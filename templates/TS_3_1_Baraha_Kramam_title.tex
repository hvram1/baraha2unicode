\begin{titlepage}
    \begin{center}
 
\begin{sanskrit}
    { \Large
    ॐ नमः परमात्मने, श्री महागणपतये नमः\\ 
    श्री गुरुभ्यो नमः,\\
     ह॒रिः॒ ॐ 
    }
    \\
    \vspace{2.5cm}
    \mbox{ \Large
    कृष्ण यजुर्वेदीय तैत्तिरीय संहितायं तृतीय काण्डे प्रथमः प्रश्नः 
    }
    
\end{sanskrit}
\end{center}

\end{titlepage}

\pagebreak

\begin{english}
    
    
    {
    
        Notes for Users of this document:\newline%
    
        Notes for Users of this document: \\
        1. Coding is more Phonetic in Baraha. \\
        2. Special care – Baraha coding uses ‘E’ and ‘O’ (in Sanskrit) fOr sOund E and O. we have used capital letters as it matches with Unicode/general conventions in this document. \\
        3. ~g is nasal of ka varga (gna) \\
        4. ~j is nasal of cha varga (jya) \\
        5. H is visargam \\
        6.  Vedic/Swara symbols – q anudAttam, \# swaritam, \$ dheerga swaritam, \& avagraha,  ~M nasal symbol \\
        7.  ushmAn codes s, S, Sha (saraswati, Sankar, puShpam) \\
        9. Confirm corrections given in TS 1.1 Pada Paatam are incorporated \\
        10. The representation of Sn has been changed to S~j as per Siksha. This will be implemented in other areas over a period of time. \\
    
    Kindly notify any major errors or inadvertent deletions to \\
    mail id- \textbf{vedavms@gmail.com}
    }
    \end{english}\par
\pagebreak

\tableofcontents

\pagebreak


