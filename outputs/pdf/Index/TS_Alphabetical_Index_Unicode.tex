\documentclass[17pt]{extarticle}
\usepackage{babel}
\usepackage{fontspec}
\usepackage{polyglossia}
\usepackage{extsizes}

\usepackage{booktabs} % To thicken table lines



\setmainlanguage{sanskrit}
\setotherlanguages{english} %% or other languages
\setlength{\parindent}{0pt}
\pagestyle{myheadings}
\newfontfamily\devanagarifont[Script=Devanagari]{AdishilaVedic}


\newcommand{\VAR}[1]{}
\newcommand{\BLOCK}[1]{}

\usepackage{longtable} % To display tables on several pages

\begin{document} 


\begin{longtable}{||p{0.4in}||p{4.9in}||p{0.9in}||} % <-- Replaces \begin{table}, alignment must be specified here (no more tabular)
    \caption{कृष्ण यजुर्वेदीय तैत्तिरीय संहिता}
    \label{tab:table1}\\
    \toprule
    \textbf{SNo} & \textbf{Beginning Padams} & \textbf{Panchasat} 
    
   
    \endfirsthead % <-- This denotes the end of the header, which will be shown on the first page only
    \toprule
    \textbf{SNo} & \textbf{Beginning Padams} & \textbf{Panchasat} 
    
   
    \endhead % <-- Everything between \endfirsthead and \endhead will be shown as a header on every page
        
    1 & अक्षन्न्॑   ।   अमी॑मदन्त   ।   हि   ।   अवेति॑   ।    & TS\_1.8.5.2       \\
    
    \hline
        
    2 & अग्ना॑विष्णू॒ इत्यग्ना᳚{-}वि॒ष्णू॒   ।   महि॑   ।   तत्   ।   वा॒म्   ।    & TS\_1.8.22.1       \\
    
    \hline
        
    3 & अग्ना॑विष्णू॒ इत्यग्ना᳚{-}वि॒ष्णू॒   ।   स॒जोष॒सेति॑ स{-}जोष॑सा   ।   इ॒माः   ।   व॒द्‌र्ध॒न्तु॒   ।    & TS\_4.7.1.1       \\
    
    \hline
        
    4 & अग्ने᳚   ।   अ॒ङ्गि॒रः॒   ।   इति॑   ।   त्रिः   ।    & TS\_6.2.7.3       \\
    
    \hline
        
    5 & अग्ने᳚   ।   आयूꣳ॑षि   ।   प॒व॒से॒   ।   एति॑   ।    & TS\_1.4.29.1       \\
    
    \hline
        
    6 & अग्ने᳚   ।   उ॒द॒ध॒ इत्यु॑द{-}धे॒   ।   या   ।   ते॒   ।    & TS\_5.5.9.1       \\
    
    \hline
        
    7 & अग्ने᳚   ।   एति॑   ।   या॒हि॒   ।   वी॒तये᳚   ।    & TS\_5.5.6.1       \\
    
    \hline
        
    8 & अग्ने᳚   ।   गोभिः॑   ।   नः॒   ।   एति॑   ।    & TS\_2.4.5.1       \\
    
    \hline
        
    9 & अग्ने᳚   ।   जा॒तान्   ।   प्रेति॑   ।   नु॒द॒   ।    & TS\_4.3.12.1 TS\_5.3.5.1       \\
    
    \hline
        
    10 & अग्ने᳚   ।   तव॑   ।   श्रवः॑   ।   वयः॑   ।    & TS\_5.2.6.1       \\
    
    \hline
        
    11 & अग्ने᳚   ।   ते॒ज॒स्वि॒न्न्   ।   ते॒ज॒स्वी   ।   त्वम्   ।    & TS\_3.3.1.1       \\
    
    \hline
        
    12 & अग्ने᳚   ।   दाः   ।   दा॒शुषे᳚   ।   र॒यिम्   ।    & TS\_2.2.12.6       \\
    
    \hline
        
    13 & अग्ने᳚   ।   भर॑न्तु   ।   चित्ति॑भि॒रिति॒ चित्ति॑{-}भिः॒   ।   सः   ।    & TS\_4.6.3.2       \\
    
    \hline
        
    14 & अग्ने᳚   ।   म॒हान्   ।   अ॒सि॒   ।   इति॑   ।    & TS\_2.5.9.1       \\
    
    \hline
        
    15 & अग्ने᳚   ।   यम्   ।   य॒ज्ञ्म्   ।   अ॒द्ध्व॒रम्   ।    & TS\_4.1.11.1       \\
    
    \hline
        
    16 & अग्ने᳚   ।   विश्वे॑भिः   ।   स्व॒नी॒केति॑ सु{-}अ॒नी॒क॒   ।   दे॒वैः   ।    & TS\_3.5.11.2       \\
    
    \hline
        
    17 & अग्रे᳚   ।   अप॑श्यत्   ।   घृ॒तप॒दीति॑ घृ॒त{-}प॒दी॒   ।   इति॑   ।    & TS\_2.6.7.4       \\
    
    \hline
        
    18 & अग्रे᳚   ।   बृ॒हन्न्   ।   उ॒षसा᳚म्   ।   ऊ॒द्‌र्ध्वः   ।    & TS\_5.2.1.5       \\
    
    \hline
        
    19 & अग्र᳚म्   ।   ए॒व   ।   स॒मा॒नाना᳚म्   ।   परीति॑   ।    & TS\_6.6.11.3       \\
    
    \hline
        
    20 & अग॑न्म   ।   सुवः॑   ।   सुवः॑   ।   अ॒ग॒न्म॒   ।    & TS\_1.6.6.1 TS\_1.7.6.1       \\
    
    \hline
        
    21 & अङ्गि॑रसः   ।   वै   ।   स॒त्रम्   ।   आ॒स॒त॒   ।    & TS\_7.1.4.1       \\
    
    \hline
        
    22 & अच्छ॑   ।   या॒ति॒   ।   तस्मा᳚त्   ।   अ॒नो॒वा॒ह्य॑मित्य॑नः{-}वा॒ह्य᳚म्   ।    & TS\_6.1.9.4       \\
    
    \hline
        
    23 & अतीति॑   ।   अ॒न्यान्   ।   अगा᳚म्   ।   न   ।    & TS\_1.3.5.1       \\
    
    \hline
        
    24 & अत्यꣳ॑हा॒ इत्यति॑{-}अꣳ॒॒हाः॒   ।   अ॒ग्नये᳚   ।   स्वाहा᳚   ।   सोमा॑य   ।    & TS\_1.8.13.3       \\
    
    \hline
        
    25 & अत्रिः॑   ।   अ॒द॒दा॒त्   ।   और्वा॑य   ।   प्र॒जामिति॑ प्र{-}जाम्   ।    & TS\_7.1.8.1       \\
    
    \hline
        
    26 & अथो॒ इति॑   ।   तेज॑सा   ।   अ॒नु॒प॒रि॒हार॒मित्य॑नु{-}प॒रि॒हार᳚म्   ।   सा॒द॒य॒ति॒   ।    & TS\_5.3.10.4       \\
    
    \hline
        
    27 & अथो॒ इति॑   ।   दे॒वताः᳚   ।   ए॒व   ।   अ॒न्वा॒रभ्येत्य॑नु{-}आ॒रभ्य॑   ।    & TS\_2.2.5.5       \\
    
    \hline
        
    28 & अथो॒ इति॑   ।   पञ्चा᳚क्ष॒रेति॒ पञ्च॑{-}अ॒क्ष॒रा॒   ।   प॒ङ्क्तिः   ।   पाङ्क्तः॑   ।    & TS\_6.1.5.2       \\
    
    \hline
        
    29 & अथो॒ इति॑   ।   रू॒पेण॑   ।   ए॒व   ।   ए॒न॒म्   ।    & TS\_5.4.2.4       \\
    
    \hline
        
    30 & अथ॑   ।   इय॑ति   ।   अथ॑   ।   इय॑ति   ।    & TS\_5.4.3.4       \\
    
    \hline
        
    31 & अथ॑र्वणः   ।   वृ॒त्र॒हण॒मिति॑ वृत्र{-}हन᳚म्   ।   पु॒र॒न्द॒रमिति॑ पुरं{-}द॒रम्   ।   तम्   ।    & TS\_4.1.3.3       \\
    
    \hline
        
    32 & अदि॑तिः   ।   पु॒त्रका॒मेति॑ पु॒त्र{-}का॒मा॒   ।   सा॒द्ध्येभ्यः॑   ।   दे॒वेभ्यः॑   ।    & TS\_6.5.6.1       \\
    
    \hline
        
    33 & अदि॑त्यै   ।   त्रयः॑   ।   रो॒हि॒तै॒ता इति॑ रोहित{-}ए॒ताः   ।   इ॒न्द्रा॒ण्यै   ।    & TS\_5.6.18.1       \\
    
    \hline
        
    34 & अद॑ब्धः   ।   गो॒पा इति॑ गो{-}पाः   ।   परीति॑   ।   पा॒हि॒   ।    & TS\_4.7.14.3       \\
    
    \hline
        
    35 & अद॑ब्धेभिः   ।   स॒वि॒तः॒   ।   पा॒युभि॒रिति॑ पा॒यु{-}भिः॒   ।   त्वम्   ।    & TS\_1.4.24.1       \\
    
    \hline
        
    36 & अधि॑पत्नी॒रित्यधि॑{-}प॒त्नीः॒   ।   नाम॑   ।   इष्ट॑काः   ।   यस्य॑   ।    & TS\_5.4.2.3       \\
    
    \hline
        
    37 & अना॑तुर॒मित्यना᳚{-}तु॒र॒म्   ।   मृ॒ड   ।   नः॒   ।   रु॒द्र॒   ।    & TS\_4.5.10.2       \\
    
    \hline
        
    38 & अना॑मृत॒ इत्यना᳚{-}मृ॒ते॒   ।   ए॒व   ।   अ॒ग्निम्   ।   चि॒नु॒ते॒   ।    & TS\_5.2.3.2       \\
    
    \hline
        
    39 & अनु॑मत्या॒ इत्यनु॑{-}म॒त्यै॒   ।   पु॒रो॒डाश᳚म्   ।   अ॒ष्टाक॑पाल॒मित्य॒ष्टा{-}क॒पा॒ल॒म्   ।   निरिति॑   ।    & TS\_1.8.1.1       \\
    
    \hline
        
    40 & अन्न॑पत॒ इत्यन्न॑{-}प॒ते॒   ।   अन्न॑स्य   ।   नः॒   ।   दे॒हि॒   ।    & TS\_4.2.3.1 TS\_5.2.2.1       \\
    
    \hline
        
    41 & अन्न॑स्य   ।   रू॒पम्   ।   रू॒पेण॑   ।   ए॒व   ।    & TS\_5.4.8.3       \\
    
    \hline
        
    42 & अन्न᳚म्   ।   अवेति॑   ।   रु॒न्धे॒   ।   पात्रा॑णि   ।    & TS\_5.6.2.3       \\
    
    \hline
        
    43 & अन्न᳚म्   ।   प्रेति॑   ।   य॒च्छ॒ति॒   ।   अ॒न्ना॒द इत्य॑न्न{-}अ॒दः   ।    & TS\_2.1.6.2       \\
    
    \hline
        
    44 & अन्न᳚म्   ।   वः॒   ।   इष॑वः   ।   नि॒मि॒ष इति॑ नि{-}मि॒षः   ।    & TS\_5.5.10.5       \\
    
    \hline
        
    45 & अन्न᳚म्   ।   स॒प्त॒द॒श इति॑ सप्त{-}द॒शः   ।   अन्न᳚म्   ।   ए॒व   ।    & TS\_5.3.4.2       \\
    
    \hline
        
    46 & अन्विति॑   ।   अह॑   ।   मासाः᳚   ।   अन्विति॑   ।    & TS\_1.7.13.1       \\
    
    \hline
        
    47 & अन्विति॑   ।   अ॒सृ॒ज्य॒त॒   ।   वै॒रा॒जम्   ।   साम॑   ।    & TS\_7.1.1.6       \\
    
    \hline
        
    48 & अन्विति॑   ।   ए॒ति॒   ।   न   ।   अ॒न्त॒मम्   ।    & TS\_6.3.9.5       \\
    
    \hline
        
    49 & अन्विति॑   ।   त्वा॒   ।   मा॒ता   ।   म॒न्य॒ता॒म्   ।    & TS\_6.1.7.7       \\
    
    \hline
        
    50 & अन्विति॑   ।   प॒शवः॑   ।   उपेति॑   ।   ति॒ष्ठ॒न्ते॒   ।    & TS\_6.6.4.2       \\
    
    \hline
        
    51 & अन्विति॑   ।   मा॒र्ष्टु॒   ।   त॒नुवः॑   ।   यत्   ।    & TS\_1.4.44.2       \\
    
    \hline
        
    52 & अन॑वरुद्ध॒स्येत्यन॑व{-}रु॒द्ध॒स्य॒   ।   अ॒श्नी॒यात्   ।   अव॑रुद्धे॒नेत्यव॑{-}रु॒द्धे॒न॒   ।   वीति॑   ।    & TS\_5.2.5.6       \\
    
    \hline
        
    53 & अपीति॑   ।   कृ॒न्ता॒मि॒   ।   अ॒स्मे इति॑   ।   रायः॑   ।    & TS\_1.2.5.2       \\
    
    \hline
        
    54 & अपीति॑   ।   द॒द्ध्या॒त्   ।   प्र॒मायु॑क॒ इति॑ प्र{-}मायु॑कः   ।   स्या॒त्   ।    & TS\_6.3.5.2       \\
    
    \hline
        
    55 & अपेति॑   ।   अम॑तिम्   ।   दु॒र्म॒तिमिति॑ दुः{-}म॒तिम्   ।   बाध॑माना   ।    & TS\_4.3.4.2       \\
    
    \hline
        
    56 & अपेति॑   ।   इ॒त॒   ।   वीति॑   ।   इ॒त॒   ।    & TS\_4.2.4.1       \\
    
    \hline
        
    57 & अपेति॑   ।   क्रा॒म॒तः॒   ।   यः   ।   अल᳚म्   ।    & TS\_2.1.1.3       \\
    
    \hline
        
    58 & अपेति॑   ।   ह॒न्ति॒   ।   इन्द्रा॑य   ।   त्रा॒त्रे   ।    & TS\_2.2.7.5       \\
    
    \hline
        
    59 & अपेति॑   ।   ह॒न्ति॒   ।   पुरु॑षःपुरुष॒ इति॒ पुरु॑षः{-}पु॒रु॒षः॒   ।   नि॒धन॒मिति॑ नि{-}धन᳚म्   ।    & TS\_6.6.3.2       \\
    
    \hline
        
    60 & अप्र॑तीत्त॒मित्यप्र॑ति{-}इ॒त्त॒म्   ।   मयि॑   ।   येन॑   ।   य॒मस्य॑   ।    & TS\_3.3.8.2       \\
    
    \hline
        
    61 & अप॑चितिमा॒नित्यप॑चिति{-}मा॒न्   ।   भ॒व॒ति॒   ।   यः   ।   ए॒वम्   ।    & TS\_5.2.2.4       \\
    
    \hline
        
    62 & अफ्सः॑   ।   नाम॑   ।   एवः॑   ।   छन्दः॑   ।    & TS\_4.3.12.2       \\
    
    \hline
        
    63 & अबें᳚   ।   अम्बा॑लि   ।   अम्बि॑के   ।   न   ।    & TS\_7.4.19.1       \\
    
    \hline
        
    64 & अभ्रा॑तृव्यः   ।   यत्   ।   इन्द्रा॑य   ।   वै॒मृ॒धाय॑   ।    & TS\_2.4.2.4       \\
    
    \hline
        
    65 & अभ॑वन्न्   ।   परेति॑   ।   असु॑राः   ।   यस्य॑   ।    & TS\_2.5.11.9       \\
    
    \hline
        
    66 & अमी॑वाः   ।   ए॒षः   ।   स्यः   ।   वा॒जी   ।    & TS\_1.7.8.3       \\
    
    \hline
        
    67 & अय॑ज्ञ्ः   ।   वै   ।   ए॒षः   ।   यः   ।    & TS\_1.5.7.1 TS\_2.5.8.1       \\
    
    \hline
        
    68 & अरि॑क्तानि   ।   पात्रा॑णि   ।   सा॒द॒य॒ति॒   ।   तस्मा᳚त्   ।    & TS\_6.4.9.5       \\
    
    \hline
        
    69 & अर्च॑न्ति   ।   अ॒र्कम्   ।   अ॒र्किणः॑   ।   ब्र॒ह्माणः॑   ।    & TS\_1.6.12.3       \\
    
    \hline
        
    70 & अवेति॑   ।   ई॒क्षे॒त॒   ।   ए॒षः   ।   वै   ।    & TS\_3.2.3.4       \\
    
    \hline
        
    71 & अवेति॑   ।   रु॒न्धे॒   ।   ए॒ताव॑त्   ।   वै   ।    & TS\_3.3.5.4       \\
    
    \hline
        
    72 & अवेति॑   ।   रु॒न्धे॒   ।   चतु॑श्चत्वारिꣳशत॒मिति॒ चतुः॑{-}च॒त्वा॒रिꣳ॒॒श॒त॒म्   ।   अन्विति॑   ।    & TS\_2.5.10.4       \\
    
    \hline
        
    73 & अवेति॑   ।   रु॒न्धे॒   ।   ज्ये॒ष्ठाः   ।   वै   ।    & TS\_3.5.10.2       \\
    
    \hline
        
    74 & अवेति॑   ।   रु॒न्धे॒   ।   ता॒र्प्यम्   ।   द॒दा॒ति॒   ।    & TS\_2.4.11.6       \\
    
    \hline
        
    75 & अवेति॑   ।   रु॒न्धे॒   ।   प॒शु॒मानिति॑ पशु{-}मान्   ।   ए॒व   ।    & TS\_6.2.6.4       \\
    
    \hline
        
    76 & अवेति॑   ।   रु॒न्धे॒   ।   मुञ्जान्॑   ।   अवेति॑   ।    & TS\_5.1.9.5       \\
    
    \hline
        
    77 & अवेति॑   ।   रु॒न्धे॒   ।   म॒द्ध्य॒तः   ।   धा॒तार᳚म्   ।    & TS\_3.4.9.3       \\
    
    \hline
        
    78 & अवेति॑   ।   रु॒न्धे॒   ।   सं॒ॅव॒थ्स॒रमिति॑ सं{-}व॒थ्स॒रम्   ।   प॒र्याल॑भ्यन्त॒ इति॑ परि{-}आल॑भ्यन्ते   ।    & TS\_2.1.2.6       \\
    
    \hline
        
    79 & अव॑ग॒तेत्यव॑{-}ग॒ता॒   ।   अ॒स्य॒   ।   विट्   ।   अन॑वगत॒मित्यन॑व{-}ग॒त॒म्   ।    & TS\_2.3.1.4       \\
    
    \hline
        
    80 & अव॑धूत॒मित्यव॑{-}धू॒त॒म्   ।   रक्षः॑   ।   अव॑धूता॒ इत्यव॑{-}धू॒ताः॒   ।   अरा॑तयः   ।    & TS\_1.1.6.1       \\
    
    \hline
        
    81 & अव॑से   ।   सखा॑यः   ।   अन्विति॑   ।   त्वा॒   ।    & TS\_6.3.11.3       \\
    
    \hline
        
    82 & अश्मन्न्॑   ।   ऊर्ज᳚म्   ।   इति॑   ।   परीति॑   ।    & TS\_5.4.4.1       \\
    
    \hline
        
    83 & अश्मन्न्॑   ।   ऊर्ज᳚म्   ।   पर्व॑ते   ।   शि॒श्रि॒या॒णाम्   ।    & TS\_4.6.1.1       \\
    
    \hline
        
    84 & अश्मा॑नम्   ।   नीति॑   ।   द॒धा॒ति॒   ।   स॒त्त्वायेति॑ सत्{-}त्वाय॑   ।    & TS\_5.4.6.5       \\
    
    \hline
        
    85 & अश्मा᳚   ।   च॒   ।   मे॒   ।   मृत्ति॑का   ।    & TS\_4.7.5.1       \\
    
    \hline
        
    86 & अश्वः॑   ।   तू॒प॒रः   ।   गो॒मृ॒ग इति॑ गो{-}मृ॒गः   ।   ते   ।    & TS\_5.5.23.1       \\
    
    \hline
        
    87 & असा॑वि   ।   सोमः॑   ।   इ॒न्द्र॒   ।   ते॒   ।    & TS\_1.4.39.1       \\
    
    \hline
        
    88 & असु॑रान्   ।   अ॒ज॒य॒न्न्   ।   तत्   ।   अप्र॑तिरथ॒स्येत्यप्र॑ति{-}र॒थ॒स्य॒   ।    & TS\_5.4.6.4       \\
    
    \hline
        
    89 & असु॑रान्   ।   जया॑म   ।   तत्   ।   नः॒   ।    & TS\_2.4.1.2       \\
    
    \hline
        
    90 & असु॑वर्ग्य॒मित्यसु॑वः{-}ग्य॒म्   ।   अ॒स्य॒   ।   तत्   ।   सु॒व॒र्ग्य॑ इति॑ सुवः{-}ग्यः॑   ।    & TS\_5.2.10.7       \\
    
    \hline
        
    91 & अहः॑   ।   एकः॑   ।   अभ॑जत   ।   अहः॑   ।    & TS\_7.4.5.2       \\
    
    \hline
        
    92 & अहः॑   ।   तस्मिन्न्॑   ।   ऐ॒न्द्र॒वा॒य॒व इत्यै᳚न्द्र{-}वा॒य॒वः   ।   गृ॒ह्य॒ते॒   ।    & TS\_7.2.8.7       \\
    
    \hline
        
    93 & अहः॑   ।   माꣳ॒॒सेन॑   ।   रात्रि᳚म्   ।   पीव॑सा   ।    & TS\_5.7.20.1       \\
    
    \hline
        
    94 & अहन्न्॑   ।   साम॑   ।   भ॒व॒ति॒   ।   इ॒यम्   ।    & TS\_7.1.4.3       \\
    
    \hline
        
    95 & अह॑रत्   ।   तस्मा᳚त्   ।   गा॒य॒त्रि॒यै   ।   उ॒भ॒यतः॑   ।    & TS\_6.2.1.4       \\
    
    \hline
        
    96 & अ॒का॒म॒य॒न्त॒   ।   प॒शून्   ।   वि॒न्दे॒म॒हि॒   ।   इति॑   ।    & TS\_6.2.8.2       \\
    
    \hline
        
    97 & अ॒कु॒र्व॒त॒   ।   तैः   ।   वै   ।   ते   ।    & TS\_6.2.1.6       \\
    
    \hline
        
    98 & अ॒कृ॒न्त॒न्न्   ।   यवे॑न   ।   संमि॑त॒मिति॒ सं{-}मि॒त॒म्   ।   तस्मा᳚त्   ।    & TS\_2.6.8.4       \\
    
    \hline
        
    99 & अ॒कृ॒ष्ट॒प॒च्यमित्य॑कृष्ट{-}प॒च्यम्   ।   इति॑   ।   उ॒भौ   ।   वा॒म्   ।    & TS\_2.4.4.3       \\
    
    \hline
        
    100 & अ॒ख्य॒त्   ।   इति॑   ।   आ॒ह॒   ।   अनु॑ख्यात्या॒ इत्यनु॑{-}ख्या॒त्यै॒   ।    & TS\_5.1.2.6       \\
    
    \hline
        
    101 & अ॒गृ॒ह्णा॒त्   ।   अस्तु॑   ।   ए॒व   ।   अ॒यम्   ।    & TS\_2.1.7.2       \\
    
    \hline
        
    102 & अ॒ग्नये᳚   ।   अनी॑कवत॒ इत्यनी॑क{-}व॒ते॒   ।   पु॒रो॒डाश᳚म्   ।   अ॒ष्टाक॑पाल॒मित्य॒ष्टा{-}क॒पा॒ल॒म्   ।    & TS\_1.8.4.1       \\
    
    \hline
        
    103 & अ॒ग्नये᳚   ।   अनी॑कवत॒ इत्यनी॑क{-}व॒ते॒   ।   रोहि॑ताञ्जि॒रिति॒ रोहि॑त{-}अ॒ञ्जिः॒   ।   अ॒न॒ड्वान्   ।    & TS\_5.5.24.1       \\
    
    \hline
        
    104 & अ॒ग्नये᳚   ।   अन्न॑वत॒ इत्यन्न॑{-}व॒ते॒   ।   पु॒रो॒डाश᳚म्   ।   अ॒ष्टाक॑पाल॒मित्य॒ष्टा{-}क॒पा॒ल॒म्   ।    & TS\_2.2.4.1       \\
    
    \hline
        
    105 & अ॒ग्नये᳚   ।   अꣳ॒॒हो॒मुच॒ इत्यꣳ॑हः{-}मुचे᳚   ।   अष्टाक॑पाल॒ इत्य॒ष्टा{-}क॒पा॒लः॒   ।   इन्द्रा॑य   ।    & TS\_7.5.22.1       \\
    
    \hline
        
    106 & अ॒ग्नये᳚   ।   कामा॑य   ।   पु॒रो॒डाश᳚म्   ।   अ॒ष्टाक॑पाल॒मित्य॒ष्टा{-}क॒पा॒ल॒म्   ।    & TS\_2.2.3.1       \\
    
    \hline
        
    107 & अ॒ग्नये᳚   ।   गा॒य॒त्राय॑   ।   त्रि॒वृत॒ इति॑ त्रि{-}वृते᳚   ।   राथ॑न्तरा॒येति॒ राथं᳚{-}त॒रा॒य॒   ।    & TS\_7.5.14.1       \\
    
    \hline
        
    108 & अ॒ग्नये᳚   ।   गृ॒हप॑तय॒ इति॑ गृ॒ह{-}प॒त॒ये॒   ।   पु॒रो॒डाश᳚म्   ।   अ॒ष्टाक॑पाल॒मित्य॒ष्टा{-}क॒पा॒ल॒म्   ।    & TS\_1.8.10.1       \\
    
    \hline
        
    109 & अ॒ग्नये᳚   ।   प॒थि॒कृत॒ इति॑ पथि{-}कृते᳚   ।   पु॒रो॒डाश᳚म्   ।   अ॒ष्टाक॑पाल॒मित्य॒ष्टा{-}क॒पा॒ल॒म्   ।    & TS\_2.2.2.1       \\
    
    \hline
        
    110 & अ॒ग्नये᳚   ।   समिति॑   ।   अ॒न॒म॒त्   ।   पृ॒थि॒व्यै   ।    & TS\_7.5.23.1       \\
    
    \hline
        
    111 & अ॒ग्नये᳚   ।   स्वाहा᳚   ।   वा॒यवे᳚   ।   स्वाहा᳚   ।    & TS\_7.1.20.1       \\
    
    \hline
        
    112 & अ॒ग्नये᳚   ।   स्वाहा᳚   ।   सोमा॑य   ।   स्वाहा᳚   ।    & TS\_7.1.14.1 TS\_7.1.16.1       \\
    
    \hline
        
    113 & अ॒ग्नय᳚   ।   प्रती॑कवत॒ इति॒ प्रती॑क{-}व॒ते॒   ।   यत्   ।   अ॒ग्नये᳚   ।    & TS\_2.4.1.4       \\
    
    \hline
        
    114 & अ॒ग्निः   ।   एका᳚क्षरे॒णेत्येक॑ {-}अ॒क्ष॒रे॒ण॒   ।   वाच᳚म्   ।   उदिति॑   ।    & TS\_1.7.11.1       \\
    
    \hline
        
    115 & अ॒ग्निः   ।   च॒   ।   इ॒दम्   ।   क॒रि॒ष्य॒थः॒   ।    & TS\_4.1.9.2       \\
    
    \hline
        
    116 & अ॒ग्निः   ।   च॒   ।   मे॒   ।   इन्द्रः॑   ।    & TS\_4.7.6.1       \\
    
    \hline
        
    117 & अ॒ग्निः   ।   च॒   ।   मे॒   ।   घ॒र्मः   ।    & TS\_4.7.9.1       \\
    
    \hline
        
    118 & अ॒ग्निः   ।   च॒   ।   वै   ।   ए॒तौ   ।    & TS\_6.2.1.7       \\
    
    \hline
        
    119 & अ॒ग्निः   ।   तेन॑   ।   ए॒व   ।   य॒ज्ञ्॒मु॒खादिति॑ यज्ञ्{-}मु॒खात्   ।    & TS\_5.1.1.3       \\
    
    \hline
        
    120 & अ॒ग्निः   ।   दे॒वेभ्यः॑   ।   अपेति॑   ।   अ॒क्रा॒म॒त्   ।    & TS\_5.4.9.1       \\
    
    \hline
        
    121 & अ॒ग्निः   ।   पृ॒थि॒व्याः   ।   य॒मः   ।   पि॒तृभ्य॒ इति॑ पि॒तृ{-}भ्यः॒   ।    & TS\_3.2.4.4       \\
    
    \hline
        
    122 & अ॒ग्निः   ।   प॒शुः   ।   आ॒सी॒त्   ।   तेन॑   ।    & TS\_5.7.26.1       \\
    
    \hline
        
    123 & अ॒ग्निः   ।   बृ॒हद्व॑या॒ इति॑ बृ॒हत्{-}व॒याः॒   ।   वि॒श्व॒जिदिति॑ विश्व{-}जित्   ।   सह॑न्त्यः   ।    & TS\_1.5.10.2       \\
    
    \hline
        
    124 & अ॒ग्निः   ।   भू॒ताना᳚म्   ।   अधि॑पति॒रित्यधि॑{-} प॒तिः॒   ।   सः   ।    & TS\_3.4.5.1       \\
    
    \hline
        
    125 & अ॒ग्निः   ।   मा॒   ।   दुरि॑ष्टा॒दिति॒ दुः{-}इ॒ष्टा॒त्   ।   पा॒तु॒   ।    & TS\_1.6.3.1       \\
    
    \hline
        
    126 & अ॒ग्निः   ।   मू॒द्‌र्धा   ।   दि॒वः   ।   क॒कुत्   ।    & TS\_4.4.4.1       \\
    
    \hline
        
    127 & अ॒ग्निः   ।   यावान्॑   ।   ए॒व   ।   अ॒ग्निः   ।    & TS\_5.2.2.5       \\
    
    \hline
        
    128 & अ॒ग्निः   ।   वृ॒त्राणि॑   ।   ज॒ङ्घ॒न॒त्   ।   द्र॒वि॒ण॒स्युः   ।    & TS\_4.3.13.1       \\
    
    \hline
        
    129 & अ॒ग्निः   ।   वै॒श्वा॒न॒रः   ।   यत्   ।   ब्रा॒ह्म॒णः   ।    & TS\_5.2.8.2       \\
    
    \hline
        
    130 & अ॒ग्निः   ।   वै॒श्व॒क॒र्म॒ण इति॑ वैश्व{-}क॒र्म॒णः   ।   सुवः॑   ।   दे॒वेषु॑   ।    & TS\_5.7.7.3       \\
    
    \hline
        
    131 & अ॒ग्निना᳚   ।   तपः॑   ।   अन्विति॑   ।   अ॒भ॒व॒त्   ।    & TS\_7.3.14.1       \\
    
    \hline
        
    132 & अ॒ग्निना᳚   ।   दे॒वेन॑   ।   पृत॑नाः   ।   ज॒या॒मि॒   ।    & TS\_3.5.3.1       \\
    
    \hline
        
    133 & अ॒ग्निना᳚   ।   र॒यिम्   ।   अ॒श्न॒व॒त्   ।   पोष᳚म्   ।    & TS\_3.1.11.1       \\
    
    \hline
        
    134 & अ॒ग्निना᳚   ।   वि॒श्वा॒षाट्   ।   सूर्ये॑ण   ।   स्व॒राडिति॑ स्व{-}राट्   ।    & TS\_4.4.8.1       \\
    
    \hline
        
    135 & अ॒ग्निना᳚   ।   वै   ।   होत्रा᳚   ।   दे॒वाः   ।    & TS\_6.3.7.1       \\
    
    \hline
        
    136 & अ॒ग्निम्   ।   ए॒व   ।   वरू॑थम्   ।   कृ॒त्वा   ।    & TS\_6.2.2.7       \\
    
    \hline
        
    137 & अ॒ग्निम्   ।   मन॑सा   ।   घृ॒तेन॑   ।   प्र॒ति॒क्ष्यन्त॒मिति॑ प्रति{-}क्ष्यन्त᳚म्   ।    & TS\_4.1.2.5       \\
    
    \hline
        
    138 & अ॒ग्निम्   ।   यु॒न॒ज्मि॒   ।   शव॑सा   ।   घृ॒तेन॑   ।    & TS\_4.7.13.1       \\
    
    \hline
        
    139 & अ॒ग्निम्   ।   वै   ।   ए॒तस्य॑   ।   शरी॑रम्   ।    & TS\_2.3.11.1       \\
    
    \hline
        
    140 & अ॒ग्नि॒ष्टो॒ममित्य॑ग्नि{-}स्तो॒मम्   ।   पूर्व᳚म्   ।   प्र॒यु॒ञ्जी॒रन्निति॑ प्र{-}यु॒ञ्जी॒रन्न्   ।   ब॒हि॒द्‌र्धेति॑ बहिः{-}धा   ।    & TS\_7.2.9.2       \\
    
    \hline
        
    141 & अ॒ग्नि॒ष्ठामित्य॑ग्नि{-}स्थाम्   ।   तस्य॑   ।   अश्रि᳚म्   ।   आ॒ह॒व॒नीये॒नेत्या᳚{-}ह॒व॒नीये॑न   ।    & TS\_6.3.4.5       \\
    
    \hline
        
    142 & अ॒ग्नि॒हो॒त्रमित्य॑ग्नि{-}हो॒त्रम्   ।   जु॒हो॒ति॒   ।   यत्   ।   ए॒व   ।    & TS\_1.5.9.1       \\
    
    \hline
        
    143 & अ॒ग्नीध॒ इत्य॑ग्नि{-}इधे᳚   ।   एति॑   ।   द॒धा॒ति॒   ।   अ॒ग्निमु॑खा॒नित्य॒ग्नि{-}मु॒खा॒न्   ।    & TS\_2.6.9.1       \\
    
    \hline
        
    144 & अ॒ग्नीषोमा॒वित्य॒ग्नी{-}सोमौ᳚   ।   इ॒ष्ट्वा   ।   स॒वि॒तार᳚म्   ।   य॒ज॒ति॒   ।    & TS\_6.1.5.3       \\
    
    \hline
        
    145 & अ॒ग्नेः   ।   आ॒ति॒थ्यम्   ।   अ॒सि॒   ।   विष्ण॑वे   ।    & TS\_1.2.10.1       \\
    
    \hline
        
    146 & अ॒ग्नेः   ।   त्रयः॑   ।   ज्यायाꣳ॑सः   ।   भ्रात॑रः   ।    & TS\_2.6.6.1       \\
    
    \hline
        
    147 & अ॒ग्नेः   ।   प॒क्ष॒तिः   ।   सर॑स्वत्यै   ।   निप॑क्षति॒रिति॒ नि{-}प॒क्ष॒तिः॒   ।    & TS\_5.7.21.1       \\
    
    \hline
        
    148 & अ॒ग्नेः   ।   भा॒गः   ।   अ॒सि॒   ।   इति॑   ।    & TS\_5.3.4.1       \\
    
    \hline
        
    149 & अ॒ग्नेः   ।   भा॒गः   ।   अ॒सि॒   ।   दी॒क्षायाः᳚   ।    & TS\_4.3.9.1       \\
    
    \hline
        
    150 & अ॒ग्नेः   ।   म॒न्वे॒   ।   प्र॒थ॒मस्य॑   ।   प्रचे॑तस॒ इति॒ प्र{-}चे॒त॒सः॒   ।    & TS\_4.7.15.1       \\
    
    \hline
        
    151 & अ॒ग्नेः   ।   वै   ।   दी॒क्षया᳚   ।   दे॒वाः   ।    & TS\_5.6.7.1       \\
    
    \hline
        
    152 & अ॒ग्ने॒   ।   एति॑   ।   रो॒ह॒   ।   अथ॑   ।    & TS\_4.2.4.4       \\
    
    \hline
        
    153 & अ॒ङ्गि॒रः॒   ।   तस्मै᳚   ।   नू॒नम्   ।   अ॒भिद्य॑व॒ इत्य॒भि{-}द्य॒वे॒   ।    & TS\_2.6.11.2       \\
    
    \hline
        
    154 & अ॒ङ्गि॒रः॒   ।   श॒तम्   ।   ते॒   ।   स॒न्तु॒   ।    & TS\_4.2.1.3       \\
    
    \hline
        
    155 & अ॒चा॒य॒त्   ।   सः   ।   अ॒म॒न्य॒त॒   ।   यः   ।    & TS\_2.1.4.6       \\
    
    \hline
        
    156 & अ॒चु॒च्य॒वुः॒   ।   इति॑   ।   आ॒ह॒   ।   य॒था॒य॒जुरिति॑ यथा{-}य॒जुः   ।    & TS\_3.3.4.2       \\
    
    \hline
        
    157 & अ॒जाया᳚म्   ।   घ॒र्मम्   ।   प्रेति॑   ।   अ॒सि॒ञ्च॒न्न्   ।    & TS\_5.4.3.3       \\
    
    \hline
        
    158 & अ॒जि॒घाꣳ॒॒स॒न्न्   ।   सः   ।   ए॒तत्   ।   रा॒क्षो॒घ्नमिति॑ राक्षः{-}घ्नम्   ।    & TS\_5.1.10.2       \\
    
    \hline
        
    159 & अ॒ज॒नय॒न्न्   ।   ये   ।   यजा॑महे   ।   इति॑   ।    & TS\_1.6.11.4       \\
    
    \hline
        
    160 & अ॒ञ्ज्ये॒तायेत्य॑ञ्जि{-}ए॒ताय॑   ।   स्वाहा᳚   ।   अ॒ञ्जि॒स॒क्थायेत्य॑ञ्जि{-}स॒क्थाय॑   ।   स्वाहा᳚   ।    & TS\_7.3.17.1       \\
    
    \hline
        
    161 & अ॒द्ध्व॒रेषु॑   ।   रा॒ज॒न्न्   ।   त्वया᳚   ।   वाज᳚म्   ।    & TS\_1.4.46.3       \\
    
    \hline
        
    162 & अ॒द्भ्य इत्य॑त्{-}भ्यः   ।   स्वाहा᳚   ।   वह॑न्तीभ्यः   ।   स्वाहा᳚   ।    & TS\_7.4.14.1       \\
    
    \hline
        
    163 & अ॒द्य   ।   वसु॑   ।   व॒स॒ति॒   ।   इति॑   ।    & TS\_2.5.3.7       \\
    
    \hline
        
    164 & अ॒द्‌र्ध॒य॒ति॒   ।   प्र॒जावा॒निति॑ प्रजा{-}वा॒न्   ।   प॒शु॒मानिति॑ पशु{-}मान्   ।   र॒यि॒मानिति॑ रयि{-}मान्   ।    & TS\_7.1.6.7       \\
    
    \hline
        
    165 & अ॒धि॒श्रय॒तीत्य॑धि{-}श्रय॑ति   ।   आज्य᳚म्   ।   च॒   ।   स्त॒बं॒य॒जुरिति॑ स्तंब{-}य॒जुः   ।    & TS\_1.6.9.4       \\
    
    \hline
        
    166 & अ॒नया᳚   ।   ए॒व   ।   ए॒नाः॒   ।   प्रेति॑   ।    & TS\_1.6.8.2       \\
    
    \hline
        
    167 & अ॒नु॒द॒त॒   ।   मनो॑जवा॒ इति॒ मनः॑{-}ज॒वाः॒   ।   पि॒तृभि॒रिति॑ पि॒तृ{-}भिः॒   ।   द॒क्षि॒ण॒तः   ।    & TS\_6.2.7.5       \\
    
    \hline
        
    168 & अ॒नु॒द॒न्त॒   ।   याः   ।   प्र॒तीचीः᳚   ।   ये   ।    & TS\_6.4.10.4       \\
    
    \hline
        
    169 & अ॒नु॒प॒रि॒क्राम॒मित्य॑नु{-}प॒रि॒क्राम᳚म्   ।   जु॒हो॒ति॒   ।   अप॑रिवर्ग॒मित्यप॑रि{-}व॒र्ग॒म्   ।   ए॒व   ।    & TS\_5.5.10.6       \\
    
    \hline
        
    170 & अ॒नु॒म्लोच॒न्तीत्य॑नु{-}म्लोच॑न्ती   ।   च॒   ।   अ॒फ्स॒रसौ᳚   ।   स॒र्पाः   ।    & TS\_4.4.3.2       \\
    
    \hline
        
    171 & अ॒न्तः   ।   अ॒मृत᳚म्   ।   अ॒फ्स्वित्य॑प्{-}सु   ।   भे॒ष॒जम्   ।    & TS\_1.7.7.2       \\
    
    \hline
        
    172 & अ॒न्तः   ।   वीति॑   ।   भा॒ति॒   ।   दे॒वाः   ।    & TS\_4.1.10.5       \\
    
    \hline
        
    173 & अ॒न्त॒र्या॒म॒पा॒त्रेणेत्य॑न्तर्याम{-}पा॒त्रेण॑   ।   सा॒वि॒त्रम्   ।   आ॒ग्र॒य॒णात्   ।   गृ॒ह्णा॒ति॒   ।    & TS\_6.5.7.1       \\
    
    \hline
        
    174 & अ॒न्नाद्य॒मित्य॑न्न{-}अद्य᳚म्   ।   आ॒त्मन्न्   ।   ध॒त्ते॒   ।   दब्धिः॑   ।    & TS\_1.6.11.6       \\
    
    \hline
        
    175 & अ॒न्यत्   ।   मुख᳚म्   ।   कु॒र्या॒त्   ।   स्रु॒वेण॑   ।    & TS\_6.2.3.3       \\
    
    \hline
        
    176 & अ॒न्या   ।   दे॒वता᳚   ।   आ॒सी॒त्   ।   सः   ।    & TS\_2.4.12.3       \\
    
    \hline
        
    177 & अ॒न्याः   ।   यन्ति॑   ।   उपेति॑   ।   य॒न्ति॒   ।    & TS\_2.5.12.2       \\
    
    \hline
        
    178 & अ॒न्ये   ।   वा॒   ।   वै   ।   नि॒धिमिति॑ नि{-}धिम्   ।    & TS\_5.6.6.2       \\
    
    \hline
        
    179 & अ॒न्ये   ।   श॒त॒रु॒द्रीय॒मिति॑ शत{-}रु॒द्रीय᳚म्   ।   हु॒त्वा   ।   गा॒वी॒धु॒कम्   ।    & TS\_5.5.9.4       \\
    
    \hline
        
    180 & अ॒न्वाभ॑ज॒दित्य॑नु{-}आभ॑जत्   ।   स॒जोषा॒ इति॑ स{-}जोषाः᳚   ।   दे॒वैः   ।   अव॑रैः   ।    & TS\_6.4.6.3       \\
    
    \hline
        
    181 & अ॒न्वाह॑र॒तीत्य॑नु{-}आह॑रति   ।   तत्   ।   अ॒न्वा॒हा॒र्य॑स्येत्य॑नु{-}आ॒हा॒र्य॑स्य   ।   अ॒न्वा॒हा॒र्य॒त्वमित्य॑न्वाहार्य{-}त्वम्   ।    & TS\_1.7.3.2       \\
    
    \hline
        
    182 & अ॒न्व॒हमित्य॑नु{-}अ॒हम्   ।   ए॒व   ।   ए॒न॒म्   ।   चि॒नु॒ते॒   ।    & TS\_5.5.6.2       \\
    
    \hline
        
    183 & अ॒न॒न्ति॒   ।   न   ।   आर्ति᳚म्   ।   एति॑   ।    & TS\_7.5.6.3       \\
    
    \hline
        
    184 & अ॒पाम्   ।   गर्भ᳚म्   ।   वीति॑   ।   अ॒द॒धा॒त्   ।    & TS\_4.6.2.4       \\
    
    \hline
        
    185 & अ॒पाम्   ।   ग्रहान्॑   ।   गृ॒ह्णा॒ति॒   ।   ए॒तत्   ।    & TS\_5.6.2.1       \\
    
    \hline
        
    186 & अ॒पाम्   ।   च॒   ।   ओष॑धीनाम्   ।   च॒   ।    & TS\_2.1.9.3       \\
    
    \hline
        
    187 & अ॒पाम्   ।   त्वा॒   ।   एमन्न्॑   ।   सा॒द॒या॒मि॒   ।    & TS\_4.3.1.1       \\
    
    \hline
        
    188 & अ॒पाम्   ।   नप्त्रे᳚   ।   ज॒षः   ।   ना॒क्रः   ।    & TS\_5.5.13.1       \\
    
    \hline
        
    189 & अ॒पा॒न॒दा इत्य॑पान{-}दाः   ।   व्या॒न॒दा इति॑ व्यान{-}दाः   ।   च॒क्षु॒र्दा इति॑ चक्षुः{-}दाः   ।   व॒र्चो॒दा इति॑ वर्चः{-}दाः   ।    & TS\_4.6.1.5       \\
    
    \hline
        
    190 & अ॒प्र॒या॒जमित्य॑प्र{-}या॒जम्   ।   उ॒द॒य॒नीय॒मित्यु॑त्{-}अ॒य॒नीय᳚म्   ।   इति॑   ।   इ॒मे   ।    & TS\_6.1.5.4       \\
    
    \hline
        
    191 & अ॒प॒क्रम्येत्य॑प{-}क्रम्य॑   ।   अ॒ति॒ष्ठ॒त्   ।   तत्   ।   रो॒हितः॑   ।    & TS\_6.1.6.6       \\
    
    \hline
        
    192 & अ॒प॒रु॒द्ध्यमा॑न॒ इत्य॑प{-}रु॒द्ध्यमा॑नः   ।   वा॒   ।   इन्द्र᳚म्   ।   ए॒व   ।    & TS\_2.2.8.5       \\
    
    \hline
        
    193 & अ॒प॒श्य॒न्न्   ।   पु॒रो॒डाश᳚म्   ।   कू॒र्मम्   ।   भू॒तम्   ।    & TS\_2.6.3.3       \\
    
    \hline
        
    194 & अ॒प॒स्याः᳚   ।   अ॒प॒श्य॒त्   ।   ताः   ।   उपेति॑   ।    & TS\_5.2.10.3       \\
    
    \hline
        
    195 & अ॒प॒हनी॒तेत्यप॑{-}हनी॑त   ।   पु॒रा   ।   अ॒स्य॒   ।   सं॒ॅव॒थ्स॒रादिति॑ सं{-}व॒थ्स॒रात्   ।    & TS\_3.2.9.5       \\
    
    \hline
        
    196 & अ॒फ्सु॒षद॒ इत्य॑फ्सु{-}सदे᳚   ।   वट्   ।   व॒न॒सद॒ इति॑ वन{-}सदे᳚   ।   वट्   ।    & TS\_4.6.1.4       \\
    
    \hline
        
    197 & अ॒फ्स्वित्य॑प्{-}सु   ।   वा॒   ।   य॒ज॒त्र॒   ।   येन॑   ।    & TS\_4.2.4.3       \\
    
    \hline
        
    198 & अ॒ब्र॒वी॒त्   ।   सः   ।   ए॒ताम्   ।   द्वि॒तीया᳚म्   ।    & TS\_5.7.5.4       \\
    
    \hline
        
    199 & अ॒भि   ।   समिति॑   ।   ए॒ति॒   ।   यत्   ।    & TS\_6.4.1.5       \\
    
    \hline
        
    200 & अ॒भि   ।   समिति॑   ।   प॒द्य॒न्ते॒   ।   द्वे इति॑   ।    & TS\_7.4.11.2       \\
    
    \hline
        
    201 & अ॒भि   ।   समिति॑   ।   भ॒व॒तः॒   ।   सः   ।    & TS\_2.2.4.7       \\
    
    \hline
        
    202 & अ॒भि॒जित्या॒ इत्य॑भि{-}जित्याः᳚   ।   इति॑   ।   आ॒हुः॒   ।   यत्   ।    & TS\_7.5.15.3       \\
    
    \hline
        
    203 & अ॒भि॒प्ल॒व इत्य॑भि{-}प्ल॒वः   ।   पूर्व᳚म्   ।   अहः॑   ।   भ॒व॒ति॒   ।    & TS\_7.1.4.2       \\
    
    \hline
        
    204 & अ॒भीति॑   ।   अ॒स्था॒त्   ।   विश्वाः᳚   ।   पृत॑नाः   ।    & TS\_4.2.8.1       \\
    
    \hline
        
    205 & अ॒भीति॑   ।   उ॒देतीत्यु॑त्{-}एति॑   ।   त्रे॒धा   ।   त॒ण्डु॒लान्   ।    & TS\_2.5.5.2       \\
    
    \hline
        
    206 & अ॒भीति॑   ।   ज॒य॒ति॒   ।   अ॒ग्नि॒चित॒मित्य॑ग्नि{-}चित᳚म्   ।   ह॒   ।    & TS\_5.4.9.4       \\
    
    \hline
        
    207 & अ॒भीति॑   ।   मृ॒श॒ति॒   ।   उपेति॑   ।   ए॒न॒म्   ।    & TS\_5.4.10.3       \\
    
    \hline
        
    208 & अ॒भीति॑   ।   वै   ।   ए॒षः   ।   ए॒तान्   ।    & TS\_2.2.2.5       \\
    
    \hline
        
    209 & अ॒भ्यारो॑ह॒तीत्य॑भि{-}आरो॑हति   ।   शरी॑रम्   ।   वै   ।   ए॒तत्   ।    & TS\_5.6.6.4       \\
    
    \hline
        
    210 & अ॒भ्या॒रोह॒तीत्य॑भि{-}आ॒रोह॑ति   ।   यथा᳚   ।   खलु॑   ।   वै   ।    & TS\_2.5.5.6       \\
    
    \hline
        
    211 & अ॒भ्य॒ङ्क्त इत्य॑भि{-}अ॒ङ्क्ते   ।   तस्यै᳚   ।   दु॒श्चर्मेति॑ दुः{-}चर्मा᳚   ।   या   ।    & TS\_2.5.1.7       \\
    
    \hline
        
    212 & अ॒भ॒व॒न्न्   ।   सः   ।   प॒र॒मे॒ष्ठी   ।   प्र॒जाप॑ति॒मिति॑ प्र॒जा{-}प॒ति॒म्   ।    & TS\_5.7.5.5       \\
    
    \hline
        
    213 & अ॒मा॒वा॒स्या॑या॒मित्य॑मा{-}वा॒स्या॑याम्   ।   प्या॒य॒य॒न्ति॒   ।   तस्मा᳚त्   ।   वार्त्र॑घ्नी॒ इति॒ वार्त्र॑{-}घ्नी॒   ।    & TS\_2.5.2.5       \\
    
    \hline
        
    214 & अ॒मित्र᳚म्   ।   अ॒र्द॒य॒   ।   कु॒वित्   ।   स्विति॑   ।    & TS\_2.6.11.3       \\
    
    \hline
        
    215 & अ॒मुष्मिन्न्॑   ।   लो॒के   ।   उपेति॑   ।   ति॒ष्ठे॒ते॒ इति॑   ।    & TS\_5.2.7.5       \\
    
    \hline
        
    216 & अ॒मुष्मिन्न्॑   ।   लो॒के   ।   ने॒नी॒ये॒र॒न्न्   ।   यत्   ।    & TS\_3.3.8.4       \\
    
    \hline
        
    217 & अ॒मूम्   ।   च॒   ।   उपेति॑   ।   द॒धा॒ति॒   ।    & TS\_5.6.8.6       \\
    
    \hline
        
    218 & अ॒मृत᳚म्   ।   हिर॑ण्यम्   ।   प्रा॒णेष्विति॑ प्र{-}अ॒नेषु॑   ।   हि॒र॒ण्य॒श॒ल्कानिति॑ हिरण्य{-}श॒ल्कान्   ।    & TS\_5.2.9.3       \\
    
    \hline
        
    219 & अ॒म॒नसे᳚   ।   स्वाहा᳚   ।   रे॒त॒स्विने᳚   ।   स्वाहा᳚   ।    & TS\_7.5.12.2       \\
    
    \hline
        
    220 & अ॒यम्   ।   पु॒रः   ।   भुवः॑   ।   तस्य॑   ।    & TS\_4.3.2.1       \\
    
    \hline
        
    221 & अ॒यम्   ।   पु॒रः   ।   हरि॑केश॒ इति॒ हरि॑{-}के॒शः॒   ।   सूर्य॑रश्मि॒रिति॒ सूर्य॑{-}र॒श्मिः॒   ।    & TS\_4.4.3.1       \\
    
    \hline
        
    222 & अ॒यम्   ।   वा॒म्   ।   मि॒त्रा॒व॒रु॒णेति॑ मित्रा{-}व॒रु॒णा॒   ।   सु॒तः   ।    & TS\_1.4.5.1       \\
    
    \hline
        
    223 & अ॒यम्   ।   वे॒नः   ।   चो॒द॒य॒त्॒   ।   पृश्नि॑गर्भा॒ इति॒ पृश्नि॑{-}ग॒र्भाः॒   ।    & TS\_1.4.8.1       \\
    
    \hline
        
    224 & अ॒य॒स्मय᳚म्   ।   वीति॑   ।   चृ॒त॒   ।   ब॒न्धम्   ।    & TS\_4.2.5.3       \\
    
    \hline
        
    225 & अ॒रे॒पसौ᳚   ।   मा   ।   य॒ज्ञ्म्   ।   हिꣳ॒॒सि॒ष्ट॒म्   ।    & TS\_4.2.5.2       \\
    
    \hline
        
    226 & अ॒र्क्ये॑ण   ।   वै   ।   स॒ह॒स्र॒श इति॑ सहस्र{-}शः   ।   प्र॒जाप॑ति॒रिति॑ प्र॒जा{-}प॒तिः॒   ।    & TS\_7.5.9.1       \\
    
    \hline
        
    227 & अ॒र्थेत॒ इत्य॑र्थ{-}इतः॑   ।   स्थ॒   ।   अ॒पाम्   ।   पतिः॑   ।    & TS\_1.8.11.1       \\
    
    \hline
        
    228 & अ॒र्द्धि   ।   अवाङ्॑   ।   प्रा॒ण इति॑ प्र{-}अ॒नः   ।   अ॒न्येषा᳚म्   ।    & TS\_6.5.2.2       \\
    
    \hline
        
    229 & अ॒र्य॒मा   ।   दे॒वता᳚   ।   फल्गु॑नी॒ इति॑   ।   नक्ष॑त्रम्   ।    & TS\_4.4.10.2       \\
    
    \hline
        
    230 & अ॒र्य॒म्णे   ।   च॒रुम्   ।   निरिति॑   ।   व॒पे॒त्   ।    & TS\_2.3.4.1       \\
    
    \hline
        
    231 & अ॒र्वाङ्   ।   य॒ज्ञ्ः   ।   समिति॑   ।   क्रा॒म॒तु॒   ।    & TS\_7.3.11.1       \\
    
    \hline
        
    232 & अ॒र॒ङ्ग॒म इत्य॑रं{-}ग॒मः   ।   एति॑   ।   मा॒   ।   प्रा॒णेनेति॑ प्र{-}अ॒नेन॑   ।    & TS\_5.6.1.4       \\
    
    \hline
        
    233 & अ॒ल॒जः   ।   आ॒न्त॒रि॒क्षः   ।   उ॒द्रः   ।   म॒द्गुः   ।    & TS\_5.5.20.1       \\
    
    \hline
        
    234 & अ॒वार्च्छ॒तीत्य॑व{-}ऋच्छ॑ति   ।   ए॒वम्   ।   अवेति॑   ।   आ॒र॒म्   ।    & TS\_2.6.3.4       \\
    
    \hline
        
    235 & अ॒व॒प॒त्   ।   तेन॑   ।   ए॒व   ।   ए॒न॒म्   ।    & TS\_2.3.4.3       \\
    
    \hline
        
    236 & अ॒व॒भृ॒थ॒य॒जूꣳषीत्य॑वभृथ{-}य॒जूꣳषि॑   ।   जु॒हो॒ति॒   ।   यत्   ।   ए॒व   ।    & TS\_6.6.3.1       \\
    
    \hline
        
    237 & अ॒व॒रुद्ध्येत्य॑व{-}रुद्ध्य॑   ।   चि॒नु॒ते॒   ।   ए॒ताः   ।   वै   ।    & TS\_5.7.10.2       \\
    
    \hline
        
    238 & अ॒श्रा॒म्य॒न्न्   ।   ते   ।   तपः॑   ।   अ॒त॒प्य॒न्त॒   ।    & TS\_5.3.5.4       \\
    
    \hline
        
    239 & अ॒श्व॒मे॒ध इत्य॑श्व{-}मे॒धः   ।   यत्   ।   ए॒ताः   ।   आहु॑ती॒रित्या{-}हु॒तीः॒   ।    & TS\_5.7.5.3       \\
    
    \hline
        
    240 & अ॒ष्टाक्ष॒रेत्य॒ष्टा{-}अ॒क्ष॒रा॒   ।   गा॒य॒त्री   ।   गा॒य॒त्रः   ।   अ॒ग्निः   ।    & TS\_5.4.4.4       \\
    
    \hline
        
    241 & अ॒ष्टाक॑पाल॒मित्य॒ष्टा{-}क॒पा॒ल॒म्   ।   निरिति॑   ।   व॒पे॒त्   ।   तेज॑स्काम॒ इति॒ तेजः॑{-}का॒मः॒   ।    & TS\_2.2.3.4       \\
    
    \hline
        
    242 & अ॒ष्टु॒   ।   इति॑   ।   आ॒ह॒   ।   मनः॑   ।    & TS\_6.4.5.5       \\
    
    \hline
        
    243 & अ॒सा॒नि॒   ।   इति॑   ।   वै   ।   अ॒ग्निः   ।    & TS\_5.5.2.2       \\
    
    \hline
        
    244 & अ॒सि॒   ।   इति॑   ।   प॒श्चात्   ।   प्र॒ति॒ष्ठेति॑ प्रति{-}स्था   ।    & TS\_5.3.4.4       \\
    
    \hline
        
    245 & अ॒सि॒   ।   इति॑   ।   समिति॑   ।   ह॒न्ति॒   ।    & TS\_6.2.7.4       \\
    
    \hline
        
    246 & अ॒सि॒   ।   ऊर्ज᳚म्   ।   मह्य᳚म्   ।   श॒स्त्रम्   ।    & TS\_3.2.7.2       \\
    
    \hline
        
    247 & अ॒सि॒   ।   जुष्टि᳚म्   ।   ते॒   ।   ग॒मे॒य॒म्   ।    & TS\_1.6.3.3       \\
    
    \hline
        
    248 & अ॒सि॒   ।   पृ॒थि॒व्याः   ।   स॒धस्था॒दिति॑ स॒ध{-}स्था॒त्   ।   अ॒ग्निम्   ।    & TS\_4.1.1.4       \\
    
    \hline
        
    249 & अ॒सि॒   ।   प्र॒जाप॑तय॒ इति॑ प्र॒जा{-}प॒त॒ये॒   ।   त्वा॒   ।   ज्योति॑ष्मते   ।    & TS\_3.5.9.2       \\
    
    \hline
        
    250 & अ॒सि॒   ।   य॒ज्ञिया᳚   ।   अ॒सि॒   ।   क्ष॒त्रिया᳚   ।    & TS\_1.2.4.2       \\
    
    \hline
        
    251 & अ॒सि॒   ।   रा॒य॒स्पो॒ष॒वनि॒रिति॑ रायस्पोष{-}वनिः॑   ।   स्वाहा᳚   ।   सिꣳ॒॒हीः   ।    & TS\_1.2.12.3       \\
    
    \hline
        
    252 & अ॒सि॒   ।   सा   ।   मा॒   ।   जि॒न्व॒   ।    & TS\_4.2.9.3       \\
    
    \hline
        
    253 & अ॒सि॒   ।   स्त॒न॒यि॒त्नु॒सनि॒रिति॑ स्तनयित्नु{-}सनिः॑   ।   अ॒सि॒   ।   वृ॒ष्टि॒सनि॒रिति॑ वृष्टि{-}सनिः॑   ।    & TS\_4.4.6.2       \\
    
    \hline
        
    254 & अ॒सि॒   ।   स॒त्यशु॑ष्म॒मिति॑ स॒त्य{-}शु॒ष्म॒म्   ।   अ॒सि॒   ।   स॒त्येन॑   ।    & TS\_1.6.1.2       \\
    
    \hline
        
    255 & अ॒सु॒र्य᳚म्   ।   ऋ॒तावा॑न॒ इत्यृ॒त{-}वा॒नः॒   ।   चय॑मानाः   ।   ऋ॒णानि॑   ।    & TS\_2.1.11.5       \\
    
    \hline
        
    256 & अ॒सौ   ।   आ॒दि॒त्यः   ।   अ॒स्मिन्न्   ।   लो॒के   ।    & TS\_7.3.10.1       \\
    
    \hline
        
    257 & अ॒सौ   ।   आ॒दि॒त्यः   ।   न   ।   वीति॑   ।    & TS\_2.1.4.1 TS\_2.1.8.1 TS\_2.2.10.1       \\
    
    \hline
        
    258 & अ॒स्थानि॑   ।   अशा॑तयत   ।   तत्   ।   पूतु॑द्रु   ।    & TS\_6.2.8.6       \\
    
    \hline
        
    259 & अ॒स्माक᳚म्   ।   वी॒राः   ।   उत्त॑र॒ इत्युत्{-}त॒रे॒   ।   भ॒व॒न्तु॒   ।    & TS\_4.6.4.4       \\
    
    \hline
        
    260 & अ॒स्मि॒न्न्   ।   चक्षुः॑   ।   ध॒त्तः॒   ।   चक्षु॑ष्मान्   ।    & TS\_2.2.9.4       \\
    
    \hline
        
    261 & अ॒स्मि॒न्न्   ।   द॒धा॒ति॒   ।   अग्ने᳚   ।   यत्   ।    & TS\_5.3.11.3       \\
    
    \hline
        
    262 & अ॒स्मि॒न्न्   ।   ह॒व्या   ।   जु॒हो॒त॒न॒   ।   प्रप्रेति॒ प्र{-}प्र॒   ।    & TS\_4.2.3.2       \\
    
    \hline
        
    263 & अ॒स्मै॒   ।   अनु॑वर्त्मान॒मित्यनु॑{-}व॒र्त्मा॒न॒म्   ।   क॒रो॒ति॒   ।   ए॒ताम्   ।    & TS\_2.2.11.2       \\
    
    \hline
        
    264 & अ॒स्मै॒   ।   इ॒मे   ।   लो॒काः   ।   ऊर्ज᳚म्   ।    & TS\_7.2.4.2       \\
    
    \hline
        
    265 & अ॒स्मै॒   ।   दान॑कामा॒ इति॒ दान॑{-}का॒माः॒   ।   प्र॒जा इति॑ प्र{-}जाः   ।   क॒रो॒ति॒   ।    & TS\_2.3.4.2       \\
    
    \hline
        
    266 & अ॒स्मै॒   ।   यः   ।   आ॒सा॒म्   ।   ए॒वम्   ।    & TS\_5.2.10.6       \\
    
    \hline
        
    267 & अ॒स्मै॒   ।   स॒वान्   ।   प्रेति॑   ।   य॒च्छ॒न्ति॒   ।    & TS\_5.6.3.2       \\
    
    \hline
        
    268 & अ॒स्य   ।   जग॑तः   ।   सु॒व॒र्दृश॒मिति॑ सुवः{-}दृश᳚म्   ।   ईशा॑नम्   ।    & TS\_2.4.14.3       \\
    
    \hline
        
    269 & अ॒स्यै॒   ।   शरी॑रम्   ।   गाम्   ।   वाव   ।    & TS\_1.7.2.2       \\
    
    \hline
        
    270 & अ॒स्य॒   ।   भ्रातृ॑व्यः   ।   भ॒व॒ति॒   ।   तौ   ।    & TS\_6.4.10.2       \\
    
    \hline
        
    271 & अ॒स्य॒   ।   सु॒म॒ताविति॑ सु{-}म॒तौ   ।   यथा᳚   ।   यू॒यम्   ।    & TS\_4.3.11.4       \\
    
    \hline
        
    272 & अ॒हम्   ।   दे॒व॒य॒ज्ययेति॑ देव{-}य॒ज्यया᳚   ।   इ॒न्द्रि॒या॒वी   ।   भू॒या॒स॒म्   ।    & TS\_1.6.11.7       \\
    
    \hline
        
    273 & अ॒हम्   ।   सुवः॑   ।   यन्तः॑   ।   न   ।    & TS\_4.6.5.2       \\
    
    \hline
        
    274 & अ॒हु॒ताद॒ इत्य॑हुत{-}अदः॑   ।   अ॒न्ये   ।   तान्   ।   अ॒ग्नि॒चिदित्य॑ग्नि{-}चित्   ।    & TS\_5.4.5.2       \\
    
    \hline
        
    275 & अ॒हो॒रा॒त्राभ्या॒मित्य॑हः{-}रा॒त्राभ्या᳚म्   ।   प॒र्जन्य᳚म्   ।   व॒र्.॒ष॒य॒तः॒   ।   अ॒ग्नये᳚   ।    & TS\_2.4.10.2       \\
    
    \hline
        
    276 & अ॒ह॒र॒त्   ।   तेन॑   ।   अ॒य॒ज॒त॒   ।   ततः॑   ।    & TS\_7.1.10.3       \\
    
    \hline
        
    277 & अꣳसा᳚भ्याम्   ।   स्वाहा᳚   ।   दो॒षभ्या॒मिति॑ दो॒ष{-}भ्या॒म्   ।   स्वाहा᳚   ।    & TS\_7.3.16.2       \\
    
    \hline
        
    278 & अꣳ॒॒शुः   ।   च॒   ।   मे॒   ।   र॒श्मिः   ।    & TS\_4.7.7.1       \\
    
    \hline
        
    279 & अꣳ॒॒शुना᳚   ।   ते॒   ।   अꣳ॒॒शुः   ।   पृ॒च्य॒ता॒म्   ।    & TS\_1.2.6.1       \\
    
    \hline
        
    280 & अꣳ॒॒शुरꣳ॑शु॒रित्यꣳ॒॒शुः{-}अꣳ॒॒शुः॒   ।   ते॒   ।   दे॒व॒   ।   सो॒म॒   ।    & TS\_1.2.11.1       \\
    
    \hline
        
    281 & अꣳ॒॒हो॒मुच॒ इत्यꣳ॑हः{-}मुचे᳚   ।   पु॒रो॒डाश᳚म्   ।   एका॑दशकपाल॒मित्येका॑दश{-}क॒पा॒ल॒म्   ।   निरिति॑   ।    & TS\_2.2.7.4       \\
    
    \hline
        
    282 & अꣳ॒॒हो॒मुच॒मित्यꣳ॑हः{-}मुच᳚म्   ।   वृ॒ष॒भम्   ।   य॒ज्ञिया॑नाम्   ।   अ॒पाम्   ।    & TS\_1.6.12.4       \\
    
    \hline
        
    283 & आकू॑ति॒मित्या{-}कू॒ति॒म्   ।   अ॒ग्निम्   ।   प्र॒युज॒मिति॑ प्र{-}युज᳚म्   ।   स्वाहा᳚   ।    & TS\_4.1.9.1       \\
    
    \hline
        
    284 & आकू᳚त्या॒ इत्या{-}कू॒त्यै॒   ।   प्र॒युज॒ इति॑ प्र{-}युजे᳚   ।   अ॒ग्नये᳚   ।   स्वाहा᳚   ।    & TS\_1.2.2.1       \\
    
    \hline
        
    285 & आगः॑   ।   कृ॒धि   ।   स्विति॑   ।   अ॒स्मान्   ।    & TS\_4.7.15.7       \\
    
    \hline
        
    286 & आपः॑   ।   उ॒न्द॒न्तु॒   ।   जी॒वसे᳚   ।   दी॒र्घा॒यु॒त्वायेति॑ दीर्घायु{-}त्वाय॑   ।    & TS\_1.2.1.1       \\
    
    \hline
        
    287 & आपः॑   ।   दे॒वीः   ।   बृ॒ह॒तीः   ।   वि॒श्वश॑भुंव॒ इति॑ वि॒श्व{-}श॒भुं॒वः॒   ।    & TS\_6.1.2.3       \\
    
    \hline
        
    288 & आपः॑   ।   वरु॑णस्य   ।   पत्न॑यः   ।   आ॒स॒न्न्   ।    & TS\_5.5.4.1       \\
    
    \hline
        
    289 & आपः॑   ।   वै   ।   इ॒दम्   ।   अग्रे᳚   ।    & TS\_7.1.5.1       \\
    
    \hline
        
    290 & आपः॑   ।   वै   ।   सर्वाः᳚   ।   दे॒वताः᳚   ।    & TS\_5.7.9.3       \\
    
    \hline
        
    291 & आप्त्यै᳚   ।   दे॒वाः   ।   वै   ।   यत्   ।    & TS\_1.7.3.3       \\
    
    \hline
        
    292 & आप्त्यै᳚   ।   न्यू॑न॒येति॒ नि{-}ऊ॒न॒या॒   ।   जु॒हो॒ति॒   ।   न्यू॑ना॒दिति॒ नि{-}ऊ॒ना॒त्   ।    & TS\_5.4.7.6       \\
    
    \hline
        
    293 & आप्त्यै᳚   ।   प॒ञ्चभि॒रिति॑ प॒ञ्च{-}भिः॒   ।   तिष्ठ॑न्तः   ।   स्तु॒व॒न्ति॒   ।    & TS\_7.5.8.4       \\
    
    \hline
        
    294 & आप॑ति॒रित्या{-}प॒तिः॒   ।   प्रा॒णमिति॑ प्र{-}अ॒नम्   ।   ए॒व   ।   प्री॒णा॒ति॒   ।    & TS\_6.2.2.3       \\
    
    \hline
        
    295 & आम॑न॒स्येत्या{-}म॒न॒स्य॒   ।   दे॒वाः॒   ।   याः   ।   स्त्रियः॑   ।    & TS\_2.3.9.2       \\
    
    \hline
        
    296 & आयन्न्॑   ।   दक्ष᳚म्   ।   दधा॑नाः   ।   ज॒नय॑न्तीः   ।    & TS\_4.1.8.6       \\
    
    \hline
        
    297 & आयुः॑   ।   ते॒   ।   आ॒यु॒र्दा इत्या॑युः{-}दाः   ।   अ॒ग्ने॒   ।    & TS\_2.5.12.1       \\
    
    \hline
        
    298 & आयुः॑   ।   द॒धा॒ति॒   ।   पु॒रस्ता᳚त्   ।   उ॒क्थस्य॑   ।    & TS\_6.5.2.3       \\
    
    \hline
        
    299 & आयुः॑   ।   द॒धा॒ति॒   ।   सर्व᳚म्   ।   आयुः॑   ।    & TS\_2.2.3.3       \\
    
    \hline
        
    300 & आयुः॑   ।   वै   ।   ए॒तत्   ।   य॒ज्ञ्स्य॑   ।    & TS\_6.5.2.1       \\
    
    \hline
        
    301 & आयु॑ष्मन्तः   ।   स॒हभ॑क्षा॒ इति॑ स॒ह{-}भ॒क्षाः॒   ।   स्या॒म॒   ।   विश्वे᳚   ।    & TS\_3.1.9.2       \\
    
    \hline
        
    302 & आय॑ना॒येत्या᳚{-}अय॑नाय   ।   स्वाहा᳚   ।   प्राय॑णा॒येति॑ प्र{-}अय॑नाय   ।   स्वाहा᳚   ।    & TS\_7.1.13.1       \\
    
    \hline
        
    303 & आर्ति᳚म्   ।   एति॑   ।   ऋ॒च्छे॒त्   ।   यत्   ।    & TS\_2.6.8.2       \\
    
    \hline
        
    304 & आसन्न्॑   ।   तानि॑   ।   तेन॑   ।   प्रेति॑   ।    & TS\_2.4.1.3       \\
    
    \hline
        
    305 & आसी᳚त्   ।   यदि॑   ।   वा॒   ।   ताव॑त्   ।    & TS\_2.4.12.2       \\
    
    \hline
        
    306 & आसी᳚त्   ।   सः   ।   क॒पिञ्ज॑लः   ।   अ॒भ॒व॒त्   ।    & TS\_2.5.1.2       \\
    
    \hline
        
    307 & आस॑त   ।   इन्धा॑नाः   ।   अ॒ग्निम्   ।   सुवः॑   ।    & TS\_4.7.13.3       \\
    
    \hline
        
    308 & आहि॑ताग्ने॒रित्याहि॑त{-}अ॒ग्नेः॒   ।   आ॒शीरित्या᳚{-}शीः   ।   यत्   ।   अ॒ग्निम्   ।    & TS\_1.5.9.7       \\
    
    \hline
        
    309 & आहु॑तय॒ इत्या{-}हु॒त॒यः॒   ।   वै   ।   ए॒तस्य॑   ।   अक्लृ॑प्ताः   ।    & TS\_3.4.8.3       \\
    
    \hline
        
    310 & आहु॑त॒मित्या{-}हु॒त॒म्   ।   प्र॒जामिति॑ प्र{-}जाम्   ।   दे॒वि॒   ।   दि॒दि॒ड्ढि॒   ।    & TS\_3.1.11.4       \\
    
    \hline
        
    311 & आ॒ग्ना॒वै॒ष्ण॒वमित्या᳚ग्ना{-}वै॒ष्ण॒वम्   ।   एका॑दशकपाल॒मित्येका॑दश{-}क॒पा॒ल॒म्   ।   निरिति॑   ।   व॒पे॒त्   ।    & TS\_2.2.9.1 TS\_2.2.9.6       \\
    
    \hline
        
    312 & आ॒ग्ने॒यः   ।   अ॒ष्टाक॑पाल॒ इत्य॒ष्टा{-}क॒पा॒लः॒   ।   सौ॒म्यः   ।   च॒रुः   ।    & TS\_7.5.21.1       \\
    
    \hline
        
    313 & आ॒ग्ने॒यः   ।   कृ॒ष्णग्री॑व॒ इति॑ कृ॒ष्ण{-}ग्री॒वः॒   ।   सा॒र॒स्व॒ती   ।   मे॒षी   ।    & TS\_5.5.22.1       \\
    
    \hline
        
    314 & आ॒ग्ने॒यम्   ।   अ॒ष्टाक॑पाल॒मित्य॒ष्टा{-}क॒पा॒ल॒म्   ।   निरिति॑   ।   व॒प॒ति॒   ।    & TS\_1.8.2.1 TS\_1.8.17.1 TS\_1.8.19.1 TS\_1.8.20.1       \\
    
    \hline
        
    315 & आ॒ग्ने॒यीम्   ।   कृ॒ष्ण॒ग्री॒वीमिति॑ कृष्ण{-}ग्री॒वीम्   ।   सꣳ॒॒हि॒तामिति॑ सं{-}हि॒ताम्   ।   ऐ॒न्द्रीम्   ।    & TS\_2.1.2.5       \\
    
    \hline
        
    316 & आ॒घा॒रमित्या᳚{-}घा॒रम्   ।   आ॒घार्येत्या᳚{-}घार्य॑   ।   ध्रु॒वाम्   ।   समिति॑   ।    & TS\_2.5.11.8       \\
    
    \hline
        
    317 & आ॒ता॒ना इत्या᳚{-}ता॒नाः   ।   तेभ्यः॑   ।   ए॒व   ।   नमः॑   ।    & TS\_6.3.8.4       \\
    
    \hline
        
    318 & आ॒त्मान᳚म्   ।   एति॑   ।   अ॒प्री॒णी॒त॒   ।   यत्   ।    & TS\_5.1.8.4       \\
    
    \hline
        
    319 & आ॒त्मान᳚म्   ।   ए॒व   ।   य॒ज्ञ्॒य॒श॒सेनेति॑ यज्ञ्{-}य॒श॒सेन॑   ।   अ॒र्प॒य॒ति॒   ।    & TS\_6.5.1.5       \\
    
    \hline
        
    320 & आ॒दायेत्या᳚{-}दाय॑   ।   तत्   ।   ए॒भ्यः॒   ।   प्रेति॑   ।    & TS\_3.3.7.2       \\
    
    \hline
        
    321 & आ॒दि॒त्यः   ।   तान्   ।   ऋष॑यः   ।   अ॒ब्रु॒व॒न्न्   ।    & TS\_5.7.5.7       \\
    
    \hline
        
    322 & आ॒दि॒त्यः   ।   वै   ।   अ॒स्मात्   ।   लो॒कात्   ।    & TS\_1.5.9.4       \\
    
    \hline
        
    323 & आ॒दि॒त्यम्   ।   गर्भ᳚म्   ।   पय॑सा   ।   स॒म॒ञ्जन्निति॑ सं{-}अ॒ञ्जन्न्   ।    & TS\_4.2.10.1       \\
    
    \hline
        
    324 & आ॒दि॒त्यम्   ।   च॒रुम्   ।   निरिति॑   ।   अ॒व॒प॒न्न्   ।    & TS\_2.3.5.3       \\
    
    \hline
        
    325 & आ॒दि॒त्यम्   ।   च॒रुम्   ।   निरिति॑   ।   व॒पे॒त्   ।    & TS\_2.2.6.1       \\
    
    \hline
        
    326 & आ॒दि॒त्याः   ।   अ॒का॒म॒य॒न्त॒   ।   उ॒भयोः᳚   ।   लो॒कयोः᳚   ।    & TS\_7.3.4.1       \\
    
    \hline
        
    327 & आ॒दि॒त्याः   ।   अ॒का॒म॒य॒न्त॒   ।   सु॒व॒र्गमिति॑ सुवः{-}गम्   ।   लो॒कम्   ।    & TS\_7.4.6.1       \\
    
    \hline
        
    328 & आ॒दि॒त्याः   ।   ते॒   ।   दे॒वाः   ।   अधि॑पतय॒ इत्यधि॑{-}प॒त॒यः॒   ।    & TS\_4.4.2.2       \\
    
    \hline
        
    329 & आ॒दि॒त्याः   ।   त्वा॒   ।   कृ॒ण्व॒न्तु॒   ।   जाग॑तेन   ।    & TS\_4.1.5.4       \\
    
    \hline
        
    330 & आ॒दि॒त्याना᳚म्   ।   ते   ।   ते॒   ।   अधि॑पतय॒ इत्यधि॑{-}प॒त॒यः॒   ।    & TS\_4.4.11.3       \\
    
    \hline
        
    331 & आ॒दि॒त्याना᳚म्   ।   प्रा॒ची॒न॒ता॒न इति॑ प्राचीन{-}ता॒नः   ।   विश्वे॑षाम्   ।   दे॒वाना᳚म्   ।    & TS\_6.1.1.4       \\
    
    \hline
        
    332 & आ॒दि॒त्येभ्यः॑   ।   भुव॑द्वद्भ्य॒ इति॒ भुव॑द्वत्{-}भ्यः॒   ।   च॒रुम्   ।   निरिति॑   ।    & TS\_2.3.1.1       \\
    
    \hline
        
    333 & आ॒न॒न्दमित्या᳚{-}न॒न्दम्   ।   न॒न्दथु॑ना   ।   काम᳚म्   ।   प्र॒त्या॒साभ्या॒मिति॑ प्रति{-}आ॒साभ्या᳚म्   ।    & TS\_5.7.19.1       \\
    
    \hline
        
    334 & आ॒यत॑न॒ इत्या᳚{-}यत॑ने   ।   श॒म॒य॒ति॒   ।   अ॒भि॒चर॒तेत्य॑भि{-}चर॑ता   ।   प्र॒ति॒लो॒ममिति॑ प्रति{-}लो॒मम्   ।    & TS\_3.4.8.5       \\
    
    \hline
        
    335 & आ॒यु॒र्दा इत्या॑युः{-}दाः   ।   अ॒ग्ने॒   ।   ह॒विषः॑   ।   जु॒षा॒णः   ।    & TS\_3.3.8.1       \\
    
    \hline
        
    336 & आ॒र्त॒वाः   ।   उ॒च्य॒न्ते॒   ।   ये   ।   ए॒वम्   ।    & TS\_7.2.6.2       \\
    
    \hline
        
    337 & आ॒र॒ण्याः   ।   उ॒भयी॑षाम्   ।   अव॑रुद्ध्या॒ इत्यव॑{-}रु॒द्ध्यै॒   ।   अन्न॑स्यान्न॒स्येत्यन्न॑स्य{-}अ॒न्न॒स्य॒   ।    & TS\_5.4.9.2       \\
    
    \hline
        
    338 & आ॒वत्   ।   अधीति॑   ।   दे॒वाः   ।   इ॒ज्या॒न्तै॒   ।    & TS\_2.5.6.6       \\
    
    \hline
        
    339 & आ॒शिर᳚म्   ।   अवेति॑   ।   न॒य॒ति॒   ।   स॒शु॒क्र॒त्वायेति॑ सशुक्र{-}त्वाय॑   ।    & TS\_6.1.6.5       \\
    
    \hline
        
    340 & आ॒शीरित्या᳚{-}शीः   ।   ग॒च्छ॒ति॒   ।   यान्   ।   का॒मये॑त   ।    & TS\_1.6.10.5       \\
    
    \hline
        
    341 & आ॒शुः   ।   त्रि॒वृदिति॑ त्रि{-}वृत्   ।   भा॒न्तः   ।   प॒ञ्च॒द॒श इति॑ पञ्च{-}द॒शः   ।    & TS\_4.3.8.1       \\
    
    \hline
        
    342 & आ॒शुः   ।   शिशा॑नः   ।   वृ॒ष॒भः   ।   न   ।    & TS\_4.6.4.1       \\
    
    \hline
        
    343 & आ॒श्वि॒नम्   ।   धू॒म्रल॑लाम॒मिति॑ धू॒म्र{-}ल॒ला॒म॒म्   ।   एति॑   ।   ल॒भे॒त॒   ।    & TS\_2.1.10.1       \\
    
    \hline
        
    344 & आ॒सि॒ष्य॒ते   ।   स्वाहा᳚   ।   आसी॑नाय   ।   स्वाहा᳚   ।    & TS\_7.1.19.2       \\
    
    \hline
        
    345 & आ॒सी॒त्   ।   प॒ञ्च॒द॒शभि॒रिति॑ पञ्चद॒श {-}भिः॒   ।   अ॒स्तु॒व॒त॒   ।   क्ष॒त्रम्   ।    & TS\_4.3.10.2       \\
    
    \hline
        
    346 & आ॒स्ता॒म्   ।   ते   ।   दे॒वाः   ।   अ॒ब्रु॒व॒न्न्   ।    & TS\_2.5.8.2       \\
    
    \hline
        
    347 & आ॒हुः   ।   ते॒   ।   त्रीणि॑   ।   दि॒वि   ।    & TS\_4.6.7.2       \\
    
    \hline
        
    348 & आ॒हुः   ।   यस्य॑   ।   इ॒माः   ।   प्र॒दिश॒ इति॑ प्र{-}दिशः॑   ।    & TS\_4.1.8.5       \\
    
    \hline
        
    349 & आ॒हुः॒   ।   प्र॒जा इति॑ प्र{-}जाः   ।   निर्भ॑क्ता॒ इति॒ निः{-}भ॒क्ताः॒   ।   अ॒नु॒त॒प्यमा॑ना॒ इत्य॑नु{-}त॒प्यमा॑नाः   ।    & TS\_3.2.8.2       \\
    
    \hline
        
    350 & आ॒ह॒   ।   अग्नी॒दित्यग्नि॑{-}इ॒त्   ।   अ॒ग्नीन्   ।   वीति॑   ।    & TS\_6.3.1.2       \\
    
    \hline
        
    351 & आ॒ह॒   ।   अ॒यम्   ।   वै   ।   प्र॒जाप॑ते॒रिति॑ प्र॒जा{-}प॒तेः॒   ।    & TS\_1.7.5.2       \\
    
    \hline
        
    352 & आ॒ह॒   ।   इति॑   ।   आ॒ह॒   ।   प्रसू᳚त्या॒ इति॒ प्र{-}सू॒त्यै॒   ।    & TS\_2.6.9.3       \\
    
    \hline
        
    353 & आ॒ह॒   ।   ऊर्जः॑   ।   हि   ।   ए॒ताः   ।    & TS\_1.5.8.2       \\
    
    \hline
        
    354 & आ॒ह॒   ।   ए॒षः   ।   वै   ।   सोम॑स्य   ।    & TS\_6.4.4.4       \\
    
    \hline
        
    355 & आ॒ह॒   ।   दे॒वाः   ।   वै   ।   क्षयः॑   ।    & TS\_3.5.2.2       \\
    
    \hline
        
    356 & आ॒ह॒   ।   द्यावा॑पृथि॒वीभ्या॒मिति॒ द्यावा᳚{-}पृ॒थि॒वीभ्या᳚म्   ।   ए॒व   ।   य॒ज्ञ्स्य॑   ।    & TS\_3.4.1.3       \\
    
    \hline
        
    357 & आ॒ह॒   ।   प्रा॒णमिति॑ प्र{-}अ॒नम्   ।   ए॒व   ।   अ॒स्मि॒न्न्   ।    & TS\_2.3.11.4       \\
    
    \hline
        
    358 & आ॒ह॒   ।   प्र॒जा इति॑ प्र{-}जाः   ।   वै   ।   इ॒न्द्रि॒यम्   ।    & TS\_6.5.8.4       \\
    
    \hline
        
    359 & आ॒ह॒   ।   ब्रह्म॑   ।   वै   ।   दे॒वाना᳚म्   ।    & TS\_6.3.6.2       \\
    
    \hline
        
    360 & आ॒ह॒   ।   यः   ।   ए॒व   ।   ए॒न॒म्   ।    & TS\_5.1.4.4       \\
    
    \hline
        
    361 & आ॒ह॒   ।   यत्   ।   ए॒व   ।   य॒ज्ञे   ।    & TS\_3.4.3.7       \\
    
    \hline
        
    362 & आ॒ह॒   ।   रास॑भः   ।   इति॑   ।   हि   ।    & TS\_5.1.5.7       \\
    
    \hline
        
    363 & आ॒ह॒   ।   समृ॑द्ध्या॒ इति॒ सं{-}ऋ॒द्ध्यै॒   ।   च॒तुभि॒रिति॑ च॒तुः{-}भिः॒   ।   अभ्रि᳚म्   ।    & TS\_5.1.1.4       \\
    
    \hline
        
    364 & इति॑   ।   अन्विति॑   ।   आ॒ह॒   ।   मासाः᳚   ।    & TS\_2.5.7.4       \\
    
    \hline
        
    365 & इति॑   ।   अ॒ब्र॒वी॒त्   ।   ते   ।   इन्द्रा॑य   ।    & TS\_2.4.2.2       \\
    
    \hline
        
    366 & इति॑   ।   आ॒शीर्प॑द॒येत्या॒शीः{-}प॒द॒या॒   ।   ऋ॒चा   ।   दक्षि॑णस्य   ।    & TS\_6.2.9.4       \\
    
    \hline
        
    367 & इति॑   ।   आ॒ह॒   ।   अन्न᳚म्   ।   वै   ।    & TS\_5.4.6.6       \\
    
    \hline
        
    368 & इति॑   ।   आ॒ह॒   ।   अ॒ग्निः   ।   वै   ।    & TS\_2.5.9.5 TS\_3.3.8.6       \\
    
    \hline
        
    369 & इति॑   ।   आ॒ह॒   ।   आ॒शिष॒मित्या᳚{-}शिष᳚म्   ।   ए॒व   ।    & TS\_6.2.2.6       \\
    
    \hline
        
    370 & इति॑   ।   आ॒ह॒   ।   तस्मा᳚त्   ।   अ॒ग्निः   ।    & TS\_5.1.7.4       \\
    
    \hline
        
    371 & इति॑   ।   आ॒ह॒   ।   तस्मा᳚त्   ।   परु॑षिपरु॒षीति॒ परु॑षि{-}प॒रु॒षि॒   ।    & TS\_5.5.6.3       \\
    
    \hline
        
    372 & इति॑   ।   आ॒ह॒   ।   दि॒वि   ।   हि   ।    & TS\_6.1.11.4       \\
    
    \hline
        
    373 & इति॑   ।   आ॒ह॒   ।   दे॒वाना᳚म्   ।   हि   ।    & TS\_6.2.2.4       \\
    
    \hline
        
    374 & इति॑   ।   आ॒ह॒   ।   पि॒तृ॒दे॒व॒त्येति॑ पितृ{-}दे॒व॒त्या᳚   ।   ए॒तेन॑   ।    & TS\_6.1.2.6       \\
    
    \hline
        
    375 & इति॑   ।   आ॒ह॒   ।   पि॒तॄन्   ।   ए॒व   ।    & TS\_3.5.2.4       \\
    
    \hline
        
    376 & इति॑   ।   आ॒ह॒   ।   प्र॒जाप॑ति॒मिति॑ प्र॒जा{-}प॒ति॒म्   ।   ए॒व   ।    & TS\_1.7.3.4       \\
    
    \hline
        
    377 & इति॑   ।   आ॒ह॒   ।   प्र॒जाप॑ति॒रिति॑ प्र॒जा{-}प॒तिः॒   ।   सर्वाः᳚   ।    & TS\_3.5.9.3       \\
    
    \hline
        
    378 & इति॑   ।   आ॒ह॒   ।   प्र॒जास्विति॑ प्र{-}जासु॑   ।   ए॒व   ।    & TS\_6.4.1.3       \\
    
    \hline
        
    379 & इति॑   ।   आ॒ह॒   ।   प्र॒जेति॑ प्र{-}जा   ।   वै   ।    & TS\_2.6.7.6       \\
    
    \hline
        
    380 & इति॑   ।   आ॒ह॒   ।   ब्रह्म॑   ।   वै   ।    & TS\_6.1.2.4       \\
    
    \hline
        
    381 & इति॑   ।   आ॒ह॒   ।   यत्   ।   ए॒व   ।    & TS\_6.1.8.3       \\
    
    \hline
        
    382 & इति॑   ।   आ॒ह॒   ।   य॒था॒य॒जुरिति॑ यथा{-}य॒जुः   ।   ए॒व   ।    & TS\_3.4.3.5       \\
    
    \hline
        
    383 & इति॑   ।   आ॒ह॒   ।   स्यो॒ना   ।   ए॒व   ।    & TS\_7.1.7.4       \\
    
    \hline
        
    384 & इति॑   ।   आ॒ह॒   ।   स॒प्तस॒प्तेति॑ स॒प्त{-}स॒प्त॒   ।   वै   ।    & TS\_1.5.4.4       \\
    
    \hline
        
    385 & इति॑   ।   इन्द्र᳚म्   ।   ए॒व   ।   दा॒तार᳚म्   ।    & TS\_2.2.8.4       \\
    
    \hline
        
    386 & इति॑   ।   इ॒माम्   ।   ए॒व   ।   इष्ट॑काम्   ।    & TS\_5.5.2.5       \\
    
    \hline
        
    387 & इति॑   ।   त्वा॒   ।   अ॒हम्   ।   परीति॑   ।    & TS\_4.2.5.4       \\
    
    \hline
        
    388 & इति॑   ।   ब्रू॒या॒त्   ।   अया॑तया॒मेत्यया॑त{-}या॒मा॒   ।   हि   ।    & TS\_2.6.3.2       \\
    
    \hline
        
    389 & इति॑   ।   मा॒   ।   क॒द्रूः   ।   अ॒वो॒च॒त्   ।    & TS\_6.1.6.2       \\
    
    \hline
        
    390 & इति॑   ।   यः   ।   अ॒प॒गु॒राता॒ इत्य॑प{-}गु॒रातै᳚   ।   श॒तेन॑   ।    & TS\_2.6.10.2       \\
    
    \hline
        
    391 & इति॑   ।   लो॒म॒तः   ।   तस्य॑   ।   मि॒मी॒त॒   ।    & TS\_6.1.9.3       \\
    
    \hline
        
    392 & इति॑   ।   वृ॒णी॒द्ध्वम्   ।   ह॒व्य॒वाह॑न॒मिति॑ हव्य{-}वाह॑नम्   ।   इति॑   ।    & TS\_2.5.8.7       \\
    
    \hline
        
    393 & इति॑   ।   सः   ।   अ॒ब्र॒वी॒त्   ।   वर᳚म्   ।    & TS\_6.4.7.2       \\
    
    \hline
        
    394 & इति॑   ।   स्वा॒हा॒का॒रेणेति॑ स्वाहा{-}का॒रेण॑   ।   प्र॒या॒जेष्विति॑ प्र{-}या॒जेषु॑   ।   य॒ज्ञ्म्   ।    & TS\_2.6.1.6       \\
    
    \hline
        
    395 & इत्   ।   अ॒नु॒म॒त॒ इत्य॑नु{-}म॒ते॒   ।   त्वम्   ।   वै॒श्वा॒न॒रः   ।    & TS\_4.7.15.6       \\
    
    \hline
        
    396 & इत्   ।   नु   ।   वा   ।   उप॑स्तीर्ण॒मियुप॑{-}स्ती॒र्ण॒म्   ।    & TS\_1.6.7.3       \\
    
    \hline
        
    397 & इन्द्रः॑   ।   पत्नि॑या   ।   मनु᳚म्   ।   अ॒या॒ज॒य॒त्   ।    & TS\_6.6.6.1       \\
    
    \hline
        
    398 & इन्द्रः॑   ।   म॒रुद्भि॒रिति॑ म॒रुत्{-}भिः॒   ।   सांॅवि॑द्ये॒नेति॒ सां{-}वि॒द्ये॒न॒   ।   माद्ध्य॑न्दिने   ।    & TS\_6.5.5.1       \\
    
    \hline
        
    399 & इन्द्रः॑   ।   वृ॒त्रम्   ।   अ॒ह॒न्न्   ।   तस्य॑   ।    & TS\_6.5.9.1       \\
    
    \hline
        
    400 & इन्द्रः॑   ।   वृ॒त्रम्   ।   अ॒ह॒न्न्   ।   सः   ।    & TS\_6.1.1.7       \\
    
    \hline
        
    401 & इन्द्रः॑   ।   वृ॒त्राय॑   ।   वज्र᳚म्   ।   उदिति॑   ।    & TS\_6.5.1.1       \\
    
    \hline
        
    402 & इन्द्रः॑   ।   वै   ।   शि॒थि॒लः   ।   इ॒व॒   ।    & TS\_7.3.7.1       \\
    
    \hline
        
    403 & इन्द्रः॑   ।   वै   ।   स॒दृङ्ङिति॑ स{-}दृङ्   ।   दे॒वता॑भिः   ।    & TS\_7.3.6.1       \\
    
    \hline
        
    404 & इन्द्रः॑   ।   व॒लस्य॑   ।   बिल᳚म्   ।   अपेति॑   ।    & TS\_2.1.5.1       \\
    
    \hline
        
    405 & इन्द्रः॑   ।   शत्रुः॑   ।   अ॒भ॒व॒त्   ।   सः   ।    & TS\_2.5.2.2       \\
    
    \hline
        
    406 & इन्द्रा॑य   ।   त्वा॒   ।   इन्द्र᳚म्   ।   जि॒न्व॒   ।    & TS\_5.3.6.2       \\
    
    \hline
        
    407 & इन्द्रा॑य   ।   राज्ञे᳚   ।   त्रयः॑   ।   शि॒ति॒पृ॒ष्ठा इति॑ शिति{-}पृ॒ष्ठाः   ।    & TS\_5.6.17.1       \\
    
    \hline
        
    408 & इन्द्रा॑य   ।   राज्ञे᳚   ।   सू॒क॒रः   ।   वरु॑णाय   ।    & TS\_5.5.11.1       \\
    
    \hline
        
    409 & इन्द्रा᳚ग्नी॒ इतीन्द्र॑{-}अ॒ग्नी॒   ।   अव्य॑थमानाम्   ।   इति॑   ।   स्व॒य॒मा॒तृ॒ण्णामिति॑ स्वयं{-}आ॒तृ॒ण्णाम्   ।    & TS\_5.3.2.1       \\
    
    \hline
        
    410 & इन्द्रा᳚ग्नी॒ इतीन्द्र॑{-}अ॒ग्नी॒   ।   अव्य॑थमानाम्   ।   इष्ट॑काम्   ।   दृꣳ॒॒ह॒त॒म्   ।    & TS\_4.3.6.1       \\
    
    \hline
        
    411 & इन्द्रा᳚ग्नी॒ इतीन्द्र॑{-}अ॒ग्नी॒   ।   एति॑   ।   ग॒त॒म्   ।   सु॒तम्   ।    & TS\_1.4.15.1       \\
    
    \hline
        
    412 & इन्द्रा᳚ग्नी॒ इतीन्द्र॑{-}अ॒ग्नी॒   ।   रो॒च॒ना   ।   दि॒वः   ।   परीति॑   ।    & TS\_4.2.11.1       \\
    
    \hline
        
    413 & इन्द्र॑   ।   म॒रु॒त्वः॒   ।   इ॒ह   ।   पा॒हि॒   ।    & TS\_1.4.18.1       \\
    
    \hline
        
    414 & इन्द्र॑स्य   ।   क्रो॒डः   ।   अदि॑त्यै   ।   पा॒ज॒स्य᳚म्   ।    & TS\_5.7.16.1       \\
    
    \hline
        
    415 & इन्द्र॑स्य   ।   वज्रः॑   ।   अ॒सि॒   ।   वार्त्र॑घ्न॒ इति॒ वार्त्र॑{-}घ्नः॒   ।    & TS\_1.8.15.1 TS\_5.7.3.1       \\
    
    \hline
        
    416 & इन्द्र᳚म्   ।   इत्   ।   हरी॒ इति॑   ।   व॒ह॒तः॒   ।    & TS\_1.4.38.1       \\
    
    \hline
        
    417 & इन्द्र᳚म्   ।   ए॒व   ।   स्वेन॑   ।   भा॒ग॒धेये॒नेति॑ भाग{-}धेये॑न   ।    & TS\_2.1.5.5       \\
    
    \hline
        
    418 & इन्द्र᳚म्   ।   वः॒   ।   वि॒श्वतः॑   ।   परीति॑   ।    & TS\_1.6.12.1 TS\_2.1.11.1       \\
    
    \hline
        
    419 & इन्द्र᳚म्   ।   वृ॒त्रम्   ।   ज॒घ्नि॒वाꣳस᳚म्   ।   मृधः॑   ।    & TS\_2.5.3.1       \\
    
    \hline
        
    420 & इषु॑मद्भ्य॒ इतीषु॑मत्{-}भ्यः॒   ।   ध॒न्वा॒विभ्य॒ इति॑ धन्वा॒वि{-}भ्यः॒   ।   च॒   ।   वः॒   ।    & TS\_4.5.3.2       \\
    
    \hline
        
    421 & इष्टि᳚म्   ।   अ॒प॒श्य॒न्न्   ।   आ॒ग्ना॒वै॒ष्ण॒वमित्या᳚ग्ना{-}वै॒ष्ण॒वम्   ।   एका॑दशकपाल॒मित्येका॑दश{-}क॒पा॒ला॒म्   ।    & TS\_2.5.4.2       \\
    
    \hline
        
    422 & इ॒च्छमा॑नः   ।   तस्मै᳚   ।   ए॒तत्   ।   भा॒ग॒धेय॒मिति॑ भाग{-}धेय᳚म्   ।    & TS\_5.7.4.2       \\
    
    \hline
        
    423 & इ॒डः   ।   य॒ज॒ति॒   ।   प॒शून्   ।   ए॒व   ।    & TS\_2.6.1.2       \\
    
    \hline
        
    424 & इ॒तासु॒रिती॒त{-}अ॒सुः॒   ।   भव॑ति   ।   जीव॑ति   ।   ए॒व   ।    & TS\_2.2.10.5       \\
    
    \hline
        
    425 & इ॒दम्   ।   अस्मि॑   ।   तत्   ।   ते॒   ।    & TS\_2.4.12.6       \\
    
    \hline
        
    426 & इ॒दम्   ।   अ॒हम्   ।   रक्ष॑सः   ।   ग्री॒वाः   ।    & TS\_6.1.8.4 TS\_6.2.10.2       \\
    
    \hline
        
    427 & इ॒दम्   ।   वा॒म्   ।   आ॒स्ये᳚   ।   ह॒विः   ।    & TS\_3.3.11.1       \\
    
    \hline
        
    428 & इ॒द्ध्मः   ।   च॒   ।   मे॒   ।   ब॒र्॒.हिः   ।    & TS\_4.7.8.1       \\
    
    \hline
        
    429 & इ॒द्ध्मे   ।   समिति॑   ।   न॒ह्ये॒त्   ।   गौः   ।    & TS\_2.2.8.2       \\
    
    \hline
        
    430 & इ॒न्द्रा॑य   ।   अन्वृ॑जव॒ इत्यनु॑{-}ऋ॒ज॒वे॒   ।   पु॒रो॒डाश᳚म्   ।   एका॑दशकपाल॒मित्येका॑दश{-}क॒पा॒ल॒म्   ।    & TS\_2.2.8.1       \\
    
    \hline
        
    431 & इ॒न्द्रा॒ग्निभ्या॒मिती᳚न्द्रा॒ग्नि{-}भ्या॒म्   ।   त्वा॒   ।   स॒युजेति॑ स{-}युजा᳚   ।   यु॒जा   ।    & TS\_4.4.5.1       \\
    
    \hline
        
    432 & इ॒न्द्रि॒यम्   ।   ए॒व   ।   माद्ध्य॑न्दिने   ।   सव॑ने   ।    & TS\_3.2.9.4       \\
    
    \hline
        
    433 & इ॒न्द्रि॒यम्   ।   ब्र॒ह्म॒व॒र्च॒समिति॑ ब्रह्म{-}व॒र्च॒सम्   ।   अ॒ध॒त्ता॒म्   ।   यः   ।    & TS\_2.3.3.2       \\
    
    \hline
        
    434 & इ॒न्द्रि॒येण॑   ।   अपेति॑   ।   क्रा॒म॒ति॒   ।   वरु॑णः   ।    & TS\_2.3.13.2       \\
    
    \hline
        
    435 & इ॒मम्   ।   ए॒व   ।   तेन॑   ।   लो॒कम्   ।    & TS\_5.2.1.7       \\
    
    \hline
        
    436 & इ॒मम्   ।   य॒म॒   ।   प्र॒स्त॒रमिति॑ प्र{-}स्त॒रम्   ।   एति॑   ।    & TS\_2.6.12.6       \\
    
    \hline
        
    437 & इ॒मान्   ।   ए॒व   ।   लो॒कान्   ।   अ॒भ्यारो॑ह॒न्तीत्य॑भि{-}आरो॑हन्ति   ।    & TS\_7.4.1.2       \\
    
    \hline
        
    438 & इ॒माम्   ।   अ॒गृ॒भ्ण॒न्न्   ।   र॒श॒नाम्   ।   ऋ॒तस्य॑   ।    & TS\_4.1.2.1       \\
    
    \hline
        
    439 & इ॒मे   ।   लो॒काः   ।   इ॒मान्   ।   ए॒व   ।    & TS\_2.5.11.6       \\
    
    \hline
        
    440 & इ॒मे इति॑   ।   वै   ।   स॒ह   ।   आ॒स्ता॒म्   ।    & TS\_3.4.3.1       \\
    
    \hline
        
    441 & इ॒यम्   ।   ए॒व   ।   सा   ।   या   ।    & TS\_4.3.11.1       \\
    
    \hline
        
    442 & इ॒यम्   ।   गीः   ।   स्वश्वा॒ इति॑ सु{-}अश्वाः᳚   ।   त्वा॒   ।    & TS\_1.2.14.4       \\
    
    \hline
        
    443 & इ॒यम्   ।   ते॒   ।   शु॒क्र॒   ।   त॒नूः   ।    & TS\_1.2.4.1       \\
    
    \hline
        
    444 & इ॒यम्   ।   वाव   ।   र॒थ॒न्त॒रमिति॑ रथं{-}त॒रम्   ।   अ॒सौ   ।    & TS\_7.4.4.3       \\
    
    \hline
        
    445 & इ॒यम्   ।   वै   ।   प्र॒जा इति॑ प्र{-}जाः   ।   प॒रा॒भव॑न्ती॒रिति॑ परा{-}भव॑न्तीः   ।    & TS\_1.7.2.4       \\
    
    \hline
        
    446 & इ॒यम्   ।   वै   ।   र॒थ॒न्त॒रमिति॑ रथं{-}त॒रम्   ।   इ॒माम्   ।    & TS\_2.6.7.2       \\
    
    \hline
        
    447 & इ॒या॒त्   ।   यत्   ।   वा॒य॒व्यः॑   ।   प॒शुः   ।    & TS\_5.5.1.4       \\
    
    \hline
        
    448 & इ॒व॒   ।   खलु॑   ।   वै   ।   प॒शवः॑   ।    & TS\_3.4.9.2       \\
    
    \hline
        
    449 & इ॒षे   ।   त्वा॒   ।   इति॑   ।   ब॒र्॒.हिः   ।    & TS\_6.3.6.1       \\
    
    \hline
        
    450 & इ॒षे   ।   त्वा॒   ।   उ॒प॒वीरित्यु॑प{-}वीः   ।   अ॒सि॒   ।    & TS\_1.3.7.1       \\
    
    \hline
        
    451 & इ॒षे   ।   त्वा॒   ।   ऊ॒र्जे   ।   त्वा॒   ।    & TS\_1.1.1.1       \\
    
    \hline
        
    452 & इ॒ष्टर्गः॑   ।   वै   ।   अ॒द्ध्व॒र्युः   ।   यज॑मानस्य   ।    & TS\_3.1.7.1       \\
    
    \hline
        
    453 & ई॒कां॒रायेती᳚म्{-}का॒राय॑   ।   स्वाहा᳚   ।   ईकृं॑ता॒येती᳚म्{-}कृ॒ता॒य॒   ।   स्वाहा᳚   ।    & TS\_7.1.19.1       \\
    
    \hline
        
    454 & ई॒क्ष॒न्ते॒   ।   प॒वित्र᳚म्   ।   वै   ।   सौ॒म्यः   ।    & TS\_6.6.7.2       \\
    
    \hline
        
    455 & ई॒म्   ।   म॒न्द्रासु॑   ।   प्र॒यसः॑   ।   वसुः॑   ।    & TS\_4.1.8.2       \\
    
    \hline
        
    456 & ई॒युः   ।   ते   ।   ये   ।   पूर्व॑तरा॒मिति॒ पूर्व॑{-}त॒रा॒म्   ।    & TS\_1.4.33.1       \\
    
    \hline
        
    457 & ई॒श्व॒रः   ।   वा॒चः   ।   वदि॑तोः   ।   सन्न्   ।    & TS\_3.4.3.4       \\
    
    \hline
        
    458 & उज्जि॑ति॒मित्युत्{-}जि॒ति॒म्   ।   अनु॑   ।   उदिति॑   ।   जे॒ष॒म्   ।    & TS\_1.6.4.2       \\
    
    \hline
        
    459 & उत्त॑रे॒ष्वित्युत्{-}त॒रे॒षु॒   ।   अहः॒ स्वित्यहः॑{-}सु॒   ।   अ॒मुतः॑   ।   अ॒र्वाञ्चः॑   ।    & TS\_3.3.6.2       \\
    
    \hline
        
    460 & उत्त॑र॒मित्युत्{-}त॒र॒म्   ।   ब॒र॒.हिषः॑   ।   प्र॒स्त॒रमिति॑ प्र{-}स्त॒रम्   ।   सा॒द॒य॒ति॒   ।    & TS\_2.6.5.3       \\
    
    \hline
        
    461 & उदिति॑   ।   आयु॑षा   ।   स्वा॒युषेति॑ सु{-}आ॒युषा᳚   ।   उदिति॑   ।    & TS\_1.2.8.1       \\
    
    \hline
        
    462 & उदिति॑   ।   उ॒   ।   त्यम्   ।   जा॒तवे॑दस॒मिति॑ जा॒त{-}वे॒द॒स॒म्   ।    & TS\_1.4.43.1       \\
    
    \hline
        
    463 & उदिति॑   ।   ए॒न॒म्   ।   उ॒त्त॒रामित्यु॑त्{-}त॒राम्   ।   न॒य॒   ।    & TS\_4.6.3.1 TS\_5.4.6.1       \\
    
    \hline
        
    464 & उदिति॑   ।   क्रा॒म॒   ।   उदिति॑   ।   अ॒क्र॒मी॒त्   ।    & TS\_5.1.3.1       \\
    
    \hline
        
    465 & उदिति॑   ।   सृ॒जे॒युः॒   ।   यत्   ।   आदि॑ष्ट॒मित्या{-}दि॒ष्ट॒म्   ।    & TS\_7.5.7.2       \\
    
    \hline
        
    466 & उदि॑ते॒ष्वित्युत्{-}इ॒ते॒षु॒   ।   नक्ष॑त्रेषु   ।   व्र॒तम्   ।   कृ॒णु॒त॒   ।    & TS\_6.1.4.4       \\
    
    \hline
        
    467 & उपेति॑   ।   अ॒न॒क्ति॒   ।   पत्नी᳚   ।   हि   ।    & TS\_6.2.9.2       \\
    
    \hline
        
    468 & उपेति॑   ।   अ॒स्य॒ति॒   ।   अस्क॑न्दाय   ।   अस्क॑न्नम्   ।    & TS\_6.3.8.3       \\
    
    \hline
        
    469 & उपेति॑   ।   अ॒ह्व॒थाः॒   ।   इति॑   ।   ह॒   ।    & TS\_1.7.2.3       \\
    
    \hline
        
    470 & उपेति॑   ।   आ॒प्नो॒ति॒   ।   कनी॑याꣳसि   ।   वै   ।    & TS\_6.6.11.5       \\
    
    \hline
        
    471 & उपेति॑   ।   इ॒या॒त्   ।   न   ।   तृ॒तीय᳚म्   ।    & TS\_5.6.8.4       \\
    
    \hline
        
    472 & उपेति॑   ।   ग॒न्त॒न॒   ।   या   ।   वः॒   ।    & TS\_1.5.11.5       \\
    
    \hline
        
    473 & उपेति॑   ।   द॒धा॒ति॒   ।   तस्मा᳚त्   ।   अ॒क्ष्ण॒या   ।    & TS\_5.2.10.5       \\
    
    \hline
        
    474 & उपेति॑   ।   द॒धा॒ति॒   ।   द्वौ   ।   त्रि॒वृता॒विति॑ त्रि{-}वृतौ᳚   ।    & TS\_5.3.3.3       \\
    
    \hline
        
    475 & उपेति॑   ।   धा॒व॒ति॒   ।   सः   ।   ए॒व   ।    & TS\_2.2.7.3       \\
    
    \hline
        
    476 & उपेति॑   ।   नः॒   ।   एति॑   ।   व॒र्त॒स्व॒   ।    & TS\_6.2.8.5       \\
    
    \hline
        
    477 & उपेति॑   ।   य॒न्ति॒   ।   प्रा॒जा॒प॒त्यमिति॑ प्राजा{-}प॒त्यम्   ।   प॒शुम्   ।    & TS\_7.5.7.4       \\
    
    \hline
        
    478 & उ॒   ।   द्या॒वा॒पृ॒थि॒वी॒ इति॑ द्यावा{-}पृ॒थि॒वी॒   ।   भ॒द्रम्   ।   अ॒भू॒त्   ।    & TS\_2.6.9.5       \\
    
    \hline
        
    479 & उ॒   ।   नः॒   ।   प॒र॒स्पा इति॑ परः{-}पाः   ।   त्वम्   ।    & TS\_3.5.11.3       \\
    
    \hline
        
    480 & उ॒क्थम्   ।   वा॒चि   ।   इन्द्रा॑य   ।   इति॑   ।    & TS\_3.2.9.2       \\
    
    \hline
        
    481 & उ॒क्थेषु॑   ।   श॒व॒सः॒   ।   प॒ते॒   ।   इष᳚म्   ।    & TS\_4.4.4.7       \\
    
    \hline
        
    482 & उ॒क्ष॒ति॒   ।   ए॒भ्यः   ।   ए॒व   ।   ए॒न॒त्   ।    & TS\_2.6.5.2       \\
    
    \hline
        
    483 & उ॒त   ।   इ॒माम्   ।   कस्मै᳚   ।   दे॒वाय॑   ।    & TS\_4.1.8.4       \\
    
    \hline
        
    484 & उ॒त   ।   यदि॑   ।   अ॒न्धः   ।   भव॑ति   ।    & TS\_2.2.4.4       \\
    
    \hline
        
    485 & उ॒त   ।   स॒दृंगिति॑ सं{-}दृक्   ।   प्र॒जाप॑ति॒रिति॑ प्र॒जा{-}प॒तिः॒   ।   प॒र॒मे॒ष्ठी   ।    & TS\_5.7.4.4       \\
    
    \hline
        
    486 & उ॒त्तिष्ठ॒न्नित्यु॑त्{-}तिष्ठन्न्॑   ।   ओज॑सा   ।   स॒ह   ।   पी॒त्वा   ।    & TS\_1.4.30.1       \\
    
    \hline
        
    487 & उ॒त्त॒म इत्यु॑त्{-}त॒मः   ।   उदिति॑   ।   बु॒द्ध्य॒स्व॒   ।   अ॒ग्ने॒   ।    & TS\_4.7.13.5       \\
    
    \hline
        
    488 & उ॒त्त॒म इत्यु॑त्{-}त॒मः   ।   हि   ।   प्रा॒ण इति॑ प्र{-}अ॒नः   ।   यदि॑   ।    & TS\_6.3.10.5       \\
    
    \hline
        
    489 & उ॒त्त॒र॒त इत्यु॑त्{-}त॒र॒तः   ।   तस्मा᳚त्   ।   स॒व्यः   ।   हस्त॑योः   ।    & TS\_5.3.3.5       \\
    
    \hline
        
    490 & उ॒त्त॒र॒तो॒ऽभि॒प्र॒या॒यीत्यु॑त्तरतः{-}अ॒भि॒प्र॒या॒यी   ।   ज॒य॒ति॒   ।   वज्रः॑   ।   वै   ।    & TS\_5.3.5.2       \\
    
    \hline
        
    491 & उ॒थ्सृज्या3मित्यु॑त्{-}सृज्या3म्   ।   न   ।   उ॒थ्सृज्या3मित्यु॑त्{-} सृज्या3म्   ।   इति॑   ।    & TS\_7.5.7.1       \\
    
    \hline
        
    492 & उ॒थ्सृ॒जन्तीत्यु॑त्{-}सृ॒जन्ति॑   ।   तु॒रीय᳚म्   ।   खलु॑   ।   वै   ।    & TS\_7.5.6.4       \\
    
    \hline
        
    493 & उ॒थ्स॒न्न॒य॒ज्ञ् इत्यु॑थ्सन्न{-}य॒ज्ञ्ः   ।   वै   ।   ए॒षः   ।   यत्   ।    & TS\_5.3.1.1       \\
    
    \hline
        
    494 & उ॒दय॑न॒मित्यु॑त्{-}अय॑नम्   ।   वेद॑   ।   प्रति॑ष्ठिते॒नेति॒ प्रति॑{-}स्थि॒ते॒न॒   ।   अरि॑ष्टेन   ।    & TS\_1.6.11.2       \\
    
    \hline
        
    495 & उ॒न्न॒त इत्यु॑त्{-}न॒तः   ।   ऋ॒ष॒भः   ।   वा॒म॒नः   ।   ते   ।    & TS\_5.6.14.1       \\
    
    \hline
        
    496 & उ॒न्न॒तमित्यु॑त्{-}न॒तम्   ।   स्या॒त्   ।   अ॒न्त॒रा   ।   ह॒वि॒द्‌र्धान॒मिति॑ हविः{-}धान᳚म्   ।    & TS\_6.2.6.3       \\
    
    \hline
        
    497 & उ॒परि॑   ।   इ॒व॒   ।   हि   ।   सु॒व॒र्ग इति॑ सुवः{-}गः   ।    & TS\_2.6.5.4       \\
    
    \hline
        
    498 & उ॒परि॑ष्टात्   ।   इ॒न्द्रि॒येण॑   ।   ए॒व   ।   अ॒स्मै॒   ।    & TS\_2.2.11.4       \\
    
    \hline
        
    499 & उ॒परि॑ष्टात्   ।   ए॒व   ।   अ॒स्मै॒   ।   छन्दाꣳ॑सि   ।    & TS\_3.4.9.5       \\
    
    \hline
        
    500 & उ॒पस्थ॒ इत्यु॒प{-}स्थे॒   ।   विश्वा॑नि   ।   अ॒ग्ने॒   ।   व॒युना॑नि   ।    & TS\_4.2.1.5       \\
    
    \hline
        
    501 & उ॒पाम॑न्त्रय॒न्तेत्यु॑प{-}अम॑न्त्रयन्त   ।   रा॒ज्येन॑   ।   पि॒तरः॑   ।   य॒मम्   ।    & TS\_2.6.6.5       \\
    
    \hline
        
    502 & उ॒पाव॑र्तत॒ इत्यु॑प{-}आव॑र्तते   ।   नक्तो॒षासा᳚   ।   इति॑   ।   कृ॒ष्णायै᳚   ।    & TS\_5.4.9.3       \\
    
    \hline
        
    503 & उ॒पाꣳ॒॒श्व॒न्त॒र्या॒मयो॒रित्यु॑पाꣳशु{-}अ॒न्त॒र्या॒मयोः᳚   ।   यत्   ।   उ॒च्चैः   ।   तत्   ।    & TS\_6.5.11.3       \\
    
    \hline
        
    504 & उ॒पो॒त्थायेत्यु॑प{-}उ॒त्थाय॑   ।   प्र॒जामिति॑ प्र{-}जाम्   ।   प॒शून्   ।   अ॒भीति॑   ।    & TS\_7.5.15.2       \\
    
    \hline
        
    505 & उ॒प॒दधा॒तीत्यु॑प{-}दधा॑ति   ।   तस्मा᳚त्   ।   सर्वान्॑   ।   ऋ॒तून्   ।    & TS\_5.3.1.3       \\
    
    \hline
        
    506 & उ॒प॒दधा॒तीत्यु॑प{-}दधा॑ति   ।   पृ॒ष्ठाना᳚म्   ।   ए॒व   ।   तेजः॑   ।    & TS\_5.3.7.2       \\
    
    \hline
        
    507 & उ॒प॒धी॒यन्त॒ इत्यु॑प{-}धी॒यन्ते᳚   ।   गच्छ॑ति   ।   स्वारा᳚ज्य॒मिति॒ स्व{-}रा॒ज्य॒म्   ।   स॒प्त   ।    & TS\_5.3.2.5       \\
    
    \hline
        
    508 & उ॒प॒नमे॒दित्यु॑प{-}नमे᳚त्   ।   अ॒ग्निः   ।   सर्वाः᳚   ।   दे॒वताः᳚   ।    & TS\_2.2.9.3       \\
    
    \hline
        
    509 & उ॒प॒नि॒धायेत्यु॑प{-}नि॒धाय॑   ।   ब्रा॒ह्म॒णम्   ।   द॒क्षि॒ण॒तः   ।   नि॒षाद्येति॑ नि{-}साद्य॑   ।    & TS\_6.4.9.3       \\
    
    \hline
        
    510 & उ॒प॒प्र॒यन्त॒ इत्यु॑प{-}प्र॒यन्तः॑   ।   अ॒द्ध्व॒रम्   ।   मन्त्र᳚म्   ।   वो॒चे॒म॒   ।    & TS\_1.5.5.1       \\
    
    \hline
        
    511 & उ॒प॒या॒मगृ॑हीत॒ इत्यु॑पया॒म{-}गृ॒ही॒तः॒   ।   अ॒सि॒   ।   अ॒न्तः   ।   य॒च्छ॒   ।    & TS\_1.4.3.1       \\
    
    \hline
        
    512 & उ॒प॒या॒मगृ॑हीत॒ इत्यु॑पया॒म{-}गृ॒ही॒तः॒   ।   अ॒सि॒   ।   इन्द्रा॑य   ।   त्वा॒   ।    & TS\_1.4.12.1       \\
    
    \hline
        
    513 & उ॒प॒या॒मगृ॑हीत॒ इत्यु॑पया॒म{-}गृ॒ही॒तः॒   ।   अ॒सि॒   ।   नृ॒षद॒मिति॑ नृ{-}सद᳚म्   ।   त्वा॒   ।    & TS\_1.7.12.1       \\
    
    \hline
        
    514 & उ॒प॒या॒मगृ॑हीत॒ इत्यु॑पया॒म{-}गृ॒ही॒तः॒   ।   अ॒सि॒   ।   प्र॒जाप॑तय॒ इति॑ प्र॒जा{-}प॒त॒ये॒   ।   त्वा॒   ।    & TS\_3.5.8.1       \\
    
    \hline
        
    515 & उ॒प॒या॒मगृ॑हीत॒ इत्यु॑पया॒म{-}गृ॒ही॒तः॒   ।   अ॒सि॒   ।   वा॒क्ष॒सदिति॑ वाक्ष{-}सत्   ।   अ॒सि॒   ।    & TS\_3.2.10.1       \\
    
    \hline
        
    516 & उ॒प॒स॒द्यन्त॒ इत्यु॑प{-}स॒द्यन्ते᳚   ।   अ॒हो॒रा॒त्राभ्या॒मित्य॑हः{-}रा॒त्राभ्या᳚म्   ।   ए॒व   ।   तत्   ।    & TS\_6.2.3.4       \\
    
    \hline
        
    517 & उ॒भयोः᳚   ।   लो॒कयोः᳚   ।   अ॒भिजि॑त्या॒ इत्य॒भि{-}जि॒त्यै॒   ।   के॒श॒श्म॒श्र्विति॑ केश{-}श्म॒श्रु   ।    & TS\_6.1.1.2       \\
    
    \hline
        
    518 & उ॒भा   ।   राध॑सः   ।   स॒ह   ।   मा॒द॒यद्ध्यै᳚   ।    & TS\_1.5.5.2       \\
    
    \hline
        
    519 & उ॒भा   ।   वा॒म्   ।   इ॒न्द्रा॒ग्नी॒ इती᳚न्द्र{-}अ॒ग्नी॒   ।   आ॒हु॒वध्यै᳚   ।    & TS\_1.1.14.1       \\
    
    \hline
        
    520 & उ॒भा   ।   हि   ।   वा॒म्   ।   सु॒हवेति॑ सु{-}हवा᳚   ।    & TS\_1.1.14.2       \\
    
    \hline
        
    521 & उ॒रुम्   ।   हि   ।   राजा᳚   ।   वरु॑णः   ।    & TS\_1.4.45.1       \\
    
    \hline
        
    522 & उ॒शन्तः॑   ।   त्वा॒   ।   ह॒वा॒म॒हे॒   ।   उ॒शन्तः॑   ।    & TS\_2.6.12.1       \\
    
    \hline
        
    523 & ऊर्क्   ।   च॒   ।   मे॒   ।   सू॒नृता᳚   ।    & TS\_4.7.4.1       \\
    
    \hline
        
    524 & ऊर्क्   ।   वै   ।   यवः॑   ।   प्रा॒णा इति॑ प्र{-}अ॒नाः   ।    & TS\_6.2.11.3       \\
    
    \hline
        
    525 & ऊ॒द्‌र्ध्वः   ।   उ॒   ।   स्विति॑   ।   नः॒   ।    & TS\_4.1.4.2       \\
    
    \hline
        
    526 & ऊ॒द्‌र्ध्वाः   ।   अ॒स्य॒   ।   स॒मिध॒ इति॑ सं{-}इधः॑   ।   भ॒व॒न्ति॒   ।    & TS\_4.1.8.1       \\
    
    \hline
        
    527 & ऊ॒र्मिः   ।   द्र॒फ्सः   ।   अ॒पाम्   ।   अ॒सि॒   ।    & TS\_4.3.4.3       \\
    
    \hline
        
    528 & ऋषि᳚म्   ।   आ॒र्॒.षे॒यम्   ।   इति॑   ।   आ॒ह॒   ।    & TS\_6.6.1.4       \\
    
    \hline
        
    529 & ऋष॑यः   ।   वै   ।   इन्द्र᳚म्   ।   प्र॒त्यक्ष॒मिति॑ प्रति{-}अक्ष᳚म्   ।    & TS\_3.5.2.1       \\
    
    \hline
        
    530 & ऋष॑यः   ।   स॒प्त   ।   धाम॑   ।   प्रि॒याणि॑   ।    & TS\_1.5.3.3       \\
    
    \hline
        
    531 & ऋ॒क्षा   ।   वा   ।   इ॒यम्   ।   अ॒लो॒मका᳚   ।    & TS\_7.4.3.1       \\
    
    \hline
        
    532 & ऋ॒ख्सा॒मे इत्यृ॑क्{-}सा॒मे   ।   वै   ।   दे॒वेभ्यः॑   ।   य॒ज्ञाय॑   ।    & TS\_6.1.3.1       \\
    
    \hline
        
    533 & ऋ॒चः   ।   प्र॒ण॒व इति॑ प्र{-}न॒वः   ।   उ॒क्थ॒शꣳ॒॒सिना॒मित्यु॑क्थ{-}शꣳ॒॒सिना᳚म्   ।   प्र॒ति॒ग॒र इति॑ प्रति{-}ग॒रः   ।    & TS\_3.2.9.6       \\
    
    \hline
        
    534 & ऋ॒चा   ।   आ॒क्रम॑ण॒मित्या᳚{-}क्रम॑णम्   ।   प्रतीति॑   ।   इष्ट॑काम्   ।    & TS\_5.5.7.2       \\
    
    \hline
        
    535 & ऋ॒च्छ॒तु॒   ।   यम्   ।   द्वि॒ष्मः   ।   इति॑   ।    & TS\_5.4.4.2       \\
    
    \hline
        
    536 & ऋ॒तवः॑   ।   ऋ॒तुषु॑   ।   ऐ॒व   ।   प्रतीति॑   ।    & TS\_1.5.7.3       \\
    
    \hline
        
    537 & ऋ॒तवः॑   ।   वै   ।   प्र॒जाका॑मा॒ इति॑ प्र॒जा{-}का॒माः॒   ।   प्र॒जामिति॑ प्र{-}जाम्   ।    & TS\_7.2.6.1       \\
    
    \hline
        
    538 & ऋ॒तस्य॑   ।   गर्भः॑   ।   प्र॒थ॒मा   ।   व्यू॒षुषीति॑ वि{-}ऊ॒षुषी᳚   ।    & TS\_4.3.11.5       \\
    
    \hline
        
    539 & ऋ॒तस्य॑   ।   र॒श्मिम्   ।   एति॑   ।   द॒दे॒   ।    & TS\_2.1.11.4       \\
    
    \hline
        
    540 & ऋ॒तावेत्यृ॒ता{-}वा॒   ।   ध॒र्ता   ।   कृ॒ष्टी॒नाम्   ।   उ॒त   ।    & TS\_1.3.14.2       \\
    
    \hline
        
    541 & ऋ॒ता॒षाट्   ।   ऋ॒तधा॒मेत्यृ॒त{-}धा॒मा॒   ।   अ॒ग्निः   ।   ग॒न्ध॒र्वः   ।    & TS\_3.4.7.1       \\
    
    \hline
        
    542 & ऋ॒तुभि॒रित्यृ॒तु{-}भिः॒   ।   सं॒ॅव॒थ्स॒रमिति॑ सं{-}व॒थ्स॒रम्   ।   विश्वाः᳚   ।   एति॑   ।    & TS\_5.1.8.6       \\
    
    \hline
        
    543 & ऋ॒तुषु॑   ।   ए॒व   ।   प्रतीति॑   ।   ति॒ष्ठ॒ति॒   ।    & TS\_6.6.3.4       \\
    
    \hline
        
    544 & ऋ॒तुषु॑   ।   ए॒व   ।   सं॒ॅव॒थ्स॒र इति॑ सं{-}व॒थ्स॒रे   ।   प्रतीति॑   ।    & TS\_7.1.10.4       \\
    
    \hline
        
    545 & ऋ॒तु॒स्था इत्यृ॑तु{-}स्थाः   ।   तस्य॑   ।   व॒स॒न्तः   ।   शिरः॑   ।    & TS\_5.7.6.6       \\
    
    \hline
        
    546 & ऋ॒तून्   ।   त॒न्व॒ते॒   ।   क॒वयः॑   ।   प्र॒जा॒न॒तीरिति॑ प्र{-}जा॒न॒तीः   ।    & TS\_4.3.11.3       \\
    
    \hline
        
    547 & ऋ॒तू॒नाम्   ।   प्री॒णा॒मि॒   ।   इति॑   ।   आ॒ह॒   ।    & TS\_1.6.11.5       \\
    
    \hline
        
    548 & ऋ॒त॒जिदित्यृ॑त{-}जित्   ।   च॒   ।   स॒त्य॒जिदिति॑ सत्य{-}जित्   ।   च॒   ।    & TS\_4.6.5.6       \\
    
    \hline
        
    549 & ऋ॒त॒प्र॒जा॒तेत्यृ॑त{-}प्र॒जा॒त॒   ।   तत्   ।   अ॒स्मासु॑   ।   द्रवि॑णम्   ।    & TS\_1.8.22.3       \\
    
    \hline
        
    550 & ऋ॒त॒व्याः᳚   ।   उपेति॑   ।   द॒धा॒ति॒   ।   ऋ॒तू॒नाम्   ।    & TS\_5.4.2.1       \\
    
    \hline
        
    551 & ऋ॒द्ध्नोति॑   ।   ए॒व   ।   यः   ।   अ॒स्य॒   ।    & TS\_1.5.1.4       \\
    
    \hline
        
    552 & ऋ॒ध्नोति॑   ।   सः   ।   पु॒न॒रा॒धेय॒मिति॑ पुनः{-}आ॒धेय᳚म्   ।   एति॑   ।    & TS\_5.4.10.5       \\
    
    \hline
        
    553 & ऋ॒ष्टि॒मन्त॒ इत्यृ॑ष्टि{-}मन्तः॑   ।   आपः॑   ।   इ॒व॒   ।   स॒द्ध्रिय॑ञ्चः   ।    & TS\_3.1.11.6       \\
    
    \hline
        
    554 & एका॑दश   ।   प्रा॒तः   ।   ग॒व्याः   ।   प॒शवः॑   ।    & TS\_5.6.22.1       \\
    
    \hline
        
    555 & एका॑दशकपाल॒मित्येका॑दश{-}क॒पा॒ल॒म्   ।   इन्द्रा॑य   ।   अ॒धि॒रा॒जायेत्य॑धि{-}रा॒जाय॑   ।   इन्द्रा॑य   ।    & TS\_2.3.6.2       \\
    
    \hline
        
    556 & एका॑दशकपाल॒मित्येका॑दश{-}क॒पा॒ल॒म्   ।   हिर॑ण्यम्   ।   दक्षि॑णा   ।   ऐ॒न्द्रम्   ।    & TS\_1.8.1.2       \\
    
    \hline
        
    557 & एका᳚   ।   च॒   ।   मे॒   ।   ति॒स्रः   ।    & TS\_4.7.11.1       \\
    
    \hline
        
    558 & एक॑या   ।   अ॒स्तु॒व॒त॒   ।   प्र॒जा इति॑ प्र{-}जाः   ।   अ॒धी॒य॒न्त॒   ।    & TS\_4.3.10.1       \\
    
    \hline
        
    559 & एक॑विꣳश॒त्येत्येक॑{-}विꣳ॒॒श॒त्या॒   ।   माषैः᳚   ।   पु॒रु॒ष॒शी॒र॒.षमिति॑ पुरुष{-}शी॒र॒.षम्   ।   अच्छ॑   ।    & TS\_5.1.8.1       \\
    
    \hline
        
    560 & एक॑स्मै   ।   स्वाहा᳚   ।   त्रि॒भ्य इति॑ त्रि{-}भ्यः   ।   स्वाहा᳚   ।    & TS\_7.2.12.1       \\
    
    \hline
        
    561 & एक॑स्मै   ।   स्वाहा᳚   ।   द्वाभ्या᳚म्   ।   स्वाहा᳚   ।    & TS\_7.2.11.1       \\
    
    \hline
        
    562 & एति॑   ।   अ॒क्रा॒न्   ।   वा॒जी   ।   पृ॒थि॒वीम्   ।    & TS\_7.5.19.1       \\
    
    \hline
        
    563 & एति॑   ।   अ॒घा॒र॒य॒त्   ।   ततः॑   ।   वै   ।    & TS\_6.3.7.2       \\
    
    \hline
        
    564 & एति॑   ।   अ॒या॒नि॒   ।   इति॑   ।   न   ।    & TS\_5.7.5.6       \\
    
    \hline
        
    565 & एति॑   ।   अ॒स्य॒   ।   वी॒रः   ।   जा॒य॒ते॒   ।    & TS\_1.5.8.4       \\
    
    \hline
        
    566 & एति॑   ।   इ॒हि॒   ।   अ॒व॒क्राम॒न्नित्य॑व{-}क्रामन्न्॑   ।   अश॑स्तीः   ।    & TS\_4.1.2.2       \\
    
    \hline
        
    567 & एति॑   ।   ए॒ति॒   ।   सोम॑च्युत॒मिति॒ सोम॑{-}च्यु॒त॒म्   ।   इति॑   ।    & TS\_6.6.1.3       \\
    
    \hline
        
    568 & एति॑   ।   ग्राव्‌ण्णः॑   ।   एति॑   ।   वा॒य॒व्या॑नि   ।    & TS\_6.3.2.3       \\
    
    \hline
        
    569 & एति॑   ।   च॒   ।   पुष्क॑रम्   ।   दि॒वः   ।    & TS\_4.2.8.2       \\
    
    \hline
        
    570 & एति॑   ।   ज॒गा॒म॒   ।   पर॑स्याः   ।   सृ॒कम्   ।    & TS\_1.6.12.5       \\
    
    \hline
        
    571 & एति॑   ।   ति॒ष्ठ॒   ।   वृ॒त्र॒ह॒न्निति॑ वृत्र{-}ह॒न्न्   ।   रथ᳚म्   ।    & TS\_1.4.37.1       \\
    
    \hline
        
    572 & एति॑   ।   द॒दी॒य॒   ।   इति॑   ।   तस्य॑   ।    & TS\_3.4.8.6       \\
    
    \hline
        
    573 & एति॑   ।   द॒दे॒   ।   इन्द्र॑स्य   ।   बा॒हुः   ।    & TS\_1.1.9.1       \\
    
    \hline
        
    574 & एति॑   ।   द॒दे॒   ।   ऋ॒तस्य॑   ।   त्वा॒   ।    & TS\_1.3.8.1       \\
    
    \hline
        
    575 & एति॑   ।   द॒दे॒   ।   ग्रावा᳚   ।   अ॒सि॒   ।    & TS\_1.4.1.1       \\
    
    \hline
        
    576 & एति॑   ।   निष॑त्त॒ इति॒ नि{-}स॒त्तः॒   ।   नमः॑   ।   ते॒   ।    & TS\_4.7.13.2       \\
    
    \hline
        
    577 & एति॑   ।   प्या॒य॒ता॒म्   ।   ध्रु॒वा   ।   घृ॒तेन॑   ।    & TS\_1.6.5.1       \\
    
    \hline
        
    578 & एति॑   ।   प्या॒य॒स्व॒   ।   म॒दि॒न्त॒म॒   ।   सोम॑   ।    & TS\_1.4.32.1       \\
    
    \hline
        
    579 & एति॑   ।   ब्रह्मन्न्॑   ।   ब्रा॒ह्म॒णः   ।   ब्र॒ह्म॒व॒र्च॒सीति॑ ब्रह्म{-}व॒र्च॒सी   ।    & TS\_7.5.18.1       \\
    
    \hline
        
    580 & एति॑   ।   मे॒   ।   गृ॒हाः   ।   भ॒व॒न्तु॒   ।    & TS\_7.3.13.1       \\
    
    \hline
        
    581 & एति॑   ।   ल॒भे॒त॒   ।   आ॒दि॒त्यान्   ।   ए॒व   ।    & TS\_2.1.2.4       \\
    
    \hline
        
    582 & एति॑   ।   ल॒भे॒त॒   ।   प्र॒जाप॑ति॒रिति॑ प्र॒जा{-}प॒तिः॒   ।   सर्वाः᳚   ।    & TS\_2.1.4.3       \\
    
    \hline
        
    583 & एति॑   ।   वा॒यो॒ इति॑   ।   भू॒ष॒   ।   शु॒चि॒पा॒ इति॑ शुचि{-}पाः॒   ।    & TS\_1.4.4.1 TS\_3.4.2.1       \\
    
    \hline
        
    584 & एति॑   ।   वि॒श॒स्व॒   ।   शम्   ।   तो॒काय॑   ।    & TS\_5.7.2.5       \\
    
    \hline
        
    585 & एति॑   ।   वृ॒ञ्जे॒   ।   सुवः॑   ।   प॒त्नी॒सं॒ॅया॒जानिति॑ पत्नी{-}सं॒ॅया॒जान्   ।    & TS\_7.3.11.3       \\
    
    \hline
        
    586 & एति॑   ।   स॒मु॒द्रात्   ।   एति॑   ।   अ॒न्तरि॑क्षात्   ।    & TS\_3.5.5.2       \\
    
    \hline
        
    587 & ए॒क॒विꣳ॒॒श इत्ये॑क{-}विꣳ॒॒शः   ।   उ॒क्थ्यः॑   ।   सौ॒रीषु॑   ।   प्रति॑ष्ठित्या॒ इति॒ प्रति॑{-}स्थि॒त्यै॒   ।    & TS\_7.2.5.6       \\
    
    \hline
        
    588 & ए॒क॒विꣳ॒॒श इत्ये॑क{-}विꣳ॒॒शः   ।   ए॒ताव॑न्तः   ।   वै   ।   दे॒व॒लो॒का इति॑ देव{-}लो॒काः   ।    & TS\_5.1.10.4       \\
    
    \hline
        
    589 & ए॒तत्   ।   कु॒र्व॒न्ति॒   ।   यत्   ।   ज्यायाꣳ॑सम्   ।    & TS\_7.2.5.5       \\
    
    \hline
        
    590 & ए॒तत्   ।   क॒रो॒ति॒   ।   यत्   ।   वेदि᳚म्   ।    & TS\_2.6.4.4       \\
    
    \hline
        
    591 & ए॒तत्   ।   वै   ।   अ॒ग्नेः   ।   शिरः॑   ।    & TS\_5.5.4.4       \\
    
    \hline
        
    592 & ए॒तत्   ।   वै   ।   अ॒पाम्   ।   ना॒म॒धेय॒मिति॑ नाम{-}धेय᳚म्   ।    & TS\_3.3.4.1       \\
    
    \hline
        
    593 & ए॒तत्   ।   वै   ।   छन्द॑साम्   ।   रू॒पम्   ।    & TS\_5.4.8.5       \\
    
    \hline
        
    594 & ए॒तत्   ।   स॒र्व॒दे॒व॒त्य॑मिति॑ सर्व{-}दे॒व॒त्य᳚म्   ।   यत्   ।   नव॑नीत॒मिति॒ नव॑{-}नी॒त॒म्   ।    & TS\_6.1.1.5       \\
    
    \hline
        
    595 & ए॒तम्   ।   युवा॑नम्   ।   परीति॑   ।   वः॒   ।    & TS\_3.3.9.1       \\
    
    \hline
        
    596 & ए॒तम्   ।   सो॒मे॒न्द्रम्   ।   श्या॒मा॒कम्   ।   च॒रुम्   ।    & TS\_2.3.2.7       \\
    
    \hline
        
    597 & ए॒तया᳚   ।   स॒ज्ञांन्येति॑ सं{-}ज्ञान्या᳚   ।   अ॒या॒ज॒य॒त्   ।   अ॒ग्नये᳚   ।    & TS\_2.2.11.6       \\
    
    \hline
        
    598 & ए॒तस्य॑   ।   स्क॒न्द॒ति॒   ।   यस्य॑   ।   पृ॒ष॒दा॒ज्यमिति॑ पृषत्{-}आ॒ज्यम्   ।    & TS\_3.2.6.3       \\
    
    \hline
        
    599 & ए॒ताः   ।   वै   ।   दे॒वताः᳚   ।   सु॒व॒र्ग्या॑ इति॑ सुवः{-}ग्याः᳚   ।    & TS\_5.4.3.5       \\
    
    \hline
        
    600 & ए॒तानि॑   ।   दि॒वि   ।   रो॒च॒नानि॑   ।   अ॒ग्निः   ।    & TS\_2.3.14.2       \\
    
    \hline
        
    601 & ए॒ताम्   ।   उपेति॑   ।   द॒धा॒ति॒   ।   इ॒यम्   ।    & TS\_5.7.2.2       \\
    
    \hline
        
    602 & ए॒ताव॑तीः   ।   वै   ।   दे॒वताः᳚   ।   ताभ्यः॑   ।    & TS\_6.4.11.2       \\
    
    \hline
        
    603 & ए॒ता॒म्   ।   वस॑वः   ।   रु॒द्राः   ।   आ॒दि॒त्याः   ।    & TS\_4.3.10.3       \\
    
    \hline
        
    604 & ए॒ति॒   ।   प्र॒जाप॑ति॒रिति॑ प्र॒जा{-}प॒तिः॒   ।   वै   ।   ए॒षः   ।    & TS\_7.2.9.3       \\
    
    \hline
        
    605 & ए॒ते   ।   अ॒न॒वा॒रम्   ।   अ॒पा॒रम्   ।   प्रेति॑   ।    & TS\_7.5.3.2       \\
    
    \hline
        
    606 & ए॒ते   ।   ए॒तान्   ।   ग्रहान्॑   ।   अ॒प॒श्य॒न्न्   ।    & TS\_6.6.8.3       \\
    
    \hline
        
    607 & ए॒ते   ।   सो॒म॒पी॒थेनेति॑ सोम{-}पी॒थेन॑   ।   आद्‌र्ध्य॑न्त   ।   दे॒वाः   ।    & TS\_6.3.1.4       \\
    
    \hline
        
    608 & ए॒ना॒म्   ।   मि॒नो॒ति॒   ।   ब्र॒ह्म॒वनि॒मिति॑ ब्रह्म{-}वनि᳚म्   ।   त्वा॒   ।    & TS\_6.2.10.5       \\
    
    \hline
        
    609 & ए॒न॒म्   ।   इति॑   ।   आ॒ह॒   ।   ओष॑धयः   ।    & TS\_5.1.5.10       \\
    
    \hline
        
    610 & ए॒न॒म्   ।   क्री॒णा॒ति॒   ।   धे॒न्वा   ।   क्री॒णा॒ति॒   ।    & TS\_6.1.10.2       \\
    
    \hline
        
    611 & ए॒न॒म्   ।   क॒रो॒ति॒   ।   ऋ॒त्विजः॑   ।   वृ॒णी॒ते॒   ।    & TS\_6.3.7.5       \\
    
    \hline
        
    612 & ए॒न॒म्   ।   दे॒वता॑भिः   ।   समिति॑   ।   अ॒द्‌र्ध॒य॒ति॒   ।    & TS\_6.1.11.3       \\
    
    \hline
        
    613 & ए॒न॒म्   ।   प॒व॒य॒ति॒   ।   परा॑ची   ।   वै   ।    & TS\_2.1.10.3       \\
    
    \hline
        
    614 & ए॒न॒म्   ।   प॒शवः॑   ।   भु॒ञ्जन्तः॑   ।   उपेति॑   ।    & TS\_6.5.9.4       \\
    
    \hline
        
    615 & ए॒न॒म्   ।   लो॒कम्   ।   ग॒म॒यि॒त्वा   ।   क्री॒णा॒ति॒   ।    & TS\_3.1.2.2       \\
    
    \hline
        
    616 & ए॒न॒म्   ।   समिति॑   ।   भ॒र॒ति॒   ।   वीति॑   ।    & TS\_1.5.4.3       \\
    
    \hline
        
    617 & ए॒न॒म्   ।   समिति॑   ।   सृ॒ज॒ति॒   ।   शान्त्यै᳚   ।    & TS\_5.1.6.2       \\
    
    \hline
        
    618 & ए॒न॒म्   ।   सौ॒श्र॒व॒साय॑   ।   जि॒न्व॒ति॒   ।   यत्   ।    & TS\_4.6.8.2       \\
    
    \hline
        
    619 & ए॒व   ।   अन्त᳚म्   ।   ग॒त्वा   ।   अप॑थात्   ।    & TS\_7.2.8.6       \\
    
    \hline
        
    620 & ए॒व   ।   अवेति॑   ।   रु॒न्धे॒   ।   प॒ञ्च॒द॒श इति॑ पञ्च{-}द॒शः   ।    & TS\_7.2.2.2       \\
    
    \hline
        
    621 & ए॒व   ।   अवेति॑   ।   रु॒न्धे॒   ।   यत्   ।    & TS\_2.3.7.3       \\
    
    \hline
        
    622 & ए॒व   ।   अ॒ग्निम्   ।   चि॒नु॒ते॒   ।   उ॒वाच॑   ।    & TS\_5.4.2.2       \\
    
    \hline
        
    623 & ए॒व   ।   अ॒स्मै॒   ।   अन्न᳚म्   ।   अवेति॑   ।    & TS\_5.6.2.6       \\
    
    \hline
        
    624 & ए॒व   ।   अ॒स्मै॒   ।   इ॒न्द्रि॒यम्   ।   प॒शून्   ।    & TS\_2.2.7.2       \\
    
    \hline
        
    625 & ए॒व   ।   अ॒स्मै॒   ।   प॒र्जन्य᳚म्   ।   व॒र्.॒ष॒य॒न्ति॒   ।    & TS\_2.4.10.3       \\
    
    \hline
        
    626 & ए॒व   ।   अ॒स्य॒   ।   अ॒पु॒वा॒यते᳚   ।   यत्   ।    & TS\_6.2.2.5       \\
    
    \hline
        
    627 & ए॒व   ।   आहु॑ति॒मित्या{-}हु॒ति॒म्   ।   आ॒यत॑नवती॒मित्या॒यत॑न{-}व॒ती॒म्   ।   क॒रो॒ति॒   ।    & TS\_5.7.3.3       \\
    
    \hline
        
    628 & ए॒व   ।   उ॒भ॒यतः॑   ।   प॒रि॒गृह्येति॑ परि{-}गृह्य॑   ।   यज॑मानः   ।    & TS\_5.5.1.5       \\
    
    \hline
        
    629 & ए॒व   ।   ए॒ताभिः॑   ।   यज॑मानः   ।   इ॒मान्   ।    & TS\_5.3.10.2       \\
    
    \hline
        
    630 & ए॒व   ।   ए॒ताम्   ।   एति॑   ।   शा॒स्ते॒   ।    & TS\_1.7.6.4       \\
    
    \hline
        
    631 & ए॒व   ।   ए॒तेन॑   ।   क॒रो॒ति॒   ।   अ॒ग्नये᳚   ।    & TS\_6.2.1.3       \\
    
    \hline
        
    632 & ए॒व   ।   ए॒न॒म्   ।   अवेति॑   ।   रु॒न्धे॒   ।    & TS\_1.5.2.5       \\
    
    \hline
        
    633 & ए॒व   ।   ए॒न॒म्   ।   उदिति॑   ।   य॒च्छ॒ते॒   ।    & TS\_5.4.6.2       \\
    
    \hline
        
    634 & ए॒व   ।   ए॒न॒म्   ।   एति॑   ।   ह॒र॒ति॒   ।    & TS\_6.3.3.3       \\
    
    \hline
        
    635 & ए॒व   ।   ए॒न॒म्   ।   दे॒वता᳚   ।   इ॒ज्यमा॑ना   ।    & TS\_2.5.4.5       \\
    
    \hline
        
    636 & ए॒व   ।   ए॒न॒म्   ।   दे॒वत॑या   ।   प्रेति॑   ।    & TS\_6.3.2.4       \\
    
    \hline
        
    637 & ए॒व   ।   ए॒न॒म्   ।   भूति᳚म्   ।   ग॒म॒य॒न्ति॒   ।    & TS\_5.3.10.3       \\
    
    \hline
        
    638 & ए॒व   ।   ए॒न॒म्   ।   यु॒न॒क्ति॒   ।   य॒ज्ञ्स्य॑   ।    & TS\_1.6.10.2       \\
    
    \hline
        
    639 & ए॒व   ।   ए॒न॒म्   ।   वैभी॑दकः   ।   इ॒द्ध्मः   ।    & TS\_2.1.8.3       \\
    
    \hline
        
    640 & ए॒व   ।   ए॒न॒म्   ।   स॒मा॒नाना᳚म्   ।   क॒रो॒ति॒   ।    & TS\_1.5.7.2       \\
    
    \hline
        
    641 & ए॒व   ।   ए॒भ्यः॒   ।   ह॒व्यम्   ।   व॒ह॒ति॒   ।    & TS\_2.5.6.3       \\
    
    \hline
        
    642 & ए॒व   ।   ए॒षः   ।   मि॒त्राय॑   ।   च॒   ।    & TS\_6.4.8.3       \\
    
    \hline
        
    643 & ए॒व   ।   तत्   ।   द्वाद॑शकपाल॒ इति॒ द्वाद॑श{-}क॒पा॒लः॒   ।   वै॒श्वा॒न॒रः   ।    & TS\_2.2.5.6       \\
    
    \hline
        
    644 & ए॒व   ।   तत्   ।   द॒धा॒ति॒   ।   अ॒न्तरि॑क्षम्   ।    & TS\_6.4.1.2       \\
    
    \hline
        
    645 & ए॒व   ।   तमः॑   ।   मृ॒त्युम्   ।   अपेति॑   ।    & TS\_5.7.5.2       \\
    
    \hline
        
    646 & ए॒व   ।   ते॒ज॒स्वी   ।   अ॒न्ना॒द इत्य॑न्न{-}अ॒दः   ।   इ॒न्द्रि॒या॒वी   ।    & TS\_2.2.5.4       \\
    
    \hline
        
    647 & ए॒व   ।   निरिति॑   ।   व॒पे॒त्   ।   हिर॑ण्यम्   ।    & TS\_2.3.2.5       \\
    
    \hline
        
    648 & ए॒व   ।   प्रतीति॑   ।   ति॒ष्ठ॒ति॒   ।   यः   ।    & TS\_6.6.10.2       \\
    
    \hline
        
    649 & ए॒व   ।   भ्रातृ॑व्यान्   ।   नु॒द॒ते॒   ।   उ॒भ॒यतः॑ प्र‌उग॒मित्यु॑भ॒यतः॑{-}प्र॒उ॒ग॒म्   ।    & TS\_5.4.11.2       \\
    
    \hline
        
    650 & ए॒व   ।   भ॒व॒ति॒   ।   अ॒ग्नये᳚   ।   अन्न॑पतय॒ इत्यन्न॑{-}प॒त॒ये॒   ।    & TS\_2.2.4.2       \\
    
    \hline
        
    651 & ए॒व   ।   भ॒व॒ति॒   ।   अ॒धि॒देव॑न॒ इत्य॑धि{-}देव॑ने   ।   जु॒हो॒ति॒   ।    & TS\_3.4.8.2       \\
    
    \hline
        
    652 & ए॒व   ।   भ॒व॒ति॒   ।   कृ॒ष्णा   ।   भ॒व॒ति॒   ।    & TS\_2.1.9.2       \\
    
    \hline
        
    653 & ए॒व   ।   मृ॒त्योः   ।   ध॒त्ते॒   ।   अवेति॑   ।    & TS\_5.4.4.5       \\
    
    \hline
        
    654 & ए॒व   ।   य॒ज्ञ्म्   ।   दु॒हे॒   ।   पु॒रस्ता᳚त्   ।    & TS\_1.7.4.3       \\
    
    \hline
        
    655 & ए॒व   ।   लो॒केषु॑   ।   प्रतीति॑   ।   ति॒ष्ठ॒न्ति॒   ।    & TS\_7.4.5.3       \\
    
    \hline
        
    656 & ए॒व   ।   वि॒शि   ।   अधीति॑   ।   ऊ॒ह॒ति॒   ।    & TS\_3.5.7.3       \\
    
    \hline
        
    657 & ए॒व   ।   वै॒श्वा॒न॒रम्   ।   अवेति॑   ।   रु॒न्धे॒   ।    & TS\_5.4.7.7       \\
    
    \hline
        
    658 & ए॒व   ।   शꣳ॒॒स॒ति॒   ।   अथ॑   ।   या   ।    & TS\_7.1.5.6       \\
    
    \hline
        
    659 & ए॒व   ।   स्वेन॑   ।   भा॒ग॒धेये॒नेति॑ भाग{-}धेये॑न   ।   उपेति॑   ।    & TS\_2.1.2.2 TS\_2.2.1.2 TS\_2.4.4.2       \\
    
    \hline
        
    660 & ए॒वम्   ।   वि॒द्वान्   ।   अ॒ग्निम्   ।   उ॒प॒तिष्ठ॑त॒ इत्यु॑प{-}तिष्ठ॑ते   ।    & TS\_1.5.9.5       \\
    
    \hline
        
    661 & ए॒वम्   ।   वि॒द्वान्   ।   द॒र्॒.श॒पू॒र्ण॒मा॒साविति॑ दर्.श{-}पू॒र्ण॒मा॒सौ   ।   यज॑ते   ।    & TS\_1.6.9.3       \\
    
    \hline
        
    662 & ए॒वम्   ।   वेद॑   ।   हन्ति॑   ।   क्षुध᳚म्   ।    & TS\_2.4.12.7       \\
    
    \hline
        
    663 & ए॒षः   ।   ते॒   ।   गा॒य॒त्रः   ।   भा॒गः   ।    & TS\_3.1.2.1       \\
    
    \hline
        
    664 & ए॒षः   ।   पतिः॑   ।   विश्वा॑नि   ।   अ॒भीति॑   ।    & TS\_6.1.11.5       \\
    
    \hline
        
    665 & ए॒षः   ।   य॒ज्ञ्ः   ।   यत्   ।   त्रै॒धा॒त॒वीय᳚म्   ।    & TS\_2.4.11.3       \\
    
    \hline
        
    666 & ए॒षः   ।   य॒ज्ञ्ः   ।   यत्   ।   द॒श॒रा॒त्र इति॑ दश{-}रा॒त्रः   ।    & TS\_7.2.5.3       \\
    
    \hline
        
    667 & ए॒षः   ।   य॒ज्ञ्स्य॑   ।   अ॒न्वा॒र॒म्भ इत्य॑नु{-} आ॒र॒म्भः   ।   अन॑वच्छित्त्या॒ इत्यन॑व{-}छि॒त्त्यै॒   ।    & TS\_6.2.1.2       \\
    
    \hline
        
    668 & ए॒षः   ।   वै   ।   अ॒ग्नेः   ।   योनिः॑   ।    & TS\_3.4.10.5       \\
    
    \hline
        
    669 & ए॒षः   ।   वै   ।   आ॒प्तः   ।   द्वा॒द॒शा॒ह इति॑ द्वादश{-}अ॒हः   ।    & TS\_7.3.3.1       \\
    
    \hline
        
    670 & ए॒षः   ।   वै   ।   दे॒व॒र॒थ इति॑ देव{-}र॒थः   ।   यत्   ।    & TS\_2.5.6.1       \\
    
    \hline
        
    671 & ए॒षः   ।   वै   ।   विव॑स्वान्   ।   आ॒दि॒त्यः   ।    & TS\_6.5.6.5       \\
    
    \hline
        
    672 & ए॒षा   ।   वै   ।   दे॒वाना᳚म्   ।   विक्रा᳚न्ति॒रिति॒ वि{-}क्रा॒न्तिः॒   ।    & TS\_2.5.6.2       \\
    
    \hline
        
    673 & ए॒षा   ।   वै   ।   प्र॒जाप॑ते॒रिति॑ प्र॒जा{-}प॒तेः॒   ।   अ॒ति॒मो॒क्षिणीत्य॑ति{-}मो॒क्षिणी᳚   ।    & TS\_6.6.9.2       \\
    
    \hline
        
    674 & ए॒षाम्   ।   लो॒काना᳚म्   ।   अन्विति॑   ।   सन्त॑त्या॒ इति॒ सं{-}त॒त्यै॒   ।    & TS\_7.1.5.5       \\
    
    \hline
        
    675 & ए॒षाम्   ।   वै   ।   ए॒तत्   ।   लो॒काना᳚म्   ।    & TS\_5.2.9.1       \\
    
    \hline
        
    676 & ऐ॒त्   ।   तम्   ।   यक्ष्मः॑   ।   आ॒र्च्छ॒त्   ।    & TS\_2.3.5.2       \\
    
    \hline
        
    677 & ऐ॒न्द्रः   ।   यत्   ।   पृश्निः॑   ।   तेन॑   ।    & TS\_2.1.3.3       \\
    
    \hline
        
    678 & ऐ॒न्द्रम्   ।   एका॑दशकपाल॒मित्येका॑दश{-}क॒पा॒ल॒म्   ।   निरिति॑   ।   व॒पे॒त्   ।    & TS\_2.2.11.1       \\
    
    \hline
        
    679 & ऐ॒न्द्रम्   ।   च॒रुम्   ।   निरिति॑   ।   व॒पे॒त्   ।    & TS\_2.2.7.1       \\
    
    \hline
        
    680 & ऐ॒न्द्रा॒ग्नमित्यै᳚न्द्र{-}अ॒ग्नम्   ।   एका॑दशकपाल॒मित्येका॑दश{-}क॒पा॒ल॒म्   ।   मा॒रु॒तीम्   ।   आ॒मिक्षा᳚म्   ।    & TS\_1.8.3.1       \\
    
    \hline
        
    681 & ऐ॒न्द्रा॒ग्नमित्यै᳚न्द्र{-}अ॒ग्नम्   ।   द्वाद॑शकपाल॒मिति॒ द्वाद॑श{-}क॒पा॒ल॒म्   ।   वै॒श्व॒दे॒वमिति॑ वैश्व{-}दे॒वम्   ।   च॒रुम्   ।    & TS\_1.8.7.1       \\
    
    \hline
        
    682 & ऐ॒न्द्र॒वा॒य॒व इत्यै᳚न्द्र{-}वा॒य॒वः   ।   गृ॒ह्य॒ते॒   ।   स्वे   ।   ए॒व   ।    & TS\_7.2.8.3       \\
    
    \hline
        
    683 & ऐ॒न्द्र॒वा॒य॒वाग्रा॒नित्यै᳚न्द्रवाय॒व{-}अ॒ग्रा॒न्   ।   गृ॒ह्णी॒या॒त्   ।   यः   ।   का॒मये॑त   ।    & TS\_7.2.7.1       \\
    
    \hline
        
    684 & ओजः॑   ।   उद्भृ॑त॒मित्युत्{-}भृ॒त॒म्   ।   वन॒स्पति॑भ्य॒ इति॒ वन॒स्पति॑{-}भ्यः॒   ।   परीति॑   ।    & TS\_4.6.6.6       \\
    
    \hline
        
    685 & ओजः॑   ।   ए॒व   ।   उ॒त्त॒र॒त इत्यु॑त्{-}त॒र॒तः   ।   ध॒त्ते॒   ।    & TS\_5.3.4.3       \\
    
    \hline
        
    686 & ओजः॑   ।   ग्री॒वाभिः॑   ।   निर्.ऋ॑ति॒मिति॒ निः{-}ऋ॒ति॒म्   ।   अ॒स्थभि॒रित्य॒स्थ{-}भिः॒   ।    & TS\_5.7.18.1       \\
    
    \hline
        
    687 & ओजः॑   ।   वै   ।   वी॒र्य᳚म्   ।   पृ॒ष्ठानि॑   ।    & TS\_7.3.5.3       \\
    
    \hline
        
    688 & ओज॑से   ।   जु॒हो॒मि॒   ।   ओ॒जो॒विदित्यो॑जः{-}वित्   ।   अ॒सि॒   ।    & TS\_3.3.1.2       \\
    
    \hline
        
    689 & ओमा॑सः   ।   च॒र्॒.ष॒णी॒धृ॒त॒ इति॑ चर्.षणि{-}धृ॒तः॒   ।   विश्वे᳚   ।   दे॒वा॒सः॒   ।    & TS\_1.4.16.1       \\
    
    \hline
        
    690 & ओष॑धयः   ।   सोम॑राज्ञी॒रिति॒ सोम॑{-}रा॒ज्ञीः॒   ।   प्रवि॑ष्टा॒ इति॒ प्र{-}वि॒ष्टाः॒   ।   पृ॒थि॒वीम्   ।    & TS\_4.2.6.5       \\
    
    \hline
        
    691 & ओष॑धीभि॒रित्योष॑धि{-}भिः॒   ।   सः   ।   अ॒हम्   ।   वाज᳚म्   ।    & TS\_4.7.12.3       \\
    
    \hline
        
    692 & ओष॑धीभ्य॒ इत्योष॑धि{-}भ्यः॒   ।   स्वाहा᳚   ।   मूले᳚भ्यः   ।   स्वाहा᳚   ।    & TS\_7.3.19.1       \\
    
    \hline
        
    693 & कः   ।   अ॒द्य   ।   यु॒ङ्क्ते॒   ।   धु॒रि   ।    & TS\_4.2.11.3       \\
    
    \hline
        
    694 & कः   ।   त्वा॒   ।   छ्य॒ति॒   ।   कः   ।    & TS\_5.2.12.1       \\
    
    \hline
        
    695 & कः   ।   त्वा॒   ।   यु॒न॒क्ति॒   ।   सः   ।    & TS\_7.5.13.1       \\
    
    \hline
        
    696 & कक्षे॑षु   ।   अ॒घा॒यव॒ इत्य॑घ{-}यवः॑   ।   तान्   ।   ते॒   ।    & TS\_4.1.10.3       \\
    
    \hline
        
    697 & कण्वाः᳚   ।   अ॒भि   ।   प्रेति॑   ।   गा॒य॒त॒   ।    & TS\_4.3.13.7       \\
    
    \hline
        
    698 & कनी॑याꣳसम्   ।   य॒ज्ञ्॒क्र॒तुमिति॑ यज्ञ्{-}क्र॒तुम्   ।   उपेति॑   ।   इ॒या॒त्   ।    & TS\_5.6.8.2       \\
    
    \hline
        
    699 & कर्म॑णे   ।   वा॒म्   ।   दे॒वेभ्यः॑   ।   श॒के॒य॒म्   ।    & TS\_1.1.4.1       \\
    
    \hline
        
    700 & कस्मा᳚त्   ।   स॒त्यात्   ।   गा॒य॒त्री   ।   कनि॑ष्ठा   ।    & TS\_6.1.6.4       \\
    
    \hline
        
    701 & काम᳚म्   ।   अ॒न्यस्य॑   ।   सः   ।   स्त्री॒षꣳ॒॒सा॒दमिति॑ स्त्री{-}सꣳ॒॒सा॒दम्   ।    & TS\_2.5.1.5       \\
    
    \hline
        
    702 & कार्.षिः॑   ।   अ॒सि॒   ।   इति॑   ।   आ॒ह॒   ।    & TS\_6.4.3.4       \\
    
    \hline
        
    703 & काष्ठा᳚म्   ।   ग॒च्छ॒त॒   ।   वाजे॑वाज॒ इति॒ वाजे᳚{-}वा॒जे॒   ।   अ॒व॒त॒   ।    & TS\_1.7.8.2       \\
    
    \hline
        
    704 & का॒मये॑त   ।   प॒शु॒मानिति॑ पशु{-}मान्   ।   स्या॒त्   ।   इति॑   ।    & TS\_1.7.1.4       \\
    
    \hline
        
    705 & का॒मये॑त   ।   वसी॑यान्   ।   स्या॒त्   ।   इति॑   ।    & TS\_5.2.8.4       \\
    
    \hline
        
    706 & का॒ले   ।   आग॑त॒ इत्या{-}ग॒ते॒   ।   वि॒जाय॑त॒ इति॑ वि{-}जाय॑ते   ।   ए॒वम्   ।    & TS\_5.5.1.7       \\
    
    \hline
        
    707 & किम्   ।   अ॒ष्ट॒मेन॑   ।   इति॑   ।   अ॒ष्टाक्ष॑रा॒मित्य॒ष्टा{-}अ॒क्ष॒रा॒म्   ।    & TS\_7.3.2.2       \\
    
    \hline
        
    708 & किम्   ।   च॒न   ।   हि॒नस्ति॑   ।   सः   ।    & TS\_2.6.8.6       \\
    
    \hline
        
    709 & किम्   ।   स्वि॒त्   ।   आ॒सी॒त्   ।   पू॒र्वचि॑त्ति॒रिति॑ पू॒र्व{-}चि॒त्तिः॒   ।    & TS\_7.4.18.1       \\
    
    \hline
        
    710 & कु॒रु॒ते॒   ।   तस्मा᳚त्   ।   आ॒हुः॒   ।   यः   ।    & TS\_2.6.1.7       \\
    
    \hline
        
    711 & कु॒रु॒ते॒   ।   सं॒ॅव॒थ्स॒र इति॑ सं{-}व॒थ्स॒रे   ।   प॒र्याग॑त॒ इति॑ परि{-}आग॑ते   ।   ए॒ताभिः॑   ।    & TS\_1.6.10.3       \\
    
    \hline
        
    712 & कु॒सु॒रु॒बिन्दः॑   ।   औद्दा॑लकि॒रित्यौत्{-}दा॒ल॒किः॒   ।   अ॒का॒म॒य॒त॒   ।   प॒शु॒मानिति॑ पशु{-}मान्   ।    & TS\_7.2.2.1       \\
    
    \hline
        
    713 & कूप्या᳚भ्यः   ।   स्वाहा᳚   ।   कूल्या᳚भ्यः   ।   स्वाहा᳚   ।    & TS\_7.4.13.1       \\
    
    \hline
        
    714 & कू॒र्मः   ।   मधु॑   ।   वाताः᳚   ।   ऋ॒ता॒य॒त इत्यृ॑त{-}य॒ते   ।    & TS\_5.2.8.6       \\
    
    \hline
        
    715 & कू॒र्मान्   ।   श॒फैः   ।   अ॒च्छला॑भिः   ।   क॒पिञ्ज॑लान्   ।    & TS\_5.7.13.1       \\
    
    \hline
        
    716 & कृत्ति॑काः   ।   नक्ष॑त्रम्   ।   अ॒ग्निः   ।   दे॒वता᳚   ।    & TS\_4.4.10.1       \\
    
    \hline
        
    717 & कृत्वः॑   ।   अ॒भि॒षुत्य॒मित्य॑भि{-}सुत्य᳚म्   ।   ब्र॒ह्म॒वा॒दिन॒ इति॑ ब्रह्म{-}वा॒दिनः॑   ।   व॒द॒न्ति॒   ।    & TS\_6.4.5.3       \\
    
    \hline
        
    718 & कृष्णः॑   ।   अ॒सि॒   ।   आ॒ख॒रे॒ष्ठ इत्या॑खरे{-}स्थः   ।   अ॒ग्नये᳚   ।    & TS\_1.1.11.1       \\
    
    \hline
        
    719 & कृ॒णु॒ष्व   ।   पाजः॑   ।   प्रसि॑ति॒मिति॒ प्र{-}सि॒ति॒म्   ।   न   ।    & TS\_1.2.14.1       \\
    
    \hline
        
    720 & कृ॒ष्णाय॑   ।   स्वाहा᳚   ।   श्वे॒ताय॑   ।   स्वाहा᳚   ।    & TS\_7.3.18.1       \\
    
    \hline
        
    721 & कृ॒ष॒ति॒   ।   त्रि॒वृत॒मिति॑ त्रि{-}वृत᳚म्   ।   ए॒व   ।   य॒ज्ञ्॒मु॒ख इति॑ यज्ञ्{-}मु॒खे   ।    & TS\_5.2.5.5       \\
    
    \hline
        
    722 & केशैः᳚   ।   शिरः॑   ।   छ॒न्नम्   ।   प्रच्यु॑त॒मिति॒ प्र{-}च्यु॒त॒म्   ।    & TS\_2.6.3.5       \\
    
    \hline
        
    723 & क्रतु᳚म्   ।   वरु॑णः   ।   वि॒क्षु   ।   अ॒ग्निम्   ।    & TS\_1.2.8.2       \\
    
    \hline
        
    724 & क्रमः॑   ।   अ॒सि॒   ।   श॒त्रू॒य॒त इति॑ शत्रु{-}य॒तः   ।   ह॒न्ता   ।    & TS\_4.2.1.2       \\
    
    \hline
        
    725 & क्रमैः᳚   ।   अतीति॑   ।   अ॒क्र॒मी॒त्   ।   वा॒जी   ।    & TS\_5.7.24.1       \\
    
    \hline
        
    726 & क्रा॒म॒तः॒   ।   तस्मा᳚त्   ।   प्राञ्चौ᳚   ।   यन्तौ᳚   ।    & TS\_6.4.10.3       \\
    
    \hline
        
    727 & क्री॒तः   ।   सोमः॑   ।   उप॑नद्ध॒ इत्युप॑{-}न॒द्धः॒   ।   नमः॑   ।    & TS\_6.1.11.6       \\
    
    \hline
        
    728 & क्री॒ते   ।   सोमे᳚   ।   मै॒त्रा॒व॒रु॒णायेति॑ मैत्रा{-}व॒रु॒णाय॑   ।   द॒ण्डम्   ।    & TS\_6.1.4.2       \\
    
    \hline
        
    729 & क्रु॒द्धः   ।   प॒रो॒वपेति॑ परा{-}उ॒पव॑   ।   म॒न्युना᳚   ।   यत्   ।    & TS\_1.5.3.2       \\
    
    \hline
        
    730 & क्रू॒रम्   ।   इ॒व॒   ।   वै   ।   अ॒स्याः॒   ।    & TS\_5.1.5.1       \\
    
    \hline
        
    731 & क्षये᳚   ।   पा॒थ   ।   दि॒वः   ।   वि॒म॒ह॒स॒ इति॑ वि{-}म॒ह॒सः॒   ।    & TS\_4.2.11.2       \\
    
    \hline
        
    732 & क्षी॒ये॒ते॒   ।   सा॒र॒स्व॒तौ   ।   होमौ᳚   ।   पु॒रस्ता᳚त्   ।    & TS\_3.5.1.4       \\
    
    \hline
        
    733 & क्षु॒रप॒वीति॑ क्षु॒र{-}प॒वि॒   ।   नाम॑   ।   व्र॒तम्   ।   येन॑   ।    & TS\_6.2.5.2       \\
    
    \hline
        
    734 & क्ष॒त्तुः   ।   गृ॒हे   ।   उ॒प॒द्ध्व॒स्त इत्यु॑प{-}ध्व॒स्तः   ।   दक्षि॑णा   ।    & TS\_1.8.9.2       \\
    
    \hline
        
    735 & क्ष॒त्रम्   ।   अन्विति॑   ।   सहः॑   ।   य॒ज॒त्र॒   ।    & TS\_1.6.12.2       \\
    
    \hline
        
    736 & क्ष॒त्रम्   ।   वि॒श्वतः॑   ।   धा॒र॒य॒   ।   इ॒दम्   ।    & TS\_4.4.12.2       \\
    
    \hline
        
    737 & क्ष॒त्रस्य॑   ।   उल्ब᳚म्   ।   अ॒सि॒   ।   क्ष॒त्रस्य॑   ।    & TS\_1.7.9.1       \\
    
    \hline
        
    738 & क॒ण्डू॒येत॑   ।   पा॒म॒न॒म्भावु॑का॒ इति॑ पामनम्{-}भावु॑काः   ।   प्र॒जा इति॑ प्र{-}जाः   ।   स्युः॒   ।    & TS\_6.1.3.8       \\
    
    \hline
        
    739 & क॒दा   ।   च॒न   ।   स्त॒रीः   ।   अ॒सि॒   ।    & TS\_1.4.22.1       \\
    
    \hline
        
    740 & क॒द्रूः   ।   च॒   ।   वै   ।   सु॒प॒र्णीति॑ सु{-}प॒र्णी   ।    & TS\_6.1.6.1       \\
    
    \hline
        
    741 & क॒रो॒ति॒   ।   अथो॒ इति॑   ।   समिति॑   ।   भ॒र॒ति॒   ।    & TS\_6.1.7.2       \\
    
    \hline
        
    742 & क॒रो॒ति॒   ।   यः   ।   सोमे॑न   ।   यज॑ते   ।    & TS\_6.4.8.2       \\
    
    \hline
        
    743 & क॒रो॒ति॒   ।   रू॒पाणि॑   ।   जु॒हो॒ति॒   ।   रू॒पैः   ।    & TS\_7.1.6.8       \\
    
    \hline
        
    744 & क॒र्णाः   ।   त्रयः॑   ।   या॒माः   ।   सौ॒म्याः   ।    & TS\_5.6.15.1       \\
    
    \hline
        
    745 & क॒र्म॒ण्यः॑   ।   सु॒दक्ष॒ इति॑ सु{-}दक्षः॑   ।   यु॒क्तग्रा॒वेति॑ यु॒क्त{-}ग्रा॒वा॒   ।   जाय॑ते   ।    & TS\_3.1.11.2       \\
    
    \hline
        
    746 & क॒ल्प॒य॒ति॒   ।   वै॒ष्ण॒वः   ।   वै   ।   दे॒वत॑या   ।    & TS\_6.3.4.4       \\
    
    \hline
        
    747 & क॒ल्या॒णी   ।   रू॒पस॑मृ॒द्धेति॑ रू॒प{-}स॒मृ॒द्धा॒   ।   सा   ।   स्या॒त्   ।    & TS\_7.1.6.6       \\
    
    \hline
        
    748 & क॒व्य॒वा॒ह॒नेति॑ कव्य{-}वा॒ह॒न॒   ।   पि॒तॄन्   ।   यक्षि॑   ।   ऋ॒ता॒वृध॒ इत्यृ॑त{-}वृधः॑   ।    & TS\_2.6.12.5       \\
    
    \hline
        
    749 & खलु॑   ।   वै   ।   य॒ज्ञेन॑   ।   यज॑मानः   ।    & TS\_7.1.3.3       \\
    
    \hline
        
    750 & खलु॑   ।   वै   ।   र॒क्षो॒हेति॑ रक्षः{-}हा   ।   अग्ने᳚   ।    & TS\_6.1.4.6       \\
    
    \hline
        
    751 & गभे॑र्ण   ।   अवि॑ज्ञाते॒नेत्यवि॑{-}ज्ञा॒ते॒न॒   ।   ब्र॒ह्म॒हेति॑ ब्रह्म{-}हा   ।   अ॒व॒भृ॒थमित्यव॑{-}भृ॒थम्   ।    & TS\_6.5.10.3       \\
    
    \hline
        
    752 & गर्भाः᳚   ।   च॒   ।   मे॒   ।   व॒थ्साः   ।    & TS\_4.7.10.1       \\
    
    \hline
        
    753 & गर्भाः᳚   ।   प्रावृ॑ताः   ।   जा॒य॒न्ते॒   ।   न   ।    & TS\_6.1.3.3       \\
    
    \hline
        
    754 & गाः॒   ।   अपा॑म   ।   सोम᳚म्   ।   अ॒मृताः᳚   ।    & TS\_3.2.5.4       \\
    
    \hline
        
    755 & गावः॑   ।   वै   ।   ए॒तत्   ।   स॒त्रम्   ।    & TS\_7.5.1.1 TS\_7.5.2.1       \\
    
    \hline
        
    756 & गा॒य॒त्रः   ।   वै   ।   ऐ॒न्द्र॒वा॒य॒व इत्यै᳚न्द्र{-}वा॒य॒वः   ।   गा॒य॒त्रम्   ।    & TS\_7.2.8.1       \\
    
    \hline
        
    757 & गा॒य॒त्रम्   ।   ब्र॒ह्म॒व॒र्च॒समिति॑ ब्रह्म{-}व॒र्च॒सम्   ।   गा॒य॒त्रि॒याम्   ।   ए॒व   ।    & TS\_7.2.6.3       \\
    
    \hline
        
    758 & गा॒य॒त्रि॒या   ।   परीति॑   ।   लि॒ख॒ति॒   ।   तेजः॑   ।    & TS\_5.1.3.5       \\
    
    \hline
        
    759 & गा॒य॒त्री   ।   त्रि॒ष्टुप्   ।   जग॑ती   ।   अ॒नु॒ष्टुगित्य॑नु{-}स्तुक्   ।    & TS\_5.2.11.1       \\
    
    \hline
        
    760 & गा॒य॒त्री   ।   यत्   ।   एका॑दशाक्ष॒रेत्येका॑दश{-}अ॒क्ष॒रा॒   ।   तेन॑   ।    & TS\_6.1.2.7       \\
    
    \hline
        
    761 & गा॒य॒त्रेण॑   ।   पु॒रस्ता᳚त्   ।   उपेति॑   ।   ति॒ष्ठ॒ते॒   ।    & TS\_5.5.8.1       \\
    
    \hline
        
    762 & गुहा᳚   ।   हि॒तम्   ।   अन्विति॑   ।   अ॒वि॒न्द॒न्न्   ।    & TS\_4.4.4.3       \\
    
    \hline
        
    763 & गृ॒णा॒हि॒   ।   घृ॒तव॒तीति॑ घृ॒त{-}व॒ती॒   ।   स॒वि॒तः॒   ।   आधि॑पत्यै॒रित्याधि॑{-}प॒त्यैः॒   ।    & TS\_4.4.12.5       \\
    
    \hline
        
    764 & गृ॒ह्णा॒ति॒   ।   ताः   ।   श्वः   ।   भू॒ते   ।    & TS\_1.6.7.2       \\
    
    \hline
        
    765 & गृ॒ह्णा॒ति॒   ।   ध्रु॒वः   ।   अ॒सि॒   ।   ध्रु॒वः   ।    & TS\_2.3.9.3       \\
    
    \hline
        
    766 & गृ॒ह्णा॒मि॒   ।   प्र॒जाभ्य॒ इति॑ प्र{-}जाभ्यः॑   ।   अ॒यम्   ।   प॒श्चात्   ।    & TS\_4.3.2.2       \\
    
    \hline
        
    767 & गृ॒ह्णी॒यात्   ।   प्र॒जामिति॑ प्र{-}जाम्   ।   प॒शून्   ।   अ॒स्य॒   ।    & TS\_6.6.11.4       \\
    
    \hline
        
    768 & गृ॒ह्णी॒यात्   ।   प्र॒त्यञ्च᳚म्   ।   य॒ज्ञ्म्   ।   अ॒ति॒ग्रा॒ह्या॑ इत्य॑ति{-}ग्रा॒ह्याः᳚   ।    & TS\_6.6.8.2       \\
    
    \hline
        
    769 & गृ॒ह्य॒ते॒   ।   इ॒यम्   ।   वै   ।   अदि॑तिः   ।    & TS\_7.5.4.2       \\
    
    \hline
        
    770 & गो॒पी॒थाय॑   ।   यदि॑   ।   नश्ये᳚त्   ।   आ॒श्वि॒नम्   ।    & TS\_2.6.3.6       \\
    
    \hline
        
    771 & गो॒ष्ठ इति॑ गो{-}स्थे   ।   रि॒री॒हि॒   ।   प्र॒जाप॑ति॒रिति॑ प्र॒जा{-}प॒तिः॒   ।   मह्य᳚म्   ।    & TS\_7.4.17.2       \\
    
    \hline
        
    772 & गौः   ।   अ॒सौ   ।   आयुः॑   ।   इ॒मान्   ।    & TS\_7.3.7.3       \\
    
    \hline
        
    773 & गौः   ।   अ॒सौ   ।   आयुः॑   ।   ए॒षु   ।    & TS\_7.3.6.2       \\
    
    \hline
        
    774 & ग्रहान्॑   ।   वै   ।   अन्विति॑   ।   प्र॒जा इति॑ प्र{-}जाः   ।    & TS\_6.5.10.1       \\
    
    \hline
        
    775 & ग्रह᳚म्   ।   स॒ह   ।   ऊ॒र्जा   ।   गृ॒ह्णा॒मि॒   ।    & TS\_3.2.6.2       \\
    
    \hline
        
    776 & ग॒च्छ॒   ।   गो॒स्थान॒मिति॑ गो{-}स्थान᳚म्   ।   वर्.ष॑तु   ।   ते॒   ।    & TS\_1.1.9.2       \\
    
    \hline
        
    777 & ग॒च्छ॒   ।   ततः॑   ।   नः॒   ।   वृष्टि᳚म्   ।    & TS\_1.1.13.2       \\
    
    \hline
        
    778 & ग॒च्छ॒ति॒   ।   न॒व॒रा॒त्र इति॑ नव{-}रा॒त्रः   ।   भ॒व॒ति॒   ।   अ॒भि॒पू॒र्वमित्य॑भि{-}पू॒र्वम्   ।    & TS\_7.2.4.3       \\
    
    \hline
        
    779 & ग॒च्छ॒न्ति॒   ।   पृ॒ष्ठानि॑   ।   हि   ।   दैवी᳚   ।    & TS\_7.4.2.3       \\
    
    \hline
        
    780 & ग॒णाः   ।   मे॒   ।   मा   ।   वीति॑   ।    & TS\_3.1.8.2       \\
    
    \hline
        
    781 & ग॒न्ध॒र्वः   ।   तस्य॑   ।   ऋ॒ख्सा॒मानीत्यृ॑क्{-}सा॒मानि॑   ।   अ॒फ्स॒रसः॑   ।    & TS\_3.4.7.2       \\
    
    \hline
        
    782 & ग॒मद्ध्ये᳚   ।   गावः॑   ।   यत्र॑   ।   भूरि॑शृङ्गा॒ इति॒ भूरि॑{-}शृ॒ङ्गाः॒   ।    & TS\_1.3.6.2       \\
    
    \hline
        
    783 & ग॒म॒य॒न्ति॒   ।   यदि॑   ।   न   ।   अ॒व॒गच्छे॒दित्य॑व{-}गच्छे᳚त्   ।    & TS\_2.3.1.5       \\
    
    \hline
        
    784 & घा॒सम्   ।   प्रेति॑   ।   य॒च्छे॒त्   ।   प्र॒हृत्येति॑ प्र{-}हृत्य॑   ।    & TS\_6.5.9.3       \\
    
    \hline
        
    785 & घृ॒तम्   ।   निरिति॑   ।   पि॒ब॒ति॒   ।   आयुः॑   ।    & TS\_2.3.11.5       \\
    
    \hline
        
    786 & घ्नन्ति॑   ।   वा   ।   ए॒तत्   ।   सोम᳚म्   ।    & TS\_6.6.7.1       \\
    
    \hline
        
    787 & घ्रा॒प॒य॒ति॒   ।   प्रा॒णमिति॑ प्र{-}अ॒नम्   ।   ए॒व   ।   अ॒स्या॒म्   ।    & TS\_5.3.2.2       \\
    
    \hline
        
    788 & चक्षु॑षी॒ इति॑   ।   वै   ।   ए॒ते इति॑   ।   य॒ज्ञ्स्य॑   ।    & TS\_2.6.2.1       \\
    
    \hline
        
    789 & चतु॑र्विꣳशति॒मिति॒ चतुः॑{-}विꣳ॒॒श॒ति॒म्   ।   अन्विति॑   ।   ब्रू॒या॒त्   ।   ब्र॒ह्म॒व॒र्च॒सका॑म॒स्येति॑ ब्रह्मवर्च॒स{-}का॒म॒स्य॒   ।    & TS\_2.5.10.3       \\
    
    \hline
        
    790 & चतु॑ष्कपाला॒ इति॒ चतुः॑{-}क॒पा॒लाः॒   ।   भ॒व॒न्ति॒   ।   चतु॑ष्पा॒दिति॒ चतुः॑{-}पा॒त्   ।   हि   ।    & TS\_2.3.12.2       \\
    
    \hline
        
    791 & चतु॑ष्पद॒ इति॒ चतुः॑{-}प॒दः॒   ।   ए॒व   ।   प॒शून्   ।   अवेति॑   ।    & TS\_6.6.11.6       \\
    
    \hline
        
    792 & चतु॑ष्पाद॒ इति॒ चतुः॑{-}पा॒दः॒   ।   हि   ।   प॒शवः॑   ।   यम्   ।    & TS\_6.3.11.5       \\
    
    \hline
        
    793 & चर्म॑   ।   अवेति॑   ।   भि॒न्द॒न्ति॒   ।   पा॒प्मान᳚म्   ।    & TS\_7.5.10.1       \\
    
    \hline
        
    794 & चात्वा॑लात्   ।   धिष्णि॑यान्   ।   उपेति॑   ।   व॒प॒ति॒   ।    & TS\_6.3.1.1       \\
    
    \hline
        
    795 & चा॒तय॑मानाः   ।   अ॒श्व॒त्थे   ।   वः॒   ।   नि॒षद॑न॒मिति॑ नि{-}सद॑नम्   ।    & TS\_4.2.6.2       \\
    
    \hline
        
    796 & चित्ति᳚म्   ।   जु॒हो॒मि॒   ।   मन॑सा   ।   घृ॒तेन॑   ।    & TS\_5.7.4.1       \\
    
    \hline
        
    797 & चि॒त्   ।   प॒था   ।   श्चोत॑न्ति   ।   कोशाः᳚   ।    & TS\_4.3.13.8       \\
    
    \hline
        
    798 & चि॒त्   ।   हि   ।   ते॒   ।   विशः॑   ।    & TS\_3.4.11.6       \\
    
    \hline
        
    799 & चि॒त्तम्   ।   च॒   ।   चित्तिः॑   ।   च॒   ।    & TS\_3.4.4.1       \\
    
    \hline
        
    800 & चि॒त्तम्   ।   स॒तां॒नेनेति॑ सं{-}ता॒नेन॑   ।   भ॒वम्   ।   य॒क्ना   ।    & TS\_1.4.36.1       \\
    
    \hline
        
    801 & चि॒त्रया᳚   ।   य॒जे॒त॒   ।   प॒शुका॑म॒ इति॑ प॒शु{-}का॒मः॒   ।   इ॒यम्   ।    & TS\_2.4.6.1       \\
    
    \hline
        
    802 & चि॒नु॒ते॒   ।   यः   ।   वै   ।   प्र॒जाप॑तय॒ इति॑ प्र॒जा{-} प॒त॒ये॒   ।    & TS\_5.7.1.2       \\
    
    \hline
        
    803 & ची॒य॒ते॒   ।   प॒शु॒मानिति॑ पशु{-}मान्   ।   ए॒व   ।   भ॒व॒ति॒   ।    & TS\_5.5.2.3       \\
    
    \hline
        
    804 & ची॒य॒ते॒   ।   यत्   ।   अ॒ग्निः   ।   येन॑   ।    & TS\_5.7.2.3       \\
    
    \hline
        
    805 & च्या॒व॒य॒ति॒   ।   वृष्णः॑   ।   अश्व॑स्य   ।   स॒दांन॒मिति॑ सं{-}दान᳚म्   ।    & TS\_2.4.9.4       \\
    
    \hline
        
    806 & च॒   ।   ए॒नाः॒   ।   ए॒वम्   ।   वेद॑   ।    & TS\_5.6.2.4       \\
    
    \hline
        
    807 & च॒   ।   क॒पाला॑नि   ।   च॒   ।   अ॒ग्नि॒हो॒त्र॒हव॒णीत्य॑ग्निहोत्र{-}हव॑नी   ।    & TS\_1.6.8.3       \\
    
    \hline
        
    808 & च॒   ।   नः॒   ।   सर्व॑वीरा॒मिति॒ सर्व॑{-}वी॒रा॒म्   ।   नीति॑   ।    & TS\_1.7.10.2       \\
    
    \hline
        
    809 & च॒   ।   मे॒   ।   व्या॒न इति॑ वि{-}अ॒नः   ।   च॒   ।    & TS\_4.7.1.2       \\
    
    \hline
        
    810 & च॒   ।   यज॑मानस्य   ।   च॒   ।   प्रा॒णमिति॑ प्र{-}अ॒नम्   ।    & TS\_5.3.7.4       \\
    
    \hline
        
    811 & च॒   ।   यज॑माने   ।   च॒   ।   इति॑   ।    & TS\_3.1.10.3       \\
    
    \hline
        
    812 & च॒   ।   वा॒जिन᳚म्   ।   सोम᳚म्   ।   राजा॑नम्   ।    & TS\_1.7.10.3       \\
    
    \hline
        
    813 & च॒   ।   विश्वे᳚   ।   दे॒वाः   ।   ऋ॒ता॒वृध॒ इत्यृ॑त{-}वृधः॑   ।    & TS\_2.4.14.5       \\
    
    \hline
        
    814 & च॒तुर्भ्य॒ इति॑ च॒तुः{-}भ्यः॒   ।   स्वाहा᳚   ।   अ॒ष्टा॒भ्यः   ।   स्वाहा᳚   ।    & TS\_7.2.15.1       \\
    
    \hline
        
    815 & च॒त॒सृभि॒रिति॑ चत॒सृ{-}भिः॒   ।   समिति॑   ।   भ॒र॒ति॒   ।   च॒त्वारि॑   ।    & TS\_5.1.4.5       \\
    
    \hline
        
    816 & च॒न   ।   प्र॒त्यव॑रोहे॒दिति॑ प्रति{-}अव॑रोहेत्   ।   न   ।   हि   ।    & TS\_5.5.4.3       \\
    
    \hline
        
    817 & च॒न्द्रमाः᳚   ।   जा॒य॒ते॒   ।   पुनः॑   ।   अ॒ग्निः   ।    & TS\_7.4.18.2       \\
    
    \hline
        
    818 & च॒रित्रा॑य   ।   अ॒ग्निः   ।   त्वा॒   ।   अ॒भीति॑   ।    & TS\_4.2.9.2       \\
    
    \hline
        
    819 & च॒रुः   ।   का॒र्यः॑   ।   अ॒स्मिन्न्   ।   ए॒व   ।    & TS\_5.5.1.6       \\
    
    \hline
        
    820 & च॒रुम्   ।   यः   ।   का॒मये॑त   ।   हिर॑ण्यम्   ।    & TS\_2.3.2.4       \\
    
    \hline
        
    821 & च॒र्॒.ष॒णीः   ।   अ॒भीति॑   ।   आ॒सा   ।   वाजे॑षु   ।    & TS\_1.3.14.7       \\
    
    \hline
        
    822 & च॒र्॒.ष॒णी॒नाम्   ।   प्रेति॑   ।   बु॒द्ध्निया᳚   ।   ई॒र॒ते॒   ।    & TS\_4.3.13.6       \\
    
    \hline
        
    823 & च॒ष्टे॒   ।   घृ॒ताचीः᳚   ।   अ॒न्त॒रा   ।   पूर्व᳚म्   ।    & TS\_4.6.3.4       \\
    
    \hline
        
    824 & छन्दः॑   ।   यत्   ।   द्वाद॑शकपाल॒ इति॒ द्वाद॑श{-}क॒पा॒लः॒   ।   भव॑ति   ।    & TS\_5.6.5.2       \\
    
    \hline
        
    825 & छन्दः॑   ।   शु॒क्रस्य॑   ।   पात्र᳚म्   ।   अ॒सि॒   ।    & TS\_3.1.6.3       \\
    
    \hline
        
    826 & छन्दाꣳ॑सि   ।   उपेति॑   ।   द॒धा॒ति॒   ।   प॒शवः॑   ।    & TS\_5.3.8.1       \\
    
    \hline
        
    827 & छन्दाꣳ॑सि   ।   विषु॑रूपा॒निति॒ विषु॑{-}रू॒पा॒न्   ।   ए॒व   ।   प॒शून्   ।    & TS\_5.3.8.3       \\
    
    \hline
        
    828 & छन्दाꣳ॑सि   ।   वृ॒ङ्क्ते॒   ।   स॒ज॒नीय॒मिति॑ स{-}ज॒नीय᳚म्   ।   शस्य᳚म्   ।    & TS\_7.5.5.2       \\
    
    \hline
        
    829 & छन्द॑साम्   ।   अ॒व॒तु॒   ।   ए॒क॒विꣳ॒॒श इत्ये॑क{-}विꣳ॒॒शः   ।   स्तोमः॑   ।    & TS\_1.8.13.2       \\
    
    \hline
        
    830 & छन्द॑साम्   ।   वी॒र्य᳚म्   ।   एति॑   ।   श्रा॒व॒य॒   ।    & TS\_3.3.7.3       \\
    
    \hline
        
    831 & छागः॑   ।   अ॒सि॒   ।   मम॑   ।   भोगा॑य   ।    & TS\_1.2.3.3       \\
    
    \hline
        
    832 & छि॒न्द्या॒त्   ।   तम्   ।   वि॒च्छिद्य॑मान॒मिति॑ वि{-}छिद्य॑मानम्   ।   अ॒द्ध्व॒र्योः   ।    & TS\_3.2.1.3       \\
    
    \hline
        
    833 & छि॒न्द्या॒म्   ।   इति॑   ।   वि॒ग्राह॒मिति॑ वि{-}ग्राह᳚म्   ।   तस्य॑   ।    & TS\_5.4.8.2       \\
    
    \hline
        
    834 & छि॒न्द॒न्ति॒   ।   यत्   ।   म॒द्ध्य॒तः   ।   प्रा॒श्नन्तीति॑ प्र{-}अ॒श्नन्ति॑   ।    & TS\_2.6.8.3       \\
    
    \hline
        
    835 & छ॒न्दो॒मा इति॑ छन्दः{-}माः   ।   ओज॑सि   ।   ए॒व   ।   वी॒र्ये᳚   ।    & TS\_7.4.2.4       \\
    
    \hline
        
    836 & छ॒न्द॒श्चित॒मिति॑ छन्दः{-}चित᳚म्   ।   चि॒न्वी॒त॒   ।   प॒शुका॑म॒ इति॑ प॒शु{-}का॒मः॒   ।   प॒शवः॑   ।    & TS\_5.4.11.1       \\
    
    \hline
        
    837 & जग॑तीम्   ।   इति॑   ।   अथो॒ इति॑   ।   खलु॑   ।    & TS\_2.6.9.4       \\
    
    \hline
        
    838 & जज्ञि॑   ।   बीज᳚म्   ।   वर्ष्टा᳚   ।   प॒र्जन्यः॑   ।    & TS\_7.5.20.1       \\
    
    \hline
        
    839 & जन॑यः   ।   ताभिः॑   ।   ए॒व   ।   ए॒ना॒म्   ।    & TS\_5.1.7.3       \\
    
    \hline
        
    840 & जाग॑तेन   ।   छन्द॑सा   ।   स॒वि॒त्रा   ।   दे॒वत॑या   ।    & TS\_5.5.8.3       \\
    
    \hline
        
    841 & जा॒तः   ।   स्यात्   ।   तेजः॑   ।   ए॒व   ।    & TS\_1.7.6.6       \\
    
    \hline
        
    842 & जा॒तम्   ।   जा॒तवे॑द॒सीति॑ जा॒त{-}वे॒द॒सि॒   ।   प्रि॒यम्   ।   शि॒शी॒त॒   ।    & TS\_3.5.11.5       \\
    
    \hline
        
    843 & जा॒ना॒ति॒   ।   सु॒म॒तिमिति॑ सु{-}म॒तिम्   ।   य॒वि॒ष्ठ॒   ।   यः   ।    & TS\_1.2.14.3       \\
    
    \hline
        
    844 & जा॒य॒ते॒   ।   असु॑राः   ।   वै   ।   दे॒वान्   ।    & TS\_6.6.4.4       \\
    
    \hline
        
    845 & जा॒य॒न्ते॒   ।   प्रा॒णा इति॑ प्र{-}अ॒नाः   ।   वै   ।   ए॒ते   ।    & TS\_6.3.1.5       \\
    
    \hline
        
    846 & जिघाꣳ॑सात्   ।   द्रु॒हः   ।   पाश᳚म्   ।   प्रतीति॑   ।    & TS\_4.3.13.4       \\
    
    \hline
        
    847 & जि॒तम्   ।   ते॒   ।   द॒क्षि॒ण॒तः   ।   वृ॒ष॒भः   ।    & TS\_2.4.14.2       \\
    
    \hline
        
    848 & जि॒न्व॒   ।   इति॑   ।   आ॒ह॒   ।   प्र॒जास्विति॑ प्र{-}जासु॑   ।    & TS\_3.5.2.5       \\
    
    \hline
        
    849 & जी॒मूत॑स्य   ।   इ॒व॒   ।   भ॒व॒ति॒   ।   प्रती॑कम्   ।    & TS\_4.6.6.1       \\
    
    \hline
        
    850 & जुष्टः॑   ।   वा॒चः   ।   भू॒या॒स॒म्   ।   जुष्टः॑   ।    & TS\_3.1.10.1       \\
    
    \hline
        
    851 & जु॒हु॒या॒त्   ।   शान्तिः॑   ।   ए॒व   ।   ए॒षा   ।    & TS\_5.7.6.5       \\
    
    \hline
        
    852 & जु॒होति॑   ।   यज॑मानम्   ।   ए॒व   ।   उ॒भ॒यतः॑   ।    & TS\_1.5.2.4       \\
    
    \hline
        
    853 & जु॒हो॒ति॒   ।   चत॑स्रः   ।   वै   ।   दिशः॑   ।    & TS\_2.4.9.2       \\
    
    \hline
        
    854 & जु॒हो॒ति॒   ।   य॒ज्ञ्स्य॑   ।   उद्य॑त्या॒ इत्युत्{-}य॒त्यै॒   ।   अ॒नु॒ष्टुबित्य॑नु{-}स्तुप्   ।    & TS\_6.1.2.5       \\
    
    \hline
        
    855 & ज्मन्न्   ।   उपेति॑   ।   वे॒त॒से   ।   अव॑त्तर॒मित्यव॑त्{-}त॒र॒म्   ।    & TS\_4.6.1.2       \\
    
    \hline
        
    856 & ज्या   ।   इ॒यम्   ।   सम॑ने   ।   पा॒रय॑न्ती   ।    & TS\_4.6.6.2       \\
    
    \hline
        
    857 & ज्यैष्ठ्य᳚म्   ।   च॒   ।   मे॒   ।   आधि॑पत्य॒मित्याधि॑{-}प॒त्य॒म्   ।    & TS\_4.7.2.1       \\
    
    \hline
        
    858 & ज्योतिः॑   ।   अ॒धा॒र॒य॒त्   ।   स॒म्राडिति॑ सम्{-}राट्   ।   ज्योतिः॑   ।    & TS\_4.2.9.5       \\
    
    \hline
        
    859 & ज्योतिः॑   ।   ध॒त्ते॒   ।   ज्योति॑ष्मन्तः   ।   अ॒स्मै॒   ।    & TS\_6.6.8.4       \\
    
    \hline
        
    860 & ज्योति॑षा   ।   अ॒भू॒व॒म्   ।   इति॑   ।   आ॒ह॒   ।    & TS\_1.7.6.3       \\
    
    \hline
        
    861 & ज्योति॑ष्टोम॒मिति॒ ज्योतिः॑{-}स्तो॒म॒म्   ।   प्र॒थ॒मम्   ।   उपेति॑   ।   य॒न्ति॒   ।    & TS\_7.4.11.1       \\
    
    \hline
        
    862 & ज्योति॑ष्मतीम्   ।   त्वा॒   ।   सा॒द॒या॒मि॒   ।   ज्यो॒ति॒ष्कृत॒मिति॑ ज्योतिः{-}कृत᳚म्   ।    & TS\_1.4.34.1       \\
    
    \hline
        
    863 & ज्योतीꣳ॑षि   ।   कु॒रु॒ते॒   ।   सु॒व॒र्गस्येति॑ सुवः{-}गस्य॑   ।   लो॒कस्य॑   ।    & TS\_5.4.1.4       \\
    
    \hline
        
    864 & ज॒घास॑   ।   सर्वा᳚   ।   ता   ।   ते॒   ।    & TS\_4.6.9.2       \\
    
    \hline
        
    865 & ज॒न॒ये॒त्   ।   एति॑   ।   व॒ह॒   ।   दे॒वान्   ।    & TS\_2.5.9.4       \\
    
    \hline
        
    866 & ज॒न॒य॒ति॒   ।   सोमे᳚   ।   सोम᳚म्   ।   अ॒भीति॑   ।    & TS\_6.5.7.3       \\
    
    \hline
        
    867 & ज॒मद॑ग्निः   ।   पुष्टि॑काम॒ इति॒ पुष्टि॑{-}का॒मः॒   ।   च॒तू॒रा॒त्रेणेति॑ चतुः{-}रा॒त्रेण॑   ।   अ॒य॒ज॒त॒   ।    & TS\_7.1.9.1       \\
    
    \hline
        
    868 & तत्   ।   आ॒प्नो॒त्   ।   इन्द्रः॑   ।   वः॒   ।    & TS\_5.6.1.3       \\
    
    \hline
        
    869 & तत्   ।   उ॒भय᳚म्   ।   आ॒प्त्वा   ।   अ॒व॒रुद्ध्येत्य॑व{-}रुद्ध्य॑   ।    & TS\_7.3.1.4       \\
    
    \hline
        
    870 & तत्   ।   पुरु॑षम्   ।   अन्विति॑   ।   प॒र्याव॑र्तन्त॒ इति॑ परि{-}आव॑र्तन्ते   ।    & TS\_7.4.11.4       \\
    
    \hline
        
    871 & तत्   ।   प्र॒जा इति॑ प्र{-}जाः   ।   अ॒भीति॑   ।   जि॒घ्र॒ति॒   ।    & TS\_6.4.11.4       \\
    
    \hline
        
    872 & तत्   ।   प॒था   ।   य॒न्ति॒   ।   स॒मा॒नम्   ।    & TS\_7.5.1.6       \\
    
    \hline
        
    873 & तत्   ।   वी॒र्य᳚म्   ।   यज॑मानः   ।   भ्रातृ॑व्यस्य   ।    & TS\_6.5.1.3       \\
    
    \hline
        
    874 & तत्   ।   हिर॑ण्यम्   ।   अ॒भ॒व॒त्   ।   तस्मा᳚त्   ।    & TS\_6.1.7.1       \\
    
    \hline
        
    875 & तपः॑   ।   अ॒स्य॒   ।   निरिति॑   ।   घ्न॒न्ति॒   ।    & TS\_3.1.1.3       \\
    
    \hline
        
    876 & तपसः॑   ।   त॒नूः   ।   अ॒सि॒   ।   प्र॒जाप॑ते॒रिति॑ प्र॒जा{-}प॒तेः॒   ।    & TS\_6.1.10.3       \\
    
    \hline
        
    877 & तप॑सा   ।   च॒   ।   एति॑   ।   अ॒ग॒च्छ॒त्   ।    & TS\_6.1.6.3       \\
    
    \hline
        
    878 & तम्   ।   ते॒   ।   दु॒श्चक्षा॒ इति॑ दुः{-}चक्षाः᳚   ।   मा   ।    & TS\_3.2.10.2       \\
    
    \hline
        
    879 & तम्   ।   निरिति॑   ।   अ॒व॒प॒त्   ।   तेन॑   ।    & TS\_2.5.3.2       \\
    
    \hline
        
    880 & तम्   ।   प्र॒त्नथा᳚   ।   पू॒र्वथा᳚   ।   वि॒श्वथा᳚   ।    & TS\_1.4.9.1       \\
    
    \hline
        
    881 & तम्   ।   भ॒ज॒   ।   सौ॒श्र॒व॒सेषु॑   ।   अ॒ग्ने॒   ।    & TS\_4.2.2.4       \\
    
    \hline
        
    882 & तम्   ।   शृ॒तम्   ।   भू॒तम्   ।   अ॒जु॒हो॒त्   ।    & TS\_6.5.9.2       \\
    
    \hline
        
    883 & तर॑ति   ।   तूर्णिः॑   ।   ह॒व्य॒वाडिति॑ हव्य{-}वाट्   ।   इति॑   ।    & TS\_2.5.9.3       \\
    
    \hline
        
    884 & तव॑   ।   अ॒ग्ने॒   ।   वा॒मीः   ।   अन्विति॑   ।    & TS\_3.5.6.3       \\
    
    \hline
        
    885 & तस्मा᳚त्   ।   आ॒ग्र॒य॒णे   ।   वाक्   ।   वीति॑   ।    & TS\_6.4.11.3       \\
    
    \hline
        
    886 & तस्मा᳚त्   ।   न   ।   अ॒भा॒गम्   ।   निरिति॑   ।    & TS\_2.6.4.2       \\
    
    \hline
        
    887 & तस्मै᳚   ।   उ॒   ।   ह॒व्यम्   ।   घृ॒तव॒दिति॑ घृ॒त{-}व॒त्   ।    & TS\_3.3.11.3       \\
    
    \hline
        
    888 & तस्मै᳚   ।   उ॒क्थ्य᳚म्   ।   ए॒व   ।   प्रेति॑   ।    & TS\_6.5.1.2       \\
    
    \hline
        
    889 & तस्य॑   ।   ते॒   ।   द॒द॒तु॒   ।   येषा᳚म्   ।    & TS\_2.3.10.2       \\
    
    \hline
        
    890 & ताम्   ।   ए॒व   ।   ए॒ना॒न्   ।   अनु॑   ।    & TS\_5.2.5.4       \\
    
    \hline
        
    891 & ताम्   ।   मि॒थु॒ने   ।   अ॒प॒श्य॒न्न्   ।   तस्या᳚म्   ।    & TS\_2.1.9.4       \\
    
    \hline
        
    892 & ताम्   ।   वाव   ।   दे॒वाः   ।   विजि॑ति॒मिति॒ वि{-}जि॒ति॒म्   ।    & TS\_2.4.2.3       \\
    
    \hline
        
    893 & ताम्   ।   सि॒नी॒वा॒ली   ।   सु॒क॒प॒र्देति॑ सु{-}क॒प॒र्दा   ।   सु॒कु॒री॒रेति॑ सु{-}कु॒री॒रा   ।    & TS\_4.1.5.3       \\
    
    \hline
        
    894 & ताम्   ।   हस्ते᳚   ।   नीति॑   ।   अ॒वे॒ष्ट॒य॒त॒   ।    & TS\_6.1.3.7       \\
    
    \hline
        
    895 & ताव॑त्   ।   उपेति॑   ।   आ॒प्नो॒ति॒   ।   यः   ।    & TS\_1.6.9.2       \\
    
    \hline
        
    896 & तिष्ठ॑ति   ।   इ॒व॒   ।   हि   ।   अ॒सौ   ।    & TS\_3.2.9.7       \\
    
    \hline
        
    897 & ति॒ष्ठ॒ति॒   ।   अ॒भीति॑   ।   दिशः॑   ।   ज॒य॒ति॒   ।    & TS\_5.6.4.5       \\
    
    \hline
        
    898 & ति॒ष्ठ॒ति॒   ।   उ॒रु   ।   अ॒न्तरि॑क्षम्   ।   अन्विति॑   ।    & TS\_6.1.11.2       \\
    
    \hline
        
    899 & ती॒र्थम्   ।   ए॒व   ।   स॒मा॒नाना᳚म्   ।   भ॒व॒ति॒   ।    & TS\_6.1.1.3       \\
    
    \hline
        
    900 & तुभ्य᳚म्   ।   इति॑   ।   आ॒ह॒   ।   षट्   ।    & TS\_5.1.5.2       \\
    
    \hline
        
    901 & तू॒ष्णीम्   ।   उपेति॑   ।   द॒धा॒ति॒   ।   न   ।    & TS\_5.2.7.4       \\
    
    \hline
        
    902 & तृ॒तीया᳚त्   ।   पुरु॑षात्   ।   सोम᳚म्   ।   न   ।    & TS\_2.1.5.6       \\
    
    \hline
        
    903 & तृ॒तीय᳚म्   ।   अ॒प॒च॒त्   ।   भोगा॑य   ।   मे॒   ।    & TS\_6.5.6.2       \\
    
    \hline
        
    904 & ते   ।   आ॒त॒स्थुरित्या᳚{-}त॒स्थुः   ।   आ॒त्मान᳚म्   ।   याः   ।    & TS\_4.2.6.4       \\
    
    \hline
        
    905 & ते   ।   पत्न॑यः   ।   लोम॑   ।   वीति॑   ।    & TS\_5.2.11.2       \\
    
    \hline
        
    906 & ते इति॑   ।   ए॒न॒म्   ।   अ॒भि   ।   समिति॑   ।    & TS\_2.5.6.5       \\
    
    \hline
        
    907 & तेज॑सा   ।   सूर्य॑स्य   ।   वर्च॑सा   ।   इन्द्र॑स्य   ।    & TS\_1.8.14.2       \\
    
    \hline
        
    908 & तेज॑स्काम॒स्येति॒ तेजः॑{-}का॒म॒स्य॒   ।   मि॒नु॒या॒त्   ।   त्रि॒वृतेति॑ त्रि{-}वृता᳚   ।   स्तोमे॑न   ।    & TS\_6.2.10.6       \\
    
    \hline
        
    909 & तेन॑   ।   पुꣳस्व॑तीः   ।   तेन॑   ।   सेन्द्रा॒ इति॒ स{-}इ॒न्द्राः॒   ।    & TS\_2.5.8.5       \\
    
    \hline
        
    910 & तेन॑   ।   लो॒कम्   ।   स्पृ॒णो॒ति॒   ।   यत्   ।    & TS\_5.6.5.3       \\
    
    \hline
        
    911 & तेभ्यः॑   ।   उ॒त्त॒र॒वे॒दिरित्यु॑त्तर{-}वे॒दिः   ।   सिꣳ॒॒हीः   ।   रू॒पम्   ।    & TS\_6.2.7.1       \\
    
    \hline
        
    912 & तेषा᳚म्   ।   असु॑राणाम्   ।   ति॒स्रः   ।   पुरः॑   ।    & TS\_6.2.3.1       \\
    
    \hline
        
    913 & तेषा᳚म्   ।   अ॒भिगू᳚र्ति॒रित्य॒भि{-}गू॒र्तिः॒   ।   नः॒   ।   इ॒न्व॒तु॒   ।    & TS\_4.6.8.3       \\
    
    \hline
        
    914 & ते॒   ।   अ॒शी॒य॒   ।   सा   ।   मे॒   ।    & TS\_1.6.3.2       \\
    
    \hline
        
    915 & ते॒   ।   दि॒शा॒मि॒   ।   तेन॑   ।   चि॒न्वा॒नः   ।    & TS\_4.2.10.4       \\
    
    \hline
        
    916 & ते॒   ।   स॒ह   ।   यु॒नज्मि॑   ।   वाच᳚म्   ।    & TS\_3.1.6.2       \\
    
    \hline
        
    917 & ते॒   ।   हिर॑ण्यशृङ्ग॒ इति॒ हिर॑ण्य{-}शृ॒ङ्गः॒   ।   अयः॑   ।   अ॒स्य॒   ।    & TS\_4.6.7.4       \\
    
    \hline
        
    918 & ते॒   ।   हु॒वे॒   ।   स॒व॒   ।   अ॒हम्   ।    & TS\_7.3.11.2       \\
    
    \hline
        
    919 & त्रयः॑   ।   इ॒मे   ।   लो॒काः   ।   इ॒मान्   ।    & TS\_6.6.2.2       \\
    
    \hline
        
    920 & त्रिषा॑हस्र॒ इति॒ त्रि{-}सा॒ह॒स्रः॒   ।   वै   ।   अ॒सौ   ।   लो॒कः   ।    & TS\_5.6.8.3       \\
    
    \hline
        
    921 & त्रि॒पदेति॑ त्रि{-}पदा᳚   ।   गा॒य॒त्री   ।   गा॒य॒त्रम्   ।   प्रा॒त॒स्स॒व॒नमिति॑ प्रातः{-}स॒व॒नम्   ।    & TS\_3.2.9.3       \\
    
    \hline
        
    922 & त्रि॒भ्य इति॑ त्रि{-}भ्यः   ।   स्वाहा᳚   ।   प॒ञ्चभ्य॒ इति॑ प॒ञ्च{-}भ्यः॒   ।   स्वाहा᳚   ।    & TS\_7.2.14.1       \\
    
    \hline
        
    923 & त्रि॒वृदिति॑ त्रि{-}वृत्   ।   अ॒ग्नि॒ष्टो॒म इत्य॑ग्नि{-}स्तो॒मः   ।   भ॒व॒ति॒   ।   तेजः॑   ।    & TS\_7.2.3.2       \\
    
    \hline
        
    924 & त्रि॒वृदिति॑ त्रि{-}वृत्   ।   हि   ।   अ॒ग्निः   ।   यत्   ।    & TS\_5.6.10.3       \\
    
    \hline
        
    925 & त्रिꣳ॒॒शत्   ।   त्रयः॑   ।   च॒   ।   ग॒णिनः॑   ।    & TS\_1.4.11.1       \\
    
    \hline
        
    926 & त्रिꣳ॒॒शद॑क्ष॒रेति॑ त्रिꣳ॒॒शत्{-}अ॒क्ष॒रा॒   ।   वि॒राडिति॑ वि{-}राट्   ।   वि॒राज॒मिति॑ वि{-}राज᳚म्   ।   आ॒प्नो॒ति॒   ।    & TS\_5.6.7.3       \\
    
    \hline
        
    927 & त्रीणि॑   ।   च॒   ।   श॒ताति॑   ।   असृ॑जन्त   ।    & TS\_5.5.2.6       \\
    
    \hline
        
    928 & त्रीणि॑   ।   वाव   ।   सव॑नानि   ।   अथ॑   ।    & TS\_3.2.2.1       \\
    
    \hline
        
    929 & त्रीन्   ।   तृ॒चान्   ।   अन्विति॑   ।   ब्रू॒या॒त्   ।    & TS\_2.5.10.1       \\
    
    \hline
        
    930 & त्रीन्   ।   शि॒ति॒पृ॒ष्ठानिति॑ शिति{-}पृ॒ष्ठान्   ।   श॒रदि॑   ।   अ॒प॒रा॒ह्ण इत्य॑पर{-}अ॒ह्ने   ।    & TS\_2.1.4.2       \\
    
    \hline
        
    931 & त्र्यवि॒रिति॑ त्रि{-}अविः॑   ।   वयः॑   ।   त्रि॒ष्टुप्   ।   छन्दः॑   ।    & TS\_4.3.5.1       \\
    
    \hline
        
    932 & त्वम्   ।   अ॒ग्ने॒   ।   बृ॒हत्   ।   वयः॑   ।    & TS\_3.4.11.1       \\
    
    \hline
        
    933 & त्वम्   ।   अ॒ग्ने॒   ।   रु॒द्रः   ।   असु॑रः   ।    & TS\_1.3.14.1       \\
    
    \hline
        
    934 & त्वम्   ।   तु॒रीया᳚   ।   व॒शिनी᳚   ।   व॒शा   ।    & TS\_3.4.2.2       \\
    
    \hline
        
    935 & त्वम्   ।   नः॒   ।   अन्त॑मः   ।   उ॒त   ।    & TS\_1.5.6.3       \\
    
    \hline
        
    936 & त्वम्   ।   सो॒म॒   ।   त॒नू॒कृद्भ्य॒ इति॑ तनू॒कृत्{-}भ्यः॒   ।   द्वेषो᳚भ्य॒ इति॒ द्वेषः॑{-}भ्यः॒   ।    & TS\_1.3.4.1       \\
    
    \hline
        
    937 & त्वया᳚   ।   श्रोत्र᳚म्   ।   गृ॒ह्णा॒मि॒   ।   प्र॒जाभ्य॒ इति॑ प्र{-}जाभ्यः॑   ।    & TS\_4.3.2.3       \\
    
    \hline
        
    938 & त्वष्टा᳚   ।   इ॒दम्   ।   विश्व᳚म्   ।   भुव॑नम्   ।    & TS\_5.1.11.4       \\
    
    \hline
        
    939 & त्वष्टा᳚   ।   रू॒पाणि॑   ।   वि॒क॒रोतीति॑ वि{-}क॒रोति॑   ।   ता॒व॒च्छ इति॑ तावत्{-}शः   ।    & TS\_1.5.9.2       \\
    
    \hline
        
    940 & त्वष्टा᳚   ।   ह॒तपु॑त्र॒ इति॑ ह॒त{-}पु॒त्रः॒   ।   वीन्द्र॒मिति॒ वि{-}इ॒न्द्र॒म्   ।   सोम᳚म्   ।    & TS\_2.4.12.1 TS\_2.5.2.1       \\
    
    \hline
        
    941 & त्वष्टुः॑   ।   सोम᳚म्   ।   अ॒भी॒षहेत्य॑भि{-}सहा᳚   ।   अ॒पि॒ब॒त्   ।    & TS\_2.3.2.6       \\
    
    \hline
        
    942 & त्वाम्   ।   अ॒ग्ने॒   ।   वृ॒ष॒भम्   ।   चेकि॑तानम्   ।    & TS\_5.7.2.1       \\
    
    \hline
        
    943 & त्वा॒   ।   ओष॑धीभ्य॒ इत्योष॑धि{-}भ्यः॒   ।   प्रेति॑   ।   उ॒क्षा॒मि॒   ।    & TS\_6.3.6.4       \\
    
    \hline
        
    944 & त्वा॒   ।   द॒द्ध्यङ्   ।   ऋषिः॑   ।   पु॒त्रः   ।    & TS\_3.5.11.4       \\
    
    \hline
        
    945 & त्वा॒   ।   ध्रज्यै᳚   ।   पू॒ष्णः   ।   रꣳह्यै᳚   ।    & TS\_1.3.10.2       \\
    
    \hline
        
    946 & त्वा॒   ।   प्रा॒ण इति॑ प्र{-}अ॒ने   ।   सा॒द॒या॒मि॒   ।   इति॑   ।    & TS\_6.4.5.7       \\
    
    \hline
        
    947 & त्वा॒   ।   प्रा॒णमिति॑ प्र{-}अ॒नम्   ।   जि॒न्व॒   ।   य॒न्ता   ।    & TS\_4.4.1.3       \\
    
    \hline
        
    948 & त्वा॒   ।   विभू॑मन॒ इति॒ वि{-}भू॒म॒ने॒   ।   ऋ॒तस्य॑   ।   त्वा॒   ।    & TS\_3.3.5.2       \\
    
    \hline
        
    949 & त्वा॒   ।   स॒त्याय॑   ।   इति॑   ।   आ॒ह॒   ।    & TS\_3.3.5.5       \\
    
    \hline
        
    950 & त्वे इति॑   ।   क्रतु᳚म्   ।   अपीति॑   ।   वृ॒ञ्ज॒न्ति॒   ।    & TS\_3.5.10.1       \\
    
    \hline
        
    951 & त॒नुवा᳚   ।   मे॒   ।   स॒ह   ।   नमः॑   ।    & TS\_5.5.9.3       \\
    
    \hline
        
    952 & त॒नू॒पा इति॑ तनू{-}पाः   ।   हि   ।   ए॒षः   ।   अग्ने᳚   ।    & TS\_1.5.7.5       \\
    
    \hline
        
    953 & त॒रणिः॑   ।   वि॒श्वद॑र्.शत॒ इति॑ वि॒श्व{-}द॒र्.॒श॒तः॒   ।   ज्यो॒ति॒ष्कृदिति॑ ज्योतिः{-}कृत्   ।   अ॒सि॒   ।    & TS\_1.4.31.1       \\
    
    \hline
        
    954 & त॒र॒   ।   यत्र॑   ।   अ॒हम्   ।   अस्मि॑   ।    & TS\_4.1.9.3       \\
    
    \hline
        
    955 & त॒र॒ति॒   ।   तर॑ति   ।   ब्र॒ह्म॒ह॒त्यामिति॑ ब्रह्म{-}ह॒त्याम्   ।   यः   ।    & TS\_5.3.12.2       \\
    
    \hline
        
    956 & दक्षि॑णः   ।   यज॑मानस्य   ।   प॒रि॒धिरिति॑ परि{-}धिः   ।   इ॒डः   ।    & TS\_1.1.11.2       \\
    
    \hline
        
    957 & दक्षि॑णः   ।   यु॒क्तः   ।   भव॑ति   ।   स॒व्यः   ।    & TS\_3.4.10.4       \\
    
    \hline
        
    958 & दक्षि॑णात्   ।   एति॑   ।   उ॒त   ।   स॒व्यात्   ।    & TS\_1.2.13.3       \\
    
    \hline
        
    959 & दधि॑   ।   हि   ।   पूर्व᳚म्   ।   क्रि॒यते᳚   ।    & TS\_2.5.3.5       \\
    
    \hline
        
    960 & दध॑त्   ।   पोष᳚म्   ।   र॒यिम्   ।   मयि॑   ।    & TS\_1.6.6.3       \\
    
    \hline
        
    961 & दब्धिः॑   ।   अ॒सि॒   ।   अद॑ब्धः   ।   भू॒या॒स॒म्   ।    & TS\_1.6.2.4       \\
    
    \hline
        
    962 & दमः॑   ।   ए॒व   ।   अ॒स्य॒   ।   ए॒षः   ।    & TS\_1.5.7.4       \\
    
    \hline
        
    963 & दवी॑यः   ।   अपेति॑   ।   से॒ध॒   ।   शत्रून्॑   ।    & TS\_4.6.6.7       \\
    
    \hline
        
    964 & दश॑   ।   समिति॑   ।   प॒द्य॒न्ते॒   ।   दशा᳚क्ष॒रेति॒ दश॑{-}अ॒क्ष॒रा॒   ।    & TS\_6.1.9.6       \\
    
    \hline
        
    965 & दह॑ति   ।   पु॒ण्य॒सम॒मिति॑ पुण्य{-}सम᳚म्   ।   भ॒व॒ति॒   ।   यदि॑   ।    & TS\_3.3.8.5       \\
    
    \hline
        
    966 & दिक्   ।   स्वाया᳚म्   ।   ए॒व   ।   दि॒शि   ।    & TS\_2.6.6.6       \\
    
    \hline
        
    967 & दिशः॑   ।   ए॒व   ।   ए॒तेन॑   ।   प्रेति॑   ।    & TS\_5.7.8.3       \\
    
    \hline
        
    968 & दि॒वः   ।   परीति॑   ।   प्र॒थ॒मम्   ।   ज॒ज्ञे॒   ।    & TS\_4.2.2.1       \\
    
    \hline
        
    969 & दि॒व्यम्   ।   ग॒च्छ॒   ।   स्वाहा᳚   ।   इति॑   ।    & TS\_6.4.1.4       \\
    
    \hline
        
    970 & दि॒शि   ।   मासाः᳚   ।   पि॒तरः॑   ।   मा॒र्ज॒य॒न्ता॒म्   ।    & TS\_1.6.5.2       \\
    
    \hline
        
    971 & दी॒र्घम्   ।   श्रवः॑   ।   दि॒वि   ।   ऐर॑यन्त   ।    & TS\_2.4.5.2       \\
    
    \hline
        
    972 & दृꣳ॒॒ह॒   ।   श्रोत्र᳚म्   ।   दृꣳ॒॒ह॒   ।   स॒जा॒तानिति॑ स{-}जा॒तान्   ।    & TS\_1.1.7.2       \\
    
    \hline
        
    973 & देवाः᳚   ।   व॒स॒व्याः॒   ।   अग्ने᳚   ।   सो॒म॒   ।    & TS\_2.4.8.1       \\
    
    \hline
        
    974 & देवाः᳚   ।   व॒स॒व्याः॒   ।   देवाः᳚   ।   श॒र्म॒ण्याः॒   ।    & TS\_2.4.10.1       \\
    
    \hline
        
    975 & देवि॑काः   ।   निरिति॑   ।   व॒पे॒त्   ।   प्र॒जाका॑म॒ इति॑ प्र॒जा{-}का॒मः॒   ।    & TS\_3.4.9.1       \\
    
    \hline
        
    976 & देवीः᳚   ।   आ॒पः॒   ।   समिति॑   ।   मधु॑मती॒रिति॒ मधु॑{-} म॒तीः॒   ।    & TS\_1.8.12.1       \\
    
    \hline
        
    977 & देव॑   ।   स॒वि॒तः॒   ।   ए॒तत्   ।   ते॒   ।    & TS\_3.2.7.1       \\
    
    \hline
        
    978 & देव॑   ।   स॒वि॒तः॒   ।   प्रेति॑   ।   सु॒व॒   ।    & TS\_1.7.7.1       \\
    
    \hline
        
    979 & दे॒भुः॒   ।   त्वया᳚   ।   व॒यम्   ।   स॒ध॒न्य॑ इति॑ सध{-}न्यः॑   ।    & TS\_1.2.14.6       \\
    
    \hline
        
    980 & दे॒वः   ।   वः॒   ।   स॒वि॒ता   ।   उदिति॑   ।    & TS\_1.1.5.1       \\
    
    \hline
        
    981 & दे॒वताः᳚   ।   अ॒भ्यारो॑ह॒न्तीत्य॑भि{-}आरो॑हन्ति   ।   त्र॒य॒स्त्रिꣳ॒॒शादिति॑ त्रयः{-}त्रिꣳ॒॒शात्   ।   त्र॒य॒स्त्रिꣳ॒॒शमिति॑ त्रय{-}त्रिꣳ॒॒शम्   ।    & TS\_7.4.3.5       \\
    
    \hline
        
    982 & दे॒वताः᳚   ।   ताः   ।   अ॒री॒र॒धा॒म॒   ।   इति॑   ।    & TS\_2.6.9.7       \\
    
    \hline
        
    983 & दे॒वताः᳚   ।   सद॑सि   ।   आर्ति᳚म्   ।   आ॒र्पय॒न्तीत्या᳚{-}अ॒र्पय॑न्ति   ।    & TS\_3.2.4.3       \\
    
    \hline
        
    984 & दे॒वता॒तेति॑ दे॒व{-}ता॒ता॒   ।   य॒ज्ञेभिः॑   ।   सू॒नो॒ इति॑   ।   स॒ह॒सः॒   ।    & TS\_4.3.13.3       \\
    
    \hline
        
    985 & दे॒वता᳚   ।   श॒तभि॑ष॒गिति॑ श॒त{-}भि॒ष॒क्   ।   नक्ष॑त्रम्   ।   इन्द्रः॑   ।    & TS\_4.4.10.3       \\
    
    \hline
        
    986 & दे॒वता᳚म्   ।   उपेति॑   ।   ए॒ति॒   ।   यः   ।    & TS\_5.1.9.4       \\
    
    \hline
        
    987 & दे॒वत॑या   ।   समृ॑द्ध्या॒ इति॒ सं{-}ऋ॒द्ध्यै॒   ।   वै॒श्व॒दे॒वमिति॑ वैश्व{-}दे॒वम्   ।   ब॒हु॒रू॒पमिति॑ बहु{-}रू॒पम्   ।    & TS\_2.1.6.4       \\
    
    \hline
        
    988 & दे॒वस्य॑   ।   अ॒हम्   ।   स॒वि॒तुः   ।   प्र॒स॒व इति॑ प्र{-}स॒वे   ।    & TS\_1.7.8.1       \\
    
    \hline
        
    989 & दे॒वस्य॑   ।   त्वा॒   ।   स॒वि॒तुः   ।   प्र॒स॒व इति॑ प्र {-}स॒वे   ।    & TS\_6.2.10.1       \\
    
    \hline
        
    990 & दे॒वस्य॑   ।   त्वा॒   ।   स॒वि॒तुः   ।   प्र॒स॒व इति॑ प्र{-}स॒वे   ।    & TS\_1.3.1.1 TS\_2.6.4.1 TS\_4.1.3.1 TS\_5.1.4.1 TS\_6.4.4.1 TS\_7.1.11.1       \\
    
    \hline
        
    991 & दे॒वा   ।   वै   ।   इ॒न्द्रि॒यम्   ।   वी॒र्या᳚म्   ।    & TS\_6.6.8.1       \\
    
    \hline
        
    992 & दे॒वाः   ।   इ॒माम्   ।   वाच᳚म्   ।   अ॒भीति॑   ।    & TS\_1.1.13.3       \\
    
    \hline
        
    993 & दे॒वाः   ।   तृ॒तीय᳚म्   ।   सव॑नम्   ।   न   ।    & TS\_6.5.7.2       \\
    
    \hline
        
    994 & दे॒वाः   ।   त्वा॒   ।   इन्द्र॑ज्येष्ठा॒ इतीन्द्र॑{-}ज्ये॒ष्ठाः॒   ।   वरु॑णराजान॒ इति॒ वरु॑ण{-}रा॒जा॒नः॒   ।    & TS\_5.5.9.5       \\
    
    \hline
        
    995 & दे॒वाः   ।   प्रा॒तः   ।   म॒द्ध्यन्दि॑ने   ।   सा॒यम्   ।    & TS\_6.2.5.4       \\
    
    \hline
        
    996 & दे॒वाः   ।   म॒नु॒ष्याः᳚   ।   पि॒तरः॑   ।   ते   ।    & TS\_2.4.1.1       \\
    
    \hline
        
    997 & दे॒वाः   ।   वै   ।   दे॒व॒यज॑न॒मिति॑ देव{-}यज॑नम्   ।   अ॒द्ध्य॒व॒सायेत्य॑धि{-}अ॒व॒साय॑   ।    & TS\_6.1.5.1       \\
    
    \hline
        
    998 & दे॒वाः   ।   वै   ।   न   ।   ऋ॒चि   ।    & TS\_2.5.7.1       \\
    
    \hline
        
    999 & दे॒वाः   ।   वै   ।   प्र॒बाहु॒गिति॑ प्र{-}बाहु॑क्   ।   ग्रहान्॑   ।    & TS\_6.6.10.1       \\
    
    \hline
        
    1000 & दे॒वाः   ।   वै   ।   मृ॒त्योः   ।   अ॒बि॒भ॒युः॒   ।    & TS\_2.3.2.1       \\
    
    \hline
        
    1001 & दे॒वाः   ।   वै   ।   यत्   ।   य॒ज्ञे   ।    & TS\_3.4.6.1 TS\_5.3.3.1 TS\_6.4.6.1 TS\_6.4.11.1 TS\_6.6.9.1       \\
    
    \hline
        
    1002 & दे॒वाः   ।   वै   ।   यत्   ।   य॒ज्ञेन॑   ।    & TS\_3.3.6.1       \\
    
    \hline
        
    1003 & दे॒वाः   ।   वै   ।   य॒ज्ञ्म्   ।   आग्नी᳚द्ध्र॒ इत्याग्नि॑{-}इ॒द्ध्रे॒   ।    & TS\_6.4.2.1       \\
    
    \hline
        
    1004 & दे॒वाः   ।   वै   ।   य॒ज्ञ्स्य॑   ।   स्व॒गा॒क॒र्तार॒मिति॑ स्वगा{-}क॒र्तार᳚म्   ।    & TS\_2.6.10.1       \\
    
    \hline
        
    1005 & दे॒वाः   ।   वै   ।   रा॒ज॒न्या᳚त्   ।   जाय॑मानात्   ।    & TS\_2.4.13.1       \\
    
    \hline
        
    1006 & दे॒वाः   ।   वै   ।   स॒त्रम्   ।   आ॒स॒त॒   ।    & TS\_2.3.3.1       \\
    
    \hline
        
    1007 & दे॒वाना᳚म्   ।   ऊ॒द्‌र्ध्वम्   ।   र॒श॒नायाः᳚   ।   एति॑   ।    & TS\_6.3.4.7       \\
    
    \hline
        
    1008 & दे॒वाना᳚म्   ।   ए॒ति॒   ।   नि॒ष्कृ॒तमिति॑ निः{-}कृ॒तम्   ।   ऋ॒तस्य॑   ।    & TS\_1.3.4.2       \\
    
    \hline
        
    1009 & दे॒वाना᳚म्   ।   तम्   ।   ए॒व   ।   अ॒स्यै॒   ।    & TS\_6.1.7.8       \\
    
    \hline
        
    1010 & दे॒वाना᳚म्   ।   वै   ।   अन्त᳚म्   ।   ज॒ग्मुषा᳚म्   ।    & TS\_7.5.8.1       \\
    
    \hline
        
    1011 & दे॒वान्   ।   उ॒पैतीत्यु॑प{-}एति॑   ।   इ॒दम्   ।   अ॒हम्   ।    & TS\_6.3.2.5       \\
    
    \hline
        
    1012 & दे॒वा॒सु॒रा इति॑ देव{-}अ॒सु॒राः   ।   ए॒षु   ।   लो॒केषु॑   ।   अ॒स्प॒र्द्ध॒न्त॒   ।    & TS\_2.1.3.1       \\
    
    \hline
        
    1013 & दे॒वा॒सु॒रा इति॑ देव{-}अ॒सु॒राः   ।   संॅय॑त्ता॒ इति॑ सं{-}य॒त्ताः॒   ।   आ॒स॒न्न्   ।   ते   ।    & TS\_1.5.1.1       \\
    
    \hline
        
    1014 & दे॒वा॒सु॒रा इति॑ देव{-}अ॒सु॒राः   ।   संॅय॑त्ता॒ इति॒ सं{-}य॒त्ताः॒   ।   आ॒स॒न्न्   ।   कनी॑याꣳसः   ।    & TS\_5.3.11.1       \\
    
    \hline
        
    1015 & दे॒वा॒सु॒रा इति॑ देव{-}अ॒सु॒राः   ।   संॅय॑त्ता॒ इति॒ सं{-}य॒त्ताः॒   ।   आ॒स॒न्न्   ।   तान्   ।    & TS\_2.3.7.1       \\
    
    \hline
        
    1016 & दे॒वा॒सु॒रा इति॑ देव{-}अ॒सु॒राः   ।   संॅय॑त्ता॒ इति॒ सं{-}य॒त्ताः॒   ।   आ॒स॒न्न्   ।   ते   ।    & TS\_2.4.2.1 TS\_5.4.1.1 TS\_6.2.2.1       \\
    
    \hline
        
    1017 & दे॒वा॒सु॒रा इति॑ देव{-}अ॒सु॒राः   ।   संॅय॑त्ता॒ इति॒ सं{-}य॒त्ताः॒   ।   आ॒स॒न्न्   ।   तेषा᳚म्   ।    & TS\_2.4.3.1       \\
    
    \hline
        
    1018 & दे॒वीः   ।   वि॒श्वदे᳚व्यावती॒रिति॑ वि॒श्वदे᳚व्य{-}व॒तीः॒   ।   पृ॒थि॒व्याः   ।   स॒धस्थ॒ इति॑ स॒ध{-}स्थे॒   ।    & TS\_4.1.6.2       \\
    
    \hline
        
    1019 & दे॒वेभिः॑   ।   इ॒न्वि॒तम्   ।   पाशा᳚त्   ।   प॒शुम्   ।    & TS\_3.1.4.4       \\
    
    \hline
        
    1020 & दे॒वेभ्यः॑   ।   ए॒व   ।   प्र॒ति॒प्रोच्येति॑ प्रति{-}प्रोच्य॑   ।   य॒ज्ञेन॑   ।    & TS\_1.6.8.4       \\
    
    \hline
        
    1021 & दे॒वेषु   ।   च॒   ।   स॒वि॒तः॒   ।   मानु॑षेषु   ।    & TS\_4.1.11.2       \\
    
    \hline
        
    1022 & दे॒वैः   ।   वसु॑भि॒रिति॒ वसु॑{-}भिः॒   ।   स॒जोषा॒ इति॑ स{-}जोषाः᳚   ।   प्री॒तम्   ।    & TS\_5.1.11.2       \\
    
    \hline
        
    1023 & दे॒व॒पु॒रा इति॑ देव{-}पु॒राः   ।   अ॒प॒श्य॒न्न्   ।   यत्   ।   द॒श॒रा॒त्र इति॑ दश{-}रा॒त्रः   ।    & TS\_7.2.5.4       \\
    
    \hline
        
    1024 & दे॒व॒यानी॒रिति॑ देव{-}यानीः᳚   ।   अ॒न्त॒रा   ।   द्यावा॑पृथि॒वी इति॒ द्यावा᳚{-}पृ॒थि॒वी   ।   वि॒यन्तीति॑ वि{-}यन्ति॑   ।    & TS\_3.5.4.2       \\
    
    \hline
        
    1025 & दे॒व॒लो॒कमिति॑ देव{-}लो॒कम्   ।   च॒   ।   ए॒व   ।   म॒नु॒ष्य॒लो॒कमिति॑ मनुष्य{-}लो॒कम्   ।    & TS\_2.5.11.3       \\
    
    \hline
        
    1026 & दे॒व॒वीत॑म॒ इति॑ देव{-}वीत॑मः   ।   वीति॑   ।   धू॒मम्   ।   अ॒ग्ने॒   ।    & TS\_4.1.3.4       \\
    
    \hline
        
    1027 & दे॒हि॒   ।   व॒र्चो॒दा इति॑ वर्चः{-} दाः   ।   अ॒ग्ने॒   ।   अ॒सि॒   ।    & TS\_1.5.5.4       \\
    
    \hline
        
    1028 & दैवी᳚   ।   सꣳ॒॒सदिति॑ सं{-}सत्   ।   ज्योतिः॑   ।   अ॒ति॒रा॒त्र इत्य॑ति{-}रा॒त्रः   ।    & TS\_7.4.2.2       \\
    
    \hline
        
    1029 & दैवी᳚म्   ।   धिय᳚म्   ।   म॒ना॒म॒हे॒   ।   सु॒मृ॒डी॒कामिति॑ सु{-}मृ॒डी॒काम्   ।    & TS\_1.2.3.1       \\
    
    \hline
        
    1030 & द्याम्   ।   इ॒मम्   ।   च॒   ।   योनि᳚म्   ।    & TS\_4.2.8.3       \\
    
    \hline
        
    1031 & द्यौः   ।   ते॒   ।   पृ॒ष्ठम्   ।   पृ॒थि॒वी   ।    & TS\_5.7.25.1       \\
    
    \hline
        
    1032 & द्य॒ति॒   ।   हृद॑यस्य   ।   अग्रे᳚   ।   अवेति॑   ।    & TS\_6.3.10.4       \\
    
    \hline
        
    1033 & द्रवि॑णम्   ।   स॒प्त॒द॒श इति॑ सप्त{-}द॒शः   ।   स्तोमः॑   ।   सः   ।    & TS\_4.3.3.2       \\
    
    \hline
        
    1034 & द्रापे᳚   ।   अन्ध॑सः   ।   प॒ते॒   ।   दरि॑द्रत्   ।    & TS\_4.5.10.1       \\
    
    \hline
        
    1035 & द्वाभ्या᳚म्   ।   स्वाहा᳚   ।   च॒तुर्भ्य॒ इति॑ च॒तुः{-}भ्यः॒   ।   स्वाहा᳚   ।    & TS\_7.2.13.1       \\
    
    \hline
        
    1036 & द्वा॒द॒शे   ।   मा॒सि   ।   शृङ्गा॑णि   ।   प्रेति॑   ।    & TS\_7.5.2.2       \\
    
    \hline
        
    1037 & द्विभा॑ग॒मिति॒ द्वि{-}भा॒ग॒म्   ।   ब्र॒ह्मणे᳚   ।   तृती॑यम्   ।   अ॒ग्नीध॒ इत्य॑ग्नि{-} इधे᳚   ।    & TS\_7.1.5.7       \\
    
    \hline
        
    1038 & द्वि॒तीयः॑   ।   गह्यः॑   ।   तृ॒तीयः॑   ।   किꣳ॒॒शि॒लः   ।    & TS\_5.5.9.2       \\
    
    \hline
        
    1039 & द्वि॒तीया᳚म्   ।   उ॒प॒धायेत्यु॑प{-}धाय॑   ।   व्या॒नेनेति॑ वि{-}अ॒नेन॑   ।   ए॒व   ।    & TS\_5.5.5.3       \\
    
    \hline
        
    1040 & द्वि॒पाद॒ इति॑ द्वि{-}पादः॑   ।   च॒   ।   चतु॑ष्पाद॒ इति॒ चतुः॑{-}पा॒दः॒   ।   च॒   ।    & TS\_5.2.9.5       \\
    
    \hline
        
    1041 & द्वि॒ष्मः   ।   यः   ।   च॒   ।   नः॒   ।    & TS\_4.4.3.3       \\
    
    \hline
        
    1042 & द्वे इति॑   ।   वाव   ।   दे॒व॒स॒त्रे इति॑ देव{-}स॒त्रे   ।   द्वा॒द॒शा॒ह इति॑ द्वादश{-}अ॒हः   ।    & TS\_7.4.5.1       \\
    
    \hline
        
    1043 & द्वेष्टि॑   ।   तम्   ।   अ॒स्मै॒   ।   प॒शुम्   ।    & TS\_6.6.4.5       \\
    
    \hline
        
    1044 & द्वेष्टि॑   ।   यः   ।   च॒   ।   ए॒न॒म्   ।    & TS\_6.3.9.3       \\
    
    \hline
        
    1045 & द॒क्ष॒क्र॒तुभ्या॒मिति॑ दक्षक्र॒तु{-}भ्या॒म्   ।   चक्षु॑र्भ्या॒मिति॒ चक्षुः॑{-}भ्या॒म्   ।   मे॒   ।   व॒र्चो॒दाविति॑ वर्चः{-}दौ   ।    & TS\_3.2.3.2       \\
    
    \hline
        
    1046 & द॒त्वते᳚   ।   स्वाहा᳚   ।   अ॒द॒न्तका॑य   ।   स्वाहा᳚   ।    & TS\_7.5.12.1       \\
    
    \hline
        
    1047 & द॒दे॒   ।   ऋ॒तस्य॑   ।   धाम्नः॑   ।   अ॒मृत॑स्य   ।    & TS\_4.2.7.2       \\
    
    \hline
        
    1048 & द॒द्ध्या॒त्   ।   क्षोधु॑कः   ।   ए॒व   ।   भ॒व॒ति॒   ।    & TS\_5.2.9.2       \\
    
    \hline
        
    1049 & द॒द्भ्य इति॑ दत्{-}भ्यः   ।   स्वाहा᳚   ।   हनू᳚भ्या॒मिति॒ हनु॑{-}भ्या॒म्   ।   स्वाहा᳚   ।    & TS\_7.3.16.1       \\
    
    \hline
        
    1050 & द॒धा॒ति॒   ।   अ॒ग्नी॒षो॒मीय॒मित्य॑ग्नी{-}सो॒मीय᳚म्   ।   एका॑दशकपाल॒मित्येका॑दश{-}क॒पा॒ल॒म्   ।   निरिति॑   ।    & TS\_2.3.3.3       \\
    
    \hline
        
    1051 & द॒धा॒ति॒   ।   दे॒व॒पु॒रा इति॑ देव{-}पु॒राः   ।   ए॒व   ।   ए॒ताः   ।    & TS\_5.7.3.2       \\
    
    \hline
        
    1052 & द॒धा॒ति॒   ।   नव॑   ।   वै   ।   पुरु॑षे   ।    & TS\_5.3.2.3       \\
    
    \hline
        
    1053 & द॒धा॒ति॒   ।   परा॑चीम्   ।   ए॒व   ।   अ॒स्मा॒त्   ।    & TS\_5.2.4.4       \\
    
    \hline
        
    1054 & द॒धा॒ति॒   ।   ब्र॒ह्म॒व॒र्च॒सीति॑ ब्रह्म{-}व॒र्च॒सी   ।   ए॒व   ।   भ॒व॒ति॒   ।    & TS\_2.1.7.7       \\
    
    \hline
        
    1055 & द॒धा॒मि॒   ।   द्यावा॑पृथि॒वी इति॒ द्यावा᳚{-}पृ॒थि॒वी   ।   अ॒न्तः   ।   उ॒रु   ।    & TS\_6.4.6.2       \\
    
    \hline
        
    1056 & द॒ध॒ति॒   ।   प्र॒जन॑ना॒येति॑ प्र{-}जन॑नाय   ।   ज्योतिः॑   ।   अ॒भितः॑   ।    & TS\_7.5.1.5       \\
    
    \hline
        
    1057 & द॒ध॒ते॒   ।   ब॒हवः॑   ।   षो॒ड॒शिनः॑   ।   भ॒व॒न्ति॒   ।    & TS\_7.4.7.3       \\
    
    \hline
        
    1058 & द॒शभ्य॒ इति॑ द॒श{-}भ्यः॒   ।   स्वाहा᳚   ।   विꣳ॒॒श॒त्यै   ।   स्वाहा᳚   ।    & TS\_7.2.17.1       \\
    
    \hline
        
    1059 & द॒ह॒न्ते॒   ।   प॒ञ्च॒द॒श॒रा॒त्रेणेति॑ पञ्चदश{-}रा॒त्रेण॑   ।   ओजः॑   ।   बल᳚म्   ।    & TS\_7.3.7.2       \\
    
    \hline
        
    1060 & धनुः॑   ।   तत्   ।   वातः॑   ।   अन्विति॑   ।    & TS\_5.5.7.3       \\
    
    \hline
        
    1061 & धाम॑   ।   इति॑   ।   आ॒ह॒   ।   ए॒षा   ।    & TS\_5.7.8.2       \\
    
    \hline
        
    1062 & धा॒त्रे   ।   पु॒रो॒डाश᳚म्   ।   द्वाद॑शकपाल॒मिति॒ द्वाद॑श{-}क॒पा॒ल॒म्   ।   निरिति॑   ।    & TS\_1.8.8.1       \\
    
    \hline
        
    1063 & धा॒व॒ति॒   ।   सः   ।   ए॒व   ।   अ॒स्मि॒न्न्   ।    & TS\_2.1.1.4       \\
    
    \hline
        
    1064 & धी॒तिभि॒रिति॑ धी॒ति{-}भिः॒   ।   हि॒तः   ।   त्वे इति॑   ।   इषः॑   ।    & TS\_4.2.7.3       \\
    
    \hline
        
    1065 & धृष्टिः॑   ।   अ॒सि॒   ।   ब्रह्म॑   ।   य॒च्छ॒   ।    & TS\_1.1.7.1       \\
    
    \hline
        
    1066 & ध्रु॒वः   ।   अ॒सि॒   ।   ध्रु॒वः   ।   अ॒हम्   ।    & TS\_1.6.2.1 TS\_1.6.10.1 TS\_2.3.9.1       \\
    
    \hline
        
    1067 & ध्रु॒वक्षि॑ती॒रिति॑ ध्रु॒व{-}क्षि॒तिः॒   ।   ध्रु॒वयो॑नि॒रिति॑ ध्रु॒व{-}यो॒निः॒   ।   ध्रु॒वा   ।   अ॒सि॒   ।    & TS\_4.3.4.1       \\
    
    \hline
        
    1068 & ध्रु॒वा   ।   अ॒सि॒   ।   ध॒रुणा᳚   ।   अस्तृ॑ता   ।    & TS\_4.2.9.1       \\
    
    \hline
        
    1069 & ध्रु॒वाम्   ।   वै   ।   रिच्य॑मानाम्   ।   य॒ज्ञ्ः   ।    & TS\_1.7.5.1       \\
    
    \hline
        
    1070 & ध्व॒र॒ति॒   ।   शो॒चिष्के॑श॒ इति॑ शो॒चिः{-}के॒शः॒   ।   तम्   ।   ई॒म॒हे॒   ।    & TS\_2.5.8.6       \\
    
    \hline
        
    1071 & ध॒ता   ।   रा॒तिः   ।   स॒वि॒ता   ।   इ॒दम्   ।    & TS\_1.4.44.1       \\
    
    \hline
        
    1072 & ध॒त्तः॒   ।   न   ।   इ॒न्द्रि॒येण॑   ।   वी॒र्ये॑ण   ।    & TS\_2.2.1.4       \\
    
    \hline
        
    1073 & ध॒त्ते॒   ।   श॒ताक्ष॑रा॒ इति॑ श॒त{-}अ॒क्ष॒राः॒   ।   भ॒व॒न्ति॒   ।   श॒तायु॒रिति॑ श॒त{-}आ॒युः॒   ।    & TS\_1.5.2.2       \\
    
    \hline
        
    1074 & ध॒र्त्राय॑   ।   गृ॒ह्णा॒मि॒   ।   वि॒शे   ।   त्वा॒   ।    & TS\_1.6.1.3       \\
    
    \hline
        
    1075 & न   ।   असो॑मया॒जीत्यसो॑म{-}या॒जी॒   ।   समिति॑   ।   न॒ये॒त्   ।    & TS\_2.5.5.1       \\
    
    \hline
        
    1076 & न   ।   अ॒न्तः   ।   ए॒ति॒   ।   तत्   ।    & TS\_2.6.9.8       \\
    
    \hline
        
    1077 & न   ।   अ॒स्य॒   ।   स॒पत्नः॑   ।   भ॒व॒ति॒   ।    & TS\_5.3.5.3       \\
    
    \hline
        
    1078 & न   ।   ए॒न॒म्   ।   शी॒त॒रू॒राविति॑ शीत{-}रू॒रौ   ।   ह॒तः॒   ।    & TS\_2.5.2.4       \\
    
    \hline
        
    1079 & न   ।   ग॒तम॑ना॒ इति॑ ग॒त{-}म॒नाः॒   ।   भ॒व॒ति॒   ।   अपेति॑   ।    & TS\_6.6.7.3       \\
    
    \hline
        
    1080 & न   ।   प्रेति॑   ।   अ॒भ॒व॒त्   ।   ते   ।    & TS\_6.6.11.2       \\
    
    \hline
        
    1081 & न   ।   य॒ज्ञ्ः   ।   प॒रा॒भव॒तीति॑ परा{-}भव॑ति   ।   न   ।    & TS\_5.4.10.4       \\
    
    \hline
        
    1082 & न   ।   रथा॑य   ।   ए॒तत्   ।   वै   ।    & TS\_6.2.6.2       \\
    
    \hline
        
    1083 & न   ।   वै   ।   ए॒षः   ।   अ॒न्यतो॑वैश्वानर॒ इत्य॒न्यतः॑{-}वै॒श्वा॒न॒रः॒   ।    & TS\_7.2.10.1       \\
    
    \hline
        
    1084 & न   ।   स॒ह   ।   आ॒सी॒त॒   ।   न   ।    & TS\_2.5.1.6       \\
    
    \hline
        
    1085 & न   ।   हि   ।   आ॒सा॒म्   ।   इष᳚म्   ।    & TS\_7.5.9.2       \\
    
    \hline
        
    1086 & न   ।   ह॒   ।   स्म॒   ।   वै   ।    & TS\_5.1.10.1       \\
    
    \hline
        
    1087 & नः॒   ।   अश्वे॑षु   ।   री॒रि॒षः॒   ।   वी॒रान्   ।    & TS\_3.4.11.3       \\
    
    \hline
        
    1088 & नः॒   ।   त्वम्   ।   सो॒म॒   ।   पि॒तृभि॒रिति॑ पि॒तृ{-}भिः॒   ।    & TS\_2.6.12.2       \\
    
    \hline
        
    1089 & नः॒   ।   द्वेष्टि॑   ।   तम्   ।   वा॒म्   ।    & TS\_5.5.10.3       \\
    
    \hline
        
    1090 & नः॒   ।   प्र॒थ॒मः   ।   अ॒न्यः   ।   अ॒न्यस्मै᳚   ।    & TS\_6.2.2.2       \\
    
    \hline
        
    1091 & नः॒   ।   म॒र्द्धीः॒   ।   एति॑   ।   तु   ।    & TS\_2.2.12.7       \\
    
    \hline
        
    1092 & नः॒   ।   री॒रि॒षः॒   ।   मा   ।   परेति॑   ।    & TS\_4.7.14.4       \\
    
    \hline
        
    1093 & नः॒   ।   सम॑नसा॒विति॒ स{-}म॒न॒सौ॒   ।   समो॑कसा॒विति॒ सं{-}ओ॒क॒सौ॒   ।   अ॒रे॒पसौ᳚   ।    & TS\_1.3.7.2       \\
    
    \hline
        
    1094 & नक्त᳚म्   ।   उ॒प॒तिष्ठ॑त॒ इत्यु॑प{-}तिष्ठ॑ते   ।   ज्योति॑षा   ।   ए॒व   ।    & TS\_1.5.9.6       \\
    
    \hline
        
    1095 & नमः॑   ।   अग्रि॑याय   ।   च॒   ।   प्र॒थ॒माय॑   ।    & TS\_4.5.5.2       \\
    
    \hline
        
    1096 & नमः॑   ।   आ॒व्या॒धिनी᳚भ्य॒ इत्या᳚{-}व्या॒धिनी᳚भ्यः   ।   वि॒विद्ध्य॑न्तीभ्य॒ इति॑ वि{-}विद्ध्य॑न्तीभ्यः   ।   च॒   ।    & TS\_4.5.4.1       \\
    
    \hline
        
    1097 & नमः॑   ।   आ॒शुषे॑णा॒येत्या॒शु{-}से॒ना॒य॒   ।   च॒   ।   आ॒शुर॑था॒येत्या॒शु{-}र॒था॒य॒   ।    & TS\_4.5.6.2       \\
    
    \hline
        
    1098 & नमः॑   ।   इ॒रि॒ण्या॑य   ।   च॒   ।   प्र॒प॒थ्या॑येति॑ प्र{-}प॒थ्या॑य   ।    & TS\_4.5.9.1       \\
    
    \hline
        
    1099 & नमः॑   ।   ऊ॒र्व्या॑य   ।   च॒   ।   सू॒र्म्या॑य   ।    & TS\_4.5.9.2       \\
    
    \hline
        
    1100 & नमः॑   ।   कूप्या॑य   ।   च॒   ।   अ॒व॒ट्या॑य   ।    & TS\_4.5.7.2       \\
    
    \hline
        
    1101 & नमः॑   ।   ज्ये॒ष्ठाय॑   ।   च॒   ।   क॒नि॒ष्ठाय॑   ।    & TS\_4.5.6.1       \\
    
    \hline
        
    1102 & नमः॑   ।   तीर्थ्या॑य   ।   च॒   ।   कूल्या॑य   ।    & TS\_4.5.8.2       \\
    
    \hline
        
    1103 & नमः॑   ।   तेभ्यः॑   ।   स्वाहा᳚   ।   समू॑ढ॒मिति॒ सं{-}ऊ॒ढ॒म्   ।    & TS\_1.8.7.2       \\
    
    \hline
        
    1104 & नमः॑   ।   ते॒   ।   रु॒द्र॒   ।   म॒न्यवे᳚   ।    & TS\_4.5.1.1       \\
    
    \hline
        
    1105 & नमः॑   ।   दु॒न्दु॒भ्या॑य   ।   च॒   ।   आ॒ह॒न॒न्या॑येत्या᳚{-}ह॒न॒न्या॑य   ।    & TS\_4.5.7.1       \\
    
    \hline
        
    1106 & नमः॑   ।   भ॒वाय॑   ।   च॒   ।   रु॒द्राय॑   ।    & TS\_4.5.5.1       \\
    
    \hline
        
    1107 & नमः॑   ।   राज्ञे᳚   ।   नमः॑   ।   वरु॑णाय   ।    & TS\_7.4.16.1       \\
    
    \hline
        
    1108 & नमः॑   ।   सह॑मानाय   ।   नि॒व्या॒धिन॒ इति॑ नि{-}व्या॒धिने᳚   ।   आ॒व्या॒धिनी॑ना॒मित्या᳚{-}व्या॒धिनी॑नाम्   ।    & TS\_4.5.3.1       \\
    
    \hline
        
    1109 & नमः॑   ।   सोमा॑य   ।   च॒   ।   रु॒द्राय॑   ।    & TS\_4.5.8.1       \\
    
    \hline
        
    1110 & नमः॑   ।   हिर॑ण्यबाहव॒ इति॒ हिर॑ण्य{-}बा॒ह॒वे॒   ।   से॒ना॒न्य॑ इति॑ सेना{-}न्ये᳚   ।   दि॒शाम्   ।    & TS\_4.5.2.1       \\
    
    \hline
        
    1111 & नये᳚त्   ।   एति॑   ।   ए॒भ्यः॒   ।   वृ॒श्च्ये॒त॒   ।    & TS\_3.2.8.4       \\
    
    \hline
        
    1112 & नर्या᳚   ।   पु॒रूणि॑   ।   अ॒ग्निः   ।   भु॒व॒त्   ।    & TS\_2.2.12.2       \\
    
    \hline
        
    1113 & नवे॑दाः   ।   यश॑स्वतीः   ।   अ॒प॒स्युवः॑   ।   न   ।    & TS\_3.1.11.5       \\
    
    \hline
        
    1114 & नवो॑नव॒ इति॒ नवः॑{-}न॒वः॒   ।   भ॒व॒ति॒   ।   जाय॑मानः   ।   अह्ना᳚म्   ।    & TS\_2.4.14.1       \\
    
    \hline
        
    1115 & नाम॑   ।   इति॑   ।   आ॒ह॒   ।   प्रा॒णा॒पा॒नाविति॑ प्राण{-}अ॒पा॒नौ   ।    & TS\_3.3.5.3       \\
    
    \hline
        
    1116 & ना॒क॒सद्भि॒रिति॑ नाक॒सत्{-}भिः॒   ।   वै   ।   दे॒वाः   ।   सु॒व॒र्गमिति॑ सुवः{-}गम्   ।    & TS\_5.3.7.1       \\
    
    \hline
        
    1117 & निरिति॑   ।   मु॒च्ये॒त॒   ।   इति॑   ।   एकै॑क॒मित्येकं᳚{-}ए॒क॒म्   ।    & TS\_5.4.5.5       \\
    
    \hline
        
    1118 & निरिति॑   ।   व॒पे॒त्   ।   यः   ।   का॒मये॑त   ।    & TS\_2.2.11.3       \\
    
    \hline
        
    1119 & निवी॑त॒मिति॒ नि{-}वी॒त॒म्   ।   म॒नु॒ष्या॑णाम्   ।   प्रा॒ची॒ना॒वी॒तमिति॑ प्राचीन{-}आ॒वी॒तम्   ।   पि॒तृ॒णाम्   ।    & TS\_2.5.11.1       \\
    
    \hline
        
    1120 & निशि॑ताया॒मिति॒ नि{-}शि॒ता॒या॒म्   ।   हि   ।   रक्षाꣳ॑सि   ।   प्रे॒रत॒ इति॑ प्र{-}ई॒रते᳚   ।    & TS\_2.2.2.3       \\
    
    \hline
        
    1121 & निहः॑   ।   अतीति॑   ।   स्रिधः॑   ।   अतीति॑   ।    & TS\_4.1.7.3       \\
    
    \hline
        
    1122 & नि॒ग्रा॒भ्या॑ इति॑ नि{-}ग्रा॒भ्याः᳚   ।   स्थ॒   ।   दे॒व॒श्रुत॒ इति॑ देव{-}श्रुतः॑   ।   आयुः॑   ।    & TS\_3.1.8.1       \\
    
    \hline
        
    1123 & नि॒युत्व॒त्येति॑ नि{-}युत्व॑त्या   ।   य॒ज॒ति॒   ।   भ्रातृ॑व्यस्य   ।   ए॒व   ।    & TS\_2.6.2.3       \\
    
    \hline
        
    1124 & नि॒युत॒मिति॑ नि{-}युत᳚म्   ।   च॒   ।   प्र॒युत॒मिति॑ प्र{-}युत᳚म्   ।   च॒   ।    & TS\_4.4.11.4       \\
    
    \hline
        
    1125 & नीति॑   ।   अ॒मा॒र्ट॒   ।   इति॑   ।   तस्मा᳚त्   ।    & TS\_7.1.1.3       \\
    
    \hline
        
    1126 & नीति॑   ।   व॒र्त॒स्व॒   ।   पुनः॑   ।   अ॒ग्ने॒   ।    & TS\_4.2.3.4       \\
    
    \hline
        
    1127 & नीति॑   ।   ह॒रा॒मि॒   ।   ते॒   ।   म॒रुद्भ्य॒ इति॑ म॒रुत्{-}भ्यः॒   ।    & TS\_1.8.4.2       \\
    
    \hline
        
    1128 & नृ॒षद॒ इति॑ नृ{-}सदे᳚   ।   वट्   ।   इति॑   ।   व्याघा॑रय॒तीति॑ वि{-}आघा॑रयति   ।    & TS\_5.4.5.1       \\
    
    \hline
        
    1129 & नेष्टः॑   ।   पत्नी᳚म्   ।   उ॒दान॒येत्यु॑त्{-}आन॑य   ।   इति॑   ।    & TS\_6.5.8.6       \\
    
    \hline
        
    1130 & न॒   ।   ए॒तु॒   ।   मनः॑   ।   पुनः॑   ।    & TS\_1.8.5.3       \\
    
    \hline
        
    1131 & न॒भ॒सः॒   ।   प॒ते॒   ।   ऊर्ज᳚म्   ।   नः॒   ।    & TS\_3.3.8.3       \\
    
    \hline
        
    1132 & न॒म॒ति॒   ।   षड॑रत्नि॒मिति॒ षट्{-}अ॒र॒त्नि॒म्   ।   प्र॒ति॒ष्ठाका॑म॒स्येति॑ प्रति॒ष्ठा{-} का॒म॒स्य॒   ।   षट्   ।    & TS\_6.3.3.6       \\
    
    \hline
        
    1133 & न॒ह्य॒ति॒   ।   अप्र॑स्रꣳसा॒येत्यप्र॑{-}स्रꣳ॒॒सा॒य॒   ।   संत॑त॒मिति॒ सं{-}त॒त॒म्   ।   अन्विति॑   ।    & TS\_2.5.7.2       \\
    
    \hline
        
    1134 & पञ्चा᳚क्ष॒रेति॒ पञ्च॑{-}अ॒क्ष॒रा॒   ।   प॒ङ्क्तिः   ।   पाङ्क्तः॑   ।   य॒ज्ञ्ः   ।    & TS\_6.4.4.2       \\
    
    \hline
        
    1135 & पञ्च॑   ।   उपेति॑   ।   द॒धा॒ति॒   ।   पञ्च॑   ।    & TS\_5.3.1.2       \\
    
    \hline
        
    1136 & पत॑यः   ।   स्या॒म॒   ।   तस्य॑   ।   व॒यम्   ।    & TS\_1.7.13.5       \\
    
    \hline
        
    1137 & पत॑ये   ।   स्वाहा᳚   ।   इति॑   ।   स्क॒न्नम्   ।    & TS\_2.6.6.4       \\
    
    \hline
        
    1138 & पन्था᳚म्   ।   अ॒नू॒वृग्भ्या॒मित्य॑नू॒वृक्{-}भ्या॒म्   ।   संत॑ति॒मिति॒ सं{-}त॒ति॒म्   ।   स्ना॒व॒न्या᳚भ्याम्   ।    & TS\_5.7.23.1       \\
    
    \hline
        
    1139 & पय॑सा   ।   जु॒हु॒यात्   ।   ग्रा॒म्यान्   ।   प॒शून्   ।    & TS\_5.4.3.2       \\
    
    \hline
        
    1140 & पय॑सा   ।   म॒हीम्   ।   गाम्   ।   स॒प्त   ।    & TS\_4.6.5.5       \\
    
    \hline
        
    1141 & पय॑स्वत्   ।   वी॒रुधा᳚म्   ।   पयः॑   ।   अ॒पाम्   ।    & TS\_1.5.10.3       \\
    
    \hline
        
    1142 & पराङ्॑   ।   ऐ॒त्   ।   तम्   ।   ए॒तया᳚   ।    & TS\_5.2.1.2       \\
    
    \hline
        
    1143 & परीति॑   ।   द॒त्त॒   ।   इ॒ह   ।   सर्वे᳚   ।    & TS\_5.7.2.4       \\
    
    \hline
        
    1144 & परीति॑   ।   न॒य॒ति॒   ।   आ॒नु॒षू॒क इत्या॑नु{-}सू॒कः   ।   भ॒व॒ति॒   ।    & TS\_2.3.4.4       \\
    
    \hline
        
    1145 & परीति॑   ।   प॒श्या॒मः॒   ।   अꣳश᳚म्   ।   एति॑   ।    & TS\_7.1.6.2       \\
    
    \hline
        
    1146 & परीति॑   ।   मि॒नु॒या॒त्   ।   स॒प्त   ।   वै   ।    & TS\_5.2.6.3       \\
    
    \hline
        
    1147 & परेति॑   ।   असु॑राः   ।   यस्य॑   ।   ए॒ताः   ।    & TS\_5.3.11.2       \\
    
    \hline
        
    1148 & परेति॑   ।   अ॒स्य॒   ।   भ्रातृ॑व्यः   ।   भ॒व॒ति॒   ।    & TS\_6.2.5.5       \\
    
    \hline
        
    1149 & परेति॑   ।   ताः   ।   भ॒ग॒व॒ इति॑ भग{-}वः॒   ।   व॒प॒   ।    & TS\_4.5.1.4       \\
    
    \hline
        
    1150 & परेति॑   ।   भ॒व॒ति॒   ।   प्र॒या॒जव॒दिति॑ प्रया॒ज{-}व॒त्   ।   ए॒व   ।    & TS\_6.1.5.5       \\
    
    \hline
        
    1151 & परेति॑   ।   वै   ।   ए॒षः   ।   य॒ज्ञ्म्   ।    & TS\_1.5.2.1       \\
    
    \hline
        
    1152 & पर्य॒ग्नीति॒ परि॑{-}अ॒ग्नि॒   ।   क॒रो॒ति॒   ।   स॒र्व॒हुत॒मिति॑ सर्व{-}हुत᳚म्   ।   ए॒व   ।    & TS\_6.3.8.1       \\
    
    \hline
        
    1153 & पर्व॑तात्   ।   एति॑   ।   सर॑स्वती   ।   य॒ज॒ता   ।    & TS\_1.8.22.2       \\
    
    \hline
        
    1154 & पव॑स्व   ।   वाज॑सातय॒ इति॒ वाज॑{-}सा॒त॒ये॒   ।   इति॑   ।   अ॒नु॒ष्टुगित्य॑नु{-}स्तुक्   ।    & TS\_5.4.12.1       \\
    
    \hline
        
    1155 & पाङ्क्तः॑   ।   य॒ज्ञ्ः   ।   दे॒वताः᳚   ।   च॒   ।    & TS\_3.5.4.4       \\
    
    \hline
        
    1156 & पात्रे᳚   ।   चतुः॑स्रक्ता॒विति॒ चतुः॑{-}स्र॒क्तौ॒   ।   स्व॒य॒म॒व॒प॒न्नाया॒ इति॑ स्वयं{-}अ॒व॒प॒न्नायै᳚   ।   शाखा॑यै   ।    & TS\_1.8.9.3       \\
    
    \hline
        
    1157 & पापी॑यान्   ।   स्या॒त्   ।   इति॑   ।   एकै॑क॒मित्येक᳚म्{-}ए॒क॒म्   ।    & TS\_5.1.1.2       \\
    
    \hline
        
    1158 & पा॒क॒य॒ज्ञ्मिति॑ पाक{-}य॒ज्ञ्म्   ।   वै   ।   अन्विति॑   ।   आहि॑ताग्ने॒रित्याहि॑त{-}अ॒ग्नेः॒   ।    & TS\_1.7.1.1       \\
    
    \hline
        
    1159 & पा॒त॒म्   ।   एति॑   ।   अ॒स्य   ।   य॒ज्ञ्स्य॑   ।    & TS\_1.2.2.2       \\
    
    \hline
        
    1160 & पा॒प्मान᳚म्   ।   अपीति॑   ।   द॒ह॒ति॒   ।   ऐ॒न्द्रेण॑   ।    & TS\_2.1.4.7       \\
    
    \hline
        
    1161 & पा॒र॒य॒   ।   आ॒दि॒त्यः   ।   ते॒   ।   वा॒जि॒न्न्   ।    & TS\_7.5.19.2       \\
    
    \hline
        
    1162 & पा॒व॒काय॑   ।   वाच᳚म्   ।   ए॒व   ।   अ॒स्मि॒न्न्   ।    & TS\_2.2.4.3       \\
    
    \hline
        
    1163 & पा॒हि॒   ।   आयुः॑   ।   मे॒   ।   पा॒हि॒   ।    & TS\_4.4.7.2       \\
    
    \hline
        
    1164 & पाꣳ॒॒सु॒रे   ।   इरा॑वती॒ इतीरा᳚{-}व॒ती॒   ।   धे॒नु॒मती॒ इति॑ धेनु{-}मती᳚   ।   हि   ।    & TS\_1.2.13.2       \\
    
    \hline
        
    1165 & पित॑रः   ।   सोम्या॑सः   ।   शि॒वे इति॑   ।   नः॒   ।    & TS\_4.6.6.4       \\
    
    \hline
        
    1166 & पि॒तरः॑   ।   अनु॑   ।   प्रेति॑   ।   स॒र्प॒न्ति॒   ।    & TS\_3.2.4.5       \\
    
    \hline
        
    1167 & पि॒तृपी॑त॒स्येति॑ पि॒तृ{-}पी॒त॒स्य॒   ।   मधु॑मत॒ इति॒ मधु॑{-}म॒तः॒   ।   उप॑हूत॒स्येत्युप॑{-}हू॒त॒स्य॒   ।   उप॑हूत॒ इत्युप॑{-}हू॒तः॒   ।    & TS\_3.2.5.3       \\
    
    \hline
        
    1168 & पि॒तृ॒णाम्   ।   इ॒दम्   ।   अ॒हम्   ।   निरिति॑   ।    & TS\_6.3.2.6       \\
    
    \hline
        
    1169 & पि॒तृ॒णाम्   ।   सद॑नम्   ।   अ॒सि॒   ।   इति॑   ।    & TS\_6.3.4.2       \\
    
    \hline
        
    1170 & पि॒शङ्गाः᳚   ।   त्रयः॑   ।   वा॒स॒न्ताः   ।   सा॒रङ्गाः᳚   ।    & TS\_5.6.23.1       \\
    
    \hline
        
    1171 & पीव॑री   ।   अ॒स्य॒   ।   जा॒या   ।   पीवा॑नः   ।    & TS\_3.2.8.5       \\
    
    \hline
        
    1172 & पुनः॑   ।   ऋ॒तुना᳚   ।   आ॒ह॒   ।   तस्मा᳚त्   ।    & TS\_6.5.3.3       \\
    
    \hline
        
    1173 & पुरु॑षः   ।   आ॒त्मा   ।   च॒   ।   शिरः॑   ।    & TS\_5.6.9.2       \\
    
    \hline
        
    1174 & पुरु॑षाणाम्   ।   रू॒पम्   ।   यत्   ।   तू॒प॒रः   ।    & TS\_5.5.1.3       \\
    
    \hline
        
    1175 & पु॒त्रम्   ।   क्षि॒तः   ।   उ॒प॒धाव॒तीत्यु॑प{-}धाव॑ति   ।   ता॒दृक्   ।    & TS\_6.5.10.2       \\
    
    \hline
        
    1176 & पु॒रस्ता᳚त्   ।   अ॒सृ॒ज॒त॒   ।   प॒शुम्   ।   म॒द्ध्य॒तः   ।    & TS\_6.3.11.6       \\
    
    \hline
        
    1177 & पु॒रस्ता᳚त्   ।   ऐ॒न्द्रस्य॑   ।   वै॒श्व॒दे॒वमिति॑ वैश्व{-}दे॒वम्   ।   एति॑   ।    & TS\_6.6.5.3       \\
    
    \hline
        
    1178 & पु॒रस्ता᳚त्   ।   प्र॒त्यञ्चः॑   ।   प॒शवः॑   ।   मेध᳚म्   ।    & TS\_5.2.8.7       \\
    
    \hline
        
    1179 & पु॒रा   ।   होता॑रः   ।   अ॒भू॒व॒न्न्   ।   तस्मा᳚त्   ।    & TS\_2.5.11.2       \\
    
    \hline
        
    1180 & पु॒री॒ष्याः᳚   ।   प्रवि॑ष्टा॒ इति॒ प्र{-}वि॒ष्टाः॒   ।   पृ॒थि॒वीम्   ।   अनु॑   ।    & TS\_5.5.7.5       \\
    
    \hline
        
    1181 & पु॒रु॒त्रेति॑ पुरु{-}त्रा   ।   च॒   ।   र॒श्मीन्   ।   अन्विति॑   ।    & TS\_4.1.2.3       \\
    
    \hline
        
    1182 & पु॒रु॒ष॒मा॒त्रेणेति॑ पुरुष{-}मा॒त्रेण॑   ।   वीति॑   ।   मि॒मी॒ते॒   ।   य॒ज्ञेन॑   ।    & TS\_5.2.5.1       \\
    
    \hline
        
    1183 & पु॒रु॒ष॒मृ॒ग इति॑ पुरुष{-}मृ॒गः   ।   च॒न्द्रम॑से   ।   गो॒धा   ।   काल॑का   ।    & TS\_5.5.15.1       \\
    
    \hline
        
    1184 & पु॒रु॒ष॒शी॒र्॒.षमिति॑ पुरुष{-}शी॒र्॒.षम्   ।   उपेति॑   ।   द॒धा॒ति॒   ।   मु॒ख॒तः   ।    & TS\_5.5.3.3       \\
    
    \hline
        
    1185 & पु॒रोह॑वि॒षीति॑ पु॒रः{-}ह॒वि॒षि॒   ।   दे॒व॒यज॑न॒ इति॑ देव{-}यज॑ने   ।   या॒ज॒ये॒त्   ।   यम्   ।    & TS\_6.2.6.1       \\
    
    \hline
        
    1186 & पु॒रो॒डाशा᳚त्   ।   क॒र॒म्भात्   ।   धा॒ना॒सो॒मादिति॑ धाना{-}सो॒मात्   ।   म॒न्थिनः॑   ।    & TS\_3.1.10.2       \\
    
    \hline
        
    1187 & पु॒रो॒डाश᳚म्   ।   अवेति॑   ।   द॒धा॒ति॒   ।   आ॒त्म॒न्वन्त॒मित्या᳚त्मन्न्{-}वन्त᳚म्   ।    & TS\_2.3.13.3       \\
    
    \hline
        
    1188 & पु॒रो॒डाश᳚म्   ।   अ॒ष्टाक॑पाल॒मित्य॒ष्टा{-}क॒पा॒ल॒म्   ।   निरिति॑   ।   व॒पे॒त्   ।    & TS\_2.2.2.2       \\
    
    \hline
        
    1189 & पु॒ष्टम्   ।   प॒शु   ।   म॒न्य॒ते॒   ।   शू॒द्रा   ।    & TS\_7.4.19.3       \\
    
    \hline
        
    1190 & पूर्वे᳚   ।   अरि॑ष्टाः   ।   स्या॒म॒   ।   त॒नुवा᳚   ।    & TS\_4.7.14.2       \\
    
    \hline
        
    1191 & पूर्वे᳚   ।   एति॑   ।   अ॒ल॒भ॒न्त॒   ।   द॒र्॒.श॒पू॒र्ण॒मा॒साविति॑ दर्.श{-}पू॒र्ण॒मा॒सौ   ।    & TS\_3.5.1.3       \\
    
    \hline
        
    1192 & पूर्व॑या   ।   अ॒भि॒जय॒तीत्य॑भि{-}जय॑ति   ।   म॒नु॒ष्य॒लो॒कमिति॑ मनुष्य{-}लो॒कम्   ।   उत्त॑र॒येत्युत्{-}त॒र॒या॒   ।    & TS\_2.5.5.4       \\
    
    \hline
        
    1193 & पू॒र्णा   ।   प॒श्चात्   ।   उ॒त   ।   पू॒र्णा   ।    & TS\_3.5.1.1       \\
    
    \hline
        
    1194 & पू॒र्वे॒द्युः   ।   यज॑ते   ।   वेदि᳚म्   ।   ए॒व   ।    & TS\_2.5.5.5       \\
    
    \hline
        
    1195 & पू॒र्व॒प॒क्ष इति॑ पूर्व{-}प॒क्षे   ।   सु॒त्या   ।   समिति॑   ।   प॒द्य॒ते॒   ।    & TS\_7.4.8.3       \\
    
    \hline
        
    1196 & पू॒षा   ।   आ॒र्द्ध्नो॒त्   ।   तस्मा᳚त्   ।   पौ॒ष्णाः   ।    & TS\_1.5.1.3       \\
    
    \hline
        
    1197 & पू॒षा   ।   प्राश्येति॑ प्र{-}अश्य॑   ।   द॒तः   ।   अ॒रु॒ण॒त्   ।    & TS\_2.6.8.5       \\
    
    \hline
        
    1198 & पू॒ष्णः   ।   व॒नि॒ष्ठुः   ।   अ॒न्धा॒हेरित्य॑न्ध{-}अ॒हेः   ।   स्थू॒र॒गु॒देति॑ स्थूर{-}गु॒दा   ।    & TS\_5.7.17.1       \\
    
    \hline
        
    1199 & पृत॑नासु   ।   जि॒ष्णु   ।   स्तौमि॑   ।   दे॒वान्   ।    & TS\_4.7.15.5       \\
    
    \hline
        
    1200 & पृश्निः॑   ।   ति॒र॒श्चीन॑पृश्नि॒रिति॑ तिर॒श्चीन॑{-}पृ॒श्निः॒   ।   ऊ॒द्‌र्ध्वपृ॑श्नि॒रित्यू॒द्‌र्ध्व{-}पृ॒श्निः॒   ।   ते   ।    & TS\_5.6.12.1       \\
    
    \hline
        
    1201 & पृ॒त॒न्य॒तः   ।   उदिति॑   ।   क्रा॒म॒   ।   म॒ह॒ते   ।    & TS\_4.1.2.4       \\
    
    \hline
        
    1202 & पृ॒थि॒वी   ।   वे॒त्तु॒   ।   अ॒धि॒षव॑ण॒मित्य॑धि{-}सव॑नम्   ।   अ॒सि॒   ।    & TS\_1.1.5.2       \\
    
    \hline
        
    1203 & पृ॒थि॒वीम्   ।   अनु॑   ।   वीति॑   ।   आ॒र॒त्   ।    & TS\_2.5.3.3       \\
    
    \hline
        
    1204 & पृ॒थि॒व्या   ।   प॒रः   ।   दे॒वेभिः॑   ।   असु॑रैः   ।    & TS\_4.6.2.3       \\
    
    \hline
        
    1205 & पृ॒थि॒व्याः   ।   स॒मि॒धा॒नमिति॑ सं{-}इ॒धा॒नम्   ।   अ॒ग्निम्   ।   रा॒यः   ।    & TS\_4.1.10.2       \\
    
    \hline
        
    1206 & पृ॒थि॒व्यै   ।   त्वा॒   ।   अ॒न्तरि॑क्षाय   ।   त्वा॒   ।    & TS\_1.3.6.1 TS\_6.3.4.1       \\
    
    \hline
        
    1207 & पृ॒थि॒व्यै   ।   मा॒   ।   पा॒हि॒   ।   ज्योतिः॑   ।    & TS\_5.7.6.2       \\
    
    \hline
        
    1208 & पृ॒थि॒व्यै   ।   स्वाहा᳚   ।   अ॒न्तरि॑क्षाय   ।   स्वाहा᳚   ।    & TS\_7.1.15.1 TS\_7.1.17.1 TS\_7.5.11.1       \\
    
    \hline
        
    1209 & पृ॒ष्ठैः   ।   ए॒व   ।   ऋ॒तून्   ।   अ॒न्वारो॑ह॒न्तीत्य॑नु{-}आरो॑हन्ति   ।    & TS\_7.2.1.2       \\
    
    \hline
        
    1210 & पृ॒ष॒तः   ।   वै॒श्व॒दे॒व इति॑ वैश्व{-}दे॒वः   ।   पि॒त्वः   ।   न्यङ्कुः॑   ।    & TS\_5.5.17.1       \\
    
    \hline
        
    1211 & प्रज्ञा᳚त्या॒ इति॒ प्र{-}ज्ञा॒त्यै॒   ।   त्रयो॑द॒शेति॒ त्रयः॑{-}द॒श॒   ।   लो॒क॒पृं॒णा इति॑ लोकं{-}पृ॒णाः   ।   उपेति॑   ।    & TS\_5.2.3.6       \\
    
    \hline
        
    1212 & प्रतीति॑   ।   ति॒ष्ठ॒ति॒   ।   एक॑चितीक॒मित्येक॑{-}चि॒ती॒क॒म्   ।   चि॒न्वी॒त॒   ।    & TS\_5.2.3.7       \\
    
    \hline
        
    1213 & प्रतीति॑   ।   ति॒ष्ठ॒ति॒   ।   घृ॒ते   ।   भ॒व॒ति॒   ।    & TS\_2.3.2.2       \\
    
    \hline
        
    1214 & प्रतीति॑   ।   ति॒ष्ठ॒न्ति॒   ।   त्र॒य॒स्त्रिꣳ॒॒शादिति॑ त्रयः{-}त्रिꣳ॒॒शात्   ।   त्र॒य॒स्त्रिꣳ॒॒शमिति॑ त्रयः{-}त्रिꣳ॒॒शम्   ।    & TS\_7.4.3.3       \\
    
    \hline
        
    1215 & प्रतीति॑   ।   ति॒ष्ठ॒न्ति॒   ।   पृ॒ष्ठम्   ।   वै   ।    & TS\_7.3.10.3       \\
    
    \hline
        
    1216 & प्रतीति॑   ।   धे॒नुम्   ।   इ॒व   ।   आ॒य॒तीमित्या᳚{-}य॒तीम्   ।    & TS\_4.4.4.2       \\
    
    \hline
        
    1217 & प्रतीति॑   ।   वै   ।   प॒रस्ता᳚त्   ।   अ॒भि॒चर॑न्त॒मित्य॑भि{-}चर॑न्तम्   ।    & TS\_2.2.9.2       \\
    
    \hline
        
    1218 & प्रतीति॑   ।   स्था॒प॒य॒ति॒   ।   क॒विः   ।   य॒ज्ञ्स्य॑   ।    & TS\_3.5.5.3       \\
    
    \hline
        
    1219 & प्रत्यु॑ष्ट॒मिति॒ प्रति॑{-}उ॒ष्ट॒म्   ।   रक्षः॑   ।   प्रत्यु॑ष्टा॒ इति॒ प्रति॑ {-}उ॒ष्टाः॒   ।   अरा॑तयः   ।    & TS\_1.1.10.1       \\
    
    \hline
        
    1220 & प्रव॑दितो॒रिति॒ प्र{-}व॒दि॒तोः॒   ।   प्रा॒त॒र॒नु॒वा॒कमिति॑ प्रातः{-}अ॒नु॒वा॒कम्   ।   उ॒पाक॑रो॒तीत्यु॑प{-}आक॑रोति   ।   याव॑ती   ।    & TS\_6.4.3.2       \\
    
    \hline
        
    1221 & प्रसि॑द्ध॒मिति॒ प्र{-}सि॒द्ध॒म्   ।   ए॒व   ।   अ॒द्ध्व॒र्युः   ।   दक्षि॑णेन   ।    & TS\_6.5.3.4       \\
    
    \hline
        
    1222 & प्राचीः᳚   ।   उपेति॑   ।   द॒धा॒ति॒   ।   प॒श्चात्   ।    & TS\_5.2.10.2       \\
    
    \hline
        
    1223 & प्राची᳚   ।   दि॒शाम्   ।   व॒स॒न्तः   ।   ऋ॒तू॒नाम्   ।    & TS\_4.3.3.1       \\
    
    \hline
        
    1224 & प्राची᳚म्   ।   अन्विति॑   ।   प्र॒दिश॒मिति॑ प्र{-}दिश᳚म्   ।   प्रेति॑   ।    & TS\_4.6.5.1 TS\_5.4.7.1       \\
    
    \hline
        
    1225 & प्राञ्च᳚म्   ।   उपेति॑   ।   द॒धा॒ति॒   ।   दा॒धार॑   ।    & TS\_5.2.7.3       \\
    
    \hline
        
    1226 & प्राशि॑त॒मिति॒ प्र{-}अ॒शि॒त॒म्   ।   मा॒   ।   हिꣳ॒॒सि॒ष्य॒ति॒   ।   इति॑   ।    & TS\_2.6.8.7       \\
    
    \hline
        
    1227 & प्रा॒चीन॑वꣳश॒मिति॑ प्रा॒चीन॑{-}वꣳ॒॒श॒म्   ।   क॒रो॒ति॒   ।   दे॒व॒म॒नु॒ष्या इति॑ देव{-}म॒नु॒ष्याः   ।   दिशः॑   ।    & TS\_6.1.1.1       \\
    
    \hline
        
    1228 & प्रा॒जा॒प॒त्या इति॑ प्राजा{-}प॒त्याः   ।   वै   ।   प॒शवः॑   ।   तेषा᳚म्   ।    & TS\_3.1.5.1       \\
    
    \hline
        
    1229 & प्रा॒ण इति॑ प्र {-}अ॒नः   ।   त्रि॒वृत॒मिति॑ त्रि{-}वृत᳚म्   ।   ए॒व   ।   प्रा॒णमिति॑ प्र {-}अ॒नम्   ।    & TS\_6.2.1.5       \\
    
    \hline
        
    1230 & प्रा॒ण इति॑ प्र{-}अ॒नः   ।   उ॒परि॑ष्टात्   ।   अ॒पा॒न इत्य॑प{-}अ॒नः   ।   यावान्॑   ।    & TS\_3.4.1.4       \\
    
    \hline
        
    1231 & प्रा॒ण इति॑ प्र{-}अ॒नः   ।   वै   ।   ए॒षः   ।   यत्   ।    & TS\_6.4.5.1 TS\_6.5.8.1       \\
    
    \hline
        
    1232 & प्रा॒णमिति॑ प्र{-}अ॒नम्   ।   अ॒स्मि॒न्न्   ।   सः   ।   द॒धा॒ति॒   ।    & TS\_7.5.6.2       \\
    
    \hline
        
    1233 & प्रा॒णमिति॑ प्र{-}अ॒नम्   ।   ए॒व   ।   प्र॒थ॒मेन॑   ।   अ॒स्पृ॒णु॒त॒   ।    & TS\_6.5.5.3       \\
    
    \hline
        
    1234 & प्रा॒णा इति॑ प्र{-}अ॒नाः   ।   उ॒प॒र॒वा इत्यु॑प{-}र॒वाः   ।   हनू॒ इति॑   ।   अ॒धि॒षव॑णे॒ इत्य॑धि{-}सव॑ने   ।    & TS\_6.2.11.4       \\
    
    \hline
        
    1235 & प्रा॒णायेति॑ प्र{-}अ॒नाय॑   ।   स्वाहा᳚   ।   व्या॒नायेति॑ वि{-}अ॒नाय॑   ।   स्वाहा᳚   ।    & TS\_7.4.21.1       \\
    
    \hline
        
    1236 & प्रा॒णा॒पा॒नयो॒रिति॑ प्राण{-}अ॒पा॒नयोः᳚   ।   विधृ॑त्या॒ इति॒ वि{-}धृ॒त्यै॒   ।   प्रा॒णा॒पा॒नाविति॑ प्राण{-}अ॒पा॒नौ   ।   वै   ।    & TS\_6.4.6.4       \\
    
    \hline
        
    1237 & प्रा॒णा॒पा॒नाभ्या॒मिति॑ प्राण{-}अ॒पा॒नाभ्या᳚म्   ।   वै   ।   ए॒ते   ।   वीति॑   ।    & TS\_7.2.7.2       \\
    
    \hline
        
    1238 & प्रा॒णेनेति॑ प्र{-}अ॒नेन॑   ।   ए॒व   ।   अ॒स्य॒   ।   अ॒पा॒नमित्य॑प{-}अ॒नम्   ।    & TS\_2.5.7.5       \\
    
    \hline
        
    1239 & प्रा॒णैरिति॑ प्र{-}अ॒नैः   ।   ए॒व   ।   ए॒न॒म्   ।   उदिति॑   ।    & TS\_5.2.2.2       \\
    
    \hline
        
    1240 & प्रा॒णैरिति॑ प्र{-}अ॒नैः   ।   ए॒व   ।   प्र॒यन्तीति॑ प्र{-}यन्ति॑   ।   प्रा॒णैरिति॑ प्र{-}अ॒नैः   ।    & TS\_3.5.10.3       \\
    
    \hline
        
    1241 & प्रा॒णैरिति॑ प्र{-}अ॒नैः   ।   दा॒धा॒र॒   ।   आसी॑नः   ।   प्रतीति॑   ।    & TS\_5.1.10.5       \\
    
    \hline
        
    1242 & प्रा॒त॒र्युजा॒विति॑ प्रातः{-}युजौ᳚   ।   वीति॑   ।   मु॒च्ये॒था॒म्   ।   अश्वि॑नौ   ।    & TS\_1.4.7.1       \\
    
    \hline
        
    1243 & प्रा॒त॒स्स॒व॒न इति॑ प्रातः{-}स॒व॒ने   ।   वै   ।   गा॒य॒त्रेण॑   ।   छन्द॑सा   ।    & TS\_7.1.2.1       \\
    
    \hline
        
    1244 & प्रि॒यः   ।   आ॒त्मा   ।   अ॒पि॒यन्त॒मित्य॑पि{-}यन्त᳚म्   ।   मा   ।    & TS\_4.6.9.4       \\
    
    \hline
        
    1245 & प्रि॒यम्   ।   पाथः॑   ।   अपीति॑   ।   इ॒हि॒   ।    & TS\_3.3.3.3       \\
    
    \hline
        
    1246 & प्रेति॑   ।   अ॒गा॒त्   ।   शस॑नम्   ।   वा॒जी   ।    & TS\_4.6.7.5       \\
    
    \hline
        
    1247 & प्रेति॑   ।   अ॒जा॒य॒त॒   ।   तस्मा᳚त्   ।   ए॒व   ।    & TS\_1.7.4.6       \\
    
    \hline
        
    1248 & प्रेति॑   ।   अ॒न्यानि॑   ।   पात्रा॑णि   ।   यु॒ज्यन्ते᳚   ।    & TS\_6.5.11.1       \\
    
    \hline
        
    1249 & प्रेति॑   ।   ई॒र॒य॒   ।   स्वे   ।   अ॒ग्ने॒   ।    & TS\_1.4.44.3       \\
    
    \hline
        
    1250 & प्रेति॑   ।   च्या॒व॒य॒न्तु॒   ।   दि॒वि   ।   दे॒वान्   ।    & TS\_3.2.8.6       \\
    
    \hline
        
    1251 & प्रेति॑   ।   च्य॒व॒स्व॒   ।   भु॒वः॒   ।   प॒ते॒   ।    & TS\_1.2.9.1       \\
    
    \hline
        
    1252 & प्रेति॑   ।   जा॒या॒म॒है॒   ।   इति॑   ।   तम्   ।    & TS\_5.5.2.7       \\
    
    \hline
        
    1253 & प्रेति॑   ।   जा॒ये॒र॒न्न्   ।   यत्   ।   अ॒न्तरि॑क्षे   ।    & TS\_5.2.7.2       \\
    
    \hline
        
    1254 & प्रेति॑   ।   दे॒वम्   ।   दे॒व्या   ।   धि॒या   ।    & TS\_3.5.11.1       \\
    
    \hline
        
    1255 & प्रेति॑   ।   मु॒ञ्च॒   ।   स्व॒स्तये᳚   ।   ये   ।    & TS\_3.2.8.3       \\
    
    \hline
        
    1256 & प्रेति॑   ।   यः   ।   ज॒ज्ञे   ।   वि॒द्वान्   ।    & TS\_2.3.14.6       \\
    
    \hline
        
    1257 & प्रेति॑   ।   य॒न्धि॒   ।   प्र॒दा॒तार॒मिति॑ प्र{-}दा॒तार᳚म्   ।   ह॒वा॒म॒हे॒   ।    & TS\_1.7.13.4       \\
    
    \hline
        
    1258 & प्रेति॑   ।   सः   ।   अ॒ग्ने॒   ।   तव॑   ।    & TS\_3.2.11.1       \\
    
    \hline
        
    1259 & प्रेति॑   ।   सु॒ला॒मि॒   ।   इति॑   ।   ते॒   ।    & TS\_7.4.19.4       \\
    
    \hline
        
    1260 & प्रेति॑   ।   हाः॒   ।   आ॒वम्   ।   अ॒न्तः   ।    & TS\_2.5.2.3       \\
    
    \hline
        
    1261 & प्र॒चर॒न्तीति॑ प्र{-}चर॑न्ति   ।   सृ॒काव॑न्त॒ इति॑ सृ॒का{-}व॒न्तः॒   ।   नि॒ष॒ङ्गिण॒ इति॑ नि{-}स॒ङ्गिनः॑   ।   ये   ।    & TS\_4.5.11.2       \\
    
    \hline
        
    1262 & प्र॒जन॑न॒मिति॑ प्र{-}जन॑नम्   ।   ज्योतिः॑   ।   अ॒ग्निः   ।   दे॒वता॑नाम्   ।    & TS\_7.1.1.1       \\
    
    \hline
        
    1263 & प्र॒जयेति॑ प्र{-}जया᳚   ।   प॒शुभि॒रिति॑ प॒शु{-}भिः॒   ।   मि॒थु॒नैः   ।   जा॒य॒ते॒   ।    & TS\_2.6.1.5       \\
    
    \hline
        
    1264 & प्र॒जव॒मिति॑ प्र{-}जव᳚म्   ।   वै   ।   ए॒तेन॑   ।   य॒न्ति॒   ।    & TS\_7.3.1.1       \\
    
    \hline
        
    1265 & प्र॒जाप॑ति॒रिति॑ प्र॒जा{-}प॒तिः॒   ।   अ॒का॒म॒य॒त॒   ।   अ॒न्ना॒द इत्य॑न्न{-}अ॒दः   ।   स्या॒म्   ।    & TS\_7.3.8.1       \\
    
    \hline
        
    1266 & प्र॒जाप॑ति॒रिति॑ प्र॒जा{-}प॒तिः॒   ।   अ॒का॒म॒य॒त॒   ।   प्रेति॑   ।   जा॒ये॒य॒   ।    & TS\_7.2.5.1 TS\_7.2.9.1       \\
    
    \hline
        
    1267 & प्र॒जाप॑ति॒रिति॑ प्र॒जा{-}प॒तिः॒   ।   अ॒का॒म॒य॒त॒   ।   प्र॒जा इति॑ प्र{-}जाः   ।   सृ॒जे॒य॒   ।    & TS\_3.1.1.1       \\
    
    \hline
        
    1268 & प्र॒जाप॑ति॒रिति॑ प्र॒जा{-}प॒तिः॒   ।   अ॒ग्निम्   ।   अ॒चि॒नु॒त॒   ।   ऋ॒तुभि॒रित्यृ॒तु{-}भिः॒   ।    & TS\_5.6.10.1       \\
    
    \hline
        
    1269 & प्र॒जाप॑ति॒रिति॑ प्र॒जा{-}प॒तिः॒   ।   अ॒ग्निम्   ।   अ॒चि॒नु॒त॒   ।   सः   ।    & TS\_5.6.6.1       \\
    
    \hline
        
    1270 & प्र॒जाप॑ति॒रिति॑ प्र॒जा{-}प॒तिः॒   ।   अ॒ग्निम्   ।   अ॒सृ॒ज॒त॒   ।   सः   ।    & TS\_5.7.10.1       \\
    
    \hline
        
    1271 & प्र॒जाप॑ति॒रिति॑ प्र॒जा{-}प॒तिः॒   ।   दू॒तीः   ।   ए॒व   ।   त्वम्   ।    & TS\_2.5.11.5       \\
    
    \hline
        
    1272 & प्र॒जाप॑ति॒रिति॑ प्र॒जा{-}प॒तिः॒   ।   दे॒वा॒सु॒रानिति॑ देव{-}अ॒सु॒रान्   ।   अ॒सृ॒ज॒त॒   ।   तत्   ।    & TS\_3.3.7.1       \\
    
    \hline
        
    1273 & प्र॒जाप॑ति॒रिति॑ प्र॒जा{-}प॒तिः॒   ।   दे॒वेभ्यः॑   ।   अ॒न्नाद्य॒मित्य॑न्न{-}अद्य᳚म्   ।   व्यादि॑श॒दिति॑ वि{-}आदि॑शत्   ।    & TS\_2.3.6.1       \\
    
    \hline
        
    1274 & प्र॒जाप॑ति॒रिति॑ प्र॒जा{-}प॒तिः॒   ।   दे॒वेभ्यः॑   ।   य॒ज्ञान्   ।   व्यादि॑श॒दिति॑ वि{-}आदि॑शत्   ।    & TS\_2.6.3.1 TS\_6.6.11.1       \\
    
    \hline
        
    1275 & प्र॒जाप॑ति॒रिति॑ प्र॒जा{-}प॒तिः॒   ।   प्र॒जा इति॑ प्र{-}जाः   ।   अ॒सृ॒ज॒त॒   ।   ताः   ।    & TS\_2.1.2.1 TS\_2.2.1.1 TS\_2.4.4.1 TS\_7.2.4.1       \\
    
    \hline
        
    1276 & प्र॒जाप॑ति॒रिति॑ प्र॒जा{-}प॒तिः॒   ।   प्र॒जा इति॑ प्र{-}जाः   ।   अ॒सृ॒ज॒त॒   ।   सः   ।    & TS\_6.6.5.1       \\
    
    \hline
        
    1277 & प्र॒जाप॑ति॒रिति॑ प्र॒जा{-}प॒तिः॒   ।   प्र॒जा इति॑ प्र{-}जाः   ।   सृ॒ष्ट्वा   ।   प्रे॒णा   ।    & TS\_5.5.2.1       \\
    
    \hline
        
    1278 & प्र॒जाप॑ति॒रिति॑ प्र॒जा{-}प॒तिः॒   ।   मन॑सा   ।   अन्धः॑   ।   अच्छे॑त॒ इत्यच्छ॑{-}इ॒तः॒   ।    & TS\_4.4.9.1       \\
    
    \hline
        
    1279 & प्र॒जाप॑ति॒रिति॑ प्र॒जा{-}प॒तिः॒   ।   य॒ज्ञान्   ।   अ॒सृ॒ज॒त॒   ।   अ॒ग्नि॒हो॒त्रमित्य॑ग्नि{-}हो॒त्रम्   ।    & TS\_1.6.9.1       \\
    
    \hline
        
    1280 & प्र॒जाप॑ति॒रिति॑ प्र॒जा{-}प॒तिः॒   ।   वरु॑णाय   ।   अश्व᳚म्   ।   अ॒न॒य॒त्   ।    & TS\_2.3.12.1       \\
    
    \hline
        
    1281 & प्र॒जाप॑ति॒रिति॑ प्र॒जा{-}प॒तिः॒   ।   वाव   ।   ज्येष्ठः॑   ।   सः   ।    & TS\_7.1.1.4       \\
    
    \hline
        
    1282 & प्र॒जाप॑ति॒रिति॑ प्र॒जा{-}प॒तिः॒   ।   वै   ।   अथ॑र्वा   ।   अ॒ग्निः   ।    & TS\_5.6.6.3       \\
    
    \hline
        
    1283 & प्र॒जाप॑ति॒रिति॑ प्र॒जा{-}प॒तिः॒   ।   सु॒व॒र्गमिति॑ सुवः{-}गम्   ।   लो॒कम्   ।   ऐ॒त्   ।    & TS\_7.3.5.1 TS\_7.4.4.1       \\
    
    \hline
        
    1284 & प्र॒जाप॑ते॒रिति॑ प्र॒जा{-}प॒तेः॒   ।   अक्षि॑   ।   अ॒श्व॒य॒त्   ।   तत्   ।    & TS\_5.3.12.1       \\
    
    \hline
        
    1285 & प्र॒जाप॑ते॒रिति॑ प्र॒जा{-}प॒तेः॒   ।   आप्त्यै᳚   ।   न्यू॑न॒येति॒ नि{-}ऊ॒न॒या॒   ।   जु॒हो॒ति॒   ।    & TS\_5.1.9.2       \\
    
    \hline
        
    1286 & प्र॒जाप॑ते॒रिति॑ प्र॒जा{-}प॒तेः॒   ।   जाय॑मानाः   ।   प्र॒जा इति॑ प्र{-}जाः   ।   जा॒ताः   ।    & TS\_3.1.4.1       \\
    
    \hline
        
    1287 & प्र॒जाप॑ते॒रिति॑ प्र॒जा{-}प॒तेः॒   ।   त्रय॑स्त्रिꣳश॒दिति॒ त्रयः॑{-}त्रिꣳ॒॒श॒त्   ।   दु॒हि॒तरः॑   ।   आ॒स॒न्न्   ।    & TS\_2.3.5.1       \\
    
    \hline
        
    1288 & प्र॒जाभ्य॒ इति॑ प्र{-}जाभ्यः॑   ।   मानु॑षीभ्यः   ।   त्वम्   ।   अ॒ङ्गि॒रः॒   ।    & TS\_4.1.4.3       \\
    
    \hline
        
    1289 & प्र॒जामिति॑ प्र {-}जाम्   ।   सो॒मा॒रौ॒द्रमिति॑ सोमा{-} रौ॒द्रम्   ।   च॒रुम्   ।   निरिति॑   ।    & TS\_2.2.10.4       \\
    
    \hline
        
    1290 & प्र॒जामिति॑ प्र{-}जाम्   ।   आपः॑   ।   वै   ।   ओष॑धयः   ।    & TS\_2.1.5.4       \\
    
    \hline
        
    1291 & प्र॒जामिति॑ प्र{-}जाम्   ।   तस्य॑   ।   अ॒न्तः   ।   य॒न्ति॒   ।    & TS\_7.1.3.2       \\
    
    \hline
        
    1292 & प्र॒जामिति॑ प्र{-}जाम्   ।   प॒शून्   ।   अवृ॑ञ्जत   ।   तस्मा᳚त्   ।    & TS\_2.4.3.3       \\
    
    \hline
        
    1293 & प्र॒जामिति॑ प्र{-}जाम्   ।   प॒शून्   ।   यज॑मानस्य   ।   शम॑यितोः   ।    & TS\_3.1.3.2       \\
    
    \hline
        
    1294 & प्र॒ज॒न॒यि॒तेति॑ प्र{-}ज॒न॒यि॒ता   ।   त्वष्टा॑रम्   ।   ए॒व   ।   स्वेन॑   ।    & TS\_2.1.8.4       \\
    
    \hline
        
    1295 & प्र॒ति॒गृह्येति॑ प्रति{-}गृह्य॑   ।   सं॒ॅव॒थ्स॒र इति॑ सं{-}व॒थ्स॒रः   ।   वै   ।   अ॒ग्निः   ।    & TS\_2.2.6.4       \\
    
    \hline
        
    1296 & प्र॒ति॒गृ॒ह्णातीति॑ प्रति{-}गृ॒ह्णाति॑   ।   ताम्   ।   प्रतीति॑   ।   गृ॒ह्णी॒या॒त्   ।    & TS\_7.1.7.3       \\
    
    \hline
        
    1297 & प्र॒ति॒चर॒तीति॑ प्रति{-}चर॑ति   ।   य॒ज्ञेन॑   ।   य॒ज्ञ्म्   ।   वा॒चा   ।    & TS\_2.2.9.7       \\
    
    \hline
        
    1298 & प्र॒ति॒पू॒रु॒षमिति॑ प्रति{-}पू॒रु॒षम्   ।   एक॑कपाला॒नित्येक॑{-}क॒पा॒ला॒न्   ।   निरिति॑   ।   व॒प॒ति॒   ।    & TS\_1.8.6.1       \\
    
    \hline
        
    1299 & प्र॒ति॒ष्ठामिति॑ प्रति{-}स्थाम्   ।   न   ।   अ॒वि॒न्द॒त॒   ।   सः   ।    & TS\_5.6.4.3       \\
    
    \hline
        
    1300 & प्र॒ति॒ष्ठामिति॑ प्रति{-}स्थाम्   ।   वेद॑   ।   प्रतीति॑   ।   ए॒व   ।    & TS\_5.7.3.4       \\
    
    \hline
        
    1301 & प्र॒त्य॒व॒रोहे॑यु॒रिति॑ प्रति{-}अ॒व॒रोहे॑युः   ।   उदिति॑   ।   वा॒   ।   माद्ये॑युः   ।    & TS\_7.3.10.4       \\
    
    \hline
        
    1302 & प्र॒थ॒मः   ।   अꣳ॒॒शुः   ।   स्कन्द॑ति   ।   सः   ।    & TS\_3.1.8.3       \\
    
    \hline
        
    1303 & प्र॒थ॒मः   ।   दे॒व॒य॒तामिति॑ देव{-}य॒ताम्   ।   इति॑   ।   आ॒ह॒   ।    & TS\_5.4.7.2       \\
    
    \hline
        
    1304 & प्र॒थ॒मे   ।   मा॒सि   ।   पृ॒ष्ठानि॑   ।   उपेति॑   ।    & TS\_7.5.3.1       \\
    
    \hline
        
    1305 & प्र॒या॒ज॒त्वमिति॑ प्रयाज{-}त्वम्   ।   यस्य॑   ।   ए॒वम्   ।   वि॒दुषः॑   ।    & TS\_2.6.1.4       \\
    
    \hline
        
    1306 & प्र॒या॒सायेति॑ प्र{-}या॒साय॑   ।   स्वाहा᳚   ।   आ॒या॒सायेत्या᳚{-}या॒साय॑   ।   स्वाहा᳚   ।    & TS\_1.4.35.1       \\
    
    \hline
        
    1307 & प्र॒युज्येति॑ प्र{-}युज्य॑   ।   न   ।   वि॒मु॒ञ्चतीति॑ वि{-}मु॒ञ्चति॑   ।   अ॒प्र॒ति॒ष्ठा॒न इत्य॑प्रति{-}स्था॒नः   ।    & TS\_2.2.6.5       \\
    
    \hline
        
    1308 & प्र॒यु॒ज्यन्त॒ इति॑ प्र{-}यु॒ज्यन्ते᳚   ।   तानि॑   ।   अन्विति॑   ।   ओष॑धयः   ।    & TS\_6.5.11.2       \\
    
    \hline
        
    1309 & प्र॒ह्रि॒यमा॑णा॒येति॑ प्र{-}ह्रि॒यमा॑णाय   ।   अन्विति॑   ।   ब्रू॒हि॒   ।   इति॑   ।    & TS\_6.3.5.4       \\
    
    \hline
        
    1310 & प॒ञ्चभ्य॒ इति॑ प॒ञ्च{-}भ्यः॒   ।   स्वाहा᳚   ।   द॒शभ्य॒ इति॑ द॒श{-}भ्यः॒   ।   स्वाहा᳚   ।    & TS\_7.2.16.1       \\
    
    \hline
        
    1311 & प॒ञ्चा॒शते᳚   ।   स्वाहा᳚   ।   श॒ताय॑   ।   स्वाहा᳚   ।    & TS\_7.2.19.1       \\
    
    \hline
        
    1312 & प॒तय॑न्तम्   ।   प॒त॒ङ्गम्   ।   शिरः॑   ।   अ॒प॒श्य॒म्   ।    & TS\_4.6.7.3       \\
    
    \hline
        
    1313 & प॒थिभि॒रिति॑ प॒थि{-}भिः॒   ।   दे॒व॒यानै॒रिति॑ देव{-}यानैः᳚   ।   इ॒ष्टा॒पू॒र्ते इती᳚ष्टा{-}पू॒र्ते   ।   कृ॒णु॒ता॒त्   ।    & TS\_5.7.7.2       \\
    
    \hline
        
    1314 & प॒दप॑ङ्क्ति॒रिति॑ प॒द{-}प॒ङ्क्तिः॒   ।   छन्दः॑   ।   अ॒क्षर॑पङ्क्ति॒रित्य॒क्षर॑{-}प॒ङ्क्तिः॒   ।   छन्दः॑   ।    & TS\_4.3.12.3       \\
    
    \hline
        
    1315 & प॒दम्   ।   सदा᳚   ।   प॒श्य॒न्ति॒   ।   सू॒रयः॑   ।    & TS\_4.2.9.4       \\
    
    \hline
        
    1316 & प॒दे इति॑   ।   अथो॒ इति॑   ।   प्रति॑ष्ठित्या॒ इति॒ प्रति॑{-}स्थि॒त्यै॒   ।   प्रेति॑   ।    & TS\_5.1.6.4       \\
    
    \hline
        
    1317 & प॒द्य॒न्ते॒   ।   चतु॑र्विꣳशति॒रिति॒ चतुः॑{-}विꣳ॒॒श॒तिः॒   ।   अ॒द्‌र्ध॒मा॒सा इत्य॑द्‌र्ध{-}मा॒साः   ।   अ॒द्‌र्ध॒मा॒सानित्य॑र्ध{-}मा॒सान्   ।    & TS\_6.2.3.5       \\
    
    \hline
        
    1318 & प॒ना॒य॒त॒   ।   मनः॑   ।   प॒श्चात्   ।   अन्विति॑   ।    & TS\_4.6.6.3       \\
    
    \hline
        
    1319 & प॒रस्ता᳚त्   ।   अ॒भीति॑   ।   द्रु॒ह्य॒ति॒   ।   न   ।    & TS\_2.2.6.3       \\
    
    \hline
        
    1320 & प॒रस्ता᳚त्   ।   अ॒र्वाची᳚म्   ।   म॒नु॒ष्याः᳚   ।   ऊर्ज᳚म्   ।    & TS\_6.2.10.3       \\
    
    \hline
        
    1321 & प॒रा॒भ॒वि॒ष्यन्त॒ इति॑ परा{-}भ॒वि॒ष्यन्तः॑   ।   म॒न्या॒म॒हे॒   ।   ततः॑   ।   मा   ।    & TS\_2.5.1.4       \\
    
    \hline
        
    1322 & प॒रि॒गृह्येति॑ परि{-}गृह्य॑   ।   दे॒वाः   ।   य॒ज्ञ्म्   ।   आ॒य॒न्न्   ।    & TS\_4.6.3.3       \\
    
    \hline
        
    1323 & प॒रि॒चा॒य्य॑मिति॑ परि{-}चा॒य्य᳚म्   ।   चि॒न्वी॒त॒   ।   ग्राम॑काम॒ इति॒ ग्राम॑{-}का॒मः॒   ।   ग्रा॒मी   ।    & TS\_5.4.11.3       \\
    
    \hline
        
    1324 & प॒रि॒द॒द्ध्यादिति॑ परि{-}द॒द्ध्यात्   ।   अन्त᳚म्   ।   य॒ज्ञ्म्   ।   ग॒म॒ये॒त्   ।    & TS\_2.4.11.2       \\
    
    \hline
        
    1325 & प॒रि॒भूरिति॑ परि{-}भूः   ।   अ॒ग्निम्   ।   प॒रि॒भूरिति॑ परि{-}भूः   ।   इन्द्र᳚म्   ।    & TS\_3.2.3.1       \\
    
    \hline
        
    1326 & प॒रि॒वे॒ष्टार॒मिति॑ परि{-}वे॒ष्टार᳚म्   ।   ते   ।   सो॒म॒पी॒थेनेति॑ सोम{-}पी॒थेन॑   ।   वीति॑   ।    & TS\_6.3.1.3       \\
    
    \hline
        
    1327 & प॒रोक्ष॒मिति॑ परः{-}अक्ष᳚म्   ।   वै   ।   अ॒न्ये   ।   दे॒वाः   ।    & TS\_1.7.3.1       \\
    
    \hline
        
    1328 & प॒र्याव॑र्तत॒ इति॑ परि{-}आव॑र्तते   ।   द॒क्षि॒णा   ।   प॒र्याव॑र्तत॒ इति॑ परि{-}आव॑र्तते   ।   स्वम्   ।    & TS\_5.2.1.3       \\
    
    \hline
        
    1329 & प॒र॒मे   ।   व्यो॑म॒न्निति॒ वि{-}ओ॒म॒न्न्   ।   इ॒मम्   ।   मा   ।    & TS\_4.2.10.2       \\
    
    \hline
        
    1330 & प॒वित्र᳚म्   ।   अतीति॑   ।   ए॒ति॒   ।   रेभन्न्॑   ।    & TS\_3.4.11.2       \\
    
    \hline
        
    1331 & प॒व॒य॒ति॒   ।   चि॒त्पति॒रिति॑ चित्{-}पतिः॑   ।   त्वा॒   ।   पु॒ना॒तु॒   ।    & TS\_6.1.1.9       \\
    
    \hline
        
    1332 & प॒व॒य॒ति॒   ।   पञ्चा᳚क्ष॒रेति॒ पञ्च॑{-}अ॒क्ष॒रा॒   ।   प॒ङ्क्तिः   ।   पाङ्क्तः॑   ।    & TS\_6.1.1.8       \\
    
    \hline
        
    1333 & प॒व॒से॒   ।   एति॑   ।   सु॒व॒   ।   ऊर्ज᳚म्   ।    & TS\_1.3.14.8       \\
    
    \hline
        
    1334 & प॒शवः॑   ।   उ॒च्य॒न्ते॒   ।   मा   ।   छन्दः॑   ।    & TS\_5.3.2.4       \\
    
    \hline
        
    1335 & प॒शवः॑   ।   वै   ।   इडा᳚   ।   स्व॒यम्   ।    & TS\_2.6.8.1       \\
    
    \hline
        
    1336 & प॒शवः॑   ।   स्युः॒   ।   यत्   ।   पर्य॑ग्निकृता॒निति॒ पर्य॑ग्नि{-}कृ॒ता॒न्   ।    & TS\_5.1.8.3       \\
    
    \hline
        
    1337 & प॒शुः   ।   इति॑   ।   यत्   ।   न   ।    & TS\_6.3.8.2       \\
    
    \hline
        
    1338 & प॒शुः   ।   वै   ।   ए॒षः   ।   यत्   ।    & TS\_5.2.10.1       \\
    
    \hline
        
    1339 & प॒शुका॑म॒ इति॑ प॒शु{-}का॒मः॒   ।   स्यात्   ।   सः   ।   ए॒तम्   ।    & TS\_2.1.1.5       \\
    
    \hline
        
    1340 & प॒शुभि॒रिति॑ प॒शु{-}भिः॒   ।   प्रा॒णानिति॑ प्र{-}अ॒नान्   ।   अ॒न्तः   ।   द॒धी॒त॒   ।    & TS\_6.4.9.4       \\
    
    \hline
        
    1341 & प॒शुभ्य॒ इति॑ प॒शु{-}भ्यः॒   ।   न   ।   अ॒न्तः   ।   ए॒ति॒   ।    & TS\_6.1.8.5       \\
    
    \hline
        
    1342 & प॒शुम्   ।   अ॒कु॒र्व॒त॒   ।   दा॒र्श्यम्   ।   य॒ज्ञ्म्   ।    & TS\_3.2.2.3       \\
    
    \hline
        
    1343 & प॒शुम्   ।   आ॒लभ्येत्या᳚{-}लभ्य॑   ।   पु॒रो॒डाश᳚म्   ।   निरिति॑   ।    & TS\_6.3.10.1       \\
    
    \hline
        
    1344 & प॒शून्   ।   ए॒व   ।   अवेति॑   ।   रु॒न्धे॒   ।    & TS\_5.4.6.3       \\
    
    \hline
        
    1345 & प॒शून्   ।   नि॒र्याच्येति॑ निः{-}याच्य॑   ।   आ॒त्मने᳚   ।   कर्म॑   ।    & TS\_5.1.2.4       \\
    
    \hline
        
    1346 & प॒शू॒नाम्   ।   यत्   ।   ऊषाः᳚   ।   द्यावा॑पृथि॒वी इति॒ द्यावा᳚{-}पृ॒थि॒वी   ।    & TS\_5.2.3.3       \\
    
    \hline
        
    1347 & प॒शोः   ।   अवेति॑   ।   द्य॒ति॒   ।   प॒शुम्   ।    & TS\_6.3.10.3       \\
    
    \hline
        
    1348 & प॒शोः   ।   वै   ।   आल॑ब्ध॒स्येत्या{-}ल॒ब्ध॒स्य॒   ।   प्रा॒णानिति॑ प्र{-}अ॒नान्   ।    & TS\_6.3.9.1       \\
    
    \hline
        
    1349 & प॒श्चात्   ।   प्राची᳚म्   ।   उ॒त्त॒मामित्यु॑त्{-}त॒माम्   ।   उपेति॑   ।    & TS\_5.3.7.3       \\
    
    \hline
        
    1350 & प॒श्चात्   ।   स॒मीची॒ इति॑   ।   ताभिः॑   ।   वै   ।    & TS\_5.2.3.5       \\
    
    \hline
        
    1351 & प॒श्य॒न्ति॒   ।   सूर्य॑स्य   ।   दे॒वाः   ।   अ॒ग्निम्   ।    & TS\_2.3.8.2       \\
    
    \hline
        
    1352 & प॒श॒पति॒रिति॑ पशु{-}पतिः॑   ।   प॒शू॒नाम्   ।   चतु॑ष्पदा॒मिति॒ चतुः॑{-}प॒दा॒म्   ।   उ॒त   ।    & TS\_3.1.4.2       \\
    
    \hline
        
    1353 & बला॑य   ।   अ॒ज॒ग॒रः   ।   आ॒खुः   ।   सृ॒ज॒या   ।    & TS\_5.5.14.1       \\
    
    \hline
        
    1354 & बा॒र्.॒ह॒स्प॒त्यम्   ।   च॒रुम्   ।   निरिति॑   ।   व॒प॒ति॒   ।    & TS\_1.8.9.1       \\
    
    \hline
        
    1355 & बा॒र्.॒ह॒स्प॒त्यम्   ।   शि॒ति॒पृ॒ष्ठमिति॑ शिति{-}पृ॒ष्ठम्   ।   एति॑   ।   ल॒भे॒त॒   ।    & TS\_2.1.6.1       \\
    
    \hline
        
    1356 & बिभ्र॑त्   ।   एति॑   ।   ग॒हि॒   ।   विकि॑रि॒देति॒ वि{-}कि॒रि॒द॒   ।    & TS\_4.5.10.5       \\
    
    \hline
        
    1357 & बिभ॑र्.षि   ।   अस्त॑वे   ।   शि॒वाम्   ।   गि॒रि॒त्रेति॑ गिरि{-}त्र॒   ।    & TS\_4.5.1.2       \\
    
    \hline
        
    1358 & बि॒ल॒धाव॑न॒ इति॑ बिल{-}धाव॑नः   ।   प्रि॒यः   ।   स्त्री॒णाम्   ।   अ॒पी॒च्यः॑   ।    & TS\_7.4.19.2       \\
    
    \hline
        
    1359 & बीज᳚म्   ।   गि॒रा   ।   च॒   ।   श्रु॒ष्टिः   ।    & TS\_4.2.5.6       \\
    
    \hline
        
    1360 & बृह॒स्पतिः॑   ।   अ॒का॒म॒य॒त॒   ।   ब्र॒ह्म॒व॒र्च॒सीति॑ ब्रह्म{-}व॒र्च॒सी   ।   स्या॒म्   ।    & TS\_7.2.3.1       \\
    
    \hline
        
    1361 & बृह॒स्पतिः॑   ।   अ॒का॒म॒य॒त॒   ।   श्रत्   ।   मे॒   ।    & TS\_7.4.1.1       \\
    
    \hline
        
    1362 & बृह॒स्पतिः॑   ।   त्वा॒   ।   सा॒द॒य॒तु॒   ।   पृ॒थि॒व्याः   ।    & TS\_4.4.6.1       \\
    
    \hline
        
    1363 & बृह॒स्पतिः॑   ।   दे॒वाना᳚म्   ।   पु॒रोहि॑त॒ इति॑ पु॒रः{-}हि॒तः॒   ।   आसी᳚त्   ।    & TS\_6.4.10.1       \\
    
    \hline
        
    1364 & बृह॒स्पतिः॑   ।   वा॒चम्   ।   इन्द्रः॑   ।   ज्ये॒ष्ठाना᳚म्   ।    & TS\_1.8.10.2       \\
    
    \hline
        
    1365 & बृह॒स्पतिः॑   ।   हे॒ती॒नाम्   ।   प्र॒ति॒ध॒र्तेति॑ प्रति{-}ध॒र्ता   ।   त्रि॒ण॒व॒त्र॒य॒स्त्रिꣳ॒॒शाविति॑ त्रिणव{-}त्र॒य॒स्त्रिꣳ॒॒शौ   ।    & TS\_4.4.2.3       \\
    
    \hline
        
    1366 & बृह॒स्पति॑सुत॒स्येति॒ बृह॒स्पति॑{-}सु॒त॒स्य॒   ।   ते॒   ।   इ॒न्दो॒ इति॑   ।   इ॒न्द्रि॒याव॑त॒ इती᳚न्द्रि॒य{-}व॒तः॒   ।    & TS\_1.4.27.1       \\
    
    \hline
        
    1367 & बृ॒हत्   ।   अ॒न्नाद्य॒स्येत्य॑न्न{-}अद्य॑स्य   ।   अव॑रुद्ध्या॒ इत्यव॑{-}रु॒द्ध्यै॒   ।   अथो॒ इति॑   ।    & TS\_7.5.8.3       \\
    
    \hline
        
    1368 & बृ॒हन्न्   ।   अद्रिः॑   ।   अ॒भ॒व॒त्   ।   तत्   ।    & TS\_3.3.9.2       \\
    
    \hline
        
    1369 & बृ॒ह॒तीः   ।   नु   ।   शक्व॑रीः   ।   इ॒मम्   ।    & TS\_4.4.12.4       \\
    
    \hline
        
    1370 & ब्रह्मा3न्   ।   त्वम्   ।   रा॒ज॒न्न्   ।   ब्र॒ह्मा   ।    & TS\_1.8.16.2       \\
    
    \hline
        
    1371 & ब्रह्म॑   ।   ज॒ज्ञा॒नम्   ।   इति॑   ।   रु॒क्मम्   ।    & TS\_5.2.7.1       \\
    
    \hline
        
    1372 & ब्रह्म॑णः   ।   पति᳚म्   ।   ए॒व   ।   स्वेन॑   ।    & TS\_2.3.3.5       \\
    
    \hline
        
    1373 & ब्रह्म॑णा   ।   ए॒व   ।   ए॒न॒म्   ।   एति॑   ।    & TS\_3.4.1.2       \\
    
    \hline
        
    1374 & ब्रा॒ह्म॒णेषु॑   ।   रुच᳚म्   ।   राज॒स्विति॒ राज॑{-}सु॒   ।   नः॒   ।    & TS\_5.7.6.4       \\
    
    \hline
        
    1375 & ब्र॒ह्म॒वा॒दिन॒ इति॑ ब्रह्म{-}वा॒दिनः॑   ।   व॒द॒न्ति॒   ।   अ॒ति॒रा॒त्र इत्य॑ति{-}रा॒त्रः   ।   प॒र॒मः   ।    & TS\_7.4.10.1       \\
    
    \hline
        
    1376 & ब्र॒ह्म॒वा॒दिन॒ इति॑ ब्रह्म{-}वा॒दिनः॑   ।   व॒द॒न्ति॒   ।   अ॒द्भिरित्य॑त्{-}भिः   ।   ह॒वीꣳषि॑   ।    & TS\_2.6.5.1       \\
    
    \hline
        
    1377 & ब्र॒ह्म॒वा॒दिन॒ इति॑ ब्रह्म{-}वा॒दिनः॑   ।   व॒द॒न्ति॒   ।   किम्   ।   द्वा॒द॒शा॒हस्येति॑ द्वादश{-}अ॒हस्य॑   ।    & TS\_7.3.2.1       \\
    
    \hline
        
    1378 & ब्र॒ह्म॒वा॒दिन॒ इति॑ ब्रह्म{-}वा॒दिनः॑   ।   व॒द॒न्ति॒   ।   वि॒चित्य॒ इति॑ वि{-}चित्यः॑   ।   सोमा3ः   ।    & TS\_6.1.9.1       \\
    
    \hline
        
    1379 & ब्र॒ह्म॒वा॒दिन॒ इति॑ ब्रह्म{-}वा॒दिनः॑   ।   व॒द॒न्ति॒   ।   सः   ।   तु   ।    & TS\_2.5.4.1 TS\_6.4.3.1 TS\_7.1.3.1       \\
    
    \hline
        
    1380 & ब्र॒ह्म॒वा॒दिन॒ इति॑ ब्रह्म{-}वा॒दिनः॑   ।   व॒द॒न्ति॒   ।   हो॒त॒व्य᳚म्   ।   दी॒क्षि॒तस्य॑   ।    & TS\_6.1.4.5       \\
    
    \hline
        
    1381 & ब्र॒ह्म॒व॒र्च॒समिति॑ ब्रह्म{-}व॒र्च॒सम्   ।   अ॒स्तु॒   ।   इति॑   ।   गा॒य॒त्रि॒या   ।    & TS\_2.5.10.2       \\
    
    \hline
        
    1382 & ब्र॒ह्म॒व॒र्च॒सिन॒ इति॑ ब्रह्म{-}व॒र्च॒सिनः॑   ।   अ॒न्ना॒दा इत्य॑न्न{-}अ॒दाः   ।   भ॒व॒न्ति॒   ।   द्वे इति॑   ।    & TS\_7.3.9.2       \\
    
    \hline
        
    1383 & ब॒द्धम्   ।   अवेति॑   ।   स्य॒ति॒   ।   व॒रु॒ण॒पा॒शादिति॑ वरुण{-}पा॒शात्   ।    & TS\_6.2.9.1       \\
    
    \hline
        
    1384 & ब॒न्धुता᳚   ।   वचो॑भि॒रिति॒ वचः॑{-}भिः॒   ।   तत्   ।   मा॒   ।    & TS\_1.2.14.5       \\
    
    \hline
        
    1385 & ब॒भू॒व॒   ।   सः   ।   प्रेति॑   ।   ज॒ज्ञे॒   ।    & TS\_1.6.6.4       \\
    
    \hline
        
    1386 & ब॒र्॒.हिषः॑   ।   अ॒हम्   ।   दे॒व॒य॒ज्ययेति॑ देव{-}य॒ज्यया᳚   ।   प्र॒जावा॒निति॑ प्र॒जा {-}वा॒न्   ।    & TS\_1.7.4.1       \\
    
    \hline
        
    1387 & ब॒र्॒.हिषः॑   ।   अ॒हम्   ।   दे॒व॒य॒ज्ययेति॑ देव{-}य॒ज्यया᳚   ।   प्र॒जावा॒निति॑ प्र॒जा{-}वा॒न्   ।    & TS\_1.6.4.1       \\
    
    \hline
        
    1388 & भक्ष॑   ।   एति॑   ।   इ॒हि॒   ।   मा॒   ।    & TS\_3.2.5.1       \\
    
    \hline
        
    1389 & भव॑ति   ।   अथ॑   ।   तत्   ।   न   ।    & TS\_7.2.7.4       \\
    
    \hline
        
    1390 & भव॑ति   ।   अनि॑ष्टम्   ।   व॒शया᳚   ।   अथ॑   ।    & TS\_6.6.6.2       \\
    
    \hline
        
    1391 & भव॑ति   ।   अ॒स्याः   ।   अन॑तिदाहा॒येत्यन॑ति{-}दा॒हा॒य॒   ।   द्वि॒गु॒णमिति॑ द्वि{-}गु॒णम्   ।    & TS\_5.2.5.3       \\
    
    \hline
        
    1392 & भव॑ति   ।   यत्   ।   आल॑ब्धाया॒मित्या{-}ल॒ब्धा॒या॒म्   ।   अ॒भ्रः   ।    & TS\_3.4.3.8       \\
    
    \hline
        
    1393 & भव॑ति   ।   व्यचि॑ष्ठम्   ।   अन्न᳚म्   ।   र॒भ॒सम्   ।    & TS\_5.1.3.3       \\
    
    \hline
        
    1394 & भव॑ति   ।   सौ॒म्याः   ।   अ॒स्य॒   ।   आहु॑तय॒ इत्या{-}हु॒त॒यः॒   ।    & TS\_3.5.7.2       \\
    
    \hline
        
    1395 & भव॑न्ति   ।   अ॒मुम्   ।   ए॒व   ।   तैः   ।    & TS\_7.3.4.2       \\
    
    \hline
        
    1396 & भव॑न्ति   ।   एक॑हाय॒न्येत्येक॑{-}हा॒य॒न्या॒   ।   क्री॒णा॒ति॒   ।   वा॒चा   ।    & TS\_6.1.6.7       \\
    
    \hline
        
    1397 & भव॑न्तीः   ।   न   ।   प्रेति॑   ।   अ॒जा॒य॒न्त॒   ।    & TS\_5.3.6.3       \\
    
    \hline
        
    1398 & भार॑ती   ।   म॒ही   ।   गृ॒णा॒ना   ।   तत्   ।    & TS\_4.1.8.3       \\
    
    \hline
        
    1399 & भा॒ग॒धेये॒नेति॑ भाग{-}धेये॑न   ।   उपेति॑   ।   धा॒व॒ति॒   ।   सः   ।    & TS\_3.4.3.3       \\
    
    \hline
        
    1400 & भि॒षजौ᳚   ।   इति॑   ।   तस्मा᳚त्   ।   ब्रा॒ह्म॒णेन॑   ।    & TS\_6.4.9.2       \\
    
    \hline
        
    1401 & भि॒षजौ᳚   ।   ताभ्या᳚म्   ।   ए॒व   ।   अ॒स्मै॒   ।    & TS\_2.3.11.3       \\
    
    \hline
        
    1402 & भुवः॑   ।   इति॑   ।   पु॒रस्ता᳚त्   ।   उपेति॑   ।    & TS\_5.2.10.4       \\
    
    \hline
        
    1403 & भुवः॑   ।   दे॒वाना᳚म्   ।   कर्म॑णा   ।   अ॒पसा᳚   ।    & TS\_7.1.18.1       \\
    
    \hline
        
    1404 & भुव॑नम्   ।   अ॒सि॑   ।   वीति॑   ।   प्र॒थ॒स्व॒   ।    & TS\_1.1.12.1       \\
    
    \hline
        
    1405 & भुव॑नस्य   ।   रेतः॑   ।   गा॒तुम्   ।   ध॒त्त॒   ।    & TS\_3.1.4.3       \\
    
    \hline
        
    1406 & भूः   ।   भुवः॑   ।   सुवः॑   ।   वस॑वः   ।    & TS\_7.4.20.1       \\
    
    \hline
        
    1407 & भूमिः॑   ।   भू॒म्ना   ।   द्यौः   ।   व॒रि॒णा   ।    & TS\_1.5.3.1 TS\_1.5.4.1       \\
    
    \hline
        
    1408 & भू॒तम्   ।   भव्य᳚म्   ।   भ॒वि॒ष्यत्   ।   वष॑ट्   ।    & TS\_7.3.12.1       \\
    
    \hline
        
    1409 & भू॒ते॒ष्ट॒का इति॑ भूत{-}इ॒ष्ट॒काः   ।   उपेति॑   ।   द॒धा॒ति॒   ।   अत्रा॒त्रेत्यत्र॑{-}अ॒त्र॒   ।    & TS\_5.6.3.1       \\
    
    \hline
        
    1410 & भू॒य॒स्कृदिति॑ भूयः{-}कृत्   ।   अ॒सि॒   ।   व॒रि॒व॒स्कृदिति॑ वरिवः{-}कृत्   ।   अ॒सि॒   ।    & TS\_4.4.7.1       \\
    
    \hline
        
    1411 & भे॒ष॒जम्   ।   ए॒व   ।   अ॒स्मै॒   ।   क॒रो॒ति॒   ।    & TS\_2.2.10.3       \\
    
    \hline
        
    1412 & भ्राजः॑   ।   अ॒सि॒   ।   दे॒वाना᳚म्   ।   धाम॑   ।    & TS\_2.4.3.2       \\
    
    \hline
        
    1413 & भ्राज॑   ।   इति॑   ।   आ॒ह॒   ।   ए॒ते   ।    & TS\_6.1.10.5       \\
    
    \hline
        
    1414 & भ॒द्रात्   ।   अ॒भीति॑   ।   श्रेयः॑   ।   प्रेति॑   ।    & TS\_3.1.1.4       \\
    
    \hline
        
    1415 & भ॒व॒   ।   यज॑मानाय   ।   शम्   ।   योः   ।    & TS\_3.2.11.3       \\
    
    \hline
        
    1416 & भ॒व॒तः॒   ।   प्रि॒यम्   ।   ए॒व   ।   ए॒न॒म्   ।    & TS\_2.2.11.5       \\
    
    \hline
        
    1417 & भ॒व॒ति॒   ।   अङ्गा॑राः   ।   ए॒व   ।   प्र॒ति॒वेष्ट॑माना॒ इति॑ प्रति{-}वेष्ट॑मानाः   ।    & TS\_3.4.8.4       \\
    
    \hline
        
    1418 & भ॒व॒ति॒   ।   अ॒न्नाद्य॒स्येत्य॑न्न{-}अद्य॑स्य   ।   अव॑रुद्ध्या॒ इत्यव॑{-}रु॒द्ध्यै॒   ।   अथो॒ इति॑   ।    & TS\_7.2.1.3       \\
    
    \hline
        
    1419 & भ॒व॒ति॒   ।   इ॒मम्   ।   स्तन᳚म्   ।   ऊर्ज॑स्वन्तम्   ।    & TS\_5.5.10.7       \\
    
    \hline
        
    1420 & भ॒व॒ति॒   ।   ऊर्क्   ।   वै   ।   उ॒दु॒बंरः॑   ।    & TS\_2.5.4.4       \\
    
    \hline
        
    1421 & भ॒व॒ति॒   ।   ज्योतिः॑   ।   ए॒व   ।   पु॒रस्ता᳚त्   ।    & TS\_7.4.6.2       \\
    
    \hline
        
    1422 & भ॒व॒ति॒   ।   ताम्   ।   अ॒फ्स्वित्य॑प्{-}सु   ।   प्रेति॑   ।    & TS\_7.1.6.4       \\
    
    \hline
        
    1423 & भ॒व॒ति॒   ।   मु॒ख॒तः   ।   ए॒व   ।   अ॒स्मि॒न्न्   ।    & TS\_2.1.10.2       \\
    
    \hline
        
    1424 & भ॒व॒ति॒   ।   यः   ।   ए॒वम्   ।   वेद॑   ।    & TS\_5.4.1.3       \\
    
    \hline
        
    1425 & भ॒व॒ति॒   ।   यः   ।   वै   ।   ए॒तासा᳚म्   ।    & TS\_5.6.2.5       \\
    
    \hline
        
    1426 & भ॒व॒ति॒   ।   यत्   ।   अ॒ष्टाक॑पाल॒ इत्य॒ष्टा{-}क॒पा॒लः॒   ।   तेन॑   ।    & TS\_2.3.3.4       \\
    
    \hline
        
    1427 & भ॒व॒ति॒   ।   य॒ज्ञ्ः   ।   दे॒वेभ्यः॑   ।   अपेति॑   ।    & TS\_5.4.1.2       \\
    
    \hline
        
    1428 & भ॒व॒ति॒   ।   वै॒श्व॒दे॒व॒त्वायेति॑ वैश्वदेव{-}त्वाय॑   ।   स॒म॒न्तमिति॑ सं{-}अ॒न्तम्   ।   प॒र्यव॑द्य॒तीति॑ परि{-}अव॑द्यति   ।    & TS\_2.3.7.4       \\
    
    \hline
        
    1429 & भ॒व॒ति॒   ।   सा॒ह॒स्री   ।   वै   ।   ए॒षा   ।    & TS\_2.1.5.2       \\
    
    \hline
        
    1430 & भ॒व॒ति॒   ।   स॒त्त्वमिति॑ सत्{-}त्वम्   ।   ए॒व   ।   ए॒न॒म्   ।    & TS\_5.2.1.6       \\
    
    \hline
        
    1431 & भ॒व॒न्ति॒   ।   एक॑विꣳशति॒मित्येक॑{-}विꣳ॒॒श॒ति॒म्   ।   सा॒मि॒धे॒नीरिति॑ सां{-}इ॒धे॒नीः   ।   अन्विति॑   ।    & TS\_5.1.8.5       \\
    
    \hline
        
    1432 & मक्षि॑का   ।   आश॑   ।   यत्   ।   वा॒   ।    & TS\_4.6.8.4       \\
    
    \hline
        
    1433 & मधुः॑   ।   च॒   ।   माध॑वः   ।   च॒   ।    & TS\_1.4.14.1 TS\_4.4.11.1       \\
    
    \hline
        
    1434 & मनः॑   ।   वि॒ष्व॒द्रिय॒गिति॑ विष्व{-}द्रिय॑क्   ।   वीति॑   ।   चा॒री॒त्   ।    & TS\_1.7.13.3       \\
    
    \hline
        
    1435 & मनुः॑   ।   पृ॒थि॒व्याः   ।   य॒ज्ञिय᳚म्   ।   ऐ॒च्छ॒त्   ।    & TS\_2.6.7.1       \\
    
    \hline
        
    1436 & मन्त्र᳚म्   ।   अ॒प॒श्य॒त्   ।   ततः॑   ।   वै   ।    & TS\_1.5.4.2       \\
    
    \hline
        
    1437 & मन्या॑सै   ।   शम्   ।   च॒   ।   नः॒   ।    & TS\_3.3.11.4       \\
    
    \hline
        
    1438 & मन॑सा   ।   मा॒   ।   भू॒तेन॑   ।   एति॑   ।    & TS\_1.6.10.6       \\
    
    \hline
        
    1439 & मम॑   ।   अ॒ग्ने॒   ।   वर्चः॑   ।   वि॒ह॒वेष्विति॑ वि{-}ह॒वेषु॑   ।    & TS\_4.7.14.1       \\
    
    \hline
        
    1440 & मम॑   ।   नाम॑   ।   प्र॒थ॒मम्   ।   जा॒त॒वे॒द॒ इति॑ जात{-}वे॒दः॒   ।    & TS\_1.5.10.1       \\
    
    \hline
        
    1441 & मयि॑   ।   अग्ने᳚   ।   पा॒व॒क॒   ।   रो॒चिषा᳚   ।    & TS\_1.5.5.3       \\
    
    \hline
        
    1442 & मयि॑   ।   गृ॒ह्णा॒मि॒   ।   अग्रे᳚   ।   अ॒ग्निम्   ।    & TS\_5.7.9.1       \\
    
    \hline
        
    1443 & मयि॑   ।   धे॒हि॒   ।   वृ॒त्रस्य॑   ।   क॒नीनि॑का   ।    & TS\_1.2.1.2       \\
    
    \hline
        
    1444 & मयि॑   ।   वः॒   ।   रायः॑   ।   श्र॒य॒न्ता॒म्   ।    & TS\_1.5.6.4       \\
    
    \hline
        
    1445 & मयि॑   ।   वी॒र्य᳚म्   ।   तत्   ।   ते॒   ।    & TS\_2.4.12.4 TS\_2.4.12.5       \\
    
    \hline
        
    1446 & मर॑ते   ।   पतिः॑   ।   न   ।   अ॒हम्   ।    & TS\_1.7.13.2       \\
    
    \hline
        
    1447 & मल᳚म्   ।   जा॒य॒ते॒   ।   एकः॑   ।   ए॒व   ।    & TS\_7.2.10.3       \\
    
    \hline
        
    1448 & मह्य᳚म्   ।   इ॒मान्   ।   इति॑   ।   अ॒ब्र॒वी॒त्   ।    & TS\_3.1.9.6       \\
    
    \hline
        
    1449 & मा   ।   छन्दः॑   ।   प्र॒मेति॑ प्र{-}मा   ।   छन्दः॑   ।    & TS\_4.3.7.1       \\
    
    \hline
        
    1450 & मा   ।   नः॒   ।   मि॒त्रः   ।   वरु॑णः   ।    & TS\_4.6.8.1       \\
    
    \hline
        
    1451 & मा   ।   नः॒   ।   हिꣳ॒॒सी॒त्   ।   ज॒नि॒ता   ।    & TS\_4.2.7.1       \\
    
    \hline
        
    1452 & मा   ।   मा॒   ।   हिꣳ॒॒सीः॒   ।   कृ॒ष्यै   ।    & TS\_1.2.2.3       \\
    
    \hline
        
    1453 & मात्रा᳚   ।   ए॒व   ।   अ॒स्यै॒   ।   सा   ।    & TS\_6.2.7.2       \\
    
    \hline
        
    1454 & माम्   ।   मा॒ता   ।   पृ॒थि॒वी   ।   हिꣳ॒॒सी॒त्   ।    & TS\_1.8.15.2       \\
    
    \hline
        
    1455 & मासाः᳚   ।   सं॒ॅव॒थ्स॒र इति॑ सं{-}व॒थ्स॒रः   ।   सं॒ॅव॒थ्स॒र इति॑ सं{-}व॒थ्स॒रः   ।   वि॒राडिति॑ वि{-}  राट्   ।    & TS\_5.6.7.2       \\
    
    \hline
        
    1456 & मासाः᳚   ।   सं॒ॅव॒थ्स॒र इति॑ सं{-}व॒थ्स॒रः   ।   सं॒ॅव॒थ्स॒रमिति॑ सं{-}व॒थ्स॒रम्   ।   ए॒व   ।    & TS\_6.3.7.3       \\
    
    \hline
        
    1457 & मा॒   ।   प्र॒ति॒ष्ठामिति॑ प्रति{-}स्थाम्   ।   ग॒म॒य॒   ।   प्र॒जयेति॑ प्र{-}जया᳚   ।    & TS\_7.1.7.2       \\
    
    \hline
        
    1458 & मा॒तरा᳚   ।   च॒   ।   एति॑   ।   मा॒   ।    & TS\_1.7.8.4       \\
    
    \hline
        
    1459 & मा॒तर᳚म्   ।   च॒   ।   पु॒त्रम्   ।   च॒   ।    & TS\_3.3.10.2       \\
    
    \hline
        
    1460 & मा॒तृत॑मा॒स्विति॑ मा॒तृ{-}त॒मा॒सु॒   ।   अ॒न्तः   ।   क्ष॒त्रस्य॑   ।   उल्ब᳚म्   ।    & TS\_1.8.12.2       \\
    
    \hline
        
    1461 & मा॒त्रे   ।   व॒थ्सम्   ।   उ॒पाव॑सृज॒तीत्यु॑प{-}अव॑सृजति   ।   सर्वे॑ण   ।    & TS\_1.7.1.3       \\
    
    \hline
        
    1462 & मा॒याः   ।   मा॒यिना᳚म्   ।   वि॒श्व॒मि॒न्वेति॑ विश्वम्{-}इ॒न्व॒   ।   त्वे इति॑   ।    & TS\_3.1.11.7       \\
    
    \hline
        
    1463 & मा॒रु॒तम्   ।   अ॒सि॒   ।   म॒रुता᳚म्   ।   ओजः॑   ।    & TS\_2.4.7.1 TS\_2.4.9.1       \\
    
    \hline
        
    1464 & मि॒त्रः   ।   अ॒सि॒   ।   वरु॑णः   ।   अ॒सि॒   ।    & TS\_1.8.16.1       \\
    
    \hline
        
    1465 & मि॒त्रम्   ।   दे॒वाः   ।   अ॒ब्रु॒व॒न्न्   ।   सोम᳚म्   ।    & TS\_6.4.8.1       \\
    
    \hline
        
    1466 & मि॒त्रावरु॑णा॒विति॑ मि॒त्रा{-}वरु॑णौ   ।   श्रोणी᳚भ्या॒मिति॒ श्रोणि॑{-}भ्या॒म्   ।   इ॒न्द्रा॒ग्नी इती᳚न्द्र{-}अ॒ग्नी   ।   शि॒ख॒ण्डाभ्या᳚म्   ।    & TS\_5.7.15.1       \\
    
    \hline
        
    1467 & मि॒त्रावरु॒णेति॑ मि॒त्रा{-}वरु॑णा   ।   ना॒थि॒तः   ।   जो॒ह॒वी॒मि॒   ।   तौ   ।    & TS\_4.7.15.3       \\
    
    \hline
        
    1468 & मि॒थु॒नम्   ।   प॒शवः॑   ।   वै   ।   ए॒ते   ।    & TS\_6.5.6.4       \\
    
    \hline
        
    1469 & मि॒थु॒नम्   ।   म॒द्ध्य॒तः   ।   द॒धा॒ति॒   ।   पुष्ट्यै᳚   ।    & TS\_2.4.6.2       \\
    
    \hline
        
    1470 & मि॒मी॒ते॒   ।   अया॑तयाम्नियायातयाम्नि॒येत्यया॑तयाम्निया{-}अ॒या॒त॒या॒म्नि॒या॒   ।   ए॒व   ।   ए॒न॒म्   ।    & TS\_6.1.9.5       \\
    
    \hline
        
    1471 & मु॒ख॒तः   ।   ए॒व   ।   अ॒स्मै॒   ।   अ॒न्नाद्य॒मित्य॑न्न {-}अद्य᳚म्   ।    & TS\_5.6.3.3       \\
    
    \hline
        
    1472 & मु॒ख॒तः   ।   ध॒त्ते॒   ।   मू॒द्‌र्ध॒न्वती॒रिति॑ मूर्धन्न्{-}वतीः᳚   ।   भ॒व॒न्ति॒   ।    & TS\_5.3.8.2       \\
    
    \hline
        
    1473 & मु॒ञ्च॒ति॒   ।   तस्मा᳚त्   ।   शी॒र्॒.ष॒तः   ।   ओष॑धयः   ।    & TS\_6.3.4.3       \\
    
    \hline
        
    1474 & मू॒द्‌र्ध॒न्वती॒रिति॑ मूद्‌र्धन्न्{-}वतीः᳚   ।   भ॒व॒न्ति॒   ।   तस्मा᳚त्   ।   पु॒रस्ता᳚त्   ।    & TS\_5.3.1.5       \\
    
    \hline
        
    1475 & मू॒र्धान᳚म्   ।   दि॒वः   ।   अ॒र॒तिम्   ।   पृ॒थि॒व्याः   ।    & TS\_1.4.13.1       \\
    
    \hline
        
    1476 & मृ॒डा॒ति॒   ।   ई॒दृशे᳚   ।   क्षेत्र॑स्य   ।   प॒ते॒   ।    & TS\_1.1.14.3       \\
    
    \hline
        
    1477 & मृ॒त्युः   ।   ग॒न्ध॒र्वः   ।   तस्य॑   ।   प्र॒जा इति॑ प्र{-}जाः   ।    & TS\_3.4.7.3       \\
    
    \hline
        
    1478 & मृ॒दा   ।   चि॒नोति॑   ।   तस्मा᳚त्   ।   अ॒ग्निः   ।    & TS\_5.7.9.4       \\
    
    \hline
        
    1479 & मेख॑ला   ।   भव॑ति   ।   ऊर्ज᳚म्   ।   ए॒व   ।    & TS\_6.1.3.4       \\
    
    \hline
        
    1480 & मेद॑सा   ।   स्रुचौ᳚   ।   प्रेति॑   ।   ऊ॒र्णो॒ति॒   ।    & TS\_6.3.11.1       \\
    
    \hline
        
    1481 & मे॒   ।   अ॒कृ॒ष्ट॒प॒च्यमित्य॑कृष्ट{-}प॒च्यम्   ।   च॒   ।   मे॒   ।    & TS\_4.7.5.2       \\
    
    \hline
        
    1482 & मे॒   ।   अ॒ति॒ग्रा॒ह्या᳚ इत्य॑ति{-}ग्रा॒ह्याः᳚   ।   च॒   ।   मे॒   ।    & TS\_4.7.7.2       \\
    
    \hline
        
    1483 & मे॒   ।   इति॑   ।   आ॒ह॒   ।   स॒वि॒तृप्र॑सूत॒ इति॑ सवि॒तृ{-}प्र॒सू॒तः॒   ।    & TS\_6.4.3.3       \\
    
    \hline
        
    1484 & मे॒   ।   इन्द्रः॑   ।   च॒   ।   मे॒   ।    & TS\_4.7.6.2       \\
    
    \hline
        
    1485 & मे॒   ।   चत॑स्रः   ।   च॒   ।   मे॒   ।    & TS\_4.7.11.2       \\
    
    \hline
        
    1486 & मे॒   ।   धन᳚म्   ।   च॒   ।   मे॒   ।    & TS\_4.7.2.2       \\
    
    \hline
        
    1487 & मे॒   ।   प्र॒भ्विति॑ प्र{-}भु   ।   च॒   ।   मे॒   ।    & TS\_4.7.4.2       \\
    
    \hline
        
    1488 & मे॒   ।   महः॑   ।   च॒   ।   मे॒   ।    & TS\_4.7.3.2       \\
    
    \hline
        
    1489 & मे॒   ।   शर्म॑   ।   च॒   ।   वर्म॑   ।    & TS\_4.4.5.2       \\
    
    \hline
        
    1490 & मे॒षः   ।   त्वा॒   ।   प॒च॒तैः   ।   अ॒व॒तु॒   ।    & TS\_7.4.12.1       \\
    
    \hline
        
    1491 & मौ॒क्   ।   अ॒ररुः॑   ।   ते॒   ।   दिव᳚म्   ।    & TS\_1.1.9.3       \\
    
    \hline
        
    1492 & म॒   ।   वे॒हत्   ।   च॒   ।   मे॒   ।    & TS\_4.7.10.2       \\
    
    \hline
        
    1493 & म॒द्ध्य॒तः   ।   ऊ॒र्जा   ।   भु॒ञ्ज॒ते॒   ।   य॒ज॒मा॒न॒लो॒क इति॑ यजमान{-}लो॒के   ।    & TS\_6.2.10.7       \\
    
    \hline
        
    1494 & म॒नु॒ष्या॑णाम्   ।   अन्न᳚म्   ।   प्र॒जाप॑ति॒मिति॑ प्र॒जा{-}प॒ति॒म्   ।   प्र॒जा इति॑ प्र{-}जाः   ।    & TS\_3.3.6.3       \\
    
    \hline
        
    1495 & म॒नु॒ष्य॒छ॒न्द॒समिति॑ मनुष्य{-}छ॒न्द॒सम्   ।   चत॑स्रः   ।   च॒   ।   अ॒ष्टौ   ।    & TS\_5.4.8.6       \\
    
    \hline
        
    1496 & म॒नो॒युज॒मिति॑ मनः{-}युज᳚म्   ।   वाजे᳚   ।   अ॒द्य   ।   हु॒वे॒म॒   ।    & TS\_4.6.2.6       \\
    
    \hline
        
    1497 & म॒युः   ।   प्रा॒जा॒प॒त्य इति॑ प्राजा{-}प॒त्यः   ।   ऊ॒लः   ।   हली᳚क्ष्णः   ।    & TS\_5.5.12.1       \\
    
    \hline
        
    1498 & म॒यो॒भूरिति॑ मयः{-}भूः   ।   वातः॑   ।   अ॒भीति॑   ।   वा॒तु॒   ।    & TS\_7.4.17.1       \\
    
    \hline
        
    1499 & म॒रुतः॑   ।   वै   ।   दे॒वाना᳚म्   ।   विशः॑   ।    & TS\_2.2.5.7       \\
    
    \hline
        
    1500 & म॒रुत्वान्॑   ।   इ॒न्द्र॒   ।   वृ॒ष॒भः   ।   रणा॑य   ।    & TS\_1.4.19.1       \\
    
    \hline
        
    1501 & म॒रुत्व॑न्तम्   ।   वृ॒ष॒भम्   ।   वा॒वृ॒धा॒नम्   ।   अक॑वारि॒मित्यक॑वा{-}अ॒रि॒म्   ।    & TS\_1.4.17.1       \\
    
    \hline
        
    1502 & म॒हान्   ।   इन्द्रः॑   ।   नृ॒वदिति॑ नृ{-}वत्   ।   एति॑   ।    & TS\_1.4.21.1       \\
    
    \hline
        
    1503 & म॒हान्   ।   इन्द्रः॑   ।   यः   ।   ओज॑सा   ।    & TS\_1.4.20.1       \\
    
    \hline
        
    1504 & म॒हान्   ।   इन्द्रः॑   ।   वज्र॑बाहु॒रिति॒ वज्र॑{-}बा॒हुः॒   ।   षो॒ड॒शी   ।    & TS\_1.4.41.1       \\
    
    \hline
        
    1505 & म॒हि॒ना   ।   दिव᳚म्   ।   मि॒त्रः   ।   ब॒भू॒व॒   ।    & TS\_4.1.6.3       \\
    
    \hline
        
    1506 & म॒हि॒नि॒   ।   स्तोमा॑सः   ।   त्वा॒   ।   वि॒चा॒रि॒णीति॑ वि{-}चा॒रि॒णि॒   ।    & TS\_2.2.12.3       \\
    
    \hline
        
    1507 & म॒ही   ।   दे॒वस्य॑   ।   स॒वि॒तुः   ।   परि॑ष्टुति॒रिति॒ परि॑{-}स्तु॒तिः॒   ।    & TS\_4.1.1.2       \\
    
    \hline
        
    1508 & म॒ही॒नाम्   ।   पयः॑   ।   अ॒सि॒   ।   विश्वे॑षाम्   ।    & TS\_3.2.6.1       \\
    
    \hline
        
    1509 & य   ।   इ॒मा   ।   विश्वा᳚   ।   भुव॑नानि   ।    & TS\_4.6.2.1       \\
    
    \hline
        
    1510 & यः   ।   अन्ति॑   ।   अग्ने᳚   ।   माकिः॑   ।    & TS\_1.2.14.2       \\
    
    \hline
        
    1511 & यः   ।   अर्व॑न्तम्   ।   जिघाꣳ॑सति   ।   तम्   ।    & TS\_7.4.15.1       \\
    
    \hline
        
    1512 & यः   ।   अꣳ॒॒शुम्   ।   गृ॒ह्णाति॑   ।   एति॑   ।    & TS\_3.3.4.3       \\
    
    \hline
        
    1513 & यः   ।   आ॒त्म॒दा इत्या᳚त्म{-}दाः   ।   ब॒ल॒दा इति॑ बल{-}दाः   ।   यस्य॑   ।    & TS\_7.5.17.1       \\
    
    \hline
        
    1514 & यः   ।   ए॒का॒द॒शः   ।   स्तनः॑   ।   ए॒व   ।    & TS\_6.6.4.6       \\
    
    \hline
        
    1515 & यः   ।   ए॒व   ।   अव॑गत॒ इत्यव॑{-}ग॒तः॒   ।   सः   ।    & TS\_6.6.5.4       \\
    
    \hline
        
    1516 & यः   ।   ए॒वम्   ।   वि॒द्वान्   ।   अ॒ग्निम्   ।    & TS\_1.5.8.5 TS\_5.7.4.3       \\
    
    \hline
        
    1517 & यः   ।   ए॒वम्   ।   वि॒द्वान्   ।   द॒श॒रा॒त्रेणेति॑ दश{-}रा॒त्रेण॑   ।    & TS\_7.2.5.2       \\
    
    \hline
        
    1518 & यः   ।   ए॒वम्   ।   वि॒द्वान्   ।   प॒ञ्च॒रा॒त्रेणेति॑ पञ्च{-}रा॒त्रेण॑   ।    & TS\_7.1.10.2       \\
    
    \hline
        
    1519 & यः   ।   ए॒वम्   ।   वेद॑   ।   ए॒षः   ।    & TS\_3.1.9.4       \\
    
    \hline
        
    1520 & यः   ।   ए॒वम्   ।   वेद॑   ।   यत्   ।    & TS\_2.5.11.4       \\
    
    \hline
        
    1521 & यः   ।   ज्ये॒ष्ठब॑न्धु॒रिति॑ ज्ये॒ष्ठ{-}ब॒न्धुः॒   ।   अप॑भूत॒ इत्यप॑{-}भू॒तः॒   ।   स्यात्   ।    & TS\_3.4.8.7       \\
    
    \hline
        
    1522 & यः   ।   त्वा॒   ।   हृ॒दा   ।   की॒रिणा᳚   ।    & TS\_1.4.46.1       \\
    
    \hline
        
    1523 & यः   ।   दे॒वाना᳚म्   ।   ना॒म॒धा इति॑ नाम{-}धाः   ।   एकः॑   ।    & TS\_4.6.2.2       \\
    
    \hline
        
    1524 & यः   ।   प्रा॒ण॒त इति॑ प्र{-}अ॒न॒तः   ।   नि॒मि॒ष॒त इति॑ नि{-}मि॒ष॒तः   ।   म॒हि॒त्वेति॑ महि{-}त्वा   ।    & TS\_7.5.16.1       \\
    
    \hline
        
    1525 & यः   ।   वि॒द्वि॒षा॒णयो॒रिति॑ वि{-}द्वि॒षा॒णयोः᳚   ।   अन्न᳚म्   ।   अत्ति॑   ।    & TS\_2.2.6.2       \\
    
    \hline
        
    1526 & यः   ।   वै   ।   अय॑थादेवत॒मित्यय॑था{-}दे॒व॒त॒म्   ।   अ॒ग्निम्   ।    & TS\_5.7.1.1       \\
    
    \hline
        
    1527 & यः   ।   वै   ।   अय॑थादेवत॒मित्यय॑था{-}दे॒व॒त॒म्   ।   य॒ज्ञ्म्   ।    & TS\_3.1.6.1       \\
    
    \hline
        
    1528 & यः   ।   वै   ।   अश्व॑स्य   ।   मेद्ध्य॑स्य   ।    & TS\_7.5.25.1       \\
    
    \hline
        
    1529 & यः   ।   वै   ।   अ॒ग्नौ   ।   अ॒ग्निः   ।    & TS\_7.5.15.1       \\
    
    \hline
        
    1530 & यः   ।   वै   ।   दे॒वान्   ।   दे॒व॒य॒श॒सेनेति॑ देव{-}य॒श॒सेन॑   ।    & TS\_3.1.9.1       \\
    
    \hline
        
    1531 & यः   ।   वै   ।   पव॑मानानाम्   ।   अ॒न्वा॒रो॒हानित्य॑नु{-}आ॒रो॒हान्   ।    & TS\_3.2.1.1       \\
    
    \hline
        
    1532 & यः   ।   वै   ।   श्र॒द्धामिति॑ श्रत्{-}धाम्   ।   अना॑र॒भ्येत्यना᳚{-}र॒भ्य॒   ।    & TS\_1.6.8.1       \\
    
    \hline
        
    1533 & यः   ।   वै   ।   स॒प्त॒द॒शमिति॑ सप्त{-}द॒शम्   ।   प्र॒जाप॑ति॒मिति॑ प्र॒जा{-}प॒ति॒म्   ।    & TS\_1.6.11.1       \\
    
    \hline
        
    1534 & यः   ।   स॒हस्रे॑ण   ।   यज॑ते   ।   प्र॒जाप॑ति॒रिति॑ प्र॒जा{-}प॒तिः॒   ।    & TS\_2.4.11.4       \\
    
    \hline
        
    1535 & यजु॑षा   ।   वै   ।   ए॒षा   ।   क्रि॒य॒ते॒   ।    & TS\_5.5.3.1       \\
    
    \hline
        
    1536 & यज॑ते   ।   अपीति॑   ।   अ॒न्यम्   ।   यज॑मानम्   ।    & TS\_1.7.1.2       \\
    
    \hline
        
    1537 & यज॑मानः   ।   अ॒हो॒रा॒त्राणीत्य॑हः{-}रा॒त्राणि॑   ।   वै   ।   ए॒तस्य॑   ।    & TS\_3.4.10.2       \\
    
    \hline
        
    1538 & यज॑मानः   ।   च॒   ।   सी॒द॒त॒   ।   प्रेद्ध॒ इति॒ प्र{-}इ॒द्धः॒   ।    & TS\_4.6.5.4       \\
    
    \hline
        
    1539 & यज॑मानः   ।   प्र॒स्त॒र इति॑ प्र{-}स्त॒रः   ।   यत्   ।   ए॒तैः   ।    & TS\_1.7.4.4       \\
    
    \hline
        
    1540 & यज॑मानः   ।   ह॒न्येत॑   ।   अ॒स्य॒   ।   य॒ज्ञ्ः   ।    & TS\_5.1.9.3       \\
    
    \hline
        
    1541 & यज॑मानस्य   ।   आ॒यत॑न॒मित्या᳚{-}यत॑नम्   ।   यत्   ।   वेदिः॑   ।    & TS\_1.7.5.3       \\
    
    \hline
        
    1542 & यज॑मानस्य   ।   यत्   ।   अना᳚र्तः   ।   उ॒दृच॒मित्यु॑त्{-}ऋच᳚म्   ।    & TS\_3.4.3.6       \\
    
    \hline
        
    1543 & यज॑मानाय   ।   चि॒नु॒या॒त्   ।   अ॒ग्निम्   ।   खलु॑   ।    & TS\_5.7.9.2       \\
    
    \hline
        
    1544 & यज॑मानाय   ।   स्वाहा᳚   ।   इति॑   ।   ते   ।    & TS\_6.2.8.3       \\
    
    \hline
        
    1545 & यत्   ।   अक्र॑न्दः   ।   प्र॒थ॒मम्   ।   जाय॑मानः   ।    & TS\_4.6.7.1       \\
    
    \hline
        
    1546 & यत्   ।   अ॒ग्नीषोमा॒वित्य॒ग्नी{-}सोमौ᳚   ।   अ॒न्त॒रा   ।   दे॒वताः᳚   ।    & TS\_2.6.2.2       \\
    
    \hline
        
    1547 & यत्   ।   अ॒ग्ने॒   ।   यानि॑   ।   कानि॑   ।    & TS\_4.1.10.1       \\
    
    \hline
        
    1548 & यत्   ।   अ॒छ॒न्दो॒ममित्य॑छन्दः{-}मम्   ।   यत्   ।   छ॒न्दो॒मा इति॑ छन्दः{-}माः   ।    & TS\_7.3.8.2       \\
    
    \hline
        
    1549 & यत्   ।   आकू॑ता॒दित्या{-}कू॒ता॒त्   ।   स॒मसु॑स्रो॒दिति॑ सं{-}असु॑स्रोत्   ।   हृ॒दः   ।    & TS\_5.7.7.1       \\
    
    \hline
        
    1550 & यत्   ।   आज्ये॑न   ।   प्र॒या॒जा इति॑ प्र{-}या॒जाः   ।   इ॒ज्यन्ते᳚   ।    & TS\_2.6.10.4       \\
    
    \hline
        
    1551 & यत्   ।   उ॒भौ   ।   वि॒मुच्येति॑ वि{-}मुच्य॑   ।   आ॒ति॒थ्यम्   ।    & TS\_6.2.1.1       \\
    
    \hline
        
    1552 & यत्   ।   एके॑न   ।   सꣳ॒॒स्था॒पय॒तीति॑ सं{-}स्था॒पय॑ति   ।   य॒ज्ञ्स्य॑   ।    & TS\_5.5.1.1       \\
    
    \hline
        
    1553 & यत्   ।   एक॑स्मिन्न्   ।   यूपे᳚   ।   द्वे इति॑   ।    & TS\_6.6.4.3       \\
    
    \hline
        
    1554 & यत्   ।   ए॒न॒म्   ।   द्यौः   ।   अज॑नयत्   ।    & TS\_1.3.14.6       \\
    
    \hline
        
    1555 & यत्   ।   कू॒र्मम्   ।   उ॒प॒दधा॒तीत्यु॑प{-}दधा॑ति   ।   यथा᳚   ।    & TS\_5.2.8.5       \\
    
    \hline
        
    1556 & यत्   ।   क॒र्ण॒गृ॒ही॒तेति॑ कर्ण{-}गृ॒ही॒ता   ।   वार्त्र॒घ्नीति॒ वार्त्र॑{-}घ्नी॒   ।   स्या॒त्   ।    & TS\_6.1.7.6       \\
    
    \hline
        
    1557 & यत्   ।   क॒लया᳚   ।   ते॒   ।   श॒फेन॑   ।    & TS\_6.1.10.1       \\
    
    \hline
        
    1558 & यत्   ।   त्रै॒धा॒त॒वीय᳚म्   ।   सर्वे॑ण   ।   ए॒व   ।    & TS\_2.4.11.5       \\
    
    \hline
        
    1559 & यत्   ।   द्वा॒द॒शा॒ह इति॑ द्वादश{-}अ॒हः   ।   ताम्   ।   वीति॑   ।    & TS\_7.3.3.2       \\
    
    \hline
        
    1560 & यत्   ।   द्वे इति॑   ।   इ॒व॒   ।   ब्रू॒यात्   ।    & TS\_2.5.9.6       \\
    
    \hline
        
    1561 & यत्   ।   द्वे इति॑   ।   रे॒त॒स्सिचा॒विति॑ रेतः{-}सिचौ᳚   ।   उ॒प॒द॒द्ध्यादित्यु॑प{-}द॒ध्यात्   ।    & TS\_5.6.8.5       \\
    
    \hline
        
    1562 & यत्   ।   नव᳚म्   ।   ऐत्   ।   तत्   ।    & TS\_2.3.10.1       \\
    
    \hline
        
    1563 & यत्   ।   पृ॒ष॒दा॒ज्यमिति॑ पृषत्{-}आ॒ज्यम्   ।   प॒शोः   ।   खलु॑   ।    & TS\_6.3.10.2       \\
    
    \hline
        
    1564 & यत्   ।   प॒शुः   ।   मा॒युम्   ।   अकृ॑त   ।    & TS\_3.1.5.2       \\
    
    \hline
        
    1565 & यत्   ।   यूप᳚म्   ।   मि॒नोति॑   ।   सु॒व॒र्गस्येति॑ सुवः{-}गस्य॑   ।    & TS\_6.3.4.8       \\
    
    \hline
        
    1566 & यत्   ।   रा॒ष्ट्र॒भृद्भि॒रिति॑ राष्ट्र॒भृत्{-}भिः॒   ।   रा॒ष्ट्रम्   ।   एति॑   ।    & TS\_3.4.6.2       \\
    
    \hline
        
    1567 & यत्   ।   वा॒   ।   स्कन्दा᳚त्   ।   आज्य॑स्य   ।    & TS\_1.6.2.2       \\
    
    \hline
        
    1568 & यत्   ।   वा॒र॒व॒न्तीये॒नेति॑ वार{-}व॒न्तीये॑न   ।   उ॒प॒तिष्ठ॑त॒ इत्यु॑प{-}तिष्ठ॑ते   ।   वा॒रय॑ते   ।    & TS\_5.5.8.2       \\
    
    \hline
        
    1569 & यत्   ।   वै   ।   अनी॑शानः   ।   भा॒रम्   ।    & TS\_6.2.5.1       \\
    
    \hline
        
    1570 & यत्   ।   वै   ।   होता᳚   ।   अ॒द्ध्व॒र्युम्   ।    & TS\_3.2.9.1       \\
    
    \hline
        
    1571 & यत्   ।   स॒माव॑त्   ।   वि॒द्व   ।   क॒था   ।    & TS\_2.5.8.4       \\
    
    \hline
        
    1572 & यत्   ।   स॒मी॒चीन᳚म्   ।   प॒शु॒शी॒र्॒.षैरिति॑ पशु{-}शी॒र्॒.षैः   ।   उ॒प॒द॒द्ध्यादित्यु॑प{-}द॒द्ध्यात्   ।    & TS\_5.2.9.6       \\
    
    \hline
        
    1573 & यत्का॑मा॒ इति॒ यत्{-}का॒माः॒   ।   ते॒   ।   जु॒हु॒मः   ।   तत्   ।    & TS\_3.2.5.7       \\
    
    \hline
        
    1574 & यत्र॑   ।   ब्र॒ह्मा   ।   यत्र॑   ।   ए॒व   ।    & TS\_2.6.9.2       \\
    
    \hline
        
    1575 & यथा᳚   ।   अस॑ति   ।   सु॒गमिति॑ सु{-}गम्   ।   मे॒षाय॑   ।    & TS\_1.8.6.2       \\
    
    \hline
        
    1576 & यथा᳚   ।   अ॒हम्   ।   यु॒ष्मान्   ।   तप॑सा   ।    & TS\_7.1.5.2       \\
    
    \hline
        
    1577 & यथा᳚   ।   वै   ।   पु॒त्रः   ।   जा॒तः   ।    & TS\_5.7.5.1       \\
    
    \hline
        
    1578 & यथा᳚   ।   वै   ।   म॒नु॒ष्याः᳚   ।   ए॒वम्   ।    & TS\_7.4.2.1       \\
    
    \hline
        
    1579 & यथा᳚   ।   वै   ।   स॒मृ॒त॒सो॒मा इति॑ समृत{-}सो॒माः   ।   ए॒वम्   ।    & TS\_1.6.7.1       \\
    
    \hline
        
    1580 & यदि॑   ।   इत॑रम्   ।   उ॒भये॑न   ।   ए॒व   ।    & TS\_6.1.9.2       \\
    
    \hline
        
    1581 & यदि॑   ।   सोमौ᳚   ।   सꣳसु॑ता॒विति॒ सं{-}सु॒तौ॒   ।   स्याता᳚म्   ।    & TS\_7.5.5.1       \\
    
    \hline
        
    1582 & यम्   ।   अब॑द्ध्नीत   ।   स॒वि॒ता   ।   सु॒केत॒ इति॑ सु{-}केतः॑   ।    & TS\_3.5.6.2       \\
    
    \hline
        
    1583 & यम्   ।   काम᳚म्   ।   का॒मय॑ते   ।   तम्   ।    & TS\_7.1.1.2       \\
    
    \hline
        
    1584 & यम्   ।   का॒मये॑त   ।   अ॒प॒शुः   ।   स्या॒त्   ।    & TS\_5.3.1.4       \\
    
    \hline
        
    1585 & यम्   ।   का॒मये॑त   ।   प॒शु॒मानिति॑ पशु{-}मान्   ।   स्या॒त्   ।    & TS\_5.2.6.4       \\
    
    \hline
        
    1586 & यम्   ।   नः॒   ।   स॒मा॒नः   ।   यम्   ।    & TS\_6.2.11.2       \\
    
    \hline
        
    1587 & यव᳚म्   ।   ग्री॒ष्माय॑   ।   ओष॑धीः   ।   व॒र्॒.षाभ्यः॑   ।    & TS\_7.2.10.2       \\
    
    \hline
        
    1588 & यस्मिन्न्॑   ।   अश्वः॑   ।   आ॒ल॒भ्यत॒ इत्या᳚{-}ल॒भ्यते᳚   ।   द्वाद॑श   ।    & TS\_5.4.12.2       \\
    
    \hline
        
    1589 & यस्य॑   ।   मुख्य॑वती॒रिति॒ मुख्य॑{-}व॒तीः॒   ।   पु॒रस्ता᳚त्   ।   उ॒प॒धी॒यन्त॒ इत्यु॑प{-}धी॒यन्ते᳚   ।    & TS\_5.3.4.6       \\
    
    \hline
        
    1590 & या   ।   तव॑   ।   त॒नूः   ।   इ॒यम्   ।    & TS\_1.2.11.2       \\
    
    \hline
        
    1591 & या   ।   वा॒म्   ।   इ॒न्द्रा॒व॒रु॒णेती᳚न्द्रा{-}व॒रु॒णा॒   ।   य॒त॒व्या᳚   ।    & TS\_2.3.13.1       \\
    
    \hline
        
    1592 & या   ।   वा॒म्   ।   कशा᳚   ।   मधु॑म॒तीति॒ मधु॑{-}म॒ती॒   ।    & TS\_1.4.6.1       \\
    
    \hline
        
    1593 & याः   ।   जा॒ताः   ।   ओष॑धयः   ।   दे॒वेभ्यः॑   ।    & TS\_4.2.6.1       \\
    
    \hline
        
    1594 & याः   ।   तृ॒तीयाः᳚   ।   पात्रा॑णि   ।   ताभिः॑   ।    & TS\_7.2.10.4       \\
    
    \hline
        
    1595 & याः   ।   ते॒   ।   अ॒ग्ने॒   ।   स॒मिध॒ इति॑ सं{-}इधः॑   ।    & TS\_5.7.8.1       \\
    
    \hline
        
    1596 & यानि॑   ।   ए॒व   ।   ए॒न॒म्   ।   भू॒तानि॑   ।    & TS\_1.7.5.4       \\
    
    \hline
        
    1597 & याम्   ।   वै   ।   अ॒द्ध्व॒र्युः   ।   च॒   ।    & TS\_3.5.9.1       \\
    
    \hline
        
    1598 & याव॑ती   ।   वै   ।   पृ॒थि॒वी   ।   तस्यै᳚   ।    & TS\_5.2.3.1       \\
    
    \hline
        
    1599 & याव॑न्तः   ।   वै   ।   दे॒वाः   ।   य॒ज्ञाय॑   ।    & TS\_6.1.2.1       \\
    
    \hline
        
    1600 & यासि॑   ।   दा॒श्वाꣳस᳚म्   ।   अच्छ॑   ।   नि॒युद्भि॒रिति॑ नि॒युत्{-}भिः॒   ।    & TS\_2.2.12.8       \\
    
    \hline
        
    1601 & या॒ज्या᳚   ।   ज॒नि॒ष्यमा॑णान्   ।   ए॒व   ।   प्रतीति॑   ।    & TS\_2.6.2.4       \\
    
    \hline
        
    1602 & या॒हि॒   ।   इति॑   ।   आ॒ह॒   ।   ज्योतिः॑   ।    & TS\_5.2.2.3       \\
    
    \hline
        
    1603 & यु॒क्ष्व   ।   हि   ।   दे॒व॒हूत॑मा॒निति॑ देव{-}हूत॑मान्   ।   अश्वान्॑   ।    & TS\_2.6.11.1       \\
    
    \hline
        
    1604 & यु॒क्ष्व   ।   हि   ।   ये   ।   तव॑   ।    & TS\_5.5.3.2       \\
    
    \hline
        
    1605 & यु॒ञ्जते᳚   ।   मनः॑   ।   उ॒त   ।   यु॒ञ्ज॒ते॒   ।    & TS\_1.2.13.1       \\
    
    \hline
        
    1606 & यु॒ञ्जा॒नः   ।   प्र॒थ॒मम्   ।   मनः॑   ।   त॒त्वाय॑   ।    & TS\_4.1.1.1       \\
    
    \hline
        
    1607 & यु॒व॒ते॒   ।   वीति॑   ।   पा॒प्मना᳚   ।   भ्रातृ॑व्येण   ।    & TS\_2.2.3.2       \\
    
    \hline
        
    1608 & यु॒ष्मानी॑तः   ।   अभ॑यम्   ।   ज्योतिः॑   ।   अ॒श्या॒म्   ।    & TS\_2.1.11.6       \\
    
    \hline
        
    1609 & यूप॑स्य   ।   यत्   ।   ऊ॒द्‌र्ध्वम्   ।   च॒षाला᳚त्   ।    & TS\_6.3.4.9       \\
    
    \hline
        
    1610 & ये   ।   अ॒ग्नयः॑   ।   सम॑नस॒ इति॒ स{-}म॒न॒सः॒   ।   अ॒न्त॒रा   ।    & TS\_4.4.11.2       \\
    
    \hline
        
    1611 & ये   ।   अ॒न्त॒श्श॒रा इत्य॑न्तः{-}श॒राः   ।   अशी᳚र्यन्त   ।   ते   ।    & TS\_6.1.3.5       \\
    
    \hline
        
    1612 & ये   ।   ई॒युः   ।   अ॒वृ॒काः   ।   ऋ॒त॒ज्ञा इत्यृ॑त{-}ज्ञाः   ।    & TS\_2.6.12.4       \\
    
    \hline
        
    1613 & ये   ।   उ॒द्यन्त॒ इत्यु॑त्{-}यन्तः॑   ।   स्तोमाः᳚   ।   श्रीः   ।    & TS\_7.1.8.2       \\
    
    \hline
        
    1614 & ये   ।   ते॒   ।   स॒र॒स्वः॒   ।   ऊ॒र्मयः॑   ।    & TS\_3.1.11.3       \\
    
    \hline
        
    1615 & ये   ।   त॒   ।   पन्था॑नः   ।   स॒वि॒तः॒   ।    & TS\_7.5.24.1       \\
    
    \hline
        
    1616 & ये   ।   दे॒वाः   ।   य॒ज्ञ्॒हन॒ इति॑ यज्ञ्{-}हनः॑   ।   य॒ज्ञ्॒मुष॒ इति॑ यज्ञ्{-}मुषः॑   ।    & TS\_3.5.4.1       \\
    
    \hline
        
    1617 & ये   ।   दे॒वाः॒   ।   दि॒वि   ।   एका॑दश   ।    & TS\_1.4.10.1       \\
    
    \hline
        
    1618 & ये   ।   पृ॒त॒न्यवः॑   ।   अ॒यम्   ।   अ॒ग्निः   ।    & TS\_4.7.13.4       \\
    
    \hline
        
    1619 & ये   ।   वा॒जिन᳚म्   ।   प॒रि॒पश्य॒न्तीति॑ परि{-}पश्य॑न्ति   ।   प॒क्वम्   ।    & TS\_4.6.9.1       \\
    
    \hline
        
    1620 & ये   ।   विश्वे᳚   ।   म॒रुतः॑   ।   जु॒नन्ति॑   ।    & TS\_3.1.11.8       \\
    
    \hline
        
    1621 & योक्त्र᳚म्   ।   गृद्ध्रा॑भिः   ।   यु॒गम्   ।   आन॑ते॒नेत्या{-}न॒ते॒न॒   ।    & TS\_5.7.14.1       \\
    
    \hline
        
    1622 & योगे॑योग॒ इति॒ योगे᳚{-}यो॒गे॒   ।   ए॒व   ।   ए॒न॒म्   ।   यु॒ङ्क्ते॒   ।    & TS\_5.1.2.2       \\
    
    \hline
        
    1623 & य॒च्छ॒ति॒   ।   प्रेति॑   ।   ह॒र॒ति॒   ।   द्वि॒तीये॑न   ।    & TS\_6.5.5.2       \\
    
    \hline
        
    1624 & य॒जे॒त॒   ।   यत्   ।   पूर्व॑या   ।   सं॒प्र॒तीति॑ सं{-}प्र॒ति   ।    & TS\_2.5.5.3       \\
    
    \hline
        
    1625 & य॒ज्ञाय॑   ।   च॒   ।   ए॒व   ।   यज॑मानाय   ।    & TS\_2.6.7.5       \\
    
    \hline
        
    1626 & य॒ज्ञिय᳚म्   ।   तेन॑   ।   त्वा॒   ।   एति॑   ।    & TS\_1.2.12.2       \\
    
    \hline
        
    1627 & य॒ज्ञिय᳚म्   ।   न   ।   अ॒वि॒न्द॒न्न्   ।   ते   ।    & TS\_2.5.6.4       \\
    
    \hline
        
    1628 & य॒ज्ञे   ।   ए॒व   ।   अ॒न्तरि॑क्षे   ।   प्रतीति॑   ।    & TS\_2.6.1.3       \\
    
    \hline
        
    1629 & य॒ज्ञे   ।   दक्षि॑णाम्   ।   ददा॑ति   ।   ताम्   ।    & TS\_1.7.1.6       \\
    
    \hline
        
    1630 & य॒ज्ञेन॑   ।   क॒ल्प॒ता॒म्   ।   व्या॒न इति॑ वि{-}अ॒नः   ।   य॒ज्ञेन॑   ।    & TS\_1.7.9.2       \\
    
    \hline
        
    1631 & य॒ज्ञेन॑   ।   वै   ।   दे॒वाः   ।   सु॒व॒र्गमिति॑ सुवः{-}गम्   ।    & TS\_6.5.3.1       \\
    
    \hline
        
    1632 & य॒ज्ञेन॑   ।   वै   ।   प्र॒जाप॑ति॒रिति॑ प्र॒जा{-}प॒तिः॒   ।   प्र॒जा इति॑ प्र{-}जाः   ।    & TS\_6.4.1.1       \\
    
    \hline
        
    1633 & य॒ज्ञेषु॑   ।   ईड्‍यः॑   ।   विश्वे᳚   ।   दे॒वाः   ।    & TS\_1.2.3.2       \\
    
    \hline
        
    1634 & य॒ज्ञ्ः   ।   दक्षि॑णाम्   ।   अ॒भीति॑   ।   अ॒द्ध्या॒य॒त्   ।    & TS\_6.1.3.6       \\
    
    \hline
        
    1635 & य॒ज्ञ्ः   ।   दे॒वेभ्यः॑   ।   निला॑यत   ।   विष्णुः॑   ।    & TS\_6.2.4.2       \\
    
    \hline
        
    1636 & य॒ज्ञ्ः   ।   वै   ।   विष्णुः॑   ।   य॒ज्ञाय॑   ।    & TS\_6.1.7.3       \\
    
    \hline
        
    1637 & य॒ज्ञ्प॑ति॒मिति॑ य॒ज्ञ्{-}प॒ति॒म्   ।   दु॒हे   ।   य॒ज्ञ्प॑ति॒रिति॑ य॒ज्ञ्{-}प॒तिः॒   ।   वा॒   ।    & TS\_3.2.7.3       \\
    
    \hline
        
    1638 & य॒ज्ञ्प॑ति॒मिति॑ य॒ज्ञ्{-}प॒ति॒म्   ।   भगा॑य   ।   दि॒व्यः   ।   ग॒न्ध॒र्वः   ।    & TS\_4.1.1.3       \\
    
    \hline
        
    1639 & य॒ज्ञ्म्   ।   आ॒प॒त्   ।   यत्   ।   छन्दाꣳ॑सि   ।    & TS\_7.2.8.2       \\
    
    \hline
        
    1640 & य॒ज्ञ्म्   ।   प्रतीति॑   ।   अ॒ति॒ष्ठि॒पा3ः   ।   य॒ज्ञ्प॒ता3विति॑ य॒ज्ञ्{-}प॒ता3उ   ।    & TS\_6.6.2.3       \\
    
    \hline
        
    1641 & य॒ज्ञ्म्   ।   वाव   ।   सः   ।   तत्   ।    & TS\_3.2.4.2       \\
    
    \hline
        
    1642 & य॒ज्ञ्म्   ।   वीति॑   ।   छि॒न्द्या॒त्   ।   ज्यो॒ति॒ष्या᳚   ।    & TS\_6.4.2.2       \\
    
    \hline
        
    1643 & य॒ज्ञ्म्   ।   वै   ।   ए॒तत्   ।   समिति॑   ।    & TS\_3.1.3.1       \\
    
    \hline
        
    1644 & य॒ज्ञ्म्   ।   शृ॒णोतु॑   ।   दे॒वः   ।   स॒वि॒ता   ।    & TS\_1.3.13.2       \\
    
    \hline
        
    1645 & य॒ज्ञ्स्य॑   ।   आ॒शीरित्या᳚{-}शीः   ।   ग॒च्छ॒ति॒   ।   यत्   ।    & TS\_1.6.10.4       \\
    
    \hline
        
    1646 & य॒ज्ञ्स्य॑   ।   घो॒षत्   ।   अ॒सि॒   ।   प्रत्यु॑ष्ट॒मिति॒ प्रति॑{-}उ॒ष्ट॒म्   ।    & TS\_1.1.2.1       \\
    
    \hline
        
    1647 & य॒ज्ञ्स्य॑   ।   शिरः॑   ।   अ॒च्छि॒द्य॒त॒   ।   ते   ।    & TS\_6.4.9.1       \\
    
    \hline
        
    1648 & य॒ज्ञ्स्य॑   ।   समि॑ष्ट्या॒ इति॒ सं{-}इ॒ष्ट्यै॒   ।   प्रा॒णा॒पा॒नाविति॑ प्राण{-}अ॒पा॒नौ   ।   वै   ।    & TS\_6.3.9.6       \\
    
    \hline
        
    1649 & य॒ज्ञ्स्य॑   ।   स्वि॑ष्ट॒मिति॒ सु{-}इ॒ष्ट॒म्   ।   श॒म॒य॒ति॒   ।   वरु॑णेन   ।    & TS\_6.6.7.4       \\
    
    \hline
        
    1650 & य॒ज्ञ्॒मु॒खमिति॑ यज्ञ्{-}मु॒खम्   ।   ए॒व   ।   पु॒रस्ता᳚त्   ।   वीति॑   ।    & TS\_5.3.3.2       \\
    
    \hline
        
    1651 & य॒ज्ञ्॒मु॒खमिति॑ यज्ञ्{-}मु॒खम्   ।   च॒तु॒ष्टो॒म इति॑ चतुः{-}स्तो॒मः   ।   य॒ज्ञ्॒मु॒खमिति॑ यज्ञ्{-}मु॒खम्   ।   ए॒व   ।    & TS\_5.3.4.5       \\
    
    \hline
        
    1652 & य॒ज्ञ्॒मु॒खेनेति॑ यज्ञ्{-}मु॒खेन॑   ।   संमि॑ता॒मिति॒ सं{-}मि॒ता॒म्   ।   एति॑   ।   प्र॒ति॒ष्ठाया॒ इति॑ प्रति{-}स्थायै᳚   ।    & TS\_2.6.4.3       \\
    
    \hline
        
    1653 & य॒ज॒ति॒   ।   प्र॒जा इति॑ प्र{-}जाः   ।   वै   ।   ब॒र्॒.हिः   ।    & TS\_6.6.3.3       \\
    
    \hline
        
    1654 & य॒तः   ।   ई॒श्व॒रः   ।   वै   ।   अश्वः॑   ।    & TS\_5.4.12.3       \\
    
    \hline
        
    1655 & य॒था॒य॒जुरिति॑ यथा{-}य॒जुः   ।   ए॒व   ।   ए॒तत्   ।   श॒तम्   ।    & TS\_1.7.6.5       \\
    
    \hline
        
    1656 & य॒न्ति॒   ।   अथो॒ इति॑   ।   अ॒नयोः᳚   ।   ए॒व   ।    & TS\_7.4.1.3       \\
    
    \hline
        
    1657 & य॒न्ति॒   ।   परा᳚ञ्चः   ।   वै   ।   ए॒ते   ।    & TS\_7.3.7.4       \\
    
    \hline
        
    1658 & य॒म॒त्वमिति॑ यम{-}त्वम्   ।   ते   ।   दे॒वाः   ।   अ॒म॒न्य॒न्त॒   ।    & TS\_2.1.4.4       \\
    
    \hline
        
    1659 & य॒वय॑   ।   अरा॑तीः   ।   पि॒तृ॒णाम्   ।   सद॑नम्   ।    & TS\_1.3.1.2       \\
    
    \hline
        
    1660 & य॒व॒त्वमिति॑ यव{-}त्वम्   ।   यत्   ।   वैक॑ङ्कतम्   ।   म॒न्थि॒पा॒त्रमिति॑ मन्थि{-}पा॒त्रम्   ।    & TS\_6.4.10.6       \\
    
    \hline
        
    1661 & रक्ष॑साम्   ।   अन॑न्ववचारा॒येत्यन॑नु{-}अ॒व॒चा॒रा॒य॒   ।   न   ।   पु॒रस्ता᳚त्   ।    & TS\_2.6.6.3       \\
    
    \hline
        
    1662 & रज॑नः   ।   वै   ।   कौ॒णे॒यः   ।   क्र॒तु॒जित॒मिति॑ क्रतु{-}जित᳚म्   ।    & TS\_2.3.8.1       \\
    
    \hline
        
    1663 & रज॑सि   ।   त॒स्थि॒वाꣳस᳚म्   ।   ऋ॒तस्य॑   ।   योनौ᳚   ।    & TS\_4.2.2.2       \\
    
    \hline
        
    1664 & रथे॑न   ।   र॒क्षो॒हेति॑ रक्षः{-}हा   ।   अ॒मित्रान्॑   ।   अ॒प॒बाध॑मान॒ इत्य॑प{-}बाध॑मानः   ।    & TS\_4.6.4.2       \\
    
    \hline
        
    1665 & रथ॑पतिभ्य॒ इति॒ रथ॑पति{-}भ्यः॒   ।   च॒   ।   वः॒   ।   नमः॑   ।    & TS\_4.5.4.2       \\
    
    \hline
        
    1666 & ररा॑णः   ।   वी॒हि   ।   मृ॒डी॒कम्   ।   सु॒हव॒ इति॑ सु{-}हवः॑   ।    & TS\_2.5.12.4       \\
    
    \hline
        
    1667 & रस॑वा॒निति॒ रस॑{-}वा॒न्   ।   ए॒व   ।   भ॒व॒ति॒   ।   अ॒ज॒क्षी॒र इत्य॑ज{-}क्षी॒रे   ।    & TS\_2.2.4.5       \\
    
    \hline
        
    1668 & रस᳚म्   ।   वृष्टि᳚म्   ।   अवेति॑   ।   रु॒न्धे॒   ।    & TS\_2.1.7.4       \\
    
    \hline
        
    1669 & राज्ञी᳚   ।   अ॒सि॒   ।   प्राची᳚   ।   दिक्   ।    & TS\_4.4.2.1       \\
    
    \hline
        
    1670 & रात्रि᳚म्   ।   प्रेति॑   ।   अ॒वि॒श॒न्न्   ।   ते   ।    & TS\_1.5.9.3       \\
    
    \hline
        
    1671 & रायः॑   ।   सु॒वीर॒ इति॑ सु{-}वीरः॑   ।   इति॑   ।   आ॒ह॒   ।    & TS\_6.3.9.4       \\
    
    \hline
        
    1672 & रा॒काम्   ।   अ॒हम्   ।   सु॒हवा॒मिति॑ सु{-}हवा᳚म्   ।   सु॒ष्टु॒तीति॑ सु{-}स्तु॒ती   ।    & TS\_3.3.11.5       \\
    
    \hline
        
    1673 & रा॒ज्याय॑   ।   सन्त᳚म्   ।   रा॒ज्यम्   ।   न   ।    & TS\_2.1.3.4       \\
    
    \hline
        
    1674 & रा॒ज॒न्न्   ।   दि॒वः   ।   आ॒चर॒न्तीत्या᳚{-}चर॑न्ति   ।   तेभिः॑   ।    & TS\_2.3.14.5       \\
    
    \hline
        
    1675 & रा॒ज॒न्यः॑   ।   वज्र॑स्य   ।   रू॒पम्   ।   समृ॑द्ध्या॒ इति॒ सं{-}ऋ॒द्ध्यै॒   ।    & TS\_6.2.5.3       \\
    
    \hline
        
    1676 & रा॒ष्ट्रका॑मा॒येति॑ रा॒ष्ट्र{-}का॒मा॒य॒   ।   हो॒त॒व्याः᳚   ।   रा॒ष्ट्रम्   ।   वै   ।    & TS\_3.4.8.1       \\
    
    \hline
        
    1677 & रि॒षः   ।   पा॒तु॒   ।   नक्त᳚म्   ।   जा॒तः   ।    & TS\_1.5.11.2       \\
    
    \hline
        
    1678 & रुक्का॑म॒ इति॒ रुक्{-}का॒मः॒   ।   छन्दाꣳ॑सि   ।   वै   ।   देवि॑काः   ।    & TS\_3.4.9.6       \\
    
    \hline
        
    1679 & रुच᳚म्   ।   अ॒द॒धुः॒   ।   यः   ।   ब्र॒ह्म॒व॒र्च॒सका॑म॒ इति॑ ब्रह्मवर्च॒स{-}का॒मः॒   ।    & TS\_2.3.2.3       \\
    
    \hline
        
    1680 & रुच᳚म्   ।   ए॒व   ।   अ॒स्मि॒न्न्   ।   द॒धा॒ति॒   ।    & TS\_2.1.2.9       \\
    
    \hline
        
    1681 & रुरुः॑   ।   रौ॒द्रः   ।   कृ॒क॒ला॒सः   ।   श॒कुनिः॑   ।    & TS\_5.5.19.1       \\
    
    \hline
        
    1682 & रुश॑त्   ।   दृ॒शे   ।   द॒दृ॒शे॒   ।   न॒क्त॒या   ।    & TS\_4.3.13.2       \\
    
    \hline
        
    1683 & रु॒द्रः   ।   इति॑   ।   अ॒ब्रु॒व॒न्न्   ।   रु॒द्रः   ।    & TS\_6.2.3.2       \\
    
    \hline
        
    1684 & रु॒द्रः   ।   खलु॑   ।   वै   ।   वा॒स्तो॒ष्प॒तिरिति॑ वास्तोः{-}प॒तिः   ।    & TS\_3.4.10.3       \\
    
    \hline
        
    1685 & रु॒द्रः   ।   वै   ।   ए॒षः   ।   यत्   ।    & TS\_5.4.3.1       \\
    
    \hline
        
    1686 & रु॒द्रम्   ।   अ॒न्तः   ।   आ॒य॒न्न्   ।   सः   ।    & TS\_6.5.6.3       \\
    
    \hline
        
    1687 & रु॒द्राः   ।   वस॑वः   ।   समिति॑   ।   इ॒न्ध॒ता॒म्   ।    & TS\_5.2.2.6       \\
    
    \hline
        
    1688 & रु॒द्राणा᳚म्   ।   आधि॑पत्य॒मित्याधि॑{-}प॒त्य॒म्   ।   चतु॑ष्पा॒दिति॒ चतुः॑{-}पा॒त्   ।   स्पृ॒तम्   ।    & TS\_4.3.9.2       \\
    
    \hline
        
    1689 & रु॒द्र॒   ।   री॒रि॒षः॒   ।   मा   ।   नः॒   ।    & TS\_4.5.10.3       \\
    
    \hline
        
    1690 & रु॒न्धे॒   ।   द॒द्ध्नः   ।   पू॒र्णाम्   ।   औदु॑बंरीम्   ।    & TS\_5.4.7.3       \\
    
    \hline
        
    1691 & रु॒न्धे॒   ।   प्र॒वेति॑ प्र{-}वा   ।   अ॒सि॒   ।   अ॒नु॒वेत्य॑नु{-}वा   ।    & TS\_3.5.2.3       \\
    
    \hline
        
    1692 & रु॒न्धे॒   ।   वै॒श्व॒दे॒वीमिति॑ वैश्व{-}दे॒वीम्   ।   ब॒हु॒रू॒पामिति॑ बहु{-}रू॒पाम्   ।   एति॑   ।    & TS\_2.1.7.5       \\
    
    \hline
        
    1693 & रु॒न्धे॒   ।   शृ॒ण्वन्ति॑   ।   ए॒न॒म्   ।   अ॒ग्निम्   ।    & TS\_5.6.10.2       \\
    
    \hline
        
    1694 & रु॒न्ध॒ते॒   ।   स॒प्त॒द॒श इति॑ सप्त{-}द॒शः   ।   भ॒व॒ति॒   ।   अ॒न्नाद्य॒स्येत्य॑न्न{-} अद्य॑स्य   ।    & TS\_7.4.3.4       \\
    
    \hline
        
    1695 & रू॒पम्   ।   यत्   ।   इष्ट॑काः   ।   रात्रि॑यै   ।    & TS\_5.7.1.3       \\
    
    \hline
        
    1696 & रू॒पेण॑   ।   ए॒व   ।   अवेति॑   ।   रु॒न्धे॒   ।    & TS\_2.1.1.6       \\
    
    \hline
        
    1697 & रेज॑ते   ।   अ॒ग्ने॒   ।   पृ॒थि॒वी   ।   म॒खेभ्यः॑   ।    & TS\_4.1.11.4       \\
    
    \hline
        
    1698 & रेतः॑   ।   धि॒षी॒य॒   ।   इति॑   ।   आ॒ह॒   ।    & TS\_1.7.4.5       \\
    
    \hline
        
    1699 & रोद॑सी॒ इति॑   ।   अ॒पृ॒णा॒त्   ।   जाय॑मानः   ।   वी॒डुम्   ।    & TS\_4.2.2.3       \\
    
    \hline
        
    1700 & रोद॑स्योः   ।   इति॑   ।   आ॒ह॒   ।   इ॒मे इति॑   ।    & TS\_5.1.5.4       \\
    
    \hline
        
    1701 & रोहि॑तः   ।   धू॒म्ररो॑हित॒ इति॑ धू॒म्र{-}रो॒हि॒तः॒   ।   क॒र्कन्धु॑रोहित॒ इति॑ क॒र्कन्धु॑{-}रो॒हि॒तः॒   ।   ते   ।    & TS\_5.6.11.1       \\
    
    \hline
        
    1702 & रोहि॑ताय   ।   स्थ॒पत॑ये   ।   वृ॒क्षाणा᳚म्   ।   पत॑ये   ।    & TS\_4.5.2.2       \\
    
    \hline
        
    1703 & र॒क्षसः॑   ।   द॒ह॒   ।   प्रति॑   ।   एति॑   ।    & TS\_4.4.4.6       \\
    
    \hline
        
    1704 & र॒क्षो॒हण॒ इति॑ रक्षः{-}हनः॑   ।   व॒ल॒ग॒हन॒ इति॑ वलग{-}हनः॑   ।   प्रेति॑   ।   उ॒क्षा॒मि॒   ।    & TS\_1.3.2.2       \\
    
    \hline
        
    1705 & र॒क्षो॒हण॒ इति॑ रक्षः{-}हनः॑   ।   व॒ल॒ग॒हन॒ इति॑ वलग{-}हनः॑   ।   वै॒ष्ण॒वान्   ।   ख॒ना॒मि॒   ।    & TS\_1.3.2.1       \\
    
    \hline
        
    1706 & र॒क्ष॒ता॒त्   ।   इ॒मम्   ।   तस्मै᳚   ।   ते॒   ।    & TS\_1.3.14.5       \\
    
    \hline
        
    1707 & र॒ज॒तम्   ।   हिर॑ण्यम्   ।   अ॒भ॒व॒त्   ।   तस्मा᳚त्   ।    & TS\_1.5.1.2       \\
    
    \hline
        
    1708 & र॒यिम्   ।   चितः॑   ।   स्थ॒   ।   प॒रि॒चित॒ इति॑ परि{-}चितः॑   ।    & TS\_4.2.7.4       \\
    
    \hline
        
    1709 & र॒श्मिः   ।   अ॒सि॒   ।   क्षया॑य   ।   त्वा॒   ।    & TS\_4.4.1.1       \\
    
    \hline
        
    1710 & र॒श्मिः   ।   इति॑   ।   ए॒व   ।   आ॒दि॒त्यम्   ।    & TS\_5.3.6.1       \\
    
    \hline
        
    1711 & र॒श॒नाम्   ।   एति॑   ।   द॒त्ते॒   ।   प्रसू᳚त्या॒ इति॒ प्र{-}सू॒त्यै॒   ।    & TS\_6.3.6.3       \\
    
    \hline
        
    1712 & लो॒कः   ।   सर॑स्वत्या   ।   या॒न्ति॒   ।   ए॒षः   ।    & TS\_7.2.1.4       \\
    
    \hline
        
    1713 & लो॒कम्   ।   अ॒जि॒गाꣳ॒॒स॒न्न्   ।   ते   ।   सु॒व॒र्गमिति॑ सुवः{-}गम्   ।    & TS\_6.5.8.2       \\
    
    \hline
        
    1714 & लो॒कम्   ।   अ॒भ्यारो॑ह॒न्तीत्य॑भि{-}आरो॑हन्ति   ।   यत्   ।   अ॒न्यतः॑   ।    & TS\_7.3.9.3       \\
    
    \hline
        
    1715 & लो॒कम्   ।   ए॒ति॒   ।   गच्छ॑ति   ।   प्र॒का॒शमिति॑ प्र{-}का॒शम्   ।    & TS\_5.3.9.2       \\
    
    \hline
        
    1716 & लो॒काना᳚म्   ।   विधृ॑त्या॒ इति॒ वि{-}धृ॒त्यै॒   ।   अ॒ग्नेः   ।   त्रयः॑   ।    & TS\_6.2.8.4       \\
    
    \hline
        
    1717 & लो॒के   ।   प॒शवः॑   ।   ये   ।   ए॒वम्   ।    & TS\_7.3.5.2       \\
    
    \hline
        
    1718 & लो॒के   ।   प॒शु॒मानिति॑ पशु{-}मान्   ।   स्या॒त्   ।   उ॒भयोः᳚   ।    & TS\_6.4.2.6       \\
    
    \hline
        
    1719 & लो॒केभ्यः॑   ।   समिति॑   ।   भ॒र॒ति॒   ।   सोमः॑   ।    & TS\_6.4.4.3       \\
    
    \hline
        
    1720 & लो॒केषु॑   ।   अधि॑   ।   प्रेति॑   ।   जा॒य॒ते॒   ।    & TS\_5.5.5.4       \\
    
    \hline
        
    1721 & ल॒भे॒त॒   ।   यः   ।   पा॒प्मना᳚   ।   गृ॒ही॒तः   ।    & TS\_2.1.3.5       \\
    
    \hline
        
    1722 & वः॒   ।   रू॒पम्   ।   अ॒भि   ।   एति॑   ।    & TS\_1.4.43.2       \\
    
    \hline
        
    1723 & वज्रः॑   ।   अ॒सि॒   ।   वार्त्र॑घ्न॒ इति॒ वार्त्र॑{-}घ्नः॒   ।   त्वया᳚   ।    & TS\_1.8.12.3       \\
    
    \hline
        
    1724 & वज्रः॑   ।   व॒ष॒ट्का॒र इति॑ वषट्{-}का॒रः   ।   त्रि॒वृत॒मिति॑ त्रि{-}वृत᳚म्   ।   ए॒व   ।    & TS\_2.6.2.5       \\
    
    \hline
        
    1725 & वज्र᳚म्   ।   ए॒व   ।   भ्रातृ॑व्येभ्यः   ।   प्रेति॑   ।    & TS\_7.4.7.2       \\
    
    \hline
        
    1726 & वद्ध्य᳚म्   ।   प्रप॑न्न॒मिति॒ प्र{-}प॒न्न॒म्   ।   न   ।   प्रति॑   ।    & TS\_6.5.8.5       \\
    
    \hline
        
    1727 & वद॑ति   ।   दक्षि॑णा   ।   अ॒सि॒   ।   इति॑   ।    & TS\_6.1.7.5       \\
    
    \hline
        
    1728 & वन॒स्पति॑भ्य॒ इति॒ वन॒स्पति॑{-}भ्यः॒   ।   स्वाहा᳚   ।   मूले᳚भ्यः   ।   स्वाहा᳚   ।    & TS\_7.3.20.1       \\
    
    \hline
        
    1729 & वन॒स्पतीन्॑   ।   प्र॒जामिति॑ प्र{-}जाम्   ।   प॒शून्   ।   तेन॑   ।    & TS\_7.4.3.2       \\
    
    \hline
        
    1730 & वन॒स्पती॑नाम्   ।   प॒श॒व्यः॑   ।   प॒शु॒मानिति॑ पशु{-}मान्   ।   ए॒व   ।    & TS\_6.3.3.5       \\
    
    \hline
        
    1731 & वयः॑   ।   वै   ।   अ॒ग्निः   ।   यत्   ।    & TS\_5.7.6.1       \\
    
    \hline
        
    1732 & वरः॑   ।   देयः॑   ।   सा   ।   हि   ।    & TS\_7.1.6.5       \\
    
    \hline
        
    1733 & वरु॑णः   ।   अ॒द्भिरित्य॑त्{-}भिः   ।   साम्ने᳚   ।   समिति॑   ।    & TS\_7.5.23.2       \\
    
    \hline
        
    1734 & वरु॑णः   ।   दशा᳚क्षरे॒णेति॒ दश॑{-}अ॒क्ष॒रे॒ण॒   ।   वि॒राज॒मिति॑ वि{-}राज᳚म्   ।   उदिति॑   ।    & TS\_1.7.11.2       \\
    
    \hline
        
    1735 & वरु॑णम्   ।   सु॒षु॒वा॒णम्   ।   अ॒न्नाद्य॒मित्य॑न्न{-}अद्य᳚म्   ।   न   ।    & TS\_2.1.9.1       \\
    
    \hline
        
    1736 & वरु॑णेन   ।   अ॒स्य॒   ।   य॒ज्ञ्म्   ।   ग्रा॒ह॒ये॒त्   ।    & TS\_6.4.2.4       \\
    
    \hline
        
    1737 & वर्चः॑   ।   इ॒दम्   ।   क्ष॒त्रम्   ।   स॒लि॒लवा॑त॒मिति॑ सलि॒ल{-}वा॒त॒म्   ।    & TS\_4.4.12.3       \\
    
    \hline
        
    1738 & वर्च॑से   ।   प॒व॒स्व॒   ।   तस्य॑   ।   मे॒   ।    & TS\_3.2.3.3       \\
    
    \hline
        
    1739 & वर्द्ध॑न्तु   ।   त्वा॒   ।   सु॒ष्टु॒तय॒ इति॑ सु{-}स्तु॒तयः॑   ।   गिरः॑   ।    & TS\_2.2.12.5       \\
    
    \hline
        
    1740 & वर्म॑   ।   च॒   ।   स्थः॒   ।   अच्छि॑द्रे॒ इति॑   ।    & TS\_4.1.3.2       \\
    
    \hline
        
    1741 & वर᳚म्   ।   वृ॒णै॒   ।   मयि॑   ।   ए॒व   ।    & TS\_2.5.2.7       \\
    
    \hline
        
    1742 & वर᳚म्   ।   वेद॑   ।   एति॑   ।   ए॒न॒म्   ।    & TS\_2.5.2.6       \\
    
    \hline
        
    1743 & वसि॑ष्ठः   ।   ह॒तपु॑त्र॒ इति॑ ह॒त{-}पु॒त्रः॒   ।   अ॒का॒म॒य॒त॒   ।   वि॒न्देय॑   ।    & TS\_7.4.7.1       \\
    
    \hline
        
    1744 & वसु॑भ्य॒ इति॒ वसु॑{-}भ्यः॒   ।   त्वा॒   ।   वसून्॑   ।   जि॒न्व॒   ।    & TS\_4.4.1.2       \\
    
    \hline
        
    1745 & वसू॑नाम्   ।   दे॒वम्   ।   राधः॑   ।   जना॑नाम्   ।    & TS\_4.4.4.5       \\
    
    \hline
        
    1746 & वसोः᳚   ।   धारा᳚म्   ।   जु॒हो॒ति॒   ।   वसोः᳚   ।    & TS\_5.4.8.1       \\
    
    \hline
        
    1747 & वसो॒ इति॑   ।   पु॒रु॒स्पृह॒मिति॑ पुरु{-}स्पृह᳚म्   ।   र॒यिम्   ।   सः   ।    & TS\_1.3.14.4       \\
    
    \hline
        
    1748 & वस्वी᳚   ।   अ॒सि॒   ।   रु॒द्रा   ।   अ॒सि॒   ।    & TS\_1.2.5.1       \\
    
    \hline
        
    1749 & वस॑वः   ।   अ॒ष्टाक्ष॒रेत्य॒ष्टा{-}अ॒क्ष॒रा॒   ।   गा॒य॒त्री   ।   एका॑दश   ।    & TS\_3.4.9.7       \\
    
    \hline
        
    1750 & वस॑वः   ।   त्वा॒   ।   धू॒प॒य॒न्तु॒   ।   गा॒य॒त्रेण॑   ।    & TS\_4.1.6.1       \\
    
    \hline
        
    1751 & वस॑वः   ।   त्वा॒   ।   प्रेति॑   ।   वृ॒ह॒न्तु॒   ।    & TS\_3.3.3.1       \\
    
    \hline
        
    1752 & वाक्   ।   ते॒   ।   एति॑   ।   प्या॒य॒ता॒म्   ।    & TS\_1.3.9.1       \\
    
    \hline
        
    1753 & वाक्   ।   वै   ।   ए॒षा   ।   यत्   ।    & TS\_6.1.7.4 TS\_6.4.7.1       \\
    
    \hline
        
    1754 & वाक्   ।   वै   ।   दे॒वेभ्यः॑   ।   अपेति॑   ।    & TS\_6.1.4.1       \\
    
    \hline
        
    1755 & वाच᳚म्   ।   अवेति॑   ।   रु॒न्ध॒ते॒   ।   भू॒मि॒दु॒न्दु॒भिमिति॑ भूमि{-}दु॒न्दु॒भिम्   ।    & TS\_7.5.9.3       \\
    
    \hline
        
    1756 & वाच᳚म्   ।   द॒धा॒ति॒   ।   प्र॒व॒दि॒तेति॑ प्र{-}व॒दि॒ता   ।   वा॒चः   ।    & TS\_2.1.2.7       \\
    
    \hline
        
    1757 & वाच᳚म्   ।   सम्   ।   प्रेति॑   ।   य॒च्छे॒त्   ।    & TS\_6.3.1.6       \\
    
    \hline
        
    1758 & वाजः॑   ।   नः॒   ।   स॒प्त   ।   प्र॒दिश॒ इति॑ प्र{-}दिशः॑   ।    & TS\_4.7.12.1       \\
    
    \hline
        
    1759 & वाजाः᳚   ।   इति॑   ।   अनि॑रुक्ता॒मित्यनिः॑{-}उ॒क्ता॒म्   ।   प्रा॒जा॒प॒त्यामिति॑ प्राजा{-}प॒त्याम्   ।    & TS\_2.5.7.3       \\
    
    \hline
        
    1760 & वाज॑स्य   ।   इ॒मम्   ।   प्र॒स॒व इति॑ प्र{-}स॒वः   ।   सु॒षु॒वे॒   ।    & TS\_1.7.10.1       \\
    
    \hline
        
    1761 & वाज॑स्य   ।   मा॒   ।   प्र॒स॒वेनेति॑ प्र{-}स॒वेन॑   ।   उ॒द्ग्रा॒भेणेत्यु॑त् {-}ग्रा॒भेण॑   ।    & TS\_1.1.13.1       \\
    
    \hline
        
    1762 & वाज᳚म्   ।   अ॒ग्ने॒   ।   तव॑   ।   ऊ॒तिभि॒रित्यू॒ति{-}भिः॒   ।    & TS\_1.5.11.3       \\
    
    \hline
        
    1763 & वाज᳚म्   ।   हनू᳚भ्या॒मिति॒ हनु॑{-}भ्या॒म्   ।   अ॒पः   ।   आ॒स्ये॑न   ।    & TS\_5.7.12.1       \\
    
    \hline
        
    1764 & वातः॑   ।   अन्विति॑   ।   वा॒तु॒   ।   ते॒   ।    & TS\_5.5.7.4       \\
    
    \hline
        
    1765 & वातः॑   ।   दे॒वता᳚   ।   सूर्यः॑   ।   दे॒वता᳚   ।    & TS\_4.3.7.2       \\
    
    \hline
        
    1766 & वाति॑   ।   त॒द्रियङ्॑   ।   अ॒ग्निः   ।   द॒ह॒ति॒   ।    & TS\_5.5.1.2       \\
    
    \hline
        
    1767 & वाते॑न   ।   अ॒स्य   ।   ह॒विषः॑   ।   त्मना᳚   ।    & TS\_1.3.8.2       \\
    
    \hline
        
    1768 & वाव   ।   ए॒तत्   ।   आ॒ह॒   ।   हिर॑ण्यम्   ।    & TS\_6.6.1.5       \\
    
    \hline
        
    1769 & वाव   ।   यः   ।   पव॑ते   ।   सः   ।    & TS\_6.1.4.3       \\
    
    \hline
        
    1770 & वाव   ।   र॒थ॒न्त॒रमिति॑ रथम्{-}त॒रम्   ।   भ॒वि॒ष्यत्   ।   बृ॒हत्   ।    & TS\_3.1.7.3       \\
    
    \hline
        
    1771 & वासः॑   ।   सर्वा॑भिः   ।   ए॒व   ।   ए॒न॒म्   ।    & TS\_6.1.9.7       \\
    
    \hline
        
    1772 & वास्तोः᳚   ।   प॒ते॒   ।   प्रतीति॑   ।   जा॒नी॒हि॒   ।    & TS\_3.4.10.1       \\
    
    \hline
        
    1773 & वा॒   ।   तु॒तोद॑   ।   स्रु॒चा   ।   इ॒व॒   ।    & TS\_4.6.9.3       \\
    
    \hline
        
    1774 & वा॒   ।   प्र॒मीये॑र॒न्निति॑ प्र{-}मीये॑रन्न्   ।   यः   ।   वा॒   ।    & TS\_2.2.2.4       \\
    
    \hline
        
    1775 & वा॒चः   ।   पत॑ये   ।   प॒व॒स्व॒   ।   वा॒जि॒न्न्   ।    & TS\_1.4.2.1       \\
    
    \hline
        
    1776 & वा॒चा   ।   मा॒   ।   इ॒न्द्रि॒येण॑   ।   एति॑   ।    & TS\_1.6.2.3       \\
    
    \hline
        
    1777 & वा॒च॒य॒ति॒   ।   ब्रह्म॑णा   ।   ए॒व   ।   क्ष॒त्रम्   ।    & TS\_5.1.10.3       \\
    
    \hline
        
    1778 & वा॒मम्   ।   अ॒द्य   ।   स॒वि॒तः॒   ।   वा॒मम्   ।    & TS\_1.4.23.1       \\
    
    \hline
        
    1779 & वा॒म्   ।   दे॒वौ॒   ।   दे॒वेषु॑   ।   अनि॑शित॒मित्यनि॑{-}शि॒त॒म्   ।    & TS\_4.7.15.4       \\
    
    \hline
        
    1780 & वा॒यवे᳚   ।   नि॒युत्व॑त॒ इति॑ नि{-}युत्व॑ते   ।   एति॑   ।   ल॒भे॒त॒   ।    & TS\_2.1.1.2       \\
    
    \hline
        
    1781 & वा॒युः   ।   अ॒सि॒   ।   प्रा॒ण इति॑ प्र{-}अ॒नः   ।   नाम॑   ।    & TS\_3.3.5.1       \\
    
    \hline
        
    1782 & वा॒युः   ।   व्यवा॒दिति॑ वि{-}अवा᳚त्   ।   तस्मा᳚त्   ।   वा॒य॒व्या᳚   ।    & TS\_3.4.3.2       \\
    
    \hline
        
    1783 & वा॒युः   ।   हि॒कं॒र्तेति॑ हिं{-}क॒र्ता   ।   अ॒ग्निः   ।   प्र॒स्तो॒तेति॑ प्र{-}स्तो॒ता   ।    & TS\_3.3.2.1       \\
    
    \hline
        
    1784 & वा॒योः   ।   प॒क्ष॒तिः   ।   सर॑स्वतः   ।   निप॑क्षति॒रिति॒ नि{-}प॒क्ष॒तिः॒   ।    & TS\_5.7.22.1       \\
    
    \hline
        
    1785 & वा॒य॒व्य᳚म्   ।   श्वे॒तम्   ।   एति॑   ।   ल॒भे॒त॒   ।    & TS\_2.1.1.1       \\
    
    \hline
        
    1786 & वा॒रु॒णः   ।   वै   ।   अ॒ग्निः   ।   उप॑नद्ध॒ इत्युप॑{-}न॒द्धः॒   ।    & TS\_5.1.6.1       \\
    
    \hline
        
    1787 & वा॒रु॒णः   ।   वै   ।   क्री॒तः   ।   सोमः॑   ।    & TS\_6.1.11.1       \\
    
    \hline
        
    1788 & वा॒रु॒णाः   ।   त्रयः॑   ।   कृ॒ष्णल॑लामा॒ इति॑ कृ॒ष्ण{-}ल॒ला॒माः॒   ।   वरु॑णाय   ।    & TS\_5.6.20.1       \\
    
    \hline
        
    1789 & वा॒रु॒णेन॑   ।   ए॒व   ।   ए॒न॒म्   ।   व॒रु॒ण॒पा॒शादिति॑ वरुण{-}पा॒शात्   ।    & TS\_2.3.11.2       \\
    
    \hline
        
    1790 & विप्राः᳚   ।   हि   ।   ए॒ते   ।   यत्   ।    & TS\_2.5.9.2       \\
    
    \hline
        
    1791 & विभ॑क्ति॒मिति॒ वि{-}भ॒क्ति॒म्   ।   क॒रो॒ति॒   ।   ब्रह्म॑   ।   ए॒व   ।    & TS\_1.5.2.3       \\
    
    \hline
        
    1792 & विश्वे᳚   ।   ए॒न॒म्   ।   अन्विति॑   ।   म॒द॒न्तु॒   ।    & TS\_4.1.7.4       \\
    
    \hline
        
    1793 & विषु॑रूपे॒ इति॒ विषु॑{-}रू॒पे॒   ।   अह॑नी॒ इति॑   ।   द्यौः   ।   इ॒व॒   ।    & TS\_4.1.11.3       \\
    
    \hline
        
    1794 & विष्णु॑मुखा॒ इति॒ विष्णु॑{-}मु॒खाः॒   ।   वै   ।   दे॒वाः   ।   छन्दो॑भि॒रिति॒ छन्दः॑{-}भिः॒   ।    & TS\_5.2.1.1       \\
    
    \hline
        
    1795 & विष्णोः᳚   ।   क्रमः॑   ।   अ॒सि॒   ।   अ॒भि॒मा॒ति॒हेत्य॑भिमाति{-}हा   ।    & TS\_4.2.1.1       \\
    
    \hline
        
    1796 & वि॒त्ताय॒नीति॑ वित्त{-}अय॑नी   ।   मे॒   ।   अ॒सि॒   ।   ति॒क्ताय॒नीति॑ तिक्त{-}अय॑नी   ।    & TS\_1.2.12.1       \\
    
    \hline
        
    1797 & वि॒धी॒यत॒ इति॑ वि{-}धी॒यते᳚   ।   अ॒र्के   ।   ए॒व   ।   तत्   ।    & TS\_5.3.4.7       \\
    
    \hline
        
    1798 & वि॒न्दते᳚   ।   प्र॒जामिति॑ प्र{-}जाम्   ।   वै॒श्वा॒न॒रम्   ।   द्वाद॑शकपाल॒मिति॒ द्वाद॑श{-}क॒पा॒ल॒म्   ।    & TS\_2.2.5.3       \\
    
    \hline
        
    1799 & वि॒प्राः॒   ।   अ॒मृ॒ताः॒   ।   ऋ॒त॒ज्ञा॒ इत्यृ॑त{-}ज्ञाः॒   ।   अ॒स्य   ।    & TS\_4.7.12.2       \\
    
    \hline
        
    1800 & वि॒भूरिति॑ वि{-}भूः   ।   अ॒सि॒   ।   प्र॒वाह॑ण॒ इति॑ प्र{-}वाह॑नः   ।   वह्निः॑   ।    & TS\_1.3.3.1       \\
    
    \hline
        
    1801 & वि॒भूरिति॑ वि{-}भूः   ।   मा॒त्रा   ।   प्र॒भूरिति॑ प्र{-}भूः   ।   पि॒त्रा   ।    & TS\_7.1.12.1       \\
    
    \hline
        
    1802 & वि॒शा॒ल इति॑ वि{-}शा॒लः   ।   भ॒व॒ति॒   ।   व्यव॑सायय॒तीति॑ वि{-}अव॑साययति   ।   ए॒व   ।    & TS\_2.1.8.5       \\
    
    \hline
        
    1803 & वि॒शे   ।   जना॑य   ।   महि॑   ।   शर्म॑   ।    & TS\_2.5.12.3       \\
    
    \hline
        
    1804 & वि॒श्वक॒र्मेति॑ वि॒श्व{-}क॒र्मा॒   ।   दि॒शाम्   ।   पतिः॑   ।   सः   ।    & TS\_5.5.5.1       \\
    
    \hline
        
    1805 & वि॒श्वक॒र्मेति॑ वि॒श्व{-}क॒र्मा॒   ।   वीति॑   ।   द्याम्   ।   और्णो᳚त्   ।    & TS\_4.6.2.5       \\
    
    \hline
        
    1806 & वि॒श्वरू॑प॒ इति॑ वि॒श्व{-}रू॒पः॒   ।   वै   ।   त्वा॒ष्ट्रः   ।   पु॒रोहि॑त॒ इति॑ पु॒रः{-}हि॒तः॒   ।    & TS\_2.5.1.1       \\
    
    \hline
        
    1807 & वि॒श्व॒च॒र्॒.ष॒ण॒ इति॑ विश्व{-}च॒र्॒.ष॒णे॒   ।   मि॒त्रस्य॑   ।   च॒र्॒.ष॒णी॒धृत॒ इति॑ चर्.षणी{-}धृतः॑   ।   श्रवः॑   ।    & TS\_3.4.11.5       \\
    
    \hline
        
    1808 & वि॒श्व॒जितीति॑ विश्व{-}जिति॑   ।   यथा᳚   ।   दु॒ग्धाम्   ।   उ॒प॒सीद॒तीत्यु॑प{-} सीद॑ति   ।    & TS\_7.2.2.3       \\
    
    \hline
        
    1809 & वि॒श॒   ।   मा   ।   ए॒षा॒म्   ।   कम्   ।    & TS\_4.6.4.5       \\
    
    \hline
        
    1810 & वि॒षू॒चीना॑नि   ।   तस्य॑   ।   उपेति॑   ।   द॒द्ध्या॒त्   ।    & TS\_5.2.9.4       \\
    
    \hline
        
    1811 & वि॒ष्णु॒क्र॒मानिति॑ विष्णु{-}क्र॒मान्   ।   क्रम॑ते   ।   सु॒व॒र्गायेति॑ सुवः{-}गाय॑   ।   हि   ।    & TS\_1.7.6.2       \\
    
    \hline
        
    1812 & विꣳ॒॒श॒त्यै   ।   स्वाहा᳚   ।   च॒त्वा॒रिꣳ॒॒शते᳚   ।   स्वाहा᳚   ।    & TS\_7.2.18.1       \\
    
    \hline
        
    1813 & वीक्ष॑माणा॒येति॑ वि{-}ईक्ष॑माणाय   ।   स्वाहा᳚   ।   वीक्षि॑ता॒येति॒ वि{-}ई॒क्षि॒ता॒य॒   ।   स्वाहा᳚   ।    & TS\_7.1.19.3       \\
    
    \hline
        
    1814 & वीति॑   ।   अ॒द्‌र्ध॒ये॒य॒म्   ।   इति॑   ।   ऊ॒द्‌र्ध्वाम्   ।    & TS\_6.3.4.6       \\
    
    \hline
        
    1815 & वीति॑   ।   आ॒र्च्छ॒त्   ।   सः   ।   आ॒त्मन्न्   ।    & TS\_6.5.11.4       \\
    
    \hline
        
    1816 & वीति॑   ।   ख्ये॒ष॒म्   ।   वै॒श्वा॒न॒रम्   ।   ज्योतिः॑   ।    & TS\_1.3.4.3       \\
    
    \hline
        
    1817 & वीति॑   ।   पाज॑सा   ।   पृ॒थुना᳚   ।   शोशु॑चानः   ।    & TS\_4.1.5.1       \\
    
    \hline
        
    1818 & वीति॑   ।   पाज॑सा   ।   वीति॑   ।   ज्योति॑षा   ।    & TS\_2.5.12.5       \\
    
    \hline
        
    1819 & वीति॑   ।   म॒द्ध्य॒मम्   ।   श्र॒था॒य॒   ।   अथ॑   ।    & TS\_4.2.1.4       \\
    
    \hline
        
    1820 & वीति॑   ।   वै   ।   ए॒तस्य॑   ।   य॒ज्ञ्ः   ।    & TS\_3.4.1.1       \\
    
    \hline
        
    1821 & वीति॑   ।   वै   ।   ए॒तौ   ।   द्वि॒षा॒ते॒ इति॑   ।    & TS\_5.2.4.1       \\
    
    \hline
        
    1822 & वीति॑   ।   व॒यम्   ।   रु॒हे॒म॒   ।   पृ॒थि॒व्याः   ।    & TS\_1.1.2.2       \\
    
    \hline
        
    1823 & वीति॑   ।   सृ॒ज॒ते॒   ।   शान्त्यै᳚   ।   अप्र॑दाहा॒येत्यप्र॑{-}दा॒हा॒य॒   ।    & TS\_1.7.6.7       \\
    
    \hline
        
    1824 & वी॒तये᳚   ।   ऋ॒तम्   ।   स॒त्यम्   ।   ओष॑धयः   ।    & TS\_4.1.4.4       \\
    
    \hline
        
    1825 & वी॒रः   ।   श॒तम॑न्यु॒रिति॑ श॒त{-}म॒न्युः॒   ।   इन्द्रः॑   ।   दु॒श्च्य॒व॒न इति॑ दुः{-}च्य॒व॒नः   ।    & TS\_4.6.4.3       \\
    
    \hline
        
    1826 & वी॒रुधः॑   ।   भ॒व॒न्ति॒   ।   ताः   ।   अ॒ग्निः   ।    & TS\_5.5.4.2       \\
    
    \hline
        
    1827 & वी॒र्ये᳚   ।   प॒शुषु॑   ।   प्रतीति॑   ।   ति॒ष्ठ॒न्ति॒   ।    & TS\_7.4.6.3       \\
    
    \hline
        
    1828 & वी॒र्य॑काम॒ इति॑ वी॒र्य॑{-}का॒मः॒   ।   स्यात्   ।   तम्   ।   ए॒तया᳚   ।    & TS\_2.3.7.2       \\
    
    \hline
        
    1829 & वी॒र्य॑सम्मित॒ इति॑ वी॒र्य॑{-}स॒म्मि॒तः॒   ।   वेणु॑ना   ।   वीति॑   ।   मि॒मी॒ते॒   ।    & TS\_5.2.5.2       \\
    
    \hline
        
    1830 & वी॒र्य᳚म्   ।   वृ॒ङ्क्ते॒   ।   पु॒रा   ।   वा॒चः   ।    & TS\_2.2.9.5       \\
    
    \hline
        
    1831 & वृ॒त्रम्   ।   अ॒ह॒न्न्   ।   सः   ।   अ॒पः   ।    & TS\_6.4.2.3       \\
    
    \hline
        
    1832 & वृ॒त्राय॑   ।   वज्र᳚म्   ।   प्रेति॑   ।   अ॒ह॒र॒त्   ।    & TS\_5.2.6.2       \\
    
    \hline
        
    1833 & वृ॒श्चे॒त्   ।   यत्   ।   अ॒क्ष॒स॒ङ्गमित्य॑क्ष{-}स॒ङ्गम्   ।   वृ॒श्चेत्   ।    & TS\_6.3.3.4       \\
    
    \hline
        
    1834 & वृ॒श्च्य॒ते॒   ।   देवीः᳚   ।   आ॒पः॒   ।   अ॒पा॒म्   ।    & TS\_6.1.4.9       \\
    
    \hline
        
    1835 & वृ॒ष्टयः॑   ।   उदिति॑   ।   ई॒र॒य॒थ॒   ।   म॒रु॒तः॒   ।    & TS\_2.4.8.2       \\
    
    \hline
        
    1836 & वृ॒ष्टि॒सनी॒रिति॑ वृष्टि{-}सनीः᳚   ।   उपेति॑   ।   द॒धा॒ति॒   ।   वृष्टि᳚म्   ।    & TS\_5.3.10.1       \\
    
    \hline
        
    1837 & वेद॑   ।   मन॑सा   ।   तु   ।   वै   ।    & TS\_5.1.3.4       \\
    
    \hline
        
    1838 & वे॒धसः॑   ।   पू॒ष्णः   ।   तानि॑   ।   अपीति॑   ।    & TS\_1.5.10.4       \\
    
    \hline
        
    1839 & वै   ।   अह्नः॑   ।   वर्णः॑   ।   यत्   ।    & TS\_6.1.3.2       \\
    
    \hline
        
    1840 & वै   ।   अ॒ग्निः   ।   यावान्॑   ।   ए॒व   ।    & TS\_5.4.10.2       \\
    
    \hline
        
    1841 & वै   ।   अ॒द्ध्व॒र्योः   ।   प्र॒ति॒ष्ठामिति॑ प्रति{-}स्थाम्   ।   वेद॑   ।    & TS\_3.1.2.3       \\
    
    \hline
        
    1842 & वै   ।   अ॒र्द्ध॒मा॒सस्येत्य॑र्द्ध{-}मा॒सस्य॑   ।   रात्र॑यः   ।   अ॒र्द्ध॒मा॒स॒श इत्य॑र्द्धमास{-}शः   ।    & TS\_2.5.8.3       \\
    
    \hline
        
    1843 & वै   ।   अ॒स्याः   ।   अ॒मृत᳚म्   ।   यत्   ।    & TS\_5.6.4.2       \\
    
    \hline
        
    1844 & वै   ।   अ॒हो॒रा॒त्राणीइत्य॑हः{-}रा॒त्राणि॑   ।   प॒शून्   ।   ए॒व   ।    & TS\_2.1.5.3       \\
    
    \hline
        
    1845 & वै   ।   आपः॑   ।   तस्मा᳚त्   ।   अ॒द्भिरित्य॑त्{-}भिः   ।    & TS\_5.6.2.2       \\
    
    \hline
        
    1846 & वै   ।   आ॒दि॒त्यः   ।   यतः॑   ।   अजा॑यत   ।    & TS\_2.1.8.2       \\
    
    \hline
        
    1847 & वै   ।   आ॒र॒ण्यम्   ।   इ॒न्द्रि॒यम्   ।   ए॒व   ।    & TS\_1.6.7.4       \\
    
    \hline
        
    1848 & वै   ।   इ॒यम्   ।   अग्रे᳚   ।   आ॒सी॒त्   ।    & TS\_6.2.4.4       \\
    
    \hline
        
    1849 & वै   ।   ए॒तत्   ।   प॒शुम्   ।   यत्   ।    & TS\_6.3.11.2       \\
    
    \hline
        
    1850 & वै   ।   ए॒तम्   ।   दी॒क्ष॒य॒न्ति॒   ।   सः   ।    & TS\_5.2.4.2       \\
    
    \hline
        
    1851 & वै   ।   ए॒तस्य॑   ।   अशा᳚न्तः   ।   योनि᳚म्   ।    & TS\_2.2.5.2       \\
    
    \hline
        
    1852 & वै   ।   ए॒ताम्   ।   पत्न॑यः   ।   अग्रे᳚   ।    & TS\_5.1.7.2       \\
    
    \hline
        
    1853 & वै   ।   ए॒ते   ।   दे॒वकृ॑त॒मिति॑ दे॒व{-}कृ॒त॒म्   ।   च॒   ।    & TS\_7.5.9.4       \\
    
    \hline
        
    1854 & वै   ।   ए॒षः   ।   अ॒स्मात्   ।   लो॒कात्   ।    & TS\_1.5.8.3       \\
    
    \hline
        
    1855 & वै   ।   ए॒षः   ।   यत्   ।   अ॒ग्निः   ।    & TS\_5.2.8.3       \\
    
    \hline
        
    1856 & वै   ।   ए॒षः   ।   सि॒सी॒र्.॒ष॒ति॒   ।   यः   ।    & TS\_2.2.4.6       \\
    
    \hline
        
    1857 & वै   ।   ग्रु॒मु॒ष्टिः   ।   स॒यो॒नि॒त्वायेति॑ सयोनि{-}त्वाय॑   ।   द्वाभ्या᳚म्   ।    & TS\_5.4.5.3       \\
    
    \hline
        
    1858 & वै   ।   ति॒ष्यः॑   ।   सोमः॑   ।   पू॒र्णमा॑स॒ इति॑ पू॒र्ण{-}मा॒सः॒   ।    & TS\_2.2.10.2       \\
    
    \hline
        
    1859 & वै   ।   तृ॒तीय᳚म्   ।   य॒ज्ञ्म्   ।   आ॒प॒त्   ।    & TS\_7.2.8.5       \\
    
    \hline
        
    1860 & वै   ।   दे॒वत॑या   ।   प॒शवः॑   ।   यत्   ।    & TS\_6.3.11.4       \\
    
    \hline
        
    1861 & वै   ।   दे॒वाः   ।   असु॑राणाम्   ।   श॒त॒त॒र्.॒हानिति॑ शत{-}त॒र्॒.हान्   ।    & TS\_5.4.7.4       \\
    
    \hline
        
    1862 & वै   ।   दे॒वाः   ।   उज्जि॑ति॒मित्युत्{-}जि॒ति॒म्   ।   अनु॑   ।    & TS\_1.7.4.2       \\
    
    \hline
        
    1863 & वै   ।   द॒श॒रा॒त्रेणेति॑ दश{-}रा॒त्रेण॑   ।   प्र॒जाप॑ति॒रिति॑ प्र॒जा{-}प॒तिः॒   ।   प्र॒जा इति॑ प्र{-}जाः   ।    & TS\_7.4.5.4       \\
    
    \hline
        
    1864 & वै   ।   निर्.ऋ॑त्या॒ इति॒ निः{-}ऋ॒त्यै॒   ।   दिक्   ।   स्वाया᳚म्   ।    & TS\_5.2.4.3       \\
    
    \hline
        
    1865 & वै   ।   पव॑मानानाम्   ।   अ॒न्वा॒रो॒हा इत्य॑नु{-}आ॒रो॒हाः   ।   तान्   ।    & TS\_3.2.1.2       \\
    
    \hline
        
    1866 & वै   ।   ब्र॒ह्म॒व॒र्च॒समिति॑ ब्रह्म{-}व॒र्च॒सम्   ।   छन्द॑साम्   ।   ए॒व   ।    & TS\_2.1.7.3       \\
    
    \hline
        
    1867 & वै   ।   मृ॒त्युब॑न्धव॒ इति॑ मृ॒त्यु{-}ब॒न्ध॒वः॒   ।   तेषा᳚म्   ।   य॒मः   ।    & TS\_5.1.8.2       \\
    
    \hline
        
    1868 & वै   ।   य॒ज्ञा॒यु॒धानीति॑ यज्ञ्{-}आ॒यु॒धानि॑   ।   य॒ज्ञ्म्   ।   ए॒व   ।    & TS\_5.4.8.4       \\
    
    \hline
        
    1869 & वै   ।   रा॒ज॒सूय॒स्येति॑ राज{-}सूय॑स्य   ।   रू॒पम्   ।   यः   ।    & TS\_5.6.3.4       \\
    
    \hline
        
    1870 & वै   ।   रे॒तो॒धा इति॑ रेतः{-}धाः   ।   अ॒ग्निः   ।   प्र॒जाना॒मिति॑ प्र{-}जाना᳚म्   ।    & TS\_2.1.2.8       \\
    
    \hline
        
    1871 & वै   ।   सु॒व॒र्ग इति॑ सुवः{-}गः   ।   लो॒कः   ।   सु॒व॒र्गमिति॑ सुवः{-}गम्   ।    & TS\_2.5.11.7       \\
    
    \hline
        
    1872 & वै   ।   स॒मु॒द्रः   ।   तस्य॑   ।   ए॒तत्   ।    & TS\_7.5.1.3       \\
    
    \hline
        
    1873 & वैक॑ङ्कतीम्   ।   एति॑   ।   द॒धा॒ति॒   ।   भाः   ।    & TS\_5.1.9.6       \\
    
    \hline
        
    1874 & वै॒श्वा॒न॒रः   ।   नः॒   ।   ऊ॒त्या   ।   आ   ।    & TS\_1.5.11.1       \\
    
    \hline
        
    1875 & वै॒श्वा॒न॒रम्   ।   द्वाद॑शकपाल॒मिति॒ द्वाद॑श{-}क॒पा॒ल॒म्   ।   निरिति॑   ।   व॒पे॒त्   ।    & TS\_2.2.5.1       \\
    
    \hline
        
    1876 & वै॒श्व॒क॒र्म॒णीरिति॑ वैश्व{-}क॒र्म॒णीः   ।   अ॒हो॒रा॒त्राणीत्य॑हः{-}रा॒त्राणि॑   ।   ए॒व   ।   अ॒स्य॒   ।    & TS\_5.6.9.3       \\
    
    \hline
        
    1877 & वै॒ष्ण॒वम्   ।   दे॒व॒य॒ज्याया॒ इति॑ देव{-}य॒ज्यायै᳚   ।   इति॑   ।   आ॒ह॒   ।    & TS\_6.3.3.2       \\
    
    \hline
        
    1878 & वै॒ष्ण॒व्या   ।   ऋ॒चा   ।   हु॒त्वा   ।   यूप᳚म्   ।    & TS\_6.3.3.1       \\
    
    \hline
        
    1879 & वै॒स॒र्ज॒नानि॑   ।   जु॒हो॒ति॒   ।   रक्ष॑साम्   ।   अप॑हत्या॒ इत्यप॑{-}ह॒त्यै॒   ।    & TS\_6.3.2.2       \\
    
    \hline
        
    1880 & व्यति॑षज्यन्त॒ इति॑ वि{-}अति॑षज्यन्ते   ।   प्र॒जयेति॑ प्र{-}जया᳚   ।   प॒शुभि॒रिति॑ प॒शु{-}भिः॒   ।   ये   ।    & TS\_7.4.3.6       \\
    
    \hline
        
    1881 & व्यवै᳚र्य॒तेति॑ वि{-}अवै᳚र्यत   ।   तस्मा᳚त्   ।   रु॒द्राः   ।   घातु॑काः   ।    & TS\_7.1.5.4       \\
    
    \hline
        
    1882 & व्याच॑ष्ट॒ इति॑ वि{-}आच॑ष्टे   ।   बृह॒स्पतिः॑   ।   त्वा॒   ।   सु॒म्ने   ।    & TS\_6.1.8.2       \\
    
    \hline
        
    1883 & व्याय॑च्छन्त॒ इति॑ वि{-}आय॑च्छन्ते   ।   येषा᳚म्   ।   सोमः॑   ।   स॒मृ॒च्छत॒ इति॑ सं{-}ऋ॒च्छते᳚   ।    & TS\_3.1.7.2       \\
    
    \hline
        
    1884 & व्यृ॑द्ध॒मिति॒ वि{-}ऋ॒द्ध॒म्   ।   वै   ।   ए॒तत्   ।   य॒ज्ञ्स्य॑   ।    & TS\_5.1.2.1       \\
    
    \hline
        
    1885 & व्र॒तम्   ।   एका᳚   ।   र॒क्ष॒ति॒   ।   दे॒व॒यू॒नामिति॑ देव{-}यू॒नाम्   ।    & TS\_4.3.11.2       \\
    
    \hline
        
    1886 & व्र॒ते   ।   तव॑   ।   अना॑गसः   ।   अदि॑तये   ।    & TS\_1.5.11.4       \\
    
    \hline
        
    1887 & व॒क्षि॒   ।   यक्षि॑   ।   च॒   ।   सः   ।    & TS\_4.6.1.3       \\
    
    \hline
        
    1888 & व॒थ्सम्   ।   एति॑   ।   ल॒भे॒त॒   ।   वा॒युः   ।    & TS\_2.1.4.8       \\
    
    \hline
        
    1889 & व॒द॒न्ति॒   ।   अ॒ग्नौ   ।   ग्रा॒म्यान्   ।   प॒शून्   ।    & TS\_5.5.5.2       \\
    
    \hline
        
    1890 & व॒द॒न्ति॒   ।   यत्   ।   घृ॒तेन॑   ।   पा॒त्नी॒व॒तमिति॑ पात्नी{-}व॒तम्   ।    & TS\_6.5.8.3       \\
    
    \hline
        
    1891 & व॒ना॒म॒हे॒   ।   धु॒क्षी॒महि॑   ।   प्र॒जामिति॑ प्र{-}जाम्   ।   इष᳚म्   ।    & TS\_1.6.4.3       \\
    
    \hline
        
    1892 & व॒पा॒मि॒   ।   म॒ही॒नाम्   ।   पयः॑   ।   अ॒सि॒   ।    & TS\_1.1.10.3       \\
    
    \hline
        
    1893 & व॒पे॒त्   ।   यम्   ।   य॒ज्ञ्ः   ।   न   ।    & TS\_3.4.9.4       \\
    
    \hline
        
    1894 & व॒पे॒त्   ।   स॒ग्रां॒ममिति॑ सं{-}ग्रा॒मम्   ।   उ॒प॒प्र॒या॒स्यन्नित्यु॑प{-}प्र॒या॒स्यन्न्   ।   इ॒न्द्रा॒ग्नी इती᳚न्द्र{-}अ॒ग्नी   ।    & TS\_2.2.1.3       \\
    
    \hline
        
    1895 & व॒रु॒ण॒पा॒शादिति॑ वरुण{-}पा॒शात्   ।   मु॒ञ्च॒ति॒   ।   एति॑   ।   त्वा॒   ।    & TS\_5.2.1.4       \\
    
    \hline
        
    1896 & व॒शाम्   ।   एति॑   ।   ल॒भे॒त॒   ।   ऐ॒न्द्रम्   ।    & TS\_2.1.4.5       \\
    
    \hline
        
    1897 & व॒ष॒ट्का॒र इति॑ वषट्{-}का॒रः   ।   वै   ।   गा॒य॒त्रि॒यै   ।   शिरः॑   ।    & TS\_2.1.7.1 TS\_3.5.7.1       \\
    
    \hline
        
    1898 & व॒ष॒ट्का॒रेणेति॑ वषट्{-}का॒रेण॑   ।   स्था॒प॒य॒ति॒   ।   त्रि॒पदेति॑ त्रि{-}पदा᳚   ।   पु॒रो॒नु॒वा॒क्येति॑ पुरः{-}अ॒नु॒वा॒क्या᳚   ।    & TS\_2.6.2.6       \\
    
    \hline
        
    1899 & शं॒ॅयोरिति॑ शं{-}योः   ।   एति॑   ।   वृ॒णी॒म॒हे॒   ।   इति॑   ।    & TS\_2.6.10.3       \\
    
    \hline
        
    1900 & शका᳚   ।   भौ॒मी   ।   पा॒न्त्रः   ।   कशः॑   ।    & TS\_5.5.18.1       \\
    
    \hline
        
    1901 & शची॑भि॒रिति॒ शचि॑{-}भिः॒   ।   दे॒वः   ।   इ॒व॒   ।   स॒वि॒ता   ।    & TS\_4.2.5.5       \\
    
    \hline
        
    1902 & शन्त॑मे॒नेति॒ शं{-}त॒मे॒न॒   ।   तया᳚   ।   दे॒वत॑या   ।   अ॒ङ्गि॒र॒स्वत्   ।    & TS\_4.3.6.2       \\
    
    \hline
        
    1903 & शम्   ।   ओष॑धीभ्य॒ इत्योषा॑ध{-}भ्यः॒   ।   शम्   ।   पृ॒थि॒व्यै   ।    & TS\_1.3.9.2       \\
    
    \hline
        
    1904 & शम्   ।   च॒   ।   मे॒   ।   मयः॑   ।    & TS\_4.7.3.1       \\
    
    \hline
        
    1905 & शम्   ।   योः   ।   अ॒र॒पः   ।   द॒धा॒त॒   ।    & TS\_2.6.12.3       \\
    
    \hline
        
    1906 & शरः॑   ।   अव॑काः   ।   वे॒त॒स॒शा॒खयेति॑ वेतस{-}शा॒खया᳚   ।   च॒   ।    & TS\_5.4.4.3       \\
    
    \hline
        
    1907 & शर्म॑   ।   य॒च्छ॒तु॒   ।   एति॑   ।   ज॒ङ्घ॒न्ति॒   ।    & TS\_4.6.6.5       \\
    
    \hline
        
    1908 & शान्त्यै᳚   ।   पा॒र्श्व॒तः   ।   एति॑   ।   छ्‌य॒ति॒   ।    & TS\_6.3.9.2       \\
    
    \hline
        
    1909 & शिरः॑   ।   प्रतीति॑   ।   द॒धा॒ति॒   ।   समिति॑   ।    & TS\_6.3.7.4       \\
    
    \hline
        
    1910 & शिरः॑   ।   वै   ।   ए॒तत्   ।   य॒ज्ञ्स्य॑   ।    & TS\_6.2.11.1       \\
    
    \hline
        
    1911 & शि॒ति॒बा॒हुरिति॑ शिति{-}बा॒हुः   ।   अ॒न्यत॑श्शितिबाहु॒रित्य॒न्यतः॑ {-}शि॒ति॒बा॒हुः॒   ।   स॒म॒न्तशि॑तिबाहु॒रिति॑ सम॒न्त{-}शि॒ति॒बा॒हुः॒   ।   ते   ।    & TS\_5.6.13.1       \\
    
    \hline
        
    1912 & शुन्ध॑द्ध्वम्   ।   दैव्या॑य   ।   कर्म॑णे   ।   दे॒व॒य॒ज्याया॒ इति॑ देव {-}य॒ज्यायै᳚   ।    & TS\_1.1.3.1       \\
    
    \hline
        
    1913 & शु॒क्रः   ।   यत्   ।   स॒प्त॒मे   ।   अहन्न्॑   ।    & TS\_7.2.8.4       \\
    
    \hline
        
    1914 & शु॒क्राः   ।   ताः   ।   नः॒   ।   आपः॑   ।    & TS\_5.6.1.2       \\
    
    \hline
        
    1915 & शु॒क्रासु॑   ।   ते॒   ।   शु॒क्र॒   ।   शु॒क्रम्   ।    & TS\_3.3.3.2       \\
    
    \hline
        
    1916 & शु॒चा   ।   अ॒र्प॒य॒ति॒   ।   तस्मा᳚त्   ।   स॒माव॑त्   ।    & TS\_5.1.4.3       \\
    
    \hline
        
    1917 & शु॒चे॒   ।   शुच॑यः   ।   च॒र॒न्ति॒   ।   तु॒वि॒म्र॒क्षास॒ इति॑ तुवि{-}म्र॒क्षासः॑   ।    & TS\_3.3.11.2       \\
    
    \hline
        
    1918 & शु॒ण्ठाः   ।   त्रयः॑   ।   वै॒ष्ण॒वाः   ।   अ॒धी॒लो॒ध॒कर्णा॒ इत्य॑धीलोध{-}कर्णाः᳚   ।    & TS\_5.6.16.1       \\
    
    \hline
        
    1919 & शूरः॑   ।   वा॒   ।   पृ॒थ्स्विति॑ पृत्{-}सु   ।   कासु॑   ।    & TS\_1.8.22.4       \\
    
    \hline
        
    1920 & शृ॒तम्   ।   कु॒रु॒त॒   ।   इति॑   ।   अ॒ब्र॒वी॒त्   ।    & TS\_2.5.3.4       \\
    
    \hline
        
    1921 & शो॒चे॒   ।   वेः   ।   त्वम्   ।   हि   ।    & TS\_4.3.13.5       \\
    
    \hline
        
    1922 & श्ये॒नाय॑   ।   पत्व॑ने   ।   स्वाहा᳚   ।   वट्   ।    & TS\_3.2.8.1       \\
    
    \hline
        
    1923 & श्रि॒ताः   ।   स॒ह॒स्र॒श इति॑ सहस्र{-}शः   ।   अवेति॑   ।   ए॒षा॒म्   ।    & TS\_4.5.1.3       \\
    
    \hline
        
    1924 & श्रि॒यै   ।   ग॒त्वा   ।   नीति॑   ।   व॒र्त॒ते॒   ।    & TS\_7.2.7.3       \\
    
    \hline
        
    1925 & श्रीः   ।   हि   ।   म॒नु॒ष्य॑स्य   ।   सु॒व॒र्ग इति॑ सुवः{-}गः   ।    & TS\_7.4.4.2       \\
    
    \hline
        
    1926 & श्रु॒तम्   ।   ग॒र्त्त॒सद॒मिति॑ गर्त्त{-}सद᳚म्   ।   युवा॑नम्   ।   मृ॒गम्   ।    & TS\_4.5.10.4       \\
    
    \hline
        
    1927 & श्रौष॑ट्   ।   इति॑   ।   उ॒पावा᳚स्रा॒गित्यु॑प{-}अवा᳚स्राक्   ।   यज॑   ।    & TS\_1.6.11.3       \\
    
    \hline
        
    1928 & श्र॒द्धाम॑ना॒ इति॑ श्र॒द्धा{-}म॒नाः॒   ।   ह॒विषा᳚   ।   ब्रह्म॑णः   ।   पति᳚म्   ।    & TS\_2.3.14.4       \\
    
    \hline
        
    1929 & श्र॒व॒स्यवः॑   ।   घृ॒तस्य॑   ।   धाराः᳚   ।   उपेति॑   ।    & TS\_1.8.22.5       \\
    
    \hline
        
    1930 & श॒तानि॑   ।   त्रय॑स्त्रिꣳशत॒मिति॒ त्रयः॑{-}त्रिꣳ॒॒श॒त॒म्   ।   च॒   ।   अथ॑   ।    & TS\_7.1.5.3       \\
    
    \hline
        
    1931 & श॒ताय॑   ।   स्वाहा᳚   ।   स॒हस्रा॑य   ।   स्वाहा᳚   ।    & TS\_7.2.20.1       \\
    
    \hline
        
    1932 & श॒तैः   ।   सोमः॑   ।   क्री॒तः   ।   भ॒व॒ति॒   ।    & TS\_7.1.6.3       \\
    
    \hline
        
    1933 & श॒त॒मू॒द्‌र्ध॒न्निति॑ शत{-}मू॒र्ध॒न्न्   ।   श॒तम्   ।   ते॒   ।   प्रा॒णा इति॑ प्र{-}अ॒नाः   ।    & TS\_4.6.5.3       \\
    
    \hline
        
    1934 & श॒भूंरिति॑ शं{-}भूः   ।   म॒यो॒भूरिति॑ मयः{-}भूः   ।   स्व॒स्ति   ।   मा॒   ।    & TS\_3.2.5.2       \\
    
    \hline
        
    1935 & श॒रदा᳚   ।   त्वा॒   ।   ऋ॒तुना᳚   ।   ह॒विषा᳚   ।    & TS\_7.1.18.2       \\
    
    \hline
        
    1936 & श॒स्तोक्थ॒स्येति॑ श॒स्त{-}उ॒क्थ॒स्य॒   ।   हरि॑वत॒ इति॒ हरि॑{-}व॒तः॒   ।   इन्द्र॑पीत॒स्येतीन्द्र॑{-}पी॒त॒स्य॒   ।   मधु॑मत॒ इति॒ मधु॑{-}म॒तः॒   ।    & TS\_3.2.5.5       \\
    
    \hline
        
    1937 & षट्   ।   प॒दानि॑   ।   अनु॑   ।   नीति॑   ।    & TS\_6.1.8.1       \\
    
    \hline
        
    1938 & ष॒ड्भिरिति॑ षट्{-}भिः   ।   दी॒क्ष॒य॒ति॒   ।   षट्   ।   वै   ।    & TS\_5.1.9.1       \\
    
    \hline
        
    1939 & ष॒ड॒हेनेति॑ षट्{-}अ॒हेन॑   ।   यन्ति॑   ।   दे॒व॒च॒क्रमिति॑ देव{-}च॒क्रम्   ।   ए॒व   ।    & TS\_7.4.11.3       \\
    
    \hline
        
    1940 & ष॒ड॒हैरिति॑ षट्{-}अ॒हैः   ।   मासान्॑   ।   स॒पांद्येति॑ सं{-}पाद्य॑   ।   अहः॑   ।    & TS\_7.5.6.1       \\
    
    \hline
        
    1941 & संप्रि॑य॒ इति॒ सं{-}प्रि॒यः॒   ।   संप्रि॑या॒ इति॒ सं{-}प्रि॒याः॒   ।   त॒नुवः॑   ।   मम॑   ।    & TS\_4.2.4.2       \\
    
    \hline
        
    1942 & संॅय॑त्त॒ इति॒ सं{-}य॒त्ते॒   ।   इ॒न्द्रि॒येण॑   ।   वै   ।   म॒न्युना᳚   ।    & TS\_2.1.3.2       \\
    
    \hline
        
    1943 & सं॒ॅवत॒ इति॑ सं{-}वतः॑   ।   अव॑रान्   ।   अ॒भि   ।   एति॑   ।    & TS\_2.6.11.4       \\
    
    \hline
        
    1944 & सं॒ॅवर॑ण॒ इति॑ सं{-}वर॑णे   ।   भ॒व॒   ।   नः॒   ।   स॒प॒त्न॒हेति॑ सपत्न{-}हा   ।    & TS\_4.1.7.2       \\
    
    \hline
        
    1945 & सं॒ॅवि॒दा॒नेति॑ सं{-}वि॒दा॒ना   ।   अग्नी॑षोमा॒वित्यग्नी᳚{-}सो॒मौ॒   ।   प्र॒थ॒मौ   ।   वी॒र्ये॑ण   ।    & TS\_3.5.1.2       \\
    
    \hline
        
    1946 & सं॒ॅव॒थ्स॒र इति॑ सं{-}व॒थ्स॒रः   ।   प्र॒यन्त॒ इति॑ प्र{-}यन्तः॑   ।   ए॒व   ।   सं॒ॅव॒थ्स॒र इति॑ सं{-}व॒थ्स॒रे   ।    & TS\_7.5.1.4       \\
    
    \hline
        
    1947 & सं॒ॅव॒थ्स॒र इति॑ सं{-}व॒थ्स॒रः   ।   ये   ।   ए॒वम्   ।   वि॒द्वाꣳसः॑   ।    & TS\_7.5.1.2       \\
    
    \hline
        
    1948 & सं॒ॅव॒थ्स॒र इति॑ सं{-}व॒थ्स॒रः   ।   वै   ।   इ॒दम्   ।   एकः॑   ।    & TS\_7.1.10.1       \\
    
    \hline
        
    1949 & सं॒ॅव॒थ्स॒रमिति॑ सं{-}व॒थ्स॒रम्   ।   उख्य᳚म्   ।   भृ॒त्वा   ।   द्वि॒तीये᳚   ।    & TS\_5.6.5.1       \\
    
    \hline
        
    1950 & सं॒ॅव॒थ्स॒रस्येति॑ सं{-}व॒थ्स॒रस्य॑   ।   यत्   ।   फ॒ल्गु॒नी॒पू॒र्ण॒मा॒स इति॑ फल्गुनी{-}पू॒र्ण॒मा॒सः   ।   मु॒ख॒तः   ।    & TS\_7.4.8.2       \\
    
    \hline
        
    1951 & सं॒ॅव॒थ्स॒रादिति॑ सं{-}व॒थ्स॒रात्   ।   अपीति॑   ।   रो॒हा॒त्   ।   इति॑   ।    & TS\_2.5.1.3       \\
    
    \hline
        
    1952 & सं॒ॅव॒थ्स॒रायेति॑ सं{-}व॒थ्स॒राय॑   ।   दी॒क्षि॒ष्यमा॑णाः   ।   ए॒का॒ष्ट॒काया॒मित्ये॑क{-}अ॒ष्ट॒काया᳚म्   ।   दी॒क्षे॒र॒न्न्   ।    & TS\_7.4.8.1       \\
    
    \hline
        
    1953 & सः   ।   अ॒ग्नेः   ।   कृष्णः॑   ।   रू॒पम्   ।    & TS\_5.2.6.5       \\
    
    \hline
        
    1954 & सः   ।   अ॒ब्र॒वी॒त्   ।   वर᳚म्   ।   वृ॒णै॒   ।    & TS\_6.3.10.6       \\
    
    \hline
        
    1955 & सः   ।   अ॒भ॒व॒त्   ।   सः   ।   अ॒बि॒भे॒त्   ।    & TS\_2.2.8.6       \\
    
    \hline
        
    1956 & सः   ।   इत्   ।   उ॒   ।   होता᳚   ।    & TS\_1.1.14.4       \\
    
    \hline
        
    1957 & सः   ।   ई॒श्व॒रः   ।   प्र॒जा इति॑ प्र{-}जाः   ।   शु॒चा   ।    & TS\_5.1.5.6       \\
    
    \hline
        
    1958 & सः   ।   उत्त॑र॒ इत्युत्{-}त॒रः॒   ।   प॒क्षः   ।   अ॒भ॒व॒त्   ।    & TS\_5.6.4.4       \\
    
    \hline
        
    1959 & सः   ।   ओष॑धीः   ।   अन्विति॑   ।   रु॒द्ध्य॒से॒   ।    & TS\_4.2.3.3       \\
    
    \hline
        
    1960 & सः   ।   नः॒   ।   मु॒ञ्च॒तु॒   ।   अꣳह॑सः   ।    & TS\_4.7.15.2       \\
    
    \hline
        
    1961 & सः   ।   पापी॑यान्   ।   भ॒वि॒ष्य॒सि॒   ।   इति॑   ।    & TS\_5.5.2.4       \\
    
    \hline
        
    1962 & सः   ।   प्र॒त्न॒वदिति॑ प्रत्न{-}वत्   ।   नीति॑   ।   काव्या᳚   ।    & TS\_2.3.14.1       \\
    
    \hline
        
    1963 & सः   ।   रि॒षः   ।   पा॒तु॒   ।   नक्त᳚म्   ।    & TS\_1.2.14.7       \\
    
    \hline
        
    1964 & सन्न्   ।   दे॒वाना᳚म्   ।   प॒वित्र᳚म्   ।   येषा᳚म्   ।    & TS\_6.4.5.4       \\
    
    \hline
        
    1965 & समाः᳚   ।   त्वा॒   ।   अ॒ग्ने॒   ।   ऋ॒तवः॑   ।    & TS\_4.1.7.1       \\
    
    \hline
        
    1966 & समिति॑   ।   अ॒भ॒व॒त्   ।   ते   ।   दे॒वाः   ।    & TS\_2.1.2.3       \\
    
    \hline
        
    1967 & समिति॑   ।   इ॒त॒म्   ।   समिति॑   ।   क॒ल्पे॒था॒म्   ।    & TS\_4.2.5.1       \\
    
    \hline
        
    1968 & समिति॑   ।   ज्योति॑षा   ।   अ॒भू॒व॒म्   ।   ऐ॒न्द्रीम्   ।    & TS\_1.6.6.2       \\
    
    \hline
        
    1969 & समिति॑   ।   ते॒   ।   मन॑सा   ।   मनः॑   ।    & TS\_1.3.10.1       \\
    
    \hline
        
    1970 & समिति॑   ।   ते॒   ।   वा॒युः   ।   मा॒त॒रिश्वा᳚   ।    & TS\_4.1.4.1       \\
    
    \hline
        
    1971 & समिति॑   ।   त्वा॒   ।   न॒ह्या॒मि॒   ।   पय॑सा   ।    & TS\_3.5.6.1       \\
    
    \hline
        
    1972 & समिति॑   ।   त्वा॒   ।   सि॒ञ्चा॒मि॒   ।   यजु॑षा   ।    & TS\_1.6.1.1       \\
    
    \hline
        
    1973 & समिति॑   ।   न॒ह्य॒ति॒   ।   ए॒व   ।   ए॒न॒म्   ।    & TS\_5.1.5.5       \\
    
    \hline
        
    1974 & समिति॑   ।   प॒श्या॒मि॒   ।   प्र॒जा इति॑ प्र{-}जाः   ।   अ॒हम्   ।    & TS\_1.5.6.1 TS\_1.5.8.1       \\
    
    \hline
        
    1975 & समिति॑   ।   व॒पा॒मि॒   ।   समिति॑   ।   आपः॑   ।    & TS\_1.1.8.1       \\
    
    \hline
        
    1976 & समि॑द्ध इति॒ सं{-}इ॒द्धः॒   ।   अ॒ञ्जन्न्   ।   कृद॑रम्   ।   म॒ती॒नाम्   ।    & TS\_5.1.11.1       \\
    
    \hline
        
    1977 & समु॑ब्जित॒ इति॒ सं{-}उ॒ब्जि॒तः॒   ।   तेषा᳚म्   ।   विशि॑प्रियाणा॒मिति॒ वि{-}शि॒प्रि॒या॒णा॒म्   ।   इष᳚म्   ।    & TS\_1.7.12.2       \\
    
    \hline
        
    1978 & समृ॑द्ध्या॒ इति॒ सं{-}ऋ॒द्ध्यै॒   ।   प्रा॒जा॒प॒त्यमिति॑ प्राजा{-}प॒त्यम्   ।   भ॒व॒ति॒   ।   प्रा॒जा॒प॒त्या इति॑ प्राजा{-}प॒त्याः   ।    & TS\_2.3.2.9       \\
    
    \hline
        
    1979 & सर्वाः᳚   ।   प्र॒जा इति॑ प्र{-}जाः   ।   प्र॒त्यङ्   ।   उदिति॑   ।    & TS\_6.5.4.2       \\
    
    \hline
        
    1980 & सर्वा॑णि   ।   छन्दाꣳ॑सि   ।   ए॒तस्या᳚म्   ।   इष्ट्या᳚म्   ।    & TS\_2.4.11.1       \\
    
    \hline
        
    1981 & सर्वा॑णि   ।   छन्दाꣳ॑सि   ।   छन्दाꣳ॑सि   ।   खलु॑   ।    & TS\_5.1.5.3       \\
    
    \hline
        
    1982 & सर्वा᳚भ्यः   ।   वै   ।   दे॒वता᳚भ्यः   ।   अ॒ग्निः   ।    & TS\_5.3.9.1       \\
    
    \hline
        
    1983 & सर्व॑स्य   ।   प्र॒ति॒शीव॒रीति॑ प्रति{-}शीव॑री   ।   भूमिः॑   ।   त्वा॒   ।    & TS\_1.4.40.1       \\
    
    \hline
        
    1984 & सर्व᳚म्   ।   वृ॒ङ्क्ते॒   ।   पौ॒र्ण॒मा॒सीमिति॑ पौर्ण{-}मा॒सीम्   ।   ए॒व   ।    & TS\_2.5.4.3       \\
    
    \hline
        
    1985 & सर॑स्वती   ।   वा॒चा   ।   ए॒व   ।   ए॒न॒म्   ।    & TS\_7.2.7.5       \\
    
    \hline
        
    1986 & सव॑नम्   ।   अ॒ष्टा॒भिः   ।   उपेति॑   ।   य॒न्ति॒   ।    & TS\_7.5.7.3       \\
    
    \hline
        
    1987 & सव॑नम्   ।   माद्ध्य॑न्दिनम्   ।   ए॒व   ।   सव॑नम्   ।    & TS\_6.4.5.2       \\
    
    \hline
        
    1988 & सा   ।   उ॒त्त॒र॒वे॒दिरित्यु॑त्तर{-}वे॒दिः   ।   अ॒ब्र॒वी॒त्   ।   सर्वान्॑   ।    & TS\_6.2.8.1       \\
    
    \hline
        
    1989 & सा   ।   वि॒राडिति॑ वि{-}राट्   ।   वि॒क्रम्येति॑ वि{-}क्रम्य॑   ।   अ॒ति॒ष्ठ॒त्   ।    & TS\_7.3.9.1       \\
    
    \hline
        
    1990 & सा   ।   वै   ।   इ॒यम्   ।   सर्वा᳚   ।    & TS\_6.2.4.5       \\
    
    \hline
        
    1991 & साम॑   ।   रा॒ज॒न्यः॑   ।   म॒नु॒ष्या॑णाम्   ।   अविः॑   ।    & TS\_7.1.1.5       \\
    
    \hline
        
    1992 & सा॒क्षादिति॑ स{-}अ॒क्षात्   ।   ए॒व   ।   सो॒म॒पी॒थमिति॑ सोम{-}पी॒थम्   ।   अवेति॑   ।    & TS\_2.3.2.8       \\
    
    \hline
        
    1993 & सा॒द्ध्याः   ।   वै   ।   दे॒वाः   ।   अ॒स्मिन्न्   ।    & TS\_6.3.5.1       \\
    
    \hline
        
    1994 & सा॒द्ध्याः   ।   वै   ।   दे॒वाः   ।   सु॒व॒र्गका॑मा॒ इति॑ सुव॒र्ग{-}का॒माः॒   ।    & TS\_7.2.1.1       \\
    
    \hline
        
    1995 & सा॒मि   ।   मा॒र्ज॒य॒न्ते॒   ।   ए॒तत्   ।   प्रतीति॑   ।    & TS\_1.7.1.5       \\
    
    \hline
        
    1996 & सा॒वि॒त्राणि॑   ।   जु॒हो॒ति॒   ।   प्रसू᳚त्या॒ इति॒ प्र{-}सू॒त्यै॒   ।   च॒तु॒र्गृ॒ही॒तेनेति॑ चतुः{-}गृ॒ही॒तेन॑   ।    & TS\_5.1.1.1       \\
    
    \hline
        
    1997 & सि॒ताय॑   ।   स्वाहा᳚   ।   असि॑ताय   ।   स्वाहा᳚   ।    & TS\_7.4.22.1       \\
    
    \hline
        
    1998 & सी॒द॒   ।   वरु॑णः   ।   अ॒सि॒   ।   धृ॒तव्र॑त॒ इति॑ धृ॒त{-}व्र॒तः॒   ।    & TS\_1.2.10.2       \\
    
    \hline
        
    1999 & सुवः॑   ।   न   ।   ज्योतिः॑   ।   अग्ने᳚   ।    & TS\_4.4.4.8       \\
    
    \hline
        
    2000 & सु॒तासः॑   ।   यत्   ।   ई॒म्   ।   स॒बाध॒ इति॑ स{-}बाधः॑   ।    & TS\_1.4.46.2       \\
    
    \hline
        
    2001 & सु॒दा॒न॒व॒ इति॑ सु{-}दा॒न॒वः॒   ।   ए॒ना   ।   वि॒श्पति॑ना   ।   अ॒भीति॑   ।    & TS\_2.3.1.3       \\
    
    \hline
        
    2002 & सु॒प्र॒जा इति॑ सु{-}प्र॒जाः   ।   प्र॒जा इति॑ प्र{-}जाः   ।   प्र॒ज॒नय॒न्निति॑ प्र{-}ज॒नयन्न्॑   ।   परीति॑   ।    & TS\_6.4.10.5       \\
    
    \hline
        
    2003 & सु॒प॒र्ण इति॑ सु{-}प॒र्णः   ।   पा॒र्ज॒न्यः   ।   हꣳ॒॒सः   ।   वृकः॑   ।    & TS\_5.5.21.1       \\
    
    \hline
        
    2004 & सु॒रेता॒ इति॑ सु{-}रेताः᳚   ।   रेतः॑   ।   धि॒षी॒य॒   ।   त्वष्टुः॑   ।    & TS\_1.6.4.4       \\
    
    \hline
        
    2005 & सु॒वा॒नः   ।   सोमः॑   ।   ऋ॒त॒युरित्यृ॑त{-}युः   ।   चि॒के॒त॒   ।    & TS\_2.2.12.4       \\
    
    \hline
        
    2006 & सु॒व॒र्ग इति॑ सुवः{-}गे   ।   लो॒के   ।   प्रतीति॑   ।   ति॒ष्ठ॒न्ति॒   ।    & TS\_7.3.10.2       \\
    
    \hline
        
    2007 & सु॒व॒र्गमिति॑ सुवः{-}गम्   ।   लो॒कम्   ।   ग॒म॒य॒ति॒   ।   वीति॑   ।    & TS\_2.6.5.6       \\
    
    \hline
        
    2008 & सु॒व॒र्गमिति॑ सुवः{-}गम्   ।   लो॒कम्   ।   न   ।   प्रेति॑   ।    & TS\_3.1.9.5       \\
    
    \hline
        
    2009 & सु॒व॒र्गमिति॑ सुवः{-}गम्   ।   लो॒कम्   ।   य॒न्ति॒   ।   परा᳚ञ्चः   ।    & TS\_7.4.2.5       \\
    
    \hline
        
    2010 & सु॒व॒र्गमिति॑ सुवः{-}गम्   ।   वै   ।   ए॒ते   ।   लो॒कम्   ।    & TS\_6.2.4.1 TS\_7.4.9.1       \\
    
    \hline
        
    2011 & सु॒व॒र्गायेति॑ सुवः{-}गाय॑   ।   वै   ।   ए॒तानि॑   ।   लो॒काय॑   ।    & TS\_6.3.2.1 TS\_6.6.1.1       \\
    
    \hline
        
    2012 & सु॒व॒र्गायेति॑ सुवः{-}गाय॑   ।   वै   ।   ए॒ते   ।   लो॒काय॑   ।    & TS\_6.5.4.1       \\
    
    \hline
        
    2013 & सु॒व॒र्गायेति॑ सुवः{-}गाय॑   ।   वै   ।   ए॒षः   ।   लो॒काय॑   ।    & TS\_5.5.7.1 TS\_5.6.8.1       \\
    
    \hline
        
    2014 & सु॒व॒र्गायेति॑ सुवः{-}गाय॑   ।   वै   ।   लो॒काय॑   ।   दे॒व॒र॒थ इति॑ देव{-}र॒थः   ।    & TS\_5.4.10.1       \\
    
    \hline
        
    2015 & सु॒शर्मेति॑ सु{-}शर्मा᳚   ।   अ॒सि॒   ।   सु॒प्र॒ति॒ष्ठा॒न इति॑ सु{-}प्र॒ति॒ष्ठा॒नः   ।   बृ॒हत्   ।    & TS\_1.4.26.1       \\
    
    \hline
        
    2016 & सु॒हि॒र॒ण्ये इति॑ सु{-}हि॒र॒ण्ये   ।   सु॒शि॒ल्पे इति॑ सु{-}शि॒ल्पे   ।   ऋ॒तस्य॑   ।   योनौ᳚   ।    & TS\_5.1.11.3       \\
    
    \hline
        
    2017 & सूर्यः॑   ।   इति॑   ।   आ॒ह॒   ।   सशु॑क्राणा॒मिति॒ स{-}शु॒क्रा॒णा॒म्   ।    & TS\_6.4.2.5       \\
    
    \hline
        
    2018 & सूर्यः॑   ।   दे॒वः   ।   दि॒वि॒षद्भ्य॒ इति॑ दिवि॒षत्{-}भ्यः॒   ।   धा॒ता   ।    & TS\_3.3.10.1       \\
    
    \hline
        
    2019 & सूर्यः॑   ।   मा॒   ।   दे॒वः   ।   दे॒वेभ्यः॑   ।    & TS\_3.5.5.1       \\
    
    \hline
        
    2020 & सू॒क्तमिति॑ सु{-}उ॒क्तम्   ।   भ॒व॒ति॒   ।   ए॒तेन॑   ।   वै   ।    & TS\_5.2.3.4       \\
    
    \hline
        
    2021 & सू॒क्त॒वाच॒ इति॑ सूक्त{-}वाचः॑   ।   पृथि॑वि   ।   मा॒तः॒   ।   मा   ।    & TS\_3.3.2.2       \\
    
    \hline
        
    2022 & सू॒द॒य॒न्तु॒   ।   ते॒   ।   पृ॒थि॒वी   ।   ते॒   ।    & TS\_5.2.12.2       \\
    
    \hline
        
    2023 & सू॒यते᳚   ।   वै   ।   ए॒षः   ।   अ॒ग्नी॒नाम्   ।    & TS\_5.6.9.1       \\
    
    \hline
        
    2024 & सृ॒ज॒ति॒   ।   य॒ज्ञ्ः   ।   वै   ।   कृ॒ष्णा॒जि॒नमिति॑ कृष्ण{-}अ॒जि॒नम्   ।    & TS\_5.1.6.3       \\
    
    \hline
        
    2025 & सृ॒ज॒तु॒   ।   जी॒वात॑वे   ।   जी॒व॒न॒स्यायै᳚   ।   अ॒ग्नेः   ।    & TS\_2.3.10.3       \\
    
    \hline
        
    2026 & सृ॒ज॒ते॒   ।   यत्   ।   वै   ।   दी॒क्षि॒तः   ।    & TS\_3.1.1.2       \\
    
    \hline
        
    2027 & सेतु॑ना   ।   अतीति॑   ।   य॒न्ति॒   ।   अ॒न्यम्   ।    & TS\_3.2.2.2       \\
    
    \hline
        
    2028 & से॒दि॒म॒   ।   अग्ने᳚   ।   स॒प॒त्न॒दम्भ॑न॒मिति॑ सपत्न{-}दम्भ॑नम्   ।   अद॑ब्धासः   ।    & TS\_1.1.10.2       \\
    
    \hline
        
    2029 & से॒न्द्र॒त्वायेति॑ सेन्द्र{-}त्वाय॑   ।   अ॒ग्नि॒हो॒त्रो॒च्छे॒ष॒णमित्य॑ग्निहोत्र{-}उ॒च्छे॒ष॒णम्   ।   अ॒भ्यात॑न॒क्तीत्य॑भि{-}आत॑नक्ति   ।   य॒ज्ञ्स्य॑   ।    & TS\_2.5.3.6       \\
    
    \hline
        
    2030 & सोमः॑   ।   अधि॑पति॒रित्यधि॑{-}प॒तिः॒   ।   स्व॒ज इति॑ स्व{-}जः   ।   अ॒व॒स्थावेत्य॑व{-}स्थावा᳚   ।    & TS\_5.5.10.2       \\
    
    \hline
        
    2031 & सोमः॑   ।   त॒नूभिः॑   ।   रु॒द्रिया॑भिः   ।   समिति॑   ।    & TS\_2.1.11.3       \\
    
    \hline
        
    2032 & सोमः॑   ।   पि॒ब॒तु॒   ।   यत्   ।   ते॒   ।    & TS\_1.4.1.2       \\
    
    \hline
        
    2033 & सोमः॑   ।   यत्   ।   भि॒न्दू॒नाम्   ।   भ॒क्षये᳚त्   ।    & TS\_6.6.3.5       \\
    
    \hline
        
    2034 & सोमः॑   ।   वै   ।   स॒हस्र᳚म्   ।   अ॒वि॒न्द॒त्   ।    & TS\_7.1.6.1       \\
    
    \hline
        
    2035 & सोमा॑य   ।   पि॒तृ॒मत॒ इति॑ पितृ{-}मते᳚   ।   पु॒रो॒डाश᳚म्   ।   षट्क॑पाल॒मिति॒ षट्{-}क॒पा॒ल॒म्   ।    & TS\_1.8.5.1       \\
    
    \hline
        
    2036 & सोमा॑य   ।   स्व॒राज्ञ्॒ इति॑ स्व{-}राज्ञे᳚   ।   अ॒नो॒वा॒हावित्य॑नः{-}वा॒हौ   ।   अ॒न॒ड्वाहौ᳚   ।    & TS\_5.6.21.1       \\
    
    \hline
        
    2037 & सोम॑स्य   ।   त्विषिः॑   ।   अ॒सि॒   ।   तव॑   ।    & TS\_1.8.14.1       \\
    
    \hline
        
    2038 & सोम᳚म्   ।   क्री॒णी॒यात्   ।   तत्   ।   अ॒भी॒षहेत्य॑भ{-}सहा᳚   ।    & TS\_6.1.10.4       \\
    
    \hline
        
    2039 & सोम᳚म्   ।   ते॒   ।   क्री॒णा॒मि॒   ।   ऊर्ज॑स्वन्तम्   ।    & TS\_1.2.7.1       \\
    
    \hline
        
    2040 & सो॒तुः   ।   बा॒हुभ्या॒मिति॑ बा॒हु{-}भ्या॒म्   ।   सुय॑त॒ इति॒ सु{-}य॒तः॒   ।   न   ।    & TS\_2.4.14.4       \\
    
    \hline
        
    2041 & सो॒म॒   ।   एति॑   ।   भूयः॑   ।   भ॒र॒   ।    & TS\_6.1.4.8       \\
    
    \hline
        
    2042 & सो॒म॒पी॒थ इति॑ सोम{-}पी॒थः   ।   समृ॑द्ध्या॒ इति॒ सं{-}ऋ॒द्ध्यै॒   ।   ब्रा॒ह्म॒ण॒स्प॒त्यमिति॑ ब्राह्मणः{-}प॒त्यम्   ।   तू॒प॒रम्   ।    & TS\_2.1.5.7       \\
    
    \hline
        
    2043 & सौ॒म्यया᳚   ।   ए॒व   ।   आहु॒त्येत्या{-}हु॒त्या॒   ।   दि॒वः   ।    & TS\_2.4.9.3       \\
    
    \hline
        
    2044 & सौ॒म्याः   ।   ताः   ।   अ॒ना॒य॒त॒ना इत्य॑ना{-}य॒त॒नाः   ।   ऐ॒न्द्र॒वा॒य॒वमित्यै᳚न्द्र{-}वा॒य॒वम्   ।    & TS\_3.1.9.3       \\
    
    \hline
        
    2045 & सौ॒म्याः   ।   त्रयः॑   ।   पि॒शङ्गाः᳚   ।   सोमा॑य   ।    & TS\_5.6.19.1       \\
    
    \hline
        
    2046 & सौ॒म्येन॑   ।   द॒धा॒ति॒   ।   प्रेति॑   ।   ज॒न॒य॒ति॒   ।    & TS\_6.6.5.2       \\
    
    \hline
        
    2047 & सौ॒री   ।   ब॒लाका᳚   ।   ऋश्यः॑   ।   म॒यूरः॑   ।    & TS\_5.5.16.1       \\
    
    \hline
        
    2048 & स्ता॒म्   ।   इति॑   ।   आ॒ह॒   ।   आ॒शिष॒मित्या᳚{-}शिष᳚म्   ।    & TS\_2.6.9.6       \\
    
    \hline
        
    2049 & स्तु॒व॒न्ति॒   ।   अनु॑श्लोके॒नेत्यनु॑{-}श्लो॒के॒न॒   ।   प॒श्चात्   ।   य॒ज्ञ्स्य॑   ।    & TS\_7.5.8.2       \\
    
    \hline
        
    2050 & स्ते॒गान्   ।   दꣳष्ट्रा᳚भ्याम्   ।   म॒ण्डूकान्॑   ।   जंभ्ये॑भिः   ।    & TS\_5.7.11.1       \\
    
    \hline
        
    2051 & स्तोम॑भागा॒ इति॒ स्तोम॑{-}भा॒गाः॒   ।   उ॒प॒दधा॒तीत्यु॑प{-}दधा॑ति   ।   प्र॒जा इति॑ प्र{-}जाः   ।   ए॒व   ।    & TS\_5.3.5.5       \\
    
    \hline
        
    2052 & स्तोम᳚म्   ।   च॒   ।   अ॒ग्नये᳚   ।   वर्.षि॑ष्ठाय   ।    & TS\_4.4.4.4       \\
    
    \hline
        
    2053 & स्त्री   ।   अ॒स्य॒   ।   जा॒ये॒त॒   ।   ऊ॒र्द्ध्वम्   ।    & TS\_2.6.5.5       \\
    
    \hline
        
    2054 & स्थ॒   ।   तेषा᳚म्   ।   वः॒   ।   द॒क्षि॒णा   ।    & TS\_5.5.10.4       \\
    
    \hline
        
    2055 & स्थ॒न॒   ।   यत्   ।   आ॒मय॑ति   ।   निरिति॑   ।    & TS\_4.2.6.3       \\
    
    \hline
        
    2056 & स्फ्यः   ।   स्व॒स्तिः   ।   वि॒घ॒न इति॑ वि{-}घ॒नः   ।   स्व॒स्तिः   ।    & TS\_3.2.4.1       \\
    
    \hline
        
    2057 & स्फ्येन॑   ।   वेदि᳚म्   ।   उदिति॑   ।   ह॒न्ति॒   ।    & TS\_6.6.4.1       \\
    
    \hline
        
    2058 & स्या॒त्   ।   इति॑   ।   नीचा᳚   ।   हस्ते॑न   ।    & TS\_6.4.5.6       \\
    
    \hline
        
    2059 & स्या॒त्   ।   इति॑   ।   स॒तं॒रामिति॑ सं{-}त॒राम्   ।   तस्य॑   ।    & TS\_5.7.10.3       \\
    
    \hline
        
    2060 & स्या॒त्   ।   रक्षाꣳ॑सि   ।   य॒ज्ञ्म्   ।   ह॒न्युः॒   ।    & TS\_5.1.3.2       \\
    
    \hline
        
    2061 & स्युः   ।   ते   ।   ए॒क॒विꣳ॒॒श॒ति॒रा॒त्रमित्ये॑कविꣳशति{-}रा॒त्रम्   ।   आ॒सी॒र॒न्न्   ।    & TS\_7.3.10.5       \\
    
    \hline
        
    2062 & स्युः   ।   ते   ।   वीति॑   ।   ब्रू॒युः॒   ।    & TS\_7.3.1.3       \\
    
    \hline
        
    2063 & स्वम्   ।   च॒म॒सः   ।   अ॒स्य॒   ।   स्वम्   ।    & TS\_3.1.2.4       \\
    
    \hline
        
    2064 & स्वान्   ।   अ॒हम्   ।   दृ॒शा॒नः   ।   रु॒क्मः   ।    & TS\_4.1.10.4       \\
    
    \hline
        
    2065 & स्वाहा᳚   ।   अ॒नु॒वर्.ष॑त॒ इत्य॑नु{-}वर्.ष॑ते   ।   स्वाहा᳚   ।   शी॒का॒यि॒ष्य॒ते   ।    & TS\_7.5.11.2       \\
    
    \hline
        
    2066 & स्वाहा᳚   ।   आ॒धिमित्या᳚{-}धिम्   ।   आधी॑ता॒येत्या{-}धी॒ता॒य॒   ।   स्वाहा᳚   ।    & TS\_7.3.15.1       \\
    
    \hline
        
    2067 & स्वाहा᳚   ।   इति॑   ।   आ॒ह॒   ।   आकू॒त्येत्या{-}कू॒त्या॒   ।    & TS\_6.1.2.2       \\
    
    \hline
        
    2068 & स्वाहा᳚   ।   ब॒हु   ।   ह॒   ।   अ॒यम्   ।    & TS\_2.4.7.2       \\
    
    \hline
        
    2069 & स्वा॒द्वीम्   ।   त्वा॒   ।   स्वा॒दुना᳚   ।   ती॒व्राम्   ।    & TS\_1.8.21.1       \\
    
    \hline
        
    2070 & स्वेन॑   ।   भा॒ग॒धेये॒नेति॑ भाग{-}धेये॑न   ।   उपेति॑   ।   धा॒व॒ति॒   ।    & TS\_2.1.6.3 TS\_2.2.1.5 TS\_2.3.1.2       \\
    
    \hline
        
    2071 & स्व॒धाया॒ इति॑ स्व{-}धायै᳚   ।   नमः॑   ।   वः॒   ।   पि॒त॒रः॒   ।    & TS\_3.2.5.6       \\
    
    \hline
        
    2072 & स्व॒य॒मा॒तृ॒ण्णामिति॑ स्वयं{-}आ॒तृ॒ण्णाम्   ।   उपेति॑   ।   द॒धा॒ति॒   ।   इ॒यम्   ।    & TS\_5.2.8.1       \\
    
    \hline
        
    2073 & स॒गंच्छ॑त॒ इति॑ सं{-}गच्छ॑ते   ।   वाज᳚म्   ।   ए॒व   ।   अ॒स्य॒   ।    & TS\_5.1.2.5       \\
    
    \hline
        
    2074 & स॒ग्रां॒ममिति॑ सं {-}ग्रा॒मम्   ।   ए॒ताम्   ।   ए॒व   ।   निरिति॑   ।    & TS\_2.2.8.3       \\
    
    \hline
        
    2075 & स॒जा॒ता इति॑ स{-}जा॒ताः   ।   विश्वान्॑   ।   ए॒व   ।   दे॒वान्   ।    & TS\_2.1.7.6       \\
    
    \hline
        
    2076 & स॒जा॒तानिति॑ स{-}जा॒तान्   ।   प्रेति॑   ।   य॒च्छ॒न्ति॒   ।   ग्रा॒मी   ।    & TS\_2.1.6.5       \\
    
    \hline
        
    2077 & स॒जित्वा॑न॒मिति॑ स{-}जित्वा॑नम्   ।   स॒दा॒सह॒मिति॑ सदा{-}सह᳚म्   ।   वर्.षि॑ष्ठम्   ।   ऊ॒तये᳚   ।    & TS\_3.4.11.4       \\
    
    \hline
        
    2078 & स॒जूरिति॑ स{-}जूः   ।   अब्दः॑   ।   अया॑वभि॒रित्यया॑व{-}भिः॒   ।   स॒जूरिति॑ स{-}जूः   ।    & TS\_5.6.4.1       \\
    
    \hline
        
    2079 & स॒जोषा॒ इति॑ स{-}जोषाः᳚   ।   इ॒न्द्र॒   ।   सग॑ण॒ इति॒ स{-}ग॒णः॒   ।   म॒रुद्भि॒रिति॑ म॒रुत्{-}भिः॒   ।    & TS\_1.4.42.1       \\
    
    \hline
        
    2080 & स॒त्यम्   ।   अ॒नयोः᳚   ।   ए॒व   ।   ए॒न॒म्   ।    & TS\_5.1.5.9       \\
    
    \hline
        
    2081 & स॒द्यः   ।   दी॒क्ष॒य॒न्ति॒   ।   स॒द्यः   ।   सोम᳚म्   ।    & TS\_1.8.18.1       \\
    
    \hline
        
    2082 & स॒न्ति॒   ।   ते   ।   ए॒षु   ।   लो॒केषु॑   ।    & TS\_3.5.4.3       \\
    
    \hline
        
    2083 & स॒न्तु॒   ।   दु॒र्मि॒त्रा इति॑ दुः{-}मि॒त्राः   ।   तस्मै᳚   ।   भू॒या॒सुः॒   ।    & TS\_1.4.45.3       \\
    
    \hline
        
    2084 & स॒न्नाः   ।   अ॒स॒न्न्   ।   इति॑   ।   उ॒प॒या॒मगृ॑हीत॒ इत्यु॑पया॒म{-}गृ॒ही॒तः॒   ।    & TS\_6.4.7.3       \\
    
    \hline
        
    2085 & स॒प्तभि॒रिति॑ स॒प्त{-}भिः॒   ।   धू॒प॒य॒ति॒   ।   स॒प्त   ।   वै   ।    & TS\_5.1.7.1       \\
    
    \hline
        
    2086 & स॒प्ता॒नाम्   ।   गि॒री॒णाम्   ।   प॒रस्ता᳚त्   ।   वि॒त्तम्   ।    & TS\_6.2.4.3       \\
    
    \hline
        
    2087 & स॒भृंत्येति॑ सं{-}भृत्य॑   ।   तेजः॑   ।   आ॒त्मन्न्   ।   द॒ध॒ते॒   ।    & TS\_7.5.8.5       \\
    
    \hline
        
    2088 & स॒मा॒न्यः॑   ।   ऋचः॑   ।   भ॒व॒न्ति॒   ।   म॒नु॒ष्य॒लो॒क इति॑ मनुष्य{-}लो॒कः   ।    & TS\_7.5.4.1       \\
    
    \hline
        
    2089 & स॒मिदिति॑ सम्{-}इत्   ।   दि॒शाम्   ।   आ॒शया᳚   ।   नः॒   ।    & TS\_4.4.12.1       \\
    
    \hline
        
    2090 & स॒मिध॒ इति॑ सं{-}इधः॑   ।   य॒ज॒ति॒   ।   व॒स॒न्तम्   ।   ए॒व   ।    & TS\_2.6.1.1       \\
    
    \hline
        
    2091 & स॒मिध॒मिति॑ सं{-}इध᳚म्   ।   एति॑   ।   ति॒ष्ठ॒   ।   गा॒य॒त्री   ।    & TS\_1.8.13.1       \\
    
    \hline
        
    2092 & स॒मिध॒मिति॑ सं{-}इध᳚म्   ।   य॒क्षि॒   ।   अ॒ग्ने॒   ।   प्रतीति॑   ।    & TS\_1.4.45.2       \\
    
    \hline
        
    2093 & स॒मि॒ष्ट॒य॒जूꣳषीति॑ समिष्ट{-}य॒जूꣳषि॑   ।   जु॒हो॒ति॒   ।   य॒ज्ञ्स्य॑   ।   समि॑ष्ट्या॒ इति॒ सं{-}इ॒ष्ट्यै॒   ।    & TS\_6.6.2.1       \\
    
    \hline
        
    2094 & स॒मीची᳚   ।   नाम॑   ।   अ॒सि॒   ।   प्राची᳚   ।    & TS\_5.5.10.1       \\
    
    \hline
        
    2095 & स॒मु॒द्रः   ।   उ॒दर᳚म्   ।   अ॒न्तरि॑क्षम्   ।   पा॒युः   ।    & TS\_7.5.25.2       \\
    
    \hline
        
    2096 & स॒मु॒द्रम्   ।   ग॒च्छ॒   ।   स्वाहा᳚   ।   अ॒न्तरि॑क्षम्   ।    & TS\_1.3.11.1       \\
    
    \hline
        
    2097 & स॒मु॒द्रिय᳚म्   ।   इति॑   ।   आ॒ह॒   ।   अ॒पाम्   ।    & TS\_5.1.5.8       \\
    
    \hline
        
    2098 & स॒वि॒तुः   ।   वरे᳚ण्यस्य   ।   चि॒त्राम्   ।   इति॑   ।    & TS\_5.4.7.5       \\
    
    \hline
        
    2099 & स॒व्यम्   ।   हि   ।   पूर्व᳚म्   ।   म॒नु॒ष्याः᳚   ।    & TS\_6.1.1.6       \\
    
    \hline
        
    2100 & स॒हर्.ष॒भेति॑ स॒ह{-}ऋ॒ष॒भा॒   ।   इति॑   ।   आ॒ह॒   ।   मि॒थु॒नम्   ।    & TS\_2.6.7.3       \\
    
    \hline
        
    2101 & स॒हस्रा॑णि   ।   स॒ह॒स्र॒श इति॑ सहस्र{-}शः   ।   ये   ।   रु॒द्राः   ।    & TS\_4.5.11.1       \\
    
    \hline
        
    2102 & स॒ह॒जानिति॑ सह{-}जान्   ।   विश्वे॑भिः   ।   दे॒वेभिः॑   ।   पृत॑नाः   ।    & TS\_3.5.3.2       \\
    
    \hline
        
    2103 & स॒ह॒सा॒व॒न्निति॑ सहसा{-}व॒न्न्   ।   परि॑ष्टौ   ।   अ॒घाय॑   ।   भू॒म॒   ।    & TS\_1.6.12.6       \\
    
    \hline
        
    2104 & स॒ह॒स्र॒त॒म्येति॑ सहस्र{-}त॒म्या᳚   ।   वै   ।   यज॑मानः   ।   सु॒व॒र्गमिति॑ सुवः{-}गम्   ।    & TS\_7.1.7.1       \\
    
    \hline
        
    2105 & सꣳश्र॑वा॒ इति॒ सं{-}श्र॒वाः॒   ।   ह॒   ।   सौ॒व॒र्च॒न॒सः   ।   तुमि॑ञ्जम्   ।    & TS\_1.7.2.1       \\
    
    \hline
        
    2106 & सꣳस्तु॑ता॒ इति॒ सं{-}स्तु॒ताः॒   ।   वि॒राज॒मिति॑ वि{-}राज᳚म्   ।   अ॒भि   ।   समिति॑   ।    & TS\_7.4.10.2       \\
    
    \hline
        
    2107 & सꣳ॒॒सृज्येति॑ सं{-}सृज्य॑   ।   पृ॒थि॒वीम्   ।   भूमि᳚म्   ।   च॒   ।    & TS\_4.1.5.2       \\
    
    \hline
        
    2108 & सꣳ॒॒हि॒तेति॑ सं{-}हि॒ता   ।   अ॒सि॒   ।   वि॒श्व॒रू॒पीरिति॑ विश्व{-}रू॒पीः   ।   एति॑   ।    & TS\_1.5.6.2       \\
    
    \hline
        
    2109 & हरिः॑   ।   अ॒सि॒   ।   हा॒रि॒यो॒ज॒न इति॑ हारि{-}यो॒ज॒नः   ।   हर्योः᳚   ।    & TS\_1.4.28.1       \\
    
    \hline
        
    2110 & हर्य॑तम्   ।   वृ॒ष॒णा॒   ।   जु॒षेथा᳚म्   ।   सु॒शर्मा॒णेति॑ सु{-}शर्मा॑णा   ।    & TS\_2.3.14.3       \\
    
    \hline
        
    2111 & हस्त॑योः   ।   त॒प॒स्वित॑र॒ इति॑ तप॒स्वि{-}त॒रः॒   ।   योनिः॑   ।   च॒तु॒र्विꣳ॒॒श इति॑ चतुः{-}विꣳ॒॒शः   ।    & TS\_5.3.3.4       \\
    
    \hline
        
    2112 & हि   ।   अश्वा᳚त्   ।   ग॒र्द॒भः   ।   अश्व᳚म्   ।    & TS\_5.1.2.3       \\
    
    \hline
        
    2113 & हि   ।   अ॒स्मा॒त्   ।   नि॒र्गृ॒ह्णातीति॑ निः{-}गृ॒ह्णाति॑   ।   चक्षुः॑   ।    & TS\_6.5.1.4       \\
    
    \hline
        
    2114 & हि   ।   ए॒तत्   ।   यत्   ।   निखा॑त॒मिति॒ नि{-}खा॒त॒म्   ।    & TS\_6.2.10.4       \\
    
    \hline
        
    2115 & हि   ।   ए॒तौ   ।   उ॒र्वशी᳚   ।   अ॒सि॒   ।    & TS\_6.3.5.3       \\
    
    \hline
        
    2116 & हि   ।   ए॒षः   ।   सन्न्   ।   मर्त्ये॑षु   ।    & TS\_6.1.4.7       \\
    
    \hline
        
    2117 & हि   ।   तत्   ।   वेद॑   ।   यतः॑   ।    & TS\_6.5.3.2       \\
    
    \hline
        
    2118 & हि   ।   तम्   ।   अन्विति॑   ।   अ॒वि॒न्द॒न्न्   ।    & TS\_2.6.6.2       \\
    
    \hline
        
    2119 & हिमाः᳚   ।   इति॑   ।   आ॒ह॒   ।   श॒तायु॒रिति॑ श॒त{-}आ॒युः॒   ।    & TS\_1.5.7.6       \\
    
    \hline
        
    2120 & हिर॑ण्यपाणि॒मिति॒ हिर॑ण्य{-}पा॒णि॒म्   ।   ऊ॒तये᳚   ।   स॒वि॒तार᳚म्   ।   उपेति॑   ।    & TS\_1.4.25.1       \\
    
    \hline
        
    2121 & हिर॑ण्यम्   ।   उ॒पास्येत्यु॑प{-}अस्य॑   ।   जु॒हो॒ति॒   ।   अ॒ग्नि॒वतीत्य॑ग्नि{-}वति॑   ।    & TS\_6.2.9.3       \\
    
    \hline
        
    2122 & हिर॑ण्यवर्णा॒ इति॒ हिर॑ण्य{-}व॒र्णाः॒   ।   शुच॑यः   ।   पा॒व॒काः   ।   यासु॑   ।    & TS\_5.6.1.1       \\
    
    \hline
        
    2123 & हि॒नो॒मि॒   ।   इन्द्रा॑विष्णू॒ इतीन्द्रा᳚{-}वि॒ष्णू॒   ।   अप॑सः   ।   पा॒रे   ।    & TS\_3.2.11.2       \\
    
    \hline
        
    2124 & हि॒र॒ण्ये॒ष्ट॒का इति॑ हिरण्य{-}इ॒ष्ट॒काः   ।   उ॒प॒दधा॒तीत्यु॑प{-}दधा॑ति   ।   इ॒मान्   ।   ए॒व   ।    & TS\_5.7.6.3       \\
    
    \hline
        
    2125 & हि॒र॒ण्य॒ग॒र्भ इति॑ हिरण्य{-}ग॒र्भः   ।   आपः॑   ।   ह॒   ।   यत्   ।    & TS\_2.2.12.1       \\
    
    \hline
        
    2126 & हिꣳ॒॒सि॒ष॒म्   ।   उ॒रु   ।   वाता॑य   ।   दे॒वस्य॑   ।    & TS\_1.1.4.2       \\
    
    \hline
        
    2127 & हिꣳ॒॒सीः॒   ।   प॒र॒मे   ।   व्यो॑म॒न्निति॒ वि{-}ओ॒म॒न्न्   ।   ग॒व॒यम्   ।    & TS\_4.2.10.3       \\
    
    \hline
        
    2128 & ही॒य॒ते॒   ।   तस्मा᳚त्   ।   द॒श॒मे   ।   अहन्न्॑   ।    & TS\_7.3.1.2       \\
    
    \hline
        
    2129 & हु॒त्वा   ।   उदिति॑   ।   गृ॒ह्णा॒ति॒   ।   सु॒व॒र्गमिति॑ सुवः{-}गम्   ।    & TS\_6.6.1.2       \\
    
    \hline
        
    2130 & हु॒वे॒   ।   तु॒राणा᳚म्   ।   एति॑   ।   यत्   ।    & TS\_2.1.11.2       \\
    
    \hline
        
    2131 & हृ॒दे   ।   त्वा॒   ।   मन॑से   ।   त्वा॒   ।    & TS\_1.3.13.1       \\
    
    \hline
        
    2132 & होता᳚   ।   स्वे   ।   ए॒व   ।   अ॒स्मै॒   ।    & TS\_5.4.11.4       \\
    
    \hline
        
    2133 & होत्राः᳚   ।   अभू᳚त्   ।   इ॒दम्   ।   विश्व॑स्य   ।    & TS\_4.2.9.6       \\
    
    \hline
        
    2134 & हो॒तः॒   ।   दो॒षा   ।   वस्तोः᳚   ।   एति॑   ।    & TS\_1.3.14.3       \\
    
    \hline
        
    2135 & ह्रि॒ये॒त॒   ।   इति॑   ।   तानि॑   ।   ए॒व   ।    & TS\_2.2.4.8       \\
    
    \hline
        
    2136 & ह॒तः   ।   षो॒ड॒शभि॒रिति॑ षोड॒श{-}भिः॒   ।   भो॒गैः   ।   अ॒सि॒ना॒त्   ।    & TS\_5.4.5.4       \\
    
    \hline
        
    2137 & ह॒र॒ति॒   ।   अ॒पाम्   ।   वै   ।   ए॒तत्   ।    & TS\_5.1.4.2       \\
    
    \hline
        
    2138 & ह॒विष्म॑तीः   ।   इ॒माः   ।   आपः॑   ।   ह॒विष्मान्॑   ।    & TS\_1.3.12.1       \\
    
    \hline
        \bottomrule
  \end{longtable}
  
\end{document}