\documentclass[17pt]{extarticle}
\usepackage{babel}
\usepackage{fontspec}
\usepackage{polyglossia}
\usepackage{extsizes}

\usepackage{booktabs} % To thicken table lines



\setmainlanguage{sanskrit}
\setotherlanguages{english} %% or other languages
\setlength{\parindent}{0pt}
\pagestyle{myheadings}
\newfontfamily\devanagarifont[Script=Devanagari]{AdishilaVedic}


\newcommand{\VAR}[1]{}
\newcommand{\BLOCK}[1]{}

\usepackage{longtable} % To display tables on several pages

\begin{document} 


\begin{longtable}{||p{0.4in}||p{4.9in}||p{0.9in}||} % <-- Replaces \begin{table}, alignment must be specified here (no more tabular)
    \caption{ कृष्ण यजुर्वेदीय तैत्तिरीय आरण्यके}
    \label{tab:table1}\\
    \toprule
    \textbf{SNo} & \textbf{Beginning Words} & \textbf{Dasini} 
    
   
    \endfirsthead % <-- This denotes the end of the header, which will be shown on the first page only
    \toprule
    \textbf{SNo} & \textbf{Beginning Words} & \textbf{Dasini} 
    
   
    \endhead % <-- Everything between \endfirsthead and \endhead will be shown as a header on every page
        
    1 & अंभस्य पारे भुवनस्य मद्ध्ये & T.A.6.1.1       \\
    
    \hline
        
    2 & अक्षिदुःखोत्थितस्यैव विप्रसन्ने कनीनिके आङ्क्ते & T.A.1.4.1       \\
    
    \hline
        
    3 & अग्नये स्वाहा विश्वेभ्यो देवेभ्यः & T.A.6.67.1       \\
    
    \hline
        
    4 & अग्नयो वायवश्चैव एतदस्य परायणं & T.A.1.8.4       \\
    
    \hline
        
    5 & अग्निं प्रणीयोप{-}समाधाय तमभित एता & T.A.1.26.1       \\
    
    \hline
        
    6 & अग्निरसि वैश्वानरोऽसि सम्ॅवथ्सरोऽसि परिवथ्सरोऽसि & T.A.7.19.1       \\
    
    \hline
        
    7 & अग्निर् यजुर्भिः सविता स्तोमैः & T.A.3.8.1       \\
    
    \hline
        
    8 & अग्निर्. होता अश्विनाऽद्ध्वर्यू त्वष्टाऽग्नीत् & T.A.3.3.1       \\
    
    \hline
        
    9 & अग्निर्वायुश्च सूर्यश्च सह सञ्च{-}स्करर्द्धिया & T.A.1.1.2       \\
    
    \hline
        
    10 & अग्निश्च जातवेदाश्च सहोजा अजिराप्रभुः & T.A.1.9.1       \\
    
    \hline
        
    11 & अग्निश्च मा मन्युश्च मन्युपतयश्च & T.A.6.31.1       \\
    
    \hline
        
    12 & अग्निष्टोमे प्रवृणक्ति एतावान्. वै & T.A.8.6.3       \\
    
    \hline
        
    13 & अग्निष्ट्वा वसुभिः पुरस्ताद्{-}रोचयतु गायत्रेण & T.A.8.5.1       \\
    
    \hline
        
    14 & अघोरेभ्योऽथ घोरेभ्यो घोरघोरतरेभ्यः सर्वेभ्यः & T.A.6.19.1       \\
    
    \hline
        
    15 & अङ्गुष्ठमात्रः पुरुषोऽङ्गुष्ठञ्च समाश्रितः ईशः & T.A.6.71.1       \\
    
    \hline
        
    16 & अच्युतां बहुलाꣳ श्रियम् स & T.A.3.11.7       \\
    
    \hline
        
    17 & अजान्. ह वै पृश्नीꣳस्तपस्यमानान् & T.A.2.9.1       \\
    
    \hline
        
    18 & अञ्जलिना पुरस्ता{-}दुपादधात् एवा ह्येवेति & T.A.1.23.5       \\
    
    \hline
        
    19 & अणुभिश्च महद्भिश्च समारूढः प्रदृश्यते & T.A.1.2.3       \\
    
    \hline
        
    20 & अणो{-}रणीयान् महतो महीया{-}नात्मा गुहायां & T.A.6.12.1       \\
    
    \hline
        
    21 & अति ताम्राणि वासाꣳसि अष्टिवज्रि & T.A.1.5.1       \\
    
    \hline
        
    22 & अत्यूर्द्ध्वाक्षोऽतिरश्चात् शिशिरः प्रदृश्यते नैव & T.A.1.6.1       \\
    
    \hline
        
    23 & अत्रिणा त्वा क्रिमे हन्मि & T.A.7.36.1       \\
    
    \hline
        
    24 & अथ वायो{-}रेकादश{-}पुरुषस्यैकादश{-}स्त्रीकस्य प्रभ्राजमानानां रुद्राणां & T.A.1.17.1       \\
    
    \hline
        
    25 & अथाग्नेरष्ट पुरुषस्य अग्नेः पूर्व{-}दिश्यस्य & T.A.1.18.1       \\
    
    \hline
        
    26 & अथादित्यस्याष्ट पुरुषस्य वसूना मादित्यानां & T.A.1.15.1       \\
    
    \hline
        
    27 & अथाद्ध्यात्मं अधराहनुः पूर्व रूपं & T.A.5.3.4       \\
    
    \hline
        
    28 & अथारुणः केतुरुपरिष्टा{-}दुपादधात् एवा हि & T.A.1.23.7       \\
    
    \hline
        
    29 & अथो आहुः सर्वेषु यज्ञ्क्रतुष्विति & T.A.1.22.10       \\
    
    \hline
        
    30 & अथो रक्षसामपहत्यै त्रिः पुनः & T.A.8.4.12       \\
    
    \hline
        
    31 & अदितिर्{-}देवा गन्धर्वा मनुष्याः पितरो{-}ऽसुरा{-}स्तेषां & T.A.6.28.1       \\
    
    \hline
        
    32 & अदित्या उष्णीषमसि वायुरस्यैडः पूषा & T.A.7.8.2       \\
    
    \hline
        
    33 & अद्ध्वरकृद्{-}देवेभ्य इत्याह यज्ञो वा & T.A.8.2.6       \\
    
    \hline
        
    34 & अद्भय{-}स्तिरोधा जायत तव वैश्रवणः & T.A.1.31.1       \\
    
    \hline
        
    35 & अद्भ्यः सम्भूतः पृथिव्यै रसाच्च & T.A.3.13.1       \\
    
    \hline
        
    36 & अधीयन्तोऽवेक्षन्ते सर्वमायुर्यन्ति न पत्न्यवेक्षेत & T.A.8.6.12       \\
    
    \hline
        
    37 & अनशनायुको भवति य एवं & T.A.8.9.8       \\
    
    \hline
        
    38 & अनाधृष्या पुरस्तात् अग्नेराधिपत्ये आयुर्मे & T.A.7.5.3       \\
    
    \hline
        
    39 & अनाधृष्या पुरस्तादिति यदेतानि यजूꣳष्याह & T.A.8.4.7       \\
    
    \hline
        
    40 & अनाभोगाः परं मृत्युं पापाः & T.A.1.8.5       \\
    
    \hline
        
    41 & अनु न जातमष्ट रोदसी & T.A.1.7.6       \\
    
    \hline
        
    42 & अन्तः प्रविष्टः शास्ता जनानां & T.A.3.11.2       \\
    
    \hline
        
    43 & अन्तरिक्षप्र उरोर्वरीयान् अशीमहि त्वा & T.A.7.7.5       \\
    
    \hline
        
    44 & अन्तरिक्षेण त्वोपयच्छामि देवानां त्वा & T.A.7.8.5       \\
    
    \hline
        
    45 & अन्तरिक्षेण त्वोपयच्छामीत्याह अन्तरिक्षेणैवैन{-}मुपयच्छति न & T.A.8.7.8       \\
    
    \hline
        
    46 & अन्तर् बहिश्च तथ् सर्वं & T.A.6.13.2       \\
    
    \hline
        
    47 & अन्तेवास्युत्तररूपं विद्या सन्धिः प्रवचनं & T.A.5.3.3       \\
    
    \hline
        
    48 & अन्नं न निन्द्यात् तद् & T.A.5.15.7       \\
    
    \hline
        
    49 & अन्नं न परिचक्षीत तद् & T.A.5.15.8       \\
    
    \hline
        
    50 & अन्नं बहु कुर्वीत तद् & T.A.5.15.9       \\
    
    \hline
        
    51 & अन्नं ब्रह्मेति व्यजानात् अन्नाद्ध्येव & T.A.5.15.2       \\
    
    \hline
        
    52 & अन्नाद्येन समिन्ताꣳ स्वाहा प्राजापत्या & T.A.7.41.7       \\
    
    \hline
        
    53 & अन्नाद्वै प्रजाः प्रजायन्ते याः & T.A.5.14.2       \\
    
    \hline
        
    54 & अन्वेति तुग्रो वक्रियान्तं आयसूयान्थ् & T.A.1.10.4       \\
    
    \hline
        
    55 & अप नः शोशुचदघमग्ने शुशुद्ध्या & T.A.4.11.1       \\
    
    \hline
        
    56 & अपरिमितम् करोति अपरिमितस्यावरुद्ध्यै परिग्रीवम् & T.A.8.3.5       \\
    
    \hline
        
    57 & अपश्यङ्गोपामनिपद्यमानम् आ च परा & T.A.7.7.1       \\
    
    \hline
        
    58 & अपश्याम युवति{-}माचरन्तीम् मृताय जीवाम् & T.A.4.12.1       \\
    
    \hline
        
    59 & अपाश्न्युष्णि{-}मपा रक्षः अपाश्न्युष्णि{-}मपा रघं & T.A.1.1.3       \\
    
    \hline
        
    60 & अपि ब्राह्मणमुखीनाः तस्मिन्नह्नः काले & T.A.1.31.5       \\
    
    \hline
        
    61 & अपूपवान् घृतवाꣳश्चरुरेह सीदतूत्तभ्नुवन् पृथिवीम् & T.A.4.8.1       \\
    
    \hline
        
    62 & अपैतु मृत्यु{-}रमृतन्न आगन् वैवस्वतो & T.A.6.45.1       \\
    
    \hline
        
    63 & अफ्सु वै वरुण आदित्यवान् & T.A.8.7.11       \\
    
    \hline
        
    64 & अभिधून्वन्तो{-}ऽभिघ्नन्त इव वातवन्तो मरुद्गणाः & T.A.1.4.2       \\
    
    \hline
        
    65 & अभिपूर्वम् प्रोक्षति अभिपूर्व{-}मेवास्मिन्{-}तेजो दधाति & T.A.8.4.2       \\
    
    \hline
        
    66 & अभ्राण्यपः प्रपद्यन्ते विद्युथ्सूर्ये समाहिता & T.A.1.8.2       \\
    
    \hline
        
    67 & अमुत्र भूयादध यद्यमस्य बृहस्पते & T.A.6.48.1       \\
    
    \hline
        
    68 & अमुत्रेतरे तस्मादिहा तप्त्रि तपाः & T.A.1.7.4       \\
    
    \hline
        
    69 & अमुष्य त्वा प्राणे सादयामि & T.A.7.10.3       \\
    
    \hline
        
    70 & अमूꣳश्च परिरक्षतः एता वाचः & T.A.1.3.3       \\
    
    \hline
        
    71 & अमृत एवैनम् प्रतिष्ठापयति वल्गुरसि & T.A.8.9.6       \\
    
    \hline
        
    72 & अमृतं ॅवा आपः अमृतस्या{-}नन्तरित्यै & T.A.1.26.7       \\
    
    \hline
        
    73 & अमृतस्य पूर्णान्तामु कलां ॅविचक्षते & T.A.3.11.5       \\
    
    \hline
        
    74 & अमृतस्य प्राणं ॅयज्ञ्मेतम् चतुर्.होतृणामात्मानं & T.A.3.11.3       \\
    
    \hline
        
    75 & अर्चिरसि शोचिरसीत्याह तेज एवास्मिन् & T.A.8.4.6       \\
    
    \hline
        
    76 & अव द्रफ्सो अꣳशमतीमतिष्ठत् इयानः & T.A.1.6.3       \\
    
    \hline
        
    77 & अव सृज पुनरग्ने पितृभ्यो & T.A.4.4.2       \\
    
    \hline
        
    78 & अवपतन्तानाꣳ रुद्राणाꣳ स्थाने स्वतेजसा & T.A.1.17.2       \\
    
    \hline
        
    79 & अश्विना घर्मम् पातꣳ हार्दिवान{-}महर्दिवाभि{-}रूतिभिरित्याह & T.A.8.8.2       \\
    
    \hline
        
    80 & अष्टयोनी{-}मष्टपुत्रां अष्टपत्नी{-}मिमां महीं अहं & T.A.1.13.1       \\
    
    \hline
        
    81 & असञ्चयवान् अग्नये वायवे सूर्याय & T.A.1.32.2       \\
    
    \hline
        
    82 & असद्वा इदमग्र आसीत् ततो & T.A.5.14.7       \\
    
    \hline
        
    83 & असन्नेव स भवति असद् & T.A.5.14.6       \\
    
    \hline
        
    84 & असाम सुमतौ यज्ञियस्य श्रियं & T.A.1.31.2       \\
    
    \hline
        
    85 & असौ खलु वावैष आदित्यः & T.A.8.11.5       \\
    
    \hline
        
    86 & असौ वै पल्वल्याः अमुष्यामेव & T.A.1.24.4       \\
    
    \hline
        
    87 & अस्कान् द्यौः पृथिवीम् अस्कानृषभो & T.A.7.13.1       \\
    
    \hline
        
    88 & अहं ॅवृक्षस्य रेरिवा कीर्तिः & T.A.5.10.1       \\
    
    \hline
        
    89 & अहं ॅवेद न मे & T.A.1.13.2       \\
    
    \hline
        
    90 & अहर् ज्योतिः केतुना जुषताम् & T.A.7.10.4       \\
    
    \hline
        
    91 & अहर्{-}दिवाभि{-}रूतिभिः अनु वान् द्यावापृथिवी & T.A.7.9.3       \\
    
    \hline
        
    92 & अहोरात्रे त्वोदीरयताम् अर्द्धमासास्त्वोदीं जयन्तु & T.A.7.26.1       \\
    
    \hline
        
    93 & आ प्यायस्व मदिन्तम सोम & T.A.3.17.1       \\
    
    \hline
        
    94 & आच्छृणत्ति देवत्राऽकः अजक्षीरेणाच्छृणत्ति परमं & T.A.8.3.9       \\
    
    \hline
        
    95 & आतनुष्व प्रतनुष्व उद्धमाधम सन्धम & T.A.1.12.1       \\
    
    \hline
        
    96 & आदित्यो वा एष एतन् & T.A.6.14.1       \\
    
    \hline
        
    97 & आदित्यो वै तेज ओजो & T.A.6.15.1       \\
    
    \hline
        
    98 & आनन्दो ब्रह्मेति व्यजानात् आनन्दा{-}द्ध्येव{-}खल्विमानि & T.A.5.15.6       \\
    
    \hline
        
    99 & आनुश्रविक एव नौ कश्यप & T.A.1.7.3       \\
    
    \hline
        
    100 & आपः पुनन्तु पृथिवीं पृथिवी & T.A.6.30.1       \\
    
    \hline
        
    101 & आपमापामपः सर्वाः अस्मा{-}दस्मादितोऽमुतः अग्निर्वायुश्च & T.A.1.21.1       \\
    
    \hline
        
    102 & आपो जनयथा च नः & T.A.7.42.5       \\
    
    \hline
        
    103 & आपो वा इदमासन्थ् सलिलमेव & T.A.1.23.1       \\
    
    \hline
        
    104 & आपो वा इदꣳ सर्वं & T.A.6.29.1       \\
    
    \hline
        
    105 & आपो वै वायोरायतनं आयतनवान् & T.A.1.22.3       \\
    
    \hline
        
    106 & आपो ह यद् बृहतीः & T.A.1.23.8       \\
    
    \hline
        
    107 & आमां मेधा सुरभिर् विश्वरूपा & T.A.6.43.1       \\
    
    \hline
        
    108 & आयतनवान् भवति आपो वा & T.A.1.22.2       \\
    
    \hline
        
    109 & आयतनवान् भवति य एवं & T.A.1.22.4       \\
    
    \hline
        
    110 & आयतनवान् भवति सम्ॅवथ्सरो वा & T.A.1.22.7       \\
    
    \hline
        
    111 & आयातु देवः सुमनाभि{-}रूतिभिर्{-}यमो हवेह & T.A.4.5.1       \\
    
    \hline
        
    112 & आयातु वरदा देवी अक्षरं & T.A.6.34.1       \\
    
    \hline
        
    113 & आयुर्{-}विश्वायुः परिपासति त्वा पूषा & T.A.4.1.2       \\
    
    \hline
        
    114 & आयुष्टे विश्वतो दधदयमग्निर्वरेण्यः पुनस्ते & T.A.2.5.1       \\
    
    \hline
        
    115 & आरण्याः पशवः कनीयाꣳसः शुचा & T.A.8.2.12       \\
    
    \hline
        
    116 & आरोगस्य स्थाने स्वतेजसा भानि & T.A.1.16.1       \\
    
    \hline
        
    117 & आरोगो भ्राजः पटरः पतङ्गः & T.A.1.7.1       \\
    
    \hline
        
    118 & आरोहतायुर्{-}जरसम् गृणाना अनुपूर्वम् ॅयतमाना & T.A.4.10.1       \\
    
    \hline
        
    119 & आशातिकाः क्रिमय इव ततः & T.A.1.8.7       \\
    
    \hline
        
    120 & आहरावद्य शृतस्य हविषो यथा & T.A.7.37.1       \\
    
    \hline
        
    121 & इत्थादुलूक आपप्तत् हिरण्याक्षो अयोमुखः & T.A.7.33.1       \\
    
    \hline
        
    122 & इदमह{-}ममुष्यामुष्यायणस्य शुचा प्राणमपि दहामीत्याह & T.A.8.10.6       \\
    
    \hline
        
    123 & इन्द्र घोषा वो वसुभिः & T.A.1.20.1       \\
    
    \hline
        
    124 & इन्द्रो राजा जगतो य & T.A.3.11.6       \\
    
    \hline
        
    125 & इन्द्रꣳ राजानꣳ सवितारमेतम् वायोरात्मानं & T.A.3.11.4       \\
    
    \hline
        
    126 & इममग्ने चमसम् मा विजीह्वरः & T.A.4.1.4       \\
    
    \hline
        
    127 & इमा नुकं भुवना सीषधेम & T.A.1.27.1       \\
    
    \hline
        
    128 & इमान् ॅलोकान् प्रत्यक्षेण कमग्निञ्चिनुते & T.A.1.22.12       \\
    
    \hline
        
    129 & इमे जीवा विमृतैराववर्{-}तिन्नभूद्{-}भद्रा देवहूतिम् & T.A.4.10.2       \\
    
    \hline
        
    130 & इमे मासा{-}श्चार्द्धमासाश्च सर्वेषां भूतानां & T.A.1.14.3       \\
    
    \hline
        
    131 & इमे वै लोका अफ्सु & T.A.1.22.8       \\
    
    \hline
        
    132 & इयं ॅवा ऋतम् तस्या & T.A.8.9.7       \\
    
    \hline
        
    133 & इयम् नारी पतिलोकम् ॅवृणाना & T.A.4.1.3       \\
    
    \hline
        
    134 & इषे पीपिहि ऊर्जे पीपिहि & T.A.7.10.1       \\
    
    \hline
        
    135 & ईयुष्टे ये पूर्वतरामपश्यन् व्युच्छन्तीमुषसं & T.A.3.18.1       \\
    
    \hline
        
    136 & ईशानः सर्वविद्याना{-} मीश्वरः सर्वभूतानां & T.A.6.21.1       \\
    
    \hline
        
    137 & उक्तꣳ स्थानं प्रमाणञ्च पुर & T.A.1.12.5       \\
    
    \hline
        
    138 & उग्रश्च धुनिश्च ध्वान्तश्च ध्वनश्च & T.A.7.25.1       \\
    
    \hline
        
    139 & उच्चैत्यव चाहभिः भूमिं पर्जन्या & T.A.1.9.6       \\
    
    \hline
        
    140 & उतो बहूनेकमहर्जहार अतन्द्रो देवः & T.A.3.14.2       \\
    
    \hline
        
    141 & उत्तमे शिखरे जाते भूम्यां & T.A.6.36.1       \\
    
    \hline
        
    142 & उत्तानायाङ्गीरसायानः वैश्वानराय रथम् वैश्वानरः & T.A.3.10.4       \\
    
    \hline
        
    143 & उत्तुद शिमिजावरि तल्पेजे तल्प & T.A.7.39.1       \\
    
    \hline
        
    144 & उत्ते तभ्नोमि पृथिवीम् त्वत्परीमम् & T.A.4.7.1       \\
    
    \hline
        
    145 & उदस्य शुष्माद् भानुर्नार्त बिभर्ति & T.A.7.17.1       \\
    
    \hline
        
    146 & उप नो मित्रावरुणाविहावतम् अन्वादीद्ध्याथामिह & T.A.7.20.3       \\
    
    \hline
        
    147 & उपानुवाक्य{-}माशु मग्निञ्चिन्वानः कमग्निञ्चिनुते इममारुण{-}केतुक & T.A.1.26.3       \\
    
    \hline
        
    148 & उभयतः सप्तेन्द्रियाणि जल्पितं त्वेव & T.A.1.2.4       \\
    
    \hline
        
    149 & उरुणसा{-}वसुतृपा{-}वुलुम्बलौ यमस्य दूतौ चरतो & T.A.4.3.2       \\
    
    \hline
        
    150 & उष्मा च नीहारश्च वृत्रस्योष्मा & T.A.1.10.7       \\
    
    \hline
        
    151 & ऊरू तदस्य यद्{-}वैश्यः पद्भ्यां & T.A.3.12.6       \\
    
    \hline
        
    152 & ऊर्द्ध्वमिम{-}मद्ध्वरङ्कृधि दिवि देवेषु होत्रा & T.A.7.7.3       \\
    
    \hline
        
    153 & ऊर्द्ध्वस्तिष्ठ ध्रुवस्त्वम् सूर्यस्य त्वा & T.A.7.3.3       \\
    
    \hline
        
    154 & ऋतं च स्वाद्ध्याय प्रवचने & T.A.5.9.1       \\
    
    \hline
        
    155 & ऋतं तपः सत्यं तपः & T.A.6.10.1       \\
    
    \hline
        
    156 & ऋतुभ्य एव यज्ञ्स्य शिरोऽवरुन्धे & T.A.8.6.2       \\
    
    \hline
        
    157 & ऋतꣳ सत्यं परं ब्रह्म & T.A.6.23.1       \\
    
    \hline
        
    158 & ऋषयः सप्तात्रिश्च यत् सर्वेऽत्रयो & T.A.1.11.2       \\
    
    \hline
        
    159 & ऋषिर्.ह दीर्घश्रुत्तमः इन्द्रस्य घर्मो & T.A.1.8.8       \\
    
    \hline
        
    160 & एको अश्वो वहति सप्तनामा & T.A.3.11.9       \\
    
    \hline
        
    161 & एतास्ते स्वधा अमृताः करोमि & T.A.4.9.1       \\
    
    \hline
        
    162 & एवमेतन्निबोधत आ मन्द्रै{-}रिन्द्र हरिभिः & T.A.1.12.2       \\
    
    \hline
        
    163 & एवा नो दूर्वे प्रतनु & T.A.6.1.8       \\
    
    \hline
        
    164 & एषा ते अग्ने समित् & T.A.7.10.5       \\
    
    \hline
        
    165 & एषा ते यमसादने स्वधा & T.A.4.7.2       \\
    
    \hline
        
    166 & एष्वेव लोकेषु प्रजा दाधार & T.A.8.9.2       \\
    
    \hline
        
    167 & ओ{-}मिति ब्रह्म ओ{-}मितीदꣳ सर्वं & T.A.5.8.1       \\
    
    \hline
        
    168 & ओं तद् ब्रह्म ओं & T.A.6.68.1       \\
    
    \hline
        
    169 & ओजोऽसि सहोऽसि बलमसि भ्राजोऽसि & T.A.6.35.1       \\
    
    \hline
        
    170 & ओमित्येकाक्षरं ब्रह्म अग्निर्देवता ब्रह्म & T.A.6.33.1       \\
    
    \hline
        
    171 & क इदं कस्मा अदात् & T.A.3.10.5       \\
    
    \hline
        
    172 & कतिधाऽवकीर्णी प्रविशति चतुर्द्धेत्याहुर्{-}ब्रह्मवादिनो मरुतः & T.A.2.18.1       \\
    
    \hline
        
    173 & कद्रुद्राय प्रचेतसे मीढुष्टमाय तव्यसे & T.A.6.25.1       \\
    
    \hline
        
    174 & कश्यपा दुदिताः सूर्याः पापान्निर्घ्नन्ति & T.A.1.8.6       \\
    
    \hline
        
    175 & कामः प्रतिग्रहीता कामꣳ समुद्रमाविश & T.A.3.10.2       \\
    
    \hline
        
    176 & कामोऽकार्.षीन् नमो नमः कामोऽकार्.षीत् & T.A.6.61.1       \\
    
    \hline
        
    177 & कुर्वाणा चीर{-}मात्मनः वासाꣳसि मम & T.A.5.4.2       \\
    
    \hline
        
    178 & कूप्या गृह्णाति ता दक्षिणत & T.A.1.24.2       \\
    
    \hline
        
    179 & कूश्माण्डैर् जुहुयाद्{-}योऽपूत इव मन्येत & T.A.2.8.1       \\
    
    \hline
        
    180 & क्वेदमभ्रं निविशते क्वायꣳ सम्ॅवथ्सरो & T.A.1.8.1       \\
    
    \hline
        
    181 & खट् फड् जहि छिन्धी & T.A.7.27.1       \\
    
    \hline
        
    182 & गर्भो देवानामित्याह गर्भो ह्येष & T.A.8.6.8       \\
    
    \hline
        
    183 & गर्भो देवानाम् पिता मतीनाम् & T.A.7.7.4       \\
    
    \hline
        
    184 & गायत्रो हि प्राणः प्राणमेव & T.A.8.4.3       \\
    
    \hline
        
    185 & गायत्रोऽसि त्रैष्टुभोऽसि जागतमसि सहोर्जो & T.A.7.8.4       \\
    
    \hline
        
    186 & गौरावस्कन्दिन्न{-}हल्यायै जार कौशिक{-}ब्राह्मण गौतमब्रुवाण & T.A.1.12.4       \\
    
    \hline
        
    187 & ग्रामे मनसा स्वाद्ध्यायमधीयीत दिवा & T.A.2.12.1       \\
    
    \hline
        
    188 & ग्रैष्मावेवास्मा ऋतू कल्पयति समग्निरग्निना & T.A.8.6.6       \\
    
    \hline
        
    189 & घर्म या ते दिवि & T.A.7.11.1 T.A.8.9.1       \\
    
    \hline
        
    190 & घर्म या ते पृथिव्यां & T.A.7.11.2       \\
    
    \hline
        
    191 & घर्मम् पात वसवो यजता & T.A.8.7.6       \\
    
    \hline
        
    192 & घर्माय शिꣳष बृहस्पतिस्त्वोपसीदतु दानवः & T.A.7.8.3       \\
    
    \hline
        
    193 & घृणिः सूर्य आदित्यो न & T.A.6.37.1       \\
    
    \hline
        
    194 & घृतं तेजो मधुमदिन्द्रियम् मय्ययमग्निर्दधातु & T.A.3.11.8       \\
    
    \hline
        
    195 & चक्षुरस्य प्रमायुकꣳ स्यात् तस्मान्नान्वीक्ष्यः & T.A.8.8.10       \\
    
    \hline
        
    196 & चतुष्टय्य आपो गृह्णाति चत्वारि & T.A.1.24.1       \\
    
    \hline
        
    197 & चन्द्रमाः षड्ढोता स ऋतून् & T.A.3.7.3       \\
    
    \hline
        
    198 & चित्तिः स्रुक् चित्तमाज्यम् वाग्वेदिः & T.A.3.1.1       \\
    
    \hline
        
    199 & चित्तꣳ सन्तानेन भवं ॅयक्ना & T.A.3.21.1       \\
    
    \hline
        
    200 & छन्दोभिरेवैनम् धूपयति अर्चिषे त्वा & T.A.8.3.6       \\
    
    \hline
        
    201 & जातवेदसे सुनवाम सोम{-}मरातीयतो निदहाति & T.A.6.2.1       \\
    
    \hline
        
    202 & जानुदघ्नी{-}मुत्तरवेदीङ्खात्वा अपां पूरयति अपां & T.A.1.25.1       \\
    
    \hline
        
    203 & ज्योतिषा ऽप्रतिख्येन सः विश्वरूपाणि & T.A.1.3.4       \\
    
    \hline
        
    204 & ज्योतिष्मतीं त्वा सादयामि ज्योतिष्कृतं & T.A.3.19.1       \\
    
    \hline
        
    205 & ठिस् ए꣡पन्सिऒन् इस् अप्पॆअरिन्ग् & T.A.6.27.1       \\
    
    \hline
        
    206 & तं मे देवा ब्रह्मणा & T.A.3.14.4       \\
    
    \hline
        
    207 & तच्छकेयं तन्मे राद्ध्यताम् वायो & T.A.7.41.4       \\
    
    \hline
        
    208 & ततो वायुरुदतिष्ठत् सा प्रतीची & T.A.1.23.6       \\
    
    \hline
        
    209 & ततो वै स दुश्चर्माऽभवत् & T.A.8.4.13       \\
    
    \hline
        
    210 & तत्पुरुषाय विद्महे महादेवाय धीमहि & T.A.6.20.1       \\
    
    \hline
        
    211 & तथा तेषु वर्तेथाः एष & T.A.5.11.5       \\
    
    \hline
        
    212 & तथ् साम्नः पयः यदजायै & T.A.8.10.2       \\
    
    \hline
        
    213 & तथ् स्मयाकानाꣳ स्मयाकत्वम् तस्माद्{-}दीक्षितेनापिगृह्य & T.A.8.1.4       \\
    
    \hline
        
    214 & तदवरुन्धे कूर्ममुपदधाति अपामेव मेधमवरुन्धे & T.A.1.25.2       \\
    
    \hline
        
    215 & तदेवर्त्तं तदु सत्यमाहु{-}स्तदेव ब्रह्म & T.A.6.1.2       \\
    
    \hline
        
    216 & तन्नाद्ध्रियत स पूतीकस्तम्बे पराक्रमत & T.A.8.2.10       \\
    
    \hline
        
    217 & तन्नो नन्दिः प्रचोदयात् तत्पुरुषाय & T.A.6.1.6       \\
    
    \hline
        
    218 & तन्नो नारसिꣳहः प्रचोदयात् भास्कराय & T.A.6.1.7       \\
    
    \hline
        
    219 & तमेकꣳ सन्तम् बहवो नाभ्यधृष्णुवन्न् & T.A.8.1.3       \\
    
    \hline
        
    220 & तमेवावरुन्धे पञ्चैते सम्भारा भवन्ति & T.A.8.2.11       \\
    
    \hline
        
    221 & तयो रेतौ वथ्सा वहोरात्रे & T.A.1.10.5       \\
    
    \hline
        
    222 & तरणिर् विश्वदर्.शतो ज्योतिष्कृदसि सूर्य & T.A.3.16.1       \\
    
    \hline
        
    223 & तस्मा इममग्र पिण्डञ्जुहोमि स & T.A.1.31.6       \\
    
    \hline
        
    224 & तस्मा एवैनञ्जुहोति एताभ्य एवैनम् & T.A.8.7.12       \\
    
    \hline
        
    225 & तस्मा{-}दुत्तरवेद्या{-}मेवोद्वासयेत् प्रजानाम् गोपीथाय पुरो & T.A.8.10.4       \\
    
    \hline
        
    226 & तस्मादग्निः सर्वा दिशोऽनु विभाति & T.A.8.3.7       \\
    
    \hline
        
    227 & तस्मादश्ञुते प्रजापतिर्वा एष द्वादशधा & T.A.8.11.6       \\
    
    \hline
        
    228 & तस्मादश्वा अजायन्त ये के & T.A.3.12.5       \\
    
    \hline
        
    229 & तस्मादिन्द्रो देवतानाम् भूयिष्ठभाक्तमः गायत्रोऽसि & T.A.8.7.5       \\
    
    \hline
        
    230 & तस्मादेवमाह यज्ञ्स्य पदे स्थ & T.A.8.3.3       \\
    
    \hline
        
    231 & तस्मान्नान्तराय्यम् आत्मनो गोपीथाय वेणुना & T.A.8.3.2       \\
    
    \hline
        
    232 & तस्मिन् राजान{-}मधिविश्रयेममिति ते अस्मै & T.A.1.7.2       \\
    
    \hline
        
    233 & तस्य धीराः परिजानन्ति योनिम् & T.A.3.13.2       \\
    
    \hline
        
    234 & तस्य वा एतस्य यज्ञ्स्य & T.A.2.14.1 T.A.2.15.1       \\
    
    \hline
        
    235 & तस्येन्द्रो वम्रिरूपेण धनुर्ज्या{-}मछिनथ्स्वयं तदिन्द्रधनुरित्यज्यं & T.A.1.5.2       \\
    
    \hline
        
    236 & तस्यैवं ॅविदुषो यज्ञ्स्यात्मा यजमानः{-}श्रध्दापत्नी & T.A.6.80.1       \\
    
    \hline
        
    237 & ता उत्तरत उपदधाति ओजसा & T.A.1.24.3       \\
    
    \hline
        
    238 & तां ॅवायुः समिन्धे सा & T.A.7.41.2       \\
    
    \hline
        
    239 & तान्नद्योऽभि समायन्ति सोरुः सती & T.A.1.2.2       \\
    
    \hline
        
    240 & ताभ्य एवैनञ्जुहोति प्रतिरेभ्यः स्वाहेत्याह & T.A.8.8.8       \\
    
    \hline
        
    241 & ताम्रो अरुणः ता अविसृष्टौ & T.A.1.10.6       \\
    
    \hline
        
    242 & तिरोधाः स्वः तिरोधा भूर्भुवस्स्वः & T.A.1.31.4       \\
    
    \hline
        
    243 & तिलाः कृष्णा{-}स्तिलाः श्वेता{-}स्तिलाः सौम्या & T.A.6.64.1       \\
    
    \hline
        
    244 & तिलाञ्जुहोमि सरसाꣳ सपिष्टान् गन्धार & T.A.6.63.1       \\
    
    \hline
        
    245 & तिस्रः क्षपस्त्रिरहा ऽतिव्रजद्भिः नासत्या & T.A.1.10.3       \\
    
    \hline
        
    246 & तेज एवास्मिन् दधाति मधु & T.A.8.2.13       \\
    
    \hline
        
    247 & तेजः प्रवर्ग्यः तेजसैव तेजः & T.A.8.10.3       \\
    
    \hline
        
    248 & तेजसैव यज्ञ्स्य शिरः सम्भरति & T.A.8.2.5       \\
    
    \hline
        
    249 & तेजसैवैनमनक्ति पृथिवीम् तपसस्त्रायस्वेति हिरण्यमुपास्यति & T.A.8.4.5       \\
    
    \hline
        
    250 & तेजसोऽस्कन्दाय इषे पीपिह्यूर्जे पीपिहीत्याह & T.A.8.8.6       \\
    
    \hline
        
    251 & तेन देवा अयजन्त साद्ध्या & T.A.3.12.4       \\
    
    \hline
        
    252 & तेषाम् मखं ॅवैष्णवं ॅयश & T.A.8.1.2       \\
    
    \hline
        
    253 & त्रिकद्रुकेभिः पतति षडुर्वी{-}रेकमिद्{-}बृहत् गायत्री & T.A.4.5.3       \\
    
    \hline
        
    254 & त्रिधा हितं पणिभिर् गुह्यमानं & T.A.6.12.3       \\
    
    \hline
        
    255 & त्रिर्.हरति त्रय इमे लोकाः & T.A.8.2.8       \\
    
    \hline
        
    256 & त्रीणि पदा निहिता गुहासु & T.A.6.1.4       \\
    
    \hline
        
    257 & त्र्यंबकं ॅयजामहे सुगन्धिं पुष्टिवर्द्धनं & T.A.6.56.1       \\
    
    \hline
        
    258 & त्वं ॅयज्ञ्स्त्वमु वेवासि सोमः & T.A.3.14.3       \\
    
    \hline
        
    259 & त्वमग्ने द्युभि{-}स्त्वमाशु{-}शुक्षणि{-}स्त्वमद्भ्य{-}स्त्वमश्मनस्परि त्वं ॅवनेभ्य{-}स्त्वमोषधीभ्य{-}स्त्वं & T.A.6.76.1       \\
    
    \hline
        
    260 & त्वष्टेद्ध्मेन विष्णुर् यज्ञेन वसव & T.A.3.8.2       \\
    
    \hline
        
    261 & त्वꣳ हि विश्वतोमुख विश्वतः & T.A.4.11.2       \\
    
    \hline
        
    262 & दक्षम् मे अन्य आवातु & T.A.7.42.2       \\
    
    \hline
        
    263 & दक्षिणपूर्व{-}स्यान्दिशि विसर्पी नरकः तस्मान्नः & T.A.1.19.1       \\
    
    \hline
        
    264 & दश प्राचीर्दश भासि दक्षिणा & T.A.7.6.1       \\
    
    \hline
        
    265 & दिक्ष्वेवैनम् प्रतिष्ठापयति देवान् घर्मपान् & T.A.8.8.4       \\
    
    \hline
        
    266 & दिवि देवेषु होत्रा यच्छेत्याह & T.A.8.6.7       \\
    
    \hline
        
    267 & दिवि धा इमं ॅयज्ञ्म् & T.A.7.9.4       \\
    
    \hline
        
    268 & दिवे स्वाहा सूर्याय स्वाहा & T.A.6.67.3       \\
    
    \hline
        
    269 & दिशो भूतिः इमानेवास्मै लोकान् & T.A.8.3.8       \\
    
    \hline
        
    270 & दीर्घमुखि दुर्.हणु मा स्म & T.A.7.32.1       \\
    
    \hline
        
    271 & दुर्भिक्षं देवलोकेषु मनूनामुदकं गृहे & T.A.1.4.3       \\
    
    \hline
        
    272 & दुर्मित्रास्तस्मै भूयासुः योऽस्मान् द्वेष्टि & T.A.7.11.9       \\
    
    \hline
        
    273 & दुहे ह वा एष & T.A.2.17.1       \\
    
    \hline
        
    274 & देवकृतस्यैनसो{-}ऽवयजनमसि स्वाहा मनुष्यकृतस्यैनसो ऽवयजनमसि & T.A.6.59.1       \\
    
    \hline
        
    275 & देवतायै वषट्काराय यच्चतुर्गृहीतम् जुहोति & T.A.8.2.2       \\
    
    \hline
        
    276 & देवपितृकार्याभ्यां न प्रमदितव्यं मातृदेवो & T.A.5.11.2       \\
    
    \hline
        
    277 & देवयन्तस्त्वेमहे उप प्रयन्तु मरुतः & T.A.7.2.2       \\
    
    \hline
        
    278 & देवस्य त्वा सवितुः प्रसव & T.A.8.7.1       \\
    
    \hline
        
    279 & देवस्य त्वा सवितुः प्रसवे & T.A.3.10.1 T.A.7.8.1       \\
    
    \hline
        
    280 & देवा वै सत्रमासत ऋद्धिपरिमितम् & T.A.8.1.1       \\
    
    \hline
        
    281 & देवानां पूरयोद्ध्या तस्याꣳ हिरण्मयः & T.A.1.27.3       \\
    
    \hline
        
    282 & देवीः पर्जन्य सूवरीः पुत्रवत्त्वाय & T.A.1.21.2       \\
    
    \hline
        
    283 & देवेभ्यः स्वाहा पितृभ्यः स्वधाऽस्तु & T.A.6.67.4       \\
    
    \hline
        
    284 & द्युतानस्त्वा मारुतो मरुद्भिरुत्तरतो रोचयत्वानुष्टुभेन & T.A.7.6.2       \\
    
    \hline
        
    285 & द्युभिरक्तुभिः परिपात{-}मस्मानरिष्टेभि{-}रश्विना सौभगेभिः तन्नो & T.A.7.42.3       \\
    
    \hline
        
    286 & द्वादश मासाः सम्ॅवथ्सरः सम्ॅवथ्सरमेवावरुन्धे & T.A.8.4.10       \\
    
    \hline
        
    287 & धियो यो नः प्रचोदयात् & T.A.1.11.3       \\
    
    \hline
        
    288 & धियो हिन्वानो धिय इन्नो & T.A.7.11.8       \\
    
    \hline
        
    289 & धुनिश्च ध्वान्तश्च ध्वनश्च ध्वनयꣳश्च & T.A.7.24.1       \\
    
    \hline
        
    290 & न सदृंशे तिष्ठति रूपमस्य & T.A.6.1.3       \\
    
    \hline
        
    291 & न हि प्रवेद सुकृतस्य & T.A.1.3.2       \\
    
    \hline
        
    292 & न ह्येष निपद्यते आ & T.A.8.6.5       \\
    
    \hline
        
    293 & नपुꣳसकं पुमाꣳस्त्र्यस्मि स्थावरोऽस्म्यथ जङ्गमः & T.A.1.11.4       \\
    
    \hline
        
    294 & नमः प्राच्यै दिशे याश्च & T.A.2.20.1       \\
    
    \hline
        
    295 & नमस्ते अस्तु मा मा & T.A.7.11.7       \\
    
    \hline
        
    296 & नमो ब्रह्मणे धारणं मे & T.A.6.9.1       \\
    
    \hline
        
    297 & नमो रुद्राय विष्णवे मृत्युर्मे & T.A.6.75.1       \\
    
    \hline
        
    298 & नमो वाचे या चोदिता & T.A.7.1.1       \\
    
    \hline
        
    299 & नमो हिरण्यबाहवे हिरण्यवर्णाय हिरण्यरूपाय & T.A.6.22.1       \\
    
    \hline
        
    300 & नाचिकेत{-}मग्निञ्चिन्वानः प्राणान् प्रत्यक्षेण कमग्निञ्चिनुते & T.A.1.22.11       \\
    
    \hline
        
    301 & नाभिर्दशमी प्राणानेव यजमाने दधाति & T.A.8.6.10       \\
    
    \hline
        
    302 & नास्याक्षो यातु सज्जति यच्छ्वेतान् & T.A.1.11.8       \\
    
    \hline
        
    303 & निघृष्वै रसमायुतैः कालैर्. हरित्वमापन्नैः & T.A.1.12.3       \\
    
    \hline
        
    304 & निजानुकामे न्यञ्जलिका अमी वाच{-}मुपासतामिति & T.A.1.6.2       \\
    
    \hline
        
    305 & निधनपतये नमः निधनपतान्तिकाय नमः & T.A.6.16.1       \\
    
    \hline
        
    306 & निष्कर्ता विह्रुतम् पुनः पुनरूर्जा, & T.A.7.20.2       \\
    
    \hline
        
    307 & नेत्यब्रवीत् पूर्वमेवाह{-}मिहासमिति तत्पुरुषस्य पुरुषत्वं & T.A.1.23.4       \\
    
    \hline
        
    308 & नैनं गरो हिनस्ति य & T.A.1.9.4       \\
    
    \hline
        
    309 & नो इतराणि ये के & T.A.5.11.3       \\
    
    \hline
        
    310 & पञ्च वा एते महायज्ञाः & T.A.2.10.1       \\
    
    \hline
        
    311 & परं मृत्यो अनु परेहि & T.A.6.46.1       \\
    
    \hline
        
    312 & परा मार्ताण्डमास्यत् सप्तभिः पुत्रै{-}रदितिः & T.A.1.13.3       \\
    
    \hline
        
    313 & परिश्रिते करोति ब्रह्मवर्चसस्य परिगृहीत्यै & T.A.8.3.1       \\
    
    \hline
        
    314 & परेयुवाꣳसम् प्रवतो महीरनु बहुभ्यः & T.A.4.1.1       \\
    
    \hline
        
    315 & पर्जन्याय प्रगायत दिवस्पुत्राय मीढुषे & T.A.1.29.1       \\
    
    \hline
        
    316 & पवित्रवन्तः परिवाजमासते पितैषां प्रत्नो & T.A.1.11.1       \\
    
    \hline
        
    317 & पशूꣳश्च मह्यमावह जीवनञ्च दिशो & T.A.6.1.5       \\
    
    \hline
        
    318 & पाङ्क्तो यज्ञ्ः यावानेव यज्ञ्ः & T.A.8.7.10       \\
    
    \hline
        
    319 & पाङ्क्तो हि यज्ञ्ः देवा & T.A.8.2.7       \\
    
    \hline
        
    320 & पादोऽस्य विश्वा भूतानि त्रिपादस्यामृतं & T.A.3.12.2       \\
    
    \hline
        
    321 & पाहि नो अग्न एकया & T.A.6.7.1       \\
    
    \hline
        
    322 & पाहि नो अग्न एनसे & T.A.6.6.1       \\
    
    \hline
        
    323 & पिता नोऽसि मा मा & T.A.7.10.6       \\
    
    \hline
        
    324 & पुनःमामैत्विन्द्रियं पुनरायुः पुनःभगः पुनः & T.A.1.30.1       \\
    
    \hline
        
    325 & पुष्करपर्णैः पुष्करदण्डैः पुष्करैश्च सꣳस्तीर्य & T.A.1.22.9       \\
    
    \hline
        
    326 & पूर्वमेवोदितम् उत्तरेणाभिगृणाति अनु वान्{-}द्यावापृथिवी & T.A.8.8.3       \\
    
    \hline
        
    327 & पूर्वमेवोदितम् उत्तरेणाभिगृणाति धियो हिन्वानो & T.A.8.9.10       \\
    
    \hline
        
    328 & पूष्णे स्वाहा पूष्णे शरसे & T.A.7.16.1       \\
    
    \hline
        
    329 & पृथिवी समित् तामग्निः समिन्धे & T.A.7.41.1       \\
    
    \hline
        
    330 & पृथिवी होता द्यौरद्ध्वर्युः रुद्रोऽग्नीत् & T.A.3.2.1       \\
    
    \hline
        
    331 & पृथिवीं तपसस्त्रायस्व अर्चिरसि शोचिरसि & T.A.7.5.2       \\
    
    \hline
        
    332 & पृथिव्यन्तरिक्षं द्यौर्{-}दिशोऽवान्तरदिशाः अग्निर्{-}वायु{-}रादित्य{-}श्चन्द्रमा नक्षत्राणि & T.A.5.7.1       \\
    
    \hline
        
    333 & पृथिव्याप स्तेजो वायु{-}राकाशा मे & T.A.6.66.1       \\
    
    \hline
        
    334 & पृष्ठानि वा अच्युतञ्च्यावयन्ति पृष्ठैरेवास्मा & T.A.8.6.4       \\
    
    \hline
        
    335 & प्र केतुना बृहता भात्यग्निराविर्{-}विश्वानि & T.A.4.3.1       \\
    
    \hline
        
    336 & प्र वाता वान्ति पतयन्ति & T.A.4.6.3       \\
    
    \hline
        
    337 & प्रजापतिं ॅवै देवाः शुक्रम् & T.A.8.10.1       \\
    
    \hline
        
    338 & प्रजापतिः प्रजया सम्ॅविदानः वीतं & T.A.3.11.12       \\
    
    \hline
        
    339 & प्रजापतिः सम्भ्रियमाणः सम्राट्थ् सम्भृतः & T.A.8.11.1       \\
    
    \hline
        
    340 & प्रजापते न त्वदेता{-}न्यन्यो विश्वा & T.A.6.54.1       \\
    
    \hline
        
    341 & प्रयासाय स्वाहाऽऽयासाय स्वाहा वियासाय & T.A.3.20.1       \\
    
    \hline
        
    342 & प्रसार्य सक्थ्यौ पतसि सव्यमक्षि & T.A.7.35.1       \\
    
    \hline
        
    343 & प्राजापत्यो वा एषोऽग्निः प्राजापत्याः & T.A.1.26.4       \\
    
    \hline
        
    344 & प्राजापत्यो हारुणिः सुपर्णेयः प्रजापतिं & T.A.6.79.1       \\
    
    \hline
        
    345 & प्राञ्चमुद्वासयति तस्मादसावादित्यः पुरस्तादुदेति शफोपयमान्धवित्राणि & T.A.8.9.3       \\
    
    \hline
        
    346 & प्राणं देवा अनुप्राणन्ति मनुष्याः & T.A.5.14.3       \\
    
    \hline
        
    347 & प्राणानां ग्रन्थिरसि रुद्रो मा & T.A.6.74.1       \\
    
    \hline
        
    348 & प्राणापान{-}व्यानोदान{-}समाना मे शुद्ध्यन्तां ज्योति & T.A.6.65.1       \\
    
    \hline
        
    349 & प्राणाय स्वाहा व्यानाय स्वाहा & T.A.7.15.1       \\
    
    \hline
        
    350 & प्राणो ब्रह्मेति व्यजानात् प्राणाद्ध्येव & T.A.5.15.3       \\
    
    \hline
        
    351 & प्राणो वा इन्द्रतमोऽग्निः प्राण & T.A.8.8.12       \\
    
    \hline
        
    352 & प्रास्मा आशा अशृण्वन्न् कामेनाजनयन् & T.A.3.15.2       \\
    
    \hline
        
    353 & प्रियेण नाम्ना समर्द्धयति कीर्तिरस्य & T.A.8.11.3       \\
    
    \hline
        
    354 & बृहस्पतये त्वा विश्वदेव्यावते स्वाहा & T.A.7.9.2       \\
    
    \hline
        
    355 & ब्रह्म जज्ञानं प्रथमं पुरस्ताद्{-}विसीमतः & T.A.6.1.10       \\
    
    \hline
        
    356 & ब्रह्म मेतु मां मधु & T.A.6.38.1       \\
    
    \hline
        
    357 & ब्रह्म मेधया मधु मेधया & T.A.6.39.1       \\
    
    \hline
        
    358 & ब्रह्म मेधवा मधु मेधवा & T.A.6.40.1       \\
    
    \hline
        
    359 & ब्रह्म वै देवानाम् बृहस्पतिः & T.A.8.7.4       \\
    
    \hline
        
    360 & ब्रह्मणा त्वा शपामि ब्रह्मणस्त्वा & T.A.7.38.1       \\
    
    \hline
        
    361 & ब्रह्मणा वीर्यावता शिवा नः & T.A.1.9.7       \\
    
    \hline
        
    362 & ब्रह्मन् प्रचरिष्यामः होतर् घर्ममभिष्टुहि & T.A.7.5.1       \\
    
    \hline
        
    363 & ब्रह्मन् प्रचरिष्यामो होतर्{-}घर्ममभिष्टुहीत्याह एष & T.A.8.4.1       \\
    
    \hline
        
    364 & ब्रह्मन् प्रवःग्येण प्रचरिष्यामः होतः & T.A.7.4.1       \\
    
    \hline
        
    365 & ब्रह्मन्नेवैनम् प्रतिष्ठापयति नेत्त्वा वातः & T.A.8.8.7       \\
    
    \hline
        
    366 & ब्रह्मयज्ञेन यक्ष्यमाणः प्राच्यां दिशि & T.A.2.11.1       \\
    
    \hline
        
    367 & ब्रह्मविदाप्नोति परं तदेषाऽभ्युक्ता सत्यं & T.A.5.14.1       \\
    
    \hline
        
    368 & ब्राह्मण एकहोता स यज्ञ्ः & T.A.3.7.1       \\
    
    \hline
        
    369 & भद्रं कर्णेभिः शृणुयाम देवाः & T.A.1.1.1       \\
    
    \hline
        
    370 & भर्ता सन् भ्रियमाणो बिभर्ति & T.A.3.14.1       \\
    
    \hline
        
    371 & भवाय नमः भवलिङ्गाय नमः & T.A.6.16.2       \\
    
    \hline
        
    372 & भिषजौ वै स्थः इदं & T.A.8.1.7       \\
    
    \hline
        
    373 & भीषाऽस्मा{-}द्वातः पवते भीषोदेति सूर्यः & T.A.5.14.8       \\
    
    \hline
        
    374 & भू{-}रन्न{-}मग्नये पृथिव्यै स्वाहा , & T.A.6.3.1       \\
    
    \hline
        
    375 & भू{-}र्भुव{-}स्सुव{-}रिति वा एता स्तिस्रो & T.A.5.5.1       \\
    
    \hline
        
    376 & भूः प्रपद्ये भुवः प्रपद्ये & T.A.2.19.1       \\
    
    \hline
        
    377 & भूरग्नये च पृथिव्यै च & T.A.6.5.1       \\
    
    \hline
        
    378 & भूरग्नये पृथिव्यै स्वाहा, भुवो & T.A.6.4.1       \\
    
    \hline
        
    379 & भूर्भुवस्सुवः ऊर्द्ध्व ऊ षुण & T.A.7.20.1       \\
    
    \hline
        
    380 & भूर्भुवस्सुवः मयि त्यदिन्द्रियम् महत् & T.A.7.21.1       \\
    
    \hline
        
    381 & भूर्भुवस्सुवो भूर्भुवस्सुवो भूर्भुवस्सुवः भुवोऽद्धायि & T.A.7.40.1       \\
    
    \hline
        
    382 & भृगुर्वै वारुणिः वरुणं पितर{-}मुपससार & T.A.5.15.1       \\
    
    \hline
        
    383 & मखाय त्वा मखस्य त्वा & T.A.7.2.3 T.A.7.2.4 T.A.7.2.5       \\
    
    \hline
        
    384 & मतिर्ह्येष कवीनाम् सम् देवो & T.A.8.6.9       \\
    
    \hline
        
    385 & मद्ध्यन्दिने प्रबलमधीयीतासौ खलु वावैष & T.A.2.13.1       \\
    
    \hline
        
    386 & मनवे तल्पम् त्वष्ट्रेऽजाम् पूष्णेऽविम् & T.A.3.10.3       \\
    
    \hline
        
    387 & मनुष्यनामैरेवैनामाह्वयति षट्थ् सम्पद्यन्ते षड्वा & T.A.8.7.2       \\
    
    \hline
        
    388 & मनुष्यो हि एष सन्{-}मनुष्यानुपैति & T.A.8.9.11       \\
    
    \hline
        
    389 & मनो ब्रह्मेति व्यजानात् मनसो & T.A.5.15.4       \\
    
    \hline
        
    390 & मनोरश्वाऽसि भूरिपुत्रेतीमा{-}मभिमृशति इयं ॅवै & T.A.8.4.8       \\
    
    \hline
        
    391 & मन्युरकार्.षीन् नमो नमः मन्युरकार्.षीन् & T.A.6.62.1       \\
    
    \hline
        
    392 & मयि मेधां मयि प्रजां & T.A.6.44.1       \\
    
    \hline
        
    393 & मयि रुक् दश पुरस्ताद्{-}रोचसे & T.A.7.6.3       \\
    
    \hline
        
    394 & मरीचयः स्वायंभुवाः ये शरीराण्य & T.A.1.27.2       \\
    
    \hline
        
    395 & मह इति ब्रह्म ब्रह्मणा & T.A.5.5.3       \\
    
    \hline
        
    396 & मह इत्यादित्यः आदित्येन वाव & T.A.5.5.2       \\
    
    \hline
        
    397 & महानाम्नीभि{-}रुदकꣳ सꣳस्पर्श्य तमाचार्यो दद्यात् & T.A.1.32.3       \\
    
    \hline
        
    398 & महाहविर्. होता सत्यहविरद्ध्वर्युः अच्युतपाजा & T.A.3.5.1       \\
    
    \hline
        
    399 & महीनाम् पयोऽसि विहितं देवत्रा & T.A.7.12.1       \\
    
    \hline
        
    400 & महेरणाय चक्षसे यो वः & T.A.6.1.12       \\
    
    \hline
        
    401 & मा छिदो मृत्यो मा & T.A.6.51.1       \\
    
    \hline
        
    402 & मा नस्तोके तनये मा & T.A.6.53.1       \\
    
    \hline
        
    403 & मा नो महान्तमुत मा & T.A.6.52.1       \\
    
    \hline
        
    404 & मा मे प्रजाया मा & T.A.1.14.2 T.A.1.14.4       \\
    
    \hline
        
    405 & मित्रस्य चर्.षणीधृतः श्रवो देवस्य & T.A.7.3.2       \\
    
    \hline
        
    406 & मित्रावरुणयोराधिपत्ये श्रोत्रम् मे दाः & T.A.7.5.4       \\
    
    \hline
        
    407 & मृत्तिके प्रतिष्ठिते सर्वं तन्मे & T.A.6.1.9       \\
    
    \hline
        
    408 & मृत्यवे स्वाहा मृत्यवे स्वाहा & T.A.6.58.1       \\
    
    \hline
        
    409 & मेधां म इन्द्रो ददातु & T.A.6.42.1       \\
    
    \hline
        
    410 & मेधादेवी जुषमाणा न आगाद् & T.A.6.41.1       \\
    
    \hline
        
    411 & य उदगान् महतोऽर्णवाद्{-}विभ्राजमानः सरिरस्य & T.A.7.42.6       \\
    
    \hline
        
    412 & य एतस्य पथो गोप्तारस्तेभ्यः & T.A.4.2.1       \\
    
    \hline
        
    413 & य एवं ॅवेद अथ & T.A.1.9.3       \\
    
    \hline
        
    414 & य एवं ॅवेद योऽपामायतनं & T.A.1.22.5       \\
    
    \hline
        
    415 & य श्छन्दसा{-}मृषभो विश्वरूपः छन्दोभ्यो{-}ऽद्ध्यमृताथ् & T.A.5.4.1       \\
    
    \hline
        
    416 & यजमानाय पीपिहि मह्यञ्ज्यैष्ठ्याय पीपिहि & T.A.7.10.2       \\
    
    \hline
        
    417 & यज्ञ्परुरन्तरियात् यजुरेव वदेत् न & T.A.8.2.3       \\
    
    \hline
        
    418 & यज्ञ्स्य पदे स्थः गायत्रेण & T.A.7.2.6       \\
    
    \hline
        
    419 & यज्ञ्ꣳ रक्षाꣳसि जिघाꣳसन्ति साम्ना & T.A.8.9.4       \\
    
    \hline
        
    420 & यज्वानो येऽप्ययज्वनः स्वर्यन्तो नापेक्षन्ते & T.A.1.27.5       \\
    
    \hline
        
    421 & यतो वाचो निवर्तन्ते अप्राप्य & T.A.5.14.4 T.A.5.14.9       \\
    
    \hline
        
    422 & यत् पृथिव्यामुद्वासयेत् पृथिवीꣳ शुचाऽर्पयेत् & T.A.8.9.5       \\
    
    \hline
        
    423 & यत् प्रत्यक् तन्{-}मनुष्याणाम् यदुदङ्ङ् & T.A.8.8.5       \\
    
    \hline
        
    424 & यत् प्रवर्ग्यः ऊङ्र्मुञ्जाः यन् & T.A.8.4.4       \\
    
    \hline
        
    425 & यत्पुरुषेण हविषा देवा यज्ञ्मतन्वत & T.A.3.12.3       \\
    
    \hline
        
    426 & यथा वृक्षस्य सपुंष्पितस्य दूराद् & T.A.6.11.1       \\
    
    \hline
        
    427 & यथ्{-}षष्ठेऽहन्{-}प्रवृज्यते ऋतुर्{-}भूत्वा सम्ॅवथ्सरमेति यथ्{-}सप्तमेऽहन्{-}प्रवृज्यते & T.A.8.12.2       \\
    
    \hline
        
    428 & यददीव्यन्नृणमहं बभूवादिथ् सन्वा संजगर & T.A.2.4.1       \\
    
    \hline
        
    429 & यदपां क्रूरं ॅयदमेद्ध्यं ॅयदशान्तं & T.A.6.1.13       \\
    
    \hline
        
    430 & यदस्याः समभरन्न् तथ् सम्राज्ञ्ः & T.A.8.1.6       \\
    
    \hline
        
    431 & यदीषितो यदि वा स्वकामी & T.A.7.31.1       \\
    
    \hline
        
    432 & यदेकादशेऽहन्{-}प्रवृज्यते इन्द्रो भूत्वा त्रिष्टुभमेति & T.A.8.12.3       \\
    
    \hline
        
    433 & यदेतद् भूतान्यन्वाविश्य दैवीं ॅवाचं & T.A.7.34.1       \\
    
    \hline
        
    434 & यदेतद्{-}वृकसो भूत्वा वाग् देव्यभिरायसि & T.A.7.30.1       \\
    
    \hline
        
    435 & यद्{-}देवा देवहेडनं देवहेळनं देवासश्चकृमा & T.A.2.3.1       \\
    
    \hline
        
    436 & यद्{-}यजुषा जुहुयात् अयथापूर्वमाहुती जुहुयात् & T.A.8.8.11       \\
    
    \hline
        
    437 & यद्{-}वल्मीकम् यद्{-}वल्मीकवपा सम्भारो भवति & T.A.8.2.9       \\
    
    \hline
        
    438 & यद्वै देवस्य सवितुः पवित्रं & T.A.4.3.3       \\
    
    \hline
        
    439 & यद्वो देवाश्चकृम जिह्वया गुरुमनसो & T.A.6.60.1       \\
    
    \hline
        
    440 & यमो ददात्व{-}वसानमस्मै नृ मुणन्तु & T.A.1.27.6       \\
    
    \hline
        
    441 & यम् ते अग्निममन्थाम वृषभायेव & T.A.4.4.1       \\
    
    \hline
        
    442 & यम् द्विष्यात् यत्र स & T.A.8.10.5       \\
    
    \hline
        
    443 & यशसा ब्रह्मवर्चसेन अन्नाद्येन समिन्तां & T.A.7.41.6       \\
    
    \hline
        
    444 & यशो जनेऽसानि स्वाहा श्रेयान. & T.A.5.4.3       \\
    
    \hline
        
    445 & यश्छन्दसा{-}मृषभो विश्वरूप{-}श्छन्दोभ्य श्छन्दाꣳस्या विवेश & T.A.6.8.1       \\
    
    \hline
        
    446 & यस्ता विजानाथ् सवितुः पिताऽसत् & T.A.1.11.5       \\
    
    \hline
        
    447 & यस्य वै कङ्कत्यग्नि{-}होत्रहवणी भवति & T.A.6.26.1       \\
    
    \hline
        
    448 & या पुरस्ताद् विद्युदापतत् तान्त & T.A.7.14.1       \\
    
    \hline
        
    449 & याश्च वासुकि वैद्युताः रजताः & T.A.1.9.2       \\
    
    \hline
        
    450 & यास्तिस्रः परमजाः इन्द्रघोषा वो & T.A.1.25.3       \\
    
    \hline
        
    451 & यास्ते अग्न आर्द्रा योनयो & T.A.7.18.1       \\
    
    \hline
        
    452 & यास्ते अग्ने घोरास्तनुवः क्षुच्च & T.A.7.22.1       \\
    
    \hline
        
    453 & युञ्जते मन उत युञ्जते & T.A.7.2.1       \\
    
    \hline
        
    454 & ये तत्र ब्राह्मणाः सम्मर.शिनः & T.A.5.11.4       \\
    
    \hline
        
    455 & ये ते सहस्रमयुतं पाशा & T.A.6.57.1       \\
    
    \hline
        
    456 & ये नखाः ते वैखानसाः & T.A.1.23.3       \\
    
    \hline
        
    457 & यो वै वसीयाꣳ सम्ॅयथानाममुपचरति & T.A.8.11.2       \\
    
    \hline
        
    458 & योऽपां पुष्पं ॅवेद पुष्पवान् & T.A.1.22.1       \\
    
    \hline
        
    459 & योऽपामायतनं ॅवेद आयतनवान् भवति & T.A.1.22.6       \\
    
    \hline
        
    460 & योऽसौ तपन्नुदेति स सर्वेषां & T.A.1.14.1       \\
    
    \hline
        
    461 & योऽस्य कौष्ठ्य जगतः पार्थिवस्यैक & T.A.4.5.2       \\
    
    \hline
        
    462 & रक्षाꣳसि हवा पुरोनुवाके तपोऽग्रमतिष्ठन्त & T.A.2.2.1       \\
    
    \hline
        
    463 & रक्षोदेवजनेभ्यः स्वाहा गृह्याभ्यः स्वाहा & T.A.6.67.2       \\
    
    \hline
        
    464 & रन्तिर्{-}नामासि दिव्यो गन्धर्वः तस्य & T.A.7.11.5       \\
    
    \hline
        
    465 & रिच्यत इव वा एष & T.A.2.16.1       \\
    
    \hline
        
    466 & रुचितमवेक्षन्ते रुचिताद्वै प्रजापतिः प्रजा & T.A.8.6.11       \\
    
    \hline
        
    467 & रुद्राय रुद्रहोत्रे स्वाहेत्याह रुद्रमेव & T.A.8.8.9       \\
    
    \hline
        
    468 & रोचितस्त्वम् देव घर्म देवेष्वसीत्याह & T.A.8.5.3       \\
    
    \hline
        
    469 & रोहिणीः पिङ्गला एकरूपाः क्षरन्तीः & T.A.3.11.10       \\
    
    \hline
        
    470 & लोकम् पृण ता अस्य & T.A.4.9.2       \\
    
    \hline
        
    471 & वज्रमेव भ्रातृव्येभ्यः प्रहरति स्तृणुत & T.A.1.26.6       \\
    
    \hline
        
    472 & वयः सुपर्णा उपसेदुरिन्द्रं प्रिय & T.A.6.73.1       \\
    
    \hline
        
    473 & वयमनुक्रामाम सुविताय नव्यसे ब्रह्मणस्त्वा & T.A.7.11.3       \\
    
    \hline
        
    474 & वरुणस्य विराट् यज्ञ्स्य पङ्क्तिः & T.A.3.9.2       \\
    
    \hline
        
    475 & वर्चसा श्रिया यशसा ब्रह्मवर्चसेन & T.A.7.41.5       \\
    
    \hline
        
    476 & वसवः प्रवृक्तः सोमोऽभिकीर्यमाणः आश्विनः & T.A.8.11.4       \\
    
    \hline
        
    477 & वाग्घोता दीक्षा पत्नी वातोऽद्ध्वर्युः & T.A.3.6.1       \\
    
    \hline
        
    478 & वाङ्म आसन्न् नसोः प्राणः & T.A.6.72.1       \\
    
    \hline
        
    479 & वातं प्राणं मनसा न्वारभामहे & T.A.6.47.1       \\
    
    \hline
        
    480 & वातरशना ह वा ऋषयः & T.A.2.7.1       \\
    
    \hline
        
    481 & वामदेवाय नमो ज्येष्ठाय नमः & T.A.6.18.1       \\
    
    \hline
        
    482 & वायुः सन्धानं इत्यधि लोकं & T.A.5.3.2       \\
    
    \hline
        
    483 & वायुस्ताꣳ अग्रे प्रमुमोक्तु देवः & T.A.3.11.13       \\
    
    \hline
        
    484 & वारेवृतꣳ ह्यासाम् तस्य ज्यामप्यादन्न् & T.A.8.1.5       \\
    
    \hline
        
    485 & विगा इन्द्र विचरन्थ् स्पाशयस्व & T.A.7.28.1       \\
    
    \hline
        
    486 & विज्ञानं ब्रह्मेति व्यजानात् विज्ञाना{-}द्ध्येव & T.A.5.15.5       \\
    
    \hline
        
    487 & विज्ञानं ॅयज्ञ्ं तनुते कर्माणि & T.A.5.14.5       \\
    
    \hline
        
    488 & विद्युन् महसो धूपयः श्वापयो & T.A.1.9.5       \\
    
    \hline
        
    489 & विधाय लोकान्. विधाय भूतानि & T.A.1.23.9       \\
    
    \hline
        
    490 & विवेशापराजिता पराङ्गेत्य पराङत्य ज्यामयी & T.A.1.27.4       \\
    
    \hline
        
    491 & विशीर्ष्णीं गृद्ध्र{-}शीर्ष्णीञ्च अपेतो निर्.ऋतिं & T.A.1.28.1       \\
    
    \hline
        
    492 & विश्वा आशा दक्षिणसदित्याह विश्वानेव & T.A.8.8.1       \\
    
    \hline
        
    493 & विश्वा हि माया अवथः & T.A.1.10.2       \\
    
    \hline
        
    494 & विषुरूपे अहनी द्यौरिवासि विश्वा & T.A.7.5.7       \\
    
    \hline
        
    495 & वीणा पण वलासितं मृतञ्जीवञ्च & T.A.1.11.6       \\
    
    \hline
        
    496 & वीर्यम् ॅवै छन्दाꣳसि वीर्येणैवैनम् & T.A.8.3.4       \\
    
    \hline
        
    497 & वृषा हरिः महान्{-}मित्रो न & T.A.8.9.9       \\
    
    \hline
        
    498 & वृष्टिकामश्चिन्वीत आपो वै वृष्टिः & T.A.1.26.5       \\
    
    \hline
        
    499 & वृष्णो अश्वस्य निष्पदसि वरुणस्त्वा & T.A.7.3.1       \\
    
    \hline
        
    500 & वेदमनूच्या{-} ऽचार्योन्तेवासिन{-}मनुशास्ति सत्यं ॅवद & T.A.5.11.1       \\
    
    \hline
        
    501 & वेदाहमेतं पुरुषं महान्तम् आदित्यवर्णं & T.A.3.12.7       \\
    
    \hline
        
    502 & वैश्वानराय प्रतिवेदयामो यदीनृणꣳ सङ्गरो & T.A.2.6.1       \\
    
    \hline
        
    503 & वैश्वानरे हविरिदम् जुहोमि साहस्र{-}मुथ्सं & T.A.4.6.1       \\
    
    \hline
        
    504 & व्यष्टभ्ना{-}द्रोदसी विष्णवेते दाधर्थ पृथिवी{-}मभितो & T.A.1.8.3       \\
    
    \hline
        
    505 & व्यसौ योऽस्मान् द्वेष्टि यञ्च & T.A.7.11.6       \\
    
    \hline
        
    506 & शन्नो देवीरभिष्टय आपो भवन्तु & T.A.7.42.4       \\
    
    \hline
        
    507 & शन्नो मित्रः शं ॅवरुणः & T.A.5.1.1 T.A.5.12.1       \\
    
    \hline
        
    508 & शन्नो वातः पवताम् मातरिश्वा & T.A.7.42.1       \\
    
    \hline
        
    509 & शल्कैरग्नि{-}मिन्धान उभौ लोकौ सनेमहं & T.A.6.50.1       \\
    
    \hline
        
    510 & शिरो वा एतद्{-}यज्ञ्स्य यत् & T.A.8.6.1       \\
    
    \hline
        
    511 & शिवेन मे सन्तिष्ठस्व स्योनेन & T.A.6.77.1       \\
    
    \hline
        
    512 & शिशुर्{-}जनधायाः शञ्च वक्षि परि & T.A.7.11.4       \\
    
    \hline
        
    513 & शीक्षां ॅव्याख्यास्यामः वर्णः स्वरः & T.A.5.2.1       \\
    
    \hline
        
    514 & शुनम् ॅवाहाः शुनम् नाराः & T.A.4.6.2       \\
    
    \hline
        
    515 & श्रद्धायां प्राणे निविश्याऽमृतꣳ हुतं & T.A.6.70.1       \\
    
    \hline
        
    516 & श्रद्धायां प्राणे निविष्टोऽमृतं जुहोमि & T.A.6.69.1       \\
    
    \hline
        
    517 & स खदिरोऽभवत् यः पशून् & T.A.8.2.4       \\
    
    \hline
        
    518 & स नः सुवः सं & T.A.6.1.15       \\
    
    \hline
        
    519 & स मा रुचितो रोचयेत्याह & T.A.8.5.2       \\
    
    \hline
        
    520 & स मे ददातु प्रजां & T.A.3.7.2 T.A.3.7.4       \\
    
    \hline
        
    521 & स य एषोऽन्तर्.{-}हृदय आकाशः & T.A.5.6.1       \\
    
    \hline
        
    522 & सतो बन्धुमसति निरविन्दन्न् हृदि & T.A.1.23.2       \\
    
    \hline
        
    523 & सत्यं परं परं सत्यं & T.A.6.78.1       \\
    
    \hline
        
    524 & सत्रिय मग्निञ्चिन्वानः कमग्निञ्चिनुते सावित्र & T.A.1.26.2       \\
    
    \hline
        
    525 & सद्योजातं प्रपद्यामि सद्योजाताय वै & T.A.6.17.1       \\
    
    \hline
        
    526 & सप्त होतार ऋत्विजः देवा & T.A.1.7.5       \\
    
    \hline
        
    527 & समुद्रा{-}दर्णवा{-}दधि सम्ॅवथ्सरो अजायत अहोरात्राणि & T.A.6.1.14       \\
    
    \hline
        
    528 & समुद्राय त्वा वाताय स्वाहा & T.A.7.9.1       \\
    
    \hline
        
    529 & सम्मा असि विमा असि & T.A.7.5.6       \\
    
    \hline
        
    530 & सम्ॅवथ्सरन्न माꣳसमश्ञीयात् न रामामुपेयात् & T.A.8.8.13       \\
    
    \hline
        
    531 & सम्ॅवथ्सरमेतद् व्रतञ्चरेत् द्वौ वा & T.A.1.32.1       \\
    
    \hline
        
    532 & सर्वभूताधिपतये नम इति अथ & T.A.1.31.3       \\
    
    \hline
        
    533 & सर्वो वै रुद्रस्तस्मै रुद्राय & T.A.6.24.1       \\
    
    \hline
        
    534 & सविता भूत्वा प्रथमेऽहन् प्रवृज्यते & T.A.8.12.1       \\
    
    \hline
        
    535 & सवितैतानि शरीराणि पृथिव्यै मातुरुपस्थ & T.A.4.7.3       \\
    
    \hline
        
    536 & सह ना ववतु सह & T.A.5.13.1       \\
    
    \hline
        
    537 & सह नौ यशः सह & T.A.5.3.1       \\
    
    \hline
        
    538 & सह वै देवानां चासुराणां & T.A.2.1.1       \\
    
    \hline
        
    539 & सहस्रवृदियं भूमिः परं ॅव्योम & T.A.1.10.1       \\
    
    \hline
        
    540 & सहस्रशीर्.षं देवं ॅविश्वाक्षं ॅविश्व & T.A.6.13.1       \\
    
    \hline
        
    541 & सहस्रशीर्.षा पुरुषः सहस्राक्षः सहस्रपात् & T.A.3.12.1       \\
    
    \hline
        
    542 & साकञ्जानाꣳ सप्तथमाहु{-}रेकजं षडुद्यमा ऋषयो & T.A.1.3.1       \\
    
    \hline
        
    543 & सावित्रम् जुहोति प्रसूत्यै चतुर्गृहीतेन & T.A.8.2.1       \\
    
    \hline
        
    544 & साऽऽदित्यꣳ समिन्धे तामहꣳ समिन्धे & T.A.7.41.3       \\
    
    \hline
        
    545 & सुवरसि सुवर्मे यच्छ दिवं & T.A.8.7.9       \\
    
    \hline
        
    546 & सुवरित्यादित्ये मह इति ब्रह्मणि & T.A.5.6.2       \\
    
    \hline
        
    547 & सुवर्णं घर्मं परिवेद वेनम् & T.A.3.11.1       \\
    
    \hline
        
    548 & सूपसदा मे भूया मा & T.A.7.5.5       \\
    
    \hline
        
    549 & सूर्यं ते चक्षुः वातं & T.A.3.4.1       \\
    
    \hline
        
    550 & सूर्यश्च मा मन्युश्च मन्युपतयश्च & T.A.6.32.1       \\
    
    \hline
        
    551 & सेनेन्द्रस्य धेना बृहस्पतेः पत्थ्या & T.A.3.9.1       \\
    
    \hline
        
    552 & सोऽजिह्वो असश्चत नैतमृषिं ॅविदित्वा & T.A.1.11.7       \\
    
    \hline
        
    553 & स्तौत्येवैनमेतत् गायत्रमसि त्रैष्टुभमसि जागतमसीति & T.A.8.4.11       \\
    
    \hline
        
    554 & स्निक्च स्नीहितिश्च स्निहितिश्च उष्णा & T.A.7.23.1       \\
    
    \hline
        
    555 & स्मृतिः प्रत्यक्ष{-}मैतिह्यं अनुमान{-}श्चतुष्टयं एतैरादित्य & T.A.1.2.1       \\
    
    \hline
        
    556 & स्वयैवैनम् देवतयोपावसृजति अश्विभ्याम् प्रदापयेत्याह & T.A.8.7.3       \\
    
    \hline
        
    557 & स्वस्ति नो मघवा करोतु & T.A.6.1.11       \\
    
    \hline
        
    558 & स्वस्तिदा विशस्पतिर् वृत्रहा विमृधो & T.A.6.55.1       \\
    
    \hline
        
    559 & स्वाहा त्वा सूर्यस्य रश्मये & T.A.8.7.7       \\
    
    \hline
        
    560 & स्वाहा मरुद्भिः परिश्रयस्वेत्याह अमुमेवादित्यं & T.A.8.4.9       \\
    
    \hline
        
    561 & स्वाहा समग्निस्तपसा गत सम् & T.A.7.7.2       \\
    
    \hline
        
    562 & हरिꣳ हरन्त{-} मनुयन्ति देवा & T.A.6.49.1       \\
    
    \hline
        
    563 & हरिꣳ हरन्तमनुयन्ति देवाः विश्वस्येशानं & T.A.3.15.1       \\
    
    \hline
        
    564 & हृदा पश्यन्ति मनसा मनीषिणः & T.A.3.11.11       \\
    
    \hline
        
    565 & हꣳसः शुचिषद् वसु{-}रन्तरिक्ष{-}सद्धोता वेदिष{-}दतिथिर्{-}दुरोणसत् & T.A.6.12.2       \\
    
    \hline
        \bottomrule
  \end{longtable}
  
\end{document}