\documentclass[17pt]{extarticle}
\usepackage{babel}
\usepackage{fontspec}
\usepackage{polyglossia}
\usepackage{extsizes}

\usepackage{booktabs} % To thicken table lines



\setmainlanguage{sanskrit}
\setotherlanguages{english} %% or other languages
\setlength{\parindent}{0pt}
\pagestyle{myheadings}
\newfontfamily\devanagarifont[Script=Devanagari]{AdishilaVedic}


\newcommand{\VAR}[1]{}
\newcommand{\BLOCK}[1]{}

\usepackage{longtable} % To display tables on several pages

\begin{document} 


\begin{longtable}{||p{0.4in}||p{0.9in}||p{4.0in}||p{0.9in}||} % <-- Replaces \begin{table}, alignment must be specified here (no more tabular)
    \caption{कृष्ण यजुर्वेदीय तैत्तिरीय ब्राह्मणे}
    \label{tab:table1}\\
    \toprule
    \textbf{SNo} & \textbf{Dasini} & \textbf{Beginning Words} & \textbf{Other Dasini beginning with same Words}
    
   
    \endfirsthead % <-- This denotes the end of the header, which will be shown on the first page only
    \toprule
    \textbf{SNo} & \textbf{Dasini} & \textbf{Beginning Words} & \textbf{Other Dasini beginning with same Words}
    
   
    \endhead % <-- Everything between \endfirsthead and \endhead will be shown as a header on every page
            1 & TB\_1.1.1.1 & ब्रह्म संधत्तं तन्मे जिन्वतम् &      \\
        \hline
            2 & TB\_1.1.1.2 & सुवीराः प्रजाः प्रजनयन् परीहि &      \\
        \hline
            3 & TB\_1.1.1.3 & व्यानꣳ संधत्तं तं मे &      \\
        \hline
            4 & TB\_1.1.1.4 & प्राणं ॅयज्ञ्पतये धत्तम् चक्षुः &      \\
        \hline
            5 & TB\_1.1.1.5 & इष{-}मूर्जमस्मासु धत्तम् प्राणान् पशुषु &      \\
        \hline
            6 & TB\_1.1.2.1 & कृत्तिका{-}स्वग्नि{-}मादधीत एतद्वा अग्नेर् नक्षत्रम् &      \\
        \hline
            7 & TB\_1.1.2.2 & अग्नि{-}नक्षत्रमित्य{-}पचायन्ति गृहान्.ह दाहुको भवति &      \\
        \hline
            8 & TB\_1.1.2.3 & तेषामनाहितो ऽग्निरासीत् अथैभ्यो वामं &      \\
        \hline
            9 & TB\_1.1.2.4 & अर्यम्णो वा एतन्नक्षत्रम् यत्पूर्वे &      \\
        \hline
            10 & TB\_1.1.2.5 & ते सुवर्गाय लोकायाग्नि{-}मचिन्वत पुरुष &      \\
        \hline
            11 & TB\_1.1.2.6 & तौ दिव्यौ श्वानावभवताम् यो &      \\
        \hline
            12 & TB\_1.1.2.7 & यद्{-}वसन्तः यो वसन्ताऽग्निमाधत्ते मुख्य &      \\
        \hline
            13 & TB\_1.1.2.8 & स्व एवैनमृतावाधाय पशुमान् भवति &      \\
        \hline
            14 & TB\_1.1.3.1 & उद्धन्ति यदेवास्या अमेद्ध्यम् तदप &      \\
        \hline
            15 & TB\_1.1.3.2 & पुष्ट्यामेव प्रजनने ऽग्निमाधत्ते अथो &      \\
        \hline
            16 & TB\_1.1.3.3 & यदस्या यज्ञियमासीत् तदमुष्या{-}मदधात् तददश्चन्द्रमसि &      \\
        \hline
            17 & TB\_1.1.3.4 & यदाखुकरीषꣳ सभांरो भवति यदेवास्य &      \\
        \hline
            18 & TB\_1.1.3.5 & अबधिरो भवति य एवं &      \\
        \hline
            19 & TB\_1.1.3.6 & कथमिदꣳ स्यादिति सोऽपश्यत् पुष्करपर्णं &      \\
        \hline
            20 & TB\_1.1.3.7 & तत् पृथिव्यै पृथिवित्वम् अभूद्वा &      \\
        \hline
            21 & TB\_1.1.3.8 & अथो शत्वांय सरेता अग्निराधेय &      \\
        \hline
            22 & TB\_1.1.3.9 & उत्तरत उपास्यत्य{-}बीभथ्सायै अति प्रयच्छति &      \\
        \hline
            23 & TB\_1.1.3.10 & देवा वा ऊर्जं ॅव्यभजन्त &      \\
        \hline
            24 & TB\_1.1.3.11 & यस्य पर्णमयः संभारो भवति &      \\
        \hline
            25 & TB\_1.1.3.12 & तच्छम्यै शमित्वम् यच्छमीमयः संभारो &      \\
        \hline
            26 & TB\_1.1.4.1 & द्वादशसु विक्रामेष्वग्निमादधीत द्वादश मासाः &      \\
        \hline
            27 & TB\_1.1.4.2 & चक्षुर् वै सत्यम् अद्रा3गित्याह &      \\
        \hline
            28 & TB\_1.1.4.3 & आग्नेयाः पशवः ऐन्द्रमहः नक्तं &      \\
        \hline
            29 & TB\_1.1.4.4 & इडा वै मानवी यज्ञानूकाशिन्यासीत् &      \\
        \hline
            30 & TB\_1.1.4.5 & यस्यैव{-}मग्निराधीयते प्रतीच्यस्य श्रीरेति भद्रो &      \\
        \hline
            31 & TB\_1.1.4.6 & प्राच्येषाꣳ श्रीरगात् भद्रा भूत्वा &      \\
        \hline
            32 & TB\_1.1.4.7 & प्रत्यस्मिन् ॅलोके स्थास्यसि अभि &      \\
        \hline
            33 & TB\_1.1.4.8 & यस्यैवमग्नि{-}राधीयते प्र प्रजया पशुभिः &      \\
        \hline
            34 & TB\_1.1.4.9 & भृगूणां त्वाऽङ्गिरसां ॅव्रतपते व्रतेना{-}दधामीति &      \\
        \hline
            35 & TB\_1.1.5.1 & प्रजापतिर् वाचः सत्य{-}मपश्यत् तेनाग्निमाधत्त &      \\
        \hline
            36 & TB\_1.1.5.2 & भूरित्याह प्रजा एव तद्{-}यजमानः &      \\
        \hline
            37 & TB\_1.1.5.3 & सुवर्गाय वा एष लोकायाधीयते &      \\
        \hline
            38 & TB\_1.1.5.4 & प्रजापतिः प्रजा असृजत ता &  TB\_2.7.9.1 TB\_3.1.4.2       \\
        \hline
            39 & TB\_1.1.5.5 & एष वै प्रजापतिः यदग्निः &      \\
        \hline
            40 & TB\_1.1.5.6 & जनिष्यमाणानेव प्रतिनुदते न्याहवनीयो गार्.हपत्य{-}मकामयत &      \\
        \hline
            41 & TB\_1.1.5.7 & यदुपर्युपरि शिरो हरेत् प्राणान् &      \\
        \hline
            42 & TB\_1.1.5.8 & तस्य त्रेधा महिमानं ॅव्यौहत् &      \\
        \hline
            43 & TB\_1.1.5.9 & यदग्निः यदश्वस्य पदेऽग्नि{-}मादद्ध्यात् रुद्राय &      \\
        \hline
            44 & TB\_1.1.5.10 & त्रीणि हवीꣳषि निर्वपति विराज &      \\
        \hline
            45 & TB\_1.1.6.1 & देवासुराः संॅयत्ता आसन्न् ते &      \\
        \hline
            46 & TB\_1.1.6.2 & तद्{-}देवा विजित्य पुनरवारुरुथ्सन्त तेऽग्नये &      \\
        \hline
            47 & TB\_1.1.6.3 & तेऽग्नये शुचये असौ वा &      \\
        \hline
            48 & TB\_1.1.6.4 & नाङ्गानि तादृगेव तत् यदेतानि &      \\
        \hline
            49 & TB\_1.1.6.5 & योऽग्निमाधत्ते ऐन्द्राग्न{-}मेकादशकपाल{-} मनु निर्वपेत् &      \\
        \hline
            50 & TB\_1.1.6.6 & यदाज्यम् अनडुहस्तण्डुलाः मिथुनमेवाव{-}रुन्धे घृते &      \\
        \hline
            51 & TB\_1.1.6.7 & यथ्सद्य एतानि हवीꣳषि निर्वपेत् &      \\
        \hline
            52 & TB\_1.1.6.8 & यथा त्रीण्यावपनानि पूरयेत् तादृक्तत् &      \\
        \hline
            53 & TB\_1.1.6.9 & ब्रह्मवादिनो वदन्ति होतव्य{-}मग्निहोत्रां3 नहोतव्या3मिति &      \\
        \hline
            54 & TB\_1.1.6.10 & अग्निमुखा{-}नेवर्तून् प्रीणाति उपबर्.हणं ददाति &      \\
        \hline
            55 & TB\_1.1.6.11 & वह्निनैव वह्नि यज्ञ्स्याव{-}रुन्धे मिथुनौ &      \\
        \hline
            56 & TB\_1.1.7.1 & घर्मः शिरस्तदयमग्निः सं प्रियः &      \\
        \hline
            57 & TB\_1.1.7.2 & अःकश्चक्षु{-}स्तदसौ सूःयस्तदयमग्निः संप्रियः पशुभिः &      \\
        \hline
            58 & TB\_1.1.7.3 & ये ते अग्ने शिवे &      \\
        \hline
            59 & TB\_1.1.8.1 & इमे वा एते लोका &      \\
        \hline
            60 & TB\_1.1.8.2 & अस्मिन्नेवैनं ॅलोके प्रतिष्ठित{-}माधत्ते वामदेव्य{-}मभिगायत &      \\
        \hline
            61 & TB\_1.1.8.3 & सोऽश्वो वारो भूत्वा पराङैत् &      \\
        \hline
            62 & TB\_1.1.8.4 & उपैनमुत्तरो यज्ञो नमति रुद्रो &      \\
        \hline
            63 & TB\_1.1.8.5 & सप्राण{-}मेवैन{-}माधत्ते स्वदितं तोकाय तनयाय &      \\
        \hline
            64 & TB\_1.1.8.6 & अन्नमेवाव रुन्धे तेन मे &      \\
        \hline
            65 & TB\_1.1.9.1 & शमीगर्भादग्निं मन्थति एषा वा &      \\
        \hline
            66 & TB\_1.1.9.2 & तस्या उच्छेषण{-}मददुः तत् प्राश्ञात् &  TB\_1.1.9.3       \\
        \hline
            67 & TB\_1.1.9.3 & तस्या उच्छेषण{-}मददुः तत् प्राश्ञात् & TB\_1.1.9.2        \\
        \hline
            68 & TB\_1.1.9.4 & उच्छेषणादेव तद्{-}रेतो धत्ते अस्थि &      \\
        \hline
            69 & TB\_1.1.9.5 & इयतीर् भवन्ति यज्ञ्परुषा संमिताः &      \\
        \hline
            70 & TB\_1.1.9.6 & एतद्वा अग्नेः प्रियं धाम &      \\
        \hline
            71 & TB\_1.1.9.7 & जगतीभिर्{-}वैश्यस्य जगती छन्दा वै &      \\
        \hline
            72 & TB\_1.1.9.8 & यन् माꣳस{-}मश्ञीयात् यथ्स्त्रिय{-}मुपेयात् निर्वीर्यः &      \\
        \hline
            73 & TB\_1.1.9.9 & नैनं प्रतिनुदन्ते ब्रह्मवादिनो वदन्ति &      \\
        \hline
            74 & TB\_1.1.9.10 & सैव साऽग्नेः सन्ततिः तं &      \\
        \hline
            75 & TB\_1.1.10.1 & प्रजापतिः प्रजा असृजत स &      \\
        \hline
            76 & TB\_1.1.10.2 & दोहा एव युष्माकमिति सा &      \\
        \hline
            77 & TB\_1.1.10.3 & सा चतुर्थ{-}मुदक्रामत् तत् प्रजापतिः &      \\
        \hline
            78 & TB\_1.1.10.4 & तामात्मनोऽधि निर्मिमीते यदग्नि{-}राधीयते तस्मा{-}देतावन्तो{-}ऽग्नय &      \\
        \hline
            79 & TB\_1.1.10.5 & पशूनेवैतेन स्पृणोति सप्रथ सभां &      \\
        \hline
            80 & TB\_1.1.10.6 & तेन सोऽस्याभीष्टः प्रीतः यथ्सभायां &      \\
        \hline
            81 & TB\_1.2.1.1 & उद्धन्यमान{-}मस्या अमेद्ध्यम् अप पाप्मानं &      \\
        \hline
            82 & TB\_1.2.1.2 & यदिदं दिवो यददः पृथिव्याः &      \\
        \hline
            83 & TB\_1.2.1.3 & वम्रीभि{-}रनुवित्तं गुहासु श्रोत्रं त &      \\
        \hline
            84 & TB\_1.2.1.4 & यत् पर्यपश्यथ् सरिरस्य मद्ध्ये &      \\
        \hline
            85 & TB\_1.2.1.5 & अति प्रयच्छं दुरितिं तरेयम् &      \\
        \hline
            86 & TB\_1.2.1.6 & पर्णमपतत् तृतीयस्यै दिवोऽधि सोऽयं &      \\
        \hline
            87 & TB\_1.2.1.7 & शमीꣳ शान्त्यै हराम्यहम् यत्ते &      \\
        \hline
            88 & TB\_1.2.1.8 & शरीरमभि सꣳ स्कृताः स्थ &      \\
        \hline
            89 & TB\_1.2.1.9 & यज्ञियैः केतुभिः सह यं &      \\
        \hline
            90 & TB\_1.2.1.10 & घृतैर् बोधयतातिथिम् आऽस्मिन्. हव्या &      \\
        \hline
            91 & TB\_1.2.1.11 & शोचिष्केशो घृतनिःणिक्{-}पावकः सुयज्ञो अग्निः &      \\
        \hline
            92 & TB\_1.2.1.12 & त्वामग्ने समिधानं ॅयविष्ठ देवा &      \\
        \hline
            93 & TB\_1.2.1.13 & इन्धानो अक्रो विदथेषु दीद्यत् &      \\
        \hline
            94 & TB\_1.2.1.14 & आरोहतं दशतꣳ शक्वरीर् मम &      \\
        \hline
            95 & TB\_1.2.1.15 & प्रजया पशुभिर्{-}ब्रह्मवर्चसेन सुवर्गे लोके &      \\
        \hline
            96 & TB\_1.2.1.16 & अग्निमश्वत्थादधि हव्यवाहम् शमीगर्भा{-}ज्जनयन.यो मयो &      \\
        \hline
            97 & TB\_1.2.1.17 & अग्नेर्भस्मास्यग्नेः पुरीषमसि सज्ञांनमसि कामधरणम् &      \\
        \hline
            98 & TB\_1.2.1.18 & कल्पेतां द्यावापृथिवी कल्पन्तामाप ओषधीः &      \\
        \hline
            99 & TB\_1.2.1.19 & अन्तरिक्षस्य पोषेण सर्वपशु{-}मादधे अजीजनन्{-}नमृतं &      \\
        \hline
            100 & TB\_1.2.1.20 & जीवात्वै पुण्याय अहं त्वदस्मि &      \\
        \hline
            101 & TB\_1.2.1.21 & अग्ने सपत्नाꣳ अप बाधमानः &      \\
        \hline
            102 & TB\_1.2.1.22 & यस्त आत्मा पशुषु प्रविष्टः &      \\
        \hline
            103 & TB\_1.2.1.23 & ऊर्जं नो धेहि द्विपदे &      \\
        \hline
            104 & TB\_1.2.1.24 & तयोः पृष्ठे सीदतु जातवेदाः &      \\
        \hline
            105 & TB\_1.2.1.25 & नर्य प्रजां मे गोपाय &      \\
        \hline
            106 & TB\_1.2.1.26 & अष्टाशफाश्च य इहाग्ने ये &      \\
        \hline
            107 & TB\_1.2.1.27 & चतुः शिखण्डा युवतिः सुपेशाः &      \\
        \hline
            108 & TB\_1.2.2.1 & नवैतान्यहानि भवन्ति नव वै &      \\
        \hline
            109 & TB\_1.2.2.2 & पशवो वा उक्थानि पशूनामव{-}रुद्ध्यै &      \\
        \hline
            110 & TB\_1.2.2.3 & सुवर्गमेव तेन लोकमभि जयन्ति &      \\
        \hline
            111 & TB\_1.2.2.4 & अतिग्राह्याः परः सामसु इमानेवैतैः &      \\
        \hline
            112 & TB\_1.2.2.5 & त्रयस्त्रिꣳशद्वै देवताः देवता एवावरुन्धते &      \\
        \hline
            113 & TB\_1.2.3.1 & संततिर्वा एते ग्रहाः यत् &      \\
        \hline
            114 & TB\_1.2.3.2 & एवꣳ संॅवथ्सरस्य पक्षसी दिवाकीर्त्यमभि &      \\
        \hline
            115 & TB\_1.2.3.3 & सप्त वै शीर्.षण्याः प्राणाः &      \\
        \hline
            116 & TB\_1.2.3.4 & सुवर्गस्य लोकस्या{-}भिजित्यै प्र वा &      \\
        \hline
            117 & TB\_1.2.4.1 & एकविꣳश एष भवति एतेन &      \\
        \hline
            118 & TB\_1.2.4.2 & देवा वा आदित्यस्य सुवर्गस्य &      \\
        \hline
            119 & TB\_1.2.4.3 & महादिवाकीर्त्यꣳ होतुः पृष्ठम् विकर्णं &      \\
        \hline
            120 & TB\_1.2.5.1 & अप्रतिष्ठां ॅवा एते गच्छन्ति &      \\
        \hline
            121 & TB\_1.2.5.2 & वैश्वदेव{-}मालभन्ते देवता एवाव{-}रुन्धते द्यावापृथिव्यां &      \\
        \hline
            122 & TB\_1.2.5.3 & मित्रेणैव यज्ञ्स्य स्विष्टꣳ शमयन्ति &      \\
        \hline
            123 & TB\_1.2.5.4 & यदतिरिक्ता{-}मेकादशिनी{-}मालभेरन्न् अप्रियं भ्रातृव्य{-}मभ्यतिरिच्येत यद् &      \\
        \hline
            124 & TB\_1.2.6.1 & प्रजापतिः प्रजाः सृष्ट्वा वृत्तोऽशयत् &      \\
        \hline
            125 & TB\_1.2.6.2 & चतुर्विꣳश{-}त्यर्द्धमासः संॅवथ्सरः यद् वा &      \\
        \hline
            126 & TB\_1.2.6.3 & त्रेधा विहितꣳ हि शिरः &      \\
        \hline
            127 & TB\_1.2.6.4 & पञ्चविꣳश आत्मा भवति तस्मान्{-}मद्ध्यतः &      \\
        \hline
            128 & TB\_1.2.6.5 & यदित इतो लोमानि दतो &      \\
        \hline
            129 & TB\_1.2.6.6 & यस्य तल्पसद्य{-}मभिजितꣳ स्यात् स &      \\
        \hline
            130 & TB\_1.2.6.7 & ब्राह्मणश्च शूद्रश्च चर्मकर्ते व्यायच्छेते &      \\
        \hline
            131 & TB\_1.3.1.1 & देवासुराः सं ॅयत्ता आसन्न् &      \\
        \hline
            132 & TB\_1.3.1.2 & ते सोम{-}मन्वविन्दन्न् तमघ्नन्न् तस्य &      \\
        \hline
            133 & TB\_1.3.1.3 & अनाग्नेयं ॅवा एतत् क्रियते &      \\
        \hline
            134 & TB\_1.3.1.4 & यथा सुप्तं बोधयति तादृगेव &      \\
        \hline
            135 & TB\_1.3.1.5 & न सभृंत्याः संभाराः न &      \\
        \hline
            136 & TB\_1.3.1.6 & तादृगेव तत् उच्चैः स्विष्टकृतमुथ् &      \\
        \hline
            137 & TB\_1.3.1.7 & तद्{-}वैश्वानरवत्{-}प्रजननवत्तर{-}मुपैतीति तदाहुः व्यृद्धं ॅवा &      \\
        \hline
            138 & TB\_1.3.2.1 & देवा वै यथादर्.शं ॅयज्ञानाहरन्त &      \\
        \hline
            139 & TB\_1.3.2.2 & तस्मिन्{-}नाजिमधावन्न् तं बृहस्पति{-}रुदजयत् तेनायजत &      \\
        \hline
            140 & TB\_1.3.2.3 & य एवं ॅविद्वान्. वाजपेयेन &      \\
        \hline
            141 & TB\_1.3.2.4 & वाज्येवैनं पीत्वा भवति आऽस्य &      \\
        \hline
            142 & TB\_1.3.2.5 & वाग्वै वाजस्य प्रसवः य &      \\
        \hline
            143 & TB\_1.3.2.6 & अप्येव नोऽत्रास्त्विति तेभ्य एता &      \\
        \hline
            144 & TB\_1.3.2.7 & तस्माद्{-}गायतश्च मत्तस्य च न &      \\
        \hline
            145 & TB\_1.3.3.1 & देवा वै यदन्यैर्{-}ग्रहैर्{-}यज्ञ्स्य नावारुन्धत &      \\
        \hline
            146 & TB\_1.3.3.2 & सर्व ऐन्द्रा भवन्ति एकधैव &      \\
        \hline
            147 & TB\_1.3.3.3 & एतन् मनुष्याणाम् यथ् सुरा &      \\
        \hline
            148 & TB\_1.3.3.4 & अन्नस्यैव शमलेन शमलं ॅयजमाना{-}दपहन्ति &      \\
        \hline
            149 & TB\_1.3.3.5 & पूर्वे सोमग्रहा गृह्यन्ते अपरे &      \\
        \hline
            150 & TB\_1.3.3.6 & सपृंचः स्थ सं मा &      \\
        \hline
            151 & TB\_1.3.3.7 & तस्माद् वाजपेययाजी पूतो मेद्ध्यो &      \\
        \hline
            152 & TB\_1.3.4.1 & ब्रह्मवादिनो वदन्ति नाग्निष्टोमो नोक्थ्यः &      \\
        \hline
            153 & TB\_1.3.4.2 & मारुत्या बृहतः स्तोत्रम् एतावन्तो &      \\
        \hline
            154 & TB\_1.3.4.3 & सुवर्गं ॅलोकꣳ षोडशिनः स्तोत्रेण &      \\
        \hline
            155 & TB\_1.3.4.4 & प्रजापतेराप्त्यै श्यामा एकरूपा भवन्ति &      \\
        \hline
            156 & TB\_1.3.4.5 & एतैः प्रचरति यज्ञ्स्याघाताय एकधा &      \\
        \hline
            157 & TB\_1.3.5.1 & सावित्रं जुहोति कर्मणः कर्मणः &      \\
        \hline
            158 & TB\_1.3.5.2 & इन्द्रस्य वज्रोऽसि वार्त्रघ्न इति &  TB\_1.7.6.8 TB\_1.7.9.1       \\
        \hline
            159 & TB\_1.3.5.3 & यदेवास्याफ्सु प्रविष्टम् तदेवावरुन्धे बहु &      \\
        \hline
            160 & TB\_1.3.5.4 & अपां न पादाशुहेमन्निति संमार्ष्टि &      \\
        \hline
            161 & TB\_1.3.6.1 & देवस्याहꣳ सवितुः प्रसवे बृहस्पतिना &      \\
        \hline
            162 & TB\_1.3.6.2 & वाजिनाꣳ साम गायते अन्नं &      \\
        \hline
            163 & TB\_1.3.6.3 & या दुन्दुभौ परमयैव वाचाऽवरां &      \\
        \hline
            164 & TB\_1.3.6.4 & सप्तदशः प्रजापतिः प्रजापते{-}राप्त्यै अर्वाऽसि &      \\
        \hline
            165 & TB\_1.3.6.5 & वाजिनो वाजं धावत काष्ठां &      \\
        \hline
            166 & TB\_1.3.6.6 & प्र वा एतेऽस्माल्{-}लोकाच्च्यवन्ते य &      \\
        \hline
            167 & TB\_1.3.6.7 & कृष्णलं कृष्णलं ॅवाजसृद्भ्यः प्रयच्छति &      \\
        \hline
            168 & TB\_1.3.6.8 & एतद्वै देवानां परममन्नम् यन्नीवाराः &      \\
        \hline
            169 & TB\_1.3.6.9 & यो वाजपेयेन यजते बार्.हस्पत्य &      \\
        \hline
            170 & TB\_1.3.7.1 & तार्प्यं ॅयजमानं परिधापयति यज्ञो &      \\
        \hline
            171 & TB\_1.3.7.2 & वाजस्याव{-}रुद्ध्यै जाय एहि सुवो &      \\
        \hline
            172 & TB\_1.3.7.3 & एवमिव हि प्रजापतिः समृद्ध्यै &      \\
        \hline
            173 & TB\_1.3.7.4 & द्वादश मासाः संॅवथ्सरः संॅवथ्सरमेव &      \\
        \hline
            174 & TB\_1.3.7.5 & यावत्{-}प्राणाः यावदेवास्यास्ति तेन सह &      \\
        \hline
            175 & TB\_1.3.7.6 & समहं प्रजया सं मया &      \\
        \hline
            176 & TB\_1.3.7.7 & पुरस्ताद्धि प्रतीचीन{-}मन्नमद्यते शीर्.षतो घ्नन्ति &      \\
        \hline
            177 & TB\_1.3.7.8 & अमृत एव सुवर्गे लोके &      \\
        \hline
            178 & TB\_1.3.8.1 & सप्तान्न{-}होमाञ्जुहोति सप्त वा अन्नानि &      \\
        \hline
            179 & TB\_1.3.8.2 & अवरुद्धेन व्यृद्ध्येत सर्वस्य समवदाय &      \\
        \hline
            180 & TB\_1.3.8.3 & पुरस्तात्{-}प्रत्यञ्च{-}मभिषिञ्चति पुरस्ताद्धि{-}प्रतीचीन{-}मन्नमद्यते शीर्.षतोऽभिषिञ्चति शीर्.षतो &      \\
        \hline
            181 & TB\_1.3.8.4 & इन्द्रिय{-}मेवास्मि{-}न्नेतेन दधाति बृहस्पतेस्त्वा साम्राज्येनाभिषिञ्चा{-}मीत्याह &      \\
        \hline
            182 & TB\_1.3.8.5 & इन्द्रियस्यावरुद्ध्यै अनिरुक्ताभिः प्रातस्सवने स्तुवते &      \\
        \hline
            183 & TB\_1.3.9.1 & नृषदं त्वेत्याह प्रजा वै &      \\
        \hline
            184 & TB\_1.3.9.2 & अफ्सुषदं त्वा घृतसद{-}मित्याह अपामेवैतेन &      \\
        \hline
            185 & TB\_1.3.9.3 & नाकसद{-}मित्याह यदावै वसीयान् भवति &      \\
        \hline
            186 & TB\_1.3.10.1 & इन्द्रो वृत्रꣳ हत्वा असुरान् &  TB\_1.7.1.6       \\
        \hline
            187 & TB\_1.3.10.2 & तमेभ्यः पुनरददुः तस्मात्{-}पितृभ्यः पूर्वेद्युः &      \\
        \hline
            188 & TB\_1.3.10.3 & एतद् वै ब्राह्मणं पुरा &      \\
        \hline
            189 & TB\_1.3.10.4 & षड्वा ऋतवः ऋतूनेव प्रीणाति &  TB\_1.6.8.8       \\
        \hline
            190 & TB\_1.3.10.5 & ऋतवः खलु वै देवाः &      \\
        \hline
            191 & TB\_1.3.10.6 & ह्लीका हि पितरः ऒष्मणो &      \\
        \hline
            192 & TB\_1.3.10.7 & पितृभ्य आवृश्च्येत अवघ्रेयमेव तन्नेव &      \\
        \hline
            193 & TB\_1.3.10.8 & नमस्करोति नमस्कारो हि पितृणाम् &      \\
        \hline
            194 & TB\_1.3.10.9 & युष्माꣳस्तेऽनु येऽस्मिॅल्लोके मां ते &      \\
        \hline
            195 & TB\_1.3.10.10 & देवानां ॅवा इतरे यज्ञाः &      \\
        \hline
            196 & TB\_1.4.1.1 & उभये वा एते प्रजापतेरद्ध्य{-}सृज्यन्त &      \\
        \hline
            197 & TB\_1.4.1.2 & देवता वा एता यजमानस्य &      \\
        \hline
            198 & TB\_1.4.1.3 & देवा एव तद्{-}देवान्{-}गच्छन्ति यच्चमसाञ्जुहोति &      \\
        \hline
            199 & TB\_1.4.1.4 & तैर्वै ते व्यावृत{-}मगच्छन्न् यद्{-}दारुमयाणि &      \\
        \hline
            200 & TB\_1.4.1.5 & तस्या एते स्तना आसन्न् &      \\
        \hline
            201 & TB\_1.4.1.6 & ऊर्जमेव तया यजमान इमां &      \\
        \hline
            202 & TB\_1.4.2.1 & युवꣳ सुराममश्विना नमुचावाऽसुरे सचा &      \\
        \hline
            203 & TB\_1.4.2.2 & वाजसनिꣳ रयिमस्मे सुवीरम् प्रशस्तं &      \\
        \hline
            204 & TB\_1.4.2.3 & यदत्र शिष्टꣳ रसिनः सुतस्य &      \\
        \hline
            205 & TB\_1.4.2.4 & यस्ते देव वरुण त्रिष्टुप्छन्दाः &      \\
        \hline
            206 & TB\_1.4.3.1 & उदस्थाद्{-}देव्यदितिर्{-}विश्वरूपी आयुर्यज्ञ्पतावधात् इन्द्राय कृण्वती &      \\
        \hline
            207 & TB\_1.4.3.2 & इमामेवास्मा उत्थापयति आयुर्यज्ञ्पतावधादित्याह आयुरेवास्मिन् &      \\
        \hline
            208 & TB\_1.4.3.3 & दुग्ध्वा ददाति न ह्यदृष्टा &      \\
        \hline
            209 & TB\_1.4.3.4 & अद्भिरेवैनदाप्नोति यो वै यज्ञ्स्यार्तेनानार्तं &      \\
        \hline
            210 & TB\_1.4.3.5 & यद्{-}युद्द्रुतस्य स्कन्देत् यत् ततोऽहुत्वा &      \\
        \hline
            211 & TB\_1.4.3.6 & वि वा एतस्य यज्ञ्श्छिद्यते &      \\
        \hline
            212 & TB\_1.4.4.1 & नि वा एतस्याहवनीयो गार्.हपत्यं &      \\
        \hline
            213 & TB\_1.4.4.2 & भागधेयेनैवैनं प्रणयति ब्राह्मण आर्.षेय &      \\
        \hline
            214 & TB\_1.4.4.3 & पुनः समन्य जुहोति अन्तेनैवान्तं &      \\
        \hline
            215 & TB\_1.4.4.4 & अथाग्निम् अथाग्निहोत्रम् यदाज्यं पुरस्ताद्धरति &      \\
        \hline
            216 & TB\_1.4.4.5 & सर्वाभिरेवैनं देवताभिरुद्धरति पराची वा &      \\
        \hline
            217 & TB\_1.4.4.6 & यस्ताम्यति अन्तमेष यज्ञ्स्य गच्छति &      \\
        \hline
            218 & TB\_1.4.4.7 & यदाहवनीय{-}मनुद्वाप्य गार्.हपत्यं मन्थेत् विच्छिन्द्यात् &      \\
        \hline
            219 & TB\_1.4.4.8 & इतः प्रथमं जज्ञे अग्निः &      \\
        \hline
            220 & TB\_1.4.4.9 & सहसे द्युम्नाय ऊर्जे पत्यायेत्याह &      \\
        \hline
            221 & TB\_1.4.4.10 & ताभ्यामेवैनꣳ समिन्धे वज्रो वै &      \\
        \hline
            222 & TB\_1.4.4.11 & पूर्वेणैवास्य यज्ञेन यज्ञ्मनु सं &      \\
        \hline
            223 & TB\_1.4.5.1 & यस्य प्रातस्सवने सोमोऽतिरिच्यते माद्ध्यन्दिनं &      \\
        \hline
            224 & TB\_1.4.5.2 & मरुत्वतीषु कुर्वन्ति तेनैव माद्ध्यन्दिनाथ्{-}सवनान्नयन्ति &      \\
        \hline
            225 & TB\_1.4.5.3 & अतिरिक्तस्य शान्त्यै बण्महाꣳ असि &      \\
        \hline
            226 & TB\_1.4.5.4 & मद्ध्यत एव यज्ञ्ꣳ समादधाति &      \\
        \hline
            227 & TB\_1.4.6.1 & एकैको वै जनताया{-}मिन्द्रः एकं &      \\
        \hline
            228 & TB\_1.4.6.2 & मरुत्वतीः प्रतिपदः मरुतो वै &      \\
        \hline
            229 & TB\_1.4.6.3 & सर्वस्माद्{-}वित्ताद्{-}वेद्यात् अभिवर्तो ब्रह्मसामं भवति &      \\
        \hline
            230 & TB\_1.4.6.4 & उक्थ्यं कुर्वीत यद्युक्थ्यः स्यात् &      \\
        \hline
            231 & TB\_1.4.6.5 & इष्टर्गः खलु वै पूर्वोऽर्ष्टुः &      \\
        \hline
            232 & TB\_1.4.6.6 & तं दक्षिणतो वेद्यै निधाय &      \\
        \hline
            233 & TB\_1.4.6.7 & अथो धुवन्त्येवैनम् अथो न्येवास्मै &  TB\_3.9.6.2       \\
        \hline
            234 & TB\_1.4.7.1 & असुर्यं ॅवा एतस्माद्{-}वर्णं कृत्वा &      \\
        \hline
            235 & TB\_1.4.7.2 & यदाहवनीय उद्वायेत् यत्तं मन्थेत् &      \\
        \hline
            236 & TB\_1.4.7.3 & अत्र वाव स निलयते &      \\
        \hline
            237 & TB\_1.4.7.4 & गार्.हपत्यो वा अग्नेर्योनिः स्वादेवैनं &      \\
        \hline
            238 & TB\_1.4.7.5 & तथ् सुवर्णꣳ हिरण्य{-}मभवत् यथ् &      \\
        \hline
            239 & TB\_1.4.7.6 & त आदारा अभवन्न् इन्द्रो &      \\
        \hline
            240 & TB\_1.4.7.7 & नीतमिश्रेण तृतीयसवने अगिष्टोमः सोमः &      \\
        \hline
            241 & TB\_1.4.8.1 & पवमानः सुवर्जनः पवित्रेण विचर्.षणिः &      \\
        \hline
            242 & TB\_1.4.8.2 & यत्ते पवित्र{-}मर्चिषि अग्ने विततमन्तरा &      \\
        \hline
            243 & TB\_1.4.8.3 & वैश्वानरो रश्मिभिर्मा पुनातु वातः &      \\
        \hline
            244 & TB\_1.4.8.4 & इदं ब्रह्म पुनीमहे यः &      \\
        \hline
            245 & TB\_1.4.8.5 & सुदुघा हि पयस्वतीः ऋषिभिः &      \\
        \hline
            246 & TB\_1.4.8.6 & ब्राह्मणेष्वमृतꣳ हितम् येन देवाः &      \\
        \hline
            247 & TB\_1.4.9.1 & प्रजा वै सत्रमासत तपस्तप्यमाना &      \\
        \hline
            248 & TB\_1.4.9.2 & तमुपोदतिष्ठन्त{-}मजुहवुः तेन{-}द्वयीमूर्जमवारुन्धत तस्माद्{-}द्विरह्नो{-}मनुष्येभ्य उपह्रियते &      \\
        \hline
            249 & TB\_1.4.9.3 & तमुपोदतिष्ठन्त{-}मजुहवुः तेन संॅवथ्सर ऊर्जमवा{-}रुन्धत &      \\
        \hline
            250 & TB\_1.4.9.4 & यद्{-}यजते यामेव देवा ऊर्जमवारुन्धत &      \\
        \hline
            251 & TB\_1.4.9.5 & यामेव पशव ऊर्जमवा{-}रुन्धत तां &      \\
        \hline
            252 & TB\_1.4.10.1 & अग्निर्वाव संॅवथ्सरः आदित्यः परिवथ्सरः &      \\
        \hline
            253 & TB\_1.4.10.2 & तस्माद्{-}वरुण{-}प्रघासैर् यजमानः परिवथ्सरीणाꣳ स्वस्ति{-}माशास्त &      \\
        \hline
            254 & TB\_1.4.10.3 & वायुमेव तदनुवथ्सर{-}माप्नोति तस्मा{-}च्छुनासीरीयेण यजमानः &      \\
        \hline
            255 & TB\_1.4.10.4 & यस्मिन्नग्निः यद्{-}वैश्वदेवेन यजते एतमेव &      \\
        \hline
            256 & TB\_1.4.10.5 & अथ गृहमेधिन{-}माप्नोति यदा गृहमेधिन{-}माप्नोति &      \\
        \hline
            257 & TB\_1.4.10.6 & अथादित्यो वरुणꣳ राजानं ॅवरुण &      \\
        \hline
            258 & TB\_1.4.10.7 & स एतं ॅलोकमजयत् यस्मिं &      \\
        \hline
            259 & TB\_1.4.10.8 & तथ् साकमेधानाꣳ साकमेधत्वम् अथर्तवः &      \\
        \hline
            260 & TB\_1.4.10.9 & अथौषधय इमं देवं त्र्यम्बकै{-}रयजन्त &      \\
        \hline
            261 & TB\_1.4.10.10 & वायोरेव सायुज्य{-}मुपैति ब्रह्मवादिनो वदन्ति &      \\
        \hline
            262 & TB\_1.5.1.1 & अग्नेः कृत्तिकाः शुक्रं परस्ता{-}ज्ज्योति{-}रवस्तात् &      \\
        \hline
            263 & TB\_1.5.1.2 & बृहस्पतेस्तिष्यः जुह्वतः परस्ताद्{-}यजमाना अवस्तात् &      \\
        \hline
            264 & TB\_1.5.1.3 & देवस्य सवितुर्. हस्तः प्रसवः &      \\
        \hline
            265 & TB\_1.5.1.4 & इन्द्रस्य रोहिणी शृणत्{-}परस्तात्{-}प्रतिशृण{-}दवस्तात् निर्.ऋत्यै{-}मूलवर्.हणी &      \\
        \hline
            266 & TB\_1.5.1.5 & वसूनाꣳ श्रविष्ठाः भूतं परस्ताद्{-}भूतिरवस्तात् &      \\
        \hline
            267 & TB\_1.5.2.1 & यत् पुण्यं नक्षत्रम् तद्{-}बट् &      \\
        \hline
            268 & TB\_1.5.2.2 & यो वै नक्षत्रियं प्रजापतिं &      \\
        \hline
            269 & TB\_1.5.2.3 & अस्मिꣳश्चा{-}मुष्मिꣳश्च यां कामयेत दुहितरं &      \\
        \hline
            270 & TB\_1.5.2.4 & यदभ्यजयन्न् तदभिजितोऽभिजित्त्वम् यं कामयेता{-}नपजय्यं &      \\
        \hline
            271 & TB\_1.5.2.5 & ते रेवत्यां प्राभवन्न् तस्माद्{-}रेवत्यां &      \\
        \hline
            272 & TB\_1.5.2.6 & देवगृहा वै नक्षत्राणि य &      \\
        \hline
            273 & TB\_1.5.2.7 & यमनक्षत्राण्यन्यानि कृत्तिकाः प्रथमम् विशाखे &      \\
        \hline
            274 & TB\_1.5.2.8 & तान्युत्तरेण अन्वेषामराथ्स्मेति तदनूराधाः ज्येष्ठमेषा{-}मवधिष्मेति &      \\
        \hline
            275 & TB\_1.5.2.9 & तच्छ्रोणा यदशृणोत् तच्छ्रविष्ठाः यच्छत{-}मभिषज्यन्न् &      \\
        \hline
            276 & TB\_1.5.3.1 & देवस्य सवितुः प्रातः प्रसवः &      \\
        \hline
            277 & TB\_1.5.3.2 & प्राचीनं मद्ध्यन्दिनात् ततो देवा &      \\
        \hline
            278 & TB\_1.5.3.3 & भगस्यापराह्णः तत्{-}पुण्यं तेजस्व्यहः तस्मादपराह्णे &      \\
        \hline
            279 & TB\_1.5.3.4 & ब्राह्मणो वा अष्टाविꣳशो नक्षत्राणाम् &      \\
        \hline
            280 & TB\_1.5.4.1 & ब्रह्मवादिनो वदन्ति कति पात्राणि &      \\
        \hline
            281 & TB\_1.5.4.2 & व्यानादुपाꣳशु सवनम् वाच ऐन्द्रवायवम् &      \\
        \hline
            282 & TB\_1.5.5.1 & ऋतमेव परमेष्ठि ऋतं नात्येति &      \\
        \hline
            283 & TB\_1.5.5.2 & तपसाऽस्यानुवर्तये शिवेनास्योपवर्तये शग्मेनास्या{-}भिवर्तये तदृतं &      \\
        \hline
            284 & TB\_1.5.5.3 & इन्द्रो मरुद्भिः सखिभिः सह &      \\
        \hline
            285 & TB\_1.5.5.4 & तदृतं तथ् सत्यम् तद्{-}व्रतं &      \\
        \hline
            286 & TB\_1.5.5.5 & शिरस्तपस्याहितम् वैश्वानरस्य तेजसा ऋतेनास्य &      \\
        \hline
            287 & TB\_1.5.5.6 & एकं मास{-}मुदसृजत् परमेष्ठी प्रजाभ्यः &      \\
        \hline
            288 & TB\_1.5.5.7 & तेनाहमस्य ब्रह्मणा निवर्तयामि जीवसे &      \\
        \hline
            289 & TB\_1.5.6.1 & देवा वै यद्{-}यज्ञे ऽकुर्वत &      \\
        \hline
            290 & TB\_1.5.6.2 & अथो परैव भवति अथ &      \\
        \hline
            291 & TB\_1.5.6.3 & अथैतन्{-}मनुर्{-}वप्त्रे मिथुन{-}मपश्यत् स श्मश्रूण्यग्रेऽवपत &      \\
        \hline
            292 & TB\_1.5.6.4 & वैश्वदेवेन चतुरो मासोऽवृञ्जतेन्द्रराजानः ताञ्छीर्.षन्नि &      \\
        \hline
            293 & TB\_1.5.6.5 & य एवं ॅविद्वाꣳ{-}श्चातुर्मास्यैर्{-}यजते भ्रातृव्यस्यैव &      \\
        \hline
            294 & TB\_1.5.6.6 & प्रजायते य एवं ॅविद्वाॅल्लोहितायसेन &      \\
        \hline
            295 & TB\_1.5.6.7 & ऋद्ध्यामेव तद्{-}वीर्य एषु लोकेषु &      \\
        \hline
            296 & TB\_1.5.7.1 & आयुषः प्राणं सन्तनु प्राणादपानं &      \\
        \hline
            297 & TB\_1.5.8.1 & इन्द्रो दधीचो अस्थभिः वृत्राण्य &      \\
        \hline
            298 & TB\_1.5.8.2 & इन्द्रमर्केभिरर्किणः इन्द्रं ॅवाणीरनूषत इन्द्र &      \\
        \hline
            299 & TB\_1.5.8.3 & उग्र उग्राभिरूतिभिः तमिन्द्रं ॅवाजयामसि &      \\
        \hline
            300 & TB\_1.5.9.1 & देवासुराः सम्ॅयत्ता आसन्न् स &  TB\_3.3.5.1       \\
        \hline
            301 & TB\_1.5.9.2 & तं ॅयज्ञ्क्रतुभिर्नान्वविन्दन्न् तमिष्टिभि{-}रन्वैच्छन्न् तमिष्टिभि{-}रन्वविन्दन्न् &      \\
        \hline
            302 & TB\_1.5.9.3 & ते तदं तमेव कृत्वोदद्रवन्न् &      \\
        \hline
            303 & TB\_ &  &           \\
        \hline
            304 & TB\_1.5.9.5 & स्रुवेणाघार{-}माघार्य तिस्रः पराची{-}राहुतीर्. हुत्वा &      \\
        \hline
            305 & TB\_1.5.9.6 & तिस्रः पराचीराहुतीर्. हुत्वा स्रुवेणोपसदं &      \\
        \hline
            306 & TB\_1.5.9.7 & तस्मादभिनीयैवाहः पशुमालभेत अह्न एव &      \\
        \hline
            307 & TB\_1.5.9.8 & पञ्चपञ्ची वै यजमानः त्वङ्मां &      \\
        \hline
            308 & TB\_1.5.10.1 & स समुद्र उत्तरतः प्राज्वलद्{-}भूम्यन्तेन &      \\
        \hline
            309 & TB\_1.5.10.2 & अथ यथ् सुवर्णरजताभ्यां कुशीभ्यां &      \\
        \hline
            310 & TB\_1.5.10.3 & तं सप्तदशेनाभि प्रास्तुवत तं &      \\
        \hline
            311 & TB\_1.5.10.4 & त्रिवृतैव तद्{-}यजमानमाददते तं त्रिवृतैव &      \\
        \hline
            312 & TB\_1.5.10.5 & तं पञ्चदशेनैव हरन्ति यावती &      \\
        \hline
            313 & TB\_1.5.10.6 & यावती सप्तदशस्य मात्रा प्रजापतिर्वै &      \\
        \hline
            314 & TB\_1.5.10.7 & मात्राꣳ सायुज्यꣳ सलोकतां गमयन्ति &      \\
        \hline
            315 & TB\_1.5.11.1 & ये वै चत्वारः स्तोमाः &      \\
        \hline
            316 & TB\_1.5.11.2 & एतानि वाव तानि ज्योतीꣳषि &      \\
        \hline
            317 & TB\_1.5.11.3 & यदश्विभ्यां धानाः पूष्णः करम्भः &      \\
        \hline
            318 & TB\_1.5.11.4 & स यदष्टाकपालान् प्रातस्सवने कुर्यात् &      \\
        \hline
            319 & TB\_1.5.12.1 & तस्यावाचोऽवपादादबिभयुः तमेतेषु सप्तसु छन्दः &      \\
        \hline
            320 & TB\_1.5.12.2 & स बृहतीमेवा{-}स्पृशत् द्वाभ्यामक्षराभ्याम् अहोरात्राभ्यामेव &      \\
        \hline
            321 & TB\_1.5.12.3 & यस्यां तत् प्रत्यतिष्ठदिति यानि &      \\
        \hline
            322 & TB\_1.5.12.4 & ते बृहती एव भूत्वा &  TB\_1.5.12.5       \\
        \hline
            323 & TB\_1.5.12.5 & ते बृहती एव भूत्वा & TB\_1.5.12.4        \\
        \hline
            324 & TB\_1.6.1.1 & अनुमत्यै पुरोडाश{-}मष्टाकपालं निर्वपति ये &      \\
        \hline
            325 & TB\_1.6.1.2 & रुद्रो भूत्वाऽग्निरनूत्थाय अद्ध्वर्युं च &      \\
        \hline
            326 & TB\_1.6.1.3 & स्वकृत इरिणे जुहोति प्रदरे &  TB\_1.7.1.9       \\
        \hline
            327 & TB\_1.6.1.4 & अन्तत एव निर्.ऋतिं निरवदयते &      \\
        \hline
            328 & TB\_1.6.1.5 & इयमेवास्मै राज्यमनुमन्यते धेनुर्{-}दक्षिणा इमामेव &      \\
        \hline
            329 & TB\_1.6.1.6 & विष्णुर्यज्ञ्ः देवताश्चैव यज्ञ्ं चाव &      \\
        \hline
            330 & TB\_1.6.1.7 & वार्त्रघ्नमेव विजित्यै हिरण्यं दक्षिणा &      \\
        \hline
            331 & TB\_1.6.1.8 & यद्{-}वही तेनाग्नेयः यदृषभः तेनैन्द्रः &      \\
        \hline
            332 & TB\_1.6.1.9 & इन्द्रियमेवाव रुन्धे ऋषभो वही &      \\
        \hline
            333 & TB\_1.6.1.10 & देवा वा ओषधीष्वाजिमयुः ता &      \\
        \hline
            334 & TB\_1.6.1.11 & प्रथमजो वथ्सो दक्षिणा समृद्ध्यै &      \\
        \hline
            335 & TB\_1.6.2.1 & वैश्वदेवेन वै प्रजापतिः प्रजा &  TB\_1.6.8.1       \\
        \hline
            336 & TB\_1.6.2.2 & सोमो रेतोऽदधात् सविता प्राजनयत् &      \\
        \hline
            337 & TB\_1.6.2.3 & स एतं प्रजापतिर्मारुतꣳ सप्तकपाल{-}मपश्यत् &      \\
        \hline
            338 & TB\_1.6.2.4 & याः पूर्वाः प्रजा असृक्षि &      \\
        \hline
            339 & TB\_1.6.2.5 & प्रजा एव तद्{-}यजमानः पोषयति &      \\
        \hline
            340 & TB\_1.6.2.6 & मामग्रे यजत मया मुखेनासुरां &      \\
        \hline
            341 & TB\_1.6.2.7 & तेऽग्निना मुखेनासुरानजयन्न् सोमेन राज्ञा &      \\
        \hline
            342 & TB\_1.6.3.1 & त्रिवृद्{-}बर्.हिर्{-}भवति माता पिता पुत्रः &      \\
        \hline
            343 & TB\_1.6.3.2 & अस्मिन्नेव तेन लोके प्रति &      \\
        \hline
            344 & TB\_1.6.3.3 & बहुरूपा हि पशवः समृद्ध्यै &      \\
        \hline
            345 & TB\_1.6.3.4 & त्रिꣳशथ् संपद्यन्ते त्रिꣳशदक्षरा विराट् &      \\
        \hline
            346 & TB\_1.6.3.5 & यदेककपाल आज्यमानयति यजमानमेव पशुभिः &      \\
        \hline
            347 & TB\_1.6.3.6 & यजमानस्यावद्येत् उद्वा माद्येद्{-}यजमानः प्र &      \\
        \hline
            348 & TB\_1.6.3.7 & यदेककपाल{-}माहवनीये जुहोति यजमानमेव सुवर्गं &      \\
        \hline
            349 & TB\_1.6.3.8 & रक्षाꣳसि यज्ञ्ꣳ हन्युः यदुदङ्ङ् &      \\
        \hline
            350 & TB\_1.6.3.9 & वाजिनो यजति अग्निर्वायुः सूर्यः &      \\
        \hline
            351 & TB\_1.6.3.10 & यदप्रहृत्य परिधीञ्जुहुयात् अन्तराधानाभ्यां घासं &      \\
        \hline
            352 & TB\_1.6.4.1 & प्रजापतिः सविता भूत्वा प्रजा &      \\
        \hline
            353 & TB\_1.6.4.2 & प्रजानामवरुणग्राहाय तासां दक्षिणो बाहुः &      \\
        \hline
            354 & TB\_1.6.4.3 & तस्मात्{-}पृथमात्रं ॅव्यꣳसौ उत्तरस्यां ॅवेद्यामुत्तरवेदिमुप &      \\
        \hline
            355 & TB\_1.6.4.4 & प्राणापानावेवाव रुन्धे ओजो बलं &      \\
        \hline
            356 & TB\_1.6.4.5 & शमीपर्णान्युप वपति घासमेवाभ्यामपि यच्छति &      \\
        \hline
            357 & TB\_1.6.5.1 & उत्तरस्यां ॅवेद्यामन्यानि हवीꣳषि सादयति &      \\
        \hline
            358 & TB\_1.6.5.2 & तत् प्रतिप्रस्थाता करोति तस्माद्{-}यच्छ्रेयान्{-}करोति &      \\
        \hline
            359 & TB\_1.6.5.3 & प्रघास्यान्. हवामह इति पत्नी{-}मुदानयति &      \\
        \hline
            360 & TB\_1.6.5.4 & भ्रातृव्य देवत्यो दक्षिणः यदाहवनीये &      \\
        \hline
            361 & TB\_1.6.5.5 & प्रत्यङ्ङेव वरुणपाशान्{-}निर्मुच्यते अक्रन्कर्म कर्मकृत &      \\
        \hline
            362 & TB\_1.6.5.6 & अफ्सु वै वरुणः साक्षादेव &      \\
        \hline
            363 & TB\_1.6.6.1 & देवासुराः सम्ॅयत्ता आसन्न् सोऽग्निरब्रवीत् &      \\
        \hline
            364 & TB\_1.6.6.2 & यदग्नयेऽनीकवते पुरोडाशमष्टाकपालं निर्वपति अग्निमेवानीकवन्तं &      \\
        \hline
            365 & TB\_1.6.6.3 & ते देवा मरुद्भ्यः सान्तपनेभ्यश्चरुं &      \\
        \hline
            366 & TB\_1.6.6.4 & आशिता एवाद्योपवसाम कस्य वाऽहेदम् &      \\
        \hline
            367 & TB\_1.6.6.5 & ततो देवा अभवन्न् पराऽसुराः &      \\
        \hline
            368 & TB\_1.6.6.6 & न प्रयाजा इज्यन्ते नानूयाजाः &      \\
        \hline
            369 & TB\_1.6.7.1 & यत्पत्नी गृहमेधीय{-}स्याश्ञीयात् गृहमेद्ध्येव स्यात् &      \\
        \hline
            370 & TB\_1.6.7.2 & ते देवा गृहमेधीयेनेष्ट्वा आशिता &      \\
        \hline
            371 & TB\_1.6.7.3 & अनु वथ्सान्. वासयन्ति भ्रातृव्यायैव &      \\
        \hline
            372 & TB\_1.6.7.4 & भागधेयेनेवैनꣳ समर्द्धयति ऋषभमाह्वयति वषट्कार &      \\
        \hline
            373 & TB\_1.6.7.5 & अथ वयं ॅवेदाम अस्मभ्यमेव &      \\
        \hline
            374 & TB\_1.6.7.6 & उद्धारं ॅवा एतमिन्द्र उदहरत &      \\
        \hline
            375 & TB\_1.6.8.1 & वैश्वदेवेन वै प्रजापतिः प्रजा & TB\_1.6.2.1        \\
        \hline
            376 & TB\_1.6.8.2 & पितृयज्ञेन सुवर्गं ॅलोकं गमयति &      \\
        \hline
            377 & TB\_1.6.8.3 & सम्ॅवथ्सरमेव प्रीणाति पितृभ्यो बर्.हिषद्भ्यो &      \\
        \hline
            378 & TB\_1.6.8.4 & अर्द्धमासानेव प्रीणाति अभिवान्यायै दुग्धे &      \\
        \hline
            379 & TB\_1.6.8.5 & एका हि पितृणाम् दक्षिणोपमन्थति &      \\
        \hline
            380 & TB\_1.6.8.6 & खाता हि देवानाम् मद्ध्यतो &      \\
        \hline
            381 & TB\_1.6.8.7 & यथ् समूलम् तत् पितृणाम् &      \\
        \hline
            382 & TB\_1.6.8.8 & षड्वा ऋतवः ऋतूनेव प्रीणाति & TB\_1.3.10.4        \\
        \hline
            383 & TB\_1.6.8.9 & मृत्युना यजमानं परि गृह्णीयात् &      \\
        \hline
            384 & TB\_1.6.9.1 & अग्नये देवेभ्यः पितृभ्यः समिद्ध्यमानायानुब्रूहीत्याह &      \\
        \hline
            385 & TB\_1.6.9.2 & यदार्.षेयं ॅवृणीत यद्धोतारम् प्रमायुको &      \\
        \hline
            386 & TB\_1.6.9.3 & यज्ञ्स्यैव चक्षुषी नान्तरेति प्राचीनावीती &      \\
        \hline
            387 & TB\_1.6.9.4 & ऋतूनाꣳ संतत्यै प्रैवैभ्यः पूर्वया &      \\
        \hline
            388 & TB\_1.6.9.5 & आ स्वधेत्या{-}श्रावयति अस्तु स्वधेति &      \\
        \hline
            389 & TB\_1.6.9.6 & ये वै यज्वानः ते &      \\
        \hline
            390 & TB\_1.6.9.7 & अथो यथाऽग्निꣳ स्विष्टकृतं ॅयजति &      \\
        \hline
            391 & TB\_1.6.9.8 & आहवनीय{-}मुपतिष्ठन्ते न्येवास्मै तद्ध्नुवते यथ्सत्याहवनीये &      \\
        \hline
            392 & TB\_1.6.9.9 & प्राणो वै सुसन्दृक् प्राणमेवात्मन्{-}दधते &      \\
        \hline
            393 & TB\_1.6.9.10 & अथो तर्पयत्येव तृप्यति प्रजया &      \\
        \hline
            394 & TB\_1.6.9.11 & ऋतूनेव प्रीणाति न पत्न्यन्वास्ते &      \\
        \hline
            395 & TB\_1.6.10.1 & प्रतिपूरुषमेककपालान् निर्वपति जाता एव &      \\
        \hline
            396 & TB\_1.6.10.2 & तद्धि रुद्रस्य भागधेयम् इमां &      \\
        \hline
            397 & TB\_1.6.10.3 & न ग्राम्यान् पशून्. हिनस्ति &      \\
        \hline
            398 & TB\_1.6.10.4 & एष ते रुद्र भागः &      \\
        \hline
            399 & TB\_1.6.10.5 & त्ःयम्बकं ॅयजामह इत्याह मृत्योः &      \\
        \hline
            400 & TB\_1.7.1.1 & एतद्{-}ब्राह्मणान्येव पञ्च हवीꣳषि अथेन्द्राय &      \\
        \hline
            401 & TB\_1.7.1.2 & द्वादशगवꣳ सीरं दक्षिणा समृद्ध्यै &      \\
        \hline
            402 & TB\_1.7.1.3 & सोऽब्रवीत् क इदं तुरीयमिति &      \\
        \hline
            403 & TB\_1.7.1.4 & वहिनी धेनुर्{-}दक्षिणा यद्{-}वहिनी तेनाग्नेयी &      \\
        \hline
            404 & TB\_1.7.1.5 & तꣳ सृष्टꣳ रक्षाꣳस्यजिघाꣳसन्न् स &      \\
        \hline
            405 & TB\_1.7.1.6 & इन्द्रो वृत्रꣳ हत्वा असुरान् & TB\_1.3.10.1        \\
        \hline
            406 & TB\_1.7.1.7 & न दिवा न नक्तमिति &      \\
        \hline
            407 & TB\_1.7.1.8 & स एतानपामार्गानजनयत् तानजुहोत् तैर्वै &      \\
        \hline
            408 & TB\_1.7.1.9 & स्वकृत इरिणे जुहोति प्रदरे & TB\_1.6.1.3        \\
        \hline
            409 & TB\_1.7.2.1 & धात्रे पुरोडाशं द्वादशकपालं निर्वपति &      \\
        \hline
            410 & TB\_1.7.2.2 & वैष्णवं त्रिकपालम् वीर्यं ॅवा &      \\
        \hline
            411 & TB\_1.7.2.3 & तेनैन्द्रः यद्वामनः तेन वैष्णवः &      \\
        \hline
            412 & TB\_1.7.2.4 & अग्निः प्रजां प्रजनयति वृद्धामिन्द्रः &      \\
        \hline
            413 & TB\_1.7.2.5 & वृद्धानिन्द्रः प्रयच्छति पौष्णश्चरुर्{-}भवति इयं &      \\
        \hline
            414 & TB\_1.7.2.6 & पवित्रं ॅवै हिरण्यम् पुनात्येवैनम् &      \\
        \hline
            415 & TB\_1.7.3.1 & रत्निनामेतानि हवीꣳषि भवन्ति एते &      \\
        \hline
            416 & TB\_1.7.3.2 & यथ् सद्यो निर्वपेत् यावतीमेकेन &      \\
        \hline
            417 & TB\_1.7.3.3 & ऐन्द्र{-}मेकादशकपालꣳ राजन्यस्य गृहे इन्द्रियमेवाव &      \\
        \hline
            418 & TB\_1.7.3.4 & नैर्.ऋतं चरुं परिवृक्त्यै गृहे &      \\
        \hline
            419 & TB\_1.7.3.5 & अन्नं ॅवै मरुतः अन्नमेवाव{-}रुन्धे &      \\
        \hline
            420 & TB\_1.7.3.6 & अन्नं ॅवै पूषा अन्नमेवाव &      \\
        \hline
            421 & TB\_1.7.3.7 & यन्न प्रति निर्वपेत् रत्निन &      \\
        \hline
            422 & TB\_1.7.3.8 & बार्.हस्पत्ये मैत्रमपि दधाति ब्रह्म &      \\
        \hline
            423 & TB\_1.7.4.1 & देवसुवामेतानि हवीꣳषि भवन्ति एतावन्तो &      \\
        \hline
            424 & TB\_1.7.4.2 & वरुणो धर्मपतीनाम् एतदेव सर्वं &      \\
        \hline
            425 & TB\_1.7.4.3 & राज्यमेवास्मिन् प्रति दधाति स्वां &      \\
        \hline
            426 & TB\_1.7.4.4 & अरातिमेवैनं तारयति असूषुदन्त यज्ञिया &      \\
        \hline
            427 & TB\_1.7.5.1 & अर्थेतः स्थेति जुहोति आहुत्यैवैना &      \\
        \hline
            428 & TB\_1.7.5.2 & ऊर्मिमन्तमेवैनं करोति वृषसेनोऽसीत्याह सेनामेवास्य &      \\
        \hline
            429 & TB\_1.7.5.3 & राष्ट्रमेव वर्चस्व्यकः सूर्यत्वचसः स्थेत्याह &      \\
        \hline
            430 & TB\_1.7.5.4 & वाशाः स्थेत्याह राष्ट्रमेव वश्यकः &      \\
        \hline
            431 & TB\_1.7.5.5 & राष्ट्रमेव तेजस्व्यकः अपामोषधीनाꣳ रसः &      \\
        \hline
            432 & TB\_1.7.6.1 & देवीरापः सं मधुमतीर्{-}मधुमतीभिः सृज्यद्ध्वमित्याह &      \\
        \hline
            433 & TB\_1.7.6.2 & शतमानं भवति शतायुः पुरुषः &      \\
        \hline
            434 & TB\_1.7.6.3 & सोमस्य ह्येतद्दात्रम् शुक्रा वः &      \\
        \hline
            435 & TB\_1.7.6.4 & स्वाहा राजसूयायेत्याह राजसूयाय ह्येना &      \\
        \hline
            436 & TB\_1.7.6.5 & यावानेव पुरुषः तस्मिन् वीर्यं &      \\
        \hline
            437 & TB\_1.7.6.6 & अग्निरेवैनं गार्.हपत्येनावति इन्द्र इन्द्रियेण &      \\
        \hline
            438 & TB\_1.7.6.7 & आभ्यामेव प्रसूतो यजमानो वज्रं &      \\
        \hline
            439 & TB\_1.7.6.8 & इन्द्रस्य वज्रोऽसि वार्त्रघ्न इति & TB\_1.3.5.2  TB\_1.7.9.1       \\
        \hline
            440 & TB\_1.7.7.1 & दिशो व्यास्थापयति दिशामभिजित्यै यदनु &      \\
        \hline
            441 & TB\_1.7.7.2 & उग्रामा तिष्ठेत्याह इन्द्रियमेवाव रुन्धे &      \\
        \hline
            442 & TB\_1.7.7.3 & मारुत एष भवति अन्नं &      \\
        \hline
            443 & TB\_1.7.7.4 & स राष्ट्रं नाभवत् स &      \\
        \hline
            444 & TB\_1.7.7.5 & षट्पुरस्ता{-}दभिषेकस्य जुहोति षडुपरिष्टात् द्वादश &      \\
        \hline
            445 & TB\_1.7.8.1 & सोमस्य त्विषिरसि तवेव मे &      \\
        \hline
            446 & TB\_1.7.8.2 & शतायुः पुरुषः शतेन्द्रियः आयुष्येवेन्द्रिये &  TB\_1.8.6.5 TB\_3.8.20.3       \\
        \hline
            447 & TB\_1.7.8.3 & सोमो राजा वरुणः देवा &      \\
        \hline
            448 & TB\_1.7.8.4 & स्वयैवैनं देवतयाऽभिषिञ्चति अग्ने{-}स्तेजसेत्याह तेज &      \\
        \hline
            449 & TB\_1.7.8.5 & ओज एवास्मिन् दधाति क्षत्राणां &      \\
        \hline
            450 & TB\_1.7.8.6 & उदङ्परेत्याग्नीद्ध्रे जुहोति एषा वै &      \\
        \hline
            451 & TB\_1.7.8.7 & प्रजापते न त्वदेतान्यन्य इति &      \\
        \hline
            452 & TB\_1.7.9.1 & इन्द्रस्य वज्रोऽसि वार्त्रघ्न इति & TB\_1.3.5.2 TB\_1.7.6.8        \\
        \hline
            453 & TB\_1.7.9.2 & षड्वा ऋतवः ऋतुभिरेवैनं ॅयुनक्ति &      \\
        \hline
            454 & TB\_1.7.9.3 & मरुतां प्रसवे जेषमित्याह मरुद्भिरेव &      \\
        \hline
            455 & TB\_1.7.9.4 & इन्द्रियमेव वीर्यमात्मन्{-}धत्ते पशूनां मन्युरसि &      \\
        \hline
            456 & TB\_1.7.9.5 & नमो मात्रे पृथिव्या इत्याहाहिं &      \\
        \hline
            457 & TB\_1.7.9.6 & त्रयोऽश्वा भवन्ति रथश्चतुर्थः तस्मा{-}च्चतुर्जुहोति &      \\
        \hline
            458 & TB\_1.7.10.1 & मित्रोऽसि वरुणोऽसीत्याह मैत्रं ॅवा &      \\
        \hline
            459 & TB\_1.7.10.2 & वैश्वदेव्यो वै प्रजाः ता &      \\
        \hline
            460 & TB\_1.7.10.3 & ब्रह्मा3न्त्वꣳ राजन्{-}ब्रह्माऽसीन्द्रोऽसि सत्यौजा इत्याह &      \\
        \hline
            461 & TB\_1.7.10.4 & मित्रोऽसि सुशेव इत्याह जगती{-}मेवैतेनाभिव्याहरति &      \\
        \hline
            462 & TB\_1.7.10.5 & नैनꣳ सत्यानृते उदिते हिꣳस्तः &      \\
        \hline
            463 & TB\_1.7.10.6 & ओदनमुद्ब्रुवते परमेष्ठी वा एषः &      \\
        \hline
            464 & TB\_1.8.1.1 & वरुणस्य सुषुवाणस्य दशधेन्द्रियं ॅवीर्यं &      \\
        \hline
            465 & TB\_1.8.1.2 & सोमेन राज्ञाऽष्टमे त्वष्ट्रा रूपेण &      \\
        \hline
            466 & TB\_1.8.2.1 & जामि वा एतत् कुर्वन्ति &      \\
        \hline
            467 & TB\_1.8.2.2 & अन्नं ॅविराट् विराजैवान्नाद्यमव रुन्धे &      \\
        \hline
            468 & TB\_1.8.2.3 & प्रजापतेराप्त्यै प्राकाशावद्ध्वर्यवे ददाति प्रकाश{-}मेवैनं &      \\
        \hline
            469 & TB\_1.8.2.4 & द्वादश{-}पष्ठौहीर्{-}ब्रह्मणे आयुरेवाव{-}रुन्धे वशां मैत्रावरुणाय &      \\
        \hline
            470 & TB\_1.8.2.5 & अनड्वाह{-}मग्नीधे वह्निर्वा अनड्वान् वह्निरग्नीत् &      \\
        \hline
            471 & TB\_1.8.3.1 & ईश्वरो वा एष दिशोऽनून्{-}मदितोः &      \\
        \hline
            472 & TB\_1.8.3.2 & आदित्यां मल्.हां गर्भिणीमा लभते &      \\
        \hline
            473 & TB\_1.8.3.3 & इन्द्रियं ॅवै गर्भः राष्ट्र{-}मेवेन्द्रियाव्यकः &      \\
        \hline
            474 & TB\_1.8.3.4 & यदश्विभ्यां पूष्णे पुरोडाशं द्वादशकपालं &      \\
        \hline
            475 & TB\_1.8.4.1 & आग्नेय{-}मष्टाकपालं निर्वपति तस्मा{-}च्छिशिरे कुरुपञ्चालाः &      \\
        \hline
            476 & TB\_1.8.4.2 & रूपाण्येव तेन कुर्वते वैश्वानरं &      \\
        \hline
            477 & TB\_1.8.4.3 & मासिमास्येतानि हवीꣳषि निरुप्याणीत्याहुः तेनैवर्तून्{-}प्रयुङ्क्त &      \\
        \hline
            478 & TB\_1.8.5.1 & इन्द्रस्य सुषुवाणस्य दशधेन्द्रियं ॅवीर्यं &      \\
        \hline
            479 & TB\_1.8.5.2 & स शार्दूलः यत्{-}कर्णयोः स &      \\
        \hline
            480 & TB\_1.8.5.3 & त्विषिमेवाव रुन्धे त्रयो ग्रहाः &      \\
        \hline
            481 & TB\_1.8.5.4 & यथ्{-}सीसम् न स्त्री न &      \\
        \hline
            482 & TB\_1.8.5.5 & तिस्रो हि रात्रीः क्रीतः &      \\
        \hline
            483 & TB\_1.8.5.6 & अनिरुक्तः प्रजापतिः प्रजापतेराप्त्यै एकयर्चा &      \\
        \hline
            484 & TB\_1.8.6.1 & यत्{-}त्रिषु यूपेष्वा{-}लभेत बहिर्द्धा ऽस्मादिन्द्रियं &      \\
        \hline
            485 & TB\_1.8.6.2 & ब्राह्मणं परिक्रीणीया{-}दुच्छेषणस्य पातारम् ब्राह्मणो &      \\
        \hline
            486 & TB\_1.8.6.3 & इन्द्रिये एवास्मै समीची दधाति &      \\
        \hline
            487 & TB\_1.8.6.4 & उत वा एषाऽश्वꣳ सूते &      \\
        \hline
            488 & TB\_1.8.6.5 & शतायुः पुरुषः शतेन्द्रियः आयुष्येवेन्द्रिये & TB\_1.7.8.2  TB\_3.8.20.3       \\
        \hline
            489 & TB\_1.8.6.6 & तन्निदधाति प्रतिष्ठित्यै पितॄन्. वा &      \\
        \hline
            490 & TB\_1.8.7.1 & अग्निष्टोममग्र आहरति यज्ञ्मुखं ॅवा &      \\
        \hline
            491 & TB\_1.8.7.2 & एतावान्. वै यज्ञ्ः यावान्{-}पवमानाः &      \\
        \hline
            492 & TB\_1.8.8.1 & उप त्वा जामयो गिर &      \\
        \hline
            493 & TB\_1.8.8.2 & यदाह पवस्व वाचो अग्रिय &      \\
        \hline
            494 & TB\_1.8.8.3 & तस्मादुद्वतीर्{-}भवन्ति सौर्यनुष्टु{-}गुत्तमा भवति सुवर्गस्य &      \\
        \hline
            495 & TB\_1.8.8.4 & तैरेव सवान्नैति यानि देवराजानां &      \\
        \hline
            496 & TB\_1.8.8.5 & विड्वा एकविꣳशः राष्ट्रꣳ सप्तदशः &      \\
        \hline
            497 & TB\_1.8.9.1 & इयं ॅवै रजता असौ &      \\
        \hline
            498 & TB\_1.8.10.1 & अप्रतिष्ठितो वा एष इत्याहुः &      \\
        \hline
            499 & TB\_1.8.10.2 & नानैवाहोरात्रयोः प्रति तिष्ठति पौर्णमास्यां &      \\
        \hline
            500 & TB\_1.8.10.3 & अपशव्यो द्विरात्र इत्याहुः द्वे &      \\
        \hline
            501 & TB\_2.1.1.1 & अङ्गिरसो वै सत्रमासत तेषां &      \\
        \hline
            502 & TB\_2.1.1.2 & तासां जग्ध्वा रुप्यन्त्यैत् तेऽब्रुवन्न् &      \\
        \hline
            503 & TB\_2.1.1.3 & स्वदन्तेऽस्मा ओषधयः ते वथ्समु{-}पावासृजन्न् &      \\
        \hline
            504 & TB\_2.1.2.1 & प्रजापति{-}रग्नि{-}मसृजत तं प्रजा अन्वसृज्यन्त &      \\
        \hline
            505 & TB\_2.1.2.2 & तद्{-}घृतमभवत् तस्माद्यस्य दक्षिणतः केशा &      \\
        \hline
            506 & TB\_2.1.2.3 & वसीय एव चेतयते तं &      \\
        \hline
            507 & TB\_2.1.2.4 & भोगायैवास्य हुतं भवति तस्या &      \\
        \hline
            508 & TB\_2.1.2.5 & सोऽग्नि{-}रबिभेत् आहुतीभिर्वै माऽऽप्नोतीति स &      \\
        \hline
            509 & TB\_2.1.2.6 & तस्मादग्निहोत्र{-}मुच्यते तद्धूयमान{-}मादित्योऽब्रवीत् मा हौषीः &      \\
        \hline
            510 & TB\_2.1.2.7 & आग्नेयी वै रात्रिः ऐन्द्रमहः &      \\
        \hline
            511 & TB\_2.1.2.8 & प्रैव तेन जायते उदिते &      \\
        \hline
            512 & TB\_2.1.2.9 & प्रैव जायते अथो यथा &      \\
        \hline
            513 & TB\_2.1.2.10 & उद्यन्तं ॅवावादित्य{-}मग्निरनु समारोहति तस्माद्धूम &      \\
        \hline
            514 & TB\_2.1.2.11 & उभाभ्यां प्रातः न देवताभ्यः &      \\
        \hline
            515 & TB\_2.1.2.12 & तूष्णीमुत्तरा{-}माहुतिं जुहोति मिथुनत्वाय प्रजात्यै &      \\
        \hline
            516 & TB\_2.1.3.1 & रुद्रो वा एषः यदग्निः &      \\
        \hline
            517 & TB\_2.1.3.2 & घर्मो वा एषोऽशान्तः अहरहः &      \\
        \hline
            518 & TB\_2.1.3.3 & तत्{-}पशव्यम् यज्जुहोति तद्ब्रह्मवर्चसि उभयमेवाकः &      \\
        \hline
            519 & TB\_2.1.3.4 & पर्यग्नि करोति रक्षसा{-}मपहत्यै त्रिः &      \\
        \hline
            520 & TB\_2.1.3.5 & पत्नीꣳ शुचा ऽर्पयेत् उदीचीन{-}मुद्वासयति &      \\
        \hline
            521 & TB\_2.1.3.6 & पशूनेवा{-}वरुन्धे सर्वान्{-}पूर्णानुन्नयति सर्वे हि &      \\
        \hline
            522 & TB\_2.1.3.7 & अन्यस्मै प्रयच्छति तादृगेव तत् &      \\
        \hline
            523 & TB\_2.1.3.8 & यदेनꣳ समयच्छत् तथ्समिधः समित्त्वम् &      \\
        \hline
            524 & TB\_2.1.3.9 & यद्{-}द्वे समिधावादद्ध्यात् भ्रातृव्यमस्मै जनयेत् &      \\
        \hline
            525 & TB\_2.1.4.1 & उत्तरावतीं ॅवै देवा आहुतिमजुहवुः &      \\
        \hline
            526 & TB\_2.1.4.2 & यस्यैवं जुह्वति भवत्येव यं &      \\
        \hline
            527 & TB\_2.1.4.3 & हुत्वोप सादयत्यजामित्वाय अथो व्यावृत्त्यै &      \\
        \hline
            528 & TB\_2.1.4.4 & यज्ञ्स्थाणु{-}मृच्छेत् अतिहाय पूर्वामाहुतिं जुहोति &      \\
        \hline
            529 & TB\_2.1.4.5 & द्विर्जुहोति अथ क्व द्वे &      \\
        \hline
            530 & TB\_2.1.4.6 & षट्थ् संपद्यन्ते षड्वा ऋतवः &      \\
        \hline
            531 & TB\_2.1.4.7 & यन्निमार्ष्टि तदोषधीनाम् यद्द्वितीयम् तत्{-}पितृणाम् &      \\
        \hline
            532 & TB\_2.1.4.8 & आत्मनो गोपीथाय निर्णेनेक्ति शुद्ध्यै &      \\
        \hline
            533 & TB\_2.1.4.9 & ब्रह्मवर्चसस्य समिद्ध्यै न बर.हिरनु &      \\
        \hline
            534 & TB\_2.1.5.1 & ब्रह्मवादिनो वदन्ति अग्निहोत्रप्रायणा यज्ञाः &      \\
        \hline
            535 & TB\_2.1.5.2 & वनस्पतय इद्ध्मः दिशः परिधयः &      \\
        \hline
            536 & TB\_2.1.5.3 & अहरेव तेन दक्षिण्यं कुरुते &      \\
        \hline
            537 & TB\_2.1.5.4 & यजमानस्या{-}पराभावाय यत्{-}प्रातः अह्न एव &      \\
        \hline
            538 & TB\_2.1.5.5 & इमामेव पूर्वया दुहे अमूमुत्तरया &      \\
        \hline
            539 & TB\_2.1.5.6 & पशुमानेव भवति दद्ध्नेन्द्रिय कामस्य &      \\
        \hline
            540 & TB\_2.1.5.7 & चतुरुन्नयति चतुरक्षरꣳ रथन्तरम् रथन्तरस्यैष &      \\
        \hline
            541 & TB\_2.1.5.8 & यो वा अग्निहोत्रस्योपसदो वेद &      \\
        \hline
            542 & TB\_2.1.5.9 & य एवं ॅवेद उपैन{-}मुपसदो &      \\
        \hline
            543 & TB\_2.1.5.10 & य एवं ॅवेद तस्य &      \\
        \hline
            544 & TB\_2.1.5.11 & तानेवोभयाꣳ स्तर्पयति त एनं &      \\
        \hline
            545 & TB\_2.1.6.1 & प्रजापति{-}रकामय{-}तात्मन् वन्मे जायेतेति सोऽजुहोत् &      \\
        \hline
            546 & TB\_2.1.6.2 & चक्षुष आदित्यः तेषाꣳ हुतादजायत &      \\
        \hline
            547 & TB\_2.1.6.3 & तनुवा अहमिति वायुः चक्षुषोऽहमित्यादित्यः &      \\
        \hline
            548 & TB\_2.1.6.4 & य एवं ॅवेद तौ &      \\
        \hline
            549 & TB\_2.1.6.5 & प्रजापतिमभि पर्यावर्तत स मृत्यो{-}रबिभेत् &      \\
        \hline
            550 & TB\_2.1.7.1 & रौद्रं गवि वायव्य{-}मुपसृष्टम् आश्विनं &      \\
        \hline
            551 & TB\_2.1.8.1 & दक्षिणत उप सृजति पितृलोकमेव &      \\
        \hline
            552 & TB\_2.1.8.2 & न सं मृशति पापवस्यसस्य &      \\
        \hline
            553 & TB\_2.1.8.3 & ऐन्द्राग्नमुप सादितम् सर्वाभ्यो वा &      \\
        \hline
            554 & TB\_2.1.9.1 & त्रयो वै प्रैयमेधा आसन्न् &      \\
        \hline
            555 & TB\_2.1.9.2 & यश्च यजुषा ऽजुहोद्यश्च तूष्णीम् &      \\
        \hline
            556 & TB\_2.1.9.3 & यस्याग्निहोत्रमहुतꣳ सूर्योऽभ्युदेति यद्{-}यन्ते स्यात् &      \\
        \hline
            557 & TB\_2.1.10.1 & यदग्नि{-}मुद्धरति वसवस्तर्ह्यग्निः तस्मिन्. यस्य &      \\
        \hline
            558 & TB\_2.1.10.2 & तस्मिन्. यस्य तथाविधे जुह्वति &      \\
        \hline
            559 & TB\_2.1.10.3 & अङ्गारा भवन्ति तेभ्योऽङ्गारेभ्योऽर्चिरुदेति प्रजापतिस्तर्ह्यग्निः &      \\
        \hline
            560 & TB\_2.1.11.1 & ऋतं त्वा सत्येन परिषिञ्चामीति &      \\
        \hline
            561 & TB\_2.2.1.1 & प्रजापतिरकामयत प्रजाः सृजेयेति स &      \\
        \hline
            562 & TB\_2.2.1.2 & प्रजापतिरेव भूत्वा प्र जायते &      \\
        \hline
            563 & TB\_2.2.1.3 & दर्भस्तम्बे जुहोति एतस्माद्वै योनेः &      \\
        \hline
            564 & TB\_2.2.1.4 & सोऽरण्यं परेत्य दर्भस्तम्ब{-}मुद्ग्रत्थ्य ब्राह्मणं &      \\
        \hline
            565 & TB\_2.2.1.5 & अग्निवान्. वै दर्भस्तम्बः अग्निवत्येव &      \\
        \hline
            566 & TB\_2.2.1.6 & अग्निमादधानो दशहोत्रा ऽरणिमवदद्ध्यात् प्रजातमेवैन{-}माधत्ते &      \\
        \hline
            567 & TB\_2.2.1.7 & अभिचरन्{-}दशहोतारं जुहुयात् नव वै &      \\
        \hline
            568 & TB\_2.2.2.1 & प्रजापतिरकामयत दर्.शपूर्णमासौ सृजेयेति स &      \\
        \hline
            569 & TB\_2.2.2.2 & ग्रहो भवति दर्.शपूर्णमासयोः सृष्टयोर्द्धृत्यै &      \\
        \hline
            570 & TB\_2.2.2.3 & पञ्चहोतारं मनसाऽनुद्रुत्या हवनीये जुहुयात् &      \\
        \hline
            571 & TB\_2.2.2.4 & तद्ग्रहस्य ग्रहत्वम् पशुबन्धेन यक्ष्यमाणः &      \\
        \hline
            572 & TB\_2.2.2.5 & सोऽस्माथ् सृष्टोऽपाक्रामत् तं ग्रहेणागृह्णात् &      \\
        \hline
            573 & TB\_2.2.2.6 & यथ् सभांराः ततो वै &      \\
        \hline
            574 & TB\_2.2.3.1 & प्रजापति{-}रकामयत प्र जायेयेति स &      \\
        \hline
            575 & TB\_2.2.3.2 & तं पूर्वपक्षे याजयेत् वसीयानेव &      \\
        \hline
            576 & TB\_2.2.3.3 & प्रजाः पशव इमे लोकाः &      \\
        \hline
            577 & TB\_2.2.3.4 & ते तपोऽतप्यन्त त आत्मन्निन्द्र{-}मपश्यन्न् &      \\
        \hline
            578 & TB\_2.2.3.5 & तं चतुर्.होत्रा प्राजनयन्न् यः &      \\
        \hline
            579 & TB\_2.2.3.6 & त आदित्या एतं पञ्चहोतार{-}मपश्यन्न् &      \\
        \hline
            580 & TB\_2.2.3.7 & क्व स्थ क्व वः &      \\
        \hline
            581 & TB\_2.2.4.1 & प्रजापति{-}रकामयत प्रजायेयेति स एतं &      \\
        \hline
            582 & TB\_2.2.4.2 & असृक्षि वा इममिति तस्य &      \\
        \hline
            583 & TB\_2.2.4.3 & सोऽन्तरिक्ष{-}मसृजत चातुर्मास्यानि सामानि स &      \\
        \hline
            584 & TB\_2.2.4.4 & प्र प्रजया पशुभिर्{-}मिथुनैर्{-}जायते स &      \\
        \hline
            585 & TB\_2.2.4.5 & तद्{-}दुर्वर्णꣳ हिरण्यमभवत् तद्{-}दुर्वर्णस्य हिरण्यस्य &      \\
        \hline
            586 & TB\_2.2.4.6 & सुवर्ण आत्मना भवति दुर्वर्णोऽस्य &      \\
        \hline
            587 & TB\_2.2.4.7 & सप्तदशेन प्राजायत य एवं &      \\
        \hline
            588 & TB\_2.2.5.1 & देवा वै वरुणमयाजयन्न् स &      \\
        \hline
            589 & TB\_2.2.5.2 & राजा त्वा वरुणो नयतु &      \\
        \hline
            590 & TB\_2.2.5.3 & वारुणो वा अश्वः स्वयैवैनं &      \\
        \hline
            591 & TB\_2.2.5.4 & अनयैवैनत्{-}प्रति गृह्णाति वैश्वानर्यार्चा रथं &      \\
        \hline
            592 & TB\_2.2.5.5 & यद्वै शिवम् तन्मयः आत्मन &      \\
        \hline
            593 & TB\_2.2.5.6 & कामो हि दाता कामः &      \\
        \hline
            594 & TB\_2.2.6.1 & अन्तो वा एष यज्ञ्स्य &      \\
        \hline
            595 & TB\_2.2.6.2 & अन्नमेवाव{-}रुन्धते मनसा प्रस्तौति मनसोद्गायति &      \\
        \hline
            596 & TB\_2.2.6.3 & चतुर्.होतॄन्. होता व्याचष्टे स्तुतमनुशꣳसति &      \\
        \hline
            597 & TB\_2.2.6.4 & तदेनं प्रकाशं गतम् प्रकाशं &      \\
        \hline
            598 & TB\_2.2.7.1 & प्रजापतिः प्रजा असृजत ताः &      \\
        \hline
            599 & TB\_2.2.7.2 & मित्रमेव भवतः प्रजापतिर् देवासुरानसृजत &      \\
        \hline
            600 & TB\_2.2.7.3 & य एवं ॅवेद अभि &      \\
        \hline
            601 & TB\_2.2.7.4 & तस्य वा इयं क्लृप्तिः &      \\
        \hline
            602 & TB\_2.2.8.1 & देवा वै चतुर्.होतृभिर्{-}यज्ञ्मतन्वत ते &      \\
        \hline
            603 & TB\_2.2.8.2 & ऋजुधैवैनममुं ॅलोकं गमयति चतुर्.होत्रा &      \\
        \hline
            604 & TB\_2.2.8.3 & इन्द्रियमेवात्मन्{-}धत्ते यो वै चतुर्.होतॄ{-}ननुसवनं &      \\
        \hline
            605 & TB\_2.2.8.4 & तृप्यति प्रजया पशुभिः उपैनं &  TB\_3.12.5.11       \\
        \hline
            606 & TB\_2.2.8.5 & इन्द्रः सप्तहोत्रा प्रजापतिर्{-}दशहोत्रा तेषां &      \\
        \hline
            607 & TB\_2.2.8.6 & आप्रीभिराप्नुवन्न् तदाप्रीणामाप्रित्वम् तमघ्नन्न् तस्य &      \\
        \hline
            608 & TB\_2.2.8.7 & तमवधिष्म पुनरिमꣳ सुवामहा इति &      \\
        \hline
            609 & TB\_2.2.8.8 & सर्वमायुरेति सोमो वै यशः &      \\
        \hline
            610 & TB\_2.2.9.1 & इदं ॅवा अग्रे नैव &      \\
        \hline
            611 & TB\_2.2.9.2 & तस्मात्{-}तेपानाज्ज्योति{-}रजायत तद्{-}भूयोऽतप्यत तस्मात्{-}तेपानादर्चि{-}रजायत तद्{-}भूयोऽतप्यत &      \\
        \hline
            612 & TB\_2.2.9.3 & स समुद्रोऽभवत् तस्माथ्{-}समुद्रस्य न &      \\
        \hline
            613 & TB\_2.2.9.4 & स कस्मा अज्ञि यद्{-}यस्या &      \\
        \hline
            614 & TB\_2.2.9.5 & य एवं ॅवेद नास्य &      \\
        \hline
            615 & TB\_2.2.9.6 & तेभ्यो मृन्मये पात्रेऽन्नमदुहत् याऽस्य &      \\
        \hline
            616 & TB\_2.2.9.7 & ताभ्यो दारुमये पात्रे पयोऽदुहत &      \\
        \hline
            617 & TB\_2.2.9.8 & तामपाहत सोऽहोरात्रयोः सन्धि{-}रभवत् सोऽकामयत &      \\
        \hline
            618 & TB\_2.2.9.9 & एते वै प्रजापतेर्दोहाः य &      \\
        \hline
            619 & TB\_2.2.9.10 & असतोऽधि मनोऽसृज्यत मनः प्रजापति{-}मसृजत &      \\
        \hline
            620 & TB\_2.2.10.1 & प्रजापति{-}रिन्द्रमसृज{-}तानुजावरं देवानाम् तं प्राहिणोत् &      \\
        \hline
            621 & TB\_2.2.10.2 & यदस्मिन्नादित्ये तदेनमब्रवीत् एतन्मे प्रयच्छ &      \\
        \hline
            622 & TB\_2.2.10.3 & विदुरेनं नाम्ना तदस्मै रुक्मं &      \\
        \hline
            623 & TB\_2.2.10.4 & चन्द्रवानेव भवति तं देवा &      \\
        \hline
            624 & TB\_2.2.10.5 & अयं ॅवा इदं परमोऽभूदिति &      \\
        \hline
            625 & TB\_2.2.10.6 & साद्ध्याः पराञ्चम् य एवं &      \\
        \hline
            626 & TB\_2.2.10.7 & पश्चात्{-}पर्यायन्न् स पश्चात्{-}पर्यवर्तयत ता &      \\
        \hline
            627 & TB\_2.2.10.8 & मुखमुत्तरतः ऊर्द्ध्वा उदायन्न् स &      \\
        \hline
            628 & TB\_2.2.11.1 & प्रजापति{-}रकामयत बहोर्भूयान्थ्{-}स्यामिति स एतं &      \\
        \hline
            629 & TB\_2.2.11.2 & तस्य प्रयुक्तीन्द्रो ऽजायत यः &      \\
        \hline
            630 & TB\_2.2.11.3 & पशुमानेव भवति सोऽकामयतर्तवो मे &      \\
        \hline
            631 & TB\_2.2.11.4 & स षड्ढोतुः सप्तहोतारं निरमिमीत &      \\
        \hline
            632 & TB\_2.2.11.5 & पद्भिर्मुखेन ते देवाः पशून्. &      \\
        \hline
            633 & TB\_2.2.11.6 & यदिदं किञ्च य एवं &      \\
        \hline
            634 & TB\_2.3.1.1 & ब्रह्मवादिनो वदन्ति किं चतुर्.होतृणां &      \\
        \hline
            635 & TB\_2.3.1.2 & प्रजापतिर्{-}दशहोता य एवं चतुर्.होतृणा{-}मृद्धिं &      \\
        \hline
            636 & TB\_2.3.1.3 & प्रत्येव तिष्ठति ब्रह्मवादिनो वदन्ति &      \\
        \hline
            637 & TB\_2.3.2.1 & दक्षिणां प्रतिग्रहीष्यन्थ्{-}सप्तदश कृत्वोऽपान्यात् आत्मानमेव &      \\
        \hline
            638 & TB\_2.3.2.2 & प्राणानेवास्योपदासयति यद्येनं पुनरुप शिक्षेयुः &      \\
        \hline
            639 & TB\_2.3.2.3 & क्लृता अस्मा ऋतव आयन्ति &      \\
        \hline
            640 & TB\_2.3.2.4 & साम्नोऽधि यजूꣳष्यसृजत यजुर्भ्योऽधि विष्णुम् &      \\
        \hline
            641 & TB\_2.3.2.5 & तदिन्द्रं ॅयश आर्च्छत् तदेनं &      \\
        \hline
            642 & TB\_2.3.3.1 & यो वा अविद्वान्{-}निवर्तयते विशीर्.षा &      \\
        \hline
            643 & TB\_2.3.3.2 & पृथिवी न्यवर्तयत सौषधीभिर्{-}वनस्पतिभि{-}रपुष्यत् वायुर्न्यवर्तयत &      \\
        \hline
            644 & TB\_2.3.4.1 & तस्य वा अग्नेर्. हिरण्यं &      \\
        \hline
            645 & TB\_2.3.4.2 & तदेतेनैव प्रत्यगृह्णात् तेन वै &  TB\_2.3.4.6       \\
        \hline
            646 & TB\_2.3.4.3 & चतुर्थमिन्द्रियस्यात्मन्{-}नुपाधत्ते य एवं ॅविद्वान् &      \\
        \hline
            647 & TB\_2.3.4.4 & अथ योऽविद्वान् प्रतिगृह्णाति पञ्चममस्येन्द्रियस्याप &      \\
        \hline
            648 & TB\_2.3.4.5 & तस्य वै मनोस्तल्पं प्रतिजग्रहुषः &      \\
        \hline
            649 & TB\_2.3.4.6 & तदेतेनैव प्रत्यगृह्णात् तेन वै & TB\_2.3.4.2        \\
        \hline
            650 & TB\_2.3.5.1 & ब्रह्मवादिनो वदन्ति यद्{-}दशहोतारः सत्रमासत &      \\
        \hline
            651 & TB\_2.3.5.2 & तेनौषधी{-}रसृजन्त यत् पञ्चहोतारः सत्रमासत &      \\
        \hline
            652 & TB\_2.3.5.3 & केनर्तून{-}कल्पयन्तेति धात्रा वै ते &      \\
        \hline
            653 & TB\_2.3.5.4 & एते वै देवा गृहपतयः &      \\
        \hline
            654 & TB\_2.3.5.5 & यद्वा इदं किं च &  TB\_3.2.2.4       \\
        \hline
            655 & TB\_2.3.5.6 & सर्वा दिशोऽभिजयति प्रजापतिर्वै दशहोतृणां &      \\
        \hline
            656 & TB\_2.3.6.1 & प्रजापतिः प्रजाः सृष्ट्वा व्यस्रꣳसत &      \\
        \hline
            657 & TB\_2.3.6.2 & ते प्रत्यशृण्वन्न् ते दर्.शपूर्णमासाभ्यामेव &      \\
        \hline
            658 & TB\_2.3.6.3 & पञ्चर्त्विजः षट्कृत्वोऽह्वयत् ऋतवः प्रत्यशृण्वन्न् &      \\
        \hline
            659 & TB\_2.3.6.4 & ता उपौहन्थ्{-}सप्तशीर्.षण्यान्{-}प्राणान् तस्माथ्{-}सौम्यस्याद्ध्वरस्य यज्ञ्क्रतोः &      \\
        \hline
            660 & TB\_2.3.7.1 & प्रजापतिः पुरुष{-}मसृजत सोऽग्निरब्रवीत् ममाय{-}मन्नमस्त्विति &      \\
        \hline
            661 & TB\_2.3.7.2 & कुसिन्धं चात्मनः स्पृणोति आदित्यस्य &      \\
        \hline
            662 & TB\_2.3.7.3 & तानि चात्मनः स्पृणोति आदित्यस्य &      \\
        \hline
            663 & TB\_2.3.7.4 & समिथ् सप्तमी सप्तहोतारमेव तद्{-}यज्ञ्क्रतु{-}माप्नोति &      \\
        \hline
            664 & TB\_2.3.8.1 & अपक्रामत गर्भिण्यः इस् ओन्ल्य् &      \\
        \hline
            665 & TB\_2.3.8.2 & तेनासुनाऽसुरानसृजत तदसुराणा{-}मसुरत्वम् य एवमसुराणा{-}मसुरत्वं &      \\
        \hline
            666 & TB\_2.3.8.3 & यन्त्यस्य पितरो हवम् स &      \\
        \hline
            667 & TB\_2.3.9.1 & यथास्थानं गर्भिण्यः इस् ऒन्ल्य् &      \\
        \hline
            668 & TB\_2.3.9.2 & यतोऽयं पवते यदभि पवते &      \\
        \hline
            669 & TB\_2.3.9.3 & अस्याः पवते इमामभि पवते &      \\
        \hline
            670 & TB\_2.3.9.4 & आदित्यमभि पवते आदित्यमभि सं &      \\
        \hline
            671 & TB\_2.3.9.5 & तस्मात्{-}पुरस्ता{-}द्वान्तम् सर्वाः प्रजाः प्रति &      \\
        \hline
            672 & TB\_2.3.9.6 & सर्वा दिशोऽनु विवाति सर्वा &      \\
        \hline
            673 & TB\_2.3.9.7 & अथ यदुत्तरतो वाति सवितैव &      \\
        \hline
            674 & TB\_2.3.9.8 & य एवास्य दक्षिणतः पाप्मानः &      \\
        \hline
            675 & TB\_2.3.9.9 & उत्तरत इतरान् पाप्मनः सचन्ते &      \\
        \hline
            676 & TB\_2.3.10.1 & प्रजापतिः सोमꣳ राजानमसृजत तं &      \\
        \hline
            677 & TB\_2.3.10.2 & प्र त्वा पद्ये सोमं &      \\
        \hline
            678 & TB\_2.3.10.3 & आऽस्यार्द्धं ॅवव्राज ताꣳ होदीक्ष्योवाच &      \\
        \hline
            679 & TB\_2.3.10.4 & यं ॅवा कामयेत प्रियः &      \\
        \hline
            680 & TB\_2.3.11.1 & ब्रह्मात्मन्{-}वदसृजत तदकामयत समात्मना पद्येयेति &      \\
        \hline
            681 & TB\_2.3.11.2 & आत्मन्{-}नात्मन्{-}नित्यामन्त्रयत तस्मै सप्तमꣳ हूतः &      \\
        \hline
            682 & TB\_2.3.11.3 & षड्ढूतो ह वै नामैषः &      \\
        \hline
            683 & TB\_2.3.11.4 & परोक्षप्रिया इव हि देवाः &      \\
        \hline
            684 & TB\_2.4.1.1 & जुष्टो दमूना अतिथिर्दुरोणे इमं &      \\
        \hline
            685 & TB\_2.4.1.2 & ताꣳस्त्वं ॅवृत्रहञ्जहि वस्वस्मभ्य{-}माभर अग्ने &      \\
        \hline
            686 & TB\_2.4.1.3 & अपापाचो अभिभूते नुदस्व अपोदीचो &      \\
        \hline
            687 & TB\_2.4.1.4 & हव्यवाह{-}मभिमातिषाऽहम् रक्षोहणं पृतनासु जिष्णुम् &      \\
        \hline
            688 & TB\_2.4.1.5 & मन्ये त्वा जातवेदसम् स &      \\
        \hline
            689 & TB\_2.4.1.6 & सो अस्माꣳ अभयतमेन नेषत् &      \\
        \hline
            690 & TB\_2.4.1.7 & प्रतिष्म देव रीषतः तपिष्ठैरजरो &      \\
        \hline
            691 & TB\_2.4.1.8 & आ सिन्धोरा परावतः दक्षं &      \\
        \hline
            692 & TB\_2.4.1.9 & प्रण आयूꣳषि तारिषत् त्वमग्ने &      \\
        \hline
            693 & TB\_2.4.1.10 & स इदं प्रति पप्रथे &      \\
        \hline
            694 & TB\_2.4.1.11 & मन्युर्भगो मन्युरेवास देवः मन्युर्. &      \\
        \hline
            695 & TB\_2.4.2.1 & चक्षुषो हेते मनसो हेते &      \\
        \hline
            696 & TB\_2.4.2.2 & तन् मृत्युर् निर्.ऋत्या सम्ॅविदानः &      \\
        \hline
            697 & TB\_2.4.2.3 & अपनह्यामि ते बाहू अपनह्याम्यास्यम् &      \\
        \hline
            698 & TB\_2.4.2.4 & वषट्कारेण वज्रेण कृत्याꣳ हन्मि &      \\
        \hline
            699 & TB\_2.4.2.5 & गिरा यज्ञ्स्य साधनम् श्रुष्टीवानं &      \\
        \hline
            700 & TB\_2.4.2.6 & तिग्मशृङ्गो वृषभः शोशुचानः प्रत्नं &      \\
        \hline
            701 & TB\_2.4.2.7 & वयꣳ सोम व्रते तव &      \\
        \hline
            702 & TB\_2.4.2.8 & सा नो देवी सुहवा &      \\
        \hline
            703 & TB\_2.4.2.9 & पर्वत इवा विचाचलिः इन्द्र &      \\
        \hline
            704 & TB\_2.4.3.1 & जुष्टी नरो ब्रह्मणा वः &      \\
        \hline
            705 & TB\_2.4.3.2 & मा त्वत्क्षेत्रा{-}ण्यरणानि गन्म वृञ्जे &      \\
        \hline
            706 & TB\_2.4.3.3 & इमं नो यज्ञ्ं ॅविहवे &      \\
        \hline
            707 & TB\_2.4.3.4 & अग्निश्च विष्णो तप उत्तमं &      \\
        \hline
            708 & TB\_2.4.3.5 & प्र यः सत्राचा मनसा &      \\
        \hline
            709 & TB\_2.4.3.6 & होतारं चित्ररथ{-}मद्ध्वरस्य यज्ञ्स्य यज्ञ्स्य &      \\
        \hline
            710 & TB\_2.4.3.7 & प्रारोचयन्मनवे केतुमह्नाम् अविन्दज्ज्योतिर्{-}बृहते रणाय &      \\
        \hline
            711 & TB\_2.4.3.8 & त्वꣳ सोम क्रतुभिः सुक्रतुर्भूः &      \\
        \hline
            712 & TB\_2.4.3.9 & अधा ते विष्णो विदुषाचिदृध्यः &      \\
        \hline
            713 & TB\_2.4.3.10 & इमा धाना घृतस्नुवः हरी &      \\
        \hline
            714 & TB\_2.4.3.11 & हरिवर्पसं गिरः आचर्.षणिप्रा वृषभो &      \\
        \hline
            715 & TB\_2.4.3.12 & अरं ते सोमस्तनुवे भवाति &      \\
        \hline
            716 & TB\_2.4.3.13 & इन्द्रागहि प्रथमो यज्ञियानाम् या &      \\
        \hline
            717 & TB\_2.4.4.1 & नक्तं जाताऽस्योषधे रामे कृष्णे &      \\
        \hline
            718 & TB\_2.4.4.2 & असिक्नियस्योषधे निरितो नाशया पृषत् &      \\
        \hline
            719 & TB\_2.4.4.3 & शुनꣳ हुवेम मघवानमिन्द्रम् अस्मिन्भरे &      \\
        \hline
            720 & TB\_2.4.4.4 & प्रोवारत मरुतो दुर्मदा इव &      \\
        \hline
            721 & TB\_2.4.4.5 & देवा भागं ॅयथा पूर्वे &      \\
        \hline
            722 & TB\_2.4.4.6 & सज्ञांनं नः स्वैः सज्ञांनमरणैः &      \\
        \hline
            723 & TB\_2.4.4.7 & अग्ने हिरण्यसंदृशः अदब्धेभिः सवितः &      \\
        \hline
            724 & TB\_2.4.4.8 & शिप्रिन् वाजानां पते शचीवस्तव &      \\
        \hline
            725 & TB\_2.4.4.9 & यजत्रा मुञ्चतेह मा यज्ञिर्वो &      \\
        \hline
            726 & TB\_2.4.4.10 & द्रुपदादिव मुञ्चतु द्रुपदादिवेन्{-}मुमुचानः स्विन्नः &      \\
        \hline
            727 & TB\_2.4.5.1 & वृषा सो अꣳशुः पवते &      \\
        \hline
            728 & TB\_2.4.5.2 & तꣳ सद्ध्रीचीरूतयो वृष्णियानि पौꣳस्यानि &      \\
        \hline
            729 & TB\_2.4.5.3 & दृढान्यौघ्नादुशमान ओजः अवाभिनत्ककुभः पर्वतानाम् &      \\
        \hline
            730 & TB\_2.4.5.4 & रजाꣳसि चित्रा वि चरन्ति &      \\
        \hline
            731 & TB\_2.4.5.5 & अवव्ययन्नसितं देव वस्वः दविद्ध्वतो &      \\
        \hline
            732 & TB\_2.4.5.6 & यो गर्भमोषधीनाम् गवां कृणोत्यर्वताम् &      \\
        \hline
            733 & TB\_2.4.5.7 & इन्द्रश्च नः शुनासीरौ इमं &      \\
        \hline
            734 & TB\_2.4.6.1 & उत नः प्रिया प्रियासु &      \\
        \hline
            735 & TB\_2.4.6.2 & तदस्य प्रियमभि पाथो अश्याम् &      \\
        \hline
            736 & TB\_2.4.6.3 & उप नः सूनवो गिरः &      \\
        \hline
            737 & TB\_2.4.6.4 & अद्ध्वरेषु नमस्यत अनाम्योज आचके &      \\
        \hline
            738 & TB\_2.4.6.5 & कनात्काभां न आभर प्रयफ्स्यन्निव &      \\
        \hline
            739 & TB\_2.4.6.6 & मुष्कयोर्निहितः सपः सृत्वेव कामस्य &      \\
        \hline
            740 & TB\_2.4.6.7 & यज्ञ्स्य काम्यः प्रियः ददामीत्यग्निर्वदति &      \\
        \hline
            741 & TB\_2.4.6.8 & अथेममन्थन्नमृतममूराः वैश्वानरं क्षेत्रजित्याय देवाः &      \\
        \hline
            742 & TB\_2.4.6.9 & अन्तरिक्षं ॅविपप्रथे दुहे द्यौः &      \\
        \hline
            743 & TB\_2.4.6.10 & रात्री व्यख्यदायती पुरुत्रा देव्यक्षभिः &      \\
        \hline
            744 & TB\_2.4.6.11 & यद् वाग्वदन्त्य{-}विचेतनानि राष्ट्री देवानां &      \\
        \hline
            745 & TB\_2.4.6.12 & ततः क्षरत्यक्षरम् तद्{-}विश्व{-}मुपजीवति इन्द्रा &      \\
        \hline
            746 & TB\_2.4.7.1 & वृषा ऽस्यꣳशुर्{-}वृषभाय गृह्यसे वृषाऽयमुग्रो &      \\
        \hline
            747 & TB\_2.4.7.2 & अस्याः पृथिव्या अद्ध्यक्षम् इममिन्द्र &      \\
        \hline
            748 & TB\_2.4.7.3 & यस्यायमृषभो हविः इन्द्राय परिणीयते &      \\
        \hline
            749 & TB\_2.4.7.4 & प्रेह्यभिप्रेहि प्रभरा सहस्व मा &      \\
        \hline
            750 & TB\_2.4.7.5 & सुवर्वतीरप एना जयेम यो &      \\
        \hline
            751 & TB\_2.4.7.6 & इन्द्रो जातो वि पुरो &      \\
        \hline
            752 & TB\_2.4.7.7 & आया हि सोमपीतये स्वारुहो &      \\
        \hline
            753 & TB\_2.4.7.9 & विश्वा आशाः पृतनाः सजंयञ्जयन्न् &      \\
        \hline
            754 & TB\_2.4.7.10 & प्र सद्यो अग्ने अत्यष्यन्यान् &      \\
        \hline
            755 & TB\_2.4.7.11 & ब्रह्म यज्ञ्स्य तन्तवः ऋत्विजो &      \\
        \hline
            756 & TB\_2.4.8.1 & स प्रत्नवन्नवीयसा अग्ने द्युम्नेन &      \\
        \hline
            757 & TB\_2.4.8.2 & अग्निर्विशां मानुषीणाम् तूर्णी रथः &      \\
        \hline
            758 & TB\_2.4.8.3 & स नो रास्व सहस्रिणः &      \\
        \hline
            759 & TB\_2.4.8.4 & उभा हि वाꣳ सुहवा &      \\
        \hline
            760 & TB\_2.4.8.5 & हविषोऽस्य नवस्य नः सुवर्विदो &      \\
        \hline
            761 & TB\_2.4.8.6 & इमे धेनू अमृतं ॅये &      \\
        \hline
            762 & TB\_2.4.8.7 & स एना विद्वान्. यक्ष्यसि &      \\
        \hline
            763 & TB\_2.5.1.1 & प्राणो रक्षति विश्वमेजत् इर्यो &      \\
        \hline
            764 & TB\_2.5.1.2 & आ न एतु पुरश्चरम् &      \\
        \hline
            765 & TB\_2.5.1.3 & वाग्देवी जुषतामिदꣳ हविः चक्षुः &      \\
        \hline
            766 & TB\_2.5.2.1 & उदेहि वाजिन्यो अस्यफ्स्वन्तः इदं &      \\
        \hline
            767 & TB\_2.5.2.2 & अस्मभ्यं द्यावापृथिवी शक्वरीभिः राष्ट्रं &      \\
        \hline
            768 & TB\_2.5.2.3 & यूयमुग्रा मरुतः पृश्निमातरः इन्द्रेण &      \\
        \hline
            769 & TB\_2.5.2.4 & सो अन्तरिक्षे रजसो विमानः &      \\
        \hline
            770 & TB\_2.5.3.1 & पुनर्न इन्द्रो मघवा ददातु &      \\
        \hline
            771 & TB\_2.5.3.2 & जेता शत्रून्. विचर्.षणिः आकूत्यै &      \\
        \hline
            772 & TB\_2.5.3.3 & सेदग्निरग्नीꣳ रत्येत्यन्यान् यत्र वाजी &      \\
        \hline
            773 & TB\_2.5.3.4 & अगृभीताः पशवः सन्तु सर्वे &      \\
        \hline
            774 & TB\_2.5.4.1 & आ नो भर भगमिन्द्र &      \\
        \hline
            775 & TB\_2.5.4.2 & अह{-}न्नहिमन्वपस्ततर्द प्र वक्षणा अभिनत् &      \\
        \hline
            776 & TB\_2.5.4.3 & यदिन्द्राहन्{-}प्रथमजा महीनाम् आन्मायिनाममिनाः प्रोत &      \\
        \hline
            777 & TB\_2.5.4.4 & नातारीरस्य समृतिं ॅवधानाम् सं &      \\
        \hline
            778 & TB\_2.5.4.5 & उदुज्जिहानो अभि काममीरयन्न् प्रपृञ्चन् &      \\
        \hline
            779 & TB\_2.5.4.6 & इमं ॅयज्ञ्मश्विना वर्द्धयन्ता इमौ &      \\
        \hline
            780 & TB\_2.5.5.1 & यज्ञो रायो यज्ञ् ईशे &      \\
        \hline
            781 & TB\_2.5.5.2 & तं त्वा भग सर्व &      \\
        \hline
            782 & TB\_2.5.5.3 & इयमेव सा या प्रथमा &      \\
        \hline
            783 & TB\_2.5.5.4 & धारावरा मरुतो धृष्णुवोजसः मृगा &      \\
        \hline
            784 & TB\_2.5.5.5 & हिरण्यवाशीरिषिरः सुवर.षाः बृहस्पतिः स &      \\
        \hline
            785 & TB\_2.5.5.6 & अरण्यान्यरण्यान्यसौ या प्रेव नश्यसि &      \\
        \hline
            786 & TB\_2.5.5.7 & उतो अरण्यानिः सायम् शकटीरिव &      \\
        \hline
            787 & TB\_2.5.6.1 & वार्त्रहत्याय शवसे पृतना{-}साह्याय च &      \\
        \hline
            788 & TB\_2.5.6.2 & शिवे ते द्यावापृथिवी उभे &      \\
        \hline
            789 & TB\_2.5.6.3 & द्रुहः पाशान्निर्.ऋत्यै चोदमोचि अहा &      \\
        \hline
            790 & TB\_2.5.6.4 & यूयं पात स्वस्तिभिः सदा &  TB\_2.5.8.5       \\
        \hline
            791 & TB\_2.5.6.5 & तां त्वा मुद्गला हविषा &      \\
        \hline
            792 & TB\_2.5.7.1 & वसूनां त्वाऽऽधीतेन रुद्राणामूर्म्या आदित्यानां &      \\
        \hline
            793 & TB\_2.5.7.2 & दीर्घायुत्वाय शतशारदाय प्रतिगृभ्णामि महते &      \\
        \hline
            794 & TB\_2.5.7.3 & प्रजापतिः प्रणेता बृहस्पतिः पुर &      \\
        \hline
            795 & TB\_2.5.7.4 & मित्रः क्षत्रं क्षत्रपतिः क्षत्रमस्मिन्. &      \\
        \hline
            796 & TB\_2.5.8.1 & स ईं पाहि य &      \\
        \hline
            797 & TB\_2.5.8.2 & अभीके चिदु लोककृत् सङ्गे &      \\
        \hline
            798 & TB\_2.5.8.3 & जुषतां प्रति मेधिरः प्र &      \\
        \hline
            799 & TB\_2.5.8.4 & मरुतो वृत्रहन्तमम् येन ज्योति{-}रजनयन्नृतावृधः &      \\
        \hline
            800 & TB\_2.5.8.5 & यूयं पात स्वस्तिभिः सदा & TB\_2.5.6.4        \\
        \hline
            801 & TB\_2.5.8.6 & अपो याचामि भेषजम् अफ्सु &      \\
        \hline
            802 & TB\_2.5.8.7 & तस्यां मे रास्व तस्यास्ते &      \\
        \hline
            803 & TB\_2.5.8.8 & ज्योतिषा त्वा वैश्वानरेणोपतिष्ठे अयं &      \\
        \hline
            804 & TB\_2.5.8.9 & देवेभ्यो हव्यं ॅवह नः &      \\
        \hline
            805 & TB\_2.5.8.10 & सैनं सश्चद्{-}देवं देवः सत्यमिन्दुं &      \\
        \hline
            806 & TB\_2.5.8.11 & भूरीणि वृत्वा हर्यश्व हꣳसि &      \\
        \hline
            807 & TB\_2.5.8.12 & अपां भूमानमुप नः सृजेह &      \\
        \hline
            808 & TB\_2.6.1.1 & स्वाद्वीं त्वा स्वादुना तीव्रां &      \\
        \hline
            809 & TB\_2.6.1.2 & दधन्वा यो नर्यो अफ्स्वन्तरा &      \\
        \hline
            810 & TB\_2.6.1.3 & इन्द्रस्य युज्यः सखा ब्रह्म &      \\
        \hline
            811 & TB\_2.6.1.4 & सरस्वत्या इन्द्राय सुत्राम्णे एष &      \\
        \hline
            812 & TB\_2.6.1.5 & उपयामगृहीतोऽस्याश्विनं तेजः सारस्वतं ॅवीर्यम् &      \\
        \hline
            813 & TB\_2.6.2.1 & सोमो राजाऽमृतꣳ सुतः ऋजीषेणाजहान्मृत्युम् &      \\
        \hline
            814 & TB\_2.6.2.2 & क्रुङ्ङाङ्गिरसो धिया ऋतेन सत्यमिन्द्रियम् &      \\
        \hline
            815 & TB\_2.6.2.3 & वेदेन रूपे व्यकरोत् सतासती &      \\
        \hline
            816 & TB\_2.6.3.1 & सुरावन्तं बर्.हिषदं सुवीरम् यज्ञ्ं &      \\
        \hline
            817 & TB\_2.6.3.2 & इमं तꣳ शुक्रं मधुमन्त{-}मिन्दुम् &      \\
        \hline
            818 & TB\_2.6.3.3 & अमीमदन्त पितरः अतीतृपन्त पितरः &      \\
        \hline
            819 & TB\_2.6.3.4 & पवित्रेण शतायुषा विश्वमायुर्{-}व्यश्नवै ।अग्न &      \\
        \hline
            820 & TB\_2.6.3.5 & ये सजाताः समनसः जीवा &      \\
        \hline
            821 & TB\_2.6.4.1 & सीसेन तन्त्रं मनसा मनीषिणः &      \\
        \hline
            822 & TB\_2.6.4.2 & अस्थि मज्जानं मासरैः कारोतरेण &      \\
        \hline
            823 & TB\_2.6.4.3 & इन्द्रः सुत्रामा हृदयेन सत्यम् &      \\
        \hline
            824 & TB\_2.6.4.4 & प्लाशीर्व्यक्तः शतधार उथ्सः दुहे &      \\
        \hline
            825 & TB\_2.6.4.5 & अविर्न मेषो नसि वीर्याय &      \\
        \hline
            826 & TB\_2.6.4.6 & केशा न शीर.षन्. यशसे &      \\
        \hline
            827 & TB\_2.6.5.1 & मित्रोऽसि वरुणोऽसि समहं ॅविश्वैर्देवैः &      \\
        \hline
            828 & TB\_2.6.5.2 & साम्राज्याय सुक्रतुः देवस्य त्वा &      \\
        \hline
            829 & TB\_2.6.5.3 & वीर्यायान्नाद्यायाभिषिञ्चामि देवस्य त्वा सवितुः &      \\
        \hline
            830 & TB\_2.6.5.4 & यशो मुखम् त्विषिः केशाश्च &      \\
        \hline
            831 & TB\_2.6.5.5 & चित्तं मे सहः बाहू &      \\
        \hline
            832 & TB\_2.6.5.6 & आनन्दनन्दावाण्डौ मे भगः सौभाग्यं &      \\
        \hline
            833 & TB\_2.6.5.7 & त्रया देवा एकादश त्रयस्त्रिꣳशाः &      \\
        \hline
            834 & TB\_2.6.5.8 & यजूꣳषि सामभिः सामान्यृग्भिः ऋचो &      \\
        \hline
            835 & TB\_2.6.6.1 & यद्{-}देवा देवहेडनम् देवासश्चकृमा वयम् &      \\
        \hline
            836 & TB\_2.6.6.2 & सूर्यो मा तस्मादेनसः विश्वान् &      \\
        \hline
            837 & TB\_2.6.6.3 & अवभृथ निचङ्कुण निचेरुरसि निचङ्कुण &      \\
        \hline
            838 & TB\_2.6.6.4 & पूतं पवित्रेणेवाज्यम् आपः शुन्धन्तु &      \\
        \hline
            839 & TB\_2.6.6.5 & तेजोऽसि तेजो मयि धेहि &      \\
        \hline
            840 & TB\_2.6.7.1 & होता यक्षथ्{-}समिधेन्द्र{-}मिडस्पदे नाभा पृथिव्या &      \\
        \hline
            841 & TB\_2.6.7.2 & वेत्वाज्यस्य होतर्यज होता यक्षदिडाभि{-}रिन्द्र{-}मीडितम् &      \\
        \hline
            842 & TB\_2.6.7.3 & वेत्वाज्यस्य होतर्यज होता यक्षदोजो &      \\
        \hline
            843 & TB\_2.6.7.4 & वीतामाज्यस्य होतर्यज होता यक्षद् &      \\
        \hline
            844 & TB\_2.6.7.5 & महीन्द्रपत्नीर्.{-}हविष्मतीः वियन्त्वाज्यस्य होतर्यज होता &      \\
        \hline
            845 & TB\_2.6.7.6. & मद्ध्वा समञ्जन्पथिभिः सुगेभिः स्वदाति &      \\
        \hline
            846 & TB\_2.6.8.1 & समिद्ध इन्द्र उषसामनीके पुरोरुचा &      \\
        \hline
            847 & TB\_2.6.8.2 & पुरदंरो मघवान्. वज्रबाहुः आयातु &      \\
        \hline
            848 & TB\_2.6.8.3 & उषासा नक्ता बृहती बृहन्तम् &      \\
        \hline
            849 & TB\_2.6.8.4 & अच्छिन्नं तन्तुं पयसा सरस्वती &      \\
        \hline
            850 & TB\_2.6.9.1 & आचर्.षणिप्रा, विवेष यन्मा तं &      \\
        \hline
            851 & TB\_2.6.10.1 & देवं बर्.हिरिन्द्रꣳ सुदेवं देवैः &      \\
        \hline
            852 & TB\_2.6.10.2 & देवी उषासा नक्ता इन्द्रं &      \\
        \hline
            853 & TB\_2.6.10.3 & वसुवने वसुधेयस्य वीतां ॅयज &      \\
        \hline
            854 & TB\_2.6.10.5 & वसुवने वसुधेयस्य वियन्तु यज &      \\
        \hline
            855 & TB\_2.6.10.6 & देव इन्द्रो वनस्पतिः हिरण्यपर्णो &  TB\_2.6.14.5       \\
        \hline
            856 & TB\_2.6.11.1 & होता यक्षथ्{-}समिधाऽग्नि{-}मिडस्पदे अश्विनेन्द्रꣳ सरस्वतीम् &      \\
        \hline
            857 & TB\_2.6.11.2 & बदरैरुपवाकाभिर् भेषजं तोक्मभिः पयः &      \\
        \hline
            858 & TB\_2.6.11.3 & होता यक्षदिडेडित आजुह्वानः सरस्वतीम् &      \\
        \hline
            859 & TB\_2.6.11.4 & भिषजाऽश्विनाऽश्वा शिशुमती भिषग्धेनुः सरस्वती &      \\
        \hline
            860 & TB\_2.6.11.5 & अश्विनेन्द्राय भेषजम् शुक्रं न &      \\
        \hline
            861 & TB\_2.6.11.6. & वियन्त्वाज्यस्य होतर्यज होता यक्षद्दैव्या &      \\
        \hline
            862 & TB\_2.6.11.7 & अश्विनेडा न भारती वाचा &      \\
        \hline
            863 & TB\_2.6.11.8 & श्रिया न मासरम् पयः &      \\
        \hline
            864 & TB\_2.6.11.9 & होता यक्षदग्निꣳ स्वाहाऽऽज्यस्य स्तोकानाम् &      \\
        \hline
            865 & TB\_2.6.11.10 & स्वाहाऽग्निꣳ होत्राज्जुषाणो अग्निर् भेषजम् &      \\
        \hline
            866 & TB\_2.6.12.1 & समिद्धो अग्निरश्विना तप्तो घर्मो &      \\
        \hline
            867 & TB\_2.6.12.2 & अधातामश्विना मधु भेषजं भिषजा &      \\
        \hline
            868 & TB\_2.6.12.3 & कवष्यो न व्यचस्वतीः अश्विभ्यां &      \\
        \hline
            869 & TB\_2.6.12.4 & दैव्या होतारा भिषजा पातमिन्द्रं &      \\
        \hline
            870 & TB\_2.6.13.1 & अश्विना हविरिन्द्रियम् नमुचेर्द्धिया सरस्वती &      \\
        \hline
            871 & TB\_2.6.13.2 & दधाना अभ्यनूषत हविषा यज्ञ्मिन्द्रियम् &      \\
        \hline
            872 & TB\_2.6.13.3 & वरुणः क्षत्रमिन्द्रियम् भगेन सविता &      \\
        \hline
            873 & TB\_2.6.14.1 & देवं बर्.हिः सरस्वती सुदेवमिन्द्रे &      \\
        \hline
            874 & TB\_2.6.14.2 & देवी उषासावश्विना भिषजेन्द्रे सरस्वती &      \\
        \hline
            875 & TB\_2.6.14.3 & देवी ऊर्जाहुती दुघे सुदुघे &      \\
        \hline
            876 & TB\_2.6.14.4 & देवी{-}स्तिस्र{-}स्तिस्रो देवीः सरस्वत्यश्विना भारतीडा &      \\
        \hline
            877 & TB\_2.6.14.5 & देव इन्द्रो वनस्पतिः हिरण्यपर्णो & TB\_2.6.10.6        \\
        \hline
            878 & TB\_2.6.14.6. & स्योनमिन्द्र ते सदः ईशायै &      \\
        \hline
            879 & TB\_2.6.15.1 & अग्निमद्य होतारमवृणीत अयꣳ सुतासुती &      \\
        \hline
            880 & TB\_2.6.15.2 & सरस्वत्यै मेषेणेन्द्रायाश्विभ्याम् इन्द्रायर्.षभेणाश्विभ्याꣳ सरस्वत्यै &      \\
        \hline
            881 & TB\_2.6.16.1 & उशन्तस्त्वा हवामह, आ नो &      \\
        \hline
            882 & TB\_2.6.16.2 & अꣳहोमुचः पितरः सोम्यासः परेऽवरेऽमृतासो &      \\
        \hline
            883 & TB\_2.6.17.1 & होता यक्षदिडस्पदे समिधानं महद्यशः &      \\
        \hline
            884 & TB\_2.6.17.2 & शुचिमिन्द्रं ॅवयोधसम् उष्णिहं छन्द &      \\
        \hline
            885 & TB\_2.6.17.3 & वेत्वाज्यस्य होतर्यज होता यक्षथ् &      \\
        \hline
            886 & TB\_2.6.17.4 & द्वारो देवीर्.हिरण्ययीः ब्रह्माण इन्द्रं &      \\
        \hline
            887 & TB\_2.6.17.5 & पष्ठवाहं गां ॅवयो दधत् &      \\
        \hline
            888 & TB\_2.6.17.6. & तिस्रो देवीर्.हिरण्ययीः भारतीर्{-}बृहतीर्महीः पतिमिन्द्रं &      \\
        \hline
            889 & TB\_2.6.17.7 & द्विपदं छन्द इहेन्द्रियम् उक्षाणं &      \\
        \hline
            890 & TB\_2.6.18.1 & समिद्धो अग्निः समिधा सुषमिद्धो &      \\
        \hline
            891 & TB\_2.6.18.2 & अनुष्टुप्छन्द इन्द्रियम् त्रिवथ्सो गौर्वयो &      \\
        \hline
            892 & TB\_2.6.18.3 & उषे यह्वी सुपेशसा विश्वे &      \\
        \hline
            893 & TB\_2.6.18.4 & विराट्छन्द इहेन्द्रियम् धेनुर्गौर्न वयो &      \\
        \hline
            894 & TB\_2.6.19.1 & वसन्तेनर्तुना देवाः वसवस्त्रिवृता स्तुतम् &      \\
        \hline
            895 & TB\_2.6.19.2 & वैरूपेण विशौजसा हविरिन्द्रे वयो &      \\
        \hline
            896 & TB\_2.6.20.1 & देवं बर्.हिरिन्द्रं ॅवयोधसम् देवं &      \\
        \hline
            897 & TB\_2.6.20.2 & देवी देवं ॅवयोधसम् उषे &      \\
        \hline
            898 & TB\_2.6.20.3 & देवी ऊर्जाहुती देवमिन्द्रं ॅवयोधसम् &      \\
        \hline
            899 & TB\_2.6.20.4 & देवी{-}स्तिस्र{-}स्तिस्रो देवीर्वयोधसम् पतिमिन्द्र{-}मवर्द्धयन्न् जगत्या &      \\
        \hline
            900 & TB\_2.6.20.5 & देवो वनस्पतिर्{-}देवमिन्द्रं ॅवयोधसम् देवो &      \\
        \hline
            901 & TB\_2.7.1.1 & त्रिवृथ्{-}स्तोमो भवति ब्रह्मवर्चसं ॅवै &      \\
        \hline
            902 & TB\_2.7.1.2 & अरुणो मिर्मिरस्त्रिशुक्रः एतद् वै &      \\
        \hline
            903 & TB\_2.7.1.3 & पुरोधामेव गच्छति तस्य प्रातस्सवने &      \\
        \hline
            904 & TB\_2.7.1.4 & प्रजापति{-}श्चतुस्त्रिꣳशो देवतानां यावतीरेव देवताः &      \\
        \hline
            905 & TB\_2.7.2.1 & यदाग्नेयो भवति अग्निमुखा ह्यृद्धिः &      \\
        \hline
            906 & TB\_2.7.2.2 & अथो य एव कश्च &      \\
        \hline
            907 & TB\_2.7.3.1 & यदाग्नेयो भवति आग्नेयो वै &      \\
        \hline
            908 & TB\_2.7.3.2 & अथ वीर्यावत्तरो भवति अथ &      \\
        \hline
            909 & TB\_2.7.3.3 & वज्रस्य वा एषोऽनुमानाय अनुमतवज्रः &      \\
        \hline
            910 & TB\_2.7.4.1 & न वै सोमेन सोमस्य &      \\
        \hline
            911 & TB\_2.7.5.1 & यो वै सोमेन सूयते &      \\
        \hline
            912 & TB\_2.7.5.2 & य एतेन यजते य &  TB\_2.7.6.2       \\
        \hline
            913 & TB\_2.7.6.1 & एष गोसवः षट्त्रिꣳश उक्थ्यो &      \\
        \hline
            914 & TB\_2.7.6.2 & य एतेन यजते य & TB\_2.7.5.2        \\
        \hline
            915 & TB\_2.7.6.3 & असौ बृहत् अनयोरेवैन{-}मनन्तर्.हित{-}मभिषिञ्चति पशुस्तोमो &      \\
        \hline
            916 & TB\_2.7.7.1 & सिꣳहे व्याघ्र उत या &      \\
        \hline
            917 & TB\_2.7.7.2 & इन्द्रं ॅया देवी सुभगा &      \\
        \hline
            918 & TB\_2.7.7.3 & इन्द्राय त्वा पयस्वते पयस्वन्तं &      \\
        \hline
            919 & TB\_2.7.7.4 & ओजस्वदस्तु मे मुखम् ओजस्वच्छिरो &      \\
        \hline
            920 & TB\_2.7.7.5 & आयुरसि तत् ते प्रयच्छामि &      \\
        \hline
            921 & TB\_2.7.7.6 & आयुरसि विश्वायुरसि सर्वायुरसि सर्वमायुरसि &      \\
        \hline
            922 & TB\_2.7.7.7 & अपां ॅयो द्रवणे रसः &      \\
        \hline
            923 & TB\_2.7.8.1 & अभिप्रेहि वीरयस्व उग्रश्चेत्ता सपत्नहा &      \\
        \hline
            924 & TB\_2.7.8.2 & अनु सोमो अन्वग्निरावीत् अनु &      \\
        \hline
            925 & TB\_2.7.9.1 & प्रजापतिः प्रजा असृजत ता & TB\_1.1.5.4  TB\_3.1.4.2       \\
        \hline
            926 & TB\_2.7.9.2 & सर्वे पुरुषाः सर्वाण्येवान्नान्यवरुन्धे सर्वान् &      \\
        \hline
            927 & TB\_2.7.9.3 & पुष्टिं तेन यत्{-}कमण्डलुम् आयुष्टेन &      \\
        \hline
            928 & TB\_2.7.9.4 & तदस्मिन्{-}नेकधाऽधात् रोहिण्यां कार्यः यद्ब्राह्मण &      \\
        \hline
            929 & TB\_2.7.9.5 & अवेत्योऽवभृथा3 ना3 इति यद्{-}दर्भपुञ्जीलैः &      \\
        \hline
            930 & TB\_2.7.10.1 & प्रजापतिरकामयत बहोर्भूयान्थ्{-}स्यामिति स एतं &      \\
        \hline
            931 & TB\_2.7.10.2 & बहुर्भवति य एतेन यजते &      \\
        \hline
            932 & TB\_2.7.11.1 & अगस्यो मरुद्भ्य उक्ष्णः प्रौक्षत् &      \\
        \hline
            933 & TB\_2.7.11.2 & एवं चतुर्थे पञ्चोत्तमेऽहन्नालभ्यन्ते वर्.षिष्ठमिव &      \\
        \hline
            934 & TB\_2.7.11.3 & य उ चैनमेवं ॅवेद &  TB\_3.8.18.7 TB\_3.9.22.3       \\
        \hline
            935 & TB\_2.7.12.1 & अस्याजरासो दमा मरित्राः अर्चद्धूमासो &      \\
        \hline
            936 & TB\_2.7.12.2 & द्वे विरूपे चरतः स्वर्थे &      \\
        \hline
            937 & TB\_2.7.12.3 & औक्षन्{-}घृतैरास्तृणन्{-}बर्.हिरस्मै आदिद्धोतारं न्यषादयन्त अग्निनाऽग्निः &      \\
        \hline
            938 & TB\_2.7.12.4 & ते स्वानिनो रुद्रिया वर.षनिर्णिजः &      \\
        \hline
            939 & TB\_2.7.12.5 & श्रुधि श्रुत्कर्ण वह्निभिः देवैरग्ने &      \\
        \hline
            940 & TB\_2.7.12.6 & दिवि श्रवो दधिरे यज्ञियासः &      \\
        \hline
            941 & TB\_2.7.13.1 & तिष्ठा हरी रथ आ &      \\
        \hline
            942 & TB\_2.7.13.2 & द्वितायो वृत्रहन्तमः विद इन्द्रः &      \\
        \hline
            943 & TB\_2.7.13.3 & भरेष्विन्द्रं सुहवं हवामहे अंहोमुचं &      \\
        \hline
            944 & TB\_2.7.13.4 & ऋष्वा त इन्द्र स्थविरस्य &      \\
        \hline
            945 & TB\_2.7.13.5 & गभस्तयो नियुतो विश्ववाराः अहरहर्भूय &      \\
        \hline
            946 & TB\_2.7.14.1 & प्रजापतिः पशूनसृजत तेऽस्माथ् सृष्टाः &      \\
        \hline
            947 & TB\_2.7.14.2 & स इन्द्रमब्रवीत् इमान्म ईफ्सेति &      \\
        \hline
            948 & TB\_2.7.14.3 & इदं ॅविष्णुर् विचक्रम इति &      \\
        \hline
            949 & TB\_2.7.15.1 & व्याघ्रोऽयमग्नौ चरति प्रविष्टः ऋषीणां &      \\
        \hline
            950 & TB\_2.7.15.2 & तस्य मृत्यौ चरति राजसूयम् &      \\
        \hline
            951 & TB\_2.7.15.3 & आऽयं भातु शवसा पञ्च &      \\
        \hline
            952 & TB\_2.7.15.4 & विश्रयस्व दिशो महीः विशस्त्वा &      \\
        \hline
            953 & TB\_2.7.15.5 & तथा त्वा सविता करत् &      \\
        \hline
            954 & TB\_2.7.15.6 & अरुणं त्वा वृकमुग्रं खजं &      \\
        \hline
            955 & TB\_2.7.16.1 & अभि प्रेहि वीरयस्व उग्रश्चेत्ता &      \\
        \hline
            956 & TB\_2.7.16.2 & मा न इन्द्राभितस्त्व{-}दृष्वारिष्टासः एवा &      \\
        \hline
            957 & TB\_2.7.16.3 & इन्द्रं ॅविश्वा अवीवृधन्न् समुद्र &      \\
        \hline
            958 & TB\_2.7.16.4 & तद्ग्रावाणः सोमसुतो मयोभुवः तदश्विना &      \\
        \hline
            959 & TB\_2.7.17.1 & ये केशिनः प्रथमाः सत्रमासत &      \\
        \hline
            960 & TB\_2.7.17.2 & देहि दक्षिणां प्रतिरस्वायुः अथा &      \\
        \hline
            961 & TB\_2.7.17.3 & तेभ्यो निधानं बहुधा व्यैच्छन्न् &      \\
        \hline
            962 & TB\_2.7.18.1 & इन्द्रं ॅवै स्वा विशो &      \\
        \hline
            963 & TB\_2.7.18.2 & यꣳ राजानं ॅविशो नापचायेयुः &      \\
        \hline
            964 & TB\_2.7.18.3 & हरन्त्यस्मै विशो बलिम् ऐनमप्रतिख्यातं &      \\
        \hline
            965 & TB\_2.7.18.4 & तद् यथा ह वै &      \\
        \hline
            966 & TB\_2.7.18.5 & सतोबृहतीषु स्तुवते सतो बृहन्न् &      \\
        \hline
            967 & TB\_3.1.1.1 & अग्निर्नः पातु कृत्तिकाः नक्षत्रं &      \\
        \hline
            968 & TB\_3.1.1.2 & सा नो यज्ञ्स्य सुविते &      \\
        \hline
            969 & TB\_3.1.1.3 & यत् ते नक्षत्रं मृगशीर.षमस्ति &      \\
        \hline
            970 & TB\_3.1.1.4 & प्रमुञ्चमानौ दुरितानि विश्वा अपाघशं &      \\
        \hline
            971 & TB\_3.1.1.5 & बृहस्पतिः प्रथमं जायमानः तिष्यं &      \\
        \hline
            972 & TB\_3.1.1.6 & ये अन्तरिक्षं पृथिवीं क्षियन्ति &      \\
        \hline
            973 & TB\_3.1.1.7 & ये अग्निदग्धा येऽनग्नि{-}दग्धाः येऽमुं &      \\
        \hline
            974 & TB\_3.1.1.8 & अर्यमा राजाऽजरस्तुविष्मान् फल्गुनीना{-}मृषभो रोरवीति &      \\
        \hline
            975 & TB\_3.1.1.9 & आयातु देवः सवितोपयातु हिरण्ययेन &      \\
        \hline
            976 & TB\_3.1.1.10 & निवेशय{-}न्नमृतान् मर्त्याꣳश्च रूपाणि पिꣳशन् &      \\
        \hline
            977 & TB\_3.1.1.11 & तन्नो वायस्तदु निष्ट्या शृणोतु &      \\
        \hline
            978 & TB\_3.1.1.12 & विषूचः शत्रू{-}नपबाधमानौ अप क्षुधं &      \\
        \hline
            979 & TB\_3.1.2.1 & ऋद्ध्यास्म हव्यैर्{-}नमसोप सद्य मित्रं &      \\
        \hline
            980 & TB\_3.1.2.2 & तस्मिन् वयममृतं दुहानाः क्षुधं &      \\
        \hline
            981 & TB\_3.1.2.3 & अहर्नो अद्य सुविते दधातु &      \\
        \hline
            982 & TB\_3.1.2.4 & यासामषाढा मधु भक्षयन्ति ता &      \\
        \hline
            983 & TB\_3.1.2.5 & यस्मिन् ब्रह्माऽभ्यजयथ् सर्वमेतत् अमुञ्च &      \\
        \hline
            984 & TB\_3.1.2.6 & महीं देवीं ॅविष्णुपत्नी{-}मजूर्याम् प्रतीचीमेनां &      \\
        \hline
            985 & TB\_3.1.2.7 & यज्ञ्ं नः पान्तु वसवः &      \\
        \hline
            986 & TB\_3.1.2.9 & अहिर् बुद्ध्नियः प्रथमान एति &      \\
        \hline
            987 & TB\_3.1.2.10 & इमानि हव्या प्रयता जुषाणा &      \\
        \hline
            988 & TB\_3.1.2.11 & यौ देवानां भिषजौ हव्यवाहौ &      \\
        \hline
            989 & TB\_3.1.3.1 & नवो नवो भवति जायमानो &      \\
        \hline
            990 & TB\_3.1.3.2 & व्युच्छन्ती दुहिता दिवः अपो &      \\
        \hline
            991 & TB\_3.1.3.3 & प्र नक्षत्राय देवाय इन्द्रायेन्दुं &      \\
        \hline
            992 & TB\_3.1.4.1 & अग्निर्वा अकामयत अन्नादो देवानां &      \\
        \hline
            993 & TB\_3.1.4.2 & प्रजापतिः प्रजा असृजत ता & TB\_1.1.5.4 TB\_2.7.9.1        \\
        \hline
            994 & TB\_3.1.4.3 & सोमो वा अकामयत ओषधीनां &      \\
        \hline
            995 & TB\_3.1.4.4 & रुद्रो वा अकामयत पशुमान्थ् &      \\
        \hline
            996 & TB\_3.1.4.5 & ऋक्षा वा इयमलोमकाऽऽसीत् साऽकामयत &      \\
        \hline
            997 & TB\_3.1.4.6 & बृहस्पतिर्वा अकामयत ब्रह्मवर्चसी स्यामिति &      \\
        \hline
            998 & TB\_3.1.4.7 & देवासुराः सम्ॅयत्ता आसन्न् ते &      \\
        \hline
            999 & TB\_3.1.4.8 & पितरो वा अकामयन्त पितृलोक &      \\
        \hline
            1000 & TB\_3.1.4.9 & अर्यमा वा अकामयत पशुमान्थ् &      \\
        \hline
            1001 & TB\_3.1.4.10 & भगो वा अकामयत भगी &      \\
        \hline
            1002 & TB\_3.1.4.11 & सविता वा अकामयत श्रन्मे &      \\
        \hline
            1003 & TB\_3.1.4.12 & त्वष्टा वा अकामयत चित्रं &      \\
        \hline
            1004 & TB\_3.1.4.13 & वायुर्वा अकामयत कामचारमेषु लोकेष्वभिजयेयमिति &      \\
        \hline
            1005 & TB\_3.1.4.14 & इन्द्राग्नी वा अकामयेताम् श्रैष्ठ्यं &      \\
        \hline
            1006 & TB\_3.1.4.15 & अथैतत्{-}पौर्णमास्या आज्यं निर्वपति कामो &      \\
        \hline
            1007 & TB\_3.1.5.1 & मित्रो वा अकामयत मित्रधेयमेषु &      \\
        \hline
            1008 & TB\_3.1.5.2 & इन्द्रो वा अकामयत ज्यैष्ठ्यं &      \\
        \hline
            1009 & TB\_3.1.5.3 & प्रजापतिर्वा अकामयत मूलं प्रजां &      \\
        \hline
            1010 & TB\_3.1.5.4 & आपो वा अकामयन्त समुद्रं &      \\
        \hline
            1011 & TB\_3.1.5.5 & विश्वे वै देवा अकामयन्त &      \\
        \hline
            1012 & TB\_3.1.5.6 & ब्रह्म वा अकामयत ब्रह्मलोकमभिजयेयमिति &      \\
        \hline
            1013 & TB\_3.1.5.7 & विष्णुर्वा अकामयत पुण्यं श्लोकं &      \\
        \hline
            1014 & TB\_3.1.5.8 & वसवो वा अकामयन्त अग्रं &      \\
        \hline
            1015 & TB\_3.1.5.9 & इन्द्रो वा अकामयत दृढो &      \\
        \hline
            1016 & TB\_3.1.5.10 & अजो वा एकपादकामयत तेजस्वी &      \\
        \hline
            1017 & TB\_3.1.5.11 & अहिर्वै बुद्ध्नियोऽकामयत इमां प्रतिष्ठां &      \\
        \hline
            1018 & TB\_3.1.5.12 & पूषा वा अकामयत पशुमान्थ् &      \\
        \hline
            1019 & TB\_3.1.5.13 & अश्विनौ वा अकामयेताम् श्रोत्रस्विनावबधिरौ &      \\
        \hline
            1020 & TB\_3.1.5.14 & यमो वा अकामयत पितृणां &      \\
        \hline
            1021 & TB\_3.1.5.15 & अथैतद{-}मावास्याया आज्यं निर्वपति कामो &      \\
        \hline
            1022 & TB\_3.1.6.1 & चन्द्रमा वा अकामयत अहोरात्रा{-}नर्द्धमासान्{-}मासा{-}नृतून्थ्{-} &      \\
        \hline
            1023 & TB\_3.1.6.2 & अहोरात्रे वा अकामयेताम् अत्यहोरात्रे &      \\
        \hline
            1024 & TB\_3.1.6.3 & उषा वा अकामयत प्रिया &      \\
        \hline
            1025 & TB\_3.1.6.4 & अथैतस्मै नक्षत्राय चरुं निर्वपति &      \\
        \hline
            1026 & TB\_3.1.6.5 & सूर्यो वा अकामयत नक्षत्राणां &      \\
        \hline
            1027 & TB\_3.1.6.6 & अथैतमदित्यै चरुं निर्वपति इयं &      \\
        \hline
            1028 & TB\_3.1.6.7 & अथैतं ॅविष्णवे चरुं निर्वपति &      \\
        \hline
            1029 & TB\_3.2.1.1 & तृतीयस्यामितो दिवि सोम आसीत् &      \\
        \hline
            1030 & TB\_3.2.1.2 & तस्मात् त्रीणित्रीणि पर्णस्य पलाशानि &      \\
        \hline
            1031 & TB\_3.2.1.3 & यत् प्राचीमाहरेत् देवलोक{-}मभिजयेत् यदुदीचीं &      \\
        \hline
            1032 & TB\_3.2.1.4 & वायव एवैनान् परि ददाति &      \\
        \hline
            1033 & TB\_3.2.1.5 & वथ्सेभ्यश्च वा एताः पुरा &      \\
        \hline
            1034 & TB\_3.2.1.6 & यजमानस्य पशून् पाहीत्याह पशुनां &      \\
        \hline
            1035 & TB\_3.2.2.1 & देवस्य त्वा सवितुः प्रसव &  TB\_3.2.8.1 TB\_3.2.9.1       \\
        \hline
            1036 & TB\_3.2.2.2 & ओषधीनामहिꣳसायै यज्ञ्स्य घोषदसीत्याह यजमान &      \\
        \hline
            1037 & TB\_3.2.2.3 & त आवहन्ति कवयः पुरस्तादित्याह &      \\
        \hline
            1038 & TB\_3.2.2.4 & यद्वा इदं किं च & TB\_2.3.5.5        \\
        \hline
            1039 & TB\_3.2.2.5 & यज्ञ्स्यानतिरेकाय वर्.षवृद्ध{-}मसीत्याह वर्.ष वृद्धा &      \\
        \hline
            1040 & TB\_3.2.2.6 & देवबर्.हिः शतवल्.शं ॅविरोहेत्याह प्रजा &      \\
        \hline
            1041 & TB\_3.2.2.7 & अदित्यै रास्नाऽसीत्याह इयं ॅवा &      \\
        \hline
            1042 & TB\_3.2.2.8 & पूषा ते ग्रन्थिं ग्रथ्नात्वित्याह &      \\
        \hline
            1043 & TB\_3.2.2.9 & ब्रह्मणैवैनद्धरति उर्वन्तरिक्ष{-}मन्विहीत्याह गत्यै देवगंममसीत्याह &      \\
        \hline
            1044 & TB\_3.2.3.1 & पूर्वेद्युरिद्ध्मा बर्.हिः करोति यज्ञ्मेवारभ्य &      \\
        \hline
            1045 & TB\_3.2.3.2 & अन्तरिक्षं ॅवै मातरिश्वनो घर्मः &      \\
        \hline
            1046 & TB\_3.2.3.3 & वसूनां पवित्रमसीत्याह प्राणा वै &      \\
        \hline
            1047 & TB\_3.2.3.4 & सौम्यः पर्णः सयोनित्वाय साक्षात् &      \\
        \hline
            1048 & TB\_3.2.3.5 & तस्मादयꣳ सर्वतः पवते हुतः &      \\
        \hline
            1049 & TB\_3.2.3.6 & पवित्रवत्यानयति अपां चैवौषधीनां च &      \\
        \hline
            1050 & TB\_3.2.3.7 & अमूमिति नाम गृह्णाति भद्रमेवासां &      \\
        \hline
            1051 & TB\_3.2.3.8 & बहु दुग्धीन्द्राय देवेभ्यो हविरिति &      \\
        \hline
            1052 & TB\_3.2.3.9 & यातयाम्ना हविषा यजेत अथो &      \\
        \hline
            1053 & TB\_3.2.3.10 & अग्निहोत्रमेवन दुह्याच्छूद्रः तद्धि नोत्{-}पुनन्ति &      \\
        \hline
            1054 & TB\_3.2.3.11 & सोममेवैनत् करोति यो वै &      \\
        \hline
            1055 & TB\_3.2.3.12 & अयस्पात्रेण वा दारुपात्रेण वाऽपिदधाति &      \\
        \hline
            1056 & TB\_3.2.4.1 & कर्मणे वां देवेभ्यः शकेयमित्याह &      \\
        \hline
            1057 & TB\_3.2.4.2 & यज्ञ्मेवारभ्य प्रणीय प्रचरति अपः &      \\
        \hline
            1058 & TB\_3.2.4.3 & अपः प्रणयति आपो वै &      \\
        \hline
            1059 & TB\_3.2.4.4 & अद्ध्वर्युं च यजमानं च &      \\
        \hline
            1060 & TB\_3.2.4.5 & अह्रुतमसि हविर्द्धान{-}मित्याहानार्त्यै दृꣳहस्व मा &      \\
        \hline
            1061 & TB\_3.2.4.6 & अश्विनौ हि देवानामद्ध्वर्यू आस्ताम् &      \\
        \hline
            1062 & TB\_3.2.4.7 & इदं देवानामिदमु नः सहेत्याह &      \\
        \hline
            1063 & TB\_3.2.5.1 & इन्द्रो वृत्रमहन्न् सोऽपः अभ्यम्रियत &      \\
        \hline
            1064 & TB\_3.2.5.2 & द्विपाद्{-}यजमानः प्रतिष्ठित्यै देवो वः &      \\
        \hline
            1065 & TB\_3.2.5.3 & प्राणैरेव प्राणान्थ् संपृणक्ति सावित्रियर्चा &      \\
        \hline
            1066 & TB\_3.2.5.4 & युष्मानिन्द्रो ऽवृणीत वृत्रतूये यूयमिन्द्रमवृणीद्ध्वं &      \\
        \hline
            1067 & TB\_3.2.5.5 & अथो रक्षसामपहत्यै शुन्धद्ध्वं दैव्याय &      \\
        \hline
            1068 & TB\_3.2.5.6 & अस्या एवैनत् त्वचं करोति &      \\
        \hline
            1069 & TB\_3.2.5.7 & अधिषवणमसि वानस्पत्यमित्याह अधिषवण{-}मेवैनत् करोति &      \\
        \hline
            1070 & TB\_3.2.5.8 & देवताभिरेवैनथ् समर्द्धयति अद्रिरसि वानस्पत्य &      \\
        \hline
            1071 & TB\_3.2.5.9 & इषमेवोर्जं ॅयजमाने दधाति द्युमद्वदत &      \\
        \hline
            1072 & TB\_3.2.5.10 & वृङ्क्त एषामिन्द्रियं ॅवीर्यम् श्रेष्ठ &      \\
        \hline
            1073 & TB\_3.2.5.11 & रक्षसां भागोऽसीत्याह तुषैरेव रक्षाꣳसि &      \\
        \hline
            1074 & TB\_3.2.6.1 & अवधूतꣳ रक्षोऽवधूता अरातय इत्याह &      \\
        \hline
            1075 & TB\_3.2.6.2 & कृष्णो रूपं कृत्वा यत् &      \\
        \hline
            1076 & TB\_3.2.6.3 & धिषणाऽसि पार्वतेयी प्रति त्वा &      \\
        \hline
            1077 & TB\_3.2.6.4 & यावदेका देवता कामयते यावदेका &      \\
        \hline
            1078 & TB\_3.2.7.1 & धृष्टिरसि ब्रह्म यच्छेत्याह धृत्यै &      \\
        \hline
            1079 & TB\_3.2.7.2 & अन्तरिक्ष एव ज्योतिर्द्धत्ते आदित्यमेवामुष्मिन् &      \\
        \hline
            1080 & TB\_3.2.7.3 & धर्मासि दिशो दृꣳहेत्याह दिश &      \\
        \hline
            1081 & TB\_3.2.7.4 & अथ द्वे अथ त्रीणि &      \\
        \hline
            1082 & TB\_3.2.7.5 & अमुष्मिन् ॅलोकेऽनुपरैति यदष्टावुप दधाति &      \\
        \hline
            1083 & TB\_3.2.7.6 & जगत्या तत् छन्दः संमितानि &      \\
        \hline
            1084 & TB\_3.2.8.1 & देवस्य त्वा सवितुः प्रसव & TB\_3.2.2.1  TB\_3.2.9.1       \\
        \hline
            1085 & TB\_3.2.8.2 & तस्मादेवमाह सꣳ रेवतीर्{-}जगतीभिर्{-}मधुमतीर्{-}मधुमतीभिः सृज्यद्ध्वमित्याह &      \\
        \hline
            1086 & TB\_3.2.8.3 & आप ओषधीर्महयन्ति तादृगेव तत् &      \\
        \hline
            1087 & TB\_3.2.8.4 & घर्मोऽसि विश्वायुरित्याह विश्वमेवायुर् यजमाने &      \\
        \hline
            1088 & TB\_3.2.8.5 & अर्द्धमासेऽर्द्धमासे प्रवृज्यते यत् पुरोडाशः &      \\
        \hline
            1089 & TB\_3.2.8.6 & रक्षसा{-}मन्तर्.हित्यै पुरोडाशं ॅवा अधिश्रितं &      \\
        \hline
            1090 & TB\_3.2.8.7 & अविदहन्तः श्रपयतेति वाचं ॅविसृजते &      \\
        \hline
            1091 & TB\_3.2.8.8 & वेदेनाभि वासयति तस्मात् केशैः &      \\
        \hline
            1092 & TB\_3.2.8.9 & प्राणाः पशवः प्राणैरेव पशून्थ् &      \\
        \hline
            1093 & TB\_3.2.8.10 & मयि तनूः संनिधद्ध्वम् अहं &      \\
        \hline
            1094 & TB\_3.2.8.11 & ततो द्वितोऽजायत स तृतीय{-}मभ्यपातयत् &      \\
        \hline
            1095 & TB\_3.2.8.12 & सूर्याभिनिम्रुक्तः कुनखिनि कुनखी श्यावदति &      \\
        \hline
            1096 & TB\_3.2.9.1 & देवस्य त्वा सवितुः प्रसव & TB\_3.2.2.1 TB\_3.2.8.1        \\
        \hline
            1097 & TB\_3.2.9.2 & तेज एवास्मिन् दधाति विषाद्वै &      \\
        \hline
            1098 & TB\_3.2.9.3 & मेद्ध्यामेवैनां देवयजनीं करोति ओषद्ध्यास्ते &      \\
        \hline
            1099 & TB\_3.2.9.4 & द्वौ वाव पुरुषौ यं &      \\
        \hline
            1100 & TB\_3.2.9.5 & भ्रातृव्यमेव पृथिव्या अपहन्ति तेऽमन्यन्त &      \\
        \hline
            1101 & TB\_3.2.9.6 & अन्तरिक्षादेवैन{-}मपहन्ति तृतीयꣳ हरति दिव &      \\
        \hline
            1102 & TB\_3.2.9.7 & क्यन्नो दास्यथेति यावथ्स्वयं परिगृह्णीथेति &      \\
        \hline
            1103 & TB\_3.2.9.8 & भवत्यात्मना पराऽस्य भ्रातृव्यो भवति &      \\
        \hline
            1104 & TB\_3.2.9.9 & प्राञ्चौ वेद्यꣳ सावुन्नयति आहवनीयस्य &      \\
        \hline
            1105 & TB\_3.2.9.10 & मूलं छिनत्ति भ्रातृव्यस्यैव मूलं &      \\
        \hline
            1106 & TB\_3.2.9.11 & प्रजापतिना यज्ञ्मुखेन संमिताम् वेदिर्देवेभ्यो &      \\
        \hline
            1107 & TB\_3.2.9.12 & पुरीषवतीं करोति प्रजा वै &      \\
        \hline
            1108 & TB\_3.2.9.13 & क्रूरमिव वा एतत् करोति &      \\
        \hline
            1109 & TB\_3.2.9.14 & यददश्चन्द्रमसि मेद्ध्यम् तदस्यामेरयति तां &      \\
        \hline
            1110 & TB\_3.2.9.15 & रक्षसामपहत्यै स्फ्यस्य वर्त्मन्थ्{-}सादयति यज्ञ्स्य &      \\
        \hline
            1111 & TB\_3.2.10.1 & वज्रो वै स्फ्यः यदन्वञ्चं &      \\
        \hline
            1112 & TB\_3.2.10.2 & यथोपधाय वृश्चन्त्यवम् हस्तावव नेनिक्ते &      \\
        \hline
            1113 & TB\_3.2.10.3 & यत् पुरस्तात् प्रत्य{-}गुपसादयेत् अन्यत्राहुतिपथादिद्ध्मं &      \\
        \hline
            1114 & TB\_3.3.1.1 & प्रत्युष्टꣳ रक्षः प्रत्युष्टा अरातय &      \\
        \hline
            1115 & TB\_3.3.1.2 & अन्तरिक्षमुपभृत् पृथिवी ध्रुवा इमे &      \\
        \hline
            1116 & TB\_3.3.1.3 & वृष्टिमेव नियच्छति अवाचीनाग्रा हि &      \\
        \hline
            1117 & TB\_3.3.1.4 & प्राचीमभ्याकारम् अग्रैरन्तरतः एवमिव ह्यन्नमद्यते &      \\
        \hline
            1118 & TB\_3.3.1.5 & स्रुग्घ्येषा प्राणो वै स्रुवः &      \\
        \hline
            1119 & TB\_3.3.2.1 & दिवः शिल्पमवततम् पृथिव्याः ककुभि &      \\
        \hline
            1120 & TB\_3.3.2.2 & स्वेनैवैनानि छन्दसा स्वया देवतया &      \\
        \hline
            1121 & TB\_3.3.2.3 & यद्येनानि पशवोऽभितिष्ठेयुः न तत् &      \\
        \hline
            1122 & TB\_3.3.2.4 & प्रतितिष्ठति प्रजया पशुभिर्यजमानः अथो &      \\
        \hline
            1123 & TB\_3.3.2.5 & नवदावो ह्येषां प्रियः यावत्प्रियो &      \\
        \hline
            1124 & TB\_3.3.3.1 & अयज्ञो वा एषः योऽपत्नीकः &      \\
        \hline
            1125 & TB\_3.3.3.2 & यत् पश्चात्{-}प्राच्यन्वासीत अनया समदं &      \\
        \hline
            1126 & TB\_3.3.3.3 & तेनैवैनां ॅव्रतमुपनयति तस्मादाहुः यश्चैवं &      \\
        \hline
            1127 & TB\_3.3.3.4 & योगक्षेमस्य क्लृप्त्यै युक्तं क्रियाता &      \\
        \hline
            1128 & TB\_3.3.3.5 & अथो अर्द्धो वा एष &      \\
        \hline
            1129 & TB\_3.3.4.1 & घृतं च वै मधु &      \\
        \hline
            1130 & TB\_3.3.4.2 & मिथुनत्वाय प्रजात्यै यद्वै पत्नी &      \\
        \hline
            1131 & TB\_3.3.4.3 & तेजो वा अग्निः तेज &      \\
        \hline
            1132 & TB\_3.3.4.4 & तद्वा अतः पवित्राभ्यामेवोत् पुनाति &      \\
        \hline
            1133 & TB\_3.3.4.5 & एषां ॅलोकानामाप्त्यै त्रिः त्र्यावृद्धि &      \\
        \hline
            1134 & TB\_3.3.4.6 & एषा हि विश्वेषां देवानां &      \\
        \hline
            1135 & TB\_3.3.5.1 & देवासुराः सम्ॅयत्ता आसन्न् स & TB\_1.5.9.1        \\
        \hline
            1136 & TB\_3.3.5.2 & अथ केनाज्यमिति सत्येनेति ब्रूयात् &      \\
        \hline
            1137 & TB\_3.3.5.3 & छन्दाꣳसि वा आज्यम् छन्दाꣳस्येव &      \\
        \hline
            1138 & TB\_3.3.5.4 & चतुष्पादः पशवः पशुष्वे{-}वोपरिष्टात्{-}प्रतितिष्ठति यजमानदेवत्या &      \\
        \hline
            1139 & TB\_3.3.5.5 & अष्टावुपभृति तस्मादष्टाशफा चतुर्द्ध्रुवायाम् तस्माच्चतुः &      \\
        \hline
            1140 & TB\_3.3.6.1 & आपो देवीरग्रेपुवो अग्रेगुव इत्याह &      \\
        \hline
            1141 & TB\_3.3.6.2 & तेनापः प्रोक्षिताः अग्निर् देवेभ्यो &      \\
        \hline
            1142 & TB\_3.3.6.3 & प्रजा एव पृथिव्यां प्रतिष्ठापयति &      \\
        \hline
            1143 & TB\_3.3.6.4 & तादृगेव तत् स्वधा पितृभ्य &      \\
        \hline
            1144 & TB\_3.3.6.5 & ऊर्जा पृथिवीं गच्छतेत्याह पृथिव्यामेवोर्जं &      \\
        \hline
            1145 & TB\_3.3.6.6 & यज्ञ्स्य धृत्यै पुरस्तात् प्रस्तरं &      \\
        \hline
            1146 & TB\_3.3.6.7 & अपरिमितं गृह्णाति अपरिमितस्यावरुद्ध्यै तस्मिन् &      \\
        \hline
            1147 & TB\_3.3.6.8 & बर्.हिः स्तृणाति प्रजा वै &      \\
        \hline
            1148 & TB\_3.3.6.9 & विश्वमेवायुर्{-}यजमाने दधाति इन्द्रस्य बाहुरसि &      \\
        \hline
            1149 & TB\_3.3.6.10 & वीतिहोत्रं त्वा कव इत्याह &      \\
        \hline
            1150 & TB\_3.3.6.11 & असौ वै जुहूः अन्तरिक्षमुपभृत् &      \\
        \hline
            1151 & TB\_3.3.7.1 & अग्निना वै होत्रा देवा &      \\
        \hline
            1152 & TB\_3.3.7.2 & ऊर्द्ध्वे समिधावादधाति अनूयाजेभ्यः समिधमतिशिनष्टि &      \\
        \hline
            1153 & TB\_3.3.7.3 & त्रिरुपवाजयति त्रयो वै प्राणाः &      \\
        \hline
            1154 & TB\_3.3.7.4 & अथो रक्षसामपहत्यै परिधीन्थ्{-}संमार्ष्टि पुनात्येवैनान् &      \\
        \hline
            1155 & TB\_3.3.7.5 & तिष्ठन्नन्यम् यथाऽनो वा रथम्ॅवा &      \\
        \hline
            1156 & TB\_3.3.7.6 & अग्निर्वै देवानां ॅयष्टा य &      \\
        \hline
            1157 & TB\_3.3.7.7 & ताभ्यामेव प्रति प्रोच्यात्याक्रामति विजिहाथां &      \\
        \hline
            1158 & TB\_3.3.7.8 & इन्द्रियमेव यजमाने दधाति समारभ्योर्द्ध्वो &      \\
        \hline
            1159 & TB\_3.3.7.9 & सुवर्गो वै लोको बृहद्भाः &      \\
        \hline
            1160 & TB\_3.3.7.10 & अग्निर्वाव पवित्रम् वृजिनमनृतं दुश्चरितम् &      \\
        \hline
            1161 & TB\_3.3.7.11 & आघार{-}माघार्य ध्रुवाꣳ समनक्ति आत्मन्नेव &      \\
        \hline
            1162 & TB\_3.3.8.1 & धिष्णिया वा एते न्युप्यन्ते &      \\
        \hline
            1163 & TB\_3.3.8.2 & यजमानायैव तल्लोकꣳ शिꣳषति नास्य &      \\
        \hline
            1164 & TB\_3.3.8.3 & तां प्रजातिं ॅयजमानोऽनु प्रजायते &      \\
        \hline
            1165 & TB\_3.3.8.4 & मुखमिव प्रत्युपह्वयेत समुंखानेव पशूनुपह्वयते &      \\
        \hline
            1166 & TB\_3.3.8.5 & यां ॅवै हस्त्यामिडामादधाति वाचः &      \\
        \hline
            1167 & TB\_3.3.8.6 & न मयाऽभागया ऽनुवक्ष्यथेति वागब्रवीत् &      \\
        \hline
            1168 & TB\_3.3.8.7 & यजमानो वै पुरोडाशः प्रजा &      \\
        \hline
            1169 & TB\_3.3.8.8 & ब्रह्मा होताऽद्ध्वर्युरग्नीत् तमभिमृशेत् इदं &      \\
        \hline
            1170 & TB\_3.3.8.9 & अग्निमुखा ह्यृद्धिः अग्निमुखामेवर्द्धिं ॅयजमान &      \\
        \hline
            1171 & TB\_3.3.8.10 & सविता यज्ञ्स्य प्रसूत्यै अथ &      \\
        \hline
            1172 & TB\_3.3.8.11 & अन्या दक्षिणा नीयन्ते यज्ञ्स्य &      \\
        \hline
            1173 & TB\_3.3.9.1 & अथ स्रुचावनुष्टुग्भ्यां ॅवाजवतीभ्यां ॅव्यूहति &      \\
        \hline
            1174 & TB\_3.3.9.2 & द्वाभ्याम् द्विप्रतिष्ठो हि वसुभ्यस्त्वा &      \\
        \hline
            1175 & TB\_3.3.9.3 & एभ्य एवैनं ॅलोकेभ्योऽनक्ति अभिपूर्वमनक्ति &      \\
        \hline
            1176 & TB\_3.3.9.4 & प्रजां ॅयोनिं मा निर्मृक्षमित्याह &      \\
        \hline
            1177 & TB\_3.3.9.5 & यावद्वा अद्ध्वर्युः प्रस्तरं प्रहरति &      \\
        \hline
            1178 & TB\_3.3.9.6 & यथायजुरेवैतत् अग्ने देव पणिभिर्{-}वीयमाण &      \\
        \hline
            1179 & TB\_3.3.9.7 & स्रुचौ सं प्रस्रावयति यदेव &      \\
        \hline
            1180 & TB\_3.3.9.8 & तानेव तेन प्रीणाति वैश्वदेव्यर्चा &      \\
        \hline
            1181 & TB\_3.3.9.9 & प्रजा वै पशवः सुम्नम् &      \\
        \hline
            1182 & TB\_3.3.9.10 & अतिरिक्ताः फलीकरणाः अतिरिक्तमाज्योच्छेषणम् अतिरिक्त &      \\
        \hline
            1183 & TB\_3.3.9.11 & अथो यद्{-}वेदश्च वेदिश्च भवतः &      \\
        \hline
            1184 & TB\_3.3.9.12 & तं कालेकाल आगते यजते &      \\
        \hline
            1185 & TB\_3.3.10.1 & यो वा अयथादेवतं ॅयज्ञ्मुपचरति &      \\
        \hline
            1186 & TB\_3.3.10.2 & न देवताभ्य आवृश्च्यते वसीयान् &  TB\_3.8.3.2       \\
        \hline
            1187 & TB\_3.3.10.3 & आशिषमेवैतामाशास्ते पूर्णपात्रे अन्ततोऽनुष्टुभा चतुष्पद्वा &      \\
        \hline
            1188 & TB\_3.3.10.4 & प्रजाः प्रजनयन्न् यद्वै यज्ञ्स्य &      \\
        \hline
            1189 & TB\_3.3.11.1 & परिवेषो वा एष वनस्पतीनाम् &      \\
        \hline
            1190 & TB\_3.3.11.2 & प्रजां पुष्टिमथो धनम् द्विपदो &      \\
        \hline
            1191 & TB\_3.3.11.3 & अथास्मै नाम गृह्य प्रहरति &      \\
        \hline
            1192 & TB\_3.3.11.4 & इन्द्रो नयतु वृत्रहा यतो &      \\
        \hline
            1193 & TB\_3.4.1.1 & ब्रह्मणे ब्राह्मणमालभते क्षत्राय राजन्यम् &      \\
        \hline
            1194 & TB\_3.4.2.1 & गीताय सूतम् नृत्ताय शैलूषम् &      \\
        \hline
            1195 & TB\_3.4.3.1 & श्रमाय कौलालम् मायायै कार्मारम् &      \\
        \hline
            1196 & TB\_3.4.4.1 & सन्धये जारम् गेहायोपपतिम् निर्.ऋत्यै &      \\
        \hline
            1197 & TB\_3.4.5.1 & नदीभ्यः पौञ्जिष्टम् ऋक्षीकाभ्यो नैषादम् &      \\
        \hline
            1198 & TB\_3.4.6.1 & उथ्सादेभ्यः कुब्जम् प्रमुदे वामनम् &      \\
        \hline
            1199 & TB\_3.4.7.1 & ऋत्यै स्तेनहृदयम् वैरहत्याय पिशुनम् &      \\
        \hline
            1200 & TB\_3.4.8.1 & भायै दार्वाहारम् प्रभाया आग्नेन्धम् &      \\
        \hline
            1201 & TB\_3.4.9.1 & अर्मेभ्यो हस्तिपम् जवायाश्वपम् पुष्ट्यै &      \\
        \hline
            1202 & TB\_3.4.10.1 & मन्यवेऽयस्तापम् क्रोधाय निसरम् शोकायाभिसरम् &      \\
        \hline
            1203 & TB\_3.4.11.1 & यम्यै यमसूम् अथर्वभ्योऽवतोकाम् सम्ॅवथ्सराय &      \\
        \hline
            1204 & TB\_3.4.12.1 & सरोभ्यो धैवरम् वेशन्ताभ्यो दाशम् &      \\
        \hline
            1205 & TB\_3.4.13.1 & प्रतिश्रुत्काया ऋतुलम् घोषाय भषम् &      \\
        \hline
            1206 & TB\_3.4.14.1 & बीभथ्सायै पौल्कसम् भूत्यै जागरणम् &      \\
        \hline
            1207 & TB\_3.4.15.1 & हसाय पुꣳश्चलूमालभते वीणावादं गणकं &      \\
        \hline
            1208 & TB\_3.4.16.1 & अक्षराजाय कितवम् कृताय सभाविनम् &      \\
        \hline
            1209 & TB\_3.4.17.1 & भूम्यै पीठसर्पिणमालभते अग्नयेऽꣳसलम् वायवे &      \\
        \hline
            1210 & TB\_3.4.18.1 & वाचे पुरुषमालभते प्राणमपानं ॅव्यानमुदानं &      \\
        \hline
            1211 & TB\_3.4.19.1 & अथैतानरूपेभ्य आलभते अतिह्रस्व{-}मतिदीर्घम् अतिकृशमत्यꣳसलम् &      \\
        \hline
            1212 & TB\_3.5.1.1 & सत्यं प्रपद्ये ऋतं प्रपद्ये &      \\
        \hline
            1213 & TB\_3.5.2.1 & प्र वो वाजा अभिद्यवः &      \\
        \hline
            1214 & TB\_3.5.2.2 & अच्छा देव विवाससि बृहदग्ने &      \\
        \hline
            1215 & TB\_3.5.2.3 & अग्ने दीद्यतं बृहत् अग्निं &      \\
        \hline
            1216 & TB\_3.5.3.1 & अग्ने महाꣳ असि ब्राह्मण &      \\
        \hline
            1217 & TB\_3.5.3.2 & चमसो देवपानः अराꣳ इवाग्ने &      \\
        \hline
            1218 & TB\_3.5.4.1 & अग्निर्. होता वेत्वग्निः होत्रं &      \\
        \hline
            1219 & TB\_3.5.5.1 & समिधो अग्न आज्यस्य वियन्तु &      \\
        \hline
            1220 & TB\_3.5.6.1 & अग्निर्{-}वृत्राणि जङ्घनत् द्रविणस्युर्{-}विपन्यया समिद्धः &      \\
        \hline
            1221 & TB\_3.5.7.1 & अग्निर्{-}मूर्द्धा दिवः ककुत् पतिः &      \\
        \hline
            1222 & TB\_3.5.7.2 & वयꣳ स्याम पतयो रयीणाम् &      \\
        \hline
            1223 & TB\_3.5.7.3 & युवꣳ सिन्धूꣳ रभिशस्ते{-}रवद्यात् अग्नीषोमा{-}वमुञ्चतं &      \\
        \hline
            1224 & TB\_3.5.7.4 & सजित्वानꣳ सदासहम् वर्.षिष्ठमूतये भर &      \\
        \hline
            1225 & TB\_3.5.7.5 & उत द्विबर्.हा अमिनः सहोभिः &      \\
        \hline
            1226 & TB\_3.5.7.6 & अयाडग्नेः प्रिया धामानि अयाट्प्रजापतेः &      \\
        \hline
            1227 & TB\_3.5.8.1 & उपहूतꣳ रथन्तरꣳ सह पृथिव्या &  TB\_3.5.13.1       \\
        \hline
            1228 & TB\_3.5.8.2 & उपहूतो भक्षः सखा उप &  TB\_3.5.13.2       \\
        \hline
            1229 & TB\_3.5.8.3 & दैव्या अद्ध्वर्यव उपहूताः उपहूता &  TB\_3.5.13.3       \\
        \hline
            1230 & TB\_3.5.9.1 & देवं बर्.हिः वसुवने वसुधेयस्य &  TB\_3.6.14.1       \\
        \hline
            1231 & TB\_3.5.10.1 & इदं द्यावापृथिवी भद्रमभूत् आर्द्ध्म &      \\
        \hline
            1232 & TB\_3.5.10.2 & वृष्टिद्यावा रीत्यापा शम्भुवौ मयोभुवौ &      \\
        \hline
            1233 & TB\_3.5.10.3 & अवीवृधत महो ज्यायोऽकृत प्रजापतिरिदं &      \\
        \hline
            1234 & TB\_3.5.10.4 & अवीवृधत महो ज्यायोऽकृत देवा &      \\
        \hline
            1235 & TB\_3.5.10.5 & उत्तरां देवयज्या{-}माशास्ते भूयो हविष्करण{-}माशास्ते &      \\
        \hline
            1236 & TB\_3.5.11.1 & तच्छं ॅयोरा वृणीमहे गातुं &      \\
        \hline
            1237 & TB\_3.5.12.1 & आ प्यायस्व सं ते &      \\
        \hline
            1238 & TB\_3.5.12.2 & अग्निर्. होता गृहपतिः स &      \\
        \hline
            1239 & TB\_3.5.13.1 & उपहूतꣳ रथन्तरꣳ सह पृथिव्या & TB\_3.5.8.1        \\
        \hline
            1240 & TB\_3.5.13.2 & उपहूतो भक्षः सखा उप & TB\_3.5.8.2        \\
        \hline
            1241 & TB\_3.5.13.3 & दैव्या अद्ध्वर्यव उपहूताः उपहूता & TB\_3.5.8.3        \\
        \hline
            1242 & TB\_3.6.1.1 & अञ्जन्ति त्वामद्ध्वरे देवयन्तः वनस्पते &      \\
        \hline
            1243 & TB\_3.6.1.2 & आरे अस्मदमतिं बाधमानः उच्छ्रयस्व &      \\
        \hline
            1244 & TB\_3.6.1.3 & जातो जायते सुदिनत्वे अह्नाम् &      \\
        \hline
            1245 & TB\_3.6.1.4 & अग्निर्{-}यज्ञ्स्य हव्यवाट् तꣳ सबाधो &      \\
        \hline
            1246 & TB\_3.6.2.1 & होता यक्षदग्निꣳ समिधा सुषमिधा &      \\
        \hline
            1247 & TB\_3.6.2.2 & होता यक्षद्{-}दुर ऋष्वाः कवष्योऽकोषधावनी{-}रुदाताभिर्{-}जिहतां &      \\
        \hline
            1248 & TB\_3.6.3.1 & समिद्धो अद्य मनुषो दुरोणे &      \\
        \hline
            1249 & TB\_3.6.3.2 & ते सुक्रतवः शुचयो धियधांः &      \\
        \hline
            1250 & TB\_3.6.3.3 & व्यचस्वतीरुर्विया विश्रयन्ताम् पतिभ्यो नजनयः &      \\
        \hline
            1251 & TB\_3.6.3.4 & प्रचोदयन्ता विदथेषु कारू प्राचीनं &      \\
        \hline
            1252 & TB\_3.6.3.5 & उपावसृजत्{-}त्मन्या समञ्जन्न् देवानां पाथ &      \\
        \hline
            1253 & TB\_3.6.4.1 & अग्निर्. होता नो अद्ध्वरे &      \\
        \hline
            1254 & TB\_3.6.5.1 & अजैदग्निः असनद्वाजं नि देवो &      \\
        \hline
            1255 & TB\_3.6.6.1 & दैव्याः शमितार उत मनुष्या &      \\
        \hline
            1256 & TB\_3.6.6.2 & सूर्यं चक्षुर्गमयतात् वातं प्राणमन्व{-}वसृजतात् &      \\
        \hline
            1257 & TB\_3.6.6.3 & शला दोषणी कश्यपेवाꣳसा अच्छिद्रे &      \\
        \hline
            1258 & TB\_3.6.6.4 & उरूकं मन्यमानाः नेद्वस्तोके तनये &      \\
        \hline
            1259 & TB\_3.6.7.1 & जुषस्व सप्रथस्तमम् वचो देवफ्सरस्तमम् &      \\
        \hline
            1260 & TB\_3.6.7.2 & श्रेष्ठं नो धेहि वार्यम् &      \\
        \hline
            1261 & TB\_3.6.8.1 & आ वृत्रहणा वृत्रहभिः शुष्मैः &      \\
        \hline
            1262 & TB\_3.6.8.2 & नान्या युवत्प्रमतिरस्ति मह्यम् स &      \\
        \hline
            1263 & TB\_3.6.9.1 & गीर्भिर्विप्रः प्रमतिमिच्छमानः ई रयिं &      \\
        \hline
            1264 & TB\_3.6.10.1 & त्वꣳ ह्यग्ने प्रथमो मनोता &      \\
        \hline
            1265 & TB\_3.6.10.2 & रुशन्तमग्निं दर्.शतं बृहन्तम् वपावन्तं &      \\
        \hline
            1266 & TB\_3.6.10.3 & स पर्येण्यः स प्रियो &      \\
        \hline
            1267 & TB\_3.6.10.4 & प्रेतीषणिमिषयन्तं पावकम् राजन्तमग्निं ॅयजतं &      \\
        \hline
            1268 & TB\_3.6.10.5 & आ यस्ततन्थ रोदसी वि &      \\
        \hline
            1269 & TB\_3.6.11.1 & आ भरतꣳ शिक्षतं ॅवज्रबाहू &      \\
        \hline
            1270 & TB\_3.6.11.2 & घासे अज्राणां ॅयवसप्रथमानाम् सुमत्क्षराणां &      \\
        \hline
            1271 & TB\_3.6.11.3 & प्रदक्षिणिद्{-}रशनया नियूय ऋतस्य वक्षि &      \\
        \hline
            1272 & TB\_3.6.11.4 & करदेवं देवो वनस्पतिः जुषतां &      \\
        \hline
            1273 & TB\_3.6.12.1 & उपोह यद्विदथं ॅवाजिनो गूः &      \\
        \hline
            1274 & TB\_3.6.12.2 & अयाड्वनस्पतेः प्रिया पाथाꣳसि अयाड्देवाना{-}माज्यपानां &      \\
        \hline
            1275 & TB\_3.6.13.1 & देवं बर्.हिः सुदेवं देवैः &      \\
        \hline
            1276 & TB\_3.6.14.1 & देवं बर्.हिः वसुवने वसुधेयस्य & TB\_3.5.9.1        \\
        \hline
            1277 & TB\_3.6.14.2 & देवा दैव्या होतारा वसुवने &      \\
        \hline
            1278 & TB\_3.6.14.3 & देवो अग्निः स्विष्टकृत् सुद्रविणा &      \\
        \hline
            1279 & TB\_3.6.15.1 & अग्निमद्य होतारमवृणीतायं ॅयजमानः पचन् &      \\
        \hline
            1280 & TB\_3.7.1.1 & सर्वान्. वा एषोऽग्नौ कामान् &      \\
        \hline
            1281 & TB\_3.7.1.2 & कामप्रीता एनं कामा अनुप्रयान्ति &      \\
        \hline
            1282 & TB\_3.7.1.3 & मनसैव यज्ञ्ꣳ सन्तनोति भूरित्याह &      \\
        \hline
            1283 & TB\_3.7.1.4 & तृतीयेन ज्योतिषा सम्ॅविशस्व सम्ॅवेशनस्तनुवै &      \\
        \hline
            1284 & TB\_3.7.1.5 & तान्. यद्दुह्यात् यातयाम्ना हविषा &      \\
        \hline
            1285 & TB\_3.7.1.6 & अथोत्तरस्मै हविषे वथ्सानपा कुर्यात् &      \\
        \hline
            1286 & TB\_3.7.1.7 & अथेतर ऐन्द्रः पुरोडाशः स्यात् &      \\
        \hline
            1287 & TB\_3.7.1.8 & ऐन्द्रं पञ्चशराव{-}मोदनं निर्वपेत् अग्निं &      \\
        \hline
            1288 & TB\_3.7.1.9 & सैव ततः प्रायश्चित्तिः अर्द्धो &      \\
        \hline
            1289 & TB\_3.7.2.1 & यद्विः षण्णेन जुहुयात् अप्रजा &      \\
        \hline
            1290 & TB\_3.7.2.2 & भूतिमेवोपैति तत् कृत्वा अन्यां &      \\
        \hline
            1291 & TB\_3.7.2.3 & तत्कृत्वा अन्यां दुग्ध्वा पुनर्. &      \\
        \hline
            1292 & TB\_3.7.2.4 & मित्रो दाधार पृथिवीमुत द्यां &      \\
        \hline
            1293 & TB\_3.7.2.5 & चतुष्पाद्भिः पशुभि{-}र्यजमानो व्यृद्ध्येत यत्र &      \\
        \hline
            1294 & TB\_3.7.2.6 & यद् दक्षिणा ब्रह्मणे च &      \\
        \hline
            1295 & TB\_3.7.2.7 & स्रुवस्य बुद्ध्नेनाभि निदद्ध्यात् मा &      \\
        \hline
            1296 & TB\_3.7.3.1 & वि वा एष इन्द्रियेण &      \\
        \hline
            1297 & TB\_3.7.3.2 & अजस्य तु नाश्ञीयात् यदजस्या{-}श्ञीयात् &      \\
        \hline
            1298 & TB\_3.7.3.3 & ब्राह्मणं तु वसत्यै नापरुन्ध्यात् &      \\
        \hline
            1299 & TB\_3.7.3.4 & यद्दर्भानद्ध्यासीत यामेवाग्ना{-}वाहुतिं जुहुयात् तामद्ध्यासीत &      \\
        \hline
            1300 & TB\_3.7.3.5 & या मेवाफ्स्वाहुतिं जुहुयात् तां &      \\
        \hline
            1301 & TB\_3.7.3.6 & गर्भꣳ स्रवन्तमगदमकः अग्नि{-}रिन्द्र{-}स्त्वष्टा बृहस्पतिः &      \\
        \hline
            1302 & TB\_3.7.3.7 & अग्निरित्याह अग्निर्वै रेतोधाः रेत &      \\
        \hline
            1303 & TB\_3.7.4.1 & याः पुरस्तात् प्रस्रवन्ति उपरिष्टाथ् &      \\
        \hline
            1304 & TB\_3.7.4.2 & सोमपीथाय संनयितुं वकल{-}मन्तरमाददे आपो &      \\
        \hline
            1305 & TB\_3.7.4.3 & अग्निं गृह्णामि सुरथं ॅयो &      \\
        \hline
            1306 & TB\_3.7.4.4 & अग्निर्. हव्यवाडिह तानावहतु पौर्णमासं &      \\
        \hline
            1307 & TB\_3.7.4.5 & स्व आयतने मनीषया इह &      \\
        \hline
            1308 & TB\_3.7.4.6 & विजयभागꣳ समिन्धतां अग्ने दीदाय &      \\
        \hline
            1309 & TB\_3.7.4.7 & रुद्रेभ्यो यज्ञ्ं प्रब्रवीमि इदमहं &      \\
        \hline
            1310 & TB\_3.7.4.8 & व्रतानां ॅव्रतपते व्रतं चरिष्यामि &      \\
        \hline
            1311 & TB\_3.7.4.9 & त्रीन् परिधीꣳस्तिस्रः समिधः यज्ञायुरनुसंचरान् &      \\
        \hline
            1312 & TB\_3.7.4.10 & आच्छेत्ता वो मा रिषं &      \\
        \hline
            1313 & TB\_3.7.4.11 & त्रिवृत् पलाशे दर्भः इयान् &      \\
        \hline
            1314 & TB\_3.7.4.12 & अयं प्राणश्चापानश्च यजमान{-}मपिगच्छतां यज्ञे &      \\
        \hline
            1315 & TB\_3.7.4.13 & शिवेयꣳ रज्जुरभिधानी अघ्निया{-}मुपसेवतां अप्रस्रꣳसाय &      \\
        \hline
            1316 & TB\_3.7.4.14 & अमृन्मयं देवपात्रं यज्ञ्स्यायुषि प्रयुज्यतां &      \\
        \hline
            1317 & TB\_3.7.4.15 & बह्वी र्भवन्तीरुप जायमानाः इह &      \\
        \hline
            1318 & TB\_3.7.4.16 & उथ्सं दुहन्ति कलशं चतुर्बिलं &      \\
        \hline
            1319 & TB\_3.7.4.17 & वथ्सेभ्यो मनुष्येभ्यः पुनर्दोहाय कल्पतां &      \\
        \hline
            1320 & TB\_3.7.4.18 & पर्णवल्कः पवित्रं सौम्यः सोमाद्धि &      \\
        \hline
            1321 & TB\_3.7.5.1 & देवा देवेषु पराक्रमद्ध्वं प्रथमा &      \\
        \hline
            1322 & TB\_3.7.5.2 & मुखमपोहामि सूर्यज्योति र्विभाहि महत &      \\
        \hline
            1323 & TB\_3.7.5.3 & घृतस्य धारया सुशेवं कल्पयामि &      \\
        \hline
            1324 & TB\_3.7.5.4 & देवाः पितरः पितरो देवाः &      \\
        \hline
            1325 & TB\_3.7.5.5 & पृथिवी माता प्रजापति र्बन्धुः &      \\
        \hline
            1326 & TB\_3.7.5.6 & आज्येन प्रत्यनज्म्येनत् तत्त आप्यायतां &      \\
        \hline
            1327 & TB\_3.7.5.7 & इडे भागं जुषस्व नः &      \\
        \hline
            1328 & TB\_3.7.5.8 & दैवीश्च मानुषीश्च अहोरात्रे मे &      \\
        \hline
            1329 & TB\_3.7.5.9 & विधेम हविषा वयं भजतां &      \\
        \hline
            1330 & TB\_3.7.5.10 & सोम्यानाꣳ सोमपीथिनां निर्भक्तो ब्राह्मणः &      \\
        \hline
            1331 & TB\_3.7.5.11 & उपनिषदे सुप्रजास्त्वाय सं पत्नी &      \\
        \hline
            1332 & TB\_3.7.5.12 & ऊर्जो भागꣳ शतक्रतू एतद्वां &      \\
        \hline
            1333 & TB\_3.7.5.13 & मा नो विदद्{-}वजना द्वेष्या &      \\
        \hline
            1334 & TB\_3.7.6.1 & परिस्तृणीत परिधत्ताग्निं परिहितो{-}ऽग्नि र्यजमानं &      \\
        \hline
            1335 & TB\_3.7.6.2 & देवेन सवित्रा प्रसूत आर्त्विज्यं &      \\
        \hline
            1336 & TB\_3.7.6.3 & प्रजापति र्विश्वेभ्यो देवेभ्यः विश्वे &      \\
        \hline
            1337 & TB\_3.7.6.4 & अन्तर्दूतश्चरति मानुषीषु चतुः शिखण्डा &      \\
        \hline
            1338 & TB\_3.7.6.5 & यो ब्रह्मणा कर्मणा द्वेष्टि &      \\
        \hline
            1339 & TB\_3.7.6.6 & सा मे धुक्ष्व यजमानाय &      \\
        \hline
            1340 & TB\_3.7.6.7 & अस्मिन्. यज्ञ् उप भूय &      \\
        \hline
            1341 & TB\_3.7.6.8 & सीदन्ती देवी सुकृतस्य लोके &      \\
        \hline
            1342 & TB\_3.7.6.9 & यत्रर्.षयः प्रथमजा ये पुराणाः &      \\
        \hline
            1343 & TB\_3.7.6.10 & इदमस्य चित्तमधरं ध्रुवायाः अहमुत्तरो &      \\
        \hline
            1344 & TB\_3.7.6.11 & अधरे मथ्सपत्नाः इयꣳ स्थाली &      \\
        \hline
            1345 & TB\_3.7.6.12 & सर्वतो मां भूतं भविष्य{-}च्छ्रयतां &      \\
        \hline
            1346 & TB\_3.7.6.13 & येनासिञ्चद् बलमिन्द्रे प्रजापतिः इदं &      \\
        \hline
            1347 & TB\_3.7.6.14 & तेन लोकान्थ् सूर्यवतो जयेम &      \\
        \hline
            1348 & TB\_3.7.6.15 & वाजजित्यायै संमार्ज्मि अग्निमन्ना{-}दमन्नाद्याय उपहूतो &      \\
        \hline
            1349 & TB\_3.7.6.16 & आयुषे वर्चसे जीवात्वै पुण्याय &      \\
        \hline
            1350 & TB\_3.7.6.17 & अग्ने यो नोऽभिदासति समानो &      \\
        \hline
            1351 & TB\_3.7.6.18 & वाजं जिगिवाꣳसं वाजिनं ॅवाजजितं &      \\
        \hline
            1352 & TB\_3.7.6.19 & दिवः खीलोऽवततः पृथिव्या अद्ध्युत्थितः &      \\
        \hline
            1353 & TB\_3.7.6.20 & सुभूतेन मे संतिष्ठस्व ब्रह्मवर्चसेन &      \\
        \hline
            1354 & TB\_3.7.6.21 & उलूखले मुसले यच्च शूर्पे &      \\
        \hline
            1355 & TB\_3.7.6.22 & उद्यन्नद्य वि नो भज &      \\
        \hline
            1356 & TB\_3.7.6.23 & अथो हारिद्रवेषु मे हरिमाणं &      \\
        \hline
            1357 & TB\_3.7.6.24 & यो नः सपत्नो यो &      \\
        \hline
            1358 & TB\_3.7.7.1 & सक्षेदं पश्य विधर्तरिदं पश्य &      \\
        \hline
            1359 & TB\_3.7.7.2 & ब्रह्मासि क्षत्रस्य योनिः क्षत्रमस्यृतस्य &      \\
        \hline
            1360 & TB\_3.7.7.3 & ज्योग्जीवा जरामशीमहि इन्द्र शाक्वर &      \\
        \hline
            1361 & TB\_3.7.7.4 & तां ते युनज्मि आऽहं &      \\
        \hline
            1362 & TB\_3.7.7.5 & तयाऽग्नि र्दीक्षया दीक्षितः ययाऽग्नि &      \\
        \hline
            1363 & TB\_3.7.7.6 & तया त्वा दीक्षया दीक्षयामि &      \\
        \hline
            1364 & TB\_3.7.7.7 & तया सोमो राजा दीक्षया &      \\
        \hline
            1365 & TB\_3.7.7.8 & दिशस्त्वा दीक्षमाणमनु दीक्षन्तां आपस्त्वा &      \\
        \hline
            1366 & TB\_3.7.7.9 & आपश्चौषधयश्च ऊर्क्च सूनृता च &      \\
        \hline
            1367 & TB\_3.7.7.10 & तपो मे प्रतिष्ठा सवितृ{-}प्रसूता &      \\
        \hline
            1368 & TB\_3.7.7.11 & विश्वे देवा यजमानश्च सीदत &      \\
        \hline
            1369 & TB\_3.7.7.12 & सख्यात्ते मा योषं सख्यान्मे &      \\
        \hline
            1370 & TB\_3.7.7.13 & परो रजास्ते पञ्चमः पादः &      \\
        \hline
            1371 & TB\_3.7.7.14 & वर्ष्मन्दिवः नाभा पृथिव्याः यथाऽयं &      \\
        \hline
            1372 & TB\_3.7.8.1 & यदस्य पारे रजसः शुक्रं &      \\
        \hline
            1373 & TB\_3.7.8.2 & प्रजाभ्यः सर्वाभ्यो मृड नमो &      \\
        \hline
            1374 & TB\_3.7.8.3 & य इदमकः तस्मै नमः &      \\
        \hline
            1375 & TB\_3.7.9.1 & अनागसस्त्वा वयं इन्द्रेण प्रेषिता &      \\
        \hline
            1376 & TB\_3.7.9.2 & युक्ताः स्थ वहत देवा &      \\
        \hline
            1377 & TB\_3.7.9.3 & प्राणश्च त्वाऽपानश्च श्रीणीतां चक्षुश्च &      \\
        \hline
            1378 & TB\_3.7.9.4 & यो देवानां देवतमस्तपोजाः तस्मै &      \\
        \hline
            1379 & TB\_3.7.9.5 & तस्य सुम्नमशीमहि तस्य भक्षमशीमहि &      \\
        \hline
            1380 & TB\_3.7.9.6 & प्र ते महे विदथे &      \\
        \hline
            1381 & TB\_3.7.9.7 & इन्द्रश्च सम्राड् वरुणश्च राजा &      \\
        \hline
            1382 & TB\_3.7.9.8 & उद्याने यत्परायणे आवर्तने विवर्तने &      \\
        \hline
            1383 & TB\_3.7.9.9 & सर्वान् पथो अनृणा आक्षीयेम &      \\
        \hline
            1384 & TB\_3.7.10.1 & उदस्तांफ्सीथ् सविता मित्रो अर्यमा &      \\
        \hline
            1385 & TB\_3.7.10.2 & दैव्यः केतु र्विश्वं भुवन{-}माविवेश &      \\
        \hline
            1386 & TB\_3.7.10.3 & देवा नो यज्ञ्{-}मृजुधा नयन्तु &      \\
        \hline
            1387 & TB\_3.7.10.4 & स्कन्नेमा विश्वा भुवना स्कन्नो &      \\
        \hline
            1388 & TB\_3.7.10.5 & अदितिरूत्या{-}ऽऽगमत् सा शन्ताची मयस्करत् &      \\
        \hline
            1389 & TB\_3.7.10.6 & अप स्रिधः तदित्पदं न &      \\
        \hline
            1390 & TB\_3.7.11.1 & ब्रह्म प्रतिष्ठा मनसो ब्रह्म &      \\
        \hline
            1391 & TB\_3.7.11.2 & यद्वो देवा अतिपादयानि वाचा &      \\
        \hline
            1392 & TB\_3.7.11.3 & इमं मे वरुण तत्त्वा &      \\
        \hline
            1393 & TB\_3.7.11.4 & ऋद्ध्यै स्वाहा समृद्ध्यै स्वाहा &      \\
        \hline
            1394 & TB\_3.7.11.5 & आप्यायय हरिवो वर्द्धमानः यदा &      \\
        \hline
            1395 & TB\_3.7.12.1 & यद्देवा देवहेडनं देवासश्चकृमा वयं &      \\
        \hline
            1396 & TB\_3.7.12.2 & ऋतेन द्यावापृथिवी ऋतेन त्वं &      \\
        \hline
            1397 & TB\_3.7.12.3 & शिश्ञै र्यदनृतं चकृमा वयं &      \\
        \hline
            1398 & TB\_3.7.12.4 & एनश्चकार यत्पिता अग्निर्मा तस्मादेनसः &      \\
        \hline
            1399 & TB\_3.7.12.5 & यदेनश्च कृमा नूतनं ॅयत्पुराणं &      \\
        \hline
            1400 & TB\_3.7.12.6 & गार्.हपत्यः प्रमुञ्चतु दुरिता यानि &      \\
        \hline
            1401 & TB\_3.7.13.1 & यत्ते ग्राव्.ण्णा चिच्छिदुः सोम &      \\
        \hline
            1402 & TB\_3.7.13.2 & त्वया तथ्सोम गुप्तमस्तु नः &      \\
        \hline
            1403 & TB\_3.7.13.3 & उपक्षरन्ति जुह्वो घृतेन प्रियाण्यङ्गानि &      \\
        \hline
            1404 & TB\_3.7.13.4 & यदागच्छात् पथिभि र्देवयानैः इष्टापूर्ते &      \\
        \hline
            1405 & TB\_3.7.14.1 & यद्दिदीक्षे मनसा यच्च वाचा &      \\
        \hline
            1406 & TB\_3.7.14.2 & शुक्रा दीक्षायै तपसो विमोचनीः &      \\
        \hline
            1407 & TB\_3.7.14.3 & अग्नेः प्रियतमꣳ हविः स्वाहा &      \\
        \hline
            1408 & TB\_3.7.14.4 & देवेभ्यः पितृभ्यः स्वाहा सोम्येभ्यः &      \\
        \hline
            1409 & TB\_3.7.14.5 & अभि नः शीयताꣳ रयिः &      \\
        \hline
            1410 & TB\_3.8.1.1 & साग्रंहण्येष्ट्या यजते इमां जनतां &      \\
        \hline
            1411 & TB\_3.8.2.1 & चतुष्टय्य आपो भवन्ति चतुश्शफो &      \\
        \hline
            1412 & TB\_3.8.2.2 & चतुश्शरावो भवति दिक्ष्वेव प्रतितिष्ठति &      \\
        \hline
            1413 & TB\_3.8.2.3 & यदाज्य{-}मुच्छिष्यते तस्मिन्{-}रशनां न्युनत्ति प्रजापतिर्वा &      \\
        \hline
            1414 & TB\_3.8.2.4 & यद्दर्भमयी रशना भवति पुनात्येवैनम् &      \\
        \hline
            1415 & TB\_3.8.3.1 & यो वै ब्रह्मणे देवेभ्यः &      \\
        \hline
            1416 & TB\_3.8.3.2 & न देवताभ्य आवृश्च्यते वसीयान् & TB\_3.3.10.2        \\
        \hline
            1417 & TB\_3.8.3.3 & तदाहुः द्वादशारत्नी रशना कर्तव्या3 &      \\
        \hline
            1418 & TB\_3.8.3.4 & यथर्.षभस्य विष्टपꣳ सꣳस्करोति तादृगेव &      \\
        \hline
            1419 & TB\_3.8.3.5 & तस्मा{-}दश्वमेधयाजी सर्वाणि भूतान्यभि भवति &      \\
        \hline
            1420 & TB\_3.8.3.6 & प्रजयैवैनं पशुभिः प्रथयति स्वाहाकृत &      \\
        \hline
            1421 & TB\_3.8.4.1 & यः पितुरनुजायाः पुत्रः स &      \\
        \hline
            1422 & TB\_3.8.4.2 & कर्म कर्मैवास्मै साधयति पौꣳश्चलेयो &      \\
        \hline
            1423 & TB\_3.8.5.1 & चत्वार ऋत्विजः समुक्षन्ति आभ्य &      \\
        \hline
            1424 & TB\_3.8.5.2 & दक्षिणत उदङ्तिष्ठन् प्रोक्षति अनेनाश्वेन &      \\
        \hline
            1425 & TB\_3.8.5.3 & बहुग्वै बह्वश्वायै बह्वजाविकायै बहुव्रीहियवायै &      \\
        \hline
            1426 & TB\_3.8.5.4 & अनेनाश्वेन मेद्ध्ये नेष्ट्वा अयं &      \\
        \hline
            1427 & TB\_3.8.6.1 & यथा वै हविषो गृहीतस्य &  TB\_3.8.8.1       \\
        \hline
            1428 & TB\_3.8.6.2 & यत्परिमिता अनुब्रूयात् परिमितमवरुन्धीत अपरिमिता &      \\
        \hline
            1429 & TB\_3.8.6.3 & अस्यामेवैनाः प्रतिष्ठापयति उवाच ह &      \\
        \hline
            1430 & TB\_3.8.6.4 & सरस्वत्यै स्वाहेत्याह सरस्वत्या एवैनं &      \\
        \hline
            1431 & TB\_3.8.6.5 & मित्राय स्वाहेत्याह मित्रायैवैनं जुहोति &      \\
        \hline
            1432 & TB\_3.8.7.1 & प्रजापतये त्वा जुष्टं प्रोक्षामीति &      \\
        \hline
            1433 & TB\_3.8.7.2 & जवमेवास्मिन्दधाति तस्मादश्वः पशूनामाशुः सारसारितमः &      \\
        \hline
            1434 & TB\_3.8.7.3 & सर्वेभ्यस्त्वा देवेभ्य इत्युपरिष्टात् सर्वे &      \\
        \hline
            1435 & TB\_3.8.8.1 & यथा वै हविषो गृहीतस्य & TB\_3.8.6.1        \\
        \hline
            1436 & TB\_3.8.8.2 & तदाहुः अनाहुतयो वा अश्वचरितानि &      \\
        \hline
            1437 & TB\_3.8.8.3 & यस्या नायतने{-}ऽन्यत्राग्ने{-}राहुतीर् जुहोति सावित्रिया &      \\
        \hline
            1438 & TB\_3.8.8.4 & यद्यज्ञ्मुखे यज्ञ्मुखे जुहुयात् पशुभि{-}र्यजमानं &      \\
        \hline
            1439 & TB\_3.8.9.1 & विभूर्मात्रा प्रभूः पित्रेत्याह इयं &      \\
        \hline
            1440 & TB\_3.8.9.2 & प्र यशः श्रैष्ठ्यमाप्नोति य &      \\
        \hline
            1441 & TB\_3.8.9.3 & आदित्यानां पत्वा{-}ऽन्विहीत्याह आदित्यानेवैनं गमयति &      \\
        \hline
            1442 & TB\_3.8.9.4 & एता वा अश्वस्य बन्धनम् &      \\
        \hline
            1443 & TB\_3.8.9.5 & राष्ट्रादेव ते व्यवच्छिद्यन्ते परा &      \\
        \hline
            1444 & TB\_3.8.10.1 & प्रजापति{-}रकामयताश्वमेधेन यजेयेति स तपोऽतप्यत &      \\
        \hline
            1445 & TB\_3.8.10.2 & सप्त जुहोति सप्त हि &      \\
        \hline
            1446 & TB\_3.8.10.3 & एकविꣳशतिं ॅवैश्वदेवानि जुहोति एकविꣳशतिर्वै &      \\
        \hline
            1447 & TB\_3.8.10.4 & त्रिꣳशतमौद्ग्रहणानि जुहोति त्रिꣳशदक्षरा विराट् &      \\
        \hline
            1448 & TB\_3.8.10.5 & यो दीक्षामति{-}रेचयति सप्ताहं प्रचरन्ति &      \\
        \hline
            1449 & TB\_3.8.11.1 & प्रजापति{-}रश्वमेध{-}मसृजत तꣳ सृष्टं न &      \\
        \hline
            1450 & TB\_3.8.11.2 & अदित्यै स्वाहाऽदित्यै मह्यै स्वाहाऽदित्यै &      \\
        \hline
            1451 & TB\_3.8.12.1 & सावित्र{-}मष्टकपालं प्रातर्निर्वपति अष्टाक्षरा गायत्री &      \\
        \hline
            1452 & TB\_3.8.12.2 & अथो माद्ध्यन्दिनमेव सवनं तेनाप्नोति &      \\
        \hline
            1453 & TB\_3.8.12.3 & चतस्रो दिशः दिग्भिरेवैनं परिगृह्णाति &      \\
        \hline
            1454 & TB\_3.8.13.1 & आ ब्रह्मन् ब्राह्मणो ब्रह्मवर्चसी &      \\
        \hline
            1455 & TB\_3.8.13.2 & अनडुह्येव वीर्यं दधाति तस्मात्पुरा &      \\
        \hline
            1456 & TB\_3.8.13.3 & यत्रैतेन यज्ञेन यजन्ते सभेयो &      \\
        \hline
            1457 & TB\_3.8.14.1 & प्रजापति{-}र्देवेभ्यो यज्ञान् व्यादिशत् स &      \\
        \hline
            1458 & TB\_3.8.14.2 & देवानेव तैर्यजमानः प्रीणाति आज्येन &      \\
        \hline
            1459 & TB\_3.8.14.3 & महतीमेव तद्देवतां प्रीणाति तण्डुलै{-}र्जुहोति &      \\
        \hline
            1460 & TB\_3.8.14.4 & रुद्रानेव तत्प्रीणाति लाजैर्जुहोति आदित्यानां &      \\
        \hline
            1461 & TB\_3.8.14.5 & विश्वानेव तद्देवान् प्रीणाति धानाभि{-}र्जुहोति &      \\
        \hline
            1462 & TB\_3.8.14.6 & प्रजापतिमेव तत्प्रीणाति मसूस्यैर्जुहोति सर्वासां &      \\
        \hline
            1463 & TB\_3.8.15.1 & प्रजापति{-}रश्वमेधमसृजत तꣳ सृष्टꣳ रक्षाꣳस्य &      \\
        \hline
            1464 & TB\_3.8.15.2 & आज्यस्य प्रतिपदं करोति प्राणो &      \\
        \hline
            1465 & TB\_3.8.15.3 & प्राणो वा आज्यम् उभयत &      \\
        \hline
            1466 & TB\_3.8.16.1 & प्रजापतिं ॅवा एष ईफ्सतीत्याहुः &      \\
        \hline
            1467 & TB\_3.8.16.2 & एकवदेव सुवर्गं ॅलोकमेति सन्ततं &      \\
        \hline
            1468 & TB\_3.8.16.3 & त्रय इमे लोकाः इमानेव &      \\
        \hline
            1469 & TB\_3.8.16.4 & समुद्रमेवाप्नोति मद्ध्याय स्वाहेत्याह मद्ध्यमेवाप्नोति &      \\
        \hline
            1470 & TB\_3.8.17.1 & विभूर्मात्रा प्रभूः पित्रेत्यश्व{-}नामानि जुहोति &      \\
        \hline
            1471 & TB\_3.8.17.2 & पृथिव्यै स्वाहाऽन्तरिक्षाय स्वाहेत्ये{-}कविꣳशिनीं दीक्षां &      \\
        \hline
            1472 & TB\_3.8.17.3 & अर्वाङ्यज्ञ्ः संक्रामत्वि{-}त्याप्ती{-}र्जुहोति सुवर्गस्य लोकस्याप्त्यै &      \\
        \hline
            1473 & TB\_3.8.17.4 & दद्भ्यः स्वाहा हनूभ्याꣳ स्वाहेत्यङ्गहोमा{-}ञ्जुहोति &      \\
        \hline
            1474 & TB\_3.8.17.5 & वनस्पतिभ्यः स्वाहेति वनस्पति{-}होमाञ्जुहोति आरण्यस्यान्नाद्यस्यावरुद्ध्यै &      \\
        \hline
            1475 & TB\_3.8.18.1 & अम्भाꣳसि जुहोति अयं ॅवै &      \\
        \hline
            1476 & TB\_3.8.18.2 & तस्य रुद्रा अधिपतयः वायुर्ज्योतिः &      \\
        \hline
            1477 & TB\_3.8.18.3 & यन्महाꣳसि जुहोति अमुमेव लोकमवरुन्धे &      \\
        \hline
            1478 & TB\_3.8.18.4 & सिताय स्वाहा{-}ऽसिताय स्वाहेति प्रमुक्ती{-}र्जुहोति &      \\
        \hline
            1479 & TB\_3.8.18.5 & यः प्राणतो य आत्मदा &      \\
        \hline
            1480 & TB\_3.8.18.6 & असौ भविष्यत् अनयोरेव लोकयोः &      \\
        \hline
            1481 & TB\_3.8.18.7 & य उ चैनमेवं ॅवेद & TB\_2.7.11.3  TB\_3.9.22.3       \\
        \hline
            1482 & TB\_3.8.19.1 & एकयूपो वैकादशिनी वा अन्येषां &      \\
        \hline
            1483 & TB\_3.8.19.2 & पाप्मा वै तेजनी पाप्मनोऽपहत्यै &      \\
        \hline
            1484 & TB\_3.8.20.1 & राज्जुदाल{-}मग्निष्ठं मिनोति भ्रूणहत्याया अपहत्यै &      \\
        \hline
            1485 & TB\_3.8.20.2 & षट्पालाशाः सोमपीथस्या{-}वरुद्ध्यै एकविꣳशतिः संपद्यन्ते &      \\
        \hline
            1486 & TB\_3.8.20.3 & शतायुः पुरुषः शतेन्द्रियः आयुष्येवेन्द्रिये & TB\_1.7.8.2 TB\_1.8.6.5        \\
        \hline
            1487 & TB\_3.8.20.4 & एषा वै वरुणस्य दिक् &      \\
        \hline
            1488 & TB\_3.8.20.5 & प्राणापाना{-}वेवास्मिन थ्सम्यञ्चौ दधाति अश्वं &      \\
        \hline
            1489 & TB\_3.8.21.1 & एकविꣳशोऽग्निर्भवति एकविꣳशः स्तोमः एकविꣳशतिर्यूपाः &      \\
        \hline
            1490 & TB\_3.8.21.2 & एकादश यूपाः यद्द्वादशो{-}ऽग्निर्भवति द्वादश &      \\
        \hline
            1491 & TB\_3.8.21.3 & दुह एवैनां तेन तदाहुः &      \\
        \hline
            1492 & TB\_3.8.21.4 & यो वा अश्वमेधे तिस्रः &      \\
        \hline
            1493 & TB\_3.8.22.1 & देवा वा अश्वमेधे पवमाने &      \\
        \hline
            1494 & TB\_3.8.22.2 & उद्गातार{-}मपरुद्ध्य अश्व{-}मुद्गीथाय वृणीते यथा &      \\
        \hline
            1495 & TB\_3.8.22.3 & वडबा उप रुन्धन्ति मिथुनत्वाय &      \\
        \hline
            1496 & TB\_3.8.23.1 & पुरुषो वै यज्ञ्ः यज्ञ्ः &      \\
        \hline
            1497 & TB\_3.8.23.2 & तस्मा{-}त्पूर्वाग्निं पुरस्ता{-}थ्स्थापयन्ति पौष्णमन्वञ्चम् अन्नं &      \\
        \hline
            1498 & TB\_3.8.23.3 & तस्माद् राजन्यो बाहुबली भावुकः &      \\
        \hline
            1499 & TB\_3.9.1.1 & प्रजापति{-}रश्वमेधमसृजत सोऽस्माथ्सृष्टो{-}ऽपाक्रामत् तमष्टादशिभिरनु प्रायुङ्क्त &      \\
        \hline
            1500 & TB\_3.9.1.2 & सम्ॅवथ्सरोऽष्टादशः यदष्टादशिन आलभ्यन्ते सम्ॅवथ्सरमेव &      \\
        \hline
            1501 & TB\_3.9.1.3 & ऋक्षीकाः पुरुषव्याघ्राः परिमोषिण आव्याधिनी{-}स्तस्करा &      \\
        \hline
            1502 & TB\_3.9.1.4 & यज्ञ्वेशसं कुर्यात् यत्पशूनालभते तेनैव &      \\
        \hline
            1503 & TB\_3.9.2.1 & प्रजापतिर{-}कामयतोभौ लोकाव{-}वरुन्धीयेति स एतानुभयान् &      \\
        \hline
            1504 & TB\_3.9.2.2 & अमुं तैः अनवरुद्धो वा &      \\
        \hline
            1505 & TB\_3.9.2.3 & सुवर्गं तु लोकं नापराद्ध्नोति &      \\
        \hline
            1506 & TB\_3.9.2.4 & विराजमेव तैराप्त्वा यजमानो{-}ऽवरुन्धे एकादश &      \\
        \hline
            1507 & TB\_3.9.3.1 & अस्मै वै लोकाय ग्राम्याः &      \\
        \hline
            1508 & TB\_3.9.3.2 & ग्राम्याꣳश्चा{-}रण्याꣳश्च उभयस्या{-}न्नाद्यस्या{-}वरुद्ध्यै उभयान् पशूनालभते &      \\
        \hline
            1509 & TB\_3.9.3.3 & अस्मिन् ॅलोके बहवः कामा &      \\
        \hline
            1510 & TB\_3.9.4.1 & युञ्जन्ति ब्रद्ध्नमित्याह असौ वा &      \\
        \hline
            1511 & TB\_3.9.4.2 & इमे वै लोकाः परितस्थुषः &      \\
        \hline
            1512 & TB\_3.9.4.3 & शोणा धृष्णू नृवाहसेत्याह अहोरात्रे &      \\
        \hline
            1513 & TB\_3.9.4.4 & परा वा एतस्य यज्ञ् &      \\
        \hline
            1514 & TB\_3.9.4.5 & भूर्भुवस्सुवरिति प्राजापत्या{-}भिरावयन्ति प्राजापत्यो वा &      \\
        \hline
            1515 & TB\_3.9.4.6 & ज्योतिश्चैवास्मै राष्ट्रं च समीची &      \\
        \hline
            1516 & TB\_3.9.4.7 & रुद्रास्त्वा{-}ऽञ्जन्तु त्रैष्टुभेन छन्दसेति वावाता &      \\
        \hline
            1517 & TB\_3.9.4.8 & यत्पत्नयः श्रिय{-}मेवास्मिन् तद्दधति नास्मात्तेज &      \\
        \hline
            1518 & TB\_3.9.5.1 & तेजसा वा एष ब्रह्मवर्चसेन &      \\
        \hline
            1519 & TB\_3.9.5.2 & उत्तरत आयतनो वै होता &      \\
        \hline
            1520 & TB\_3.9.5.3 & दिवमेव वृष्टिमवरुन्धे किꣳ स्विदासीद्{-}बृहद्वय &      \\
        \hline
            1521 & TB\_3.9.5.4 & कः स्विदेकाकी चरतीत्याह असौ &      \\
        \hline
            1522 & TB\_3.9.5.5 & अयं ॅवै लोक आवपनं &      \\
        \hline
            1523 & TB\_3.9.6.1 & अप वा एतस्मा{-}त्प्राणाः क्रामन्ति &      \\
        \hline
            1524 & TB\_3.9.6.2 & अथो धुवन्त्येवैनम् अथो न्येवास्मै & TB\_1.4.6.7        \\
        \hline
            1525 & TB\_3.9.6.3 & ये यज्ञे धुवनं तन्वते &      \\
        \hline
            1526 & TB\_3.9.6.4 & सुवर्गमेवैनां ॅलोकं गमयति आऽहमजानि &      \\
        \hline
            1527 & TB\_3.9.6.5 & यथा यजुरेवैतत् त्रय्यः सूच्यो &      \\
        \hline
            1528 & TB\_3.9.7.1 & अप वा एतस्माच्छ्री राष्ट्रं &  TB\_3.9.14.1       \\
        \hline
            1529 & TB\_3.9.7.2 & श्रियमेवा{-}वरुन्धे शीते वाते पुनन्नि{-}वेत्याह &      \\
        \hline
            1530 & TB\_3.9.7.3 & शूद्रा यदर्यजारा न पोषाय &      \\
        \hline
            1531 & TB\_3.9.7.4 & राष्ट्रं पसः राष्ट्रमव विश्याहन्ति &      \\
        \hline
            1532 & TB\_3.9.7.5 & प्रसुलामीति ते पिता गभे &      \\
        \hline
            1533 & TB\_3.9.8.1 & प्रजापतिः प्रजाः सृष्ट्वा प्रेणाऽनु &      \\
        \hline
            1534 & TB\_3.9.8.2 & वैराजो वै पुरुषः विराज{-}मेवालभते &      \\
        \hline
            1535 & TB\_3.9.8.3 & यज्ञो वै गौः यज्ञ्मेवालभते &      \\
        \hline
            1536 & TB\_3.9.9.1 & प्रथमेन वा एष स्तोमन &      \\
        \hline
            1537 & TB\_3.9.9.2 & उतेव ग्राम्याः उतेवारण्याः अहरेव &      \\
        \hline
            1538 & TB\_3.9.9.3 & तेनैवोभयान् पशूनवरुन्धे प्राजापत्या भवन्ति &      \\
        \hline
            1539 & TB\_3.9.10.1 & प्रजापति{-}रकामयत महानन्नादः स्यामिति स &      \\
        \hline
            1540 & TB\_3.9.11.1 & वैश्वदेवो वा अश्वः तं &      \\
        \hline
            1541 & TB\_3.9.11.2 & चतुर्दशैता{-}ननुवाकाञ्जुहोत्य{-}नन्तरित्यै प्रयासाय स्वाहेति पञ्चदशम् &      \\
        \hline
            1542 & TB\_3.9.11.3 & पराऽसुराः यथ्स्विष्टकृद्भ्यो लोहितं जुहोति &      \\
        \hline
            1543 & TB\_3.9.11.4 & अश्वशफेन द्वितीया{-}माहुतिं जुहोति पशवो &      \\
        \hline
            1544 & TB\_3.9.12.1 & अश्वस्य वा आलब्धस्य मेध &      \\
        \hline
            1545 & TB\_3.9.12.2 & बार्.हताः पशवः सा पशूनां &      \\
        \hline
            1546 & TB\_3.9.12.3 & अश्वस्तोमीयꣳ हुत्वा द्विपदा जुहोति &      \\
        \hline
            1547 & TB\_3.9.13.1 & प्रजापति{-}रश्वमेध{-}मसृजत सो{-}ऽस्माथ्सृष्टो{-}ऽपाक्रामत् तं ॅयज्ञ्{-}क्रतुभि{-}रन्वैच्छत् &      \\
        \hline
            1548 & TB\_3.9.13.2 & इयं ॅवै सविता यो &      \\
        \hline
            1549 & TB\_3.9.13.3 & यत्प्रातरिष्टिभि{-}र्यजते अश्वमेव तदन्विच्छति यथ्सायं &      \\
        \hline
            1550 & TB\_3.9.14.1 & अप वा एतस्माच्छ्री राष्ट्रं & TB\_3.9.7.1        \\
        \hline
            1551 & TB\_3.9.14.2 & प्रभ्रꣳशुका{-}ऽस्माच्छ्रीः स्यात् न वै &      \\
        \hline
            1552 & TB\_3.9.14.3 & न वै ब्राह्मणे राष्ट्रं &      \\
        \hline
            1553 & TB\_3.9.14.4 & इष्टापूर्ते{-}नैवैनꣳ स समद्र्धयति इत्यजिना &      \\
        \hline
            1554 & TB\_3.9.15.1 & सर्वेषु वा एषु लोकेषु &      \\
        \hline
            1555 & TB\_3.9.15.2 & अशनया मृत्युरेव तमेवामुष्मिन् ॅलोकेऽवयजते &      \\
        \hline
            1556 & TB\_3.9.15.3 & मृत्युमे वा हुत्या तर्पयित्वा &      \\
        \hline
            1557 & TB\_3.9.16.1 & वारुणो वा अश्वः तं &      \\
        \hline
            1558 & TB\_3.9.16.2 & धर्मो वा अधिपतिः धर्ममेवा{-}वरुन्धे &      \\
        \hline
            1559 & TB\_3.9.16.3 & प्र वा एष एभ्यो &      \\
        \hline
            1560 & TB\_3.9.16.4 & ब्रह्मक्षत्रे एवावरुन्धे यदाऽश्विनो भवति &      \\
        \hline
            1561 & TB\_3.9.17.1 & यद्यश्वमुपतपद्विन्देत् आग्नेय{-}मष्टाकपालं निर्वपेत् सौम्यं &      \\
        \hline
            1562 & TB\_3.9.17.2 & ताभिरेवैनं भिषज्यति यथ्सावित्रो भवति &      \\
        \hline
            1563 & TB\_3.9.17.3 & रौद्रं चरुं निर्वपेत् यदि &      \\
        \hline
            1564 & TB\_3.9.17.4 & अग्नये{-}ऽꣳहोमुचेऽष्टाकपालः सौर्यं पयः वायव्य &      \\
        \hline
            1565 & TB\_3.9.17.5 & यस्याश्वो मेधाय प्रोक्षितो{-}ऽद्ध्येति सौर्यं &      \\
        \hline
            1566 & TB\_3.9.18.1 & तदाहुः द्वादश ब्रह्मौदनान् थ्सꣳस्थित &      \\
        \hline
            1567 & TB\_3.9.18.2 & साऽऽप्ता भवति यातयाम्नी क्रूरीकृतेव &      \\
        \hline
            1568 & TB\_3.9.19.1 & एष वै विभूर्नाम यज्ञ्ः &      \\
        \hline
            1569 & TB\_3.9.19.2 & सर्वꣳ ह वै तत्र &      \\
        \hline
            1570 & TB\_3.9.19.3 & यत्रैतेन यज्ञेन यजन्ते एष &      \\
        \hline
            1571 & TB\_3.9.20.1 & तार्प्येणाश्वꣳ संज्ञ्पयन्ति यज्ञो वै &      \\
        \hline
            1572 & TB\_3.9.20.2 & रुक्मो भवति सुवर्गस्य लोकस्या{-}नुख्यात्यै &      \\
        \hline
            1573 & TB\_3.9.20.3 & अन्तरिक्षं कृत्यधीवासेन दिवꣳ हिरण्यकशिपुना &      \\
        \hline
            1574 & TB\_3.9.21.1 & आदित्याश्चाङ्गिरसश्च सुवर्गे लोकेऽस्पद्र्धन्त तेऽङ्गिरस &      \\
        \hline
            1575 & TB\_3.9.21.2 & यच्छ्वयदरुरासीत् तस्मादर्वा नाम यथ्सद्यो &      \\
        \hline
            1576 & TB\_3.9.21.3 & योनि{-}मानायतनवान् भवति स एवं &      \\
        \hline
            1577 & TB\_3.9.22.1 & प्रजापतिं ॅवै देवाः पितरम् &      \\
        \hline
            1578 & TB\_3.9.22.2 & तस्मादश्वमेधः वेदुको{-}ऽश्वमाशुं भवति य &      \\
        \hline
            1579 & TB\_3.9.22.3 & य उ चैनमेवं ॅवेद & TB\_2.7.11.3 TB\_3.8.18.7        \\
        \hline
            1580 & TB\_3.9.22.4 & प्राजापत्येनैव यज्ञेन यजते कामप्रेण &      \\
        \hline
            1581 & TB\_3.9.23.1 & यो वा अश्वस्य मेद्ध्यस्य &      \\
        \hline
            1582 & TB\_3.9.23.2 & यद्दर्.शपूर्णमासौ यजते अश्वस्यैव मेद्ध्यस्य &      \\
        \hline
            1583 & TB\_3.10.1.1 & संज्ञानं ॅविज्ञानं प्रज्ञानं जानदभिजानत् &      \\
        \hline
            1584 & TB\_3.10.1.2 & आवेशय{-}न्निवेशयन्थ्सम्ॅवेशनः सꣳशान्तः शान्तः आभवन् &      \\
        \hline
            1585 & TB\_3.10.1.3 & कान्ता काम्या कामजाता ऽऽयुष्मती &      \\
        \hline
            1586 & TB\_3.10.1.4 & अरुणो ऽरुणरजाः पुण्डरीको विश्वजिद{-}भिजित् &      \\
        \hline
            1587 & TB\_3.10.2.1 & भूरग्निं च पृथिवीं च &      \\
        \hline
            1588 & TB\_3.10.3.1 & त्वमेव त्वां ॅवेत्थ योऽसि &      \\
        \hline
            1589 & TB\_3.10.4.1 & सम्ॅवथ्सरोऽसि परिवथ्सरोऽसि इदावथ्सरोऽसी{-}दुवथ्सरोऽसि इद्वथ्सरोऽसि &      \\
        \hline
            1590 & TB\_3.10.4.2 & अहोरात्रा{-}णीष्टकाः ऋषभोऽसि स्वर्गो लोकः &      \\
        \hline
            1591 & TB\_3.10.4.3 & इन्दु र्दक्षः श्येन ऋतावा &      \\
        \hline
            1592 & TB\_3.10.5.1 & भूर्भुवः स्वः ओजो बलं &      \\
        \hline
            1593 & TB\_3.10.6.1 & राज्ञी विराज्ञी सम्राज्ञी स्वराज्ञी &      \\
        \hline
            1594 & TB\_3.10.7.1 & असवे स्वाहा वसवे स्वाहा &      \\
        \hline
            1595 & TB\_3.10.8.1 & विपश्चिते पवमानाय गायत मही &      \\
        \hline
            1596 & TB\_3.10.8.2 & ये ते सहस्रमयुतं पाशाः &      \\
        \hline
            1597 & TB\_3.10.8.3 & अन्धो जागृविः प्राण असावेहि &  TB\_3.11.5.4       \\
        \hline
            1598 & TB\_3.10.8.4 & सुहस्तः सुवासाः शूषो नामास्य{-}मृतो &      \\
        \hline
            1599 & TB\_3.10.8.5 & प्राणो हृदये हृदयं मयि &      \\
        \hline
            1600 & TB\_3.10.8.6 & मनो हृदये हृदयं मयि &      \\
        \hline
            1601 & TB\_3.10.8.7 & रेतो हृदये हृदयं मयि &      \\
        \hline
            1602 & TB\_3.10.8.8 & लोमानि हृदये हृदयं मयि &      \\
        \hline
            1603 & TB\_3.10.8.9 & मूद्र्धा हृदये हृदयं मयि &      \\
        \hline
            1604 & TB\_3.10.8.10 & आत्मा हृदये हृदयं मयि &      \\
        \hline
            1605 & TB\_3.10.9.1 & प्रजापति र्देवानसृजत ते पाप्मना &      \\
        \hline
            1606 & TB\_3.10.9.2 & वृश्चतश्च सैषा मीमाꣳसाऽग्निहोत्र एव &      \\
        \hline
            1607 & TB\_3.10.9.3 & वृश्चतश्च अत्यꣳहो हारुणिः ब्रह्मचारिणे &      \\
        \hline
            1608 & TB\_3.10.9.4 & स कस्मिन् प्रतिष्ठित इति &      \\
        \hline
            1609 & TB\_3.10.9.5 & कस्मिन्नु तप इति बल &      \\
        \hline
            1610 & TB\_3.10.9.6 & तस्माथ् सावित्रे न सम्ॅवदेत &      \\
        \hline
            1611 & TB\_3.10.9.7 & अथ यदाह प्रस्तुतं ॅविष्टुतं &      \\
        \hline
            1612 & TB\_3.10.9.8 & अथ यदाह पवित्रं पवयिष्यन्{-}थ्सहस्वान्{-}थ्सहीयानरुणो &      \\
        \hline
            1613 & TB\_3.10.9.9 & एष सम्ॅवथ्सरः अथ यदाह &      \\
        \hline
            1614 & TB\_3.10.9.10 & सर्व{-}मायुरेति अभि स्वर्गं ॅलोकं &      \\
        \hline
            1615 & TB\_3.10.9.11 & स ह हꣳसो हिरण्मयो &      \\
        \hline
            1616 & TB\_3.10.9.12 & सर्वम्बत गौतमो वेद यः &      \\
        \hline
            1617 & TB\_3.10.9.13 & स होवाच मा भैषी &      \\
        \hline
            1618 & TB\_3.10.9.14 & एतद्वै सावित्र{-}स्याष्टाक्षरं पदꣳ श्रियाभिषिक्तं &      \\
        \hline
            1619 & TB\_3.10.9.15 & तदेतत् परि यद्देवचक्रं आर्द्रं &      \\
        \hline
            1620 & TB\_3.10.10.1 & इयं ॅवाव सरघा तस्या &      \\
        \hline
            1621 & TB\_3.10.10.2 & न हास्यैता अग्नौ मधु &      \\
        \hline
            1622 & TB\_3.10.10.3 & यो ह वै मुहूर्तानां &      \\
        \hline
            1623 & TB\_3.10.10.4 & नाद्र्धमसेषु न मासेष्वा{-}र्तिमार्च्छति य &      \\
        \hline
            1624 & TB\_3.10.10.5 & इदानीं तदानीमिति एते वै &      \\
        \hline
            1625 & TB\_3.10.11.1 & कश्चिद्धवा अस्मा{-}ल्लोकात् प्रेत्य आत्मानं &      \\
        \hline
            1626 & TB\_3.10.11.2 & स स्वं ॅलोकं प्रति &      \\
        \hline
            1627 & TB\_3.10.11.3 & तस्य हैवा{-}होरात्राणि अमुष्मिन् ॅलोके &      \\
        \hline
            1628 & TB\_3.10.11.4 & तꣳ ह त्रीन् गिरि{-}रूपान &      \\
        \hline
            1629 & TB\_3.10.11.5 & तस्मै हैतमग्निं सावित्रमुवाच तं &      \\
        \hline
            1630 & TB\_3.10.11.6 & यावन्तꣳ ह वै त्रय्या &      \\
        \hline
            1631 & TB\_3.10.11.7 & इन्द्रस्यैव सायुज्यꣳ सलोकता{-}माप्नोति य &      \\
        \hline
            1632 & TB\_3.11.1.1 & लोकोऽसि स्वर्गोऽसि अनन्तोऽस्य पारोऽसि &      \\
        \hline
            1633 & TB\_3.11.1.2 & तपोऽसि लोके श्रितं तेजसः &      \\
        \hline
            1634 & TB\_3.11.1.3 & तेजोऽसि तपसि श्रितं समुद्रस्य &      \\
        \hline
            1635 & TB\_3.11.1.4 & समुद्रोऽसि तेजसि श्रितः अपां &      \\
        \hline
            1636 & TB\_3.11.1.5 & आपः स्थ समुद्रे श्रिताः &      \\
        \hline
            1637 & TB\_3.11.1.6 & पृथिव्य{-}स्यफ्सु श्रिता अग्नेः प्रतिष्ठा &      \\
        \hline
            1638 & TB\_3.11.1.7 & अग्निरसि पृथिव्याꣳ श्रितः अन्तरिक्षस्य &      \\
        \hline
            1639 & TB\_3.11.1.8 & अन्तरिक्ष{-}मस्यग्नौ श्रितं वायोः प्रतिष्ठा &      \\
        \hline
            1640 & TB\_3.11.1.9 & वायुरस्यन्तरिक्षे श्रितः दिवः प्रतिष्ठा &      \\
        \hline
            1641 & TB\_3.11.1.10 & द्यौरसि वायौ श्रिता आदित्यस्य &      \\
        \hline
            1642 & TB\_3.11.1.11 & आदित्योऽसि दिवि श्रितः चन्द्रमसः &      \\
        \hline
            1643 & TB\_3.11.1.12 & चन्द्रमा अस्यादित्ये श्रितः नक्षत्राणां &      \\
        \hline
            1644 & TB\_3.11.1.13 & नक्षत्राणि स्थ चन्द्रमसि श्रितानि &      \\
        \hline
            1645 & TB\_3.11.1.14 & सम्ॅवथ्सरोऽसि नक्षत्रेषु श्रितः ऋतूनां &      \\
        \hline
            1646 & TB\_3.11.1.15 & ऋतवः स्थ सम्ॅवथ्सरे श्रिताः &      \\
        \hline
            1647 & TB\_3.11.1.16 & मासाः स्थर्तुषु श्रिताः अद्र्धमासानां &      \\
        \hline
            1648 & TB\_3.11.1.17 & अद्र्धमासाः स्थ मासु श्रिताः &      \\
        \hline
            1649 & TB\_3.11.1.18 & अहोरात्रे स्थोऽद्र्धमासेषु श्रिते भूतस्य &      \\
        \hline
            1650 & TB\_3.11.1.19 & पौर्णमास्यष्टका ऽमावास्या अन्नादाः स्थान्नदुघो &      \\
        \hline
            1651 & TB\_3.11.1.20 & राडसि बृहती श्रीरसीन्द्रपत्नी धर्मपत्नी &      \\
        \hline
            1652 & TB\_3.11.1.21 & ओजोऽसि सहोऽसि बलमसि भ्राजोऽसि &      \\
        \hline
            1653 & TB\_3.11.2.1 & त्वमग्ने रुद्रो असुरो महो &      \\
        \hline
            1654 & TB\_3.11.2.2 & षष्ठाः सप्तमेषु श्रयद्ध्वं सप्तमा &      \\
        \hline
            1655 & TB\_3.11.2.3 & षोडशाः सप्तदशेषु श्रयद्ध्वं सप्तदशा &      \\
        \hline
            1656 & TB\_3.11.2.4 & षड्विꣳशाः सप्तविꣳशेषु श्रयद्ध्वं सप्तविꣳशा &      \\
        \hline
            1657 & TB\_3.11.3.1 & अग्नाविष्णू सजोषसा इमा वर्द्धन्तु &      \\
        \hline
            1658 & TB\_3.11.4.1 & अन्नपते ऽन्नस्य नो देहि &      \\
        \hline
            1659 & TB\_3.11.4.2 & मरुतो गणानां पतयः रुद्र &      \\
        \hline
            1660 & TB\_3.11.5.1 & सप्त ते अग्ने समिधः &      \\
        \hline
            1661 & TB\_3.11.5.2 & इन्द्रꣳ स दिशां देवं &      \\
        \hline
            1662 & TB\_3.11.5.3 & ऊद्र्ध्वा दिक् बृहस्पति र्देवता &      \\
        \hline
            1663 & TB\_3.11.5.4 & अन्धो जागृविः प्राण असावेहि & TB\_3.10.8.3        \\
        \hline
            1664 & TB\_3.11.6.1 & यत्तेऽचितं ॅयदु चितन्ते अग्ने &      \\
        \hline
            1665 & TB\_3.11.6.2 & तया देवतया{-}ऽङ्गिरस्वद्{-}ध्रुवा सीद ता &      \\
        \hline
            1666 & TB\_3.11.6.3 & यो दीदाय समिद्धः स्वे &      \\
        \hline
            1667 & TB\_3.11.6.4 & देवान् देवायते यज अग्निर्. &      \\
        \hline
            1668 & TB\_3.11.7.1 & अयं ॅवाव यः पवते &      \\
        \hline
            1669 & TB\_3.11.7.2 & अथ यथ् सम्ॅवाति तदस्य &      \\
        \hline
            1670 & TB\_3.11.7.3 & हिरण्यं ॅवा अग्ने र्नाचिकेतस्यायतनं &      \\
        \hline
            1671 & TB\_3.11.7.4 & एवमेव स तेजसा यशसा &      \\
        \hline
            1672 & TB\_3.11.7.5 & अनन्तꣳ ह वा अपार{-}मक्षय्यं &      \\
        \hline
            1673 & TB\_3.11.8.1 & उशन्. ह वै वाजश्रवसः &      \\
        \hline
            1674 & TB\_3.11.8.2 & गौतम कुमारमिति स होवाच &      \\
        \hline
            1675 & TB\_3.11.8.3 & प्रजां त इति किं &      \\
        \hline
            1676 & TB\_3.11.8.4 & किं प्रथमाꣳ रात्रि{-}माश्ना इति &      \\
        \hline
            1677 & TB\_3.11.8.5 & इष्टापूर्तयो र्मे ऽक्षितिं ब्रूहीति &      \\
        \hline
            1678 & TB\_3.11.8.6 & अप पुन र्मृत्युं जयति &      \\
        \hline
            1679 & TB\_3.11.8.7 & त{-}दस्मै नै वाच्छदयत् त{-}दात्म{-}न्नेव &      \\
        \hline
            1680 & TB\_3.11.8.8 & दक्षाय त्वा दक्षिणां प्रतिगृह्णामीति &      \\
        \hline
            1681 & TB\_3.11.9.1 & तꣳ हैत{-}मेके पशुबन्ध एवोत्तरवेद्यां &      \\
        \hline
            1682 & TB\_3.11.9.2 & कामेन समर्द्धयति अथ हैनं &      \\
        \hline
            1683 & TB\_3.11.9.3 & यथान्युप्त{-}मेवोपदधे ततो वै स &      \\
        \hline
            1684 & TB\_3.11.9.4 & पञ्च दक्षिणतः पञ्च पश्चात् &      \\
        \hline
            1685 & TB\_3.11.9.5 & सप्त पुरस्तात् तिस्रो दक्षिणतः &      \\
        \hline
            1686 & TB\_3.11.9.6 & त्रिवृत् प्रजननं उपस्थो योनि &      \\
        \hline
            1687 & TB\_3.11.9.7 & ज्यैष्ठ्यं गच्छति योऽग्निं नाचिकेतं &      \\
        \hline
            1688 & TB\_3.11.9.8 & तेजस्वी यशस्वी ब्रह्मवर्चसी स्यामिति &      \\
        \hline
            1689 & TB\_3.11.9.9 & भूयिष्ठ{-}मेवास्मै श्रद्दधते भूयिष्ठा दक्षिणा &      \\
        \hline
            1690 & TB\_3.11.10.1 & यां प्रथमा{-}मिष्टका{-}मुपदधाति इमं तया &      \\
        \hline
            1691 & TB\_3.11.10.2 & अथो या अमुष्मिन् ॅलोके &      \\
        \hline
            1692 & TB\_3.11.10.3 & ग्रीष्मो दक्षिणः पक्षः वर्.षा &      \\
        \hline
            1693 & TB\_3.11.10.4 & य उ चैन{-}मेवं ॅवेद &      \\
        \hline
            1694 & TB\_3.12.1.1 & तुभ्यन्ता अङ्गिरस्तमा श्याम तङ्काममग्ने &      \\
        \hline
            1695 & TB\_3.12.2.1 & देवेभ्यो वै स्वर्गो लोकस्तिरोऽभवत् &      \\
        \hline
            1696 & TB\_3.12.2.2 & तमाशा{-}ऽब्रवीत् प्रजापत आशया वै &      \\
        \hline
            1697 & TB\_3.12.2.3 & तं कामोऽब्रवीत् प्रजापते कामेन &      \\
        \hline
            1698 & TB\_3.12.2.4 & तं ब्रह्माऽब्रवीत् प्रजापते ब्रह्मणा &      \\
        \hline
            1699 & TB\_3.12.2.5 & तं ॅयज्ञोऽब्रवीत् प्रजापते यज्ञेन &      \\
        \hline
            1700 & TB\_3.12.2.6 & तमापो{-}ऽब्रुवन्न् प्रजापतेऽफ्सु वै सर्वे &      \\
        \hline
            1701 & TB\_3.12.2.7 & तमग्नि र्बलिमानब्रवीत् प्रजापते ऽग्नये &      \\
        \hline
            1702 & TB\_3.12.2.8 & तमनुवित्तिर{-}ब्रवीत् प्रजापते स्वर्गं ॅवै &      \\
        \hline
            1703 & TB\_3.12.2.9 & ता वा एताः सप्त &      \\
        \hline
            1704 & TB\_3.12.3.1 & तपसा देवा देवता{-}मग्र आयन्न् &      \\
        \hline
            1705 & TB\_3.12.3.2 & सा नो जुषाणोप यज्ञ्{-}मागात् &      \\
        \hline
            1706 & TB\_3.12.3.3 & यथा देवैः सधमादं मदेम &      \\
        \hline
            1707 & TB\_3.12.3.4 & सङ्कल्पजूतिं देवं ॅविपश्चिं मनो &      \\
        \hline
            1708 & TB\_3.12.4.1 & देवेभ्यो वै स्वर्गो लोकस्तिरो &      \\
        \hline
            1709 & TB\_3.12.4.2 & तं तपोऽ ब्रवीत् प्रजापते &      \\
        \hline
            1710 & TB\_3.12.4.3 & तꣳ श्रद्धा{-}ऽब्रवीत् प्रजापते श्रद्धया &      \\
        \hline
            1711 & TB\_3.12.4.4 & तꣳ सत्य{-}मब्रवीत् प्रजापते सत्येन &      \\
        \hline
            1712 & TB\_3.12.4.5 & तं मनोऽब्रवीत् प्रजापते मनसा &      \\
        \hline
            1713 & TB\_3.12.4.6 & तं चरण{-}मब्रवीत् प्रजापते चरणेन &      \\
        \hline
            1714 & TB\_3.12.4.7 & ता वा एताः पञ्च &      \\
        \hline
            1715 & TB\_3.12.5.1 & ब्रह्म वै चतुर्.होतारः चतुर्.होतृभ्योऽधि &      \\
        \hline
            1716 & TB\_3.12.5.2 & अग्निर्. होता पञ्चहोतॄणां वाग्घोता &      \\
        \hline
            1717 & TB\_3.12.5.3 & एतान्. योऽद्ध्यै{-}त्यछदिर्दर्.शे यावत्तरसं स्वरेति &      \\
        \hline
            1718 & TB\_3.12.5.4 & एतैरधिवादमपाजयत् अथो विश्वं पाप्मानं &      \\
        \hline
            1719 & TB\_3.12.5.5 & पुरस्ता{-}द्दशहोतार{-}मुदञ्च{-}मुपदधाति यावत्पदं हृदयं ॅयजुषी &      \\
        \hline
            1720 & TB\_3.12.5.6 & सदेवमग्निं चिनुते रथसमिंत{-}श्चेतव्यः वज्रो &      \\
        \hline
            1721 & TB\_3.12.5.7 & अन्तरिक्ष{-}मुक्थ्येन स्वरतिरात्रेण सर्वान् ॅलोकानहीनेन &      \\
        \hline
            1722 & TB\_3.12.5.8 & असावादित्य एकविꣳशः अमुमेवा{-}दित्यमाप्नोति शतं &      \\
        \hline
            1723 & TB\_3.12.5.9 & सर्वस्याप्त्यै सर्वस्या{-}वरुद्ध्यै यदि न &      \\
        \hline
            1724 & TB\_3.12.5.10 & हिरण्यं ददाति हिरण्य{-}ज्योतिरेव स्वर्गं &      \\
        \hline
            1725 & TB\_3.12.5.11 & तृप्यति प्रजया पशुभिः उपैनं & TB\_2.2.8.4        \\
        \hline
            1726 & TB\_3.12.5.12 & हिरण्येष्टको भवति यावदुत्तम{-}मङ्गुलि{-}काण्डं ॅयज्ञ् &      \\
        \hline
            1727 & TB\_3.12.5.13 & यावदेव वीर्यं तदस्मिन् दधाति &      \\
        \hline
            1728 & TB\_3.12.6.1 & यच्चामृतं ॅयच्च मर्त्यं यच्च &      \\
        \hline
            1729 & TB\_3.12.6.2 & सङ्ख्याता देवमायया सर्वास्ताः यावन्त &      \\
        \hline
            1730 & TB\_3.12.6.3 & सर्वास्ताः यावन्तो{-}ऽश्मानोऽस्यां पृथिव्यां प्रतिष्ठासु &      \\
        \hline
            1731 & TB\_3.12.6.4 & यावन्तो वनस्पतयः अस्यां पृथिव्यामधि &      \\
        \hline
            1732 & TB\_3.12.6.5 & देवत्रा यच्च मानुषं सर्वास्ताः &      \\
        \hline
            1733 & TB\_3.12.6.6 & सर्वास्ताः सर्वꣳ हिरण्यꣳ रजतं &      \\
        \hline
            1734 & TB\_3.12.7.1 & सर्वा दिशो दिक्षु यच्चान्त &      \\
        \hline
            1735 & TB\_3.12.7.2 & गन्धर्वा{-}फ्सरसश्च ये सर्वास्ताः सर्वा{-}नुदारान्{-}थ्सलिलान् &      \\
        \hline
            1736 & TB\_3.12.7.3 & सर्वास्ताः सर्वान् मरीचीन. विततान् &      \\
        \hline
            1737 & TB\_3.12.7.4 & याश्च कूप्या याश्च नाद्याः &      \\
        \hline
            1738 & TB\_3.12.7.5 & अन्तरिक्षचरञ्च यत् सर्वास्ताः अग्निं &      \\
        \hline
            1739 & TB\_3.12.8.1 & सर्वां दिवꣳ सर्वान् देवान् &      \\
        \hline
            1740 & TB\_3.12.8.2 & अथर्वाङ्गिरसश्च ये सर्वास्ताः इतिहासपुराणञ्च &      \\
        \hline
            1741 & TB\_3.12.8.3 & सर्वास्ताः अहोरात्राणि सर्वाणि अद्र्धमासाꣳश्च &      \\
        \hline
            1742 & TB\_3.12.9.1 & ऋचां प्राची महती दिगुच्यते &      \\
        \hline
            1743 & TB\_3.12.9.2 & सर्वं तेज स्सामरूप्यꣳ ह &      \\
        \hline
            1744 & TB\_3.12.9.3 & तप आसीद् गृहपतिः ब्रह्म &      \\
        \hline
            1745 & TB\_3.12.9.4 & अपानो विद्वा{-}नावृतः प्रति प्रातिष्ठ{-}दद्ध्वरे &      \\
        \hline
            1746 & TB\_3.12.9.5 & ऊर्ग्{-}राजान{-}मुदवहत् ध्रुव गोपः सहो{-}ऽभवत् &      \\
        \hline
            1747 & TB\_3.12.9.6 & इद्ध्मꣳ ह क्षुच् चैभ्य &      \\
        \hline
            1748 & TB\_3.12.9.7 & विश्वसृजः प्रथमाः सत्र{-}मासत सहस्र &      \\
        \hline
            1749 & TB\_3.12.9.8 & तस्यैवात्मा पदवित्तं ॅविदित्वा न &      \\
        \hline
        \bottomrule
  \end{longtable}
  
\end{document}