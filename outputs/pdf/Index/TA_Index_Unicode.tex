\documentclass[17pt]{extarticle}
\usepackage{babel}
\usepackage{fontspec}
\usepackage{polyglossia}
\usepackage{extsizes}

\usepackage{booktabs} % To thicken table lines



\setmainlanguage{sanskrit}
\setotherlanguages{english} %% or other languages
\setlength{\parindent}{0pt}
\pagestyle{myheadings}
\newfontfamily\devanagarifont[Script=Devanagari]{AdishilaVedic}


\newcommand{\VAR}[1]{}
\newcommand{\BLOCK}[1]{}

\usepackage{longtable} % To display tables on several pages

\begin{document} 


\begin{longtable}{||p{0.4in}||p{0.9in}||p{4.0in}||p{0.9in}||} % <-- Replaces \begin{table}, alignment must be specified here (no more tabular)
    \caption{ कृष्ण यजुर्वेदीय तैत्तिरीय आरण्यके}
    \label{tab:table1}\\
    \toprule
    \textbf{SNo} &  \textbf{Dasini} & \textbf{Beginning Words} & \textbf{Other Dasini beginning with same Words}
    
   
    \endfirsthead % <-- This denotes the end of the header, which will be shown on the first page only
    \toprule
    \textbf{SNo} & \textbf{Dasini} & \textbf{Beginning Words} & \textbf{Other Dasini beginning with same Words}
    
   
    \endhead % <-- Everything between \endfirsthead and \endhead will be shown as a header on every page
            1 & T.A.1.1.1 & भद्रं कर्णेभिः शृणुयाम देवाः &      \\
        \hline
            2 & T.A.1.1.2 & अग्निर्वायुश्च सूर्यश्च सह सञ्च{-}स्करर्द्धिया &      \\
        \hline
            3 & T.A.1.1.3 & अपाश्न्युष्णि{-}मपा रक्षः अपाश्न्युष्णि{-}मपा रघं &      \\
        \hline
            4 & T.A.1.2.1 & स्मृतिः प्रत्यक्ष{-}मैतिह्यं अनुमान{-}श्चतुष्टयं एतैरादित्य &      \\
        \hline
            5 & T.A.1.2.2 & तान्नद्योऽभि समायन्ति सोरुः सती &      \\
        \hline
            6 & T.A.1.2.3 & अणुभिश्च महद्भिश्च समारूढः प्रदृश्यते &      \\
        \hline
            7 & T.A.1.2.4 & उभयतः सप्तेन्द्रियाणि जल्पितं त्वेव &      \\
        \hline
            8 & T.A.1.3.1 & साकञ्जानाꣳ सप्तथमाहु{-}रेकजं षडुद्यमा ऋषयो &      \\
        \hline
            9 & T.A.1.3.2 & न हि प्रवेद सुकृतस्य &      \\
        \hline
            10 & T.A.1.3.3 & अमूꣳश्च परिरक्षतः एता वाचः &      \\
        \hline
            11 & T.A.1.3.4 & ज्योतिषा ऽप्रतिख्येन सः विश्वरूपाणि &      \\
        \hline
            12 & T.A.1.4.1 & अक्षिदुःखोत्थितस्यैव विप्रसन्ने कनीनिके आङ्क्ते &      \\
        \hline
            13 & T.A.1.4.2 & अभिधून्वन्तो{-}ऽभिघ्नन्त इव वातवन्तो मरुद्गणाः &      \\
        \hline
            14 & T.A.1.4.3 & दुर्भिक्षं देवलोकेषु मनूनामुदकं गृहे &      \\
        \hline
            15 & T.A.1.5.1 & अति ताम्राणि वासाꣳसि अष्टिवज्रि &      \\
        \hline
            16 & T.A.1.5.2 & तस्येन्द्रो वम्रिरूपेण धनुर्ज्या{-}मछिनथ्स्वयं तदिन्द्रधनुरित्यज्यं &      \\
        \hline
            17 & T.A.1.6.1 & अत्यूर्द्ध्वाक्षोऽतिरश्चात् शिशिरः प्रदृश्यते नैव &      \\
        \hline
            18 & T.A.1.6.2 & निजानुकामे न्यञ्जलिका अमी वाच{-}मुपासतामिति &      \\
        \hline
            19 & T.A.1.6.3 & अव द्रफ्सो अꣳशमतीमतिष्ठत् इयानः &      \\
        \hline
            20 & T.A.1.7.1 & आरोगो भ्राजः पटरः पतङ्गः &      \\
        \hline
            21 & T.A.1.7.2 & तस्मिन् राजान{-}मधिविश्रयेममिति ते अस्मै &      \\
        \hline
            22 & T.A.1.7.3 & आनुश्रविक एव नौ कश्यप &      \\
        \hline
            23 & T.A.1.7.4 & अमुत्रेतरे तस्मादिहा तप्त्रि तपाः &      \\
        \hline
            24 & T.A.1.7.5 & सप्त होतार ऋत्विजः देवा &      \\
        \hline
            25 & T.A.1.7.6 & अनु न जातमष्ट रोदसी &      \\
        \hline
            26 & T.A.1.8.1 & क्वेदमभ्रं निविशते क्वायꣳ सम्ॅवथ्सरो &      \\
        \hline
            27 & T.A.1.8.2 & अभ्राण्यपः प्रपद्यन्ते विद्युथ्सूर्ये समाहिता &      \\
        \hline
            28 & T.A.1.8.3 & व्यष्टभ्ना{-}द्रोदसी विष्णवेते दाधर्थ पृथिवी{-}मभितो &      \\
        \hline
            29 & T.A.1.8.4 & अग्नयो वायवश्चैव एतदस्य परायणं &      \\
        \hline
            30 & T.A.1.8.5 & अनाभोगाः परं मृत्युं पापाः &      \\
        \hline
            31 & T.A.1.8.6 & कश्यपा दुदिताः सूर्याः पापान्निर्घ्नन्ति &      \\
        \hline
            32 & T.A.1.8.7 & आशातिकाः क्रिमय इव ततः &      \\
        \hline
            33 & T.A.1.8.8 & ऋषिर्.ह दीर्घश्रुत्तमः इन्द्रस्य घर्मो &      \\
        \hline
            34 & T.A.1.9.1 & अग्निश्च जातवेदाश्च सहोजा अजिराप्रभुः &      \\
        \hline
            35 & T.A.1.9.2 & याश्च वासुकि वैद्युताः रजताः &      \\
        \hline
            36 & T.A.1.9.3 & य एवं ॅवेद अथ &      \\
        \hline
            37 & T.A.1.9.4 & नैनं गरो हिनस्ति य &      \\
        \hline
            38 & T.A.1.9.5 & विद्युन् महसो धूपयः श्वापयो &      \\
        \hline
            39 & T.A.1.9.6 & उच्चैत्यव चाहभिः भूमिं पर्जन्या &      \\
        \hline
            40 & T.A.1.9.7 & ब्रह्मणा वीर्यावता शिवा नः &      \\
        \hline
            41 & T.A.1.10.1 & सहस्रवृदियं भूमिः परं ॅव्योम &      \\
        \hline
            42 & T.A.1.10.2 & विश्वा हि माया अवथः &      \\
        \hline
            43 & T.A.1.10.3 & तिस्रः क्षपस्त्रिरहा ऽतिव्रजद्भिः नासत्या &      \\
        \hline
            44 & T.A.1.10.4 & अन्वेति तुग्रो वक्रियान्तं आयसूयान्थ् &      \\
        \hline
            45 & T.A.1.10.5 & तयो रेतौ वथ्सा वहोरात्रे &      \\
        \hline
            46 & T.A.1.10.6 & ताम्रो अरुणः ता अविसृष्टौ &      \\
        \hline
            47 & T.A.1.10.7 & उष्मा च नीहारश्च वृत्रस्योष्मा &      \\
        \hline
            48 & T.A.1.11.1 & पवित्रवन्तः परिवाजमासते पितैषां प्रत्नो &      \\
        \hline
            49 & T.A.1.11.2 & ऋषयः सप्तात्रिश्च यत् सर्वेऽत्रयो &      \\
        \hline
            50 & T.A.1.11.3 & धियो यो नः प्रचोदयात् &      \\
        \hline
            51 & T.A.1.11.4 & नपुꣳसकं पुमाꣳस्त्र्यस्मि स्थावरोऽस्म्यथ जङ्गमः &      \\
        \hline
            52 & T.A.1.11.5 & यस्ता विजानाथ् सवितुः पिताऽसत् &      \\
        \hline
            53 & T.A.1.11.6 & वीणा पण वलासितं मृतञ्जीवञ्च &      \\
        \hline
            54 & T.A.1.11.7 & सोऽजिह्वो असश्चत नैतमृषिं ॅविदित्वा &      \\
        \hline
            55 & T.A.1.11.8 & नास्याक्षो यातु सज्जति यच्छ्वेतान् &      \\
        \hline
            56 & T.A.1.12.1 & आतनुष्व प्रतनुष्व उद्धमाधम सन्धम &      \\
        \hline
            57 & T.A.1.12.2 & एवमेतन्निबोधत आ मन्द्रै{-}रिन्द्र हरिभिः &      \\
        \hline
            58 & T.A.1.12.3 & निघृष्वै रसमायुतैः कालैर्. हरित्वमापन्नैः &      \\
        \hline
            59 & T.A.1.12.4 & गौरावस्कन्दिन्न{-}हल्यायै जार कौशिक{-}ब्राह्मण गौतमब्रुवाण &      \\
        \hline
            60 & T.A.1.12.5 & उक्तꣳ स्थानं प्रमाणञ्च पुर &      \\
        \hline
            61 & T.A.1.13.1 & अष्टयोनी{-}मष्टपुत्रां अष्टपत्नी{-}मिमां महीं अहं &      \\
        \hline
            62 & T.A.1.13.2 & अहं ॅवेद न मे &      \\
        \hline
            63 & T.A.1.13.3 & परा मार्ताण्डमास्यत् सप्तभिः पुत्रै{-}रदितिः &      \\
        \hline
            64 & T.A.1.14.1 & योऽसौ तपन्नुदेति स सर्वेषां &      \\
        \hline
            65 & T.A.1.14.2 & मा मे प्रजाया मा &  T.A.1.14.4       \\
        \hline
            66 & T.A.1.14.3 & इमे मासा{-}श्चार्द्धमासाश्च सर्वेषां भूतानां &      \\
        \hline
            67 & T.A.1.14.4 & मा मे प्रजाया मा & T.A.1.14.2        \\
        \hline
            68 & T.A.1.15.1 & अथादित्यस्याष्ट पुरुषस्य वसूना मादित्यानां &      \\
        \hline
            69 & T.A.1.16.1 & आरोगस्य स्थाने स्वतेजसा भानि &      \\
        \hline
            70 & T.A.1.17.1 & अथ वायो{-}रेकादश{-}पुरुषस्यैकादश{-}स्त्रीकस्य प्रभ्राजमानानां रुद्राणां &      \\
        \hline
            71 & T.A.1.17.2 & अवपतन्तानाꣳ रुद्राणाꣳ स्थाने स्वतेजसा &      \\
        \hline
            72 & T.A.1.18.1 & अथाग्नेरष्ट पुरुषस्य अग्नेः पूर्व{-}दिश्यस्य &      \\
        \hline
            73 & T.A.1.19.1 & दक्षिणपूर्व{-}स्यान्दिशि विसर्पी नरकः तस्मान्नः &      \\
        \hline
            74 & T.A.1.20.1 & इन्द्र घोषा वो वसुभिः &      \\
        \hline
            75 & T.A.1.21.1 & आपमापामपः सर्वाः अस्मा{-}दस्मादितोऽमुतः अग्निर्वायुश्च &      \\
        \hline
            76 & T.A.1.21.2 & देवीः पर्जन्य सूवरीः पुत्रवत्त्वाय &      \\
        \hline
            77 &  &  &          \\
        \hline
            78 & T.A.1.22.1 & योऽपां पुष्पं ॅवेद पुष्पवान् &      \\
        \hline
            79 & T.A.1.22.2 & आयतनवान् भवति आपो वा &      \\
        \hline
            80 & T.A.1.22.3 & आपो वै वायोरायतनं आयतनवान् &      \\
        \hline
            81 & T.A.1.22.4 & आयतनवान् भवति य एवं &      \\
        \hline
            82 & T.A.1.22.5 & य एवं ॅवेद योऽपामायतनं &      \\
        \hline
            83 & T.A.1.22.6 & योऽपामायतनं ॅवेद आयतनवान् भवति &      \\
        \hline
            84 & T.A.1.22.7 & आयतनवान् भवति सम्ॅवथ्सरो वा &      \\
        \hline
            85 & T.A.1.22.8 & इमे वै लोका अफ्सु &      \\
        \hline
            86 & T.A.1.22.9 & पुष्करपर्णैः पुष्करदण्डैः पुष्करैश्च सꣳस्तीर्य &      \\
        \hline
            87 & T.A.1.22.10 & अथो आहुः सर्वेषु यज्ञ्क्रतुष्विति &      \\
        \hline
            88 & T.A.1.22.11 & नाचिकेत{-}मग्निञ्चिन्वानः प्राणान् प्रत्यक्षेण कमग्निञ्चिनुते &      \\
        \hline
            89 & T.A.1.22.12 & इमान् ॅलोकान् प्रत्यक्षेण कमग्निञ्चिनुते &      \\
        \hline
            90 & T.A.1.23.1 & आपो वा इदमासन्थ् सलिलमेव &      \\
        \hline
            91 & T.A.1.23.2 & सतो बन्धुमसति निरविन्दन्न् हृदि &      \\
        \hline
            92 & T.A.1.23.3 & ये नखाः ते वैखानसाः &      \\
        \hline
            93 & T.A.1.23.4 & नेत्यब्रवीत् पूर्वमेवाह{-}मिहासमिति तत्पुरुषस्य पुरुषत्वं &      \\
        \hline
            94 & T.A.1.23.5 & अञ्जलिना पुरस्ता{-}दुपादधात् एवा ह्येवेति &      \\
        \hline
            95 & T.A.1.23.6 & ततो वायुरुदतिष्ठत् सा प्रतीची &      \\
        \hline
            96 & T.A.1.23.7 & अथारुणः केतुरुपरिष्टा{-}दुपादधात् एवा हि &      \\
        \hline
            97 & T.A.1.23.8 & आपो ह यद् बृहतीः &      \\
        \hline
            98 & T.A.1.23.9 & विधाय लोकान्. विधाय भूतानि &      \\
        \hline
            99 & T.A.1.24.1 & चतुष्टय्य आपो गृह्णाति चत्वारि &      \\
        \hline
            100 & T.A.1.24.2 & कूप्या गृह्णाति ता दक्षिणत &      \\
        \hline
            101 & T.A.1.24.3 & ता उत्तरत उपदधाति ओजसा &      \\
        \hline
            102 & T.A.1.24.4 & असौ वै पल्वल्याः अमुष्यामेव &      \\
        \hline
            103 & T.A.1.25.1 & जानुदघ्नी{-}मुत्तरवेदीङ्खात्वा अपां पूरयति अपां &      \\
        \hline
            104 & T.A.1.25.2 & तदवरुन्धे कूर्ममुपदधाति अपामेव मेधमवरुन्धे &      \\
        \hline
            105 & T.A.1.25.3 & यास्तिस्रः परमजाः इन्द्रघोषा वो &      \\
        \hline
            106 & T.A.1.26.1 & अग्निं प्रणीयोप{-}समाधाय तमभित एता &      \\
        \hline
            107 & T.A.1.26.2 & सत्रिय मग्निञ्चिन्वानः कमग्निञ्चिनुते सावित्र &      \\
        \hline
            108 & T.A.1.26.3 & उपानुवाक्य{-}माशु मग्निञ्चिन्वानः कमग्निञ्चिनुते इममारुण{-}केतुक &      \\
        \hline
            109 & T.A.1.26.4 & प्राजापत्यो वा एषोऽग्निः प्राजापत्याः &      \\
        \hline
            110 & T.A.1.26.5 & वृष्टिकामश्चिन्वीत आपो वै वृष्टिः &      \\
        \hline
            111 & T.A.1.26.6 & वज्रमेव भ्रातृव्येभ्यः प्रहरति स्तृणुत &      \\
        \hline
            112 & T.A.1.26.7 & अमृतं ॅवा आपः अमृतस्या{-}नन्तरित्यै &      \\
        \hline
            113 & T.A.1.27.1 & इमा नुकं भुवना सीषधेम &      \\
        \hline
            114 & T.A.1.27.2 & मरीचयः स्वायंभुवाः ये शरीराण्य &      \\
        \hline
            115 & T.A.1.27.3 & देवानां पूरयोद्ध्या तस्याꣳ हिरण्मयः &      \\
        \hline
            116 & T.A.1.27.4 & विवेशापराजिता पराङ्गेत्य पराङत्य ज्यामयी &      \\
        \hline
            117 & T.A.1.27.5 & यज्वानो येऽप्ययज्वनः स्वर्यन्तो नापेक्षन्ते &      \\
        \hline
            118 & T.A.1.27.6 & यमो ददात्व{-}वसानमस्मै नृ मुणन्तु &      \\
        \hline
            119 & T.A.1.28.1 & विशीर्ष्णीं गृद्ध्र{-}शीर्ष्णीञ्च अपेतो निर्.ऋतिं &      \\
        \hline
            120 & T.A.1.29.1 & पर्जन्याय प्रगायत दिवस्पुत्राय मीढुषे &      \\
        \hline
            121 & T.A.1.30.1 & पुनःमामैत्विन्द्रियं पुनरायुः पुनःभगः पुनः &      \\
        \hline
            122 & T.A.1.31.1 & अद्भय{-}स्तिरोधा जायत तव वैश्रवणः &      \\
        \hline
            123 & T.A.1.31.2 & असाम सुमतौ यज्ञियस्य श्रियं &      \\
        \hline
            124 & T.A.1.31.3 & सर्वभूताधिपतये नम इति अथ &      \\
        \hline
            125 & T.A.1.31.4 & तिरोधाः स्वः तिरोधा भूर्भुवस्स्वः &      \\
        \hline
            126 & T.A.1.31.5 & अपि ब्राह्मणमुखीनाः तस्मिन्नह्नः काले &      \\
        \hline
            127 & T.A.1.31.6 & तस्मा इममग्र पिण्डञ्जुहोमि स &      \\
        \hline
            128 & T.A.1.32.1 & सम्ॅवथ्सरमेतद् व्रतञ्चरेत् द्वौ वा &      \\
        \hline
            129 & T.A.1.32.2 & असञ्चयवान् अग्नये वायवे सूर्याय &      \\
        \hline
            130 & T.A.1.32.3 & महानाम्नीभि{-}रुदकꣳ सꣳस्पर्श्य तमाचार्यो दद्यात् &      \\
        \hline
            131 & T.A.2.1.1 & सह वै देवानां चासुराणां &      \\
        \hline
            132 & T.A.2.2.1 & रक्षाꣳसि हवा पुरोनुवाके तपोऽग्रमतिष्ठन्त &      \\
        \hline
            133 & T.A.2.3.1 & यद्{-}देवा देवहेडनं देवहेळनं देवासश्चकृमा &      \\
        \hline
            134 & T.A.2.4.1 & यददीव्यन्नृणमहं बभूवादिथ् सन्वा संजगर &      \\
        \hline
            135 & T.A.2.5.1 & आयुष्टे विश्वतो दधदयमग्निर्वरेण्यः पुनस्ते &      \\
        \hline
            136 & T.A.2.6.1 & वैश्वानराय प्रतिवेदयामो यदीनृणꣳ सङ्गरो &      \\
        \hline
            137 & T.A.2.7.1 & वातरशना ह वा ऋषयः &      \\
        \hline
            138 & T.A.2.8.1 & कूश्माण्डैर् जुहुयाद्{-}योऽपूत इव मन्येत &      \\
        \hline
            139 & T.A.2.9.1 & अजान्. ह वै पृश्नीꣳस्तपस्यमानान् &      \\
        \hline
            140 & T.A.2.10.1 & पञ्च वा एते महायज्ञाः &      \\
        \hline
            141 & T.A.2.11.1 & ब्रह्मयज्ञेन यक्ष्यमाणः प्राच्यां दिशि &      \\
        \hline
            142 & T.A.2.12.1 & ग्रामे मनसा स्वाद्ध्यायमधीयीत दिवा &      \\
        \hline
            143 & T.A.2.13.1 & मद्ध्यन्दिने प्रबलमधीयीतासौ खलु वावैष &      \\
        \hline
            144 & T.A.2.14.1 & तस्य वा एतस्य यज्ञ्स्य &  T.A.2.15.1       \\
        \hline
            145 & T.A.2.15.1 & तस्य वा एतस्य यज्ञ्स्य & T.A.2.14.1        \\
        \hline
            146 & T.A.2.16.1 & रिच्यत इव वा एष &      \\
        \hline
            147 & T.A.2.17.1 & दुहे ह वा एष &      \\
        \hline
            148 & T.A.2.18.1 & कतिधाऽवकीर्णी प्रविशति चतुर्द्धेत्याहुर्{-}ब्रह्मवादिनो मरुतः &      \\
        \hline
            149 & T.A.2.19.1 & भूः प्रपद्ये भुवः प्रपद्ये &      \\
        \hline
            150 & T.A.2.20.1 & नमः प्राच्यै दिशे याश्च &      \\
        \hline
            151 & T.A.3.1.1 & चित्तिः स्रुक् चित्तमाज्यम् वाग्वेदिः &      \\
        \hline
            152 & T.A.3.2.1 & पृथिवी होता द्यौरद्ध्वर्युः रुद्रोऽग्नीत् &      \\
        \hline
            153 & T.A.3.3.1 & अग्निर्. होता अश्विनाऽद्ध्वर्यू त्वष्टाऽग्नीत् &      \\
        \hline
            154 & T.A.3.4.1 & सूर्यं ते चक्षुः वातं &      \\
        \hline
            155 & T.A.3.5.1 & महाहविर्. होता सत्यहविरद्ध्वर्युः अच्युतपाजा &      \\
        \hline
            156 & T.A.3.6.1 & वाग्घोता दीक्षा पत्नी वातोऽद्ध्वर्युः &      \\
        \hline
            157 & T.A.3.7.1 & ब्राह्मण एकहोता स यज्ञ्ः &      \\
        \hline
            158 & T.A.3.7.2 & स मे ददातु प्रजां &  T.A.3.7.4       \\
        \hline
            159 & T.A.3.7.3 & चन्द्रमाः षड्ढोता स ऋतून् &      \\
        \hline
            160 & T.A.3.7.4 & स मे ददातु प्रजां & T.A.3.7.2        \\
        \hline
            161 & T.A.3.8.1 & अग्निर् यजुर्भिः सविता स्तोमैः &      \\
        \hline
            162 & T.A.3.8.2 & त्वष्टेद्ध्मेन विष्णुर् यज्ञेन वसव &      \\
        \hline
            163 & T.A.3.9.1 & सेनेन्द्रस्य धेना बृहस्पतेः पत्थ्या &      \\
        \hline
            164 & T.A.3.9.2 & वरुणस्य विराट् यज्ञ्स्य पङ्क्तिः &      \\
        \hline
            165 & T.A.3.10.1 & देवस्य त्वा सवितुः प्रसवे &  T.A.7.8.1       \\
        \hline
            166 & T.A.3.10.2 & कामः प्रतिग्रहीता कामꣳ समुद्रमाविश &      \\
        \hline
            167 & T.A.3.10.3 & मनवे तल्पम् त्वष्ट्रेऽजाम् पूष्णेऽविम् &      \\
        \hline
            168 & T.A.3.10.4 & उत्तानायाङ्गीरसायानः वैश्वानराय रथम् वैश्वानरः &      \\
        \hline
            169 & T.A.3.10.5 & क इदं कस्मा अदात् &      \\
        \hline
            170 & T.A.3.11.1 & सुवर्णं घर्मं परिवेद वेनम् &      \\
        \hline
            171 & T.A.3.11.2 & अन्तः प्रविष्टः शास्ता जनानां &      \\
        \hline
            172 & T.A.3.11.3 & अमृतस्य प्राणं ॅयज्ञ्मेतम् चतुर्.होतृणामात्मानं &      \\
        \hline
            173 & T.A.3.11.4 & इन्द्रꣳ राजानꣳ सवितारमेतम् वायोरात्मानं &      \\
        \hline
            174 & T.A.3.11.5 & अमृतस्य पूर्णान्तामु कलां ॅविचक्षते &      \\
        \hline
            175 & T.A.3.11.6 & इन्द्रो राजा जगतो य &      \\
        \hline
            176 & T.A.3.11.7 & अच्युतां बहुलाꣳ श्रियम् स &      \\
        \hline
            177 & T.A.3.11.8 & घृतं तेजो मधुमदिन्द्रियम् मय्ययमग्निर्दधातु &      \\
        \hline
            178 & T.A.3.11.9 & एको अश्वो वहति सप्तनामा &      \\
        \hline
            179 & T.A.3.11.10 & रोहिणीः पिङ्गला एकरूपाः क्षरन्तीः &      \\
        \hline
            180 & T.A.3.11.11 & हृदा पश्यन्ति मनसा मनीषिणः &      \\
        \hline
            181 & T.A.3.11.12 & प्रजापतिः प्रजया सम्ॅविदानः वीतं &      \\
        \hline
            182 & T.A.3.11.13 & वायुस्ताꣳ अग्रे प्रमुमोक्तु देवः &      \\
        \hline
            183 & T.A.3.12.1 & सहस्रशीर्.षा पुरुषः सहस्राक्षः सहस्रपात् &      \\
        \hline
            184 & T.A.3.12.2 & पादोऽस्य विश्वा भूतानि त्रिपादस्यामृतं &      \\
        \hline
            185 & T.A.3.12.3 & यत्पुरुषेण हविषा देवा यज्ञ्मतन्वत &      \\
        \hline
            186 & T.A.3.12.4 & तेन देवा अयजन्त साद्ध्या &      \\
        \hline
            187 & T.A.3.12.5 & तस्मादश्वा अजायन्त ये के &      \\
        \hline
            188 & T.A.3.12.6 & ऊरू तदस्य यद्{-}वैश्यः पद्भ्यां &      \\
        \hline
            189 & T.A.3.12.7 & वेदाहमेतं पुरुषं महान्तम् आदित्यवर्णं &      \\
        \hline
            190 & T.A.3.13.1 & अद्भ्यः सम्भूतः पृथिव्यै रसाच्च &      \\
        \hline
            191 & T.A.3.13.2 & तस्य धीराः परिजानन्ति योनिम् &      \\
        \hline
            192 & T.A.3.14.1 & भर्ता सन् भ्रियमाणो बिभर्ति &      \\
        \hline
            193 & T.A.3.14.2 & उतो बहूनेकमहर्जहार अतन्द्रो देवः &      \\
        \hline
            194 & T.A.3.14.3 & त्वं ॅयज्ञ्स्त्वमु वेवासि सोमः &      \\
        \hline
            195 & T.A.3.14.4 & तं मे देवा ब्रह्मणा &      \\
        \hline
            196 & T.A.3.15.1 & हरिꣳ हरन्तमनुयन्ति देवाः विश्वस्येशानं &      \\
        \hline
            197 & T.A.3.15.2 & प्रास्मा आशा अशृण्वन्न् कामेनाजनयन् &      \\
        \hline
            198 & T.A.3.16.1 & तरणिर् विश्वदर्.शतो ज्योतिष्कृदसि सूर्य &      \\
        \hline
            199 & T.A.3.17.1 & आ प्यायस्व मदिन्तम सोम &      \\
        \hline
            200 & T.A.3.18.1 & ईयुष्टे ये पूर्वतरामपश्यन् व्युच्छन्तीमुषसं &      \\
        \hline
            201 & T.A.3.19.1 & ज्योतिष्मतीं त्वा सादयामि ज्योतिष्कृतं &      \\
        \hline
            202 & T.A.3.20.1 & प्रयासाय स्वाहाऽऽयासाय स्वाहा वियासाय &      \\
        \hline
            203 & T.A.3.21.1 & चित्तꣳ सन्तानेन भवं ॅयक्ना &      \\
        \hline
            204 & T.A.4.1.1 & परेयुवाꣳसम् प्रवतो महीरनु बहुभ्यः &      \\
        \hline
            205 & T.A.4.1.2 & आयुर्{-}विश्वायुः परिपासति त्वा पूषा &      \\
        \hline
            206 & T.A.4.1.3 & इयम् नारी पतिलोकम् ॅवृणाना &      \\
        \hline
            207 & T.A.4.1.4 & इममग्ने चमसम् मा विजीह्वरः &      \\
        \hline
            208 & T.A.4.2.1 & य एतस्य पथो गोप्तारस्तेभ्यः &      \\
        \hline
            209 & T.A.4.3.1 & प्र केतुना बृहता भात्यग्निराविर्{-}विश्वानि &      \\
        \hline
            210 & T.A.4.3.2 & उरुणसा{-}वसुतृपा{-}वुलुम्बलौ यमस्य दूतौ चरतो &      \\
        \hline
            211 & T.A.4.3.3 & यद्वै देवस्य सवितुः पवित्रं &      \\
        \hline
            212 & T.A.4.4.1 & यम् ते अग्निममन्थाम वृषभायेव &      \\
        \hline
            213 & T.A.4.4.2 & अव सृज पुनरग्ने पितृभ्यो &      \\
        \hline
            214 & T.A.4.5.1 & आयातु देवः सुमनाभि{-}रूतिभिर्{-}यमो हवेह &      \\
        \hline
            215 & T.A.4.5.2 & योऽस्य कौष्ठ्य जगतः पार्थिवस्यैक &      \\
        \hline
            216 & T.A.4.5.3 & त्रिकद्रुकेभिः पतति षडुर्वी{-}रेकमिद्{-}बृहत् गायत्री &      \\
        \hline
            217 & T.A.4.6.1 & वैश्वानरे हविरिदम् जुहोमि साहस्र{-}मुथ्सं &      \\
        \hline
            218 & T.A.4.6.2 & शुनम् ॅवाहाः शुनम् नाराः &      \\
        \hline
            219 & T.A.4.6.3 & प्र वाता वान्ति पतयन्ति &      \\
        \hline
            220 & T.A.4.7.1 & उत्ते तभ्नोमि पृथिवीम् त्वत्परीमम् &      \\
        \hline
            221 & T.A.4.7.2 & एषा ते यमसादने स्वधा &      \\
        \hline
            222 & T.A.4.7.3 & सवितैतानि शरीराणि पृथिव्यै मातुरुपस्थ &      \\
        \hline
            223 & T.A.4.8.1 & अपूपवान् घृतवाꣳश्चरुरेह सीदतूत्तभ्नुवन् पृथिवीम् &      \\
        \hline
            224 & T.A.4.9.1 & एतास्ते स्वधा अमृताः करोमि &      \\
        \hline
            225 & T.A.4.9.2 & लोकम् पृण ता अस्य &      \\
        \hline
            226 & T.A.4.10.1 & आरोहतायुर्{-}जरसम् गृणाना अनुपूर्वम् ॅयतमाना &      \\
        \hline
            227 & T.A.4.10.2 & इमे जीवा विमृतैराववर्{-}तिन्नभूद्{-}भद्रा देवहूतिम् &      \\
        \hline
            228 & T.A.4.11.1 & अप नः शोशुचदघमग्ने शुशुद्ध्या &      \\
        \hline
            229 & T.A.4.11.2 & त्वꣳ हि विश्वतोमुख विश्वतः &      \\
        \hline
            230 & T.A.4.12.1 & अपश्याम युवति{-}माचरन्तीम् मृताय जीवाम् &      \\
        \hline
            231 & T.A.5.1.1 & शन्नो मित्रः शं ॅवरुणः &  T.A.5.12.1       \\
        \hline
            232 & T.A.5.2.1 & शीक्षां ॅव्याख्यास्यामः वर्णः स्वरः &      \\
        \hline
            233 & T.A.5.3.1 & सह नौ यशः सह &      \\
        \hline
            234 & T.A.5.3.2 & वायुः सन्धानं इत्यधि लोकं &      \\
        \hline
            235 & T.A.5.3.3 & अन्तेवास्युत्तररूपं विद्या सन्धिः प्रवचनं &      \\
        \hline
            236 & T.A.5.3.4 & अथाद्ध्यात्मं अधराहनुः पूर्व रूपं &      \\
        \hline
            237 & T.A.5.4.1 & य श्छन्दसा{-}मृषभो विश्वरूपः छन्दोभ्यो{-}ऽद्ध्यमृताथ् &      \\
        \hline
            238 & T.A.5.4.2 & कुर्वाणा चीर{-}मात्मनः वासाꣳसि मम &      \\
        \hline
            239 & T.A.5.4.3 & यशो जनेऽसानि स्वाहा श्रेयान. &      \\
        \hline
            240 & T.A.5.5.1 & भू{-}र्भुव{-}स्सुव{-}रिति वा एता स्तिस्रो &      \\
        \hline
            241 & T.A.5.5.2 & मह इत्यादित्यः आदित्येन वाव &      \\
        \hline
            242 & T.A.5.5.3 & मह इति ब्रह्म ब्रह्मणा &      \\
        \hline
            243 & T.A.5.6.1 & स य एषोऽन्तर्.{-}हृदय आकाशः &      \\
        \hline
            244 & T.A.5.6.2 & सुवरित्यादित्ये मह इति ब्रह्मणि &      \\
        \hline
            245 & T.A.5.7.1 & पृथिव्यन्तरिक्षं द्यौर्{-}दिशोऽवान्तरदिशाः अग्निर्{-}वायु{-}रादित्य{-}श्चन्द्रमा नक्षत्राणि &      \\
        \hline
            246 & T.A.5.8.1 & ओ{-}मिति ब्रह्म ओ{-}मितीदꣳ सर्वं &      \\
        \hline
            247 & T.A.5.9.1 & ऋतं च स्वाद्ध्याय प्रवचने &      \\
        \hline
            248 & T.A.5.10.1 & अहं ॅवृक्षस्य रेरिवा कीर्तिः &      \\
        \hline
            249 & T.A.5.11.1 & वेदमनूच्या{-} ऽचार्योन्तेवासिन{-}मनुशास्ति सत्यं ॅवद &      \\
        \hline
            250 & T.A.5.11.2 & देवपितृकार्याभ्यां न प्रमदितव्यं मातृदेवो &      \\
        \hline
            251 & T.A.5.11.3 & नो इतराणि ये के &      \\
        \hline
            252 & T.A.5.11.4 & ये तत्र ब्राह्मणाः सम्मर.शिनः &      \\
        \hline
            253 & T.A.5.11.5 & तथा तेषु वर्तेथाः एष &      \\
        \hline
            254 & T.A.5.12.1 & शन्नो मित्रः शं ॅवरुणः & T.A.5.1.1        \\
        \hline
            255 & T.A.5.13.1 & सह ना ववतु सह &      \\
        \hline
            256 & T.A.5.14.1 & ब्रह्मविदाप्नोति परं तदेषाऽभ्युक्ता सत्यं &      \\
        \hline
            257 & T.A.5.14.2 & अन्नाद्वै प्रजाः प्रजायन्ते याः &      \\
        \hline
            258 & T.A.5.14.3 & प्राणं देवा अनुप्राणन्ति मनुष्याः &      \\
        \hline
            259 & T.A.5.14.4 & यतो वाचो निवर्तन्ते अप्राप्य &  T.A.5.14.9       \\
        \hline
            260 & T.A.5.14.5 & विज्ञानं ॅयज्ञ्ं तनुते कर्माणि &      \\
        \hline
            261 & T.A.5.14.6 & असन्नेव स भवति असद् &      \\
        \hline
            262 & T.A.5.14.7 & असद्वा इदमग्र आसीत् ततो &      \\
        \hline
            263 & T.A.5.14.8 & भीषाऽस्मा{-}द्वातः पवते भीषोदेति सूर्यः &      \\
        \hline
            264 & T.A.5.14.9 & यतो वाचो निवर्तन्ते अप्राप्य & T.A.5.14.4        \\
        \hline
            265 & T.A.5.15.1 & भृगुर्वै वारुणिः वरुणं पितर{-}मुपससार &      \\
        \hline
            266 & T.A.5.15.2 & अन्नं ब्रह्मेति व्यजानात् अन्नाद्ध्येव &      \\
        \hline
            267 & T.A.5.15.3 & प्राणो ब्रह्मेति व्यजानात् प्राणाद्ध्येव &      \\
        \hline
            268 & T.A.5.15.4 & मनो ब्रह्मेति व्यजानात् मनसो &      \\
        \hline
            269 & T.A.5.15.5 & विज्ञानं ब्रह्मेति व्यजानात् विज्ञाना{-}द्ध्येव &      \\
        \hline
            270 & T.A.5.15.6 & आनन्दो ब्रह्मेति व्यजानात् आनन्दा{-}द्ध्येव{-}खल्विमानि &      \\
        \hline
            271 & T.A.5.15.7 & अन्नं न निन्द्यात् तद् &      \\
        \hline
            272 & T.A.5.15.8 & अन्नं न परिचक्षीत तद् &      \\
        \hline
            273 & T.A.5.15.9 & अन्नं बहु कुर्वीत तद् &      \\
        \hline
            274 & T.A.6.1.1 & अंभस्य पारे भुवनस्य मद्ध्ये &      \\
        \hline
            275 & T.A.6.1.2 & तदेवर्त्तं तदु सत्यमाहु{-}स्तदेव ब्रह्म &      \\
        \hline
            276 & T.A.6.1.3 & न सदृंशे तिष्ठति रूपमस्य &      \\
        \hline
            277 & T.A.6.1.4 & त्रीणि पदा निहिता गुहासु &      \\
        \hline
            278 & T.A.6.1.5 & पशूꣳश्च मह्यमावह जीवनञ्च दिशो &      \\
        \hline
            279 & T.A.6.1.6 & तन्नो नन्दिः प्रचोदयात् तत्पुरुषाय &      \\
        \hline
            280 & T.A.6.1.7 & तन्नो नारसिꣳहः प्रचोदयात् भास्कराय &      \\
        \hline
            281 & T.A.6.1.8 & एवा नो दूर्वे प्रतनु &      \\
        \hline
            282 & T.A.6.1.9 & मृत्तिके प्रतिष्ठिते सर्वं तन्मे &      \\
        \hline
            283 & T.A.6.1.10 & ब्रह्म जज्ञानं प्रथमं पुरस्ताद्{-}विसीमतः &      \\
        \hline
            284 & T.A.6.1.11 & स्वस्ति नो मघवा करोतु &      \\
        \hline
            285 & T.A.6.1.12 & महेरणाय चक्षसे यो वः &      \\
        \hline
            286 & T.A.6.1.13 & यदपां क्रूरं ॅयदमेद्ध्यं ॅयदशान्तं &      \\
        \hline
            287 & T.A.6.1.14 & समुद्रा{-}दर्णवा{-}दधि सम्ॅवथ्सरो अजायत अहोरात्राणि &      \\
        \hline
            288 & T.A.6.1.15 & स नः सुवः सं &      \\
        \hline
            289 & T.A.6.2.1 & जातवेदसे सुनवाम सोम{-}मरातीयतो निदहाति &      \\
        \hline
            290 & T.A.6.3.1 & भू{-}रन्न{-}मग्नये पृथिव्यै स्वाहा , &      \\
        \hline
            291 & T.A.6.4.1 & भूरग्नये पृथिव्यै स्वाहा, भुवो &      \\
        \hline
            292 & T.A.6.5.1 & भूरग्नये च पृथिव्यै च &      \\
        \hline
            293 & T.A.6.6.1 & पाहि नो अग्न एनसे &      \\
        \hline
            294 & T.A.6.7.1 & पाहि नो अग्न एकया &      \\
        \hline
            295 & T.A.6.8.1 & यश्छन्दसा{-}मृषभो विश्वरूप{-}श्छन्दोभ्य श्छन्दाꣳस्या विवेश &      \\
        \hline
            296 & T.A.6.9.1 & नमो ब्रह्मणे धारणं मे &      \\
        \hline
            297 & T.A.6.10.1 & ऋतं तपः सत्यं तपः &      \\
        \hline
            298 & T.A.6.11.1 & यथा वृक्षस्य सपुंष्पितस्य दूराद् &      \\
        \hline
            299 & T.A.6.12.1 & अणो{-}रणीयान् महतो महीया{-}नात्मा गुहायां &      \\
        \hline
            300 & T.A.6.12.2 & हꣳसः शुचिषद् वसु{-}रन्तरिक्ष{-}सद्धोता वेदिष{-}दतिथिर्{-}दुरोणसत् &      \\
        \hline
            301 & T.A.6.12.3 & त्रिधा हितं पणिभिर् गुह्यमानं &      \\
        \hline
            302 & T.A.6.13.1 & सहस्रशीर्.षं देवं ॅविश्वाक्षं ॅविश्व &      \\
        \hline
            303 & T.A.6.13.2 & अन्तर् बहिश्च तथ् सर्वं &      \\
        \hline
            304 & T.A.6.14.1 & आदित्यो वा एष एतन् &      \\
        \hline
            305 & T.A.6.15.1 & आदित्यो वै तेज ओजो &      \\
        \hline
            306 & T.A.6.16.1 & निधनपतये नमः निधनपतान्तिकाय नमः &      \\
        \hline
            307 & T.A.6.16.2 & भवाय नमः भवलिङ्गाय नमः &      \\
        \hline
            308 & T.A.6.17.1 & सद्योजातं प्रपद्यामि सद्योजाताय वै &      \\
        \hline
            309 & T.A.6.18.1 & वामदेवाय नमो ज्येष्ठाय नमः &      \\
        \hline
            310 & T.A.6.19.1 & अघोरेभ्योऽथ घोरेभ्यो घोरघोरतरेभ्यः सर्वेभ्यः &      \\
        \hline
            311 & T.A.6.20.1 & तत्पुरुषाय विद्महे महादेवाय धीमहि &      \\
        \hline
            312 & T.A.6.21.1 & ईशानः सर्वविद्याना{-} मीश्वरः सर्वभूतानां &      \\
        \hline
            313 & T.A.6.22.1 & नमो हिरण्यबाहवे हिरण्यवर्णाय हिरण्यरूपाय &      \\
        \hline
            314 & T.A.6.23.1 & ऋतꣳ सत्यं परं ब्रह्म &      \\
        \hline
            315 & T.A.6.24.1 & सर्वो वै रुद्रस्तस्मै रुद्राय &      \\
        \hline
            316 & T.A.6.25.1 & कद्रुद्राय प्रचेतसे मीढुष्टमाय तव्यसे &      \\
        \hline
            317 & T.A.6.26.1 & यस्य वै कङ्कत्यग्नि{-}होत्रहवणी भवति &      \\
        \hline
            318 & T.A.6.27.1 & ठिस् ए꣡पन्सिऒन् इस् अप्पॆअरिन्ग् &      \\
        \hline
            319 & T.A.6.28.1 & अदितिर्{-}देवा गन्धर्वा मनुष्याः पितरो{-}ऽसुरा{-}स्तेषां &      \\
        \hline
            320 & T.A.6.29.1 & आपो वा इदꣳ सर्वं &      \\
        \hline
            321 & T.A.6.30.1 & आपः पुनन्तु पृथिवीं पृथिवी &      \\
        \hline
            322 & T.A.6.31.1 & अग्निश्च मा मन्युश्च मन्युपतयश्च &      \\
        \hline
            323 & T.A.6.32.1 & सूर्यश्च मा मन्युश्च मन्युपतयश्च &      \\
        \hline
            324 & T.A.6.33.1 & ओमित्येकाक्षरं ब्रह्म अग्निर्देवता ब्रह्म &      \\
        \hline
            325 & T.A.6.34.1 & आयातु वरदा देवी अक्षरं &      \\
        \hline
            326 & T.A.6.35.1 & ओजोऽसि सहोऽसि बलमसि भ्राजोऽसि &      \\
        \hline
            327 & T.A.6.36.1 & उत्तमे शिखरे जाते भूम्यां &      \\
        \hline
            328 & T.A.6.37.1 & घृणिः सूर्य आदित्यो न &      \\
        \hline
            329 & T.A.6.38.1 & ब्रह्म मेतु मां मधु &      \\
        \hline
            330 & T.A.6.39.1 & ब्रह्म मेधया मधु मेधया &      \\
        \hline
            331 & T.A.6.40.1 & ब्रह्म मेधवा मधु मेधवा &      \\
        \hline
            332 & T.A.6.41.1 & मेधादेवी जुषमाणा न आगाद् &      \\
        \hline
            333 & T.A.6.42.1 & मेधां म इन्द्रो ददातु &      \\
        \hline
            334 & T.A.6.43.1 & आमां मेधा सुरभिर् विश्वरूपा &      \\
        \hline
            335 & T.A.6.44.1 & मयि मेधां मयि प्रजां &      \\
        \hline
            336 & T.A.6.45.1 & अपैतु मृत्यु{-}रमृतन्न आगन् वैवस्वतो &      \\
        \hline
            337 & T.A.6.46.1 & परं मृत्यो अनु परेहि &      \\
        \hline
            338 & T.A.6.47.1 & वातं प्राणं मनसा न्वारभामहे &      \\
        \hline
            339 & T.A.6.48.1 & अमुत्र भूयादध यद्यमस्य बृहस्पते &      \\
        \hline
            340 & T.A.6.49.1 & हरिꣳ हरन्त{-} मनुयन्ति देवा &      \\
        \hline
            341 & T.A.6.50.1 & शल्कैरग्नि{-}मिन्धान उभौ लोकौ सनेमहं &      \\
        \hline
            342 & T.A.6.51.1 & मा छिदो मृत्यो मा &      \\
        \hline
            343 & T.A.6.52.1 & मा नो महान्तमुत मा &      \\
        \hline
            344 & T.A.6.53.1 & मा नस्तोके तनये मा &      \\
        \hline
            345 & T.A.6.54.1 & प्रजापते न त्वदेता{-}न्यन्यो विश्वा &      \\
        \hline
            346 & T.A.6.55.1 & स्वस्तिदा विशस्पतिर् वृत्रहा विमृधो &      \\
        \hline
            347 & T.A.6.56.1 & त्र्यंबकं ॅयजामहे सुगन्धिं पुष्टिवर्द्धनं &      \\
        \hline
            348 & T.A.6.57.1 & ये ते सहस्रमयुतं पाशा &      \\
        \hline
            349 & T.A.6.58.1 & मृत्यवे स्वाहा मृत्यवे स्वाहा &      \\
        \hline
            350 & T.A.6.59.1 & देवकृतस्यैनसो{-}ऽवयजनमसि स्वाहा मनुष्यकृतस्यैनसो ऽवयजनमसि &      \\
        \hline
            351 & T.A.6.60.1 & यद्वो देवाश्चकृम जिह्वया गुरुमनसो &      \\
        \hline
            352 & T.A.6.61.1 & कामोऽकार्.षीन् नमो नमः कामोऽकार्.षीत् &      \\
        \hline
            353 & T.A.6.62.1 & मन्युरकार्.षीन् नमो नमः मन्युरकार्.षीन् &      \\
        \hline
            354 & T.A.6.63.1 & तिलाञ्जुहोमि सरसाꣳ सपिष्टान् गन्धार &      \\
        \hline
            355 & T.A.6.64.1 & तिलाः कृष्णा{-}स्तिलाः श्वेता{-}स्तिलाः सौम्या &      \\
        \hline
            356 & T.A.6.65.1 & प्राणापान{-}व्यानोदान{-}समाना मे शुद्ध्यन्तां ज्योति &      \\
        \hline
            357 & T.A.6.66.1 & पृथिव्याप स्तेजो वायु{-}राकाशा मे &      \\
        \hline
            358 & T.A.6.67.1 & अग्नये स्वाहा विश्वेभ्यो देवेभ्यः &      \\
        \hline
            359 & T.A.6.67.2 & रक्षोदेवजनेभ्यः स्वाहा गृह्याभ्यः स्वाहा &      \\
        \hline
            360 & T.A.6.67.3 & दिवे स्वाहा सूर्याय स्वाहा &      \\
        \hline
            361 & T.A.6.67.4 & देवेभ्यः स्वाहा पितृभ्यः स्वधाऽस्तु &      \\
        \hline
            362 & T.A.6.68.1 & ओं तद् ब्रह्म ओं &      \\
        \hline
            363 & T.A.6.69.1 & श्रद्धायां प्राणे निविष्टोऽमृतं जुहोमि &      \\
        \hline
            364 & T.A.6.70.1 & श्रद्धायां प्राणे निविश्याऽमृतꣳ हुतं &      \\
        \hline
            365 & T.A.6.71.1 & अङ्गुष्ठमात्रः पुरुषोऽङ्गुष्ठञ्च समाश्रितः ईशः &      \\
        \hline
            366 & T.A.6.72.1 & वाङ्म आसन्न् नसोः प्राणः &      \\
        \hline
            367 & T.A.6.73.1 & वयः सुपर्णा उपसेदुरिन्द्रं प्रिय &      \\
        \hline
            368 & T.A.6.74.1 & प्राणानां ग्रन्थिरसि रुद्रो मा &      \\
        \hline
            369 & T.A.6.75.1 & नमो रुद्राय विष्णवे मृत्युर्मे &      \\
        \hline
            370 & T.A.6.76.1 & त्वमग्ने द्युभि{-}स्त्वमाशु{-}शुक्षणि{-}स्त्वमद्भ्य{-}स्त्वमश्मनस्परि त्वं ॅवनेभ्य{-}स्त्वमोषधीभ्य{-}स्त्वं &      \\
        \hline
            371 & T.A.6.77.1 & शिवेन मे सन्तिष्ठस्व स्योनेन &      \\
        \hline
            372 & T.A.6.78.1 & सत्यं परं परं सत्यं &      \\
        \hline
            373 & T.A.6.79.1 & प्राजापत्यो हारुणिः सुपर्णेयः प्रजापतिं &      \\
        \hline
            374 & T.A.6.80.1 & तस्यैवं ॅविदुषो यज्ञ्स्यात्मा यजमानः{-}श्रध्दापत्नी &      \\
        \hline
            375 & T.A.7.1.1 & नमो वाचे या चोदिता &      \\
        \hline
            376 & T.A.7.2.1 & युञ्जते मन उत युञ्जते &      \\
        \hline
            377 & T.A.7.2.2 & देवयन्तस्त्वेमहे उप प्रयन्तु मरुतः &      \\
        \hline
            378 & T.A.7.2.3 & मखाय त्वा मखस्य त्वा &  T.A.7.2.4 T.A.7.2.5       \\
        \hline
            379 & T.A.7.2.4 & मखाय त्वा मखस्य त्वा & T.A.7.2.3  T.A.7.2.5       \\
        \hline
            380 & T.A.7.2.5 & मखाय त्वा मखस्य त्वा & T.A.7.2.3 T.A.7.2.4        \\
        \hline
            381 & T.A.7.2.6 & यज्ञ्स्य पदे स्थः गायत्रेण &      \\
        \hline
            382 & T.A.7.3.1 & वृष्णो अश्वस्य निष्पदसि वरुणस्त्वा &      \\
        \hline
            383 & T.A.7.3.2 & मित्रस्य चर्.षणीधृतः श्रवो देवस्य &      \\
        \hline
            384 & T.A.7.3.3 & ऊर्द्ध्वस्तिष्ठ ध्रुवस्त्वम् सूर्यस्य त्वा &      \\
        \hline
            385 & T.A.7.4.1 & ब्रह्मन् प्रवःग्येण प्रचरिष्यामः होतः &      \\
        \hline
            386 & T.A.7.5.1 & ब्रह्मन् प्रचरिष्यामः होतर् घर्ममभिष्टुहि &      \\
        \hline
            387 & T.A.7.5.2 & पृथिवीं तपसस्त्रायस्व अर्चिरसि शोचिरसि &      \\
        \hline
            388 & T.A.7.5.3 & अनाधृष्या पुरस्तात् अग्नेराधिपत्ये आयुर्मे &      \\
        \hline
            389 & T.A.7.5.4 & मित्रावरुणयोराधिपत्ये श्रोत्रम् मे दाः &      \\
        \hline
            390 & T.A.7.5.5 & सूपसदा मे भूया मा &      \\
        \hline
            391 & T.A.7.5.6 & सम्मा असि विमा असि &      \\
        \hline
            392 & T.A.7.5.7 & विषुरूपे अहनी द्यौरिवासि विश्वा &      \\
        \hline
            393 & T.A.7.6.1 & दश प्राचीर्दश भासि दक्षिणा &      \\
        \hline
            394 & T.A.7.6.2 & द्युतानस्त्वा मारुतो मरुद्भिरुत्तरतो रोचयत्वानुष्टुभेन &      \\
        \hline
            395 & T.A.7.6.3 & मयि रुक् दश पुरस्ताद्{-}रोचसे &      \\
        \hline
            396 & T.A.7.7.1 & अपश्यङ्गोपामनिपद्यमानम् आ च परा &      \\
        \hline
            397 & T.A.7.7.2 & स्वाहा समग्निस्तपसा गत सम् &      \\
        \hline
            398 & T.A.7.7.3 & ऊर्द्ध्वमिम{-}मद्ध्वरङ्कृधि दिवि देवेषु होत्रा &      \\
        \hline
            399 & T.A.7.7.4 & गर्भो देवानाम् पिता मतीनाम् &      \\
        \hline
            400 & T.A.7.7.5 & अन्तरिक्षप्र उरोर्वरीयान् अशीमहि त्वा &      \\
        \hline
            401 & T.A.7.8.1 & देवस्य त्वा सवितुः प्रसवे & T.A.3.10.1        \\
        \hline
            402 & T.A.7.8.2 & अदित्या उष्णीषमसि वायुरस्यैडः पूषा &      \\
        \hline
            403 & T.A.7.8.3 & घर्माय शिꣳष बृहस्पतिस्त्वोपसीदतु दानवः &      \\
        \hline
            404 & T.A.7.8.4 & गायत्रोऽसि त्रैष्टुभोऽसि जागतमसि सहोर्जो &      \\
        \hline
            405 & T.A.7.8.5 & अन्तरिक्षेण त्वोपयच्छामि देवानां त्वा &      \\
        \hline
            406 & T.A.7.9.1 & समुद्राय त्वा वाताय स्वाहा &      \\
        \hline
            407 & T.A.7.9.2 & बृहस्पतये त्वा विश्वदेव्यावते स्वाहा &      \\
        \hline
            408 & T.A.7.9.3 & अहर्{-}दिवाभि{-}रूतिभिः अनु वान् द्यावापृथिवी &      \\
        \hline
            409 & T.A.7.9.4 & दिवि धा इमं ॅयज्ञ्म् &      \\
        \hline
            410 & T.A.7.10.1 & इषे पीपिहि ऊर्जे पीपिहि &      \\
        \hline
            411 & T.A.7.10.2 & यजमानाय पीपिहि मह्यञ्ज्यैष्ठ्याय पीपिहि &      \\
        \hline
            412 & T.A.7.10.3 & अमुष्य त्वा प्राणे सादयामि &      \\
        \hline
            413 & T.A.7.10.4 & अहर् ज्योतिः केतुना जुषताम् &      \\
        \hline
            414 & T.A.7.10.5 & एषा ते अग्ने समित् &      \\
        \hline
            415 & T.A.7.10.6 & पिता नोऽसि मा मा &      \\
        \hline
            416 & T.A.7.11.1 & घर्म या ते दिवि &  T.A.8.9.1       \\
        \hline
            417 & T.A.7.11.2 & घर्म या ते पृथिव्यां &      \\
        \hline
            418 & T.A.7.11.3 & वयमनुक्रामाम सुविताय नव्यसे ब्रह्मणस्त्वा &      \\
        \hline
            419 & T.A.7.11.4 & शिशुर्{-}जनधायाः शञ्च वक्षि परि &      \\
        \hline
            420 & T.A.7.11.5 & रन्तिर्{-}नामासि दिव्यो गन्धर्वः तस्य &      \\
        \hline
            421 & T.A.7.11.6 & व्यसौ योऽस्मान् द्वेष्टि यञ्च &      \\
        \hline
            422 & T.A.7.11.7 & नमस्ते अस्तु मा मा &      \\
        \hline
            423 & T.A.7.11.8 & धियो हिन्वानो धिय इन्नो &      \\
        \hline
            424 & T.A.7.11.9 & दुर्मित्रास्तस्मै भूयासुः योऽस्मान् द्वेष्टि &      \\
        \hline
            425 & T.A.7.12.1 & महीनाम् पयोऽसि विहितं देवत्रा &      \\
        \hline
            426 & T.A.7.13.1 & अस्कान् द्यौः पृथिवीम् अस्कानृषभो &      \\
        \hline
            427 & T.A.7.14.1 & या पुरस्ताद् विद्युदापतत् तान्त &      \\
        \hline
            428 & T.A.7.15.1 & प्राणाय स्वाहा व्यानाय स्वाहा &      \\
        \hline
            429 & T.A.7.16.1 & पूष्णे स्वाहा पूष्णे शरसे &      \\
        \hline
            430 & T.A.7.17.1 & उदस्य शुष्माद् भानुर्नार्त बिभर्ति &      \\
        \hline
            431 & T.A.7.18.1 & यास्ते अग्न आर्द्रा योनयो &      \\
        \hline
            432 & T.A.7.19.1 & अग्निरसि वैश्वानरोऽसि सम्ॅवथ्सरोऽसि परिवथ्सरोऽसि &      \\
        \hline
            433 & T.A.7.20.1 & भूर्भुवस्सुवः ऊर्द्ध्व ऊ षुण &      \\
        \hline
            434 & T.A.7.20.2 & निष्कर्ता विह्रुतम् पुनः पुनरूर्जा, &      \\
        \hline
            435 & T.A.7.20.3 & उप नो मित्रावरुणाविहावतम् अन्वादीद्ध्याथामिह &      \\
        \hline
            436 & T.A.7.21.1 & भूर्भुवस्सुवः मयि त्यदिन्द्रियम् महत् &      \\
        \hline
            437 & T.A.7.22.1 & यास्ते अग्ने घोरास्तनुवः क्षुच्च &      \\
        \hline
            438 & T.A.7.23.1 & स्निक्च स्नीहितिश्च स्निहितिश्च उष्णा &      \\
        \hline
            439 & T.A.7.24.1 & धुनिश्च ध्वान्तश्च ध्वनश्च ध्वनयꣳश्च &      \\
        \hline
            440 & T.A.7.25.1 & उग्रश्च धुनिश्च ध्वान्तश्च ध्वनश्च &      \\
        \hline
            441 & T.A.7.26.1 & अहोरात्रे त्वोदीरयताम् अर्द्धमासास्त्वोदीं जयन्तु &      \\
        \hline
            442 & T.A.7.27.1 & खट् फड् जहि छिन्धी &      \\
        \hline
            443 & T.A.7.28.1 & विगा इन्द्र विचरन्थ् स्पाशयस्व &      \\
        \hline
            444 & T.A.7.30.1 & यदेतद्{-}वृकसो भूत्वा वाग् देव्यभिरायसि &      \\
        \hline
            445 & T.A.7.31.1 & यदीषितो यदि वा स्वकामी &      \\
        \hline
            446 & T.A.7.32.1 & दीर्घमुखि दुर्.हणु मा स्म &      \\
        \hline
            447 & T.A.7.33.1 & इत्थादुलूक आपप्तत् हिरण्याक्षो अयोमुखः &      \\
        \hline
            448 & T.A.7.34.1 & यदेतद् भूतान्यन्वाविश्य दैवीं ॅवाचं &      \\
        \hline
            449 & T.A.7.35.1 & प्रसार्य सक्थ्यौ पतसि सव्यमक्षि &      \\
        \hline
            450 & T.A.7.36.1 & अत्रिणा त्वा क्रिमे हन्मि &      \\
        \hline
            451 & T.A.7.37.1 & आहरावद्य शृतस्य हविषो यथा &      \\
        \hline
            452 & T.A.7.38.1 & ब्रह्मणा त्वा शपामि ब्रह्मणस्त्वा &      \\
        \hline
            453 & T.A.7.39.1 & उत्तुद शिमिजावरि तल्पेजे तल्प &      \\
        \hline
            454 & T.A.7.40.1 & भूर्भुवस्सुवो भूर्भुवस्सुवो भूर्भुवस्सुवः भुवोऽद्धायि &      \\
        \hline
            455 & T.A.7.41.1 & पृथिवी समित् तामग्निः समिन्धे &      \\
        \hline
            456 & T.A.7.41.2 & तां ॅवायुः समिन्धे सा &      \\
        \hline
            457 & T.A.7.41.3 & साऽऽदित्यꣳ समिन्धे तामहꣳ समिन्धे &      \\
        \hline
            458 & T.A.7.41.4 & तच्छकेयं तन्मे राद्ध्यताम् वायो &      \\
        \hline
            459 & T.A.7.41.5 & वर्चसा श्रिया यशसा ब्रह्मवर्चसेन &      \\
        \hline
            460 & T.A.7.41.6 & यशसा ब्रह्मवर्चसेन अन्नाद्येन समिन्तां &      \\
        \hline
            461 & T.A.7.41.7 & अन्नाद्येन समिन्ताꣳ स्वाहा प्राजापत्या &      \\
        \hline
            462 & T.A.7.42.1 & शन्नो वातः पवताम् मातरिश्वा &      \\
        \hline
            463 & T.A.7.42.2 & दक्षम् मे अन्य आवातु &      \\
        \hline
            464 & T.A.7.42.3 & द्युभिरक्तुभिः परिपात{-}मस्मानरिष्टेभि{-}रश्विना सौभगेभिः तन्नो &      \\
        \hline
            465 & T.A.7.42.4 & शन्नो देवीरभिष्टय आपो भवन्तु &      \\
        \hline
            466 & T.A.7.42.5 & आपो जनयथा च नः &      \\
        \hline
            467 & T.A.7.42.6 & य उदगान् महतोऽर्णवाद्{-}विभ्राजमानः सरिरस्य &      \\
        \hline
            468 & T.A.8.1.1 & देवा वै सत्रमासत ऋद्धिपरिमितम् &      \\
        \hline
            469 & T.A.8.1.2 & तेषाम् मखं ॅवैष्णवं ॅयश &      \\
        \hline
            470 & T.A.8.1.3 & तमेकꣳ सन्तम् बहवो नाभ्यधृष्णुवन्न् &      \\
        \hline
            471 & T.A.8.1.4 & तथ् स्मयाकानाꣳ स्मयाकत्वम् तस्माद्{-}दीक्षितेनापिगृह्य &      \\
        \hline
            472 & T.A.8.1.5 & वारेवृतꣳ ह्यासाम् तस्य ज्यामप्यादन्न् &      \\
        \hline
            473 & T.A.8.1.6 & यदस्याः समभरन्न् तथ् सम्राज्ञ्ः &      \\
        \hline
            474 & T.A.8.1.7 & भिषजौ वै स्थः इदं &      \\
        \hline
            475 & T.A.8.2.1 & सावित्रम् जुहोति प्रसूत्यै चतुर्गृहीतेन &      \\
        \hline
            476 & T.A.8.2.2 & देवतायै वषट्काराय यच्चतुर्गृहीतम् जुहोति &      \\
        \hline
            477 & T.A.8.2.3 & यज्ञ्परुरन्तरियात् यजुरेव वदेत् न &      \\
        \hline
            478 & T.A.8.2.4 & स खदिरोऽभवत् यः पशून् &      \\
        \hline
            479 & T.A.8.2.5 & तेजसैव यज्ञ्स्य शिरः सम्भरति &      \\
        \hline
            480 & T.A.8.2.6 & अद्ध्वरकृद्{-}देवेभ्य इत्याह यज्ञो वा &      \\
        \hline
            481 & T.A.8.2.7 & पाङ्क्तो हि यज्ञ्ः देवा &      \\
        \hline
            482 & T.A.8.2.8 & त्रिर्.हरति त्रय इमे लोकाः &      \\
        \hline
            483 & T.A.8.2.9 & यद्{-}वल्मीकम् यद्{-}वल्मीकवपा सम्भारो भवति &      \\
        \hline
            484 & T.A.8.2.10 & तन्नाद्ध्रियत स पूतीकस्तम्बे पराक्रमत &      \\
        \hline
            485 & T.A.8.2.11 & तमेवावरुन्धे पञ्चैते सम्भारा भवन्ति &      \\
        \hline
            486 & T.A.8.2.12 & आरण्याः पशवः कनीयाꣳसः शुचा &      \\
        \hline
            487 & T.A.8.2.13 & तेज एवास्मिन् दधाति मधु &      \\
        \hline
            488 & T.A.8.3.1 & परिश्रिते करोति ब्रह्मवर्चसस्य परिगृहीत्यै &      \\
        \hline
            489 & T.A.8.3.2 & तस्मान्नान्तराय्यम् आत्मनो गोपीथाय वेणुना &      \\
        \hline
            490 & T.A.8.3.3 & तस्मादेवमाह यज्ञ्स्य पदे स्थ &      \\
        \hline
            491 & T.A.8.3.4 & वीर्यम् ॅवै छन्दाꣳसि वीर्येणैवैनम् &      \\
        \hline
            492 & T.A.8.3.5 & अपरिमितम् करोति अपरिमितस्यावरुद्ध्यै परिग्रीवम् &      \\
        \hline
            493 & T.A.8.3.6 & छन्दोभिरेवैनम् धूपयति अर्चिषे त्वा &      \\
        \hline
            494 & T.A.8.3.7 & तस्मादग्निः सर्वा दिशोऽनु विभाति &      \\
        \hline
            495 & T.A.8.3.8 & दिशो भूतिः इमानेवास्मै लोकान् &      \\
        \hline
            496 & T.A.8.3.9 & आच्छृणत्ति देवत्राऽकः अजक्षीरेणाच्छृणत्ति परमं &      \\
        \hline
            497 & T.A.8.4.1 & ब्रह्मन् प्रचरिष्यामो होतर्{-}घर्ममभिष्टुहीत्याह एष &      \\
        \hline
            498 & T.A.8.4.2 & अभिपूर्वम् प्रोक्षति अभिपूर्व{-}मेवास्मिन्{-}तेजो दधाति &      \\
        \hline
            499 & T.A.8.4.3 & गायत्रो हि प्राणः प्राणमेव &      \\
        \hline
            500 & T.A.8.4.4 & यत् प्रवर्ग्यः ऊङ्र्मुञ्जाः यन् &      \\
        \hline
            501 & T.A.8.4.5 & तेजसैवैनमनक्ति पृथिवीम् तपसस्त्रायस्वेति हिरण्यमुपास्यति &      \\
        \hline
            502 & T.A.8.4.6 & अर्चिरसि शोचिरसीत्याह तेज एवास्मिन् &      \\
        \hline
            503 & T.A.8.4.7 & अनाधृष्या पुरस्तादिति यदेतानि यजूꣳष्याह &      \\
        \hline
            504 & T.A.8.4.8 & मनोरश्वाऽसि भूरिपुत्रेतीमा{-}मभिमृशति इयं ॅवै &      \\
        \hline
            505 & T.A.8.4.9 & स्वाहा मरुद्भिः परिश्रयस्वेत्याह अमुमेवादित्यं &      \\
        \hline
            506 & T.A.8.4.10 & द्वादश मासाः सम्ॅवथ्सरः सम्ॅवथ्सरमेवावरुन्धे &      \\
        \hline
            507 & T.A.8.4.11 & स्तौत्येवैनमेतत् गायत्रमसि त्रैष्टुभमसि जागतमसीति &      \\
        \hline
            508 & T.A.8.4.12 & अथो रक्षसामपहत्यै त्रिः पुनः &      \\
        \hline
            509 & T.A.8.4.13 & ततो वै स दुश्चर्माऽभवत् &      \\
        \hline
            510 & T.A.8.5.1 & अग्निष्ट्वा वसुभिः पुरस्ताद्{-}रोचयतु गायत्रेण &      \\
        \hline
            511 & T.A.8.5.2 & स मा रुचितो रोचयेत्याह &      \\
        \hline
            512 & T.A.8.5.3 & रोचितस्त्वम् देव घर्म देवेष्वसीत्याह &      \\
        \hline
            513 & T.A.8.6.1 & शिरो वा एतद्{-}यज्ञ्स्य यत् &      \\
        \hline
            514 & T.A.8.6.2 & ऋतुभ्य एव यज्ञ्स्य शिरोऽवरुन्धे &      \\
        \hline
            515 & T.A.8.6.3 & अग्निष्टोमे प्रवृणक्ति एतावान्. वै &      \\
        \hline
            516 & T.A.8.6.4 & पृष्ठानि वा अच्युतञ्च्यावयन्ति पृष्ठैरेवास्मा &      \\
        \hline
            517 & T.A.8.6.5 & न ह्येष निपद्यते आ &      \\
        \hline
            518 & T.A.8.6.6 & ग्रैष्मावेवास्मा ऋतू कल्पयति समग्निरग्निना &      \\
        \hline
            519 & T.A.8.6.7 & दिवि देवेषु होत्रा यच्छेत्याह &      \\
        \hline
            520 & T.A.8.6.8 & गर्भो देवानामित्याह गर्भो ह्येष &      \\
        \hline
            521 & T.A.8.6.9 & मतिर्ह्येष कवीनाम् सम् देवो &      \\
        \hline
            522 & T.A.8.6.10 & नाभिर्दशमी प्राणानेव यजमाने दधाति &      \\
        \hline
            523 & T.A.8.6.11 & रुचितमवेक्षन्ते रुचिताद्वै प्रजापतिः प्रजा &      \\
        \hline
            524 & T.A.8.6.12 & अधीयन्तोऽवेक्षन्ते सर्वमायुर्यन्ति न पत्न्यवेक्षेत &      \\
        \hline
            525 & T.A.8.7.1 & देवस्य त्वा सवितुः प्रसव &      \\
        \hline
            526 & T.A.8.7.2 & मनुष्यनामैरेवैनामाह्वयति षट्थ् सम्पद्यन्ते षड्वा &      \\
        \hline
            527 & T.A.8.7.3 & स्वयैवैनम् देवतयोपावसृजति अश्विभ्याम् प्रदापयेत्याह &      \\
        \hline
            528 & T.A.8.7.4 & ब्रह्म वै देवानाम् बृहस्पतिः &      \\
        \hline
            529 & T.A.8.7.5 & तस्मादिन्द्रो देवतानाम् भूयिष्ठभाक्तमः गायत्रोऽसि &      \\
        \hline
            530 & T.A.8.7.6 & घर्मम् पात वसवो यजता &      \\
        \hline
            531 & T.A.8.7.7 & स्वाहा त्वा सूर्यस्य रश्मये &      \\
        \hline
            532 & T.A.8.7.8 & अन्तरिक्षेण त्वोपयच्छामीत्याह अन्तरिक्षेणैवैन{-}मुपयच्छति न &      \\
        \hline
            533 & T.A.8.7.9 & सुवरसि सुवर्मे यच्छ दिवं &      \\
        \hline
            534 & T.A.8.7.10 & पाङ्क्तो यज्ञ्ः यावानेव यज्ञ्ः &      \\
        \hline
            535 & T.A.8.7.11 & अफ्सु वै वरुण आदित्यवान् &      \\
        \hline
            536 & T.A.8.7.12 & तस्मा एवैनञ्जुहोति एताभ्य एवैनम् &      \\
        \hline
            537 & T.A.8.8.1 & विश्वा आशा दक्षिणसदित्याह विश्वानेव &      \\
        \hline
            538 & T.A.8.8.2 & अश्विना घर्मम् पातꣳ हार्दिवान{-}महर्दिवाभि{-}रूतिभिरित्याह &      \\
        \hline
            539 & T.A.8.8.3 & पूर्वमेवोदितम् उत्तरेणाभिगृणाति अनु वान्{-}द्यावापृथिवी &      \\
        \hline
            540 & T.A.8.8.4 & दिक्ष्वेवैनम् प्रतिष्ठापयति देवान् घर्मपान् &      \\
        \hline
            541 & T.A.8.8.5 & यत् प्रत्यक् तन्{-}मनुष्याणाम् यदुदङ्ङ् &      \\
        \hline
            542 & T.A.8.8.6 & तेजसोऽस्कन्दाय इषे पीपिह्यूर्जे पीपिहीत्याह &      \\
        \hline
            543 & T.A.8.8.7 & ब्रह्मन्नेवैनम् प्रतिष्ठापयति नेत्त्वा वातः &      \\
        \hline
            544 & T.A.8.8.8 & ताभ्य एवैनञ्जुहोति प्रतिरेभ्यः स्वाहेत्याह &      \\
        \hline
            545 & T.A.8.8.9 & रुद्राय रुद्रहोत्रे स्वाहेत्याह रुद्रमेव &      \\
        \hline
            546 & T.A.8.8.10 & चक्षुरस्य प्रमायुकꣳ स्यात् तस्मान्नान्वीक्ष्यः &      \\
        \hline
            547 & T.A.8.8.11 & यद्{-}यजुषा जुहुयात् अयथापूर्वमाहुती जुहुयात् &      \\
        \hline
            548 & T.A.8.8.12 & प्राणो वा इन्द्रतमोऽग्निः प्राण &      \\
        \hline
            549 & T.A.8.8.13 & सम्ॅवथ्सरन्न माꣳसमश्ञीयात् न रामामुपेयात् &      \\
        \hline
            550 & T.A.8.9.1 & घर्म या ते दिवि & T.A.7.11.1        \\
        \hline
            551 & T.A.8.9.2 & एष्वेव लोकेषु प्रजा दाधार &      \\
        \hline
            552 & T.A.8.9.3 & प्राञ्चमुद्वासयति तस्मादसावादित्यः पुरस्तादुदेति शफोपयमान्धवित्राणि &      \\
        \hline
            553 & T.A.8.9.4 & यज्ञ्ꣳ रक्षाꣳसि जिघाꣳसन्ति साम्ना &      \\
        \hline
            554 & T.A.8.9.5 & यत् पृथिव्यामुद्वासयेत् पृथिवीꣳ शुचाऽर्पयेत् &      \\
        \hline
            555 & T.A.8.9.6 & अमृत एवैनम् प्रतिष्ठापयति वल्गुरसि &      \\
        \hline
            556 & T.A.8.9.7 & इयं ॅवा ऋतम् तस्या &      \\
        \hline
            557 & T.A.8.9.8 & अनशनायुको भवति य एवं &      \\
        \hline
            558 & T.A.8.9.9 & वृषा हरिः महान्{-}मित्रो न &      \\
        \hline
            559 & T.A.8.9.10 & पूर्वमेवोदितम् उत्तरेणाभिगृणाति धियो हिन्वानो &      \\
        \hline
            560 & T.A.8.9.11 & मनुष्यो हि एष सन्{-}मनुष्यानुपैति &      \\
        \hline
            561 & T.A.8.10.1 & प्रजापतिं ॅवै देवाः शुक्रम् &      \\
        \hline
            562 & T.A.8.10.2 & तथ् साम्नः पयः यदजायै &      \\
        \hline
            563 & T.A.8.10.3 & तेजः प्रवर्ग्यः तेजसैव तेजः &      \\
        \hline
            564 & T.A.8.10.4 & तस्मा{-}दुत्तरवेद्या{-}मेवोद्वासयेत् प्रजानाम् गोपीथाय पुरो &      \\
        \hline
            565 & T.A.8.10.5 & यम् द्विष्यात् यत्र स &      \\
        \hline
            566 & T.A.8.10.6 & इदमह{-}ममुष्यामुष्यायणस्य शुचा प्राणमपि दहामीत्याह &      \\
        \hline
            567 & T.A.8.11.1 & प्रजापतिः सम्भ्रियमाणः सम्राट्थ् सम्भृतः &      \\
        \hline
            568 & T.A.8.11.2 & यो वै वसीयाꣳ सम्ॅयथानाममुपचरति &      \\
        \hline
            569 & T.A.8.11.3 & प्रियेण नाम्ना समर्द्धयति कीर्तिरस्य &      \\
        \hline
            570 & T.A.8.11.4 & वसवः प्रवृक्तः सोमोऽभिकीर्यमाणः आश्विनः &      \\
        \hline
            571 & T.A.8.11.5 & असौ खलु वावैष आदित्यः &      \\
        \hline
            572 & T.A.8.11.6 & तस्मादश्ञुते प्रजापतिर्वा एष द्वादशधा &      \\
        \hline
            573 & T.A.8.12.1 & सविता भूत्वा प्रथमेऽहन् प्रवृज्यते &      \\
        \hline
            574 & T.A.8.12.2 & यथ्{-}षष्ठेऽहन्{-}प्रवृज्यते ऋतुर्{-}भूत्वा सम्ॅवथ्सरमेति यथ्{-}सप्तमेऽहन्{-}प्रवृज्यते &      \\
        \hline
            575 & T.A.8.12.3 & यदेकादशेऽहन्{-}प्रवृज्यते इन्द्रो भूत्वा त्रिष्टुभमेति &      \\
        \hline
        \bottomrule
  \end{longtable}
  
\end{document}