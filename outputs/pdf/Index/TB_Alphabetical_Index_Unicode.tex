\documentclass[17pt]{extarticle}
\usepackage{babel}
\usepackage{fontspec}
\usepackage{polyglossia}
\usepackage{extsizes}

\usepackage{booktabs} % To thicken table lines



\setmainlanguage{sanskrit}
\setotherlanguages{english} %% or other languages
\setlength{\parindent}{0pt}
\pagestyle{myheadings}
\newfontfamily\devanagarifont[Script=Devanagari]{AdishilaVedic}


\newcommand{\VAR}[1]{}
\newcommand{\BLOCK}[1]{}

\usepackage{longtable} % To display tables on several pages

\begin{document} 


\begin{longtable}{||p{0.4in}||p{4.9in}||p{0.9in}||} % <-- Replaces \begin{table}, alignment must be specified here (no more tabular)
    \caption{कृष्ण यजुर्वेदीय तैत्तिरीय ब्राह्मणे}
    \label{tab:table1}\\
    \toprule
    \textbf{SNo} & \textbf{Beginning Words} & \textbf{Dasini} 
    
   
    \endfirsthead % <-- This denotes the end of the header, which will be shown on the first page only
    \toprule
    \textbf{SNo} & \textbf{Beginning Words} & \textbf{Dasini} 
    
   
    \endhead % <-- Everything between \endfirsthead and \endhead will be shown as a header on every page
        
    1 & अःकश्चक्षु{-}स्तदसौ सूःयस्तदयमग्निः संप्रियः पशुभिः & TB\_1.1.7.2       \\
    
    \hline
        
    2 & अक्षराजाय कितवम् कृताय सभाविनम् & TB\_3.4.16.1       \\
    
    \hline
        
    3 & अगस्यो मरुद्भ्य उक्ष्णः प्रौक्षत् & TB\_2.7.11.1       \\
    
    \hline
        
    4 & अगृभीताः पशवः सन्तु सर्वे & TB\_2.5.3.4       \\
    
    \hline
        
    5 & अग्नये देवेभ्यः पितृभ्यः समिद्ध्यमानायानुब्रूहीत्याह & TB\_1.6.9.1       \\
    
    \hline
        
    6 & अग्नये{-}ऽꣳहोमुचेऽष्टाकपालः सौर्यं पयः वायव्य & TB\_3.9.17.4       \\
    
    \hline
        
    7 & अग्नाविष्णू सजोषसा इमा वर्द्धन्तु & TB\_3.11.3.1       \\
    
    \hline
        
    8 & अग्नि{-}नक्षत्रमित्य{-}पचायन्ति गृहान्.ह दाहुको भवति & TB\_1.1.2.2       \\
    
    \hline
        
    9 & अग्निं गृह्णामि सुरथं ॅयो & TB\_3.7.4.3       \\
    
    \hline
        
    10 & अग्निः प्रजां प्रजनयति वृद्धामिन्द्रः & TB\_1.7.2.4       \\
    
    \hline
        
    11 & अग्निना वै होत्रा देवा & TB\_3.3.7.1       \\
    
    \hline
        
    12 & अग्निमद्य होतारमवृणीत अयꣳ सुतासुती & TB\_2.6.15.1       \\
    
    \hline
        
    13 & अग्निमद्य होतारमवृणीतायं ॅयजमानः पचन् & TB\_3.6.15.1       \\
    
    \hline
        
    14 & अग्निमश्वत्थादधि हव्यवाहम् शमीगर्भा{-}ज्जनयन.यो मयो & TB\_1.2.1.16       \\
    
    \hline
        
    15 & अग्निमादधानो दशहोत्रा ऽरणिमवदद्ध्यात् प्रजातमेवैन{-}माधत्ते & TB\_2.2.1.6       \\
    
    \hline
        
    16 & अग्निमुखा ह्यृद्धिः अग्निमुखामेवर्द्धिं ॅयजमान & TB\_3.3.8.9       \\
    
    \hline
        
    17 & अग्निमुखा{-}नेवर्तून् प्रीणाति उपबर्.हणं ददाति & TB\_1.1.6.10       \\
    
    \hline
        
    18 & अग्निरसि पृथिव्याꣳ श्रितः अन्तरिक्षस्य & TB\_3.11.1.7       \\
    
    \hline
        
    19 & अग्निरित्याह अग्निर्वै रेतोधाः रेत & TB\_3.7.3.7       \\
    
    \hline
        
    20 & अग्निरेवैनं गार्.हपत्येनावति इन्द्र इन्द्रियेण & TB\_1.7.6.6       \\
    
    \hline
        
    21 & अग्निर्{-}मूर्द्धा दिवः ककुत् पतिः & TB\_3.5.7.1       \\
    
    \hline
        
    22 & अग्निर्{-}यज्ञ्स्य हव्यवाट् तꣳ सबाधो & TB\_3.6.1.4       \\
    
    \hline
        
    23 & अग्निर्{-}वृत्राणि जङ्घनत् द्रविणस्युर्{-}विपन्यया समिद्धः & TB\_3.5.6.1       \\
    
    \hline
        
    24 & अग्निर्. हव्यवाडिह तानावहतु पौर्णमासं & TB\_3.7.4.4       \\
    
    \hline
        
    25 & अग्निर्. होता गृहपतिः स & TB\_3.5.12.2       \\
    
    \hline
        
    26 & अग्निर्. होता नो अद्ध्वरे & TB\_3.6.4.1       \\
    
    \hline
        
    27 & अग्निर्. होता पञ्चहोतॄणां वाग्घोता & TB\_3.12.5.2       \\
    
    \hline
        
    28 & अग्निर्. होता वेत्वग्निः होत्रं & TB\_3.5.4.1       \\
    
    \hline
        
    29 & अग्निर्नः पातु कृत्तिकाः नक्षत्रं & TB\_3.1.1.1       \\
    
    \hline
        
    30 & अग्निर्वा अकामयत अन्नादो देवानां & TB\_3.1.4.1       \\
    
    \hline
        
    31 & अग्निर्वाव पवित्रम् वृजिनमनृतं दुश्चरितम् & TB\_3.3.7.10       \\
    
    \hline
        
    32 & अग्निर्वाव संॅवथ्सरः आदित्यः परिवथ्सरः & TB\_1.4.10.1       \\
    
    \hline
        
    33 & अग्निर्विशां मानुषीणाम् तूर्णी रथः & TB\_2.4.8.2       \\
    
    \hline
        
    34 & अग्निर्वै देवानां ॅयष्टा य & TB\_3.3.7.6       \\
    
    \hline
        
    35 & अग्निवान्. वै दर्भस्तम्बः अग्निवत्येव & TB\_2.2.1.5       \\
    
    \hline
        
    36 & अग्निश्च विष्णो तप उत्तमं & TB\_2.4.3.4       \\
    
    \hline
        
    37 & अग्निष्टोममग्र आहरति यज्ञ्मुखं ॅवा & TB\_1.8.7.1       \\
    
    \hline
        
    38 & अग्निहोत्रमेवन दुह्याच्छूद्रः तद्धि नोत्{-}पुनन्ति & TB\_3.2.3.10       \\
    
    \hline
        
    39 & अग्ने दीद्यतं बृहत् अग्निं & TB\_3.5.2.3       \\
    
    \hline
        
    40 & अग्ने महाꣳ असि ब्राह्मण & TB\_3.5.3.1       \\
    
    \hline
        
    41 & अग्ने यो नोऽभिदासति समानो & TB\_3.7.6.17       \\
    
    \hline
        
    42 & अग्ने सपत्नाꣳ अप बाधमानः & TB\_1.2.1.21       \\
    
    \hline
        
    43 & अग्ने हिरण्यसंदृशः अदब्धेभिः सवितः & TB\_2.4.4.7       \\
    
    \hline
        
    44 & अग्नेः कृत्तिकाः शुक्रं परस्ता{-}ज्ज्योति{-}रवस्तात् & TB\_1.5.1.1       \\
    
    \hline
        
    45 & अग्नेः प्रियतमꣳ हविः स्वाहा & TB\_3.7.14.3       \\
    
    \hline
        
    46 & अग्नेर्भस्मास्यग्नेः पुरीषमसि सज्ञांनमसि कामधरणम् & TB\_1.2.1.17       \\
    
    \hline
        
    47 & अङ्गारा भवन्ति तेभ्योऽङ्गारेभ्योऽर्चिरुदेति प्रजापतिस्तर्ह्यग्निः & TB\_2.1.10.3       \\
    
    \hline
        
    48 & अङ्गिरसो वै सत्रमासत तेषां & TB\_2.1.1.1       \\
    
    \hline
        
    49 & अच्छा देव विवाससि बृहदग्ने & TB\_3.5.2.2       \\
    
    \hline
        
    50 & अच्छिन्नं तन्तुं पयसा सरस्वती & TB\_2.6.8.4       \\
    
    \hline
        
    51 & अजस्य तु नाश्ञीयात् यदजस्या{-}श्ञीयात् & TB\_3.7.3.2       \\
    
    \hline
        
    52 & अजैदग्निः असनद्वाजं नि देवो & TB\_3.6.5.1       \\
    
    \hline
        
    53 & अजो वा एकपादकामयत तेजस्वी & TB\_3.1.5.10       \\
    
    \hline
        
    54 & अञ्जन्ति त्वामद्ध्वरे देवयन्तः वनस्पते & TB\_3.6.1.1       \\
    
    \hline
        
    55 & अति प्रयच्छं दुरितिं तरेयम् & TB\_1.2.1.5       \\
    
    \hline
        
    56 & अतिग्राह्याः परः सामसु इमानेवैतैः & TB\_1.2.2.4       \\
    
    \hline
        
    57 & अतिरिक्तस्य शान्त्यै बण्महाꣳ असि & TB\_1.4.5.3       \\
    
    \hline
        
    58 & अतिरिक्ताः फलीकरणाः अतिरिक्तमाज्योच्छेषणम् अतिरिक्त & TB\_3.3.9.10       \\
    
    \hline
        
    59 & अत्र वाव स निलयते & TB\_1.4.7.3       \\
    
    \hline
        
    60 & अथ केनाज्यमिति सत्येनेति ब्रूयात् & TB\_3.3.5.2       \\
    
    \hline
        
    61 & अथ गृहमेधिन{-}माप्नोति यदा गृहमेधिन{-}माप्नोति & TB\_1.4.10.5       \\
    
    \hline
        
    62 & अथ द्वे अथ त्रीणि & TB\_3.2.7.4       \\
    
    \hline
        
    63 & अथ यथ् सम्ॅवाति तदस्य & TB\_3.11.7.2       \\
    
    \hline
        
    64 & अथ यथ् सुवर्णरजताभ्यां कुशीभ्यां & TB\_1.5.10.2       \\
    
    \hline
        
    65 & अथ यदाह पवित्रं पवयिष्यन्{-}थ्सहस्वान्{-}थ्सहीयानरुणो & TB\_3.10.9.8       \\
    
    \hline
        
    66 & अथ यदाह प्रस्तुतं ॅविष्टुतं & TB\_3.10.9.7       \\
    
    \hline
        
    67 & अथ यदुत्तरतो वाति सवितैव & TB\_2.3.9.7       \\
    
    \hline
        
    68 & अथ योऽविद्वान् प्रतिगृह्णाति पञ्चममस्येन्द्रियस्याप & TB\_2.3.4.4       \\
    
    \hline
        
    69 & अथ वयं ॅवेदाम अस्मभ्यमेव & TB\_1.6.7.5       \\
    
    \hline
        
    70 & अथ वीर्यावत्तरो भवति अथ & TB\_2.7.3.2       \\
    
    \hline
        
    71 & अथ स्रुचावनुष्टुग्भ्यां ॅवाजवतीभ्यां ॅव्यूहति & TB\_3.3.9.1       \\
    
    \hline
        
    72 & अथर्वाङ्गिरसश्च ये सर्वास्ताः इतिहासपुराणञ्च & TB\_3.12.8.2       \\
    
    \hline
        
    73 & अथाग्निम् अथाग्निहोत्रम् यदाज्यं पुरस्ताद्धरति & TB\_1.4.4.4       \\
    
    \hline
        
    74 & अथादित्यो वरुणꣳ राजानं ॅवरुण & TB\_1.4.10.6       \\
    
    \hline
        
    75 & अथास्मै नाम गृह्य प्रहरति & TB\_3.3.11.3       \\
    
    \hline
        
    76 & अथेतर ऐन्द्रः पुरोडाशः स्यात् & TB\_3.7.1.7       \\
    
    \hline
        
    77 & अथेममन्थन्नमृतममूराः वैश्वानरं क्षेत्रजित्याय देवाः & TB\_2.4.6.8       \\
    
    \hline
        
    78 & अथैतं ॅविष्णवे चरुं निर्वपति & TB\_3.1.6.7       \\
    
    \hline
        
    79 & अथैतत्{-}पौर्णमास्या आज्यं निर्वपति कामो & TB\_3.1.4.15       \\
    
    \hline
        
    80 & अथैतद{-}मावास्याया आज्यं निर्वपति कामो & TB\_3.1.5.15       \\
    
    \hline
        
    81 & अथैतन्{-}मनुर्{-}वप्त्रे मिथुन{-}मपश्यत् स श्मश्रूण्यग्रेऽवपत & TB\_1.5.6.3       \\
    
    \hline
        
    82 & अथैतमदित्यै चरुं निर्वपति इयं & TB\_3.1.6.6       \\
    
    \hline
        
    83 & अथैतस्मै नक्षत्राय चरुं निर्वपति & TB\_3.1.6.4       \\
    
    \hline
        
    84 & अथैतानरूपेभ्य आलभते अतिह्रस्व{-}मतिदीर्घम् अतिकृशमत्यꣳसलम् & TB\_3.4.19.1       \\
    
    \hline
        
    85 & अथो अर्द्धो वा एष & TB\_3.3.3.5       \\
    
    \hline
        
    86 & अथो तर्पयत्येव तृप्यति प्रजया & TB\_1.6.9.10       \\
    
    \hline
        
    87 & अथो धुवन्त्येवैनम् अथो न्येवास्मै & TB\_1.4.6.7 TB\_3.9.6.2       \\
    
    \hline
        
    88 & अथो परैव भवति अथ & TB\_1.5.6.2       \\
    
    \hline
        
    89 & अथो माद्ध्यन्दिनमेव सवनं तेनाप्नोति & TB\_3.8.12.2       \\
    
    \hline
        
    90 & अथो य एव कश्च & TB\_2.7.2.2       \\
    
    \hline
        
    91 & अथो यथाऽग्निꣳ स्विष्टकृतं ॅयजति & TB\_1.6.9.7       \\
    
    \hline
        
    92 & अथो यद्{-}वेदश्च वेदिश्च भवतः & TB\_3.3.9.11       \\
    
    \hline
        
    93 & अथो या अमुष्मिन् ॅलोके & TB\_3.11.10.2       \\
    
    \hline
        
    94 & अथो रक्षसामपहत्यै परिधीन्थ्{-}संमार्ष्टि पुनात्येवैनान् & TB\_3.3.7.4       \\
    
    \hline
        
    95 & अथो रक्षसामपहत्यै शुन्धद्ध्वं दैव्याय & TB\_3.2.5.5       \\
    
    \hline
        
    96 & अथो शत्वांय सरेता अग्निराधेय & TB\_1.1.3.8       \\
    
    \hline
        
    97 & अथो हारिद्रवेषु मे हरिमाणं & TB\_3.7.6.23       \\
    
    \hline
        
    98 & अथोत्तरस्मै हविषे वथ्सानपा कुर्यात् & TB\_3.7.1.6       \\
    
    \hline
        
    99 & अथौषधय इमं देवं त्र्यम्बकै{-}रयजन्त & TB\_1.4.10.9       \\
    
    \hline
        
    100 & अदितिरूत्या{-}ऽऽगमत् सा शन्ताची मयस्करत् & TB\_3.7.10.5       \\
    
    \hline
        
    101 & अदित्यै रास्नाऽसीत्याह इयं ॅवा & TB\_3.2.2.7       \\
    
    \hline
        
    102 & अदित्यै स्वाहाऽदित्यै मह्यै स्वाहाऽदित्यै & TB\_3.8.11.2       \\
    
    \hline
        
    103 & अद्ध्वरेषु नमस्यत अनाम्योज आचके & TB\_2.4.6.4       \\
    
    \hline
        
    104 & अद्ध्वर्युं च यजमानं च & TB\_3.2.4.4       \\
    
    \hline
        
    105 & अद्भिरेवैनदाप्नोति यो वै यज्ञ्स्यार्तेनानार्तं & TB\_1.4.3.4       \\
    
    \hline
        
    106 & अद्र्धमासाः स्थ मासु श्रिताः & TB\_3.11.1.17       \\
    
    \hline
        
    107 & अधरे मथ्सपत्नाः इयꣳ स्थाली & TB\_3.7.6.11       \\
    
    \hline
        
    108 & अधा ते विष्णो विदुषाचिदृध्यः & TB\_2.4.3.9       \\
    
    \hline
        
    109 & अधातामश्विना मधु भेषजं भिषजा & TB\_2.6.12.2       \\
    
    \hline
        
    110 & अधिषवणमसि वानस्पत्यमित्याह अधिषवण{-}मेवैनत् करोति & TB\_3.2.5.7       \\
    
    \hline
        
    111 & अनडुह्येव वीर्यं दधाति तस्मात्पुरा & TB\_3.8.13.2       \\
    
    \hline
        
    112 & अनड्वाह{-}मग्नीधे वह्निर्वा अनड्वान् वह्निरग्नीत् & TB\_1.8.2.5       \\
    
    \hline
        
    113 & अनन्तꣳ ह वा अपार{-}मक्षय्यं & TB\_3.11.7.5       \\
    
    \hline
        
    114 & अनयैवैनत्{-}प्रति गृह्णाति वैश्वानर्यार्चा रथं & TB\_2.2.5.4       \\
    
    \hline
        
    115 & अनागसस्त्वा वयं इन्द्रेण प्रेषिता & TB\_3.7.9.1       \\
    
    \hline
        
    116 & अनाग्नेयं ॅवा एतत् क्रियते & TB\_1.3.1.3       \\
    
    \hline
        
    117 & अनिरुक्तः प्रजापतिः प्रजापतेराप्त्यै एकयर्चा & TB\_1.8.5.6       \\
    
    \hline
        
    118 & अनु वथ्सान्. वासयन्ति भ्रातृव्यायैव & TB\_1.6.7.3       \\
    
    \hline
        
    119 & अनु सोमो अन्वग्निरावीत् अनु & TB\_2.7.8.2       \\
    
    \hline
        
    120 & अनुमत्यै पुरोडाश{-}मष्टाकपालं निर्वपति ये & TB\_1.6.1.1       \\
    
    \hline
        
    121 & अनुष्टुप्छन्द इन्द्रियम् त्रिवथ्सो गौर्वयो & TB\_2.6.18.2       \\
    
    \hline
        
    122 & अनेनाश्वेन मेद्ध्ये नेष्ट्वा अयं & TB\_3.8.5.4       \\
    
    \hline
        
    123 & अन्तत एव निर्.ऋतिं निरवदयते & TB\_1.6.1.4       \\
    
    \hline
        
    124 & अन्तरिक्ष एव ज्योतिर्द्धत्ते आदित्यमेवामुष्मिन् & TB\_3.2.7.2       \\
    
    \hline
        
    125 & अन्तरिक्ष{-}मस्यग्नौ श्रितं वायोः प्रतिष्ठा & TB\_3.11.1.8       \\
    
    \hline
        
    126 & अन्तरिक्ष{-}मुक्थ्येन स्वरतिरात्रेण सर्वान् ॅलोकानहीनेन & TB\_3.12.5.7       \\
    
    \hline
        
    127 & अन्तरिक्षं कृत्यधीवासेन दिवꣳ हिरण्यकशिपुना & TB\_3.9.20.3       \\
    
    \hline
        
    128 & अन्तरिक्षं ॅविपप्रथे दुहे द्यौः & TB\_2.4.6.9       \\
    
    \hline
        
    129 & अन्तरिक्षं ॅवै मातरिश्वनो घर्मः & TB\_3.2.3.2       \\
    
    \hline
        
    130 & अन्तरिक्षचरञ्च यत् सर्वास्ताः अग्निं & TB\_3.12.7.5       \\
    
    \hline
        
    131 & अन्तरिक्षमुपभृत् पृथिवी ध्रुवा इमे & TB\_3.3.1.2       \\
    
    \hline
        
    132 & अन्तरिक्षस्य पोषेण सर्वपशु{-}मादधे अजीजनन्{-}नमृतं & TB\_1.2.1.19       \\
    
    \hline
        
    133 & अन्तरिक्षादेवैन{-}मपहन्ति तृतीयꣳ हरति दिव & TB\_3.2.9.6       \\
    
    \hline
        
    134 & अन्तर्दूतश्चरति मानुषीषु चतुः शिखण्डा & TB\_3.7.6.4       \\
    
    \hline
        
    135 & अन्तो वा एष यज्ञ्स्य & TB\_2.2.6.1       \\
    
    \hline
        
    136 & अन्धो जागृविः प्राण असावेहि & TB\_3.10.8.3 TB\_3.11.5.4       \\
    
    \hline
        
    137 & अन्नं ॅविराट् विराजैवान्नाद्यमव रुन्धे & TB\_1.8.2.2       \\
    
    \hline
        
    138 & अन्नं ॅवै पूषा अन्नमेवाव & TB\_1.7.3.6       \\
    
    \hline
        
    139 & अन्नं ॅवै मरुतः अन्नमेवाव{-}रुन्धे & TB\_1.7.3.5       \\
    
    \hline
        
    140 & अन्नपते ऽन्नस्य नो देहि & TB\_3.11.4.1       \\
    
    \hline
        
    141 & अन्नमेवाव रुन्धे तेन मे & TB\_1.1.8.6       \\
    
    \hline
        
    142 & अन्नमेवाव{-}रुन्धते मनसा प्रस्तौति मनसोद्गायति & TB\_2.2.6.2       \\
    
    \hline
        
    143 & अन्नस्यैव शमलेन शमलं ॅयजमाना{-}दपहन्ति & TB\_1.3.3.4       \\
    
    \hline
        
    144 & अन्यस्मै प्रयच्छति तादृगेव तत् & TB\_2.1.3.7       \\
    
    \hline
        
    145 & अन्या दक्षिणा नीयन्ते यज्ञ्स्य & TB\_3.3.8.11       \\
    
    \hline
        
    146 & अप पुन र्मृत्युं जयति & TB\_3.11.8.6       \\
    
    \hline
        
    147 & अप वा एतस्मा{-}त्प्राणाः क्रामन्ति & TB\_3.9.6.1       \\
    
    \hline
        
    148 & अप वा एतस्माच्छ्री राष्ट्रं & TB\_3.9.7.1 TB\_3.9.14.1       \\
    
    \hline
        
    149 & अप स्रिधः तदित्पदं न & TB\_3.7.10.6       \\
    
    \hline
        
    150 & अपः प्रणयति आपो वै & TB\_3.2.4.3       \\
    
    \hline
        
    151 & अपक्रामत गर्भिण्यः इस् ओन्ल्य् & TB\_2.3.8.1       \\
    
    \hline
        
    152 & अपनह्यामि ते बाहू अपनह्याम्यास्यम् & TB\_2.4.2.3       \\
    
    \hline
        
    153 & अपरिमितं गृह्णाति अपरिमितस्यावरुद्ध्यै तस्मिन् & TB\_3.3.6.7       \\
    
    \hline
        
    154 & अपशव्यो द्विरात्र इत्याहुः द्वे & TB\_1.8.10.3       \\
    
    \hline
        
    155 & अपां न पादाशुहेमन्निति संमार्ष्टि & TB\_1.3.5.4       \\
    
    \hline
        
    156 & अपां भूमानमुप नः सृजेह & TB\_2.5.8.12       \\
    
    \hline
        
    157 & अपां ॅयो द्रवणे रसः & TB\_2.7.7.7       \\
    
    \hline
        
    158 & अपानो विद्वा{-}नावृतः प्रति प्रातिष्ठ{-}दद्ध्वरे & TB\_3.12.9.4       \\
    
    \hline
        
    159 & अपापाचो अभिभूते नुदस्व अपोदीचो & TB\_2.4.1.3       \\
    
    \hline
        
    160 & अपो याचामि भेषजम् अफ्सु & TB\_2.5.8.6       \\
    
    \hline
        
    161 & अप्येव नोऽत्रास्त्विति तेभ्य एता & TB\_1.3.2.6       \\
    
    \hline
        
    162 & अप्रतिष्ठां ॅवा एते गच्छन्ति & TB\_1.2.5.1       \\
    
    \hline
        
    163 & अप्रतिष्ठितो वा एष इत्याहुः & TB\_1.8.10.1       \\
    
    \hline
        
    164 & अफ्सु वै वरुणः साक्षादेव & TB\_1.6.5.6       \\
    
    \hline
        
    165 & अफ्सुषदं त्वा घृतसद{-}मित्याह अपामेवैतेन & TB\_1.3.9.2       \\
    
    \hline
        
    166 & अबधिरो भवति य एवं & TB\_1.1.3.5       \\
    
    \hline
        
    167 & अभि नः शीयताꣳ रयिः & TB\_3.7.14.5       \\
    
    \hline
        
    168 & अभि प्रेहि वीरयस्व उग्रश्चेत्ता & TB\_2.7.16.1       \\
    
    \hline
        
    169 & अभिचरन्{-}दशहोतारं जुहुयात् नव वै & TB\_2.2.1.7       \\
    
    \hline
        
    170 & अभिप्रेहि वीरयस्व उग्रश्चेत्ता सपत्नहा & TB\_2.7.8.1       \\
    
    \hline
        
    171 & अभीके चिदु लोककृत् सङ्गे & TB\_2.5.8.2       \\
    
    \hline
        
    172 & अमीमदन्त पितरः अतीतृपन्त पितरः & TB\_2.6.3.3       \\
    
    \hline
        
    173 & अमुं तैः अनवरुद्धो वा & TB\_3.9.2.2       \\
    
    \hline
        
    174 & अमुष्मिन् ॅलोकेऽनुपरैति यदष्टावुप दधाति & TB\_3.2.7.5       \\
    
    \hline
        
    175 & अमूमिति नाम गृह्णाति भद्रमेवासां & TB\_3.2.3.7       \\
    
    \hline
        
    176 & अमृत एव सुवर्गे लोके & TB\_1.3.7.8       \\
    
    \hline
        
    177 & अमृन्मयं देवपात्रं यज्ञ्स्यायुषि प्रयुज्यतां & TB\_3.7.4.14       \\
    
    \hline
        
    178 & अम्भाꣳसि जुहोति अयं ॅवै & TB\_3.8.18.1       \\
    
    \hline
        
    179 & अयं प्राणश्चापानश्च यजमान{-}मपिगच्छतां यज्ञे & TB\_3.7.4.12       \\
    
    \hline
        
    180 & अयं ॅवा इदं परमोऽभूदिति & TB\_2.2.10.5       \\
    
    \hline
        
    181 & अयं ॅवाव यः पवते & TB\_3.11.7.1       \\
    
    \hline
        
    182 & अयं ॅवै लोक आवपनं & TB\_3.9.5.5       \\
    
    \hline
        
    183 & अयज्ञो वा एषः योऽपत्नीकः & TB\_3.3.3.1       \\
    
    \hline
        
    184 & अयस्पात्रेण वा दारुपात्रेण वाऽपिदधाति & TB\_3.2.3.12       \\
    
    \hline
        
    185 & अयाडग्नेः प्रिया धामानि अयाट्प्रजापतेः & TB\_3.5.7.6       \\
    
    \hline
        
    186 & अयाड्वनस्पतेः प्रिया पाथाꣳसि अयाड्देवाना{-}माज्यपानां & TB\_3.6.12.2       \\
    
    \hline
        
    187 & अरं ते सोमस्तनुवे भवाति & TB\_2.4.3.12       \\
    
    \hline
        
    188 & अरण्यान्यरण्यान्यसौ या प्रेव नश्यसि & TB\_2.5.5.6       \\
    
    \hline
        
    189 & अरातिमेवैनं तारयति असूषुदन्त यज्ञिया & TB\_1.7.4.4       \\
    
    \hline
        
    190 & अरुणं त्वा वृकमुग्रं खजं & TB\_2.7.15.6       \\
    
    \hline
        
    191 & अरुणो मिर्मिरस्त्रिशुक्रः एतद् वै & TB\_2.7.1.2       \\
    
    \hline
        
    192 & अरुणो ऽरुणरजाः पुण्डरीको विश्वजिद{-}भिजित् & TB\_3.10.1.4       \\
    
    \hline
        
    193 & अर्थेतः स्थेति जुहोति आहुत्यैवैना & TB\_1.7.5.1       \\
    
    \hline
        
    194 & अर्द्धमासानेव प्रीणाति अभिवान्यायै दुग्धे & TB\_1.6.8.4       \\
    
    \hline
        
    195 & अर्द्धमासेऽर्द्धमासे प्रवृज्यते यत् पुरोडाशः & TB\_3.2.8.5       \\
    
    \hline
        
    196 & अर्मेभ्यो हस्तिपम् जवायाश्वपम् पुष्ट्यै & TB\_3.4.9.1       \\
    
    \hline
        
    197 & अर्यमा राजाऽजरस्तुविष्मान् फल्गुनीना{-}मृषभो रोरवीति & TB\_3.1.1.8       \\
    
    \hline
        
    198 & अर्यमा वा अकामयत पशुमान्थ् & TB\_3.1.4.9       \\
    
    \hline
        
    199 & अर्यम्णो वा एतन्नक्षत्रम् यत्पूर्वे & TB\_1.1.2.4       \\
    
    \hline
        
    200 & अर्वाङ्यज्ञ्ः संक्रामत्वि{-}त्याप्ती{-}र्जुहोति सुवर्गस्य लोकस्याप्त्यै & TB\_3.8.17.3       \\
    
    \hline
        
    201 & अवधूतꣳ रक्षोऽवधूता अरातय इत्याह & TB\_3.2.6.1       \\
    
    \hline
        
    202 & अवभृथ निचङ्कुण निचेरुरसि निचङ्कुण & TB\_2.6.6.3       \\
    
    \hline
        
    203 & अवरुद्धेन व्यृद्ध्येत सर्वस्य समवदाय & TB\_1.3.8.2       \\
    
    \hline
        
    204 & अवव्ययन्नसितं देव वस्वः दविद्ध्वतो & TB\_2.4.5.5       \\
    
    \hline
        
    205 & अविदहन्तः श्रपयतेति वाचं ॅविसृजते & TB\_3.2.8.7       \\
    
    \hline
        
    206 & अविर्न मेषो नसि वीर्याय & TB\_2.6.4.5       \\
    
    \hline
        
    207 & अवीवृधत महो ज्यायोऽकृत देवा & TB\_3.5.10.4       \\
    
    \hline
        
    208 & अवीवृधत महो ज्यायोऽकृत प्रजापतिरिदं & TB\_3.5.10.3       \\
    
    \hline
        
    209 & अवेत्योऽवभृथा3 ना3 इति यद्{-}दर्भपुञ्जीलैः & TB\_2.7.9.5       \\
    
    \hline
        
    210 & अशनया मृत्युरेव तमेवामुष्मिन् ॅलोकेऽवयजते & TB\_3.9.15.2       \\
    
    \hline
        
    211 & अश्वशफेन द्वितीया{-}माहुतिं जुहोति पशवो & TB\_3.9.11.4       \\
    
    \hline
        
    212 & अश्वस्तोमीयꣳ हुत्वा द्विपदा जुहोति & TB\_3.9.12.3       \\
    
    \hline
        
    213 & अश्वस्य वा आलब्धस्य मेध & TB\_3.9.12.1       \\
    
    \hline
        
    214 & अश्विना हविरिन्द्रियम् नमुचेर्द्धिया सरस्वती & TB\_2.6.13.1       \\
    
    \hline
        
    215 & अश्विनेडा न भारती वाचा & TB\_2.6.11.7       \\
    
    \hline
        
    216 & अश्विनेन्द्राय भेषजम् शुक्रं न & TB\_2.6.11.5       \\
    
    \hline
        
    217 & अश्विनौ वा अकामयेताम् श्रोत्रस्विनावबधिरौ & TB\_3.1.5.13       \\
    
    \hline
        
    218 & अश्विनौ हि देवानामद्ध्वर्यू आस्ताम् & TB\_3.2.4.6       \\
    
    \hline
        
    219 & अष्टावुपभृति तस्मादष्टाशफा चतुर्द्ध्रुवायाम् तस्माच्चतुः & TB\_3.3.5.5       \\
    
    \hline
        
    220 & अष्टाशफाश्च य इहाग्ने ये & TB\_1.2.1.26       \\
    
    \hline
        
    221 & असतोऽधि मनोऽसृज्यत मनः प्रजापति{-}मसृजत & TB\_2.2.9.10       \\
    
    \hline
        
    222 & असवे स्वाहा वसवे स्वाहा & TB\_3.10.7.1       \\
    
    \hline
        
    223 & असावादित्य एकविꣳशः अमुमेवा{-}दित्यमाप्नोति शतं & TB\_3.12.5.8       \\
    
    \hline
        
    224 & असिक्नियस्योषधे निरितो नाशया पृषत् & TB\_2.4.4.2       \\
    
    \hline
        
    225 & असुर्यं ॅवा एतस्माद्{-}वर्णं कृत्वा & TB\_1.4.7.1       \\
    
    \hline
        
    226 & असृक्षि वा इममिति तस्य & TB\_2.2.4.2       \\
    
    \hline
        
    227 & असौ बृहत् अनयोरेवैन{-}मनन्तर्.हित{-}मभिषिञ्चति पशुस्तोमो & TB\_2.7.6.3       \\
    
    \hline
        
    228 & असौ भविष्यत् अनयोरेव लोकयोः & TB\_3.8.18.6       \\
    
    \hline
        
    229 & असौ वै जुहूः अन्तरिक्षमुपभृत् & TB\_3.3.6.11       \\
    
    \hline
        
    230 & अस्थि मज्जानं मासरैः कारोतरेण & TB\_2.6.4.2       \\
    
    \hline
        
    231 & अस्मभ्यं द्यावापृथिवी शक्वरीभिः राष्ट्रं & TB\_2.5.2.2       \\
    
    \hline
        
    232 & अस्मिन् ॅलोके बहवः कामा & TB\_3.9.3.3       \\
    
    \hline
        
    233 & अस्मिन्. यज्ञ् उप भूय & TB\_3.7.6.7       \\
    
    \hline
        
    234 & अस्मिन्नेव तेन लोके प्रति & TB\_1.6.3.2       \\
    
    \hline
        
    235 & अस्मिन्नेवैनं ॅलोके प्रतिष्ठित{-}माधत्ते वामदेव्य{-}मभिगायत & TB\_1.1.8.2       \\
    
    \hline
        
    236 & अस्मिꣳश्चा{-}मुष्मिꣳश्च यां कामयेत दुहितरं & TB\_1.5.2.3       \\
    
    \hline
        
    237 & अस्मै वै लोकाय ग्राम्याः & TB\_3.9.3.1       \\
    
    \hline
        
    238 & अस्या एवैनत् त्वचं करोति & TB\_3.2.5.6       \\
    
    \hline
        
    239 & अस्याः पवते इमामभि पवते & TB\_2.3.9.3       \\
    
    \hline
        
    240 & अस्याः पृथिव्या अद्ध्यक्षम् इममिन्द्र & TB\_2.4.7.2       \\
    
    \hline
        
    241 & अस्याजरासो दमा मरित्राः अर्चद्धूमासो & TB\_2.7.12.1       \\
    
    \hline
        
    242 & अस्यामेवैनाः प्रतिष्ठापयति उवाच ह & TB\_3.8.6.3       \\
    
    \hline
        
    243 & अह{-}न्नहिमन्वपस्ततर्द प्र वक्षणा अभिनत् & TB\_2.5.4.2       \\
    
    \hline
        
    244 & अहरेव तेन दक्षिण्यं कुरुते & TB\_2.1.5.3       \\
    
    \hline
        
    245 & अहर्नो अद्य सुविते दधातु & TB\_3.1.2.3       \\
    
    \hline
        
    246 & अहिर् बुद्ध्नियः प्रथमान एति & TB\_3.1.2.9       \\
    
    \hline
        
    247 & अहिर्वै बुद्ध्नियोऽकामयत इमां प्रतिष्ठां & TB\_3.1.5.11       \\
    
    \hline
        
    248 & अहोरात्रा{-}णीष्टकाः ऋषभोऽसि स्वर्गो लोकः & TB\_3.10.4.2       \\
    
    \hline
        
    249 & अहोरात्रे वा अकामयेताम् अत्यहोरात्रे & TB\_3.1.6.2       \\
    
    \hline
        
    250 & अहोरात्रे स्थोऽद्र्धमासेषु श्रिते भूतस्य & TB\_3.11.1.18       \\
    
    \hline
        
    251 & अह्रुतमसि हविर्द्धान{-}मित्याहानार्त्यै दृꣳहस्व मा & TB\_3.2.4.5       \\
    
    \hline
        
    252 & अꣳहोमुचः पितरः सोम्यासः परेऽवरेऽमृतासो & TB\_2.6.16.2       \\
    
    \hline
        
    253 & आ न एतु पुरश्चरम् & TB\_2.5.1.2       \\
    
    \hline
        
    254 & आ नो भर भगमिन्द्र & TB\_2.5.4.1       \\
    
    \hline
        
    255 & आ प्यायस्व सं ते & TB\_3.5.12.1       \\
    
    \hline
        
    256 & आ ब्रह्मन् ब्राह्मणो ब्रह्मवर्चसी & TB\_3.8.13.1       \\
    
    \hline
        
    257 & आ भरतꣳ शिक्षतं ॅवज्रबाहू & TB\_3.6.11.1       \\
    
    \hline
        
    258 & आ यस्ततन्थ रोदसी वि & TB\_3.6.10.5       \\
    
    \hline
        
    259 & आ वृत्रहणा वृत्रहभिः शुष्मैः & TB\_3.6.8.1       \\
    
    \hline
        
    260 & आ सिन्धोरा परावतः दक्षं & TB\_2.4.1.8       \\
    
    \hline
        
    261 & आ स्वधेत्या{-}श्रावयति अस्तु स्वधेति & TB\_1.6.9.5       \\
    
    \hline
        
    262 & आग्नेय{-}मष्टाकपालं निर्वपति तस्मा{-}च्छिशिरे कुरुपञ्चालाः & TB\_1.8.4.1       \\
    
    \hline
        
    263 & आग्नेयाः पशवः ऐन्द्रमहः नक्तं & TB\_1.1.4.3       \\
    
    \hline
        
    264 & आग्नेयी वै रात्रिः ऐन्द्रमहः & TB\_2.1.2.7       \\
    
    \hline
        
    265 & आघार{-}माघार्य ध्रुवाꣳ समनक्ति आत्मन्नेव & TB\_3.3.7.11       \\
    
    \hline
        
    266 & आचर्.षणिप्रा, विवेष यन्मा तं & TB\_2.6.9.1       \\
    
    \hline
        
    267 & आच्छेत्ता वो मा रिषं & TB\_3.7.4.10       \\
    
    \hline
        
    268 & आज्यस्य प्रतिपदं करोति प्राणो & TB\_3.8.15.2       \\
    
    \hline
        
    269 & आज्येन प्रत्यनज्म्येनत् तत्त आप्यायतां & TB\_3.7.5.6       \\
    
    \hline
        
    270 & आत्मनो गोपीथाय निर्णेनेक्ति शुद्ध्यै & TB\_2.1.4.8       \\
    
    \hline
        
    271 & आत्मन्{-}नात्मन्{-}नित्यामन्त्रयत तस्मै सप्तमꣳ हूतः & TB\_2.3.11.2       \\
    
    \hline
        
    272 & आत्मा हृदये हृदयं मयि & TB\_3.10.8.10       \\
    
    \hline
        
    273 & आदित्यमभि पवते आदित्यमभि सं & TB\_2.3.9.4       \\
    
    \hline
        
    274 & आदित्यां मल्.हां गर्भिणीमा लभते & TB\_1.8.3.2       \\
    
    \hline
        
    275 & आदित्यानां पत्वा{-}ऽन्विहीत्याह आदित्यानेवैनं गमयति & TB\_3.8.9.3       \\
    
    \hline
        
    276 & आदित्याश्चाङ्गिरसश्च सुवर्गे लोकेऽस्पद्र्धन्त तेऽङ्गिरस & TB\_3.9.21.1       \\
    
    \hline
        
    277 & आदित्योऽसि दिवि श्रितः चन्द्रमसः & TB\_3.11.1.11       \\
    
    \hline
        
    278 & आनन्दनन्दावाण्डौ मे भगः सौभाग्यं & TB\_2.6.5.6       \\
    
    \hline
        
    279 & आप ओषधीर्महयन्ति तादृगेव तत् & TB\_3.2.8.3       \\
    
    \hline
        
    280 & आपः स्थ समुद्रे श्रिताः & TB\_3.11.1.5       \\
    
    \hline
        
    281 & आपश्चौषधयश्च ऊर्क्च सूनृता च & TB\_3.7.7.9       \\
    
    \hline
        
    282 & आपो देवीरग्रेपुवो अग्रेगुव इत्याह & TB\_3.3.6.1       \\
    
    \hline
        
    283 & आपो वा अकामयन्त समुद्रं & TB\_3.1.5.4       \\
    
    \hline
        
    284 & आप्यायय हरिवो वर्द्धमानः यदा & TB\_3.7.11.5       \\
    
    \hline
        
    285 & आप्रीभिराप्नुवन्न् तदाप्रीणामाप्रित्वम् तमघ्नन्न् तस्य & TB\_2.2.8.6       \\
    
    \hline
        
    286 & आभ्यामेव प्रसूतो यजमानो वज्रं & TB\_1.7.6.7       \\
    
    \hline
        
    287 & आया हि सोमपीतये स्वारुहो & TB\_2.4.7.7       \\
    
    \hline
        
    288 & आयातु देवः सवितोपयातु हिरण्ययेन & TB\_3.1.1.9       \\
    
    \hline
        
    289 & आयुरसि तत् ते प्रयच्छामि & TB\_2.7.7.5       \\
    
    \hline
        
    290 & आयुरसि विश्वायुरसि सर्वायुरसि सर्वमायुरसि & TB\_2.7.7.6       \\
    
    \hline
        
    291 & आयुषः प्राणं सन्तनु प्राणादपानं & TB\_1.5.7.1       \\
    
    \hline
        
    292 & आयुषे वर्चसे जीवात्वै पुण्याय & TB\_3.7.6.16       \\
    
    \hline
        
    293 & आरे अस्मदमतिं बाधमानः उच्छ्रयस्व & TB\_3.6.1.2       \\
    
    \hline
        
    294 & आरोहतं दशतꣳ शक्वरीर् मम & TB\_1.2.1.14       \\
    
    \hline
        
    295 & आवेशय{-}न्निवेशयन्थ्सम्ॅवेशनः सꣳशान्तः शान्तः आभवन् & TB\_3.10.1.2       \\
    
    \hline
        
    296 & आशिता एवाद्योपवसाम कस्य वाऽहेदम् & TB\_1.6.6.4       \\
    
    \hline
        
    297 & आशिषमेवैतामाशास्ते पूर्णपात्रे अन्ततोऽनुष्टुभा चतुष्पद्वा & TB\_3.3.10.3       \\
    
    \hline
        
    298 & आहवनीय{-}मुपतिष्ठन्ते न्येवास्मै तद्ध्नुवते यथ्सत्याहवनीये & TB\_1.6.9.8       \\
    
    \hline
        
    299 & आऽयं भातु शवसा पञ्च & TB\_2.7.15.3       \\
    
    \hline
        
    300 & आऽस्यार्द्धं ॅवव्राज ताꣳ होदीक्ष्योवाच & TB\_2.3.10.3       \\
    
    \hline
        
    301 & इडा वै मानवी यज्ञानूकाशिन्यासीत् & TB\_1.1.4.4       \\
    
    \hline
        
    302 & इडे भागं जुषस्व नः & TB\_3.7.5.7       \\
    
    \hline
        
    303 & इतः प्रथमं जज्ञे अग्निः & TB\_1.4.4.8       \\
    
    \hline
        
    304 & इदं देवानामिदमु नः सहेत्याह & TB\_3.2.4.7       \\
    
    \hline
        
    305 & इदं द्यावापृथिवी भद्रमभूत् आर्द्ध्म & TB\_3.5.10.1       \\
    
    \hline
        
    306 & इदं ब्रह्म पुनीमहे यः & TB\_1.4.8.4       \\
    
    \hline
        
    307 & इदं ॅवा अग्रे नैव & TB\_2.2.9.1       \\
    
    \hline
        
    308 & इदं ॅविष्णुर् विचक्रम इति & TB\_2.7.14.3       \\
    
    \hline
        
    309 & इदमस्य चित्तमधरं ध्रुवायाः अहमुत्तरो & TB\_3.7.6.10       \\
    
    \hline
        
    310 & इदानीं तदानीमिति एते वै & TB\_3.10.10.5       \\
    
    \hline
        
    311 & इद्ध्मꣳ ह क्षुच् चैभ्य & TB\_3.12.9.6       \\
    
    \hline
        
    312 & इन्दु र्दक्षः श्येन ऋतावा & TB\_3.10.4.3       \\
    
    \hline
        
    313 & इन्द्रं ॅया देवी सुभगा & TB\_2.7.7.2       \\
    
    \hline
        
    314 & इन्द्रं ॅविश्वा अवीवृधन्न् समुद्र & TB\_2.7.16.3       \\
    
    \hline
        
    315 & इन्द्रं ॅवै स्वा विशो & TB\_2.7.18.1       \\
    
    \hline
        
    316 & इन्द्रः सप्तहोत्रा प्रजापतिर्{-}दशहोत्रा तेषां & TB\_2.2.8.5       \\
    
    \hline
        
    317 & इन्द्रः सुत्रामा हृदयेन सत्यम् & TB\_2.6.4.3       \\
    
    \hline
        
    318 & इन्द्रमर्केभिरर्किणः इन्द्रं ॅवाणीरनूषत इन्द्र & TB\_1.5.8.2       \\
    
    \hline
        
    319 & इन्द्रश्च नः शुनासीरौ इमं & TB\_2.4.5.7       \\
    
    \hline
        
    320 & इन्द्रश्च सम्राड् वरुणश्च राजा & TB\_3.7.9.7       \\
    
    \hline
        
    321 & इन्द्रस्य युज्यः सखा ब्रह्म & TB\_2.6.1.3       \\
    
    \hline
        
    322 & इन्द्रस्य रोहिणी शृणत्{-}परस्तात्{-}प्रतिशृण{-}दवस्तात् निर्.ऋत्यै{-}मूलवर्.हणी & TB\_1.5.1.4       \\
    
    \hline
        
    323 & इन्द्रस्य वज्रोऽसि वार्त्रघ्न इति & TB\_1.3.5.2 TB\_1.7.6.8 TB\_1.7.9.1       \\
    
    \hline
        
    324 & इन्द्रस्य सुषुवाणस्य दशधेन्द्रियं ॅवीर्यं & TB\_1.8.5.1       \\
    
    \hline
        
    325 & इन्द्रस्यैव सायुज्यꣳ सलोकता{-}माप्नोति य & TB\_3.10.11.7       \\
    
    \hline
        
    326 & इन्द्रागहि प्रथमो यज्ञियानाम् या & TB\_2.4.3.13       \\
    
    \hline
        
    327 & इन्द्राग्नी वा अकामयेताम् श्रैष्ठ्यं & TB\_3.1.4.14       \\
    
    \hline
        
    328 & इन्द्राय त्वा पयस्वते पयस्वन्तं & TB\_2.7.7.3       \\
    
    \hline
        
    329 & इन्द्रिय{-}मेवास्मि{-}न्नेतेन दधाति बृहस्पतेस्त्वा साम्राज्येनाभिषिञ्चा{-}मीत्याह & TB\_1.3.8.4       \\
    
    \hline
        
    330 & इन्द्रियं ॅवै गर्भः राष्ट्र{-}मेवेन्द्रियाव्यकः & TB\_1.8.3.3       \\
    
    \hline
        
    331 & इन्द्रियमेव यजमाने दधाति समारभ्योर्द्ध्वो & TB\_3.3.7.8       \\
    
    \hline
        
    332 & इन्द्रियमेव वीर्यमात्मन्{-}धत्ते पशूनां मन्युरसि & TB\_1.7.9.4       \\
    
    \hline
        
    333 & इन्द्रियमेवात्मन्{-}धत्ते यो वै चतुर्.होतॄ{-}ननुसवनं & TB\_2.2.8.3       \\
    
    \hline
        
    334 & इन्द्रियमेवाव रुन्धे ऋषभो वही & TB\_1.6.1.9       \\
    
    \hline
        
    335 & इन्द्रियस्यावरुद्ध्यै अनिरुक्ताभिः प्रातस्सवने स्तुवते & TB\_1.3.8.5       \\
    
    \hline
        
    336 & इन्द्रिये एवास्मै समीची दधाति & TB\_1.8.6.3       \\
    
    \hline
        
    337 & इन्द्रो जातो वि पुरो & TB\_2.4.7.6       \\
    
    \hline
        
    338 & इन्द्रो दधीचो अस्थभिः वृत्राण्य & TB\_1.5.8.1       \\
    
    \hline
        
    339 & इन्द्रो नयतु वृत्रहा यतो & TB\_3.3.11.4       \\
    
    \hline
        
    340 & इन्द्रो मरुद्भिः सखिभिः सह & TB\_1.5.5.3       \\
    
    \hline
        
    341 & इन्द्रो वा अकामयत ज्यैष्ठ्यं & TB\_3.1.5.2       \\
    
    \hline
        
    342 & इन्द्रो वा अकामयत दृढो & TB\_3.1.5.9       \\
    
    \hline
        
    343 & इन्द्रो वृत्रमहन्न् सोऽपः अभ्यम्रियत & TB\_3.2.5.1       \\
    
    \hline
        
    344 & इन्द्रो वृत्रꣳ हत्वा असुरान् & TB\_1.3.10.1 TB\_1.7.1.6       \\
    
    \hline
        
    345 & इन्द्रꣳ स दिशां देवं & TB\_3.11.5.2       \\
    
    \hline
        
    346 & इन्धानो अक्रो विदथेषु दीद्यत् & TB\_1.2.1.13       \\
    
    \hline
        
    347 & इमं तꣳ शुक्रं मधुमन्त{-}मिन्दुम् & TB\_2.6.3.2       \\
    
    \hline
        
    348 & इमं नो यज्ञ्ं ॅविहवे & TB\_2.4.3.3       \\
    
    \hline
        
    349 & इमं मे वरुण तत्त्वा & TB\_3.7.11.3       \\
    
    \hline
        
    350 & इमं ॅयज्ञ्मश्विना वर्द्धयन्ता इमौ & TB\_2.5.4.6       \\
    
    \hline
        
    351 & इमा धाना घृतस्नुवः हरी & TB\_2.4.3.10       \\
    
    \hline
        
    352 & इमानि हव्या प्रयता जुषाणा & TB\_3.1.2.10       \\
    
    \hline
        
    353 & इमामेव पूर्वया दुहे अमूमुत्तरया & TB\_2.1.5.5       \\
    
    \hline
        
    354 & इमामेवास्मा उत्थापयति आयुर्यज्ञ्पतावधादित्याह आयुरेवास्मिन् & TB\_1.4.3.2       \\
    
    \hline
        
    355 & इमे धेनू अमृतं ॅये & TB\_2.4.8.6       \\
    
    \hline
        
    356 & इमे वा एते लोका & TB\_1.1.8.1       \\
    
    \hline
        
    357 & इमे वै लोकाः परितस्थुषः & TB\_3.9.4.2       \\
    
    \hline
        
    358 & इयं ॅवाव सरघा तस्या & TB\_3.10.10.1       \\
    
    \hline
        
    359 & इयं ॅवै रजता असौ & TB\_1.8.9.1       \\
    
    \hline
        
    360 & इयं ॅवै सविता यो & TB\_3.9.13.2       \\
    
    \hline
        
    361 & इयतीर् भवन्ति यज्ञ्परुषा संमिताः & TB\_1.1.9.5       \\
    
    \hline
        
    362 & इयमेव सा या प्रथमा & TB\_2.5.5.3       \\
    
    \hline
        
    363 & इयमेवास्मै राज्यमनुमन्यते धेनुर्{-}दक्षिणा इमामेव & TB\_1.6.1.5       \\
    
    \hline
        
    364 & इष{-}मूर्जमस्मासु धत्तम् प्राणान् पशुषु & TB\_1.1.1.5       \\
    
    \hline
        
    365 & इषमेवोर्जं ॅयजमाने दधाति द्युमद्वदत & TB\_3.2.5.9       \\
    
    \hline
        
    366 & इष्टर्गः खलु वै पूर्वोऽर्ष्टुः & TB\_1.4.6.5       \\
    
    \hline
        
    367 & इष्टापूर्तयो र्मे ऽक्षितिं ब्रूहीति & TB\_3.11.8.5       \\
    
    \hline
        
    368 & इष्टापूर्ते{-}नैवैनꣳ स समद्र्धयति इत्यजिना & TB\_3.9.14.4       \\
    
    \hline
        
    369 & ईश्वरो वा एष दिशोऽनून्{-}मदितोः & TB\_1.8.3.1       \\
    
    \hline
        
    370 & उक्थ्यं कुर्वीत यद्युक्थ्यः स्यात् & TB\_1.4.6.4       \\
    
    \hline
        
    371 & उग्र उग्राभिरूतिभिः तमिन्द्रं ॅवाजयामसि & TB\_1.5.8.3       \\
    
    \hline
        
    372 & उग्रामा तिष्ठेत्याह इन्द्रियमेवाव रुन्धे & TB\_1.7.7.2       \\
    
    \hline
        
    373 & उच्छेषणादेव तद्{-}रेतो धत्ते अस्थि & TB\_1.1.9.4       \\
    
    \hline
        
    374 & उत द्विबर्.हा अमिनः सहोभिः & TB\_3.5.7.5       \\
    
    \hline
        
    375 & उत नः प्रिया प्रियासु & TB\_2.4.6.1       \\
    
    \hline
        
    376 & उत वा एषाऽश्वꣳ सूते & TB\_1.8.6.4       \\
    
    \hline
        
    377 & उतेव ग्राम्याः उतेवारण्याः अहरेव & TB\_3.9.9.2       \\
    
    \hline
        
    378 & उतो अरण्यानिः सायम् शकटीरिव & TB\_2.5.5.7       \\
    
    \hline
        
    379 & उत्तरत आयतनो वै होता & TB\_3.9.5.2       \\
    
    \hline
        
    380 & उत्तरत इतरान् पाप्मनः सचन्ते & TB\_2.3.9.9       \\
    
    \hline
        
    381 & उत्तरत उपास्यत्य{-}बीभथ्सायै अति प्रयच्छति & TB\_1.1.3.9       \\
    
    \hline
        
    382 & उत्तरस्यां ॅवेद्यामन्यानि हवीꣳषि सादयति & TB\_1.6.5.1       \\
    
    \hline
        
    383 & उत्तरां देवयज्या{-}माशास्ते भूयो हविष्करण{-}माशास्ते & TB\_3.5.10.5       \\
    
    \hline
        
    384 & उत्तरावतीं ॅवै देवा आहुतिमजुहवुः & TB\_2.1.4.1       \\
    
    \hline
        
    385 & उथ्सं दुहन्ति कलशं चतुर्बिलं & TB\_3.7.4.16       \\
    
    \hline
        
    386 & उथ्सादेभ्यः कुब्जम् प्रमुदे वामनम् & TB\_3.4.6.1       \\
    
    \hline
        
    387 & उदङ्परेत्याग्नीद्ध्रे जुहोति एषा वै & TB\_1.7.8.6       \\
    
    \hline
        
    388 & उदस्तांफ्सीथ् सविता मित्रो अर्यमा & TB\_3.7.10.1       \\
    
    \hline
        
    389 & उदस्थाद्{-}देव्यदितिर्{-}विश्वरूपी आयुर्यज्ञ्पतावधात् इन्द्राय कृण्वती & TB\_1.4.3.1       \\
    
    \hline
        
    390 & उदुज्जिहानो अभि काममीरयन्न् प्रपृञ्चन् & TB\_2.5.4.5       \\
    
    \hline
        
    391 & उदेहि वाजिन्यो अस्यफ्स्वन्तः इदं & TB\_2.5.2.1       \\
    
    \hline
        
    392 & उद्गातार{-}मपरुद्ध्य अश्व{-}मुद्गीथाय वृणीते यथा & TB\_3.8.22.2       \\
    
    \hline
        
    393 & उद्धन्ति यदेवास्या अमेद्ध्यम् तदप & TB\_1.1.3.1       \\
    
    \hline
        
    394 & उद्धन्यमान{-}मस्या अमेद्ध्यम् अप पाप्मानं & TB\_1.2.1.1       \\
    
    \hline
        
    395 & उद्धारं ॅवा एतमिन्द्र उदहरत & TB\_1.6.7.6       \\
    
    \hline
        
    396 & उद्यन्तं ॅवावादित्य{-}मग्निरनु समारोहति तस्माद्धूम & TB\_2.1.2.10       \\
    
    \hline
        
    397 & उद्यन्नद्य वि नो भज & TB\_3.7.6.22       \\
    
    \hline
        
    398 & उद्याने यत्परायणे आवर्तने विवर्तने & TB\_3.7.9.8       \\
    
    \hline
        
    399 & उप त्वा जामयो गिर & TB\_1.8.8.1       \\
    
    \hline
        
    400 & उप नः सूनवो गिरः & TB\_2.4.6.3       \\
    
    \hline
        
    401 & उपक्षरन्ति जुह्वो घृतेन प्रियाण्यङ्गानि & TB\_3.7.13.3       \\
    
    \hline
        
    402 & उपनिषदे सुप्रजास्त्वाय सं पत्नी & TB\_3.7.5.11       \\
    
    \hline
        
    403 & उपयामगृहीतोऽस्याश्विनं तेजः सारस्वतं ॅवीर्यम् & TB\_2.6.1.5       \\
    
    \hline
        
    404 & उपहूतो भक्षः सखा उप & TB\_3.5.8.2 TB\_3.5.13.2       \\
    
    \hline
        
    405 & उपहूतꣳ रथन्तरꣳ सह पृथिव्या & TB\_3.5.8.1 TB\_3.5.13.1       \\
    
    \hline
        
    406 & उपावसृजत्{-}त्मन्या समञ्जन्न् देवानां पाथ & TB\_3.6.3.5       \\
    
    \hline
        
    407 & उपैनमुत्तरो यज्ञो नमति रुद्रो & TB\_1.1.8.4       \\
    
    \hline
        
    408 & उपोह यद्विदथं ॅवाजिनो गूः & TB\_3.6.12.1       \\
    
    \hline
        
    409 & उभये वा एते प्रजापतेरद्ध्य{-}सृज्यन्त & TB\_1.4.1.1       \\
    
    \hline
        
    410 & उभा हि वाꣳ सुहवा & TB\_2.4.8.4       \\
    
    \hline
        
    411 & उभाभ्यां प्रातः न देवताभ्यः & TB\_2.1.2.11       \\
    
    \hline
        
    412 & उरूकं मन्यमानाः नेद्वस्तोके तनये & TB\_3.6.6.4       \\
    
    \hline
        
    413 & उलूखले मुसले यच्च शूर्पे & TB\_3.7.6.21       \\
    
    \hline
        
    414 & उशन्. ह वै वाजश्रवसः & TB\_3.11.8.1       \\
    
    \hline
        
    415 & उशन्तस्त्वा हवामह, आ नो & TB\_2.6.16.1       \\
    
    \hline
        
    416 & उषा वा अकामयत प्रिया & TB\_3.1.6.3       \\
    
    \hline
        
    417 & उषासा नक्ता बृहती बृहन्तम् & TB\_2.6.8.3       \\
    
    \hline
        
    418 & उषे यह्वी सुपेशसा विश्वे & TB\_2.6.18.3       \\
    
    \hline
        
    419 & ऊद्र्ध्वा दिक् बृहस्पति र्देवता & TB\_3.11.5.3       \\
    
    \hline
        
    420 & ऊर्ग्{-}राजान{-}मुदवहत् ध्रुव गोपः सहो{-}ऽभवत् & TB\_3.12.9.5       \\
    
    \hline
        
    421 & ऊर्जं नो धेहि द्विपदे & TB\_1.2.1.23       \\
    
    \hline
        
    422 & ऊर्जमेव तया यजमान इमां & TB\_1.4.1.6       \\
    
    \hline
        
    423 & ऊर्जा पृथिवीं गच्छतेत्याह पृथिव्यामेवोर्जं & TB\_3.3.6.5       \\
    
    \hline
        
    424 & ऊर्जो भागꣳ शतक्रतू एतद्वां & TB\_3.7.5.12       \\
    
    \hline
        
    425 & ऊर्द्ध्वे समिधावादधाति अनूयाजेभ्यः समिधमतिशिनष्टि & TB\_3.3.7.2       \\
    
    \hline
        
    426 & ऊर्मिमन्तमेवैनं करोति वृषसेनोऽसीत्याह सेनामेवास्य & TB\_1.7.5.2       \\
    
    \hline
        
    427 & ऋक्षा वा इयमलोमकाऽऽसीत् साऽकामयत & TB\_3.1.4.5       \\
    
    \hline
        
    428 & ऋक्षीकाः पुरुषव्याघ्राः परिमोषिण आव्याधिनी{-}स्तस्करा & TB\_3.9.1.3       \\
    
    \hline
        
    429 & ऋचां प्राची महती दिगुच्यते & TB\_3.12.9.1       \\
    
    \hline
        
    430 & ऋजुधैवैनममुं ॅलोकं गमयति चतुर्.होत्रा & TB\_2.2.8.2       \\
    
    \hline
        
    431 & ऋतं त्वा सत्येन परिषिञ्चामीति & TB\_2.1.11.1       \\
    
    \hline
        
    432 & ऋतमेव परमेष्ठि ऋतं नात्येति & TB\_1.5.5.1       \\
    
    \hline
        
    433 & ऋतवः खलु वै देवाः & TB\_1.3.10.5       \\
    
    \hline
        
    434 & ऋतवः स्थ सम्ॅवथ्सरे श्रिताः & TB\_3.11.1.15       \\
    
    \hline
        
    435 & ऋतूनाꣳ संतत्यै प्रैवैभ्यः पूर्वया & TB\_1.6.9.4       \\
    
    \hline
        
    436 & ऋतूनेव प्रीणाति न पत्न्यन्वास्ते & TB\_1.6.9.11       \\
    
    \hline
        
    437 & ऋतेन द्यावापृथिवी ऋतेन त्वं & TB\_3.7.12.2       \\
    
    \hline
        
    438 & ऋत्यै स्तेनहृदयम् वैरहत्याय पिशुनम् & TB\_3.4.7.1       \\
    
    \hline
        
    439 & ऋद्ध्यामेव तद्{-}वीर्य एषु लोकेषु & TB\_1.5.6.7       \\
    
    \hline
        
    440 & ऋद्ध्यास्म हव्यैर्{-}नमसोप सद्य मित्रं & TB\_3.1.2.1       \\
    
    \hline
        
    441 & ऋद्ध्यै स्वाहा समृद्ध्यै स्वाहा & TB\_3.7.11.4       \\
    
    \hline
        
    442 & ऋष्वा त इन्द्र स्थविरस्य & TB\_2.7.13.4       \\
    
    \hline
        
    443 & एकं मास{-}मुदसृजत् परमेष्ठी प्रजाभ्यः & TB\_1.5.5.6       \\
    
    \hline
        
    444 & एकयूपो वैकादशिनी वा अन्येषां & TB\_3.8.19.1       \\
    
    \hline
        
    445 & एकवदेव सुवर्गं ॅलोकमेति सन्ततं & TB\_3.8.16.2       \\
    
    \hline
        
    446 & एकविꣳश एष भवति एतेन & TB\_1.2.4.1       \\
    
    \hline
        
    447 & एकविꣳशतिं ॅवैश्वदेवानि जुहोति एकविꣳशतिर्वै & TB\_3.8.10.3       \\
    
    \hline
        
    448 & एकविꣳशोऽग्निर्भवति एकविꣳशः स्तोमः एकविꣳशतिर्यूपाः & TB\_3.8.21.1       \\
    
    \hline
        
    449 & एका हि पितृणाम् दक्षिणोपमन्थति & TB\_1.6.8.5       \\
    
    \hline
        
    450 & एकादश यूपाः यद्द्वादशो{-}ऽग्निर्भवति द्वादश & TB\_3.8.21.2       \\
    
    \hline
        
    451 & एकैको वै जनताया{-}मिन्द्रः एकं & TB\_1.4.6.1       \\
    
    \hline
        
    452 & एतद् वै ब्राह्मणं पुरा & TB\_1.3.10.3       \\
    
    \hline
        
    453 & एतद्{-}ब्राह्मणान्येव पञ्च हवीꣳषि अथेन्द्राय & TB\_1.7.1.1       \\
    
    \hline
        
    454 & एतद्वा अग्नेः प्रियं धाम & TB\_1.1.9.6       \\
    
    \hline
        
    455 & एतद्वै देवानां परममन्नम् यन्नीवाराः & TB\_1.3.6.8       \\
    
    \hline
        
    456 & एतद्वै सावित्र{-}स्याष्टाक्षरं पदꣳ श्रियाभिषिक्तं & TB\_3.10.9.14       \\
    
    \hline
        
    457 & एतन् मनुष्याणाम् यथ् सुरा & TB\_1.3.3.3       \\
    
    \hline
        
    458 & एता वा अश्वस्य बन्धनम् & TB\_3.8.9.4       \\
    
    \hline
        
    459 & एतानि वाव तानि ज्योतीꣳषि & TB\_1.5.11.2       \\
    
    \hline
        
    460 & एतान्. योऽद्ध्यै{-}त्यछदिर्दर्.शे यावत्तरसं स्वरेति & TB\_3.12.5.3       \\
    
    \hline
        
    461 & एतावान्. वै यज्ञ्ः यावान्{-}पवमानाः & TB\_1.8.7.2       \\
    
    \hline
        
    462 & एते वै देवा गृहपतयः & TB\_2.3.5.4       \\
    
    \hline
        
    463 & एते वै प्रजापतेर्दोहाः य & TB\_2.2.9.9       \\
    
    \hline
        
    464 & एतैः प्रचरति यज्ञ्स्याघाताय एकधा & TB\_1.3.4.5       \\
    
    \hline
        
    465 & एतैरधिवादमपाजयत् अथो विश्वं पाप्मानं & TB\_3.12.5.4       \\
    
    \hline
        
    466 & एनश्चकार यत्पिता अग्निर्मा तस्मादेनसः & TB\_3.7.12.4       \\
    
    \hline
        
    467 & एभ्य एवैनं ॅलोकेभ्योऽनक्ति अभिपूर्वमनक्ति & TB\_3.3.9.3       \\
    
    \hline
        
    468 & एवं चतुर्थे पञ्चोत्तमेऽहन्नालभ्यन्ते वर्.षिष्ठमिव & TB\_2.7.11.2       \\
    
    \hline
        
    469 & एवमिव हि प्रजापतिः समृद्ध्यै & TB\_1.3.7.3       \\
    
    \hline
        
    470 & एवमेव स तेजसा यशसा & TB\_3.11.7.4       \\
    
    \hline
        
    471 & एवꣳ संॅवथ्सरस्य पक्षसी दिवाकीर्त्यमभि & TB\_1.2.3.2       \\
    
    \hline
        
    472 & एष गोसवः षट्त्रिꣳश उक्थ्यो & TB\_2.7.6.1       \\
    
    \hline
        
    473 & एष ते रुद्र भागः & TB\_1.6.10.4       \\
    
    \hline
        
    474 & एष वै प्रजापतिः यदग्निः & TB\_1.1.5.5       \\
    
    \hline
        
    475 & एष वै विभूर्नाम यज्ञ्ः & TB\_3.9.19.1       \\
    
    \hline
        
    476 & एष सम्ॅवथ्सरः अथ यदाह & TB\_3.10.9.9       \\
    
    \hline
        
    477 & एषा वै वरुणस्य दिक् & TB\_3.8.20.4       \\
    
    \hline
        
    478 & एषा हि विश्वेषां देवानां & TB\_3.3.4.6       \\
    
    \hline
        
    479 & एषां ॅलोकानामाप्त्यै त्रिः त्र्यावृद्धि & TB\_3.3.4.5       \\
    
    \hline
        
    480 & ऐन्द्र{-}मेकादशकपालꣳ राजन्यस्य गृहे इन्द्रियमेवाव & TB\_1.7.3.3       \\
    
    \hline
        
    481 & ऐन्द्रं पञ्चशराव{-}मोदनं निर्वपेत् अग्निं & TB\_3.7.1.8       \\
    
    \hline
        
    482 & ऐन्द्राग्नमुप सादितम् सर्वाभ्यो वा & TB\_2.1.8.3       \\
    
    \hline
        
    483 & ओज एवास्मिन् दधाति क्षत्राणां & TB\_1.7.8.5       \\
    
    \hline
        
    484 & ओजस्वदस्तु मे मुखम् ओजस्वच्छिरो & TB\_2.7.7.4       \\
    
    \hline
        
    485 & ओजोऽसि सहोऽसि बलमसि भ्राजोऽसि & TB\_3.11.1.21       \\
    
    \hline
        
    486 & ओदनमुद्ब्रुवते परमेष्ठी वा एषः & TB\_1.7.10.6       \\
    
    \hline
        
    487 & ओषधीनामहिꣳसायै यज्ञ्स्य घोषदसीत्याह यजमान & TB\_3.2.2.2       \\
    
    \hline
        
    488 & औक्षन्{-}घृतैरास्तृणन्{-}बर्.हिरस्मै आदिद्धोतारं न्यषादयन्त अग्निनाऽग्निः & TB\_2.7.12.3       \\
    
    \hline
        
    489 & कः स्विदेकाकी चरतीत्याह असौ & TB\_3.9.5.4       \\
    
    \hline
        
    490 & कथमिदꣳ स्यादिति सोऽपश्यत् पुष्करपर्णं & TB\_1.1.3.6       \\
    
    \hline
        
    491 & कनात्काभां न आभर प्रयफ्स्यन्निव & TB\_2.4.6.5       \\
    
    \hline
        
    492 & करदेवं देवो वनस्पतिः जुषतां & TB\_3.6.11.4       \\
    
    \hline
        
    493 & कर्म कर्मैवास्मै साधयति पौꣳश्चलेयो & TB\_3.8.4.2       \\
    
    \hline
        
    494 & कर्मणे वां देवेभ्यः शकेयमित्याह & TB\_3.2.4.1       \\
    
    \hline
        
    495 & कल्पेतां द्यावापृथिवी कल्पन्तामाप ओषधीः & TB\_1.2.1.18       \\
    
    \hline
        
    496 & कवष्यो न व्यचस्वतीः अश्विभ्यां & TB\_2.6.12.3       \\
    
    \hline
        
    497 & कश्चिद्धवा अस्मा{-}ल्लोकात् प्रेत्य आत्मानं & TB\_3.10.11.1       \\
    
    \hline
        
    498 & कस्मिन्नु तप इति बल & TB\_3.10.9.5       \\
    
    \hline
        
    499 & कान्ता काम्या कामजाता ऽऽयुष्मती & TB\_3.10.1.3       \\
    
    \hline
        
    500 & कामप्रीता एनं कामा अनुप्रयान्ति & TB\_3.7.1.2       \\
    
    \hline
        
    501 & कामेन समर्द्धयति अथ हैनं & TB\_3.11.9.2       \\
    
    \hline
        
    502 & कामो हि दाता कामः & TB\_2.2.5.6       \\
    
    \hline
        
    503 & किं प्रथमाꣳ रात्रि{-}माश्ना इति & TB\_3.11.8.4       \\
    
    \hline
        
    504 & कुसिन्धं चात्मनः स्पृणोति आदित्यस्य & TB\_2.3.7.2       \\
    
    \hline
        
    505 & कृत्तिका{-}स्वग्नि{-}मादधीत एतद्वा अग्नेर् नक्षत्रम् & TB\_1.1.2.1       \\
    
    \hline
        
    506 & कृष्णलं कृष्णलं ॅवाजसृद्भ्यः प्रयच्छति & TB\_1.3.6.7       \\
    
    \hline
        
    507 & कृष्णो रूपं कृत्वा यत् & TB\_3.2.6.2       \\
    
    \hline
        
    508 & केनर्तून{-}कल्पयन्तेति धात्रा वै ते & TB\_2.3.5.3       \\
    
    \hline
        
    509 & केशा न शीर.षन्. यशसे & TB\_2.6.4.6       \\
    
    \hline
        
    510 & क्यन्नो दास्यथेति यावथ्स्वयं परिगृह्णीथेति & TB\_3.2.9.7       \\
    
    \hline
        
    511 & क्रुङ्ङाङ्गिरसो धिया ऋतेन सत्यमिन्द्रियम् & TB\_2.6.2.2       \\
    
    \hline
        
    512 & क्रूरमिव वा एतत् करोति & TB\_3.2.9.13       \\
    
    \hline
        
    513 & क्लृता अस्मा ऋतव आयन्ति & TB\_2.3.2.3       \\
    
    \hline
        
    514 & क्व स्थ क्व वः & TB\_2.2.3.7       \\
    
    \hline
        
    515 & खाता हि देवानाम् मद्ध्यतो & TB\_1.6.8.6       \\
    
    \hline
        
    516 & गन्धर्वा{-}फ्सरसश्च ये सर्वास्ताः सर्वा{-}नुदारान्{-}थ्सलिलान् & TB\_3.12.7.2       \\
    
    \hline
        
    517 & गभस्तयो नियुतो विश्ववाराः अहरहर्भूय & TB\_2.7.13.5       \\
    
    \hline
        
    518 & गर्भꣳ स्रवन्तमगदमकः अग्नि{-}रिन्द्र{-}स्त्वष्टा बृहस्पतिः & TB\_3.7.3.6       \\
    
    \hline
        
    519 & गार्.हपत्यः प्रमुञ्चतु दुरिता यानि & TB\_3.7.12.6       \\
    
    \hline
        
    520 & गार्.हपत्यो वा अग्नेर्योनिः स्वादेवैनं & TB\_1.4.7.4       \\
    
    \hline
        
    521 & गिरा यज्ञ्स्य साधनम् श्रुष्टीवानं & TB\_2.4.2.5       \\
    
    \hline
        
    522 & गीताय सूतम् नृत्ताय शैलूषम् & TB\_3.4.2.1       \\
    
    \hline
        
    523 & गीर्भिर्विप्रः प्रमतिमिच्छमानः ई रयिं & TB\_3.6.9.1       \\
    
    \hline
        
    524 & गौतम कुमारमिति स होवाच & TB\_3.11.8.2       \\
    
    \hline
        
    525 & ग्रहो भवति दर्.शपूर्णमासयोः सृष्टयोर्द्धृत्यै & TB\_2.2.2.2       \\
    
    \hline
        
    526 & ग्राम्याꣳश्चा{-}रण्याꣳश्च उभयस्या{-}न्नाद्यस्या{-}वरुद्ध्यै उभयान् पशूनालभते & TB\_3.9.3.2       \\
    
    \hline
        
    527 & ग्रीष्मो दक्षिणः पक्षः वर्.षा & TB\_3.11.10.3       \\
    
    \hline
        
    528 & घर्मः शिरस्तदयमग्निः सं प्रियः & TB\_1.1.7.1       \\
    
    \hline
        
    529 & घर्मो वा एषोऽशान्तः अहरहः & TB\_2.1.3.2       \\
    
    \hline
        
    530 & घर्मोऽसि विश्वायुरित्याह विश्वमेवायुर् यजमाने & TB\_3.2.8.4       \\
    
    \hline
        
    531 & घासे अज्राणां ॅयवसप्रथमानाम् सुमत्क्षराणां & TB\_3.6.11.2       \\
    
    \hline
        
    532 & घृतं च वै मधु & TB\_3.3.4.1       \\
    
    \hline
        
    533 & घृतस्य धारया सुशेवं कल्पयामि & TB\_3.7.5.3       \\
    
    \hline
        
    534 & घृतैर् बोधयतातिथिम् आऽस्मिन्. हव्या & TB\_1.2.1.10       \\
    
    \hline
        
    535 & चक्षुर् वै सत्यम् अद्रा3गित्याह & TB\_1.1.4.2       \\
    
    \hline
        
    536 & चक्षुष आदित्यः तेषाꣳ हुतादजायत & TB\_2.1.6.2       \\
    
    \hline
        
    537 & चक्षुषो हेते मनसो हेते & TB\_2.4.2.1       \\
    
    \hline
        
    538 & चतस्रो दिशः दिग्भिरेवैनं परिगृह्णाति & TB\_3.8.12.3       \\
    
    \hline
        
    539 & चतुः शिखण्डा युवतिः सुपेशाः & TB\_1.2.1.27       \\
    
    \hline
        
    540 & चतुरुन्नयति चतुरक्षरꣳ रथन्तरम् रथन्तरस्यैष & TB\_2.1.5.7       \\
    
    \hline
        
    541 & चतुर्.होतॄन्. होता व्याचष्टे स्तुतमनुशꣳसति & TB\_2.2.6.3       \\
    
    \hline
        
    542 & चतुर्थमिन्द्रियस्यात्मन्{-}नुपाधत्ते य एवं ॅविद्वान् & TB\_2.3.4.3       \\
    
    \hline
        
    543 & चतुर्दशैता{-}ननुवाकाञ्जुहोत्य{-}नन्तरित्यै प्रयासाय स्वाहेति पञ्चदशम् & TB\_3.9.11.2       \\
    
    \hline
        
    544 & चतुर्विꣳश{-}त्यर्द्धमासः संॅवथ्सरः यद् वा & TB\_1.2.6.2       \\
    
    \hline
        
    545 & चतुश्शरावो भवति दिक्ष्वेव प्रतितिष्ठति & TB\_3.8.2.2       \\
    
    \hline
        
    546 & चतुष्टय्य आपो भवन्ति चतुश्शफो & TB\_3.8.2.1       \\
    
    \hline
        
    547 & चतुष्पादः पशवः पशुष्वे{-}वोपरिष्टात्{-}प्रतितिष्ठति यजमानदेवत्या & TB\_3.3.5.4       \\
    
    \hline
        
    548 & चतुष्पाद्भिः पशुभि{-}र्यजमानो व्यृद्ध्येत यत्र & TB\_3.7.2.5       \\
    
    \hline
        
    549 & चत्वार ऋत्विजः समुक्षन्ति आभ्य & TB\_3.8.5.1       \\
    
    \hline
        
    550 & चन्द्रमा अस्यादित्ये श्रितः नक्षत्राणां & TB\_3.11.1.12       \\
    
    \hline
        
    551 & चन्द्रमा वा अकामयत अहोरात्रा{-}नर्द्धमासान्{-}मासा{-}नृतून्थ्{-} & TB\_3.1.6.1       \\
    
    \hline
        
    552 & चन्द्रवानेव भवति तं देवा & TB\_2.2.10.4       \\
    
    \hline
        
    553 & चमसो देवपानः अराꣳ इवाग्ने & TB\_3.5.3.2       \\
    
    \hline
        
    554 & चित्तं मे सहः बाहू & TB\_2.6.5.5       \\
    
    \hline
        
    555 & छन्दाꣳसि वा आज्यम् छन्दाꣳस्येव & TB\_3.3.5.3       \\
    
    \hline
        
    556 & जगतीभिर्{-}वैश्यस्य जगती छन्दा वै & TB\_1.1.9.7       \\
    
    \hline
        
    557 & जगत्या तत् छन्दः संमितानि & TB\_3.2.7.6       \\
    
    \hline
        
    558 & जनिष्यमाणानेव प्रतिनुदते न्याहवनीयो गार्.हपत्य{-}मकामयत & TB\_1.1.5.6       \\
    
    \hline
        
    559 & जवमेवास्मिन्दधाति तस्मादश्वः पशूनामाशुः सारसारितमः & TB\_3.8.7.2       \\
    
    \hline
        
    560 & जातो जायते सुदिनत्वे अह्नाम् & TB\_3.6.1.3       \\
    
    \hline
        
    561 & जामि वा एतत् कुर्वन्ति & TB\_1.8.2.1       \\
    
    \hline
        
    562 & जीवात्वै पुण्याय अहं त्वदस्मि & TB\_1.2.1.20       \\
    
    \hline
        
    563 & जुषतां प्रति मेधिरः प्र & TB\_2.5.8.3       \\
    
    \hline
        
    564 & जुषस्व सप्रथस्तमम् वचो देवफ्सरस्तमम् & TB\_3.6.7.1       \\
    
    \hline
        
    565 & जुष्टी नरो ब्रह्मणा वः & TB\_2.4.3.1       \\
    
    \hline
        
    566 & जुष्टो दमूना अतिथिर्दुरोणे इमं & TB\_2.4.1.1       \\
    
    \hline
        
    567 & जेता शत्रून्. विचर्.षणिः आकूत्यै & TB\_2.5.3.2       \\
    
    \hline
        
    568 & ज्यैष्ठ्यं गच्छति योऽग्निं नाचिकेतं & TB\_3.11.9.7       \\
    
    \hline
        
    569 & ज्योग्जीवा जरामशीमहि इन्द्र शाक्वर & TB\_3.7.7.3       \\
    
    \hline
        
    570 & ज्योतिश्चैवास्मै राष्ट्रं च समीची & TB\_3.9.4.6       \\
    
    \hline
        
    571 & ज्योतिषा त्वा वैश्वानरेणोपतिष्ठे अयं & TB\_2.5.8.8       \\
    
    \hline
        
    572 & त आदारा अभवन्न् इन्द्रो & TB\_1.4.7.6       \\
    
    \hline
        
    573 & त आदित्या एतं पञ्चहोतार{-}मपश्यन्न् & TB\_2.2.3.6       \\
    
    \hline
        
    574 & त आवहन्ति कवयः पुरस्तादित्याह & TB\_3.2.2.3       \\
    
    \hline
        
    575 & त{-}दस्मै नै वाच्छदयत् त{-}दात्म{-}न्नेव & TB\_3.11.8.7       \\
    
    \hline
        
    576 & तं कामोऽब्रवीत् प्रजापते कामेन & TB\_3.12.2.3       \\
    
    \hline
        
    577 & तं कालेकाल आगते यजते & TB\_3.3.9.12       \\
    
    \hline
        
    578 & तं चतुर्.होत्रा प्राजनयन्न् यः & TB\_2.2.3.5       \\
    
    \hline
        
    579 & तं चरण{-}मब्रवीत् प्रजापते चरणेन & TB\_3.12.4.6       \\
    
    \hline
        
    580 & तं तपोऽ ब्रवीत् प्रजापते & TB\_3.12.4.2       \\
    
    \hline
        
    581 & तं त्वा भग सर्व & TB\_2.5.5.2       \\
    
    \hline
        
    582 & तं दक्षिणतो वेद्यै निधाय & TB\_1.4.6.6       \\
    
    \hline
        
    583 & तं पञ्चदशेनैव हरन्ति यावती & TB\_1.5.10.5       \\
    
    \hline
        
    584 & तं पूर्वपक्षे याजयेत् वसीयानेव & TB\_2.2.3.2       \\
    
    \hline
        
    585 & तं ब्रह्माऽब्रवीत् प्रजापते ब्रह्मणा & TB\_3.12.2.4       \\
    
    \hline
        
    586 & तं मनोऽब्रवीत् प्रजापते मनसा & TB\_3.12.4.5       \\
    
    \hline
        
    587 & तं सप्तदशेनाभि प्रास्तुवत तं & TB\_1.5.10.3       \\
    
    \hline
        
    588 & तं ॅयज्ञोऽब्रवीत् प्रजापते यज्ञेन & TB\_3.12.2.5       \\
    
    \hline
        
    589 & तं ॅयज्ञ्क्रतुभिर्नान्वविन्दन्न् तमिष्टिभि{-}रन्वैच्छन्न् तमिष्टिभि{-}रन्वविन्दन्न् & TB\_1.5.9.2       \\
    
    \hline
        
    590 & तच्छं ॅयोरा वृणीमहे गातुं & TB\_3.5.11.1       \\
    
    \hline
        
    591 & तच्छम्यै शमित्वम् यच्छमीमयः संभारो & TB\_1.1.3.12       \\
    
    \hline
        
    592 & तच्छ्रोणा यदशृणोत् तच्छ्रविष्ठाः यच्छत{-}मभिषज्यन्न् & TB\_1.5.2.9       \\
    
    \hline
        
    593 & ततः क्षरत्यक्षरम् तद्{-}विश्व{-}मुपजीवति इन्द्रा & TB\_2.4.6.12       \\
    
    \hline
        
    594 & ततो देवा अभवन्न् पराऽसुराः & TB\_1.6.6.5       \\
    
    \hline
        
    595 & ततो द्वितोऽजायत स तृतीय{-}मभ्यपातयत् & TB\_3.2.8.11       \\
    
    \hline
        
    596 & तत् पृथिव्यै पृथिवित्वम् अभूद्वा & TB\_1.1.3.7       \\
    
    \hline
        
    597 & तत् प्रतिप्रस्थाता करोति तस्माद्{-}यच्छ्रेयान्{-}करोति & TB\_1.6.5.2       \\
    
    \hline
        
    598 & तत्{-}पशव्यम् यज्जुहोति तद्ब्रह्मवर्चसि उभयमेवाकः & TB\_2.1.3.3       \\
    
    \hline
        
    599 & तत्कृत्वा अन्यां दुग्ध्वा पुनर्. & TB\_3.7.2.3       \\
    
    \hline
        
    600 & तथा त्वा सविता करत् & TB\_2.7.15.5       \\
    
    \hline
        
    601 & तथ् साकमेधानाꣳ साकमेधत्वम् अथर्तवः & TB\_1.4.10.8       \\
    
    \hline
        
    602 & तथ् सुवर्णꣳ हिरण्य{-}मभवत् यथ् & TB\_1.4.7.5       \\
    
    \hline
        
    603 & तदस्मिन्{-}नेकधाऽधात् रोहिण्यां कार्यः यद्ब्राह्मण & TB\_2.7.9.4       \\
    
    \hline
        
    604 & तदस्य प्रियमभि पाथो अश्याम् & TB\_2.4.6.2       \\
    
    \hline
        
    605 & तदाहुः अनाहुतयो वा अश्वचरितानि & TB\_3.8.8.2       \\
    
    \hline
        
    606 & तदाहुः द्वादश ब्रह्मौदनान् थ्सꣳस्थित & TB\_3.9.18.1       \\
    
    \hline
        
    607 & तदाहुः द्वादशारत्नी रशना कर्तव्या3 & TB\_3.8.3.3       \\
    
    \hline
        
    608 & तदिन्द्रं ॅयश आर्च्छत् तदेनं & TB\_2.3.2.5       \\
    
    \hline
        
    609 & तदृतं तथ् सत्यम् तद्{-}व्रतं & TB\_1.5.5.4       \\
    
    \hline
        
    610 & तदेतत् परि यद्देवचक्रं आर्द्रं & TB\_3.10.9.15       \\
    
    \hline
        
    611 & तदेतेनैव प्रत्यगृह्णात् तेन वै & TB\_2.3.4.2 TB\_2.3.4.6       \\
    
    \hline
        
    612 & तदेनं प्रकाशं गतम् प्रकाशं & TB\_2.2.6.4       \\
    
    \hline
        
    613 & तद् यथा ह वै & TB\_2.7.18.4       \\
    
    \hline
        
    614 & तद्{-}घृतमभवत् तस्माद्यस्य दक्षिणतः केशा & TB\_2.1.2.2       \\
    
    \hline
        
    615 & तद्{-}दुर्वर्णꣳ हिरण्यमभवत् तद्{-}दुर्वर्णस्य हिरण्यस्य & TB\_2.2.4.5       \\
    
    \hline
        
    616 & तद्{-}देवा विजित्य पुनरवारुरुथ्सन्त तेऽग्नये & TB\_1.1.6.2       \\
    
    \hline
        
    617 & तद्{-}वैश्वानरवत्{-}प्रजननवत्तर{-}मुपैतीति तदाहुः व्यृद्धं ॅवा & TB\_1.3.1.7       \\
    
    \hline
        
    618 & तद्ग्रहस्य ग्रहत्वम् पशुबन्धेन यक्ष्यमाणः & TB\_2.2.2.4       \\
    
    \hline
        
    619 & तद्ग्रावाणः सोमसुतो मयोभुवः तदश्विना & TB\_2.7.16.4       \\
    
    \hline
        
    620 & तद्धि रुद्रस्य भागधेयम् इमां & TB\_1.6.10.2       \\
    
    \hline
        
    621 & तद्वा अतः पवित्राभ्यामेवोत् पुनाति & TB\_3.3.4.4       \\
    
    \hline
        
    622 & तनुवा अहमिति वायुः चक्षुषोऽहमित्यादित्यः & TB\_2.1.6.3       \\
    
    \hline
        
    623 & तन् मृत्युर् निर्.ऋत्या सम्ॅविदानः & TB\_2.4.2.2       \\
    
    \hline
        
    624 & तन्निदधाति प्रतिष्ठित्यै पितॄन्. वा & TB\_1.8.6.6       \\
    
    \hline
        
    625 & तन्नो वायस्तदु निष्ट्या शृणोतु & TB\_3.1.1.11       \\
    
    \hline
        
    626 & तप आसीद् गृहपतिः ब्रह्म & TB\_3.12.9.3       \\
    
    \hline
        
    627 & तपसा देवा देवता{-}मग्र आयन्न् & TB\_3.12.3.1       \\
    
    \hline
        
    628 & तपसाऽस्यानुवर्तये शिवेनास्योपवर्तये शग्मेनास्या{-}भिवर्तये तदृतं & TB\_1.5.5.2       \\
    
    \hline
        
    629 & तपो मे प्रतिष्ठा सवितृ{-}प्रसूता & TB\_3.7.7.10       \\
    
    \hline
        
    630 & तपोऽसि लोके श्रितं तेजसः & TB\_3.11.1.2       \\
    
    \hline
        
    631 & तमग्नि र्बलिमानब्रवीत् प्रजापते ऽग्नये & TB\_3.12.2.7       \\
    
    \hline
        
    632 & तमनुवित्तिर{-}ब्रवीत् प्रजापते स्वर्गं ॅवै & TB\_3.12.2.8       \\
    
    \hline
        
    633 & तमवधिष्म पुनरिमꣳ सुवामहा इति & TB\_2.2.8.7       \\
    
    \hline
        
    634 & तमापो{-}ऽब्रुवन्न् प्रजापतेऽफ्सु वै सर्वे & TB\_3.12.2.6       \\
    
    \hline
        
    635 & तमाशा{-}ऽब्रवीत् प्रजापत आशया वै & TB\_3.12.2.2       \\
    
    \hline
        
    636 & तमुपोदतिष्ठन्त{-}मजुहवुः तेन संॅवथ्सर ऊर्जमवा{-}रुन्धत & TB\_1.4.9.3       \\
    
    \hline
        
    637 & तमुपोदतिष्ठन्त{-}मजुहवुः तेन{-}द्वयीमूर्जमवारुन्धत तस्माद्{-}द्विरह्नो{-}मनुष्येभ्य उपह्रियते & TB\_1.4.9.2       \\
    
    \hline
        
    638 & तमेभ्यः पुनरददुः तस्मात्{-}पितृभ्यः पूर्वेद्युः & TB\_1.3.10.2       \\
    
    \hline
        
    639 & तया त्वा दीक्षया दीक्षयामि & TB\_3.7.7.6       \\
    
    \hline
        
    640 & तया देवतया{-}ऽङ्गिरस्वद्{-}ध्रुवा सीद ता & TB\_3.11.6.2       \\
    
    \hline
        
    641 & तया सोमो राजा दीक्षया & TB\_3.7.7.7       \\
    
    \hline
        
    642 & तयाऽग्नि र्दीक्षया दीक्षितः ययाऽग्नि & TB\_3.7.7.5       \\
    
    \hline
        
    643 & तयोः पृष्ठे सीदतु जातवेदाः & TB\_1.2.1.24       \\
    
    \hline
        
    644 & तस्मा{-}त्पूर्वाग्निं पुरस्ता{-}थ्स्थापयन्ति पौष्णमन्वञ्चम् अन्नं & TB\_3.8.23.2       \\
    
    \hline
        
    645 & तस्मा{-}दश्वमेधयाजी सर्वाणि भूतान्यभि भवति & TB\_3.8.3.5       \\
    
    \hline
        
    646 & तस्मात् त्रीणित्रीणि पर्णस्य पलाशानि & TB\_3.2.1.2       \\
    
    \hline
        
    647 & तस्मात्{-}तेपानाज्ज्योति{-}रजायत तद्{-}भूयोऽतप्यत तस्मात्{-}तेपानादर्चि{-}रजायत तद्{-}भूयोऽतप्यत & TB\_2.2.9.2       \\
    
    \hline
        
    648 & तस्मात्{-}पुरस्ता{-}द्वान्तम् सर्वाः प्रजाः प्रति & TB\_2.3.9.5       \\
    
    \hline
        
    649 & तस्मात्{-}पृथमात्रं ॅव्यꣳसौ उत्तरस्यां ॅवेद्यामुत्तरवेदिमुप & TB\_1.6.4.3       \\
    
    \hline
        
    650 & तस्माथ् सावित्रे न सम्ॅवदेत & TB\_3.10.9.6       \\
    
    \hline
        
    651 & तस्मादग्निहोत्र{-}मुच्यते तद्धूयमान{-}मादित्योऽब्रवीत् मा हौषीः & TB\_2.1.2.6       \\
    
    \hline
        
    652 & तस्मादभिनीयैवाहः पशुमालभेत अह्न एव & TB\_1.5.9.7       \\
    
    \hline
        
    653 & तस्मादयꣳ सर्वतः पवते हुतः & TB\_3.2.3.5       \\
    
    \hline
        
    654 & तस्मादश्वमेधः वेदुको{-}ऽश्वमाशुं भवति य & TB\_3.9.22.2       \\
    
    \hline
        
    655 & तस्मादुद्वतीर्{-}भवन्ति सौर्यनुष्टु{-}गुत्तमा भवति सुवर्गस्य & TB\_1.8.8.3       \\
    
    \hline
        
    656 & तस्मादेवमाह सꣳ रेवतीर्{-}जगतीभिर्{-}मधुमतीर्{-}मधुमतीभिः सृज्यद्ध्वमित्याह & TB\_3.2.8.2       \\
    
    \hline
        
    657 & तस्माद् राजन्यो बाहुबली भावुकः & TB\_3.8.23.3       \\
    
    \hline
        
    658 & तस्माद् वाजपेययाजी पूतो मेद्ध्यो & TB\_1.3.3.7       \\
    
    \hline
        
    659 & तस्माद्{-}गायतश्च मत्तस्य च न & TB\_1.3.2.7       \\
    
    \hline
        
    660 & तस्माद्{-}वरुण{-}प्रघासैर् यजमानः परिवथ्सरीणाꣳ स्वस्ति{-}माशास्त & TB\_1.4.10.2       \\
    
    \hline
        
    661 & तस्मिन् वयममृतं दुहानाः क्षुधं & TB\_3.1.2.2       \\
    
    \hline
        
    662 & तस्मिन्{-}नाजिमधावन्न् तं बृहस्पति{-}रुदजयत् तेनायजत & TB\_1.3.2.2       \\
    
    \hline
        
    663 & तस्मिन्. यस्य तथाविधे जुह्वति & TB\_2.1.10.2       \\
    
    \hline
        
    664 & तस्मै हैतमग्निं सावित्रमुवाच तं & TB\_3.10.11.5       \\
    
    \hline
        
    665 & तस्य त्रेधा महिमानं ॅव्यौहत् & TB\_1.1.5.8       \\
    
    \hline
        
    666 & तस्य प्रयुक्तीन्द्रो ऽजायत यः & TB\_2.2.11.2       \\
    
    \hline
        
    667 & तस्य मृत्यौ चरति राजसूयम् & TB\_2.7.15.2       \\
    
    \hline
        
    668 & तस्य रुद्रा अधिपतयः वायुर्ज्योतिः & TB\_3.8.18.2       \\
    
    \hline
        
    669 & तस्य वा अग्नेर्. हिरण्यं & TB\_2.3.4.1       \\
    
    \hline
        
    670 & तस्य वा इयं क्लृप्तिः & TB\_2.2.7.4       \\
    
    \hline
        
    671 & तस्य वै मनोस्तल्पं प्रतिजग्रहुषः & TB\_2.3.4.5       \\
    
    \hline
        
    672 & तस्य सुम्नमशीमहि तस्य भक्षमशीमहि & TB\_3.7.9.5       \\
    
    \hline
        
    673 & तस्य हैवा{-}होरात्राणि अमुष्मिन् ॅलोके & TB\_3.10.11.3       \\
    
    \hline
        
    674 & तस्या उच्छेषण{-}मददुः तत् प्राश्ञात् & TB\_1.1.9.2 TB\_1.1.9.3       \\
    
    \hline
        
    675 & तस्या एते स्तना आसन्न् & TB\_1.4.1.5       \\
    
    \hline
        
    676 & तस्यां मे रास्व तस्यास्ते & TB\_2.5.8.7       \\
    
    \hline
        
    677 & तस्यावाचोऽवपादादबिभयुः तमेतेषु सप्तसु छन्दः & TB\_1.5.12.1       \\
    
    \hline
        
    678 & तस्यैवात्मा पदवित्तं ॅविदित्वा न & TB\_3.12.9.8       \\
    
    \hline
        
    679 & ता उपौहन्थ्{-}सप्तशीर्.षण्यान्{-}प्राणान् तस्माथ्{-}सौम्यस्याद्ध्वरस्य यज्ञ्क्रतोः & TB\_2.3.6.4       \\
    
    \hline
        
    680 & ता वा एताः पञ्च & TB\_3.12.4.7       \\
    
    \hline
        
    681 & ता वा एताः सप्त & TB\_3.12.2.9       \\
    
    \hline
        
    682 & तां ते युनज्मि आऽहं & TB\_3.7.7.4       \\
    
    \hline
        
    683 & तां त्वा मुद्गला हविषा & TB\_2.5.6.5       \\
    
    \hline
        
    684 & तां प्रजातिं ॅयजमानोऽनु प्रजायते & TB\_3.3.8.3       \\
    
    \hline
        
    685 & तादृगेव तत् उच्चैः स्विष्टकृतमुथ् & TB\_1.3.1.6       \\
    
    \hline
        
    686 & तादृगेव तत् स्वधा पितृभ्य & TB\_3.3.6.4       \\
    
    \hline
        
    687 & तानि चात्मनः स्पृणोति आदित्यस्य & TB\_2.3.7.3       \\
    
    \hline
        
    688 & तानेव तेन प्रीणाति वैश्वदेव्यर्चा & TB\_3.3.9.8       \\
    
    \hline
        
    689 & तानेवोभयाꣳ स्तर्पयति त एनं & TB\_2.1.5.11       \\
    
    \hline
        
    690 & तान्. यद्दुह्यात् यातयाम्ना हविषा & TB\_3.7.1.5       \\
    
    \hline
        
    691 & तान्युत्तरेण अन्वेषामराथ्स्मेति तदनूराधाः ज्येष्ठमेषा{-}मवधिष्मेति & TB\_1.5.2.8       \\
    
    \hline
        
    692 & ताभिरेवैनं भिषज्यति यथ्सावित्रो भवति & TB\_3.9.17.2       \\
    
    \hline
        
    693 & ताभ्यामेव प्रति प्रोच्यात्याक्रामति विजिहाथां & TB\_3.3.7.7       \\
    
    \hline
        
    694 & ताभ्यामेवैनꣳ समिन्धे वज्रो वै & TB\_1.4.4.10       \\
    
    \hline
        
    695 & ताभ्यो दारुमये पात्रे पयोऽदुहत & TB\_2.2.9.7       \\
    
    \hline
        
    696 & तामपाहत सोऽहोरात्रयोः सन्धि{-}रभवत् सोऽकामयत & TB\_2.2.9.8       \\
    
    \hline
        
    697 & तामात्मनोऽधि निर्मिमीते यदग्नि{-}राधीयते तस्मा{-}देतावन्तो{-}ऽग्नय & TB\_1.1.10.4       \\
    
    \hline
        
    698 & तार्प्यं ॅयजमानं परिधापयति यज्ञो & TB\_1.3.7.1       \\
    
    \hline
        
    699 & तार्प्येणाश्वꣳ संज्ञ्पयन्ति यज्ञो वै & TB\_3.9.20.1       \\
    
    \hline
        
    700 & तासां जग्ध्वा रुप्यन्त्यैत् तेऽब्रुवन्न् & TB\_2.1.1.2       \\
    
    \hline
        
    701 & ताꣳस्त्वं ॅवृत्रहञ्जहि वस्वस्मभ्य{-}माभर अग्ने & TB\_2.4.1.2       \\
    
    \hline
        
    702 & तिग्मशृङ्गो वृषभः शोशुचानः प्रत्नं & TB\_2.4.2.6       \\
    
    \hline
        
    703 & तिष्ठन्नन्यम् यथाऽनो वा रथम्ॅवा & TB\_3.3.7.5       \\
    
    \hline
        
    704 & तिष्ठा हरी रथ आ & TB\_2.7.13.1       \\
    
    \hline
        
    705 & तिस्रः पराचीराहुतीर्. हुत्वा स्रुवेणोपसदं & TB\_1.5.9.6       \\
    
    \hline
        
    706 & तिस्रो देवीर्.हिरण्ययीः भारतीर्{-}बृहतीर्महीः पतिमिन्द्रं & TB\_2.6.17.6.       \\
    
    \hline
        
    707 & तिस्रो हि रात्रीः क्रीतः & TB\_1.8.5.5       \\
    
    \hline
        
    708 & तुभ्यन्ता अङ्गिरस्तमा श्याम तङ्काममग्ने & TB\_3.12.1.1       \\
    
    \hline
        
    709 & तूष्णीमुत्तरा{-}माहुतिं जुहोति मिथुनत्वाय प्रजात्यै & TB\_2.1.2.12       \\
    
    \hline
        
    710 & तृतीयस्यामितो दिवि सोम आसीत् & TB\_3.2.1.1       \\
    
    \hline
        
    711 & तृतीयेन ज्योतिषा सम्ॅविशस्व सम्ॅवेशनस्तनुवै & TB\_3.7.1.4       \\
    
    \hline
        
    712 & तृप्यति प्रजया पशुभिः उपैनं & TB\_2.2.8.4 TB\_3.12.5.11       \\
    
    \hline
        
    713 & ते तदं तमेव कृत्वोदद्रवन्न् & TB\_1.5.9.3       \\
    
    \hline
        
    714 & ते तपोऽतप्यन्त त आत्मन्निन्द्र{-}मपश्यन्न् & TB\_2.2.3.4       \\
    
    \hline
        
    715 & ते देवा गृहमेधीयेनेष्ट्वा आशिता & TB\_1.6.7.2       \\
    
    \hline
        
    716 & ते देवा मरुद्भ्यः सान्तपनेभ्यश्चरुं & TB\_1.6.6.3       \\
    
    \hline
        
    717 & ते प्रत्यशृण्वन्न् ते दर्.शपूर्णमासाभ्यामेव & TB\_2.3.6.2       \\
    
    \hline
        
    718 & ते बृहती एव भूत्वा & TB\_1.5.12.4 TB\_1.5.12.5       \\
    
    \hline
        
    719 & ते रेवत्यां प्राभवन्न् तस्माद्{-}रेवत्यां & TB\_1.5.2.5       \\
    
    \hline
        
    720 & ते सुक्रतवः शुचयो धियधांः & TB\_3.6.3.2       \\
    
    \hline
        
    721 & ते सुवर्गाय लोकायाग्नि{-}मचिन्वत पुरुष & TB\_1.1.2.5       \\
    
    \hline
        
    722 & ते सोम{-}मन्वविन्दन्न् तमघ्नन्न् तस्य & TB\_1.3.1.2       \\
    
    \hline
        
    723 & ते स्वानिनो रुद्रिया वर.षनिर्णिजः & TB\_2.7.12.4       \\
    
    \hline
        
    724 & तेज एवास्मिन् दधाति विषाद्वै & TB\_3.2.9.2       \\
    
    \hline
        
    725 & तेजसा वा एष ब्रह्मवर्चसेन & TB\_3.9.5.1       \\
    
    \hline
        
    726 & तेजस्वी यशस्वी ब्रह्मवर्चसी स्यामिति & TB\_3.11.9.8       \\
    
    \hline
        
    727 & तेजो वा अग्निः तेज & TB\_3.3.4.3       \\
    
    \hline
        
    728 & तेजोऽसि तपसि श्रितं समुद्रस्य & TB\_3.11.1.3       \\
    
    \hline
        
    729 & तेजोऽसि तेजो मयि धेहि & TB\_2.6.6.5       \\
    
    \hline
        
    730 & तेन लोकान्थ् सूर्यवतो जयेम & TB\_3.7.6.14       \\
    
    \hline
        
    731 & तेन सोऽस्याभीष्टः प्रीतः यथ्सभायां & TB\_1.1.10.6       \\
    
    \hline
        
    732 & तेनापः प्रोक्षिताः अग्निर् देवेभ्यो & TB\_3.3.6.2       \\
    
    \hline
        
    733 & तेनासुनाऽसुरानसृजत तदसुराणा{-}मसुरत्वम् य एवमसुराणा{-}मसुरत्वं & TB\_2.3.8.2       \\
    
    \hline
        
    734 & तेनाहमस्य ब्रह्मणा निवर्तयामि जीवसे & TB\_1.5.5.7       \\
    
    \hline
        
    735 & तेनैन्द्रः यद्वामनः तेन वैष्णवः & TB\_1.7.2.3       \\
    
    \hline
        
    736 & तेनैवैनां ॅव्रतमुपनयति तस्मादाहुः यश्चैवं & TB\_3.3.3.3       \\
    
    \hline
        
    737 & तेनैवोभयान् पशूनवरुन्धे प्राजापत्या भवन्ति & TB\_3.9.9.3       \\
    
    \hline
        
    738 & तेनौषधी{-}रसृजन्त यत् पञ्चहोतारः सत्रमासत & TB\_2.3.5.2       \\
    
    \hline
        
    739 & तेभ्यो निधानं बहुधा व्यैच्छन्न् & TB\_2.7.17.3       \\
    
    \hline
        
    740 & तेभ्यो मृन्मये पात्रेऽन्नमदुहत् याऽस्य & TB\_2.2.9.6       \\
    
    \hline
        
    741 & तेषामनाहितो ऽग्निरासीत् अथैभ्यो वामं & TB\_1.1.2.3       \\
    
    \hline
        
    742 & तेऽग्नये शुचये असौ वा & TB\_1.1.6.3       \\
    
    \hline
        
    743 & तेऽग्निना मुखेनासुरानजयन्न् सोमेन राज्ञा & TB\_1.6.2.7       \\
    
    \hline
        
    744 & तैरेव सवान्नैति यानि देवराजानां & TB\_1.8.8.4       \\
    
    \hline
        
    745 & तैर्वै ते व्यावृत{-}मगच्छन्न् यद्{-}दारुमयाणि & TB\_1.4.1.4       \\
    
    \hline
        
    746 & तौ दिव्यौ श्वानावभवताम् यो & TB\_1.1.2.6       \\
    
    \hline
        
    747 & त्ःयम्बकं ॅयजामह इत्याह मृत्योः & TB\_1.6.10.5       \\
    
    \hline
        
    748 & त्रय इमे लोकाः इमानेव & TB\_3.8.16.3       \\
    
    \hline
        
    749 & त्रयस्त्रिꣳशद्वै देवताः देवता एवावरुन्धते & TB\_1.2.2.5       \\
    
    \hline
        
    750 & त्रया देवा एकादश त्रयस्त्रिꣳशाः & TB\_2.6.5.7       \\
    
    \hline
        
    751 & त्रयो वै प्रैयमेधा आसन्न् & TB\_2.1.9.1       \\
    
    \hline
        
    752 & त्रयोऽश्वा भवन्ति रथश्चतुर्थः तस्मा{-}च्चतुर्जुहोति & TB\_1.7.9.6       \\
    
    \hline
        
    753 & त्रिरुपवाजयति त्रयो वै प्राणाः & TB\_3.3.7.3       \\
    
    \hline
        
    754 & त्रिवृतैव तद्{-}यजमानमाददते तं त्रिवृतैव & TB\_1.5.10.4       \\
    
    \hline
        
    755 & त्रिवृत् पलाशे दर्भः इयान् & TB\_3.7.4.11       \\
    
    \hline
        
    756 & त्रिवृत् प्रजननं उपस्थो योनि & TB\_3.11.9.6       \\
    
    \hline
        
    757 & त्रिवृथ्{-}स्तोमो भवति ब्रह्मवर्चसं ॅवै & TB\_2.7.1.1       \\
    
    \hline
        
    758 & त्रिवृद्{-}बर्.हिर्{-}भवति माता पिता पुत्रः & TB\_1.6.3.1       \\
    
    \hline
        
    759 & त्रिꣳशतमौद्ग्रहणानि जुहोति त्रिꣳशदक्षरा विराट् & TB\_3.8.10.4       \\
    
    \hline
        
    760 & त्रिꣳशथ् संपद्यन्ते त्रिꣳशदक्षरा विराट् & TB\_1.6.3.4       \\
    
    \hline
        
    761 & त्रीणि हवीꣳषि निर्वपति विराज & TB\_1.1.5.10       \\
    
    \hline
        
    762 & त्रीन् परिधीꣳस्तिस्रः समिधः यज्ञायुरनुसंचरान् & TB\_3.7.4.9       \\
    
    \hline
        
    763 & त्रेधा विहितꣳ हि शिरः & TB\_1.2.6.3       \\
    
    \hline
        
    764 & त्वमग्ने रुद्रो असुरो महो & TB\_3.11.2.1       \\
    
    \hline
        
    765 & त्वमेव त्वां ॅवेत्थ योऽसि & TB\_3.10.3.1       \\
    
    \hline
        
    766 & त्वया तथ्सोम गुप्तमस्तु नः & TB\_3.7.13.2       \\
    
    \hline
        
    767 & त्वष्टा वा अकामयत चित्रं & TB\_3.1.4.12       \\
    
    \hline
        
    768 & त्वामग्ने समिधानं ॅयविष्ठ देवा & TB\_1.2.1.12       \\
    
    \hline
        
    769 & त्विषिमेवाव रुन्धे त्रयो ग्रहाः & TB\_1.8.5.3       \\
    
    \hline
        
    770 & त्वꣳ सोम क्रतुभिः सुक्रतुर्भूः & TB\_2.4.3.8       \\
    
    \hline
        
    771 & त्वꣳ ह्यग्ने प्रथमो मनोता & TB\_3.6.10.1       \\
    
    \hline
        
    772 & तꣳ श्रद्धा{-}ऽब्रवीत् प्रजापते श्रद्धया & TB\_3.12.4.3       \\
    
    \hline
        
    773 & तꣳ सत्य{-}मब्रवीत् प्रजापते सत्येन & TB\_3.12.4.4       \\
    
    \hline
        
    774 & तꣳ सद्ध्रीचीरूतयो वृष्णियानि पौꣳस्यानि & TB\_2.4.5.2       \\
    
    \hline
        
    775 & तꣳ सृष्टꣳ रक्षाꣳस्यजिघाꣳसन्न् स & TB\_1.7.1.5       \\
    
    \hline
        
    776 & तꣳ ह त्रीन् गिरि{-}रूपान & TB\_3.10.11.4       \\
    
    \hline
        
    777 & तꣳ हैत{-}मेके पशुबन्ध एवोत्तरवेद्यां & TB\_3.11.9.1       \\
    
    \hline
        
    778 & दक्षाय त्वा दक्षिणां प्रतिगृह्णामीति & TB\_3.11.8.8       \\
    
    \hline
        
    779 & दक्षिणत उदङ्तिष्ठन् प्रोक्षति अनेनाश्वेन & TB\_3.8.5.2       \\
    
    \hline
        
    780 & दक्षिणत उप सृजति पितृलोकमेव & TB\_2.1.8.1       \\
    
    \hline
        
    781 & दक्षिणां प्रतिग्रहीष्यन्थ्{-}सप्तदश कृत्वोऽपान्यात् आत्मानमेव & TB\_2.3.2.1       \\
    
    \hline
        
    782 & दद्भ्यः स्वाहा हनूभ्याꣳ स्वाहेत्यङ्गहोमा{-}ञ्जुहोति & TB\_3.8.17.4       \\
    
    \hline
        
    783 & दधन्वा यो नर्यो अफ्स्वन्तरा & TB\_2.6.1.2       \\
    
    \hline
        
    784 & दधाना अभ्यनूषत हविषा यज्ञ्मिन्द्रियम् & TB\_2.6.13.2       \\
    
    \hline
        
    785 & दर्भस्तम्बे जुहोति एतस्माद्वै योनेः & TB\_2.2.1.3       \\
    
    \hline
        
    786 & दिवः खीलोऽवततः पृथिव्या अद्ध्युत्थितः & TB\_3.7.6.19       \\
    
    \hline
        
    787 & दिवः शिल्पमवततम् पृथिव्याः ककुभि & TB\_3.3.2.1       \\
    
    \hline
        
    788 & दिवमेव वृष्टिमवरुन्धे किꣳ स्विदासीद्{-}बृहद्वय & TB\_3.9.5.3       \\
    
    \hline
        
    789 & दिवि श्रवो दधिरे यज्ञियासः & TB\_2.7.12.6       \\
    
    \hline
        
    790 & दिशस्त्वा दीक्षमाणमनु दीक्षन्तां आपस्त्वा & TB\_3.7.7.8       \\
    
    \hline
        
    791 & दिशो व्यास्थापयति दिशामभिजित्यै यदनु & TB\_1.7.7.1       \\
    
    \hline
        
    792 & दीर्घायुत्वाय शतशारदाय प्रतिगृभ्णामि महते & TB\_2.5.7.2       \\
    
    \hline
        
    793 & दुग्ध्वा ददाति न ह्यदृष्टा & TB\_1.4.3.3       \\
    
    \hline
        
    794 & दुह एवैनां तेन तदाहुः & TB\_3.8.21.3       \\
    
    \hline
        
    795 & दृढान्यौघ्नादुशमान ओजः अवाभिनत्ककुभः पर्वतानाम् & TB\_2.4.5.3       \\
    
    \hline
        
    796 & देव इन्द्रो वनस्पतिः हिरण्यपर्णो & TB\_2.6.10.6 TB\_2.6.14.5       \\
    
    \hline
        
    797 & देवं बर्.हिः वसुवने वसुधेयस्य & TB\_3.5.9.1 TB\_3.6.14.1       \\
    
    \hline
        
    798 & देवं बर्.हिः सरस्वती सुदेवमिन्द्रे & TB\_2.6.14.1       \\
    
    \hline
        
    799 & देवं बर्.हिः सुदेवं देवैः & TB\_3.6.13.1       \\
    
    \hline
        
    800 & देवं बर्.हिरिन्द्रं ॅवयोधसम् देवं & TB\_2.6.20.1       \\
    
    \hline
        
    801 & देवं बर्.हिरिन्द्रꣳ सुदेवं देवैः & TB\_2.6.10.1       \\
    
    \hline
        
    802 & देवगृहा वै नक्षत्राणि य & TB\_1.5.2.6       \\
    
    \hline
        
    803 & देवता वा एता यजमानस्य & TB\_1.4.1.2       \\
    
    \hline
        
    804 & देवताभिरेवैनथ् समर्द्धयति अद्रिरसि वानस्पत्य & TB\_3.2.5.8       \\
    
    \hline
        
    805 & देवत्रा यच्च मानुषं सर्वास्ताः & TB\_3.12.6.5       \\
    
    \hline
        
    806 & देवबर्.हिः शतवल्.शं ॅविरोहेत्याह प्रजा & TB\_3.2.2.6       \\
    
    \hline
        
    807 & देवसुवामेतानि हवीꣳषि भवन्ति एतावन्तो & TB\_1.7.4.1       \\
    
    \hline
        
    808 & देवस्य त्वा सवितुः प्रसव & TB\_3.2.2.1 TB\_3.2.8.1 TB\_3.2.9.1       \\
    
    \hline
        
    809 & देवस्य सवितुः प्रातः प्रसवः & TB\_1.5.3.1       \\
    
    \hline
        
    810 & देवस्य सवितुर्. हस्तः प्रसवः & TB\_1.5.1.3       \\
    
    \hline
        
    811 & देवस्याहꣳ सवितुः प्रसवे बृहस्पतिना & TB\_1.3.6.1       \\
    
    \hline
        
    812 & देवा एव तद्{-}देवान्{-}गच्छन्ति यच्चमसाञ्जुहोति & TB\_1.4.1.3       \\
    
    \hline
        
    813 & देवा देवेषु पराक्रमद्ध्वं प्रथमा & TB\_3.7.5.1       \\
    
    \hline
        
    814 & देवा दैव्या होतारा वसुवने & TB\_3.6.14.2       \\
    
    \hline
        
    815 & देवा नो यज्ञ्{-}मृजुधा नयन्तु & TB\_3.7.10.3       \\
    
    \hline
        
    816 & देवा भागं ॅयथा पूर्वे & TB\_2.4.4.5       \\
    
    \hline
        
    817 & देवा वा अश्वमेधे पवमाने & TB\_3.8.22.1       \\
    
    \hline
        
    818 & देवा वा आदित्यस्य सुवर्गस्य & TB\_1.2.4.2       \\
    
    \hline
        
    819 & देवा वा ऊर्जं ॅव्यभजन्त & TB\_1.1.3.10       \\
    
    \hline
        
    820 & देवा वा ओषधीष्वाजिमयुः ता & TB\_1.6.1.10       \\
    
    \hline
        
    821 & देवा वै चतुर्.होतृभिर्{-}यज्ञ्मतन्वत ते & TB\_2.2.8.1       \\
    
    \hline
        
    822 & देवा वै यथादर्.शं ॅयज्ञानाहरन्त & TB\_1.3.2.1       \\
    
    \hline
        
    823 & देवा वै यदन्यैर्{-}ग्रहैर्{-}यज्ञ्स्य नावारुन्धत & TB\_1.3.3.1       \\
    
    \hline
        
    824 & देवा वै यद्{-}यज्ञे ऽकुर्वत & TB\_1.5.6.1       \\
    
    \hline
        
    825 & देवा वै वरुणमयाजयन्न् स & TB\_2.2.5.1       \\
    
    \hline
        
    826 & देवाः पितरः पितरो देवाः & TB\_3.7.5.4       \\
    
    \hline
        
    827 & देवानां ॅवा इतरे यज्ञाः & TB\_1.3.10.10       \\
    
    \hline
        
    828 & देवानेव तैर्यजमानः प्रीणाति आज्येन & TB\_3.8.14.2       \\
    
    \hline
        
    829 & देवान् देवायते यज अग्निर्. & TB\_3.11.6.4       \\
    
    \hline
        
    830 & देवासुराः सं ॅयत्ता आसन्न् & TB\_1.3.1.1       \\
    
    \hline
        
    831 & देवासुराः संॅयत्ता आसन्न् ते & TB\_1.1.6.1       \\
    
    \hline
        
    832 & देवासुराः सम्ॅयत्ता आसन्न् ते & TB\_3.1.4.7       \\
    
    \hline
        
    833 & देवासुराः सम्ॅयत्ता आसन्न् स & TB\_1.5.9.1 TB\_3.3.5.1       \\
    
    \hline
        
    834 & देवासुराः सम्ॅयत्ता आसन्न् सोऽग्निरब्रवीत् & TB\_1.6.6.1       \\
    
    \hline
        
    835 & देवी उषासा नक्ता इन्द्रं & TB\_2.6.10.2       \\
    
    \hline
        
    836 & देवी उषासावश्विना भिषजेन्द्रे सरस्वती & TB\_2.6.14.2       \\
    
    \hline
        
    837 & देवी ऊर्जाहुती दुघे सुदुघे & TB\_2.6.14.3       \\
    
    \hline
        
    838 & देवी ऊर्जाहुती देवमिन्द्रं ॅवयोधसम् & TB\_2.6.20.3       \\
    
    \hline
        
    839 & देवी देवं ॅवयोधसम् उषे & TB\_2.6.20.2       \\
    
    \hline
        
    840 & देवी{-}स्तिस्र{-}स्तिस्रो देवीः सरस्वत्यश्विना भारतीडा & TB\_2.6.14.4       \\
    
    \hline
        
    841 & देवी{-}स्तिस्र{-}स्तिस्रो देवीर्वयोधसम् पतिमिन्द्र{-}मवर्द्धयन्न् जगत्या & TB\_2.6.20.4       \\
    
    \hline
        
    842 & देवीरापः सं मधुमतीर्{-}मधुमतीभिः सृज्यद्ध्वमित्याह & TB\_1.7.6.1       \\
    
    \hline
        
    843 & देवेन सवित्रा प्रसूत आर्त्विज्यं & TB\_3.7.6.2       \\
    
    \hline
        
    844 & देवेभ्यः पितृभ्यः स्वाहा सोम्येभ्यः & TB\_3.7.14.4       \\
    
    \hline
        
    845 & देवेभ्यो वै स्वर्गो लोकस्तिरो & TB\_3.12.4.1       \\
    
    \hline
        
    846 & देवेभ्यो वै स्वर्गो लोकस्तिरोऽभवत् & TB\_3.12.2.1       \\
    
    \hline
        
    847 & देवेभ्यो हव्यं ॅवह नः & TB\_2.5.8.9       \\
    
    \hline
        
    848 & देवो अग्निः स्विष्टकृत् सुद्रविणा & TB\_3.6.14.3       \\
    
    \hline
        
    849 & देवो वनस्पतिर्{-}देवमिन्द्रं ॅवयोधसम् देवो & TB\_2.6.20.5       \\
    
    \hline
        
    850 & देहि दक्षिणां प्रतिरस्वायुः अथा & TB\_2.7.17.2       \\
    
    \hline
        
    851 & दैवीश्च मानुषीश्च अहोरात्रे मे & TB\_3.7.5.8       \\
    
    \hline
        
    852 & दैव्यः केतु र्विश्वं भुवन{-}माविवेश & TB\_3.7.10.2       \\
    
    \hline
        
    853 & दैव्या अद्ध्वर्यव उपहूताः उपहूता & TB\_3.5.8.3 TB\_3.5.13.3       \\
    
    \hline
        
    854 & दैव्या होतारा भिषजा पातमिन्द्रं & TB\_2.6.12.4       \\
    
    \hline
        
    855 & दैव्याः शमितार उत मनुष्या & TB\_3.6.6.1       \\
    
    \hline
        
    856 & दोहा एव युष्माकमिति सा & TB\_1.1.10.2       \\
    
    \hline
        
    857 & द्यौरसि वायौ श्रिता आदित्यस्य & TB\_3.11.1.10       \\
    
    \hline
        
    858 & द्रुपदादिव मुञ्चतु द्रुपदादिवेन्{-}मुमुचानः स्विन्नः & TB\_2.4.4.10       \\
    
    \hline
        
    859 & द्रुहः पाशान्निर्.ऋत्यै चोदमोचि अहा & TB\_2.5.6.3       \\
    
    \hline
        
    860 & द्वादश मासाः संॅवथ्सरः संॅवथ्सरमेव & TB\_1.3.7.4       \\
    
    \hline
        
    861 & द्वादश{-}पष्ठौहीर्{-}ब्रह्मणे आयुरेवाव{-}रुन्धे वशां मैत्रावरुणाय & TB\_1.8.2.4       \\
    
    \hline
        
    862 & द्वादशगवꣳ सीरं दक्षिणा समृद्ध्यै & TB\_1.7.1.2       \\
    
    \hline
        
    863 & द्वादशसु विक्रामेष्वग्निमादधीत द्वादश मासाः & TB\_1.1.4.1       \\
    
    \hline
        
    864 & द्वाभ्याम् द्विप्रतिष्ठो हि वसुभ्यस्त्वा & TB\_3.3.9.2       \\
    
    \hline
        
    865 & द्वारो देवीर्.हिरण्ययीः ब्रह्माण इन्द्रं & TB\_2.6.17.4       \\
    
    \hline
        
    866 & द्वितायो वृत्रहन्तमः विद इन्द्रः & TB\_2.7.13.2       \\
    
    \hline
        
    867 & द्विपदं छन्द इहेन्द्रियम् उक्षाणं & TB\_2.6.17.7       \\
    
    \hline
        
    868 & द्विपाद्{-}यजमानः प्रतिष्ठित्यै देवो वः & TB\_3.2.5.2       \\
    
    \hline
        
    869 & द्विर्जुहोति अथ क्व द्वे & TB\_2.1.4.5       \\
    
    \hline
        
    870 & द्वे विरूपे चरतः स्वर्थे & TB\_2.7.12.2       \\
    
    \hline
        
    871 & द्वौ वाव पुरुषौ यं & TB\_3.2.9.4       \\
    
    \hline
        
    872 & धर्मासि दिशो दृꣳहेत्याह दिश & TB\_3.2.7.3       \\
    
    \hline
        
    873 & धर्मो वा अधिपतिः धर्ममेवा{-}वरुन्धे & TB\_3.9.16.2       \\
    
    \hline
        
    874 & धात्रे पुरोडाशं द्वादशकपालं निर्वपति & TB\_1.7.2.1       \\
    
    \hline
        
    875 & धारावरा मरुतो धृष्णुवोजसः मृगा & TB\_2.5.5.4       \\
    
    \hline
        
    876 & धिषणाऽसि पार्वतेयी प्रति त्वा & TB\_3.2.6.3       \\
    
    \hline
        
    877 & धिष्णिया वा एते न्युप्यन्ते & TB\_3.3.8.1       \\
    
    \hline
        
    878 & धृष्टिरसि ब्रह्म यच्छेत्याह धृत्यै & TB\_3.2.7.1       \\
    
    \hline
        
    879 & न ग्राम्यान् पशून्. हिनस्ति & TB\_1.6.10.3       \\
    
    \hline
        
    880 & न दिवा न नक्तमिति & TB\_1.7.1.7       \\
    
    \hline
        
    881 & न देवताभ्य आवृश्च्यते वसीयान् & TB\_3.3.10.2 TB\_3.8.3.2       \\
    
    \hline
        
    882 & न प्रयाजा इज्यन्ते नानूयाजाः & TB\_1.6.6.6       \\
    
    \hline
        
    883 & न मयाऽभागया ऽनुवक्ष्यथेति वागब्रवीत् & TB\_3.3.8.6       \\
    
    \hline
        
    884 & न वै ब्राह्मणे राष्ट्रं & TB\_3.9.14.3       \\
    
    \hline
        
    885 & न वै सोमेन सोमस्य & TB\_2.7.4.1       \\
    
    \hline
        
    886 & न सं मृशति पापवस्यसस्य & TB\_2.1.8.2       \\
    
    \hline
        
    887 & न सभृंत्याः संभाराः न & TB\_1.3.1.5       \\
    
    \hline
        
    888 & न हास्यैता अग्नौ मधु & TB\_3.10.10.2       \\
    
    \hline
        
    889 & नक्तं जाताऽस्योषधे रामे कृष्णे & TB\_2.4.4.1       \\
    
    \hline
        
    890 & नक्षत्राणि स्थ चन्द्रमसि श्रितानि & TB\_3.11.1.13       \\
    
    \hline
        
    891 & नदीभ्यः पौञ्जिष्टम् ऋक्षीकाभ्यो नैषादम् & TB\_3.4.5.1       \\
    
    \hline
        
    892 & नमस्करोति नमस्कारो हि पितृणाम् & TB\_1.3.10.8       \\
    
    \hline
        
    893 & नमो मात्रे पृथिव्या इत्याहाहिं & TB\_1.7.9.5       \\
    
    \hline
        
    894 & नर्य प्रजां मे गोपाय & TB\_1.2.1.25       \\
    
    \hline
        
    895 & नवदावो ह्येषां प्रियः यावत्प्रियो & TB\_3.3.2.5       \\
    
    \hline
        
    896 & नवैतान्यहानि भवन्ति नव वै & TB\_1.2.2.1       \\
    
    \hline
        
    897 & नवो नवो भवति जायमानो & TB\_3.1.3.1       \\
    
    \hline
        
    898 & नाकसद{-}मित्याह यदावै वसीयान् भवति & TB\_1.3.9.3       \\
    
    \hline
        
    899 & नाङ्गानि तादृगेव तत् यदेतानि & TB\_1.1.6.4       \\
    
    \hline
        
    900 & नातारीरस्य समृतिं ॅवधानाम् सं & TB\_2.5.4.4       \\
    
    \hline
        
    901 & नाद्र्धमसेषु न मासेष्वा{-}र्तिमार्च्छति य & TB\_3.10.10.4       \\
    
    \hline
        
    902 & नानैवाहोरात्रयोः प्रति तिष्ठति पौर्णमास्यां & TB\_1.8.10.2       \\
    
    \hline
        
    903 & नान्या युवत्प्रमतिरस्ति मह्यम् स & TB\_3.6.8.2       \\
    
    \hline
        
    904 & नि वा एतस्याहवनीयो गार्.हपत्यं & TB\_1.4.4.1       \\
    
    \hline
        
    905 & निवेशय{-}न्नमृतान् मर्त्याꣳश्च रूपाणि पिꣳशन् & TB\_3.1.1.10       \\
    
    \hline
        
    906 & नीतमिश्रेण तृतीयसवने अगिष्टोमः सोमः & TB\_1.4.7.7       \\
    
    \hline
        
    907 & नृषदं त्वेत्याह प्रजा वै & TB\_1.3.9.1       \\
    
    \hline
        
    908 & नैनं प्रतिनुदन्ते ब्रह्मवादिनो वदन्ति & TB\_1.1.9.9       \\
    
    \hline
        
    909 & नैनꣳ सत्यानृते उदिते हिꣳस्तः & TB\_1.7.10.5       \\
    
    \hline
        
    910 & नैर्.ऋतं चरुं परिवृक्त्यै गृहे & TB\_1.7.3.4       \\
    
    \hline
        
    911 & पञ्च दक्षिणतः पञ्च पश्चात् & TB\_3.11.9.4       \\
    
    \hline
        
    912 & पञ्चपञ्ची वै यजमानः त्वङ्मां & TB\_1.5.9.8       \\
    
    \hline
        
    913 & पञ्चर्त्विजः षट्कृत्वोऽह्वयत् ऋतवः प्रत्यशृण्वन्न् & TB\_2.3.6.3       \\
    
    \hline
        
    914 & पञ्चविꣳश आत्मा भवति तस्मान्{-}मद्ध्यतः & TB\_1.2.6.4       \\
    
    \hline
        
    915 & पञ्चहोतारं मनसाऽनुद्रुत्या हवनीये जुहुयात् & TB\_2.2.2.3       \\
    
    \hline
        
    916 & पत्नीꣳ शुचा ऽर्पयेत् उदीचीन{-}मुद्वासयति & TB\_2.1.3.5       \\
    
    \hline
        
    917 & पद्भिर्मुखेन ते देवाः पशून्. & TB\_2.2.11.5       \\
    
    \hline
        
    918 & परा वा एतस्य यज्ञ् & TB\_3.9.4.4       \\
    
    \hline
        
    919 & पराऽसुराः यथ्स्विष्टकृद्भ्यो लोहितं जुहोति & TB\_3.9.11.3       \\
    
    \hline
        
    920 & परिवेषो वा एष वनस्पतीनाम् & TB\_3.3.11.1       \\
    
    \hline
        
    921 & परिस्तृणीत परिधत्ताग्निं परिहितो{-}ऽग्नि र्यजमानं & TB\_3.7.6.1       \\
    
    \hline
        
    922 & परो रजास्ते पञ्चमः पादः & TB\_3.7.7.13       \\
    
    \hline
        
    923 & परोक्षप्रिया इव हि देवाः & TB\_2.3.11.4       \\
    
    \hline
        
    924 & पर्णमपतत् तृतीयस्यै दिवोऽधि सोऽयं & TB\_1.2.1.6       \\
    
    \hline
        
    925 & पर्णवल्कः पवित्रं सौम्यः सोमाद्धि & TB\_3.7.4.18       \\
    
    \hline
        
    926 & पर्यग्नि करोति रक्षसा{-}मपहत्यै त्रिः & TB\_2.1.3.4       \\
    
    \hline
        
    927 & पर्वत इवा विचाचलिः इन्द्र & TB\_2.4.2.9       \\
    
    \hline
        
    928 & पवमानः सुवर्जनः पवित्रेण विचर्.षणिः & TB\_1.4.8.1       \\
    
    \hline
        
    929 & पवित्रं ॅवै हिरण्यम् पुनात्येवैनम् & TB\_1.7.2.6       \\
    
    \hline
        
    930 & पवित्रवत्यानयति अपां चैवौषधीनां च & TB\_3.2.3.6       \\
    
    \hline
        
    931 & पवित्रेण शतायुषा विश्वमायुर्{-}व्यश्नवै ।अग्न & TB\_2.6.3.4       \\
    
    \hline
        
    932 & पशवो वा उक्थानि पशूनामव{-}रुद्ध्यै & TB\_1.2.2.2       \\
    
    \hline
        
    933 & पशुमानेव भवति दद्ध्नेन्द्रिय कामस्य & TB\_2.1.5.6       \\
    
    \hline
        
    934 & पशुमानेव भवति सोऽकामयतर्तवो मे & TB\_2.2.11.3       \\
    
    \hline
        
    935 & पशूनेवा{-}वरुन्धे सर्वान्{-}पूर्णानुन्नयति सर्वे हि & TB\_2.1.3.6       \\
    
    \hline
        
    936 & पशूनेवैतेन स्पृणोति सप्रथ सभां & TB\_1.1.10.5       \\
    
    \hline
        
    937 & पश्चात्{-}पर्यायन्न् स पश्चात्{-}पर्यवर्तयत ता & TB\_2.2.10.7       \\
    
    \hline
        
    938 & पष्ठवाहं गां ॅवयो दधत् & TB\_2.6.17.5       \\
    
    \hline
        
    939 & पाप्मा वै तेजनी पाप्मनोऽपहत्यै & TB\_3.8.19.2       \\
    
    \hline
        
    940 & पितरो वा अकामयन्त पितृलोक & TB\_3.1.4.8       \\
    
    \hline
        
    941 & पितृभ्य आवृश्च्येत अवघ्रेयमेव तन्नेव & TB\_1.3.10.7       \\
    
    \hline
        
    942 & पितृयज्ञेन सुवर्गं ॅलोकं गमयति & TB\_1.6.8.2       \\
    
    \hline
        
    943 & पुनः समन्य जुहोति अन्तेनैवान्तं & TB\_1.4.4.3       \\
    
    \hline
        
    944 & पुनर्न इन्द्रो मघवा ददातु & TB\_2.5.3.1       \\
    
    \hline
        
    945 & पुरदंरो मघवान्. वज्रबाहुः आयातु & TB\_2.6.8.2       \\
    
    \hline
        
    946 & पुरस्ता{-}द्दशहोतार{-}मुदञ्च{-}मुपदधाति यावत्पदं हृदयं ॅयजुषी & TB\_3.12.5.5       \\
    
    \hline
        
    947 & पुरस्तात्{-}प्रत्यञ्च{-}मभिषिञ्चति पुरस्ताद्धि{-}प्रतीचीन{-}मन्नमद्यते शीर्.षतोऽभिषिञ्चति शीर्.षतो & TB\_1.3.8.3       \\
    
    \hline
        
    948 & पुरस्ताद्धि प्रतीचीन{-}मन्नमद्यते शीर्.षतो घ्नन्ति & TB\_1.3.7.7       \\
    
    \hline
        
    949 & पुरीषवतीं करोति प्रजा वै & TB\_3.2.9.12       \\
    
    \hline
        
    950 & पुरुषो वै यज्ञ्ः यज्ञ्ः & TB\_3.8.23.1       \\
    
    \hline
        
    951 & पुरोधामेव गच्छति तस्य प्रातस्सवने & TB\_2.7.1.3       \\
    
    \hline
        
    952 & पुष्टिं तेन यत्{-}कमण्डलुम् आयुष्टेन & TB\_2.7.9.3       \\
    
    \hline
        
    953 & पुष्ट्यामेव प्रजनने ऽग्निमाधत्ते अथो & TB\_1.1.3.2       \\
    
    \hline
        
    954 & पूतं पवित्रेणेवाज्यम् आपः शुन्धन्तु & TB\_2.6.6.4       \\
    
    \hline
        
    955 & पूर्वे सोमग्रहा गृह्यन्ते अपरे & TB\_1.3.3.5       \\
    
    \hline
        
    956 & पूर्वेणैवास्य यज्ञेन यज्ञ्मनु सं & TB\_1.4.4.11       \\
    
    \hline
        
    957 & पूर्वेद्युरिद्ध्मा बर्.हिः करोति यज्ञ्मेवारभ्य & TB\_3.2.3.1       \\
    
    \hline
        
    958 & पूषा ते ग्रन्थिं ग्रथ्नात्वित्याह & TB\_3.2.2.8       \\
    
    \hline
        
    959 & पूषा वा अकामयत पशुमान्थ् & TB\_3.1.5.12       \\
    
    \hline
        
    960 & पृथिवी न्यवर्तयत सौषधीभिर्{-}वनस्पतिभि{-}रपुष्यत् वायुर्न्यवर्तयत & TB\_2.3.3.2       \\
    
    \hline
        
    961 & पृथिवी माता प्रजापति र्बन्धुः & TB\_3.7.5.5       \\
    
    \hline
        
    962 & पृथिव्य{-}स्यफ्सु श्रिता अग्नेः प्रतिष्ठा & TB\_3.11.1.6       \\
    
    \hline
        
    963 & पृथिव्यै स्वाहाऽन्तरिक्षाय स्वाहेत्ये{-}कविꣳशिनीं दीक्षां & TB\_3.8.17.2       \\
    
    \hline
        
    964 & पौर्णमास्यष्टका ऽमावास्या अन्नादाः स्थान्नदुघो & TB\_3.11.1.19       \\
    
    \hline
        
    965 & प्र ते महे विदथे & TB\_3.7.9.6       \\
    
    \hline
        
    966 & प्र त्वा पद्ये सोमं & TB\_2.3.10.2       \\
    
    \hline
        
    967 & प्र नक्षत्राय देवाय इन्द्रायेन्दुं & TB\_3.1.3.3       \\
    
    \hline
        
    968 & प्र प्रजया पशुभिर्{-}मिथुनैर्{-}जायते स & TB\_2.2.4.4       \\
    
    \hline
        
    969 & प्र यः सत्राचा मनसा & TB\_2.4.3.5       \\
    
    \hline
        
    970 & प्र यशः श्रैष्ठ्यमाप्नोति य & TB\_3.8.9.2       \\
    
    \hline
        
    971 & प्र वा एतेऽस्माल्{-}लोकाच्च्यवन्ते य & TB\_1.3.6.6       \\
    
    \hline
        
    972 & प्र वा एष एभ्यो & TB\_3.9.16.3       \\
    
    \hline
        
    973 & प्र वो वाजा अभिद्यवः & TB\_3.5.2.1       \\
    
    \hline
        
    974 & प्र सद्यो अग्ने अत्यष्यन्यान् & TB\_2.4.7.10       \\
    
    \hline
        
    975 & प्रघास्यान्. हवामह इति पत्नी{-}मुदानयति & TB\_1.6.5.3       \\
    
    \hline
        
    976 & प्रचोदयन्ता विदथेषु कारू प्राचीनं & TB\_3.6.3.4       \\
    
    \hline
        
    977 & प्रजया पशुभिर्{-}ब्रह्मवर्चसेन सुवर्गे लोके & TB\_1.2.1.15       \\
    
    \hline
        
    978 & प्रजयैवैनं पशुभिः प्रथयति स्वाहाकृत & TB\_3.8.3.6       \\
    
    \hline
        
    979 & प्रजा एव तद्{-}यजमानः पोषयति & TB\_1.6.2.5       \\
    
    \hline
        
    980 & प्रजा एव पृथिव्यां प्रतिष्ठापयति & TB\_3.3.6.3       \\
    
    \hline
        
    981 & प्रजा वै पशवः सुम्नम् & TB\_3.3.9.9       \\
    
    \hline
        
    982 & प्रजा वै सत्रमासत तपस्तप्यमाना & TB\_1.4.9.1       \\
    
    \hline
        
    983 & प्रजां त इति किं & TB\_3.11.8.3       \\
    
    \hline
        
    984 & प्रजां पुष्टिमथो धनम् द्विपदो & TB\_3.3.11.2       \\
    
    \hline
        
    985 & प्रजां ॅयोनिं मा निर्मृक्षमित्याह & TB\_3.3.9.4       \\
    
    \hline
        
    986 & प्रजाः पशव इमे लोकाः & TB\_2.2.3.3       \\
    
    \hline
        
    987 & प्रजाः प्रजनयन्न् यद्वै यज्ञ्स्य & TB\_3.3.10.4       \\
    
    \hline
        
    988 & प्रजानामवरुणग्राहाय तासां दक्षिणो बाहुः & TB\_1.6.4.2       \\
    
    \hline
        
    989 & प्रजापतये त्वा जुष्टं प्रोक्षामीति & TB\_3.8.7.1       \\
    
    \hline
        
    990 & प्रजापति र्देवानसृजत ते पाप्मना & TB\_3.10.9.1       \\
    
    \hline
        
    991 & प्रजापति र्विश्वेभ्यो देवेभ्यः विश्वे & TB\_3.7.6.3       \\
    
    \hline
        
    992 & प्रजापति{-}रकामय{-}तात्मन् वन्मे जायेतेति सोऽजुहोत् & TB\_2.1.6.1       \\
    
    \hline
        
    993 & प्रजापति{-}रकामयत प्र जायेयेति स & TB\_2.2.3.1       \\
    
    \hline
        
    994 & प्रजापति{-}रकामयत प्रजायेयेति स एतं & TB\_2.2.4.1       \\
    
    \hline
        
    995 & प्रजापति{-}रकामयत बहोर्भूयान्थ्{-}स्यामिति स एतं & TB\_2.2.11.1       \\
    
    \hline
        
    996 & प्रजापति{-}रकामयत महानन्नादः स्यामिति स & TB\_3.9.10.1       \\
    
    \hline
        
    997 & प्रजापति{-}रकामयताश्वमेधेन यजेयेति स तपोऽतप्यत & TB\_3.8.10.1       \\
    
    \hline
        
    998 & प्रजापति{-}रग्नि{-}मसृजत तं प्रजा अन्वसृज्यन्त & TB\_2.1.2.1       \\
    
    \hline
        
    999 & प्रजापति{-}रश्वमेध{-}मसृजत तꣳ सृष्टं न & TB\_3.8.11.1       \\
    
    \hline
        
    1000 & प्रजापति{-}रश्वमेध{-}मसृजत सो{-}ऽस्माथ्सृष्टो{-}ऽपाक्रामत् तं ॅयज्ञ्{-}क्रतुभि{-}रन्वैच्छत् & TB\_3.9.13.1       \\
    
    \hline
        
    1001 & प्रजापति{-}रश्वमेधमसृजत तꣳ सृष्टꣳ रक्षाꣳस्य & TB\_3.8.15.1       \\
    
    \hline
        
    1002 & प्रजापति{-}रश्वमेधमसृजत सोऽस्माथ्सृष्टो{-}ऽपाक्रामत् तमष्टादशिभिरनु प्रायुङ्क्त & TB\_3.9.1.1       \\
    
    \hline
        
    1003 & प्रजापति{-}रिन्द्रमसृज{-}तानुजावरं देवानाम् तं प्राहिणोत् & TB\_2.2.10.1       \\
    
    \hline
        
    1004 & प्रजापति{-}र्देवेभ्यो यज्ञान् व्यादिशत् स & TB\_3.8.14.1       \\
    
    \hline
        
    1005 & प्रजापति{-}श्चतुस्त्रिꣳशो देवतानां यावतीरेव देवताः & TB\_2.7.1.4       \\
    
    \hline
        
    1006 & प्रजापतिं ॅवा एष ईफ्सतीत्याहुः & TB\_3.8.16.1       \\
    
    \hline
        
    1007 & प्रजापतिं ॅवै देवाः पितरम् & TB\_3.9.22.1       \\
    
    \hline
        
    1008 & प्रजापतिः पशूनसृजत तेऽस्माथ् सृष्टाः & TB\_2.7.14.1       \\
    
    \hline
        
    1009 & प्रजापतिः पुरुष{-}मसृजत सोऽग्निरब्रवीत् ममाय{-}मन्नमस्त्विति & TB\_2.3.7.1       \\
    
    \hline
        
    1010 & प्रजापतिः प्रजा असृजत ता & TB\_1.1.5.4 TB\_2.7.9.1 TB\_3.1.4.2       \\
    
    \hline
        
    1011 & प्रजापतिः प्रजा असृजत ताः & TB\_2.2.7.1       \\
    
    \hline
        
    1012 & प्रजापतिः प्रजा असृजत स & TB\_1.1.10.1       \\
    
    \hline
        
    1013 & प्रजापतिः प्रजाः सृष्ट्वा प्रेणाऽनु & TB\_3.9.8.1       \\
    
    \hline
        
    1014 & प्रजापतिः प्रजाः सृष्ट्वा वृत्तोऽशयत् & TB\_1.2.6.1       \\
    
    \hline
        
    1015 & प्रजापतिः प्रजाः सृष्ट्वा व्यस्रꣳसत & TB\_2.3.6.1       \\
    
    \hline
        
    1016 & प्रजापतिः प्रणेता बृहस्पतिः पुर & TB\_2.5.7.3       \\
    
    \hline
        
    1017 & प्रजापतिः सविता भूत्वा प्रजा & TB\_1.6.4.1       \\
    
    \hline
        
    1018 & प्रजापतिः सोमꣳ राजानमसृजत तं & TB\_2.3.10.1       \\
    
    \hline
        
    1019 & प्रजापतिना यज्ञ्मुखेन संमिताम् वेदिर्देवेभ्यो & TB\_3.2.9.11       \\
    
    \hline
        
    1020 & प्रजापतिमभि पर्यावर्तत स मृत्यो{-}रबिभेत् & TB\_2.1.6.5       \\
    
    \hline
        
    1021 & प्रजापतिमेव तत्प्रीणाति मसूस्यैर्जुहोति सर्वासां & TB\_3.8.14.6       \\
    
    \hline
        
    1022 & प्रजापतिर{-}कामयतोभौ लोकाव{-}वरुन्धीयेति स एतानुभयान् & TB\_3.9.2.1       \\
    
    \hline
        
    1023 & प्रजापतिरकामयत दर्.शपूर्णमासौ सृजेयेति स & TB\_2.2.2.1       \\
    
    \hline
        
    1024 & प्रजापतिरकामयत प्रजाः सृजेयेति स & TB\_2.2.1.1       \\
    
    \hline
        
    1025 & प्रजापतिरकामयत बहोर्भूयान्थ्{-}स्यामिति स एतं & TB\_2.7.10.1       \\
    
    \hline
        
    1026 & प्रजापतिरेव भूत्वा प्र जायते & TB\_2.2.1.2       \\
    
    \hline
        
    1027 & प्रजापतिर् वाचः सत्य{-}मपश्यत् तेनाग्निमाधत्त & TB\_1.1.5.1       \\
    
    \hline
        
    1028 & प्रजापतिर्{-}दशहोता य एवं चतुर्.होतृणा{-}मृद्धिं & TB\_2.3.1.2       \\
    
    \hline
        
    1029 & प्रजापतिर्वा अकामयत मूलं प्रजां & TB\_3.1.5.3       \\
    
    \hline
        
    1030 & प्रजापते न त्वदेतान्यन्य इति & TB\_1.7.8.7       \\
    
    \hline
        
    1031 & प्रजापतेराप्त्यै प्राकाशावद्ध्वर्यवे ददाति प्रकाश{-}मेवैनं & TB\_1.8.2.3       \\
    
    \hline
        
    1032 & प्रजापतेराप्त्यै श्यामा एकरूपा भवन्ति & TB\_1.3.4.4       \\
    
    \hline
        
    1033 & प्रजाभ्यः सर्वाभ्यो मृड नमो & TB\_3.7.8.2       \\
    
    \hline
        
    1034 & प्रजायते य एवं ॅविद्वाॅल्लोहितायसेन & TB\_1.5.6.6       \\
    
    \hline
        
    1035 & प्रण आयूꣳषि तारिषत् त्वमग्ने & TB\_2.4.1.9       \\
    
    \hline
        
    1036 & प्रतितिष्ठति प्रजया पशुभिर्यजमानः अथो & TB\_3.3.2.4       \\
    
    \hline
        
    1037 & प्रतिपूरुषमेककपालान् निर्वपति जाता एव & TB\_1.6.10.1       \\
    
    \hline
        
    1038 & प्रतिश्रुत्काया ऋतुलम् घोषाय भषम् & TB\_3.4.13.1       \\
    
    \hline
        
    1039 & प्रतिष्म देव रीषतः तपिष्ठैरजरो & TB\_2.4.1.7       \\
    
    \hline
        
    1040 & प्रत्यङ्ङेव वरुणपाशान्{-}निर्मुच्यते अक्रन्कर्म कर्मकृत & TB\_1.6.5.5       \\
    
    \hline
        
    1041 & प्रत्यस्मिन् ॅलोके स्थास्यसि अभि & TB\_1.1.4.7       \\
    
    \hline
        
    1042 & प्रत्युष्टꣳ रक्षः प्रत्युष्टा अरातय & TB\_3.3.1.1       \\
    
    \hline
        
    1043 & प्रत्येव तिष्ठति ब्रह्मवादिनो वदन्ति & TB\_2.3.1.3       \\
    
    \hline
        
    1044 & प्रथमजो वथ्सो दक्षिणा समृद्ध्यै & TB\_1.6.1.11       \\
    
    \hline
        
    1045 & प्रथमेन वा एष स्तोमन & TB\_3.9.9.1       \\
    
    \hline
        
    1046 & प्रदक्षिणिद्{-}रशनया नियूय ऋतस्य वक्षि & TB\_3.6.11.3       \\
    
    \hline
        
    1047 & प्रभ्रꣳशुका{-}ऽस्माच्छ्रीः स्यात् न वै & TB\_3.9.14.2       \\
    
    \hline
        
    1048 & प्रमुञ्चमानौ दुरितानि विश्वा अपाघशं & TB\_3.1.1.4       \\
    
    \hline
        
    1049 & प्रसुलामीति ते पिता गभे & TB\_3.9.7.5       \\
    
    \hline
        
    1050 & प्राचीनं मद्ध्यन्दिनात् ततो देवा & TB\_1.5.3.2       \\
    
    \hline
        
    1051 & प्राचीमभ्याकारम् अग्रैरन्तरतः एवमिव ह्यन्नमद्यते & TB\_3.3.1.4       \\
    
    \hline
        
    1052 & प्राच्येषाꣳ श्रीरगात् भद्रा भूत्वा & TB\_1.1.4.6       \\
    
    \hline
        
    1053 & प्राजापत्येनैव यज्ञेन यजते कामप्रेण & TB\_3.9.22.4       \\
    
    \hline
        
    1054 & प्राञ्चौ वेद्यꣳ सावुन्नयति आहवनीयस्य & TB\_3.2.9.9       \\
    
    \hline
        
    1055 & प्राणं ॅयज्ञ्पतये धत्तम् चक्षुः & TB\_1.1.1.4       \\
    
    \hline
        
    1056 & प्राणश्च त्वाऽपानश्च श्रीणीतां चक्षुश्च & TB\_3.7.9.3       \\
    
    \hline
        
    1057 & प्राणाः पशवः प्राणैरेव पशून्थ् & TB\_3.2.8.9       \\
    
    \hline
        
    1058 & प्राणानेवास्योपदासयति यद्येनं पुनरुप शिक्षेयुः & TB\_2.3.2.2       \\
    
    \hline
        
    1059 & प्राणापाना{-}वेवास्मिन थ्सम्यञ्चौ दधाति अश्वं & TB\_3.8.20.5       \\
    
    \hline
        
    1060 & प्राणापानावेवाव रुन्धे ओजो बलं & TB\_1.6.4.4       \\
    
    \hline
        
    1061 & प्राणैरेव प्राणान्थ् संपृणक्ति सावित्रियर्चा & TB\_3.2.5.3       \\
    
    \hline
        
    1062 & प्राणो रक्षति विश्वमेजत् इर्यो & TB\_2.5.1.1       \\
    
    \hline
        
    1063 & प्राणो वा आज्यम् उभयत & TB\_3.8.15.3       \\
    
    \hline
        
    1064 & प्राणो वै सुसन्दृक् प्राणमेवात्मन्{-}दधते & TB\_1.6.9.9       \\
    
    \hline
        
    1065 & प्राणो हृदये हृदयं मयि & TB\_3.10.8.5       \\
    
    \hline
        
    1066 & प्रारोचयन्मनवे केतुमह्नाम् अविन्दज्ज्योतिर्{-}बृहते रणाय & TB\_2.4.3.7       \\
    
    \hline
        
    1067 & प्रेतीषणिमिषयन्तं पावकम् राजन्तमग्निं ॅयजतं & TB\_3.6.10.4       \\
    
    \hline
        
    1068 & प्रेह्यभिप्रेहि प्रभरा सहस्व मा & TB\_2.4.7.4       \\
    
    \hline
        
    1069 & प्रैव जायते अथो यथा & TB\_2.1.2.9       \\
    
    \hline
        
    1070 & प्रैव तेन जायते उदिते & TB\_2.1.2.8       \\
    
    \hline
        
    1071 & प्रोवारत मरुतो दुर्मदा इव & TB\_2.4.4.4       \\
    
    \hline
        
    1072 & प्लाशीर्व्यक्तः शतधार उथ्सः दुहे & TB\_2.6.4.4       \\
    
    \hline
        
    1073 & बदरैरुपवाकाभिर् भेषजं तोक्मभिः पयः & TB\_2.6.11.2       \\
    
    \hline
        
    1074 & बर्.हिः स्तृणाति प्रजा वै & TB\_3.3.6.8       \\
    
    \hline
        
    1075 & बहु दुग्धीन्द्राय देवेभ्यो हविरिति & TB\_3.2.3.8       \\
    
    \hline
        
    1076 & बहुग्वै बह्वश्वायै बह्वजाविकायै बहुव्रीहियवायै & TB\_3.8.5.3       \\
    
    \hline
        
    1077 & बहुरूपा हि पशवः समृद्ध्यै & TB\_1.6.3.3       \\
    
    \hline
        
    1078 & बहुर्भवति य एतेन यजते & TB\_2.7.10.2       \\
    
    \hline
        
    1079 & बह्वी र्भवन्तीरुप जायमानाः इह & TB\_3.7.4.15       \\
    
    \hline
        
    1080 & बार्.हताः पशवः सा पशूनां & TB\_3.9.12.2       \\
    
    \hline
        
    1081 & बार्.हस्पत्ये मैत्रमपि दधाति ब्रह्म & TB\_1.7.3.8       \\
    
    \hline
        
    1082 & बीभथ्सायै पौल्कसम् भूत्यै जागरणम् & TB\_3.4.14.1       \\
    
    \hline
        
    1083 & बृहस्पतिः प्रथमं जायमानः तिष्यं & TB\_3.1.1.5       \\
    
    \hline
        
    1084 & बृहस्पतिर्वा अकामयत ब्रह्मवर्चसी स्यामिति & TB\_3.1.4.6       \\
    
    \hline
        
    1085 & बृहस्पतेस्तिष्यः जुह्वतः परस्ताद्{-}यजमाना अवस्तात् & TB\_1.5.1.2       \\
    
    \hline
        
    1086 & ब्रह्म प्रतिष्ठा मनसो ब्रह्म & TB\_3.7.11.1       \\
    
    \hline
        
    1087 & ब्रह्म यज्ञ्स्य तन्तवः ऋत्विजो & TB\_2.4.7.11       \\
    
    \hline
        
    1088 & ब्रह्म वा अकामयत ब्रह्मलोकमभिजयेयमिति & TB\_3.1.5.6       \\
    
    \hline
        
    1089 & ब्रह्म वै चतुर्.होतारः चतुर्.होतृभ्योऽधि & TB\_3.12.5.1       \\
    
    \hline
        
    1090 & ब्रह्म संधत्तं तन्मे जिन्वतम् & TB\_1.1.1.1       \\
    
    \hline
        
    1091 & ब्रह्मक्षत्रे एवावरुन्धे यदाऽश्विनो भवति & TB\_3.9.16.4       \\
    
    \hline
        
    1092 & ब्रह्मणे ब्राह्मणमालभते क्षत्राय राजन्यम् & TB\_3.4.1.1       \\
    
    \hline
        
    1093 & ब्रह्मणैवैनद्धरति उर्वन्तरिक्ष{-}मन्विहीत्याह गत्यै देवगंममसीत्याह & TB\_3.2.2.9       \\
    
    \hline
        
    1094 & ब्रह्मवर्चसस्य समिद्ध्यै न बर.हिरनु & TB\_2.1.4.9       \\
    
    \hline
        
    1095 & ब्रह्मवादिनो वदन्ति अग्निहोत्रप्रायणा यज्ञाः & TB\_2.1.5.1       \\
    
    \hline
        
    1096 & ब्रह्मवादिनो वदन्ति कति पात्राणि & TB\_1.5.4.1       \\
    
    \hline
        
    1097 & ब्रह्मवादिनो वदन्ति किं चतुर्.होतृणां & TB\_2.3.1.1       \\
    
    \hline
        
    1098 & ब्रह्मवादिनो वदन्ति नाग्निष्टोमो नोक्थ्यः & TB\_1.3.4.1       \\
    
    \hline
        
    1099 & ब्रह्मवादिनो वदन्ति यद्{-}दशहोतारः सत्रमासत & TB\_2.3.5.1       \\
    
    \hline
        
    1100 & ब्रह्मवादिनो वदन्ति होतव्य{-}मग्निहोत्रां3 नहोतव्या3मिति & TB\_1.1.6.9       \\
    
    \hline
        
    1101 & ब्रह्मा होताऽद्ध्वर्युरग्नीत् तमभिमृशेत् इदं & TB\_3.3.8.8       \\
    
    \hline
        
    1102 & ब्रह्मा3न्त्वꣳ राजन्{-}ब्रह्माऽसीन्द्रोऽसि सत्यौजा इत्याह & TB\_1.7.10.3       \\
    
    \hline
        
    1103 & ब्रह्मात्मन्{-}वदसृजत तदकामयत समात्मना पद्येयेति & TB\_2.3.11.1       \\
    
    \hline
        
    1104 & ब्रह्मासि क्षत्रस्य योनिः क्षत्रमस्यृतस्य & TB\_3.7.7.2       \\
    
    \hline
        
    1105 & ब्राह्मणं तु वसत्यै नापरुन्ध्यात् & TB\_3.7.3.3       \\
    
    \hline
        
    1106 & ब्राह्मणं परिक्रीणीया{-}दुच्छेषणस्य पातारम् ब्राह्मणो & TB\_1.8.6.2       \\
    
    \hline
        
    1107 & ब्राह्मणश्च शूद्रश्च चर्मकर्ते व्यायच्छेते & TB\_1.2.6.7       \\
    
    \hline
        
    1108 & ब्राह्मणेष्वमृतꣳ हितम् येन देवाः & TB\_1.4.8.6       \\
    
    \hline
        
    1109 & ब्राह्मणो वा अष्टाविꣳशो नक्षत्राणाम् & TB\_1.5.3.4       \\
    
    \hline
        
    1110 & भगस्यापराह्णः तत्{-}पुण्यं तेजस्व्यहः तस्मादपराह्णे & TB\_1.5.3.3       \\
    
    \hline
        
    1111 & भगो वा अकामयत भगी & TB\_3.1.4.10       \\
    
    \hline
        
    1112 & भरेष्विन्द्रं सुहवं हवामहे अंहोमुचं & TB\_2.7.13.3       \\
    
    \hline
        
    1113 & भवत्यात्मना पराऽस्य भ्रातृव्यो भवति & TB\_3.2.9.8       \\
    
    \hline
        
    1114 & भागधेयेनेवैनꣳ समर्द्धयति ऋषभमाह्वयति वषट्कार & TB\_1.6.7.4       \\
    
    \hline
        
    1115 & भागधेयेनैवैनं प्रणयति ब्राह्मण आर्.षेय & TB\_1.4.4.2       \\
    
    \hline
        
    1116 & भायै दार्वाहारम् प्रभाया आग्नेन्धम् & TB\_3.4.8.1       \\
    
    \hline
        
    1117 & भिषजाऽश्विनाऽश्वा शिशुमती भिषग्धेनुः सरस्वती & TB\_2.6.11.4       \\
    
    \hline
        
    1118 & भूतिमेवोपैति तत् कृत्वा अन्यां & TB\_3.7.2.2       \\
    
    \hline
        
    1119 & भूम्यै पीठसर्पिणमालभते अग्नयेऽꣳसलम् वायवे & TB\_3.4.17.1       \\
    
    \hline
        
    1120 & भूयिष्ठ{-}मेवास्मै श्रद्दधते भूयिष्ठा दक्षिणा & TB\_3.11.9.9       \\
    
    \hline
        
    1121 & भूरग्निं च पृथिवीं च & TB\_3.10.2.1       \\
    
    \hline
        
    1122 & भूरित्याह प्रजा एव तद्{-}यजमानः & TB\_1.1.5.2       \\
    
    \hline
        
    1123 & भूरीणि वृत्वा हर्यश्व हꣳसि & TB\_2.5.8.11       \\
    
    \hline
        
    1124 & भूर्भुवः स्वः ओजो बलं & TB\_3.10.5.1       \\
    
    \hline
        
    1125 & भूर्भुवस्सुवरिति प्राजापत्या{-}भिरावयन्ति प्राजापत्यो वा & TB\_3.9.4.5       \\
    
    \hline
        
    1126 & भृगूणां त्वाऽङ्गिरसां ॅव्रतपते व्रतेना{-}दधामीति & TB\_1.1.4.9       \\
    
    \hline
        
    1127 & भोगायैवास्य हुतं भवति तस्या & TB\_2.1.2.4       \\
    
    \hline
        
    1128 & भ्रातृव्य देवत्यो दक्षिणः यदाहवनीये & TB\_1.6.5.4       \\
    
    \hline
        
    1129 & भ्रातृव्यमेव पृथिव्या अपहन्ति तेऽमन्यन्त & TB\_3.2.9.5       \\
    
    \hline
        
    1130 & मद्ध्यत एव यज्ञ्ꣳ समादधाति & TB\_1.4.5.4       \\
    
    \hline
        
    1131 & मद्ध्वा समञ्जन्पथिभिः सुगेभिः स्वदाति & TB\_2.6.7.6.       \\
    
    \hline
        
    1132 & मनसैव यज्ञ्ꣳ सन्तनोति भूरित्याह & TB\_3.7.1.3       \\
    
    \hline
        
    1133 & मनो हृदये हृदयं मयि & TB\_3.10.8.6       \\
    
    \hline
        
    1134 & मन्यवेऽयस्तापम् क्रोधाय निसरम् शोकायाभिसरम् & TB\_3.4.10.1       \\
    
    \hline
        
    1135 & मन्युर्भगो मन्युरेवास देवः मन्युर्. & TB\_2.4.1.11       \\
    
    \hline
        
    1136 & मन्ये त्वा जातवेदसम् स & TB\_2.4.1.5       \\
    
    \hline
        
    1137 & मयि तनूः संनिधद्ध्वम् अहं & TB\_3.2.8.10       \\
    
    \hline
        
    1138 & मरुतां प्रसवे जेषमित्याह मरुद्भिरेव & TB\_1.7.9.3       \\
    
    \hline
        
    1139 & मरुतो गणानां पतयः रुद्र & TB\_3.11.4.2       \\
    
    \hline
        
    1140 & मरुतो वृत्रहन्तमम् येन ज्योति{-}रजनयन्नृतावृधः & TB\_2.5.8.4       \\
    
    \hline
        
    1141 & मरुत्वतीः प्रतिपदः मरुतो वै & TB\_1.4.6.2       \\
    
    \hline
        
    1142 & मरुत्वतीषु कुर्वन्ति तेनैव माद्ध्यन्दिनाथ्{-}सवनान्नयन्ति & TB\_1.4.5.2       \\
    
    \hline
        
    1143 & महतीमेव तद्देवतां प्रीणाति तण्डुलै{-}र्जुहोति & TB\_3.8.14.3       \\
    
    \hline
        
    1144 & महादिवाकीर्त्यꣳ होतुः पृष्ठम् विकर्णं & TB\_1.2.4.3       \\
    
    \hline
        
    1145 & महीं देवीं ॅविष्णुपत्नी{-}मजूर्याम् प्रतीचीमेनां & TB\_3.1.2.6       \\
    
    \hline
        
    1146 & महीन्द्रपत्नीर्.{-}हविष्मतीः वियन्त्वाज्यस्य होतर्यज होता & TB\_2.6.7.5       \\
    
    \hline
        
    1147 & मा त्वत्क्षेत्रा{-}ण्यरणानि गन्म वृञ्जे & TB\_2.4.3.2       \\
    
    \hline
        
    1148 & मा न इन्द्राभितस्त्व{-}दृष्वारिष्टासः एवा & TB\_2.7.16.2       \\
    
    \hline
        
    1149 & मा नो विदद्{-}वजना द्वेष्या & TB\_3.7.5.13       \\
    
    \hline
        
    1150 & मात्राꣳ सायुज्यꣳ सलोकतां गमयन्ति & TB\_1.5.10.7       \\
    
    \hline
        
    1151 & मामग्रे यजत मया मुखेनासुरां & TB\_1.6.2.6       \\
    
    \hline
        
    1152 & मारुत एष भवति अन्नं & TB\_1.7.7.3       \\
    
    \hline
        
    1153 & मारुत्या बृहतः स्तोत्रम् एतावन्तो & TB\_1.3.4.2       \\
    
    \hline
        
    1154 & मासाः स्थर्तुषु श्रिताः अद्र्धमासानां & TB\_3.11.1.16       \\
    
    \hline
        
    1155 & मासिमास्येतानि हवीꣳषि निरुप्याणीत्याहुः तेनैवर्तून्{-}प्रयुङ्क्त & TB\_1.8.4.3       \\
    
    \hline
        
    1156 & मित्रः क्षत्रं क्षत्रपतिः क्षत्रमस्मिन्. & TB\_2.5.7.4       \\
    
    \hline
        
    1157 & मित्रमेव भवतः प्रजापतिर् देवासुरानसृजत & TB\_2.2.7.2       \\
    
    \hline
        
    1158 & मित्राय स्वाहेत्याह मित्रायैवैनं जुहोति & TB\_3.8.6.5       \\
    
    \hline
        
    1159 & मित्रेणैव यज्ञ्स्य स्विष्टꣳ शमयन्ति & TB\_1.2.5.3       \\
    
    \hline
        
    1160 & मित्रो दाधार पृथिवीमुत द्यां & TB\_3.7.2.4       \\
    
    \hline
        
    1161 & मित्रो वा अकामयत मित्रधेयमेषु & TB\_3.1.5.1       \\
    
    \hline
        
    1162 & मित्रोऽसि वरुणोऽसि समहं ॅविश्वैर्देवैः & TB\_2.6.5.1       \\
    
    \hline
        
    1163 & मित्रोऽसि वरुणोऽसीत्याह मैत्रं ॅवा & TB\_1.7.10.1       \\
    
    \hline
        
    1164 & मित्रोऽसि सुशेव इत्याह जगती{-}मेवैतेनाभिव्याहरति & TB\_1.7.10.4       \\
    
    \hline
        
    1165 & मिथुनत्वाय प्रजात्यै यद्वै पत्नी & TB\_3.3.4.2       \\
    
    \hline
        
    1166 & मुखमपोहामि सूर्यज्योति र्विभाहि महत & TB\_3.7.5.2       \\
    
    \hline
        
    1167 & मुखमिव प्रत्युपह्वयेत समुंखानेव पशूनुपह्वयते & TB\_3.3.8.4       \\
    
    \hline
        
    1168 & मुखमुत्तरतः ऊर्द्ध्वा उदायन्न् स & TB\_2.2.10.8       \\
    
    \hline
        
    1169 & मुष्कयोर्निहितः सपः सृत्वेव कामस्य & TB\_2.4.6.6       \\
    
    \hline
        
    1170 & मूद्र्धा हृदये हृदयं मयि & TB\_3.10.8.9       \\
    
    \hline
        
    1171 & मूलं छिनत्ति भ्रातृव्यस्यैव मूलं & TB\_3.2.9.10       \\
    
    \hline
        
    1172 & मृत्युना यजमानं परि गृह्णीयात् & TB\_1.6.8.9       \\
    
    \hline
        
    1173 & मृत्युमे वा हुत्या तर्पयित्वा & TB\_3.9.15.3       \\
    
    \hline
        
    1174 & मेद्ध्यामेवैनां देवयजनीं करोति ओषद्ध्यास्ते & TB\_3.2.9.3       \\
    
    \hline
        
    1175 & य इदमकः तस्मै नमः & TB\_3.7.8.3       \\
    
    \hline
        
    1176 & य उ चैन{-}मेवं ॅवेद & TB\_3.11.10.4       \\
    
    \hline
        
    1177 & य उ चैनमेवं ॅवेद & TB\_2.7.11.3 TB\_3.8.18.7 TB\_3.9.22.3       \\
    
    \hline
        
    1178 & य एतेन यजते य & TB\_2.7.5.2 TB\_2.7.6.2       \\
    
    \hline
        
    1179 & य एवं ॅविद्वान्. वाजपेयेन & TB\_1.3.2.3       \\
    
    \hline
        
    1180 & य एवं ॅविद्वाꣳ{-}श्चातुर्मास्यैर्{-}यजते भ्रातृव्यस्यैव & TB\_1.5.6.5       \\
    
    \hline
        
    1181 & य एवं ॅवेद अभि & TB\_2.2.7.3       \\
    
    \hline
        
    1182 & य एवं ॅवेद उपैन{-}मुपसदो & TB\_2.1.5.9       \\
    
    \hline
        
    1183 & य एवं ॅवेद तस्य & TB\_2.1.5.10       \\
    
    \hline
        
    1184 & य एवं ॅवेद तौ & TB\_2.1.6.4       \\
    
    \hline
        
    1185 & य एवं ॅवेद नास्य & TB\_2.2.9.5       \\
    
    \hline
        
    1186 & य एवास्य दक्षिणतः पाप्मानः & TB\_2.3.9.8       \\
    
    \hline
        
    1187 & यं ॅवा कामयेत प्रियः & TB\_2.3.10.4       \\
    
    \hline
        
    1188 & यः पितुरनुजायाः पुत्रः स & TB\_3.8.4.1       \\
    
    \hline
        
    1189 & यः प्राणतो य आत्मदा & TB\_3.8.18.5       \\
    
    \hline
        
    1190 & यच्चामृतं ॅयच्च मर्त्यं यच्च & TB\_3.12.6.1       \\
    
    \hline
        
    1191 & यच्छ्वयदरुरासीत् तस्मादर्वा नाम यथ्सद्यो & TB\_3.9.21.2       \\
    
    \hline
        
    1192 & यजत्रा मुञ्चतेह मा यज्ञिर्वो & TB\_2.4.4.9       \\
    
    \hline
        
    1193 & यजमानस्य पशून् पाहीत्याह पशुनां & TB\_3.2.1.6       \\
    
    \hline
        
    1194 & यजमानस्या{-}पराभावाय यत्{-}प्रातः अह्न एव & TB\_2.1.5.4       \\
    
    \hline
        
    1195 & यजमानस्यावद्येत् उद्वा माद्येद्{-}यजमानः प्र & TB\_1.6.3.6       \\
    
    \hline
        
    1196 & यजमानायैव तल्लोकꣳ शिꣳषति नास्य & TB\_3.3.8.2       \\
    
    \hline
        
    1197 & यजमानो वै पुरोडाशः प्रजा & TB\_3.3.8.7       \\
    
    \hline
        
    1198 & यजूꣳषि सामभिः सामान्यृग्भिः ऋचो & TB\_2.6.5.8       \\
    
    \hline
        
    1199 & यज्ञियैः केतुभिः सह यं & TB\_1.2.1.9       \\
    
    \hline
        
    1200 & यज्ञो रायो यज्ञ् ईशे & TB\_2.5.5.1       \\
    
    \hline
        
    1201 & यज्ञो वै गौः यज्ञ्मेवालभते & TB\_3.9.8.3       \\
    
    \hline
        
    1202 & यज्ञ्ं नः पान्तु वसवः & TB\_3.1.2.7       \\
    
    \hline
        
    1203 & यज्ञ्मेवारभ्य प्रणीय प्रचरति अपः & TB\_3.2.4.2       \\
    
    \hline
        
    1204 & यज्ञ्वेशसं कुर्यात् यत्पशूनालभते तेनैव & TB\_3.9.1.4       \\
    
    \hline
        
    1205 & यज्ञ्स्थाणु{-}मृच्छेत् अतिहाय पूर्वामाहुतिं जुहोति & TB\_2.1.4.4       \\
    
    \hline
        
    1206 & यज्ञ्स्य काम्यः प्रियः ददामीत्यग्निर्वदति & TB\_2.4.6.7       \\
    
    \hline
        
    1207 & यज्ञ्स्य धृत्यै पुरस्तात् प्रस्तरं & TB\_3.3.6.6       \\
    
    \hline
        
    1208 & यज्ञ्स्यानतिरेकाय वर्.षवृद्ध{-}मसीत्याह वर्.ष वृद्धा & TB\_3.2.2.5       \\
    
    \hline
        
    1209 & यज्ञ्स्यैव चक्षुषी नान्तरेति प्राचीनावीती & TB\_1.6.9.3       \\
    
    \hline
        
    1210 & यतोऽयं पवते यदभि पवते & TB\_2.3.9.2       \\
    
    \hline
        
    1211 & यत् ते नक्षत्रं मृगशीर.षमस्ति & TB\_3.1.1.3       \\
    
    \hline
        
    1212 & यत् पर्यपश्यथ् सरिरस्य मद्ध्ये & TB\_1.2.1.4       \\
    
    \hline
        
    1213 & यत् पश्चात्{-}प्राच्यन्वासीत अनया समदं & TB\_3.3.3.2       \\
    
    \hline
        
    1214 & यत् पुण्यं नक्षत्रम् तद्{-}बट् & TB\_1.5.2.1       \\
    
    \hline
        
    1215 & यत् पुरस्तात् प्रत्य{-}गुपसादयेत् अन्यत्राहुतिपथादिद्ध्मं & TB\_3.2.10.3       \\
    
    \hline
        
    1216 & यत् प्राचीमाहरेत् देवलोक{-}मभिजयेत् यदुदीचीं & TB\_3.2.1.3       \\
    
    \hline
        
    1217 & यत्{-}त्रिषु यूपेष्वा{-}लभेत बहिर्द्धा ऽस्मादिन्द्रियं & TB\_1.8.6.1       \\
    
    \hline
        
    1218 & यत्ते ग्राव्.ण्णा चिच्छिदुः सोम & TB\_3.7.13.1       \\
    
    \hline
        
    1219 & यत्ते पवित्र{-}मर्चिषि अग्ने विततमन्तरा & TB\_1.4.8.2       \\
    
    \hline
        
    1220 & यत्तेऽचितं ॅयदु चितन्ते अग्ने & TB\_3.11.6.1       \\
    
    \hline
        
    1221 & यत्पत्नयः श्रिय{-}मेवास्मिन् तद्दधति नास्मात्तेज & TB\_3.9.4.8       \\
    
    \hline
        
    1222 & यत्पत्नी गृहमेधीय{-}स्याश्ञीयात् गृहमेद्ध्येव स्यात् & TB\_1.6.7.1       \\
    
    \hline
        
    1223 & यत्परिमिता अनुब्रूयात् परिमितमवरुन्धीत अपरिमिता & TB\_3.8.6.2       \\
    
    \hline
        
    1224 & यत्प्रातरिष्टिभि{-}र्यजते अश्वमेव तदन्विच्छति यथ्सायं & TB\_3.9.13.3       \\
    
    \hline
        
    1225 & यत्रर्.षयः प्रथमजा ये पुराणाः & TB\_3.7.6.9       \\
    
    \hline
        
    1226 & यत्रैतेन यज्ञेन यजन्ते एष & TB\_3.9.19.3       \\
    
    \hline
        
    1227 & यत्रैतेन यज्ञेन यजन्ते सभेयो & TB\_3.8.13.3       \\
    
    \hline
        
    1228 & यथर्.षभस्य विष्टपꣳ सꣳस्करोति तादृगेव & TB\_3.8.3.4       \\
    
    \hline
        
    1229 & यथा त्रीण्यावपनानि पूरयेत् तादृक्तत् & TB\_1.1.6.8       \\
    
    \hline
        
    1230 & यथा देवैः सधमादं मदेम & TB\_3.12.3.3       \\
    
    \hline
        
    1231 & यथा यजुरेवैतत् त्रय्यः सूच्यो & TB\_3.9.6.5       \\
    
    \hline
        
    1232 & यथा वै हविषो गृहीतस्य & TB\_3.8.6.1 TB\_3.8.8.1       \\
    
    \hline
        
    1233 & यथा सुप्तं बोधयति तादृगेव & TB\_1.3.1.4       \\
    
    \hline
        
    1234 & यथान्युप्त{-}मेवोपदधे ततो वै स & TB\_3.11.9.3       \\
    
    \hline
        
    1235 & यथायजुरेवैतत् अग्ने देव पणिभिर्{-}वीयमाण & TB\_3.3.9.6       \\
    
    \hline
        
    1236 & यथास्थानं गर्भिण्यः इस् ऒन्ल्य् & TB\_2.3.9.1       \\
    
    \hline
        
    1237 & यथोपधाय वृश्चन्त्यवम् हस्तावव नेनिक्ते & TB\_3.2.10.2       \\
    
    \hline
        
    1238 & यथ् सद्यो निर्वपेत् यावतीमेकेन & TB\_1.7.3.2       \\
    
    \hline
        
    1239 & यथ् सभांराः ततो वै & TB\_2.2.2.6       \\
    
    \hline
        
    1240 & यथ् समूलम् तत् पितृणाम् & TB\_1.6.8.7       \\
    
    \hline
        
    1241 & यथ्{-}सीसम् न स्त्री न & TB\_1.8.5.4       \\
    
    \hline
        
    1242 & यथ्सद्य एतानि हवीꣳषि निर्वपेत् & TB\_1.1.6.7       \\
    
    \hline
        
    1243 & यदग्नयेऽनीकवते पुरोडाशमष्टाकपालं निर्वपति अग्निमेवानीकवन्तं & TB\_1.6.6.2       \\
    
    \hline
        
    1244 & यदग्नि{-}मुद्धरति वसवस्तर्ह्यग्निः तस्मिन्. यस्य & TB\_2.1.10.1       \\
    
    \hline
        
    1245 & यदग्निः यदश्वस्य पदेऽग्नि{-}मादद्ध्यात् रुद्राय & TB\_1.1.5.9       \\
    
    \hline
        
    1246 & यदतिरिक्ता{-}मेकादशिनी{-}मालभेरन्न् अप्रियं भ्रातृव्य{-}मभ्यतिरिच्येत यद् & TB\_1.2.5.4       \\
    
    \hline
        
    1247 & यदत्र शिष्टꣳ रसिनः सुतस्य & TB\_1.4.2.3       \\
    
    \hline
        
    1248 & यददश्चन्द्रमसि मेद्ध्यम् तदस्यामेरयति तां & TB\_3.2.9.14       \\
    
    \hline
        
    1249 & यदप्रहृत्य परिधीञ्जुहुयात् अन्तराधानाभ्यां घासं & TB\_1.6.3.10       \\
    
    \hline
        
    1250 & यदभ्यजयन्न् तदभिजितोऽभिजित्त्वम् यं कामयेता{-}नपजय्यं & TB\_1.5.2.4       \\
    
    \hline
        
    1251 & यदश्विभ्यां धानाः पूष्णः करम्भः & TB\_1.5.11.3       \\
    
    \hline
        
    1252 & यदश्विभ्यां पूष्णे पुरोडाशं द्वादशकपालं & TB\_1.8.3.4       \\
    
    \hline
        
    1253 & यदस्मिन्नादित्ये तदेनमब्रवीत् एतन्मे प्रयच्छ & TB\_2.2.10.2       \\
    
    \hline
        
    1254 & यदस्य पारे रजसः शुक्रं & TB\_3.7.8.1       \\
    
    \hline
        
    1255 & यदस्या यज्ञियमासीत् तदमुष्या{-}मदधात् तददश्चन्द्रमसि & TB\_1.1.3.3       \\
    
    \hline
        
    1256 & यदाखुकरीषꣳ सभांरो भवति यदेवास्य & TB\_1.1.3.4       \\
    
    \hline
        
    1257 & यदागच्छात् पथिभि र्देवयानैः इष्टापूर्ते & TB\_3.7.13.4       \\
    
    \hline
        
    1258 & यदाग्नेयो भवति अग्निमुखा ह्यृद्धिः & TB\_2.7.2.1       \\
    
    \hline
        
    1259 & यदाग्नेयो भवति आग्नेयो वै & TB\_2.7.3.1       \\
    
    \hline
        
    1260 & यदाज्य{-}मुच्छिष्यते तस्मिन्{-}रशनां न्युनत्ति प्रजापतिर्वा & TB\_3.8.2.3       \\
    
    \hline
        
    1261 & यदाज्यम् अनडुहस्तण्डुलाः मिथुनमेवाव{-}रुन्धे घृते & TB\_1.1.6.6       \\
    
    \hline
        
    1262 & यदार्.षेयं ॅवृणीत यद्धोतारम् प्रमायुको & TB\_1.6.9.2       \\
    
    \hline
        
    1263 & यदाह पवस्व वाचो अग्रिय & TB\_1.8.8.2       \\
    
    \hline
        
    1264 & यदाहवनीय उद्वायेत् यत्तं मन्थेत् & TB\_1.4.7.2       \\
    
    \hline
        
    1265 & यदाहवनीय{-}मनुद्वाप्य गार्.हपत्यं मन्थेत् विच्छिन्द्यात् & TB\_1.4.4.7       \\
    
    \hline
        
    1266 & यदित इतो लोमानि दतो & TB\_1.2.6.5       \\
    
    \hline
        
    1267 & यदिदं किञ्च य एवं & TB\_2.2.11.6       \\
    
    \hline
        
    1268 & यदिदं दिवो यददः पृथिव्याः & TB\_1.2.1.2       \\
    
    \hline
        
    1269 & यदिन्द्राहन्{-}प्रथमजा महीनाम् आन्मायिनाममिनाः प्रोत & TB\_2.5.4.3       \\
    
    \hline
        
    1270 & यदुपर्युपरि शिरो हरेत् प्राणान् & TB\_1.1.5.7       \\
    
    \hline
        
    1271 & यदेककपाल आज्यमानयति यजमानमेव पशुभिः & TB\_1.6.3.5       \\
    
    \hline
        
    1272 & यदेककपाल{-}माहवनीये जुहोति यजमानमेव सुवर्गं & TB\_1.6.3.7       \\
    
    \hline
        
    1273 & यदेनश्च कृमा नूतनं ॅयत्पुराणं & TB\_3.7.12.5       \\
    
    \hline
        
    1274 & यदेनꣳ समयच्छत् तथ्समिधः समित्त्वम् & TB\_2.1.3.8       \\
    
    \hline
        
    1275 & यदेवास्याफ्सु प्रविष्टम् तदेवावरुन्धे बहु & TB\_1.3.5.3       \\
    
    \hline
        
    1276 & यद् दक्षिणा ब्रह्मणे च & TB\_3.7.2.6       \\
    
    \hline
        
    1277 & यद् वाग्वदन्त्य{-}विचेतनानि राष्ट्री देवानां & TB\_2.4.6.11       \\
    
    \hline
        
    1278 & यद्{-}देवा देवहेडनम् देवासश्चकृमा वयम् & TB\_2.6.6.1       \\
    
    \hline
        
    1279 & यद्{-}द्वे समिधावादद्ध्यात् भ्रातृव्यमस्मै जनयेत् & TB\_2.1.3.9       \\
    
    \hline
        
    1280 & यद्{-}यजते यामेव देवा ऊर्जमवारुन्धत & TB\_1.4.9.4       \\
    
    \hline
        
    1281 & यद्{-}युद्द्रुतस्य स्कन्देत् यत् ततोऽहुत्वा & TB\_1.4.3.5       \\
    
    \hline
        
    1282 & यद्{-}वसन्तः यो वसन्ताऽग्निमाधत्ते मुख्य & TB\_1.1.2.7       \\
    
    \hline
        
    1283 & यद्{-}वही तेनाग्नेयः यदृषभः तेनैन्द्रः & TB\_1.6.1.8       \\
    
    \hline
        
    1284 & यद्दर्.शपूर्णमासौ यजते अश्वस्यैव मेद्ध्यस्य & TB\_3.9.23.2       \\
    
    \hline
        
    1285 & यद्दर्भमयी रशना भवति पुनात्येवैनम् & TB\_3.8.2.4       \\
    
    \hline
        
    1286 & यद्दर्भानद्ध्यासीत यामेवाग्ना{-}वाहुतिं जुहुयात् तामद्ध्यासीत & TB\_3.7.3.4       \\
    
    \hline
        
    1287 & यद्दिदीक्षे मनसा यच्च वाचा & TB\_3.7.14.1       \\
    
    \hline
        
    1288 & यद्देवा देवहेडनं देवासश्चकृमा वयं & TB\_3.7.12.1       \\
    
    \hline
        
    1289 & यद्यज्ञ्मुखे यज्ञ्मुखे जुहुयात् पशुभि{-}र्यजमानं & TB\_3.8.8.4       \\
    
    \hline
        
    1290 & यद्यश्वमुपतपद्विन्देत् आग्नेय{-}मष्टाकपालं निर्वपेत् सौम्यं & TB\_3.9.17.1       \\
    
    \hline
        
    1291 & यद्येनानि पशवोऽभितिष्ठेयुः न तत् & TB\_3.3.2.3       \\
    
    \hline
        
    1292 & यद्वा इदं किं च & TB\_2.3.5.5 TB\_3.2.2.4       \\
    
    \hline
        
    1293 & यद्विः षण्णेन जुहुयात् अप्रजा & TB\_3.7.2.1       \\
    
    \hline
        
    1294 & यद्वै शिवम् तन्मयः आत्मन & TB\_2.2.5.5       \\
    
    \hline
        
    1295 & यद्वो देवा अतिपादयानि वाचा & TB\_3.7.11.2       \\
    
    \hline
        
    1296 & यन् माꣳस{-}मश्ञीयात् यथ्स्त्रिय{-}मुपेयात् निर्वीर्यः & TB\_1.1.9.8       \\
    
    \hline
        
    1297 & यन्त्यस्य पितरो हवम् स & TB\_2.3.8.3       \\
    
    \hline
        
    1298 & यन्न प्रति निर्वपेत् रत्निन & TB\_1.7.3.7       \\
    
    \hline
        
    1299 & यन्निमार्ष्टि तदोषधीनाम् यद्द्वितीयम् तत्{-}पितृणाम् & TB\_2.1.4.7       \\
    
    \hline
        
    1300 & यन्महाꣳसि जुहोति अमुमेव लोकमवरुन्धे & TB\_3.8.18.3       \\
    
    \hline
        
    1301 & यमनक्षत्राण्यन्यानि कृत्तिकाः प्रथमम् विशाखे & TB\_1.5.2.7       \\
    
    \hline
        
    1302 & यमो वा अकामयत पितृणां & TB\_3.1.5.14       \\
    
    \hline
        
    1303 & यम्यै यमसूम् अथर्वभ्योऽवतोकाम् सम्ॅवथ्सराय & TB\_3.4.11.1       \\
    
    \hline
        
    1304 & यशो मुखम् त्विषिः केशाश्च & TB\_2.6.5.4       \\
    
    \hline
        
    1305 & यश्च यजुषा ऽजुहोद्यश्च तूष्णीम् & TB\_2.1.9.2       \\
    
    \hline
        
    1306 & यस्त आत्मा पशुषु प्रविष्टः & TB\_1.2.1.22       \\
    
    \hline
        
    1307 & यस्ताम्यति अन्तमेष यज्ञ्स्य गच्छति & TB\_1.4.4.6       \\
    
    \hline
        
    1308 & यस्ते देव वरुण त्रिष्टुप्छन्दाः & TB\_1.4.2.4       \\
    
    \hline
        
    1309 & यस्मिन् ब्रह्माऽभ्यजयथ् सर्वमेतत् अमुञ्च & TB\_3.1.2.5       \\
    
    \hline
        
    1310 & यस्मिन्नग्निः यद्{-}वैश्वदेवेन यजते एतमेव & TB\_1.4.10.4       \\
    
    \hline
        
    1311 & यस्य तल्पसद्य{-}मभिजितꣳ स्यात् स & TB\_1.2.6.6       \\
    
    \hline
        
    1312 & यस्य पर्णमयः संभारो भवति & TB\_1.1.3.11       \\
    
    \hline
        
    1313 & यस्य प्रातस्सवने सोमोऽतिरिच्यते माद्ध्यन्दिनं & TB\_1.4.5.1       \\
    
    \hline
        
    1314 & यस्या नायतने{-}ऽन्यत्राग्ने{-}राहुतीर् जुहोति सावित्रिया & TB\_3.8.8.3       \\
    
    \hline
        
    1315 & यस्यां तत् प्रत्यतिष्ठदिति यानि & TB\_1.5.12.3       \\
    
    \hline
        
    1316 & यस्याग्निहोत्रमहुतꣳ सूर्योऽभ्युदेति यद्{-}यन्ते स्यात् & TB\_2.1.9.3       \\
    
    \hline
        
    1317 & यस्यायमृषभो हविः इन्द्राय परिणीयते & TB\_2.4.7.3       \\
    
    \hline
        
    1318 & यस्याश्वो मेधाय प्रोक्षितो{-}ऽद्ध्येति सौर्यं & TB\_3.9.17.5       \\
    
    \hline
        
    1319 & यस्यैव{-}मग्निराधीयते प्रतीच्यस्य श्रीरेति भद्रो & TB\_1.1.4.5       \\
    
    \hline
        
    1320 & यस्यैवं जुह्वति भवत्येव यं & TB\_2.1.4.2       \\
    
    \hline
        
    1321 & यस्यैवमग्नि{-}राधीयते प्र प्रजया पशुभिः & TB\_1.1.4.8       \\
    
    \hline
        
    1322 & या दुन्दुभौ परमयैव वाचाऽवरां & TB\_1.3.6.3       \\
    
    \hline
        
    1323 & या मेवाफ्स्वाहुतिं जुहुयात् तां & TB\_3.7.3.5       \\
    
    \hline
        
    1324 & यां प्रथमा{-}मिष्टका{-}मुपदधाति इमं तया & TB\_3.11.10.1       \\
    
    \hline
        
    1325 & यां ॅवै हस्त्यामिडामादधाति वाचः & TB\_3.3.8.5       \\
    
    \hline
        
    1326 & याः पुरस्तात् प्रस्रवन्ति उपरिष्टाथ् & TB\_3.7.4.1       \\
    
    \hline
        
    1327 & याः पूर्वाः प्रजा असृक्षि & TB\_1.6.2.4       \\
    
    \hline
        
    1328 & यातयाम्ना हविषा यजेत अथो & TB\_3.2.3.9       \\
    
    \hline
        
    1329 & यामेव पशव ऊर्जमवा{-}रुन्धत तां & TB\_1.4.9.5       \\
    
    \hline
        
    1330 & यावती सप्तदशस्य मात्रा प्रजापतिर्वै & TB\_1.5.10.6       \\
    
    \hline
        
    1331 & यावत्{-}प्राणाः यावदेवास्यास्ति तेन सह & TB\_1.3.7.5       \\
    
    \hline
        
    1332 & यावदेका देवता कामयते यावदेका & TB\_3.2.6.4       \\
    
    \hline
        
    1333 & यावदेव वीर्यं तदस्मिन् दधाति & TB\_3.12.5.13       \\
    
    \hline
        
    1334 & यावद्वा अद्ध्वर्युः प्रस्तरं प्रहरति & TB\_3.3.9.5       \\
    
    \hline
        
    1335 & यावन्तो वनस्पतयः अस्यां पृथिव्यामधि & TB\_3.12.6.4       \\
    
    \hline
        
    1336 & यावन्तꣳ ह वै त्रय्या & TB\_3.10.11.6       \\
    
    \hline
        
    1337 & यावानेव पुरुषः तस्मिन् वीर्यं & TB\_1.7.6.5       \\
    
    \hline
        
    1338 & याश्च कूप्या याश्च नाद्याः & TB\_3.12.7.4       \\
    
    \hline
        
    1339 & यासामषाढा मधु भक्षयन्ति ता & TB\_3.1.2.4       \\
    
    \hline
        
    1340 & युक्ताः स्थ वहत देवा & TB\_3.7.9.2       \\
    
    \hline
        
    1341 & युञ्जन्ति ब्रद्ध्नमित्याह असौ वा & TB\_3.9.4.1       \\
    
    \hline
        
    1342 & युवꣳ सिन्धूꣳ रभिशस्ते{-}रवद्यात् अग्नीषोमा{-}वमुञ्चतं & TB\_3.5.7.3       \\
    
    \hline
        
    1343 & युवꣳ सुराममश्विना नमुचावाऽसुरे सचा & TB\_1.4.2.1       \\
    
    \hline
        
    1344 & युष्मानिन्द्रो ऽवृणीत वृत्रतूये यूयमिन्द्रमवृणीद्ध्वं & TB\_3.2.5.4       \\
    
    \hline
        
    1345 & युष्माꣳस्तेऽनु येऽस्मिॅल्लोके मां ते & TB\_1.3.10.9       \\
    
    \hline
        
    1346 & यूयं पात स्वस्तिभिः सदा & TB\_2.5.6.4 TB\_2.5.8.5       \\
    
    \hline
        
    1347 & यूयमुग्रा मरुतः पृश्निमातरः इन्द्रेण & TB\_2.5.2.3       \\
    
    \hline
        
    1348 & ये अग्निदग्धा येऽनग्नि{-}दग्धाः येऽमुं & TB\_3.1.1.7       \\
    
    \hline
        
    1349 & ये अन्तरिक्षं पृथिवीं क्षियन्ति & TB\_3.1.1.6       \\
    
    \hline
        
    1350 & ये केशिनः प्रथमाः सत्रमासत & TB\_2.7.17.1       \\
    
    \hline
        
    1351 & ये ते अग्ने शिवे & TB\_1.1.7.3       \\
    
    \hline
        
    1352 & ये ते सहस्रमयुतं पाशाः & TB\_3.10.8.2       \\
    
    \hline
        
    1353 & ये यज्ञे धुवनं तन्वते & TB\_3.9.6.3       \\
    
    \hline
        
    1354 & ये वै चत्वारः स्तोमाः & TB\_1.5.11.1       \\
    
    \hline
        
    1355 & ये वै यज्वानः ते & TB\_1.6.9.6       \\
    
    \hline
        
    1356 & ये सजाताः समनसः जीवा & TB\_2.6.3.5       \\
    
    \hline
        
    1357 & येनासिञ्चद् बलमिन्द्रे प्रजापतिः इदं & TB\_3.7.6.13       \\
    
    \hline
        
    1358 & यो गर्भमोषधीनाम् गवां कृणोत्यर्वताम् & TB\_2.4.5.6       \\
    
    \hline
        
    1359 & यो दीक्षामति{-}रेचयति सप्ताहं प्रचरन्ति & TB\_3.8.10.5       \\
    
    \hline
        
    1360 & यो दीदाय समिद्धः स्वे & TB\_3.11.6.3       \\
    
    \hline
        
    1361 & यो देवानां देवतमस्तपोजाः तस्मै & TB\_3.7.9.4       \\
    
    \hline
        
    1362 & यो नः सपत्नो यो & TB\_3.7.6.24       \\
    
    \hline
        
    1363 & यो ब्रह्मणा कर्मणा द्वेष्टि & TB\_3.7.6.5       \\
    
    \hline
        
    1364 & यो वा अग्निहोत्रस्योपसदो वेद & TB\_2.1.5.8       \\
    
    \hline
        
    1365 & यो वा अयथादेवतं ॅयज्ञ्मुपचरति & TB\_3.3.10.1       \\
    
    \hline
        
    1366 & यो वा अविद्वान्{-}निवर्तयते विशीर्.षा & TB\_2.3.3.1       \\
    
    \hline
        
    1367 & यो वा अश्वमेधे तिस्रः & TB\_3.8.21.4       \\
    
    \hline
        
    1368 & यो वा अश्वस्य मेद्ध्यस्य & TB\_3.9.23.1       \\
    
    \hline
        
    1369 & यो वाजपेयेन यजते बार्.हस्पत्य & TB\_1.3.6.9       \\
    
    \hline
        
    1370 & यो वै नक्षत्रियं प्रजापतिं & TB\_1.5.2.2       \\
    
    \hline
        
    1371 & यो वै ब्रह्मणे देवेभ्यः & TB\_3.8.3.1       \\
    
    \hline
        
    1372 & यो वै सोमेन सूयते & TB\_2.7.5.1       \\
    
    \hline
        
    1373 & यो ह वै मुहूर्तानां & TB\_3.10.10.3       \\
    
    \hline
        
    1374 & योगक्षेमस्य क्लृप्त्यै युक्तं क्रियाता & TB\_3.3.3.4       \\
    
    \hline
        
    1375 & योनि{-}मानायतनवान् भवति स एवं & TB\_3.9.21.3       \\
    
    \hline
        
    1376 & योऽग्निमाधत्ते ऐन्द्राग्न{-}मेकादशकपाल{-} मनु निर्वपेत् & TB\_1.1.6.5       \\
    
    \hline
        
    1377 & यौ देवानां भिषजौ हव्यवाहौ & TB\_3.1.2.11       \\
    
    \hline
        
    1378 & यꣳ राजानं ॅविशो नापचायेयुः & TB\_2.7.18.2       \\
    
    \hline
        
    1379 & रक्षसा{-}मन्तर्.हित्यै पुरोडाशं ॅवा अधिश्रितं & TB\_3.2.8.6       \\
    
    \hline
        
    1380 & रक्षसां भागोऽसीत्याह तुषैरेव रक्षाꣳसि & TB\_3.2.5.11       \\
    
    \hline
        
    1381 & रक्षसामपहत्यै स्फ्यस्य वर्त्मन्थ्{-}सादयति यज्ञ्स्य & TB\_3.2.9.15       \\
    
    \hline
        
    1382 & रक्षाꣳसि यज्ञ्ꣳ हन्युः यदुदङ्ङ् & TB\_1.6.3.8       \\
    
    \hline
        
    1383 & रजाꣳसि चित्रा वि चरन्ति & TB\_2.4.5.4       \\
    
    \hline
        
    1384 & रत्निनामेतानि हवीꣳषि भवन्ति एते & TB\_1.7.3.1       \\
    
    \hline
        
    1385 & राजा त्वा वरुणो नयतु & TB\_2.2.5.2       \\
    
    \hline
        
    1386 & राज्जुदाल{-}मग्निष्ठं मिनोति भ्रूणहत्याया अपहत्यै & TB\_3.8.20.1       \\
    
    \hline
        
    1387 & राज्ञी विराज्ञी सम्राज्ञी स्वराज्ञी & TB\_3.10.6.1       \\
    
    \hline
        
    1388 & राज्यमेवास्मिन् प्रति दधाति स्वां & TB\_1.7.4.3       \\
    
    \hline
        
    1389 & राडसि बृहती श्रीरसीन्द्रपत्नी धर्मपत्नी & TB\_3.11.1.20       \\
    
    \hline
        
    1390 & रात्री व्यख्यदायती पुरुत्रा देव्यक्षभिः & TB\_2.4.6.10       \\
    
    \hline
        
    1391 & राष्ट्रं पसः राष्ट्रमव विश्याहन्ति & TB\_3.9.7.4       \\
    
    \hline
        
    1392 & राष्ट्रमेव तेजस्व्यकः अपामोषधीनाꣳ रसः & TB\_1.7.5.5       \\
    
    \hline
        
    1393 & राष्ट्रमेव वर्चस्व्यकः सूर्यत्वचसः स्थेत्याह & TB\_1.7.5.3       \\
    
    \hline
        
    1394 & राष्ट्रादेव ते व्यवच्छिद्यन्ते परा & TB\_3.8.9.5       \\
    
    \hline
        
    1395 & रुक्मो भवति सुवर्गस्य लोकस्या{-}नुख्यात्यै & TB\_3.9.20.2       \\
    
    \hline
        
    1396 & रुद्रानेव तत्प्रीणाति लाजैर्जुहोति आदित्यानां & TB\_3.8.14.4       \\
    
    \hline
        
    1397 & रुद्रास्त्वा{-}ऽञ्जन्तु त्रैष्टुभेन छन्दसेति वावाता & TB\_3.9.4.7       \\
    
    \hline
        
    1398 & रुद्रेभ्यो यज्ञ्ं प्रब्रवीमि इदमहं & TB\_3.7.4.7       \\
    
    \hline
        
    1399 & रुद्रो भूत्वाऽग्निरनूत्थाय अद्ध्वर्युं च & TB\_1.6.1.2       \\
    
    \hline
        
    1400 & रुद्रो वा अकामयत पशुमान्थ् & TB\_3.1.4.4       \\
    
    \hline
        
    1401 & रुद्रो वा एषः यदग्निः & TB\_2.1.3.1       \\
    
    \hline
        
    1402 & रुशन्तमग्निं दर्.शतं बृहन्तम् वपावन्तं & TB\_3.6.10.2       \\
    
    \hline
        
    1403 & रूपाण्येव तेन कुर्वते वैश्वानरं & TB\_1.8.4.2       \\
    
    \hline
        
    1404 & रेतो हृदये हृदयं मयि & TB\_3.10.8.7       \\
    
    \hline
        
    1405 & रौद्रं गवि वायव्य{-}मुपसृष्टम् आश्विनं & TB\_2.1.7.1       \\
    
    \hline
        
    1406 & रौद्रं चरुं निर्वपेत् यदि & TB\_3.9.17.3       \\
    
    \hline
        
    1407 & लोकोऽसि स्वर्गोऽसि अनन्तोऽस्य पारोऽसि & TB\_3.11.1.1       \\
    
    \hline
        
    1408 & लोमानि हृदये हृदयं मयि & TB\_3.10.8.8       \\
    
    \hline
        
    1409 & वज्रस्य वा एषोऽनुमानाय अनुमतवज्रः & TB\_2.7.3.3       \\
    
    \hline
        
    1410 & वज्रो वै स्फ्यः यदन्वञ्चं & TB\_3.2.10.1       \\
    
    \hline
        
    1411 & वडबा उप रुन्धन्ति मिथुनत्वाय & TB\_3.8.22.3       \\
    
    \hline
        
    1412 & वथ्सेभ्यश्च वा एताः पुरा & TB\_3.2.1.5       \\
    
    \hline
        
    1413 & वथ्सेभ्यो मनुष्येभ्यः पुनर्दोहाय कल्पतां & TB\_3.7.4.17       \\
    
    \hline
        
    1414 & वनस्पतय इद्ध्मः दिशः परिधयः & TB\_2.1.5.2       \\
    
    \hline
        
    1415 & वनस्पतिभ्यः स्वाहेति वनस्पति{-}होमाञ्जुहोति आरण्यस्यान्नाद्यस्यावरुद्ध्यै & TB\_3.8.17.5       \\
    
    \hline
        
    1416 & वम्रीभि{-}रनुवित्तं गुहासु श्रोत्रं त & TB\_1.2.1.3       \\
    
    \hline
        
    1417 & वयꣳ सोम व्रते तव & TB\_2.4.2.7       \\
    
    \hline
        
    1418 & वयꣳ स्याम पतयो रयीणाम् & TB\_3.5.7.2       \\
    
    \hline
        
    1419 & वरुणः क्षत्रमिन्द्रियम् भगेन सविता & TB\_2.6.13.3       \\
    
    \hline
        
    1420 & वरुणस्य सुषुवाणस्य दशधेन्द्रियं ॅवीर्यं & TB\_1.8.1.1       \\
    
    \hline
        
    1421 & वरुणो धर्मपतीनाम् एतदेव सर्वं & TB\_1.7.4.2       \\
    
    \hline
        
    1422 & वर्ष्मन्दिवः नाभा पृथिव्याः यथाऽयं & TB\_3.7.7.14       \\
    
    \hline
        
    1423 & वषट्कारेण वज्रेण कृत्याꣳ हन्मि & TB\_2.4.2.4       \\
    
    \hline
        
    1424 & वसन्तेनर्तुना देवाः वसवस्त्रिवृता स्तुतम् & TB\_2.6.19.1       \\
    
    \hline
        
    1425 & वसवो वा अकामयन्त अग्रं & TB\_3.1.5.8       \\
    
    \hline
        
    1426 & वसीय एव चेतयते तं & TB\_2.1.2.3       \\
    
    \hline
        
    1427 & वसुवने वसुधेयस्य वियन्तु यज & TB\_2.6.10.5       \\
    
    \hline
        
    1428 & वसुवने वसुधेयस्य वीतां ॅयज & TB\_2.6.10.3       \\
    
    \hline
        
    1429 & वसूनां त्वाऽऽधीतेन रुद्राणामूर्म्या आदित्यानां & TB\_2.5.7.1       \\
    
    \hline
        
    1430 & वसूनां पवित्रमसीत्याह प्राणा वै & TB\_3.2.3.3       \\
    
    \hline
        
    1431 & वसूनाꣳ श्रविष्ठाः भूतं परस्ताद्{-}भूतिरवस्तात् & TB\_1.5.1.5       \\
    
    \hline
        
    1432 & वहिनी धेनुर्{-}दक्षिणा यद्{-}वहिनी तेनाग्नेयी & TB\_1.7.1.4       \\
    
    \hline
        
    1433 & वह्निनैव वह्नि यज्ञ्स्याव{-}रुन्धे मिथुनौ & TB\_1.1.6.11       \\
    
    \hline
        
    1434 & वाग्देवी जुषतामिदꣳ हविः चक्षुः & TB\_2.5.1.3       \\
    
    \hline
        
    1435 & वाग्वै वाजस्य प्रसवः य & TB\_1.3.2.5       \\
    
    \hline
        
    1436 & वाचे पुरुषमालभते प्राणमपानं ॅव्यानमुदानं & TB\_3.4.18.1       \\
    
    \hline
        
    1437 & वाजं जिगिवाꣳसं वाजिनं ॅवाजजितं & TB\_3.7.6.18       \\
    
    \hline
        
    1438 & वाजजित्यायै संमार्ज्मि अग्निमन्ना{-}दमन्नाद्याय उपहूतो & TB\_3.7.6.15       \\
    
    \hline
        
    1439 & वाजसनिꣳ रयिमस्मे सुवीरम् प्रशस्तं & TB\_1.4.2.2       \\
    
    \hline
        
    1440 & वाजस्याव{-}रुद्ध्यै जाय एहि सुवो & TB\_1.3.7.2       \\
    
    \hline
        
    1441 & वाजिनाꣳ साम गायते अन्नं & TB\_1.3.6.2       \\
    
    \hline
        
    1442 & वाजिनो यजति अग्निर्वायुः सूर्यः & TB\_1.6.3.9       \\
    
    \hline
        
    1443 & वाजिनो वाजं धावत काष्ठां & TB\_1.3.6.5       \\
    
    \hline
        
    1444 & वाज्येवैनं पीत्वा भवति आऽस्य & TB\_1.3.2.4       \\
    
    \hline
        
    1445 & वायव एवैनान् परि ददाति & TB\_3.2.1.4       \\
    
    \hline
        
    1446 & वायुमेव तदनुवथ्सर{-}माप्नोति तस्मा{-}च्छुनासीरीयेण यजमानः & TB\_1.4.10.3       \\
    
    \hline
        
    1447 & वायुरस्यन्तरिक्षे श्रितः दिवः प्रतिष्ठा & TB\_3.11.1.9       \\
    
    \hline
        
    1448 & वायुर्वा अकामयत कामचारमेषु लोकेष्वभिजयेयमिति & TB\_3.1.4.13       \\
    
    \hline
        
    1449 & वायोरेव सायुज्य{-}मुपैति ब्रह्मवादिनो वदन्ति & TB\_1.4.10.10       \\
    
    \hline
        
    1450 & वारुणो वा अश्वः तं & TB\_3.9.16.1       \\
    
    \hline
        
    1451 & वारुणो वा अश्वः स्वयैवैनं & TB\_2.2.5.3       \\
    
    \hline
        
    1452 & वार्त्रघ्नमेव विजित्यै हिरण्यं दक्षिणा & TB\_1.6.1.7       \\
    
    \hline
        
    1453 & वार्त्रहत्याय शवसे पृतना{-}साह्याय च & TB\_2.5.6.1       \\
    
    \hline
        
    1454 & वाशाः स्थेत्याह राष्ट्रमेव वश्यकः & TB\_1.7.5.4       \\
    
    \hline
        
    1455 & वि वा एतस्य यज्ञ्श्छिद्यते & TB\_1.4.3.6       \\
    
    \hline
        
    1456 & वि वा एष इन्द्रियेण & TB\_3.7.3.1       \\
    
    \hline
        
    1457 & विजयभागꣳ समिन्धतां अग्ने दीदाय & TB\_3.7.4.6       \\
    
    \hline
        
    1458 & विड्वा एकविꣳशः राष्ट्रꣳ सप्तदशः & TB\_1.8.8.5       \\
    
    \hline
        
    1459 & विदुरेनं नाम्ना तदस्मै रुक्मं & TB\_2.2.10.3       \\
    
    \hline
        
    1460 & विधेम हविषा वयं भजतां & TB\_3.7.5.9       \\
    
    \hline
        
    1461 & विपश्चिते पवमानाय गायत मही & TB\_3.10.8.1       \\
    
    \hline
        
    1462 & विभूर्मात्रा प्रभूः पित्रेत्यश्व{-}नामानि जुहोति & TB\_3.8.17.1       \\
    
    \hline
        
    1463 & विभूर्मात्रा प्रभूः पित्रेत्याह इयं & TB\_3.8.9.1       \\
    
    \hline
        
    1464 & वियन्त्वाज्यस्य होतर्यज होता यक्षद्दैव्या & TB\_2.6.11.6.       \\
    
    \hline
        
    1465 & विराजमेव तैराप्त्वा यजमानो{-}ऽवरुन्धे एकादश & TB\_3.9.2.4       \\
    
    \hline
        
    1466 & विराट्छन्द इहेन्द्रियम् धेनुर्गौर्न वयो & TB\_2.6.18.4       \\
    
    \hline
        
    1467 & विश्रयस्व दिशो महीः विशस्त्वा & TB\_2.7.15.4       \\
    
    \hline
        
    1468 & विश्वमेवायुर्{-}यजमाने दधाति इन्द्रस्य बाहुरसि & TB\_3.3.6.9       \\
    
    \hline
        
    1469 & विश्वसृजः प्रथमाः सत्र{-}मासत सहस्र & TB\_3.12.9.7       \\
    
    \hline
        
    1470 & विश्वा आशाः पृतनाः सजंयञ्जयन्न् & TB\_2.4.7.9       \\
    
    \hline
        
    1471 & विश्वानेव तद्देवान् प्रीणाति धानाभि{-}र्जुहोति & TB\_3.8.14.5       \\
    
    \hline
        
    1472 & विश्वे देवा यजमानश्च सीदत & TB\_3.7.7.11       \\
    
    \hline
        
    1473 & विश्वे वै देवा अकामयन्त & TB\_3.1.5.5       \\
    
    \hline
        
    1474 & विषूचः शत्रू{-}नपबाधमानौ अप क्षुधं & TB\_3.1.1.12       \\
    
    \hline
        
    1475 & विष्णुर्यज्ञ्ः देवताश्चैव यज्ञ्ं चाव & TB\_1.6.1.6       \\
    
    \hline
        
    1476 & विष्णुर्वा अकामयत पुण्यं श्लोकं & TB\_3.1.5.7       \\
    
    \hline
        
    1477 & वीतामाज्यस्य होतर्यज होता यक्षद् & TB\_2.6.7.4       \\
    
    \hline
        
    1478 & वीतिहोत्रं त्वा कव इत्याह & TB\_3.3.6.10       \\
    
    \hline
        
    1479 & वीर्यायान्नाद्यायाभिषिञ्चामि देवस्य त्वा सवितुः & TB\_2.6.5.3       \\
    
    \hline
        
    1480 & वृङ्क्त एषामिन्द्रियं ॅवीर्यम् श्रेष्ठ & TB\_3.2.5.10       \\
    
    \hline
        
    1481 & वृद्धानिन्द्रः प्रयच्छति पौष्णश्चरुर्{-}भवति इयं & TB\_1.7.2.5       \\
    
    \hline
        
    1482 & वृश्चतश्च अत्यꣳहो हारुणिः ब्रह्मचारिणे & TB\_3.10.9.3       \\
    
    \hline
        
    1483 & वृश्चतश्च सैषा मीमाꣳसाऽग्निहोत्र एव & TB\_3.10.9.2       \\
    
    \hline
        
    1484 & वृषा सो अꣳशुः पवते & TB\_2.4.5.1       \\
    
    \hline
        
    1485 & वृषा ऽस्यꣳशुर्{-}वृषभाय गृह्यसे वृषाऽयमुग्रो & TB\_2.4.7.1       \\
    
    \hline
        
    1486 & वृष्टिद्यावा रीत्यापा शम्भुवौ मयोभुवौ & TB\_3.5.10.2       \\
    
    \hline
        
    1487 & वृष्टिमेव नियच्छति अवाचीनाग्रा हि & TB\_3.3.1.3       \\
    
    \hline
        
    1488 & वेत्वाज्यस्य होतर्यज होता यक्षथ् & TB\_2.6.17.3       \\
    
    \hline
        
    1489 & वेत्वाज्यस्य होतर्यज होता यक्षदिडाभि{-}रिन्द्र{-}मीडितम् & TB\_2.6.7.2       \\
    
    \hline
        
    1490 & वेत्वाज्यस्य होतर्यज होता यक्षदोजो & TB\_2.6.7.3       \\
    
    \hline
        
    1491 & वेदेन रूपे व्यकरोत् सतासती & TB\_2.6.2.3       \\
    
    \hline
        
    1492 & वेदेनाभि वासयति तस्मात् केशैः & TB\_3.2.8.8       \\
    
    \hline
        
    1493 & वैराजो वै पुरुषः विराज{-}मेवालभते & TB\_3.9.8.2       \\
    
    \hline
        
    1494 & वैरूपेण विशौजसा हविरिन्द्रे वयो & TB\_2.6.19.2       \\
    
    \hline
        
    1495 & वैश्वदेव{-}मालभन्ते देवता एवाव{-}रुन्धते द्यावापृथिव्यां & TB\_1.2.5.2       \\
    
    \hline
        
    1496 & वैश्वदेवेन चतुरो मासोऽवृञ्जतेन्द्रराजानः ताञ्छीर्.षन्नि & TB\_1.5.6.4       \\
    
    \hline
        
    1497 & वैश्वदेवेन वै प्रजापतिः प्रजा & TB\_1.6.2.1 TB\_1.6.8.1       \\
    
    \hline
        
    1498 & वैश्वदेवो वा अश्वः तं & TB\_3.9.11.1       \\
    
    \hline
        
    1499 & वैश्वदेव्यो वै प्रजाः ता & TB\_1.7.10.2       \\
    
    \hline
        
    1500 & वैश्वानरो रश्मिभिर्मा पुनातु वातः & TB\_1.4.8.3       \\
    
    \hline
        
    1501 & वैष्णवं त्रिकपालम् वीर्यं ॅवा & TB\_1.7.2.2       \\
    
    \hline
        
    1502 & व्यचस्वतीरुर्विया विश्रयन्ताम् पतिभ्यो नजनयः & TB\_3.6.3.3       \\
    
    \hline
        
    1503 & व्याघ्रोऽयमग्नौ चरति प्रविष्टः ऋषीणां & TB\_2.7.15.1       \\
    
    \hline
        
    1504 & व्यानादुपाꣳशु सवनम् वाच ऐन्द्रवायवम् & TB\_1.5.4.2       \\
    
    \hline
        
    1505 & व्यानꣳ संधत्तं तं मे & TB\_1.1.1.3       \\
    
    \hline
        
    1506 & व्युच्छन्ती दुहिता दिवः अपो & TB\_3.1.3.2       \\
    
    \hline
        
    1507 & व्रतानां ॅव्रतपते व्रतं चरिष्यामि & TB\_3.7.4.8       \\
    
    \hline
        
    1508 & शतमानं भवति शतायुः पुरुषः & TB\_1.7.6.2       \\
    
    \hline
        
    1509 & शतायुः पुरुषः शतेन्द्रियः आयुष्येवेन्द्रिये & TB\_1.7.8.2 TB\_1.8.6.5 TB\_3.8.20.3       \\
    
    \hline
        
    1510 & शमीगर्भादग्निं मन्थति एषा वा & TB\_1.1.9.1       \\
    
    \hline
        
    1511 & शमीपर्णान्युप वपति घासमेवाभ्यामपि यच्छति & TB\_1.6.4.5       \\
    
    \hline
        
    1512 & शमीꣳ शान्त्यै हराम्यहम् यत्ते & TB\_1.2.1.7       \\
    
    \hline
        
    1513 & शरीरमभि सꣳ स्कृताः स्थ & TB\_1.2.1.8       \\
    
    \hline
        
    1514 & शला दोषणी कश्यपेवाꣳसा अच्छिद्रे & TB\_3.6.6.3       \\
    
    \hline
        
    1515 & शिप्रिन् वाजानां पते शचीवस्तव & TB\_2.4.4.8       \\
    
    \hline
        
    1516 & शिरस्तपस्याहितम् वैश्वानरस्य तेजसा ऋतेनास्य & TB\_1.5.5.5       \\
    
    \hline
        
    1517 & शिवे ते द्यावापृथिवी उभे & TB\_2.5.6.2       \\
    
    \hline
        
    1518 & शिवेयꣳ रज्जुरभिधानी अघ्निया{-}मुपसेवतां अप्रस्रꣳसाय & TB\_3.7.4.13       \\
    
    \hline
        
    1519 & शिश्ञै र्यदनृतं चकृमा वयं & TB\_3.7.12.3       \\
    
    \hline
        
    1520 & शुक्रा दीक्षायै तपसो विमोचनीः & TB\_3.7.14.2       \\
    
    \hline
        
    1521 & शुचिमिन्द्रं ॅवयोधसम् उष्णिहं छन्द & TB\_2.6.17.2       \\
    
    \hline
        
    1522 & शुनꣳ हुवेम मघवानमिन्द्रम् अस्मिन्भरे & TB\_2.4.4.3       \\
    
    \hline
        
    1523 & शूद्रा यदर्यजारा न पोषाय & TB\_3.9.7.3       \\
    
    \hline
        
    1524 & शोचिष्केशो घृतनिःणिक्{-}पावकः सुयज्ञो अग्निः & TB\_1.2.1.11       \\
    
    \hline
        
    1525 & शोणा धृष्णू नृवाहसेत्याह अहोरात्रे & TB\_3.9.4.3       \\
    
    \hline
        
    1526 & श्रमाय कौलालम् मायायै कार्मारम् & TB\_3.4.3.1       \\
    
    \hline
        
    1527 & श्रियमेवा{-}वरुन्धे शीते वाते पुनन्नि{-}वेत्याह & TB\_3.9.7.2       \\
    
    \hline
        
    1528 & श्रिया न मासरम् पयः & TB\_2.6.11.8       \\
    
    \hline
        
    1529 & श्रुधि श्रुत्कर्ण वह्निभिः देवैरग्ने & TB\_2.7.12.5       \\
    
    \hline
        
    1530 & श्रेष्ठं नो धेहि वार्यम् & TB\_3.6.7.2       \\
    
    \hline
        
    1531 & षट्थ् संपद्यन्ते षड्वा ऋतवः & TB\_2.1.4.6       \\
    
    \hline
        
    1532 & षट्पालाशाः सोमपीथस्या{-}वरुद्ध्यै एकविꣳशतिः संपद्यन्ते & TB\_3.8.20.2       \\
    
    \hline
        
    1533 & षट्पुरस्ता{-}दभिषेकस्य जुहोति षडुपरिष्टात् द्वादश & TB\_1.7.7.5       \\
    
    \hline
        
    1534 & षड्ढूतो ह वै नामैषः & TB\_2.3.11.3       \\
    
    \hline
        
    1535 & षड्वा ऋतवः ऋतुभिरेवैनं ॅयुनक्ति & TB\_1.7.9.2       \\
    
    \hline
        
    1536 & षड्वा ऋतवः ऋतूनेव प्रीणाति & TB\_1.3.10.4 TB\_1.6.8.8       \\
    
    \hline
        
    1537 & षड्विꣳशाः सप्तविꣳशेषु श्रयद्ध्वं सप्तविꣳशा & TB\_3.11.2.4       \\
    
    \hline
        
    1538 & षष्ठाः सप्तमेषु श्रयद्ध्वं सप्तमा & TB\_3.11.2.2       \\
    
    \hline
        
    1539 & षोडशाः सप्तदशेषु श्रयद्ध्वं सप्तदशा & TB\_3.11.2.3       \\
    
    \hline
        
    1540 & स इदं प्रति पप्रथे & TB\_2.4.1.10       \\
    
    \hline
        
    1541 & स इन्द्रमब्रवीत् इमान्म ईफ्सेति & TB\_2.7.14.2       \\
    
    \hline
        
    1542 & स ईं पाहि य & TB\_2.5.8.1       \\
    
    \hline
        
    1543 & स एतं प्रजापतिर्मारुतꣳ सप्तकपाल{-}मपश्यत् & TB\_1.6.2.3       \\
    
    \hline
        
    1544 & स एतं ॅलोकमजयत् यस्मिं & TB\_1.4.10.7       \\
    
    \hline
        
    1545 & स एतानपामार्गानजनयत् तानजुहोत् तैर्वै & TB\_1.7.1.8       \\
    
    \hline
        
    1546 & स एना विद्वान्. यक्ष्यसि & TB\_2.4.8.7       \\
    
    \hline
        
    1547 & स कस्मा अज्ञि यद्{-}यस्या & TB\_2.2.9.4       \\
    
    \hline
        
    1548 & स कस्मिन् प्रतिष्ठित इति & TB\_3.10.9.4       \\
    
    \hline
        
    1549 & स नो रास्व सहस्रिणः & TB\_2.4.8.3       \\
    
    \hline
        
    1550 & स पर्येण्यः स प्रियो & TB\_3.6.10.3       \\
    
    \hline
        
    1551 & स प्रत्नवन्नवीयसा अग्ने द्युम्नेन & TB\_2.4.8.1       \\
    
    \hline
        
    1552 & स बृहतीमेवा{-}स्पृशत् द्वाभ्यामक्षराभ्याम् अहोरात्राभ्यामेव & TB\_1.5.12.2       \\
    
    \hline
        
    1553 & स यदष्टाकपालान् प्रातस्सवने कुर्यात् & TB\_1.5.11.4       \\
    
    \hline
        
    1554 & स राष्ट्रं नाभवत् स & TB\_1.7.7.4       \\
    
    \hline
        
    1555 & स शार्दूलः यत्{-}कर्णयोः स & TB\_1.8.5.2       \\
    
    \hline
        
    1556 & स षड्ढोतुः सप्तहोतारं निरमिमीत & TB\_2.2.11.4       \\
    
    \hline
        
    1557 & स समुद्र उत्तरतः प्राज्वलद्{-}भूम्यन्तेन & TB\_1.5.10.1       \\
    
    \hline
        
    1558 & स समुद्रोऽभवत् तस्माथ्{-}समुद्रस्य न & TB\_2.2.9.3       \\
    
    \hline
        
    1559 & स स्वं ॅलोकं प्रति & TB\_3.10.11.2       \\
    
    \hline
        
    1560 & स ह हꣳसो हिरण्मयो & TB\_3.10.9.11       \\
    
    \hline
        
    1561 & स होवाच मा भैषी & TB\_3.10.9.13       \\
    
    \hline
        
    1562 & संज्ञानं ॅविज्ञानं प्रज्ञानं जानदभिजानत् & TB\_3.10.1.1       \\
    
    \hline
        
    1563 & संततिर्वा एते ग्रहाः यत् & TB\_1.2.3.1       \\
    
    \hline
        
    1564 & सक्षेदं पश्य विधर्तरिदं पश्य & TB\_3.7.7.1       \\
    
    \hline
        
    1565 & सख्यात्ते मा योषं सख्यान्मे & TB\_3.7.7.12       \\
    
    \hline
        
    1566 & सङ्कल्पजूतिं देवं ॅविपश्चिं मनो & TB\_3.12.3.4       \\
    
    \hline
        
    1567 & सङ्ख्याता देवमायया सर्वास्ताः यावन्त & TB\_3.12.6.2       \\
    
    \hline
        
    1568 & सजित्वानꣳ सदासहम् वर्.षिष्ठमूतये भर & TB\_3.5.7.4       \\
    
    \hline
        
    1569 & सज्ञांनं नः स्वैः सज्ञांनमरणैः & TB\_2.4.4.6       \\
    
    \hline
        
    1570 & सतोबृहतीषु स्तुवते सतो बृहन्न् & TB\_2.7.18.5       \\
    
    \hline
        
    1571 & सत्यं प्रपद्ये ऋतं प्रपद्ये & TB\_3.5.1.1       \\
    
    \hline
        
    1572 & सदेवमग्निं चिनुते रथसमिंत{-}श्चेतव्यः वज्रो & TB\_3.12.5.6       \\
    
    \hline
        
    1573 & सन्धये जारम् गेहायोपपतिम् निर्.ऋत्यै & TB\_3.4.4.1       \\
    
    \hline
        
    1574 & सपृंचः स्थ सं मा & TB\_1.3.3.6       \\
    
    \hline
        
    1575 & सप्त जुहोति सप्त हि & TB\_3.8.10.2       \\
    
    \hline
        
    1576 & सप्त ते अग्ने समिधः & TB\_3.11.5.1       \\
    
    \hline
        
    1577 & सप्त पुरस्तात् तिस्रो दक्षिणतः & TB\_3.11.9.5       \\
    
    \hline
        
    1578 & सप्त वै शीर्.षण्याः प्राणाः & TB\_1.2.3.3       \\
    
    \hline
        
    1579 & सप्तदशः प्रजापतिः प्रजापते{-}राप्त्यै अर्वाऽसि & TB\_1.3.6.4       \\
    
    \hline
        
    1580 & सप्तदशेन प्राजायत य एवं & TB\_2.2.4.7       \\
    
    \hline
        
    1581 & सप्तान्न{-}होमाञ्जुहोति सप्त वा अन्नानि & TB\_1.3.8.1       \\
    
    \hline
        
    1582 & सप्राण{-}मेवैन{-}माधत्ते स्वदितं तोकाय तनयाय & TB\_1.1.8.5       \\
    
    \hline
        
    1583 & समहं प्रजया सं मया & TB\_1.3.7.6       \\
    
    \hline
        
    1584 & समिथ् सप्तमी सप्तहोतारमेव तद्{-}यज्ञ्क्रतु{-}माप्नोति & TB\_2.3.7.4       \\
    
    \hline
        
    1585 & समिद्ध इन्द्र उषसामनीके पुरोरुचा & TB\_2.6.8.1       \\
    
    \hline
        
    1586 & समिद्धो अग्निः समिधा सुषमिद्धो & TB\_2.6.18.1       \\
    
    \hline
        
    1587 & समिद्धो अग्निरश्विना तप्तो घर्मो & TB\_2.6.12.1       \\
    
    \hline
        
    1588 & समिद्धो अद्य मनुषो दुरोणे & TB\_3.6.3.1       \\
    
    \hline
        
    1589 & समिधो अग्न आज्यस्य वियन्तु & TB\_3.5.5.1       \\
    
    \hline
        
    1590 & समुद्रमेवाप्नोति मद्ध्याय स्वाहेत्याह मद्ध्यमेवाप्नोति & TB\_3.8.16.4       \\
    
    \hline
        
    1591 & समुद्रोऽसि तेजसि श्रितः अपां & TB\_3.11.1.4       \\
    
    \hline
        
    1592 & सम्ॅवथ्सरमेव प्रीणाति पितृभ्यो बर्.हिषद्भ्यो & TB\_1.6.8.3       \\
    
    \hline
        
    1593 & सम्ॅवथ्सरोऽष्टादशः यदष्टादशिन आलभ्यन्ते सम्ॅवथ्सरमेव & TB\_3.9.1.2       \\
    
    \hline
        
    1594 & सम्ॅवथ्सरोऽसि नक्षत्रेषु श्रितः ऋतूनां & TB\_3.11.1.14       \\
    
    \hline
        
    1595 & सम्ॅवथ्सरोऽसि परिवथ्सरोऽसि इदावथ्सरोऽसी{-}दुवथ्सरोऽसि इद्वथ्सरोऽसि & TB\_3.10.4.1       \\
    
    \hline
        
    1596 & सरस्वत्या इन्द्राय सुत्राम्णे एष & TB\_2.6.1.4       \\
    
    \hline
        
    1597 & सरस्वत्यै मेषेणेन्द्रायाश्विभ्याम् इन्द्रायर्.षभेणाश्विभ्याꣳ सरस्वत्यै & TB\_2.6.15.2       \\
    
    \hline
        
    1598 & सरस्वत्यै स्वाहेत्याह सरस्वत्या एवैनं & TB\_3.8.6.4       \\
    
    \hline
        
    1599 & सरोभ्यो धैवरम् वेशन्ताभ्यो दाशम् & TB\_3.4.12.1       \\
    
    \hline
        
    1600 & सर्व ऐन्द्रा भवन्ति एकधैव & TB\_1.3.3.2       \\
    
    \hline
        
    1601 & सर्व{-}मायुरेति अभि स्वर्गं ॅलोकं & TB\_3.10.9.10       \\
    
    \hline
        
    1602 & सर्वं तेज स्सामरूप्यꣳ ह & TB\_3.12.9.2       \\
    
    \hline
        
    1603 & सर्वतो मां भूतं भविष्य{-}च्छ्रयतां & TB\_3.7.6.12       \\
    
    \hline
        
    1604 & सर्वमायुरेति सोमो वै यशः & TB\_2.2.8.8       \\
    
    \hline
        
    1605 & सर्वम्बत गौतमो वेद यः & TB\_3.10.9.12       \\
    
    \hline
        
    1606 & सर्वस्माद्{-}वित्ताद्{-}वेद्यात् अभिवर्तो ब्रह्मसामं भवति & TB\_1.4.6.3       \\
    
    \hline
        
    1607 & सर्वस्याप्त्यै सर्वस्या{-}वरुद्ध्यै यदि न & TB\_3.12.5.9       \\
    
    \hline
        
    1608 & सर्वा दिशो दिक्षु यच्चान्त & TB\_3.12.7.1       \\
    
    \hline
        
    1609 & सर्वा दिशोऽनु विवाति सर्वा & TB\_2.3.9.6       \\
    
    \hline
        
    1610 & सर्वा दिशोऽभिजयति प्रजापतिर्वै दशहोतृणां & TB\_2.3.5.6       \\
    
    \hline
        
    1611 & सर्वां दिवꣳ सर्वान् देवान् & TB\_3.12.8.1       \\
    
    \hline
        
    1612 & सर्वान् पथो अनृणा आक्षीयेम & TB\_3.7.9.9       \\
    
    \hline
        
    1613 & सर्वान्. वा एषोऽग्नौ कामान् & TB\_3.7.1.1       \\
    
    \hline
        
    1614 & सर्वाभिरेवैनं देवताभिरुद्धरति पराची वा & TB\_1.4.4.5       \\
    
    \hline
        
    1615 & सर्वास्ताः अहोरात्राणि सर्वाणि अद्र्धमासाꣳश्च & TB\_3.12.8.3       \\
    
    \hline
        
    1616 & सर्वास्ताः यावन्तो{-}ऽश्मानोऽस्यां पृथिव्यां प्रतिष्ठासु & TB\_3.12.6.3       \\
    
    \hline
        
    1617 & सर्वास्ताः सर्वान् मरीचीन. विततान् & TB\_3.12.7.3       \\
    
    \hline
        
    1618 & सर्वास्ताः सर्वꣳ हिरण्यꣳ रजतं & TB\_3.12.6.6       \\
    
    \hline
        
    1619 & सर्वे पुरुषाः सर्वाण्येवान्नान्यवरुन्धे सर्वान् & TB\_2.7.9.2       \\
    
    \hline
        
    1620 & सर्वेभ्यस्त्वा देवेभ्य इत्युपरिष्टात् सर्वे & TB\_3.8.7.3       \\
    
    \hline
        
    1621 & सर्वेषु वा एषु लोकेषु & TB\_3.9.15.1       \\
    
    \hline
        
    1622 & सर्वꣳ ह वै तत्र & TB\_3.9.19.2       \\
    
    \hline
        
    1623 & सविता यज्ञ्स्य प्रसूत्यै अथ & TB\_3.3.8.10       \\
    
    \hline
        
    1624 & सविता वा अकामयत श्रन्मे & TB\_3.1.4.11       \\
    
    \hline
        
    1625 & सहसे द्युम्नाय ऊर्जे पत्यायेत्याह & TB\_1.4.4.9       \\
    
    \hline
        
    1626 & सा चतुर्थ{-}मुदक्रामत् तत् प्रजापतिः & TB\_1.1.10.3       \\
    
    \hline
        
    1627 & सा नो जुषाणोप यज्ञ्{-}मागात् & TB\_3.12.3.2       \\
    
    \hline
        
    1628 & सा नो देवी सुहवा & TB\_2.4.2.8       \\
    
    \hline
        
    1629 & सा नो यज्ञ्स्य सुविते & TB\_3.1.1.2       \\
    
    \hline
        
    1630 & सा मे धुक्ष्व यजमानाय & TB\_3.7.6.6       \\
    
    \hline
        
    1631 & साग्रंहण्येष्ट्या यजते इमां जनतां & TB\_3.8.1.1       \\
    
    \hline
        
    1632 & साद्ध्याः पराञ्चम् य एवं & TB\_2.2.10.6       \\
    
    \hline
        
    1633 & साम्नोऽधि यजूꣳष्यसृजत यजुर्भ्योऽधि विष्णुम् & TB\_2.3.2.4       \\
    
    \hline
        
    1634 & साम्राज्याय सुक्रतुः देवस्य त्वा & TB\_2.6.5.2       \\
    
    \hline
        
    1635 & सावित्र{-}मष्टकपालं प्रातर्निर्वपति अष्टाक्षरा गायत्री & TB\_3.8.12.1       \\
    
    \hline
        
    1636 & सावित्रं जुहोति कर्मणः कर्मणः & TB\_1.3.5.1       \\
    
    \hline
        
    1637 & साऽऽप्ता भवति यातयाम्नी क्रूरीकृतेव & TB\_3.9.18.2       \\
    
    \hline
        
    1638 & सिताय स्वाहा{-}ऽसिताय स्वाहेति प्रमुक्ती{-}र्जुहोति & TB\_3.8.18.4       \\
    
    \hline
        
    1639 & सिꣳहे व्याघ्र उत या & TB\_2.7.7.1       \\
    
    \hline
        
    1640 & सीदन्ती देवी सुकृतस्य लोके & TB\_3.7.6.8       \\
    
    \hline
        
    1641 & सीसेन तन्त्रं मनसा मनीषिणः & TB\_2.6.4.1       \\
    
    \hline
        
    1642 & सुदुघा हि पयस्वतीः ऋषिभिः & TB\_1.4.8.5       \\
    
    \hline
        
    1643 & सुभूतेन मे संतिष्ठस्व ब्रह्मवर्चसेन & TB\_3.7.6.20       \\
    
    \hline
        
    1644 & सुरावन्तं बर्.हिषदं सुवीरम् यज्ञ्ं & TB\_2.6.3.1       \\
    
    \hline
        
    1645 & सुवर्गं तु लोकं नापराद्ध्नोति & TB\_3.9.2.3       \\
    
    \hline
        
    1646 & सुवर्गं ॅलोकꣳ षोडशिनः स्तोत्रेण & TB\_1.3.4.3       \\
    
    \hline
        
    1647 & सुवर्गमेव तेन लोकमभि जयन्ति & TB\_1.2.2.3       \\
    
    \hline
        
    1648 & सुवर्गमेवैनां ॅलोकं गमयति आऽहमजानि & TB\_3.9.6.4       \\
    
    \hline
        
    1649 & सुवर्गस्य लोकस्या{-}भिजित्यै प्र वा & TB\_1.2.3.4       \\
    
    \hline
        
    1650 & सुवर्गाय वा एष लोकायाधीयते & TB\_1.1.5.3       \\
    
    \hline
        
    1651 & सुवर्गो वै लोको बृहद्भाः & TB\_3.3.7.9       \\
    
    \hline
        
    1652 & सुवर्ण आत्मना भवति दुर्वर्णोऽस्य & TB\_2.2.4.6       \\
    
    \hline
        
    1653 & सुवर्वतीरप एना जयेम यो & TB\_2.4.7.5       \\
    
    \hline
        
    1654 & सुवीराः प्रजाः प्रजनयन् परीहि & TB\_1.1.1.2       \\
    
    \hline
        
    1655 & सुहस्तः सुवासाः शूषो नामास्य{-}मृतो & TB\_3.10.8.4       \\
    
    \hline
        
    1656 & सूर्यं चक्षुर्गमयतात् वातं प्राणमन्व{-}वसृजतात् & TB\_3.6.6.2       \\
    
    \hline
        
    1657 & सूर्याभिनिम्रुक्तः कुनखिनि कुनखी श्यावदति & TB\_3.2.8.12       \\
    
    \hline
        
    1658 & सूर्यो मा तस्मादेनसः विश्वान् & TB\_2.6.6.2       \\
    
    \hline
        
    1659 & सूर्यो वा अकामयत नक्षत्राणां & TB\_3.1.6.5       \\
    
    \hline
        
    1660 & सेदग्निरग्नीꣳ रत्येत्यन्यान् यत्र वाजी & TB\_2.5.3.3       \\
    
    \hline
        
    1661 & सैनं सश्चद्{-}देवं देवः सत्यमिन्दुं & TB\_2.5.8.10       \\
    
    \hline
        
    1662 & सैव ततः प्रायश्चित्तिः अर्द्धो & TB\_3.7.1.9       \\
    
    \hline
        
    1663 & सैव साऽग्नेः सन्ततिः तं & TB\_1.1.9.10       \\
    
    \hline
        
    1664 & सो अन्तरिक्षे रजसो विमानः & TB\_2.5.2.4       \\
    
    \hline
        
    1665 & सो अस्माꣳ अभयतमेन नेषत् & TB\_2.4.1.6       \\
    
    \hline
        
    1666 & सोमपीथाय संनयितुं वकल{-}मन्तरमाददे आपो & TB\_3.7.4.2       \\
    
    \hline
        
    1667 & सोममेवैनत् करोति यो वै & TB\_3.2.3.11       \\
    
    \hline
        
    1668 & सोमस्य त्विषिरसि तवेव मे & TB\_1.7.8.1       \\
    
    \hline
        
    1669 & सोमस्य ह्येतद्दात्रम् शुक्रा वः & TB\_1.7.6.3       \\
    
    \hline
        
    1670 & सोमेन राज्ञाऽष्टमे त्वष्ट्रा रूपेण & TB\_1.8.1.2       \\
    
    \hline
        
    1671 & सोमो राजा वरुणः देवा & TB\_1.7.8.3       \\
    
    \hline
        
    1672 & सोमो राजाऽमृतꣳ सुतः ऋजीषेणाजहान्मृत्युम् & TB\_2.6.2.1       \\
    
    \hline
        
    1673 & सोमो रेतोऽदधात् सविता प्राजनयत् & TB\_1.6.2.2       \\
    
    \hline
        
    1674 & सोमो वा अकामयत ओषधीनां & TB\_3.1.4.3       \\
    
    \hline
        
    1675 & सोम्यानाꣳ सोमपीथिनां निर्भक्तो ब्राह्मणः & TB\_3.7.5.10       \\
    
    \hline
        
    1676 & सोऽग्नि{-}रबिभेत् आहुतीभिर्वै माऽऽप्नोतीति स & TB\_2.1.2.5       \\
    
    \hline
        
    1677 & सोऽन्तरिक्ष{-}मसृजत चातुर्मास्यानि सामानि स & TB\_2.2.4.3       \\
    
    \hline
        
    1678 & सोऽब्रवीत् क इदं तुरीयमिति & TB\_1.7.1.3       \\
    
    \hline
        
    1679 & सोऽरण्यं परेत्य दर्भस्तम्ब{-}मुद्ग्रत्थ्य ब्राह्मणं & TB\_2.2.1.4       \\
    
    \hline
        
    1680 & सोऽश्वो वारो भूत्वा पराङैत् & TB\_1.1.8.3       \\
    
    \hline
        
    1681 & सोऽस्माथ् सृष्टोऽपाक्रामत् तं ग्रहेणागृह्णात् & TB\_2.2.2.5       \\
    
    \hline
        
    1682 & सौम्यः पर्णः सयोनित्वाय साक्षात् & TB\_3.2.3.4       \\
    
    \hline
        
    1683 & स्कन्नेमा विश्वा भुवना स्कन्नो & TB\_3.7.10.4       \\
    
    \hline
        
    1684 & स्योनमिन्द्र ते सदः ईशायै & TB\_2.6.14.6.       \\
    
    \hline
        
    1685 & स्रुग्घ्येषा प्राणो वै स्रुवः & TB\_3.3.1.5       \\
    
    \hline
        
    1686 & स्रुचौ सं प्रस्रावयति यदेव & TB\_3.3.9.7       \\
    
    \hline
        
    1687 & स्रुवस्य बुद्ध्नेनाभि निदद्ध्यात् मा & TB\_3.7.2.7       \\
    
    \hline
        
    1688 & स्रुवेणाघार{-}माघार्य तिस्रः पराची{-}राहुतीर्. हुत्वा & TB\_1.5.9.5       \\
    
    \hline
        
    1689 & स्व आयतने मनीषया इह & TB\_3.7.4.5       \\
    
    \hline
        
    1690 & स्व एवैनमृतावाधाय पशुमान् भवति & TB\_1.1.2.8       \\
    
    \hline
        
    1691 & स्वकृत इरिणे जुहोति प्रदरे & TB\_1.6.1.3 TB\_1.7.1.9       \\
    
    \hline
        
    1692 & स्वदन्तेऽस्मा ओषधयः ते वथ्समु{-}पावासृजन्न् & TB\_2.1.1.3       \\
    
    \hline
        
    1693 & स्वयैवैनं देवतयाऽभिषिञ्चति अग्ने{-}स्तेजसेत्याह तेज & TB\_1.7.8.4       \\
    
    \hline
        
    1694 & स्वाद्वीं त्वा स्वादुना तीव्रां & TB\_2.6.1.1       \\
    
    \hline
        
    1695 & स्वाहा राजसूयायेत्याह राजसूयाय ह्येना & TB\_1.7.6.4       \\
    
    \hline
        
    1696 & स्वाहाऽग्निꣳ होत्राज्जुषाणो अग्निर् भेषजम् & TB\_2.6.11.10       \\
    
    \hline
        
    1697 & स्वेनैवैनानि छन्दसा स्वया देवतया & TB\_3.3.2.2       \\
    
    \hline
        
    1698 & हरन्त्यस्मै विशो बलिम् ऐनमप्रतिख्यातं & TB\_2.7.18.3       \\
    
    \hline
        
    1699 & हरिवर्पसं गिरः आचर्.षणिप्रा वृषभो & TB\_2.4.3.11       \\
    
    \hline
        
    1700 & हविषोऽस्य नवस्य नः सुवर्विदो & TB\_2.4.8.5       \\
    
    \hline
        
    1701 & हव्यवाह{-}मभिमातिषाऽहम् रक्षोहणं पृतनासु जिष्णुम् & TB\_2.4.1.4       \\
    
    \hline
        
    1702 & हसाय पुꣳश्चलूमालभते वीणावादं गणकं & TB\_3.4.15.1       \\
    
    \hline
        
    1703 & हिरण्यं ददाति हिरण्य{-}ज्योतिरेव स्वर्गं & TB\_3.12.5.10       \\
    
    \hline
        
    1704 & हिरण्यं ॅवा अग्ने र्नाचिकेतस्यायतनं & TB\_3.11.7.3       \\
    
    \hline
        
    1705 & हिरण्यवाशीरिषिरः सुवर.षाः बृहस्पतिः स & TB\_2.5.5.5       \\
    
    \hline
        
    1706 & हिरण्येष्टको भवति यावदुत्तम{-}मङ्गुलि{-}काण्डं ॅयज्ञ् & TB\_3.12.5.12       \\
    
    \hline
        
    1707 & हुत्वोप सादयत्यजामित्वाय अथो व्यावृत्त्यै & TB\_2.1.4.3       \\
    
    \hline
        
    1708 & होता यक्षथ्{-}समिधाऽग्नि{-}मिडस्पदे अश्विनेन्द्रꣳ सरस्वतीम् & TB\_2.6.11.1       \\
    
    \hline
        
    1709 & होता यक्षथ्{-}समिधेन्द्र{-}मिडस्पदे नाभा पृथिव्या & TB\_2.6.7.1       \\
    
    \hline
        
    1710 & होता यक्षदग्निꣳ समिधा सुषमिधा & TB\_3.6.2.1       \\
    
    \hline
        
    1711 & होता यक्षदग्निꣳ स्वाहाऽऽज्यस्य स्तोकानाम् & TB\_2.6.11.9       \\
    
    \hline
        
    1712 & होता यक्षदिडस्पदे समिधानं महद्यशः & TB\_2.6.17.1       \\
    
    \hline
        
    1713 & होता यक्षदिडेडित आजुह्वानः सरस्वतीम् & TB\_2.6.11.3       \\
    
    \hline
        
    1714 & होता यक्षद्{-}दुर ऋष्वाः कवष्योऽकोषधावनी{-}रुदाताभिर्{-}जिहतां & TB\_3.6.2.2       \\
    
    \hline
        
    1715 & होतारं चित्ररथ{-}मद्ध्वरस्य यज्ञ्स्य यज्ञ्स्य & TB\_2.4.3.6       \\
    
    \hline
        
    1716 & ह्लीका हि पितरः ऒष्मणो & TB\_1.3.10.6       \\
    
    \hline
        \bottomrule
  \end{longtable}
  
\end{document}