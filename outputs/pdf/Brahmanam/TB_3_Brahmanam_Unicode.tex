\documentclass[17pt]{extarticle}
\usepackage{babel}
\usepackage{fontspec}
\usepackage{polyglossia}
\usepackage{extsizes}



\setmainlanguage{sanskrit}
\setotherlanguages{english} %% or other languages
\setlength{\parindent}{0pt}
\pagestyle{myheadings}
\newfontfamily\devanagarifont[Script=Devanagari]{AdishilaVedic}


\newcommand{\VAR}[1]{}
\newcommand{\BLOCK}[1]{}




\begin{document}
\begin{titlepage}
    \begin{center}
 
\begin{sanskrit}
    { \Large
    ॐ नमः परमात्मने, श्री महागणपतये नमः
श्री गुरुभ्यो नमः, ह॒रिः॒ ॐ 
    }
    \\
    \vspace{2.5cm}
    \mbox{ \Huge
    3.      कृष्ण यजुर्वेदीय तैत्तिरीय ब्राह्मणे तृतीयाष्टकं    }
\end{sanskrit}
\end{center}

\end{titlepage}
\tableofcontents
\pagebreak
ॐ नमः परमात्मने, श्री महागणपतये नमः
श्री गुरुभ्यो नमः, ह॒रिः॒ ॐ \newline
3.      कृष्ण यजुर्वेदीय तैत्तिरीय ब्राह्मणे तृतीयाष्टकं  \newline
     \addcontentsline{toc}{section}{ 3.1     प्रथमः प्रपाठकः-नक्षत्रेष्टीनां याज्याऽनुवाक्या ब्राह्मणानि}
     \markright{ 3.1     प्रथमः प्रपाठकः-नक्षत्रेष्टीनां याज्याऽनुवाक्या ब्राह्मणानि \hfill https://www.vedavms.in \hfill}
     \section*{ 3.1     प्रथमः प्रपाठकः-नक्षत्रेष्टीनां याज्याऽनुवाक्या ब्राह्मणानि }
                \textbf{ 3.1.1      अनुवाकं   1 - देवनक्षत्राणां याज्यानुवाक्याः} \newline
                                \textbf{ TB 3.1.1.1} \newline
                  अ॒ग्निर्नः॑ पातु॒ कृत्ति॑काः । नक्ष॑त्रं दे॒वमि॑न्द्रि॒यम् । इ॒दमा॑सां ॅविचक्ष॒णम् । ह॒विरा॒सं जु॑होतन ॥ यस्य॒ भान्ति॑ र॒श्मयो॒ यस्य॑ के॒तवः॑ । यस्ये॒मा विश्वा॒ भुव॑नानि॒ सर्वा᳚ । स कृत्ति॑काभिर॒भि स॒म्ॅवसा॑नः । अ॒ग्निर्नो॑ दे॒वः सु॑वि॒ते द॑धातु ॥ प्र॒जाप॑ते रोहि॒णी वे॑तु॒ पत्नी᳚ । वि॒श्वरू॑पा बृह॒ती चि॒त्रभा॑नुः \textbf{ 1} \newline
                  \newline
                                \textbf{ TB 3.1.1.2} \newline
                  सा नो॑ य॒ज्ञ्स्य॑ सुवि॒ते द॑धातु । यथा॒ जीवे॑म श॒रदः॒ सवी॑राः ॥ रो॒हि॒णी दे॒व्युद॑गात् पु॒रस्ता᳚त् । विश्वा॑ रू॒पाणि॑ प्रति॒मोद॑माना । प्र॒जाप॑तिꣳ ह॒विषा॑ व॒र्द्धय॑न्ती । प्रि॒या दे॒वाना॒-मुप॑यातु य॒ज्ञ्म् ॥ सोमो॒ राजा॑ मृगशी॒र॒.षेण॒ आगन्न्॑ । शि॒वं नक्ष॑त्रं प्रि॒यम॑स्य॒ धाम॑ । आ॒प्याय॑मानो बहु॒॒धा जने॑षु । रेतः॑ प्र॒जां ॅयज॑माने दधातु । \textbf{ 2} \newline
                  \newline
                                \textbf{ TB 3.1.1.3} \newline
                  यत् ते॒ नक्ष॑त्रं मृगशी॒॒र॒.षमस्ति॑ । प्रि॒यꣳ रा॑जन् प्रि॒यत॑मं प्रि॒याणा᳚म् । तस्मै॑ ते सोम ह॒विषा॑ विधेम । शन्न॑ एधि द्वि॒पदे॒ शं चतु॑ष्पदे ॥ आ॒र्द्रया॑ रु॒द्रः प्रथ॑मान एति । श्रेष्ठो॑ दे॒वानां॒ पति॑रघ्नि॒याना᳚म् । नक्ष॑त्रमस्य ह॒विषा॑ विधेम । मा नः॑ प्र॒जाꣳ री॑रिष॒न् मोत वी॒रान् ॥ हे॒ती रु॒द्रस्य॒ परि॑णो वृणक्तु । आ॒र्द्रा नक्ष॑त्रं जुषताꣳ ह॒विर्नः॑ \textbf{ 3} \newline
                  \newline
                                \textbf{ TB 3.1.1.4} \newline
                  प्र॒मु॒ञ्चमा॑नौ दुरि॒तानि॒ विश्वा᳚ । अपा॒घशꣳ॑ सं नुदता॒मरा॑तिम् ॥ पुन॑र्नो दे॒व्यदि॑तिः स्पृणोतु । पुन॑र्वसू नः॒ पुन॒रेतां᳚ ॅय॒ज्ञ्म् । पुन॑र्नो दे॒वा अ॒भिय॑न्तु॒ सर्वे᳚ । पुनः॑ पुनर्वो ह॒विषा॑ यजामः ॥ ए॒वा न दे॒व्यदि॑तिरन॒र्वा । विश्व॑स्य भ॒र्त्री जग॑तः प्रति॒ष्ठा । पुन॑र्वसू ह॒विषा॑ व॒र्द्धय॑न्ती । प्रि॒यं दे॒वाना॒-मप्ये॑तु॒ पाथः॑ । \textbf{ 4} \newline
                  \newline
                                \textbf{ TB 3.1.1.5} \newline
                  बृह॒स्पतिः॑ प्रथ॒मं जाय॑मानः । ति॒ष्यं॑ नक्ष॑त्रम॒भि संब॑भूव । श्रेष्ठो॑ दे॒वानां॒ पृत॑नासु जि॒ष्णुः । दिशोऽनु॒ सर्वा॒ अभ॑यन्नो अस्तु ॥ ति॒ष्यः॑ पु॒रस्ता॑दु॒त म॑द्ध्य॒तो नः॑ । बृह॒स्पति॑ र्नः॒ परि॑पातु प॒श्चात् । बाधे॑ता॒न् द्वेषो॒ अभ॑यं कृणुताम् । सु॒वीर्य॑स्य॒ पत॑यः स्याम ॥ इ॒दꣳ स॒र्पेभ्यो॑ ह॒विर॑स्तु॒ जुष्ट᳚म् । आ॒श्रे॒षा येषा॑-मनु॒यन्ति॒ चेतः॑ \textbf{ 5} \newline
                  \newline
                                \textbf{ TB 3.1.1.6} \newline
                  ये अ॒न्तरि॑क्षं पृथि॒वीं क्षि॒यन्ति॑ । तेनः॑ स॒र्पासो॒ हव॒माग॑मिष्ठाः ॥ ये रो॑च॒ने सूर्य॒स्यापि॑ स॒र्पाः । ये दिवं॑ दे॒वीमनु॑ स॒ञ्चर॑न्ति । येषा॑माश्रे॒षा अ॑नु॒यन्ति॒ काम᳚म् । तेभ्यः॑ स॒र्पेभ्यो॒ मधु॑मज्जुहोमि ॥ उप॑हूताः पि॒तरो॒ ये म॒घासु॑ । मनो॑जवसः सु॒कृतः॑ सुकृ॒त्याः । ते नो॒ नक्ष॑त्रे॒ हव॒माग॑मिष्ठाः । स्व॒धाभि॑ र्य॒ज्ञ्ं प्रय॑तं जुषन्ताम् । \textbf{ 6} \newline
                  \newline
                                \textbf{ TB 3.1.1.7} \newline
                  ये अ॑ग्निद॒ग्धा येऽन॑ग्नि-दग्धाः । ये॑ऽमुं ॅलो॒कं पि॒तरः॑ क्षि॒यन्ति॑ । याꣳश्च॑ वि॒द्मयाꣳ उ॑ च॒ न प्र॑वि॒द्म । म॒घासु॑ य॒ज्ञ्ꣳ सुकृ॑तं जुषन्ताम् ॥ गवां॒ पतिः॒ फल्गु॑नीनामसि॒ त्वम् । तद॑र्यमन् वरुण मित्र॒ चारु॑ । तं त्वा॑ व॒यꣳ स॑नि॒तारꣳ॑ सनी॒नाम् । जी॒वा जीव॑न्त॒मुप॒ सम्ॅवि॑शेम ॥ येने॒मा विश्वा॒ भुव॑नानि॒ सञ्जि॑ता । यस्य॑ दे॒वा अ॑नु स॒म्ॅयन्ति॒ चेतः॑ \textbf{ 7} \newline
                  \newline
                                \textbf{ TB 3.1.1.8} \newline
                  अ॒र्य॒मा राजा॒ऽजर॒स्तुवि॑ष्मान् । फल्गु॑नीना-मृष॒भो रो॑रवीति ॥ श्रेष्ठो॑ दे॒वानां᳚ भगवो भगासि । तत्त्वा॑ विदुः॒ फल्गु॑नी॒स्तस्य॑ वित्तात् । अ॒स्मभ्यं॑ क्ष॒त्रम॒जरꣳ॑ सु॒वीर्य᳚म् । गोम॒दश्व॑-व॒दुप॒-सन्नु॑दे॒ह ॥ भगो॑ ह दा॒ता भग॒ इत् प्र॑दा॒ता । भगो॑ दे॒वीः फल्गु॑नी॒रावि॑वेश । भग॒स्येत्तं प्र॑स॒वं ग॑मेम । यत्र॑ दे॒वैः स॑ध॒मादं॑ मदेम । \textbf{ 8} \newline
                  \newline
                                \textbf{ TB 3.1.1.9} \newline
                  आया॑तु दे॒वः स॑वि॒तोप॑यातु । हि॒र॒ण्यये॑न सु॒वृता॒ रथे॑न । वह॒न्॒. हस्तꣳ॑ सु॒भगं॑ विद्म॒नाप॑सम् । प्र॒यच्छ॑न्तं॒ पपु॑रिं॒ पुण्य॒मच्छ॑ ॥ हस्तः॒ प्रय॑च्छ त्व॒मृतं॒ ॅवसी॑यः । दक्षि॑णेन॒ प्रति॑गृभ्णीम एनत् । दा॒तार॑म॒द्य स॑वि॒ता वि॑देय ।यो नो॒ हस्ता॑य प्रसु॒वाति॑ य॒ज्ञ्म् ॥ त्वष्टा॒ नक्ष॑त्रम॒भ्ये॑ति चि॒त्राम् । सु॒भꣳ स॑सं ॅयुव॒तिꣳ रोच॑मानाम् \textbf{ 9} \newline
                  \newline
                                \textbf{ TB 3.1.1.10} \newline
                  नि॒वे॒शय॑-न्न॒मृता॒न् मर्त्याꣳ॑श्च । रू॒पाणि॑ पिꣳ॒॒शन् भुव॑नानि॒ विश्वा᳚ ॥ तन्न॒स्त्वष्टा॒ तदु॑ चि॒त्रा विच॑ष्टाम् । तन्नक्ष॑त्रं भूरि॒दा अ॑स्तु॒ मह्य᳚म् । तन्नः॑ प्र॒जां ॅवी॒रव॑तीꣳ सनोतु । गोभि॑र्नो॒ अश्वैः॒ सम॑नक्तु य॒ज्ञ्म् ॥ वा॒युर्नक्ष॑त्र-म॒भ्ये॑ति॒ निष्ट्या᳚म् । ति॒ग्मशृ॑ङ्गो वृष॒भो रोरु॑वाणः । स॒मी॒रय॒न् भुव॑ना मात॒रिश्वा᳚ । अप॒ द्वेषाꣳ॑सि नुदता॒मरा॑तीः । \textbf{ 10} \newline
                  \newline
                                \textbf{ TB 3.1.1.11} \newline
                  तन्नो॑ वा॒यस्तदु॒ निष्ट्या॑ शृणोतु । तन्नक्ष॑त्रं भूरि॒दा अ॑स्तु॒ मह्य᳚म् । तन्नो॑ दे॒वासो॒ अनु॑ जानन्तु॒ काम᳚म् । यथा॒ तरे॑म दुरि॒तानि॒ विश्वा᳚ ॥ दू॒रम॒स्मच्छत्र॑वो यन्तु भी॒ताः । तदि॑न्द्रा॒ग्नी कृ॑णुतां॒ तद् विशा॑खे । तन्नो॑ दे॒वा अनु॑मदन्तु य॒ज्ञ्म् । प॒श्चात् पु॒रस्ता॒-दभ॑यन्नो अस्तु ॥ नक्ष॑त्राणा॒-मधि॑पत्नी॒ विशा॑खे । श्रेष्ठा॑विन्द्रा॒ग्नी भुव॑नस्य गो॒पौ \textbf{ 11} \newline
                  \newline
                                \textbf{ TB 3.1.1.12} \newline
                  विषू॑चः॒ शत्रू॑-नप॒बाध॑मानौ । अप॒ क्षुधं॑ नुदता॒मरा॑तिम् ॥ पू॒र्णा प॒श्चादु॒त पू॒र्णा पु॒रस्ता᳚त् । उन्म॑द्ध्य॒तः पौ᳚र्णमा॒सी जि॑गाय । तस्यां᳚ दे॒वा अधि॑ स॒म्ॅवस॑न्तः । उ॒त्त॒मे नाक॑ इ॒ह मा॑दयन्ताम् ॥ पृ॒थ्वी सु॒वर्चा॑ युव॒तिः स॒जोषाः᳚ । पौ॒र्ण॒मा॒स्युद॑गा॒-च्छोभ॑माना । आ॒प्या॒यय॑न्ती दुरि॒तानि॒ विश्वा᳚ । उ॒रुं दुहां॒ ॅयज॑मानाय य॒ज्ञ्म् ( ) । \textbf{ 12} \newline
                  \newline
                                    (चि॒त्रभा॑नु॒र्- यज॑माने दधातु - ह॒विर्नः॒ - पाथ॒ - श्चेतो॑ - जुषन्तां॒ - चेतो॑ - मदेम॒ - रोच॑माना॒ - मरा॑तीर् - गो॒पौ - य॒ज्ञ्ं) \textbf{(A1)} \newline \newline
                \textbf{ 3.1.2     अनुवाकं   2 - यमनक्षत्राणां याज्यानुवाक्याः} \newline
                                \textbf{ TB 3.1.2.1} \newline
                  ऋ॒द्ध्यास्म॑ ह॒व्यैर्-नम॑सोप॒ सद्य॑ । मि॒त्रं दे॒वं मि॑त्र॒धेय॑न्नो अस्तु । अ॒नू॒रा॒धान्. ह॒विषा॑ व॒र्द्धय॑न्तः । श॒तं जी॑वेम श॒रदः॒ सवी॑राः ॥ चि॒त्रं नक्ष॑त्र॒-मुद॑गात् पु॒रस्ता᳚त् । अ॒नू॒रा॒धास॒ इति॒ यद् वद॑न्ति । तन्मि॒त्र ए॑ति प॒थिभि॑ र्देव॒यानैः᳚ । हि॒र॒ण्ययै॒ र्वित॑तै-र॒न्तरि॑क्षे ॥ इन्द्रो᳚ ज्ये॒ष्ठामनु॒ नक्ष॑त्रमेति । यस्मि॑न् वृ॒त्रं ॅवृ॑त्र॒ तूर्ये॑ त॒तार॑ \textbf{ 13} \newline
                  \newline
                                \textbf{ TB 3.1.2.2} \newline
                  तस्मि॑न् व॒यम॒मृतं॒ दुहा॑नाः । क्षुधं॑ तरेम॒ दुरि॑तिं॒ दुरि॑ष्टिम् ॥ पु॒र॒न्द॒राय॑ वृष॒भाय॑ धृ॒ष्णवे᳚ । अषा॑ढाय॒ सह॑मानाय मी॒ढुषे᳚ । इन्द्रा॑य ज्ये॒ष्ठा मधु॑म॒द्-दुहा॑ना । उ॒रुं कृ॑णोतु॒ यज॑मानाय लो॒कम् ॥ मूलं॑ प्र॒जां ॅवी॒रव॑तीं ॅविदेय । परा᳚च्येतु॒ निर्.ऋ॑तिः परा॒चा । गोभि॒ र्नक्ष॑त्रं प॒शुभिः॒ सम॑क्तम् । अह॑र्भूया॒द्-यज॑मानाय॒ मह्य᳚म् । \textbf{ 14} \newline
                  \newline
                                \textbf{ TB 3.1.2.3} \newline
                  अह॑र्नो अ॒द्य सु॑वि॒ते द॑धातु । मूलं॒ नक्ष॑त्र॒मिति॒ यद् वद॑न्ति । परा॑चीं ॅवा॒चा निर्.ऋ॑तिं नुदामि । शि॒वं प्र॒जायै॑ शि॒वम॑स्तु॒ मह्य᳚म् ॥ या दि॒व्या आपः॒ पय॑सा सं बभू॒वुः । या अ॒न्तरि॑क्ष उ॒त पार्थि॑वी॒र्याः । यासा॑मषा॒ढा अ॑नु॒यन्ति॒ काम᳚म् । ता न॒ आपः॒ शꣳ स्यो॒ना भ॑वन्तु ॥ याश्च॒ कूप्या॒ याश्च॑ ना॒द्याः᳚ समु॒द्रियाः᳚ । याश्च॑ वैश॒न्तीरु॒त प्रा॑स॒चीर्याः \textbf{ 15} \newline
                  \newline
                                \textbf{ TB 3.1.2.4} \newline
                  यासा॑मषा॒ढा मधु॑ भ॒क्षय॑न्ति । ता न॒ आपः॒ शꣳ स्यो॒ना भ॑वन्तु ॥ तन्नो॒ विश्वे॒ उप॑ शृण्वन्तु दे॒वाः । तद॑षा॒ढा अ॒भि सम्ॅय॑न्तु य॒ज्ञ्म् । तन्नक्ष॑त्रं प्रथतां प॒शुभ्यः॑ । कृ॒षिर् वृ॒ष्टिर् यज॑मानाय कल्पताम् ॥ शु॒भ्राः क॒न्या॑ युव॒तयः॑ सु॒पेश॑सः । क॒र्म॒कृतः॑ सु॒कृतो॑ वी॒र्या॑वतीः । विश्वा᳚न् दे॒वान्. ह॒विषा॑ व॒र्द्धय॑न्तीः । अ॒षा॒ढाः काम॒मुप॑यान्तु य॒ज्ञ्म् । \textbf{ 16} \newline
                  \newline
                                \textbf{ TB 3.1.2.5} \newline
                  यस्मि॒न् ब्रह्मा॒ऽभ्यज॑य॒थ् सर्व॑मे॒तत् । अ॒मुञ्च॑ लो॒क मि॒दमू॑च॒ सर्व᳚म् । तन्नो॒ नक्ष॑त्रमभि॒जिद् वि॒जित्य॑ । श्रियं॑ दधा॒-त्वहृ॑णीयमानम् ॥ उ॒भौ लो॒कौ ब्रह्म॑णा॒ सञ्जि॑ते॒मौ । तन्नो॒ नक्ष॑त्रमभि॒जिद् विच॑ष्टाम् । तस्मि॑न् व॒यं पृत॑नाः॒ सं ज॑येम । तन्नो॑ दे॒वासो॒ अनु॑जानन्तु॒ काम᳚म् ॥ शृ॒ण्वन्ति॑ श्रो॒णा-म॒मृत॑स्य गो॒पाम् । पुण्या॑मस्या॒ उप॑शृणोमि॒ वाच᳚म् \textbf{ 17} \newline
                  \newline
                                \textbf{ TB 3.1.2.6} \newline
                  म॒हीं दे॒वीं ॅविष्णु॑पत्नी-मजू॒र्याम् । प्र॒तीची॑मेनाꣳ ह॒विषा॑ यजामः ॥ त्रे॒धा विष्णु॑-रुरुगा॒यो विच॑क्रमे । म॒हीं दिवं॑ पृथि॒वी-म॒न्तरि॑क्षम् । तच्छ्रो॒णैति॒ श्रव॑ इ॒च्छमा॑ना । पुण्यꣳ॒॒ श्लोकं॒ ॅयज॑मानाय कृण्व॒ती ॥ अ॒ष्टौ दे॒वा वस॑वः सो॒म्यासः॑ । चत॑स्रो दे॒वी र॒जराः॒ श्रवि॑ष्ठाः । ते य॒ज्ञ्ं पा᳚न्तु॒ रज॑सः प॒रस्ता᳚त् । स॒म्ॅव॒थ्स॒रीण॑-म॒मृतꣳ॑ स्व॒स्ति । \textbf{ 18} \newline
                  \newline
                                \textbf{ TB 3.1.2.7} \newline
                  य॒ज्ञ्ं नः॑ पान्तु॒ वस॑वः पु॒रस्ता᳚त् । द॒क्षि॒ण॒तो॑-ऽभिय॑न्तु॒ श्रवि॑ष्ठाः । पुण्यं॒ नक्ष॑त्रम॒भि सम्ॅवि॑शाम । मा नो॒ अरा॑ति-र॒घशꣳ॒॒सा गन्न्॑ ॥ क्ष॒त्रस्य॒ राजा॒ वरु॑णोऽधिरा॒जः । नक्ष॑त्राणाꣳ श॒तभि॑ष॒ग् वसि॑ष्ठः । तौ दे॒वेभ्यः॑ कृणुतो दी॒र्घमायुः॑ । श॒तꣳ स॒हस्रा॑ भेष॒जानि॑ धत्तः ॥ य॒ज्ञ्न्नो॒ राजा॒ वरु॑ण॒ उप॑यातु । तन्नो॒ विश्वे॑ अ॒भि सम्ॅय॑न्तु दे॒वाः । 19(10)ट्.भ्.3.1.2.8तन्नो॒ नक्ष॑त्रꣳ श॒तभि॑षग् जुषा॒णम् । दी॒र्घमायुः॒ प्रति॑रद् भेष॒जानि॑ ॥ अ॒ज एक॑पा॒दुद॑गात् पु॒रस्ता᳚त् । विश्वा॑ भू॒तानि॑ प्रति॒मोद॑मानः । तस्य॑ दे॒वाः प्र॑स॒वं ॅय॑न्ति॒ सर्वे᳚ । प्रो॒ष्ठ॒प॒दासो॑ अ॒मृत॑स्य गो॒पाः ॥  वि॒भ्राज॑मानः समिधा॒न उ॒ग्रः । आऽन्तरि॑क्ष-मरुह॒द-ग॒न्द्याम् । तꣳ सूर्यं॑ दे॒व-म॒जमेक॑पादम् । प्रो॒ष्ठ॒प॒दासो॒ अनु॑यन्ति॒ सर्वे᳚ । \textbf{ 20} \newline
                  \newline
                                \textbf{ TB } \newline
                   \textbf{ 0} \newline
                  \newline
                                \textbf{ TB 3.1.2.9} \newline
                  अहि॑र् बु॒द्ध्नियः॒ प्रथ॑मान एति । श्रेष्ठो॑ दे॒वाना॑मु॒त मानु॑षाणाम् । तं ब्रा᳚ह्म॒णाः सो॑म॒पाः सो॒म्यासः॑ । प्रो॒ष्ठ॒प॒दासो॑ अ॒भि र॑क्षन्ति॒ सर्वे᳚ ॥ च॒त्वार॒ एक॑म॒भि कर्म॑ दे॒वाः । प्रो॒ष्ठ॒प॒दास॒ इति॒ यान्. वद॑न्ति । ते बु॒द्धनियं॑ परि॒षद्यꣳ॑ स्तु॒वन्तः॑ । अहिꣳ॑ रक्षन्ति॒ नम॑सोप॒सद्य॑ ॥ पू॒षा रे॒वत्यन्वे॑ति॒ पन्था᳚म् । पु॒ष्टि॒पती॑ पशु॒पा वाज॑बस्त्यौ \textbf{ 21} \newline
                  \newline
                                \textbf{ TB 3.1.2.10} \newline
                  इ॒मानि॑ ह॒व्या प्रय॑ता जुषा॒णा । सु॒गैर्नो॒ यानै॒रुप॑यातां ॅय॒ज्ञ्म् ॥ क्षु॒द्रान् प॒शून् र॑क्षतु रे॒वती॑ नः । गावो॑ नो॒ अश्वाꣳ॒॒ अन्वे॑तु पू॒षा । अन्नꣳ॒॒ रक्ष॑न्तौ बहु॒॒धा विरू॑पम् । वाजꣳ॑ सनुतां॒ ॅयज॑मानाय य॒ज्ञ्म् ॥ तद॒श्विना॑-वश्व॒युजोप॑याताम् । शुभं॒ गमि॑ष्ठौ सु॒यमे॑भि॒रश्वैः᳚ । स्वं नक्ष॑त्रꣳ ह॒विषा॒ यज॑न्तौ । मद्ध्वा॒ संपृ॑क्तौ॒ यजु॑षा॒ सम॑क्तौ । \textbf{ 22} \newline
                  \newline
                                \textbf{ TB 3.1.2.11} \newline
                  यौ दे॒वानां᳚ भि॒षजौ॑ हव्यवा॒हौ । विश्व॑स्य दू॒ता-व॒मृत॑स्य गो॒पौ । तौ नक्ष॑त्रं जुजुषा॒णो-प॑याताम् । नमो॒ऽश्विभ्यां᳚ कृणुमो-ऽश्व॒युग्भ्या᳚म् ॥ अप॑ पा॒प्मानं॒ भर॑णीर्भरन्तु । तद् य॒मो राजा॒ भग॑वा॒न्॒. विच॑ष्टाम् । लो॒कस्य॒ राजा॑ मह॒तो म॒हान्. हि । सु॒गं नः॒ पन्था॒मभ॑यं कृणोतु ॥ यस्मि॒न् नक्ष॑त्रे य॒म एति॒ राजा᳚ । यस्मि॑-न्नेनम॒भ्यषि॑चंन्त दे॒वाः ( ) । तद॑स्य चि॒त्रꣳ ह॒विषा॑ यजाम । अप॑ पा॒प्मानं॒ भर॑णीर् भरन्तु ॥ "नि॒वेश॑नी॒ {1}" "यत्ते॑ दे॒वा अद॑धुः {2}" । \textbf{ 23} \newline
                  \newline
                                    (त॒तार॒ - मह्यं॑ - प्रास॒चीर्या - या᳚न्तु य॒ज्ञ्ं - ॅवाचꣳ॑ - स्व॒स्ति - दे॒वा-अनु॑ यन्ति॒ सर्वे॒-वाज॑बस्त्यौ॒-सम॑क्तौ-दे॒वा स्त्रीणि॑ च) \textbf{(A2)} \newline \newline
                \textbf{ 3.1.3     अनुवाकं   3 - चान्द्रमसादीनां सप्तनामिष्टीनां याज्यानुवाक्याः} \newline
                                \textbf{ TB 3.1.3.1} \newline
                  "नवो॑ नवो भवति॒ जाय॑मानो॒ {3]" "यमा॑दि॒त्या अꣳ॒॒शुमा᳚प्या॒यय॑न्ति {4}"॥ ये विरू॑पे॒ सम॑नसा स॒म्ॅव्यय॑न्ती । स॒मा॒नं तन्तुं॑ परि तात॒नाते᳚ । वि॒भू प्र॒भू अ॑नु॒भू वि॒श्वतो॑ हुवे । ते नो॒ नक्ष॑त्रे॒ हव॒माग॑मेतम् ॥ व॒यं दे॒वी ब्रह्म॑णा सम्ॅविदा॒नाः । सु॒रत्ना॑सो दे॒ववी॑तिं॒ दधा॑नाः । अ॒हो॒रा॒त्रे ह॒विषा॑ व॒र्द्धय॑न्तः । अति॑ पा॒प्मान॒-मति॑मुक्त्या गमेम ॥ प्रत्यु॑वदृश्याय॒ती \textbf{ 24} \newline
                  \newline
                                \textbf{ TB 3.1.3.2} \newline
                  व्यु॒च्छन्ती॑ दुहि॒ता दि॒वः । अ॒पो म॒ही वृ॑णुते॒ चक्षु॑षा ॥ तमो॒ ज्योति॑ष्कृणोति सू॒नरी᳚ । उदु॒स्त्रियाः᳚ सचते॒ सूर्यः॑ । स चा॑ उ॒द्यन्नक्ष॑त्र-मर्चि॒मत् । तवेदु॑षो॒ व्युषि॒ सूर्य॑स्य च ॥ सं भ॒क्तेन॑ गमेमहि । तन्नो॒ नक्ष॑त्र-मर्चि॒मत् । भा॒नु॒मत्तेज॑ उ॒च्चर॑त् । उप॑ य॒ज्ञ्-मि॒हाग॑मत् । \textbf{ 25} \newline
                  \newline
                                \textbf{ TB 3.1.3.3} \newline
                  प्र नक्ष॑त्राय दे॒वाय॑ । इन्द्रा॒येन्दुꣳ॑ हवामहे । स नः॑ सवि॒ता सु॑वथ् स॒निम् । पु॒ष्टि॒दां ॅवी॒रव॑त्तमम् ॥ "उदु॒त्यं {5}" "चि॒त्रम् {6}"॥ "अदि॑तिर्न उरुष्यतु {7}" "म॒हीमू॒षु मा॒तर᳚म् {8} ॥ "इ॒दं ॅविष्णुः॒{9}" "प्रतद्विष्णुः॑ {10]" ॥ "अ॒ग्निर्मू॒र्द्धा {11}" "भुवः॑ {12}" ॥ "अनु॑ नो॒ऽद्यानु॑मति॒ {13}" "रन्विद॑नुमते॒ त्वम् {14}" ॥ "ह॒व्य॒वाहꣳ॒॒ {15}" "स्वि॑ष्टम् {16}" ( ) । \textbf{ 26} \newline
                  \newline
                                    (आ॒य॒त्य॑ - गम॒थ् - स्वि॑ष्टम्) \textbf{(A3)} \newline \newline
                \textbf{ 3.1.4     अनुवाकं   4 - देव नक्षत्रेष्टिब्राह्मणम्} \newline
                                \textbf{ TB 3.1.4.1} \newline
                  अ॒ग्निर्वा अ॑कामयत । अ॒न्ना॒दो दे॒वानाꣳ॑ स्या॒मिति॑ । स ए॒तम॒ग्नये॒ कृत्ति॑काभ्यः पुरो॒डाश॑-म॒ष्टाक॑पालं॒ निर॑वपत् । ततो॒ वै सो᳚ऽन्ना॒दो दे॒वाना॑मभवत् । अ॒ग्निर्वै दे॒वाना॑मन्ना॒दः ॥ यथा॑ ह॒ वा अ॒ग्निर् दे॒वना॑-मन्ना॒दः । ए॒वꣳ ह॒ वा ए॒ष म॑नु॒ष्या॑णां भवति । य ए॒तेन॑ ह॒विषा॒ यज॑ते । य उ॑ चैनदे॒वं ॅवेद॑ ॥ सोऽत्र॑ जुहोति । अ॒ग्नये॒ स्वाहा॒ कृत्ति॑काभ्यः॒ स्वाहा᳚ । अ॒म्बायै॒ स्वाहा॑ दु॒लायै॒ स्वाहा᳚ । नि॒त॒त्न्यै स्वाहा॒ ऽभ्रय॑न्त्यै॒ स्वाहा᳚ । मे॒घय॑न्त्यै॒ स्वाहा॑ व॒र॒.षय॑न्त्यै॒ स्वाहा᳚ । चु॒पु॒णीका॑यै॒ स्वाहेति॑ । \textbf{ 27} \newline
                  \newline
                                \textbf{ TB 3.1.4.2} \newline
                  प्र॒जाप॑तिः प्र॒जा अ॑सृजत । ता अ॑स्माथ् सृ॒ष्टाः परा॑चीरायन्न् । तासाꣳ॑ रोहि॒णी-म॒भ्य॑द्ध्यायत् । सो॑ऽकामयत । उप॒ माऽऽव॑र्तेत । समे॑नया गच्छे॒येति॑ । स ए॒तं प्र॒जाप॑तये रोहि॒ण्यै च॒रुं निर॑वपत् । ततो॒ वै सा तमु॒पाव॑र्तत । समे॑नया ऽगच्छत ॥ उप॑ ह॒ वा ए॑नं प्रि॒यमाव॑र्तते । सं प्रि॒येण॑ गच्छते । य ए॒तेन॑ ह॒विषा॒ यज॑ते । य उ॑ चैनदे॒वं ॅवेद॑ ॥ सोऽत्र॑ जुहोति । प्र॒जाप॑तये॒ स्वाहा॑ रोहि॒ण्यै स्वाहा᳚ । रोच॑मानायै॒ स्वाहा᳚ प्र॒जाभ्यः॒ स्वाहेति॑ । \textbf{ 28} \newline
                  \newline
                                \textbf{ TB 3.1.4.3} \newline
                  सोमो॒ वा अ॑कामयत । ओष॑धीनाꣳ रा॒ज्य-म॒भिज॑येय॒मिति॑ । स ए॒तꣳ सोमा॑य मृगशी॒र॒.षाय॑ श्यामा॒कं च॒रुं पय॑सि॒ निर॑वपत् । ततो॒ वै स ओष॑धीनाꣳ रा॒ज्यम॒भ्य॑जयत् । स॒मा॒नानाꣳ॑ ह॒ वै रा॒ज्यम॒भिज॑यति । य ए॒तेन॑ ह॒विषा॒ यज॑ते । य उ॑ चैनदे॒वं ॅवेद॑ । सोऽत्र॑ जुहोति । सोमा॑य॒ स्वाहा॑ मृगशी॒र॒.षाय॒ स्वाहा᳚ । इ॒न्व॒काभ्यः॒ स्वाहौष॑धीभ्यः॒ स्वाहा᳚ । रा॒ज्याय॒ स्वाहा॒ ऽभिजि॑त्यै॒ स्वाहेति॑ । \textbf{ 29} \newline
                  \newline
                                \textbf{ TB 3.1.4.4} \newline
                  रु॒द्रो वा अ॑कामयत । प॒शु॒मान्थ् स्या॒मिति॑ । स ए॒तꣳ रु॒द्राया॒र्द्रायै॒ प्रैय॑ङ्गवं च॒रुं पय॑सि॒ निर॑वपत् । ततो॒ वै स प॑शु॒मान॑भवत् । प॒शु॒मान्. ह॒ वै भ॑वति । य ए॒तेन॑ ह॒विषा॒ यज॑ते । य उ॑ चैनदे॒वं ॅवेद॑ । सोऽत्र॑ जुहोति । रु॒द्राय॒ स्वाहा॒ ऽऽर्द्रायै॒ स्वाहा᳚ । पिन्व॑मानायै॒ स्वाहा॑ प॒शुभ्यः॒ स्वाहेति॑ । \textbf{ 30} \newline
                  \newline
                                \textbf{ TB 3.1.4.5} \newline
                  ऋ॒क्षा वा इ॒यम॑लो॒मका॑ऽऽसीत् । साऽका॑मयत । ओष॑धीभि॒र्-वन॒स्पति॑भिः॒ प्रजा॑ये॒येति॑ । सैतमदि॑त्यै॒ पुन॑र्वसुभ्यां च॒रुं निर॑वपत् । ततो॒ वा इ॒यमोष॑धीभि॒र्-वन॒स्पति॑भिः॒ प्राजा॑यत । प्रजा॑यते ह॒ वै प्र॒जया॑ प॒शुभिः॑ । य ए॒तेन॑ ह॒विषा॒ यज॑ते । य उ॑ चैनदे॒वं ॅवेद॑ । सोऽत्र॑ जुहोति । अदि॑त्यै॒ स्वाहा॒ पुन॑र्वसुभ्याम् । स्वाहाऽऽभू᳚त्यै॒ स्वाहा॒ प्रजा᳚त्यै॒ स्वाहेति॑ । \textbf{ 31} \newline
                  \newline
                                \textbf{ TB 3.1.4.6} \newline
                  बृह॒स्पति॒र्वा अ॑कामयत । ब्र॒ह्म॒व॒र्च॒सी स्या॒मिति॑ । स ए॒तं बृह॒स्पत॑ये ति॒ष्या॑य नैवा॒रं च॒रुं पय॑सि॒ निर॑वपत् । ततो॒ वै स ब्र॑ह्मवर्च॒स्य॑भवत् । ब्र॒ह्म॒व॒र्च॒सी ह॒ वै भ॑वति । य ए॒तेन॑ ह॒विषा॒ यज॑ते । य उ॑ चैनदे॒वं ॅवेद॑ । सोऽत्र॑ जुहोति । बृह॒स्पत॑ये॒ स्वाहा॑ ति॒ष्या॑य॒ स्वाहा᳚ । ब्र॒ह्म॒व॒र्च॒साय॒ स्वाहेति॑ । \textbf{ 32} \newline
                  \newline
                                \textbf{ TB 3.1.4.7} \newline
                  दे॒वा॒सु॒राः सम्ॅय॑त्ता आसन्न् । ते दे॒वाः स॒र्पेभ्य॑ आश्रे॒षाभ्य॒ आज्ये॑ कर॒म्भं निर॑वपन्न् । ताने॒ताभि॑रे॒व दे॒वता॑-भि॒रुपा॑नयन्न् । ए॒ताभि॑र्ह॒ वै दे॒वता॑भिर्-द्वि॒षन्तं॒ भ्रातृ॑व्य॒-मुप॑नयति । य ए॒तेन॑ ह॒विषा॒ यज॑ते । य उ॑ चैनदे॒वं ॅवेद॑ । सोऽत्र॑ जुहोति । स॒र्पेभ्यः॒ स्वाहा᳚ऽऽश्रे॒षाभ्यः॒ स्वाहा᳚ । द॒न्द॒शूके᳚भ्यः॒ स्वाहेति॑ । \textbf{ 33} \newline
                  \newline
                                \textbf{ TB 3.1.4.8} \newline
                  पि॒तरो॒ वा अ॑कामयन्त । पि॒तृ॒लो॒क ऋ॑द्ध्नुया॒मेति॑ । त ए॒तं पि॒तृभ्यो॑ म॒घाभ्यः॑ पुरो॒डाशꣳ॒॒ षट्क॑पालं॒ निर॑वपन्न् । ततो॒ वै ते पि॑तृलो॒क आ᳚र्द्ध्नुवन्न् । पि॒तृ॒लो॒के ह॒ वा ऋ॑ध्नोति । य ए॒तेन॑ ह॒विषा॒ यज॑ते । य उ॑ चैनदे॒वं ॅवेद॑ । सोऽत्र॑ जुहोति । पि॒तृभ्यः॒ स्वाहा॑ म॒घाभ्यः॑ । स्वाहा॑ ऽन॒घाभ्यः॒ स्वाहा॑ ऽग॒दाभ्यः॑ । स्वाहा॑ ऽरुन्ध॒तीभ्यः॒ स्वाहेति॑ । \textbf{ 34} \newline
                  \newline
                                \textbf{ TB 3.1.4.9} \newline
                  अ॒र्य॒मा वा अ॑कामयत । प॒शु॒मान्थ् स्या॒मिति॑ । स ए॒तम॑र्य॒म्णे फल्गु॑नीभ्यां च॒रुं निर॑वपत् । ततो॒ वै स प॑शु॒मान॑भवत् । प॒शु॒मान्. ह॒ वै भ॑वति । य ए॒तेन॑ ह॒विषा॒ यज॑ते । य उ॑ चैनदे॒वं ॅवेद॑ । सोऽत्र॑ जुहोति । अ॒र्य॒म्णे स्वाहा॒ फल्गु॑नीभ्याꣳ॒॒ स्वाहा᳚ । प॒शुभ्यः॒ स्वाहेति॑ । \textbf{ 35} \newline
                  \newline
                                \textbf{ TB 3.1.4.10} \newline
                  भगो॒ वा अ॑कामयत । भ॒गी श्रे॒ष्ठी दे॒वानाꣳ॑ स्या॒मिति॑ । स ए॒तं भगा॑य॒ फल्गु॑नीभ्यां च॒रुं निर॑वपत् । ततो॒ वै स भ॒गी श्रे॒ष्ठी दे॒वाना॑मभवत् । भ॒गी ह॒ वै श्रे॒ष्ठी स॑मा॒नानां᳚ भवति । य ए॒तेन॑ ह॒विषा॒ यज॑ते । य उ॑ चैनदे॒वं ॅवेद॑ । सोऽत्र॑ जुहोति । भगा॑य॒ स्वाहा॒ फल्गु॑नीभ्याꣳ॒॒ स्वाहा᳚ । श्रैष्ठ्या॑य॒ स्वाहेति॑ । \textbf{ 36} \newline
                  \newline
                                \textbf{ TB 3.1.4.11} \newline
                  स॒वि॒ता वा अ॑कामयत । श्रन्मे॑ दे॒वा दधी॑रन्न् । स॒वि॒ता स्या॒मिति॑ । स ए॒तꣳ स॑वि॒त्रे हस्ता॑य पुरो॒डाशं॒ द्वाद॑शकपालं॒ निर॑वपदाशू॒नां ॅव्री॑ही॒णाम् । ततो॒ वै तस्मै॒ श्रद्दे॒वा अद॑धत । स॒वि॒ताऽभ॑वत् । श्रद्ध॒ वा अ॑स्मै मनु॒ष्या॑ दधते । स॒वि॒ता स॑मा॒नानां᳚ भवति । य ए॒तेन॑ ह॒विषा॒ यज॑ते । य उ॑ चैनदे॒वं ॅवेद॑ । सोऽत्र॑ जुहोति । स॒वि॒त्रे स्वाहा॒ हस्ता॑य । स्वाहा॑ दद॒ते स्वाहा॑ पृण॒ते । स्वाहा᳚ प्र॒यच्छ॑ते॒ स्वाहा᳚ प्रतिगृभ्ण॒ते स्वाहेति॑ । \textbf{ 37} \newline
                  \newline
                                \textbf{ TB 3.1.4.12} \newline
                  त्वष्टा॒ वा अ॑कामयत । चि॒त्रं प्र॒जां ॅवि॑न्दे॒येति॑ । स ए॒तं त्वष्ट्रे॑ चि॒त्रायै॑ पुरो॒डाश॑म॒ष्टा-क॑पालं॒ निर॑वपत् । ततो॒ वै स चि॒त्रं प्र॒जाम॑विन्दत । चि॒त्रꣳ ह॒ वै प्र॒जां ॅवि॑न्दते । य ए॒तेन॑ ह॒विषा॒ यज॑ते । य उ॑ चैनदे॒वं ॅवेद॑ । सोऽत्र॑ जुहोति । त्वष्ट्रे॒ स्वाहा॑ चि॒त्रायै॒ स्वाहा᳚ । चैत्रा॑य॒ स्वाहा᳚ प्र॒जायै॒ स्वाहेति॑ । \textbf{ 38} \newline
                  \newline
                                \textbf{ TB 3.1.4.13} \newline
                  वा॒युर्वा अ॑कामयत । का॒म॒चार॑मे॒षु लो॒केष्व॒भिज॑येय॒मिति॑ । स ए॒तद्-वा॒यवे॒ निष्ट्या॑यै गृ॒ष्ट्यै दु॒ग्धं पयो॒ निर॑वपत् । ततो॒ वै स का॑म॒चार॑मे॒षु लो॒केष्व॒भ्य॑जयत् । का॒म॒चारꣳ॑ ह॒ वा ए॒षु लो॒केष्व॒भिज॑यति । य ए॒तेन॑ ह॒विषा॒ यज॑ते । य उ॑ चैनदे॒वं ॅवेद॑ । सोऽत्र॑ जुहोति । वा॒यवे॒ स्वाहा॒ निष्ट्या॑यै॒ स्वाहा᳚ । का॒म॒चारा॑य॒ स्वाहा॒ऽभिजि॑त्यै॒ स्वाहेति॑ । \textbf{ 39} \newline
                  \newline
                                \textbf{ TB 3.1.4.14} \newline
                  इ॒न्द्रा॒ग्नी वा अ॑कामयेताम् । श्रैष्ठ्यं॑ दे॒वाना॑-म॒भिज॑ये॒वेति॑ । तावे॒तमि॑न्द्रा॒ग्निभ्यां॒ ॅविशा॑खाभ्यां पुरो॒डाश॒-मेका॑दशकपालं॒ निर॑वपताम् । ततो॒ वै तौ श्रैष्ठ्यं॑ दे॒वाना॑-म॒भ्य॑जयताम् । श्रैष्ठ्यꣳ॑ ह॒ वै स॑मा॒नाना॑म॒भिज॑यति । य ए॒तेन॑ ह॒विषा॒ यज॑ते । य उ॑ चैनदे॒वं ॅवेद॑ । सोऽत्र॑ जुहोति । इ॒न्द्रा॒ग्निभ्याꣳ॒॒ स्वाहा॒ विशा॑खाभ्याꣳ॒॒ स्वाहा᳚ । श्रैष्ठ्या॑य॒ स्वाहा॒ ऽभिजि॑त्यै॒ स्वाहेति॑ । \textbf{ 40} \newline
                  \newline
                                \textbf{ TB 3.1.4.15} \newline
                  अथै॒तत्-पौ᳚र्णमा॒स्या आज्यं॒ निर्व॑पति । कामो॒ वै पौ᳚र्णमा॒सी । काम॒ आज्य᳚म् । कामे॑नै॒व कामꣳ॒॒ सम॑र्द्धयति । क्षि॒प्रमे॑नꣳ॒॒ स काम॒ उप॑नमति । येन॒ कामे॑न॒ यज॑ते । सोऽत्र॑ जुहोति । पौ॒र्ण॒मा॒स्यै स्वाहा॒ कामा॑य॒ स्वाहाऽऽग॑त्यै॒ स्वाहेति॑ । \textbf{ 41} \newline
                  \newline
                                    श्पॆचिअल् खोर्वै इन्दिचतिन्ग् थे नुम्बेर् ऒf "वाक्यम्स्" इन् एअच् डसिनि ऒf 4थ् आनुवकम्  (अ॒ग्निः पञ्च॑दश - प्र॒जाप॑तिः॒ षोड॑श॒ - सोम॒ एका॑दश - रु॒द्रो दश॒ - र्.क्षैका॑दश॒ - बृह॒स्पति॒र् दश॑ - देवासु॒रा नव॑ - पि॒तर॒ एका॑दशा-र्य॒मा-भगो॒ दश॑ दश-सवि॒ता चतु॑र्दश॒-त्वष्टा॑-वा॒यु- रि॑न्द्रा॒ग्नी दश॑ द॒शा-थै॒तत् पौ᳚र्णमा॒स्या अ॒ष्टौ पञ्च॑दश ) \textbf{(A4)} \newline \newline
                \textbf{ 3.1.5     अनुवाकं   5 - यमनक्षत्रेष्टिब्राह्मणम्} \newline
                                \textbf{ TB 3.1.5.1} \newline
                  मि॒त्रो वा अ॑कामयत । मि॒त्र॒धेय॑मे॒षु लो॒केष्व॒भि ज॑येय॒मिति॑ । स ए॒तं मि॒त्राया॑नूरा॒धेभ्य॑श्च॒रुं निर॑वपत् । ततो॒ वै स मि॑त्र॒धेय॑मे॒षु लो॒केष्व॒भ्य॑जयत् । मि॒त्र॒धेयꣳ॑ ह॒ वा ए॒षु लो॒केष्व॒भिज॑यति । य ए॒तेन॑ ह॒विषा॒ यज॑ते । य उ॑ चैनदे॒वं ॅवेद॑ । सोऽत्र॑ जुहोति । मि॒त्राय॒ स्वाहा॑ ऽनूरा॒धेभ्यः॒ स्वाहा᳚ । मि॒त्र॒धेया॑य॒ स्वाहा॒ ऽभिजि॑त्यै॒ स्वाहेति॑ । \textbf{ 42} \newline
                  \newline
                                \textbf{ TB 3.1.5.2} \newline
                  इन्द्रो॒ वा अ॑कामयत । ज्यैष्ठ्यं॑ दे॒वाना॑-म॒भिज॑येय॒मिति॑ । स ए॒तमिन्द्रा॑य ज्ये॒ष्ठायै॑ पुरो॒डाश॒-मेका॑दशकपालं॒ निर॑वपन्-म॒हाव्री॑हीणाम् । ततो॒ वै स ज्यैष्ठ्यं॑ दे॒वाना॑म॒भ्य॑जयत् । ज्यैष्ठ्यꣳ॑ ह॒ वै स॑मा॒नाना॑-म॒भिज॑यति । य ए॒तेन॑ ह॒विषा॒ यज॑ते । य उ॑ चैनदे॒वं ॅवेद॑ । सोऽत्र॑ जुहोति । इन्द्रा॑य॒ स्वाहा᳚ ज्ये॒ष्ठायै॒ स्वाहा᳚ । ज्यैष्ठ्या॑य॒ स्वाहा॒ ऽभिजि॑त्यै॒ स्वाहेति॑ । \textbf{ 43} \newline
                  \newline
                                \textbf{ TB 3.1.5.3} \newline
                  प्र॒जाप॑ति॒र्वा अ॑कामयत । मूलं॑ प्र॒जां ॅवि॑न्दे॒येति॑ । स ए॒तं प्र॒जाप॑तये॒ मूला॑य च॒रुं निर॑वपत् । ततो॒ वै स मूलं॑ प्र॒जाम॑विन्दत । मूलꣳ॑ ह॒ वै प्र॒जां ॅवि॑न्दते । य ए॒तेन॑ ह॒विषा॒ यज॑ते । य उ॑ चैनदे॒वं ॅवेद॑ । सोऽत्र॑ जुहोति । प्र॒जाप॑तये॒ स्वाहा॒ मूला॑य॒ स्वाहा᳚ । प्र॒जायै॒ स्वाहेति॑ । \textbf{ 44} \newline
                  \newline
                                \textbf{ TB 3.1.5.4} \newline
                  आपो॒ वा अ॑कामयन्त । स॒मु॒द्रं काम॑म॒ भिज॑ये॒मेति॑ । ता ए॒तम॒द्भ्यो॑ऽषा॒ढाभ्य॑श्च॒रुं निर॑वपन्न् । ततो॒ वै ताः स॑मु॒द्रं काम॑म॒भ्य॑जयन्न् । स॒मु॒द्रꣳ ह॒ वै काम॑म॒भिज॑यति । य ए॒तेन॑ ह॒विषा॒ यज॑ते । य उ॑ चैनदे॒वं ॅवेद॑ । सोऽत्र॑ जुहोति । अ॒द्भ्यः स्वाहा॑ ऽषा॒ढाभ्यः॒ स्वाहा᳚ । स॒मु॒द्राय॒ स्वाहा॒ कामा॑य॒ स्वाहा᳚ । अ॒भिजि॑त्यै॒ स्वाहेति॑ । \textbf{ 45} \newline
                  \newline
                                \textbf{ TB 3.1.5.5} \newline
                  विश्वे॒ वै दे॒वा अ॑कामयन्त । अ॒न॒प॒ज॒य्यं ज॑ये॒मेति॑ । त ए॒तं ॅविश्वे᳚भ्यो दे॒वेभ्यो॑ऽषा॒ढाभ्य॑श्च॒रुं निर॑वपन्न् । ततो॒ वै ते॑ऽनपज॒य्य-म॑जयन्न् । अ॒न॒प॒ज॒य्यꣳ ह॒ वै ज॑यति । य ए॒तेन॑ ह॒विषा॒ यज॑ते । य उ॑ चैनदे॒वं ॅवेद॑ । सोऽत्र॑ जुहोति । विश्वे᳚भ्यो दे॒वेभ्यः॒ स्वाहा॑ ऽषा॒ढाभ्यः॒ स्वाहा᳚ । अ॒न॒प॒ज॒य्याय॒ स्वाहा॒ जित्यै॒ स्वाहेति॑ । \textbf{ 46} \newline
                  \newline
                                \textbf{ TB 3.1.5.6} \newline
                  ब्रह्म॒ वा अ॑कामयत । ब्र॒ह्म॒लो॒कम॒भिज॑येय॒मिति॑ । तदे॒तं ब्रह्म॑णेऽभि॒जिते॑ च॒रुं निर॑वपत् । ततो॒ वै तद्ब्र॑ह्मलो॒कम॒भ्य॑जयत् । ब्र॒ह्म॒लो॒कꣳ ह॒ वा अ॒भिज॑यति । य ए॒तेन॑ ह॒विषा॒ यज॑ते । य उ॑ चैनदे॒वं ॅवेद॑ । सोऽत्र॑ जुहोति । ब्रह्म॑णे॒ स्वाहा॑ ऽभि॒जिते॒ स्वाहा᳚ । ब्र॒ह्म॒लो॒काय॒ स्वाहा॒ ऽभिजि॑त्यै॒ स्वाहेति॑ । \textbf{ 47} \newline
                  \newline
                                \textbf{ TB 3.1.5.7} \newline
                  विष्णु॒र्वा अ॑कामयत । पुण्यꣳ॒॒ श्लोकꣳ॑ शृण्वीय । न मा॑ पा॒पी की॒र्तिराग॑च्छे॒दिति॑ । स ए॒तं ॅविष्ण॑वे श्रो॒णायै॑ पुरो॒डाशं॑ त्रिकपा॒लं निर॑वपत् । ततो॒ वै स पुण्यꣳ॒॒ श्लोक॑म शृणुत । नैनं॑ पा॒पी की॒र्तिरा- ग॑च्छत् । पुण्यꣳ॑ ह॒ वै श्लोकꣳ॑ शृणुते । नैनं॑ पा॒पी की॒र्तिराग॑च्छति । य ए॒तेन॑ ह॒विषा॒ यज॑ते । य उ॑ चैनदे॒वं ॅवेद॑ । सोऽत्र॑ जुहोति । विष्ण॑वे॒ स्वाहा᳚ श्रो॒णायै॒ स्वाहा᳚ । श्लोका॑य॒ स्वाहा᳚ श्रु॒ताय॒ स्वाहेति॑ । \textbf{ 48} \newline
                  \newline
                                \textbf{ TB 3.1.5.8} \newline
                  वस॑वो॒ वा अ॑कामयन्त । अग्रं॑ दे॒वता॑नां॒ परी॑या॒मेति॑ । त ए॒तं ॅवसु॑भ्यः॒ श्रवि॑ष्ठाभ्यः पुरो॒डाश॑म॒ष्टाक॑पालं॒ निर॑वपन्न् । ततो॒ वै तेऽग्रं॑ दे॒वता॑नां॒ पर्या॑यन्न् । अग्रꣳ॑ ह॒ वै स॑मा॒नानां॒ पर्ये॑ति । य ए॒तेन॑ ह॒विषा॒ यज॑ते । य उ॑ चैनदे॒वं ॅवेद॑ । सोऽत्र॑ जुहोति । वसु॑भ्यः॒ स्वाहा॒ श्रवि॑ष्ठाभ्यः॒ स्वाहा᳚ । अग्रा॑य॒ स्वाहा॒ परी᳚त्यै॒ स्वाहेति॑ । \textbf{ 49} \newline
                  \newline
                                \textbf{ TB 3.1.5.9} \newline
                  इन्द्रो॒ वा अ॑कामयत । दृ॒ढो ऽशि॑थिलः स्या॒मिति॑ । स ए॒तं ॅवरु॑णाय श॒तभि॑षजे भेष॒जेभ्यः॑ पुरो॒डाशं॒ दश॑कपालं॒ निर॑वपत्-कृ॒ष्णानां᳚ ॅव्रीही॒णाम् । ततो॒ वै स दृ॒ढोऽशि॑थिलोऽभवत् । दृ॒ढो ह॒ वा अशि॑थिलो भवति । य ए॒तेन॑ ह॒विषा॒ यज॑ते । य उ॑ चैनदे॒वं ॅवेद॑ । सोऽत्र॑ जुहोति । वरु॑णाय॒ स्वाहा॑ श॒तभि॑षजे॒ स्वाहा᳚ । भे॒ष॒जेभ्यः॒ स्वाहेति॑ । \textbf{ 50} \newline
                  \newline
                                \textbf{ TB 3.1.5.10} \newline
                  अ॒जो वा एक॑पादकामयत । ते॒ज॒स्वी ब्र॑ह्मवर्च॒सी स्या॒मिति॑ । स ए॒तम॒जायैक॑पदे प्रोष्ठप॒देभ्य॑-श्च॒रुं निर॑वपत् । ततो॒ वै स ते॑ज॒स्वी ब्र॑ह्मवर्च॒स्य॑भवत् । ते॒ज॒स्वी ह॒ वै ब्र॑ह्मवर्च॒सी भ॑वति । य ए॒तेन॑ ह॒विषा॒ यज॑ते । य उ॑ चैनदे॒वं ॅवेद॑ । सोऽत्र॑ जुहोति । अ॒जायैक॑पदे॒ स्वाहा᳚ प्रोष्ठप॒देभ्यः॒ स्वाहा᳚ । तेज॑से॒ स्वाहा᳚ ब्रह्मवर्च॒साय॒ स्वाहेति॑ । \textbf{ 51} \newline
                  \newline
                                \textbf{ TB 3.1.5.11} \newline
                  अहि॒र्वै बु॒द्ध्नियो॑ऽकामयत । इ॒मां प्र॑ति॒ष्ठां ॅवि॑न्दे॒येति॑ । स ए॒तमह॑ये बु॒द्ध्निया॑य प्रोष्ठप॒देभ्यः॑ पुरो॒डाशं॒ भूमि॑कपालं॒ निर॑वपत् । ततो॒ वै स इ॒मां प्र॑ति॒ष्ठा-म॑विन्दत । इ॒माꣳ ह॒ वै प्र॑ति॒ष्ठां ॅवि॑न्दते । य ए॒तेन॑ ह॒विषा॒ यज॑ते । य उ॑ चैनदे॒वं ॅवेद॑ । सोऽत्र॑ जुहोति । अह॑ये बु॒द्ध्निया॑य॒ स्वाहा᳚ प्रोष्ठप॒देभ्यः॒ स्वाहा᳚ । प्र॒ति॒ष्ठायै॒ स्वाहेति॑ । \textbf{ 52} \newline
                  \newline
                                \textbf{ TB 3.1.5.12} \newline
                  पू॒षा वा अ॑कामयत । प॒शु॒मान्थ् स्या॒मिति॑ । स ए॒तं पू॒ष्णे रे॒वत्यै॑ च॒रुं निर॑वपत् । ततो॒ वै स प॑शु॒मान॑भवत् । प॒शु॒मान्. ह॒ वै भ॑वति । य ए॒तेन॑ ह॒विषा॒ यज॑ते । य उ॑ चैनदे॒वं ॅवेद॑ । सोऽत्र॑ जुहोति । पू॒ष्णे स्वाहा॑ रे॒वत्यै॒ स्वाहा᳚ । प॒शुभ्यः॒ स्वाहेति॑ । \textbf{ 53} \newline
                  \newline
                                \textbf{ TB 3.1.5.13} \newline
                  अ॒श्विनौ॒ वा अ॑कामयेताम् । श्रो॒त्र॒स्विना॒वब॑धिरौ स्या॒वेति॑ । तावे॒तम॒श्विभ्या॑-मश्व॒युग्भ्यां᳚ पुरो॒डाशं॑ द्विकपा॒लं निर॑वपताम् । ततो॒ वै तौ श्रो᳚त्र॒स्विना॒वब॑धिरा-वभवताम् । श्रो॒त्र॒स्वी ह॒ वा अब॑धिरो भवति । य ए॒तेन॑ ह॒विषा॒ यज॑ते । य उ॑ चैनदे॒वं ॅवेद॑ । सोऽत्र॑ जुहोति । अ॒श्विभ्याꣳ॒॒ स्वाहा᳚ऽश्व॒युग्भ्याꣳ॒॒ स्वाहा᳚ । श्रोत्रा॑य॒ स्वाहा॒ श्रुत्यै॒ स्वाहेति॑ । \textbf{ 54} \newline
                  \newline
                                \textbf{ TB 3.1.5.14} \newline
                  य॒मो वा अ॑कामयत । पि॒तृ॒णाꣳ रा॒ज्य-म॒भिज॑येय॒मिति॑ । स ए॒तं ॅय॒माया॑प॒भर॑णीभ्यश्च॒रुं निर॑वपत् । ततो॒ वै स पि॑तृ॒णाꣳ रा॒ज्यम॒भ्य॑जयत् । स॒मा॒नानाꣳ॑ ह॒ वै रा॒ज्यम॒भिज॑यति । य ए॒तेन॑ ह॒विषा॒ यज॑ते । य उ॑ चैनदे॒वं ॅवेद॑ । सोऽत्र॑ जुहोति । य॒माय॒ स्वाहा॑ ऽप॒भर॑णीभ्यः॒ स्वाहा᳚ । रा॒ज्याय॒ स्वाहा॒ ऽभिजि॑त्यै स्वाहेति॑ । \textbf{ 55} \newline
                  \newline
                                \textbf{ TB 3.1.5.15} \newline
                  अथै॒तद॑-मा॒वास्या॑या॒ आज्यं॒ निर्व॑पति । कामो॒ वा अ॑मावा॒स्या᳚ । काम॒ आज्य᳚म् । कामे॑नै॒व कामꣳ॒॒ सम॑र्द्धयति । क्षि॒प्रमे॑नꣳ॒॒ स काम॒ उप॑नमति । येन॒ कामे॑न॒ यज॑ते । सोऽत्र॑ जुहोति । अ॒मा॒वा॒स्या॑यै॒ स्वाहा॒ कामा॑य॒ स्वाहा ऽऽग॑त्यै॒ स्वाहेति॑ । \textbf{ 56} \newline
                  \newline
                                    श्पॆचिअल् खोर्वै इन्दिचतिन्ग् थे नुम्बेर् ऒf "वाक्यम्स्" इन् एअच् डसिनि ऒf 5थ् आनुवकम् (मि॒त्र - इन्द्रः॑ - प्र॒जाप॑ति॒र् दश॑ द॒शा - प॒ एका॑दश॒ - विश्वे॒ ब्रह्म॒ दश॑दश॒ - विष्णु॒ स्त्रयो॑दश॒ - वस॑व॒ - इन्द्रो॒ऽ - जोऽहि॒र्वै - 
बु॒द्ध्नियः॑ - पु॒षा - ऽश्विनौ॑ - य॒मो दश॑द॒शा - थै॒तद॑ - मावा॒स्या॑या अ॒ष्टौ पञ्च॑दश ) \textbf{(A5)} \newline \newline
                \textbf{ 3.1.6     अनुवाकं   6 - चान्द्रमसादीष्टिब्राह्मणम्} \newline
                                \textbf{ TB 3.1.6.1} \newline
                  च॒न्द्रमा॒ वा अ॑कामयत । अ॒हो॒रा॒त्रा-न॑र्द्धमा॒सान्-मासा॑-नृ॒तून्थ्- स॑म्ॅवथ्स॒र-मा॒प्त्वा । च॒न्द्रम॑सः॒ सायु॑ज्यꣳ सलो॒कता॑-माप्नुया॒मिति॑ । स ए॒तं च॒न्द्रम॑से प्रती॒दृश्या॑यै पुरो॒डाशं॒ पञ्च॑दशकपालं॒ निर॑वपत् । ततो॒ वै सो॑ऽहोरा॒त्रा-न॑र्द्धमा॒सान्-मासा॑-नृ॒तून्थ्-स॑म्ॅवथ्स॒र-मा॒प्त्वा । च॒न्द्रम॑सः॒ सायु॑ज्यꣳ सलो॒कता॑माप्नोत् । अ॒हो॒रा॒त्रान्. ह॒ वा अ॑र्द्धमा॒सान् मासा॑न्-ऋ॒तून्थ्-स॑म्ॅवथ्स॒र-मा॒प्त्वा । च॒न्द्रम॑सः॒ सायु॑ज्यꣳ सलो॒कता॑माप्नोति । य ए॒तेन॑ ह॒विषा॒ यज॑ते । य उ॑ चैनदे॒वं ॅवेद॑ । सोऽत्र॑ जुहोति । च॒न्द्रम॑से॒ स्वाहा᳚ प्रती॒दृश्या॑यै॒ स्वाहा᳚ । अ॒हो॒रा॒त्रेभ्यः॒ स्वाहा᳚ ऽर्द्धमा॒सेभ्यः॒ स्वाहा᳚ । मासे᳚भ्यः॒ स्वाह॒र्तुभ्यः॒ स्वाहा᳚ । स॒म्ॅव॒थ्स॒राय॒ स्वाहेति॑ । \textbf{ 57} \newline
                  \newline
                                \textbf{ TB 3.1.6.2} \newline
                  अ॒हो॒रा॒त्रे वा अ॑कामयेताम् । अत्य॑होरा॒त्रे मु॑च्येवहि । न ना॑वहोरा॒त्रे आ᳚प्नुयाता॒मिति॑ । ते ए॒तम॑हो-रा॒त्राभ्यां᳚ च॒रुं निर॑वपताम् ।  द्व॒यानां᳚ ॅव्रीही॒णाम् । शु॒क्लानां᳚ च कृ॒ष्णानां᳚ च । स॒वा॒त्योर्दु॒ग्धे । श्वे॒तायै॑ च कृ॒ष्णायै॑ च । ततो॒ वै ते अत्य॑होरा॒त्रे अ॑मुच्येते । नैने॑ अहोरा॒त्रे आ᳚प्नुताम् । अति॑ ह॒ वा अ॑होरा॒त्रे मु॑च्यते । नैन॑महोरा॒त्रे आ᳚प्नुतः । य ए॒तेन॑ ह॒विषा॒ यज॑ते । य उ॑ चैनदे॒वंॅवेद॑ । सोऽत्र॑ जुहोति । अह्ने॒ स्वाहा॒ रात्रि॑यै॒ स्वाहा᳚ । अति॑मुक्त्यै॒ स्वाहेति॑ । \textbf{ 58} \newline
                  \newline
                                \textbf{ TB 3.1.6.3} \newline
                  उ॒षा वा अ॑कामयत । प्रि॒या ऽऽदि॒त्यस्य॑ सु॒भगा᳚ स्या॒मिति॑ । सैतमु॒षसे॑ च॒रुं निर॑वपत् । ततो॒ वै सा प्रि॒या ऽऽदि॒त्यस्य॑ सु॒भगा॑ ऽभवत् । प्रि॒यो ह॒ वै स॑मा॒नानाꣳ॑ सु॒भगो॑ भवति । य ए॒तेन॑ ह॒विषा॒ यज॑ते । य उ॑ चैनदे॒वं ॅवेद॑ । सोऽत्र॑ जुहोति । उ॒षसे॒ स्वाहा॒ व्यु॑ष्ट्यै॒ स्वाहा᳚ । व्यू॒षुष्यै॒ स्वाहा᳚ व्यु॒च्छन्त्यै॒ स्वाहा᳚ । व्यु॑ष्टायै॒ स्वाहेति॑ । \textbf{ 59} \newline
                  \newline
                                \textbf{ TB 3.1.6.4} \newline
                  अथै॒तस्मै॒ नक्ष॑त्राय च॒रुं निर्व॑पति । यथा॒ त्वं दे॒वाना॒मसि॑ । ए॒वम॒हं म॑नु॒ष्या॑णां भूयास॒मिति॑ । यथा॑ ह॒ वा ए॒तद्-दे॒वाना᳚म् । ए॒वꣳ ह॒ वा ए॒ष म॑नु॒ष्या॑णां भवति । य ए॒तेन॑ ह॒विषा॒ यज॑ते । य उ॑ चैनदे॒वं ॅवेद॑ । सोऽत्र॑ जुहोति । नक्ष॑त्राय॒ स्वाहो॑देष्य॒ते स्वाहा᳚ । उ॒द्य॒ते स्वाहोदि॑ताय॒ स्वाहा᳚ । हर॑से॒ स्वाहा॒ भर॑से॒ स्वाहा᳚ । भ्राज॑से॒ स्वाहा॒ तेज॑से॒ स्वाहा᳚ । तप॑से॒ स्वाहा᳚ ब्रह्मवर्च॒साय॒ स्वाहेति॑ । \textbf{ 60} \newline
                  \newline
                                \textbf{ TB 3.1.6.5} \newline
                  सूर्यो॒ वा अ॑कामयत । नक्ष॑त्राणां प्रति॒ष्ठा स्या॒मिति॑ । स ए॒तꣳ सूर्या॑य॒ नक्ष॑त्रेभ्यश्च॒रुं निर॑वपत् । ततो॒ वै स नक्ष॑त्राणां प्रति॒ष्ठा ऽभ॑वत् । प्र॒ति॒ष्ठा ह॒ वै स॑मा॒नानां᳚ भवति । य ए॒तेन॑ ह॒विषा॒ यज॑ते । य उ॑ चैनदे॒वं ॅवेद॑ । सोऽत्र॑ जुहोति । सूर्या॑य॒ स्वाहा॒ नक्ष॑त्रेभ्यः॒ स्वाहा᳚ । प्र॒ति॒ष्ठायै॒ स्वाहेति॑ । \textbf{ 61} \newline
                  \newline
                                \textbf{ TB 3.1.6.6} \newline
                  अथै॒तमदि॑त्यै च॒रुं निर्व॑पति । इ॒यं ॅवा अदि॑तिः । अ॒स्यामे॒व प्रति॑ तिष्ठति । सोऽत्र॑ जुहोति । अदि॑त्यै॒ स्वाहा᳚ प्रति॒ष्ठायै॒ स्वाहेति॑ । \textbf{ 62} \newline
                  \newline
                                \textbf{ TB 3.1.6.7} \newline
                  अथै॒तं ॅविष्ण॑वे च॒रुं निर्व॑पति । य॒ज्ञो वै विष्णुः॑ । य॒ज्ञ् ए॒वान्त॒तः प्रति॑ तिष्ठति । सोऽत्र॑ जुहोति । विष्ण॑वे॒ स्वाहा॑ य॒ज्ञाय॒ स्वाहा᳚ । प्र॒ति॒ष्ठायै॒ स्वाहेति॑ । \textbf{ 63} \newline
                  \newline
                                                        \textbf{special korvai} \newline
              (स॒वि॒ताऽऽशू॒नां ॅव्री॑ही॒णामिन्द्रो॑ म॒हाव्री॑हीणा॒मिन्द्रः॑ कृ॒ष्णानों᳚ ॅव्रीही॒णाम॑होरा॒त्रे द्व॒यानां᳚ ॅव्रीही॒णाम् । पि॒तरः॒ षट्क॑पालꣳ सवि॒ता द्वाद॑शकपालमिन्द्रा॒ग्नी एका॑दशकपाल॒मिन्द्र॒ एका॑दशकपाल॒मिन्द्रो॒ दश॑कपालं॒ ॅविष्णु॑स्त्रिकपा॒लमहि॒र् भूमि॑कपालम॒श्विनौ᳚ द्विकपा॒लं च॒न्द्रमाः॒ पञ्च॑दशकपाल-म॒ग्निरत्वष्टा॒ वस॑वो॒ऽष्टाक॑पालम॒न्यत्र॑ च॒रुम् । रु॒द्रो᳚ऽर्य॒मा पू॒षा प॑शु॒मान्थ् स्याꣳ॒॒ सोमो॑ रु॒द्रो बृह॒स्पतिः॒ पय॑सि वा॒युः पयः॒ सोमो॑ वा॒युरि॑न्द्रा॒ग्नी मि॒त्र इन्द्र॒ आपो॒ ब्रह्म॑ य॒मो॑ऽभिजि॑त्यै॒ त्वष्टा᳚ प्र॒जाप॑तिः प्र॒जायै॑ पौर्णमा॒स्या अ॑मावा॒स्या॑या॒ आग॑त्यै॒ विश्वे॒ जित्या॑ अ॒श्विनौ॒ श्रुत्यै᳚ । ब्रह्म॒ तदे॒तं ॅविष्णुः॒ स ए॒तं ॅवा॒युः स ए॒तदाप॒स्ताः । पि॒तरो॒ विश्वे॒ वस॑वोऽकामयन्त॒मेति॒ त ए॒तं निर॑वपन्न् । आपो॑ऽकामयन्त॒ मेति॒ ता ए॒तं निर॑वपन्न् । इ॒न्द्रा॒ग्नी अ॒श्विना॑वकामयेतां॒ ॅवेति॒ तावे॒तं निर॑वपदाम् । अ॒हो॒रा॒त्रे वा अ॑कामयेता॒मिति॒ ते ए॒तं निर॑वपताम् । अ॒न्यत्रा॑कामयत॒मिति॒ स ए॒तं निर॑वपत् । इ॒न्द्रा॒ग्नी श्रैष्ठ्य॒मिन्द्रो॒ ज्यैष्ठ्य॒मिन्द्रो॑ दृ॒ढः । आहिः॒ सूर्योऽदि॑त्यै॒ विष्ण॑वे प्रति॒ष्ठायै᳚ । सोमो॑ य॒मः स॑मा॒नाना᳚म् । अ॒ग्निर्नो॑ रीरिषद॒न्यत्र॑ रीरिषः) \newline
                            \textbf{special korvai} \newline
              (वि॒दुषो॑ ह॒विषा॒ य उ॑ चैनदे॒वं ॅवेद॑ । इष्टि॑भि॒र्य उ॑ चैना ए॒वं ॅवेद॑ । अ॒न्यत्र॑ ब्राह्म॒णे य उ॑ चैनमे॒वं ॅवेद॑ ) \newline
                                (च॒न्द्रमाः॒ पञ्च॑दशा - होरा॒त्रे स॒प्तद॑ - शो॒षा एका॑द॒शा - थै॒तस्मै॒ नक्ष॑त्ताय॒ त्रयो॑दश॒ - सूर्यो॒ दशा - थै॒तमदि॑त्यै॒ पञ्चा - थै॒तं ॅविष्ण॑वे॒ षट्थ् स॒प्त) \textbf{(A6)} \newline \newline
                \textbf{Prapaataka korvai with starting  Padams of 1 to 6 Anuvaakams :-} \newline
        (अ॒ग्निर्न॑-ऋ॒द्ध्यास्म॒-नवो॑नवो॒-ऽग्निर्-मि॒त्र - श्च॒न्द्रमाः॒ षट्) \newline

        \textbf{korvai with starting Words of 1, 11, 21 Series of Dasinis :-} \newline
        (अ॒ग्निर्न॒ - स्तन्नो॑ वा॒यु - रहि॑र्बू॒द्ध्निय॑ - ऋ॒क्षा वा इ॒य - मथै॒तत् पौ᳚र्णमा॒स्या - अ॒जो वा एक॑पा॒थ् - सूर्य॒ स्त्रिष॑ष्टिः) \newline

        \textbf{first and last  Word 3rd Ashtakam 1st Prapaatakam :-} \newline
        (अ॒ग्निर्नः॑ - पातु प्रति॒ष्ठायै॒ स्वाहेति॑ ) \newline 

       

        ॥ हरिः॑ ॐ ॥
॥ कृष्ण यजुर्वेदीय तैत्तिरीय ब्राह्मणे तृतीयाष्टके प्रथमः प्रपाठकः समाप्तः ॥

Appendix (of Expansions)
ट्.भ्.3.1.2.11 - "नि॒वेश॑नी॒ {1}" "यत्ते॑ दे॒वा अद॑धुः {2}"
नि॒वेश॑नी स॒ङ्गम॑नी॒ वसू॑नां॒ ॅविश्वा॑ रू॒पाणि॒ वसू᳚न्यावे॒शय॑न्ती । 
स॒ह॒स्र॒पो॒षꣳ सु॒भगा॒ ररा॑णा॒ सा न॒ आग॒न् वर्च॑सा सम्ॅविदा॒ना ॥ {1}

यत्ते॑ दे॒वा अद॑धुर् भाग॒धेय॒ममा॑वास्ये स॒म्ॅवस॑न्तो महि॒त्वा । 
सानो॑ य॒ज्ञ्ं पि॑पृहि विश्ववारे र॒यिं नो॑ धेहि सुभगे सु॒वीरं᳚ ॥ {2}
( Both {1] and {2] appearing in T.S.3.5.1.1) 

ट्.भ्.3.1.3.1 - "नवो॑ नवो भवति॒ जाय॑मानो॒ {3]" 
"यमा॑दि॒त्या अꣳ॒॒शुमा᳚प्या॒यय॑न्ति {4}" 
नवो॑नवो भवति॒ जाय॑मा॒नोऽह्नां᳚ के॒तुरु॒षसा॑ मे॒त्यग्रे᳚ । 
भा॒गं दे॒वेभ्यो॒ विद॑धात्या॒यन् प्रच॒न्द्रमा᳚स्तिरति दी॒र्घमायुः॑ ॥ {3]

यमा॑दि॒त्या अꣳ॒॒शुमा᳚प्या॒यय॑न्ति॒ यमक्षि॑त॒-मक्षि॑तयः॒ पिब॑न्ति । 
तेन॑ नो॒ राजा॒ वरु॑णो॒ बृह॒स्पति॒रा प्या॑ययन्तु॒ भुव॑नस्य गो॒पाः ॥ {4]
(Both {3] and {4] appearing in T.S.2.4.14.1)

ट्.भ्.3.1.3.3 - "उदु॒त्यं {5}" "चि॒त्रम् {6}" 
उदु॒ त्यं जा॒तवे॑दसं दे॒वं ॅव॑हन्ति के॒तवः॑ । दृ॒शे विश्वा॑य॒ सूर्यं᳚ ॥ {5}

चि॒त्रं दे॒वाना॒-मुद॑गा॒दनी॑कं॒ चक्षु॑र् मि॒त्रस्य॒ वरु॑णस्या॒ऽग्नेः । 
आऽ प्रा॒ द्यावा॑पृथि॒वी अ॒न्तरि॑क्षꣳ॒॒ सूर्य॑ आ॒त्मा जग॑तस्त॒स्थुष॑श्च ॥ {6}
( Both {5] and {6] appearing in T.S.1.4.43.1

ट्.भ्.3.1.3.3 - "अदि॑तिर्न उरुष्य॒तु {7}" "म॒हीमू॒षु मा॒तर᳚म् {8}" 
अदि॑तिर्न उरुष्य॒त्वदि॑तिः॒ शर्म॑ यच्छतु । अदि॑तिः पा॒त्वꣳ ह॑सः ॥ {7}

म॒हीमू॒षु मा॒तरꣳ॑ सुव्र॒ताना॑मृ॒तस्य॒ पत्नी॒मव॑से हुवेम । 
तु॒वि॒क्ष॒त्राम॒जर॑न्तीमुरू॒चीꣳ सु॒शर्मा॑ण॒मदि॑तिꣳ सु॒प्रणी॑तिं ॥ {8}
( Both {7] and {8] appearing in T.S.1.5.11.5) 

ट्.भ्.3.1.3.3 - "इ॒दं ॅविष्णुः॒ {9}" "प्रतद्विष्णुः॑ {10}" 
इ॒दं ॅविष्णु॒र् विच॑क्रमे त्रे॒धा नि द॑धे प॒दं । 
स मू॑ढमस्य पाꣳ सु॒॒रे ॥ {9} 
(AppEaring in T.S.1.2.13.1)

प्रतद्-विष्णुः॑ स्तवते वी॒र्या॑य । मृ॒गो न भी॒मः कु॑च॒रो गि॑रि॒ष्ठाः । 
यस्यो॒रुषु॑ त्रि॒षु वि॒क्रम॑णेषु । अधि॑क्षि॒यन्ति॒ भुव॑नानि॒ विश्वा᳚ ॥ {10}
(appearing in T.B.2.4.3.4) 

ट्.भ्.3.1.3.3 - "अ॒ग्निर्मू॒र्धा {11}" "भुवः॑ {12}" 
अ॒ग्निर्मू॒र्धा दि॒वः क॒कुत् पतिः॑ पृथि॒व्या अ॒यं । 
अ॒पाꣳ रेताꣳ॑सि जिन्वति ॥ {11}

भुवो॑ य॒ज्ञ्स्य॒ रज॑सश्च ने॒ता यत्रा॑ नि॒युद्भिः॒ सच॑से शि॒वाभिः॑ । 
दि॒वि मू॒र्धानं॑ दधिषे सुव॒र्॒.षां जि॒ह्वाम॑ग्ने चकृषे हव्य॒वाहं᳚ ॥ {12}
( Both {11] and {12] appearing in T.S.4.4.4.1

ट्.भ्.3.1.3.3 -"अनु॑ नो॒ऽद्यानु॑मति॒ {13}" "रन्विद॑नुमते॒ त्वम् {14}" 
अनु॑ नो॒ऽद्याऽनु॑मतिर् य॒ज्ञ्ं दे॒वेषु॑ मन्यतां । 
अ॒ग्निश्च॑ हव्य॒वाह॑नो॒ भव॑तां दा॒शुषे॒ मयः॑ ॥ {13}

अन्विद॑नुमते॒ त्वं मन्या॑सै॒ शञ्च॑नः कृधि । 
क्रत्वे॒ दक्षा॑य नो हिनु॒ प्रण॒ आयूꣳ॑षि तारिषः ॥ {14}
( Both {13] and {14] appearing in T.S.3.3.11.3)
ट्.भ्.3.1.3.3 - "ह॒व्य॒वाहꣳ॒॒ {15}" "स्वि॑ष्टम् {16}" 
ह॒व्य॒वाह॑-मभिमाति॒षाऽह᳚म् । र॒क्षो॒हणं॒ पृत॑नासु जि॒ष्णुम् । 
ज्योति॑ष्मन्तं॒ दीद्य॑तं॒ पुर॑न्धिम् । अ॒ग्निꣳ स्वि॑ष्ट॒कृत॒मा हु॑वेम ॥ {15}

स्वि॑ष्टमग्ने अ॒भि तत् पृ॑णाहि । विश्वा॑ देव॒ पृत॑ना अ॒भिष्य । 
उ॒रुं नः॒ पन्थां᳚ प्रदि॒शन्वि भा॑हि । ज्योति॑ष्मद्धेह्य॒जरं॑ न॒ आयुः॑ ॥ {16}
(Both {15] and {16] appearing in T.B.2.4.1.4) 
====================================== \newline
        \pagebreak
        
        
        
     \addcontentsline{toc}{section}{ 3.2     द्वितीयः प्रपाठकः - दशपूर्णमासेष्टिब्राह्मणम्}
     \markright{ 3.2     द्वितीयः प्रपाठकः - दशपूर्णमासेष्टिब्राह्मणम् \hfill https://www.vedavms.in \hfill}
     \section*{ 3.2     द्वितीयः प्रपाठकः - दशपूर्णमासेष्टिब्राह्मणम् }
                \textbf{ 3.2.1     अनुवाकं   1 - वथ्सापाकरणम्} \newline
                                \textbf{ TB 3.2.1.1} \newline
                  तृ॒तीय॑स्यामि॒तो दि॒वि सोम॑ आसीत् । तं गा॑य॒त्र्याह॑रत् । तस्य॑ प॒र्णम॑च्छिद्यत । तत् प॒र्णो॑ऽभवत् । तत् प॒र्णस्य॑ पर्ण॒त्वम् । ब्रह्म॒ वै प॒र्णः । यत् प॑र्णशा॒खया॑ व॒थ्सान॑पाक॒रोति॑ । ब्रह्म॑णै॒वैना॑न॒पाक॑रोति ॥ गा॒य॒त्रो वै प॒र्णः । गा॒य॒त्राः प॒शवः॑ \textbf{ 1} \newline
                  \newline
                                \textbf{ TB 3.2.1.2} \newline
                  तस्मा॒त् त्रीणि॑त्रीणि प॒र्णस्य॑ पला॒शानि॑ । त्रि॒पदा॑ गाय॒त्री । यत् प॑र्णश॒खया॒ गाः प्रा॒र्पय॑ति । स्वयै॒वैना॑ दे॒वत॑या॒ प्रार्प॑यति । यं का॒मये॑ता प॒शुः स्या॒दिति॑ । अ॒प॒र्णां तस्मै॒ शुष्का᳚ग्रा॒माह॑रेत् । अ॒प॒शुरे॒व भ॑वति । यं का॒मये॑त पशु॒मान्थ् स्या॒दिति॑ । ब॒हु॒प॒र्णां तस्मै॑ बहुशा॒खामाह॑रेत् । प॒शु॒मन्त॑-मे॒वैनं॑ करोति । \textbf{ 2} \newline
                  \newline
                                \textbf{ TB 3.2.1.3} \newline
                  यत् प्राची॑मा॒हरे᳚त् । दे॒व॒लो॒क-म॒भिज॑येत् । यदुदी॑चीं मनुष्यलो॒कम् । प्राची॒मुदी॑ची॒माह॑रति । उ॒भयो᳚र् लो॒कयो॑-र॒भिजि॑त्यै ॥ इ॒षे त्वो॒र्जे त्वेत्या॑ह । इष॑मे॒वोर्जं॒ ॅयज॑माने दधाति ॥ वा॒यवः॒ स्थेत्या॑ह । वा॒युर्वा अ॒न्तरि॑क्ष॒-स्याद्ध्य॑क्षाः । अ॒न्त॒रि॒क्ष॒दे॒व॒त्याः᳚ खलु॒ वै प॒शवः॑ \textbf{ 3} \newline
                  \newline
                                \textbf{ TB 3.2.1.4} \newline
                  वा॒यव॑ ए॒वैना॒न् परि॑ ददाति । प्र वा ए॑नाने॒तदा-क॑रोति । यदाह॑ । वा॒यवः॒ स्थेत्यु॑पा॒यवः॒ स्थेत्या॑ह । यज॑मानायै॒व प॒शूनुप॑ह्वयते ॥ दे॒वो वः॑ सवि॒ता प्रार्प॑य॒त्वित्या॑ह॒ प्रसू᳚त्यै ॥ श्रेष्ठ॑तमाय॒ कर्म॑ण॒ इत्या॑ह । य॒ज्ञो हि श्रेष्ठ॑तमं॒ कर्म॑ । तस्मा॑दे॒वमा॑ह ॥ आप्या॑यद्ध्वमघ्निया देवभा॒गमित्या॑ह \textbf{ 4} \newline
                  \newline
                                \textbf{ TB 3.2.1.5} \newline
                  व॒थ्सेभ्य॑श्च॒ वा ए॒ताः पु॒रा म॑नु॒ष्ये᳚भ्य॒-श्चाप्या॑यन्त । दे॒वेभ्य॑ ए॒वैना॒ इन्द्रा॒याप्या॑यति ॥ ऊर्ज॑स्वतीः॒ पय॑स्वती॒रित्या॑ह । ऊर्जꣳ॒ ॒हि पयः॑ स॒म्भर॑न्ति ॥ प्र॒जाव॑ती-रनमी॒वा अ॑य॒क्ष्मा इत्या॑ह॒ प्रजा᳚त्यै । मावः॑ स्ते॒न ई॑शत॒ माऽघशꣳ॑स॒ इत्या॑ह॒ गुप्त्यै᳚ ॥ रु॒द्रस्य॑ हे॒तिः परि॑ वो वृण॒क्त्वित्या॑ह । रु॒द्रा दे॒वैना᳚-स्त्रायते ॥ ध्रु॒वा अ॒स्मिन् गोप॑तौ स्यात ब॒ह्वीरित्या॑ह । ध्रु॒वा ए॒वास्मि॑न् ब॒ह्वीः क॑रोति ( ) । \textbf{ 5} \newline
                  \newline
                                \textbf{ TB 3.2.1.6} \newline
                  यज॑मानस्य प॒शून् पा॒हीत्या॑ह । प॒शु॒नां गो॑पी॒थाय॑ । तस्मा᳚थ्सा॒यं प॒शव॒ उप॑ स॒माव॑र्तन्ते ॥ अन॑धः सादयति । गर्भा॑णां॒ धृत्या॒ अप्र॑पादाय । तस्मा॒द्गर्भाः᳚ प्र॒जाना॒मप्र॑पादुकाः । उ॒परी॑व॒ निद॑धाति । उ॒परी॑व॒ हि सु॑व॒र्गो लो॒कः । सु॒व॒र्गस्य॑ लो॒कस्य॒ सम॑ष्ट्यै । \textbf{ 6} \newline
                  \newline
                                    (प॒शवः॑ - करोति - प॒शवो॑ - देवभा॒गमित्या॑ह - करोति॒ - +नव॑ च) \textbf{(A1)} \newline \newline
                \textbf{ 3.2.2     अनुवाकं   2 - बर्हिराहरणम्} \newline
                                \textbf{ TB 3.2.2.1} \newline
                  दे॒वस्य॑ त्वा सवि॒तुः प्र॑स॒व इत्य॑श्वप॒र्॒.शुमाद॑त्ते॒ प्रसू᳚त्यै । अ॒श्विनो᳚र्-बा॒हुभ्या॒मित्या॑ह । अ॒श्विनौ॒ हि दे॒वाना॑मद्ध्व॒र्यू आस्ता᳚म् । पू॒ष्णो हस्ता᳚भ्या॒मित्या॑ह॒ यत्यै᳚ ॥ यो वा ओष॑धीः पर्व॒शो वेद॑ । नैनाः॒ स हि॑नस्ति । प्र॒जाप॑ति॒र्वा ओष॑धीः पर्व॒शो वे॑द । स ए॑ना॒ न हि॑नस्ति । अ॒श्व॒प॒र्श्वा ब॒र्॒.हिरच्छै॑ति । प्रा॒जा॒प॒त्यो वा अश्वः॑ सयोनि॒त्वाय॑ \textbf{ 7} \newline
                  \newline
                                \textbf{ TB 3.2.2.2} \newline
                  ओष॑धीना॒महिꣳ॑सायै ॥ य॒ज्ञ्स्य॑ घो॒षद॒सीत्या॑ह । यज॑मान ए॒व र॒यिं द॑धाति । प्रत्यु॑ष्टꣳ॒॒ रक्षः॒ प्रत्यु॑ष्टा॒ अरा॑तय॒ इत्या॑ह । रक्ष॑सा॒-मप॑हत्यै ॥ प्रेयम॑गाद्धि॒षणा॑ ब॒र्॒.हिरच्छेत्या॑ह । वि॒द्या वै धि॒षणा᳚ । वि॒द्ययै॒वैन॒दच्छै॑ति ॥ मनु॑ना कृ॒ता स्व॒धया॒ वित॒ष्टेत्या॑ह । मा॒न॒वी हि पर्.शुः॑ स्व॒धाकृ॑ता \textbf{ 8} \newline
                  \newline
                                \textbf{ TB 3.2.2.3} \newline
                  त आव॑हन्ति क॒वयः॑ पु॒रस्ता॒दित्या॑ह । शु॒श्रु॒वाꣳसो॒ वै क॒वयः॑ । य॒ज्ञ्ः पु॒रस्ता᳚त् । मु॒ख॒त ए॒व य॒ज्ञ्मार॑भते । अथो॒ यदे॒तदु॒क्त्वा यतः॒ कुत॑श्चा॒हर॑ति । तत् प्राच्या॑ ए॒व दि॒शो भ॑वति ॥ दे॒वेभ्यो॒ जुष्ट॑मि॒ह ब॒र्॒.हिरा॒सद॒ इत्या॑ह । ब॒र्॒.हिषः॒ समृ॑द्ध्यै । कर्म॒णोऽन॑पराधाय ॥ दे॒वानां᳚ परिषू॒तम॒सीत्या॑ह \textbf{ 9} \newline
                  \newline
                                \textbf{ TB 3.2.2.4} \newline
                  यद्वा इ॒दं किं च॑ । तद्-दे॒वानां᳚ परिषू॒तम् । अथो॒ यथा॒ वस्य॑से प्रति॒ प्रोच्याहे॒दं क॑रिष्या॒मीति॑ । ए॒वमे॒व तद॑द्ध्व॒र्युर्-दे॒वेभ्यः॑ प्रति॒ प्रोच्य॑ ब॒र्॒.हिर्दा॑ति । आ॒त्मनोऽहिꣳ॑सायै ॥ याव॑तः स्त॒म्बान् प॑रि दि॒शेत् । यत्तेषा॑-मुच्छिꣳ॒॒ष्यात् । अति॒ तद्-य॒ज्ञ्स्य॑ रेचयेत् । एकꣳ॑ स्त॒म्बं परि॑दिशेत् । तꣳ सर्वं॑ दायात् \textbf{ 10} \newline
                  \newline
                                \textbf{ TB 3.2.2.5} \newline
                  य॒ज्ञ्स्यान॑तिरेकाय ॥ व॒र्॒.षवृ॑द्ध-म॒सीत्या॑ह । व॒र्॒.ष वृ॑द्धा॒ वा ओष॑धयः ॥ देव॑ बर्.हि॒रित्या॑ह । दे॒वेभ्य॑ ए॒वैन॑त् करोति । मा त्वा॒ऽन्वङ्मा ति॒र्यगित्या॒हा हिꣳ॑सायै । पर्व॑ ते राद्ध्यास॒मित्या॒-हर्द्ध्यै᳚ ॥ आ॒च्छे॒त्ता ते॒ मा रि॑ष॒मित्या॑ह । नास्या॒त्मनो॑ मीयते । य ए॒वं ॅवेद॑ । \textbf{ 11} \newline
                  \newline
                                \textbf{ TB 3.2.2.6} \newline
                  देव॑बर्.हिः श॒तव॑ल्.शं॒ ॅविरो॒हेत्या॑ह । प्र॒जा वै ब॒र्॒.हिः । प्र॒जानां᳚ प्र॒जन॑नाय ॥ स॒हस्र॑वल्.शा॒ वि व॒यꣳ रु॑हे॒मेत्या॑ह । आ॒शिष॑मे॒वैतामाशा᳚स्ते ॥ पृ॒थि॒व्याः स॒पृंचः॑ पा॒हीत्या॑ह॒ प्रति॑ष्ठित्यै ॥ अयु॑ङ्गायुङ्गान्-मु॒ष्टीन् ॅलु॑नोति । मि॒थु॒न॒त्वाय॒ प्रजा᳚त्यै ॥ सु॒स॒भृंता᳚ त्वा॒ संभ॑रा॒मीत्या॑ह । ब्रह्म॑णै॒वैन॒थ्- संभ॑रति । \textbf{ 12} \newline
                  \newline
                                \textbf{ TB 3.2.2.7} \newline
                  अदि॑त्यै॒ रास्ना॒ऽसीत्या॑ह । इ॒यं ॅवा अदि॑तिः । अ॒स्या ए॒वैन॒द्-रास्नां᳚ करोति ॥ इ॒न्द्रा॒ण्यै स॒नंह॑न॒मित्या॑ह । इ॒न्द्रा॒णी वा अग्रे॑ दे॒वता॑नाꣳ॒॒ सम॑नह्यत । सा ऽऽर्द्ध्नो᳚त् । ऋद्ध्यै॒ संन॑ह्यति । प्र॒जा वै ब॒र्॒.हिः । प्र॒जाना॒मप॑रावापाय । तस्मा॒थ्-स्नाव॑संतताः प्र॒जा जा॑यन्ते । \textbf{ 13} \newline
                  \newline
                                \textbf{ TB 3.2.2.8} \newline
                  पू॒षा ते᳚ ग्र॒न्थिं ग्र॑थ्ना॒त्वित्या॑ह । पुष्टि॑मे॒व यज॑माने दधाति ॥ स ते॒ माऽऽ स्था॒दित्या॒हाहिꣳ॑सायै ॥ प॒श्चात् प्राञ्च॒मुप॑गूहति । प॒श्चाद् वै प्रा॒चीनꣳ॒॒ रेतो॑ धीयते । प॒श्चादे॒वास्मै᳚ प्रा॒चीनꣳ॒॒ रेतो॑ दधाति ॥ इन्द्र॑स्य त्वा बा॒हुभ्या॒-मुद्य॑च्छ॒ इत्या॑ह । इ॒न्द्रि॒यमे॒व यज॑माने दधाति ॥ बृह॒स्पते᳚र्मू॒र्द्ध्ना ह॑रा॒मीत्या॑ह । ब्रह्म॒ वै दे॒वानां॒ बृह॒स्पतिः॑ \textbf{ 14} \newline
                  \newline
                                \textbf{ TB 3.2.2.9} \newline
                  ब्रह्म॑णै॒वैन॑द्धरति ॥ उ॒र्व॑न्तरि॑क्ष॒-मन्वि॒हीत्या॑ह॒ गत्यै᳚ ॥ दे॒व॒गं॒मम॒सीत्या॑ह । दे॒वाने॒वैन॑द्गमयति ॥ अन॑धः सादयति । गर्भा॑णां॒ धृत्या॒ अप्र॑पादाय । तस्मा॒द्-गर्भाः᳚ प्र॒जाना॒मप्र॑पादुकाः । उ॒परी॑व॒ निद॑धाति । उ॒परी॑व॒ हि सु॑व॒र्गो लो॒कः । सु॒व॒र्गस्य॑ लो॒कस्य॒ सम॑ष्ट्यै ( ) । \textbf{ 15} \newline
                  \newline
                                    (स॒यो॒नि॒त्वाय॑ - स्व॒धाकृ॑ता॒ - ऽसीत्या॑ह - दाया॒द् - वेद॑ - भरति - जायन्ते॒ - बृह॒स्पतिः॒ - सम॑ष्ट्यै) \textbf{(A2)} \newline \newline
                \textbf{ 3.2.3     अनुवाकं   3 - दोहनम्} \newline
                                \textbf{ TB 3.2.3.1} \newline
                  पू॒र्वे॒द्युरि॒द्ध्मा ब॒र्॒.हिः क॑रोति । य॒ज्ञ्मे॒वारभ्य॑ गृही॒त्वोप॑वसति ॥ प्र॒जाप॑तिर् य॒ज्ञ्म॑सृजत । तस्यो॒खे अ॑स्रꣳसेताम् । य॒ज्ञो वै प्र॒जाप॑तिः । यथ् सा᳚न्नाय्यो॒खे भव॑तः । य॒ज्ञ्स्यै॒व तदु॒खे उप॑दधा॒त्य-प्र॑स्रꣳसाय ॥ शुन्ध॑द्ध्वं॒ दैव्या॑य॒ कर्म॑णे देवय॒ज्याया॒ इत्या॑ह । दे॒व॒य॒ज्याया॑ ए॒वैना॑नि शुन्धति ॥ मा॒त॒रिश्व॑नो घ॒र्मो॑ऽसीत्या॑ह \textbf{ 16} \newline
                  \newline
                                \textbf{ TB 3.2.3.2} \newline
                  अ॒न्तरि॑क्षं॒ ॅवै मा॑त॒रिश्व॑नो घ॒र्मः । ए॒षां ॅलो॒कानां॒ ॅविधृ॑त्यै ॥ द्यौर॑सि पृथि॒व्य॑सीत्या॑ह । दि॒वश्च॒ ह्ये॑षा पृ॑थि॒व्याश्च॒ संभृ॑ता । यदु॒खा । तस्मा॑दे॒वमा॑ह ॥ वि॒श्वधा॑या असि पर॒मेण॒ धाम्नेत्या॑ह । वृष्टि॒र्वै वि॒श्वधा॑याः । वृष्टि॑मे॒वाव॑रुन्धे । दृꣳह॑स्व॒ मा ह्वा॒रित्या॑ह॒ धृत्यै᳚ । \textbf{ 17} \newline
                  \newline
                                \textbf{ TB 3.2.3.3} \newline
                  वसू॑नां प॒वित्र॑म॒सीत्या॑ह । प्रा॒णा वै वस॑वः । तेषां॒ ॅवा ए॒तद्-भा॑ग॒धेय᳚म् । यत् प॒वित्र᳚म् । तेभ्य॑ ए॒वैन॑त् करोति ॥ श॒तधा॑रꣳ स॒हस्र॑धार॒मित्या॑ह । प्रा॒णेष्वे॒वायु॑र्-दधाति सर्व॒त्वाय॑ ॥ त्रि॒वृत् प॑लाशशा॒खायां᳚ दर्भ॒मयं॑ भवति । त्रि॒वृद्वै प्रा॒णः । त्रि॒वृत॑मे॒व प्रा॒णं म॑द्ध्य॒तो यज॑माने दधाति \textbf{ 18} \newline
                  \newline
                                \textbf{ TB 3.2.3.4} \newline
                  सौ॒म्यः प॒र्णः स॑योनि॒त्वाय॑ । सा॒क्षात् प॒वित्रं॑ द॒र्भाः । प्राख्सा॒यमधि॒ निद॑धाति । तत् प्रा॑णापा॒नयो॑ रू॒पम् । ति॒र्यक्प्रा॒तः । तद्-दर्.श॑स्य रू॒पम् । दा॒र्श्यꣳ ह्ये॑तदहः॑ । अन्नं॒ ॅवै च॒न्द्रमाः᳚ । अन्नं॑ प्रा॒णाः । उ॒भय॑मे॒वोपै॒त्यजा॑मित्वाय \textbf{ 19} \newline
                  \newline
                                \textbf{ TB 3.2.3.5} \newline
                  तस्मा॑द॒यꣳ स॒र्वतः॑ पवते ॥ हु॒तः स्तो॒को हु॒तो द्र॒फ्स इत्या॑ह॒ प्रति॑ष्ठित्यै । ह॒विषोऽस्क॑न्दाय । न हि हु॒तꣳ स्वाहा॑कृतꣳ॒॒ स्कन्द॑ति । दि॒वि नाको॒ नामा॒ग्निः । तस्य॑ वि॒प्रुषो॑ भाग॒धेय᳚म् । अ॒ग्नये॑ बृह॒ते नाका॒येत्या॑ह । नाक॑मे॒वाग्निं भा॑ग॒धेये॑न॒ सम॑र्द्धयति । स्वाहा॒ द्यावा॑पृथि॒वीभ्या॒मित्या॑ह । द्यावा॑पृथि॒व्योरे॒वैन॒त्-प्रति॑ष्ठापयति \textbf{ 20} \newline
                  \newline
                                \textbf{ TB 3.2.3.6} \newline
                  प॒वित्र॑व॒त्यान॑यति । अ॒पां चै॒वौष॑धीनां च॒ रसꣳ॒॒ सꣳसृ॑जति । अथो॒ ओष॑धीष्वे॒व प॒शून् प्रति॑ष्ठापयति ॥ अ॒न्वा॒रभ्य॒ वाचं॑ ॅयच्छति । य॒ज्ञ्स्य॒ धृत्यै᳚ । धा॒रय॑न्नास्ते । धा॒रय॑न्त इव॒ हि दु॒हन्ति॑ । काम॑धुक्ष॒ इत्या॒हातृ॒तीय॑स्यै । त्रय॑ इ॒मे लो॒काः । इ॒माने॒व लो॒कान्. यज॑मानो दुहे \textbf{ 21} \newline
                  \newline
                                \textbf{ TB 3.2.3.7} \newline
                  अ॒मूमिति॒ नाम॑ गृह्णाति । भ॒द्रमे॒वासां॒ कर्मा॒विष्क॑रोति ॥ सा वि॒श्वायुः॒ सा वि॒श्वव्य॑चाः॒ सा वि॒श्वक॒र्मेत्या॑ह । इ॒यं ॅवै वि॒श्वायुः॑ । अ॒न्तरि॑क्षं ॅवि॒श्वव्य॑चाः । अ॒सौ वि॒श्वक॑र्मा । इ॒माने॒वैताभि॑र्लो॒कान्. य॑थापू॒र्वं दु॑हे । अथो॒ यथा᳚ प्रदा॒त्रे पुण्य॑मा॒शास्ते᳚ । ए॒वमे॒वैना॑ ए॒तदुप॑स्तौति । तस्मा॒त्-प्रादा॒दित्यु॒न्नीय॒ वन्द॑माना उपस्तु॒वन्तः॑ प॒शून् दु॑हन्ति \textbf{ 22} \newline
                  \newline
                                \textbf{ TB 3.2.3.8} \newline
                  ब॒हु दु॒ग्धीन्द्रा॑य दे॒वेभ्यो॑ ह॒विरिति॒ वाचं॒ ॅविसृ॑जते । य॒था॒ दे॒व॒तमे॒व प्रसौ॑ति । दैव्य॑स्य च मानु॒षस्य॑ च॒ व्यावृ॑त्त्यै । त्रिरा॑ह । त्रिष॑त्या॒ हि दे॒वाः । अवा॑चं ॅय॒मोऽन॑न्वा-र॒भ्योत्त॑राः । अप॑रिमितमे॒वाव॑रुन्धे ॥ न दा॑रुपा॒त्रेण॑ दुह्यात् । अ॒ग्नि॒वद्वै दा॑रुपा॒त्रम् । यद्-दा॑रुपा॒त्रेण॑ दु॒ह्यात् \textbf{ 23} \newline
                  \newline
                                \textbf{ TB 3.2.3.9} \newline
                  या॒तया᳚म्ना ह॒विषा॑ यजेत । अथो॒ खल्वा॑हुः । पु॒रो॒डाश॑मुखानि॒ वै ह॒वीꣳषि॑ । नेत इ॑तः पुरो॒डाशꣳ॑ ह॒विषो॒ यामो॒ऽस्तीति॑ । काम॑मे॒व दा॑रुपा॒त्रेण॑ दुह्यात् ॥ शू॒द्र ए॒व न दु॑ह्यात् । अस॑तो॒ वा ए॒ष संभू॑तः । यच्छू॒द्रः । अह॑विरे॒व तदित्या॑हुः । यच्छू॒द्रो दोग्धीति॑ \textbf{ 24} \newline
                  \newline
                                \textbf{ TB 3.2.3.10} \newline
                  अ॒ग्नि॒हो॒त्रमे॒वन दु॑ह्याच्छू॒द्रः । तद्धि नोत्-पु॒नन्ति॑ । य॒दा खलु॒ वै प॒वित्र॑म॒त्येति॑ । अथ॒ तद्ध॒विरिति॑ ॥ संपृ॑च्यद्ध्व-मृतावरी॒रित्या॑ह । अ॒पां चै॒वौष॑धीनां च॒ रसꣳ॒॒ सꣳसृ॑जति । तस्मा॑द॒पां चौष॑धीनां च॒ रस॒मुप॑ जीवामः । म॒न्द्रा धन॑स्य सा॒तय॒ इत्या॑ह । पुष्टि॑मे॒व यज॑माने दधाति ॥ सोमे॑न॒ त्वाऽऽत॑न॒च्मीन्द्रा॑य॒ दधीत्या॑ह \textbf{ 25} \newline
                  \newline
                                \textbf{ TB 3.2.3.11} \newline
                  सोम॑मे॒वैन॑त् करोति । यो वै सोमं॑ भक्षयि॒त्वा । सं॒ॅव॒थ्स॒रꣳ सोमं॒ न पिब॑ति । पु॒न॒र्भक्ष्यो᳚ऽस्य सोमपी॒थो भ॑वति ॥ सोमः॒ खलु॒ वै सा᳚नां॒य्यम् । य ए॒वं ॅवि॒द्वान्थ् सा᳚नां॒य्यं पिब॑ति । अ॒पु॒न॒र्भक्ष्यो᳚ऽस्य सोमपी॒थो भ॑वति । न मृ॒न्मये॒नापि॑दद्ध्यात् । यन् मृ॒न्मये॑नापिद॒द्ध्यात् । पि॒तृ॒दे॒व॒त्यꣳ॑ स्यात् \textbf{ 26} \newline
                  \newline
                                \textbf{ TB 3.2.3.12} \newline
                  अ॒य॒स्पा॒त्रेण॑ वा दारुपा॒त्रेण॒ वाऽपि॑दधाति । तद्धि सदे॑वम् ॥ उ॒द॒न्वद्- भ॑वति । आपो॒ वै र॑क्षो॒घ्नीः । रक्ष॑सा॒-मप॑हत्यै ॥ अद॑स्तमसि॒ विष्ण॑वे॒ त्वेत्या॑ह । य॒ज्ञो वै विष्णुः॑ । य॒ज्ञायै॒वैन॒दद॑स्तं करोति । विष्णो॑ ह॒व्यꣳ र॑क्ष॒स्वेत्या॑ह॒ गुप्त्यै᳚ ॥ अन॑धः सादयति ( ) । गर्भा॑णां॒ धृत्या॒ अप्र॑पादाय । तस्मा॒द्-गर्भाः᳚ प्र॒जाना॒म प्र॑पादुकाः । उ॒परी॑व॒ निद॑धाति । उ॒परी॑व॒ हि सु॑व॒र्गो लो॒कः । सु॒व॒र्गस्य॑ लो॒कस्य॒ सम॑ष्ट्यै । \textbf{ 27} \newline
                  \newline
                                    (अ॒सीत्या॑ह॒ - धृत्यै॒ - यज॑माने दधा॒ - त्यजा॑मित्वाय - स्थापयति - दुहे - दुहन्ति - दु॒ह्याद् - दोग्धीति॒ - दधीत्या॑ह - स्याथ् - सादयति॒ पञ्च॑ च) \textbf{(A3)} \newline \newline
                \textbf{ 3.2.4     अनुवाकं   4 - हविर्निर्वापः} \newline
                                \textbf{ TB 3.2.4.1} \newline
                  कर्म॑णे वां दे॒वेभ्यः॑ शकेय॒मित्या॑ह॒ शक्त्यै᳚ । य॒ज्ञ्स्य॒ वै सन्त॑ति॒मनु॑ प्र॒जाः प॒शवो॒ यज॑मानस्य॒ सन्ता॑यन्ते । य॒ज्ञ्स्य॒ विच्छि॑त्ति॒मनु॑ प्र॒जाः प॒शवो॒ यज॑मानस्य॒ विच्छि॑द्यन्ते । य॒ज्ञ्स्य॒ सन्त॑तिरसि य॒ज्ञ्स्य॑ त्वा॒ संत॑त्यै स्तृणामि॒ सन्त॑त्यै त्वा य॒ज्ञ्स्येत्याह॑व॒नीया॒थ् सन्त॑नोति । यज॑मानस्य प्र॒जायै॑ पशू॒नाꣳ सन्त॑त्यै ॥ अ॒पः प्रण॑यति । श्र॒द्धा वा आपः॑ । श्र॒द्धामे॒वारभ्य॑ प्र॒णीय॒ प्रच॑रति । अ॒पः प्रण॑यति । य॒ज्ञो वा आपः॑ \textbf{ 28} \newline
                  \newline
                                \textbf{ TB 3.2.4.2} \newline
                  य॒ज्ञ्मे॒वारभ्य॑ प्र॒णीय॒ प्रच॑रति । अ॒पः प्रण॑यति । वज्रो॒ वा आपः॑ । वज्र॑मे॒व भ्रातृ॑व्येभ्यः प्र॒हृत्य॑ प्र॒णीय॒ प्रच॑रति । अ॒पः प्रण॑यति । आपो॒ वै र॑क्षो॒घ्नीः । रक्ष॑सा॒-मप॑हत्यै । अ॒पः प्रण॑यति । आपो॒ वै दे॒वानां᳚ प्रि॒यं धाम॑ । दे॒वाना॑मे॒व प्रि॒यं धाम॑ प्र॒णीय॒ प्रच॑रति \textbf{ 29} \newline
                  \newline
                                \textbf{ TB 3.2.4.3} \newline
                  अ॒पः प्रण॑यति । आपो॒ वै सर्वा॑ दे॒वताः᳚ । दे॒वता॑ ए॒वारभ्य॑ प्र॒णीय॒ प्रच॑रति ॥ वेषा॑य॒ त्वेत्या॑ह । वेषा॑य॒ ह्ये॑नदाद॒त्ते ॥ प्रत्यु॑ष्टꣳ॒॒ रक्षः॒ प्रत्यु॑ष्टा॒ अरा॑तय॒ इत्या॑ह । रक्ष॑सा॒-मप॑हत्यै ॥ धूर॒सीत्या॑ह । ए॒ष वै धुर्यो॒ऽग्निः । तं ॅयदनु॑पस्पृश्याती॒यात् \textbf{ 30} \newline
                  \newline
                                \textbf{ TB 3.2.4.4} \newline
                  अ॒द्ध्व॒र्युं च॒ यज॑मानं च॒ प्रद॑हेत् । उ॒प॒स्पृश्यात्ये॑ति । अ॒द्ध्व॒र्योश्च॒ यज॑मानस्य॒ चाप्र॑दाहाय ॥ धूर्व॒ तम्ॅयो᳚ऽस्मान् धूर्व॑ति॒ तं धू᳚र्व॒ यं ॅव॒यं धूर्वा॑म॒ इत्या॑ह । द्वौ वाव पुरु॑षौ । यं चै॒व धूर्व॑ति । यश्चै॑नं॒ धूर्व॑ति । तावु॒भौ शु॒चाऽर्प॑यति ॥ त्वं दे॒वाना॑मसि॒ सस्नि॑तमं॒ पप्रि॑तमं॒ जुष्ट॑तमं॒ ॅवह्नि॑तमं देव॒हूत॑म॒मित्या॑ह । य॒था॒ य॒जुरे॒वै तत् \textbf{ 31} \newline
                  \newline
                                \textbf{ TB 3.2.4.5} \newline
                  अह्रु॑तमसि हवि॒र्द्धान॒-मित्या॒हाना᳚र्त्यै । दृꣳह॑स्व॒ मा ह्वा॒रित्या॑ह॒ धृत्यै᳚ ॥ मि॒त्रस्य॑ त्वा॒ चक्षु॑षा॒ प्रेक्ष॒ इत्या॑ह मित्र॒त्वाय॑ । मा भेर्मा सम्ॅवि॑क्था॒ मा त्वा॑ हिꣳसिष॒-मित्या॒हाहिꣳ॑सायै । यद्वै किं च॒ वातो॒ नाभि॒वाति॑ । तथ् सर्वं॑ ॅवरुणदेव॒त्य᳚म् ॥ उ॒रु वाता॒येत्या॑ह । अवा॑रुणमे॒वैन॑त् करोति ॥ दे॒वस्य॑ त्वा सवि॒तुः प्र॑स॒व इत्या॑ह॒ प्रसू᳚त्यै । अ॒श्विनो᳚र्-बा॒हुभ्या॒मित्या॑ह \textbf{ 32} \newline
                  \newline
                                \textbf{ TB 3.2.4.6} \newline
                  अ॒श्विनौ॒ हि दे॒वाना॑मद्ध्व॒र्यू आस्ता᳚म् । पू॒ष्णो हस्ता᳚भ्या॒मित्या॑ह॒ यत्यै᳚ । अ॒ग्नये॒ जुष्टं॒ निर्व॑पा॒मीत्या॑ह । अ॒ग्नय॑ ए॒वैना॒ञ्जुष्टं॒ निर्व॑पति । त्रिर्यजु॑षा । त्रय॑ इ॒मे लो॒काः । ए॒षां ॅलो॒काना॒माप्त्यै᳚ । तू॒ष्णीं च॑तु॒र्थम् । अप॑रिमितमे॒वाव॑रुन्धे । स ए॒वमे॒वानु॑पू॒र्वꣳ ह॒वीꣳषि॒ निर्व॑पति । \textbf{ 33} \newline
                  \newline
                                \textbf{ TB 3.2.4.7} \newline
                  इ॒दं दे॒वाना॑मि॒दमु॑ नः स॒हेत्या॑ह॒ व्यावृ॑त्त्यै ॥ स्फा॒त्यै त्वा॒ नारा᳚त्या॒ इत्या॑ह॒ गुप्त्यै᳚ । तम॑सीव॒ वा ए॒षो᳚ऽन्तश्च॑रति । यः प॑री॒णहि॑ ॥ सुव॑र॒भि विख्ये॑षं ॅवैश्वान॒रं ज्योति॒रित्या॑ह । सुव॑रे॒वाभि वि प॑श्यति वैश्वान॒रं ज्योतिः॑ । द्यावा॑पृथि॒वी ह॒विषि॑ गृही॒त उद॑वेपेताम् ॥ दृꣳह॑न्तां॒ दुर्या॒ द्यावा॑पृथि॒व्योरित्या॑ह । गृ॒हाणां॒ द्यावा॑पृथि॒व्योर्-द्धृत्यै᳚ ॥ उ॒र्व॑न्तरि॑क्ष॒-मन्वि॒हीत्या॑ह॒ गत्यै᳚ ( ) ॥ अदि॑त्यास्त्वो॒पस्थे॑ सादया॒मीत्या॑ह । इ॒यं ॅवा अदि॑तिः । अ॒स्या ए॒वैन॑दु॒पस्थे॑ सादयति ॥ अग्ने॑ ह॒व्यꣳ र॑क्ष॒स्वेत्या॑ह॒ गुप्त्यै᳚ । \textbf{ 34} \newline
                  \newline
                                    (य॒ज्ञो वा आपो॒ - धाम॑ प्र॒णीय॒ प्रच॑र - त्यती॒या - दे॒तद् - बा॒हुभ्या॒मित्या॑ह - ह॒विꣳषि॒ निर्व॑पति॒ - गत्यै॑ च॒त्वारि॑ च) \textbf{(A4)} \newline \newline
                \textbf{ 3.2.5     अनुवाकं   5 - व्रीह्यवघातः} \newline
                                \textbf{ TB 3.2.5.1} \newline
                  इन्द्रो॑ वृ॒त्रम॑हन्न् । सो॑ऽपः । अ॒भ्य॑म्रियत । तासां॒ ॅयन्मेद्ध्यं॑ ॅय॒ज्ञियꣳ॒॒ सदे॑व॒मासी᳚त् । तदपोद॑क्रामत् । ते द॒र्भा अ॑भवन्न् । यद्-द॒र्भैर॒प उ॑त्पु॒नाति॑ । या ए॒व मेद्ध्या॑ य॒ज्ञियाः॒ सदे॑वा॒ आपः॑ । ताभि॑रे॒वैना॒ उत्पु॑नाति । द्वाभ्या॒मुत्पु॑नाति \textbf{ 35} \newline
                  \newline
                                \textbf{ TB 3.2.5.2} \newline
                  द्वि॒पाद्-यज॑मानः॒ प्रति॑ष्ठित्यै ॥ दे॒वो वः॑ सवि॒तोत्-पु॑ना॒त्वित्या॑ह । स॒वि॒तृप्र॑सूत ए॒वैना॒ उत्पु॑नाति । अच्छि॑द्रेण प॒वित्रे॒णेत्या॑ह । अ॒सौ वा आ॑दि॒त्योऽच्छि॑द्रं प॒वित्र᳚म् । तेनै॒वैना॒ उत्पु॑नाति । वसोः॒ सूर्य॑स्य र॒श्मिभि॒रित्या॑ह । प्रा॒णा वा आपः॑ । प्रा॒णा वस॑वः । प्रा॒णा र॒श्मयः॑ \textbf{ 36} \newline
                  \newline
                                \textbf{ TB 3.2.5.3} \newline
                  प्रा॒णैरे॒व प्रा॒णान्थ् संपृ॑णक्ति । सा॒वि॒त्रि॒यर्चा । स॒वि॒तृप्र॑सूतं मे॒ कर्मा॑स॒दिति॑ । स॒वि॒तृप्र॑सूतमे॒वास्य॒ कर्म॑ भवति । प॒च्छो गा॑यत्रि॒या त्रि॑ष्षमृद्ध॒त्वाय॑ ॥ आपो॑ देवीरग्रेपुवो अग्रेगुव॒ इत्या॑ह । रू॒प-मे॒वासा॑मे॒तन्-म॑हि॒मानं॒ ॅव्याच॑ष्टे । अग्र॑ इ॒मं ॅय॒ज्ञ्ं न॑य॒ताग्रे॑ य॒ज्ञ्प॑ति॒मित्या॑ह । अग्रे॑ ए॒व य॒ज्ञ्ं न॑यन्ति । अग्रे॑ य॒ज्ञ्प॑तिम् \textbf{ 37} \newline
                  \newline
                                \textbf{ TB 3.2.5.4} \newline
                  यु॒ष्मानिन्द्रो॑ ऽवृणीत वृत्र॒तूये॑ यू॒यमिन्द्र॑मवृणीद्ध्वं ॅवृत्र॒तूर्य॒ इत्या॑ह । वृ॒त्रꣳ ह॑ हनि॒ष्यन्निन्द्र॒ आपो॑ वव्रे । आपो॒ हेन्द्रं॑ ॅवव्रिरे । स॒ज्ञां-मे॒वासा॑मे॒तथ्-सामा॑नं॒ ॅव्याच॑ष्टे ॥ प्रोक्षि॑ताः॒ स्थेत्या॑ह । तेनापः॒ प्रोक्षि॑ताः ॥ अ॒ग्नये॑ वो॒ जुष्टं॒ प्रोक्षा᳚-म्य॒ग्नीषोमा᳚भ्या॒-मित्या॑ह । य॒था॒ दे॒व॒तमे॒वैना॒न् प्रोक्ष॑ति । त्रिः प्रोक्ष॑ति । त्र्या॑वृ॒द्धि य॒ज्ञ्ः \textbf{ 38} \newline
                  \newline
                                \textbf{ TB 3.2.5.5} \newline
                  अथो॒ रक्ष॑सा॒मप॑हत्यै । शुन्ध॑द्ध्वं॒ दैव्या॑य॒ कर्म॑णे देवय॒ज्याया॒ इत्या॑ह । दे॒व॒य॒ज्याया॑ ए॒वैना॑नि शुन्धति । त्रिः प्रोक्ष॑ति । त्र्या॑वृ॒द्धि य॒ज्ञ्ः । अथो॑ मेद्ध्य॒त्वाय॑ । अव॑धूतꣳ॒॒ रक्षोऽव॑धूता॒ अरा॑तय॒ इत्या॑ह । रक्ष॑सा॒-मप॑हत्यै । अदि॑त्या॒स्त्वग॒सीत्या॑ह । इ॒यं ॅवा अदि॑तिः \textbf{ 39} \newline
                  \newline
                                \textbf{ TB 3.2.5.6} \newline
                  अ॒स्या ए॒वैन॒त् त्वचं॑ करोति । प्रति॑ त्वा पृथि॒वी वे॒त्त्वित्या॑ह॒ प्रति॑ष्ठित्यै ॥ पु॒रस्ता᳚त् प्रती॒चीन॑ग्रीव॒-मुत्त॑रलो॒मोप॑स्तृणाति मेद्ध्य॒त्वाय॑ । तस्मा᳚त् पु॒रस्ता᳚त् प्र॒त्यञ्चः॑ प॒शवो॒ मेध॒मुप॑तिष्ठन्ते । तस्मा᳚त् प्र॒जा मृ॒गं ग्राहु॑काः । य॒ज्ञो दे॒वेभ्यो॒ निला॑यत । कृष्णो॑ रू॒पं कृ॒त्वा । यत् कृ॑ष्णाजि॒ने ह॒विर॑द्ध्यव॒हन्ति॑ । य॒ज्ञादे॒व-तद्-य॒ज्ञ्ं प्रयु॑ङ्क्ते । ह॒विषोऽस्क॑न्दाय । \textbf{ 40} \newline
                  \newline
                                \textbf{ TB 3.2.5.7} \newline
                  अ॒धि॒षव॑णमसि वानस्प॒त्यमित्या॑ह । अ॒धि॒षव॑ण-मे॒वैन॑त् करोति । प्रति॒ त्वाऽदि॑त्या॒स्त्वग्वे॒त्त्वित्या॑ह सय॒त्वाय॑ ॥ अ॒ग्नेस्त॒नू-र॒सीत्या॑ह । अ॒ग्नेर्वा ए॒षा त॒नूः । यदोष॑धयः । वा॒चो वि॒सर्ज॑न॒मित्या॑ह । य॒दा हि प्र॒जा ओष॑धीना-म॒श्नन्ति॑ । अथ॒ वाचं॒ ॅविसृ॑जन्ते । दे॒ववी॑तये त्वा गृह्णा॒मीत्या॑ह \textbf{ 41} \newline
                  \newline
                                \textbf{ TB 3.2.5.8} \newline
                  दे॒वता॑भिरे॒वैन॒थ् सम॑र्द्धयति ॥ अद्रि॑रसि वानस्प॒त्य इत्या॑ह । ग्रावा॑ण-मे॒वैन॑त् करोति । स इ॒दं दे॒वेभ्यो॑ ह॒व्यꣳ सु॒शमि॑ शमि॒ष्वेत्या॑ह॒ शान्त्यै᳚ ॥ हवि॑ष्कृ॒देहीत्या॑ह । य ए॒व दे॒वानाꣳ॑ हवि॒ष्कृतः॑ । तान्. ह्व॑यति । त्रिर्ह्व॑यति । त्रिष॑त्या॒ हि दे॒वाः ॥ इष॒मा व॒दोर्ज॒मा व॒देत्या॑ह \textbf{ 42} \newline
                  \newline
                                \textbf{ TB 3.2.5.9} \newline
                  इष॑मे॒वोर्जं॒ ॅयज॑माने दधाति । द्यु॒मद्व॑दत व॒यꣳ स॑घां॒तं जे॒ष्मेत्या॑ह॒ भ्रातृ॑व्याभिभूत्यै ॥ मनोः᳚ श्र॒द्धादे॑वस्य॒ यज॑मानस्यासुर॒घ्नी वाक् । य॒ज्ञा॒यु॒धेषु॒ प्रवि॑ष्टाऽऽसीत् । तेऽसु॑रा॒ याव॑न्तो यज्ञायु॒धाना॑मु॒द्वद॑ता-मु॒पाशृ॑ण्वन्न् । ते परा॑ऽभवन्न् । तस्मा॒थ् स्वानां॒ मद्ध्ये॑ऽव॒साय॑ यजेत । याव॑न्तोऽस्य॒ भ्रातृ॑व्या यज्ञायु॒धाना॑-मु॒द्वद॑ता-मुपशृ॒ण्वन्ति॑ । ते परा॑ भवन्ति । उ॒च्चैः स॒माह॑न्त॒वा आ॑ह॒ विजि॑त्यै \textbf{ 43} \newline
                  \newline
                                \textbf{ TB 3.2.5.10} \newline
                  वृ॒ङ्क्त ए॑षामिन्द्रि॒यं ॅवी॒र्य᳚म् । श्रेष्ठ॑ एषां भवति ॥ व॒र्॒.ष वृ॑द्धमसि॒ प्रति॑ त्वा व॒र्॒.षवृ॑द्धं ॅवे॒त्त्वित्या॑ह । व॒र्॒.षवृ॑द्धा॒ वा ओष॑धयः । व॒र्॒.षवृ॑द्धा इ॒षीकाः॒ समृ॑द्ध्यै । य॒ज्ञ्ꣳ रक्षाꣳ॒॒स्यनु॒ प्रावि॑शन्न् । तान्य॒स्ना प॒शुभ्यो॑ नि॒रवा॑दयन्त । तुषै॒रोष॑धीभ्यः ॥ परा॑पूतꣳ॒॒ रक्षः॒ परा॑पूता॒ अरा॑तय॒ इत्या॑ह । रक्ष॑सा॒मप॑हत्यै \textbf{ 44} \newline
                  \newline
                                \textbf{ TB 3.2.5.11} \newline
                  रक्ष॑सां भा॒गो॑ऽसीत्या॑ह । तुषै॑रे॒व रक्षाꣳ॑सि नि॒रव॑दयते । अ॒प उप॑स्पृशति मेद्ध्य॒त्वाय॑ । वा॒युर्वो॒ वि वि॑न॒क्त्वित्या॑ह । प॒वित्रं॒ ॅवै वा॒युः । पु॒नात्ये॒वैनान्॑ । अ॒न्तरि॑क्षादिव॒ वा ए॒ते प्रस्क॑न्दन्ति । ये शूर्पा᳚त् । दे॒वो वः॑ सवि॒ता हिर॑ण्यपाणिः॒ प्रति॑गृह्णा॒त्वित्या॑ह॒ प्रति॑ष्ठित्यै । ह॒विषोऽस्क॑न्दाय ( ) । त्रिष्फ॒लीक॑र्त॒वा आ॑ह । त्र्या॑वृ॒द्धि य॒ज्ञ्ः । अथो॑ मेद्ध्य॒त्वाय॑ । \textbf{ 45} \newline
                  \newline
                                    (द्वाभ्या॒मुत् पु॑नाति - र॒श्मयो॑ - नय॒न्त्यग्रे॑ य॒ज्ञ्प॑तिं - ॅय॒ज्ञो - ऽदि॑ति॒ - रस्क॑न्दाय - गृह्णा॒मीत्या॑ह - व॒देत्या॑ह॒ - विजि॑त्या॒ - अप॑हत्या॒ - अस्क॑न्दाय॒ त्रीणि॑ च) \textbf{(A5)} \newline \newline
                \textbf{ 3.2.6     अनुवाकं   6 - तण्डुलपेषणम्} \newline
                                \textbf{ TB 3.2.6.1} \newline
                  अव॑धूतꣳ॒॒ रक्षोऽव॑धूता॒ अरा॑तय॒ इत्या॑ह । रक्ष॑सा॒मप॑हत्यै । अदि॑त्या॒स्त्वग॒सीत्या॑ह । इ॒यं ॅवा अदि॑तिः । अ॒स्या ए॒वैन॒त् त्वचं॑ करोति । प्रति॑ त्वा पृथि॒वी वे॒त्त्वित्या॑ह॒ प्रति॑ष्ठित्यै । पु॒रस्ता᳚त् प्रती॒चीन॑ग्रीव॒-मुत्त॑रलो॒मोप॑स्तृणाति मेद्ध्य॒त्वाय॑ । तस्मा᳚त् पु॒रस्ता᳚त् प्र॒त्यञ्चः॑ प॒शवो॒ मेध॒मुप॑तिष्ठन्ते । तस्मा᳚त् प्र॒जा मृ॒गं ग्राहु॑काः । य॒ज्ञो दे॒वेभ्यो॒ निला॑यत \textbf{ 46} \newline
                  \newline
                                \textbf{ TB 3.2.6.2} \newline
                  कृष्णो॑ रू॒पं कृ॒त्वा । यत् कृ॑ष्णाजि॒ने ह॒विर॑धिपि॒नष्टि॑ । य॒ज्ञादे॒व तद् य॒ज्ञ्ं प्रयु॑ङ्क्ते । ह॒विषोऽस्क॑न्दाय ॥ द्यावा॑पृथि॒वी स॒हास्ता᳚म् । ते श॑म्यामा॒त्रमेक॒मह॒र्व्यैताꣳ॑ शम्यामा॒त्रमेक॒महः॑ ॥ दि॒वः स्क॑म्भ॒निर॑सि॒ प्रति॒ त्वा ऽदि॑त्या॒स्त्वग्वे॒त्त्वित्या॑ह । द्यावा॑पृथि॒व्योर् वीत्यै᳚ । धि॒षणा॑ऽसि पर्व॒त्या प्रति॑ त्वा दि॒वः स्क॑म्भ॒निर्वे॒त्त्वित्या॑ह । द्यावा॑पृथि॒व्योर् विधृ॑त्यै । \textbf{ 47} \newline
                  \newline
                                \textbf{ TB 3.2.6.3} \newline
                  धि॒षणा॑ऽसि पार्वते॒यी प्रति॑ त्वा पर्व॒तिर्वे॒त्त्वित्या॑ह । द्यावा॑पृथि॒व्योर्द्धृत्यै᳚ ॥ दे॒वस्य॑ त्वा सवि॒तुः प्र॑स॒व इत्या॑ह॒ प्रसू᳚त्यै । अ॒श्विनो᳚र् बा॒हुभ्या॒मित्या॑ह । अ॒श्विनौ॒ हि दे॒वाना॑मद्ध्व॒र्यू आस्ता᳚म् । पू॒ष्णो हस्ता᳚भ्या॒मित्या॑ह॒ यत्यै᳚ । अधि॑वपा॒मीत्या॑ह । य॒था॒दे॒व॒त-मे॒वैना॒नधि॑ वपति ॥ धा॒न्य॑मसि धिनु॒हि दे॒वानित्या॑ह । ए॒तस्य॒ यजु॑षो वी॒र्ये॑ण \textbf{ 48} \newline
                  \newline
                                \textbf{ TB 3.2.6.4} \newline
                  याव॒देका॑ दे॒वता॑ का॒मय॑ते॒ याव॒देका᳚ । ताव॒दाहु॑तिः प्रथते । न हि तदस्ति॑ । यत्ताव॑दे॒व स्यात् । याव॑ज्जु॒होति॑ ॥ प्रा॒णाय॑ त्वाऽपा॒नाय॒ त्वेत्या॑ह । प्रा॒णाने॒व यज॑माने दधाति । दी॒र्घामनु॒ प्रसि॑ति॒मायु॑षे धा॒मित्या॑ह । आयु॑रे॒वास्मि॑न् दधाति । अ॒न्तरि॑क्षादिव॒ वा ए॒तानि॒ प्रस्क॑न्दन्ति ( ) । यानि॑ दृ॒षदः॑ । दे॒वो वः॑ सवि॒ता हिर॑ण्यपाणिः॒ प्रति॑ गृह्णा॒त्वित्या॑ह॒ प्रति॑ष्ठित्यै । ह॒विषोऽस्क॑न्दाय ॥ अस॑म्ॅवपन्ती पिꣳ॑षा॒णूनि॑ कुरुता॒दित्या॑ह मेद्ध्य॒त्वाय॑ । \textbf{ 49} \newline
                  \newline
                                    (निला॑यत॒ - विधृ॑त्यै - वी॒र्ये॑ण - स्कन्दन्ति च॒त्वारि॑ च ) \textbf{(A6)} \newline \newline
                \textbf{ 3.2.7     अनुवाकं   7 - कपालोपधानम्} \newline
                                \textbf{ TB 3.2.7.1} \newline
                  धृष्टि॑रसि॒ ब्रह्म॑ य॒च्छेत्या॑ह॒ धृत्यै᳚ ॥ अपा᳚ग्ने॒ ऽग्निमा॒मादं॑ जहि॒ निष्क्र॒व्यादꣳ॑ से॒धा दे॑व॒यजं॑ ॅव॒हेत्या॑ह । य ए॒वामात्क्र॒व्यात् । तम॑प॒हत्य॑ । मेद्ध्ये॒ऽग्नौ क॒पाल॒मुप॑दधाति ॥ निर्द॑ग्धꣳ॒॒ रक्षो॒ निर्द॑ग्धा॒ अरा॑तय॒ इत्या॑ह । रक्षाꣳ॑स्ये॒व निर्द॑हति । अ॒ग्नि॒वत्युप॑दधाति । अ॒स्मिन्ने॒व लो॒के ज्योति॑र्द्धत्ते । अङ्गा॑र॒मधि॑-वर्तयति \textbf{ 50} \newline
                  \newline
                                \textbf{ TB 3.2.7.2} \newline
                  अ॒न्तरि॑क्ष ए॒व ज्योति॑र्द्धत्ते । आ॒दि॒त्यमे॒वामुष्मि॑न् ॅलो॒के ज्योति॑र्द्धत्ते । ज्योति॑ष्मन्तोऽस्मा इ॒मे लो॒का भ॑वन्ति । य ए॒वं ॅवेद॑ ॥ ध्रु॒वम॑सि पृथि॒वीं दृꣳ॒॒हेत्या॑ह । पृ॒थि॒वीमे॒वैतेन॑ दृꣳहति । ध॒र्त्रम॑स्य॒न्तरि॑क्षं दृꣳ॒॒हेत्या॑ह । अ॒न्तरि॑क्षमे॒वैतेन॑ दृꣳहति । ध॒रुण॑मसि॒ दिवं॑ दृꣳ॒॒हेत्या॑ह । दिव॑मे॒वैतेन॑ दृꣳहति \textbf{ 51} \newline
                  \newline
                                \textbf{ TB 3.2.7.3} \newline
                  धर्मा॑सि॒ दिशो॑ दृꣳ॒॒हेत्या॑ह । दिश॑ ए॒वैतेन॑ दृꣳहति । इ॒माने॒वैतैर् लो॒कान् दृꣳ॑हति । दृꣳ ह॑न्तेऽस्मा इ॒मे लो॒काः प्र॒जया॑ प॒शुभिः॑ । य ए॒वं ॅवेद॑ ॥ त्रीण्यग्रे॑ क॒पाला॒न्युप॑दधाति । त्रय॑ इ॒मे लो॒काः । ए॒षां ॅलो॒काना॒माप्त्यै᳚ । एक॒मग्रे॑ क॒पाल॒मुप॑दधाति । एकं॒ ॅवा अग्रे॑ क॒पालं॒ पुरु॑षस्य स॒भंव॑ति \textbf{ 52} \newline
                  \newline
                                \textbf{ TB 3.2.7.4} \newline
                  अथ॒ द्वे । अथ॒ त्रीणि॑ । अथ॑ च॒त्वारि॑ । अथा॒ष्टौ । तस्मा॑द॒ष्टाक॑पालं॒ पुरु॑षस्य॒ शिरः॑ । यदे॒वं क॒पाला᳚न्युप॒ दधा॑ति । य॒ज्ञो वै प्र॒जाप॑तिः । य॒ज्ञ्मे॒व प्र॒जाप॑तिꣳ॒॒ सꣳस्क॑रोति । आ॒त्मान॑मे॒व तथ् सꣳस्क॑रोति । तꣳ सꣳस्कृ॑तमा॒त्मान᳚म् \textbf{ 53} \newline
                  \newline
                                \textbf{ TB 3.2.7.5} \newline
                  अ॒मुष्मि॑न् ॅलो॒केऽनु॒परै॑ति ॥ यद॒ष्टावु॑प॒ दधा॑ति । गा॒य॒त्रि॒या तथ् संमि॑तम् । यन्नव॑ । त्रि॒वृता॒ तत् । यद् दश॑ । वि॒राजा॒ तत् । यदेका॑दश । त्रि॒ष्टुभा॒ तत् । यद्द्वाद॑श \textbf{ 54} \newline
                  \newline
                                \textbf{ TB 3.2.7.6} \newline
                  जग॑त्या॒ तत् । छन्दः॑ संमितानि॒ स उ॑प॒दध॑त् क॒पाला॑नि । इ॒मान् ॅलो॒कान॑नुपू॒र्वं दिशो॒ विधृ॑त्यै दृꣳहति । अथायुः॑ प्रा॒णान् प्र॒जां प॒शून्. यज॑माने दधाति । स॒जा॒तान॑स्मा अ॒भितो॑ बहु॒लान् क॑रोति ॥ चितः॒ स्थेत्या॑ह । य॒था॒ य॒जुरे॒वैतत् ॥ भृगू॑णा॒मङ्गि॑रसां॒ तप॑सा तप्यद्ध्व॒मित्या॑ह । दे॒वता॑ना-मे॒वैना॑नि॒ तप॑सा तपति । तानि॒ ततः॒ सꣳस्थि॑ते ( ) । यानि॑ घ॒र्मे क॒पाला᳚न् युपचि॒न्वन्ति॑ वे॒धस॒ इति॒ चतु॑ष्पदय॒र्चा विमु॑ञ्चति । चतु॑ष्पादः प॒शवः॑ । प॒शुष्वे॒वोपरि॑ष्टा॒त् प्रति॑तिष्ठति । \textbf{ 55} \newline
                  \newline
                                    (व॒र्त॒य॒ति॒ - दिव॑मे॒वैतेन॑ दृꣳहति - स॒म्भव॑ति॒ - तꣳ सꣳस्कृ॑तमा॒त्मानं॒ - द्वाद॑श॒ - सꣳस्थि॑ते॒ त्रीणि॑ च) \textbf{(A7)} \newline \newline
                \textbf{ 3.2.8     अनुवाकं   8 - पुरोडाशानिष्पादनम्} \newline
                                \textbf{ TB 3.2.8.1} \newline
                  दे॒वस्य॑ त्वा सवि॒तुः प्र॑स॒व इत्या॑ह॒ प्रसू᳚त्यै । अ॒श्विनो᳚र् बा॒हुभ्या॒मित्या॑ह । अ॒श्विनौ॒ हि दे॒वाना॑मद्ध्व॒र्यू आस्ता᳚म् । पू॒ष्णो हस्ता᳚भ्या॒मित्या॑ह॒ यत्यै᳚ । सम्ॅव॑पा॒मीत्या॑ह । य॒था॒ दे॒व॒तमे॒वैना॑नि॒ सम्ॅव॑पति ॥ समापो॑ अ॒द्भिर॑ग्मत॒ समोष॑धयो॒ रसे॒नेत्या॑ह । आपो॒ वा ओष॑धीर्-जिन्वन्ति । ओष॑धयो॒ऽपो जि॑न्वन्ति । अ॒न्या वा ए॒तासा॑म॒न्या जि॑न्वन्ति \textbf{ 56} \newline
                  \newline
                                \textbf{ TB 3.2.8.2} \newline
                  तस्मा॑दे॒वमा॑ह । सꣳ रे॒वती॒र्-जग॑तीभि॒र्-मधु॑मती॒र्-मधु॑मतीभिः सृज्यद्ध्व॒मित्या॑ह । आपो॒ वै रे॒वतीः᳚ । प॒शवो॒ जग॑तीः । ओष॑धयो॒ मधु॑मतीः । आप॒ ओष॑धीः प॒शून् । ताने॒वास्मा॑ एक॒धा सꣳ॒॒सृज्य॑ । मधु॑मतः करोति । अ॒द्भ्यः परि॒ प्रजा॑ताः स्थ॒ सम॒द्भिः पृ॑च्यद्ध्व॒मिति॑ प॒र्याप्ला॑वयति । यथा॒ सुवृ॑ष्ट इ॒माम॑नु वि॒सृत्य॑ \textbf{ 57} \newline
                  \newline
                                \textbf{ TB 3.2.8.3} \newline
                  आप॒ ओष॑धीर्म॒हय॑न्ति । ता॒दृगे॒व तत् ॥ जन॑यत्यै त्वा॒ सम्ॅयौ॒मीत्या॑ह । प्र॒जा ए॒वैतेन॑ दाधार । अ॒ग्नये᳚ त्वा॒ऽग्नीषोमा᳚भ्या॒मित्या॑ह॒ व्यावृ॑त्त्यै । म॒खस्य॒ शिरो॒ऽसीत्या॑ह । य॒ज्ञो वै म॒खः । तस्यै॒तच्छिरः॑ । यत् पु॑रो॒डाशः॑ । तस्मा॑दे॒वमा॑ह \textbf{ 58} \newline
                  \newline
                                \textbf{ TB 3.2.8.4} \newline
                  घ॒र्मो॑ऽसि वि॒श्वायु॒रित्या॑ह । विश्व॑मे॒वायु॒र् यज॑माने दधाति । उ॒रु प्र॑थस्वो॒रु ते॑ य॒ज्ञ्प॑तिः प्रथता॒मित्या॑ह । यज॑मानमे॒व प्र॒जया॑ प॒शुभिः॑ प्रथयति । त्वचं॑ गृह्णी॒ष्वेत्या॑ह । सर्व॑मे॒वैनꣳ॒॒ सत॑नुं करोति । अथा॒प आ॒नीय॒ परि॑ मार्ष्टि । माꣳ॒॒स ए॒व तत् त्वचं॑ दधाति । तस्मा᳚त् त्व॒चा माꣳ॒॒सं छ॒न्नम् ॥ घ॒र्मो वा ए॒षोऽशा᳚न्तः \textbf{ 59} \newline
                  \newline
                                \textbf{ TB 3.2.8.5} \newline
                  अ॒र्द्ध॒मा॒से᳚ऽर्द्धमासे॒ प्रवृ॑ज्यते । यत् पु॑रो॒डाशः॑ । स ई᳚श्व॒रो यज॑मानꣳ शु॒चा प्र॒दहः॑ । पर्य॑ग्नि करोति । प॒शुमे॒वैन॑मकः । शान्त्या॒ अप्र॑दाहाय । त्रिः पर्य॑ग्नि करोति । त्र्या॑वृ॒द्धि य॒ज्ञ्ः । अथो॒ रक्ष॑सा॒मप॑हत्यै ॥ अ॒न्तरि॑तꣳ॒॒ रक्षो॒ऽन्तरि॑ता॒ अरा॑तय॒ इत्या॑ह \textbf{ 60} \newline
                  \newline
                                \textbf{ TB 3.2.8.6} \newline
                  रक्ष॑सा-म॒न्तर्.हि॑त्यै । पु॒रो॒डाशं॒ ॅवा अधि॑श्रितꣳ॒॒ रक्षाꣳ॑स्यजिघाꣳसन्न् । दि॒वि नाको॒ नामा॒ग्नी र॑क्षो॒हा । स ए॒वास्मा॒द्-रक्षाꣳ॒॒स्यपा॑हन्न् । दे॒वस्त्वा॑ सवि॒ता श्र॑पय॒त्वित्या॑ह । स॒वि॒तृप्र॑सूत ए॒वैनꣳ॑ श्रपयति । वर्.षि॑ष्ठे॒ अधि॒ नाक॒ इत्या॑ह । रक्ष॑सा॒मप॑हत्यै । अ॒ग्निस्ते॑ त॒नुवं॒ माऽति॑ धा॒गित्या॒हान॑तिदाहाय । अग्ने॑ ह॒व्यꣳ र॑क्ष॒स्वेत्या॑ह॒ गुप्त्यै᳚ \textbf{ 61} \newline
                  \newline
                                \textbf{ TB 3.2.8.7} \newline
                  अवि॑दहन्तः श्रपय॒तेति॒ वाचं॒ ॅविसृ॑जते । य॒ज्ञ्मे॒व ह॒वीꣳष्य॑भि-व्या॒हृत्य॒ प्रत॑नुते । पु॒रो॒रुच॒मवि॑दाहाय॒ शृत्यै॑ करोति ॥ म॒स्तिष्को॒ वै पु॑रो॒डाशः॑ । तं ॅयन्नाभि॑वा॒सये᳚त् । आ॒विर्म॒स्तिष्कः॑ स्यात् । अ॒भिवा॑सयति । तस्मा॒द्-गुहा॑ म॒स्तिष्कः॑ । भस्म॑ना॒ऽभि वा॑सयति । तस्मा᳚न् माꣳ॒॒सेनास्थि॑ छ॒न्नम् \textbf{ 62} \newline
                  \newline
                                \textbf{ TB 3.2.8.8} \newline
                  वे॒देना॒भि वा॑सयति । तस्मा॒त् केशैः॒ शिरः॑ छ॒न्नम् । अख॑लतिभावुको भवति । य ए॒वं ॅवेद॑ ॥ प॒शोर्वै प्र॑ति॒मा पु॑रो॒डाशः॑ । स नाय॒जुष्क॑-मभि॒वास्यः॑ । वृथे॑व स्यात् । ई॒श्व॒रा यज॑मानस्य प॒शवः॒ प्रमे॑तोः ॥ सं ब्रह्म॑णा पृच्य॒स्वेत्या॑ह । प्रा॒णा वै ब्रह्म॑ \textbf{ 63} \newline
                  \newline
                                \textbf{ TB 3.2.8.9} \newline
                  प्रा॒णाः प॒शवः॑ । प्रा॒णैरे॒व प॒शून्थ् संपृ॑णक्ति । न प्र॒मायु॑का भवन्ति । यज॑मानो॒ वै पु॑रो॒डाशः॑ । प्र॒जा प॒शवः॒ पुरी॑षम् । यदे॒वम॑भि वा॒सय॑ति । यज॑मानमे॒व प्र॒जया॑ प॒शुभिः॒ सम॑र्द्धयति ॥ दे॒वा वै ह॒विर् भृ॒त्वाऽब्रु॑वन्न् । कस्मि॑न्नि॒दं म्र॑क्ष्यामह॒ इति॑ । सो᳚ऽग्निर॑ब्रवीत् \textbf{ 64} \newline
                  \newline
                                \textbf{ TB 3.2.8.10} \newline
                  मयि॑ त॒नूः संनिध॑द्ध्वम् । अ॒हं ॅव॒स्तं ज॑नयिष्यामि । यस्मि॑न् म्र॒क्ष्यद्ध्व॒ इति॑ । ते दे॒वा अ॒ग्नौ त॒नूः सं न्य॑दधत । तस्मा॑दाहुः । अ॒ग्निः सर्वा॑ दे॒वता॒ इति॑ । सोऽङ्गा॑रेणा॒पः । अ॒भ्य॑पातयत् । तत॑ एक॒तो॑ऽजायत । स द्वि॒तीय॑म॒भ्य॑पातयत् \textbf{ 65} \newline
                  \newline
                                \textbf{ TB 3.2.8.11} \newline
                  ततो᳚ द्वि॒तो॑ऽजायत । स तृ॒तीय॑-म॒भ्य॑पातयत् । तत॑-स्त्रि॒तो॑ऽजायत । यद॒द्भ्योऽजा॑यन्त । तदा॒प्याना॑माप्य॒त्वम् । यदा॒त्मभ्यो ऽजा॑यन्त । तदा॒त्म्याना॑-मात्म्य॒त्वम् ॥ ते दे॒वा आ॒प्येष्व॑मृजत । आ॒प्या अ॑मृजत॒ सूर्या᳚भ्युदिते । सूर्या᳚भ्युदितः॒ सूर्या॑भिनिम्रुक्ते \textbf{ 66} \newline
                  \newline
                                \textbf{ TB 3.2.8.12} \newline
                  सूर्या॑भिनिम्रुक्तः कुन॒खिनि॑ । कु॒न॒खी श्या॒वद॑ति । श्या॒वद॑न्नग्रदिधि॒षौ । अ॒ग्र॒दि॒धि॒षुः प॑रिवि॒त्ते । प॒रि॒वि॒त्तो वी॑र॒हणि॑ । वी॒र॒हा ब्र॑ह्म॒हणि॑ । तद्ब्र॑ह्म॒हणं॒ नात्य॑च्यवत । अ॒न्त॒र्वे॒दि निन॑य॒त्यव॑रुद्ध्यै । उल्मु॑केना॒भि गृ॑ह्णाति शृत॒त्वाय॑ । शृ॒तका॑मा इव॒ हि दे॒वाः ( ) । \textbf{ 67} \newline
                  \newline
                                    (अ॒न्या जि॑न्वन्-त्यनुवि॒सृत्यै॒-वमा॒-हाशा᳚न्त-आह॒-गुप्त्यै॑-छ॒न्नं - ब्रह्मा᳚-ब्रवीद्-द्वि॒तीय॑म॒भ्य॑पातय॒थ्-सूर्या॑भिनिम्रुक्ते-दे॒वाः) \textbf{(A8)} \newline \newline
                \textbf{ 3.2.9     अनुवाकं   9 - वेदिकरणम्} \newline
                                \textbf{ TB 3.2.9.1} \newline
                  दे॒वस्य॑ त्वा सवि॒तुः प्र॑स॒व इति॒ स्फ्यमाद॑त्ते॒ प्रसू᳚त्यै । अ॒श्विनो᳚र्-बा॒हुभ्या॒मित्या॑ह । अ॒श्विनौ॒ हि दे॒वाना॑मद्ध्व॒र्यू आस्ता᳚म् । पू॒ष्णो हस्ता᳚भ्या॒मित्या॑ह॒ यत्यै᳚ ॥ आद॑द॒ इन्द्र॑स्य बा॒हुर॑सि॒ दक्षि॑ण॒ इत्या॑ह । इ॒न्द्रि॒यमे॒व यज॑माने दधाति । स॒हस्र॑भृष्टिः श॒तते॑जा॒ इत्या॑ह । रू॒पमे॒वास्यै॒तन्-म॑हि॒मानं॒ ॅव्याच॑ष्टे ॥ वा॒युर॑सि ति॒ग्मते॑जा॒ इत्या॑ह । तेजो॒ वै वा॒युः \textbf{ 68} \newline
                  \newline
                                \textbf{ TB 3.2.9.2} \newline
                  तेज॑ ए॒वास्मि॑न् दधाति ॥ वि॒षाद्वै नामा॑सु॒र आ॑सीत् । सो॑ऽबिभेत् । य॒ज्ञेन॑ मा दे॒वा अ॒भिभ॑विष्य॒न्तीति॑ । स पृ॑थि॒वीम॒भ्य॑वमीत् । सा ऽमे॒द्ध्या ऽभ॑वत् । अथो॒ यदिन्द्रो॑ वृ॒त्रमहन्न्॑ । तस्य॒ लोहि॑तं पृथि॒वीमनु॒ व्य॑धावत् । सा ऽमे॒द्ध्या ऽभ॑वत् । पृथि॑वि देवयज॒नीत्या॑ह \textbf{ 69} \newline
                  \newline
                                \textbf{ TB 3.2.9.3} \newline
                  मेद्ध्या॑मे॒वैनां᳚ देव॒यज॑नीं करोति । ओष॑द्ध्यास्ते॒ मूलं॒ मा हिꣳ॑सिष॒मित्या॑ह । ओष॑धीना॒-महिꣳ॑सायै ॥ व्र॒जं ग॑च्छ गो॒स्थान॒मित्या॑ह । छन्दाꣳ॑सि॒ वै व्र॒जो गो॒स्थानः॑ । छन्दाꣳ॑स्ये॒वास्मै᳚ व्र॒जं गो॒स्थानं॑ करोति ॥ वर्.ष॑तु ते॒ द्यौरित्या॑ह । वृष्टि॒र्वै द्यौः । वृष्टि॑मे॒वाव॑रुन्धे ॥ ब॒धा॒न दे॑व सवितः पर॒मस्यां᳚ परा॒वतीत्या॑ह \textbf{ 70} \newline
                  \newline
                                \textbf{ TB 3.2.9.4} \newline
                  द्वौ वाव पुरु॑षौ । यं चै॒व द्वेष्टि॑ । यश्चै॑नं॒ द्वेष्टि॑ । तावु॒भौ ब॑द्ध्नाति पर॒मस्यां᳚ परा॒वति॑ श॒तेन॒ पाशैः᳚ । यो᳚ऽस्मान् द्वेष्टि॒ यं च॑ व॒यं द्वि॒ष्मस्तमतो॒ मा मौ॒गित्या॒हानि॑म्रुक्त्यै ॥ अ॒ररु॒र्वै नामा॑सु॒र आ॑सीत् । स पृ॑थि॒व्यामुप॑म्लुप्तोऽशयत् । तं दे॒वा अप॑हतो॒ऽररुः॑ पृथि॒व्या इति॑ पृथि॒व्या अपा᳚घ्नन्न् । भ्रातृ॑व्यो॒ वा अ॒ररुः॑ । अप॑हतो॒ऽररुः॑ पृथि॒व्या इति॒ यदाह॑ \textbf{ 71} \newline
                  \newline
                                \textbf{ TB 3.2.9.5} \newline
                  भ्रातृ॑व्यमे॒व पृ॑थि॒व्या अप॑हन्ति । ते॑ऽमन्यन्त । दिवं॒ ॅवा अ॒यमि॒तः प॑तिष्य॒तीति॑ । तम॒ररु॑स्ते॒ दिवं॒ मा स्का॒निति॑ दि॒वः पर्य॑बाधन्त । भ्रातृ॑व्यो॒ वा अ॒ररुः॑ । अ॒ररु॑स्ते॒ दिवं॒ मा स्का॒निति॒ यदाह॑ । भ्रातृ॑व्यमे॒व दि॒वः परि॑बाधते । स्त॒म्ब॒ य॒जुर्. ह॑रति । पृ॒थि॒व्या ए॒व भ्रातृ॑व्य॒-मप॑हन्ति । द्वि॒तीयꣳ॑ हरति \textbf{ 72} \newline
                  \newline
                                \textbf{ TB 3.2.9.6} \newline
                  अ॒न्तरि॑क्षादे॒वैन॒-मप॑हन्ति । तृ॒तीयꣳ॑ हरति । दि॒व ए॒वैन॒-मप॑हन्ति । तू॒ष्णीं च॑तु॒र्थꣳ ह॑रति । अप॑रिमितादे॒वैन॒-मप॑हन्ति ॥ असु॑राणां॒ ॅवा इ॒यमग्र॑ आसीत् । याव॒दासी॑नः परा॒ पश्य॑ति । ताव॑द्-दे॒वाना᳚म् । ते दे॒वा अ॑ब्रुवन्न् । अस्त्वे॒व नो॒ऽस्यामपीति॑ \textbf{ 73} \newline
                  \newline
                                \textbf{ TB 3.2.9.7} \newline
                  क्य॑न्नो दास्य॒थेति॑ । याव॑थ्स्व॒यं प॑रिगृह्णी॒थेति॑ । ते वस॑व॒स्त्वेति॑ दक्षिण॒तः पर्य॑गृह्णन्न् । रु॒द्रास्त्वेति॑ प॒श्चात् । आ॒दि॒त्या-स्त्वेत्यु॑त्तर॒तः । ते᳚ऽग्निना॒ प्राञ्चो॑ऽजयन्न् । वसु॑भिर् दक्षि॒णा । रु॒द्रैः प्र॒त्यञ्चः॑ । आ॒दि॒त्यैरुद॑ञ्चः । यस्यै॒वं ॅवि॒दुषो॒ वेदिं॑ परि गृ॒ह्णन्ति॑ \textbf{ 74} \newline
                  \newline
                                \textbf{ TB 3.2.9.8} \newline
                  भव॑त्या॒त्मना᳚ । परा᳚ऽस्य॒ भ्रातृ॑व्यो भवति ॥ दे॒वस्य॑ सवि॒तुः स॒व इत्या॑ह॒ प्रसू᳚त्यै । कर्म॑ कृण्वन्ति वे॒धस॒ इत्या॑ह । इ॒षि॒तꣳ हि कर्म॑ क्रि॒यते᳚ । पृ॒थि॒व्यै मेद्ध्यं॑ चामे॒द्ध्यं च॒ व्युद॑क्रामताम् । प्रा॒चीन॑-मुदी॒चीनं॒ मेद्ध्य᳚म् । प्र॒ती॒चीनं॑ दक्षि॒णा ऽमे॒द्ध्यम् । प्राची॒मुदी॑चीं प्रव॒णां क॑रोति । मेद्ध्या॑मे॒वैनां᳚ देव॒यज॑नीं करोति \textbf{ 75} \newline
                  \newline
                                \textbf{ TB 3.2.9.9} \newline
                  प्राञ्चौ॑ वेद्यꣳ॒॒ सावुन्न॑यति । आ॒ह॒व॒नीय॑स्य॒ परि॑गृहीत्यै । प्र॒तीची॒ श्रोणी᳚ । गार्.ह॑पत्यस्य॒ परि॑गृहीत्यै । अथो॑ मिथुन॒त्वाय॑ । उद्ध॑न्ति । यदे॒वास्या॑ अमे॒द्ध्यम् । तदप॑हन्ति । उद्ध॑न्ति । तस्मा॒दोष॑धयः॒ परा॑भवन्ति \textbf{ 76} \newline
                  \newline
                                \textbf{ TB 3.2.9.10} \newline
                  मूलं॑ छिनत्ति । भ्रातृ॑व्यस्यै॒व मूलं॑ छिनत्ति । मूलं॒ ॅवा अ॑ति॒तिष्ठ॒द्-रक्षाꣳ॒॒स्यनूत् पि॑पते । यद्धस्ते॑न छि॒न्द्यात् । कु॒न॒खिनीः᳚ प्र॒जाः स्युः॑ । स्फ्येन॑ छिनत्ति । वज्रो॒ वै स्फ्यः । वज्रे॑णै॒व य॒ज्ञाद्-रक्षाꣳ॒॒स्यप॑हन्ति । पि॒तृ॒दे॒व॒त्या ऽति॑खाता । इय॑तीं खनति \textbf{ 77} \newline
                  \newline
                                \textbf{ TB 3.2.9.11} \newline
                  प्र॒जाप॑तिना यज्ञ्मु॒खेन॒ संमि॑ताम् ॥ वेदि॑र्दे॒वेभ्यो॒ निला॑यत । तां च॑तुरङ्गु॒ले ऽन्व॑विन्दन्न् । तस्मा᳚च्चतुरङ्गु॒लं खेया᳚ । च॒तु॒रङ्गु॒लं ख॑नति । च॒तु॒र॒ङ्गु॒ले ह्योष॑धयः प्रति॒तिष्ठ॑न्ति । आप्र॑ति॒ष्ठायै॑ खनति । यज॑मानमे॒व प्र॑ति॒ष्ठां ग॑मयति । द॒क्षि॒ण॒तो वर्.षी॑यसीं करोति । दे॒व॒यज॑नस्यै॒व रू॒पम॑कः \textbf{ 78} \newline
                  \newline
                                \textbf{ TB 3.2.9.12} \newline
                  पुरी॑षवतीं करोति । प्र॒जा वै प॒शवः॒ पुरी॑षम् । प्र॒जयै॒वैनं॑ प॒शुभिः॒ पुरी॑षवन्तं करोति । उत्त॑रं परिग्रा॒हं परि॑गृह्णाति । ए॒ताव॑ती॒ वै पृ॑थि॒वी । याव॑ती॒ वेदिः॑ । तस्या॑ ए॒ताव॑त ए॒व भ्रातृ॑व्यं नि॒र्भज्य॑ । आ॒त्मन॒ उत्त॑रं परिग्रा॒हं परि॑गृह्णाति । ऋ॒त-म॑स्यृत॒सद॑न-मस्यृत॒श्रीर॒सीत्या॑ह । य॒था॒ य॒जुरे॒वै तत् । \textbf{ 79} \newline
                  \newline
                                \textbf{ TB 3.2.9.13} \newline
                  क्रू॒रमि॑व॒ वा ए॒तत् क॑रोति । यद्वेदिं॑ क॒रोति॑ । धा अ॑सि स्व॒धा अ॒सीति॑ योयुप्यते॒ शान्त्यै᳚ । उ॒र्वी चासि॒ वस्वी॑ चा॒सीत्या॑ह । उ॒र्वीमे॒वैनां॒ ॅवस्वीं᳚ करोति । पु॒रा क्रू॒रस्य॑ वि॒सृपो॑ विरफ्शि॒न्नित्या॑ह मेद्ध्य॒त्वाय॑ । उ॒दा॒दाय॑ पृथि॒वीं जी॒रदा॑नु॒र्यामैर॑यं च॒न्द्रम॑सि स्व॒धाभि॒रित्या॑ह । यदे॒वास्या॑ अमे॒द्ध्यम् । तद॑प॒हत्य॑ । मेद्ध्यां᳚ देव॒यज॑नीं कृ॒त्वा \textbf{ 80} \newline
                  \newline
                                \textbf{ TB 3.2.9.14} \newline
                  यद॒दश्च॒न्द्रम॑सि॒ मेद्ध्य᳚म् । तद॒स्यामेर॑यति । तां धीरा॑सो अनु॒दृश्य॑ यजन्त॒ इत्या॒हानु॑ख्यात्यै ॥ प्रोक्ष॑णी॒रासा॑दय । इ॒द्ध्माब॒र्॒.हि-रुप॑सादय । स्रु॒वं च॒ स्रुच॑श्च॒ संमृ॑ड्ढि । पत्नीꣳ॒॒ संन॑ह्य । आज्ये॑नो॒देहीत्या॑हानुपू॒र्वता॑यै । प्रोक्ष॑णी॒रासा॑दयति । आपो॒ वै र॑क्षो॒घ्नीः \textbf{ 81} \newline
                  \newline
                                \textbf{ TB 3.2.9.15} \newline
                  रक्ष॑सा॒मप॑हत्यै । स्फ्यस्य॒ वर्त्मन्᳚थ्-सादयति । य॒ज्ञ्स्य॒ सन्त॑त्यै । उ॒वाच॒ हासि॑तो दैव॒लः । ए॒ताव॑ती॒र्वा अ॒मुष्मि॑न् ॅलो॒क आप॑ आसन्न् । याव॑तीः॒ प्रोक्ष॑णी॒रिति॑ । तस्मा᳚द् ब॒ह्वीरा॒साद्याः᳚ । स्फ्यमु॒दस्यन्न्॑ । यं द्वि॒ष्यात्तं ध्या॑येत् । शु॒चैवैन॑-मर्पयति ( ) । \textbf{ 82} \newline
                  \newline
                                    (वै वा॒यु - रा॑ह - परा॒वतीत्या॒हा - ह॑ - द्वि॒तीयꣳ॑ हर॒ - तीति॑ - परिगृ॒ह्णन्ति॑ - देव॒यज॑नीं करोति - भवन्ति - खनत्य - क - रे॒तत् - कृ॒त्वा - र॑क्षो॒घ्नी - र॑र्पयति) \textbf{(A9)} \newline \newline
                \textbf{ 3.2.10    अनुवाकं   10 - इध्माबर्हिः सादनम्} \newline
                                \textbf{ TB 3.2.10.1} \newline
                  वज्रो॒ वै स्फ्यः । यद॒न्वञ्चं॑ धा॒रये᳚त् । वज्रे᳚ऽद्ध्व॒र्युः क्ष॑ण्वीत । पु॒रस्ता᳚त् ति॒र्यञ्चं॑ धारयति । वज्रो॒ वै स्फ्यः । वज्रे॑णै॒व य॒ज्ञ्स्य॑ दक्षिण॒तो रक्षाꣳ॒॒स्यप॑हन्ति । अ॒ग्निभ्यां॒ प्राच॑श्च प्र॒तीच॑श्च । स्फ्येनोदी॑चश्चा-ध॒राच॑श्च । स्फ्येन॒ वा ए॒ष वज्रे॑णा॒स्यै पा॒प्मानं॒ भ्रातृ॑व्य-मप॒हत्य॑ । उ॒त्क॒रेऽधि॒ प्रवृ॑श्चति \textbf{ 83} \newline
                  \newline
                                \textbf{ TB 3.2.10.2} \newline
                  यथो॑प॒धाय॑ वृ॒श्चन्त्य॒वम् । हस्ता॒वव॑ नेनिक्ते । आ॒त्मान॑मे॒व प॑वयते । स्फ्यं प्रक्षा॑लयति मेद्ध्य॒त्वाय॑ । अथो॑ पा॒प्मन॑ ए॒व भ्रातृ॑व्यस्य न्य॒ङ्गं छि॑नत्ति ॥ इ॒द्ध्मा-ब॒र्॒.हि-रुप॑सादयति॒ युक्त्यै᳚ । य॒ज्ञ्स्य॑ मिथुन॒त्वाय॑ । अथो॑ पुरो॒रुच॑मे॒वैतां द॑धाति । उत्त॑रस्य॒ कर्म॒णोऽनु॑ख्यात्यै । न पु॒रस्ता᳚त् प्र॒त्यगुप॑सादयेत् \textbf{ 84} \newline
                  \newline
                                \textbf{ TB 3.2.10.3} \newline
                  यत् पु॒रस्ता᳚त् प्र॒त्य-गु॑पसा॒दये᳚त् । अ॒न्यत्रा॑हुतिप॒थादि॒द्ध्मं प्रति॑पादयेत् । प्र॒जा वै ब॒र्॒.हिः । अप॑राद्ध्नुयाद्-ब॒र्॒.हिषा᳚ प्र॒जानां᳚ प्र॒जन॑नम् । प॒श्चात्-प्रागुप॑सादयति । आ॒हु॒ति॒प॒थेने॒द्ध्मं प्रति॑पादयति । स॒प्रं॒त्ये॑व ब॒र्॒.हिषा᳚ प्र॒जानां᳚ प्र॒जन॑न॒मुपै॑ति । दक्षि॑णमि॒द्ध्मम् । उत्त॑रं ब॒र॒.हिः । आ॒त्मा वा इ॒द्ध्मः ( ) । प्र॒जा ब॒र्॒.हिः । प्र॒जा ह्या᳚त्मन॒ उत्त॑रतरा ती॒र्थे । ततो॒ मेध॑मुप॒नीय॑ । य॒था॒ दे॒व॒तमे॒वैन॒त्-प्रति॑ष्ठापयति । प्रति॑तिष्ठति प्र॒जया॑ प॒शुभि॒र् यज॑मानः । \textbf{ 85} \newline
                  \newline
                                    (वृ॒श्च॒ति॒ - सा॒द॒ये॒ - दि॒द्ध्मः पञ्च॑ च) \textbf{(A10)} \newline \newline
                \textbf{Prapaataka korvai with starting  Words of 1 to 10 Anuvaakams :-} \newline
        (तृतीय॑स्यां - दे॒वस्या᳚श्वप॒र्॒.शुं ॅयो वै - पू᳚र्वे॒द्युः - कर्म॑णे वा॒ - मिन्द्रो॑ वृ॒त्रम॑ह॒न्थ् सो॑ऽपो - ऽव॑धूतं॒ - धृष्टि॑र् - दे॒वस्येत्या॑ह॒ सम्ॅव॑पामि - दे॒वस्य॒ स्फ्यमाद॑दे॒ - वज्रो॒ वै स्फ्यो दश॑) \newline

        \textbf{korvai with starting Words of 1, 11, 21 Series of Dasinis :-} \newline
        (तृ॒तीय॑स्यां - ॅय॒ज्ञ्स्यान॑तिरेकाय - प॒वित्र॑वत्य - ध्व॒र्युं चा॑ - धि॒षव॑णमस्य॒ - न्तरि॑क्ष ए॒व - रक्ष॑साम॒न्तर्.हि॑त्यै॒ - द्वौ वाव पुरु॑षो॒ - यद॒दश्च॒न्द्रम॑सि॒ मेद्ध्यं॒ पञ्चाशी॑तिः) \newline

        \textbf{first and last  Word 3rd Ashtakam 2nd Prapaatakam :-} \newline
        (तृ॒तीय॑स्यां॒ - ॅयज॑मानः) \newline 

       

        ॥ हरिः॑ ॐ ॥
॥ कृष्ण यजुर्वेदीय तैत्तिरीय ब्राह्मणे तृतीयाष्टके द्वितीयः प्रपाठकः समाप्तः ॥ \newline
        \pagebreak
        
        
        
     \addcontentsline{toc}{section}{ 3.3     तृतीयाष्टके तृतीय प्रपाठकः - दर्.शपूर्णमासेष्टिब्राह्मणम्}
     \markright{ 3.3     तृतीयाष्टके तृतीय प्रपाठकः - दर्.शपूर्णमासेष्टिब्राह्मणम् \hfill https://www.vedavms.in \hfill}
     \section*{ 3.3     तृतीयाष्टके तृतीय प्रपाठकः - दर्.शपूर्णमासेष्टिब्राह्मणम् }
                \textbf{ 3.3.1     अनुवाकं   1 - स्त्रुख्संमार्गः} \newline
                                \textbf{ TB 3.3.1.1} \newline
                  प्रत्यु॑ष्टꣳ॒॒ रक्षः॒ प्रत्यु॑ष्टा॒ अरा॑तय॒ इत्या॑ह । रक्ष॑सा॒-मप॑हत्यै । अ॒ग्नेर्व॒स्तेजि॑ष्ठेन॒ तेज॑सा॒ निष्ट॑पा॒मीत्या॑ह मेद्ध्य॒त्वाय॑ । स्रुचः॒ संमा᳚र्ष्टि । स्रु॒वमग्रे᳚ । पुमाꣳ॑ समे॒वाभ्यः॒ सꣳश्य॑ति मिथुन॒त्वाय॑ । अथ॑ जु॒हूम् । अथो॑प॒भृत᳚म् । अथ॑ ध्रु॒वाम् । अ॒सौ वै जु॒हूः \textbf{ 1} \newline
                  \newline
                                \textbf{ TB 3.3.1.2} \newline
                  अ॒न्तरि॑क्षमुप॒भृत् । पृ॒थि॒वी ध्रु॒वा । इ॒मे वै लो॒काः स्रुचः॑ । वृष्टिः॑ स॒मांर्ज॑नानि । वृष्टि॒र्वा इ॒मान् ॅलो॒कान॑नुपू॒र्वं क॑ल्पयति । ते ततः॑ क्लृ॒प्ताः समे॑धन्ते । समे॑धन्तेऽस्मा इ॒मे लो॒काः प्र॒जया॑ प॒शुभिः॑ । य ए॒वं ॅवेद॑ ॥ यदि॑ का॒मये॑त॒ वर्.षु॑कः प॒र्जन्यः॑ स्या॒दिति॑ । अ॒ग्र॒तः संमृ॑ज्यात् \textbf{ 2} \newline
                  \newline
                                \textbf{ TB 3.3.1.3} \newline
                  वृष्टि॑मे॒व निय॑च्छति । अ॒वा॒चीना᳚ग्रा॒ हि वृष्टिः॑ । यदि॑ का॒मये॒ताव॑र्.षुकः स्या॒दिति॑ । मू॒ल॒तः संमृ॑ज्यात् । वृष्टि॑मे॒वोद्य॑च्छति । तदु॒ वा आ॑हुः । अ॒ग्र॒त ए॒वोपरि॑ष्टा॒थ् संमृ॑ज्यात् । मू॒ल॒तो॑ऽधस्ता᳚त् । तद॑नुपू॒र्वं क॑ल्पते । वर्.षु॑को भव॒तीति॑ । \textbf{ 3} \newline
                  \newline
                                \textbf{ TB 3.3.1.4} \newline
                  प्राची॑मभ्या॒कार᳚म् । अग्रै॑रन्तर॒तः । ए॒वमि॑व॒ ह्यन्न॑म॒द्यते᳚ । अथो॒ अग्रा॒द्वा ओष॑धीना॒मूर्जं॑ प्र॒जा उप॑जीवन्ति । ऊ॒र्ज ए॒वा-न्नाद्य॒स्याव॑रुद्ध्यै । अ॒धस्ता᳚त् प्र॒तीची᳚म् । द॒ण्डमु॑त्तम॒तः । मूले॑न॒ मूलं॒ प्रति॑ष्ठित्यै । तस्मा॑दर॒त्नौ प्राञ्च्यु॒परि॑ष्टा॒ल् लोमा॑नि । प्र॒त्यञ्च्य॒धस्ता᳚त् \textbf{ 4} \newline
                  \newline
                                \textbf{ TB 3.3.1.5} \newline
                  स्रुग्घ्ये॑षा ॥ प्रा॒णो वै स्रु॒वः । जु॒हूर्दक्षि॑णो॒ हस्तः॑ । उ॒प॒भृथ्स॒व्यः । आ॒त्मा ध्रु॒वा । अन्नꣳ॑ स॒मांर्ज॑नानि । मु॒ख॒तो वै प्रा॒णो॑ऽपा॒नो भू॒त्वा । आ॒त्मान॒मन्नं॑ प्र॒विश्य॑ । बा॒ह्य॒तस्त॒नुवꣳ॑ शुभयति । तस्मा᳚थ् स्रु॒वमे॒वाग्रे॒ संमा᳚र्ष्टि ( ) ॥ मु॒ख॒तो हि प्रा॒णो॑ऽपा॒नो भू॒त्वा । आ॒त्मान॒मन्न॑-मावि॒शति॑ । तौ प्रा॑णापा॒नौ । अव्य॑र्द्धुकः प्राणापा॒नाभ्यां᳚ भवति । य ए॒वं ॅवेद॑ । \textbf{ 5} \newline
                  \newline
                                    (जु॒हुर् - मृ॑ज्याद् - भव॒तीति॑ - प्र॒त्यञ्च्य॒धस्ता᳚न् - मार्ष्टि॒ पञ्च॑ च) \textbf{(A1)} \newline \newline
                \textbf{ 3.3.2     अनुवाकं   2 - सम्मार्जनानामग्नौ प्रहरणम्} \newline
                                \textbf{ TB 3.3.2.1} \newline
                  दि॒वः शिल्प॒मव॑ततम् । पृ॒थि॒व्याः क॒कुभि॑ श्रि॒तम् । तेन॑ व॒यꣳ स॒हस्र॑वल्.शेन । स॒पत्नं॑ नाशयामसि॒ स्वाहेति॑ स्रुख्स॒मांर्ज॑नान्य॒ग्नौ प्रह॑रति । आपो॒ वै द॒र्भाः । रू॒पमे॒वैषा॑मे॒तन् म॑हि॒मानं॒ ॅव्याच॑ष्टे ॥ अ॒नु॒ष्टुभ॒र्चा । आनु॑ष्टुभः प्र॒जाप॑तिः । प्रा॒जा॒प॒त्यो वे॒दः । वे॒दस्याग्रꣳ॑ स्रुख्स॒मांर्ज॑नानि \textbf{ 6} \newline
                  \newline
                                \textbf{ TB 3.3.2.2} \newline
                  स्वेनै॒वैना॑नि॒ छन्द॑सा । स्वया॑ दे॒वत॑या॒ सम॑र्द्धयति । अथो॒ ऋग्वाव योषा᳚ । द॒र्भो वृषा᳚ । तन्मि॑थु॒नम् । मि॒थु॒नमे॒वास्य॒ तद्-य॒ज्ञे क॑रोति प्र॒जन॑नाय । प्रजा॑यते प्र॒जया॑ प॒शुभि॒र्-यज॑मानः ॥ तान्येके॒ वृथै॒वापा᳚स्यन्ति । तत्तथा॒ न का᳚र्यम् । आर॑ब्धस्य य॒ज्ञिय॑स्य॒ कर्म॑णः॒ स वि॑दो॒हः \textbf{ 7} \newline
                  \newline
                                \textbf{ TB 3.3.2.3} \newline
                  यद्ये॑नानि प॒शवो॑ऽभि॒तिष्ठे॑युः । न तत् प॒शुभ्यः॒ कम् । अ॒द्भिर्मा᳚र्-जयि॒त्वोत्क॒रे न्य॑स्येत् । यद्वै य॒ज्ञिय॑स्य॒ कर्म॑णो॒ऽन्यत्राहु॑तीभ्यः स॒न्तिष्ठ॑ते । उ॒त्क॒रो वाव तस्य॑ प्रति॒ष्ठा । ए॒ताꣳ हि तस्मै᳚ प्रति॒ष्ठां दे॒वाः स॒मभ॑रन्न् । यद॒द्भिर्मा॒र्जय॑ति । तेन॑ शा॒न्तम् । यदु॑त्क॒रे न्य॒स्यति॑ । प्र॒ति॒ष्ठामे॒वैना॑नि॒ तद्-ग॑मयति \textbf{ 8} \newline
                  \newline
                                \textbf{ TB 3.3.2.4} \newline
                  प्रति॑तिष्ठति प्र॒जया॑ प॒शुभि॒र्यज॑मानः ॥ अथो᳚ स्त॒म्बस्य॒ वा ए॒तद्-रू॒पम् । यथ् स्रु॑ख्स॒मांर्ज॑नानि । स्त॒म्ब॒शो वा ओष॑धयः । तासां᳚ जरत्क॒क्षे प॒शवो॒ न र॑मन्ते । अप्रि॑यो॒ ह्ये॑षां जरत्क॒क्षः । याव॑दप्रियो ह॒ वै ज॑रत्क॒क्षः प॑शू॒नाम् । ताव॑दप्रियः पशू॒नां भ॑वति । यस्यै॒तान्य॒न्यत्रा॒ग्नेर्-दध॑ति । न॒व॒दाव्या॑सु॒ वा ओष॑धीषु प॒शवो॑ रमन्ते \textbf{ 9} \newline
                  \newline
                                \textbf{ TB 3.3.2.5} \newline
                  न॒व॒दा॒वो ह्ये॑षां प्रि॒यः । याव॑त्प्रियो ह॒ वै न॑वदा॒वः प॑शू॒नाम् । ताव॑त्प्रियः पशू॒नां भ॑वति । यस्यै॒तान्य॒ग्नौ प्र॒हर॑न्ति । तस्मा॑दे॒तान्य॒ग्नावे॒व प्रह॑रेत् । य॒त॒रस्मिन्᳚थ्-संमृ॒ज्यात् । प॒शू॒नां धृत्यै᳚ ॥ यो भू॒ताना॒मधि॑पतिः । रु॒द्रस्त॑न्तिच॒रो वृषा᳚ । प॒शून॒स्माकं॒ मा हिꣳ॑सीः ( ) । ए॒तद॑स्तु हु॒तं तव॒ स्वाहेत्य॑ग्नि स॒मांर्ज॑नान्य॒ग्नौ प्रह॑रति । ए॒षा वा ए॒तेषां॒ ॅयोनिः॑ । ए॒षा प्र॑ति॒ष्ठा । स्वामे॒वैना॑नि॒ योनि᳚म् । स्वां प्र॑ति॒ष्ठां ग॑मयति । प्रति॑तिष्ठति प्र॒जया॑ प॒शुभि॒र्यज॑मानः । \textbf{ 10} \newline
                  \newline
                                    (वे॒दस्याग्रꣳ॑ स्रुख्स॒मांर्ज॑नानि - विदो॒हो - ग॑मयति - प॒शवो॑ रमन्ते - हिꣳसीः॒ षट्च॑) \textbf{(A2)} \newline \newline
                \textbf{ 3.3.3     अनुवाकं   3 - पत्नीसंनहनम्} \newline
                                \textbf{ TB 3.3.3.1} \newline
                  अय॑ज्ञो॒ वा ए॒षः । यो॑ऽप॒त्नीकः॑ । न प्र॒जाः प्रजा॑येरन्न् । पत्न्यन्वा᳚स्ते । य॒ज्ञ्मे॒वाकः॑ । प्र॒जानां᳚ प्र॒जन॑नाय । यत् तिष्ठ॑न्ती स॒नंह्ये॑त । प्रि॒यं ज्ञा॒तिꣳ रु॑न्ध्यात् । आसी॑ना॒ संन॑ह्यते । आसी॑ना॒ ह्ये॑षा वी॒र्यं॑ क॒रोति॑ \textbf{ 11} \newline
                  \newline
                                \textbf{ TB 3.3.3.2} \newline
                  यत् प॒श्चात्-प्राच्य॒न्वासी॑त । अ॒नया॑ स॒मदं॑ दधीत । दे॒वानां॒ पत्नि॑या स॒मदं॑ दधीत । देशा᳚द्-दक्षिण॒त उदी॒च्यन्वा᳚स्ते । आ॒त्मनो॑ गोपी॒थाय॑ ॥ आ॒शासा॑ना सौमन॒समित्या॑ह । मेद्ध्या॑मे॒वैनां॒ केव॑लीं कृ॒त्वा । आ॒शिषा॒ सम॑र्द्धयति । अ॒ग्नेरनु॑व्रता भू॒त्वा संन॑ह्ये सुकृ॒ताय॒ कमित्या॑ह । ए॒तद्वै पत्नि॑यै व्रतोप॒नय॑नम् \textbf{ 12} \newline
                  \newline
                                \textbf{ TB 3.3.3.3} \newline
                  तेनै॒वैनां᳚ ॅव्र॒तमुप॑नयति । तस्मा॑दाहुः । यश्चै॒वं ॅवेद॒ यश्च॒न । योक्त्र॑मे॒व यु॑ते । यम॒न्वास्ते᳚ । तस्या॒मुष्मि॑न् ॅलो॒के भ॑व॒तीति॒ योक्त्रे॑ण । यद् योक्त्र᳚म् । स योगः॑ । यदास्ते᳚ । स क्षेमः॑ \textbf{ 13} \newline
                  \newline
                                \textbf{ TB 3.3.3.4} \newline
                  यो॒ग॒क्षे॒मस्य॒ क्लृप्त्यै᳚ ॥ यु॒क्तं क्रि॑याता आ॒शीः कामे॑ युज्याता॒ इति॑ । आ॒शिषः॒ समृ॑द्ध्यै । ग्र॒न्थिं ग्र॑थ्नाति । आ॒शिष॑ ए॒वास्यां॒ परि॑गृह्णाति । पुमा॒न्॒. वै ग्र॒न्थिः । स्त्री पत्नी᳚ । तन् मि॑थु॒नम् । मि॒थु॒नमे॒वास्य॒ तद्-य॒ज्ञे क॑रोति प्र॒जन॑नाय । प्रजा॑यते प्र॒जया॑ प॒शुभि॒र् यज॑मानः \textbf{ 14} \newline
                  \newline
                                \textbf{ TB 3.3.3.5} \newline
                  अथो॑ अ॒र्द्धो वा ए॒ष आ॒त्मनः॑ । यत् पत्नी᳚ । य॒ज्ञ्स्य॒ धृत्या॒ अशि॑थिलंभावाय ॥ सु॒प्र॒जस॑स्त्वा व॒यꣳ सु॒पत्नी॒रुप॑सेदि॒मेत्या॑ह । य॒ज्ञ्मे॒व तन् मि॑थु॒नी क॑रोति । ऊ॒नेऽति॑रिक्तं धीयाता॒ इति॒ प्रजा᳚त्यै ॥ म॒ही॒नां पयो॒ऽस्योष॑धीनाꣳ॒॒ रस॒ इत्या॑ह । रू॒पमे॒वास्यै॒-तन्म॑हि॒मानं॒ ॅव्याच॑ष्टे । तस्य॒ तेऽक्षी॑यमाणस्य॒ निर्व॑पामि देवय॒ज्याया॒ इत्या॑ह । आ॒शिष॑मे॒वैतामाशा᳚स्ते ( ) । \textbf{ 15} \newline
                  \newline
                                    (क॒रोति॑ - व्रतोप॒नय॑नं॒ - क्षेमो॒ - यज॑मानः - शास्ते) \textbf{(A3)} \newline \newline
                \textbf{ 3.3.4     अनुवाकं   4 - आज्योत्पवनम्} \newline
                                \textbf{ TB 3.3.4.1} \newline
                  घृ॒तं च॒ वै मधु॑ च प्र॒जाप॑तिरासीत् । यतो॒ मद्ध्वा॑सीत् । ततः॑ प्र॒जा अ॑सृजत । तस्मा॒न् मधु॑षि प्र॒जन॑नमिवास्ति । तस्मा॒न् मधु॑षा॒ न प्रच॑रन्ति । या॒तया॑म॒ हि । आज्ये॑न॒ प्रच॑रन्ति । य॒ज्ञो वा आज्य᳚म् । य॒ज्ञेनै॒व य॒ज्ञ्ं प्रच॑र॒न्त्यया॑तयामत्वाय । पत्न्यवे᳚क्षते \textbf{ 16} \newline
                  \newline
                                \textbf{ TB 3.3.4.2} \newline
                  मि॒थु॒न॒त्वाय॒ प्रजा᳚त्यै । यद्वै पत्नी॑ य॒ज्ञ्स्य॑ क॒रोति॑ । मि॒थु॒नं तत् । अथो॒ पत्नि॑या ए॒वैष य॒ज्ञ्स्या᳚न्वार॒म्भोऽन॑वच्छित्त्यै ॥ अ॒मे॒द्ध्यं ॅवा ए॒तत् क॑रोति । यत् पत्न्य॒वेक्ष॑ते । गार्.ह॑प॒त्ये ऽधि॑श्रयति मेद्ध्य॒त्वाय॑ । आ॒ह॒व॒नीय॑-म॒भ्युद्द्र॑वति । य॒ज्ञ्स्य॒ सन्त॑त्यै । तेजो॑ऽसि॒ तेजोऽनु॒ प्रेहीत्या॑ह \textbf{ 17} \newline
                  \newline
                                \textbf{ TB 3.3.4.3} \newline
                  तेजो॒ वा अ॒ग्निः । तेज॒ आज्य᳚म् । तेज॑सै॒व तेजः॒ सम॑र्द्धयति । अ॒ग्निस्ते॒ तेजो॒ मा विनै॒दित्या॒हाहिꣳ॑सायै । स्फ्यस्य॒ वर्त्मन्᳚थ्सादयति । य॒ज्ञ्स्य॒ सन्त॑त्यै । अ॒ग्नेर्जि॒ह्वाऽसि॑ सु॒भूर्दे॒वाना॒मित्या॑ह । य॒था॒ य॒जुरे॒वैतत् । धाम्ने॑धाम्ने दे॒वेभ्यो॒ यजु॑षेयजुषे भ॒वेत्या॑ह । आ॒शिष॑मे॒वैतामाशा᳚स्ते । \textbf{ 18} \newline
                  \newline
                                \textbf{ TB 3.3.4.4} \newline
                  तद्वा अतः॑ प॒वित्रा᳚भ्यामे॒वोत् पु॑नाति । यज॑मानो॒ वा आज्य᳚म् । प्रा॒णा॒पा॒नौ प॒वित्रे᳚ । यज॑मान ए॒व प्रा॑णापा॒नौ द॑धाति । पु॒न॒रा॒हार᳚म् । ए॒वमि॑व॒ हि प्रा॑णापा॒नौ स॒चंर॑तः । शु॒क्रम॑सि॒ ज्योति॑रसि॒ तेजो॒ऽसीत्या॑ह । रू॒पमे॒वास्यै॒ तन् म॑हि॒मानं॒ ॅव्याच॑ष्टे । त्रिर्यजु॑षा । त्रय॑ इ॒मे लो॒काः \textbf{ 19} \newline
                  \newline
                                \textbf{ TB 3.3.4.5} \newline
                  ए॒षां ॅलो॒काना॒माप्त्यै᳚ । त्रिः । त्र्या॑वृ॒द्धि य॒ज्ञ्ः । अथो॑ मेद्ध्य॒त्वाय॑ । अथाज्य॑वतीभ्याम॒पः । रू॒पमे॒वासा॑मे॒तद्-वर्णं॑ दधाति । अपि॒ वा उ॒ताहुः॑ । यथा॑ ह॒ वै योषा॑ सु॒वर्णꣳ॒॒ हिर॑ण्यं पेश॒लं बिभ्र॑ती रू॒पाण्यास्ते᳚ । ए॒वमे॒ता ए॒तर्.हीति॑ । आपो॒ वै सर्वा॑ दे॒वताः᳚ \textbf{ 20} \newline
                  \newline
                                \textbf{ TB 3.3.4.6} \newline
                  ए॒षा हि विश्वे॑षां दे॒वानां᳚ त॒नूः । यदाज्य᳚म् ॥ तत्रो॒भयो᳚र्मीमाꣳ॒॒सा । जा॒मि स्यात् । यद्-यजु॒षाऽऽज्यं॒ ॅयजु॑षा॒ऽप उ॑त्पुनी॒यात् । छन्द॑सा॒ऽप उत्पु॑ना॒त्यजा॑मित्वाय । अथो॑ मिथुन॒त्वाय॑ । सा॒वि॒त्रि॒यर्चा । स॒वि॒तृप्र॑सूतं मे॒ कर्मा॑स॒दिति॑ । स॒वि॒तृप्र॑सूतमे॒वास्य॒ कर्म॑ भवति ( ) । प॒च्छो गा॑यत्रि॒या त्रिः॑ षमृद्ध॒त्वाय॑ । अ॒द्भिरे॒वौष॑धीः॒ संन॑यति । ओष॑धीभिः प॒शून् । प॒शुभि॒र्यज॑मानम् । शु॒क्रं त्वा॑ शु॒क्रायां॒ ज्योति॑स्त्वा॒ ज्योति॑ष्य॒र्चिस्त्वा॒ ऽर्चिषीत्या॑ह सर्व॒त्वाय॑ । पर्या᳚प्त्या॒ अन॑न्तरायाय । \textbf{ 21} \newline
                  \newline
                                    (ई॒क्ष॒त॒- आ॒ह॒-शा॒स्ते॒-लो॒का - दे॒वता॑ - भवति॒ षट्च॑) \textbf{(A4)} \newline \newline
                \textbf{ 3.3.5     अनुवाकं   5 - आज्यग्रहणम्} \newline
                                \textbf{ TB 3.3.5.1} \newline
                  दे॒वा॒सु॒राः सम्ॅय॑त्ता आसन्न् । स ए॒तमिन्द्र॒ आज्य॑स्यावका॒श-म॑पश्यत् । तेनावै᳚क्षत । ततो॑ दे॒वा अभ॑वन्न् । पराऽसु॑राः । य ए॒वं ॅवि॒द्वानाज्य॑-म॒वेक्ष॑ते । भव॑त्या॒त्मना᳚ । परा᳚ऽस्य॒ भ्रातृ॑व्यो भवति । ब्र॒ह्म॒वा॒दिनो॑ वदन्ति । यदाज्ये॑ना॒न्यानि॑ ह॒वीꣳष्य॑-भिघा॒रय॑ति \textbf{ 22} \newline
                  \newline
                                \textbf{ TB 3.3.5.2} \newline
                  अथ॒ केनाज्य॒मिति॑ । स॒त्येनेति॑ ब्रूयात् । चक्षु॒र्वै स॒त्यम् । स॒त्येनै॒वैन॑द॒-भिघा॑रयति । ई॒श्व॒रो वा ए॒षो᳚ऽन्धो भवि॑तोः । यश्चक्षु॒षा ऽऽज्य॑म॒वेक्ष॑ते । नि॒मील्यावे᳚क्षेत । दा॒धारा॒त्मञ्चक्षुः॑ । अ॒भ्याज्यं॑ घारयति ॥ आज्यं॑ गृह्णाति \textbf{ 23} \newline
                  \newline
                                \textbf{ TB 3.3.5.3} \newline
                  छन्दाꣳ॑सि॒ वा आज्य᳚म् । छन्दाꣳ॑स्ये॒व प्री॑णाति । च॒तुर्जु॒ह्वां गृ॑ह्णाति । चतु॑ष्पादः प॒शवः॑ । प॒शूने॒वाव॑रुन्धे । अ॒ष्टावु॑प॒भृति॑ । अ॒ष्टाक्ष॑रा गाय॒त्री । गा॒य॒त्रः प्रा॒णः । प्रा॒णमे॒व प॒शुषु॑ दधाति । च॒तुर्द्ध्रु॒वाया᳚म् \textbf{ 24} \newline
                  \newline
                                \textbf{ TB 3.3.5.4} \newline
                  चतु॑ष्पादः प॒शवः॑ । प॒शुष्वे॒-वोपरि॑ष्टा॒त्-प्रति॑तिष्ठति । य॒ज॒मा॒न॒दे॒व॒त्या॑ वै जु॒हूः । भ्रा॒तृ॒व्य॒दे॒व॒त्यो॑प॒भृत् । च॒तुर्जु॒ह्वां गृ॒ह्णन्भूयो॑ गृह्णीयात् । अ॒ष्टावु॑प॒भृति॑ गृ॒ह्णन् कनी॑यः । यज॑मानायै॒व भ्रातृ॑व्य॒मुप॑स्तिं करोति ॥गौर्वै स्रुचः॑ । च॒तुर्जु॒ह्वां गृ॑ह्णाति । तस्मा॒च्चतु॑ष्पदी \textbf{ 25} \newline
                  \newline
                                \textbf{ TB 3.3.5.5} \newline
                  अ॒ष्टावु॑प॒भृति॑ । तस्मा॑द॒ष्टाश॑फा । च॒तुर्द्ध्रु॒वाया᳚म् । तस्मा॒च्चतुः॑ स्तना । गामे॒व तथ् सꣳस्क॑रोति । साऽस्मै॒ सꣳस्कृ॒तेष॒मूर्जं॑ दुहे । यज्जु॒ह्वां गृ॒ह्णाति॑ । प्र॒या॒जेभ्य॒स्तत् । यदु॑प॒भृति॑ । प्र॒या॒जा॒नू॒या॒जेभ्य॒स्तत् ( ) । सर्व॑स्मै॒ वा ए॒तद्-य॒ज्ञाय॑ गृह्यते । यद्ध्रु॒वाया॒माज्य᳚म् । \textbf{ 26} \newline
                  \newline
                                    (अ॒भि॒घा॒रय॑ति - गृह्णाति - ध्रु॒वायां॒ - चतु॑ष्पदी॒ - प्रयाजानूया॒जेभ्य॒स्तद् द्वे॒ च॑) \textbf{(A5)} \newline \newline
                \textbf{ 3.3.6     अनुवाकं   6 - स्त्रुख्सादनम्} \newline
                                \textbf{ TB 3.3.6.1} \newline
                  आपो॑ देवीरग्रेपुवो अग्रेगुव॒ इत्या॑ह । रू॒पमे॒वासा॑मे॒तन्-म॑हि॒मानं॒ ॅव्याच॑ष्टे । अग्र॑ इ॒मं ॅय॒ज्ञ्ं न॑य॒ताग्रे॑ य॒ज्ञ्प॑ति॒मित्या॑ह । अग्र॑ ए॒व य॒ज्ञ्ं न॑यन्ति । अग्रे॑ य॒ज्ञ्प॑तिम् । यु॒ष्मानिन्द्रो॑ऽवृणीत वृत्र॒तूर्ये॑ यू॒यमिन्द्र॑मवृणीद्ध्वं ॅवृत्र॒तूर्य॒ इत्या॑ह । वृ॒त्रꣳ ह॑ हनि॒ष्यन्निन्द्र॒ आपो॑ वव्रे । आपो॒ हेन्द्रं॑ ॅवव्रिरे । स॒ज्ञां-मे॒वासा॑-मे॒तथ्-सामा॑नं॒ ॅव्याच॑ष्टे । प्रोक्षि॑ताः॒ स्थेत्या॑ह \textbf{ 27} \newline
                  \newline
                                \textbf{ TB 3.3.6.2} \newline
                  तेनापः॒ प्रोक्षि॑ताः ॥ अ॒ग्निर् दे॒वेभ्यो॒ निला॑यत । कृष्णो॑ रू॒पं कृ॒त्वा । स वन॒स्पती॒न् प्रावि॑शत् । कृष्णो᳚ऽस्याखरे॒ष्ठो᳚ऽग्नये᳚ त्वा॒ स्वाहेत्या॑ह । अ॒ग्नय॑ ए॒वैनं॒ जुष्टं॑ करोति । अथो॑ अ॒ग्नेरे॒व मेध॒मव॑रुन्धे ॥ वेदि॑रसि ब॒र्.॒हिषे᳚ त्वा॒ स्वाहेत्या॑ह । प्र॒जा वै ब॒र्.॒हिः । पृ॒थि॒वी वेदिः॑ \textbf{ 28} \newline
                  \newline
                                \textbf{ TB 3.3.6.3} \newline
                  प्र॒जा ए॒व पृ॑थि॒व्यां प्रति॑ष्ठापयति । ब॒र्.॒हिर॑सि स्रु॒ग्भ्यस्त्वा॒ स्वाहेत्या॑ह । प्र॒जा वै ब॒र्.॒हिः । यज॑मानः॒ स्रुचः॑ । यज॑मानमे॒व प्र॒जासु॒ प्रति॑ष्ठापयति ॥ दि॒वे त्वा॒ऽन्तरि॑क्षाय त्वा पृथि॒व्यै त्वेति॑ ब॒र्.॒हिरा॒साद्य॒ प्रोक्ष॑ति । ए॒भ्य ए॒वैनं॑ ॅलो॒केभ्यः॒ प्रोक्ष॑ति । अथ॒ ततः॑ स॒ह स्रु॒चा पु॒रस्ता᳚त् प्र॒त्यञ्चं॑ ग्र॒न्थिं प्रत्यु॑क्षति । प्र॒जा वै ब॒र्.॒हिः । यथा॒ सूत्यै॑ का॒ल आपः॑ पु॒रस्ता॒द्-यन्ति॑ \textbf{ 29} \newline
                  \newline
                                \textbf{ TB 3.3.6.4} \newline
                  ता॒दृगे॒व तत् ॥ स्व॒धा पि॒तृभ्य॒ इत्या॑ह । स्व॒धा॒का॒रो हि पि॑तृ॒णाम् । ऊर्ग्भ॑व बर्.हि॒षद्भ्य॒ इति॒ दक्षि॑णायै॒ श्रोणे॒रोत्त॑रस्यै॒ निन॑यति॒ सन्त॑त्यै । मासा॒ वै पि॒तरो॑ बर्.हि॒षदः॑ । मासा॑ने॒व प्री॑णाति । मासा॒ वा ओष॑धीर्-व॒र्द्धय॑न्ति । मासाः᳚ पचन्ति॒ समृ॑द्ध्यै । अन॑तिस्कन्दन्. ह प॒र्जन्यो॑ वर्.षति । यत्रै॒तदे॒वं क्रि॒यते᳚ । \textbf{ 30} \newline
                  \newline
                                \textbf{ TB 3.3.6.5} \newline
                  ऊ॒र्जा पृ॑थि॒वीं ग॑च्छ॒तेत्या॑ह । पृ॒थि॒व्यामे॒वोर्जं॑ दधाति । तस्मा᳚त् पृथि॒व्या ऊ॒र्जा भु॑ञ्जते । ग्र॒न्थिं ॅविस्रꣳ॑सयति । प्रज॑नयत्ये॒व तत् । ऊ॒र्द्ध्वं प्राञ्च॒मुद्गू॑ढं प्र॒त्यञ्च॒माय॑च्छति । तस्मा᳚त् प्रा॒चीनꣳ॒॒ रेतो॑ धीयते । प्र॒तीचीः᳚ प्र॒जा जा॑यन्ते ॥ विष्णोः॒ स्तूपो॒ऽसीत्या॑ह । य॒ज्ञो वै विष्णुः॑ \textbf{ 31} \newline
                  \newline
                                \textbf{ TB 3.3.6.6} \newline
                  य॒ज्ञ्स्य॒ धृत्यै᳚ । पु॒रस्ता᳚त् प्रस्त॒रं गृ॑ह्णाति । मुख्य॑मे॒वैनं॑ करोति । इय॑न्तं गृह्णाति । प्र॒जाप॑तिना यज्ञ्मु॒खेन॒ संमि॑तम् । इय॑न्तं गृह्णाति । य॒ज्ञ्॒प॒रुषा॒ संमि॑तम् । इय॑न्तं गृह्णाति । ए॒ताव॒द्वै पुरु॑षे वी॒र्य᳚म् । वी॒र्य॑ संमितम् \textbf{ 32} \newline
                  \newline
                                \textbf{ TB 3.3.6.7} \newline
                  अप॑रिमितं गृह्णाति । अप॑रिमित॒स्याव॑रुद्ध्यै । तस्मि॑न् प॒वित्रे॒ अपि॑सृजति । यज॑मानो॒ वै प्र॑स्त॒रः । प्रा॒णा॒पा॒नौ प॒वित्रे᳚ । यज॑मान ए॒व प्रा॑णापा॒नौ द॑धाति ॥ ऊर्णा᳚म्रदसं त्वा स्तृणा॒मीत्या॑ह । य॒था॒य॒जुरे॒वै तत् । स्वा॒स॒स्थं दे॒वेभ्य॒ इत्या॑ह । दे॒वेभ्य॑ ए॒वैन॑थ् स्वास॒स्थं क॑रोति \textbf{ 33} \newline
                  \newline
                                \textbf{ TB 3.3.6.8} \newline
                  ब॒र्॒.॒हिः स्तृ॑णाति । प्र॒जा वै ब॒र्.॒हिः । पृ॒थि॒वी वेदिः॑ । प्र॒जा ए॒व पृ॑थि॒व्यां प्रति॑ष्ठापयति ॥ अन॑तिदृश्नꣳ स्तृणाति । प्र॒जयै॒वैनं॑ प॒शुभि॒रन॑तिदृश्नं करोति । धा॒रय॑न् प्रस्त॒रं प॑रि॒धीन् परि॑ दधाति । यज॑मानो॒ वै प्र॑स्त॒रः । यज॑मान ए॒व तथ् स्व॒यं प॑रि॒धीन् परि॑ दधाति ॥ ग॒न्ध॒र्वो॑ऽसि वि॒श्वाव॑सु॒रित्या॑ह \textbf{ 34} \newline
                  \newline
                                \textbf{ TB 3.3.6.9} \newline
                  विश्व॑मे॒वायु॒र्-यज॑माने दधाति । इन्द्र॑स्य बा॒हुर॑सि॒ दक्षि॑ण॒ इत्या॑ह । इ॒न्द्रि॒यमे॒व यज॑माने दधाति । मि॒त्रावरु॑णौ त्वोत्तर॒तः परि॑धत्ता॒मित्या॑ह । प्रा॒णा॒पा॒नौ मि॒त्रावरु॑णौ । प्रा॒णा॒पा॒ना-वे॒वास्मि॑न् दधाति । सूर्य॑स्त्वा पु॒रस्ता᳚त् पा॒त्वित्या॑ह । रक्ष॑सा॒-मप॑हत्यै । कस्या᳚श्चिद॒भिश॑स्त्या॒ इत्या॑ह । अप॑रिमितादे॒वैनं॑ पाति । \textbf{ 35} \newline
                  \newline
                                \textbf{ TB 3.3.6.10} \newline
                  वी॒तिहो᳚त्रं त्वा कव॒ इत्या॑ह । अ॒ग्निमे॒व हो॒त्रेण॒ सम॑र्द्धयति । द्यु॒मन्तꣳ॒॒ समि॑धीम॒हीत्या॑ह॒ समि॑द्ध्यै । अग्ने॑ बृ॒हन्त॑मद्ध्व॒र इत्या॑ह॒ वृद्ध्यै᳚ । वि॒शो य॒न्त्रे स्थ॒ इत्या॑ह । वि॒शां ॅयत्यै᳚ । उ॒दी॒चीना᳚ग्रे॒ निद॑धाति॒ प्रति॑ष्ठित्यै । वसू॑नाꣳ रु॒द्राणा॑-मादि॒त्यानाꣳ॒॒ सद॑सि सी॒देत्या॑ह । दे॒वता॑नामे॒व सद॑ने प्रस्त॒रꣳ सा॑दयति ॥ जु॒हूर॑सि घृ॒ताची॒ नाम्नेत्या॑ह \textbf{ 36} \newline
                  \newline
                                \textbf{ TB 3.3.6.11} \newline
                  अ॒सौ वै जु॒हूः । अ॒न्तरि॑क्षमुप॒भृत् । पृ॒थि॒वी ध्रु॒वा । तासा॑मे॒तदे॒व प्रि॒यं नाम॑ । यद्-घृ॒ताचीति॑ । यद्-घृ॒ताचीत्याह॑ । प्रि॒येणै॒वैना॒ नाम्ना॑ सादयति ॥ ए॒ता अ॑सदन्थ्-सुकृ॒तस्य॑ लो॒क इत्या॑ह । स॒त्यं ॅवै सु॑कृ॒तस्य॑ लो॒कः । स॒त्य ए॒वैनाः᳚ सुकृ॒तस्य॑ लो॒के सा॑दयति ( ) । ता वि॑ष्णो पा॒हीत्या॑ह । य॒ज्ञो वै विष्णुः॑ । य॒ज्ञ्स्य॒ धृत्यै᳚ । पा॒हि य॒ज्ञ्ं पा॒हि य॒ज्ञ्प॑तिं पा॒हि मां ॅय॑ज्ञ्॒निय॒मित्या॑ह । य॒ज्ञाय॒ यज॑मानाया॒त्मने᳚ । तेभ्य॑ ए॒वाशिष॒माशा॒स्तेऽना᳚र्त्यै । \textbf{ 37} \newline
                  \newline
                                    (स्थेत्या॑ह - पृथि॒वी वेदि॒र् - यन्ति॑ - क्रि॒यते॒ - विष्णु॑र् - वी॒र्य॑संमितं - करो - त्याह - पाति॒ - नाम्नेत्या॑ह - लो॒के सा॑दयति॒ षट्च॑) \textbf{(A6)} \newline \newline
                \textbf{ 3.3.7     अनुवाकं   7 - आघारः} \newline
                                \textbf{ TB 3.3.7.1} \newline
                  अ॒ग्निना॒ वै होत्रा᳚ । दे॒वा असु॑रा-न॒भ्य॑भवन्न् । अ॒ग्नये॑ समि॒द्ध्यमा॑ना॒या-नु॑ब्रू॒हीत्या॑ह॒ भ्रातृ॑व्याभिभूत्यै । एक॑विꣳशति-मिद्ध्मदा॒रूणि॑ भवन्ति । ए॒क॒विꣳ॒॒शो वै पुरु॑षः । पुरु॑ष॒स्याप्त्यै᳚ । पञ्च॑दशेद्ध्मदा॒रूण्य॒-भ्याद॑धाति । पञ्च॑दश॒ वा अ॑र्द्धमा॒सस्य॒ रात्र॑यः । अ॒र्द्ध॒मा॒स॒शः स॑म्ॅवथ्स॒र आ᳚प्यते । त्रीन्-प॑रि॒धीन्-परि॑-दधाति \textbf{ 38} \newline
                  \newline
                                \textbf{ TB 3.3.7.2} \newline
                  ऊ॒र्द्ध्वे स॒मिधा॒वाद॑धाति । अ॒नू॒या॒जेभ्यः॑ स॒मिध॒मति॑शिनष्टि । षट्थ्सं प॑द्यन्ते । षड्वा ऋ॒तवः॑ । ऋ॒तूने॒व प्री॑णाति । वे॒देनोप॑वाजयति । प्रा॒जा॒प॒त्यो वै वे॒दः । प्रा॒जा॒प॒त्यः प्रा॒णः । यज॑मान आहव॒नीयः॑ । यज॑मान ए॒व प्रा॒णं द॑धाति \textbf{ 39} \newline
                  \newline
                                \textbf{ TB 3.3.7.3} \newline
                  त्रिरुप॑वाजयति । त्रयो॒ वै प्रा॒णाः । प्रा॒णाने॒वास्मि॑न् दधाति । वे॒देनो॑प॒यत्य॑ स्रु॒वेण॑ प्राजाप॒त्य-मा॑घा॒र-माघा॑रयति । य॒ज्ञो वै प्र॒जाप॑तिः । य॒ज्ञ्मे॒व प्र॒जाप॑तिं मुख॒त आर॑भते । अथो᳚ प्र॒जाप॑तिः॒ सर्वा॑ दे॒वताः᳚ । सर्वा॑ ए॒व दे॒वताः᳚ प्रीणाति ॥ अ॒ग्निम॑ग्नीत् त्रिस्त्रिः॒ संमृ॒ड्ढीत्या॑ह । त्र्या॑वृ॒द्धि य॒ज्ञ्ः \textbf{ 40} \newline
                  \newline
                                \textbf{ TB 3.3.7.4} \newline
                  अथो॒ रक्ष॑सा॒मप॑हत्यै । प॒रि॒धीन्थ्-संमा᳚र्ष्टि । पु॒नात्ये॒वैनान्॑ । त्रिस्त्रिः॒ संमा᳚र्ष्टि । त्र्या॑वृ॒द्धि य॒ज्ञ्ः । अथो॑ मेद्ध्य॒त्वाय॑ । अथो॑ ए॒ते वै दे॑वाश्वाः । दे॒वा॒श्वाने॒व तथ् संमा᳚र्ष्टि । सु॒व॒र्गस्य॑ लो॒कस्य॒ सम॑ष्ट्यै । आसी॑नो॒ऽन्य-मा॑घा॒र-माघा॑रयति \textbf{ 41} \newline
                  \newline
                                \textbf{ TB 3.3.7.5} \newline
                  तिष्ठ॑न्न॒न्यम् । यथाऽनो॑ वा॒ रथ॑म्ॅवा यु॒ञ्ज्यात् । ए॒वमे॒व तद॑द्ध्व॒र्युर्-य॒ज्ञ्ं ॅयु॑नक्ति । सु॒व॒र्गस्य॑ लो॒कस्या॒भ्यू᳚ढ्यै । वह॑न्त्येनं ग्रा॒म्याः प॒शवः॑ । य ए॒वं ॅवेद॑ ॥ भुव॑नमसि॒ विप्र॑थ॒स्वेत्या॑ह । य॒ज्ञो वै भुव॑नम् । य॒ज्ञ् ए॒व यज॑मानं प्र॒जया॑ प॒शुभिः॑ प्रथयति । अग्ने॒ यष्ट॑रि॒दं नम॒ इत्या॑ह \textbf{ 42} \newline
                  \newline
                                \textbf{ TB 3.3.7.6} \newline
                  अ॒ग्निर्वै दे॒वानां॒ ॅयष्टा᳚ । य ए॒व दे॒वानां॒ ॅयष्टा᳚ । तस्मा॑ ए॒व नम॑स्करोति ॥ जुह्वेह्य॒ग्निस्त्वा᳚ ह्वयति देवय॒ज्याया॒ उप॑भृ॒देहि॑ दे॒वस्त्वा॑ सवि॒ता ह्व॑यति देवय॒ज्याया॒ इत्या॑ह । आ॒ग्ने॒यी वै जु॒हूः । सा॒वि॒त्र्यु॑प॒भृत् । ताभ्या॑मे॒वैने॒ प्रसू॑त॒ आद॑त्ते ॥ अग्ना॑विष्णू॒ मा वा॒मव॑क्रमिष॒मित्या॑ह । अ॒ग्निः पु॒रस्ता᳚त् । विष्णु॑र्य॒ज्ञ्ः प॒श्चात् \textbf{ 43} \newline
                  \newline
                                \textbf{ TB 3.3.7.7} \newline
                  ताभ्या॑मे॒व प्र॑ति॒ प्रोच्या॒त्याक्रा॑मति । विजि॑हाथां॒ मा मा॒ संता᳚प्त॒मित्या॒हाहिꣳ॑सायै । लो॒कं मे॑ लोककृतौ कृणुत॒मित्या॑ह । आ॒शिष॑मे॒वैतामाशा᳚स्ते ॥ विष्णोः॒ स्थान॑म॒सीत्या॑ह । य॒ज्ञो वै विष्णुः॑ । ए॒तत् खलु॒ वै दे॒वाना॒-मप॑राजित-मा॒यत॑नम् । यद्-य॒ज्ञ्ः । दे॒वाना॑मे॒वाप॑राजित आ॒यत॑ने तिष्ठति ॥ इ॒त इन्द्रो॑ अकृणोद्-वी॒र्या॑णीत्या॑ह \textbf{ 44} \newline
                  \newline
                                \textbf{ TB 3.3.7.8} \newline
                  इ॒न्द्रि॒यमे॒व यज॑माने दधाति । स॒मा॒रभ्यो॒र्द्ध्वो अ॑द्ध्व॒रो दि॑वि॒स्पृश॒मित्या॑ह॒ वृद्ध्यै᳚ । आ॒घा॒र-मा॑घा॒र्यमा॑ण॒मनु॑ समा॒रभ्य॑ । ए॒तस्मि॑न् का॒ले दे॒वाः सु॑व॒र्गं ॅलो॒कमा॑यन्न् । सा॒क्षादे॒व यज॑मानः सुव॒र्गं ॅलो॒कमे॑ति । अथो॒ समृ॑द्धेनै॒व य॒ज्ञेन॒ यज॑मानः सुव॒र्गं ॅलो॒कमे॑ति । अह्रु॑तो य॒ज्ञो य॒ज्ञ्प॑ते॒रित्या॒हाना᳚र्त्यै । इन्द्रा॑वा॒न्थ्-स्वाहेत्या॑ह । इ॒न्द्रि॒यमे॒व यज॑माने दधाति । बृ॒हद्भा इत्या॑ह \textbf{ 45} \newline
                  \newline
                                \textbf{ TB 3.3.7.9} \newline
                  सु॒व॒र्गो वै लो॒को बृ॒हद्भाः । सु॒व॒र्गस्य॑ लो॒कस्य॒ सम॑ष्ट्यै ॥ य॒ज॒मा॒न॒दे॒व॒त्या॑ वै जु॒हूः । भ्रा॒तृ॒व्य॒दे॒व॒त्यो॑प॒भृत् । प्रा॒ण आ॑घा॒रः । यथ् सꣳ॑स्प॒र्.शये᳚त् । भ्रातृ॑व्येऽस्य प्रा॒णं द॑द्ध्यात् । असꣳ॑स्पर्.शयन्न॒त्याक्रा॑मति । यज॑मान ए॒व प्रा॒णं द॑धाति । पा॒हिमा᳚ऽग्ने॒ दुश्च॑रिता॒दा मा॒ सुच॑रिते भ॒जेत्या॑ह \textbf{ 46} \newline
                  \newline
                                \textbf{ TB 3.3.7.10} \newline
                  अ॒ग्निर्वाव प॒वित्र᳚म् । वृ॒जि॒नमनृ॑तं॒ दुश्च॑रितम् । ऋ॒जु॒क॒र्मꣳ स॒त्यꣳ सुच॑रितम् । अ॒ग्निरे॒वैनं॑ ॅवृजि॒नाद-नृ॑ता॒द्-दुश्च॑रितात्पाति । ऋ॒जु॒क॒र्मे स॒त्ये सुच॑रिते भजति । तस्मा॑दे॒वमाशा᳚स्ते । आ॒त्मनो॑ गोपी॒थाय॑ ॥ शिरो॒ वा ए॒तद्-य॒ज्ञ्स्य॑ । यदा॑घा॒रः । आ॒त्मा ध्रु॒वा \textbf{ 47} \newline
                  \newline
                                \textbf{ TB 3.3.7.11} \newline
                  आ॒घा॒र-मा॒घार्य॑ ध्रु॒वाꣳ सम॑नक्ति । आ॒त्मन्ने॒व य॒ज्ञ्स्य॒ शिरः॒ प्रति॑दधाति । द्विः सम॑नक्ति । द्वौ हि प्रा॑णापा॒नौ । तदा॑हुः । त्रिरे॒व सम॑ञ्ज्यात् । त्रिधा॑तु॒ हि शिर॒ इति॑ । शिर॑ इवै॒तद्-य॒ज्ञ्स्य॑ । अथो॒ त्रयो॒ वै प्रा॒णाः । प्रा॒णाने॒वास्मि॑न् दधाति ( ) ॥ म॒खस्य॒ शिरो॑ऽसि॒ संज्योति॑षा॒ ज्योति॑रङ्क्ता॒मित्या॑ह । ज्योति॑रे॒वास्मा॑ उ॒परि॑ष्टाद्-दधाति । सु॒व॒र्गस्य॑ लो॒कस्यानु॑ख्यात्यै । \textbf{ 48} \newline
                  \newline
                                    (परि॑दधाति - प्रा॒णं द॑धाति॒ - हि य॒ज्ञो - घा॑रयति॒ - नम॒ इत्या॑ह - प॒श्चाद् - वी॒र्या॑णीत्या॑ह॒ - भा इत्या॑ह - भ॒जेत्या॑ह - ध्रु॒ - वैवास्मि॑न् दधाति॒ त्रीणि॑ च) \textbf{(A7)} \newline \newline
                \textbf{ 3.3.8     अनुवाकं   8 - इडाभक्षणम्} \newline
                                \textbf{ TB 3.3.8.1} \newline
                  धिष्णि॑या॒ वा ए॒ते न्यु॑प्यन्ते । यद्ब्र॒ह्मा । यद्धोता᳚ । यद॑द्ध्व॒र्युः । यद॒ग्नीत् । यद्-यज॑मानः । तान्. यद॑न्तरे॒यात् । यज॑मानस्य प्रा॒णान्थ्-संक॑र्.षेत् । प्र॒मायु॑कः स्यात् । पु॒रो॒डाश॑मप॒गृह्य॒ संच॑रत्यद्ध्व॒र्युः \textbf{ 49} \newline
                  \newline
                                \textbf{ TB 3.3.8.2} \newline
                  यज॑मानायै॒व तल्लो॒कꣳ शिꣳ॑षति । नास्य॑ प्रा॒णान्थ्-संक॑र्.षति । न प्र॒मायु॑को भवति । पु॒रस्ता᳚त् प्र॒त्यङ्ङासी॑नः । इडा॑या॒ इडा॒माद॑धाति । हस्त्याꣳ॒॒ होत्रे᳚ । प॒शवो॒ वा इडा᳚ । प॒शवः॒ पुरु॑षः । प॒शुष्वे॒व प॒शून् प्रति॑ष्ठापयति । इडा॑यै॒ वा ए॒षा प्रजा॑तिः \textbf{ 50} \newline
                  \newline
                                \textbf{ TB 3.3.8.3} \newline
                  तां प्रजा॑तिं॒ ॅयज॑मा॒नोऽनु॒ प्रजा॑यते । द्विर॒ङ्गुला॑वनक्ति॒ पर्व॑णोः । द्वि॒पाद्-यज॑मानः॒ प्रति॑ष्ठित्यै । स॒कृदुप॑स्तृणाति । द्विराद॑धाति । स॒कृद॒भिघा॑रयति । च॒तुः संप॑द्यते । च॒त्वारि॒ वै प॒शोः प्र॑ति॒ष्ठाना॑नि । यावा॑ने॒व प॒शुः । तमुप॑ह्वयते \textbf{ 51} \newline
                  \newline
                                \textbf{ TB 3.3.8.4} \newline
                  मुख॑मिव॒ प्रत्युप॑ह्वयेत । स॒मुं॒खाने॒व प॒शूनुप॑ह्वयते । प॒शवो॒ वा इडा᳚ । तस्मा॒थ् साऽन्वा॒रभ्या᳚ । अ॒द्ध्व॒र्युणा॑ च॒ यज॑मानेन च । उप॑हूतः पशु॒मान॑सा॒नीत्या॑ह । उप॒ ह्ये॑नौ॒ ह्वय॑ते॒ होता᳚ । इडा॑यै दे॒वता॑ना-मुपह॒वे । उप॑हूतः पशु॒मान् भ॑वति । य ए॒वं ॅवेद॑ \textbf{ 52} \newline
                  \newline
                                \textbf{ TB 3.3.8.5} \newline
                  यां ॅवै हस्त्या॒मिडा॑मा॒दधा॑ति । वा॒चः सा भा॑ग॒धेय᳚म् । यामु॑प॒ह्वय॑ते । प्रा॒णानाꣳ॒॒ सा । वाचं॑ चै॒व प्रा॒णाꣳश्चाव॑रुन्धे ॥ अथ॒ वा ए॒तर्ह्युप॑हूताया॒मिडा॑याम् । पु॒रो॒डाश॑स्यै॒व ब॑र्हि॒षदो॑ मीमाꣳ॒॒सा । यज॑मानं दे॒वा अ॑ब्रुवन्न् । ह॒विर्नो॒ निर्व॒पेति॑ । नाहम॑भा॒गो निर्व॑फ्स्या॒मीत्य॑ब्रवीत् \textbf{ 53} \newline
                  \newline
                                \textbf{ TB 3.3.8.6} \newline
                  न मया॑ऽभा॒गया ऽनु॑वक्ष्य॒थेति॒ वाग॑ब्रवीत् । नाहम॑भा॒गा पु॑रोऽनुवा॒क्या॑ भविष्या॒मीति॑ पुरोऽनुवा॒क्या᳚ । नाहम॑भा॒गा या॒ज्या॑ भविष्या॒मीति॑ या॒ज्या᳚ । न मया॑ऽभा॒गेन॒ वष॑ट्करिष्य॒थेति॑ वषट्का॒रः । यद्-य॑जमानभा॒गं नि॒धाय॑ पुरो॒डाशं॑ बर्.हि॒षदं॑ क॒रोति॑ । ताने॒व तद्-भा॒गिनः॑ करोति । च॒तु॒र्द्धा क॑रोति । चत॑स्रो॒ दिशः॑ । दि॒क्ष्वे॑व प्रति॑तिष्ठति । ब॒र्.॒हि॒षदं॑ करोति \textbf{ 54} \newline
                  \newline
                                \textbf{ TB 3.3.8.7} \newline
                  यज॑मानो॒ वै पु॑रो॒डाशः॑ । प्र॒जा ब॒र्.॒हिः । यज॑मानमे॒व प्र॒जासु॒ प्रति॑ष्ठापयति । तस्मा॑द॒स्थ्नाऽन्याः प्र॒जाः प्र॑ति॒तिष्ठ॑न्ति । माꣳ॒॒सेना॒न्याः । अथो॒ खल्वा॑हुः । दक्षि॑णा॒ वा ए॒ता ह॑विर्-य॒ज्ञ्स्या᳚न्तर्वे॒द्यव॑रुद्ध्यन्ते । यत् पु॑रो॒डाशं॑ बर्.हि॒षदं॑ क॒रोतीति॑ । च॒तु॒र्द्धा क॑रोति । च॒त्वारो॒ ह्ये॑ते ह॑विर्-य॒ज्ञ्स्य॒र्त्विजः॑ \textbf{ 55} \newline
                  \newline
                                \textbf{ TB 3.3.8.8} \newline
                  ब्र॒ह्मा होता᳚ऽद्ध्व॒र्युर॒ग्नीत् । तम॒भिमृ॑शेत् । इ॒दं ब्र॒ह्मणः॑ । इ॒दꣳ होतुः॑ । इ॒दम॑द्ध्व॒र्योः । इ॒दम॒ग्नीध॒ इति॑ । यथै॒वादः सौ॒म्ये᳚ऽद्ध्व॒रे । आ॒देश॑मृ॒त्विग्भ्यो॒ दक्षि॑णा नी॒यन्ते᳚ । ता॒दृगे॒व तत् । अ॒ग्नीधे᳚ प्रथ॒मायाद॑धाति \textbf{ 56} \newline
                  \newline
                                \textbf{ TB 3.3.8.9} \newline
                  अ॒ग्निमु॑खा॒ ह्यृद्धिः॑ । अ॒ग्निमु॑खामे॒वर्द्धिं॒ ॅयज॑मान ऋद्ध्नोति ॥ स॒कृदु॑प॒स्तीर्य॒ द्विरा॒दध॑त् । उ॒प॒स्तीर्य॒ द्विर॒भिघा॑रयति । षट्थ् संप॑द्यन्ते । षड्वा ऋ॒तवः॑ । ऋ॒तूने॒व प्री॑णाति । वे॒देन॑ ब्र॒ह्मणे᳚ ब्रह्मभा॒गं परि॑हरति । प्रा॒जा॒प॒त्यो वै वे॒दः । प्रा॒जा॒प॒त्यो ब्र॒ह्मा \textbf{ 57} \newline
                  \newline
                                \textbf{ TB 3.3.8.10} \newline
                  स॒वि॒ता य॒ज्ञ्स्य॒ प्रसू᳚त्यै । अथ॒ काम॑म॒न्येन॑ । ततो॒ होत्रे᳚ । मद्ध्यं॒ ॅवा ए॒तद्-य॒ज्ञ्स्य॑ । यद्धोता᳚ । म॒द्ध्य॒त ए॒व य॒ज्ञ्ं प्री॑णाति । अथा᳚द्ध्व॒र्यवे᳚ । प्र॒ति॒ष्ठा वा ए॒षा य॒ज्ञ्स्य॑ । यद॑द्ध्व॒र्युः । तस्मा᳚द्धविर्-य॒ज्ञ्स्यै॒तामे॒वावृत॒मनु॑ \textbf{ 58} \newline
                  \newline
                                \textbf{ TB 3.3.8.11} \newline
                  अ॒न्या दक्षि॑णा नीयन्ते । य॒ज्ञ्स्य॒ प्रति॑ष्ठित्यै । अ॒ग्निम॑ग्नीथ् स॒कृथ् स॑कृ॒थ् संमृ॒ड्ढीत्या॑ह । परा॑ङिव॒ ह्ये॑तर्.हि॑ य॒ज्ञ्ः ॥ इ॒षि॒ता दैव्या॒ होता॑र॒ इत्या॑ह । इ॒षि॒तꣳ हि कर्म॑ क्रि॒यते᳚ । भ॒द्र॒वाच्या॑य॒ प्रेषि॑तो॒ मानु॑षः सूक्तवा॒काय॑ सू॒क्ता ब्रू॒हीत्या॑ह । आ॒शिष॑मे॒वैतामाशा᳚स्ते । स्व॒गा दैव्या॒ होतृ॑भ्य॒ इत्या॑ह । य॒ज्ञ्मे॒व तथ्स्व॒गा क॑रोति ( ) । स्व॒स्तिर्-मानु॑षेभ्य॒ इत्या॑ह । आ॒शिष॑मे॒वैतामाशा᳚स्ते । श॒म्ॅयोर्-ब्रू॒हीत्या॑ह । श॒म्ॅयुमे॒व बा॑र्.हस्प॒त्यं भा॑ग॒धेये॑न॒ सम॑र्द्धयति । \textbf{ 59} \newline
                  \newline
                                    (च॒र॒त्य॒द्ध्व॒र्युः - प्रजा॑तिर् - ह्वयते॒ - वेदा᳚ - ब्रवीद् - बर्.हि॒षदं॑ करो-त्यृ॒त्विजो॑-दधाति-ब्र॒ह्मा-ऽनु॑-करोति च॒त्वारि॑ च) \textbf{(A8)} \newline \newline
                \textbf{ 3.3.9     अनुवाकं   9 - प्रस्तरप्रहारादिकम्} \newline
                                \textbf{ TB 3.3.9.1} \newline
                  अथ॒ स्रुचा॑वनु॒ष्टुग्भ्यां॒ ॅवाज॑वतीभ्यां॒ ॅव्यू॑हति । प्र॒ति॒ष्ठा वा अ॑नु॒ष्टुक् । अन्नं॒ ॅवाजः॒ प्रति॑ष्ठित्यै । अ॒न्नाद्य॒स्याव॑रुद्ध्यै । प्राचीं᳚ जु॒हूमू॑हति । जा॒ताने॒व भ्रातृ॑व्या॒न् प्रणु॑दते । प्र॒तीची॑मुप॒भृत᳚म् । ज॒नि॒ष्यमा॑णाने॒व प्रति॑नुदते । स विषू॑च ए॒वापोह्य॑ स॒पत्ना॒न्॒. यज॑मानः । अ॒स्मिन् ॅलो॒के प्रति॑तिष्ठति \textbf{ 60} \newline
                  \newline
                                \textbf{ TB 3.3.9.2} \newline
                  द्वाभ्या᳚म् । द्विप्र॑तिष्ठो॒ हि ॥ वसु॑भ्यस्त्वा रु॒द्रेभ्य॑स्त्वाऽऽदि॒त्येभ्य॒स्त्वेत्या॑ह । य॒था॒य॒जुरे॒वै तत् । स्रु॒क्षु प्र॑स्त॒रम॑नक्ति । इ॒मे वै लो॒काः स्रुचः॑ । यज॑मानः प्रस्त॒रः । यज॑मानमे॒व तेज॑साऽनक्ति । त्रे॒धाऽन॑क्ति । त्रय॑ इ॒मे लो॒काः \textbf{ 61} \newline
                  \newline
                                \textbf{ TB 3.3.9.3} \newline
                  ए॒भ्य ए॒वैनं॑ ॅलो॒केभ्यो॑ऽनक्ति । अ॒भि॒पू॒र्वम॑नक्ति । अ॒भि॒पू॒र्वमे॒व यज॑मानं॒ तेज॑साऽनक्ति ॥ अ॒क्तꣳ रिहा॑णा॒ इत्या॑ह । तेजो॒ वा आज्य᳚म् । यज॑मानः प्रस्त॒रः । यज॑मानमे॒व तेज॑साऽनक्ति । वि॒यन्तु॒ वय॒ इत्या॑ह । वय॑ ए॒वैनं॑ कृ॒त्वा । सु॒व॒र्गं ॅलो॒कं ग॑मयति \textbf{ 62} \newline
                  \newline
                                \textbf{ TB 3.3.9.4} \newline
                  प्र॒जां ॅयोनिं॒ मा निर्मृ॑क्ष॒मित्या॑ह । प्र॒जायै॑ गोपी॒थाय॑ ॥ आप्या॑यन्ता॒माप॒ ओष॑धय॒ इत्या॑ह । आप॑ ए॒वौष॑धी॒राप्या॑ययति । म॒रुतां॒ पृष॑तयः॒ स्थेत्या॑ह । म॒रुतो॒ वै वृष्ट्या॑ ईशते । वृष्टि॑मे॒वाव॑रुन्धे । दिवं॑ गच्छ॒ ततो॑ नो॒ वृष्टि॒मेर॒येत्या॑ह । वृष्टि॒र्वै द्यौः । वृष्टि॑मे॒वाव॑रुन्धे \textbf{ 63} \newline
                  \newline
                                \textbf{ TB 3.3.9.5} \newline
                  याव॒द्वा अ॑द्ध्व॒र्युः प्र॑स्त॒रं प्र॒हर॑ति । ताव॑द॒स्यायु॑र्मीयते । आ॒यु॒ष्पा अ॑ग्ने॒ऽस्यायु॑र्मे पा॒हीत्या॑ह । आयु॑रे॒वात्मन्-ध॑त्ते । याव॒द्वा अ॑द्ध्व॒र्युः प्र॑स्त॒रं प्र॒हर॑ति । ताव॑दस्य॒ चक्षु॑र्मीयते । च॒क्षु॒ष्पा अ॑ग्नेऽसि॒ चक्षु॑र्मे पा॒हीत्या॑ह । चक्षु॑रे॒वात्मन्-ध॑त्ते ॥ ध्रु॒वाऽसीत्या॑ह॒ प्रति॑ष्ठित्यै ॥ यं प॑रि॒धिं प॒र्यध॑त्था॒ इत्या॑ह \textbf{ 64} \newline
                  \newline
                                \textbf{ TB 3.3.9.6} \newline
                  य॒था॒य॒जुरे॒वैतत् । अग्ने॑ देव प॒णिभि॑र्-वी॒यमा॑ण॒ इत्या॑ह । अ॒ग्नय॑ ए॒वैनं॒ जुष्टं॑ करोति । तं त॑ ए॒तमनु॒ जोषं॑ भरा॒मीत्या॑ह । स॒जा॒ताने॒वास्मा॒ अनु॑कान्करोति । नेदे॒ष त्वद॑पचे॒तया॑ता॒ इत्या॒हानु॑ख्यात्यै ॥ य॒ज्ञ्स्य॒ पाथ॒ उप॒ समि॑त॒मित्या॑ह । भू॒मान॑मे॒वोपै॑ति । प॒रि॒धीन्-प्रह॑रति । य॒ज्ञ्स्य॒ समि॑ष्ट्यै \textbf{ 65} \newline
                  \newline
                                \textbf{ TB 3.3.9.7} \newline
                  स्रुचौ॒ सं प्रस्रा॑वयति । यदे॒व तत्र॑ क्रू॒रम् । तत्-तेन॑ शमयति । जु॒ह्वामु॑प॒भृत᳚म् । य॒ज॒मा॒न॒दे॒व॒त्या॑ वै जु॒हूः । भ्रा॒तृ॒व्य॒दे॒व॒त्यो॑प॒भृत् । यज॑मानायै॒व भ्रातृ॑व्य॒मुप॑स्तिं करोति ॥ सꣳ॒॒स्रा॒वभा॑गाः॒ स्थेत्या॑ह । वस॑वो॒ वै रु॒द्रा आ॑दि॒त्याः सꣳ॑स्रा॒वभा॑गाः । तेषां॒ तद्-भा॑ग॒धेय᳚म् \textbf{ 66} \newline
                  \newline
                                \textbf{ TB 3.3.9.8} \newline
                  ताने॒व तेन॑ प्रीणाति । वै॒श्व॒दे॒व्यर्चा । ए॒ते हि विश्वे॑ दे॒वाः । त्रि॒ष्टुग्-भ॑वति । इ॒न्द्रि॒यं ॅवै त्रि॒ष्टुक् । इ॒न्द्रि॒यमे॒व यज॑माने दधाति ॥ अ॒ग्नेर्वा॒मप॑न्नगृहस्य॒ सद॑सि सादया॒मीत्या॑ह । इ॒यं ॅवा अ॒ग्निरप॑न्नगृहः । अ॒स्या ए॒वैने॒ सद॑ने सादयति । सु॒म्नाय॑ सुम्निनी सु॒म्ने मा॑ धत्त॒मित्या॑ह \textbf{ 67} \newline
                  \newline
                                \textbf{ TB 3.3.9.9} \newline
                  प्र॒जा वै प॒शवः॑ सु॒म्नम् । प्र॒जामे॒व प॒शूना॒त्मन्ध॑त्ते । धु॒रि धु॒र्यौ॑ पात॒मित्या॑ह । जा॒या॒प॒त्योर्-गो॑पी॒थाय॑ ॥ अग्ने॑ऽदब्धायोऽशीततनो॒ इत्या॑ह । य॒था॒य॒जुरे॒वैतत् । पा॒हि मा॒ऽद्य दि॒वः पा॒हि प्रसि॑त्यै पा॒हि दुरि॑ष्ट्यै पा॒हि दु॑रद्म॒न्यै पा॒हि दुश्च॑रिता॒दित्या॑ह । आ॒शिष॑मे॒वैतामाशा᳚स्ते । अवि॑षं नः पि॒तुं कृ॑णु सु॒षदा॒ योनिꣳ॒॒ स्वाहेती᳚द्ध्म-स॒म्ॅवृश्च॑नान्यन्वाहार्य॒पच॑ने ऽभ्या॒धाय॑ फलीकरणहो॒मं जु॑होति । अति॑रिक्तानि॒ वा इ॑द्ध्म स॒म्ॅवृश्च॑नानि \textbf{ 68} \newline
                  \newline
                                \textbf{ TB 3.3.9.10} \newline
                  अति॑रिक्ताः फली॒कर॑णाः । अति॑रिक्तमाज्योच्छेष॒णम् । अति॑रिक्त ए॒वाति॑रिक्तं दधाति । अथो॒ अति॑रिक्ते-नै॒वाति॑रिक्त-मा॒प्त्वाऽव॑रुन्धे ॥ वेदि॑र्-दे॒वेभ्यो॒ निला॑यत । तां ॅवे॒देनान्व॑विन्दन्न् । वे॒देन॒ वेदिं॑ ॅविविदुः पृथि॒वीम् । सा प॑प्रथे पृथि॒वी पार्थि॑वानि । गर्भं॑ बिभर्ति॒ भुव॑नेष्व॒न्तः । ततो॑ य॒ज्ञो जा॑यते विश्व॒दानि॒रिति॑ पु॒रस्ता᳚थ् स्तम्बय॒जुषो॑ वे॒देन॒ वेदिꣳ॒॒ संमा॒ष्ट्र्यनु॑वित्त्यै \textbf{ 69} \newline
                  \newline
                                \textbf{ TB 3.3.9.11} \newline
                  अथो॒ यद्-वे॒दश्च॒ वेदि॑श्च॒ भव॑तः । मि॒थु॒न॒त्वाय॒ प्रजा᳚त्यै । प्र॒जाप॑ते॒र्वा ए॒तानि॒ श्मश्रू॑णि । यद्-वे॒दः । पत्नि॑या उ॒पस्थ॒ आस्य॑ति । मि॒थु॒नमे॒व क॑रोति । वि॒न्दते᳚ प्र॒जाम् । वे॒दꣳ होताऽऽह॑व॒नीया᳚थ्-स्तृ॒णन्ने॑ति । य॒ज्ञ्मे॒व तथ् सन्त॑नो॒त्योत्त॑रस्मा-दर्द्धमा॒सात् । तꣳ सन्त॑त॒मुत्त॑रेऽर्द्धमा॒स आल॑भते \textbf{ 70} \newline
                  \newline
                                \textbf{ TB 3.3.9.12} \newline
                  तं का॒लेका॑ल॒ आग॑ते यजते ॥ ब्र॒ह्म॒वा॒दिनो॑ वदन्ति । स त्वा अ॑द्ध्व॒र्युः स्या᳚त् । यो यतो॑ य॒ज्ञ्ं प्र॑यु॒ङ्क्ते । तदे॑नं प्रतिष्ठा॒पय॒तीति॑ । वाता॒द्वा अ॑द्ध्व॒र्युर्य॒ज्ञ्ं प्रयु॑ङ्क्ते । देवा॑ गातुविदो गा॒तुं ॅवि॒त्वा गा॒तुमि॒तेत्या॑ह । यत॑ ए॒व य॒ज्ञ्ं प्र॑यु॒ङ्क्ते । तदे॑नं॒ प्रति॑ष्ठापयति । प्रति॑तिष्ठति प्र॒जया॑ प॒शुभि॒र्-यज॑मानः ( ) । \textbf{ 71} \newline
                  \newline
                                    (ति॒ष्ठ॒ती॒ - मे लो॒का - ग॑मयति॒ - द्यौर् वृष्टि॑मे॒वाव॑रुन्धे - प॒र्यध॑त्था॒ इत्या॑ह॒ - समि॑ष्ट्यै - भाग॒धेयं॑ - धत्त॒मित्या॑ह॒ - वा इ॑द्ध्म स॒म्ॅवृश्च॑ना॒ - न्यनु॑वित्त्यै - लभते॒ - यज॑मानः) \textbf{(A9)} \newline \newline
                \textbf{ 3.3.10    अनुवाकं   10 - पत्नीयोक्रविमोकः} \newline
                                \textbf{ TB 3.3.10.1} \newline
                  यो वा अय॑थादेवतं ॅय॒ज्ञ्मु॑प॒चर॑ति । आ दे॒वता᳚भ्यो वृश्च्यते । पापी॑यान् भवति । यो य॑थादेव॒तम् । न दे॒वता᳚भ्य॒ आवृ॑श्च्यते । वसी॑यान् भवति । वा॒रु॒णो वै पाशः॑ । इ॒मं ॅविष्या॑मि॒ वरु॑णस्य॒ पाश॒मित्या॑ह । व॒रु॒ण॒पा॒शादे॒वैनां᳚ मुञ्चति । स॒वि॒तृप्र॑सूतो यथादेव॒तम् \textbf{ 72} \newline
                  \newline
                                \textbf{ TB 3.3.10.2} \newline
                  न दे॒वता᳚भ्य॒ आवृ॑श्च्यते । वसी॑यान् भवति । धा॒तुश्च॒ योनौ॑ सुकृ॒तस्य॑ लो॒क इत्या॑ह । अ॒ग्निर्वै धा॒ता । पुण्यं॒ कर्म॑ सुकृ॒तस्य॑ लो॒कः । अ॒ग्निरे॒वैनां᳚ धा॒ता । पुण्ये॒ कर्म॑णि सुकृ॒तस्य॑ लो॒के द॑धाति । स्यो॒नं मे॑ स॒ह पत्या॑ करो॒मीत्या॑ह । आ॒त्मन॑श्च॒ यज॑मानस्य॒ चाना᳚त्यै स॒त्वांय॑ ॥ समायु॑षा॒ संप्र॒जयेत्या॑ह \textbf{ 73} \newline
                  \newline
                                \textbf{ TB 3.3.10.3} \newline
                  आ॒शिष॑मे॒वैतामाशा᳚स्ते पूर्णपा॒त्रे । अ॒न्त॒तो॑ऽनु॒ष्टुभा᳚ । चतु॑ष्प॒द्वा ए॒तच्छन्दः॒ प्रति॑ष्ठितं॒ पत्नि॑यै पूर्णपा॒त्रे भ॑वति । अ॒स्मिन् ॅलो॒के प्रति॑तिष्ठा॒नीति॑ । अ॒स्मिन्ने॒व लो॒के प्रति॑तिष्ठति । अथो॒ वाग्वा अ॑नु॒ष्टुक् । वाङ्मि॑थु॒नम् । आपो॒ रेतः॑ प्र॒जन॑नम् । ए॒तस्मा॒द्वै मि॑थु॒नाद्वि॒द्योत॑मानः स्त॒नय॑न्वर्.षति । रेतः॑ सि॒ञ्चन्न् \textbf{ 74} \newline
                  \newline
                                \textbf{ TB 3.3.10.4} \newline
                  प्र॒जाः प्र॑ज॒नयन्न्॑ ॥ यद्वै य॒ज्ञ्स्य॒ ब्रह्म॑णा यु॒ज्यते᳚ । ब्रह्म॑णा॒ वै तस्य॑ विमो॒कः । अ॒द्भिः शान्तिः॑ । विमु॑क्तं॒ ॅवा ए॒तर्.हि॒ योक्त्रं॒ ब्रह्म॑णा । आ॒दायै॑न॒त्पत्नी॑ स॒हाप उप॑गृह्णीते॒ शान्त्यै᳚ । अ॒ञ्ज॒लौ पू᳚र्णपा॒त्रमान॑यति । रेत॑ ए॒वास्यां᳚ प्र॒जां द॑धाति । प्र॒जया॒ हि म॑नु॒ष्यः॑ पू॒र्णः । मुखं॒ ॅविमृ॑ष्टे ( ) । अ॒व॒भृ॒थस्यै॒व रू॒पं कृ॒त्वोत्ति॑ष्ठति । \textbf{ 75} \newline
                  \newline
                                    (स॒वि॒तृप्र॑सूतो यथादेव॒तं-प्र॒जयेत्या॑ह-सि॒ञ्चन्-मृ॑ष्ट॒ एकं॑ च) \textbf{(A10)} \newline \newline
                \textbf{ 3.3.11    अनुवाकं   11 - उपवेषोपगूहनम्} \newline
                                \textbf{ TB 3.3.11.1} \newline
                  प॒रि॒वे॒षो वा ए॒ष वन॒स्पती॑नाम् । यदु॑पवे॒षः । य ए॒वं ॅवेद॑ । वि॒न्दते॑ परिवे॒ष्टार᳚म् । तमु॑त्क॒रे ॥ यं दे॒वा म॑नु॒ष्ये॑षु । उ॒प॒वे॒षमधा॑रयन्न् । ये अ॒स्मदप॑चेतसः । तान॒स्मभ्य॑-मि॒हाकु॑रु । उप॑वे॒षोप॑विड्ढि नः \textbf{ 76} \newline
                  \newline
                                \textbf{ TB 3.3.11.2} \newline
                  प्र॒जां पुष्टि॒मथो॒ धन᳚म् । द्वि॒पदो॑ न॒श्चतु॑ष्पदः । ध्रु॒वान-न॑पगान्-कु॒र्विति॑ पु॒रस्ता᳚त् प्र॒त्यञ्च॒मुप॑गूहति । तस्मा᳚त् पु॒रस्ता᳚त् प्र॒त्यञ्चः॑ शू॒द्रा अव॑स्यन्ति । स्थ॒वि॒म॒त उप॑गूहति । अप्र॑तिवादिन ए॒वैना᳚न् कुरुते ॥ धृष्टि॒र्वा उ॑पवे॒षः । शु॒चार्तो वज्रो॒ ब्रह्म॑णा॒ सꣳशि॑तः । योप॑वे॒षे शुक् । साऽमुमृ॑च्छतु॒ यं द्वि॒ष्म इति॑ \textbf{ 77} \newline
                  \newline
                                \textbf{ TB 3.3.11.3} \newline
                  अथा᳚स्मै नाम॒ गृह्य॒ प्रह॑रति । निर॒मुं नु॑द॒ ओक॑सः । स॒पत्नो॒ यः पृ॑त॒न्यति॑ । नि॒र्बा॒द्ध्ये॑न ह॒विषा᳚ । इन्द्र॑ एणं॒ परा॑शरीत् । इ॒हि ति॒स्रः प॑रा॒वतः॑ । इ॒हि पञ्च॒जनाꣳ॒॒ अति॑ । इ॒हि ति॒स्रोऽति॑ रोच॒ना याव॑त् । सूर्यो॒ अस॑द्-दि॒वि । प॒र॒मां त्वा॑ परा॒वत᳚म् \textbf{ 78} \newline
                  \newline
                                \textbf{ TB 3.3.11.4} \newline
                  इन्द्रो॑ नयतु वृत्र॒हा । यतो॒ न पुन॒राय॑सि । श॒श्व॒तीभ्यः॒ समा᳚भ्य॒ इति॑ । त्रि॒वृद्वा ए॒ष वज्रो॒ ब्रह्म॑णा॒ सꣳशि॑तः । शु॒चैवैनं॑ ॅवि॒द्ध्वा । ए॒भ्यो लो॒केभ्यो॑ नि॒र्णुद्य॑ । वज्रे॑ण॒ ब्रह्म॑णा स्तृणुते । ह॒तो॑ ऽसावव॑धिष्मा॒-मुमित्या॑ह॒ स्तृत्यै᳚ । यं द्वि॒ष्यात्तं ध्या॑येत् । शु॒चैवैन॑-मर्पयति ( ) । \textbf{ 79} \newline
                  \newline
                                    (नो॒-द्वि॒ष्म इति॑-परा॒वत॑-मर्पयति) \textbf{(A11)} \newline \newline
                \textbf{Prapaataka korvai with starting  Words of 1 to 11 Anuvaakams :-} \newline
        (प्रत्यु॑ष्टं - दि॒वः शिल्प॒ - मय॑ज्ञो - धृ॒तं च॑ - देवासु॒राः स ए॒तमिन्द्र॒ - आपो॑ देवी - र॒ग्निना॒ - धिष्णि॑या॒ - अथ॒ स्रुचौ॒ - यो वा अय॑थादेवतं - परिवे॒षो वा एका॑दश) \newline

        \textbf{korvai with starting Words of 1, 11, 21 Series of Dasinis :-} \newline
        (प्रत्यु॑ष्ट॒ - मय॑ज्ञ् - ए॒षा हि विश्वे॑षां दे॒वाना॑ - मू॒र्जा पृ॑थि॒वी - मथो॒ रक्ष॑सां॒ - तां प्रजा॑तिं॒ - द्वाभ्यां॒ - तं का॒लेका॑ले॒ नव॑सप्ततिः) \newline

        \textbf{first and last  Word 3rd Ashtakam 3rd Prapaatakam :-} \newline
        (प्रत्यु॑ष्टꣳ - शु॒चैवैन॑मर्पयति) \newline 

       

        ॥ हरिः॑ ॐ ॥
॥ कृष्ण यजुर्वेदीय तैत्तिरीय ब्राह्मणे तृतीयाष्टके तृतीय प्रपाठकः समाप्तः ॥
=================================== \newline
        \pagebreak
        
        
        
     \addcontentsline{toc}{section}{ 3.4     तृतीयाष्टके चतुर्थः प्रपाठकः - पुरुषमेधः}
     \markright{ 3.4     तृतीयाष्टके चतुर्थः प्रपाठकः - पुरुषमेधः \hfill https://www.vedavms.in \hfill}
     \section*{ 3.4     तृतीयाष्टके चतुर्थः प्रपाठकः - पुरुषमेधः }
                \textbf{ 3.4.1     अनुवाकं   1 - पुरुषमेधः} \newline
                                \textbf{ TB 3.4.1.1} \newline
                  ब्रह्म॑णे ब्राह्म॒णमाल॑भते । क्ष॒त्राय॑ राज॒न्य᳚म् । म॒रुद्भ्यो॒ वैश्य᳚म् । तप॑से शू॒द्रम् । तम॑से॒ तस्क॑रम् । नार॑काय वीर॒हण᳚म् । पा॒प्मने᳚ क्ली॒बम् । आ॒क्र॒याया॑यो॒गूम् । कामा॑य पुꣳश्च॒लूम् । अति॑क्रुष्टाय माग॒धम् । \textbf{ 1} \newline
                  \newline
                                     \textbf{(A1)} \newline \newline
                \textbf{ 3.4.2     अनुवाकं   2 –} \newline
                                \textbf{ TB 3.4.2.1} \newline
                  गी॒ताय॑ सू॒तम् । नृ॒त्ताय॑ शैलू॒षम् । धर्मा॑य सभाच॒रम् । न॒र्माय॑ रे॒भम् । नरि॑ष्ठायै भीम॒लम् । हसा॑य॒ कारि᳚म् । आ॒न॒न्दाय॑ स्त्रीष॒खम् । प्र॒मुदे॑ कुमारीपु॒त्रम् । मे॒धायै॑ रथका॒रम् । धैर्या॑य॒ तक्षा॑णम् । \textbf{ 2} \newline
                  \newline
                                     \textbf{(A2)} \newline \newline
                \textbf{ 3.4.3     अनुवाकं   3 –} \newline
                                \textbf{ TB 3.4.3.1} \newline
                  श्रमा॑य कौला॒लम् । मा॒यायै॑ कार्मा॒रम् । रू॒पाय॑ मणिका॒रम् । शुभे॑ व॒पम् । श॒र॒व्या॑या इषुका॒रम् । हे॒त्यै ध॑न्वका॒रम् । कर्म॑णे ज्याका॒रम् । दि॒ष्टाय॑ रज्जुस॒र्गम् । मृ॒त्यवे॑ मृग॒युम् । अन्त॑काय श्व॒नितं᳚ । \textbf{ 3} \newline
                  \newline
                                     \textbf{(A3)} \newline \newline
                \textbf{ 3.4.4     अनुवाकं   4 –} \newline
                                \textbf{ TB 3.4.4.1} \newline
                  स॒न्धये॑ जा॒रम् । गे॒हायो॑पप॒तिम् । निर्.ऋ॑त्यै परिवि॒त्तम् । आर्त्यै॑ परिविविदा॒नम् । अरा᳚द्ध्यै दिधिषू॒पति᳚म् । प॒वित्रा॑य भि॒षज᳚म् । प्र॒ज्ञाना॑य नक्षत्रद॒र॒.शम् । निष्कृ॑त्यै पेशस्का॒रीम् । बला॑यो प॒दाम् । वर्णा॑यानू॒रुध᳚म् । \textbf{ 4} \newline
                  \newline
                                     \textbf{(A4)} \newline \newline
                \textbf{ 3.4.5     अनुवाकं   5 –} \newline
                                \textbf{ TB 3.4.5.1} \newline
                  न॒दीभ्यः॑ पौञ्जि॒ष्टम् । ऋ॒क्षीका᳚भ्यो॒ नैषा॑दम् । पु॒रु॒ष॒व्या॒घ्राय॑ दु॒र्मद᳚म् । प्र॒युद्भ्य॒ उन्म॑त्तम् । ग॒न्ध॒र्वा॒फ्स॒राभ्यो॒ व्रात्य᳚म् । स॒र्प॒दे॒व॒ज॒नेभ्यो ऽप्र॑तिपदम् । अवे᳚भ्यः कित॒वम् । इ॒र्यता॑या॒ अकि॑तवम् । पि॒शा॒चेभ्यो॑ बिदलका॒रम् । या॒तु॒धाने᳚भ्यः कण्टकका॒रम् । \textbf{ 5} \newline
                  \newline
                                     \textbf{(A5)} \newline \newline
                \textbf{ 3.4.6     अनुवाकं   6 –} \newline
                                \textbf{ TB 3.4.6.1} \newline
                  उ॒थ्सा॒देभ्यः॑ कु॒ब्जम् । प्र॒मुदे॑ वाम॒नम् । द्वा॒र्भ्यः स्रा॒मम् । स्वप्ना॑या॒न्धम् । अध॑र्माय बधि॒रम् । स॒ज्ञांना॑य स्मरका॒रीम् । प्र॒का॒मोद्या॑योप॒सद᳚म् । आ॒शि॒क्षायै᳚ प्र॒श्निन᳚म् । उ॒प॒शि॒क्षाया॑ अभिप्र॒श्निन᳚म् । म॒र्यादा॑यै प्रश्नविवा॒कम् । \textbf{ 6} \newline
                  \newline
                                     \textbf{(A6)} \newline \newline
                \textbf{ 3.4.7     अनुवाकं   7 –} \newline
                                \textbf{ TB 3.4.7.1} \newline
                  ऋत्यै᳚ स्ते॒नहृ॑दयम् । वैर॑हत्याय॒ पिशु॑नम् । विवि॑त्त्यै क्ष॒त्तार᳚म् । औप॑द्रष्टाय संग्रही॒तार᳚म् । बला॑यानुच॒रम् । भू॒म्ने प॑रिष्क॒न्दम् । प्रि॒याय॑ प्रियवा॒दिन᳚म् । अरि॑ष्ट्या अश्वसा॒दम् । मेधा॑य वासः पल्पू॒लीम् । प्र॒का॒माय॑ रजयि॒त्रीम् । \textbf{ 7} \newline
                  \newline
                                     \textbf{(A7)} \newline \newline
                \textbf{ 3.4.8     अनुवाकं   8 –} \newline
                                \textbf{ TB 3.4.8.1} \newline
                  भायै॑ दार्वाहा॒रम् । प्र॒भाया॑ आग्ने॒न्धम् । नाक॑स्य पृ॒ष्ठाया॑भिषे॒क्तार᳚म् । ब्र॒द्ध्नस्य॑ वि॒ष्टपा॑य पात्रनिर्णे॒गम् । दे॒व॒लो॒काय॑ पेशि॒तार᳚म् । म॒नु॒ष्य॒लो॒काय॑ प्रकरि॒तार᳚म् । सर्वे᳚भ्यो लो॒केभ्य॑ उपसे॒क्तार᳚म् । अव॑र्त्यै व॒धायो॑पमन्थि॒तार᳚म् । सुव॒र्गाय॑ लो॒काय॑ भाग॒दुघ᳚म् । वर्.षि॑ष्ठाय॒ नाका॑य परिवे॒ष्टार᳚म् । \textbf{ 8} \newline
                  \newline
                                     \textbf{(A8)} \newline \newline
                \textbf{ 3.4.9     अनुवाकं   9 –} \newline
                                \textbf{ TB 3.4.9.1} \newline
                  अर्मे᳚भ्यो हस्ति॒पम् । ज॒वाया᳚श्व॒पम् । पुष्ट्यै॑ गोपा॒लम् । तेज॑सेऽजपा॒लम् । वी॒र्या॑याविपा॒लम् । इरा॑यै की॒नाश᳚म् । की॒लाला॑य सुराका॒रम् । भ॒द्राय॑ गृह॒पम् । श्रेय॑से वित्त॒धम् । अद्ध्य॑क्षायानुक्ष॒त्तार᳚म् । \textbf{ 9} \newline
                  \newline
                                     \textbf{(A9)} \newline \newline
                \textbf{ 3.4.10    अनुवाकं   10 –} \newline
                                \textbf{ TB 3.4.10.1} \newline
                  म॒न्यवे॑ऽयस्ता॒पम् । क्रोधा॑य निस॒रम् । शोका॑याभिस॒रम् । उ॒त्कू॒ल॒वि॒कू॒लाभ्यां᳚ त्रि॒स्थिन᳚म् । योगा॑य यो॒क्तार᳚म् । क्षेमा॑य विमो॒क्तार᳚म् । वपु॑षे मानस्कृ॒तम् । शीला॑याञ्जनीका॒रम् । निर्.ऋ॑त्यै कोशका॒रीम् । य॒माया॒सूम् । \textbf{ 10} \newline
                  \newline
                                     \textbf{(A10)} \newline \newline
                \textbf{ 3.4.11    अनुवाकं   11 –} \newline
                                \textbf{ TB 3.4.11.1} \newline
                  य॒म्यै॑ यम॒सूम् । अथ॑र्व॒भ्योऽव॑तोकाम् । स॒म्ॅव॒थ्स॒राय॑ पर्या॒रिणी᳚म् । प॒रि॒व॒थ्स॒रायावि॑जाताम् । इ॒दा॒व॒थ्स॒राया॑प॒स्कद्व॑रीम् । इ॒द्व॒थ्स॒राया॒तीत्व॑रीम् । व॒थ्स॒राय॒ विज॑र्जराम् । स॒र्व॒थ्सं॒राय॒ पलि॑क्नीम् । वना॑य वन॒पम् । अ॒न्यतो॑रण्याय दाव॒पम् । \textbf{ 11} \newline
                  \newline
                                     \textbf{(A11)} \newline \newline
                \textbf{ 3.4.12    अनुवाकं   12 –} \newline
                                \textbf{ TB 3.4.12.1} \newline
                  सरो᳚भ्यो धैव॒रम् । वेश॑न्ताभ्यो॒ दाश᳚म् । उ॒प॒स्थाव॑रीभ्यो॒ बैन्द᳚म् । न॒ड्व॒लाभ्यः॑ शौष्क॒लम् । पा॒र्या॑य कैव॒र्तम् । अ॒वा॒र्या॑य मार्गा॒रम् । ती॒र्थेभ्य॑ आ॒न्दम् । विष॑मेभ्यो मैना॒लम् । स्वने᳚भ्यः॒ पर्ण॑कम् । गुहा᳚भ्यः॒ किरा॑तम् । सानु॑भ्यो॒ जम्भ॑कम् । पर्व॑तेभ्यः॒ किं पू॑रुषम् । \textbf{ 12} \newline
                  \newline
                                     \textbf{(A12)} \newline \newline
                \textbf{ 3.4.13    अनुवाकं   13 –} \newline
                                \textbf{ TB 3.4.13.1} \newline
                  प्र॒ति॒श्रुत्का॑या ऋतु॒लम् । घोषा॑य भ॒षम् । अन्ता॑य बहुवा॒दिन᳚म् । अ॒न॒न्ताय॒ मूक᳚म् । मह॑से वीणावा॒दम् । क्रोशा॑य तूणव॒द्ध्मम् । आ॒क्र॒न्दाय॑ दुन्दुभ्याघा॒तम् । अ॒व॒र॒स्प॒राय॑ शङ्ख॒द्ध्मम् । ऋ॒भुभ्यो॑ऽजिनसन्धा॒यम् । सा॒द्ध्येभ्य॑श्चर्म॒म्णम् । \textbf{ 13} \newline
                  \newline
                                     \textbf{(A13)} \newline \newline
                \textbf{ 3.4.14    अनुवाकं   14 –} \newline
                                \textbf{ TB 3.4.14.1} \newline
                  बी॒भ॒थ्सायै॑ पौल्क॒सम् । भूत्यै॑ जागर॒णम् । अभू᳚त्यै स्वप॒नम् । तु॒लायै॑ वाणि॒जम् । वर्णा॑य हिरण्यका॒रम् । विश्वे᳚भ्यो दे॒वेभ्यः॑ सिद्ध्म॒लम् । प॒श्चा॒द्-दो॒षाय॑ ग्ला॒वम् । ऋत्यै॑ जनवा॒दिन᳚म् । व्यृ॑द्ध्या अपग॒ल्भम् । सꣳ॒॒श॒राय॑ प्र॒च्छिद᳚म् । \textbf{ 14} \newline
                  \newline
                                     \textbf{(A14)} \newline \newline
                \textbf{ 3.4.15    अनुवाकं   15 –} \newline
                                \textbf{ TB 3.4.15.1} \newline
                  हसा॑य पुꣳश्च॒लूमाल॑भते । वी॒णा॒वा॒दं गण॑कं गी॒ताय॑ । याद॑से शाबु॒ल्याम् । न॒र्माय॑ भद्रव॒तीम् । तू॒ण॒व॒द्ध्मं ग्रा॑म॒ण्यं॑ पाणिसंघा॒तं नृ॒त्ताय॑ । मोदा॑यानु॒क्रोश॑कम् । आ॒न॒न्दाय॑ तल॒वम् । \textbf{ 15} \newline
                  \newline
                                     \textbf{(A15)} \newline \newline
                \textbf{ 3.4.16    अनुवाकं   16 –} \newline
                                \textbf{ TB 3.4.16.1} \newline
                  अ॒क्ष॒रा॒जाय॑ कित॒वम् । कृ॒ताय॑ सभा॒विन᳚म् । त्रेता॑या आदिनवद॒र्॒.शम् । द्वा॒प॒राय॑ बहिः॒ सद᳚म् । कल॑ये सभास्था॒णुम् । दु॒ष्कृ॒ताय॑ च॒रका॑चार्यम् । अद्ध्व॑ने ब्रह्मचा॒रिण᳚म् । पि॒शा॒चेभ्यः॑ सैल॒गम् । पि॒पा॒सायै॑ गोव्य॒च्छम् । निर्.ऋ॑त्यै गोघा॒तम् । क्षु॒धे गो॑विक॒र्तम् । क्षु॒त्तृ॒ष्णाभ्यां॒ तम् । यो गां ॅवि॒कृन्त॑न्तं माꣳ॒॒सं भिक्ष॑माण उप॒तिष्ठ॑ते । \textbf{ 16} \newline
                  \newline
                                     \textbf{(A16)} \newline \newline
                \textbf{ 3.4.17    अनुवाकं   17 –} \newline
                                \textbf{ TB 3.4.17.1} \newline
                  भूम्यै॑ पीठस॒र्पिण॒माल॑भते । अ॒ग्नयेऽꣳ॑स॒लम् । वा॒यवे॑ चाण्डा॒लम् । अ॒न्तरि॑क्षाय वꣳशन॒र्तिन᳚म् । दि॒वे ख॑ल॒तिम् । सूर्या॑य हर्य॒क्षम् । च॒न्द्रम॑से मिर्मि॒रम् । नक्ष॑त्रेभ्यः कि॒लास᳚म् । अह्ने॑ शु॒क्लं पि॑ङ्ग॒लम् । रात्रि॑यै कृ॒ष्णं पि॑ङ्गा॒क्षम् । \textbf{ 17} \newline
                  \newline
                                     \textbf{(A17)} \newline \newline
                \textbf{ 3.4.18    अनुवाकं   18 –} \newline
                                \textbf{ TB 3.4.18.1} \newline
                  वा॒चे पुरु॑ष॒माल॑भते । प्रा॒णम॑पा॒नं ॅव्या॒नमु॑दा॒नꣳ स॑मा॒नं तान्. वा॒यवे᳚ । सूर्या॑य॒ चक्षु॒राल॑भते । मन॑श्च॒न्द्रम॑से । दि॒ग्म्यः श्रोत्र᳚म् । प्र॒जाप॑तये॒ पुरु॑षम् । \textbf{ 18} \newline
                  \newline
                                     \textbf{(A18)} \newline \newline
                \textbf{ 3.4.19    अनुवाकं   19 –} \newline
                                \textbf{ TB 3.4.19.1} \newline
                  अथै॒तानरू॑पेभ्य॒ आल॑भते । अति॑ह्रस्व॒-मति॑दीर्घम् । अति॑कृश॒मत्यꣳ॑सलम् । अति॑शुक्ल॒-मति॑कृष्णम् । अति॑श्लक्ष्ण॒-मति॑लोमशम् । अति॑किरिट॒-मति॑दन्तुरम् । अति॑मिर्मिर॒-मति॑मेमिषम् । आ॒शायै॑ जा॒मिम् । प्र॒ती॒क्षायै॑ कुमा॒रीम् । \textbf{ 19} \newline
                  \newline
                                     \textbf{(A19)} \newline \newline
                \textbf{Prapaataka korvai with starting  Words of 1 to 19 Anuvaakams :-} \newline
        (ब्रह्म॑णे - गी॒ताय॒ - श्रमा॑य - स॒न्धये॑ - न॒दीभ्य॑ - उथ्सा॒देभ्य॒ -ऋत्यै॒ - भाया॒ - अर्मे᳚भ्यो - म॒न्यवे॑ - य॒म्यै॑ दश॑ दश॒ - सरो᳚भ्यो॒ द्वाद॑श - प्रति॒श्रुत्का॑यै - बीभ॒थ्सायै॒ दश॑ दश॒ - हसा॑य स॒प्ता-क्ष॑रा॒जाय॒ त्रयो॑दश॒ - भूम्यै॒ दश॑ - वा॒चे षडथ॒ - नवैका॒न्न विꣳ॑शतिः ) \newline

        \textbf{korvai with starting Words of 1, 11, 21 Series of Dasinis :-} \newline
        (ब्रह्म॑णे - य॒म्यै॑ नव॑दश) \newline

        \textbf{first and last  Word 3rd Ashtakam 4th Prapaatakam :-} \newline
        (ब्रह्म॑णे - कुमा॒रीम्) \newline 

       

        ॥ हरिः॑ ॐ ॥
॥ कृष्ण यजुर्वेदीय तैत्तिरीय ब्राह्मणे तृतीयाष्टके चतुर्थ प्रपाठकः समाप्तः ॥
================================== \newline
        \pagebreak
        
        
        
     \addcontentsline{toc}{section}{ 3.5     तृतीयाष्टके पञ्चमः प्रपाठकः - इष्टिहौत्रम्}
     \markright{ 3.5     तृतीयाष्टके पञ्चमः प्रपाठकः - इष्टिहौत्रम् \hfill https://www.vedavms.in \hfill}
     \section*{ 3.5     तृतीयाष्टके पञ्चमः प्रपाठकः - इष्टिहौत्रम् }
                \textbf{ 3.5.1     अनुवाकं   1 - होतुर्जपः पुरस्ताथ् सामिधेनीनाम्} \newline
                                \textbf{ TB 3.5.1.1} \newline
                  स॒त्यं प्रप॑द्ये । ऋ॒तं प्रप॑द्ये । अ॒मृतं॒ प्रप॑द्ये । प्र॒जाप॑तेः प्रि॒यां त॒नुव॒मना᳚र्तां॒ प्रप॑द्ये । इ॒दम॒हं प॑ञ्चद॒शेन॒ वज्रे॑ण । द्वि॒षन्तं॒ भ्रातृ॑व्य॒मव॑क्रामामि । यो᳚ऽस्मान् द्वेष्टि॑ । यं च॑ व॒यं द्वि॒ष्मः । भूर्भुवः॒ सुवः॑ । हिम् । \textbf{ 1} \newline
                  \newline
                                    (स॒त्यं दश॑) \textbf{(A1)} \newline \newline
                \textbf{ 3.5.2     अनुवाकं   2 - सामिधेनीम्रन्वाह} \newline
                                \textbf{ TB 3.5.2.1} \newline
                  प्र वो॒ वाजा॑ अ॒भिद्य॑वः । ह॒विष्म॑न्तो घृ॒ताच्या᳚ । दे॒वाञ्जि॑गाति सुम्न॒युः ॥ अग्न॒ आया॑हि वी॒तये᳚ । गृ॒णा॒नो ह॒व्यदा॑तये । नि होता॑ सथ्सि ब॒र॒.हिषि॑ ॥ तं त्वा॑ स॒मिद्भि॑रङ्गिरः । घृ॒तेन॑ वर्द्धयामसि । बृ॒हच्छो॑चा यविष्ठ्य ॥ स नः॑ पृ॒थु श्र॒वाय्य᳚म् \textbf{ 2} \newline
                  \newline
                                \textbf{ TB 3.5.2.2} \newline
                  अच्छा॑ देव विवाससि । बृ॒हद॑ग्ने सु॒वीर्य᳚म् ॥ ई॒डेन्यो॑ नम॒स्य॑स्ति॒रः । तमाꣳ॑सि दर्.श॒तः । सम॒ग्निरि॑द्ध्यते॒ वृषा᳚ ॥ वृषो॑ अ॒ग्निः समि॑द्ध्यते । अश्वो॒ न दे॑व॒वाह॑नः । तꣳ ह॒विष्म॑न्त ईडते ॥ वृष॑णं त्वा व॒यं ॅवृषन्न्॑ । वृषा॑णः॒ समि॑धीमहि \textbf{ 3} \newline
                  \newline
                                \textbf{ TB 3.5.2.3} \newline
                  अग्ने॒ दीद्य॑तं बृ॒हत् ॥ अ॒ग्निं दू॒तं ॅवृ॑णीमहे । होता॑रं ॅवि॒श्ववे॑दसम् । अ॒स्य य॒ज्ञ्स्य॑ सु॒क्रतु᳚म् ॥ स॒मि॒द्ध्यमा॑नो अद्ध्व॒रे । अ॒ग्निः पा॑व॒क ईड्यः॑ । शो॒चिष्के॑श॒स्तमी॑महे ॥ समि॑द्धो अग्न आहुत । दे॒वान्. य॑क्षि स्वद्ध्वर । त्वꣳ हि ह॑व्य॒वाडसि॑ ( ) ॥ आजु॑होत दुव॒स्यत॑ । अ॒ग्निं प्र॑य॒त्य॑द्ध्व॒रे । वृ॒णी॒द्ध्वꣳ ह॑व्य॒वाह॑नम् ॥ त्वं ॅवरु॑ण उ॒त मि॒त्रो अ॑ग्ने । त्वां ॅव॑र्द्धन्ति म॒तिभि॒र्वसि॑ष्ठाः । त्वे वसु॑ सुषण॒नानि॑ सन्तु । यू॒यं पा॑त स्व॒स्तिभिः॒ सदा॑ नः । \textbf{ 4} \newline
                  \newline
                                    (श्र॒वाय्य॑ - मिधीम॒ - ह्यसि॑ स॒प्त च॑) \textbf{(A2)} \newline \newline
                \textbf{ 3.5.3     अनुवाकं   3 - प्ररमुक्त्वा निविदोऽनूच्य देवता आवाहयति} \newline
                                \textbf{ TB 3.5.3.1} \newline
                  अग्ने॑ म॒हाꣳ अ॑सि ब्राह्मण भारत । असा॒वसौ᳚ ॥ दे॒वेद्धो॒ मन्वि॑द्धः । ऋषि॑ष्टुतो॒ विप्रा॑नुमदितः । क॒वि॒श॒स्तो ब्रह्म॑सꣳशितो घृ॒ताह॑वनः । प्र॒णीर्-य॒ज्ञाना᳚म् । र॒थीर॑द्ध्व॒राणा᳚म् । अ॒तूर्तो॒ होता᳚ । तूर्णि॑र्.हव्य॒वाट् । आस्पात्रं॑ जु॒हूर्-दे॒वाना᳚म् \textbf{ 5} \newline
                  \newline
                                \textbf{ TB 3.5.3.2} \newline
                  च॒म॒सो दे॑व॒पानः॑ । अ॒राꣳ इ॑वाग्ने ने॒मिर्दे॒वाꣳस्त्वं प॑रि॒भूर॑सि । आव॑ह दे॒वान्. यज॑मानाय ॥ अ॒ग्निम॑ग्न॒ आव॑ह । सोम॒माव॑ह । अ॒ग्निमाव॑ह । प्र॒जाप॑ति॒माव॑ह । अ॒ग्नीषोमा॒वाव॑ह । इ॒न्द्रा॒ग्नी आव॑ह । इन्द्र॒माव॑ह ( ) । म॒हे॒न्द्रमाव॑ह । दे॒वाꣳ आ᳚ज्य॒पाꣳ आव॑ह । अ॒ग्निꣳ हो॒त्रायाव॑ह । स्वं म॑हि॒मान॒माव॑ह । आ चा᳚ग्ने दे॒वान्. वह॑ । सु॒यजा॑ च यज जातवेदः । \textbf{ 6} \newline
                  \newline
                                    (दे॒वाना॒-मिन्द्र॒माव॑ह॒ षट्च॑) \textbf{(A3)} \newline \newline
                \textbf{ 3.5.4     अनुवाकं   4 - स्रुचावादापयति} \newline
                                \textbf{ TB 3.5.4.1} \newline
                  अ॒ग्निर्. होता॒ वेत्व॒ग्निः । हो॒त्रं ॅवे᳚त्तु प्रावि॒त्रम् । स्मो व॒यम् । सा॒धु ते॑ यजमान दे॒वता᳚ । घृ॒तव॑ती-मद्ध्वर्यो॒ स्रुच॒माऽस्य॑स्व । दे॒वा॒युवं॑ ॅवि॒श्ववा॑राम् । ईडा॑महै दे॒वाꣳ ई॒डेन्यान्॑ । न॒म॒स्याम॑ नम॒स्यान्॑ । यजा॑म य॒ज्ञियान्॑ । \textbf{ 7} \newline
                  \newline
                                    (अ॒ग्निर्. होता॒ नव॑) \textbf{(A4)} \newline \newline
                \textbf{ 3.5.5     अनुवाकं   5 - प्रयाजाः} \newline
                                \textbf{ TB 3.5.5.1} \newline
                  स॒मिधो॑ अग्न॒ आज्य॑स्य वियन्तु । तनू॒नपा॑दग्न॒ आज्य॑स्य वेतु । इ॒डो अ॑ग्न॒ आज्य॑स्य वियन्तु । ब॒र्॒.हिर॑ग्न॒ आज्य॑स्य वेतु । स्वाहा॒ऽग्निम् । स्वाहा॒ सोम᳚म् । स्वाहा॒ऽग्निम् । स्वाहा᳚ प्र॒जाप॑तिम् । स्वाहा॒ऽग्नीषोमौ᳚ । स्वाहे᳚न्द्रा॒ग्नी ( ) । स्वाहेन्द्र᳚म् । स्वाहा॑ महे॒न्द्रम् । स्वाहा॑ दे॒वाꣳ आ᳚ज्य॒पान् । स्वाहा॒ऽग्निꣳ हो॒त्राज्जु॑षा॒णाः । अग्न॒ आज्य॑स्य वियन्तु । \textbf{ 8} \newline
                  \newline
                                    (इ॒न्द्रा॒ग्नी पञ्च॑ च) \textbf{(A5)} \newline \newline
                \textbf{ 3.5.6     अनुवाकं   6 - अज्यभागयोः पुरोनुवाक्ये} \newline
                                \textbf{ TB 3.5.6.1} \newline
                  अ॒ग्निर्-वृ॒त्राणि॑ जङ्घनत् । द्र॒वि॒ण॒स्युर्-वि॑प॒न्यया᳚ । समि॑द्धः शु॒क्र आहु॑तः । जु॒षा॒णो अ॒ग्निराज्य॑स्य वेतु । त्वꣳ सो॑मासि॒ सत्प॑तिः । त्वꣳ राजो॒त वृ॑त्र॒हा । त्वं भ॒द्रो अ॑सि॒ क्रतुः॑ । जु॒षा॒णः सोम॒ आज्य॑स्य ह॒विषो॑ वेतु । अ॒ग्निः प्र॒त्नेन॒ जन्म॑ना । शुम्भा॑नस्त॒नुवꣳ॒॒ स्वाम् ( ) । क॒विर्-विप्रे॑ण वावृधे । जु॒षा॒णो अ॒ग्निराज्य॑स्य वेतु । सोम॑ गी॒र्भिष्ट्वा॑ व॒यम् । व॒र्द्धया॑मो वचो॒विदः॑ । सु॒मृ॒डी॒को न॒ आवि॑श । जु॒षा॒णः सोम॒ आज्य॑स्य ह॒विषो॑ वेतु । \textbf{ 9} \newline
                  \newline
                                    (स्वाꣳ षट्च॑) \textbf{(A6)} \newline \newline
                \textbf{ 3.5.7     अनुवाकं   7 - हविषां याज्यानुवाक्याः} \newline
                                \textbf{ TB 3.5.7.1} \newline
                  अ॒ग्निर्-मू॒र्द्धा दि॒वः क॒कुत् । पतिः॑ पृथि॒व्या अ॒यम् । अ॒पाꣳ रेताꣳ॑सि जिन्वति । भुवो॑ य॒ज्ञ्स्य॒ रज॑सश्च ने॒ता । यत्रा॑ नि॒युद्भिः॒ सच॑से शि॒वाभिः॑ । दि॒वि मू॒र्द्धानं॑ दधिषे सुव॒र्॒.षाम् । जि॒ह्वाम॑ग्ने चकृषे हव्य॒वाह᳚म् । प्रजा॑पते॒ न त्वदे॒तान्य॒न्यः । विश्वा॑ जा॒तानि॒ परि॒ ता ब॑भूव । यत्-का॑मास्ते जुहु॒मस्तन्नो॑ अस्तु \textbf{ 10} \newline
                  \newline
                                \textbf{ TB 3.5.7.2} \newline
                  व॒यꣳ स्या॑म॒ पत॑यो रयी॒णाम् । स वे॑द पु॒त्रः पि॒तरꣳ॒॒ स मा॒तर᳚म् । स सू॒नुर्भु॑व॒थ्स भु॑व॒त्पुन॑र्मघः । स द्यामौर्णो॑द॒न्तरि॑क्षꣳ॒॒ स सुवः॑ । स विश्वा॒ भुवो॑ अभव॒थ्स आऽभ॑वत् । अग्नी॑षोमा॒ सवे॑दसा । सहू॑ती वनतं॒ गिरः॑ । सं दे॑व॒त्रा ब॑भूवथुः । यु॒वमे॒तानि॑ दि॒वि रो॑च॒नानि॑ । अ॒ग्निश्च॑ सोम॒ सक्र॑तू अधत्तम् \textbf{ 11} \newline
                  \newline
                                \textbf{ TB 3.5.7.3} \newline
                  यु॒वꣳ सिन्धूꣳ॑ र॒भिश॑स्ते-रव॒द्यात् । अग्नी॑षोमा॒-वमु॑ञ्चतं गृभी॒तान् । इन्द्रा᳚ग्नी रोच॒ना दि॒वः । परि॒ वाजे॑षु भूषथः । तद्वां᳚ चेति॒ प्रवी॒र्य᳚म् । श्नथ॑द्वृ॒त्रमु॒त स॑नोति॒ वाज᳚म् । इन्द्रा॒ यो अ॒ग्नी सहु॑री सप॒र्यात् । इ॒र॒ज्यन्ता॑ वस॒व्य॑स्य॒ भूरेः᳚ । सह॑स्तमा॒ सह॑सा वाज॒यन्ता᳚ । एन्द्र॑ सान॒सिꣳ र॒यिम् \textbf{ 12} \newline
                  \newline
                                \textbf{ TB 3.5.7.4} \newline
                  स॒जित्वा॑नꣳ सदा॒सह᳚म् । वर्.षि॑ष्ठमू॒तये॑ भर । प्रस॑साहिषे पुरुहूत॒ शत्रून्॑ । ज्येष्ठ॑स्ते॒ शुष्म॑ इ॒ह रा॒तिर॑स्तु । इन्द्राभ॑र॒ दक्षि॑णेना॒ वसू॑नि । पतिः॒ सिन्धू॑नामसि रे॒वती॑नाम् । म॒हाꣳ इन्द्रो॒ य ओज॑सा । प॒र्जन्यो॑ वृष्टि॒माꣳ इ॑व । स्तोमै᳚र्व॒थ्सस्य॑ वावृधे । म॒हां इन्द्रो॑ नृ॒वदाच॑र्.षणि॒प्राः \textbf{ 13} \newline
                  \newline
                                \textbf{ TB 3.5.7.5} \newline
                  उ॒त द्वि॒बर्.हा॑ अमि॒नः सहो॑भिः । अ॒स्म॒द्रिय॑ग्वावृधे वी॒र्या॑य । उ॒रुः पृ॒थुः सुकृ॑तः क॒र्तृभि॑र्भूत् । पि॒प्री॒हि दे॒वाꣳ उ॑श॒तो य॑विष्ठ । वि॒द्वां ऋतूꣳर्. ऋ॑तुपते यजे॒ह । ये दैव्या॑ ऋ॒त्विज॒स्तेभि॑रग्ने । त्वꣳ होतॄ॑णाम॒स्याय॑जिष्ठः ॥ अ॒ग्निꣳ स्वि॑ष्ट॒कृत᳚म् । अया॑ड॒ग्निर॒ग्नेः प्रि॒या धामा॑नि । अया॒ट्थ्सोम॑स्य प्रि॒या धामा॑नि \textbf{ 14} \newline
                  \newline
                                \textbf{ TB 3.5.7.6} \newline
                  अया॑ड॒ग्नेः प्रि॒या धामा॑नि । अया᳚ट्प्र॒जाप॑तेः प्रि॒या धामा॑नि । अया॑ड॒ग्नीषोम॑योः प्रि॒या धामा॑नि । अया॑डिन्द्राग्नि॒योः प्रि॒या धामा॑नि । अया॒डिन्द्र॑स्य प्रि॒या धामा॑नि । अया᳚ण्महे॒न्द्रस्य॑ प्रि॒या धामा॑नि । अया᳚ड्दे॒वाना॑-माज्य॒पानां᳚ प्रि॒या धामा॑नि । यक्ष॑द॒ग्नेर्. होतुः॑ प्रि॒या धामा॑नि । यक्ष॒थ्स्वं म॑हि॒मान᳚म् । आय॑जता॒मेज्या॒ इषः॑ ( ) । कृ॒णोतु॒ सो अ॑द्ध्व॒रा जा॒तवे॑दाः । जु॒षताꣳ॑ ह॒विः । अग्ने॒ यद॒द्य वि॒शो अ॑द्ध्वरस्य होतः । पाव॑क शोचे॒ वेष्ट्वꣳ हि यज्वा᳚ । ऋ॒ता य॑जासि महि॒ना वि यद्भूः । ह॒व्या व॑ह यविष्ठ॒ या ते॑ अ॒द्य । \textbf{ 15} \newline
                  \newline
                                    (अ॒स्त्व॒ - ध॒त्तꣳ॒ - र॒यिं - च॑र्.षणि॒प्राः - सोम॑स्य प्रि॒या धामा॒ - नीषः॒ षट्च॑) \textbf{(A7)} \newline \newline
                \textbf{ 3.5.8     अनुवाकं   8 - इडोपाह्वानाम्} \newline
                                \textbf{ TB 3.5.8.1} \newline
                  उप॑हूतꣳ रथन्त॒रꣳ स॒ह पृ॑थि॒व्या । उप॑ मा रथन्त॒रꣳ स॒ह पृ॑थि॒व्या ह्व॑यताम् । उप॑हूतं ॅवामदे॒व्यꣳ स॒हान्तरि॑क्षेण । उप॑ मा वामदे॒व्यꣳ स॒हान्तरि॑क्षेण ह्वयताम् । उप॑ हूतं बृ॒हथ्स॒ह दि॒वा । उप॑ मा बृ॒हथ्स॒ह दि॒वा ह्व॑यताम् । उप॑हूताः स॒प्त होत्राः᳚ । उप॑ मा स॒प्त होत्रा᳚ ह्वयन्ताम् । उप॑हूता धे॒नुः स॒हर्.ष॑भा । उप॑ मा धे॒नुः स॒हर्.ष॑भा ह्वयताम् \textbf{ 16} \newline
                  \newline
                                \textbf{ TB 3.5.8.2} \newline
                  उप॑हूतो भ॒क्षः सखा᳚ । उप॑ मा भ॒क्षः सखा᳚ ह्वयताम् । उप॑हू॒ताॅ(4) हो । इडोप॑हूता । उप॑हू॒तेडा᳚ । उपो॑ अ॒स्माꣳ इडा᳚ ह्वयताम् । इडोप॑हूता । उप॑हू॒तेडा᳚ । मा॒न॒वी घृ॒तप॑दी मैत्रावरु॒णी । ब्रह्म॑ दे॒वकृ॑त॒मुप॑हूतम् \textbf{ 17} \newline
                  \newline
                                \textbf{ TB 3.5.8.3} \newline
                  दैव्या॑ अद्ध्व॒र्यव॒ उप॑हूताः । उप॑हूता मनु॒ष्याः᳚ । य इ॒मं ॅय॒ज्ञ्मवान्॑ । ये य॒ज्ञ्प॑तिं॒ ॅवर्द्धान्॑ । उप॑हूते॒ द्यावा॑पृथि॒वी । पू॒र्व॒जे ऋ॒ताव॑री । दे॒वी दे॒वपु॑त्रे । उप॑हूतो॒ऽयं ॅयज॑मानः । उत्त॑रस्यां देवय॒ज्याया॒-मुप॑हूतः । भूय॑सि हवि॒ष्कर॑ण॒ उप॑हूतः ( ) । दि॒व्ये धाम॒न्नुप॑हूतः । इ॒दं मे॑ दे॒वा ह॒विर्जु॑षन्ता॒मिति॒ तस्मि॒न्नुप॑हूतः । विश्व॑मस्य प्रि॒यमुप॑हूतम् । विश्व॑स्य प्रि॒यस्योप॑हूत॒स्योप॑हूतः । \textbf{ 18} \newline
                  \newline
                                    (स॒हर्.ष॑भा ह्वयता॒ - मुप॑हूतꣳ - हवि॒ष्कर॑ण॒ उप॑हूतश्च॒त्वारि॑ च) \textbf{(A8)} \newline \newline
                \textbf{ 3.5.9     अनुवाकं   9 - अनूयाजाः} \newline
                                \textbf{ TB 3.5.9.1} \newline
                  दे॒वं ब॒र्॒.हिः । व॒सु॒वने॑ वसु॒धेय॑स्य वेतु । दे॒वो नरा॒शꣳ सः॑ । व॒सु॒वने॑ वसु॒धेय॑स्य वेतु । दे॒वो अ॒ग्निः स्वि॑ष्ट॒कृत् । सु॒द्रवि॑णा म॒न्द्रः क॒विः । स॒त्यम॑न्माऽऽय॒जी होता᳚ । होतु॑र्. होतु॒राय॑जीयान् । अग्ने॒ यान्-दे॒वानया᳚ट् । याꣳ अपि॑प्रेः ( ) । ये ते॑ हो॒त्रे अम॑थ्सत । ताꣳ स॑स॒नुषीꣳ॒॒ होत्रां᳚ देवं ग॒माम् । दि॒वि दे॒वेषु॑ य॒ज्ञ्मेर॑ये॒मम् । स्वि॒ष्ट॒कृच्चाग्ने॒ होताऽभूः᳚ । व॒सु॒वने॑ वसु॒धेय॑स्य नमोवा॒के वीहि॑ । \textbf{ 19} \newline
                  \newline
                                    (अपि॑प्रेः॒ पञ्च॑ च) \textbf{(A9)} \newline \newline
                \textbf{ 3.5.10    अनुवाकं   10 - सूक्तावाकः} \newline
                                \textbf{ TB 3.5.10.1} \newline
                  इ॒दं द्या॑वापृथिवी भ॒द्रम॑भूत् । आर्द्ध्म॑ सूक्तवा॒कम् । उ॒त न॑मोवा॒कम् । ऋ॒द्ध्यास्म॑ सू॒क्तोच्य॑मग्ने । त्वꣳ सू᳚क्त॒वाग॑सि । उप॑श्रितो दि॒वः पृ॑थि॒व्योः । ओम॑न्वती ते॒ऽस्मिन्. य॒ज्ञे य॑जमान॒ द्यावा॑पृथि॒वी स्ता᳚म् । श॒गं॒ये जी॒रदा॑नू । अत्र॑स्नू॒ अप्र॑वेदे । उ॒रुग॑व्यूती अभयं॒ कृतौ᳚ \textbf{ 20} \newline
                  \newline
                                \textbf{ TB 3.5.10.2} \newline
                  वृ॒ष्टिद्या॑वा री॒त्या॑पा । श॒म्भुवौ॑ मयो॒भुवौ᳚ । ऊर्ज॑स्वती च॒ पय॑स्वती च । सू॒प॒च॒र॒णा च॑ स्वधिचर॒णा च॑ ।  तयो॑रा॒विदि॑ ॥ अ॒ग्निरि॒दꣳ ह॒विर॑जुषत । अवी॑वृधत॒ महो॒ ज्यायो॑ऽकृत । सोम॑ इ॒दꣳ ह॒विर॑जुषत । अवी॑वृधत॒ महो॒ ज्यायो॑ऽकृत । अ॒ग्निरि॒दꣳ ह॒विर॑जुषत \textbf{ 21} \newline
                  \newline
                                \textbf{ TB 3.5.10.3} \newline
                  अवी॑वृधत॒ महो॒ ज्यायो॑ऽकृत । प्र॒जाप॑तिरि॒दꣳ ह॒विर॑जुषत । अवी॑वृधत॒ महो॒ ज्यायो॑ऽकृत । अ॒ग्नीषोमा॑वि॒दꣳ ह॒विर॑जुषेताम् । अवी॑वृधेतां॒ महो॒ ज्यायो᳚ऽक्राताम् । इ॒न्द्रा॒ग्नी इ॒दꣳ ह॒विर॑जुषेताम् । अवी॑वृधेतां॒ महो॒ ज्यायो᳚ऽक्राताम् । इन्द्र॑ इ॒दꣳ ह॒विर॑जुषत । अवी॑वृधत॒ महो॒ ज्यायो॑ऽकृत । म॒हे॒न्द्र इ॒दꣳ ह॒विर॑जुषत \textbf{ 22} \newline
                  \newline
                                \textbf{ TB 3.5.10.4} \newline
                  अवी॑वृधत॒ महो॒ ज्यायो॑ऽकृत । दे॒वा आ᳚ज्य॒पा आज्य॑मजुषन्त । अवी॑वृधन्त॒ महो॒ ज्यायो᳚ऽक्रत । अ॒ग्निर्.हो॒त्रेणे॒दꣳ ह॒विर॑जुषत । अवी॑वृधत॒ महो॒ ज्यायो॑ऽकृत । अ॒स्यामृध॒-द्धोत्रा॑यां देवंग॒माया᳚म् । आशा᳚स्ते॒ऽयं ॅयज॑मानो॒ऽसौ ॥ आयु॒राशा᳚स्ते । सु॒प्र॒जा॒स्त्व-माशा᳚स्ते । स॒जा॒त॒व॒न॒स्या-माशा᳚स्ते \textbf{ 23} \newline
                  \newline
                                \textbf{ TB 3.5.10.5} \newline
                  उत्त॑रां देवय॒ज्या-माशा᳚स्ते । भूयो॑ हवि॒ष्कर॑ण॒-माशा᳚स्ते । दि॒व्यं धामाशा᳚स्ते । विश्वं॑ प्रि॒यमाशा᳚स्ते । यद॒नेन॑ ह॒विषा ऽऽशा᳚स्ते । तद॑श्या॒त्-तदृ॑द्ध्यात् । तद॑स्मै दे॒वा रा॑सन्ताम् । तद॒ग्निर्-दे॒वो दे॒वेभ्यो॒ वन॑ते । व॒यम॒ग्नेर् मानु॑षाः । इ॒ष्टं च॑ वी॒तं च॑ ( ) । उ॒भे च॑ नो॒ द्यावा॑पृथि॒वी अꣳह॑सस्पाताम् । इ॒ह गति॑र्-वा॒मस्ये॒दं च॑ । नमो॑ दे॒वेभ्यः॑ । \textbf{ 24} \newline
                  \newline
                                    (अ॒भ॒यं॒कृता॑वकृ - ता॒ग्निरि॒दꣳ ह॒विर॑जुषत - महे॒न्द्र इ॒दꣳ ह॒विर॑जुषत - सजातवन॒स्यामाशा᳚स्ते - वी॒तं च॒ त्रीणि॑ च) \textbf{(A10)} \newline \newline
                \textbf{ 3.5.11    अनुवाकं   11 - शंयुवाकः} \newline
                                \textbf{ TB 3.5.11.1} \newline
                  तच्छं॒ ॅयोरा वृ॑णीमहे । गा॒तुं ॅय॒ज्ञाय॑ । गा॒तुं ॅय॒ज्ञ्प॑तये । दैवी᳚ स्व॒स्तिर॑स्तु नः । स्व॒स्तिर् मानु॑षेभ्यः । ऊ॒र्द्ध्वं जि॑गातु भेष॒जम् । शं नो॑ अस्तु द्वि॒पदे᳚ । शं चतु॑ष्पदे । \textbf{ 25} \newline
                  \newline
                                    (तच्छं॒ ॅयोर॒ष्टौ) \textbf{(A11)} \newline \newline
                \textbf{ 3.5.12    अनुवाकं   12 - पत्नीसंयाजानां याज्यानुवाक्याः} \newline
                                \textbf{ TB 3.5.12.1} \newline
                  “आ प्या॑यस्व॒{1}” “सं ते᳚ {2}” ॥ “इ॒ह त्वष्टा॑रमग्रि॒यं{3}” “तन्न॑स्तु॒रीप᳚म् {4}” ॥ दे॒वानां॒ पत्नी॑रुश॒तीर॑वन्तु नः । प्राव॑न्तु नस्तु॒जये॒ वाज॑सातये । याः पार्थि॑वासो॒ या अ॒पामपि॑ व्र॒ते । ता नो॑ देवीः सुहवाः॒ शर्म॑ यच्छत ॥ उ॒त ग्ना वि॑यन्तु दे॒वप॑त्नीः । इ॒न्द्रा॒ण्य॑ग्नाय्य॒श्विनी॒ राट् । आ रोद॑सी वरुणा॒नी शृ॑णोतु । वि॒यन्तु॑ दे॒वीर्य ऋ॒तुर्जनी॑नाम् ( ) । \textbf{ 26} \newline
                  \newline
                                \textbf{ TB 3.5.12.2} \newline
                  अ॒ग्निर्. होता॑ गृ॒हप॑तिः॒ स राजा᳚ । विश्वा॑ वेद॒ जनि॑मा जा॒तवे॑दाः । दे॒वाना॑मु॒त यो मर्त्या॑नाम् । यजि॑ष्ठः॒ स प्रय॑जतामृ॒तावा᳚ ॥ व॒यमु॑ त्वा गृहपते॒ जना॑नाम् । अग्ने॒ अक॑र्म स॒मिधा॑ बृ॒हन्त᳚म् । अ॒स्थू॒रिणो॒ गार्.ह॑पत्यानि सन्तु । ति॒ग्मेन॑ न॒स्तेज॑सा॒ सꣳशि॑शाधि । \textbf{ 27} \newline
                  \newline
                                    (जनी॑ना - +म॒ष्टौ च॑) \textbf{(A12)} \newline \newline
                \textbf{ 3.5.13    अनुवाकं   13 - पत्नीसंयाजे सोमाह्वानम्} \newline
                                \textbf{ TB 3.5.13.1} \newline
                  उप॑हूतꣳ रथन्त॒रꣳ स॒ह पृ॑थि॒व्या । उप॑ मा रथन्त॒रꣳ स॒ह पृ॑थि॒व्या ह्व॑यताम् । उप॑हूतं ॅवामदे॒व्यꣳ स॒हान्तरि॑क्षेण । उप॑ मा वामदे॒व्यꣳ स॒हान्तरि॑क्षेण ह्वयताम् । उप॑हूतं बृ॒हथ्स॒ह दि॒वा । उप॑ मा बृ॒हथ्स॒ह दि॒वा ह्व॑यताम् । उप॑हूताः स॒प्त होत्राः᳚ । उप॑ मा स॒प्त होत्रा᳚ ह्वयन्ताम् । उप॑हूता धे॒नुः स॒हर्.ष॑भा । उप॑ मा धे॒नुः स॒हर्.ष॑भा ह्वयताम् \textbf{ 28} \newline
                  \newline
                                \textbf{ TB 3.5.13.2} \newline
                  उप॑हूतो भ॒क्षः सखा᳚ । उप॑ मा भ॒क्षः सखा᳚ ह्वयताम् । उप॑हू॒ताॅ(4) हो । इडोप॑हूता । उप॑हू॒तेडा᳚ । उपो॑ अ॒स्माꣳ इडा᳚ ह्वयताम् । इडोप॑हूता । उप॑हू॒तेडा᳚ । मा॒न॒वी घृ॒तप॑दी मैत्रावरु॒णी । ब्रह्म॑ दे॒वकृ॑त॒मुप॑हूतम् \textbf{ 29} \newline
                  \newline
                                \textbf{ TB 3.5.13.3} \newline
                  दैव्या॑ अद्ध्व॒र्यव॒ उप॑हूताः । उप॑हूता मनु॒ष्याः᳚ । य इ॒मं ॅय॒ज्ञ्मवान्॑ । ये य॒ज्ञ्प॑त्नीं॒ ॅवर्द्धान्॑ । उप॑हूते॒ द्यावा॑पृथि॒वी । पू॒र्व॒जे ऋ॒ताव॑री । दे॒वी दे॒वपु॑त्रे । उप॑हूते॒यं ॅयज॑माना । इ॒न्द्रा॒णीवा॑विध॒वा । अदि॑तिरिव सुपु॒त्रा ( ) । उत्त॑रस्यां देवय॒ज्याया॒-मुप॑हूता । भूय॑सि हवि॒ष्कर॑ण॒ उप॑हूता । दि॒व्ये धाम॒न्नुप॑हूता । इ॒दं मे॑ दे॒वा ह॒विर्-जु॑षन्ता॒मिति॒ तस्मि॒न्-नुप॑हूता । विश्व॑मस्याः प्रि॒यमुप॑हूतम् । विश्व॑स्य प्रि॒यस्योप॑हूत॒स्योप॑हूता । \textbf{ 30} \newline
                  \newline
                                    (स॒हर्.ष॑भा ह्वयता॒ - मुप॑हूतꣳ - सुपु॒त्रा षट्च॑) \textbf{(A13)} \newline \newline
                \textbf{Prapaataka korvai with starting  Words of 1 to 13 Anuvaakams :-} \newline
        (स॒त्यं - प्रवो - ऽग्ने॑ म॒हा - न॒ग्निर्. होता॑ - स॒मिधो॒ - ऽग्निर् वृ॒त्राण्य॒ - ग्निर् मू॒र्द्धो - प॑हूतं - दे॒वं ब॒र्॒.हि - रि॒दं द्या॑वापृथिवी॒ -तच्छ॒म्ॅयो- रा प्या॑य॒स्वो - प॑हूतं॒ त्रयो॑दश) \newline

        \textbf{korvai with starting Words of 1, 11, 21 Series of Dasinis :-} \newline
        (स॒त्यं - ॅव॒यꣳ स्या॑म - वृ॒ष्टिद्या॑वा त्रिꣳ॒॒शत्) \newline

        \textbf{first and last  Word 3rd Ashtakam 5th Prapaatakam :-} \newline
        (स॒त्य-मुप॑हूता) \newline 

       

        ॥ हरिः॑ ॐ ॥
॥ कृष्ण यजुर्वेदीय तैत्तिरीय ब्राह्मणे तृतीयाष्टके पञ्चम प्रपाठकः समाप्तः ॥
Appendix (of Expansions)
ट्.भ्.3.5.12.1 - "आ प्या॑यस्व॒ {1}" "सं ते᳚ {2}" 
आ प्या॑यस्व॒ समे॑तु ते वि॒श्वतः॑ सोम॒ वृष्णि॑यं । 
भवा॒ वाज॑स्य सङ्ग॒थे ॥ {1}
सं ते॒ पयाꣳ ॑सि॒ समु॑ यन्तु॒ वाजाः॒ सं ॅवृष्णि॑या-न्यभिमाति॒षाहः॑ । 
आ॒प्याय॑मानो अ॒मृता॑य सोम दि॒वि श्रवाꣳ॑स्युत्त॒मानि॑ धिष्व ॥ {2} 
(Both {1} and {2} appearing in TS 4.2.7.4) 

ट्.भ्.3.5.12.1 - "इ॒ह त्वष्टा॑रमग्रि॒यं {3}" "तन्न॑स्तु॒रीप᳚म् {4}" 
इ॒ह त्वष्टा॑रमग्रि॒यं ॅवि॒श्वरू॑प॒मुप॑ ह्वये । अ॒स्माक॑मस्तु॒ केव॑लः ॥ {3}
तन्न॑स्तु॒रीप॒मध॑ पोषयि॒त्नु देव॑ त्वष्ट॒र्वि र॑रा॒णः स्य॑स्व । 
यतो॑ वी॒रः क॑र्म॒ण्यः॑ सु॒दक्षो॑ यु॒क्तग्रा॑वा॒ जाय॑ते दे॒वका॑मः ॥ {4}
(Both {3} and {4} appearing in TS 3.1.11.1) 
====================================== \newline
        \pagebreak
        
        
        
     \addcontentsline{toc}{section}{ 3.6     तृतीयाष्टके षष्ठः प्रपाठकः - पशुहौत्रम्}
     \markright{ 3.6     तृतीयाष्टके षष्ठः प्रपाठकः - पशुहौत्रम् \hfill https://www.vedavms.in \hfill}
     \section*{ 3.6     तृतीयाष्टके षष्ठः प्रपाठकः - पशुहौत्रम् }
                \textbf{ 3.6.1     अनुवाकं   1 - यूपसंस्काराः} \newline
                                \textbf{ TB 3.6.1.1} \newline
                  अ॒ञ्जन्ति॒ त्वाम॑द्ध्व॒रे दे॑व॒यन्तः॑ । वन॑स्पते॒ मधु॑ना॒ दैव्ये॑न । यदू॒र्द्ध्वस्ति॑ष्ठा॒द्-द्रवि॑णे॒ह ध॑त्तात् । यद्वा॒ क्षयो॑ मा॒तुर॒स्या उ॒पस्थे᳚ ॥ उच्छ्र॑यस्व वनस्पते । वर्ष्म॑न् पृथि॒व्या अधि॑ । सुमि॑ती मी॒यमा॑नः । वर्चो॑धा य॒ज्ञ्वा॑हसे ॥ समि॑द्धस्य॒ श्रय॑माणः पु॒रस्ता᳚त् । ब्रह्म॑ वन्वा॒नो अ॒जरꣳ॑ सु॒वीर᳚म् \textbf{ 1} \newline
                  \newline
                                \textbf{ TB 3.6.1.2} \newline
                  आ॒रे अ॒स्मदम॑तिं॒ बाध॑मानः । उच्छ्र॑यस्व मह॒ते सौभ॑गाय ॥ ऊ॒र्द्ध्व ऊ॒षुण॑ ऊ॒तये᳚ । तिष्ठा॑ दे॒वो न स॑वि॒ता । ऊ॒र्द्ध्वो वाज॑स्य॒ सनि॑ता॒ यद॒ञ्जिभिः॑ । वा॒घद्भि॑र्-वि॒ह्वया॑महे ॥ ऊ॒र्द्ध्वो नः॑ पा॒ह्यꣳह॑सो॒ नि के॒तुना᳚ । विश्वꣳ॒॒ सम॒त्रिणं॑ दह । कृ॒धी न॑ ऊ॒र्द्ध्वाञ्च॒ रथा॑य जी॒वसे᳚ । वि॒दा दे॒वेषु॑ नो॒ दुवः॑ । \textbf{ 2} \newline
                  \newline
                                \textbf{ TB 3.6.1.3} \newline
                  जा॒तो जा॑यते सुदिन॒त्वे अह्ना᳚म् । सम॒र्य आ वि॒दथे॒ वर्द्ध॑मानः । पु॒नन्ति॒ धीरा॑ अ॒पसो॑ मनी॒षा । दे॒व॒या विप्र॒ उदि॑यर्ति॒ वाच᳚म् ॥ युवा॑ सु॒वासाः॒ परि॑वीत॒ आगा᳚त् । स उ॒ श्रेया᳚न् भवति॒ जाय॑मानः । तं धीरा॑सः क॒वय॒ उन्न॑यन्ति । स्वा॒धियो॒ मन॑सा देव॒यन्तः॑ ॥ पृ॒थु॒पाजा॒ अम॑र्त्यः । घृ॒तनि॑र्णि॒ख्-स्वा॑हुतः ( ) \textbf{ 3} \newline
                  \newline
                                \textbf{ TB 3.6.1.4} \newline
                  अ॒ग्निर्-य॒ज्ञ्स्य॑ हव्य॒वाट् ॥ तꣳ स॒बाधो॑ य॒तस्रु॑चः । इ॒त्था धि॒या य॒ज्ञ्व॑न्तः । आच॑क्रुर॒ग्नि-मू॒तये᳚ ॥ त्वं ॅवरु॑ण उ॒त मि॒त्रो अ॑ग्ने । त्वां ॅव॑र्द्धन्ति म॒तिभि॒र्-वसि॑ष्ठाः । त्वे वसु॑ सुषण॒नानि॑ सन्तु । यू॒यं पा॑त स्व॒स्तिभिः॒ सदा॑ नः । \textbf{ 4} \newline
                  \newline
                                    (सु॒वीरं॒ - दुवः॒ - स्वा॑हुतो॒ - +ऽष्टौ च॑) \textbf{(A1)} \newline \newline
                \textbf{ 3.6.2     अनुवाकं   2 - प्रयाजविषया मैत्रावरुणप्रैषाः} \newline
                                \textbf{ TB 3.6.2.1} \newline
                  होता॑ यक्षद॒ग्निꣳ स॒मिधा॑ सुष॒मिधा॒ समि॑द्धं॒ नाभा॑ पृथि॒व्याः स॑गं॒थे वा॒मस्य॑ । वर्ष्म॑न्दि॒व इ॒डस्प॒दे वेत्वाज्य॑स्य॒ होत॒र्यज॑ ॥ होता॑ यक्ष॒त्-तनू॒नपा॑त॒-मदि॑ते॒र्-गर्भं॒ भुव॑नस्य गो॒पाम् । मद्ध्वा॒ऽद्य दे॒वो दे॒वेभ्यो॑ देव॒याना᳚न्प॒थो अ॑नक्तु॒ वेत्वाज्य॑स्य॒ होत॒र्यज॑ ॥ होता॑ यक्ष॒न्-नरा॒शꣳसं॑ नृश॒स्त्रं नॄꣳः प्र॑णेत्रम् । गोभि॑र्-व॒पावा॒न्थ्-स्याद्-वी॒रैः शक्ती॑वा॒न्-रथैः᳚ प्रथम॒यावा॒ हिर॑ण्यैश्च॒न्द्री वेत्वाज्य॑स्य॒ होत॒र्यज॑ ॥ होता॑ यक्षद॒ग्निमि॒ड ई॑डि॒तो दे॒वो दे॒वाꣳ आव॑क्षद्-दू॒तो ह॑व्य॒वाडमू॑रः । उपे॒मं ॅय॒ज्ञ्मुपे॒मां दे॒वो दे॒वहू॑तिमवतु॒ वेत्वाज्य॑स्य॒ होत॒र्यज॑ ॥ होता॑ यक्षद्-ब॒र्॒.हिः सु॒ष्टरी॒मोर्ण॑म्रदा अ॒स्मिन्. य॒ज्ञे वि च॒ प्र च॑ प्रथताꣳ स्वास॒स्थं दे॒वेभ्यः॑ । एमे॑नद॒द्य वस॑वो रु॒द्रा आ॑दित्याः स॑दन्तु प्रि॒यमिन्द्र॑स्यास्तु॒ वेत्वाज्य॑स्य॒ होत॒र्यज॑ । \textbf{ 5} \newline
                  \newline
                                \textbf{ TB 3.6.2.2} \newline
                  होता॑ यक्ष॒द्-दुर॑ ऋ॒ष्वाः क॑व॒ष्योऽको॑षधावनी॒-रुदाता॑भि॒र्-जिह॑तां॒ ॅवि पक्षो॑भिः श्रयन्ताम् । सु॒प्रा॒य॒णा अ॒स्मिन्. य॒ज्ञे विश्र॑यन्ता-मृता॒वृधो॑ वि॒यन्त्वाज्य॑स्य॒ होत॒र्यज॑ ॥ होता॑ यक्षदु॒षासा॒नक्ता॑ बृह॒ती सु॒पेश॑सा॒ नॄꣳः पति॑भ्यो॒ योनिं॑ कृण्वा॒ने । सꣳ॒॒स्मय॑माने॒ इन्द्रे॑ण दे॒वैरेदं ब॒र्॒.हिः सी॑दतां ॅवी॒तामाज्य॑स्य॒ होत॒र्यज॑ ॥ होता॑ यक्ष॒द्-दैव्या॒ होता॑रा म॒न्द्रा पोता॑रा क॒वी प्रचे॑तसा । स्वि॑ष्टम॒द्यान्यः क॑रदि॒षा स्व॑भिगूर्तम॒न्य ऊ॒र्जा सत॑वसे॒मं ॅय॒ज्ञ्ं दि॒वि दे॒वेषु॑ धत्तां ॅवी॒तामाज्य॑स्य॒ होत॒र्यज॑ ॥ होता॑ यक्षत्-ति॒स्रो दे॒वीर॒पसा॑म॒पस्त॑मा॒ अच्छि॑द्र-म॒द्येदमप॑-स्तन्वताम् । दे॒वेभ्यो॑ दे॒वीर्-दे॒वमपो॑ वि॒यन्त्वाज्य॑स्य॒ होत॒र्यज॑ ॥ होता॑ यक्ष॒त्-त्वष्टा॑र॒-मचि॑ष्टु॒मपा॑कꣳ रेतो॒धां ॅविश्र॑वसं ॅयशो॒धाम् । पु॒रु॒रूप॒मका॑मकर्.शनꣳ सु॒पोषः॒ पोषैः॒ स्याथ् सु॒वीरो॑ वी॒रैर्-वेत्वाज्य॑स्य॒ होत॒र्यज॑ ( ) ॥ होता॑ यक्ष॒द्-वन॒स्पति॑-मु॒पाव॑स्रक्षद्धि॒यो जो॒ष्टारꣳ॑ श॒शम॒न्नरः॑ । स्वदा॒थ् स्वधि॑तिर्. ऋतु॒थाऽद्य दे॒वो दे॒वेभ्यो॑ ह॒व्याऽवा॒ड्-वेत्वाज्य॑स्य॒ होत॒र्यज॑ ॥ होता॑ यक्षद॒ग्निꣳ स्वाहाऽऽज्य॑स्य॒ स्वाहा॒ मेद॑सः॒ स्वाहा᳚ स्तो॒कानाꣳ॒॒ स्वाहा॒ स्वाहा॑ कृतीनाꣳ॒॒ स्वाहा॑ ह॒व्यसू᳚क्तीनाम् । स्वाहा॑ दे॒वाꣳ आ᳚ज्य॒पान्थ् स्वाहा॒ऽग्निꣳ हो॒त्राज्जु॑षा॒णा अग्न॒ आज्य॑स्य वियन्तु॒ होत॒र्यज॑ । \textbf{ 6} \newline
                  \newline
                                                        \textbf{special korvai} \newline
              (अ॒ग्निं तनू॒नपा॑तं॒ नरा॒शꣳस॑म॒ग्निमि॒ड ई॑डि॒तो ब॒र्॒.हिर्दुर॑ उ॒षासा॒नक्ता॒ दैव्या॑ ति॒स्रस्त्वष्टा॑रं॒ ॅवन॒स्पति॑म॒ग्निम् ) (पञ्च॒ वेत्वेको॑ वि॒यन्तु॒ द्विर्वी॒तामेको॑ वि॒यन्तु॒ द्विर्वेत्वेको॑ वियन्तु॒ होत॒र्यज॑) \newline
                                (प्रि॒यमिन्द्र॑स्यास्तु॒ वेत्वाज्य्॑स्य॒ होत॒र्यज॑ - सु॒वीरो॑ वी॒रैर् वेत्वाज्य॑स्य॒ होत॒र्यज॑ च॒त्वारि॑ च) \textbf{(A2)} \newline \newline
                \textbf{ 3.6.3     अनुवाकं   3 - आप्रिय प्रयाजानां याज्याः} \newline
                                \textbf{ TB 3.6.3.1} \newline
                  समि॑द्धो अ॒द्य मनु॑षो दुरो॒णे । दे॒वो दे॒वान्. य॑जसि जातवेदः । आ च॒ वह॑ मित्रमह-श्चिकि॒त्वान् । त्वं दू॒तः क॒विर॑सि॒ प्रचे॑ताः ॥ तनू॑नपात्प॒थ ऋ॒तस्य॒ यानान्॑ । मद्ध्वा॑ सम॒ञ्जन्थ् स्व॑दया सुजिह्व । मन्मा॑नि धी॒भिरु॒त य॒ज्ञ्मृ॒न्धन्न् । दे॒व॒त्रा च॑ कृणुह्यद्ध्व॒रं नः॑ ॥ नरा॒शꣳस॑स्य महि॒मान॑मेषाम् । उप॑स्तोषाम यज॒तस्य॑ य॒ज्ञिः \textbf{ 7} \newline
                  \newline
                                \textbf{ TB 3.6.3.2} \newline
                  ते सु॒क्रत॑वः॒ शुच॑यो धिय॒धांः । स्वद॑न्तु दे॒वा उ॒भया॑नि ह॒व्या ॥ आ॒जुह्वा॑न॒ ईड्यो॒ वन्द्य॑श्च । आया᳚ह्यग्ने॒ वसु॑भिः स॒जोषाः᳚ । त्वं दे॒वाना॑मसि यह्व॒ होता᳚ । स ए॑नान्. यक्षीषि॒तो यजी॑यान् ॥ प्रा॒चीनं॑ ब॒र्॒.हिः प्र॒दिशा॑ पृथि॒व्याः । वस्तो॑र॒स्या वृ॑ज्यते॒ अग्रे॒ अह्ना᳚म् । व्यु॑ प्रथते वित॒रं ॅवरी॑यः । दे॒वेभ्यो॒ अदि॑तये स्यो॒नम् । \textbf{ 8} \newline
                  \newline
                                \textbf{ TB 3.6.3.3} \newline
                  व्यच॑स्वतीरुर्वि॒या विश्र॑यन्ताम् । पति॑भ्यो॒ नजन॑यः॒ शुम्भ॑मानाः । देवी᳚र्द्वारो बृहतीर्-विश्वमिन्वाः । दे॒वेभ्यो॑ भवथ सुप्राय॒णाः ॥ आ सु॒ष्वय॑न्ती यज॒ते उपा॑के । उ॒षासा॒नक्ता॑ सदतां॒ नि योनौ᳚ । दि॒व्ये योष॑णे बृह॒ती सु॑रु॒क्मे । अधि॒ श्रियꣳ॑ शुक्र॒पिशं॒ दधा॑ने ॥ दैव्या॒ होता॑रा प्रथ॒मा सु॒वाचा᳚ । मिमा॑ना य॒ज्ञ्ं मनु॑षो॒ यज॑द्ध्यै \textbf{ 9} \newline
                  \newline
                                \textbf{ TB 3.6.3.4} \newline
                  प्र॒चो॒दय॑न्ता वि॒दथे॑षु का॒रू । प्रा॒चीनं॒ ज्योतिः॑ प्र॒दिशा॑ दि॒शन्ता᳚ ॥ आनो॑ य॒ज्ञ्ं भार॑ती॒ तूय॑मेतु । इडा॑ मनु॒ष्वदि॒ह चे॒तय॑न्ती । ति॒स्रो दे॒वीर्-ब॒र्॒.हिरेदꣳ स्यो॒नम् । सर॑स्वती॒ स्वप॑सः सदन्तु ॥ य इ॒मे द्यावा॑पृथि॒वी जनि॑त्री । रू॒पैरपिꣳ॑श॒द्-भुव॑नानि॒ विश्वा᳚ । तम॒द्य हो॑तरिषि॒तो यजी॑यान् । दे॒वं त्वष्टा॑रमि॒ह य॑क्षि वि॒द्वान् ( ) । \textbf{ 10} \newline
                  \newline
                                \textbf{ TB 3.6.3.5} \newline
                  उ॒पाव॑सृज॒त्-त्मन्या॑ सम॒ञ्जन्न् । दे॒वानां॒ पाथ॑ ऋतु॒था ह॒वीꣳषि॑ । वन॒स्पतिः॑ शमि॒ता दे॒वो अ॒ग्निः । स्वद॑न्तु ह॒व्यं मधु॑ना घृ॒तेन॑ ॥ स॒द्यो जा॒तो व्य॑मिमीत य॒ज्ञ्म् । अ॒ग्निर्-दे॒वाना॑-मभवत्-पुरो॒गाः । अ॒स्य होतुः॑ प्र॒दिश्यृ॒तस्य॑ वा॒चि । स्वाहा॑कृतꣳ ह॒विर॑दन्तु दे॒वाः । \textbf{ 11} \newline
                  \newline
                                    (य॒ज्ञिः - स्यो॒नं - ॅयज॑द्ध्यै - वि॒द्वान॒ - +ष्टौ च॑) \textbf{(A3)} \newline \newline
                \textbf{ 3.6.4     अनुवाकं   4 - पर्यग्निकरणीया ऋचः} \newline
                                \textbf{ TB 3.6.4.1} \newline
                  अ॒ग्निर्. होता॑ नो अद्ध्व॒रे । वा॒जी सन्-परि॑णीयते । दे॒वो दे॒वेषु॑ य॒ज्ञियः॑ ॥ परि॑ त्रिवि॒ष्ट्य॑द्ध्व॒रम् । यात्य॒ग्नी र॒थीरि॑व । आ दे॒वेषु॒ प्रयो॒ दध॑त् ॥ परि॒ वाज॑पतिः क॒विः । अ॒ग्निर्. ह॒व्या न्य॑क्रमीत् । दध॒द्-रत्ना॑नि दा॒शुषे᳚ । \textbf{ 12} \newline
                  \newline
                                    (अ॒ग्निर् होता॑ नो॒ नव॑) \textbf{(A4)} \newline \newline
                \textbf{ 3.6.5     अनुवाकं   5 - मैत्रावरुणप्रैषः} \newline
                                \textbf{ TB 3.6.5.1} \newline
                  अजै॑द॒ग्निः । अस॑न॒द्वाजं॒ नि । दे॒वो दे॒वेभ्यो॑ ह॒व्याऽवा᳚ट् । प्राञ्जो॑भिर्. हिन्वा॒नः । धेना॑भिः॒ कल्प॑मानः । य॒ज्ञ्स्यायुः॑ प्रति॒रन्न् । उप॒ प्रेष्य॑ होतः । ह॒व्या दे॒वेभ्यः॑ । \textbf{ 13} \newline
                  \newline
                                    (अजै॑द॒ष्टौ) \textbf{(A5)} \newline \newline
                \textbf{ 3.6.6     अनुवाकं   6 - होतुरध्रिगुप्रैषः} \newline
                                \textbf{ TB 3.6.6.1} \newline
                  दैव्याः᳚ शमितार उ॒त म॑नुष्या॒ आर॑भद्ध्वम् । उप॑ नयत॒ मेद्ध्या॒ दुरः॑ । आ॒शासा॑ना॒ मेध॑पतिभ्यां॒ मेध᳚म् । प्रास्मा॑ अ॒ग्निं भ॑रत । स्तृ॒णी॒त ब॒र्॒.हिः । अन्वे॑नं मा॒ता म॑न्यताम् । अनु॑ पि॒ता । अनु॒ भ्राता॒ सग॑र्भ्यः । अनु॒ सखा॒ सयू᳚थ्यः ॥ उ॒दी॒चीनाꣳ॑ अस्य प॒दो निध॑त्तात् \textbf{ 14} \newline
                  \newline
                                \textbf{ TB 3.6.6.2} \newline
                  सूर्यं॒ चक्षु॑र्गमयतात् । वातं॑ प्रा॒णम॒न्व-व॑सृजतात् । दिशः॒ श्रोत्र᳚म् । अ॒न्तरि॑क्ष॒मसु᳚म् । पृ॒थि॒वीꣳ शरी॑रम् । ए॒क॒धाऽस्य॒ त्वच॒माच्छ्य॑तात् । पु॒रा नाभ्या॑ अपि॒शसो॑ व॒पामुत्खि॑दतात् । अ॒न्तरे॒वोष्माणं॑ ॅवारयतात् । श्ये॒नम॑स्य॒ वक्षः॑ कृणुतात् । प्र॒शसा॑ बा॒हू \textbf{ 15} \newline
                  \newline
                                \textbf{ TB 3.6.6.3} \newline
                  श॒ला दो॒षणी᳚ । क॒श्यपे॒वाꣳसा᳚ । अच्छि॑द्रे॒ श्रोणी᳚ । क॒वषो॒रू स्रे॒कप॑र्णाऽष्ठी॒वन्ता᳚ । षड्विꣳ॑शतिरस्य॒ वङ्क्र॑यः । ता अ॑नु॒ष्ठ्योच्च्या॑वयतात् । गात्रं॑ गात्रम॒स्यानू॑नं कृणुतात् । ऊ॒व॒द्ध्य॒गो॒हं पार्थि॑वं खनतात् । अ॒स्ना रक्षः॒ सꣳसृ॑जतात् । व॒नि॒ष्ठुम॑स्य॒ मा रा॑विष्ट \textbf{ 16} \newline
                  \newline
                                \textbf{ TB 3.6.6.4} \newline
                  उरू॑कं॒ मन्य॑मानाः ॥ नेद्व॑स्तो॒के तन॑ये । रवि॑ता॒ रव॑च्छमितारः । अद्ध्रि॑गो शमी॒द्ध्वम् । सु॒शमि॑ शमीद्ध्वम् । श॒मी॒द्ध्व-म॑द्ध्रिगो ॥ अद्ध्रि॑गु॒श्चापा॑पश्च । उ॒भौ दे॒वानाꣳ॑ शमि॒तारौ᳚ । तावि॒मं प॒शुꣳ श्र॑पयतां प्रवि॒द्वाꣳसौ᳚ । यथा॑ यथाऽस्य॒ श्रप॑णं॒ तथा॑तथा ( ) । \textbf{ 17} \newline
                  \newline
                                    [ध॒त्ता॒द् - बा॒हू - मा रा॑विष्ट॒ - तथा॑तथा ( ) ] \textbf{(A6)} \newline \newline
                \textbf{ 3.6.7     अनुवाकं   7 - स्तोकविषयं मैत्रावरुणानुवचनम्} \newline
                                \textbf{ TB 3.6.7.1} \newline
                  जु॒षस्व॑ स॒प्रथ॑स्तमम् । वचो॑ दे॒वफ्स॑रस्तमम् । ह॒व्या जुह्वा॑न आ॒सनि॑ ॥ इ॒मं नो॑ य॒ज्ञ्म॒मृते॑षु धेहि । इ॒मा ह॒व्या जा॑तवेदो जुषस्व । स्तो॒काना॑मग्ने॒ मेद॑सो घृ॒तस्य॑ । होतः॒ प्राशा॑न प्रथ॒मो नि॒षद्य॑ ॥ घृ॒तव॑न्तः पावक ते । स्तो॒काःश्चो॑तन्ति॒ मेद॑सः । स्वध॑र्मं दे॒ववी॑तये \textbf{ 18} \newline
                  \newline
                                \textbf{ TB 3.6.7.2} \newline
                  श्रेष्ठं॑ नो धेहि॒ वार्य᳚म् ॥ तुभ्यꣳ॑ स्तो॒का घृ॑त॒श्चुतः॑ । अग्ने॒ विप्रा॑य सन्त्य । ऋषिः॒ श्रेष्ठः॒ समि॑द्ध्यसे । य॒ज्ञ्स्य॑ प्रावि॒ता भ॑व ॥ तुभ्यꣳ॑ श्चोतन्त्यद्ध्रिगो शचीवः । स्तो॒कासो॑ अग्ने॒ मेद॑सो घृ॒तस्य॑ । क॒वि॒श॒स्तो बृ॑ह॒ता भा॒नुनाऽऽगाः᳚ । ह॒व्या जु॑षस्व मेधिर ॥ ओजि॑ष्ठं ते मद्ध्य॒तो मेद॒ उद्भृ॑तम् ( ) । प्रते॑ व॒यं द॑दामहे । श्चोत॑न्ति ते वसो स्तो॒का अधि॑ त्व॒चि । प्रति॒ तान्दे॑व॒शो वि॑हि । \textbf{ 19} \newline
                  \newline
                                    (दे॒ववी॑तय॒-उद्भृ॑तं॒ त्रीणि॑ च) \textbf{(A7)} \newline \newline
                \textbf{ 3.6.8     अनुवाकं   8 - वपा पुरोडाशस्विष्टकृतां पुरोनुवाक्याः प्रैषाश्च} \newline
                                \textbf{ TB 3.6.8.1} \newline
                  आ वृ॑त्रहणा वृत्र॒हभिः॒ शुष्मैः᳚ । इन्द्र॑ या॒तं नमो॑भिरग्ने अ॒र्वाक् । यु॒वꣳ राधो॑भि॒रक॑वेभिरिन्द्र । अग्ने॑ अ॒स्मे भ॑वतमुत्त॒मेभिः॑ ॥ होता॑ यक्षदिन्द्रा॒ग्नी । छाग॑स्य व॒पाया॒ मेद॑सः । जु॒षेताꣳ॑ ह॒विः । होत॒र्यज॑ ॥ वि ह्यख्य॒न्-मन॑सा॒ वस्य॑ इ॒च्छन्न् । इन्द्रा᳚ग्नी ज्ञा॒स उ॒त वा॑ सजा॒तान् \textbf{ 20} \newline
                  \newline
                                \textbf{ TB 3.6.8.2} \newline
                  नान्या यु॒वत्प्रम॑तिरस्ति॒ मह्य᳚म् । स वां॒ धियं॑ ॅवाज॒यन्ती॑मतक्षम् ॥ होता॑ यक्षदिन्द्रा॒ग्नी । पु॒रो॒डाश॑स्य जु॒षेताꣳ॑ ह॒विः । होत॒र्यज॑ ॥ त्वामी॑डते अजि॒रं दू॒त्या॑य । ह॒विष्म॑न्तः॒ सद॒मिन्-मानु॑षासः । यस्य॑ दे॒वैरास॑दो ब॒र्॒.हिर॑ग्ने । अहा᳚न्यस्मै सु॒दिना॑ भवन्तु ॥ होता॑ यक्षद॒ग्निम् ( ) । पु॒रो॒डाश॑स्य जु॒षताꣳ॑ ह॒विः । होत॒र्यज॑ । \textbf{ 21} \newline
                  \newline
                                    (स॒जा॒ता - न॒ग्निं द्वे च॑) \textbf{(A8)} \newline \newline
                \textbf{ 3.6.9     अनुवाकं   9 - तेषामेव याज्याः} \newline
                                \textbf{ TB 3.6.9.1} \newline
                  गी॒र्भिर्विप्रः॒ प्रम॑तिमि॒च्छमा॑नः । ई॒॑ र॒यिं ॅय॒शसं॑ पूर्व॒भाज᳚म् । इन्द्रा᳚ग्नी वृत्रहणा सुवज्रा । प्रणो॒ नव्ये॑भिस्तिरतं दे॒ष्णैः ॥ माच्छे᳚द्म र॒श्मीꣳरिति॒ नाध॑मानाः । पि॒तृ॒णाꣳ शक्ती॑रनु॒ यच्छ॑मानाः । इ॒न्द्रा॒ग्निभ्यां॒ कं ॅवृष॑णो मदन्ति । ता ह्यद्री॑ धि॒षणा॑या उ॒पस्थे᳚ ॥ अ॒ग्निꣳ सु॑दी॒तिꣳ सु॒दृशं॑ गृ॒णन्तः॑ । न॒म॒स्याम॒स्त्वेड्यं॑ जातवेदः ( ) । त्वां दू॒तम॑र॒तिꣳ ह॑व्य॒वाह᳚म् । दे॒वा अ॑कृण्वन्-न॒मृत॑स्य॒ नाभि᳚म् । \textbf{ 22} \newline
                  \newline
                                    (जा॒त॒वे॒दो॒ द्वे च॑) \textbf{(A9)} \newline \newline
                \textbf{ 3.6.10    अनुवाकं   10 - मनोतासूक्तम्} \newline
                                \textbf{ TB 3.6.10.1} \newline
                  त्वꣳ ह्य॑ग्ने प्रथ॒मो म॒नोता᳚ । अ॒स्या धि॒यो अभ॑वो दस्म॒ होता᳚ । त्वꣳ सीं᳚ ॅवृषन्-नकृणोर्-दु॒ष्टरी॑तु । सहो॒ विश्व॑स्मै॒ सह॑से॒ सह॑द्ध्यै ॥ अधा॒ होता॒ न्य॑सीदो॒ यजी॑यान् । इ॒डस्प॒द इ॒षय॒न्नीड्यः॒ सन्न् । तं त्वा॒ नरः॑ प्रथ॒मं दे॑व॒यन्तः॑ । म॒हो रा॒ये चि॒तय॑न्तो॒ अनु॑ ग्मन्न् ॥ वृ॒तेव॒ यन्तं॑ ब॒हुभि॑र्-वस॒व्यैः᳚ । त्वे र॒यिं जा॑गृ॒वाꣳसो॒ अनु॑ ग्मन्न् \textbf{ 23} \newline
                  \newline
                                \textbf{ TB 3.6.10.2} \newline
                  रुश॑न्तम॒ग्निं द॑र्.श॒तं बृ॒हन्त᳚म् । व॒पाव॑न्तं ॅवि॒श्वहा॑ दीदि॒वाꣳस᳚म् ॥ प॒दं दे॒वस्य॒ नम॑सा वि॒यन्तः॑ । श्र॒व॒स्यवः॒ श्रव॑ आप॒न्नमृ॑क्तम् । नामा॑नि चिद्-दधिरे य॒ज्ञिया॑नि । भ॒द्रायां᳚ ते रणयन्त॒ सन्दृ॑ष्टौ ॥ त्वां ॅव॑र्द्धन्ति क्षि॒तयः॑ पृथि॒व्याम् । त्वꣳ राय॑ उ॒भया॑सो॒ जना॑नाम् । त्वं त्रा॒ता त॑रणे॒ चेत्यो॑ भूः । पि॒ता मा॒ता सद॒मिन्-मानु॑षाणाम् । \textbf{ 24} \newline
                  \newline
                                \textbf{ TB 3.6.10.3} \newline
                  स प॒र्येण्यः॒ स प्रि॒यो वि॒क्ष्व॑ग्निः । होता॑ म॒न्द्रो निष॑सादा॒ यजी॑यान् । तं त्वा॑ व॒यं दम॒ आ दी॑दि॒वाꣳस᳚म् । उप॑ज्ञु॒बाधो॒ नम॑सा सदेम ॥ तं त्वा॑ व॒यꣳ सु॒धियो॒ नव्य॑मग्ने । सु॒म्ना॒यव॑ ईमहे देव॒यन्तः॑ । त्वं ॅविशो॑ अनयो॒ दीद्या॑नः । दि॒वो अ॑ग्ने बृह॒ता रो॑च॒नेन॑ ॥ वि॒शां क॒विं ॅवि॒श्पतिꣳ॒॒ शश्व॑तीनाम् । नि॒तोश॑नं ॅवृष॒भं च॑र्.षणी॒नाम् \textbf{ 25} \newline
                  \newline
                                \textbf{ TB 3.6.10.4} \newline
                  प्रेती॑षणिमि॒षय॑न्तं पाव॒कम् । राज॑न्तम॒ग्निं ॅय॑ज॒तꣳ र॑यी॒णाम् ॥ सो अ॑ग्न ईजे शश॒मे च॒ मर्तः॑ । यस्त॒ आन॑ट्थ्-स॒मिधा॑ ह॒व्यदा॑तिम् । य आहु॑तिं॒ परि॒ वेदा॒ नमो॑भिः । विश्वेथ्स वा॒मा द॑धते॒ त्वोतः॑ ॥ अ॒स्मा उ॑ ते॒ महि॑ म॒हे वि॑धेम । नमो॑भिरग्ने स॒मिधो॒त ह॒व्यैः । वेदी॑ सूनो सहसो गी॒र्भिरु॒क्थैः । आ ते॑ भ॒द्रायाꣳ॑ सुम॒तौ य॑तेम । \textbf{ 26} \newline
                  \newline
                                \textbf{ TB 3.6.10.5} \newline
                  आ यस्त॒तन्थ॒ रोद॑सी॒ वि भा॒सा । श्रवो॑भिश्च श्रव॒स्य॑स्तरु॑त्रः । बृ॒हद्भि॒र्-वाजैः॒ स्थवि॑रेभिर॒स्मे । रे॒वद्भि॑रग्ने वित॒रं ॅविभा॑हि ॥ नृ॒वद्व॑सो॒ सद॒मिद्धे᳚ह्य॒स्मे । भूरि॑ तो॒काय॒ तन॑याय प॒र्श्वः । पू॒र्वीरिषो॑ बृह॒तीरा॒रे अ॑घाः । अ॒स्मे भ॒द्रा सौ᳚श्रव॒सानि॑ सन्तु ॥ पु॒रूण्य॑ग्ने पुरु॒धा त्वा॒या । वसू॑नि राजन्-व॒सुता॑ते अश्याम् ( ) । पु॒रूणि॒ हि त्वे पु॑रुवार॒ सन्ति॑ । अग्ने॒ वसु॑ विध॒ते राज॑नि॒ त्वे । \textbf{ 27} \newline
                  \newline
                                    (जा॒गृ॒वाꣳसो॒ अनु॑ ग्म॒न् - मानु॑षाणां - चर्.षणी॒नां - ॅय॑ते - माश्यां॒ द्वे च॑) \textbf{(A10)} \newline \newline
                \textbf{ 3.6.11    अनुवाकं   11 - हविर्वनस्पतिस्विष्टकृतां पुरोनुवाक्याः प्रैषाश्च} \newline
                                \textbf{ TB 3.6.11.1} \newline
                  आ भ॑रतꣳ शिक्षतं ॅवज्रबाहू । अ॒स्माꣳ इ॑न्द्राग्नी अवतꣳ॒॒ शची॑भिः ।इ॒मे नु ते र॒श्मयः॒ सूर्य॑स्य । येभिः॑ सपि॒त्वं पि॒तरो॑ न॒ आयन्न्॑ ॥ होता॑ यक्षदिन्द्रा॒ग्नी । छाग॑स्य ह॒विष॒ आत्ता॑म॒द्य । म॒द्ध्य॒तो मेद॒ उद्भृ॑तम् । पु॒रा द्वेषो᳚भ्यः । पु॒रा पौरु॑षेय्या गृ॒भः । घस्ता᳚न्नू॒नम् \textbf{ 28} \newline
                  \newline
                                \textbf{ TB 3.6.11.2} \newline
                  घा॒से अ॑ज्राणां॒ ॅयव॑सप्रथमानाम् । सु॒मत्क्ष॑राणाꣳ श॒तरु॑द्रियाणाम् । अ॒ग्नि॒ष्वा॒त्तानां॒ पीवो॑पवसनानाम् । पा॒र्श्व॒तः श्रो॑णि॒तः शि॑ताम॒त उ॑थ्साद॒तः । अङ्गा॑दङ्गा॒दव॑त्तानाम् । कर॑त ए॒वेन्द्रा॒ग्नी । जु॒षेताꣳ॑ ह॒विः । होत॒र्यज॑ ॥ दे॒वेभ्यो॑ वनस्पते ह॒वीꣳषि॑ । हिर॑ण्यपर्ण प्र॒दिव॑स्ते॒ अर्थ᳚म् \textbf{ 29} \newline
                  \newline
                                \textbf{ TB 3.6.11.3} \newline
                  प्र॒द॒क्षि॒णिद्-र॑श॒नया॑ नि॒यूय॑ । ऋ॒तस्य॑ वक्षि प॒थिभी॒ रजि॑ष्ठैः ॥ होता॑ यक्ष॒द्-वन॒स्पति॑म॒भि हि । पि॒ष्टत॑मया॒ रभि॑ष्ठया रश॒नयाऽऽधि॑त । यत्रे᳚न्द्राग्नि॒योः छाग॑स्य ह॒विषः॑ प्रि॒या धामा॑नि । यत्र॒ वन॒स्पतेः᳚ प्रि॒या पाथाꣳ॑सि । यत्र॑ दे॒वाना॑माज्य॒पानां᳚ प्रि॒या धामा॑नि । यत्रा॒ग्नेर्. होतुः॑ प्रि॒या धामा॑नि । तत्रै॒तं प्र॒स्तु-त्ये॑वो-प॒स्तुत्ये॑वो॒पाव॑स्रक्षत् । रभी॑याꣳ समिव कृ॒त्वी \textbf{ 30} \newline
                  \newline
                                \textbf{ TB 3.6.11.4} \newline
                  कर॑दे॒वं दे॒वो वन॒स्पतिः॑ । जु॒षताꣳ॑ ह॒विः । होत॒र्यज॑ ॥ पि॒प्री॒हि दे॒वाꣳ उ॑श॒तो य॑विष्ठ । वि॒द्वां ऋ॒तूꣳर्. ऋ॑तुपते यजे॒ह । ये दैव्या॑ ऋ॒त्विज॒स्तेभि॑रग्ने । त्वꣳ होतॄ॑णाम॒स्याय॑जिष्ठः ॥ होता॑ यक्षद॒ग्निꣳ स्वि॑ष्ट॒कृत᳚म् । अया॑ड॒ग्नि-रि॑न्द्राग्नि॒योः छाग॑स्य ह॒विषः॑ प्रि॒या धामा॑नि । अया॒ड्वन॒स्पतेः᳚ प्रि॒या पाथाꣳ॑सि ( ) । अया᳚ड्दे॒वाना॑-माज्य॒पानां᳚ प्रि॒या धामा॑नि । यक्ष॑द॒ग्नेर्. होतुः॑ प्रि॒या धामा॑नि । यक्ष॒थ् स्वं म॑हि॒मान᳚म् । आय॑जता॒मेज्या॒ इषः॑ । कृ॒णोतु॒ सो अ॑द्ध्व॒रा जा॒तवे॑दाः । जु॒षताꣳ॑ ह॒विः । होत॒र्यज॑ । \textbf{ 31} \newline
                  \newline
                                    (नू॒न - मर्थं॑ - कृ॒त्वी - पाथाꣳ॑सि स॒प्त च॑) \textbf{(A11)} \newline \newline
                \textbf{ 3.6.12    अनुवाकं   12 - तेषामेव याज्याः} \newline
                                \textbf{ TB 3.6.12.1} \newline
                  उपो॑ह॒ यद्वि॒दथं॑ ॅवा॒जिनो॒ गूः । गी॒र्भिर्-विप्राः॒ प्रम॑ति-मि॒च्छमा॑नाः । अ॒र्वन्तो॒ न काष्ठां॒ नक्ष॑माणाः । इ॒न्द्रा॒ग्नी जोहु॑वतो॒ नर॒स्ते ॥ वन॑स्पते रश॒नया॑ऽभि॒धाय॑ । पि॒ष्टत॑मया व॒युना॑नि वि॒द्वान् । वह॑ देव॒त्रा दि॑धिषो ह॒वीꣳषि॑ । प्र च॑ दा॒तार॑म॒मृते॑षु वोचः ॥ अ॒ग्निꣳ स्वि॑ष्ट॒कृत᳚म् । अया॑ड॒ग्नि-रि॑न्द्राग्नि॒योः छाग॑स्य ह॒विषः॑ प्रि॒या धामा॑नि \textbf{ 32} \newline
                  \newline
                                \textbf{ TB 3.6.12.2} \newline
                  अया॒ड्वन॒स्पतेः᳚ प्रि॒या पाथाꣳ॑सि । अया᳚ड्दे॒वाना॑-माज्य॒पानां᳚ प्रि॒या धामा॑नि । यक्ष॑द॒ग्नेर्. होतुः॑ प्रि॒या धामा॑नि । यक्ष॒थ् स्वं म॑हि॒मान᳚म् । आय॑जता॒मेज्या॒ इषः॑ । कृ॒णोतु॒ सो अ॑द्ध्व॒रा जा॒तवे॑दाः । जु॒षताꣳ॑ ह॒विः । अग्ने॒ यद॒द्य वि॒शो अ॑द्ध्वरस्य होतः । पाव॑क शोचे॒ वेष्ट्वꣳ हि यज्वा᳚ । ऋ॒ता य॑जासि महि॒ना वि यद्-भूः ( ) । ह॒व्या व॑ह यविष्ठ॒ या ते॑ अ॒द्य । \textbf{ 33} \newline
                  \newline
                                    (धामा॑नि॒ - भुरेकं॑ च) \textbf{(A12)} \newline \newline
                \textbf{ 3.6.13    अनुवाकं   13 - अनूयाजप्रैषाः} \newline
                                \textbf{ TB 3.6.13.1} \newline
                  दे॒वं ब॒र्॒.हिः सु॑दे॒वं दे॒वैः स्याथ् सु॒वीरं॑ ॅवी॒रैर्वस्तो᳚र्-वृ॒ज्येता॒क्तोः प्रभ्रि॑-ये॒तात्य॒न्यान्-रा॒या ब॒र्॒.हिष्म॑तो मदेम वसु॒वने॑ वसु॒धेय॑स्य वेतु॒ यज॑ ॥ दे॒वीर्-द्वारः॑ संघा॒ते वि॒ड्वीर्याम॑ञ्छिथि॒रा ध्रु॒वा दे॒वहू॑तौ व॒थ्स ई॑मेना॒स्तरु॑ण॒ आमि॑मीयात्-कुमा॒रो वा॒ नव॑जातो॒ मैना॒ अर्वा॑ रे॒णुक॑काटः॒ पृण॑ग्वसु॒वने॑ वसु॒धेय॑स्य वियन्तु॒ यज॑ ॥ दे॒वी उ॒षासा॒ नक्ताऽद्या॒स्मिन्. य॒ज्ञे प्र॑य॒त्य॑ह्वेता॒मपि॑ नू॒नं दैवी॒र्विशः॒ प्राया॑सिष्टाꣳ॒॒ सुप्री॑ते॒ सुधि॑ते वसु॒वने॑ वसु॒धेय॑स्य वीतां॒ ॅयज॑ ॥ दे॒वी जोष्ट्री॒ वसु॑धिती॒ ययो॑र॒न्याऽघा द्वेषाꣳ॑सि यु॒यव॒दाऽन्या व॑क्ष॒द्-वसु॒ वार्या॑णि॒ यज॑मानाय वसु॒वने॑ वसु॒धेय॑स्य वीतां॒ ॅयज॑ ॥ दे॒वी ऊ॒र्जाहु॑ती॒ इष॒मूर्ज॑म॒न्या व॑क्ष॒थ्सग्धिꣳ॒॒ सपी॑तिम॒न्या नवे॑न॒ पूर्वं॒ दय॑मानाः॒ स्याम॑ पुरा॒णेन॒ नवं॒ तामूर्ज॑मू॒र्जाहु॑ती ऊ॒र्जय॑माने अधातां ॅवसु॒वने॑ वसु॒धेय॑स्य वीतां॒ ॅयज॑ ॥ दे॒वा दैव्या॒ होता॑रा॒ नेष्टा॑रा॒ पोता॑रा ह॒ताघ॑शꣳसा-वाभ॒रद्व॑सू वसु॒वने॑ वसु॒धेय॑स्य वीतां॒ ॅयज॑ ॥ दे॒वीस्ति॒स्रस्ति॒स्रो दे॒वीरिडा॒ सर॑स्वती॒ भार॑ती॒ द्यां भार॑त्यादि॒त्यै-र॑स्पृक्ष॒थ्-सर॑स्वती॒मꣳ रु॒द्रैर् य॒ज्ञ्मा॑वी-दि॒हैवेड॑या॒ वसु॑मत्या सध॒मादं॑ मदेम वसु॒वने॑ वसु॒धेय॑स्य वियन्तु॒ यज॑ ॥ दे॒वो नरा॒शꣳस॑-स्त्रिशी॒र्॒.षा ष॑ड॒क्षः श॒तमिदे॑नꣳ शितिपृ॒ष्ठा आद॑धति स॒हस्र॑मीं॒ प्रव॑हन्ति मि॒त्रावरु॒णेद॑स्य हो॒त्रमर्.ह॑तो॒ बृह॒स्पतिः॑ स्तो॒त्रम॒श्विना ऽऽद्ध्व॑र्यवं ॅवसु॒वने॑ वसु॒धेय॑स्य वेतु॒ यज॑ ॥दे॒वो वन॒स्पति॑र्. व॒र्॒.षप्रा॑वा घृ॒तनि॑र्णि॒ग्-द्या-मग्रे॒णास्पृ॑क्ष॒दाऽन्तरि॑क्षं॒ मद्ध्ये॑नाप्राः पृथि॒वी-मुप॑रेणादृꣳहीद्-वसु॒वने॑ वसु॒धेय॑स्य वेतु॒ यज॑ ॥ दे॒वं ब॒र्॒.हिर्वारि॑तीनां नि॒धे धा॑सि॒ प्रच्यु॑तीना॒म-प्र॑च्युतं निकाम॒धर॑णं पुरुस्पा॒र्॒.हं ॅयश॑स्वदे॒ना ब॒र्॒.हिषा॒ऽन्या ब॒र्॒.हीꣳष्य॒भिष्या॑म वसु॒वने॑ वसु॒धेय॑स्य वेतु॒ यज॑ ( ) ॥ दे॒वो अ॒ग्निः स्वि॑ष्ट॒कृथ् सु॒द्रवि॑णा म॒न्द्रः क॒विः स॒त्यम॑न्माऽऽय॒जी होता॒ होतु॑र्. होतु॒राय॑जीया॒नग्ने॒ यान्-दे॒वानया॒ड्याꣳ अपि॑प्रे॒र्ये ते॑ हो॒त्रे अम॑थ्सत॒ ताꣳ स॑स॒नुषीꣳ॒॒ होत्रां᳚ देवं ग॒मां दि॒वि दे॒वेषु॑ य॒ज्ञ्मेर॑ये॒मꣳ स्वि॑ष्ट॒कृच्चाग्ने॒ होताऽभू᳚र्वसु॒वने॑ वसु॒धेय॑स्य नमोवा॒के वीहि॒ यज॑ । \textbf{ 34} \newline
                  \newline
                                    (यजैकं॑ च) \textbf{(A13)} \newline \newline
                \textbf{ 3.6.14    अनुवाकं   14 - अनूयाजयाज्याः} \newline
                                \textbf{ TB 3.6.14.1} \newline
                  दे॒वं ब॒र्॒.हिः । व॒सु॒वने॑ वसु॒धेय॑स्य वेतु । दे॒वीर्द्वारः॑ । व॒सु॒वने॑ वसु॒धेय॑स्य वियन्तु । दे॒वी उ॒षासा॒ नक्ता᳚ । व॒सु॒वने॑ वसु॒धेय॑स्य वीताम् । दे॒वी जोष्ट्री᳚ । व॒सु॒वने॑ वसु॒धेय॑स्य वीताम् । दे॒वी ऊ॒र्जाहु॑ती । व॒सु॒वने॑ वसु॒धेय॑स्य वीताम् \textbf{ 35} \newline
                  \newline
                                \textbf{ TB 3.6.14.2} \newline
                  दे॒वा दैव्या॒ होता॑रा । व॒सु॒वने॑ वसु॒धेय॑स्य वीताम् । दे॒वीस्ति॒स्र-स्ति॒स्रो दे॒वीः । व॒सु॒वने॑ वसु॒धेय॑स्य वियन्तु । दे॒वो नरा॒शꣳसः॑ । व॒सु॒वने॑ वसु॒धेय॑स्य वेतु । दे॒वो वन॒स्पतिः॑ । व॒सु॒वने॑ वसु॒धेय॑स्य वेतु । दे॒वं ब॒र्॒.हिर्वारि॑तीनाम् । व॒सु॒वने॑ वसु॒धेय॑स्य वेतु \textbf{ 36} \newline
                  \newline
                                \textbf{ TB 3.6.14.3} \newline
                  दे॒वो अ॒ग्निः स्वि॑ष्ट॒कृत् । सु॒द्रवि॑णा म॒न्द्रः क॒विः । स॒त्यम॑न्माऽऽय॒जी होता᳚ । होतु॑र्. होतु॒राय॑जीयान् । अग्ने॒ यान्दे॒वानया᳚ट् । याꣳ अपि॑प्रेः । ये ते॑ हो॒त्रे अम॑थ्सत । ताꣳ स॑स॒नुषीꣳ॒॒ होत्रां᳚ देवं ग॒माम् । दि॒वि दे॒वेषु॑ य॒ज्ञ्मेर॑ये॒मम् । स्वि॒ष्ट॒कृच्चाग्ने॒ होताऽभूः᳚ ( ) । व॒सु॒वने॑ वसु॒धेय॑स्य नमोवा॒के वीहि॑ । \textbf{ 37} \newline
                  \newline
                                    (वी॒तां॒ - ॅवे॒त्व - भू॒रेकं॑ च) \textbf{(A14)} \newline \newline
                \textbf{ 3.6.15    अनुवाकं   15 - सूक्तवाकप्रैषाः} \newline
                                \textbf{ TB 3.6.15.1} \newline
                  अ॒ग्निम॒द्य होता॑रमवृणीता॒यं ॅयज॑मानः॒ पच॑न् प॒क्तीः पच॑न् पुरो॒डाशं॑ ब॒द्ध्नन्-नि॑न्द्रा॒ग्निभ्यां॒ छागꣳ॑ सूप॒स्था अ॒द्य दे॒वो वन॒स्पति॑-रभव-दिन्द्रा॒ग्निभ्यां॒ छागे॒नाघ॑स्तां॒ तं मे॑द॒स्तः प्रति॑ पच॒ताऽग्र॑भीष्टा॒मवी॑वृधेतां पुरो॒डाशे॑न॒ त्वाम॒द्यर्.ष॑ आर्.षेयर्.षीणां नपादवृणीता॒यं ॅयज॑मानो ब॒हुभ्य॒ आ संग॑तेभ्य ए॒ष मे॑ दे॒वेषु॒ वसु॒ वार्या य॑क्ष्यत॒ इति॒ ता या दे॒वा दे॑व॒-दाना॒न्यदु॒स्तान्य॑स्मा॒ आ च॒ शास्वा च॑ गुरस्वेषि॒तश्च॑ होत॒रसि॑ भद्र॒वाच्या॑य॒ प्रेषि॑तो॒ मानु॑षः सूक्तवा॒काय॑ सू॒क्ता ब्रू॑हि । \textbf{ 38} \newline
                  \newline
                                    (अ॒ग्निम॒द्यैक᳚म्) \textbf{(A15)} \newline \newline
                \textbf{Prapaataka korvai with starting  Words of 1 to 15 Anuvaakams :-} \newline
        (अ॒ञ्जन्ति॒ - होता॑ यक्ष॒थ् - समि॑द्वो अ॒द्या - ग्नि - रजै॒द् - दैव्या॑ - जु॒षस् - वा वृ॑त्रहणा - गी॒र्भि - स्त्वꣳ ह्या - भ॑रत॒ - मुपो॑ह॒ यद् - दे॒वं ब॒र्॒.हिः सु॑दे॒वं - दे॒वं ब॒र्॒.हि - र॒ग्निम॒द्य पञ्च॑दश) \newline

        \textbf{korvai with starting Words of 1, 11, 21 Series of Dasinis :-} \newline
        (अ॒ञ्जन् - त्यु॒पाव॑सृज॒न् - नान्या यु॒वत् - कर॑दे॒व म॒ष्टात्रिꣳ॑शत्) \newline

        \textbf{first and last  Word 3rd Ashtakam 6th Prapaatakam :-} \newline
        (अ॒ञ्जन्ति॑ - सू॒क्ता ब्रू॑हि) \newline 

       

        ॥ हरिः॑ ॐ ॥
॥ कृष्ण यजुर्वेदीय तैत्तिरीय ब्राह्मणे तृतीयाष्टके षष्ठः प्रपाठकः समाप्तः ॥
================= \newline
        \pagebreak
        
        
        
     \addcontentsline{toc}{section}{ 3.7     तैत्तिरीय ब्राह्मणे तृतियाष्टके सप्तमः प्रपाठकः अच्छिद्रं काण्डं प्रायश्चित्तेन मन्त्रैश्च यज्ञ्च्छिद्रपूरणादच्छिद्रमुच्यते}
     \markright{ 3.7     तैत्तिरीय ब्राह्मणे तृतियाष्टके सप्तमः प्रपाठकः अच्छिद्रं काण्डं प्रायश्चित्तेन मन्त्रैश्च यज्ञ्च्छिद्रपूरणादच्छिद्रमुच्यते \hfill https://www.vedavms.in \hfill}
     \section*{ 3.7     तैत्तिरीय ब्राह्मणे तृतियाष्टके सप्तमः प्रपाठकः अच्छिद्रं काण्डं प्रायश्चित्तेन मन्त्रैश्च यज्ञ्च्छिद्रपूरणादच्छिद्रमुच्यते }
                \textbf{ 3.7.1     अनुवाकं   1 - दर्शपूर्णमासेष्टिविषयाणि कानि चित्प्रायश्चित्तानि} \newline
                                \textbf{ TB 3.7.1.1} \newline
                  सर्वा॒न्॒. वा ए॒षो᳚ऽग्नौ कामा॒न् प्रवे॑शयति । यो᳚-ऽग्नीन॑न्वा॒धाय॑ व्र॒तमु॒पैति॑ । स यदनि॑ष्ट्वा प्रया॒यात् । अका॑मप्रीता एनं॒ कामा॒ नानु॒ प्रया॑युः । अ॒ते॒जा अ॑वी॒र्यः॑ स्यात् । स जु॑हुयात् । तुभ्यं॒ ता अ॑ङ्गिरस्तम । विश्वाः᳚ सुक्षि॒तयः॒ पृथ॑क् । अग्ने॒ कामा॑य येमिर॒ इति॑ । कामा॑ने॒वास्मि॑न् दधाति \textbf{ 1} \newline
                  \newline
                                \textbf{ TB 3.7.1.2} \newline
                  काम॑प्रीता एनं॒ कामा॒ अनु॒प्रया᳚न्ति । ते॒ज॒स्वी वी॒र्या॑वान् भवति ॥सन्त॑ति॒र्वा ए॒षा य॒ज्ञ्स्य॑ । यो᳚-ऽग्नीन॑न्वा॒धाय॑ व्र॒तमु॒पैति॑ । स यदु॒द्वाय॑ति । विच्छि॑त्ति-रे॒वास्य॒ सा ॥ तं प्राञ्च॑मु॒द्धृत्य॑ । मन॒सोप॑तिष्ठेत । मनो॒ वै प्र॒जाप॑तिः । प्रा॒जा॒प॒त्यो य॒ज्ञ्ः \textbf{ 2} \newline
                  \newline
                                \textbf{ TB 3.7.1.3} \newline
                  मन॑सै॒व य॒ज्ञ्ꣳ सन्त॑नोति । भूरित्या॑ह । भू॒तो वै प्र॒जाप॑तिः । भूति॑मे॒वोपै॑ति ॥ वि वा ए॒ष इ॑न्द्रि॒येण॑ वी॒र्ये॑णर्द्ध्यते । यस्या-हि॑ताग्ने-र॒ग्निर॑प॒क्षाय॑ति । याव॒च्छम्य॑या प्र॒विद्ध्ये᳚त् । यदि॒ ताव॑दप॒क्षाये᳚त् । तꣳ संभ॑रेत् । इ॒दं त॒ एकं॑ प॒र उ॑त॒ एकं᳚ \textbf{ 3} \newline
                  \newline
                                \textbf{ TB 3.7.1.4} \newline
                  तृ॒तीये॑न॒ ज्योति॑षा॒ सम्ॅवि॑शस्व । स॒म्ॅवेश॑नस्त॒नुवै॒ चारु॑रेधि । प्रि॒ये दे॒वानां᳚ पर॒मे ज॒नित्र॒ इति॑ । ब्रह्म॑णै॒वैनꣳ॒॒ संभ॑रति । सैव ततः॒ प्राय॑श्चित्तिः । यदि॑ परस्त॒-राम॑प॒क्षाये᳚त् । अ॒नु॒ प्र॒यायाव॑स्येत् । सो ए॒व ततः॒ प्राय॑श्चित्तिः ॥ ओष॑धी॒र्वा ए॒तस्य॑ प॒शून् पयः॒ प्रवि॑शति । यस्य॑ ह॒विषे॑ व॒थ्सा अ॒पाकृ॑ता॒ धय॑न्ति \textbf{ 4} \newline
                  \newline
                                \textbf{ TB 3.7.1.5} \newline
                  तान्. यद्दु॒ह्यात् । या॒तया᳚म्ना ह॒विषा॑ यजेत । यन्न दु॒ह्यात् । य॒ज्ञ्॒प॒रु-र॒न्तरि॑यात् । वा॒य॒व्यां᳚ ॅयवा॒गूं निर्व॑पेत् । वा॒युर्वै पय॑सः प्रदापयि॒ता । स ए॒वास्मै॒ पयः॒ प्रदा॑पयति । पयो॒ वा ओष॑धयः । पयः॒ पयः॑ । पय॑सै॒वास्मै॒ पयोऽव॑रुन्धे \textbf{ 5} \newline
                  \newline
                                \textbf{ TB 3.7.1.6} \newline
                  अथोत्त॑रस्मै ह॒विषे॑ व॒थ्सान॒पा कु॑र्यात् । सैव ततः॒ प्राय॑श्चित्तिः ॥ अ॒न्य॒त॒रान्. वा ए॒ष दे॒वान् भा॑ग॒धेये॑न॒ व्य॑र्द्धयति । ये यज॑मानस्य सा॒यं गृ॒हमा॒-गच्छ॑न्ति । यस्य॑ सायं दु॒ग्धꣳ ह॒विरार्ति॑-मा॒र्च्छति॑ । इन्द्रा॑य व्री॒हीन्नि॒-रुप्यो-प॑वसेत् । पयो॒ वा ओष॑धयः । पय॑ ए॒वारभ्य॑ गृही॒त्वो-प॑वसति । यत्प्रा॒तः स्यात् । तच्छृ॒तं कु॑र्यात् \textbf{ 6} \newline
                  \newline
                                \textbf{ TB 3.7.1.7} \newline
                  अथेत॑र ऐ॒न्द्रः पु॑रो॒डाशः॑ स्यात् । इ॒न्द्रि॒ये ए॒वास्मै॑ स॒मीची॑ दधाति । पयो॒ वा ओष॑धयः । पयः॒ पयः॑ । पय॑सै॒वास्मै॒ पयोऽव॑रुन्धे । अथोत्त॑रस्मै ह॒विषे॑ व॒थ्सान॒पा कु॑र्यात् । सैव ततः॒ प्राय॑श्चित्तिः ॥ उ॒भया॒न्॒. वा ए॒ष दे॒वान् भा॑ग॒धेये॑न॒ व्य॑र्द्धयति । ये यज॑मानस्य सा॒यं च॑ प्रा॒तश्च॑ गृ॒हमा॒-गच्छ॑न्ति । यस्यो॒भयꣳ॑ ह॒विरार्त्ति॑-मा॒र्च्छति॑ \textbf{ 7} \newline
                  \newline
                                \textbf{ TB 3.7.1.8} \newline
                  ऐ॒न्द्रं पञ्च॑शराव-मोद॒नं निर्व॑पेत् । अ॒ग्निं दे॒वता॑नां प्रथ॒मं ॅय॑जेत् । अ॒ग्निमु॑खा ए॒व दे॒वताः᳚ प्रीणाति । अ॒ग्निं ॅवा अन्व॒न्या दे॒वताः᳚ । इन्द्र॒मन्व॒न्याः । ता ए॒वोभयीः᳚ प्रीणाति । पयो॒ वा ओष॑धयः । पयः॒ पयः॑ । पय॑सै॒वास्मै॒ पयोऽव॑रुन्धे । अथोत्त॑रस्मै ह॒विषे॑ व॒थ्सान॒पा कु॑र्यात् \textbf{ 8} \newline
                  \newline
                                \textbf{ TB 3.7.1.9} \newline
                  सैव ततः॒ प्राय॑श्चित्तिः ॥ अ॒र्द्धो वा ए॒तस्य॑ य॒ज्ञ्स्य॑ मीयते । यस्य॒ व्रत्येऽह॒न् पत्न्य॑ना-लम्भु॒का भव॑ति । ताम॑प॒रुद्ध्य॑ यजेत । सर्वे॑णै॒व य॒ज्ञेन॑ यजते । तामि॒ष्ट्वो-प॑ह्वयेत । अमू॒हम॑स्मि । सा त्वं । द्यौर॒हं । पृ॒थि॒वी त्वं ( ) । सामा॒हं । ऋक्त्वं । तावेहि॒ संभ॑वाव । स॒ह रेतो॑ दधावहै । पुꣳ॒॒से पु॒त्राय॒ वेत्त॑वै । रा॒यस्पोषा॑य सुप्रजा॒स्त्वाय॑ सु॒वीर्या॒येति॑ । अ॒र्द्ध ए॒वैना॒मु-प॑ह्वयते । सैव ततः॒ प्राय॑श्चित्तिः । \textbf{ 9} \newline
                  \newline
                                                        \textbf{special korvai} \newline
              (सर्वा॒न्॒. वि वै यदि॑ परस्त॒रामोष॑धीरन् यत॒रानु॒भया॑न॒र्धो वै ) \newline
                                (द॒धा॒ति॒ - य॒ज्ञ् - उ॑त॒ एकं॒ - धय॑न्ति - रुन्धे - कुर्या - दा॒र्छ - त्य॒पाकु॑र्यात् - पृथि॒वी त्वम॒ष्टौ च॑) \textbf{(A1)} \newline \newline
                \textbf{ 3.7.2     अनुवाकं   2 -अग्निहोत्रसाम्नाय्यसाधारणानि प्रायश्चित्तानि} \newline
                                \textbf{ TB 3.7.2.1} \newline
                  यद्विः ष॑ण्णेन जुहु॒यात् । अप्र॑जा अप॒शुर्यज॑मानः स्यात् । यदना॑यतने नि॒नये᳚त् । अ॒ना॒य॒त॒न-स्स्या᳚त् । प्रा॒जा॒प॒त्यय॒र्चा व॑ल्मीक-व॒पाया॒म-व॑नयेत् । प्रा॒जा॒प॒त्यो वै व॒ल्मीकः॑ । य॒ज्ञ्ः प्र॒जाप॑तिः । प्र॒जाप॑तावे॒व य॒ज्ञ्ं प्रति॑ष्ठापयति । भूरित्या॑ह । भू॒तो वै प्र॒जाप॑तिः \textbf{ 10} \newline
                  \newline
                                \textbf{ TB 3.7.2.2} \newline
                  भूति॑मे॒वोपै॑ति । तत् कृ॒त्वा । अ॒न्यां दु॒ग्ध्वा पुन॑र्. होत॒व्यं᳚ । सैव ततः॒ प्राय॑श्चित्तिः ॥ यत्की॒टा-व॑पन्नेन जुहु॒यात् । अप्र॑जा अप॒शु-र्यज॑मानः स्यात् । यदना॑यतने नि॒नये᳚त् । अ॒ना॒य॒त॒नः स्या᳚त् । म॒द्ध्य॒मेन॑ प॒र्णेन॑ द्यावा-पृथि॒व्य॑-य॒र्चाऽन्तः॑ परि॒धि निन॑येत् । द्यावा॑पृथि॒व्यो-रे॒वैन॒त्-प्रति॑ष्ठापयति \textbf{ 11} \newline
                  \newline
                                \textbf{ TB 3.7.2.3} \newline
                  तत्कृ॒त्वा । अ॒न्यां दु॒ग्ध्वा पुन॑र्. होत॒व्यं᳚ । सैव ततः॒ प्राय॑श्चित्तिः ॥ यदव॑वृष्टेन जुहु॒यात् । अप॑रूप-मस्या॒त्म-ञ्जा॑येत । कि॒लासो॑ वा॒ स्याद॑र्.श॒सो वा᳚ । यत् प्रत्ये॒यात् । य॒ज्ञ्ं ॅविच्छि॑न्द्यात् । स जु॑हुयात् । मि॒त्रो जना᳚न् कल्पयति प्रजा॒नन्न् \textbf{ 12} \newline
                  \newline
                                \textbf{ TB 3.7.2.4} \newline
                  मि॒त्रो दा॑धार पृथि॒वीमु॒त द्यां । मि॒त्रः कृ॒ष्टीर-नि॑मिषा॒ ऽभिच॑ष्टे । स॒त्याय॑ ह॒व्यं घृ॒तव॑-ज्जुहो॒तेति॑ । मि॒त्रेणै॒वैन॑त् कल्पयति । तत्कृ॒त्वा । अ॒न्यां दु॒ग्ध्वा पुन॑र्.होत॒व्यं᳚ । सैव ततः॒ प्राय॑श्चित्तिः ॥ यत् पूर्व॑स्या॒-माहु॑त्याꣳ हु॒ताया॒-मुत्त॒रा-ऽऽहु॑तिः॒ स्कन्दे᳚त् । द्वि॒पाद्भिः॑ प॒शुभि॒-र्यज॑मानो॒ व्यृ॑॑द्ध्येत । यदुत्त॑रया॒ ऽभिजु॑हु॒यात् \textbf{ 13} \newline
                  \newline
                                \textbf{ TB 3.7.2.5} \newline
                  चतु॑ष्पाद्भिः प॒शुभि॒-र्यज॑मानो॒ व्यृ॑द्ध्येत । यत्र॒ वेत्थ॑ वनस्पते दे॒वानां॒ गुह्या॒ नामा॑नि । तत्र॑ ह॒व्यानि॑ गाम॒येति॑ वानस्प॒त्य-य॒र्चा स॒मिध॑-मा॒धाय॑ । तू॒ष्णीमे॒व पुन॑-र्जुहुयात् । वन॒स्पति॑नै॒व य॒ज्ञ्स्यार्तां॒ चाना᳚र्तां॒ चाहु॑ती॒ विदा॑धार । तत्कृ॒त्वा । अ॒न्यां दु॒ग्ध्वा पुन॑र्. होत॒व्यं᳚ । सैव ततः॒ प्राय॑श्चित्तिः ॥ यत्पु॒रा प्र॑या॒जेभ्यः॒ प्राङङ्गा॑रः॒ स्कन्दे᳚त् । अ॒द्ध्व॒र्यवे॑ च॒ यज॑मानाय॒ चाकꣳ॑ स्यात् \textbf{ 14} \newline
                  \newline
                                \textbf{ TB 3.7.2.6} \newline
                  यद् द॑क्षि॒णा । ब्र॒ह्मणे॑ च॒ यज॑मानाय॒ चाकꣳ॑ स्यात् । यत् प्र॒त्यक् । होत्रे॑ च॒ पत्नि॑यै च॒ यज॑मानाय॒ चाकꣳ॑ स्यात् । यदुदङ्॑ । अ॒ग्नीधे॑ च प॒शुभ्य॑श्च॒ यज॑मानाय॒ चाकꣳ॑ स्यात् । यद॑भिजुहु॒यात् । रु॒द्रो᳚ऽस्य प॒शून् घातु॑कः स्यात् । यन्नाभि॑ जुहु॒यात् । अशा᳚न्तः॒ प्रह्रि॑येत । \textbf{ 15} \newline
                  \newline
                                \textbf{ TB 3.7.2.7} \newline
                  स्रु॒वस्य॒ बुद्ध्ने॑नाभि॒ निद॑द्ध्यात् । मा त॑मो॒ मा य॒ज्ञ्स्त॑म॒न्मा यज॑मान-स्तमत् । नम॑स्ते अस्त्वाय॒ते । नमो॑ रुद्र पराय॒ते । नमो॒ यत्र॑ नि॒षीद॑सि । अ॒मुं मा हिꣳ॑सीर॒मुं मा हिꣳ॑सी॒रिति॒ येन॒ स्कन्दे᳚त् ॥ तं प्रह॑रेत् । स॒हस्र॑शृङ्गो वृष॒भो जा॒तवे॑दाः । स्तोम॑पृष्ठो घृ॒तवा᳚न्-थ्सु॒प्रती॑कः । मा नो॑ हासीन्मेत्थि॒तो (हासीर्मेत्थि॒तो) नेत्त्वा॒ जहा॑म ( ) । गो॒पो॒षं नो॑ वीरपो॒षं च॑ य॒च्छेति॑ । ब्रह्म॑णै॒वैनं॒ प्रह॑रति । सैव ततः॒ प्राय॑श्चित्तिः । \textbf{ 16} \newline
                  \newline
                                                        \textbf{special korvai} \newline
              (यद्विःष॑ण्णेन प्राजाप॒त्यया॒ यत् की॒टा म॑द्ध्य॒मेन॒ यदव॑वृष्टेन॒ यत् पूर्व॑स्यां॒ ॅयत् पु॒रा प्र॑या॒जेभ्य॒ प्रङ्ङ्गा॑रो॒ यद् द॑क्षि॒णा यत् प्र॒त्यग्यदुदङ्ङ्॑) \newline
                                (वै प्र॒जाप॑तिः - स्थापयति - प्रजा॒नन् - न॒भिजु॑हु॒याथ् - स्या᳚द् - ध्रियेत॒ - जहा॑म॒ त्रिणि॑ च) \textbf{(A2)} \newline \newline
                \textbf{ 3.7.3     अनुवाकं   3 -मुख्याग्न्यसंभवेऽनुकल्पाः होमाधाराः} \newline
                                \textbf{ TB 3.7.3.1} \newline
                  वि वा ए॒ष इ॑न्द्रि॒येण॑ वी॒र्ये॑णर्द्ध्यते । यस्या-हि॑ताग्ने-र॒ग्नि-र्म॒थ्यमा॑नो॒ न जाय॑ते । यत्रा॒न्यं पश्ये᳚त् । तत॑ आ॒हृत्य॑ होत॒व्यं᳚ । अ॒ग्ना-वे॒वास्या᳚-ग्निहो॒त्रꣳ हु॒तं भ॑वति ॥ यद्य॒न्यं न वि॒न्देत् । अ॒जायाꣳ॑ होत॒व्यं᳚ । आ॒ग्ने॒यी वा ए॒षा । यद॒जा । अ॒ग्ना-वे॒वास्या᳚-ग्निहो॒त्रꣳ हु॒तं भ॑वति \textbf{ 17} \newline
                  \newline
                                \textbf{ TB 3.7.3.2} \newline
                  अ॒जस्य॒ तु नाश्ञी॑यात् । यद॒जस्या᳚-श्ञी॒यात् । यामे॒वाग्ना-वाहु॑तिं जुहु॒यात् । ताम॑द्यात् । तस्मा॑द॒जस्य॒ नाश्यं᳚ ॥ यद्य॒जां न वि॒न्देत् । ब्रा॒ह्म॒णस्य॒ दक्षि॑णे॒ हस्ते॑ होत॒व्यं᳚ । ए॒ष वा अ॒ग्नि र्वै᳚श्वान॒रः । यद्ब्रा᳚ह्म॒णः । अ॒ग्ना-वे॒वास्या᳚-ग्निहो॒त्रꣳ हु॒तं भ॑वति \textbf{ 18} \newline
                  \newline
                                \textbf{ TB 3.7.3.3} \newline
                  ब्रा॒ह्म॒णं तु व॑स॒त्यै॑ नाप॑रुन्ध्यात् । यद्ब्रा᳚ह्म॒णं ॅव॑स॒त्या अ॑परु॒न्ध्यात् । यस्मि॑न्ने॒-वाग्नावाहु॑तिं जुहु॒यात् । तं भा॑ग॒धेये॑न॒ व्य॑र्द्धयेत् । तस्मा᳚द् ब्राह्म॒णो व॑स॒त्यै॑ नाप॒रुद्ध्यः॑ ॥ यदि॑ ब्राह्म॒णं न वि॒न्देत् । द॒र्भ॒स्त॒बें हो॑त॒व्यं᳚ । अ॒ग्नि॒वान्. वै द॑र्भस्त॒बंः । अ॒ग्नावे॒-वास्या᳚-ग्निहो॒त्रꣳ हु॒तं भ॑वति । द॒र्भाꣳस्तु नाद्ध्या॑सीत \textbf{ 19} \newline
                  \newline
                                \textbf{ TB 3.7.3.4} \newline
                  यद्द॒र्भान॒द्ध्यासी॑त । यामे॒वाग्ना-वाहु॑तिं जुहु॒यात् । तामद्ध्या॑सीत । तस्मा᳚द् द॒र्भा नाद्ध्या॑सित॒व्याः᳚ ॥ यदि॑ द॒र्भान्न वि॒न्देत् । अ॒फ्सु हो॑त॒व्यं᳚ । आपो॒ वै सर्वा॑ दे॒वताः᳚ । दे॒वता᳚स्वे॒ वास्या᳚-ग्निहो॒त्रꣳ हु॒तं भ॑वति । आप॒स्तु न परि॑चक्षीत । यदापः॑ परि॒चक्षी॑त \textbf{ 20} \newline
                  \newline
                                \textbf{ TB 3.7.3.5} \newline
                  या मे॒वाफ्स्वाहु॑तिं जुहु॒यात् । तां परि॑चक्षीत । तस्मा॒दापो॒ न प॑रि॒चक्ष्याः᳚ ॥ मेद्ध्या॑ च॒ वा ए॒तस्या॑मे॒द्ध्या च॑ त॒नुवौ॒ सꣳसृ॑ज्येते । यस्या-हि॑ताग्ने-र॒न्यैर॒ग्निभि॑-र॒ग्नयः॑ सꣳसृ॒ज्यन्ते᳚ । अ॒ग्नये॒ विवि॑चये पुरो॒डाश॑-म॒ष्टाक॑पालं॒ निर्व॑पेत् । मेद्ध्यां᳚ चै॒वास्या॑-मे॒द्ध्यां च॑ त॒नुवौ॒ व्याव॑र्तयति ॥ अ॒ग्नये᳚ व्र॒तप॑तये पुरो॒डाश॑-म॒ष्टाक॑पालं॒ निर्व॑पेत् । अ॒ग्निमे॒व व्र॒तप॑तिꣳ॒॒ स्वेन॑ भाग॒धेये॒-नोप॑धावति । स ए॒वैनं॑ ॅव्र॒तमा-ल॑भंयति । \textbf{ 21} \newline
                  \newline
                                \textbf{ TB 3.7.3.6} \newline
                  गर्भꣳ॒॒ स्रव॑न्तमग॒दम॑कः । अ॒ग्नि-रिन्द्र॒-स्त्वष्टा॒ बृह॒स्पतिः॑ । पृ॒थि॒व्या-मव॑चुश्चो-तै॒तत् । नाभि-प्राप्नो॑ति॒ निर्.ऋ॑तिं परा॒चैः ॥ रेतो॒ वा ए॒तद्वाजि॑न॒-माहि॑ताग्नेः । यद॑ग्निहो॒त्रं । तद्यथ्स्रवे᳚त् । रेतो᳚ऽस्य॒ वाजि॑नꣳ स्रवेत् । गर्भꣳ॒॒ स्रव॑न्तम-ग॒दम॑क॒-रित्या॑ह । रेत॑ ए॒वास्मि॒न् वाजि॑नं दधाति । \textbf{ 22} \newline
                  \newline
                                \textbf{ TB 3.7.3.7} \newline
                  अ॒ग्निरित्या॑ह । अ॒ग्निर्वै रे॑तो॒धाः । रेत॑ ए॒व तद्द॑धाति । इन्द्र॒ इत्या॑ह । इ॒न्द्रि॒य-मे॒वास्मि॑न्-दधाति । त्वष्टेत्या॑ह । त्वष्टा॒ वै प॑शू॒नां मि॑थु॒नानाꣳ॑ रूप॒कृत् । रू॒पम॒व प॒शुषु॑ दधाति । बृह॒स्पति॒रित्या॑ह । ब्रह्म॒ वै दे॒वानां॒ बृह॒स्पतिः॑ ( ) । ब्रह्म॑णै॒वास्मै᳚ प्र॒जाः प्रज॑नयति ॥ पृ॒थि॒व्या-मव॑चुश्चोतै॒त-दित्या॑ह । अ॒स्या-मे॒वैन॒त् प्रति॑ष्ठापयति ॥ नाभि प्राप्नो॑ति॒ निर्.ऋ॑तिं परा॒चैरित्या॑ह । रक्ष॑सा॒मप॑हत्यै । \textbf{ 23} \newline
                  \newline
                                                        \textbf{special korvai} \newline
              (वि वै यद् य॒न्यम॒जायां᳚ ब्राह्म॒णस्य॑ दर्भस्त॒म्बे᳚ऽफ्सु हो॑त॒व्य᳚म्) \newline
                                (अ॒जाऽग्नावे॒वास्या᳚ग्निहो॒त्रꣳ हु॒तं भ॑वति - भव - त्यासीत - परि॒चक्षी॑त - लम्भयति - दधाति - दे॒वानां॒ बृह॒स्पतिः॒ पञ्च॑ च) \textbf{(A3)} \newline \newline
                \textbf{ 3.7.4     अनुवाकं   4 -ऐष्टिकयाजमानमन्त्राः} \newline
                                \textbf{ TB 3.7.4.1} \newline
                  याः पु॒रस्ता᳚त् प्र॒स्रव॑न्ति । उ॒परि॑ष्टाथ् स॒र्वत॑श्च॒ याः । ताभी॑ र॒श्मिप॑वित्राभिः । श्र॒द्धां ॅय॒ज्ञ्मार॑भे ॥ देवा॑ गातुविदः । गा॒तुं ॅय॒ज्ञाय॑ विन्दत । मन॑स॒स्पति॑ना दे॒वेन॑ । वाता᳚द्-य॒ज्ञ्ः प्रयु॑ज्यतां ॥ तृ॒तीय॑स्यै दि॒वः । गा॒य॒त्रि॒या सोम॒ आभृ॑तः \textbf{ 24} \newline
                  \newline
                                \textbf{ TB 3.7.4.2} \newline
                  सो॒म॒पी॒थाय॒ संन॑यितुं । वक॑ल॒-मन्त॑र॒माद॑दे ॥ आपो॑ देवीः शु॒द्धाः स्थ॑ । इ॒मा पात्रा॑णि शुन्धत । उ॒पा॒त॒ङ्क्या॑य दे॒वानां᳚ । प॒र्ण॒व॒ल्कमु॒त शु॑न्धत ॥ पयो॑ गृ॒हेषु॒ पयो॑ अघ्नि॒यासु॑ । पयो॑ व॒थ्सेषु॒ पय॒ इन्द्रा॑य ह॒विषे᳚ ध्रियस्व । गा॒य॒त्री प॑र्णव॒ल्केन॑ । पयः सोमं॑ करोत्वि॒मं । \textbf{ 25} \newline
                  \newline
                                \textbf{ TB 3.7.4.3} \newline
                  अ॒ग्निं गृ॑ह्णामि सु॒रथं॒ ॅयो म॑यो॒ भूः । य उ॒द्यन्त॑मा॒ रोह॑ति॒ सूर्य॒मह्ने᳚ । आ॒दि॒त्यं ज्योति॑षां॒ ज्योति॑रुत्त॒मं । श्वो य॒ज्ञाय॑ रमतां दे॒वता᳚भ्यः ॥वसू᳚न् रु॒द्राना॑दि॒त्यान् । इन्द्रे॑ण स॒ह दे॒वताः᳚ । ताः पूर्वः॒ परि॑गृह्णामि । स्व आ॒यत॑ने मनी॒षया᳚ ॥ इ॒मामूर्जं॑ पञ्चद॒शीं ॅये प्रवि॑ष्टाः । तान्दे॒वान् परि॑गृह्णामि॒ पूर्वः॑ \textbf{ 26} \newline
                  \newline
                                \textbf{ TB 3.7.4.4} \newline
                  अ॒ग्निर्. ह॑व्य॒वाडि॒ह तानाव॑हतु । पौ॒र्ण॒मा॒सꣳ ह॒विरि॒दमे॑षां॒ मयि॑ । आ॒मा॒वा॒स्यꣳ॑ ह॒विरि॒दमे॑षां॒ मयि॑ ॥ अ॒न्त॒राऽग्नी प॒शवः॑ । दे॒व॒सꣳ॒॒ सद॒माग॑मन्न् । तान् पूर्वः॒ परि॑गृह्णामि । स्व आ॒यत॑ने मनी॒षया᳚ ॥ इ॒ह प्र॒जा वि॒श्वरू॑पा रमन्तां । अ॒ग्निं गृ॒हप॑तिम॒भि स॒म्ॅवसा॑नाः । ताः पूर्वः॒ परि॑गृह्णामि \textbf{ 27} \newline
                  \newline
                                \textbf{ TB 3.7.4.5} \newline
                  स्व आ॒यत॑ने मनी॒षया᳚ ॥ इ॒ह प॒शवो॑ वि॒श्वरू॑पा रमन्तां । अ॒ग्निं गृ॒हप॑तिम॒भि स॒म्ॅवसा॑नाः । तान् पूर्वः॒ परि॑गृह्णामि । स्व आ॒यत॑ने मनी॒षया᳚ ॥ अ॒यं पि॑तृ॒णाम॒ग्निः । अवा᳚ड्ढ॒व्या पि॒तृभ्य॒ आ । तं पूर्वः॒ परि॑गृह्णामि । अवि॑षं नः पि॒तुं क॑रत् ॥ अज॑स्रं॒ त्वाꣳ स॑भापा॒लाः \textbf{ 28} \newline
                  \newline
                                \textbf{ TB 3.7.4.6} \newline
                  वि॒ज॒यभा॑गꣳ॒॒ समि॑न्धतां । अग्ने॑ दी॒दाय॑ मे सभ्य । विजि॑त्यै श॒रदः॑ श॒तं ॥ अन्न॑मावस॒थीयं᳚ । अ॒भिह॑राणि श॒रदः॑ श॒तं । आ॒व॒स॒थे श्रियं॒ मन्त्रं᳚ । अहि॑र्बु॒द्ध्नियो॒ निय॑च्छतु ॥ इ॒दम॒ह-म॒ग्नि ज्ये᳚ष्ठेभ्यः । वसु॑भ्यो य॒ज्ञ्ं प्रब्र॑वीमि । इ॒दम॒ह-मिन्द्र॑ ज्येष्ठेभ्यः \textbf{ 29} \newline
                  \newline
                                \textbf{ TB 3.7.4.7} \newline
                  रु॒द्रेभ्यो॑ य॒ज्ञ्ं प्रब्र॑वीमि । इ॒दम॒हं ॅवरु॑ण-ज्येष्ठेभ्यः । आ॒दि॒त्येभ्यो॑ य॒ज्ञ्ं प्रब्र॑वीमि ॥ पय॑स्वती॒-रोष॑धयः । पय॑स्वद्वी॒रुधां॒ पयः॑ । अ॒पां पय॑सो॒ यत्पयः॑ । तेन॒ मामि॑न्द्र॒ सꣳसृ॑ज ॥ अग्ने᳚ व्रतपते व्र॒तं च॑रिष्यामि । तच्छ॑केयं॒ तन्मे॑ राद्ध्यतां । वायो᳚ व्रतपत॒ आदि॑त्य व्रतपते \textbf{ 30} \newline
                  \newline
                                \textbf{ TB 3.7.4.8} \newline
                  व्र॒तानां᳚ ॅव्रतपते व्र॒तं च॑रिष्यामि । तच्छ॑केयं॒ तन्मे॑ राद्ध्यतां ॥ इ॒मां प्राची॒मुदी॑चीं । इष॒मूर्ज॑म॒भि सꣳस्कृ॑तां । ब॒हु॒प॒र्णामशु॑ष्काग्रां । हरा॑मि पशु॒पाम॒हं ॥ यत्कृष्णो॑ रू॒पं कृ॒त्वा । प्रावि॑श॒स्त्वं ॅवन॒स्पतीन्॑ । तत॒स्त्वामे॑कविꣳशति॒धा । संभ॑रामि सुसं॒भृता᳚ । \textbf{ 31} \newline
                  \newline
                                \textbf{ TB 3.7.4.9} \newline
                  त्रीन् प॑रि॒धीꣳस्ति॒स्रः स॒मिधः॑ । य॒ज्ञायु॑रनुसंच॒रान् । उ॒प॒वे॒षं मेक्ष॑णं॒ धृष्टिं᳚ । संभ॑रामि सुस॒भृंता᳚ ॥ या जा॒ता ओष॑धयः । दे॒वेभ्य॑ स्त्रियु॒गं पु॒रा । तासां॒ पर्व॑ राद्ध्यासं । प॒रि॒स्त॒र-मा॒हरन्न्॑ ॥ अ॒पां मेद्ध्यं॑ ॅय॒ज्ञियं᳚ । सदे॑वꣳ शि॒वम॑स्तु मे \textbf{ 32} \newline
                  \newline
                                \textbf{ TB 3.7.4.10} \newline
                  आ॒च्छे॒त्ता वो॒ मा रि॑षं । जीवा॑नि श॒रदः॑ श॒तं ॥ अप॑रिमितानां॒ परि॑मिताः । संन॑ह्ये सुकृ॒ताय॒कं । एनो॒ मा निगां᳚ कत॒मच्च॒ नाहं । पुन॑रु॒त्थाय॑ बहु॒ला भ॑वन्तु ॥ स॒कृ॒दा॒च्छि॒न्नं ब॒र्॒.हिरूर्णा॑मृदु । स्यो॒नं पि॒तृभ्य॑स्त्वा भराम्य॒हं । अ॒स्मिन् थ्सी॑दन्तु मे पि॒तरः॑ सो॒म्याः । पि॒ता॒म॒हाः प्रपि॑तामहाश्चानु॒गैः स॒ह । \textbf{ 33} \newline
                  \newline
                                \textbf{ TB 3.7.4.11} \newline
                  त्रि॒वृत् प॑ला॒शे द॒र्भः । इया᳚न् प्रादे॒श स॑मिंतः । य॒ज्ञे प॒वित्रं॒ पोतृ॑तमं । पयो॑ ह॒व्यं क॑रोतु मे ॥ इ॒मौ प्रा॑णापा॒नौ । य॒ज्ञ्स्याङ्गा॑नि सर्व॒शः । आ॒प्या॒यय॑न्तौ॒ संच॑रतां । प॒वित्रे॑ हव्य॒शोध॑ने ॥ प॒वित्रे᳚ स्थो वैष्ण॒वी । वा॒युर्वां॒ मन॑सा पुनातु । \textbf{ 34} \newline
                  \newline
                                \textbf{ TB 3.7.4.12} \newline
                  अ॒यं प्रा॒णश्चा॑पा॒नश्च॑ । यज॑मान॒-मपि॑गच्छतां । य॒ज्ञे ह्यभू॑तां॒ पोता॑रौ । प॒वित्रे॑ हव्य॒शोध॑ने ॥ त्वया॒ वेदिं॑ ॅविविदुः पृथि॒वीं । त्वया॑ य॒ज्ञो जा॑यते विश्व॒दानिः॑ । अच्छि॑द्रं ॅय॒ज्ञ्मन्वे॑षि वि॒द्वान् । त्वया॒ होता॒ सन्त॑नोत्यर्द्धमा॒सान् ॥ त्र॒य॒स्त्रिꣳ॒॒शो॑ऽसि॒ तन्तू॑नां । प॒वित्रे॑ण स॒हाग॑हि \textbf{ 35} \newline
                  \newline
                                \textbf{ TB 3.7.4.13} \newline
                  शि॒वेयꣳ रज्जु॑रभि॒धानी᳚ । अ॒घ्नि॒या-मुप॑सेवतां ॥ अप्र॑स्रꣳसाय य॒ज्ञ्स्य॑ । उ॒खे उप॑दधाम्य॒हं । प॒शुभिः॒ संनी॑तं बिभृतां । इन्द्रा॑य शृ॒तं दधि॑ ॥ उ॒प॒वे॒षो॑ऽसि य॒ज्ञाय॑ । त्वां प॑रिवे॒ष-म॑धारयन्न् । इन्द्रा॑य ह॒विः कृ॒ण्वन्तः॑ । शि॒वः श॒ग्मो भ॑वासि नः । \textbf{ 36} \newline
                  \newline
                                \textbf{ TB 3.7.4.14} \newline
                  अमृ॑न्मयं देवपा॒त्रं । य॒ज्ञ्स्यायु॑षि॒ प्रयु॑ज्यतां । ति॒रः॒ प॒वि॒त्रमति॑नीताः । आपो॑ धारय॒ माऽति॑गुः ॥ दे॒वेन॑ सवि॒त्रोत्पू॑ताः । वसोः॒ सूर्य॑स्य र॒श्मिभिः॑ । गां दो॑हपवि॒त्रे रज्जुं᳚ । सर्वा॒ पात्रा॑णि शुन्धत ॥ ए॒ता आच॑रन्ति॒ मधु॑म॒द्दुहा॑नाः । प्र॒जाव॑ती-र्य॒शसो॑ वि॒श्वरू॑पाः \textbf{ 37} \newline
                  \newline
                                \textbf{ TB 3.7.4.15} \newline
                  ब॒ह्वी र्भव॑न्ती॒रुप॒ जाय॑मानाः । इ॒ह व॒ इन्द्रो॑ रमयतु गावः ॥ पू॒षा स्थ॑ ॥ अ॒य॒क्ष्मा वः॑ प्र॒जया॒ सꣳसृ॑जामि । रा॒यस्पोषे॑ण बहु॒ला भव॑न्तीः । ऊर्जं॒ पयः॒ पिन्व॑माना घृ॒तं च॑ । जी॒वो जीव॑न्ती॒रुप॑ वः सदेयं ॥ द्यौश्चे॒मं ॅय॒ज्ञ्ं पृ॑थि॒वी च॒ संदु॑हातां । धा॒ता सोमे॑न स॒ह वाते॑न वा॒युः । यज॑मानाय॒ द्रवि॑णं दधातु । \textbf{ 38} \newline
                  \newline
                                \textbf{ TB 3.7.4.16} \newline
                  उथ्सं॑ दुहन्ति क॒लशं॒ चतु॑र्बिलं । इडां᳚ दे॒वीं मधु॑मतीꣳ सुव॒र्विदं᳚ । तदि॑न्द्रा॒ग्नी जि॑न्वतꣳ सू॒नृता॑वत् । तद्-यज॑मान-ममृत॒त्वे द॑धातु ॥ काम॑धुक्षः॒ प्रणो᳚ ब्रूहि । इन्द्रा॑य ह॒विरि॑न्द्रि॒यं ॥ अ॒मूं ॅयस्यां᳚ दे॒वानां᳚ । म॒नु॒ष्या॑णां॒ पयो॑ हि॒तं ॥ ब॒हु दु॒ग्धीन्द्रा॑य दे॒वेभ्यः॑ । ह॒व्य-माप्या॑यतां॒ पुनः॑ \textbf{ 39} \newline
                  \newline
                                \textbf{ TB 3.7.4.17} \newline
                  व॒थ्सेभ्यो॑ मनु॒ष्ये᳚भ्यः । पु॒न॒र्दो॒हाय॑ कल्पतां ॥ य॒ज्ञ्स्य॒ सन्त॑तिरसि । य॒ज्ञ्स्य॑ त्वा॒ सन्त॑ति॒मनु॒सन्त॑नोमि ॥ अद॑स्तमसि॒ विष्ण॑वे त्वा । य॒ज्ञायापि॑ दधाम्य॒हं । अ॒द्भिररि॑क्तेन॒ पात्रे॑ण । याः पू॒ताः प॑रि॒शेर॑ते ॥ अ॒यं पयः॒ सोमं॑ कृ॒त्वा । स्वां ॅयोनि॒-मपि॑गच्छतु \textbf{ 40} \newline
                  \newline
                                \textbf{ TB 3.7.4.18} \newline
                  प॒र्ण॒व॒ल्कः प॒वित्रं᳚ । सौ॒म्यः सोमा॒द्धि निर्मि॑तः ॥ इ॒मौ प॒र्णं च॑ द॒र्भं च॑ । दे॒वानाꣳ॑ हव्य॒शोध॑नौ । प्रा॒त॒र्वे॒षाय॑ गोपाय । विष्णो॑ ह॒व्यꣳ हि रक्ष॑सि ॥ उ॒भाव॒ग्नी उ॑पस्तृण॒ते । दे॒वता॒ उप॑वसन्तु मे । अ॒हं ग्रा॒म्या-नुप॑वसामि । मह्यं॒ गोप॑तये प॒शून् ( ) । \textbf{ 41} \newline
                  \newline
                                                        \textbf{special korvai} \newline
              याः पु॒रस्ता॑दि॒मामूर्ज॑मि॒ह प्र॒जा इ॒ह प॒शवो॒ऽयं पि॑तृ॒णाम॒ग्निः \newline
                                [आभृ॑त - इ॒मं - गृ॑ह्णामि॒ पूर्व॒ - स्ताः पुर्वः॒ परि॑गृह्णामि - सभापा॒ला - इन्द्र॑ज्येष्ठेभ्य॒-आदि॑त्य व्रतपते-सुस॒भ्रुंता॑ - मे - स॒ह - पु॑नातु - गहि - नो - वि॒श्वरू॑पा - दधातु॒ - पुन॑र् - गच्छतु - प॒शुन् ( ) ] \textbf{(A4)} \newline \newline
                \textbf{ 3.7.5     अनुवाकं   5 -ऐष्टिकयाजमानमन्त्राः} \newline
                                \textbf{ TB 3.7.5.1} \newline
                  देवा॑ दे॒वेषु॒ परा᳚क्रमद्ध्वं । प्रथ॑मा द्वि॒तीये॑षु । द्विती॑या-स्तृ॒तीये॑षु । त्रिरे॑कादशा इ॒ह मा॑ऽवत । इ॒दꣳ श॑केयं॒ ॅयदि॒दं क॒रोमि॑ । आ॒त्मा क॑रोत्वा॒त्मने᳚ । इ॒दं क॑रिष्ये भेष॒जं । इ॒दं मे॑ विश्वभेषजा । अश्वि॑ना॒ प्राव॑तं ॅयु॒वं ॥ इ॒दम॒हꣳ सेना॑या अ॒भीत्व॑र्यै \textbf{ 42} \newline
                  \newline
                                \textbf{ TB 3.7.5.2} \newline
                  मुख॒मपो॑हामि ॥ सूर्य॑ज्योति॒ र्विभा॑हि । म॒ह॒त इ॑न्द्रि॒याय॑ ॥ आप्या॑यतां घृ॒तयो॑निः । अ॒ग्निर्. ह॒व्या-ऽनु॑मन्यतां । खम॑ङ्क्ष्व॒ त्वच॑मङ्क्ष्व । सु॒रू॒पं त्वा॑ वसु॒विदं᳚ । प॒शू॒नां तेज॑सा । अ॒ग्नये॒ जुष्ट॑-म॒भिघा॑रयामि ॥ स्यो॒नं ते॒ सद॑नं करोमि \textbf{ 43} \newline
                  \newline
                                \textbf{ TB 3.7.5.3} \newline
                  घृ॒तस्य॒ धार॑या सु॒शेवं॑ कल्पयामि ॥ तस्मि᳚न्-थ्सीदा॒मृते॒ प्रति॑तिष्ठ । व्री॒ही॒णां मे॑ध सुमन॒स्यमा॑नः ॥ आ॒र्द्रः प्र॑थस्नु॒र् भुव॑नस्य गो॒पाः । शृ॒त उथ्स्ना॑ति जनि॒ता म॑ती॒नां ॥ यस्त॑ आ॒त्मा प॒शुषु॒ प्रवि॑ष्टः । दे॒वानां᳚ ॅवि॒ष्ठामनु॒ यो वि॑त॒स्थे । आ॒त्म॒न्वान् थ्सो॑म घृ॒तवा॒न॒. हि भू॒त्वा । दे॒वान् ग॑च्छ॒ सुव॑र्विन्द॒ यज॑मानाय॒ मह्यं᳚ ॥ इरा॒ भूतिः॑ पृथि॒व्यै रसो॒ मोत्क्र॑मीत् । \textbf{ 44} \newline
                  \newline
                                \textbf{ TB 3.7.5.4} \newline
                  देवाः᳚ पितरः॒ पित॑रो देवाः । यो॑ऽहम॑स्मि॒ स सन्. य॑जे । यस्या᳚स्मि॒ न तम॒न्तरे॑मि । स्वं म॑ इ॒ष्टꣳ स्वं द॒त्तं । स्वं पू॒र्तꣳ स्वꣳ श्रा॒न्तं । स्वꣳ हु॒तं । तस्य॑ मे॒ऽग्नि-रु॑पद्र॒ष्टा । वा॒युरु॑पश्रो॒ता । आ॒दि॒त्यो॑-ऽनुख्या॒ता । द्यौः पि॒ता \textbf{ 45} \newline
                  \newline
                                \textbf{ TB 3.7.5.5} \newline
                  पृ॒थि॒वी मा॒ता । प्र॒जाप॑ति॒ र्बन्धुः॑ । य ए॒वास्मि॒ स सन्. य॑जे ॥ मा भेर्मा सम्ॅवि॑क्था॒ मा त्वा॑ हिꣳसिषं । मा ते॒ तेजो-ऽप॑क्रमीत् । भ॒र॒तमुद्ध॑रे॒ मनु॑षिञ्च । अ॒व॒दाना॑नि ते प्र॒त्यव॑दास्यामि । नम॑स्ते अस्तु॒ मा मा॑ हिꣳसीः ॥ यद॑व॒दाना॑नि तेऽव॒द्यन्न् । विलो॒माका॑र्.षमा॒त्मनः॑ \textbf{ 46} \newline
                  \newline
                                \textbf{ TB 3.7.5.6} \newline
                  आज्ये॑न॒ प्रत्य॑नज्म्येनत् । तत्त॒ आप्या॑यतां॒ पुनः॑ ॥ अज्या॑यो यवमा॒त्रात् । आ॒व्या॒धात् कृ॑त्यतामि॒दं । मा रू॑रुपाम य॒ज्ञ्स्य॑ । शु॒द्धꣳ स्वि॑ष्टमि॒दꣳ ह॒विः ॥ मनु॑ना दृ॒ष्टां घृ॒तप॑दीं । मि॒त्रावरु॑णसमीरितां । द॒क्षि॒णा॒र्द्धादस॑भिंन्दन्न् । अव॑द्या-म्येक॒तो मु॑खां । \textbf{ 47} \newline
                  \newline
                                \textbf{ TB 3.7.5.7} \newline
                  इडे॑ भा॒गं जु॑षस्व नः । जिन्व॒ गा जिन्वार्व॑तः । तस्या᳚स्ते भक्षि॒वाणः॑ स्याम । स॒र्वात्मा॑नः स॒र्वग॑णाः ॥ ब्रद्ध्न॒ पिन्व॑स्व । दद॑तो मे॒ मा क्षा॑यि । कु॒र्व॒तो मे॒ मोप॑दसत् । दि॒शां क्लृप्ति॑रसि । दिशो॑ मे कल्पन्तां । कल्प॑न्तां मे॒ दिशः॑ \textbf{ 48} \newline
                  \newline
                                \textbf{ TB 3.7.5.8} \newline
                  दैवी᳚श्च॒ मानु॑षीश्च । अ॒हो॒रा॒त्रे मे॑ कल्पेतां । अ॒र्द्ध॒मा॒सा मे॑ कल्पन्तां । मासा॑ मे कल्पन्तां । ऋ॒तवो॑ मे कल्पन्तां । स॒म्ॅव॒थ्स॒रो मे॑ कल्पतां । क्लृप्ति॑रसि॒ कल्प॑तां मे ॥ आशा॑नां त्वाऽऽशापा॒लेभ्यः॑ । च॒तुर्भ्यो॑ अ॒मृते᳚भ्यः । इ॒दं भू॒तस्या-द्ध्य॑क्षेभ्यः \textbf{ 49} \newline
                  \newline
                                \textbf{ TB 3.7.5.9} \newline
                  वि॒धेम॑ ह॒विषा॑ व॒यं । भज॑तां भा॒गी भा॒गं । माऽभा॒गो-ऽभ॑क्त । निर॑भा॒गं भ॑जामः । अ॒पस्पि॑न्व । ओष॑धीर्जिन्व । द्वि॒पात्पा॑हि । चतु॑ष्पादव । दि॒वो वृष्टि॒मेर॑य । ब्रा॒ह्म॒णाना॑-मि॒दꣳ ह॒विः \textbf{ 50} \newline
                  \newline
                                \textbf{ TB 3.7.5.10} \newline
                  सो॒म्यानाꣳ॑ सोमपी॒थिनां᳚ । निर्भ॒क्तो ब्रा᳚ह्मणः । नेहा-ब्रा᳚ह्मणस्यास्ति ॥ सम॑ङ्क्तां ब॒र॒.हिर्. ह॒विषा॑ घृ॒तेन॑ । समा॑दि॒त्यै र्वसु॑भिः॒ सं म॒रुद्भिः॑ । समिन्द्रे॑ण॒ विश्वे॑भि र्दे॒वेभि॑रङ्क्तां । दि॒व्यं नभो॑ गच्छतु॒ यथ् स्वाहा᳚ ॥ इ॒न्द्रा॒णी वा॑ विध॒वा भू॑यासं । अदि॑तिरिव सुपु॒त्रा । अ॒स्थू॒रि त्वा॑ गार्.हपत्य \textbf{ 51} \newline
                  \newline
                                \textbf{ TB 3.7.5.11} \newline
                  उप॒निष॑दे सुप्रजा॒स्त्वाय॑ ॥ सं पत्नी॒ पत्या॑ सुकृ॒तेन॑ गच्छतां । य॒ज्ञ्स्य॑ यु॒क्तौ धुर्या॑वभूतां । स॒जां॒ना॒नौ विज॑हता॒-मरा॑तीः । दि॒वि ज्योति॑र॒ज-र॒मार॑भेतां ॥ दश॑ ते त॒नुवो॑ यज्ञ् य॒ज्ञियाः᳚ । ताः प्री॑णातु॒ यज॑मानो घृ॒तेन॑ । ना॒रि॒ष्ठयोः᳚ प्र॒शिष॒मीड॑मानः । दे॒वानां॒ दैव्येऽपि॒ यज॑मानो॒-ऽमृतो॑-ऽभूत् ॥ यं ॅवां᳚ दे॒वा अ॑कल्पयन्न् \textbf{ 52} \newline
                  \newline
                                \textbf{ TB 3.7.5.12} \newline
                  ऊ॒र्जो भा॒गꣳ श॑तक्रतू । ए॒तद्वां॒ तेन॑ प्रीणानि । तेन॑ तृप्यतमꣳहहौ ॥ अ॒हं दे॒वानाꣳ॑ सु॒कृता॑मस्मि लो॒के । ममे॒दमि॒ष्टं न मिथु॑ र्भवाति । अ॒हं ना॑रि॒ष्ठा-वनु॑यजामि वि॒द्वान् । यदा᳚भ्या॒-मिन्द्रो॒ अद॑धा-द्भाग॒धेयं᳚ ॥ अदा॑रसृद्भवत देव सोम । अ॒स्मिन्. य॒ज्ञे म॑रुतो मृडता नः । मा नो॑ विदद॒भि भा॒मो अश॑स्तिः \textbf{ 53} \newline
                  \newline
                                \textbf{ TB 3.7.5.13} \newline
                  मा नो॑ विदद्-व॒जना॒ द्वेष्या॒ या ॥ ऋ॒ष॒भं ॅवा॒जिनं॑ ॅव॒यं । पू॒र्णमा॑सं ॅयजामहे । स नो॑ दोहताꣳ सु॒वीर्यं᳚ । रा॒यस्पोषꣳ॑ सह॒स्रिणं᳚ । प्रा॒णाय॑ सु॒राध॑से । पू॒र्णमा॑साय॒ स्वाहा᳚ ॥ अ॒मा॒वा॒स्या॑ सु॒भगा॑ सु॒शेवा᳚ । धे॒नुरि॑व॒ भूय॑ आ॒प्याय॑माना । सा नो॑ दोहताꣳ सु॒वीर्यं᳚ ( ) । रा॒यस्पोषꣳ॑ सह॒स्रिणं᳚ । अ॒पा॒नाय॑ सु॒राध॑से । अ॒मा॒वा॒स्या॑यै॒ स्वाहा᳚ ॥ अ॒भिस्तृ॑णीहि॒ परि॑धेहि॒ वेदिं᳚ । जा॒मिं मा हिꣳ॑सीरमु॒या शया॑ना । हो॒तृ॒षद॑ना॒ हरि॑ताः सु॒वर्णाः᳚ । नि॒ष्का इ॒मे यज॑मानस्य ब्र॒द्ध्ने । \textbf{ 54} \newline
                  \newline
                                    (अ॒भीत्व॑र्यै - करोमि - क्रमीत् - पि॒ता - ऽऽत्मन॑ - एक॒तोमु॑खां - मे॒ दिशो - ऽध्य॑क्षेभ्यो - ह॒विर् - गा॑र्.हपत्या - कल्पय॒ - न्नश॑स्तिः॒ - सा नो॑ दोहताꣳ सु॒वीर्यꣳ॑ स॒प्त च॑) \textbf{(A5)} \newline \newline
                \textbf{ 3.7.6     अनुवाकं   6 -ऐष्टिकयाजमानमन्त्राः} \newline
                                \textbf{ TB 3.7.6.1} \newline
                  परि॑स्तृणीत॒ परि॑धत्ता॒ग्निं । परि॑हितो॒-ऽग्नि र्यज॑मानं भुनक्तु । अ॒पाꣳ रस॒ ओष॑धीनाꣳ सु॒वर्णः॑ । नि॒ष्का इ॒मे यज॑मानस्य सन्तु काम॒दुघाः᳚ । अ॒मुत्रा॒मुष्मि॑न् ॅलो॒के ॥ भूप॑ते॒ भुव॑नपते । म॒ह॒तो भू॒तस्य॑ पते । ब्र॒ह्माणं॑ त्वा वृणीमहे ॥ अ॒हं भूप॑तिर॒हं भुव॑नपतिः । अ॒हं म॑ह॒तो भू॒तस्य॒ पतिः॑ \textbf{ 55} \newline
                  \newline
                                \textbf{ TB 3.7.6.2} \newline
                  दे॒वेन॑ सवि॒त्रा प्रसू॑त॒ आर्त्वि॑ज्यं करिष्यामि । देव॑ सवितरे॒तं त्वा॑ वृणते । बृह॒स्पतिं॒ दैव्यं॑ ब्र॒ह्माणं᳚ । तद॒हं मन॑से॒ प्रब्र॑वीमि । मनो॑ गायत्रि॒यै । गा॒य॒त्री त्रि॒ष्टुभे᳚ । त्रि॒ष्टुब् जग॑त्यै । जग॑त्यनु॒ष्टुभे᳚ । अ॒नु॒ष्टुक् प॒ङ्क्त्यै । प॒ङ्क्तिः प्र॒जाप॑तये \textbf{ 56} \newline
                  \newline
                                \textbf{ TB 3.7.6.3} \newline
                  प्र॒जाप॑ति॒ र्विश्वे᳚भ्यो दे॒वेभ्यः॑ । विश्वे॑ दे॒वा बृह॒स्पत॑ये । बृह॒स्पति॒र् ब्रह्म॑णे । ब्रह्म॒ भूर्भुवः॒ सुवः॑ । बृह॒स्पति॑ र्दे॒वानां᳚ ब्र॒ह्मा । अ॒हं म॑नु॒ष्या॑णां । बृह॑स्पते य॒ज्ञ्ं गो॑पाय ॥ इ॒दं तस्मै॑ ह॒र्म्यं क॑रोमि । यो वो॑ देवा॒श्चर॑ति ब्रह्म॒चर्यं᳚ । मे॒धा॒वी दि॒क्षु मन॑सा तप॒स्वी \textbf{ 57} \newline
                  \newline
                                \textbf{ TB 3.7.6.4} \newline
                  अ॒न्तर्दू॒तश्च॑रति॒ मानु॑षीषु ॥ चतुः॑ शिखण्डा युव॒तिः सु॒पेशाः᳚ । घृ॒तप्र॑तीका॒ भुव॑नस्य॒ मद्ध्ये᳚ । म॒र्मृ॒ज्यमा॑ना मह॒ते सौभ॑गाय । मह्यं॑ धुक्ष्व॒ यज॑मानाय॒ कामान्॑ ॥ भूमि॑ र्भू॒त्वा म॑हि॒मानं॑ पुपोष । ततो॑ दे॒वी व॑र्द्धयते॒ पयाꣳ॑सि । य॒ज्ञिया॑ य॒ज्ञ्ं ॅविच॒यन्ति॒ शंच॑ । ओष॑धी॒राप॑ इ॒ह शक्व॑रीश्च ॥ यो मा॑ हृ॒दा मन॑सा॒ यश्च॑ वा॒चा \textbf{ 58} \newline
                  \newline
                                \textbf{ TB 3.7.6.5} \newline
                  यो ब्रह्म॑णा॒ कर्म॑णा॒ द्वेष्टि॑ देवाः । यः श्रु॒तेन॒ हृद॑येनेष्ण॒ता च॑ । तस्ये᳚न्द्र॒ वज्रे॑ण॒ शिर॑श्छिनद्मि ॥ ऊर्णा॑मृदु॒ प्रथ॑मानꣳ स्यो॒नं । दे॒वेभ्यो॒ जुष्टꣳ॒॒ सद॑नाय ब॒र्॒.हिः । सु॒व॒र्गे लो॒के यज॑मानꣳ॒॒ हि धे॒हि । मां नाक॑स्य पृ॒ष्ठे प॑र॒मे व्यो॑मन्न् ॥ चतुः॑ शिखण्डा युव॒तिः सु॒पेशाः᳚ । घृ॒तप्र॑तीका व॒युना॑नि वस्ते । सा स्ती॒र्यमा॑णा मह॒ते सौभ॑गाय \textbf{ 59} \newline
                  \newline
                                \textbf{ TB 3.7.6.6} \newline
                  सा मे॑ धुक्ष्व॒ यज॑मानाय॒ कामान्॑ । शि॒वा च॑ मे श॒ग्मा चै॑धि । स्यो॒ना च॑ मे सु॒षदा॑ चैधि । ऊर्ज॑स्वती च मे॒ पय॑स्वती चैधि । इष॒मूर्जं॑ मे पिन्वस्व । ब्रह्म॒ तेजो॑ मे पिन्वस्व । क्ष॒त्रमोजो॑ मे पिन्वस्व । विशं॒ पुष्टिं॑ मे पिन्वस्व । आयु॑र॒न्नाद्यं॑ मे पिन्वस्व । प्र॒जां प॒शून् मे॑ पिन्वस्व । \textbf{ 60} \newline
                  \newline
                                \textbf{ TB 3.7.6.7} \newline
                  अ॒स्मिन्. य॒ज्ञ् उप॒ भूय॒ इन्नु मे᳚ । अवि॑क्षोभाय परि॒धीन्द॑धामि । ध॒र्ता ध॒रुणो॒ धरी॑यान् । अ॒ग्नि र्द्वेषाꣳ॑सि॒ निरि॒तो नु॑दातै ॥ विच्छि॑नद्मि॒ विधृ॑तीभ्याꣳ स॒पत्नान्॑ । जा॒तान् भ्रातृ॑व्या॒न॒. ये च॑ जनि॒ष्यमा॑णाः । वि॒शो य॒न्त्राभ्यां॒ ॅविध॑माम्येनान् । अ॒हꣳस्वाना॑-मुत्त॒मो॑ऽसानि देवाः । वि॒शो य॒न्त्रे नु॒दमा॑ने॒ अरा॑तिं । विश्वं॑ पा॒प्मान॒मम॑तिं दुर्मरा॒युं । \textbf{ 61} \newline
                  \newline
                                \textbf{ TB 3.7.6.8} \newline
                  सीद॑न्ती दे॒वी सु॑कृ॒तस्य॑ लो॒के । धृती᳚ स्थो॒ विधृ॑ती॒ स्वधृ॑ती । प्रा॒णान्मयि॑ धारयतं । प्र॒जां मयि॑ धारयतं । प॒शून्मयि॑ धारयतं ॥ अ॒यं प्र॑स्त॒र उ॒भय॑स्य ध॒र्ता । ध॒र्ता प्र॑या॒जाना॑-मु॒तानू॑या॒जानां᳚ । स दा॑धार स॒मिधो॑ वि॒श्वरू॑पाः । तस्मि॒न्थ् स्रुचो॒ अद्ध्यासा॑दयामि ॥ आरो॑ह प॒थो जु॑हु देव॒यानान्॑ \textbf{ 62} \newline
                  \newline
                                \textbf{ TB 3.7.6.9} \newline
                  यत्रर्.ष॑यः प्रथम॒जा ये पु॑रा॒णाः । हिर॑ण्यपक्षाऽजि॒रा संभृ॑ताङ्गा । वहा॑सि मा सु॒कृतां॒ ॅयत्र॑ लो॒काः ॥ अवा॒हं बा॑ध उप॒भृता॑ स॒पत्नान्॑ । जा॒तान् भ्रातृ॑व्या॒न्॒. ये च॑ जनि॒ष्यमा॑णाः । दोहै॑ य॒ज्ञ्ꣳ सु॒दुघा॑मिव धे॒नुं । अ॒हमुत्त॑रो भूयासं । अध॑रे॒ मथ्स॒पत्नाः᳚ ॥ यो मा॑ वा॒चा मन॑सा दुर्मरा॒युः । हृ॒दा-ऽरा॑ती॒या-द॑भि॒दास॑-दग्ने \textbf{ 63} \newline
                  \newline
                                \textbf{ TB 3.7.6.10} \newline
                  इ॒दम॑स्य चि॒त्तमध॑रं ध्रु॒वायाः᳚ । अ॒हमुत्त॑रो भूयासं । अध॑रे॒ मथ्स॒पत्नाः᳚ ॥ ऋ॒ष॒भो॑ऽसि शाक्व॒रः । घृ॒ताची॑नाꣳ सू॒नुः । प्रि॒येण॒ नाम्ना᳚ प्रि॒ये सद॑सि सीद ॥ स्यो॒नो मे॑ सीद सु॒षदः॑ पृथि॒व्यां । प्रथ॑यि प्र॒जया॑ प॒शुभिः॑ सुव॒र्गे लो॒के । दि॒वि सी॑द पृथि॒व्या-म॒न्तरि॑क्षे । अ॒हमुत्त॑रो भूयासं \textbf{ 64} \newline
                  \newline
                                \textbf{ TB 3.7.6.11} \newline
                  अध॑रे॒ मथ्स॒पत्नाः᳚ ॥ इ॒यꣳ स्था॒ली घृ॒तस्य॑ पू॒र्णा । अच्छि॑न्नपयाः श॒तधा॑र॒ उथ्सः॑ । मा॒रु॒तेन॒ शर्म॑णा॒ दैव्ये॑न ॥ य॒ज्ञो॑ऽसि स॒र्वतः॑ श्रि॒तः । स॒र्वतो॒ मां भू॒तं भ॑वि॒ष्य-च्छ्र॑यतां । श॒तं मे॑ सन्त्वा॒शिषः॑ । स॒हस्रं॑ मे सन्तु सू॒नृताः᳚ । इरा॑वतीः पशु॒मतीः᳚ । प्र॒जाप॑तिरसि स॒र्वतः॑ श्रि॒तः \textbf{ 65} \newline
                  \newline
                                \textbf{ TB 3.7.6.12} \newline
                  स॒र्वतो॒ मां भू॒तं भ॑वि॒ष्य-च्छ्र॑यतां । श॒तं मे॑ सन्त्वा॒शिषः॑ । स॒हस्रं॑ मे सन्तु सू॒नृताः᳚ । इरा॑वतीः पशु॒मतीः᳚ ॥ इ॒दमि॑न्द्रि॒य-म॒मृतं॑ ॅवी॒र्यं᳚ । अ॒नेनेन्द्रा॑य प॒शवो॑-ऽचिकिथ्सन्न् । तेन॑ देवा अव॒तोप॒ मां । इ॒हेष॒मूर्जं॒ ॅयशः॒ सह॒ ओजः॑ सनेयं । शृ॒तं मयि॑ श्रयतां ॥ यत् पृ॑थि॒वी-मच॑र॒त्तत्-प्रवि॑ष्टं \textbf{ 66} \newline
                  \newline
                                \textbf{ TB 3.7.6.13} \newline
                  येनासि॑ञ्च॒द् बल॒मिन्द्रे᳚ प्र॒जाप॑तिः । इ॒दं तच्छु॒क्रं मधु॑ वा॒जिनी॑वत् । येनो॒-परि॑ष्टा॒दधि॑नोन्-महे॒न्द्रं । दधि॒ मां धि॑नोतु ॥ अ॒यं ॅवे॒दः पृ॑थि॒वी-मन्व॑विन्दत् ।गुहा॑ स॒तीं गह॑ने॒ गह्व॑रेषु । स वि॑न्दतु॒ यज॑मानाय लो॒कं । अच्छि॑द्रं ॅय॒ज्ञ्ं भूरि॑कर्मा करोतु ॥ अ॒यं ॅय॒ज्ञ्ः सम॑सदद्ध॒विष्मान्॑ । ऋ॒चा साम्ना॒ यजु॑षा दे॒वता॑भिः \textbf{ 67} \newline
                  \newline
                                \textbf{ TB 3.7.6.14} \newline
                  तेन॑ लो॒कान्थ् सूर्य॑वतो जयेम । इन्द्र॑स्य स॒ख्य-म॑मृत॒त्व-म॑श्यां ॥ यो नः॒ कनी॑य इ॒ह का॒मया॑तै । अ॒स्मिन्. य॒ज्ञे यज॑मानाय॒ मह्यं᳚ । अप॒ तमि॑न्द्रा॒ग्नी भुव॑नान्नुदेतां । अ॒हं प्र॒जां ॅवी॒रव॑तीं ॅविदेय ॥ अग्ने॑ वाजजित् । वाजं॑ त्वा सरि॒ष्यन्तं᳚ । वाजं॑ जे॒ष्यन्तं᳚ । वा॒जिनं॑ ॅवाज॒जितं᳚ \textbf{ 68} \newline
                  \newline
                                \textbf{ TB 3.7.6.15} \newline
                  वा॒ज॒जि॒त्यायै॒ संमा᳚र्ज्मि । अ॒ग्निम॑न्ना॒-दम॒न्नाद्या॑य ॥ उप॑हूतो॒ द्यौः पि॒ता । उप॒ मां द्यौः पि॒ता ह्व॑यतां । अ॒ग्नि-राग्नी᳚द्ध्रात् । आयु॑षे॒ वर्च॑से । जी॒वात्वै पुण्या॑य । उप॑हूता पृथि॒वी मा॒ता । उप॒ मां मा॒ता पृ॑थि॒वी ह्व॑यतां । अ॒ग्नि-राग्नी᳚द्ध्रात् \textbf{ 69} \newline
                  \newline
                                \textbf{ TB 3.7.6.16} \newline
                  आयु॑षे॒ वर्च॑से । जी॒वात्वै पुण्या॑य ॥ मनो॒ ज्योति॑ र्जुषता॒माज्यं᳚ । विच्छि॑न्नं ॅय॒ज्ञ्ꣳ समि॒मं द॑धातु । बृह॒स्पति॑-स्तनुतामि॒मं नः॑ । विश्वे॑ दे॒वा इ॒ह मा॑दयन्तां ॥ यं ते॑ अग्न आवृ॒श्चामि॑ । अ॒हं ॅवा᳚ क्षिपि॒तश्चरन्न्॑ । प्र॒जां च॒ तस्य॒ मूलं॑ च ।नी॒चैर्दे॑वा॒ निवृ॑श्चत \textbf{ 70} \newline
                  \newline
                                \textbf{ TB 3.7.6.17} \newline
                  अग्ने॒ यो नो॑ऽभि॒दास॑ति । स॒मा॒नो यश्च॒ निष्ट्यः॑ । इ॒द्ध्मस्ये॑व प्र॒क्षाय॑तः । मा तस्योच्छे॑षि॒ किंच॒न । यो मां द्वेष्टि॑ जातवेदः । यं चा॒हं द्वेष्मि॒ यश्च॒ मां । सर्वाꣳ॒॒स्तान॑ग्ने॒ संद॑ह । याꣳश्चा॒हं द्वेष्मि॒ ये च॒ मां ॥ अग्ने॑ वाजजित् । वाजं॑ त्वा ससृ॒वाꣳसं᳚ \textbf{ 71} \newline
                  \newline
                                \textbf{ TB 3.7.6.18} \newline
                  वाजं॑ जिगि॒वाꣳसं᳚ । वा॒जिनं॑ ॅवाज॒जितं᳚ । वा॒ज॒जि॒त्यायै॒ संमा᳚र्ज्मि । अ॒ग्नि-म॑न्ना॒द-म॒न्नाद्या॑य ॥ वेदि॑ र्ब॒र॒.हिः शृ॒तꣳ ह॒विः ।इ॒द्ध्मः प॑रि॒धयः॒ स्रुचः॑ । आज्यं॑ ॅय॒ज्ञ् ऋचो॒ यजुः॑ । या॒ज्या᳚श्च वषट्का॒राः । सं मे॒ संन॑तयो नमन्तां । इ॒द्ध्म॒सं॒-नह॑ने हु॒ते । \textbf{ 72} \newline
                  \newline
                                \textbf{ TB 3.7.6.19} \newline
                  दि॒वः खीलोऽव॑ततः । पृ॒थि॒व्या अद्ध्युत्थि॑तः । तेना॑ स॒हस्र॑काण्डेन । द्वि॒षन्तꣳ॑ शोचयामसि । द्वि॒षन्मे॑ ब॒हु शो॑चतु । ओष॑धे॒ मो अ॒हꣳ शु॑चं ॥ यज्ञ्॒ नम॑स्ते यज्ञ् । नमो॒ नम॑श्च ते यज्ञ् । शि॒वेन॑ मे॒ संति॑ष्ठस्व । स्यो॒नेन॑ मे॒ संति॑ष्ठस्व \textbf{ 73} \newline
                  \newline
                                \textbf{ TB 3.7.6.20} \newline
                  सु॒भू॒तेन॑ मे॒ संति॑ष्ठस्व । ब्र॒ह्म॒व॒र्च॒सेन॑ मे॒ संति॑ष्ठस्व । य॒ज्ञ्स्यर्द्धि॒मनु॒ संति॑ष्ठस्व । उप॑ ते यज्ञ्॒ नमः॑ । उप॑ ते॒ नमः॑ । उप॑ ते॒ नमः॑ ॥ त्रिष्फ॒ली क्रि॒यमा॑णानां । यो न्य॒ङ्गो अ॑व॒शिष्य॑ते । रक्ष॑सां भाग॒धेयं᳚ । आप॒स्तत् प्रव॑हतादि॒तः । \textbf{ 74} \newline
                  \newline
                                \textbf{ TB 3.7.6.21} \newline
                  उ॒लूख॑ले॒ मुस॑ले॒ यच्च॒ शूर्पे᳚ । आ॒शि॒श्लेष॑ दृ॒षदि॒ यत्क॒पाले᳚ । अ॒व॒प्रुषो॑ वि॒प्रुषः॒ सम्ॅय॑जामि । विश्वे॑ दे॒वा ह॒विरि॒दं जु॑षन्तां । य॒ज्ञे या वि॒प्रुषः॒ सन्ति॑ ब॒ह्वीः । अ॒ग्नौ ताः सर्वाः॒ स्वि॑ष्टाः॒ सुहु॑ता जुहोमि ॥ उ॒द्यन्न॒द्य मि॑त्रमहः । स॒पत्ना᳚न्मे अनीनशः । दिवै॑नान्. वि॒द्युता॑ जहि । नि॒म्रोच॒न्नध॑रान् कृधि । \textbf{ 75} \newline
                  \newline
                                \textbf{ TB 3.7.6.22} \newline
                  उ॒द्यन्न॒द्य वि नो॑ भज । पि॒ता पु॒त्रेभ्यो॒ यथा᳚ । दी॒र्घा॒यु॒त्वस्य॑ हेशिषे । तस्य॑ नो देहि सूर्य ॥ उ॒द्यन्न॒द्य मि॑त्रमहः । आ॒रोह॒न्नुत्त॑रां॒ दिवं᳚ । हृ॒द्रो॒गं मम॑ सूर्य । ह॒रि॒माणं॑ च नाशय ॥ शुके॑षु मे हरि॒माणं᳚ । रो॒प॒णाका॑सु दद्ध्मसि \textbf{ 76} \newline
                  \newline
                                \textbf{ TB 3.7.6.23} \newline
                  अथो॑ हारिद्र॒वेषु॑ मे । ह॒रि॒माणं॒ निद॑द्ध्मसि ॥ उद॑गाद॒य-मा॑दि॒त्यः । विश्वे॑न॒ सह॑सा स॒ह । द्वि॒षन्तं॒ मम॑ र॒न्धयन्न्॑ । मो अ॒हं द्वि॑ष॒तो र॑धं ॥ यो नः॒ शपा॒दश॑पतः । यश्च॑ नः॒ शप॑तः॒ शपा᳚त् । उ॒षाश्च॒ तस्मै॑ नि॒म्रुक्च॑ । सर्वं॑ पा॒पꣳ समू॑हतां ( ) । \textbf{ 77} \newline
                  \newline
                                \textbf{ TB 3.7.6.24} \newline
                  यो नः॑ स॒पत्नो॒ यो रणः॑ । मर्तो॑ऽभि॒दास॑ति देवाः। इ॒द्ध्मस्ये॑व प्र॒क्षाय॑तः । मा तस्योच्छे॑षि॒ किञ्च॒न ॥ अव॑सृष्टः॒ परा॑पत । श॒रो ब्रह्म॑सꣳशितः । गच्छा॒ऽमित्रा॒न् प्रवि॑श । मैषां॒ कंच॒नोच्छि॑षः । \textbf{ 78} \newline
                  \newline
                                    (पतिः॑ - प्र॒जाप॑तये - तप॒स्वी - वा॒चा - सौभ॑गाय - प॒शून्मे॑ पिन्वस्व - दुर्मरा॒युं - दे॑व॒याना॑ - नग्ने॒ - ऽन्तरि॑क्षे॒ऽहमुत्त॑रो भूयासं -प्र॒जाप॑तिरसि स॒र्वतः॑ श्रि॒तः - प्रवि॑ष्टं - दे॒वता॑भिर् - वाज॒जितं॑ - 
पृथि॒वी ह्व॑यताम॒ग्निराग्नी᳚द्ध्राद् - वृश्चत - ससृ॒वाꣳसꣳ॑ - हु॒ते - स्यो॒नेन॑ मे॒ सम्ति॑ष्ठस्वे॒ - तः - कृ॑धि- दध्म - स्यूहता - +म॒ष्टौ च॑) \textbf{(A6)} \newline \newline
                \textbf{ 3.7.7     अनुवाकं   7 -सोमाङ्ग.भूतेषु मन्त्रेषु-दीक्षामारभ्याग्नीषोमयि पशुपर्यन्ते प्रयोगे शेषभूता मन्त्राः} \newline
                                \textbf{ TB 3.7.7.1} \newline
                  सक्षे॒दं प॑श्य । विध॑र्तरि॒दं प॑श्य । नाके॒दं प॑श्य । र॒मतिः॒ पनि॑ष्ठा । ऋ॒तं ॅवर्.षि॑ष्ठं । अ॒मृता॒ यान्या॒हुः । सूर्यो॒ वरि॑ष्ठो अ॒क्षभि॒र्विभा॑ति । अन॒ द्यावा॑पृथि॒वी दे॒वपु॑त्रे ॥ दी॒क्षाऽसि॒ तप॑सो॒ योनिः॑ । तपो॑ऽसि॒ ब्रह्म॑णो॒ योनिः॑ \textbf{ 79} \newline
                  \newline
                                \textbf{ TB 3.7.7.2} \newline
                  ब्रह्मा॑सि क्ष॒त्रस्य॒ योनिः॑ । क्ष॒त्रम॑स्यृ॒तस्य॒ योनिः॑ । ऋ॒तम॑सि॒ भूरार॑भे । श्र॒द्धां मन॑सा । दी॒क्षां तप॑सा । विश्व॑स्य॒ भुव॑न॒स्याधि॑पत्नीं । सर्वे॒ कामा॒ यज॑मानस्य सन्तु ॥ वातं॑ प्रा॒णं मन॑सा॒ऽन्वार॑भामहे । प्र॒जाप॑तिं॒ ॅयो भुव॑नस्य गो॒पाः । स नो॑ मृ॒त्योस्त्रा॑यतां॒ पात्वꣳह॑सः \textbf{ 80} \newline
                  \newline
                                \textbf{ TB 3.7.7.3} \newline
                  ज्योग्जी॒वा ज॒राम॑शीमहि ॥ इन्द्र॑ शाक्वर गाय॒त्रीं प्रप॑द्ये । तां ते॑ युनज्मि । इन्द्र॑ शाक्वर त्रि॒ष्टुभं॒ प्रप॑द्ये । तां ते॑ युनज्मि । इन्द्र॑ शाक्वर॒ जग॑तीं॒ प्रप॑द्ये । तां ते॑ युनज्मि । इन्द्र॑ शाक्वरानु॒ष्टुभं॒ प्रप॑द्ये । तां ते॑ युनज्मि । इन्द्र॑ शाक्वर प॒ङ्क्तिं प्रप॑द्ये \textbf{ 81} \newline
                  \newline
                                \textbf{ TB 3.7.7.4} \newline
                  तां ते॑ युनज्मि ॥ आऽहं दी॒क्षाम॑रुह-मृ॒तस्य॒ पत्नीं᳚ । गा॒य॒त्रेण॒ छन्द॑सा॒ ब्रह्म॑णा च । ऋ॒तꣳ स॒त्ये॑ऽधायि । स॒त्यमृ॒ते॑ऽधायि । ऋ॒तं च॑ मे स॒त्यं चा॑भूतां । ज्योति॑रभूवꣳ॒॒ सुव॑रगमं । सु॒व॒र्गं ॅलो॒कं नाक॑स्य पृ॒ष्ठं । ब्र॒द्ध्नस्य॑ वि॒ष्टप॑मगमं ॥ पृ॒थि॒वी दी॒क्षा \textbf{ 82} \newline
                  \newline
                                \textbf{ TB 3.7.7.5} \newline
                  तया॒ऽग्नि र्दी॒क्षया॑ दीक्षि॒तः । यया॒ऽग्नि र्दी॒क्षया॑ दीक्षि॒तः । तया᳚ त्वा दी॒क्षया॑ दीक्षयामि । अ॒न्तरि॑क्षं दी॒क्षा । तया॑ वा॒यु र्दी॒क्षया॑ दीक्षि॒तः । यया॑ वा॒यु र्दी॒क्षया॑ दीक्षि॒तः । तया᳚ त्वा दी॒क्षया॑ दीक्षयामि । द्यौ र्दी॒क्षा । तया॑ऽऽदि॒त्यो दी॒क्षया॑ दीक्षि॒तः । यया॑ऽऽदि॒त्यो दी॒क्षया॑ दीक्षि॒तः \textbf{ 83} \newline
                  \newline
                                \textbf{ TB 3.7.7.6} \newline
                  तया᳚ त्वा दी॒क्षया॑ दीक्षयामि । दिशो॑ दी॒क्षा । तया॑ च॒न्द्रमा॑ दी॒क्षया॑ दीक्षि॒तः । यया॑ च॒न्द्रमा॑ दी॒क्षया॑ दीक्षि॒तः । तया᳚ त्वा दी॒क्षया॑ दीक्षयामि । आपो॑ दी॒क्षा । तया॒ वरु॑णो॒ राजा॑ दी॒क्षया॑ दीक्षि॒तः । यया॒ वरु॑णो॒ राजा॑ दी॒क्षया॑ दीक्षि॒तः । तया᳚ त्वा दी॒क्षया॑ दीक्षयामि । ओष॑धयो दी॒क्षा \textbf{ 84} \newline
                  \newline
                                \textbf{ TB 3.7.7.7} \newline
                  तया॒ सोमो॒ राजा॑ दी॒क्षया॑ दीक्षि॒तः। यया॒ सोमो॒ राजा॑ दी॒क्षया॑ दीक्षि॒तः । तया᳚ त्वा दी॒क्षया॑ दीक्षयामि । वाग् दी॒क्षा । तया᳚ प्रा॒णो दी॒क्षया॑ दीक्षि॒तः । यया᳚ प्रा॒णो दी॒क्षया॑ दीक्षि॒तः । तया᳚ त्वा दी॒क्षया॑ दीक्षयामि । पृ॒थि॒वी त्वा॒ दीक्ष॑माण॒मनु॑ दीक्षतां । अ॒न्तरि॑क्षं त्वा॒ दीक्ष॑माण॒मनु॑ दीक्षतां । द्यौस्त्वा॒ दीक्ष॑माण॒मनु॑ दीक्षतां \textbf{ 85} \newline
                  \newline
                                \textbf{ TB 3.7.7.8} \newline
                  दिश॑स्त्वा॒ दीक्ष॑माण॒मनु॑ दीक्षन्तां । आप॑स्त्वा॒ दीक्ष॑माण॒मनु॑ दीक्षन्तां । ओष॑धयस्त्वा॒ दीक्ष॑माण॒मनु॑ दीक्षन्तां । वाक्त्वा॒ दीक्ष॑माण॒मनु॑ दीक्षतां । ऋच॑स्त्वा॒ दीक्ष॑माण॒मनु॑ दीक्षन्तां । सामा॑नि त्वा॒ दीक्ष॑माण॒मनु॑ दीक्षन्तां । यजूꣳ॑षि त्वा॒ दीक्ष॑माण॒मनु॑ दीक्षन्तां । अह॑श्च॒ रात्रि॑श्च । कृ॒षिश्च॒ वृष्टि॑श्च । त्विषि॒श्चा-प॑चितिश्च \textbf{ 86} \newline
                  \newline
                                \textbf{ TB 3.7.7.9} \newline
                  आप॒श्चौष॑धयश्च । ऊर्क्च॑ सू॒नृता॑ च । तास्त्वा॒ दीक्ष॑माण॒मनु॑ दीक्षन्तां ॥ स्वे दक्षे॒ दक्ष॑पिते॒ह सी॑द । दे॒वानाꣳ॑ सु॒म्नो म॑ह॒ते रणा॑य । स्वा॒स॒स्थ-स्त॒नुवा॒ सम्ॅवि॑शस्व । पि॒तेवै॑धि सू॒नव॒ आ सु॒शेवः॑ । शि॒वो मा॑ शि॒वमावि॑श । स॒त्यं म॑ आ॒त्मा । श्र॒द्धा मे ऽक्षि॑तिः \textbf{ 87} \newline
                  \newline
                                \textbf{ TB 3.7.7.10} \newline
                  तपो॑ मे प्रति॒ष्ठा । स॒वि॒तृ-प्र॑सूता मा॒ दिशो॑ दीक्षयन्तु । स॒त्यम॑स्मि ॥ अ॒हं त्वद॑स्मि॒ मद॑सि॒ त्वमे॒तत् । ममा॑सि॒ योनि॒स्तव॒ योनि॑रस्मि । ममै॒व सन्वह॑ ह॒व्यान्य॑ग्ने । पु॒त्रः पि॒त्रे लो॑क॒कृ-ज्जा॑तवेदः ॥ आ॒जुह्वा॑नः सु॒प्रती॑कः पु॒रस्ता᳚त् । अग्ने॒ स्वां ॅयोनि॒मासी॑द सा॒द्ध्या । अ॒स्मिन्थ् स॒धस्थे॒ अद्ध्युत्त॑रस्मिन्न् \textbf{ 88} \newline
                  \newline
                                \textbf{ TB 3.7.7.11} \newline
                  विश्वे॑ देवा॒ यज॑मानश्च सीदत ॥ एक॑मि॒षे विष्णु॒स्त्वा-ऽन्वे॑तु । द्वे ऊ॒र्जे विष्णु॒स्त्वाऽन्वे॑तु । त्रीणि॑ व्र॒ताय॒ विष्णु॒स्त्वा-ऽन्वे॑तु । च॒त्वारि॒ मायो॑ भवाय॒ विष्णु॒स्त्वा-ऽन्वे॑तु । पञ्च॑ प॒शुभ्यो॒ विष्णु॒स्त्वा-ऽन्वे॑तु । षड्रा॒यस्पोषा॑य॒ विष्णु॒स्त्वा-ऽन्वे॑तु । स॒प्त स॒प्तभ्यो॒ होत्रा᳚भ्यो॒ विष्णु॒स्त्वा-ऽन्वे॑तु ॥ सखा॑यः स॒प्तप॑दा अभूम । स॒ख्यं ते॑ गमेयं \textbf{ 89} \newline
                  \newline
                                \textbf{ TB 3.7.7.12} \newline
                  स॒ख्यात्ते॒ मा यो॑षं । स॒ख्यान्मे॒ मा यो᳚ष्ठाः ॥ साऽसि॑ सुब्रह्मण्ये । तस्या᳚स्ते पृथि॒वी पादः॑ । साऽसि॑ सुब्रह्मण्ये । तस्या᳚स्ते॒ऽन्तरि॑क्षं॒ पादः॑ । साऽसि॑ सुब्रह्मण्ये । तस्या᳚स्ते॒ द्यौः पादः॑ । साऽसि॑ सुब्रह्मण्ये । तस्या᳚स्ते॒ दिशः॒ पादः॑ \textbf{ 90} \newline
                  \newline
                                \textbf{ TB 3.7.7.13} \newline
                  प॒रो र॑जास्ते पञ्च॒मः पादः॑ । सा न॒ इष॒मूर्जं॑ धुक्ष्व । तेज॑ इन्द्रि॒यं । ब्र॒ह्म॒व॒र्च॒स-म॒न्नाद्यं᳚ ॥ विमि॑मे त्वा॒ पय॑स्वतीं । दे॒वानां᳚ धे॒नुꣳ सु॒दुघा॒-मन॑प-स्फुरन्तीं । इन्द्रः॒ सोमं॑ पिबतु । क्षेमो॑ अस्तु नः ॥ इ॒मां न॑राः कृणुत॒ वेदि॒मेत्य॑ । वसु॑मतीꣳ रु॒द्रव॑ती-मादि॒त्यव॑तीं \textbf{ 91} \newline
                  \newline
                                \textbf{ TB 3.7.7.14} \newline
                  वर्ष्म॑न्दि॒वः । नाभा॑ पृथि॒व्याः । यथा॒ऽयं ॅयज॑मानो॒ न रिष्ये᳚त् । दे॒वस्य॑ सवि॒तुः स॒वे ॥ चतुः॑ शिखण्डा युव॒तिः सु॒पेशाः᳚ । घृ॒तप्र॑तीका॒ भुव॑नस्य॒ मद्ध्ये᳚ । तस्याꣳ॑ सुप॒र्णावधि॒ यौ निवि॑ष्टौ । तयो᳚ र्दे॒वाना॒मधि॑ भाग॒धेयं᳚ ॥ अप॒ जन्यं॑ भ॒यं नु॑द । अप च॒क्राणि॑ वर्तय ( ) । गृ॒हꣳ सोम॑स्य गच्छतं ॥ न वा उ॑वे॒तन्म्रि॑यसे॒ न रि॑ष्यसि । दे॒वाꣳ इदे॑षि प॒थिभिः॑ सु॒गेभिः॑ । यत्र॒ यन्ति॑ सु॒कृतो॒ नापि॑ दु॒ष्कृतः॑ । तत्र॑ त्वा दे॒वः स॑वि॒ता द॑धातु । \textbf{ 92} \newline
                  \newline
                                    (ब्रह्म॑णो॒ योनि॒ - रꣳह॑सः - प॒ङ्क्तिं प्रप॑द्ये - दी॒क्षा - ययो॑ऽऽदि॒त्यो दी॒क्षया॑ दीक्षि॒त - स्तया᳚ त्वा दी॒क्षया॑ दीक्षया॒म्योष॑धयो दी॒क्षा - द्यौस्त्वा॒ दीक्ष॑माण॒मनु॑ दीक्षता॒ - मप॑चिति॒श्चा - क्षि॑ति॒ - रुत्त॑रस्मिन् - गमेयं॒ - दिशः॒ पाद॑ - आदि॒त्यव॑तीं - ॅवर्तय॒ पञ्च॑ च) \textbf{(A7)} \newline \newline
                \textbf{ 3.7.8     अनुवाकं   8 -पशुविषया अच्छिद्रमन्त्राः} \newline
                                \textbf{ TB 3.7.8.1} \newline
                  यद॒स्य पा॒रे रज॑सः । शु॒क्रं ज्योति॒-रजा॑यत । तन्नः॑ पर्.ष॒दति॒ द्विषः॑ । अग्ने॑ वैश्वानर॒ स्वाहा᳚ ॥ यस्मा᳚द्भी॒षा-ऽवा॑शिष्ठाः । ततो॑ नो॒ अभ॑यं कृधि । प्र॒जाभ्यः॒ सर्वा᳚भ्यो मृड । नमो॑ रु॒द्राय॑ मी॒ढुषे᳚ ॥ यस्मा᳚द्भी॒षा न्यष॑दः । ततो॑ नो॒ अभ॑यं कृधि \textbf{ 93} \newline
                  \newline
                                \textbf{ TB 3.7.8.2} \newline
                  प्र॒जाभ्यः॒ सर्वा᳚भ्यो मृड । नमो॑ रु॒द्राय॑ मी॒ढुषे᳚ ॥ उदु॑स्र तिष्ठ॒ प्रति॑तिष्ठ॒ मा रि॑षः । मेमं ॅय॒ज्ञ्ं ॅयज॑मानं च रीरिषः । सु॒व॒र्गे लो॒के यज॑मानꣳ॒॒ हि धे॒हि । शं न॑ एधि द्वि॒पदे॒ शं चतु॑ष्पदे ॥ यस्मा᳚द्भी॒षा-ऽवे॑पिष्ठाः प॒लायि॑ष्ठाः स॒मज्ञा᳚स्थाः । ततो॑ नो॒ अभ॑यं कृधि । प्र॒जाभ्यः॒ सर्वा᳚भ्यो मृड । नमो॑ रु॒द्राय॑ मी॒ढुषे᳚ । \textbf{ 94} \newline
                  \newline
                                \textbf{ TB 3.7.8.3} \newline
                  य इ॒दमकः॑ । तस्मै॒ नमः॑ । तस्मै॒ स्वाहा᳚ ॥ “ न वा उ॑ वे॒तन्म्रि॑यसे{1}“ । “आशा॑नां त्वा॒{2}“ “विश्वा॒ आशाः᳚{3}" ॥ य॒ज्ञ्स्य॒ हि स्थ ऋ॒त्वियौ᳚ । इन्द्रा᳚ग्नी॒ चेत॑नस्य च । हु॒ता॒हु॒तस्य॑ तृप्यतं । अहु॑तस्य हु॒तस्य॑ च । हु॒तस्य॒ चाहु॑तस्य च ( ) । अहु॑तस्य हु॒तस्य॑ च । इन्द्रा᳚ग्नी अ॒स्य सोम॑स्य । वी॒तं पि॑बतं जु॒षेथां᳚ ॥ मा यज॑मानं॒ तमो॑ विदत् । मर्त्विजो॒ मो इ॒माः प्र॒जाः । मा यः सोम॑मि॒मं पिबा᳚त् । सꣳसृ॑ष्टमु॒भयं॑ कृ॒तं । \textbf{ 95} \newline
                  \newline
                                    (कृ॒धि॒ - मी॒ढुषे - ऽहु॑तस्य च स॒प्त च॑) \textbf{(A8)} \newline \newline
                \textbf{ 3.7.9     अनुवाकं   9 -उपांश्वभिषवा मन्त्राः} \newline
                                \textbf{ TB 3.7.9.1} \newline
                  अ॒ना॒गस॑स्त्वा व॒यं । इन्द्रे॑ण॒ प्रेषि॑ता॒ उप॑ । वा॒युष्टे॑ अस्त्वꣳश॒भूः । मि॒त्रस्ते॑ अस्त्वꣳश॒भूः । वरु॑णस्ते अस्त्वꣳश॒भूः ॥ अपां᳚ क्षया॒ ऋत॑स्य गर्भाः । भुव॑नस्य गोपाः॒ श्येना॑ अतिथयः । पर्व॑तानां ककुभः प्र॒युतो॑ नपातारः । व॒ग्नुनेन्द्रꣳ॑ ह्वयत । घोषे॒णामी॑-वाꣳश्चातयत \textbf{ 96} \newline
                  \newline
                                \textbf{ TB 3.7.9.2} \newline
                  यु॒क्ताः स्थ॒ वह॑त ॥ दे॒वा ग्रावा॑ण॒ इन्दु॒रिन्द्र॒ इत्य॑वादिषुः । एन्द्र॑मचुच्यवुः पर॒मस्याः᳚ परा॒वत॑: । आऽस्माथ् स॒धस्था᳚त् । ओरोर॒न्तरि॑क्षात् । आ सु॑भू॒तम॑सुषवुः । ब्र॒ह्म॒व॒र्च॒सं म॒ आऽसु॑षवुः । स॒म॒रे रक्षाꣳ॑स्यवधिषुः । अप॑हतं ब्रह्म॒ज्यस्य॑ ॥ वाक्च॑ त्वा॒ मन॑श्च श्रिणीतां \textbf{ 97} \newline
                  \newline
                                \textbf{ TB 3.7.9.3} \newline
                  प्रा॒णश्च॑ त्वाऽपा॒नश्च॑ श्रीणीतां । चक्षु॑श्च त्वा॒ श्रोत्रं॑ च श्रीणीतां । दक्ष॑श्च त्वा॒ बलं॑ च श्रीणीतां । ओज॑श्च त्वा॒ सह॑श्च श्रीणीतां । आयु॑श्च त्वा ज॒रा च॑ श्रीणीतां । आ॒त्मा च॑ त्वा त॒नूश्च॑ श्रीणीतां । शृ॒तो॑ऽसि शृ॒तं कृ॑तः । शृ॒ताय॑ त्वा शृ॒तेभ्य॑स्त्वा ॥ यमिन्द्र॑मा॒हु र्वरु॑णं॒ ॅयमा॒हुः । यं मि॒त्रमा॒हुर् यमु॑ स॒त्यमा॒हुः \textbf{ 98} \newline
                  \newline
                                \textbf{ TB 3.7.9.4} \newline
                  यो दे॒वानां᳚ दे॒वत॑मस्तपो॒जाः । तस्मै᳚ त्वा॒ तेभ्य॑स्त्वा ॥ मयि॒ त्यदि॑न्द्रि॒यं म॒हत् । मयि॒ दक्षो॒ मयि॒ क्रतुः॑ । मयि॑ धायि सु॒वीर्यं᳚ । त्रिशु॑ग्घ॒र्मो विभा॑तु मे । आकू᳚त्या॒ मन॑सा स॒ह । वि॒राजा॒ ज्योति॑षा स॒ह । य॒ज्ञेन॒ पय॑सा स॒ह । तस्य॒ दोह॑मशीमहि \textbf{ 99} \newline
                  \newline
                                \textbf{ TB 3.7.9.5} \newline
                  तस्य॑ सु॒म्नम॑शीमहि । तस्य॑ भ॒क्षम॑शीमहि । वाग्जु॑षा॒णा सोम॑स्य तृप्यतु ॥ “मि॒त्रो जना॒न्{4}“ “प्र स मि॑त्र{5}“ ॥ यस्मा॒न्न जा॒तः परो॑ अ॒न्यो अस्ति॑ । य आ॑वि॒वेश॒ भुव॑नानि॒ विश्वा᳚ । प्र॒जाप॑तिः प्र॒जया॑ सम्ॅविदा॒नः । त्रीणि॒ ज्योतीꣳ॑षि सचते॒ स षो॑ड॒शी ॥ ए॒ष ब्र॒ह्मा य ऋ॒त्वियः॑ । इन्द्रो॒ नाम॑ श्रु॒तो ग॒णे । \textbf{ 100} \newline
                  \newline
                                \textbf{ TB 3.7.9.6} \newline
                  प्र ते॑ म॒हे वि॒दथे॑ ऽशꣳसिषꣳ॒॒ हरी᳚ । य ऋ॒त्वियः॒ प्र ते॑ वन्वे । व॒नुषो॑ हर्य॒तं मदं᳚ ॥ इन्द्रो॒ नाम॑ घृ॒तं न यः । हरि॑भि॒श्चारु॒ सेच॑ते । श्रु॒तो ग॒ण आ त्वा॑ विशन्तु । हरि॑वर्पस॒गिंरः॑ ॥ इन्द्राधि॑प॒ते-ऽधि॑पति॒स्त्वं दे॒वाना॑मसि । अधि॑पतिं॒ मां । आयु॑ष्मन्तं॒ ॅवर्च॑स्वन्तं मनु॒ष्ये॑षु कुरु । \textbf{ 101} \newline
                  \newline
                                \textbf{ TB 3.7.9.7} \newline
                  इन्द्र॑श्च स॒म्राड् वरु॑णश्च॒ राजा᳚ । तौ ते॑ भ॒क्षं च॑क्रतु॒रग्र॑ ए॒तं । तयो॒रनु॑ भ॒क्षं भ॑क्षयामि । वाग्जु॑षा॒णा सोम॑स्य तृप्यतु ॥ प्र॒जाप॑ति र्वि॒श्वक॑र्मा । तस्य॒ मनो॑ दे॒वं ॅय॒ज्ञेन॑ राद्ध्यासं । अ॒र्थे॒ गा अ॒स्य ज॑हितः । अ॒व॒सान॑पते-ऽव॒सानं॑ मे विन्द ॥ नमो॑ रु॒द्राय॑ वास्तो॒ष्पत॑ये । आय॑ने वि॒द्रव॑णे \textbf{ 102} \newline
                  \newline
                                \textbf{ TB 3.7.9.8} \newline
                  उ॒द्याने॒ यत्प॒राय॑णे । आ॒वर्त॑ने वि॒वर्त॑ने । यो गो॑पा॒यति॒ तꣳ हु॑वे ॥ यान्य॑पा॒मित्या॒-न्यप्र॑तीत्ता॒-न्यस्मि॑ । य॒मस्य॑ ब॒लिना॒ चरा॑मि । इ॒हैव सन्तः॒ प्रति॒ तद्या॑तयामः । जी॒वा जी॒वेभ्यो॒ निह॑राम एनत् ॥ अ॒नृ॒णा अ॒स्मिन्न॑-नृ॒णाः पर॑स्मिन्न् । तृ॒तीये॑ लो॒के अ॑नृ॒णाः स्या॑म । ये दे॑व॒याना॑ उ॒त पि॑तृ॒याणाः᳚ \textbf{ 103} \newline
                  \newline
                                \textbf{ TB 3.7.9.9} \newline
                  सर्वा᳚न् प॒थो अ॑नृ॒णा आक्षी॑येम ॥ इ॒दमू॒ नु श्रेयो॑-ऽव॒सान॒-माग॑न्म । शि॒वे नो॒ द्यावा॑पृथि॒वी उ॒भे इ॒मे । गोम॒द्धन॑-व॒दश्व॑व॒-दूर्ज॑स्वत् । सु॒वीरा॑ वी॒रैरनु॒ संच॑रेम ॥ अ॒र्कः प॒वित्रꣳ॒॒ रज॑सो वि॒मानः॑ । पु॒नाति॑ दे॒वानां॒ भुव॑नानि॒ विश्वा᳚ । द्यावा॑पृथि॒वी पय॑सा सम्ॅविदा॒ने । घृ॒तं दु॑हाते अ॒मृतं॒ प्रपी॑ने ॥ प॒वित्र॑म॒र्को रज॑सो वि॒मानः॑ ( ) । पु॒नाति॑ दे॒वानां॒ भुव॑नानि॒ विश्वा᳚ । सुव॒ र्ज्योति॒ र्यशो॑ म॒हत् । अ॒शी॒महि॑ गा॒धमु॒त प्र॑ति॒ष्ठां । \textbf{ 104} \newline
                  \newline
                                    (चा॒त॒य॒त॒ - श्री॒णी॒ताꣳ॒॒ - स॒त्यमा॒हु - र॑शीमहि - ग॒णे - कु॑रु - वि॒द्रव॑णे - पितृ॒याणा॑ - अ॒र्को रज॑सो वि॒मान॒स्त्रीणि॑ च) \textbf{(A9)} \newline \newline
                \textbf{ 3.7.10    अनुवाकं   10 -सौमिकप्रायाश्चित्तमन्त्राः} \newline
                                \textbf{ TB 3.7.10.1} \newline
                  उद॑स्तांफ्सीथ् सवि॒ता मि॒त्रो अ॑र्य॒मा । सर्वा॑-न॒मित्रा॑-नवधीद्यु॒गेन॑ । बृ॒हन्तं॒ माम॑कर-द्वी॒रव॑न्तं । र॒थ॒न्त॒रे श्र॑यस्व॒ स्वाहा॑ पृथि॒व्यां । वा॒म॒दे॒व्ये श्र॑यस्व॒ स्वाहा॒-ऽन्तरि॑क्षे । बृ॒ह॒ति श्र॑यस्व॒ स्वाहा॑ दि॒वि । बृ॒ह॒ता त्वोप॑स्तभ्नोमि ॥ आ त्वा॑ ददे॒ यश॑से वी॒र्या॑य च । अ॒स्मास्व॑-घ्निया यू॒यं द॑धाथेन्द्रि॒यं पयः॑ ॥ यस्ते᳚ द्र॒फ्सो यस्त॑ उद॒र्॒.षः \textbf{ 105} \newline
                  \newline
                                \textbf{ TB 3.7.10.2} \newline
                  दैव्यः॑ के॒तु र्विश्वं॒ भुव॑न-मावि॒वेश॑ । स नः॑ पा॒ह्यरि॑ष्ट्यै॒ स्वाहा᳚ ॥ अनु॑ मा॒ सर्वो॑ य॒ज्ञो॑-ऽयमे॑तु । विश्वे॑ दे॒वा म॒रुतः॒ सामा॒र्कः । आ॒प्रिय॒ ॑007आ;छन्दाꣳ॑सि नि॒विदो॒ यजूꣳ॑षि । अ॒स्यै पृ॑थि॒व्यै यद्य॒ज्ञियं᳚ ॥ प्र॒जाप॑ते र्वर्त॒निमनु॑ वर्तस्व । अनु॑ वी॒रैरनु॑ राद्ध्याम॒ गोभिः॑ । अन्वश्वै॒रनु॒ सर्वै॑रु पु॒ष्टैः । अनु॑ प्र॒जया-ऽन्वि॑न्द्रि॒येण॑ \textbf{ 106} \newline
                  \newline
                                \textbf{ TB 3.7.10.3} \newline
                  दे॒वा नो॑ य॒ज्ञ्-मृ॑जु॒धा न॑यन्तु ॥ प्रति॑ क्ष॒त्रे प्रति॑ तिष्ठामि रा॒ष्ट्रे । प्रत्यश्वे॑षु॒ प्रति॑तिष्ठामि॒ गोषु॑ । प्रति॑ प्र॒जायां॒ प्रति॑ तिष्ठामि॒ भव्ये᳚ । विश्व॑म॒न्याऽभि॑ वावृ॒धे । तद॒न्यस्या॒-मधि॑श्रि॒तं । दि॒वे च॑ वि॒श्वक॑र्मणे । पृ॒थि॒व्यै चा॑करं॒ नमः॑ ॥ अस्का॒न्द्यौः पृ॑थि॒वीं । अस्का॑नृष॒भो युवा॒ गाः \textbf{ 107} \newline
                  \newline
                                \textbf{ TB 3.7.10.4} \newline
                  स्क॒न्नेमा विश्वा॒ भुव॑ना । स्क॒न्नो य॒ज्ञ्ः प्रज॑नयतु । अस्का॒नज॑नि॒ प्राज॑नि । आ स्क॒न्नाज्जा॑यते॒ वृषा᳚ । स्क॒न्नात् प्रज॑निषीमहि ॥ ये दे॒वा येषा॑मि॒दं भा॑ग॒धेयं॑ ब॒भूव॑ । येषां᳚ प्रया॒जा उ॒तानू॑या॒जाः । इन्द्र॑ज्येष्ठेभ्यो॒ वरु॑णराजभ्यः । अ॒ग्निहो॑तृभ्यो दे॒वेभ्यः॒ स्वाहा᳚ ॥ उ॒त त्या नो॒ दिवा॑ म॒तिः \textbf{ 108} \newline
                  \newline
                                \textbf{ TB 3.7.10.5} \newline
                  अदि॑तिरू॒त्या-ऽऽग॑मत् । सा शन्ता॑ची॒ मय॑स्करत् । अप॒ स्रिधः॑ ॥ उ॒त त्या दैव्या॑ भि॒षजा᳚ । शं न॑स्करतो अ॒श्विना᳚ । यू॒याता॑-म॒स्मद्रपः॑ । अप॒ स्रिधः॑ ॥ शम॒ग्नि-र॒ग्निभि॑स्करत् । शं न॑स्तपतु॒ सूर्यः॑ । शं ॅवातो॑ वात्वर॒पाः \textbf{ 109} \newline
                  \newline
                                \textbf{ TB 3.7.10.6} \newline
                  अप॒ स्रिधः॑ ॥ तदित्प॒दं न विचि॑केत वि॒द्वान् । यन्मृ॒तः पुन॑र॒प्येति॑ जी॒वान् । त्रि॒वृद्यद् भुव॑नस्य रथ॒वृत् । जी॒वो गर्भो॒ न मृ॒तः स जी॑वात् ॥ प्रत्य॑स्मै॒ पिपी॑षते । वि॒श्वा॑नि वि॒दुषे॑ भर । अ॒रं॒ ग॒माय॒ जग्म॑वे । अप॑श्चाद्दध्वने॒ नरे᳚ (ओर् अप॑श्चाद्दघ्वने॒ नरे᳚) ॥ इन्दु॒रिन्दु॒म-वा॑गात् ( ) । इन्दो॒रिन्द्रो॑-ऽपात् । तस्य॑ त इन्द॒विन्द्र॑-पीतस्य॒ मधु॑मतः । उप॑हूत॒स्योप॑हूतो भक्षयामि । \textbf{ 110} \newline
                  \newline
                                    (उ॒द॒र्॒.ष - इ॑न्द्रि॒येण॒ - गा - म॒ति - र॑र॒पा - अ॑गा॒त्रीणि॑ च) \textbf{(A10)} \newline \newline
                \textbf{ 3.7.11    अनुवाकं   11 -दर्शपूर्णमासप्रायश्चित्तमन्त्राः} \newline
                                \textbf{ TB 3.7.11.1} \newline
                  ब्रह्म॑ प्रति॒ष्ठा मन॑सो॒ ब्रह्म॑ वा॒चः । ब्रह्म॑ य॒ज्ञानाꣳ॑ ह॒विषा॒-माज्य॑स्य । अति॑रिक्तं॒ कर्म॑णो॒ यच्च॑ ही॒नं । य॒ज्ञ्ः पर्वा॑णि प्रति॒रन्ने॑ति क॒ल्पयन्न्॑ । स्वाहा॑कृ॒ता-ऽऽहु॑तिरेतु दे॒वान् ॥ आश्रा॑वितम॒त्याश्रा॑वितं । वष॑ट्कृत-म॒त्यनू᳚क्तं च य॒ज्ञे । अति॑रिक्तं॒ कर्म॑णो॒ यच्च॑ ही॒नं । य॒ज्ञ्ः पर्वा॑णि प्रति॒रन्ने॑ति क॒ल्पयन्न्॑ । स्वाहा॑कृ॒ता-ऽऽहु॑तिरेतु दे॒वान् । \textbf{ 111} \newline
                  \newline
                                \textbf{ TB 3.7.11.2} \newline
                  यद्वो॑ देवा अतिपा॒दया॑नि । वा॒चा चि॒त्प्रय॑तं देव॒ हेड॑नं । अ॒रा॒यो अ॒स्माꣳ अ॒भि-दु॑च्छुना॒यते᳚। अ॒न्यत्रा॒स्मन्-म॑रुत॒स्त-न्निधे॑तन ॥ त॒तं म॒ आप॒स्तदु॑ तायत॒ पुनः॑ । स्वादि॑ष्ठा धी॒तिरु॒चथा॑य शस्यते । अ॒यꣳ स॑मु॒द्र उ॒त वि॒श्वभे॑षजः । स्वाहा॑-कृतस्य॒ समु॑तृप्णुत र्भुवः ॥ “उद्व॒यं तम॑स॒स्परि॑{6}” । “उदु॒ त्यं{7}” “चि॒त्रं{8}” \textbf{ 112} \newline
                  \newline
                                \textbf{ TB 3.7.11.3} \newline
                  “इ॒मं मे॑ वरुण॒{9}” “तत्त्वा॑ यामि{10}” । “त्वं नो॑ अग्ने॒{11}” “स त्वं नो॑ अग्ने{12}” । “त्वम॑ग्ने अ॒याऽसि॒{13}” “प्रजा॑पते{14}” ॥ इ॒मं जी॒वेभ्यः॑ परि॒धिं द॑धामि । मैषां नु॑ गा॒दप॑रो॒ अर्द्ध॑मे॒तं । श॒तं जी॑वन्तु श॒रदः॑ पुरू॒चीः । ति॒रो मृ॒त्युं द॑धतां॒ पर्व॑तेन ॥ इ॒ष्टेभ्यः॒ स्वाहा॒ वष॒डनि॑ष्टेभ्यः॒ स्वाहा᳚ । भे॒ष॒जं दुरि॑ष्ट्यै॒ स्वाहा॒ निष्कृ॑त्यै॒ स्वाहा᳚ । दौरा᳚र्द्ध्यै॒ स्वाहा॒ दैवी᳚भ्य-स्त॒नूभ्यः॒ स्वाहा᳚ \textbf{ 113} \newline
                  \newline
                                \textbf{ TB 3.7.11.4} \newline
                  ऋद्ध्यै॒ स्वाहा॒ समृ॑द्ध्यै॒ स्वाहा᳚ ॥ यत॑ इन्द्र॒ भया॑महे । ततो॑ नो॒ अभ॑यं कृधि । मघ॑वञ्छ॒ग्धि तव॒ तन्न॑ ऊ॒तये᳚ । वि द्विषो॒ वि मृधो॑ जहि ॥ स्व॒स्ति॒दा वि॒शस्पतिः॑ । वृ॒त्र॒हा विमृधो॑ व॒शी । वृषेन्द्रः॑ पु॒र ए॑तु नः । स्व॒स्ति॒दा अ॑भयं क॒रः ॥ आ॒भि र्गी॒र्भि र्यदतो॑ न ऊ॒नं \textbf{ 114} \newline
                  \newline
                                \textbf{ TB 3.7.11.5} \newline
                  आप्या॑यय हरिवो॒ वर्द्ध॑मानः । य॒दा स्तो॒तृभ्यो॒ महि॑ गो॒त्रा रु॒जासि॑ । भू॒यि॒ष्ठ॒भाजो॒ अध॑ ते स्याम ॥ अना᳚ज्ञातं॒ ॅयदाज्ञा॑तं । य॒ज्ञ्स्य॑ क्रि॒यते॒ मिथु॑ । अग्ने॒ तद॑स्य कल्पय । त्वꣳ हि वेत्थ॑ यथात॒थं ॥ पुरु॑ष संमितो य॒ज्ञ्ः । य॒ज्ञ्ः पुरु॑ष संमितः । अग्ने॒ तद॑स्य कल्पय ( ) । त्वꣳ हि वेत्थ॑ यथात॒थं ॥ यत् पा॑क॒त्रा मन॑सा दी॒नद॑क्षा॒ न । य॒ज्ञ्स्य॑ म॒न्वते॒ मर्ता॑सः । अ॒ग्निष्टद्धोता᳚ कृतु॒विद्वि॑जा॒नन्न् । यजि॑ष्ठो दे॒वाꣳ ऋ॑तु॒शो य॑जाति । \textbf{ 115} \newline
                  \newline
                                    (दे॒वाꣳ - श्चि॒त्रं - त॒नूभ्यः॒ स्वाहो॒ - नं - पुरु॑षसंमि॒तोऽग्ने॒ तद॑स्य कल्पय॒ पञ्च॑ च) \textbf{(A11)} \newline \newline
                \textbf{ 3.7.12    अनुवाकं   12 -अग्निष्टोमादौ यजमानजप्या मन्त्रविशेषाः} \newline
                                \textbf{ TB 3.7.12.1} \newline
                  यद्दे॑वा देव॒हेड॑नं । देवा॑सश्चकृ॒मा व॒यं । आदि॑त्या॒-स्तस्मा᳚न्मा मुञ्चत । ऋ॒तस्य॒र्तेन॒ मामु॒त ॥ देवा॑ जीवनका॒म्या यत् । वा॒चाऽनृ॑तमूदि॒म । अ॒ग्निर्मा॒ तस्मा॒देन॑सः । गार्.ह॑पत्यः॒ प्रमु॑ञ्चतु । दु॒रि॒ता यानि॑ चकृ॒म । क॒रोतु॒ माम॑ने॒ नसं᳚ । \textbf{ 116} \newline
                  \newline
                                \textbf{ TB 3.7.12.2} \newline
                  ऋ॒तेन॑ द्यावापृथिवी । ऋ॒तेन॒ त्वꣳ स॑रस्वति । ऋ॒तान्मा॑ मुञ्च॒ताꣳह॑सः । यद॒न्य कृ॑तमारि॒म ॥ स॒जा॒त॒शꣳ॒॒सादु॒त वा॑ जामिशꣳ॒॒सात् । ज्याय॑सः॒ शꣳसा॑दु॒त वा॒ कनी॑यसः । अना᳚ज्ञातं दे॒वकृ॑तं॒ ॅयदेनः॑ । तस्मा॒त्त्वम॒स्मा-ञ्जा॑तवेदो मुमुग्धि ॥ यद्वा॒चा यन्मन॑सा । बा॒हुभ्या॑-मू॒रुभ्या॑-मष्ठी॒वद्भ्यां᳚ \textbf{ 117} \newline
                  \newline
                                \textbf{ TB 3.7.12.3} \newline
                  शि॒श्ञै र्यदनृ॑तं चकृ॒मा व॒यं । अ॒ग्निर्मा॒ तस्मा॒देन॑सः ॥ यद्धस्ता᳚भ्यां च॒कर॒ किल्बि॑षाणि । अ॒क्षाणां᳚ ॅव॒ग्नुमु॑प॒ जिघ्न॑मानः । दू॒रे॒प॒श्या च॑ राष्ट्र॒भृच्च॑ । तान्य॑फ्स॒रसा॒-वनु॑दत्ता-मृ॒णानि॑ ॥ अदी᳚व्यन्नृ॒णं ॅयद॒हं च॒कार॑ । यद्वाऽदा᳚स्यन्थ् संज॒गारा॒ जने᳚भ्यः । अ॒ग्निर्मा॒ तस्मा॒देन॑सः ॥ यन्मयि॑ मा॒ता गर्भे॑ स॒ति \textbf{ 118} \newline
                  \newline
                                \textbf{ TB 3.7.12.4} \newline
                  एन॑श्च॒कार॒ यत्पि॒ता । अ॒ग्निर्मा॒ तस्मा॒देन॑सः ॥ यदा॑ पि॒पेष॑ मा॒तरं॑ पि॒तरं᳚ । पु॒त्रः प्रमु॑दितो॒ धयन्न्॑ । अहिꣳ॑सितौ पि॒तरौ॒ मया॒ तत् । तद॑ग्ने अनृ॒णो भ॑वामि ॥ यद॒न्तरि॑क्षं पृथि॒वीमु॒त द्यां । यन्मा॒तरं॑ पि॒तरं॑ ॅवा जिहिꣳसि॒म । अ॒ग्निर्मा॒ तस्मा॒देन॑सः ॥ यदा॒शसा॑ नि॒शसा॒ यत्प॑रा॒शसा᳚ \textbf{ 119} \newline
                  \newline
                                \textbf{ TB 3.7.12.5} \newline
                  यदेन॑श्च कृ॒मा नूत॑नं॒ ॅयत्पु॑रा॒णं । अ॒ग्निर्मा॒ तस्मा॒देन॑सः ॥ अति॑क्रामामि दुरि॒तं ॅयदेनः॑ । जहा॑मि रि॒प्रं प॑र॒मे स॒धस्थे᳚ । यत्र॒ यन्ति॑ सु॒कृतो॒ नापि॑ दु॒ष्कृतः॑ । तमारो॑हामि सु॒कृतां॒ नु लो॒कं ॥ त्रि॒ते दे॒वा अ॑मृजतै॒-तदेनः॑ । त्रि॒त ए॒तन्म॑नु॒ष्ये॑षु मामृजे । ततो॑ मा॒ यदि॒ किंचि॑दान॒शे । अ॒ग्निर्मा॒ तस्मा॒देन॑सः \textbf{ 120} \newline
                  \newline
                                \textbf{ TB 3.7.12.6} \newline
                  गार्.ह॑पत्यः॒ प्रमु॑ञ्चतु । दु॒रि॒ता यानि॑ चकृ॒म । क॒रोतु॒ माम॑ने॒नसं᳚ ॥ दि॒वि जा॒ता अ॒फ्सु जा॒ताः । या जा॒ता ओष॑धीभ्यः । अथो॒ या अ॑ग्नि॒जा आपः॑ । ता नः॑ शुन्धन्तु॒ शुन्ध॑नीः ॥ यदापो॒ नक्तं॑ दुरि॒तं चरा॑म । यद्वा॒ दिवा॒ नूत॑नं॒ ॅयत्पु॑रा॒णं । हिर॑ण्य-वर्णा॒स्तत॒ उत्पु॑नीत नः ( ) ॥ “इ॒मं मे॑ वरुण॒{15}” “तत्त्वा॑ यामि{16}” । “त्वं नो॑ अग्ने॒{17}” “स त्वं नो॑ अग्ने{18}” । “त्वम॑ग्ने अ॒याऽसि॑{19}” । \textbf{ 121} \newline
                  \newline
                                                        \textbf{special korvai} \newline
              (यद् दे॑वा॒ गार्.ह॑पत्यो॒ यद्धस्ता᳚भ्यां॒ ॅयन्मयि॑ मा॒ता यदा॑पि॒पेष॒ ॅयद॒न्तरि॑क्षं॒ ॅयदा॒शसाऽति॑क्रामामि त्रि॒ते दे॒वा दि॒वि जा॒ता अ॒फ्सु जा॒ता यदाप॑ इ॒मं मे॑ वरुणा॒ तत्त्वा॑ यामि॒ त्वं नो॑ अग्ने॒ स त्वं नो॑ अग्ने॒ त्वम॑ग्ने अ॒याऽसि॑) \newline
                                (अ॒ने॒नस॑ - मष्ठी॒वद्भ्याꣳ॑ - स॒ति - प॑रा॒शसा॑ऽऽन॒शे᳚ - ऽग्निर्मा॒ तस्मा॒देन॑सः - पुनीत न॒स्त्रीणि॑ च) \textbf{(A12)} \newline \newline
                \textbf{ 3.7.13    अनुवाकं   13 -अवभृथे कर्मणि ऋजीषप्रोक्षणमन्त्राः} \newline
                                \textbf{ TB 3.7.13.1} \newline
                  यत्ते॒ ग्राव्.ण्णा॑ चिच्छि॒दुः सो॑म राजन्न् । प्रि॒याण्यङ्गा॑नि॒ स्वधि॑ता॒ परूꣳ॑षि । तथ्संध॒थ् स्वाज्ये॑नो॒त व॑र्द्धयस्व । अ॒ना॒गसो॒ अध॒मिथ्स॒क्षंये॑म ॥ यत्ते॒ ग्रावा॑ बा॒हुच्यु॑तो॒ अचु॑च्यवुः । नरो॒ यत्ते॑ दुदु॒हु र्दक्षि॑णेन । तत्त॒ आप्या॑यतां॒ तत्ते᳚ । निष्ट्या॑यतां देव सोम ॥ यत्ते॒ त्वचं॑ बिभि॒दुर्यच्च॒ योनिं᳚ । यदा॒स्थाना॒त् प्रच्यु॑तो॒ वेन॑सि॒ त्मना᳚ \textbf{ 122} \newline
                  \newline
                                \textbf{ TB 3.7.13.2} \newline
                  त्वया॒ तथ्सो॑म गु॒प्तम॑स्तु नः । सा नः॑ स॒धां-ऽस॑त्पर॒मे व्यो॑मन्न् ॥ अहा॒च्छरी॑रं॒ पय॑सा स॒मेत्य॑ । अ॒न्यो᳚ऽन्यो भवति॒ वर्णो॑ अस्य । तस्मि॑न् व॒यमुप॑हूता॒स्तव॑ स्मः । आ नो॑ भज॒ सद॑सि वि॒श्वरू॑पे ॥ नृ॒चक्षाः॒ सोम॑ उ॒त शु॒श्रुग॑स्तु । मा नो॒ विहा॑सी॒द्गिर॑ आवृणा॒नः । अना॑गा-स्त॒नुवो॑ वावृधा॒नः । आ नो॑ रू॒पं ॅव॑हतु॒ जाय॑मानः । \textbf{ 123} \newline
                  \newline
                                \textbf{ TB 3.7.13.3} \newline
                  उप॑क्षरन्ति जु॒ह्वो॑ घृ॒तेन॑ । प्रि॒याण्यङ्गा॑नि॒ तव॑ व॒र्द्धय॑न्तीः । तस्मै॑ ते सोम॒ नम॒ इद्वष॑ट्च । उप॑ मा राजन्थ् सुकृ॒ते ह्व॑यस्व ॥ सं प्रा॑णापा॒नाभ्याꣳ॒॒ समु॒ चक्षु॑षा॒ त्वं । सꣳ श्रोत्रे॑ण गच्छस्व सोम राजन्न् । यत्त॒ आस्थि॑तꣳ॒॒ शमु॒ तत्ते॑ अस्तु । जा॒नी॒तान्नः॑ स॒गंम॑ने पथी॒नां ॥ ए॒तं जा॑नीतात्पर॒मे व्यो॑मन्न् । वृकाः᳚ सधस्था वि॒द रू॒पम॑स्य \textbf{ 124} \newline
                  \newline
                                \textbf{ TB 3.7.13.4} \newline
                  यदा॒गच्छा᳚त् प॒थिभि॑ र्देव॒यानैः᳚ । इ॒ष्टा॒पू॒र्ते कृ॑णुता-दा॒विर॑स्मै ॥ अरि॑ष्टो राजन्नग॒दः परे॑हि । नम॑स्ते अस्तु॒ चक्ष॑से रघूय॒ते । नाक॒मारो॑ह स॒ह यज॑मानेन । सूर्यं॑ गच्छतात्पर॒मे व्यो॑मन्न् ॥ अभू᳚द्दे॒वः स॑वि॒ता वन्द्यो॒ नु नः॑ । इ॒दानी॒मह्न॑ उप॒वाच्यो॒ नृभिः॑ । वि यो रत्ना॒ भज॑ति मान॒वेभ्यः॑ । श्रेष्ठं॑ नो॒ अत्र॒ द्रवि॑णं॒ ॅयथा॒ दध॑त् ( ) ॥ उप॑ नो मित्रावरुणा-वि॒हाव॑तं । अ॒न्वादी᳚द्ध्या-थामि॒ह नः॑ सखाया । आ॒दि॒त्यानां॒ प्रसि॑तिर्. हे॒तिः । उ॒ग्रा श॒तापा᳚ष्ठा घ॒विषा॒ परि॑णो वृणक्तु ॥ “आप्या॑यस्व॒{20}” “सं ते᳚{21}” । \textbf{ 125} \newline
                  \newline
                                    (त्मना॒ - जाय॑मानो - ऽस्य॒ - दध॒त् पञ्च॑ च) \textbf{(A13)} \newline \newline
                \textbf{ 3.7.14    अनुवाकं   14 -अवभृथे कर्मणि सेचनादिमन्त्राः} \newline
                                \textbf{ TB 3.7.14.1} \newline
                  यद्दि॑दी॒क्षे मन॑सा॒ यच्च॑ वा॒चा । यद्वा᳚ प्रा॒णैश्चक्षु॑षा॒ यच्च॒ श्रोत्रे॑ण । यद्रेत॑सा मिथु॒नेना-प्या॒त्मना᳚ । अ॒द्भ्यो लो॒का द॑धि॒रे तेज॑ इन्द्रि॒यं । शु॒क्रा दी॒क्षायै॒ तप॑सो वि॒मोच॑नीः । आपो॑ विमो॒क्त्री र्मयि॒ तेज॑ इन्द्रि॒यं ॥ यदृ॒चा साम्ना॒ यजु॑षा । प॒शू॒नां चर्म॑न्. ह॒विषा॑ दिदी॒क्षे । यच्छन्दो॑भि॒-रोष॑धीभि॒-र्वन॒स्पतौ᳚ । अ॒द्भ्यो लो॒का द॑धि॒रे (द॑धिरे॒) तेज॑ इन्द्रि॒यं \textbf{ 126} \newline
                  \newline
                                \textbf{ TB 3.7.14.2} \newline
                  शु॒क्रा दी॒क्षायै॒ तप॑सो वि॒मोच॑नीः । आपो॑ विमो॒क्त्री र्मयि॒ तेज॑ इन्द्रि॒यं ॥ येन॒ ब्रह्म॒ येन॑ क्ष॒त्रं । येने᳚न्द्रा॒ग्नी प्र॒जाप॑तिः॒ सोमो॒ वरु॑णो॒ येन॒ राजा᳚ । विश्वे॑ दे॒वा ऋष॑यो॒ येन॑ प्रा॒णाः । अ॒द्भ्यो लो॒का द॑धिरे॒ तेज॑ इन्द्रि॒यं । शु॒क्रा दी॒क्षायै॒ तप॑सो वि॒मोच॑नीः । आपो॑ विमो॒क्त्री र्मयि॒ तेज॑ इन्द्रि॒यं ॥ अ॒पां पुष्प॑म॒स्यो-ष॑धीनाꣳ॒॒ रसः॑ । सोम॑स्य प्रि॒यं धाम॑ \textbf{ 127} \newline
                  \newline
                                \textbf{ TB 3.7.14.3} \newline
                  अ॒ग्नेः प्रि॒यत॑मꣳ ह॒विः स्वाहा᳚ ॥ अ॒पां पुष्प॑-म॒स्योष॑धीनाꣳ॒॒ रसः॑ । सोम॑स्य प्रि॒यं धाम॑ । इन्द्र॑स्य प्रि॒यत॑मꣳ ह॒विः स्वाहा᳚ । अ॒पां पुष्प॑म॒स्यो-ष॑धीनाꣳ॒॒ रसः॑ । सोम॑स्य प्रि॒यं धाम॑ । विश्वे॑षां दे॒वानां᳚ प्रि॒यत॑मꣳ ह॒विः स्वाहा᳚ ॥ व॒यꣳ सो॑म व्र॒ते तव॑ । मन॑स्त॒नूषु॒ पिप्र॑तः । प्र॒जाव॑न्तो अशीमहि । \textbf{ 128} \newline
                  \newline
                                \textbf{ TB 3.7.14.4} \newline
                  दे॒वेभ्यः॑ पि॒तृभ्यः॒ स्वाहा᳚ । सो॒म्येभ्यः॑ पि॒तृभ्यः॒ स्वाहा᳚ । क॒व्येभ्यः॑ पि॒तृभ्यः॒ स्वाहा᳚ ॥ देवा॑स इ॒ह मा॑दयद्ध्वं । सोम्या॑स इ॒ह मा॑दयद्ध्वं । कव्या॑स इ॒ह मा॑दयद्ध्वं ॥ अन॑न्तरिताः पि॒तरः॑ सो॒म्याः सो॑मपी॒थात् ॥ अपै॑तु मृ॒त्युर॒मृतं॑ न॒ आगन्न्॑ । वै॒व॒स्व॒तो नो॒ अभ॑यं कृणोतु । प॒र्णं ॅवन॒स्पते॑रिव \textbf{ 129} \newline
                  \newline
                                \textbf{ TB 3.7.14.5} \newline
                  अ॒भि नः॑ शीयताꣳ र॒यिः । सच॑तां नः॒ शची॒पतिः॑ ॥ परं॑ मृत्यो॒ अनु॒परे॑हि॒ पन्थां᳚ । यस्ते॒ स्व इत॑रो देव॒याना᳚त् । चक्षु॑ष्मते शृण्व॒ते ते᳚ ब्रवीमि । मा नः॑ प्र॒जाꣳ री॑रिषो॒ मोत वी॒रान् ॥ इ॒दमू॒ नु श्रेयो॑ऽव॒सान॒-माग॑न्म । यद्गो॒जिद्-ध॑न॒जि-द॑श्व॒जिद्यत् । प॒र्णं ॅवन॒स्पते॑रिव । अ॒भि नः॑ शीयताꣳ र॒यिः ( ) । सच॑तां नः॒ शची॒पतिः॑ । \textbf{ 130} \newline
                  \newline
                                    (वन॒स्पता॑व॒द्भ्यो लो॒का द॑धिरे॒ तेज॑ इन्द्रि॒यं धामा॑ - शीमही - वा॒ - भि नः॑ शीयताꣳ र॒यिरेकं॑ च) \textbf{(A14)} \newline \newline
                \textbf{PrapAtaka Korvai with starting  words of 1 to 14 anuvAkams :-} \newline
        (सर्वा॒न्॒. - यद् विष्ष॑ण्णेन॒ - वि वै - याः पु॒रस्ता॒द् - देवा॑ दे॒वेषु॒ - परि॑स्तणित॒ - सक्षे॒दं - ॅयद॒स्य पा॒रे॑ - ऽना॒गस॒ - उद॑स्तां फ्सी॒द् - ब्रह्म॑ प्रति॒ष्ठा - यद् दे॑व॒ - यत्ते॒ ग्राव.ण्णा॒ - यद् दि॑दी॒क्षे चतु॑र्दश) \newline

        \textbf{korvai with starting words of 1, 11, 21 series of daSinis :-} \newline
        (सर्वा॒न् - भूति॑मे॒व - यामे॒वाफ्स्वाहु॑तिम् - ॅव्र॒तानां᳚ - पर्णव॒ल्कः - सो॒म्याना॑ - म॒स्मिन्. य॒ज्ञे - ऽग्ने॒ यो नो॒ - ज्योग्जी॒वाः - प॒रोर॑जास्ते॒ - प्र ते॑ म॒हे - ब्रह्म॑ प्रति॒ष्ठा - गार्.ह॑पत्या स्त्रिꣳ॒॒शदु॑त्तरश॒तं) \newline

        \textbf{first and last  word 3rd aShtakam 6th prapATakam :-} \newline
        (सर्वा॒न्॒. वै - शची॒पतिः॑) \newline 

       

        ॥ हरिः॑ ॐ ॥॥ कृष्ण यजुर्वेदीय तैत्तिरीय ब्राह्मणे तृतीयाष्टके सप्तमः प्रपाठकः समाप्तः ॥

=====================================
Appendix (of Expansions)
ट्.भ्.3.7.8.3 "न वा उ॑ वे॒तन्म्रि॑यसे{1}" 
न वा उ॑वे॒तन्म्रि॑यसे॒ न रि॑ष्यसि दे॒वाꣳ इदे॑षि प॒थिभिः॑ सु॒गेभिः॑ । 
हरी॑ ते॒ युञ्जा॒ पृष॑ती अभूता॒मुपा᳚स्थाद्-वा॒जी धु॒रि रास॑भस्य ॥ {1}
(Appearing in T.S.4.6.9.4)

ट्.भ्.3.7.8.3 "आशा॑नां त्वा॒{2}", "विश्वा॒ आशाः᳚{3}"
आशा॑नां त्वा ऽऽशापा॒लेभ्यः॑ । च॒तुर्भ्यो॑ अ॒मृते᳚भ्यः । 
इ॒दं भू॒तस्याध्य॑क्षेभ्यः । वि॒धेम॑ ह॒विषा॑ व॒यम् ॥ {2}

विश्वा॒ आशा॒ मधु॑ना॒ सꣳसृ॑जामि । अ॒न॒मी॒वा आप॒ ओष॑धयो भवन्तु । 
अ॒यं ॅयज॑मानो॒ मृधो॒ व्य॑स्यताम् । अगृ॑भीताः प॒शवः॑ सन्तु॒ सर्वे᳚ ॥ {3}
(BOth {2} and {3} appEaring in T.B.2.5.3.3)

ट्.भ्.3.7.9.5 - "मि॒त्रो जना॒न्{4}" "प्र स मि॑त्र{5}" 
मि॒त्रो जनान्॑ यातयति प्रजा॒नन् मि॒त्रो दा॑धार पृथि॒वीमु॒त द्यां । 
मि॒त्रः कृ॒ष्टीरनि॑मिषा॒ऽभि च॑ष्टे स॒त्याय॑ ह॒व्यं घृ॒तव॑द्-विधेम ॥ {4}

प्र स मि॑त्र॒ मर्तो॑ अस्तु॒ प्रय॑स्वा॒न्. यस्त॑ आदित्य॒ शिक्ष॑ति व्र॒तेन॑ । 
न ह॑न्यते॒ न जी॑यते॒ त्वोतो॒ नैन॒मꣳहो॑ अश्नो॒त्यन्ति॑तो॒ न दू॒रात् ॥ {5}
(BOth {4} and {5} appEaring in T.S.3.4.11.5

ट्.भ्.3.7.11.2 - "उद्व॒यं तम॑स॒स्परि॑{6}" 
उद्व॒यं तम॑स॒स्परि॒ पश्य॑न्तो॒ ज्योति॒रुत्त॑रं । 
दे॒वं दे॑व॒त्रा सूर्य॒मग॑न्म॒ ज्योति॑रुत्त॒मं ॥ {6}
(Appearing in T.S.4.1.7.4)

ट्.भ्.3.7.11.2 - "उदु॒ त्यं{7}" "चि॒त्रं{8}" 
उदु॒ त्यं जा॒तवे॑दसं दे॒वं ॅव॑हन्ति के॒तवः॑ । दृ॒शे विश्वा॑य॒ सूर्यं᳚ ॥ {7}

चि॒त्रं दे॒वाना॒-मुद॑गा॒दनी॑कं॒ चक्षु॑र् मि॒त्रस्य॒ वरु॑णस्या॒ऽग्नेः । 
आऽ प्रा॒ द्यावा॑पृथि॒वी अ॒न्तरि॑क्षꣳ॒॒ सूर्य॑ आ॒त्मा जग॑तस्त॒स्थुष॑श्च ॥ {8}
( BOth {7} and {8] appEaring in T.S.1.4.43.1)

ट्.भ्.3.7.11.3 - "इ॒मं मे॑ वरुण॒{9}" "तत्त्वा॑ यामि{10}" 
इ॒मं मे॑ वरुण श्रुधी॒ हव॑म॒द्या च॑ मृडय । त्वाम॑व॒स्युरा च॑के ॥ {9}

तत्त्वा॑ यामि॒ ब्रह्म॑णा॒ वन्द॑मान॒-स्तदा शा᳚स्ते॒ यज॑मानो ह॒विर्भिः॑ । 
अहे॑डमानो वरुणे॒ह बो॒ध्युरु॑शꣳस॒ मा न॒ आयुः॒ प्रमो॑षीः ॥ {10} 
( BOth {9} and {10] appEaring in T.S.2.1.11.6)

ट्.भ्.3.7.11.3 - "त्वं नो॑ अग्ने॒{11}" "स त्वं नो॑ अग्ने{12}" 
त्वं नो॑ अग्ने॒ वरु॑णस्य वि॒द्वान् दे॒वस्य॒ हेडोऽव॑ यासि सीष्ठाः । 
यजि॑ष्ठो॒ वह्नि॑ तमः॒ शोशु॑चानो॒ विश्वा॒ द्वेषाꣳ॑सि॒ प्रमु॑मुग्ध्य॒स्मत् ॥ {11}

स त्वंनो॑ अग्नेऽव॒मो भ॑वो॒ती नेदि॑ष्ठो अ॒स्या उ॒षसो॒ व्यु॑ष्टौ । 
अव॑ यक्ष्व नो॒ वरु॑णꣳ॒॒ ररा॑णो वी॒हि मृ॑डी॒कꣳ सु॒हवो॑ न एधि ॥ {12}
( BOth {11} and {12] appEaring in T.S.2.5.12.3)

ट्.भ्.3.7.11.3 - "त्वम॑ग्ने अ॒याऽसि॒{13}" "प्रजा॑पते{14}" 
त्वम॑ग्ने अ॒याऽसि॑ । अ॒या सन्मन॑सा हि॒तः । अ॒या सन्.ह॒व्यमू॑हिषे । 
अ॒या नो॑ धेहि भेष॒जम् । इ॒ष्टो अ॒ग्निराहु॑तः । स्वाहा॑कृतः पिपर्तु नः । 
स्व॒गा दे॒वेभ्य॑ इ॒दं नमः॑ ॥ {13} 
( {13} appEaring in T.B.2.4.1.9 )

प्रजा॑पते॒ न त्वदे॒तान्य॒न्यो विश्वा॑ जा॒तानि॒ परि॒ ता ब॑भूव । 
यत्का॑मास्ते जुहु॒मस्तन्नो॑ अस्तु व॒यꣳ स्या॑म॒ पत॑यो रयी॒णाम् ॥ {14}
( {14} appEaring in T.S.1.8.14.2)

ट्.भ्. 3.7.12.6 - "इ॒मं मे॑ वरुण॒{15}" "तत्त्वा॑ यामि{16}" 
ईतॆम् नॊ.{15} इस् समॆ अस् {9} अबॊवॆ
ईतॆम् नॊ.{16} इस् समॆ अस् {10} अबॊवॆ

ट्.भ्.3.7.12.6 - "त्वं नो॑ अग्ने॒{17}" "स त्वं नो॑ अग्ने{18}" 
ईतॆम् नॊ.{17} इस् समॆ अस् {11} अबॊवॆ 
ईतॆम् नॊ.{18} इस् समॆ अस् {12} अबॊवॆ 

ट्.भ्.3.7.12.6- "त्वम॑ग्ने अ॒याऽसि॑{19}" 
ईतॆम् नॊ.{19} इस् समॆ अस् {13} अबॊवॆ

ट्.भ्.3.7.13.4 - "आप्या॑यस्व॒{20}" "सं ते᳚{21}"
आ प्या॑यस्व॒ समॆ॑तु ते वि॒श्वतः॑ सोम॒ वृष्णि॑यं । 
भवा॒ वाज॑स्य सङ्ग॒थे ॥ {20} 

सं ते॒ पयाꣳ॑सि॒ समु॑ यन्तु॒ वाजाः॒ सं ॅवृष्णि॑या-न्यभिमाति॒षाहः॑ ।
आ॒प्याय॑मानो अ॒मृता॑य सोम दि॒वि श्रवाꣳ॑स्युत्त॒मानि॑ धिष्व ॥ {21}
(BOth {20} and {21} appEaring in TS 4.2.7.4) \newline
        \pagebreak
        
        
        
     \addcontentsline{toc}{section}{ 3.8     तैत्तिरीय ब्राह्मणे तृतीयाष्टके अष्टम: प्रपाठक: (अश्वमेधब्राह्मणं वैश्वदेवं काण्डं)}
     \markright{ 3.8     तैत्तिरीय ब्राह्मणे तृतीयाष्टके अष्टम: प्रपाठक: (अश्वमेधब्राह्मणं वैश्वदेवं काण्डं) \hfill https://www.vedavms.in \hfill}
     \section*{ 3.8     तैत्तिरीय ब्राह्मणे तृतीयाष्टके अष्टम: प्रपाठक: (अश्वमेधब्राह्मणं वैश्वदेवं काण्डं) }
                \textbf{ 3.8.1     अनुवाकं   1 -सांग्रहण्येष्ट्यादयः याजमानसंस्काराः} \newline
                                \textbf{ TB 3.8.1.1} \newline
                  सा॒ग्रं॒ह॒ण्येष्ट्या॑ यजते । इ॒मां ज॒नताꣳ॒॒ संग॑ह्णा॒नीति॑ ॥ द्वाद॑शारत्नी रश॒ना भ॑वति । द्वाद॑श॒ मासाः᳚ सम्ॅवथ्स॒रः । स॒म्ॅव॒थ्स॒रमे॒वा व॑रुन्धे । मौ॒ञ्जी भ॑वति । ऊर्ग्वै मुञ्जाः᳚ । ऊर्ज॑मे॒ वा व॑रुन्धे ॥ चि॒त्रा नक्ष॑त्रं भवति । चि॒त्रं ॅवा ए॒तत्कर्म॑ । 1(10) ट्.भ्.3.8.1.2 यद॑श्वमे॒धः समृ॑द्ध्यै ॥ पुण्य॑नाम देव॒यज॑न-म॒द्ध्यव॑स्यति । पुण्या॑मे॒व तेन॑ की॒र्तिम॒भिज॑यति ॥ अप॑दाती-नृ॒त्विजः॑ स॒माव॑ह॒न्त्या सु॑ब्रह्म॒ण्यायाः᳚ । सु॒व॒र्गस्य॑ लो॒कस्य॒ सम॑ष्ट्यै ॥ के॒श॒श्म॒श्रु व॑पते । न॒खानि॒ निकृ॑न्तते । द॒तो धा॑वते । स्नाति॑ । अह॑तं॒ ॅवासः॒ परि॑धत्ते ( ) । पा॒प्मनोऽप॑हत्यै । वाचं॑ ॅय॒त्वोप॑वसति । सु॒व॒र्गस्य॑ लो॒कस्य॒ गुप्त्यै᳚ । रात्रिं॑ जाग॒रय॑न्त आसते । सु॒व॒र्गस्य॑ लो॒कस्य॒ सम॑ष्ट्यै । \textbf{ 2} \newline
                  \newline
                                    (कर्म॑ - धत्ते॒ पञ्च॑ च) \textbf{(A1)} \newline \newline
                \textbf{ 3.8.2     अनुवाकं   2 -ब्रह्मौदनाभिधानम्} \newline
                                \textbf{ TB 3.8.2.1} \newline
                  चतु॑ष्टय्य॒ आपो॑ भवन्ति । चतु॑श्शफो॒ वा अश्वः॑ प्राजाप॒त्यः समृ॑द्ध्यै ॥ ता दि॒ग्भ्यः स॒माभृ॑ता भवन्ति । दि॒क्षु वा आपः॑ । अन्नं॒ ॅवा आपः॑ । अ॒द्भ्यो वा अन्नं॑ जायते । यदे॒वाद्भ्योऽन्नं॒ जाय॑ते । तदव॑रुन्धे ॥ तासु॑ ब्रह्मौद॒नं प॑चति । रेत॑ ए॒व तद्द॑धाति । \textbf{ 3} \newline
                  \newline
                                \textbf{ TB 3.8.2.2} \newline
                  चतु॑श्शरावो भवति । दि॒क्ष्वे॑व प्रति॑तिष्ठति ॥ उ॒भ॒यतो॑ रु॒क्मौ भ॑वतः । उ॒भ॒यत॑ ए॒वास्मि॒न्-रुचं॑ दधाति । उद्ध॑रति शृत॒त्वाय॑ । स॒र्पिष्वा᳚न् भवति मेद्ध्य॒त्वाय॑ । च॒त्वार॑ आर्.षे॒याः प्राश्न॑न्ति । दि॒शामे॒व ज्योति॑षि जुहोति । च॒त्वारि॒ हिर॑ण्यानि ददाति । दि॒शाम॒व ज्योतीꣳ॒॒ष्य व॑रुन्धे । \textbf{ 4} \newline
                  \newline
                                \textbf{ TB 3.8.2.3} \newline
                  यदाज्य॑-मु॒च्छिष्य॑ते । तस्मि॑न्-रश॒नां न्यु॑नत्ति । प्र॒जाप॑ति॒र्वा ओ॑द॒नः । रेत॒ आज्य᳚म् । यदाज्ये॑ रश॒नां न्यु॒नत्ति॑ । प्र॒जाप॑तिमे॒व रेत॑सा॒ सम॑र्द्धयति ॥ द॒र्भ॒मयी॑ रश॒ना भ॑वति । ब॒हु वा ए॒ष कु॑च॒रो॑-ऽमे॒द्ध्यमुप॑गच्छति । यदश्वः॑ । प॒वित्रं॒ ॅवै द॒र्भाः \textbf{ 5} \newline
                  \newline
                                \textbf{ TB 3.8.2.4} \newline
                  यद्द॑र्भ॒मयी॑ रश॒ना भव॑ति । पु॒नात्ये॒वैन᳚म् । पू॒तमे॑नं॒ मेद्ध्य॒-माल॑भते ॥ अश्व॑स्य॒ वा आल॑ब्धस्य महि॒मोद॑-क्रामत् । स म॒हर्त्वि॑जः॒ प्रावि॑शत् । तन्म॒हर्त्वि॑जां महर्त्वि॒क्त्वम् । यन्म॒हर्त्वि॑जः प्रा॒श्नन्ति॑ । म॒हि॒मान॑-मे॒वास्मि॒न्-तद्द॑धति ॥ अश्व॑स्य॒ वा आल॑ब्धस्य॒ रेत॒ उद॑क्रामत् । तथ्सु॒वर्णꣳ॒॒ हिर॑ण्य-मभवत् ( ) । यथ्सु॒वर्णꣳ॒॒ हिर॑ण्यं॒ ददा॑ति । रेत॑ ए॒व तद्द॑धाति ॥ ओ॒द॒ने द॑दाति । रेतो॒ वा ओ॑द॒नः । रेतो॒ हिर॑ण्यम् । रेत॑सै॒वास्मि॒न् रेतो॑ दधाति । \textbf{ 6} \newline
                  \newline
                                    (द॒धा॒ति॒ - रु॒न्धे॒ - द॒र्भा - अ॑भव॒थ् षट्च॑) \textbf{(A2)} \newline \newline
                \textbf{ 3.8.3     अनुवाकं   3 -रशनयाऽश्वबन्धनम्} \newline
                                \textbf{ TB 3.8.3.1} \newline
                  यो वै ब्रह्म॑णे दे॒वेभ्यः॑ प्र॒जाप॑त॒ये-ऽप्र॑ति-प्रो॒च्याश्वं॒ मेद्ध्यं॑ ब॒द्ध्नाति॑ । आ दे॒वता᳚भ्यो वृश्च्यते । पापी॑यान्भवति । यः प्र॑ति॒प्रोच्य॑ । न दे॒वता᳚भ्य॒ आवृ॑श्च्यते । वसी॑यान्भवति ॥ यदाह॑ । ब्रह्म॒न्नश्वं॒ मेद्ध्यं॑ भन्थ्स्यामि दे॒वेभ्यः॑ प्र॒जाप॑तये॒ तेन॑ राद्ध्यास॒मिति॑ । ब्रह्म॒ वै ब्र॒ह्मा । ब्रह्म॑ण ए॒व दे॒वेभ्यः॑ प्र॒जाप॑तये प्रति॒प्रोच्याश्वं॒ मेद्ध्यं॑ बद्ध्नाति \textbf{ 7} \newline
                  \newline
                                \textbf{ TB 3.8.3.2} \newline
                  न दे॒वता᳚भ्य॒ आवृ॑श्च्यते । वसी॑यान् भवति ॥ दे॒वस्य॑ त्वा सवि॒तुः प्र॑स॒व इति॑ रश॒ना-माद॑त्ते॒ प्रसू᳚त्यै । अ॒श्विनो᳚-र्बा॒हुभ्या॒-मित्या॑ह । अ॒श्विनौ॒ हि दे॒वाना॑-मद्ध्व॒र्यू आस्ता᳚म् । पू॒ष्णो हस्ता᳚भ्या॒-मित्या॑ह॒ यत्यै᳚ ॥ व्यृ॑द्धं॒ ॅवा ए॒तद्य॒ज्ञ्स्य॑ । यद॑य॒जुष्के॑ण क्रि॒यते᳚ । इ॒माम॑गृभ्णन्-रश॒ना-मृ॒तस्ये-त्यधि॑ वदति॒ यजु॑ष्कृत्यै । य॒ज्ञ्स्य॒ समृ॑द्ध्यै । \textbf{ 8} \newline
                  \newline
                                \textbf{ TB 3.8.3.3} \newline
                  तदा॑हुः । द्वाद॑शारत्नी रश॒ना क॑र्त॒व्या(3) त्रयो॑दशार॒त्नी(3)रिति॑ । ऋ॒ष॒भो वा ए॒ष ऋ॑तू॒नाम् । यथ्स॑म्ॅवथ्स॒रः । तस्य॑ त्रयोद॒शो मासो॑ वि॒ष्टप᳚म् । ऋ॒ष॒भ ए॒ष य॒ज्ञाना᳚म् । यद॑श्वमे॒धः । यथा॒ वा ऋ॑ष॒भस्य॑ वि॒ष्टप᳚म् । ए॒वमे॒तस्य॑ वि॒ष्टप᳚म् । त्र॒यो॒द॒शम॑र॒त्निꣳ र॑श॒नाया॑-मु॒पाद॑धाति \textbf{ 9} \newline
                  \newline
                                \textbf{ TB 3.8.3.4} \newline
                  यथ॑र्.ष॒भस्य॑ वि॒ष्टपꣳ॑ सꣳस्क॒रोति॑ । ता॒दृगे॒व तत् ॥ पूर्व॒ आयु॑षि वि॒दथे॑षु क॒व्येत्या॑ह । आयु॑रे॒वास्मि॑न् दधाति ॥ तया॑ दे॒वाः सु॒तमाब॑भूवु॒-रित्या॑ह । भूति॑मे॒वोपाव॑र्तते ॥ ऋ॒तस्य॒ सामन᳚थ् स॒रमा॒रप॒न्तीत्या॑ह । स॒त्यं ॅवा ऋ॒तम् । स॒त्येनै॒वैन॑ मृ॒तेनार॑भते ॥ अ॒भि॒धा अ॒सीत्या॑ह \textbf{ 10} \newline
                  \newline
                                \textbf{ TB 3.8.3.5} \newline
                  तस्मा॑-दश्वमेधया॒जी सर्वा॑णि भू॒तान्य॒भि भ॑वति । भुव॑न-म॒सीत्या॑ह । भू॒मान॑-मे॒वोप॑ति । य॒न्ता-ऽसीत्या॑ह । य॒न्तार॑मे॒वैनं॑ करोति । ध॒र्ताऽसीत्या॑ह । ध॒र्तार॑मे॒वैनं॑ करोति । सो᳚ऽग्निं ॅवै᳚श्वान॒र-मित्या॑ह । अ॒ग्ना-वे॒वैनं॑ ॅवैश्वान॒रे जु॑होति । सप्र॑थस॒मित्या॑ह \textbf{ 11} \newline
                  \newline
                                \textbf{ TB 3.8.3.6} \newline
                  प्र॒जयै॒वैनं॑ प॒शुभिः॑ प्रथयति । स्वाहा॑कृत॒ इत्या॑ह । होम॑ ए॒वास्य॒षः । पृ॒थि॒व्या-मित्या॑ह । अ॒स्यामे॒वैनं॒ प्रति॑ष्ठापयति । य॒न्ता राड्य॒न्ताऽसि॒ यम॑नो ध॒र्ताऽसि॑ ध॒रुण॒ इत्या॑ह । रू॒पमे॒वास्यै॒-तन्म॑हि॒मानं॒ ॅव्याच॑ष्टे । कृ॒ष्यै त्वा॒ क्षेमा॑य त्वा र॒य्यै त्वा॒ पोषा॑य॒ त्वेत्या॑ह । आ॒शिष॑मे॒वैतामाशा᳚स्ते ॥स्व॒गा त्वा॑ दे॒वेभ्य॒ इत्या॑ह ( ) । दे॒वेभ्य॑ ए॒वैनꣳ॑ स्व॒गा क॑रोति । स्वाहा᳚ त्वा प्र॒जाप॑तय॒ इत्या॑ह । प्रा॒जा॒प॒त्यो वा अश्वः॑ । यस्या॑ ए॒व दे॒वता॑या आल॒भ्यत᳚ । तयै॒वैनꣳ॒॒ सम॑र्द्धयति । \textbf{ 12} \newline
                  \newline
                                    (ब॒ध्ना॒ति॒ - समृ॑द्ध्या - उ॒पद॑धा - त्य॒सीत्या॑ह॒ - सप्र॑थस॒मीत्या॑ह - दे॒वेभ्य॒ इत्या॑ह॒ पञ्च॑ च) \textbf{(A3)} \newline \newline
                \textbf{ 3.8.4     अनुवाकं   4 -अश्वस्य जलेऽवगाहनम्} \newline
                                \textbf{ TB 3.8.4.1} \newline
                  यः पि॒तुर॑नु॒जायाः᳚ पु॒त्रः । स पु॒रस्ता᳚न्नयति । यो मा॒तुर॑नु॒जायाः᳚ पु॒त्रः । स प॒श्चान्न॑यति । विष्व॑ञ्चमे॒वास्मा᳚त्-पा॒प्मानं॒ ॅविवृ॑हतः ॥ यो अर्व॑न्तं॒ जिघाꣳ॑सति॒ तम॒भ्य॑मीति॒ वरु॑ण॒ इति॒ श्वानं॑ चतुर॒क्षं प्रसौ॑ति । प॒रो मर्तः॑ प॒रः श्वेति॒ शुन॑श्चतु-र॒क्षस्य॒ प्रह॑न्ति । श्वेव॒ वै पा॒प्मा भ्रातृ॑व्यः । पा॒प्मान॑-मे॒वास्य॒ भ्रातृ॑व्यꣳ हन्ति ॥ सै॒द्ध्र॒कं मुस॑लं भवति \textbf{ 13} \newline
                  \newline
                                \textbf{ TB 3.8.4.2} \newline
                  कर्म॑ कर्मै॒वास्मै॑ साधयति ॥ पौꣳ॒॒श्च॒ले॒यो ह॑न्ति । पुꣳ॒॒श्च॒ल्वां ॅवै दे॒वाः शुचं॒ न्य॑दधुः । शु॒चैवास्य॒ शुचꣳ॑ हन्ति ॥ पा॒प्मा वा ए॒तमी᳚फ्स॒-तीत्या॑हुः । यो᳚ऽश्वमे॒धेन॒ यज॑त॒ इति॑ । अश्व॑स्याधस्प॒दमुपा᳚स्यति । व॒ज्री वा अश्वः॑ प्राजाप॒त्यः । वज्रे॑णै॒व पा॒प्मानं॒ भ्रातृ॑व्य॒-मव॑क्रामति ॥ द॒क्षि॒णा-ऽप॑प्लावयति । 14(10) ट्.भ्.3.8.4.3 पा॒प्मान॑-मे॒वास्मा॒-च्छम॑ल॒-मप॑प्लावयति ॥ ऐ॒षी॒क उ॑दू॒हो भ॑वति । आयु॒र्वा इ॒षीकाः᳚ । आयु॑रे॒-वास्मि॑न् दधति । अ॒मृतं॒ ॅवा इ॒षीकाः᳚ । अ॒मृत॑मे॒-वास्मि॑न् दधति ॥ वे॒त॒स॒शा॒खोप॒संब॑द्धा भवति । अ॒फ्सु-यो॑नि॒र्वा अश्वः॑ । अ॒फ्सु॒जो वे॑त॒सः । स्वादे॒वैनं॒ ॅयोने॒र्निर्मि॑मीते ( ) ॥ पु॒रस्ता᳚त्-प्र॒त्यञ्च॑-म॒भ्युदू॑हति । पु॒रस्ता॑-दे॒वास्मि॑न् प्र॒तीच्य॒ मृतं॑ दधाति ॥ अ॒हं च॒ त्वं च॑ वृत्रह॒न्निति॑ ब्र॒ह्मा यज॑मानस्य॒ हस्तं॑ गृह्णाति । ब्र॒ह्म॒क्ष॒त्रे ए॒व संद॑धाति । अ॒भि क्रत्वे᳚न्द्र भू॒रध॒ ज्मन्नित्य॑-ध्व॒र्यु-र्यज॑मानं ॅवाचयत्य॒-भिजि॑त्यै । \textbf{ 15} \newline
                  \newline
                                    (भ॒व॒ति॒ - प्ला॒व॒य॒ति॒ - मि॒मी॒ते॒ पञ्च॑ च) \textbf{(A4)} \newline \newline
                \textbf{ 3.8.5     अनुवाकं   5 -अश्वस्य प्रोक्षणं महर्त्विजाम्} \newline
                                \textbf{ TB 3.8.5.1} \newline
                  च॒त्वार॑ ऋ॒त्विजः॒ समु॑क्षन्ति । आ॒भ्य ए॒वैनं॑ चत॒सृभ्यो॑ दि॒ग्भ्यो॑-ऽभिसमी॑रयन्ति ॥ श॒तेन॑ राजपु॒त्रैः स॒हाद्ध्व॒र्युः । पु॒रस्ता᳚त्-प्र॒त्यङ्तिष्ठ॒न् प्रोक्ष॑ति । अ॒नेनाश्वे॑न॒ मेद्ध्ये॑ने॒ष्ट्वा । अ॒यꣳ राजा॑ वृ॒त्रं ॅव॑द्ध्या॒दिति॑ । रा॒ज्यं ॅवा अ॑द्ध्व॒र्युः । क्ष॒त्रꣳ रा॑जपु॒त्रः । रा॒ज्येनै॒-वास्मि॑न् क्ष॒त्रं द॑धाति ॥ श॒तेना॑-रा॒जभि॑रु॒ग्रैः स॒ह ब्र॒ह्मा \textbf{ 16} \newline
                  \newline
                                \textbf{ TB 3.8.5.2} \newline
                  द॒क्षि॒ण॒त उद॒ङ्तिष्ठ॒न् प्रोक्ष॑ति । अ॒नेनाश्वे॑न॒ मेद्ध्ये॑ने॒ष्ट्वा । अ॒यꣳ राजा᳚-ऽप्रतिधृ॒ष्यो᳚-ऽस्त्विति॑ । बलं॒ ॅवै ब्र॒ह्मा । बल॑मरा॒जोग्रः । बल॑नै॒-वास्मि॒न् बलं॑ दधाति ॥ श॒तेन॑ सूतग्राम॒णिभिः॑ स॒ह होता᳚ । प॒श्चात्-प्राङ्तिष्ठ॒न् प्रोक्ष॑ति । अ॒नेनाश्वे॑न॒ मेद्ध्ये॑ने॒ष्ट्वा । अ॒यꣳ राजा॒ऽस्यै वि॒शः \textbf{ 17} \newline
                  \newline
                                \textbf{ TB 3.8.5.3} \newline
                  ब॒हु॒ग्वै ब॑ह्व॒श्वायै॑ बह्वजावि॒कायै᳚ । ब॒हु॒व्री॒हि॒य॒वायै॑ बहुमाषति॒लायै᳚ । ब॒हु॒हि॒र॒ण्यायै॑ बहुह॒स्तिका॑यै । ब॒हु॒दा॒स॒पू॒रु॒षायै॑ रयि॒मत्यै॒ पुष्टि॑मत्यै । ब॒हु॒रा॒य॒स्पो॒षायै॒ राजा॒ऽस्त्विति॑ । भू॒मा वै होता᳚ । भू॒मा सू॑तग्राम॒ण्यः॑ । भू॒म्नैवास्मि॑न् भू॒मानं॑ दधाति ॥ श॒तेन॑ क्षत्त-संग्रही॒तृभिः॑ स॒होद्गा॒ता । उ॒त्त॒र॒तो द॑क्षि॒णा तिष्ठ॒न् प्रोक्ष॑ति \textbf{ 18} \newline
                  \newline
                                \textbf{ TB 3.8.5.4} \newline
                  अ॒नेनाश्वे॑न॒ मेद्ध्ये॑ ने॒ष्ट्वा । अ॒यꣳ राजा॒ सर्व॒मायु॑रे॒त्विति॑ । आयु॒र्वा उ॑द्गा॒ता । आयुः॑ क्षत्त-संग्रही॒तारः॑ । आयु॑षै॒-वास्मि॒न्नायु॑-र्दधाति ॥ श॒तꣳ श॑तं भवन्ति । श॒तायुः॒ पुरु॑षः श॒तेन्द्रि॑यः । आयु॑ष्ये॒वेन्द्रि॒ये प्रति॑तिष्ठति ॥ च॒तुः॒ श॒ता भ॑वन्ति । चत॑स्रो॒ दिशः॑ ( ) । दि॒क्ष्वे॑व प्रति॑तिष्ठति । \textbf{ 19} \newline
                  \newline
                                    (ब्र॒ह्मा - वि॒श - उ॑क्षति॒ - दिश॒ एकं॑ च) \textbf{(A5)} \newline \newline
                \textbf{ 3.8.6     अनुवाकं   6 -भूमौ पततां बिन्दूनामभिमन्त्रणम्} \newline
                                \textbf{ TB 3.8.6.1} \newline
                  यथा॒ वै ह॒विषो॑ गृही॒तस्य॒ स्कन्द॑ति । ए॒वं ॅवा ए॒तदश्व॑स्य स्कन्दति । यन्नि॒क्तमना॑-लब्ध-मुथ्सृ॒जन्ति॑ । यथ्स्तोक्या॑ अ॒न्वाह॑ । स॒र्व॒हुत॑-मे॒वैनं॑ करो॒त्य-स्क॑न्दाय । अस्क॑न्नꣳ॒॒ हि तत् । यद्धु॒तस्य॒ स्कन्द॑ति ॥ स॒हस्र॒-मन्वा॑ह । स॒हस्र॑ संमितः सुव॒र्गो लो॒कः । सु॒व॒र्गस्य॑ लो॒कस्या॒-भिजि॑त्यै । \textbf{ 20} \newline
                  \newline
                                \textbf{ TB 3.8.6.2} \newline
                  यत्परि॑मिता अनुब्रू॒यात् । परि॑मित॒मव॑रुन्धीत । अप॑रिमिता॒ अन्वा॑ह । अप॑रिमितः सुव॒र्गो लो॒कः । सु॒व॒र्गस्य॑ लो॒कस्य॒ सम॑ष्ट्यै ॥ स्तोक्या॑ जुहोति । या ए॒व वर्ष्या॒ आपः॑ । ता अव॑रुन्धे ॥ अ॒स्यां जु॑होति । इ॒यं ॅवा अ॒ग्निर्वै᳚श्वान॒रः \textbf{ 21} \newline
                  \newline
                                \textbf{ TB 3.8.6.3} \newline
                  अ॒स्यामे॒वैनाः॒ प्रति॑ष्ठापयति ॥ उ॒वाच॑ ह प्र॒जाप॑तिः । स्तोक्या॑सु॒ वा अ॒हम॑श्वमे॒धꣳ सꣳस्था॑पयामि । तेन॒ ततः॒ सꣳस्थि॑तेन चरा॒मीति॑ ॥ अ॒ग्नये॒ स्वाहेत्या॑ह । अ॒ग्नय॑ ए॒वैनं॑ जुहोति ॥ सोमा॑य॒ स्वाहेत्या॑ह । सोमा॑यै॒वैनं॑ जुहोति । स॒वि॒त्रे स्वाहेत्या॑ह । स॒वि॒त्र ए॒वैनं॑ जुहोति \textbf{ 22} \newline
                  \newline
                                \textbf{ TB 3.8.6.4} \newline
                  सर॑स्वत्यै॒ स्वाहेत्या॑ह । सर॑स्वत्या ए॒वैनं॑ जुहोति । पू॒ष्णे स्वाहेत्या॑ह । पू॒ष्ण ए॒वैनं॑ जुहोति । बृह॒स्पत॑ये॒ स्वाहेत्या॑ह । बृह॒स्पत॑य ए॒वैनं॑ जुहोति । अ॒पां मोदा॑य॒ स्वाहेत्या॑ह । अ॒द्भ्य ए॒वैनं॑ जुहोति । वा॒यवे॒ स्वाहेत्या॑ह । वा॒यव॑ ए॒वैनं॑ जुहोति \textbf{ 23} \newline
                  \newline
                                \textbf{ TB 3.8.6.5} \newline
                  मि॒त्राय॒ स्वाहेत्या॑ह । मि॒त्रायै॒वैनं॑ जुहोति । वरु॑णाय॒ स्वाहेत्या॑ह । वरु॑णायै॒वैनं॑ जुहोति ॥ ए॒ताभ्य॑ ए॒वैनं॑ दे॒वता᳚भ्यो जुहोति ॥ दश॑ दश स॒पांदं॑ जुहोति । दशा᳚क्षरा वि॒राट् । अन्नं॑ ॅवि॒राट् । वि॒राजै॒-वान्नाद्य॒मव॑रुन्धे ॥ प्र वा ए॒षो᳚-ऽस्माल्लो॒का-च्च्य॑वते ( ) । यः परा॑ची॒राहु॑ती-र्जु॒होति॑ । पुनः॑ पुनरभ्या॒वर्तं॑ जुहोति । अ॒स्मिन्ने॒व लो॒के प्रति॑तिष्ठति ॥ ए॒ताꣳ ह॒ वाव सो᳚ऽश्वमे॒धस्य॒ सꣳस्थि॑ति मुवा॒चा-स्क॑न्दाय । अस्क॑न्नꣳ॒॒ हि तत् । यद्य॒ज्ञ्स्य॒ सꣳस्थि॑तस्य॒ स्कन्द॑ति । \textbf{ 24} \newline
                  \newline
                                    (अ॒भिजि॑त्यै - वैश्वान॒रः - स॑वि॒त्र ए॒वैन॑म् जुहोति - वा॒यव॑ ए॒वैन॑म् जुहोति - च्यवते॒ षट्च॑) \textbf{(A6)} \newline \newline
                \textbf{ 3.8.7     अनुवाकं   7 -अथाध्वर्युप्रोक्षणम्} \newline
                                \textbf{ TB 3.8.7.1} \newline
                  प्र॒जाप॑तये त्वा॒ जुष्टं॒ प्रोक्षा॒मीति॑ पुर॒स्ता᳚त्-प॒त्यङ्तिष्ठ॒न् प्रोक्ष॑ति । प्र॒जाप॑ति॒र्वै दे॒वाना॑-मन्ना॒दो वी॒र्या॑वान् । अ॒न्नाद्य॑-मे॒वास्मि॑न् वी॒र्यं॑ दधाति । तस्मा॒दश्वः॑ पशू॒ना-म॑न्ना॒दो वी॒र्या॑वत्तमः ॥ इ॒न्द्रा॒ग्निभ्यां॒ त्वेति॑ दक्षिण॒तः । इ॒न्द्रा॒ग्नी वै दे॒वाना॒मोजि॑ष्ठौ॒ बलि॑ष्ठौ । ओज॑ ए॒वास्मि॒न् बलं॑ दधाति । तस्मा॒दश्वः॑ पशू॒नामोजि॑ष्ठो॒ बलि॑ष्ठः ॥ वा॒यवे॒ त्वेति॑ प॒श्चात् । वा॒युर्वै दे॒वाना॑मा॒शुः सा॑रसा॒रित॑मः \textbf{ 25} \newline
                  \newline
                                \textbf{ TB 3.8.7.2} \newline
                  ज॒वमे॒वास्मि॑न्दधाति । तस्मा॒दश्वः॑ पशू॒नामा॒शुः सा॑रसा॒रित॑मः ॥ विश्वे᳚भ्यस्त्वा दे॒वेभ्य॒ इत्यु॑त्तर॒तः । विश्वे॒ वै दे॒वा दे॒वानां᳚ ॅयश॒स्वित॑माः । यश॑ ए॒वास्मि॑न्दधाति । तस्मा॒दश्वः॑ पशू॒नां ॅय॑श॒स्वित॑मः ॥ दे॒वेभ्य॒-स्त्वेत्य॒धस्ता᳚त् । दे॒वा वै दे॒वाना॒-मप॑चिततमाः । अप॑चिति-मे॒वास्मि॑-न्दधाति । तस्मा॒दश्वः॑ पशू॒ना-मप॑चिततमः । \textbf{ 26} \newline
                  \newline
                                \textbf{ TB 3.8.7.3} \newline
                  सर्वे᳚भ्यस्त्वा दे॒वेभ्य॒ इत्यु॒परि॑ष्टात् । सर्वे॒ वै दे॒वास्त्विषि॑मन्तो हर॒स्विनः॑ । त्विषि॑मे॒-वास्मि॒न्॒. हरो॑ दधाति । तस्मा॒दश्वः॑ पशू॒नां त्विषि॑मान् हर॒स्वित॑मः ॥ दि॒वे त्वा॒ऽन्तरि॑क्षाय त्वा पृथि॒व्यै त्वेत्या॑ह । ए॒भ्य ए॒वैनं॑ ॅलो॒केभ्यः॒ प्रोक्ष॑ति ॥ स॒ते त्वाऽस॑ते त्वा॒-ऽद्भ्यस्त्वौ-ष॑धीभ्यस्त्वा॒ विश्वे᳚भ्यस्त्वा भू॒तेभ्य॒ इत्या॑ह । तस्मा॑दश्वमेधया॒जिनꣳ॒॒ सर्वा॑णि भू॒तान्युप॑ जीवन्ति ॥ ब्र॒ह्म॒वा॒दिनो॑ वदन्ति । यत्प्रा॑जाप॒त्योऽश्वः॑ ( ) । अथ॒ कस्मा॑देन-म॒न्याभ्यो॑ दे॒वता॒भ्योऽपि॒ प्रोक्ष॒तीति॑ । अश्वे॒ वै सर्वा॑ दे॒वता॑ अ॒न्वाय॑त्ताः । तं ॅयद्विश्वे᳚भ्यस्त्वा भू॒तेभ्य॒ इति॑ प्रा॒क्षति॑ । दे॒वता॑ ए॒वास्मि॑न्-न॒न्वाया॑तयति । तस्मा॒दश्वे॒ सर्वा॑ दे॒वता॑ अ॒न्वाय॑त्ताः । \textbf{ 27} \newline
                  \newline
                                    (सा॒र॒सा॒रित॒मो - ऽप॑चिततमः - प्राजाप॒त्योऽश्व॒ पञ्च॑ च) \textbf{(A7)} \newline \newline
                \textbf{ 3.8.8     अनुवाकं   8 -अश्वचरितानामश्वरूपाणां च होमाः} \newline
                                \textbf{ TB 3.8.8.1} \newline
                  यथा॒ वै ह॒विषो॑ गृही॒तस्य॒ स्कन्द॑ति । ए॒वं ॅवा ए॒तदश्व॑स्य स्कन्दति । यत्प्रोक्षि॑त॒-मना॑लब्ध-मुथ्सृ॒जन्ति॑ । यद॑श्व-चरि॒तानि॑ जु॒होति॑ । स॒र्व॒हुत॑-मे॒वैनं॑ करो॒त्य-स्क॑न्दाय । अस्क॑न्नꣳ॒॒ हि तत् । यद्धु॒तस्य॒ स्कन्द॑ति ॥ ईं॒का॒राय॒ स्वाहें कृ॑ताय॒ स्वाहेत्या॑ह । ए॒तानि॒ वा अ॑श्वचरि॒तानि॑ । च॒रि॒तैरे॒वैनꣳ॒॒ सम॑र्द्धयति । \textbf{ 28} \newline
                  \newline
                                \textbf{ TB 3.8.8.2} \newline
                  तदा॑हुः । अना॑हुतयो॒ वा अ॑श्वचरि॒तानि॑ । नैता हो॑त॒व्या॑ इति॑ । अथो॒ खल्वा॑हुः । हो॒त॒व्या॑ ए॒व । अत्र॒ वावैवं ॅवि॒द्वान॑श्वमे॒धꣳ सꣳस्था॑पयति । यद॑श्व-चरि॒तानि॑ जु॒होति॑ । तस्मा᳚-द्धोत॒व्या॑ इति॑ ॥ ब॒हि॒र्द्धा वा ए॑नमे॒तदा॒यत॑नाद्-दधाति । भ्रातृ॑व्यमस्मै जनयति \textbf{ 29} \newline
                  \newline
                                \textbf{ TB 3.8.8.3} \newline
                  यस्या॑ नायत॒ने᳚-ऽन्यत्रा॒ग्ने-राहु॑तीर् जु॒होति॑ । सा॒वि॒त्रि॒या इष्ट्याः᳚ पु॒रस्ता᳚थ्स्विष्ट॒ कृतः॑ । आ॒ह॒व॒नीये᳚-ऽश्वचरि॒तानि॑ जुहोति । आ॒यत॑न ए॒वास्याहु॑तीर् जुहोति । नास्मै॒ भ्रातृ॑व्यं जनयति ॥ तदा॑हुः । य॒ज्ञ्॒मु॒खे य॑ज्ञ्मुखे होत॒व्याः᳚ । य॒ज्ञ्स्य॒ क्लृप्त्यै᳚ । सु॒व॒र्गस्य॑ लो॒कस्या-नु॑ख्यात्या॒ इति॑ ॥ अथो॒ खल्वा॑हुः \textbf{ 30} \newline
                  \newline
                                \textbf{ TB 3.8.8.4} \newline
                  यद्य॑ज्ञ्मु॒खे य॑ज्ञ्मुखे जुहु॒यात् । प॒शुभि॒-र्यज॑मानं॒ ॅव्य॑द्र्धयेत् । अव॑ सुव॒र्गाल्लो॒का-त्प॑द्येत । पापी॑यान् थ्स्या॒दिति॑ । स॒कृदे॒व हो॑त॒व्याः᳚ । न यज॑मानं प॒शुभि॒-र्व्य॑द्र्धयति । अ॒भि सु॑व॒र्गं ॅलो॒कं ज॑यति । न पापी॑यान् भवति ॥ अ॒ष्टाच॑त्वारिꣳशतमश्व रू॒पाणि॑ जुहोति । अ॒ष्टाच॑त्वारिꣳशदक्षरा॒ जग॑ती ( ) । जाग॒तोऽश्वः॑ प्राजाप॒त्यः समृ॑द्ध्यै ॥ एक॒मति॑रिक्तं जुहोति । तस्मा॒देकः॑ प्र॒जास्वद्र्धु॑कः । \textbf{ 31} \newline
                  \newline
                                    (अ॒द्र्ध॒य॒ति॒ - ज॒न॒य॒ति॒ - खल्वा॑हु॒र् - जग॑ती॒ त्रीणि॑ च) \textbf{(A8)} \newline \newline
                \textbf{ 3.8.9     अनुवाकं   9 -अश्वस्य नाम्नामभिवाचनं, उथ्सर्गश्च} \newline
                                \textbf{ TB 3.8.9.1} \newline
                  वि॒भूर्मा॒त्रा प्र॒भूः पि॒त्रेत्या॑ह । इ॒यं ॅवै मा॒ता । अ॒सौ पि॒ता । आ॒भ्यामे॒वैनं॒ परि॑ददाति ॥ अश्वो॑ऽसि॒ हयो॒ऽसीत्या॑ह । शास्त्ये॒वैन॑मे॒तत् । तस्मा᳚च्छि॒ष्टाः प्र॒जा जा॑यन्ते ॥ अत्यो॒ऽसीत्या॑ह । तस्मा॒दश्वः॒ सर्वा᳚न् प॒शूनत्ये॑ऽति । तस्मा॒दश्वः॒ सर्वे॑षां पशू॒नाꣳ श्रैष्ठ्यं॑ गच्छति । \textbf{ 32} \newline
                  \newline
                                \textbf{ TB 3.8.9.2} \newline
                  प्र यशः॒ श्रैष्ठ्य॑माप्नोति । य ए॒वं ॅवेद॑ ॥ नरो॒ऽस्यर्वा॑ऽसि॒ सप्ति॑रसि वा॒ज्य॑सीत्या॑ह । रू॒पमे॒वास्यै॒तन्म॑हि॒मानं॒ ॅव्याच॑ष्टे ॥ ययु॒र्नामा॒-ऽसीत्या॑ह । ए॒तद्वा अश्व॑स्य प्रि॒यं ना॑म॒धेय᳚म् । प्रि॒येणै॒वैनं॑ नाम॒धेये॑ना॒भिव॑दति । तस्मा॒दप्या॑मि॒त्रौ स॒गंत्य॑ । नाम्ना॒ चेद्ध्वये॑ते । मि॒त्रमे॒व भ॑वतः । \textbf{ 33} \newline
                  \newline
                                \textbf{ TB 3.8.9.3} \newline
                  आ॒दि॒त्यानां॒ पत्वा-ऽन्वि॒हीत्या॑ह । आ॒दि॒त्याने॒वैनं॑ गमयति ॥ अ॒ग्नये॒ स्वाहा॒ स्वाहे᳚न्द्रा॒ग्निभ्या॒मिति॑ पूर्वहो॒मा-ञ्जु॑होति । पूव॑ ए॒व द्वि॒षन्तं॒ भ्रातृ॑व्य॒-मति॑क्रामति ॥ भूर॑सि भु॒वे त्वा॒ भव्या॑य त्वा भविष्य॒ते त्वेत्युथ्सृ॑जति सर्व॒त्वाय॑ ॥ देवा॑ आशापाला ए॒तं दे॒वेभ्योऽश्वं॒ मेधा॑य॒ प्रोक्षि॑तं गोपाय॒तेत्या॑ह । श॒तं ॅवै तल्प्या॑ राजपु॒त्रा दे॒वा आ॑शापा॒लाः । तेभ्य॑ ए॒वैनं॒ परि॑ददाति ॥ ई॒श्व॒रो वा अश्वः॒ प्रमु॑क्तः॒ परां᳚ परा॒वतं॒ गन्तोः᳚ । इ॒ह धृतिः॒ स्वाहे॒ह विधृ॑तिः॒ स्वाहे॒ह रन्तिः॒ स्वाहे॒ह रम॑तिः॒ स्वाहेति॑ चतृ॒षु प॒थ्सु जु॑होति \textbf{ 34} \newline
                  \newline
                                \textbf{ TB 3.8.9.4} \newline
                  ए॒ता वा अश्व॑स्य॒ बन्ध॑नम् । ताभि॑रे॒वैनं॑ बद्ध्नाति । तस्मा॒दश्वः॒ प्रमु॑क्तो॒ बन्ध॑न॒-माग॑च्छति । तस्मा॒दश्वः॒ प्रमु॑क्तो॒ बन्ध॑नं॒ न ज॑हाति ॥ रा॒ष्ट्रं ॅवा अ॑श्वमे॒धः । रा॒ष्ट्रे खलु॒ वा ए॒ते व्याय॑च्छन्ते । येऽश्वं॒ मेद्ध्यꣳ॒॒ रक्ष॑न्ति । तेषां॒ ॅय उ॒दृचं॒ गच्छ॑न्ति । रा॒ष्ट्रादे॒व ते रा॒ष्ट्रं ग॑च्छन्ति । अ॒थ य उ॒दृचं॒ न गच्छ॑न्ति ( ) \textbf{ 35} \newline
                  \newline
                                \textbf{ TB 3.8.9.5} \newline
                  रा॒ष्ट्रादे॒व ते व्यव॑च्छिद्यन्ते । परा॒ वा ए॒ष सि॑च्यते । यो॑-ऽब॒लो᳚-ऽश्वमे॒धेन॒ यज॑ते । यद॒मित्रा॒ अश्वं॑ ॅवि॒न्देरन्न्॑ । ह॒न्येता᳚स्य य॒ज्ञ्ः । च॒तु॒श्श॒ता र॑क्षन्ति । य॒ज्ञ्स्याघाताय ॥ अथा॒न्य-मा॒नीय॒ प्रोक्षे॑युः । सैव ततः॒ प्राय॑श्चित्तिः । \textbf{ 36} \newline
                  \newline
                                    (ग॒च्छ॒ति॒ - भ॒व॒तः॒ - प॒थ्सु जु॑होति॒ - न गच्छ॑न्ति॒ - +नव॑ च) \textbf{(A9)} \newline \newline
                \textbf{ 3.8.10    अनुवाकं   10 -दीक्षाभिधानम् तत्र वैश्वदेवहोमः} \newline
                                \textbf{ TB 3.8.10.1} \newline
                  प्र॒जाप॑ति-रकामयताश्वमे॒धेन॑ यजे॒येति॑ । स तपो॑ऽतप्यत । तस्य॑ तेपा॒नस्य॑ । स॒प्तात्मनो॑ दे॒वता॒ उद॑क्रामन्न् । सा दी॒क्षा-ऽभ॑वत् । स ए॒तानि॑ वैश्वदे॒वान्य॑पश्यत् । तान्य॑ जुहोत् । तैर्वै स दी॒क्षामवा॑रुन्ध । यद्वै᳚श्व दे॒वानि॑ जु॒होति॑ । दी॒क्षामे॒व-तैर्यज॑मा॒नो-ऽव॑रुन्धे । \textbf{ 37} \newline
                  \newline
                                \textbf{ TB 3.8.10.2} \newline
                  स॒प्त जु॑होति । स॒प्त हि ता दे॒वता॑ उ॒दक्रा॑मन्न् ॥ अ॒न्व॒हं जु॑होति । अ॒न्व॒हमे॒व दी॒क्षामव॑रुन्धे ॥ त्रीणि॑ वैश्वदे॒वानि॑ जुहोति । च॒त्वार्यौ᳚द्ग्रह॒णानि॑ । स॒प्त संप॑द्यन्ते । स॒प्त वै शी॑र.ष॒ण्याः᳚ प्रा॒णाः । प्रा॒णा दी॒क्षा । प्रा॒णैरे॒व प्रा॒णान्दी॒क्षामव॑रुन्धे । \textbf{ 38} \newline
                  \newline
                                \textbf{ TB 3.8.10.3} \newline
                  एक॑विꣳशतिं ॅवैश्वदे॒वानि॑ जुहोति । एक॑विꣳशति॒र्वै दे॑वलो॒काः । द्वाद॑श॒ मासाः॒ पञ्च॒र्तवः॑ । त्रय॑ इ॒मे लो॒काः । अ॒सावा॑दि॒त्य ए॑कविꣳ॒॒शः ॥ ए॒ष सु॑व॒र्गो लो॒कः । तद्दैव्यं॑ क्ष॒त्त्रम् । सा श्रीः । तद्ब्र॒द्ध्नस्य॑ वि॒ष्टप᳚म् । तथ्स्वारा᳚ज्य-मुच्यते । \textbf{ 39} \newline
                  \newline
                                \textbf{ TB 3.8.10.4} \newline
                  त्रिꣳ॒॒शत॑मौद्ग्रह॒णानि॑ जुहोति । त्रिꣳ॒॒शद॑क्षरा वि॒राट् । अन्नं॑ ॅवि॒राट् । वि॒राजै॒वान्नाद्य॒मव॑रुन्धे ॥ त्रे॒धा वि॒भज्य॑ दे॒वतां᳚ जुहोति । त्र्या॑वृतो॒ वै दे॒वाः । त्र्या॑वृत इ॒मे लो॒काः । ए॒षां ॅलो॒काना॒माप्त्यै᳚ । ए॒षां ॅलो॒कानां॒ क्लृप्त्यै᳚ ॥ अप॒ वा ए॒तस्मा᳚त्प्रा॒णाः क्रा॑मन्ति \textbf{ 40} \newline
                  \newline
                                \textbf{ TB 3.8.10.5} \newline
                  यो दी॒क्षाम॑ति-रे॒चय॑ति । स॒प्ता॒हं प्रच॑रन्ति । स॒प्त वै शी॑र्.ष॒ण्याः᳚ प्रा॒णाः । प्रा॒णा दी॒क्षा । प्रा॒णैरे॒व प्रा॒णान्दी॒क्षामव॑रुन्धे ॥ पू॒र्णा॒हु॒ति-मु॑त्त॒मां जु॑होति । सर्वं॒ ॅवै पू᳚र्णाहु॒तिः । सर्व॑मे॒वाप्नो॑ति । अथो॑ इ॒यं ॅवै पू᳚र्णाहु॒तिः । अ॒स्यामे॒व प्रति॑तिष्ठति ( ) । \textbf{ 41} \newline
                  \newline
                                    (रु॒न्धे॒ - प्रा॒णान् दी॒क्षामव॑रुन्ध - उच्यते - क्रामन्ति - तिष्ठति) \textbf{(A10)} \newline \newline
                \textbf{ 3.8.11    अनुवाकं   11 -वैश्वदेवहोममन्त्रव्याख्यानम्} \newline
                                \textbf{ TB 3.8.11.1} \newline
                  प्रजाप॑ति-रश्वमे॒ध-म॑सृजत । तꣳ सृ॒ष्टं न किञ्च॒नोद॑यच्छत् । तं ॅवै᳚श्वदे॒वान्ये॒-वोद॑यच्छन्न् । यद्वै᳚श्वदे॒वानि॑ जु॒होति॑ । य॒ज्ञ्स्योद्य॑त्यै ॥ स्वाहा॒-ऽऽधिमाधी॑ताय॒ स्वाहा᳚ । स्वाहा-ऽऽधी॑तं॒ मन॑से॒ स्वाहा᳚ । स्वाहा॒ मनः॑ प्र॒जाप॑तये॒ स्वाहा᳚ । काय॒ स्वाहा॒ कस्मै॒ स्वाहा॑ कत॒मस्मै॒ स्वाहेति॑ प्राजाप॒त्ये मुख्ये॑ भवतः । प्र॒जाप॑ति-मुखाभिरे॒वैनं॑ दे॒वता॑भि॒रुद्य॑च्छते । \textbf{ 42} \newline
                  \newline
                                \textbf{ TB 3.8.11.2} \newline
                  अदि॑त्यै॒ स्वाहाऽदि॑त्यै म॒ह्यै᳚ स्वाहाऽदि॑त्यै सुमृडी॒कायै॒ स्वाहेत्या॑ह । इ॒यं ॅवा अदि॑तिः । अ॒स्या ए॒वैनं॑ प्रति॒ष्ठायोद्य॑च्छते ॥ सर॑स्वत्यै॒ स्वाहा॒ सर॑स्वत्यै बृह॒त्यै᳚ स्वाहा॒ सर॑स्वत्यै पाव॒कायै॒ स्वाहेत्या॑ह । वाग्वै सर॑स्वती । वा॒चैवैन॒मुद्य॑च्छते ॥ पू॒ष्णे स्वाहा॑ पू॒ष्णे प्र॑प॒थ्या॑य॒ स्वाहा॑ पू॒ष्णे न॒रन्धि॑षाय॒ स्वाहेत्या॑ह । प॒शवो॒ वै पू॒षा । प॒शुभि॑रे॒वैन॒-मुद्य॑च्छते ॥ त्वष्ट्रे॒ स्वाहा॒ त्वष्ट्र॑ तु॒रीपा॑य॒ स्वाहा॒ त्वष्ट्रे॑ पुरु॒रूपा॑य॒ स्वाहेत्या॑ह ( ) । त्वष्टा॒ वै प॑शू॒नां मि॑थु॒नानाꣳ॑ रूप॒कृत् । रू॒पमे॒व प॒शुषु॑ दधाति । अथो॑ रू॒पैरे॒वैन॒-मुद्य॑च्छते ॥ विष्ण॑वे॒ स्वाहा॒ विष्ण॑वे निखुर्य॒पाय॒ स्वाहा॒ विष्ण॑वे निभूय॒पाय॒ स्वाहेत्या॑ह । य॒ज्ञो वै विष्णुः॑ । य॒ज्ञायै॒वैन॒-मुद्य॑च्छते ॥ पू॒र्णा॒हु॒ति-मु॑त्त॒मां जु॑होति । प्रत्युत्त॑ब्ध्यै सय॒त्वाय॑ । \textbf{ 43} \newline
                  \newline
                                    (य॒च्छ॒ते॒ - पु॒रु॒रूपा॑य॒ स्वाहेत्या॑हा॒ष्टौ च॑) \textbf{(A11)} \newline \newline
                \textbf{ 3.8.12    अनुवाकं   12 -अश्वसंचरणवथ्सरे प्रतिदिनं देवयजनदेशे कर्तव्यमिष्टित्रयमभिधीयते} \newline
                                \textbf{ TB 3.8.12.1} \newline
                  सा॒वि॒त्र-म॒ष्टक॑पालं प्रा॒तर्निर्व॑पति । अ॒ष्टाक्ष॑रा गाय॒त्री । गा॒य॒त्रं प्रा॑तस्सव॒नं । प्रा॒त॒स्स॒व॒ना दे॒वैनं॑ गायत्रि॒या ॑007आ;छन्द॒सोऽधि॒ निर्मि॑मीते । अथो᳚ प्रातस्सव॒नमे॒व तेना᳚प्नोति । गा॒य॒त्रीं छन्दः॑ ॥ स॒वि॒त्रे प्र॑सवि॒त्र एका॑दश-कपालं म॒द्ध्यन्दि॑ने । एका॑दशाक्षरा त्रि॒ष्टुप् । त्रैष्टु॑भं॒ माद्ध्य॑न्दिनꣳ॒॒ सव॑नम् । माद्ध्य॑न्दिना-दे॒वैनꣳ॒॒ सव॑नात्त्रि॒ष्टुभः॒-छन्द॒सोऽधि॒ निर्मि॑मीते \textbf{ 44} \newline
                  \newline
                                \textbf{ TB 3.8.12.2} \newline
                  अथो॒ माद्ध्य॑न्दिनमे॒व सव॑नं॒ तेना᳚प्नोति । त्रि॒ष्टुभं॒ छन्दः॑ ॥ स॒वि॒त्र आ॑सवि॒त्रे द्वाद॑श-कपाल-मपरा॒ह्णे । द्वाद॑शाक्षरा॒ जग॑ती । जाग॑तं तृतीय सव॒नम् । तृ॒ती॒य॒ स॒व॒नादे॒वैनं॒ जग॑त्या॒ ॑007आ;छन्द॒सोऽधि॒ निर्मि॑मीते । अथो॑ तृतीय सव॒नमे॒व तेना᳚प्नोति । जग॑तीं॒ छन्दः॑ ॥ ई॒श्व॒रो वा अश्वः॒ प्रमु॑क्तः॒ परां᳚ परा॒वतं॒ गन्तोः᳚ । इ॒ह धृतिः॒ स्वाहे॒ह विधृ॑तिः॒ स्वाहे॒ह रन्तिः॒ स्वाहे॒ह रम॑तिः॒ स्वाहेति॒ चत॑स्र॒ आहु॑ती-र्जुहोति ( ) \textbf{ 45} \newline
                  \newline
                                \textbf{ TB 3.8.12.3} \newline
                  चत॑स्रो॒ दिशः॑ । दि॒ग्भिरे॒वैनं॒ परि॑गृह्णाति ॥ आश्व॑त्थो व्र॒जो भ॑वति । प्र॒जाप॑ति-र्दे॒वेभ्यो॒ निला॑यत । अश्वो॑ रू॒पं कृ॒त्वा । सो᳚ऽश्व॒त्थे स॑म्ॅवथ्स॒र-म॑तिष्ठत् । तद॑श्व॒त्थस्या᳚-श्वत्थ॒त्वम् । यदाश्व॑त्थो व्र॒जो भव॑ति । स्व ए॒वैनं॒ ॅयोनौ॒ प्रति॑ष्ठापयति । \textbf{ 46} \newline
                  \newline
                                    (त्रि॒ष्टुभ॒ ॑श्छन्द॒सोऽधि॒ निर्मि॑मीते - जुहोति॒ - नव॑ च) \textbf{(A12)} \newline \newline
                \textbf{ 3.8.13    अनुवाकं   13 -संवथ्सराद्र्ध्वमुख्यस्याग्रेरूपस्थानम्} \newline
                                \textbf{ TB 3.8.13.1} \newline
                  आ ब्र॑ह्मन् ब्राह्म॒णो ब्र॑ह्मवर्च॒सी जा॑यता॒-मित्या॑ह । ब्रा॒ह्म॒ण ए॒व ब्र॑ह्मवर्च॒सं दधाति । तस्मा᳚त्पु॒रा ब्रा᳚ह्म॒णो ब्र॑ह्मवर्च॒स्य॑ जायत ॥ आऽस्मिन् रा॒ष्ट्रे रा॑ज॒न्य॑ इष॒व्यः॑ शूरो॑ महार॒थो जा॑यता॒-मित्या॑ह । रा॒ज॒न्य॑ ए॒व शौ॒र्यं म॑हि॒मानं॑ दधाति । तस्मा᳚त्पु॒रा रा॑ज॒न्य॑ इष॒व्यः॑ शूरो॑ महार॒थो॑-ऽजायत ॥ दोग्ध्री॑ धे॒नुरित्या॑ह । धे॒न्वामे॒व पयो॑ दधाति । तस्मा᳚त्पु॒रा दोग्ध्री॑ धे॒नुर॑जायत । वोढा॑ऽन॒ड्वा-नित्या॑ह \textbf{ 47} \newline
                  \newline
                                \textbf{ TB 3.8.13.2} \newline
                  अ॒न॒डुह्ये॒व वी॒र्यं॑ दधाति । तस्मा᳚त्पु॒रा वोढा॑ऽन॒ड्वान॑जायत । आ॒शुः सप्ति॒रित्या॑ह । अश्व॑ ए॒व ज॒वं द॑धाति । तस्मा᳚त्पु॒रा-ऽऽशुरश्वो॑-ऽजायत । पुर॑धिं॒र्योषेत्या॑ह । यो॒षित्ये॒व रू॒पं द॑धाति । तस्मा॒थ्स्त्री यु॑व॒तिः प्रि॒या भावु॑का ॥ जि॒ष्णू र॑थे॒ष्ठा इत्या॑ह । आ ह॒ वै तत्र॑ जि॒ष्णू र॑थे॒ष्ठा जा॑यते \textbf{ 48} \newline
                  \newline
                                \textbf{ TB 3.8.13.3} \newline
                  यत्रै॒तेन॑ य॒ज्ञेन॒ यज॑न्ते ॥ स॒भेयो॒ युवेत्या॑ह । यो वै पू᳚र्ववय॒सी । स स॒भेयो॒ युवा᳚ । तस्मा॒द्युवा॒ पुमा᳚न् प्रि॒यो भावु॑कः ॥ आऽस्य यज॑मानस्य वी॒रो जा॑यता॒-मित्या॑ह । आ ह॒ वै तत्र॒ यज॑मानस्य वी॒रो जा॑यते । यत्रै॒तेन॑ य॒ज्ञेन॒ यज॑न्ते ॥ नि॒का॒मेनि॑कामे नः प॒र्जन्यो॑ वर्.ष॒त्वित्या॑ह । नि॒का॒मेनि॑कामे ह॒ वै तत्र॑ प॒र्जन्यो॑ वर्.षति ( ) । यत्रै॒तेन॑ य॒ज्ञेन॒ यज॑न्ते ॥ फ॒लिन्यो॑ न॒ ओष॑धयः पच्यन्ता॒-मित्या॑ह । फ॒लिन्यो॑ ह॒ वै तत्रौष॑धयः पच्यन्ते । यत्रै॒तेन॑ य॒ज्ञेन॒ यज॑न्ते । यो॒ग॒क्षे॒मो नः॑ कल्पता॒-मित्या॑ह । कल्प॑ते ह॒ वै तत्र॑ प्र॒जाभ्यो॑ योगक्षे॒मः । यत्रै॒तेन॑ य॒ज्ञेन॒ यज॑न्ते । \textbf{ 49} \newline
                  \newline
                                    (अ॒न॒ड्वानित्या॑ह - जायते - वर्.षति स॒प्त च॑) \textbf{(A13)} \newline \newline
                \textbf{ 3.8.14    अनुवाकं   14 -अन्नहोमाः. त्रिरात्ररूपस्याश्वमेधस्य प्रथमदिनरात्रौ} \newline
                                \textbf{ TB 3.8.14.1} \newline
                  प्र॒जाप॑ति-र्दे॒वेभ्यो॑ य॒ज्ञान् व्यादि॑शत् । स आ॒त्म-न्न॑श्वमे॒धम॑धत्त । तं दे॒वा अ॑ब्रुवन्न् । ए॒ष वाव य॒ज्ञ्ः । यद॑श्वमे॒धः । अप्ये॒व नोऽत्रा॒स्त्विति॑ । तेभ्य॑ ए॒तान॑न्नहो॒मान् प्राय॑च्छत् । तान॑जुहोत् । तैर्वै स दे॒वान॑ प्रीणात् । यद॑न्न-हो॒माञ्जु॒होति॑ \textbf{ 50} \newline
                  \newline
                                \textbf{ TB 3.8.14.2} \newline
                  दे॒वाने॒व तैर्यज॑मानः प्रीणाति ॥ आज्ये॑न जुहोति । अ॒ग्नेर्वा ए॒तद्रू॒पम् । यदाज्य᳚म् । यदाज्ये॑न जु॒होति॑ । अ॒ग्निमे॒व तत्प्री॑णाति ॥ मधु॑ना जुहोति । म॒ह॒त्यै वा ए॒तद्दे॒वता॑यै रू॒पम् । यन्मधु॑ । यन्मधु॑ना जु॒होति॑ \textbf{ 51} \newline
                  \newline
                                \textbf{ TB 3.8.14.3} \newline
                  म॒ह॒तीमे॒व तद्दे॒वतां᳚ प्रीणाति । त॒ण्डु॒लै-र्जु॑होति । वसू॑नां॒ ॅवा ए॒तद्रू॒पम् । यत्त॑ण्डु॒लाः । यत्त॑ण्डु॒लै-र्जु॒होति॑ । वसू॑ने॒व तत्प्री॑णाति । पृथु॑कै-र्जु॒होति॑ (र्जुहोति) । रु॒द्राणां॒ ॅवा ए॒तद्रू॒पम् । यत्पृथु॑काः । यत्पृथु॑कै-र्जु॒होति॑ \textbf{ 52} \newline
                  \newline
                                \textbf{ TB 3.8.14.4} \newline
                  रु॒द्राने॒व तत्प्री॑णाति । ला॒जैर्जु॑होति । आ॒दि॒त्यानां॒ ॅवा ए॒तद्रू॒पम् । यल्ला॒जाः । यल्ला॒जै-र्जु॒होति॑ । आ॒दि॒त्याने॒व तत्प्री॑णाति । क॒रम्बै᳚-र्जुहोति । विश्वे॑षां॒ ॅवा ए॒तद्दे॒वानाꣳ॑ रू॒पम् । यत्क॒रम्बाः᳚ । यत्क॒रम्बै᳚-र्जु॒होति॑ \textbf{ 53} \newline
                  \newline
                                \textbf{ TB 3.8.14.5} \newline
                  विश्वा॑ने॒व तद्दे॒वान् प्री॑णाति । धा॒नाभि॑-र्जुहोति । नक्ष॑त्राणां॒ ॅवा ए॒तद्रू॒पम् । यद्धा॒नाः । यद्धा॒नाभि॑-र्जु॒होति॑ । नक्ष॑त्राण्ये॒व तत्प्री॑णाति । सक्तु॑भि-र्जुहोति । प्र॒जाप॑ते॒र्वा ए॒तद्रू॒पम् । यथ्सक्त॑वः । यथ्सक्तु॑भि-र्जु॒होति॑ \textbf{ 54} \newline
                  \newline
                                \textbf{ TB 3.8.14.6} \newline
                  प्र॒जाप॑तिमे॒व तत्प्री॑णाति । म॒सूस्यै᳚र्जुहोति । सर्वा॑सां॒ ॅवा ए॒तद्दे॒वता॑नाꣳ रू॒पम् । यन्म॒सूस्या॑नि । यन्म॒सूस्यै᳚-र्जु॒होति॑ । सर्वा॑ ए॒व तद्दे॒वताः᳚ प्रीणाति । प्रि॒य॒ङ्गु॒त॒ण्डु॒लै-र्जु॑होति । प्रि॒याङ्गा॑ ह॒ वै नामै॒ते । ए॒तैर्वै दे॒वा अश्व॒स्याङ्गा॑नि॒ सम॑दधुः । यत्प्रि॑यङ्गुतण्डु॒लै-र्जु॒होति॑ ( ) । अश्व॑स्यै॒वाङ्गा॑नि॒ संद॑धाति ॥ दशान्ना॑नि जुहोति । दशा᳚क्षरा वि॒राट् । वि॒राट्-कृ॒थ्स्नस्या॒-न्नाद्य॒स्या-व॑रुद्ध्यै । \textbf{ 55} \newline
                  \newline
                                                        \textbf{special korvai} \newline
              (अ॒न्न॒हो॒मानाज्ये॑ना॒ग्नेर्मधु॑ना तण्डु॒लैः पृथु॑कैर्ला॒जैः क॒रम्बै᳚द्र्धा॒नाभिः॒ सक्तु॑भिर्म॒सूस्यैः᳚ प्रियङ्गुतण्डु॒लैर् दशान्ना॑नि॒ द्वाद॑श) \newline
                                (जु॒होति॒ - मधु॑ना जु॒होति॒ - पृथु॑कैर्जु॒होति॑ - क॒रम्बै᳚र्जु॒होति॒ - सक्तु॑भिर्जु॒होति॑ - प्रियङ्गुतण्डु॒लैर्जु॒होति॑ च॒त्वारि॑ च) \textbf{(A14)} \newline \newline
                \textbf{ 3.8.15    अनुवाकं   15 -तत्प्रकारविशेष} \newline
                                \textbf{ TB 3.8.15.1} \newline
                  प्र॒जाप॑ति-रश्वमे॒धम॑सृजत । तꣳ सृ॒ष्टꣳ रक्षाꣳ॑स्य जिघाꣳसन्न् । स ए॒तान् प्र॒जाप॑तिर्नक्तꣳ हो॒मान॑-पश्यत् । तान॑ जुहोत् । तैर्वै स य॒ज्ञाद् रक्षाꣳ॒॒स्यपा॑हन्न् । यन्न॑क्तꣳ हो॒माञ्जु॒होति॑ । य॒ज्ञादे॒व तैर्यज॑मानो॒ रक्षाꣳ॒॒स्यप॑हन्ति ॥ आज्ये॑न जुहोति । वज्रो॒ वा आज्य᳚म् । वज्रे॑णै॒व य॒ज्ञाद् रक्षाꣳ॒॒स्यप॑हन्ति । \textbf{ 56} \newline
                  \newline
                                \textbf{ TB 3.8.15.2} \newline
                  आज्य॑स्य प्रति॒पदं॑ करोति । प्रा॒णो वा आज्य᳚म् । मु॒ख॒त ए॒वास्य॑ प्रा॒णं द॑धाति ॥ अ॒न्न॒हो॒मा-ञ्जु॑होति । शरी॑रवदे॒वाव॑रुन्धे ॥ व्य॒त्यासं॑ जुहोति । उ॒भय॒स्या व॑रुद्ध्यै ॥ नक्तं॑ जुहोति । रक्ष॑सा॒मप॑हत्यै ॥ आज्ये॑नान् त॒तो जु॑होति \textbf{ 57} \newline
                  \newline
                                \textbf{ TB 3.8.15.3} \newline
                  प्रा॒णो वा आज्य᳚म् । उ॒भ॒यत॑ ए॒वास्य॑ प्रा॒णं द॑धाति । पु॒रस्ता᳚-च्चो॒परि॑ष्टाच्च ॥ एक॑स्मै॒ स्वाहेत्या॑ह । अ॒स्मिन्ने॒व लो॒के प्रति॑तिष्ठति । द्वाभ्याꣳ॒॒ स्वाहेत्या॑ह । अ॒मुष्मि॑न्ने॒व लो॒के प्रति॑तिष्ठति । उ॒भयो॑रे॒व लो॒कयाः॒ प्रति॑तिष्ठति । अ॒स्मिꣳश्चा॒-मुष्मिꣳ॑श्च ॥ श॒ताय॒ स्वाहेत्या॑ह ( ) । श॒तायु॒र्वै पुरु॑षः श॒तवी᳚र्यः । आयु॑रे॒व वी॒र्य॑मव॑रुन्धे । स॒हस्रा॑य॒ स्वाहेत्या॑ह । आयु॒र्वै स॒हस्र᳚म् । आयु॑रे॒वा-व॑रुन्धे ॥ सर्व॑स्मै॒ स्वाहेत्या॑ह । अप॑रिमितमे॒वा व॑रुन्धे । \textbf{ 58} \newline
                  \newline
                                    (ए॒व य॒ज्ञ्द् रक्षाꣳ॒॒स्यप॑हन् - त्यन्त॒तो जु॑होति - श॒ताय॒ स्वाहेत्या॑ह सप्त च॑) \textbf{(A15)} \newline \newline
                \textbf{ 3.8.16    अनुवाकं   16 -विवरणमेतयोः} \newline
                                \textbf{ TB 3.8.16.1} \newline
                  प्र॒जाप॑तिं॒ ॅवा ए॒ष ई᳚फ्स॒तीत्या॑हुः । यो᳚ऽश्वमे॒धेन॒ यज॑त॒ इति॑ । अथो॑ आहुः । सर्वा॑णि भू॒तानीति॑ ॥ एक॑स्मै॒ स्वाहेत्या॑ह । प्र॒जाप॑ति॒र्वा एकः॑ । तमे॒वाप्नो॑ति ॥ एक॑स्मै॒ स्वाहा॒ द्वाभ्याꣳ॒॒ स्वाहेत्य॑भि-पू॒र्वमाहु॑ती-र्जुहोति । अ॒भि॒पू॒र्वमे॒व स॑व॒र्गं ॅलो॒कमे॑ति ॥ ए॒को॒त्त॒रं जु॑होति \textbf{ 59} \newline
                  \newline
                                \textbf{ TB 3.8.16.2} \newline
                  ए॒क॒वदे॒व सु॑व॒र्गं ॅलो॒कमे॑ति ॥ सन्त॑तं जुहोति । सु॒व॒र्गस्य॑ लो॒कस्य॒ सन्त॑त्यै ॥ श॒ताय॒ स्वाहेत्या॑ह । श॒तायु॒र्वै पुरु॑षः श॒तवी᳚र्यः । आयु॑रे॒व वी॒र्य॑मव॑रुन्धे । स॒हस्रा॑य॒ स्वाहेत्या॑ह । आयु॒र्वै स॒हस्र᳚म् । आयु॑रे॒वा व॑रुन्धे ॥ अ॒युता॑य॒ स्वाहा॑ नि॒युता॑य॒ स्वाहा᳚ प्र॒युता॑य॒ स्वाहेत्या॑ह \textbf{ 60} \newline
                  \newline
                                \textbf{ TB 3.8.16.3} \newline
                  त्रय॑ इ॒मे लो॒काः । इ॒माने॒व लो॒कानव॑रुन्धे ॥ अर्बु॑दाय॒ स्वाहेत्या॑ह । वाग्वा अर्बु॑दम् । वाच॑मे॒वा व॑रुन्धे ॥ न्य॑र्बुदाय॒ स्वाहेत्या॑ह । यो वै वा॒चो भू॒मा । तन्न्य॑र्बुदम् । वा॒च ए॒व भू॒मान॒मव॑रुन्धे ॥ स॒मु॒द्राय॒ स्वाहेत्या॑ह \textbf{ 61} \newline
                  \newline
                                \textbf{ TB 3.8.16.4} \newline
                  स॒मु॒द्रमे॒वाप्नो॑ति । मद्ध्या॑य॒ स्वाहेत्या॑ह । मद्ध्य॑मे॒वाप्नो॑ति । अन्ता॑य॒ स्वाहेत्या॑ह । अन्त॑मे॒-वाप्नो॑ति । प॒रा॒द्र्धाय॒ स्वाहेत्या॑ह । प॒रा॒द्र्धमे॒वाप्नो॑ति ॥ उ॒षसे॒ स्वाहा॒ व्यु॑ष्ट्यै॒ स्वाहेत्या॑ह । रात्रि॒र्वा उ॒षाः । अह॒र्व्यु॑ष्टिः ( ) । अ॒हो॒रा॒त्रे ए॒वाव॑रुन्धे । अथो॑ अहोरा॒त्रयो॑रे॒व प्रति॑तिष्ठति ॥ ता यदु॒भयी॒र्दिवा॑ वा॒ नक्तं॑ ॅवा जुहु॒यात् । अ॒हो॒रा॒त्रे मो॑हयेत् । उ॒षसे॒ स्वाहा॒ व्यु॑ष्ट्यै॒ स्वाहो॑देष्य॒ते स्वाहो᳚द्य॒ते स्वाहेत्यनु॑दिते जुहोति । उदि॑ताय॒ स्वाहा॑ सुव॒र्गाय॒ स्वाहा॑ लो॒काय॒ स्वाहेत्युदि॑ते जुहोति । अ॒हो॒रा॒त्रयो॒-रव्य॑तिमोहाय । \textbf{ 62} \newline
                  \newline
                                    (ए॒को॒त्त॒रम् जु॑होति - प्र॒युता॑य॒ स्वाहेत्या॑ह - समु॒द्राय॒ स्वाहेत्या॒हाह॒र् - व्यु॑ष्टिः स॒प्त च॑) \textbf{(A16)} \newline \newline
                \textbf{ 3.8.17    अनुवाकं   17 -सप्तमकाण्डगतान्नहोमानुवाका व्याख्यायन्ते} \newline
                                \textbf{ TB 3.8.17.1} \newline
                  वि॒भूर्मा॒त्रा प्र॒भूः पि॒त्रेत्य॑श्व-ना॒मानि॑ जुहोति । उ॒भयो॑रे॒वैनं॑ ॅलो॒कयो᳚-र्नाम॒धेयं॑ गमयति ॥ आय॑नाय॒ स्वाहा॒ प्राय॑णाय॒ स्वाहेत्यु॑द्-द्रा॒वाञ्जु॑होति । सर्व॑मे॒वैन॒-मस्क॑न्नꣳ सुव॒र्गं ॅलो॒कं ग॑मयति ॥ अ॒ग्नये॒ स्वाहा॒ सोमा॑य॒ स्वाहेति॑ पूर्वहा॒मा-ञ्जु॑होति । पूर्व॑ ए॒व द्वि॒षन्तं॒ भ्रातृ॑व्य॒-मति॑क्रामति ॥ पृ॒थि॒व्यै स्वाहा॒ऽन्तरि॑क्षाय॒ स्वाहेत्या॑ह । य॒था॒ य॒जुरे॒वैतत् ॥ अ॒ग्नये॒ स्वाहा॒ सोमा॑य॒ स्वाहेति॑ पूर्वदी॒क्षा जु॑होति । पूर्व॑ ए॒व द्वि॒षन्तं॒ भ्रातृ॑व्य॒-मति॑क्रामति । \textbf{ 63} \newline
                  \newline
                                \textbf{ TB 3.8.17.2} \newline
                  पृ॒थि॒व्यै स्वाहा॒ऽन्तरि॑क्षाय॒ स्वाहेत्ये॑-कविꣳ॒॒शिनीं᳚ दी॒क्षां जु॑होति । एक॑विꣳशति॒र्वै दे॑वलो॒काः । द्वाद॑श॒ मासाः॒ पञ्च॒र्तवः॑ । त्रय॑ इ॒मे लो॒काः । अ॒सावा॑दि॒त्य ए॑कविꣳ॒॒शः । ए॒ष सु॑व॒र्गो लो॒कः । सु॒व॒र्गस्य॑ लो॒कस्य॒ सम॑ष्ट्यै ॥ भुवो॑ दे॒वानां॒ कर्म॒णेत्यृ॑तु-दी॒क्षा जु॑होति । ऋ॒तूने॒-वास्मै॑ कल्पयति ॥ अ॒ग्नये॒ स्वाहा॑ वा॒यवे॒ स्वाहेति॑ जुहो॒त्य-न॑न्तरित्यै । \textbf{ 64} \newline
                  \newline
                                \textbf{ TB 3.8.17.3} \newline
                  अ॒र्वाङ्य॒ज्ञ्ः संक्रा॑म॒त्वि-त्याप्ती᳚-र्जुहोति । सु॒व॒र्गस्य॑ लो॒कस्याप्त्यै᳚ ॥ भू॒तं भव्यं॑ भवि॒ष्यदिति॒ पर्या᳚प्ती-र्जुहोति । सु॒व॒र्गस्य॑ लो॒कस्य॒ पर्या᳚प्त्यै ॥ आ मे॑ गृ॒हा भ॑व॒न्त्वित्या॒भू-र्जु॑होति । सु॒व॒र्गस्य॑ लो॒कस्या भू᳚त्यै ॥ अ॒ग्निना॒ तपो-ऽन्व॑भव॒दित्य॑नु॒भू-र्जु॑होति । सु॒व॒र्गस्य॑ लो॒कस्यानु॑भूत्यै ॥ स्वाहा॒-ऽऽधिमाधी॑ताय॒ स्वाहेति॒ सम॑स्तानि वैश्वदे॒वानि॑ जुहोति । सम॑स्तमे॒व द्वि॒षन्तं॒ भ्रातृ॑व्य॒-मति॑क्रामति । \textbf{ 65} \newline
                  \newline
                                \textbf{ TB 3.8.17.4} \newline
                  द॒द्भ्यः स्वाहा॒ हनू᳚भ्याꣳ॒॒ स्वाहेत्य॑ङ्गहो॒मा-ञ्जु॑होति । अङ्गे॑ अङ्गे॒ वै पुरु॑षस्य पा॒प्मोप॑श्लिष्टः । अङ्गा॑दङ्गा दे॒वैनं॑ पा॒प्मन॒स्तेन॑ मुञ्चति ॥ अ॒ञ्ज्ये॒ताय॒ स्वाहा॑ कृ॒ष्णाय॒ स्वाहा᳚ श्वे॒ताय॒ स्वाहेत्य॑-श्वरू॒पाणि॑ जुहोति । रू॒पैरे॒वैनꣳ॒॒ सम॑द्र्धयति ॥ ओष॑धीभ्यः॒ स्वाहा॒ मूले᳚भ्यः॒ स्वाहेत्यो॑षधि हो॒माञ्जु॑होति । द्व॒य्यो वा ओष॑धयः । पुष्पे᳚भ्यो॒ऽन्याः फलं॑ गृ॒ह्णन्ति॑ । मूले᳚भ्यो॒ऽन्याः । ता ए॒वोभयी॒रव॑रुन्धे । \textbf{ 66} \newline
                  \newline
                                \textbf{ TB 3.8.17.5} \newline
                  वन॒स्पति॑भ्यः॒ स्वाहेति॑ वनस्पति-हो॒माञ्जु॑होति । आ॒र॒ण्यस्या॒न्नाद्य॒स्याव॑रुद्ध्यै ॥ मे॒षस्त्वा॑ पच॒तै-र॑व॒त्वित्य-पा᳚व्यानि जुहोति । प्रा॒णा वै दे॒वा अपा᳚व्याः । प्रा॒णाने॒वाव॑रुन्धे ॥ कूप्या᳚भ्यः॒ स्वाहा॒ऽद्भ्यः स्वाहेत्य॒पाꣳ होमा᳚ञ्जुहोति । अ॒फ्सु वा आपः॑ । अन्नं॒ ॅवा आपः॑ । अ॒द्भ्यो वा अन्नं॑ जायते । यदे॒वाद्भ्योऽन्नं॒ जाय॑ते ( ) । तदव॑रुन्धे । \textbf{ 67} \newline
                  \newline
                                    (पू॒र्व॒दी॒क्षा जु॑होति॒ पूर्व॑ ए॒व द्वि॒षन्त॒म् भ्रातृ॑व्य॒मति॑क्राम॒ - त्यन॑न्तरित्यै - क्रामति - रुन्धे॒ - जाय॑त॒ एकं॑ च) \textbf{(A17)} \newline \newline
                \textbf{ 3.8.18    अनुवाकं   18 -सप्तमकाण्डगतान्नहोमानुवाका व्याख्यायन्ते} \newline
                                \textbf{ TB 3.8.18.1} \newline
                  अम्भाꣳ॑सि जुहोति । अ॒यं ॅवै लो॒को-ऽम्भाꣳ॑सि । तस्य॒ वस॒वो-ऽधि॑पतयः । अ॒ग्निर्ज्योतिः॑ । यदम्भाꣳ॑सि जु॒होति॑ । इ॒ममे॒व लो॒कमव॑रुन्धे । वसू॑नाꣳ॒॒ सायु॑ज्यं गच्छति । अ॒ग्निं ज्योति॒रव॑रुन्धे ॥ नभाꣳ॑सि जुहोति । अ॒न्तरि॑क्षं ॅवै नभाꣳ॑सि \textbf{ 68} \newline
                  \newline
                                \textbf{ TB 3.8.18.2} \newline
                  तस्य॑ रु॒द्रा अधि॑पतयः । वा॒युर्ज्योतिः॑ । यन्नभाꣳ॑सि जु॒होति॑ । अ॒न्तरि॑क्षमे॒वा-व॑रुन्धे । रु॒द्राणाꣳ॒॒ सायु॑ज्यं गच्छति । वा॒युं ज्योति॒रव॑रुन्धे । महाꣳ॑सि जुहोति । अ॒सौ वै लो॒को महाꣳ॑सि । तस्या॑दि॒त्या अधि॑पतयः। सूर्यो॒ ज्योतिः॑ \textbf{ 69} \newline
                  \newline
                                \textbf{ TB 3.8.18.3} \newline
                  यन्महाꣳ॑सि जु॒होति॑ । अ॒मुमे॒व लो॒कमव॑रुन्धे । आ॒दि॒त्यानाꣳ॒॒ सायु॑ज्यं गच्छति । सूर्यं॒ ज्योति॒रव॑रुन्धे ॥ नमो॒ राज्ञ्॒ नमो॒ वरु॑णा॒येति॑ य॒व्यानि॑ जुहोति । अ॒न्नाद्य॒स्याव॑रुद्ध्यै ॥ म॒यो॒भूर्वातो॑ अ॒भि वा॑तू॒स्रा इति॑ ग॒व्यानि॑ जुहोति । प॒शू॒नामव॑रुद्ध्यै ॥ प्रा॒णाय॒ स्वाहा᳚ व्या॒नाय॒ स्वाहेति॑ सन्तति-हो॒माञ्जु॑होति । सु॒व॒र्गस्य॑ लो॒कस्य॒ सन्त॑त्यै । \textbf{ 70} \newline
                  \newline
                                \textbf{ TB 3.8.18.4} \newline
                  सि॒ताय॒ स्वाहा-ऽसि॑ताय॒ स्वाहेति॒ प्रमु॑क्ती-र्जुहोति । सु॒व॒र्गस्य॑ लो॒कस्य॒ प्रमु॑क्त्यै ॥ पृ॒थि॒व्यै स्वाहा॒-ऽन्तरि॑क्षाय॒ स्वाहेत्या॑ह । य॒था॒ य॒जुरे॒वैतत् ॥ द॒त्त्वते॒ स्वाहा॑-ऽद॒न्तका॑य॒ स्वाहेति॑ शरीर-हो॒माञ्जु॑होति । पि॒तृ॒लो॒कमे॒व तैर्यज॑मा॒नो-ऽव॑रुन्धे ॥ कस्त्वा॑ युनक्ति॒ स त्वा॑ युन॒क्त्विति॑ परि॒धीन्. यु॑नक्ति । इ॒मे वै लो॒काः प॑रि॒धयः॑ । इ॒माने॒वास्मै॑ लो॒कान्. यु॑नक्ति । सु॒व॒र्गस्य॑ लो॒कस्य॒ सम॑ष्ट्यै । \textbf{ 71} \newline
                  \newline
                                \textbf{ TB 3.8.18.5} \newline
                  यः प्रा॑ण॒तो य आ᳚त्म॒दा इति॑ महि॒मानौ॑ जुहोति । सु॒व॒र्गो वै लो॒को महः॑ । सु॒व॒र्गमे॒व ताभ्यां᳚ ॅलो॒कं ॅयज॑मा॒नोऽव॑रुन्धे ॥ आ ब्रह्म॑न् ब्राह्म॒णो ब्र॑ह्मवच॒र्सी जा॑यता॒मिति॒ सम॑स्तानि ब्रह्मवर्च॒सानि॑ जुहोति । ब्र॒ह्म॒व॒र्च॒समे॒व तैर्यज॑मा॒नो ऽव॑रुन्धे ॥ जज्ञि॒ बीज॒मिति॑ जुहो॒त्य-न॑न्तरित्यै ॥ अ॒ग्नये॒ सम॑नम-त्पृथि॒व्यै समन॑म॒दिति॑ सन्नति हो॒माञ्जु॑होति । सु॒व॒र्गस्य॑ लो॒कस्य॒ संन॑त्यै ॥ भू॒ताय॒ स्वाहा॑ भविष्य॒ते स्वाहेति॑ भूताभ॒व्यौ होमौ॑ जुहोति । अ॒यं ॅवै लो॒को भू॒तम् \textbf{ 72} \newline
                  \newline
                                \textbf{ TB 3.8.18.6} \newline
                  असौ भ॑वि॒ष्यत् । अ॒नयो॑रे॒व लो॒कयोः॒ प्रति॑तिष्ठति । सर्व॒स्याप्त्यै᳚ । सर्व॒स्याव॑रुद्ध्यै ॥ यदक्र॑न्दः प्रथ॒मं जाय॑मान॒ इत्य॑श्वस्तो॒मीयं॑ जुहोति । सर्व॒स्याप्त्यै᳚ । सर्व॑स्य॒ जित्यै᳚ ॥ सर्व॑मे॒व तेना᳚प्नोति । सर्वं॑ जयति । यो᳚ऽश्वमे॒धेन॒ यज॑ते ( ) \textbf{ 73} \newline
                  \newline
                                \textbf{ TB 3.8.18.7} \newline
                  य उ॑ चैनमे॒वं ॅवेद॑ ॥ य॒ज्ञ्ꣳ रक्षाꣳ॑स्य जिघाꣳसन्न् । स ए॒तान् प्र॒जाप॑ति-र्नक्तꣳ हो॒मान॑पश्यत् । तान॑जुहोत् । तैर्वै स य॒ज्ञाद्रक्षाꣳ॒॒स्यपा॑हन्न् । यन्न॑क्तꣳ हो॒माञ्जु॒होति॑ । य॒ज्ञादे॒व तैर्यज॑मानो॒ रक्षाꣳ॒॒स्यप॑हन्ति ॥ उ॒षसे॒ स्वाहा॒ व्यु॑ष्ट्यै॒ स्वाहेत्य॑न्त॒तो जु॑होति । सु॒व॒र्गस्य॑ लो॒कस्य॒ सम॑ष्ट्यै । \textbf{ 74} \newline
                  \newline
                                    (वै नभाꣳ॑सि॒ - सूर्यो॒ ज्योतिः॒ - सन्त॑त्यै॒ - सम॑ष्ट्यै - भु॒तम् - ॅयज॑ते॒ - +नव॑ च) \textbf{(A18)} \newline \newline
                \textbf{ 3.8.19    अनुवाकं   19 -औपसथ्यदिने यूपप्रयोगाः} \newline
                                \textbf{ TB 3.8.19.1} \newline
                  ए॒क॒यू॒पो वै॑काद॒शिनी॑ वा । अ॒न्येषां᳚ ॅय॒ज्ञानां॒ ॅयूपा॑ भवन्ति । ए॒क॒वि॒॒शिन्य॑-श्वमे॒धस्य॑ । सु॒व॒र्गस्य॑ लो॒कस्या॒-भिजि॑त्यै ॥ बै॒ल॒.वो वा॑ खादि॒रो वा॑ पाला॒शो वा᳚ । अ॒न्येषां᳚ ॅयज्ञ्क्रतू॒नां ॅयूपा॑ भवन्ति । राज्जु॑दाल॒ एक॑विꣳशत्य-रत्निरश्वमे॒धस्य॑ । सु॒व॒र्गस्य॑ लो॒कस्य॒ सम॑ष्ट्यै ॥ नान्येषां᳚ पशू॒नां ते॑ज॒न्या अ॑व॒द्यन्ति॑ । अव॑द्य॒न्त्यश्व॑स्य \textbf{ 75} \newline
                  \newline
                                \textbf{ TB 3.8.19.2} \newline
                  पा॒प्मा वै ते॑ज॒नी । पा॒प्मनोऽप॑हत्यै ॥ प्ल॒क्ष॒शा॒खाया॑म॒न्येषां᳚ पशू॒नाम॑व॒द्यन्ति॑ । वे॒त॒स॒-शा॒खाया॒-मश्व॑स्य । अ॒फ्सुयो॑नि॒र्वा अश्वः॑ । अ॒फ्सु॒जो वे॑त॒सः । स्व ए॒वास्य॒ योना॒व-व॑द्यति ॥ यूपे॑षु ग्रा॒म्यान् प॒शून्नि॑-यु॒ञ्जन्ति॑ । आ॒रो॒केष्वा॑-र॒ण्यान्धा॑रयन्ति । प॒शू॒नां ॅव्यावृ॑त्त्यै ( ) ॥ आ ग्रा॒म्यान् प॒शून्-ॅलभ॑न्ते । प्रार॒ण्यान्थ्सृ॑जन्ति । पा॒प्मनो-ऽप॑हत्यै । \textbf{ 76} \newline
                  \newline
                                    (अश्व॑स्य॒ - व्यावृ॑त्त्य॒ त्रीणि॑ च) \textbf{(A19)} \newline \newline
                \textbf{ 3.8.20    अनुवाकं   20 -यूपानां स्थानादयः} \newline
                                \textbf{ TB 3.8.20.1} \newline
                  राज्जु॑दाल-मग्नि॒ष्ठं मि॑नोति । भ्रू॒ण॒ह॒त्याया॒ अप॑हत्यै ॥ पौतु॑द्रवाव॒भितो॑ भवतः । पुण्य॑स्य ग॒न्धस्याव॑रुद्ध्यै । भ्रू॒ण॒ह॒त्या-मे॒वास्मा॑-दप॒हत्य॑ । पुण्ये॑न ग॒न्धेनो॑भ॒यतः॒ परि॑गृह्णाति ॥ षड्बै॒॒ल्॒.वा भ॑वन्ति । ब्र॒ह्म॒व॒र्च॒सस्या-व॑रुद्ध्यै ॥ षट्खा॑दि॒राः । तेज॒सो-ऽव॑रुद्ध्यै । \textbf{ 77} \newline
                  \newline
                                \textbf{ TB 3.8.20.2} \newline
                  षट्पा॑ला॒शाः । सो॒म॒पी॒थस्या-व॑रुद्ध्यै ॥ एक॑विꣳशतिः॒ संप॑द्यन्ते । एक॑विꣳशति॒र्वै दे॑वलो॒काः । द्वाद॑श॒ मासाः॒ पञ्च॒र्तवः॑ । त्रय॑ इ॒मे लो॒काः । अ॒सावा॑दि॒त्य ए॑कविꣳ॒॒शः । ए॒ष सु॑व॒र्गो लो॒कः । सु॒व॒र्गस्य॑ लो॒कस्य॒ सम॑ष्ट्यै ॥ श॒तं प॒शवो॑ भवन्ति \textbf{ 78} \newline
                  \newline
                                \textbf{ TB 3.8.20.3} \newline
                  श॒तायुः॒ पुरु॑षः श॒तेन्द्रि॑यः । आयु॑ष्ये॒वेन्द्रि॒ये प्रति॑तिष्ठति ॥ सर्वं॒ ॅवा अ॑श्वमे॒द्ध्याप्नो॑ति । अप॑रिमिता भवन्ति । अप॑रिमित॒स्या-व॑रुद्ध्यै ॥ ब्र॒ह्म॒वा॒दिनो॑ वदन्ति । कस्मा᳚थ्स॒त्यात् । द॒क्षि॒ण॒तो᳚-ऽन्येषां᳚ पशू॒नाम॑व॒द्यन्ति॑ । उ॒त्त॒र॒तो-ऽश्व॒स्येति॑ । वा॒रु॒णो वा अश्वः॑ \textbf{ 79} \newline
                  \newline
                                \textbf{ TB 3.8.20.4} \newline
                  ए॒षा वै वरु॑णस्य॒ दिक् । स्वाया॑मे॒वास्य॑ दि॒श्यव॑द्यति । यदित॑रेषां पशू॒नाम॑व॒द्यति॑ । श॒त॒दे॒व॒त्यं॑ तेनाव॑रुन्धे ॥ चि॒ते᳚-ऽग्ना-वधि॑वैत॒से कटेऽश्वं॑ चिनोति । अ॒फ्सु-यो॑नि॒र्वा अश्वः॑ । अ॒फ्सु॒जो वे॑त॒सः । स्व ए॒वैनं॒ ॅयोनौ॒ प्रति॑ष्ठापयति ॥ पु॒रस्ता᳚-त्प्र॒त्यञ्चं॑ तूप॒रं चि॑नोति । प॒श्चात्प्रा॒चीनं॑ गोमृ॒गम् \textbf{ 80} \newline
                  \newline
                                \textbf{ TB 3.8.20.5} \newline
                  प्रा॒णा॒पा॒ना-वे॒वास्मिन᳚ थ्स॒म्यञ्चौ॑ दधाति ॥ अश्वं॑ तूप॒रं गो॑मृ॒गमिति॑ सर्व॒हुत॑ ए॒ताञ्जु॑होति । ए॒षां ॅलो॒काना॑-म॒भिजि॑त्यै ॥ आ॒त्मना॒-ऽभिजु॑होति । सात्मा॑नमे॒वैनꣳ॒॒ सत॑नुं करोति ॥ सात्मा॒ऽमुष्मि॑न् ॅला॒के भ॑वति । य ए॒वं ॅवेद॑ । अथो॒ वसो॑रे॒व धारां॒ तेनाव॑रुन्धे ॥ इ॒लु॒वर्दा॑य॒ स्वाहा॑ बलि॒वर्दा॑य॒ स्वाहेत्या॑ह । स॒म्ॅव॒थ्स॒रो वा इ॑लु॒वर्दः॑ ( ) । प॒रि॒व॒थ्स॒रो ब॑लि॒वर्दः॑ । स॒म्ॅव॒थ्स॒रादे॒व प॑रिवथ्स॒रादायु॒र व॑रुन्धे । आयु॑रे॒वास्मि॑न्-दधाति । तस्मा॑दश्व-मेधया॒जी ज॒रसा॑ वि॒स्रसा॒ऽमुं ॅलो॒कमे॑ति । \textbf{ 81} \newline
                  \newline
                                    (तेज॒सोऽव॑रुद्धै - भव॒न् - त्यश्वो॑ - गोमृ॒ग - मि॑लु॒वर्द॑ श्च॒त्वारि॑ च) \textbf{(A20)} \newline \newline
                \textbf{ 3.8.21    अनुवाकं   21 -चेतव्याग्नयादेर्विशेषः} \newline
                                \textbf{ TB 3.8.21.1} \newline
                  ए॒क॒विꣳ॒॒शो᳚ऽग्निर्भ॑वति । ए॒क॒विꣳ॒॒शः स्तोमः॑ । एक॑विꣳशति॒र्यूपाः᳚ ॥ यथा॒ वा अश्वा॑ वर्.र्ष॒भा वा॒ वृषा॑णः सꣳस्फु॒रेरन्न्॑ । ए॒वमे॒तथ्स्तोमाः॒ सꣳस्फु॑रन्ते । यदे॑कविꣳ॒॒शाः । ते यथ्स॑मृ॒च्छेरन्न्॑ । ह॒न्येता᳚स्य य॒ज्ञ्ः ॥ द्वा॒द॒श ए॒वाग्निः स्या॒दित्या॑हुः । द्वा॒द॒शः स्तोमः॑ \textbf{ 82} \newline
                  \newline
                                \textbf{ TB 3.8.21.2} \newline
                  एका॑दश॒ यूपाः᳚ ॥ यद्द्वा॑द॒शो᳚-ऽग्निर्भव॑ति । द्वाद॑श॒ मासाः᳚ सम्ॅवथ्स॒रः । स॒म्ॅव॒थ्स॒रेणै॒वास्मा॒ अन्न॒मव॑रुन्धे ॥ यद्दश॒ यूपा॒ भव॑न्ति । दशा᳚क्षरा वि॒राट् । अन्नं॑ ॅवि॒राट् । वि॒राजै॒वान्नाद्य॒-मव॑रुन्धे । य ए॑काद॒शः । स्तन॑ ए॒वास्यै॒ सः \textbf{ 83} \newline
                  \newline
                                \textbf{ TB 3.8.21.3} \newline
                  दु॒ह ए॒वैनां॒ तेन॑ ॥ तदा॑हुः । यद्द्वा॑द॒शो᳚ऽग्निः स्या᳚द्द्वाद॒शः स्तोम॒ एका॑दश॒ यूपाः᳚ । यथा॒ स्थूरि॑णा या॒यात् । ता॒दृक्तत् ॥ ए॒क॒विꣳ॒॒श ए॒वाग्निः स्या॒दित्या॑हुः । ए॒क॒विꣳ॒॒शः स्तोमः॑ । एक॑विꣳशति॒र्यूपाः᳚ । यथा॒ प्रष्टि॑भि॒र्याति॑ । ता॒दृगे॒व तत् । \textbf{ 84} \newline
                  \newline
                                \textbf{ TB 3.8.21.4} \newline
                  यो वा अ॑श्वमे॒धे ति॒स्रः क॒कुभो॒ वेद॑ । क॒कुद्ध॒ राज्ञां᳚ भवति । ए॒क॒विꣳ॒॒शो᳚-ऽग्निर्भ॑वति । ए॒क॒विꣳ॒॒शः स्तोमः॑ । एक॑विꣳशति॒र्यूपाः᳚ । ए॒ता वा अ॑श्वम॒धे ति॒स्रः क॒कुभः॑ । य ए॒वं ॅवेद॑ । क॒कुद्ध॒ राज्ञां᳚ भवति ॥ यो वा अ॑श्वमे॒धे त्रीणि॑ शी॒र्॒.षाणि॒ वेद॑ । शिरो॑ ह॒ राज्ञां᳚ भवति ( ) । ए॒क॒विꣳ॒॒शो᳚-ऽग्निर्भ॑वति । ए॒क॒विꣳ॒॒शः स्तोमः॑ । एक॑विꣳशति॒र्यूपाः᳚ । ए॒तानि॒ वा अ॑श्वमे॒धे त्रीणि॑ शी॒र॒.षाणि॑ । य ए॒वं ॅवेद॑ । शिरो॑ ह॒ राज्ञां᳚ भवति । \textbf{ 85} \newline
                  \newline
                                    (द्वा॒द॒शः स्तोमः॒-स- ए॒व तच् - छिरो॑ ह॒ राज्ञां᳚ भवति॒ षट्च॑) \textbf{(A21)} \newline \newline
                \textbf{ 3.8.22    अनुवाकं   22 -उक्थ्याख्ये द्वितीयेऽहनि बहिष्पवमाने अश्वस्योद्रातृत्वम्} \newline
                                \textbf{ TB 3.8.22.1} \newline
                  दे॒वा वा अ॑श्वमे॒धे पव॑माने । सु॒व॒र्गं ॅलो॒कं न प्राजा॑नन्न् । तमश्वः॒ प्राजा॑नात् । यद॑श्वम॒धे-ऽश्वे॑न॒ मेद्ध्ये॒नोद॑ञ्चो बहिष्पवमा॒नꣳ सर्प॑न्ति । सु॒व॒र्गस्य॑ लो॒कस्य॒ प्रज्ञा᳚त्यै । न वै म॑नु॒ष्यः॑ सुव॒र्गं लो॒कमञ्ज॑सा वेद । अश्वो॒ वै सु॑व॒र्गं ॅलो॒कमञ्ज॑सा वेद ॥ यदु॑द्गा॒-तोद्गाये᳚त् । यथाऽक्षे᳚त्रज्ञो॒ऽन्येन॑ प॒था प्र॑तिपा॒दये᳚त् । ता॒दृक्तत् \textbf{ 86} \newline
                  \newline
                                \textbf{ TB 3.8.22.2} \newline
                  उ॒द्गा॒तार॑-मप॒रुद्ध्य॑ । अश्व॑-मुद्गी॒थाय॑ वृणीते । यथा᳚ क्षेत्र॒ज्ञो-ऽञ्ज॑सा॒ नय॑ति । ए॒वमे॒वैन॒मश्वः॑ सुव॒र्गं ॅलो॒कमञ्ज॑सा नयति ॥ पुच्छ॑म॒न्वार॑भन्ते । सु॒व॒र्गस्य॑ लो॒कस्य॒ सम॑ष्ट्यै ॥ हिं क॑रोति । सामै॒वाकः॑ । हिं क॑रोति । उ॒द्गी॒थ ए॒वास्य॒ सः । \textbf{ 87} \newline
                  \newline
                                \textbf{ TB 3.8.22.3} \newline
                  वड॑बा॒ उप॑ रुन्धन्ति । मि॒थु॒न॒त्वाय॒ प्रजा᳚त्यै । अथो॒ यथो॑पगा॒तार॑ उप॒गाय॑न्ति । ता॒दृगे॒व तत् ॥ उद॑गासी॒दश्वो॒ मेद्ध्य॒ इत्या॑ह । प्रा॒जा॒प॒त्यो वा अश्वः॑ । प्र॒जाप॑ति-रुद्गी॒थः । उ॒द्गी॒थमे॒वाव॑रुन्धे । अथो॑ ऋख्सा॒मयो॑रे॒व प्रति॑तिष्ठति ॥ हिर॑ण्येनो॒पाक॑रोति ( ) । ज्योति॒र्वै हिर॑ण्यम् । ज्योति॑रे॒व मु॑ख॒तो द॑धाति । यज॑माने च प्र॒जासु॑ च । अथो॒ हिर॑ण्यज्योतिरे॒व यज॑मानः सुव॒र्गं ॅलो॒कमे॑ति । \textbf{ 88} \newline
                  \newline
                                    (तथ् - स - उ॒पाक॑रोति च॒त्वारि॑ च) \textbf{(A22)} \newline \newline
                \textbf{ 3.8.23    अनुवाकं   23 -अश्वे पर्यग्न्यप्रयन्तानां पशूनां नियोजनं} \newline
                                \textbf{ TB 3.8.23.1} \newline
                  पुरु॑षो॒ वै य॒ज्ञ्ः । य॒ज्ञ्ः प्र॒जाप॑तिः । यदश्वे॑ प॒शून्नि॑यु॒ञ्जन्ति॑ । य॒ज्ञादे॒व तद्य॒ज्ञ्ं प्रयु॑ङ्क्ते ॥ अश्वं॑ तूप॒रं गो॑मृ॒गम् । तान॑ग्नि॒ष्ठ आल॑भते । से॒ना॒मु॒खमे॒व तथ्सꣳश्य॑ति । तस्मा᳚द् राजमु॒खं भी॒ष्मं भावु॑कम् ॥ आ॒ग्ने॒यं कृ॒ष्णग्री॑वं पु॒रस्ता᳚-ल्ल॒लाटे᳚ । पू॒र्वा॒ग्निमे॒व तं कु॑रुते \textbf{ 89} \newline
                  \newline
                                \textbf{ TB 3.8.23.2} \newline
                  तस्मा᳚-त्पूर्वा॒ग्निं पु॒रस्ता᳚-थ्स्थापयन्ति ॥ पौ॒ष्णम॒न्वञ्च᳚म् । अन्नं॒ ॅवै पू॒षा । तस्मा᳚-त्पूर्वा॒ग्ना-वा॑हा॒र्य॑-माह॑रन्ति ॥ ऐ॒न्द्रा॒पौ॒ष्ण मु॒परि॑ष्टात् । ऐ॒न्द्रो वै रा॑ज॒न्योऽन्नं॑ पू॒षा । अ॒न्नाद्ये॑नै॒वैन॑-मुभ॒यतः॒ परि॑गृह्णाति । तस्मा᳚द् राज॒न्यो᳚न्ना॒दो भावु॑कः ॥ आ॒ग्ने॒यौ कृ॒ष्णग्री॑वौ बाहु॒वोः । बा॒हु॒वोरे॒व वी॒र्यं॑ धत्ते \textbf{ 90} \newline
                  \newline
                                \textbf{ TB 3.8.23.3} \newline
                  तस्मा᳚द् राज॒न्यो॑ बाहुब॒ली भावु॑कः ॥ त्वा॒ष्ट्रौ लो॑मशस॒क्थौ स॒क्थ्योः । स॒क्थ्योरे॒व वी॒र्यं॑ धत्ते । तस्मा᳚द् राज॒न्य॑ ऊरुब॒ली भावु॑कः ॥ शि॒ति॒पृ॒ष्ठौ बा॑र्.हस्प॒त्यौ पृ॒ष्ठे । ब्र॒ह्म॒व॒र्च॒स-मे॒वोपरि॑ष्टाद्धत्ते । अथो॑ क॒वचे॑ ए॒वैते अ॒भितः॒ पर्यू॑हते । तस्मा᳚द् राज॒न्यः॑ सन्न॑द्धो वी॒र्यं॑ करोति ॥ धा॒त्रे पृ॑षोद॒र-म॒धस्ता᳚त् । प्र॒ति॒ष्ठामे॒वैतां कु॑रुते ( ) । अथो॑ इ॒यं ॅवै धा॒ता । अ॒स्यामे॒व प्रति॑तिष्ठति ॥ सौ॒र्यं ब॒लक्षं॒ पुच्छे᳚ । उ॒थ्से॒धमे॒व तं कु॑रुते । तस्मा॑दुथ्से॒धं भ॒ये प्र॒जा अ॒भिसꣳश्र॑यन्ति । \textbf{ 91} \newline
                  \newline
                                    (कु॒रु॒ते॒ - ध॒त्ते॒ - कु॒रु॒ते॒ पञ्च॑ च) \textbf{(A23)} \newline \newline
                \textbf{Prapaataka korvai with starting  words of 1 to 23 anuvAkams -} \newline
        (सा॒ग्रं॒ह॒ण्या - चतु॑ष्टय्यो॒ - यो वै - यः पि॒तु - श्च॒त्वारा॒ - यथा॑ नि॒क्तम् - प्र॒जाप॑तये त्वा॒- यथा॒ प्रोक्षि॑तं - वि॒भूरा॑ह- प्र॒जाप॑तिरकामयताश्वमे॒धेन॑ - प्र॒जाप॑ति॒र्न किञ्च॒ न - सा॑वि॒त्र - मा ब्रह्म॑न् - प्र॒जाप॑तिर् दे॒वेभ्यः॑ - प्र॒जाप॑ति॒ रक्षाꣳ॑सि - प्र॒जाप॑तिमीफ्सति - वि॒भुर॑श्वना॒मान्य - म्भाꣳ॑ - स्येकयू॒पो - राज्जु॑दाल - मेकविꣳ॒॒शो - दे॒वाः - पुरु॑ष॒ स्त्रयो॑विꣳशतिः) \newline

        \textbf{korvai with starting words of 1, 11, 21 series of daSinis :-} \newline
        (सा॒ग्रं॒ह॒ण्या - तस्मा॑दश्वमेधया॒जी - यत् परि॑मिता॒ - यद् य॑ज्ञ्मु॒खे - यो दी॒क्षाम् - दे॒वाने॒व - त्रय॑ इ॒मे - सि॒ताय॑ - प्राणापा॒नावे॒वास्मि॒न् - तस्मा᳚द् राज॒न्य॑ एक॑नवतिः) \newline

        \textbf{first and last  word 3rd aShtakam 8th prapATakam :-} \newline
        (सा॒ग्रं॒ह॒ण्या - सꣳश्र॑यन्ति) \newline 

       

        ॥ हरिः॑ ॐ ॥
॥ कृष्ण यजुर्वेदीय तैत्तिरीय ब्राह्मणे तृतीयाष्टके अष्टम: प्रपाठकः समाप्तः ॥
+++++++++++++++++++++++++++ \newline
        \pagebreak
        
        
        
     \addcontentsline{toc}{section}{ 3.9     तैत्तिरीय ब्राह्मणे तृतीयाष्टके नवम: प्रपाठक: (अश्वमेधस्य द्वितीय-तृतीया हर्विधानम्)}
     \markright{ 3.9     तैत्तिरीय ब्राह्मणे तृतीयाष्टके नवम: प्रपाठक: (अश्वमेधस्य द्वितीय-तृतीया हर्विधानम्) \hfill https://www.vedavms.in \hfill}
     \section*{ 3.9     तैत्तिरीय ब्राह्मणे तृतीयाष्टके नवम: प्रपाठक: (अश्वमेधस्य द्वितीय-तृतीया हर्विधानम्) }
                \textbf{ 3.9.1     अनुवाकं   1 -अष्टादशिनां ग्राभ्याणां आरण्Yआनां च पशूनां प्रयोगः} \newline
                                \textbf{ TB 3.9.1.1} \newline
                  प्र॒जाप॑ति-रश्वमे॒धम॑सृजत । सो᳚ऽस्माथ्सृ॒ष्टो-ऽपा᳚क्रामत् । तम॑ष्टाद॒शिभि॒रनु॒ प्रायु॑ङ्क्त । तमा᳚प्नोत् । तमा॒प्त्वा ऽष्टा॑द॒शिभि॒रवा॑रुन्ध । यद॑ष्टाद॒शिन॑ आल॒भ्यन्ते᳚ । य॒ज्ञ्मे॒व तैरा॒प्त्वा यज॑मा॒नो-ऽव॑रुन्धे ॥ स॒म्ॅव॒थ्स॒रस्य॒ वा ए॒षा प्र॑ति॒मा । यद॑ष्टाद॒शिनः॑ । द्वाद॑श॒ मासाः॒ पञ्च॒र्तवः॑ \textbf{ 1} \newline
                  \newline
                                \textbf{ TB 3.9.1.2} \newline
                  स॒म्ॅव॒थ्स॒रो᳚ऽष्टाद॒शः । यद॑ष्टाद॒शिन॑ आल॒भ्यन्ते॑ । स॒म्ॅव॒थ्स॒रमे॒व तैरा॒प्त्वा यजमा॒नो-ऽव॑रुन्धे ॥ अ॒ग्नि॒ष्ठे᳚ ऽन्यान् प॒शूनु॑पा-क॒रोति॑ । इत॑रेषु॒ यूपे᳚ष्वष्टाद॒शिनो-ऽजा॑मित्वाय ॥ नव॑ न॒वाल॑भ्यन्ते सवीर्य॒त्वाय॑ ॥ यदा॑र॒ण्यैः सꣳ॑ स्था॒पये᳚त् । व्यव॑स्येतां पिता पु॒त्रौ । व्यद्ध्वा॑नः क्रामेयुः । विदू॑रं॒ ग्राम॑यो र्ग्रामा॒न्तौ स्या॑ताम् \textbf{ 2} \newline
                  \newline
                                \textbf{ TB 3.9.1.3} \newline
                  ऋ॒क्षीकाः᳚ पुरुषव्या॒घ्राः प॑रिमो॒षिण॑ आव्या॒धिनी॒-स्तस्क॑रा॒ अर॑ण्ये॒ष्वा-जा॑येरन्न् ॥ तदा॑हुः । अप॑शवो॒ वा ए॒ते । यदा॑र॒ण्याः । यदा॑र॒ण्यैः सꣳ॑स्था॒पये᳚त । क्षि॒प्रे यज॑मान॒-मर॑ण्यं मृ॒तꣳ ह॑रेयुः । अर॑ण्यायतना॒ ह्या॑र॒ण्याः प॒शव॒ इति॑ । यत्प॒शून्नालभे॑त । अन॑वरुद्धा अस्य प॒शवः॑ स्युः । यत्पर्य॑ग्नि-कृतानुथ्सृ॒जेत् \textbf{ 3} \newline
                  \newline
                                \textbf{ TB 3.9.1.4} \newline
                  य॒ज्ञ्॒वे॒श॒सं कु॑र्यात् ॥ यत्प॒शूना॒लभ॑ते । तेनै॒व प॒शूनव॑रुन्धे । यत्पर्य॑ग्नि-कृतानुथ्सृ॒जत्य-य॑ज्ञ्वेशसाय । अव॑रुद्धा अस्य प॒शवो॒ भव॑न्ति । न य॑ज्ञ्वेश॒ संभ॑वति । न यज॑मान॒-मर॑ण्यं मृ॒तꣳ ह॑रन्ति ॥ ग्रा॒म्यैः सꣳस्था॑पयति । ए॒ते वै प॒शवः॒ क्षेमो॒ नाम॑ । सं पि॑ता पु॒त्राव व॑स्यतः ( ) । समद्ध्वा॑नः क्रामन्ति । स॒म॒न्ति॒कं ग्राम॑यो-र्ग्रामा॒न्तौ भ॑वतः । नर्क्षीकाः᳚ पुरुषव्या॒घ्राः प॑रिमो॒षिण॑ आव्या॒धिनी॒-स्तस्क॑रा॒ अर॑ण्ये॒॒ष्वा जा॑यन्ते । \textbf{ 4} \newline
                  \newline
                                    (ऋ॒तवः॑ - स्याता - मुथ्सृ॒जेथ् - स्य॑त॒स्त्रीणि॑ च) \textbf{(A1)} \newline \newline
                \textbf{ 3.9.2     अनुवाकं   2 -चातुर्मास्यपशूनां प्रयोगः} \newline
                                \textbf{ TB 3.9.2.1} \newline
                  प्र॒जाप॑तिर-कामयतो॒भौ लो॒काव-व॑रुन्धी॒येति॑ । स ए॒तानु॒भया᳚न् प॒शून॑ पश्यत् । ग्रा॒म्याꣳश्चा॑-र॒ण्याꣳश्च॑ । तानाल॑भत । तैर्वै स उ॒भौ लो॒काव वा॑रुन्ध । ग्रा॒म्यैरे॒व प॒शुभि॑रि॒मं ॅलो॒कम वा॑रुन्ध । आ॒र॒ण्यैर॒मुम् । यद्ग्रा॒म्यान् प॒शूना॒लभ॑ते । इ॒ममे॒व तैर्लो॒कमव॑रुन्धे । यदा॑र॒ण्यान् \textbf{ 5} \newline
                  \newline
                                \textbf{ TB 3.9.2.2} \newline
                  अ॒मुं तैः ॥ अन॑वरुद्धो॒ वा ए॒तस्य॑ सम्ॅवथ्स॒र इत्या॑हुः । य इ॒त इ॒तश्चातुर्मा॒स्यानि॑ स॒म्ॅवथ्स॒रं प्र॑यु॒ङ्क्त इति॑ । ए॒तावा॒न्॒. वै स॑म्ॅवथ्स॒रः । यच्चा॑तुर्मा॒स्यानि॑ । यदे॒ते चातु॑र्मा॒स्याः प॒शव॑ आल॒भ्यन्ते᳚ । प्र॒त्यक्ष॑मे॒व तैः स॑म्ॅवथ्स॒रं यज॑मा॒नो-ऽव॑रुन्धे । ॥वि वा ए॒ष प्र॒जया॑ प॒शुभि॑र्. ऋद्ध्यते । यः स॑म्ॅवथ्स॒रं प्र॑यु॒ङ्क्ते । स॒म्ॅव॒थ्स॒रः सु॑व॒र्गो लो॒कः \textbf{ 6} \newline
                  \newline
                                \textbf{ TB 3.9.2.3} \newline
                  सु॒व॒र्गं तु लो॒कं नाप॑राद्ध्नोति । प्र॒जा वै प॒शव॑ एकाद॒शिनी᳚ । यदे॒त ऐ॑कादशि॒नाः प॒शव॑ आल॒भ्यन्ते᳚ । सा॒क्षादे॒व प्र॒जां प॒शून्. यज॑मा॒नो-ऽव॑रुन्धे ॥ प्र॒जाप॑ति-र्वि॒राज॑मसृजत । सा सृ॒ष्टाऽश्व॑मे॒धं प्रावि॑शत् । तां द॒शिभि॒रनु॒ प्रायु॑ङ्क्त । तामा᳚प्नोत् । तामा॒प्त्वा द॒शिभि॒रवा॑रुन्ध । यद्द॒शिन॑ आल॒भ्यन्ते᳚ \textbf{ 7} \newline
                  \newline
                                \textbf{ TB 3.9.2.4} \newline
                  वि॒राज॑मे॒व तैरा॒प्त्वा यज॑मा॒नो-ऽव॑रुन्धे ॥ एका॑दश द॒शत॒ आल॑भ्यन्ते । एका॑दशाक्षरा त्रि॒ष्टुप् । त्रैष्टु॑भाः प॒शवः॑ । प॒शूने॒वाव॑रुन्धे ॥ वै॒श्व॒दे॒वो वा अश्वः॑। ना॒ना॒दे॒व॒त्याः᳚ प॒शवो॑ भवन्ति । अश्व॑स्य सर्व॒त्वाय॑ । नाना॑रूपा भवन्ति । तस्मा॒न्नाना॑रूपाः प॒शवः॑( ) । ब॒हु॒रू॒पा भ॑वन्ति । तस्मा᳚द् बहुरू॒पाः प॒शवः॒ समृ॑द्ध्यै । \textbf{ 8} \newline
                  \newline
                                    (आ॒र॒ण्यान्-ॅलो॒को-द॒शिन॑ आल॒भ्यन्ते॒-नाना॑रूपाः प॒शवो॒ द्वे च॑) \textbf{(A2)} \newline \newline
                \textbf{ 3.9.3     अनुवाकं   3 -रोहितादीनां पशूनां वपाहोमसाहित्यम्} \newline
                                \textbf{ TB 3.9.3.1} \newline
                  अ॒स्मै वै लो॒काय॑ ग्रा॒म्याः प॒शव॒ आल॑भ्यन्ते । अ॒मुष्मा॑ आर॒ण्याः । यद्ग्रा॒म्यान् प॒शूना॒लभ॑ते । इ॒ममे॒व तैर्लो॒कमव॑रुन्धे । यदा॑र॒ण्यान् । अ॒मुं तैः । उ॒भया᳚न् प॒शूनाल॑भते । ग्रा॒म्याꣳश्चा॑-र॒ण्याꣳश्च॑ । उ॒भयो᳚र् लो॒कयो॒र व॑रुद्ध्यै । उ॒भया᳚न् प॒शूनाल॑भते \textbf{ 9} \newline
                  \newline
                                \textbf{ TB 3.9.3.2} \newline
                  ग्रा॒म्याꣳश्चा॑-र॒ण्याꣳश्च॑ । उ॒भय॑स्या॒-न्नाद्य॒स्या-व॑रुद्ध्यै । उभया᳚न् प॒शूनालभते । ग्राम्याꣳश्चा॑र॒ण्याꣳश्च॑ । उ॒भये॑षां पशू॒नामव॑रुद्ध्यै ॥ त्रय॑स्त्रयो भवन्ति । त्रय॑ इ॒मे लो॒काः । ए॒षां ॅलो॒काना॒माप्त्यै᳚ ॥ ब्र॒ह्म॒वा॒दिनो॑ वदन्ति । कस्मा᳚थ्स॒त्यात् \textbf{ 10} \newline
                  \newline
                                \textbf{ TB 3.9.3.3} \newline
                  अ॒स्मिन् ॅलो॒के ब॒हवः॒ कामा॒ इति॑ । यथ्स॑मा॒नीभ्यो॑ दे॒वता᳚भ्यो॒-ऽन्ये᳚ऽन्ये प॒शव॑ आल॒भ्यन्ते᳚ । अ॒स्मिन्ने॒व तल्लो॒के कामा᳚न्दधाति । तस्मा॑द॒स्मिन्-ॅलो॒के ब॒हवः॒ कामाः᳚ ॥ त्र॒या॒णां त्र॑याणाꣳ स॒ह व॒पा जु॑होति । त्र्या॑वृतो॒ वै दे॒वाः । त्र्या॑वृत इ॒मे लो॒काः । ए॒षां ॅलो॒काना॒माप्त्यै᳚ । ए॒षां ॅलो॒कानां॒ क्लृप्त्यै᳚ ॥ पर्य॑ग्नि-कृता-नार॒ण्या-नुथ्सृ॑ज॒न्त्यहिꣳ॑सायै ( ) । \textbf{ 11} \newline
                  \newline
                                    (अव॑रुद्ध्या उ॒भया᳚न् प॒शूनाल॑भते - स॒त्या - दहिꣳ॑सायै) \textbf{(A3)} \newline \newline
                \textbf{ 3.9.4     अनुवाकं   4 -अश्वस्य रथयोजनालंकारादयः} \newline
                                \textbf{ TB 3.9.4.1} \newline
                  यु॒ञ्जन्ति॑ ब्र॒द्ध्नमित्या॑ह । अ॒सौ वा आ॑दि॒त्यो ब्र॒द्ध्नः । आ॒दि॒त्य-मे॒वास्मै॑ युनक्ति । अ॒रु॒षमित्या॑ह । अ॒ग्निर्वा अ॑रु॒षः । अ॒ग्निमे॒वास्मै॑ युनक्ति । चर॑न्त॒मित्या॑ह । वा॒युर्वै चरन्न्॑ । वा॒युमे॒वास्मै॑ युनक्ति । परि॑त॒स्थुष॒ इत्या॑ह \textbf{ 12} \newline
                  \newline
                                \textbf{ TB 3.9.4.2} \newline
                  इ॒मे वै लो॒काः परि॑त॒स्थुषः॑ । इ॒मने॒वास्मै॑ लो॒कान्. यु॑नक्ति ॥ रोच॑न्ते रोच॒ना दि॒वीत्या॑ह । नक्ष॑त्राणि॒ वै रो॑च॒ना दि॒वि । नक्ष॑त्राण्ये॒वास्मै॑ रोचयति ॥ यु॒ञ्जन्त्य॑स्य॒ काम्येत्या॑ह । कामा॑ने॒वास्मै॑ युनक्ति ॥ हरी॒ विप॑क्ष॒सेत्या॑ह । इ॒मे वै हरी॒ विप॑क्षसा । इ॒मे ए॒वास्मै॑ युनक्ति । \textbf{ 13} \newline
                  \newline
                                \textbf{ TB 3.9.4.3} \newline
                  शोणा॑ धृ॒ष्णू नृ॒वाह॒सेत्या॑ह । अ॒हो॒रा॒त्रे वै नृ॒वाह॑सा । अ॒हो॒रा॒त्रे ए॒वास्मै॑ युनक्ति ॥ ए॒ता ए॒वास्मै॑ दे॒वता॑ युनक्ति । सु॒व॒र्गस्य॑ लो॒कस्य॒ सम॑ष्ट्यै ॥ के॒तुं कृ॒ण्वन्न॑-के॒तव॒ इति॑ द्ध्व॒जं प्रति॑मुञ्चति । यश॑ ए॒वैनꣳ॒॒ राज्ञां᳚ गमयति ॥ जी॒मूत॑स्येव भवति॒ प्रती॑क॒मित्या॑ह । य॒था॒ य॒जुरे॒वैतत् ॥ ये ते॒ पन्था॑नः सवितः पू॒र्व्यास॒ इत्य॑द्ध्व॒र्यु-र्यज॑मानं ॅवाचयत्य॒भिजि॑त्यै । \textbf{ 14} \newline
                  \newline
                                \textbf{ TB 3.9.4.4} \newline
                  परा॒ वा ए॒तस्य॑ य॒ज्ञ् ए॑ति । यस्य॑ प॒शुरु॒पा-कृ॑तो॒-ऽन्यत्र॒ वेद्या॒ एति॑ । एतꣳ स्तो॑तरे॒तेन॑ प॒था पुन॒रश्व॒-माव॑र्तयासि न॒ इत्या॑ह । वा॒युर्वै स्तोता᳚ । वा॒युमे॒वास्य॑ प॒रस्ता᳚द्-दधा॒त्यावृ॑त्त्यै ॥ यथा॒ वै ह॒विषो॑ गृही॒तस्य॒ स्कन्द॑ति । ए॒वं ॅवा ए॒तदश्व॑स्य स्कन्दति । यद॑स्यो॒पाकृ॑तस्य॒ लोमा॑नि॒ शीय॑न्ते । यद्वाले॑षु का॒चाना॒वय॑न्ति । लोमा᳚न्ये॒वास्य॒ तथ्संभ॑रन्ति । \textbf{ 15} \newline
                  \newline
                                \textbf{ TB 3.9.4.5} \newline
                  भूर्भुव॒स्सुव॒रिति॑ प्राजाप॒त्या-भि॒राव॑यन्ति । प्रा॒जा॒प॒त्यो वा अश्वः॑ । स्वयै॒वैनं॑ दे॒वत॑या॒ सम॑द्र्धयन्ति ॥ भूरिति॒ महि॑षी । भुव॒ इति॑ वा॒वाता᳚ । सुव॒रिति॑ परिवृ॒क्ती । ए॒षां ॅलो॒काना॑-म॒भिजि॑त्यै ॥ हि॒र॒ण्ययाः᳚ का॒चा भ॑वन्ति । ज्योति॒र्वै हिर॑ण्यम् । रा॒ष्ट्रम॑श्वमे॒धः \textbf{ 16} \newline
                  \newline
                                \textbf{ TB 3.9.4.6} \newline
                  ज्योति॑श्चै॒वास्मै॑ रा॒ष्ट्रं च॑ स॒मीची॑ दधाति ॥ स॒हस्रं॑ भवन्ति । स॒हस्र॑ संमितः सुव॒र्गो लो॒कः । सु॒व॒र्गस्य॑ लो॒कस्या॒-भिजि॑त्यै ॥ अप॒ वा ए॒तस्मा॒त्तेज॑ इन्द्रि॒यं प॒शवः॒ श्रीः क्रा॑मन्ति । यो᳚ऽश्वमे॒धेन॒ यज॑ते । वस॑वस्त्वा-ऽञ्जन्तु गाय॒त्रेण॒ छन्द॒सेति॒ महि॑ष्य॒भ्य॑नक्ति । तेजो॒ वा आज्य᳚म् । तेजो॑ गाय॒त्री । तेज॑सै॒वास्मै॒ तेजोऽव॑रुन्धे \textbf{ 17} \newline
                  \newline
                                \textbf{ TB 3.9.4.7} \newline
                  रु॒द्रास्त्वा᳚-ऽञ्जन्तु॒ त्रैष्टु॑भेन॒ छन्द॒सेति॑ वा॒वाता᳚ । तेजो॒ वा आज्य᳚म् । इ॒न्द्रि॒यं त्रि॒ष्टुप् । तेज॑सै॒वास्मा॑ इन्द्रि॒यमव॑रुन्धे । आ॒दि॒त्या-स्त्वा᳚-ऽञ्जन्तु॒ जाग॑तेन॒ छन्द॒सेति॑ परिवृ॒क्ती । तेजो॒ वा आज्य᳚म् । प॒शवो॒ जग॑ती । तेज॑सै॒वास्मै॑ प॒शूनव॑रुन्धे ॥ पत्न॑यो॒-ऽभ्य॑ञ्जन्ति । श्रि॒या वा ए॒तद् रू॒पम् \textbf{ 18} \newline
                  \newline
                                \textbf{ TB 3.9.4.8} \newline
                  यत्पत्न॑यः । श्रिय॑-मे॒वास्मि॒न् तद्द॑धति । नास्मा॒त्तेज॑ इन्द्रि॒यं प॒शवः॒ श्रीरप॑क्रामन्ति ॥ लाजी(3)ञ्छाची(3)न्. यशो॑ म॒माॅ(4) इत्यति॑रिक्त॒-मन्न॒मश्वा॑यो॒ पाह॑रन्ति । प्र॒जामे॒वान्ना॒दीं कु॑र्वते ॥ ए॒तद्दे॑वा॒ अन्न॑मत्तै॒त-दन्न॑मद्धि प्रजापत॒ इत्या॑ह । प्र॒जाया॑मे॒वान्नाद्यं॑ दधते ॥ यदि॒ नाव॒जिघ्र᳚त् । अ॒ग्निः प॒शुरा॑सी॒दित्यव॑-घ्रापयेत् । अव॑ है॒व जि॑घ्रति ( ) ॥ आक्रान॑. वा॒जी क्रमै॒रत्य॑-क्रमीद्वा॒जी द्यौस्ते॑ पृ॒ष्ठं पृ॑थि॒वी स॒धस्थ॒मित्यश्व॒-मनु॑मन्त्रयते । ए॒षां ॅलो॒काना॑-म॒भिजि॑त्यै ॥ समि॑द्धो अ॒ञ्जन्कृद॑रं मती॒ना-मित्यश्व॑स्या॒ प्रियो॑ भवन्ति सरूप॒त्वाय॑ । \textbf{ 19} \newline
                  \newline
                                    (परि॑त॒स्थुष॒ इत्या॑हे॒ - मे ए॒वास्मै॑ युनक्त्य॒ - भिजि॑त्यै-भर-न्त्यश्वमे॒धो - रु॑न्धे - रू॒पम् - जि॑घ्रति॒ त्रीणि॑ च) \textbf{(A4)} \newline \newline
                \textbf{ 3.9.5     अनुवाकं   5 -ब्रह्मोद्यनामकः होतृब्राह्मणोस्संवादः} \newline
                                \textbf{ TB 3.9.5.1} \newline
                  तेज॑सा॒ वा ए॒ष ब्र॑ह्मवर्च॒सेन॒ व्यृ॑द्ध्यते । यो᳚-ऽश्वमे॒धेन॒ यज॑ते । होता॑ च ब्र॒ह्मा च॑ ब्र॒ह्मोद्यं॑ ॅवदतः । तेज॑सा चै॒वैनं॑ ब्रह्मवर्च॒सेन॑ च॒ सम॑द्र्धयतः ॥ द॒क्षि॒ण॒तो ब्र॒ह्मा भ॑वति । द॒क्षि॒ण॒त आ॑यतनो॒ वै ब्र॒ह्मा । बा॒र्॒.ह॒स्प॒त्यो वै ब्र॒ह्मा । ब्र॒ह्म॒व॒र्च॒समे॒वास्य॑ दक्षिण॒तो (दक्षि॒णतो) द॑धाति । तस्मा॒द्-दक्षि॒णोऽद्र्धो᳚ ब्रह्मवर्च॒सित॑रः ॥ उ॒त्त॒र॒तो होता॑ भवति \textbf{ 20} \newline
                  \newline
                                \textbf{ TB 3.9.5.2} \newline
                  उ॒त्त॒र॒त आ॑यतनो॒ वै होता᳚ । आ॒ग्ने॒यो वै होता᳚ । तेजो॒ वा अ॒ग्निः । तेज॑ ए॒वास्यो᳚त्तर॒तो द॑धाति । तस्मा॒दुत्त॒रोऽद्र्ध॑-स्तेज॒स्वित॑रः ॥ यूप॑म॒भितो॑ वदतः । य॒ज॒मा॒न॒दे॒व॒त्यो॑ वै यूपः॑ । यज॑मानमे॒व तेज॑सा च ब्रह्मवर्च॒सेन॑ च॒ सम॑द्र्धयतः ॥ किꣳ स्वि॑दासीत्-पू॒र्वचि॑त्ति॒-रित्या॑ह । द्यौर्वै वृष्टिः॑ पू॒र्वचि॑त्तिः \textbf{ 21} \newline
                  \newline
                                \textbf{ TB 3.9.5.3} \newline
                  दिव॑मे॒व वृष्टि॒मव॑रुन्धे ॥ किꣳ स्वि॑दासीद्-बृ॒हद्वय॒ इत्या॑ह । अश्वो॒ वै बृ॒हद्वयः॑ । अश्व॑मे॒वाव॑रुन्धे ॥ किꣳ स्वि॑दासीत्-पिशङ्गि॒लेत्या॑ह । रात्रि॒र्वै पि॑शङ्गि॒ला । रात्रि॑मे॒वाव॑रुन्धे ॥ किꣳ स्वि॑दासीत्-पिलिप्पि॒लेत्या॑ह । श्रीर्वै पि॑लिप्पि॒ला । अ॒न्नाद्य॑मे॒वाव॑रुन्धे । \textbf{ 22} \newline
                  \newline
                                \textbf{ TB 3.9.5.4} \newline
                  कः स्वि॑देका॒की च॑र॒तीत्या॑ह । अ॒सौ वा आ॑दि॒त्य ए॑का॒की च॑रति । तेज॑ ए॒वाव॑रुन्धे ॥ क उ॑ स्विज्जायते॒ पुन॒रित्या॑ह । च॒न्द्रमा॒ वै जा॑यते॒ पुनः॑ । आयु॑रे॒वाव॑रुन्धे ॥ किꣳ स्वि॑द्धि॒मस्य॑ भेष॒जमित्या॑ह । अ॒ग्निर्वै हि॒मस्य॑ भेष॒जम् । ब्र॒ह्म॒व॒र्च॒समे॒वाव॑रुन्धे ॥ किꣳ स्वि॑दा॒-वप॑नं म॒हदित्या॑ह \textbf{ 23} \newline
                  \newline
                                \textbf{ TB 3.9.5.5} \newline
                  अ॒यं ॅवै लो॒क आ॒वप॑नं म॒हत् । अ॒स्मिन्ने॒व लो॒के प्रति॑तिष्ठति ॥ पृ॒च्छामि॑ त्वा॒ पर॒मन्तं॑ पृथि॒व्या इत्या॑ह । वेदि॒र्वै परोऽन्तः॑ पृथि॒व्याः । वेदि॑मे॒वाव॑रुन्धे ॥ पृ॒च्छामि॑ त्वा॒ भुव॑नस्य॒ नाभि॒मित्या॑ह । य॒ज्ञो वै भुव॑नस्य॒ नाभिः॑ । य॒ज्ञ्मे॒वाव॑रुन्धे ॥ पृ॒च्छामि॑ त्वा॒ वृष्णो॒ अश्व॑स्य॒ रेत॒ इत्या॑ह । सोमो॒ वै वृष्णो॒ अश्व॑स्य॒ रेतः॑ ( ) । सो॒म॒पी॒थ-मे॒वाव॑रुन्धे ॥ पृ॒च्छामि॑ वा॒चः प॑र॒मं ॅव्यो॑मेत्या॑ह । ब्रह्म॒ वै वा॒चः प॑र॒मं ॅव्यो॑म । ब्र॒ह्म॒व॒र्च॒स-मे॒वाव॑रुन्धे । \textbf{ 24} \newline
                  \newline
                                    (होता॑ भवति॒ - वै वृष्टिः॑ पू॒र्वचि॑त्ति - र॒न्नाद्य॑मे॒वाव॑रुन्धे - म॒हदित्या॑ह॒ - सोमो॒ वै वृष्णो॒ अश्व॑स्य॒ रेत॑श्च॒त्वारि॑ च) \textbf{(A5)} \newline \newline
                \textbf{ 3.9.6     अनुवाकं   6 -अश्वस्य मृतोपचारः संज्ञ्पनप्रकारः} \newline
                                \textbf{ TB 3.9.6.1} \newline
                  अप॒ वा ए॒तस्मा᳚-त्प्रा॒णाः क्रा॑मन्ति । यो᳚ऽश्वमे॒धेन॒ यज॑ते । प्रा॒णाय॒ स्वाहा᳚ व्या॒नाय॒ स्वाहेति॑ संज्ञ्॒प्यमा॑न॒ आहु॑ती-र्जुहोति । प्रा॒णाने॒वास्मि॑न्दधाति । नास्मा᳚त्प्रा॒णा अप॑क्रामन्ति ॥ अव॑न्तीः॒-स्थाव॑न्ती-स्त्वा-ऽवन्तु । प्रि॒यं त्वा᳚ प्रि॒याणा᳚म् । वर्.षि॑ष्ठ॒माप्यानाम् । नि॒धी॒नां त्वा॑ निधि॒पतिꣳ॑ हवामहे वसोम॒-मेत्या॑ह । अपै॒वास्मै॒ तद्ध्नु॑वते \textbf{ 25} \newline
                  \newline
                                \textbf{ TB 3.9.6.2} \newline
                  अथो॑ धु॒वन्त्ये॒वैन᳚म् । अथो॒ न्ये॑वास्मै᳚ ह्नुवते ॥ त्रिः परि॑यन्ति । त्रय॑ इ॒मे लो॒काः । ए॒भ्य ए॒वैनं॑ ॅलो॒केभ्यो॑ धुवते ॥ त्रिः पुनः॒ परि॑यन्ति । षट्-थ्संप॑द्यन्ते । षड्वा ऋ॒तवः॑ । ऋ॒तुभि॑रे॒वैनं॑ धुवते ॥ अप॒ वा ए॒तेभ्यः॑ प्रा॒णाः क्रा॑मन्ति \textbf{ 26} \newline
                  \newline
                                \textbf{ TB 3.9.6.3} \newline
                  ये य॒ज्ञे धुव॑नं त॒न्वते᳚ । न॒व॒कृत्वः॒ परि॑यन्ति । नव॒ वै पुरु॑षे प्रा॒णाः । प्रा॒णाने॒वात्मन्द॑धते । नैभ्यः॑ प्रा॒णा अप॑क्रामन्ति ॥ अम्बे॒ अम्बा॒ल्यम्बि॑क॒ इति॒ पत्नी॑मु॒दान॑यति । अह्व॑तै॒वैना᳚म् ॥ सुभ॑गे॒ काम्पी॑ल-वासि॒नीत्या॑ह । तप॑ ए॒वैना॒-मुप॑नयति ॥ सु॒व॒र्गे लो॒के संप्रोर्ण्वा॑था॒-मित्या॑ह \textbf{ 27} \newline
                  \newline
                                \textbf{ TB 3.9.6.4} \newline
                  सु॒व॒र्गमे॒वैनां᳚ ॅलो॒कं ग॑मयति ॥ आऽहम॑जानि गर्भ॒धमा त्वम॑जासि गर्भ॒धमित्या॑ह । प्र॒जा वै प॒शवो॒ गर्भः॑ । प्र॒जामे॒व प॒शूना॒त्मन्ध॑त्ते ॥ दे॒वा वा अ॑श्वमे॒धे पव॑माने । सु॒व॒र्गं ॅलो॒कं न प्राजा॑नन्न् । तमश्वः॒ प्राजा॑नात् । यथ्सू॒चीभि॑-रसिप॒था-न्क॒ल्पय॑न्ति । सु॒व॒र्गस्य॑ लो॒कस्य॒ प्रज्ञा᳚त्यै ॥ गा॒य॒त्री त्रि॒ष्टुब्जग॒तीत्या॑ह \textbf{ 28} \newline
                  \newline
                                \textbf{ TB 3.9.6.5} \newline
                  य॒था॒ य॒जुरे॒वैतत् ॥ त्र॒य्यः सू॒च्यो॑ भवन्ति । अ॒य॒स्मय्यो॑ रज॒ता हरि॑ण्यः । अ॒स्य वै लो॒कस्य॑ रू॒पम॑य॒स्मय्यः॑ । अ॒न्तरि॑क्षस्य रज॒ताः । दि॒वो हरि॑ण्यः । दिशो॒ वा अ॑य॒स्मय्यः॑ । अ॒वा॒न्त॒र॒दि॒शा र॑ज॒ताः । ऊ॒द्र्ध्वा हरि॑ण्यः । दिश॑ ए॒वास्मै॑ कल्पयति ( ) ॥ कस्त्वा᳚-छ्यति॒ कस्त्वा॒ विशा॒स्ती-त्या॒हा हिꣳ॑सायै । \textbf{ 29} \newline
                  \newline
                                    (ह्नु॒व॒ते॒-क्रा॒म॒-न्त्यू॒र्ण्वा॒था॒मित्या॑ह॒-जग॒तीत्या॑ह-कल्पय॒त्येकं॑ च) \textbf{(A6)} \newline \newline
                \textbf{ 3.9.7     अनुवाकं   7 -तत्र मन्त्राः} \newline
                                \textbf{ TB 3.9.7.1} \newline
                  अप॒ वा ए॒तस्मा॒च्छ्री रा॒ष्ट्रं क्रा॑मति । यो᳚ऽश्वमे॒धेन॒ यज॑ते । ऊ॒द्र्ध्वामे॑ना॒-मुच्छ्र॑यता॒-दित्या॑ह । श्रीर्वै रा॒ष्ट्रम॑श्वमे॒धः । श्रिय॑मे॒वास्म॑ रा॒ष्ट्रमू॒द्र्ध्व-मुच्छ्र॑यति ॥ वे॒णु॒भा॒रं गि॒रावि॒-वेत्या॑ह । रा॒ष्ट्रं ॅवै भा॒रः । रा॒ष्ट्रमे॒वास्मै॒ पर्यू॑हति ॥ अथा᳚स्या॒ मद्ध्य॑मेधता॒-मित्या॑ह । श्रीर्वै रा॒ष्ट्रस्य॒ मद्ध्य᳚म् \textbf{ 30} \newline
                  \newline
                                \textbf{ TB 3.9.7.2} \newline
                  श्रिय॑मे॒वा-व॑रुन्धे ॥ शी॒ते वाते॑ पु॒नन्नि॒-वेत्या॑ह । क्षेमो॒ वै रा॒ष्ट्रस्य॑ शी॒तो वातः॑ । क्षेम॑मे॒वा-व॑रुन्धे ॥ यद्ध॑रि॒णी यव॒मत्तीत्या॑ह । विड्वै ह॑रि॒णी । रा॒ष्ट्रं ॅयवः॑ । विशं॑ चै॒वास्मै॑ रा॒ष्ट्रं च॑ स॒मीची॑ दधाति ॥ न पु॒ष्टं प॒शु म॑न्यत॒ इत्या॑ह । तस्मा॒द् राजा॑ प॒शून्न पुष्य॑ति । \textbf{ 31} \newline
                  \newline
                                \textbf{ TB 3.9.7.3} \newline
                  शू॒द्रा यदर्य॑जारा॒ न पोषा॑य धनाय॒तीत्या॑ह । तस्मा᳚-द्वैशीपु॒त्रं नाभिषि॑ञ्चन्ते ॥ इ॒यं ॅय॒का श॑कुन्ति॒केत्या॑ह । विड्वै श॑कुन्ति॒का । रा॒ष्ट्रम॑श्वमे॒धः । विशं॑ चै॒वास्मै॑ रा॒ष्ट्रं च॑ स॒मीची॑ दधाति ॥ आ॒हल॒मिति॒ सर्प॒तीत्या॑ह । तस्मा᳚द् रा॒ष्ट्राय॒ विशः॑ सर्पन्ति ॥ आह॑तं ग॒भे पस॒ इत्या॑ह । विड्वै गभः॑ \textbf{ 32} \newline
                  \newline
                                \textbf{ TB 3.9.7.4} \newline
                  रा॒ष्ट्रं पसः॑ । रा॒ष्ट्रम॒व वि॒श्याह॑न्ति । तस्मा᳚द् रा॒ष्ट्रं ॅविशं॒ घातु॑कम् ॥ मा॒ता च॑ ते पि॒ता च॑ त॒ इत्या॑ह । इ॒यं ॅवै मा॒ता । अ॒सौ पि॒ता । आ॒भ्यामे॒वैनं॒ परि॑ददाति ॥ अग्रं॑ ॅवृ॒क्षस्य॑ रोहत॒ इत्या॑ह । श्रीर्वै व॒क्षस्याग्र᳚म् । श्रिय॑मे॒वाव॑रुन्धे । \textbf{ 33} \newline
                  \newline
                                \textbf{ TB 3.9.7.5} \newline
                  प्रसु॑ला॒मीति॑ ते पि॒ता ग॒भे मु॒ष्टिम॑तꣳसय॒-दित्या॑ह । विड्वै गभः॑ । रा॒ष्ट्रं मु॒ष्टिः । रा॒ष्ट्रमे॒व वि॒श्याह॑न्ति । तस्मा᳚द् रा॒ष्ट्रं ॅविशं॒ घातु॑कम् ॥ अप॒ वा ए॒तेभ्यः॑ प्रा॒णाः क्रा॑मन्ति । ये य॒ज्ञेऽपू॑तं॒ ॅवद॑न्ति । द॒धि॒क्राव्.ण्णो॑ अकारिष॒मिति॑ सुरभि॒-मती॒मृचं॑ ॅवदन्ति । प्रा॒णा वै सु॑र॒भयः॑ । प्रा॒णाने॒वात्मन्द॑धते ( ) । नैभ्यः॑ प्रा॒णा अप॑क्रामन्ति ॥ आपो॒ हिष्ठा म॑यो॒भुव॒ इत्य॒द्भिर्मा᳚र्जयन्ते । आपो॒ वै सर्वा॑ दे॒वताः᳚ । दे॒वता॑भिरे॒-वात्मानं॑ पवयन्ते । \textbf{ 34} \newline
                  \newline
                                    (रा॒ष्ट्रस्य॒ मद्ध्य॒म् - पुष्य॑ति॒ - गभो॑ - रुन्धे - दधते च॒त्वारि॑ च) \textbf{(A7)} \newline \newline
                \textbf{ 3.9.8     अनुवाकं   8 -अश्वमेधस्य तत्पशूनां च प्रशंसा} \newline
                                \textbf{ TB 3.9.8.1} \newline
                  प्र॒जाप॑तिः प्र॒जाः सृ॒ष्ट्वा प्रे॒णाऽनु॒ प्रावि॑शत् । ताभ्यः॒ पुनः॒ संभ॑वितुं॒ नाश॑क्नोत् । सो᳚ऽब्रवीत् । ऋ॒द्ध्नव॒दिथ्सः । यो मे॒तः पुनः॑ स॒भंर॒दिति॑ । तं दे॒वा अ॑श्वमे॒धेनै॒व सम॑भरन्न् । ततो॒ वै त आ᳚द्र्ध्नुवन्न । यो᳚ऽश्वमे॒धेन॒ यज॑ते । प्र॒जाप॑तिमे॒व संभ॑रत्यृ॒द्ध्नोति॑ ॥ पुरु॑ष॒माल॑भते \textbf{ 35} \newline
                  \newline
                                \textbf{ TB 3.9.8.2} \newline
                  वै॒रा॒जो वै पुरु॑षः । वि॒राज॑-मे॒वाल॑भते । अथो॒ अन्नं॒ ॅवै वि॒राट् । अन्न॑मे॒वाव॑रुन्धे ॥ अश्व॒माल॑भते । प्रा॒जा॒प॒त्यो वा अश्वः॑ । प्र॒जाप॑ति-मे॒वाल॑भते । अथो॒ श्रीर्वा एक॑शफम् । श्रिय॑-मे॒वाव॑रुन्धे ॥ गामाल॑भते \textbf{ 36} \newline
                  \newline
                                \textbf{ TB 3.9.8.3} \newline
                  य॒ज्ञो वै गौः । य॒ज्ञ्मे॒वाल॑भते । अथो॒ अन्नं॒ ॅवै गौः । अन्न॑मे॒वाव॑रुन्धे ॥ अ॒जा॒वी आल॑भते भू॒म्ने । अथो॒ पुष्टि॒र्वै भू॒मा । पुष्टि॑मे॒वाव॑रुन्धे ॥ पर्य॑ग्निकृतं॒ पुरु॑षं चार॒ण्याꣳ-श्चोथ् सृ॑ज॒न्त्यहिꣳ॑सायै ॥ उ॒भौ वा ए॒तौ प॒शू आल॑भ्येते । यश्चा॑व॒मो यश्च॑ पर॒मः ( ) । ते᳚ऽस्यो॒भये॑ य॒ज्ञे ब॒द्धाः । अ॒भीष्टा॑ अ॒भिप्री॑ताः । अ॒भिजि॑ता अ॒भिहु॑ता भवन्ति ॥ नैनं॑ द॒ङ्क्ष्णवः॑ प॒शवो॑ य॒ज्ञे ब॒द्धाः । अ॒भीष्टा॑ अ॒भिप्री॑ताः । अ॒भिजि॑ता अ॒भिहु॑ता हिꣳसन्ति । यो᳚ऽश्वमे॒धेन॒ यज॑ते । य उ॑ चैनमे॒वं ॅवेद॑ । \textbf{ 37} \newline
                  \newline
                                    (ल॒भ॒ते॒ - गामाल॑भते - पर॒मो᳚ऽष्टौ च॑) \textbf{(A8)} \newline \newline
                \textbf{ 3.9.9     अनुवाकं   9 -उत्तमेऽहनि प्रशवः} \newline
                                \textbf{ TB 3.9.9.1} \newline
                  प्र॒थ॒मेन॒ वा ए॒ष स्तोम॑न रा॒द्ध्वा । च॒तु॒ष्टो॒मेन॑ कृ॒तेनाया॑ना॒-मुत्त॒रेऽहन्न्॑ । ए॒क॒विꣳ॒॒शे प्र॑ति॒ष्ठायां॒ प्रति॑तिष्ठति ॥ ए॒क॒विꣳ॒॒शा-त्प्र॑ति॒ष्ठाया॑ ऋ॒तून॒न्वारो॑हति । ऋ॒तवो॒ वै पृ॒ष्ठानि॑ । ऋ॒तवः॑ सम्ॅवथ्स॒रः । ऋ॒तुष्वे॒व स॑म्ॅवथ्स॒रे प्र॑ति॒ष्ठाय॑ । दे॒वता॑ अ॒भ्यारो॑हति । शक्व॑रयः पृ॒ष्ठं भ॑वन्त्य॒-न्यद॑न्य॒-च्छन्दः॑ । अ॒न्ये᳚ऽन्ये॒ वा ए॒ते प॒शव॒ आल॑भ्यन्ते \textbf{ 38} \newline
                  \newline
                                \textbf{ TB 3.9.9.2} \newline
                  उ॒तेव॑ ग्रा॒म्याः । उ॒तेवा॑र॒ण्याः । अह॑रे॒व रू॒पेण॒ सम॑द्र्धयति । अथो॒ अह्न॑ ए॒वैष ब॒लिर्.ह्रि॑यते ॥ तदा॑हुः । अप॑शवो॒ वा ए॒ते । यद॑जा॒-वय॑श्चार॒ण्याश्च॑ । ए॒ते वै सर्वे॑ प॒शवः॑ । यद्ग॒व्या इति॑ । ग॒व्यान् प॒शूनु॑त्त॒मेऽह॒न्नाल॑भते \textbf{ 39} \newline
                  \newline
                                \textbf{ TB 3.9.9.3} \newline
                  तेनै॒वोभया᳚न् प॒शूनव॑रुन्धे ॥ प्रा॒जा॒प॒त्या भ॑वन्ति । अन॑भिजितस्या॒-भिजि॑त्यै ॥ सौ॒रीर्नव॑ श्वे॒ताव॒शा अ॑नूब॒न्ध्या॑ भवन्ति । अ॒न्त॒त ए॒व ब॑ह्मवर्च॒समव॑रुन्धे ॥ सोमा॑य स्व॒राज्ञे॑-ऽनोवा॒हा-व॑न॒ड्वाहा॒विति॑ द्व॒न्द्विनः॑ प॒शूनाल॑भते । अ॒हो॒रा॒त्राणा॑-म॒भिजि॑त्यै ॥ प॒शुभि॒र्वा ए॒ष व्यृ॑द्ध्यते । यो᳚ऽश्वमे॒धेन॒ यज॑ते । छ॒ग॒लं क॒ल्माषं॑ किकिदी॒विं ॅवि॑दी॒गय॒मिति॑ त्वा॒ष्ट्रान् प॒शूनाल॑भते ( ) । प॒शुभि॑रे॒वात्मानꣳ॒॒ सम॑द्र्धयति ॥ ऋ॒तुभि॒र्वा ए॒ष व्यृ॑द्ध्यते । यो᳚ऽश्वमे॒धेन॒ यज॑ते । पि॒शङ्गा॒-स्त्रयो॑ वास॒न्ता इत्यृ॑तु-प॒शूनाल॑भते । ऋ॒तुभि॑रे॒-वात्मानꣳ॒॒ सम॑द्र्धयति ॥ आ वा ए॒ष प॒शुभ्यो॑ वृश्च्यते । यो᳚ऽश्वमे॒धेन॒ यज॑ते । पर्य॑ग्निकृता॒ उथ्सृ॑ज॒न्त्यना᳚व्रस्काय । \textbf{ 40} \newline
                  \newline
                                    (ल॒भ्य॒न्ते॒ - ल॒भ॒ते॒ - त्वा॒ष्ट्रान् प॒शूनाल॑भते॒ऽष्टौ च॑) \textbf{(A9)} \newline \newline
                \textbf{ 3.9.10    अनुवाकं   10 -महिमाभिधानौ प्रहौ} \newline
                                \textbf{ TB 3.9.10.1} \newline
                  प्र॒जाप॑ति-रकामयत म॒हान॑न्ना॒दः स्या॒मिति॑ । स ए॒ताव॑श्वमे॒धे म॑हि॒माना॑वपश्यत् । ताव॑गृह्णीत । ततो॒ वै स म॒हान॑न्ना॒दो॑-ऽभवत् । यः का॒मये॑त म॒हान॑न्ना॒दः स्या॒मिति॑ । स ए॒ताव॑श्वमे॒धे म॑हि॒मानौ॑ गृह्णीत । म॒हाने॒वान्ना॒दो भ॑वति ॥ य॒ज॒मा॒न॒दे॒व॒त्या॑ वै व॒पा । राजा॑ महि॒मा । यद्व॒पां म॑हि॒म्नोभ॒यतः॑ परि॒यज॑ति ( ) । यज॑मानमे॒व रा॒ज्येनो॑भ॒यतः॒ परि॑गृह्णाति ॥ पु॒रस्ता᳚-थ्स्वाहाकारा॒ वा अ॒न्ये दे॒वाः । उ॒परि॑ष्टा-थ्स्वाहाकारा अ॒न्ये । ते वा ए॒तेऽश्व॑ ए॒व मेद्ध्य॑ उ॒भये-ऽव॑रुद्ध्यन्ते । यद्व॒पां म॑हि॒म्नोभ॒यतः॑ परि॒यज॑ति । ताने॒वोभया᳚न् प्रीणाति । \textbf{ 41} \newline
                  \newline
                                    (प॒रि॒यज॑ति॒ षट्च॑) \textbf{(A10)} \newline \newline
                \textbf{ 3.9.11    अनुवाकं   11 -शरीरहोमाः स्विष्टकृदादयश्च} \newline
                                \textbf{ TB 3.9.11.1} \newline
                  वै॒श्व॒दे॒वो वा अश्वः॑ । तं ॅयत्प्रा॑जाप॒त्यं कु॒र्यात् । या दे॒वता॒ अपि॑भागाः । ता भा॑ग॒धेये॑न॒ व्य॑द्र्धयेत् । दे॒वता᳚भ्यः स॒मदं॑ दद्ध्यात् । स्ते॒गान्-दꣳष्ट्रा᳚भ्यां म॒ण्डूका॒-ञ्जम्भ्ये॑-भि॒रिति॑ । आज्य॑-मव॒दानं॑ कृ॒त्वा प्रति॑सं॒ख्याय॒-माहु॑ती-र्जुहोति । या ए॒व दे॒वता॒ अपि॑भागाः । ता भा॑ग॒धेये॑न॒ सम॑द्र्धयति । न दे॒वता᳚भ्यः स॒मदं॑ दधाति । \textbf{ 42} \newline
                  \newline
                                \textbf{ TB 3.9.11.2} \newline
                  चतु॑र्दशै॒ता-न॑नुवा॒काञ्जु॑हो॒त्य-न॑न्तरित्यै ॥ प्र॒या॒साय॒ स्वाहेति॑ पञ्चद॒शम् । पञ्च॑दश॒ वा अ॑द्र्धमा॒सस्य॒ रात्र॑यः । अ॒द्र्ध॒मा॒स॒शः स॑म्ॅवथ्स॒र आ᳚प्यते ॥ दे॒वा॒सु॒राः सम्ॅयं॑त्ता आसन्न् । ते᳚ऽब्रुवन्न॒ग्नयः॑ स्विष्ट॒कृतः॑ । अश्व॑स्य॒ मेद्ध्य॑स्य व॒यमु॑द्धा॒र-मुद्ध॑रामहै । अथै॒तान॒भि-भ॑वा॒मेति॑ । ते लोहि॑त॒-मुद॑हरन्त । ततो॑ दे॒वा अभ॑वन्न् \textbf{ 43} \newline
                  \newline
                                \textbf{ TB 3.9.11.3} \newline
                  पराऽसु॑राः । यथ्स्वि॑ष्ट॒कृद्भ्यो॒ लोहि॑तं जु॒होति॒ भ्रातृ॑व्याभिभूत्यै । भव॑त्या॒त्मना᳚ । परा᳚ऽस्य॒ भ्रातृ॑व्यो भवति ॥ गो॒मृ॒ग॒क॒ण्ठेन॑ प्रथ॒मा-माहु॑तिं जुहोति । प॒शवो॒ वै गो॑मृ॒गः । रु॒द्रो᳚ऽग्निः स्वि॑ष्ट॒कृत् । रु॒द्रादे॒व प॒शून॒न्तर्द॑धाति । अथो॒ यत्रै॒षा-ऽऽहु॑तिर्. हू॒यते᳚ । न तत्र॑ रु॒द्रः प॒शून॒भिम॑न्यते । \textbf{ 44} \newline
                  \newline
                                \textbf{ TB 3.9.11.4} \newline
                  अ॒श्व॒श॒फेन॑ द्वि॒तीया॒-माहु॑तिं जुहोति । प॒शवो॒ वा एक॑शफम् । रु॒द्रो᳚ऽग्निः स्वि॑ष्ट॒कृत् । रु॒द्रादे॒व प॒शून॒न्तर्द॑धाति । अथो॒ यत्रै॒षा-ऽऽहु॑तिर.ह॒यते᳚ । न तत्र॑ रु॒द्रः प॒शून॒भिम॑न्यते ॥ अ॒य॒स्मये॑न कम॒ण्डलु॑ना तृ॒तीया᳚म् । आहु॑तिं जुहोत्या-य॒स्यो॑ वै प्र॒जाः । रु॒द्रो᳚ऽग्निः स्वि॑ष्ट॒कृत् । रु॒द्रादे॒व प्र॒जा अ॒न्तर्द॑धाति ( ) । अथो॒ यत्रै॒षा-ऽऽहु॑तिर्. हू॒यते᳚ । न तत्र॑ रु॒द्रः प्र॒जा अ॒भिम॑न्यते । \textbf{ 45} \newline
                  \newline
                                    (द॒धा॒ - त्यभ॑वन् - मन्यते - प्र॒जा अ॒न्तर् द॑धाति॒ द्वे च॑) \textbf{(A11)} \newline \newline
                \textbf{ 3.9.12    अनुवाकं   12 -तदुभयहोममध्यवर्त्यश्चस्तोमीयहोमः} \newline
                                \textbf{ TB 3.9.12.1} \newline
                  अश्व॑स्य॒ वा आल॑ब्धस्य॒ मेध॒ उद॑क्रामत् । तद॑श्वस्तो॒मीय॑-मभवत् । यद॑श्वस्तो॒मीयं॑ जु॒होति॑ । समे॑धमे॒वैन॒-माल॑भते ॥ आज्ये॑न जुहोति । मेधो॒ वा आज्य᳚म् । मेधो᳚ऽश्वस्तो॒मीय᳚म् । मेधे॑नै॒वास्मि॒न् मेधं॑ दधाति ॥ षटत्रꣳ॑शतं जुहोति । षटत्रꣳ॑शदक्षरा बृह॒ती \textbf{ 46} \newline
                  \newline
                                \textbf{ TB 3.9.12.2} \newline
                  बार्.ह॑ताः प॒शवः॑ । सा प॑शू॒नां मात्रा᳚ । प॒शूने॒व मात्र॑या॒ सम॑द्र्धयति ॥ ता यद्भूय॑सीर्वा॒ कनी॑यसीर्वा जुहु॒यात् । प॒शून्मात्र॑या॒ व्य॑द्र्धयेत् । षटत्रꣳ॑शतं जुहोति । षटत्रꣳ॑शदक्षरा बृह॒ती । बार्.ह॑ताः प॒शवः॑ । सा प॑शू॒नां मात्रा᳚ । प॒शूने॒व मात्र॑या॒ सम॑द्र्धयति । \textbf{ 47} \newline
                  \newline
                                \textbf{ TB 3.9.12.3} \newline
                  अ॒श्व॒स्तो॒मीयꣳ॑ हु॒त्वा द्वि॒पदा॑ जुहोति । द्वि॒पाद्वै पुरु॑षो॒ द्विप्र॑तिष्ठः । तदे॑नं प्रति॒ष्ठया॒ सम॑द्र्धयति ॥ तदा॑हुः । अ॒श्व॒स्तो॒मीयं॒ पूर्वꣳ॑ होत॒व्या(3)न् द्वि॒पदा(3) इति॑ । अश्वो॒ वा अ॑श्वस्तो॒मीय᳚म् । पुरु॑षो द्वि॒पदाः᳚ । अ॒श्व॒स्तो॒मीयꣳ॑ हु॒त्वा द्वि॒पदा॑ जुहोति । तस्मा᳚द्-द्वि॒पाच्चतु॑ष्पादमत्ति । अथो᳚ द्वि॒पद्ये॒व चतु॑ष्पदः॒ प्रति॑ष्ठापयति ( ) ॥ द्वि॒पदा॑ हु॒त्वा । नान्यामुत्त॑रा॒-माहु॑तिं जुहुयात् । यद॒न्यामुत्त॑रा॒-माहु॑तिं जुहु॒यात् । प्र प्र॑ति॒ष्ठाया᳚श्च्यवेत । द्वि॒पदा॑ अन्त॒तो ज॑होति॒ प्रति॑ष्ठित्यै । \textbf{ 48} \newline
                  \newline
                                    (बृ॒ह॒ - त्य॑द्र्धयति - स्थापयति॒ पञ्च॑ च) \textbf{(A12)} \newline \newline
                \textbf{ 3.9.13    अनुवाकं   13 -संवथ्सरानुष्ठानमिष्टीनाम्} \newline
                                \textbf{ TB 3.9.13.1} \newline
                  प्र॒जाप॑ति-रश्वमे॒ध-म॑सृजत । सो᳚-ऽस्माथ्सृ॒ष्टो-ऽपा᳚क्रामत् । तं ॅय॑ज्ञ्-क्र॒तुभि॒-रन्वै᳚च्छत् । तं ॅय॑ज्ञ्-क्र॒तुभि॒-र्नान्व॑विन्दत् । तमिष्टि॑भि॒-रन्वै᳚च्छत् । तमिष्टि॑भि॒-रन्व॑विन्दत् । तदिष्टी॑ना-मिष्टि॒त्वम् । यथ्सं॑ॅवथ्स॒र-मिष्टि॑भि॒-र्यज॑ते । अश्व॑मे॒व तदन्वि॑च्छति ॥ सा॒वि॒त्रियो॑ भवन्ति \textbf{ 49} \newline
                  \newline
                                \textbf{ TB 3.9.13.2} \newline
                  इ॒यं ॅवै स॑वि॒ता । यो वा अ॒स्यां नश्य॑ति॒ यो नि॒लय॑ते । अ॒स्यां ॅवाव तं ॅवि॑न्दन्ति । न वा इ॒मां कश्च॒नेत्या॑हुः । ति॒र्यङ्नोद्र्ध्वो-ऽत्ये॑तुमर्.ह॒तीति॑ । यथ्सा॑वि॒त्रियो॒ भव॑न्ति । स॒वि॒तृ-प्र॑सूत ए॒वैन॑मिच्छति ॥ ई॒श्व॒रो वा अश्वः॒ प्रमु॑क्तः॒ परां᳚ परा॒वतं॒ गन्तोः᳚ । यथ्सा॒यं धृती᳚र्जु॒होति॑ । अश्व॑स्य॒ यत्यै॒ धृत्यै᳚ । \textbf{ 50} \newline
                  \newline
                                \textbf{ TB 3.9.13.3} \newline
                  यत्प्रा॒तरिष्टि॑भि॒-र्यज॑ते । अश्व॑मे॒व तदन्वि॑च्छति । यथ्सा॒यं धृती᳚र्जु॒होति॑ । अश्व॑स्यै॒व यत्यै॒ धृत्यै᳚ । तस्मा᳚थ्सा॒यं प्र॒जाः क्षे॒म्या॑ भवन्ति ॥ यत्प्रा॒त-रिष्टि॑भि॒-र्यज॑ते । अश्व॑मे॒व तदन्वि॑च्छति । तस्मा॒द्दिवा॑ नष्टै॒ष ए॑ति ॥ यत्प्रा॒तरिष्टि॑भि॒-र्यज॑ते सा॒यं धृती᳚र्जु॒होति॑ । अ॒हो॒रा॒त्राभ्या॑-मे॒वैन॒-मन्वि॑च्छति ( ) । अथो॑ अहोरा॒त्राभ्या॑-मे॒वास्मै॑ योगक्षे॒मं क॑ल्पयति । \textbf{ 51} \newline
                  \newline
                                    (भ॒व॒न्ति॒ - धृत्या॑ - एन॒ मन्वि॑च्छ॒त्येकं॑ च) \textbf{(A13)} \newline \newline
                \textbf{ 3.9.14    अनुवाकं   14 -तास्विष्टिषु ब्राह्मणराजन्ययोर्गानम्} \newline
                                \textbf{ TB 3.9.14.1} \newline
                  अप॒ वा ए॒तस्मा॒च्छ्री रा॒ष्ट्रं क्रा॑मति । यो᳚ऽश्वमे॒धेन॒ यज॑ते । ब्रा॒ह्म॒णौ वी॑णागा॒थिनौ॑ गायतः । श्रि॒या वा ए॒तद् रू॒पम् । यद्वीणा᳚ । श्रिय॑-मे॒वास्मि॒-न्तद्ध॑त्तः । य॒दा खलु॒ वै पुरु॑षः॒ श्रिय॑मश्नु॒ते । वीणा᳚ऽस्मै वाद्यते ॥ तदा॑हुः । यदु॒भौ ब्रा᳚ह्म॒णौ गाये॑ताम् \textbf{ 52} \newline
                  \newline
                                \textbf{ TB 3.9.14.2} \newline
                  प्र॒भ्रꣳशु॑का-ऽस्मा॒च्छ्रीः स्या᳚त् । न वै ब्रा᳚ह्म॒णे श्री र॑मत॒ इति॑ । ब्रा॒ह्म॒णो᳚ऽन्यो गाये᳚त् । रा॒ज॒न्यो᳚ऽन्यः । ब्रह्म॒ वै ब्रा᳚ह्म॒णः । क्ष॒त्रꣳ रा॑ज॒न्यः॑ । तथा॑ हास्य॒ ब्रह्म॑णा च क्ष॒त्रेण॑ चोभ॒यतः॒ श्रीः परि॑गृहीता भवति ॥ तदा॑हुः । यदु॒भौ दिवा॒ गाये॑ताम् । अपा᳚स्माद् रा॒॒ष्ट्रं क्रा॑मेत् \textbf{ 53} \newline
                  \newline
                                \textbf{ TB 3.9.14.3} \newline
                  न वै ब्रा᳚ह्म॒णे रा॒ष्ट्रꣳ र॑मत॒ इति॑ । य॒दा खलु॒ वै राजा॑ का॒मय॑ते । अथ॑ ब्राह्म॒णं जि॑नाति । दिवा᳚ ब्राह्म॒णो गा॑येत् । नक्तꣳ॑ राज॒न्यः॑ । ब्रह्म॑णो॒ वै रू॒पमहः॑ । क्ष॒त्रस्य॒ रात्रिः॑ । तथा॑ हास्य॒ ब्रह्म॑णा च क्ष॒त्रेण॑ चोभ॒यतो॑ रा॒ष्ट्रं परि॑गृहीतं भवति ॥ इत्य॑ददा॒ इत्य॑यजथा॒ इत्य॑पच॒ इति॑ ब्राह्म॒णो गाये᳚त् । इ॒ष्टा॒पू॒र्तं ॅवै ब्रा᳚ह्म॒णस्य॑ \textbf{ 54} \newline
                  \newline
                                \textbf{ TB 3.9.14.4} \newline
                  इ॒ष्टा॒पू॒र्ते-नै॒वैनꣳ॒॒ स सम॑द्र्धयति ॥ इत्य॑जिना॒ इत्य॑युद्ध्यथा॒ इत्य॒मुꣳ स॑ग्रां॒म-म॑ह॒न्निति॑ राज॒न्यः॑ । यु॒द्धं ॅवै रा॑ज॒न्य॑स्य । यु॒द्धेनै॒वैनꣳ॒॒ स सम॑द्र्धयति ॥ अक्लृ॑प्ता॒ वा ए॒तस्य॒र्तव॒ इत्या॑हुः । यो᳚ऽश्वमे॒धेन॒ यज॑त॒ इति॑ । ति॒स्रो᳚ऽन्यो गाय॑ति ति॒स्रो᳚ऽन्यः । षट्थ्संप॑द्यन्ते । षड्वा ऋ॒तवः॑ । ऋ॒तूने॒वास्मै॑ कल्पयतः ( ) ॥ ताभ्याꣳ॑ सꣳ॒॒स्थाया᳚म् । अ॒नो॒यु॒क्ते च॑ श॒ते च॑ ददाति । श॒तायुः॒ पुरु॑षः श॒तेन्द्रि॑यः । आयु॑ष्ये॒वेन्द्रि॒ये प्रति॑तिष्ठति । \textbf{ 55} \newline
                  \newline
                                    (गाये॑ताम् - क्रामेद् - ब्राह्म॒णस्य॑ - कल्पयतश्च॒त्वारि॑ च) \textbf{(A14)} \newline \newline
                \textbf{ 3.9.15    अनुवाकं   15 -अवभृथहोमविशेषाः} \newline
                                \textbf{ TB 3.9.15.1} \newline
                  सर्वे॑षु॒ वा ए॒षु लो॒केषु॑ मृ॒त्यवो॒ऽन्वाय॑त्ताः । तेभ्यो॒ यदाहु॑ती॒र्न जु॑हु॒यात् । लो॒के लो॑क एनं मृ॒त्युर्वि॑न्देत् । मृ॒त्यवे॒ स्वाहा॑ मृ॒त्यवे॒ स्वाहेत्य॑भि पू॒र्वमाहु॑ती-र्जुहोति । लो॒काल्लो॑कादे॒व मृ॒त्युमव॑यजते । नैनं॑ ॅलो॒के ला॑के मृ॒त्युर्वि॑न्दति ॥ यद॒मुष्मै॒ स्वाहा॒ऽमुष्मै॒ स्वाहेति॒ जुह्व॑थ्सं॒ चक्षी॑त । ब॒हुं मृ॒त्युम॒मित्रं॑ कुर्वीत । मृ॒त्यवे॒ स्वाहेत्येक॑स्मा ए॒वैका᳚-ञ्जुहुयात् । एको॒ वा अ॒मुष्मि॑न् ॅला॒के मृ॒त्युः \textbf{ 56} \newline
                  \newline
                                \textbf{ TB 3.9.15.2} \newline
                  अ॒श॒न॒या॒ मृ॒त्युरे॒व । तमे॒वामुष्मि॑न् ॅलो॒केऽव॑यजते ॥ भ्रू॒ण॒ह॒त्यायै॒ स्वाहेत्य॑वभृ॒थ आहु॑तिं जुहोति । भ्रू॒ण॒ह॒त्यामे॒ वा व॑यजते ॥ तदा॑हुः । यद्भ्रू॑णह॒त्या ऽपा॒त्र्याऽथ॑ । कस्मा᳚द्-य॒ज्ञेऽपि॑ क्रियत॒ इति॑ ॥ अमृ॑त्यु॒र्वा अ॒न्यो भ्रू॑णह॒त्याया॒ इत्या॑हुः । भ्रू॒ण॒ह॒त्या वाव मृ॒त्युरिति॑ । यद्भ्रू॑णह॒त्यायै॒ स्वाहेत्य॑वभृ॒थ आहुतिं॑ जु॒होति॑ । \textbf{ 57} \newline
                  \newline
                                \textbf{ TB 3.9.15.3} \newline
                  मृ॒त्युमे॒ वा हु॑त्या तर्पयि॒त्वा प॑रि॒पाणं॑ कृ॒त्वा । भ्रू॒ण॒घ्ने भे॑ष॒जं क॑रोति ॥ ए॒ताꣳ ह॒ वै मु॑ण्डि॒भ औ॑दन्य॒वः । भ्रू॒ण॒ह॒त्यायै॒ प्राय॑श्चित्तिं ॅवि॒दाञ्च॑कार । यो हा॒स्यापि॑ प्र॒जायां᳚ ब्राह्म॒णꣳ हन्ति॑ । सर्व॑स्मै॒ तस्मै॑ भेष॒जं क॑रोति ॥ जु॒म्ब॒काय॒ स्वाहेत्य॑वभृ॒थ उ॑त्त॒मामाहु॑तिं जुहोति । वरु॑णो॒ वै जु॑म्ब॒कः । अ॒न्त॒त ए॒व वरु॑ण॒-मव॑यजते ॥ ख॒ल॒तेर्वि॑क्लि॒धस्य॑ शु॒क्लस्य॑ पिङ्गा॒क्षस्य॑ मू॒द्र्धञ्जु॑होति ( ) । ए॒तद्वै वरु॑णस्य रू॒पं । रू॒पेणै॒व वरु॑ण॒मव॑यजते । \textbf{ 58} \newline
                  \newline
                                    (लो॒के मृ॒त्युर् - जु॒होति॑ - मू॒द्र्धञ्जु॑होति॒ द्वे च॑) \textbf{(A15)} \newline \newline
                \textbf{ 3.9.16    अनुवाकं   16 -उपाकरणमन्त्रव्याख्यानादिः} \newline
                                \textbf{ TB 3.9.16.1} \newline
                  वा॒रु॒णो वा अश्वः॑ । तं दे॒वत॑या॒ व्य॑द्र्धयति । यत्प्रा॑जाप॒त्यं क॒रोति॑ । नमो॒ राज्ञे॒ नमो॒ वरु॑णा॒येत्या॑ह । वा॒रु॒णो वा अश्वः॑ । स्वयै॒वैनं॑ दे॒वत॑या॒ सम॑द्र्धयति ॥ नमोऽश्वा॑य॒ नमः॑ प्र॒जाप॑तय॒ इत्या॑ह । प्रा॒जा॒प॒त्यो वा अश्वः॑ । स्वयै॒वैनं॑ दे॒वत॑या॒ सम॑द्र्धयति ॥ नमोऽधि॑पतय॒ इत्या॑ह \textbf{ 59} \newline
                  \newline
                                \textbf{ TB 3.9.16.2} \newline
                  धर्मो॒ वा अधि॑पतिः । धर्म॑मे॒वा-व॑रुन्धे ॥ अधि॑पति-र॒स्यधि॑पतिं मा कु॒र्वधि॑पतिर॒हं प्र॒जानां᳚ भूयास॒-मित्या॑ह । अधि॑पति-म॒वैनꣳ॑ समा॒नानां᳚ करोति ॥ मां ध॑हि॒ मयि॑ ध॒हीत्या॑ह । आ॒शिष॑मे॒वैतामाशा᳚स्ते ॥ उ॒पाकृ॑ताय॒ स्वाहेत्यु॒पाकृ॑ते जुहोति । आल॑ब्धाय॒ स्वाहेति॒ नियु॑क्ते जुहोति । हु॒ताय॒ स्वाहेति॑ हु॒ते जु॑होति । ए॒षां ॅलो॒काना॑-म॒भिजि॑त्यै । \textbf{ 60} \newline
                  \newline
                                \textbf{ TB 3.9.16.3} \newline
                  प्र वा ए॒ष ए॒भ्यो लो॒केभ्य॑श्च्यवते । यो᳚ऽश्वमे॒धेन॒ यज॑ते । आ॒ग्ने॒य-मै᳚न्द्रा॒ग्न-मा᳚श्वि॒नम् । तान् प॒शूनाल॑भते॒ प्रति॑ष्ठित्यै ॥ यदा᳚ऽग्ने॒यो भव॑ति । अ॒ग्निः सर्वा॑ दे॒वताः᳚ । दे॒वता॑ ए॒वाव॑रुन्धे ॥ ब्रह्म॒ वा अ॒ग्निः । क्ष॒त्रमिन्द्रः॑ । यदै᳚न्द्रा॒ग्नो भव॑ति \textbf{ 61} \newline
                  \newline
                                \textbf{ TB 3.9.16.4} \newline
                  ब्र॒ह्म॒क्ष॒त्रे ए॒वाव॑रुन्धे ॥ यदा᳚ऽश्वि॒नो भव॑ति । आ॒शिषा॒मव॑रुद्ध्यै ॥ त्रयो॑ भवन्ति । त्रय॑ इ॒मे लो॒काः । ए॒ष्वे॑व लो॒केषु॒ प्रति॑तिष्ठति ॥ अ॒ग्नये-ऽꣳ॑हो॒मुचे॒-ऽष्टाक॑पाल॒ इति॒ दश॑हविष॒मिष्टिं॒ निर्व॑पति । दशा᳚क्षरा वि॒राट् । अन्नं॑ ॅवि॒राट् । वि॒राजै॒-वान्नाद्य॒-मव॑रुन्धे ( ) ॥ अ॒ग्नेर्म॑न्वे प्रथ॒मस्य॒ प्रचे॑तस॒ इति॑ याज्यानुवा॒क्या॑ भवन्ति सर्व॒त्वाय॑ । \textbf{ 62} \newline
                  \newline
                                    (अधि॑पतय॒ इत्या॑ - हा॒भिजि॑त्या - ऐन्द्रा॒ग्नो भव॑ति - रुन्ध॒ एकं॑ च) \textbf{(A16)} \newline \newline
                \textbf{ 3.9.17    अनुवाकं   17 -अश्वस्य रोगादिनिमित्तं प्रायश्चित्तम्} \newline
                                \textbf{ TB 3.9.17.1} \newline
                  यद्यश्व॑मुप॒तप॑द्वि॒न्देत् । आ॒ग्ने॒य-म॒ष्टाक॑पालं॒ निर्व॑पेत् । सौ॒म्यं च॒रुम् । सा॒वि॒त्र-म॒ष्टाक॑पालम् ॥ यदा᳚ग्ने॒यो भव॑ति । अ॒ग्निः सर्वा॑ दे॒वताः᳚ । दे॒वता॑-भिरे॒वैनं॑ भिषज्यति ॥ यथ्सौ॒म्यो भव॑ति । सोमो॒ वा ओष॑धीनाꣳ॒॒ राजा᳚ । याभ्य॑ ए॒वैनं॑ ॅवि॒न्दति॑ \textbf{ 63} \newline
                  \newline
                                \textbf{ TB 3.9.17.2} \newline
                  ताभि॑रे॒वैनं॑ भिषज्यति ॥ यथ्सा॑वि॒त्रो भव॑ति । स॒वि॒तृप्र॑सूत ए॒वैनं॑ भिषज्यति ॥ ए॒ताभि॑रे॒वैनं॑ दे॒वता॑भिर्-भिषज्यति । अ॒ग॒दो है॒व भ॑वति ॥ पौ॒ष्णं च॒रुं निर्व॑पेत् । यदि॑ श्लो॒णः स्यात् । पू॒षा वै श्लौण्य॑स्य भि॒षक् । स ए॒वैनं॑ भिषज्यति । अश्लो॑णो है॒व भ॑वति । \textbf{ 64} \newline
                  \newline
                                \textbf{ TB 3.9.17.3} \newline
                  रौ॒द्रं च॒रुं निर्व॑पेत् । यदि॑ मह॒ती दे॒वता॑-ऽभि॒मन्ये॑त । ए॒त॒द्दे॒व॒त्यो॑ वा अश्वः॑ । स्वयै॒वैनं॑ दे॒वत॑या भिषज्यति । अ॒ग॒दो है॒व भ॑वति ॥ वै॒श्वा॒न॒रं द्वाद॑श-कपालं॒ निर्व॑पेन्मृगाख॒रे यदि॒ नागच्छे᳚त् । इ॒यं ॅवा अ॒ग्निर्वै᳚श्वान॒रः । इ॒यमे॒वैन॑-म॒र्चिभ्यां᳚ परि॒रोध॒मान॑यति । आ है॒व सुत्य॒मह॑-र्गच्छति ॥ यद्य॑धी॒यात् \textbf{ 65} \newline
                  \newline
                                \textbf{ TB 3.9.17.4} \newline
                  अ॒ग्नये-ऽꣳ॑हो॒मुचे॒ऽष्टाक॑पालः । सौ॒र्यं पयः॑ । वा॒य॒व्य॑ आज्य॑भागः ॥ यज॑मानो॒ वा अश्वः॑ । अꣳह॑सा॒ वा ए॒ष गृ॑ही॒तः । यस्याश्वो॒ मेधा॑य॒ प्रोक्षि॑तो॒-ऽद्ध्येति॑ । यदꣳ॑हो॒मुचे॑ नि॒र्वप॑ति । अꣳह॑स ए॒व तेन॑ मुच्यते ॥ यज॑मानो॒ वा अश्वः॑ । रेत॑सा॒ वा ए॒ष व्यृ॑द्ध्यते \textbf{ 66} \newline
                  \newline
                                \textbf{ TB 3.9.17.5} \newline
                  यस्याश्वो॒ मेधा॑य॒ प्रोक्षि॑तो॒-ऽद्ध्येति॑ । सौ॒र्यꣳ रेतः॑ । यथ्सौ॒र्यं पयो॒ भव॑ति । रेत॑सै॒वैनꣳ॒॒ स सम॑द्र्धयति ॥ यज॑मानो॒ वा अश्वः॑ । गर्भै॒र्वा ए॒ष व्यृ॑द्ध्यते । यस्याश्वो॒ मेधा॑य॒ प्रोक्षि॑तो॒-ऽद्ध्येति॑ । वा॒य॒व्या॑ गर्भाः᳚ । यद्वा॑य॒व्य॑ आज्य॑भागो॒ भव॑ति । गर्भै॑रे॒वैनꣳ॒॒ स सम॑द्र्धयति ( ) ॥ अथो॒ यस्यै॒षा-ऽश्व॑मे॒धे प्राय॑श्चित्तः क्रि॒यते᳚ । इ॒ष्ट्वा वसी॑यान् भवति । \textbf{ 67} \newline
                  \newline
                                    (वि॒न्द - त्यश्लो॑णो है॒व भ॑व - त्यधी॒या - दृ॑द्ध्यते॒ - गर्भै॑रे॒वैनꣳ॒॒ स सम॑द्र्धयति॒ द्वे च॑) \textbf{(A17)} \newline \newline
                \textbf{ 3.9.18    अनुवाकं   18 -ब्रह्मौदना उच्यन्ते} \newline
                                \textbf{ TB 3.9.18.1} \newline
                  तदा॑हुः । द्वाद॑श ब्रह्मौद॒नान् थ्सꣳस्थि॑त॒ निर्व॑पेत् । द्वा॒द॒शभि॒र्वेष्टि॑भि॒-र्यजे॒तेति॑ ॥ यदिष्टि॑-भि॒र्यजे॑त । उ॒प॒नामु॑क एनं ॅय॒ज्ञ्ः स्या᳚त् । पापी॑याꣳ॒॒स्तु स्या᳚त् । आ॒प्तानि॒ वा ए॒तस्य॒ छन्दाꣳ॑सि । य ई॑जा॒नः । तानि॒ क ए॒ताव॑दाशु॒ पुनः॒ प्रयु॑ञ्जी॒तेति॑ । सर्वा॒ वै सꣳस्थि॑ते य॒ज्ञे वागा᳚प्यते \textbf{ 68} \newline
                  \newline
                                \textbf{ TB 3.9.18.2} \newline
                  साऽऽप्ता भ॑वति या॒तया᳚म्नी । क्रू॒रीकृ॑तेव॒ हि भव॒त्यरु॑ष्कृता । सा न पुनः॑ प्र॒युज्येत्या॑हुः ॥ द्वाद॑शै॒व ब्र॑ह्मौद॒नान् थ्सꣳस्थि॑ते॒ निर्व॑पेत् । प्र॒जाप॑ति॒र्वा ओ॑द॒नः । य॒ज्ञ्ः प्र॒जाप॑तिः । उ॒प॒नामु॑क एनं ॅय॒ज्ञो भ॑वति । न पापी॑यान् भवति ॥ द्वाद॑श भवन्ति । द्वाद॑श॒ मासाः᳚ सम्ॅवथ्स॒रः ( ) । स॒म्ॅव॒थ्स॒र ए॒व प्रति॑तिष्ठति । \textbf{ 69} \newline
                  \newline
                                    (आ॒प्य॒ते॒ - स॒म्ॅव॒थ्स॒र एकं॑ च) \textbf{(A18)} \newline \newline
                \textbf{ 3.9.19    अनुवाकं   19 -विभुत्वादिभिः द्वादशभिर्गुणौरश्वमेधप्रशंसा} \newline
                                \textbf{ TB 3.9.19.1} \newline
                  ए॒ष वै वि॒भूर्नाम॑ य॒ज्ञ्ः । सर्वꣳ॑ ह॒ वै तत्र॑ वि॒भु भ॑वति । यत्रै॒तेन॑ य॒ज्ञेन॒ यज॑न्ते ॥ ए॒ष वै प्र॒भूर्नाम॑ य॒ज्ञ्ः । सर्वꣳ॑ ह॒ वै तत्र॑ प्र॒भु भ॑वति । यत्रै॒तेन॑ य॒ज्ञेन॒ यज॑न्ते ॥ ए॒ष वा ऊर्ज॑स्वा॒न्नाम॑ य॒ज्ञ्ः । सर्वꣳ॑ ह॒ वै तत्रोर्ज॑स्व-द्भवति । यत्रै॒तेन॑ य॒ज्ञेन॒ यज॑न्ते ॥ ए॒ष वै पय॑स्वा॒न्नाम॑ य॒ज्ञ्ः \textbf{ 70} \newline
                  \newline
                                \textbf{ TB 3.9.19.2} \newline
                  सर्वꣳ॑ ह॒ वै तत्र॒ पय॑स्व-द्भवति । यत्रै॒तेन॑ य॒ज्ञेन॒ यज॑न्ते ॥ ए॒ष वै विधृ॑तो॒ नाम॑ य॒ज्ञ्ः । सर्वꣳ॑ ह॒ वै तत्र॒ विधृ॑तं भवति । यत्रै॒तेन॑ य॒ज्ञेन॒ यज॑न्ते ॥ ए॒ष वै व्यावृ॑त्तो॒ नाम॑ य॒ज्ञ्ः । सर्वꣳ॑ ह॒ वै तत्र॒ व्यावृ॑त्तं भवति । यत्रै॒तेन॑ य॒ज्ञेन॒ यज॑न्ते ॥ ए॒ष वै प्रति॑ष्ठितो॒ नाम॑ य॒ज्ञ्ः । सर्वꣳ॑ ह॒ वै तत्र॒ प्रति॑ष्ठितं भवति \textbf{ 71} \newline
                  \newline
                                \textbf{ TB 3.9.19.3} \newline
                  यत्रै॒तेन॑ य॒ज्ञेन॒ यज॑न्ते ॥ ए॒ष वै ते॑ज॒स्वी नाम॑ य॒ज्ञ्ः । सर्वꣳ॑ ह॒ वै तत्र॑ तेज॒स्वि भ॑वति । यत्रै॒तेन॑ य॒ज्ञेन॒ यज॑न्ते ॥ ए॒॒ष वै ब्र॑ह्मवर्च॒सी नाम॑ य॒ज्ञ्ः । आ ह॒ वै तत्र॑ ब्राह्म॒णो ब्र॑ह्मवर्च॒सी जा॑यते । यत्रै॒तेन॑ य॒ज्ञेन॒ यज॑न्ते ॥ ए॒ष वा अ॑तिव्या॒धी नाम॑ य॒ज्ञ्ः । आ ह॒ वै तत्र॑ राज॒न्यो॑-ऽतिव्या॒धी जा॑यते । यत्रै॒तेन॑ य॒ज्ञेन॒ यज॑न्ते ( ) ॥ ए॒ष वै दी॒र्घो नाम॑ य॒ज्ञ्ः । दी॒र्घायु॑षो ह॒ वै तत्र॑ मनु॒ष्या॑ भवन्ति । यत्रै॒तेन॑ य॒ज्ञेन॒ यज॑न्ते ॥ ए॒ष वै क्लृ॒प्तो नाम॑ य॒ज्ञ्ः । कल्प॑ते ह॒ वै तत्र॑ प्र॒जाभ्यो॑ योगक्षे॒मः । यत्रै॒तेन॑ य॒ज्ञेन॒ यज॑न्ते । \textbf{ 72} \newline
                  \newline
                                                        \textbf{special korvai} \newline
              (ए॒ष वै वि॒भूः प्र॒भूरूर्ज॑स्वा॒न् पय॑स्वा॒न्॒. विधृ॑तो॒ व्यावृ॑त्तः॒ प्रति॑ष्ठितस्तेज॒स्वी ब्र॑ह्मवर्च॒स्य॑तिव्या॒धी दी॒र्घः क्लु॒प्तो द्वाद॑श) \newline
                                (पय॑स्वा॒न्नाम॑ य॒ज्ञ्ः - प्रति॑ष्ठितं भवति॒ - यत्रै॒तेन॑ य॒ज्ञेन॒ यज॑न्ते॒ षट्च॑) \textbf{(A19)} \newline \newline
                \textbf{ 3.9.20    अनुवाकं   20 -अश्वसंज्ञ्पनप्रकारः} \newline
                                \textbf{ TB 3.9.20.1} \newline
                  ता॒र्प्येणाश्वꣳ॒॒ संज्ञ्॑पयन्ति । य॒ज्ञो वै ता॒र्प्यम् । य॒ज्ञेनै॒वैनꣳ॒॒ सम॑द्र्धयन्ति ॥ या॒मेन॒ साम्ना᳚ प्रस्तो॒ता-ऽनूप॑तिष्ठते । य॒म॒लो॒क-मे॒वैनं॑ गमयति ॥ ता॒र्प्ये च॑ कृत्त्यधीवा॒से चाश्वꣳ॒॒ संज्ञ्॑पयन्ति । ए॒तद्वै प॑शू॒नाꣳ रू॒पम् । रू॒पेणै॒व प॒शूनव॑रुन्धे ॥ हि॒र॒ण्य॒क॒शि॒पु भ॑वति । तेज॒सो-ऽव॑रुद्ध्यै । \textbf{ 73} \newline
                  \newline
                                \textbf{ TB 3.9.20.2} \newline
                  रु॒क्मो भ॑वति । सु॒व॒र्गस्य॑ लो॒कस्या-नु॑ख्यात्यै ॥ अश्वो॑ भवति । प्र॒जाप॑ते॒-राप्त्यै᳚ ॥ अ॒स्य वै लो॒कस्य॑ रू॒पं ता॒र्प्यम् । अ॒न्तरि॑क्षस्य कृत्त्यधीवा॒सः । दि॒वो हि॑रण्यकशि॒पु । आ॒दि॒त्यस्य॑ रु॒क्मः । प्र॒जाप॑ते॒रश्वः॑ । इ॒ममे॒व लो॒कं ता॒र्प्येणा᳚प्नोति ( ) \textbf{ 74} \newline
                  \newline
                                \textbf{ TB 3.9.20.3} \newline
                  अ॒न्तरि॑क्षं कृत्यधीवा॒सेन॑ । दिवꣳ॑ हिरण्यकशि॒पुना᳚ । आ॒दि॒त्यꣳ रु॒क्मेण॑ । अश्वे॑नै॒व मेद्ध्ये॑न प्र॒जाप॑तेः॒ सायु॑ज्यꣳ सलो॒कता॑-माप्नोति ॥ ए॒तासा॑मे॒व दे॒वता॑नाꣳ॒॒ सायु॑ज्यम् । सा॒र्ष्टिताꣳ॑ समान लो॒कता॑माप्नोति । यो᳚ऽश्वमे॒धेन॒ यज॑ते । य उ॑ चैनमे॒वं ॅवेद॑ । \textbf{ 75} \newline
                  \newline
                                    (अव॑रुद्ध्या - आप्नोत्य॒ -+ष्टौ च॑) \textbf{(A20)} \newline \newline
                \textbf{ 3.9.21    अनुवाकं   21 -उत्तरवेद्युपवापः} \newline
                                \textbf{ TB 3.9.21.1} \newline
                  आ॒दि॒त्याश्चाङ्गि॑रसश्च सुव॒र्गे लो॒के᳚ऽस्पद्र्धन्त । तेऽङ्गि॑रस आदि॒त्येभ्यः॑ । अ॒मुमा॑दि॒त्य-मश्वꣳ॑ श्वे॒तं भू॒तं दक्षि॑णा-मनयन्न् । ते᳚ऽब्रुवन्न् । यं नोऽन᳚ष्ट । स वर्यो॑ऽभू॒दिति॑ । तस्मा॒दश्वꣳ॒॒ सव॒र्येत्या-ह्व॑यन्ति । तस्मा᳚द्य॒ज्ञे वरो॑ दीयते ॥ यत्प्र॒जाप॑ति॒-राल॒ब्धो-ऽश्वो-ऽभ॑वत् । तस्मा॒दश्वो॒ नाम॑ । \textbf{ 76} \newline
                  \newline
                                \textbf{ TB 3.9.21.2} \newline
                  यच्छ्वय॒दरु॒रासी᳚त् । तस्मा॒दर्वा॒ नाम॑ ॥ यथ्स॒द्यो वाजा᳚न् थ्स॒मज॑यत् । तस्मा᳚द्वा॒जी नाम॑ ॥ यदसु॑राणां ॅलो॒कानाद॑त्त । तस्मा॑दादि॒त्यो नाम॑ ॥ अ॒ग्निर्वा अ॑श्वमे॒धस्य॒ योनि॑रा॒यत॑नम् । सूर्यो॒-ऽग्नेर्योनि॑-रा॒यत॑नम् । यद॑श्वमे॒धे᳚-ऽग्नौ चित्य॑ उत्तरवे॒दि-मु॑प॒वप॑ति । योनि॑-मन्तमे॒वैन॑-मा॒यत॑नवन्तं करोति । \textbf{ 77} \newline
                  \newline
                                \textbf{ TB 3.9.21.3} \newline
                  योनि॑-माना॒यत॑नवान् भवति । स ए॒वं ॅवेद॑ ॥ प्रा॒णा॒पा॒नौ वा ए॒तौ दे॒वाना᳚म् । यद॑र्काश्वमे॒धौ । प्रा॒णा॒पा॒नावे॒वाव॑रुन्धे ॥ ओजो॒ बलं॒ ॅवा ए॒तौ दे॒वाना᳚म् । यद॑र्काश्वमे॒धौ । ओजो॒ बल॑मे॒वाव॑रुन्धे ॥ अ॒ग्निर्वा अ॑श्वमे॒धस्य॒ योनि॑-रा॒यत॑नम् । सूर्यो॒-ऽग्नेर्योनि॑-रा॒यत॑नम् ( ) । यद॑श्वमे॒धे᳚-ऽग्नौ चित्य॑ उत्तरवे॒दिं चि॒नोति॑ । ताव॑र्काश्वमे॒धौ । अ॒र्का॒श्व॒मे॒धावे॒वा व॑रुन्धे । अथो॑ अर्काश्व-मे॒धयो॑रे॒व प्रति॑तिष्ठति । \textbf{ 78} \newline
                  \newline
                                    (नाम॑ - करोति॒ - सूर्यो॒ऽग्नेर्योनि॑रा॒यत॑नम् च॒त्वारि॑ च) \textbf{(A21)} \newline \newline
                \textbf{ 3.9.22    अनुवाकं   22 -ऋषभालम्भः} \newline
                                \textbf{ TB 3.9.22.1} \newline
                  प्र॒जाप॑तिं॒ ॅवै दे॒वाः पि॒तर᳚म् । प॒शुं भू॒तं मेधा॒याल॑भन्त । तमा॒लभ्योपा॑वसन्न् । प्रा॒तर्यष्टा᳚स्मह॒ इति॑ । एकं॒ ॅवा ए॒तद्दे॒वाना॒महः॑ । यथ्स॑म्ॅवथ्स॒रः । तस्मा॒दश्वः॑ पु॒रस्ता᳚-थ्सम्ॅवथ्स॒र आल॑भ्यते ॥ यत्प्र॒जाप॑ति॒-राल॒ब्धो-ऽश्वोऽभ॑वत् । तस्मा॒दश्वः॑ । यथ्स॒द्यो मेधोऽभ॑वत् \textbf{ 79} \newline
                  \newline
                                \textbf{ TB 3.9.22.2} \newline
                  तस्मा॑दश्वमे॒धः ॥ वेदु॒को-ऽश्व॑मा॒शुं भ॑वति । य ए॒वं वेद॑ ॥ यद्वै तत्प्र॒जाप॑ति॒-राल॒ब्धो-ऽश्वोऽभ॑वत् । तस्मा॒दश्वः॑ प्र॒जाप॑तेः पशू॒ना-मनु॑रूपतमः ॥ आऽस्य॑ पु॒त्रः प्रति॑रूपो जायते । य ए॒वं ॅवेद॑ ॥ सर्वा॑णि भू॒तानि॑ स॒भृंत्याल॑भते । समे॑नं दे॒वा-स्तेज॑से ब्रह्मवर्च॒साय॑ भरन्ति । यो᳚ऽश्वमे॒धेन॒ यज॑ते \textbf{ 80} \newline
                  \newline
                                \textbf{ TB 3.9.22.3} \newline
                  य उ॑ चैनमे॒वं ॅवेद॑ ॥ ए॒तद्वै तद्दे॒वा ए॒तां दे॒वता᳚म् । प॒शुं भू॒तं मेधा॒याल॑भन्त । य॒ज्ञ्म॒व । य॒ज्ञेन॑ य॒ज्ञ्म॑यजन्त दे॒वाः । का॒म॒प्रं य॒ज्ञ्म॑ कुर्वत । ते॑ऽमृत॒त्वम॑कामयन्त । त॑ऽमृत॒त्व-म॑गच्छन्न् ॥ यो᳚ऽश्वमे॒धेन॒ यज॑ते । दे॒वाना॑मे॒वा-य॑नेनैति \textbf{ 81} \newline
                  \newline
                                \textbf{ TB 3.9.22.4} \newline
                  प्रा॒जा॒प॒त्येनै॒व य॒ज्ञेन॑ यजते काम॒प्रेण॑ । अपु॑नर्मारमे॒व ग॑च्छति ॥ ए॒तस्य॒ वै रू॒पेण॑ पु॒रस्ता᳚-त्प्राजाप॒त्य-मृ॑ष॒भं तू॑प॒रंब॑हुरू॒प-माल॑भते । सर्वे᳚भ्यः॒ कामे᳚भ्यः । सर्व॒स्याप्त्यै᳚ । सर्व॑स्य॒ जित्यै᳚ ॥ सर्व॑मे॒व तेना᳚प्नोति । सर्वं॑ जयति । यो᳚ऽश्वमे॒धेन॒ यज॑ते । य उ॑ चैनमे॒वं ॅवेद॑ ( ) । \textbf{ 82} \newline
                  \newline
                                    (मेधोऽभ॑व॒द् - यज॑त - एति॒ - वेद॑) \textbf{(A22)} \newline \newline
                \textbf{ 3.9.23    अनुवाकं   23 -अश्वावयवेषूपासनम्} \newline
                                \textbf{ TB 3.9.23.1} \newline
                  यो वा अश्व॑स्य॒ मेद्ध्य॑स्य॒ लोम॑नी॒ वेद॑ । अश्व॑स्यै॒व मेद्ध्य॑स्य॒ लोमं॑ ॅलोमञ्जुहोति । अ॒हो॒रा॒त्रे वा अश्व॑स्य॒ मेद्ध्य॑स्य॒ लोम॑नी । यथ्सा॒यं प्रा॑तर्जु॒होति॑ । अश्व॑स्यै॒व मेद्ध्य॑स्य॒ लोमं॑ ॅलोमञ्जुहोति । ए॒तद॑नुकृति हस्म॒ वै पु॒रा । अश्व॑स्य॒ मेद्ध्य॑स्य॒ लोमं॑ ॅलोमञ्जुह्वति ॥ यो वा अश्व॑स्य॒ मेद्ध्य॑स्य प॒दे वेद॑ । अश्व॑स्यै॒व मेद्ध्य॑स्य प॒दे प॑दे जुहोति । द॒र्॒.श॒पू॒र्ण॒मा॒सौ वा अश्व॑स्य॒ मेद्ध्य॑स्य प॒दे \textbf{ 83} \newline
                  \newline
                                \textbf{ TB 3.9.23.2} \newline
                  यद्द॑र्.शपूर्णमा॒सौ यज॑ते । अश्व॑स्यै॒व मेद्ध्य॑स्य प॒दे प॑दे जुहोति । ए॒तद॑नुकृति ह स्म॒ वै पु॒रा । अश्व॑स्य॒ मेद्ध्य॑स्य प॒दे प॑दे जुह्वति ॥ यो वा अश्व॑स्य॒ मेद्ध्य॑स्य वि॒वर्त॑नं॒ ॅवेद॑ । अश्व॑स्यै॒व मेद्ध्य॑स्य वि॒वर्त॑ने विवर्तने जुहोति । अ॒सौ वा आ॑दि॒त्योऽश्वः॑ । स आ॑हव॒नीय॒-माग॑च्छति । तद्विव॑र्तते । यद॑ग्निहो॒त्रं जु॒होति॑ ( ) । अश्व॑स्यै॒व मेद्ध्य॑स्य वि॒वर्त॑ने विवर्तने जुहोति । ए॒तद॑नुकृति ह स्म॒ वै पु॒रा । अश्व॑स्य॒ मेद्ध्य॑स्य वि॒वर्त॑ने विवर्तने जुह्वति । \textbf{ 84} \newline
                  \newline
                                    (प॒दे - अ॑ग्निहो॒त्रं जु॒होति॒ त्रीणि॑ च) \textbf{(A23)} \newline \newline
                \textbf{PrapAtaka Korvai with starting  words of 1 to 23 anuvAkams :-} \newline
        (प्र॒जाप॑ति॒स्तम॑ष्टाद॒शिभिः॑ - प्र॒जाप॑तिरकामयतो॒भा - व॒स्मै - यु॒ञ्जन्ति॒-तेज॒सा - ऽप॑ प्रा॒णा - अप॒ श्रीरू॒द्र्ध्वम् - प्र॒जाप॑तिः प्रे॒णाऽनु॑ - प्रथ॒मेन॑ - प्र॒जाप॑तिरकामयत म॒हान्. - वै᳚श्वदे॒वो वा अश्वो - ऽश्व॑स्य - प्र॒जाप॑ति॒स्तं ॅय॑ज्ञ्क्र॒तुभि॒ - रप॒ श्रीर्ब्रा᳚ह्म॒णौ - सर्वे॑षु- वारु॒णो - यद्यश्व॒म् - तदा॑हु- रे॒ष वै वि॒भू - स्ता॒र्प्येणा॑- दि॒त्याः - प्र॒जाप॑तिं पि॒तरं॒ - ॅयो वा अश्व॑स्य॒ मेद्ध्य॑स्य॒ लोम॑नी॒ त्रयो॑विꣳशतिः) \newline

        \textbf{korvai with starting words of 1, 11, 21 series of daSinis :-} \newline
        (प्र॒जाप॑तिर॒ - स्मिन् ॅलो॒के - उ॑त्तर॒तः - श्रिय॑मे॒व - प्र॒जाप॑तिरकामयत म॒हान्.-यत् प्रा॒त - प्र वा ए॒ष ए॒भ्यो लो॒केभ्यः॒- सर्वꣳ॑ ह॒ वै तत्र॒ पय॑स्व॒द्य -उ॑ चैनमे॒वं ॅवेद॑ च॒त्वार्यशी॑तिः) \newline

        \textbf{first and last  word 3rd aShtakam 9th prapATakam :-} \newline
        (प्र॒जाप॑तिरश्वमे॒धं - जु॑ह्वति) \newline 

       

        ॥ हरिः॑ ॐ ॥
॥ कृष्ण यजुर्वेदीय तैत्तिरीय ब्राह्मणे तृतीयाष्टके नवम: प्रपाठकः समाप्तः ॥
++++++++++++++++++++ \newline
        \pagebreak
        
        
        
     \addcontentsline{toc}{section}{ 3.10    तैत्तिरीय यजुर्ब्राह्मणे काठके प्रथमः प्रश्नः( सावित्रचयनम्)}
     \markright{ 3.10    तैत्तिरीय यजुर्ब्राह्मणे काठके प्रथमः प्रश्नः( सावित्रचयनम्) \hfill https://www.vedavms.in \hfill}
     \section*{ 3.10    तैत्तिरीय यजुर्ब्राह्मणे काठके प्रथमः प्रश्नः( सावित्रचयनम्) }
                \textbf{ 3.10.1    अनुवाकं   1 -अनुवाकेषु नवसु लेखासु नाभ्यां चेष्टकोपधानं स्वयमातृण्णोपधानं चोच्यते} \newline
                                \textbf{ TB 3.10.1.1} \newline
                  सं॒ज्ञानं॑ ॅवि॒ज्ञानं॑ प्र॒ज्ञानं॑ जा॒नद॑भिजा॒नत् । स॒कंल्प॑मानं प्र॒कल्प॑मान-मुप॒कल्प॑मान॒-मुप॑क्लृप्तं क्लृ॒प्तं ।श्रेयो॒ वसी॑य आ॒यथ्संभू॑तं भू॒तं ॥ चि॒त्रः के॒तुः प्र॒भाना॒भान्थ् स॒भांन् । ज्योति॑ष्माꣳ॒॒-स्तेज॑स्वाना॒तपꣳ॒॒-स्तप॑न्नभि॒तपन्न्॑ । रो॒च॒नो रोच॑मानः-शोभ॒नः-शोभ॑मानः क॒ल्याणः॑ ॥ दर्.शा॑ दृ॒ष्टा द॑र्.श॒ता वि॒श्वरू॑पा सुदर्.श॒ना । आ॒प्याय॑माना॒ प्याय॑माना॒-प्याया॑ सू॒नृतेरा᳚ । आ॒पूर्य॑माणा॒ पूर्य॑माणा पू॒रय॑न्ती पू॒र्णा पौ᳚र्णमा॒सी ॥ दा॒ता प्र॑दा॒ताऽऽन॒न्दो मोदः॑ प्रमो॒दः \textbf{ 1} \newline
                  \newline
                                \textbf{ TB 3.10.1.2} \newline
                  आ॒वे॒शय॑-न्निवे॒शयन्᳚थ्स॒म्ॅवेश॑नः॒ सꣳशा᳚न्तः शा॒न्तः । आ॒भव॑न् प्र॒भवन्᳚-थ्स॒भंव॒न्-थ्संभू॑तो भू॒तः ॥प्रस्तु॑तं॒ ॅविष्टु॑तꣳ॒॒ सꣳस्तु॑तं क॒ल्याणं॑ ॅवि॒श्वरू॑पं । शु॒क्रम॒मृतं॑ तेज॒स्वि तेजः॒ समि॑द्धं । अ॒रु॒णं भा॑नु॒मन् मरी॑चिमदभि॒तप॒त् तप॑स्वत् ॥ स॒वि॒ता प्र॑सवि॒ता दी॒प्तो दी॒पय॒न् दीप्य॑मानः । ज्वलं॑ ज्वलि॒ता तप॑न् वि॒तपन्᳚-थ्स॒न्तपन्न्॑ । रो॒च॒नो रोच॑मानः शु॒भूंः शुंभ॑मानो वा॒मः ॥ सु॒ता सु॑न्व॒ती प्रसु॑ता सू॒यमा॑ना ऽभिषू॒यमा॑णा । पीती᳚ प्र॒पा स॒पां तृप्ति॑-स्त॒र्पय॑न्ती \textbf{ 2} \newline
                  \newline
                                \textbf{ TB 3.10.1.3} \newline
                  का॒न्ता का॒म्या का॒मजा॒ता ऽऽयु॑ष्मती काम॒दुघा᳚ ॥ अ॒भि॒शा॒स्ता ऽनु॑म॒न्ता-ऽऽन॒न्दो मोदः॑ प्रमो॒दः । आ॒सा॒दय॑न् निषा॒दयन्᳚-थ्सꣳ॒॒साद॑नः॒ सꣳस॑न्नः स॒न्नः । आ॒भू र्वि॒भूः प्र॒भूः श॒भूं र्भुवः॑ ॥ प॒वित्रं॑ पवयि॒ष्यन् पू॒तो मेद्ध्यः॑ । यशो॒ यश॑स्वा-ना॒युर॒मृतः॑ । जी॒वो जी॑वि॒ष्यन् थ्स्व॒र्गो लो॒कः ॥ सह॑स्वा॒न् थ्सही॑या॒-नोज॑स्वा॒न् थ्सह॑मानः । जय॑न्न-भि॒जयन्᳚-थ्सु॒द्रवि॑णो द्रविणो॒दाः । आ॒र्द्रप॑वित्रो॒ हरि॑केशो॒ मोदः॑ प्रमो॒दः । \textbf{ 3} \newline
                  \newline
                                \textbf{ TB 3.10.1.4} \newline
                  अ॒रु॒णो॑ ऽरु॒णर॑जाः पु॒ण्डरी॑को विश्व॒जिद॑-भि॒जित् । आ॒र्द्रः पिन्व॑मा॒नो ऽन्न॑वा॒न्-रस॑वा॒-निरा॑वान् । स॒र्वौ॒ष॒धः स॑भं॒रो मह॑स्वान् ॥ ए॒ज॒त्का जो॑व॒त्काः । क्षु॒ल्ल॒काः शि॑पिविष्ट॒काः । स॒रि॒स्त्र॒राः सु॒शेर॑वः । अ॒जि॒रासो॑ गमि॒ष्णवः॑ ॥ इ॒दानीं᳚ त॒दानी॑-मे॒तर्.हि॑ क्षि॒प्रम॑जि॒रं । आ॒शु र्नि॑मे॒षः फ॒णो द्रव॑न्नति॒-द्रवन्न्॑ । त्वरꣳ॒॒ स्त्वर॑माण आ॒शुराशी॑याञ्ज॒वः {आ॒शुराशी॑यान् ज॒वः} ( ) ॥ अ॒ग्नि॒ष्टो॒म उ॒क्थ्यो॑ ऽतिरा॒त्रो द्वि॑रा॒त्र-स्त्रि॑रा॒त्र-श्च॑तूरा॒त्रः । अ॒ग्निर्. ऋ॒तुः सूर्य॑ ऋ॒तुश्च॒न्द्रमा॑ ऋ॒तुः ॥ प्र॒जाप॑तिः सम्ॅवथ्स॒रो म॒हान्कः । \textbf{ 4} \newline
                  \newline
                                    (प्र॒मो॒द - स्त॒र्पय॑न्ती - प्रमो॒दो - ज॒वस्त्रीणि॑ च) \textbf{(A1)} \newline \newline
                \textbf{ 3.10.2    अनुवाकं   2 - अनुवाकेषु नवसु लेखासु नाभ्यां चेष्टकोपधानं स्वयमातृण्णोपधानं चोच्यते} \newline
                                \textbf{ TB 3.10.2.1} \newline
                  भूर॒ग्निं च॑ पृथि॒वीं च॒ मां च॑ । त्रीꣳश्च॑ लो॒कान् थ्स॑म्ॅवथ्स॒रं च॑ । प्र॒जाप॑तिस्त्वा सादयतु । तया॑ दे॒वत॑या ऽङ्गिर॒स्वद्-ध्रु॒वा सी॑द ॥ भुवो॑ वा॒युं चा॒न्तरि॑क्षं च॒ मां च॑ । त्रीꣳश्च॑ लो॒कान् थ्स॑म्ॅवथ्स॒रं च॑ । प्र॒जाप॑तिस्त्वा सादयतु । तया॑ दे॒वत॑याऽ ङ्गिंर॒स्वद् ध्रु॒वा सी॑द । स्व॑रादि॒त्यं च॒ दिवं॑ च॒ मां च॑ । त्रीꣳश्च॑ लो॒कान् थ्स॑म्ॅवथ्स॒रं च॑ ( ) । प्र॒जाप॑तिस्त्वा सादयतु । तया॑ दे॒वत॑या ऽङ्गिंर॒स्वद् ध्रु॒वा सी॑द । भूर्भुवः॒ स्व॑-श्च॒न्द्रम॑सं च॒ दिश॑श्च॒ मां च॑ । त्रीꣳश्च॑ लो॒कान् थ्स॑वंथ्स॒रं च॑ । प्र॒जाप॑तिस्त्वा सादयतु । तया॑ दे॒वत॑या ऽङ्गिर॒स्वद् ध्रु॒वा सी॑द । \textbf{ 5} \newline
                  \newline
                                    (स॒म्ॅव॒थ्स॒रं॒ च॒ षट् च॑) \textbf{(A2)} \newline \newline
                \textbf{ 3.10.3    अनुवाकं   3 -अनुवाकेषु नवसु लेखासु नाभ्यां चेष्टकोपधानं स्वयमातृण्णोपधानं चोच्यते} \newline
                                \textbf{ TB 3.10.3.1} \newline
                  त्वमे॒व त्वां ॅवे᳚त्थ॒ यो॑ऽसि॒ सोऽसि॑ । त्वमे॒व त्वाम॑चैषीः । चि॒तश्चासि॒ संचि॑त-श्चास्यग्ने । ए॒तावाꣳ॒॒श्चासि॒ भूयाꣳ॑श्चास्यग्ने ।यत्ते॑ अग्ने॒ न्यू॑नं॒ ॅयदु॒ तेऽति॑रिक्तं । आ॒दि॒त्या-स्तदंङ्गि॑रस-श्चिन्वन्तु । विश्वे॑ ते दे॒वाश्चिति॒-मापू॑रयन्तु । चि॒तश्चासि॒ संचि॑त-श्चास्यग्ने । ए॒तावाꣳ॒॒श्चासि॒ भूयाꣳ॑श्चास्यग्ने । मा ते॑ अग्ने ऽच॒येन॒ माऽति॑ च॒ येनायु॒रावृ॑क्षि ( ) । सव॑र्षां॒ ज्योति॑षां॒ ज्योति॒ र्यद॒दावु॒देति॑ । तप॑सो जा॒तमनि॑-भृष्ट॒मोजः॑ । तत्ते॒ ज्योति॑रिष्टके । तेन॑ मे तप । तेन॑ मे ज्वल । तेन॑ मे दीदिहि । याव॑द्दे॒वाः । याव॒दसा॑ति॒ सूर्यः॑ । याव॑दु॒तापि॒ ब्रह्म॑ । \textbf{ 6} \newline
                  \newline
                                    (आवृ॑क्षि॒ नव॑ च) \textbf{(A3)} \newline \newline
                \textbf{ 3.10.4    अनुवाकं   4 -अनुवाकेषु नवसु लेखासु नाभ्यां चेष्टकोपधानं स्वयमातृण्णोपधानं चोच्यते} \newline
                                \textbf{ TB 3.10.4.1} \newline
                  स॒म्ॅव॒थ्स॒रो॑ऽसि परिवथ्स॒रो॑ऽसि । इ॒दा॒व॒थ्स॒रो॑ऽसी-दुवथ्स॒रो॑ऽसि । इ॒द्व॒थ्स॒रो॑ऽसि वथ्स॒रो॑ऽसि । तस्य॑ ते वस॒न्तः शिरः॑ । ग्री॒ष्मो दक्षि॑णः प॒क्षः । व॒र॒.षाः पुच्छं᳚ । श॒रदुत्त॑रः प॒क्षः । हे॒म॒न्तो मद्ध्यं᳚ । पू॒र्व॒प॒क्षाश्चित॑यः । अ॒प॒र॒प॒क्षाः पुरी॑षं \textbf{ 7} \newline
                  \newline
                                \textbf{ TB 3.10.4.2} \newline
                  अ॒हो॒रा॒त्रा-णीष्ट॑काः । ऋ॒ष॒भो॑ऽसि स्व॒र्गो लो॒कः । यस्यां᳚ दि॒शि म॒हीय॑से । ततो॑ नो॒ मह॒ आव॑ह । वा॒यु र्भू॒त्वा सर्वा॒ दिश॒ आवा॑हि । सर्वा॒ दिशोऽनु॒ विवा॑हि । सर्वा॒ दिशोऽनु॒ सम्ॅवा॑हि । चित्त्या॒ चिति॒-मापृ॑ण । अचि॑त्त्या॒ चिति॒-मापृ॑ण । चिद॑सि समु॒द्रयो॑निः \textbf{ 8} \newline
                  \newline
                                \textbf{ TB 3.10.4.3} \newline
                  इन्दु॒ र्दक्षः॑ श्ये॒न ऋ॒तावा᳚ । हिर॑ण्यपक्षः शकु॒नो भु॑र॒ण्युः । म॒हान् थ्स॒धस्थे᳚ ध्रु॒व आनिष॑त्तः । नम॑स्ते अस्तु॒ मा मा॑ हिꣳसीः । एति॒ प्रेति॒ वीति॒ समित्युदिति॑ । दिवं॑ मे यच्छ । अ॒न्तरि॑क्षं मे यच्छ । पृ॒थि॒वीं मे॑ यच्छ । पृ॒थि॒वीं मे॑ यच्छ । अ॒न्तरि॑क्षं मे यच्छ ( ) । दिवं॑ मे यच्छ । अह्ना॒ प्रसा॑रय । रात्र्या॒ सम॑च । रात्र्या॒ प्रसा॑रय । अह्ना॒ सम॑च । कामं॒ प्रसा॑रय । कामꣳ॒॒ सम॑च । \textbf{ 9} \newline
                  \newline
                                    (पुरी॑षꣳ - समु॒द्रयो॑निः - पृथि॒वीं मे॑ यच्छा॒न्तरि॑क्षं मे यच्छ स॒प्त च॑) \textbf{(A4)} \newline \newline
                \textbf{ 3.10.5    अनुवाकं   5 -अनुवाकेषु नवसु लेखासु नाभ्यां चेष्टकोपधानं स्वयमातृण्णोपधानं चोच्यते} \newline
                                \textbf{ TB 3.10.5.1} \newline
                  भूर्भुवः॒ स्वः॑ । ओजो॒ बलं᳚ । ब्रह्म॑ क्ष॒त्रं । यशो॑ म॒हत् । स॒त्यं तपो॒ नाम॑ । रू॒प-म॒मृतं᳚ । चक्षुः॒ श्रोत्रं᳚ । मन॒ आयुः॑ ।विश्वं॒ ॅयशो॑ म॒हः । स॒मं तपो॒ हरो॒ भाः ( ) । जा॒तवे॑दा॒ यदि॑ वा पाव॒कोऽसि॑ । वै॒श्वा॒न॒रो यदि॑ वा वैद्यु॒तोऽसि॑ । शं प्र॒जाभ्यो॒ यज॑मानाय लो॒कं । ऊर्जं॒ पुष्टिं॒ दद॑-द॒भ्याव॑वृथ्स्व । \textbf{ 10} \newline
                  \newline
                                    (भाश्च॒त्वारि॑ च) \textbf{(A5)} \newline \newline
                \textbf{ 3.10.6    अनुवाकं   6 -अनुवाकेषु नवसु लेखासु नाभ्यां चेष्टकोपधानं स्वयमातृण्णोपधानं चोच्यते} \newline
                                \textbf{ TB 3.10.6.1} \newline
                  राज्ञी॑ विराज्ञी᳚ । स॒म्राज्ञी᳚ स्व॒राज्ञी᳚ । अ॒र्चिः शो॒चिः । तपो॒ हरो॒ भाः । अ॒ग्नि-रिन्द्रो॒ बृह॒स्पतिः॑ । विश्वे॑ दे॒वा भुव॑नस्य गो॒पाः । ते मा॒ सर्वे॒ यश॑सा॒ सꣳसृ॑जन्तु । \textbf{ 11} \newline
                  \newline
                                    (राज्ञीन्द्रो॑ मा स॒प्त \textbf{(A6)} \newline \newline
                \textbf{ 3.10.7    अनुवाकं   7 -अनुवाकेषु नवसु लेखासु नाभ्यां चेष्टकोपधानं स्वयमातृण्णोपधानं चोच्यते} \newline
                                \textbf{ TB 3.10.7.1} \newline
                  अस॑वे॒ स्वाहा॒ वस॑वे॒ स्वाहा᳚ । विभु॑वे॒ स्वाहा॒ विव॑स्वते॒ स्वाहा᳚ । अ॒भि॒भुवे॒ स्वाहा-ऽधि॑पतये॒ स्वाहा᳚ । दि॒वांपत॑ये॒ स्वाहा ऽꣳ॑हस्प॒त्याय॒ स्वाहा᳚ । चा॒क्षु॒ष्म॒त्याय॒ स्वाहा᳚ ज्योतिष्म॒त्याय॒ स्वाहा᳚ । राज्ञे॒ स्वाहा॑ वि॒राज्ञे॒ स्वाहा᳚ । स॒म्राज्ञे॒ स्वाहा᳚ स्व॒राज्ञे॒ स्वाहा᳚ । शूषा॑य॒ स्वाहा॒ सूर्या॑य॒ स्वाहा᳚ । च॒न्द्रम॑से॒ स्वाहा॒ ज्योति॑षे॒ स्वाहा᳚ । सꣳ॒॒सर्पा॑य॒ स्वाहा॑ क॒ल्याणा॑य॒ स्वाहा᳚ ( ) । अर्जु॑नाय॒ स्वाहा᳚ । \textbf{ 12} \newline
                  \newline
                                    (क॒ल्याणा॑य॒ स्वाहैकं॑ च) \textbf{(A7)} \newline \newline
                \textbf{ 3.10.8    अनुवाकं   8 -अनुवाकेषु नवसु लेखासु नाभ्यां चेष्टकोपधानं स्वयमातृण्णोपधानं चोच्यते} \newline
                                \textbf{ TB 3.10.8.1} \newline
                  वि॒प॒श्चिते॒ पव॑मानाय गायत । म॒ही न धारा ऽत्यन्धो॑ अर्.षति । अहि॑र्.ह जी॒र्णामति॑सर्पति॒ त्वचं᳚ । अत्यो॒ न क्रीड॑न्न सर॒द्वृषा॒ हरिः॑ । उ॒प॒या॒मगृ॑हीतोऽसि मृ॒त्यवे᳚ त्वा॒ जुष्टं॑ गृह्णामि । ए॒ष ते॒ योनि॑र्मृ॒त्यवे᳚ त्वा ॥ अप॑ मृ॒त्यु-मप॒क्षुधं᳚ । अपे॒तः श॒पथं॑ जहि । अधा॑ नो अग्न॒ आव॑ह । रा॒यस्पोषꣳ॑ सह॒स्रिणं᳚ \textbf{ 13} \newline
                  \newline
                                \textbf{ TB 3.10.8.2} \newline
                  ये ते॑ स॒हस्र॑म॒युतं॒ पाशाः᳚ । मृत्यो॒ मर्त्या॑य॒ हन्त॑वे । तान्. य॒ज्ञ्स्य॑ मा॒यया᳚ । सर्वा॒नव॑ यजामहे ॥ भ॒क्षो᳚ ऽस्यमृत भ॒क्षः । तस्य॑ ते मृ॒त्यु-पी॑तस्या॒-मृत॑वतः । स्व॒गा-कृ॑तस्य॒ मधु॑मतः । उप॑हूत॒स्यो-प॑हूतो भक्षयामि ॥ म॒न्द्रा ऽभिभू॑तिः के॒तु र्य॒ज्ञानां॒ ॅवाक् । असा॒वेहि॑ \textbf{ 14} \newline
                  \newline
                                \textbf{ TB 3.10.8.3} \newline
                  अ॒न्धो जागृ॑विः प्राण । असा॒वेहि॑ । ब॒धि॒र आ᳚क्रन्दयित-रपान । असा॒वेहि॑ । अ॒ह॒स्तोस्त्वा॒ चक्षुः॑ । असा॒वेहि॑ । अ॒पा॒दाशो॒ मनः॑ । असा॒वेहि॑ । कवे॒ विप्र॑चित्ते॒ श्रोत्र॑ । असा॒वेहि॑ \textbf{ 15} \newline
                  \newline
                                \textbf{ TB 3.10.8.4} \newline
                  सु॒ह॒स्तः सु॑वा॒साः । शू॒षो नामा᳚स्य॒-मृतो॒ मर्त्ये॑षु । तन्त्वा॒ऽहं तथा॒ वेद॑ । असा॒वेहि॑ । अ॒ग्निर्मे॑ वा॒चि श्रि॒तः । वाग्घृद॑ये । हृद॑यं॒ मयि॑ । अ॒हम॒मृते᳚ । अ॒मृतं॒ ब्रह्म॑णि । वा॒युर्मे᳚ प्रा॒णे श्रि॒तः \textbf{ 16} \newline
                  \newline
                                \textbf{ TB 3.10.8.5} \newline
                  प्रा॒णो हृद॑ये । हृद॑यं॒ मयि॑ । अ॒हम॒मृते᳚ । अ॒मृतं॒ ब्रह्म॑णि ।सूर्यो॑ मे॒ चक्षु॑षि श्रि॒तः । चक्षु॒र्॒. हृद॑ये । हृद॑यं॒ मयि॑ । अ॒हम॒मृते᳚ । अ॒मृतं॒ ब्रह्म॑णि । च॒न्द्रमा॑ मे॒ मन॑सि श्रि॒तः \textbf{ 17} \newline
                  \newline
                                \textbf{ TB 3.10.8.6} \newline
                  मनो॒ हृद॑ये । हृद॑यं॒ मयि॑ । अ॒हम॒मृते᳚ । अ॒मृतं॒ ब्रह्म॑णि ।दिशो॑ मे॒ श्रोत्रे᳚ श्रि॒ताः । श्रोत्रꣳ॒॒ हृद॑ये । हृद॑यं॒ मयि॑ । अ॒हम॒मृते᳚ । अ॒मृतं॒ ब्रह्म॑णि । आपो॑ मे॒ रेत॑सि श्रि॒ताः \textbf{ 18} \newline
                  \newline
                                \textbf{ TB 3.10.8.7} \newline
                  रेतो॒ हृद॑ये । हृद॑यं॒ मयि॑ । अ॒हम॒मृते᳚ । अ॒मृतं॒ ब्रह्म॑णि ।पृ॒थि॒वी मे॒ शरी॑रे श्रि॒ता । शरी॑रꣳ॒॒ हृद॑ये । हृद॑यं॒ मयि॑ । अ॒हम॒मृते᳚ । अ॒मृतं॒ ब्रह्म॑णि । ओ॒ष॒धि॒व॒न॒स्प॒तयो॑ मे॒ लोम॑सु श्रि॒ताः \textbf{ 19} \newline
                  \newline
                                \textbf{ TB 3.10.8.8} \newline
                  लोमा॑नि॒ हृद॑ये । हृद॑यं॒ मयि॑ । अ॒हम॒मृते᳚ । अ॒मृतं॒ ब्रह्म॑णि । इन्द्रो॑ मे॒ बले᳚ श्रि॒तः । बलꣳ॒॒ हृद॑ये । हृद॑यं॒ मयि॑ । अ॒हम॒मृते᳚ । अ॒मृतं॒ ब्रह्म॑णि । प॒र्जन्यो॑ मे मू॒र्द्ध्नि श्रि॒तः \textbf{ 20} \newline
                  \newline
                                \textbf{ TB 3.10.8.9} \newline
                  मू॒द्र्धा हृद॑ये । हृद॑यं॒ मयि॑ । अ॒हम॒मृते᳚ । अ॒मृतं॒ ब्रह्म॑णि ।ईशा॑नो मे म॒न्यौ श्रि॒तः । म॒न्युर्. हृद॑ये । हृद॑यं॒ मयि॑ । अ॒हम॒मृते᳚ । अ॒मृतं॒ ब्रह्म॑णि । आ॒त्मा म॑ आ॒त्मनि॑ श्रि॒तः( ) \textbf{ 21} \newline
                  \newline
                                \textbf{ TB 3.10.8.10} \newline
                  आ॒त्मा हृद॑ये । हृद॑यं॒ मयि॑ । अ॒हम॒मृते᳚ । अ॒मृतं॒ ब्रह्म॑णि ।पुन॑र्म आ॒त्मा पुन॒रायु॒रागा᳚त् । पुनः॑ प्रा॒णः पुन॒राकू॑त॒-मागा᳚त् ।वै॒श्वा॒न॒रो र॒श्मिभि॑ र्वावृधा॒नः । अ॒न्तस्ति॑ष्ठत्व॒मृत॑स्य गो॒पाः । \textbf{ 22} \newline
                  \newline
                                                        \textbf{special korvai} \newline
              (अ॒ग्निर् वा॒युः सूर्य॑ श्च॒न्द्रमा॒ दिश॒ आपः॑ पृथि॒ व्यो॑षधिवनस्प॒तय॒ इन्द्रः॑ प॒र्जन्य॒ ईशा॑न आ॒त्मा पुन॑र्मे॒ त्रयो॑दश) \newline
                                (स॒ह॒स्रिण॑ - मिहि॒ - श्रोत्रासा॒वेहि॑ - प्रा॒णे श्रि॒तो - मन॑सि श्रि॒तो - रेत॑सि श्रि॒ता - लोम॑सु श्रि॒ता - मू॒र्द्ध्नि श्रि॒त - आ॒त्मनि॑ श्रि॒तो᳚ - +ऽष्टौ च॑) \textbf{(A8)} \newline \newline
                \textbf{ 3.10.9    अनुवाकं   9 -अनुवाकेषु नवसु लेखासु नाभ्यां चेष्टकोपधानं स्वयमातृण्णोपधानं चोच्यते} \newline
                                \textbf{ TB 3.10.9.1} \newline
                  प्र॒जाप॑ति र्दे॒वान॑सृजत । ते पा॒प्मना॒ संदि॑ता अजायन्त । तान् व्य॑द्यत् । यद् व्य॒द्यत् । तस्मा᳚द् वि॒द्युत् । तम॑वृश्चत् । यदवृ॑श्चत् । तस्मा॒द् वृष्टिः॑ । तस्मा॒द् यत्रै॒ते दे॒वते॑ अभि॒प्राप्नु॑तः । विच॑ है॒वास्य॒ तत्र॑ पा॒प्मान॒न्द्यतः॑ \textbf{ 23} \newline
                  \newline
                                \textbf{ TB 3.10.9.2} \newline
                  वृ॒श्चत॑श्च ॥ सैषा मी॑माꣳ॒॒साऽग्नि॑हो॒त्र ए॒व स॑पंन्ना । अथो॑ आहुः । सर्वे॑षु यज्ञ्क्र॒तुष्विति॑ ॥ होष्य॑न्न॒प उप॑स्पृशेत् । विद्यु॑दसि॒ विद्य॑मे पा॒प्मान॒मिति॑ । अथ॑ हु॒त्वोप॑स्पृशेत् । वृष्टि॑रसि॒ वृश्च॑ मे पा॒प्मान॒मिति॑ ।य॒क्ष्यमा॑णो वे॒ष्ट्वा वा᳚ । वि च॑ है॒वास्यै॒ते दे॒वते॑ पा॒प्मानं॒ द्यतः॑ \textbf{ 24} \newline
                  \newline
                                \textbf{ TB 3.10.9.3} \newline
                  वृ॒श्चत॑श्च ॥ अ॒त्यꣳ॒॒हो हारु॑णिः । ब्र॒ह्म॒चा॒रिणे᳚ प्र॒श्नान् प्रोच्य॒ प्रजि॑घाय ।परे॑हि । प्ल॒क्षं दैयां᳚ (दैय्यां᳚) पातिं पृच्छ । वेत्थ॑ सावि॒त्रां(3) न वे॒त्था(3) इति॑ ।तमा॒गत्य॑ पप्रच्छ । आ॒चार्यो॑ मा॒ प्राहै॑षीत् ।वेत्थ॑ सावि॒त्रां(3) न वे॒त्था(3) इति॑ । सहो॑वाच॒ वेदेति॑ । \textbf{ 25} \newline
                  \newline
                                \textbf{ TB 3.10.9.4} \newline
                  स कस्मि॒न् प्रति॑ष्ठित॒ इति॑ । प॒रोर॑ज॒सीति॑ ।कस्तद्यत् प॒रोर॑जा॒ इति॑ । ए॒ष वाव स प॒रोर॑जा॒ इति॑ होवाच । य ए॒ष तप॑ति । ए॒षो᳚ ऽर्वाग्र॑जा॒ इति॑ । स कस्मि॑न् त्वे॒ष इति॑ । स॒त्य इति॑ । किं तथ्स॒त्यमिति॑ । तप॒ इति॑ \textbf{ 26} \newline
                  \newline
                                \textbf{ TB 3.10.9.5} \newline
                  कस्मि॒न्नु तप॒ इति॑ । बल॒ इति॑ । किं तद् ब॒लमिति॑ । प्रा॒ण इति॑ । मास्म॑ प्रा॒णमति॑ पृच्छ॒ इति॑ माऽऽचा॒र्यो᳚ ऽब्रवी॒दिति॑ होवाच ब्रह्मचा॒री ॥ सहो॑वाच प्ल॒क्षो दैयां᳚ (दैय्यां᳚) पातिः । यद्वै ब्र॑ह्मचारिन् प्रा॒णमत्य॑प्रक्ष्यः । मू॒द्र्धा ते॒ व्यप॑तिष्यत् ।अ॒हमु॑त आचा॒र्या-च्छ्रेया᳚न् भविष्यामि । यो मा॑ सावि॒त्रे स॒मवा॑दि॒ष्टेति॑ \textbf{ 27} \newline
                  \newline
                                \textbf{ TB 3.10.9.6} \newline
                  तस्मा᳚थ् सावि॒त्रे न सम्ॅव॑देत ॥ स यो ह॒वै सा॑वि॒त्रं ॅवि॒दुषा॑ सावि॒त्रे स॒म्ॅवद॑ते । सहा᳚स्मि॒-ञ्छ्रियं॑ दधाति ।अनु॑ ह॒वा अ॑स्मा अ॒सौ तप॒ञ्छ्रियं॑ मन्यते ।अन्व॑स्मै॒ श्रीस्तपो॑ मन्यते । अन्व॑स्मै॒ तपो॒ बलं॑ मन्यते । अन्व॑स्मै॒ बलं॑ प्रा॒णं म॑न्यते ॥ स यदाह॑ ।स॒ज्ञांनं॑ ॅवि॒ज्ञानं॒ दर्.शा॑ दृ॒ष्टेति॑ । ए॒ष ए॒व तत् । \textbf{ 28} \newline
                  \newline
                                \textbf{ TB 3.10.9.7} \newline
                  अथ॒ यदाह॑ । प्रस्तु॑तं॒ ॅविष्टु॑तꣳ सु॒ता सु॑न्व॒तीति॑ । ए॒ष ए॒व तत् ॥ ए॒ष ह्ये॑व तान्यहा॑नि । ए॒ष रात्र॑यः ॥ अथ॒ यदाह॑ । चि॒त्रः के॒तु र्दा॒ता प्र॑दा॒ता स॑वि॒ता प्र॑सवि॒ता ऽभि॑शा॒स्ता ऽनु॑म॒न्तेति॑ । ए॒ष ए॒व तत् । ए॒ष ह्ये॑व तेऽह्नो॑ मुहू॒र्ताः । ए॒ष रात्रेः᳚ । \textbf{ 29} \newline
                  \newline
                                \textbf{ TB 3.10.9.8} \newline
                  अथ॒ यदाह॑ । प॒वित्रं॑ पवयि॒ष्यन्-थ्सह॑स्वा॒न्-थ्सही॑यानरु॒णो॑ ऽरु॒णर॑जा॒ इति॑ । ए॒ष ए॒व तत् । ए॒ष ह्ये॑व ते᳚ ऽद्र्धमा॒साः । ए॒ष मासाः᳚ ॥ अथ॒ यदाह॑ । अ॒ग्नि॒ष्टो॒म उ॒क्थ्यो᳚ऽग्निर्. ऋ॒तुः प्र॒जाप॑तिः सम्ॅवथ्स॒र इति॑ । ए॒ष ए॒व तत् । ए॒ष ह्ये॑व ते य॑ज्ञ्क्र॒तवः॑ । ए॒ष ऋ॒तवः॑ \textbf{ 30} \newline
                  \newline
                                \textbf{ TB 3.10.9.9} \newline
                  ए॒ष स॑म्ॅवथ्स॒रः ॥ अथ॒ यदाह॑ । इ॒दानीं᳚ त॒दानी॒मिति॑ । ए॒ष ए॒व तत् । ए॒ष ह्ये॑व ते मु॑हू॒र्तानां᳚ मुहू॒र्ताः ॥ ज॒न॒को ह॒ वै दे॑हः । अ॒हो॒रा॒त्रैः स॒माज॑गाम । तꣳ हो॑चुः । यो वा अ॒स्मान्. वेद॑ । वि॒जह॑त्-पा॒प्मान॑मेति \textbf{ 31} \newline
                  \newline
                                \textbf{ TB 3.10.9.10} \newline
                  सर्व॒-मायु॑रेति । अ॒भि स्व॒र्गं ॅलो॒कं ज॑यति । नास्या॒-मुष्मि॑न् ॅलो॒केऽन्नं॑ क्षीयत॒ इति॑ ॥ वि॒जह॑द्ध॒ वै पा॒प्मान॑मेति । सर्व॒मायु॑रेति । अ॒भि स्व॒र्गं ॅलो॒कं ज॑यति । नास्या॒-मुष्मि॑न् ॅलो॒के ऽन्नं॑ क्षीयते । य ए॒वं ॅवेद॑ ॥ अही॑ना॒ हाश्व॑त्थ्यः । सा॒वि॒त्रं ॅवि॒दां च॑कार \textbf{ 32} \newline
                  \newline
                                \textbf{ TB 3.10.9.11} \newline
                  स ह॑ हꣳ॒॒सो हि॑र॒ण्मयो॑ भू॒त्वा । स्व॒र्गं ॅलो॒कमि॑याय । आ॒दि॒त्यस्य॒ सायु॑ज्यं ॥ हꣳ॒॒सो ह॒वै हि॑र॒ण्मयो॑ भू॒त्वा ।स्व॒र्गं ॅलो॒कमे॑ति । आ॒दि॒त्यस्य॒ सायु॑ज्यं । य ए॒वं ॅवेद॑ ॥ दे॒व॒भा॒गो ह॑ श्रौत॒र्॒.षः । सा॒वि॒त्रं ॅवि॒दां च॑कार । तꣳ ह॒ वागदृ॑श्यमा॒ना ऽभ्यु॑वाच \textbf{ 33} \newline
                  \newline
                                \textbf{ TB 3.10.9.12} \newline
                  सर्व॑म्बत गौत॒मो वे॑द । यः सा॑वि॒त्रं ॅवेदेति॑ । स हो॑वाच । कैषा वाग॒सीति॑ । अ॒यम॒हꣳ सा॑वि॒त्रः । दे॒वाना॑-मुत्त॒मो लो॒कः । गुह्यं॒ महो॒ बिभ्र॒दिति॑ । ए॒ताव॑ति ह गौत॒मः । य॒ज्ञो॒प॒वी॒तं कृ॒त्वा-ऽधो निप॑पात । नमो॒ नम॒ इति॑ \textbf{ 34} \newline
                  \newline
                                \textbf{ TB 3.10.9.13} \newline
                  स हो॑वाच । मा भै॑षी र्गौतम । जि॒तो वै ते॑ लो॒क इति॑ ॥ तस्मा॒द्ये के च॑ सावि॒त्रं ॅवि॒दुः । सर्वे॒ ते जि॒तलो॑काः ॥ स यो ह॒वै सा॑वि॒त्र-स्या॒ष्टाक्ष॑रं प॒दꣳ श्रि॒या ऽभिषि॑क्तं॒ ॅवेद॑ । श्रि॒या है॒वाभिषि॑च्यते । घृणि॒रिति॒ द्वे अ॒क्षरे᳚ । सूर्य॒ इति॒ त्रीणि॑ । आ॒दि॒त्य इति॒ त्रीणि॑ \textbf{ 35} \newline
                  \newline
                                \textbf{ TB 3.10.9.14} \newline
                  ए॒तद्वै सा॑वि॒त्र-स्या॒ष्टाक्ष॑रं प॒दꣳ श्रि॒याभिषि॑क्तं । य ए॒वं ॅवेद॑ ।श्रि॒या है॒वाभिषि॑च्यते ॥ तदे॒-तदृ॒चा-ऽभ्यु॑क्तं । ऋ॒चो अ॒क्षरे॑ पर॒मे व्यो॑मन्न् । यस्मि॑न् दे॒वा अधि॒ विश्वे॑ निषे॒दुः ।यस्तन्न वेद॒ किमृ॒चा क॑रिष्यति । य इत् तद्वि॒दुस्त इ॒मे समा॑सत॒ इति॑ ॥ न ह॒वा ए॒तस्य॒र्चा न यजु॑षा॒ न साम्ना-ऽर्थो᳚ऽस्ति । यः सा॑वि॒त्रं ॅवेद॑ । \textbf{ 36} \newline
                  \newline
                                \textbf{ TB 3.10.9.15} \newline
                  तदे॒तत् प॑रि॒ यद्दे॑वच॒क्रं । आ॒र्द्रं पिन्व॑मानꣳ स्व॒र्गे लो॒क ए॑ति ।वि॒जह॒द् विश्वा॑ भू॒तानि॑ स॒पंश्य॑त् ॥ आ॒र्द्रो ह॒ वै पिन्व॑मानः स्व॒र्गे लो॒क ए॑ति । वि॒जह॒न् विश्वा॑ भू॒तानि॑ स॒पंश्यन्न्॑ । य ए॒वं ॅवेद॑ ॥ शू॒षो ह॒वै वा᳚र्ष्णे॒यः । आ॒दि॒त्येन॑ स॒माज॑गाम । तꣳ हो॑वाच । एहि॑ सावि॒त्रं ॅवि॑द्धि ( ) । अ॒यं ॅवै स्व॒र्ग्यो᳚ऽग्निः पा॑रयि॒ष्णु-र॒मृता॒-थ्संभू॑त॒ इति॑ । ए॒ष वाव स सा॑वि॒त्रः । य ए॒ष तप॑ति । एहि॒ मां ॅवि॑द्धि । इति॑ है॒वैनं॒ तदु॑वाच । \textbf{ 37} \newline
                  \newline
                                                        \textbf{special korvai} \newline
              (प्र॒जाप॑तिर्दे॒वान् स॒ज्ञांनं॒ प्रस्तु॑तं॒ तान्यहा᳚न्ये॒ष रात्र॑यास्चि॒त्र के॒तुस्तेऽह्नो॑ मुहू॒र्तो रात्रेः᳚ प॒वित्रं॒ ते᳚ऽर्ध मा॒सा अ॑ग्निष्टो॒मो य॑ज्ञ्क्र॒तव॑ इ॒दानीं᳚ मुहू॒र्तानां᳚ जन॒कोऽही॑ना देवभा॒गः कैषा वाङ्म शू॒षो ह॒ वै षोड॑श) \newline
                                (द्यतो॒ - द्यतो॒ - वेदेति॒ - तप॒ इति॑ - स॒मवा॑दि॒ष्टेति॒ - तद् - रात्रेर्॑. - ऋ॒तवः॑ - एति - चकारो - वाच॒ - नम॒ इत्या॑ - दि॒त्य इति॒ त्रीणि॑ - सावि॒त्रं ॅवेद॑ - विद्धि॒ पञ्च॑ च) \textbf{(A9)} \newline \newline
                \textbf{ 3.10.10   अनुवाकं   10 -अनुवाकेषु नवसु लेखासु नाभ्यां चेष्टकोपधानं स्वयमातृण्णोपधानं चोच्यते} \newline
                                \textbf{ TB 3.10.10.1} \newline
                  इ॒यं ॅवाव स॒रघा᳚ । तस्या॑ अ॒ग्निरे॒व सा॑र॒घं मधु॑ । या ए॒ताः पू᳚र्वपक्षा-परप॒क्षयो॒ रात्र॑यः । ता म॑धु॒कृतः॑ । यान्यहा॑नि । ते म॑धुवृ॒षाः ।स यो ह॒ वा ए॒ता म॑धु॒कृत॑श्च मधुवृ॒षाꣳश्च॒ वेद॑ । कु॒र्वन्ति॑ हास्यै॒ता अ॒ग्नौ मधु॑ । नास्ये᳚ष्टापू॒र्तं ध॑यन्ति ॥ अथ॒ यो न वेद॑ \textbf{ 38} \newline
                  \newline
                                \textbf{ TB 3.10.10.2} \newline
                  न हा᳚स्यै॒ता अ॒ग्नौ मधु॑ कुर्वन्ति । धय॑न्त्यस्येष्टापू॒र्तं ॥यो ह॒ वा अ॑होरा॒त्राणा᳚-न्नाम॒धेया॑नि॒ वेद॑ । नाहो॑रा॒त्रेष्वा-र्ति॒मार्च्छ॑ति । स॒ज्ञांनं॑ ॅवि॒ज्ञानं॒ दर्.शा॑ दृ॒ष्टेति॑ । ए॒ता-व॑नुवा॒कौ पू᳚र्वप॒क्षस्या॑-होरा॒त्राणां᳚ नाम॒धेया॑नि । प्रस्तु॑तं॒ ॅविष्टु॑तꣳ सु॒ता सु॑न्व॒तीति॑ । ए॒ताव॑नु-वा॒काव॑-परप॒क्षस्या॑-होरा॒त्राणां᳚ नाम॒धेया॑नि । नाहो॑रा॒त्रेष्वा-र्ति॒मार्च्छ॑ति । य ए॒वं ॅवेद॑ । \textbf{ 39} \newline
                  \newline
                                \textbf{ TB 3.10.10.3} \newline
                  यो ह॒ वै मु॑हू॒र्तानां᳚ नाम॒धेया॑नि॒ वेद॑ । न मु॑हू॒र्तेष्वा-र्ति॒मार्च्छ॑ति । चि॒त्रः के॒तु र्दा॒ता प्र॑दा॒ता स॑वि॒ता प्र॑सवि॒ता ऽभि॑शा॒स्ता ऽनु॑म॒न्तेति॑ । ए॒ते॑ऽनुवा॒का मु॑हू॒र्तानां᳚ नाम॒धेया॑नि । न मु॑हू॒र्तेष्वा-र्ति॒मार्च्छ॑ति । य ए॒वं ॅवेद॑ ॥ यो ह॒ वा अ॑द्र्धमा॒सा-ना᳚ञ्च॒ मासा॑नाञ्च नाम॒धेया॑नि॒ वेद॑ ।नाद्र्ध॑मा॒सेषु॒ न मासे॒ष्वा-र्ति॒मार्च्छ॑ति ।प॒वित्रं॑ पवयि॒ष्यन्-थ्सह॑स्वा॒न्-थ्सही॑यानरु॒णो॑ ऽरु॒णर॑जा॒ इति॑ ।ए॒ते॑ऽनुवा॒का अ॑द्र्धमा॒साना᳚ञ्च॒ मासा॑नाञ्च नाम॒धेया॑नि \textbf{ 40} \newline
                  \newline
                                \textbf{ TB 3.10.10.4} \newline
                  नाद्र्ध॑म॒सेषु॒ न मासे॒ष्वा-र्ति॒मार्च्छ॑ति । य ए॒वं ॅवेद॑ ॥यो ह॒ वै य॑ज्ञ्क्रतू॒नां च॑र्तू॒नां च॑ सम्ॅवथ्स॒रस्य॑ च नाम॒धेया॑नि॒ वेद॑ ।न य॑ज्ञ्क्र॒तुषु॒ नर्तुषु॒ न स॑म्ॅवथ्स॒र आर्ति॒मार्च्छ॑ति । अ॒ग्नि॒ष्टो॒म उ॒क्थ्यो᳚ऽग्निर्. ऋ॒तुः प्र॒जाप॑तिः सम्ॅवथ्स॒र इति॑ ।ए॒ते॑ ऽनुवा॒का य॑ज्ञ्क्रतू॒नां च॑र्तू॒नां च॑ सम्ॅवथ्स॒रस्य॑ च नाम॒धेया॑नि ।न य॑ज्ञ्क्र॒तुषु॒ नर्तुषु॒ न स॑म्ॅवथ्स॒र आर्ति॒मार्च्छ॑ति । य ए॒वं ॅवेद॑ ॥ यो ह॒ वै मु॑हू॒र्तानां᳚ मुहू॒र्तान्. वेद॑ ।न मु॑हू॒र्तानां᳚ मुहू॒र्तेष्वार्ति॒ मार्च्छ॑ति ( ) \textbf{ 41} \newline
                  \newline
                                \textbf{ TB 3.10.10.5} \newline
                  इ॒दानीं᳚ त॒दानी॒मिति॑ । ए॒ते वै मु॑हू॒र्तानां᳚ मुहू॒र्ताः । न मु॑हू॒र्तानां᳚ मुहू॒र्तेष्वार्ति॒-मार्च्छ॑ति । य ए॒वं ॅवेद॑ ॥अथो॒ यथा᳚ क्षेत्र॒ज्ञो भू॒त्वा ऽनु॑ प्र॒विश्यान्न॒-मत्ति॑ । ए॒वमे॒वैतान् क्षे᳚त्र॒ज्ञो भू॒त्वा ऽनु॑ प्र॒विश्या-न्न॑मत्ति । स ए॒तेषा॑मे॒व स॑लो॒कताꣳ॒॒ सायु॑ज्य-मश्नुते । अप॑ पुनर्मृ॒त्युं ज॑यति । य ए॒वं ॅवेद॑ । \textbf{ 42} \newline
                  \newline
                                                        \textbf{special korvai} \newline
              (इ॒यम॑होरा॒त्राणाꣳ॑ स॒ज्ञांनं॑ पूर्वप॒क्षस्य॒ प्रस्तु॑तमपरप॒क्षस्य॑ मुहू॒र्तानां᳚ चि॒त्रः के॒तुर॑द्र्धमा॒सानां᳚ प॒वित्रं॑ ॅयज्ञ्क्रतू॒नाम॑ग्निष्टो॒मो य॑ज्ञ्क्रतू॒नामि॒दानीं᳚ मुहू॒र्तानां᳚ मुहू॒र्तान्. वे॒देदानी॒मथो॒ द्वाद॑श) \newline
                                (न वेदै॒ - वं वेदा॑ - नुवा॒का अ॑द्र्धमा॒सानां᳚ च॒ मसा॑नां च नाम॒धेया॑नि - मुहू॒र्तेष्वार्ति॒मार्च्छ॑ति॒ - +नव॑ च) \textbf{(A10)} \newline \newline
                \textbf{ 3.10.11   अनुवाकं   11 -अनुवाकेषु नवसु लेखासु नाभ्यां चेष्टकोपधानं स्वयमातृण्णोपधानं चोच्यते} \newline
                                \textbf{ TB 3.10.11.1} \newline
                  कश्चि॑द्ध॒वा अ॒स्मा-ल्लो॒कात् प्रेत्य॑ । आ॒त्मानं॑ ॅवेद । अ॒य-म॒ह-म॒स्मीति॑ । कश्चि॒थ्स्वं ॅलो॒कं न प्रति॒ प्रजा॑नाति ॥ अ॒ग्निमु॑ग्धो है॒व धु॒मता᳚न्तः । स्वं ॅलो॒कं न प्रति॒ प्रजा॑नाति ।अथ॒ यो है॒वैत-म॒ग्निꣳ सा॑वि॒त्रं ॅवेद॑ । स ए॒वास्मा ल्लो॒कात् प्रेत्य॑ । आ॒त्मानं॑ ॅवेद । अ॒य-म॒ह-म॒स्मीति॑ \textbf{ 43} \newline
                  \newline
                                \textbf{ TB 3.10.11.2} \newline
                  स स्वं ॅलो॒कं प्रति॒ प्रजा॑नाति ॥ ए॒ष उ॑ वे॒वैनं॒ तथ्सा॑वि॒त्रः । स्व॒र्गं ॅलो॒क-म॒भिव॑हति ॥ अ॒हो॒रा॒त्रै र्वा इदꣳ स॒युग्भिः॑ क्रियते । इ॒ति॒ रा॒त्राया॑ दीक्षिषत । इ॒ति॒ रा॒त्राय॑ व्र॒त-मुपा॑गु॒रिति॑ । तानि॒ हाने॑वं ॅवि॒दुषः॑ । अ॒मुष्मि॑न् ॅलो॒के शे॑व॒धिं ध॑यन्ति ।धी॒तꣳ है॒व स शे॑व॒धि-मनु॒परै॑ति ॥अथ॒ यो है॒वैतम॒ग्निꣳ सा॑वि॒त्रं ॅवेद॑ \textbf{ 44} \newline
                  \newline
                                \textbf{ TB 3.10.11.3} \newline
                  तस्य॑ है॒वा-हो॑रा॒त्राणि॑ । अ॒मुष्मि॑न् ॅलो॒के शे॑व॒धिं न ध॑यन्ति ।अधी॑तꣳ है॒व स शे॑व॒धि-मनु॒परै॑ति ॥ भ॒रद्वा॑जो ह त्रि॒भिरायु॑र्भि र्ब्रह्म॒चर्य॑मुवास । तꣳ ह॒ जीर्णिꣳ॒॒ स्थवि॑रꣳ॒॒ शया॑नं । इन्द्र॑ उप॒व्रज्यो॑ वाच । भर॑द्वाज । यत्ते॑ चतु॒र्थमायु॑र्द॒द्यां । किमे॑नेन कुर्या॒ इति॑ । ब्र॒ह्म॒चर्य॑मे॒वैने॑न चरेय॒मिति॑ होवाच \textbf{ 45} \newline
                  \newline
                                \textbf{ TB 3.10.11.4} \newline
                  तꣳ ह॒ त्रीन् गि॒रि-रू॑पा॒न वि॑ज्ञातानिव दर्.श॒यां च॑कार । तेषाꣳ॒॒ हैकै॑कस्मान् मु॒ष्टिनाऽऽद॑दे । स हो॑वाच ।भर॑द्वा॒जेत्या॒मन्त्र्य॑ । वेदा॒ वा ए॒ते । अ॒न॒न्ता वै वेदाः᳚ । ए॒तद्वा ए॒तै स्त्रि॒भि-रायु॑र्भि॒-रन्व॑वोचथाः । अथ॑ त॒ इत॑र॒-दन॑नूक्तमे॒व । एही॒मं ॅवि॑द्धि । अ॒यं ॅवै स॑र्व वि॒द्येति॑ \textbf{ 46} \newline
                  \newline
                                \textbf{ TB 3.10.11.5} \newline
                  तस्मै॑ है॒तम॒ग्निꣳ सा॑वि॒त्रमु॑वाच । तꣳ स वि॑दि॒त्वा । अ॒मृतो॑ भू॒त्वा । स्व॒र्गं ॅलो॒कमि॑याय । आ॒दि॒त्यस्य॒ सायु॑ज्यं ॥ अ॒मृतो॑ है॒व भू॒त्वा । स्व॒र्गं ॅलो॒कमे॑ति । आ॒दि॒त्यस्य॒ सायु॑ज्यं । य ए॒वं ॅवेद॑ ॥ ए॒षो ए॒व त्रयी॑ वि॒द्या \textbf{ 47} \newline
                  \newline
                                \textbf{ TB 3.10.11.6} \newline
                  याव॑न्तꣳ ह॒ वै त्र॒य्या वि॒द्यया॑ लो॒कं ज॑यति । ताव॑न्तं ॅलो॒कं ज॑यति । य ए॒वं ॅवेद॑ ॥ अ॒ग्ने र्वा ए॒तानि॑ नाम॒धेया॑नि ।अ॒ग्नेरे॒व सायु॑ज्यꣳ सलो॒कता॑-माप्नोति ।य ए॒वं ॅवेद॑ । वा॒यो र्वा ए॒तानि॑ नाम॒धेया॑नि । वा॒योरे॒व सायु॑ज्यꣳ सलो॒कता॑-माप्नोति । य ए॒वं ॅवेद॑ । इन्द्र॑स्य॒ वा ए॒तानि॑ नाम॒धेया॑नि \textbf{ 48} \newline
                  \newline
                                \textbf{ TB 3.10.11.7} \newline
                  इन्द्र॑स्यै॒व सायु॑ज्यꣳ सलो॒कता॑-माप्नोति । य ए॒वं ॅवेद॑ ।बृह॒स्पते॒ र्वा ए॒तानि॑ नाम॒धेया॑नि । बृह॒स्पते॑रे॒व सायु॑ज्यꣳ सलो॒कता॑-माप्नोति ।य ए॒वं ॅवेद॑ । प्र॒जाप॑ते॒ र्वा ए॒तानि॑ नाम॒धेया॑नि । प्र॒जाप॑तेरे॒व सायु॑ज्यꣳ सलो॒कता॑-माप्नोति । य ए॒वं ॅवेद॑ ।ब्रह्म॑णो॒ वा ए॒तानि॑ नाम॒धेया॑नि । ब्रह्म॑ण ए॒व सायु॑ज्यꣳ सलो॒कता॑-माप्नोति ( ) । य ए॒वं ॅवेद॑ ॥ स वा ए॒षो᳚ऽग्नि-र॑पक्षपु॒च्छो वा॒युरे॒व । तस्या॒ग्नि र्मुखं᳚ । अ॒सावा॑दि॒त्यः शिरः॑ । स यदे॒ते दे॒वते॒ अन्त॑रेण । तथ्सर्वꣳ॑ सीव्यति । तस्मा᳚थ् सावि॒त्रः । \textbf{ 49} \newline
                  \newline
                                                        \textbf{special korvai} \newline
              (अ॒ग्नेर्वा॒योरिन्द्र॑स्य॒ बृह॒स्पतेः᳚ प्र॒जाप॑ते॒र् ब्रह्म॑णः॒ स वै स॒प्त) \newline
                                (अ॒यम॒हम॒स्मीति॒ - वेद॑ - होवाच - सर्ववि॒द्येति॑ - वि॒द्ये - न्द्र॑स्य॒ वा ए॒तानि॑ नाम॒धेया॑नि॒ - ब्रह्म॑ण ए॒व सायु॑ज्यꣳ सलो॒कता॑माप्नोति स॒प्त च॑) । \textbf{(A11)} \newline \newline
                \textbf{PrapAtaka Korvai with starting  words of 1 to 10 anuvAkams :-} \newline
        (सं॒ज्ञानं॒ - भू - स्त्वमे॒व - स॑म्ॅवथ्स॒रो॑ऽसि॒ - भू - राज्ञ्य - स॑व-विप॒श्चिते᳚ - प्र॒जाप॑तिर् दे॒वा - नि॒यं ॅवाव स॒रघा॒ - कश्चि॒द्धैका॑दश) \newline

        \textbf{korvai with starting words of 1, 11, 21 series of daSinis :-} \newline
        (सं॒ज्ञानꣳ॒॒ - राज्ञी॑ - मू॒द्र्धा हृद॑य - ए॒ष स॑म्ॅवथ्स॒रो - नाद्र्ध॑मा॒सेषु॒ नव॑चत्वारिꣳशत्) \newline

        \textbf{first and last  word 1st prapATakam of Kaatakam:-} \newline
        (सं॒ज्ञानꣳ॑ - सावि॒त्रः) \newline 

       

        ॥ हरिः॑ ॐ ॥
॥ इति तैत्तरीय यजुब्राह्मणे काठके प्रथमः प्रश्नः समाप्तः ॥
++++++++++++++++++++++++++++++++++++++++++ \newline
        \pagebreak
        
        
        
     \addcontentsline{toc}{section}{ 3.11     तैत्तरीय यजुर्ब्राह्मणे काठके द्वितीयः प्रश्नः नाचिकेतचयनम्}
     \markright{ 3.11     तैत्तरीय यजुर्ब्राह्मणे काठके द्वितीयः प्रश्नः नाचिकेतचयनम् \hfill https://www.vedavms.in \hfill}
     \section*{ 3.11     तैत्तरीय यजुर्ब्राह्मणे काठके द्वितीयः प्रश्नः नाचिकेतचयनम् }
                \textbf{ 3.11.1    अनुवाकं   1 -इष्टकोपधानमन्त्राः} \newline
                                \textbf{ TB 3.11.1.1} \newline
                  लो॒को॑ऽसि स्व॒र्गो॑ऽसि । अ॒न॒न्तो᳚ऽस्य पा॒रो॑ऽसि । अक्षि॑तोऽस्य क्ष॒य्यो॑ऽसि । तप॑सः प्रति॒ष्ठा । त्वयी॒दम॒न्तः । विश्वं॑ ॅय॒क्षं ॅविश्वं॑ भू॒तं ॅविश्वꣳ॑ सुभू॒तं । विश्व॑स्य भ॒र्ता विश्व॑स्य जनयि॒ता । तन्त्वोप॑दधे काम॒दुघ॒-मक्षि॑तं । प्र॒जाप॑तिस्त्वा सादयतु । तया॑ दे॒वत॑या-ऽङ्गिर॒स्वद्-ध्रु॒वा सी॑द । \textbf{ 1} \newline
                  \newline
                                \textbf{ TB 3.11.1.2} \newline
                  तपो॑ऽसि लो॒के श्रि॒तं । तेज॑सः प्रति॒ष्ठा । त्वयी॒दम॒न्तः । विश्वं॑ ॅय॒क्षं ॅविश्वं॑ भू॒तं ॅविश्वꣳ॑ सुभू॒तं । विश्व॑स्य भ॒र्तृ विश्व॑स्य जनयि॒तृ । तत् त्वोप॑दधे काम॒दुघ॒-मक्षि॑तं । प्र॒जाप॑तिस्त्वा सादयतु ।तया॑ दे॒वत॑या-ऽङ्गिर॒स्वद्-ध्रु॒वा सी॑द । \textbf{ 2} \newline
                  \newline
                                \textbf{ TB 3.11.1.3} \newline
                  तेजो॑ऽसि॒ तप॑सि श्रि॒तं । स॒मु॒द्रस्य॑ प्रति॒ष्ठा ।त्वयी॒दम॒न्तः । विश्वं॑ ॅय॒क्षं ॅविश्वं॑ भू॒तं ॅविश्वꣳ॑ सुभू॒तं ।विश्व॑स्य भ॒र्तृविश्व॑स्य जनयि॒तृ । तत् त्वोप॑दधे काम॒दुघ॒-मक्षि॑तं ।प्र॒जाप॑तिस्त्वा सादयतु । तया॑ दे॒वत॑या-ऽङ्गिर॒स्वद्-ध्रु॒वा सी॑द । \textbf{ 3} \newline
                  \newline
                                \textbf{ TB 3.11.1.4} \newline
                  स॒मु॒द्रो॑ऽसि॒ तेज॑सि श्रि॒तः । अ॒पां प्र॑ति॒ष्ठा । त्वयी॒दम॒न्तः । विश्वं॑ ॅय॒क्षं ॅविश्वं॑ भू॒तं ॅविश्वꣳ॑ सुभू॒तं ।विश्व॑स्य भ॒र्ता विश्व॑स्य जनयि॒ता ।तं त्वोप॑दधे काम॒दुघ॒-मक्षि॑तं । प्र॒जाप॑तिस्त्वा सादयतु । तया॑ दे॒वत॑या-ऽङ्गिर॒स्वद्-ध्रु॒वा सी॑द । \textbf{ 4} \newline
                  \newline
                                \textbf{ TB 3.11.1.5} \newline
                  आपः॑ स्थ समु॒द्रे श्रि॒ताः । पृ॒थि॒व्याः प्र॑ति॒ष्ठा यु॒ष्मासु॑ । इ॒दम॒न्तः । विश्वं॑ ॅय॒क्षं ॅविश्वं॑ भू॒तं ॅविश्वꣳ॑ सुभू॒तं । विश्व॑स्य भ॒र्त्र्यो॑ विश्व॑स्य जनयि॒त्र्यः॑ । ता व॒ उप॑दधे काम॒दुघा॒ अक्षि॑ताः । प्र॒जाप॑तिस्त्वा सादयतु ।तया॑ दे॒वत॑या-ऽङ्गिर॒स्वद्-ध्रु॒वा सी॑द । \textbf{ 5} \newline
                  \newline
                                \textbf{ TB 3.11.1.6} \newline
                  पृ॒थि॒व्य॑-स्य॒फ्सु श्रि॒ता । अ॒ग्नेः प्र॑ति॒ष्ठा । त्वयी॒दम॒न्तः । विश्वं॑ ॅय॒क्षं ॅविश्वं॑ भू॒तं ॅविश्वꣳ॑ सुभू॒तं । विश्व॑स्य भ॒र्त्री विश्व॑स्य जनयि॒त्री । तां त्वोप॑दधे काम॒दुघा॒-मक्षि॑तां । प्र॒जाप॑तिस्त्वा सादयतु । तया॑ दे॒वत॑या-ऽङ्गिर॒स्वद्-ध्रु॒वा सी॑द । \textbf{ 6} \newline
                  \newline
                                \textbf{ TB 3.11.1.7} \newline
                  अ॒ग्निर॑सि पृथि॒व्याꣳ श्रि॒तः । अ॒न्तरि॑क्षस्य प्रति॒ष्ठा । त्वयी॒दम॒न्तः । विश्वं॑ ॅय॒क्षं ॅविश्वं॑ भू॒तं ॅविश्वꣳ॑ सुभू॒तं ।विश्व॑स्य भ॒र्ता विश्व॑स्य जनयि॒ता । तं त्वोप॑दधे काम॒दुघ॒-मक्षि॑तं । प्र॒जाप॑तिस्त्वा सादयतु । तया॑ दे॒वत॑या-ऽङ्गिर॒स्वद्-ध्रु॒वा सी॑द । \textbf{ 7} \newline
                  \newline
                                \textbf{ TB 3.11.1.8} \newline
                  अ॒न्तरि॑क्ष-मस्य॒ग्नौ श्रि॒तं । वा॒योः प्र॑ति॒ष्ठा । त्वयी॒दम॒न्तः ।विश्वं॑ ॅय॒क्षं ॅविश्वं॑ भू॒तं ॅविश्वꣳ॑ सुभू॒तं । विश्व॑स्य भ॒र्तृ विश्व॑स्य जनयि॒तृ ।तत् त्वोप॑दधे काम॒दुघ॒-मक्षि॑तं । प्र॒जाप॑तिस्त्वा सादयतु ।तया॑ दे॒वत॑या-ऽङ्गिर॒स्वद्-ध्रु॒वा सी॑द । \textbf{ 8} \newline
                  \newline
                                \textbf{ TB 3.11.1.9} \newline
                  वा॒युर॑स्य॒न्तरि॑क्षे श्रि॒तः । दि॒वः प्र॑ति॒ष्ठा । त्वयी॒दम॒न्तः । विश्वं॑ ॅय॒क्षं ॅविश्वं॑ भू॒तं ॅविश्वꣳ॑ सुभू॒तं । विश्व॑स्य भ॒र्ता विश्व॑स्य जनयि॒ता ।तं त्वोप॑दधे काम॒दुघ॒-मक्षि॑तं । प्र॒जाप॑तिस्त्वा सादयतु ।तया॑ दे॒वत॑या-ऽङ्गिर॒स्वद्-ध्रु॒वा सी॑द । \textbf{ 9} \newline
                  \newline
                                \textbf{ TB 3.11.1.10} \newline
                  द्यौर॑सि वा॒यौ श्रि॒ता । आ॒दि॒त्यस्य॑ प्रति॒ष्ठा । त्वयी॒दम॒न्तः । विश्वं॑ ॅय॒क्षं ॅविश्वं॑ भू॒तं ॅविश्वꣳ॑ सुभू॒तं । विश्व॑स्य भ॒र्त्री विश्व॑स्य जनयि॒त्री ।तां त्वोप॑दधे काम॒दुघा॒-मक्षि॑तां । प्र॒जाप॑तिस्त्वा सादयतु ।तया॑ दे॒वत॑या-ऽङ्गिर॒स्वद्-ध्रु॒वा सी॑द । \textbf{ 10} \newline
                  \newline
                                \textbf{ TB 3.11.1.11} \newline
                  आ॒दि॒त्यो॑ऽसि दि॒वि श्रि॒तः । च॒न्द्रम॑सः प्रति॒ष्ठा । त्वयी॒दम॒न्तः । विश्वं॑ ॅय॒क्षं ॅविश्वं॑ भू॒तं ॅविश्वꣳ॑ सुभू॒तं । विश्व॑स्य भ॒र्ता विश्व॑स्य जनयि॒ता ।तं त्वोप॑दधे काम॒दुघ॒-मक्षि॑तं । प्र॒जाप॑तिस्त्वा सादयतु । तया॑ दे॒वत॑या-ऽङ्गिर॒स्वद्-ध्रु॒वा सी॑द । \textbf{ 11} \newline
                  \newline
                                \textbf{ TB 3.11.1.12} \newline
                  च॒न्द्रमा॑ अस्यादि॒त्ये श्रि॒तः । नक्ष॑त्राणां प्रति॒ष्ठा । त्वयी॒दम॒न्तः । विश्वं॑ ॅय॒क्षं ॅविश्वं॑ भू॒तं ॅविश्वꣳ॑ सुभू॒तं । विश्व॑स्य भ॒र्ता विश्व॑स्य जनयि॒ता । तं त्वोप॑दधे काम॒दुघ॒-मक्षि॑तं । प्र॒जाप॑तिस्त्वा सादयतु । तया॑ दे॒वत॑या-ऽङ्गिर॒स्वद्-ध्रु॒वा सी॑द । \textbf{ 12} \newline
                  \newline
                                \textbf{ TB 3.11.1.13} \newline
                  नक्ष॑त्राणि स्थ च॒न्द्रम॑सि श्रि॒तानि॑ । स॒म्ॅव॒थ्स॒रस्य॑ प्रति॒ष्ठा यु॒ष्मासु॑ । इ॒दम॒न्तः । विश्वं॑ ॅय॒क्षं ॅविश्वं॑ भू॒तं ॅविश्वꣳ॑ सुभू॒तं ।विश्व॑स्य भ॒र्तॄणि॒ विश्व॑स्य जनयि॒तॄणि॑ ।तानि॑ व॒ उप॑दधे काम॒दुघा॒-न्यक्षि॑तानि । प्र॒जाप॑तिस्त्वा सादयतु ।तया॑ दे॒वत॑या-ऽङ्गिर॒स्वद्-ध्रु॒वा सी॑द । \textbf{ 13} \newline
                  \newline
                                \textbf{ TB 3.11.1.14} \newline
                  स॒म्ॅव॒थ्स॒रो॑ऽसि॒ नक्ष॑त्रेषु श्रि॒तः । ऋ॒तू॒नां प्र॑ति॒ष्ठा । त्वयी॒दम॒न्तः ।विश्वं॑ ॅय॒क्षं ॅविश्वं॑ भू॒तं ॅविश्वꣳ॑ सुभू॒तं । विश्व॑स्य भ॒र्ता विश्व॑स्य जनयि॒ता । तं त्वोप॑दधे काम॒दुघ॒-मक्षि॑तं । प्र॒जाप॑तिस्त्वा सादयतु ।तया॑ दे॒वत॑या-ऽङ्गिर॒स्वद्-ध्रु॒वा सी॑द । \textbf{ 14} \newline
                  \newline
                                \textbf{ TB 3.11.1.15} \newline
                  ऋ॒तवः॑ स्थ सम्ॅवथ्स॒रे श्रि॒ताः । मासा॑नां प्रति॒ष्ठा यु॒ष्मासु॑ । इ॒दम॒न्तः । विश्वं॑ ॅय॒क्षं ॅविश्वं॑ भू॒तं ॅविश्वꣳ॑ सुभू॒तं ।विश्व॑स्य भ॒र्तारो॒ विश्व॑स्य जनयि॒तारः॑ ।तान्. व॒ उप॑दधे काम॒दुघा॒-नक्षि॑तान् । प्र॒जाप॑तिस्त्वा सादयतु ।तया॑ दे॒वत॑या-ऽङ्गिर॒स्वद्-ध्रु॒वा सी॑द । \textbf{ 15} \newline
                  \newline
                                \textbf{ TB 3.11.1.16} \newline
                  मासाः᳚ स्थ॒र्तुषु॑ श्रि॒ताः । अ॒द्र्ध॒मा॒सानां᳚ प्रति॒ष्ठा यु॒ष्मासु॑ । इ॒दम॒न्तः । विश्वं॑ ॅय॒क्षं ॅविश्वं॑ भू॒तं ॅविश्वꣳ॑ सुभू॒तं । विश्व॑स्य भ॒र्तारो॒ विश्व॑स्य जनयि॒तारः॑ । तान्. व॒ उप॑दधे काम॒दुघा॒-नक्षि॑तान् । प्र॒जाप॑तिस्त्वा सादयतु ।तया॑ दे॒वत॑या-ऽङ्गिर॒स्वद्-ध्रु॒वा सी॑द । \textbf{ 16} \newline
                  \newline
                                \textbf{ TB 3.11.1.17} \newline
                  अ॒द्र्ध॒मा॒साः स्थ॑ मा॒सु श्रि॒ताः । अ॒हो॒रा॒त्रयोः᳚ प्रति॒ष्ठा यु॒ष्मासु॑ । इ॒दम॒न्तः । विश्वं॑ ॅय॒क्षं ॅविश्वं॑ भू॒तं ॅविश्वꣳ॑ सुभू॒तं ।विश्व॑स्य भ॒र्तारो॒ विश्व॑स्य जनयि॒तारः॑ ।तान्. व॒ उप॑दधे काम॒दुघा॒-नक्षि॑तान् । प्र॒जाप॑तिस्त्वा सादयतु ।तया॑ दे॒वत॑या-ऽङ्गिर॒स्वद्-ध्रु॒वा सी॑द । \textbf{ 17} \newline
                  \newline
                                \textbf{ TB 3.11.1.18} \newline
                  अ॒हो॒रा॒त्रे स्थो᳚ऽद्र्धमा॒सेषु॑ श्रि॒ते । भू॒तस्य॑ प्रति॒ष्ठे भव्य॑स्य प्रति॒ष्ठे । यु॒वयो॑-रि॒दम॒न्तः । विश्वं॑ ॅय॒क्षं ॅविश्वं॑ भू॒तं ॅविश्वꣳ॑ सुभू॒तं ।विश्व॑स्य भ॒र्त्र्यौ॑ विश्व॑स्य जनयि॒त्र्यौ᳚ । ते वा॒मुप॑दधे काम॒दुघे॒ अक्षि॑ते । प्र॒जाप॑तिस्त्वा सादयतु । तया॑ दे॒वत॑या-ऽङ्गिर॒स्वद्-ध्रु॒वा सी॑द । \textbf{ 18} \newline
                  \newline
                                \textbf{ TB 3.11.1.19} \newline
                  पौ॒र्ण॒मा॒स्यष्ट॑का ऽमावा॒स्या᳚ । अ॒न्ना॒दाः स्था᳚न्न॒दुघो॑ यु॒ष्मासु॑ । इ॒दम॒न्तः । विश्वं॑ ॅय॒क्षं ॅविश्वं॑ भू॒तं ॅविश्वꣳ॑ सुभू॒तं ।विश्व॑स्य भ॒र्त्र्यो॑ विश्व॑स्य जनयि॒त्र्यः॑ ।ता व॒ उप॑दधे काम॒दुघा॒-अक्षि॑ताः । प्र॒जाप॑तिस्त्वा सादयतु ।तया॑ दे॒वत॑या-ऽङ्गिर॒स्वद्-ध्रु॒वा सी॑द । \textbf{ 19} \newline
                  \newline
                                \textbf{ TB 3.11.1.20} \newline
                  राड॑सि बृह॒ती श्रीर॒सीन्द्र॑पत्नी॒ धर्म॑पत्नी । विश्वं॑ भू॒तमनु॒ प्रभू॑ता ।त्वयी॒दम॒न्तः । विश्वं॑ ॅय॒क्षं ॅविश्वं॑ भू॒तं ॅविश्वꣳ॑ सुभू॒तं ।विश्व॑स्य भ॒र्त्री विश्व॑स्य जनयि॒त्री । तां त्वोप॑दधे काम॒दुघा॒-मक्षि॑तां । प्र॒जाप॑तिस्त्वा सादयतु । तया॑ दे॒वत॑या-ऽङ्गिर॒स्वद्-ध्रु॒वा सी॑द । \textbf{ 20} \newline
                  \newline
                                \textbf{ TB 3.11.1.21} \newline
                  ओजो॑ऽसि॒ सहो॑ऽसि । बल॑मसि॒ भ्राजो॑ऽसि ।दे॒वानां॒ धामा॒मृतं᳚ । अम॑र्त्य-स्तपो॒जाः । त्वयी॒दम॒न्तः । विश्वं॑ ॅय॒क्षं ॅविश्वं॑ भू॒तं ॅविश्वꣳ॑ सुभू॒तं । विश्व॑स्य भ॒र्ता विश्व॑स्य जनयि॒ता । तं त्वोप॑दधे काम॒दुघ॒-मक्षि॑तं । प्र॒जाप॑तिस्त्वा सादयतु । तया॑ दे॒वत॑या-ऽङ्गिर॒स्वद्-ध्रु॒वा सी॑द ( ) । \textbf{ 21} \newline
                  \newline
                                                        \textbf{special korvai} \newline
              लो॒को॑ऽसि भ॒र्ता तम् । तप॒स्तेजो॑ऽसि भ॒र्तृ तत् । स॒मु॒द्रो॑ऽसि भ॒र्ता तम् । आपः॑ स्थ भ॒र्त्र्य॑स्ता वः॑ । पृ॒थि॒वी भ॒र्त्री ताम् । अ॒ग्निर॑सि भ॒र्ता तम् । अ॒न्तरि॑क्षं भ॒र्तृ तत् । वा॒युर॑सि भ॒र्ता तम् । द्यौर॑सि भ॒र्त्री ताम् । आ॒दि॒त्यश्च॒न्द्रमा॑ भ॒र्ता तम् । नक्ष॑त्राणि स्थ भ॒र्तॄणि॒ तानि॑ वः । स॒म्ॅव॒थ्स॒रो॑ऽसि भ॒र्ता तम् । ऋ॒तवो॒ मासा॑ अद्र्धमा॒सा भ॒र्तार॒स्तान्. वः॑ । अ॒हो॒रा॒त्रे भ॒र्त्र्यौ॑ ते वा᳚म् । पौ॒र्ण॒मा॒सी भ॒र्त्र्य॑स्ता वः॑ । राड॑सि भ॒र्त्री ताम् । ओजो॑ऽसि भ॒र्ता तमेक॑विꣳशतिः । \newline
                                लो॒क - स्तप॒ - स्तेजः॑ - समु॒द्र - आपः॑ - पृथि॒व्य॑ - ग्नि - र॒न्तरि॑क्षम् - वा॒युर् - द्यौ - रा॑दि॒त्य - श्च॒न्द्रमा॒ - नक्ष॑त्राणि - सम्ॅवथ्स॒र - ऋ॒तवो॒ - मासा॑ - अद्र्धमा॒सा - अ॑होरा॒त्रे - पौ᳚र्णमा॒सी - राड॒ - स्योजो॒ऽस्येक॑विꣳशतिः \textbf{(A1)} \newline \newline
                \textbf{ 3.11.2    अनुवाकं   2 -चतुर्भिरनुवाकैः चतस्र आहुतीर्जुहोति} \newline
                                \textbf{ TB 3.11.2.1} \newline
                  त्वम॑ग्ने रु॒द्रो असु॑रो म॒हो दि॒वः । त्वꣳ शर्द्धो॒ मारु॑तं पृ॒क्ष ई॑शिषे । त्वं ॅवातै॑ररु॒णैर्या॑सि शंग॒यः ।त्वं पू॒षा वि॑ध॒तः पा॑सि॒ नु त्मना᳚ । देवा॑ दे॒वेषु॑ श्रयद्ध्वं । प्रथ॑मा द्वि॒तीये॑षु श्रयद्ध्वं । द्विती॑या-स्तृ॒तीये॑षु श्रयद्ध्वं । तृती॑या-श्चतु॒र्थेषु॑ श्रयद्ध्वं । च॒तु॒र्थाः प॑ञ्च॒मेषु॑ श्रयद्ध्वं । प॒ञ्च॒माः ष॒ष्ठेषु॑ श्रयद्ध्वं \textbf{ 22} \newline
                  \newline
                                \textbf{ TB 3.11.2.2} \newline
                  ष॒ष्ठाः स॑प्त॒मेषु॑ श्रयद्ध्वं । स॒प्त॒मा अ॑ष्ट॒मेषु॑ श्रयद्ध्वं । अ॒ष्ट॒मा न॑व॒मेषु॑ श्रयद्ध्वं । न॒व॒मा द॑श॒मेषु॑ श्रयद्ध्वं । द॒श॒मा ए॑काद॒शेषु॑ श्रयद्ध्वं । ए॒क॒द॒शा द्वा॑द॒शेषु॑ श्रयद्ध्वं । द्वा॒द॒शा-स्त्र॑योद॒शेषु॑ श्रयद्ध्वं । त्र॒यो॒द॒शा-श्च॑तुर्द॒शेषु॑ श्रयद्ध्वं । च॒तु॒र्द॒शाः प॑ञ्चद॒शेषु॑ श्रयद्ध्वं । प॒ञ्च॒द॒शाः षो॑ड॒शेषु॑ श्रयद्ध्वं \textbf{ 23} \newline
                  \newline
                                \textbf{ TB 3.11.2.3} \newline
                  षो॒ड॒शाः स॑प्तद॒शेषु॑ श्रयद्ध्वं । स॒प्त॒द॒शा अ॑ष्टाद॒शेषु॑ श्रयद्ध्वं । अ॒ष्टा॒द॒शा ए॑कान्नविꣳ॒॒शेषु॑ श्रयद्ध्वं । ए॒का॒न्न॒विꣳ॒॒शा विꣳ॒॒शेषु॑ श्रयद्ध्वं । विꣳ॒॒शा ए॑कविꣳ॒॒शेषु॑ श्रयद्ध्वं । ए॒क॒विꣳ॒॒शा द्वा॑विꣳ॒॒शेषु॑ श्रयद्ध्वं । द्वा॒विꣳ॒॒शा स्त्र॑योविꣳ॒॒शेषु॑ श्रयद्ध्वं । त्र॒यो॒विꣳ॒॒शा श्च॑तुर्विꣳ॒॒शेषु॑ श्रयद्ध्वं । च॒तु॒र्विꣳ॒॒शाः प॑ञ्चविꣳ॒॒शेषु॑ श्रयद्ध्वं । प॒ञ्च॒विꣳ॒॒शाः ष॑ड्विꣳ॒॒शेषु॑ श्रयद्ध्वं \textbf{ 24} \newline
                  \newline
                                \textbf{ TB 3.11.2.4} \newline
                  ष॒ड्विꣳ॒॒शाः स॑प्तविꣳ॒॒शेषु॑ श्रयद्ध्वं । स॒प्त॒विꣳ॒॒शा अ॑ष्टाविꣳ॒॒शेषु॑ श्रयद्ध्वं । अ॒ष्टा॒विꣳ॒॒शा ए॑कान्नत्रिꣳ॒॒शेषु॑ श्रयद्ध्वं । ए॒का॒न्न॒त्रिꣳ॒॒शा स्त्रिꣳ॒॒शेषु॑ श्रयद्ध्वं । त्रिꣳ॒॒शा ए॑कत्रिꣳ॒॒शेषु॑ श्रयद्ध्वं । ए॒क॒त्रिꣳ॒॒शा द्वा᳚त्रिꣳ॒॒शेषु॑ श्रयद्ध्वं । द्वा॒त्रिꣳ॒॒शा स्त्र॑यस्त्रिꣳ॒॒शेषु॑ श्रयद्ध्वं । देवा᳚स्त्रिरे-कादशा॒ स्त्रिस्त्र॑यस्त्रिꣳशाः । उत्त॑रे भवत । उत्त॑र वर्त्मान॒ उत्त॑र सत्वानः ( ) । यत्का॑म इ॒दं जु॒होमि॑ । तन्मे॒ समृ॑द्ध्यतां । व॒यꣳ स्या॑म॒ पत॑यो रयी॒णां । भूर्भुव॒स्वः॑ स्वाहा᳚ । \textbf{ 25} \newline
                  \newline
                                    (ष॒ष्ठेषु॑ श्रयद्धवꣳ - षोड॒शेषु॑ श्रयद्धवꣳ - षड्विꣳ॒॒शेषु॑ श्रयद्ध्व॒ - मुत्त॑रसत्वानश्च॒त्वारि॑ च) \textbf{(A2)} \newline \newline
                \textbf{ 3.11.3    अनुवाकं   3 -चतुर्भिरनुवाकैः चतस्र आहुतीर्जुहोति} \newline
                                \textbf{ TB 3.11.3.1} \newline
                  अग्ना॑विष्णू स॒जोष॑सा । इ॒मा व॑र्द्धन्तु वा॒ङ्गिरः॑ । द्यु॒म्नै र्वाजे॑भि॒-राग॑तं । राज्ञी॑ वि॒राज्ञी᳚ । स॒म्राज्ञी᳚ स्व॒राज्ञी᳚ । अ॒र्चिः शो॒चिः । तपो॒ हरो॒ भाः । अ॒ग्निः सोमो॒ बृह॒स्पतिः॑ ।विश्वे॑ दे॒वा भुव॑नस्य गो॒पाः । ते सर्वे॑ स॒ङ्गत्य॑ ( ) । इ॒दं मे॒ प्राव॑ता॒ वचः॑ । व॒यꣳ स्या॑म॒ पत॑यो रयी॒णां । भूर्-भुव॒स्-स्व॑स् स्वाहा᳚ । \textbf{ 26} \newline
                  \newline
                                    (स॒ङ्गत्य॒ त्रीणि॑ च) \textbf{(A3)} \newline \newline
                \textbf{ 3.11.4    अनुवाकं   4 -चतुर्भिरनुवाकैः चतस्र आहुतीर्जुहोति} \newline
                                \textbf{ TB 3.11.4.1} \newline
                  अन्न॑प॒ते ऽन्न॑स्य नो देहि । अ॒न॒मी॒वस्य॑ शु॒ष्मिणः॑ । प्रप्र॑दा॒तार॑न्तारिषः । ऊर्जं॑ नो धेहि द्वि॒पदे॒ चतु॑ष्पदे । अग्ने॑ पृथिवीपते । सोम॑ वीरुधां पते । त्वष्टः॑ समिधां पते । विष्ण॑वाशानां पते । मित्र॑ सत्यानां पते । वरु॑ण धर्मणां पते \textbf{ 27} \newline
                  \newline
                                \textbf{ TB 3.11.4.2} \newline
                  म॒रुतो॑ गणानां पतयः । रुद्र॑ पशूनां पते । इन्द्रौ॑जसां पते । बृह॑स्पते ब्रह्मणस्पते । आ रु॒चा रो॑चे॒ऽहꣳ स्व॒यं । रु॒चा रु॑रुचे॒-रोच॑मानः । अ॒तीत्या॒दः स्व॑राभ॑रे॒ह । तस्मि॒न॒. योनौ᳚ प्रज॒नौ प्रजा॑येय । व॒यꣳ स्या॑म॒ पत॑यो रयी॒णां । भूर्भुवः॒ स्वः॑ स्वाहा᳚ ( ) । \textbf{ 28} \newline
                  \newline
                                    [वरु॑ण धर्मणां पते॒ स्वः॑ - स्वाहा᳚ ( ) ] \textbf{(A4)} \newline \newline
                \textbf{ 3.11.5    अनुवाकं   5 -चतुर्भिरनुवाकैः चतस्र आहुतीर्जुहोति} \newline
                                \textbf{ TB 3.11.5.1} \newline
                  स॒प्त ते॑ अग्ने स॒मिधः॑ स॒प्त जि॒ह्वाः । स॒प्तर्.ष॑यः स॒प्तधाम॑ प्रि॒याणि॑ । स॒प्त होत्रा॑ अनु वि॒द्वान् । स॒प्त योनी॒-रापृ॑णस्वा घृ॒तेन॑ । प्राची॒ दिक् । अ॒ग्नि र्दे॒वता᳚ । अ॒ग्निꣳ स दि॒शां दे॒वं दे॒वता॑ना-मृच्छतु । यो मै॒तस्यै॑ दि॒शो॑ ऽभि॒दास॑ति । द॒क्षि॒णा दिक् । इन्द्रो॑ दे॒वता᳚ \textbf{ 29} \newline
                  \newline
                                \textbf{ TB 3.11.5.2} \newline
                  इन्द्रꣳ॒॒ स दि॒शां दे॒वं दे॒वता॑ना-मृच्छतु । यो मै॒तस्यै॑ दि॒शो॑ ऽभि॒दास॑ति । प्र॒तीची॒ दिक् । सोमो॑ दे॒वता᳚ । सोमꣳ॒॒ स दि॒शां दे॒वं दे॒वता॑ना-मृच्छतु । यो मै॒तस्यै॑ दि॒शो॑ ऽभि॒दास॑ति । उदी॑ची॒ दिक् । मि॒त्रावरु॑णौ दे॒वता᳚ ।मि॒त्रावरु॑णौ॒ स दि॒शां दे॒वौ दे॒वता॑ना-मृच्छतु । यो मै॒तस्यै॑ दि॒शो॑ ऽभि॒दास॑ति \textbf{ 30} \newline
                  \newline
                                \textbf{ TB 3.11.5.3} \newline
                  ऊ॒द्र्ध्वा दिक् । बृह॒स्पति॑ र्दे॒वता᳚ । बृह॒स्पतिꣳ॒॒ स दि॒शां दे॒वं दे॒वता॑ना-मृच्छतु । यो मै॒तस्यै॑ दि॒शो॑ ऽभि॒दास॑ति । इ॒यन्दिक् । अदि॑ति र्दे॒वता᳚ । अदि॑तिꣳ॒ ॒ स दि॒शां दे॒वीं दे॒वता॑ना-मृच्छतु । यो मै॒तस्यै॑ दि॒शो॑ ऽभि॒दास॑ति । पुरु॑षो॒ दिक् । पुरु॑षो मे॒ कामा॒न् थ्सम॑र्द्धयतु ( ) \textbf{ 31} \newline
                  \newline
                                \textbf{ TB 3.11.5.4} \newline
                  अ॒न्धो जागृ॑विः प्राण । असा॒वेहि॑ । ब॒धि॒र आ᳚क्रन्दयित-रपान । असा॒वेहि॑ । उ॒षस॑-मुषस-मशीय । अ॒हमसो॒ ज्योति॑रशीय ।अ॒हमसो॒ ऽपो॑शीय । व॒यꣳ स्या॑म॒ पत॑यो रयी॒णां । भूर्भुवः॒ स्वः॑ स्वाहा᳚ । \textbf{ 32} \newline
                  \newline
                                    (द॒क्षि॒णा दिगिन्द्रो॑ दे॒वता॑ - मि॒त्रावरु॑णौ॒ स दि॒शां दे॒वौ दे॒वता॑नामृच्छतु॒ 
यो मै॒तस्यै॑ दि॒शो॑ऽभि॒दा - स॑त्यद्र्धयतु॒ - +नव॑ च) \textbf{(A5)} \newline \newline
                \textbf{ 3.11.6    अनुवाकं   6 - उपस्थानम्} \newline
                                \textbf{ TB 3.11.6.1} \newline
                  यत्तेऽचि॑तं॒ ॅयदु॑ चि॒तन्ते॑ अग्ने । यत्त॑ ऊ॒नं ॅयदु॒ तेऽति॑रिक्तं । आ॒दि॒त्यास्त-दङ्गि॑रस-श्चिन्वन्तु । विश्वे॑ ते दे॒वाश्चिति॒-मापू॑रयन्तु । चि॒तश्चासि॒ सञ्चि॑तश्चा-स्यग्ने । ए॒तावाꣳ॒॒श्चासि॒ भूयाꣳ॑श्चास्यग्ने ॥ लो॒कं पृ॑ण च्छि॒द्रं पृ॑ण । अथो॑ सीद शि॒वा त्वं । इ॒न्द्रा॒ग्नी त्वा॒ बृह॒स्पतिः॑ । अ॒स्मिन्. योना॑वसीषदन्न् \textbf{ 33} \newline
                  \newline
                                \textbf{ TB 3.11.6.2} \newline
                  तया॑ दे॒वत॑या-ऽङ्गिर॒स्वद्-ध्रु॒वा सी॑द ॥ ता अ॑स्य॒ सूद॑दोहसः ।सोमꣳ॑ श्रीणन्ति॒ पृश्न॑यः । जन्मं॑ दे॒वानां॒ ॅविशः॑ ।त्रि॒ष्वा रो॑च॒ने दि॒वः । तया॑ दे॒वत॑या-ऽङ्गिर॒स्वद्-ध्रु॒वा सी॑द ॥अग्ने॑ दे॒वाꣳ इ॒हाव॑ह । ज॒ज्ञा॒नो वृ॒क्त ब॑र्.हिषे । असि॒ होता॑ न॒ ईड्यः॑ ॥ अग॑न्म म॒हा मन॑सा॒ यवि॑ष्ठं \textbf{ 34} \newline
                  \newline
                                \textbf{ TB 3.11.6.3} \newline
                  यो दी॒दाय॒ समि॑द्धः॒ स्वे दु॑रो॒णे । चि॒त्रभा॑नू॒ रोद॑सी अ॒न्तरु॒र्वी ।स्वा॑हुतं ॅवि॒श्वतः॑ प्र॒त्यञ्चं᳚ ॥ मे॒धा॒का॒रं ॅवि॒दथ॑स्य प्र॒साध॑नं ।अ॒ग्निꣳ होता॑रं परि॒भूत॑मं म॒तिं । त्वामर्भ॑स्य ह॒विषः॑ समा॒नमित् । त्वां म॒हो वृ॑णते॒ नरो॒ नान्यं-त्वत् ॥ म॒नु॒ष्वत्त्वा॒ निधी॑महि । म॒नु॒ष्वथ् समि॑धीमहि । अग्ने॑ मनु॒ष्व-द॑ङ्गिरः \textbf{ 35} \newline
                  \newline
                                \textbf{ TB 3.11.6.4} \newline
                  दे॒वान् दे॑वाय॒ते य॑ज ॥ अ॒ग्निर्. हि वा॒जिनं॑ ॅवि॒शे । ददा॑ति वि॒श्वच॑र्.षणिः । अ॒ग्नी रा॒ये स्वा॒भुवं᳚ । स प्री॒तो या॑ति॒ वार्यं᳚ । इषꣳ॑ स्तो॒तृभ्य॒ आभ॑र ॥पृ॒ष्टो दि॒वि पृ॒ष्टो अ॒ग्निः पृ॑थि॒व्यां । पृ॒ष्टो विश्वा॒ ओष॑धी॒-रावि॑वेश । वै॒श्वा॒न॒रः सह॑सा पृ॒ष्टो अ॒ग्निः । स नो॒ दिवा॒ स रि॒षः पा॑तु॒ नक्तं᳚ ( ) । \textbf{ 36} \newline
                  \newline
                                    (अ॒सी॒ष॒द॒न्॒. - यवि॑ष्ठ - मङ्गिरो॒ - नक्तं᳚ ) \textbf{(A6)} \newline \newline
                \textbf{ 3.11.7    अनुवाकं   7 -नाचिकेतब्राह्मणं तत्र आग्निदेवतोपासनम्} \newline
                                \textbf{ TB 3.11.7.1} \newline
                  अ॒यं ॅवाव यः पव॑ते । सो᳚ऽग्नि ना॑र्चिके॒तः । स यत् प्राङ् पव॑ते । तद॑स्य॒ शिरः॑ । अथ॒ यद् द॑क्षि॒णा । स दक्षि॑णः प॒क्षः ।अथ॒ यत् प्र॒त्यक् । तत् पुच्छं᳚ । य दुदङ्ङ्॑ । स उत्त॑रः प॒क्षः \textbf{ 37} \newline
                  \newline
                                \textbf{ TB 3.11.7.2} \newline
                  अथ॒ यथ् स॒म्ॅवाति॑ । तद॑स्य स॒मञ्च॑नं च प्र॒सार॑णं च । अथो॑ स॒पंदे॒वास्य॒ सा ॥ सꣳ ह॒ वा अ॑स्मै॒ स कामः॑ पद्यते । यत् का॑मो॒ यज॑ते । यो᳚ऽग्निं ना॑चिके॒तं चि॑नु॒ते ।य उ॑ चैन-मे॒वं ॅवेद॑ ॥ यो ह॒ वा अ॒ग्ने र्ना॑चिके॒तस्या॒यत॑नं प्रति॒॒ष्ठां ॅवेद॑ । आ॒यत॑नवान् भवति । गच्छ॑ति प्रति॒ष्ठां \textbf{ 38} \newline
                  \newline
                                \textbf{ TB 3.11.7.3} \newline
                  हिर॑ण्यं॒ ॅवा अ॒ग्ने र्ना॑चिके॒तस्या॒यत॑नं प्रति॒ष्ठा । य ए॒वं ॅवेद॑ ।आ॒यत॑नवान् भवति । गच्छ॑ति प्रति॒ष्ठां ॥यो ह॒ वा अ॒ग्ने र्ना॑चिके॒तस्य॒ शरी॑रं॒ ॅवेद॑ । स श॑रीर ए॒व स्व॒र्गं ॅलो॒कमे॑ति । हिर॑ण्यं॒ ॅवा अ॒ग्ने र्ना॑चिके॒तस्य॒ शरी॑रं । य ए॒वं ॅवेद॑ । स श॑रीर ए॒व स्व॒र्गं ॅलो॒कमे॑ति ॥ अथो॒ यथा॑ रु॒क्म उत्-त॑प्तो भा॒य्यात् \textbf{ 39} \newline
                  \newline
                                \textbf{ TB 3.11.7.4} \newline
                  ए॒वमे॒व स तेज॑सा॒ यश॑सा । अ॒स्मिꣳश्च॑ लो॒के॑-ऽमुष्मिꣳ॑श्च भाति ॥ उ॒रवो॑ ह॒ वै नामै॒ते लो॒काः । ये-ऽव॑रेणादि॒त्यं । अथ॑ है॒ते वरी॑याꣳसो लो॒काः । ये परे॑णादि॒त्यं । अन्त॑वन्तꣳ ह॒ वा ए॒ष क्ष॒य्यं ॅलो॒कं ज॑यति । यो-ऽव॑रेणादि॒त्यं । अथ॑ है॒षो॑-ऽन॒न्त-म॑पा॒र-म॑क्ष॒य्यं ॅलो॒कं ज॑यति । यः परे॑णादि॒त्यं ( ) । \textbf{ 40} \newline
                  \newline
                                \textbf{ TB 3.11.7.5} \newline
                  अ॒न॒न्तꣳ ह॒ वा अ॑पा॒र-म॑क्ष॒य्यं ॅलो॒कं ज॑यति । यो᳚ऽग्निं ना॑चिके॒तं चि॑नु॒ते । य उ॑ चैन-मे॒वं ॅवेद॑ ॥ अथो॒ यथा॒ रथे॒ तिष्ठ॒न् पक्ष॑सी पर्या॒-वर्त्त॑माने प्र॒त्यपे᳚क्षते । ए॒व-म॑होरा॒त्रे प्र॒त्यपे᳚क्षते । नास्या॑-होरा॒त्रे लो॒कमा᳚प्नुतः । यो᳚ऽग्निं ना॑चि॒केतं चि॑नु॒ते । य उ॑ चैन-मे॒वं ॅवेद॑ । \textbf{ 41} \newline
                  \newline
                                    (उत्त॑रः प॒क्षो - गच्छ॑ति प्रति॒ष्ठां - भा॒य्याद् - यः परे॑णादि॒त्य - +म॒ष्टौ च॑) \textbf{(A7)} \newline \newline
                \textbf{ 3.11.8    अनुवाकं   8 -नाचिकेतोपाख्यानम्} \newline
                                \textbf{ TB 3.11.8.1} \newline
                  उ॒शन्. ह॒ वै वा॑जश्रव॒सः स॑र्ववेद॒सं द॑दौ । तस्य॑ ह॒ नचि॑केता॒ नाम॑ पु॒त्र आ॑स । तꣳ ह॑ कुमा॒रꣳ सन्तं᳚ । दक्षि॑णासु नी॒यमा॑नासु श्र॒द्धा ऽऽवि॑वेश । स हो॑वाच । तत॒ कस्मै॒ मां दा᳚स्य॒सीति॑ । द्वि॒तीयं॑ तृ॒तीयं᳚ ॥ तꣳ ह॒ परी॑त उवाच ।मृ॒त्यवे᳚ त्वा ददा॒मीति॑ ॥ तꣳ ह॒ स्मोत्थि॑तं॒ ॅवाग॒भिव॑दति \textbf{ 42} \newline
                  \newline
                                \textbf{ TB 3.11.8.2} \newline
                  गौत॑म कुमा॒रमिति॑ । स हो॑वाच । परे॑हि मृ॒त्यो र्गृ॒हान् । मृ॒त्यवे॒ वै त्वा॑ऽदा॒-मिति॑ ॥ तं ॅवै प्र॒वस॑न्तं ग॒न्तासीति॑ होवाच । तस्य॑ स्म ति॒स्रो रात्री॒-रना᳚श्वान् गृ॒हे व॑सतात् । स यदि॑ त्वा पृ॒च्छेत् । कुमा॑र॒ कति॒ रात्री॑-रवाथ्सी॒-रिति॑ । ति॒स्र इति॒ प्रति॑ ब्रूतात् । किं प्र॑थ॒माꣳ रात्रि॑-माश्ना॒ इति॑ \textbf{ 43} \newline
                  \newline
                                \textbf{ TB 3.11.8.3} \newline
                  प्र॒जां त॒ इति॑ । किं द्वि॒तीया॒-मिति॑ । प॒शूꣳस्त॒ इति॑ । किं तृ॒तीया॒-मिति॑ । सा॒धु॒कृ॒त्यां त॒ इति॑ ॥ तं ॅवै प्र॒वस॑न्तं जगाम । तस्य॑ ह ति॒स्रो रात्री॒-रना᳚श्वान् गृ॒ह उ॑वास । तमा॒गत्य॑ पप्रच्छ । कुमा॑र॒ कति॒ रात्री॑-रवाथ्सी॒-रिति॑ । ति॒स्र इति॒ प्रत्यु॑वाच \textbf{ 44} \newline
                  \newline
                                \textbf{ TB 3.11.8.4} \newline
                  किं प्र॑थ॒माꣳ रात्रि॑-माश्ना॒ इति॑ । प्र॒जां त॒ इति॑ । किं द्वि॒तीया॒-मिति॑ । प॒शूꣳस्त॒ इति॑ । किं तृ॒तीया॒-मिति॑ । सा॒धु॒कृ॒त्यां त॒ इति॑ ॥ नम॑स्ते अस्तु भगव॒ इति॑ होवाच । वरं॑ ॅवृणी॒ष्वेति॑ ॥ पि॒तर॑मे॒व जीव॑न्नया॒नीति॑ ॥ द्वि॒तीयं॑ ॅवृणी॒ष्वेति॑ \textbf{ 45} \newline
                  \newline
                                \textbf{ TB 3.11.8.5} \newline
                  इ॒ष्टा॒पू॒र्तयो॒ र्मे ऽक्षि॑तिं ब्रू॒हीति॑ होवाच । तस्मै॑ है॒तम॒ग्निं ना॑चिके॒त-मु॑वाच । ततो॒ वै तस्ये᳚-ष्टापू॒र्ते ना क्षी॑येते ॥ नास्ये᳚ष्टा-पू॒र्ते क्षी॑येते । योऽ᳚ग्निं ना॑चिके॒तं चि॑नु॒ते । य उ॑ चैन-मे॒वं ॅवेद॑ ॥ तृ॒तीयं॑ ॅवृणी॒ष्वेति॑ । पु॒न॒ मृ॒र्त्यो र्मे ऽप॑जितिं ब्रू॒हीति॑ होवाच । तस्मै॑ है॒तम॒ग्निं ना॑चिके॒त-मु॑वाच । ततो॒ वै सोऽप॑ पुन र्मृ॒त्यु-म॑जयत् \textbf{ 46} \newline
                  \newline
                                \textbf{ TB 3.11.8.6} \newline
                  अप॑ पुन र्मृ॒त्युं ज॑यति । योऽ᳚ग्निं ना॑चिके॒तं चि॑नु॒ते ॥ य उ॑ चैन-मे॒वं ॅवेद॑ ॥ प्र॒जाप॑ति॒ र्वै प्र॒जाका॑म॒-स्तपो॑-ऽतप्यत ।स हिर॑ण्य॒-मुदा᳚स्यत् । त-द॒ग्नौ प्रास्य॑त् । त-द॑स्मै॒ नाच्छ॑दयत् । तद् द्वि॒तीयं॒ प्रास्य॑त् । त-द॑स्मै॒ नै वाच्छ॑दयत् । तत् तृ॒तीयं॒ प्रास्य॑त् \textbf{ 47} \newline
                  \newline
                                \textbf{ TB 3.11.8.7} \newline
                  त-द॑स्मै॒ नै वाच्छ॑दयत् । त-दा॒त्म-न्ने॒व हृ॑द॒य्ये᳚ऽग्नौ वै᳚श्वान॒रे प्रास्य॑त् । त-द॑स्मा अच्छदयत् । तस्मा॒द्धिर॑ण्यं॒ कनि॑ष्ठं॒ धना॑नां । भु॒ञ्जत् प्रि॒यत॑मं । हृ॒द॒य॒जꣳ हि । स वै तमे॒व नावि॑न्दत् । यस्मै॒ तां दक्षि॑णा॒-मने᳚ष्यत् । ताꣳ स्वायै॒व हस्ता॑य॒ दक्षि॑णायानयत् । तां प्रत्य॑गृह्णात् \textbf{ 48} \newline
                  \newline
                                \textbf{ TB 3.11.8.8} \newline
                  दक्षा॑य त्वा॒ दक्षि॑णां॒ प्रति॑गृह्णा॒मीति॑ । सो॑ ऽदक्षत॒ दक्षि॑णां प्रति॒गृह्य॑ । दक्ष॑ते ह॒ वै दक्षि॑णां प्रति॒गृह्य॑ । य ए॒वं ॅवेद॑ ॥ ए॒तद्ध॑ स्म॒ वै तद् वि॒द्वाꣳसो॑ वाजश्रव॒सा गोत॑माः । अप्य॑नूदे॒श्यां᳚ दक्षि॑णां॒ प्रति॑गृह्णन्ति । उ॒भये॑न व॒यं द॑क्षिष्यामह ए॒व दक्षि॑णां प्रति॒गृह्येति॑ । ते॑ ऽदक्षन्त॒ दक्षि॑णां प्रति॒गृह्य॑ ।दक्ष॑ते ह॒ वै दक्षि॑णां प्रति॒गृह्य॑ । य ए॒वं ॅवेद॑ ( ) । प्र हा॒न्यं ॅव्ली॑नाति । \textbf{ 49} \newline
                  \newline
                                    (व॒द॒ - त्या॒श्ना॒ इत् - यु॑वाच - द्वि॒तीयं॑ ॅवृणी॒ष्वे - त्य॑जयत् - तृ॒तीयं॒ प्रास्य॑ - दगृह्णा॒द् - य ए॒वं ॅवेदैकं॑ च) \textbf{(A8)} \newline \newline
                \textbf{ 3.11.9    अनुवाकं   9 -चयनप्रयोगः} \newline
                                \textbf{ TB 3.11.9.1} \newline
                  तꣳ है॒त-मेके॑ पशुब॒न्ध ए॒वोत्त॑रवे॒द्यां चि॑न्वते । उ॒त्त॒र॒वे॒दिस॑म्मित ए॒षो᳚ऽग्निरिति॒ वद॑न्तः । तन्न तथा॑ कु॒र्यात् । ए॒त-म॒ग्निं कामे॑न॒ व्य॑र्द्धयेत् । स ए॑नं॒ कामे॑न॒ व्यृ॑द्धः । कामे॑न॒ व्य॑र्द्धयेत् ।सौ॒म्ये वावै न॑मध्व॒रे चि॑न्वी॒त । यत्र॑ वा॒ भूयि॑ष्ठा॒ आहु॑तयो हू॒येरन्न्॑ । ए॒त-म॒ग्निं कामे॑न॒ सम॑र्द्धयति । स ए॑नं॒ कामे॑न॒ समृ॑द्धः \textbf{ 50} \newline
                  \newline
                                \textbf{ TB 3.11.9.2} \newline
                  कामे॑न॒ सम॑र्द्धयति ॥ अथ॑ हैनं पु॒रर्.ष॑यः । उ॒त्त॒र॒वे॒द्या-मे॒व स॒त्रिय॑-मचिन्वत । ततो॒ वै तेऽवि॑न्दन्त प्र॒जां ।अ॒भि स्व॒र्गं ॅलो॒क-म॑जयन्न् । वि॒न्दत॑ ए॒व प॒जां । अ॒भि स्व॒र्गं ॅलो॒कं ज॑यति । यो᳚ऽग्निं ना॑चिके॒तं चि॑नु॒ते ।य उ॑ चैन-मे॒वं ॅवेद॑ ॥ अथ॑ हैनं ॅवा॒युर्. ऋद्धि॑कामः \textbf{ 51} \newline
                  \newline
                                \textbf{ TB 3.11.9.3} \newline
                  य॒था॒न्यु॒प्त-मे॒वोप॑दधे । ततो॒ वै स ए॒ता-मृद्धि॑-माद्र्ध्नोत् ।यामि॒दं ॅवा॒युर्. ऋ॒द्धः । ए॒ता-मृद्धि॑-मृद्ध्नोति । यामि॒दं ॅवा॒युर्. ऋ॒द्धः । यो᳚ऽग्निं ना॑चिके॒तं चि॑नु॒ते । य उ॑ चैन-मे॒वं ॅवेद॑ ॥ अथ॑ हैनं गोब॒लो वार्ष्णः॑ प॒शुका॑मः । पांक्त॑मे॒व चि॑क्ये । पञ्च॑ पु॒रस्ता᳚त् \textbf{ 52} \newline
                  \newline
                                \textbf{ TB 3.11.9.4} \newline
                  पञ्च॑ दक्षिण॒तः । पञ्च॑ प॒श्चात् । पञ्चो᳚त्तर॒तः । एकां॒ मद्ध्ये᳚ ।ततो॒ वै स स॒हस्रं॑ प॒शून् प्राप्नो᳚त् । प्र स॒हस्रं॑ प॒शू-ना᳚प्नोति ।यो᳚ऽग्निं ना॑चिके॒तं चि॑नु॒ते । य उ॑ चैन-मे॒वं ॅवेद॑ ॥ अथ॑ हैनं प्र॒जाप॑ति॒ र्ज्यैष्ठ्य॑कामो॒ यश॑स्कामः प्र॒जन॑नकामः । त्रि॒वृत॑मे॒व चि॑क्ये \textbf{ 53} \newline
                  \newline
                                \textbf{ TB 3.11.9.5} \newline
                  स॒प्त पु॒रस्ता᳚त् । ति॒स्रो द॑क्षिण॒तः । स॒प्त प॒श्चात् । ति॒स्र उ॑त्तर॒तः । एकां॒ मद्ध्ये᳚ । ततो॒ वै स प्र यशो॒ ज्यैष्ठ्य॑-माप्नोत् । ए॒तां प्रजा॑तिं॒ प्राजा॑यत । यामि॒दं प्र॒जाः प्र॒जाय॑न्ते । त्रि॒वृद् वै ज्यैष्ठ्यं᳚ । मा॒ता पि॒ता पु॒त्रः \textbf{ 54} \newline
                  \newline
                                \textbf{ TB 3.11.9.6} \newline
                  त्रि॒वृत् प्र॒जन॑नं । उ॒पस्थो॒ योनि॑ र्मद्ध्य॒मा । प्रयशो॒ ज्यैष्ठ्य॑-माप्नोति । ए॒तां प्र॑जातिं॒ प्रजा॑यते । यामि॒दं प्र॒जाः प्र॒जाय॑न्ते । यो᳚ऽग्निं ना॑चिके॒तं चि॑नु॒ते । य उ॑ चैनमे॒वं ॅवेद॑ ॥ अथ॑ हैन॒-मिन्द्रो॒ ज्यैष्ठ्य॑कामः ।ऊ॒द्र्ध्वा ए॒वोप॑दधे । ततो॒ वै स ज्यैष्ठ्य॑-मगच्छत् \textbf{ 55} \newline
                  \newline
                                \textbf{ TB 3.11.9.7} \newline
                  ज्यैष्ठ्यं॑ गच्छति । यो᳚ऽग्निं ना॑चिके॒तं चि॑नु॒ते । य उ॑ चैन-मे॒वं ॅवेद॑ ॥ अथ॑ हैन-म॒सावा॑दि॒त्यः स्व॒र्गका॑मः ।प्राची॑रे॒वोप॑दधे । ततो॒ वै सो॑ऽभि स्व॒र्गं ॅलो॒क-म॑जयत् । अ॒भि स्व॒र्गं ॅलो॒कं ज॑यति । यो᳚ऽग्निं ना॑चिके॒तं चि॑नु॒ते ।य उ॑ चैन-मे॒वं ॅवेद॑ ॥ स यदी॒च्छेत् \textbf{ 56} \newline
                  \newline
                                \textbf{ TB 3.11.9.8} \newline
                  ते॒ज॒स्वी य॑श॒स्वी ब्र॑ह्मवर्च॒सी स्या॒मिति॑ । प्राङा होतु॒र्द्धिष्ण्या॒-दुथ्स॑र्पेत् । येयं प्रागा॒द् यश॑स्वती । सा मा॒ प्रोर्णो॑तु ।तेज॑सा॒ यश॑सा ब्रह्मवर्च॒सेनेति॑ । ते॒ज॒स्व्ये॑व य॑श॒स्वी ब्र॑ह्मवर्च॒सी भ॑वति ॥ अथ॒ यदी॒च्छेत् । भूयि॑ष्ठं मे॒ श्रद्द॑धीरन्न् । भूयि॑ष्ठा॒ दक्षि॑णा नयेयु॒रिति॑ । दक्षि॑णासु नी॒यमा॑नासु॒ प्राच्येहि॒ प्राच्ये॒हीति॒ प्राची॑ जुषा॒णा वेत्वाज्य॑स्य॒ स्वाहेति॑ स्रु॒वेणो॑-प॒हत्या॑ हव॒नीये॑ जुहुयात् \textbf{ 57} \newline
                  \newline
                                \textbf{ TB 3.11.9.9} \newline
                  भूयि॑ष्ठ-मे॒वास्मै॒ श्रद्द॑धते । भूयि॑ष्ठा॒ दक्षि॑णा नयन्ति ॥ पुरी॑ष-मुप॒धाय॑ । चि॒ति॒-क्लृ॒प्तिभि॑-रभि॒मृश्य॑ । अ॒ग्निं प्र॒णीयो॑पसमा॒धाय॑ । चत॑स्र ए॒ता आहु॑ती र्जुहोति । त्वम॑ग्ने रु॒द्र इति॑ शतरु॒द्रीय॑स्य रू॒पं । अग्ना॑-विष्णू॒ इति॑ वसो॒द्र्धारा॑याः । अन्न॑पत॒ इत्य॑न्न हो॒मः । स॒प्त ते॑ अग्ने स॒मिधः॑ स॒प्त जि॒ह्वा इति॑ विश्व॒प्रीः ( ) । \textbf{ 58} \newline
                  \newline
                                                        \textbf{special korvai} \newline
              (पुरर्.ष॑यो वा॒युर् गो॑ब॒लः स॒हस्रं॑ प्र॒जाप॑ति स्त्रि॒वृदिन्द्रो॒ ऽसावा॑दि॒त्यः स यदी॒च्छेत् ) \newline
                                [समृ॑द्ध॒ - ऋद्धि॑कामः - प॒रस्ता᳚च् - चिक्ये - पु॒त्रो॑ - ऽगच्छ - दि॒च्छे - ज्जु॑हुयाद् - विश्व॒प्रिः ( ) ] \textbf{(A9)} \newline \newline
                \textbf{ 3.11.10   अनुवाकं   10 -तत्प्रशंसा} \newline
                                \textbf{ TB 3.11.10.1} \newline
                  यां प्र॑थ॒मा-मिष्ट॑का-मुप॒दधा॑ति । इ॒मं तया॑ लो॒क-म॒भि ज॑यति ।अथो॒ या अ॒स्मिन् ॅलो॒के दे॒वताः᳚ । तासाꣳ॒॒ सायु॑ज्यꣳ सलो॒कता॑-माप्नोति । यां द्वि॒तीया॑-मुप॒दधा॑ति । अ॒न्त॒रि॒क्ष॒लो॒कं तया॒-ऽभि जय॑ति । अथो॒ या अ॑न्तरिक्ष लो॒के दे॒वताः᳚ । तासाꣳ॒॒ सायु॑ज्यꣳ सलो॒कता॑-माप्नोति । यां तृ॒तीया॑-मुप॒दधा॑ति । अ॒मुं तया॑ लो॒क-म॒भिज॑यति \textbf{ 59} \newline
                  \newline
                                \textbf{ TB 3.11.10.2} \newline
                  अथो॒ या अ॒मुष्मि॑न् ॅलो॒के दे॒वताः᳚ । तासाꣳ॒॒ सायु॑ज्यꣳ सलो॒कता॑-माप्नोति । अथो॒ या अ॒मूरित॑रा अ॒ष्टाद॑श । य ए॒वामी उ॒रव॑श्च॒ वरी॑याꣳसश्च लो॒काः ।ताने॒व ताभि॑-र॒भिज॑यति ॥ का॒म॒चारो॑ ह॒ वा अ॑स्यो॒रुषु॑ च॒ वरी॑यस्सु च लो॒केषु॑ भवति । यो᳚ऽग्निं ना॑चिके॒तं चि॑नु॒ते । य उ॑ चैन-मे॒वं ॅवेद॑ ॥ स॒म्ॅव॒थ्स॒रो वा अ॒ग्नि ना॑र्चिके॒तः । तस्य॑ वस॒न्तः शिरः॑ \textbf{ 60} \newline
                  \newline
                                \textbf{ TB 3.11.10.3} \newline
                  ग्री॒ष्मो दक्षि॑णः प॒क्षः । व॒र्॒.षा उत्त॑रः । श॒रत् पुच्छं᳚ । मासाः᳚ कर्मका॒राः । अ॒हो॒रा॒त्रे श॑तरु॒द्रीयं᳚ । प॒र्जन्यो॒ वसो॒द्र्धारा᳚ ।यथा॒ वै प॒र्जन्यः॒ सुवृ॑ष्टं ॅवृ॒ष्ट्वा । प्र॒जाभ्यः॒ सर्वा॒न् कामा᳚न्थ् संपू॒रय॑ति । ए॒व-मे॒व स तस्य॒ सर्वा॒न् कामा॒न्थ् संपू॑रयति । यो᳚ऽग्निं ना॑चिके॒तं चि॑नु॒ते \textbf{ 61} \newline
                  \newline
                                \textbf{ TB 3.11.10.4} \newline
                  य उ॑ चैन-मे॒वं ॅवेद॑ ॥ स॒म्ॅव॒थ्स॒रो वा अ॒ग्नि र्ना॑चिके॒तः । तस्य॑ वस॒न्तः शिरः॑ । ग्री॒ष्मो दक्षि॑णः प॒क्षः ।व॒॒र्॒.षाः पुच्छं᳚ । श॒रदुत्त॑रः प॒क्षः । हे॒म॒न्तो मद्ध्यं᳚ । पू॒र्व॒प॒क्षाश्चित॑यः । अ॒प॒र॒प॒क्षाः पुरी॑षं । अ॒हो॒रा॒त्राणीष्ट॑काः ( ) ।ए॒ष वाव सो᳚ऽग्नि-र॑ग्नि॒मयः॑ पुनर्ण॒वः । अ॒ग्नि॒मयो॑ ह॒ वै पु॑नर्ण॒वो भू॒त्वा । स्व॒र्गं ॅलो॒क-मे॑ति । आ॒दि॒त्यस्य॒ सायु॑ज्यं । यो᳚ऽग्निं ना॑चिके॒तं चि॑नु॒ते । य उ॑ चैन-मे॒वं ॅवेद॑ । \textbf{ 62} \newline
                  \newline
                                    अ॒मुं तया॑ लो॒कम॒भिज॑यति॒ - शिर॑ - श्चिनु॒त - इष्ट॑काः॒ षट्च॑) \textbf{(A10)} \newline \newline
                \textbf{PrapAtaka Korvai with starting  words of 1 to 10 anuvAkams :-} \newline
        (लो॒क - स्त्वम॒ग्ने - ऽग्ना॑विष्णू॒ - अन्न॑पते - स॒प्त ते॑ अग्ने॒ - यत्ते ऽचि॑तम॒ - यं ॅवा - वोशन्. ह॒ वै - तꣳ है॒तं - ॅयां प्र॑थ॒मामिष्ट॑कां॒ दश॑) \newline

        \textbf{korvai with starting words of 1, 11, 21 series of daSinis :-} \newline
        (लो॒क - आ॑दि॒त्य - ओजो᳚ - ऽस्यू॒द्र्ध्वा दिग॑ - न॒न्तꣳ ह॒ वै - कामे॑न - ग्री॒ष्मो द्विष॑ष्टिः) \newline

        \textbf{first and last  word 2nd prapATakam of Kaatakam:-} \newline
        (लो॒को - वेद॑) \newline 

       

        ॥ हरिः॑ ॐ ॥
॥ तैत्तरीय यजुब्राह्मणे काठके द्वितीयः प्रश्नः समाप्तः ॥
+++++++++++++++++++++++++++++++++ \newline
        \pagebreak
        
        
        
     \addcontentsline{toc}{section}{ 3.12     तैत्तरीय यजुर्ब्राह्मणे काठके तृतीयः प्रश्नः  चातुर्होत्रचयनं वैश्वसृजचयनं च}
     \markright{ 3.12     तैत्तरीय यजुर्ब्राह्मणे काठके तृतीयः प्रश्नः  चातुर्होत्रचयनं वैश्वसृजचयनं च \hfill https://www.vedavms.in \hfill}
     \section*{ 3.12     तैत्तरीय यजुर्ब्राह्मणे काठके तृतीयः प्रश्नः  चातुर्होत्रचयनं वैश्वसृजचयनं च }
                \textbf{ 3.12.1    अनुवाकं   1 -वैश्वसृजचयनाङ्गभूताः दिवश्श्येनय इष्टयः} \newline
                                \textbf{ TB 3.12.1.1} \newline
                  “तुभ्य॒न्ता अ॑ङ्गिरस्तमा॒{21}“ “श्याम॒ तङ्काम॑मग्ने {22}” । “आशा॑नां त्वा॒{23}” “विश्वा॒ आशाः᳚{24}” । “अनु॑ नो॒ऽद्यानु॑मति॒{25}” “रन्विद॑नुमते॒ त्वं{26}” । “कामो॑ भू॒तस्य॒{27}” “काम॒स्तदग्रे᳚{28}” । “ब्रह्म॑ जज्ञा॒नं{29}” “पि॒ता वि॒राजां᳚{30}” । “य॒ज्ञो रा॒यो॑{31}” “ऽयं ॅय॒ज्ञ्ः{32}” । “आपो॑ भ॒द्रा{33}” “आदित्प॑श्यामि{34}” । “तुभ्यं॑ भरन्ति॒{35]” “यो दे॒ह्यः{36}” । “पूर्वं॑ देवा॒ अप॑रेण{37}” “प्राणापा॒नौ{38}” । “ह॒व्य॒वाहꣳ॒॒{39}” “स्वि॑ष्टं{40}” । \textbf{ 1} \newline
                  \newline
                                    (तुभ्यं॒ दश॑) \textbf{(A1)} \newline \newline
                \textbf{ 3.12.2    अनुवाकं   2 -तासां ब्राह्मणम्} \newline
                                \textbf{ TB 3.12.2.1} \newline
                  दे॒वेभ्यो॒ वै स्व॒र्गो लो॒कस्ति॒रो॑ऽभवत् । ते प्र॒जाप॑ति-मब्रुवन्न् । प्रजा॑पते स्व॒र्गो वै नो॑ लो॒कस्ति॒रो॑ऽभूत् । तमन्वि॒च्छेति॑ । तं ॅय॑ज्ञ्क्र॒तुभि॒-रन्वै᳚च्छत् । तं ॅय॑ज्ञ्क्र॒तुभि॒-र्नान्व॑विन्दत् ।तमिष्टि॑भि॒-रन्वै᳚च्छत् । तमिष्टि॑भि॒-रन्व॑विन्दत् ।तदिष्टी॑ना-मिष्टि॒त्वं । एष्ट॑यो ह॒वै नाम॑ । ता इष्ट॑य॒ इत्याच॑क्षते प॒रोक्षे॑ण । प॒रोक्ष॑ प्रिया इव॒ हि दे॒वाः । \textbf{ 2} \newline
                  \newline
                                \textbf{ TB 3.12.2.2} \newline
                  तमाशा᳚-ऽब्रवीत् । प्रजा॑पत आ॒शया॒ वै श्रा᳚म्यसि ।अ॒हमु॒ वा आशा᳚ऽस्मि । मां नु य॑जस्व । अथ॑ ते स॒त्याऽऽशा॑ भविष्यति । अनु॑ स्व॒र्गं ॅलो॒कं ॅवे॒थ्स्यसीति॑ । स ए॒तम॒ग्नये॒ कामा॑य पुरो॒डाश॑-म॒ष्टाक॑पालं॒ निर॑वपत् ।आ॒शायै॑ च॒रुं । अनु॑मत्यै च॒रुं । ततो॒ वै तस्य॑ स॒त्या-ऽऽशा॑ ऽभवत् । अनु॑ स्व॒र्गं ॅलो॒क-म॑विन्दत् । स॒त्या ह॒ वा अ॒स्याशा॑ भवति । अनु॑ स्व॒र्गं ॅलो॒कं ॅवि॑न्दति । य ए॒तेन॑ ह॒विषा॒ यज॑ते । य उ॑ चैनदे॒वं ॅवेद॑ ॥ सोऽत्र॑ जुहोति । अ॒ग्नये॒ कामा॑य॒ स्वाहा॒-ऽऽशायै॒ स्वाहा᳚ । अनु॑मत्यै॒ स्वाहा᳚ प्र॒जाप॑तये॒ स्वाहा᳚ । स्व॒र्गाय॑ लो॒काय॒ स्वाहा॒ऽग्नये᳚ स्विष्ट॒कृते॒ स्वाहेति॑ । \textbf{ 3} \newline
                  \newline
                                \textbf{ TB 3.12.2.3} \newline
                  तं कामो᳚ऽब्रवीत् । प्रजा॑पते॒ कामे॑न॒ वै श्रा᳚म्यसि । अ॒हमु॒वै कामो᳚ऽस्मि । मां नु य॑जस्व । अथ॑ ते स॒त्यः कामो॑ भविष्यति । अनु॑ स्व॒र्गं ॅलो॒कं ॅवे॒थ्स्यसीति॑ । स ए॒तम॒ग्नये॒ कामा॑य पुरो॒डाश॑-म॒ष्टाक॑पालं॒ निर॑वपत् । कामा॑य च॒रुं । अनु॑मत्यै च॒रुं । ततो॒ वै तस्य॑ स॒त्यः कामो॑ऽभवत् । अनु॑ स्व॒र्गं ॅलो॒कम॑विन्दत् । स॒त्यो ह॒ वा अ॑स्य॒ कामो॑ भवति ।अनु॑ स्व॒र्गं ॅलो॒कं ॅवि॑न्दति । य ए॒तेन॑ ह॒विषा॒ यज॑ते ।य उ॑ चैनदे॒वं ॅवेद॑ । सोऽत्र॑ जुहोति । अ॒ग्नये॒ कामा॑य॒ स्वाहा॒ कामा॑य॒ स्वाहा᳚ । अनु॑मत्यै॒ स्वाहा᳚ प्र॒जाप॑तये॒ स्वाहा᳚ ।स्व॒र्गाय॑ लो॒काय॒ स्वाहा॒ऽग्नये᳚ स्विष्ट॒कृते॒ स्वाहेति॑ । \textbf{ 4} \newline
                  \newline
                                \textbf{ TB 3.12.2.4} \newline
                  तं ब्रह्मा᳚ऽब्रवीत् । प्रजा॑पते॒ ब्रह्म॑णा॒ वैश्रा᳚म्यसि । अ॒हमु॒ वै ब्रह्मा᳚ऽस्मि । मां नु य॑जस्व । अथ॑ ते ब्रह्म॒ण्वान्. य॒ज्ञो भ॑विष्यति । अनु॑ स्व॒र्गं ॅलो॒कं ॅवे॒थ्स्यसीति॑ । स ए॒तम॒ग्नये॒ कामा॑य पुरो॒डाश॑-म॒ष्टाक॑पालं॒ निर॑वपत् । ब्रह्म॑णे च॒रुं । अनु॑मत्यै च॒रुं । ततो॒ वै तस्य॑ ब्रह्म॒ण्वान्. य॒ज्ञो॑ऽभवत् । अनु॑ स्व॒र्गं ॅलो॒कम॑विन्दत् । ब्र॒ह्म॒ण्वान्. ह॒ वा अ॑स्य य॒ज्ञो भ॑वति । अनु॑ स्व॒र्गं ॅलो॒कं ॅवि॑न्दति । य ए॒तेन॑ ह॒विषा॒ यज॑ते । य उ॑चैनदे॒वं ॅवेद॑ । सोऽत्र॑ जुहोति । अ॒ग्नये॒ कामा॑य॒ स्वाहा॒ ब्रह्म॑णे॒ स्वाहा᳚ । अनु॑मत्यै॒ स्वाहा᳚ प्र॒जाप॑तये॒ स्वाहा᳚ । स्व॒र्गाय॑ लो॒काय॒ स्वाहा॒ ऽग्नये᳚ स्विष्ट॒कृते॒ स्वाहेति॑ । \textbf{ 5} \newline
                  \newline
                                \textbf{ TB 3.12.2.5} \newline
                  तं ॅय॒ज्ञो᳚ऽब्रवीत् । प्रजा॑पते य॒ज्ञेन॒ वै श्रा᳚म्यसि । अ॒हमु॒ वै य॒ज्ञो᳚ऽस्मि । मां नु य॑जस्व । अथ॑ ते स॒त्यो य॒ज्ञो भ॑वष्यति । अनु॑ स्व॒र्गं ॅलो॒कं ॅवे॒थ्स्यसीति॑ । स ए॒तम॒ग्नये॒ कामा॑य पुरो॒डाश॑-म॒ष्टाक॑पालं॒ निर॑वपत् । य॒ज्ञाय॑ च॒रुं । अनु॑मत्यै च॒रुं । ततो॒ वै तस्य॑ स॒त्यो य॒ज्ञो॑ऽभवत् । अनु॑ स्व॒र्गं ॅलो॒कम॑विन्दत् । स॒त्यो ह॒ वा अ॑स्य य॒ज्ञो भ॑वति । अनु॑ स्व॒र्गं ॅलो॒कं ॅवि॑न्दति । य ए॒तेन॑ ह॒विषा॒ यज॑ते । य उ॑चैनदे॒वं ॅवेद॑ । सोऽत्र॑ जुहोति । अ॒ग्नये॒ कामा॑य॒ स्वाहा॑ य॒ज्ञाय॒ स्वाहा᳚ । अनु॑मत्यै॒ स्वाहा᳚ प्र॒जाप॑तये॒ स्वाहा᳚ । स्व॒र्गाय॑ लो॒काय॒ स्वाहा॒ऽग्नये᳚ स्विष्ट॒कृते॒ स्वाहेति॑ । \textbf{ 6} \newline
                  \newline
                                \textbf{ TB 3.12.2.6} \newline
                  तमापो᳚-ऽब्रुवन्न् । प्रजा॑पते॒ऽफ्सु वै सर्वे॒ कामाः᳚ श्रि॒ताः । व॒यमु॒ वा आप॑स्स्मः । अ॒स्मान्नु य॑जस्व । अथ॒ त्वयि॒ सर्वे॒ कामाः᳚ श्रयिष्यन्ते । अनु॑ स्व॒र्गं ॅलो॒कं ॅवे॒थ्स्यसीति॑ । स ए॒तम॒ग्नये॒ कामा॑य पुरो॒डाश॑-म॒ष्टाक॑पालं॒ निर॑वपत् । अ॒द्भ्यश्च॒रुं । अनु॑मत्यै च॒रुं । ततो॒ वै तस्मि॒न्थ्-सर्वे॒ कामा॑ अश्रयन्त । अनु॑ स्व॒र्गं ॅलो॒कम॑विन्दत् । सर्वे॑ ह॒ वा अ॑स्मि॒न् कामाः᳚ श्रयन्ते । अनु॑ स्व॒र्गं ॅलो॒कं ॅवि॑न्दति । य ए॒तेन॑ ह॒विषा॒ यज॑ते । य उ॑चैनदे॒वं ॅवेद॑ । सोऽत्र॑ जुहोति । अ॒ग्नये॒ कामा॑य॒ स्वाहा॒ ऽद्भ्यः स्वाहा᳚ ।अनु॑मत्यै॒ स्वाहा᳚ प्र॒जाप॑तये॒ स्वाहा᳚ । स्व॒र्गाय॑ लो॒काय॒ स्वाहा॒ऽग्नये᳚ स्विष्ट॒कृते॒ स्वाहेति॑ । \textbf{ 7} \newline
                  \newline
                                \textbf{ TB 3.12.2.7} \newline
                  तम॒ग्नि र्ब॑लि॒मान॑ब्रवीत् । प्रजा॑पते॒ ऽग्नये॒ वै ब॑लि॒मते॒ सर्वा॑णि भू॒तानि॑ ब॒लिꣳ ह॑रन्ति । अ॒हमु॒ वा अ॒ग्नि र्ब॑लि॒मान॑स्मि । मां नु य॑जस्व । अथ॑ ते॒ सर्वा॑णि भू॒तानि॑ ब॒लिꣳ ह॑रिष्यन्ति । अनु॑ स्व॒र्गं ॅलो॒कं ॅवे॒थ्स्यसीति॑ । स ए॒तम॒ग्नये॒ कामा॑य पुरो॒डाश॑-म॒ष्टाक॑पाल॒-न्निर॑वपत् । अ॒ग्नये॑ बलि॒मते॑ च॒रुं । अनु॑मत्यै च॒रुं । ततो॒ वै तस्मै॒ सर्वा॑णि भू॒तानि॑ ब॒लि-म॑हरन्न् । अनु॑ स्व॒र्गं ॅलो॒कम॑विन्दत् । सर्वा॑णि ह॒ वा अ॑स्मै भू॒तानि॑ ब॒लिꣳ ह॑रन्ति । अनु॑ स्व॒र्गं ॅलो॒कं ॅवि॑न्दति । य ए॒तेन॑ ह॒विषा॒ यज॑ते । य उ॑चैनदे॒वं ॅवेद॑ । सोऽत्र॑ जुहोति ।अ॒ग्नये॒ कामा॑य॒ स्वाहा॒ ऽग्नये॑ बलि॒मते॒ स्वाहा᳚ । अनु॑मत्यै॒ स्वाहा᳚ प्र॒जाप॑तये॒ स्वाहा᳚ । स्व॒र्गाय॑ लो॒काय॒ स्वाहा॒ऽग्नये᳚ स्विष्ट॒कृते॒ स्वाहेति॑ । \textbf{ 8} \newline
                  \newline
                                \textbf{ TB 3.12.2.8} \newline
                  तमनु॑वित्तिर-ब्रवीत् । प्रजा॑पते स्व॒र्गं ॅवै लो॒कमनु॑-विविथ्ससि । अ॒हमु॒ वा अनु॑वित्तिरस्मि । मां नु य॑जस्व । अथ॑ ते स॒त्या ऽनु॑वित्ति-र्भविष्यति । अनु॑ स्व॒र्गं ॅलो॒कं ॅवे॒थ्स्यसीति॑ । स ए॒तम॒ग्नये॒ कामा॑य पुरो॒डाश॑-म॒ष्टाक॑पालं॒ निर॑वपत् । अनु॑वित्त्यै च॒रुं । अनु॑मत्यै च॒रुं । ततो॒ वै तस्य॑ स॒त्या ऽनु॑वित्तिरभवत् । अनु॑ स्व॒र्गं ॅलो॒क-म॑विन्दत् । स॒त्या ह॒ वा अ॒स्यानु॑वित्ति-र्भवति । अनु॑ स्व॒र्गं ॅलो॒कं ॅवि॑न्दति । य ए॒तेन॑ ह॒विषा॒ यज॑ते । य उ॑चैनदे॒वं ॅवेद॑ । सोऽत्र॑ जुहोति । अ॒ग्नये॒ कामा॑य॒ स्वाहा-ऽनु॑वित्त्यै॒ स्वाहा᳚ । अनु॑मत्यै॒ स्वाहा᳚ प्र॒जाप॑तये॒ स्वाहा᳚ । स्व॒र्गाय॑ लो॒काय॒ स्वाहा॒ऽग्नये᳚ स्विष्ट॒कृते॒ स्वाहेति॑ । \textbf{ 9} \newline
                  \newline
                                \textbf{ TB 3.12.2.9} \newline
                  ता वा ए॒ताः स॒प्त स्व॒र्गस्य॑ लो॒कस्य॒ द्वारः॑ । दि॒वः श्ये॑न॒यो ऽनु॑वित्तयो॒ नाम॑ । आशा᳚ प्रथ॒माꣳ र॑क्षति । कामो᳚ द्वि॒तीयां᳚ । ब्रह्म॑ तृ॒तीयां᳚ । य॒ज्ञ्श्च॑तु॒र्थीं । आपः॑ पञ्च॒मीं । अ॒ग्नि र्ब॑लि॒मान्-थ्ष॒ष्ठीं । अनु॑वित्तिः सप्त॒मीं । अनु॑ ह॒ वै स्व॒र्गं ॅलो॒कं ॅवि॑न्दति । का॒म॒चारो᳚ऽस्य स्व॒र्गे लो॒के भ॑वति । य ए॒ताभि॒-रिष्टि॑भि॒-र्यज॑ते । य उ॑चैना ए॒वं ॅवेद॑ । तास्व॑न्वि॒ष्टि । प॒ष्ठौ॒ही॒ व॒रां द॑द्यात् कꣳ॒॒ सञ्च॑ । स्त्रियै॑ चाभा॒रꣳ समृ॑द्ध्यै । \textbf{ 10} \newline
                  \newline
                                     \textbf{(A2)} \newline \newline
                \textbf{ 3.12.3    अनुवाकं   3 -उपाघा नामेष्टयः} \newline
                                \textbf{ TB 3.12.3.1} \newline
                  तप॑सा दे॒वा दे॒वता॒-मग्र॑ आयन्न् । तप॒सर्.ष॑यः॒ स्व॑रन्व॑विन्दन्न् ।तप॑सा स॒पत्ना॒न् प्रणु॑दा॒मारा॑तीः । येने॒दं ॅविश्वं॒ परि॑भूतं॒ ॅयदस्ति॑ ॥प्र॒थ॒म॒जं दे॒वꣳ ह॒विषा॑ विधेम । स्व॒य॒भुं ब्रह्म॑ पर॒मं तपो॒ यत् ।स ए॒व पु॒त्रः स पि॒ता स मा॒ता । तपो॑ ह य॒क्षं प्र॑थ॒मꣳ संब॑भूव ॥श्र॒द्धया दे॑वो देव॒त्व-म॑श्नुते । श्र॒द्धा प्र॑ति॒ष्ठा लो॒कस्य॑ दे॒वी \textbf{ 11} \newline
                  \newline
                                \textbf{ TB 3.12.3.2} \newline
                  सा नो॑ जुषा॒णोप॑ य॒ज्ञ्-मागा᳚त् । काम॑वथ्सा॒ ऽमृतं॒ दुहा॑ना ॥श्र॒द्धा दे॒वी प्र॑थम॒जा ऋ॒तस्य॑ । विश्व॑स्य भ॒र्त्री जग॑तः प्रति॒ष्ठा ।ताꣳ श्र॒द्धाꣳ ह॒विषा॑ यजामहे । सा नो॑ लो॒क-म॒मृतं॑ दधातु ।ईशा॑ना दे॒वी भुव॑न॒स्याधि॑पत्नी ॥ आगा᳚थ्स॒त्यꣳ ह॒विरि॒दं जु॑षा॒णं ।यस्मा᳚द्दे॒वा ज॑ज्ञिरे॒ भुव॑नञ्च॒ विश्वे᳚ । तस्मै॑ विधेम ह॒विषा॑ घृ॒तेन॑ \textbf{ 12} \newline
                  \newline
                                \textbf{ TB 3.12.3.3} \newline
                  यथा॑ दे॒वैः स॑ध॒मादं॑ मदेम ॥ यस्य॑ प्रति॒ष्ठो-र्व॑न्तरि॑क्षं ।यस्मा᳚द्दे॒वा ज॑ज्ञिरे॒ भुव॑नञ्च॒ सर्वे᳚ । तथ्स॒त्य-मर्च॒दुप॑य॒ज्ञ्ं न॒ आगा᳚त् । ब्रह्माहु॑ती॒-रुप॒मोद॑मानं ॥ मन॑सो॒ वशे॒ सर्व॑मि॒दं ब॑भूव । नान्यस्य॒ मनो॒ वश॒मन्वि॑याय । भी॒ष्मो हि दे॒वः सह॑सः॒ सही॑यान् । स नो॑ जुषा॒ण उप॑ य॒ज्ञ्मागा᳚त् ॥ आकू॑तीना॒-मधि॑पतिं॒ चेत॑सां च \textbf{ 13} \newline
                  \newline
                                \textbf{ TB 3.12.3.4} \newline
                  स॒ङ्क॒ल्पजू॑तिं दे॒वं ॅवि॑प॒श्चिं । मनो॒ राजा॑नमि॒ह व॒र्द्धय॑न्तः ।उ॒प॒ह॒वे᳚ ऽस्य सुम॒तौ स्या॑म ॥ चर॑णं प॒वित्रं॒ ॅवित॑तं पुरा॒णं ।येन॑ पू॒तस्तर॑ति दुष्कृ॒तानि॑ । तेन॑ प॒वित्रे॑ण शु॒द्धेन॑ पू॒ताः ।अति॑ पा॒प्मान॒-मरा॑तिं तरेम ॥ लो॒कस्य॒ द्वार॑-मर्चि॒मत् प॒वित्रं᳚ ।ज्योति॑ष्म॒द् भ्राज॑मानं॒ मह॑स्वत् । अ॒मृत॑स्य॒ धारा॑ बहु॒धा दोह॑मानं ( ) । चर॑णं नो लो॒के सुधि॑तां दधातु ॥ “अ॒ग्नि र्मू॒द्र्धा{41}” “भुवः॑{42}” ॥“अनु॑ नो॒ऽद्यानु॑मति॒{43}” “रन्विद॑नुमते॒त्वं{44}” ॥ “ह॒व्य॒वाहꣳ॒॒{45}” “स्वि॑ष्टं{46}” । \textbf{ 14} \newline
                  \newline
                                    (दे॒वी - घृ॒तेन॒ - चेत॑सां च॒ - दोह॑मानं च॒त्वारि॑ च) \textbf{(A3)} \newline \newline
                \textbf{ 3.12.4    अनुवाकं   4 -तासां ब्राह्मणम्} \newline
                                \textbf{ TB 3.12.4.1} \newline
                  दे॒वेभ्यो॒ वै स्व॒र्गो लो॒कस्ति॒रो॑ ऽभवत् । ते प्र॒जाप॑ति-मब्रुवन्न् । प्रजा॑पते स्व॒र्गो वै नो॑ लो॒कस्ति॒रो॑ऽभूत् । तमन्वि॒च्छेति॑ ।तं ॅय॑ज्ञ्क्र॒तुभि॒-रन्वै᳚च्छत् । तं ॅय॑ज्ञ्क्र॒तुभि॒ र्नान्व॑विन्दत् ।तमिष्टि॑भि॒-रन्वै᳚च्छत् । तमिष्टि॑भि॒-रन्व॑विन्दत् । तदिष्टी॑ना-मिष्टि॒त्वं । एष्ट॑यो ह॒ वै नाम॑ । ता इष्ट॑य॒ इत्या च॑क्षते प॒रोक्षे॑ण । प॒रोक्ष॑प्रिया इव॒ हि दे॒वाः । \textbf{ 15} \newline
                  \newline
                                \textbf{ TB 3.12.4.2} \newline
                  तं तपो᳚ऽ ब्रवीत् । प्रजा॑पते॒ तप॑सा॒ वै श्रा᳚म्यसि । अ॒हमु॒ वै तपो᳚ऽ स्मि । मां नु य॑जस्व । अथ॑ ते स॒त्यं तपो॑ भविष्यति । अनु॑ स्व॒र्गं ॅलो॒कं ॅवे॒थ्स्यसीति॑ ।स ए॒तमा᳚ग्ने॒य-म॒ष्टाक॑पालं॒ निर॑वपत् । तप॑से च॒रुं । अनु॑मत्यै च॒रुं । ततो॒ वै तस्य॑ स॒त्यं तपो॑ ऽभवत् । अनु॑ स्व॒र्गं ॅलो॒कम॑विन्दत् । स॒त्य ꣳ ह॒ वा अ॑स्य॒ तपो॑ भवति । अनु॑ स्व॒र्गं ॅलो॒कं ॅवि॑न्दति । य ए॒तेन॑ ह॒विषा॒ यज॑ते । य उ॑ चैनदे॒वं ॅवेद॑ । सोऽत्र॑ जुहोति । अ॒ग्नये॒ स्वाहा॒ तप॑से॒ स्वाहा᳚ । अनु॑मत्यै॒ स्वाहा᳚ प्र॒जाप॑तये॒ स्वाहा᳚ ।स्व॒र्गाय॑ लो॒काय॒ स्वाहा॒ऽग्नये᳚ स्विष्ट॒कृते॒ स्वाहेति॑ । \textbf{ 16} \newline
                  \newline
                                \textbf{ TB 3.12.4.3} \newline
                  तꣳ श्र॒द्धा-ऽब्र॑वीत् । प्रजा॑पते श्र॒द्धया॒ वै श्रा᳚म्यसि । अ॒हमु॒ वै श्र॒द्धाऽस्मि॑ । मां नु य॑जस्व । अथ॑ ते स॒त्या श्र॒द्धा भ॑विष्यति । अनु॑ स्व॒र्गं ॅलो॒कं ॅवे॒थ्स्यसीति॑ । स ए॒तमा᳚ऽग्ने॒य-म॒ष्टाक॑पालं॒ निर॑वपत् । श्र॒द्धायै॑ च॒रुं । अनु॑मत्यै च॒रुं । ततो॒ वै तस्य॑ स॒त्या श्र॒द्धाऽभ॑वत् । अनु॑ स्व॒र्गं ॅलो॒कम॑विन्दत् । स॒त्या ह॒ वा अ॑स्य श्र॒द्धा भ॑वति । अनु॑ स्व॒र्गं ॅलो॒कं ॅवि॑न्दति । य ए॒तेन॑ ह॒विषा॒ यज॑ते । य उ॑ चैनदे॒वं ॅवेद॑ । सोऽत्र॑ जुहोति । अ॒ग्नये॒ स्वाहा᳚ श्र॒द्धायै॒ स्वाहा᳚ ।अनु॑मत्यै॒ स्वाहा᳚ प्र॒जाप॑तये॒ स्वाहा᳚ ।स्व॒र्गाय॑ लो॒काय॒ स्वाहा॒ऽग्नये᳚ स्विष्ट॒कृते॒ स्वाहेति॑ । \textbf{ 17} \newline
                  \newline
                                \textbf{ TB 3.12.4.4} \newline
                  तꣳ स॒त्य-म॑ब्रवीत् । प्रजा॑पते स॒त्येन॒ वै श्रा᳚म्यसि । अ॒हमु॒ वै स॒त्यम॑स्मि । मां नु य॑जस्व । अथ॑ ते स॒त्यꣳ स॒त्यं भ॑विष्यति । अनु॑ स्व॒र्गं ॅलो॒कं ॅवे॒थ्स्यसीति॑ । स ए॒तमा᳚ऽग्ने॒य-म॒ष्टाक॑पालं॒ निर॑वपत् । स॒त्याय॑ च॒रुं । अनु॑मत्यै च॒रुं । ततो॒ वै तस्य॑ स॒त्यꣳ स॒त्यम॑भवत् । अनु॑ स्व॒र्गं ॅलो॒कम॑विन्दत् । स॒त्यꣳ ह॒ वा अ॑स्य स॒त्यं भ॑वति । अनु॑ स्व॒र्गं ॅलो॒कं ॅवि॑न्दति । य ए॒तेन॑ ह॒विषा॒ यज॑ते । य उ॑चैनदे॒वं ॅवेद॑ । सोऽत्र॑ जुहोति । अ॒ग्नये॒ स्वाहा॑ स॒त्याय॒ स्वाहा᳚ । अनु॑मत्यै॒ स्वाहा᳚ प्र॒जाप॑तये॒ स्वाहा᳚ ।स्व॒र्गाय॑ लो॒काय॒ स्वाहा॒ऽग्नये᳚ स्विष्ट॒कृते॒ स्वाहेति॑ । \textbf{ 18} \newline
                  \newline
                                \textbf{ TB 3.12.4.5} \newline
                  तं मनो᳚ऽब्रवीत् । प्रजा॑पते॒ मन॑सा॒ वै श्रा᳚म्यसि । अ॒हमु॒ वै मनो᳚ऽस्मि । मां नु य॑जस्व । अथ॑ ते स॒त्यं मनो॑ भविष्यति । अनु॑ स्व॒र्गं ॅलो॒कं ॅवे॒थ्स्यसीति॑ । स ए॒तमा᳚ऽग्ने॒य-म॒ष्टाक॑पालं॒ निर॑वपत् । मन॑से च॒रुं । अनु॑मत्यै च॒रुं । ततो॒ वै तस्य॑ स॒त्यं मनो॑ऽभवत् । अनु॑ स्व॒र्गं ॅलो॒कम॑विन्दत् । स॒त्यꣳ ह॒ वा अ॑स्य॒ मनो॑ भवति । अनु॑ स्व॒र्गं ॅलो॒कं ॅवि॑न्दति । य ए॒तेन॑ ह॒विषा॒ यज॑ते । य उ॑चैन दे॒वं ॅवेद॑ । सोऽत्र॑ जुहोति । अ॒ग्नये॒ स्वाहा॒ मन॑से॒ स्वाहा᳚ ।अनु॑मत्यै॒ स्वाहा᳚ प्र॒जाप॑तये॒ स्वाहा᳚ ।स्व॒र्गाय॑ लो॒काय॒ स्वाहा॒ऽग्नये᳚ स्विष्ट॒कृते॒ स्वाहेति॑ । \textbf{ 19} \newline
                  \newline
                                \textbf{ TB 3.12.4.6} \newline
                  तं चर॑ण-मब्रवीत् । प्रजा॑पते॒ चर॑णेन॒ वै श्रा᳚म्यसि । अ॒हमु॒ वै चर॑णमस्मि । मां नु य॑जस्व । अथ॑ ते स॒त्यं चर॑णं भविष्यति । अनु॑ स्व॒र्गं ॅलो॒कं ॅवे॒थ्स्यसीति॑ ।स ए॒तमा᳚ऽग्ने॒य-म॒ष्टाक॑पालं॒ निर॑वपत् । चर॑णाय च॒रुं । अनु॑मत्यै च॒रुं । ततो॒ वै तस्य॑ स॒त्यं चर॑णमभवत् । अनु॑ स्व॒र्गं ॅलो॒कम॑विन्दत् । स॒त्यꣳ ह॒ वा अ॑स्य॒ चर॑णं भवति । अनु॑स्व॒र्गं ॅलो॒कं ॅवि॑न्दति । य ए॒तेन॑ ह॒विषा॒ यज॑ते । य उ॑चैनदे॒वं ॅवेद॑ । सोऽत्र॑ जुहोति । अ॒ग्नये॒ स्वाहा॒ चर॑णाय॒ स्वाहा᳚ । अनु॑मत्यै॒ स्वाहा᳚ प्र॒जाप॑तये॒ स्वाहा᳚ ।स्व॒र्गाय॑ लो॒काय॒ स्वाहा॒ऽग्नये᳚ स्विष्ट॒कृते॒ स्वाहेति॑ । \textbf{ 20} \newline
                  \newline
                                \textbf{ TB 3.12.4.7} \newline
                  ता वा ए॒ताः पञ्च॑ स्व॒र्गस्य॑ लो॒कस्य॒ द्वारः॑ । अपा॑घा॒ अनु॑वित्तयो॒ नाम॑ । तपः॑ प्रथ॒माꣳ र॑क्षति । श्र॒द्धा द्वि॒तीयां᳚ । स॒त्यं तृ॒॒तीयां᳚ । मन॑श्चतु॒र्थीं । चर॑णं पञ्च॒मीं ।अनु॑ ह॒ वै स्व॒र्गं ॅलो॒कं ॅवि॑न्दति । का॒म॒चारो᳚ऽस्य स्व॒र्गे लो॒के भ॑वति । य ए॒ताभि॒-रिष्टि॑भि॒-र्यज॑ते । य उ॑चैना ए॒वं ॅवेद॑ ।तास्व॑न्वि॒ष्टि । प॒ष्ठौ॒ही॒ व॒रां द॑द्यात् कꣳ॒॒सञ्च॑ । स्त्रियै॑ चाभा॒रꣳ समृ॑द्ध्यै । \textbf{ 21} \newline
                  \newline
                                     \textbf{(A4)} \newline \newline
                \textbf{ 3.12.5    अनुवाकं   5 -चातुर्होत्रचयनम्} \newline
                                \textbf{ TB 3.12.5.1} \newline
                  ब्रह्म॒ वै चतु॑र्.होतारः । चतु॑र्.होतृ॒भ्योऽधि॑ य॒ज्ञो निर्मि॑तः । नैनꣳ॑ श॒प्तं । नाभिच॑रित॒-माग॑च्छति । य ए॒वं ॅवेद॑ ॥ यो ह॒ वै चतु॑र्.होतृणां चतुर्.होतृ॒त्वं ॅवेद॑ । अथो॒ पञ्च॑ होतृत्वं । सर्वा॑ हास्मै॒ दिशः॑ कल्पन्ते ।वा॒चस्पति॒र्॒. होता॒ दश॑होतॄणां । पृ॒थि॒वी होता॒ चतु॑र्.होतॄणां \textbf{ 22} \newline
                  \newline
                                \textbf{ TB 3.12.5.2} \newline
                  अ॒ग्निर्. होता॒ पञ्च॑होतॄणां । वाग्घोता॒ षड्ढो॑तॄणां ।म॒हाह॑वि॒र्॒. होता॑ स॒प्तहो॑तॄणां । ए॒तद्वै चतु॑र्.होतृणां चतुर्.होतृ॒त्वं ।अथो॒ पञ्च॑होतृत्वं । सर्वा॑ हास्मै॒ दिशः॑ कल्पन्ते । य ए॒वं ॅवेद॑ ॥ए॒षा वै स॑र्व वि॒द्या । ए॒तद् भे॑ष॒जं ।ए॒षा प॒ङ्क्तिः स्व॒र्गस्य॑ लो॒कस्यां᳚ ज॒साय॑निः स्रु॒तिः \textbf{ 23} \newline
                  \newline
                                \textbf{ TB 3.12.5.3} \newline
                  ए॒तान्. योऽद्ध्यै-त्यछ॑दिर्द॒र्॒.शे याव॑त्त॒रसं᳚ । स्व॑रेति । अ॒न॒प॒ब्र॒वः सर्व॒मायु॑रेति । वि॒न्दते᳚ प्र॒जां । रा॒यस्पोषं॑ गौप॒त्यं । ब्र॒ह्म॒व॒र्च॒सी भ॑वति ॥ ए॒तान्. योऽद्ध्यैति॑ । स्पृ॒णोत्या॒त्मानं᳚ । प्र॒जां पि॒तॄन् ॥ ए॒तान्. वा अ॑रु॒ण औ॑पवे॒शि र्वि॒दाञ्च॑कार \textbf{ 24} \newline
                  \newline
                                \textbf{ TB 3.12.5.4} \newline
                  ए॒तैर॑धिवा॒दमपा॑जयत् । अथो॒ विश्वं॑ पा॒प्मानं᳚ । स्व॑र्ययौ । ए॒तान्. योऽद्ध्यैति॑ । अ॒धि॒वा॒दं ज॑यति । अथो॒ विश्वं॑ पा॒प्मानं᳚ । स्व॑रेति ॥ ए॒तैर॒ग्निं चि॑न्वीत स्व॒र्गका॑मः ।ए॒तैरायु॑ष्कामः । प्र॒जा प॒शुका॑मो वा । \textbf{ 25} \newline
                  \newline
                                \textbf{ TB 3.12.5.5} \newline
                  पु॒रस्ता॒-द्दश॑होतार॒-मुद॑ञ्च॒-मुप॑दधाति यावत्प॒दं । हृद॑यं॒ ॅयजु॑षी॒ पत्न्यौ॑ च । द॒क्षि॒ण॒तः प्राञ्चं॒ चतु॑र्.होतारं ।प॒श्चादुद॑ञ्चं॒ पञ्च॑ होतारं । उ॒त्त॒र॒तः प्राञ्चꣳ॒॒ षड्ढो॑तारं ।उ॒परि॑ष्टा॒त् प्राञ्चꣳ॑ स॒प्तहो॑तारं । हृद॑यं॒ ॅयजूꣳ॑षि॒ पत्न्य॑श्च । य॒था॒व॒का॒शं ग्रहान्॑ । य॒था॒व॒का॒शं प्र॑तिग्र॒हान् ॅलो॑कं पृ॒णाश्च॑ ।सर्वा॑ हास्यै॒ता दे॒वताः᳚ प्री॒ता अ॒भीष्टा॑ भवन्ति \textbf{ 26} \newline
                  \newline
                                \textbf{ TB 3.12.5.6} \newline
                  सदे॑वम॒ग्निं चि॑नुते ॥ र॒थस॑मिंत-श्चेत॒व्यः॑ । वज्रो॒ वै रथः॑ ।वज्रे॑णै॒व पा॒प्मानं॒ भ्रातृ॑व्यꣳ स्तृणुते । प॒क्षः स॑मिंत-श्चेत॒व्यः॑ । ए॒तावा॒न॒. वै रथः॑ । याव॑त्प॒क्षः । र॒थस॑मिंतमे॒व चि॑नुते ॥ इ॒ममे॒व लो॒कं प॑शुब॒न्धेना॒भिज॑यति । अथो॑ अग्निष्टो॒मेन॑ \textbf{ 27} \newline
                  \newline
                                \textbf{ TB 3.12.5.7} \newline
                  अ॒न्तरि॑क्ष-मु॒क्थ्ये॑न । स्व॑रतिरा॒त्रेण॑ । सर्वा᳚न् ॅलो॒कान॑ही॒नेन॑ । अथो॑ स॒त्रेण॑ ॥ वरो॒ दक्षि॑णा । वरे॑णै॒व वरꣳ॑ स्पृणोति ।आ॒त्मा हि वरः॑ ॥ एक॑विꣳशति॒ र्दक्षि॑णा ददाति । ए॒क॒विꣳ॒॒शो वा इ॒तः स्व॒र्गो लो॒कः । प्रस्व॒र्गं ॅलो॒कमा᳚प्नोति \textbf{ 28} \newline
                  \newline
                                \textbf{ TB 3.12.5.8} \newline
                  अ॒सावा॑दि॒त्य ए॑कविꣳ॒॒शः । अ॒मुमे॒वा-दि॒त्यमा᳚प्नोति ॥ श॒तं द॑दाति । श॒तायुः॒ पुरु॑षः श॒तेन्द्रि॑यः । आयु॑ष्ये॒वेन्द्रि॒ये प्रति॑तिष्ठति ॥ स॒हस्रं॑ ददाति ।स॒हस्र॑ संमितः स्व॒र्गो लो॒कः । स्व॒र्गस्य॑ लो॒कस्या॒-भिजि॑त्यै ॥ अ॒न्वि॒ष्ट॒कं दक्षि॑णा ददाति । सर्वा॑णि॒ वयाꣳ॑सि \textbf{ 29} \newline
                  \newline
                                \textbf{ TB 3.12.5.9} \newline
                  सर्व॒स्याप्त्यै᳚ । सर्व॒स्या-व॑रुद्ध्यै ॥ यदि॒ न वि॒न्देत॑ । म॒न्थाने॑ताव॒तो द॑द्यादोद॒नान्. वा᳚ । अ॒श्नु॒ते तं कामं᳚ । यस्मै॒ कामा॑या॒-ग्निश्ची॒यते᳚ ॥ प॒ष्ठौ॒हीं त्व॒न्तर्व॑तीं दद्यात् ।सा हि सर्वा॑णि॒ वयाꣳ॑सि । सर्व॒स्याप्त्यै᳚ । सर्व॒स्या-व॑रुद्ध्यै । \textbf{ 30} \newline
                  \newline
                                \textbf{ TB 3.12.5.10} \newline
                  हिर॑ण्यं ददाति । हिर॑ण्य-ज्योतिरे॒व स्व॒र्गं ॅलो॒कमे॑ति ॥ वासो॑ ददाति । तेनायुः॒ प्रति॑रते ॥ वे॒दि॒तृ॒ती॒ये य॑जेत । त्रिष॑त्या॒ हि दे॒वाः । स स॑त्यम॒ग्निं चि॑नुते ॥ तदे॒तत् प॑शुब॒न्धे ब्राह्म॑णं ब्रूयात् । नेत॑रेषु य॒ज्ञेषु॑ ॥ यो ह॒ वै चतु॑र्.होतॄ-ननुसव॒नं त॑र्पयित॒व्यान्॑. वेद॑ \textbf{ 31} \newline
                  \newline
                                \textbf{ TB 3.12.5.11} \newline
                  तृप्य॑ति प्र॒जया॑ प॒शुभिः॑ । उपै॑नꣳ सोमपी॒थो न॑मति ।ए॒ते वै चतु॑र्.होतारो ऽनुसव॒नं त॑र्पयित॒व्याः᳚ । ये ब्रा᳚ह्म॒णा ब॑हु॒विदः॑ ।तेभ्यो॒ यद्दक्षि॑णा॒ न नये᳚त् । दुरि॑ष्टꣳ स्यात् । अ॒ग्निम॑स्य वृञ्जीरन्न् । तेभ्यो॑ यथा श्र॒द्धं द॑द्यात् । स्वि॑ष्टमे॒वैतत्क्रि॑यते । नास्या॒ग्निं ॅवृ॑ञ्जते । \textbf{ 32} \newline
                  \newline
                                \textbf{ TB 3.12.5.12} \newline
                  हि॒र॒ण्ये॒ष्ट॒को भ॑वति । याव॑दुत्त॒म-म॑ङ्गुलि-का॒ण्डं ॅय॑ज्ञ् प॒रुषा॒ संमि॑तं ।तेजो॒ हिर॑ण्यं ॥ यदि॒ हिर॑ण्यं॒ न वि॒न्देत् । शर्क॑रा अ॒क्ता उप॑दद्ध्यात् । तेजो॑ घृ॒तं । सते॑जसमे॒-वाग्निं चि॑नुते ॥ अ॒ग्निञ्चि॒त्वा सौ᳚त्राम॒ण्या य॑जेत मैत्रावरु॒ण्या वा᳚ । वी॒र्ये॑ण॒ वा ए॒ष व्यृ॑द्ध्यते । यो᳚ऽग्निञ्चि॑नु॒ते ( ) \textbf{ 33} \newline
                  \newline
                                \textbf{ TB 3.12.5.13} \newline
                  याव॑दे॒व वी॒र्यं᳚ । तद॑स्मिन् दधाति ॥ ब्रह्म॑णः॒ सायु॑ज्यꣳ सलो॒कता॑-माप्नोति । ए॒तासा॑मे॒व दे॒वता॑नाꣳ॒॒ सायु॑ज्यं । सा॒र्ष्टिताꣳ॑ समान लो॒कता॑-माप्नोति । य ए॒तम॒ग्निं चि॑नु॒ते । य उ॑चैनमे॒वं ॅवेद॑ ॥ ए॒तदे॒व सा॑वि॒त्रे ब्राह्म॑णं । अथो॑ नाचिके॒ते । \textbf{ 34} \newline
                  \newline
                                    (होता॒ चतु॑र्.होतॄणाꣳ - स्रु॒ति - श्च॑कार - वा - भवन्त्य - ग्निष्टो॒मेना᳚ - प्नोति॒ - वयाꣳ॑सि॒ - वयाꣳ॑सि॒ - सर्व॒स्याप्त्यै॒ सर्व॒स्याव॑रुद्ध्यै॒ - वेद॑ - वृञ्जते - चिनु॒ते - +नव॑ च) \textbf{(A5)} \newline \newline
                \textbf{ 3.12.6    अनुवाकं   6 -वैश्वसृजचयनम्} \newline
                                \textbf{ TB 3.12.6.1} \newline
                  यच्चा॒मृतं॒ ॅयच्च मर्त्यं᳚ । यच्च॒ प्राणि॑ति॒ यच्च॒ न । सर्वा॒स्ता इष्ट॑काः कृ॒त्वा । उप॑ काम॒दुघा॑ दधे । तेनर्.षि॑णा॒ तेन॒ ब्रह्म॑णा । तया॑ दे॒वत॑या-ऽङ्गिर॒स्वद्-ध्रु॒वा सी॑द ॥ सर्वाः॒ स्त्रियः॒ सर्वा᳚न् पुꣳ॒॒सः । सर्वं॒ न स्त्री॑-पुमञ्च॒ यत् ।सर्वा॒स्ताः ॥ याव॑न्तः पाꣳ॒॒सवो॒ भूमेः᳚ \textbf{ 35} \newline
                  \newline
                                \textbf{ TB 3.12.6.2} \newline
                  सङ्ख्या॑ता देवमा॒यया᳚ । सर्वा॒स्ताः ॥ याव॑न्त॒ ऊषाः᳚ पशू॒नां । पृ॒थि॒व्यां पुष्टि॑ र्हि॒ताः । सर्वा॒स्ताः ॥ याव॑तीः॒ सिक॑ताः॒ सर्वाः᳚ ।अ॒फ्स्व॑न्तश्च॒ याः श्रि॒ताः । सर्वा॒स्ताः ॥ याव॑तीः॒ शर्क॑रा॒ धृत्यै᳚ । अ॒स्यां पृ॑थि॒व्यामधि॑ \textbf{ 36} \newline
                  \newline
                                \textbf{ TB 3.12.6.3} \newline
                  सर्वा॒स्ताः ॥ याव॒न्तो-ऽश्मा॑नो॒ऽस्यां पृ॑थि॒व्यां । प्र॒ति॒ष्ठासु॒ प्रति॑ष्ठिताः । सर्वा॒स्ताः ॥ याव॑ती र्वी॒रुधः॒ सर्वाः᳚ ।विष्ठि॑ताः पृथि॒वीमनु॑ । सर्वा॒स्ताः ॥ याव॑ती॒रोष॑धीः॒ सर्वाः᳚ ।विष्ठि॑ताः पृथि॒वीमनु॑ । सर्वा॒स्ताः । \textbf{ 37} \newline
                  \newline
                                \textbf{ TB 3.12.6.4} \newline
                  याव॑न्तो॒ वन॒स्पत॑यः । अ॒स्यां पृ॑थि॒व्यामधि॑ । सर्वा॒स्ताः ॥याव॑न्तो ग्रा॒म्याः प॒शवः॒ सर्वे᳚ । आ॒र॒ण्याश्च॒ ये । सर्वा॒स्ताः ॥ ये द्वि॒पाद॒-श्चतु॑ष्पादः । अ॒पाद॑ उदरस॒र्पिणः॑ ।सर्वा॒स्ताः ॥ याव॒दाञ्ज॑-नमु॒च्यते᳚ \textbf{ 38} \newline
                  \newline
                                \textbf{ TB 3.12.6.5} \newline
                  दे॒व॒त्रा यच्च॑ मानु॒षं । सर्वा॒स्ताः ॥ याव॑त्कृ॒ष्णाय॑सꣳ॒॒ सर्वं᳚ ।दे॒व॒त्रा यच्च॑ मानु॒षं । सर्वा॒स्ताः ॥ याव॑ ल्लो॒हाय॑सꣳ॒॒ सर्वं᳚ ।दे॒व॒त्रा यच्च॑ मानु॒षं । सर्वा॒स्ताः ॥ सर्वꣳ॒॒ सीसꣳ॒॒ सर्वं॒ त्रपु॑ ।दे॒व॒त्रा यच्च॑ मानु॒षं \textbf{ 39} \newline
                  \newline
                                \textbf{ TB 3.12.6.6} \newline
                  सर्वा॒स्ताः ॥ सर्वꣳ॒॒ हिर॑ण्यꣳ रज॒तं । दे॒व॒त्रा यच्च॑ मानु॒षं ।सर्वा॒स्ताः ॥ सर्वꣳ॒॒ सुव॑र्णꣳ॒॒ हरि॑तं । दे॒व॒त्रा यच्च॑ मानु॒षं । सर्वा॒स्ता इष्ट॑काः कृ॒त्वा । उप॑काम॒दुघा॑ दधे । तेनर्.षि॑णा॒ तेन॒ ब्रह्म॑णा । तया॑ दे॒वत॑या-ऽङ्गिर॒स्वद्-ध्रु॒वा सी॑द ( ) । \textbf{ 40} \newline
                  \newline
                                                        \textbf{special korvai} \newline
              (यच्च॒ स्त्रियः॑ पाꣳ॒॒सव॒ ऊषाः॒ सिक॑ताः॒ शर्क॑रा॒ अश्मा॑नो वी॒रुध॒ ओष॑धी॒र् वन॒स्पत॑यो ग्रा॒म्या ये द्वि॒पादो॒ याव॒दाञ्ज॑नं॒ याव॑त् कृ॒ष्णाय॑संॅलो॒हाय॑सꣳ॒॒ सीसꣳ॒॒ हिर॑ण्यꣳ॒॒ सुव॑र्णꣳ॒॒ हरि॑तम॒ष्टाद॑श) \newline
                                (भूमे॒ - रधि॒ - विष्ठि॑ताः पृथि॒वीमनु॒ सर्वा॒स्ता - उ॒च्यत॑ - मानु॒षꣳ - सी॑द ( ) ) \textbf{(A6)} \newline \newline
                \textbf{ 3.12.7    अनुवाकं   7 -वैश्वसृजचयनम्} \newline
                                \textbf{ TB 3.12.7.1} \newline
                  सर्वा॒ दिशो॑ दि॒क्षु । यच्चा॒न्त र्भू॒तं प्रति॑ष्ठितं । सर्वा॒स्ता इष्ट॑काः कृ॒त्वा । उप॑ काम॒दुघा॑ दधे ।तेनर्.षि॑णा॒ तेन॒ ब्रह्म॑णा । तया॑ दे॒वत॑या-ऽङ्गिर॒स्वद्-ध्रु॒वा सी॑द ॥अ॒न्तरि॑क्षञ्च॒ केव॑लं । यच्चा॒स्मि-न्न॑न्त॒राहि॑तं ।सर्वा॒स्ताः ॥ आ॒न्त॒रि॒क्ष्य॑श्च॒ याः प्र॒जाः \textbf{ 41} \newline
                  \newline
                                \textbf{ TB 3.12.7.2} \newline
                  ग॒न्ध॒र्वा॒॒-फ्स॒रस॑श्च॒ ये । सर्वा॒स्ताः ॥ सर्वा॑-नुदा॒रान्-थ्स॒लिलान्॑ ।अ॒न्तरि॑क्षे॒ प्रति॑ष्ठितान् । सर्वा॒स्ताः ॥ सर्वा॑-नुदा॒रान्-थ्स॑लि॒लान् ।स्था॒व॒राः प्रो॒ष्या᳚श्च॒ ये । सर्वा॒स्ताः ॥ सर्वां॒ धुनिꣳ॒॒ सर्वा᳚न्-ध्वꣳ॒॒सान् । हि॒मो यच्च॑ शी॒यते᳚ \textbf{ 42} \newline
                  \newline
                                \textbf{ TB 3.12.7.3} \newline
                  सर्वा॒स्ताः ॥ सर्वा॒न् मरी॑ची॒न॒. वित॑तान् । नी॒हा॒रो यच्च॑ शी॒यते᳚ । सर्वा॒स्ताः ॥ सर्वा॑ वि॒द्युतः॒ सर्वा᳚न्थ् स्तनयि॒त्नून् । ह्रा॒दुनी॒ र्यच्च॑ शी॒यते᳚ । सर्वा॒स्ताः ॥ सर्वाः॒ स्रव॑न्तीः स॒रितः॑ । सर्व॑मफ्सु च॒रञ्च॒ यत् । सर्वा॒स्ताः । \textbf{ 43} \newline
                  \newline
                                \textbf{ TB 3.12.7.4} \newline
                  याश्च॒ कूप्या॒ याश्च॑ ना॒द्याः᳚ समु॒द्रियाः᳚ । याश्च॑ वैश॒न्तीरु॒त प्रा॑स॒चीर्याः । सर्वा॒स्ताः ॥ ये चो॒त्तिष्ठ॑न्ति जी॒मूताः᳚ । याश्च॒ वर्.ष॑न्ति वृ॒ष्टयः॑ । सर्वा॒स्ताः ॥ तप॒स्तेज॑ आका॒शं । यच्चा॑का॒शे प्रति॑ष्ठितं । सर्वा॒स्ताः ॥ वा॒युं ॅवयाꣳ॑सि॒ सर्वा॑णि \textbf{ 44} \newline
                  \newline
                                \textbf{ TB 3.12.7.5} \newline
                  अ॒न्त॒रि॒क्ष॒च॒रञ्च॒ यत् । सर्वा॒स्ताः ॥ अ॒ग्निꣳ सूर्यं॑ च॒न्द्रं । मि॒त्रं ॅवरु॑णं॒ भगं᳚ । सर्वा॒स्ताः ॥ स॒त्यꣳ श्र॒द्धां तपो॒ दमं᳚ । नाम॑ रू॒पञ्च॑ भू॒तानां᳚ । सर्वा॒स्ता इष्ट॑काः कृ॒त्वा । उप॑ काम॒दुघा॑ दधे । तेनर्.षि॑णा॒ तेन॒ ब्रह्म॑णा ( ) । तया॑ दे॒वत॑या-ऽङ्गिर॒स्वद्-ध्रु॒वा सी॑द । \textbf{ 45} \newline
                  \newline
                                                        \textbf{special korvai} \newline
              (दिशो॒ऽन्तरि॑क्षमान्तरि॒क्ष्य॑ उदा॒रानु॑दा॒रान् धुनिं॒ मरी॑चीन्. वि॒द्युतः॒ स्रव॑न्ती॒र्याश्च॒ ये च॒ तपो॑ वा॒युम॒ग्निꣳ स॒त्यं पञ्च॑दश) \newline
                                (प्र॒जा-हि॒मो यच्च॑ शी॒यते॒-सर्वा॒स्ताः-सर्वा॑णि॒-ब्रह्म॒णैकं॑ च) \textbf{(A7)} \newline \newline
                \textbf{ 3.12.8    अनुवाकं   8 -वैश्वसृजचयनम्} \newline
                                \textbf{ TB 3.12.8.1} \newline
                  सर्वां॒ दिवꣳ॒॒ सर्वा᳚न् दे॒वान् दि॒वि । यच्चा॒न्त र्भू॒तं प्रति॑ष्ठितं ।सर्वा॒स्ता इष्ट॑काः कृ॒त्वा । उप॑ काम॒दुघा॑ दधे । तेनर्.षि॑णा॒ तेन॒ ब्रह्म॑णा । तया॑ दे॒वत॑या-ऽङ्गिर॒स्वद-ध्रु॒वा सी॑द ॥ याव॑ती॒ स्तार॑काः॒ सर्वाः᳚ । वित॑ता रोच॒ने दि॒वि । सर्वा॒स्ताः ॥ ऋचो॒ यजूꣳ॑षि॒ सामा॑नि \textbf{ 46} \newline
                  \newline
                                \textbf{ TB 3.12.8.2} \newline
                  अ॒थ॒र्वा॒ङ्गि॒रस॑श्च॒ ये । सर्वा॒स्ताः ॥ इ॒ति॒हा॒स॒पु॒रा॒णञ्च॑ ।स॒र्प॒दे॒व॒ज॒नाश्च॒ ये । सर्वा॒स्ताः ॥ ये च॑ लो॒का ये चा॑लो॒काः ।अ॒न्त र्भू॒तं प्रति॑ष्ठितं । सर्वा॒स्ताः ॥ यच्च॒ ब्रह्म॒ यच्चा᳚ब्र॒ह्म ।अ॒न्त र्ब्र॒ह्मन् प्रति॑ष्ठितं \textbf{ 47} \newline
                  \newline
                                \textbf{ TB 3.12.8.3} \newline
                  सर्वा॒स्ताः ॥ अ॒हो॒रा॒त्राणि॒ सर्वा॑णि । अ॒द्र्ध॒मा॒साꣳश्च॒ केव॑लान् ।सर्वा॒स्ताः ॥ सर्वा॑नृ॒तून्-थ्सर्वा᳚न् मा॒सान् । स॒म्ॅव॒थ्स॒रञ्च॒ केव॑लं ।सर्वा॒स्ताः ॥ सर्वं॒ भूतꣳ॒॒ सर्वं॒ भव्यं᳚ । यच्चा॒तोऽधि॑ भवि॒ष्यति॑ । सर्वा॒स्ता इष्ट॑काः कृ॒त्वा ( ) । उप॑ काम॒दुघा॑ दधे । तेनर्.षि॑णा॒ तेन॒ ब्रह्म॑णा । तया॑ दे॒वत॑या-ऽङ्गिर॒स्वद्-ध्रु॒वा सी॑द । \textbf{ 48} \newline
                  \newline
                                                        \textbf{special korvai} \newline
              (दिवं॒ तार॑का॒ ऋच॑ इतिहासपुरा॒णं च॒ ये च॒ यच्चा॑होरा॒त्राण्यृ॒तून् भूतं॒ नव॑) \newline
                                (सामा॑नि - ब्र॒ह्मन् प्रति॑ष्ठितं - कृ॒त्वा त्रीणि॑ च) \textbf{(A8)} \newline \newline
                \textbf{ 3.12.9    अनुवाकं   9 -ब्रह्मा सदस्यासीनो वौश्वसृजान् व्याचष्टे} \newline
                                \textbf{ TB 3.12.9.1} \newline
                  ऋ॒चां प्राची॑ मह॒ती दिगु॑च्यते । दक्षि॑णा-माहु॒ र्यजु॑षामपा॒रां ।अथ॑र्वणा॒-मग्ङि॑रसां प्र॒तीची᳚ । साम्ना॒-मुदी॑ची मह॒ती दिगु॑च्यते ॥ ऋ॒ग्भिः पू᳚र्वा॒ह्णे दि॒वि दे॒व ई॑यते । य॒जु॒र्वे॒दे ति॑ष्ठति॒ मद्ध्ये॒ अह्नः॑ ।सा॒म॒वे॒दे-ना᳚स्तम॒ये मही॑यते । वेदै॒-रशू᳚न्य-स्त्रि॒भिरे॑ति॒ सूर्यः॑ ॥ ऋ॒ग्भ्यो जा॒ताꣳ स॑र्व॒शो मूर्त्ति॑-माहुः । सर्वा॒ गति॑ र्याजु॒षी है॒व शश्व॑त् \textbf{ 49} \newline
                  \newline
                                \textbf{ TB 3.12.9.2} \newline
                  सर्वं॒ तेज॑ स्सामरू॒प्यꣳ ह॑ शश्वत् । सर्वꣳ॑ हे॒दं ब्रह्म॑णा है॒व सृ॒ष्टं ॥ ऋ॒ग्भ्यो जा॒तं ॅवैश्यं॒ ॅवर्ण॑-माहुः । य॒जु॒र्वे॒दं क्ष॑त्रि॒यस्या॑-हु॒र्योनिं᳚ । सा॒म॒वे॒दो ब्रा᳚ह्म॒णानां॒ प्रसू॑तिः । पूर्वे॒ पूर्वे᳚भ्यो॒ वच॑ ए॒त-दू॑चुः ॥ आ॒द॒र्॒.श-म॒ग्निं चि॑न्वा॒नाः । पूर्वे॑ विश्व॒ सृजो॒ऽमृताः᳚ । श॒तं व॑र्.ष सह॒स्राणि॑ । दी॒क्षि॒ताः स॒त्र-मा॑सत । \textbf{ 50} \newline
                  \newline
                                \textbf{ TB 3.12.9.3} \newline
                  तप॑ आसीद् गृ॒हप॑तिः । ब्रह्म॑ ब्र॒ह्माभ॑वथ् स्व॒यं । स॒त्यꣳ ह॒ हो तै॑षा॒मासी᳚त् । यद् वि॑श्व॒सृज॒ आस॑त ॥अ॒मृत॑-मेभ्य॒ उद॑गायत् । स॒हस्रं॑ परिवथ्स॒रान् । भू॒तꣳ ह॑ प्रस्तो॒ तैषा॒मासी᳚त् । भ॒वि॒ष्यत् प्रति॑ चाहरत् ॥प्रा॒णो अ॑द्ध्व॒र्यु-र॑भवत् । इ॒दꣳ सर्वꣳ॒॒ सिषा॑सतां \textbf{ 51} \newline
                  \newline
                                \textbf{ TB 3.12.9.4} \newline
                  अ॒पा॒नो वि॒द्वा-ना॒वृतः॑ । प्रति॒ प्राति॑ष्ठ-दद्ध्व॒रे ॥ आ॒र्त॒वा उ॑पगा॒तारः॑ । स॒द॒स्या॑ ऋ॒तवो॑-ऽभवन्न् । अ॒र्द्ध॒मा॒साश्च॒ मासा᳚श्च । च॒म॒सा॒-द्ध्व॒र्य॒वो-ऽभ॑वन्न् ॥ अशꣳ॑ स॒द् ब्रह्म॑ण॒-स्तेजः॑ । अ॒च्छा॒वा॒को ऽभ॑व॒द् यशः॑ । ऋ॒त-मे॑षां प्रशा॒स्ता-ऽऽसी᳚त् । यद् वि॑श्व॒ सृज॒ आस॑त । \textbf{ 52} \newline
                  \newline
                                \textbf{ TB 3.12.9.5} \newline
                  ऊर्ग्-राजा॑न॒-मुद॑वहत् । ध्रु॒व॒ गो॒पः सहो॑-ऽभवत् । ओजो॒-ऽभ्य॑ष्टौ॒द् ग्राव्.ण्णः॑ । यद् वि॑श्व॒ सृज॒ आस॑त ॥अप॑चितिः पो॒त्रीया॑-मयजत् । ने॒ष्ट्रीया॑-मयज॒त् त्विषिः॑ । आग्नी᳚द्ध्राद् वि॒दुषी॑ स॒त्यं । श्र॒द्धा है॒वा य॑जथ् स्व॒यं ॥ इरा॒ पत्नी॑ विश्व॒ सृजां᳚ । आकू॑ति-रपिनड्ढ॒विः \textbf{ 53} \newline
                  \newline
                                \textbf{ TB 3.12.9.6} \newline
                  इ॒द्ध्मꣳ ह॒ क्षुच् चै᳚भ्य उ॒ग्रे । तृ॒ष्णा चा व॑हतामु॒भे ॥ वागे॑षाꣳ सुब्रह्म॒ण्या-ऽऽसी᳚त् । छ॒न्दो॒ यो॒गान्. वि॑जान॒ती ।क॒ल्प॒त॒न्त्राणि॑ तन्वा॒ना-ऽहः॑ । सꣳ॒॒स्थाश्च॑ सर्व॒शः ॥ अ॒हो॒रा॒त्रे प॑शुपा॒ल्यौ । मु॒हू॒र्त्ताः प्रे॒ष्या॑ अभवन्न् । मृ॒त्युस् तद॑-भव-द्धा॒ता । श॒मि॒तो ग्रो वि॒शां पतिः॑ । \textbf{ 54} \newline
                  \newline
                                \textbf{ TB 3.12.9.7} \newline
                  वि॒श्व॒सृजः॑ प्रथ॒माः स॒त्र-मा॑सत । स॒हस्र॑ समं॒ प्रसु॑तेन॒ यन्तः॑ ।ततो॑ ह जज्ञे॒ भुव॑नस्य गो॒पाः । हि॒र॒ण्मयः॑ श॒कुनि॒ र्ब्रह्म॒ नाम॑ ॥ येन॒ सूर्य॒-स्तप॑ति॒ तेज॑से॒-द्धः । पि॒ता पु॒त्रेण॑ पितृ॒मान्. योनि॑योनौ । नावे॑दविन् मनुते॒ तं बृ॒हन्तं᳚ । स॒र्वा॒नु॒भु-मा॒त्मानꣳ॑ संपरा॒ये ॥ ए॒ष नि॒त्यो म॑हि॒मा ब्रा᳚ह्म॒णस्य॑ । न कर्म॑णा वर्द्धते॒ नो कनी॑यान् \textbf{ 55} \newline
                  \newline
                                \textbf{ TB 3.12.9.8} \newline
                  तस्यै॒वात्मा प॑द॒वित्तं ॅवि॑दित्वा । न कर्म॑णा लिप्यते॒ पाप॑केन ॥पञ्च॑पञ्चा॒शत॑-स्त्रि॒वृतः॑ सम्ॅवथ्स॒राः । पञ्च॑पञ्चा॒शतः॑ पञ्चद॒शाः । पञ्च॑पञ्चा॒शतः॑ सप्तद॒शाः । पञ्च॑पञ्चा॒शत॑ एकविꣳ॒॒शाः । वि॒श्व॒सृजाꣳ॑ स॒हस्र॑ सम्ॅवथ्सरं ॥ ए॒तेन॒ वै वि॑श्व॒सृज॑ इ॒दं ॅविश्व॑-मसृजन्त । यद् विश्व॒-मसृ॑जन्त । तस्मा᳚द् विश्व॒सृजः॑ ( ) ॥ विश्व॑-मेना॒-ननु॒ प्रजा॑यते । ब्रह्म॑णः॒ सायु॑ज्यꣳ सलो॒कतां᳚ ॅयन्ति । ए॒तासा॑मे॒व दे॒वता॑नाꣳ॒॒ सायु॑ज्यं । सा॒र्ष्टिताꣳ॑ समान लो॒कतां᳚ ॅयन्ति । य ए॒त-दु॑प॒यन्ति॑ । ये चै॑न॒त् प्राहुः॑ । येभ्य॑-श्चैन॒त् प्राहुः॑ ॥ ॐ । \textbf{ 56} \newline
                  \newline
                                    (शश्व॑ - दासत॒ - सिषा॑सता॒ - मास॑त - ह॒विष् - पतिः॒ - कनी॑या॒न् - तस्मा᳚द् विश्व॒सृजो॒ऽष्टौ च॑) \textbf{(A9)} \newline \newline
                \textbf{PrapAtaka Korvai with starting  words of 1 to 9 anuvAkams :-} \newline
        (तुभ्यं॑ - दे॒वेभ्य॒ - स्तप॑सा - दे॒वेभ्यो॒ - ब्रह्म॒ वै चतु॑र्.होतारो॒ - यच्चा॒मृतꣳ॒॒ - सर्वा॒ दिशो॑ दि॒क्षु - सर्वां॒ दिव॑ - मृ॒चां प्राची॒ नव॑) \newline

        \textbf{korvai with starting words of 1, 11, 21 series of daSinis :-} \newline
        (तुभ्यं॒ - तप॑सा॒ - ता वा ए॒ताः पञ्च॒ - हिर॑ण्यं ददाति॒ - सर्वा॒ दिश॒ - स्तप॑ आसीद्गृ॒हप॑तिः॒ षट्प॑ञ्चा॒शत्) \newline

        \textbf{first and last  word 3rd prapATakam of kAThakam:-} \newline
        (तुभ्य॒-म्ॐ) \newline 

       

        ॥ हरिः॑ ॐ ॥
॥ तैत्तिरीय यजुर्ब्राह्मणे काठके तृतीय प्रश्नः समाप्तः ॥

॥ इति तैत्तिरीय यजुर्ब्राह्मणे काठकं समाप्तं ॥
==================


Appendix (of Expansions)
ट्.भ्.3.12.1.1 - तुभ्य॒न्ता अ॑ङ्गिरस्तमा॒{21} श्याम॒ तङ्काम॑मग्ने {22} 
तुभ्यं॒ ता अ॑ङ्गिरस्तम॒ विश्वाः᳚ सुक्षि॒तयः॒ पृथ॑क् । 
अग्ने॒ कामा॑य येमिरे ॥ {21}

अ॒श्याम॒ तं काम॑मग्ने॒ तवो॒ त्य॑श्याम॑ र॒यिꣳ र॑यिवः सु॒वीरं᳚ । 
अ॒श्याम॒ वाज॑म॒भि वा॒जय॑न्तो॒ ऽश्याम॑ द्यु॒म्नम॑जरा॒जरं॑ ते ॥
(BOth {21} and {22} appearing in TS 1.3.14.3) 

ट्.भ्.3.12.1.1 - आशा॑नां त्वा॒{23} विश्वा॒ आशाः᳚{24} 
आशा॑नां त्वा ऽऽशापा॒लेभ्यः॑ । च॒तुर्भ्यो॑ अ॒मृते᳚भ्यः । 
इ॒दं भू॒तस्याध्य॑क्षेभ्यः । वि॒धेम॑ ह॒विषा॑ व॒यम् ॥ {23}

विश्वा॒ आशा॒ मधु॑ना॒ सꣳसृ॑जामि । अ॒न॒मी॒वा आप॒ ओष॑धयो भवन्तु । 
अ॒यं ॅयज॑मानो॒ मृधो॒ व्य॑स्यताम् अगृ॑भीताः प॒शवः॑ सन्तु॒ सर्वे᳚ ॥ {24}
(BOth {23} and {24} appEaring in T.B.2.5.3.3)

ट्.भ्.3.12.1.1 - अनु॑ नो॒ऽद्यानु॑मति॒{25} रन्विद॑नुमते॒ त्वं{26} 
अनु॑ नो॒ऽद्याऽनु॑मतिर्य॒ज्ञ्ं दे॒वेषु॑ मन्यतां । 
अ॒ग्निश्च॑ हव्य॒वाह॑नो॒ भव॑तां दा॒शुषे॒ मयः॑ ॥ {25}

अन्विद॑नुमते॒ त्वं मन्या॑सै॒ शञ्च॑नः कृधि । 
क्रत्वे॒ दक्षा॑य नो हिनु॒ प्रण॒ आयूꣳ॑षि तारिषः ॥ {26}
(BOth {25} and {26} appEaring in T.S.3.11.3.3)

ट्.भ्.3.12.1.1 - कामो॑ भू॒तस्य॒{27} काम॒स्तदग्रे᳚{28} 
कामो॑ भू॒तस्य॒ भव्य॑स्य । स॒म्राडेको॒ विरा॑जति । स इ॒दं प्रति॑ पप्रथे । 
ऋ॒तूनुथ् सृ॑जते व॒शी ॥ {27}

काम॒स्तदग्रे॒ सम॑वर्त॒ताधि॑ । मन॑सो॒ रेतः॑ प्रथ॒मं ॅयदासी᳚त् । 
स॒तो बन्धु॒मस॑ति॒ निर॑विन्दन्न् । हृ॒दि प्र॒तीष्या॑ क॒वयो॑ मनी॒षा ॥ {28}
(BOth {27} and {28} appEaring in T.B.2.4.1.9 and T.B.2.4.1.10

ट्.भ्.3.12.1.1 - ब्रह्म॑ जज्ञा॒नं{29} पि॒ता वि॒राजां᳚{30} 
ब्रह्म॑ जज्ञा॒नं प्र॑थ॒मं पु॒रस्ता॒द्वि सी॑म॒तः सु॒रुचो॑ वे॒न आ॑वः । 
स बु॒ध्निया॑ उप॒मा अ॑स्य वि॒ष्ठाः स॒तश्च॒ योनि॒मस॑तश्च॒ विवः॑ ॥ {29}
({29} appEaring in T.S.4.2.8.2)

पि॒ता वि॒राजा॑मृष॒भो र॑यी॒णां । अ॒न्तरि॑क्षं ॅवि॒श्वरू॑प॒ आवि॑वेश । 
तम॒र्कै-र॒भ्य॑र्चन्ति व॒थ्सं । ब्रह्म॒ सन्तं॒ ब्रह्म॑णा व॒र्द्धय॑न्तः ॥ {30}
({30} appEaring in T.B.2.8.8.9)

ट्.भ्.3.12.1.1 - य॒ज्ञो रा॒यो॑{31} ऽयं ॅय॒ज्ञ्ः{32} 
य॒ज्ञो रा॒यो य॒ज्ञ् ई॑शे॒ वसू॑नाम् । य॒ज्ञ्ः स॒स्याना॑मु॒त सु॑क्षिती॒नाम् । 
य॒ज्ञ् इ॒ष्टः पू॒र्वचि॑त्तिं दधातु । य॒ज्ञो ब्र॑ह्म॒ण्वाꣳ अप्ये॑तु दे॒वान् ॥ {31}

अ॒यं ॅय॒ज्ञो व॑र्धतां॒ गोभि॒रश्वैः᳚ । इ॒यं ॅवेदिः॑ स्वप॒त्या सु॒वीरा᳚ । 
इ॒दं ब॒र्॒.हिरति॑ ब॒र्॒.हीꣳष्य॒न्या । इ॒मं ॅय॒ज्ञ्ं ॅविश्वे॑ अवन्तु दे॒वाः ॥ {32}
(BOth {31} and {32} appEaring in T.B.2.5.5.1)

ट्.भ्.3.12.1.1 - आपो॑ भ॒द्रा{33} आदित्प॑श्यामि{34} 
आपो॑ भ॒द्रा घृ॒तमिदाप॑ आसुर॒ग्नी-षोमौ॑ बिभ्र॒त्याप॒ इत् ताः । 
ती॒व्रो रसो॑ मधु॒पृचा॑ मरंग॒म आ मा᳚ प्रा॒णेन॑ स॒ह वर्च॑सा गन्न् ॥ {33}
आदित् प॑श्याम्यु॒त वा॑ शृणो॒म्या मा॒ घोषो॑ गच्छति॒ वाङ्न॑ आसां । 
मन्ये॑ भेजा॒नो अ॒मृत॑स्य॒ तर्.हि॒ हिर॑ण्यवर्णा॒ अतृ॑पं ॅय॒दा वः॑ ॥ {34} 
(BOth {33} and {34} appEaring in T.S.5.6.1.3 and 5.6.1.4)

ट्.भ्.3.12.1.1 - तुभ्यं॑ भरन्ति॒{35] यो दे॒ह्यः{36} 
तुभ्यं॑ भरन्ति क्षि॒तयो॑ यविष्ठ । ब॒लिम॑ग्ने॒ अन्ति॑ त॒ ओत दू॒रात् । 
आ भन्दि॑ष्ठस्य सुम॒तिं चि॑किद्धि । बृ॒हत्ते॑ अग्ने॒ महि॒ शर्म॑ भ॒द्रम् ॥ {35} 

यो दे॒ह्यो अन॑मयद्वध॒स्नैः । यो अर्य॑पत्नी-रु॒षस॑श्च॒कार॑ । 
स नि॒रुध्या॒ नहु॑षो य॒ह्वो अ॒ग्निः । विश॑श्चक्रे बलि॒हृतः॒ सहो॑भिः ॥ {36} 
(BOth {35} and {36} appEaring in T.B.2.4.7.9)

ट्.भ्.3.12.1.1 - पूर्वं॑ देवा॒ अप॑रेण{37} प्राणापा॒नौ{38} 
पूर्वं॑ देवा॒ अप॑रेणा-नु॒पश्य॒ञ्जन्म॑भिः । जन्मा॒न्यव॑रैः॒ परा॑णि । 
वेदा॑नि देवा अ॒यम॒स्मीति॒ माम् । अ॒हꣳ हि॒त्वा शरी॑रं ज॒रसः॑ 
प॒रस्ता᳚त् ॥ {37}

प्रा॒णा॒पा॒नौ चक्षुः॒ श्रोत्र᳚म् । वाचं॒ मन॑सि॒ संभृ॑ताम् । 
हि॒त्वा शरी॑रं ज॒रसः॑ प॒रस्ता᳚त् । आ भूतिं॒ भूतिं॑ ॅव॒यम॑श्नवामहै ॥ {38}
(BOth {37} and {38} appEaring in T.B.2.5.6.5)

ट्.भ्.3.12.1.1 - ह॒व्य॒वाह॒॒{39} स्वि॑ष्टं{40} 
ह॒व्य॒वाह॑-मभिमाति॒षाऽह᳚म् । र॒क्षो॒हणं॒ पृत॑नासु जि॒ष्णुम् । 
ज्योति॑ष्मन्तं॒ दीद्य॑तं॒ पुर॑न्धिम् । अ॒ग्निꣳ स्वि॑ष्ट॒कृत॒मा हु॑वेम ॥ {39}

स्वि॑ष्टमग्ने अ॒भि तत् पृ॑णाहि । विश्वा॑ देव॒ पृत॑ना अ॒भिष्य । 
उ॒रुं नः॒ पन्थां᳚ प्रदि॒शन्वि भा॑हि । ज्योति॑ष्मद्धेह्य॒जरं॑ न॒ आयुः॑ ॥ {40}
(BOth {39} and {40} appEaring in T.B.2.4.1.4)


ट्.भ्.3.12.3.4 - अ॒ग्नि र्मू॒र्धा{41} भुवः॑{42} 
अ॒ग्निर्मू॒र्धा दि॒वः क॒कुत् पतिः॑ पृथि॒व्या अ॒यं । 
अ॒पाꣳ रेताꣳ॑सि जिन्वति ॥ {41}

भुवो॑ य॒ज्ञ्स्य॒ रज॑सश्च ने॒ता यत्रा॑ नि॒युद्भिः॒ सच॑से शि॒वाभिः॑ । 
दि॒वि मू॒र्धानं॑ दधिषे सुव॒र्॒.षां जि॒ह्वाम॑ग्ने चकृषे हव्य॒वाहं᳚ ॥ {42}
(BOth {41} and {42} appEaring in T.S.4.4.4.1)

ट्.भ्.3.12.3.4 - अनु॑ नो॒ऽद्यानु॑मति॒{43} रन्विद॑नुमते॒त्वं{44} 
समॆ अस् {25} अन्द् {26} अबॊवॆ

ट्.भ्.3.12.3.4 - ह॒व्य॒वाह॒॒{45} स्वि॑ष्टं{46}
समॆ अस् {39} अन्द् {40} अबॊवॆ

आनुवकम् 6 
in the entire 6th Anuvakam " sarvAqstAH" appearing 16 times in short form which are highlighted by us in the  text for identification. TheExpansion is as follows and is common for the all the 16 appearances.
सर्वा॒स्ता इष्ट॑काः कृ॒त्वा । उ॑प काम॒दुघा॑ दधे । तेनर्.षि॑णा॒ तेन॒ ब्रह्म॑णा । 
तया॑ दे॒वत॑या-ऽङ्गिर॒स्वद्-ध्रु॒वा सी॑द ॥
(Appearing in T.B.3.12.6.1)

आनुवकम् 7 
in the entire 7th Anuvakam " sarvAqstAH" appearing 13 times in short form which are highlighted by us in the  text for identification. 
ठॆ ए꣡पन्सिऒन् इस् अस् fऒल्लॊव्स् अन्द् इस् चॊम्मॊन् fऒर् थॆ अल्ल् थॆ 13 अप्पॆअरन्चॆस्.
सर्वा॒स्ता इष्ट॑काः कृ॒त्वा । उ॑प काम॒दुघा॑ दधे । तेनर्.षि॑णा॒ तेन॒ ब्रह्म॑णा । तया॑ दे॒वत॑या-ऽङ्गिर॒स्वद्-ध्रु॒वा सी॑द ॥
(Appearing in T.B.3.12.7.1)

आनुवकम् 8 
in the entire 8th Anuvakam " sarvAqstAH" appearing 7 times in short form which are highlighted by us in the  text for identification. The Expansion is as follows and is common for the all the 7 appearances.
सर्वा॒स्ता इष्ट॑काः कृ॒त्वा । उ॑प काम॒दुघा॑ दधे । तेनर्.षि॑णा॒ तेन॒ ब्रह्म॑णा । तया॑ दे॒वत॑या-ऽङ्गिर॒स्वद्-ध्रु॒वा सी॑द ॥
(Appearing in T.B.3.12.8.1) \newline
        \pagebreak
        
        
        

\end{document}
