\documentclass[17pt]{extarticle}
\usepackage{babel}
\usepackage{fontspec}
\usepackage{polyglossia}
\usepackage{extsizes}



\setmainlanguage{sanskrit}
\setotherlanguages{english} %% or other languages
\setlength{\parindent}{0pt}
\pagestyle{myheadings}
\newfontfamily\devanagarifont[Script=Devanagari]{AdishilaVedic}


\newcommand{\VAR}[1]{}
\newcommand{\BLOCK}[1]{}




\begin{document}
\begin{titlepage}
    \begin{center}
 
\begin{sanskrit}
    { \Large
    ॐ नमः परमात्मने, श्री महागणपतये नमः
श्री गुरुभ्यो नमः, ह॒रिः॒ ॐ 
    }
    \\
    \vspace{2.5cm}
    \mbox{ \Huge
    2       कृष्ण यजुर्वेदीय तैत्तिरीय ब्राह्मणे द्वितीयाष्टकं    }
\end{sanskrit}
\end{center}

\end{titlepage}
\tableofcontents
\pagebreak
ॐ नमः परमात्मने, श्री महागणपतये नमः
श्री गुरुभ्यो नमः, ह॒रिः॒ ॐ \newline
2       कृष्ण यजुर्वेदीय तैत्तिरीय ब्राह्मणे द्वितीयाष्टकं  \newline
     \addcontentsline{toc}{section}{ 2.1      द्वितीयाष्टके प्रथमः प्रपाठकः - अग्निहोत्रब्राह्मणम्}
     \markright{ 2.1      द्वितीयाष्टके प्रथमः प्रपाठकः - अग्निहोत्रब्राह्मणम् \hfill https://www.vedavms.in \hfill}
     \section*{ 2.1      द्वितीयाष्टके प्रथमः प्रपाठकः - अग्निहोत्रब्राह्मणम् }
                \textbf{ 2.1.1      अनुवाकं   1 - अग्निहोत्रस्योपोद्धातः} \newline
                                \textbf{ TB 2.1.1.1} \newline
                  अङ्गि॑रसो॒ वै स॒त्रमा॑सत । तेषां॒ पृश्नि॑र्-घर्म॒धुगा॑सीत् । सर्जी॒षेणा॑जीवत् । ते᳚ऽब्रुवन्न् । कस्मै॒ नु स॒त्रमा᳚स्महे । ये᳚ऽस्या ओष॑धी॒र्न ज॒नया॑म॒ इति॑ । ते दि॒वो वृष्टि॑-मसृजन्त । याव॑न्तः स्तो॒का अ॒वाप॑द्यन्त । ताव॑ती॒रोष॑धयो ऽजायन्त ॥ ता जा॒ताः पि॒तरो॑ वि॒षेणा॑लिम्पन्न् \textbf{ 1} \newline
                  \newline
                                \textbf{ TB 2.1.1.2} \newline
                  तासां᳚ ज॒ग्ध्वा रुप्य॒न्त्यैत् । ते᳚ऽब्रुवन्न् । क इ॒दमि॒त्थ-म॑क॒रिति॑ । व॒यं भा॑ग॒धेय॑-मि॒च्छमा॑ना॒ इति॑ पि॒तरो᳚ऽब्रुवन्न् । किं ॅवो॑ भाग॒धेय॒मिति॑ । अ॒ग्नि॒हो॒त्र ए॒व नोऽप्य॒स्त्वित्य॑-ब्रुवन्न् । तेभ्य॑ ए॒तद्-भा॑ग॒धेयं॒ प्राय॑च्छन्न् । यद्धु॒त्वा नि॒मार्ष्टि॑ । ततो॒ वै त ओष॑धीर-स्वदयन्न् ॥ य ए॒वं ॅवेद॑ \textbf{ 2} \newline
                  \newline
                                \textbf{ TB 2.1.1.3} \newline
                  स्वद॑न्तेऽस्मा॒ ओष॑धयः । ते व॒थ्समु॒-पावा॑सृजन्न् । इ॒दं नो॑ ह॒व्यं प्रदा॑प॒येति॑ । सो᳚ऽब्रवी॒द्वरं॑ ॅवृणै । दश॑ मा॒ रात्री᳚र्जा॒तं न दो॑हन्न् । आ॒स॒ङ्ग॒वं मा॒त्रा स॒ह च॑रा॒णीति॑ । तस्मा᳚द्-व॒थ्सं जा॒तं दश॒ रात्री॒र्न दु॑हन्ति । आ॒स॒ङ्ग॒वं मा॒त्रा स॒ह चर॑ति । वारे॑वृतꣳ॒॒ह्य॑स्य । तस्मा᳚द्-व॒थ्सꣳ सꣳ॑सृष्टध॒यꣳ रु॒द्रो घातु॑कः ( ) । अति॒ हि स॒न्धां धय॑ति । \textbf{ 3} \newline
                  \newline
                                    (अ॒लि॒म्प॒न् - वेद॒ - घातु॑क॒ एकं॑ च) \textbf{(A1)} \newline \newline
                \textbf{ 2.1.2      अनुवाकं   2 - अग्निहोत्रनिरूपणम्} \newline
                                \textbf{ TB 2.1.2.1} \newline
                  प्र॒जाप॑ति-र॒ग्नि-म॑सृजत । तं प्र॒जा अन्व॑सृज्यन्त । तम॑भा॒ग उपा᳚स्त । सो᳚ऽस्य प्र॒जाभि॒-रपा᳚क्रामत् । तम॑व॒रुरु॑थ्-समा॒नोऽन्वै᳚त् । तम॑व॒रुधं॒ नाश॑क्नोत् । स तपो॑ ऽतप्यत । सो᳚ऽग्नि-रुपा॑रम॒ताता॑पि॒ वै स्य प्र॒जाप॑ति॒रिति॑ । स र॒राटा॒दुद॑मृष्ट \textbf{ 4} \newline
                  \newline
                                \textbf{ TB 2.1.2.2} \newline
                  तद्-घृ॒तम॑भवत् । तस्मा॒द्यस्य॑ दक्षिण॒तः केशा॒ उन्मृ॑ष्टाः । तां ज्ये᳚ष्ठल॒क्ष्मी प्रा॑जाप॒त्येत्या॑हुः । यद्-र॒राटा॑दु॒दमृ॑ष्ट । तस्मा᳚द्-र॒राटे॒ केशा॒ न स॑न्ति । तद॒ग्नौ प्रागृ॑ह्णात् । तद्व्य॑चिकि॒थ्सत् । जु॒हवा॒नी(3) मा हौ॒षा(3)मिति॑ । तद्-वि॑चिकि॒थ्सायै॒ जन्म॑ ॥ य ए॒वं ॅवि॒द्वान्. वि॑चि॒किथ्स॑ति \textbf{ 5} \newline
                  \newline
                                \textbf{ TB 2.1.2.3} \newline
                  वसी॑य ए॒व चे॑तयते ॥ तं ॅवाग॒भ्य॑वद-ज्जु॒हुधीति॑ । सो᳚ऽब्रवीत् । कस्त्वम॒सीति॑ । स्वैव ते॒ वागित्य॑ब्रवीत् । सो॑ऽजुहो॒थ्-स्वाहेति॑ । तथ्स्वा॑हाका॒रस्य॒ जन्म॑ ॥ य ए॒वꣳ स्वा॑हाका॒रस्य॒ जन्म॒ वेद॑ । क॒रोति॑ स्वाहाका॒रेण॑ वी॒र्य᳚म् ॥ यस्यै॒वं ॅवि॒दुषः॑ स्वाहाका॒रेण॒ जुह्व॑ति \textbf{ 6} \newline
                  \newline
                                \textbf{ TB 2.1.2.4} \newline
                  भोगा॑यै॒वास्य॑ हु॒तं भ॑वति ॥ तस्या॒ आहु॑त्यै॒ पुरु॑ष-मसृजत । द्वि॒तीय॑-मजुहोत् । सोऽश्व॑-मसृजत । तृ॒तीय॑-मजुहोत् । स गा॑मसृजत । च॒तु॒र्थ-म॑जुहोत् । सोऽवि॑मसृजत । प॒ञ्च॒म-म॑जुहोत् । सो॑ऽजाम॑सृजत \textbf{ 7} \newline
                  \newline
                                \textbf{ TB 2.1.2.5} \newline
                  सो᳚ऽग्नि-र॑बिभेत् । आहु॑तीभि॒र्वै मा᳚ऽऽप्नो॒तीति॑ । स प्र॒जाप॑तिं॒ पुनः॒ प्रावि॑शत् । तं प्र॒जाप॑ति-रब्रवीत् । जाय॒स्वेति॑ । सो᳚ऽब्रवीत् । किं भा॑ग॒धेय॑म॒भि ज॑निष्य॒ इति॑ । तुभ्य॑मे॒वेदꣳ हू॑याता॒ इत्य॑ब्रवीत् । स ए॒तद्-भा॑ग॒धेय॑म॒भ्य॑जायत । यद॑ग्निहो॒त्रम् \textbf{ 8} \newline
                  \newline
                                \textbf{ TB 2.1.2.6} \newline
                  तस्मा॑दग्निहो॒त्र-मु॑च्यते ॥ तद्धू॒यमा॑न-मादि॒त्यो᳚ऽब्रवीत् । मा हौ॑षीः । उ॒भयो॒र्वै-ना॑वे॒तदिति॑ । सो᳚ऽग्नि-र॑ब्रवीत् । क॒थं नौ॑ होष्य॒न्तीति॑ । सा॒यमे॒व तुभ्यं॑ जु॒हवन्न्॑ । प्रा॒तर्मह्य॒-मित्य॑ब्रवीत् । तस्मा॑द॒ग्नये॑ सा॒यꣳ हू॑यते । सूर्या॑य प्रा॒तः । \textbf{ 9} \newline
                  \newline
                                \textbf{ TB 2.1.2.7} \newline
                  आ॒ग्ने॒यी वै रात्रिः॑ । ऐ॒न्द्रमहः॑ । यदनु॑दिते॒ सूर्ये᳚ प्रा॒तर्जु॑हु॒यात् । उ॒भय॑मे॒वाग्ने॒यꣳ स्या᳚त् । उदि॑ते॒ सूर्ये᳚ प्रा॒तर्जु॑होति । तथा॒ऽग्नये॑ सा॒यꣳ हू॑यते । सूर्या॑य प्रा॒तः ॥ रात्रिं॒ ॅवा अनु॑ प्र॒जाः प्र जा॑यन्ते । अह्ना॒ प्रति॑ तिष्ठन्ति । यथ्सा॒यं जु॒होति॑ \textbf{ 10} \newline
                  \newline
                                \textbf{ TB 2.1.2.8} \newline
                  प्रैव तेन॑ जायते । उदि॑ते॒ सूर्ये᳚ प्रा॒तर्-जु॑होति । प्रत्ये॒व तेन॑ तिष्ठति ॥ प्र॒जाप॑तिर-कामयत॒ प्रजा॑ये॒येति॑ । स ए॒तद॑ग्निहो॒त्रं मि॑थु॒न-म॑पश्यत् । तदुदि॑ते॒ सूर्ये॑ऽजुहोत् । यजु॑षा॒ ऽन्यत् । तू॒ष्णीम॒न्यत् । ततो॒ वै स प्राजा॑यत । यस्यै॒वं ॅवि॒दुष॒ उदि॑ते॒ सूर्ये᳚ऽग्निहो॒त्रं जुह्व॑ति \textbf{ 11} \newline
                  \newline
                                \textbf{ TB 2.1.2.9} \newline
                  प्रैव जा॑यते । अथो॒ यथा॒ दिवा᳚ प्रजा॒नन्नेति॑ । ता॒दृगे॒व तत् ॥ अथो॒ खल्वा॑हुः । यस्य॒ वै द्वौ पुण्यौ॑ गृ॒हे वस॑तः । यस्तयो॑र॒न्यꣳ रा॒धय॑त्य॒न्यं न । उ॒भौ वाव स तावृ॑च्छ॒तीति॑ । अ॒ग्निं ॅवावादि॒त्यः सा॒यं प्रवि॑शति । तस्मा॑द॒ग्निर्दू॒रान्नक्तं॑ ददृशे ।उ॒भे हि तेज॑सी स॒पंद्ये॑ते \textbf{ 12} \newline
                  \newline
                                \textbf{ TB 2.1.2.10} \newline
                  उ॒द्यन्तं॒ ॅवावादि॒त्य-म॒ग्निरनु॑ स॒मारो॑हति । तस्मा᳚द्धू॒म ए॒वाग्नेर्दिवा॑ ददृशे । यद॒ग्नये॑ सा॒यं जु॑हु॒यात् । आ सूर्या॑य वृश्च्येत । यथ्सूर्या॑य प्रा॒तर्-जु॑हु॒यात् । आऽग्नये॑ वृश्च्येत । दे॒वता᳚भ्यः स॒मदं॑ दद्ध्यात् । अ॒ग्निर्ज्योति॒र्-ज्योतिः॒ सूर्यः॒ स्वाहेत्ये॒व सा॒यꣳ हो॑त॒व्य᳚म् । सूर्यो॒ ज्योति॒र् ज्योति॑र॒ग्निः स्वाहेति॑ प्रा॒तः । तथो॒भाभ्याꣳ॑ सा॒यꣳ हू॑यते \textbf{ 13} \newline
                  \newline
                                \textbf{ TB 2.1.2.11} \newline
                  उ॒भाभ्यां᳚ प्रा॒तः । न दे॒वता᳚भ्यः स॒मदं॑ दधाति ॥ अ॒ग्निर्ज्योति॒रित्या॑ह । अ॒ग्निर्वै रे॑तो॒धाः ॥ प्र॒जा ज्योति॒रित्या॑ह । प्र॒जा ए॒वास्मै॒ प्र ज॑नयति ॥ सूर्यो॒ ज्योति॒रित्या॑ह । प्र॒जास्वे॒व प्रजा॑तासु॒ रेतो॑ दधाति ॥ ज्योति॑र॒ग्निः स्वाहेत्या॑ह । प्र॒जा ए॒व प्रजा॑ता अ॒स्यां प्रति॑ष्ठापयति । \textbf{ 14} \newline
                  \newline
                                \textbf{ TB 2.1.2.12} \newline
                  तू॒ष्णीमुत्त॑रा॒-माहु॑तिं जुहोति । मि॒थु॒न॒त्वाय॒ प्रजा᳚त्यै ॥ यदुदि॑ते॒ सूर्ये᳚ प्रा॒तर्-जु॑हु॒यात् । यथाऽति॑थये॒ प्रद्रु॑ताय शू॒न्याया॑वस॒था-या॑हा॒र्यꣳ॑ हर॑न्ति । ता॒दृगे॒व तत् ।क्वाह॒ तत॒स्तद्-भव॒तीत्या॑हुः । यथ्स न वेद॑ । यस्मै॒ तद्धर॒न्तीति॑ । तस्मा॒द्-यदौ॑ष॒सं जु॒होति॑ । तदे॒व स॑प्रं॒ति ( ) । अथो॒ यथा॒ प्रार्थ॑मौष॒सं प॑रि॒ वेवे᳚ष्टि । ता॒दृगे॒व तत् । \textbf{ 15} \newline
                  \newline
                                    (अ॒मृ॒ष्ट॒ - वि॒चि॒किथ्स॑ति॒ - जुह्व॑ - त्य॒जाम॑सृज - ताग्निहो॒त्रꣳ - सूर्या॑य प्रा॒तर् - जु॒होति॒ - जुह्व॑ति - सं॒पद्ये॑ते - हूयते - स्थापयति - संप्र॒ति द्वे च॑) \textbf{(A2)} \newline \newline
                \textbf{ 2.1.3      अनुवाकं   3 - उपयोक्ष्यमाणस्य हविषः संस्कारः} \newline
                                \textbf{ TB 2.1.3.1} \newline
                  रु॒द्रो वा ए॒षः । यद॒ग्निः । पत्नी᳚ स्था॒ली । यन्मद्ध्ये॒ऽग्ने-र॑धि॒श्रये᳚त् । रु॒द्राय॒ पत्नी॒मपि॑ दद्ध्यात् । प्र॒मायु॑का स्यात् । उदी॒चोऽङ्गा॑रान्नि॒रूह्याधि॑ श्रयति । पत्नि॑यै गोपी॒थाय॑ ॥ व्य॑न्तान् करोति । तथा॒ पत्न्य-प्र॑मायुका भवति । \textbf{ 16} \newline
                  \newline
                                \textbf{ TB 2.1.3.2} \newline
                  घ॒र्मो वा ए॒षोऽशा᳚न्तः । अह॑रहः॒ प्रवृ॑ज्यते । यद॑ग्निहो॒त्रम् । प्रति॑षिञ्चेत् प॒शुका॑मस्य । शा॒न्तमि॑व॒ हि प॑श॒व्य᳚म् ॥ न प्रति॑षिञ्चेद्-ब्रह्मवर्च॒सका॑मस्य । समि॑द्धमिव॒ हि ब्र॑ह्मवर्च॒सम् ॥ अथो॒ खलु॑ । प्र॒ति॒षिच्य॑मे॒व । यत्-प्र॑तिषि॒ञ्चति॑ \textbf{ 17} \newline
                  \newline
                                \textbf{ TB 2.1.3.3} \newline
                  तत्-प॑श॒व्य᳚म् । यज्जु॒होति॑ । तद्ब्र॑ह्मवर्च॒सि । उ॒भय॑मे॒वाकः॑ ॥ प्रच्यु॑तं॒ ॅवा ए॒तद॒स्मा-ल्लो॒कात् । अग॑तं देवलो॒कम् । यच्छृ॒तꣳ ह॒विरन॑भिघारितम् । अ॒भिद्यो॑तयति । अ॒भ्ये॑वैन॑द्घारयति । अथो॑ देव॒त्रैवैन॑द्-गमयति । \textbf{ 18} \newline
                  \newline
                                \textbf{ TB 2.1.3.4} \newline
                  पर्य॑ग्नि करोति । रक्ष॑सा॒-मप॑हत्यै ॥ त्रिः पर्य॑ग्नि करोति । त्र्या॑वृ॒द्धि य॒ज्ञ्ः । अथो॑ मेद्ध्य॒त्वाय॑ ॥ यत्-प्रा॒चीन॑-मुद्वा॒सये᳚त् । यज॑मानꣳ शु॒चा ऽर्प॑येत् । यद्-द॑क्षि॒णा । पि॒तृ॒दे॒व॒त्यꣳ॑ स्यात् । यत्-प्र॒त्यक् \textbf{ 19} \newline
                  \newline
                                \textbf{ TB 2.1.3.5} \newline
                  पत्नीꣳ॑ शु॒चा ऽर्प॑येत् । उ॒दी॒चीन॒-मुद्वा॑सयति । ए॒षा वै दे॑वमनु॒ष्याणाꣳ॑ शा॒न्ता दिक् । तामे॒वैन॒दनूद्-वा॑सयति॒ शान्त्यै᳚ ॥ वर्त्म॑ करोति । य॒ज्ञ्स्य॒ संत॑त्यै ॥ निष्ट॑पति । उपै॒व तथ्स्तृ॑णाति ॥ च॒तुरुन्न॑यति । चतु॑ष्पादः प॒शवः॑ \textbf{ 20} \newline
                  \newline
                                \textbf{ TB 2.1.3.6} \newline
                  प॒शूने॒वा-व॑रुन्धे ॥ सर्वा᳚न्-पू॒र्णानुन्न॑यति । सर्वे॒ हि पुण्या॑ रा॒द्धाः ॥ अ॒नूच॒ उन्न॑यति । प्र॒जाया॑ अनूचीन॒त्वाय॑ । अ॒नूच्ये॒वास्य॑ प्र॒जा ऽर्द्धु॑का भवति ॥ सं मृ॑शति॒ व्यावृ॑त्त्यै ॥ नाहो᳚ष्य॒न्नुप॑ सादयेत् । यदहो᳚ष्यन्नुप सा॒दये᳚त् । यथा॒ ऽन्यस्मा॑ उपनि॒धाय॑ \textbf{ 21} \newline
                  \newline
                                \textbf{ TB 2.1.3.7} \newline
                  अ॒न्यस्मै᳚ प्र॒यच्छ॑ति । ता॒दृगे॒व तत् । आऽस्मै॑ वृश्च्येत । यदे॒व गार्.ह॑पत्येऽधि॒श्रय॑ति । तेन॒ गार्.ह॑पत्यं प्रीणाति ॥ अ॒ग्निर॑बिभेत् । आहु॑तयो॒ माऽत्ये᳚ष्य॒न्तीति॑ । स ए॒ताꣳ स॒मिध॑-मपश्यत् । तामाऽध॑त्त । ततो॒ वा अ॒ग्नावाहु॑तयोऽद्ध्रियन्त \textbf{ 22} \newline
                  \newline
                                \textbf{ TB 2.1.3.8} \newline
                  यदे॑नꣳ स॒मय॑च्छत् । तथ्स॒मिधः॑ समि॒त्त्वम् । स॒मिध॒माद॑धाति । समे॒वैनं॑ ॅयच्छति । आहु॑तीनां॒ धृत्यै᳚ । अथो॑ अग्निहो॒त्रमे॒वेद्ध्मव॑त्-करोति । आहु॑तीनां॒ प्रति॑ष्ठित्यै ॥ ब्र॒ह्म॒वा॒दिनो॑ वदन्ति । यदेकाꣳ॑ स॒मिध॑मा॒धाय॒ द्वे आहु॑ती जु॒होति॑ । अथ॒ कस्याꣳ॑ स॒मिधि॑ द्वि॒तीया॒-माहु॑तिं जुहो॒तीति॑ \textbf{ 23} \newline
                  \newline
                                \textbf{ TB 2.1.3.9} \newline
                  यद्-द्वे स॒मिधा॑वाद॒द्ध्यात् । भ्रातृ॑व्यमस्मै जनयेत् । एकाꣳ॑ स॒मिध॑मा॒धाय॑ । यजु॑षा॒ ऽन्यामाहु॑तिं जुहोति । उ॒भे ए॒व स॒मिद्व॑ती॒ आहु॑ती जुहोति । नास्मै॒ भ्रातृ॑व्यं जनयति ॥ आदी᳚प्तायां जुहोति । समि॑द्धमिव॒ हि ब्र॑ह्मवर्च॒सम् । अथो॒ यथाऽति॑थिं॒ ज्योति॑ष्कृ॒त्वा प॑रि॒वेवे᳚ष्टि । ता॒दृगे॒व तत् ( ) ॥ च॒तुरुन्न॑यति । द्विर्जु॑होति । तस्मा᳚द्-द्वि॒पा-च्चतु॑ष्पादमत्ति । अथो᳚ द्वि॒पद्ये॒व चतु॑ष्पदः॒ प्रति॑ष्ठापयति । \textbf{ 24} \newline
                  \newline
                                    (भ॒व॒ति॒ - प्र॒ति॒षि॒ञ्चति॑ - गमयति - प्र॒त्यक् - प॒शव॑ - उपनि॒धा - या᳚द्घ्रिय॒न् - तेति॒ - तच्च॒त्वारि॑ च) \textbf{(A3)} \newline \newline
                \textbf{ 2.1.4      अनुवाकं   4 - उपयुक्तस्य हविषः संस्कारः} \newline
                                \textbf{ TB 2.1.4.1} \newline
                  उ॒त्त॒राव॑तीं॒ ॅवै दे॒वा आहु॑ति॒मजु॑हवुः । अवा॑ची॒मसु॑राः । ततो॑ दे॒वा अभ॑वन्न् । पराऽसु॑राः । यं का॒मये॑त॒ वसी॑यान्थ् स्या॒दिति॑ । कनी॑य॒स्तस्य॒ पूर्वꣳ॑ हु॒त्वा । उत्त॑रं॒ भूयो॑ जुहुयात् । ए॒षा वा उ॑त्त॒राव॒त्याहु॑तिः । तां दे॒वा अ॑जुहवुः । तत॒स्ते॑ऽभवन्न् \textbf{ 25} \newline
                  \newline
                                \textbf{ TB 2.1.4.2} \newline
                  यस्यै॒वं जुह्व॑ति । भव॑त्ये॒व ॥ यं का॒मये॑त॒ पापी॑यान्थ् स्या॒दिति॑ । भूय॒स्तस्य॒ पूर्वꣳ॑ हु॒त्वा । उत्त॑रं॒ कनी॑यो जुहुयात् । ए॒षा वा अवा॒च्याहु॑तिः । तामसु॑रा अजुहवुः । तत॒स्ते परा॑ऽभवन्न् । यस्यै॒वं जुह्व॑ति । परै॒व भ॑वति । \textbf{ 26} \newline
                  \newline
                                \textbf{ TB 2.1.4.3} \newline
                  हु॒त्वोप॑ सादय॒त्यजा॑मित्वाय । अथो॒ व्यावृ॑त्त्यै ॥ गार्.ह॑पत्यं॒ प्रती᳚क्षते । अन॑नुद्ध्यायिन-मे॒वैनं॑ करोति ॥ अ॒ग्नि॒हो॒त्रस्य॒ वै स्था॒णुर॑स्ति । तं ॅय ऋ॒च्छेत् । य॒ज्ञ्॒स्था॒णुमृ॑च्छेत् । ए॒ष वा अ॑ग्निहो॒त्रस्य॑ स्था॒णुः । यत्-पूर्वाऽऽहु॑तिः । तां ॅयदुत्त॑रया॒ऽभि जु॑हु॒यात् \textbf{ 27} \newline
                  \newline
                                \textbf{ TB 2.1.4.4} \newline
                  य॒ज्ञ्॒स्था॒णु-मृ॑च्छेत् । अ॒ति॒हाय॒ पूर्वा॒माहु॑तिं जुहोति । य॒ज्ञ्॒स्था॒णुमे॒व परि॑ वृणक्ति । अथो॒ भ्रातृ॑व्यमे॒वाप्त्वा ऽति॑क्रामति ॥ अ॒वा॒चीनꣳ॑ सा॒यमुप॑ मार्ष्टि । रेत॑ ए॒व तद्-द॑धाति । ऊ॒र्द्ध्वं प्रा॒तः । प्र ज॑नयत्ये॒व तत् ॥ ब्र॒ह्म॒वा॒दिनो॑ वदन्ति । च॒तुरुन्न॑यति \textbf{ 28} \newline
                  \newline
                                \textbf{ TB 2.1.4.5} \newline
                  द्विर्जु॑होति । अथ॒ क्व॑ द्वे आहु॑ती भवत॒ इति॑ । अ॒ग्नौ वै᳚श्वान॒र इति॑ ब्रूयात् । ए॒ष वा अ॒ग्निर्वै᳚श्वान॒रः । यद्ब्रा᳚ह्म॒णः । हु॒त्वा द्विः प्राश्ना॑ति । अ॒ग्नावे॒व वै᳚श्वान॒रे द्वे आहु॑ती जुहोति ॥ द्विर्जु॒होति॑ । द्विर्निमा᳚र्ष्टि । द्विः प्राश्ना॑ति \textbf{ 29} \newline
                  \newline
                                \textbf{ TB 2.1.4.6} \newline
                  षट्थ् संप॑द्यन्ते । षड्वा ऋ॒तवः॑ । ऋ॒तूने॒व प्री॑णाति । ब्र॒ह्म॒वा॒दिनो॑ वदन्ति । किं॒ दे॒व॒त्य॑-मग्निहो॒त्रमिति॑ । वै॒श्व॒दे॒वमिति॑ ब्रूयात् । यद्-यजु॑षा जु॒होति॑ । तदै᳚न्द्रा॒ग्नम् । यत्-तू॒ष्णीम् । तत्-प्रा॑जाप॒त्यम् \textbf{ 30} \newline
                  \newline
                                \textbf{ TB 2.1.4.7} \newline
                  यन्नि॒मार्ष्टि॑ । तदोष॑धीनाम् । यद्द्वि॒तीय᳚म् । तत्-पि॑तृ॒णाम् । यत्-प्राश्ना॑ति । तद्-गर्भा॑णाम् । तस्मा॒द्-गर्भा॒ अन॑श्नन्तो वर्द्धन्ते । यदा॒चाम॑ति । तन्म॑नु॒ष्या॑णाम् ॥ उद॑ङ्पर्या॒वृत्याचा॑मति \textbf{ 31} \newline
                  \newline
                                \textbf{ TB 2.1.4.8} \newline
                  आ॒त्मनो॑ गोपी॒थाय॑ ॥ निर्णे॑नेक्ति॒ शुद्ध्यै᳚ ॥ निष्ट॑पति स्व॒गाकृ॑त्यै । उद्-दि॑शति । स॒प्त॒र्.॒षीने॒व प्री॑णाति ॥ द॒क्षि॒णा प॒र्याव॑र्तते । स्वमे॒व वी॒र्य॑मनु॑ प॒र्याव॑र्तते । तस्मा॒द्-दक्षि॒णोऽर्द्ध॑ आ॒त्मनो॑ वी॒र्या॑वत्तरः । अथो॑ आदि॒त्यस्यै॒वावृत॒मनु॑ प॒र्याव॑र्तते ॥ हु॒त्वोप॒ समि॑न्धे \textbf{ 32} \newline
                  \newline
                                \textbf{ TB 2.1.4.9} \newline
                  ब्र॒ह्म॒व॒र्च॒सस्य॒ समि॑द्ध्यै ॥ न ब॒र॒.हिरनु॒ प्र ह॑रेत् । असꣳ॑स्थितो॒ वा ए॒ष य॒ज्ञ्ः । यद॑ग्निहो॒त्रम् । यद॑नु प्र॒हरे᳚त् । य॒ज्ञ्ं ॅविच्छि॑न्द्यात् । तस्मा॒न्नानु॑ प्र॒हृत्य᳚म् । य॒ज्ञ्स्य॒ सन्त॑त्यै ॥ अ॒पो नि न॑यति । अ॒व॒भृ॒थस्यै॒व रू॒पम॑कः ( ) । \textbf{ 33} \newline
                  \newline
                                    (अ॒भ॒व॒न् - भ॒व॒ति॒ - जु॒हु॒यान् - न॑यति - मार्ष्टि॒ द्विः प्राश्ना॑ति - प्राजाप॒त्य - माचा॑मती - न्धे - ऽकः) \textbf{(A4)} \newline \newline
                \textbf{ 2.1.5      अनुवाकं   5 - काम्यानि होमद्रव्याणि} \newline
                                \textbf{ TB 2.1.5.1} \newline
                  ब्र॒ह्म॒वा॒दिनो॑ वदन्ति । अ॒ग्नि॒हो॒त्रप्रा॑यणा य॒ज्ञाः । किं प्रा॑यण-मग्निहो॒त्रमिति॑ । व॒थ्सो वा अ॑ग्निहो॒त्रस्य॒ प्राय॑णम् । अ॒ग्नि॒हो॒त्रं ॅयज्ञाना᳚म् ॥ तस्य॑ पृथि॒वी सदः॑ । अ॒न्तरि॑क्ष॒-माग्नी᳚द्ध्रम् । द्यौर्.-ह॑वि॒र्द्धान᳚म् । दि॒व्या आपः॒ प्रोक्ष॑णयः । ओष॑धयो ब॒र॒.हिः \textbf{ 34} \newline
                  \newline
                                \textbf{ TB 2.1.5.2} \newline
                  वन॒स्पत॑य इ॒द्ध्मः । दिशः॑ परि॒धयः॑ । आ॒दि॒त्यो यूपः॑ । यज॑मानः प॒शुः । स॒मु॒द्रो॑ऽवभृ॒थः । स॒म्ॅव॒थ्स॒रः स्व॑गाका॒रः । तस्मा॒दाहि॑ताग्नेः॒ सर्व॑मे॒व ब॑र्हि॒ष्यं॑ द॒त्तं भ॑वति । यथ् सा॒यं जु॒होति॑ । रात्रि॑मे॒व तेन॑ दक्षि॒ण्यां᳚ कुरुते । यत्-प्रा॒तः \textbf{ 35} \newline
                  \newline
                                \textbf{ TB 2.1.5.3} \newline
                  अह॑रे॒व तेन॑ दक्षि॒ण्यं॑ कुरुते । यत्-ततो॒ ददा॑ति । सा दक्षि॑णा ॥ याव॑न्तो॒ वै दे॒वा अहु॑त॒मादन्न्॑ । ते परा॑ऽभवन्न् । त ए॒तद॑ग्निहो॒त्रꣳ सर्व॑स्यै॒व स॑मव॒दाया॑जुहवुः । तस्मा॑दाहुः । अ॒ग्नि॒हो॒त्रं ॅवै दे॒वा गृ॒हाणां॒ निष्कृ॑ति-मपश्य॒न्निति॑ । यथ्-सा॒यं जु॒होति॑ । रात्रि॑या ए॒व तद्धु॒ताद्या॑य \textbf{ 36} \newline
                  \newline
                                \textbf{ TB 2.1.5.4} \newline
                  यज॑मान॒स्या-प॑राभावाय । यत्-प्रा॒तः । अह्न॑ ए॒व तद्धु॒ताद्या॑य । यज॑मान॒स्या-प॑राभावाय । यत्-ततो॒ऽश्नाति॑ । हु॒तमे॒व तत् ॥ द्वयोः॒ पय॑सा जुहुयात्-प॒शुका॑मस्य । ए॒तद्वा अ॑ग्निहो॒त्रं मि॑थु॒नम् । य ए॒वं ॅवेद॑ । प्र प्र॒जया॑ प॒शुभि॑र्-मिथु॒नैर्-जा॑यते \textbf{ 37} \newline
                  \newline
                                \textbf{ TB 2.1.5.5} \newline
                  इ॒मामे॒व पूर्व॑या दु॒हे । अ॒मूमुत्त॑रया । अ॒धि॒श्रित्योत्त॑र॒मा न॑यति । योना॑वे॒व तद्-रेतः॑ सिञ्चति प्र॒जन॑ने ॥ आज्ये॑न जुहुया॒त्-तेज॑स्कामस्य । तेजो॒ वा आज्य᳚म् । ते॒ज॒स्व्ये॑व भ॑वति । पय॑सा प॒शुका॑मस्य । ए॒तद्वै प॑शू॒नाꣳ रू॒पम् । रू॒पेणै॒वास्मै॑ प॒शूनव॑ रुन्धे \textbf{ 38} \newline
                  \newline
                                \textbf{ TB 2.1.5.6} \newline
                  प॒शु॒माने॒व भ॑वति । द॒द्ध्नेन्द्रि॒य का॑मस्य । इ॒न्द्रि॒यं ॅवै दधि॑ । इ॒न्द्रि॒या॒व्ये॑व भ॑वति । य॒वा॒ग्वा᳚ ग्राम॑कामस्यौष॒धा वै म॑नु॒ष्याः᳚ । भा॒ग॒धेये॑नै॒वास्मै॑ सजा॒तानव॑-रुन्धे । ग्रा॒म्ये॑व भ॑वति ॥अय॑ज्ञो॒ वा ए॒षः । यो॑सा॒मा \textbf{ 39} \newline
                  \newline
                                \textbf{ TB 2.1.5.7} \newline
                  च॒तुरुन्न॑यति । चतु॑रक्षरꣳ रथन्त॒रम् । र॒थ॒न्त॒रस्यै॒ष वर्णः॑ । उ॒परी॑व हरति । अ॒न्तरि॑क्षं ॅवामदे॒व्यम् । वा॒म॒दे॒व्यस्यै॒ष वर्णः॑ । द्विर्जु॑होति । द्व्य॑क्षरं बृ॒हत् । बृ॒ह॒त ए॒ष वर्णः॑ । अ॒ग्नि॒हो॒त्रमे॒व तथ् साम॑न्वत्-करोति । \textbf{ 40} \newline
                  \newline
                                \textbf{ TB 2.1.5.8} \newline
                  यो वा अ॑ग्निहो॒त्रस्यो॑प॒सदो॒ वेद॑ । उपै॑नमुप॒सदो॑ नमन्ति । वि॒न्दत॑ उपस॒त्तार᳚म् । उ॒न्नीयोप॑ सादयति । पृ॒थि॒वीमे॒व प्री॑णाति । हो॒ष्यन्नुप॑ सादयति । अ॒न्तरि॑क्षमे॒व प्री॑णाति । हु॒त्वोप॑ सादयति । दिव॑मे॒व प्री॑णाति । ए॒ता वा अ॑ग्निहो॒त्रस्यो॑प॒सदः॑ \textbf{ 41} \newline
                  \newline
                                \textbf{ TB 2.1.5.9} \newline
                  य ए॒वं ॅवेद॑ । उपै॑न-मुप॒सदो॑ नमन्ति । वि॒न्दत॑ उपस॒त्तार᳚म् ॥ यो वा अ॑ग्निहो॒त्रस्याश्रा॑वितं प्र॒त्याश्रा॑वितꣳ॒॒ होता॑रं ब्र॒ह्माणं॑ ॅवषट्का॒रं ॅवेद॑ । तस्य॒ त्वे॑व हु॒तम् । प्रा॒णो वा अ॑ग्निहो॒त्रस्याश्रा॑वितम् । अ॒पा॒नः प्र॒त्याश्रा॑वितम् । मनो॒ होता᳚ । चक्षु॑र्-ब्र॒ह्मा । नि॒मे॒षो व॑षट्का॒रः \textbf{ 42} \newline
                  \newline
                                \textbf{ TB 2.1.5.10} \newline
                  य ए॒वं ॅवेद॑ । तस्य॒ त्वे॑व हु॒तम् ॥ सा॒यं॒ ॅयावा॑नश्च॒ वै दे॒वाः प्रा॑त॒र्यावा॑णश्चा-ग्निहो॒त्रिणो॑ गृ॒हमाग॑च्छन्ति । तान्. यन्न त॒र्पये᳚त् । प्र॒जया᳚ऽस्य प॒शुभि॒र्वि ति॑ष्ठेरन्न् । यत्त॒र्पये᳚त् । तृ॒प्ता ए॑नं प्र॒जया॑ प॒शुभि॑स्तर्पयेयुः ॥ स॒जूर्दे॒वैः सा॒यं ॅयाव॑भि॒रिति॑ सा॒यꣳ सं मृ॑शति । स॒जूर्दे॒वैः प्रा॒तर्याव॑भि॒रिति॑ प्रा॒तः । ये चै॒व दे॒वाः सा॑यं॒ ॅयावा॑नो॒ ये च॑ प्रात॒र्यावा॑णः \textbf{ 43} \newline
                  \newline
                                \textbf{ TB 2.1.5.11} \newline
                  ताने॒वोभयाꣳ॑ स्तर्पयति । त ए॑नं तृ॒प्ताः प्र॒जया॑ प॒शुभि॑ स्तर्पयन्ति ॥ अ॒रु॒णो ह॑ स्मा॒हौप॑वेशिः । अ॒ग्नि॒हो॒त्र ए॒वाहꣳ सा॒यं प्रा॑त॒र्वज्रं॒ भ्रातृ॑व्येभ्यः॒ प्र ह॑रामि । तस्मा॒न्-मत्पापी॑याꣳसो॒ भ्रातृ॑व्या॒ इति॑ । च॒तुरुन्न॑यति । द्विर्जु॑होति । स॒मिथ्-स॑प्त॒मी । स॒प्तप॑दा॒ शक्व॑री । शा॒क्व॒रो वज्रः॑ ( ) । अ॒ग्नि॒हो॒त्र ए॒व तथ्सा॒यं प्रा॑त॒र्वज्रं॒ ॅयज॑मानो॒ भ्रातृ॑व्याय॒ प्र ह॑रति । भव॑त्या॒त्मना᳚ । परा᳚ऽस्य॒ भ्रातृ॑व्यो भवति । \textbf{ 44} \newline
                  \newline
                                    (ब॒र्॒.हिः - प्रा॒तर्. - हु॒ताद्या॑य - जायते - रुन्धे - ऽसा॒मा - क॑रो - त्ये॒ता वा अ॑ग्निहो॒त्रस्यो॑ प॒सदो॑ - वषट्का॒र - श्च॑ प्रात॒र्यावा॑णो॒ - वज्र॒स्त्रीणि॑ च) \textbf{(A5)} \newline \newline
                \textbf{ 2.1.6     अनुवाकं   6 - अभ्युद्द्रवणं, होमाभावप्रतीकारश्च} \newline
                                \textbf{ TB 2.1.6.1} \newline
                  प्र॒जाप॑ति-रकामय-तात्म॒न् वन्मे॑ जाये॒तेति॑ । सो॑ऽजुहोत् । तस्या᳚त्म॒न् वद॑जायत । अ॒ग्निर्-वा॒युरा॑-दि॒त्यः । ते᳚ऽब्रुवन्न् । प्र॒जाप॑ति-रहौषी-दात्म॒न्-वन्मे॑ जाये॒तेति॑ । तस्य॑ व॒यम॑जनिष्महि । जाय॑तां न आत्म॒न्-वदिति॒ ते॑ऽजुहवुः । प्रा॒णाना॑म॒ग्निः । त॒नुवै॑ वा॒युः \textbf{ 45} \newline
                  \newline
                                \textbf{ TB 2.1.6.2} \newline
                  चक्षु॑ष आदि॒त्यः । तेषाꣳ॑ हु॒ताद॑जायत॒ गौरे॒व ॥ तस्यै॒ पय॑सि॒ व्याय॑च्छन्त । मम॑ हु॒ताद॑जनि॒ ममेति॑ । ते प्र॒जाप॑तिं प्र॒श्नमा॑यन्न् । स आ॑दि॒त्यो᳚ऽग्नि-म॑ब्रवीत् । य॒त॒रो नौ॒ जया᳚त् । तन्नौ॑ स॒हास॒दिति॑ । कस्यै कोऽहौ॑षी॒दिति॑ प्र॒जाप॑ति-रब्रवी॒त्-कस्यै क॒ इति॑ । प्रा॒णाना॑म॒हमित्य॒ग्निः \textbf{ 46} \newline
                  \newline
                                \textbf{ TB 2.1.6.3} \newline
                  त॒नुवा॑ अ॒हमिति॑ वा॒युः । चक्षु॑षो॒ऽहमित्या॑दि॒त्यः । य ए॒व प्रा॒णाना॒महौ॑षीत् । तस्य॑ हु॒ताद॑ज॒नीति॑ । अ॒ग्नेर्. हु॒ताद॑ज॒नीति॑ । तद॑ग्निहो॒त्रस्या᳚-ग्निहोत्र॒त्वम् । गौर्वा अ॑ग्निहो॒त्रम् । य ए॒वं ॅवेद॒ गौर॑ग्निहो॒त्रमिति॑ । प्रा॒णा॒पा॒नाभ्या॑-मे॒वाग्निꣳ सम॑र्द्धयति । अव्य॑र्द्धुकः प्राणापा॒नाभ्यां᳚ भवति \textbf{ 47} \newline
                  \newline
                                \textbf{ TB 2.1.6.4} \newline
                  य ए॒वं ॅवेद॑ ॥ तौ वा॒युर॑ब्रवीत् । अनु॒ मा भ॑जत॒मिति॑ । यदे॒व गार्.ह॑पत्येऽधि॒श्रित्या॑-हव॒नीय॑-म॒भ्यु॑द्द्रवान्॑ । तेन॒ त्वां प्री॑णा॒नित्य॑-ब्रूताम् । तस्मा॒द्-यद्-गार्.ह॑पत्येऽधि॒-श्रित्या॑हव॒नीय॑-म॒भ्यु॑द्द्रव॑ति । वा॒युमे॒व तेन॑ प्रीणाति ॥ प्र॒जाप॑तिर्-दे॒वताः᳚ सृ॒जमा॑नः । अ॒ग्निमे॒व दे॒वता॑नां प्रथ॒म-म॑सृजत । सो᳚ऽन्यदा॑ल॒म्-भ्य॑मवि॑त्त्वा \textbf{ 48} \newline
                  \newline
                                \textbf{ TB 2.1.6.5} \newline
                  प्र॒जाप॑तिम॒भि प॒र्याव॑र्तत । स मृ॒त्यो-र॑बिभेत् । सो॑ऽमुमा॑दि॒त्य-मा॒त्मनो॒ निर॑मिमीत । तꣳहु॒त्वा परा᳚ङ्प॒र्याव॑र्तत । ततो॒ वै स मृ॒त्युमपा॑जयत् । अप॑ मृ॒त्युं ज॑यति । य ए॒वं ॅवेद॑ ॥ तस्मा॒द्-यस्यै॒वं ॅवि॒दुषः॑ । उ॒तैका॒हमु॒त द्व्य॒हं न जुह्व॑ति । हु॒तमे॒वास्य॑ भवति ( ) । अ॒सौ ह्या॑दि॒त्यो᳚ऽग्निहो॒त्रम् । \textbf{ 49} \newline
                  \newline
                                    (त॒नुवै॑ वा॒यु - र॒ग्निर् - भ॑व॒ - त्यवि॑त्त्वा - भव॒त्येकं॑ च) \textbf{(A6)} \newline \newline
                \textbf{ 2.1.7     अनुवाकं   7 - हविषः सर्वदेवतासंबन्धेन प्रशंसा} \newline
                                \textbf{ TB 2.1.7.1} \newline
                  रौ॒द्रं गवि॑ । वा॒य॒व्य॑-मुप॑सृष्टम् । आ॒श्वि॒नं दु॒ह्यमा॑नम् । सौ॒म्यं दु॒ग्धम् । वा॒रु॒णमधि॑श्रितम् । वै॒श्व॒दे॒वा भि॒न्दवः॑ । पौ॒ष्णमुद॑न्तम् । सा॒र॒स्व॒तं ॅवि॒ष्यन्द॑मानम् । मै॒त्रꣳ शरः॑ । धा॒तुरुद्वा॑सितम् ( ) । बृह॒स्पते॒रुन्नी॑तम् । स॒वि॒तुः प्रक्रा᳚न्तम् । द्या॒वा॒पृ॒थि॒व्यꣳ॑ ह्रि॒यमा॑णम् । ऐ॒न्द्रा॒ग्न-मुप॑सन्नम् । अ॒ग्नेः पूर्वा ऽऽहु॑तिः । प्र॒जाप॑ते॒-रुत्त॑रा । ऐ॒न्द्रꣳ हु॒तम् । \textbf{ 50} \newline
                  \newline
                                    (उद्वा॑सितꣳ स॒प्त च॑) \textbf{(A7)} \newline \newline
                \textbf{ 2.1.8     अनुवाकं   8 - दोहनप्रकारोऽग्निहोत्रहोत्रश्च} \newline
                                \textbf{ TB 2.1.8.1} \newline
                  द॒क्षि॒ण॒त उप॑ सृजति । पि॒तृ॒लो॒कमे॒व तेन॑ जयति ॥ प्राची॒माव॑र्तयति । दे॒व॒लो॒कमे॒व तेन॑ जयति । उदी॑चीमा॒वृत्य॑ दोग्धि । म॒नु॒ष्य॒लो॒कमे॒व तेन॑ जयति ॥ पूर्वौ॑ दुह्याज्ज्ये॒ष्ठस्य॑ ज्यैष्ठिने॒यस्य॑ । यो वा॑ ग॒तश्रीः॒ स्यात् । अप॑रौ दुह्यात्-कनि॒ष्ठस्य॑ कानिष्ठिने॒यस्य॑ । यो वा॒ बुभू॑षेत् । \textbf{ 51} \newline
                  \newline
                                \textbf{ TB 2.1.8.2} \newline
                  न सं मृ॑शति । पा॒प॒व॒स्य॒सस्य॒ व्यावृ॑त्त्यै ॥ वा॒य॒व्यं॑ ॅवा ए॒तदुप॑सृष्टम् । आ॒श्वि॒नं दु॒ह्यमा॑नम् । मै॒त्रं दु॒ग्धम् । अ॒र्य॒म्ण उ॑द्वा॒स्यमा॑नम् । त्वा॒ष्ट्र-मु॑न्नी॒यमा॑नम् । बृह॒स्पते॒-रुन्नी॑तम् । स॒वि॒तुः प्रक्रा᳚न्तम् । द्या॒वा॒पृ॒थि॒व्यꣳ॑ ह्रि॒यमा॑णम् । \textbf{ 52} \newline
                  \newline
                                \textbf{ TB 2.1.8.3} \newline
                  ऐ॒न्द्रा॒ग्नमुप॑ सादितम् । सर्वा᳚भ्यो॒ वा ए॒ष दे॒वता᳚भ्यो जुहोति । यो᳚ऽग्निहो॒त्रं जु॒होति॑ ॥ यथा॒ खलु॒ वै धे॒नुं ती॒र्थे त॒र्पय॑ति । ए॒वम॑ग्निहो॒त्री यज॑मानं तर्पयति । तृप्य॑ति प्र॒जया॑ प॒शुभिः॑ । प्रसु॑व॒र्गं ॅलो॒कं जा॑नाति । पश्य॑ति पु॒त्रम् । पश्य॑ति॒ पौत्र᳚म् । प्र प्र॒जया॑ प॒शुभि॑र्-मिथु॒नैर्-जा॑यते ( ) । यस्यै॒वं ॅवि॒दुषो᳚ऽग्निहो॒त्रं जुह्व॑ति । य उ॑ चैन दे॒वं ॅवेद॑ । \textbf{ 53} \newline
                  \newline
                                    (बभू॑षे- द्ध्रि॒यमा॑णं - जायते॒ द्वे च॑) \textbf{(A8)} \newline \newline
                \textbf{ 2.1.9     अनुवाकं   9 - अग्निहोत्रस्यासंसृष्टहोममन्त्रौ} \newline
                                \textbf{ TB 2.1.9.1} \newline
                  त्रयो॒ वै प्रै॑यमे॒धा आ॑सन्न् । तेषां॒ त्रिरेको᳚ऽग्निहो॒त्र-म॑जुहोत् । द्विरेकः॑ । स॒कृदेकः॑ । तेषां॒ ॅयस्त्रिरजु॑होत् । स ऋ॒चाऽजु॑होत् । यो द्विः । स यजु॑षा । यः स॒कृत् । स तू॒ष्णीम् \textbf{ 54} \newline
                  \newline
                                \textbf{ TB 2.1.9.2} \newline
                  यश्च॒ यजु॒षा ऽजु॑हो॒द्यश्च॑ तू॒ष्णीम् । तावु॒भावा᳚र्द्ध्नुताम् । तस्मा॒द्-यजु॒षा ऽऽहु॑तिः॒ पूर्वा॑ होत॒व्या᳚ । तू॒ष्णीमुत्त॑रा । उ॒भे ए॒वर्द्धी अव॑रुन्धे । अ॒ग्निर्-ज्योति॒र् ज्योति॑र॒ग्निः स्वाहेति॑ सा॒यं जु॑होति । रेत॑ ए॒व तद्-द॑धाति । सूर्यो॒ ज्योति॒र्-ज्योतिः॒ सूर्यः॒ स्वाहेति॑ प्रा॒तः । रेत॑ ए॒व हि॒तं प्र ज॑नयति ॥ रेतो॒ वा ए॒तस्य॑ हि॒तं न प्र जा॑यते \textbf{ 55} \newline
                  \newline
                                \textbf{ TB 2.1.9.3} \newline
                  यस्या᳚ग्निहो॒त्रमहु॑तꣳ॒॒ सूर्यो॒ऽभ्यु॑देति॑ । यद्-यन्ते॒ स्यात् । उ॒न्नीय॒ प्राङु॒दाद्र॑वेत् । स उ॑प॒साद्यातमि॑तो-रासीत । स य॒दा ताम्ये᳚त् । अथ॒ भूः स्वाहेति॑ जुहुयात् । प्र॒जाप॑ति॒र्वै भू॒तः । तमे॒वो-पा॑सरत् । स ए॒वैनं॒ तत॒ उन्न॑यति । नार्ति॒मार्च्छ॑ति॒ यज॑मानः ( ) । \textbf{ 56} \newline
                  \newline
                                    (तू॒ष्णीं - जा॑यते॒ - यज॑मानः) \textbf{(A9)} \newline \newline
                \textbf{ 2.1.10    अनुवाकं   10 - वह्नेरवस्थाविशेषानुसारेण होमः} \newline
                                \textbf{ TB 2.1.10.1} \newline
                  यद॒ग्नि-मु॒द्धर॑ति । वस॑व॒स्तर्ह्य॒ग्निः । तस्मि॒न्॒. यस्य॒ तथा॑विधे॒ जुह्व॑ति । वसु॑ष्वे॒वास्या᳚ग्निहो॒त्रꣳ हु॒तं भ॑वति ॥ निहि॑तो धूपा॒यञ्छे॑ते । रु॒द्रास्तह्य॒र्ग्निः । तस्मि॒न्.॒ यस्य॒ तथा॑विधे॒ जुह्व॑ति । रु॒द्रेष्वे॒वास्या᳚ग्निहो॒त्रꣳ हु॒तं भ॑वति ॥ प्र॒थ॒ममि॒द्ध्मम॒र्चिरा ल॑भते । आ॒दि॒त्यास्तर्ह्य॒ग्निः \textbf{ 57} \newline
                  \newline
                                \textbf{ TB 2.1.10.2} \newline
                  तस्मि॒न्॒. यस्य॒ तथा॑विधे॒ जुह्व॑ति । आ॒दि॒त्येष्वे॒वास्या᳚ग्निहो॒त्रꣳ हु॒तं भ॑वति ॥ सर्व॑ ए॒व स॑र्व॒श इ॒द्ध्म आदी᳚प्तो भवति । विश्वे॑ दे॒वास्तह्य॒र्ग्निः । तस्मि॒न्॒. यस्य॒ तथा॑विधे॒ जुह्व॑ति । विश्वे᳚ष्वे॒वास्य॑ दे॒वेष्व॑ग्निहो॒त्रꣳ हु॒तं भ॑वति ॥ नि॒त॒राम॒र्चिरु॒पावै॑ति लोहि॒नीके॑व भवति । इन्द्र॒स्तर्ह्य॒ग्निः । तस्मि॒न्॒. यस्य॒ तथा॑विधे॒ जुह्व॑ति । इन्द्र॑ ए॒वास्या᳚ग्निहो॒त्रꣳ हु॒तं भ॑वति । \textbf{ 58} \newline
                  \newline
                                \textbf{ TB 2.1.10.3} \newline
                  अङ्गा॑रा भवन्ति । तेभ्योऽङ्गा॑रेभ्यो॒ऽर्चिरुदे॑ति । प्र॒जाप॑ति॒स्तर्ह्य॒ग्निः । तस्मि॒न्॒. यस्य॒ तथा॑विधे॒ जुह्व॑ति । प्र॒जाप॑तावे॒वास्या᳚ग्निहो॒त्रꣳ हु॒तं भ॑वति ॥ शरोऽङ्गा॑रा॒ अद्ध्यू॑हन्ते । ब्रह्म॒-तर्ह्य॒ग्निः । तस्मि॒न्.॒ यस्य॒ तथा॑विधे॒ जुह्व॑ति । ब्रह्म॑न्ने॒वास्या᳚ग्निहो॒त्रꣳ हु॒तं भ॑वति ॥ वसु॑षु रु॒द्रेष्वा॑दि॒त्येषु॒ विश्वे॑षु दे॒वेषु॑ ( ) । इन्द्रे᳚ प्र॒जाप॑तौ॒ ब्रह्मन्न्॑ । अप॑रिवर्गमे॒वास्यै॒तासु॑ दे॒वता॑सु हु॒तं भ॑वति । यस्यै॒वं ॅवि॒दुषो᳚ऽग्निहो॒त्रं जुह्व॑ति । य उ॑ चैन दे॒वं ॅवेद॑ । \textbf{ 59} \newline
                  \newline
                                                        \textbf{special korvai} \newline
              (यद॒ग्निं निहि॑तः प्रथ॒मꣳ सर्व॑ ए॒व नि॑त॒रामङ्गा॑राः॒ शरोऽङ्गा॑रा॒ ब्रह्म॒वसु॑ष्व॒ष्टौ) \newline
                                (आ॒दि॒त्यास्तर्ह्य॒ग्निर् - इन्द्र॑ ए॒वास्या᳚ग्निहो॒त्रꣳ हु॒तं भ॑वति - दे॒वेषु॑ च॒त्वारि॑ च) \textbf{(A10)} \newline \newline
                \textbf{ 2.1.11    अनुवाकं   11 - कालभेदेन समन्त्रकं परिषेचनम्} \newline
                                \textbf{ TB 2.1.11.1} \newline
                  ऋ॒तं त्वा॑ स॒त्येन॒ परि॑षिञ्चा॒मीति॑ सा॒यं परि॑षिञ्चति । स॒त्यं त्व॒र्तेन॒ परि॑षिञ्चा॒मीति॑ प्रा॒तः । अ॒ग्निर्वा ऋ॒तम् । अ॒सावा॑दि॒त्यः स॒त्यम् । अ॒ग्निमे॒व तदा॑दि॒त्येन॑ सा॒यं परि॑षिञ्चति । अ॒ग्निना॑ ऽऽदि॒त्यं प्रा॒तस्सः । याव॑दहोरा॒त्रे भव॑तः । ताव॑दस्य लो॒कस्य॑ । नार्ति॒र्न रिष्टिः॑ । नान्तो॒ न प॑र्य॒न्तो᳚ऽस्ति ( ) । यस्यै॒वं ॅवि॒दुषो᳚ऽग्निहो॒त्रं जुह्व॑ति । य उ॑ चैन दे॒वं ॅवेद॑ । \textbf{ 60} \newline
                  \newline
                                    (अ॒स्ति॒ द्वे च॑) \textbf{(A11)} \newline \newline
                \textbf{PrapAtaka Korvai with starting  words of 1 to11 anuvAkams :-} \newline
        (अङ्गि॑रसः - प्र॒जाप॑तिर॒ग्निꣳ - रु॒द्र - उ॑त्त॒राव॑तीं - ब्रह्मवा॒दिनो᳚ ऽग्निहो॒त्रप्रा॑यणा य॒ज्ञाः - प्र॒जाप॑तिरकामयतात्म॒न्वद् - रौ॒द्रं गवि॑ -दक्षिण॒त - स्त्रयो॒ वै - यद॒ग्नि - मृ॒तं त्वा॑ स॒त्येनैका॑दश) \newline

        \textbf{korvai with starting words of 1, 11, 21 series of daSinis:-} \newline
        (अङ्गि॑रसः॒ - प्रैव तेन॑ - प॒शूने॒व - यन्नि॒मार्ष्टि॒ - यो वा अ॑ग्निहो॒त्रस्यो॑प॒सदो॑ - दक्षिण॒तः ष॑ष्टिः ) \newline

        \textbf{first and last  word - 1st prapATakam, 2nd aShTakam,:-} \newline
        (अङ्गि॑रसो॒ - य उ॑ चैन दे॒वं ॅवेद॑) \newline 

       

        ॥ हरिः॑ ॐ ॥
॥ कृष्ण यजुर्वेदीय तैत्तिरीय ब्राह्मणे द्वितीयाष्टके प्रथमः प्रपाठकः समाप्तः ॥
============= \newline
        \pagebreak
        
        
        
     \addcontentsline{toc}{section}{ 2.2     द्वितीयाष्टके द्वितीयः प्रपाठकः - होतृब्राह्मणम्}
     \markright{ 2.2     द्वितीयाष्टके द्वितीयः प्रपाठकः - होतृब्राह्मणम् \hfill https://www.vedavms.in \hfill}
     \section*{ 2.2     द्वितीयाष्टके द्वितीयः प्रपाठकः - होतृब्राह्मणम् }
                \textbf{ 2.2.1     अनुवाकं   1 - दशहोतृमन्त्रस्य क्रत्वर्थपुरुषार्थप्रयोगौ} \newline
                                \textbf{ TB 2.2.1.1} \newline
                  प्र॒जाप॑तिरकामयत प्र॒जाः सृ॑जे॒येति॑ । स ए॒तं दश॑होतार-मपश्यत् । तं मन॑सा ऽनु॒द्रुत्य॑ दर्भस्त॒म्बे॑ऽजुहोत् । ततो॒ वै स प्र॒जा अ॑सृजत । ता अ॑स्माथ्सृ॒ष्टा अपा᳚क्रामन्न् । ता ग्रहे॑णागृह्णात् । तद्ग्रह॑स्य ग्रह॒त्वम् । यः का॒मये॑त॒ प्रजा॑ये॒येति॑ । स दश॑होतारं॒ मन॑सा ऽनु॒द्रुत्य॑ दर्भस्त॒म्बे जु॑हुयात् । प्र॒जाप॑ति॒र्वै दश॑होता \textbf{ 1} \newline
                  \newline
                                \textbf{ TB 2.2.1.2} \newline
                  प्र॒जाप॑तिरे॒व भू॒त्वा प्र जा॑यते ॥ मन॑सा जुहोति । मन॑ इव॒ हि प्र॒जाप॑तिः । प्र॒जाप॑ते॒राप्त्यै᳚ ॥ पू॒र्णया॑ जुहोति । पू॒र्ण इ॑व॒ हि प्र॒जाप॑तिः । प्र॒जाप॑ते॒राप्त्यै᳚ ॥ न्यू॑नया जुहोति । न्यू॑ना॒द्धि प्र॒जाप॑तिः प्र॒जा असृ॑जत । प्र॒जानाꣳ॒॒ सृष्ट्यै᳚ । \textbf{ 2} \newline
                  \newline
                                \textbf{ TB 2.2.1.3} \newline
                  द॒र्भ॒स्त॒म्बे जु॑होति । ए॒तस्मा॒द्वै योनेः᳚ प्र॒जाप॑तिः प्र॒जा अ॑सृजत । यस्मा॑दे॒व योनेः᳚ प्र॒जाप॑तिः प्र॒जा असृ॑जत । तस्मा॑दे॒व योनेः॒ प्र जा॑यते ॥ ब्रा॒ह्म॒णो द॑क्षिण॒त उपा᳚स्ते । ब्रा॒ह्म॒णो वै प्र॒जाना॑-मुपद्र॒ष्टा । उ॒प॒द्र॒ष्टु॒मत्ये॒व प्र जा॑यते ॥ ग्रहो॑ भवति । प्र॒जानाꣳ॑ सृ॒ष्टानां॒ धृत्यै᳚ ॥ यं ब्रा᳚ह्म॒णं ॅवि॒द्यां ॅवि॒द्वाꣳसं॒ ॅयशो॒ नर्च्छेत् । \textbf{ 3} \newline
                  \newline
                                \textbf{ TB 2.2.1.4} \newline
                  सोऽर॑ण्यं प॒रेत्य॑ । द॒र्भ॒स्त॒म्ब-मु॒द्ग्रत्थ्य॑ । ब्रा॒ह्म॒णं द॑क्षिण॒तो नि॒षाद्य॑ । चतु॑र्.होतॄ॒न् व्याच॑क्षीत । ए॒तद्वै दे॒वानां᳚ पर॒मं गुह्यं॒ ब्रह्म॑ । यच्चतु॑र्.होतारः । तदे॒व प्र॑का॒शं ग॑मयति । तदे॑नं प्रका॒शं ग॒तम् । प्र॒का॒शं प्र॒जानां᳚ गमयति ॥ द॒र्भ॒स्त॒म्ब-मु॒द्ग्रत्थ्य॒ व्याच॑ष्टे \textbf{ 4} \newline
                  \newline
                                \textbf{ TB 2.2.1.5} \newline
                  अ॒ग्नि॒वान्. वै द॑र्भस्त॒म्बः । अ॒ग्नि॒वत्ये॒व व्याच॑ष्टे ॥ ब्रा॒ह्म॒णो द॑क्षिण॒त उपा᳚स्ते । ब्रा॒ह्म॒णो वै प्र॒जाना॑-मुपद्र॒ष्टा । उ॒प॒द्र॒ष्टु॒-मत्ये॒वैनं॒ ॅयश॑ ऋच्छति ॥ ई॒श्व॒रं तं ॅयशोऽर्तो॒रित्या॑हुः । यस्यान्ते᳚ व्या॒चष्ट॒ इति॑ । वर॒स्तस्मै॒ देयः॑ । यदे॒वैनं॒ तत्रो॑प॒ नम॑ति । तदे॒वाव॑ रुन्धे । \textbf{ 5} \newline
                  \newline
                                \textbf{ TB 2.2.1.6} \newline
                  अ॒ग्निमा॒दधा॑नो॒ दश॑होत्रा॒ ऽरणि॒मव॑दद्ध्यात् । प्रजा॑तमे॒वैन॒-माध॑त्ते । तेनै॒वोद्द्रुत्या᳚ग्निहो॒त्रं जु॑हुयात् । प्रजा॑तमे॒वैन॑ज्जुहोति । ह॒विर्नि॑र्व॒फ्स्यन्-दश॑होतारं॒ ॅव्याच॑क्षीत । प्रजा॑तमे॒वैनं॒ निर्व॑पति । सा॒मि॒धे॒नी-र॑नुव॒क्ष्यन्-दश॑होतारं॒ ॅव्याच॑क्षीत । सा॒मि॒धे॒नीरे॒व सृ॒ष्ट्वाऽऽरभ्य॒ प्र त॑नुते । अथो॑ य॒ज्ञो वै दश॑होता । य॒ज्ञ्मे॒व त॑नुते । \textbf{ 6} \newline
                  \newline
                                \textbf{ TB 2.2.1.7} \newline
                  अ॒भि॒चर॒न्-दश॑होतारं जुहुयात् । नव॒ वै पुरु॑षे प्रा॒णाः । नाभि॑र्दश॒मी । सप्रा॑णमे॒वैन॑-म॒भिच॑रति । ए॒ताव॒द्वै पुरु॑षस्य॒ स्वम् । याव॑त् प्रा॒णाः । याव॑दे॒वास्यास्ति॑ । तद॒भिच॑रति ॥ स्वकृ॑त॒ इरि॑णे जुहोति प्रद॒रे वा᳚ । ए॒तद्वा अ॒स्यै निर्.ऋ॑तिगृहीतम् ( ) । निर्.ऋ॑ति गृहीत ए॒वैनं॒ निर्.ऋ॑त्या ग्राहयति ॥ यद् वा॒चः क्रू॒रम् । तेन॒ वष॑ट्करोति । वा॒च ए॒वैनं॑ क्रू॒रेण॒ प्र वृ॑श्चति । ता॒जगार्ति॒मार्च्छ॑ति । \textbf{ 7} \newline
                  \newline
                                    (दश॑होता॒ - सृष्ट्या॑ - ऋ॒च्छेद् - व्याच॑ष्टे - रुन्ध - ए॒व त॑नुते॒ - निर्.ऋ॑तिगृहीतं॒ पञ्च॑ च) \textbf{(A1)} \newline \newline
                \textbf{ 2.2.2      अनुवाकं   2 - चतुर्होत्रादिमन्त्राणां क्रत्वर्थप्रयोगः} \newline
                                \textbf{ TB 2.2.2.1} \newline
                  प्र॒जाप॑तिरकामयत दर्.शपूर्णमा॒सौ सृ॑जे॒येति॑ । स ए॒तं चतु॑र्.होतार-मपश्यत् । तं मन॑साऽनु॒द्रुत्या॑ हव॒नीये॑ऽजुहोत् । ततो॒ वै स द॑र्.शपूर्णमा॒साव॑सृजत । ताव॑स्माथ् सृ॒ष्टावपा᳚क्रामताम् । तौ ग्रहे॑णा गृह्णात् । तद्ग्रह॑स्य ग्रह॒त्वम् । द॒र्॒.श॒पू॒र्ण॒मा॒सा-वा॒लभ॑मानः । चतु॑र्.होतारं॒ मन॑साऽनु॒द्रुत्या॑-हव॒नीये॑ जुहुयात् । द॒र्॒.श॒पू॒र्ण॒मा॒सावे॒व सृ॒ष्ट्वाऽऽरभ्य॒ प्र त॑नुते \textbf{ 8} \newline
                  \newline
                                \textbf{ TB 2.2.2.2} \newline
                  ग्रहो॑ भवति । द॒र्॒.श॒पू॒र्ण॒मा॒सयोः᳚ सृ॒ष्टयो॒र्द्धृत्यै᳚ ॥ सो॑ऽकामयत चातुर्मा॒स्यानि॑ सृजे॒येति॑ । स ए॒तं पञ्च॑होतार-मपश्यत् । तं मन॑साऽनु॒द्रुत्या॑ हव॒नीये॑ऽजुहोत् । ततो॒ वै स चा॑तुर्मा॒स्यान्य॑सृजत । तान्य॑स्माथ्-सृ॒ष्टान्यपा᳚क्रामन्न् । तानि॒ ग्रहे॑णागृह्णात् । तद्ग्रह॑स्य ग्रह॒त्वम् । चा॒तु॒र्मा॒स्यान्या॒लभ॑मानः \textbf{ 9} \newline
                  \newline
                                \textbf{ TB 2.2.2.3} \newline
                  पञ्च॑होतारं॒ मन॑साऽनु॒द्रुत्या॑ हव॒नीये॑ जुहुयात् । चा॒तु॒र्मा॒स्यान्ये॒व सृ॒ष्ट्वाऽऽरभ्य॒ प्र त॑नुते । ग्रहो॑ भवति । चा॒तु॒र्मा॒स्यानाꣳ॑ सृ॒ष्टानां॒ धृत्यै᳚ ॥ सो॑ऽकामयत पशुब॒न्धꣳ सृ॑जे॒येति॑ । स ए॒तꣳ षड्ढो॑तार-मपश्यत् । तं मन॑साऽनु॒द्रुत्या॑ हव॒नीये॑ऽजुहोत् । ततो॒ वै स प॑शुब॒न्धम॑सृजत । सो᳚ऽस्माथ् सृ॒ष्टो ऽपा᳚क्रामत् । तं ग्रहे॑णागृह्णात् \textbf{ 10} \newline
                  \newline
                                \textbf{ TB 2.2.2.4} \newline
                  तद्ग्रह॑स्य ग्रह॒त्वम् । प॒शु॒ब॒न्धेन॑ य॒क्ष्यमा॑णः । षड्ढो॑तारं॒ मन॑सा ऽनु॒द्रुत्या॑ हव॒नीये॑ जुहुयात् । प॒शु॒ब॒न्धमे॒व सृ॒ष्ट्वाऽऽरभ्य॒ प्र त॑नुते । ग्रहो॑ भवति । प॒शु॒ब॒न्धस्य॑ सृ॒ष्टस्य॒ धृत्यै᳚ ॥ सो॑ऽकामयत सौ॒म्य-म॑द्ध्व॒रꣳ सृ॑जे॒येति॑ । स ए॒तꣳ स॒प्तहो॑तार-मपश्यत् । तं मन॑साऽनु॒द्रुत्या॑ हव॒नीये॑ ऽजुहोत् । ततो॒ वै स सौ॒म्य-म॑द्ध्व॒र-म॑सृजत \textbf{ 11} \newline
                  \newline
                                \textbf{ TB 2.2.2.5} \newline
                  सो᳚ऽस्माथ् सृ॒ष्टोऽपा᳚क्रामत् । तं ग्रहे॑णागृह्णात् । तद्ग्रह॑स्य ग्रह॒त्वम् । दी॒क्षि॒ष्यमा॑णः । स॒प्तहो॑तारं॒ मन॑साऽनु॒द्रुत्या॑ हव॒नीये॑ जुहुयात् । सौ॒म्यमे॒वाद्ध्व॒रꣳ सृ॒ष्ट्वाऽऽरभ्य॒ प्र त॑नुते । ग्रहो॑ भवति । सौ॒म्यस्या᳚-द्ध्व॒रस्य॑ सृ॒ष्टस्य॒ धृत्यै᳚ ॥ दे॒वेभ्यो॒ वै य॒ज्ञो न प्राभ॑वत् । तमे॑ताव॒च्छः सम॑भरन्न् \textbf{ 12} \newline
                  \newline
                                \textbf{ TB 2.2.2.6} \newline
                  यथ् स॑भां॒राः । ततो॒ वै तेभ्यो॑ य॒ज्ञ्ः प्राभ॑वत् । यथ् स॑भां॒रा भव॑न्ति । य॒ज्ञ्स्य॒ प्रभू᳚त्यै ॥ आ॒ति॒त्थ्यमा॒साद्य॒ व्याच॑ष्टे । य॒ज्ञ्॒मु॒खं ॅवा आ॑ति॒थ्यम् । मु॒ख॒त ए॒व य॒ज्ञ्ꣳ स॒भृंत्य॒ प्र त॑नुते ॥ अय॑ज्ञो॒ वा ए॒षः । यो॑ऽप॒त्नीकः॑ । न प्र॒जाः प्र जा॑येरन्न् ( ) । पत्नी॒र्व्याच॑ष्टे । य॒ज्ञ्मे॒वाकः॑ । प्र॒जानां᳚ प्र॒जन॑नाय ॥ उ॒प॒सथ्सु॒ व्याच॑ष्टे । ए॒तद्वै पत्नी॑ना-मा॒यत॑नम् । स्व ए॒वैना॑ आ॒यत॒नेऽव॑ कल्पयति । \textbf{ 13} \newline
                  \newline
                                    (त॒नु॒त॒ - आ॒लभ॑मानो - ऽगृह्णा - दसृज - ताभर - ञ्जायेर॒न्थ् षट्च॑) \textbf{(A2)} \newline \newline
                \textbf{ 2.2.3     अनुवाकं   3 - चतुर्होतृपञ्चहोतृमन्त्रग्रहभागयोः पुरुषार्थ प्रयोगः} \newline
                                \textbf{ TB 2.2.3.1} \newline
                  प्र॒जाप॑ति-रकामयत॒ प्र जा॑ये॒येति॑ । स तपो॑ऽतप्यत । स त्रि॒वृतꣳ॒॒ स्तोम॑-मसृजत । तं प॑ञ्चद॒शस्तोमो॑ मद्ध्य॒त उद॑तृणत् । तौ पू᳚र्वप॒क्षश्चा॑परप॒क्षश्चा॑भवताम् । पू॒र्व॒प॒क्षं दे॒वा अन्वसृ॑ज्यन्त । अ॒प॒र॒प॒क्ष-मन्वसु॑राः । ततो॑ दे॒वा अभ॑वन्न् । पराऽसु॑राः । यं का॒मये॑त॒ वसी॑यान्थ् स्या॒दिति॑ \textbf{ 14} \newline
                  \newline
                                \textbf{ TB 2.2.3.2} \newline
                  तं पू᳚र्वप॒क्षे या॑जयेत् । वसी॑याने॒व भ॑वति । यं का॒मये॑त॒ पापी॑यान्थ्-स्या॒दिति॑ । तम॑परप॒क्षे या॑जयेत् । पापी॑याने॒व भ॑वति । तस्मा᳚त् पूर्वप॒क्षो॑ ऽपरप॒क्षात्-क॑रु॒ण्य॑तरः ॥ प्र॒जाप॑ति॒र्वै दश॑होता । चतु॑र्.होता॒ पञ्च॑होता । षड्ढो॑ता स॒प्तहो॑ता । ऋ॒तवः॑ सम्ॅवथ्स॒रः \textbf{ 15} \newline
                  \newline
                                \textbf{ TB 2.2.3.3} \newline
                  प्र॒जाः प॒शव॑ इ॒मे लो॒काः । य ए॒वं प्र॒जाप॑तिं ब॒होर्भूयाꣳ॑सं॒ ॅवेद॑ । ब॒होरे॒व भूया᳚न् भवति ॥ प्र॒जाप॑तिर्-देवासु॒रान॑सृजत । स इन्द्र॒मपि॒ नासृ॑जत । तं दे॒वा अ॑ब्रुवन्न् । इन्द्रं॑ नो जन॒येति॑ । सो᳚ऽब्रवीत् । यथा॒ ऽहं ॅयु॒ष्माꣳ स्तप॒सा ऽसृ॑क्षि । ए॒वमिन्द्रं॑ जनयद्ध्व॒मिति॑ \textbf{ 16} \newline
                  \newline
                                \textbf{ TB 2.2.3.4} \newline
                  ते तपो॑ऽतप्यन्त । त आ॒त्मन्निन्द्र॑-मपश्यन्न् । तम॑ब्रुवन्न् । जाय॒स्वेति॑ । सो᳚ऽब्रवीत् ।किं भा॑ग॒धेय॑-म॒भिज॑निष्य॒ इति॑ । ऋ॒तून्थ् स॑म्ॅवथ्स॒रम् । प्र॒जाः प॒शून् । इ॒माल् ॅलो॒कानित्य॑ब्रुवन्न् । तं ॅवै माऽऽहु॑त्या॒ प्रज॑नय॒तेत्य॑ब्रवीत् \textbf{ 17} \newline
                  \newline
                                \textbf{ TB 2.2.3.5} \newline
                  तं चतु॑र्.होत्रा॒ प्राज॑नयन्न् ॥ यः का॒मये॑त वी॒रो म॒ आ जा॑ये॒तेति॑ । स चतु॑र्.होतारं जुहुयात् । प्र॒जाप॑ति॒र्वै चतु॑र्.होता । प्र॒जाप॑तिरे॒व भू॒त्वा प्रजा॑यते । ज॒जन॒दिन्द्र॑-मिन्द्रि॒याय॒ स्वाहेति॒ ग्रहे॑ण जुहोति । आऽस्य॑ वी॒रो जा॑यते । वी॒रꣳ हि दे॒वा ए॒तया ऽऽहु॑त्या॒ प्राज॑नयन्न् ॥ आ॒दि॒त्याश्चाङ्गि॑रसश्च सुव॒र्गे लो॒के᳚ ऽस्पर्द्धन्त । व॒यं पूर्वे॑ सुव॒र्गं ॅलो॒कमि॑याम व॒यं पूर्व॒ इति॑ \textbf{ 18} \newline
                  \newline
                                \textbf{ TB 2.2.3.6} \newline
                  त आ॑दि॒त्या ए॒तं पञ्च॑होतार-मपश्यन्न् । तं पु॒रा प्रा॑तरनुवा॒कादाग्नी᳚द्ध्रे ऽजुहवुः । ततो॒ वै ते पूर्वे॑ सुव॒र्गं ॅलो॒कमा॑यन्न् । यः सु॑व॒र्गका॑मः॒ स्यात् । स पञ्च॑होतारं पु॒रा प्रा॑तरनुवा॒कादाग्नी᳚द्ध्रे जुहुयात् । स॒म्ॅव॒थ्स॒रो वै पञ्च॑होता । स॒म्ॅव॒थ्स॒रः सु॑व॒र्गो लो॒कः । स॒म्ॅव॒थ्स॒र ए॒वर्तुषु॑ प्रति॒ष्ठाय॑ । सु॒व॒र्गं ॅलो॒कमे॑ति ॥ ते᳚ब्रुव॒न्-नङ्गि॑रस आदि॒त्यान् \textbf{ 19} \newline
                  \newline
                                \textbf{ TB 2.2.3.7} \newline
                  क्व॑ स्थ । क्व॑ वः स॒द्भ्यो ह॒व्यं ॅव॑क्ष्याम॒ इति॑ । छन्दः॒ स्वित्य॑ब्रुवन्न् । गा॒य॒त्रि॒यां त्रि॒ष्टुभि॒-जग॑त्या॒मिति॑ । तस्मा॒च्छन्दः॑ सु स॒द्भ्य आ॑दि॒त्येभ्यः॑ । आ॒ङ्गी॒र॒सीः प्र॒जा ह॒व्यं ॅव॑हन्ति । वह॑न्त्यस्मै प्र॒जा ब॒लिम् । ऐन॒मप्र॑तिख्यातं गच्छति । य ए॒वं ॅवेद॑ ॥ द्वाद॑श॒ मासाः॒ पञ्च॒र्तवः॑ ( ) । त्रय॑ इ॒मे लो॒काः । अ॒सावा॑दि॒त्य ए॑कविꣳ॒॒शः । ए॒तस्मि॒न् वा ए॒ष श्रि॒तः । ए॒तस्मि॒न्-प्रति॑ष्ठितः । य ए॒वमे॒तꣳ श्रि॒तं प्रति॑ष्ठितं॒ ॅवेद॑ । प्रत्ये॒व ति॑ष्ठति । \textbf{ 20} \newline
                  \newline
                                    (स्या॒दिति॑ - सम्ॅवथ्स॒रो - ज॑नयद्ध्व॒मिती - त्य॑ब्रवी॒त् - पूर्व॒ इत्या॑ - दि॒त्यान् - ऋ॒तवः॒ षट्च॑) \textbf{(A3)} \newline \newline
                \textbf{ 2.2.4     अनुवाकं   4 - जगथ् सृष्टिकथनमुखेन होतृमन्त्रप्रशंसा} \newline
                                \textbf{ TB 2.2.4.1} \newline
                  प्र॒जाप॑ति-रकामयत॒ प्रजा॑ये॒येति॑ । स ए॒तं दश॑होतार-मपश्यत् । तेन॑ दश॒धा ऽऽत्मानं॑ ॅवि॒धाय॑ । दश॑होत्रा ऽतप्यत । तस्य॒ चित्तिः॒ स्रुगासी᳚त् । चि॒त्तमाज्य᳚म् । तस्यै॒ता॑वत्ये॒व वागासी᳚त् । ए॒तावान्॑. यज्ञ्क्र॒तुः ॥ स चतु॑र्.होतार-मसृजत । सो॑ऽनन्दत् \textbf{ 21} \newline
                  \newline
                                \textbf{ TB 2.2.4.2} \newline
                  असृ॑क्षि॒ वा इ॒ममिति॑ । तस्य॒ सोमो॑ ह॒विरासी᳚त् । स चतु॑र्.होत्रा ऽतप्यत । सो॑ऽताम्यत् । स भूरिति॒ व्याह॑रत् । स भूमि॑मसृजत । अ॒ग्नि॒हो॒त्रं द॑र्.शपूर्णमा॒सौ यजूꣳ॑षि । स द्वि॒तीय॑मतप्यत । सो॑ऽताम्यत् । स भुव॒ इति॒ व्याह॑रत् \textbf{ 22} \newline
                  \newline
                                \textbf{ TB 2.2.4.3} \newline
                  सो᳚ऽन्तरि॑क्ष-मसृजत । चा॒तु॒र्मा॒स्यानि॒ सामा॑नि । स तृ॒तीय॑मतप्यत । सो॑ऽताम्यत् । स सुव॒रिति॒ व्याह॑रत् । स दिव॑मसृजत । अ॒ग्नि॒ष्टो॒म-मु॒क्थ्य॑-मतिरा॒त्रमृचः॑ । ए॒ता वै व्याहृ॑तय इ॒मे लो॒काः । इ॒मान् खलु॒ वै लो॒काननु॑ प्र॒जाः प॒शव॒श्छन्दाꣳ॑सि॒ प्राजा॑यन्त । य ए॒वमे॒ताः प्र॒जाप॑तेः प्रथ॒मा व्याहृ॑तीः॒ प्रजा॑ता॒ वेद॑ \textbf{ 23} \newline
                  \newline
                                \textbf{ TB 2.2.4.4} \newline
                  प्र प्र॒जया॑ प॒शुभि॑र्-मिथु॒नैर्-जा॑यते ॥ स पञ्च॑होतार-मसृजत । स ह॒विर्ना-वि॑न्दत । तस्मै॒ सोम॑-स्त॒नुवं॒ प्राय॑च्छत् । ए॒तत् ते॑ ह॒विरिति॑ । स पञ्च॑होत्रा ऽतप्यत । सो॑ऽताम्यत् । स प्र॒त्यङ्ङ॑बाधत । सोऽसु॑रा-नसृजत । तद॒स्या प्रि॑यमासीत् \textbf{ 24} \newline
                  \newline
                                \textbf{ TB 2.2.4.5} \newline
                  तद्-दु॒र्वर्णꣳ॒॒ हिर॑ण्यमभवत् । तद्-दु॒र्वर्ण॑स्य॒ हिर॑ण्यस्य॒ जन्म॑ । स द्वि॒तीय॑मतप्यत । सो॑ऽताम्यत् । स प्राङ॑बाधत । स दे॒वान॑सृजत । तद॑स्य प्रि॒यम॑सीत् । तथ् सु॒वर्णꣳ॒॒ हिर॑ण्यमभवत् । तथ् सु॒वर्ण॑स्य॒ हिर॑ण्यस्य॒ जन्म॑ । य ए॒वꣳ सु॒वर्ण॑स्य॒ हिर॑ण्यस्य॒ जन्म॒ वेद॑ \textbf{ 25} \newline
                  \newline
                                \textbf{ TB 2.2.4.6} \newline
                  सु॒वर्ण॑ आ॒त्मना॑ भवति । दु॒र्वर्णो᳚ऽस्य॒ भ्रातृ॑व्यः । तस्मा᳚थ् सु॒वर्णꣳ॒॒ हिर॑ण्यं भा॒र्य᳚म् । सु॒वर्ण॑ ए॒व भ॑वति । ऐनं॑ प्रि॒यं ग॑च्छति॒ नाप्रि॑यम् ॥ स स॒प्तहो॑तार-मसृजत । स स॒प्तहो᳚त्रै॒व सु॑व॒र्गं ॅलो॒कमै᳚त् ॥ त्रि॒ण॒वेन॒ स्तोमे॑नै॒भ्यो लो॒केभ्योऽसु॑रा॒न् प्राणु॑दत । त्र॒य॒स्त्रिꣳ॒॒शेन॒ प्रत्य॑तिष्ठत् । ए॒क॒विꣳ॒॒शेन॒ रुच॑मधत्त \textbf{ 26} \newline
                  \newline
                                \textbf{ TB 2.2.4.7} \newline
                  स॒प्त॒द॒शेन॒ प्राजा॑यत । य ए॒वं ॅवि॒द्वान्थ् सोमे॑न॒ यज॑ते । स॒प्तहो᳚त्रै॒व सु॑व॒र्गं ॅलो॒कमे॑ति । त्रि॒ण॒वेन॒ स्तोमे॑नै॒भ्यो लो॒केभ्यो॒ भ्रातृ॑व्या॒न्-प्रणु॑दते । त्र॒य॒स्त्रिꣳ॒॒शेन॒ प्रति॑ तिष्ठति । ए॒क॒विꣳ॒॒शेन॒ रुचं॑ धत्ते । स॒प्त॒द॒शेन॒ प्रजा॑यते । तस्मा᳚थ्-सप्तद॒शः स्तोमो॒ न नि॒र्॒.हृत्यः॑ । प्र॒जाप॑ति॒ र्वै स॑प्तद॒शः । प्र॒जाप॑तिमे॒व म॑द्ध्य॒तो ध॑त्ते॒ प्रजा᳚त्यै ( ) । \textbf{ 27} \newline
                  \newline
                                    (अ॒न॒न्द॒द् - भुव॒ इति॒ व्याह॑र॒द् - वेदा॑ - सी॒द् - वेदा॑ - धत्त॒ - प्रजा᳚त्यै) \textbf{(A4)} \newline \newline
                \textbf{ 2.2.5     अनुवाकं   5 - दक्षिणाप्रतिग्रहमन्त्रव्याख्यानम्} \newline
                                \textbf{ TB 2.2.5.1} \newline
                  दे॒वा वै वरु॑णमयाजयन्न् । स यस्यै॑ यस्यै दे॒वता॑यै॒ दक्षि॑णा॒मन॑यत् । ताम॑व्लीनात् । ते᳚ऽब्रुवन्न् । व्या॒वृत्य॒ प्रति॑ गृह्णाम । तथा॑ नो॒ दक्षि॑णा॒ न व्ले᳚ष्य॒तीति॑ । ते व्या॒वृत्य॒ प्रत्य॑गृह्णन्न् । ततो॒ वै तां दक्षि॑णा॒ नाव्ली॑नात् । य ए॒वं ॅवि॒द्वान् व्या॒वृत्य॒ दक्षि॑णां प्रतिगृ॒ह्णाति॑ । नैनं॒ दक्षि॑णा व्लीनाति । \textbf{ 28} \newline
                  \newline
                                \textbf{ TB 2.2.5.2} \newline
                  राजा᳚ त्वा॒ वरु॑णो नयतु देवि दक्षिणे॒ऽग्नये॒ हिर॑ण्य॒मित्या॑ह । आ॒ग्ने॒यं ॅवै हिर॑ण्यम् । स्वयै॒वैन॑द्-दे॒वत॑या॒ प्रति॑ गृह्णाति । सोमा॑य॒ वास॒ इत्या॑ह । सौ॒म्यं ॅवै वासः॑ । स्वयै॒वैन॑द्-दे॒वत॑या॒ प्रति॑ गृह्णाति । रु॒द्राय॒ गामित्या॑ह । रौ॒द्री वै गौः । स्वयै॒वैनां᳚ दे॒वत॑या॒ प्रति॑ गृह्णाति । वरु॑णा॒याश्व॒मित्या॑ह \textbf{ 29} \newline
                  \newline
                                \textbf{ TB 2.2.5.3} \newline
                  वा॒रु॒णो वा अश्वः॑ । स्वयै॒वैनं॑ दे॒वत॑या॒ प्रति॑ गृह्णाति । प्रा॒जाप॑तये॒ पुरु॑ष॒मित्या॑ह । प्रा॒जा॒प॒त्यो वै पुरु॑षः । स्वयै॒वैनं॑ दे॒वत॑या॒ प्रति॑ गृह्णाति । मन॑वे॒ तल्प॒मित्या॑ह । मा॒न॒वो वै तल्पः॑ । स्वयै॒वैनं॑ दे॒वत॑या॒ प्रति॑ गृह्णाति । उ॒त्ता॒नाया᳚ङ्गीर॒सायान॒ इत्या॑ह । इ॒यं ॅवा उ॑त्ता॒न आ᳚ङ्गीर॒सः \textbf{ 30} \newline
                  \newline
                                \textbf{ TB 2.2.5.4} \newline
                  अ॒नयै॒वैन॒त्-प्रति॑ गृह्णाति ॥ वै॒श्वा॒न॒र्यार्चा रथं॒ प्रति॑ गृह्णाति । वै॒श्वा॒न॒रो वै दे॒वत॑या॒ रथः॑ । स्वयै॒वैनं॑ दे॒वत॑या॒ प्रति॑ गृह्णाति ॥ तेना॑मृत॒त्वम॑श्या॒-मित्या॑ह । अ॒मृत॑मे॒वात्मन्-ध॑त्ते ॥ वयो॑ दा॒त्र इत्या॑ह । वय॑ ए॒वैनं॑ कृ॒त्वा । सु॒व॒र्गं ॅलो॒कं ग॑मयति ॥ मयो॒ मह्य॑मस्तु प्रतिग्रही॒त्र इत्या॑ह \textbf{ 31} \newline
                  \newline
                                \textbf{ TB 2.2.5.5} \newline
                  यद्वै शि॒वम् । तन्मयः॑ । आ॒त्मन॑ ए॒वैषा परी᳚त्तिः ॥ क इ॒दं कस्मा॑ अदा॒दित्या॑ह । प्र॒जाप॑ति॒र्वै कः । स प्र॒जाप॑तये ददाति ॥ कामः॒ कामा॒येत्या॑ह । कामे॑न॒ हि ददा॑ति । कामे॑न प्रतिगृ॒ह्णाति॑ ॥ कामो॑ दा॒ता कामः॑ प्रति ग्रही॒तेत्या॑ह \textbf{ 32} \newline
                  \newline
                                \textbf{ TB 2.2.5.6} \newline
                  कामो॒ हि दा॒ता । कामः॑ प्रति ग्रही॒ता ॥ कामꣳ॑ समु॒द्र-मावि॒शेत्या॑ह । स॒मु॒द्र इ॑व॒ हि कामः॑ । नेव॒ हि काम॒स्यान्तोऽस्ति॑ । न स॑मु॒द्रस्य॑ ॥ कामे॑न त्वा॒ प्रति॑ गृह्णा॒मीत्या॑ह । येन॒ कामे॑न प्रति गृ॒ह्णाति॑ । स ए॒वैन॑म॒मुष्मि॑ल् ॅलो॒के काम॒ आग॑च्छति ॥ कामै॒तत्त॑ ए॒षा ते॑ काम॒ दक्षि॒णेत्या॑ह ( ) । काम॑ ए॒व तद्-यज॑मानो॒ऽमुष्मि॑ल् ॅलो॒के दक्षि॑णामिच्छति । न प्र॑तिग्रही॒तरि॑ ॥ य ए॒वं ॅवि॒द्वान्-दक्षि॑णां प्रतिगृ॒ह्णाति॑ । अ॒नृ॒णामे॒वैनां॒ प्रति॑ गृह्णाति । \textbf{ 33} \newline
                  \newline
                                    (व्ली॒ना॒त्य - श्व॒मित्या॑हा - ङ्गिर॒सः - प्र॑तिग्रही॒त्र इत्या॑ह - प्रतिग्रही॒तेत्या॑ह॒ - दक्षि॒णेत्या॑ह च॒त्वारि॑ च) \textbf{(A5)} \newline \newline
                \textbf{ 2.2.6     अनुवाकं   6 - द्वादशाहदशमेऽहनि होतृमन्त्राः} \newline
                                \textbf{ TB 2.2.6.1} \newline
                  अन्तो॒ वा ए॒ष य॒ज्ञ्स्य॑ । यद्-द॑श॒ममहः॑ । द॒श॒मेऽह᳚न्थ्-सर्परा॒ज्ञिया॑ ऋ॒ग्भिः स्तु॑वन्ति । य॒ज्ञ्स्यै॒वान्तं॑ ग॒त्वा । अ॒न्नाद्य॒मव॑रुन्धते ॥ ति॒सृभिः॑ स्तुवन्ति । त्रय॑ इ॒मे लो॒काः । ए॒भ्य ए॒व लो॒केभ्यो॒ऽन्नाद्य॒मव॑रुन्धते । पृश्नि॑वतीर्-भवन्ति । अन्नं॒ ॅवै पृश्नि॑ \textbf{ 34} \newline
                  \newline
                                \textbf{ TB 2.2.6.2} \newline
                  अन्न॑मे॒वाव॑-रुन्धते ॥ मन॑सा॒ प्रस्तौ॑ति । मन॒सोद्गा॑यति । मन॑सा॒ प्रति॑ हरति । मन॑ इव॒ हि प्र॒जाप॑तिः । प्र॒जाप॑ते॒राप्त्यै᳚ ॥ दे॒वा वै स॒र्पाः । तेषा॑मि॒यꣳ राज्ञी᳚ । यथ्-स॑र्परा॒ज्ञिया॑ ऋ॒ग्भिः स्तु॒वन्ति॑ । अ॒स्यामे॒व प्रति॑ तिष्ठन्ति । \textbf{ 35} \newline
                  \newline
                                \textbf{ TB 2.2.6.3} \newline
                  चतु॑र्.होतॄ॒न्॒. होता॒ व्याच॑ष्टे । स्तु॒तमनु॑शꣳसति॒ शान्त्यै᳚ ॥ अन्तो॒ वा ए॒ष य॒ज्ञ्स्य॑ । यद्-द॑श॒ममहः॑ । ए॒तत् खलु॒ वै दे॒वानां᳚ पर॒मं गुह्यं॒ ब्रह्म॑ । यच्चतु॑र्.होतारः । द॒श॒मेऽहꣳ॒॒-श्चतु॑र्.होतॄ॒न् व्याच॑ष्टे । य॒ज्ञ्स्यै॒वान्तं॑ ग॒त्वा । प॒र॒मं दे॒वानां॒ गुह्यं॒ ब्रह्माव॑रुन्धे । तदे॒व प्र॑का॒शं ग॑मयति \textbf{ 36} \newline
                  \newline
                                \textbf{ TB 2.2.6.4} \newline
                  तदे॑नं प्रका॒शं ग॒तम् । प्र॒का॒शं प्र॒जानां᳚ गमयति ॥ वाचं॑ ॅयच्छति । य॒ज्ञ्स्य॒ धृत्यै᳚ ॥ य॒ज॒मा॒न॒दे॒व॒त्यं॑ ॅवा अहः॑ । भ्रा॒तृ॒व्य॒दे॒व॒त्या॑ रात्रिः॑ । अह्ना॒ रात्रिं॑ ध्यायेत् । भ्रातृ॑व्यस्यै॒व तल्लो॒कं ॅवृ॑ङ्क्ते । यद्दिवा॒ वाचं॑ ॅविसृ॒जेत् । अह॒र्भ्रातृ॑व्या॒योच्छिꣳ॑षेत् ( ) । यन्नक्तं॑ ॅविसृ॒जेत् । रात्रिं॒ भ्रातृ॑व्या॒योच्छिꣳ॑षेत् । अ॒धि॒वृ॒क्ष॒सू॒र्ये वाचं॒ ॅविसृ॑जति । ए॒ताव॑न्तमे॒वास्मै॑ लो॒कमुच्छिꣳ॑षति । याव॑दादि॒त्यो᳚ऽस्त॒मेति॑ । \textbf{ 37} \newline
                  \newline
                                    (पृश्नि॑ - तिष्ठन्ति - गमयति - शिꣳषे॒त् पञ्च॑ च) \textbf{(A6)} \newline \newline
                \textbf{ 2.2.7     अनुवाकं   7 - सप्तहोतृमन्त्रसाध्ययज्ञ्प्रशंसा} \newline
                                \textbf{ TB 2.2.7.1} \newline
                  प्र॒जाप॑तिः प्र॒जा अ॑सृजत । ताः सृ॒ष्टाः सम॑श्लिष्यन्न् । ता रू॒पेणानु॒ प्रावि॑शत् । तस्मा॑दाहुः । रू॒पं ॅवै प्र॒जाप॑ति॒रिति॑ । ता नाम्नाऽनु॒ प्रावि॑शत् । तस्मा॑दाहुः । नाम॒ वै प्र॒जाप॑ति॒रिति॑ । तस्मा॒दप्या॑ ऽमि॒त्रौ स॒गंत्य॑ । नाम्ना॒ चेद्ध्वये॑ते \textbf{ 38} \newline
                  \newline
                                \textbf{ TB 2.2.7.2} \newline
                  मि॒त्रमे॒व भ॑वतः ॥ प्र॒जाप॑तिर् देवासु॒रान॑सृजत । स इन्द्र॒मपि॒ नासृ॑जत । तं दे॒वा अ॑ब्रुवन्न् । इन्द्रं॑ नो जन॒येति॑ । स आ॒त्म-न्निन्द्र॑-मपश्यत् । तम॑सृजत । तं त्रि॒ष्टुग्-वी॒र्यं॑ भू॒त्वाऽनु॒ प्रावि॑शत् । तस्य॒ वज्रः॑ पञ्चद॒शो हस्त॒ आप॑द्यत । तेनो॒दय्यासु॑रान॒भ्य॑ भवत् । \textbf{ 39} \newline
                  \newline
                                \textbf{ TB 2.2.7.3} \newline
                  य ए॒वं ॅवेद॑ । अ॒भि भ्रातृ॑व्यान् भवति ॥ ते दे॒वा असु॑रैर्-वि॒जित्य॑ । सु॒व॒र्गं ॅलो॒कमा॑यन्न् । ते॑ऽमुष्मि॑ल् ॅलो॒के व्य॑क्षुद्ध्यन्न् । ते᳚ऽब्रुवन्न् । अ॒मुतः॑ प्रदानं॒ ॅवा उप॑जिजीवि॒मेति॑ । ते स॒प्तहो॑तारं ॅय॒ज्ञ्ं ॅवि॒धाया॒यास्य᳚म् । आ॒ङ्गी॒र॒सं प्राहि॑ण्वन्न् । ए॒तेना॒मुत्र॑ कल्प॒येति॑ \textbf{ 40} \newline
                  \newline
                                \textbf{ TB 2.2.7.4} \newline
                  तस्य॒ वा इ॒यं क्लृप्तिः॑ । यदि॒दं किं च॑ । य ए॒वं ॅवेद॑ । कल्प॑तेऽस्मै । स वा अ॒यं म॑नु॒ष्ये॑षु य॒ज्ञ्ः स॒प्तहो॑ता । अ॒मुत्र॑ स॒द्भ्यो दे॒वेभ्यो॑ ह॒व्यं ॅव॑हति । य ए॒वं ॅवेद॑ । उपै॑नं ॅय॒ज्ञो न॑मति ॥ सो॑ऽमन्यत । अ॒भि वा इ॒मे᳚ऽस्मा-ल्लो॒काद॒मुंॅलो॒कं क॑मिष्यन्त॒ इति॑ ( ) । स वाच॑स्पते॒ हृदिति॒ व्याह॑रत् । तस्मा᳚त् पु॒त्रो हृद॑यम् । तस्मा॑द॒स्मा-ल्लो॒काद॒मुं ॅलो॒कं नाभिका॑मयन्ते । पु॒त्रो हि हृद॑यम् । \textbf{ 41} \newline
                  \newline
                                    (ह्वये॑त - अभवत् - कल्प॒ये - तीति॑ च॒त्वारि॑ च) \textbf{(A7)} \newline \newline
                \textbf{ 2.2.8     अनुवाकं   8 - होतृमन्त्राणां सोमयागाङ्गत्वम्} \newline
                                \textbf{ TB 2.2.8.1} \newline
                  दे॒वा वै चतु॑र्.होतृभिर्-य॒ज्ञ्म॑तन्वत । ते वि पा॒प्मना॒ भ्रातृ॑व्ये॒णाज॑यन्त । अ॒भि सु॑व॒र्गं ॅलो॒कम॑जयन्न् । य ए॒वं ॅवि॒द्वाꣳ-श्चतु॑र्.होतृभिर्य॒ज्ञ्ं त॑नु॒ते । वि पा॒प्मना॒ भ्रातृ॑व्येण जयते । अ॒भि सु॑व॒र्गं ॅलो॒कं ज॑यति ॥ षड्ढो᳚त्रा प्राय॒णीय॒मासा॑दयति । अ॒मुष्मै॒ वै लो॒काय॒ षड्ढो॑ता । घ्नन्ति॒ खलु॒ वा ए॒तथ् सोम᳚म् । यद॑भिषु॒ण्वन्ति॑ \textbf{ 42} \newline
                  \newline
                                \textbf{ TB 2.2.8.2} \newline
                  ऋ॒जु॒धैवैन॑म॒मुं ॅलो॒कं ग॑मयति । चतु॑र्.होत्रा ऽऽति॒त्थ्यम् । यशो॒ वै चतु॑र्.होता । यश॑ ए॒वात्मन्-ध॑त्ते । पञ्च॑होत्रा प॒शुमुप॑ सादयति । सु॒व॒र्ग्यो॑ वै पञ्च॑होता । यज॑मानः प॒शुः । यज॑मानमे॒व सु॑व॒र्गं ॅलो॒कं ग॑मयति । ग्रहा᳚न्-गृही॒त्वा स॒प्तहो॑तारं जुहोति । इ॒न्द्रि॒यं ॅवै स॒प्तहो॑ता \textbf{ 43} \newline
                  \newline
                                \textbf{ TB 2.2.8.3} \newline
                  इ॒न्द्रि॒यमे॒वात्मन्-ध॑त्ते ॥ यो वै चतु॑र्.होतॄ-ननुसव॒नं त॒र्पय॑ति । तृप्य॑ति प्र॒जया॑ प॒शुभिः॑ । उपै॑नꣳ सोमपी॒थो न॑मति । ब॒हि॒ष्पव॒मा॒ने दश॑होतारं॒ ॅव्याच॑क्षीत । माद्ध्य॑न्दिने॒ पव॑माने॒ चतु॑र्.होतारम् । आर्भ॑वे॒ पव॑माने॒ पञ्च॑होतारम् । पि॒तृ॒य॒ज्ञे षड्ढो॑तारम् । य॒ज्ञा॒य॒ज्ञिय॑स्य स्तो॒त्रे स॒प्तहो॑तारम् । अ॒नु॒स॒व॒नमे॒वैनाꣳ॑ स्तर्पयति \textbf{ 44} \newline
                  \newline
                                \textbf{ TB 2.2.8.4} \newline
                  तृप्य॑ति प्र॒जया॑ प॒शुभिः॑ । उपै॑नꣳ सोमपी॒थो न॑मति ॥ दे॒वा वै चतु॑र्.होतृभिः स॒त्रमा॑सत । ऋद्धि॑परिमितं॒ ॅयश॑स्कामाः । ते᳚ऽब्रुवन्न् । यन्नः॑ प्रथ॒मं ॅयश॑ ऋ॒च्छात् । सर्वे॑षां न॒स्तथ्-स॒हास॒दिति॑ । सोम॒श्चतु॑र्.होत्रा । अ॒ग्निः पञ्च॑होत्रा । धा॒ता षड्ढो᳚त्रा \textbf{ 45} \newline
                  \newline
                                \textbf{ TB 2.2.8.5} \newline
                  इन्द्रः॑ स॒प्तहो᳚त्रा । प्र॒जाप॑ति॒र्-दश॑होत्रा । तेषाꣳ॒॒ सोमꣳ॒॒ राजा॑नं॒ ॅयश॑ आर्च्छत् ॥ तन्न्य॑कामयत । तेनापा᳚क्रामत् । तेन॑ प्र॒लाय॑मचरत् । तं दे॒वाः प्रै॒षैः प्रैष॑मैच्छन्न् । तत्-प्रै॒षाणां᳚ प्रैष॒त्वम् । नि॒विद्भि॒र्न्य॑वेदयन्न् । तन्नि॒विदां᳚ निवि॒त्त्वम् \textbf{ 46} \newline
                  \newline
                                \textbf{ TB 2.2.8.6} \newline
                  आ॒प्रीभि॑राप्नुवन्न् । तदा॒प्रीणा॑माप्रि॒त्वम् । तम॑घ्नन्न् । तस्य॒ यशो॒ व्य॑गृह्णत । ते ग्रहा॑ अभवन्न् । तद्ग्रहा॑णां ग्रह॒त्वम् । यस्यै॒वं ॅवि॒दुषो॒ ग्रहा॑ गृ॒ह्यन्ते᳚ । तस्य॒ त्वे॑व गृ॑ही॒ताः ॥ ते᳚ऽब्रुवन्न् । यो वै नः॒ श्रेष्ठोऽभू᳚त् \textbf{ 47} \newline
                  \newline
                                \textbf{ TB 2.2.8.7} \newline
                  तम॑वधिष्म । पुन॑रि॒मꣳ सु॑वामहा॒ इति॑ । तं छन्दो॑भि-रसुवन्त । तच्छन्द॑सां छन्द॒स्त्वम् । साम्ना॒ समान॑यन्न् । तथ् साम्नः॑ साम॒त्वम् । उ॒क्थै-रुद॑स्थापयन्न् । तदु॒क्थाना॑-मुक्थ॒त्वम् । य ए॒वं ॅवेद॑ । प्रत्ये॒व ति॑ष्ठति \textbf{ 48} \newline
                  \newline
                                \textbf{ TB 2.2.8.8} \newline
                  सर्व॒मायु॑रेति ॥ सोमो॒ वै यशः॑ । य ए॒वं ॅवि॒द्वान्थ्-सोम॑मा॒गच्छ॑ति । यश॑ ए॒वैन॑मृच्छति । तस्मा॑दाहुः । यश्चै॒वं ॅवेद॒ यश्च॒ न । तावु॒भौ सोम॒-माग॑च्छतः । सोमो॒ हि यशः॑ । तं त्वाऽव यश॑ ऋच्छ॒तीत्या॑हुः । यः सोमे॒ सोमं॒ प्राहेति॑ ( ) । तस्मा॒थ्-सोमे॒ सोमः॒ प्रोच्यः॑ । यश॑ ए॒वैन॑मृच्छति । \textbf{ 49} \newline
                  \newline
                                    (अ॒भि॒षु॒ण्वन्ति॑ - स॒प्तहो॑ता - तर्पयति॒ - षड्ढो᳚त्रा - निवि॒त्त्व - मभू᳚त् - तिष्ठति॒ - प्राहेति॒ द्वे च॑) \textbf{(A8)} \newline \newline
                \textbf{ 2.2.9     अनुवाकं   9 - होतृमन्त्रोत्पत्तिकथनप्रसङ्गेन जगथ् सृष्टिः} \newline
                                \textbf{ TB 2.2.9.1} \newline
                  इ॒दं ॅवा अग्रे॒ नैव किंच॒नासी᳚त् । न द्यौरा॑सीत् । न पृ॑थि॒वी । नान्तरि॑क्षम् । तदस॑दे॒व सन्मनो॑ ऽकुरुत॒ स्यामिति॑ । तद॑तप्यत । तस्मा᳚त्-तेपा॒नाद्धू॒मो॑ऽजायत । तद्-भूयो॑ऽतप्यत । तस्मा᳚त्-तेपा॒नाद॒ग्नि-र॑जायत । तद्-भूयो॑ऽतप्यत \textbf{ 50} \newline
                  \newline
                                \textbf{ TB 2.2.9.2} \newline
                  तस्मा᳚त्-तेपा॒नाज्ज्योति॑-रजायत । तद्-भूयो॑ऽतप्यत । तस्मा᳚त्-तेपा॒नाद॒र्चि-र॑जायत । तद्-भूयो॑ऽतप्यत । तस्मा᳚त्-तेपा॒नान्मरी॑चयोऽजायन्त । तद्-भूयो॑ऽतप्यत । तस्मा᳚त्-तेपा॒नादु॑दा॒रा अ॑जायन्त । तद्-भूयो॑ऽतप्यत । तद॒भ्रमि॑व॒ सम॑हन्यत ॥ तद्-व॒स्तिम॑भिनत् \textbf{ 51} \newline
                  \newline
                                \textbf{ TB 2.2.9.3} \newline
                  स स॑मु॒द्रो॑ऽभवत् । तस्मा᳚थ्-समु॒द्रस्य॒ न पि॑बन्ति । प्र॒जन॑नमिव॒ हि मन्य॑न्ते । तस्मा᳚त् प॒शोर्-जाय॑माना॒दापः॑ पु॒रस्ता᳚द्-यन्ति ॥ तद्-दश॑हो॒ताऽन्व॑सृज्यत । प्र॒जाप॑ति॒र्वै दश॑ होता ॥ य ए॒वं तप॑सो वी॒र्यं॑ ॅवि॒द्वाꣳस्तप्य॑ते । भव॑त्ये॒व ॥ तद्वा इ॒दमापः॑ सलि॒ल-मा॑सीत् । सो॑ऽरोदीत्-प्र॒जाप॑तिः \textbf{ 52} \newline
                  \newline
                                \textbf{ TB 2.2.9.4} \newline
                  स कस्मा॑ अज्ञि । यद्-य॒स्या अप्र॑तिष्ठाया॒ इति॑ । यद॒फ्स्व॑-वाप॑द्यत । सा पृ॑थि॒व्य॑भवत् । यद्व्यमृ॑ष्ट । तद॒न्तरि॑क्ष-मभवत् । यदू॒र्द्ध्व-मु॒दमृ॑ष्ट । सा द्यौर॑भवत् । यदरो॑दीत् । तद॒नयो॑ रोद॒स्त्वम् \textbf{ 53} \newline
                  \newline
                                \textbf{ TB 2.2.9.5} \newline
                  य ए॒वं ॅवेद॑ । नास्य॑ गृ॒हे रु॑दन्ति । ए॒तद्वा ए॒षां ॅलो॒कानां॒ जन्म॑ । य ए॒वमे॒षां ॅलो॒कानां॒ जन्म॒ वेद॑ । नैषु लो॒के-ष्वार्ति॒मार्च्छ॑ति । स इ॒मां प्र॑ति॒ष्ठा-म॑विन्दत ॥ स इ॒मां प्र॑ति॒ष्ठां ॅवि॒त्त्वा ऽका॑मयत॒ प्रजा॑ये॒येति॑ । स तपो॑ऽतप्यत । सो᳚ऽन्तर्वा॑नभवत् । स ज॒घना॒दसु॑रा-नसृजत \textbf{ 54} \newline
                  \newline
                                \textbf{ TB 2.2.9.6} \newline
                  तेभ्यो॑ मृ॒न्मये॒ पात्रेऽन्न॑मदुहत् । याऽस्य॒ सा त॒नूरासी᳚त् । तामपा॑हत । सा तमि॑स्राऽभवत् ॥ सो॑ऽकामयत॒ प्रजा॑ये॒येति॑ । स तपो॑ऽतप्यत । सो᳚ऽन्तर्वा॑-नभवत् । स प्र॒जन॑नादे॒व प्र॒जा अ॑सृजत । तस्मा॑दि॒मा भूयि॑ष्ठाः । प्र॒जन॑ना॒द्ध्ये॑ना॒ असृ॑जत \textbf{ 55} \newline
                  \newline
                                \textbf{ TB 2.2.9.7} \newline
                  ताभ्यो॑ दारु॒मये॒ पात्रे॒ पयो॑ऽदुहत । याऽस्य॒ सा त॒नूरासी᳚त् । तामपा॑हत । सा जोथ्स्ना॑ ऽभवत् ॥ सो॑ऽकामयत॒ प्रजा॑ये॒येति॑ । सतपो॑ऽतप्यत । सो᳚ऽन्तर्वा॑-नभवत् । स उ॑पप॒क्षाभ्या॑-मे॒वर्तून॑सृजत । तेभ्यो॑ रज॒ते पात्रे॑ घृ॒तम॑दुहत् । याऽस्य॒ सा त॒नूरासी᳚त् \textbf{ 56} \newline
                  \newline
                                \textbf{ TB 2.2.9.8} \newline
                  तामपा॑हत । सो॑ऽहोरा॒त्रयोः᳚ स॒न्धि-र॑भवत् ॥ सो॑ऽकामयत॒ प्रजा॑ये॒येति॑ । स तपो॑ऽतप्यत । सो᳚ऽन्तर्वा॑-नभवत् । स मुखा᳚द्-दे॒वा-न॑सृजत । तेभ्यो॒ हरि॑ते॒ पात्रे॒ सोम॑मदुहत् । याऽस्य॒ सा त॒नूरासी᳚त् । तामपा॑हत । तदह॑रभवत् । \textbf{ 57} \newline
                  \newline
                                \textbf{ TB 2.2.9.9} \newline
                  ए॒ते वै प्र॒जाप॑ते॒र्दोहाः᳚ । य ए॒वं ॅवेद॑ । दु॒ह ए॒व प्र॒जाः ॥ दिवा॒ वै नो॑ऽभू॒दिति॑ । तद्-दे॒वानां᳚ देव॒त्वम् । य ए॒वं दे॒वानां᳚ देव॒त्वं ॅवेद॑ । दे॒ववा॑ने॒व भ॑वति ॥ ए॒तद्वा अ॑होरा॒त्राणां॒ जन्म॑ । य ए॒वम॑होरा॒त्राणां॒ जन्म॒ वेद॑ । नाहो॑रा॒त्रेष्वार्ति॒मार्च्छ॑ति । \textbf{ 58} \newline
                  \newline
                                \textbf{ TB 2.2.9.10} \newline
                  अस॒तोऽधि॒ मनो॑ऽसृज्यत । मनः॑ प्र॒जाप॑ति-मसृजत । प्र॒जाप॑तिः प्र॒जा अ॑सृजत । तद्वा इ॒दं मन॑स्ये॒व प॑र॒मं प्रति॑ष्ठितम् । यदि॒दं किंच॑ । तदे॒तच्छ्वो॑वस्य॒ संनाम॒ ब्रह्म॑ ॥ व्यु॒च्छन्ती᳚ व्युच्छन्त्यस्मै॒ वस्य॑सी वस्यसी॒ व्यु॑च्छति । प्रजा॑यते प्र॒जया॑ प॒शुभिः॑ । प्र प॑रमे॒ष्ठिनो॒ मात्रा॑-माप्नोति । य ए॒वं ॅवेद॑ ( ) । \textbf{ 59} \newline
                  \newline
                                                        \textbf{special korvai} \newline
              (इ॒दं धू॒मो᳚ ऽग्निर् ज्योति॑ र॒र्चिर् मरी॑चय उदा॒रा स्तद॒भ्रꣳ / स ज॒घना॒थ् सा तमि॑स्रा॒ स प्र॒जन॑ना॒थ् सा जोथ्स्ना॒ स उ॑पप॒क्षाभ्याꣳ॒॒ सो॑ ऽहोरा॒त्रयोः᳚ स॒न्धिः समुखा॒त् तदह॑र् दे॒ववा᳚न् / मृ॒न्मये॑ दारु॒मये॑ रज॒ते हरि॑ते॒ तेभ्य॒स्ताभ्यो॒ द्वे तेभ्योऽन्नं॒ पयो॑ घृ॒तꣳ सोम᳚म्) \newline
                                (अ॒ग्निर॑जायत॒ तद्भूयो॑ऽतप्यता - भिनद - रोदीत् प्र॒जाप॑ती - रोद॒स्त्व - म॑सृज॒ - तासृ॑जत - घृ॒तम॑दुह॒द्याऽस्य॒ सा त॒नूरासी॒ - दह॑रभवद् - ऋच्छति॒ - वेद॑) \textbf{(A9)} \newline \newline
                \textbf{ 2.2.10    अनुवाकं   10 - देवसृष्टिमध्येऽवस्थितस्येन्द्रस्य देवताधिपतित्वम्} \newline
                                \textbf{ TB 2.2.10.1} \newline
                  प्र॒जाप॑ति॒-रिन्द्र॑मसृज-तानुजाव॒रं दे॒वाना᳚म् । तं प्राहि॑णोत् । परे॑हि । ए॒तेषां᳚ दे॒वाना॒-मधि॑पतिरे॒धीति॑ । तं दे॒वा अ॑ब्रुवन्न् । कस्त्वमसि॑ । व॒यं ॅवै त्वच्छ्रेयाꣳ॑सः स्म॒ इति॑ । सो᳚ऽब्रवीत् । कस्त्वमसि॑ व॒यं ॅवै त्वच्छ्रेयाꣳ॑सः स्म॒ इति॑ मा दे॒वा अ॑वोच॒न्निति॑ ॥ अथ॒ वा इ॒दं तर्.हि॑ प्र॒जाप॑तौ॒ हर॑ आसीत् \textbf{ 60} \newline
                  \newline
                                \textbf{ TB 2.2.10.2} \newline
                  यद॒स्मिन्ना॑दि॒त्ये । तदे॑नमब्रवीत् । ए॒तन्मे॒ प्रय॑च्छ । अथा॒हमे॒तेषां᳚ दे॒वाना॒-मधि॑पतिर्-भविष्या॒मीति॑ । को॑हꣳ-स्या॒मित्य॑ब्रवीत् । ए॒तत् प्र॒दायेति॑ । ए॒तथ्-स्या॒ इत्य॑ब्रवीत् । यदे॒तद्ब्रवी॒षीति॑ । को ह॒ वै नाम॑ प्र॒जाप॑तिः । य ए॒वं ॅवेद॑ \textbf{ 61} \newline
                  \newline
                                \textbf{ TB 2.2.10.3} \newline
                  वि॒दुरे॑नं॒ नाम्ना᳚ । तद॑स्मै रु॒क्मं कृ॒त्वा प्रत्य॑मुञ्चत् । ततो॒ वा इन्द्रो॑ दे॒वाना॒-मधि॑पति-रभवत् । य ए॒वं ॅवेद॑ । अधि॑पतिरे॒व स॑मा॒नानां᳚ भवति ॥ सो॑ऽमन्यत । किं किं॒ ॅवा अ॑कर॒मिति॑ । स च॒न्द्रं म॒ आह॒रेति॒ प्राल॑पत् । तच्च॒न्द्रम॑स-श्चन्द्रम॒स्त्वम् । य ए॒वं ॅवेद॑ \textbf{ 62} \newline
                  \newline
                                \textbf{ TB 2.2.10.4} \newline
                  च॒न्द्रवा॑ने॒व भ॑वति ॥ तं दे॒वा अ॑ब्रुवन्न् । सु॒वीर्यो॑ ऽमर्या॒ यथा॑ गोपा॒यत॒ इति॑ । तथ् सूर्य॑स्य सूर्य॒त्वम् । य ए॒वं ॅवेद॑ । नैनं॑ दभ्नोति ॥ कश्च॒ नास्मि॒न्वा इ॒दमि॑न्द्रि॒यं प्रत्य॑स्था॒दिति॑ । तदिन्द्र॑-स्येन्द्र॒त्वम् । य ए॒वं ॅवेद॑ । इ॒न्द्रि॒या॒व्ये॑व भ॑वति । \textbf{ 63} \newline
                  \newline
                                \textbf{ TB 2.2.10.5} \newline
                  अ॒यं ॅवा इ॒दं प॑र॒मो॑ऽभू॒दिति॑ । तत्-प॑रमे॒ष्ठिनः॑ परमेष्ठि॒त्वम् । य ए॒वं ॅवेद॑ । प॒र॒मामे॒व काष्ठां᳚ गच्छति ॥ तं दे॒वाः स॑म॒न्तं पर्य॑विशन्न् । वस॑वः पु॒रस्ता᳚त् । रु॒द्रा द॑क्षिण॒तः । आ॒दि॒त्याः प॒श्चात् । विश्वे॑ दे॒वा उ॑त्तर॒तः । अङ्गि॑रसः प्र॒त्यञ्च᳚म् \textbf{ 64} \newline
                  \newline
                                \textbf{ TB 2.2.10.6} \newline
                  सा॒द्ध्याः परा᳚ञ्चम् । य ए॒वं ॅवेद॑ । उपै॑नꣳ समा॒नाः सं ॅवि॑शन्ति ॥ स प्र॒जाप॑तिरे॒व भू॒त्वा प्र॒जा आव॑यत् । ता अ॑स्मै॒ नाति॑ष्ठन्ता॒न्नाद्या॑य । ता मुखं॑ पु॒रस्ता॒त्-पश्य॑न्तीः । द॒क्षि॒ण॒तः पर्या॑यन्न् । स द॑क्षिण॒तः पर्य॑वर्तयत । ता मुखं॑ पु॒रस्ता॒त्-पश्य॑न्तीः । मुखं॑ दक्षिण॒तः \textbf{ 65} \newline
                  \newline
                                \textbf{ TB 2.2.10.7} \newline
                  प॒श्चात्-पर्या॑यन्न् । स प॒श्चात्-पर्य॑वर्तयत । ता मुखं॑ पु॒रस्ता॒त्-पश्य॑न्तीः । मुखं॑ दक्षिण॒तः । मुखं॑ प॒श्चात् । उ॒त्त॒र॒तः पर्या॑यन्न् । स उ॑त्तर॒तः पर्य॑वर्तयत । ता मुखं॑ पु॒रस्ता॒त्-पश्य॑न्तीः । मुखं॑ दक्षिण॒तः । मुखं॑ प॒श्चात् ( ) \textbf{ 66} \newline
                  \newline
                                \textbf{ TB 2.2.10.8} \newline
                  मुख॑मुत्तर॒तः । ऊ॒र्द्ध्वा उदा॑यन्न् । स उ॒परि॑ष्टा॒न्न्य॑-वर्तयत । ताः स॒र्वतो॑मुखो भू॒त्वा ऽऽव॑यत् । ततो॒ वै तस्मै᳚ प्र॒जा अति॑ष्ठन्ता॒न्नाद्या॑य ॥ य ए॒वं ॅवि॒द्वान् परि॑ च व॒र्तय॑ते॒ नि च॑ । प्र॒जाप॑तिरे॒व भु॒त्वा प्र॒जा अ॑त्ति । तिष्ठ॑न्तेऽस्मै प्र॒जा अ॒न्नाद्या॑य । अ॒न्ना॒द ए॒व भ॑वति । \textbf{ 67} \newline
                  \newline
                                    (आ॒सी॒द् - वेद॑ - चन्द्रम॒स्त्वं ॅय ए॒वं ॅवेदे᳚ - न्द्रिया॒व्ये॑व भ॑वति - प्र॒त्यञ्च॒ - मुखं॑ दक्षिण॒तो - मुखं॑ प॒श्चान् - +नव॑ च) \textbf{(A10)} \newline \newline
                \textbf{ 2.2.11    अनुवाकं   11 - होतृमन्त्राणां पुरुषार्थप्रयोगः} \newline
                                \textbf{ TB 2.2.11.1} \newline
                  प्र॒जाप॑ति-रकामयत ब॒होर्भूया᳚न्थ्-स्या॒मिति॑ । स ए॒तं दश॑होतार-मपश्यत् । तं प्रायु॑ङ्क्त । तस्य॒ प्रयु॑क्ति ब॒होर्भूया॑नभवत् । यः का॒मये॑त ब॒होर्भूया᳚न्थ्-स्या॒मिति॑ । स दश॑होतारं॒ प्रयु॑ञ्जीत । ब॒होरे॒व भूया᳚न्-भवति ॥ सो॑ऽकामयत वी॒रो म॒ आजा॑ये॒तेति॑ । स दश॑होतु॒श्चतु॑र्.होतारं॒ निर॑मिमीत । तं प्रायु॑ङ्क्त \textbf{ 68} \newline
                  \newline
                                \textbf{ TB 2.2.11.2} \newline
                  तस्य॒ प्रयु॒क्तीन्द्रो॑ ऽजायत । यः का॒मये॑त वी॒रो म॒ आजा॑ये॒तेति॑ । स चतु॑र्.होतारं॒ प्रयु॑ञ्जीत । आऽस्य॑ वी॒रो जा॑यते ॥ सो॑ऽकामयत पशु॒मान्थ्-स्या॒मिति॑ । स चतु॑र्.होतुः॒ पञ्च॑होतारं॒ निर॑मिमीत । तं प्रायु॑ङ्क्त । तस्य॒ प्रयु॑क्ति पशु॒मान॑भवत् । यः का॒मये॑त पशु॒मान्थ्-स्या॒मिति॑ । स पञ्च॑होतारं॒ प्रयु॑ञ्जीत \textbf{ 69} \newline
                  \newline
                                \textbf{ TB 2.2.11.3} \newline
                  प॒शु॒माने॒व भ॑वति ॥ सो॑ऽकामयत॒र्तवो॑ मे कल्पेर॒न्निति॑ । स पञ्च॑होतुः॒ षड्ढो॑तारं॒ निर॑मिमीत । तं प्रायु॑ङ्क्त । तस्य॒ प्रयु॑क्त्यृ॒तवो᳚ऽस्मा अकल्पन्त । यः का॒मये॑त॒र्तवो॑ मे कल्पेर॒न्निति॑ । स षड्ढो॑तारं॒ प्रयु॑ञ्जीत । कल्प॑न्तेऽस्मा ऋ॒तवः॑ ॥ सो॑ऽकामयत सोम॒पः सो॑मया॒जी स्यां᳚ । आ मे॑ सोम॒पः सो॑मया॒जी जा॑ये॒तेति॑ \textbf{ 70} \newline
                  \newline
                                \textbf{ TB 2.2.11.4} \newline
                  स षड्ढो॑तुः स॒प्तहो॑तारं॒ निर॑मिमीत । तं प्रायु॑ङ्क्त । तस्य॒ प्रयु॑क्ति सोम॒पः सो॑मया॒ज्य॑भवत् । आऽस्य॑ सोम॒पः सो॑मया॒ज्य॑जायत । यः का॒मये॑त सोम॒पः सो॑मया॒जी स्या᳚म् । आ मे॑ सोम॒पः सो॑मया॒जी जा॑ये॒तेति॑ । स स॒प्तहो॑तारं॒ प्रयु॑ञ्जीत । सो॒म॒प ए॒व सो॑मया॒जी भ॑वति । आऽस्य॑ सोम॒पः सो॑मया॒जी जा॑यते ॥ स वा ए॒ष प॒शुः प॑ञ्च॒धा प्रति॑ तिष्ठति \textbf{ 71} \newline
                  \newline
                                \textbf{ TB 2.2.11.5} \newline
                  प॒द्भिर्मुखे॑न ॥ ते दे॒वाः प॒शून्. वि॒त्त्वा । सु॒व॒र्गं ॅलो॒कमा॑यन्न् ॥ ते॑ऽमुष्मि॑ल् ॅलो॒के व्य॑क्षुद्ध्यन्न् । ते᳚ऽब्रुवन्न् । अ॒मुतः॑ प्रदानं॒ ॅवा उप॑जिजीवि॒मेति॑ । ते स॒प्तहो॑तारं ॅय॒ज्ञ्ं ॅवि॒धाया॒यास्यं᳚ । आ॒ङ्गी॒र॒सं प्राहि॑ण्वन्न् । ए॒तेना॒मुत्र॑ कल्प॒येति॑ । तस्य॒ वा इ॒यं क्लृप्तिः॑ \textbf{ 72} \newline
                  \newline
                                \textbf{ TB 2.2.11.6} \newline
                  यदि॒दं किञ्च॑ । य ए॒वं ॅवेद॑ । कल्प॑तेऽस्मै । स वा अ॒यं म॑नु॒ष्ये॑षु य॒ज्ञ्ः स॒प्तहो॑ता । अ॒मुत्र॑ स॒द्भ्यो दे॒वेभ्यो॑ ह॒व्यं ॅव॑हति । य ए॒वं ॅवेद॑ । उपै॑नं ॅय॒ज्ञो न॑मति ॥ यो वै चतु॑र्.होतृणां नि॒दानं॒ ॅवेद॑ । नि॒दान॑वान् भवति । अ॒ग्नि॒हो॒त्रं ॅवै दश॑होतुर्नि॒दान᳚म् ( ) । द॒र्.॒श॒पू॒र्ण॒मा॒सौ चतु॑र्.होतुः । चा॒तु॒र्मा॒स्यानि॒ पञ्च॑ होतुः । प॒शु॒ब॒न्धः षड्ढो॑तुः । सौ॒म्यो᳚ऽद्ध्व॒रः स॒प्तहो॑तुः । ए॒तद्वै चतु॑र्.होतृणां नि॒दान᳚म् । य ए॒वं ॅवेद॑ । नि॒दान॑वान् भवति । \textbf{ 73} \newline
                  \newline
                                    (अ॒मि॒मी॒त॒ तं प्रायु॑ङ्क्त॒ - पञ्च॑होतारं॒ प्रयु॑ञ्जित - जाये॒तेति॑ - तिष्ठति॒ - क्लृप्ति॒र् - दश॑होतुर् नि॒दानꣳ॑ स॒प्त च॑ ) \textbf{(A11)} \newline \newline
                \textbf{PrapAtaka Korvai with starting  words of 1 to11 anuvAkams :-} \newline
        (प्र॒जाप॑तिरकामयत प्र॒जाः सृ॑जे॒येति॑ - प्र॒जाप॑तिरकामयत दर्.शपूर्णमा॒सौ सृ॑जे॒येति॑ - प्र॒जाप॑तिरकामयत॒ प्रजा॑ये॒येति॒ स तपः॒ स त्रि॒वृतं॑ - प्र॒जाप॑तिरकामयत॒ दश॑होतारं॒ तेन॑ दश॒धाऽऽत्मानं॑ - दे॒वा वै वरु॑ण॒ - मन्तो॒ वै - प्र॒जाप॑ति॒स्ताः सृ॒ष्टाः सम॑श्लिष्यन् - दे॒वा वै चतु॑र्.होतृभि - रि॒दं ॅवा अग्रे᳚ - प्र॒जाप॑ति॒रिन्द्रं॑ - प्र॒जाप॑तिरकामयत ब॒होर्भूया॒नेका॑दश) \newline

        \textbf{korvai with starting words of 1, 11, 21 series of daSinis :-} \newline
        (प्र॒जाप॑ति॒ - स्तद्ग्रह॑स्य - प्र॒जा॑पतिरकामयता॒ - नयै॒वैन॒त् - तस्य॒ वा इ॒यं क्लृप्ति॒ - स्तस्मा᳚त् तेपा॒नाज्ज्योति॒र् - यद॒स्मिन्ना॑दि॒त्ये -सषड्ढो॑तुः स॒प्तहो॑तारं॒ त्रिस॑प्ततिः) \newline

        \textbf{first and last  word - 2nd aShTakam , 2nd prapATakam :-} \newline
        (प्र॒जाप॑तिरकामयत - नि॒दान॑वान्भवति) \newline 

       

        ॥ हरिः॑ ॐ ॥
॥ कृष्ण यजुर्वेदीय तैत्तिरीय ब्राह्मणे द्वितीयाष्टके द्वितीयः प्रपाठकः समाप्तः ॥

================== \newline
        \pagebreak
        
        
        
     \addcontentsline{toc}{section}{ 2.3     द्वितीयाष्टके तृतीयः प्रपाठकः - होतृब्राह्मणशेषः}
     \markright{ 2.3     द्वितीयाष्टके तृतीयः प्रपाठकः - होतृब्राह्मणशेषः \hfill https://www.vedavms.in \hfill}
     \section*{ 2.3     द्वितीयाष्टके तृतीयः प्रपाठकः - होतृब्राह्मणशेषः }
                \textbf{ 2.3.1    अनुवाकं   1 - सर्वेषां होतृमन्त्राणां चतुर्होतृशब्देन व्यवहार} \newline
                                \textbf{ TB 2.3.1.1} \newline
                  ब्र॒ह्म॒वा॒दिनो॑ वदन्ति । किं चतु॑र्.होतृणां चतुर्.होतृ॒त्वमिति॑ । यदे॒वैषु च॑तु॒र्द्धा होता॑रः । तेन॒ चतु॑र्.होतारः । तस्मा॒च्चतु॑र्.होतार उच्यन्ते । तच्चतु॑र्.होतृणां चतुर्.होतृ॒त्वम् ॥ सोमो॒ वै चतु॑र्.होता । अ॒ग्निः पञ्च॑होता । धा॒ता षड्ढो॑ता । इन्द्रः॑ स॒प्तहो॑ता \textbf{ 1} \newline
                  \newline
                                \textbf{ TB 2.3.1.2} \newline
                  प्र॒जाप॑ति॒र्-दश॑होता । य ए॒वं चतु॑र्.होतृणा॒-मृद्धिं॒ ॅवेद॑ । ऋ॒ध्नोत्ये॒व ॥ य ए॑षामे॒वं ब॒न्धुतां॒ ॅवेद॑ । बन्धु॑मान्भवति । य ए॑षामे॒वं क्लृप्तिं॒ ॅवेद॑ । कल्प॑तेऽस्मै । य ए॑षामे॒व-मा॒यत॑नं॒ ॅवेद॑ । आ॒यत॑नवान्-भवति । य ए॑षामे॒वं प्र॑ति॒ष्ठां ॅवेद॑ \textbf{ 2} \newline
                  \newline
                                \textbf{ TB 2.3.1.3} \newline
                  प्रत्ये॒व ति॑ष्ठति ॥ ब्र॒ह्म॒वा॒दिनो॑ वदन्ति । दश॑होता॒ चतु॑र्.होता । पञ्च॑होता॒ षड्ढो॑ता स॒प्तहो॑ता । अथ॒ कस्मा॒च्चतु॑र्.होतार उच्यन्त॒ इति॑ । इन्द्रो॒ वै चतु॑र्.होता । इन्द्रः॒ खलु॒ वै श्रेष्ठो॑ दे॒वता॑ना-मुप॒देश॑नात् । य ए॒वमिन्द्रꣳ॒॒ श्रेष्ठं॑ दे॒वता॑ना-मुप॒देश॑ना॒द्-वेद॑ । वसि॑ष्ठः समा॒नानां᳚ भवति । तस्मा॒च्छ्रेष्ठ॑मा॒यन्तं॑ प्रथ॒मेनै॒वानु॑बुद्ध्यन्ते ( ) । अ॒यमागन्न्॑ । अ॒यमवा॑सा॒दिति॑ । की॒र्तिर॑स्य॒ पूर्वाऽऽग॑च्छति ज॒नता॑माय॒तः । अथो॑ एनं प्रथ॒मेनै॒वानु॑ बुद्ध्यन्ते । अ॒यमागन्न्॑ । अ॒यमवा॑सा॒दिति॑ । \textbf{ 3} \newline
                  \newline
                                    (स॒प्तहो॑ता - प्रति॒ष्ठां वेद॑ - बुद्ध्यन्ते॒ षट्च॑) \textbf{(A1)} \newline \newline
                \textbf{ 2.3.2      अनुवाकं   2 - नैमित्तिको होतृमन्त्रहोमः} \newline
                                \textbf{ TB 2.3.2.1} \newline
                  दक्षि॑णां प्रतिग्रही॒ष्यन्थ्-स॒प्तद॑श॒ कृत्वोऽपा᳚न्यात् । आ॒त्मान॑मे॒व समि॑न्धे । तेज॑से वी॒र्या॑य । अथो᳚ प्र॒जाप॑तिरे॒वैनां᳚ भू॒त्वा प्रति॑ गृह्णाति । आ॒त्मनो ऽना᳚र्त्यै ॥ यद्ये॑न॒मार्त्-वि॑ज्याद्-वृ॒तꣳ सन्तं॑ नि॒र्.हरे॑रन्न् । आग्नी᳚द्ध्रे जुहुया॒द्-दश॑होतारम् । च॒तु॒र्गृ॒ही॒तेनाज्ये॑न । पु॒रस्ता᳚त्-प्र॒त्यङ्तिष्ठन्न्॑ । प्र॒ति॒लो॒मं ॅवि॒ग्राह᳚म् \textbf{ 4} \newline
                  \newline
                                \textbf{ TB 2.3.2.2} \newline
                  प्रा॒णाने॒वास्योप॑दासयति ॥ यद्ये॑नं॒ पुन॑रुप॒ शिक्षे॑युः । आग्नी᳚द्ध्र ए॒व जु॑हुया॒द्-दश॑होतारम् । च॒तु॒र्गृ॒ही॒तेनाज्ये॑न । प॒श्चात् प्राङासी॑नः । अ॒नु॒लो॒म-मवि॑ग्राहम् । प्रा॒णाने॒वास्मै॑ कल्पयति ॥ प्राय॑श्चित्ती॒-वाग्घोते-त्यृ॑तुमु॒ख ऋ॑तुमुखे जुहोति । ऋ॒तूने॒वास्मै॑ कल्पयति । कल्प॑न्तेऽस्मा ऋ॒तवः॑ \textbf{ 5} \newline
                  \newline
                                \textbf{ TB 2.3.2.3} \newline
                  क्लृ॒ता अ॑स्मा ऋ॒तव॒ आय॑न्ति ॥ षड्ढो॑ता॒ वै भू॒त्वा प्र॒जाप॑तिरि॒दꣳ सर्व॑मसृजत । स मनो॑ऽसृजत । मन॒सोऽधि॑ गाय॒त्रीम॑सृजत । तद्गा॑य॒त्रीं ॅयश॑ आर्च्छत् । तामाऽल॑भत । गा॒य॒त्रि॒या अधि॒ छन्दाꣳ॑स्य सृजत । छन्दो॒भ्योऽधि॒ साम॑ । तथ् साम॒ यश॑ आर्च्छत् । तदाऽल॑भत \textbf{ 6} \newline
                  \newline
                                \textbf{ TB 2.3.2.4} \newline
                  साम्नोऽधि॒ यजूꣳ॑ष्यसृजत । यजु॒र्भ्योऽधि॒ विष्णु᳚म् । तद्-विष्णुं॒ ॅयश॑ आर्च्छत् । तमाऽल॑भत । वि॒ष्णोरद्ध्योष॑धीरसृजत । ओष॑धी॒भ्योऽधि॒ सोम᳚म् । तथ् सोमं॒ ॅयश॑ आर्च्छत् । तमाऽल॑भत । सोमा॒दधि॑ प॒शून॑सृजत । प॒शुभ्योऽधीन्द्र᳚म् \textbf{ 7} \newline
                  \newline
                                \textbf{ TB 2.3.2.5} \newline
                  तदिन्द्रं॒ ॅयश॑ आर्च्छत् । तदे॑नं॒ नाति॒ प्राच्य॑वत । इन्द्र॑ इव यश॒स्वी भ॑वति । य ए॒वं ॅवेद॑ । नैनं॒ ॅयशोऽति॒ प्रच्य॑वते ॥ यद्वा इ॒दं किं च॑ । तथ् सर्व॑मुत्ता॒न ए॒वाङ्गी॑र॒सः प्रत्य॑गृह्णात् । तदे॑नं॒ प्रति॑गृहीतं॒ नाहि॑नत् । यत् किंच॑ प्रति गृह्णी॒यात् । तथ्-सर्व॑मुत्ता॒नस्त्वा᳚ङ्गीर॒सः प्रति॑गृह्णा॒त्वित्ये॒व प्रति॑गृह्णीयात् ( ) । इ॒यं ॅवा उ॑त्ता॒न आ᳚ङ्गीर॒सः । अ॒नयै॒वैन॒त्-प्रति॑ गृह्णाति । नैनꣳ॑ हिनस्ति ॥ ब॒र॒.हिषा॒ प्रती॑या॒द्गां ॅवाऽश्वं॑ ॅवा । ए॒तद्वै प॑शू॒नां प्रि॒यं धाम॑ । प्रि॒येणै॒वैनं॒ धाम्ना॒ प्रत्ये॑ति । \textbf{ 8} \newline
                  \newline
                                    (वि॒ग्राह॑ - मृ॒तव॒ - स्तदाऽल॑भ॒ - तेन्द्रं॑ - गृह्णीया॒थ् षट्च॑) \textbf{(A2)} \newline \newline
                \textbf{ 2.3.3      अनुवाकं   3 - केशवपनाङ्गं पुष्टिवेदनं} \newline
                                \textbf{ TB 2.3.3.1} \newline
                  यो वा अवि॑द्वान्-निव॒र्तय॑ते । विशी॑र्.षा॒ सपा᳚प्मा॒ ऽमुष्मि॑ल् ॅलो॒के भ॑वति । अथ॒ यो वि॒द्वान्-नि॑व॒र्तय॑ते । सशी॑र्.षा॒ विपा᳚प्मा॒ ऽमुष्मि॑ल् ॅलो॒के भ॑वति ॥ दे॒वता॒ वै स॒प्त पुष्टि॑कामा॒ न्य॑वर्तयन्त । अ॒ग्निश्च॑ पृथि॒वी च॑ । वा॒युश्चा॒न्तरि॑क्षं च । आ॒दि॒त्यश्च॒ द्यौश्च॑ च॒न्द्रमाः᳚ । अ॒ग्निर्न्य॑वर्तयत । स सा॑ह॒स्र-म॑पुष्यत् \textbf{ 9} \newline
                  \newline
                                \textbf{ TB 2.3.3.2} \newline
                  पृ॒थि॒वी न्य॑वर्तयत । सौष॑धीभि॒र्-वन॒स्पति॑भि-रपुष्यत् । वा॒युर्न्य॑वर्तयत । स मरी॑चीभि-रपुष्यत् । अ॒न्तरि॑क्षं॒ न्य॑वर्तयत । तद्वयो॑भि-रपुष्यत् । आ॒दि॒त्यो न्य॑वर्तयत । स र॒श्मिभि॑-रपुष्यत् । द्यौर्न्य॑वर्तयत । सा नक्ष॑त्रै-रपुष्यत् ( ) । च॒न्द्रमा॒ न्य॑वर्तयत । सो॑ऽहोरा॒त्रै-र॑र्द्धमा॒सैर्-मासैर्.॑ ऋ॒तुभिः॑ सम्ॅवथ्स॒रेणा॑-पुष्यत् । तान्पोषा᳚न्-पुष्यति । याꣳस्तेऽपु॑ष्यन्न् । य ए॒वं ॅवि॒दान्नि च॑ व॒र्तय॑ते॒ परि॑ च । \textbf{ 10} \newline
                  \newline
                                    (अ॒पु॒ष्य॒न् - नक्ष॑त्रैरपुष्य॒त् पञ्च॑ च) \textbf{(A3)} \newline \newline
                \textbf{ 2.3.4      अनुवाकं   4 - प्रतिग्रहमन्त्रमहिमज्ञानम्} \newline
                                \textbf{ TB 2.3.4.1} \newline
                  तस्य॒ वा अ॒ग्नेर्. हिर॑ण्यं प्रतिजग्र॒हुषः॑ । अ॒र्द्धमि॑न्द्रि॒यस्या-पा᳚क्रामत् । तदे॒तेनै॒व प्रत्य॑गृह्णात् । तेन॒ वै सो᳚ऽर्द्धमि॑न्द्रि॒य-स्या॒त्मन्-नु॒पाध॑त्त । अ॒र्द्धमि॑न्द्रि॒यस्या॒त्मन्-नु॒पाध॑त्ते । य ए॒वं ॅवि॒द्वान्. हिर॑ण्यं प्रति गृ॒ह्णाति॑ । अथ॒ यो ऽवि॑द्वान् प्रति गृ॒ह्णाति॑ । अ॒र्द्धम॑स्येन्द्रि॒यस्याप॑ क्रामति ॥ तस्य॒ वै सोम॑स्य॒ वासः॑ प्रतिजग्र॒हुषः॑ । तृती॑यमिन्द्रि॒यस्या-पा᳚क्रामत् \textbf{ 11} \newline
                  \newline
                                \textbf{ TB 2.3.4.2} \newline
                  तदे॒तेनै॒व प्रत्य॑गृह्णात् । तेन॒ वै स तृती॑यमिन्द्रि॒यस्या॒त्मन्-नु॒पाध॑त्त । तृती॑यमिन्द्रि॒यस्या॒त्मन्-नु॒पाध॑त्ते । य ए॒वं ॅवि॒द्वान्. वासः॑ प्रति गृ॒ह्णाति॑ । अथ॒ योऽवि॑द्वान् प्रति गृ॒ह्णाति॑ । तृती॑यमस्येन्द्रि॒यस्याप॑ क्रामति ॥ तस्य॒ वै रु॒द्रस्य॒ गां प्र॑तिजग्र॒हुषः॑ । च॒तु॒र्थमि॑न्द्रि॒यस्या-पा᳚क्रामत् । तामे॒तेनै॒व प्रत्य॑गृह्णात् । तेन॒ वै स च॑तु॒र्थमि॑न्द्रि॒यस्या॒त्मन्-नु॒पाध॑त्त \textbf{ 12} \newline
                  \newline
                                \textbf{ TB 2.3.4.3} \newline
                  च॒तु॒र्थमि॑न्द्रि॒यस्या॒त्मन्-नु॒पाध॑त्ते । य ए॒वं ॅवि॒द्वान् गां प्र॑ति गृ॒ह्णाति॑ । अथ॒ योऽवि॑द्वान् प्रति गृ॒ह्णाति॑ । च॒तु॒र्थम॑स्येन्द्रि॒यस्याप॑ क्रामति ॥ तस्य॒ वै वरु॑ण॒स्याश्व॑म् प्रतिजग्र॒हुषः॑ । प॒ञ्च॒ममि॑न्द्रि॒यस्या-पा᳚क्रामत् । तमे॒तेनै॒व प्रत्य॑गृह्णात् । तेन॒ वै स प॑ञ्च॒ममि॑न्द्रि॒यस्या॒त्मन्-नु॒पाध॑त्त । प॒ञ्च॒ममि॑न्द्रि॒यस्या॒त्मन्-नु॒पाध॑त्ते । य ए॒वं ॅवि॒द्वानश्वं॑ प्रति गृ॒ह्णाति॑ \textbf{ 13} \newline
                  \newline
                                \textbf{ TB 2.3.4.4} \newline
                  अथ॒ योऽवि॑द्वान् प्रतिगृ॒ह्णाति॑ । प॒ञ्च॒मम॑स्येन्द्रि॒यस्याप॑ क्रामति ॥ तस्य॒ वै प्र॒जाप॑तेः॒ पुरु॑षं प्रतिजग्र॒हुषः॑ । ष॒ष्ठमि॑न्द्रि॒यस्या-पा᳚क्रामत् । तमे॒तेनै॒व प्रत्य॑गृह्णात् । तेन॒ वै स ष॒ष्ठमि॑न्द्रि॒यस्या॒त्मन्-नु॒पाध॑त्त । ष॒ष्ठमि॑न्द्रि॒यस्या॒त्मन्-नु॒पाध॑त्ते । य ए॒वं ॅवि॒द्वान् पुरु॑षं प्रति गृ॒ह्णाति॑ । अथ॒ योऽवि॑द्वान् प्रति गृ॒ह्णाति॑ । ष॒ष्ठम॑स्येन्द्रि॒यस्याप॑-क्रामति । \textbf{ 14} \newline
                  \newline
                                \textbf{ TB 2.3.4.5} \newline
                  तस्य॒ वै मनो॒स्तल्पं॑ प्रतिजग्र॒हुषः॑ । स॒प्त॒ममि॑न्द्रि॒यस्या-पा᳚क्रामत् । तमे॒तेनै॒व प्रत्य॑गृह्णात् । तेन॒ वै स स॑प्त॒ममि॑न्द्रि॒य-स्या॒त्मन्-नु॒पाध॑त्त । स॒प्त॒ममि॑न्द्रि॒यस्या॒त्मन्-नु॒पाध॑त्ते । य ए॒वं ॅवि॒द्वाꣳस्तल्पं॑ प्रतिगृ॒ह्णाति॑ । अथ॒ योऽवि॑द्वान् प्रति गृ॒ह्णाति॑ । स॒प्त॒मम॑स्येन्द्रि॒यस्याप॑-क्रामति ॥ तस्य॒ वा उ॑त्ता॒नस्या᳚ङ्गीर॒सस्या प्रा॑णत् प्रतिजग्र॒हुषः॑ । अ॒ष्ट॒ममि॑न्द्रि॒यस्या-पा᳚क्रामत् \textbf{ 15} \newline
                  \newline
                                \textbf{ TB 2.3.4.6} \newline
                  तदे॒तेनै॒व प्रत्य॑गृह्णात् । तेन॒ वै सो᳚ऽष्ट॒ममि॑न्द्रि॒यस्या॒त्मन्-नु॒पाध॑त्ते । अ॒ष्ट॒ममि॑न्द्रि॒यस्या॒त्मन्-नु॒पाध॑त्ते । य ए॒वं ॅवि॒द्वान प्रा॑णत् प्रति गृ॒ह्णाति॑ । अथ॒ योऽवि॑द्वान् प्रतिगृ॒ह्णाति॑ । अ॒ष्ट॒मम॑स्येन्द्रि॒यस्याप॑ क्रामति ॥ यद्वा इ॒दं किं च॑ । तथ् सर्व॑-मुत्ता॒न ए॒वाङ्गी॑र॒सः प्रत्य॑ गृह्णात् । तदे॑नं॒ प्रति॑गृहीतं॒ नाहि॑नत् । यत् किं च॑ प्रति गृह्णी॒यात् ( ) । तथ् सर्व॑-मुत्ता॒न-स्त्वा᳚ङ्गीर॒सः प्रति॑गृह्णा॒त्वित्ये॒व प्रति॑ गृह्णीयात् । इ॒यं ॅवा उ॑त्ता॒न आ᳚ङ्गीर॒सः । अ॒नयै॒वैन॒त्-प्रति॑गृह्णाति । नैनꣳ॑ हिनस्ति । \textbf{ 16} \newline
                  \newline
                                                        \textbf{special korvai} \newline
              (तस्य॒ वा अ॒ग्नेर् हिर॑ण्यꣳ॒॒ सोम॑स्य॒ वास॒स्तदे॒तेन॑ रु॒द्रस्य॒ गां तामे॒तेन॒ वरु॑ण॒स्याश्वं॑ प्र॒जाप॑तेः॒ पुरु॑षं॒ मनो॒स्तल्पं॒ तमे॒तेनो᳚त्ता॒नस्य॒ तदे॒तेनाप्रा॑ण॒द्यद् वै ) (अ॒र्द्धं तृती॑यमष्ट॒मं तच्च॑तु॒र्थं तां प॑ञ्च॒मꣳ ष॒॒ष्ठꣳ स॑प्त॒मन्तम् । तदे॒तेन॒ द्वे तामे॒तेनैकं॒ तमे॒तेन॒ त्रीणि॒ तदे॒तेनैकं᳚ ) \newline
                                (तृती॑यमिन्द्रि॒यस्यापा᳚क्रामच् - चतु॒र्थमि॑न्द्रि॒यस्या॒त्मन्नु॒पाध॒ - त्ताश्वं॑ प्रतिगृ॒ह्णाति॑ - ष॒ष्ठ॑मस्येन्द्रि॒यस्याप॑क्राम -त्यष्ट॒ममि॑न्द्रि॒यस्यापा᳚क्रामत् - प्रति गृह्णी॒याच् च॒त्वारि॑ च) \textbf{(A4)} \newline \newline
                \textbf{ 2.3.5      अनुवाकं   5 - प्रश्नोत्तररूपो ब्रह्मविषयस्संवादः} \newline
                                \textbf{ TB 2.3.5.1} \newline
                  ब्र॒ह्म॒वा॒दिनो॑ वदन्ति ॥ यद्-दश॑होतारः स॒त्रमास॑त । केन॒ ते गृ॒हप॑तिना ऽऽर्द्ध्नुवन्न् । केन॑ प्र॒जा अ॑सृज॒न्तेति॑ । प्र॒जाप॑तिना॒ वै ते गृ॒हप॑तिना ऽऽर्द्ध्नुवन्न् । तेन॑ प्र॒जा अ॑सृजन्त ॥ यच्चतु॑र्.होतारः स॒त्रमास॑त । केन॒ ते गृ॒हप॑तिना ऽऽर्द्ध्नुवन्न् । केनौष॑धी-रसृज॒न्तेति॑ । सोमे॑न॒ वै ते गृ॒हप॑तिना ऽऽर्द्ध्नुवन्न् \textbf{ 17} \newline
                  \newline
                                \textbf{ TB 2.3.5.2} \newline
                  तेनौष॑धी-रसृजन्त ॥ यत् पञ्च॑होतारः स॒त्रमास॑त । केन॒ ते गृ॒हप॑तिना ऽऽर्द्ध्नुवन्न् । केनै॒भ्यो लो॒केभ्योऽसु॑रा॒न् प्राणु॑दन्त । केनै॑षां प॒शून॑वृञ्ज॒तेति॑ । अ॒ग्निना॒ वै ते गृ॒हप॑तिना ऽऽर्द्ध्नुवन्न् । तेनै॒भ्यो लो॒केभ्योऽसु॑रा॒न्-प्राणु॑दन्त । तेनै॑षां प॒शून॑वृञ्जत ॥ यथ् षड्ढो॑तारः स॒त्रमास॑त । केन॒ ते गृ॒हप॑तिना ऽऽर्द्ध्नुवन्न् \textbf{ 18} \newline
                  \newline
                                \textbf{ TB 2.3.5.3} \newline
                  केन॒र्तून॑-कल्पय॒न्तेति॑ । धा॒त्रा वै ते गृ॒हप॑तिना ऽऽर्द्ध्नुवन्न् । तेन॒र्तून॑कल्पयन्त ॥ यथ् स॒प्तहो॑तारः स॒त्रमास॑त । केन॒ ते गृ॒हप॑तिना ऽऽर्द्ध्नुवन्न् । केन॒ सुव॑रायन्न् । केने॒माल् ॅलो॒कान्थ्-सम॑तन्व॒न्निति॑ । अ॒र्य॒म्णा वै ते गृ॒हप॑तिना ऽऽर्द्ध्नुवन्न् । तेन॒ सुव॑रायन्न् । तेने॒माल् ॅलो॒कान्थ्-सम॑तन्व॒न्निति॑ । \textbf{ 19} \newline
                  \newline
                                \textbf{ TB 2.3.5.4} \newline
                  ए॒ते वै दे॒वा गृ॒हप॑तयः । तान्. य ए॒वं ॅवि॒द्वान् । अप्य॒न्यस्य॑ गार्.हप॒ते दीक्ष॑ते । अ॒वा॒न्त॒रमे॒व स॒त्रिणा॑-मृद्ध्नोति ॥ यो वा अ॑र्य॒मणं॒ ॅवेद॑ । दान॑कामा अस्मै प्र॒जा भ॑वन्ति । य॒ज्ञो वा अ॑र्य॒मा । आर्या॑वस॒तिरिति॒ वै तमा॑हु॒र्यं प्र॒शꣳस॑न्ति ॥ आर्या॑ वस॒तिर्भ॑वति । य ए॒वं ॅवेद॑ । \textbf{ 20} \newline
                  \newline
                                \textbf{ TB 2.3.5.5} \newline
                  यद्वा इ॒दं किं च॑ । तथ् सर्वं॒ चतु॑र्.होतारः । चतु॑र्.होतृ॒भ्योऽधि॑ य॒ज्ञो निमि॑र्तः । स य ए॒वं ॅवि॒द्वान्. वि॒वदे॑त । अ॒हमे॒व भूयो॑ वेद । यश्चतु॑र्.होतॄ॒न्॒. वेदेति॑ । स ह्ये॑व भूयो॒ वेद॑ । यश्चतु॑र्.होतॄ॒न्.॒ वेद॑ ॥ यो वै चतु॑र्.होतृणाꣳ॒॒ होतॄ॒न्॒. वेद॑ । सर्वा॑सु प्र॒जास्वन्न॑मत्ति \textbf{ 21} \newline
                  \newline
                                \textbf{ TB 2.3.5.6} \newline
                  सर्वा॒ दिशो॒ऽभिज॑यति । प्र॒जाप॑ति॒र्वै दश॑होतृणाꣳ॒॒ होता᳚ । सोम॒श्चतु॑र्.होतृणाꣳ॒॒ होता᳚ । अ॒ग्निः पञ्च॑होतृणाꣳ॒॒ होता᳚ । धा॒ता षड्ढो॑तृणाꣳ॒॒ होता᳚ । अ॒र्य॒मा स॒प्तहो॑तृणाꣳ॒॒ होता᳚ । ए॒ते वै चतु॑र्.होतृणाꣳ॒॒ होता॑रः । तान्. य ए॒वं ॅवेद॑ । सर्वा॑सु प्र॒जास्वन्न॑मत्ति । सर्वा॒ दिशो॒ऽभि ज॑यति ( ) । \textbf{ 22} \newline
                  \newline
                                    (आ॒र्द्ध्नु॒व॒ - न्ना॒ध्नु॒व॒ - न्नित्ये॒ - वं ॅवेदा᳚ - त्ति॒ - सर्वा॒ दिशो॒भि ज॑यति वै तेन॑ स॒त्रं केन॑) \textbf{(A5)} \newline \newline
                \textbf{ 2.3.6     अनुवाकं   6 –  होतृप्रशंसाप्रसङ्गेन क्रतूनामृत्विग् विशेषव्यवस्थापनं, अग्निहोत्रस्य प्राशस्त्यं च} \newline
                                \textbf{ TB 2.3.6.1} \newline
                  प्र॒जाप॑तिः प्र॒जाः सृ॒ष्ट्वा व्य॑स्रꣳसत । स हृद॑यं भू॒तो॑ऽशयत् । आत्म॒न्॒. हा(3) इत्यह्व॑यत् । आपः॒ प्रत्य॑शृण्वन्न् । ता अ॑ग्निहो॒त्रेणै॒व य॑ज्ञ्क्र॒तुनोप॑ प॒र्याव॑र्तन्त । ताः कुसि॑न्ध॒मुपौ॑हन्न् । तस्मा॑दग्निहो॒त्रस्य॑ यज्ञ्क्र॒तोः । एक॑ ऋ॒त्विक् ॥ च॒तु॒ष्कृत्वोऽह्व॑यत् । अ॒ग्निर्-वा॒यु-रा॑दि॒त्य-श्च॒न्द्रमाः᳚ \textbf{ 23} \newline
                  \newline
                                \textbf{ TB 2.3.6.2} \newline
                  ते प्रत्य॑शृण्वन्न् । ते द॑र्.शपूर्णमा॒साभ्या॑मे॒व य॑ज्ञ्क्र॒तुनोप॑ प॒र्याव॑र्तन्त । त उपौ॑हꣳ-श्च॒त्वार्यङ्गा॑नि । तस्मा᳚द्-दर्.शपूर्णमा॒सयो᳚र् यज्ञ्क्र॒तोः । च॒त्वार॑ ऋ॒त्विजः॑ ॥ प॒ञ्च॒ कृत्वोऽह्व॑यत् । प॒शवः॒ प्रत्य॑शृण्वन्न् । ते चा॑तुर्मा॒स्यैरे॒व य॑ज्ञ्क्र॒तुनोप॑ प॒र्याव॑र्तन्त । त उपौ॑ह॒ल् ॅलोम॑ छ॒वीं माꣳ॒॒समस्थि॑ म॒ज्जान᳚म् । तस्मा᳚च्चातुर्मा॒स्यानां᳚ ॅयज्ञ्क्र॒तोः \textbf{ 24} \newline
                  \newline
                                \textbf{ TB 2.3.6.3} \newline
                  पञ्च॒र्त्विजः॑ ॥ ष॒ट्कृत्वोऽह्व॑यत् । ऋ॒तवः॒ प्रत्य॑शृण्वन्न् । ते प॑शुब॒न्धेनै॒व य॑ज्ञ्क्र॒तुनोप॑ प॒र्याव॑र्तन्त । त उपौ॑ह॒न्थ्-स्तना॑वा॒ण्डौ शि॒श्नमवा᳚ञ्चं प्रा॒णम् । तस्मा᳚त्-पशुब॒न्धस्य॑ यज्ञ्क्र॒तोः । षडृ॒त्विजः॑ ॥ स॒प्त॒कृत्वोऽह्व॑यत् । होत्राः॒ प्रत्य॑शृण्वन्न् । ताः सौ॒म्येनै॒वाद्ध्व॒रेण॑ यज्ञ्क्र॒तुनोप॑ प॒र्याव॑र्तन्त \textbf{ 25} \newline
                  \newline
                                \textbf{ TB 2.3.6.4} \newline
                  ता उपौ॑हन्थ्-स॒प्तशी॑र्.ष॒ण्या᳚न्-प्रा॒णान् । तस्मा᳚थ्-सौ॒म्यस्या᳚द्ध्व॒रस्य॑ यज्ञ्क्र॒तोः । स॒प्त होत्राः॒ प्राची॒र्वष॑ट्कुर्वन्ति ॥ द॒श॒कृत्वोऽह्व॑यत् । तपः॒ प्रत्य॑शृणोत् । तत्-कर्म॑णै॒व स॑म्ॅवथ्स॒रेण॒ सर्वै᳚र्यज्ञ्क्र॒तुभि॒रुप॑ प॒र्याव॑र्तत । तथ् सर्व॑मा॒त्मान॒-मप॑रिवर्ग॒-मुपौ॑हत् । तस्मा᳚थ्- सम्ॅवथ्स॒रे सर्वे॑ यज्ञ्क्र॒तवोऽव॑रुद्ध्यन्ते ॥ तस्मा॒द्-दश॑होता॒ चतु॑र्.होता । पञ्च॑होता॒ षड्ढो॑ता स॒प्तहो॑ता ( ) । एक॑होत्रे ब॒लिꣳ ह॑रन्ति । हर॑न्त्यस्मै प्र॒जा ब॒लिम् । ऐन॒मप्र॑तिख्यातं गच्छति । य ए॒वं ॅवेद॑ । \textbf{ 26} \newline
                  \newline
                                    (च॒न्द्रमा᳚ - श्चातुर्मा॒स्यानां᳚ ॅयज्ञ्क्र॒तो - र॑द्ध्व॒रेण॑ यज्ञ्क्र॒तुनोप॑ प॒र्याव॑र्तन्त - स॒प्तहो॑ता च॒त्वारि॑ च) \textbf{(A6)} \newline \newline
                \textbf{ 2.3.7     अनुवाकं   7 - अग्निहोत्रस्य सर्वक्रतुरुद्भावसाधनेन प्रशंसा} \newline
                                \textbf{ TB 2.3.7.1} \newline
                  प्र॒जाप॑तिः॒ पुरु॑ष-मसृजत । सो᳚ऽग्निर॑ब्रवीत् । ममा॒य-मन्न॑म॒स्त्विति॑ । सो॑ऽबिभेत् । सर्वं॒ ॅवै मा॒ऽयं प्रध॑क्ष्य॒तीति॑ । स ए॒ताꣳ-श्चतु॑र्.होतॄनात्म॒-स्पर॑णानपश्यत् । तान॑जुहोत् । तैर्वै स आ॒त्मान॑-मस्पृणोत् ॥ यद॑ग्निहो॒त्रं जु॒होति॑ । एक॑होतारमे॒व तद्-य॑ज्ञ्क्र॒तु-मा᳚प्नोत्यग्निहो॒त्रम् \textbf{ 27} \newline
                  \newline
                                \textbf{ TB 2.3.7.2} \newline
                  कुसि॑न्धं चा॒त्मनः॑ स्पृ॒णोति॑ । आ॒दि॒त्यस्य॑ च॒ सायु॑ज्यं गच्छति ॥ च॒तुरुन्न॑यति । चतु॑र्.होतारमे॒व तद्-य॑ज्ञ्क्र॒तु-मा᳚प्नोति दर्.शपूर्णमा॒सौ । च॒त्वारि॑ चा॒त्मनोऽङ्गा॑नि स्पृ॒णोति॑ । आ॒दि॒त्यस्य॑ च॒ सायु॑ज्यं गच्छति ॥ च॒तुरुन्न॑यति । स॒मित्प॑ञ्च॒मी । पञ्च॑होतारमे॒व तद्-य॑ज्ञ्क्र॒तु-मा᳚प्नोति चातुर्मा॒स्यानि॑ । लोम॑ छ॒वीं माꣳ॒॒समस्थि॑ म॒ज्जान᳚म् \textbf{ 28} \newline
                  \newline
                                \textbf{ TB 2.3.7.3} \newline
                  तानि॑ चा॒त्मनः॑ स्पृ॒णोति॑ । आ॒दि॒त्यस्य॑ च॒ सायु॑ज्यं गच्छति ॥ च॒तुरुन्न॑यति । द्विर्जु॑होति । षड्ढो॑तारमे॒व तद्-य॑ज्ञ्क्र॒तु-मा᳚प्नोति पशुब॒न्धम् । स्तना॑वा॒ण्डौ शि॒श्नमवा᳚ञ्चं प्रा॒णम् । तानि॑ चा॒त्मनः॑ स्पृ॒णोति॑ । आ॒दि॒त्यस्य॑ च॒ सायु॑ज्यं गच्छति । च॒तुरुन्न॑यति । द्विर्जु॑होति \textbf{ 29} \newline
                  \newline
                                \textbf{ TB 2.3.7.4} \newline
                  स॒मिथ् स॑प्त॒मी । स॒प्तहो॑तारमे॒व तद्-य॑ज्ञ्क्र॒तु-मा᳚प्नोति सौ॒म्य-म॑द्ध्व॒रम् । स॒प्त चा॒त्मनः॑ शीर्.ष॒ण्या᳚न्-प्रा॒णान्थ्-स्पृ॒णोति॑ । आ॒दि॒त्यस्य॑ च॒ सायु॑ज्यं गच्छति ॥ च॒तुरुन्न॑यति । द्विर्जु॒होति॑ । द्विर्निमा᳚र्ष्टि । द्विः प्राश्ना॑ति । दश॑होतारमे॒व तद्-य॑ज्ञ्क्र॒तु-मा᳚प्नोति सम्ॅवथ्स॒रम् । सर्वं॑ चा॒त्मान॒-मप॑रिवर्गꣳ स्पृ॒णोति॑ ( ) । आ॒दि॒त्यस्य॑ च॒ सायु॑ज्यं गच्छति । \textbf{ 30} \newline
                  \newline
                                    (अ॒ग्नि॒हो॒त्रं - म॒ज्जानं॒ - द्विर्जु॑हो॒ - त्यप॑रिवर्गꣳ स्पृ॒णोत्येकं॑ च ) \textbf{(A7)} \newline \newline
                \textbf{ 2.3.8      अनुवाकं   8 - सृष्टिसाधनत्वेन होतृमन्त्रासामर्थ्य प्रशंसा} \newline
                                \textbf{ TB 2.3.8.1} \newline
                  [(अप॑क्रामत गर्भि॒ण्यः) इस् ओन्ल्य् इन्स्त्रुच्तिओन् , नोत् अ मन्त्र]प्र॒जाप॑ति-रकामयत॒ प्रजा॑ये॒येति॑ । स तपो॑ऽतप्यत । सो᳚ऽन्तर्वा॑नभवत् । स हरि॑तः श्या॒वो॑ऽभवत् । तस्मा॒थ्-स्त्र्य॑न्तर्व॑त्नी । हरि॑णी स॒ती श्या॒वा भ॑वति । स वि॒जाय॑मानो॒ गर्भे॑णाताम्यत् । स ता॒न्तः कृ॒ष्णः श्या॒वो॑ऽभवत् । तस्मा᳚त् ता॒न्तः कृ॒ष्णः श्या॒वो भ॑वति । तस्यासु॑रे॒वाजी॑वत् \textbf{ 31} \newline
                  \newline
                                \textbf{ TB 2.3.8.2} \newline
                  तेनासु॒नाऽसु॑रानसृजत । तदसु॑राणा-मसुर॒त्वम् । य ए॒वमसु॑राणा-मसुर॒त्वं ॅवेद॑ । असु॑माने॒व भ॑वति । नैन॒मसु॑र्जहाति ॥ सोऽसु॑रान्थ् सृ॒ष्ट्वा पि॒तेवा॑मन्यत । तदनु॑ पि॒तॄन॑सृजत । तत्-पि॑तृ॒णां पि॑तृ॒त्वम् । य ए॒वं पि॑तृ॒णां पि॑तृ॒त्वं ॅवेद॑ । पि॒तेवै॒व स्वानां᳚ भवति \textbf{ 32} \newline
                  \newline
                                \textbf{ TB 2.3.8.3} \newline
                  यन्त्य॑स्य पि॒तरो॒ हव᳚म् ॥ स पि॒तॄन्थ् सृ॒ष्ट्वा ऽम॑नस्यत् । तदनु॑ मनु॒ष्या॑-नसृजत । तन्म॑नु॒ष्या॑णां मनुष्य॒त्वम् । य ए॒वं म॑नु॒ष्या॑णां मनुष्य॒त्वं ॅवेद॑ । म॒न॒स्व्ये॑व भ॑वति । नैनं॒ मनु॑र्जहाति ॥ तस्मै॑ मनु॒ष्या᳚न्थ् ससृजा॒नाय॑ । दिवा॑ देव॒त्रा ऽभ॑वत् । तदनु॑ दे॒वान॑सृजत ( ) । तद्-दे॒वानां᳚ देव॒त्वम् । य ए॒वं दे॒वानां᳚ देव॒त्वं ॅवेद॑ । दिवा॑ है॒वास्य॑ देव॒त्रा भ॑वति ॥ तानि॒ वा ए॒तानि॑ च॒त्वार्यम्भाꣳ॑सि । दे॒वा म॑नु॒ष्याः᳚ पि॒तरोऽसु॑राः ॥ तेषु॒ सर्वे॒ष्वम्भो॒ नभ॑ इव भवति । य ए॒वं ॅवेद॑ । \textbf{ 33} \newline
                  \newline
                                    (अ॒जी॒व॒थ् - स्वानां᳚ भवति - दे॒वान॑सृजत स॒प्त च॑) \textbf{(A8)} \newline \newline
                \textbf{ 2.3.9      अनुवाकं   9 - वायुरूपत्वज्ञानेन होतृमन्त्राणां प्रशंसा} \newline
                                \textbf{ TB 2.3.9.1} \newline
                  [(य॒था॒स्था॒नं ग॑र्भि॒ण्यः॑) इस् ऒन्ल्य् इन्स्त्रुच्तिऒन् , नॊत् अ मन्त्र]ब्र॒ह्म॒वा॒दिनो॑ वदन्ति । यो वा इ॒मं ॅवि॒द्यात् । यतो॒ऽयं पव॑ते । यद॑भि॒ पव॑ते । यद॑भि स॒पंव॑ते । सर्व॒मायु॑रियात् । न पु॒रा ऽऽयु॑षः॒ प्रमी॑येत । प॒शु॒मान्थ् स्या᳚त् । वि॒न्देत॑ प्र॒जाम् ॥ यो वा इ॒मं ॅवेद॑ \textbf{ 34} \newline
                  \newline
                                \textbf{ TB 2.3.9.2} \newline
                  यतो॒ऽयं पव॑ते । यद॑भि॒ पव॑ते । यद॑भि स॒पंव॑ते । सर्व॒मायु॑रेति । न पु॒राऽऽयु॑षः॒ प्रमी॑यते । प॒शु॒मान् भ॑वति । वि॒न्दते᳚ प्र॒जाम् ॥ अ॒द्भ्यः प॑वते । अ॒पो॑ऽभि प॑वते । अ॒पो॑ऽभि संप॑वते । \textbf{ 35} \newline
                  \newline
                                \textbf{ TB 2.3.9.3} \newline
                  अ॒स्याः प॑वते । इ॒माम॒भि प॑वते । इ॒माम॒भि संप॑वते । अ॒ग्नेः प॑वते । अ॒ग्निम॒भि प॑वते । अ॒ग्निम॒भि सं प॑वते । अ॒न्तरि॑क्षात् पवते । अ॒न्तरि॑क्षम॒भि प॑वते । अ॒न्तरि॑क्षम॒भि सं प॑वते । आ॒दि॒त्यात् प॑वते \textbf{ 36} \newline
                  \newline
                                \textbf{ TB 2.3.9.4} \newline
                  आ॒दि॒त्यम॒भि प॑वते । आ॒दि॒त्यम॒भि सं प॑वते । द्योः प॑वते । दिव॑म॒भि प॑वते ॥ दिव॑म॒भि सं प॑वते । दि॒ग्भ्यः प॑वते । दिशो॒ऽभि प॑वते । दिशो॒ऽभि सं प॑वते । स यत्-पु॒रस्ता॒द्वाति॑ । प्रा॒ण ए॒व भू॒त्वा पु॒रस्ता᳚द्वाति \textbf{ 37} \newline
                  \newline
                                \textbf{ TB 2.3.9.5} \newline
                  तस्मा᳚त्-पु॒रस्ता॒-द्वान्त᳚म् । सर्वाः᳚ प्र॒जाः प्रति॑ नन्दन्ति । प्रा॒णो हि प्रि॒यः प्र॒जाना᳚म् । प्रा॒ण इ॑व प्रि॒यः प्र॒जानां᳚ भवति । य ए॒वं ॅवेद॑ । स वा ए॒ष प्रा॒ण ए॒व ॥ अथ॒ यद्-द॑क्षिण॒तो वाति॑ । मा॒त॒रिश्वै॒व भू॒त्वा द॑क्षिण॒तो वा॑ति । तस्मा᳚द्-दक्षिण॒तो वान्तं॑ ॅवि॒द्यात् । सर्वा॒ दिश॒ आवा॑ति \textbf{ 38} \newline
                  \newline
                                \textbf{ TB 2.3.9.6} \newline
                  सर्वा॒ दिशोऽनु॒ विवा॑ति । सर्वा॒ दिशोऽनु॒ सम्ॅवा॒तीति॑ । स वा ए॒ष मा॑त॒रिश्वै॒व ॥ अथ॒ यत्-प॒श्चाद्वाति॑ । पव॑मान ए॒व भू॒त्वा प॒श्चाद्वा॑ति । पू॒तम॑स्मा॒ आह॑रन्ति । पू॒तमुप॑हरन्ति । पू॒तम॑श्नाति । य ए॒वं ॅवेद॑ । स वा ए॒ष पव॑मान ए॒व । \textbf{ 39} \newline
                  \newline
                                \textbf{ TB 2.3.9.7} \newline
                  अथ॒ यदु॑त्तर॒तो वाति॑ । स॒वि॒तैव भू॒त्वोत्त॑र॒तो वा॑ति । स॒वि॒तेव॒ स्वानां᳚ भवति । य ए॒वं ॅवेद॑ । स वा ए॒ष स॑वि॒तैव ॥ ते य ए॑नं पु॒रस्ता॑दा॒यन्त॑-मुप॒वद॑न्ति । य ए॒वास्य॑ पु॒रस्ता᳚त्-पा॒प्मानः॑ । ताꣳस्ते ऽप॑घ्नन्ति । पु॒रस्ता॒दित॑रान्-पा॒प्मनः॑ सचन्ते ॥ अथ॒ य ए॑नं दक्षिण॒त आ॒यन्त॑-मुप॒वद॑न्ति \textbf{ 40} \newline
                  \newline
                                \textbf{ TB 2.3.9.8} \newline
                  य ए॒वास्य॑ दक्षिण॒तः पा॒प्मानः॑ । ताꣳस्ते ऽप॑घ्नन्ति । द॒क्षि॒ण॒त इत॑रान् पा॒प्मनः॑ सचन्ते ॥ अथ॒ य ए॑नं प॒श्चादा॒यन्त॑-मुप॒वद॑न्ति । य ए॒वास्य॑ प॒श्चात्-पा॒प्मानः॑ । ताꣳस्ते ऽप॑घ्नन्ति । प॒श्चादित॑रान्-पा॒प्मनः॑ सचन्ते ॥ अथ॒ य ए॑नमुत्तर॒त आ॒यन्त॑मुप॒ वद॑न्ति । य ए॒वास्यो᳚त्तर॒तः पा॒प्मानः॑ । ताꣳस्ते ऽप॑घ्नन्ति \textbf{ 41} \newline
                  \newline
                                \textbf{ TB 2.3.9.9} \newline
                  उ॒त्त॒र॒त इत॑रान् पा॒प्मनः॑ सचन्ते ॥ तस्मा॑ दे॒वं ॅवि॒द्वान् । वीव॑ नृत्येत् । प्रेव॑ चलेत् । व्यस्ये॑वा॒क्ष्यौ भा॑षेत । म॒ण्टये॑दिव । क्रा॒थये॑दिव । शृ॒ङ्गा॒येते॑व ॥ उ॒त मोप॑वदेयुः । उ॒त मे॑ पा॒प्मान॒-मप॑हन्यु॒रिति॑ ( ) ॥ स यां दिशꣳ॑ स॒निमे॒ष्यन्थ्-स्यात् । य॒दा तां दिशं॒ ॅवातो॑ वा॒यात् । अथ॒ प्रवे॒यात् । प्र वा॑ धापयेत् (धावयेत्) । सा॒तमे॒व र॑दि॒तं ॅव्यू॑ढं ग॒न्धम॒भि प्रच्य॑वते । आऽस्य॒ तं ज॑नप॒दं पूर्वा॑ की॒र्तिर्ग॑च्छति ॥दान॑कामा अस्मै प्र॒जा भ॑वन्ति । य ए॒वं ॅवेद॑ । \textbf{ 42} \newline
                  \newline
                                    (वेद॒ - संप॑वत - आदि॒त्यात् प॑वते - वा॒त्या - वा᳚त्ये॒ - ष पव॑मान ए॒व - द॑क्षिण॒त आ॒यन्त॑मुप॒ वद॑न् - त्युत्तर॒तः पा॒प्मान॒स्ताꣳस्ते ऽप॑घ्न॒ - न्तीत्य॒ष्टौ च॑) \textbf{(A9)} \newline \newline
                \textbf{ 2.3.10    अनुवाकं   10 - दशहोतृमन्त्राणां काम्यप्रयोगः} \newline
                                \textbf{ TB 2.3.10.1} \newline
                  प्र॒जाप॑तिः॒ सोमꣳ॒॒ राजा॑नमसृजत । तं त्रयो॒ वेदा॒ अन्व॑सृज्यन्त । तान्. हस्ते॑ऽकुरुत । अथ॒ ह सीता॑ सावि॒त्री । सोमꣳ॒॒ राजा॑नं चकमे । श्र॒द्धामु॒ स च॑कमे । सा ह॑ पि॒तरं॑ प्र॒जाप॑ति॒-मुप॑ससार । तꣳ हो॑वाच । नम॑स्ते अस्तु भगवः । उप॑ त्वाऽयानि \textbf{ 43} \newline
                  \newline
                                \textbf{ TB 2.3.10.2} \newline
                  प्र त्वा॑ पद्ये ॥ सोमं॒ ॅवै राजा॑नं कामये । श्र॒द्धामु॒ स का॑मयत॒ इति॑ । तस्या॑ उ॒ ह स्था॑ग॒रम॑लङ्का॒रं क॑ल्पयि॒त्वा । दश॑होतारं पु॒रस्ता᳚द्-व्या॒ख्याय॑ । चतु॑र्.होतारं दक्षिण॒तः । पञ्च॑ होतारं प॒श्चात् । षड्ढो॑तार-मुत्तर॒तः । स॒प्तहो॑तार-मु॒परि॑ष्टात् । स॒म्भा॒रैश्च॒ पत्नि॑भिश्च॒ मुखे॑ऽल॒ङ्कृत्य॑ \textbf{ 44} \newline
                  \newline
                                \textbf{ TB 2.3.10.3} \newline
                  आऽस्यार्द्धं ॅव॑व्राज । ताꣳ हो॒दीक्ष्यो॑वाच । उप॒ माऽऽव॑र्त॒स्वेति॑ । तꣳ हो॑वाच । भोगं॒ तु म॒ आच॑क्ष्व । ए॒तन् म॒ आच॑क्ष्व । यत्ते॑ पा॒णाविति॑ । तस्या॑ उ॒ ह त्रीन्. वेदा॒न् प्रद॑दौ । तस्मा॒दु ह॒ स्त्रियो॒ भोग॒मैव हा॑रयन्ते ॥ स यः का॒मये॑त प्रि॒यः स्या॒मिति॑ \textbf{ 45} \newline
                  \newline
                                \textbf{ TB 2.3.10.4} \newline
                  यं ॅवा॑ का॒मये॑त प्रि॒यः स्या॒दिति॑ । तस्मा॑ ए॒तꣳ स्था॑ग॒र-म॑लङ्का॒रं क॑ल्पयि॒त्वा । दश॑होतारं पु॒रस्ता᳚द्-व्या॒ख्याय॑ । चतु॑र्.होतारं दक्षिण॒तः । पञ्च॑होतारं प॒श्चात् । षड्ढो॑तार-मुत्तर॒तः । स॒प्तहो॑ता-रमु॒परि॑ष्टात् । स॒म्भा॒रैश्च॒ पत्नि॑भिश्च॒ मुखे॑ऽल॒ङ्कृत्य॑ । आऽस्यार्द्धं ॅव्र॑जेत् । प्रि॒यो है॒व भ॑वति ( ) । \textbf{ 46} \newline
                  \newline
                                    (अ॒या॒ - न्य॒ल॒ङ्कृत्य॑ - स्या॒मिति॑ - भवति) \textbf{(A10)} \newline \newline
                \textbf{ 2.3.11    अनुवाकं   11 - दशहोत्रादिमन्त्रनाम्नां प्रवृत्तिनिमित्तम्} \newline
                                \textbf{ TB 2.3.11.1} \newline
                  ब्रह्मा᳚त्म॒न्-वद॑सृजत । तद॑कामयत । समा॒त्मना॑ पद्ये॒येति॑ । आत्म॒न्-नात्म॒न्-नित्याम॑न्त्रयत । तस्मै॑ दश॒मꣳ हू॒तः प्रत्य॑शृणोत् । स दश॑हूतोऽभवत् । दश॑हूतो ह॒ वै नामै॒षः । तं ॅवा ए॒तं दश॑हूतꣳ॒॒ सन्त᳚म् । दश॑हो॒तेत्याच॑क्षते प॒रोक्षे॑ण । प॒रोक्ष॑प्रिया इव॒ हि दे॒वाः । \textbf{ 47} \newline
                  \newline
                                \textbf{ TB 2.3.11.2} \newline
                  आत्म॒न्-नात्म॒न्-नित्याम॑न्त्रयत । तस्मै॑ सप्त॒मꣳ हू॒तः प्रत्य॑शृणोत् । स स॒प्तहू॑तोऽभवत् । स॒प्तहू॑तो ह॒ वै नामै॒षः । तं ॅवा ए॒तꣳ स॒प्तहू॑तꣳ॒॒ सन्त᳚म् । स॒प्तहो॒तेत्याच॑क्षते प॒रोक्षे॑ण । प॒रोक्ष॑प्रिया इव॒ हि दे॒वाः ॥ आत्म॒न्-नात्म॒न्-नित्याम॑न्त्रयत । तस्मै॑ ष॒ष्ठꣳ हू॒तः प्रत्य॑शृणोत् । स षड्ढू॑तोऽभवत् \textbf{ 48} \newline
                  \newline
                                \textbf{ TB 2.3.11.3} \newline
                  षड्ढू॑तो ह॒ वै नामै॒षः । तं ॅवा ए॒तꣳ षड्ढू॑तꣳ॒॒ सन्त᳚म् । षड्ढो॒तेत्याच॑क्षते प॒रोक्षे॑ण । प॒रोक्ष॑प्रिया इव॒ हि दे॒वाः ॥ आत्म॒न्-नात्म॒न्-नित्याम॑न्त्रयत । तस्मै॑ पञ्च॒मꣳ हू॒तः प्रत्य॑ शृणोत् । स पञ्च॑ हूतोऽभवत् । पञ्च॑हूतो ह॒ वै नामै॒षः । तं ॅवा ए॒तं पञ्च॑हूतꣳ॒॒ सन्त᳚म् । पञ्च॑हो॒तेत्याच॑क्षते प॒रोक्षे॑ण \textbf{ 49} \newline
                  \newline
                                \textbf{ TB 2.3.11.4} \newline
                  प॒रोक्ष॑प्रिया इव॒ हि दे॒वाः ॥ आत्म॒न्-नात्म॒न्-नित्याम॑न्त्रयत । तस्मै॑ चतु॒र्थꣳ हू॒तः प्रत्य॑शृणोत् । स चतु॑र्.हूतोऽभवत् । चतु॑र्.हूतो ह॒ वै नामै॒षः । तं ॅवा ए॒तं चतु॑र्.हूतꣳ॒॒ सन्त᳚म् । चतु॑र्.हो॒तेत्याच॑क्षते प॒रोक्षे॑ण । प॒रोक्ष॑प्रिया इव॒ हि दे॒वाः ॥ तम॑ब्रवीत् । त्वं ॅवै मे॒ नेदि॑ष्ठꣳ हू॒तः प्रत्य॑श्रौषीः ( ) । त्वयै॑-नानाख्या॒तार॒ इति॑ । तस्मा॒न्नु है॑नाꣳ॒॒-श्चतु॑र्.होतार॒ इत्याच॑क्षते । तस्मा᳚-च्छुश्रू॒षुः पु॒त्राणाꣳ॒॒ हृद्य॑तमः । नेदि॑ष्ठो॒ हृद्य॑तमः । नेदि॑ष्ठो॒ ब्रह्म॑णो भवति । य ए॒वं ॅवेद॑ । \textbf{ 50} \newline
                  \newline
                                    (दे॒वाः - षड्ढू॑तोऽभव॒त् - पञ्च॑हो॒तेत्याच॑क्षते प॒रोक्षे॑णा - श्रौषीः॒ षट्च॑) \textbf{(A11)} \newline \newline
                \textbf{PrapAtaka Korvai with starting  words of 1 to11 anuvAkams :-} \newline
        (ब्र॒ह्म॒वा॒दिनः॒ किं - दक्षि॑णां॒ - ॅयो वा अवि॑द्वा॒न् - तस्य॒ वै - ब्र॑ह्मवा॒दिनो॒ यद् दश॑होतारः - प्र॒जाप॑ति॒ व्यस्रं - प्र॒जाप॑तिः॒ पुरु॑षं - प्र॒जाप॑तिरकामयत॒ स तपः॒ सो᳚ऽन्तर्वा᳚न् - ब्रह्मवा॒दिनो॒ यो वा इ॒मं वि॒द्यात् - प्र॒जाप॑तिः॒ सोमꣳ॒॒ राजा॑नं॒ - ब्रह्मा᳚त् म॒न्वदेका॑दश) \newline

        \textbf{korvai with starting words of 1, 11, 21 series of daSinis :-} \newline
        (ब्र॒ह्म॒वा॒दिन॒ - स्तस्य॒ वा अ॒ग्नेर् - यद्वा इ॒दं किंच॑ - प्र॒जाप॑तिरकामयत॒ -य ए॒वास्य॑ दक्षिण॒तः प॑ञ्चा॒शत्) \newline

        \textbf{first and last  word - 2nd aShTakam 3rd prapATakam :-} \newline
        (ब्र॒ह्म॒वा॒दिनो॒ - य ए॒वं ॅवेद॑ ) \newline 

       

        ॥ हरिः॑ ॐ ॥॥ कृष्ण यजुर्वेदीय तैत्तिरीय ब्राह्मणे द्वितीयाष्टके तृतीयः प्रपाठकः समाप्तः ॥
============================ \newline
        \pagebreak
        
        
        
     \addcontentsline{toc}{section}{ 2.4     द्वितीयाष्टके चतुर्थः प्रपाठकः - उपहोमाः}
     \markright{ 2.4     द्वितीयाष्टके चतुर्थः प्रपाठकः - उपहोमाः \hfill https://www.vedavms.in \hfill}
     \section*{ 2.4     द्वितीयाष्टके चतुर्थः प्रपाठकः - उपहोमाः }
                \textbf{ 2.4.1     अनुवाकं   1 - उपहोमाः} \newline
                                \textbf{ TB 2.4.1.1} \newline
                  जुष्टो॒ दमू॑ना॒ अति॑थिर्दुरो॒णे । इ॒मं नो॑ य॒ज्ञ्मुप॑ याहि वि॒द्वान् । विश्वा॑ अग्नेऽभि॒ युजो॑ वि॒हत्य॑ । श॒त्रू॒य॒तामाभ॑रा॒ भोज॑नानि ॥ अग्ने॒ शर्द्ध॑ मह॒ते सौभ॑गाय । तव॑ द्यु॒म्नान्यु॑त्त॒मानि॑ सन्तु । सं जा᳚स्प॒त्यꣳ सु॒यम॒मा कृ॑णुष्व । श॒त्रू॒य॒ताम॒भिति॑ष्ठा॒ महाꣳ॑सि ॥ अग्ने॒ यो नो॒ऽभितो॒ जनः॑ । वृको॒ वारो॒ जिघाꣳ॑सति \textbf{ 1} \newline
                  \newline
                                \textbf{ TB 2.4.1.2} \newline
                  ताꣳस्त्वं ॅवृ॑त्रहञ्जहि । वस्व॒स्मभ्य॒-माभ॑र ॥ अग्ने॒ यो नो॑ऽभि॒दास॑ति । स॒मा॒नो यश्च॒ निष्ट्यः॑ । इ॒द्ध्मस्ये॑व प्र॒क्षाय॑तः । मा तस्योच्छे॑षि॒ किञ्च॒न ॥ त्वमि॑न्द्राभि॒भूर॑सि । दे॒वो विज्ञा॑तवीर्यः । वृ॒त्र॒हा पु॑रु॒चेत॑नः ॥ अप॒ प्राच॑ इन्द्र॒ विश्वाꣳ॑ अ॒मित्रान्॑ \textbf{ 2} \newline
                  \newline
                                \textbf{ TB 2.4.1.3} \newline
                  अपापा॑चो अभिभूते नुदस्व । अपोदी॑चो॒ अप॑ शूरध॒राच॑ ऊ॒रौ । यथा॒ तव॒ शर्म॒न्मदे॑म ॥ तमिन्द्रं॑ ॅवाजयामसि । म॒हे वृ॒त्राय॒ हन्त॑वे । स वृषा॑ वृष॒भो भु॑वत् ॥ यु॒जे रथं॑ ग॒वेष॑णꣳ॒॒ हरि॑भ्याम् । उप॒ ब्रह्मा॑णि जुजुषा॒ण-म॑स्थुः । विबा॑धिष्टा॒स्य रोद॑सी महि॒त्वा । इन्द्रो॑ वृ॒त्राण्य॑प्र॒ती ज॑घ॒न्वान् । \textbf{ 3} \newline
                  \newline
                                \textbf{ TB 2.4.1.4} \newline
                  ह॒व्य॒वाह॑-मभिमाति॒षाऽह᳚म् । र॒क्षो॒हणं॒ पृत॑नासु जि॒ष्णुम् । ज्योति॑ष्मन्तं॒ दीद्य॑तं॒ पुर॑न्धिम् । अ॒ग्निꣳ स्वि॑ष्ट॒कृत॒मा हु॑वेम ॥ स्वि॑ष्टमग्ने अ॒भि तत् पृ॑णाहि । विश्वा॑ देव॒ पृत॑ना अ॒भिष्य । उ॒रुं नः॒ पन्थां᳚ प्रदि॒शन्वि भा॑हि । ज्योति॑ष्मद्धेह्य॒जरं॑ न॒ आयुः॑ ॥ त्वाम॑ग्ने ह॒विष्म॑न्तः । दे॒वं मर्ता॑स ईडते \textbf{ 4} \newline
                  \newline
                                \textbf{ TB 2.4.1.5} \newline
                  मन्ये᳚ त्वा जा॒तवे॑दसम् । स ह॒व्या व॑क्ष्याऽनु॒षक् ॥ विश्वा॑नि नो दु॒र्गहा॑ जातवेदः । सिन्धुं॒ न ना॒वा दु॑रि॒ताऽति॑पर्.षि । अग्ने॑ अत्रि॒वन् मन॑सा गृणा॒नः । अ॒स्माकं॑ बोद्ध्यवि॒ता त॒नूना᳚म् ॥ पू॒षा गा अन्वे॑तु नः । पू॒षा र॑क्ष॒त्वर्व॑तः । पू॒षा वाजꣳ॑ सनोतु नः ॥ पू॒षेमा आशा॒ अनु॑वेद॒ सर्वाः᳚ \textbf{ 5} \newline
                  \newline
                                \textbf{ TB 2.4.1.6} \newline
                  सो अ॒स्माꣳ अभ॑यतमेन नेषत् । स्व॒स्ति॒दा अघृ॑णिः॒ सर्व॑वीरः । अप्र॑युच्छन् पु॒र ए॑तु॒ प्रजा॒नन्न् ॥ त्वम॑ग्ने स॒प्रथा॑ असि । जुष्टो॒ होता॒ वरे᳚ण्यः । त्वया॑ य॒ज्ञ्ं ॅवित॑न्वते ॥ अ॒ग्नी रक्षाꣳ॑सि सेधति । शु॒क्रशो॑चि॒रम॑र्त्यः । शुचिः॑ पाव॒क ईड्यः॑ ॥ अग्ने॒ रक्षा॑णो॒ अꣳह॑सः \textbf{ 6} \newline
                  \newline
                                \textbf{ TB 2.4.1.7} \newline
                  प्रति॑ष्म देव॒ रीष॑तः । तपि॑ष्ठैर॒जरो॑ दह ॥ अग्ने॒ हꣳसि॒ न्य॑त्रिण᳚म् । दीद्य॒न् मर्त्ये॒ष्वा । स्वे क्षये॑ शुचिव्रत ॥ आ वा॑त वाहि भेष॒जम् । वि वा॑त वाहि॒ यद्रपः॑ । त्वꣳहि वि॒श्व भे॑षजः । दे॒वानां᳚ दू॒त ईय॑से ॥ द्वावि॒मौ वाता॑ वातः \textbf{ 7} \newline
                  \newline
                                \textbf{ TB 2.4.1.8} \newline
                  आ सिन्धो॒रा प॑रा॒वतः॑ । दक्षं॑ मे अ॒न्य आ॒वातु॑ । परा॒ऽन्यो वा॑तु॒ यद्रपः॑ ॥ यद॒दो वा॑त ते गृ॒हे । अ॒मृत॑स्य नि॒धिर्.हि॒तः । ततो॑ नो देहि जी॒वसे᳚ । ततो॑ नो धेहि भेष॒जम् । ततो॑ नो॒ मह॒ आव॑ह ॥ वात॒ आवा॑तु भेष॒जम् । श॒म्भूर् म॑यो॒ भूर्नो॑ हृ॒दे \textbf{ 8} \newline
                  \newline
                                \textbf{ TB 2.4.1.9} \newline
                  प्रण॒ आयूꣳ॑षि तारिषत् ॥ त्वम॑ग्ने अ॒याऽसि॑ । अ॒या सन्मन॑सा हि॒तः । अ॒या सन्.ह॒व्यमू॑हिषे । अ॒या नो॑ धेहि भेष॒जम् ॥ इ॒ष्टो अ॒ग्निराहु॑तः । स्वाहा॑कृतः पिपर्तु नः । स्व॒गा दे॒वेभ्य॑ इ॒दं नमः॑ ॥ कामो॑ भू॒तस्य॒ भव्य॑स्य । स॒म्राडेको॒ विरा॑जति \textbf{ 9} \newline
                  \newline
                                \textbf{ TB 2.4.1.10} \newline
                  स इ॒दं प्रति॑ पप्रथे । ऋ॒तूनुथ् सृ॑जते व॒शी ॥ काम॒स्तदग्रे॒ सम॑वर्त॒ताधि॑ । मन॑सो॒ रेतः॑ प्रथ॒मं ॅयदासी᳚त् । स॒तो बन्धु॒मस॑ति॒ निर॑विन्दन्न् । हृ॒दि प्र॒तीष्या॑ क॒वयो॑ मनी॒षा ॥ त्वया॑ मन्यो स॒रथ॑-मारु॒जन्तः॑ । हर्.ष॑माणासो धृष॒ता म॑रुत्वः । ति॒ग्मेष॑व॒ आयु॑धा सꣳ॒॒शिशा॑नाः । उप॒प्रय॑न्ति॒ नरो॑ अ॒ग्निरू॑पाः । \textbf{ 10} \newline
                  \newline
                                \textbf{ TB 2.4.1.11} \newline
                  म॒न्युर्भगो॑ म॒न्युरे॒वास॑ दे॒वः । म॒न्युर्. होता॒ वरु॑णो वि॒श्ववे॑दाः । म॒न्युं ॅविश॑ ईडते देव॒यन्तीः᳚ । पा॒हि नो॑ मन्यो॒ तप॑सा॒ श्रमे॑ण ॥ त्वम॑ग्ने व्रत॒भृच्छुचिः॑ । दे॒वाꣳ आसा॑दया इ॒ह । अग्ने॑ ह॒व्याय॒ वोढ॑वे ॥ व्र॒ता नु बिभ्र॑द्व्रत॒पा अदा᳚भ्यः । यजा॑ नो दे॒वाꣳ अ॒जरः॑ सु॒वीरः॑ । दध॒द्-रत्ना॑नि सुविदा॒नो अ॑ग्ने ( ) । गो॒पा॒य नो॑ जी॒वसे॑ जातवेदः । \textbf{ 11} \newline
                  \newline
                                    (जिघाꣳ॑स - त्य॒मित्रा᳚ - ञ्जघ॒न्वा - नी॑डते॒ - सर्वा॒ - अꣳह॑सो - वातो - हृ॒दे - रा॑ज - त्य॒ग्निरू॑पाः - सुविदा॒नो अ॑ग्न॒ एकं॑ च ) \textbf{(A1)} \newline \newline
                \textbf{ 2.4.2      अनुवाकं   2 - उपहोमाः} \newline
                                \textbf{ TB 2.4.2.1} \newline
                  चक्षु॑षो हेते॒ मन॑सो हेते । वाचो॑ हेते॒ ब्रह्म॑णो हेते । यो मा॑ऽघा॒यु-र॑भि॒दास॑ति । तम॑ग्ने मे॒न्याऽमे॒निं कृ॑णु ॥ यो मा॒ चक्षु॑षा॒ यो मन॑सा । यो वा॒चा ब्रह्म॑णा ऽघा॒यु-र॑भि॒दास॑ति । तया᳚ऽग्ने॒ त्वं मे॒न्या । अ॒मुम॑मे॒निं कृ॑णु ॥ यत् किंचा॒सौ मन॑सा॒ यच्च॑ वा॒चा । य॒ज्ञिर् जु॒होति॒ यजु॑षा ह॒विर्भिः॑ \textbf{ 12} \newline
                  \newline
                                \textbf{ TB 2.4.2.2} \newline
                  तन् मृ॒त्युर् निर्.ऋ॑त्या सम्ॅविदा॒नः । पु॒राऽऽदि॒ष्टादाहु॑तीरस्य हन्तु ॥ या॒तु॒धाना॒ निर्.ऋ॑ति॒रादु॒ रक्षः॑ । ते अ॑स्यघ्नं॒ त्वनृ॑तेन स॒त्यम् । इन्द्रे॑षिता॒ आज्य॑मस्य मथ्नन्तु । मातथ्-समृ॑द्धि॒ यद॒सौ क॒रोति॑ ॥ हन्मि॑ ते॒ऽहं कृ॒तꣳ ह॒विः । यो मे॑ घो॒रमची॑ कृतः । अपा᳚ञ्चौ त उ॒भौ बा॒हू । अप॑नह्याम्या॒स्य᳚म् । \textbf{ 13} \newline
                  \newline
                                \textbf{ TB 2.4.2.3} \newline
                  अप॑नह्यामि ते बा॒हू । अप॑नह्याम्या॒स्य᳚म् । अ॒ग्नेर् दे॒वस्य॒ ब्रह्म॑णा । सर्वं॑ तेऽवधिषं कृ॒तम् ॥ पु॒रा ऽमुष्य॑ वषट्का॒रात् । य॒ज्ञ्ं दे॒वेषु॑ नस्कृधि । स्वि॑ष्टम॒स्माकं॑ भूयात् । माऽस्मान् प्राप॒न्-नरा॑तयः ॥ अन्ति॑ दू॒रे स॒तो अ॑ग्ने । भ्रातृ॑व्यस्या ऽभि॒दास॑तः \textbf{ 14} \newline
                  \newline
                                \textbf{ TB 2.4.2.4} \newline
                  व॒ष॒ट्का॒रेण॒ वज्रे॑ण । कृ॒त्याꣳ ह॑न्मि कृ॒ताम॒हम् ॥ यो मा॒ नक्तं॒ दिवा॑ सा॒यम् । प्रा॒तश्चाह्नो॑ नि॒पीय॑ति । अ॒द्या तमि॑न्द्र॒ वज्रे॑ण । भ्रातृ॑व्यं पादयामसि ॥ इन्द्र॑स्य गृ॒हो॑ऽसि॒ तं त्वा᳚ । प्रप॑द्ये॒ सगुः॒ साश्वः॑ । स॒ह यन्मे॒ अस्ति॒ तेन॑ ॥ ईडे॑ अ॒ग्निं ॅवि॑प॒श्चित᳚म् \textbf{ 15} \newline
                  \newline
                                \textbf{ TB 2.4.2.5} \newline
                  गि॒रा य॒ज्ञ्स्य॒ साध॑नम् । श्रु॒ष्टी॒वानं॑ धि॒तावा॑नम् ॥ अग्ने॑ श॒केम॑ ते व॒यम् । यमं॑ दे॒वस्य॑ वा॒जिनः॑ । अति॒ द्वेषाꣳ॑सि तरेम ॥ अव॑तं मा॒ सम॑नसौ॒ समो॑कसौ । सचे॑तसौ॒ सरे॑तसौ । उ॒भौ माम॑वतं जातवेदसौ । शि॒वौ भ॑वतम॒द्य नः॑ ॥ स्व॒यं कृ॑ण्वा॒नः सु॒गमप्र॑यावम् \textbf{ 16} \newline
                  \newline
                                \textbf{ TB 2.4.2.6} \newline
                  ति॒ग्मशृ॑ङ्गो वृष॒भः शोशु॑चानः । प्र॒त्नꣳ स॒धस्थ॒-मनु॒पश्य॑मानः । आ तन्तु॑म॒ग्निर्-दि॒व्यं त॑तान ॥ त्वं न॒स्तन्तु॑रु॒त सेतु॑रग्ने । त्वं पन्था॑ भवसि देव॒यानः॑ । त्वया᳚ ऽग्ने पृ॒ष्ठं ॅव॒यमारु॑हेम । अथा॑ दे॒वैः स॑ध॒मादं॑ मदेम ॥ उदु॑त्त॒मं मु॑मुग्धि नः । वि पाशं॑ मद्ध्य॒मं चृ॑त । अवा॑ध॒मानि॑ जी॒वसे᳚ । \textbf{ 17} \newline
                  \newline
                                \textbf{ TB 2.4.2.7} \newline
                  व॒यꣳ सो॑म व्र॒ते तव॑ । मन॑स्त॒नूषु॒ बिभ्र॑तः । प्र॒जा व॑न्तो अशीमहि ॥ इ॒न्द्रा॒णी दे॒वी सु॒भगा॑ सु॒पत्नी᳚ । उदꣳशे॑न पति॒विद्ये॑ जिगाय । त्रिꣳ॒॒शद॑स्या ज॒घनं॒ ॅयोज॑नानि । उ॒पस्थ॒ इन्द्रꣳ॒॒ स्थवि॑रं बिभर्ति ॥ सेना॑ ह॒ नाम॑ पृथि॒वी ध॑नञ्ज॒या । वि॒श्वव्य॑चा॒ अदि॑तिः॒ सूर्य॑त्वक् । इ॒न्द्रा॒णी दे॒वी प्रा॒सहा॒ ददा॑ना \textbf{ 18} \newline
                  \newline
                                \textbf{ TB 2.4.2.8} \newline
                  सा नो॑ दे॒वी सु॒हवा॒ शर्म॑ यच्छतु ॥ आ त्वा॑ ऽहार्.षम॒न्तर॑भूः । ध्रु॒वस्ति॒ष्ठावि॑चाचलिः । विश॑स्त्वा॒ सर्वा॑ वाञ्छन्तु । मा त्वद्-रा॒ष्ट्र-मधि॑भ्रशत् ॥ ध्रु॒वा द्यौर् द्ध्रु॒वा पृ॑थि॒वी । ध्रु॒वं ॅविश्व॑मि॒दं जग॑त् । ध्रु॒वा ह॒ पर्व॑ता इ॒मे । ध्रु॒वो राजा॑ वि॒शाम॒यम् ॥ इ॒हैवैधि॒ मा व्य॑थिष्ठाः \textbf{ 19} \newline
                  \newline
                                \textbf{ TB 2.4.2.9} \newline
                  पर्व॑त इ॒वा वि॑चाचलिः । इन्द्र॑ इवे॒ह ध्रु॒वस्ति॑ष्ठ । इ॒ह रा॒ष्ट्र-मु॑धारय ॥ अ॒भिति॑ष्ठ पृतन्य॒तः । अध॑रे सन्तु॒ शत्र॑वः । इन्द्र॑ इव वृत्र॒हाति॑ष्ठ । अ॒पः क्षेत्रा॑णि स॒जंयन्न्॑ ॥ इन्द्र॑ एणमदीधरत् । ध्रु॒वं ध्रु॒वेण॑ ह॒विषा᳚ । तस्मै॑ दे॒वा अधि॑ब्रवन्न् ( ) । अ॒यं च॒ ब्रह्म॑ण॒स्पतिः॑ । \textbf{ 20} \newline
                  \newline
                                    (ह॒विर्भि॑ - रा॒स्य॑ - मभि॒ दास॑तो - विप॒श्चित॒ - मप्र॑यावं - जी॒वसे॒ - ददा॑ना - व्यथिष्ठा - ब्रव॒न्नेकं॑ च) \textbf{(A2)} \newline \newline
                \textbf{ 2.4.3      अनुवाकं   3 - उपहोमाः} \newline
                                \textbf{ TB 2.4.3.1} \newline
                  जुष्टी॑ नरो॒ ब्रह्म॑णा वः पितृ॒णाम् । अक्ष॑मव्ययं॒ न किला॑ रिषाथ । यच्छक्व॑रीषु बृह॒ता रवे॑ण । इन्द्रे॒ शुष्म॒मद॑धाथा वसिष्ठाः ॥ पा॒व॒का नः॒ सर॑स्वती । वाजे॑भिर्-वा॒जिनी॑वती । य॒ज्ञ्ं ॅव॑ष्टु धि॒या व॑सुः ॥ सर॑स्वत्य॒भि नो॑ नेषि॒ वस्यः॑ । मा प॑स्फरीः॒ पय॑सा॒ मा न॒ आ ध॑क् । जु॒षस्व॑ नः स॒ख्या॑ वे॒श्या॑ च \textbf{ 21} \newline
                  \newline
                                \textbf{ TB 2.4.3.2} \newline
                  मा त्वत्क्षेत्रा॒-ण्यर॑णानि गन्म ॥ वृ॒ञ्जे ह॒विर्नम॑सा ब॒र्॒.हिर॒ग्नौ । अया॑मि॒ स्रुग् घृ॒तव॑ती सुवृ॒क्तिः । अम्य॑क्षि॒ सद्म॒ सद॑ने पृथि॒व्याः । अश्रा॑यि य॒ज्ञ्ः सूर्ये॒ न चक्षुः॑ ॥ इ॒हार्वाञ्च॒मति॑ह्वये । इन्द्रं॒ जैत्रा॑य॒ जेत॑वे । अ॒स्माक॑मस्तु॒ केव॑लः ॥ अ॒र्वाञ्च॒मिन्द्र॑म॒मुतो॑ हवामहे । यो गो॒जिद्-ध॑न॒जि-द॑श्व॒जिद्यः \textbf{ 22} \newline
                  \newline
                                \textbf{ TB 2.4.3.3} \newline
                  इ॒मं नो॑ य॒ज्ञ्ं ॅवि॑ह॒वे जु॑षस्व । अ॒स्य कु॑र्मो हरिवो मे॒दिनं॑ त्वा ॥ अस॑मृंष्टो जायसे मातृ॒वोः शुचिः॑ । म॒न्द्रः क॒विरुद॑तिष्ठो॒ विव॑स्वतः । घृ॒तेन॑ त्वा ऽवर्द्धयन्नग्न आहुत । धू॒मस्ते॑ के॒तुर॑भवद्-दि॒वि श्रि॒तः ॥ अ॒ग्निरग्रे᳚ प्रथ॒मो दे॒वता॑नाम् । सम्ॅया॑ताना-मुत्त॒मो विष्णु॑रासीत् । यज॑मानाय परि॒गृह्य॑ दे॒वान् । दी॒क्षये॒दꣳ ह॒विराग॑च्छतं नः । \textbf{ 23} \newline
                  \newline
                                \textbf{ TB 2.4.3.4} \newline
                  अ॒ग्निश्च॑ विष्णो॒ तप॑ उत्त॒मं म॒हः । दी॒क्षा॒पा॒लेभ्यो॒ वन॑तꣳ॒॒ हि श॑क्रा । विश्वै᳚र् दे॒वैर् य॒ज्ञियैः᳚ सम्ॅविदा॒नौ । दी॒क्षाम॒स्मै यज॑मानाय धत्तम् ॥ प्रतद्-विष्णुः॑ स्तवते वी॒र्या॑य । मृ॒गो न भी॒मः कु॑च॒रो गि॑रि॒ष्ठाः । यस्यो॒रुषु॑ त्रि॒षु वि॒क्रम॑णेषु । अधि॑क्षि॒यन्ति॒ भुव॑नानि॒ विश्वा᳚ ॥ नू मर्तो॑ दयते सनि॒ष्यन्. यः । विष्ण॑व उरुगा॒याय॒ दाश॑त् \textbf{ 24} \newline
                  \newline
                                \textbf{ TB 2.4.3.5} \newline
                  प्र यः स॒त्राचा॒ मन॑सा॒ यजा॑ तै । ए॒ताव॑न्तं॒ नर्य॑मा॒विवा॑सात् ॥ विच॑क्रमे पृथि॒वीमे॒ष ए॒ताम् । क्षेत्रा॑य॒ विष्णु॒र् मनु॑षे दश॒स्यन्न् । ध्रु॒वासो॑ अस्य की॒रयो॒ जना॑सः । उ॒रु॒क्षि॒तिꣳ सु॒जनि॑मा चकार ॥ त्रिर्दे॒वः पृ॑थि॒वीमे॒ष ए॒ताम् । विच॑क्रमे श॒तर्च॑सं महि॒त्वा । प्र विष्णु॑-रस्तु त॒वस॒स्त वी॑यान् । त्वे॒षꣳ ह्य॑स्य॒ स्थवि॑रस्य॒ नाम॑ । \textbf{ 25} \newline
                  \newline
                                \textbf{ TB 2.4.3.6} \newline
                  होता॑रं चि॒त्रर॑थ-मद्ध्व॒रस्य॑ । य॒ज्ञ्स्य॑ यज्ञ्स्य के॒तुꣳ रुश॑न्तम् । प्रत्य॑र्द्धिं दे॒वस्य॑ देवस्य म॒ह्ना । श्रि॒या त्व॑ऽग्निमति॑थिं॒ जना॑नाम् ॥ आ नो॒ विश्वा॑भिरू॒तिभिः॑ स॒जोषाः᳚ । ब्रह्म॑ जुषा॒णो ह॑र्यश्व याहि । वरी॑वृज॒थ् स्थवि॑रेभिः सुशिप्र । अ॒स्मे दध॒द्-वृष॑णꣳ॒॒ शुष्म॑मिन्द्र ॥ इन्द्रः॑ सुव॒र्॒.षा ज॒नय॒न्नहा॑नि । जि॒गायो॒शिग्भिः॒ पृत॑ना अभि॒श्रीः \textbf{ 26} \newline
                  \newline
                                \textbf{ TB 2.4.3.7} \newline
                  प्रारो॑चय॒न्मन॑वे के॒तुमह्ना᳚म् । अवि॑न्द॒ज्ज्योति॑र्-बृह॒ते रणा॑य ॥ अश्वि॑ना॒वव॑से॒ निह्व॑ये वाम् । आ नू॒नं ॅया॑तꣳ सुकृ॒ताय॑ विप्रा । प्रा॒त॒र्यु॒क्तेन॑ सु॒वृता॒ रथे॑न । उ॒पाग॑च्छत॒मव॒सा ऽऽग॑तं नः ॥ अ॒वि॒ष्टं धी॒ष्वश्वि॑ना न आ॒सु । प्र॒जाव॒द्-रेतो॒ अह्र॑यं नो अस्तु । आवां᳚ तो॒के तन॑ये॒ तूतु॑जानाः । सु॒रत्ना॑सो दे॒ववी॑तिं गमेम । \textbf{ 27} \newline
                  \newline
                                \textbf{ TB 2.4.3.8} \newline
                  त्वꣳ सो॑म॒ क्रतु॑भिः सु॒क्रतु॑र्भूः । त्वं दक्षैः᳚ सु॒दक्षो॑ वि॒श्ववे॑दाः । त्वं ॅवृषा॑ वृष॒त्वेभि॑र् महि॒त्वा । द्यु॒म्नेभि॑र्-द्यु॒म्न्य॑भवो नृ॒चक्षाः᳚ ॥ अषा॑ढं ॅयु॒थ्सु पृत॑नासु॒ पप्रि᳚म् । सु॒व॒र॒.षाम॒फ्स्वां ॅवृ॒जन॑स्य गो॒पाम् । भ॒रे॒षु॒जाꣳ सु॑क्षि॒तिꣳ सु॒श्रव॑सम् । जय॑न्तं॒ त्वामनु॑ मदेम सोम ॥ भवा॑ मि॒त्रो न शेव्यो॑ घृ॒तासु॑तिः । विभू॑तद्युम्न एव॒या उ॑ स॒प्रथाः᳚ \textbf{ 28} \newline
                  \newline
                                \textbf{ TB 2.4.3.9} \newline
                  अधा॑ ते विष्णो वि॒दुषा॑चि॒दृध्यः॑ । स्तोमो॑ य॒ज्ञ्स्य॒ राद्ध्यो॑ ह॒विष्म॑तः ॥ यः पू॒र्व्याय॑ वे॒धसे॒ नवी॑यसे । सु॒मज्जा॑नये॒ विष्ण॑वे॒ ददा॑शति । यो जा॒तम॒स्य म॑ह॒तो म॒हि ब्रवा᳚त् । सेदुः॒ श्रवो॑भिर्-यु॒ज्यं॑ चिद॒भ्य॑सत् ॥ तमु॑ स्तोतारः पू॒र्व्यं ॅयथा॑ वि॒द ऋ॒तस्य॑ । गर्भꣳ॑ ह॒विषा॑ पिपर्तन । आऽस्य॑ जा॒नन्तो॒ नाम॑ चिद्विवक्तन । बृ॒हत्ते॑ विष्णो सुम॒तिं भ॑जामहे । \textbf{ 29} \newline
                  \newline
                                \textbf{ TB 2.4.3.10} \newline
                  इ॒मा धा॒ना घृ॑त॒स्नुवः॑ । हरी॑ इ॒होप॑वक्षतः । इन्द्रꣳ॑ सु॒खत॑मे॒ रथे᳚ ॥ “ए॒ष ब्र॒ह्मा{1}”, प्र ते॑ म॒हे । वि॒दथे॑ शꣳ॒सिषꣳ॒॒ हरी᳚ । य ऋ॒त्वियः॒ प्र ते॑ वन्वे । व॒नुषो॑ हर्य॒तं मद᳚म् । इन्द्रो॒ नाम॑ घृ॒तं न यः । हरि॑भि॒श्चारु॒ सेच॑ते । श्रु॒तो ग॒ण आ त्वा॑ विशन्तु \textbf{ 30} \newline
                  \newline
                                \textbf{ TB 2.4.3.11} \newline
                  हरि॑वर्पसं॒ गिरः॑ ॥ आच॑र्.षणि॒प्रा वृ॑ष॒भो जना॑नाम् । राजा॑ कृष्टी॒नां पु॑रुहू॒त इन्द्रः॑ । स्तु॒तः श्र॑व॒स्यन्नव॒सोप॑म॒द्रिक् । यु॒क्त्वा हरी॒ वृष॒णा ऽऽया᳚ह्य॒र्वाङ् ॥ प्र यथ् सिन्ध॑वः प्रस॒वं ॅयदायन्न्॑ । आपः॑ समु॒द्रꣳ र॒त्थ्ये॑व जग्मुः । अत॑श्चि॒दिन्द्रः॒ सद॑सो॒ वरी॑यान् । यदीꣳ॒॒ सोमः॑ पृ॒णाति॑ दु॒ग्धो अꣳ॒॒शुः ॥ ह्वया॑मसि॒ त्वेन्द्र॑ या॒ह्य॑र्वाङ् \textbf{ 31} \newline
                  \newline
                                \textbf{ TB 2.4.3.12} \newline
                  अरं॑ ते॒ सोम॑स्त॒नुवे॑ भवाति । शत॑क्रतो मा॒दय॑स्वा सु॒तेषु॑ । प्रास्माꣳ अ॑व॒ पृत॑नासु॒ प्र यु॒थ्सु ॥ इन्द्रा॑य॒ सोमाः᳚ प्र॒दिवो॒ विदा॑नाः । ऋ॒भुर्-येभि॒र्-वृष॑पर्वा॒-विहा॑याः । प्र॒य॒म्यमा॑णा॒न् प्रति॒ षूगृ॑भाय । इन्द्र॒ पिब॒ वृष॑धूतस्य॒ वृष्णः॑ ॥ अहे॑डमान॒ उप॑याहि य॒ज्ञ्म् । तुभ्यं॑ पवन्त॒ इन्द॑वः सु॒तासः॑ । गावो॒ न व॑ज्रिन्थ् स्व॒मोको॒ अच्छ॑ \textbf{ 32} \newline
                  \newline
                                \textbf{ TB 2.4.3.13} \newline
                  इन्द्राग॑हि प्रथ॒मो यज्ञिया॑नाम् ॥ या ते॑ का॒कुथ् सुकृ॑ता॒ या वरि॑ष्ठा । यया॒ शश्व॒त्पिब॑सि॒ मद्ध्व॑ ऊ॒र्मिम् । तया॑ पाहि॒ प्र ते॑ अद्ध्व॒र्युर॑स्थात् । सं ते॒ वज्रो॑ वर्ततामिन्द्र ग॒व्युः ॥ प्रा॒त॒र्युजा॒ विबो॑धय । अश्वि॑ना॒वेह ग॑च्छतम् । अ॒स्य सोम॑स्य पी॒तये᳚ ॥ प्रा॒त॒र्यावा॑णा प्रथ॒मा य॑जद्ध्वम् । पु॒रा गृद्ध्रा॒दर॑रुषः पिबाथः ( ) । प्रा॒तर्.हि य॒ज्ञ्म॒श्विना॒ दधा॑ते । प्रशꣳ॑ सन्ति क॒वयः॑ पूर्व॒भाजः॑ ॥ प्रा॒तर्य॑जद्ध्वम॒श्विना॑ हिनोत । न सा॒यम॑स्ति देव॒या अजु॑ष्टम् । उ॒तान्यो अ॒स्मद्-य॑जते॒ विचा॑यः । पूर्वः॑ पूर्वो॒ यज॑मानो॒ वनी॑यान् । \textbf{ 33} \newline
                  \newline
                                    (चा॒ - श्व॒जिद्यो-ग॑च्छतं नो॒ - दाश॒न् - नामा॑ - भि॒श्रीर् - ग॑मेम - स॒प्रथा॑ - भजामहे - विशन्तु - या॒ह्य॑र्वाङ् - च्छ॑ - पिबाथः॒ षट्च॑) \textbf{(A3)} \newline \newline
                \textbf{ 2.4.4      अनुवाकं   4 - उपहोमाः} \newline
                                \textbf{ TB 2.4.4.1} \newline
                  न॒क्तं॒ जा॒ताऽस्यो॑षधे । रामे॒ कृष्णे॒ असि॑क्नि च । इ॒दꣳ र॑जनि रजय । कि॒लासं॑ पलि॒तं च॒ यत् ॥ कि॒लासं॑ च पलि॒तं च॑ । निरि॒तो ना॑शया॒ पृष॑त् । आ नः॒ स्वो अ॑श्नुतां॒ ॅवर्णः॑ । परा᳚ श्वे॒तानि॑ पातय ॥ असि॑तं ते नि॒लय॑नम् । आ॒स्थान॒मसि॑तं॒ तव॑ \textbf{ 34} \newline
                  \newline
                                \textbf{ TB 2.4.4.2} \newline
                  असि॑क्नियस्योषधे । निरि॒तो ना॑शया॒ पृष॑त् ॥ अ॒स्थि॒जस्य॑ कि॒लास॑स्य । त॒नू॒जस्य॑ च॒ यत्त्व॒चि । कृ॒त्यया॑ कृ॒तस्य॒ ब्रह्म॑णा । लक्ष्म॑ श्वे॒तम॑नीनशम् ॥ सरू॑पा॒ नाम॑ ते मा॒ता । सरू॑पो॒ नाम॑ ते पि॒ता । सरू॑पा ऽस्योषधे॒ सा । सरू॑पमि॒दं कृ॑धि । \textbf{ 35} \newline
                  \newline
                                \textbf{ TB 2.4.4.3} \newline
                  शु॒नꣳ हु॑वेम म॒घवा॑न॒मिन्द्र᳚म् । अ॒स्मिन्भरे॒ नृत॑मं॒ ॅवाज॑सातौ । शृ॒ण्वन्त॑-मु॒ग्रमू॒तये॑ स॒मथ्सु॑ । घ्नन्तं॑ ॅवृ॒त्राणि॑ स॒जिंतं॒ धना॑नाम् ॥ धू॒नु॒थ द्यां पर्व॑तान्दा॒शुषे॒ वसु॑ । नि वो॒ वना॑ जिहते॒ याम॑नो भि॒या । को॒पय॑थ पृथि॒वीं पृ॑श्निमातरः । यु॒धे यदु॑ग्राः॒ पृष॑ती॒-रयु॑ग्ध्वम् ॥ प्रवे॑पयन्ति॒ पर्व॑तान् । विवि॑ञ्चन्ति॒ वन॒स्पतीन्॑ \textbf{ 36} \newline
                  \newline
                                \textbf{ TB 2.4.4.4} \newline
                  प्रोवा॑रत मरुतो दु॒र्मदा॑ इव । देवा॑सः॒ सर्व॑या वि॒शा ॥ पु॒रु॒त्रा हि स॒दृङ्ङसि॑ । विशो॒ विश्वा॒ अनु॑ प्र॒भु । स॒मथ्सु॑ त्वा हवामहे ॥ स॒मथ्स्व॒ग्नि-मव॑से । वा॒ज॒यन्तो॑ हवामहे । वाजे॑षु चि॒त्ररा॑धसम् ॥ संग॑च्छद्ध्वꣳ॒॒ सम्ॅव॑दद्ध्वम् । सं ॅवो॒ मनाꣳ॑सि जानताम् \textbf{ 37} \newline
                  \newline
                                \textbf{ TB 2.4.4.5} \newline
                  दे॒वा भा॒गं ॅयथा॒ पूर्वे᳚ । स॒जां॒ना॒ना उ॒पास॑त ॥ स॒मा॒नो मन्त्रः॒ समि॑तिः समा॒नी । स॒मा॒नं मनः॑ स॒ह चि॒त्तमे॑षाम् । स॒मा॒नं केतो॑ अ॒भि-सꣳर॑भद्ध्वम् । स॒ज्ञांने॑न वो ह॒विषा॑ यजामः ॥ स॒मा॒नी व॒ आकू॑तिः । स॒मा॒ना हृद॑यानि वः । स॒मा॒नम॑स्तु वो॒ मनः॑ । यथा॑ वः॒ सुस॒हाऽस॑ति । \textbf{ 38} \newline
                  \newline
                                \textbf{ TB 2.4.4.6} \newline
                  स॒ज्ञांनं॑ नः॒ स्वैः । स॒ज्ञांन॒मर॑णैः । स॒ज्ञांन॑मश्विना यु॒वम् । इ॒हास्मासु॒ निय॑च्छतम् ॥ स॒ज्ञांनं॑ मे॒ बृह॒स्पतिः॑ । स॒ज्ञांनꣳ॑ सवि॒ता क॑रत् । स॒ज्ञांन॑मश्विना यु॒वम् । इ॒ह मह्यं॒ निय॑च्छतम् ॥ उप॑च्छा॒यामि॑व॒ घृणेः᳚ । अग॑न्म॒ शर्म॑ ते व॒यम् \textbf{ 39} \newline
                  \newline
                                \textbf{ TB 2.4.4.7} \newline
                  अग्ने॒ हिर॑ण्यसंदृशः ॥ अद॑ब्धेभिः सवितः पा॒युभि॒ष्ट्वम् । शि॒वेभि॑र॒द्य परि॑पाहि नो॒ गय᳚म् । हिर॑ण्यजिह्वः सुवि॒ताय॒ नव्य॑से । रक्षा॒माकि॑र्नो अ॒घशꣳ॑स ईशत ॥ मदे॑मदे॒ हि नो॑ द॒दुः । यू॒था गवा॑मृजु॒क्रतुः॑ । संगृ॑भाय पु॒रू श॒ता । उ॒भ॒या ह॒स्त्या वसु॑ । शि॒शी॒हि रा॒य आभ॑र । \textbf{ 40} \newline
                  \newline
                                \textbf{ TB 2.4.4.8} \newline
                  शिप्रि॑न् वाजानां पते । शची॑व॒स्तव॑ दꣳ॒॒सना᳚ । आ तू न॑ इन्द्र भाजय । गोष्वश्वे॑षु शु॒भ्रुषु॑ । स॒हस्रे॑षु तुवीमघ ॥ यद्-दे॑वा देव॒ हेड॑नम् । देवा॑सश्चकृ॒मा व॒यम् । आदि॑त्या॒स्तस्मा᳚न्मा यू॒यम् । ऋ॒तस्य॒र्तेन॑ मुञ्चत ॥ ऋ॒तस्य॒र्ते-ना॑दित्याः \textbf{ 41} \newline
                  \newline
                                \textbf{ TB 2.4.4.9} \newline
                  यज॑त्रा मु॒ञ्चते॒ह मा᳚ । य॒ज्ञिर्वो॑ यज्ञ्वाहसः । आ॒शिक्ष॑न्तो॒ न शे॑किम ॥ मेद॑स्वता॒ यज॑मानाः । स्रु॒चाऽऽज्ये॑न॒ जुह्व॑तः । अ॒का॒मा वो॑ विश्वे देवाः । शिक्ष॑न्तो॒ नोप॑ शेकिम ॥ यदि॒ दिवा॒ यदि॒ नक्त᳚म् । एन॑ एन॒स्योऽक॑रत् । भू॒तं मा॒ तस्मा॒द्भव्यं॑ च ( ) \textbf{ 42} \newline
                  \newline
                                \textbf{ TB 2.4.4.10} \newline
                  द्रु॒प॒दादि॑व मुञ्चतु ॥ द्रु॒प॒दादि॒वेन्-मु॑मुचा॒नः । स्वि॒न्नः स्ना॒त्वी मला॑दिव । पू॒तं प॒वित्रे॑णे॒ वाज्य᳚म् । विश्वे॑ मुञ्चन्तु॒ मैन॑सः ॥ उद्व॒यं तम॑स॒स्परि॑ । पश्य॑न्तो॒ ज्योति॒रुत्त॑रम् । दे॒वं दे॑व॒त्रा सूर्य᳚म् । अग॑न्म॒ ज्योति॑रुत्त॒मम् । \textbf{ 43} \newline
                  \newline
                                    (तव॑ - कृधि॒ - वन॒स्पती᳚ - ञ्जानता॒ - मस॑ति - व॒यं - भ॑रा - दित्या - श्च॒ - +नव॑ च) \textbf{(A4)} \newline \newline
                \textbf{ 2.4.5      अनुवाकं   5 - उपहोमाः} \newline
                                \textbf{ TB 2.4.5.1} \newline
                  वृषा॒ सो अꣳ॒॒शुः प॑वते ह॒विष्मा॒न्थ्-सोमः॑ । इन्द्र॑स्य भा॒ग ऋ॑त॒युः श॒तायुः॑ । स मा॒ वृषा॑णं ॅवृष॒भं कृ॑णोतु । प्रि॒यं ॅवि॒शाꣳ सर्व॑वीरꣳ सु॒वीर᳚म् ॥ कस्य॒ वृषा॑ सु॒ते सचा᳚ । नि॒युत्वा᳚न् वृष॒भो र॑णत् । वृ॒त्र॒हा सोम॑पीतये ॥ यस्ते॑ शृङ्ग वृषो नपात् । प्रण॑पात् कुण्ड॒पाय्यः॑ । न्य॑स्मिन्दद्ध्र॒ आ मनः॑ । \textbf{ 44} \newline
                  \newline
                                \textbf{ TB 2.4.5.2} \newline
                  तꣳ स॒द्ध्रीची॑रू॒तयो॒ वृष्णि॑यानि । पौꣳस्या॑नि नि॒युतः॑ सश्चु॒रिन्द्र᳚म् । स॒मु॒द्रं न सिन्ध॑व उ॒क्थशु॑ष्माः । उ॒रु॒व्यच॑सं॒ गिर॒ आवि॑शन्ति ॥ इन्द्रा॑य॒ गिरो॒ अनि॑शितसर्गाः । अ॒पः प्रैर॑य॒न्थ् सग॑रस्य बु॒द्दध्नात् । यो अक्षे॑णेव च॒क्रिया॒ शची॑भिः । विष्व॑क्त॒स्तम्भ॑ पृथि॒वीमु॒त द्याम् ॥ अक्षो॑दय॒च्छव॑सा॒ क्षाम॑ बु॒द्ध्नम् । वार्णवा॑त॒स्तवि॑षीभि॒रिन्द्रः॑ \textbf{ 45} \newline
                  \newline
                                \textbf{ TB 2.4.5.3} \newline
                  दृ॒ढान्यौ᳚घ्नादु॒शमा॑न॒ ओजः॑ । अवा॑भिनत्क॒कुभः॒ पर्व॑तानाम् ॥ आ नो॑ अग्ने सुके॒तुना᳚ । र॒यिं ॅवि॒श्वायु॑पोषसम् । मा॒र्डी॒कं धे॑हि जी॒वसे᳚ ॥ त्वꣳ सो॑म म॒हे भग᳚म् । त्वं ॅयून॑ ऋताय॒ते । दक्षं॑ दधासि जी॒वसे᳚ ॥ रथं॑ ॅयुञ्जते म॒रुतः॑ शु॒भे स॒गम् । सूरो॒ न मि॑त्रावरुणा॒ गवि॑ष्टिषु \textbf{ 46} \newline
                  \newline
                                \textbf{ TB 2.4.5.4} \newline
                  रजाꣳ॑सि चि॒त्रा वि च॑रन्ति त॒न्यवः॑ । दि॒वः स॑म्राजा॒ पय॑सा न उक्षतम् ॥ वाचꣳ॒॒ सु मि॑त्रावरुणा॒विरा॑वतीम् । प॒र्जन्य॑श्चि॒त्रां ॅव॑दति॒ त्विषी॑मतीम् । अ॒भ्रा व॑सत मरुतः सु मा॒यया᳚ । द्यां ॅव॑र्.षयत-मरु॒णाम॑रे॒पस᳚म् ॥ अयु॑क्त स॒प्त शु॒न्ध्युवः॑ । सूरो॒ रथ॑स्य न॒प्त्रियः॑ । ताभि॑र्याति॒ स्वयु॑क्तिभिः ॥ वहि॑ष्ठेभिर्वि॒हर॑न्. यासि॒ तन्तु᳚म् \textbf{ 47} \newline
                  \newline
                                \textbf{ TB 2.4.5.5} \newline
                  अ॒व॒व्यय॒न्नसि॑तं देव॒ वस्वः॑ । दवि॑द्ध्वतो र॒श्मयः॒ सूर्य॑स्य । चर्मे॒वावा॑धु॒स्तमो॑ अ॒फ्स्व॑न्तः ॥ प॒र्जन्या॑य॒ प्रगा॑यत । दि॒वस्पु॒त्राय॑ मी॒ढुषे᳚ । स नो॑ य॒वस॑मिच्छतु ॥ अच्छा॑ वद त॒वसं॑ गी॒र्भिरा॒भिः । स्तु॒हि प॒र्जन्यं॒ नम॒सा वि॑वास । कनि॑क्रदद्-वृष॒भो जी॒रदा॑नुः । रेतो॑ दधा॒त्वोष॑धीषु॒ गर्भ᳚म् । \textbf{ 48} \newline
                  \newline
                                \textbf{ TB 2.4.5.6} \newline
                  यो गर्भ॒मोष॑धीनाम् । गवां᳚ कृ॒णोत्यर्व॑ताम् । प॒र्जन्यः॑ पुरु॒षीणा᳚म् ॥ तस्मा॒ इदा॒स्ये॑ ह॒विः । जु॒होता॒ मधु॑मत्तमम् । इडां᳚ नः स॒म्ॅयतं॑ करत् ॥ ति॒स्रो यद॑ग्ने श॒रद॒स्त्वामित् । शुचिं॑ घृ॒तेन॒ शुच॑यः सप॒र्यन्न् । नामा॑नि चिद्दधिरे य॒ज्ञिया॑नि । असू॑दयन्त त॒नुवः॒ सुजा॑ताः । \textbf{ 49} \newline
                  \newline
                                \textbf{ TB 2.4.5.7} \newline
                  इन्द्र॑श्च नः शुनासीरौ । इ॒मं ॅय॒ज्ञ्ं मि॑मिक्षतम् । गर्भं॑ धत्तꣳ स्व॒स्तये᳚ ॥ ययो॑रि॒दं ॅविश्वं॒ भुव॑न-मावि॒वेश॑ । ययो॑रान॒न्दो निहि॑तो॒ मह॑श्च । शुना॑सीरावृ॒तुभिः॑ सम्ॅविदा॒नौ । इन्द्र॑वन्तौ ह॒विरि॒दं जु॑षेथाम् ॥ आ घा॒ ये अ॒ग्निमि॑न्ध॒ते । स्तृ॒णन्ति॑ ब॒र्॒.हिरा॑नु॒षक् । येषा॒मिन्द्रो॒ युवा॒ सखा᳚ ( ) ॥ अग्न॒ इन्द्र॑श्च मे॒दिना᳚ । ह॒थो वृ॒त्राण्य॑प्र॒ति । यु॒वꣳ हि वृ॑त्र॒हन्त॑मा ॥ याभ्याꣳ॒॒ सुव॒रज॑य॒न्नग्र॑ ए॒व । यावा॑तस्थ॒तुर्-भुव॑नस्य॒ मद्ध्ये᳚ । प्रच॑र्.ष॒णी वृ॑षणा॒ वज्र॑बाहू । अ॒ग्नी इन्द्रा॑ वृत्र॒हणा॑ हुवे वाम् । \textbf{ 50} \newline
                  \newline
                                    (मन॒ - इन्द्रो॒ - गावि॑ष्टिषु॒ - तन्तुं॒ - गर्भꣳ॒॒ - सुजा॑ताः॒ - सखा॑ स॒प्त च॑) \textbf{(A5)} \newline \newline
                \textbf{ 2.4.6      अनुवाकं   6 - उपहोमाः} \newline
                                \textbf{ TB 2.4.6.1} \newline
                  उ॒त नः॑ प्रि॒या प्रि॒यासु॑ । स॒प्तस्वसा॒ सुजु॑ष्टा । सर॑स्वती॒ स्तोम्या॑ ऽभूत् ॥ इ॒मा जुह्वा॑ना यु॒ष्मदा नमो॑भिः । प्रति॒ स्तोमꣳ॑ सरस्वति जुषस्व । तव॒ शर्म॑न् प्रि॒यत॑मे॒ दधा॑नाः । उप॑स्थेयाम शर॒णं न वृ॒क्षम् ॥ त्रीणि॑ प॒दा विच॑क्रमे । विष्णु॑र्गो॒पा अदा᳚भ्यः । ततो॒ धर्मा॑णि धा॒रयन्न्॑ । \textbf{ 51} \newline
                  \newline
                                \textbf{ TB 2.4.6.2} \newline
                  तद॑स्य प्रि॒यम॒भि पाथो॑ अश्याम् । नरो॒ यत्र॑ देव॒यवो॒ मद॑न्ति । उ॒रु॒क्र॒मस्य॒ स हि बन्धु॑रि॒त्था । विष्णोः᳚ प॒दे प॑र॒मे मद्ध्व॒ उथ्सः॑ ॥ क्र॒त्वा॒ दा अ॑स्थु॒ श्रेष्ठः॑ । अ॒द्य त्वा॑ व॒न्वन्थ् सु॒रेक्णाः᳚ । मर्त॑ आनाश सुवृ॒क्तिम् ॥ इ॒मा ब्र॑ह्म ब्रह्मवाह । प्रि॒या त॒ आ ब॒र॒.हिः सी॑द । वी॒हि सू॑र पुरो॒डाश᳚म् । \textbf{ 52} \newline
                  \newline
                                \textbf{ TB 2.4.6.3} \newline
                  उप॑ नः सू॒नवो॒ गिरः॑ । शृ॒ण्वन्त्व॒मृतस्य॒ ये । सु॒मृ॒डी॒का भ॑वन्तु नः ॥ अ॒द्या नो॑ देव सवितः । प्र॒जाव॑थ् सावीः॒ सौभ॑गम् । परा॑ दुः॒ष्वप्नि॑यꣳ सुव ॥ विश्वा॑नि देव सवितः । दु॒रि॒तानि॒ परा॑सुव । यद्-भ॒द्रं तन्म॒ आसु॑व ॥ शुचि॑म॒र्कैर्-बृह॒स्पति᳚म् \textbf{ 53} \newline
                  \newline
                                \textbf{ TB 2.4.6.4} \newline
                  अ॒द्ध्व॒रेषु॑ नमस्यत । अ॒ना॒म्योज॒ आच॑के ॥ या धा॒रय॑न्त दे॒वा सु॒दक्षा॒ दक्ष॑पितारा । अ॒सु॒र्या॑य॒ प्रम॑हसा ॥ स इत्क्षेति॒ सुधि॑त॒ ओक॑सि॒ स्वे । तस्मा॒ इडा॑ पिन्वते विश्व॒दानी᳚ । तस्मै॒ विशः॑ स्व॒यमे॒वान॑मन्ति । यस्मि॑न् ब्र॒ह्मा राज॑नि॒ पूर्व॒ एति॑ ॥ सकू॑तिमिन्द्र॒ सच्यु॑तिम् । सच्यु॑तिं ज॒घन॑च्युतिम् \textbf{ 54} \newline
                  \newline
                                \textbf{ TB 2.4.6.5} \newline
                  क॒नात्का॒भां न॒ आभ॑र । प्र॒य॒फ्स्यन्नि॑व स॒क्थ्यौ᳚ ॥ विन॑ इन्द्र॒ मृधो॑ जहि । कनी॑खुनदिव सा॒पयन्न्॑ । अ॒भि नः॒ सुष्टु॑तिं नय ॥ प्र॒जाप॑तिः स्त्रि॒यां ॅयशः॑ । मु॒ष्कयो॑रदधा॒थ् सप᳚म् । काम॑स्य॒ तृप्ति॑मान॒न्दम् । तस्या᳚ग्ने भाजये॒ह मा᳚ ॥ मोदः॑ प्रमो॒द आ॑न॒न्दः \textbf{ 55} \newline
                  \newline
                                \textbf{ TB 2.4.6.6} \newline
                  मु॒ष्कयो॒र्निहि॑तः॒ सपः॑ । सृ॒त्वेव॒ काम॑स्य तृप्याणि । दक्षि॑णानां प्रतिग्र॒हे ॥ मन॑सश्चि॒त्त-माकू॑तिम् । वा॒चः स॒त्य-म॑शीमहि । प॒शू॒नाꣳ रू॒पमन्न॑स्य । यशः॒ श्रीः श्र॑यतां॒ मयि॑ ॥ यथा॒ ऽहम॒स्या अतृ॑पꣳ स्त्रि॒यै पुमान्॑ । यथा॒ स्त्री तृप्य॑ति पुꣳ॒॒सि प्रि॒ये प्रि॒या । ए॒वं भग॑स्य तृप्याणि \textbf{ 56} \newline
                  \newline
                                \textbf{ TB 2.4.6.7} \newline
                  य॒ज्ञ्स्य॒ काम्यः॑ प्रि॒यः ॥ ददा॒मीत्य॒ग्निर्व॑दति । तथेति॑ वा॒युरा॑ह॒ तत् । हन्तेति॑ स॒त्यं च॒न्द्रमाः᳚ । आ॒दि॒त्यः स॒त्यमोमिति॑ ॥ आप॒स्तथ्-स॒त्यमाभ॑रन्न् । यशो॑ य॒ज्ञ्स्य॒ दक्षि॑णाम् । अ॒सौ मे॒ कामः॒ समृ॑द्ध्यताम् ॥ न हि स्पश॒मवि॑दन्-न॒न्यम॒स्मात् । वै॒श्वा॒न॒रात् पु॑र ए॒तार॑म॒ग्नेः \textbf{ 57} \newline
                  \newline
                                \textbf{ TB 2.4.6.8} \newline
                  अथे॑ममन्थन्न॒मृत॒ममू॑राः । वै॒श्वा॒न॒रं क्षे᳚त्र॒जित्या॑य दे॒वाः ॥ येषा॑मि॒मे पूर्वे॒ अर्मा॑स॒ आसन्न्॑ । अ॒यू॒पाः सद्म॒ विभृ॑ता पु॒रूणि॑ । वैश्वा॑नर॒ त्वया॒ ते नु॒त्ताः । पृ॒थि॒वी-म॒न्याम॒भित॑स्थु॒र्-जना॑सः ॥ पृ॒थि॒वीं मा॒तरं॑ म॒हीम् । अ॒न्तरि॑क्ष॒-मुप॑ब्रुवे । बृ॒ह॒तीमू॒तये॒ दिव᳚म् ॥ विश्वं॑ बिभर्ति पृथि॒वी \textbf{ 58} \newline
                  \newline
                                \textbf{ TB 2.4.6.9} \newline
                  अ॒न्तरि॑क्षं॒ ॅविप॑प्रथे । दु॒हे द्यौर् बृ॑ह॒ती पयः॑ ॥ न ता न॑शन्ति॒ न द॑भाति॒ तस्क॑रः । नैना॑ अमि॒त्रो व्यथि॒राद॑धर्.षति । दे॒वाꣳश्च॒ याभि॒र् यज॑ते॒ ददा॑ति च । ज्योगित्ताभिः॑ सचते॒ गोप॑तिः स॒ह ॥ न ता अर्वा॑ रे॒णुक॑काटो अश्नुते । न सꣳ॑ स्कृत॒त्र-मुप॑यन्ति॒ ता अ॒भि । उ॒रु॒गा॒यमभ॑यं॒ तस्य॒ ता अनु॑ । गावो॒ मर्त्य॑स्य॒ विच॑रन्ति॒ यज्व॑नः । \textbf{ 59} \newline
                  \newline
                                \textbf{ TB 2.4.6.10} \newline
                  रात्री॒ व्य॑ख्यदाय॒ती । पु॒रु॒त्रा दे॒व्य॑क्षभिः॑ । विश्वा॒ अधि॒ श्रियो॑ऽधित ॥ उप॑ ते॒ गा इ॒वाक॑रम् । वृ॒णी॒ष्व दु॑हितर् दिवः । रात्री॒ स्तोमं॒ न जि॒ग्युषी᳚ ॥ दे॒वीं ॅवाच॑मजनयन्त दे॒वाः । तां ॅवि॒श्वरू॑पाः प॒शवो॑ वदन्ति । सा नो॑ म॒न्द्रेष॒मूर्जं॒ दुहा॑ना । धे॒नुर्वाग॒स्मानुप॒ सुष्टु॒तैतु॑ । \textbf{ 60} \newline
                  \newline
                                \textbf{ TB 2.4.6.11} \newline
                  यद् वाग्वद॑न्त्य-विचेत॒नानि॑ । राष्ट्री॑ दे॒वानां᳚ निष॒साद॑ म॒न्द्रा । चत॑स्र॒ ऊर्जं॑ दुदुहे॒ पयाꣳ॑सि । क्व॑ स्वि दस्याः पर॒मं ज॑गाम ॥ गौ॒री मि॑माय सलि॒लानि॒ तक्ष॑ती । एक॑पदी द्वि॒पदी॒ सा चतु॑ष्पदी । अ॒ष्टाप॑दी॒ नव॑पदी बभू॒वुषी᳚ । स॒हस्रा᳚क्षरा पर॒मे व्यो॑मन्न् ॥ तस्याꣳ॑ समु॒द्रा अधि॒ वि क्ष॑रन्ति । तेन॑ जीवन्ति प्र॒दिश॒श्चत॑स्रः \textbf{ 61} \newline
                  \newline
                                \textbf{ TB 2.4.6.12} \newline
                  ततः॑ क्षरत्य॒क्षर᳚म् । तद्-विश्व॒-मुप॑जीवति ॥ इन्द्रा॒ सूरा॑ ज॒नय॑न्-वि॒श्वक॑र्मा । म॒रुत्वाꣳ॑ अस्तु ग॒णवा᳚न्थ्-सजा॒तवान्॑ । अ॒स्य स्नु॒षा-श्वशु॑रस्य॒ प्रशि॑ष्टिम् । स॒पत्ना॒ वाचं॒ मन॑सा॒ उपा॑सताम् ॥ इन्द्रः॒ सूरो॑ अतर॒द्-रजाꣳ॑सि । स्नु॒षा स॒पत्नाः॒ श्वशु॑रो॒ऽयम॑स्तु । अ॒यꣳ शत्रू᳚ञ्जयतु॒ जर्.हृ॑षाणः । अ॒यं ॅवाजं॑ जयतु॒ वाज॑सातौ ( ) ॥ अ॒ग्निः क्ष॑त्र॒भृदनि॑भृष्ट॒मोजः॑ । स॒ह॒स्रियो॑ दीप्यता॒-मप्र॑युच्छन्न् । वि॒भ्राज॑मानः समिधा॒न उ॒ग्रः । आऽन्तरि॑क्षमरुह॒द-ग॒न्द्याम् । \textbf{ 62} \newline
                  \newline
                                    (धा॒रय॑न् - पुरो॒डाशं॒ - बृह॒स्पतिं॑ - ज॒घन॑च्युति - मान॒न्दो - भग॑स्य तृप्या - ण्य॒ग्नेः - पृ॑थि॒वी - यज्व॑न - एतु - प्र॒दिश॒श्चत॑स्रो॒ - वाज॑सातौ च॒त्वारि॑ च) \textbf{(A6)} \newline \newline
                \textbf{ 2.4.7      अनुवाकं   7 - उपहोमाः} \newline
                                \textbf{ TB 2.4.7.1} \newline
                  वृषा᳚ ऽस्यꣳ॒॒शुर्-वृ॑ष॒भाय॑ गृह्यसे । वृषा॒ऽयमु॒ग्रो नृ॒चक्ष॑से । दि॒व्यः क॑र्म॒ण्यो॑ हि॒तो बृ॒हन्नाम॑ । वृ॒ष॒भस्य॒ या क॒कुत् ॥ वि॒षू॒वान्. वि॑ष्णो भवतु । अ॒यं ॅयो मा॑म॒को वृषा᳚ । अथो॒ इन्द्र॑ इव दे॒वेभ्यः॑ । विब्र॑वीतु॒ जने᳚भ्यः ॥ आयु॑ष्मन्तं॒ ॅवर्च॑स्वन्तम् । अथो॒ अधि॑पतिं ॅवि॒शाम् \textbf{ 63} \newline
                  \newline
                                \textbf{ TB 2.4.7.2} \newline
                  अ॒स्याः पृ॑थि॒व्या अद्ध्य॑क्षम् । इ॒ममि॑न्द्र वृष॒भं कृ॑णु ॥ यः सु॒शृङ्गः॑ सुवृष॒भः । क॒ल्याणो॒ द्रोण॒ आहि॑तः । कार्.षी॑वलप्रगाणेन । वृ॒ष॒भेण॑ यजामहे ॥ वृ॒ष॒भेण॒ यज॑मानाः । अक्रू॑रेणेव स॒र्पिषा᳚ । मृध॑श्च॒ सर्वा॒ इन्द्रे॑ण । पृत॑नाश्च जयामसि । \textbf{ 64} \newline
                  \newline
                                \textbf{ TB 2.4.7.3} \newline
                  यस्या॒यमृ॑ष॒भो ह॒विः । इन्द्रा॑य परिणी॒यते᳚ । जया॑ति॒ शत्रु॑मा॒यन्त᳚म् । अथो॑ हन्ति पृतन्य॒तः ॥ नृ॒णामह॑ प्र॒णीरस॑त् । अग्र॑ उद्भिन्द॒ताम॑सत् ॥ इन्द्र॒ शुष्मं॑ त॒नुवा॒ मेर॑यस्व । नी॒चा विश्वा॑ अ॒भिति॑ष्ठा॒भिमा॑तीः । नि शृ॑णीह्याबा॒धं ॅयो नो॒ अस्ति॑ । उ॒रुं नो॑ लो॒कं कृ॑णुहि जीरदानो । \textbf{ 65} \newline
                  \newline
                                \textbf{ TB 2.4.7.4} \newline
                  प्रेह्य॒भिप्रेहि॒ प्रभ॑रा॒ सह॑स्व । मा विवे॑नो॒ वि शृ॑णुष्वा॒ जने॑षु । उदी॑डि॒तो वृ॑षभ॒ तिष्ठ॒ शुष्मैः᳚ । इन्द्र॒ शत्रू᳚न् पु॒रो अ॒स्माक॑ युद्ध्य ॥ अग्ने॒ जेता॒ त्वं ज॑य । शत्रू᳚न्थ् सहस॒ ओज॑सा । वि शत्रू॒न्॒. विमृधो॑ नुद ॥ ए॒तं ते॒ स्तोमं॑ तुविजात॒ विप्रः॑ । रथं॒ न धीरः॒ स्वपा॑ अतक्षम् । यदीद॑ग्ने॒ प्रति॒ त्वं दे॑व॒ हर्याः᳚ \textbf{ 66} \newline
                  \newline
                                \textbf{ TB 2.4.7.5} \newline
                  सुव॑र्वतीर॒प ए॑ना जयेम ॥ यो घृ॒तेना॒भिमा॑नितः । इन्द्र॒ जैत्रा॑य जज्ञिषे । स नः॒ संका॑सु पारय । पृ॒त॒ना॒साह्ये॑षु च ॥ इन्द्रो॑ जिगाय पृथि॒वीम् । अ॒न्तरि॑क्षꣳ॒॒ सुव॑र्म॒हत् । वृ॒त्र॒हा पु॑रु॒चेत॑नः ॥ इन्द्रो॑ जिगाय॒ सह॑सा॒ सहाꣳ॑सि । इन्द्रो॑ जिगाय॒ पृत॑नानि॒ विश्वा᳚ \textbf{ 67} \newline
                  \newline
                                \textbf{ TB 2.4.7.6} \newline
                  इन्द्रो॑ जा॒तो वि पुरो॑ रुरोज । स नः॑ पर॒स्पा वरि॑वः कृणोतु ॥ अ॒यं कृ॒त्नुरगृ॑भीतः । वि॒श्व॒जि-दु॒द्भिदिथ्-सोमः॑ । ऋषि॒र् विप्रः॒ काव्ये॑न ॥ वा॒युर॑ग्रे॒गा य॑ज्ञ्॒प्रीः । सा॒कं ग॒न्मन॑सा य॒ज्ञ्म् । शि॒वो नि॒युद्भिः॑ शि॒वाभिः॑ ॥ वायो॑ शु॒क्रो अ॑यामि ते । मद्ध्वो॒ अग्रं॒ दिवि॑ष्टिषु \textbf{ 68} \newline
                  \newline
                                \textbf{ TB 2.4.7.7} \newline
                  आया॑ हि॒ सोम॑पीतये । स्वा॒रु॒हो दे॑व नि॒युत्व॑ता ॥ इ॒ममि॑न्द्र वर्द्धय क्ष॒त्रिया॑णाम् । अ॒यं ॅवि॒शां ॅवि॒श्पति॑रस्तु॒ राजा᳚ । अ॒स्मा इ॑न्द्र॒ महि॒ वर्चाꣳ॑सि धेहि । अ॒व॒र्चसं॑ कृणुहि॒ शत्रु॑मस्य ॥ इ॒ममाभ॑ज॒ ग्रामे॒ अश्वे॑षु॒ गोषु॑ । निर॒मुं भ॑ज॒ यो॑ऽमित्रो॑ अस्य । वर्.ष्म॑न् क्ष॒त्रस्य॑ क॒कुभि॑ श्रयस्व । ततो॑ न उ॒ग्रो विभ॑जा॒ वसू॑नि ॥ 69(10)ट्.भ्.2.4.7.8अ॒स्मे द्या॑वापृथिवी॒ भूरि॑ वा॒मम् । सं दु॑हाथां घर्म॒दुघे॑व धे॒नुः । अ॒यꣳ राजा᳚ प्रि॒य इन्द्र॑स्य भूयात् । प्रि॒यो गवा॒मोष॑धीना-मु॒तापाम् ॥ यु॒नज्मि॑ त उत्त॒राव॑न्त॒-मिन्द्र᳚म् । येन॒ जया॑सि॒ न परा॒जया॑सै । स त्वा॑ ऽकरेकवृष॒भꣳ स्वाना᳚म् । अथो॑ राजन्नुत्त॒मं मा॑न॒वाना᳚म् ॥ उत्त॑र॒स्त्वमध॑रे ते स॒पत्नाः᳚ । एक॑वृषा॒ इन्द्र॑सखा जिगी॒वान् \textbf{ 70} \newline
                  \newline
                                \textbf{ TB } \newline
                   \textbf{ 0} \newline
                  \newline
                                \textbf{ TB 2.4.7.9} \newline
                  विश्वा॒ आशाः॒ पृत॑नाः स॒जंय॒ञ्जयन्न्॑ । अ॒भिति॑ष्ठ शत्रूय॒तः स॑हस्व ॥ तुभ्यं॑ भरन्ति क्षि॒तयो॑ यविष्ठ । ब॒लिम॑ग्ने॒ अन्ति॑ त॒ ओत दू॒रात् । आ भन्दि॑ष्ठस्य सुम॒तिं चि॑किद्धि । बृ॒हत्ते॑ अग्ने॒ महि॒ शर्म॑ भ॒द्रम् ॥ यो दे॒ह्यो अन॑मयद्वध॒स्नैः । यो अर्य॑पत्नी-रु॒षस॑श्च॒कार॑ । स नि॒रुद्ध्या॒ नहु॑षो य॒ह्वो अ॒ग्निः । विश॑श्चक्रे बलि॒हृतः॒ सहो॑भिः । \textbf{ 71} \newline
                  \newline
                                \textbf{ TB 2.4.7.10} \newline
                  प्र स॒द्यो अ॑ग्ने॒ अत्य᳚ष्य॒न्यान् । आ॒विर्यस्मै॒ चारु॑तरो ब॒भूथ॑ । ई॒डेन्यो॑ वपु॒ष्यो॑ वि॒भावा᳚ । प्रि॒यो वि॒शामति॑थि॒र्-मानु॑षीणाम् ॥ ब्रह्म॑ ज्येष्ठा वी॒र्या॑ संभृ॑तानि । ब्रह्माग्रे॒ ज्येष्ठं॒ दिव॒मात॑तान । ऋ॒तस्य॒ ब्रह्म॑ प्रथ॒मोत ज॑ज्ञे । तेना॑र्.हति॒ ब्रह्म॑णा॒ स्पर्द्धि॑तुं॒ कः ॥ ब्रह्म॒ स्रुचो॑ घृ॒तव॑तीः । ब्रह्म॑णा॒ स्वर॑वो मि॒ताः \textbf{ 72} \newline
                  \newline
                                \textbf{ TB 2.4.7.11} \newline
                  ब्रह्म॑ य॒ज्ञ्स्य॒ तन्त॑वः । ऋ॒त्विजो॒ ये ह॑वि॒ष्कृतः॑ ॥ शृङ्गा॑णी॒वेच्छृ॒ङ्गिणाꣳ॒॒ संद॑दृश्रिरे । च॒षाल॑वन्तः॒ स्वर॑वः पृथि॒व्याम् । ते दे॒वासः॒ स्वर॑वस्तस्थि॒वाꣳसः॑ । नमः॒ सखि॑भ्यः स॒न्नान्मा ऽव॑गात ॥ अ॒भि॒भूर॒ग्निर॑तर॒द्-रजाꣳ॑सि । स्पृधो॑ वि॒हत्य॒ पृत॑ना अभि॒श्रीः । जु॒षा॒णो म॒ आहु॑तिं मा महिष्ट । ह॒त्वा स॒पत्ना॒न्॒. वरि॑वस्करन्नः ( ) ॥ ईशा॑नं त्वा॒ भुव॑नाना-मभि॒श्रिय᳚म् । स्तौम्य॑ग्न उरु॒कृतꣳ॑ सु॒वीर᳚म् । ह॒विर् जु॑षा॒णः स॒पत्नाꣳ॑ अभि॒भूर॑सि । ज॒हि शत्रूꣳ॒॒रप॒ मृधो॑ नुदस्व । \textbf{ 73} \newline
                  \newline
                                    (वि॒शां - ज॑यामसि - जीरदानो॒ - हर्या॒ - विश्वा॒ - दिवि॑ष्टिषु॒ - वसू॑नि - जिगी॒वान्थ् - सहो॑भिर् - मि॒ता - न॑ श्च॒त्वारि॑ च) \textbf{(A7)} \newline \newline
                \textbf{ 2.4.8      अनुवाकं   8 - उपहोमाः} \newline
                                \textbf{ TB 2.4.8.1} \newline
                  स प्र॑त्न॒वन्नवी॑यसा । अग्ने᳚ द्यु॒म्नेन॑ स॒म्ॅयता᳚ । बृ॒हत् ततन्थ भा॒नुना᳚ ॥ नवं॒ नु स्तोम॑म॒ग्नये᳚ । दि॒वः श्ये॒नाय॑ जीजनम् । वसोः᳚ कु॒विद्व॒नाति॑ नः ॥ स्वा॒रु॒हा यस्य॒ श्रियो॑ दृ॒शे । र॒यिर्वी॒रव॑तो यथा । अग्रे॑ य॒ज्ञ्स्य॒ चेत॑तः ॥ अदा᳚भ्यः पुर ए॒ता \textbf{ 74} \newline
                  \newline
                                \textbf{ TB 2.4.8.2} \newline
                  अ॒ग्निर्वि॒शां मानु॑षीणाम् । तूर्णी॒ रथः॒ सदा॒ नवः॑ ॥ नवꣳ॒॒ सोमा॑य वा॒जिने᳚ । आज्यं॒ पय॑सोऽजनि । जुष्टꣳ॒॒ शुचि॑तमं॒ ॅवसु॑ ॥ नवꣳ॑ सोम जुषस्व नः । पी॒यूष॑स्ये॒ह तृ॑प्णुहि । यस्ते॑ भा॒ग ऋ॒ताव॒यम् ॥ नव॑स्य सोम ते व॒यम् । आ सु॑म॒तिं ॅवृ॑णीमहे \textbf{ 75} \newline
                  \newline
                                \textbf{ TB 2.4.8.3} \newline
                  स नो॑ रास्व सह॒स्रिणः॑ ॥ नवꣳ॑ ह॒विर्जु॑षस्व नः । ऋ॒तुभिः॑ सोम॒ भूत॑मम् । तद॒ङ्ग प्रति॑हर्य नः । राजन᳚थ् सोम स्व॒स्तये᳚ ॥ नवꣳ॒॒ स्तोमं॒ नवꣳ॑ ह॒विः । इ॒न्द्रा॒ग्निभ्यां॒ निवे॑दय । तज्जु॑षेताꣳ॒॒ सचे॑तसा ॥ शुचिं॒ नु स्तोमं॒ नव॑जातम॒द्य । इन्द्रा᳚ग्नी वृत्रहणा जु॒षेथा᳚म् \textbf{ 76} \newline
                  \newline
                                \textbf{ TB 2.4.8.4} \newline
                  उ॒भा हि वाꣳ॑ सु॒हवा॒ जोह॑वीमि । ता वाजꣳ॑ स॒द्य उ॑श॒ते धेष्ठा᳚ ॥ अ॒ग्निरिन्द्रो॒ नव॑स्य नः । अ॒स्य ह॒व्यस्य॑ तृप्यताम् । इ॒ह दे॒वौ स॑ह॒स्रिणौ᳚ ॥ य॒ज्ञ्ं न॒ आ हि गच्छ॑ताम् । वसु॑मन्तꣳ सुव॒र्विद᳚म् । अ॒स्य ह॒व्यस्य॑ तृप्यताम् । अ॒ग्निरिन्द्रो॒ नव॑स्य नः ॥ विश्वा᳚न् दे॒वाꣳ स्त॑र्पयत \textbf{ 77} \newline
                  \newline
                                \textbf{ TB 2.4.8.5} \newline
                  ह॒विषो॒ऽस्य नव॑स्य नः । सु॒व॒र्विदो॒ हि ज॑ज्ञि॒रे ॥ एदं ब॒र्॒.हिः सु॒ष्टरी॑मा॒ नवे॑न । अ॒यं ॅय॒ज्ञो यज॑मानस्य भा॒गः । अ॒यं ब॑भूव॒ भुव॑नस्य॒ गर्भः॑ । विश्वे॑ दे॒वा इ॒दम॒द्याग॑मिष्ठाः ॥ इ॒मे नु द्यावा॑पृथि॒वी स॒मीची᳚ । त॒न्वा॒ने य॒ज्ञ्ं पु॑रु॒पेश॑सं धि॒या । आऽस्मै॑ पृणीतां॒ भुव॑नानि॒ विश्वा᳚ । प्र॒जां पुष्टि॑म॒मृतं॒ नवे॑न । \textbf{ 78} \newline
                  \newline
                                \textbf{ TB 2.4.8.6} \newline
                  इ॒मे धे॒नू अ॒मृतं॒ ॅये दु॒हाते᳚ । पय॑स्व-त्युत्त॒रामे॑तु॒ पुष्टिः॑ । इ॒मं ॅय॒ज्ञ्ं जु॒षमा॑णे॒ नवे॑न । स॒मीची॒ द्यावा॑पृथि॒वी घृ॒ताची᳚ ॥ यवि॑ष्ठो हव्य॒वाह॑नः । चि॒त्रभा॑नुर्-घृ॒तासु॑तिः । नव॑जातो॒ विरो॑चसे । अग्ने॒ तत्ते॑ महित्व॒नम् ॥ त्वम॑ग्ने दे॒वता᳚भ्यः । भा॒गे दे॑व॒ न मी॑यसे \textbf{ 79} \newline
                  \newline
                                \textbf{ TB 2.4.8.7} \newline
                  स ए॑ना वि॒द्वान्. य॑क्ष्यसि । नवꣳ॒॒ स्तोमं॑ जुषस्व नः ॥ अ॒ग्निः प्र॑थ॒मः प्राश्ना॑तु । स हि वेद॒ यथा॑ ह॒विः । शि॒वा अ॒स्मभ्य॒-मोष॑धीः । कृ॒णोतु॑ वि॒श्वच॑र्.षणिः ॥ भ॒द्रान्नः॒ श्रेयः॒ सम॑नैष्ट देवाः । त्वया॑ऽव॒सेन॒ सम॑शीमहि त्वा । स नो॑ मयो॒भूः पि॑तो॒ आवि॑शस्व । शं तो॒काय॑ त॒नुवे᳚ स्यो॒नः ( ) ॥ ए॒तमु॒ त्यं मधु॑ना॒ सम्ॅयु॑तं॒ ॅयव᳚म् । सर॑स्वत्या॒ अधि॑म॒नाव॑चर्कृषुः । इन्द्र॑ आसी॒थ् सीर॑पतिः श॒तक्र॑तुः । की॒नाशा॑ आसन् म॒रुतः॑ सु॒दान॑वः । \textbf{ 80} \newline
                  \newline
                                    (पु॒र॒ ए॒ता - वृ॑णीमहे - जु॒षेथां᳚-तर्पय-ता॒मृतं॒ नवे॑न-मीयसे-स्यो॒न श्च॒त्वारि॑ च) \textbf{(A8)} \newline \newline
                \textbf{PrapAtaka Korvai with starting  words of 1 to 8 anuvAkams :-} \newline
        (जुष्ट॒ - श्चक्षु॑षो॒ - जुष्टी॑ नरो - नक्तं जा॒ता - वृषा॒ स - उ॒त नो॒ - वृषा᳚ऽस्यꣳ॒॒शुः - स प्र॑त्न॒वद॒ष्टौ) \newline

        \textbf{korvai with starting words of 1, 11, 21 series of :-} \newline
        (जुष्टो॑ - म॒न्युर् भगो॒ - जुष्टी॑ नरो॒ - हरि॑वर्पसं॒ गिरः॒ - शिप्रि॑न् वाजाना - मु॒त नो॒ - यद् वाग्वद॑न्ती॒ - विश्वा॒ आशा॒ अशी॑तिः ) \newline

        \textbf{first and last  word - 2nd aShTakam 3rd prapATakam :-} \newline
        (जुष्टः॑ - सु॒दान॑वः ) \newline 

       

        ॥ हरिः॑ ॐ ॥
॥ कृष्ण यजुर्वेदीय तैत्तिरीय ब्राह्मणे द्वितीयाष्टके चतुर्थः प्रपाठकः समाप्तः ॥
=======================


Appendix
ट्.भ्.2.4.3.10 "ए॒ष ब्र॒ह्मा{1}" 
ए॒ष ब्र॒ह्मा य ऋ॒त्वियः॑ । इन्द्रो॒ नाम॑ श्रु॒तो ग॒णे ॥
(Appearing in TB 3-7-9-5)

============================== \newline
        \pagebreak
        
        
        
     \addcontentsline{toc}{section}{ 2.5     द्वितीयाष्टके पञ्चमः प्रपाठकः - उपहोमशेषः}
     \markright{ 2.5     द्वितीयाष्टके पञ्चमः प्रपाठकः - उपहोमशेषः \hfill https://www.vedavms.in \hfill}
     \section*{ 2.5     द्वितीयाष्टके पञ्चमः प्रपाठकः - उपहोमशेषः }
                \textbf{ 2.5.1     अनुवाकं   1 -  उपहोमाः} \newline
                                \textbf{ TB 2.5.1.1} \newline
                  प्रा॒णो र॑क्षति॒ विश्व॒मेज॑त् । इर्यो॑ भू॒त्वा ब॑हु॒धा ब॒हूनि॑ । स इथ्सर्वं॒ ॅव्या॑नशे ॥ यो दे॒वो दे॒वेषु॑ वि॒भूर॒न्तः । आवृ॑दू॒दात् क्षेत्रिय॑द्ध्व॒गद्वृषा᳚ । तमित् प्रा॒णं मन॒सो प॑शिक्षत । अग्रं॑ दे॒वाना॑-मि॒दम॑त्तु नो ह॒विः ॥ मन॑स॒श्चित्ते॒दम् । भू॒तं भव्यं॑ च गुप्यते । तद्धि दे॒वेष्व॑ग्रि॒यम् । \textbf{ 1} \newline
                  \newline
                                \textbf{ TB 2.5.1.2} \newline
                  आ न॑ एतु पुरश्च॒रम् । स॒ह दे॒वैरि॒मꣳ हव᳚म् । मनः॒ श्रेय॑सि श्रेयसि । कर्म॑न्. य॒ज्ञ्प॑तिं॒ दध॑त् ॥ जु॒षतां᳚ मे॒ वागि॒दꣳ ह॒विः । वि॒राड्दे॒वी पु॒रोहि॑ता । ह॒व्य॒वाडन॑ पायिनी ॥ यया॑ रू॒पाणि॑ बहु॒धा वद॑न्ति । पेशाꣳ॑सि दे॒वाः प॑र॒मे ज॒नित्रे᳚ । सानो॑ वि॒राडन॑पस्फुरन्ती \textbf{ 2} \newline
                  \newline
                                \textbf{ TB 2.5.1.3} \newline
                  वाग्दे॒वी जु॑षतामि॒दꣳ ह॒विः ॥ चक्षु॑र् दे॒वानां॒ ज्योति॑र॒मृते॒ न्य॑क्तम् । अ॒स्य वि॒ज्ञाना॑य बहु॒धा निधी॑यते । तस्य॑ सु॒म्न-म॑शीमहि । मा नो॑ हासीद्-विचक्ष॒णम् ॥ आयु॒रिन्नः॒ प्रती᳚र्यताम् । अन॑न्धा॒-श्चक्षु॑षा व॒यम् । जी॒वा ज्योति॑रशीमहि । सुव॒र्ज्योति॑-रु॒तामृत᳚म् ॥ श्रोत्रे॑ण भ॒द्र मु॒त शृ॑ण्वन्ति स॒त्यम् ( ) ।  श्रोत्रे॑ण॒ वाचं॑ बहु॒धोद्य मा॑नाम् । श्रोत्रे॑ण॒ मोद॑श्च॒ मह॑श्च श्रूयते । श्रोत्रे॑ण॒ सर्वा॒ दिश॒ आशृ॑णोमि ॥ येन॒ प्राच्या॑ उ॒त द॑क्षि॒णा । प्र॒तीच्यै॑ दि॒शः शृ॒ण्वन्त्यु॑त्त॒रात् । तदिच्छ्रोत्रं॑ बहु॒धोद्यमा॑नम् । अ॒रान्न ने॒मिः परि॒ सर्वं॑ बभूव । \textbf{ 3} \newline
                  \newline
                                    (अ॒ग्रि॒ - यमन॑पस्फुरन्ती - स॒त्यꣳ स॒प्त च॑) \textbf{(A1)} \newline \newline
                \textbf{ 2.5.2     अनुवाकं   2 -  उपहोमाः} \newline
                                \textbf{ TB 2.5.2.1} \newline
                  उ॒देहि॑ वाजि॒न्यो अ॑स्य॒फ्स्व॑न्तः । इ॒दꣳ रा॒ष्ट्रमावि॑श सू॒नृता॑वत् । यो रोहि॑तो॒ विश्व॑मि॒दं ज॒जान॑ । स नो॑ रा॒ष्ट्रेषु॒ सुधि॑तान्दधातु ॥ रोहꣳ॑ रोहꣳ॒॒ रोहि॑त॒ आरु॑रोह । प्र॒जाभि॒र्वृद्धिं॑ ज॒नुषा॑मु॒पस्थ᳚म् । ताभिः॒ सꣳर॑ब्धो अविद॒थ्षडु॒र्वीः । गा॒तुं प्र॒पश्य॑न्नि॒ह रा॒ष्ट्रमाहाः᳚ ॥ आहा॑र्.षीद्-रा॒ष्ट्रमि॒ह रोहि॑तः । मृधो॒ व्या᳚स्थ॒दभ॑यं नो अस्तु \textbf{ 4} \newline
                  \newline
                                \textbf{ TB 2.5.2.2} \newline
                  अ॒स्मभ्यं॑ द्यावापृथिवी॒ शक्व॑रीभिः । रा॒ष्ट्रं दु॑हाथामि॒ह रे॒वती॑भिः ॥ विम॑मर्.श॒ रोहि॑तो वि॒श्वरू॑पः । स॒मा॒च॒क्रा॒णः प्र॒रुहो॒ रुह॑श्च । दिव॑गं॒त्वाय॑ मह॒ता म॑हि॒म्ना । वि नो॑ रा॒ष्ट्रमु॑नत्तु॒ पय॑सा॒ स्वेन॑ ॥ यास्ते॒ विश॒स्तप॑सा संबभू॒वुः । गा॒य॒त्रं ॅव॒थ्समनु॒ तास्त॒ आगुः॑ । तास्त्वा वि॑शन्तु॒ मह॑सा॒ स्वेन॑ । सं मा॑ता पु॒त्रो अ॒भ्ये॑तु॒ रोहि॑तः । \textbf{ 5} \newline
                  \newline
                                \textbf{ TB 2.5.2.3} \newline
                  यू॒यमु॑ग्रा मरुतः पृश्निमातरः । इन्द्रे॑ण स॒युजा॒ प्रमृ॑णीथ॒ शत्रून्॑ । आ वो॒ रोहि॑तो अशृणोदभिद्यवः । त्रिस॑प्तासो मरुतः स्वादुसंमुदः ॥ रोहि॑तो॒ द्यावा॑पृथि॒वी ज॑जान । तस्मिꣳ॒॒स्तन्तुं॑ परमे॒ष्ठी त॑तान । तस्मि॑ञ्छिश्रिये अ॒ज एक॑पात् । अदृꣳ॑ ह॒द्द्यावा॑पृथि॒वी बले॑न ॥ रोहि॑तो॒ द्यावा॑पृथि॒वी अ॑दृꣳहत् । तेन॒ सुवः॑ स्तभि॒तं तेन॒ नाकः॑ \textbf{ 6} \newline
                  \newline
                                \textbf{ TB 2.5.2.4} \newline
                  सो अ॒न्तरि॑क्षे॒ रज॑सो वि॒मानः॑ । तेन॑ दे॒वाः सुव॒रन्व॑विन्दन्न् ॥ सु॒शेवं॑ त्वा भा॒नवो॑ दीदि॒वाꣳस᳚म् । सम॑ग्रासो जु॒ह्वो॑ जातवेदः । उ॒क्षन्ति॑ त्वा वा॒जिन॒मा घृ॒तेन॑ । सꣳस॑मग्ने युवसे॒ भोज॑नानि ॥ अग्ने॒ शर्द्ध॑ मह॒ते सौभ॑गाय । तव॑ द्यु॒म्ना-न्यु॑त्त॒मानि॑ सन्तु । सं जा᳚स्प॒त्यꣳ सु॒यम॒माकृ॑णुष्व । श॒त्रू॒य॒ताम॒भिति॑ष्ठा॒ महा॑सि ( ) । \textbf{ 7} \newline
                  \newline
                                    (अ॒- स्त्वे॒तु॒ रोहि॑तो॒ - नाको॒ - महाꣳ॑सि) \textbf{(A2)} \newline \newline
                \textbf{ 2.5.3     अनुवाकं   3 -  उपहोमाः} \newline
                                \textbf{ TB 2.5.3.1} \newline
                  पुन॑र्न॒ इन्द्रो॑ म॒घवा॑ ददातु । धना॑नि श॒क्रो घन्यः॑ सु॒राधाः᳚ । अ॒र्वा॒चीनं॑ कृणुतां ॅयाचि॒तो मनः॑ । श्रु॒ष्टी नो॑ अ॒स्य ह॒विषो॑ जुषा॒णः ॥ यानि॑ नो जि॒नन् धना॑नि । ज॒हर्थ॑ शूर म॒न्युना᳚ । इन्द्रानु॑विन्द न॒स्तानि॑ । अ॒नेन॑ ह॒विषा॒ पुनः॑ ॥ इन्द्र॒ आशा᳚भ्यः॒ परि॑ । सर्वा॒भ्योऽभ॑यं करत् \textbf{ 8} \newline
                  \newline
                                \textbf{ TB 2.5.3.2} \newline
                  जेता॒ शत्रू॒न्॒. विच॑र्.षणिः ॥ आकू᳚त्यै त्वा॒ कामा॑य त्वा स॒मृधे᳚ त्वा । पु॒रो द॑धे अमृत॒त्वाय॑ जी॒वसे᳚ ॥ आकू॑तिम॒स्याव॑से । काम॑मस्य॒ समृ॑द्ध्यै । इन्द्र॑स्य युञ्जते॒ धियः॑ ॥ आकू॑तिं दे॒वीं मन॑सः पु॒रो द॑धे । य॒ज्ञ्स्य॑ मा॒ता सु॒हवा॑ मे अस्तु । यदि॒च्छामि॒ मन॑सा॒ सका॑मः । वि॒देय॑मेन॒द्धृद॑ये॒ निवि॑ष्टम् । \textbf{ 9} \newline
                  \newline
                                \textbf{ TB 2.5.3.3} \newline
                  सेद॒ग्निर॒ग्नीꣳ रत्ये᳚त्य॒न्यान् । यत्र॑ वा॒जी तन॑यो वी॒डुपा॑णिः । स॒हस्र॑पाथा अ॒क्षरा॑ स॒मेति॑ ॥ आशा॑नां त्वा ऽऽशापा॒लेभ्यः॑ । च॒तुर्भ्यो॑ अ॒मृते᳚भ्यः । इ॒दं भू॒तस्याद्ध्य॑क्षेभ्यः । वि॒धेम॑ ह॒विषा॑ व॒यम् ॥ विश्वा॒ आशा॒ मधु॑ना॒ सꣳसृ॑जामि । अ॒न॒मी॒वा आप॒ ओष॑धयो भवन्तु । अ॒यं ॅयज॑मानो॒ मृधो॒ व्य॑स्यताम् ( ) \textbf{ 10} \newline
                  \newline
                                \textbf{ TB 2.5.3.4} \newline
                  अगृ॑भीताः प॒शवः॑ सन्तु॒ सर्वे᳚ ॥ अ॒ग्निः सोमो॒ वरु॑णो मि॒त्र इन्द्रः॑ । बृह॒स्पतिः॑ सवि॒ता यः स॑ह॒स्री । पू॒षा नो॒ गोभि॒रव॑सा॒ सर॑स्वती । त्वष्टा॑ रू॒पाणि॒ सम॑नक्तु य॒ज्ञिः ॥ त्वष्टा॑ रू॒पाणि॒ दध॑ती॒ सर॑स्वती । पू॒षा भगꣳ॑ सवि॒ता नो॑ ददातु । बृह॒स्पति॒र्-दद॒दिन्द्रः॑ स॒हस्र᳚म् । मि॒त्रो दा॒ता वरु॑णः॒ सोमो॑ अ॒ग्निः । \textbf{ 11} \newline
                  \newline
                                    (क॒र॒न् - निवि॑ष्ट - मस्यतां॒ - +नव॑ च) \textbf{(A3)} \newline \newline
                \textbf{ 2.5.4     अनुवाकं   4 -  उपहोमाः} \newline
                                \textbf{ TB 2.5.4.1} \newline
                  आ नो॑ भर॒ भग॑मिन्द्र द्यु॒मन्त᳚म् । नि ते॑ दे॒ष्णस्य॑ धीमहि प्ररे॒के । उ॒र्व इ॑व पप्रथे॒ कामो॑ अ॒स्मे । तमापृ॑णा वसुपते॒ वसू॑नाम् ॥ इ॒मं कामं॑ मन्दया॒ गोभि॒रश्वैः᳚ । च॒न्द्रव॑ता॒ राध॑सा प॒प्रथ॑श्च । सु॒व॒र्यवो॑ म॒तिभि॒स्तुभ्यं॒ ॅविप्राः᳚ । इन्द्रा॑य॒ वाहः॑ कुशि॒कासो॑ अक्रन्न् ॥ इन्द्र॑स्य॒ नु वी॒र्या॑णि॒ प्रवो॑चम् । यानि॑ च॒कार॑ प्रथ॒मानि॑ व॒ज्री \textbf{ 12} \newline
                  \newline
                                \textbf{ TB 2.5.4.2} \newline
                  अह॒-न्नहि॒मन्व॒पस्त॑तर्द । प्र व॒क्षणा॑ अभिन॒त् पर्व॑तानाम् ॥ अह॒न्नहिं॒ पर्व॑ते शिश्रिया॒णम् । त्वष्टा᳚ऽस्मै॒ वज्रꣳ॑ स्व॒र्यं॑ ततक्ष । वा॒श्रा इ॑व धे॒नवः॒ स्यन्द॑मानाः । अञ्जः॑ समु॒द्र-मव॑जग्मु॒रापः॑ ॥ वृ॒षा॒यमा॑णोऽवृणीत॒ सोम᳚म् । त्रिक॑द्रुकेष्व-पिबथ् सु॒तस्य॑ । आ साय॑कं म॒घवा॑ दत्त॒ वज्र᳚म् । अह॑न्नेनं प्रथम॒जा मही॑नाम् । \textbf{ 13} \newline
                  \newline
                                \textbf{ TB 2.5.4.3} \newline
                  यदिन्द्राह॑न्-प्रथम॒जा मही॑नाम् । आन्मा॒यिना॒ममि॑नाः॒ प्रोत मा॒याः । आथ् सूर्यं॑ ज॒नय॒न् द्यामु॒षास᳚म् । ता॒दीक्ना॒ शत्रू॒न्न किला॑विविथ्से ॥ अह॑न् वृ॒त्रं ॅवृ॑त्र॒तरं॒ ॅव्यꣳस᳚म् । इन्द्रो॒ वज्रे॑ण मह॒ता व॒धेन॑ । स्कन्धाꣳ॑सीव॒ कुलि॑शेना॒ विवृ॑क्णा । अहिः॑ शयत उप॒पृक्-पृ॑थि॒व्याम् ॥ अ॒यो॒द्ध्येव दु॒र्मद॒ आ हि जु॒ह्वे । म॒हा॒वी॒रं तु॑ विबा॒ध-मृ॑जी॒षम् \textbf{ 14} \newline
                  \newline
                                \textbf{ TB 2.5.4.4} \newline
                  नाता॑रीरस्य॒ समृ॑तिं ॅव॒धाना᳚म् । सꣳ रु॒जानाः᳚ पिपिष॒ इन्द्र॑शत्रुः ॥ विश्वो॒ विहा॑या अर॒तिः । वसु॑र्दधे॒ हस्ते॒ दक्षि॑णे । त॒रणि॒र्न शि॑श्रथत् । श्र॒व॒स्य॑या॒ न शि॑श्रथत् । विश्व॑स्मा॒ इदि॑षुद्ध्य॒से । दे॒व॒त्रा ह॒व्यमूहि॑षे । विश्व॑स्मा॒ इथ्सु॒कृते॒ वार॑मृण्वति । अ॒ग्निर् द्वारा॒ व्यृ॑ण्वति । \textbf{ 15} \newline
                  \newline
                                \textbf{ TB 2.5.4.5} \newline
                  उदु॒ज्जिहा॑नो अ॒भि काम॑मी॒रयन्न्॑ । प्र॒पृ॒ञ्चन् विश्वा॒ भुव॑नानि पू॒र्वथा᳚ । आ के॒तुना॒ सुष॑मिद्धो॒ यजि॑ष्ठः । कामं॑ नो अग्ने अ॒भिह॑र्य दि॒ग्भ्यः ॥ जु॒षा॒णो ह॒व्यम॒मृते॑षु दू॒ढ्यः॑ । आ नो॑ र॒यिं ब॑हु॒लाङ् गोम॑ती॒मिष᳚म् । निधे॑हि॒ यक्ष॑द॒मृते॑षु॒ भूषन्न्॑ ॥ अश्वि॑ना य॒ज्ञ्माग॑तम् । दा॒शुषः॒ पुरु॑दꣳससा । पू॒षा र॑क्षतु नो र॒यिम् । \textbf{ 16} \newline
                  \newline
                                \textbf{ TB 2.5.4.6} \newline
                  इ॒मं ॅय॒ज्ञ्म॒श्विना॑ व॒र्द्धय॑न्ता । इ॒मौ र॒यिं ॅयज॑मानाय धत्तम् । इ॒मौ प॒शून् र॑क्षतां ॅवि॒श्वतो॑ नः । पू॒षा नः॑ पातु॒ सद॒म-प्र॑युच्छन्न् ॥ प्र ते॑ म॒हे स॑रस्वति । सुभ॑गे॒ वाजि॑नीवति । स॒त्य॒वाचे॑ भरे म॒तिम् ॥ इ॒दं ते॑ ह॒व्यं घृ॒तव॑थ्-सरस्वति । स॒त्य॒वाचे॒ प्रभ॑रेमा ह॒वीꣳषि॑ । इ॒मानि॑ ते दुरि॒ता सौभ॑गानि ( ) । तेभि॑र्व॒यꣳ सु॒भगा॑सः स्याम । \textbf{ 17} \newline
                  \newline
                                    (व॒ज् - य्रही॑ना - मृजी॒षं - ॅव्यृ॑ण्वति - रक्षतु नो र॒यिꣳ - सौभ॑गा॒न्येकं॑ च) \textbf{(A4)} \newline \newline
                \textbf{ 2.5.5     अनुवाकं   5 -  उपहोमाः} \newline
                                \textbf{ TB 2.5.5.1} \newline
                  य॒ज्ञो रा॒यो य॒ज्ञ् ई॑शे॒ वसू॑नाम् । य॒ज्ञ्ः स॒स्याना॑मु॒त सु॑क्षिती॒नाम् । य॒ज्ञ् इ॒ष्टः पू॒र्वचि॑त्तिं दधातु । य॒ज्ञो ब्र॑ह्म॒ण्वाꣳ अप्ये॑तु दे॒वान् ॥ अ॒यं ॅय॒ज्ञो व॑र्द्धतां॒ गोभि॒रश्वैः᳚ । इ॒यं ॅवेदिः॑ स्वप॒त्या सु॒वीरा᳚ । इ॒दं ब॒र्॒.हिरति॑ ब॒र॒.हीꣳष्य॒न्या । इ॒मं ॅय॒ज्ञ्ं ॅविश्वे॑ अवन्तु दे॒वाः ॥ भग॑ ए॒व भग॑वाꣳ अस्तु देवाः । तेन॑ व॒यं भग॑वन्तः स्याम \textbf{ 18} \newline
                  \newline
                                \textbf{ TB 2.5.5.2} \newline
                  तं त्वा॑ भग॒ सर्व॒ इज्जो॑हवीमि । स नो॑ भग पुर ए॒ता भ॑वे॒ह ॥ भग॒ प्रणे॑त॒र् भग॒ सत्य॑राधः । भगे॒मां धिय॒मुद॑व॒ दद॑न्नः । भग॒ प्रणो॑ जनय॒ गोभि॒रश्वैः᳚ । भग॒ प्र नृभि॑र् नृ॒वन्तः॑ स्याम ॥ शश्व॑तीः॒ समा॒ उप॑यन्ति लो॒काः । शश्व॑तीः॒ समा॒ उप॑य॒न्त्यापः॑ । इ॒ष्टं पू॒र्तꣳ शश्व॑तीनाꣳ॒॒ समा॑नाꣳ शाश्व॒तेन॑ । ह॒विषे॒ष्ट्वा ऽन॒न्तं ॅलो॒कं पर॒मारु॑रोह । \textbf{ 19} \newline
                  \newline
                                \textbf{ TB 2.5.5.3} \newline
                  इ॒यमे॒व सा या प्र॑थ॒मा व्यौच्छ॑त् । सा रू॒पाणि॑ कुरुते॒ पञ्च॑ दे॒वी ॥ द्वे स्वसा॑रौ वयत॒स्तन्त्र॑मे॒तत् । स॒ना॒तनं॒ ॅवित॑तꣳ॒॒ षण्म॑यूखम् । अवा॒न्याꣳ स्तन्तू᳚न् कि॒रतो॑ ध॒त्तो अ॒न्यान् । नाव॑पृ॒ज्याते॒ न ग॑माते॒ अन्त᳚म् ॥ आ वो॑ यन्तूदवा॒हासो॑ अ॒द्य । वृष्टिं॒ ॅये विश्वे॑ म॒रुतो॑ जु॒नन्ति॑ । अ॒यं ॅयो अ॒ग्निर् म॑रुतः॒ समि॑द्धः । ए॒तं जु॑षद्ध्वं कवयो युवानः । \textbf{ 20} \newline
                  \newline
                                \textbf{ TB 2.5.5.4} \newline
                  धा॒रा॒व॒रा म॒रुतो॑ धृ॒ष्णुवो॑जसः । मृ॒गा न भी॒मास्त॑वि॒षेभि॑-रू॒र्मिभिः॑ । अ॒ग्नयो॒ न शु॑शुचा॒ना ऋ॑जी॒षिणः॑ । भ्रुमिं॒ धम॑न्त॒ उप॒ गा अ॑वृण्वत ॥ "विच॑क्रमे॒{1}", "त्रिर्दे॒वः{2}"॥ आवे॒धसं॒ नील॑पृष्ठं बृ॒हन्त᳚म् । बृह॒स्पतिꣳ॒॒ सद॑ने सादयद्ध्वम् । सा॒दद्यो॑निं॒ दम॒ आदी॑दि॒वाꣳस᳚म् । हिर॑ण्यवर्णमरु॒षꣳ स॑पेम ॥ स हि शुचिः॑ श॒तप॑त्रः॒ स शु॒न्ध्यूः \textbf{ 21} \newline
                  \newline
                                \textbf{ TB 2.5.5.5} \newline
                  हिर॑ण्यवाशीरिषि॒रः सु॑व॒र॒.षाः । बृह॒स्पतिः॒ स स्वा॑वे॒श ऋ॒ष्वाः । पू॒रू सखि॑भ्य आ सु॒तिं क॑रिष्ठः ॥ पूषꣳ॒॒स्तव॑ व्र॒ते व॒यम् । न रि॑ष्येम क॒दा च॒न । स्तो॒तार॑स्त इ॒ह स्म॑सि ॥ यास्ते॑ पूष॒न्नावो॑ अ॒न्तः स॑मु॒द्रे । हि॒र॒ण्ययी॑र॒न्तरि॑क्षे॒ चर॑न्ति । याभि॑र्यासि दू॒त्याꣳ सूर्य॑स्य । कामे॑न कृ॒तः श्रव॑ इ॒च्छमा॑नः । \textbf{ 22} \newline
                  \newline
                                \textbf{ TB 2.5.5.6} \newline
                  अर॑ण्या॒न्यर॑ण्यान्य॒सौ । या प्रेव॒ नश्य॑सि । क॒था ग्रामं॒ न पृ॑च्छसि । न त्वा॒ भीरि॑व विन्दती(3) ॥ वृ॒षा॒र॒वाय॒ वद॑ते । यदु॒पाव॑ति चिच्चि॒कः । आ॒घा॒टीभि॑रिव धा॒वयन्न्॑ । अ॒र॒ण्या॒निर्-म॑हीयते ॥ उ॒त गाव॑ इवादन्न् । उ॒तो वेश्मे॑व दृश्यते \textbf{ 23} \newline
                  \newline
                                \textbf{ TB 2.5.5.7} \newline
                  उ॒तो अ॑रण्या॒निः सा॒यम् । श॒क॒टीरि॑व सर्जति ॥ गाम॒ङ्गैष॒ आह्व॑यति । दार्व॒ङ्गैष॒ उपा॑वधीत् । वस॑न्नरण्या॒न्याꣳ सा॒यम् । अक्रु॑क्ष॒दिति॑ मन्यते ॥ न वा अ॑रण्या॒निर्.ह॑न्ति । अ॒न्यश्चेन्नाभि॒गच्छ॑ति । स्वा॒दोः फल॑स्य ज॒ग्ध्वा । यत्र॒ कामं॒ निप॑द्यते ( ) ॥ आञ्ज॑नगन्धीꣳ सुर॒भीम् । ब॒ह्व॒न्नामकृ॑षीवलाम् । प्राहं मृ॒गाणां᳚ मा॒तर᳚म् । अ॒र॒ण्या॒नीम॑शꣳ-सिषम् । \textbf{ 24} \newline
                  \newline
                                    (स्या॒म॒ - रु॒रो॒ह॒ - यु॒वा॒नः॒ - शु॒न्ध् - यूरि॒च्छमा॑नो - दृश्यते॒ - निप॑द्यते च॒त्वारि॑ च) \textbf{(A5)} \newline \newline
                \textbf{ 2.5.6     अनुवाकं   6 -  उपहोमाः} \newline
                                \textbf{ TB 2.5.6.1} \newline
                  वार्त्र॑हत्याय॒ शव॑से । पृ॒त॒ना॒-साह्या॑य च । इन्द्र॒त्वा-ऽऽव॑र्तयामसि ॥ सु॒ब्रह्मा॑णं ॅवी॒रव॑न्तं बृ॒हन्त᳚म् । उ॒रुं ग॑भी॒रं पृ॒थुबु॑द्ध्नमिन्द्र । श्रु॒तर्.षि॑-मु॒ग्रम॑भि-माति॒षाह᳚म् । अ॒स्मभ्यं॑ चि॒त्रं ॅवृष॑णꣳ र॒यिं दाः᳚ ॥ क्षे॒त्रि॒यै त्वा॒ निर्.ऋ॑त्यै त्वा । द्रु॒हो मु॑ञ्चामि॒ वरु॑णस्य॒ पाशा᳚त् । अ॒ना॒गसं॒ ब्रह्म॑णे त्वा करोमि \textbf{ 25} \newline
                  \newline
                                \textbf{ TB 2.5.6.2} \newline
                  शि॒वे ते॒ द्यावा॑पृथि॒वी उ॒भे इ॒मे ॥ शं ते॑ अ॒ग्निः स॒हाद्भिर॑स्तु । शं द्यावा॑पृथि॒वी स॒हौष॑धीभिः । शम॒न्तरि॑क्षꣳ स॒ह वाते॑न ते । शं ते॒ चत॑स्रः प्र॒दिशो॑ भवन्तु ॥ या दैवी॒श्चत॑स्रः प्र॒दिशः॑ । वात॑पत्नीर॒भि सूर्यो॑ विच॒ष्टे । तासां᳚ त्वा ज॒रस॒ आद॑धामि । प्र यक्ष्म॑ एतु॒ निर्.ऋ॑तिं परा॒चैः ॥ अमो॑चि॒ यक्ष्मा᳚द्-दुरि॒ता-दव॑र्त्यै \textbf{ 26} \newline
                  \newline
                                \textbf{ TB 2.5.6.3} \newline
                  द्रु॒हः पाशा॒न्निर्.ऋ॑त्यै॒ चोद॑मोचि । अहा॒ अव॑र्ति॒मवि॑दथ् स्यो॒नम् । अप्य॑भूद्-भ॒द्रे सु॑कृ॒तस्य॑ लो॒के ॥ सूर्य॑मृ॒तं तम॑सो॒ ग्राह्या॒ यत् । दे॒वा अमु॑ञ्च॒-न्नसृ॑ज॒न्व्ये॑नसः । ए॒वम॒हमि॒मं क्षे᳚त्रि॒याज्जा॑मिशꣳ॒॒सात् । द्रु॒हो मु॑ञ्चामि॒ वरु॑णस्य॒ पाशा᳚त् ॥ बृह॑स्पते यु॒वमिन्द्र॑श्च॒ वस्वः॑ । दि॒व्यस्ये॑शाथे उ॒त पार्थि॑वस्य । ध॒त्तꣳ र॒यिꣳ स्तु॑व॒ते की॒रये॑ चित् \textbf{ 27} \newline
                  \newline
                                \textbf{ TB 2.5.6.4} \newline
                  यू॒यं पा॑त स्व॒स्तिभिः॒ सदा॑ नः ॥ दे॒वा॒युध॒मिन्द्र॒माजोहु॑वानाः । वि॒श्वा॒वृध॑म॒भि ये रक्ष॑माणाः । येन॑ ह॒ता दी॒र्घमद्ध्वा॑न॒मायन्न्॑ । अ॒न॒न्तमर्थ॒मनि॑वर्थ्-स्यमानाः ॥ यत्ते॑ सुजाते हि॒मव॑थ्सु भेष॒जम् । म॒यो॒भूः शंत॑मा॒ यद्धृ॒दोऽसि॑ । ततो॑ नो देहि सीबले ॥ अ॒दो गि॒रिभ्यो॒ अधि॒ यत्प्र॒धाव॑सि । सꣳ॒॒शोभ॑माना क॒न्ये॑व शुभ्रे \textbf{ 28} \newline
                  \newline
                                \textbf{ TB 2.5.6.5} \newline
                  तां त्वा॒ मुद्ग॑ला ह॒विषा॑ वर्द्धयन्ति । सा नः॑ सीबले र॒यिमाभा॑जये॒ह ॥ पूर्वं॑ देवा॒ अप॑रेणा-नु॒पश्य॒ञ्जन्म॑भिः । जन्मा॒न्यव॑रैः॒ परा॑णि । वेदा॑नि देवा अ॒यम॒स्मीति॒ माम् । अ॒हꣳ हि॒त्वा शरी॑रं ज॒रसः॑ प॒रस्ता᳚त् ॥ प्रा॒णा॒पा॒नौ चक्षुः॒ श्रोत्र᳚म् । वाचं॒ मन॑सि॒ संभृ॑ताम् । हि॒त्वा शरी॑रं ज॒रसः॑ प॒रस्ता᳚त् । आ भूतिं॒ भूतिं॑ ॅव॒यम॑श्नवामहै ( ) ॥ इ॒मा ए॒व ता उ॒षसो॒ याः प्र॑थ॒मा व्यौच्छन्न्॑ । ता दे॒व्यः॑ कुर्वते॒ पञ्च॑ रू॒पा । शश्व॑ती॒र्नाव॑पृज्यन्ति । न ग॑म॒न्त्यन्त᳚म् । \textbf{ 29} \newline
                  \newline
                                    (क॒रो॒ - म्यव॑र्त्यै - चि - च्छुभ्रे - ऽश्नवामहै च॒त्वारि॑ च) \textbf{(A6)} \newline \newline
                \textbf{ 2.5.7     अनुवाकं   7 -  उपहोमाः} \newline
                                \textbf{ TB 2.5.7.1} \newline
                  वसू॑नां॒ त्वाऽऽधी॑तेन । रु॒द्राणा॑मू॒र्म्या । आ॒दि॒त्यानां॒ तेज॑सा । विश्वे॑षां दे॒वानां॒ क्रतु॑ना । म॒रुता॒मेम्ना॑ जुहोमि॒ स्वाहा᳚ ॥ अ॒भिभू॑ति-र॒हमाग॑मम् । इन्द्र॑सखा स्वा॒युधः॑ । आस्वाशा॑सु दु॒ष्षहः॑ ॥ इ॒दं ॅवर्चो॑ अ॒ग्निना॑ द॒त्तमागा᳚त् । यशो॒ भर्गः॒ सह॒ ओजो॒ बलं॑ च \textbf{ 30} \newline
                  \newline
                                \textbf{ TB 2.5.7.2} \newline
                  दी॒र्घा॒यु॒त्वाय॑ श॒तशा॑रदाय । प्रति॑गृभ्णामि मह॒ते वी॒र्या॑य ॥ आयु॑रसि वि॒श्वायु॑रसि । स॒र्वायु॑रसि॒ सर्व॒मायु॑रसि । सर्वं॑ म॒ आयु॑र्भूयात् । सर्व॒मायु॑र्गेषम् ॥ भूर्भुवः॒ सुवः॑ ॥ अ॒ग्निर्-धर्मे॑णान्ना॒दः । मृ॒त्युर्-धर्मे॒णान्न॑पतिः । ब्रह्म॑ क्ष॒त्रꣳ स्वाहा᳚ । \textbf{ 31} \newline
                  \newline
                                \textbf{ TB 2.5.7.3} \newline
                  प्र॒जाप॑तिः प्रणे॒ता । बृह॒स्पतिः॑ पुर ए॒ता । य॒मः पन्थाः᳚ । च॒न्द्रमाः᳚ पुनर॒सुः स्वाहा᳚ ॥ अ॒ग्निर॑न्ना॒दोऽन्न॑पतिः । अ॒न्नाद्य॑म॒स्मिन्. य॒ज्ञे यज॑मानाय ददातु॒ स्वाहा᳚ ॥ सोमो॒ राजा॒ राज॑पतिः । रा॒ज्यम॒स्मिन्. य॒ज्ञे यज॑मानाय ददातु॒ स्वाहा᳚ ॥ वरु॑णः स॒म्राट् थ्स॒म्राट्प॑तिः । साम्रा᳚ज्यम॒स्मिन्. य॒ज्ञे यज॑मानाय ददातु॒ स्वाहा᳚ । \textbf{ 32} \newline
                  \newline
                                \textbf{ TB 2.5.7.4} \newline
                  मि॒त्रः क्ष॒त्रं क्ष॒त्रप॑तिः । क्ष॒त्रम॒स्मिन्. य॒ज्ञे यज॑मानाय ददातु॒ स्वाहा᳚ ॥ इन्द्रो॒ बलं॒ बल॑पतिः । बल॑म॒स्मिन्. य॒ज्ञे यज॑मानाय ददातु॒ स्वाहा᳚ ॥ बृह॒स्पति॒र् ब्रह्म॒ ब्रह्म॑पतिः । ब्रह्मा॒स्मिन्. य॒ज्ञे यज॑मानाय ददातु॒ स्वाहा᳚ ॥ स॒वि॒ता रा॒ष्ट्रꣳ रा॒ष्ट्रप॑तिः । रा॒ष्ट्रम॒स्मिन्. य॒ज्ञे यज॑मानाय ददातु॒ स्वाहा᳚ ॥ पू॒षा वि॒शां ॅविट्प॑तिः । विश॑म॒स्मिन्. य॒ज्ञे यज॑मानाय ददातु॒ स्वाहा᳚ ( ) ॥ सर॑स्वती॒ पुष्टिः॒ पुष्टि॑पत्नी । पुष्टि॑म॒स्मिन्. य॒ज्ञे यज॑मानाय ददातु॒ स्वाहा᳚ ॥ त्वष्टा॑ पशू॒नां मि॑थु॒नानाꣳ॑ रूप॒कृद्-रू॒पप॑तिः । रू॒पेणा॒स्मिन्. य॒ज्ञे यज॑मानाय प॒शून् द॑दातु॒ स्वाहा᳚ । \textbf{ 33} \newline
                  \newline
                                                        \textbf{special korvai} \newline
              (अ॒ग्निः सोमो॒ वरु॑णो मि॒त्र इन्द्रो॒ बृह॒स्पतिः॑ सवि॒ता पू॒षा सरस्वती॒ त्वष्टा॒ दश॑) \newline
                                (च - स्वाहा॒ - साम्रा᳚ज्यम॒स्मिन्. य॒ज्ञे यज॑मानाय ददातु॒ स्वाहा॒ - विश॑म॒स्मिन्. य॒ज्ञे यज॑मानाय ददातु॒ स्वाहा॑ च॒त्वारि॑ च) \textbf{(A7)} \newline \newline
                \textbf{ 2.5.8     अनुवाकं   8 -  उपहोमाः} \newline
                                \textbf{ TB 2.5.8.1} \newline
                  स ईं᳚ पाहि॒ य ऋ॑जी॒षी तरु॑त्रः । यः शिप्र॑वान् वृष॒भो यो म॑ती॒नाम् । यो गो᳚त्र॒भिद्-व॑ज्र॒भृद्यो ह॑रि॒ष्ठाः । स इ॑न्द्र चि॒त्राꣳ अ॒भितृ॑न्धि॒ वाजान्॑ ॥ आ ते॒ शुष्मो॑ वृष॒भ ए॑तु प॒श्चात् । ओत्त॒राद॑ध॒रागा पु॒रस्ता᳚त् । आ वि॒श्वतो॑ अ॒भि समे᳚त्व॒र्वाङ् । इन्द्र॑ द्यु॒म्नꣳ सुव॑र्वद्धेह्य॒स्मे ॥ प्रोष्व॑स्मै पुरोर॒थम् । इन्द्रा॑य शू॒षम॑र्चत \textbf{ 34} \newline
                  \newline
                                \textbf{ TB 2.5.8.2} \newline
                  अ॒भीके॑ चिदु लोक॒कृत् । स॒ङ्गे स॒मथ्सु॑ वृत्र॒हा । अ॒स्माकं॑ बोधि चोदि॒ता । नभ॑न्ता-मन्य॒केषा᳚म् । ज्या॒का अधि॒ धन्व॑सु ॥ इन्द्रं॑ ॅव॒यꣳ शु॑ना॒सीर᳚म् । अ॒स्मिन्. य॒ज्ञे ह॑वामहे । आवाजै॒रुप॑ नो गमत् ॥ इन्द्रा॑य॒ शुना॒सीरा॑य । स्रु॒चा जु॑हुत नो ह॒विः \textbf{ 35} \newline
                  \newline
                                \textbf{ TB 2.5.8.3} \newline
                  जु॒षतां॒ प्रति॒ मेधि॑रः ॥ प्र ह॒व्यानि॑ घृ॒तव॑न्त्यस्मै । हर्य॑श्वाय भरता स॒जोषाः᳚ । इन्द्र॒र्तुभि॒र् ब्रह्म॑णा वावृधा॒नः । शु॒ना॒सी॒री ह॒विरि॒दं जु॑षस्व ॥ वयः॑ सुप॒र्णा उप॑सेदु॒रिन्द्र᳚म् । प्रि॒यमे॑धा॒ ऋष॑यो॒ नाध॑मानाः । अप॑ ध्वा॒न्तमू᳚र्णु॒हि पू॒र्द्धि चक्षुः॑ । मु॒मु॒ग्ध्य॑स्मान् नि॒धये॑व ब॒द्धान् ॥ बृ॒हदिन्द्रा॑य गायत \textbf{ 36} \newline
                  \newline
                                \textbf{ TB 2.5.8.4} \newline
                  मरु॑तो वृत्र॒हन्त॑मम् । येन॒ ज्योति॒-रज॑नयन्नृता॒वृधः॑ । दे॒वं दे॒वाय॒ जागृ॑वि ॥ का मि॒हैकाः॒ क इ॒मे प॑त॒ङ्गाः । मा॒न्था॒लाः कुलि॒ परि॑ मा पतन्ति । अना॑वृतैना॒न्-प्रध॑मन्तु दे॒वाः । सौप॑र्णं॒ चक्षु॑स्त॒नुवा॑ विदेय ॥ ए॒वा व॑न्दस्व॒ वरु॑णं बृ॒हन्त᳚म् । न॒म॒स्या धीर॑म॒मृत॑स्य गो॒पाम् । स नः॒ शर्म॑ त्रि॒वरू॑थं॒ ॅवियꣳ॑सत् \textbf{ 37} \newline
                  \newline
                                \textbf{ TB 2.5.8.5} \newline
                  यू॒यं पा॑त स्व॒स्तिभिः॒ सदा॑ नः ॥ नाके॑ सुप॒र्णमुप॒ यत्-पत॑न्तम् । हृ॒दा वेन॑न्तो अ॒भ्यच॑क्षत त्वा । हिर॑ण्यपक्षं॒ ॅवरु॑णस्य दू॒तम् । य॒मस्य॒ योनौ॑ शकु॒नं भु॑र॒ण्युम् ॥ शं नो॑ दे॒वीर॒भिष्ट॑ये । आपो॑ भवन्तु पी॒तये᳚ । शं ॅयोर॒भिस्र॑वन्तु नः ॥ ईशा॑ना॒ वार्या॑णाम् । क्षय॑न्ती-श्चर्.षणी॒नाम् \textbf{ 38} \newline
                  \newline
                                \textbf{ TB 2.5.8.6} \newline
                  अ॒पो या॑चामि भेष॒जम् ॥ अ॒फ्सु मे॒ सोमो॑ अब्रवीत् । अ॒न्तर् विश्वा॑नि भेष॒जा । अ॒ग्निं च॑ वि॒श्वश॑म्भुवम् । आप॑श्च वि॒श्वभे॑षजीः ॥ यद॒फ्सु ते॑ सरस्वति । गोष्वश्वे॑षु॒ यन्मधु॑ । तेन॑ मे वाजिनीवति । मुख॑मङ्ग्धि सरस्वति ॥ या सर॑स्वती वैशम्भ॒ल्या \textbf{ 39} \newline
                  \newline
                                \textbf{ TB 2.5.8.7} \newline
                  तस्यां᳚ मे रास्व । तस्या᳚स्ते भक्षीय । तस्या᳚स्ते भूयिष्ठ॒भाजो॑ भूयास्म ॥ अ॒हं त्वद॑स्मि॒ मद॑सि॒ त्वमे॒तत् । ममा॑सि॒ योनि॒स्तव॒ योनि॑रस्मि । ममे॒व सन्वह॑ ह॒व्यान्य॑ग्ने । पु॒त्रः पि॒त्रे लो॑क॒कृज्जा॑तवेदः ॥ इ॒हैव सन्तत्र॒ सन्तं॑ त्वा ऽग्ने । प्रा॒णेन॑ वा॒चा मन॑सा बिभर्मि । ति॒रो मा॒ सन्त॒मायु॒र्मा प्रहा॑सीत् \textbf{ 40} \newline
                  \newline
                                \textbf{ TB 2.5.8.8} \newline
                  ज्योति॑षा त्वा वैश्वान॒रेणोप॑तिष्ठे ॥ अ॒यं ते॒ योनि॑र्.ऋ॒त्वियः॑ । यतो॑ जा॒तो अरो॑चथाः । तं जा॒नन्न॑ग्न॒ आरो॑ह । अथा॑ नो वर्द्धया र॒यिम् ॥ या ते॑ अग्ने य॒ज्ञिया॑ त॒नूस्तये-ह्यारो॑हा॒त्माऽऽत्मान᳚म् । अच्छा॒ वसू॑नि कृ॒ण्वन्न॒स्मे नर्या॑ पु॒रूणि॑ । य॒ज्ञो भू॒त्वा य॒ज्ञ्मासी॑द॒ स्वां ॅयोनि᳚म् । जात॑वेदो॒ भुव॒ आजाय॑मानः॒ सक्ष॑य॒ एहि॑ ॥ उ॒पाव॑रोह जातवेदः॒ पुन॒स्त्वम् \textbf{ 41} \newline
                  \newline
                                \textbf{ TB 2.5.8.9} \newline
                  दे॒वेभ्यो॑ ह॒व्यं ॅव॑ह नः प्रजा॒नन्न् । आयुः॑ प्र॒जाꣳ र॒यिम॒स्मासु॑ धेहि । अज॑स्रो दीदिहि नो दुरो॒णे ॥ तमिन्द्रं॑ जोहवीमि म॒घवा॑नमु॒ग्रम् । स॒त्रादधा॑न॒-मप्र॑तिष्कुतꣳ॒॒ शवाꣳ॑सि । मꣳहि॑ष्ठो गी॒र्भिरा च॑ य॒ज्ञियो॑ ऽव॒वर्त॑त् । रा॒ये नो॒ विश्वा॑ सु॒पथा॑ कृणोतु व॒ज्री ॥ त्रिक॑द्रुकेषु महि॒षो यवा॑शिरं तुवि॒शुष्म॑स्तृ॒पत् । सोम॑मपिब॒द्-विष्णु॑ना सु॒तं ॅयथाव॑शत् । स ईं᳚ ममाद॒ महि॒ कर्म॒ कर्त॑वे म॒हामु॒रुम् \textbf{ 42} \newline
                  \newline
                                \textbf{ TB 2.5.8.10} \newline
                  सैनꣳ॑ सश्चद्-दे॒वं दे॒वः स॒त्यमिन्दुꣳ॑ स॒त्य इन्द्रः॑ ॥ वि॒दद्यती॑ स॒रमा॑ रु॒ग्णमद्रेः᳚ । महि॒ पाथः॑ पू॒र्व्यꣳ स॒द्ध्रिय॑क्कः । अग्रं॑ नयथ् सु॒पद्यक्ष॑राणाम् । अच्छा॒ रवं॑ प्रथ॒मा जा॑न॒ती गा᳚त् ॥ वि॒दद्-गव्यꣳ॑ स॒रमा॑ दृ॒ढमू॒र्वम् । येना॒ नुकं॒ मानु॑षी॒ भोज॑ते॒ विट् । आ ये विश्वा᳚ स्वप॒त्यानि॑ च॒क्रुः । कृ॒ण्वा॒नासो॑ अमृत॒त्वाय॑ गा॒तुम् ॥ त्वं नृभि॑र्नृपते दे॒वहू॑तौ \textbf{ 43} \newline
                  \newline
                                \textbf{ TB 2.5.8.11} \newline
                  भूरी॑णि वृ॒त्वा ह॑र्यश्व हꣳसि । त्वं निद॑स्युं॒ चुमु॑रिम् । धुनिं॒ चास्वा॑पयो द॒भीत॑ये सु॒हन्तु॑ ॥ ए॒वा पा॑हि प्र॒त्नथा॒ मन्द॑तु त्वा । श्रु॒धि ब्रह्म॑ वावृधस्वो॒त गी॒र्भिः । आ॒विः सूर्यं॑ कृणु॒हि पी॒पिही॒षः । ज॒हि शत्रूꣳ॑र॒भि गा इ॑न्द्र तृन्धि ॥ अग्ने॒ बाध॑स्व॒ विमृधो॑ नुदस्व । अपामी॑वा॒ अप॒ रक्षाꣳ॑सि सेध । अ॒स्माथ्-स॑मु॒द्राद्-बृ॑ह॒तो दि॒वो नः॑ \textbf{ 44} \newline
                  \newline
                                \textbf{ TB 2.5.8.12} \newline
                  अ॒पां भू॒मान॒मुप॑ नः सृजे॒ह ॥ यज्ञ्॒ प्रति॑तिष्ठ सुम॒तौ सु॒शेवा॒ आ त्वा᳚ । वसू॑नि पुरु॒धा वि॑शन्तु । दी॒र्घमायु॒र् यज॑मानाय कृ॒ण्वन्न् । अथा॒मृते॑न जरि॒तार॑मङ्ग्धि ॥ इन्द्रः॑ शु॒नाव॒द्-वित॑नोति॒ सीर᳚म् । स॒म्ॅव॒थ्स॒रस्य॑ प्रति॒माण॑मे॒तत् । अ॒र्कस्य॒ ज्योति॒स्तदिदा॑स॒ ज्येष्ठ᳚म् । स॒म्ॅव॒थ्स॒रꣳ शु॒नव॒थ्सीर॑मे॒तत् ॥ इन्द्र॑स्य॒ राधः॒ प्रय॑तं पु॒रु त्मना᳚ ( ) । तद॑र्करू॒पं ॅवि॒मिमा॑नमेति । द्वाद॑शारे॒ प्रति॑तिष्ठ॒तीद्-वृषा᳚ ॥ अ॒श्वा॒यन्तो॑ ग॒व्यन्तो॑ वा॒जय॑न्तः । हवा॑महे॒ त्वोप॑गन्त॒ वा उ॑ । आ॒भूष॑न्तस्त्वा सुम॒तौ नवा॑याम् । व॒यमि॑न्द्र त्वा शु॒नꣳ हु॑वेम । \textbf{ 45} \newline
                  \newline
                                    (अ॒र्च॒त॒ - ह॒विर् - गा॑यत - यꣳसच् - चर्.षणी॒नां - ॅवै॑शम्भ॒ल्या - हा॑सी॒त् - त्व - मु॒रुं - दे॒वहू॑तौ - नः॒ - त्मना॒ षट् च॑) \textbf{(A8)} \newline \newline
                \textbf{PrapAtaka Korvai with starting  words of 1 to 8 anuvAkams :-} \newline
        (प्रा॒ण - उ॒देहि॒ - पुन॑र्न॒ - आ नो॑ भर - य॒ज्ञो रा॒यो - वार्त्र॑हत्याय॒ - वसू॑नाꣳ॒॒ - स ईं॑ पाह्य॒ ष्टौ) \newline

        \textbf{korvai with starting words of 1, 11, 21 series of daSinis :-} \newline
        (प्रा॒णो र॑क्ष॒ - त्यगृ॑भीता - धाराव॒रा म॒रुतो॑ - दीर्घायु॒त्वाय॒ - ज्योति॑षा त्वा॒ पञ्च॑चत्वारिꣳशत् ) \newline

        \textbf{first and last  word 2nd aShTakam 5th prapAtakam :-} \newline
        (प्रा॒णः - शु॒नꣳ हु॑वेम) \newline 

       

        ॥ हरिः॑ ॐ ॥
॥ कृष्ण यजुर्वेदीय तैत्तिरीय ब्राह्मणे द्वितीयाष्टके पञ्चमः प्रपाठकः समाप्तः ॥

Appendix 
ट्.भ्.2.5.5.4 - "विच॑क्रमे॒ {1}", "त्रिर्दे॒वः{2}" 
विच॑क्रमे पृथि॒वीमे॒ष ए॒ताम् । क्षेत्रा॑य॒ विष्णु॒र् मनु॑षे दश॒स्यन्न् । 
ध्रु॒वासो॑ अस्य की॒रयो॒ जना॑सः । उ॒रु॒क्षि॒तिꣳ सु॒जनि॑मा चकार ॥ {1}

त्रिर्दे॒वः पृ॑थि॒वीमे॒ष ए॒ताम् । विच॑क्रमे श॒तर्च॑सं महि॒त्वा । 
प्र विष्णु॑-रस्तु त॒वस॒स्त वी॑यान् । त्वे॒षꣳ ह्य॑स्य॒ स्थवि॑रस्य॒ नाम॑ ॥ {2}
Both appearing in T.B.2.4.3.5 \newline
        \pagebreak
        
        
        
     \addcontentsline{toc}{section}{ 2.6     द्वितीयाष्टके षष्ठः प्रपाठकः - सौत्रामणिः कौकिली होत्रं च}
     \markright{ 2.6     द्वितीयाष्टके षष्ठः प्रपाठकः - सौत्रामणिः कौकिली होत्रं च \hfill https://www.vedavms.in \hfill}
     \section*{ 2.6     द्वितीयाष्टके षष्ठः प्रपाठकः - सौत्रामणिः कौकिली होत्रं च }
                \textbf{ 2.6.1      अनुवाकं   1 - ग्रहाः} \newline
                                \textbf{ TB 2.6.1.1} \newline
                  स्वा॒द्वीं त्वा᳚ स्वा॒दुना᳚ । ती॒व्रां ती॒व्रेण॑ । अ॒मृता॑-म॒मृते॑न । मधु॑मतीं॒ मधु॑मता । सृ॒जामि॒ सꣳ सोमे॑न । सोमो᳚ऽस्य॒श्विभ्यां᳚ पच्यस्व । सर॑स्वत्यै पच्यस्व । इन्द्रा॑य सु॒त्राम्णे॑ पच्यस्व ॥ परी॒तोषि॑ञ्चता सु॒तम् । सोमो॒ य उ॑त्त॒मꣳ ह॒विः \textbf{ 1} \newline
                  \newline
                                \textbf{ TB 2.6.1.2} \newline
                  द॒ध॒न्वा यो नर्यो॑ अ॒फ्स्व॑न्तरा । सु॒षाव॒ सोम॒मद्रि॑भिः ॥ पु॒नातु॑ ते परि॒स्रुत᳚म् । सोमꣳ॒॒ सूर्य॑स्य दुहि॒ता । वारे॑ण॒ शश्व॑ता॒ तना᳚ । वा॒युः पू॒तः प॒वित्रे॑ण । प्राङ्ख्सोमो॒ अति॑द्रुतः । इन्द्र॑स्य॒ युज्यः॒ सखा᳚ ॥ वा॒युः पू॒तः प॒वित्रे॑ण । प्र॒त्यङ्ख्सोमो॒ आति॑द्रुतः \textbf{ 2} \newline
                  \newline
                                \textbf{ TB 2.6.1.3} \newline
                  इन्द्र॑स्य॒ युज्यः॒ सखा᳚ ॥ ब्रह्म॑ क्ष॒त्रं प॑वते॒ तेज॑ इन्द्रि॒यम् । सुर॑या॒ सोमः॑ सु॒त आसु॑तो॒ मदा॑य । शु॒क्रेण॑ देव दे॒वताः᳚ पिपृग्धि । रसे॒नान्नं॒ ॅयज॑मानाय धेहि ॥ कु॒विद॒ङ्ग यव॑मन्तो॒ यवं॑ चित् । यथा॒ दान्त्य॑नुपू॒र्वं ॅवि॒यूय॑ । इ॒हेहै॑षां कृणुत॒ भोज॑नानि । ये ब॒र॒.हिषो॒ नमो॑वृक्तिं॒ न ज॒ग्मुः ॥ उ॒प॒या॒म-गृ॑हीतोऽस्य॒श्विभ्यां᳚ त्वा॒ जुष्टं॑ गृह्णामि \textbf{ 3} \newline
                  \newline
                                \textbf{ TB 2.6.1.4} \newline
                  सर॑स्वत्या॒ इन्द्रा॑य सु॒त्राम्णे᳚ ॥ ए॒ष ते॒ योनि॒स्तेज॑से त्वा । वी॒र्या॑य त्वा॒ बला॑य त्वा ॥ तेजो॑ऽसि॒ तेजो॒ मयि॑ धेहि । वी॒र्य॑मसि वी॒र्यं॑ मयि॑ धेहि । बल॑मसि॒ बलं॒ मयि॑ धेहि ॥ नाना॒ हि वां᳚ दे॒वहि॑तꣳ॒॒ सदः॑ कृ॒तम् । मा सꣳसृ॑क्षाथां पर॒मे व्यो॑मन्न् । सुरा॒ त्वमसि॑ शु॒ष्मिणी॒ सोम॑ ए॒षः । मा मा॑ हिꣳसीः॒ स्वां ॅयोनि॑मावि॒शन्न् । \textbf{ 4} \newline
                  \newline
                                \textbf{ TB 2.6.1.5} \newline
                  उ॒प॒या॒मगृ॑हीतोऽस्याश्वि॒नं तेजः॑ । सा॒र॒स्व॒तं ॅवी॒र्य᳚म् । ऐ॒न्द्रं बल᳚म् ॥ ए॒ष ते॒ योनि॒र्मोदा॑य त्वा । आ॒न॒न्दाय॑ त्वा॒ मह॑से त्वा ॥ ओजो॒ऽस्योजो॒ मयि॑ धेहि । म॒न्युर॑सि म॒न्युं मयि॑ धेहि । महो॑ऽसि॒ महो॒ मयि॑ धेहि । सहो॑ऽसि॒ सहो॒ मयि॑ धेहि ॥ या व्या॒घ्रं ॅविषू॑चिका ( ) । उ॒भौ वृकं॑ च॒ रक्ष॑ति । श्ये॒नं प॑त॒त्रिणꣳ॑ सिꣳ॒॒हम् । सेमं पा॒त्वꣳह॑सः ॥ सं॒ पृचः॑ स्थ॒ सं मा॑ भ॒द्रेण॑ पृङ्क्त । वि॒पृचः॑ स्थ॒ वि मा॑ पा॒प्मना॑ पृङ्क्त । \textbf{ 5} \newline
                  \newline
                                    (ह॒विः - प्र॒त्यङ्ख्सोमो॒ आति॑द्रुतो - गृह्णा - म्यावि॒शन् - विषू॑चिका॒ पञ्च॑ च) \textbf{(A1)} \newline \newline
                \textbf{ 2.6.2      अनुवाकं   2 - ग्रहोपस्थानम्} \newline
                                \textbf{ TB 2.6.2.1} \newline
                  सोमो॒ राजा॒ऽमृतꣳ॑ सु॒तः । ऋ॒जी॒षेणा॑जहान्मृ॒त्युम् । ऋ॒तेन॑ स॒त्यमि॑न्द्रि॒यम् । विपानꣳ॑ शु॒क्रमन्ध॑सः । इन्द्र॑स्येन्द्रि॒यम् । इ॒दं पयो॒ऽमृतं॒ मधु॑ ॥ सोम॑म॒द्भ्यो व्य॑पिबत् । छन्द॑सा हꣳ॒॒सः शु॑चि॒षत् । ऋ॒तेन॑ स॒त्यमि॑न्द्रि॒यम् । अ॒द्भ्यः क्षी॒रं ॅव्य॑पिबत् \textbf{ 6} \newline
                  \newline
                                \textbf{ TB 2.6.2.2} \newline
                  क्रुङ्ङा᳚ङ्गिर॒सो धि॒या । ऋ॒तेन॑ स॒त्यमि॑न्द्रि॒यम् । अन्ना᳚त् परि॒स्रुतो॒ रस᳚म् । ब्रह्म॑णा॒ व्य॑पिबत् क्ष॒त्रम् । ऋ॒तेन॑ स॒त्यमि॑न्द्रि॒यम् ॥ रेतो॒ मूत्रं॒ ॅविज॑हाति । योनिं॑ प्रवि॒शदि॑न्द्रि॒यम् । गर्भो॑ ज॒रायु॒णाऽऽवृ॑तः । उल्बं॑ जहाति॒ जन्म॑ना । ऋ॒तेन॑ स॒त्यमि॑न्द्रि॒यम् \textbf{ 7} \newline
                  \newline
                                \textbf{ TB 2.6.2.3} \newline
                  वेदे॑न रू॒पे व्य॑करोत् । स॒ता॒स॒ती प्र॒जाप॑तिः । ऋ॒तेन॑ स॒त्यमि॑न्द्रि॒यम् ॥ सोमे॑न॒ सोमौ॒ व्य॑पिबत् । सु॒ता॒सु॒तौ प्र॒जाप॑तिः । ऋ॒तेन॑ स॒त्यमि॑न्द्रि॒यम् । दृ॒ष्ट्वा रू॒पे व्याक॑रोत् । स॒त्या॒नृ॒ते प्र॒जाप॑तिः । अश्र॑द्धा॒-मनृ॒तेऽद॑धात् । श्र॒द्धाꣳ स॒त्ये प्र॒जाप॑तिः ( ) । ऋ॒तेन॑ स॒त्यमि॑न्द्रि॒यम् ॥ दृ॒ष्ट्वा प॑रि॒स्रुतो॒ रस᳚म् । शु॒क्रेण॑ शु॒क्रं ॅव्य॑पिबत् । पयः॒ सोमं॑ प्र॒जाप॑तिः । ऋ॒तेन॑ स॒त्यमि॑न्द्रि॒यम् । विपानꣳ॑ शु॒क्रमन्ध॑सः । इन्द्र॑स्येन्द्रि॒यम् । इ॒दं पयो॒ऽमृतं॒ मधु॑ । \textbf{ 8} \newline
                  \newline
                                                        \textbf{special korvai} \newline
              (सोमो॒ राजा॒ विपानꣳ॒॒ सोम॑म॒द्भ्योऽन्ना॒द् रेतो॒ मूत्रं॒ ॅवेदे॑न सतास॒ती सोमे॑न सुतासु॒तौ दृ॒ष्ट्वा रू॒पे दृ॒ष्ट्वा प॑रि॒स्रुतो॒ रसं॒ ॅविपानं॒ दश॑ ) \newline
                                (अ॒द्भ्यः क्षी॒रं ॅव्य॑पिब॒ - ज्जन्म॑न॒र्तेन॑ स॒त्यमि॑न्द्रि॒यꣳ - श्र॒द्धाꣳ स॒त्ये प्र॒जाप॑तिर॒ष्टौ च॑) \textbf{(A2)} \newline \newline
                \textbf{ 2.6.3     अनुवाकं   3 - ग्रहहोमः} \newline
                                \textbf{ TB 2.6.3.1} \newline
                  सुरा॑वन्तं बर्.हि॒षदꣳ॑ सु॒वीर᳚म् । य॒ज्ञ्ꣳ हि॑न्वन्ति महि॒षा नमो॑भिः । दधा॑नाः॒ सोमं॑ दि॒वि दे॒वता॑सु । मदे॒मेन्द्रं॒ ॅयज॑मानाः स्व॒र्काः ॥ यस्ते॒ रसः॒ संभृ॑त॒ ओष॑धीषु । सोम॑स्य॒ शुष्मः॒ सुर॑या सु॒तस्य॑ । तेन॑ जिन्व॒ यज॑मानं॒ मदे॑न । सर॑स्वती-म॒श्विना॒-विन्द्र॑-म॒ग्निम् ॥ यम॒श्विना॒ नमु॑चेरासु॒रादधि॑ । सर॑स्व॒त्यस॑नोदिन्द्रि॒याय॑ \textbf{ 9} \newline
                  \newline
                                \textbf{ TB 2.6.3.2} \newline
                  इ॒मं तꣳ शु॒क्रं मधु॑मन्त॒-मिन्दु᳚म् । सोमꣳ॒॒ राजा॑नमि॒ह भ॑क्षयामि ॥ यदत्र॑ रि॒प्तꣳ र॒सिनः॑ सु॒तस्य॑ । यदिन्द्रो॒ अपि॑ब॒च्छची॑भिः । अ॒हं तद॑स्य॒ मन॑सा शि॒वेन॑ । सोमꣳ॒॒ राजा॑नमि॒ह भ॑क्षयामि ॥ पि॒तृभ्यः॑ स्वधा॒विभ्यः॑ स्व॒धा नमः॑ । पि॒ता॒म॒हेभ्यः॑ स्वधा॒विभ्यः॑ स्व॒धा नमः॑ । प्रपि॑तामहेभ्यः स्वधा॒विभ्यः॑ स्व॒धा नमः॑ । अक्ष॑न् पि॒तरः॑ \textbf{ 10} \newline
                  \newline
                                \textbf{ TB 2.6.3.3} \newline
                  अमी॑मदन्त पि॒तरः॑ । अती॑तृपन्त पि॒तरः॑ । अमी॑मृजन्त पि॒तरः॑ । पित॑रः॒ शुन्ध॑द्ध्वम् ॥ पु॒नन्तु॑ मा पि॒तरः॑ सो॒म्यासः॑ । पु॒नन्तु॑ मा पिताम॒हाः । पु॒नन्तु॒ प्रपि॑तामहाः । प॒वित्रे॑ण श॒तायु॑षा । पु॒नन्तु॑ मा पिताम॒हाः । पु॒नन्तु॒ प्रपि॑तामहाः \textbf{ 11} \newline
                  \newline
                                \textbf{ TB 2.6.3.4} \newline
                  प॒वित्रे॑ण श॒तायु॑षा ॥ विश्व॒मायु॒र्-व्य॑श्नवै ।"अग्न॒ आयूꣳ॑षि पव॒से {3}" "ऽग्ने॒ पव॑स्व{4}"। "पव॑मानः॒ सुव॒र्जनः॑{5}" "पु॒नन्तु॑ मा देवज॒नाः{6}"। "जात॑वेदः प॒वित्र॑व॒द्{7}" "यत्ते॑ प॒वित्र॑म॒र्चिषि॑{8}"। "उ॒भाभ्यां᳚ देव सवितर्{9}" "वैश्वदे॒वी पु॑न॒ती{10}"॥ ये स॑मा॒नाः सम॑नसः । पि॒तरो॑ यम॒राज्ये᳚ । तेषां᳚ ॅलो॒कः स्व॒धा नमः॑ । य॒ज्ञो दे॒वेषु॑ कल्पताम् । \textbf{ 12} \newline
                  \newline
                                \textbf{ TB 2.6.3.5} \newline
                  ये स॑जा॒ताः सम॑नसः । जी॒वा जी॒वेषु॑ माम॒काः । तेषाꣳ॒॒ श्रीर्मयि॑ कल्पताम् । अ॒स्मिल् ॅलो॒के श॒तꣳ समाः᳚ ॥ द्वे स्रु॒ती अ॑शृणवं पितृ॒णाम् । अ॒हं दे॒वाना॑मु॒त मर्त्या॑नाम् । याभ्या॑मि॒दं ॅविश्व॒मेज॒थ्-समे॑ति । यद॑न्त॒रा पि॒तरं॑ मा॒तरं॑ च ॥ इ॒दꣳ ह॒विः प्र॒जन॑नं मे अस्तु । दश॑वीरꣳ स॒र्वग॑णꣳ स्व॒स्तये᳚ ( ) । आ॒त्म॒सनि॑ प्रजा॒सनि॑ । प॒शु॒सन्य॑भय॒सनि॑ लोक॒सनि॑ । अ॒ग्निः प्र॒जां ब॑हु॒लां मे॑ करोतु । अन्नं॒ पयो॒ रेतो॑ अ॒स्मासु॑ धत्त । रा॒यस्पोष॒-मिष॒मूर्ज॑म॒स्मासु॑ दीधर॒थ्-स्वाहा᳚ । \textbf{ 13} \newline
                  \newline
                                    (इ॒न्द्रि॒याय॑ - पि॒तरः॑ - श॒तायु॑षा पु॒नन्तु॑ मा पिताम॒हाः पु॒नन्तु॒ प्रपि॑तामहाः-कल्पताꣳ-स्व॒स्तये॒ पञ्च॑ च) \textbf{(A3)} \newline \newline
                \textbf{ 2.6.4     अनुवाकं   4 - उपहोमाः} \newline
                                \textbf{ TB 2.6.4.1} \newline
                  सीसे॑न॒ तन्त्रं॒ मन॑सा मनी॒षिणः॑ । ऊ॒र्णा॒सू॒त्रेण॑ क॒वयो॑ वयन्ति । अ॒श्विना॑ य॒ज्ञ्ꣳ स॑वि॒ता सर॑स्वती । इन्द्र॑स्य रू॒पं ॅवरु॑णो भिष॒ज्यन्न् ॥ तद॑स्य रू॒पम॒मृतꣳ॒॒ शची॑भिः । ति॒स्रो द॑धुर्दे॒वताः᳚ सꣳररा॒णाः । लोमा॑नि॒ शष्पै᳚र् बहु॒धा न तोक्म॑भिः । त्वग॑स्य माꣳ॒॒सम॑भव॒न्न ला॒जाः ॥ तद॒श्विना॑ भि॒षजा॑ रु॒द्रव॑र्तनी । सर॑स्वती वयति॒ पेशो॒ अन्त॑रः \textbf{ 14} \newline
                  \newline
                                \textbf{ TB 2.6.4.2} \newline
                  अस्थि॑ म॒ज्जानं॒ मास॑रैः । का॒रो॒त॒रेण॒ दध॑तो॒ गवां᳚ त्व॒चि ॥ सर॑स्वती॒ मन॑सा पेश॒लं ॅवसु॑ । नास॑त्याभ्यां ॅवयति दर्.श॒तं ॅवपुः॑ । रसं॑ परि॒स्रुता॒ न रोहि॑तम् । न॒ग्नहु॒र्द्धीर॒स्तस॑रं॒ न वेम॑ ॥ पय॑सा शु॒क्रम॒मृतं॑ ज॒नित्र᳚म् । सुर॑या॒ मूत्रा᳚ज्जनयन्ति॒ रेतः॑ । अपाम॑तिं दुर्म॒तिं बाध॑मानाः । ऊव॑द्ध्यं॒ ॅवातꣳ॑ स॒बुवं॒ तदा॒रात् । \textbf{ 15} \newline
                  \newline
                                \textbf{ TB 2.6.4.3} \newline
                  इन्द्रः॑ सु॒त्रामा॒ हृद॑येन स॒त्यम् । पु॒रो॒डाशे॑न सवि॒ता ज॑जान । यकृ॑त् क्लो॒मानं॒ ॅवरु॑णो भिष॒ज्यन्न् । मत॑स्ने वाय॒व्यै᳚र्न मि॑नाति पि॒त्तम् ॥ आ॒न्त्राणि॑ स्था॒ली मधु॒ पिन्व॑माना । गुदा॒ पात्रा॑णि सु॒दुघा॒ न धे॒नुः । श्ये॒नस्य॒ पत्रं॒ न प्ली॒हा शची॑भिः । आ॒स॒न्दी नाभि॑रु॒दरं॒ न मा॒ता ॥ कु॒म्भो व॑नि॒ष्ठुर्ज॑नि॒ता शची॑भिः । यस्मि॒न्नग्रे॒ योन्यां॒ गर्भो॑ अ॒न्तः \textbf{ 16} \newline
                  \newline
                                \textbf{ TB 2.6.4.4} \newline
                  प्ला॒शीर्व्य॑क्तः श॒तधा॑र॒ उथ्सः॑ । दु॒हे न कु॒म्भीꣳ स्व॒धां पि॒तृभ्यः॑ ॥ मुखꣳ॒॒ सद॑स्य॒ शिर॒ इथ्सदे॑न । जि॒ह्वा प॒वित्र॑म॒श्विना॒ सꣳ सर॑स्वती । चप्पं॒ न पा॒युर्भि॒षग॑स्य॒ वालः॑ । व॒स्तिर्न शेपो॒ हर॑सा तर॒स्वी ॥ अ॒श्विभ्यां॒ चक्षु॑र॒मृतं॒ ग्रहा᳚भ्याम् । छागे॑न॒ तेजो॑ ह॒विषा॑ शृ॒तेन॑ । पक्ष्मा॑णि गो॒धूमैः॒ क्व॑लैरु॒तानि॑ । पेशो॒ न शु॒क्लमसि॑तं ॅवसाते । \textbf{ 17} \newline
                  \newline
                                \textbf{ TB 2.6.4.5} \newline
                  अवि॒र्न मे॒षो न॒सि वी॒र्या॑य । प्रा॒णस्य॒ पन्था॑ अ॒मृतो॒ ग्रहा᳚भ्याम् । सर॑स्व॒त्युप॒वाकै᳚र्व्या॒नम् । नस्या॑नि ब॒र॒.हिर् बद॑रैर्-जजान ॥ इन्द्र॑स्य रू॒पमृ॑ष॒भो बला॑य । कर्णा᳚भ्याꣳ॒॒ श्रोत्र॑म॒मृतं॒ ग्रहा᳚भ्याम् । यवा॒ न ब॒र॒.हिर् भ्रु॒वि केस॑राणि । क॒र्कन्धु॑ जज्ञे॒ मधु॑ सार॒घं मुखा᳚त् ॥ आ॒त्मन्नु॒पस्थे॒ न वृक॑स्य॒ लोम॑ । मुखे॒ श्मश्रू॑णि॒ न व्या᳚घ्रलो॒मम् \textbf{ 18} \newline
                  \newline
                                \textbf{ TB 2.6.4.6} \newline
                  केशा॒ न शी॒र॒.षन्. यश॑से श्रि॒यै शिखा᳚ । सिꣳ॒॒हस्य॒ लोम॒ त्विषि॑-रिन्द्रि॒याणि॑ ॥ अङ्गा᳚न्या॒त्मन्-भि॒षजा॒ तद॒श्विना᳚ । आ॒त्मान॒मङ्गैः॒ सम॑धा॒थ् सर॑स्वती । इन्द्र॑स्य रू॒पꣳ श॒तमा॑न॒मायुः॑ । च॒न्द्रेण॒ ज्योति॑र॒मृतं॒ दधा॑ना ॥ सर॑स्वती॒ योन्यां॒ गर्भ॑म॒न्तः । अ॒श्विभ्यां॒ पत्नी॒ सुकृ॑तं बिभर्ति । अ॒पाꣳ रसे॑न॒ वरु॑णो॒ न साम्ना᳚ । इन्द्रꣳ॑ श्रि॒यै ज॒नय॑न्न॒फ्सु राजा᳚ ( ) ॥ तेजः॑ पशू॒नाꣳ ह॒विरि॑न्द्रि॒याव॑त् । प॒रि॒स्रुता॒ पय॑सा सार॒घं मधु॑ । अ॒श्विभ्यां᳚ दु॒ग्धं भि॒षजा॒ सर॑स्वत्या सुतासु॒ताभ्या᳚म् । अ॒मृतः॒ सोम॒ इन्दुः॑ । \textbf{ 19} \newline
                  \newline
                                    (अन्त॑र - आ॒रा - द॒न्तर् - व॑साते - व्याघ्रलो॒मꣳ - राजा॑ च॒त्वारि॑ च) \textbf{(A4)} \newline \newline
                \textbf{ 2.6.5     अनुवाकं   5 - अभिषेकः} \newline
                                \textbf{ TB 2.6.5.1} \newline
                  मि॒त्रो॑ऽसि॒ वरु॑णोऽसि । सम॒हं ॅविश्वै᳚र्दे॒वैः ॥ क्ष॒त्रस्य॒ नाभि॑रसि । क्ष॒त्रस्य॒ योनि॑रसि ॥ स्यो॒नामासी॑द । सु॒षदा॒मासी॑द । मा त्वा॑ हिꣳसीत् । मा मा॑ हिꣳसीत् ॥ निष॑साद धृ॒तव्र॑तो॒ वरु॑णः । प॒स्त्या᳚स्वा \textbf{ 20} \newline
                  \newline
                                \textbf{ TB 2.6.5.2} \newline
                  साम्रा᳚ज्याय सु॒क्रतुः॑ ॥ दे॒वस्य॑ त्वा सवि॒तुः प्र॑स॒वे । अ॒श्विनो᳚र् बा॒हुभ्या᳚म् । पू॒ष्णो हस्ता᳚भ्याम् । अ॒श्विनो॒र्-भैष॑ज्येन । तेज॑से ब्रह्मवर्च॒साया॒भिषि॑ञ्चामि ॥ दे॒वस्य॑ त्वा सवि॒तुः प्र॑स॒वे । अ॒श्विनो᳚र्-बा॒हुभ्या᳚म् । पू॒ष्णो हस्ता᳚भ्याम् । सर॑स्वत्यै॒ भैष॑ज्येन \textbf{ 21} \newline
                  \newline
                                \textbf{ TB 2.6.5.3} \newline
                  वी॒र्या॑या॒न्नाद्या॑या॒भिषि॑ञ्चामि ॥ दे॒वस्य॑ त्वा सवि॒तुः प्र॑स॒वे । अ॒श्विनो᳚र् बा॒हुभ्या᳚म् । पू॒ष्णो हस्ता᳚भ्याम् । इन्द्र॑स्येन्द्रि॒येण॑ । श्रि॒यै यश॑से॒ बला॑या॒भिषि॑ञ्चामि ॥ को॑ऽसि कत॒मो॑ऽसि । कस्मै᳚ त्वा॒ काय॑ त्वा ॥ सुश्लो॒काॅ(4) सुम॑ड्ग॒लाॅ(4) सत्य॑रा॒जा(3)न् ॥ शिरो॑ मे॒ श्रीः \textbf{ 22} \newline
                  \newline
                                \textbf{ TB 2.6.5.4} \newline
                  यशो॒ मुख᳚म् । त्विषिः॒ केशा᳚श्च॒ श्मश्रू॑णि । राजा॑ मे प्रा॒णो॑ऽमृत᳚म् । स॒म्राट् चक्षुः॑ । वि॒राट्छ्रोत्र᳚म् । जि॒ह्वा मे॑ भ॒द्रम् । वाङ्महः॑ । मनो॑ म॒न्युः । स्व॒राड्भामः॑ । मोदाः᳚ प्रमो॒दा अ॒ङ्गुली॒रङ्गा॑नि \textbf{ 23} \newline
                  \newline
                                \textbf{ TB 2.6.5.5} \newline
                  चि॒त्तं मे॒ सहः॑ । बा॒हू मे॒ बल॑मिन्द्रि॒यम् । हस्तौ॑ मे॒ कर्म॑ वी॒र्य᳚म् । आ॒त्मा क्ष॒त्रमुरो॒ मम॑ । पृ॒ष्टीर्मे॑ रा॒ष्ट्रमु॒दर॒मꣳसौ᳚ । ग्री॒वाश्च॒ श्रोण्यौ᳚ । ऊ॒रू अ॑र॒त्नी जानु॑नी । विशो॒ मेऽङ्गा॑नि स॒र्वतः॑ । नाभि॑र्मे चि॒त्तं ॅवि॒ज्ञान᳚म् । पा॒युर्मेऽप॑चितिर्भ॒सत् \textbf{ 24} \newline
                  \newline
                                \textbf{ TB 2.6.5.6} \newline
                  आ॒न॒न्द॒न॒न्दावा॒ण्डौ मे᳚ । भगः॒ सौभा᳚ग्यं॒ पसः॑ ॥ जङ्घा᳚भ्यां प॒द्भ्यां धर्मो᳚ऽस्मि । वि॒शि राजा॒ प्रति॑ष्ठितः ॥ प्रति॑ क्ष॒त्रे प्रति॑तिष्ठामि रा॒ष्ट्रे । प्रत्यश्वे॑षु॒ प्रति॑तिष्ठामि॒ गोषु॑ । प्रत्यङ्गे॑षु॒ प्रति॑तिष्ठाम्या॒त्मन्न् । प्रति॑ प्रा॒णेषु॒ प्रति॑तिष्ठामि पु॒ष्टे । प्रति॒ द्यावा॑पृथि॒व्योः । प्रतिति॑ष्ठामि य॒ज्ञे । \textbf{ 25} \newline
                  \newline
                                \textbf{ TB 2.6.5.7} \newline
                  त्र॒या दे॒वा एका॑दश । त्र॒य॒स्त्रिꣳ॒॒शाः सु॒राध॑सः । बृह॒स्पति॑ पुरोहिताः । दे॒वस्य॑ सवि॒तुः स॒वे । दे॒वा दे॒वैर॑वन्तु मा । प्र॒थ॒मा द्वि॒तीयैः᳚ । द्वि॒तीया᳚स्तृ॒तीयैः᳚ । तृ॒तीयाः᳚ स॒त्येन॑ । स॒त्यं ॅय॒ज्ञेन॑ ।य॒ज्ञो यजु॑र्भिः \textbf{ 26} \newline
                  \newline
                                \textbf{ TB 2.6.5.8} \newline
                  यजूꣳ॑षि॒ साम॑भिः । सामा᳚न्यृ॒ग्भिः । ऋचो॑ या॒ज्या॑भिः । या॒ज्या॑ वषट्का॒रैः । व॒ष॒ट्का॒रा आहु॑तिभिः । आहु॑तयो मे॒ कामा॒न्थ् सम॑र्द्धयन्तु ॥ भूः स्वाहा᳚ ॥ लोमा॑नि॒ प्रय॑ति॒र्मम॑ । त्वङ्म॒ आन॑ति॒राग॑तिः । माꣳ॒॒सं म॒ उप॑नतिः ( ) । व स्वस्थि॑ । म॒ज्जा म॒ आन॑तिः । \textbf{ 27} \newline
                  \newline
                                    (प॒स्त्या᳚स्वा - सर॑स्वत्यै॒ भैष॑ज्येन॒ - श्री - रङ्गा॑नि - भ॒सद् - य॒ज्ञे - य॒ज्ञो यजु॑र्भि॒ - रुप॑नति॒र्द्वे च॑) \textbf{(A5)} \newline \newline
                \textbf{ 2.6.6     अनुवाकं   6 - अवभृथः} \newline
                                \textbf{ TB 2.6.6.1} \newline
                  यद्-दे॑वा देव॒हेड॑नम् । देवा॑सश्चकृ॒मा व॒यम् । अ॒ग्निर्मा॒ तस्मा॒देन॑सः । विश्वा᳚न् मुञ्च॒त्वꣳह॑सः ॥ यदि॒ दिवा॒ यदि॒ नक्त᳚म् । एनाꣳ॑सि चकृ॒मा व॒यम् । वा॒युर्मा॒ तस्मा॒देन॑सः । विश्वा᳚न् मुञ्च॒त्वꣳह॑सः ॥ यदि॒ जाग्र॒द्-यदि॒ स्वप्ने᳚ । एनाꣳ॑सि चकृ॒मा व॒यम् \textbf{ 28} \newline
                  \newline
                                \textbf{ TB 2.6.6.2} \newline
                  सूर्यो॑ मा॒ तस्मा॒देन॑सः । विश्वा᳚न् मुञ्च॒त्वꣳह॑सः ॥ यद्ग्रामे॒ यदर॑ण्ये । यथ् स॒भायां॒ ॅयदि॑न्द्रि॒ये । यच्छू॒द्रे यद॒र्ये᳚ । एन॑श्चकृ॒मा व॒यम् । यदेक॒स्याधि॒ धर्म॑णि । तस्या॑व॒यज॑नमसि ॥ यदापो॒ अघ्नि॑या॒ वरु॒णेति॒ शपा॑महे । ततो॑ वरुण नो मुञ्च । \textbf{ 29} \newline
                  \newline
                                \textbf{ TB 2.6.6.3} \newline
                  अव॑भृथ निचङ्कुण निचे॒रुर॑सि निचङ्कुण । अव॑ दे॒वैर्-दे॒वकृ॑त॒मेनो॑ऽयाट् । अव॒ मर्त्यै॒र्मर्त्य॑कृतम् । उ॒रोरा नो॑ देव रि॒षस्पा॑हि ॥ सु॒मि॒त्रा न॒ आप॒ ओष॑धयः सन्तु । दु॒र्मि॒त्रास्तस्मै॑ भुयासुः । यो᳚ऽस्मान् द्वेष्टि॑ । यं च॑ व॒यं द्वि॒ष्मः ॥ द्रु॒प॒दादि॒वेन्-मु॑मुचा॒नः । स्वि॒न्नः स्ना॒त्वी मला॑दिव \textbf{ 30} \newline
                  \newline
                                \textbf{ TB 2.6.6.4} \newline
                  पू॒तं प॒वित्रे॑णे॒वाज्य᳚म् । आपः॑ शुन्धन्तु॒ मैन॑सः ॥ उद्व॒यं तम॑स॒स्परि॑ । पश्य॑न्तो॒ ज्योति॒रुत्त॑रम् । दे॒वं दे॑व॒त्रा सूर्य᳚म् । अग॑न्म॒ ज्योति॑रुत्त॒मम् ॥ प्रति॑युतो॒ वरु॑णस्य॒ पाशः॑ । प्रत्य॑स्तो॒ वरु॑णस्य॒ पाशः॑ ॥ एधो᳚ऽस्येधिषी॒महि॑ । स॒मिद॑सि \textbf{ 31} \newline
                  \newline
                                \textbf{ TB 2.6.6.5} \newline
                  तेजो॑ऽसि॒ तेजो॒ मयि॑ धेहि ॥ अ॒पो अन्व॑चारिषम् । रसे॑न॒ सम॑सृक्ष्महि । पय॑स्वाꣳ अग्न॒ आऽग॑मम् । तं मा॒ सꣳसृ॑ज॒ वर्च॑सा । प्र॒जया॑ च॒ धने॑न च ॥ स॒माव॑वर्ति पृथि॒वी । समु॒षाः । समु॒ सूर्यः॑ । समु॒ विश्व॑मि॒दं जग॑त् ( ) । वै॒श्वा॒न॒र ज्यो॑तिर्-भूयासम् । वि॒भुं कामं॒ ॅव्य॑श्नवै ॥ भूः स्वाहा᳚ । \textbf{ 32} \newline
                  \newline
                                    (स्वप्न॒ एनाꣳ॑सि चकृ॒मा व॒यं - मु॑ञ्च॒ - मला॑दिव - स॒मिद॑सि॒ - जग॒त् त्रीणि॑ च) \textbf{(A6)} \newline \newline
                \textbf{ 2.6.7     अनुवाकं   7 - ऐन्द्रे प्रश्नौ प्रयाजानामेकादश मैत्रावरुणप्रैषाः} \newline
                                \textbf{ TB 2.6.7.1} \newline
                  होता॑ यक्षथ्-स॒मिधेन्द्र॑-मि॒डस्प॒दे । नाभा॑ पृथि॒व्या अधि॑ । दि॒वो वर्ष्म॒न्थ् समि॑द्ध्यते । ओजि॑ष्ठश्चर्.षणी॒सहान्॑ । वेत्वाज्य॑स्य॒ होत॒र्यज॑ ॥ होता॑ यक्ष॒त्तनू॒नपा॑तम् । ऊ॒तिभि॒र्-जेता॑र॒-मप॑राजितम् । इन्द्रं॑ दे॒वꣳ सु॑व॒र्विद᳚म् । प॒थिभि॒र् मधु॑मत्तमैः । नरा॒शꣳसे॑न॒ तेज॑सा \textbf{ 33} \newline
                  \newline
                                \textbf{ TB 2.6.7.2} \newline
                  वेत्वाज्य॑स्य॒ होत॒र्यज॑ ॥ होता॑ यक्ष॒दिडा॑भि॒-रिन्द्र॑-मीडि॒तम् । आ॒जुह्वा॑न॒-मम॑र्त्यम् । दे॒वो दे॒वैः सवी᳚र्यः । वज्र॑हस्तः पुरंद॒रः । वेत्वाज्य॑स्य॒ होत॒र्यज॑ ॥ होता॑ यक्षद्-ब॒र॒.हिषीन्द्रं॑ निषद्व॒रम् । वृ॒ष॒भं नर्या॑पसम् । वसु॑भी रु॒द्रैरा॑दि॒त्यैः । स॒युग्भि॑र्-ब॒र॒.हिरा ऽस॑दत् \textbf{ 34} \newline
                  \newline
                                \textbf{ TB 2.6.7.3} \newline
                  वेत्वाज्य॑स्य॒ होत॒र्यज॑ ॥ होता॑ यक्ष॒दोजो॒ न वी॒र्य᳚म् । सहो॒ द्वार॒ इन्द्र॑मवर्द्धयन्न् । सु॒प्रा॒य॒णा विश्र॑यन्तामृता॒वृधः॑ । द्वार॒ इन्द्रा॑य मी॒ढुषे᳚ । वि॒यन्त्वाज्य॑स्य॒ होत॒र्यज॑ ॥ होता॑ यक्षदु॒षे इन्द्र॑स्य धे॒नू । सु॒दुघे॑ मा॒तरौ॑ म॒ही । सवा॒तरौ॒ न तेज॑सी । व॒थ्समिन्द्र॑-मवर्द्धताम् \textbf{ 35} \newline
                  \newline
                                \textbf{ TB 2.6.7.4} \newline
                  वी॒तामाज्य॑स्य॒ होत॒र्यज॑ ॥ होता॑ यक्ष॒द् दैव्या॒ होता॑रा । भि॒षजा॒ सखा॑या । ह॒विषेन्द्रं॑ भिषज्यतः । क॒वी दे॒वौ प्रचे॑तसौ । इन्द्रा॑य धत्त इन्द्रि॒यम् । वी॒तामाज्य॑स्य॒ होत॒र्यज॑ ॥ होता॑ यक्षत्ति॒स्रो दे॒वीः । त्रय॑स्त्रि॒धात॑वो॒ऽपसः॑ । इडा॒ सर॑स्वती॒ भार॑ती \textbf{ 36} \newline
                  \newline
                                \textbf{ TB 2.6.7.5} \newline
                  म॒हीन्द्र॑पत्नीर्.-ह॒विष्म॑तीः । वि॒यन्त्वाज्य॑स्य॒ होत॒र्यज॑ ॥ होता॑ यक्ष॒त्त्वष्टा॑र॒-मिन्द्रं॑ दे॒वम् । भि॒षजꣳ॑ सु॒यजं॑ घृत॒श्रिय᳚म् । पु॒रु॒रूपꣳ॑ सु॒रेत॑सं म॒घोनि᳚म् । इन्द्रा॑य॒ त्वष्टा॒ दध॑दिन्द्रि॒याणि॑ । वेत्वाज्य॑स्य॒ होत॒र्यज॑ ॥ होता॑ यक्ष॒द्-वन॒स्पति᳚म् । श॒मि॒तारꣳ॑ श॒तक्र॑तुम् । धि॒यो जो॒ष्टार॑-मिन्द्रि॒यम् \textbf{ 37} \newline
                  \newline
                                \textbf{ TB 2.6.7.6.} \newline
                  मद्ध्वा॑ सम॒ञ्जन्प॒थिभिः॑ सु॒गेभिः॑ । स्वदा॑ति ह॒व्यं मधु॑ना घृ॒तेन॑ । वेत्वाज्य॑स्य॒ होत॒र्यज॑ ॥ होता॑ यक्ष॒दिन्द्रꣳ॒॒ स्वाहाऽऽज्य॑स्य । स्वाहा॒ मेद॑सः । स्वाहा᳚ स्तो॒काना᳚म् । स्वाहा॒ स्वाहा॑कृतीनाम् । स्वाहा॑ ह॒व्यसू᳚क्तीनाम् । स्वाहा॑ दे॒वाꣳ आ᳚ज्य॒पान् । स्वाहेन्द्रꣳ॑ हो॒त्राज्जु॑षा॒णाः ( ) । इन्द्र॒ आज्य॑स्य वियन्तु । होत॒र्यज॑ । \textbf{ 38} \newline
                  \newline
                                                        \textbf{special korvai} \newline
              (स॒मिधेन्द्रं॒ तनू॒नपा॑त॒मिडा॑भिर् ब॒र्॒.हिष्योज॑ उ॒षे दैव्या॑ ति॒स्रस्त्वष्टा॑रं॒ ॅवन॒स्पति॒मिन्द्र᳚म् । स॒मिधेन्द्रं॑ च॒तुर्वेत्वेको॑ वि॒यन्तु॒ द्विर्वी॒तामेको॑ वि॒यन्तु॒ द्विर्वेत्वेको॑ वि॒यन्तु॒ होत॒र्यज॑ ) \newline
                                (तेज॑सा-ऽसद-दवर्द्धतां॒-भार॑ती-न्द्रि॒यं-जु॑षा॒णा द्वे च॑) \textbf{(A7)} \newline \newline
                \textbf{ 2.6.8     अनुवाकं   8 - ऐन्द्रे प्रश्नौ प्रयाजयाज्याः} \newline
                                \textbf{ TB 2.6.8.1} \newline
                  समि॑द्ध॒ इन्द्र॑ उ॒षसा॒मनी॑के । पु॒रो॒रुचा॑ पूर्व॒कृद्वा॑वृधा॒नः । त्रि॒भिर्दे॒वैस्त्रिꣳ॒॒शता॒ वज्र॑बाहुः । ज॒घान॑ वृ॒त्रं ॅवि दुरो॑ ववार ॥ नरा॒शꣳसः॒ प्रति॒ शूरो॒ मिमा॑नः । तनू॒नपा॒त्-प्रति॑ य॒ज्ञ्स्य॒ धाम॑ । गोभि॑र्-व॒पावा॒न्-मधु॑ना सम॒ञ्जन्न् । हिर॑ण्यैश्च॒न्द्री य॑जति॒ प्रचे॑ताः ॥ ई॒डि॒तो दे॒वैर्.हरि॑वाꣳ अभि॒ष्टिः । आ॒जुह्वा॑नो ह॒विषा॒ शर्द्ध॑मानः \textbf{ 39} \newline
                  \newline
                                \textbf{ TB 2.6.8.2} \newline
                  पु॒र॒दं॒रो म॒घवा॒न्॒. वज्र॑बाहुः । आया॑तु य॒ज्ञ्मुप॑ नो जुषा॒णः ॥ जु॒षा॒णो ब॒र॒.हिर्.हरि॑वान्न॒ इन्द्रः॑ । प्रा॒चीनꣳ॑ सीदत्प्र॒दिशा॑ पृथि॒व्याः । उ॒रु॒व्यचाः॒ प्रथ॑मानꣳ स्यो॒नम् । आ॒दि॒त्यैर॒क्तं ॅवसु॑भिः स॒जोषाः᳚ ॥ इन्द्रं॒ दुरः॑ कव॒ष्यो॑ धाव॑मानाः । वृषा॑णं ॅयन्तु॒ जन॑यः सु॒पत्नीः᳚ । द्वारो॑ दे॒वीर॒भितो॒ विश्र॑यन्ताम् । सु॒वीरा॑ वी॒रं प्रथ॑माना॒ महो॑भिः । \textbf{ 40} \newline
                  \newline
                                \textbf{ TB 2.6.8.3} \newline
                  उ॒षासा॒ नक्ता॑ बृह॒ती बृ॒हन्त᳚म् । पय॑स्वती सु॒दुघे॒ शूर॒मिन्द्र᳚म् । पेश॑स्वती॒ तन्तु॑ना स॒म्ॅव्यय॑न्ती । दे॒वानां᳚ दे॒वं ॅय॑जतः सुरु॒क्मे ॥ दैव्या॒ मिमा॑ना॒ मन॑सा पुरु॒त्रा । होता॑रा॒विन्द्रं॑ प्रथ॒मा सु॒वाचा᳚ । मू॒र्द्धन्. य॒ज्ञ्स्य॒ मधु॑ना॒ दधा॑ना । प्रा॒चीनं॒ ज्योति॑र्.ह॒विषा॑ वृधातः ॥ ति॒स्रो दे॒वीर्.ह॒विषा॒ वर्द्ध॑मानाः । इन्द्रं॑ जुषा॒णा वृष॑णं॒ न पत्नीः᳚ \textbf{ 41} \newline
                  \newline
                                \textbf{ TB 2.6.8.4} \newline
                  अच्छि॑न्नं॒ तन्तुं॒ पय॑सा॒ सर॑स्वती । इडा॑ दे॒वी भार॑ती वि॒श्वतू᳚र्तिः ॥ त्वष्टा॒ दध॒दिन्द्रा॑य॒ शुष्म᳚म् । अपा॒कोऽचि॑ष्टुर्य॒शसे॑ पु॒रूणि॑ । वृषा॒ यज॒न् वृष॑णं॒ भूरि॑रेताः । मू॒र्द्धन्. य॒ज्ञ्स्य॒ सम॑नक्तु दे॒वान् ॥ वन॒स्पति॒-रव॑सृष्टो॒ न पाशैः᳚ । त्मन्या॑ सम॒ञ्जञ्छ॑मि॒ता न दे॒वः । इन्द्र॑स्य ह॒व्यैर्ज॒ठरं॑ पृणा॒नः । स्वदा॑ति ह॒व्यं मधु॑ना घृ॒तेन॑ ( ) ॥ स्तो॒काना॒मिन्दुं॒ प्रति॒ शूर॒ इन्द्रः॑ । वृ॒षा॒यमा॑णो वृष॒भ-स्तु॑रा॒षाट् । घृ॒त॒प्रुषा॒ मधु॑ना ह॒व्यमु॒न्दन्न् । मू॒र्द्धन्. य॒ज्ञ्स्य॑ जुषताꣳ॒॒ स्वाहा᳚ । \textbf{ 42} \newline
                  \newline
                                    (शर्द्ध॑मानो॒ - महो॑भिः॒ - पत्नी᳚र् - घृ॒तेन॑ च॒त्वारि॑ च) \textbf{(A8)} \newline \newline
                \textbf{ 2.6.9     अनुवाकं   9 - ऐन्द्रे प्रश्नौ वपादीनां याज्यानुवाक्याः} \newline
                                \textbf{ TB 2.6.9.1} \newline
                  "आच॑र्.षणि॒प्रा{11}", "वि॒वेष॒ यन्मा᳚{12}"॥ "तꣳ स॒द्ध्रीचीः᳚{13}"॥ स॒त्यमित्तन्न त्वावाꣳ॑ अ॒न्यो अस्ति॑ । इन्द्र॑ दे॒वो न मर्त्यो॒ ज्यायान्॑ । अह॒न्नहिं॑ परि॒शया॑न॒मर्णः॑ । अवा॑सृजो॒ऽपो अच्छा॑ समु॒द्रम् ॥ प्रस॑साहिषे पुरुहूत॒ शत्रून्॑ । ज्येष्ठ॑स्ते॒ शुष्म॑ इ॒ह रा॒तिर॑स्तु । इन्द्राभ॑र॒ दक्षि॑णेना॒ वसू॑नि । पतिः॒ सिन्धू॑नामसि रे॒वती॑नाम् ( ) ॥ स शेवृ॑ध॒मधि॑धा द्यु॒म्नम॒स्मे । महि॑ क्ष॒त्रं ज॑ना॒षाडि॑न्द्र॒ तव्य᳚म् । रक्षा॑ च नो म॒घोनः॑ पा॒हि सू॒रीन् । रा॒ये च॑ नः स्वप॒त्या इ॒षे धाः᳚ । \textbf{ 43} \newline
                  \newline
                                    (रे॒वती॑नां च॒त्वारि॑ च) \textbf{(A9)} \newline \newline
                \textbf{ 2.6.10    अनुवाकं   10 - ऐन्द्रे प्रश्नौ अनुयाजानामेकादश मैत्रावरुणप्रैषाः} \newline
                                \textbf{ TB 2.6.10.1} \newline
                  दे॒वं ब॒र्॒.हिरिन्द्रꣳ॑ सुदे॒वं दे॒वैः । वी॒रव॑थ्-स्ती॒र्णं ॅवेद्या॑मवर्द्धयत् । वस्तो᳚र्वृ॒तं प्राक्तो᳚र्भृ॒तम् । रा॒या ब॒र्॒.हिष्म॒तोऽत्य॑गात् । व॒सु॒वने॑ वसु॒धेय॑स्य वेतु॒ यज॑ ॥ दे॒वीर्द्वार॒ इन्द्रꣳ॑ संघा॒ते । वि॒ड्वीर्याम॑न्नवर्द्धयन्न् । आ व॒थ्सेन॒ तरु॑णेन कुमा॒रेण॑ च मीवि॒ता अपार्वा॑णम् । रे॒णुक॑काटं नुदन्ताम् । व॒सु॒वने॑ वसु॒धेय॑स्य वियन्तु॒ यज॑ । \textbf{ 44} \newline
                  \newline
                                \textbf{ TB 2.6.10.2} \newline
                  दे॒वी उ॒षासा॒ नक्ता᳚ । इन्द्रं॑ ॅय॒ज्ञे प्र॑य॒त्य॑ह्वेताम् । दैवी॒र्विशः॒ प्राया॑सिष्टाम् । सुप्री॑ते॒ सुधि॑ते अभूताम् । व॒सु॒वने॑ वसु॒धेय॑स्य वीतां॒ ॅयज॑ ॥ दे॒वी जोष्ट्री॒ वसु॑धिती । दे॒वमिन्द्र॑मवर्द्धताम् । अया᳚व्य॒न्याऽघा द्वेषाꣳ॑सि । आऽन्या वा᳚क्षी॒द् वसु॒ वार्या॑णि । यज॑मानाय शिक्षि॒ते \textbf{ 45} \newline
                  \newline
                                \textbf{ TB 2.6.10.3} \newline
                  व॒सु॒वने॑ वसु॒धेय॑स्य वीतां॒ ॅयज॑ ॥ दे॒वी ऊ॒र्जाहु॑ती॒ दुघे॑ सु॒दुघे᳚ । पय॒सेन्द्र॑मवर्द्धताम् । इष॒मूर्ज॑म॒न्या ऽवा᳚क्षीत् । सग्धिꣳ॒॒ सपी॑तिम॒न्या । नवे॑न॒ पूर्वं॒ दय॑माने । पु॒रा॒णेन॒ नव᳚म् । अधा॑ता॒-मूर्ज॑मू॒र्जाहु॑ती॒ वसु॒ वार्या॑णि । यज॑मानाय शिक्षि॒ते । व॒सु॒वने॑ वसु॒धेय॑स्य वीतां॒ ॅयज॑ ॥ 46(10)ट्.भ्.2.6.10.4दे॒वा दैव्या॒ होता॑रा । दे॒वमिन्द्र॑मवर्द्धताम् । ह॒ताघ॑शꣳ सा॒वाभा᳚र्ष्टां॒ ॅवसु॒ वार्या॑णि । यज॑मानाय शिक्षि॒तौ । व॒सु॒वने॑ वसु॒धेय॑स्य वीतां॒ ॅयज॑ ॥ दे॒वीस्ति॒स्रस्ति॒स्रो दे॒वीः । पति॒मिन्द्र॑मवर्द्धयन्न् । अस्पृ॑क्ष॒द्भार॑ती॒ दिव᳚म् । रु॒द्रैर्य॒ज्ञ्ꣳ सर॑स्वती । इडा॒ वसु॑मती गृ॒हान् \textbf{ 47} \newline
                  \newline
                                \textbf{ TB } \newline
                   \textbf{ 0} \newline
                  \newline
                                \textbf{ TB 2.6.10.5} \newline
                  व॒सु॒वने॑ वसु॒धेय॑स्य वियन्तु॒ यज॑ ॥ दे॒व इन्द्रो॒ नरा॒शꣳसः॑ । त्रि॒व॒रू॒थ-स्त्रि॑वन्धु॒रः । दे॒वमिन्द्र॑मवर्द्धयत् । श॒तेन॑ शितिपृ॒ष्ठाना॒माहि॑तः । स॒हस्रे॑ण॒ प्रव॑र्तते । मि॒त्रावरु॒णेद॑स्य हो॒त्रमर्.ह॑तः । बृह॒स्पतिः॑ स्तो॒त्रम् । अ॒श्विनाऽऽद्ध्व॑र्यवम् । व॒सु॒वने॑ वसु॒धेय॑स्य वेतु॒ यज॑ । \textbf{ 48} \newline
                  \newline
                                \textbf{ TB 2.6.10.6} \newline
                  दे॒व इन्द्रो॒ वन॒स्पतिः॑ । हिर॑ण्यपर्णो॒ मधु॑शाखः सुपिप्प॒लः । दे॒वमिन्द्र॑मवर्द्धयत् । दिव॒मग्रे॑णाप्रात् । आऽन्तरि॑क्षं पृथि॒वीम॑दृꣳहीत् । व॒सु॒वने॑ वसु॒धेय॑स्य वेतु॒ यज॑ ॥ दे॒वं ब॒र॒.हिर्वारि॑तीनाम् । दे॒वमिन्द्र॑मवर्द्धयत् । स्वा॒स॒स्थ-मिन्द्रे॒णास॑न्नम् । अ॒न्या ब॒र्॒.हीꣳष्य॒भ्य॑भूत् ( ) । व॒सु॒वने॑ वसु॒धेय॑स्य वेतु॒ यज॑ ॥ दे॒वो अ॒ग्निः स्वि॑ष्ट॒कृत् । दे॒वमिन्द्र॑मवर्द्धयत् । स्वि॑ष्टं कु॒र्वन्थ् स्वि॑ष्ट॒कृत् । स्वि॑ष्टम॒द्य क॑रोतु नः । व॒सु॒वने॑ वसु॒धेय॑स्य वेतु॒ यजे॑ । \textbf{ 49} \newline
                  \newline
                                                        \textbf{special korvai} \newline
              दे॒वं ब॒र्॒.हिर् दे॒वीर्द्वारो॑ दे॒वी उ॒षासा॒ नक्ता॑ दे॒वी जोष्ट्री॑ दे॒वी ऊ॒र्जाहु॑ती दे॒वा दैव्या॒ होता॑रा शिक्षि॒तौ दे॒वीस्ति॒स्रस्ति॒स्रो दे॒वीः पतिं॑ दे॒व इन्द्रो॒ नरा॒शꣳसो॑ दे॒व इन्द्रो॒ वन॒स्पति॑र् दे॒वं ब॒र्॒.हिर्वारि॑तीनां दे॒वो अ॒ग्निः स्वि॑ष्ट॒कृद् दे॒वम् । वे॒तु॒ वि॒य॒न्तु॒ च॒तुर्वी॑ता॒मेको॑ वियन्तु च॒तुर्वे॑तु । अ॒व॒र्द्ध॒य॒द॒व॒र्द्ध॒य॒न् त्रिर॑वर्द्धता॒मेको॑ ऽवर्द्धयꣳ श्च॒तुर॑वर्द्धयत् । वस्तो॒रा व॒थ्सेन॒ दैवी॒रया॒वीषꣳ॑ ह॒तास्पृ॑क्षच्छ॒तेन॒ दिव॒मिन्द्रꣳ॑ स्वास॒स्थꣳ स्वि॑ष्टम् । शिक्षि॒ते शि॑क्षि॒ते शि॑क्षि॒तौ । \newline
                                (वि॒य॒न्तु॒ यज॑ - शिक्षि॒ते - शि॑क्षि॒ते व॑सु॒वने॑ वसु॒धेय॑स्य वीतां॒ ॅयज॑-गृ॒हान्.-वे॑तु॒ यजा॑-भू॒थ् षट्च॑) \textbf{(A10)} \newline \newline
                \textbf{ 2.6.11    अनुवाकं   11 - पशुत्रये प्रय जार्था मैत्रावरुणप्रैषाः} \newline
                                \textbf{ TB 2.6.11.1} \newline
                  होता॑ यक्षथ्-स॒मिधा॒ऽग्नि-मि॒डस्प॒दे । अ॒श्विनेन्द्रꣳ॒॒ सर॑स्वतीम् । अ॒जो धू॒म्रो न गो॒धूमैः॒ क्व॑लैर्भेष॒जम् । मधु॒ शष्पै॒र्न तेज॑ इन्द्रि॒यम् । पयः॒ सोमः॑ परि॒स्रुता॑ घृ॒तं मधु॑ । वि॒यन्त्वाज्य॑स्य॒ होत॒र्यज॑ ॥ होता॑ यक्ष॒त्तनू॒नपा॒थ् सर॑स्वती । अवि॑र्मे॒षो न भे॑ष॒जम् । प॒था मधु॑म॒ता ऽऽभ॑रन्न् । अ॒श्विनेन्द्रा॑य वी॒र्य᳚म् \textbf{ 50} \newline
                  \newline
                                \textbf{ TB 2.6.11.2} \newline
                  बद॑रैरुप॒वाका॑भिर् भेष॒जं तोक्म॑भिः । पयः॒ सोमः॑ परि॒स्रुता॑ घृ॒तं मधु॑ । वि॒यन्त्वाज्य॑स्य॒ होत॒र्यज॑ ॥ होता॑ यक्ष॒न्नरा॒शꣳस॒नं न॒ग्नहु᳚म् । पतिꣳ॒॒ सुरा॑यै भेष॒जम् । मे॒षः सर॑स्वती भि॒षक् । रथो॒ न च॒न्द्र्य॑श्विनो᳚र्व॒पा इन्द्र॑स्य वी॒र्य᳚म् । बद॑रैरुप॒वाका॑भिर् भेष॒जं तोक्म॑भिः । पयः॒ सोमः॑ परि॒स्रुता॑ घृ॒तं मधु॑ । वि॒यन्त्वाज्य॑स्य॒ होत॒र्यज॑ । \textbf{ 51} \newline
                  \newline
                                \textbf{ TB 2.6.11.3} \newline
                  होता॑ यक्षदि॒डेडि॒त आ॒जुह्वा॑नः॒ सर॑स्वतीम् । इन्द्रं॒ बले॑न व॒र्द्धयन्न्॑ । ऋ॒ष॒भेण॒ गवे᳚न्द्रि॒यम् । अ॒श्विनेन्द्रा॑य वी॒र्य᳚म् । यवैः᳚ क॒र्कन्धु॑भिः । मधु॑ ला॒जैर्न मास॑रम् । पयः॒ सोमः॑ परि॒स्रुता॑ घृ॒तं मधु॑ । वि॒यन्त्वाज्य॑स्य॒ होत॒र्यज॑ ॥ होता॑ यक्षद्-ब॒र॒.हिः सु॒ष्टरी॒मोऽर्ण॑म्रदाः । भि॒षङ्नास॑त्या \textbf{ 52} \newline
                  \newline
                                \textbf{ TB 2.6.11.4} \newline
                  भि॒षजा॒ऽश्विनाऽश्वा॒ शिशु॑मती । भि॒षग्धे॒नुः सर॑स्वती । भि॒षग्दु॒ह इन्द्रा॑य भेष॒जम् । पयः॒ सोमः॑ परि॒स्रुता॑ घृ॒तं मधु॑ । वि॒यन्त्वाज्य॑स्य॒ होत॒र्यज॑ ॥ होता॑ यक्ष॒द्दुरो॒ दिशः॑ । क॒व॒ष्यो॑ न व्यच॑स्वतीः । अ॒श्विभ्यां॒ न दुरो॒ दिशः॑ । इन्द्रो॒ न रोद॑सी॒ दुघे᳚ । दु॒हे कामा॒न्थ् सर॑स्वती \textbf{ 53} \newline
                  \newline
                                \textbf{ TB 2.6.11.5} \newline
                  अ॒श्विनेन्द्रा॑य भेष॒जम् । शु॒क्रं न ज्योति॑रिन्द्रि॒यम् । पयः॒ सोमः॑ परि॒स्रुता॑ घृ॒तं मधु॑ । वि॒यन्त्वाज्य॑स्य॒ होत॒र्यज॑ ॥ होता॑ यक्षथ् सु॒पेश॑सो॒षे नक्तं॒ दिवा᳚ । अ॒श्विना॑ संजाना॒ने । सम॑ञ्जाते॒ सर॑स्वत्या । त्विषि॒मिन्द्रे॒ न भे॑ष॒जम् । श्ये॒नो न रज॑सा हृ॒दा । पयः॒ सोमः॑ परि॒स्रुता॑ घृ॒तं मधु॑ \textbf{ 54} \newline
                  \newline
                                \textbf{ TB 2.6.11.6.} \newline
                  वि॒यन्त्वाज्य॑स्य॒ होत॒र्यज॑ ॥ होता॑ यक्ष॒द्दैव्या॒ होता॑रा भि॒षजा॒ऽश्विना᳚ । इन्द्रं॒ न जागृ॑वी॒ दिवा॒ नक्तं॒ न भे॑ष॒जैः । शूषꣳ॒॒ सर॑स्वती भि॒षक् । सीसे॑न दुह इन्द्रि॒यम् । पयः॒ सोमः॑ परि॒स्रुता॑ घृ॒तं मधु॑ । वि॒यन्त्वाज्य॑स्य॒ होत॒र्यज॑ ॥ होता॑ यक्षत् ति॒स्रो दे॒वीर्न भे॑ष॒जम् । त्रय॑स्त्रि॒धात॑वो॒ऽपसः॑ । रू॒पमिन्द्रे॑ हिर॒ण्यय᳚म् \textbf{ 55} \newline
                  \newline
                                \textbf{ TB 2.6.11.7} \newline
                  अ॒श्विनेडा॒ न भार॑ती । वा॒चा सर॑स्वती । मह॒ इन्द्रा॑य दधुरिन्द्रि॒यम् । पयः॒ सोमः॑ परि॒स्रुता॑ घृ॒तं मधु॑ । वि॒यन्त्वाज्य॑स्य॒ होत॒र्यज॑ ॥ होता॑ यक्ष॒त्त्वष्टा॑र॒-मिन्द्र॑म॒श्विना᳚ । भि॒षजं॒ न सर॑स्वतीम् । ओजो॒ न जू॒तिरि॑न्द्रि॒यम् । वृको॒ न र॑भ॒सो भि॒षक् । यशः॒ सुर॑या भेष॒जम् \textbf{ 56} \newline
                  \newline
                                \textbf{ TB 2.6.11.8} \newline
                  श्रि॒या न मास॑रम् । पयः॒ सोमः॑ परि॒स्रुता॑ घृ॒तं मधु॑ । वि॒यन्त्वाज्य॑स्य॒ होत॒र्यज॑ ॥ होता॑ यक्ष॒द्-वन॒स्पति᳚म् । श॒मि॒तारꣳ॑ श॒तक्र॑तुम् । भी॒मं न म॒न्युꣳ राजा॑नं ॅव्या॒घ्रं नम॑सा॒ऽश्विना॒ भामं᳚ । सर॑स्वती भि॒षक् । इन्द्रा॑य दुह इन्द्रि॒यम् । पयः॒ सोमः॑ परि॒स्रुता॑ घृ॒तं मधु॑ । वि॒यन्त्वाज्य॑स्य॒ होत॒र्यज॑ । \textbf{ 57} \newline
                  \newline
                                \textbf{ TB 2.6.11.9} \newline
                  होता॑ यक्षद॒ग्निꣳ स्वाहाऽऽज्य॑स्य स्तो॒काना᳚म् । स्वाहा॒ मेद॑सां॒ पृथ॑क् । स्वाहा॒ छाग॑म॒श्विभ्या᳚म् । स्वाहा॑ मे॒षꣳ सर॑स्वत्यै । स्वाहा॑र्.ष॒भ-मिन्द्रा॑य सिꣳ॒॒हाय॒ सह॑सेन्द्रि॒यम् । स्वाहा॒ऽग्निं न भे॑ष॒जम् । स्वाहा॒ सोम॑मिन्द्रि॒यम् । स्वाहेन्द्रꣳ॑ सु॒त्रामा॑णꣳ सवि॒तारं॒ ॅवरु॑णं भि॒षजां॒ पति᳚म् । स्वाहा॒ वन॒स्पतिं॑ प्रि॒यं पाथो॒ न भे॑ष॒जम् । स्वाहा॑ दे॒वाꣳ आ᳚ज्य॒पान् \textbf{ 58} \newline
                  \newline
                                \textbf{ TB 2.6.11.10} \newline
                  स्वाहा॒ऽग्निꣳ हो॒त्राज्जु॑षा॒णो अ॒ग्निर् भे॑ष॒जम् । पयः॒ सोमः॑ परि॒स्रुता॑ घृ॒तं मधु॑ । वि॒यन्त्वाज्य॑स्य॒ होत॒र्यज॑ ॥ होता॑ यक्षद॒श्विना॒ सर॑स्वती॒मिन्द्रꣳ॑ सु॒त्रामा॑णम् । इ॒मे सोमाः᳚ सु॒रामा॑णः । छागै॒र्न मे॒षैर्.ऋ॑ष॒भैः सु॒ताः । शष्पै॒र्न तोक्म॑भिः । ला॒जैर्मह॑स्वन्तः । मदा॒ मास॑रेण॒ परि॑ष्कृताः । शु॒क्राः पय॑स्वन्तो॒ऽमृताः᳚ ( ) । प्रस्थि॑ता वो मधु॒श्चुतः॑ । तान॒श्विना॒ सर॑स्व॒तीन्द्रः॑ सु॒त्रामा॑ वृत्र॒हा । जु॒षन्ताꣳ॑ सौ॒म्यं मधु॑ । पिब॑न्तु॒ मद॑न्तु वि॒यन्तु॒ सोम᳚म् । होत॒र्यज॑ । \textbf{ 59} \newline
                  \newline
                                                        \textbf{special korvai} \newline
              स॒मिधा॒ऽग्निꣳ षट् । तनू॒नपा᳚थ् स॒प्त । नरा॒शꣳस॒मृषिः॑ । इ॒डेडि॒तो यवै॑र॒ष्टौ । ब॒र्॒.हिः स॒प्त । दुरो॒ऽश्विना॑ शु॒क्रन्नव॑ । सु॒पेश॑सो॒षे नक्त॒मृषिः॑। दैव्या॒ होता॑रा॒ सीसे॑न॒ रसः॑ । ति॒स्रस्त्वष्टा॑रम॒ष्टाव॑ष्टौ । वन॒स्पति॒मृषिः॑ । अ॒ग्निं त्रयो॑दश । अ॒श्विना॒ द्वाद॑श॒ त्रयो॑दश । स॒मिधा॒ऽग्निं बद॑रै॒र् बद॑रै॒र् यवै॑र॒श्विना॒ त्विषि॑म॒श्विना॒ न भे॑ष॒जꣳ रू॒पम॒श्विना॑ भी॒मं भाम᳚म् । \newline
                                (वी॒र्यं - ॅवि॒यन्त्वाज्य॑स्य॒ होत॒र्यज॒ - नास॑त्या॒ - सर॑स्वती॒ - मधु॑ - हिर॒ण्ययं॑ - भेष॒जं - ॅवि॒यन्त्वाज्य॑स्य॒ होत॒र्यजा᳚ - ज्य॒पा - न॒मृताः॒ पञ्च॑ च) \textbf{(A11)} \newline \newline
                \textbf{ 2.6.12    अनुवाकं   12 - पशुत्रये प्रयाजयाज्या आप्रियः} \newline
                                \textbf{ TB 2.6.12.1} \newline
                  समि॑द्धो अ॒ग्निर॑श्विना । त॒प्तो घ॒र्मो वि॒राट्थ् सु॒तः । दु॒हे धे॒नुः सर॑स्वती । सोमꣳ॑ शु॒क्रमि॒हेन्द्रि॒यम् ॥ त॒नू॒पा भि॒षजा॑ सु॒ते । अ॒श्विनो॒भा सर॑स्वती । मद्ध्वा॒ रजाꣳ॑सीन्द्रि॒यम् । इन्द्रा॑य प॒थिभि॑र् वहान् ॥ इन्द्रा॒येन्दुꣳ॒॒ सर॑स्वती । नरा॒शꣳसे॑न न॒ग्नहुः॑ \textbf{ 60} \newline
                  \newline
                                \textbf{ TB 2.6.12.2} \newline
                  अधा॑ताम॒श्विना॒ मधु॑ । भे॒ष॒जं भि॒षजा॑ सु॒ते ॥ आ॒जुह्वा॑ना॒ सर॑स्वती । इन्द्रा॑येन्द्रि॒याणि॑ वी॒र्य᳚म् । इडा॑भिरश्विना॒विष᳚म् । समूर्जꣳ॒॒ सꣳ र॒यिं द॑धुः ॥ अश्वि॑ना॒ नमु॑चेः सु॒तम् । सोमꣳ॑ शु॒क्रं प॑रि॒स्रुता᳚ । सर॑स्वती॒ तमाभ॑रत् । ब॒र्॒.हिषेन्द्रा॑य॒ पात॑वे । \textbf{ 61} \newline
                  \newline
                                \textbf{ TB 2.6.12.3} \newline
                  क॒व॒ष्यो॑ न व्यच॑स्वतीः । अ॒श्विभ्यां॒ न दुरो॒ दिशः॑ । इन्द्रो॒ न रोद॑सी॒ दुघे᳚ । दु॒हे कामा॒न्थ्-सर॑स्वती ॥ उ॒षासा॒ नक्त॑मश्विना । दिवेन्द्रꣳ॑ सा॒यमि॑न्द्रि॒यैः । स॒जां॒ना॒ने सु॒पेश॑सा । सम॑ञ्जाते॒ सर॑स्वत्या ॥ पा॒तं नो॑ अश्विना॒ दिवा᳚ । पा॒हि नक्तꣳ॑ सरस्वति \textbf{ 62} \newline
                  \newline
                                \textbf{ TB 2.6.12.4} \newline
                  दैव्या॑ होतारा भिषजा । पा॒तमिन्द्रꣳ॒॒ सचा॑ सु॒ते ॥ ति॒स्रस्त्रे॒धा सर॑स्वती । अ॒श्विना॒ भार॒तीडा᳚ । ती॒व्रं प॑रि॒स्रुता॒ सोम᳚म् । इन्द्रा॑य सुषवु॒र्मद᳚म् ॥ अश्वि॑ना भेष॒जं मधु॑ । भे॒ष॒जं नः॒ सर॑स्वती । इन्द्रे॒ त्वष्टा॒ यशः॒ श्रिय᳚म् । रू॒पꣳ रू॑पमधुः सु॒ते ( ) ॥ ऋ॒तु॒थेन्द्रो॒ वन॒स्पतिः॑ । श॒श॒मा॒नः प॑रि॒स्रुता᳚ । की॒लाल॑म॒श्विभ्यां॒ मधु॑ । दु॒हे धे॒नुः सर॑स्वती ॥ गोभि॒र्न सोम॑मश्विना । मास॑रेण परि॒ष्कृता᳚ । सम॑धाताꣳ॒॒ सर॑स्वत्या । स्वाहेन्द्रे॑ सु॒तं मधु॑ । \textbf{ 63} \newline
                  \newline
                                    (न॒ग्नहुः॒ - पात॑वे - सरस्व - त्यधुः सु॒ते᳚ऽष्टौ च॑) \textbf{(A12)} \newline \newline
                \textbf{ 2.6.13    अनुवाकं   13 - पशुत्रये वपादीनां याज्यानुवाक्याः} \newline
                                \textbf{ TB 2.6.13.1} \newline
                  अ॒श्विना॑ ह॒विरि॑न्द्रि॒यम् । नमु॑चेर्द्धि॒या सर॑स्वती । आ शु॒क्रमा॑सु॒राद्व॒सु । म॒घमिन्द्रा॑य जभ्रिरे ॥ यम॒श्विना॒ सर॑स्वती । ह॒विषेन्द्र॒-मव॑र्द्धयन्न् । स बि॑भेद ब॒लं (व॒लं) म॒घम् । नमु॑चावासु॒रे सचा᳚ । तमिन्द्रं॑ प॒शवः॒ सचा᳚ ॥ अ॒श्विनो॒भा सर॑स्वती \textbf{ 64} \newline
                  \newline
                                \textbf{ TB 2.6.13.2} \newline
                  दधा॑ना अ॒भ्य॑नूषत । ह॒विषा॑ य॒ज्ञ्मि॑न्द्रि॒यम् ॥ य इन्द्र॑ इन्द्रि॒यं द॒धुः । स॒वि॒ता वरु॑णो॒ भगः॑ । स सु॒त्रामा॑ ह॒विष्प॑तिः । यज॑मानाय सश्चत ॥ स॒वि॒ता वरु॑णो॒ दध॑त् । यज॑मानाय दा॒शुषे᳚ । आद॑त्त॒ नमु॑चे॒र्वसु॑ । सु॒त्रामा॒ बल॑मिन्द्रि॒यम् । \textbf{ 65} \newline
                  \newline
                                \textbf{ TB 2.6.13.3} \newline
                  वरु॑णः क्ष॒त्रमि॑न्द्रि॒यम् । भगे॑न सवि॒ता श्रिय᳚म् । सु॒त्रामा॒ यश॑सा॒ बल᳚म् । दधा॑ना य॒ज्ञ्मा॑शत ॥ अश्वि॑ना॒ गोभि॑रिन्द्रि॒यम् । अश्वे॑भिर् वी॒र्यं॑ बल᳚म् । ह॒विषेन्द्रꣳ॒॒ सर॑स्वती । यज॑मान-मवर्द्धयन्न् ॥ ता नास॑त्या सु॒पेश॑सा । हिर॑ण्यवर्तनी॒ नरा᳚ ( ) । सर॑स्वती ह॒विष्म॑ती । इन्द्र॒ कम॑र्सु नोऽवत ॥ ता भि॒षजा॑ सु॒कर्म॑णा । सा सु॒दुघा॒ सर॑स्वती । स वृ॑त्र॒हा श॒तक्र॑तुः । इन्द्रा॑य दधुरिन्द्रि॒यम् । \textbf{ 66} \newline
                  \newline
                                    (उ॒भा सर॑स्वती॒ - बल॑मिन्द्रि॒यं - नरा॒ षट्च॑) \textbf{(A13)} \newline \newline
                \textbf{ 2.6.14    अनुवाकं   14 - पशुत्रये अनूयाजप्रैषाः} \newline
                                \textbf{ TB 2.6.14.1} \newline
                  दे॒वं ब॒र्॒.हिः सर॑स्वती । सु॒दे॒वमिन्द्रे॑ अ॒श्विना᳚ । तेजो॒ न चक्षु॑र॒क्ष्योः । ब॒र्॒.हिषा॑ दधुरिन्द्रि॒यम् । व॒सु॒वने॑ वसु॒धेय॑स्य वियन्तु॒ यज॑ ॥ दे॒वीर्द्वारो॑ अ॒श्विना᳚ । भि॒षजेन्द्रे॒ सर॑स्वती । प्रा॒णं न वी॒र्यं॑ न॒सि । द्वारो॑ दधुरिन्द्रि॒यम् । व॒सु॒वने॑ वसु॒धेय॑स्य वियन्तु॒ यज॑ । \textbf{ 67} \newline
                  \newline
                                \textbf{ TB 2.6.14.2} \newline
                  दे॒वी उ॒षासा॑व॒श्विना᳚ । भि॒षजेन्द्रे॒ सर॑स्वती । बलं॒ न वाच॑मा॒स्ये᳚ । उ॒षाभ्यां᳚ दधुरिन्द्रि॒यम् । व॒सु॒वने॑ वसु॒धेय॑स्य वियन्तु॒ यज॑ ॥ दे॒वी जोष्ट्री॑ अ॒श्विना᳚ । सु॒त्रामेन्द्रे॒ सर॑स्वती । श्रोत्रं॒ न कर्ण॑यो॒र्यशः॑ । जोष्ट्री᳚भ्यां दधुरिन्द्रि॒यम् । व॒सु॒वने॑ वसु॒धेय॑स्य वियन्तु॒ यज॑ । \textbf{ 68} \newline
                  \newline
                                \textbf{ TB 2.6.14.3} \newline
                  दे॒वी ऊ॒र्जाहु॑ती॒ दुघे॑ सु॒दुघे᳚ । पय॒सेन्द्रꣳ॒॒ सर॑स्वत्य॒श्विना॑ भि॒षजा॑ऽवत । शु॒क्रं न ज्योतिः॒ स्तन॑यो॒राहु॑ती धत्त इन्द्रि॒यम् । व॒सु॒वने॑ वसु॒धेय॑स्य वियन्तु॒ यज॑ ॥ दे॒वा दे॒वानां᳚ भि॒षजा᳚ । होता॑रा॒विन्द्र॑-म॒श्विना᳚ । व॒ष॒ट्का॒रैः सर॑स्वती । त्विषिं॒ न हृद॑ये म॒तिम् । होतृ॑भ्यां दधुरिन्द्रि॒यम् । व॒सु॒वने॑ वसु॒धेय॑स्य वियन्तु॒ यज॑ । \textbf{ 69} \newline
                  \newline
                                \textbf{ TB 2.6.14.4} \newline
                  दे॒वी-स्ति॒स्र-स्ति॒स्रो दे॒वीः । सर॑स्वत्य॒श्विना॒ भार॒तीडा᳚ । शूषं॒ न मद्ध्ये॒ नाभ्या᳚म् । इन्द्रा॑य दधुरिन्द्रि॒यम् । व॒सु॒वने॑ वसु॒धेय॑स्य वियन्तु॒ यज॑ ॥ दे॒व इन्द्रो॒ नरा॒शꣳसः॑ । त्रि॒व॒रू॒थः सर॑स्वत्या॒ ऽश्विभ्या॑मीयते॒ रथः॑ । रेतो॒ न रू॒पम॒मृतं॑ ज॒नित्र᳚म् । इन्द्रा॑य॒ त्वष्टा॒ दध॑दिन्द्रि॒याणि॑ । व॒सु॒वने॑ वसु॒धेय॑स्य वियन्तु॒ यज॑ । \textbf{ 70} \newline
                  \newline
                                \textbf{ TB 2.6.14.5} \newline
                  दे॒व इन्द्रो॒ वन॒स्पतिः॑ । हिर॑ण्यपर्णो अ॒श्विभ्या᳚म् । सर॑स्वत्याः सुपिप्प॒लः । इन्द्रा॑य पच्यते॒ मधु॑ । ओजो॒ न जू॒तिमृ॑ष॒भो न भाम᳚म् । वन॒स्पति॑र्नो॒ दध॑दिन्द्रि॒याणि॑ । व॒सु॒वने॑ वसु॒धेय॑स्य वियन्तु॒ यज॑ ॥ दे॒वं ब॒र्॒.हिर्वारि॑तीनाम् । अ॒द्ध्व॒रे स्ती॒र्णम॒श्विभ्या᳚म् । ऊर्ण॑म्रदाः॒ सर॑स्वत्याः \textbf{ 71} \newline
                  \newline
                                \textbf{ TB 2.6.14.6.} \newline
                  स्यो॒नमि॑न्द्र ते॒ सदः॑ । ई॒शायै॑ म॒न्युꣳ राजा॑नं ब॒र्॒.हिषा॑ दधुरिन्द्रि॒यम् । व॒सु॒वने॑ वसु॒धेय॑स्य वियन्तु॒ यज॑ ॥ दे॒वो अ॒ग्निः स्वि॑ष्ट॒कृत् । दे॒वान्.य॑क्षद्-यथाय॒थम् । होता॑रा॒-विन्द्र॑म॒श्विना᳚ । वा॒चा वाचꣳ॒॒ सर॑स्वतीम् । अ॒ग्निꣳ सोमꣳ॑ स्विष्ट॒कृत् । स्वि॑ष्ट॒ इन्द्रः॑ सु॒त्रामा॑ सवि॒ता वरु॑णो भि॒षक् । इ॒ष्टो दे॒वो वन॒स्पतिः॑ ( ) । स्वि॑ष्टा दे॒वा आ᳚ज्य॒पाः । इ॒ष्टो अ॒ग्निर॒ग्निना᳚ । होता॑ हो॒त्रे स्वि॑ष्ट॒कृत् । यशो॒ न दध॑दिन्द्रि॒यम् । ऊर्ज॒मप॑चितिꣳ स्व॒धाम् । व॒सु॒वने॑ वसु॒धेय॑स्य वियन्तु॒ यज॑ । \textbf{ 72} \newline
                  \newline
                                                        \textbf{special korvai} \newline
              (दे॒वं ब॒र्॒.हिर् दे॒वीर् द्वारो॑ दे॒वी उ॒षासा॑व॒श्विना॑ दे॒वी जोष्ट्र॑ दे॒वी ऊ॒र्जाहू॑ती दे॒वा दे॒वानां᳚ भि॒षजा॑ वषट्का॒रैर् दे॒वीस्तिस्रस्ति॒स्रो दे॒वीः सर॑स्वती दे॒व इन्द्रो॒ नरा॒शꣳसो॑ दे॒व इन्द्रो॒ वन॒स्पति॑र्दे॒वं ब॒र्॒.हिर्वारि॑तीनां दे॒वोअ॒ग्निः स्वि॑ष्ट॒कृद् दे॒वान्) । \newline
                            \textbf{special korvai} \newline
              (स॒मिधा॒ऽग्निं दे॒वं ब॒र्॒.हिः सर॑स्वत्या॒श्विना॒ सर्वं॑ ॅवियन्तु । द्वार॑स्ति॒स्रः संर्व॑ वियन्तु । अ॒ज इन्द्र॒मोजो॒ऽग्निं परः॒ सर॑स्वतीम् । नक्तं॒पूर्वः॒ सर॑स्वति । अ॒न्यत्र॒ सर॑स्वति । भि॒षक् पूर्वं॑ दुह इन्द्रि॒यं । अ॒न्यत्र॑ दधुरिन्द्रि॒यम् । सौ॒त्रा॒म॒ण्याꣳ सु॑तासु॒ती । अ॒ञ्जन्त्य॒यं ॅयज॑मानः) \newline
                                (द्वारो॑ दधुरिन्द्रि॒यं ॅव॑सु॒वने॑ वसु॒धेय॑स्य वियन्तु॒ यज॒ - जोष्ट्री᳚भ्यां दधुरिन्द्रि॒यं ॅव॑सु॒वने॑ वसु॒धेय॑स्य वियन्तु॒ यज॒ -होतृ॑भ्यां दधुरिन्द्रि॒यं ॅव॑सु॒वने॑ वसु॒धेय॑स्य वियन्तु॒ यजे᳚ - न्द्रि॒याणि॑ वसु॒वने॑ वसु॒धेय॑स्य वियन्तु॒ यज॒ - सर॑स्वत्या॒ - वन॒स्पतिः॒ षट्च॑) \textbf{(A14)} \newline \newline
                \textbf{ 2.6.15    अनुवाकं   15 - सूक्तवाकप्रैषः} \newline
                                \textbf{ TB 2.6.15.1} \newline
                  अ॒ग्निम॒द्य होता॑रमवृणीत । अ॒यꣳ सु॑तासु॒ती यज॑मानः । पच॑न् प॒क्तीः । पच॑न् पुरो॒डाशान्॑ । गृ॒ह्णन् ग्रहान्॑ । ब॒द्ध्नन्न॒श्विभ्यां॒ छागꣳ॒॒ सर॑स्वत्या॒ इन्द्रा॑य । ब॒द्ध्नन्थ् सर॑स्वत्यै मे॒षमिन्द्रा॑या॒श्विभ्या᳚म् । ब॒द्ध्नन्निन्द्रा॑यर्.ष॒भ-म॒श्विभ्याꣳ॒॒ सर॑स्वत्यै । सू॒प॒स्था अ॒द्य दे॒वो वन॒स्पति॑रभवत् । अ॒श्विभ्यां॒ छागे॑न॒ सर॑स्वत्या॒ इन्द्रा॑य \textbf{ 73} \newline
                  \newline
                                \textbf{ TB 2.6.15.2} \newline
                  सर॑स्वत्यै मे॒षेणेन्द्रा॑या॒श्विभ्या᳚म् । इन्द्रा॑यर्.ष॒भेणा॒श्विभ्याꣳ॒॒ सर॑स्वत्यै । अक्षꣳ॒॒स्तान्मे॑द॒स्तः प्रति॑ पच॒ताऽग्र॑भीषुः । अवी॑वृधन्त॒ ग्रहैः᳚ । अपा॑ताम॒श्विना॒ सर॑स्व॒तीन्द्रः॑ सु॒त्रामा॑ वृत्र॒हा । सोमा᳚न्थ्-सु॒राम्णः॑ । उपो॑ उक्थाम॒दाः श्रौ॒द्विमदा॑ अदन्न् । अवी॑वृधन् ताङ्गू॒षैः । त्वाम॒द्यर्.ष॑ आर्.षेयर्.षीणां नपादवृणीत । अ॒यꣳ सु॑तासु॒ती यज॑मानः ( ) । ब॒हुभ्य॒ आसंग॑तेभ्यः । ए॒ष मे॑ दे॒वेषु॒ वसु॒ वार्या य॑क्ष्यत॒ इति॑ । ता या दे॒वा दे॑व॒दाना॒न्यदुः॑ । तान्य॑स्मा॒ आ च॒ शास्व॑ । आ च॑ गुरस्व । इ॒षि॒तश्च॑ होत॒रसि॑ भद्र॒वाच्या॑य॒ प्रेषि॑तो॒ मानु॑षः । सू॒क्त॒वा॒काय॑ सू॒क्ता ब्रू॑हि । \textbf{ 74} \newline
                  \newline
                                    (इन्द्रा॑य॒ - यज॑मानः स॒प्त च॑) \textbf{(A15)} \newline \newline
                \textbf{ 2.6.16    अनुवाकं   16 - पितृयज्ञ्याज्यानुवाक्याः} \newline
                                \textbf{ TB 2.6.16.1} \newline
                  "उ॒शन्त॑स्त्वा हवामह॒{14}", "आ नो॑ अग्ने सुके॒तुना᳚{15}"। "त्वꣳ सो॑म म॒हे भगं॒{16}","त्वꣳ सो॑म॒ प्रचि॑कितो मनी॒षा{17}"। "त्वया॒ हि नः॑ पि॒तरः॑ सोम॒ पूर्वे॒{18}", "त्वꣳ सो॑म पि॒तृभिः॑ सम्ॅविदा॒नः{19}" । "बर्.हि॑षदः पितर॒{20}", "आऽहं पि॒तॄन्{21}" । "उप॑हूताः पि॒तरो{22}", "ऽग्नि॑ष्वात्ताः पितरः{23}" ॥ अ॒ग्नि॒ष्वा॒त्तानृ॑तु॒मतो॑ हवामहे । नरा॒शꣳसे॑ सोमपी॒थं ॅय आ॒शुः । ते नो॒ अर्व॑न्तः सु॒हवा॑ भवन्तु । शं नो॑ भवन्तु द्वि॒पदे॒ शं चतु॑ष्पदे ॥ ये अ॑ग्निष्वा॒त्ता येऽन॑ग्निष्वात्ताः \textbf{ 75} \newline
                  \newline
                                \textbf{ TB 2.6.16.2} \newline
                  अꣳ॒॒हो॒मुचः॑ पि॒तरः॑ सो॒म्यासः॑ । परेऽव॑रे॒ऽमृता॑सो॒ भव॑न्तः । अधि॑ब्रुवन्तु॒ ते अ॑वन्त्व॒स्मान् ॥ वा॒न्या॑यै दु॒ग्धे जु॒षमा॑णाः कर॒म्भम् । उ॒दीरा॑णा॒ अव॑रे॒ परे॑ च । अ॒ग्नि॒ष्वा॒त्ता ऋ॒तुभिः॑ सम्ॅविदा॒नाः । इन्द्र॑वन्तो ह॒विरि॒दं जु॑षन्ताम् ॥ "यद॑ग्ने कव्यवाहन॒{24}, "त्वम॑ग्न ईडि॒तो जा॑तवेदः{25}" । "मात॑लीक॒व्यैः{26}" ॥ ये ता॑तृ॒पुर्दे॑व॒त्रा जेह॑मानाः ( ) । हो॒त्रा॒वृधः॒ स्तोम॑तष्टासो अ॒र्कैः । आऽग्ने॑ याहि सुवि॒दत्रे॑भिर॒र्वाङ् । स॒त्यैः क॒व्यैः पि॒तृभि॑र्-घर्म॒सद्भिः॑ ॥ ह॒व्य॒वाह॑म॒जरं॑ पुरुप्रि॒यम् । अ॒ग्निं घृ॒तेन॑ ह॒विषा॑ सप॒र्यन्न् । उपा॑सदं कव्य॒वाहं॑ पितृ॒णाम् । स नः॑ प्र॒जां ॅवी॒रव॑तीꣳ॒॒ समृ॑ण्वतु । \textbf{ 76} \newline
                  \newline
                                    (अन॑ग्निष्वात्ता॒ - जेह॑मानाः स॒प्त च॑) \textbf{(A16)} \newline \newline
                \textbf{ 2.6.17    अनुवाकं   17 - ऐन्द्रो पशौ एकादश प्रयाजाः} \newline
                                \textbf{ TB 2.6.17.1} \newline
                  होता॑ यक्षदि॒डस्प॒दे । स॒मि॒धा॒नं म॒हद्यशः॑ । सुष॑मिद्धं॒ ॅवरे᳚ण्यम् । अ॒ग्निमिन्द्रं॑ ॅवयो॒धस᳚म् । गा॒य॒त्रीं छन्द॑ इन्द्रि॒यम् । त्र्यविं॒ गां ॅवयो॒ दध॑त् । वेत्वाज्य॑स्य॒ होत॒र्यज॑ ॥ होता॑ यक्ष॒च्छुचि॑व्रतम् । तनू॒नपा॑तमु॒द्भिद᳚म् । यं गर्भ॒मदि॑तिर्द॒धे \textbf{ 77} \newline
                  \newline
                                \textbf{ TB 2.6.17.2} \newline
                  शुचि॒मिन्द्रं॑ ॅवयो॒धस᳚म् । उ॒ष्णिहं॒ छन्द॑ इन्द्रि॒यम् । दि॒त्य॒वाहं॒ गां ॅवयो॒ दध॑त् । वेत्वाज्य॑स्य॒ होत॒र्यज॑ ॥ होता॑ यक्षदी॒डेन्य᳚म् । ई॒डि॒तं ॅवृ॑त्र॒हन्त॑मम् । इडा॑भि॒रीड्यꣳ॒॒ सहः॑ । सोम॒मिन्द्रं॑ ॅवयो॒धस᳚म् । अ॒नु॒ष्टुभं॒ छन्द॑ इन्द्रि॒यम् । त्रि॒व॒थ्सं गां ॅवयो॒ दध॑त् \textbf{ 78} \newline
                  \newline
                                \textbf{ TB 2.6.17.3} \newline
                  वेत्वाज्य॑स्य॒ होत॒र्यज॑ ॥ होता॑ यक्षथ् सुबर्.हि॒षद᳚म् । पू॒ष॒ण्वन्त॒-मम॑र्त्यम् । सीद॑न्तं ब॒र्॒.हिषि॑ प्रि॒ये । अ॒मृतेन्द्रं॑ ॅवयो॒धस᳚म् । बृ॒ह॒तीं छन्द॑ इन्द्रि॒यम् । पञ्चा॑विं॒ गां ॅवयो॒ दध॑त् । वेत्वाज्य॑स्य॒ होत॒र्यज॑ ॥ होता॑ यक्ष॒द् व्यच॑स्वतीः । सु॒प्रा॒य॒णा ऋ॑ता॒वृधः॑ \textbf{ 79} \newline
                  \newline
                                \textbf{ TB 2.6.17.4} \newline
                  द्वारो॑ दे॒वीर्.हि॑र॒ण्ययीः᳚ । ब्र॒ह्माण॒ इन्द्रं॑ ॅवयो॒धस᳚म् । प॒ङ्क्तिं छन्द॑ इ॒हेन्द्रि॒यम् । तु॒र्य॒वाहं॒ गां ॅवयो॒ दध॑त् । वेत्वाज्य॑स्य॒ होत॒र्यज॑ ॥ होता॑ यक्षथ्-सु॒पेश॑से । सु॒शि॒ल्पे बृ॑ह॒ती उ॒भे । नक्तो॒षासा॒ न द॑र्.श॒ते । विश्व॒मिन्द्रं॑ ॅवयो॒धस᳚म् । त्रि॒ष्टुभं॒ छन्द॑ इन्द्रि॒यम् \textbf{ 80} \newline
                  \newline
                                \textbf{ TB 2.6.17.5} \newline
                  प॒ष्ठ॒वाहं॒ गां ॅवयो॒ दध॑त् । वेत्वाज्य॑स्य॒ होत॒र्यज॑ ॥ होता॑ यक्ष॒त् प्रचे॑तसा । दे॒वाना॑मुत्त॒मं ॅयशः॑ । होता॑रा॒ दैव्या॑ क॒वी । स॒युजेन्द्रं॑ ॅवयो॒धस᳚म् । जग॑तीं॒ छन्द॑ इ॒हेन्द्रि॒यम् । अ॒न॒ड्वाहं॒ गां ॅवयो॒ दध॑त् । वेत्वाज्य॑स्य॒ होत॒र्यज॑ ॥ होता॑ यक्ष॒त् पेश॑स्वतीः \textbf{ 81} \newline
                  \newline
                                \textbf{ TB 2.6.17.6.} \newline
                  ति॒स्रो दे॒वीर्.हि॑र॒ण्ययीः᳚ । भार॑तीर्-बृह॒तीर्म॒हीः । पति॒मिन्द्रं॑ ॅवयो॒धस᳚म् । वि॒राजं॒ छन्द॑ इ॒हेन्द्रि॒यम् । धे॒नुं गां न वयो॒ दध॑त् । वेत्वाज्य॑स्य॒ होत॒र्यज॑ ॥ होता॑ यक्षथ् सु॒रेत॑सम् । त्वष्टा॑रं पुष्टि॒वर्द्ध॑नम् । रू॒पाणि॒ बिभ्र॑तं॒ पृथ॑क् । पुष्टि॒मिन्द्रं॑ ॅवयो॒धस᳚म् \textbf{ 82} \newline
                  \newline
                                \textbf{ TB 2.6.17.7} \newline
                  द्वि॒पदं॒ छन्द॑ इ॒हेन्द्रि॒यम् । उ॒क्षाणं॒ गां न वयो॒ दध॑त् । वेत्वाज्य॑स्य॒ होत॒र्यज॑ ॥ होता॑ यक्षच्छ॒तक्र॑तुम् । हिर॑ण्यपर्ण-मु॒क्थिन᳚म् । र॒श॒नां बिभ्र॑तं ॅव॒शिम् । भग॒मिन्द्रं॑ ॅवयो॒धस᳚म् । क॒कुभं॒ छन्द॑ इ॒हेन्द्रि॒यम् । व॒शां ॅवे॒हतं॒ गां न वयो॒ दध॑त् । वेत्वाज्य॑स्य॒ होत॒र्यज॑ ( ) ॥ होता॑ यक्ष॒थ्-स्वाहा॑कृतीः । अ॒ग्निं गृ॒हप॑तिं॒ पृथ॑क् । वरु॑णं भेष॒जं क॒विम् । क्ष॒त्रमिन्द्रं॑ ॅवयो॒धस᳚म् । अति॑च्छन्दसं॒ छन्द॑ इन्द्रि॒यम् । बृ॒हदृ॑ष॒भं गां ॅवयो॒ दध॑त् । वेत्वाज्य॑स्य॒ होत॒र्यज॑ । \textbf{ 83} \newline
                  \newline
                                                        \textbf{special korvai} \newline
              इ॒डस्प॒दे᳚ऽग्निं गा॑य॒त्रीं त्र्यवि᳚म् । शुचि॑व्रतꣳ॒॒ शुचि॑मु॒ष्णिहं॑ दित्य॒वाह᳚म् । ई॒डेन्यꣳ॒॒ सोम॑-मन॒ष्टुभं॑ त्रिव॒थ्सम् । सु॒ब॒र॒.हि॒षद॑म॒मृतेन्द्रं॑ बृह॒तीं पञ्चा॑विम् । व्यच॑स्वतीः सुप्राय॒णा द्वारो᳚ ब्रा॒ह्माणः॑ पा॒ङ्क्तिमि॒ह तु॑र्य॒वाह᳚म् । सु॒पेश॑से॒ विश्व॒मिन्द्रं॑ त्रि॒ष्टुभं॑ पष्ठ॒वाह᳚म् । प्रचे॑तसा स॒युजेन्द्रं॒ जग॑तीमि॒हान॒ड्वाह᳚म् । पेश॑स्वतीस्ति॒स्त्रो भार॑तीः॒ पतिं॑ ॅवि॒राज॑मि॒ह धे॒नुं न । सु॒रेत॑सं॒ त्वष्टा॑रं॒ पुष्टिं॑ द्वि॒पद॑मि॒होक्षाणं॒ न । श॒तक्र॑तुं॒ भग॒मिन्द्रं॑ क॒कुभ॑मि॒ह व॒शां ॅवे॒हतं॒ गां न । स्वाहा॑कृतीः क्ष॒त्रमति॑च्छन्दसं बृ॒हदृ॑ष॒भं गां ॅवयः॑ । इ॒न्द्रि॒यमुषि॑वसु॒नव॑द॒शेहे᳚न्द्रिय॒मष्ट॑ नव दश॒ गां न वयो॒ दध॒थ् सर्व॑वेतु । \newline
                                (द॒धे - दध॑द् - ऋता॒वृध॑ - इन्द्रि॒यं - पेश॑स्वतीर् - वयो॒धसं॒ - ॅवेत्वाज्य॑स्य॒ होत॒र्यज॑ स॒प्त च॑) \textbf{(A17)} \newline \newline
                \textbf{ 2.6.18    अनुवाकं   18 - ऐन्द्रो पशौ प्रयाजयाज्याः} \newline
                                \textbf{ TB 2.6.18.1} \newline
                  समि॑द्धो अ॒ग्निः स॒मिधा᳚ । सुष॑मिद्धो॒ वरे᳚ण्यः । गा॒य॒त्री छन्द॑ इन्द्रि॒यम् । त्र्यवि॒र्गौर्वयो॑ दधुः ॥ तनू॒नपा॒-च्छुचि॑व्रतः । त॒नू॒नपाच्च॒ (त॒नू॒पाच्च) सर॑स्वती । उ॒ष्णिक्छन्द॑ इन्द्रि॒यम् । दि॒त्य॒वाड्गौर्वयो॑ दधुः ॥ इडा॑भिर॒ग्निरीड्यः॑ । सोमो॑ दे॒वो अम॑र्त्यः \textbf{ 84} \newline
                  \newline
                                \textbf{ TB 2.6.18.2} \newline
                  अ॒नु॒ष्टुप्छन्द॑ इन्द्रि॒यम् । त्रि॒व॒थ्सो गौर्वयो॑ दधुः ॥ सु॒ब॒र्.॒हिर॒ग्निः पू॑ष॒ण्वान् । स्ती॒र्ण-ब॑र्.हि॒रम॑र्त्यः । बृ॒ह॒ती छन्द॑ इन्द्रि॒यम् । पञ्चा॑वि॒र् गौर्वयो॑ दधुः ॥ दुरो॑ दे॒वीर्दिशो॑ म॒हीः । ब्र॒ह्मा दे॒वो बृह॒स्पतिः॑ । प॒ङ्क्तिः छन्द॑ इ॒हेन्द्रि॒यम् । तु॒र्य॒वाड् गौर्वयो॑ दधुः । \textbf{ 85} \newline
                  \newline
                                \textbf{ TB 2.6.18.3} \newline
                  उ॒षे य॒ह्वी सु॒पेश॑सा । विश्वे॑ दे॒वा अम॑र्त्याः । त्रि॒ष्टुप् छन्द॑ इन्द्रि॒यम् । प॒ष्ठ॒वाद्गौर्वयो॑ दधुः ॥ दैव्या॑ होतारा भिषजा । इन्द्रे॑ण स॒युजा॑ यु॒जा । जग॑ती॒ छन्द॑ इ॒हेन्द्रि॒यम् । अ॒न॒ड्वान्-गौर्वयो॑ दधुः ॥ ति॒स्र इडा॒ सर॑स्वती । भार॑ती म॒रुतो॒ विशः॑ \textbf{ 86} \newline
                  \newline
                                \textbf{ TB 2.6.18.4} \newline
                  वि॒राट्छन्द॑ इ॒हेन्द्रि॒यम् । धे॒नुर्गौर्न वयो॑ दधुः ॥ त्वष्टा॑ तु॒रीपो॒ अद्भु॑तः । इ॒न्द्रा॒ग्नी पु॑ष्टि॒वर्द्ध॑ना । द्वि॒पाच्छन्द॑ इ॒हेन्द्रि॒यम् । उ॒क्षा गौर्न वयो॑ दधुः ॥ श॒मि॒ता नो॒ वन॒स्पतिः॑ । स॒वि॒ता प्र॑सु॒वन्भग᳚म् । क॒कुच्छन्द॑ इ॒हेन्द्रि॒यम् । व॒शा वे॒हद्गौर्न वयो॑ दधुः ( ) ॥ स्वाहा॑ य॒ज्ञ्ं ॅवरु॑णः । सु॒क्ष॒त्रो भे॑ष॒जं क॑रत् । अतिः॑ छन्दाः॒ छन्द॑ इन्द्रि॒यम् । बृ॒हदृ॑ष॒भो गौर्वयो॑ दधुः । \textbf{ 87} \newline
                  \newline
                                    (अम॑र्त्य - स्तुर्य॒वाड्गौर्वयो॑ दधु॒र् - विशो॑ - व॒शा वे॒हद्गौर्न वयो॑ दधुश्च॒त्वारि॑ च) \textbf{(A18)} \newline \newline
                \textbf{ 2.6.19    अनुवाकं   19 - ऐन्द्रो पशौ वपादीनां यज्यानुवाक्याः} \newline
                                \textbf{ TB 2.6.19.1} \newline
                  व॒स॒न्तेन॒र्तुना॑ दे॒वाः । वस॑वस्त्रि॒वृता᳚ स्तु॒तम् । र॒थ॒न्त॒रेण॒ तेज॑सा । ह॒विरिन्द्रे॒ वयो॑ दधुः ॥ ग्री॒ष्मेण॑ दे॒वा ऋ॒तुना᳚ । रु॒द्राः प॑ञ्चद॒शे स्तु॒तम् । बृ॒ह॒ता यश॑सा॒ बल᳚म् । ह॒विरिन्द्रे॒ वयो॑ दधुः ॥ व॒र॒.षाभि॑र्. ऋ॒तुना॑ऽऽदि॒त्याः । स्तोमे॑ सप्तद॒शे स्तु॒तम् \textbf{ 88} \newline
                  \newline
                                \textbf{ TB 2.6.19.2} \newline
                  वै॒रू॒पेण॑ वि॒शौज॑सा । ह॒विरिन्द्रे॒ वयो॑ दधुः ॥ शा॒र॒देन॒र्तुना॑ दे॒वाः । ए॒क॒विꣳ॒॒श ऋ॒भवः॑ स्तु॒तम् । वै॒रा॒जेन॑ श्रि॒या श्रिय᳚म् । ह॒विरिन्द्रे॒ वयो॑ दधुः ॥ हे॒म॒न्तेन॒र्तुना॑ दे॒वाः । म॒रुत॑स्त्रिण॒वे स्तु॒तम् । बले॑न॒ शक्व॑रीः॒ सहः॑ । ह॒विरिन्द्रे॒ वयो॑ दधुः ( ) ॥ शै॒शि॒रेण॒र्तुना॑ दे॒वाः । त्र॒य॒स्त्रिꣳ॒॒शे॑ ऽमृतꣳ॑ स्तु॒तम् । स॒त्येन॑ रे॒वतीः᳚ क्ष॒त्रम् । ह॒विरिन्द्रे॒ वयो॑ दधुः । \textbf{ 89} \newline
                  \newline
                                                        \textbf{special korvai} \newline
              (व॒स॒न्तेन॑ ग्री॒ष्मेण॑ व॒र्॒.षाभिः॑ शार॒देन॑ हेम॒न्तेन॑ शैशि॒रेण॒ षट् ) \newline
                                (स्तोमे॑ सप्तद॒शे स्तु॒तꣳ - सहो॑ ह॒विरिन्द्रे॒ वयो॑ दधुश्च॒त्वारि॑ च) \textbf{(A19)} \newline \newline
                \textbf{ 2.6.20    अनुवाकं   20 - ऐन्द्रो पशौ अनूयाजानां मैत्रावरिणप्रैषाः} \newline
                                \textbf{ TB 2.6.20.1} \newline
                  दे॒वं ब॒र्॒.हिरिन्द्रं॑ ॅवयो॒धस᳚म् । दे॒वं दे॒वम॑वर्द्धयत् । गा॒य॒त्रि॒या छन्द॑सेन्द्रि॒यम् । तेज॒ इन्द्रे॒ वयो॒ दध॑त् । व॒सु॒वने॑ वसु॒धेय॑स्य वेतु॒ यज॑ ॥ दे॒वीर्द्वारो॑ दे॒वमिन्द्रं॑ ॅवयो॒धस᳚म् । दे॒वीर्दे॒व-म॑वर्द्धयन्न् । उ॒ष्णिहा॒ छन्द॑सेन्द्रि॒यम् । प्रा॒णमिन्द्रे॒ वयो॒ दध॑त् । व॒सु॒वने॑ वसु॒धेय॑स्य वियन्तु॒ यज॑ । \textbf{ 90} \newline
                  \newline
                                \textbf{ TB 2.6.20.2} \newline
                  दे॒वी दे॒वं ॅव॑यो॒धस᳚म् । उ॒षे इन्द्र॑मवर्द्धताम् । अ॒नु॒ष्टुभा॒ छन्द॑सेन्द्रि॒यम् । वाच॒मिन्द्रे॒ वयो॒ दध॑त् । व॒सु॒वने॑ वसु॒धेय॑स्य वीतां॒ ॅयज॑ ॥ दे॒वी जोष्ट्री॑ दे॒वमिन्द्रं॑ ॅवयो॒धस᳚म् । दे॒वी दे॒वम॑वर्द्धताम् । बृ॒ह॒त्या छन्द॑सेन्द्रि॒यम् । श्रोत्र॒मिन्द्रे॒ वयो॒ दध॑त् । व॒सु॒वने॑ वसु॒धेय॑स्य वीतां॒ ॅयज॑ । \textbf{ 91} \newline
                  \newline
                                \textbf{ TB 2.6.20.3} \newline
                  दे॒वी ऊ॒र्जाहु॑ती दे॒वमिन्द्रं॑ ॅवयो॒धस᳚म् । दे॒वी दे॒वम॑वर्द्धताम् । प॒ङ्क्त्या छन्द॑सेन्द्रि॒यम् । शु॒क्रमिन्द्रे॒ वयो॒ दध॑त् । व॒सु॒वने॑ वसु॒धेय॑स्य वीतां॒ ॅयज॑ ॥ दे॒वा दैव्या॒ होता॑रा दे॒वमिन्द्रं॑ ॅवयो॒धस᳚म् । दे॒वा दे॒वम॑वर्द्धताम् । त्रि॒ष्टुभा॒ छन्द॑सेन्द्रि॒यम् । त्विषि॒मिन्द्रे॒ वयो॒ दध॑त् । व॒सु॒वने॑ वसु॒धेय॑स्य वीतां॒ ॅयज॑ । \textbf{ 92} \newline
                  \newline
                                \textbf{ TB 2.6.20.4} \newline
                  दे॒वी-स्ति॒स्र-स्ति॒स्रो दे॒वीर्व॑यो॒धस᳚म् । पति॒मिन्द्र॑-मवर्द्धयन्न् । जग॑त्या॒ छन्द॑सेन्द्रि॒यम् । बल॒मिन्द्रे॒ वयो॒ दध॑त् । व॒सु॒वने॑ वसु॒धेय॑स्य वियन्तु॒ यज॑ ॥ दे॒वो नरा॒शꣳसो॑ दे॒वमिन्द्रं॑ ॅवयो॒धस᳚म् । दे॒वो दे॒वम॑वर्द्धयत् । वि॒राजा॒ छन्द॑सेन्द्रि॒यम् । रेत॒ इन्द्रे॒ वयो॒ दध॑त् । व॒सु॒वने॑ वसु॒धेय॑स्य वेतु॒ यज॑ । \textbf{ 93} \newline
                  \newline
                                \textbf{ TB 2.6.20.5} \newline
                  दे॒वो वन॒स्पति॑र्-दे॒वमिन्द्रं॑ ॅवयो॒धस᳚म् । दे॒वो दे॒वम॑वर्द्धयत् । द्वि॒पदा॒ छन्द॑सेन्द्रि॒यम् । भग॒मिन्द्रे॒ वयो॒ दध॑त् । व॒सु॒वने॑ वसु॒धेय॑स्य वेतु॒ यज॑ ॥ दे॒वं ब॒र्॒.हिर्वारि॑तीनां दे॒वमिन्द्रं॑ ॅवयो॒धस᳚म् । दे॒वो (दे॒वं) दे॒वम॑वर्द्धयत् । क॒कुभा॒ छन्द॑सेन्द्रि॒यम् । यश॒ इन्द्रे॒ वयो॒ दध॑त् । व॒सु॒वने॑ वसु॒धेय॑स्य वेतु॒ यज॑ ( ) ॥ दे॒वो अ॒ग्निः स्वि॑ष्ट॒कृद्-दे॒वमिन्द्रं॑ ॅवयो॒धस᳚म् । दे॒वो दे॒वम॑वर्द्धयत् । अति॑च्छन्दसा॒ छन्द॑सेन्द्रि॒यम् । क्ष॒त्रमिन्द्रे॒ वयो॒ दध॑त् । व॒सु॒वने॑ वसु॒धेय॑स्य वेतु॒ यज॑ । \textbf{ 94} \newline
                  \newline
                                                        \textbf{special korvai} \newline
              (दे॒वं ब॒र्॒.हिर् गा॑यत्रि॒या तेजः॑ । दे॒वीर्द्वार॑ उ॒ष्णिहा᳚ प्रा॒णम् । दे॒वी दे॒वमु॒षे अ॑नु॒ष्टुभा॒ वाच᳚म् । दे॒वी जोष्ट्री॑ बृह॒त्या श्रोत्र᳚म् । दे॒वी ऊ॒र्जाहू॑ती प॒ङ्क्त्या शु॒क्रम् । दे॒वा दैव्या॒ होता॑रा त्रि॒ष्टुभा॒ त्विषि᳚म् । दे॒वीस्ति॒स्रस्ति॒स्रो दे॒वीः पतिं॒ जग॑त्या॒ बल᳚म् । दे॒वो नरा॒शꣳसो॑ वि॒राजा॒ रेतः॑ । दे॒वो वन॒स्पति॑र् द्वि॒पदा॒ भग᳚म् । दे॒वं ब॒र्॒.हिर् वारि॑तीनां क॒कुभा॒ यशः॑ । दे॒वो अ॒ग्निः स्वि॑ष्ट॒कृदति॑च्छन्दसा क्ष॒त्रम् । वे॒तु॒ वि॒य॒न्तु॒ च॒तुर् वी॑ता॒मेको॑ वियन्तु च॒तुर्वे॑तु । अ॒व॒र्द्ध॒य॒द॒व॒र्द्ध॒यꣳ॒॒ श्च॒तुर॑वर्द्धता॒मेको॑ ऽवर्द्धयꣳ श्च॒तुर॑वर्द्धयत् ।) \newline
                                (वि॒य॒न्तु॒ यज॑ - वीतां॒ ॅयज॑-वीतां॒ ॅयज॑-वेतु॒ यज॑-वेतु॒ यज॒ पञ्च॑ च) \textbf{(A20)} \newline \newline
                \textbf{PrapAtaka Korvai with starting  words of 1 to 20 anuvAkams :-} \newline
        (स्वा॒द्वीं त्वा॒ - सोमः॒ - सुरा॑वन्तꣳ॒॒ - सीसे॑न - मि॒त्रो॑ऽसि॒ - यद् दे॑वा॒ - होता॑ यक्षथ् स॒मिधेन्द्रꣳ॒॒ - समि॑द्ध॒ इन्द्र॒ - आच॑र्.षणि॒प्रा - दे॒वं ब॒र्॒.हिरिन्द्रꣳ॑ सुदे॒वꣳ - होता॑ यक्षथ् स॒मिधा॒ऽग्निꣳ - समि॑द्धो अ॒ग्निर॑श्विना॒ - ऽश्विना॑ ह॒विरि॑न्द्रि॒यं - दे॒वं ब॒र्॒.हिः सर॑स्व - त्या॒ग्निम॒ - द्योशन्तो॒ - होता॑ यक्षदि॒डस्प॒दे - समि॑द्धो अ॒ग्निः स॒मिधा॑ - वस॒न्तेन॑ - दे॒वं ब॒र्॒.हिरिन्द्रं॑ ॅवयो॒धसं॑ ॅविꣳश॒तिः) \newline

        \textbf{korvai with starting words of 1, 11, 21 series of daSinis :-} \newline
        (स्वा॒द्वीं त्वाऽ - मी॑ मदन्त पि॒तरः॒ - साम्रा᳚ज्याय - पू॒तं प॒वित्रे॑णे॒ वाज्य॑ - मु॒षासा॒नक्ता॒ - बद॑रै॒ - रधा॑ताम॒श्विना॑ - दे॒व इन्द्रो॒ वन॒स्पतिः॑ - पष्ठ॒वाहं॒ गां - दे॒वी दे॒वं ॅव॑यो॒धसं॒ चतु॑र्नवतिः ) \newline

        \textbf{first and last  word 2nd aShTakam 6th prapAtakam :-} \newline
        (स्वा॒द्वीन्त्वा॑ - वेतु॒ यज॑) \newline 

       

        ॥ हरिः॑ ॐ ॥
॥ कृष्ण यजुर्वेदीय तैत्तिरीय ब्राह्मणे द्वितीयाष्टके षष्ठः प्रपाठकः समाप्तः ॥
Appendix (of Expansions)
ट्.भ्.2.6.3.4 "अग्न॒ आयूꣳ॑षि पव॒से {3}","ऽग्ने॒ पव॑स्व{4}" 
अग्न॒ आयूꣳ॑षि पवस॒ आ सु॒वोर्ज॒मिषं॑ च नः । 
आ॒रे बा॑धस्व दु॒च्छुनां᳚ ॥ {3}

अग्ने॒ पव॑स्व॒ स्वपा॑ अ॒स्मे वर्चः॑ सु॒वीर्यं᳚ । 
दध॒त् पोषꣳ॑ र॒यिं मयि॑ ॥ {4}
(Appearing in T.S.1.3.14.8 )
ट्.भ्.2.6.3.4 "पव॑मानः॒ सुव॒र्जनः॑{5}" "पु॒नन्तु॑ मा देवज॒नाः{6}" 
पव॑मानः॒ सुव॒र्जनः॑ । प॒वित्रे॑ण॒ विच॑र्.षणिः । 
यः पोता॒ स पु॑नातु मा ॥ {5}

पु॒नन्तु॑ मा देवज॒नाः । पु॒नन्तु॒ मन॑ वो धि॒या । 
पु॒नन्तु॒ विश्व॑ आ॒यवः॑ ॥ {6} 
(Both appearing in T.B.1.4.8.1)


ट्.भ्.2.6.3.4 "जात॑वेदः प॒वित्र॑व॒द्{7}" "यत्ते॑ प॒वित्र॑म॒र्चिषि॑{8}"
जात॑वेदः प॒वित्र॑वत् । प॒वित्रे॑ण पुनाहि मा । 
शु॒क्रेण॑ देव॒ दीद्य॑त् । अग्ने॒ क्रत्वा॒ क्रतूꣳ॒॒रनु॑ ॥ {7}
यत्ते॑ प॒वित्र॑-म॒र्चिषि॑ । अग्ने॒ वित॑तमन्त॒रा । 
ब्रह्म॒ तेन॑ पुनीमहे ॥ {8}
(Both appearing in T.B.1.4.8.2)

ट्.भ्.2.6.3.4 "उ॒भाभ्यां᳚ देव सवितर्{9}" "वैश्वदे॒वी पु॑न॒ती{10}" 
उ॒भाभ्यां᳚ देव सवितः । प॒वित्रे॑ण स॒वेन॑ च । 
इ॒दं ब्रह्म॑ पुनीमहे ॥ {9}

वै॒श्व॒दे॒वी पु॑न॒ती दे॒व्यागा᳚त् । 
यस्यै॑ ब॒ह्वी-स्त॒नुवो॑ वी॒तपृ॑ष्ठाः । 
तया॒ मद॑न्तः सध॒माद्ये॑षु । 
व॒यꣳ स्या॑म॒ पत॑यो रयी॒णाम् ॥ {10} 
(Both appearing in T.B.1.4.8.2)
ट्.भ्.2.6.9.1 "आच॑र्.षणि॒प्रा{11}" 
आच॑र्.षणि॒प्रा वृ॑ष॒भो जना॑नाम् । 
राजा॑ कृष्टी॒नां पु॑रुहू॒त इन्द्रः॑ । 
स्तु॒तः श्र॑व॒स्यन्नव॒सोप॑म॒द्रिक् । 
यु॒क्त्वा हरी॒ वृष॒णा ऽऽया᳚ह्य॒र्वाङ् ॥ {11}
(Appearing in T.B.2.4.3.11)
ट्.भ्.2.6.9.1 "वि॒वेष॒ यन्मा᳚{12}" 
वि॒वेष॒ यन्मा॑ धि॒षणा॑ ज॒जान॒ स्तवै॑ 
पु॒रा पार्या॒दिन्द्र॒ मह्नः॑ । 
अꣳह॑सो॒ यत्र॑ पी॒पर॒द्यथा॑ नो ना॒वेव॒ 
यान्त॑ मु॒भये॑ हवन्ते ॥ {12}
(Appearing in T.S.1.6.12.3)
ट्.भ्.2.6.9.1 "तꣳ स॒ध्रीचीः᳚{13}" 
तꣳ स॒ध्रीची॑रू॒तयो॒ वृष्णि॑यानि । 
पौꣳस्या॑नि नि॒युतः॑ सश्चु॒रिन्द्र᳚म् । 
स॒मु॒द्रं न सिन्ध॑व उ॒क्थशु॑ष्माः । उ॒रु॒व्यच॑स॒गिंर॒ आवि॑शन्ति ॥ {13}
(Appearing in T.B.2.4.5.2)
ट्.भ्.2.6.16.1 "उ॒शन्त॑स्त्वा हवामह॒{14}" 
उ॒शन्त॑स्त्वा हवामह उ॒शन्तः॒ समि॑धीमहि । 
उ॒शन्नु॑श॒त आ व॑ह पि॒तॄन् ह॒विषे॒ अत्त॑वे ॥ {14}
(Appearing in T.S. 2.6.12.1)
ट्.भ्.2.6.16.1 "आ नो॑ अग्ने सुके॒तुना᳚{15}""त्वꣳ सो॑म म॒हे भगं॒{16}"
आ नो॑ अग्ने सुके॒तुना᳚ । र॒यिं ॅवि॒श्वायु॑पोषसम् । 
मा॒र्डी॒कं धे॑हि जी॒वसे᳚ ॥ {15}

त्वꣳ सो॑म म॒हे भग᳚म् । त्वं ॅयून॑ ऋताय॒ते । 
दक्षं॑ दधासि जी॒वसे᳚ ॥ {16}
(Both appearing in T.B.2.4.5.3 )
ट्.भ्.2.6.16.1 "त्वꣳ सो॑म॒ प्रचि॑कितो मनी॒षा{17}", 
"त्वया॒ हि नः॑ पि॒तरः॑ सोम॒ पूर्वे॒{18}" 
त्वꣳ सो॑म॒ प्रचि॑कितो मनी॒षा त्वꣳ रजि॑ष्ठ॒मनु॑ नेषि॒ पन्थां᳚ । 
तव॒ प्रणी॑ती पि॒तरो॑ न इन्दो दे॒वेषु॒ रत्न॑म 
भजन्त॒ धीराः᳚ ॥ {17}

त्वया॒ हि नः॑ पि॒तरः॑ सोम॒ पूर्वे॒ कर्मा॑णि च॒क्रुः प॑वमान॒ धीराः᳚ । व॒न्वन्नवा॑तः परि॒धीꣳ रपो᳚र्णु वी॒रेभि॒रश्वै᳚र्म॒घवा॑ भवा नः ॥ {18}
(Both appearing in T.S.2.6.12.1)
ट्.भ्.2.6.16.1 "त्वꣳ सो॑म पि॒तृभिः॑ सम्ॅविदा॒नः{19}" 
त्वꣳ सो॑म पि॒तृभिः॑ संॅविदा॒नोऽनु॒ द्यावा॑पृथि॒वी आ त॑तन्थ । तस्मै॑ त इन्दो ह॒विषा॑ विधेम व॒यꣳ स्या॑म॒ पत॑यो रयी॒णां ॥ {19}
(Appearing in T.S.2.6.12.2)
ट्.भ्.2.6.16.1 "बर्.हि॑षदः पितर॒{20}", "आऽहं पि॒तॄन्{21}", "उप॑हूताः पि॒तरो{22}" 
बर्.हि॑षदः पितर ऊ॒त्य॑र्वागि॒मा वो॑ ह॒व्या च॑कृमा जु॒षध्वं᳚ । 
त आ ग॒ताऽव॑सा॒ शं त॑मे॒नाथा॒स्मभ्यꣳ॒॒ शं ॅयोर॑र॒पो द॑धात ॥ {20}
(Appearing in T.S.2.6.12.2)

आऽहं पि॒तॄन्थ् सु॑वि॒दत्राꣳ॑ अविथ्सि॒ नपा॑तं च वि॒क्रम॑णं च॒ विष्णोः᳚ । 
ब॒र्.हि॒षदो॒ ये स्व॒धया॑ सु॒तस्य॒ भज॑न्त पि॒त्वस्त इ॒हाऽऽ ग॑मिष्ठाः ॥ {21}
(Appearing in T.S.2.6.12.3)
उप॑हूताः पि॒तरः॑ सो॒म्यासो॑ बर्.हि॒ष्ये॑षु नि॒धिषु॑ प्रि॒येषु॑ । त आ ग॑मन्तु॒ त इ॒ह श्रु॑व॒न्त्वधि॑ ब्रुवन्तु॒ ते अ॑वन्त्व॒स्मान् ॥ {22}
(Appearing in T.S.2.6.12.3)

ट्.भ्.2.6.16.1 "ऽग्नि॑ष्वात्ताः पितरः{23}" 
अग्नि॑ष्वात्ताः पितर॒ एह ग॑च्छत॒ सदः॑ सदः सदत सुप्रणीतयः । अ॒त्ता ह॒वीꣳषि॒ प्रय॑तानि ब॒र्॒.हिष्यथा॑ र॒यिꣳ सर्व॑वीरं दधातन ॥ {23}
(Appearing in T.S.2.6.12.2)
ट्.भ्.2.6.16.2 "यद॑ग्ने कव्यवाहन॒{24}, 
"त्वम॑ग्न ईडि॒तो जा॑तवेदः{25}", "मात॑लीक॒व्यैः{26}" 
यद॑ग्ने कव्यवाहन पि॒तॄन् यक्ष्यृ॑ता॒वृधः॑ । 
प्र च॑ ह॒व्यानि॑ वक्ष्यसि दे॒वेभ्य॑श्च पि॒तृभ्य॒ आ ॥ {24]
त्वम॑ग्न ईडि॒तो जा॑तवे॒दोऽवा᳚ड्ढ॒व्यानि॑ सुर॒भीणि॑ कृ॒त्वा । प्रादाः᳚ पि॒तृभ्यः॑ स्व॒धया॒ ते अ॑क्षन्न॒द्धि त्वं दे॑व॒ प्रय॑ता ह॒वीꣳषि॑ । {25}
मात॑ली क॒व्यैर्य॒मो अङ्गि॑रोभि॒ र्बृह॒स्पति॒र् ऋक्व॑भि र्वावृधा॒नः । याꣳश्च॑ दे॒वा वा॑वृ॒धुर्ये च॑ दे॒वान्थ् स्वाहा॒ऽन्ये स्व॒धया॒ऽन्ये म॑दन्ति ॥ {26}
(ThE abOvE 3 appearing in T.S.2.6.12.5)
============================ \newline
        \pagebreak
        
        
        
     \addcontentsline{toc}{section}{ 2.7     द्वितीयाष्टके सप्तमः प्रपाठकः - सवा एकाहविशेषाः}
     \markright{ 2.7     द्वितीयाष्टके सप्तमः प्रपाठकः - सवा एकाहविशेषाः \hfill https://www.vedavms.in \hfill}
     \section*{ 2.7     द्वितीयाष्टके सप्तमः प्रपाठकः - सवा एकाहविशेषाः }
                \textbf{ 2.7.1     अनुवाकं   1 - बृहस्पतिसवः} \newline
                                \textbf{ TB 2.7.1.1} \newline
                  त्रि॒वृथ्-स्तोमो॑ भवति । ब्र॒ह्म॒व॒र्च॒सं ॅवै त्रि॒वृत् । ब्र॒ह्म॒व॒र्च॒समे॒वाव॑रुन्धे ॥ अ॒ग्नि॒ष्टो॒मः सोमो॑ भवति । ब्र॒ह्म॒व॒र्च॒सं ॅवा अ॑ग्निष्टो॒मः । ब्र॒ह्म॒व॒र्च॒समे॒वाव॑रुन्धे ॥ र॒थ॒तंरꣳ॒ साम॑ भवति । ब्र॒ह्म॒व॒र्च॒सं ॅवै र॑थन्त॒रम् । ब्र॒ह्म॒व॒र्च॒समे॒वाव॑रुन्धे ॥ प॒रि॒स्र॒जी होता॑ भवति \textbf{ 1} \newline
                  \newline
                                \textbf{ TB 2.7.1.2} \newline
                  अ॒रु॒णो मि॑र्मि॒रस्त्रिशु॑क्रः । ए॒तद् वै ब्र॑ह्मवर्च॒सस्य॑ रू॒पम् । रू॒पेणै॒व ब्र॑ह्मवर्च॒समव॑रुन्धे ॥ बृह॒स्पति॑-रकामयत दे॒वानां᳚ पुरो॒धां ग॑च्छेय॒मिति॑ । स ए॒तं बृ॑हस्पतिस॒वम॑पश्यत् । तमाऽह॑रत् । तेना॑यजत । ततो॒ वै स दे॒वानां᳚ पुरो॒धा-म॑गच्छत् । यः पु॑रो॒धाका॑मः॒ स्यात् । स बृ॑हस्पतिस॒वेन॑ यजेत \textbf{ 2} \newline
                  \newline
                                \textbf{ TB 2.7.1.3} \newline
                  पु॒रो॒धामे॒व ग॑च्छति ॥ तस्य॑ प्रातस्सव॒ने स॒न्नेषु॑ नाराशꣳ॒॒सेषु॑ । एका॑दश॒ दक्षि॑णा नीयन्ते । एका॑दश॒ माद्ध्य॑दिंने॒ सव॑ने स॒न्नेषु॑ नाराशꣳ॒॒सेषु॑ । एका॑दश तृतीयसव॒ने स॒न्नेषु॑ नाराशꣳ॒॒सेषु॑ । त्रय॑स्त्रिꣳश॒थ्-संप॑द्यन्ते । त्रय॑स्त्रिꣳश॒द् वै दे॒वताः᳚ । दे॒वता॑ ए॒वाव॑रुन्धे ॥ अश्व॑श्चतुस्त्रिꣳ॒॒शः । प्रा॒जा॒प॒त्यो वा अश्वः॑ \textbf{ 3} \newline
                  \newline
                                \textbf{ TB 2.7.1.4} \newline
                  प्र॒जाप॑ति-श्चतुस्त्रिꣳ॒॒शो दे॒वता॑नां । याव॑तीरे॒व दे॒वताः᳚ । ता ए॒वाव॑रुन्धे ॥ कृ॒ष्णा॒जि॒ने॑ ऽभिषि॑ञ्चति । ब्रह्म॑णो॒ वा ए॒तद्-रू॒पम् । यत् कृ॑ष्णाजि॒नम् । ब्र॒ह्म॒व॒र्च॒सेनै॒वैनꣳ॒॒ सम॑र्द्धयति ॥ आज्ये॑ना॒भिषि॑ञ्चति । तेजो॒ वा आज्य᳚म् । तेज॑ ए॒वास्मि॑न् दधाति ( ) । \textbf{ 4} \newline
                  \newline
                                    (होता॑ भवति - यजेत॒ - वा अश्वो॑ - दधाति) \textbf{(A1)} \newline \newline
                \textbf{ 2.7.2      अनुवाकं   2 - वैश्यसवः} \newline
                                \textbf{ TB 2.7.2.1} \newline
                  यदा᳚ग्ने॒यो भव॑ति । अ॒ग्निमु॑खा॒ ह्यृद्धिः॑ ॥ अथ॒ यत् पौ॒ष्णः । पुष्टि॒र्वै पू॒षा । पुष्टि॒र्-वैश्य॑स्य । पुष्टि॑मे॒वाव॑रुन्धे ॥ प्र॒स॒वाय॑ सावि॒त्रः ॥ अथ॒ यत् त्वा॒ष्ट्रः । त्वष्टा॒ हि रू॒पाणि॑ विक॒रोति॑ ॥ नि॒र्व॒रु॒ण॒त्वाय॑ वारु॒णः \textbf{ 5} \newline
                  \newline
                                \textbf{ TB 2.7.2.2} \newline
                  अथो॒ य ए॒व कश्च॒ सन्थ्सू॒यते᳚ । स हि वा॑रु॒णः ॥ अथ॒ यद् वै᳚श्वदे॒वः । वै॒श्व॒दे॒वो हि वैश्यः॑ ॥ अथ॒ यन् मा॑रु॒तः । मा॒रु॒तो हि वैश्यः॑ ॥ स॒प्तैतानि॑ ह॒वीꣳषि॑ भवन्ति । स॒प्तग॑णा॒ वै म॒रुतः॑ ॥ पृश्निः॑ पष्ठौ॒ही मा॑रु॒त्याल॑भ्यते । विड्वै म॒रुतः॑ ( ) ॥ विश॑ ए॒वैतन्-म॑द्ध्य॒तो॑ऽभिषि॑च्यते । तस्मा॒द् वा ए॒ष वि॒शः प्रि॒यः । वि॒शो हि म॑द्ध्य॒तो॑ऽभिषि॒च्यते᳚ ॥ ऋ॒ष॒भ॒च॒र्मेऽद्ध्य॒भिषि॑ञ्चति । स हि प्र॑जनयि॒ता ॥ द॒द्ध्नाऽभिषि॑ञ्चति । ऊर्ग्वा अ॒न्नाद्यं॒ दधि॑ । ऊ॒र्जैवैन॑-म॒न्नाद्ये॑न॒ सम॑र्द्धयति । \textbf{ 6} \newline
                  \newline
                                    (वा॒रु॒णो - विड्वै म॒रुतो॒ऽष्टौ च॑) \textbf{(A2)} \newline \newline
                \textbf{ 2.7.3      अनुवाकं   3 - ब्राह्मणसवः} \newline
                                \textbf{ TB 2.7.3.1} \newline
                  यदा᳚ग्ने॒यो भव॑ति । आ॒ग्ने॒यो वै ब्रा᳚ह्म॒णः ॥ अथ॒ यथ् सौ॒म्यः । सौ॒म्यो हि ब्रा᳚ह्म॒णः ॥ प्र॒स॒वायै॒व सा॑वि॒त्रः ॥ अथ॒ यद् बा॑र्.हस्प॒त्यः । ए॒तद्वै ब्रा᳚ह्म॒णस्य॑ वाक्प॒तीय᳚म् ॥ अथ॒ यद॑ग्नीषो॒मीयः॑ । आ॒ग्ने॒यो वै ब्रा᳚ह्म॒णः । तौ य॒दा स॒गंच्छे॑ते \textbf{ 7} \newline
                  \newline
                                \textbf{ TB 2.7.3.2} \newline
                  अथ॑ वी॒र्या॑वत्तरो भवति ॥ अथ॒ यथ् सा॑रस्व॒तः । ए॒तद्धि प्र॒त्यक्षं॑ ब्राह्म॒णस्य॑ वाक्प॒तीय᳚म् ॥ नि॒र्व॒रु॒ण॒त्वायै॒व वा॑रु॒णः । अथो॒ य ए॒व कश्च॒ सन्थ्सू॒यते᳚ । स हि वा॑रु॒णः ॥ अथ॒ यद् द्या॑वापृथि॒व्यः॑ । इन्द्रो॑ वृ॒त्राय॒ वज्र॒मुद॑यच्छत् । तं द्यावा॑पृथि॒वी नान्व॑मन्येताम् । तमे॒तेनै॒व भा॑ग॒धेये॒ना-न्व॑मन्येताम् \textbf{ 8} \newline
                  \newline
                                \textbf{ TB 2.7.3.3} \newline
                  वज्र॑स्य॒ वा ए॒षो॑ऽनुमा॒नाय॑ । अनु॑मतवज्रः सूयाता॒ इति॑ ॥ अ॒ष्टावे॒तानि॑ ह॒वीꣳषि॑ भवन्ति । अ॒ष्टाक्ष॑रा गाय॒त्री । गा॒य॒त्री ब्र॑ह्मवर्च॒सम् । गा॒य॒त्रि॒यैव ब्र॑ह्मवर्च॒समव॑रुन्धे ॥ हिर॑ण्येन घृ॒तमुत् पु॑नाति । तेज॑स ए॒व रु॒चे ॥ कृ॒ष्णा॒जि॒ने॑ऽभिषि॑ञ्चति । ब्रह्म॑णो॒ वा ए॒तदृ॑ख्सा॒मयो॑ रू॒पम् ( ) । यत् कृ॑ष्णाजि॒नम् । ब्रह्म॑न्ने॒वैन॑-मृख्सा॒मयो॒-रद्ध्य॒भिषि॑ञ्चति ॥ घृ॒तेना॒-भिषि॑ञ्चति । तथा॑ वी॒र्या॑वत्तरो भवति । \textbf{ 9} \newline
                  \newline
                                    (स॒गंच्छे॑ते - भाग॒धेये॒नान्व॑मन्येताꣳ- रू॒पं च॒त्वारि॑ च) \textbf{(A3)} \newline \newline
                \textbf{ 2.7.4      अनुवाकं   4 - सोमसवः} \newline
                                \textbf{ TB 2.7.4.1} \newline
                  न वै सोमे॑न॒ सोम॑स्य स॒वो᳚ऽस्ति । ह॒तो ह्ये॑षः । अ॒भिषु॑तो॒ ह्ये॑षः । न हि ह॒तः सू॒यते᳚ । सौ॒मीꣳ सू॒तव॑शा॒माल॑भते । सोमो॒ वै रे॑तो॒धाः । रेत॑ ए॒व तद्-द॑धाति ॥ सौ॒म्यर्चा ऽभिषि॑ञ्चति । रे॒तो॒धा ह्य॑षा । रेतः॒ सोमः॑ ( ) । रेत॑ ए॒वास्मि॑न् दधाति ॥ यत् किञ्च राज॒सूय॑मृ॒ते सोम᳚म् । तथ् सर्वं॑ भवति ॥ अषा॑ढं ॅयु॒थ्सु पृत॑नासु॒ पप्रि᳚म् । सु॒व॒र॒.षाम॒फ्स्वां ॅवृ॒जन॑स्य गो॒पाम् । भ॒रे॒षु॒जाꣳ सु॑क्षि॒तिꣳ सु॒श्रव॑सम् । जय॑न्तं॒ त्वामनु॑ मदेम सोम । \textbf{ 10} \newline
                  \newline
                                    (रेतः॒ सोमः॑ स॒प्त च॑ ) \textbf{(A4)} \newline \newline
                \textbf{ 2.7.5      अनुवाकं   5 - पृथिसवः} \newline
                                \textbf{ TB 2.7.5.1} \newline
                  यो वै सोमे॑न सू॒यते᳚ । स दे॑वस॒वः । यः प॒शुना॑ सू॒यते᳚ । स दे॑वस॒वः । य इष्ट्या॑ सू॒यते᳚ । स म॑नुष्यस॒वः । ए॒तं ॅवै पृथ॑ये दे॒वाः प्राय॑च्छन्न् । ततो॒ वै सोऽप्या॑र॒ण्यानां᳚ पशू॒नाम॑सूयत । याव॑तीः॒ किय॑तीश्च प्र॒जा वाचं॒ ॅवद॑न्ति । तासाꣳ॒॒ सर्वा॑साꣳ सूयते \textbf{ 11} \newline
                  \newline
                                \textbf{ TB 2.7.5.2} \newline
                  य ए॒तेन॒ यज॑ते । य उ॑ चैनमे॒वं ॅवेद॑ ॥ ना॒रा॒शꣳ॒॒स्यर्चा ऽभिषि॑ञ्चति । म॒नु॒ष्या॑ वै नरा॒शꣳसः॑ । नि॒ह्नुत्य॒ वावैतत् । अथा॒भिषि॑ञ्चति ॥ यत् किञ्च॑ राज॒सूय॑-मनुत्तरवे॒दीक᳚म् । तथ् सर्वं॑ भवति ॥ ये मे॑ पञ्चा॒शतं॑ द॒दुः । अश्वा॑नाꣳ स॒धस्तु॑तिः ( ) । द्यु॒मद॑ग्ने॒ महि॒ श्रवः॑ । बृ॒हत् कृ॑धि म॒घोना᳚म् । नृ॒वद॑मृत नृ॒णाम् । \textbf{ 12} \newline
                  \newline
                                    (सू॒य॒ते॒ - स॒धस्तु॑ति॒स्त्रीणि॑ च) \textbf{(A5)} \newline \newline
                \textbf{ 2.7.6      अनुवाकं   6 - गोसवः} \newline
                                \textbf{ TB 2.7.6.1} \newline
                  ए॒ष गो॑स॒वः । ष॒ट्त्रिꣳ॒॒श उ॒क्थ्यो॑ बृ॒हथ् सा॑मा । पव॑माने कण्वरथन्त॒रं भ॑वति ॥ यो वै वा॑ज॒पेयः॑ । स स॑म्राट्थ्स॒वः । यो रा॑ज॒सूयः॑ । स व॑रुणस॒वः । प्र॒जाप॑तिः॒ स्वारा᳚ज्यं परमे॒ष्ठी । स्वारा᳚ज्यं॒ गौरे॒व ॥ गौरि॑व भवति \textbf{ 13} \newline
                  \newline
                                \textbf{ TB 2.7.6.2} \newline
                  य ए॒तेन॒ यज॑ते । य उ॑ चैनमे॒वं ॅवेद॑ ॥ उ॒भे बृ॑हद्-रथन्त॒रे भ॑वतः । तद्धि स्वारा᳚ज्यम् ॥ अ॒युतं॒ दक्षि॑णाः । तद्धि स्वारा᳚ज्यम् ॥ प्र॒ति॒धुषा॒ ऽभिषि॑ञ्चति । तद्धि स्वारा᳚ज्यम् ॥ अनु॑द्धते॒ वेद्यै॑ दक्षिण॒त आ॑हव॒नीय॑स्य बृह॒तः स्तो॒त्रं प्रत्य॒भिषि॑ञ्चति । इ॒यं ॅवाव र॑थन्त॒रम् \textbf{ 14} \newline
                  \newline
                                \textbf{ TB 2.7.6.3} \newline
                  अ॒सौ बृ॒हत् । अ॒नयो॑रे॒वैन॒-मन॑न्तर्.हित-म॒भिषि॑ञ्चति ॥ प॒शु॒स्तो॒मो वा ए॒षः । तेन॑ गोस॒वः । ष॒ट्त्रिꣳ॒॒शः सर्वः॑ ॥ रे॒वज्जा॒तः सह॑सा वृ॒द्धः । क्ष॒त्राणां᳚ क्षत्र॒भृत्त॑मो वयो॒धाः । म॒हान् म॑हि॒त्वे त॑स्तभा॒नः । क्ष॒त्रे रा॒ष्ट्रे च॑ जागृहि । प्र॒जाप॑तेस्त्वा परमे॒ष्ठिनः॒ स्वारा᳚ज्येना॒भिषि॑ञ्चा॒-मीत्या॑ह ( ) ॥ स्वारा᳚ज्य-मे॒वैनं॑ गमयति । \textbf{ 15} \newline
                  \newline
                                    (इ॒व॒ भ॒व॒ति॒ - र॒थ॒न्त॒र - मा॒हैकं॑ च) \textbf{(A6)} \newline \newline
                \textbf{ 2.7.7      अनुवाकं   7 - ओदनसवमन्त्राः} \newline
                                \textbf{ TB 2.7.7.1} \newline
                  सिꣳ॒॒हे व्या॒घ्र उ॒त या पृदा॑कौ । त्विषि॑र॒ग्नौ ब्रा᳚ह्म॒णे सूर्ये॒ या । इन्द्रं॒ ॅया दे॒वी सु॒भगा॑ ज॒जान॑ । सा न॒ आग॒न् वर्च॑सा सम्ॅविदा॒ना ॥ या रा॑ज॒न्ये॑ दुन्दु॒भावाय॑तायाम् । अश्व॑स्य॒ क्रन्द्ये॒ पुरु॑षस्य मा॒यौ । इन्द्रं॒ ॅया दे॒वी सु॒भगा॑ ज॒जान॑ । सा न॒ आग॒न् वर्च॑सा सम्ॅविदा॒ना ॥ या ह॒स्तिनि॑ द्वी॒पिनि॒ या हिर॑ण्ये । त्विषि॒-रश्वे॑षु॒ पुरु॑षेषु॒ गोषु॑ \textbf{ 16} \newline
                  \newline
                                \textbf{ TB 2.7.7.2} \newline
                  इन्द्रं॒ ॅया दे॒वी सु॒भगा॑ ज॒जान॑ । सा न॒ आग॒न् वर्च॑सा सम्ॅविदा॒ना ॥ रथे॑ अ॒क्षेषु॑ वृष॒भस्य॒ वाजे᳚ । वाते॑ प॒र्जन्ये॒ वरु॑णस्य॒ शुष्मे᳚ । इन्द्रं॒ ॅया दे॒वी सु॒भगा॑ ज॒जान॑ । सा न॒ आग॒न् वर्च॑सा सम्ॅविदा॒ना ॥ राड॑सि वि॒राड॑सि । स॒म्राड॑सि स्व॒राड॑सि ॥ इन्द्रा॑य त्वा॒ तेज॑स्वते॒ तेज॑स्वन्तꣳ श्रीणामि । इन्द्रा॑य॒ त्वौज॑स्वत॒ ओज॑सवन्तꣳ श्रीणामि \textbf{ 17} \newline
                  \newline
                                \textbf{ TB 2.7.7.3} \newline
                  इन्द्रा॑य त्वा॒ पय॑स्वते॒ पय॑स्वन्तꣳ श्रीणामि । इन्द्रा॑य॒ त्वाऽऽयु॑ष्मत॒ आयु॑ष्मन्तꣳ श्रीणामि ॥ तेजो॑ऽसि । तत् ते॒ प्रय॑च्छामि । तेज॑स्वदस्तु मे॒ मुख᳚म् । तेज॑स्व॒च्छिरो॑ अस्तु मे । तेज॑स्वान्. वि॒श्वतः॑ प्र॒त्यङ् । तेज॑सा॒ संपि॑पृग्धि मा ॥ ओजो॑ऽसि । तत् ते॒ प्रय॑च्छामि \textbf{ 18} \newline
                  \newline
                                \textbf{ TB 2.7.7.4} \newline
                  ओज॑स्वदस्तु मे॒ मुख᳚म् । ओज॑स्व॒च्छिरो॑ अस्तु मे । ओज॑स्वान्. वि॒श्वतः॑ प्र॒त्यङ् । ओज॑सा॒ संपि॑पृग्धि मा ॥ पयो॑ऽसि । तत्ते॒ प्रय॑च्छामि । पय॑स्वदस्तु मे॒ मुख᳚म् । पय॑स्व॒च्छिरो॑ अस्तु मे । पय॑स्वान्. वि॒श्वतः॑ प्र॒त्यङ् । पय॑सा॒ संपि॑पृग्धि मा । \textbf{ 19} \newline
                  \newline
                                \textbf{ TB 2.7.7.5} \newline
                  आयु॑रसि । तत् ते॒ प्रय॑च्छामि । आयु॑ष्मदस्तु मे॒ मुख᳚म् । आयु॑ष्म॒च्छिरो॑ अस्तु मे । आयु॑ष्मान्. वि॒श्वतः॑ प्र॒त्यङ् । आयु॑षा॒ संपि॑पृग्धि मा ॥ इ॒मम॑ग्न॒ आयु॑षे॒ वर्च॑से कृधि । प्रि॒यꣳ रेतो॑ वरुण सोम राजन्न् । मा॒तेवा᳚स्मा अदिते॒ शर्म॑ यच्छ । विश्वे॑ देवा॒ जर॑दष्टि॒र्यथा ऽस॑त् । \textbf{ 20} \newline
                  \newline
                                \textbf{ TB 2.7.7.6} \newline
                  आयु॑रसि वि॒श्वायु॑रसि । स॒र्वायु॑रसि॒ सर्व॒मायु॑रसि ॥ यतो॒ वातो॒ मनो॑ जवाः । यतः॒ क्षर॑न्ति॒ सिन्ध॑वः । तासां᳚ त्वा॒ सर्वा॑साꣳ रु॒चा । अ॒भिषि॑ञ्चामि॒ वर्च॑सा ॥ स॒मु॒द्र इ॑वासि ग॒ह्मना᳚ । सोम॑ इवा॒स्यदा᳚भ्यः । अ॒ग्निरि॑व वि॒श्वतः॑ प्र॒त्यङ् । सूर्य॑ इव॒ ज्योति॑षा वि॒भूः । \textbf{ 21} \newline
                  \newline
                                \textbf{ TB 2.7.7.7} \newline
                  अ॒पां ॅयो द्रव॑णे॒ रसः॑ । तम॒हम॒स्मा आ॑मुष्याय॒णाय॑ । तेज॑से ब्रह्मवर्च॒साय॑ गृह्णामि । अ॒पां ॅय ऊ॒र्मौ रसः॑ । तम॒हम॒स्मा आ॑मुष्याय॒णाय॑ । ओज॑से वी॒र्या॑य गृह्णामि । अ॒पां ॅयो म॑द्ध्य॒तो रसः॑ । तम॒हम॒स्मा आ॑मुष्याय॒णाय॑ । पुष्ट्यै᳚ प्र॒जन॑नाय गृह्णामि । अ॒पां ॅयो य॒ज्ञियो॒ रसः॑ ( ) । तम॒हम॒स्मा आ॑मुष्याय॒णाय॑ । आयु॑षे दीर्घायु॒त्वाय॑ गृह्णामि । \textbf{ 22} \newline
                  \newline
                                    (गोष् - वोज॑स्वन्तꣳ श्रीणा॒ - म्योजो॑ऽसि॒ तत्ते॒ प्रय॑च्छामि॒ - पय॑सा॒ संपि॑पृग्धि॒ मा-स॑द् - वि॒भूर् - य॒ज्ञियो॒ रसो॒ द्वे च॑) \textbf{(A7)} \newline \newline
                \textbf{ 2.7.8     अनुवाकं   8 - ओदनसवगता स्थारोहणमन्त्राः} \newline
                                \textbf{ TB 2.7.8.1} \newline
                  अ॒भिप्रेहि॑ वी॒रय॑स्व । उ॒ग्रश्चेत्ता॑ सपत्न॒हा ॥ आति॑ष्ठ मित्र॒वर्द्ध॑नः । तुभ्यं॑ दे॒वा अधि॑ब्रवन्न् ॥ "अ॒ङ्कौ न्य॒ङ्काव॒भित॒{27}” "आति॑ष्ठ वृत्रह॒न्-रथ᳚म्{28}" ॥ आ॒तिष्ठ॑न्तं॒ परि॒ विश्वे॑ अभूषन्न् । श्रियं॒ ॅवसा॑नश्चरति॒ स्वरो॑चाः । म॒हत्-तद॒स्यासु॑रस्य॒ नाम॑ । आ वि॒श्वरू॑पो अ॒मृता॑नि तस्थौ ॥ अनु॒ त्वेन्द्रो॑ मद॒त्वनु॒ बृह॒स्पतिः॑ \textbf{ 23} \newline
                  \newline
                                \textbf{ TB 2.7.8.2} \newline
                  अनु॒ सोमो॒ अन्व॒ग्निरा॑वीत् । अनु॑ त्वा॒ विश्वे॑ दे॒वा अ॑वन्तु । अनु॑ स॒प्त राजा॑नो॒ य उ॒ताभिषि॑क्ताः ॥ अनु॑ त्वा मि॒त्रावरु॑णावि॒हाव॑तम् । अनु॒ द्यावा॑पृथि॒वी वि॒श्वश॑भूं । सूर्यो॒ अहो॑भि॒रनु॑ त्वाऽवतु । च॒न्द्रमा॒ नक्ष॑त्रै॒रनु॑ त्वाऽवतु ॥ द्यौश्च॑ त्वा पृथि॒वी च॒ प्रचे॑तसा । शु॒क्रो बृ॒हद्-दक्षि॑णा त्वा पिपर्तु । अनु॑ स्व॒धा चि॑किताꣳ॒॒ सोमो॑ अ॒ग्निः ( ) । आऽयं पृ॑णक्तु॒ रज॑सी उ॒पस्थ᳚म् । \textbf{ 24} \newline
                  \newline
                                    (बृह॒स्पतिः॒ - सोमो॑ अ॒ग्निरेकं॑ च) \textbf{(A8)} \newline \newline
                \textbf{ 2.7.9     अनुवाकं   9 - ओदनसवः} \newline
                                \textbf{ TB 2.7.9.1} \newline
                  प्र॒जाप॑तिः प्र॒जा अ॑सृजत । ता अ॑स्माथ्-सृ॒ष्टाः परा॑चीरायन्न् । स ए॒तं प्र॒जाप॑तिरोद॒न-म॑पश्यत् । सोऽन्नं॑ भू॒तो॑ऽतिष्ठत् । ता अ॒न्यत्रा॒न्नाद्य॒मवि॑त्त्वा । प्र॒जाप॑तिं प्र॒जा उ॒पाव॑र्तन्त । अन्न॑मे॒वैनं॑ भू॒तं पश्य॑न्तीः प्र॒जा उ॒पाव॑र्तन्ते । य ए॒तेन॒ यज॑ते । य उ॑ चैनमे॒वं ॅवेद॑ ॥ सर्वा॒ण्यन्ना॑नि भवन्ति \textbf{ 25} \newline
                  \newline
                                \textbf{ TB 2.7.9.2} \newline
                  सर्वे॒ पुरु॑षाः । सर्वा᳚ण्ये॒वान्ना॒न्यव॑रुन्धे । सर्वा॒न् पुरु॑षान् ॥ राड॑सि वि॒राड॒सीत्या॑ह । स्वारा᳚ज्यमे॒वैनं॑ गमयति ॥ यद्धिर॑ण्यं॒ ददा॑ति । तेज॒स्तेनाव॑रुन्धे । यत्ति॑सृध॒न्वम् । वी॒र्यं॑ तेन॑ । यदष्ट्रा᳚म् \textbf{ 26} \newline
                  \newline
                                \textbf{ TB 2.7.9.3} \newline
                  पुष्टिं॒ तेन॑ । यत्-क॑म॒ण्डलु᳚म् । आयु॒ष्टेन॑ ॥ यद्धिर॑ण्यमाब॒द्ध्नाति॑ । ज्योति॒र्वै हिर॑ण्यम् । ज्योति॑रे॒वास्मि॑न्-दधाति । अथो॒ तेजो॒ वै हिर॑ण्यम् । तेज॑ ए॒वात्मन् ध॑त्ते ॥ यदो॑द॒नं प्रा॒श्नाति॑ । ए॒तदे॒व सर्व॑मव॒रुद्ध्य॑ \textbf{ 27} \newline
                  \newline
                                \textbf{ TB 2.7.9.4} \newline
                  तद॑स्मिन्-नेक॒धाऽधा᳚त् ॥ रो॒हि॒ण्यां का॒र्यः॑ । यद्ब्रा᳚ह्म॒ण ए॒व रो॑हि॒णी । तस्मा॑दे॒व । अथो॒ वर्ष्मै॒वैनꣳ॑ समा॒नानां᳚ करोति ॥ उ॒द्य॒ता सूर्ये॑ण का॒र्यः॑ । उ॒द्यन्तं॒ ॅवा ए॒तꣳ सर्वाः᳚ प्र॒जाः प्रति॑नन्दन्ति ॥ दि॒दृ॒क्षेण्यो॑ दर्.श॒नीयो॑ भवति । य ए॒वं ॅवेद॑ ॥ ब्र॒ह्म॒वा॒दिनो॑ वदन्ति \textbf{ 28} \newline
                  \newline
                                \textbf{ TB 2.7.9.5} \newline
                  अ॒वेत्यो॑ऽवभृ॒था(3) ना(3) इति॑ । यद्-द॑र्भपुञ्जी॒लैः प॒वय॑ति । तथ् स्वि॑दे॒वावै॑ति । तन्नावै॑ति ॥ त्रि॒भिः प॑वयति । त्रय॑ इ॒मे लो॒काः । ए॒भिरे॒वैनं॑ ॅलो॒कैः प॑वयति । अथो॑ अ॒पां ॅवा ए॒तत् तेजो॒ वर्चः॑ । यद्-द॒र्भाः । यद्-द॑र्भपुञ्जी॒लैः प॒वय॑ति ( ) । अ॒पामे॒वैनं॒ तेज॑सा॒ वर्च॑सा॒ऽभिषि॑ञ्चति । \textbf{ 29} \newline
                  \newline
                                    (भ॒व॒ - न्त्यष्ट्रा॑ - मव॒रुद्ध्य॑ - वदन्ति - द॒र्भा यद् द॑र्भपुञ्जी॒लैः प॒वय॒त्येकं॑ च ) \textbf{(A9)} \newline \newline
                \textbf{ 2.7.10     अनुवाकं   10 - पञ्चशारदीयविधिः} \newline
                                \textbf{ TB 2.7.10.1} \newline
                  प्र॒जाप॑तिरकामयत ब॒होर्भूया᳚न्थ्-स्या॒मिति॑ । स ए॒तं प॑ञ्चशार॒दीय॑-मपश्यत् । तमाऽह॑रत् । तेना॑यजत । ततो॒ वै स ब॒होर्भूया॑-नभवत् । यः का॒मये॑त ब॒होर्भूया᳚न्थ्-स्या॒मिति॑ । स प॑ञ्चशार॒दीये॑न यजेत । ब॒होरे॒व भूया᳚न् भवति ॥ म॒रु॒थ् स्तो॒मो वा ए॒षः । म॒रुतो॒ हि दे॒वानां॒ भूयि॑ष्ठाः । \textbf{ 30} \newline
                  \newline
                                \textbf{ TB 2.7.10.2} \newline
                  ब॒हुर्भ॑वति । य ए॒तेन॒ यज॑ते । य उ॑ चैनमे॒वं ॅवेद॑ ॥ प॒ञ्च॒शा॒र॒दीयो॑ भवति । पञ्च॒ वा ऋ॒तवः॑ सम्ॅवथ्स॒रः । ऋ॒तुष्वे॒व सं॑ॅवथ्स॒रे प्रति॑तिष्ठति । अथो॒ पञ्चा᳚क्षरा प॒ङ्क्तिः । पाङ्क्तो॑ य॒ज्ञ्ः । य॒ज्ञ्मे॒वाव॑रुन्धे ॥ स॒प्त॒द॒शꣳ स्तोमा॒ नाति॑यन्ति ( ) । स॒प्त॒द॒शः प्र॒जाप॑तिः । प्र॒जाप॑ते॒राप्त्यै᳚ । \textbf{ 31} \newline
                  \newline
                                    (भूयि॑ष्ठा - यन्ति॒ द्वे च॑) \textbf{(A10)} \newline \newline
                \textbf{ 2.7.11     अनुवाकं   11 - पञ्चशारदीयगतपशुविधिः} \newline
                                \textbf{ TB 2.7.11.1} \newline
                  अ॒गस्यो॑ म॒रुद्भ्य॑ उ॒क्ष्णः प्रौक्ष॑त् । तानिन्द्र॒ आद॑त्त । त ए॑नं॒ ॅवज्र॑मु॒द्यत्या॒भ्या॑यन्त । तान॒गस्त्य॑श्चै॒वेन्द्र॑श्च कयाशु॒भीये॑नाशमयताम् । ताञ्छा॒न् तानुपा᳚ह्वयत । यत् क॑याशु॒भीयं॒ भव॑ति॒ शान्त्यै᳚ ॥ तस्मा॑दे॒त ऐ᳚न्द्रा मारु॒ता उ॒क्षाणः॑ सव॒नीया॑ भवन्ति । त्रयः॑ प्रथ॒मेऽह॒न्नाल॑भ्यन्ते । ए॒वं द्वि॒तीये᳚ । ए॒वं तृ॒तीये᳚ \textbf{ 32} \newline
                  \newline
                                \textbf{ TB 2.7.11.2} \newline
                  ए॒वं च॑तु॒र्थे । पञ्चो᳚त्त॒मेऽह॒न्नाल॑भ्यन्ते । वर्.षि॑ष्ठमिव॒ ह्ये॑तदहः॑ ॥ वर्.षि॑ष्ठः समा॒नानां᳚ भवति । य ए॒तेन॒ यज॑ते । य उ॑ चैनमे॒वं ॅवेद॑ ॥ स्वारा᳚ज्यं॒ ॅवा ए॒ष य॒ज्ञ्ः । ए॒तेन॒ वा एक॒यावा॑ कान्द॒मः स्वारा᳚ज्य-मगच्छत् ॥ स्वारा᳚ज्यं गच्छति । य ए॒तेन॒ यज॑ते \textbf{ 33} \newline
                  \newline
                                \textbf{ TB 2.7.11.3} \newline
                  य उ॑ चैनमे॒वं ॅवेद॑ ॥ मा॒रु॒तो वा ए॒ष स्तोमः॑ । ए॒तेन॒ वै म॒रुतो॑ दे॒वानां॒ भूयि॑ष्ठा अभवन्न् । भूयि॑ष्ठः समा॒नानां᳚ भवति । य ए॒तेन॒ यज॑ते । य उ॑ चैनमे॒वं ॅवेद॑ ॥ प॒ञ्च॒शा॒र॒दीयो॒ वा ए॒ष य॒ज्ञ्ः । आ प॑ञ्च॒मात्-पुरु॑षा॒दन्न॑मत्ति । य ए॒तेन॒ यज॑ते । य उ॑ चैनमे॒वं ॅवेद॑ ( ) ॥ स॒प्त॒द॒शꣳ स्तोमा॒ नाति॑यन्ति । स॒प्त॒द॒शः प्र॒जाप॑तिः । प्र॒जाप॑तेरे॒व नैति॑ । \textbf{ 34} \newline
                  \newline
                                                        \textbf{special korvai} \newline
              (अ॒गस्त्यः॒ स्वारा᳚ज्यं मारु॒तः प॑ञ्चशार॒दीयो॒ वा ए॒ष य॒ज्ञ्ः स॑प्तद॒शंप्र॒जाप॑तेरे॒व नैति॑) \newline
                                (तृ॒तीये॑ - गच्छति॒ य ए॒तेन॒ यज॑ते - ऽत्ति॒ य ए॒तेन॒ यज॑ते॒ य उ॑ चैनमे॒वं ॅवेद॒ त्रीणि॑ च) \textbf{(A11)} \newline \newline
                \textbf{ 2.7.12     अनुवाकं   12 - अग्निष्टुद्यागे ग्रहाणां पुरोरुचः} \newline
                                \textbf{ TB 2.7.12.1} \newline
                  अ॒स्याजरा॑सो द॒मा म॒रित्राः᳚ । अ॒र्चद्धू॑मासो अ॒ग्नयः॑ पाव॒काः । श्वि॒ची॒चयः॑ श्वा॒त्रासो॑ भुर॒ण्यवः॑ । व॒न॒र॒.षदो॑ वा॒यवो॒ न सोमाः᳚ ॥ यजा॑ नो मि॒त्रावरु॑णा । यजा॑ दे॒वाꣳ ऋ॒तं बृ॒हत् । अग्ने॒ यक्षि॒ स्वं दम᳚म् ॥ अश्वि॑ना॒ पिब॑तꣳ सु॒तम् । दीद्य॑ग्नी शुचि व्रता । ऋ॒तुना॑ यज्ञ्वाहसा । \textbf{ 35} \newline
                  \newline
                                \textbf{ TB 2.7.12.2} \newline
                  द्वे विरू॑पे चरतः॒ स्वर्थे᳚ । अ॒न्याऽन्या॑ व॒थ्समुप॑धापयेते । हरि॑र॒न्यस्यां॒ भव॑ति स्व॒धावान्॑ । शु॒क्रो अ॒न्यस्यां᳚ ददृशे सु॒वर्चाः᳚ ॥ पू॒र्वा॒प॒रं च॑रतो मा॒ययै॒तौ । शिशू॒ क्रीड॑न्तौ॒ परि॑यातो अद्ध्व॒रम् । विश्वा᳚न्य॒न्यो भुव॑नाऽभि॒चष्टे᳚ । ऋ॒तून॒न्यो वि॒दध॑ज्जायते॒ पुनः॑ ॥ त्रीणि॑ श॒ता त्रीष॒हस्रा᳚ण्य॒ग्निम् । त्रिꣳ॒॒शच्च॑ दे॒वा नव॑ चासपर्यन्न् \textbf{ 36} \newline
                  \newline
                                \textbf{ TB 2.7.12.3} \newline
                  औक्ष॑न्-घृ॒तैरास्तृ॑णन्-ब॒र्॒.हिर॑स्मै । आदिद्धोता॑रं॒ न्य॑षादयन्त ॥ अ॒ग्निना॒ऽग्निः समि॑द्ध्यते । क॒विर्-गृ॒हप॑ति॒र्युवा᳚ । ह॒व्य॒वाड्-जु॒ह्वा᳚स्यः ॥ अ॒ग्निर्-दे॒वानां᳚ ज॒ठर᳚म् । पू॒तद॑क्षः क॒विक्र॑तुः । दे॒वो दे॒वेभि॒राग॑मत् ॥ अ॒ग्नि॒श्रियो॑ म॒रुतो॑ वि॒श्वकृ॑ष्टयः । आ त्वे॒षमु॒ग्रमव॑ ईमहे व॒यम् \textbf{ 37} \newline
                  \newline
                                \textbf{ TB 2.7.12.4} \newline
                  ते स्वा॒निनो॑ रु॒द्रिया॑ व॒र॒.षनि॑र्णिजः । सिꣳ॒॒हा न हे॒षक्र॑तवः सु॒दान॑वः ॥ यदु॑त्त॒मे म॑रुतो मद्ध्य॒मे वा᳚ । यद्वा॑ऽव॒मे सु॑भगासो दि॒विष्ठ । ततो॑ नो रुद्रा उ॒त वा॒ऽन्वस्य॑ । अग्ने॑ वि॒त्ताद्ध॒विषो॒ यद्यजा॑मः ॥ ईडे॑ अ॒ग्निꣳ स्वव॑सं॒ नमो॑भिः । इ॒ह प्र॑स॒प्तो विच॑यत्कृ॒तं नः॑ । रथै॑रिव॒ प्रभ॑रे वाज॒यद्भिः॑ । प्र॒द॒क्षि॒णिन्-म॒रुताꣳ॒॒ स्तोम॑मृद्ध्याम् । \textbf{ 38} \newline
                  \newline
                                \textbf{ TB 2.7.12.5} \newline
                  श्रु॒धि श्रु॑त्कर्ण॒ वह्नि॑भिः । दे॒वैर॑ग्ने स॒याव॑भिः । आसी॑दन्तु ब॒र॒.हिषि॑ । मि॒त्रो वरु॑णो अर्य॒मा । प्रा॒त॒र्यावा॑णो अद्ध्व॒रम् ॥ विश्वे॑षा॒-मदि॑तिर्-य॒ज्ञिया॑नाम् । विश्व॑षा॒-मति॑थि॒र्-मानु॑षाणाम् । अ॒ग्निर्-दे॒वाना॒मव॑ आवृणा॒नः । सु॒मृ॒डी॒को भ॑वतु वि॒श्ववे॑दाः ॥ त्वे अ॑ग्ने सुम॒तिं भिक्ष॑माणाः \textbf{ 39} \newline
                  \newline
                                \textbf{ TB 2.7.12.6} \newline
                  दि॒वि श्रवो॑ दधिरे य॒ज्ञिया॑सः । नक्ता॑ च च॒क्रुरु॒षसा॒ विरू॑पे । कृ॒ष्णं च॒ वर्ण॑मरु॒णं च॒ सन्धुः॑ ॥ त्वाम॑ग्न आदि॒त्यास॑ आ॒स्य᳚म् । त्वां जि॒ह्वाꣳ शुच॑यश्चक्रिरे कवे । त्वाꣳ रा॑ति॒षाचो॑ अद्ध्व॒रेषु॑ सश्चिरे । त्वे दे॒वा ह॒विर॑द॒न्त्याहु॑तम् ॥ नि त्वा॑ य॒ज्ञ्स्य॒ साध॑नम् । अग्ने॒ होता॑र-मृ॒त्विज᳚म् । व॒नु॒ष्वद्-दे॑व धीमहि॒ प्रचे॑तसम् ( ) । जी॒रं दू॒तमम॑र्त्यम् । \textbf{ 40} \newline
                  \newline
                                    (य॒ज्ञ्॒वा॒ह॒सा॒ - ऽस॒प॒र्य॒न् - व॒य - मृ॑द्ध्यां॒ - भिक्ष॑माणाः॒ - प्रचे॑तस॒मेकं॑ च) \textbf{(A12)} \newline \newline
                \textbf{ 2.7.13     अनुवाकं   13 - इन्द्रस्तुद्यागे ग्रहाणां पुरोरुचः} \newline
                                \textbf{ TB 2.7.13.1} \newline
                  तिष्ठा॒ हरी॒ रथ॒ आ यु॒ज्यमा॑ना या॒हि । वा॒युर्न नि॒युतो॑ नो॒ अच्छ॑ । पिबा॒स्यन्धो॑ अ॒भिसृ॑ष्टो अ॒स्मे । इन्द्र॒ स्वाहा॑ ररि॒मा ते॒ मदा॑य ॥ कस्य॒ वृषा॑ सु॒ते सचा᳚ । नि॒युत्वा᳚न् वृष॒भो र॑णत् । वृ॒त्र॒हा सोम॑पीतये ॥ इन्द्रं॑ ॅव॒यं म॑हाध॒ने । इन्द्र॒मर्भे॑ हवामहे । युजं॑ ॅवृ॒त्रेषु॑ व॒ज्रिण᳚म् । \textbf{ 41} \newline
                  \newline
                                \textbf{ TB 2.7.13.2} \newline
                  द्वि॒तायो वृ॑त्र॒हन्त॑मः । वि॒द इन्द्रः॑ श॒तक्र॑तुः । उप॑ नो॒ हरि॑भिः सु॒तम् ॥ स सूर॒ आज॒नय॒ञ्ज्योति॒रिन्द्र᳚म् । अ॒या धि॒या त॒रणि॒रद्रि॑बर्.हाः । ऋ॒तेन॑ शु॒ष्मीनव॑मानो अ॒र्कैः । व्यु॑स्रिधो॑ अ॒स्रो अद्रि॑र्बिभेद ॥ उ॒त त्यदा॒श्वश्वि॑यम् । यदि॑न्द्र॒ नाहु॑षी॒ष्वा । अग्रे॑ वि॒क्षु प्रतीद॑यत् । \textbf{ 42} \newline
                  \newline
                                \textbf{ TB 2.7.13.3} \newline
                  भरे॒ष्विन्द्रꣳ॑ सु॒हवꣳ॑ हवामहे । अꣳ॒॒हो॒मुचꣳ॑ सु॒कृतं॒ दैव्य॒ञ्जन᳚म् । अ॒ग्निं मि॒त्रं ॅवरु॑णꣳ सा॒तये॒ भग᳚म् । द्यावा॑पृथि॒वी म॒रुतः॑ स्व॒स्तये᳚ ॥ म॒हि क्षेत्रं॑ पु॒रुश्च॒न्द्रं ॅविवि॒द्वान् । आदिथ् सखि॑भ्यश्च॒ रथꣳ॒॒ समै॑रत् । इन्द्रो॒ नृभि॑रजन॒द्-दीद्या॑नः सा॒कम् । सूर्य॑मु॒षसं॑ गा॒तुम॒ग्निम् ॥ उ॒रुं नो॑ लो॒कमनु॑नेषि वि॒द्वान् । सुव॑र्व॒-ज्ज्योति॒रभ॑यꣳ स्व॒स्ति \textbf{ 43} \newline
                  \newline
                                \textbf{ TB 2.7.13.4} \newline
                  ऋ॒ष्वा त॑ इन्द्र॒ स्थवि॑रस्य बा॒हू । उप॑स्थेयाम शर॒णा बृ॒हन्ता᳚ ॥ आ नो॒ विश्वा॑भिरू॒तिभिः॑ स॒जोषाः᳚ । ब्रह्म॑ जुषा॒णो ह॑र्यश्व याहि । वरी॑वृज॒थ् स्थवि॑रेभिः सुशिप्र । अ॒स्मे दध॒द्-वृष॑णꣳ॒॒ शुष्म॑मिन्द्र ॥ इन्द्रा॑य॒ गाव॑ आ॒शिर᳚म् । दु॒दु॒ह्रे व॒ज्रिणे॒ मधु॑ । यथ् सी॑मुपह्व॒रेऽवि॒दत् ॥ तास्ते॑ वज्रिन् धे॒नवो॑ जोजयुर्नः ( ) \textbf{ 44} \newline
                  \newline
                                \textbf{ TB 2.7.13.5} \newline
                  गभ॑स्तयो नि॒युतो॑ वि॒श्ववा॑राः । अह॑रह॒र्भूय॒ इज्जोगु॑वानाः । पू॒र्णा इ॑न्द्र क्षु॒मतो॒ भोज॑नस्य ॥ इ॒मां ते॒ धियं॒ प्रभ॑रे म॒हो म॒हीम् । अ॒स्य स्तो॒त्रे धि॒षणा॒ यत्त॑ आन॒जे । तमु॑थ्स॒वे च॑ प्रस॒वे च॑ सास॒हिम् । इन्द्रं॑ दे॒वासः॒ शव॑सा मद॒न्ननु॑ । \textbf{ 45} \newline
                  \newline
                                    (व॒ज्रिण॑ - मयथ् - स्व॒स्ति - जो॑जयुर्नः - +स॒प्त च॑) \textbf{(A13)} \newline \newline
                \textbf{ 2.7.14     अनुवाकं   14 - अप्तोर्यामाविधिः} \newline
                                \textbf{ TB 2.7.14.1} \newline
                  प्र॒जाप॑तिः प॒शून॑सृजत । ते᳚ऽस्माथ् सृ॒ष्टाः परा᳚ञ्च आयन्न् । तान॑ग्निष्टो॒मेन॒ नाप्नो᳚त् । तानु॒क्थ्ये॑न॒ नाप्नो᳚त् । तान्थ् षो॑ड॒शिना॒ नाप्नो᳚त् । तान्रात्रि॑या॒ नाप्नो᳚त् । तान्थ् स॒न्धिना॒ नाप्नो᳚त् । सो᳚ऽग्निम॑ब्रवीत् । इ॒मान्म॑ ई॒फ्सेति॑ । तान॒ग्निस्त्रि॒वृता॒ स्तोमे॑न॒ नाप्नो᳚त् \textbf{ 46} \newline
                  \newline
                                \textbf{ TB 2.7.14.2} \newline
                  स इन्द्र॑मब्रवीत् । इ॒मान्म॑ ई॒फ्सेति॑ । तानिन्द्रः॑ पञ्चद॒शेन॒ स्तोमे॑न॒ नाप्नो᳚त् । स विश्वा᳚न् दे॒वा-न॑ब्रवीत् । इ॒मान्म॑ ईफ्स॒तेति॑ । तान्. विश्वे॑ दे॒वाः स॑प्तद॒शेन॒ स्तोमे॑न॒ नाप्नु॑वन्न् । स विष्णु॑मब्रवीत् । इ॒मान्म॑ ई॒फ्सेति॑ । तान्. विष्णु॑रेकविꣳ॒॒शेन॒ स्तोमे॑नाप्नोत् । वा॒र॒व॒न्तीये॑नावारयत ( ) \textbf{ 47} \newline
                  \newline
                                \textbf{ TB 2.7.14.3} \newline
                  इ॒दं ॅविष्णु॒र् विच॑क्रम॒ इति॒ व्य॑क्रमत ॥ यस्मा᳚त् प॒शवः॒ प्र प्रेव॒ भ्रꣳशे॑रन्न् । स ए॒तेन॑ यजेत ॥ यदाप्नो᳚त् । तद॒प्तोर्याम॑स्याप्तोर्याम॒त्वम् ॥ ए॒तेन॒ वै दे॒वा जैत्वा॑नि जि॒त्वा । यं काम॒-मका॑मयन्त॒ तमा᳚प्नुवन्न् । यं कामं॑ का॒मय॑ते । तमे॒तेना᳚प्नोति । \textbf{ 48} \newline
                  \newline
                                    (स्तोमे॑न॒ नाप्नो॑ - दवारयत॒ - +नव॑ च) \textbf{(A14)} \newline \newline
                \textbf{ 2.7.15     अनुवाकं   15 - राजाभिषेकः} \newline
                                \textbf{ TB 2.7.15.1} \newline
                  व्या॒घ्रो॑ऽयम॒ग्नौ च॑रति॒ प्रवि॑ष्टः । ऋषी॑णां पु॒त्रो अ॑भिशस्ति॒पा अ॒यम् । न॒म॒स्का॒रेण॒ नम॑सा ते जुहोमि । मा दे॒वानां᳚ मिथु॒या क॑र्म भा॒गम् ॥ सावी॒र॒.हि दे॑व प्रस॒वाय॑ पित्रे । व॒र्ष्माण॑मस्मै वरि॒माण॑मस्मै । अथा॒स्मभ्यꣳ॑ सवितः स॒र्वता॑ता । दि॒वे दि॑व॒ आसु॑वा॒ भूरि॑प॒श्वः ॥ भू॒तो भू॒तेषु॑ चरति॒ प्रवि॑ष्टः । स भू॒ताना॒-मधि॑पतिर्-बभूव \textbf{ 49} \newline
                  \newline
                                \textbf{ TB 2.7.15.2} \newline
                  तस्य॑ मृ॒त्यौ च॑रति राज॒सूय᳚म् । स राजा॑ रा॒ज्य-मनु॑मन्यतामि॒दम् ॥ येभिः॒ शिल्पैः᳚ पप्रथा॒नामदृꣳ॑हत् । येभि॒द्र्याम॒भ्यपिꣳ॑शत्-प्र॒जाप॑तिः । येभि॒र्वाचं॑ ॅवि॒श्वरू॑पाꣳ स॒मव्य॑यत् । तेने॒मम॑ग्न इ॒ह वर्च॑सा॒ सम॑ङ्ग्धि ॥ येभि॑रादि॒त्यस्तप॑ति॒ प्रके॒तुभिः॑ । येभिः॒ सूर्यो॑ ददृ॒शे चि॒त्रभा॑नुः । येभि॒र्वाचं॑ पुष्क॒लेभि॒रव्य॑यत् । तेने॒मम॑ग्न इ॒ह वर्च॑सा॒ सम॑ङ्ग्धि । \textbf{ 50} \newline
                  \newline
                                \textbf{ TB 2.7.15.3} \newline
                  आऽयं भा॑तु॒ शव॑सा॒ पञ्च॑ कृ॒ष्टीः । इन्द्र॑ इव ज्ये॒ष्ठो भ॑वतु प्र॒जावान्॑ । अ॒स्मा अ॑स्तु पुष्क॒लं चि॒त्रभा॑नु । आऽयं पृ॑णक्तु॒ रज॑सी उ॒पस्थ᳚म् ॥ यत्ते॒ शिल्पं॑ कश्यप रोच॒नाव॑त् । इ॒न्द्रि॒याव॑त्-पुष्क॒लं चि॒त्रभा॑नु । यस्मि॒न्थ् सूर्या॒ अर्पि॑ताः स॒प्त सा॒कम् । तस्मि॒न्-राजा॑न॒-मधि॒विश्र॑ये॒मम् ॥ द्यौर॑सि पृथि॒व्य॑सि ॥ व्या॒घ्रो वैया॒घ्रेऽधि॑ \textbf{ 51} \newline
                  \newline
                                \textbf{ TB 2.7.15.4} \newline
                  विश्र॑यस्व॒ दिशो॑ म॒हीः । विश॑स्त्वा॒ सर्वा॑ वाञ्छन्तु । मा त्वद्-रा॒ष्ट्रमधि॑भ्रशत् ॥ या दि॒व्या आपः॒ पय॑सा संबभू॒वुः । या अ॒न्तरि॑क्ष उ॒त पार्थि॑वी॒र्याः । ता सां᳚ त्वा॒ सर्वा॑साꣳ रु॒चा । अ॒भिषि॑ञ्चामि॒ वर्च॑सा । अ॒भि त्वा॒ वर्च॑सा सिचं दि॒व्येन॑ । पय॑सा स॒ह । यथाऽऽसा॑ राष्ट्र॒वर्द्ध॑नः \textbf{ 52} \newline
                  \newline
                                \textbf{ TB 2.7.15.5} \newline
                  तथा᳚ त्वा सवि॒ता क॑रत् । इन्द्रं॒ ॅविश्वा॑ अवीवृधन्न् । स॒मु॒द्र व्य॑चसं॒ गिरः॑ । र॒थीत॑मꣳ रथी॒नाम् । वाजा॑नाꣳ॒॒ सत् प॑तिं॒ पति᳚म् । वस॑वस्त्वा पु॒रस्ता॑द॒भिषि॑ञ्चन्तु गाय॒त्रेण॒ छन्द॑सा । रु॒द्रास्त्वा॑ दक्षिण॒तो॑ऽभिषि॑ञ्चन्तु॒ त्रैष्टु॑भेन॒ छन्द॑सा । आ॒दि॒त्यास्त्वा॑ प॒श्चाद॒भिषि॑ञ्चन्तु॒ जाग॑तेन॒ छन्द॑सा । विश्वे᳚ त्वा दे॒वा उ॑त्तर॒तो॑ऽभिषि॑ञ्च॒न्-त्वानु॑ष्टुभेन॒ छन्द॑सा । बृह॒स्पति॑-स्त्वो॒परि॑ष्टा-द॒भिषि॑ञ्चतु॒ पाङ्क्ते॑न॒ छन्द॑सा । \textbf{ 53} \newline
                  \newline
                                \textbf{ TB 2.7.15.6} \newline
                  अ॒रु॒णं त्वा॒ वृक॑मु॒ग्रं ख॑जं क॒रम् । रोच॑मानं म॒रुता॒मग्रे॑ अ॒र्चिषः॑ । सूर्य॑वन्तं म॒घवा॑नं ॅविषास॒हिम् । इन्द्र॑मु॒क्थ्येषु॑ नाम॒हूत॑मꣳ हुवेम ॥ प्र बा॒हवा॑ सिसृतं जी॒वसे॑ नः । आ नो॒ गव्यू॑तिमुक्षतं घृ॒तेन॑ । आ नो॒ जने᳚ श्रवयतं ॅयुवाना । श्रु॒तं मे॑ मित्रावरुणा॒ हवे॒मा ॥ इन्द्र॑स्य ते वीर्य॒कृतः॑ । बा॒हू उ॒पाव॑हरामि ( ) । \textbf{ 54} \newline
                  \newline
                                    (ब॒भू॒ - वाव्य॑य॒त् तेने॒मम॑ग्न इ॒ह वर्च॑सा॒ सम॑ङ्ग्धि॒ - वैया॒घ्रेऽधि॑ - राष्ट्र॒वर्द्ध॑नः॒ - पाङ्क्ते॑न॒ छन्द॑सो॒ - पाव॑हरामि) \textbf{(A15)} \newline \newline
                \textbf{ 2.7.16     अनुवाकं   16 - राजाभिषेकाङ्गं रथारोहणम्} \newline
                                \textbf{ TB 2.7.16.1} \newline
                  अ॒भि प्रेहि॑ वी॒रय॑स्व । उ॒ग्रश्चेत्ता॑ सपत्न॒हा ॥ आति॑ष्ठ वृत्र॒हन्त॑मः । तुभ्यं॑ दे॒वा अधि॑ब्रवन्न् ॥ अ॒ङ्कौ न्य॒ङ्काव॒भितो॒ रथं॒ ॅयौ । ध्वा॒न्तं ॅवा॑ता॒ग्रमनु॑सं॒ चर॑न्तौ । दू॒रे हे॑ति-रिन्द्रि॒यावा᳚न् पत॒त्री । ते नो॒ऽग्नयः॒ पप्र॑यः पारयन्तु ॥ नम॑स्त ऋषे गद । अव्य॑थायै  (अव्य॑धायै)  त्वा स्व॒धायै᳚ त्वा \textbf{ 55} \newline
                  \newline
                                \textbf{ TB 2.7.16.2} \newline
                  मा न॑ इन्द्रा॒भित॒स्त्व-दृ॒ष्वारि॑ष्टासः । ए॒वा ब्र॑ह्म॒न्तवेद॑स्तु ॥ तिष्ठा॒ रथे॒ अधि॒ यद्-वज्र॑हस्तः ॥ आ र॒श्मीन्दे॑व युवसे॒ स्वश्वः॑ ॥ "आति॑ष्ठ वृत्रहन् {29}", "ना॒तिष्ठ॑न्तं॒ परि॑ {30}" । "अनु॒ त्वेन्द्रो॑ मद॒ {31]". "त्वनु॑ त्वा मि॒त्रावरु॑णौ {32}" ॥ द्यौश्च॑ त्वा पृथि॒वी च॒ प्रचे॑तसा । शु॒क्रो बृ॒हद्-दक्षि॑णा त्वा पिपर्तु । अनु॑ स्व॒धा चिकिताꣳ॒॒ सोमो॑ अ॒ग्निः । अनु॑ त्वाऽवतु सवि॒ता स॒वेन॑ । \textbf{ 56} \newline
                  \newline
                                \textbf{ TB 2.7.16.3} \newline
                  इन्द्रं॒ ॅविश्वा॑ अवीवृधन्न् । स॒मु॒द्र व्य॑चसं॒ गिरः॑ । र॒थीत॑मꣳ रथी॒नाम् । वाजा॑नाꣳ॒॒ सत् प॑तिं॒ पति᳚म् ॥ परि॑ मा से॒न्या घोषाः᳚ । ज्यानां᳚ ॅवृञ्जन्तु गृ॒द्ध्नवः॑ । मे॒थि॒ष्ठाः पिन्व॑माना इ॒ह । मां गोप॑तिम॒भि सम्ॅवि॑शन्तु ॥ तन्मेऽनु॑मति॒-रनु॑मन्यताम् । तन्मा॒ता पृ॑थि॒वी तत् पि॒ता द्यौः \textbf{ 57} \newline
                  \newline
                                \textbf{ TB 2.7.16.4} \newline
                  तद्ग्रावा॑णः सोम॒सुतो॑ मयो॒भुवः॑ । तद॑श्विना शृणुतꣳ सौभगा यु॒वम् ॥ "अव॑ते॒ हेड॒ {33}", "उदु॑त्त॒मम् {34}" ॥ ए॒ना व्या॒घ्रं प॑रिषस्वजा॒नाः । सिꣳ॒॒हꣳ हि॑न्वन्ति मह॒ते सौभ॑गाय । स॒मु॒द्रं न सु॒हवं॑ तस्थि॒वाꣳस᳚म् । म॒र्मृ॒ज्यन्ते᳚ द्वी॒पिन॑म॒फ्स्व॑न्तः ॥ उद॒सावे॑तु॒ सूर्यः॑ । उदि॒दं मा॑म॒कं ॅवचः॑ । उदि॑हि देव सूर्य ( ) । स॒ह व॒ग्नुना॒ मम॑ । अ॒हं ॅवा॒चो वि॒वाच॑नम् । मयि॒ वाग॑स्तु धर्ण॒सिः । यन्तु॑ न॒दयो॒ वर्.ष॑न्तु प॒र्जन्याः᳚ । सु॒पि॒प्प॒ला ओष॑धयो भवन्तु ॥ अन्न॑वतामोद॒नव॑तामा॒मिक्ष॑वताम् । ए॒षाꣳ राजा॑ भूयासम् । \textbf{ 58} \newline
                  \newline
                                    (स्व॒धायै᳚ त्वा - स॒वेन॒ - द्यौः - सु᳚र्य स॒प्त च॑) \textbf{(A16)} \newline \newline
                \textbf{ 2.7.17     अनुवाकं   17 - राजाभिषेकाङ्गं वपनम्} \newline
                                \textbf{ TB 2.7.17.1} \newline
                  ये के॒शिनः॑ प्रथ॒माः स॒त्रमास॑त । येभि॒राभृ॑तं॒ ॅयदि॒दं ॅवि॒रोच॑ते । तेभ्यो॑ जुहोमि बहु॒धा घृ॒तेन॑ । रा॒यस्पोषे॑णे॒मं ॅवर्च॑सा॒ सꣳसृ॑जाथ ॥ नर्ते ब्रह्म॑ण॒स्तप॑सो विमो॒कः । द्वि॒नाम्नी॑ दी॒क्षा व॒शिनी॒ ह्यु॑ग्रा । प्रकेशाः᳚ सु॒वते॑ का॒ण्डिनो॑ भवन्ति । तेषां᳚ ब्र॒ह्मेदीशे॒ वप॑नस्य॒ नान्यः ॥ आरो॑ह॒ प्रोष्ठं॒ ॅविष॑हस्व॒ शत्रून्॑ । अवा᳚स्राग्दी॒क्षा व॒शिनी॒ ह्यु॑ग्रा \textbf{ 59} \newline
                  \newline
                                \textbf{ TB 2.7.17.2} \newline
                  दे॒हि दक्षि॑णां॒ प्रति॑र॒स्वायुः॑ । अथा॑ मुच्यस्व॒ वरु॑णस्य॒ पाशा᳚त् ॥ येनाव॑पथ् सवि॒ता क्षु॒रेण॑ । सोम॑स्य॒ राज्ञो॒ वरु॑णस्य वि॒द्वान् । तेन॑ ब्रह्माणो वपते॒दम॒स्योर्जेमम् । र॒य्या वर्च॑सा॒ सꣳसृ॑जाथ ॥ मा ते॒ केशा॒ननु॑ गा॒द्वर्च॑ ए॒तत् । तथा॑ धा॒ता क॑रोतु ते । तुभ्य॒मिन्द्रो॒ बृह॒स्पतिः॑ । स॒वि॒ता वर्च॒ आद॑धात् । \textbf{ 60} \newline
                  \newline
                                \textbf{ TB 2.7.17.3} \newline
                  तेभ्यो॑ नि॒धानं॑ बहु॒धा व्यैच्छन्न्॑ । अ॒न्त॒रा द्यावा॑पृथि॒वी अ॒पः सुवः॑ । द॒र्भ॒स्त॒म्बे वी॒र्य॑कृते नि॒धाय॑ । पौꣳस्ये॑ने॒मं ॅवर्च॑सा॒ सꣳसृ॑जाथ ॥ बलं॑ ते बाहु॒वोः स॑वि॒ता द॑धातु । सोम॑स्त्वाऽनक्तु॒ पय॑सा घृ॒तेन॑ । स्त्री॒षु रू॒पम॑श्विनै॒त-न्निध॑त्तम् । पौꣳस्ये॑ने॒मं वर्च॑सा॒ सꣳसृ॑जाथ ॥ यथ् सी॒मन्तं॒ कङ्क॑तस्ते लि॒लेख॑ । यद्वा᳚ क्षु॒रः प॑रिव॒वर्ज॒ वपꣳ॑स्ते ( ) । स्त्री॒षु रू॒पम॑श्विनै॒त-न्निध॑त्तम् । पौꣳस्ये॑ने॒मꣳ सꣳसृ॑जाथो वी॒र्ये॑ण । \textbf{ 61} \newline
                  \newline
                                                        \textbf{special korvai} \newline
              (ये के॒शिनो॒ नर्ते मा ते॒ बलं॒ ॅयथ् सी॒मन्तं॒ पञ्च॑) \newline
                                (अवा᳚स्राग्दी॒क्षा वा॒शिनी॒ ह्यु॑ग्रा-ऽऽद॑धाद्-व॒वर्ज॒ वपꣳ॑स्ते॒ द्वे च॑ ) \textbf{(A17)} \newline \newline
                \textbf{ 2.7.18     अनुवाकं   18 - विद्यनाख्य एकाहविशेषः} \newline
                                \textbf{ TB 2.7.18.1} \newline
                  इन्द्रं॒ ॅवै स्वा विशो॑ म॒रुतो॒ नापा॑चायन्न् । सोऽन॑पचाय्यमान ए॒तं ॅवि॑घ॒नम॑पश्यत् । तमाऽह॑रत् । तेना॑यजत । तेनै॒वासां॒ तꣳ सꣳ॑स्त॒म्भं ॅव्य॑हन्न् । यद् व्यहन्न्॑ । तद्-वि॑घ॒नस्य॑ विघन॒त्वम् ॥ वि पा॒प्मानं॒ भ्रातृ॑व्यꣳ हते । य ए॒तेन॒ यज॑ते । य उ॑ चैनमे॒वं ॅवेद॑ । \textbf{ 62} \newline
                  \newline
                                \textbf{ TB 2.7.18.2} \newline
                  यꣳ राजा॑नं॒ ॅविशो॒ नाप॒चाये॑युः । यो वा᳚ ब्राह्म॒णस्तम॑सा पा॒प्मना॒ प्रावृ॑तः॒ स्यात् । स ए॒तेन॑ यजेत । वि॒घ॒नेनै॒वैन॑द्-वि॒हत्य॑ । वि॒शामाधि॑पत्यं गच्छति ॥ तस्य॒ द्वे द्वा॑द॒शे स्तो॒त्रे भव॑तः । द्वे च॑तुर्विꣳ॒॒शे । औद्भि॑द्यमे॒व तत् । ए॒तद्वै क्ष॒त्रस्यौद्भि॑द्यम् । यद॑स्मै॒ स्वा विशो॑ ब॒लिꣳ हर॑न्ति । \textbf{ 63} \newline
                  \newline
                                \textbf{ TB 2.7.18.3} \newline
                  हर॑न्त्यस्मै॒ विशो॑ ब॒लिम् । ऐन॒मप्र॑तिख्यातं गच्छति । य ए॒वं ॅवेद॑ ॥ प्र॒बाहु॒ग्वा अग्रे᳚ क्ष॒त्राण्याते॑पुः । तेषा॒मिन्द्रः॑ क्ष॒त्राण्याद॑त्त । न वा इ॒मानि॑ क्ष॒त्राण्य॑ भूव॒न्निति॑ । तन्नक्ष॑त्राणां नक्षत्र॒त्वम् । आ श्रेय॑सो॒ भ्रातृ॑व्यस्य॒ तेज॑ इन्द्रि॒यं द॑त्ते । य ए॒तेन॒ यज॑ते । य उ॑ चैनमे॒वं ॅवेद॑ । \textbf{ 64} \newline
                  \newline
                                \textbf{ TB 2.7.18.4} \newline
                  तद् यथा॑ ह॒ वै स॑चा॒क्रिणौ॒ कप्ल॑कावु॒पा-व॑हितौ॒ स्याता᳚म् । ए॒वमे॒तौ यु॒ग्मन्तौ॒ स्तोमौ᳚ । अ॒युक्षु॒ स्तोमे॑षु क्रियेते । पा॒प्मनोऽप॑हत्यै ॥ अप॑ पा॒प्मानं॒ भ्रातृ॑व्यꣳ हते । य ए॒तेन॒ यज॑ते । य उ॑ चैनमे॒वं ॅवेद॑ ॥ तद् यथा॑ ह॒ वै सू॑तग्राम॒ण्यः॑ । ए॒वं छन्दाꣳ॑सि । तेष्व॒सावा॑दि॒त्यो बृ॑ह॒तीर॒भ्यू॑ढः \textbf{ 65} \newline
                  \newline
                                \textbf{ TB 2.7.18.5} \newline
                  स॒तोबृ॑हतीषु स्तुवते स॒तो बृ॑हन्न् । प्र॒जया॑ प॒शुभि॑-रसा॒नीत्ये॒व ॥ व्यति॑षक्ताभिः स्तुवते । व्यति॑षक्तं॒ ॅवै क्ष॒त्रं ॅवि॒शा । वि॒शैवैनं॑ क्ष॒त्रेण॒ व्यति॑षजति ॥ व्यति॑षक्ताभिः स्तुवते । व्यति॑षक्तो॒ वै ग्रा॑म॒णीः स॑जा॒तैः । स॒जा॒तैरे॒वैनं॒ ॅव्यति॑षजति ॥ व्यति॑षक्ताभिः स्तुवते । व्यति॑षक्तो॒ वै पुरु॑षः पा॒प्मभिः॑ ( ) । व्यति॑षक्ता-भिरे॒वास्य॑ पा॒प्मनो॑ नुदते । \textbf{ 66} \newline
                  \newline
                                    (वेद॒ - हर॑न् - त्येनमे॒वं ॅवेदा॒ - भ्यू॑ढः - पा॒प्मभि॒रेकं॑ च) \textbf{(A18)} \newline \newline
                \textbf{PrapAtaka Korvai with starting  words of 1 to 18 anuvAkams :-} \newline
        (त्रि॒वृद् - यदा᳚ग्ने॒यो᳚ऽग्निमु॑खा॒ ह्यृद्धि॒र् - यदा᳚ग्ने॒य आ᳚ग्ने॒यो - न वै सोमे॑न॒ - यो वै सोमे॑ - नै॒ष गो॑स॒वः - सिꣳ॒॒हे॑ - ऽभिप्रेहि॑ मित्र॒वर्द्ध॑नः - प्र॒जाप॑ति॒स्ता ओ॑द॒नं - प्र॒जाप॑तिरकामयत ब॒होर् भूया॑ - न॒गस्त्यो॒ - ऽस्याजरा॑स॒ - स्तिष्ठा॒ हरी᳚ - प्र॒जाप॑तिः प॒शून् - व्या॒घ्रो॑ऽयम॒ - भिप्रेहि॑ वृत्र॒हन्त॑मो॒ - ये के॒शिन॒ - इन्द्रं॒ ॅवा अ॒ष्टाद॑श ) \newline

        \textbf{korvai with starting words of 1, 11, 21 series of daSinis :-} \newline
        (त्रि॒वृद् - यो वै सोमे॒ - नायु॑रसि वि॒श्वायु॑र् - ब॒हुर् भ॑वति॒ - तिष्ठा॒ हरी॒ रथ॒ - आऽयं भा॑तु॒ - तेभ्यो॑ नि॒धानꣳ॒॒ षट्थ्ष॑ष्टिः) \newline

        \textbf{first and last  word 2nd aShTakam 6th prapAtakam :-} \newline
        (त्रि॒वृत् - पा॒प्मनो॑ नुदते) \newline 

       

        ॥ हरिः॑ ॐ ॥
॥ कृष्ण यजुर्वेदीय तैत्तिरीय ब्राह्मणे द्वितीयाष्टके सप्तमः प्रपाठकः समाप्तः ॥

Appendix (of Expansions)
ट्.भ्.2.7.8.1 "अ॒ङ्कौ न्य॒ङ्काव॒भित॒{27} 
अ॒ङ्कौन्य॒ङ्का व॒भितो॒ रथं॒ ॅयौ ध्वा॒न्तं ॅवा॑ता॒ग्रमनु॑ स॒चंर॑न्तौ 
दू॒रे हे॑तिरिन्द्रि॒यावा᳚न् पत॒त्री ते नो॒ऽग्नयः॒ पप्र॑यः पारयन्तु ॥ {27}
(Appearing in T.S. 1.7.7.2), 
(AlsO samE appearing abovE, in T.B.2.7.16.1 ) \newline
        \pagebreak
        
        
        

\end{document}
