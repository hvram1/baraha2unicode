\documentclass[17pt]{extarticle}
\usepackage{babel}
\usepackage{fontspec}
\usepackage{polyglossia}
\usepackage{extsizes}



\setmainlanguage{sanskrit}
\setotherlanguages{english} %% or other languages
\setlength{\parindent}{0pt}
\pagestyle{myheadings}
\newfontfamily\devanagarifont[Script=Devanagari]{AdishilaVedic}


\newcommand{\VAR}[1]{}
\newcommand{\BLOCK}[1]{}




\begin{document}
\begin{titlepage}
    \begin{center}
 
\begin{sanskrit}
    { \Large
    ॐ नमः परमात्मने, श्री महागणपतये नमः, श्री गुरुभ्यो नमः
ह॒रिः॒ ॐ 
    }
    \\
    \vspace{2.5cm}
    \mbox{ \Huge
    6.3      षष्ठकाण्डे तृतीयः प्रश्नः - सोममन्त्रब्राह्मणनिरूपणं   }
\end{sanskrit}
\end{center}

\end{titlepage}
\tableofcontents

ॐ नमः परमात्मने, श्री महागणपतये नमः, श्री गुरुभ्यो नमः
ह॒रिः॒ ॐ \newline
6.3      षष्ठकाण्डे तृतीयः प्रश्नः - सोममन्त्रब्राह्मणनिरूपणं \newline

\addcontentsline{toc}{section}{ 6.3      षष्ठकाण्डे तृतीयः प्रश्नः - सोममन्त्रब्राह्मणनिरूपणं}
\markright{ 6.3      षष्ठकाण्डे तृतीयः प्रश्नः - सोममन्त्रब्राह्मणनिरूपणं \hfill https://www.vedavms.in \hfill}
\section*{ 6.3      षष्ठकाण्डे तृतीयः प्रश्नः - सोममन्त्रब्राह्मणनिरूपणं }
                                \textbf{ TS 6.3.1.1} \newline
                  चात्वा॑लात् । धिष्णि॑यान् । उपेति॑ । व॒प॒ति॒ । योनिः॑ । वै । य॒ज्ञ्स्य॑ । चात्वा॑लम् । य॒ज्ञ्स्य॑ । स॒यो॒नि॒त्वायेति॑ सयोनि - त्वाय॑ । दे॒वाः । वै । य॒ज्ञ्म् । परेति॑ । अ॒ज॒य॒न्त॒ । तम् । आग्नी᳚᳚द्ध्रा॒दित्याग्नि॑ - इ॒द्ध्रा॒त् । पुनः॑ । अपेति॑ । अ॒ज॒य॒न्न् । ए॒तत् । वै । य॒ज्ञ्स्य॑ । अप॑राजित॒मित्यप॑रा - जि॒त॒म् । यत् । आग्नी᳚द्ध्र॒मित्याग्नि॑ - इ॒द्ध्र॒म् । यत् । आग्नी᳚द्ध्रा॒दित्याग्नि॑ - इ॒द्ध्रा॒त् । धिष्णि॑यान् । वि॒हर॒तीति॑ वि-हर॑ति । यत् । ए॒व । य॒ज्ञ्स्य॑ । अप॑राजित॒मित्यप॑रा - जि॒त॒म् । ततः॑ । ए॒व । ए॒न॒म् । पुनः॑ । त॒नु॒ते॒ । प॒रा॒जित्येति॑ परा-जित्य॑ । इ॒व॒ । खलु॑ । वै । ए॒ते । य॒न्ति॒ । ये । ब॒हि॒ष्प॒व॒मा॒नमिति॑ बहिः - प॒व॒मा॒नम् । सर्प॑न्ति । ब॒हि॒ष्प॒व॒मा॒न इति॑ बहिः - प॒व॒मा॒ने । स्तु॒ते । \textbf{  1} \newline
                  \newline
                                \textbf{ TS 6.3.1.2} \newline
                  आ॒ह॒ । अग्नी॒दित्यग्नि॑ - इ॒त् । अ॒ग्नीन् । वीति॑ । ह॒र॒ । ब॒र॒.हिः । स्तृ॒णा॒हि॒ । पु॒रो॒डाशान्॑ । अल᳚म् । कु॒रु॒ । इति॑ । य॒ज्ञ्म् । ए॒व । अ॒प॒जित्येत्य॑प-जित्य॑ । पुनः॑ । त॒न्वा॒नाः । य॒न्ति॒ । अङ्गा॑रैः । द्वे इति॑ । सव॑ने॒ इति॑ । वीति॑ । ह॒र॒ति॒ । श॒लाका॑भिः । तृ॒तीय᳚म् । स॒शु॒क्र॒त्वायेति॑ सशुक्र - त्वाय॑ । अथो॒ इति॑ । समिति॑ । भ॒र॒ति॒ । ए॒व । ए॒न॒त् । धिष्णि॑याः । वै । अ॒मुष्मिन्न्॑ । लो॒के । सोम᳚म् । अ॒र॒क्ष॒न्न् । तेभ्यः॑ । अधीति॑ । सोम᳚म् । एति॑ । अ॒ह॒र॒न्न् । तम् । अ॒न्व॒वाय॒न्नित्य॑नु-अ॒वायन्न्॑ । तम् । परीति॑ । अ॒वि॒श॒न्न् । य । ए॒वम् । वेद॑ । वि॒न्दते᳚ । \textbf{  2} \newline
                  \newline
                                \textbf{ TS 6.3.1.3} \newline
                  प॒रि॒वे॒ष्टार॒मिति॑ परि - वे॒ष्टार᳚म् । ते । सो॒म॒पी॒थेनेति॑ सोम - पी॒थेन॑ । वीति॑ । आ॒द्‌र्ध्य॒न्त॒ । ते । दे॒वेषु॑ । सो॒म॒पी॒थमिति॑ सोम - पी॒थम् । ऐ॒च्छ॒न्त॒ । तान् । दे॒वाः । अ॒ब्रु॒व॒न्न् । द्वे द्वे॒ इति॒ द्वे-द्वे॒ । नाम॑नी॒ इति॑ । कु॒रु॒द्ध्व॒म् । अथ॑ । प्रेति॑ । वा॒ । आ॒फ्स्यथ॑ । न । वा॒ । इति॑ । अ॒ग्नयः॑ । वै । अथ॑ । धिष्णि॑याः । तस्मा᳚त् । द्वि॒नामेति॑ द्वि - नामा᳚ । ब्रा॒ह्म॒णः । अद्‌र्धु॑कः । तेषा᳚म् । ये । नेदि॑ष्ठम् । प॒र्यवि॑श॒निति॑ परि - अवि॑शन्न् । ते । सो॒म॒पी॒थमिति॑ सोम - पी॒थम् । प्रेति॑ । आ॒प्नु॒व॒न्न् । आ॒ह॒व॒नीय॒ इत्या᳚ - ह॒व॒नीयः॑ । आ॒ग्नी॒द्ध्रीय॒ इत्या᳚ग्नि - इ॒ध्रीयः॑ । हो॒त्रीयः॑ । मा॒र्जा॒लीयः॑ । तस्मा᳚त् । तेषु॑ । जु॒ह्व॒ति॒ । अ॒ति॒हायेत्य॑ति-हाय॑ । वष॑ट् । क॒रो॒ति॒ । वीति॑ । हि । \textbf{  3} \newline
                  \newline
                                \textbf{ TS 6.3.1.4} \newline
                  ए॒ते । सो॒म॒पी॒थेनेति॑ सोम - पी॒थेन॑ । आद्‌र्ध्य॑न्त । दे॒वाः । वै । याः । प्राचीः᳚ । आहु॑ती॒रित्या - हु॒तीः॒ । अजु॑हवुः । ये । पु॒रस्ता᳚त् । असु॑राः । आसन्न्॑ । तान् । ताभिः॑ । प्रेति॑ । अ॒नु॒द॒न्त॒ । याः । प्र॒तीचीः᳚ । ये । प॒श्चात् । असु॑राः । आसन्न्॑ । तान् । ताभिः॑ । अपेति॑ । अ॒नु॒द॒न्त॒ । प्राचीः᳚ । अ॒न्याः । आहु॑तय॒ इत्या - हु॒त॒यः॒ । हू॒यन्ते᳚ । प्र॒त्यङ् । आसी॑नः । धिष्णि॑यान् । व्याघा॑रय॒तीति॑ वि - आघा॑रयति । प॒श्चात् । च॒ । ए॒व । पु॒रस्ता᳚त् । च॒ । यज॑मानः । भ्रातृ॑व्यान् । प्रेति॑ । नु॒द॒ते॒ । तस्मा᳚त् । परा॑चीः । प्र॒जा इति॑ प्र-जाः । प्रेति॑ । वी॒य॒न्ते॒ । प्र॒तीचीः᳚ । \textbf{  4} \newline
                  \newline
                                \textbf{ TS 6.3.1.5} \newline
                  जा॒य॒न्ते॒ । प्रा॒णा इति॑ प्र-अ॒नाः । वै । ए॒ते । यत् । धिष्णि॑याः । यत् । अ॒द्ध्व॒र्युः । प्र॒त्यङ् । धिष्णि॑यान् । अ॒ति॒सर्पे॒दित्य॑ति - सर्पे᳚त् । प्रा॒णानिति॑ प्र - अ॒नान् । समिति॑ । क॒र्॒.षे॒त् । प्र॒मायु॑क॒ इति॑ प्र - मायु॑कः । स्या॒त् । नाभिः॑ । वै । ए॒षा । य॒ज्ञ्स्य॑ । यत् । होता᳚ । ऊ॒द्‌र्ध्वः । खलु॑ । वै । नाभ्यै᳚ । प्रा॒ण इति॑ प्र - अ॒नः । अवाङ्॑ । अ॒पा॒न इत्य॑प - अ॒नः । यत् । अ॒द्ध्व॒र्युः । प्र॒त्यङ् । होता॑रम् । अ॒ति॒सर्पे॒दित्य॑ति - सर्पे᳚त् । अ॒पा॒न इत्य॑प - अ॒ने । प्रा॒णमिति॑ प्र - अ॒नम् । द॒द्ध्या॒त् । प्र॒मायु॑क॒ इति॑ प्र - मायु॑कः । स्या॒त् । न । अ॒द्ध्व॒र्युः । उपेति॑ । गा॒ये॒त् । वाग्वी᳚र्य॒ इति॒ वाक् - वी॒र्यः॒ । वै । अ॒द्ध्व॒र्युः । यत् । अ॒द्ध्व॒र्युः । उ॒प॒गाये॒दित्यु॑प - गाये᳚त् । उ॒द्गा॒त्र इत्यु॑त् - गा॒त्रे । \textbf{  5} \newline
                  \newline
                                \textbf{ TS 6.3.1.6} \newline
                  वाच᳚म् । सम् । प्रेति॑ । य॒च्छे॒त् । उ॒प॒दासु॒केत्यु॑प-दासु॑का । अ॒स्य॒ । वाक् । स्या॒त् । ब्र॒ह्म॒वा॒दिन॒ इति॑ ब्रह्म - वा॒दिनः॑ । व॒द॒न्ति॒ । न । असꣳ॑स्थित॒ इत्यसं᳚ - स्थि॒ते॒ । सोमे᳚ । अ॒द्ध्व॒र्युः । प्र॒त्यङ् । सदः॑ । अतीति॑ । इ॒या॒त् । अथ॑ । क॒था । दा॒क्षि॒णानि॑ । होतु᳚म् । ए॒ति॒ । यामः॑ । हि । सः । तेषा᳚म् । कस्मै᳚ । अह॑ । दे॒वाः । याम᳚म् । वा॒ । अया॑मम् । वा॒ । अन्विति॑ । ज्ञा॒स्य॒न्ति॒ । इति॑ । उत्त॑रे॒णेत्युत्- त॒रे॒ण॒ । आग्नी᳚द्ध्र॒मित्याग्नि॑ - इ॒द्ध्र॒म् । प॒रीत्येति॑ परि - इत्य॑ । जु॒हो॒ति॒ । दा॒क्षि॒णानि॑ । न । प्रा॒णानिति॑ प्र - अ॒नान् । समिति॑ । क॒र्॒.ष॒ति॒ । नीति॑ । अ॒न्ये । धिष्णि॑याः । उ॒प्यन्ते᳚ ( ) । न । अ॒न्ये । यान् । नि॒वप॒तीति॑ नि - वप॑ति । तेन॑ । तान् । प्री॒णा॒ति॒ । यान् । न । नि॒वप॒तीति॑ नि - वप॑ति । यत् । अ॒नु॒दि॒शतीत्य॑नु - दि॒शति॑ । तेन॑ । तान् ॥ \textbf{  6 } \newline
                  \newline
                      (स्तु॒ते - वि॒न्दते॒ - हि - वी॑यन्ते प्र॒तीची॑ - रुद्ग्रा॒त्र - उ॒प्यन्ते॒ - चतु॑र्दश च)  \textbf{(A1)} \newline \newline
                                \textbf{ TS 6.3.2.1} \newline
                  सु॒व॒र्गायेति॑ सुवः - गाय॑ । वै । ए॒तानि॑ । लो॒काय॑ । हू॒य॒न्ते॒ । यत् । वै॒स॒र्ज॒नानि॑ । द्वाभ्या᳚म् । गार्.ह॑पत्य॒ इति॒ गार्.ह॑ - प॒त्ये॒ । जु॒हो॒ति॒ । द्वि॒पादिति॑ द्वि - पात् । यज॑मानः । प्रति॑ष्ठित्या॒ इति॒ प्रति॑ - स्थि॒त्यै॒ । आग्नी᳚द्ध्र॒ इत्याग्नि॑-इ॒द्ध्रे॒ । जु॒हो॒ति॒ । अ॒न्तरि॑क्षे । ए॒व । एति॑ । क्र॒म॒ते॒ । आ॒ह॒व॒नीय॒ इत्या᳚ - ह॒व॒नीये᳚ । जु॒हो॒ति॒ । सु॒व॒र्गमिति॑ सुवः - गम् । ए॒व । ए॒न॒म् । लो॒कम् । ग॒म॒य॒ति॒ । दे॒वान् । वै । सु॒व॒र्गमिति॑ सुवः - गम् । लो॒कम् । य॒तः । रक्षाꣳ॑सि । अ॒जि॒घाꣳ॒॒स॒न्न् । ते । सोमे॑न । राज्ञा᳚ । रक्षाꣳ॑सि । अ॒प॒हत्येत्य॑प - हत्य॑ । अ॒प्तुम् । आ॒त्मान᳚म् । कृ॒त्वा । सु॒व॒र्गमिति॑ सुवः - गम् । लो॒कम् । आ॒य॒न्न् । रक्ष॑साम् । अनु॑पलाभा॒येत्यनु॑प - ला॒भा॒य॒ । आत्तः॑ । सोमः॑ । भ॒व॒ति॒ । अथ॑ । \textbf{  7} \newline
                  \newline
                                \textbf{ TS 6.3.2.2} \newline
                  वै॒स॒र्ज॒नानि॑ । जु॒हो॒ति॒ । रक्ष॑साम् । अप॑हत्या॒ इत्यप॑ - ह॒त्यै॒ । त्वम् । सो॒म॒ । त॒नू॒कृद्भ्य॒ इति॑ तनू॒कृत् - भ्यः॒ । इति॑ । आ॒ह॒ । त॒नू॒कृदिति॑ तनू - कृत् । हि । ए॒षः । द्वेषो᳚भ्य॒ इति॒ द्वेषः॑ - भ्यः॒ । अ॒न्यकृ॑तेभ्य॒ इत्य॒न्य - कृ॒ते॒भ्यः॒ । इति॑ । आ॒ह॒ । अ॒न्यकृ॑ता॒नीत्य॒न्य - कृ॒ता॒नि॒ । हि । रक्षाꣳ॑सि । उ॒रु । य॒न्ता । अ॒सि॒ । वरू॑थम् । इति॑ । आ॒ह॒ । उ॒रु । नः॒ । कृ॒धि॒ । इति॑ । वाव । ए॒तत् । आ॒ह॒ । जु॒षा॒णः । अ॒प्तुः । आज्य॑स्य । वे॒तु॒ । इति॑ । आ॒ह॒ । अ॒प्तुम् । ए॒व । यज॑मानम् । कृ॒त्वा । सु॒व॒र्गमिति॑ सुवः - गम् । लो॒कम् । ग॒म॒य॒ति॒ । रक्ष॑साम् । अनु॑पलाभा॒येत्यनु॑प - ला॒भा॒य॒ । एति॑ । सोम᳚म् । द॒द॒ते॒ । \textbf{  8} \newline
                  \newline
                                \textbf{ TS 6.3.2.3} \newline
                  एति॑ । ग्राव्‌ण्णः॑ । एति॑ । वा॒य॒व्या॑नि । एति॑ । द्रो॒ण॒क॒ल॒शमिति॑ द्रोण - क॒ल॒शम् । उदिति॑ । पत्नी᳚म् । एति॑ । न॒य॒न्ति॒ । अन्विति॑ । अनाꣳ॑सि । प्रेति॑ । व॒र्त॒य॒न्ति॒ । याव॑त् । ए॒व । अ॒स्य॒ । अस्ति॑ । तेन॑ । स॒ह । सु॒व॒र्गमिति॑ सुवः - गम् । लो॒कम् । ए॒ति॒ । नय॑व॒त्येति॒ नय॑ - व॒त्या॒ । ऋ॒चा । आग्नी᳚द्ध्र॒ इत्याग्नि॑ - इ॒द्ध्रे॒ । जु॒हो॒ति॒ । सु॒व॒र्गस्येति॑ सुवः - गस्य॑ । लो॒कस्य॑ । अ॒भिनी᳚त्या॒ इत्य॒भि-नी॒त्यै॒ । ग्राव्‌ण्णः॑ । वा॒य॒व्या॑नि । द्रो॒ण॒क॒ल॒शमिति॑ द्रोण-क॒ल॒शम् । आग्नी᳚द्ध्र॒ इत्याग्नि॑ - इ॒द्ध्रे॒ । उपेति॑ । वा॒स॒य॒ति॒ । वीति॑ । हि । ए॒न॒म् । तैः । गृ॒ह्णते᳚ । यत् । स॒ह । उ॒प॒वा॒सये॒दित्यु॑प - वा॒सये᳚त् । अ॒पु॒वा॒येत॑ । सौ॒म्य । ऋ॒चा । प्रेति॑ । पा॒द॒य॒ति॒ । स्वया᳚ । \textbf{  9} \newline
                  \newline
                                \textbf{ TS 6.3.2.4} \newline
                  ए॒व । ए॒न॒म् । दे॒वत॑या । प्रेति॑ । पा॒द॒य॒ति॒ । अदि॑त्याः । सदः॑ । अ॒सि॒ । अदि॑त्याः । सदः॑ । एति॑ । सी॒द॒ । इति॑ । आ॒ह॒ । य॒था॒य॒जुरिति॑ यथा-य॒जुः । ए॒व । ए॒तत् । यज॑मानः । वै । ए॒तस्य॑ । पु॒रा । गो॒प्ता । भ॒व॒ति॒ । ए॒षः । वः॒ । दे॒व॒ । स॒वि॒तः॒ । सोमः॑ । इति॑ । आ॒ह॒ । स॒वि॒तृप्र॑सूत॒ इति॑ सवि॒तृ-प्र॒सू॒तः॒ । ए॒व । ए॒न॒म् । दे॒वता᳚भ्यः । सम् । प्रेति॑ । य॒च्छ॒ति॒ । ए॒तत् । त्वम् । सो॒म॒ । दे॒वः । दे॒वान् । उपेति॑ । अ॒गाः॒ । इति॑ । आ॒ह॒ । दे॒वः । हि । ए॒षः । सन्न् । \textbf{  10} \newline
                  \newline
                                \textbf{ TS 6.3.2.5} \newline
                  दे॒वान् । उ॒पैतीत्यु॑प - एति॑ । इ॒दम् । अ॒हम् । म॒नु॒ष्यः॑ । म॒नु॒ष्यान्॑ । इति॑ । आ॒ह॒ । म॒नु॒ष्यः॑ । हि । ए॒षः । सन्न् । म॒नु॒ष्यान्॑ । उ॒पैतीत्यु॑प - एति॑ । यत् । ए॒तत् । यजुः॑ । न । ब्रू॒यात् । अप्र॑जा॒ इत्यप्र॑ - जाः॒ । अ॒प॒शुः । यज॑मानः । स्या॒त् । स॒ह । प्र॒जयेति॑ प्र - जया᳚ । स॒ह । रा॒यः । पोषे॑ण । इति॑ । आ॒ह॒ । प्र॒जयेति॑ प्र - जया᳚ । ए॒व । प॒शुभि॒रिति॑ प॒शु-भिः॒ । स॒ह । इ॒मम् । लो॒कम् । उ॒पाव॑र्तत॒ इत्यु॑प - आव॑र्तते । नमः॑ । दे॒वेभ्यः॑ । इति॑ । आ॒ह॒ । न॒म॒स्का॒र इति॑ नमः - का॒रः । हि । दे॒वाना᳚म् । स्व॒धेति॑ स्व - धा । पि॒तृभ्य॒ इति॑ पि॒तृ - भ्यः॒ । इति॑ । आ॒ह॒ । स्व॒धा॒का॒र इति॑ स्वधा - का॒रः । हि । \textbf{  11} \newline
                  \newline
                                \textbf{ TS 6.3.2.6} \newline
                  पि॒तृ॒णाम् । इ॒दम् । अ॒हम् । निरिति॑ । वरु॑णस्य । पाशा᳚त् । इति॑ । आ॒ह॒ । व॒रु॒ण॒पा॒शादिति॑ वरुण - पा॒शात् । ए॒व । निरिति॑ । मु॒च्य॒ते॒ । अग्ने᳚ । व्र॒त॒प॒त॒ इति॑ व्रत - प॒ते॒ । आ॒त्मनः॑ । पूर्वा᳚ । त॒नूः । आ॒देयेत्या᳚ - देया᳚ । इति॑ । आ॒हुः॒ । कः । हि । तत् । वेद॑ । यत् । वसी॑यान् । स्वे । वशे᳚ । भू॒ते । पुनः॑ । वा॒ । ददा॑ति । न । वा॒ । इति॑ । ग्रावा॑णः । वै । सोम॑स्य । राज्ञ्ः॑ । म॒लि॒म्लु॒से॒नेति॑ मलिम्लु - से॒ना । यः । ए॒वम् । वि॒द्वान् । ग्राव्‌ण्णः॑  । आग्नी᳚द्ध्र॒ इत्याग्नि॑ - इ॒द्ध्रे॒ । उ॒प॒वा॒सय॒तीत्यु॑प - वा॒सय॑ति । न । ए॒न॒म् । म॒लि॒म्लु॒से॒नेति॑ मलिम्लु - से॒ना । वि॒न्द॒ति॒ ( ) ॥ \textbf{  12 } \newline
                  \newline
                      (अथ॑-ददते॒ - स्वया॒ - सन्थ् - स्व॑धाका॒रो हि - वि॑न्दति)  \textbf{(A2)} \newline \newline
                                \textbf{ TS 6.3.3.1} \newline
                  वै॒ष्ण॒व्या । ऋ॒चा । हु॒त्वा । यूप᳚म् । अच्छ॑ । ए॒ति॒ । वै॒ष्ण॒वः । वै । दे॒वत॑या । यूपः॑ । स्वया᳚ । ए॒व । ए॒न॒म् । दे॒वत॑या । अच्छ॑ । ए॒ति॒ । अतीति॑ । अ॒न्यान् । अगा᳚म् । न । अ॒न्यान् । उपेति॑ । अ॒गा॒म् । इति॑ । आ॒ह॒ । अतीति॑ । हि । अ॒न्यान् । एति॑ । न । अ॒न्यान् । उ॒पैतीत्यु॑प-एति॑ । अ॒र्वाक् । त्वा॒ । परैः᳚ । अ॒वि॒द॒म् । प॒रः । अव॑रैः । इति॑ । आ॒ह॒ । अ॒र्वाक् । हि । ए॒न॒म् । परैः᳚ । वि॒न्दति॑ । प॒रः । अव॑रैः । तम् । त्वा॒ । जु॒षे॒ । \textbf{  13} \newline
                  \newline
                                \textbf{ TS 6.3.3.2} \newline
                  वै॒ष्ण॒वम् । दे॒व॒य॒ज्याया॒ इति॑ देव - य॒ज्यायै᳚ । इति॑ । आ॒ह॒ । दे॒व॒य॒ज्याया॒ इति॑ देव-य॒ज्यायै᳚ । हि । ए॒न॒म् । जु॒षते᳚ । दे॒वः । त्वा॒ । स॒वि॒ता । मद्ध्वा᳚ । अ॒न॒क्तु॒ । इति॑ । आ॒ह॒ । तेज॑सा । ए॒व । ए॒न॒म् । अ॒न॒क्ति॒ । ओष॑धे । त्राय॑स्व । ए॒न॒म् । स्वधि॑त॒ इति॒ स्व-धि॒ते॒ । मा । ए॒न॒म् । हिꣳ॒॒सीः॒ । इति॑ । आ॒ह॒ । वज्रः॑ । वै । स्वधि॑ति॒रिति॒ स्व - धि॒तिः॒ । शान्त्यै᳚ । स्वधि॑ते॒रिति॒ स्व - धि॒तेः॒ । वृ॒क्षस्य॑ । बिभ्य॑तः । प्र॒थ॒मेन॑ । शक॑लेन । स॒ह । तेजः॑ । परेति॑ । प॒त॒ति॒ । यः । प्र॒थ॒मः । शक॑लः । प॒रा॒पते॒दिति॑ परा - पते᳚त् । तम् । अपि॑ । एति॑ । ह॒रे॒त् । सते॑जस॒मिति॒ स - ते॒ज॒स॒म् । \textbf{  14} \newline
                  \newline
                                \textbf{ TS 6.3.3.3} \newline
                  ए॒व । ए॒न॒म् । एति॑ । ह॒र॒ति॒ । इ॒मे । वै । लो॒काः । यूपा᳚त् । प्र॒य॒त इति॑ प्र - य॒तः । बि॒भ्य॒ति॒ । दिव᳚म् । अग्रे॑ण । मा । ले॒खीः॒ । अ॒न्तरि॑क्षम् । मद्ध्ये॑न । मा । हिꣳ॒॒सीः॒ । इति॑ । आ॒ह॒ । ए॒भ्यः । ए॒व । ए॒न॒म् । लो॒केभ्यः॑ । श॒म॒य॒ति॒ । वन॑स्पते । श॒तव॑ल्.श॒ इति॑ श॒त - व॒ल॒.शः॒ । वीति॑ । रो॒ह॒ । इति॑ । आ॒व्रश्च॑न॒ इत्या᳚ - व्रश्च॑ने । जु॒हो॒ति॒ । तस्मा᳚त् । आ॒व्रश्च॑ना॒दित्या᳚ - व्रश्च॑नात् । वृ॒क्षाणा᳚म् । भूयाꣳ॑सः । उदिति॑ । ति॒ष्ठ॒न्ति॒ । स॒हस्र॑वल्.शा॒ इति॑ स॒हस्र॑ - व॒ल॒.शाः॒ । वीति॑ । व॒यम् । रु॒हे॒म॒ । इति॑ । आ॒ह॒ । आ॒शिष॒मित्या᳚-शिष᳚म् । ए॒व । ए॒ताम् । एति॑ । शा॒स्ते॒ । अन॑क्षसङ्ग॒मित्यन॑क्ष - स॒ङ्गम् । \textbf{  15} \newline
                  \newline
                                \textbf{ TS 6.3.3.4} \newline
                  वृ॒श्चे॒त् । यत् । अ॒क्ष॒स॒ङ्गमित्य॑क्ष - स॒ङ्गम् । वृ॒श्चेत् । अ॒ध॒ई॒षमित्य॑धः - ई॒षम् । यज॑मानस्य । प्र॒मायु॑क॒मिति॑ प्र-मायु॑कम् । स्या॒त् । यम् । का॒मये॑त । अप्र॑तिष्ठित॒ इत्यप्र॑ति - स्थि॒तः॒ । स्या॒त् । इति॑ । आ॒रो॒हमित्या᳚ - रो॒हम् । तस्मै᳚ । वृ॒श्चे॒त् । ए॒षः । वै । वन॒स्पती॑नाम् । अप्र॑तिष्ठित॒ इत्यप्र॑ति - स्थि॒तः॒ । अप्र॑तिष्ठित॒ इत्यप्र॑ति - स्थि॒तः॒ । ए॒व । भ॒व॒ति॒ । यम् । का॒मये॑त । अ॒प॒शुः । स्या॒त् । इति॑ । अ॒प॒र्णम् । तस्मै᳚ । शुष्का᳚ग्र॒मिति॒ शुष्क॑ - अ॒ग्र॒म् । वृ॒श्चे॒त् । ए॒षः । वै । वन॒स्पती॑नाम् । अ॒प॒श॒व्यः । अ॒प॒शुः । ए॒व । भ॒व॒ति॒ । यम् । का॒मये॑त । प॒शु॒मानिति॑ पशु - मान् । स्या॒त् । इति॑ । ब॒हु॒प॒र्णमिति॑ बहु - प॒र्णम् । तस्मै᳚ । ब॒हु॒शा॒खमिति॑ बहु - शा॒खम् । वृ॒श्चे॒त् । ए॒षः । वै । \textbf{  16} \newline
                  \newline
                                \textbf{ TS 6.3.3.5} \newline
                  वन॒स्पती॑नाम् । प॒श॒व्यः॑ । प॒शु॒मानिति॑ पशु - मान् । ए॒व । भ॒व॒ति॒ । प्रति॑ष्ठित॒मिति॒ प्रति॑ - स्थि॒त॒म् । वृ॒श्चे॒त् । प्र॒ति॒ष्ठाका॑म॒स्येति॑ प्रति॒ष्ठा- का॒म॒स्य॒ । ए॒षः । वै । वन॒स्पती॑नाम् । प्रति॑ष्ठित॒ इति॒ प्रति॑ - स्थि॒तः॒ । यः । स॒मे । भूम्यै᳚ । स्वात् । योनेः᳚ । रू॒ढः । प्रतीति॑ । ए॒व । ति॒ष्ठ॒ति॒ । यः । प्र॒त्यङ् । उप॑नत॒ इत्युप॑-न॒तः॒ । तम् । वृ॒श्चे॒त् । सः । हि । मेध᳚म् । अ॒भीति॑ । उप॑नत॒ इत्युप॑ - न॒तः॒ । पञ्चा॑रत्नि॒मिति॒ पञ्च॑- अ॒र॒त्नि॒म् । तस्मै᳚ । वृ॒श्चे॒त् । यम् । का॒मये॑त । उपेति॑ । ए॒न॒म् । उत्त॑र॒ इत्युत् - त॒रः॒ । य॒ज्ञ्ः । न॒मे॒त् । इति॑ । पञ्चा᳚क्ष॒रेति॒ पञ्च॑-आ॒क्ष॒रा॒ । प॒ङ्क्तिः । पाङ्क्तः॑ । य॒ज्ञ्ः । उपेति॑ । ए॒न॒म् । उत्त॑र॒ इत्युत् - त॒रः॒ । य॒ज्ञ्ः । \textbf{  17} \newline
                  \newline
                                \textbf{ TS 6.3.3.6} \newline
                  न॒म॒ति॒ । षड॑रत्नि॒मिति॒ षट् - अ॒र॒त्नि॒म् । प्र॒ति॒ष्ठाका॑म॒स्येति॑ प्रति॒ष्ठा- का॒म॒स्य॒ । षट् । वै । ऋ॒तवः॑ । ऋ॒तुषु॑ । ए॒व । प्रतीति॑ । ति॒ष्ठ॒ति॒ । स॒प्तार॑त्नि॒मिति॑ स॒प्त - अ॒र॒त्नि॒म् । प॒शुका॑म॒स्येति॑ प॒शु - का॒म॒स्य॒ । स॒प्तप॒देति॑ स॒प्त-प॒दा॒ । शक्व॑री । प॒शवः॑ । शक्व॑री । प॒शून् । ए॒व । अवेति॑ । रु॒न्धे॒ । नवा॑रत्नि॒मिति॒ नव॑ - अ॒र॒त्नि॒म् । तेज॑स्काम॒स्येति॒ तेजः॑ - का॒म॒स्य॒ । त्रि॒वृतेति॑ त्रि - वृता᳚ । स्तोमे॑न । सम्मि॑त॒मिति॒ सं - मि॒त॒म् । तेजः॑ । त्रि॒वृदिति॑ त्रि - वृत् । ते॒ज॒स्वी । ए॒व । भ॒व॒ति॒ । एका॑दशारत्नि॒मित्येका॑दश-अ॒र॒त्नि॒म् । इ॒न्द्रि॒यका॑म॒स्येती᳚न्द्रि॒य-का॒म॒स्य॒ । एका॑दशाक्ष॒रेत्येका॑दश - अ॒क्ष॒रा॒ । त्रि॒ष्टुक् । इ॒न्द्रि॒यम् । त्रि॒ष्टुक् । इ॒न्द्रि॒या॒वी । ए॒व । भ॒व॒ति॒ । पञ्च॑दशारत्नि॒मिति॒ पञ्च॑दश - अ॒र॒त्नि॒म् । भ्रातृ॑व्यवत॒ इति॒ भ्रातृ॑व्य - व॒तः॒ । प॒ञ्च॒द॒श इति॑ पञ्च - द॒शः । वज्रः॑ । भ्रातृ॑व्याभिभूत्या॒ इति॒ भ्रातृ॑व्य - अ॒भि॒भू॒त्यै॒ । स॒प्तद॑शारत्नि॒मिति॑ स॒प्तद॑श - अ॒र॒त्नि॒म् । प्र॒जाका॑म॒स्येति॑ प्र॒जा - का॒म॒स्य॒ । स॒प्त॒द॒श इति॑ सप्त - द॒शः । प्र॒जाप॑ति॒रिति॑ प्र॒जा-प॒तिः॒ । प्र॒जाप॑ते॒रिति॑ प्र॒जा - प॒तेः॒ । आप्त्यै᳚ ( ) । एक॑विꣳशत्यरत्नि॒मित्येक॑विꣳशति - अ॒र॒त्नि॒म् । प्र॒ति॒ष्ठाका॑म॒स्येति॑ प्रति॒ष्ठा - का॒म॒स्य॒ । ए॒क॒विꣳ॒॒श इत्ये॑क - विꣳ॒॒शः । स्तोमा॑नाम् । प्र॒ति॒ष्ठेति॑ प्रति - स्था । प्रति॑ष्ठित्या॒ इति॒ प्रति॑ - स्थि॒त्यै॒ । अ॒ष्टाश्रि॒रित्य॒ष्टा - अ॒श्रिः॒ । भ॒व॒ति॒ । अ॒ष्टाक्ष॒रेत्य॒ष्टा - अ॒क्ष॒रा॒ । गा॒य॒त्री । तेजः॑ । गा॒य॒त्री । गा॒य॒त्री । य॒ज्ञ्॒मु॒खमिति॑ यज्ञ् - मु॒खम् । तेज॑सा । ए॒व । गा॒य॒त्रि॒या । य॒ज्ञ्॒मु॒खेनेति॑ यज्ञ् - मु॒खेन॑ । संमि॑त॒ इति॒ सं - मि॒तः॒ ॥ \textbf{  18 } \newline
                  \newline
                      (जु॒षे॒ - सते॑जस॒ - मन॑क्षसङ्गं - बहुशा॒खं ॅवृ॑श्चेदे॒ष वै - य॒ज्ञ् उपै॑न॒मुत्त॑रो य॒ज्ञ् - आप्त्या॒ - एका॒न्नविꣳ॑श॒तिश्च॑)  \textbf{(A3)} \newline \newline
                                \textbf{ TS 6.3.4.1} \newline
                  पृ॒थि॒व्यै । त्वा॒ । अ॒न्तरि॑क्षाय । त्वा॒ । दि॒वे । त्वा॒ । इति॑ । आ॒ह॒ । ए॒भ्यः । ए॒व । ए॒न॒म् । लो॒केभ्यः॑ । प्रेति॑ । उ॒क्ष॒ति॒ । परा᳚ञ्चम् । प्रेति॑ । उ॒क्ष॒ति॒ । पराङ्॑ । इ॒व॒ । हि । सु॒व॒र्ग इति॑ सुवः-गः । लो॒कः । क्रू॒रम् । इ॒व॒ । वै । ए॒तत् । क॒रो॒ति॒ । यत् । खन॑ति । अ॒पः । अवेति॑ । न॒य॒ति॒ । शान्त्यै᳚ । यव॑मती॒रिति॒ यव॑-म॒तीः॒ । अवेति॑ । न॒य॒ति॒ । ऊर्क् । वै । यवः॑ । यज॑मानेन । यूपः॑ । संमि॑त॒ इति॒ सं - मि॒तः॒ । यावान्॑ । ए॒व । यज॑मानः । ताव॑तीम् । ए॒व । अ॒स्मि॒न्न् । ऊर्ज᳚म् । द॒धा॒ति॒ । \textbf{  19} \newline
                  \newline
                                \textbf{ TS 6.3.4.2} \newline
                  पि॒तृ॒णाम् । सद॑नम् । अ॒सि॒ । इति॑ । ब॒र्॒.हिः । अवेति॑ । स्तृ॒णा॒ति॒ । पि॒तृ॒दे॒व॒त्य॑मिति॑ पितृ - दे॒व॒त्य᳚म् । हि । ए॒तत् । यत् । निखा॑त॒मिति॒ नि - खा॒त॒म् । यत् । ब॒र॒.हिः । अन॑वस्ती॒र्येत्यन॑व - स्ती॒र्य॒ । मि॒नु॒यात् । पि॒तृ॒दे॒वत्य॑ इति॑ पितृ-दे॒व॒त्यः॑ । निखा॑त॒ इति॒ नि-खा॒तः॒ । स्या॒त् । ब॒र॒.हिः । अ॒व॒स्तीर्येत्य॑व - स्तीर्य॑ । मि॒नो॒ति॒ । अ॒स्याम् । ए॒व । ए॒न॒म् । मि॒नो॒ति॒ । यू॒प॒श॒क॒लमिति॑ यूप - श॒क॒लम् । अवेति॑ । अ॒स्य॒ति॒ । सते॑जस॒मिति॒ स - ते॒ज॒स॒म् । ए॒व । ए॒न॒म् । मि॒नो॒ति॒ । दे॒वः । त्वा॒ । स॒वि॒ता । मद्ध्वा᳚ । अ॒न॒क्तु॒ । इति॑ । आ॒ह॒ । तेज॑सा । ए॒व । ए॒न॒म् । अ॒न॒क्ति॒ । सु॒पि॒प्प॒लाभ्य॒ इति॑ सु-पि॒प्प॒लाभ्यः॑ । त्वा॒ । ओष॑धीभ्य॒ इत्योष॑धि - भ्यः॒ । इति॑ । च॒षाल᳚म् । प्रतीति॑ । \textbf{  20} \newline
                  \newline
                                \textbf{ TS 6.3.4.3} \newline
                  मु॒ञ्च॒ति॒ । तस्मा᳚त् । शी॒र्॒.ष॒तः । ओष॑धयः । फल᳚म् । गृ॒ह्ण॒न्ति॒ । अ॒नक्ति॑ । तेजः॑ । वै । आज्य᳚म् । यज॑मानेन । अ॒ग्नि॒ष्ठेत्य॑ग्नि-स्था । अश्रिः॑ । संमि॒तेति॒ सं - मि॒ता॒ । यत् । अ॒ग्नि॒ष्ठामित्य॑ग्नि- स्थाम् । अश्रि᳚म् । अ॒न॒क्ति॒ । यज॑मानम् । ए॒व । तेज॑सा । अ॒न॒क्ति॒ । आ॒न्तमित्या᳚ - अ॒न्तम् । अ॒न॒क्ति॒ । आ॒न्तमित्या᳚ - अ॒न्तम् । ए॒व । यज॑मानम् । तेज॑सा । अ॒न॒क्ति॒ । स॒र्वतः॑ । परीति॑ । मृ॒श॒ति॒ । अप॑रिवर्ग॒मित्यप॑रि - व॒र्ग॒म् । ए॒व । अ॒स्मि॒न्न् । तेजः॑ । द॒धा॒ति॒ । उदिति॑ । दिव᳚म् । स्त॒भा॒न॒ । एति॑ । अ॒न्तरि॑क्षम् । पृ॒ण॒ । इति॑ । आ॒ह॒ । ए॒षाम् । लो॒काना᳚म् । विधृ॑त्या॒ इति॒ वि - धृ॒त्यै॒ । वै॒ष्ण॒व्या । ऋ॒चा । \textbf{  21} \newline
                  \newline
                                \textbf{ TS 6.3.4.4} \newline
                  क॒ल्प॒य॒ति॒ । वै॒ष्ण॒वः । वै । दे॒वत॑या । यूपः॑ । स्वया᳚ । ए॒व । ए॒न॒म् । दे॒वत॑या । क॒ल्प॒य॒ति॒ । द्वाभ्या᳚म् । क॒ल्प॒य॒ति॒ । द्वि॒पादिति॑ द्वि - पात् । यज॑मानः । प्रति॑ष्ठित्या॒ इति॒ प्रति॑ - स्थि॒त्यै॒ । यम् । का॒मये॑त । तेज॑सा । ए॒न॒म् । दे॒वता॑भिः । इ॒न्द्रि॒येण॑ । वीति॑ । अ॒द्‌र्ध॒ये॒य॒म् । इति॑ । अ॒ग्नि॒ष्ठामित्य॑ग्नि - स्थाम् । तस्य॑ । अश्रि᳚म् । आ॒ह॒व॒नीया॒दित्या᳚-ह॒व॒नीया᳚त् । इ॒त्थम् । वा॒ । इ॒त्थम् । वा॒ । अतीति॑ । ना॒व॒ये॒त् । तेज॑सा । ए॒व । ए॒न॒म् । दे॒वता॑भिः । इ॒न्द्रि॒येण॑ । वीति॑ । अ॒द्‌र्ध॒य॒ति॒ । यम् । का॒मये॑त । तेज॑सा । ए॒न॒म् । दे॒वता॑भिः । इ॒न्द्रि॒येण॑ । समिति॑ । अ॒द्‌र्ध॒ये॒य॒म् । इति॑ । \textbf{  22} \newline
                  \newline
                                \textbf{ TS 6.3.4.5} \newline
                  अ॒ग्नि॒ष्ठामित्य॑ग्नि - स्थाम् । तस्य॑ । अश्रि᳚म् । आ॒ह॒व॒नीये॒नेत्या᳚ - ह॒व॒नीये॑न । समिति॑ । मि॒नु॒या॒त् । तेज॑सा । ए॒व । ए॒न॒म् । दे॒वता॑भिः । इ॒न्द्रि॒येण॑ । समिति॑ । अ॒द्‌र्ध॒य॒ति॒ । ब्र॒ह्म॒वनि॒मिति॑ ब्रह्म - वनि᳚म् । त्वा॒ । क्ष॒त्र॒वनि॒मिति॑ क्षत्र-वनि᳚म् । इति॑ । आ॒ह॒ । य॒था॒य॒जुरिति॑ यथा-य॒जुः । ए॒व । ए॒तत् । परीति॑ । व्य॒य॒ति॒ । ऊर्क् । वै । र॒श॒ना । यज॑मानेन । यूपः॑ । संमि॑त॒ इति॒ सं-मि॒तः॒ । यज॑मानम् । ए॒व । ऊ॒र्जा । समिति॑ । अ॒द्‌र्ध॒य॒ति॒ । ना॒भि॒द॒घ्न इति॑ नाभि - द॒घ्ने । परीति॑ । व्य॒य॒ति॒ । ना॒भि॒द॒घ्न इति॑ नाभि - द॒घ्ने । ए॒व । अ॒स्मै॒ । ऊर्ज᳚म् । द॒धा॒ति॒ । तस्मा᳚त् । ना॒भि॒द॒घ्न इति॑ नाभि - द॒घ्ने । ऊ॒र्जा । भु॒ञ्ज॒ते॒ । यम् । का॒मये॑त । ऊ॒र्जा । ए॒न॒म् । \textbf{  23} \newline
                  \newline
                                \textbf{ TS 6.3.4.6} \newline
                  वीति॑ । अ॒द्‌र्ध॒ये॒य॒म् । इति॑ । ऊ॒द्‌र्ध्वाम् । वा॒ । तस्य॑ । अवा॑चीम् । वा॒ । अवेति॑ । ऊ॒हे॒त् । ऊ॒र्जा । ए॒व । ए॒न॒म् । वीति॑ । अ॒द्‌र्ध॒य॒ति॒ । यदि॑ । का॒मये॑त । वर्.षु॑कः । प॒र्जन्यः॑ । स्या॒त् । इति॑ । अवा॑चीम् । अवेति॑ । ऊ॒हे॒त् । वृष्टि᳚म् । ए॒व । नीति॑ । य॒च्छ॒ति॒ । यदि॑ । का॒मये॑त । अव॑र्.षुकः । स्या॒त् । इति॑ । ऊ॒द्‌र्ध्वाम् । उदिति॑ । ऊ॒हे॒त् । वृष्टि᳚म् । ए॒व । उदिति॑ । य॒च्छ॒ति॒ । पि॒तृ॒णाम् । निखा॑त॒मिति॒ नि - खा॒त॒म् । म॒नु॒ष्या॑णाम् । ऊ॒द्‌र्ध्वम् । निखा॑ता॒दिति॒ नि - खा॒ता॒त् । एति॑ । र॒श॒नायाः᳚ । ओष॑धीनाम् । र॒श॒ना । विश्वे॑षां । \textbf{  24} \newline
                  \newline
                                \textbf{ TS 6.3.4.7} \newline
                  दे॒वाना᳚म् । ऊ॒द्‌र्ध्वम् । र॒श॒नायाः᳚ । एति॑ । च॒षाला᳚त् । इन्द्र॑स्य । च॒षाल᳚म् । सा॒द्ध्याना᳚म् । अति॑रिक्त॒मित्यति॑ - रि॒क्त॒म् । सः । वै । ए॒षः । स॒र्व॒दे॒व॒त्य॑ इति॑ सर्व-दे॒व॒त्यः॑ । यत् । यूपः॑ । यत् । यूप᳚म् । मि॒नोति॑ । सर्वाः᳚ । ए॒व । दे॒वताः᳚ । प्री॒णा॒ति॒ । य॒ज्ञेन॑ । वै । दे॒वाः । सु॒व॒र्गमिति॑ सुवः-गम् । लो॒कम् । आ॒य॒न्न् । ते । अ॒म॒न्य॒न्त॒ । म॒नु॒ष्याः᳚ । नः॒ । अ॒न्वाभ॑विष्य॒न्तीत्य॑नु-आभ॑विष्यन्ति । इति॑ । ते । यूपे॑न । यो॒प॒यि॒त्वा । सु॒व॒र्गमिति॑ सुवः-गम् । लो॒कम् । आ॒य॒न्न् । तम् । ऋष॑यः । यूपे॑न । ए॒व । अनु॑ । प्रेति॑ । अ॒जा॒न॒न्न् । तत् । यूप॑स्य । यू॒प॒त्वमिति॑ यूप- त्वम् । \textbf{  25} \newline
                  \newline
                                \textbf{ TS 6.3.4.8} \newline
                  यत् । यूप᳚म् । मि॒नोति॑ । सु॒व॒र्गस्येति॑ सुवः - गस्य॑ । लो॒कस्य॑ । प्रज्ञा᳚त्या॒ इति॒ प्र - ज्ञा॒त्यै॒ । पु॒रस्ता᳚त् । मि॒नो॒ति॒ । पु॒रस्ता᳚त् । हि । य॒ज्ञ्स्य॑ । प्र॒ज्ञा॒यत॒ इति॑ प्र - ज्ञा॒यते᳚ । अप्र॑ज्ञात॒मित्यप्र॑ - ज्ञा॒त॒म् । हि । तत् । यत् । अति॑पन्न॒ इत्यति॑ - प॒न्ने॒ । आ॒हुः । इ॒दम् । का॒र्य᳚म् । आ॒सी॒त् । इति॑ । सा॒द्ध्याः । वै । दे॒वाः । य॒ज्ञ्म् । अतीति॑ । अ॒म॒न्य॒न्त॒ । तान् । य॒ज्ञ्ः । न । अ॒स्पृ॒श॒त् । तान् । यत् । य॒ज्ञ्स्य॑ । अति॑रिक्त॒मित्यति॑ - रि॒क्त॒म् । आसी᳚त् । तत् । अ॒स्पृ॒श॒त् । अति॑रिक्त॒मित्यति॑ - रि॒क्त॒म् । वै । ए॒तत् । य॒ज्ञ्स्य॑ । यत् । अ॒ग्नौ । अ॒ग्निम् । म॒थि॒त्वा । प्र॒हर॒तीति॑ प्र - हर॑ति । अति॑रिक्त॒मित्यति॑ - रि॒क्त॒म् । ए॒तत् । \textbf{  26} \newline
                  \newline
                                \textbf{ TS 6.3.4.9} \newline
                  यूप॑स्य । यत् । ऊ॒द्‌र्ध्वम् । च॒षाला᳚त् । तेषा᳚म् । तत् । भा॒ग॒धेय॒मिति॑ भाग - धेय᳚म् । तान् । ए॒व । तेन॑ । प्री॒णा॒ति॒ । दे॒वाः । वै । सꣳस्थि॑त॒ इति॒ सं-स्थि॒ते॒ । सोमे᳚ । प्रेति॑ । स्रुचः॑ । अह॑रन्न् । प्रेति॑ । यूप᳚म् । ते । अ॒म॒न्य॒न्त॒ । य॒ज्ञ्॒वे॒श॒समिति॑ यज्ञ् - वे॒श॒स॒म् । वै । इ॒दम् । कु॒र्मः॒ । इति॑ । ते । प्र॒स्त॒रमिति॑ प्र - स्त॒रम् । स्रु॒चाम् । नि॒ष्क्रय॑ण॒मिति॑ निः - क्रय॑णम् । अ॒प॒श्य॒न्न् । स्वरु᳚म् । यूप॑स्य । सꣳस्थि॑त॒ इति॒ सं - स्थि॒ते॒ । सोमे᳚ । प्रेति॑ । प्र॒स्त॒रमिति॑ प्र-स्त॒रम् । हर॑ति । जु॒होति॑ । स्वरु᳚म् । अय॑ज्ञ्वेशसा॒येत्यय॑ज्ञ्-वे॒श॒सा॒य॒ ॥ \textbf{  27} \newline
                  \newline
                      (द॒धा॒ति॒ - प्रत्यृ॒ - चा - सम॑र्द्धयेय॒मित्यू॒ - र्जैनं॒ - ॅविश्वे॑षां - ॅयूप॒त्व - मति॑रिक्तमे॒तद् - द्विच॑त्वारिꣳशच्च)  \textbf{(A4)} \newline \newline
                                \textbf{ TS 6.3.5.1} \newline
                  सा॒द्ध्याः । वै । दे॒वाः । अ॒स्मिन्न् । लो॒के । आ॒स॒न्न् । न । अ॒न्यत् । किम् । च॒न । मि॒षत् । ते । अ॒ग्निम् । ए॒व । अ॒ग्नये᳚ । मेधा॑य । एति॑ । अ॒ल॒भ॒न्त॒ । न । हि । अ॒न्यत् । आ॒ल॒भ्यं॑मित्या᳚-ल॒भ्यं᳚म् । अवि॑न्दन्न् । ततः॑ । वै । इ॒माः । प्र॒जा इति॑ प्र - जाः । प्रेति॑ । अ॒जा॒य॒न्त॒ । यत् । अ॒ग्नौ । अ॒ग्निम् । म॒थि॒त्वा । प्र॒हर॒तीति॑ प्र - हर॑ति । प्र॒जाना॒मिति॑ प्र - जाना᳚म् । प्र॒जन॑ना॒येति॑ प्र-जन॑नाय । रु॒द्रः । वै । ए॒षः । यत् । अ॒ग्निः । यज॑मानः । प॒शुः । यत् । प॒शुम् । आ॒लभ्येत्या᳚ - लभ्य॑ । अ॒ग्निम् । मन्थे᳚त् । रु॒द्राय॑ । यज॑मानम् । \textbf{  28} \newline
                  \newline
                                \textbf{ TS 6.3.5.2} \newline
                  अपीति॑ । द॒द्ध्या॒त् । प्र॒मायु॑क॒ इति॑ प्र-मायु॑कः । स्या॒त् । अथो॒ इति॑ । खलु॑ । आ॒हुः॒ । अ॒ग्निः । सर्वाः᳚ । दे॒वताः᳚ । ह॒विः । ए॒तत् । यत् । प॒शुः । इति॑ । यत् । प॒शुम् । आ॒लभ्येत्या᳚-लभ्य॑ । अ॒ग्निम् । मन्थ॑ति । ह॒व्याय॑ । ए॒व । आस॑न्ना॒येत्या-स॒न्ना॒य॒ । सर्वाः᳚ । दे॒वताः᳚ । ज॒न॒य॒ति॒ । उ॒पा॒कृत्येत्यु॑प - आ॒कृत्य॑ । ए॒व । मन्थ्यः॑ । तत् । न । इ॒व॒ । आल॑ब्ध॒मित्या - ल॒ब्ध॒म् । न । इ॒व॒ । आल॑ब्ध॒मित्यना᳚ - ल॒ब्ध॒म् । अ॒ग्नेः । ज॒नित्र᳚म् । अ॒सि॒ । इति॑ । आ॒ह॒ । अ॒ग्नेः । हि । ए॒तत् । ज॒नित्र᳚म् । वृष॑णौ । स्थः॒ । इति॑ । आ॒ह॒ । वृष॑णौ । \textbf{  29} \newline
                  \newline
                                \textbf{ TS 6.3.5.3} \newline
                  हि । ए॒तौ । उ॒र्वशी᳚ । अ॒सि॒ । आ॒युः । अ॒सि॒ । इति॑ । आ॒ह॒ । मि॒थु॒न॒त्वायेति॑ मिथुन - त्वाय॑ । घृ॒तेन॑ । अ॒क्ते इति॑ । वृष॑णम् । द॒धा॒था॒म् । इति॑ । आ॒ह॒ । वृष॑णम् । हि । ए॒ते इति॑ । दधा॑ते॒ इति॑ । ये इति॑ । अ॒ग्निम् । गा॒य॒त्रम् । छन्दः॑ । अनु॑ । प्रेति॑ । जा॒य॒स्व॒ । इति॑ । आ॒ह॒ । छन्दो॑भि॒रिति॒ छन्दः॑ - भिः॒ । ए॒व । ए॒न॒म् । प्रेति॑ । ज॒न॒य॒ति॒ । अ॒ग्नये᳚ । म॒थ्यमा॑नाय । अन्विति॑ । ब्रू॒हि॒ । इति॑ । आ॒ह॒ । सा॒वि॒त्रीम् । ऋच᳚म् । अन्विति॑ । आ॒ह॒ । स॒वि॒तृप्र॑सूत॒ इति॑ सवि॒तृ-प्र॒सू॒तः॒ । ए॒व । ए॒न॒म् । म॒न्थ॒ति॒ । जा॒ताय॑ । अन्विति॑ । ब्रू॒हि॒ । \textbf{  30} \newline
                  \newline
                                \textbf{ TS 6.3.5.4} \newline
                  प्र॒ह्रि॒यमा॑णा॒येति॑ प्र - ह्रि॒यमा॑णाय । अन्विति॑ । ब्रू॒हि॒ । इति॑ । आ॒ह॒ । काण्डे॑काण्ड॒ इति॒ काण्डे᳚ - का॒ण्डे॒ । ए॒व । ए॒न॒म् । क्रि॒यमा॑णे । समिति॑ । अ॒द्‌र्ध॒य॒ति॒ । गा॒य॒त्रीः । सर्वाः᳚ । अन्विति॑ । आ॒ह॒ । गा॒य॒त्रछ॑न्दा॒ इति॑ गाय॒त्र-छ॒न्दाः॒ । वै । अ॒ग्निः । स्वेन॑ । ए॒व । ए॒न॒म् । छन्द॑सा । समिति॑ । अ॒द्‌र्ध॒य॒ति॒ । अ॒ग्निः । पु॒रा । भव॑ति । अ॒ग्निम् । म॒थि॒त्वा । प्रेति॑ । ह॒र॒ति॒ । तौ । स॒भंव॑न्ता॒विति॑ सं - भव॑न्तौ । यज॑मानम् । अ॒भि । समिति॑ । भ॒व॒तः॒ । भव॑तम् । नः॒ । सम॑नसा॒विति॒ स - म॒न॒सौ॒ । इति॑ । आ॒ह॒ । शान्त्यै᳚ । प्र॒हृत्येति॑ प्र-हृत्य॑ । जु॒हो॒ति॒ । जा॒ताय॑ । ए॒व । अ॒स्मै॒ । अन्न᳚म् । अपीति॑ ( ) । द॒धा॒ति॒ । आज्ये॑न । जु॒हो॒ति॒ । ए॒तत् । वै । अ॒ग्नेः । प्रि॒यम् । धाम॑ । यत् । आज्य᳚म् । प्रि॒येण॑ । ए॒व । ए॒न॒म् । धाम्ना᳚ । समिति॑ । अ॒द्‌र्ध॒य॒ति॒ । अथो॒ इति॑ । तेज॑सा ॥ \textbf{  31} \newline
                  \newline
                      (यज॑मान-माह॒ वृष॑णौ-जा॒तायानु॑ ब्रू॒ह्या-प्य॒ -ष्टाद॑श च)  \textbf{(A5)} \newline \newline
                                \textbf{ TS 6.3.6.1} \newline
                  इ॒षे । त्वा॒ । इति॑ । ब॒र्॒.हिः । एति॑ । द॒त्ते॒ । इ॒च्छते᳚ । इ॒व॒ । हि । ए॒षः । यः । यज॑ते । उ॒प॒वीरित्यु॑प - वीः । अ॒सि॒ । इति॑ । आ॒ह॒ । उपेति॑ । हि । ए॒ना॒न् । आ॒क॒रोतीत्या᳚ - क॒रोति॑ । उपो॒ इति॑ । दे॒वान् । दैवीः᳚ । विशः॑ । प्रेति॑ । अ॒गुः॒ । इति॑ । आ॒ह॒ । दैवीः᳚ । हि । ए॒ताः । विशः॑ । स॒तीः । दे॒वान् । उ॒प॒यन्तीत्यु॑प - यन्ति॑ । वह्नीः᳚ । उ॒शिजः॑ । इति॑ । आ॒ह॒ । ऋ॒त्विजः॑ । वै । वह्न॑यः । उ॒शिजः॑ । तस्मा᳚त् । ए॒वम् । आ॒ह॒ । बृह॑स्पते । धा॒रय॑ । वसू॑नि । इति॑ । \textbf{  32} \newline
                  \newline
                                \textbf{ TS 6.3.6.2} \newline
                  आ॒ह॒ । ब्रह्म॑ । वै । दे॒वाना᳚म् । बृह॒स्पतिः॑ । ब्रह्म॑णा । ए॒व । अ॒स्मै॒ । प॒शून् । अवेति॑ । रु॒न्धे॒ । ह॒व्या । ते॒ । स्व॒द॒न्ता॒म् । इति॑ । आ॒ह॒ । स्व॒दय॑ति । ए॒व । ए॒ना॒न् । देव॑ । त्व॒ष्टः॒ । वसु॑ । र॒ण्व॒ । इति॑ । आ॒ह॒ । त्वष्टा᳚ । वै । प॒शू॒नाम् । मि॒थु॒नाना᳚म् । रू॒प॒कृदिति॑ रूप - कृत् । रू॒पम् । ए॒व । प॒शुषु॑ । द॒धा॒ति॒ । रेव॑तीः । रम॑द्ध्वम् । इति॑ । आ॒ह॒ । प॒शवः॑ । वै । रे॒वतीः᳚ । प॒शून् । ए॒व । अ॒स्मै॒ । र॒म॒य॒ति॒ । दे॒वस्य॑ । त्वा॒ । स॒वि॒तुः । प॒स॒व इति॑ प्र - स॒वे । इति॑ । \textbf{  33} \newline
                  \newline
                                \textbf{ TS 6.3.6.3} \newline
                  र॒श॒नाम् । एति॑ । द॒त्ते॒ । प्रसू᳚त्या॒ इति॒ प्र - सू॒त्यै॒ । अ॒श्विनोः᳚ । बा॒हुभ्या॒मिति॑ बा॒हु-भ्या॒म् । इति॑ । आ॒ह॒ । अ॒श्विनौ᳚ । हि । दे॒वाना᳚म् । अ॒द्ध्व॒र्यू इति॑ । आस्ता᳚म् । पू॒ष्णः । हस्ता᳚भ्याम् । इति॑ । आ॒ह॒ । यत्यै᳚ । ऋ॒तस्य॑ । त्वा॒ । दे॒व॒ह॒वि॒रिति॑ देव - ह॒विः॒ । पाशे॑न । एति॑ । र॒भे॒ । इति॑ । आ॒ह॒ । स॒त्यम् । वै । ऋ॒तम् । स॒त्येन॑ । ए॒व । ए॒न॒म् । ऋ॒तेन॑ । एति॑ । र॒भ॒ते॒ । अ॒क्ष्ण॒या । परीति॑ । ह॒र॒ति॒ । वद्ध्य᳚म् । हि । प्र॒त्यञ्च᳚म् । प्र॒ति॒मु॒ञ्चन्तीति॑ प्रति - मु॒ञ्चन्ति॑ । व्यावृ॑त्त्या॒ इति॑ वि - आवृ॑त्त्यै । धर्.ष॑ । मानु॑षान् । इति॑ । नीति॑ । यु॒न॒क्ति॒ । धृत्यै᳚ । अ॒द्भ्य इत्य॑त् - भ्यः । \textbf{  34} \newline
                  \newline
                                \textbf{ TS 6.3.6.4} \newline
                  त्वा॒ । ओष॑धीभ्य॒ इत्योष॑धि - भ्यः॒ । प्रेति॑ । उ॒क्षा॒मि॒ । इति॑ । आ॒ह॒ । अ॒द्भ्य इत्य॑त् - भ्यः । हि । ए॒षः । ओष॑धीभ्य॒ इत्योष॑धि - भ्यः॒ । स॒भंव॒तीति॑ सं - भव॑ति । यत् । प॒शुः । अ॒पाम् । पे॒रुः । अ॒सि॒ । इति॑ । आ॒ह॒ । ए॒षः । हि । अ॒पाम् । पा॒ता । यः । मेधा॑य । आ॒र॒भ्यत॒ इत्या᳚-र॒भ्यते᳚ । स्वा॒त्तम् । चि॒त् । सदे॑व॒मिति॒ स - दे॒व॒म् । ह॒व्यम् । आपः॑ । दे॒वीः॒ । स्वद॑त । ए॒न॒म् । इति॑ । आ॒ह॒ । स्व॒दय॑ति । ए॒व । ए॒न॒म् । उ॒परि॑ष्टात् । प्रेति॑ । उ॒क्ष॒ति॒ । उ॒परि॑ष्टात् । ए॒व । ए॒न॒म् । मेद्ध्य᳚म् । क॒रो॒ति॒ । पा॒यय॑ति । अ॒न्त॒र॒तः । ए॒व । ए॒न॒म् ( ) । मेद्ध्य᳚म् । क॒रो॒ति॒ । अ॒धस्ता᳚त् । उपेति॑ । उ॒क्ष॒ति॒ । स॒र्वतः॑ । ए॒व । ए॒न॒म् । मेद्ध्य᳚म् । क॒रो॒ति॒ ॥ \textbf{  35 } \newline
                  \newline
                      (वसू॒नीति॑-प्रस॒व इत्य॒-द्भ्यो᳚-ऽन्तर॒त ए॒वैनं॒ - दश॑ च)  \textbf{(A6)} \newline \newline
                                \textbf{ TS 6.3.7.1} \newline
                  अ॒ग्निना᳚ । वै । होत्रा᳚ । दे॒वाः । असु॑रान् । अ॒भीति॑ । अ॒भ॒व॒न्न् । अ॒ग्नये᳚ । स॒मि॒द्ध्यमा॑ना॒येति॑ सं - इ॒द्ध्यमा॑नाय । अन्विति॑ । ब्रू॒हि॒ । इति॑ । आ॒ह॒ । भ्रातृ॑व्याभिभूत्या॒ इति॒ भ्रातृ॑व्य-अ॒भि॒भू॒त्यै॒ । स॒प्तद॒शेति॑ स॒प्त - द॒श॒ । सा॒मि॒धे॒नीरिति॑ सां - इ॒धे॒नीः । अन्विति॑ । आ॒ह॒ । स॒प्त॒द॒श इति॑ सप्त - द॒शः । प्र॒जाप॑ति॒रिति॑ प्र॒जा - प॒तिः॒ । प्र॒जाप॑ते॒रिति॑ प्र॒जा - प॒तेः॒ । आप्त्यै᳚ । स॒प्तद॒शेति॑ स॒प्त - द॒श॒ । अन्विति॑ । आ॒ह॒ । द्वाद॑श । मासाः᳚ । पञ्च॑ । ऋ॒तवः॑ । सः । सं॒ॅव॒थ्स॒र इति॑ सं - व॒थ्स॒रः । सं॒ॅव॒थ्स॒रमिति॑ सं - व॒थ्स॒रम् । प्र॒जा इति॑ प्र - जाः । अनु॑ । प्रेति॑ । जा॒य॒न्ते॒ । प्र॒जाना॒मिति॑ प्र-जाना᳚म् । प्र॒जन॑ना॒येति॑ प्र-जन॑नाय । दे॒वाः । वै । सा॒मि॒धे॒नीरिति॑ सां-इ॒धे॒नीः । अ॒नूच्येत्य॑नु - उच्य॑ । य॒ज्ञ्म् । न । अन्विति॑ । अ॒प॒श्य॒न्न् । सः । प्र॒जाप॑ति॒रिति॑ प्र॒जा-प॒तिः॒ । तू॒ष्णीम् । आ॒घा॒रमित्या᳚-घा॒रम् । \textbf{  36} \newline
                  \newline
                                \textbf{ TS 6.3.7.2} \newline
                  एति॑ । अ॒घा॒र॒य॒त् । ततः॑ । वै । दे॒वाः । य॒ज्ञ्म् । अन्विति॑ । अ॒प॒श्य॒न्न् । यत् । तू॒ष्णीम् । आ॒घा॒रमित्या᳚ - घा॒रम् । आ॒घा॒रय॒तीत्या᳚ - घा॒रय॑ति । य॒ज्ञ्स्य॑ । अनु॑ख्यात्या॒ इत्यनु॑ - ख्या॒त्यै॒ । असु॑रेषु । वै । य॒ज्ञ्ः । आ॒सी॒त् । तम् । दे॒वाः । तू॒ष्णीꣳ॒॒हो॒मेनेति॑ तूष्णीम्-हो॒मेन॑ । अ॒वृ॒ञ्ज॒त॒ । यत् । तू॒ष्णीम् । आ॒घा॒रमित्या᳚ - घा॒रम् । आ॒घा॒रय॒तीत्या᳚ - घा॒रय॑ति । भ्रातृ॑व्यस्य । ए॒व । तत् । य॒ज्ञ्म् । वृ॒ङ्क्ते॒ । प॒रि॒धीनिति॑ परि - धीन् । समिति॑ । मा॒र्ष्टि॒ । पु॒नाति॑ । ए॒व । ए॒ना॒न् । त्रिस्त्रि॒रिति॒ त्रिः - त्रिः॒ । समिति॑ । मा॒र्ष्टि॒ । त्र्या॑वृ॒दिति॒ त्रि - आ॒वृ॒त् । हि । य॒ज्ञ्ः । अथो॒ इति॑ । रक्ष॑साम् । अप॑हत्या॒ इत्यप॑ - ह॒त्यै॒ । द्वाद॑श । समिति॑ । प॒द्य॒न्ते॒ । द्वाद॑श । \textbf{  37} \newline
                  \newline
                                \textbf{ TS 6.3.7.3} \newline
                  मासाः᳚ । सं॒ॅव॒थ्स॒र इति॑ सं-व॒थ्स॒रः । सं॒ॅव॒थ्स॒रमिति॑ सं-व॒थ्स॒रम् । ए॒व । प्री॒णा॒ति॒ । अथो॒ इति॑ । सं॒ॅव॒थ्स॒रमिति॑ सं - व॒थ्स॒रम् । ए॒व । अ॒स्मै॒ । उपेति॑ । द॒धा॒ति॒ । सु॒व॒र्गस्येति॑ सुवः - गस्य॑ । लो॒कस्य॑ । सम॑ष्ट्या॒ इति॒ सं - अ॒ष्ट्यै॒ । शिरः॑ । वै । ए॒तत् । य॒ज्ञ्स्य॑ । यत् । आ॒घा॒र इत्या᳚ - घा॒रः । अ॒ग्निः । सर्वाः᳚ । दे॒वताः᳚ । यत् । आ॒घा॒रमित्या᳚-घा॒रम् । आ॒घा॒रय॒तीत्या᳚-घा॒रय॑ति । शी॒र्॒.ष॒तः । ए॒व । य॒ज्ञ्स्य॑ । यज॑मानः । सर्वाः᳚ । दे॒वताः᳚ । अवेति॑ । रु॒न्धे॒ । शिरः॑ । वै । ए॒तत् । य॒ज्ञ्स्य॑ । यत् । आ॒घा॒र इत्या᳚ - घा॒रः । आ॒त्मा । प॒शुः । आ॒घा॒रमित्या᳚ - घा॒रम् । आ॒घार्येत्या᳚ - घार्य॑ । प॒शुम् । समिति॑ । अ॒न॒क्ति॒ । आ॒त्मन्न् । ए॒व । य॒ज्ञ्स्य॑ । \textbf{  38} \newline
                  \newline
                                \textbf{ TS 6.3.7.4} \newline
                  शिरः॑ । प्रतीति॑ । द॒धा॒ति॒ । समिति॑ । ते॒ । प्रा॒ण इति॑ प्र - अ॒नः । वा॒युना᳚ । ग॒च्छ॒ता॒म् । इति॑ । आ॒ह॒ । वा॒यु॒दे॒व॒त्य॑ इति॑ वायु-दे॒व॒त्यः॑ । वै । प्रा॒ण इति॑ प्र - अ॒नः । वा॒यौ । ए॒व । अ॒स्य॒ । प्रा॒णमिति॑ प्र - अ॒नम् । जु॒हो॒ति॒ । समिति॑ । यज॑त्रैः । अङ्गा॑नि । समिति॑ । य॒ज्ञ्प॑ति॒रिति॑ य॒ज्ञ् - प॒तिः॒ । आ॒शिषेत्या᳚ - शिषा᳚ । इति॑ । आ॒ह॒ । य॒ज्ञ्प॑ति॒मिति॑ य॒ज्ञ् - प॒ति॒म् । ए॒व । अ॒स्य॒ । आ॒शिष॒मित्या᳚-शिष᳚म् । ग॒म॒य॒ति॒ । वि॒श्वरू॑प॒ इति॑ वि॒श्व - रू॒पः॒ । वै । त्वा॒ष्ट्रः । उ॒परि॑ष्टात् । प॒शुम् । अ॒भीति॑ । अ॒व॒मी॒त् । तस्मा᳚त् । उ॒परि॑ष्टात् । प॒शोः । न । अवेति॑ । द्य॒न्ति॒ । यत् । उ॒परि॑ष्टात् । प॒शुम् । स॒म॒नक्तीति॑ सं - अ॒नक्ति॑ । मेद्ध्य᳚म् । ए॒व । \textbf{  39} \newline
                  \newline
                                \textbf{ TS 6.3.7.5} \newline
                  ए॒न॒म् । क॒रो॒ति॒ । ऋ॒त्विजः॑ । वृ॒णी॒ते॒ । छन्दाꣳ॑सि । ए॒व । वृ॒णी॒ते॒ । स॒प्त । वृ॒णी॒ते॒ । स॒प्त । ग्रा॒म्याः । प॒शवः॑ । स॒प्त । आ॒र॒ण्याः । स॒प्त । छन्दाꣳ॑सि । उ॒भय॑स्य । अव॑रुद्ध्या॒ इत्यव॑ - रु॒द्ध्यै॒ । एका॑दश । प्र॒या॒जानिति॑ प्र - या॒जान् । य॒ज॒ति॒ । दश॑ । वै । प॒शोः । प्रा॒णा इति॑ प्र-अ॒नाः । आ॒त्मा । ए॒का॒द॒शः । यावान्॑ । ए॒व । प॒शुः । तम् । प्रेति॑ । य॒ज॒ति॒ । व॒पाम् । एकः॑ । परीति॑ । श॒ये॒ । आ॒त्मा । ए॒व । आ॒त्मान᳚म् । परीति॑ । श॒ये॒ । वज्रः॑ । वै । स्वधि॑ति॒रिति॒ स्व - धि॒तिः॒ । वज्रः॑ । यू॒प॒श॒क॒ल इति॑ यूप - श॒क॒लः । घृ॒तम् । खलु॑ । वै ( ) । दे॒वाः । वज्र᳚म् । कृ॒त्वा । सोम᳚म् । अ॒घ्न॒न्न् । घृ॒तेन॑ । अ॒क्तौ । प॒शुम् । त्रा॒ये॒था॒म् । इति॑ । आ॒ह॒ । वज्रे॑ण । ए॒व । ए॒न॒म् । वशे᳚ । कृ॒त्वा । एति॑ । ल॒भ॒ते॒ ॥ \textbf{  40} \newline
                  \newline
                      (आ॒घा॒रं - प॑द्यन्ते॒ द्वाद॑शा॒ - ऽऽत्मन्ने॒व य॒ज्ञ्स्य॒ - मेध्य॑मे॒व - खलु॒ वा - अ॒ष्टाद॑श च)  \textbf{(A7)} \newline \newline
                                \textbf{ TS 6.3.8.1} \newline
                  पर्य॒ग्नीति॒ परि॑ - अ॒ग्नि॒ । क॒रो॒ति॒ । स॒र्व॒हुत॒मिति॑ सर्व-हुत᳚म् । ए॒व । ए॒न॒म् । क॒रो॒ति॒ । अस्क॑न्दाय । अस्क॑न्नम् । हि । तत् । यत् । हु॒तस्य॑ । स्कन्द॑ति । त्रिः । पर्य॒ग्नीति॒ परि॑ - अ॒ग्नि॒ । क॒रो॒ति॒ । त्र्या॑वृ॒दिति॒ त्रि - आ॒वृ॒त् । हि । य॒ज्ञ्ः । अथो॒ इति॑ । रक्ष॑साम् । अप॑हत्या॒ इत्यप॑ - ह॒त्यै॒ । ब्र॒ह्म॒वा॒दिन॒ इति॑ ब्रह्म-वा॒दिनः॑ । व॒द॒न्ति॒ । अ॒न्वा॒रभ्य॒ इत्य॑नु - आ॒रभ्यः॑ । प॒शू(3)ः । न । अ॒न्वा॒रभ्या(3) इत्य॑नु -आ॒रभ्या(3)ः । इति॑ । मृ॒त्यवे᳚ । वै । ए॒षः । नी॒य॒ते॒ । यत् ।  प॒शुः । तम् । यत् । अ॒न्वा॒रभे॒तेत्य॑नु-आ॒रभे॑त । प्र॒मायु॑क॒ इति॑ प्र-मायु॑कः । यज॑मानः । स्या॒त् । अथो॒ इति॑ । खलु॑ । आ॒हुः॒ । सु॒व॒र्गायेति॑ सुवः - गाय॑ । वै । ए॒षः । लो॒काय॑ । नी॒य॒ते॒ । यत् । \textbf{  41} \newline
                  \newline
                                \textbf{ TS 6.3.8.2} \newline
                  प॒शुः । इति॑ । यत् । न । अ॒न्वा॒रभे॒तेत्य॑नु - आ॒रभे॑त । सु॒व॒र्गादिति॑ सुवः - गात् । लो॒कात् । यज॑मानः । ही॒ये॒त॒ । व॒पा॒श्रप॑णीभ्या॒मिति॑ वपा - श्रप॑णीभ्याम् । अन्वार॑भत॒ इत्य॑नु - आर॑भते । तत् । न । इ॒व॒ । अ॒न्वार॑ब्ध॒मित्य॑नु-आर॑ब्धम् । न । इ॒व॒ । अन॑न्वारब्ध॒मित्यन॑नु-आ॒र॒ब्ध॒म् । उप॑ । प्रेति॑ । इ॒ष्य॒ । हो॒तः॒ । ह॒व्या । दे॒वेभ्यः॑ । इति॑ । आ॒ह॒ । इ॒षि॒तम् । हि । कर्म॑ । क्रि॒यते᳚ । रेव॑तीः । य॒ज्ञ्प॑ति॒मिति॑ य॒ज्ञ् - प॒ति॒म् । प्रि॒य॒धेति॑ प्रिय - धा । एति॑ । वि॒श॒त॒ । इति॑ । आ॒ह॒ । य॒था॒य॒जुरिति॑ यथा - य॒जुः । ए॒व । ए॒तत् । अ॒ग्निना᳚ । पु॒रस्ता᳚त् । ए॒ति॒ । रक्ष॑साम् । अप॑हत्या॒ इत्यप॑ - ह॒त्यै॒ । पृ॒थि॒व्याः । स॒पृंच॒ इति॑ सं - पृचः॑ । पा॒हि॒ । इति॑ । ब॒र्॒.हिः । \textbf{  42} \newline
                  \newline
                                \textbf{ TS 6.3.8.3} \newline
                  उपेति॑ । अ॒स्य॒ति॒ । अस्क॑न्दाय । अस्क॑न्नम् । हि । तत् । यत् । ब॒र्॒.हिषि॑ । स्कन्द॑ति । अथो॒ इति॑ । ब॒र्॒.हि॒षद॒मिति॑ बर्.हि - सद᳚म् । ए॒व । ए॒न॒म् । क॒रो॒ति॒ । पराङ्॑ । एति॑ । व॒र्त॒ते॒ । अ॒द्ध्व॒र्युः । प॒शोः । स॒ज्ञ्ं॒प्यमा॑ना॒दिति॑ सं - ज्ञ्॒प्यमा॑नात् । प॒शुभ्य॒ इति॑ प॒शु - भ्यः॒ । ए॒व । तत् । नीति॑ । ह्नु॒ते॒ । आ॒त्मनः॑ । अना᳚व्रस्का॒येत्यना᳚-व्र॒स्का॒य॒ । गच्छ॑ति । श्रिय᳚म् । प्रेति॑ । प॒शून् । आ॒प्नो॒ति॒ । यः । ए॒वम् । वेद॑ । प॒श्चाल्लो॒केति॑ प॒श्चात् - लो॒का॒ । वै । ए॒षा । प्राची᳚ । उ॒दानी॑यत॒ इत्यु॑त् - आनी॑यते । यत् । पत्नी᳚ । नमः॑ । ते॒ । आ॒ता॒नेत्या᳚ - ता॒न॒ । इति॑ । आ॒ह॒ । आ॒दि॒त्यस्य॑ । वै । र॒श्मयः॑ । \textbf{  43} \newline
                  \newline
                                \textbf{ TS 6.3.8.4} \newline
                  आ॒ता॒ना इत्या᳚ - ता॒नाः । तेभ्यः॑ । ए॒व । नमः॑ । क॒रो॒ति॒ । अ॒न॒र्वा । प्रेति॑ । इ॒हि॒ । इति॑ । आ॒ह॒ । भ्रातृ॑व्यः । वै । अर्वा᳚ । भ्रातृ॑व्यापनुत्त्या॒ इति॒ भ्रातृ॑व्य - अ॒प॒नु॒त्त्यै॒ । घृ॒तस्य॑ । कु॒ल्याम् । अन्विति॑ । स॒ह । प्र॒जयेति॑ प्र - जया᳚ । स॒ह । रा॒यः । पोषे॑ण । इति॑ । आ॒ह॒ । आ॒शिष॒मित्या᳚ - शिष᳚म् । ए॒व । ए॒ताम् । एति॑ । शा॒स्ते॒ । आपः॑ ।दे॒वीः॒ । शु॒द्धा॒यु॒व॒ इति॑ शुद्ध - यु॒वः॒ । इति॑ । आ॒ह॒ । य॒था॒य॒जुरिति॑ यथा - य॒जुः । ए॒व । ए॒तत् ॥ \textbf{  44 } \newline
                  \newline
                      (लो॒काय॑ नीयते॒ यद् - ब॒र॒.ही - र॒श्मयः॑ - स॒प्तत्रिꣳ॑शच्च)  \textbf{(A8)} \newline \newline
                                \textbf{ TS 6.3.9.1} \newline
                  प॒शोः । वै । आल॑ब्ध॒स्येत्या - ल॒ब्ध॒स्य॒ । प्रा॒णानिति॑ प्र - अ॒नान् । शुक् । ऋ॒च्छ॒ति॒ । वाक् । ते॒ । एति॑ । प्या॒य॒ता॒म् । प्रा॒ण इति॑ प्र - अ॒नः । ते॒ । एति॑ । प्या॒य॒ता॒म् । इति॑ । आ॒ह॒ । प्रा॒णेभ्य॒ इति॑ प्र - अ॒नेभ्यः॑ । ए॒व । अ॒स्य॒ । शुच᳚म् । श॒म॒य॒ति॒ । सा । प्रा॒णेभ्य॒ इति॑ प्र - अ॒नेभ्यः॑ । अधीति॑ । पृ॒थि॒वीम् । शुक् । प्रेति॑ । वि॒श॒ति॒ । शम् । अहो᳚भ्या॒मित्यहः॑ - भ्या॒म् । इति॑ । नीति॑ । न॒य॒ति॒ । अ॒हो॒रा॒त्राभ्या॒मित्य॑हः-रा॒त्राभ्या᳚म् । ए॒व । पृ॒थि॒व्यै । शुच᳚म् । श॒म॒य॒ति॒ । ओष॑धे । त्राय॑स्व । ए॒न॒म् । स्वधि॑त॒ इति॒ स्व - धि॒ते॒ । मा । ए॒न॒म् । हिꣳ॒॒सीः॒ । इति॑ । आ॒ह॒ । वज्रः॑ । वै । स्वधि॑ति॒रिति॒ स्व - धि॒तिः॒ । \textbf{  45 } \newline
                  \newline
                                \textbf{ TS 6.3.9.2} \newline
                  शान्त्यै᳚ । पा॒र्श्व॒तः । एति॑ । छ्‌य॒ति॒ । म॒द्ध्य॒तः । हि । म॒नु॒ष्याः᳚ । आ॒च्छ्यन्तीत्या᳚-छ्यन्ति॑ । ति॒र॒श्चीन᳚म् । एति॑ । छ्य॒ति॒ । अ॒नू॒चीन᳚म् । हि । म॒नु॒ष्याः᳚ । आ॒च्छ्यन्तीत्या᳚ - छ्यन्ति॑ । व्यावृ॑त्त्या॒ इति॑ वि - आवृ॑त्त्यै । रक्ष॑साम् । भा॒गः । अ॒सि॒ । इति॑ । स्थ॒वि॒म॒तः । ब॒र्॒.हिः । अ॒क्त्वा । अपेति॑ । अ॒स्य॒ति॒ । अ॒स्ना । ए॒व । रक्षाꣳ॑सि । नि॒रव॑दयत॒ इति॑ निः - अव॑दयते । इ॒दम् । अ॒हम् । रक्षः॑ । अ॒ध॒मम् । तमः॑ । न॒या॒मि॒ । यः । अ॒स्मान् । द्वेष्टि॑ । यम् । च॒ । व॒यम् । द्वि॒ष्मः । इति॑ । आ॒ह॒ । द्वौ । वाव । पुरु॑षौ । यम् । च॒ । ए॒व । \textbf{  46} \newline
                  \newline
                                \textbf{ TS 6.3.9.3} \newline
                  द्वेष्टि॑ । यः । च॒ । ए॒न॒म् । द्वेष्टि॑ । तौ । उ॒भौ । अ॒ध॒मम् । तमः॑ । न॒य॒ति॒ । इ॒षे । त्वा॒ । इति॑ । व॒पाम् । उदिति॑ । खि॒द॒ति॒ । इ॒च्छते᳚ । इ॒व॒ । हि । ए॒षः । यः । यज॑ते । यत् । उ॒प॒तृ॒न्द्यादित्यु॑प - तृ॒न्द्यात् । रु॒द्रः । अ॒स्य॒ । प॒शून् । घातु॑कः । स्या॒त् । यत् । न । उ॒प॒तृ॒न्द्यादित्यु॑प - तृ॒न्द्यात् । अय॑ता । स्या॒त् । अ॒न्यया᳚ । उ॒प॒तृ॒णत्तीत्यु॑प - तृ॒णत्ति॑ । अ॒न्यया᳚ । न । धृत्यै᳚ । घृ॒तेन॑ । द्या॒वा॒पृ॒थि॒वी॒ इति॑ द्यावा-पृ॒थि॒वी॒ । प्रेति॑ । ऊ॒र्ण्वा॒था॒म् । इति॑ । आ॒ह॒ । द्यावा॑पृथि॒वी इति॒ द्यावा᳚ - पृ॒थि॒वी । ए॒व । रसे॑न । अ॒न॒क्ति॒ । अच्छि॑न्नः । \textbf{  47} \newline
                  \newline
                                \textbf{ TS 6.3.9.4} \newline
                  रायः॑ । सु॒वीर॒ इति॑ सु - वीरः॑ । इति॑ । आ॒ह॒ । य॒था॒य॒जुरिति॑ यथा - य॒जुः । ए॒व । ए॒तत् । क्रू॒रम् । इ॒व॒ । वै । ए॒तत् । क॒रो॒ति॒ । यत् । व॒पाम् । उ॒त्खि॒दतीत्यु॑त् - खि॒दति॑ । उ॒रु । अ॒न्तरि॑क्षम् । अन्विति॑ । इ॒हि॒ । इति॑ । आ॒ह॒ । शान्त्यै᳚ । प्रेति॑ । वै । ए॒षः । अ॒स्मात् । लो॒कात् । च्य॒व॒ते॒ । यः । प॒शुम् । मृ॒त्यवे᳚ । नी॒यमा॑नम् । अ॒न्वा॒रभ॑त॒ इत्य॑नु - आ॒रभ॑ते । व॒पा॒श्रप॑णी॒ इति॑ वपा - श्रप॑णी । पुनः॑ । अ॒न्वार॑भत॒ इत्य॑नु - आर॑भते । अ॒स्मिन्न् । ए॒व । लो॒के । प्रतीति॑ । ति॒ष्ठ॒ति॒ । अ॒ग्निना᳚ । पु॒रस्ता᳚त् । ए॒ति॒ । रक्ष॑साम् । अप॑हत्या॒ इत्यप॑-ह॒त्यै॒ । अथो॒ इति॑ । दे॒वताः᳚ । ए॒व । ह॒व्येन॑ । \textbf{  48} \newline
                  \newline
                                \textbf{ TS 6.3.9.5} \newline
                  अन्विति॑ । ए॒ति॒ । न । अ॒न्त॒मम् । अङ्गा॑रम् । अतीति॑ । ह॒रे॒त् । यत् । अ॒न्त॒मम् । अङ्गा॑रम् । अ॒ति॒हरे॒दित्य॑ति - हरे᳚त् । दे॒वताः᳚ । अतीति॑ । म॒न्ये॒त॒ । वायो॒ इति॑ । वीति॑ । इ॒हि॒ । स्तो॒काना᳚म् । इति॑ । आ॒ह॒ । तस्मा᳚त् । विभ॑क्ता॒ इति॒ वि - भ॒क्ताः॒ । स्तो॒काः । अवेति॑ । प॒द्य॒न्ते॒ । अग्र᳚म् । वै । ए॒तत् । प॒शू॒नाम् । यत् । व॒पा । अग्र᳚म् । ओष॑धीनाम् । ब॒र्॒.हिः । अग्रे॑ण । ए॒व । अग्र᳚म् । समिति॑ । अ॒द्‌र्ध॒य॒ति॒ । अथो॒ इति॑ । ओष॑धीषु । ए॒व । प॒शून् । प्रतीति॑ । स्था॒प॒य॒ति॒ । स्वाहा॑कृतीभ्य॒ इति॒ स्वाहा॑कृति - भ्यः॒ । प्रेति॑ । इ॒ष्य॒ । इति॑ । आ॒ह॒ । \textbf{  49} \newline
                  \newline
                                \textbf{ TS 6.3.9.6} \newline
                  य॒ज्ञ्स्य॑ । समि॑ष्ट्या॒ इति॒ सं-इ॒ष्ट्यै॒ । प्रा॒णा॒पा॒नाविति॑ प्राण-अ॒पा॒नौ । वै । ए॒तौ । प॒शू॒नाम् । यत् । पृ॒ष॒दा॒ज्यमिति॑ पृषत् - आ॒ज्यम् । आ॒त्मा । व॒पा । पृ॒ष॒दा॒ज्यमिति॑ पृषत् - आ॒ज्यम् । अ॒भि॒घार्येत्य॑भि - घार्य॑ । व॒पाम् । अ॒भीति॑ । घा॒र॒य॒ति॒ । आ॒त्मन् । ए॒व । प॒शू॒नाम् । प्रा॒णा॒पा॒नाविति॑ प्राण - अ॒पा॒नौ । द॒धा॒ति॒ । स्वाहा᳚ । ऊ॒द्‌र्ध्वन॑भस॒मित्यू॒द्‌र्ध्व - न॒भ॒स॒म् । मा॒रु॒तम् । ग॒च्छ॒त॒म् । इति॑ । आ॒ह॒ । ऊ॒द्‌र्ध्वन॑भा॒ इत्यू॒द्‌र्ध्व - न॒भाः॒ । ह॒ । स्म॒ । वै । मा॒रु॒तः । दे॒वाना᳚म् । व॒पा॒श्रप॑णी॒ इति॑ वपा - श्रप॑णी । प्रेति॑ । ह॒र॒ति॒ । तेन॑ । ए॒व । ए॒ने॒ इति॑ । प्रेति॑ । ह॒र॒ति॒ । विषू॑ची॒ इति॑ । प्रेति॑ । ह॒र॒ति॒ । तस्मा᳚त् । विष्व॑ञ्चौ । प्रा॒णा॒पा॒नाविति॑ प्राण - अ॒पा॒नौ ॥ \textbf{  50} \newline
                  \newline
                      (स्वधि॑ति - श्चै॒वा - च्छि॑न्नो - ह॒व्येने॒ - ष्येत्या॑ह॒ - षट्च॑त्वारिꣳशच्च)  \textbf{(A9)} \newline \newline
                                \textbf{ TS 6.3.10.1} \newline
                  प॒शुम् । आ॒लभ्येत्या᳚ - लभ्य॑ । पु॒रो॒डाश᳚म् । निरिति॑ । व॒प॒ति॒ । समे॑ध॒मिति॒ स - मे॒ध॒म् । ए॒व । ए॒न॒म् । एति॑ । ल॒भ॒ते॒ । व॒पया᳚ । प्र॒चर्येति॑ प्र-चर्य॑ । पु॒रो॒डाशे॑न । प्रेति॑ । च॒र॒ति॒ । ऊर्क् । वै । पु॒रो॒डाशः॑ । ऊर्ज᳚म् । ए॒व । प॒शू॒नाम् । म॒द्ध्य॒तः । द॒धा॒ति॒ । अथो॒ इति॑ । प॒शोः । ए॒व । छि॒द्रम् । अपीति॑ । द॒धा॒ति॒ । पृ॒ष॒दा॒ज्यस्येति॑ पृषत् - आ॒ज्यस्य॑ । उ॒प॒हत्येत्यु॑प - हत्य॑ । त्रिः । पृ॒च्छ॒ति॒ । शृ॒तम् । ह॒वी(3)ः । श॒मि॒तः॒ । इति॑ । त्रिष॑त्या॒ इति॒ त्रि - स॒त्याः॒ । हि । दे॒वाः । यः । अशृ॑तम् । शृ॒तम् । आह॑ । सः । एन॑सा । प्रा॒णा॒पा॒नाविति॑ प्राण - अ॒पा॒नौ । वै । ए॒तौ । प॒शू॒नां । \textbf{  51} \newline
                  \newline
                                \textbf{ TS 6.3.10.2} \newline
                  यत् । पृ॒ष॒दा॒ज्यमिति॑ पृषत् - आ॒ज्यम् । प॒शोः । खलु॑ । वै । आल॑ब्ध॒स्येत्या - ल॒ब्ध॒स्य॒ । हृद॑यम् । आ॒त्मा । अ॒भि । समिति॑ । ए॒ति॒ । यत् । पृ॒ष॒दा॒ज्येनेति॑ पृषत् - आ॒ज्येन॑ । हृद॑यम् । अ॒भि॒घा॒रय॒तीत्य॑भि - घा॒रय॑ति । आ॒त्मन्न् । ए॒व । प॒शू॒नाम् । प्रा॒णा॒पा॒नाविति॑ प्राण - अ॒पा॒नौ । द॒धा॒ति॒ । प॒शुना᳚ । वै । दे॒वाः । सु॒व॒र्गमिति॑ सुवः - गम् । लो॒कम् । आ॒य॒न्न् । ते । अ॒म॒न्य॒न्त॒ । म॒नु॒ष्याः᳚ । नः॒ । अ॒न्वाभ॑विष्य॒न्तीत्य॑नु-आभ॑विष्यन्ति । इति॑ । तस्य॑ । शिरः॑ । छि॒त्त्वा । मेध᳚म् । प्रेति॑ । अ॒क्षा॒र॒य॒न्न् । सः । प्र॒क्षः । अ॒भ॒व॒त् । तत् । प्र॒क्षस्य॑ । प्र॒क्ष॒त्वमिति॑ प्रक्ष - त्वम् । यत् । प्ल॒क्ष॒शा॒खेति॑ प्लक्ष - शा॒खा । उ॒त्त॒र॒ब॒र्॒.हिरित्युत्त॑र - ब॒र॒.हिः । भव॑ति । समे॑ध॒स्येति॒ स - मे॒ध॒स्य॒ । ए॒व । \textbf{  52} \newline
                  \newline
                                \textbf{ TS 6.3.10.3} \newline
                  प॒शोः । अवेति॑ । द्य॒ति॒ । प॒शुम् । वै । ह्रि॒यमा॑णम् । रक्षाꣳ॑सि । अन्विति॑ । स॒च॒न्ते॒ । अ॒न्त॒रा । यूप᳚म् । च॒ । आ॒ह॒व॒नीय॒मित्या᳚ - ह॒व॒नीय᳚म् । च॒ । ह॒र॒ति॒ । रक्ष॑साम् । अप॑हत्या॒ इत्यप॑ - ह॒त्यै॒ । प॒शोः । वै । आल॑ब्ध॒स्येत्या - ल॒ब्ध॒स्य॒ । मनः॑ । अपेति॑ । क्रा॒म॒ति॒ । म॒नोता॑यै । ह॒विषः॑ । अ॒व॒दी॒यमा॑न॒स्येत्य॑व - दी॒यमा॑नस्य । अन्विति॑ । ब्रू॒हि॒ । इति॑ । आ॒ह॒ । मनः॑ । ए॒व । अ॒स्य॒ । अवेति॑ । रु॒न्धे॒ । एका॑दश । अ॒व॒दाना॒नीत्य॑व - दाना॑नि । अवेति॑ । द्य॒ति॒ । दश॑ । वै । प॒शोः । प्रा॒णा इति॑ प्र - अ॒नाः । आ॒त्मा । ए॒का॒द॒शः । यावान्॑ । ए॒व । प॒शुः । तस्य॑ । अवेति॑ । \textbf{  53} \newline
                  \newline
                                \textbf{ TS 6.3.10.4} \newline
                  द्य॒ति॒ । हृद॑यस्य । अग्रे᳚ । अवेति॑ । द्य॒ति॒ । अथ॑ । जि॒ह्वायाः᳚ । अथ॑ । वक्ष॑सः । यत् । वै । हृद॑येन । अ॒भि॒गच्छ॒तीत्य॑भि - गच्छ॑ति । तत् । जि॒ह्वया᳚ । व॒द॒ति॒ । यत् । जि॒ह्वया᳚ । वद॑ति । तत् । उर॑सः । अधि॑ । निरिति॑ । व॒द॒ति॒ । ए॒तत् । वै । प॒शोः । य॒था॒पू॒र्वमिति॑ यथा - पू॒र्वम् । यस्य॑ । ए॒वम् । अ॒व॒दायेत्य॑व - दाय॑ । य॒था॒काम॒मिति॑ यथा - काम᳚म् । उत्त॑रेषा॒मित्युत् - त॒रे॒षा॒म् । अ॒व॒द्यतीत्य॑व - द्यति॑ । य॒था॒पू॒र्वमिति॑ यथा - पू॒र्वम् । ए॒व । अ॒स्य॒ । प॒शोः । अव॑त्तम् । भ॒व॒ति॒ । म॒द्ध्य॒तः । गु॒दस्य॑ । अवेति॑ । द्य॒ति॒ । म॒द्ध्य॒तः । हि । प्रा॒ण इति॑ प्र - अ॒नः । उ॒त्त॒मस्येत्यु॑त् - त॒मस्य॑ । अवेति॑ । द्य॒ति॒ । \textbf{  54} \newline
                  \newline
                                \textbf{ TS 6.3.10.5} \newline
                  उ॒त्त॒म इत्यु॑त् - त॒मः । हि । प्रा॒ण इति॑ प्र - अ॒नः । यदि॑ । इत॑रम् । यदि॑ । इत॑रम् । उ॒भय᳚म् । ए॒व । अजा॑मि । जाय॑मानः । वै । ब्रा॒ह्म॒णः । त्रि॒भिरिति॑ त्रि - भिः । ऋ॒ण॒वेत्यृ॑ण - वा । जा॒य॒ते॒ । ब्र॒ह्म॒चर्ये॒णेति॑ ब्रह्म - चर्ये॑ण । ऋषि॑भ्य॒ इत्यृषि॑ - भ्यः॒ । य॒ज्ञेन॑ । दे॒वेभ्यः॑ । प्र॒जयेति॑ प्र - जया᳚ । पि॒तृभ्य॒ इति॑ पि॒तृ - भ्यः॒ । ए॒षः । वै । अ॒नृ॒णः । यः । पु॒त्री । यज्वा᳚ । ब्र॒ह्म॒चा॒रि॒वा॒सीति॑ ब्रह्मचारि - वा॒सी । तत् । अ॒व॒दानै॒रित्य॑व - दानैः᳚ । ए॒व । अवेति॑ । द॒य॒ते॒ । तत् । अ॒व॒दाना॑ना॒मित्य॑व - दाना॑नाम् । अ॒व॒दा॒न॒त्वमित्य॑वदान - त्वम् । दे॒वा॒सु॒रा इति॑ देव - अ॒सु॒राः । संॅय॑त्ता॒ इति॒ सं - य॒त्ताः॒ । आ॒स॒न्न् । ते । दे॒वाः । अ॒ग्निम् । अ॒ब्रु॒व॒न्न् । त्वया᳚ । वी॒रेण॑ । असु॑रान् । अ॒भीति॑ । भ॒वा॒म॒ । इति॑ । \textbf{  55} \newline
                  \newline
                                \textbf{ TS 6.3.10.6} \newline
                  सः । अ॒ब्र॒वी॒त् । वर᳚म् । वृ॒णै॒ । प॒शोः । उ॒द्धा॒रमित्यु॑त् - हा॒रम् । उदिति॑ । ह॒रै॒ । इति॑ । सः । ए॒तम् । उ॒द्धा॒रमित्यु॑त् - हा॒रम् । उदिति॑ । अ॒ह॒र॒त॒ । दोः । पू॒र्वा॒द्‌र्धस्येति॑ पूर्व - अ॒द्‌र्धस्य॑ । गु॒दम् । म॒द्ध्य॒तः । श्रोणि᳚म् । ज॒घ॒ना॒द्‌र्धस्येति॑ जघन-अ॒द्‌र्धस्य॑ । ततः॑ । दे॒वाः । अभ॑वन्न् । परेति॑ । असु॑राः । यत् । त्र्य॒ङ्गाणा॒मिति॑ त्रि - अ॒ङ्गाना᳚म् । स॒म॒व॒द्यतीति॑ सं - अ॒व॒द्यति॑ । भ्रातृ॑व्याभिभूत्या॒ इति॒ भ्रातृ॑व्य-अ॒भि॒भू॒त्यै॒ । भव॑ति । आ॒त्मना᳚ । परेति॑ । अ॒स्य॒ । भ्रातृ॑व्यः । भ॒व॒ति॒ । अ॒क्ष्ण॒या । अवेति॑ । द्य॒ति॒ । तस्मा᳚त् । अ॒क्ष्ण॒या । प॒शवः॑ । अङ्गा॑नि । प्रेति॑ । ह॒र॒न्ति॒ । प्रति॑ष्ठित्या॒ इति॒ प्रति॑ - स्थि॒त्यै॒ ॥ \textbf{  56 } \newline
                  \newline
                      (ए॒तौ प॑शू॒नाꣳ - समे॑धस्यै॒व - तस्याऽवो᳚ - त्त॒मस्याव॑ द्य॒ती - ति॒ - पञ्च॑चत्वारिꣳशच्च)  \textbf{(A10)} \newline \newline
                                \textbf{ TS 6.3.11.1} \newline
                  मेद॑सा । स्रुचौ᳚ । प्रेति॑ । ऊ॒र्णो॒ति॒ । मेदो॑रूपा॒ इति॒ मेदः॑ - रू॒पाः॒ । वै । प॒शवः॑ । रू॒पम् । ए॒व । प॒शुषु॑ । द॒धा॒ति॒ । यू॒षन्न् । अ॒व॒धायेत्य॑व-धाय॑ । प्रेति॑ । ऊ॒र्णो॒ति॒ । रसः॑ । वै । ए॒षः । प॒शू॒नाम् । यत् । यूः । रस᳚म् । ए॒व । प॒शुषु॑ । द॒धा॒ति॒ । पा॒र्श्वेन॑ । व॒सा॒हो॒ममिति॑ वसा - हो॒मम् । प्रेति॑ । यौ॒ति॒ । मद्ध्य᳚म् । वै । ए॒तत् । प॒शू॒नाम् । यत् । पा॒र्श्वम् । रसः॑ । ए॒षः । प॒शू॒नाम् । यत् । वसा᳚ । यत् । पा॒र्श्वेन॑ । व॒सा॒हो॒ममिति॑ वसा - हो॒मम् । प्र॒यौतीति॑ प्र - यौति॑ । म॒द्ध्य॒तः । ए॒व । प॒शू॒नाम् । रस᳚म् । द॒धा॒ति॒ । घ्नन्ति॑ । \textbf{  57} \newline
                  \newline
                                \textbf{ TS 6.3.11.2} \newline
                  वै । ए॒तत् । प॒शुम् । यत् । स॒ज्ञ्ं॒पय॒न्तीति॑ सं - ज्ञ्॒पय॑न्ति । ऐ॒न्द्रः । खलु॑ । वै । दे॒वत॑या । प्रा॒ण इति॑ प्र - अ॒नः । ऐ॒न्द्रः । अ॒पा॒न इत्य॑प -अ॒नः । ऐ॒न्द्रः । प्रा॒ण इति॑ प्र - अ॒नः । अङ्गे॑अङ्ग॒ इत्यङ्गे᳚ - अ॒ङ्गे॒ । नीति॑ । दे॒द्ध्य॒त् । इति॑ । आ॒ह॒ । प्रा॒णा॒पा॒नाविति॑ प्राण - अ॒पा॒नौ । ए॒व । प॒शुषु॑ । द॒धा॒ति॒ । देव॑ । त्व॒ष्टः॒ । भूरि॑ । ते॒ । सꣳस॒मिति॒ सं-स॒म् । ए॒तु॒ । इति॑ । आ॒ह॒ । त्वा॒ष्ट्राः । हि । दे॒वत॑या । प॒शवः॑ । विषु॑रूपा॒ इति॒ विषु॑ - रू॒पाः॒ । यत् । सल॑क्ष्माण॒ इति॒ स-ल॒क्ष्मा॒णः॒ । भव॑थ । इति॑ । आ॒ह॒ । विषु॑रूपा॒ इति॒ विषु॑-रू॒पाः॒ । हि । ए॒ते । सन्तः॑ । सल॑क्ष्माण॒ इति॒ स - ल॒क्ष्मा॒णः॒ । ए॒तर्.हि॑ । भव॑न्ति । दे॒व॒त्रेति॑ देव - त्रा । यन्त᳚म् । \textbf{  58} \newline
                  \newline
                                \textbf{ TS 6.3.11.3} \newline
                  अव॑से । सखा॑यः । अन्विति॑ । त्वा॒ । मा॒ता । पि॒तरः॑ । म॒द॒न्तु॒ । इति॑ । आ॒ह॒ । अनु॑मत॒मित्यनु॑ - म॒त॒म् । ए॒व । ए॒न॒म् । मा॒त्रा । पि॒त्रा । सु॒व॒र्गमिति॑ सुवः - गम् । लो॒कम् । ग॒म॒य॒ति॒ । अ॒द्‌र्ध॒र्च इत्य॑द्‌र्ध-ऋ॒चे । व॒सा॒हो॒ममिति॑ वसा - हो॒मम् । जु॒हो॒ति॒ । अ॒सौ । वै । अ॒द्‌र्ध॒र्च इत्य॑द्‌र्ध-ऋ॒चः । इ॒यम् । अ॒द्‌र्ध॒र्च इत्य॑द्‌र्ध - ऋ॒चः । इ॒मे इति॑ । ए॒व । रसे॑न । अ॒न॒क्ति॒ । दिशः॑ । जु॒हो॒ति॒ । दिशः॑ । ए॒व । रसे॑न । अ॒न॒क्ति॒ । अथो॒ इति॑ । दि॒ग्भ्य इति॑ दिक् - भ्यः । ए॒व । ऊर्ज᳚म् । रस᳚म् । अवेति॑ । रु॒न्धे॒ । प्रा॒णा॒पा॒नाविति॑ प्राण - अ॒पा॒नौ । वै । ए॒तौ । प॒शू॒नाम् । यत् । पृ॒ष॒दा॒ज्यमिति॑ पृषत् - आ॒ज्यम् । वा॒न॒स्प॒त्याः । खलु॑ । \textbf{  59} \newline
                  \newline
                                \textbf{ TS 6.3.11.4} \newline
                  वै । दे॒वत॑या । प॒शवः॑ । यत् । पृ॒ष॒दा॒ज्यस्येति॑ पृषत् - आ॒ज्यस्य॑ । उ॒प॒हत्येत्यु॑प - हत्य॑ । आह॑ । वन॒स्पत॑ये । अन्विति॑ । ब्रू॒हि॒ । वन॒स्पत॑ये । प्रेति॑ । इ॒ष्य॒ । इति॑ । प्रा॒णा॒पा॒नाविति॑ प्राण - अ॒पा॒नौ । ए॒व । प॒शुषु॑ । द॒धा॒ति॒ । अ॒न्यस्या᳚न्य॒स्येत्य॒न्यस्य॑ - अ॒न्य॒स्य॒ । स॒म॒व॒त्तमिति॑ सं - अ॒व॒त्तम् । स॒मव॑द्य॒तीति॑ सं-अव॑द्यति । तस्मा᳚त् । नाना॑रूपा॒ इति॒ नाना᳚-रू॒पाः । प॒शवः॑ । यू॒ष्णा । उपेति॑ । सि॒ञ्च॒ति॒ । रसः॑ । वै । ए॒षः । प॒शू॒नाम् । यत् । यूः । रस᳚म् । ए॒व । प॒शुषु॑ । द॒धा॒ति॒ । इडा᳚म् । उपेति॑ । ह्व॒य॒ते॒ । प॒शवः॑ । वै । इडा᳚ । प॒शून् । ए॒व । उपेति॑ । ह्व॒य॒ते॒ । च॒तुः । उपेति॑ । ह्व॒य॒ते॒ । \textbf{  60} \newline
                  \newline
                                \textbf{ TS 6.3.11.5} \newline
                  चतु॑ष्पाद॒ इति॒ चतुः॑ - पा॒दः॒ । हि । प॒शवः॑ । यम् । का॒मये॑त । अ॒प॒शुः । स्या॒त् । इति॑ । अ॒मे॒दस्क॒मित्य॑मे॒दः-क॒म् । तस्मै᳚ । एति॑ । द॒द्ध्या॒त् । मेदो॑रूपा॒ इति॒ मेदः॑-रू॒पाः॒ । वै । प॒शवः॑ । रू॒पेण॑ । ए॒व । ए॒न॒म् । प॒शुभ्य॒ इति॑ प॒शु-भ्यः॒ । निरिति॑ । भ॒ज॒ति॒ । अ॒प॒शुः । ए॒व । भ॒व॒ति॒ । यम् । का॒मये॑त । प॒शु॒मानिति॑ पशु - मान् । स्या॒त् । इति॑ । मेद॑स्वत् । तस्मै᳚ । एति॑ । द॒द्ध्या॒त् । मेदो॑रूपा॒ इति॒ मेदः॑ - रू॒पाः॒ । वै । प॒शवः॑ । रू॒पेण॑ । ए॒व । अ॒स्मै॒ । प॒शून् । अवेति॑ । रु॒न्धे॒ । प॒शु॒मानिति॑ पशु - मान् । ए॒व । भ॒व॒ति॒ । प्र॒जाप॑ति॒रिति॑ प्र॒जा-प॒तिः॒ । य॒ज्ञ्म् । अ॒सृ॒ज॒त॒ । सः । आज्यं᳚ । \textbf{  61} \newline
                  \newline
                                \textbf{ TS 6.3.11.6} \newline
                  पु॒रस्ता᳚त् । अ॒सृ॒ज॒त॒ । प॒शुम् । म॒द्ध्य॒तः । पृ॒ष॒दा॒ज्यमिति॑ पृषत् - आ॒ज्यम् । प॒श्चात् । तस्मा᳚त् । आज्ये॑न । प्र॒या॒जा इति॑ प्र - या॒जाः । इ॒ज्य॒न्ते॒ । प॒शुना᳚ । म॒द्ध्य॒तः । पृ॒ष॒दा॒ज्येनेति॑ पृषत् - आ॒ज्येन॑ । अ॒नू॒या॒जा इत्य॑नु - या॒जाः । तस्मा᳚त् । ए॒तत् । मि॒श्रम् । इ॒व॒ । प॒श्चा॒थ्सृ॒ष्टमिति॑ पश्चात् - सृ॒ष्टम् । हि । एका॑दश । अ॒नू॒या॒जानित्य॑नु - या॒जान् । य॒ज॒ति॒ । दश॑ । वै । प॒शोः । प्रा॒णा इति॑ प्र - अ॒नाः । आ॒त्मा । ए॒का॒द॒शः । यावान्॑ । ए॒व । प॒शुः । तम् । अन्विति॑ । य॒ज॒ति॒ । घ्नन्ति॑ । वै । ए॒तत् । प॒शुम् । यत् । स॒ज्ञ्ं॒पय॒न्तीति॑ सं - ज्ञ्॒पय॑न्ति । प्रा॒णा॒पा॒नाविति॑ प्राण - अ॒पा॒नौ । खलु॑ । वै । ए॒तौ । प॒शू॒नाम् । यत् । पृ॒ष॒दा॒ज्यमिति॑ पृषत् - आ॒ज्यम् । यत् । पृ॒ष॒दा॒ज्येनेति॑ पृषत् - आ॒ज्येन॑ ( ) । अ॒नू॒या॒जानित्य॑नु - या॒जान् । यज॑ति । प्रा॒णा॒पा॒नाविति॑ प्राण - अ॒पा॒नौ । ए॒व । प॒शुषु॑ । द॒धा॒ति॒ ॥ \textbf{  62 } \newline
                  \newline
                      (घ्नन्ति॒ - यन्तं॒ - खलु॑ - च॒तुरुप॑ ह्वयत॒ - आज्यं॒ - ॅयत् पृ॑षदा॒ज्येन॒ - षट् च॑)  \textbf{(A11)} \newline \newline
\textbf{praSna korvai with starting padams of 1 to 11 anuvAkams :-} \newline
(चात्वा॑लाथ् - सुव॒र्गाय॒ यद् वै॑सर्ज॒नानि॑ - वैष्ण॒व्यर्चा - पृ॑थि॒व्यै - सा॒ध्या - इ॒षे त्वे - त्य॒ग्निना॒ - पर्य॑ग्नि - प॒शोः - प॒शुमा॒लभ्य॒ - मेद॑सा॒ स्रुचा॒ - वेका॑दश) \newline

\textbf{korvai with starting padams of1, 11, 21 series of pa~jcAtis :-} \newline
(चात्वा॑लाद् - दे॒वानु॒पैति॑ - मुञ्चति - प्रह्रि॒यमा॑णाय॒ - पर्य॑ग्नि - प॒शुमा॒लभ्य॒ - चतु॑ष्पादो॒ - द्विष॑ष्टिः) \newline

\textbf{first and last padam of third praSnam of 6th KANDam} \newline
(चात्वा॑लात् - प॒शुषु॑ दधाति) \newline 


॥ हरिः॑ ॐ ॥
॥ कृष्ण यजुर्वेदीय तैत्तिरीय संहितायां षष्ठकाण्डे तृतीयः प्रश्नः समाप्तः ॥ \newline
\pagebreak
\pagebreak
        


\end{document}
