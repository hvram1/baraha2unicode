\documentclass[17pt]{extarticle}
\usepackage{babel}
\usepackage{fontspec}
\usepackage{polyglossia}
\usepackage{extsizes}



\setmainlanguage{sanskrit}
\setotherlanguages{english} %% or other languages
\setlength{\parindent}{0pt}
\pagestyle{myheadings}
\newfontfamily\devanagarifont[Script=Devanagari]{AdishilaVedic}


\newcommand{\VAR}[1]{}
\newcommand{\BLOCK}[1]{}




\begin{document}
\begin{titlepage}
    \begin{center}
 
\begin{sanskrit}
    { \Large
    ॐ नमः परमात्मने, श्री महागणपतये नमः, श्री गुरुभ्यो नमः ॥ ह॒रिः॒ ॐ 
    }
    \\
    \vspace{2.5cm}
    \mbox{ \Huge
    5.2      पञ्चमकाण्डे द्वितीयः प्रश्नः - चित्युपक्रमाभिधानं   }
\end{sanskrit}
\end{center}

\end{titlepage}
\tableofcontents

ॐ नमः परमात्मने, श्री महागणपतये नमः, श्री गुरुभ्यो नमः
ह॒रिः॒ ॐ \newline
5.2      पञ्चमकाण्डे द्वितीयः प्रश्नः - चित्युपक्रमाभिधानं \newline

\addcontentsline{toc}{section}{ 5.2      पञ्चमकाण्डे द्वितीयः प्रश्नः - चित्युपक्रमाभिधानं}
\markright{ 5.2      पञ्चमकाण्डे द्वितीयः प्रश्नः - चित्युपक्रमाभिधानं \hfill https://www.vedavms.in \hfill}
\section*{ 5.2      पञ्चमकाण्डे द्वितीयः प्रश्नः - चित्युपक्रमाभिधानं }
                                \textbf{ TS 5.2.1.1} \newline
                  विष्णु॑मुखा॒ इति॒ विष्णु॑ - मु॒खाः॒ । वै । दे॒वाः । छन्दो॑भि॒रिति॒ छन्दः॑ - भिः॒ । इ॒मान् । लो॒कान् । अ॒न॒प॒ज॒य्यमित्य॑नप - ज॒य्यम् । अ॒भीति॑ । अ॒ज॒य॒न्न् । यत् । वि॒ष्णु॒क्र॒मानिति॑ विष्णु-क्र॒मान् । क्रम॑ते । विष्णुः॑ । ए॒व । भू॒त्वा । यज॑मानः । छन्दो॑भि॒रिति॒ छन्दः॑ - भिः॒ । इ॒मान् । लो॒कान् । अ॒न॒प॒ज॒य्यमित्य॑नप-ज॒य्यम् । अ॒भीति॑ । ज॒य॒ति॒ । विष्णोः᳚ । क्रमः॑ । अ॒सि॒ । अ॒भि॒मा॒ति॒हेत्य॑भिमाति-हा । इति॑ । आ॒ह॒ । गा॒य॒त्री । वै । पृ॒थि॒वी । त्रैष्टु॑भम् । अ॒न्तरि॑क्षम् । जाग॑ती । द्यौः । आनु॑ष्टुभी॒रित्यानु॑ - स्तु॒भीः॒ । दिशः॑ । छन्दो॑भि॒रिति॒ छन्दः॑ - भिः॒ । ए॒व । इ॒मान् । लो॒कान् । य॒था॒पू॒र्वमिति॑ यथा - पू॒र्वम् । अ॒भीति॑ । ज॒य॒ति॒ । प्र॒जाप॑ति॒रिति॑ प्र॒जा - प॒तिः॒ । अ॒ग्निम् । अ॒सृ॒ज॒त॒ । सः॒ । अ॒स्मा॒त् । सृ॒ष्टः । \textbf{  1} \newline
                  \newline
                                \textbf{ TS 5.2.1.2} \newline
                  पराङ्॑ । ऐ॒त् । तम् । ए॒तया᳚ । अन्विति॑ । ऐ॒त् । अक्र॑न्दत् । इति॑ । तया᳚ । वै । सः । अ॒ग्नेः । प्रि॒यम् । धाम॑ । अवेति॑ । अ॒रु॒न्ध॒ । यत् । ए॒ताम् । अ॒न्वाहेत्य॑नु-आह॑ । अ॒ग्नेः । ए॒व । ए॒तया᳚ । प्रि॒यम् । धाम॑ । अवेति॑ । रु॒न्धे॒ । ई॒श्व॒रः । वै । ए॒षः । पराङ्॑ । प्र॒दघ॒ इति॑ प्र -दघः॑ । यः । वि॒ष्णु॒क्र॒मानिति॑ विष्णु - क्र॒मान् । क्रम॑ते । च॒त॒सृभि॒रिति॑ चत॒सृ - भिः॒ । एति॑ । व॒र्त॒ते॒ । च॒त्वारि॑ । छन्दाꣳ॑सि । छन्दाꣳ॑सि । खलु॑ । वै । अ॒ग्नेः । प्रि॒या । त॒नूः । प्रि॒याम् । ए॒व । अ॒स्य॒ । त॒नुव᳚म् । अ॒भीति॑ । \textbf{  2} \newline
                  \newline
                                \textbf{ TS 5.2.1.3} \newline
                  प॒र्याव॑र्तत॒ इति॑ परि - आव॑र्तते । द॒क्षि॒णा । प॒र्याव॑र्तत॒ इति॑ परि - आव॑र्तते । स्वम् । ए॒व । वी॒र्य᳚म् । अन्विति॑ । प॒र्याव॑र्तत॒ इति॑ परि - आव॑र्तते । तस्मा᳚त् । दक्षि॑णः । अद्‌र्धः॑ । आ॒त्मनः॑ । वी॒र्या॑वत्तर॒ इति॑ वी॒र्या॑वत् - त॒रः॒ । अथो॒ इति॑ । आ॒दि॒त्यस्य॑ । ए॒व । आ॒वृत॒मित्या᳚ - वृत᳚म् । अन्विति॑ । प॒र्याव॑र्तत॒ इति॑ परि - आव॑र्तते । शुन॒श्शेप᳚म् । आजी॑गर्तिम् । वरु॑णः । अ॒गृ॒ह्णा॒त् । सः । ए॒ताम् । वा॒रु॒णीम् । अ॒प॒श्य॒त् । तया᳚ । वै । सः । आ॒त्मान᳚म् । व॒रु॒ण॒पा॒शादिति॑ वरुण - पा॒शात् । अ॒मु॒ञ्च॒त् । वरु॑णः । वै । ए॒तम् । गृ॒ह्णा॒ति॒ । यः । उ॒खाम् । प्र॒ति॒मु॒ञ्चत॒ इति॑ प्रति - मु॒ञ्चते᳚ । उदिति॑ । उ॒त्त॒ममित्यु॑त् - त॒मम् । व॒रु॒ण॒ । पाश᳚म् । अ॒स्मत् । इति॑ । आ॒ह॒ । आ॒त्मान᳚म् । ए॒व । ए॒तया᳚ । \textbf{  3} \newline
                  \newline
                                \textbf{ TS 5.2.1.4} \newline
                  व॒रु॒ण॒पा॒शादिति॑ वरुण-पा॒शात् । मु॒ञ्च॒ति॒ । एति॑ । त्वा॒ । अ॒हा॒र्॒.ष॒म् । इति॑ । आ॒ह॒ । एति॑ । हि । ए॒न॒म् । हर॑ति । ध्रु॒वः । ति॒ष्ठ॒ । अवि॑चाचलि॒रित्यवि॑ - चा॒च॒लिः॒ । इति॑ । आ॒ह॒ । प्रति॑ष्ठित्या॒ इति॒ प्रति॑ - स्थि॒त्यै॒ । विशः॑ । त्वा॒ । सर्वाः᳚ । वा॒ञ्छ॒न्तु॒ । इति॑ । आ॒ह॒ । वि॒शा । ए॒व । ए॒न॒म् । समिति॑ । अ॒द्‌र्ध॒य॒ति॒ । अ॒स्मिन्न् । रा॒ष्ट्रम् । अधीति॑ । श्र॒य॒ । इति॑ । आ॒ह॒ । रा॒ष्ट्रम् । ए॒व । अ॒स्मि॒न्न् । ध्रु॒वम् । अ॒कः॒ । यम् । का॒मये॑त । रा॒ष्ट्रम् । स्या॒त् । इति॑ । तम् । मन॑सा । ध्या॒ये॒त् । रा॒ष्ट्रम् । ए॒व । भ॒व॒ति॒ । \textbf{  4} \newline
                  \newline
                                \textbf{ TS 5.2.1.5} \newline
                  अग्रे᳚ । बृ॒हन्न् । उ॒षसा᳚म् । ऊ॒द्‌र्ध्वः । अ॒स्था॒त् । इति॑ । आ॒ह॒ । अग्र᳚म् । ए॒व । ए॒न॒म् । स॒मा॒नाना᳚म् । क॒रो॒ति॒ । नि॒र्ज॒ग्मि॒वानिति॑ निः-ज॒ग्मि॒वान् । तम॑सः । इति॑ । आ॒ह॒ । तमः॑ । ए॒व । अ॒स्मा॒त् । अपेति॑ । ह॒न्ति॒ । ज्योति॑षा । एति॑ । अ॒गा॒त् । इति॑ । आ॒ह॒ । ज्योतिः॑ । ए॒व । अ॒स्मि॒न्न् । द॒धा॒ति॒ । च॒त॒सृभि॒रिति॑ चत॒सृ - भिः॒ । सा॒द॒य॒ति॒ । च॒त्वारि॑ । छन्दाꣳ॑सि । छन्दो॑भि॒रिति॒ छन्दः॑ - भिः॒ । ए॒व । अति॑च्छन्द॒सेत्यति॑ - छ॒न्द॒सा॒ । उ॒त्त॒मयेत्यु॑त् - त॒मया᳚ । वर्ष्म॑ । वै । ए॒षा । छन्द॑साम् । यत् । अति॑च्छन्दा॒ इत्यति॑ - छ॒न्दाः॒ । वर्ष्म॑ । ए॒व । ए॒न॒म् । स॒मा॒नाना᳚म् । क॒रो॒ति॒ । सद्व॒तीति॒ सत् - व॒ती॒ । \textbf{  5} \newline
                  \newline
                                \textbf{ TS 5.2.1.6} \newline
                  भ॒व॒ति॒ । स॒त्त्वमिति॑ सत् - त्वम् । ए॒व । ए॒न॒म् । ग॒म॒य॒ति॒ । वा॒थ्स॒प्रेणेति॑ वाथ्स - प्रेण॑ । उपेति॑ । ति॒ष्ठ॒ते॒ । ए॒तेन॑ । वै । व॒थ्स॒प्रीरिति॑ वथ्स - प्रीः । भा॒ल॒न्द॒नः । अ॒ग्नेः । प्रि॒यम् । धाम॑ । अवेति॑ । अ॒रु॒न्ध॒ । अ॒ग्नेः । ए॒व । ए॒तेन॑ । प्रि॒यम् । धाम॑ । अवेति॑ । रु॒न्धे॒ । ए॒का॒द॒शम् । भ॒व॒ति॒ । ए॒क॒धेत्ये॑क - धा । ए॒व । यज॑माने । वी॒र्य᳚म् । द॒धा॒ति॒ । स्तोमे॑न । वै । दे॒वाः । अ॒स्मिन्न् । लो॒के । आ॒द्‌र्ध्नु॒व॒न्न् । छन्दो॑भि॒रिति॒ छन्दः॑ - भिः॒ । अ॒मुष्मिन्न्॑ । स्तोम॑स्य । इ॒व॒ । खलु॑ । वै । ए॒तत् । रू॒पम् । यत् । वा॒थ्स॒प्रमिति॑ वाथ्स-प्रम् । यत् । वा॒थ्स॒प्रेणेति॑ वाथ्स-प्रेण॑ । उ॒प॒तिष्ठ॑त॒ इत्यु॑प-तिष्ठ॑ते । \textbf{  6} \newline
                  \newline
                                \textbf{ TS 5.2.1.7} \newline
                  इ॒मम् । ए॒व । तेन॑ । लो॒कम् । अ॒भीति॑ । ज॒य॒ति॒ । यत् । वि॒ष्णु॒क्र॒मानिति॑ विष्णु - क्र॒मान् । क्रम॑ते । अ॒मुम् । ए॒व । तैः । लो॒कम् । अ॒भीति॑ । ज॒य॒ति॒ । पू॒र्वे॒द्युः । प्रेति॑ । क्रा॒म॒ति॒ । उ॒त्त॒रे॒द्युः । उपेति॑ । ति॒ष्ठ॒ते॒ । तस्मा᳚त् । योगे᳚ । अ॒न्यासा᳚म् । प्र॒जाना॒मिति॑ प्र - जाना᳚म् । मनः॑ । क्षेमे᳚ । अ॒न्यासा᳚म् । तस्मा᳚त् । या॒या॒व॒रः । क्षे॒म्यस्य॑ । ई॒शे॒ । तस्मा᳚त् । या॒या॒व॒रः । क्षे॒म्यम् । अ॒द्ध्यव॑स्य॒तीत्य॑धि - अव॑स्यति । मु॒ष्टी इति॑ । क॒रो॒ति॒ । वाच᳚म् । य॒च्छ॒ति॒ । य॒ज्ञ्स्य॑ । धृत्यै᳚ ॥ \textbf{  7 } \newline
                  \newline
                      (सृ॒ष्टो᳚ऽ - (1॒) भ्ये॑ - तया॑ - भवति॒ - सद्व॑त्यु - प॒तिष्ठ॑ते॒ - द्विच॑त्वारिꣳशच्च)  \textbf{(A1)} \newline \newline
                                \textbf{ TS 5.2.2.1} \newline
                  अन्न॑पत॒ इत्यन्न॑ - प॒ते॒ । अन्न॑स्य । नः॒ । दे॒हि॒ । इति॑ । आ॒ह॒ । अ॒ग्निः । वै । अन्न॑पति॒रित्यन्न॑ - प॒तिः॒ । सः । ए॒व । अ॒स्मै॒ । अन्न᳚म् । प्रेति॑ । य॒च्छ॒ति॒ । अ॒न॒मी॒वस्य॑ । शु॒ष्मिणः॑ । इति॑ । आ॒ह॒ । अ॒य॒क्ष्मस्य॑ । इति॑ । वाव । ए॒तत् । आ॒ह॒ । प्रेति॑ । प्र॒दा॒तार॒मिति॑ प्र - दा॒तार᳚म् । ता॒रि॒षः॒ । ऊर्ज᳚म् । नः॒ । धे॒हि॒ । द्वि॒पद॒ इति॑ द्वि - पदे᳚ । चतु॑ष्पद॒ इति॒ चतुः॑ - प॒दे॒ । इति॑ । आ॒ह॒ । आ॒शिष॒मित्या᳚ - शिष᳚म् । ए॒व । ए॒ताम् । एति॑ । शा॒स्ते॒ । उदिति॑ । उ॒ । त्वा॒ । विश्वे᳚ । दे॒वाः । इति॑ । आ॒ह॒ । प्रा॒णा इति॑ प्र-अ॒नाः । वै । विश्वे᳚ । दे॒वाः । \textbf{  8} \newline
                  \newline
                                \textbf{ TS 5.2.2.2} \newline
                  प्रा॒णैरिति॑ प्र - अ॒नैः । ए॒व । ए॒न॒म् । उदिति॑ । य॒च्छ॒ते॒ । अग्ने᳚ । भर॑न्तु । चित्ति॑भि॒रिति॒ चित्ति॑ - भिः॒ । इति॑ । आ॒ह॒ । यस्मै᳚ । ए॒व । ए॒न॒म् । चि॒त्ताय॑ । उ॒द्यच्छ॑त॒ इत्यु॑त् - यच्छ॑ते । तेन॑ । ए॒व । ए॒न॒म् । समिति॑ । अ॒द्‌र्ध॒य॒ति॒ । च॒त॒सृभि॒रिति॑ चत॒सृ - भिः॒ । एति॑ । सा॒द॒य॒ति॒ । च॒त्वारि॑ । छन्दाꣳ॑सि । छन्दो॑भि॒रिति॒ छन्दः॑ - भिः॒ । ए॒व । अति॑च्छन्द॒सेत्यति॑ - छ॒न्द॒सा॒ । उ॒त्त॒मयेत्यु॑त् - त॒मया᳚ । वर्ष्म॑ । वै । ए॒षा । छन्द॑साम् । यत् । अति॑च्छन्दा॒ इत्यति॑ - छ॒न्दाः॒ । वर्ष्म॑ । ए॒व । ए॒न॒म् । स॒मा॒नाना᳚म् । क॒रो॒ति॒ । सद्व॒तीति॒ सत् - व॒ती॒ । भ॒व॒ति॒ । स॒त्त्वमिति॑ सत्-त्वम् । ए॒व । ए॒न॒म् । ग॒म॒य॒ति॒ । प्रेति॑ । इत् । अ॒ग्ने॒ । ज्योति॑ष्मान् । \textbf{  9} \newline
                  \newline
                                \textbf{ TS 5.2.2.3} \newline
                  या॒हि॒ । इति॑ । आ॒ह॒ । ज्योतिः॑ । ए॒व । अ॒स्मि॒न्न् । द॒धा॒ति॒ । त॒नुवा᳚ । वै । ए॒षः । हि॒न॒स्ति॒ । यम् । हि॒नस्ति॑ । मा । हिꣳ॒॒सीः॒ । त॒नुवा᳚ । प्र॒जा इति॑ प्र - जाः । इति॑ । आ॒ह॒ । प्र॒जाभ्य॒ इति॑ प्र - जाभ्यः॑ । ए॒व । ए॒न॒म् । श॒म॒य॒ति॒ । रक्षाꣳ॑सि । वै । ए॒तत् । य॒ज्ञ्म् । स॒च॒न्ते॒ । यत् । अनः॑ । उ॒थ्सर्ज॒तीत्यु॑त्-सर्ज॑ति । अक्र॑न्दत् । इति॑ । अन्विति॑ । आ॒ह॒ । रक्ष॑साम् । अप॑हत्या॒ इत्यप॑ - ह॒त्यै॒ । अन॑सा । व॒ह॒न्ति॒ । अप॑चिति॒मित्यप॑ - चि॒ति॒म् । ए॒व । अ॒स्मि॒न्न् । द॒धा॒ति॒ । तस्मा᳚त् । अ॒न॒स्वी । च॒ । र॒थी । च॒ । अति॑थीनाम् । अप॑चिततमा॒वित्यप॑चित - त॒मौ॒ । \textbf{  10} \newline
                  \newline
                                \textbf{ TS 5.2.2.4} \newline
                  अप॑चितिमा॒नित्यप॑चिति - मा॒न् । भ॒व॒ति॒ । यः । ए॒वम् । वेद॑ । स॒मिधेति॑ सं - इधा᳚ । अ॒ग्निम् । दु॒व॒स्य॒त॒ । इति॑ । घृ॒ता॒नु॒षि॒क्तामिति॑ घृत - अ॒नु॒षि॒क्ताम् । अव॑सित॒ इत्यव॑ - सि॒ते॒ । स॒मिध॒मिति॑ सं - इध᳚म् । एति॑ । द॒धा॒ति॒ । यथा᳚ । अति॑थये । आग॑ता॒येत्या - ग॒ता॒य॒ । स॒र्पिष्व॑त् । आ॒ति॒थ्यम् । क्रि॒यते᳚ । ता॒दृक् । ए॒व । तत् । गा॒य॒त्रि॒या । ब्रा॒ह्म॒णस्य॑ । गा॒य॒त्रः । हि । ब्रा॒ह्म॒णः । त्रि॒ष्टुभा᳚ । रा॒ज॒न्य॑स्य । त्रैष्टु॑भः । हि । रा॒ज॒न्यः॑ । अ॒फ्स्वित्य॑प्-सु । भस्म॑ । प्रेति॑ । वे॒श॒य॒ति॒ । अ॒फ्सुयो॑नि॒रितिय॒फ्सु - यो॒निः॒ । वै । अ॒ग्निः । स्वाम् । ए॒व । ए॒न॒म् । योनि᳚म् । ग॒म॒य॒ति॒ । ति॒सृभि॒रिति॑ ति॒सृ-भिः॒ । प्रेति॑ । वे॒श॒य॒ति॒ । त्रि॒वृदिति॑ त्रि - वृत् । वै । \textbf{  11} \newline
                  \newline
                                \textbf{ TS 5.2.2.5} \newline
                  अ॒ग्निः । यावान्॑ । ए॒व । अ॒ग्निः । तम् । प्र॒ति॒ष्ठामिति॑ प्रति - स्थाम् । ग॒म॒य॒ति॒ । परेति॑ । वै । ए॒षः । अ॒ग्निम् । व॒प॒ति॒ । यः । अ॒फ्स्वित्य॑प् - सु । भस्म॑ । प्र॒वे॒शय॒तीति॑ प्र - वे॒शय॑ति । ज्योति॑ष्मतीभ्याम् । अवेति॑ । द॒धा॒ति॒ । ज्योतिः॑ । ए॒व । अ॒स्मि॒न्न् । द॒धा॒ति॒ । द्वाभ्या᳚म् । प्रति॑ष्ठित्या॒ इति॒ प्रति॑ - स्थि॒त्यै॒ । परेति॑ । वै । ए॒षः । प्र॒जामिति॑ प्र - जाम् । प॒शून् । व॒प॒ति॒ । यः । अ॒फ्स्वित्य॑प् - सु । भस्म॑ । प्र॒वे॒शय॒तीति॑ प्र - वे॒शय॑ति । पुनः॑ । ऊ॒र्जा । स॒ह । र॒य्या । इति॑ । पुनः॑ । उ॒दैतीत्यु॑त् - ऐति॑ । प्र॒जामिति॑ प्र - जाम् । ए॒व । प॒शून् । आ॒त्मन्न् । ध॒त्ते॒ । पुनः॑ । त्वा॒ । आ॒दि॒त्याः । \textbf{  12} \newline
                  \newline
                                \textbf{ TS 5.2.2.6} \newline
                  रु॒द्राः । वस॑वः । समिति॑ । इ॒न्ध॒ता॒म् । इति॑ । आ॒ह॒ । ए॒ताः । वै । ए॒तम् । दे॒वताः᳚ । अग्रे᳚ । समिति॑ । ऐ॒न्ध॒त॒ । ताभिः॑ । ए॒व । ए॒न॒म् । समिति॑ । इ॒न्धे॒ । बोध॑ । सः । बो॒धि॒ । इति॑ । उपेति॑ । ति॒ष्ठ॒ते॒ । बो॒धय॑ति । ए॒व । ए॒न॒म् । तस्मा᳚त् । सु॒प्त्वा । प्र॒जा इति॑ प्र - जाः । प्रेति॑ । बु॒द्ध्य॒न्ते॒ । य॒था॒स्था॒नमिति॑ यथा - स्था॒नम् । उपेति॑ । ति॒ष्ठ॒ते॒ । तस्मा᳚त् । य॒था॒स्था॒नमिति॑ यथा - स्था॒नम् । प॒शवः॑ । पुनः॑ । एत्येत्या᳚ - इत्य॑ । उपेति॑ । ति॒ष्ठ॒न्ते॒ ॥ \textbf{  13} \newline
                  \newline
                      (वै विश्वे॑ दे॒वा - ज्योति॑ष्मा॒ - नप॑चिततमौ - त्रि॒वृद्वा - आ॑दि॒त्या - द्विच॑त्वारिꣳशच्च)  \textbf{(A2)} \newline \newline
                                \textbf{ TS 5.2.3.1} \newline
                  याव॑ती । वै । पृ॒थि॒वी । तस्यै᳚ । य॒मः । आधि॑पत्य॒मित्याधि॑ - प॒त्य॒म् । परीति॑ । इ॒या॒य॒ । यः । वै । य॒मम् । दे॒व॒यज॑न॒मिति॑ देव - यज॑नम् । अ॒स्याः । अनि॑र्या॒च्येत्यनिः॑ - या॒च्य॒ । अ॒ग्निम् । चि॒नु॒ते । य॒माय॑ । ए॒न॒म् । सः । चि॒नु॒ते॒ । अपेति॑ । इ॒त॒ । इति॑ । अ॒द्ध्यव॑सायय॒तीत्य॑धि - अव॑साययति । य॒मम् । ए॒व । दे॒व॒यज॑न॒मिति॑ देव-यज॑नम् । अ॒स्यै । नि॒र्याच्येति॑ निः - याच्य॑ । आ॒त्मने᳚ । अ॒ग्निम् । चि॒नु॒ते॒ । इ॒ष्व॒ग्रेणेती॑षु - अ॒ग्रेण॑ । वै । अ॒स्याः । अना॑मृत॒मित्यना᳚ - मृ॒त॒म् । इ॒च्छन्तः॑ । न । अ॒वि॒न्द॒न्न् । ते । दे॒वाः । ए॒तत् । यजुः॑ । अ॒प॒श्य॒न्न् । अपेति॑ । इ॒त॒ । इति॑ । यत् । ए॒तेन॑ । अ॒द्ध्य॒व॒सा॒यय॒तीत्य॑धि - अ॒व॒सा॒यय॑ति । \textbf{  14} \newline
                  \newline
                                \textbf{ TS 5.2.3.2} \newline
                  अना॑मृत॒ इत्यना᳚ - मृ॒ते॒ । ए॒व । अ॒ग्निम् । चि॒नु॒ते॒ । उदिति॑ । ह॒न्ति॒ । यत् । ए॒व । अ॒स्याः॒ । अ॒मे॒द्ध्यम् । तत् । अपेति॑ । ह॒न्ति॒ । अ॒पः । अवेति॑ । उ॒क्ष॒ति॒ । शान्त्यै᳚ । सिक॑ताः । नीति॑ । व॒प॒ति॒ । ए॒तत् । वै । अ॒ग्नेः । वै॒श्वा॒न॒रस्य॑ । रू॒पम् । रू॒पेण॑ । ए॒व । वै॒श्वा॒न॒रम् । अवेति॑ । रु॒न्धे॒ । ऊषान्॑ । नीति॑ । व॒प॒ति॒ । पुष्टिः॑ । वै । ए॒षा । प्र॒जन॑न॒मिति॑ प्र - जन॑नम् । यत् । ऊषाः᳚ । पुष्ट्या᳚म् । ए॒व । प्र॒जन॑न॒ इति॑ प्र - जन॑ने । अ॒ग्निम् । चि॒नु॒ते॒ । अथो॒ इति॑ । स॒ज्ञांन॒ इति॑ सं - ज्ञाने᳚ । ए॒व । स॒ज्ञांन॒मिति॑ सं-ज्ञान᳚म् । हि । ए॒तत् । \textbf{  15} \newline
                  \newline
                                \textbf{ TS 5.2.3.3} \newline
                  प॒शू॒नाम् । यत् । ऊषाः᳚ । द्यावा॑पृथि॒वी इति॒ द्यावा᳚ - पृ॒थि॒वी । स॒ह । आ॒स्ता॒म् । ते इति॑ । वि॒य॒ती इति॑ वि - य॒ती । अ॒ब्रू॒ता॒म् । अस्तु॑ । ए॒व । नौ॒ । स॒ह । य॒ज्ञिय᳚म् । इति॑ । यत् । अ॒मुष्याः᳚ । य॒ज्ञिय᳚म् । आसी᳚त् । तत् । अ॒स्याम् । अ॒द॒धा॒त् । ते । ऊषाः᳚ । अ॒भ॒व॒न्न् । यत् । अ॒स्याः । य॒ज्ञिय᳚म् । आसी᳚त् । तत् । अ॒मुष्या᳚म् । अ॒द॒धा॒त् । तत् । अ॒दः । च॒न्द्रम॑सि । कृ॒ष्णम् । ऊषान्॑ । नि॒वप॒न्निति॑ नि-वपन्न्॑ । अ॒दः । ध्या॒ये॒त् । द्यावा॑पृथि॒व्योरिति॒ द्यावा᳚ - पृ॒थि॒व्योः । ए॒व । य॒ज्ञिये᳚ । अ॒ग्निम् । चि॒नु॒ते॒ । अ॒यम् । सः । अ॒ग्निः । इति॑ । वि॒श्वामि॑त्र॒स्येति॑ वि॒श्व - मि॒त्र॒स्य॒ । \textbf{  16} \newline
                  \newline
                                \textbf{ TS 5.2.3.4} \newline
                  सू॒क्तमिति॑ सु - उ॒क्तम् । भ॒व॒ति॒ । ए॒तेन॑ । वै । वि॒श्वामि॑त्र॒ इति॑ वि॒श्व - मि॒त्रः॒ । अ॒ग्नेः । प्रि॒यम् । धाम॑ । अवेति॑ । अ॒रु॒न्ध॒ । अ॒ग्नेः । ए॒व । ए॒तेन॑ । प्रि॒यम् । धाम॑ । अवेति॑ । रु॒न्धे॒ । छन्दो॑भि॒रिति॒ छन्दः॑ - भिः॒ । वै । दे॒वाः । सु॒व॒र्गमिति॑ सुवः - गम् । लो॒कम् । आ॒य॒न्न् । चत॑स्रः । प्राचीः᳚ । उपेति॑ । द॒धा॒ति॒ । च॒त्वारि॑ । छन्दाꣳ॑सि । छन्दो॑भि॒रिति॒ छन्दः॑ - भिः॒ । ए॒व । तत् । यज॑मानः । सु॒व॒र्गमिति॑ सुवः - गम् । लो॒कम् । ए॒ति॒ । तेषा᳚म् । सु॒व॒र्गमिति॑ सुवः - गम् । लो॒कम् । य॒ताम् । दिशः॑ । समिति॑ । अ॒व्ली॒य॒न्त॒ । ते । द्वे इति॑ । पु॒रस्ता᳚त् । स॒मीची॒ इति॑ । उपेति॑ । अ॒द॒ध॒त॒ । द्वे इति॑ । \textbf{  17} \newline
                  \newline
                                \textbf{ TS 5.2.3.5} \newline
                  प॒श्चात् । स॒मीची॒ इति॑ । ताभिः॑ । वै । ते । दिशः॑ । अ॒दृꣳ॒॒ह॒न्न् । यत् । द्वे इति॑ । पु॒रस्ता᳚त् । स॒मीची॒ इति॑ । उ॒प॒दधा॒तीत्यु॑प - दधा॑ति । द्वे इति॑ । प॒श्चात् । स॒मीची॒ इति॑ । दि॒शाम् । विधृ॑त्या॒ इति॒ वि-धृ॒त्यै॒ । अथो॒ इति॑ । प॒शवः॑ । वै । छन्दाꣳ॑सि । प॒शून् । ए॒व । अ॒स्मै॒ । स॒मीचः॑ । द॒धा॒ति॒ । अ॒ष्टौ । उपेति॑ । द॒धा॒ति॒ । अ॒ष्टाक्ष॒रेत्य॒ष्टा-अ॒क्ष॒रा॒ । गा॒य॒त्री । गा॒य॒त्रः । अ॒ग्निः । यावान्॑ । ए॒व । अ॒ग्निः । तम् । चि॒नु॒ते॒ । अ॒ष्टौ । उपेति॑ । द॒धा॒ति॒ । अ॒ष्टाक्ष॒रेत्य॒ष्टा-अ॒क्ष॒रा॒ । गा॒य॒त्री । गा॒य॒त्री । सु॒व॒र्गमिति॑ सुवः - गम् । लो॒कम् । अञ्ज॑सा । वे॒द॒ । सु॒व॒र्गस्येति॑ सुवः - गस्य॑ । लो॒कस्य॑ । \textbf{  18} \newline
                  \newline
                                \textbf{ TS 5.2.3.6} \newline
                  प्रज्ञा᳚त्या॒ इति॒ प्र - ज्ञा॒त्यै॒ । त्रयो॑द॒शेति॒ त्रयः॑ - द॒श॒ । लो॒क॒पृं॒णा इति॑ लोकं - पृ॒णाः । उपेति॑ । द॒धा॒ति॒ । एक॑विꣳशति॒रित्येक॑ - विꣳ॒॒श॒तिः॒ । समिति॑ । प॒द्य॒न्ते॒ । प्र॒ति॒ष्ठेति॑ प्रति - स्था । वै । ए॒क॒विꣳ॒॒श इत्ये॑क - विꣳ॒॒शः । प्र॒ति॒ष्ठेति॑ प्रति - स्था । गार्.ह॑पत्य॒ इति॒ गार्.ह॑ - प॒त्यः॒ । ए॒क॒विꣳ॒॒शस्येत्ये॑क - विꣳ॒॒शस्य॑ । ए॒व । प्र॒ति॒ष्ठामिति॑ प्रति-स्थाम् । गार्.ह॑पत्य॒मिति॒ गार्.ह॑ - प॒त्य॒म् । अनु॑ । प्रतीति॑ । ति॒ष्ठ॒ति॒ । प्रतीति॑ । अ॒ग्निम् । चि॒क्या॒नः । ति॒ष्ठ॒ति॒ । यः । ए॒वम् । वेद॑ । पञ्च॑चितीक॒मिति॒ पञ्च॑-चि॒ती॒क॒म् । चि॒न्वी॒त॒ । प्र॒थ॒मम् । चि॒न्वा॒नः । पाङ्क्तः॑ । य॒ज्ञ्ः । पाङ्क्ताः᳚ । प॒शवः॑ । य॒ज्ञ्म् । ए॒व । प॒शून् । अवेति॑ । रु॒न्धे॒ । त्रिचि॑तीक॒मिति॒ त्रि - चि॒ती॒क॒म् । चि॒न्वी॒त॒ । द्वि॒तीय᳚म् । चि॒न्वा॒नः । त्रयः॑ । इ॒मे । लो॒काः । ए॒षु । ए॒व । लो॒केषु॑ । \textbf{  19} \newline
                  \newline
                                \textbf{ TS 5.2.3.7} \newline
                  प्रतीति॑ । ति॒ष्ठ॒ति॒ । एक॑चितीक॒मित्येक॑ - चि॒ती॒क॒म् । चि॒न्वी॒त॒ । तृ॒तीय᳚म् । चि॒न्वा॒नः । ए॒क॒धेत्ये॑क - धा । वै । सु॒व॒र्ग इति॑ सुवः - गः । लो॒कः । ए॒क॒वृतेत्ये॑क - वृता᳚ । ए॒व । सु॒व॒र्गमिति॑ सुवः - गम् । लो॒कम् । ए॒ति॒ । पुरी॑षेण । अ॒भीति॑ । ऊ॒ह॒ति॒ । तस्मा᳚त् । माꣳ॒॒सेन॑ । अस्थि॑ । छ॒न्नम् । न । दु॒श्चर्मेति॑ दुः - चर्मा᳚ । भ॒व॒ति॒ । यः । ए॒वम् । वेद॑ । पञ्च॑ । चित॑यः । भ॒व॒न्ति॒ । प॒ञ्चभि॒रिति॑ प॒ञ्च - भिः॒ । पुरी॑षैः । अ॒भीति॑ । ऊ॒ह॒ति॒ । दश॑ । समिति॑ । प॒द्य॒न्ते॒ । दशा᳚क्ष॒रेति॒ दश॑ - अ॒क्ष॒रा॒ । वि॒राडिति॑ वि-राट् । अन्न᳚म् । वि॒राडिति॑ वि - राट् । वि॒राजीति॑ वि - राजि॑ । ए॒व । अ॒न्नाद्य॒ इत्य॑न्न - अद्ये᳚ । प्रतीति॑ । ति॒ष्ठ॒ति॒ ॥ \textbf{  20 } \newline
                  \newline
                      (अ॒ध्य॒व॒सा॒यय॑ति॒ - ह्ये॑त - द्वि॒श्वामि॑त्रस्या - दधत॒ द्वे - लो॒कस्य॑ - लो॒केषु॑ -स॒प्तच॑त्वारिꣳशच्च)  \textbf{(A3)} \newline \newline
                                \textbf{ TS 5.2.4.1} \newline
                  वीति॑ । वै । ए॒तौ । द्वि॒षा॒ते॒ इति॑ । यः । च॒ । पु॒रा । अ॒ग्निः । यः । च॒ । उ॒खाया᳚म् । समिति॑ । इ॒त॒म् । इति॑ । च॒त॒सृभि॒रिति॑ चत॒सृ-भिः॒ । सम् । नीति॑ । व॒प॒ति॒ । च॒त्वारि॑ । छन्दाꣳ॑सि । छन्दाꣳ॑सि । खलु॑ । वै । अ॒ग्नेः । प्रि॒या । त॒नूः । प्रि॒यया᳚ । ए॒व । ए॒नौ॒ । त॒नुवा᳚ । समिति॑ । शा॒स्ति॒ । समिति॑ । इ॒त॒म् । इति॑ । आ॒ह॒ । तस्मा᳚त् । ब्रह्म॑णा । क्ष॒त्रम् । समिति॑ । ए॒ति॒ । यत् । स॒न्युंप्येति॑ सं - न्युप्य॑ । वि॒हर॒तीति॑ वि - हर॑ति । तस्मा᳚त् । ब्रह्म॑णा । क्ष॒त्रम् । वीति॑ । ए॒ति॒ । ऋ॒तुभि॒रित्यृ॒तु - भिः॒ । \textbf{  21} \newline
                  \newline
                                \textbf{ TS 5.2.4.2} \newline
                  वै । ए॒तम् । दी॒क्ष॒य॒न्ति॒ । सः । ऋ॒तुभि॒रित्यृ॒तु - भिः॒ । ए॒व । वि॒मुच्य॒ इति॑ वि - मुच्यः॑ । मा॒ता । इ॒व॒ । पु॒त्रम् । पृ॒थि॒वी । पु॒री॒ष्य᳚म् । इति॑ । आ॒ह॒ । ऋ॒तुभि॒रित्यृ॒तु - भिः॒ । ए॒व । ए॒न॒म् । दी॒क्ष॒यि॒त्वा । ऋ॒तुभि॒रित्यृ॒तु-भिः॒ । वीति॑ । मु॒ञ्च॒ति॒ । वै॒श्वा॒न॒र्या । शि॒क्य᳚म् । एति॑ । द॒त्ते॒ । स्व॒दय॑ति । ए॒व । ए॒न॒त्॒ । नै॒र्॒.ऋ॒तीरिति॑ नैः-ऋ॒तीः । कृ॒ष्णाः । ति॒स्रः । तुष॑पक्वा॒ इति॒ तुष॑ - प॒क्वाः॒ । भ॒व॒न्ति॒ । निर्.ऋ॑त्या॒ इति॒ निः - ऋ॒त्यै॒ । वै । ए॒तत् । भा॒ग॒धेय॒मिति॑ भाग - धेय᳚म् । यत् । तुषाः᳚ । निर्.ऋ॑त्या॒ इति॒ निः-ऋ॒त्यै॒ । रू॒पम् । कृ॒ष्णम् । रू॒पेण॑ । ए॒व । निर्.ऋ॑ति॒मिति॒ निः - ऋ॒ति॒म् । नि॒रव॑दयत॒ इति॑ निः - अव॑दयते । इ॒माम् । दिश᳚म् । य॒न्ति॒ । ए॒षा । \textbf{  22} \newline
                  \newline
                                \textbf{ TS 5.2.4.3} \newline
                  वै । निर्.ऋ॑त्या॒ इति॒ निः - ऋ॒त्यै॒ । दिक् । स्वाया᳚म् । ए॒व । दि॒शि । निर्.ऋ॑ति॒मिति॒ निः - ऋ॒ति॒म् । नि॒रव॑दयत॒ इति॑ निः - अव॑दयते । स्वकृ॑त॒ इति॒ स्व - कृ॒ते॒ । इरि॑णे । उपेति॑ । द॒धा॒ति॒ । प्र॒द॒र इति॑ प्र - द॒रे । वा॒ । ए॒तत् । वै । निर्.ऋ॑त्या॒ इति॒ निः - ऋ॒त्याः॒ । आ॒यत॑न॒मित्या᳚ - यत॑नम् । स्वे । ए॒व । आ॒यत॑न॒ इत्या᳚ - यत॑ने । निर्.ऋ॑ति॒मिति॒ निः - ऋ॒ति॒म् । नि॒रव॑दयत॒ इति॑ निः - अव॑दयते । शि॒क्य᳚म् । अ॒भि । उपेति॑ । द॒धा॒ति॒ । नै॒र्॒.ऋ॒त इति॑ नैः - ऋ॒तः । वै । पाशः॑ । सा॒क्षादिति॑ स - अ॒क्षात् । ए॒व । ए॒न॒म् । नि॒र्॒.ऋ॒ति॒पा॒शादिति॑ निर्.ऋति - पा॒शात् । मु॒ञ्च॒ति॒ । ति॒स्रः । उपेति॑ । द॒धा॒ति॒ । त्रे॒धा॒वि॒हि॒त इति॑ त्रेधा - वि॒हि॒तः । वै । पुरु॑षः । यावान्॑ । ए॒व । पुरु॑षः । तस्मा᳚त् । निर्.ऋ॑ति॒मिति॒ निः - ऋ॒ति॒म् । अवेति॑ । य॒ज॒ते॒ । परा॑चीः । उपेति॑ । \textbf{  23} \newline
                  \newline
                                \textbf{ TS 5.2.4.4} \newline
                  द॒धा॒ति॒ । परा॑चीम् । ए॒व । अ॒स्मा॒त् । निर्.ऋ॑ति॒मिति॒ निः - ऋ॒ति॒म् । प्रेति॑ । नु॒द॒ते॒ । अप्र॑तीक्ष॒मित्यप्र॑ति - ई॒क्ष॒म् । एति॑ । य॒न्ति॒ । निर्.ऋ॑त्या॒ इति॒ निः - ऋ॒त्याः॒ । अ॒न्तर्.हि॑त्या॒ इत्य॒न्तः - हि॒त्यै॒ । मा॒र्ज॒यि॒त्वा । उपेति॑ । ति॒ष्ठ॒न्ते॒ । मे॒द्ध्य॒त्वायेति॑ मेद्ध्य - त्वाय॑ । गार्.ह॑पत्य॒मिति॒ गार्.ह॑ - प॒त्य॒म् । उपेति॑ । ति॒ष्ठ॒न्ते॒ । नि॒र्॒.ऋ॒ति॒लो॒क इति॑ निर्.ऋति - लो॒के । ए॒व । च॒रि॒त्वा । पू॒ताः । दे॒व॒लो॒कमिति॑ देव -  लो॒कम् । उ॒पाव॑र्तन्त॒ इत्यु॑प - आव॑र्तन्ते । एक॑या । उपेति॑ । ति॒ष्ठ॒न्ते॒ । ए॒क॒धेत्ये॑क - धा । ए॒व । यज॑माने । वी॒र्य᳚म् । द॒ध॒ति॒ । नि॒वेश॑न॒ इति॑ नि-वेश॑नः । स॒ङ्गम॑न॒ इति॑ सं- गम॑नः । वसू॑नाम् । इति॑ । आ॒ह॒ । प्र॒जेति॑ प्र - जा । वै ।   प॒शवः॑ । वसु॑ । प्र॒जयेति॑ प्र - जया᳚ । ए॒व । ए॒न॒म् । प॒शुभि॒रिति॑ प॒शु - भिः॒ । समिति॑ । अ॒द्‌र्ध॒य॒न्ति॒ ॥ \textbf{  24} \newline
                  \newline
                      (ऋ॒तुभि॑ - रे॒षा- परा॑ची॒रुपा॒ - ष्टाच॑त्वारिꣳशच्च)  \textbf{(A4)} \newline \newline
                                \textbf{ TS 5.2.5.1} \newline
                  पु॒रु॒ष॒मा॒त्रेणेति॑ पुरुष - मा॒त्रेण॑ । वीति॑ । मि॒मी॒ते॒ । य॒ज्ञेन॑ । वै । पुरु॑षः । सम्मि॑त॒ इति॒ सं - मि॒तः॒ । य॒ज्ञ्॒प॒रुषेति॑ यज्ञ्-प॒रुषा᳚ । ए॒व । ए॒न॒म् । वीति॑ । मि॒मी॒ते॒ । यावान्॑ । पुरु॑षः । ऊ॒द्‌र्ध्वबा॑हु॒रित्यु॒द्‌र्ध्व-बा॒हुः॒ । तावान्॑ । भ॒व॒ति॒ । ए॒ताव॑त् । वै । पुरु॑षे । वी॒र्य᳚म् । वी॒र्ये॑ण । ए॒व । ए॒न॒म् । वीति॑ । मि॒मी॒ते॒ । प॒क्षी । भ॒व॒ति॒ । न । हि । अ॒प॒क्षः । पति॑तुम् । अर्.ह॑ति । अ॒र॒त्निना᳚ । प॒क्षौ । द्राघी॑याꣳसौ । भ॒व॒तः॒ । तस्मा᳚त् । प॒क्षप्र॑वयाꣳ॒॒सीति॑ प॒क्ष - प्र॒व॒याꣳ॒॒सि॒ । वयाꣳ॑सि । व्या॒म॒मा॒त्राविति॑ व्याम - मा॒त्रौ । प॒क्षौ । च॒ । पुच्छ᳚म् । च॒ । भ॒व॒ति॒ । ए॒ताव॑त् । वै । पुरु॑षे । वी॒र्य᳚म् । \textbf{  25} \newline
                  \newline
                                \textbf{ TS 5.2.5.2} \newline
                  वी॒र्य॑सम्मित॒ इति॑ वी॒र्य॑ - स॒म्मि॒तः॒ । वेणु॑ना । वीति॑ । मि॒मी॒ते॒ । आ॒ग्ने॒यः । वै । वेणुः॑ । स॒यो॒नि॒त्वायेति॑ सयोनि - त्वाय॑ । यजु॑षा । यु॒न॒क्ति॒ । यजु॑षा । कृ॒ष॒ति॒ । व्यावृ॑त्त्या॒ इति॑ वि - आवृ॑त्त्यै । ष॒ड्ग॒वेनेति॑ षट् - ग॒वेन॑ । कृ॒ष॒ति॒ । षट् । वै । ऋ॒तवः॑ । ऋ॒तुभि॒रित्यृ॒तु - भिः॒ । ए॒व । ए॒न॒म् । कृ॒ष॒ति॒ । यत् । द्वा॒द॒श॒ग॒वेनेति॑ द्वादश - ग॒वेन॑ । सं॒ॅव॒थ्स॒रेणेति॑ सं - व॒थ्स॒रेण॑ । ए॒व । इ॒यम् । वै । अ॒ग्नेः । अ॒ति॒दा॒हादित्य॑ति - दा॒हात् । अ॒बि॒भे॒त् । सा । ए॒तत् । द्वि॒गु॒णमिति॑ द्वि - गु॒णम् । अ॒प॒श्य॒त् । कृ॒ष्टम् । च॒ । अकृ॑ष्टम् । च॒ । ततः॑ । वै । इ॒माम् । न । अतीति॑ । अ॒द॒ह॒त् । यत् । कृ॒ष्टम् । च॒ । अकृ॑ष्टम् । च॒ । \textbf{  26} \newline
                  \newline
                                \textbf{ TS 5.2.5.3} \newline
                  भव॑ति । अ॒स्याः । अन॑तिदाहा॒येत्यन॑ति - दा॒हा॒य॒ । द्वि॒गु॒णमिति॑ द्वि - गु॒णम् । तु । वै । अ॒ग्निम् । उद्य॑न्तु॒मित्युत्-य॒न्तु॒म् । अ॒र्.॒ह॒ति॒ । इति॑ । आ॒हुः॒ । यत् । कृ॒ष्टम् । च॒ । अकृ॑ष्टम् । च॒ । भव॑ति । अ॒ग्नेः । उद्य॑त्या॒ इत्युत् - य॒त्यै॒ । ए॒ताव॑न्तः । वै । प॒शवः॑ । द्वि॒पाद॒ इति॑ द्वि - पादः॑ । च॒ । चतु॑ष्पाद॒ इति॒ चतुः॑ - पा॒दः॒ । च॒ । तान् । यत् । प्राचः॑ । उ॒थ्सृ॒जेदित्यु॑त् - सृ॒जेत् । रु॒द्राय॑ । अपीति॑ । द॒द्ध्या॒त् । यत् । द॒क्षि॒णा । पि॒तृभ्य॒ इति॑ पि॒तृ - भ्यः॒ । नीति॑ । धु॒वे॒त् । यत् । प्र॒तीचः॑ । रक्षाꣳ॑सि । ह॒न्युः॒ । उदी॑चः । उदिति॑ । सृ॒ज॒ति॒ । ए॒षा । वै । दे॒व॒म॒नु॒ष्याणा॒मिति॑ देव - म॒नु॒ष्याणा᳚म् । शा॒न्ता । दिक् । \textbf{  27} \newline
                  \newline
                                \textbf{ TS 5.2.5.4} \newline
                  ताम् । ए॒व । ए॒ना॒न् । अनु॑ । उदिति॑ । सृ॒ज॒ति॒ । अथो॒ इति॑ । खलु॑ । इ॒माम् । दिश᳚म् । उदिति॑ । सृ॒ज॒ति॒ । अ॒सौ । वै । आ॒दि॒त्यः । प्रा॒ण इति॑ प्र - अ॒नः । प्रा॒णमिति॑ प्र - अ॒नम् । ए॒व । ए॒ना॒न् । अनु॑ । उदिति॑ । सृ॒ज॒ति॒ । द॒क्षि॒णा । प॒र्याव॑र्तन्त॒ इति॑ परि - आव॑र्तन्ते । स्वम् । ए॒व । वी॒र्य᳚म् । अन्विति॑ । प॒र्याव॑र्तन्त॒ इति॑ परि - आव॑र्तन्ते । तस्मा᳚त् । दक्षि॑णः । अद्‌र्धः॑ । आ॒त्मनः॑ । वी॒र्या॑वत्तर॒ इति॑ वी॒र्या॑वत् - त॒रः॒ । अथो॒ इति॑ । आ॒दि॒त्यस्य॑ । ए॒व । आ॒वृत॒मित्या᳚ - वृत᳚म् । अन्विति॑ । प॒र्याव॑र्तन्त॒ इति॑ परि-आव॑र्तन्ते । तस्मा᳚त् । परा᳚ञ्चः । प॒शवः॑ । वीति॑ । ति॒ष्ठ॒न्ते॒ । प्र॒त्यञ्चः॑ । एति॑ । व॒र्त॒न्ते॒ । ति॒स्रस्ति॑स्र॒ इति॑ ति॒स्रः - ति॒स्रः॒ । सीताः᳚ । \textbf{  28} \newline
                  \newline
                                \textbf{ TS 5.2.5.5} \newline
                  कृ॒ष॒ति॒ । त्रि॒वृत॒मिति॑ त्रि - वृत᳚म् । ए॒व । य॒ज्ञ्॒मु॒ख इति॑ यज्ञ्-मु॒खे । वीति॑ । या॒त॒य॒ति॒ । ओष॑धीः । व॒प॒ति॒ । ब्रह्म॑णा । अन्न᳚म् । अवेति॑ । रु॒न्धे॒ । अ॒र्के । अ॒र्कः । ची॒य॒ते॒ । च॒तु॒र्द॒शभि॒रिति॑ चतुर्द॒श - भिः॒ । व॒प॒ति॒ । स॒प्त । ग्रा॒म्याः । ओष॑धयः । स॒प्त । आ॒र॒ण्याः । उ॒भयी॑षाम् । अव॑रुद्ध्या॒ इत्यव॑ - रु॒द्ध्यै॒ । अन्न॑स्यान्न॒स्येत्यन्न॑स्य - अ॒न्न॒स्य॒ । व॒प॒ति॒ । अन्न॑स्यान्न॒स्येत्यन्न॑स्य-अ॒न्न॒स्य॒ । अव॑रुद्ध्या॒ इत्यव॑-रु॒द्ध्यै॒ । कृ॒ष्टे । व॒प॒ति॒ । कृ॒ष्टे । हि । ओष॑धयः । प्र॒ति॒तिष्ठ॒न्तीति॑ प्रति-तिष्ठ॑न्ति । अ॒नु॒सी॒तमित्य॑नु - सी॒तम् । व॒प॒ति॒ । प्रजा᳚त्या॒ इति॒ प्र - जा॒त्यै॒ । द्वा॒द॒शस्विति॑ द्वाद॒श - सु॒ । सीता॑सु । व॒प॒ति॒ । द्वाद॑श । मासाः᳚ । सं॒ॅव॒थ्स॒र इति॑ सं - व॒थ्स॒रः । सं॒ॅव॒थ्स॒रेणेति॑ सं - व॒थ्स॒रेण॑ । ए॒व । अ॒स्मै॒ । अन्न᳚म् । प॒च॒ति॒ । यत् । अ॒ग्नि॒चिदित्य॑ग्नि - चित् । \textbf{  29} \newline
                  \newline
                                \textbf{ TS 5.2.5.6} \newline
                  अन॑वरुद्ध॒स्येत्यन॑व-रु॒द्ध॒स्य॒ । अ॒श्नी॒यात् । अव॑रुद्धे॒नेत्यव॑ - रु॒द्धे॒न॒ । वीति॑ । ऋ॒द्ध्ये॒त॒ । ये । वन॒स्पती॑नाम् । फ॒ल॒ग्रह॑य॒ इति॑ फल-ग्रह॑यः । तान् । इ॒द्ध्मे । अपि॑ । प्रेति॑ । उ॒क्षे॒त् । अन॑वरुद्ध॒स्येत्यन॑व-रु॒द्ध॒स्य॒ । अव॑रुद्ध्या॒ इत्यव॑ - रु॒द्ध्यै॒ । दि॒ग्भ्य इति॑ दिक् - भ्यः । लो॒ष्टान् । समिति॑ । अ॒स्य॒ति॒ । दि॒शाम् । ए॒व । वी॒र्य᳚म् । अ॒व॒रुद्ध्येत्य॑व-रुद्ध्य॑ । दि॒शाम् । वी॒र्ये᳚ । अ॒ग्निम् । चि॒नु॒ते॒ । यम् । द्वि॒ष्यात् । यत्र॑ । सः । स्यात् । तस्यै᳚ । दि॒शः । लो॒ष्टम् । एति॑ । ह॒रे॒त् । इष᳚म् । ऊर्ज᳚म् । अ॒हम् । इ॒तः । एति॑ । द॒दे॒ । इति॑ । इष᳚म् । ए॒व । ऊर्ज᳚म् । तस्यै᳚ । दि॒शः । अवेति॑ ( ) । रु॒न्धे॒ । क्षोधु॑कः । भ॒व॒ति॒ । यः । तस्या᳚म् । दि॒शि । भव॑ति । उ॒त्त॒र॒वे॒दिमित्यु॑त्तर - वे॒दिम् । उपेति॑ । व॒प॒ति॒ । उ॒त्त॒र॒वे॒द्यामित्यु॑त्तर - वे॒द्याम् । हि । अ॒ग्निः । ची॒यते᳚ । अथो॒ इति॑ । प॒शवः॑ । वै । उ॒त्त॒र॒वे॒दिरित्यु॑त्तर - वे॒दिः । प॒शून् । ए॒व । अवेति॑ । रु॒न्धे॒ । अथो॒ इति॑ । य॒ज्ञ्॒प॒रुष॒ इति॑ यज्ञ् - प॒रुषः॑ । अन॑न्तरित्या॒ इत्यन॑न्तः - इ॒त्यै॒ ॥ \textbf{  30} \newline
                  \newline
                      (च॒ भ॒व॒त्ये॒ताव॒द्वै पुरु॑षे वी॒र्यं॑ - ॅयत्कृ॒ष्टञ्चाऽकृ॑ष्टञ्च॒ - दिख्- सीता॑ - अग्नि॒चि - दव॒ - पञ्च॑विꣳशतिश्च)  \textbf{(A5)} \newline \newline
                                \textbf{ TS 5.2.6.1} \newline
                  अग्ने᳚ । तव॑ । श्रवः॑ । वयः॑ । इति॑ । सिक॑ताः । नीति॑ । व॒प॒ति॒ । ए॒तत् । वै । अ॒ग्नेः । वै॒श्वा॒न॒रस्य॑ । सू॒क्तमिति॑ सु - उ॒क्तम् । सू॒क्तेनेति॑ सु - उ॒क्तेन॑ । ए॒व । वै॒श्वा॒न॒रम् । अवेति॑ । रु॒न्धे॒ । ष॒ड्भिरिति॑ षट् - भिः । नीति॑ । व॒प॒ति॒ । षट् । वै । ऋ॒तवः॑ । सं॒ॅव॒थ्स॒र इति॑ सं - व॒थ्स॒रः । सं॒ॅव॒थ्स॒र इति॑ सं - व॒थ्स॒रः । अ॒ग्निः । वै॒श्वा॒न॒रः । सा॒क्षादिति॑ स - अ॒क्षात् । ए॒व । वै॒श्वा॒न॒रम् । अवेति॑ । रु॒न्धे॒ । स॒मु॒द्रम् । वै । नाम॑ । ए॒तत् । छन्दः॑ । स॒मु॒द्रम् । अन्विति॑ । प्र॒जा इति॑ प्र - जाः । प्रेति॑ । जा॒य॒न्ते॒ । यत् । ए॒तेन॑ । सिक॑ताः । नि॒वप॒तीति॑ नि - वप॑ति । प्र॒जाना॒मिति॑ प्र - जाना᳚म् । प्र॒जन॑ना॒येति॑ प्र - जन॑नाय । इन्द्रः॑ । \textbf{  31} \newline
                  \newline
                                \textbf{ TS 5.2.6.2} \newline
                  वृ॒त्राय॑ । वज्र᳚म् । प्रेति॑ । अ॒ह॒र॒त् । सः । त्रे॒धा । वीति॑ । अ॒भ॒व॒त् । स्फ्यः । तृती॑यम् । रथः॑ । तृती॑यम् । यूपः॑ । तृती॑यम् । ये । अ॒न्त॒श्श॒रा इत्य॑न्तः-श॒राः । अशी᳚र्यन्त । ताः । शर्क॑राः । अ॒भ॒व॒न्न् । तत् । शर्क॑राणाम् । श॒र्क॒र॒त्वमिति॑ शर्कर - त्वम् । वज्रः॑ । वै । शर्क॑राः । प॒शुः । अ॒ग्निः । यत् । शर्क॑राभिः । अ॒ग्निम् । प॒रि॒मि॒नोतीति॑ परि - मि॒नोति॑ । वज्रे॑ण । ए॒व । अ॒स्मै॒ । प॒शून् । परीति॑ । गृ॒ह्णा॒ति॒ । तस्मा᳚त् । वज्रे॑ण । प॒शवः॑ । परि॑गृहीता॒ इति॒ परि॑ - गृ॒ही॒ताः॒ । तस्मा᳚त् । स्थेयान्॑ । अस्थे॑यसः । न । उपेति॑ । ह॒र॒ते॒ । त्रि॒स॒प्ताभि॒रिति॑ त्रि - स॒प्ताभिः॑ । प॒शुका॑म॒स्येति॑ प॒शु - का॒म॒स्य॒ । \textbf{  32} \newline
                  \newline
                                \textbf{ TS 5.2.6.3} \newline
                  परीति॑ । मि॒नु॒या॒त् । स॒प्त । वै । शी॒र्॒.ष॒ण्याः᳚ । प्रा॒णा इति॑ प्र-अ॒नाः । प्रा॒णा इति॑ प्र - अ॒नाः । प॒शवः॑ । प्रा॒णैरिति॑ प्र - अ॒नैः । ए॒व । अ॒स्मै॒ । प॒शून् । अवेति॑ । रु॒न्धे॒ । त्रि॒ण॒वाभि॒रिति॑ त्रि - न॒वाभिः॑ । भ्रातृ॑व्यवत॒ इति॒ भ्रातृ॑व्य - व॒तः॒ । त्रि॒वृत॒मिति॑ त्रि - वृत᳚म् । ए॒व । वज्र᳚म् । स॒भृंत्येति॑ सं - भृत्य॑ । भ्रातृ॑व्याय । प्रेति॑ । ह॒र॒ति॒ । स्तृत्यै᳚ । अप॑रिमिताभि॒रित्यप॑रि - मि॒ता॒भिः॒ । परीति॑ । मि॒नु॒या॒त् । अप॑रिमित॒स्येत्यप॑रि - मि॒त॒स्य॒ । अव॑रुद्ध्या॒ इत्यव॑-रु॒द्ध्यै॒ । यम् । का॒मये॑त । अ॒प॒शुः । स्या॒त् । इति॑ । अप॑रिमि॒त्येत्यप॑रि - मि॒त्य॒ । तस्य॑ । शर्क॑राः । सिक॑ताः । वीति॑ । ऊ॒हे॒त् । अप॑रिगृहीत॒ इत्यप॑रि - गृ॒ही॒ते॒ । ए॒व । अ॒स्य॒ । वि॒षू॒चीन᳚म् । रेतः॑ । परेति॑ । सि॒ञ्च॒ति॒ । अ॒प॒शुः । ए॒व । भ॒व॒ति॒ । \textbf{  33} \newline
                  \newline
                                \textbf{ TS 5.2.6.4} \newline
                  यम् । का॒मये॑त । प॒शु॒मानिति॑ पशु - मान् । स्या॒त् । इति॑ । प॒रि॒मित्येति॑ परि-मित्य॑ । तस्य॑ । शर्क॑राः । सिक॑ताः । वीति॑ । ऊ॒हे॒त् । परि॑गृहीत॒ इति॒ परि॑ - गृ॒ही॒ते॒ । ए॒व । अ॒स्मै॒ । स॒मी॒चीन᳚म् । रेतः॑ । सि॒ञ्च॒ति॒ । प॒शु॒मानिति॑ पशु-मान् । ए॒व । भ॒व॒ति॒ । सौ॒म्या । वीति॑ । ऊ॒ह॒ति॒ । सोमः॑ । वै । रे॒तो॒धा इति॑ रेतः - धाः । रेतः॑ । ए॒व । तत् । द॒धा॒ति॒ । गा॒य॒त्रि॒या । ब्रा॒ह्म॒णस्य॑ । गा॒य॒त्रः । हि । ब्रा॒ह्म॒णः । त्रि॒ष्टुभा᳚ । रा॒ज॒न्य॑स्य । त्रैष्टु॑भः । हि । रा॒ज॒न्यः॑ । शं॒ॅयुमिति॑ शं - युम् । बा॒र्॒.ह॒स्प॒त्यम् । मेधः॑ । न । उपेति॑ । अ॒न॒म॒त् । सः । अ॒ग्निम् । प्रेति॑ । अ॒वि॒श॒त् । \textbf{  34} \newline
                  \newline
                                \textbf{ TS 5.2.6.5} \newline
                  सः । अ॒ग्नेः । कृष्णः॑ । रू॒पम् । कृ॒त्वा । उदिति॑ । आ॒य॒त॒ । सः । अश्व᳚म् । प्रेति॑ । अ॒वि॒श॒त् । सः । अश्व॑स्य । अ॒वा॒न्त॒र॒श॒फ इत्य॑वान्तर - श॒फः । अ॒भ॒व॒त् । यत् । अश्व᳚म् । आ॒क्र॒मय॒तीत्या᳚ - क्र॒मय॑ति । यः । ए॒व । मेधः॑ । अश्व᳚म् । प्रेति॑ । अवि॑शत् । तम् । ए॒व । अवेति॑ । रु॒न्धे॒ । प्र॒जाप॑ति॒नेति॑ प्र॒जा - प॒ति॒ना॒ । अ॒ग्निः । चे॒त॒व्यः॑ । इति॑ । आ॒हुः॒ । प्रा॒जा॒प॒त्य इति॑ प्राजा - प॒त्यः । अश्वः॑ । यत् । अश्व᳚म् । आ॒क्र॒मय॒तीत्या᳚-क्र॒मय॑ति । प्र॒जाप॑ति॒नेति॑ प्र॒जा - प॒ति॒ना॒ । ए॒व । अ॒ग्निम् । चि॒नु॒ते॒ । पु॒ष्क॒र॒प॒र्णमिति॑ पुष्कर - प॒र्णम् । उपेति॑ । द॒धा॒ति॒ । योनिः॑ । वै । अ॒ग्नेः । पु॒ष्क॒र॒प॒र्णमिति॑ पुष्कर - प॒र्णम् । सयो॑नि॒मिति॒ स-यो॒नि॒म् ( ) । ए॒व । अ॒ग्निम् । चि॒नु॒ते॒ । अ॒पाम् । पृ॒ष्ठम् । अ॒सि॒ । इति॑ । उपेति॑ । द॒धा॒ति॒ । अ॒पाम् । वै । ए॒तत् । पृ॒ष्ठम् । यत् । पु॒ष्क॒र॒प॒र्णमिति॑ पुष्कर - प॒र्णम् । रू॒पेण॑ । ए॒व । एन॒त्॒ । उपेति॑ । द॒धा॒ति॒ ॥ \textbf{  35} \newline
                  \newline
                      (इन्द्रः॑ - प॒शुका॑मस्य - भवत्य - विश॒थ् - सयो॑निं - ॅविꣳश॒तिश्च॑)  \textbf{(A6)} \newline \newline
                                \textbf{ TS 5.2.7.1} \newline
                  ब्रह्म॑ । ज॒ज्ञा॒नम् । इति॑ । रु॒क्मम् । उपेति॑ । द॒धा॒ति॒ । ब्रह्म॑मुखा॒ इति॒ ब्रह्म॑ - मु॒खाः॒ । वै । प्र॒जाप॑ति॒रिति॑ प्र॒जा - प॒तिः॒ । प्र॒जा इति॑ प्र - जाः । अ॒सृ॒ज॒त॒ । ब्रह्म॑मुखा॒ इति॒ ब्रह्म॑ - मु॒खाः॒ । ए॒व । तत् । प्र॒जा इति॑ प्र - जाः । यज॑मानः । सृ॒ज॒ते॒ । ब्रह्म॑ । ज॒ज्ञा॒नम् । इति॑ । आ॒ह॒ । तस्मा᳚त् । ब्रा॒ह्म॒णः । मुख्यः॑ । मुख्यः॑ । भ॒व॒ति॒ । यः । ए॒वम् । वेद॑ । ब्र॒ह्म॒वा॒दिन॒ इति॑ ब्रह्म - वा॒दिनः॑ । व॒द॒न्ति॒ । न । पृ॒थि॒व्याम् । न । अ॒न्तरि॑क्षे । न । दि॒वि । अ॒ग्निः । चे॒त॒व्यः॑ । इति॑ । यत् । पृ॒थि॒व्याम् । चि॒न्वी॒त । पृ॒थि॒वीम् । शु॒चा । अ॒र्प॒ये॒त् । न । ओष॑धयः । न । वन॒स्पत॑यः । \textbf{  36} \newline
                  \newline
                                \textbf{ TS 5.2.7.2} \newline
                  प्रेति॑ । जा॒ये॒र॒न्न् । यत् । अ॒न्तरि॑क्षे । चि॒न्वी॒त । अ॒न्तरि॑क्षम् । शु॒चा । अ॒र्प॒ये॒त् । न । वयाꣳ॑सि । प्रेति॑ । जा॒ये॒र॒न्न् । यत् । दि॒वि । चि॒न्वी॒त । दिव᳚म् । शु॒चा । अ॒र्प॒ये॒त् । न । प॒र्जन्यः॑ । व॒र्॒.षे॒त् । रु॒क्मम् । उपेति॑ । द॒धा॒ति॒ । अ॒मृत᳚म् । वै । हिर॑ण्यम् । अ॒मृते᳚ । ए॒व । अ॒ग्निम् । चि॒नु॒ते॒ । प्रजा᳚त्या॒ इति॒ प्र - जा॒त्यै॒ । हि॒र॒ण्मय᳚म् । पुरु॑षम् । उपेति॑ । द॒धा॒ति॒ । य॒ज॒मा॒न॒लो॒कस्येति॑ यजमान - लो॒कस्य॑ । विधृ॑त्या॒ इति॒ वि-धृ॒त्यै॒ । यत् । इष्ट॑कायाः । आतृ॑ण्ण॒मित्या-तृ॒ण्ण॒म् । अ॒नू॒प॒द॒द्ध्यादित्य॑नु - उ॒प॒द॒द्ध्यात् । प॒शू॒नाम् । च॒ । यज॑मानस्य । च॒ । प्रा॒णमिति॑ प्र - अ॒नम् । अपीति॑ । द॒द्ध्या॒त् । द॒क्षि॒ण॒तः । \textbf{  37} \newline
                  \newline
                                \textbf{ TS 5.2.7.3} \newline
                  प्राञ्च᳚म् । उपेति॑ । द॒धा॒ति॒ । दा॒धार॑ । य॒ज॒मा॒न॒लो॒कमिति॑ यजमान - लो॒कम् । न । प॒शू॒नाम् । च॒ । यज॑मानस्य । च॒ । प्रा॒णमिति॑ प्र - अ॒नम् । अपीति॑ । द॒धा॒ति॒ । अथो॒ इति॑ । खलु॑ । इष्ट॑कायाः । आतृ॑ण्ण॒मित्या - तृ॒ण्ण॒म् । अनु॑ । उपेति॑ । द॒धा॒ति॒ । प्रा॒णाना॒मिति॑ प्र - अ॒नाना᳚म् । उथ्सृ॑ष्ट्या॒ इत्युत् - सृ॒ष्ट्यै॒ । द्र॒फ्सः । च॒स्क॒न्द॒ । इति॑ । अ॒भीति॑ । मृ॒श॒ति॒ । होत्रा॑सु । ए॒व । ए॒न॒म् । प्रतीति॑ । स्था॒प॒य॒ति॒ । स्रुचौ᳚ । उपेति॑ । द॒धा॒ति॒ । आज्य॑स्य । पू॒र्णाम् । का॒र्ष्म॒र्य॒मयी॒मिति॑ कार्ष्मर्य - मयी᳚म् । द॒द्ध्नः । पू॒र्णाम् । औदु॑बंरीम् । इ॒यम् । वै । का॒र्ष्म॒र्य॒मयीति॑ कार्ष्मर्य - मयी᳚ । अ॒सौ । औदु॑बंरी । इ॒मे इति॑ । ए॒व । उपेति॑ । ध॒त्ते॒ । \textbf{  38} \newline
                  \newline
                                \textbf{ TS 5.2.7.4} \newline
                  तू॒ष्णीम् । उपेति॑ । द॒धा॒ति॒ । न । हि । इ॒मे इति॑ । यजु॑षा । आप्तु᳚म् । अर्.ह॑ति । दक्षि॑णाम् । का॒र्ष्म॒र्य॒मयी॒मिति॑ कार्ष्मर्य - मयी᳚म् । उत्त॑रा॒मित्युत् - त॒रा॒म् । औदु॑बंरीम् । तस्मा᳚त् । अ॒स्याः । अ॒सौ । उत्त॒रेत्युत् - त॒रा॒ । आज्य॑स्य । पू॒र्णाम् । का॒र्ष्म॒र्य॒मयी॒मिति॑ कार्ष्मर्य - मयी᳚म् । वज्रः॑ । वै । आज्य᳚म् । वज्रः॑ । का॒र्ष्म॒र्यः॑ । वज्रे॑ण । ए॒व । य॒ज्ञ्स्य॑ । द॒क्षि॒ण॒तः । रक्षाꣳ॑सि । अपेति॑ । ह॒न्ति॒ । द॒द्ध्नः । पू॒र्णाम् । औदु॑बंरीम् । प॒शवः॑ । वै । दधि॑ । ऊर्क् । उ॒दु॒बंरः॑ । प॒शुषु॑ । ए॒व । ऊर्ज᳚म् । द॒धा॒ति॒ । पू॒र्णे इति॑ । उपेति॑ । द॒धा॒ति॒ । पू॒र्णे इति॑ । ए॒व । ए॒न॒म् । \textbf{  39} \newline
                  \newline
                                \textbf{ TS 5.2.7.5} \newline
                  अ॒मुष्मिन्न्॑ । लो॒के । उपेति॑ । ति॒ष्ठे॒ते॒ इति॑ । वि॒राजीति॑ वि - राजि॑ । अ॒ग्निः । चे॒त॒व्यः॑ । इति॑ । आ॒हुः॒ । स्रुक् । वै । वि॒राडिति॑ वि-राट् । यत् । स्रुचौ᳚ । उ॒प॒दधा॒तीत्यु॑प - दधा॑ति । वि॒राजीति॑ वि - राजि॑ । ए॒व । अ॒ग्निम् । चि॒नु॒ते॒ । य॒ज्ञ्॒मु॒खे य॑ज्ञ्मुख॒ इति॑ यज्ञ्मु॒खे-य॒ज्ञ्॒मु॒खे॒ । वै । क्रि॒यमा॑णे । य॒ज्ञ्म् । रक्षाꣳ॑सि । जि॒घाꣳ॒॒स॒न्ति॒ । य॒ज्ञ्॒मु॒खमिति॑ यज्ञ् - मु॒खम् । रु॒क्मः । यत् । रु॒क्मम् । व्या॒घा॒रय॒तीति॑ वि - आ॒घा॒रय॑ति । य॒ज्ञ्॒मु॒खादिति॑ यज्ञ् - मु॒खात् । ए॒व । रक्षाꣳ॑सि । अपेति॑ । ह॒न्ति॒ । प॒ञ्चभि॒रिति॑ प॒ञ्च - भिः॒ । व्याघा॑रय॒तीति॑ वि - आघा॑रयति । पाङ्क्तः॑ । य॒ज्ञ्ः । यावान्॑ । ए॒व । य॒ज्ञ्ः । तस्मा᳚त् । रक्षाꣳ॑सि । अपेति॑ । ह॒न्ति॒ । अ॒क्ष्ण॒या । व्याघा॑रय॒तीति॑ वि - आघा॑रयति । तस्मा᳚त् । अ॒क्ष्ण॒या ( ) । प॒शवः॑ । अङ्गा॑नि । प्रेति॑ । ह॒र॒न्ति॒ । प्रति॑ष्ठित्या॒ इति॒ प्रति॑ - स्थि॒त्यै॒ ॥ \textbf{  40 } \newline
                  \newline
                       (वन॒स्पत॑यो- दक्षिण॒तो- ध॑त्त- एनं॒- तस्मा॑ दक्ष्ण॒या-पञ्च॑ च)  \textbf{(A7)} \newline \newline
                                \textbf{ TS 5.2.8.1} \newline
                  स्व॒य॒मा॒तृ॒ण्णामिति॑ स्वयं - आ॒तृ॒ण्णाम् । उपेति॑ । द॒धा॒ति॒ । इ॒यम् । वै । स्व॒य॒मा॒तृ॒ण्णेति॑ स्वयं-आ॒तृ॒ण्णा । इ॒माम् । ए॒व । उपेति॑ । ध॒त्ते॒ । अश्व᳚म् । उपेति॑ । घ्रा॒प॒य॒ति॒ । प्रा॒णमिति॑ प्र-अ॒नम् । ए॒व । अ॒स्या॒म् । द॒धा॒ति॒ । अथो॒ इति॑ । प्रा॒जा॒प॒त्य इति॑ प्रजा - प॒त्यः । वै । अश्वः॑ । प्र॒जाप॑ति॒नेति॑ प्र॒जा - प॒ति॒ना॒ । ए॒व । अ॒ग्निम् । चि॒नु॒ते॒ । प्र॒थ॒मा । इष्ट॑का । उ॒प॒धी॒यमा॒नेत्यु॑प-धी॒यमा॑ना । प॒शू॒नाम् । च॒ । यज॑मानस्य । च॒ । प्रा॒णमिति॑ प्र - अ॒नम् । अपीति॑ । द॒धा॒ति॒ । स्व॒य॒मा॒तृ॒ण्णेति॑ स्वयं - आ॒तृ॒ण्णा । भ॒व॒ति॒ । प्रा॒णाना॒मिति॑ प्र - अ॒नाना᳚म् । उथ्सृ॑ष्ट्या॒ इत्युत् - सृ॒ष्ट्यै॒ । अथो॒ इति॑ । सु॒व॒र्गस्येति॑ सुवः-गस्य॑ । लो॒कस्य॑ । अनु॑ख्यात्या॒ इत्यनु॑ - ख्या॒त्यै॒ । अ॒ग्नौ । अ॒ग्निः । चे॒त॒व्यः॑ । इति॑ । आ॒हुः॒ । ए॒षः । वै । \textbf{  41} \newline
                  \newline
                                \textbf{ TS 5.2.8.2} \newline
                  अ॒ग्निः । वै॒श्वा॒न॒रः । यत् । ब्रा॒ह्म॒णः । तस्मै᳚ । प्र॒थ॒माम् । इष्ट॑काम् । यजु॑ष्कृता॒मिति॒ यजुः॑ - कृ॒ता॒म् । प्रेति॑ । य॒च्छे॒त् । ताम् । ब्रा॒ह्म॒णः । च॒ । उपेति॑ । द॒द्ध्या॒ता॒म् । अ॒ग्नौ । ए॒व । तत् । अ॒ग्निम् । चि॒नु॒ते॒ । ई॒श्व॒रः । वै । ए॒षः । आर्ति᳚म् । आर्तो॒रित्या-अ॒र्तोः॒ । यः । अवि॑द्वान् । इष्ट॑काम् । उ॒प॒द॒धा॒तीत्यु॑प - दधा॑ति । त्रीन् । वरान्॑ । द॒द्या॒त् । त्रयः॑ । वै । प्रा॒णा इति॑ प्र - अ॒नाः । प्रा॒णाना॒मिति॑ प्र - अ॒नाना᳚म् । स्पृत्यै᳚ । द्वौ । ए॒व । देयौ᳚ । द्वौ । हि । प्रा॒णाविति॑ प्र - अ॒नौ । एकः॑ । ए॒व । देयः॑ । एकः॑ । हि । प्रा॒ण इति॑ प्र - अ॒नः । प॒शुः । \textbf{  42} \newline
                  \newline
                                \textbf{ TS 5.2.8.3} \newline
                  वै । ए॒षः । यत् । अ॒ग्निः । न । खलु॑ । वै । प॒शवः॑ । आय॑वसे । र॒म॒न्ते॒ । दू॒र्वे॒ष्ट॒कामिति॑ दूर्वा - इ॒ष्ट॒काम् । उपेति॑ । द॒धा॒ति॒ । प॒शू॒नाम् । धृत्यै᳚ । द्वाभ्या᳚म् । प्रति॑ष्ठित्या॒ इति॒ प्रति॑ - स्थि॒त्यै॒ । काण्डा᳚त्काण्डा॒दिति॒ काण्डा᳚त् - का॒ण्डा॒त् । प्र॒रोह॒न्तीति॑ प्र-रोह॑न्ती । इति॑ । आ॒ह॒ । काण्डे॑नकाण्डे॒नेति॒ काण्डे॑न - का॒ण्डे॒न॒ । हि । ए॒षा । प्र॒ति॒तिष्ठ॒तीति॑ प्रति - तिष्ठ॑ति । ए॒वा । नः॒ । दू॒र्वे॒ । प्रेति॑ । त॒नु॒ । स॒हस्रे॑ण । श॒तेन॑ । च॒ । इति॑ । आ॒ह॒ । सा॒ह॒स्रः । प्र॒जाप॑ति॒रिति॑ प्र॒जा - प॒तिः॒ । प्र॒जाप॑ते॒रिति॑ प्र॒जा - प॒तेः॒ । आप्त्यै᳚ । दे॒व॒ल॒क्ष्ममिति॑ देव-ल॒क्ष्मम् । वै । त्र्या॒लि॒खि॒तेति॑ त्रि - आ॒लि॒खि॒ता । ताम् । उत्त॑रलक्ष्माण॒मित्युत्त॑र - ल॒क्ष्मा॒ण॒म् । दे॒वाः । उपेति॑ । अ॒द॒ध॒त॒ । अध॑रलक्ष्माण॒मित्यध॑र - ल॒क्ष्मा॒ण॒म् । असु॑राः । यम् । \textbf{  43} \newline
                  \newline
                                \textbf{ TS 5.2.8.4} \newline
                  का॒मये॑त । वसी॑यान् । स्या॒त् । इति॑ । उत्त॑रलक्ष्माण॒मित्युत्त॑र-ल॒क्ष्मा॒ण॒म् । तस्य॑ । उपेति॑ । द॒द्ध्या॒त् । वसी॑यान् । ए॒व । भ॒व॒ति॒ । यम् । का॒मये॑त । पापी॑यान् । स्या॒त् । इति॑ । अध॑रलक्ष्माण॒मित्यध॑र- ल॒क्ष्मा॒ण॒म् । तस्य॑ । उपेति॑ । द॒द्ध्या॒त् । अ॒सु॒र॒यो॒निमित्य॑सुर - यो॒निम् । ए॒व । ए॒न॒म् । अनु॑ । परेति॑ । भा॒व॒य॒ति॒ । पापी॑यान् । भ॒व॒ति॒ । त्र्या॒लि॒खि॒तेति॑ त्रि - आ॒लि॒खि॒ता । भ॒व॒ति॒ । इ॒मे । वै । लो॒काः । त्र्या॒लि॒खि॒तेति॑ त्रि - आ॒लि॒खि॒ता । ए॒भ्यः । ए॒व । लो॒केभ्यः॑ । भ्रातृ॑व्यम् । अ॒न्तः । ए॒ति॒ । अङ्गि॑रसः । सु॒व॒र्गमिति॑ सुवः - गम् । लो॒कम् । य॒तः । पु॒रो॒डाशः॑ । कू॒र्मः । भू॒त्वा । अनु॑ । प्रेति॑ । अ॒स॒र्प॒त् । \textbf{  44} \newline
                  \newline
                                \textbf{ TS 5.2.8.5} \newline
                  यत् । कू॒र्मम् । उ॒प॒दधा॒तीत्यु॑प - दधा॑ति । यथा᳚ । क्षे॒त्र॒विदिति॑ क्षेत्र - वित् । अञ्ज॑सा । नय॑ति । ए॒वम् । ए॒व । ए॒न॒म् । कू॒र्मः । सु॒व॒र्गमिति॑ सुवः - गम् । लो॒कम् । अञ्ज॑सा । न॒य॒ति॒ । मेधः॑ । वै । ए॒षः । प॒शू॒नाम् । यत् । कू॒र्मः । यत् । कू॒र्मम् । उ॒प॒दधा॒तीत्यु॑प - दधा॑ति । स्वम् । ए॒व । मेध᳚म् । पश्य॑न्तः । प॒शवः॑ । उपेति॑ । ति॒ष्ठ॒न्ते॒ । श्म॒शा॒नम् । वै । ए॒तत् । क्रि॒य॒ते॒ । यत् । मृ॒ताना᳚म् । प॒शू॒नाम् । शी॒र्॒.षाणि॑ । उ॒प॒धी॒यन्त॒ इत्यु॑प - धी॒यन्ते᳚ । यत् । जीव॑न्तम् । कू॒र्मम् । उ॒प॒दधा॒तीत्यु॑प - दधा॑ति । तेन॑ । अश्म॑शानचि॒दित्यश्म॑शान - चि॒त् । वा॒स्त॒व्यः॑ । वै । ए॒षः । यत् । \textbf{  45} \newline
                  \newline
                                \textbf{ TS 5.2.8.6} \newline
                  कू॒र्मः । मधु॑ । वाताः᳚ । ऋ॒ता॒य॒त इत्यृ॑त - य॒ते । इति॑ । द॒द्ध्ना । म॒धु॒मि॒श्रेणेति॑ मधु - मि॒श्रेण॑ । अ॒भीति॑ । अ॒न॒क्ति॒ । स्व॒दय॑ति । ए॒व । ए॒न॒म् । ग्रा॒म्यम् । वै । ए॒तत् । अन्न᳚म् । यत् । दधि॑ । आ॒र॒ण्यम् । मधु॑ । यत् । द॒द्ध्ना । म॒धु॒मि॒श्रेणेति॑ मधु - मि॒श्रेण॑ । अ॒भ्य॒नक्तीत्य॑भि - अ॒नक्ति॑ । उ॒भय॑स्य । अव॑रुद्ध्या॒ इत्यव॑-रु॒द्ध्यै॒ । म॒ही । द्यौः । पृ॒थि॒वी । च॒ । नः॒ । इति॑ । आ॒ह॒ । आ॒भ्याम् । ए॒व । ए॒न॒म् । उ॒भ॒यतः॑ । परीति॑ । गृ॒ह्णा॒ति॒ । प्राञ्च᳚म् । उपेति॑ । द॒धा॒ति॒ । सु॒व॒र्गस्येति॑ सुवः - गस्य॑ । लो॒कस्य॑ । सम॑ष्ट्या॒ इति॒ सं-अ॒ष्ट्यै॒ । पु॒रस्ता᳚त् । प्र॒त्यञ्च᳚म् । उपेति॑ । द॒धा॒ति॒ । तस्मा᳚त् । \textbf{  46} \newline
                  \newline
                                \textbf{ TS 5.2.8.7} \newline
                  पु॒रस्ता᳚त् । प्र॒त्यञ्चः॑ । प॒शवः॑ । मेध᳚म् । उपेति॑ । ति॒ष्ठ॒न्ते॒ । यः । वै । अप॑नाभि॒मित्यप॑ - ना॒भि॒म् । अ॒ग्निम् । चि॒नु॒ते । यज॑मानस्य । नाभि᳚म् । अनु॑ । प्रेति॑ । वि॒श॒ति॒ । सः । ए॒न॒म् । ई॒श्व॒रः । हिꣳसि॑तोः । उ॒लूख॑लम् । उपेति॑ । द॒धा॒ति॒ । ए॒षा । वै । अ॒ग्नेः । नाभिः॑ । सना॑भि॒मिति॒ स - ना॒भि॒म् । ए॒व । अ॒ग्निम् । चि॒नु॒त॒ । अहिꣳ॑सायै । औदु॑बंरम् । भ॒व॒ति॒ । ऊर्क् । वै । उ॒दु॒बंरः॑ । ऊर्ज᳚म् । ए॒व । अवेति॑ । रु॒न्धे॒ । म॒द्ध्य॒तः । उपेति॑ । द॒धा॒ति॒ । म॒द्ध्य॒तः । ए॒व । अ॒स्मै॒ । ऊर्ज᳚म् । द॒धा॒ति॒ । तस्मा᳚त् ( ) । म॒द्ध्य॒तः । ऊ॒र्जा । भु॒ञ्ज॒ते॒ । इय॑त् । भ॒व॒ति॒ । प्र॒जाप॑ति॒नेति॑ प्र॒जा - प॒ति॒ना॒ । य॒ज्ञ्॒मु॒खेनेति॑ यज्ञ् - मु॒खेन॑ । सम्मि॑त॒मिति॒ सं - मि॒त॒म् । अवेति॑ । ह॒न्ति॒ । अन्न᳚म् । ए॒व । अ॒कः॒ । वै॒ष्ण॒व्या । ऋ॒चा । उपेति॑ । द॒धा॒ति॒ । विष्णुः॑ । वै । य॒ज्ञ्ः । वै॒ष्ण॒वाः । वन॒स्पत॑यः । य॒ज्ञे । ए॒व । य॒ज्ञ्म् । प्रतीति॑ । स्था॒प॒य॒ति॒ ॥ \textbf{  47} \newline
                  \newline
                      (ए॒ष वै - प॒शु - र्य - म॑सर्प - दे॒ष यत् - तस्मा॒त् - तस्मा᳚थ् - स॒प्तविꣳ॑शतिश्च)  \textbf{(A8)} \newline \newline
                                \textbf{ TS 5.2.9.1} \newline
                  ए॒षाम् । वै । ए॒तत् । लो॒काना᳚म् । ज्योतिः॑ । संभृ॑त॒मिति॒ सं-भृ॒त॒म् । यत् । उ॒खा । यत् । उ॒खाम् । उ॒प॒दधा॒तीत्यु॑प-दधा॑ति । ए॒भ्यः । ए॒व । लो॒केभ्यः॑ । ज्योतिः॑ । अवेति॑ । रु॒न्धे॒ । म॒द्ध्य॒तः । उपेति॑ । द॒धा॒ति॒ । म॒द्ध्य॒तः । ए॒व । अ॒स्मै॒ । ज्योतिः॑ । द॒धा॒ति॒ । तस्मा᳚त् । म॒द्ध्य॒तः । ज्योतिः॑ । उपेति॑ । आ॒स्म॒हे॒ । सिक॑ताभिः । पू॒र॒य॒ति॒ । ए॒तत् । वै । अ॒ग्नेः । वै॒श्वा॒न॒रस्य॑ । रू॒पम् । रू॒पेण॑ । ए॒व । वै॒श्वा॒न॒रम् । अवेति॑ । रु॒न्धे॒ । यम् । का॒मये॑त । क्षोधु॑कः । स्या॒त् । इति॑ । ऊ॒नाम् । तस्य॑ । उपेति॑ । \textbf{  48} \newline
                  \newline
                                \textbf{ TS 5.2.9.2} \newline
                  द॒द्ध्या॒त् । क्षोधु॑कः । ए॒व । भ॒व॒ति॒ । यम् । का॒मये॑त । अनु॑पदस्य॒दित्यनु॑प - द॒स्य॒त् । अन्न᳚म् । अ॒द्या॒त् । इति॑ । पू॒र्णाम् । तस्य॑ । उपेति॑ । द॒द्ध्या॒त् । अनु॑पदस्य॒दित्यनु॑प - द॒स्य॒त् । ए॒व । अन्न᳚म् । अ॒त्ति॒ । स॒हस्र᳚म् । वै । प्रतीति॑ । पुरु॑षः । प॒शू॒नाम् । य॒च्छ॒ति॒ । स॒हस्र᳚म् । अ॒न्ये । प॒शवः॑ । मद्ध्ये᳚ । पु॒रु॒ष॒शी॒र्॒.षमिति॑ पुरुष - शी॒र्॒.षम् । उपेति॑ । द॒धा॒ति॒ । स॒वी॒र्य॒त्वायेति॑ सवीर्य-त्वाय॑ । उ॒खाया᳚म् । अपीति॑ । द॒धा॒ति॒ । प्र॒ति॒ष्ठामिति॑ प्रति - स्थाम् । ए॒व । ए॒न॒त् । ग॒म॒य॒ति॒ । व्यृ॑द्ध॒मिति॒ वि - ऋ॒द्ध॒म् । वै । ए॒तत् । प्रा॒णैरिति॑ प्र - अ॒नैः । अ॒मे॒द्ध्यम् । यत् । पु॒रु॒ष॒शी॒र्॒.षमिति॑ पुरुष - शी॒र्॒.षम् । अ॒मृत᳚म् । खलु॑ । वै । प्रा॒णा इति॑ प्र - अ॒नाः । \textbf{  49} \newline
                  \newline
                                \textbf{ TS 5.2.9.3} \newline
                  अ॒मृत᳚म् । हिर॑ण्यम् । प्रा॒णेष्विति॑ प्र - अ॒नेषु॑ । हि॒र॒ण्य॒श॒ल्कानिति॑ हिरण्य - श॒ल्कान् । प्रतीति॑ । अ॒स्य॒ति॒ । प्र॒ति॒ष्ठामिति॑ प्रति-स्थाम् । ए॒व । ए॒न॒त् । ग॒म॒यि॒त्वा । प्रा॒णैरिति॑ प्र-अ॒नैः । समिति॑ । अ॒द्‌र्ध॒य॒ति॒ । द॒द्ध्ना । म॒धु॒मि॒श्रेणेति॑ मधु-मि॒श्रेण॑ । पू॒र॒य॒ति॒ । म॒ध॒व्यः॑ । अ॒सा॒नि॒ । इति॑ । शृ॒ता॒त॒ङ्क्ये॑नेति॑ शृत - आ॒त॒ङ्क्ये॑न । मे॒द्ध्य॒त्वायेति॑ मेद्ध्य - त्वाय॑ । ग्रा॒म्यम् । वै । ए॒तत् । अन्न᳚म् । यत् । दधि॑ । आ॒र॒ण्यम् । मधु॑ । यत् । द॒द्ध्ना । म॒धु॒मि॒श्रेणेति॑ मधु - मि॒श्रेण॑ । पू॒रय॑ति । उ॒भय॑स्य । अव॑रुद्ध्या॒ इत्यव॑ - रु॒द्ध्यै॒ । प॒शु॒शी॒र्॒.षाणीति॑ पशु - शी॒र्॒.षाणि॑ । उपेति॑ । द॒धा॒ति॒ । प॒शवः॑ । वै । प॒शु॒शी॒र्॒.षाणीति॑ पशु - शी॒र्॒.षाणि॑ । प॒शून् । ए॒व । अवेति॑ । रु॒न्धे॒ । यम् । का॒मये॑त । अ॒प॒शुः । स्या॒त् । इति॑ । \textbf{  50 } \newline
                  \newline
                                \textbf{ TS 5.2.9.4} \newline
                  वि॒षू॒चीना॑नि । तस्य॑ । उपेति॑ । द॒द्ध्या॒त् । विषू॑चः । ए॒व । अ॒स्मा॒त् । प॒शून् । द॒धा॒ति॒ । अ॒प॒शुः । ए॒व । भ॒व॒ति॒ । यम् । का॒मये॑त । प॒शु॒मानिति॑ पशु - मान् । स्या॒त् । इति॑ । स॒मी॒चीना॑नि । तस्य॑ । उपेति॑ । द॒द्ध्या॒त् । स॒मीचः॑ । ए॒व । अ॒स्मै॒ । प॒शून् । द॒धा॒ति॒ । प॒शु॒मानिति॑ पशु - मान् । ए॒व । भ॒व॒ति॒ । पु॒रस्ता᳚त् । प्र॒ती॒चीन᳚म् । अश्व॑स्य । उपेति॑ । द॒धा॒ति॒ । प॒श्चात् । प्रा॒चीन᳚म् । ऋ॒ष॒भस्य॑ । अप॑शवः । वै । अ॒न्ये । गो॒ अ॒श्वेभ्य॒ इति॑ गो -अ॒श्वेभ्यः॑ । प॒शवः॑ । गो॒ अ॒श्वानिति॑ गो - अ॒श्वान् । ए॒व । अ॒स्मै॒ । स॒मीचः॑ । द॒धा॒ति॒ । ए॒ताव॑न्तः । वै । प॒शवः॑ । \textbf{  51} \newline
                  \newline
                                \textbf{ TS 5.2.9.5} \newline
                  द्वि॒पाद॒ इति॑ द्वि - पादः॑ । च॒ । चतु॑ष्पाद॒ इति॒ चतुः॑ - पा॒दः॒ । च॒ । तान् । वै । ए॒तत् । अ॒ग्नौ । प्रेति॑ । द॒धा॒ति॒ । यत् । प॒शु॒शी॒र्॒.षाणीति॑ पशु - शी॒र्॒.षाणि॑ । उ॒प॒दधा॒तीत्यु॑प - दधा॑ति । अ॒मुम् । आ॒र॒ण्यम् । अन्विति॑ । ते॒ । दि॒शा॒मि॒ । इति॑ । आ॒ह॒ । ग्रा॒म्येभ्यः॑ । ए॒व । प॒शुभ्य॒ इति॑ प॒शु - भ्यः॒ । आ॒र॒ण्यान् । प॒शून् । शुच᳚म् । अनू॑ । उदिति॑ । सृ॒ज॒ति॒ । तस्मा᳚त् । स॒माव॑त् । प॒शू॒नाम् । प्र॒जाय॑मानाना॒मिति॑ प्र - जाय॑मानानाम् । आ॒र॒ण्याः । प॒शवः॑ । कनी॑याꣳसः । शु॒चा । हि । ऋ॒ताः । स॒र्प॒शी॒र्॒.षमिति॑ सर्प - शी॒र्॒.षम् । उपेति॑ । द॒धा॒ति॒ । या । ए॒व । स॒र्पे । त्विषिः॑ । ताम् । ए॒व । अवेति॑ । रु॒न्धे॒ । \textbf{  52} \newline
                  \newline
                                \textbf{ TS 5.2.9.6} \newline
                  यत् । स॒मी॒चीन᳚म् । प॒शु॒शी॒र्॒.षैरिति॑ पशु - शी॒र्॒.षैः । उ॒प॒द॒द्ध्यादित्यु॑प - द॒द्ध्यात् । ग्रा॒म्यान् । प॒शून् । दꣳशु॑काः । स्युः॒ । यत् । वि॒षू॒चीन᳚म् । आ॒र॒ण्यान् । यजुः॑ । ए॒व । व॒दे॒त् । अवेति॑ । ताम् । त्विषि᳚म् । रु॒न्धे॒ । या । स॒र्पे । न । ग्रा॒म्यान् । प॒शून् । हि॒नस्ति॑ । न । आ॒र॒ण्यान् । अथो॒ इति॑ । खलु॑ । उ॒प॒धेय॒मित्यु॑प - धेय᳚म् । ए॒व । यत् । उ॒प॒दधा॒तीत्यु॑प - दधा॑ति । तेन॑ । ताम् । त्विषि᳚म् । अवेति॑ । रु॒न्धे॒ । या । स॒र्पे । यत् । यजुः॑ । वद॑ति । तेन॑ । शा॒न्तम् ॥ \textbf{  53} \newline
                  \newline
                      (ऊ॒नान्तस्योप॑ - प्रा॒णाः - स्या॒दिति॒ - वै प॒शवो॑ - रुन्धे॒ - चतु॑श्चत्वारिꣳशच्च)  \textbf{(A9)} \newline \newline
                                \textbf{ TS 5.2.10.1} \newline
                  प॒शुः । वै । ए॒षः । यत् । अ॒ग्निः । योनिः॑ । खलु॑ । वै । ए॒षा । प॒शोः । वीति॑ । क्रि॒य॒ते॒ । यत् । प्रा॒चीन᳚म् । ऐ॒ष्ट॒कात् । यजुः॑ । क्रि॒यते᳚ । रेतः॑ । अ॒प॒स्याः᳚ । अ॒प॒स्याः᳚ । उपेति॑ । द॒धा॒ति॒ । योनौ᳚ । ए॒व । रेतः॑ । द॒धा॒ति॒ । पञ्च॑ । उपेति॑ । द॒धा॒ति॒ । पाङ्क्ताः᳚ । प॒शवः॑ । प॒शून् । ए॒व । अ॒स्मै॒ । प्रेति॑ । ज॒न॒य॒ति॒ । पञ्च॑ । द॒क्षि॒ण॒तः । वज्रः॑ । वै । अ॒प॒स्याः᳚ । वज्रे॑ण । ए॒व । य॒ज्ञ्स्य॑ । द॒क्षि॒ण॒तः । रक्षाꣳ॑सि । अपेति॑ । ह॒न्ति॒ । पञ्च॑ । प॒श्चात् । \textbf{  54} \newline
                  \newline
                                \textbf{ TS 5.2.10.2} \newline
                  प्राचीः᳚ । उपेति॑ । द॒धा॒ति॒ । प॒श्चात् । वै । प्रा॒चीन᳚म् । रेतः॑ । धी॒य॒ते॒ । प॒श्चात् । ए॒व । अ॒स्मै॒ । प्रा॒चीन᳚म् । रेतः॑ । द॒धा॒ति॒ । पञ्च॑ । पु॒रस्ता᳚त् । प्र॒तीचीः᳚ । उपेति॑ । द॒धा॒ति॒ । पञ्च॑ । प॒श्चात् । प्राचीः᳚ । तस्मा᳚त् । प्रा॒चीन᳚म् । रेतः॑ । धी॒य॒ते॒ । प्र॒तीचीः᳚ । प्र॒जा इति॑ प्र-जाः । जा॒य॒न्ते॒ । पञ्च॑ । उ॒त्त॒र॒त इत्यु॑त् - त॒र॒तः । छ॒न्द॒स्याः᳚ । प॒शवः॑ । वै । छ॒न्द॒स्याः᳚ । प॒शून् । ए॒व । प्रजा॑ता॒निति॒ प्र - जा॒ता॒न् । स्वम् । आ॒यत॑न॒मित्या᳚ - यत॑नम् । अ॒भि । परीति॑ । ऊ॒ह॒ते॒ । इ॒यम् । वै । अ॒ग्नेः । अ॒ति॒दा॒हादित्य॑ति - दा॒हात् । अ॒बि॒भे॒त् । सा । ए॒ताः । \textbf{  55} \newline
                  \newline
                                \textbf{ TS 5.2.10.3} \newline
                  अ॒प॒स्याः᳚ । अ॒प॒श्य॒त् । ताः । उपेति॑ । अ॒ध॒त्त॒ । ततः॑ । वै । इ॒माम् । न । अतीति॑ । अ॒द॒ह॒त् । यत् । अ॒प॒स्याः᳚ । उ॒प॒दधा॒तीत्यु॑प - दधा॑ति । अ॒स्याः । अन॑तिदाहा॒येत्यन॑ति - दा॒हा॒य॒ । उ॒वाच॑ । ह॒ । इ॒यम् । अद॑त् । इत् । सः । ब्रह्म॑णा । अन्न᳚म् । यस्य॑ । ए॒ताः । उ॒प॒धी॒यान्ता॒ इत्यु॑प - धी॒यान्तै᳚ । यः । उ॒ । च॒ । ए॒नाः॒ । ए॒वम् । वेद॑त् । इति॑ । प्रा॒ण॒भृत॒ इति॑ प्राण-भृतः॑ । उपेति॑ । द॒धा॒ति॒ । रेत॑सि । ए॒व । प्रा॒णानिति॑ प्र - अ॒नान् । द॒धा॒ति॒ । तस्मा᳚त् । वदन्न्॑ । प्रा॒णन्निति॑ प्र - अ॒नन्न् । पश्यन्न्॑ । शृ॒ण्वन्न् । प॒शुः । जा॒य॒ते॒ । अ॒यम् । पु॒रः । \textbf{  56} \newline
                  \newline
                                \textbf{ TS 5.2.10.4} \newline
                  भुवः॑ । इति॑ । पु॒रस्ता᳚त् । उपेति॑ । द॒धा॒ति॒ । प्रा॒णमिति॑ प्र - अ॒नम् । ए॒व । ए॒ताभिः॑ । दा॒धा॒र॒ । अ॒यम् । द॒क्षि॒णा । वि॒श्वक॒र्मेति॑ वि॒श्व-क॒र्मा॒ । इति॑ । द॒क्षि॒ण॒तः । मनः॑ । ए॒व । ए॒ताभिः॑ । दा॒धा॒र॒ । अ॒यम् । प॒श्चात् । वि॒श्वव्य॑चा॒ इति॑ वि॒श्व - व्य॒चाः॒ । इति॑ । प॒श्चात् । चक्षुः॑ । ए॒व । ए॒ताभिः॑ । दा॒धा॒र॒ । इ॒दम् । उ॒त्त॒रादित्यु॑त् - त॒रात् । सुवः॑ । इति॑ । उ॒त्त॒र॒त इत्यु॑त् - त॒र॒तः । श्रोत्र᳚म् । ए॒व । ए॒ताभिः॑ । दा॒धा॒र॒ । इ॒यम् । उ॒परि॑ । म॒तिः । इति॑ । उ॒परि॑ष्टात् । वाच᳚म् । ए॒व । ए॒ताभिः॑ । दा॒धा॒र॒ । दश॑द॒शेति॒ दश॑-द॒श॒ । उपेति॑ । द॒धा॒ति॒ । स॒वी॒र्य॒त्वायेति॑ सवीर्य-त्वाय॑ । अ॒क्ष्ण॒या । \textbf{  57} \newline
                  \newline
                                \textbf{ TS 5.2.10.5} \newline
                  उपेति॑ । द॒धा॒ति॒ । तस्मा᳚त् । अ॒क्ष्ण॒या । प॒शवः॑ । अङ्गा॑नि । प्रेति॑ । ह॒र॒न्ति॒ । प्रति॑ष्ठित्या॒ इति॒ प्रति॑ - स्थि॒त्यै॒ । याः । प्राचीः᳚ । ताभिः॑ । वसि॑ष्ठः । आ॒द्‌र्ध्नो॒त् । याः । द॒क्षि॒णा । ताभिः॑ । भ॒रद्वा॑जः । याः । प्र॒तीचीः᳚ । ताभिः॑ । वि॒श्वामि॑त्र॒ इति॑ वि॒श्व - मि॒त्रः॒ । याः । उदी॑चीः । ताभिः॑ । ज॒मद॑ग्निः । याः । ऊ॒द्‌र्ध्वाः । ताभिः॑ । वि॒श्वक॒र्मेति॑ वि॒श्व - क॒र्मा॒ । यः । ए॒वम् । ए॒तासा᳚म् । ऋद्धि᳚म् । वेद॑ । ऋ॒द्ध्नोति॑ । ए॒व । यः । आ॒सा॒म् । ए॒वम् । ब॒न्धुता᳚म् । वेद॑ । बन्धु॑मा॒निति॒ बन्धु॑ - मा॒न् । भ॒व॒ति॒ । यः । आ॒सा॒म् । ए॒वम् । क्लृप्ति᳚म् । वेद॑ । कल्प॑ते । \textbf{  58} \newline
                  \newline
                                \textbf{ TS 5.2.10.6} \newline
                  अ॒स्मै॒ । यः । आ॒सा॒म् । ए॒वम् । आ॒यत॑न॒मित्या᳚ - यत॑नम् । वेद॑ । आ॒यत॑नवा॒नित्या॒यत॑न - वा॒न् । भ॒व॒ति॒ । यः । आ॒सा॒म् । ए॒वम् । प्र॒ति॒ष्ठामिति॑ प्रति - स्थाम् । वेद॑ । प्रतीति॑ । ए॒व । ति॒ष्ठ॒ति॒ । प्रा॒ण॒भृत॒ इति॑ प्राण - भृतः॑ । उ॒प॒धायेत्यु॑प - धाय॑ । सं॒ॅयत॒ इति॑ सं - यतः॑ । उपेति॑ । द॒धा॒ति॒ । प्रा॒णा॒निति॑ प्र - अ॒नान् । ए॒व । अ॒स्मि॒न्न् । धि॒त्वा । सं॒ॅयद्भि॒रिति॑ सं॒ॅयत् - भिः॒ । समिति॑ । य॒च्छ॒ति॒ । तत् । सं॒ॅयता॒मिति॑ सं - यता᳚म् । सं॒ॅय॒त्त्वमिति॑ संॅयत् - त्वम् । अथो॒ इति॑ । प्रा॒ण इति॑ प्र - अ॒ने । ए॒व । अ॒पा॒नमित्य॑प - अ॒नम् । द॒धा॒ति॒ । तस्मा᳚त् । प्रा॒णा॒पा॒नाविति॑ प्राण - अ॒पा॒नौ । समिति॑ । च॒र॒तः॒ । विषू॑चीः । उपेति॑ । द॒धा॒ति॒ । तस्मा᳚त् । विष्व॑ञ्चौ । प्रा॒णा॒पा॒नाविति॑ प्राण-अ॒पा॒नौ । यत् । वै । अ॒ग्नेः । असं॑ॅयत॒मित्यसं᳚ - य॒त॒म् । \textbf{  59} \newline
                  \newline
                                \textbf{ TS 5.2.10.7} \newline
                  असु॑वर्ग्य॒मित्यसु॑वः - ग्य॒म् । अ॒स्य॒ । तत् । सु॒व॒र्ग्य॑ इति॑ सुवः - ग्यः॑ । अ॒ग्निः । यत् । सं॒ॅयत॒ इति॑ सं - यतः॑ । उ॒प॒दधा॒तीत्यु॑प - दधा॑ति । समिति॑ । ए॒व । ए॒न॒म् । य॒च्छ॒ति॒ । सु॒व॒र्ग्य॑मिति॑ सुवः - ग्य᳚म् । ए॒व । अ॒कः॒ । त्र्यवि॒रिति॑ त्रि - अविः॑ । वयः॑ । कृ॒तम् । अया॑नाम् । इति॑ । आ॒ह॒ । वयो॑भि॒रिति॒ वयः॑ - भिः॒ । ए॒व । अयान्॑ । अवेति॑ । रु॒न्धे॒ । अयैः᳚ । वयाꣳ॑सि । स॒र्वतः॑ । वा॒यु॒मती॒रिति॑ वायु - मतीः᳚ । भ॒व॒न्ति॒ । तस्मा᳚त् । अ॒यम् । स॒र्वतः॑ । प॒व॒ते॒ ॥ \textbf{  60 } \newline
                  \newline
                       (प॒श्चा - दे॒ताः - पु॒रो᳚ - ऽक्ष्ण॒या - कल्प॒ते - ऽसं॑ ॅयतं॒ - पञ्च॑त्रिꣳशच्च)  \textbf{(A10)} \newline \newline
                                \textbf{ TS 5.2.11.1} \newline
                  गा॒य॒त्री । त्रि॒ष्टुप् । जग॑ती । अ॒नु॒ष्टुगित्य॑नु-स्तुक् । प॒ङ्क्त्या᳚ । स॒ह ॥ बृ॒ह॒ती । उ॒ष्णिहा᳚ । क॒कुत् । सू॒चीभिः॑ । शि॒म्य॒न्तु॒ । त्वा॒ ॥ द्वि॒पदेति॑ द्वि - पदा᳚ । या । चतु॑ष्प॒देति॒ चतुः॑ - प॒दा॒ । त्रि॒पदेति॑ त्रि - पदा᳚ । या । च॒ । षट्प॒देति॒ षट् - प॒दा॒ ॥ सछ॑न्दा॒ इति॒ स - छ॒न्दाः॒ । या । च॒ । विच्छ॑न्दा॒ इति॒ वि - छ॒न्दाः॒ । सू॒चीभिः॑ । शि॒म्य॒न्तु॒ । त्वा॒ ॥ म॒हाना᳚म्नी॒रिति॑ म॒हा - ना॒म्नीः॒ । रे॒वत॑यः । विश्वाः᳚ । आशाः᳚ । प्र॒सूव॑री॒रिति॑ प्र - सूव॑रीः ॥ मेघ्‌याः᳚ । वि॒द्युत॒ इति॑ वि - द्युतः॑ । वाचः॑ । सू॒चीभिः॑ । शि॒म्य॒न्तु॒ । त्वा॒ ॥ र॒ज॒ताः । हरि॑णीः । सीसाः᳚ । युजः॑ । यु॒ज्य॒न्ते॒ । कर्म॑भि॒रिति॒ कर्म॑ -  भिः॒ ॥ अश्व॑स्य । वा॒जिनः॑ । त्व॒चि । सू॒चीभिः॑ । शि॒म्य॒न्तु॒ । त्वा॒ ॥ नारीः᳚ । \textbf{  61} \newline
                  \newline
                                \textbf{ TS 5.2.11.2} \newline
                  ते । पत्न॑यः । लोम॑ । वीति॑ । चि॒न्व॒न्तु॒ । म॒नी॒षया᳚ ॥ दे॒वाना᳚म् । पत्नीः᳚ । दिशः॑ । सू॒चीभिः॑ । शि॒म्य॒न्तु॒ । त्वा॒ ॥ कु॒वित् । अ॒ङ्ग । यव॑मन्त॒ इति॒ यव॑ - म॒न्तः॒ । यव᳚म् । चि॒त् । यथा᳚ । दान्ति॑ । अ॒नु॒पू॒र्वमित्य॑नु - पू॒र्वम् । वि॒यूयेति॑ वि - यूय॑ ॥ इ॒हेहेती॒ह - इ॒ह॒ । ए॒षा॒म् । कृ॒णु॒त॒ । भोज॑नानि । ये । ब॒र्॒.हिषः॑ । नमो॑वृक्ति॒मिति॒ नमः॑ - वृ॒क्ति॒म् । न । ज॒ग्मुः ॥ \textbf{  62} \newline
                  \newline
                      (नारी᳚ - स्त्रिꣳ॒॒शच्च॑)  \textbf{(A11)} \newline \newline
                                \textbf{ TS 5.2.12.1} \newline
                  कः । त्वा॒ । छ्य॒ति॒ । कः । त्वा॒ । वीति॑ । शा॒स्ति॒ । कः । ते॒ । गात्रा॑णि । शि॒म्य॒ति॒ ॥ कः । उ॒ । ते॒ । श॒मि॒ता । क॒विः ॥ ऋ॒तवः॑ । ते॒ । ऋ॒तु॒धेत्यृ॑तु - धा । परुः॑ । श॒मि॒तारः॑ । वीति॑ । शा॒स॒तु॒ ॥ सं॒ॅव॒थ्स॒रस्येति॑ सं - व॒थ्स॒रस्य॑ । धाय॑सा । शिमी॑भिः । शि॒म्य॒न्तु॒ । त्वा॒ ॥ दैव्याः᳚ । अ॒द्ध्व॒र्यवः॑ । त्वा॒ । छ्यन्तु॑ । वीति॑ । च॒ । शा॒स॒तु॒ ॥ गात्रा॑णि । प॒र्व॒श इति॑ पर्व - शः । ते॒ । शिमाः᳚ । कृ॒ण्व॒न्तु॒ । शिम्य॑न्तः ॥ अ॒द्‌र्ध॒मा॒सा इत्य॑द्‌र्ध - मा॒साः । परूꣳ॑षि । ते॒ । मासाः᳚ । छ्य॒न्तु॒ । शिम्य॑न्तः ॥ अ॒हो॒रा॒त्राणीत्य॑हः - रा॒त्राणि॑ । म॒रुतः॑ । विलि॑ष्ट॒मिति॒ वि - लि॒ष्ट॒म् । \textbf{  63 } \newline
                  \newline
                                \textbf{ TS 5.2.12.2} \newline
                  सू॒द॒य॒न्तु॒ । ते॒ ॥ पृ॒थि॒वी । ते॒ । अ॒न्तरि॑क्षेण । वा॒युः । छि॒द्रम् । भि॒ष॒ज्य॒तु॒ ॥ द्यौः । ते॒ । नक्ष॑त्रैः । स॒ह । रू॒पम् । कृ॒णो॒तु॒ । सा॒धु॒या ॥ शम् । ते॒ । परे᳚भ्यः । गात्रे᳚भ्यः । शम् । अ॒स्तु॒ । अव॑रेभ्यः ॥ शम् । अ॒स्थभ्य॒ इत्य॒स्थ - भ्यः॒ । म॒ज्जभ्य॒ इति॑ म॒ज्ज - भ्यः॒ । शम् । उ॒ । ते॒ । त॒नुवे᳚ । भु॒व॒त् ॥ \textbf{  64} \newline
                  \newline
                      (विलि॑ष्टं - त्रिꣳ॒॒शच्च॑)  \textbf{(A12)} \newline \newline
\textbf{praSna korvai with starting padams of 1 to 12 anuvAkams :-} \newline
(विष्णु॑मुखा॒ - अन्न॑पते॒ - याव॑ती॒ - वि वै - पु॑रुषमा॒त्रेणा - ऽग्ने॒ तव॒ श्रवो॒ वयो॒ - ब्रह्म॑ जज्ञा॒नꣳ - स्व॑यमातृ॒ण्णा - मे॒षां ॅवै - प॒शु - र्गा॑य॒त्री - कस्त्वा॒ - द्वाद॑श ) \newline

\textbf{korvai with starting padams of1, 11, 21 series of pa~jcAtis :-} \newline
(विष्णु॑मुखा॒ - अप॑चितिमा॒न्॒ - वि वा ए॒ता - वग्ने॒ तव॑ - स्वयमातृ॒ण्णां - ॅवि॑षू॒चीना॑नि - गाय॒त्री - चतु॑ष्षष्टिः) \newline

\textbf{first and last padam of second praSnam of 5th kANDam} \newline
(विष्णु॑मुखा - स्त॒नुवे॑ भुवत् ) \newline 


॥ हरिः॑ ॐ ॥॥ कृष्ण यजुर्वेदीय तैत्तिरीय संहितायां पञ्चमकाण्डे द्वितीयः प्रश्नः समाप्तः ॥
------------------------------------ \newline
\pagebreak
\pagebreak
        


\end{document}
