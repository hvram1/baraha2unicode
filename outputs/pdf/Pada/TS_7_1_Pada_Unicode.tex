\documentclass[17pt]{extarticle}
\usepackage{babel}
\usepackage{fontspec}
\usepackage{polyglossia}
\usepackage{extsizes}



\setmainlanguage{sanskrit}
\setotherlanguages{english} %% or other languages
\setlength{\parindent}{0pt}
\pagestyle{myheadings}
\newfontfamily\devanagarifont[Script=Devanagari]{AdishilaVedic}


\newcommand{\VAR}[1]{}
\newcommand{\BLOCK}[1]{}




\begin{document}
\begin{titlepage}
    \begin{center}
 
\begin{sanskrit}
    { \Large
    ॐ नमः परमात्मने, श्री महागणपतये नमः, श्री गुरुभ्यो नमः
ह॒रिः॒ ॐ 
    }
    \\
    \vspace{2.5cm}
    \mbox{ \Huge
    7.1      सप्तमकाण्डे प्रथमः प्रश्नः- अश्वमेधगतमन्त्राणामभिधानं   }
\end{sanskrit}
\end{center}

\end{titlepage}
\tableofcontents

ॐ नमः परमात्मने, श्री महागणपतये नमः, श्री गुरुभ्यो नमः
ह॒रिः॒ ॐ \newline
7.1      सप्तमकाण्डे प्रथमः प्रश्नः- अश्वमेधगतमन्त्राणामभिधानं \newline

\addcontentsline{toc}{section}{ 7.1      सप्तमकाण्डे प्रथमः प्रश्नः- अश्वमेधगतमन्त्राणामभिधानं}
\markright{ 7.1      सप्तमकाण्डे प्रथमः प्रश्नः- अश्वमेधगतमन्त्राणामभिधानं \hfill https://www.vedavms.in \hfill}
\section*{ 7.1      सप्तमकाण्डे प्रथमः प्रश्नः- अश्वमेधगतमन्त्राणामभिधानं }
                                \textbf{ TS 7.1.1.1} \newline
                  प्र॒जन॑न॒मिति॑ प्र-जन॑नम् । ज्योतिः॑ । अ॒ग्निः । दे॒वता॑नाम् । ज्योतिः॑ । वि॒राडिति॑ वि - राट् । छन्द॑साम् । ज्योतिः॑ । वि॒राडिति॑ वि - राट् । वा॒चः । अ॒ग्नौ । समिति॑ । ति॒ष्ठ॒ते॒ । वि॒राज॒मिति॑ वि - राज᳚म् । अ॒भि । समिति॑ । प॒द्य॒ते॒ । तस्मा᳚त् । तत् । ज्योतिः॑ । उ॒च्य॒ते॒ । द्वौ । स्तोमौ᳚ । प्रा॒त॒स्स॒व॒नमिति॑ प्रातः - स॒व॒नम् । व॒ह॒तः॒ । यथा᳚ । प्रा॒ण इति॑ प्र - अ॒नः । च॒ । अ॒पा॒न इत्य॑प - अ॒नः । च॒ । द्वौ । माद्ध्य॑न्दिनम् । सव॑नम् । यथा᳚ । चक्षुः॑ । च॒ । श्रोत्र᳚म् । च॒ । द्वौ । तृ॒ती॒य॒स॒व॒नमिति॑ तृतीय - स॒व॒नम् । यथा᳚ । वाक् । च॒ । प्र॒ति॒ष्ठेति॑ प्रति - स्था । च॒ । पुरु॑षसम्मित॒ इति॒ पुरु॑ष - स॒म्मि॒तः॒ । वै । ए॒षः । य॒ज्ञ्ः । अस्थू॑रिः । \textbf{  1} \newline
                  \newline
                                \textbf{ TS 7.1.1.2} \newline
                  यम् । काम᳚म् । का॒मय॑ते । तम् । ए॒तेन॑ । अ॒भीति॑ । अ॒श्नु॒ते॒ । सर्व᳚म् । हि । अस्थू॑रिणा । अ॒भ्य॒श्नु॒त इत्य॑भि - अ॒श्नु॒ते । अ॒ग्नि॒ष्टो॒मेनेत्य॑ग्नि - स्तो॒मेन॑ । वै । प्र॒जाप॑ति॒रिति॑ प्र॒जा - प॒तिः॒ । प्र॒जा इति॑ प्र-जाः । अ॒सृ॒ज॒त॒ । ताः । अ॒ग्नि॒ष्टो॒मेनेत्य॑ग्नि-स्तो॒मेन॑ । ए॒व । परीति॑ । अ॒गृ॒ह्णा॒त् । तासा᳚म् । परि॑गृहीताना॒मिति॒ परि॑ - गृ॒ही॒ता॒ना॒म् । अ॒श्व॒त॒रः । अतीति॑ । अ॒प्र॒व॒त॒ । तस्य॑ । अ॒नु॒हायेत्य॑नु - हाय॑ । रेतः॑ । एति॑ । अ॒द॒त्त॒ । तत् । ग॒र्द॒भे । नीति॑ । अ॒मा॒र्ट् । तस्मा᳚त् । ग॒र्द॒भः । द्वि॒रेता॒ इति॑ द्वि - रेताः᳚ । अथो॒ इति॑ । आ॒हुः॒ । वड॑बायाम् । नीति॑ । अ॒मा॒र्ट्॒ । इति॑ । तस्मा᳚त् । वड॑बा । द्वि॒रेता॒ इति॑ द्वि - रेताः᳚ । अथो॒ इति॑ । आ॒हुः॒ । ओष॑धीषु । \textbf{  2} \newline
                  \newline
                                \textbf{ TS 7.1.1.3} \newline
                  नीति॑ । अ॒मा॒र्ट॒ । इति॑ । तस्मा᳚त् । ओष॑धयः । अन॑भ्यक्ता॒ इत्यन॑भि - अ॒क्ताः॒ । रे॒भ॒न्ति॒ । अथो॒ इति॑ । आ॒हुः॒ । प्र॒जास्विति॑ प्र - जासु॑ । नीति॑ । अ॒मा॒र्ट्॒ । इति॑ । तस्मा᳚त् । य॒मौ । जा॒ये॒ते॒ इति॑ । तस्मा᳚त् । अ॒श्व॒त॒रः । न । प्रेति॑ । जा॒य॒ते॒ । आत्त॑रेता॒ इत्यात्त॑ - रे॒ताः॒ । हि । तस्मा᳚त् । ब॒र्॒.हिषि॑ । अन॑वक्लृप्त॒ इत्यन॑व - क्लृ॒प्तः॒ । स॒र्व॒वे॒द॒स इति॑ सर्व-वे॒द॒से । वा॒ । स॒हस्रे᳚ । वा॒ । अव॑क्लृप्त॒ इत्यव॑-क्लृ॒प्तः॒ । अतीति॑ । हि । अप्र॑वत । यः । ए॒वम् । वि॒द्वान् । अ॒ग्नि॒ष्टो॒मेनेत्य॑ग्नि- स्तो॒मेन॑ । यज॑ते । प्रेति॑ । अजा॑ताः । प्र॒जा इति॑ प्र-जाः । ज॒नय॑ति । परीति॑ । प्रजा॑ता॒ इति॒ प्र - जा॒ताः॒ । गृ॒ह्णा॒ति॒ । तस्मा᳚त् । आ॒हुः॒ । ज्ये॒ष्ठ॒य॒ज्ञ् इति॑ ज्येष्ठ - य॒ज्ञ्ः । इति॑ । \textbf{  3} \newline
                  \newline
                                \textbf{ TS 7.1.1.4} \newline
                  प्र॒जाप॑ति॒रिति॑ प्र॒जा - प॒तिः॒ । वाव । ज्येष्ठः॑ । सः । हि । ए॒तेन॑ । अग्रे᳚ । अय॑जत । प्र॒जाप॑ति॒रिति॑ प्र॒जा - प॒तिः॒ । अ॒का॒म॒य॒त॒ । प्रेति॑ । जा॒ये॒य॒ । इति॑ । सः । मु॒ख॒तः । त्रि॒वृत॒मिति॑ त्रि - वृत᳚म् । निरिति॑ । अ॒मि॒मी॒त॒ । तम् । अ॒ग्निः । दे॒वता᳚ । अन्विति॑ । अ॒सृ॒ज्य॒त॒ । गा॒य॒त्री । छन्दः॑ । र॒थ॒न्त॒रमिति॑ रथं - त॒रम् । साम॑ । ब्रा॒ह्म॒णः । म॒नु॒ष्या॑णाम् । अ॒जः । प॒शू॒नाम् । तस्मा᳚त् । ते । मुख्याः᳚ । मु॒ख॒तः । हि । असृ॑ज्यन्त । उर॑सः । बा॒हुभ्या॒मिति॑ बा॒हु - भ्या॒म् । प॒ञ्च॒द॒शमिति॑ पञ्च - द॒शम् । निरिति॑ । अ॒मि॒मी॒त॒ । तम् । इन्द्रः॑ । दे॒वता᳚ । अन्विति॑ । अ॒सृ॒ज्य॒त॒ । त्रि॒ष्टुप् । छन्दः॑ । बृ॒हत् । \textbf{  4} \newline
                  \newline
                                \textbf{ TS 7.1.1.5} \newline
                  साम॑ । रा॒ज॒न्यः॑ । म॒नु॒ष्या॑णाम् । अविः॑ । प॒शू॒नाम् । तस्मा᳚त् । ते । वी॒र्या॑वन्त॒ इति॑ वी॒र्य॑ - व॒न्तः॒ । वी॒र्या᳚त् । हि । असृ॑ज्यन्त । म॒द्ध्य॒तः । स॒प्त॒द॒शमिति॑ सप्त - द॒शम् । निरिति॑ । अ॒मि॒मी॒त॒ । तम् । विश्वे᳚ । दे॒वाः । दे॒वताः᳚ । अन्विति॑ । अ॒सृ॒ज्य॒न्त॒ । जग॑ती । छन्दः॑ । वै॒रू॒पम् । साम॑ । वैश्यः॑ । म॒नु॒ष्या॑णाम् । गावः॑ । प॒शू॒नाम् । तस्मा᳚त् । ते । आ॒द्याः᳚ । अ॒न्न॒धाना॒दित्य॑न्न - धाना᳚त् । हि । असृ॑ज्यन्त । तस्मा᳚त् । भूयाꣳ॑सः । अ॒न्येभ्यः॑ । भूयि॑ष्ठाः । हि । दे॒वताः᳚ । अन्विति॑ । असृ॑ज्यन्त । प॒त्तः । ए॒क॒विꣳ॒॒शमित्ये॑क - विꣳ॒॒शम् । निरिति॑ । अ॒मि॒मी॒त॒ । तम् । अ॒नु॒ष्टुबित्य॑नु - स्तुप् । छन्दः॑ । \textbf{  5} \newline
                  \newline
                                \textbf{ TS 7.1.1.6} \newline
                  अन्विति॑ । अ॒सृ॒ज्य॒त॒ । वै॒रा॒जम् । साम॑ । शू॒द्रः । म॒नु॒ष्या॑णाम् । अश्वः॑ । प॒शू॒नाम् । तस्मा᳚त् । तौ । भू॒त॒स॒ङ्क्रा॒मिणा॒विति॑ भूत - स॒ङ्क्रा॒मिणौ᳚ । अश्वः॑ । च॒ । शू॒द्रः । च॒ । तस्मा᳚त् । शू॒द्रः । य॒ज्ञे । अन॑वक्लृप्त॒ इत्यन॑व - क्लृ॒प्तः॒ । न । हि । दे॒वताः᳚ । अन्विति॑ । असृ॑ज्यत । तस्मा᳚त् । पादौ᳚ । उपेति॑ । जी॒व॒तः॒ । प॒त्तः । हि । असृ॑ज्येताम् । प्रा॒णा इति॑ प्र - अ॒नाः । वै । त्रि॒वृदिति॑ त्रि - वृत् । अ॒द्‌र्ध॒मा॒सा इत्य॑द्‌र्ध - मा॒साः । प॒ञ्च॒द॒श इति॑ पञ्च - द॒शः । प्र॒जाप॑ति॒रिति॑ प्र॒जा - प॒तिः॒ । स॒प्त॒द॒श इति॑ सप्त - द॒शः । त्रयः॑ । इ॒मे । लो॒काः । अ॒सौ । आ॒दि॒त्यः । ए॒क॒विꣳ॒॒श इत्ये॑क - विꣳ॒॒शः । ए॒तस्मिन्न्॑ । वै । ए॒ते । श्रि॒ताः । ए॒तस्मिन्न्॑ । प्रति॑ष्ठिता॒ इति॒ प्रति॑ - स्थि॒ताः॒ ( ) । यः । ए॒वम् । वेद॑ । ए॒तस्मिन्न्॑ । ए॒व । श्र॒य॒ते॒ । ए॒तस्मिन्न्॑ । प्रतीति॑ । ति॒ष्ठ॒ति॒ ॥ \textbf{  6} \newline
                  \newline
                      (अस्थू॑रि॒ - रोष॑धीषु - ज्येष्ठय॒ज्ञ् इति॑ - बृ॒ह - द॑नु॒ष्टुप् छन्दः॒ - प्रति॑ष्ठिता॒ - नव॑ च)  \textbf{(A1)} \newline \newline
                                \textbf{ TS 7.1.2.1} \newline
                  प्रा॒त॒स्स॒व॒न इति॑ प्रातः - स॒व॒ने । वै । गा॒य॒त्रेण॑ । छन्द॑सा । त्रि॒वृत॒ इति॑ त्रि - वृते᳚ । स्तोमा॑य । ज्योतिः॑ । दध॑त् । ए॒ति॒ । त्रि॒वृतेति॑ त्रि - वृता᳚ । ब्र॒ह्म॒व॒र्च॒सेनेति॑ ब्रह्म - व॒र्च॒सेन॑ । प॒ञ्च॒द॒शायेति॑ पञ्च - द॒शाय॑ । ज्योतिः॑ । दध॑त् । ए॒ति॒ । प॒ञ्च॒द॒शेनेति॑ पञ्च-द॒शेन॑ । ओज॑सा । वी॒र्ये॑ण । स॒प्त॒द॒शायेति॑ सप्त - द॒शाय॑ । ज्योतिः॑ । दध॑त् । ए॒ति॒ । स॒प्त॒द॒शेनेति॑ सप्त - द॒शेन॑ । प्रा॒जा॒प॒त्येनेति॑ प्राजा - प॒त्येन॑ । प्र॒जन॑ने॒नेति॑ प्र - जन॑नेन । ए॒क॒विꣳ॒॒शायेत्ये॑क - विꣳ॒॒शाय॑ । ज्योतिः॑ । दध॑त् । ए॒ति॒ । स्तोमः॑ । ए॒व । तत् । स्तोमा॑य । ज्योतिः॑ । दध॑त् । ए॒ति॒ । अथो॒ इति॑ । स्तोमे᳚ । ए॒व । स्तोम᳚म् । अ॒भि । प्रेति॑ । न॒य॒ति॒ । याव॑न्तः । वै । स्तोमाः᳚ । ताव॑न्तः । कामाः᳚ । ताव॑न्तः । लो॒काः ( ) । ताव॑न्ति । ज्योतीꣳ॑षि । ए॒ताव॑तः । ए॒व । स्तोमान्॑ । ए॒ताव॑तः । कामान्॑ । ए॒ताव॑तः । लो॒कान् । ए॒ताव॑न्ति । ज्योतीꣳ॑षि । अवेति॑ । रु॒न्धे॒ ॥ \textbf{  7} \newline
                  \newline
                      (ताव॑न्तो लो॒का - स्त्रयो॑दश च)  \textbf{(A2)} \newline \newline
                                \textbf{ TS 7.1.3.1} \newline
                  ब्र॒ह्म॒वा॒दिन॒ इति॑ ब्रह्म - वा॒दिनः॑ । व॒द॒न्ति॒ । सः । तु । वै । य॒जे॒त॒ । यः । अ॒ग्नि॒ष्टो॒मेनेत्य॑ग्नि - स्तो॒मेन॑ । यज॑मानः । अथ॑ । सर्व॑स्तोमे॒नेति॒ सर्व॑ - स्तो॒मे॒न॒ । यजे॑त । इति॑ । यस्य॑ । त्रि॒वृत॒मिति॑ त्रि - वृत᳚म् । अ॒न्त॒र्यन्तीत्य॑न्तः-यन्ति॑ । प्रा॒णानिति॑ प्र - अ॒नान् । तस्य॑ । अ॒न्तः । य॒न्ति॒ । प्रा॒णेष्विति॑ प्र - अ॒नेषु॑ । मे॒ । अपीति॑ । अ॒स॒त् । इति॑ । खलु॑ । वै । य॒ज्ञेन॑ । यज॑मानः । य॒ज॒ते॒ । यस्य॑ । प॒ञ्च॒द॒शमिति॑ पञ्च - द॒शम् । अ॒न्त॒र्यन्तीत्य॑न्तः - यन्ति॑ । वी॒र्य᳚म् । तस्य॑ । अ॒न्तः । य॒न्ति॒ । वी॒र्ये᳚ । मे॒ । अपीति॑ । अ॒स॒त् । इति॑ । खलु॑ । वै । य॒ज्ञेन॑ । यज॑मानः । य॒ज॒ते॒ । यस्य॑ । स॒प्त॒द॒शमिति॑ सप्त - द॒शम् । अ॒न्त॒र्यन्तीत्य॑न्तः - यन्ति॑ । \textbf{  8} \newline
                  \newline
                                \textbf{ TS 7.1.3.2} \newline
                  प्र॒जामिति॑ प्र - जाम् । तस्य॑ । अ॒न्तः । य॒न्ति॒ । प्र॒जाया॒मिति॑ प्र - जाया᳚म् । मे॒ । अपीति॑ । अ॒स॒त् । इति॑ । खलु॑ । वै । य॒ज्ञेन॑ । यज॑मानः । य॒ज॒ते॒ । यस्य॑ । ए॒क॒विꣳ॒॒शमित्ये॑क - विꣳ॒॒शम् । अ॒न्त॒र्यन्तित्य॑न्तः - यन्ति॑ । प्र॒ति॒ष्ठामिति॑ प्रति - स्थाम् । तस्य॑ । अ॒न्तः । य॒न्ति॒ । प्र॒ति॒ष्ठाया॒मिति॑ प्रति - स्थाया᳚म् । मे॒ । अपीति॑ । अ॒स॒त् । इति॑ । खलु॑ । वै । य॒ज्ञेन॑ । यज॑मानः । य॒ज॒ते॒ । यस्य॑ । त्रि॒ण॒वमिति॑ त्रि - न॒वम् । अ॒न्त॒र्यन्तीत्य॑न्तः - यन्ति॑ । ऋ॒तून् । च॒ । तस्य॑ । न॒क्ष॒त्रिया᳚म् । च॒ । वि॒राज॒मिति॑ वि-राज᳚म् । अ॒न्तः । य॒न्ति॒ । ऋ॒तुषु॑ । मे॒ । अपीति॑ । अ॒स॒त् । न॒क्ष॒त्रिया॑याम् । च॒ । वि॒राजीति॑ वि - राजि॑ । इति॑ । \textbf{  9} \newline
                  \newline
                                \textbf{ TS 7.1.3.3} \newline
                  खलु॑ । वै । य॒ज्ञेन॑ । यज॑मानः । य॒ज॒ते॒ । यस्य॑ । त्र॒य॒स्त्रिꣳ॒॒शमिति॑ त्रयः - त्रिꣳ॒॒शम् । अ॒न्त॒र्यन्तित्य॑न्तः - यन्ति॑ । दे॒वताः᳚ । तस्य॑ । अ॒न्तः । य॒न्ति॒ । दे॒वता॑सु । मे॒ । अपीति॑ । अ॒स॒त् । इति॑ । खलु॑ । वै । य॒ज्ञेन॑ । यज॑मानः । य॒ज॒ते॒ । यः । वै । स्तोमा॑नाम् । अ॒व॒मम् । प॒र॒मता᳚म् । गच्छ॑न्तम् । वेद॑ । प॒र॒मता᳚म् । ए॒व । ग॒च्छ॒ति॒ । त्रि॒वृदिति॑ त्रि - वृत् । वै । स्तोमा॑नाम् । अ॒व॒मः । त्रि॒वृदिति॑ त्रि - वृत् । प॒र॒मः । यः । ए॒वम् । वेद॑ । प॒र॒मता᳚म् । ए॒व । ग॒च्छ॒ति॒ ॥ \textbf{  10} \newline
                  \newline
                      (स॒प्त॒द॒शम॑न्त॒र्यन्ति॑ - वि॒राजीति॒ - चतु॑श्चत्वारिꣳशच्च)  \textbf{(A3)} \newline \newline
                                \textbf{ TS 7.1.4.1} \newline
                  अङ्गि॑रसः । वै । स॒त्रम् । आ॒स॒त॒ । ते । सु॒व॒र्गमिति॑ सुवः - गम् । लो॒कम् । आ॒य॒न्न् । तेषा᳚म् । ह॒विष्मान्॑ । च॒ । ह॒वि॒ष्कृदिति॑ हविः - कृत् । च॒ । अ॒ही॒ये॒ता॒म् । तौ । अ॒का॒म॒ये॒ता॒म् । सु॒व॒र्गमिति॑ सुवः - गम् । लो॒कम् । इ॒या॒व॒ । इति॑ । तौ । ए॒तम् । द्वि॒रा॒त्रमिति॑ द्वि - रा॒त्रम् । अ॒प॒श्य॒ता॒म् । तम् । एति॑ । अ॒ह॒र॒ता॒म् । तेन॑ । अ॒य॒जे॒ता॒म् । ततः॑ । वै । तौ । सु॒व॒र्गमिति॑ सुवः - गम् । लो॒कम् । ऐ॒ता॒म् । यः । ए॒वम् । वि॒द्वान् । द्वि॒रा॒त्रेणेति॑ द्वि - रा॒त्रेण॑ । यज॑ते । सु॒व॒र्गमिति॑ सुवः - गम् । ए॒व । लो॒कम् । ए॒ति॒ । तौ । ऐता᳚म् । पूर्वे॑ण । अह्ना᳚ । अग॑च्छताम् । उत्त॑रे॒णेत्युत् - त॒रे॒ण॒ । \textbf{  11} \newline
                  \newline
                                \textbf{ TS 7.1.4.2} \newline
                  अ॒भि॒प्ल॒व इत्य॑भि - प्ल॒वः । पूर्व᳚म् । अहः॑ । भ॒व॒ति॒ । गतिः॑ । उत्त॑र॒मित्युत् - त॒र॒म् । ज्योति॑ष्टोम॒ इति॒ ज्योति॑ - स्तो॒मः॒ । अ॒ग्नि॒ष्टो॒म इत्य॑ग्नि-स्तो॒मः । पूर्व᳚म् । अहः॑ । भ॒व॒ति॒ । तेजः॑ । तेन॑ । अवेति॑ । रु॒न्धे॒ । सर्व॑स्तोम॒ इति॒ सर्व॑-स्तो॒मः॒ । अ॒ति॒रा॒त्र इत्य॑ति-रा॒त्रः । उत्त॑र॒मित्युत् - त॒र॒म् । सर्व॑स्य । आप्त्यै᳚ । सर्व॑स्य । अव॑रुद्ध्या॒ इत्यव॑ - रु॒द्ध्यै॒ । गा॒य॒त्रम् । पूर्वे᳚ । अहन्न्॑ । साम॑ । भ॒व॒ति॒ । तेजः॑ । वै । गा॒य॒त्री । गा॒य॒त्री । ब्र॒ह्म॒व॒र्च॒समिति॑ ब्रह्म - व॒र्च॒सम् । तेजः॑ । ए॒व । ब्र॒ह्म॒व॒र्च॒समिति॑ ब्रह्म - व॒र्च॒सम् । आ॒त्मन्न् । ध॒त्ते॒ । त्रैष्टु॑भम् । उत्त॑र॒ इत्युत् - त॒रे॒ । ओजः॑ । वै । वी॒र्य᳚म् । त्रि॒ष्टुक् । ओजः॑ । ए॒व । वी॒र्य᳚म् । आ॒त्मन्न् । ध॒त्ते॒ । र॒थ॒न्त॒रमिति॑ रथं - त॒रम् । पूर्वे᳚ । \textbf{  12} \newline
                  \newline
                                \textbf{ TS 7.1.4.3} \newline
                  अहन्न्॑ । साम॑ । भ॒व॒ति॒ । इ॒यम् । वै । र॒थ॒न्त॒रमिति॑ रथं - त॒रम् । अ॒स्याम् । ए॒व । प्रतीति॑ । ति॒ष्ठ॒ति॒ । बृ॒हत् । उत्त॑र॒ इत्युत् - त॒रे॒ । अ॒सौ । वै । बृ॒हत् । अ॒मुष्या᳚म् । ए॒व । प्रतीति॑ । ति॒ष्ठ॒ति॒ । तत् । आ॒हुः॒ । क्व॑ । जग॑ती । च॒ । अ॒नु॒ष्टुबित्य॑नु - स्तुप् । च॒ । इति॑ । वै॒खा॒न॒सम् । पूर्वे᳚ । अहन्न्॑ । साम॑ । भ॒व॒ति॒ । तेन॑ । जग॑त्यै । न । ए॒ति॒ । षो॒ड॒शि । उत्त॑र॒ इत्युत् - त॒रे॒ । तेन॑ । अ॒नु॒ष्टुभ॒ इत्य॑नु - स्तुभः॑ । अथ॑ । आ॒हुः॒ । यत् । स॒मा॒ने । अ॒द्‌र्ध॒मा॒स इत्य॑द्‌र्ध - मा॒से । स्याता᳚म् । अ॒न्य॒त॒रस्य॑ । अह्नः॑ । वी॒र्य᳚म् । अन्विति॑ ( ) । प॒द्ये॒त॒ । इति॑ । अ॒मा॒वा॒स्या॑या॒मित्य॑मा - वा॒स्या॑याम् । पूर्व᳚म् । अहः॑ । भ॒व॒ति॒ । उत्त॑रस्मि॒न्नित्युत् - त॒र॒स्मि॒न्न् । उत्त॑र॒मित्युत् - त॒र॒म् । नाना᳚ । ए॒व । अ॒द्‌र्ध॒मा॒सयो॒रित्य॑द्‌र्ध - मा॒सयोः᳚ । भ॒व॒तः॒ । नाना॑वीर्ये॒ इति॒ नाना᳚ - वी॒र्ये॒ । भ॒व॒तः॒ । ह॒विष्म॑न्निधन॒मिति॑ ह॒विष्म॑त् - नि॒ध॒न॒म् । पूर्व᳚म् । अहः॑ । भ॒व॒ति॒ । ह॒वि॒ष्कृन्नि॑धन॒मिति॑ हवि॒ष्कृत् - नि॒ध॒न॒म् । उत्त॑र॒मित्युत् - त॒र॒म् । प्रति॑ष्ठित्या॒ इति॒ प्रति॑ - स्थि॒त्यै॒ ॥ \textbf{  13} \newline
                  \newline
                      (उत्त॑रेण - रथन्त॒रं पूर्वे - ऽन्वे - क॑विꣳशतिश्च)  \textbf{(A4)} \newline \newline
                                \textbf{ TS 7.1.5.1} \newline
                  आपः॑ । वै । इ॒दम् । अग्रे᳚ । स॒लि॒लम् । आ॒सी॒त् । तस्मिन्न्॑ । प्र॒जाप॑ति॒रिति॑ प्र॒जा - प॒तिः॒ । वा॒युः । भू॒त्वा । अ॒च॒र॒त् । सः । इ॒माम् । अ॒प॒श्य॒त् । ताम् । व॒रा॒हः । भू॒त्वा । एति॑ । अ॒ह॒र॒त् । ताम् । वि॒श्वक॒र्मेति॑ वि॒श्व-क॒र्मा॒ । भू॒त्वा । वीति॑ । अ॒मा॒र्ट्॒ । सा । अ॒प्र॒थ॒त॒ । सा । पृ॒थि॒वी । अ॒भ॒व॒त् । तत् । पृ॒थि॒व्यै । पृ॒थि॒वि॒त्वमिति॑ पृ॒थिवि - त्वम् । तस्या᳚म् । अ॒श्रा॒म्य॒त् । प्र॒जाप॑ति॒रिति॑ प्र॒जा-प॒तिः॒ । सः । दे॒वान् । अ॒सृ॒ज॒त॒ । वसून्॑ । रु॒द्रान् । आ॒दि॒त्यान् । ते । दे॒वाः । प्र॒जाप॑ति॒मिति॑ प्र॒जा - प॒ति॒म् । अ॒ब्रु॒व॒न्न् । प्रेति॑ । जा॒या॒म॒है॒ । इति॑ । सः । अ॒ब्र॒वी॒त् । \textbf{  14} \newline
                  \newline
                                \textbf{ TS 7.1.5.2} \newline
                  यथा᳚ । अ॒हम् । यु॒ष्मान् । तप॑सा । असृ॑क्षि । ए॒वम् । तप॑सि । प्र॒जन॑न॒मिति॑ प्र - जन॑नम् । इ॒च्छ॒द्ध्व॒म् । इति॑ । तेभ्यः॑ । अ॒ग्निम् । आ॒यत॑न॒मित्या᳚ - यत॑नम् । प्रेति॑ । अ॒य॒च्छ॒त् । ए॒तेन॑ । आ॒यत॑ने॒नेत्या᳚ - यत॑नेन । श्रा॒म्य॒त॒ । इति॑ । ते । अ॒ग्निना᳚ । आ॒यत॑ने॒नेत्या᳚ - यत॑नेन । अ॒श्रा॒म्य॒न्न् । ते । सं॒ॅव॒थ्स॒र इति॑ सं - व॒थ्स॒रे । एका᳚म् । गाम् । अ॒सृ॒ज॒न्त॒ । ताम् । वसु॑भ्य॒ इति॒ वसु॑ - भ्यः॒ । रु॒द्रेभ्यः॑ । आ॒दि॒त्येभ्यः॑ । प्रेति॑ । अ॒य॒च्छ॒न्न् । ए॒ताम् । र॒क्ष॒द्ध्व॒म् । इति॑ । ताम् । वस॑वः । रु॒द्राः । आ॒दि॒त्याः । अ॒र॒क्ष॒न्त॒ । सा । वसु॑भ्य॒ इति॒ वसु॑ - भ्यः॒ । रु॒द्रेभ्यः॑ । आ॒दि॒त्येभ्यः॑ । प्रेति॑ । अ॒जा॒य॒त॒ । त्रीणि॑ । च॒ । \textbf{  15} \newline
                  \newline
                                \textbf{ TS 7.1.5.3} \newline
                  श॒तानि॑ । त्रय॑स्त्रिꣳशत॒मिति॒ त्रयः॑ - त्रिꣳ॒॒श॒त॒म् । च॒ । अथ॑ । सा । ए॒व । स॒ह॒स्र॒त॒मीति॑ सहस्र - त॒मी । अ॒भ॒व॒त् । ते । दे॒वाः । प्र॒जाप॑ति॒मिति॑ प्र॒जा - प॒ति॒म् । अ॒ब्रु॒व॒न्न् । स॒हस्रे॑ण । नः॒ । या॒ज॒य॒ । इति॑ । सः । अ॒ग्नि॒ष्टो॒मेनेत्य॑ग्नि-स्तो॒मेन॑ । वसून्॑ । अ॒या॒ज॒य॒त् । ते । इ॒मम् । लो॒कम् । अ॒ज॒य॒न्न् । तत् । च॒ । अ॒द॒दुः॒ । सः । उ॒क्थ्ये॑न । रु॒द्रान् । अ॒या॒ज॒य॒त् । ते । अ॒न्तरि॑क्षम् । अ॒ज॒य॒न्न् । तत् । च॒ । अ॒द॒दुः॒ । सः । अ॒ति॒रा॒त्रेणेत्य॑ति - रा॒त्रेण॑ । आ॒दि॒त्यान् । अ॒या॒ज॒य॒त् । ते । अ॒मुम् । लो॒कम् । अ॒ज॒य॒न्न् । तत् । च॒ । अ॒द॒दुः॒ । तत् । अ॒न्तरि॑क्षम् । \textbf{  16} \newline
                  \newline
                                \textbf{ TS 7.1.5.4} \newline
                  व्यवै᳚र्य॒तेति॑ वि - अवै᳚र्यत । तस्मा᳚त् । रु॒द्राः । घातु॑काः । अ॒ना॒य॒त॒ना इत्य॑ना - य॒त॒नाः । हि । तस्मा᳚त् । आ॒हुः॒ । शि॒थि॒लम् । वै । म॒द्ध्य॒मम् । अहः॑ । त्रि॒रा॒त्रस्येति॑ त्रि - रा॒त्रस्य॑ । वीति॑ । हि । तत् । अ॒वैर्य॒तेत्य॑व - ऐर्य॑त । इति॑ । त्रैष्टु॑भम् । म॒द्ध्य॒मस्य॑ । अह्नः॑ । आज्य᳚म् । भ॒व॒ति॒ । सं॒ॅयाना॒निति॑ सम् - याना॑नि । सू॒क्तानीति॑ सु - उ॒क्तानि॑ । शꣳ॒॒स॒ति॒ । षो॒ड॒शिन᳚म् । शꣳ॒॒स॒ति॒ । अह्नः॑ । धृत्यै᳚ । अशि॑थिलम्भावा॒येत्यशि॑थिलं - भा॒वा॒य॒ । तस्मा᳚त् । त्रि॒रा॒त्रस्येति॑ त्रि - रा॒त्रस्य॑ । अ॒ग्नि॒ष्टो॒म इत्य॑ग्नि - स्तो॒मः । ए॒व । प्र॒थ॒मम् । अहः॑ । स्या॒त् । अथ॑ । उ॒क्थ्यः॑ । अथ॑ । अ॒ति॒रा॒त्र इत्य॑ति - रा॒त्रः । ए॒षाम् । लो॒काना᳚म् । विधृ॑त्या॒ इति॒ वि - धृ॒त्यै॒ । त्रीणि॑त्री॒णीति॒ त्रीणि॑ - त्री॒णि॒ । श॒तानि॑ । अ॒नू॒ची॒ना॒हमित्य॑नूचीन - अ॒हम् । अव्य॑वच्छिन्ना॒नीत्यवि॑ - अ॒व॒च्छि॒न्ना॒नि॒ । द॒दा॒ति॒ । \textbf{  17} \newline
                  \newline
                                \textbf{ TS 7.1.5.5} \newline
                  ए॒षाम् । लो॒काना᳚म् । अन्विति॑ । सन्त॑त्या॒ इति॒ सं - त॒त्यै॒ । द॒शत᳚म् । न । वीति॑ । छि॒न्द्या॒त् । वि॒राज॒मिति॑ वि - राज᳚म् । न । इत् । वि॒च्छि॒नदा॒नीति॑ वि - छि॒नदा॑नि । इति॑ । अथ॑ । या । स॒ह॒स्र॒त॒मीति॑ सहस्र - त॒मी । आसी᳚त् । तस्या᳚म् । इन्द्रः॑ । च॒ । विष्णुः॑ । च॒ । व्याय॑च्छेता॒मिति॑ वि - आय॑च्छेताम् । सः । इन्द्रः॑ । अ॒म॒न्य॒त॒ । अ॒नया᳚ । वै । इ॒दम् । विष्णुः॑ । स॒हस्र᳚म् । व॒र्क्ष्य॒ते॒ । इति॑ । तस्या᳚म् । अ॒क॒ल्पे॒ता॒म् । द्विभा॑ग॒ इति॒ द्वि - भा॒गे॒ । इन्द्रः॑ । तृती॑ये । विष्णुः॑ । तत् । वै । ए॒षा । अ॒भ्यनू᳚च्यत॒ इत्य॑भि-अनू᳚च्यते । उ॒भा । जि॒ग्य॒थुः॒ । इति॑ । ताम् । वै । ए॒ताम् । अ॒च्छा॒वा॒कः । \textbf{  18} \newline
                  \newline
                                \textbf{ TS 7.1.5.6} \newline
                  ए॒व । शꣳ॒॒स॒ति॒ । अथ॑ । या । स॒ह॒स्र॒त॒मीति॑ सहस्र - त॒मी । सा । होत्रे᳚ । देया᳚ । इति॑ । होता॑रम् । वै । अ॒भ्यति॑रिच्यत॒ इत्य॑भि-अति॑रिच्यते । यत् । अ॒ति॒रिच्य॑त॒ इत्य॑ति - रिच्य॑ते । होता᳚ । अना᳚प्तस्य । आ॒प॒यि॒ता । अथ॑ । आ॒हुः॒ । उ॒न्ने॒त्र इत्यु॑त् - ने॒त्रे । देया᳚ । इति॑ । अति॑रि॒क्तेत्यति॑ - रि॒क्ता॒ । वै । ए॒षा । स॒हस्र॑स्य । अति॑रिक्त॒ इत्यति॑ - रि॒क्तः॒ । उ॒न्ने॒तेत्यु॑त् - ने॒ता । ऋ॒त्विजा᳚म् । अथ॑ । आ॒हुः॒ । सर्वे᳚भ्यः । स॒द॒स्ये᳚भ्यः । देया᳚ । इति॑ । अथ॑ । आ॒हुः॒ । उ॒दा॒कृत्येत्यु॑त् - आ॒कृत्या᳚ । सा । वश᳚म् । च॒रे॒त् । इति॑ । अथ॑ । आ॒हुः॒ । ब्र॒ह्मणे᳚ । च॒ । अ॒ग्नीध॒ इत्य॑ग्नि - इधे᳚ । च॒ । देया᳚ । इति॑ । \textbf{  19} \newline
                  \newline
                                \textbf{ TS 7.1.5.7} \newline
                  द्विभा॑ग॒मिति॒ द्वि - भा॒ग॒म् । ब्र॒ह्मणे᳚ । तृती॑यम् । अ॒ग्नीध॒ इत्य॑ग्नि- इधे᳚ । ऐ॒न्द्रः । वै । ब्र॒ह्मा । वै॒ष्ण॒वः । अ॒ग्नीदित्य॑ग्नि-इत् । यथा᳚ । ए॒व । तौ । अक॑ल्पेताम् । इति॑ । अथ॑ । आ॒हुः॒ । या । क॒ल्या॒णी । ब॒हु॒रू॒पेति॑ बहु - रू॒पा । सा । देया᳚ । इति॑ । अथ॑ । आ॒हुः॒ । या । द्वि॒रू॒पेति॑ द्वि - रू॒पा । उ॒भ॒यत॑ए॒नीत्य॑भ॒यतः॑ - ए॒नी॒ । सा । देया᳚ । इति॑ । स॒हस्र॑स्य । परि॑गृहीत्या॒ इति॒ परि॑-गृ॒ही॒त्यै॒ । तत् । वै । ए॒तत् । स॒हस्र॑स्य । अय॑नम् । स॒हस्र᳚म् । स्तो॒त्रीयाः᳚ । स॒हस्र᳚म् । दक्षि॑णाः । स॒हस्र॑संमित॒ इति॑ स॒हस्र॑ - स॒मिं॒तः॒ । सु॒व॒र्ग इति॑ सुवः - गः । लो॒कः । सु॒व॒र्गस्येति॑ सुवः - गस्य॑ । लो॒कस्य॑ । अ॒भिजि॑त्या॒ इत्य॒भि - जि॒त्यै॒ ॥ \textbf{  20} \newline
                  \newline
                      (अ॒ब्र॒वी॒ - च्च॒ - तद॒न्तरि॑क्षं - ददात्य - च्छावा॒क - श्च॒ देयेति॑ - स॒प्तच॑त्वारिꣳशच्च)  \textbf{(A5)} \newline \newline
                                \textbf{ TS 7.1.6.1} \newline
                  सोमः॑ । वै । स॒हस्र᳚म् । अ॒वि॒न्द॒त् । तम् । इन्द्रः॑ । अन्विति॑ । अ॒न्वि॒न्द॒त् । तौ । य॒मः । न्याग॑च्छ॒दिति॑ नि - आग॑च्छत् । तौ । अ॒ब्र॒वी॒त् । अस्तु॑ । मे॒ । अत्र॑ । अपीति॑ । इति॑ । अस्तु॑ । ही(3) । इति॑ । अ॒ब्रू॒ता॒म् । सः । य॒मः । एक॑स्याम् । वी॒र्य᳚म् । परीति॑ । अ॒प॒श्य॒त् । इ॒यम् । वै । अ॒स्य । स॒हस्र॑स्य । वी॒र्य᳚म् । बि॒भ॒र्ति॒ । इति॑ । तौ । अ॒ब्र॒वी॒त् । इ॒यम् । मम॑ । अस्तु॑ । ए॒तत् । यु॒वयोः᳚ । इति॑ । तौ । अ॒ब्रू॒ता॒म् । सर्वे᳚ । वै । ए॒तत् । ए॒तस्या᳚म् । वी॒र्य᳚म् । \textbf{  21} \newline
                  \newline
                                \textbf{ TS 7.1.6.2} \newline
                  परीति॑ । प॒श्या॒मः॒ । अꣳश᳚म् । एति॑ । ह॒रा॒म॒है॒ । इति॑ । तस्या᳚म् । अꣳश᳚म् ।  एति॑ । अ॒ह॒र॒न्त॒ । ताम् । अ॒फ्स्वित्य॑प् - सु । प्रेति॑ । अ॒वे॒श॒य॒न्न् । सोमा॑य । उ॒देहीत्यु॑त् - एहि॑ । इति॑ । सा । रोहि॑णी । पि॒ङ्ग॒ला । एक॑हाय॒नीत्येक॑ - हा॒य॒नी॒ । रू॒पम् । कृ॒त्वा । त्रय॑स्त्रिꣳश॒तेति॒ त्रयः॑ - त्रिꣳ॒॒श॒ता॒ । च॒ । त्रि॒भिरिति॑ त्रि - भिः । च॒ । श॒तैः । स॒ह । उ॒दैदित्यु॑त् - ऐत् । तस्मा᳚त् । रोहि॑ण्या । पि॒ङ्ग॒लया᳚ । एक॑हाय॒न्येत्येक॑ - हा॒य॒न्या॒ । सोम᳚म् । क्री॒णी॒या॒त् । यः । ए॒वम् । वि॒द्वान् । रोहि॑ण्या । पि॒ङ्ग॒लया᳚ । एक॑हाय॒न्येत्येक॑ - हा॒य॒न्या॒ । सोम᳚म् । क्री॒णाति॑ । त्रय॑स्त्रिꣳश॒तेति॒ त्रयः॑ - त्रिꣳ॒॒श॒ता॒ । च॒ । ए॒व । अ॒स्य॒ । त्रि॒भिरिति॑ त्रि - भिः । च॒ । \textbf{  22 } \newline
                  \newline
                                \textbf{ TS 7.1.6.3} \newline
                  श॒तैः । सोमः॑ । क्री॒तः । भ॒व॒ति॒ । सुक्री॑ते॒नेति॒ सु-क्री॒ते॒न॒ । य॒ज॒ते॒ । ताम् । अ॒फ्स्वित्य॑प् - सु । प्रेति॑ । अ॒वे॒श॒य॒न्न् । इन्द्रा॑य । उ॒देहीत्यु॑त् - एहि॑ । इति॑ । सा । रोहि॑णी । ल॒क्ष्म॒णा । प॒ष्ठौ॒ही । वार्त्र॒घ्नीति॒ वार्त्र॑ - घ्नी॒ । रू॒पम् । कृ॒त्वा । त्रय॑स्त्रिꣳश॒तेति॒ त्रयः॑ - त्रिꣳ॒॒श॒ता॒ । च॒ । त्रि॒भिरिति॑ त्रि - भिः । च॒ । श॒तैः । स॒ह । उ॒दैदित्यु॑त् - ऐत् । तस्मा᳚त् । रोहि॑णीम् । ल॒क्ष्म॒णाम् । प॒ष्ठौ॒हीम् । वार्त्र॑घ्नी॒मिति॒ वार्त्र॑ - घ्नी॒म् । द॒द्या॒त् । यः । ए॒वम् । वि॒द्वान् । रोहि॑णीम् । ल॒क्ष्म॒णाम् । प॒ष्ठौ॒हीम् । वार्त्र॑घ्नी॒मिति॒ वार्त्र॑ - घ्नी॒म् । ददा॑ति । त्रय॑स्त्रिꣳश॒दिति॒ त्रयः॑ - त्रिꣳ॒॒श॒त् । च॒ । ए॒व॒ । अ॒स्य॒ । त्रीणि॑ । च॒ । श॒तानि॑ । सा । द॒त्ता । \textbf{  23} \newline
                  \newline
                                \textbf{ TS 7.1.6.4} \newline
                  भ॒व॒ति॒ । ताम् । अ॒फ्स्वित्य॑प् - सु । प्रेति॑ । अ॒वे॒श॒य॒न्न् । य॒माय॑ । उ॒देहीत्यु॑त् - एहि॑ । इति॑ । सा । जर॑ती । मू॒र्खा । त॒ज्ज॒घ॒न्येति॑ तत् - ज॒घ॒न्या । रू॒पम् । कृ॒त्वा । त्रय॑स्त्रिꣳश॒तेति॒ त्रयः॑-त्रिꣳ॒॒श॒ता॒ । च॒ । त्रि॒भिरिति॑ त्रि - भिः । च॒ । श॒तैः । स॒ह । उ॒दैदित्यु॑त् - ऐत् । तस्मा᳚त् । जर॑तीम् । मू॒र्खाम् । त॒ज्ज॒घ॒न्यामिति॑ तत् - ज॒घ॒न्याम् । अ॒नु॒स्तर॑णी॒मित्य॑नु - स्तर॑णीम् । कु॒र्वी॒त॒ । यः । ए॒वम् । वि॒द्वान् । जर॑तीम् । मू॒र्खाम् । त॒ज्ज॒घ॒न्यामिति॑ तत् - ज॒घ॒न्याम् । अ॒नु॒स्तर॑णी॒मित्यु॑नु - स्तर॑णीम् । कु॒रु॒ते । त्रय॑स्त्रिꣳश॒दिति॒ त्रयः॑ - त्रिꣳ॒॒श॒त् । च॒ । ए॒व । अ॒स्य॒ । त्रीणि॑ । च॒ । श॒तानि॑ । सा । अ॒मुष्मिन्न्॑ । लो॒के । भ॒व॒ति॒ । वाक् । ए॒व । स॒ह॒स्र॒त॒मीति॑ सहस्र - त॒मी । तस्मा᳚त् । \textbf{  24} \newline
                  \newline
                                \textbf{ TS 7.1.6.5} \newline
                  वरः॑ । देयः॑ । सा । हि । वरः॑ । स॒हस्र᳚म् । अ॒स्य॒ । सा । द॒त्ता । भ॒व॒ति॒ । तस्मा᳚त् । वरः॑ । न । प्र॒ति॒गृह्य॒ इति॑ प्रति - गृह्यः॑ । सा । हि । वरः॑ । स॒हस्र᳚म् । अ॒स्य॒ । प्रति॑गृहीत॒मिति॒ प्रति॑ - गृ॒ही॒त॒म् । भ॒व॒ति॒ । इ॒यम् । वरः॑ । इति॑ । ब्रू॒या॒त् । अथ॑ । अ॒न्याम् । ब्रू॒या॒त् । इ॒यम् । मम॑ । इति॑ । तथा᳚ । अ॒स्य॒ । तत् । स॒हस्र᳚म् । अप्र॑तिगृहीत॒मित्यप्र॑ति - गृ॒ही॒त॒म् । भ॒व॒ति॒ । उ॒भ॒य॒त॒ए॒नीत्यु॑भयतः - ए॒नी । स्या॒त् । तत् । आ॒हुः॒ । अ॒न्य॒त॒ए॒नीत्य॑न्यतः - ए॒नी । स्या॒त् । स॒हस्र᳚म् । प॒रस्ता᳚त् । एत᳚म् । इति॑ । या । ए॒व । वरः॑ । \textbf{  25} \newline
                  \newline
                                \textbf{ TS 7.1.6.6} \newline
                  क॒ल्या॒णी । रू॒पस॑मृ॒द्धेति॑ रू॒प - स॒मृ॒द्धा॒ । सा । स्या॒त् । सा । हि । वरः॑ । समृ॑द्ध्या॒ इति॒ सं - ऋ॒द्ध्यै॒ । ताम् । उत्त॑रे॒णेत्युत् - त॒रे॒ण॒ । आग्नी᳚द्ध्र॒मित्याग्नि॑ - इ॒द्ध्र॒म् । प॒र्या॒णीयेति॑ परि - आ॒नीय॑ । आ॒ह॒व॒नीय॒स्येत्या᳚ - ह॒व॒नीय॑स्य । अन्ते᳚ । द्रो॒ण॒क॒ल॒शमिति॑ द्रोण - क॒ल॒शम् । अवेति॑ । घ्रा॒प॒ये॒त् । एति॑ । जि॒घ्र॒ । क॒लश᳚म् । म॒हि॒ । उ॒रुधा॒रेत्यु॒रु - धा॒रा॒ । पय॑स्वती । एति॑ । त्वा॒ । वि॒श॒न्तु॒ । इन्द॑वः । स॒मु॒द्रम् । इ॒व॒ । सिन्ध॑वः । सा । मा॒ । स॒हस्रे᳚ । एति॑ । भ॒ज॒ । प्र॒जयेति॑ प्र - जया᳚ । प॒शुभि॒रिति॑ प॒शु-भिः॒ । स॒ह । पुनः॑ । मा॒ । एति॑ । वि॒श॒ता॒त् । र॒यिः । इति॑ । प्र॒जयेति॑ प्र - जया᳚ । ए॒व । ए॒न॒म् । प॒शुभि॒रिति॑ प॒शु - भिः॒ । र॒य्या । समिति॑ । \textbf{  26} \newline
                  \newline
                                \textbf{ TS 7.1.6.7} \newline
                  अ॒द्‌र्ध॒य॒ति॒ । प्र॒जावा॒निति॑ प्रजा - वा॒न् । प॒शु॒मानिति॑ पशु - मान् । र॒यि॒मानिति॑ रयि - मान् । भ॒व॒ति॒ । यः । ए॒वम् । वेद॑ । तया᳚ । स॒ह । आग्नी᳚द्ध्र॒मित्याग्नि॑ - इ॒द्ध्र॒म् । प॒रेत्येति॑ परा - इत्य॑ । पु॒रस्ता᳚त् । प्र॒तीच्या᳚म् । तिष्ठ॑न्त्याम् । जु॒हु॒या॒त् । उ॒भा । जि॒ग्य॒थुः॒ । न । परेति॑ । ज॒ये॒थे॒ इति॑ । न । परेति॑ । जि॒ग्ये॒ । क॒त॒रः । च॒न । ए॒नोः॒ ॥ इन्द्रः॑ । च॒ । वि॒ष्णो॒ इति॑ । यत् । अप॑स्पृधेथाम् । त्रे॒धा । स॒हस्र᳚म् । वीति॑ । तत् । ऐ॒र॒ये॒था॒म् । इति॑ । त्रे॒धा॒वि॒भ॒क्तमिति॑ त्रेधा - वि॒भ॒क्तम् । वै । त्रि॒रा॒त्र इति॑ त्रि - रा॒त्रे । स॒हस्र᳚म् । सा॒ह॒स्रीम् । ए॒व । ए॒ना॒म् । क॒रो॒ति॒ । स॒हस्र॑स्य । ए॒व । ए॒ना॒म् । मात्रा᳚म् । \textbf{  27} \newline
                  \newline
                                \textbf{ TS 7.1.6.8} \newline
                  क॒रो॒ति॒ । रू॒पाणि॑ । जु॒हो॒ति॒ । रू॒पैः । ए॒व । ए॒ना॒म् । समिति॑ । अ॒द्‌र्ध॒य॒ति॒ । तस्याः᳚ ।      उ॒पो॒त्थायेत्यु॑प - उ॒त्थाय॑ । कर्ण᳚म् । एति॑ । ज॒पे॒त् । इडे᳚ । रन्ते᳚ । अदि॑ते । सर॑स्वति । प्रिये᳚ । प्रेय॑सि । महि॑ । विश्रु॒तीति॒ वि - श्रु॒ति॒ । ए॒तानि॑ । ते॒ । अ॒घ्नि॒ये॒ । नामा॑नि । सु॒कृत॒मिति॑ सु - कृत᳚म् । मा॒ । दे॒वेषु॑ । ब्रू॒ता॒त् । इति॑ । दे॒वेभ्यः॑ । ए॒व । ए॒न॒म् । एति॑ । वे॒द॒य॒ति॒ । अन्विति॑ । ए॒न॒म् । दे॒वाः । बु॒द्ध्य॒न्ते॒ ॥ \textbf{  28 } \newline
                  \newline
                      ( ए॒तदे॒तस्यां᳚ ॅवी॒र्य॑ - मस्य त्रि॒भिश्च॑ - द॒त्ता - स॑हस्रत॒मी तस्मा॑ - दे॒व वरः॒ - सं - मात्रा॒ - मेका॒न्नच॑त्वारिꣳ॒॒शच्च॑)  \textbf{(A6)} \newline \newline
                                \textbf{ TS 7.1.7.1} \newline
                  स॒ह॒स्र॒त॒म्येति॑ सहस्र - त॒म्या᳚ । वै । यज॑मानः । सु॒व॒र्गमिति॑ सुवः - गम् । लो॒कम् । ए॒ति॒ । सा । ए॒न॒म् । सु॒व॒र्गमिति॑ सुवः - गम् । लो॒कम् । ग॒म॒य॒ति॒ । सा । मा॒ । सु॒व॒र्गमिति॑ सुवः - गम् । लो॒कम् । ग॒म॒य॒ । इति॑ । आ॒ह॒ । सु॒व॒र्गमिति॑ सुवः - गम् । ए॒व । ए॒न॒म् । लो॒कम् । ग॒म॒य॒ति॒ । सा । मा॒ । ज्योति॑ष्मन्तम् । लो॒कम् । ग॒म॒य॒ । इति॑ । आ॒ह॒ । ज्योति॑ष्मन्तम् । ए॒व । ए॒न॒म् । लो॒कम् । ग॒म॒य॒ति॒ । सा । मा॒ । सर्वान्॑ । पुण्यान्॑ । लो॒कान् । ग॒म॒य॒ । इति॑ । आ॒ह॒ । सर्वान्॑ । ए॒व । ए॒न॒म् । पुण्यान्॑ । लो॒कान् । ग॒म॒य॒ति॒ । सा । \textbf{  29} \newline
                  \newline
                                \textbf{ TS 7.1.7.2} \newline
                  मा॒ । प्र॒ति॒ष्ठामिति॑ प्रति - स्थाम् । ग॒म॒य॒ । प्र॒जयेति॑ प्र - जया᳚ । प॒शुभि॒रिति॑ प॒शु - भिः॒ । स॒ह । पुनः॑ । मा॒ । एति॑ । वि॒श॒ता॒त् । र॒यिः । इति॑ । प्र॒जयेति॑ प्र - जया᳚ । ए॒व । ए॒न॒म् । प॒शुभि॒रिति॑ प॒शु - भिः॒ । र॒य्याम् । प्रतीति॑ । स्था॒प॒य॒ति॒ । प्र॒जावा॒निति॑ प्र॒जा - वा॒न् । प॒शु॒मानिति॑ पशु - मान् । र॒यि॒मानिति॑ रयि - मान् । भ॒व॒ति॒ । यः । ए॒वम् । वेद॑ । ताम् । अ॒ग्नीध॒ इत्य॑ग्नि - इधे᳚ । वा॒ । ब्र॒ह्मणे᳚ । वा॒ । होत्रे᳚ । वा॒ । उ॒द्गा॒त्र इत्यु॑त् - गा॒त्रे । वा॒ । अ॒द्ध्व॒र्यवे᳚ । वा॒ । द॒द्या॒त् । स॒हस्र᳚म् । अ॒स्य॒ । सा । द॒त्ता । भ॒व॒ति॒ । स॒हस्र᳚म् । अ॒स्य॒ । प्रति॑गृहीत॒मिति॒ प्रति॑ - गृ॒ही॒त॒म् । भ॒व॒ति॒ । यः । ताम् । अवि॑द्वान् । \textbf{  30} \newline
                  \newline
                                \textbf{ TS 7.1.7.3} \newline
                  प्र॒ति॒गृ॒ह्णातीति॑ प्रति - गृ॒ह्णाति॑ । ताम् । प्रतीति॑ । गृ॒ह्णी॒या॒त् । एका᳚ । अ॒सि॒ । न । स॒हस्र᳚म् । एका᳚म् । त्वा॒ । भू॒ताम् । प्रतीति॑ । गृ॒ह्णा॒मि॒ । न । स॒हस्र᳚म् । एका᳚ । मा॒ । भू॒ता । एति॑ । वि॒श॒ । मा । स॒हस्र᳚म् । इति॑ । एका᳚म् । ए॒व । ए॒ना॒म् । भू॒ताम् । प्रतीति॑ । गृ॒ह्णा॒ति॒ । न । स॒हस्र᳚म् । यः । ए॒वम् । वेद॑ । स्यो॒ना । अ॒सि॒ । सु॒षदेति॑ सु-सदा᳚ । सु॒शेवेति॑ सु - शेवा᳚ । स्यो॒ना । मा॒ । एति॑ । वि॒श॒ । सु॒षदेति॑ सु - सदा᳚ । मा॒ । एति॑ । वि॒श॒ । सु॒शेवेति॑ सु - शेवा᳚ । मा॒ । एति॑ । वि॒श॒ । \textbf{  31} \newline
                  \newline
                                \textbf{ TS 7.1.7.4} \newline
                  इति॑ । आ॒ह॒ । स्यो॒ना । ए॒व । ए॒न॒म् । सु॒षदेति॑ सु - सदा᳚ । सु॒शेवेति॑ सु - शेवा᳚ । भू॒ता । एति॑ । वि॒श॒ति॒ । न । ए॒न॒म् । हि॒न॒स्ति॒ । ब्र॒ह्म॒वा॒दिन॒ इति॑ ब्रह्म - वा॒दिनः॑ । व॒द॒न्ति॒ । स॒हस्र᳚म् । स॒ह॒स्र॒त॒मीति॑ सहस्र - त॒मी । अन्विति॑ । ए॒ती(3) । स॒ह॒स्र॒त॒मीमिति॑ सहस्र - त॒मीम् । स॒हस्रा(3)म् । इति॑ । यत् । प्राची᳚म् । उ॒थ्सृ॒जेदित्यु॑त् - सृ॒जेत् । स॒हस्र᳚म् । स॒ह॒स्र॒त॒मीति॑ सहस्र - त॒मी । अन्विति॑ । इ॒या॒त् । तत् । स॒हस्र᳚म् । अ॒प्र॒ज्ञा॒त्रमित्य॑प्र - ज्ञा॒त्रम् । सु॒व॒र्गमिति॑ सुवः - गम् । लो॒कम् । न । प्रेति॑ । जा॒नी॒या॒त् । प्र॒तीची᳚म् । उदिति॑ । सृ॒ज॒ति॒ । ताम् । स॒हस्र᳚म् । अन्विति॑ । प॒र्याव॑र्तत॒ इति॑ परि - आव॑र्तते । सा । प्र॒जा॒न॒तीति॑ प्र - जा॒न॒ती । सु॒व॒र्गमिति॑ सुवः - गम् । लो॒कम् । ए॒ति॒ । यज॑मानम् ( ) । अ॒भि । उदिति॑ । सृ॒ज॒ति॒ । क्षि॒प्रे । स॒हस्र᳚म् । प्रेति॑ । जा॒य॒ते॒ । उ॒त्त॒मेत्यु॑त् - त॒मा । नी॒यते᳚ । प्र॒थ॒मा । दे॒वान् । ग॒च्छ॒ति॒ ॥ \textbf{  32} \newline
                  \newline
                      (लो॒कान् ग॑मयति॒ सा - ऽवि॑द्वान्थ् - सु॒शेवा॒ माऽऽ वि॑श॒ - यज॑मानं॒ - द्वाद॑श च)  \textbf{(A7)} \newline \newline
                                \textbf{ TS 7.1.8.1} \newline
                  अत्रिः॑ । अ॒द॒दा॒त् । और्वा॑य । प्र॒जामिति॑ प्र - जाम् । पु॒त्रका॑मा॒येति॑ पु॒त्र - का॒मा॒य॒ । सः । रि॒रि॒चा॒नः । अ॒म॒न्य॒त॒ । निर्वी᳚र्य॒ इति॒ निः - वी॒र्यः॒ । शि॒थि॒लः । या॒तया॒मेति॑ या॒त - या॒मा॒ । सः । ए॒तम् । च॒तू॒रा॒त्रमिति॑ चतुः - रा॒त्रम् । अ॒प॒श्य॒त् । तम् । एति॑ । अ॒ह॒र॒त् । तेन॑ । अ॒य॒ज॒त॒ । ततः॑ । वै । तस्य॑ । च॒त्वारः॑ । वी॒राः । एति॑ । अ॒जा॒य॒न्त॒ । सुहो॒तेति॒ सु - हो॒ता॒ । सू᳚द्गा॒तेति॒ सु - उ॒द्गा॒ता॒ । स्व॑द्ध्वर्यु॒रिति॒ सु - अ॒द्ध्व॒र्युः॒ । सुस॑भेय॒ इति॒ सु - स॒भे॒यः॒ । यः । ए॒वम् । वि॒द्वान् । च॒तू॒रा॒त्रेणेति॑ चतुः - रा॒त्रेण॑ । यज॑ते । एति॑ । अ॒स्य॒ । च॒त्वारः॑ । वी॒राः । जा॒य॒न्ते॒ । सुहो॒तेति॒ सु - हो॒ता॒ । सू᳚द्गा॒तेति॒ सु - उ॒द्गा॒ता॒ । स्व॑द्ध्वर्यु॒रिति॒ सु - अ॒द्ध्व॒र्युः॒ । सुस॑भेय॒ इति॒ सु - स॒भे॒यः॒ । ये । च॒तु॒र्विꣳ॒॒शा इति॑ चतुः - विꣳ॒॒शाः । पव॑मानाः । ब्र॒ह्म॒व॒र्च॒समिति॑ ब्रह्म - व॒र्च॒सम् । तत् । \textbf{  33} \newline
                  \newline
                                \textbf{ TS 7.1.8.2} \newline
                  ये । उ॒द्यन्त॒ इत्यु॑त् - यन्तः॑ । स्तोमाः᳚ । श्रीः । सा । अत्रि᳚म् । श्र॒द्धादे॑व॒मिति॑ श्र॒द्धा - दे॒व॒म् । यज॑मानम् । च॒त्वारि॑ । वी॒र्या॑णि । न । उपेति॑ । अ॒न॒म॒न्न् । तेजः॑ । इ॒न्द्रि॒यम् । ब्र॒ह्म॒व॒र्च॒समिति॑ ब्रह्म-व॒र्च॒सम् । अ॒न्नाद्य॒मित्य॑न्न - अद्य᳚म् । सः । ए॒तान् । च॒तुरः॑ । चतु॑ष्टोमा॒निति॒ चतुः॑ - स्तो॒मा॒न्न् । सोमान्॑ । अ॒प॒श्य॒त् । तान् । एति॑ । अ॒ह॒र॒त् । तैः । अ॒य॒ज॒त॒ । तेजः॑ । ए॒व । प्र॒थ॒मेन॑ । अवेति॑ । अ॒रु॒न्ध॒ । इ॒न्द्रि॒यम् । द्वि॒तीये॑न । ब्र॒ह्म॒व॒र्च॒समिति॑ ब्रह्म - व॒र्च॒सम् । तृ॒तीये॑न । अ॒न्नाद्य॒मित्य॑न्न -अद्य᳚म् । च॒तु॒र्थेन॑ । यः । ए॒वम् । वि॒द्वान् । च॒तुरः॑ । चतु॑ष्टोमा॒निति॒ चतुः॑ - स्तो॒मा॒न् । सोमान्॑ । आ॒हर॒तीत्या᳚ - हर॑ति । तैः । यज॑ते । तेजः॑ । ए॒व ( ) । प्र॒थ॒मेन॑ । अवेति॑ । रु॒न्धे॒ । इ॒न्द्रि॒यम् । द्वि॒तीये॑न । ब्र॒ह्म॒व॒र्च॒समिति॑ ब्रह्म - व॒र्च॒सम् । तृ॒तीये॑न । अ॒न्नाद्य॒मित्य॑न्न - अद्य᳚म् । च॒तु॒र्थेन॑ । याम् । ए॒व । अत्रिः॑ । ऋद्धि᳚म् । आद्‌र्ध्नो᳚त् । ताम् । ए॒व । यज॑मानः । ऋ॒द्ध्नो॒ति॒ ॥ \textbf{  34} \newline
                  \newline
                      ( तत्-तेज॑ ए॒वा-ष्टाद॑श च)  \textbf{(A8)} \newline \newline
                                \textbf{ TS 7.1.9.1} \newline
                  ज॒मद॑ग्निः । पुष्टि॑काम॒ इति॒ पुष्टि॑ - का॒मः॒ । च॒तू॒रा॒त्रेणेति॑ चतुः - रा॒त्रेण॑ । अ॒य॒ज॒त॒ । सः । ए॒तान् । पोषान्॑ । अ॒पु॒ष्य॒त् । तस्मा᳚त् । प॒लि॒तौ । जाम॑दग्नियौ । न । समिति॑ । जा॒ना॒ते॒ इति॑ । ए॒तान् । ए॒व । पोषान्॑ । पु॒ष्य॒ति॒ । यः । ए॒वम् । वि॒द्वान् । च॒तू॒रा॒त्रेणेति॑ चतुः - रा॒त्रेण॑ । यज॑ते । पु॒रो॒डा॒शिन्यः॑ । उ॒प॒सद॒ इत्यु॑प - सदः॑ । भ॒व॒न्ति॒ । प॒शवः॑ । वै । पु॒रो॒डाशः॑ । प॒शून् । ए॒व । अवेति॑ । रु॒न्धे॒ । अन्न᳚म् । वै । पु॒रो॒डाशः॑ । अन्न᳚म् । ए॒व । अवेति॑ । रु॒न्धे॒ । अ॒न्ना॒द इत्य॑न्न - अ॒दः । प॒शु॒मानिति॑ पशु - मान् । भ॒व॒ति॒ । यः । ए॒वम् । वि॒द्वान् । च॒तू॒रा॒त्रेणेति॑ चतुः-रा॒त्रेण॑ । यज॑ते ॥ \textbf{  35} \newline
                  \newline
                      (ज॒मद॑ग्नि - र॒ष्टाच॑त्वारिꣳशत्)  \textbf{(A9)} \newline \newline
                                \textbf{ TS 7.1.10.1} \newline
                  सं॒ॅव॒थ्स॒र इति॑ सं - व॒थ्स॒रः । वै । इ॒दम् । एकः॑ । आ॒सी॒त् । सः । अ॒का॒म॒य॒त॒ । ऋ॒तून् । सृ॒जे॒य॒ । इति॑ । सः । ए॒तम् । प॒ञ्च॒रा॒त्रमिति॑ पञ्च - रा॒त्रम् । अ॒प॒श्य॒त् । तम् । एति॑ । अ॒ह॒र॒त् । तेन॑ । अ॒य॒ज॒त॒ । ततः॑ । वै । सः । ऋ॒तून् । अ॒सृ॒ज॒त॒ । यः । ए॒वम् । वि॒द्वान् । प॒ञ्च॒रा॒त्रेणेति॑ पञ्च - रा॒त्रेण॑ । यज॑ते । प्रेति॑ । ए॒व । जा॒य॒ते॒ । ते । ऋ॒तवः॑ । सृ॒ष्टाः । न । व्याव॑र्त॒न्तेति॑ वि - आव॑र्तन्त । ते । ए॒तम् । प॒ञ्च॒रा॒त्रमिति॑ पञ्च - रा॒त्रम् । अ॒प॒श्य॒न्न् । तम् । एति॑ । अ॒ह॒र॒न्न् । तेन॑ । अ॒य॒ज॒न्त॒ । ततः॑ । वै । ते । व्याव॑र्त॒न्तेति॑ वि - आव॑र्तन्त । \textbf{  36} \newline
                  \newline
                                \textbf{ TS 7.1.10.2} \newline
                  यः । ए॒वम् । वि॒द्वान् । प॒ञ्च॒रा॒त्रेणेति॑ पञ्च-रा॒त्रेण॑ । यज॑ते । वीति॑ । पा॒प्मना᳚ । भ्रातृ॑व्येण । एति॑ । व॒र्त॒ते॒ । सार्व॑सेनि॒रिति॒ सार्व॑ - से॒निः॒ । शौ॒चे॒यः । अ॒का॒म॒य॒त॒ । प॒शु॒मानिति॑ पशु - मान् । स्या॒म् । इति॑ । सः । ए॒तम् । प॒ञ्च॒रा॒त्रमिति॑ पञ्च - रा॒त्रम् । एति॑ । अ॒ह॒र॒त् । तेन॑ । अ॒य॒ज॒त॒ । ततः॑ । वै । सः । स॒हस्र᳚म् । प॒शून् । प्रेति॑ । आ॒प्नो॒त् । यः । ए॒वम् । वि॒द्वान् । प॒ञ्च॒रा॒त्रेणेति॑ पञ्च-रा॒त्रेण॑ । यज॑ते । प्रेति॑ । स॒हस्र᳚म् । प॒शून् । आ॒प्नो॒ति॒ । ब॒ब॒रः । प्रावा॑हणिः । अ॒का॒म॒य॒त॒ । वा॒चः । प्र॒व॒दि॒तेति॑ प्र - व॒दि॒ता । स्या॒म् । इति॑ । सः । ए॒तम् । प॒ञ्च॒रा॒त्रमिति॑ पञ्च - रा॒त्रम् । एति॑ । \textbf{  37} \newline
                  \newline
                                \textbf{ TS 7.1.10.3} \newline
                  अ॒ह॒र॒त् । तेन॑ । अ॒य॒ज॒त॒ । ततः॑ । वै । सः । वा॒चः । प्र॒व॒दि॒तेति॑ प्र - व॒दि॒ता । अ॒भ॒व॒त् । यः । ए॒वम् । वि॒द्वान् । प॒ञ्च॒रा॒त्रेणेति॑ पञ्च - रा॒त्रेण॑ । यज॑ते । प्र॒व॒दि॒तेति॑ प्र - व॒दि॒ता । ए॒व । वा॒चः । भ॒व॒ति॒ । अथो॒ इति॑ । ए॒न॒म् । वा॒चः । पतिः॑ । इति॑ । आ॒हुः॒ । अना᳚प्तः । च॒तू॒रा॒त्र इति॑ चतुः - रा॒त्रः । अति॑रिक्त॒ इत्यति॑ - रि॒क्तः॒ । ष॒ड्रा॒त्र इति॑ षट् - रा॒त्रः । अथ॑ । वै । ए॒षः । स॒प्रं॒तीति॑ सं - प्र॒ति । य॒ज्ञ्ः । यत् । प॒ञ्च॒रा॒त्र इति॑ पञ्च - रा॒त्रः । यः । ए॒वम् । वि॒द्वान् । प॒ञ्च॒रा॒त्रेणेति॑ पञ्च - रा॒त्रेण॑ । यज॑ते । स॒म्प्र॒तीति॑ सं - प्र॒ति । ए॒व । य॒ज्ञेन॑ । य॒ज॒ते॒ । प॒ञ्च॒रा॒त्र इति॑ पञ्च - रा॒त्रः । भ॒व॒ति॒ । पञ्च॑ । वै । ऋ॒तवः॑ । सं॒ॅव॒थ्स॒र इति॑ सं - व॒थ्स॒रः । \textbf{  38} \newline
                  \newline
                                \textbf{ TS 7.1.10.4} \newline
                  ऋ॒तुषु॑ । ए॒व । सं॒ॅव॒थ्स॒र इति॑ सं - व॒थ्स॒रे । प्रतीति॑ । ति॒ष्ठ॒न्ति॒ । अथो॒ इति॑ । पञ्चा᳚क्ष॒रेति॒ पञ्च॑ - अ॒क्ष॒रा॒ । प॒ङ्क्तिः । पाङ्क्तः॑ । य॒ज्ञ्ः । य॒ज्ञ्म् । ए॒व । अवेति॑ । रु॒न्धे॒ । त्रि॒वृदिति॑ त्रि - वृत् । अ॒ग्नि॒ष्टो॒म इत्य॑ग्नि - स्तो॒मः । भ॒व॒ति॒ । तेजः॑ । ए॒व । अवेति॑ । रु॒न्धे॒ । प॒ञ्च॒द॒श इति॑ पञ्च - द॒शः । भ॒व॒ति॒ । इ॒न्द्रि॒यम् । ए॒व । अवेति॑ । रु॒न्धे॒ । स॒प्त॒द॒श इति॑ सप्त - द॒शः । भ॒व॒ति॒ । अ॒न्नाद्य॒स्येत्य॑न्न - अद्य॑स्य । अव॑रुद्ध्या॒ इत्यव॑ - रु॒द्ध्यै॒ । अथो॒ इति॑ । प्रेति॑ । ए॒व । तेन॑ । जा॒य॒ते॒ । प॒ञ्च॒विꣳ॒॒श इति॑ पञ्च - विꣳ॒॒शः । अ॒ग्नि॒ष्टो॒म इत्य॑ग्नि - स्तो॒मः । भ॒व॒ति॒ । प्र॒जाप॑ते॒रिति॑ प्र॒जा-प॒तेः॒ । आप्त्यै᳚ । म॒हा॒व्र॒तवा॒निति॑ महाव्र॒त-वा॒न् । अ॒न्नाद्य॒स्येत्य॑न्न - अद्य॑स्य । अव॑रुद्ध्या॒ इत्यव॑ - रु॒द्ध्यै॒ । वि॒श्व॒जिदिति॑ विश्व - जित् । सर्व॑पृष्ठ॒ इति॒ सर्व॑ - पृ॒ष्ठः॒ । अ॒ति॒रा॒त्र इत्य॑ति - रा॒त्रः । भ॒व॒ति॒ । सर्व॑स्य । अ॒भिजि॑त्या॒ इत्य॒भि-जि॒त्यै॒ ( ) ॥ \textbf{  39} \newline
                  \newline
                      (ते व्याव॑र्तन्त - प्रवदि॒ता स्या॒मिति॒ स ए॒तं प॑ञ्चरा॒त्रमा - सं॑ॅवथ्स॒रो॑- भिजि॑त्यै)  \textbf{(A10)} \newline \newline
                                \textbf{ TS 7.1.11.1} \newline
                  दे॒वस्य॑ । त्वा॒ । स॒वि॒तुः । प्र॒स॒व इति॑ प्र - स॒वे । अ॒श्विनोः᳚ । बा॒हुभ्या॒मिति॑ बा॒हु - भ्या॒म् । पू॒ष्णः । हस्ता᳚भ्याम् । एति॑ । द॒दे॒ । इ॒माम् । अ॒गृ॒भ्ण॒न्न् । र॒श॒नाम् । ऋ॒तस्य॑ । पूर्वे᳚ । आयु॑षि । वि॒दथे॑षु । क॒व्या ॥ तया᳚ । दे॒वाः । सु॒तम् । एति॑ । ब॒भू॒वुः॒ । ऋ॒तस्य॑ । सामन्न्॑ । स॒रम् । आ॒रप॒न्तीत्या᳚ - रप॑न्ती ॥ अ॒भि॒धा इत्य॑भि - धाः । अ॒सि॒ । भुव॑नम् । अ॒सि॒ । य॒न्ता । अ॒सि॒ । ध॒र्ता । अ॒सि॒ । सः । अ॒ग्निम् । वै॒श्वा॒न॒रम् । सप्र॑थस॒मिति॒ स - प्र॒थ॒स॒म् । ग॒च्छ॒ । स्वाहा॑कृत॒ इति॒ स्वाहा᳚ - कृ॒तः॒ । पृ॒थि॒व्याम् । य॒न्ता । राट् । य॒न्ता । अ॒सि॒ । यम॑नः । ध॒र्ता । अ॒सि॒ । ध॒रुणः॑ ( ) । कृ॒ष्यै । त्वा॒ । क्षेमा॑य । त्वा॒ । र॒य्यै । त्वा॒ । पोषा॑य । त्वा॒ । पृ॒थि॒व्यै । त्वा॒ । अ॒न्तरि॑क्षाय । त्वा॒ । दि॒वे । त्वा॒ । स॒ते । त्वा॒ । अस॑ते । त्वा॒ । अ॒द्भ्य इत्य॑त् - भ्यः । त्वा॒ । ओष॑धीभ्य॒ इत्योष॑धि - भ्यः॒ । त्वा॒ । विश्वे᳚भ्यः । त्वा॒ । भू॒तेभ्यः॑ ॥ \textbf{  40 } \newline
                  \newline
                      (ध॒रुणः॒ - पञ्च॑विꣳशतिश्च)  \textbf{(A11)} \newline \newline
                                \textbf{ TS 7.1.12.1} \newline
                  वि॒भूरिति॑ वि - भूः । मा॒त्रा । प्र॒भूरिति॑ प्र - भूः । पि॒त्रा । अश्वः॑ । अ॒सि॒ । हयः॑ । अ॒सि॒ । अत्यः॑ । अ॒सि॒ । नरः॑ । अ॒सि॒ । अर्वा᳚ । अ॒सि॒ । सप्तिः॑ । अ॒सि॒ । वा॒जी । अ॒सि॒ । वृषा᳚ । अ॒सि॒ । नृ॒मणा॒ इति॑ नृ - मनाः᳚ । अ॒सि॒ । ययुः॑ । नाम॑ । अ॒सि॒ । आ॒दि॒त्याना᳚म् । पत्व॑ । अन्विति॑ । इ॒हि॒ । अ॒ग्नये᳚ । स्वाहा᳚ । स्वाहा᳚ । इ॒न्द्रा॒ग्निभ्या॒मिती᳚न्द्रा॒ग्नि - भ्या॒म् । स्वाहा᳚ । प्र॒जाप॑तय॒ इति॑ प्र॒जा - प॒त॒ये॒ । स्वाहा᳚ । विश्वे᳚भ्यः । दे॒वेभ्यः॑ । स्वाहा᳚ । सर्वा᳚भ्यः । दे॒वेता᳚भ्यः । इ॒ह । धृतिः॑ । स्वाहा᳚ । इ॒ह । विधृ॑ति॒रिति॒ वि - धृ॒तिः॒ । स्वाहा᳚ । इ॒ह । रन्तिः॑ । स्वाहा᳚ ( ) । इ॒ह । रम॑तिः । स्वाहा᳚ । भूः । अ॒सि॒ । भु॒वे । त्वा॒ । भव्या॑य । त्वा॒ । भ॒वि॒ष्य॒ते । त्वा॒ । विश्वे᳚भ्यः । त्वा॒ । भू॒तेभ्यः॑ । देवाः᳚ । आ॒शा॒पा॒ला॒ इत्या॑शा - पा॒लाः॒ । ए॒तम् । दे॒वेभ्यः॑ । अश्व᳚म् । मेधा॑य । प्रोक्षि॑त॒मिति॒ प्र - उ॒क्षि॒त॒म् । गो॒पा॒य॒त॒ ॥ \textbf{  41} \newline
                  \newline
                      (रन्तिः॒ स्वाहा॒ - द्वाविꣳ॑शतिश्च)  \textbf{(A12)} \newline \newline
                                \textbf{ TS 7.1.13.1} \newline
                  आय॑ना॒येत्या᳚ - अय॑नाय । स्वाहा᳚ । प्राय॑णा॒येति॑ प्र - अय॑नाय । स्वाहा᳚ । उ॒द्द्रा॒वायेत्यु॑त् - द्रा॒वाय॑ । स्वाहा᳚ । उद्द्रु॑ता॒येत्युत्-द्रु॒ता॒य॒ । स्वाहा᳚ । शू॒का॒रायेति॑ शू-का॒राय॑ । स्वाहा᳚ । शूकृ॑ता॒येति॒ शू-कृ॒ता॒य॒ । स्वाहा᳚ । पला॑यिताय । स्वाहा᳚ । आ॒पला॑यिता॒येत्या᳚ - पला॑यिताय । स्वाहा᳚ । आ॒वल्ग॑त॒ इत्या᳚ - वल्ग॑ते । स्वाहा᳚ । प॒रा॒वल्ग॑त॒ इति॑ परा - वल्ग॑ते । स्वाहा᳚ । आ॒य॒त इत्या᳚ - य॒ते । स्वाहा᳚ । प्र॒य॒त इति॑ प्र - य॒ते । स्वाहा᳚ । सर्व॑स्मै । स्वाहा᳚ ॥ \textbf{  42} \newline
                  \newline
                      (आय॑ना॒योत्त॑रमा॒पला॑यिताय॒ षड्विꣳ॑शतिः)  \textbf{(A13)} \newline \newline
                                \textbf{ TS 7.1.14.1} \newline
                  अ॒ग्नये᳚ । स्वाहा᳚ । सोमा॑य । स्वाहा᳚ । वा॒यवे᳚ । स्वाहा᳚ । अ॒पाम् । मोदा॑य । स्वाहा᳚ । स॒वि॒त्रे । स्वाहा᳚ । सर॑स्वत्यै । स्वाहा᳚ । इन्द्रा॑य । स्वाहा᳚ । बृह॒स्पत॑ये । स्वाहा᳚ । मि॒त्राय॑ । स्वाहा᳚ । वरु॑णाय । स्वाहा᳚ । सर्व॑स्मै । स्वाहा᳚ ॥ \textbf{  43} \newline
                  \newline
                      (अ॒ग्नये॑ वा॒यवे॒ऽपां मोदा॒येन्द्रा॑य॒ त्रयो॑विꣳशतिः)  \textbf{(A14)} \newline \newline
                                \textbf{ TS 7.1.15.1} \newline
                  पृ॒थि॒व्यै । स्वाहा᳚ । अ॒न्तरि॑क्षाय । स्वाहा᳚ । दि॒वे । स्वाहा᳚ । सूर्या॑य । स्वाहा᳚ । च॒न्द्रम॑से । स्वाहा᳚ । नक्ष॑त्रेभ्यः । स्वाहा᳚ । प्राच्यै᳚ । दि॒शे । स्वाहा᳚ । दक्षि॑णायै । दि॒शे । स्वाहा᳚ । प्र॒तीच्यै᳚ । दि॒शे । स्वाहा᳚ । उदी᳚च्यै । दि॒शे । स्वाहा᳚ । ऊ॒र्ध्वायै᳚ । दि॒शे । स्वाहा᳚ । दि॒ग्भ्य इति॑ दिक् - भ्यः । स्वाहा᳚ । अ॒वा॒न्त॒र॒दि॒शाभ्य॒ इत्य॑वान्तर - दि॒शाभ्यः॑ । स्वाहा᳚ । समा᳚भ्यः । स्वाहा᳚ । श॒रद्भ्य॒ इति॑ श॒रत् - भ्यः॒ । स्वाहा᳚ । अ॒हो॒रा॒त्रेभ्य॒ इत्य॑हः - रा॒त्रेभ्यः॑ । स्वाहा᳚ । अ॒द्‌र्ध॒मा॒सेभ्य॒ इत्य॑द्‌र्ध-मा॒सेभ्यः॑ । स्वाहा᳚ । मासे᳚भ्यः । स्वाहा᳚ । ऋ॒तुभ्य॒ इत्यृ॒तु - भ्यः॒ । स्वाहा᳚ । सं॒ॅव॒थ्स॒रायेति॑ सं - व॒थ्स॒राय॑ । स्वाहा᳚ । सर्व॑स्मै । स्वाहा᳚ ॥ \textbf{  44} \newline
                  \newline
                      (पृ॒थि॒व्यै सूर्या॑य॒ नक्ष॑त्रेभ्यः॒ प्राच्यै॑ स॒प्तच॑त्वारिꣳशत्)  \textbf{(A15)} \newline \newline
                                \textbf{ TS 7.1.16.1} \newline
                  अ॒ग्नये᳚ । स्वाहा᳚ । सोमा॑य । स्वाहा᳚ । स॒वि॒त्रे । स्वाहा᳚ । सर॑स्वत्यै । स्वाहा᳚ । पू॒ष्णे । स्वाहा᳚ । बृह॒स्पत॑ये । स्वाहा᳚ । अ॒पाम् । मोदा॑य । स्वाहा᳚ । वा॒यवे᳚ । स्वाहा᳚ । मि॒त्राय॑ । स्वाहा᳚ । वरु॑णाय । स्वाहा᳚ । सर्व॑स्मै । स्वाहा᳚ ॥ \textbf{  45} \newline
                  \newline
                      (अ॒ग्नये॑ सवि॒त्रे पू॒ष्णे॑ऽपां मोदा॑य वा॒यवे॒ त्रयो॑विꣳशतिः)  \textbf{(A16)} \newline \newline
                                \textbf{ TS 7.1.17.1} \newline
                  पृ॒थि॒व्यै । स्वाहा᳚ । अ॒न्तरि॑क्षाय । स्वाहा᳚ । दि॒वे । स्वाहा᳚ । अ॒ग्नये᳚ । स्वाहा᳚ । सोमा॑य । स्वाहा᳚ । सूर्या॑य । स्वाहा᳚ । च॒न्द्रम॑से । स्वाहा᳚ । अह्ने᳚ । स्वाहा᳚ । रात्रि॑यै । स्वाहा᳚ । ऋ॒जवे᳚ । स्वाहा᳚ । सा॒धवे᳚ । स्वाहा᳚ । सु॒क्षि॒त्या इति॑ सु - क्षि॒त्यै । स्वाहा᳚ । क्षु॒धे । स्वाहा᳚ । आ॒शि॒ति॒म्ने । स्वाहा᳚ । रोगा॑य । स्वाहा᳚ । हि॒माय॑ । स्वाहा᳚ । शी॒ताय॑ । स्वाहा᳚ । आ॒त॒पायेत्या᳚ - त॒पाय॑ । स्वाहा᳚ । अर॑ण्याय । स्वाहा᳚ । सु॒व॒र्गायेति॑ सुवः - गाय॑ । स्वाहा᳚ । लो॒काय॑ । स्वाहा᳚ । सर्व॑स्मै । स्वाहा᳚ ॥ \textbf{  46} \newline
                  \newline
                      (पृ॒थि॒व्या अ॒ग्नयेऽह्ने॒ रात्रि॑यै॒ चतु॑श्चत्वारिꣳशत्)  \textbf{(A17)} \newline \newline
                                \textbf{ TS 7.1.18.1} \newline
                  भुवः॑ । दे॒वाना᳚म् । कर्म॑णा । अ॒पसा᳚ । ऋ॒तस्य॑ । प॒थ्या᳚ । अ॒सि॒ । वसु॑भि॒रिति॒ वसु॑ - भिः॒ । दे॒वेभिः॑ । दे॒वत॑या । गा॒य॒त्रेण॑ । त्वा॒ । छन्द॑सा । यु॒न॒ज्मि॒ । व॒स॒न्तेन॑ । त्वा॒ । ऋ॒तुना᳚ । ह॒विषा᳚ । दी॒क्ष॒या॒मि॒ । रु॒द्रेभिः॑ । दे॒वेभिः॑ । दे॒वत॑या । त्रैष्टु॑भेन । त्वा॒ । छन्द॑सा । यु॒न॒ज्मि॒ । ग्री॒ष्मेण॑ । त्वा॒ । ऋ॒तुना᳚ । ह॒विषा᳚ । दी॒क्ष॒या॒मि॒ । आ॒दि॒त्येभिः॑ । दे॒वेभिः॑ । दे॒वत॑या । जाग॑तेन । त्वा॒ । छन्द॑सा । यु॒न॒ज्मि॒ । व॒र्॒.षाभिः॑ । त्वा॒ । ऋ॒तुना᳚ । ह॒विषा᳚ । दी॒क्ष॒या॒मि॒ । विश्वे॑भिः । दे॒वेभिः॑ । दे॒वत॑या । आनु॑ष्टुभे॒नेत्यानु॑ - स्तु॒भे॒न॒ । त्वा॒ । छन्द॑सा । यु॒न॒ज्मि॒ । \textbf{  47} \newline
                  \newline
                                \textbf{ TS 7.1.18.2} \newline
                  श॒रदा᳚ । त्वा॒ । ऋ॒तुना᳚ । ह॒विषा᳚ । दी॒क्ष॒या॒मि॒ । अङ्गि॑रोभि॒रित्यङ्गि॑रः- भिः॒ । दे॒वेभिः॑ । दे॒वत॑या । पाङ्क्ते॑न । त्वा॒ । छन्द॑सा । यु॒न॒ज्मि॒ । हे॒म॒न्त॒शि॒शि॒राभ्या॒मिति॑ हेमन्त - शि॒शि॒राभ्या᳚म् । त्वा॒ । ऋ॒तुना᳚ । ह॒विषा᳚ । दी॒क्ष॒या॒मि॒ । एति॑ । अ॒हम् । दी॒क्षाम् । अ॒रु॒ह॒म् । ऋ॒तस्य॑ । पत्नी᳚म् । गा॒य॒त्रेण॑ । छन्द॑सा । ब्रह्म॑णा । च॒ । ऋ॒तम् । स॒त्ये । अ॒धा॒म् । स॒त्यम् । ऋ॒ते । अ॒धा॒म् ॥ म॒हीम् । उ॒ । स्विति॑ । सु॒त्रामा॑ण॒मिति॑ सु - त्रामा॑णम् । इ॒ह । धृतिः॑ । स्वाहा᳚ । इ॒ह । विधृ॑ति॒रिति॒ वि - धृ॒तिः॒ । स्वाहा᳚ । इ॒ह । रन्तिः॑ । स्वाहा᳚ । इ॒ह । रम॑तिः । स्वाहा᳚ ॥ \textbf{  48} \newline
                  \newline
                      (आनु॑ष्टुभेन त्वा॒ छन्द॑सा युन॒ज्म्ये - का॒न्न प॑ञ्चा॒शच्च॑)  \textbf{(A18)} \newline \newline
                                \textbf{ TS 7.1.19.1} \newline
                  ई॒कां॒रायेती᳚म् - का॒राय॑ । स्वाहा᳚ । ईकृं॑ता॒येती᳚म् - कृ॒ता॒य॒ । स्वाहा᳚ । क्रन्द॑ते । स्वाहा᳚ । अ॒व॒क्रन्द॑त॒ इत्य॑व - क्रन्द॑ते । स्वाहा᳚ । प्रोथ॑ते । स्वाहा᳚ । प्र॒प्रोथ॑त॒ इति॑ प्र - प्रोथ॑ते । स्वाहा᳚ । ग॒न्धाय॑ । स्वाहा᳚ । घ्रा॒ताय॑ । स्वाहा᳚ । प्रा॒णायेति॑ प्र - अ॒नाय॑ । स्वाहा᳚ । व्या॒नायेति॑ वि - अ॒नाय॑ । स्वाहा᳚ । अ॒पा॒नायेत्य॑प - अ॒नाय॑ । स्वाहा᳚ । स॒दीं॒यमा॑ना॒येति॑ सं-दी॒यमा॑नाय । स्वाहा᳚ । संदि॑ता॒येति॒ सं-दि॒ता॒य॒ । स्वाहा᳚ । वि॒चृ॒त्यमा॑ना॒येति॑ वि - चृ॒त्यमा॑नाय । स्वाहा᳚ । विचृ॑त्ता॒येति॒ वि - चृ॒त्ता॒य॒ । स्वाहा᳚ । प॒ला॒यि॒ष्यमा॑णाय । स्वाहा᳚ । पला॑यिताय । स्वाहा᳚ । उ॒प॒रꣳ॒॒स्य॒त इत्यु॑प - रꣳ॒॒स्य॒ते । स्वाहा᳚ । उप॑रता॒येत्युप॑ - र॒ता॒य॒ । स्वाहा᳚ । नि॒वे॒क्ष्य॒त इति॑ नि - वे॒क्ष्य॒ते । स्वाहा᳚ । नि॒वि॒शमा॑ना॒येति॑ नि - वि॒शमा॑नाय । स्वाहा᳚ । निवि॑ष्टा॒येति॒ नि - वि॒ष्टा॒य॒ । स्वाहा᳚ । नि॒ष॒थ्स्य॒त इति॑ नि - स॒थ्स्य॒ते । स्वाहा᳚ । नि॒षीद॑त॒ इति॑ नि - सीद॑ते । स्वाहा᳚ । निष॑ण्णा॒येति॒ नि - स॒न्ना॒य॒ । स्वाहा᳚ । \textbf{  49} \newline
                  \newline
                                \textbf{ TS 7.1.19.2} \newline
                  आ॒सि॒ष्य॒ते । स्वाहा᳚ । आसी॑नाय । स्वाहा᳚ । आ॒सि॒ताय॑ । स्वाहा᳚ । नि॒प॒थ्स्य॒त इति॑ नि - प॒थ्स्य॒ते । स्वाहा᳚ । नि॒पद्य॑माना॒येति॑ नि - पद्य॑मानाय । स्वाहा᳚ । निप॑न्ना॒येति॒ नि - प॒न्ना॒य॒ । स्वाहा᳚ । श॒यि॒ष्य॒ते । स्वाहा᳚ । शया॑नाय । स्वाहा᳚ । श॒यि॒ताय॑ । स्वाहा᳚ । स॒म्मी॒लि॒ष्य॒त इति॑ सं - मी॒लि॒ष्य॒ते । स्वाहा᳚ । स॒म्मील॑त॒ इति॑ सं - मील॑ते । स्वाहा᳚ । सम्मी॑लिता॒येति॒ सं - मी॒लि॒ता॒य॒ । स्वाहा᳚ । स्व॒फ्स्य॒ते । स्वाहा᳚ । स्व॒प॒ते । स्वाहा᳚ । सु॒प्ताय॑ । स्वाहा᳚ । प्र॒भो॒थ्स्य॒त इति॑ प्र - भो॒थ्स्य॒ते । स्वाहा᳚ । प्र॒बुद्ध्य॑माना॒येति॑ प्र - बुद्ध्य॑मानाय । स्वाहा᳚ । प्रबु॑द्धा॒येति॒ प्र - बु॒द्धा॒य॒ । स्वाहा᳚ । जा॒ग॒रि॒ष्य॒ते । स्वाहा᳚ । जाग्र॑ते । स्वाहा᳚ । जा॒ग॒रि॒ताय॑ । स्वाहा᳚ । शुश्रू॑षमाणाय । स्वाहा᳚ । शृ॒ण्व॒ते । स्वाहा᳚ । श्रु॒ताय॑ । स्वाहा᳚ । वी॒क्षि॒ष्य॒त इति॑ वि - ई॒क्षि॒ष्य॒ते । स्वाहा᳚ । \textbf{  50} \newline
                  \newline
                                \textbf{ TS 7.1.19.3} \newline
                  वीक्ष॑माणा॒येति॑ वि-ईक्ष॑माणाय । स्वाहा᳚ । वीक्षि॑ता॒येति॒ वि-ई॒क्षि॒ता॒य॒ । स्वाहा᳚ । सꣳ॒॒हा॒स्य॒त इति॑ सं - हा॒स्य॒ते । स्वाहा᳚ । स॒जिंहा॑ना॒येति॑ सं - जिहा॑नाय । स्वाहा᳚ । उ॒ज्जिहा॑ना॒येत्यु॑त्- जिहा॑नाय । स्वाहा᳚ । वि॒व॒र्थ्स्य॒त इति॑ वि - व॒र्थ्स्य॒ते । स्वाहा᳚ । वि॒वर्त॑माना॒येति॑ वि - वर्त॑मानाय । स्वाहा᳚ । विवृ॑त्ता॒येति॒ वि-वृ॒त्ता॒य॒ । स्वाहा᳚ । उ॒त्था॒स्य॒त इत्यु॑त् - स्था॒स्य॒ते । स्वाहा᳚ । उ॒त्तिष्ठ॑त॒ इत्यु॑त्- तिष्ठ॑ते । स्वाहा᳚ । उत्थि॑ता॒येत्युत् - स्थि॒ता॒य॒ । स्वाहा᳚ । वि॒ध॒वि॒ष्य॒त इति॑ वि - ध॒वि॒ष्य॒ते । स्वाहा᳚ । वि॒धू॒न्वा॒नायेति॑ वि - धू॒न्वा॒नाय॑ । स्वाहा᳚ । विधू॑ता॒येति॒ वि - धू॒ता॒य॒ । स्वाहा᳚ । उ॒त्क्रꣳ॒॒स्य॒त इत्यु॑त्- क्रꣳ॒॒स्य॒ते । स्वाहा᳚ । उ॒त्क्राम॑त॒ इत्यु॑त् - क्राम॑ते । स्वाहा᳚ । उत्क्रा᳚न्ता॒येत्युत् - क्रा॒न्ता॒य॒ । स्वाहा᳚ । च॒ङ्क्र॒मि॒ष्य॒ते । स्वाहा᳚ । च॒ङ्क्र॒म्यमा॑णाय । स्वाहा᳚ । च॒ङ्क्र॒मि॒ताय॑ । स्वाहा᳚ । क॒ण्डू॒यि॒ष्य॒ते । स्वाहा᳚ । क॒ण्डू॒यमा॑नाय । स्वाहा᳚ । क॒ण्डू॒यि॒ताय॑ । स्वाहा᳚ । नि॒क॒षि॒ष्य॒त इति॑ नि - क॒षि॒ष्य॒ते । स्वाहा᳚ । नि॒कष॑माणा॒येति॑ नि - कष॑माणाय । स्वाहा᳚ ( ) । निक॑षिता॒येति॒ नि - क॒षि॒ता॒य॒ । स्वाहा᳚ । यत् । अत्ति॑ । तस्मै᳚ । स्वाहा᳚ । यत् । पिब॑ति । तस्मै᳚ । स्वाहा᳚ । यत् । मेह॑ति । तस्मै᳚ । स्वाहा᳚ । यत् । शकृ॑त् । क॒रोति॑ । तस्मै᳚ । स्वाहा᳚ । रेत॑से । स्वाहा᳚ । प्र॒जाभ्य॒ इति॑ प्र - जाभ्यः॑ । स्वाहा᳚ । प्र॒जन॑ना॒येति॑ प्र - जन॑नाय । स्वाहा᳚ । सर्व॑स्मै । स्वाहा᳚ ॥ \textbf{  51 } \newline
                  \newline
                      (निष॑ण्णाय॒ स्वाहा॑ - वीक्षिष्य॒ते स्वाहा॑ - नि॒कष॑माणाय॒ स्वाहा॑ - स॒प्तविꣳ॑शतिश्च)  \textbf{(A19)} \newline \newline
                                \textbf{ TS 7.1.20.1} \newline
                  अ॒ग्नये᳚ । स्वाहा᳚ । वा॒यवे᳚ । स्वाहा᳚ । सूर्या॑य । स्वाहा᳚ । ऋ॒तम् । अ॒सि॒ । ऋ॒तस्य॑ । ऋ॒तम् । अ॒सि॒ । स॒त्यम् । अ॒सि॒ । स॒त्यस्य॑ । स॒त्यम् । अ॒सि॒ । ऋ॒तस्य॑ । पन्थाः᳚ । अ॒सि॒ । दे॒वाना᳚म् । छा॒या । अ॒मृत॑स्य । नाम॑ । तत् । स॒त्यम् । यत् । त्वम् । प्र॒जाप॑ति॒रिति॑ प्र॒जा - प॒तिः॒ । असि॑ । अधीति॑ । यत् । अ॒स्मि॒न्न् । वा॒जिनि॑ । इ॒व॒ । शुभः॑ । स्पद्‌र्ध॑न्ते । दिवः॑ । सूर्ये॑ण । विशः॑ । अ॒पः । वृ॒णा॒नः । प॒व॒ते॒ । क॒व्यन्न् । प॒शुम् । न । गो॒पा इति॑ गो - पाः । इर्यः॑ । परि॒ज्मेति॒ परि॑ - ज्मा॒ ॥ \textbf{  52} \newline
                  \newline
                      (अ॒ग्नये॑ वा॒यवे॒ सूर्या॑या॒ - ऽष्टाच॑त्वारिꣳशत्)  \textbf{(A20)} \newline \newline
\textbf{praSna korvai with starting padams of 1 to 20 anuvAkams :-} \newline
(प्र॒जन॑नं - प्रातस्सव॒ने वै - ब्र॑ह्मवा॒दिनः॒ स त्वा - अङ्गि॑रस॒- आपो॒ वै - सोमो॒ वै - स॑हस्रत॒म्या - त्रि॑ - र्ज॒मद॑ग्निः - संॅवथ्स॒रो - दे॒वस्य॑ -वि॒भू - राय॑नाया॒- ऽग्नये॑ - पृथि॒व्या - अ॒ग्नये॑ - पृथि॒व्यै - भुव॑ - ईकां॒राया॒ - ग्नये॑ वा॒यवे॒ सूर्या॑य - विꣳश॒तिः ) \newline

\textbf{korvai with starting padams of1, 11, 21 series of pa~jcAtis :-} \newline
(प्र॒जन॑न॒ - मङ्गि॑रसः॒ - सोमो॒ वै - प्र॑तिगृ॒ह्णाति॑ - वि॒भू - र्वीक्ष॑माणाय॒ - द्विप॑ञ्चा॒शत्) \newline

\textbf{first and last padam of first praSnam of 7th kANDam} \newline
(प्र॒जन॑नं॒ - परि॑ज्मा) \newline 


॥ हरिः॑ ॐ ॥
॥ कृष्ण यजुर्वेदीय तैत्तिरीय संहितायां सप्तमकाण्डे प्रथमः प्रश्नः समाप्तः ॥ \newline
\pagebreak
7.1.1   AppEndix\\7.1.18.2 - म॒हीमू॒षु>1 सु॒त्रामा॑ण >2\\म॒हीमू॒षु मा॒तरꣳ॑ सुव्र॒ताना॑मृ॒तस्य॒ पत्नी॒मव॑से हुवेम । \\तु॒वि॒क्ष॒त्राम॒जर॑न्तीमुरू॒चीꣳ सु॒शर्मा॑ण॒मदि॑तिꣳ सु॒प्रणी॑तिं ॥ \\\\सु॒त्रामा॑णं पृथि॒वीं द्याम॑ने॒हसꣳ॑ सु॒शर्मा॑ण॒ मदि॑तिꣳ सु॒प्रणी॑तिं । \\दैवीं॒ नावꣳ॑ स्वरि॒त्रामना॑गस॒मस्र॑वन्ती॒मा रु॑हेमा स्व॒स्तये᳚ ॥\\(आप्पेअरिन्ग् इन् ट्.श्.1.5.11.5)\\========================\\
\pagebreak
        


\end{document}
