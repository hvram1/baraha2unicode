\documentclass[17pt]{extarticle}
\usepackage{babel}
\usepackage{fontspec}
\usepackage{polyglossia}
\usepackage{extsizes}



\setmainlanguage{sanskrit}
\setotherlanguages{english} %% or other languages
\setlength{\parindent}{0pt}
\pagestyle{myheadings}
\newfontfamily\devanagarifont[Script=Devanagari]{AdishilaVedic}


\newcommand{\VAR}[1]{}
\newcommand{\BLOCK}[1]{}




\begin{document}
\begin{titlepage}
    \begin{center}
 
\begin{sanskrit}
    { \Large
    ॐ नमः परमात्मने, श्री महागणपतये नमः, श्री गुरुभ्यो नमः
ह॒रिः॒ ॐ 
    }
    \\
    \vspace{2.5cm}
    \mbox{ \Huge
    7.5      सप्तमकाण्डे पञ्चमः प्रश्नः - सत्रविशेषाभिधानं   }
\end{sanskrit}
\end{center}

\end{titlepage}
\tableofcontents

ॐ नमः परमात्मने, श्री महागणपतये नमः, श्री गुरुभ्यो नमः
ह॒रिः॒ ॐ \newline
7.5      सप्तमकाण्डे पञ्चमः प्रश्नः - सत्रविशेषाभिधानं \newline

\addcontentsline{toc}{section}{ 7.5      सप्तमकाण्डे पञ्चमः प्रश्नः - सत्रविशेषाभिधानं}
\markright{ 7.5      सप्तमकाण्डे पञ्चमः प्रश्नः - सत्रविशेषाभिधानं \hfill https://www.vedavms.in \hfill}
\section*{ 7.5      सप्तमकाण्डे पञ्चमः प्रश्नः - सत्रविशेषाभिधानं }
                                \textbf{ TS 7.5.1.1} \newline
                  गावः॑ । वै । ए॒तत् । स॒त्रम् । आ॒स॒त॒ । अ॒शृ॒ङ्गाः । स॒तीः । शृङ्गा॑णि । नः॒ । जा॒य॒न्तै॒ । इति॑ । कामे॑न । तासा᳚म् । दश॑ । मासाः᳚ । निष॑ण्णा॒ इति॒ नि - स॒न्नाः॒ । आसन्न्॑ । अथ॑ । शृङ्गा॑णि । अ॒जा॒य॒न्त॒ । ताः । उदिति॑ । अ॒ति॒ष्ठ॒न्न् । अरा᳚थ्स्म । इति॑ । अथ॑ । यासा᳚म् । न । अजा॑यन्त । ताः । सं॒ॅव॒थ्स॒रमिति॑ सं -  व॒थ्स॒रम् । आ॒प्त्वा । उदिति॑ । अ॒ति॒ष्ठ॒न्न् । अरा᳚थ्स्म । इति॑ । यासा᳚म् । च॒ । अजा॑यन्त । यासा᳚म् । च॒ । न । ताः । उ॒भयीः᳚ । उदिति॑ । अ॒ति॒ष्ठ॒न्न् । अरा᳚थ्स्म । इति॑ । गो॒स॒त्रमिति॑ गो - स॒त्रम् । वै । \textbf{  1} \newline
                  \newline
                                \textbf{ TS 7.5.1.2} \newline
                  सं॒ॅव॒थ्स॒र इति॑ सं - व॒थ्स॒रः । ये । ए॒वम् । वि॒द्वाꣳसः॑ । सं॒ॅव॒थ्स॒रमिति॑ सं - व॒थ्स॒रम् । उ॒प॒यन्तीत्यु॑प - यन्ति॑ । ऋ॒द्ध्नु॒वन्ति॑ । ए॒व । तस्मा᳚त् । तू॒प॒रा । वार्.षि॑कौ । मासौ᳚ । पर्त्वा᳚ । च॒र॒ति॒ । स॒त्राभि॑जित॒मिति॑ स॒त्र - अ॒भि॒जि॒त॒म् । हि । अ॒स्यै॒ । तस्मा᳚त् । सं॒ॅव॒थ्स॒र॒सद॒ इति॑ संॅवथ्सर - सदः॑ । यत् । किम् । च॒ । गृ॒हे । क्रि॒यते᳚ । तत् । आ॒प्तम् । अव॑रुद्ध॒मित्यव॑-रु॒द्ध॒म् । अ॒भिजि॑त॒मित्य॒भि - जि॒त॒म् । क्रि॒य॒ते॒ । स॒मु॒द्रम् । वै । ए॒ते । प्रेति॑ । प्ल॒व॒न्ते॒ । ये । सं॒ॅव॒थ्स॒रमिति॑ सं - व॒थ्स॒रम् । उ॒प॒यन्तीत्यु॑प-यन्ति॑ । यः । वै । स॒मु॒द्रस्य॑ । पा॒रम् । न । पश्य॑ति । न । वै । सः । ततः॑ । उदिति॑ । ए॒ति॒ । सं॒ॅव॒थ्स॒र इति॑ सं - व॒थ्स॒रः । \textbf{  2} \newline
                  \newline
                                \textbf{ TS 7.5.1.3} \newline
                  वै । स॒मु॒द्रः । तस्य॑ । ए॒तत् । पा॒रम् । यत् । अ॒ति॒रा॒त्रावित्य॑ति-रा॒त्रौ । ये । ए॒वम् । वि॒द्वाꣳसः॑ । सं॒ॅव॒थ्स॒रमिति॑ सं - व॒थ्स॒रम् । उ॒प॒यन्तीत्यु॑प - यन्ति॑ । अना᳚र्ताः । ए॒व । उ॒दृच॒मित्यु॑त् - ऋच᳚म् । ग॒च्छ॒न्ति॒ । इ॒यम् । वै । पूर्वः॑ । अ॒ति॒रा॒त्र इत्य॑ति - रा॒त्रः । अ॒सौ । उत्त॑र॒ इत्युत् - त॒रः॒ । मनः॑ । पूर्वः॑ । वाक् । उत्त॑र॒ इत्युत् - त॒रः॒ । प्रा॒ण इति॑ प्र -अ॒नः । पूर्वः॑ । अ॒पा॒न इत्य॑प -   अ॒नः । उत्त॑र॒ इत्युत् -त॒रः॒ । प्र॒रोध॑न॒मिति॑ प्र - रोध॑नम् । पूर्वः॑ । उ॒दय॑न॒मित्यु॑त्-अय॑नम् । उत्त॑र॒ इत्युत् - त॒रः॒ । ज्योति॑ष्टोम॒ इति॒ ज्योतिः॑-स्तो॒मः॒ । वै॒श्वा॒न॒रः । अ॒ति॒रा॒त्र इत्य॑ति - रा॒त्रः । भ॒व॒ति॒ । ज्योतिः॑ । ए॒व । पु॒रस्ता᳚त् । द॒ध॒ते॒ । सु॒व॒र्गस्येति॑ सुवः - गस्य॑ । लो॒कस्य॑ । अनु॑ख्यात्या॒ इत्यनु॑ - ख्या॒त्यै॒ । च॒तु॒र्विꣳ॒॒श इति॑ चतुः - विꣳ॒॒शः । प्रा॒य॒णीय॒ इति॑ प्र - अ॒य॒नीयः॑ । भ॒व॒ति॒ । चतु॑र्विꣳशति॒रिति॒ चतुः॑ - विꣳ॒॒श॒तिः॒ । अ॒द्‌र्ध॒मा॒सा इत्य॑द्‌र्ध - मा॒साः । \textbf{  3} \newline
                  \newline
                                \textbf{ TS 7.5.1.4} \newline
                  सं॒ॅव॒थ्स॒र इति॑ सं - व॒थ्स॒रः । प्र॒यन्त॒ इति॑ प्र - यन्तः॑ । ए॒व । सं॒ॅव॒थ्स॒र इति॑ सं - व॒थ्स॒रे । प्रतीति॑ । ति॒ष्ठ॒न्ति॒ । तस्य॑ । त्रीणि॑ । च॒ । श॒तानि॑ । ष॒ष्टिः । च॒ । स्तो॒त्रीयाः᳚ । ताव॑तीः । सं॒ॅव॒थ्स॒रस्येति॑ सं - व॒थ्स॒रस्य॑ । रात्र॑यः । उ॒भे इति॑ । ए॒व । सं॒ॅव॒थ्स॒रस्येति॑ सं - व॒थ्स॒रस्य॑ । रू॒पे इति॑ । आ॒प्नु॒व॒न्ति॒ । ते । सꣳस्थि॑त्या॒ इति॒ सं-स्थि॒त्यै॒ । अरि॑ष्ट्यै । उत्त॑रै॒रित्युत् - त॒रैः॒ । अहो॑भि॒रित्यहः॑-भिः॒ । च॒र॒न्ति॒ । ष॒ड॒हा इति॑ षट् - अ॒हाः । भ॒व॒न्ति॒ । षट् । वै । ऋ॒तवः॑ । सं॒ॅव॒थ्स॒र इति॑ सं - व॒थ्स॒रः । ऋ॒तुषु॑ । ए॒व । सं॒ॅव॒थ्स॒र इति॑ सं-व॒थ्स॒रे । प्रतीति॑ । ति॒ष्ठ॒न्ति॒ । गौः । च॒ । आयुः॑ । च॒ । म॒द्ध्य॒तः । स्तोमौ᳚ । भ॒व॒तः॒ । सं॒ॅव॒थ्स॒रस्येति॑ सं - व॒थ्स॒रस्य॑ । ए॒व । तत् । मि॒थु॒नम् । म॒द्ध्य॒तः । \textbf{  4} \newline
                  \newline
                                \textbf{ TS 7.5.1.5} \newline
                  द॒ध॒ति॒ । प्र॒जन॑ना॒येति॑ प्र - जन॑नाय । ज्योतिः॑ । अ॒भितः॑ । भ॒व॒ति॒ । वि॒मोच॑न॒मिति॑ वि - मोच॑नम् । ए॒व । तत् । छन्दाꣳ॑सि । ए॒व । तत् । वि॒मोक॒मिति॑ वि - मोक᳚म् । य॒न्ति॒ । अथो॒ इति॑ । उ॒भ॒यतो᳚ज्योति॒षेत्यु॑भ॒यतः॑-ज्यो॒ति॒षा॒ । ए॒व । ष॒ड॒हेनेति॑ षट्-अ॒हेन॑ । सु॒व॒र्गमिति॑ सुवः - गम् । लो॒कम् । य॒न्ति॒ । ब्र॒ह्म॒वा॒दिन॒ इति॑ ब्रह्म - वा॒दिनः॑ । व॒द॒न्ति॒ । आस॑ते । केन॑ । य॒न्ति॒ । इति॑ । दे॒व॒याने॒नेति॑ देव - याने॑न । प॒था । इति॑ । ब्रू॒या॒त् । छन्दाꣳ॑सि । वै । दे॒व॒यान॒ इति॑ देव - यानः॑ । पन्थाः᳚ । गा॒य॒त्री । त्रि॒ष्टुप् । जग॑ती । ज्योतिः॑ । वै । गा॒य॒त्री । गौः । त्रि॒ष्टुक् । आयुः॑ । जग॑ती । यत् । ए॒ते । स्तोमाः᳚ । भव॑न्ति । द॒व॒याने॒नेति॑ देव-याने॑न । ए॒व । \textbf{  5} \newline
                  \newline
                                \textbf{ TS 7.5.1.6} \newline
                  तत् । प॒था । य॒न्ति॒ । स॒मा॒नम् । साम॑ । भ॒व॒ति॒ । दे॒व॒लो॒क इति॑ देव - लो॒कः । वै । साम॑ । दे॒व॒लो॒कादिति॑ देव - लो॒कात् । ए॒व । न । य॒न्ति॒ । अ॒न्या‌अ॑न्या॒ इत्य॒न्याः - अ॒न्याः॒ । ऋचः॑ । भ॒व॒न्ति॒ । म॒नु॒ष्य॒लो॒क इति॑ मनुष्य - लो॒कः । वै । ऋचः॑ । म॒नु॒ष्य॒लो॒कादिति॑ मनुष्य - लो॒कात् । ए॒व । अ॒न्यम॑न्य॒मित्य॒न्यम् - अ॒न्य॒म् । दे॒व॒लो॒कमिति॑ देव - लो॒कम् । अ॒भ्या॒रोह॑न्त॒ इत्य॑भि-आ॒रोह॑न्तः । य॒न्ति॒ । अ॒भि॒व॒र्त इत्य॑भि  -  व॒र्तः । ब्र॒ह्म॒सा॒ममिति॑ ब्रह्म - सा॒मम् । भ॒व॒ति॒ । सु॒व॒र्गस्येति॑ सुवः - गस्य॑ । लो॒कस्य॑ । अ॒भिवृ॑त्या॒ इत्य॒भि - वृ॒त्यै॒ । अ॒भि॒जिदित्य॑भि - जित् । भ॒व॒ति॒ । सु॒व॒र्गस्येति॑ सुवः - गस्य॑ । लो॒कस्य॑ । अ॒भिजि॑त्या॒ इत्य॒भि - जि॒त्यै॒ । वि॒श्व॒जिदिति॑ विश्व - जित् । भ॒व॒ति॒ । विश्व॑स्य । जित्यै᳚ । मा॒सिमा॒सीति॑ मा॒सि - मा॒सि॒ । पृ॒ष्ठानि॑ । उपेति॑ । य॒न्ति॒ । मा॒सिमा॒सीति॑ मा॒सि - मा॒सि॒ । अ॒ति॒ग्रा॒ह्या॑ इत्य॑ति - ग्रा॒ह्याः᳚ । गृ॒ह्य॒न्ते॒ । मा॒सिमा॒सीति॑ मा॒सि - मा॒सि॒ । ए॒व । वी॒र्य᳚म् ( ) । द॒ध॒ति॒ । मा॒साम् । प्रति॑ष्ठित्या॒ इति॒ प्रति॑ - स्थि॒त्यै॒ । उ॒परि॑ष्टात् । मा॒साम् । पृ॒ष्ठानि॑ । उपेति॑ । य॒न्ति॒ । तस्मा᳚त् । उ॒परि॑ष्टात् । ओष॑धयः । फल᳚म् । गृ॒ह्ण॒न्ति॒ ॥ \textbf{  6 } \newline
                  \newline
                      (गो॒स॒त्रं ॅवा - ए॑ति संॅवथ्स॒रो᳚ - ऽर्द्धमा॒सा - मि॑थु॒नं म॑द्ध्य॒तो - दे॑व॒याने॑नै॒व - वी॒र्यं॑ - त्रयो॑दश च)  \textbf{(A1)} \newline \newline
                                \textbf{ TS 7.5.2.1} \newline
                  गावः॑ । वै । ए॒तत् । स॒त्रम् । आ॒स॒त॒ । अ॒शृ॒ङ्गाः । स॒तीः । शृङ्गा॑णि । सिषा॑सन्तीः । तासा᳚म् । दश॑ । मासाः᳚ । निष॑ण्णा॒ इति॒ नि - स॒न्नाः॒ । आसन्न्॑ । अथ॑ । शृङ्गा॑णि । अ॒जा॒य॒न्त॒ । ताः । अ॒ब्रु॒व॒न्न् । अरा᳚थ्स्म । उदिति॑ । ति॒ष्ठा॒म॒ । अवेति॑ । तम् । काम᳚म् । अ॒रु॒थ्स्म॒हि॒ । येन॑ । कामे॑न । न्यष॑दा॒मेति॑ नि - अस॑दाम । इति॑ । तासा᳚म् । उ॒ । तु । वै । अ॒ब्रु॒व॒न्न् । अ॒द्‌र्धाः । वा॒ । याव॑तीः । वा॒ । आसा॑महै । ए॒व । इ॒मौ । द्वा॒द॒शौ । मासौ᳚ । सं॒ॅव॒थ्स॒रमिति॑ सं - व॒थ्स॒रम् । स॒पांद्येति॑ सं - पाद्य॑ । उदिति॑ । ति॒ष्ठा॒म॒ । इति॑ । तासा᳚म् । \textbf{  7} \newline
                  \newline
                                \textbf{ TS 7.5.2.2} \newline
                  द्वा॒द॒शे । मा॒सि । शृङ्गा॑णि । प्रेति॑ । अ॒व॒र्त॒न्त॒ । श्र॒द्धयेति॑ श्रत्-धया᳚ । वा॒ । अश्र॑द्ध॒येत्यश्र॑त् - ध॒या॒ । वा॒ । ताः । इ॒माः । याः । तू॒प॒राः । उ॒भय्यः॑ । वाव । ताः । आ॒द्‌र्ध्नु॒व॒न्न् । याः । च॒ । शृङ्गा॑णि । अस॑न्वन्न् । याः । च॒ । ऊर्ज᳚म् । अ॒वारु॑न्ध॒तेत्य॑व - अरु॑न्धत । ऋ॒द्ध्नोति॑ । द॒शस्विति॑ द॒श - सु॒ । मा॒सु । उ॒त्तिष्ठ॒न्नित्यु॑त्-तिष्ठन्न्॑ । ऋ॒द्ध्नोति॑ । द्वा॒द॒शस्विति॑ द्वाद॒श - सु॒ । यः । ए॒वम् । वेद॑ । प॒देन॑ । खलु॑ । वै । ए॒ते । य॒न्ति॒ । वि॒न्दति॑ । खलु॑ । वै । प॒देन॑ । यन्न् । तत् । वै । ए॒तत् । ऋ॒द्धम् । अय॑नम् । तस्मा᳚त् ( ) । ए॒तत् । गो॒सनीति॑ गो - सनि॑ ॥ \textbf{  8} \newline
                  \newline
                      (ति॒ष्ठा॒मेति॒ तासां॒ - तस्मा॒द् - द्वे च॑)  \textbf{(A2)} \newline \newline
                                \textbf{ TS 7.5.3.1} \newline
                  प्र॒थ॒मे । मा॒सि । पृ॒ष्ठानि॑ । उपेति॑ । य॒न्ति॒ । म॒द्ध्य॒मे । उपेति॑ । य॒न्ति॒ । उ॒त्त॒म इत्यु॑त् - त॒मे । उपेति॑ । य॒न्ति॒ । तत् । आ॒हुः॒ । याम् । वै । त्रिः । एक॑स्य । अह्नः॑ । उ॒प॒सीद॒न्तीत्यु॑प - सीद॑न्ति । द॒ह्रम् । वै । सा । अप॑राभ्याम् । दोहा᳚भ्याम् । दु॒हे॒ । अथ॑ । कुतः॑ । सा । धो॒क्ष्य॒ते॒ । याम् । द्वाद॑श । कृत्वः॑ । उ॒प॒सीद॒न्तीत्यु॑प - सीद॑न्ति । इति॑ । सं॒ॅव॒थ्स॒रमिति॑ सं - व॒थ्स॒रम् । स॒पांद्येति॑ सं - पाद्य॑ । उ॒त्त॒म इत्यु॑त् - त॒मे । मा॒सि । स॒कृत् । पृ॒ष्ठानि॑ । उपेति॑ । इ॒युः॒ । तत् । यज॑मानाः । य॒ज्ञ्म् । प॒शून् । अवेति॑ । रु॒न्ध॒ते॒ । स॒मु॒द्रम् । वै । \textbf{  9} \newline
                  \newline
                                \textbf{ TS 7.5.3.2} \newline
                  ए॒ते । अ॒न॒वा॒रम् । अ॒पा॒रम् । प्रेति॑ । प्ल॒व॒न्ते॒ । ये । सं॒ॅव॒थ्स॒रमिति॑ सं - व॒थ्स॒रम् । उ॒प॒यन्तीत्यु॑प - यन्ति॑ । यत् । बृ॒ह॒द्र॒थ॒न्त॒रे इति॑ बृहत् - र॒थ॒न्त॒रे । अ॒न्वर्जे॑यु॒रित्य॑नु - अर्जे॑युः । यथा᳚ । मद्ध्ये᳚ । स॒मु॒द्रस्य॑ । प्ल॒वम् । अ॒न्वर्जे॑यु॒रित्य॑नु - अर्जे॑युः । ता॒दृक् । तत् । अनु॑थ्सर्ग॒मित्यनु॑त् - स॒र्ग॒म् । बृ॒ह॒द्र॒थ॒न्त॒राभ्या॒मिति॑ बृहत् - र॒थ॒न्त॒राभ्या᳚म् । इ॒त्वा । प्र॒ति॒ष्ठामिति॑ प्रति -  स्थाम् । ग॒च्छ॒न्ति॒ । सर्वे᳚भ्यः । वै । कामे᳚भ्यः । स॒न्धिरिति॑ सं - धिः । दु॒हे॒ । तत् । यज॑मानाः ।  सर्वान्॑ । कामान्॑ । अवेति॑ । रु॒न्ध॒ते॒ ॥ \textbf{  10} \newline
                  \newline
                      (स॒मु॒द्रं ॅवै - चतु॑स्त्रिꣳशच्च)  \textbf{(A3)} \newline \newline
                                \textbf{ TS 7.5.4.1} \newline
                  स॒मा॒न्यः॑ । ऋचः॑ । भ॒व॒न्ति॒ । म॒नु॒ष्य॒लो॒क इति॑ मनुष्य - लो॒कः । वै । ऋचः॑ । म॒नु॒ष्य॒लो॒कादिति॑ मनुष्य - लो॒कात् । ए॒व । न । य॒न्ति॒ । अ॒न्यद॑न्य॒दित्य॒न्यत् - अ॒न्य॒त् । साम॑ । भ॒व॒ति॒ । दे॒व॒लो॒क इति॑ देव - लो॒कः । वै । साम॑ । दे॒व॒लो॒कादिति॑ देव - लो॒कात् । ए॒व । अ॒न्यम॑न्य॒मित्य॒न्यं - अ॒न्य॒म् । म॒नु॒ष्य॒लो॒कमिति॑ मनुष्य - लो॒कम् । प्र॒त्य॒व॒रोह॑न्त॒ इति॑ प्रति - अ॒व॒रोह॑न्तः । य॒न्ति॒ । जग॑तीम् । अग्रे᳚ । उपेति॑ । य॒न्ति॒ । जग॑तीम् । वै । छन्दाꣳ॑सि । प्र॒त्यव॑रोह॒न्तीति॑ प्रति - अव॑रोहन्ति । आ॒ग्र॒य॒णम् । ग्रहाः᳚ । बृ॒हत् । पृ॒ष्ठानि॑ । त्र॒य॒स्त्रिꣳ॒॒शमिति॑ त्रयः - त्रिꣳ॒॒शम् । स्तोमाः᳚ । तस्मा᳚त् । ज्यायाꣳ॑सम् । कनी॑यान् । प्र॒त्यव॑रोह॒तीति॑ प्रति - अव॑रोहति । वै॒श्व॒क॒र्म॒ण इति॑ वैश्व - क॒र्म॒णः । गृ॒ह्य॒ते॒ । विश्वा॑नि । ए॒व । तेन॑ । कर्मा॑णि । यज॑मानाः । अवेति॑ । रु॒न्ध॒ते॒ । आ॒दि॒त्यः । \textbf{  11} \newline
                  \newline
                                \textbf{ TS 7.5.4.2} \newline
                  गृ॒ह्य॒ते॒ । इ॒यम् । वै । अदि॑तिः । अ॒स्याम् । ए॒व । प्रतीति॑ । ति॒ष्ठ॒न्ति॒ । अ॒न्यो᳚न्य॒ इत्य॒न्यः - अ॒न्यः॒ । गृ॒ह्ये॒ते॒ इति॑ । मि॒थु॒न॒त्वायेति॑ मिथुन - त्वाय॑ । प्रजा᳚त्या॒ इति॒ प्र - जा॒त्यै॒ । अ॒वा॒न्त॒रमित्य॑व - अ॒न्त॒रम् । वै । द॒श॒रा॒त्रेणेति॑ दश - रा॒त्रेण॑ । प्र॒जाप॑ति॒रिति॑ प्र॒जा - प॒तिः॒ । प्र॒जा इति॑ प्र - जाः । अ॒सृ॒ज॒त॒ । यत् । द॒श॒रा॒त्र इति॑ दश - रा॒त्रः । भव॑ति । प्र॒जा इति॑ प्र - जाः । ए॒व । तत् । यज॑मानाः । सृ॒ज॒न्ते॒ । ए॒ताम् । ह॒ । वै । उ॒द॒ङ्कः । शौ॒ल्बा॒य॒नः । स॒त्रस्य॑ । ऋद्धि᳚म् । उ॒वा॒च॒ । यत् । द॒श॒रा॒त्र इति॑ दश - रा॒त्रः । यत् । द॒श॒रा॒त्र इति॑ दश - रा॒त्रः । भव॑ति । स॒त्रस्य॑ । ऋद्ध्यै᳚ । अथो॒ इति॑ । यत् । ए॒व । पूर्वे॑षु । अह॒स्स्वित्यहः॑ - सु॒ । विलो॒मेति॒ वि - लो॒म॒ । क्रि॒यते᳚ । तस्य॑ । ए॒व ( ) । ए॒षा । शान्तिः॑ ॥ \textbf{  12} \newline
                  \newline
                      (आ॒दि॒त्य - स्तस्यै॒व - द्वे च॑)  \textbf{(A4)} \newline \newline
                                \textbf{ TS 7.5.5.1} \newline
                  यदि॑ । सोमौ᳚ । सꣳसु॑ता॒विति॒ सं - सु॒तौ॒ । स्याता᳚म् । म॒ह॒ति । रात्रि॑यै । प्रा॒त॒र॒नु॒वा॒कमिति॑ प्रातः - अ॒नु॒वा॒कम् । उ॒पाकु॑र्या॒दित्यु॑प - आकु॑र्यात् । पूर्वः॑ । वाच᳚म् । पूर्वः॑ । दे॒वताः᳚ । पूर्वः॑ । छन्दाꣳ॑सि । वृ॒ङ्क्ते॒ । वृष॑ण्वती॒मिति॒ वृषण्॑ - व॒ती॒म् । प्र॒ति॒पद॒मिति॑ प्रति - पद᳚म् । कु॒र्या॒त् । प्रा॒त॒स्स॒व॒नादिति॑ प्रातः-स॒व॒नात् । ए॒व । ए॒षा॒म् । इन्द्र᳚म् । वृ॒ङ्क्ते॒ । अथो॒ इति॑ । खलु॑ । आ॒हुः॒ । स॒व॒न॒मु॒खेस॑वनमुख॒ इति॑ सवनमु॒खे - स॒व॒न॒मु॒खे॒ । का॒र्या᳚ । इति॑ । स॒व॒न॒मु॒खाथ्स॑वनमुखा॒दिति॑ सवनमु॒खात् - स॒व॒न॒मु॒खा॒त् । ए॒व । ए॒षा॒म् । इन्द्र᳚म् । वृ॒ङ्क्ते॒ । सं॒ॅवे॒शायेति॑ सं - वे॒शाय॑ । उ॒प॒वे॒शायेत्यु॑प - वे॒शाय॑ । गा॒य॒त्रि॒याः । त्रि॒ष्टुभः॑ । जग॑त्याः । अ॒नु॒ष्टुभ॒ इत्य॑नु - स्तुभः॑ । प॒ङ्क्त्याः । अ॒भिभू᳚त्या॒ इत्य॒भि - भू॒त्यै॒ । स्वाहा᳚ । छन्दाꣳ॑सि । वै । सं॒ॅवे॒श इति॑ सं - वे॒शः । उ॒प॒वे॒श इत्यु॑प - वे॒शः । छन्दो॑भि॒रिति॒ छन्दः॑ - भिः॒ । ए॒व । ए॒षा॒म् । \textbf{  13} \newline
                  \newline
                                \textbf{ TS 7.5.5.2} \newline
                  छन्दाꣳ॑सि । वृ॒ङ्क्ते॒ । स॒ज॒नीय॒मिति॑ स - ज॒नीय᳚म् । शस्य᳚म् । वि॒ह॒व्य॑मिति॑ वि - ह॒व्य᳚म् । शस्य᳚म् । अ॒गस्त्य॑स्य । क॒या॒शु॒भीय॒मिति॑ कया-शु॒भीय᳚म् । शस्य᳚म् । ए॒ताव॑त् । वै । अ॒स्ति॒ । याव॑त् । ए॒तत् । याव॑त् । ए॒व । अस्ति॑ । तत् । ए॒षा॒म् । वृ॒ङ्क्ते॒ । यदि॑ । प्रा॒त॒स्स॒व॒न इति॑ प्रातः - स॒व॒ने । क॒लशः॑ । दीर्ये॑त । वै॒ष्ण॒वीषु॑ । शि॒पि॒वि॒ष्टव॑ती॒ष्विति॑ शिपिवि॒ष्ट - व॒ती॒षु॒ । स्तु॒वी॒र॒न्न् । यत् । वै । य॒ज्ञ्स्य॑ । अ॒ति॒रिच्य॑त॒ इत्य॑ति - रिच्य॑ते । विष्णु᳚म् । तत् । शि॒पि॒वि॒ष्टमिति॑ शिपि - वि॒ष्टम् । अ॒भि । अतीति॑ । रि॒च्य॒ते॒ । तत् । विष्णुः॑ । शि॒पि॒वि॒ष्ट इति॑ शिपि - वि॒ष्टः । अति॑रिक्त॒ इत्यति॑- रि॒क्ते॒ । ए॒व । अति॑रिक्त॒मित्यति॑ - रि॒क्त॒म् । द॒धा॒ति॒ । अथो॒ इति॑ । अति॑रिक्ते॒नेत्यति॑ - रि॒क्ते॒न॒ । ए॒व । अति॑रिक्त॒मित्यति॑ - रि॒क्त॒म् । आ॒प्त्वा । अवेति॑ ( ) । रु॒न्ध॒ते॒ । यदि॑ । म॒द्ध्यन्दि॑ने । दीर्ये॑त । व॒ष॒ट्का॒रनि॑धन॒मिति॑ वषट्का॒र - नि॒ध॒न॒म् । साम॑ । कु॒र्युः॒ । व॒ष॒ट्का॒र इति॑ वषट् - का॒रः । वै । य॒ज्ञ्स्य॑ । प्र॒ति॒ष्ठेति॑ प्रति - स्था । प्र॒ति॒ष्ठामिति॑ प्रति - स्थाम् । ए॒व । ए॒न॒त् । ग॒म॒य॒न्ति॒ । यदि॑ । तृ॒ती॒य॒स॒व॒न इति॑ तृतीय - स॒व॒ने । ए॒तत् । ए॒व ॥ \textbf{  14} \newline
                  \newline
                      (छन्दो॑भिरे॒वैषा॒ - मवै - का॒न्नविꣳ॑श॒तिश्च॑)  \textbf{(A5)} \newline \newline
                                \textbf{ TS 7.5.6.1} \newline
                  ष॒ड॒हैरिति॑ षट् - अ॒हैः । मासान्॑ । स॒पांद्येति॑ सं - पाद्य॑ । अहः॑ । उदिति॑ । सृ॒ज॒न्ति॒ । ष॒ड॒हैरिति॑ षट् - अ॒हैः । हि । मासान्॑ । स॒पंश्य॒न्तीति॑ सं - पश्य॑न्ति । अ॒द्‌र्ध॒मा॒सैरित्य॑द्‌र्ध - मा॒सैः । मासान्॑ । स॒पांद्येति॑ सं - पाद्य॑ । अहः॑ । उदिति॑ । सृ॒ज॒न्ति॒ । अ॒द्‌र्ध॒मा॒सैरित्य॑द्‌र्ध - मा॒सैः । हि । मासान्॑ । स॒पंश्य॒न्तीति॑ सं - पश्य॑न्ति । अ॒मा॒वा॒स्य॑येत्य॑मा - वा॒स्य॑या । मासान्॑ । स॒पांद्येति॑ सं - पाद्य॑ । अहः॑ । उदिति॑ । सृ॒ज॒न्ति॒ । अ॒मा॒वा॒स्य॑येत्य॑मा - वा॒स्य॑या । हि । मासान्॑ । स॒पंश्य॒न्तीति॑ सं-पश्य॑न्ति । पौ॒र्ण॒मा॒स्येति॑ पौर्ण-मा॒स्या । मासान्॑ । स॒पांद्येति॑ सं - पाद्य॑ । अहः॑ । उदिति॑ । सृ॒ज॒न्ति॒ । पौ॒र्ण॒मा॒स्येति॑ पौर्ण - मा॒स्या । हि । मासान्॑ । स॒पंश्य॒न्तीति॑ सं - पश्य॑न्ति । यः । वै । पू॒र्णे । आ॒सि॒ञ्चतीत्या᳚ - सि॒ञ्चति॑ ।परेति॑ । सः । सि॒ञ्च॒ति॒ । यः । पू॒र्णात् । उ॒दच॒तीत्यु॑त् - अच॑ति । \textbf{  15} \newline
                  \newline
                                \textbf{ TS 7.5.6.2} \newline
                  प्रा॒णमिति॑ प्र - अ॒नम् । अ॒स्मि॒न्न् । सः । द॒धा॒ति॒ । यत् । पौ॒र्ण॒मा॒स्येति॑ पौर्ण - मा॒स्या । मासान्॑ । स॒पांद्येति॑ सं - पाद्य॑ । अहः॑ । उ॒थ्सृ॒जन्तीत्यु॑त् - सृ॒जन्ति॑ । सं॒ॅव॒थ्स॒रायेति॑ सं-व॒थ्स॒राय॑ । ए॒व । तत् । प्रा॒णमिति॑ प्र - अ॒नम् । द॒ध॒ति॒ । तत् । अन्विति॑ । स॒त्रिणः॑ । प्रेति॑ । अ॒न॒न्ति॒ । यत् । अहः॑ । न । उ॒थ्सृ॒जेयु॒रित्यु॑त्-सृ॒जेयुः॑ । यथा᳚ । दृतिः॑ । उप॑नद्ध॒ इत्युप॑ - न॒द्धः॒ । वि॒पत॒तीति॑ वि - पत॑ति । ए॒वम् । सं॒ॅव॒थ्स॒र इति॑ सं - व॒थ्स॒रः । वीति॑ । प॒ते॒त् । आर्ति᳚म् । एति॑ । ऋ॒च्छे॒युः॒ । यत् । पौ॒र्ण॒मा॒स्येति॑ पौर्ण - मा॒स्या । मासान्॑ । स॒पांद्येति॑ सं - पाद्य॑ । अहः॑ । उ॒थ्सृ॒जन्तीत्यु॑त् - सृ॒जन्ति॑ । सं॒ॅव॒थ्स॒रायेति॑ सं - व॒थ्स॒राय॑ । ए॒व । तत् । उ॒दा॒नमित्यु॑त् - अ॒नम् । द॒ध॒ति॒ । तत् । अन्विति॑ । स॒त्रिणः॑ । उदिति॑ । \textbf{  16} \newline
                  \newline
                                \textbf{ TS 7.5.6.3} \newline
                  अ॒न॒न्ति॒ । न । आर्ति᳚म् । एति॑ । ऋ॒च्छ॒न्ति॒ । पू॒र्णमा॑स॒ इति॑ पू॒र्ण - मा॒से॒ । वै । दे॒वाना᳚म् । सु॒तः । यत् । पौ॒र्ण॒मा॒स्येति॑ पौर्ण - मा॒स्या । मासान्॑ । स॒पांद्येति॑ सं - पाद्य॑ । अहः॑ । उ॒थ्सृ॒जन्तीत्य॑त् - सृ॒जन्ति॑ । दे॒वाना᳚म् । ए॒व । तत् । य॒ज्ञेन॑ । य॒ज्ञ्म् । प्र॒त्यव॑रोह॒न्तीति॑ प्रति - अव॑रोहन्ति । वीति॑ । वै । ए॒तत् । य॒ज्ञ्म् । छि॒न्द॒न्ति॒ । यत् । ष॒ड॒हस॑न्तत॒मिति॑ षड॒ह - स॒न्त॒त॒म् । सन्त᳚म् । अथ॑ । अहः॑ । उ॒थ्सृ॒जन्तीत्यु॑त् - सृ॒जन्ति॑ । प्रा॒जा॒प॒त्यमिति॑ प्राजा-प॒त्यम् । प॒शुम् । एति॑ । ल॒भ॒न्ते॒ । प्र॒जाप॑ति॒रिति॑ प्र॒जा-प॒तिः॒ । सर्वाः᳚ । दे॒वताः᳚ । दे॒वता॑भिः । ए॒व । य॒ज्ञ्म् । समिति॑ । त॒न्व॒न्ति॒ । यन्ति॑ । वै । ए॒ते । सव॑नात् । ये । अहः॑ । \textbf{  17} \newline
                  \newline
                                \textbf{ TS 7.5.6.4} \newline
                  उ॒थ्सृ॒जन्तीत्यु॑त् - सृ॒जन्ति॑ । तु॒रीय᳚म् । खलु॑ । वै । ए॒तत् । सव॑नम् । यत् । सा॒न्ना॒य्यमिति॑ सां-ना॒य्यम् । यत् । सा॒न्ना॒य्यमिति॑ सां-ना॒य्यम् । भव॑ति । तेन॑ । ए॒व । सव॑नात् । न । य॒न्ति॒ । स॒मु॒प॒हूयेति॑ सं - उ॒प॒हूय॑ । भ॒क्ष॒य॒न्ति॒ । ए॒तथ्सो॑मपीथा॒ इत्ये॒तत् - सो॒म॒पी॒थाः॒ । हि । ए॒तर्.हि॑ । य॒था॒य॒त॒नमिति॑ यथा - अ॒य॒त॒नम् । वै । ए॒तेषा᳚म् । स॒व॒न॒भाज॒ इति॑ सवन - भाजः॑ । दे॒वताः᳚ । ग॒च्छ॒न्ति॒ । ये । अहः॑ । उ॒थ्सृ॒जन्तीत्यु॑त् - सृ॒जन्ति॑ । अ॒नु॒स॒व॒नमित्य॑नु - स॒व॒नम् । पु॒रो॒डाशान्॑ । निरिति॑ । व॒प॒न्ति॒ । य॒था॒य॒त॒नादिति॑ यथा - आ॒य॒त॒नात् । ए॒व । स॒व॒न॒भाज॒ इति॑ सवन - भाजः॑ । दे॒वताः᳚ । अवेति॑ । रु॒न्ध॒ते॒ । अ॒ष्टाक॑पाला॒नित्य॒ष्टा - क॒पा॒ला॒न् । प्रा॒त॒स्स॒व॒न इति॑ प्रातः - स॒व॒ने । एका॑दशकपाला॒नित्येका॑दश - क॒पा॒ला॒न्न् । माद्ध्य॑न्दिने । सव॑ने । द्वाद॑शकपाला॒निति॒ द्वाद॑श - क॒पा॒ला॒न्न् । तृ॒ती॒य॒स॒व॒न इति॑ तृतीय - स॒व॒ने । छन्दाꣳ॑सि । ए॒व । आ॒प्त्वा ( ) । अवेति॑ । रु॒न्ध॒ते॒ । वै॒श्व॒दे॒वमिति॑ वैश्व-दे॒वम् । च॒रुम् । तृ॒ती॒य॒स॒व॒न इति॑ तृतीय-स॒व॒ने । निरिति॑ । व॒प॒न्ति॒ । वै॒श्व॒दे॒वमिति॑ वैश्व - दे॒वम् । वै । तृ॒ती॒य॒स॒व॒नमिति॑ तृतीय - स॒व॒नम् । तेन॑ । ए॒व । तृ॒ती॒य॒स॒व॒नादिति॑ तृतीय - स॒व॒नात् । न । य॒न्ति॒ ॥ \textbf{  18} \newline
                  \newline
                      (उ॒दच॒ - त्यु - द्येऽह॑ - रा॒प्त्वा - पञ्च॑दश च)  \textbf{(A6)} \newline \newline
                                \textbf{ TS 7.5.7.1} \newline
                  उ॒थ्सृज्या(3)मित्यु॑त् - सृज्या(3)म् । न । उ॒थ्सृज्या(3)मित्यु॑त्- सृज्या(3)म् । इति॑ । मी॒माꣳ॒॒स॒न्ते॒ । ब्र॒ह्म॒वा॒दिन॒ इति॑ ब्रह्म - वा॒दिनः॑ । तत् । उ॒ । आ॒हुः॒ । उ॒थ्सृज्य॒मित्यु॑त् - सृज्य᳚म् । ए॒व । इति॑ । अ॒मा॒वा॒स्या॑या॒मित्य॑मा - वा॒स्या॑याम् । च॒ । पौ॒र्ण॒मा॒स्यामिति॑ पौर्ण - मा॒स्याम् । च॒ । उ॒थ्सृज्य॒मित्यु॑त् - सृज्य᳚म् । इति॑ । आ॒हुः॒ । ए॒ते इति॑ । हि । य॒ज्ञ्म् । वह॑तः । इति॑ । ते इति॑ । तु । वाव । न । उ॒थ्सृज्ये॒ इत्यु॑त् - सृज्ये᳚ । इति॑ । आ॒हुः॒ । ये इति॑ । अ॒वा॒न्त॒रमित्य॑व - अ॒न्त॒रम् । य॒ज्ञ्म् । भे॒जाते॒ इति॑ । इति॑ । या । प्र॒थ॒मा । व्य॑ष्ट॒केति॒ वि - अ॒ष्ट॒का॒ । तस्या᳚म् । उ॒थ्सृज्य॒मित्यु॑त् - सृज्य᳚म् । इति॑ । आ॒हुः॒ । ए॒षः । वै । मा॒सः । वि॒श॒र इति॑ वि - श॒रः । इति॑ । न । आदि॑ष्ट॒मित्या - दि॒ष्ट॒म् । \textbf{  19} \newline
                  \newline
                                \textbf{ TS 7.5.7.2} \newline
                  उदिति॑ । सृ॒जे॒युः॒ । यत् । आदि॑ष्ट॒मित्या - दि॒ष्ट॒म् । उ॒थ्सृ॒जेयु॒रित्यु॑त् - सृ॒जेयुः॑ । या॒दृशे᳚ । पुनः॑ । प॒र्या॒प्ला॒व इति॑ परि - आ॒प्ला॒वे । मद्ध्ये᳚ । ष॒ड॒हस्येति॑ षट् - अ॒हस्य॑ । स॒पंद्ये॒तेति॑ सं - पद्ये॑त । ष॒ड॒हैरिति॑ षट् - अ॒हैः । मासान्॑ । स॒पांद्येति॑ सं - पाद्य॑ । यत् । स॒प्त॒मम् । अहः॑ । तस्मिन्न्॑ । उदिति॑ । सृ॒जे॒युः॒ । तत् । अ॒ग्नये᳚ । वसु॑मत॒ इति॒ वसु॑ - म॒ते॒ । पु॒रो॒डाश᳚म् । अ॒ष्टाक॑पाल॒मित्य॒ष्टा - क॒पा॒ल॒म् । निरितिः॑ । व॒पे॒युः॒ । ऐ॒न्द्रम् । दधि॑ । इन्द्रा॑य । म॒रुत्व॑ते । पु॒रो॒डाश᳚म् । एका॑दशकपाल॒मित्येका॑दश - क॒पा॒ल॒म् । वै॒श्व॒दे॒वमिति॑ वैश्व-दे॒वम् । द्वाद॑शकपाल॒मिति॒ द्वाद॑श - क॒पा॒ल॒म् । अ॒ग्नेः । वै । वसु॑मत॒ इति॒ वसु॑ - म॒तः॒ । प्रा॒त॒स्स॒व॒नमिति॑ प्रातः - स॒व॒नम् । यत् । अ॒ग्नये᳚ । वसु॑मत॒ इति॒ वसु॑-म॒ते॒ । पु॒रो॒डाश᳚म् । अ॒ष्टाक॑पाल॒मित्य॒ष्टा - क॒पा॒ल॒म् । नि॒र्वप॒न्तीति॑ निः - वप॑न्ति । दे॒वता᳚म् । ए॒व । तत् । भा॒गिनी᳚म् । कु॒र्वन्ति॑ । \textbf{  20} \newline
                  \newline
                                \textbf{ TS 7.5.7.3} \newline
                  सव॑नम् । अ॒ष्टा॒भिः । उपेति॑ । य॒न्ति॒ । यत् । ऐ॒न्द्रम् । दधि॑ । भव॑ति । इन्द्र᳚म् । ए॒व । तत् । भा॒ग॒धेया॒दिति॑ भाग-धेया᳚त् । न । च्या॒व॒य॒न्ति॒ । इन्द्र॑स्य । वै । म॒रुत्व॑तः । माद्ध्य॑न्दिनम् । सव॑नम् । यत् । इन्द्रा॑य । म॒रुत्व॑ते । पु॒रो॒डाश᳚म् । एका॑दशकपाल॒मित्येका॑दश - क॒पा॒ल॒म् । नि॒र्वप॒न्तीति॑ निः-वप॑न्ति । दे॒वता᳚म् । ए॒व । तत् । भा॒गिनी᳚म् । कु॒र्वन्ति॑ । सव॑नम् । ए॒का॒द॒शभि॒रित्ये॑काद॒श - भिः॒ । उपेति॑ । य॒न्ति॒ । विश्वे॑षाम् । वै । दे॒वाना᳚म् । ऋ॒भु॒मता॒मित्यृ॑भु - मता᳚म् । तृ॒ती॒य॒स॒व॒नमिति॑ तृतीय-स॒व॒नम् । यत् । वै॒श्व॒दे॒वमिति॑ वैश्व-दे॒वम् । द्वाद॑शकपाल॒मिति॒ द्वाद॑श - क॒पा॒ल॒म् । नि॒र्वप॒न्तीति॑ निः - वप॑न्ति । दे॒वताः᳚ । ए॒व । तत् । भा॒गिनीः᳚ । कु॒र्वन्ति॑ । सव॑नम् । द्वा॒द॒शभि॒रिति॑ द्वाद॒श - भिः॒ । \textbf{  21} \newline
                  \newline
                                \textbf{ TS 7.5.7.4} \newline
                  उपेति॑ । य॒न्ति॒ । प्रा॒जा॒प॒त्यमिति॑ प्राजा - प॒त्यम् । प॒शुम् । एति॑ । ल॒भ॒न्ते॒ । य॒ज्ञ्ः । वै । प्र॒जाप॑ति॒रिति॑ प्र॒जा - प॒तिः॒ । य॒ज्ञ्स्य॑ । अन॑नुसर्गा॒येत्यन॑नु - स॒र्गा॒य॒ । अ॒भि॒व॒र्त इत्य॑भि - व॒र्तः । इ॒तः । षट् । मा॒सः । ब्र॒ह्म॒सा॒ममिति॑ ब्रह्म - सा॒मम् । भ॒व॒ति॒ । ब्रह्म॑ । वै । अ॒भि॒व॒र्त इत्य॑भि - व॒र्तः । ब्रह्म॑णा । ए॒व । तत् । सु॒व॒र्गमिति॑ सुवः - गम् । लो॒कम् । अ॒भि॒व॒र्तय॑न्त॒ इत्य॑भि - व॒र्तय॑न्तः । य॒न्ति॒ । प्र॒ति॒कू॒लमिति॑ प्रति - कू॒लम् । इ॒व॒ । हि । इ॒तः । सु॒व॒र्ग इति॑ सुवः - गः । लो॒कः । इन्द्र॑ । क्रतु᳚म् । नः॒ । एति॑ । भ॒र॒ । पि॒ता । पु॒त्रेभ्यः॑ । यथा᳚ ॥ शिक्ष॑ । नः॒ । अ॒स्मिन्न् । पु॒रु॒हू॒तेति॑ पुरु - हू॒त॒ । याम॑नि । जी॒वाः । ज्योतिः॑ । अ॒शी॒म॒हि॒ । इति॑ ( ) । अ॒मुतः॑ । आ॒य॒तामित्या᳚ - य॒ताम् । षट् । मा॒सः । ब्र॒ह्म॒सा॒ममिति॑ ब्रह्म-सा॒मम् । भ॒व॒ति॒ । अ॒यम् । वै । लो॒कः । ज्योतिः॑ । प्र॒जेति॑ प्र-जा । ज्योतिः॑ । इ॒मम् । ए॒व । तत् । लो॒कम् । पश्य॑न्तः । अ॒भि॒वद॑न्त॒इत्य॑भि - वद॑न्तः । एति॑ । य॒न्ति॒ ॥ \textbf{  22} \newline
                  \newline
                      (नाऽऽदि॑ष्टं - कु॒र्वन्ति॑ - द्वाद॒शभि॒ - रिति॑- विꣳश॒तिश्च॑)  \textbf{(A7)} \newline \newline
                                \textbf{ TS 7.5.8.1} \newline
                  दे॒वाना᳚म् । वै । अन्त᳚म् । ज॒ग्मुषा᳚म् । इ॒न्द्रि॒यम् । वी॒र्य᳚म् । अपेति॑ । अ॒क्रा॒म॒त् । तत् । क्रो॒शेन॑ । अवेति॑ । अ॒रु॒न्ध॒त॒ । तत् । क्रो॒शस्य॑ । क्रो॒श॒त्वमिति॑ क्रोश - त्वम् । यत् । क्रो॒शेन॑ । चात्वा॑लस्य । अन्ते᳚ । स्तु॒वन्ति॑ । य॒ज्ञ्स्य॑ । ए॒व । अन्त᳚म् । ग॒त्वा । इ॒न्द्रि॒यम् । वी॒र्य᳚म् । अवेति॑ । रु॒न्ध॒ते॒ । स॒त्रस्यद्‌र्ध्या᳚ । आ॒ह॒व॒नीय॒स्येत्या᳚ - ह॒व॒नीय॑स्य । अन्ते᳚ । स्तु॒व॒न्ति॒ । अ॒ग्निम् । ए॒व । उ॒प॒द्र॒ष्टार॒मित्यू॑प - द्र॒ष्टार᳚म् । कृ॒त्वा । ऋद्धि᳚म् । उपेति॑ । य॒न्ति॒ । प॒जाप॑त॒र्॒.हृद॑येन । ह॒वि॒द्‌र्धान॒ इति॑ हविः - धाने᳚ । अ॒न्तः । स्तु॒व॒न्ति॒ । प्रे॒माण᳚म् । ए॒व । अ॒स्य॒ । ग॒च्छ॒न्ति॒ । श्लो॒केन॑ । पु॒रस्ता᳚त् । सद॑सः । \textbf{  23} \newline
                  \newline
                                \textbf{ TS 7.5.8.2} \newline
                  स्तु॒व॒न्ति॒ । अनु॑श्लोके॒नेत्यनु॑ - श्लो॒के॒न॒ । प॒श्चात् । य॒ज्ञ्स्य॑ । ए॒व । अन्त᳚म् । ग॒त्वा । श्लो॒क॒भाज॒ इति॑ श्लोक - भाजः॑ । भ॒व॒न्ति॒ । न॒वभि॒रिति॑ न॒व - भिः॒ । अ॒द्ध्व॒र्युः । उदिति॑ । गा॒य॒ति॒ । नव॑ । वै । पुरु॑षे । प्रा॒णा इति॑ प्र - अ॒नाः । प्रा॒णानिति॑ प्र - अ॒नान् । ए॒व । यज॑मानेषु । द॒धा॒ति॒ । सर्वाः᳚ । ऐ॒न्द्रियः॑ । भ॒व॒न्ति॒ । प्रा॒णेष्विति॑ प्र - अ॒नेषु॑ । ए॒व । इ॒न्द्रि॒यम् । द॒ध॒ति॒ । अप्र॑तिहृताभि॒रित्यप्र॑ति - हृ॒ता॒भिः॒ । उदिति॑ । गा॒य॒ति॒ । तस्मा᳚त् । पुरु॑षः । सर्वा॑णि । अ॒न्यानि॑ । शी॒र्ष्णः । अङ्गा॑नि । प्रतीति॑ । अ॒च॒ति॒ । शिरः॑ । ए॒व । न । प॒ञ्च॒द॒शमिति॑ पञ्च - द॒शम् । र॒थ॒न्त॒रमिति॑ रथं-त॒रम् । भ॒व॒ति॒ । इ॒न्द्रि॒यम् । ए॒व । अवेति॑ । रु॒न्ध॒ते॒ । स॒प्त॒द॒शमिति॑ सप्त - द॒शम् । \textbf{  24} \newline
                  \newline
                                \textbf{ TS 7.5.8.3} \newline
                  बृ॒हत् । अ॒न्नाद्य॒स्येत्य॑न्न - अद्य॑स्य । अव॑रुद्ध्या॒ इत्यव॑ - रु॒द्ध्यै॒ । अथो॒ इति॑ । प्रेति॑ । ए॒व । तेन॑ । जा॒य॒न्ते॒ । ए॒क॒विꣳ॒॒शमित्ये॑क - विꣳ॒॒शम् । भ॒द्रम् । द्वि॒पदा॒स्विति॑ द्वि - पदा॑सु । प्रति॑ष्ठित्या॒ इति॒ प्रति॑ - स्थि॒त्यै॒ । पत्न॑यः । उपेति॑ । गा॒य॒न्ति॒ । मि॒थु॒न॒त्वायेति॑ मिथुन - त्वाय॑ । प्रजा᳚त्या॒ इति॒ प्र - जा॒त्यै॒ । प्र॒जाप॑ति॒रिति॑ प्र॒जा - प॒तिः॒ । प्र॒जा इति॑ प्र-जाः । अ॒सृ॒ज॒त॒ । सः । अ॒का॒म॒य॒त॒ । आ॒साम् । अ॒हम् । रा॒ज्यम् । परीति॑ । इ॒या॒म् । इति॑ । तासा᳚म् । रा॒ज॒नेन॑ । ए॒व । रा॒ज्यम् । परीति॑ । ऐ॒त् । तत् । रा॒ज॒नस्य॑ । रा॒ज॒न॒त्वमिति॑ राजन - त्वम् । यत् । रा॒ज॒नम् । भव॑ति । प्र॒जाना॒मिति॑ प्र - जाना᳚म् । ए॒व । तत् । यज॑मानाः । रा॒ज्यम् । परीति॑ । य॒न्ति॒ । प॒ञ्च॒विꣳ॒॒शमिति॑ पञ्च - विꣳ॒॒शम् । भ॒व॒ति॒ । प्र॒जाप॑ते॒रिति॑ प्र॒जा - प॒तेः॒ । \textbf{  25} \newline
                  \newline
                                \textbf{ TS 7.5.8.4} \newline
                  आप्त्यै᳚ । प॒ञ्चभि॒रिति॑ प॒ञ्च - भिः॒ । तिष्ठ॑न्तः । स्तु॒व॒न्ति॒ । दे॒व॒लो॒कमिति॑ देव - लो॒कम् । ए॒व । अ॒भीति॑ । ज॒य॒न्ति॒ । प॒ञ्चभि॒रिति॑ प॒ञ्च - भिः॒ । आसी॑नाः । म॒नु॒ष्य॒लो॒कमिति॑ मनुष्य - लो॒कम् । ए॒व । अ॒भीति॑ । ज॒य॒न्ति॒ । दश॑ । समिति॑ । प॒द्य॒न्ते॒ । दशा᳚क्ष॒रेति॒ दश॑ - अ॒क्ष॒रा॒ । वि॒राडिति॑ वि - राट् । अन्न᳚म् । वि॒राडिति॑ वि - राट् । वि॒राजेति॑ वि - राजा᳚ । ए॒व । अ॒न्नाद्य॒मित्य॑न्न -अद्य᳚म् । अवेति॑ । रु॒न्ध॒ते॒ । प॒ञ्च॒धेति॑ पञ्च - धा । वि॒नि॒षद्येति॑ वि - नि॒षद्य॑ । स्तु॒व॒न्ति॒ । पञ्च॑ । दिशः॑ । दि॒क्षु । ए॒व । प्रतीति॑ । ति॒ष्ठ॒न्ति॒ । एकै॑क॒येत्येक॑या - ए॒क॒या॒ । अस्तु॑तया । स॒माय॒न्तीति॑ सं - आय॑न्ति । दि॒ग्भ्य इति॑ दिक् - भ्यः । ए॒व । अ॒न्नाद्य॒मित्य॑न्न - अद्य᳚म् । समिति॑ । भ॒र॒न्ति॒ । ताभिः॑ । उ॒द्गा॒तेत्यु॑त् - गा॒ता । उदिति॑ । गा॒य॒ति॒ । दि॒ग्भ्य इति॑ दिक् - भ्यः । ए॒व । अ॒न्नाद्य॒मित्य॑न्न - अद्य᳚म् । \textbf{  26} \newline
                  \newline
                                \textbf{ TS 7.5.8.5} \newline
                  स॒भृंत्येति॑ सं - भृत्य॑ । तेजः॑ । आ॒त्मन्न् । द॒ध॒ते॒ । तस्मा᳚त् । एकः॑ । प्रा॒ण इति॑ प्र - अ॒नः । सर्वा॑णि । अङ्गा॑नि । अ॒व॒ति॒ । अथो॒ इति॑ । यथा᳚ । सु॒प॒र्ण इति॑ सु - प॒र्णः । उ॒त्प॒ति॒ष्यन्नित्यु॑त् - प॒ति॒ष्यन्न् । शिरः॑ । उ॒त्त॒ममित्यु॑त् - त॒मम् । कु॒रु॒ते । ए॒वम् । ए॒व । तत् । यज॑मानाः । प्र॒जाना॒मिति॑ प्र - जाना᳚म् । उ॒त्त॒मा इत्यु॑त् - त॒माः । भ॒व॒न्ति॒ । आ॒स॒न्दीमित्या᳚ - स॒न्दीम् । उ॒द्गा॒तेत्यु॑त् - गा॒ता । एति॑ । रो॒ह॒ति॒ । साम्रा᳚ज्य॒मिति॒ सां - रा॒ज्य॒म् । ए॒व । ग॒च्छ॒न्ति॒ । प्ले॒ङ्खम् । होता᳚ । नाक॑स्य । ए॒व । पृ॒ष्ठम् । रो॒ह॒न्ति॒ । कू॒र्चौ । अ॒द्ध्व॒र्युः । ब्र॒द्ध्नस्य॑ । ए॒व । वि॒ष्टप᳚म् । ग॒च्छ॒न्ति॒ । ए॒ताव॑न्तः । वै । दे॒व॒लो॒का इति॑ देव - लो॒काः । तेषु॑ । ए॒व । य॒था॒पू॒र्वमिति॑ यथा - पू॒र्वम् । प्रतीति॑ ( ) । ति॒ष्ठ॒न्ति॒ । अथो॒ इति॑ । आ॒क्रम॑ण॒मित्या᳚ - क्रम॑णम् । ए॒व । तत् । सेतु᳚म् । यज॑मानाः । कु॒र्व॒ते॒ । सु॒व॒र्गस्येति॑ सुवः-गस्य॑ । लो॒कस्य॑ । सम॑ष्ट्या॒ इति॒ सं - अ॒ष्ट्यै॒ ॥ \textbf{  27} \newline
                  \newline
                      (सद॑सः-सप्तद॒शं-प्र॒जाप॑ते-र्गायति दि॒ग्भ्य ए॒वान्नाद्यं॒-प्रत्ये-का॑दश च)  \textbf{(A8)} \newline \newline
                                \textbf{ TS 7.5.9.1} \newline
                  अ॒र्क्ये॑ण । वै । स॒ह॒स्र॒श इति॑ सहस्र - शः । प्र॒जाप॑ति॒रिति॑ प्र॒जा - प॒तिः॒ । प्र॒जा इति॑ प्र - जाः । अ॒सृ॒ज॒त॒ । ताभ्यः॑ । इला᳚न्देन । इरा᳚म् । लूता᳚म् । अवेति॑ । अ॒रु॒न्ध॒ । यत् । अ॒र्क्य᳚म् । भव॑ति । प्र॒जा इति॑ प्र - जाः । ए॒व । तत् । यज॑मानाः । सृ॒ज॒न्ते॒ । इला᳚न्दम् । भ॒व॒ति॒ । प्र॒जाभ्य॒ इति॑ प्र - जाभ्यः॑ । ए॒व । सृ॒ष्टाभ्यः॑ । इरा᳚म् । लूता᳚म् । अवेति॑ । रु॒न्ध॒ते॒ । तस्मा᳚त् । याम् । समा᳚म् । स॒त्रम् । समृ॑द्ध॒मिति॒ सं - ऋ॒द्ध॒म् । क्षोधु॑काः । ताम् । समा᳚म् । प्र॒जा इति॑ प्र - जाः । इष᳚म् । हि । आ॒सा॒म् । ऊर्ज᳚म् । आ॒दद॑त॒ इत्या᳚ - दद॑ते । याम् । समा᳚म् । व्यृ॑द्ध॒मिति॒ वि - ऋ॒द्ध॒म् । अक्षो॑धुकाः । ताम् । समा᳚म् । प्र॒जा इति॑ प्र - जाः । \textbf{  28} \newline
                  \newline
                                \textbf{ TS 7.5.9.2} \newline
                  न । हि । आ॒सा॒म् । इष᳚म् । ऊर्ज᳚म् । आ॒दद॑त॒ इत्या᳚ - दद॑ते । उ॒त्क्रो॒दमित्यु॑त् - क्रो॒दम् । कु॒र्व॒ते॒ । यथा᳚ । ब॒न्धात् । मु॒मु॒चा॒नाः । उ॒त्क्रो॒दमित्यु॑त् - क्रो॒दम् । कु॒र्वते᳚ । ए॒वम् । ए॒व । तत् । यज॑मानाः । दे॒व॒ब॒न्धादिति॑ देव-ब॒न्धात् । मु॒मु॒चा॒नाः । उ॒त्क्रो॒दमित्यु॑त् - क्रो॒दम् । कु॒र्व॒ते॒ । इष᳚म् । ऊर्ज᳚म् । आ॒त्मन्न् । दधा॑नाः । वा॒णः । श॒तत॑न्तु॒रिति॑ श॒त - त॒न्तुः॒ । भ॒व॒ति॒ । श॒तायु॒रिति॑ श॒त - आ॒युः॒ । पुरु॑षः । श॒तेन्द्रि॑य॒ इति॑ श॒त - इ॒न्द्रि॒यः॒ । आयु॑षि । ए॒व । इ॒न्द्रि॒ये । प्रतीति॑ । ति॒ष्ठ॒न्ति॒ । आ॒जिम् । धा॒व॒न्ति॒ । अन॑भिजित॒स्येत्यन॑भि - जि॒त॒स्य॒ । अ॒भिजि॑त्या॒ इत्य॒भि - जि॒त्यै॒ । दु॒न्दु॒भीन् । स॒माघ्न॒न्तीति॑ सं - आघ्न॑न्ति । प॒र॒मा । वै । ए॒षा । वाक् । या । दु॒न्दु॒भौ । प॒र॒माम् । ए॒व । \textbf{  29} \newline
                  \newline
                                \textbf{ TS 7.5.9.3} \newline
                  वाच᳚म् । अवेति॑ । रु॒न्ध॒ते॒ । भू॒मि॒दु॒न्दु॒भिमिति॑ भूमि - दु॒न्दु॒भिम् । एति॑ । घ्न॒न्ति॒ । या । ए॒व । इ॒माम् । वाक् । प्रवि॒ष्टेति॒ प्र - वि॒ष्टा॒ । ताम् । ए॒व । अवेति॑ । रु॒न्ध॒ते॒ । अथो॒ इति॑ । इ॒माम् । ए॒व । ज॒य॒न्ति॒ । सर्वाः᳚ । वाचः॑ । व॒द॒न्ति॒ । सर्वा॑साम् । वा॒चाम् । अव॑रुद्ध्या॒ इत्यव॑ - रु॒द्ध्यै॒ । आ॒र्द्रे । चर्मन्न्॑ । व्याय॑च्छेते॒ इति॑ वि-आय॑च्छेते । इ॒न्द्रि॒यस्य॑ । अव॑रुद्ध्या॒ इत्यव॑ - रु॒द्ध्यै॒ । एति॑ । अ॒न्यः । क्रोश॑ति । प्रेति॑ । अ॒न्यः । शꣳ॒॒स॒ति॒ । यः । आ॒क्रोश॒तीत्या᳚-क्रोश॑ति । पु॒नाति॑ । ए॒व । ए॒ना॒न् । सः । यः । प्र॒शꣳस॒तीति॑ प्र - शꣳस॑ति । पू॒तेषु॑ । ए॒व । अ॒न्नाद्य॒मित्य॑न्न - अद्य᳚म् । द॒धा॒ति॒ । ऋषि॑कृत॒मित्यृषि॑-कृ॒त॒म् । च॒ । \textbf{  30} \newline
                  \newline
                                \textbf{ TS 7.5.9.4} \newline
                  वै । ए॒ते । दे॒वकृ॑त॒मिति॑ दे॒व-कृ॒त॒म् । च॒ । पूर्वैः᳚ । मासैः᳚ । अवेति॑ । रु॒न्ध॒ते॒ । यत् । भू॒ते॒च्छदा॒मिति॑ भूते - छदा᳚म् । सामा॑नि । भव॑न्ति । उ॒भय॑स्य । अव॑रुद्ध्या॒ इत्यव॑ - रु॒द्ध्यै॒ । यन्ति॑ । वै । ए॒ते । मि॒थु॒नात् । ये । सं॒ॅव॒थ्स॒रमिति॑ सं-व॒थ्स॒रम् । उ॒प॒यन्तीत्यु॑प-यन्ति॑ । अ॒न्त॒र्वे॒दीत्य॑न्तः - वे॒दि । मि॒थु॒नौ । समिति॑ । भ॒व॒तः॒ । तेन॑ । ए॒व । मि॒थु॒नात् । न । य॒न्ति॒ ॥ \textbf{  31} \newline
                  \newline
                      (व्यृ॑द्ध॒मक्षो॑धुका॒स्ताꣳ समां᳚ प्र॒जाः - प॑र॒मामे॒व - च॑ - त्रिꣳ॒॒शच्च॑)  \textbf{(A9)} \newline \newline
                                \textbf{ TS 7.5.10.1} \newline
                  चर्म॑ । अवेति॑ । भि॒न्द॒न्ति॒ । पा॒प्मान᳚म् । ए॒व । ए॒षा॒म् । अवेति॑ । भि॒न्द॒न्ति॒ । मा । अपेति॑ । रा॒थ्सीः॒ । मा । अतीति॑ । व्या॒थ्सीः॒ । इति॑ । आ॒ह॒ । स॒प्रं॒तीति॑ सं - प्र॒ति । ए॒व । ए॒षा॒म् । पा॒प्मान᳚म् । अवेति॑ । भि॒न्द॒न्ति॒ । उ॒द॒कु॒म्भानित्यु॑द-कु॒म्भान् । अ॒धि॒नि॒धायेत्य॑धि - नि॒धाय॑ । दा॒स्यः॑ । मा॒र्जा॒लीय᳚म् । परीति॑ । नृ॒त्य॒न्ति॒ । प॒दः । नि॒घ्न॒तीरिति॑ नि - घ्न॒तीः । इ॒दंम॑धु॒मिती॒दं - म॒धु॒म् । गाय॑न्त्यः । मधु॑ । वै । दे॒वाना᳚म् । प॒र॒मम् । अ॒न्नाद्य॒मित्य॑न्न - अद्य᳚म् । प॒र॒मम् । ए॒व । अ॒न्नाद्य॒मित्य॑न्न - अद्य᳚म् । अवेति॑ । रु॒न्ध॒ते॒ । प॒दः । नीति॑ । घ्न॒न्ति॒ । म॒ही॒याम् । ए॒व । ए॒षु॒ । द॒ध॒ति॒ ॥ \textbf{  32} \newline
                  \newline
                      (चर्मै - का॒न्नप॑ञ्चा॒शत्)  \textbf{(A10)} \newline \newline
                                \textbf{ TS 7.5.11.1} \newline
                  पृ॒थि॒व्यै । स्वाहा᳚ । अ॒न्तरि॑क्षाय । स्वाहा᳚ । दि॒वे । स्वाहा᳚ । स॒प्ल्ॐ॒ष्य॒त इति॑ सं - प्लो॒ष्य॒ते । स्वाहा᳚ । स॒प्लंव॑माना॒येति॑ सं - प्लव॑मानाय । स्वाहा᳚ । संप्लु॑ता॒येति॒ सं-प्लु॒ता॒य॒ । स्वाहा᳚ । मे॒घा॒यि॒ष्य॒ते । स्वाहा᳚ । मे॒घा॒य॒त इति॑ मेघ - य॒ते । स्वाहा᳚ । मे॒घि॒ताय॑ । स्वाहा᳚ । मे॒घाय॑ । स्वाहा᳚ । नी॒हा॒राय॑ । स्वाहा᳚ । नि॒हाका॑या॒ इति॑ नि-हाका॑यै । स्वाहा᳚ । प्रा॒स॒चाय॑ । स्वाहा᳚ । प्र॒च॒लाका॑या॒ इति॑ प्र - च॒लाका॑यै । स्वाहा᳚ । वि॒द्यो॒ति॒ष्य॒त इति॑ वि - द्यो॒ति॒ष्य॒ते । स्वाहा᳚ । वि॒द्योत॑माना॒येति॑ वि - द्योत॑मानाय । स्वाहा᳚ । सं॒ॅवि॒द्योत॑माना॒येति॑ सं-वि॒द्योत॑मानाय । स्वाहा᳚ । स्त॒न॒यि॒ष्य॒ते । स्वाहा᳚ । स्त॒नय॑ते । स्वाहा᳚ । उ॒ग्रम् । स्त॒नय॑ते । स्वाहा᳚ । व॒र्.॒षि॒ष्य॒ते । स्वाहा᳚ । वर्.ष॑ते । स्वाहा᳚ । अ॒भि॒वर्.ष॑त॒ इत्य॑भि - वर्.ष॑ते । स्वाहा᳚ । प॒रि॒वर्.ष॑त॒ इति॑ परि - वर्.ष॑ते । स्वाहा᳚ । सं॒ॅवर्.ष॑त॒ इति॑ सं - वर्.ष॑ते । \textbf{  33} \newline
                  \newline
                                \textbf{ TS 7.5.11.2} \newline
                  स्वाहा᳚ । अ॒नु॒वर्.ष॑त॒ इत्य॑नु - वर्.ष॑ते । स्वाहा᳚ । शी॒का॒यि॒ष्य॒ते । स्वाहा᳚ । शी॒का॒य॒त इति॑ शीक - य॒ते । स्वाहा᳚ । शी॒कि॒ताय॑ । स्वाहा᳚ । प्रो॒षि॒ष्य॒ते । स्वाहा᳚ । प्रु॒ष्ण॒ते । स्वाहा᳚ । प॒रि॒प्रु॒ष्ण॒त इति॑ परि - प्रु॒ष्ण॒ते । स्वाहा᳚ । उ॒द्ग्र॒ही॒ष्य॒त इत्यु॑त् - ग्र॒ही॒ष्य॒ते । स्वाहा᳚ । उ॒द्गृ॒ह्ण॒त इत्यु॑त् - गृ॒ह्ण॒ते । स्वाहा᳚ । उद्गृ॑हीता॒येत्युत् -   गृ॒ही॒ता॒य॒ । स्वाहा᳚ । वि॒प्लो॒ष्य॒त इति॑ वि - प्लो॒ष्य॒ते । स्वाहा᳚ । वि॒प्लव॑माना॒येति॑ वि - प्लव॑मानाय । स्वाहा᳚ । विप्लु॑ता॒येति॒ वि - प्लु॒ता॒य॒ । स्वाहा᳚ । आ॒त॒फ्स्य॒त इत्या᳚ - त॒फ्स्य॒ते । स्वाहा᳚ । आ॒तप॑त॒ इत्या᳚ - तप॑ते । स्वाहा᳚ । उ॒ग्रम् । आ॒तप॑त॒ इत्या᳚ - तप॑ते । स्वाहा᳚ । ऋ॒ग्भ्य इत्यृ॑क् - भ्यः । स्वाहा᳚ । यजु॑र्भ्य॒ इति॒ यजुः॑ - भ्यः॒ । स्वाहा᳚ । साम॑भ्य॒ इति॒ साम॑ - भ्यः॒ । स्वाहा᳚ । अङ्गि॑रोभ्य॒ इत्यङ्गि॑रः -भ्यः॒ । स्वाहा᳚ । वेदे᳚भ्यः । स्वाहा᳚ । गाथा᳚भ्यः । स्वाहा᳚ । ना॒रा॒शꣳ॒॒सीभ्यः॑ । स्वाहा᳚ । रैभी᳚भ्यः । स्वाहा॑ ( ) । सर्व॑स्मै । स्वाहा᳚ ॥ \textbf{  34} \newline
                  \newline
                      (सं॒ॅवर्.ष॑ते॒ - रैभी᳚भ्यः॒ स्वाहा॒ - द्वे च॑)  \textbf{(A11)} \newline \newline
                                \textbf{ TS 7.5.12.1} \newline
                  द॒त्वते᳚ । स्वाहा᳚ । अ॒द॒न्तका॑य । स्वाहा᳚ । प्रा॒णिने᳚ । स्वाहा᳚ । अ॒प्रा॒णाय॑ । स्वाहा᳚ । मुख॑वत॒ इति॒ मुख॑-व॒ते॒ । स्वाहा᳚ । अ॒मु॒खाय॑ । स्वाहा᳚ । नासि॑कवत॒ इति॒ नासि॑क - व॒ते॒ । स्वाहा᳚ । अ॒ना॒सि॒काय॑ । स्वाहा᳚ । अ॒क्ष॒ण्वत॒ इत्य॑क्षण्-वते᳚ । स्वाहा᳚ । अ॒न॒क्षिका॑य । स्वाहा᳚ । क॒र्णिने᳚ । स्वाहा᳚ । अ॒क॒र्णका॑य । स्वाहा᳚ । शी॒र्.॒ष॒ण्वत॒ इति॑ शीर्.षण् - वते᳚ । स्वाहा᳚ । अ॒शी॒र्॒.षका॑य । स्वाहा᳚ । प॒द्वत॒ इति॑ पत् - वते᳚ । स्वाहा᳚ । अ॒पा॒दका॑य । स्वाहा᳚ । प्रा॒ण॒त इति॑ प्र-अ॒न॒ते । स्वाहा᳚ । अप्रा॑णत॒ इत्यप्र॑ - अ॒न॒ते॒ । स्वाहा᳚ । वद॑ते । स्वाहा᳚ । अव॑दते । स्वाहा᳚ । पश्य॑ते । स्वाहा᳚ । अप॑श्यते । स्वाहा᳚ । शृ॒ण्व॒ते । स्वाहा᳚ । अशृ॑ण्वते । स्वाहा᳚ । म॒न॒स्विने᳚ । स्वाहा᳚ । \textbf{  35} \newline
                  \newline
                                \textbf{ TS 7.5.12.2} \newline
                  अ॒म॒नसे᳚ । स्वाहा᳚ । रे॒त॒स्विने᳚ । स्वाहा᳚ । अ॒रे॒तस्का॒येत्य॑रे॒तः - का॒य॒ । स्वाहा᳚ । प्र॒जाभ्य॒ इति॑ प्र - जाभ्यः॑ । स्वाहा᳚ । प्र॒जन॑ना॒येति॑ प्र - जन॑नाय । स्वाहा᳚ । लोम॑वत॒ इति॒ लोम॑ - व॒ते॒ । स्वाहा᳚ । अ॒लो॒मका॑य । स्वाहा᳚ । त्व॒चे । स्वाहा᳚ । अ॒त्वक्का॑य । स्वाहा᳚ । चर्म॑ण्वत॒ इति॒ चर्मण्॑ -   व॒ते॒ । स्वाहा᳚ । अ॒च॒र्मका॑य । स्वाहा᳚ । लोहि॑तवत॒ इति॒ लोहि॑त - व॒ते॒ । स्वाहा᳚ । अ॒लो॒हि॒ताय॑ । स्वाहा᳚ । माꣳ॒॒स॒न्वत॒ इति॑ माꣳसन्न्-वते᳚ । स्वाहा᳚ । अ॒माꣳ॒॒सका॑य । स्वाहा᳚ । स्नाव॑भ्य॒ इति॒ स्नाव॑ - भ्यः॒ । स्वाहा᳚ । अ॒स्ना॒वका॑य । स्वाहा᳚ । अ॒स्थ॒न्वत॒ इत्य॑स्थन्न् - वते᳚ । स्वाहा᳚ । अ॒न॒स्थिका॑य । स्वाहा᳚ । म॒ज्ज॒न्वत॒ इति॑ मज्जन्न् - वते᳚ । स्वाहा᳚ । अ॒म॒ज्जका॑य । स्वाहा᳚ । अ॒ङ्गिने᳚ । स्वाहा᳚ । अ॒न॒ङ्गाय॑ । स्वाहा᳚ । आ॒त्मने᳚ । स्वाहा᳚ । अना᳚त्मने । स्वाहा᳚ ( ) । सर्व॑स्मै । स्वाहा᳚ ॥ \textbf{  36} \newline
                  \newline
                      (म॒न॒स्विने॒ स्वाहा - ऽना᳚त्मने॒ स्वाहा॒ - द्वे च॑)  \textbf{(A12)} \newline \newline
                                \textbf{ TS 7.5.13.1} \newline
                  कः । त्वा॒ । यु॒न॒क्ति॒ । सः । त्वा॒ । यु॒न॒क्तु॒ । विष्णुः॑ । त्वा॒ । यु॒न॒क्तु॒ । अ॒स्य । य॒ज्ञ्स्य॑ । ऋद्ध्यै᳚ । मह्य᳚म् । सन्न॑त्या॒ इति॒ सं - न॒त्यै॒ । अ॒मुष्मै᳚ । कामा॑य । आयु॑षे । त्वा॒ । प्रा॒णायेति॑ प्र - अ॒नाय॑ । त्वा॒ । अ॒पा॒नायेत्य॑प - अ॒नाय॑ । त्वा॒ । व्या॒नायेति॑ वि - अ॒नाय॑ । त्वा॒ । व्यु॑ष्ट्या॒ इति॒ वि - उ॒ष्ट्यै॒ । त्वा॒ । र॒य्यै । त्वा॒ । राध॑से । त्वा॒ । घोषा॑य । त्वा॒ । पोषा॑य । त्वा॒ । आ॒रा॒द्घो॒षायेत्या॑रात् - घो॒षाय॑ । त्वा॒ । प्रच्यु॑त्या॒ इति॒ प्र - च्यु॒त्यै॒ । त्वा॒ ॥ \textbf{  37} \newline
                  \newline
                      (कस्त्वा॒ - ऽष्टात्रिꣳ॑शत्)  \textbf{(A13)} \newline \newline
                                \textbf{ TS 7.5.14.1} \newline
                  अ॒ग्नये᳚ । गा॒य॒त्राय॑ । त्रि॒वृत॒ इति॑ त्रि-वृते᳚ । राथ॑न्तरा॒येति॒ राथं᳚-त॒रा॒य॒ । वा॒स॒न्ताय॑ । अ॒ष्टाक॑पाल॒ इत्य॒ष्टा - क॒पा॒लः॒ । इन्द्रा॑य । त्रैष्टु॑भाय । प॒ञ्च॒द॒शायेति॑ पञ्च - द॒शाय॑ । बार्.ह॑ताय । ग्रैष्मा॑य । एका॑दशकपाल॒ इत्येका॑दश - क॒पा॒लः॒ । विश्वे᳚भ्यः । दे॒वेभ्यः॑ । जाग॑तेभ्यः । स॒प्त॒द॒शेभ्य॒ इति॑ सप्त - द॒शेभ्यः॑ । वै॒रू॒पेभ्यः॑ । वार्.षि॑केभ्यः । द्वाद॑शकपाल॒ इति॒ द्वाद॑श - क॒पा॒लः॒ । मि॒त्रावरु॑णाभ्या॒मिति॑ मि॒त्रा - वरु॑णाभ्याम् । आनु॑ष्टुभाभ्या॒मित्यानु॑- स्तु॒भा॒भ्या॒म् । ए॒क॒विꣳ॒॒शाभ्या॒मित्ये॑क-विꣳ॒॒शाभ्या᳚म् । वै॒रा॒जाभ्या᳚म् । शा॒र॒दाभ्या᳚म् । प॒य॒स्या᳚ । बृह॒स्पत॑ये । पाङ्क्ता॑य । त्रि॒ण॒वायेति॑ त्रि - न॒वाय॑ । शा॒क्व॒राय॑ । हैम॑न्तिकाय । च॒रुः । स॒वि॒त्रे । आ॒ति॒च्छ॒न्द॒सायेत्या॑ति - छ॒न्द॒साय॑ । त्र॒य॒स्त्रिꣳ॒॒शायेति॑ त्रयः - त्रिꣳ॒॒शाय॑ । रै॒व॒ताय॑ । शै॒शि॒राय॑ । द्वाद॑शकपाल॒ इति॒ द्वाद॑श -क॒पा॒लः॒ । अदि॑त्यै । विष्णु॑पत्न्या॒ इति॒ विष्णु॑ - प॒त्न्यै॒ । च॒रुः । अ॒ग्नये᳚ । वै॒श्वा॒न॒राय॑ । द्वाद॑शकपाल॒ इति॒ द्वाद॑श - क॒पा॒लः॒ । अनु॑मत्या॒ इत्यनु॑ - म॒त्यै॒ । च॒रुः । का॒यः । एक॑कपाल॒ इत्येक॑-क॒पा॒लः॒ ॥ \textbf{  38} \newline
                  \newline
                      (अ॒ग्नयेऽदि॑त्या॒ अनु॑मत्यै - स॒प्तच॑त्वारिꣳशत्)  \textbf{(A14)} \newline \newline
                                \textbf{ TS 7.5.15.1} \newline
                  यः । वै । अ॒ग्नौ । अ॒ग्निः । प्र॒ह्रि॒यत॒ इति॑ प्र - ह्रि॒यते᳚ । यः । च॒ । सोमः॑ । राजा᳚ । तयोः᳚ । ए॒षः । आ॒ति॒थ्यम् । यत् । अ॒ग्नी॒षो॒मीय॒ इत्य॑ग्नी - सो॒मीयः॑ । अथ॑ । ए॒षः । रु॒द्रः । यः । ची॒यते᳚ । यत् । सञ्चि॑त॒ इति॒ सं - चि॒ते॒ । अ॒ग्नौ । ए॒तानि॑ । ह॒वीꣳषि॑ । न । नि॒र्वपे॒दिति॑ निः - वपे᳚त् । ए॒षः । ए॒व । रु॒द्रः । अशा᳚न्तः । उ॒पो॒त्थायेत्यु॑प - उ॒त्थाय॑ । प्र॒जामिति॑ प्र - जाम् । प॒शून् । यज॑मानस्य । अ॒भीति॑ । म॒न्ये॒त॒ । यत् । सञ्चि॑त॒ इति॒ सं-चि॒ते॒ । अ॒ग्नौ । ए॒तानि॑ । ह॒वीꣳषि॑ । नि॒र्वप॒तीति॑ निः - वप॑ति । भा॒ग॒धेये॒नेति॑ भाग - धेये॑न । ए॒व । ए॒न॒म् । श॒म॒य॒ति॒ । न । अ॒स्य॒ । रु॒द्रः । अशा᳚न्तः । \textbf{  39} \newline
                  \newline
                                \textbf{ TS 7.5.15.2} \newline
                  उ॒पो॒त्थायेत्यु॑प - उ॒त्थाय॑ । प्र॒जामिति॑ प्र-जाम् । प॒शून् । अ॒भीति॑ । म॒न्य॒ते॒ । दश॑ । ह॒वीꣳषि॑ । भ॒व॒न्ति॒ । नव॑ । वै । पुरु॑षे । प्रा॒णा इति॑ प्र - अ॒नाः । नाभिः॑ । द॒श॒मी । प्रा॒णानिति॑ प्र - अ॒नान् । ए॒व । यज॑माने । द॒धा॒ति॒ । अथो॒ इति॑ । दशा᳚क्ष॒रेति॒ दश॑ - अ॒क्ष॒रा॒ । वि॒राडिति॑ वि - राट् । अन्न᳚म् । वि॒राडिति॑ वि - राट् । वि॒राजीति॑ वि - राजि॑ । ए॒व । अ॒न्नाद्य॒ इत्य॑न्न - अद्ये᳚ । प्रतीति॑ । ति॒ष्ठ॒ति॒ । ऋ॒तुभि॒रित्यृ॒तु - भिः॒ । वै । ए॒षः । छन्दो॑भि॒रिति॒ छन्दः॑- भिः॒ । स्तोमैः᳚ । पृ॒ष्ठैः । चे॒त॒व्यः॑ । इति॑ । आ॒हुः॒ । यत् । ए॒तानि॑ । ह॒वीꣳषि॑ । नि॒र्वप॒तीति॑ निः - वप॑ति । ऋ॒तुभि॒रित्यृ॒तु - भिः॒ । ए॒व । ए॒न॒म् । छन्दो॑भि॒रिति॒ छन्दः॑ - भिः॒ । स्तोमैः᳚ । पृ॒ष्ठैः । चि॒नु॒ते॒ । दिशः॑ । सु॒षु॒वा॒णेन॑ । \textbf{  40} \newline
                  \newline
                                \textbf{ TS 7.5.15.3} \newline
                  अ॒भि॒जित्या॒ इत्य॑भि - जित्याः᳚ । इति॑ । आ॒हुः॒ । यत् । ए॒तानि॑ । ह॒वीꣳषि॑ । नि॒र्वप॒तीति॑ निः - वप॑ति । दि॒शाम् । अ॒भिजि॑त्या॒ इत्य॒भि - जि॒त्यै॒ । ए॒तया᳚ । वै । इन्द्र᳚म् । दे॒वाः । अ॒या॒ज॒य॒न्न् । तस्मा᳚त् । इ॒न्द्र॒स॒व इती᳚न्द्र - स॒वः । ए॒तया᳚ । मनु᳚म् । म॒नु॒ष्याः᳚ । तस्मा᳚त् । म॒नु॒स॒व इति॑ मनु - स॒वः । यथा᳚ । इन्द्रः॑ । दे॒वाना᳚म् । यथा᳚ । मनुः॑ । म॒नु॒ष्या॑णाम् । ए॒वम् । भ॒व॒ति॒ । यः । ए॒वम् । वि॒द्वान् । ए॒तया᳚ । इष्ट्या᳚ । यज॑ते । दिग्व॑ती॒रिति॒ दिक्-व॒तीः॒ । पु॒रो॒ऽनु॒वा॒क्या॑ इति॑ पुरः-अ॒नु॒वा॒क्याः᳚ । भ॒व॒न्ति॒ । सर्वा॑साम् । दि॒शाम् । अ॒भिजि॑त्या॒ इत्य॒भि - जि॒त्यै॒ ॥ \textbf{  41} \newline
                  \newline
                      (अशा᳚न्तः - सुषुवा॒णेनै - क॑चत्वारिꣳशच्च)  \textbf{(A15)} \newline \newline
                                \textbf{ TS 7.5.16.1} \newline
                  यः । प्रा॒ण॒त इति॑ प्र - अ॒न॒तः । नि॒मि॒ष॒त इति॑ नि - मि॒ष॒तः । म॒हि॒त्वेति॑ महि - त्वा । एकः॑ । इत् । राजा᳚ । जग॑तः । ब॒भूव॑ ॥ यः । ईशे᳚ । अ॒स्य । द्वि॒पद॒ इति॑ द्वि - पदः॑ । चतु॑ष्पद॒ इति॒ चतुः॑ - प॒दः॒ । कस्मै᳚ । दे॒वाय॑ । ह॒विषा᳚ । वि॒धे॒म॒ ॥ उ॒प॒या॒मगृ॑हीत॒ इत्यु॑पया॒म-गृ॒ही॒तः॒ । अ॒सि॒ । प्र॒जाप॑तय॒ इति॑ प्र॒जा - प॒त॒ये॒ । त्वा॒ । जुष्ट᳚म् । गृ॒ह्णा॒मि॒ । तस्य॑ । ते॒ । द्यौः । म॒हि॒मा । नक्ष॑त्राणि । रू॒पम् । आ॒दि॒त्यः । ते॒ । तेजः॑ । तस्मै᳚ । त्वा॒ । म॒हि॒म्ने । प्र॒जाप॑तय॒ इति॑ प्र॒जा - प॒त॒ये॒ । स्वाहा᳚ ॥ \textbf{  42} \newline
                  \newline
                      (यः प्रा॑ण॒तो द्यौरा॑दि॒त्यो᳚ - ऽष्टात्रिꣳ॑शत् )  \textbf{(A16)} \newline \newline
                                \textbf{ TS 7.5.17.1} \newline
                  यः । आ॒त्म॒दा इत्या᳚त्म - दाः । ब॒ल॒दा इति॑ बल - दाः । यस्य॑ । विश्वे᳚ । उ॒पास॑त॒ इत्यु॑प - आस॑ते । प्र॒शिष॒मिति॑ प्र-शिष᳚म् । यस्य॑ । दे॒वाः ॥ यस्य॑ । छा॒या । अ॒मृत᳚म् । यस्य॑ । मृ॒त्युः । कस्मै᳚ । दे॒वाय॑ । ह॒विषा᳚ । वि॒धे॒म॒ ॥ उ॒प॒या॒मगृ॑हीत॒ इत्यु॑पया॒म - गृ॒ही॒तः॒ । अ॒सि॒ । प्र॒जाप॑तय॒ इति॑ प्र॒जा - प॒त॒ये॒ । त्वा॒ । जुष्ट᳚म् । गृ॒ह्णा॒मि॒ । तस्य॑ । ते॒ । पृ॒थि॒वी । म॒हि॒मा । ओष॑धयः । वन॒स्पत॑यः । रू॒पम् । अ॒ग्निः । ते॒ । तेजः॑ । तस्मै᳚ । त्वा॒ । म॒हि॒म्ने । प्र॒जाप॑तय॒ इति॑ प्र॒जा - प॒त॒ये॒ । स्वाहा᳚ ॥ \textbf{  43 } \newline
                  \newline
                      (य आ᳚त्म॒दाः पृ॑थि॒व्य॑ग्नि-रेका॒न्नच॑त्वारिꣳ॒॒शत्)  \textbf{(A17)} \newline \newline
                                \textbf{ TS 7.5.18.1} \newline
                  एति॑ । ब्रह्मन्न्॑ । ब्रा॒ह्म॒णः । ब्र॒ह्म॒व॒र्च॒सीति॑ ब्रह्म-व॒र्च॒सी । जा॒य॒ता॒म् । एति॑ । अ॒स्मिन्न् । रा॒ष्ट्रे । रा॒ज॒न्यः॑ । इ॒ष॒व्यः॑ । शूरः॑ । म॒हा॒र॒थ इति॑ महा - र॒थः । जा॒य॒ता॒म् । दोग्ध्री᳚ । धे॒नुः । वोढा᳚ । अ॒न॒ड्वान् । आ॒शुः । सप्तिः॑ । पुर॑न्धिः । योषा᳚ । जि॒ष्णूः । र॒थे॒ष्ठा इति॑ रथे - स्थाः । स॒भेयः॑ । युवा᳚ । एति॑ । अ॒स्य । यज॑मानस्य । वी॒रः । जा॒य॒ता॒म् । नि॒का॒मेनि॑काम॒ इति॑ निका॒मे - नि॒का॒मे॒ । नः॒ । प॒र्जन्यः॑ । व॒र्.॒ष॒तु॒ । फ॒लिन्यः॑ । नः॒ । ओष॑धयः । प॒च्य॒न्ता॒म् । यो॒ग॒क्षे॒म इति॑ योग - क्षे॒मः । नः॒ । क॒ल्प॒ता॒म् ॥ \textbf{  44} \newline
                  \newline
                      (आ ब्रह्म॒ - न्नेक॑चत्वारिꣳशत् ) \textbf{(A18)} \newline \newline
                                \textbf{ TS 7.5.19.1} \newline
                  एति॑ । अ॒क्रा॒न् । वा॒जी । पृ॒थि॒वीम् । अ॒ग्निम् । युज᳚म् । अ॒कृ॒त॒ । वा॒जी । अर्वा᳚ । एति॑ । अ॒क्रा॒न् । वा॒जी । अ॒न्तरि॑क्षम् । वा॒युम् । युज᳚म् । अ॒कृ॒त॒ । वा॒जी । अर्वा᳚ । द्याम् । वा॒जी । एति॑ । अ॒क्रꣳ॒॒स्त॒ । सूर्य᳚म् । युज᳚म् । अ॒कृ॒त॒ । वा॒जी । अर्वा᳚ । अ॒ग्निः । ते॒ । वा॒जि॒न्न् । युङ् । अन्विति॑ । त्वा॒ । एति॑ । र॒भे॒ । स्व॒स्ति । मा॒ । समिति॑ । पा॒र॒य॒ । वा॒युः । ते॒ । वा॒जि॒न्न् । युङ् । अन्विति॑ । त्वा॒ । एति॑ । र॒भे॒ । स्व॒स्ति । मा॒ । समिति॑ । \textbf{  45} \newline
                  \newline
                                \textbf{ TS 7.5.19.2} \newline
                  पा॒र॒य॒ । आ॒दि॒त्यः । ते॒ । वा॒जि॒न्न् । युङ् । अन्विति॑ । त्वा॒ । एति॑ । र॒भे॒ । स्व॒स्ति । मा॒ । समिति॑ । पा॒र॒य॒ । प्रा॒ण॒धृगिति॑ प्राण - धृक् । अ॒सि॒ । प्रा॒णमिति॑ प्र - अ॒नम् । मे॒ । दृꣳ॒॒ह॒ । व्या॒न॒धृगिति॑ व्यान - धृक् । अ॒सि॒ । व्या॒नमिति॑ वि - अ॒नम् । मे॒ । दृꣳ॒॒ह॒ । अ॒पा॒न॒धृगित्य॑पान - धृक् । अ॒सि॒ । अ॒पा॒नमित्य॑प - अ॒नम् । मे॒ । दृꣳ॒॒ह॒ । चक्षुः॑ । अ॒सि॒ । चक्षुः॑ । मयि॑ । धे॒हि॒ । श्रोत्र᳚म् । अ॒सि॒ । श्रोत्र᳚म् । मयि॑ । धे॒हि॒ । आयुः॑ । अ॒सि॒ । आयुः॑ । मयि॑ । धे॒हि॒ ॥ \textbf{  46} \newline
                  \newline
                      (वा॒युस्ते॑ वाजि॒न्॒ युङ्ङनु॒ त्वा ऽऽ र॑भे स्व॒स्ति मा॒ सं - त्रिच॑त्वारिꣳशच्च)  \textbf{(A19)} \newline \newline
                                \textbf{ TS 7.5.20.1} \newline
                  जज्ञि॑ । बीज᳚म् । वर्ष्टा᳚ । प॒र्जन्यः॑ । पक्ता᳚ । स॒स्यम् । सु॒पि॒प्प॒ला इति॑ सु - पि॒प्प॒लाः । ओष॑धयः । स्व॒धि॒च॒र॒णेति॑ सु - अ॒धि॒च॒र॒णा । इ॒यम् । सू॒प॒स॒द॒न इति॑ सु - उ॒प॒स॒द॒नः । अ॒ग्निः । स्व॒द्ध्य॒क्षमिति॑ सु - अ॒द्ध्य॒क्षम् । अ॒न्तरि॑क्षम् । सु॒पा॒व इति॑ सु-पा॒वः । पव॑मानः । सू॒प॒स्था॒नेति॑ सु - उ॒प॒स्था॒ना । द्यौः । शि॒वम् । अ॒सौ । तपन्न्॑ । य॒था॒पू॒र्वमिति॑ यथा - पू॒र्वम् । अ॒हो॒रा॒त्रे इत्य॑हः - रा॒त्रे । प॒ञ्च॒द॒शिन॒ इति॑ पञ्च - द॒शिनः॑ । अ॒द्‌र्ध॒मा॒सा इत्य॑द्‌र्ध - मा॒साः । त्रिꣳ॒॒शिनः॑ । मासाः᳚ । क्लृ॒प्ताः । ऋ॒तवः॑ । शा॒न्तः । सं॒ॅव॒थ्स॒र इति॑ सं - व॒थ्स॒रः ॥ \textbf{  47} \newline
                  \newline
                      (जज्ञि॒ बीज॒ - मेक॑त्रिꣳशत्)  \textbf{(A20)} \newline \newline
                                \textbf{ TS 7.5.21.1} \newline
                  आ॒ग्ने॒यः । अ॒ष्टाक॑पाल॒ इत्य॒ष्टा - क॒पा॒लः॒ । सौ॒म्यः । च॒रुः । सा॒वि॒त्रः । अ॒ष्टाक॑पाल॒ इत्य॒ष्टा - क॒पा॒लः॒ । पौ॒ष्णः । च॒रुः । रौ॒द्रः । च॒रुः । अ॒ग्नये᳚ । वै॒श्वा॒न॒राय॑ । द्वाद॑शकपाल॒ इति॒ द्वाद॑श - क॒पा॒लः॒ । मृ॒गा॒ख॒र इति॑ मृग - आ॒ख॒रे । यदि॑ । न । आ॒गच्छे॒दित्या᳚-गच्छे᳚त् । अ॒ग्नये᳚ । अꣳ॒॒हो॒मुच॒ इत्यꣳ॑हः-मुचे᳚ । अ॒ष्टाक॑पाल॒ इत्य॒ष्टा-क॒पा॒लः॒ । सौ॒र्यम् । पयः॑ । वा॒य॒व्यः॑ । आज्य॑भाग॒ इत्याज्य॑ - भा॒गः॒ ॥ \textbf{  48} \newline
                  \newline
                      (आ॒ग्ने॒य - श्चतु॑र्विꣳशतिः)  \textbf{(A21)} \newline \newline
                                \textbf{ TS 7.5.22.1} \newline
                  अ॒ग्नये᳚ । अꣳ॒॒हो॒मुच॒ इत्यꣳ॑हः-मुचे᳚ । अष्टाक॑पाल॒ इत्य॒ष्टा-क॒पा॒लः॒ । इन्द्रा॑य । अꣳ॒॒हो॒मुच॒ इत्यꣳ॑हः - मुचे᳚ । एका॑दशकपाल॒ इत्येका॑दश - क॒पा॒लः॒ । मि॒त्रावरु॑णाभ्या॒मिति॑ मि॒त्रा - वरु॑णाभ्याम् । आ॒गो॒मुग्भ्या॒मित्या॑गो॒मुक् - भ्या॒म् । प॒य॒स्या᳚ । वा॒यो॒सा॒वि॒त्र इति॑ वायो - सा॒वि॒त्रः । आ॒गो॒मुग्भ्या॒मित्या॑गो॒मुक् - भ्या॒म् । च॒रुः । अ॒श्विभ्या॒मित्य॒श्वि - भ्या॒म् । अ॒गो॒मुग्भ्या॒मित्या॑गो॒मुक् - भ्या॒म् । धा॒नाः । म॒रुद्भ्य॒ इति॑ म॒रुत् - भ्यः॒ । ए॒नो॒मुग्भ्य॒ इत्ये॑नो॒मुक्-भ्यः॒ । स॒प्तक॑पाल॒ इति॑ स॒प्त - क॒पा॒लः॒ । विश्वे᳚भ्यः । दे॒वेभ्यः॑ । ए॒नो॒मुग्भ्य॒ इत्ये॑नो॒मुक् - भ्यः॒ । द्वाद॑शकपाल॒ इति॒ द्वाद॑श - क॒पा॒लः॒ । अनु॑मत्या॒ इत्यनु॑ - म॒त्यै॒ । च॒रुः । अ॒ग्नये᳚ । वै॒श्वा॒न॒राय॑ । द्वाद॑शकपाल॒ इति॒ द्वाद॑श - क॒पा॒लः॒ । द्यावा॑पृथि॒वीभ्या॒मिति॒ द्यावा᳚ - पृ॒थि॒वीभ्या᳚म् । अꣳ॒॒हो॒मुग्भ्या॒मित्यꣳ॑हो॒मुक् - भ्या॒म् । द्वि॒क॒पा॒ल इति॑ द्वि - क॒पा॒लः ॥ \textbf{  49} \newline
                  \newline
                      (अ॒ग्नयेऽꣳ॑हो॒मुचे᳚ - त्रिꣳ॒॒शत्)  \textbf{(A22)} \newline \newline
                                \textbf{ TS 7.5.23.1} \newline
                  अ॒ग्नये᳚ । समिति॑ । अ॒न॒म॒त् । पृ॒थि॒व्यै । समिति॑ । अ॒न॒म॒त् । यथा᳚ । अ॒ग्निः । पृ॒थि॒व्या । स॒मन॑म॒दिति॑ सं - अन॑मत् । ए॒वम् । मह्य᳚म् । भ॒द्राः । संन॑तय॒ इति॒ सं - न॒त॒यः॒ । समिति॑ । न॒म॒न्तु॒ । वा॒यवे᳚ । समिति॑ । अ॒न॒म॒त् । अ॒न्तरि॑क्षाय । समिति॑ । अ॒न॒म॒त् । यथा᳚ । वा॒युः । अ॒न्तरि॑क्षेण । सूर्या॑य । समिति॑ । अ॒न॒म॒त् । दि॒वे । समिति॑ । अ॒न॒म॒त् । यथा᳚ । सूर्यः॑ । दि॒वा । च॒न्द्रम॑से । समिति॑ । अ॒न॒म॒त् । नक्ष॑त्रेभ्यः । समिति॑ । अ॒न॒म॒त् । यथा᳚ । च॒न्द्रमाः᳚ । नक्ष॑त्रैः । वरु॑णाय । समिति॑ । अ॒न॒म॒त् । अ॒द्भ्य इत्य॑त् - भ्यः । समिति॑ । अ॒न॒म॒त्॒ । यथा᳚ । \textbf{  50} \newline
                  \newline
                                \textbf{ TS 7.5.23.2} \newline
                  वरु॑णः । अ॒द्भिरित्य॑त् - भिः । साम्ने᳚ । समिति॑ । अ॒न॒म॒त् । ऋ॒चे । समिति॑ । अ॒न॒म॒त् । यथा᳚ । साम॑ । ऋ॒चा । ब्रह्म॑णे । समिति॑ । अ॒न॒म॒त् । क्ष॒त्राय॑ । समिति॑ । अ॒न॒म॒त् । यथा᳚ । ब्रह्म॑ । क्ष॒त्रेण॑ । राज्ञे᳚ । समिति॑ । अ॒न॒म॒त् । वि॒शे । समिति॑ । अ॒न॒म॒त् । यथा᳚ । राजा᳚ । वि॒शा । रथा॑य । समिति॑ । अ॒न॒म॒त् । अश्वे᳚भ्यः । समिति॑ । अ॒न॒म॒त् । यथा᳚ । रथः॑ । अश्वैः᳚ । प्र॒जाप॑तय॒ इति॑ प्र॒जा - प॒त॒ये॒ । समिति॑ । अ॒न॒म॒त् । भू॒तेभ्यः॑ । समिति॑ । अ॒न॒म॒त् । यथा᳚ । प्र॒जाप॑ति॒रिति॑ प्र॒जा - प॒तिः॒ । भू॒तैः । स॒मन॑म॒दिति॑ सं-अन॑मत् । ए॒वम् । मह्य᳚म् ( ) । भ॒द्राः । संन॑तय॒ इति॒ सं - न॒त॒यः॒ । समिति॑ । न॒म॒न्तु॒ ॥ \textbf{  51} \newline
                  \newline
                      (अ॒द्भ्यः सम॑नम॒द्यथा॒-मह्यं॑-च॒त्वारि॑ च)  \textbf{(A23)} \newline \newline
                                \textbf{ TS 7.5.24.1} \newline
                  ये । त॒ । पन्था॑नः । स॒वि॒तः॒ । पू॒र्व्यासः॑ । अ॒रे॒णवः॑ । वित॑ता॒ इति॒ वि - त॒ताः॒ । अ॒न्तरि॑क्षे ॥ तेभिः॑ । नः॒ । अ॒द्य । प॒थिभि॒रिति॑ प॒थि - भिः॒ । सु॒गेभि॒रिति॑ सु - गेभिः॑ । रक्ष॑ । च॒ । नः॒ । अधीति॑ । च॒ । दे॒व॒ । ब्रू॒हि॒ ॥ नमः॑ । अ॒ग्नये᳚ । पृ॒थि॒वि॒क्षित॒ इति॑ पृथिवि-क्षिते᳚ । लो॒क॒स्पृत॒ इति॑ लोक - स्पृते᳚ । लो॒कम् । अ॒स्मै । यज॑मानाय । दे॒हि॒ । नमः॑ । वा॒यवे᳚ । अ॒न्त॒रि॒क्ष॒क्षित॒ इत्य॑न्तरिक्ष - क्षिते᳚ । लो॒क॒स्पृत॒ इति॑ लोक - स्पृते᳚ । लो॒कम् । अ॒स्मै । यज॑मानाय । दे॒हि॒ । नमः॑ । सूर्या॑य । दि॒वि॒क्षित॒ इति॑ दिवि - क्षिते᳚ । लो॒क॒स्पृत॒ इति॑ लोक - स्पृते᳚ । लो॒कम् । अ॒स्मै । यज॑मानाय । दे॒हि॒ ॥ \textbf{  52} \newline
                  \newline
                      (ये ते॒ - चतु॑श्चत्वारिꣳशत्)  \textbf{(A24)} \newline \newline
                                \textbf{ TS 7.5.25.1} \newline
                  यः । वै । अश्व॑स्य । मेद्ध्य॑स्य । शिरः॑ । वेद॑ । शी॒र्.॒ष॒ण्वानिति॑ शीर्.षण्-वान् । मेद्ध्यः॑ । भ॒व॒ति॒ । उ॒षाः । वै । अश्व॑स्य । मेद्ध्य॑स्य । शिरः॑ । सूर्यः॑ । चक्षुः॑ । वातः॑ । प्रा॒ण इति॑ प्र - अ॒नः । च॒न्द्रमाः᳚ । श्रोत्र᳚म् । दिशः॑ । पादाः᳚ । अ॒वा॒न्त॒र॒दि॒शा इत्य॑वान्तर - दि॒शाः । पर्.श॑वः । अ॒हो॒रा॒त्रे इत्य॑हः - रा॒त्रे । नि॒मे॒ष इति॑ नि - मे॒षः । अ॒द्‌र्ध॒मा॒सा इत्य॑द्‌र्ध - मा॒साः । पर्वा॑णि । मासाः᳚ । स॒धांना॒नीति॑ सं - धाना॑नि । ऋ॒तवः॑ । अङ्गा॑नि । सं॒ॅव॒थ्स॒र इति॑ सं - व॒थ्स॒रः । आ॒त्मा । र॒श्मयः॑ । केशाः᳚ । नक्ष॑त्राणि । रू॒पम् । तार॑काः । अ॒स्थानि॑ । नभः॑ । माꣳ॒॒सानि॑ । ओष॑धयः । लोमा॑नि । वन॒स्पत॑यः । वालाः᳚ । अ॒ग्निः । मुख᳚म् । वै॒श्वा॒न॒रः । व्यात्त॒मिति॑ वि - आत्त᳚म् । \textbf{  53} \newline
                  \newline
                                \textbf{ TS 7.5.25.2} \newline
                  स॒मु॒द्रः । उ॒दर᳚म् । अ॒न्तरि॑क्षम् । पा॒युः । द्यावा॑पृथि॒वी इति॒ द्यावा᳚ - पृ॒थि॒वी । आ॒ण्डौ । ग्रावा᳚ । शेपः॑ । सोमः॑ । रेतः॑ । यत् । ज॒ञ्ज॒भ्यते᳚ । तत् । वीति॑ । द्यो॒त॒ते॒ । यत् । वि॒धू॒नु॒त इति॑ वि-धू॒नु॒ते । तत् । स्त॒न॒य॒ति॒ । यत् । मेह॑ति । तत् । व॒र्.॒ष॒ति॒ । वाक् । ए॒व । अ॒स्य॒ । वाक् । अहः॑ । वै । अश्व॑स्य । जाय॑मानस्य । म॒हि॒मा । पु॒रस्ता᳚त् । जा॒य॒ते॒ । रात्रिः॑ । ए॒न॒म् । म॒हि॒मा । प॒श्चात् । अन्विति॑ । जा॒य॒ते॒ । ए॒तौ । वै । म॒हि॒मानौ᳚ । अश्व᳚म् । अ॒भितः॑ । समिति॑ । ब॒भू॒व॒तुः॒ । हयः॑ । दे॒वान् । अ॒व॒ह॒त् ( ) । अर्वा᳚ । असु॑रान् । वा॒जी । ग॒न्ध॒र्वान् । अश्वः॑ । म॒नु॒ष्यान्॑ । स॒मु॒द्रः । वै । अश्व॑स्य । योनिः॑ । स॒मु॒द्रः । बन्धुः॑ ॥ \textbf{  54} \newline
                  \newline
                      (व्यात्त॑ - मवह॒द् - द्वाद॑श च )  \textbf{(A25)} \newline \newline
\textbf{praSna korvai with starting padams of 1 to 11 Anuvaakams :-} \newline
(दे॒वा॒सु॒राः-परा॒-भूमि॒-र्भूमि॑-रुपप्र॒यन्तः॒-सं प॑श्या॒-म्यय॑ज्ञ्ः॒-सं प॑श्या-म्यग्निहो॒त्रं-मम॒ नाम॑-वैश्वान॒र-एका॑दश) । \newline

\textbf{korvai with starting padams of1, 11, 21 Series Of Panchaatis :-} \newline
(दे॒वा॒सु॒राः-क्रु॒द्धः-सं प॑श्यामि॒-सं प॑श्यामि॒-नक्त॒-मुप॑गन्त॒-नैक॑पञ्चा॒शत्) \newline

\textbf{first and last padam of Fifth Prasnam :-} \newline
(दे॒वा॒सु॒राः-पा॑रयि॒ष्णुं) \newline 


॥ हरिः॑ ॐ ॥॥ कृष्ण यजुर्वेदीय तैत्तिरीय संहितायां पद पाठे प्रथमकाण्डे पञ्चमः प्रश्नः समाप्तः ॥
---------------------------------------- \newline
\pagebreak
\pagebreak
        


\end{document}
