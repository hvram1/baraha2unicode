\documentclass[17pt]{extarticle}
\usepackage{babel}
\usepackage{fontspec}
\usepackage{polyglossia}
\usepackage{extsizes}



\setmainlanguage{sanskrit}
\setotherlanguages{english} %% or other languages
\setlength{\parindent}{0pt}
\pagestyle{myheadings}
\newfontfamily\devanagarifont[Script=Devanagari]{AdishilaVedic}


\newcommand{\VAR}[1]{}
\newcommand{\BLOCK}[1]{}




\begin{document}
\begin{titlepage}
    \begin{center}
 
\begin{sanskrit}
    { \Large
    ॐ नमः परमात्मने, श्री महागणपतये नमः, श्री गुरुभ्यो नमः ॥ ह॒रिः॒ ॐ 
    }
    \\
    \vspace{2.5cm}
    \mbox{ \Huge
    5.5      पञ्चमकाण्डे पञ्चमः प्रश्नः - वायव्यपश्वाद्यानं निरूपणं   }
\end{sanskrit}
\end{center}

\end{titlepage}
\tableofcontents

ॐ नमः परमात्मने, श्री महागणपतये नमः, श्री गुरुभ्यो नमः
ह॒रिः॒ ॐ \newline
5.5      पञ्चमकाण्डे पञ्चमः प्रश्नः - वायव्यपश्वाद्यानं निरूपणं \newline

\addcontentsline{toc}{section}{ 5.5      पञ्चमकाण्डे पञ्चमः प्रश्नः - वायव्यपश्वाद्यानं निरूपणं}
\markright{ 5.5      पञ्चमकाण्डे पञ्चमः प्रश्नः - वायव्यपश्वाद्यानं निरूपणं \hfill https://www.vedavms.in \hfill}
\section*{ 5.5      पञ्चमकाण्डे पञ्चमः प्रश्नः - वायव्यपश्वाद्यानं निरूपणं }
                                \textbf{ TS 5.5.1.1} \newline
                  यत् । एके॑न । सꣳ॒॒स्था॒पय॒तीति॑ सं - स्था॒पय॑ति । य॒ज्ञ्स्य॑ । सन्त॑त्या॒ इति॒ सं - त॒त्यै॒ । अवि॑च्छेदा॒येत्यवि॑ - छे॒दा॒य॒ । ऐ॒न्द्राः । प॒शवः॑ । ये । मु॒ष्क॒राः । यत् । ऐ॒न्द्राः । सन्तः॑ । अ॒ग्निभ्य॒ इत्य॒ग्नि - भ्यः॒ । आ॒ल॒भ्यन्त॒ इत्या᳚ - ल॒भ्यन्ते᳚ । दे॒वता᳚भ्यः । स॒मद॒मिति॑ स - मद᳚म् । द॒धा॒ति॒ । आ॒ग्ने॒यीः । त्रि॒ष्टुभः॑ । या॒ज्या॒नु॒वा॒क्या॑ इति॑ याज्या - अ॒नु॒वा॒क्याः᳚ । कु॒र्या॒त् । यत् । आ॒ग्ने॒यीः । तेन॑ । आ॒ग्ने॒याः । यत् । त्रि॒ष्टुभः॑ । तेन॑ । ऐ॒न्द्राः । समृ॑द्ध्या॒ इति॒ सं - ऋ॒द्ध्यै॒ । न । दे॒वता᳚भ्यः । स॒मद॒मिति॑ स - मद᳚म् । द॒धा॒ति॒ । वा॒यवे᳚ । नि॒युत्व॑त॒ इति॑ नि-युत्व॑ते । तू॒प॒रम् । एति॑ । ल॒भ॒ते॒ । तेजः॑ । अ॒ग्नेः । वा॒युः । तेज॑से । ए॒षः । एति॑ । ल॒भ्य॒ते॒ । तस्मा᳚त् । य॒द्रियङ्॑ । वा॒युः । \textbf{  1} \newline
                  \newline
                                \textbf{ TS 5.5.1.2} \newline
                  वाति॑ । त॒द्रियङ्॑ । अ॒ग्निः । द॒ह॒ति॒ । स्वम् । ए॒व । तत् । तेजः॑ । अन्विति॑ । ए॒ति॒ । यत् । न । नि॒युत्व॑त॒ इति॑ नि - युत्व॑ते । स्यात् । उदिति॑ । मा॒द्ये॒त् । यज॑मानः । नि॒युत्व॑त॒ इति॑ नि - युत्व॑ते । भ॒व॒ति॒ । यज॑मानस्य । अनु॑न्मादा॒येत्यनु॑त् - मा॒दा॒य॒ । वा॒यु॒मती॒ इति॑ वायु - मती᳚ । श्वे॒तव॑ती॒ इति॑ श्वे॒त - व॒ती॒ । या॒ज्या॒नु॒वा॒क्ये॑ इति॑ याज्या - अ॒नु॒वा॒क्ये᳚ । भ॒व॒तः॒ । स॒ते॒ज॒स्त्वायेति॑ सतेजः - त्वाय॑ । हि॒र॒ण्य॒ग॒र्भ इति॑ हिरण्य - ग॒र्भः । समिति॑ । अ॒व॒र्त॒त॒ । अग्रे᳚ । इति॑ । आ॒घा॒रमित्या᳚ - घा॒रम् । एति॑ । घा॒र॒य॒ति॒ । प्र॒जाप॑ति॒रिति॑ प्र॒जा-प॒तिः॒ । वै । हि॒र॒ण्य॒ग॒र्भ इति॑ हिरण्य - ग॒र्भः । प्र॒जाप॑ते॒रिति॑ प्र॒जा - प॒तेः॒ । अ॒नु॒रू॒प॒त्वायेत्य॑नुरूप - त्वाय॑ । सर्वा॑णि । वै । ए॒षः । रू॒पाणि॑ । प॒शू॒नाम् । प्रति॑ । एति॑ । ल॒भ्य॒ते॒ । यत् । श्म॒श्रु॒णः । तत् । \textbf{  2} \newline
                  \newline
                                \textbf{ TS 5.5.1.3} \newline
                  पुरु॑षाणाम् । रू॒पम् । यत् । तू॒प॒रः । तत् । अश्वा॑नाम् । यत् । अ॒न्यतो॑द॒न्नित्य॒न्यतः॑ - द॒न्न् । तत् । गवा᳚म् । यत् । अव्याः᳚ । इ॒व॒ । श॒फाः । तत् । अवी॑नाम् । यत् । अ॒जः । तत् । अ॒जाना᳚म् । वा॒युः । वै । प॒शू॒नाम् । प्रि॒यम् । धाम॑ । यत् । वा॒य॒व्यः॑ । भव॑ति । ए॒तम् । ए॒व । ए॒न॒म् । अ॒भीति॑ । स॒ञ्जा॒ना॒ना इति॑ सं - जा॒ना॒नाः । प॒शवः॑ । उपेति॑ । ति॒ष्ठ॒न्ते॒ । वा॒य॒व्यः॑ । का॒र्या(3)ः । प्रा॒जा॒प॒त्या(3) इति॑ प्राजा - प॒त्या(3)ः । इति॑ । आ॒हुः॒ । यत् । वा॒य॒व्य᳚म् । कु॒र्यात् । प्र॒जाप॑ते॒रिति॑ प्र॒जा - प॒तेः॒ । इ॒या॒त् । यत् । प्रा॒जा॒प॒त्यमिति॑ प्राजा - प॒त्यम् । कु॒र्यात् । वा॒योः । \textbf{  3} \newline
                  \newline
                                \textbf{ TS 5.5.1.4} \newline
                  इ॒या॒त् । यत् । वा॒य॒व्यः॑ । प॒शुः । भव॑ति । तेन॑ । वा॒योः । न । ए॒ति॒ । यत् । प्रा॒जा॒प॒त्य इति॑ प्राजा - प॒त्यः । पु॒रो॒डाशः॑ । भव॑ति । तेन॑ । प्र॒जाप॑ते॒रिति॑ प्र॒जा - प॒तेः॒ । न । ए॒ति॒ । यत् । द्वाद॑शकपाल॒ इति॒ द्वाद॑श - क॒पा॒लः॒ । तेन॑ । वै॒श्वा॒न॒रात् । न । ए॒ति॒ । आ॒ग्ना॒वै॒ष्ण॒वमित्या᳚ग्ना - वै॒ष्ण॒वम् । एका॑दशकपाल॒मित्येका॑दश- क॒पा॒ल॒म् । निरिति॑ । व॒प॒ति॒ । दी॒क्षि॒ष्यमा॑णः । अ॒ग्निः । सर्वाः᳚ । दे॒वताः᳚ । विष्णुः॑ । य॒ज्ञ्ः । दे॒वताः᳚ । च॒ । ए॒व । य॒ज्ञ्म् । च॒ । एति॑ । र॒भ॒ते॒ । अ॒ग्निः । अ॒व॒मः । दे॒वता॑नाम् । विष्णुः॑ । प॒र॒मः । यत् । आ॒ग्ना॒वै॒ष्ण॒वमित्या᳚ग्ना - वै॒ष्ण॒वम् । एका॑दशकपाल॒मित्येका॑दश-क॒पा॒ल॒म् । नि॒र्वप॒तीति॑ निः - वप॑ति । दे॒वताः᳚ । \textbf{  4} \newline
                  \newline
                                \textbf{ TS 5.5.1.5} \newline
                  ए॒व । उ॒भ॒यतः॑ । प॒रि॒गृह्येति॑ परि-गृह्य॑ । यज॑मानः । अवेति॑ । रु॒न्धे॒ । पु॒रो॒डाशे॑न । वै । दे॒वाः । अ॒मुष्मिन्न्॑ । लो॒के । आ॒द्र्ध्नु॒व॒न्न् । च॒रुणा᳚ । अ॒स्मिन्न् । यः । का॒मये॑त । अ॒मुष्मिन्न्॑ । लो॒के । ऋ॒द्ध्नु॒या॒म् । इति॑ । सः । पु॒रो॒डाश᳚म् । कु॒र्वी॒त॒ । अ॒मुष्मिन्न्॑ । ए॒व । लो॒के । ऋ॒द्ध्नो॒ति॒ । यत् । अ॒ष्टाक॑पाल॒ इत्य॒ष्टा - क॒पा॒लः॒ । तेन॑ । आ॒ग्ने॒यः । यत् । त्रि॒क॒पा॒ल इति॑ त्रि - क॒पा॒लः । तेन॑ । वै॒ष्ण॒वः । समृ॑द्ध्या॒ इति॒ सं - ऋ॒द्ध्यै॒ । यः । का॒मये॑त । अ॒स्मिन्न् । लो॒के । ऋ॒द्ध्नु॒या॒म् । इति॑ । सः । च॒रुम् । कु॒र्वी॒त॒ । अ॒ग्नेः । घृ॒तम् । विष्णोः᳚ । त॒ण्डु॒लाः । तस्मा᳚त् । \textbf{  5} \newline
                  \newline
                                \textbf{ TS 5.5.1.6} \newline
                  च॒रुः । का॒र्यः॑ । अ॒स्मिन्न् । ए॒व । लो॒के । ऋ॒द्ध्नो॒ति॒ । आ॒दि॒त्यः । भ॒व॒ति॒ । इ॒यम् । वै । अदि॑तिः । अ॒स्याम् । ए॒व । प्रतीति॑ । ति॒ष्ठ॒ति॒ । अथो॒ इति॑ । अ॒स्याम् । ए॒व । अधीति॑ । य॒ज्ञ्म् । त॒नु॒ते॒ । यः । वै । सं॒ॅव॒थ्स॒रमिति॑ सं - व॒थ्स॒रम् । उख्य᳚म् । अभृ॑त्वा । अ॒ग्निम् । चि॒नु॒ते । यथा᳚ । सा॒मि । गर्भः॑ । अ॒व॒पद्य॑त॒ इत्य॑व - पद्य॑ते । ता॒दृक् । ए॒व । तत् । आर्ति᳚म् । एति॑ । ऋ॒च्छे॒त् । वै॒श्वा॒न॒रम् । द्वाद॑शकपाल॒मिति॒ द्वाद॑श - क॒पा॒ल॒म् । पु॒रस्ता᳚त् । निरिति॑ । व॒पे॒त् । सं॒ॅव॒थ्स॒र इति॑ सं - व॒थ्स॒रः । वै । अ॒ग्निः । वै॒श्वा॒न॒रः । यथा᳚ । सं॒ॅव॒थ्स॒रमिति॑ सं - व॒थ्स॒रम् । आ॒प्त्वा । \textbf{  6} \newline
                  \newline
                                \textbf{ TS 5.5.1.7} \newline
                  का॒ले । आग॑त॒ इत्या - ग॒ते॒ । वि॒जाय॑त॒ इति॑ वि - जाय॑ते । ए॒वम् । ए॒व । सं॒ॅव॒थ्स॒रमिति॑ सं - व॒थ्स॒रम् । आ॒प्त्वा । का॒ले । आग॑त॒ इत्या - ग॒ते॒ । अ॒ग्निम् । चि॒नु॒ते॒ । न । आर्ति᳚म् । एति॑ । ऋ॒च्छ॒ति॒ । ए॒षा । वै । अ॒ग्नेः । प्रि॒या । त॒नूः । यत् । वै॒श्वा॒न॒रः । प्रि॒याम् । ए॒व । अ॒स्य॒ । त॒नुव᳚म् । अवेति॑ । रु॒न्धे॒ । त्रीणि॑ । ए॒तानि॑ । ह॒वीꣳ षि॑ । भ॒व॒न्ति॒ । त्रयः॑ । इ॒मे । लो॒काः । ए॒षाम् । लो॒काना᳚म् । रोहा॑य ॥ \textbf{  7 } \newline
                  \newline
                      (य॒द्रिय॑ङ् वा॒यु - र्यच्छ्म॑श्रु॒णस्तद् - वा॒यो - र्नि॒र्वप॑ति दे॒वता॒ - स्तस्मा॑ - दा॒प्त्वा - ष्टात्रिꣳ॑शच्च )  \textbf{(A1)} \newline \newline
                                \textbf{ TS 5.5.2.1} \newline
                  प्र॒जाप॑ति॒रिति॑ प्र॒जा - प॒तिः॒ । प्र॒जा इति॑ प्र - जाः । सृ॒ष्ट्वा । प्रे॒णा । अनु॑ । प्रेति॑ । अ॒वि॒श॒त् । ताभ्यः॑ । पुनः॑ । संभ॑वितु॒मिति॒ सं - भ॒वि॒तु॒म् । न । अ॒श॒क्नो॒त् । सः । अ॒ब्र॒वी॒त् । ऋ॒द्ध्नव॑त् । इत् । सः । यः । मा॒ । इ॒तः । पुनः॑ । स॒चिं॒नव॒दिति॑ सं - चि॒नव॑त् । इति॑ । तम् । दे॒वाः । समिति॑ । अ॒चि॒न्व॒न्न् । ततः॑ । वै । ते । आ॒द्‌र्ध्नु॒व॒न्न् । यत् । स॒मचि॑न्व॒न्निति॑ सं - अचि॑न्वन्न् । तत् । चित्य॑स्य । चि॒त्य॒त्वमिति॑ चित्य - त्वम् । यः । ए॒वम् । वि॒द्वान् । अ॒ग्निम् । चि॒नु॒ते । ऋ॒द्ध्नोति॑ । ए॒व । कस्मै᳚ । कम् । अ॒ग्निः । ची॒य॒ते॒ । इति॑ । आ॒हुः॒ । अ॒ग्नि॒वानित्य॑ग्नि - वान् । \textbf{  8} \newline
                  \newline
                                \textbf{ TS 5.5.2.2} \newline
                  अ॒सा॒नि॒ । इति॑ । वै । अ॒ग्निः । ची॒य॒ते॒ । अ॒ग्नि॒वानित्य॑ग्नि - वान् । ए॒व । भ॒व॒ति॒ । कस्मै᳚ । कम् । अ॒ग्निः । ची॒य॒ते॒ । इति॑ । आ॒हुः॒ । दे॒वाः । मा॒ । वे॒द॒न्न् । इति॑ । वै । अ॒ग्निः । ची॒य॒ते॒ । वि॒दुः । ए॒न॒म् । दे॒वाः । कस्मै᳚ । कम् । अ॒ग्निः । ची॒य॒ते॒ । इति॑ । आ॒हुः॒ । गृ॒ही । अ॒सा॒नि॒ । इति॑ । वै । अ॒ग्निः । ची॒य॒ते॒ । गृ॒ही । ए॒व । भ॒व॒ति॒ । कस्मै᳚ । कम् । अ॒ग्निः । ची॒य॒ते॒ । इति॑ । आ॒हुः॒ । प॒शु॒मानिति॑ पशु - मान् । अ॒सा॒नि॒ । इति॑ । वै । अ॒ग्निः । \textbf{  9} \newline
                  \newline
                                \textbf{ TS 5.5.2.3} \newline
                  ची॒य॒ते॒ । प॒शु॒मानिति॑ पशु - मान् । ए॒व । भ॒व॒ति॒ । कस्मै᳚ । कम् । अ॒ग्निः । ची॒य॒ते॒ । इति॑ । आ॒हुः॒ । स॒प्त । मा॒ । पुरु॑षाः । उपेति॑ । जी॒वा॒न् । इति॑ । वै । अ॒ग्निः । ची॒य॒ते॒ । त्रयः॑ । प्राञ्चः॑ । त्रयः॑ । प्र॒त्यञ्चः॑ । आ॒त्मा । स॒प्त॒मः । ए॒ताव॑न्तः । ए॒व । ए॒न॒म् । अ॒मुष्मिन्न्॑ । लो॒के । उपेति॑ । जी॒व॒न्ति॒ । प्र॒जाप॑ति॒रिति॑ प्र॒जा - प॒तिः॒ । अ॒ग्निम् । अ॒चि॒की॒ष॒त॒ । तम् । पृ॒थि॒वी । अ॒ब्र॒वी॒त् । न । मयि॑ । अ॒ग्निम् । चे॒ष्य॒से॒ । अतीति॑ । मा॒ । ध॒क्ष्य॒ति॒ । सा । त्वा॒ । अ॒ति॒द॒ह्यमा॒नेत्य॑ति - द॒ह्यमा॑ना । वीति॑ । ध॒वि॒ष्ये॒ । \textbf{  10} \newline
                  \newline
                                \textbf{ TS 5.5.2.4} \newline
                  सः । पापी॑यान् । भ॒वि॒ष्य॒सि॒ । इति॑ । सः । अ॒ब्र॒वी॒त् । तथा᳚ । वै । अ॒हम् । क॒रि॒ष्या॒मि॒ । यथा᳚ । त्वा॒ । न । अ॒ति॒ध॒क्ष्यतीत्य॑ति-ध॒क्ष्यति॑ । इति॑ । सः । इ॒माम् । अ॒भीति॑ । अ॒मृ॒श॒त् । प्र॒जाप॑ति॒रिति॑ प्र॒जा-प॒तिः॒ । त्वा॒ । सा॒द॒य॒त॒ । तया᳚ । दे॒वत॑या । अ॒ङ्गि॒र॒स्वत् । ध्रु॒वा । सी॒द॒ । इति॑ । इ॒माम् । ए॒व । इष्ट॑काम् । कृ॒त्वा । उपेति॑ । अ॒ध॒त्त॒ । अन॑तिदाहा॒येत्यन॑ति - दा॒हा॒य॒ । यत् । प्रतीति॑ । अ॒ग्निम् । चि॒न्वी॒त । तत् । अ॒भीति॑ । मृ॒शे॒त् । प्र॒जाप॑ति॒रिति॑ प्र॒जा-प॒तिः॒ । त्वा॒ । सा॒द॒य॒तु॒ । तया᳚ । दे॒वत॑या । अ॒ङ्गि॒र॒स्वत् । ध्रु॒वा । सी॒द॒ । \textbf{  11} \newline
                  \newline
                                \textbf{ TS 5.5.2.5} \newline
                  इति॑ । इ॒माम् । ए॒व । इष्ट॑काम् । कृ॒त्वा । उपेति॑ । ध॒त्ते॒ । अन॑तिदाहा॒येत्यन॑ति - दा॒हा॒य॒ । प्र॒जाप॑ति॒रिति॑ प्र॒जा - प॒तिः॒ । अ॒का॒म॒य॒त॒ । प्रेति॑ । जा॒ये॒य॒ । इति॑ । सः । ए॒तम् । उख्य᳚म् । अ॒प॒श्य॒त् । तम् । सं॒ॅव॒थ्स॒रमिति॑ सं - व॒थ्स॒रम् । अ॒बि॒भः॒ । ततः॑ । वै । सः । प्रेति॑ । अ॒जा॒य॒त॒ । तस्मा᳚त् । सं॒ॅव॒थ्स॒रमिति॑ सं-व॒थ्स॒रम् । भा॒र्यः॑ । प्रेति॑ । ए॒व । जा॒य॒ते॒ । तम् । वस॑वः । अ॒ब्रु॒व॒न्न् । प्रेति॑ । त्वम् । अ॒ज॒नि॒ष्ठाः॒ । व॒यम् । प्रेति॑ । जा॒या॒म॒है॒ । इति॑ । तम् । वसु॑भ्य॒ इति॒ वसु॑ - भ्यः॒ । प्रेति॑ । अ॒य॒च्छ॒त् । तम् । त्रीणि॑ । अहा॑नि । अ॒बि॒भ॒रुः॒ । तेन॑ । \textbf{  12} \newline
                  \newline
                                \textbf{ TS 5.5.2.6} \newline
                  त्रीणि॑ । च॒ । श॒ताति॑ । असृ॑जन्त । त्रय॑स्त्रिꣳशत॒मिति॒ त्रयः॑- त्रिꣳ॒॒श॒त॒म् । च॒ । तस्मा᳚त् । त्र्य॒हमिति॑ त्रि - अ॒हम् । भा॒र्यः॑ । प्रेति॑ । ए॒व । जा॒य॒ते॒ । तान् । रु॒द्राः । अ॒ब्रु॒व॒न्न् । प्रेति॑ । यू॒यम् । अ॒ज॒नि॒ढ्व॒म् । व॒यम् । प्रेति॑ । जा॒या॒म॒है॒ । इति॑ । तम् । रु॒द्रेभ्यः॑ । प्रेति॑ । अ॒य॒च्छ॒न्न् । तम् । षट् । अहा॑नि । अ॒बि॒भ॒रुः॒ । तेन॑ । त्रीणि॑ । च॒ । श॒तानि॑ । असृ॑जन्त । त्रय॑स्त्रिꣳशत॒मिति॒ त्रयः॑ - त्रिꣳ॒॒श॒त॒म् । च॒ । तस्मा᳚त् । ष॒ड॒हमिति॑ षट् - अ॒हम् । भा॒र्यः॑ । प्रेति॑ । ए॒व । जा॒य॒ते॒ । तान् । आ॒दि॒त्याः । अ॒ब्रु॒व॒न्न् । प्रेति॑ । यू॒यम् । अ॒ज॒नि॒ढ्व॒म् । व॒यम् । \textbf{  13} \newline
                  \newline
                                \textbf{ TS 5.5.2.7} \newline
                  प्रेति॑ । जा॒या॒म॒है॒ । इति॑ । तम् । आ॒दि॒त्येभ्यः॑ । प्रेति॑ । अ॒य॒च्छ॒न्न् । तम् । द्वाद॑श । अहा॑नि । अ॒बि॒भ॒रुः॒ । तेन॑ । त्रीणि॑ । च॒ । श॒तानि॑ । असृ॑जन्त । त्रय॑स्त्रिꣳशत॒मिति॒ त्रयः॑ - त्रिꣳ॒॒श॒त॒म् । च॒ । तस्मा᳚त् । द्वा॒द॒शा॒हमिति॑ द्वादश - अ॒हम् । भा॒र्यः॑ । प्रेति॑ । ए॒व । जा॒य॒ते॒ । तेन॑ । वै । ते । स॒हस्र᳚म् । अ॒सृ॒ज॒न्त॒ । उ॒खाम् । स॒ह॒स्र॒त॒मीमिति॑ सहस्र - त॒मीम् । यः । ए॒वम् । उख्य᳚म् । सा॒ह॒स्रम् । वेद॑ । प्रेति॑ । स॒हस्र᳚म् । प॒शून् । आ॒प्नो॒ति॒ ॥ \textbf{  14} \newline
                  \newline
                      (अ॒ग्नि॒वान् - प॑शु॒मान॑सा॒नीति॒ वा अ॒ग्नि - र्द्ध॑विष्ये - मृशेत् प्र॒जाप॑तिस्त्वा सादयतु॒ तया॑ दे॒वत॑याऽङ्गिर॒स्व ध्रु॒वा सी॑द॒ - तेन॒ - ताना॑दि॒त्या अ॑ब्रुव॒न् प्र यू॒यम॑जनिढ्वं ॅव॒यं - च॑त्वारिꣳ॒॒शच्च॑)  \textbf{(A2)} \newline \newline
                                \textbf{ TS 5.5.3.1} \newline
                  यजु॑षा । वै । ए॒षा । क्रि॒य॒ते॒ । यजु॑षा । प॒च्य॒ते॒ । यजु॑षा । वीति॑ । मु॒च्य॒ते॒ । यत् । उ॒खा । सा । वै । ए॒षा । ए॒तर्.हि॑ । या॒तया॒म्नीति॑ या॒त - या॒म्नी॒ । सा । न । पुनः॑ । प्र॒युज्येति॑ प्र - युज्या᳚ । इति॑ । आ॒हुः॒ । अग्ने᳚ । यु॒क्ष्व । हि । ये । तव॑ । यु॒क्ष्व । हि । दे॒व॒हूत॑मा॒निति॑ देव - हूत॑मान् । इति॑ । उ॒खाया᳚म् । जु॒हो॒ति॒ । तेन॑ । ए॒व । ए॒ना॒म् । पुनः॑ । प्रेति॑ । यु॒ङ्क्ते॒ । तेन॑ । अया॑तया॒म्नीत्यया॑त - या॒म्नी॒ । यः । वै । अ॒ग्निम् । योगे᳚ । आग॑त॒ इत्या - ग॒ते॒ । यु॒नक्ति॑ । यु॒ङ्क्ते । यु॒ञ्जा॒नेषु॑ । अग्ने᳚ । \textbf{  15} \newline
                  \newline
                                \textbf{ TS 5.5.3.2} \newline
                  यु॒क्ष्व । हि । ये । तव॑ । यु॒क्ष्व । हि । दे॒व॒हूत॑मा॒निति॑ देव - हूत॑मान् । इति॑ । आ॒ह॒ । ए॒षः । वै । अ॒ग्नेः । योगः॑ । तेन॑ । ए॒व । ए॒न॒म् । यु॒न॒क्ति॒ । यु॒ङ्क्ते । यु॒ञ्जा॒नेषु॑ । ब्र॒ह्म॒वा॒दिन॒ इति॑ ब्रह्म - वा॒दिनः॑ । व॒द॒न्ति॒ । न्यङ्॑ । अ॒ग्निः । चे॒त॒व्या(3)ः । उ॒त्ता॒ना(3) इत्यु॑त् - ता॒ना(3)ः । इति॑ । वय॑साम् । वै । ए॒षः । प्र॒ति॒मयेति॑ प्रति - मया᳚ । ची॒य॒ते॒ । यत् । अ॒ग्निः । यत् । न्य॑ञ्चम् । चि॒नु॒यात् । पृ॒ष्टि॒तः । ए॒न॒म् । आहु॑तय॒ इत्या - हु॒त॒यः॒ । ऋ॒च्छे॒युः॒ । यत् । उ॒त्ता॒नमित्यु॑त्-ता॒नम् । न । पति॑तुम् । श॒क्नु॒या॒त् । असु॑वर्ग्य॒ इत्यसु॑वः-ग्यः॒ । अ॒स्य॒ । स्या॒त् । प्रा॒चीन᳚म् । उ॒त्ता॒नमित्यु॑त्-ता॒नम् । \textbf{  16} \newline
                  \newline
                                \textbf{ TS 5.5.3.3} \newline
                  पु॒रु॒ष॒शी॒र्॒.षमिति॑ पुरुष-शी॒र्॒.षम् । उपेति॑ । द॒धा॒ति॒ । मु॒ख॒तः । ए॒व । ए॒न॒म् । आहु॑तय॒ इत्या - हु॒त॒यः॒ । ऋ॒च्छ॒न्ति॒ । न । उ॒त्ता॒नमित्यु॑त्- ता॒नम् । चि॒नु॒ते॒ । सु॒व॒र्ग्य॑ इति॑ सुवः - ग्यः॑ । अ॒स्य॒ । भ॒व॒ति॒ । सौ॒र्या । जु॒हो॒ति॒ । चक्षुः॑ । ए॒व । अ॒स्मि॒न्न् । प्रतीति॑ । द॒धा॒ति॒ । द्विः । जु॒हो॒ति॒ । द्वे इति॑ । हि । चक्षु॑षी॒ इति॑ । स॒मा॒न्या । जु॒हो॒ति॒ । स॒मा॒नम् । हि । चक्षुः॑ । समृ॑द्ध्या॒ इति॒ सं - ऋ॒द्ध्यै॒ । दे॒वा॒सु॒रा इति॑ देव - अ॒सु॒राः । संॅय॑त्ता॒ इति॒ सं - य॒त्ताः॒ । आ॒स॒न्न् । ते । वा॒मम् । वसु॑ । सम् । नीति॑ । अ॒द॒ध॒त॒ । तत् । दे॒वाः । वा॒म॒भृतेति॑ वाम - भृता᳚ । अ॒वृ॒ञ्ज॒त॒ । तत् । वा॒म॒भृत॒ इति॑ वाम - भृतः॑ । वा॒म॒भृ॒त्त्वमिति॑ वामभृत् - त्वम् । यत् । वा॒म॒भृत॒मिति॑ वाम - भृत᳚म् ( ) । उ॒प॒दधा॒तीत्यु॑प - दधा॑ति । वा॒मम् । ए॒व । तया᳚ । वसु॑ । यज॑मानः । भ्रातृ॑व्यस्य । वृ॒ङ्क्ते॒ । हिर॑ण्यमू॒द्‌र्ध्नीति॒ हिर॑ण्य - मू॒द्‌र्ध्नी॒ । भ॒व॒ति॒ । ज्योतिः॑ । वै । हिर॑ण्यम् । ज्योतिः॑ । वा॒मम् । ज्योति॑षा । ए॒व । अ॒स्य॒ । ज्योतिः॑ । वा॒मम् । वृ॒ङ्क्ते॒ । द्वि॒य॒जुरिति॑ द्वि-य॒जुः । भ॒व॒ति॒ । प्रति॑ष्ठित्या॒ इति॒ प्रति॑-स्थि॒त्यै॒ ॥ \textbf{  17 } \newline
                  \newline
                      (यु॒ञ्जा॒नेष्वग्ने᳚-प्रा॒चीन॑मुत्ता॒नं - ॅवा॑म॒भृतं॒ - चतु॑र्विꣳशतिश्च)  \textbf{(A3)} \newline \newline
                                \textbf{ TS 5.5.4.1} \newline
                  आपः॑ । वरु॑णस्य । पत्न॑यः । आ॒स॒न्न् । ताः । अ॒ग्निः । अ॒भीति॑ । अ॒द्ध्या॒य॒त् । ताः । समिति॑ । अ॒भ॒व॒त् । तस्य॑ । रेतः॑ । परेति॑ । अ॒प॒त॒त् । तत् । इ॒यम् । अ॒भ॒व॒त् । यत् । द्वि॒तीय᳚म् । प॒राप॑त॒दिति॑ परा - अप॑तत् । तत् । अ॒सौ । अ॒भ॒व॒त् । इ॒यम् । वै । वि॒राडिति॑ वि - राट् । अ॒सौ । स्व॒राडिति॑ स्व - राट् । यत् । वि॒राजा॒विति॑ वि - राजौ᳚ । उ॒प॒दधा॒तीत्यु॑प-दधा॑ति । इ॒मे इति॑ । ए॒व । उपेति॑ । ध॒त्ते॒ । यत् । वै । अ॒सौ । रेतः॑ । सि॒ञ्चति॑ । तत् । अ॒स्याम् । प्रतीति॑ । ति॒ष्ठ॒ति॒ । तत् । प्रेति॑ । जा॒य॒ते॒ । ताः । ओष॑धयः । \textbf{  18} \newline
                  \newline
                                \textbf{ TS 5.5.4.2} \newline
                  वी॒रुधः॑ । भ॒व॒न्ति॒ । ताः । अ॒ग्निः । अ॒त्ति॒ । यः । ए॒वम् । वेद॑ । प्रेति॑ । ए॒व । जा॒य॒ते॒ । अ॒न्ना॒द इत्य॑न्न - अ॒दः । भ॒व॒ति॒ । यः । रे॒त॒स्वी । स्यात् । प्र॒थ॒माया᳚म् । तस्य॑ । चित्या᳚म् । उ॒भे इति॑ । उपेति॑ । द॒द्ध्या॒त् । इ॒मे इति॑ । ए॒व । अ॒स्मै॒ । स॒मीची॒ इति॑ । रेतः॑ । सि॒ञ्च॒तः॒ । यः । सि॒क्तरे॑ता॒ इति॑ सि॒क्त - रे॒ताः॒ । स्यात् । प्र॒थ॒माया᳚म् । तस्य॑ । चित्या᳚म् । अ॒न्याम् । उपेति॑ । द॒द्ध्या॒त् । उ॒त्त॒माया॒मित्यु॑त्-त॒माया᳚म् । अ॒न्याम् । रेतः॑ । ए॒व । अ॒स्य॒ । सि॒क्तम् । आ॒भ्याम् । उ॒भ॒यतः॑ । परीति॑ । गृ॒ह्णा॒ति॒ । सं॒ॅव॒थ्स॒रमिति॑ सं - व॒थ्स॒रम् । न । कम् । \textbf{  19} \newline
                  \newline
                                \textbf{ TS 5.5.4.3} \newline
                  च॒न । प्र॒त्यव॑रोहे॒दिति॑ प्रति - अव॑रोहेत् । न । हि । इ॒मे इति॑ । कम् । च॒न । प्र॒त्य॒व॒रोह॑त॒ इति॑ प्रति - अ॒व॒रोह॑तः । तत् । ए॒न॒योः॒ । व्र॒तम् । यः । वै । अप॑शीर्.षाण॒मित्यप॑ - शी॒र्.॒षा॒ण॒म् । अ॒ग्निम् । चि॒नु॒ते । अप॑शी॒र्॒.षेत्यप॑ - शी॒र्.॒षा॒ । अ॒मुष्मिन्न्॑ । लो॒के । भ॒व॒ति॒ । यः । सशी॑र्.षाण॒मिति॒ स - शी॒र्.॒षा॒ण॒म् । चि॒नु॒ते । सशी॒र्॒.षेति॒ स - शी॒र्.॒षा॒ । अ॒मुष्मिन्न्॑ । लो॒के । भ॒व॒ति॒ । चित्ति᳚म् । जु॒हो॒मि॒ । मन॑सा । घृ॒तेन॑ । यथा᳚ । दे॒वाः । इ॒ह । आ॒गम॒न्नित्या᳚ - गमन्न्॑ । वी॒तिहो᳚त्रा॒ इति॑ वी॒ति - हो॒त्राः॒ । ऋ॒ता॒वृध॒ इत्यृ॑त - वृधः॑ । स॒मु॒द्रस्य॑ । व॒युन॑स्य । पत्मन्न्॑ । जु॒होमि॑ । वि॒श्वक॑र्मण॒ इति॑ वि॒श्व - क॒र्म॒णे॒ । विश्वा᳚ । अहा᳚ । अम॑र्त्यम् । ह॒विः । इति॑ । स्व॒य॒मा॒तृ॒ण्णामिति॑ स्वयं - आ॒तृ॒ण्णाम् । उ॒प॒धायेत्यु॑प - धाय॑ । जु॒हो॒ति॒ । \textbf{  20} \newline
                  \newline
                                \textbf{ TS 5.5.4.4} \newline
                  ए॒तत् । वै । अ॒ग्नेः । शिरः॑ । सशी॑र्.षाण॒मिति॒ स - शी॒र्.॒षा॒ण॒म् । ए॒व । अ॒ग्निम् । चि॒नु॒ते॒ । सशी॒र्.॒षेति॒ स - शी॒र्.॒षा॒ । अ॒मुष्मिन्न्॑ । लो॒के । भ॒व॒ति॒ । यः । ए॒वम् । वेद॑ । सु॒व॒र्गायेति॑ सुवः - गाय॑ । वै । ए॒षः । लो॒काय॑ । ची॒य॒ते॒ । यत् । अ॒ग्निः । तस्य॑ । यत् । अय॑थापूर्व॒मित्यय॑था - पू॒र्व॒म् । क्रि॒यते᳚ । असु॑वर्ग्य॒मित्यसु॑वः-ग्य॒म् । अ॒स्य॒ । तत् । सु॒व॒ग्य॑ इति॑ सुवः - ग्यः॑ । अ॒ग्निः । चिति᳚म् । उ॒प॒धायेत्यु॑प - धाय॑ । अ॒भीति॑ । मृ॒शे॒त् । चित्ति᳚म् । अचि॑त्तिम् । चि॒न॒व॒त् । वीति॑ । वि॒द्वान् । पृ॒ष्ठा । इ॒व॒ । वी॒ता । वृ॒जि॒ना । च॒ । मर्तान्॑ । रा॒ये । च॒ । नः॒ । स्व॒प॒त्यायेति॑ सु - अ॒प॒त्याय॑ ( ) । दे॒व॒ । दिति᳚म् । च॒ । रास्व॑ । अदि॑तिम् । उ॒रु॒ष्य॒ । इति॑ । य॒था॒पू॒र्वमिति॑ यथा - पू॒र्वम् । ए॒व । ए॒ना॒म् । उपेति॑ । ध॒त्ते॒ । प्राञ्च᳚म् । ए॒न॒म् । चि॒नु॒ते॒ । सु॒व॒र्ग्य॑ इति॑ सुवः - ग्यः॑ । अ॒स्य॒ । भ॒व॒ति॒ ॥ \textbf{  21 } \newline
                  \newline
                      (ओष॑धयः॒ - कं - जु॑होति - स्वप॒त्याया॒ - ष्टाद॑श च)  \textbf{(A4)} \newline \newline
                                \textbf{ TS 5.5.5.1} \newline
                  वि॒श्वक॒र्मेति॑ वि॒श्व - क॒र्मा॒ । दि॒शाम् । पतिः॑ । सः । नः॒ । प॒शून् । पा॒तु॒ । सः । अ॒स्मान् । पा॒तु॒ । तस्मै᳚ । नमः॑ । प्र॒जाप॑ति॒रिति॑ प्र॒जा - प॒तिः॒ । रु॒द्रः । वरु॑णः । अ॒ग्निः । दि॒शाम् । पतिः॑ । सः । नः॒ । प॒शून् । पा॒तु॒ । सः । अ॒स्मान् । पा॒तु॒ । तस्मै᳚ । नमः॑ । ए॒ताः । वै । दे॒वताः᳚ । ए॒तेषा᳚म् । प॒शू॒नाम् । अधि॑पतय॒ इत्यधि॑ - प॒त॒यः॒ । ताभ्यः॑ । वै । ए॒षः । एति॑ । वृ॒श्च्य॒ते॒ । यः । प॒शु॒शी॒र्॒.षाणीति॑ पशु - शी॒र्॒.षाणि॑ । उ॒प॒दधा॒तीत्यु॑प - दधा॑ति । हि॒र॒ण्ये॒ष्ट॒का इति॑ हिरण्य - इ॒ष्ट॒काः । उपेति॑ । द॒धा॒ति॒ । ए॒ताभ्यः॑ । ए॒व । दे॒वता᳚भ्यः । नमः॑ । क॒रो॒ति॒ । ब्र॒ह्म॒वा॒दिन॒ इति॑ ब्रह्म - वा॒दिनः॑ । \textbf{  22} \newline
                  \newline
                                \textbf{ TS 5.5.5.2} \newline
                  व॒द॒न्ति॒ । अ॒ग्नौ । ग्रा॒म्यान् । प॒शून् । प्रेति॑ । द॒धा॒ति॒ । शु॒चा । आ॒र॒ण्यान् । अ॒र्प॒य॒ति॒ । किम् । ततः॑ । उदिति॑ । शिꣳ॒॒ष॒ति॒ । इति॑ । यत् । हि॒र॒ण्ये॒ष्ट॒का इति॑ हिरण्य - इ॒ष्ट॒काः । उ॒प॒दधा॒तीत्यु॑प-दधा॑ति । अ॒मृत᳚म् । वै । हिर॑ण्यम् । अ॒मृते॑न । ए॒व । ग्रा॒म्येभ्यः॑ । प॒शुभ्य॒ इति॑ प॒शु - भ्यः॒ । भे॒ष॒जम् । क॒रो॒ति॒ । न । ए॒ना॒न् । हि॒न॒स्ति॒ । प्रा॒ण इति॑ प्र - अ॒नः । वै । प्र॒थ॒मा । स्व॒य॒मा॒तृ॒ण्णेति॑ स्वयं -  आ॒तृ॒ण्णा । व्या॒न इति॑ वि - अ॒नः । द्वि॒तीया᳚ । अ॒पा॒न इत्य॑प-अ॒नः । तृ॒तीया᳚ । अनु॑ । प्रेति॑ । अ॒न्या॒त् । प्र॒थ॒माम् । स्व॒य॒मा॒तृ॒ण्णामिति॑ स्वयं - आ॒तृ॒ण्णाम् । उ॒प॒धायेत्यु॑प - धाय॑ । प्रा॒णेनेति॑ प्र - अ॒नेन॑ । ए॒व । प्रा॒णमिति॑ प्र - अ॒नम् । समिति॑ । अ॒द्‌र्ध॒य॒ति॒ । वीति॑ । अ॒न्या॒त् । \textbf{  23} \newline
                  \newline
                                \textbf{ TS 5.5.5.3} \newline
                  द्वि॒तीया᳚म् । उ॒प॒धायेत्यु॑प - धाय॑ । व्या॒नेनेति॑ वि - अ॒नेन॑ । ए॒व । व्या॒नमिति॑ वि - अ॒नम् । समिति॑ । अ॒द्‌र्ध॒य॒ति॒ । अपेति॑ । अ॒न्या॒त् । तृ॒तीया᳚म् । उ॒प॒धायेत्यु॑प - धाय॑ । अ॒पा॒नेनेत्य॑प - अ॒नेन॑ । ए॒व । अ॒पा॒नमित्य॑प - अ॒नम् । समिति॑ । अ॒द्‌र्ध॒य॒ति॒ । अथो॒ इति॑ । प्रा॒णैरिति॑ प्र - अ॒नैः । ए॒व । ए॒न॒म् । समिति॑ । इ॒न्धे॒ । भूः । भुवः॑ । सुवः॑ । इति॑ । स्व॒य॒मा॒तृ॒ण्णा इति॑ स्वयं-आ॒तृ॒ण्णाः । उपेति॑ । द॒धा॒ति॒ । इ॒मे । वै । लो॒काः । स्व॒य॒मा॒तृ॒ण्णा इति॑ स्वयं-आ॒तृ॒ण्णाः । ए॒ताभिः॑ । खलु॑ । वै । व्याहृ॑तीभि॒रिति॒ व्याहृ॑ति - भिः॒ । प्र॒जाप॑ति॒रिति॑ प्र॒जा - प॒तिः॒ । प्रेति॑ । अ॒जा॒य॒त॒ । यत् । ए॒ताभिः॑ । व्याहृ॑तीभि॒रिति॒ व्याहृ॑ति - भिः । स्व॒य॒मा॒तृ॒ण्णा इति॑ स्वयं - आ॒तृ॒ण्णाः । उ॒प॒दधा॒तीत्यु॑प - दधा॑ति । इ॒मान् । ए॒व । लो॒कान् । उ॒प॒धायेत्यु॑प - धाय॑ । ए॒षु । \textbf{  24} \newline
                  \newline
                                \textbf{ TS 5.5.5.4} \newline
                  लो॒केषु॑ । अधि॑ । प्रेति॑ । जा॒य॒ते॒ । प्रा॒णायेति॑ प्र - अ॒नाय॑ । व्या॒नायेति॑ वि - अ॒नाय॑ । अ॒पा॒नायेत्य॑प - अ॒नाय॑ । वा॒चे । त्वा॒ । चक्षु॑षे । त्वा॒ । तया᳚ । दे॒वत॑या । अ॒ङ्गि॒र॒स्वत् । ध्रु॒वा । सी॒द॒ । अ॒ग्निना᳚ । वै । दे॒वाः । सु॒व॒र्गमिति॑ सुवः - गम् । लो॒कम् । अ॒जि॒गाꣳ॒॒स॒न्न् । तेन॑ । पति॑तुम् । न । अ॒श॒क्नु॒व॒न्न् । ते । ए॒ताः । चत॑स्रः । स्व॒य॒मा॒तृ॒ण्णा इति॑ स्वयं - अ॒तृ॒ण्णाः । अ॒प॒श्य॒न्न् । ताः । दि॒क्षु । उपेति॑ । अ॒द॒ध॒त॒ । तेन॑ । स॒र्वत॑श्चक्षु॒षेति॑ स॒र्वतः॑ - च॒क्षु॒षा॒ । सु॒व॒र्गमिति॑ सुवः - गम् । लो॒कम् । आ॒य॒न्न् । यत् । चत॑स्रः । स्व॒य॒मा॒तृ॒ण्णा इति॑ स्वयं - आ॒तृ॒ण्णाः । दि॒क्षु । उ॒प॒दधा॒तीत्यु॑प - दधा॑ति । सर्वत॑श्चक्षु॒षेति॑ स॒र्वतः॑ - च॒क्षु॒षा॒ । ए॒व । तत् । अ॒ग्निना᳚ । यज॑मानः ( ) । सु॒व॒र्गमिति॑ सुवः - गम् । लो॒कम् । ए॒ति॒ ॥ \textbf{  25} \newline
                  \newline
                      (ब्र॒ह्म॒वा॒दिनो॒ - व्य॑न्या - दे॒षु - यज॑मान॒ - स्त्रीणि॑ च)  \textbf{(A5)} \newline \newline
                                \textbf{ TS 5.5.6.1} \newline
                  अग्ने᳚ । एति॑ । या॒हि॒ । वी॒तये᳚ । इति॑ । आ॒ह॒ । अह्व॑त । ए॒व । ए॒न॒म् । अ॒ग्निम् । दू॒तम् । वृ॒णी॒म॒हे॒ । इति॑ । आ॒ह॒ । हू॒त्वा । ए॒व । ए॒न॒म् । वृ॒णी॒ते॒ । अ॒ग्निना᳚ । अ॒ग्निः । समिति॑ । इ॒द्ध्य॒ते॒ । इति॑ । आ॒ह॒ । समिति॑ । इ॒न्धे॒ । ए॒व । ए॒न॒म् । अ॒ग्निः । वृ॒त्राणि॑ । ज॒ङ्घ॒न॒त् । इति॑ । आ॒ह॒ । समि॑द्ध॒ इति॒ सं - इ॒द्धे॒ । ए॒व । अ॒स्मि॒न्न् । इ॒न्द्रि॒यम् । द॒धा॒ति॒ । अ॒ग्नेः । स्तोम᳚म् । म॒ना॒म॒हे॒ । इति॑ । आ॒ह॒ । म॒नु॒ते । ए॒व । ए॒न॒म् । ए॒तानि॑ । वै । अह्ना᳚म् । रू॒पाणि॑ । \textbf{  26} \newline
                  \newline
                                \textbf{ TS 5.5.6.2} \newline
                  अ॒न्व॒हमित्य॑नु - अ॒हम् । ए॒व । ए॒न॒म् । चि॒नु॒ते॒ । अवेति॑ । अह्ना᳚म् । रू॒पाणि॑ । रु॒न्धे॒ । ब्र॒ह्म॒वा॒दिन॒ इति॑ ब्रह्म-वा॒दिनः॑ । व॒द॒न्ति॒ । कस्मा᳚त् । स॒त्यात् । या॒तया᳚म्नी॒रिति॑ या॒त - या॒म्नीः॒ । अ॒न्याः । इष्ट॑काः । अया॑तया॒म्नीत्यया॑त-या॒म्नी॒ । लो॒क॒पृं॒णेति॑ लोकं - पृ॒णा । इति॑ । ऐ॒न्द्रा॒ग्नीत्यै᳚न्द्र-अ॒ग्नी । हि । बा॒र्.॒ह॒स्प॒त्या । इति॑ । ब्रू॒या॒त् । इ॒न्द्रा॒ग्नी इती᳚न्द्र - अ॒ग्नी । च॒ । हि । दे॒वाना᳚म् । बृह॒स्पतिः॑ । च॒ । अया॑तयामान॒ इत्यया॑त - या॒मा॒नः॒ । अ॒नु॒च॒रव॒तीत्य॑नुच॒र - व॒ती॒ । भ॒व॒ति॒ । अजा॑मित्वा॒येत्यजा॑मि-त्वा॒य॒ । अ॒नु॒ष्टुभेत्य॑नु - स्तुभा᳚ । अन्विति॑ । च॒र॒ति॒ । आ॒त्मा । वै । लो॒क॒पृं॒णेति॑ लोकं - पृ॒णा । प्रा॒ण इति॑ प्र - अ॒नः । अ॒नु॒ष्टुबित्य॑नु - स्तुप् । तस्मा᳚त् । प्रा॒ण इति॑ प्र - अ॒नः । सर्वा॑णि । अङ्गा॑नि । अन्विति॑ । च॒र॒ति॒ । ताः । अ॒स्य॒ । सूद॑दोहस॒ इति॒ सूद॑ - दो॒ह॒सः॒ । \textbf{  27} \newline
                  \newline
                                \textbf{ TS 5.5.6.3} \newline
                  इति॑ । आ॒ह॒ । तस्मा᳚त् । परु॑षिपरु॒षीति॒ परु॑षि - प॒रु॒षि॒ । रसः॑ । सोम᳚म् । श्री॒ण॒न्ति॒ । पृश्न॑यः । इति॑ । आ॒ह॒ । अन्न᳚म् । वै । पृश्नि॑ । अन्न᳚म् । ए॒व । अवेति॑ । रु॒न्धे॒ । अ॒र्कः । वै । अ॒ग्निः । अ॒र्कः । अन्न᳚म् । अन्न᳚म् । ए॒व । अवेति॑ । रु॒न्धे॒ । जन्मन्न्॑ । दे॒वाना᳚म् । विशः॑ । त्रि॒षु । एति॑ । रो॒च॒ने । दि॒वः । इति॑ । आ॒ह॒ । इ॒मान् । ए॒व । अ॒स्मै॒ । लो॒कान् । ज्योति॑ष्मतः । क॒रो॒ति॒ । यः । वै । इष्ट॑कानाम् । प्र॒ति॒ष्ठामिति॑ प्रति - स्थाम् । वेद॑ । प्रतीति॑ । ए॒व । ति॒ष्ठ॒ति॒ । तया᳚ ( ) । दे॒वत॑या । अ॒ङ्गि॒र॒स्वत् । ध्रु॒वा । सी॒द॒ । इति॑ । आ॒ह॒ । ए॒षा । वै । इष्ट॑कानाम् । प्र॒ति॒ष्ठेति॑ प्रति - स्था । यः । ए॒वम् । वेद॑ । प्रतीति॑ । ए॒व । ति॒ष्ठ॒ति॒ ॥ \textbf{  28 } \newline
                  \newline
                      (रू॒पाणि॒ - सूद॑दोहस॒ - स्तया॒ - षोड॑श च)  \textbf{(A6)} \newline \newline
                                \textbf{ TS 5.5.7.1} \newline
                  सु॒व॒र्गायेति॑ सुवः - गाय॑ । वै । ए॒षः । लो॒काय॑ । ची॒य॒ते॒ । यत् । अ॒ग्निः । वज्रः॑ । ए॒का॒द॒शिनी᳚ । यत् । अ॒ग्नौ । ए॒का॒द॒शिनी᳚म् । मि॒नु॒यात् । वज्रे॑ण । ए॒न॒म् । सु॒व॒र्गादिति॑ सुवः - गात् । लो॒कात् । अ॒न्तः । द॒द्ध्या॒त् । यत् । न । मि॒नु॒यात् । स्वरु॑भि॒रिति॒ स्वरु॑-भिः॒ । प॒शून् । वीति॑ । अ॒द्‌र्ध॒ये॒त् । ए॒क॒यू॒पमित्ये॑क - यू॒पम् । मि॒नो॒ति॒ । न । ए॒न॒म् । वज्रे॑ण । सु॒व॒र्गादिति॑ सुवः - गात् । लो॒कात् । अ॒न्त॒र्दधा॒तीत्य॑न्तः - दधा॑ति । न । स्वरु॑भि॒रिति॒ स्वरु॑-भिः॒ । प॒शून् । वीति॑ । अ॒द्‌र्ध॒य॒ति॒ । वीति॑ । वै । ए॒षः । इ॒न्द्रि॒येण॑ । वी॒र्ये॑ण । ऋ॒द्ध्य॒ते॒ । यः । अ॒ग्निम् । चि॒न्वन्न् । अ॒धि॒क्राम॒तीत्य॑धि - क्राम॑ति । ऐ॒न्द्रि॒या । \textbf{  29} \newline
                  \newline
                                \textbf{ TS 5.5.7.2} \newline
                  ऋ॒चा । आ॒क्रम॑ण॒मित्या᳚ - क्रम॑णम् । प्रतीति॑ । इष्ट॑काम् । उपेति॑ । द॒द्ध्या॒त् । न । इ॒न्द्रि॒येण॑ । वी॒र्ये॑ण । वीति॑ । ऋ॒द्ध्य॒ते॒ । रु॒द्रः । वै । ए॒षः । यत् । अ॒ग्निः । तस्य॑ । ति॒स्रः । श॒र॒व्याः᳚ । प्र॒तीची᳚ । ति॒रश्ची᳚ । अ॒नूची᳚ । ताभ्यः॑ । वै । ए॒षः । एति॑ । वृ॒श्च्य॒ते॒ । यः । अ॒ग्निम् । चि॒नु॒ते । अ॒ग्निम् । चि॒त्वा । ति॒सृ॒ध॒न्वमिति॑ तिसृ - ध॒न्वम् । अया॑चितम् । ब्रा॒ह्म॒णाय॑ । द॒द्या॒त् । ताभ्यः॑ । ए॒व । नमः॑ । क॒रो॒ति॒ । अथो॒ इति॑ । ताभ्यः॑ । ए॒व । आ॒त्मान᳚म् । निरिति॑ । क्री॒णी॒ते॒ । यत् । ते॒ । रु॒द्र॒ । पु॒रः । \textbf{  30} \newline
                  \newline
                                \textbf{ TS 5.5.7.3} \newline
                  धनुः॑ । तत् । वातः॑ । अन्विति॑ । वा॒तु॒ । ते॒ । तस्मै᳚ । ते॒ । रु॒द्र॒ । सं॒ॅव॒थ्स॒रेणेति॑ सं - व॒थ्स॒रेण॑ । नमः॑ । क॒रो॒मि॒ । यत् । ते॒ । रु॒द्र॒ । द॒क्षि॒णा । धनुः॑ । तत् । वातः॑ । अन्विति॑ । वा॒तु॒ । ते॒ । तस्मै᳚ । ते॒ । रु॒द्र॒ । प॒रि॒व॒थ्स॒रेणेति॑ परि - व॒थ्स॒रेण॑ । नमः॑ । क॒रो॒मि॒ । यत् । ते॒ । रु॒द्र॒ । प॒श्चात् । धनुः॑ । तत् । वातः॑ । अन्विति॑ । वा॒तु॒ । ते॒ । तस्मै᳚ । ते॒ । रु॒द्र॒ । इ॒दा॒व॒थ्स॒रेण॑ । नमः॑ । क॒रो॒मि॒ । यत् । ते॒ । रु॒द्र॒ । उ॒त्त॒रादित्यु॑त् - त॒रात् । धनुः॑ । तत् । \textbf{  31} \newline
                  \newline
                                \textbf{ TS 5.5.7.4} \newline
                  वातः॑ । अन्विति॑ । वा॒तु॒ । ते॒ । तस्मै᳚ । ते॒ । रु॒द्र॒ । इ॒दु॒व॒थ्स॒रेणेती॑दु- व॒थ्स॒रेण॑ । नमः॑ । क॒रो॒मि॒ । यत् । ते॒ । रु॒द्र॒ । उ॒परि॑ । धनुः॑ । तत् । वातः॑ । अन्विति॑ । वा॒तु॒ । ते॒ । तस्मै᳚ । ते॒ । रु॒द्र॒ । व॒थ्स॒रेण॑ । नमः॑ । क॒रो॒मि॒ । रु॒द्रः । वै । ए॒षः । यत् । अ॒ग्निः । सः । यथा᳚ । व्या॒घ्रः । क्रु॒द्धः । तिष्ठ॑ति । ए॒वम् । वै । ए॒षः । ए॒तर्.हि॑ । सञ्चि॑त॒मिति॒ सं - चि॒त॒म् । ए॒तैः । उपेति॑ । ति॒ष्ठ॒ते॒ । न॒म॒स्का॒रैरिति॑ नमः-का॒रैः । ए॒व । ए॒न॒म् । श॒म॒य॒ति॒ । ये । अ॒ग्नयः॑ । \textbf{  32} \newline
                  \newline
                                \textbf{ TS 5.5.7.5} \newline
                  पु॒री॒ष्याः᳚ । प्रवि॑ष्टा॒ इति॒ प्र - वि॒ष्टाः॒ । पृ॒थि॒वीम् । अनु॑ ॥ तेषा᳚म् । त्वम् । अ॒सि॒ । उ॒त्त॒म इत्यु॑त् - त॒मः । प्रेति॑ । नः॒ । जी॒वात॑वे । सु॒व॒ ॥ आप᳚म् । त्वा॒ । अ॒ग्ने॒ । मन॑सा । आप᳚म् । त्वा॒ । अ॒ग्ने॒ । तप॑सा । आप᳚म् । त्वा॒ । अ॒ग्ने॒ । दी॒क्षया᳚ । आप᳚म् । त्वा॒ । अ॒ग्ने॒ । उ॒प॒सद्भि॒रित्यु॑प॒सत् - भिः॒ । आप᳚म् । त्वा॒ । अ॒ग्ने॒ । सु॒त्यया᳚ । आप᳚म् । त्वा॒ । अ॒ग्ने॒ । दक्षि॑णाभिः । आप᳚म् । त्वा॒ । अ॒ग्ने॒ । अ॒व॒भृ॒थेनेत्य॑व - भृ॒थेन॑ । आप᳚म् । त्वा॒ । अ॒ग्ने॒ । व॒शया᳚ । आप᳚म् । त्वा॒ । अ॒ग्ने॒ । स्व॒गा॒का॒रेणेति॑ स्वगा - का॒रेण॑ । इति॑ । आ॒ह॒ ( ) । ए॒षा । वै । अ॒ग्नेः । आप्तिः॑ । तया᳚ । ए॒व । ए॒न॒म् । आ॒प्नो॒ति॒ ॥ \textbf{  33 } \newline
                  \newline
                      (ऐ॒न्द्रि॒या - पु॒र - उ॑त्त॒राद्धनु॒स्त- द॒ग्नय॑ - आहा॒ - ष्टौ च॑)  \textbf{(A7)} \newline \newline
                                \textbf{ TS 5.5.8.1} \newline
                  गा॒य॒त्रेण॑ । पु॒रस्ता᳚त् । उपेति॑ । ति॒ष्ठ॒ते॒ । प्रा॒णमिति॑ प्र - अ॒नम् । ए॒व । अ॒स्मि॒न्न् । द॒धा॒ति॒ । बृ॒ह॒द्र॒थ॒न्त॒राभ्या॒मिति॑ बृहत् - र॒थ॒न्त॒राभ्या᳚म् । प॒क्षौ । ओजः॑ । ए॒व । अ॒स्मि॒न्न् । द॒धा॒ति॒ । ऋ॒तु॒स्थाय॑ज्ञाय॒ज्ञिये॑न । पुच्छ᳚म् । ऋ॒तुषु॑ । ए॒व । प्रतीति॑ । ति॒ष्ठ॒ति॒ । पृ॒ष्ठैः । उपेति॑ । ति॒ष्ठ॒ते॒ । तेजः॑ । वै । पृ॒ष्ठानि॑ । तेजः॑ । ए॒व । अ॒स्मि॒न्न् । द॒धा॒ति॒ । प्र॒जाप॑ति॒रिति॑ प्र॒जा - प॒तिः॒ । अ॒ग्निम् । अ॒सृ॒ज॒त॒ । सः । अ॒स्मा॒त् । सृ॒ष्टः । पराङ्॑ । ऐ॒त् । तम् । वा॒र॒व॒न्तीये॒नेति॑ वार - व॒न्तीये॑न । अ॒वा॒र॒य॒त॒ । तत् । वा॒र॒व॒न्तीय॒स्येति॑ वार - व॒न्तीय॑स्य । वा॒र॒व॒न्ती॒य॒त्वमिति॑ वारवन्तीय - त्वम् । श्यै॒तेन॑ । श्ये॒ती । अ॒कु॒रु॒त॒ । तत् । श्यै॒तस्य॑ । श्यै॒त॒त्वमिति॑ श्यैत - त्वम् । \textbf{  34} \newline
                  \newline
                                \textbf{ TS 5.5.8.2} \newline
                  यत् । वा॒र॒व॒न्तीये॒नेति॑ वार - व॒न्तीये॑न । उ॒प॒तिष्ठ॑त॒ इत्यु॑प - तिष्ठ॑ते । वा॒रय॑ते । ए॒व । ए॒न॒म् । श्यै॒तेन॑ । श्ये॒ती । कु॒रु॒ते॒ । प्र॒जाप॑ते॒र्.हृद॑येन । अ॒पि॒प॒क्षमित्य॑पि - प॒क्षम् । प्रति॑ । उपेति॑ । ति॒ष्ठ॒ते॒ । प्रे॒माण᳚म् । ए॒व । अ॒स्य॒ । ग॒च्छ॒ति॒ । प्राच्या᳚ । त्वा॒ । दि॒शा । सा॒द॒या॒मि॒ । गा॒य॒त्रेण॑ । छन्द॑सा । अ॒ग्निना᳚ । दे॒वत॑या । अ॒ग्नेः । शी॒र्ष्णा । अ॒ग्नेः । शिरः॑ । उपेति॑ । द॒धा॒मि॒ । दक्षि॑णया । त्वा॒ । दि॒शा । सा॒द॒या॒मि॒ । त्रैष्टु॑भेन । छन्द॑सा । इन्द्रे॑ण । दे॒वत॑या । अ॒ग्नेः । प॒क्षेण॑ । अ॒ग्नेः । प॒क्षम् । उपेति॑ । द॒धा॒मि॒ । प्र॒तीच्या᳚ । त्वा॒ । दि॒शा । सा॒द॒या॒मि॒ । \textbf{  35} \newline
                  \newline
                                \textbf{ TS 5.5.8.3} \newline
                  जाग॑तेन । छन्द॑सा । स॒वि॒त्रा । दे॒वत॑या । अ॒ग्नेः । पुच्छे॑न । अ॒ग्नेः । पुच्छ᳚म् । उपेति॑ । द॒धा॒मि॒ । उदी᳚च्या । त्वा॒ । दि॒शा । सा॒द॒या॒मि॒ । आनु॑ष्टुभे॒नेत्यानु॑ - स्तु॒भे॒न॒ । छन्द॑सा । मि॒त्रावरु॑णाभ्या॒मिति॑ मि॒त्रा-वरु॑णाभ्याम् । दे॒वत॑या । अ॒ग्नेः । प॒क्षेण॑ । अ॒ग्नेः । प॒क्षम् । उपेति॑ । द॒धा॒मि॒ । ऊ॒द्‌र्ध्वया᳚ । त्वा॒ । दि॒शा । सा॒द॒या॒मि॒ । पाङ्क्ते॑न । छन्द॑सा । बृह॒स्पति॑ना । दे॒वत॑या । अ॒ग्नेः । पृ॒ष्ठेन॑ । अ॒ग्नेः । पृ॒ष्ठम् । उपेति॑ । द॒धा॒मि॒ । यः । वै । अपा᳚त्मान॒मित्यप॑-आ॒त्मा॒न॒म् । अ॒ग्निम् । चि॒नु॒ते । अपा॒त्मेत्यप॑ - आ॒त्मा॒ । अ॒मुष्मिन्न्॑ । लो॒के । भ॒व॒ति॒ । यः । सात्मा॑न॒मिति॒ स - आ॒त्मा॒न॒म् । चि॒नु॒ते ( ) । सात्मेति॒ स-आ॒त्मा॒ । अ॒मुष्मिन्न्॑ । लो॒के । भ॒व॒ति॒ । आ॒त्मे॒ष्ट॒का इत्या᳚त्म-इ॒ष्ट॒काः । उपेति॑ । द॒धा॒ति॒ । ए॒षः । वै । अ॒ग्नेः । आ॒त्मा । सात्मा॑न॒मिति॒ स-आ॒त्मा॒न॒म् । ए॒व । अ॒ग्निम् । चि॒नु॒ते॒ । सात्मेति॒ स - आ॒त्मा॒ । अ॒मुष्मिन्न्॑ । लो॒के । भ॒व॒ति॒ । यः । ए॒वम् । वेद॑ ॥ \textbf{  36 } \newline
                  \newline
                      (श्यै॒त॒त्वं - प्र॒तीच्या᳚ त्वा दि॒शा सा॑दयामि॒ - यः सात्मा॑नं चिनु॒ते - द्वाविꣳ॑शतिश्च)  \textbf{(A8)} \newline \newline
                                \textbf{ TS 5.5.9.1} \newline
                  अग्ने᳚ । उ॒द॒ध॒ इत्यु॑द - धे॒ । या । ते॒ । इषुः॑ । यु॒वा । नाम॑ । तया᳚ । नः॒ । मृ॒ड॒ । तस्याः᳚ । ते॒ । नमः॑ । तस्याः᳚ । ते॒ । उपेति॑ । जीव॑न्तः । भू॒या॒स्म॒ । अग्ने᳚ । दु॒द्ध्र॒ । ग॒ह्य॒ । किꣳ॒॒शि॒ल॒ । व॒न्य॒ । या । ते॒ । इषुः॑ । यु॒वा । नाम॑ । तया᳚ । नः॒ । मृ॒ड॒ । तस्याः᳚ । ते॒ । नमः॑ । तस्याः᳚ । ते॒ । उपेति॑ । जीव॑न्तः । भू॒या॒स्म॒ । पञ्च॑ । वै । ए॒ते । अ॒ग्नयः॑ । यत् । चित॑यः । उ॒द॒धिरित्यु॑द - धिः । ए॒व । नाम॑ । प्र॒थ॒मः । दु॒द्ध्रः । \textbf{  37} \newline
                  \newline
                                \textbf{ TS 5.5.9.2} \newline
                  द्वि॒तीयः॑ । गह्यः॑ । तृ॒तीयः॑ । किꣳ॒॒शि॒लः । च॒तु॒र्थः । वन्यः॑ । प॒ञ्च॒मः । तेभ्यः॑ । यत् । आहु॑ती॒रित्या - हु॒तीः॒ । न । जु॒हु॒यात् । अ॒द्ध्व॒र्युम् । च॒ । यज॑मानम् । च॒ । प्रेति॑ । द॒हे॒युः॒ । यत् । ए॒ताः । आहु॑ती॒रित्या - हु॒तीः॒ । जु॒होति॑ । भा॒ग॒धेये॒नेति॑ भाग-धेये॑न । ए॒व । ए॒ना॒न् । श॒म॒य॒ति॒ । न । आर्ति᳚म् । एति॑ । ऋ॒च्छ॒ति॒ । अ॒द्ध्व॒र्युः । न । यज॑मानः । वाक् । मे॒ । आ॒सन्न् । न॒सोः । प्रा॒ण इति॑ प्र - अ॒नः । अ॒क्ष्योः । चक्षुः॑ । कर्ण॑योः । श्रोत्र᳚म् । बा॒हु॒वोः । बल᳚म् । ऊ॒रु॒वोः । ओजः॑ । अरि॑ष्टा । विश्वा॑नि । अङ्गा॑नि । त॒नूः । \textbf{  38} \newline
                  \newline
                                \textbf{ TS 5.5.9.3} \newline
                  त॒नुवा᳚ । मे॒ । स॒ह । नमः॑ । ते॒ । अ॒स्तु॒ । मा । मा॒ । हिꣳ॒॒सीः॒ । अपेति॑ । वै । ए॒तस्मा᳚त् । प्रा॒णा इति॑ प्र - अ॒नाः । क्रा॒म॒न्ति॒ । यः । अ॒ग्निम् । चि॒न्वन्न् । अ॒धि॒क्राम॒तीत्य॑धि - क्राम॑ति । वाक् । मे॒ । आ॒सन्न् । न॒सोः । प्रा॒ण इति॑ प्र - अ॒नः । इति॑ । आ॒ह॒ । प्रा॒णानिति॑ प्र - अ॒नान् । ए॒व । आ॒त्मन्न् । ध॒त्ते॒ । यः । रु॒द्र ः । अ॒ग्नौ । यः । अ॒फ्स्वित्य॑प् - सु । यः । ओष॑धीषु । यः । रु॒द्रः । विश्वा᳚ । भुव॑ना । आ॒वि॒वेशेत्या᳚ - वि॒वेश॑ । तस्मै᳚ । रु॒द्राय॑ । नमः॑ । अ॒स्तु॒ । आहु॑तिभागा॒ इत्याहु॑ति - भा॒गाः॒ । वै । अ॒न्ये । रु॒द्राः । ह॒विर्भा॑गा॒ इति॑ ह॒विः - भा॒गाः॒ । \textbf{  39} \newline
                  \newline
                                \textbf{ TS 5.5.9.4} \newline
                  अ॒न्ये । श॒त॒रु॒द्रीय॒मिति॑ शत - रु॒द्रीय᳚म् । हु॒त्वा । गा॒वी॒धु॒कम् । च॒रुम् । ए॒तेन॑ । यजु॑षा । च॒र॒माया᳚म् । इष्ट॑कायाम् । नीति॑ । द॒द्ध्या॒त् । भा॒ग॒धेये॒नेति॑ भाग - धेये॑न । ए॒व । ए॒न॒म् । श॒म॒य॒ति॒ । तस्य॑ । तु । वै । श॒त॒रु॒द्रीय॒मिति॑ शत - रु॒द्रीय᳚म् । हु॒तम् । इति॑ । आ॒हुः॒ । यस्य॑ । ए॒तत् । अ॒ग्नौ । क्रि॒यते᳚ । इति॑ । वस॑वः । त्वा॒ । रु॒द्रैः । पु॒रस्ता᳚त् । पा॒न्तु॒ । पि॒तरः॑ । त्वा॒ । य॒मरा॑जान॒ इति॑ य॒म - रा॒जा॒नः॒ । पि॒तृभि॒रिति॑ पि॒तृ - भिः॒ । द॒क्षि॒ण॒तः । पा॒न्तु॒ । आ॒दि॒त्याः । त्वा॒ । विश्वैः᳚ । दे॒वैः । प॒श्चात् । पा॒न्तु॒ । द्यु॒ता॒नः । त्वा॒ । मा॒रु॒तः । म॒रुद्भि॒रिति॑ म॒रुत् - भिः॒ । उ॒त्त॒र॒त इत्यु॑त् - त॒र॒तः । पा॒तु॒ । \textbf{  40} \newline
                  \newline
                                \textbf{ TS 5.5.9.5} \newline
                  दे॒वाः । त्वा॒ । इन्द्र॑ज्येष्ठा॒ इतीन्द्र॑ - ज्ये॒ष्ठाः॒ । वरु॑णराजान॒ इति॒ वरु॑ण - रा॒जा॒नः॒ । अ॒धस्ता᳚त् । च॒ । उ॒परि॑ष्ठात् । च॒ । पा॒न्तु॒ । न । वै । ए॒तेन॑ । पू॒तः । न । मेद्ध्यः॑ । न । प्रोक्षि॑त॒ इति॒ प्र - उ॒क्षि॒तः॒ । यत् । ए॒न॒म् । अतः॑ । प्रा॒चीन᳚म् । प्रो॒क्षतीति॑ प्र - उ॒क्षति॑ । यथ् । सञ्चि॑त॒मिति॒ सं - चि॒त॒म् । आज्ये॑न । प्रो॒क्षतीति॑ प्र - उ॒क्षति॑ । तेन॑ । पू॒तः । तेन॑ । मेद्ध्यः॑ । तेन॑ । प्रोक्षि॑त॒ इति॒ प्र-उ॒क्षि॒तः॒ ॥ \textbf{  41} \newline
                  \newline
                      (दु॒ध्र - स्त॒नू - र्ह॒विर्भा॑गाः - पातु॒ - द्वात्रिꣳ॑शच्च)  \textbf{(A9)} \newline \newline
                                \textbf{ TS 5.5.10.1} \newline
                  स॒मीची᳚ । नाम॑ । अ॒सि॒ । प्राची᳚ । दिक् । तस्याः᳚ । ते॒ । अ॒ग्निः । अधि॑पति॒रित्यधि॑ - प॒तिः॒ । अ॒सि॒तः । र॒क्षि॒ता । यः । च॒ । अधि॑पति॒रित्यधि॑-प॒तिः॒ । यः । च॒ । गो॒प्ता । ताभ्या᳚म् । नमः॑ । तौ । नः॒ । मृ॒ड॒य॒ता॒म् । ते । यम् । द्वि॒ष्मः । यः । च॒ । नः॒ । द्वेष्टि॑ । तम् । वा॒म् । जम्भे᳚ । द॒धा॒मि॒ । ओ॒ज॒स्विनी᳚ । नाम॑ । अ॒सि॒ । द॒क्षि॒णा । दिक् । तस्याः᳚ । ते॒ । इन्द्रः॑ । अधि॑पति॒रित्यधि॑ - प॒तिः॒ । पृदा॑कुः । प्राची᳚ । नाम॑ । अ॒सि॒ । प्र॒तीची᳚ । दिक् । तस्याः᳚ । ते॒ । \textbf{  42} \newline
                  \newline
                                \textbf{ TS 5.5.10.2} \newline
                  सोमः॑ । अधि॑पति॒रित्यधि॑ - प॒तिः॒ । स्व॒ज इति॑ स्व - जः । अ॒व॒स्थावेत्य॑व - स्थावा᳚ । नाम॑ । अ॒सि॒ । उदी॑ची । दिक् । तस्याः᳚ । ते॒ । वरु॑णः । अधि॑पति॒रित्यधि॑-प॒तिः॒ । ति॒रश्व॑राजि॒रिति॑ ति॒रश्व॑-रा॒जिः॒ । अधि॑प॒त्नीत्यधि॑-प॒त्नी॒ । नाम॑ । अ॒सि॒ । बृ॒ह॒ती । दिक् । तस्याः᳚ । ते॒ । बृह॒स्पतिः॑ । अधि॑पति॒रित्यधि॑ - प॒तिः॒ । श्वि॒त्रः । व॒शिनी᳚ । नाम॑ । अ॒सि॒ । इ॒यं । दिक् । तस्याः᳚ । ते॒ । य॒मः । अधि॑पति॒रित्यधि॑-प॒तिः॒ । क॒ल्माष॑ग्रीव॒ इति॑ क॒ल्माष॑ - ग्री॒वः॒ । र॒क्षि॒ता । यः । च॒ । अधि॑पति॒रित्यधि॑-प॒तिः॒ । यः । च॒ । गो॒प्ता । ताभ्या᳚म् । नमः॑ । तौ । नः॒ । मृ॒ड॒य॒ता॒म् । ते । यम् । द्वि॒ष्मः । यः । च॒ । \textbf{  43} \newline
                  \newline
                                \textbf{ TS 5.5.10.3} \newline
                  नः॒ । द्वेष्टि॑ । तम् । वा॒म् । जम्भे᳚ । द॒धा॒मि॒ । ए॒ताः । वै । दे॒वताः᳚ । अ॒ग्निम् । चि॒तम् । र॒क्ष॒न्ति॒ । ताभ्यः॑ । यत् । आहु॑ती॒रित्या - हु॒तीः॒ । न । जु॒हु॒यात् । अ॒द्ध्व॒र्युम् । च॒ । यज॑मानम् । च॒ । ध्या॒ये॒युः॒ । यत् । ए॒ताः । आहु॑ती॒रित्या - हु॒तीः॒ । जु॒होति॑ । भा॒ग॒धेये॒नेति॑ भाग-धेये॑न । ए॒व । ए॒ना॒न् । श॒म॒य॒ति॒ । न । आर्ति᳚म् । एति॑ । ऋ॒च्छ॒ति॒ । अ॒द्ध्व॒र्युः । न । यज॑मानः । हे॒तयः॑ । नाम॑ । स्थ॒ । तेषा᳚म् । वः॒ । पु॒रः । गृ॒हाः । अ॒ग्निः । वः॒ । इष॑वः । स॒लि॒लः । नि॒लि॒पां इति॑ नि-लि॒पांः । नाम॑ । \textbf{  44} \newline
                  \newline
                                \textbf{ TS 5.5.10.4} \newline
                  स्थ॒ । तेषा᳚म् । वः॒ । द॒क्षि॒णा । गृ॒हाः । पि॒तरः॑ । वः॒ । इष॑वः । सग॑रः । व॒ज्रिणः॑ । नाम॑ । स्थ॒ । तेषा᳚म् । वः॒ । प॒श्चात् । गृ॒हाः । स्वप्नः॑ । वः॒ । इष॑वः । गह्व॑रः । अ॒व॒स्थावा॑न॒ इत्य॑व - स्थावा॑नः । नाम॑ । स्थ॒ । तेषा᳚म् । वः॒ । उ॒त्त॒रादित्यु॑त् - त॒रात् । गृ॒हाः । आपः॑ । वः॒ । इष॑वः । स॒मु॒द्रः । अधि॑पतय॒ इत्यधि॑ - प॒त॒यः॒ । नाम॑ । स्थ॒ । तेषा᳚म् । वः॒ । उ॒परि॑ । गृ॒हाः । व॒र्.॒षम् । वः॒ । इष॑वः । अव॑स्वान् । क्र॒व्याः । नाम॑ । स्थ॒ । पार्थि॑वाः । तेषा᳚म् । वः॒ । इ॒ह । गृ॒हाः । \textbf{  45} \newline
                  \newline
                                \textbf{ TS 5.5.10.5} \newline
                  अन्न᳚म् । वः॒ । इष॑वः । नि॒मि॒ष इति॑ नि - मि॒षः । वा॒त॒ना॒ममिति॑ वात - ना॒मम् । तेभ्यः॑ । वः॒ । नमः॑ । ते । नः॒ । मृ॒ड॒य॒त॒ । ते । यम् । द्वि॒ष्मः । यः । च॒ । नः॒ । द्वेष्टि॑ । तम् । वः॒ । जम्भे᳚ । द॒धा॒मि॒ । हु॒ताद॒ इति॑ हुत - अदः॑ । वै । अ॒न्ये । दे॒वाः । अ॒हु॒ताद॒ इत्य॑हुत - अदः॑ । अ॒न्ये । तान् । अ॒ग्नि॒चिदित्य॑ग्नि - चित् । ए॒व । उ॒भयान्॑ । प्री॒णा॒ति॒ । द॒द्ध्ना । म॒धु॒मि॒श्रेणेति॑ मधु - मि॒श्रेण॑ । ए॒ताः । आहु॑ती॒रित्या-हु॒तीः॒ । जु॒हो॒ति॒ । भा॒ग॒धेये॒नेति॑ भाग - धेये॑न । ए॒व । ए॒ना॒न् । प्री॒णा॒ति॒ । अथो॒ इति॑ । खलु॑ । आ॒हुः॒ । इष्ट॑काः । वै । दे॒वाः । अ॒हु॒ताद॒ इत्य॑हुत - अदः॑ । इति॑ । \textbf{  46} \newline
                  \newline
                                \textbf{ TS 5.5.10.6} \newline
                  अ॒नु॒प॒रि॒क्राम॒मित्य॑नु - प॒रि॒क्राम᳚म् । जु॒हो॒ति॒ । अप॑रिवर्ग॒मित्यप॑रि - व॒र्ग॒म् । ए॒व । ए॒ना॒न् । प्री॒णा॒ति॒ । इ॒मम् । स्तन᳚म् । ऊर्ज॑स्वन्तम् । ध॒य॒ । अ॒पाम् । प्रप्या॑त॒मिति॒ प्र-प्या॒त॒म् । अ॒ग्ने॒ । स॒रि॒रस्य॑ । मद्ध्ये᳚ ॥ उथ्स᳚म् । जु॒ष॒स्व॒ । मधु॑मन्त॒मिति॒ मधु॑-म॒न्त॒म् । ऊ॒र्व॒ । स॒मु॒द्रिय᳚म् । सद॑नम् । एति॑ । वि॒श॒स्व॒ ॥ यः । वै । अ॒ग्निम् । प्र॒युज्येति॑ प्र - युज्य॑ । न । वि॒मु॒ञ्चतीति॑ वि - मु॒ञ्चति॑ । यथा᳚ । अश्वः॑ । यु॒क्तः । अवि॑मुच्यमान॒ इत्यवि॑ - मु॒च्य॒मा॒नः॒ । क्षुद्ध्यन्न्॑ । प॒रा॒भव॒तीति॑ परा - भव॑ति । ए॒वम् । अ॒स्य॒ । अ॒ग्निः । परेति॑ । भ॒व॒ति॒ । तम् । प॒रा॒भव॑न्त॒मिति॑ परा - भव॑न्तम् । यज॑मानः । अनु॑ । परेति॑ । भ॒व॒ति॒ । सः । अ॒ग्निम् । चि॒त्वा । लू॒क्षः । \textbf{  47} \newline
                  \newline
                                \textbf{ TS 5.5.10.7} \newline
                  भ॒व॒ति॒ । इ॒मम् । स्तन᳚म् । ऊर्ज॑स्वन्तम् । ध॒य॒ । अ॒पाम् । इति॑ । आज्य॑स्य । पू॒र्णाम् । स्रुच᳚म् । जु॒हो॒ति॒ । ए॒षः । वै । अ॒ग्नेः । वि॒मो॒क इति॑ वि - मो॒कः । वि॒मुच्येति॑ वि - मुच्य॑ । ए॒व । अ॒स्मै॒ । अन्न᳚म् । अपीति॑ । द॒धा॒ति॒ । तस्मा᳚त् । आ॒हुः॒ । यः । च॒ । ए॒वम् । वेद॑ । यः । च॒ । न । सु॒धाय॒मिति॑ सु - धाय᳚म् । ह॒ । वै । वा॒जी । सुहि॑त॒ इति॒ सु - हि॒तः॒ । द॒धा॒ति॒ । इति॑ । अ॒ग्निः । वाव । वा॒जी । तम् । ए॒व । तत् । प्री॒णा॒ति॒ । सः । ए॒न॒म् । प्री॒तः । प्री॒णा॒ति॒ । वसी॑यान् । भ॒व॒ति॒ ( ) ॥ \textbf{  48} \newline
                  \newline
                      (प्र॒तीची॒ दिक्तस्या᳚स्ते-द्वि॒ष्मो यश्च॑-निलि॒म्पा ना-मे॒ ह गृ॒हा-इति॑-लू॒क्षो-वसी॑यान् भवति)  \textbf{(A10)} \newline \newline
                                \textbf{ TS 5.5.11.1} \newline
                  इन्द्रा॑य । राज्ञे᳚ । सू॒क॒रः । वरु॑णाय । राज्ञे᳚ । कृष्णः॑ । य॒माय॑ । राज्ञे᳚ । ऋश्यः॑ । ऋ॒ष॒भाय॑ । राज्ञे᳚ । ग॒व॒यः । शा॒र्दू॒लाय॑ । राज्ञे᳚ । गौ॒रः । पु॒रु॒ष॒रा॒जायेति॑ पुरुष - रा॒जाय॑ । म॒र्कटः॑ । क्षि॒प्र॒श्ये॒नस्येति॑ क्षिप्र - श्ये॒नस्य॑ । वर्ति॑का । नील॑ङ्गोः । क्रिमिः॑ । सोम॑स्य । राज्ञ्ः॑ । कु॒लु॒ङ्गः । सिन्धोः᳚ । शिꣳ॒॒शु॒मारः॑ । हि॒मव॑त॒ इति॑ हि॒म - व॒तः॒ । ह॒स्ती ॥ \textbf{ } \newline
                  \newline
                      इन्द्रा॑य॒ राज्ञे॑ सूक॒रो वरु॑णाय॒ राज्ञे॒ कृष्णो॑ य॒माय॒ राज्ञ्॒ ऋश्य॑ ऋष॒भाय॒ राज्ञे॑ गव॒यः शा᳚र्दू॒लाय॒ राज्ञे॑ गौ॒रः पु॑रुषरा॒जाय॑ म॒र्कटः॑ क्षिप्रश्ये॒नस्य॒ वर्ति॑का॒ नील॑ङ्गोः॒ क्रिमिः॒ सोम॑स्य॒ राज्ञ्ः॑ कुलु॒ङ्गः सिन्धोः᳚ शिꣳशु॒मारो॑ हि॒मव॑तो ह॒स्ती ॥ 49 (इन्द्रा॑य॒ राज्ञे॒-ऽष्टाविꣳ॑शतिः)  \textbf{(A11)} \newline \newline
                                \textbf{ TS 5.5.12.1} \newline
                  म॒युः । प्रा॒जा॒प॒त्य इति॑ प्राजा - प॒त्यः । ऊ॒लः । हली᳚क्ष्णः । वृ॒ष॒दꣳ॒॒शः । ते । धा॒तुः । सर॑स्वत्यै । शारिः॑ । श्ये॒ता । पु॒रु॒ष॒वागिति॑ पुरुष-वाक् । सर॑स्वते । शुकः॑ । श्ये॒तः । पु॒रु॒ष॒वागिति॑ पुरुष-वाक् । आ॒र॒ण्यः । अ॒जः । न॒कु॒लः । शका᳚ । ते । पौ॒ष्णाः । वा॒चे । क्रौ॒ञ्चः ॥ \textbf{  50} \newline
                  \newline
                      (म॒यु - स्त्रयो॑विꣳशतिः)  \textbf{(A12)} \newline \newline
                                \textbf{ TS 5.5.13.1} \newline
                  अ॒पाम् । नप्त्रे᳚ । ज॒षः । ना॒क्रः । मक॑रः । कु॒ली॒कयः॑ । ते । अकू॑पारस्य । वा॒चे । पै॒ङ्ग॒रा॒ज इति॑ पैङ्ग-रा॒जः । भगा॑य । कु॒षीत॑कः । आ॒ती । वा॒ह॒सः । दर्वि॑दा । ते । वा॒य॒व्याः᳚ । दि॒ग्भ्य इति॑ दिक्-भ्यः । च॒क्र॒वा॒कः ॥ \textbf{  51} \newline
                  \newline
                      (अ॒पा - मेका॒न्नविꣳ॑श॒तिः)  \textbf{(A13)} \newline \newline
                                \textbf{ TS 5.5.14.1} \newline
                  बला॑य । अ॒ज॒ग॒रः । आ॒खुः । सृ॒ज॒या । श॒यण्ड॑कः । ते । मै॒त्राः । मृ॒त्यवे᳚ । अ॒सि॒तः । म॒न्यवे᳚ । स्व॒ज इति॑ स्व - जः । कु॒भीं॒नस॒ इति॑ कुंभी - नसः॑ । पु॒ष्क॒र॒सा॒द इति॑ पुष्कर - सा॒दः । लो॒हि॒ता॒हिरिति॑ लोहित-अ॒हिः । ते । त्वा॒ष्ट्राः । प्र॒ति॒श्रुत्का॑या॒ इति॑ प्रति - श्रुत्का॑यै । वा॒ह॒सः ॥ \textbf{  52 } \newline
                  \newline
                      (बला॑या॒ - ष्टाद॑श)  \textbf{(A14)} \newline \newline
                                \textbf{ TS 5.5.15.1} \newline
                  पु॒रु॒ष॒मृ॒ग इति॑ पुरुष - मृ॒गः । च॒न्द्रम॑से । गो॒धा । काल॑का । दा॒र्वा॒घा॒ट इति॑ दारु-आ॒घा॒तः । ते । वन॒स्पती॑नाम् । ए॒णी । अह्ने᳚ । कृष्णः॑ । रात्रि॑यै । पि॒कः । क्ष्विङ्काः᳚ । नील॑शी॒र्ष्णीति॒ नील॑ - शी॒र्ष्णी॒ । ते । अ॒र्य॒म्णे । धा॒तुः । क॒त्क॒टः ॥ \textbf{  53} \newline
                  \newline
                      (पु॒रु॒ष॒मृ॒गो᳚-ऽष्टाद॑श)  \textbf{(A15)} \newline \newline
                                \textbf{ TS 5.5.16.1} \newline
                  सौ॒री । ब॒लाका᳚ । ऋश्यः॑ । म॒यूरः॑ । श्ये॒नः । ते । ग॒न्ध॒र्वाणा᳚म् । वसू॑नाम् । क॒पिञ्ज॑लः । रु॒द्राणा᳚म् । ति॒त्ति॒रिः । रो॒हित् । कु॒ण्डृ॒णाची᳚ । गो॒लत्ति॑का । ताः । अ॒फ्स॒रसा᳚म् । अर॑ण्याय । सृ॒म॒रः ॥ \textbf{  54 } \newline
                  \newline
                      (सौ॒-र्य॑ष्टाद॑श)  \textbf{(A16)} \newline \newline
                                \textbf{ TS 5.5.17.1} \newline
                  पृ॒ष॒तः । वै॒श्व॒दे॒व इति॑ वैश्व - दे॒वः । पि॒त्वः । न्यङ्कुः॑ । कशः॑ । ते । अनु॑मत्या॒ इत्यनु॑ - म॒त्यै॒ । अ॒न्य॒वा॒प इत्य॑न्य - वा॒पः । अ॒द्‌र्ध॒मा॒साना॒मित्य॑द्‌र्ध - मा॒साना᳚म् । मा॒साम् । क॒श्यपः॑ । क्वयिः॑ । कु॒टरुः॑ । दा॒त्यौ॒हः । ते । सि॒नी॒वा॒ल्यै । बृह॒स्पत॑ये । शि॒त्पु॒टः ॥ \textbf{  55 } \newline
                  \newline
                      (पृ॒षता᳚- ऽष्टाद॑श)  \textbf{(A17)} \newline \newline
                                \textbf{ TS 5.5.18.1} \newline
                  शका᳚ । भौ॒मी । पा॒न्त्रः । कशः॑ । मा॒न्थी॒लवः॑ । ते । पि॒तृ॒णाम् । ऋ॒तू॒नाम् । जह॑का । सं॒ॅव॒थ्स॒रायेति॑ सं-व॒थ्स॒राय॑ । लोपा᳚ । क॒पोतः॑ । उलू॑कः । श॒शः । ते । नै॒र्.॒ऋ॒ता इति॑ नैः - ऋ॒ताः । कृ॒क॒वाकुः॑ । सा॒वि॒त्रः ॥ \textbf{  56 } \newline
                  \newline
                      (शका॒ - ऽष्टाद॑श )  \textbf{(A18)} \newline \newline
                                \textbf{ TS 5.5.19.1} \newline
                  रुरुः॑ । रौ॒द्रः । कृ॒क॒ला॒सः । श॒कुनिः॑ । पिप्प॑का । ते । श॒र॒व्या॑यै । ह॒रि॒णः । मा॒रु॒तः । ब्रह्म॑णे । शा॒र्गः । त॒रक्षुः॑ । कृ॒ष्णः । श्वा । च॒तु॒र॒क्ष इति॑ चतुः - अ॒क्षः । ग॒र्द॒भः । ते । इ॒त॒र॒ज॒नाना॒मिती॑तर-ज॒नाना᳚म् । अ॒ग्नये᳚ । धूंक्ष्णा᳚ ॥ \textbf{  57 } \newline
                  \newline
                      (रुरु॑ - र्विꣳश॒तिः)  \textbf{(A19)} \newline \newline
                                \textbf{ TS 5.5.20.1} \newline
                  अ॒ल॒जः । आ॒न्त॒रि॒क्षः । उ॒द्रः । म॒द्गुः । प्ल॒वः । ते । अ॒पाम् । अदि॑त्यै । हꣳ॒॒स॒साचि॒रिति॑ हꣳस - साचिः॑ । इ॒न्द्रा॒ण्यै । कीर्.शा᳚ । गृध्रः॑ । शि॒ति॒क॒क्षीति॑ शिति - क॒क्षी । वा॒द्‌र्ध्रा॒ण॒सः । ते । दि॒व्याः । द्या॒वा॒पृ॒थि॒व्येति॑ द्यावा - पृ॒थि॒व्या᳚ । श्वा॒विदिति॑ श्व - वित् ॥ \textbf{  58 } \newline
                  \newline
                      (अ॒ल॒जो᳚ - ऽष्टाद॑श )  \textbf{(A20)} \newline \newline
                                \textbf{ TS 5.5.21.1} \newline
                  सु॒प॒र्ण इति॑ सु - प॒र्णः । पा॒र्ज॒न्यः । हꣳ॒॒सः । वृकः॑ । वृ॒ष॒दꣳ॒॒शः । ते । ऐ॒न्द्राः । अ॒पाम् । उ॒द्रः । अ॒र्य॒म्णे । लो॒पा॒शः । सिꣳ॒॒हः । न॒कु॒लः । व्या॒घ्रः । ते । म॒हे॒न्द्रायेति॑ महा - इ॒न्द्राय॑ । कामा॑य । पर॑स्वान् ॥ \textbf{  59} \newline
                  \newline
                      (सु॒प॒णो᳚ - ऽष्टाद॑श)  \textbf{(A21)} \newline \newline
                                \textbf{ TS 5.5.22.1} \newline
                  आ॒ग्ने॒यः । कृ॒ष्णग्री॑व॒ इति॑ कृ॒ष्ण - ग्री॒वः॒ । सा॒र॒स्व॒ती । मे॒षी । ब॒भ्रुः । सौ॒म्यः । पौ॒ष्णः । श्या॒मः । शि॒ति॒पृ॒ष्ठ इति॑ शिति -पृ॒ष्ठः । बा॒र्.॒ह॒स्प॒त्यः । शि॒ल्पः । वै॒श्व॒दे॒व इति॑ वैश्व - दे॒वः । ऐ॒न्द्रः । अ॒रु॒णः । मा॒रु॒तः । क॒ल्माषः॑ । ऐ॒न्द्रा॒ग्न इत्यै᳚न्द्र - अ॒ग्नः । सꣳ॒॒हि॒त इति॑ सं - हि॒तः । अ॒धोरा॑म॒ इत्य॒धः - रा॒मः॒ । सा॒वि॒त्रः । वा॒रु॒णः । पेत्वः॑ ॥ \textbf{  60} \newline
                  \newline
                      (आ॒ग्ने॒यो - द्वाविꣳ॑शतिः)  \textbf{(A22)} \newline \newline
                                \textbf{ TS 5.5.23.1} \newline
                  अश्वः॑ । तू॒प॒रः । गो॒मृ॒ग इति॑ गो - मृ॒गः । ते । प्रा॒जा॒प॒त्या इति॑ प्राजा - प॒त्याः । आ॒ग्ने॒यौ । कृ॒ष्णग्री॑वा॒विति॑ कृ॒ष्ण - ग्री॒वौ॒ । त्वा॒ष्ट्रौ । लो॒म॒श॒स॒क्थाविति॑ लोमश - स॒क्थौ । शि॒ति॒पृ॒ष्ठाविति॑ शिति - पृ॒ष्ठौ । बा॒र्.॒ह॒स्प॒त्यौ । धा॒त्रे । पृ॒षो॒द॒र इति॑ पृष - उ॒द॒रः । सौ॒र्यः । ब॒लक्षः॑ । पेत्वः॑ ॥ \textbf{  61 } \newline
                  \newline
                      (अश्वः॒ - षोड॑श)  \textbf{(A23)} \newline \newline
                                \textbf{ TS 5.5.24.1} \newline
                  अ॒ग्नये᳚ । अनी॑कवत॒ इत्यनी॑क-व॒ते॒ । रोहि॑ताञ्जि॒रिति॒ रोहि॑त-अ॒ञ्जिः॒ । अ॒न॒ड्वान् । अ॒धोरा॑मा॒वित्य॒धः - रा॒मौ॒ । सा॒वि॒त्रौ । पौ॒ष्णौ । र॒ज॒तना॑भी॒ इति॑ रज॒त - ना॒भी॒ । वै॒श्व॒दे॒वाविति॑ वैश्व-दे॒वौ । पि॒शङ्गौ᳚ । तू॒प॒रौ । मा॒रु॒तः । क॒ल्माषः॑ । आ॒ग्ने॒यः । कृ॒ष्णः । अ॒जः । सा॒र॒स्व॒ती । मे॒षी । वा॒रु॒णः । कृ॒ष्णः । एक॑शितिपा॒दित्येक॑-शि॒ति॒पा॒त् । पेत्वः॑ ॥ \textbf{  62} \newline
                  \newline
                      (अ॒ग्नये॒ - द्वाविꣳ॑शतिः)  \textbf{(A24)} \newline \newline
\textbf{praSna korvai with starting padams of 1 to 24 anuvAkams :-} \newline
(यदेके॑न - प्र॒जाप॑तिः प्रे॒णाऽनु॒ - यजु॒षा - ऽऽपो॑ - वि॒श्वक॒र्मा - ऽग्न॒ आ या॑हि - सुव॒र्गाय॒ वज्रो॑ - गाय॒त्रेणा - ग्न॑ उदधे - स॒मीची - न्द्रा॑य - म॒यु - र॒पां - बला॑य - पुरुषमृ॒गः - सौ॒री - पृ॑ष॒तः - शका॒ - रुरु॑ - रल॒जः - सु॑प॒र्ण - आ᳚ग्ने॒यो - ऽश्वो॒ - ऽग्नयेऽनी॑कवते॒ - चतु॑र्विꣳशतिः) \newline

\textbf{korvai with starting padams of1, 11, 21 series of pa~jcAtis :-} \newline
(यदेके॑न॒ - स पापी॑या - ने॒तद्वा अ॒ग्ने - र्धनु॒स्तद् - दे॒वास्त्वेन्द्र॑ज्येष्ठा - अ॒पां नप्त्रे - ऽश्व॑स्तूप॒रो - द्विष॑ष्टिः) \newline

\textbf{first and last padam of fifth praSnam of 5th kANDam} \newline
(यदेके॒ - नैक॑शितिपा॒त् पेत्वः॑) \newline 


॥ हरिः॑ ॐ ॥
॥ कृष्ण यजुर्वेदीय तैत्तिरीय संहितायां पञ्चमकाण्डे पञ्चमः प्रश्नः समाप्तः ॥
------------------------------------ \newline
\pagebreak
\pagebreak
        


\end{document}
