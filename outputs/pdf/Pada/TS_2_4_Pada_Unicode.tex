\documentclass[17pt]{extarticle}
\usepackage{babel}
\usepackage{fontspec}
\usepackage{polyglossia}
\usepackage{extsizes}



\setmainlanguage{sanskrit}
\setotherlanguages{english} %% or other languages
\setlength{\parindent}{0pt}
\pagestyle{myheadings}
\newfontfamily\devanagarifont[Script=Devanagari]{AdishilaVedic}


\newcommand{\VAR}[1]{}
\newcommand{\BLOCK}[1]{}




\begin{document}
\begin{titlepage}
    \begin{center}
 
\begin{sanskrit}
    { \Large
    ॐ नमः परमात्मने, श्री महागणपतये नमः, श्री गुरुभ्यो नमः
ह॒रिः॒ ॐ 
    }
    \\
    \vspace{2.5cm}
    \mbox{ \Huge
    2.4      द्वितीयकाण्डे चतुर्थः प्रश्नः - इष्टिविधानं   }
\end{sanskrit}
\end{center}

\end{titlepage}
\tableofcontents

ॐ नमः परमात्मने, श्री महागणपतये नमः, 
श्री गुरुभ्यो नमः, ह॒रिः॒ ॐ \newline
2.4      द्वितीयकाण्डे चतुर्थः प्रश्नः - इष्टिविधानं \newline

\addcontentsline{toc}{section}{ 2.4      द्वितीयकाण्डे चतुर्थः प्रश्नः - इष्टिविधानं}
\markright{ 2.4      द्वितीयकाण्डे चतुर्थः प्रश्नः - इष्टिविधानं \hfill https://www.vedavms.in \hfill}
\section*{ 2.4      द्वितीयकाण्डे चतुर्थः प्रश्नः - इष्टिविधानं }
                                \textbf{ TS 2.4.1.1} \newline
                  दे॒वाः । म॒नु॒ष्याः᳚ । पि॒तरः॑ । ते । अ॒न्यतः॑ । आ॒स॒न्न् । असु॑राः । रक्षाꣳ॑सि । पि॒शा॒चाः । ते । अ॒न्यतः॑ । तेषा᳚म् । दे॒वाना᳚म् । उ॒त । यत् । अल्प᳚म् । लोहि॑तम् । अकु॑र्वन्न् । तत् । रक्षाꣳ॑सि । रात्री॑भि॒रिति॒ रात्रि॑-भिः॒ । अ॒सु॒भ्न॒न्न् । तान् । सु॒ब्धान् । मृ॒तान् । अ॒भि । वीति॑ । औ॒च्छ॒त् । ते । दे॒वाः । अ॒वि॒दुः॒ । यः । वै । नः॒ । अ॒यम् । म्रि॒यते᳚ । रक्षाꣳ॑सि । वै । इ॒मम् । घ्न॒न्ति॒ । इति॑ । ते । रक्षाꣳ॑सि । उपेति॑ । अ॒म॒न्त्र॒य॒न्त॒ । तानि॑ । अ॒ब्रु॒व॒न्न् । वर᳚म् । वृ॒णा॒म॒है॒ । यत् । \textbf{  1} \newline
                  \newline
                                \textbf{ TS 2.4.1.2} \newline
                  असु॑रान् । जया॑म । तत् । नः॒ । स॒ह । अ॒स॒त् । इति॑ । ततः॑ । वै । दे॒वाः । असु॑रान् । अ॒ज॒य॒न्न् । ते । असु॑रान् । जि॒त्वा । रक्षाꣳ॑सि । अपेति॑ । अ॒नु॒द॒न्त॒ । तानि॑ । रक्षाꣳ॑सि । अनृ॑तम् । अ॒क॒र्त॒ । इति॑ । स॒म॒न्तमिति॑ सं - अ॒न्तम् । दे॒वान् । परीति॑ । अ॒वि॒श॒न्न् । ते । दे॒वाः । अ॒ग्नौ । अ॒ना॒थ॒न्त॒ । ते । अ॒ग्नये᳚ । प्रव॑त॒ इति॒ प्र - व॒ते॒ । पु॒रो॒डाश᳚म् । अ॒ष्टाक॑पाल॒मित्य॒ष्टा - क॒पा॒ल॒म् । निरिति॑ । अ॒व॒प॒न्न् । अ॒ग्नये᳚ । वि॒बा॒धव॑त॒ इति॑ विबा॒ध - व॒ते॒ । अ॒ग्नये᳚ । प्रती॑कवत॒ इति॒ प्रती॑क - व॒ते॒ । यत् । अ॒ग्नये᳚ । प्रव॑त॒ इति॒ प्र - व॒ते॒ । नि॒रव॑प॒न्निति॑ निः - अव॑पन्न् । यानि॑ । ए॒व । पु॒रस्ता᳚त् । रक्षाꣳ॑सि । \textbf{  2} \newline
                  \newline
                                \textbf{ TS 2.4.1.3} \newline
                  आसन्न्॑ । तानि॑ । तेन॑ । प्रेति॑ । अ॒नु॒द॒न्त॒ । यत् । अ॒ग्नये᳚ । वि॒बा॒धव॑त॒ इति॑ विबा॒ध - व॒ते॒ । यानि॑ । ए॒व । अ॒भितः॑ । रक्षाꣳ॑सि । आसन्न्॑ । तानि॑ । तेन॑ । वीति॑ । अ॒बा॒ध॒न्त॒ । यत् । अ॒ग्नये᳚ । प्रती॑कवत॒ इति॒ प्रती॑क - व॒ते॒ । यानि॑ । ए॒व । प॒श्चात् । रक्षाꣳ॑सि । आसन्न्॑ । तानि॑ । तेन॑ ।   अपेति॑ । अ॒नु॒द॒न्त॒ । ततः॑ । दे॒वाः । अभ॑वन्न् । परेति॑ । असु॑राः । यः । भ्रातृ॑व्यवा॒निति॒ भ्रातृ॑व्य - वा॒न् । स्यात् । सः । स्पर्द्ध॑मानः । ए॒तया᳚ । इष्ट्या᳚ । य॒जे॒त॒ । अ॒ग्नये᳚ । प्रव॑त॒ इति॒ प्र - व॒ते॒ । पु॒रो॒डाश᳚म् । अ॒ष्टाक॑पाल॒मित्य॒ष्टा - क॒पा॒ल॒म् । निरिति॑ । व॒पे॒त् । अ॒ग्नये᳚ । वि॒बा॒धव॑त॒ इति॑ विबा॒ध - व॒ते॒ । \textbf{  3} \newline
                  \newline
                                \textbf{ TS 2.4.1.4} \newline
                  अ॒ग्नय᳚ । प्रती॑कवत॒ इति॒ प्रती॑क - व॒ते॒ । यत् । अ॒ग्नये᳚ । प्रव॑त॒ इति॒ प्र - व॒ते॒ । नि॒र्वप॒तीति॑ निः - वप॑ति । यः । ए॒व । अ॒स्मा॒त् । श्रेयान्॑ । भ्रातृ॑व्यः । तम् । तेन॑ । प्रेति॑ । नु॒द॒ते॒ । यत् । अ॒ग्नये᳚ । वि॒बा॒धव॑त॒ इति॑ विबा॒ध - व॒ते॒ । यः । ए॒व । ए॒ने॒न॒ । स॒दृङ्ङिति॑ स - दृङ् । तम् । तेन॑ । वीति॑ । बा॒ध॒ते॒ । यत् । अ॒ग्नये᳚ । प्रती॑कवत॒ इति॒ प्रती॑क - व॒ते॒ । यः । ए॒व । अ॒स्मा॒त् । पापी॑यान् । तम् । तेन॑ । अपेति॑ । नु॒द॒ते॒ । प्रेति॑ । श्रेयाꣳ॑सम् । भ्रातृ॑व्यम् । नु॒द॒ते॒ । अतीति॑ । स॒दृश᳚म् । क्रा॒म॒ति॒ । न । ए॒न॒म् । पापी॑यान् । आ॒प्नो॒ति॒ । यः । ए॒वम् ( ) । वि॒द्वान् । ए॒तया᳚ । इष्ट्या᳚ । यज॑ते ॥ \textbf{  4 } \newline
                  \newline
                      (वृ॒णा॒म॒है॒ यत् - पु॒रस्ता॒द् रक्षाꣳ॑सि- वपेद॒ग्नये॑ विबा॒धव॑त - ए॒वं - च॒त्वारि॑ च)  \textbf{(A1)} \newline \newline
                                \textbf{ TS 2.4.2.1} \newline
                  दे॒वा॒सु॒रा इति॑ देव - अ॒सु॒राः । संॅय॑त्ता॒ इति॒ सं - य॒त्ताः॒ । आ॒स॒न्न् । ते । दे॒वाः । अ॒ब्रु॒व॒न्न् । यः । नः॒ । वी॒र्या॑वत्तम॒ इति॑ वी॒र्या॑वत् - त॒मः॒ । तम् । अन्विति॑ । स॒मार॑भामहा॒ इति॑ सं - आर॑भामहै । इति॑ । ते । इन्द्र᳚म् । अ॒ब्रु॒व॒न्न् । त्वम् । वै । नः॒ । वी॒र्या॑वत्तम॒ इति॑ वी॒र्या॑वत् - त॒मः॒ । अ॒सि॒ । त्वाम् । अन्विति॑ । स॒मार॑भामहा॒ इति॑ सं-आर॑भामहै । इति॑ । सः । अ॒ब्र॒वी॒त् । ति॒स्रः । मे॒ । इ॒माः । त॒नुवः॑ । वी॒र्या॑वती॒रिति॑ वी॒र्य॑ - व॒तीः॒ । ताः । प्री॒णी॒त॒ । अथ॑ । असु॑रान् । अ॒भीति॑ । भ॒वि॒ष्य॒थ॒ । इति॑ । ताः । वै । ब्रू॒हि॒ । इति॑ । अ॒ब्रु॒व॒न्न् । इ॒यम् । अꣳ॒॒हो॒मुगित्यꣳ॑हः -मुक् । इ॒यम् । वि॒मृ॒धेति॑ वि - मृ॒धा । इ॒यम् । इ॒न्द्रि॒याव॒तीती᳚न्द्रि॒य - व॒ती॒ । \textbf{  5} \newline
                  \newline
                                \textbf{ TS 2.4.2.2} \newline
                  इति॑ । अ॒ब्र॒वी॒त् । ते । इन्द्रा॑य । अꣳ॒॒हो॒मुच॒ इत्यꣳ॑हः - मुचे᳚ । पु॒रो॒डाश᳚म् । एका॑दशकपाल॒मित्येका॑दश - क॒पा॒ल॒म् । निरिति॑ । अ॒व॒प॒न्न् । इन्द्रा॑य । वै॒मृ॒धाय॑ । इन्द्रा॑य । इ॒न्द्रि॒याव॑त॒ इती᳚न्द्रि॒य - व॒ते॒ । यत् । इन्द्रा॑य । अꣳ॒॒हो॒मुच॒ इत्यꣳ॑हः - मुचे᳚ । नि॒रव॑प॒न्निति॑ निः - अव॑पन्न् । अꣳह॑सः । ए॒व । तेन॑ । अ॒मु॒च्य॒न्त॒ । यत् । इन्द्रा॑य । वै॒मृ॒धाय॑ । मृधः॑ । ए॒व । तेन॑ । अपेति॑ । अ॒घ्न॒त॒ । यत् । इन्द्रा॑य । इ॒न्द्रि॒याव॑त॒ इती᳚न्द्रि॒य - व॒ते॒ । इ॒न्द्रि॒यम् । ए॒व । तेन॑ । आ॒त्मन्न् । अ॒द॒ध॒त॒ । त्रय॑स्त्रिꣳशत्कपाल॒मिति॒ त्रय॑स्त्रिꣳशत् - क॒पा॒ल॒म् । पु॒रो॒डाश᳚म् । निरिति॑ । अ॒व॒प॒न्न् । त्रय॑स्त्रिꣳश॒दिति॒ त्रयः॑ - त्रिꣳ॒॒श॒त् । वै । दे॒वताः᳚ । ताः । इन्द्रः॑ । आ॒त्मन्न् । अन्विति॑ । स॒मार॑भंय॒तेति॑ सं - आर॑भंयत । भूत्यै᳚ । \textbf{  6} \newline
                  \newline
                                \textbf{ TS 2.4.2.3} \newline
                  ताम् । वाव । दे॒वाः । विजि॑ति॒मिति॒ वि - जि॒ति॒म् । उ॒त्त॒मामित्यु॑त् - त॒माम् । असु॑रैः । वीति॑ । अ॒ज॒य॒न्त॒ । यः । भ्रातृ॑व्यवा॒निति॒ भ्रातृ॑व्य - वा॒न् । स्यात् । सः । स्पर्द्ध॑मानः । ए॒तया᳚ । इष्ट्या᳚ । य॒जे॒त॒ । इन्द्रा॑य । अꣳ॒॒हो॒मुच॒ इत्यꣳ॑हः - मुचे᳚ । पु॒रो॒डाश᳚म् । एका॑दशकपाल॒मित्येका॑दश - क॒पा॒ल॒म् । निरिति॑ । व॒पे॒त् । इन्द्रा॑य । वै॒मृ॒धाय॑ । इन्द्रा॑य । इ॒न्द्रि॒याव॑त॒ इती᳚न्द्रि॒य - व॒ते॒ । अꣳह॑सा । वै । ए॒षः । गृ॒ही॒तः । यस्मा᳚त् । श्रेयान्॑ । भ्रातृ॑व्यः । यत् । इन्द्रा॑य । अꣳ॒॒हो॒मुच॒ इत्यꣳ॑हः - मुचे᳚ । नि॒र्वप॒तीति॑ निः - वप॑ति । अꣳह॑सः । ए॒व । तेन॑ । मु॒च्य॒ते॒ । मृ॒धा । वै । ए॒षः । अ॒भिष॑ण्ण॒ इत्य॒भि - स॒न्नः॒ । यस्मा᳚त् । स॒मा॒नेषु॑ । अ॒न्यः । श्रेयान॑ । उ॒त । \textbf{  7} \newline
                  \newline
                                \textbf{ TS 2.4.2.4} \newline
                  अभ्रा॑तृव्यः । यत् । इन्द्रा॑य । वै॒मृ॒धाय॑ । मृधः॑ । ए॒व । तेन॑ । अपेति॑ । ह॒ते॒ । यत् । इन्द्रा॑य । इ॒न्द्रि॒याव॑त॒ इती᳚न्द्रि॒य - व॒ते॒ । इ॒न्द्रि॒यम् । ए॒व । तेन॑ । आ॒त्मन्न् । ध॒त्ते॒ । त्रय॑स्त्रिꣳशत्कपाल॒मिति॒ त्रय॑स्त्रिꣳशत् - क॒पा॒ल॒म् । पु॒रो॒डाश᳚म् । निरिति॑ । व॒प॒ति॒ । त्रय॑स्त्रिꣳश॒दिति॒ त्रयः॑ - त्रिꣳ॒॒श॒त् । वै । दे॒वताः᳚ । ताः । ए॒व । यज॑मानः । आ॒त्मन्न् । अन्विति॑ । स॒मार॑भंयत॒ इति॑ सं - आर॑भंयते । भूत्यै᳚ । सा । वै । ए॒षा । विजि॑ति॒रिति॒ वि - जि॒तिः॒ । नाम॑ । इष्टिः॑ । यः । ए॒वम् । वि॒द्वान् । ए॒तया᳚ । इष्ट्या᳚ । यज॑ते । उ॒त्त॒मामित्यु॑त् - त॒माम् । ए॒व । विजि॑ति॒मिति॒ वि - जि॒ति॒म् । भ्रातृ॑व्येण । वीति॑ । ज॒य॒ते॒ ॥ \textbf{  8} \newline
                  \newline
                       \textbf{(A2)} \newline \newline
                                \textbf{ TS 2.4.3.1} \newline
                  दे॒वा॒सु॒रा इति॑ देव - अ॒सु॒राः । संॅय॑त्ता॒ इति॒ सं - य॒त्ताः॒ । आ॒स॒न्न् । तेषा᳚म् । गा॒य॒त्री । ओजः॑ । बल᳚म् । इ॒न्द्रि॒यम् । वी॒र्य᳚म् । प्र॒जामिति॑ प्र - जाम् । प॒शून् । स॒गृंह्येति॑ सं - गृह्य॑ । आ॒दायेत्या᳚ - दाय॑ । अ॒प॒क्रम्येत्य॑प - क्रम्य॑ । अ॒ति॒ष्ठ॒त् । ते । अ॒म॒न्य॒न्त॒ । य॒त॒रान् । वै । इ॒यम् । उ॒पा॒व॒र्थ्स्यतीत्यु॑प - आ॒व॒र्थ्स्यति॑ । ते । इ॒दम् । भ॒वि॒ष्य॒न्ति॒ । इति॑ । ताम् । वीति॑ । अ॒ह्व॒य॒न्त॒ । विश्व॑कर्म॒न्निति॒ विश्व॑ - क॒र्म॒न्न् । इति॑ । दे॒वाः । दाभि॑ । इति॑ । असु॑राः । सा । न । अ॒न्य॒त॒रान् । च॒न । उ॒पाव॑र्त॒तेत्यु॑प - आव॑र्तत । ते । दे॒वाः । ए॒तत् । यजुः॑ । अ॒प॒श्य॒न्न् । ओजः॑ । अ॒सि॒ । सहः॑ । अ॒सि॒ । बल᳚म् । अ॒सि॒ । \textbf{  9} \newline
                  \newline
                                \textbf{ TS 2.4.3.2} \newline
                  भ्राजः॑ । अ॒सि॒ । दे॒वाना᳚म् । धाम॑ । नाम॑ । अ॒सि॒ । विश्व᳚म् । अ॒सि॒ । वि॒श्वायु॒रिति॑ वि॒श्व - आ॒युः॒ । सर्व᳚म् । अ॒सि॒ । स॒र्वायु॒रिति॑ स॒र्व - आ॒युः॒ । अ॒भि॒भूरित्य॑भि - भूः । इति॑ । वाव । दे॒वाः । असु॑राणाम् । ओजः॑ । बल᳚म् । इ॒न्द्रि॒यम् । वी॒र्य᳚म् । प्र॒जामिति॑ प्र-जाम् । प॒शून् । अ॒वृ॒ञ्ज॒त॒ । यत् । गा॒य॒त्री । अ॒प॒क्रम्येत्य॑प - क्रम्य॑ । अति॑ष्ठत् । तस्मा᳚त् । ए॒ताम् । गा॒य॒त्री । इति॑ । इष्टि᳚म् । आ॒हुः॒ । सं॒ॅव॒थ्स॒र इति॑ सं - व॒थ्स॒रः । वै । गा॒य॒त्री । सं॒ॅव॒थ्स॒र इति॑ सं - व॒थ्स॒रः । वै । तत् । अ॒प॒क्रम्येत्य॑प - क्रम्य॑ । अ॒ति॒ष्ठ॒त् । यत् । ए॒तया᳚ । दे॒वाः । असु॑राणाम् । ओजः॑ । बल᳚म् । इ॒न्द्रि॒यम् । वी॒र्य᳚म् । \textbf{  10} \newline
                  \newline
                                \textbf{ TS 2.4.3.3} \newline
                  प्र॒जामिति॑ प्र - जाम् । प॒शून् । अवृ॑ञ्जत । तस्मा᳚त् । ए॒ताम् । सं॒ॅव॒र्ग इति॑ सं - व॒र्गः । इति॑ । इष्टि᳚म् । आ॒हुः॒ । यः । भ्रातृ॑व्यवा॒निति॒ भ्रातृ॑व्य - वा॒न् । स्यात् । सः । स्पर्द्ध॑मानः । ए॒तया᳚ । इष्ट्या᳚ । य॒जे॒त॒ । अ॒ग्नये᳚ । सं॒ॅव॒र्गायेति॑ सं - व॒र्गाय॑ । पु॒रो॒डाश᳚म् । अ॒ष्टाक॑पाल॒मित्य॒ष्टा - क॒पा॒ल॒म् । निरिति॑ । व॒पे॒त् । तम् । शृ॒तम् । आस॑न्न॒मित्या - स॒न्न॒म् । ए॒तेन॑ । यजु॑षा । अ॒भीति॑ । मृ॒शे॒त् । ओजः॑ । ए॒व । बल᳚म् । इ॒न्द्रि॒यम् । वी॒र्य᳚म् । प्र॒जामिति॑ प्र - जाम् । प॒शून् । भ्रातृ॑व्यस्य । वृ॒ङ्क्ते॒ । भव॑ति । आ॒त्मना᳚ । परेति॑ । अ॒स्य॒ । भ्रातृ॑व्यः । भ॒व॒ति॒ ॥ \textbf{ } \newline
                  \newline
                      प्र॒जां प॒शूनवृ॑ञ्जत॒ तस्मा॑दे॒ताꣳ सं॑ॅव॒र्ग इतीष्टि॑माहु॒र्यो भ्रातृ॑व्यवा॒न्थ्‌स्याथ् सस्पर्द्ध॑मान ए॒तयेष्‌ट्या॑ यजेता॒ग्नये॑ संॅव॒र्गाय॑ पुरो॒डाश॑म॒ष्टाक॑पालं॒ निर्व॑पे॒त्‌तꣳशृ॒तमास॑न्नमे॒तेन॒ यजु॑षा॒ऽभि मृ॑शे॒दोज॑ ए॒व बल॑मिन्द्रि॒यं ॅवी॒र्यं॑ प्र॒जां प॒शून् भ्रातृ॑व्यस्य वृङ्क्ते॒ भव॑त्या॒त्मना॒ परा᳚स्य॒ भ्रातृ॑व्यो भवति ॥ 11  (बल॑मस्ये॒ - तया॑ दे॒वा असु॑राणा॒मोजो॒ बल॑मिन्द्रि॒यं ॅवी॒र्यं॑ - पञ्च॑चत्वारिꣳशच्च)  \textbf{(A3)} \newline \newline
                                \textbf{ TS 2.4.4.1} \newline
                  प्र॒जाप॑ति॒रिति॑ प्र॒जा - प॒तिः॒ । प्र॒जा इति॑ प्र-जाः । अ॒सृ॒ज॒त॒ । ताः । अ॒स्मा॒त् । सृ॒ष्टाः । परा॑चीः । आ॒य॒न्न् । ताः । यत्र॑ । अव॑सन्न् । ततः॑ । ग॒र्मुत् । उदिति॑ । अ॒ति॒ष्ठ॒त् । ताः । बृह॒स्पतिः॑ । च॒ । अ॒न्ववै॑ता॒मित्य॑नु - अवै॑ताम् । सः । अ॒ब्र॒वी॒त् । बृह॒स्पतिः॑ । अ॒नया᳚ । त्वा॒ । प्रेति॑ । ति॒ष्ठा॒नि॒ । अथ॑ । त्वा॒ । प्र॒जा इति॑ प्र - जाः । उ॒पाव॑र्थ्स्य॒न्तीत्यु॑प - आव॑र्थ्स्यन्ति । इति॑ । तम् । प्रेति॑ । अ॒ति॒ष्ठ॒त् । ततः॑ । वै । प्र॒जाप॑ति॒मिति॑ प्र॒जा - प॒ति॒म् । प्र॒जा इति॑ प्र - जाः । उ॒पाव॑र्त॒न्तेत्यु॑प - आव॑र्तन्त । यः । प्र॒जाका॑म॒ इति॑ प्र॒जा - का॒मः॒ । स्यात् । तस्मै᳚ । ए॒तम् । प्रा॒जा॒प॒त्यमिति॑ प्राजा - प॒त्यम् । गा॒र्मु॒तम् । च॒रुम् । निरिति॑ । व॒पे॒त् । प्र॒जाप॑ति॒मिति॑ प्र॒जा - प॒ति॒म् । \textbf{  12} \newline
                  \newline
                                \textbf{ TS 2.4.4.2} \newline
                  ए॒व । स्वेन॑ । भा॒ग॒धेये॒नेति॑ भाग - धेये॑न । उपेति॑ । धा॒व॒ति॒ । सः । ए॒व । अ॒स्मै॒ । प्र॒जामिति॑ प्र - जाम् । प्रेति॑ । ज॒न॒य॒ति॒ । प्र॒जाप॑ति॒रिति॑ प्र॒जा - प॒तिः॒ । प॒शून् । अ॒सृ॒ज॒त॒ । ते । अ॒स्मा॒त् । सृ॒ष्टाः । परा᳚ञ्चः । आ॒य॒न्न् । ते । यत्र॑ । अव॑सन्न् । ततः॑ । ग॒र्मुत् । उदिति॑ । अ॒ति॒ष्ठ॒त् । तान् । पू॒षा । च॒ । अ॒न्ववै॑ता॒मित्य॑नु - अवै॑ताम् । सः । अ॒ब्र॒वी॒त् । पू॒षा । अ॒नया᳚ । मा॒ । प्रेति॑ । ति॒ष्ठ॒ । अथ॑ । त्वा॒ । प॒शवः॑ । उ॒पाव॑र्थ्स्य॒न्तीत्यु॑प - आव॑र्थ्स्यन्ति । इति॑ । माम् । प्रेति॑ । ति॒ष्ठ॒ । इति॑ । सोमः॑ । अ॒ब्र॒वी॒त् । मम॑ । वै । \textbf{  13} \newline
                  \newline
                                \textbf{ TS 2.4.4.3} \newline
                  अ॒कृ॒ष्ट॒प॒च्यमित्य॑कृष्ट - प॒च्यम् । इति॑ । उ॒भौ । वा॒म् । प्रेति॑ । ति॒ष्ठा॒नि॒ । इति॑ । अ॒ब्र॒वी॒त् । तौ । प्रेति॑ । अ॒ति॒ष्ठ॒त् । ततः॑ । वै । प्र॒जाप॑ति॒मिति॑ प्र॒जा - प॒ति॒म् । प॒शवः॑ । उ॒पाव॑र्त॒न्तेत्यु॑प-आव॑र्तन्त । यः । प॒शुका॑म॒ इति॑ प॒शु - का॒मः॒ । स्यात् । तस्मै᳚ । ए॒तम् । सो॒मा॒पौ॒ष्णमिति॑ सोमा - पौ॒ष्णम् । गा॒र्मु॒तम् । च॒रुम् । निरिति॑ । व॒पे॒त् । सो॒मा॒पू॒षणा॒विति॑ सोमा - पू॒षणौ᳚ । ए॒व । स्वेन॑ । भा॒ग॒धेये॒नेति॑ भाग- धेये॑न । उपेति॑ । धा॒व॒ति॒ । तौ । ए॒व । अ॒स्मै॒ । प॒शून् । प्रेति॑ । ज॒न॒य॒तः॒ । सोमः॑ । वै । रे॒तो॒धा इति॑ रेतः - धाः । पू॒षा । प॒शू॒नाम् । प्र॒ज॒न॒यि॒तेति॑ प्र - ज॒न॒यि॒ता । सोमः॑ । ए॒व । अ॒स्मै॒ । रेतः॑ । दधा॑ति । पू॒षा ( ) । प॒शून् । प्रेति॑ । ज॒न॒य॒ति॒ ॥ \textbf{  14} \newline
                  \newline
                      (व॒पे॒त् प्र॒जाप॑तिं॒ - ॅवै - दधा॑ति पू॒षा - त्रीणि॑ च)  \textbf{(A4)} \newline \newline
                                \textbf{ TS 2.4.5.1} \newline
                  अग्ने᳚ । गोभिः॑ । नः॒ । एति॑ । ग॒हि॒ । इन्दो॒ इति॑ । पु॒ष्ट्या । जु॒ष॒स्व॒ । नः॒ ॥ इन्द्रः॑ । ध॒र्ता । गृ॒हेषु॑ । नः॒ ॥ स॒वि॒ता । यः । स॒ह॒स्रियः॑ । सः । नः॒ । गृ॒हेषु॑ । रा॒र॒ण॒त् ॥ एति॑ । पू॒षा । ए॒तु॒ । एति॑ । वसु॑ ॥ धा॒ता । द॒दा॒तु॒ । नः॒ । र॒यिम् । ईशा॑नः । जग॑तः । पतिः॑ ॥ सः । नः॒ । पू॒र्णेन॑ । वा॒व॒न॒त् ॥ त्वष्टा᳚ । यः । वृ॒ष॒भः । वृषा᳚ । सः । नः॒ । गृ॒हेषु॑ । रा॒र॒ण॒त् ॥ स॒हस्रे॑ण । अ॒युते॑न । च॒ । येन॑ । दे॒वाः । अ॒मृत᳚म् । \textbf{  15} \newline
                  \newline
                                \textbf{ TS 2.4.5.2} \newline
                  दी॒र्घम् । श्रवः॑ । दि॒वि । ऐर॑यन्त ॥ रायः॑ । पो॒ष॒ । त्वम् । अ॒स्मभ्य॒मित्य॒स्म - भ्य॒म् । गवा᳚म् । कु॒ल्मिम् । जी॒वसे᳚ । एति॑ । यु॒व॒स्व॒ ॥ अ॒ग्निः । गृ॒हप॑ति॒रिति॑ गृ॒ह - प॒तिः॒ । सोमः॑ । वि॒श्व॒वनि॒रिति॑ विश्व - वनिः॑ । स॒वि॒ता । सु॒मे॒धा इति॑ सु-मे॒धाः । स्वाहा᳚ ॥ अग्ने᳚ । गृ॒ह॒प॒त॒ इति॑ गृह - प॒ते॒ । यः । ते॒ । घृत्यः॑ । भा॒गः । तेन॑ । सहः॑ । ओजः॑ । आ॒क्रम॑माणा॒येत्या᳚-क्रम॑माणाय । धे॒हि॒ । श्रष्ठ्या᳚त् । प॒थः । मा । यो॒ष॒म् । मू॒र्द्धा । भू॒या॒स॒म् । स्वाहा᳚ ॥ \textbf{  16} \newline
                  \newline
                      (अ॒मृत॑ - म॒ष्टात्रिꣳ॑शच्च)  \textbf{(A5)} \newline \newline
                                \textbf{ TS 2.4.6.1} \newline
                  चि॒त्रया᳚ । य॒जे॒त॒ । प॒शुका॑म॒ इति॑ प॒शु - का॒मः॒ । इ॒यम् । वै । चि॒त्रा । यत् । वै । अ॒स्याम् । विश्व᳚म् । भू॒तम् । अधीति॑ । प्र॒जाय॑त॒ इति॑ प्र - जाय॑ते । तेन॑ । इ॒यम् । चि॒त्रा । यः । ए॒वम् । वि॒द्वान् । चि॒त्रया᳚ । प॒शुका॑म॒ इति॑ प॒शु - का॒मः॒ । यज॑ते । प्रेति॑ । प्र॒जयेति॑ प्र - जया᳚ । प॒शुभि॒रिति॑ प॒शु - भिः॒ । मि॒थु॒नैः । जा॒य॒ते॒ । प्रेति॑ । ए॒व । आ॒ग्ने॒येन॑ । वा॒प॒य॒ति॒ । रेतः॑ । सौ॒म्येन॑ । द॒धा॒ति॒ । रेतः॑ । ए॒व । हि॒तम् । त्वष्टा᳚ । रू॒पाणि॑ । वीति॑ । क॒रो॒ति॒ । सा॒र॒स्व॒तौ । भ॒व॒तः॒ । ए॒तत् । वै । दैव्य᳚म् । मि॒थु॒नम् । दैव्य᳚म् । ए॒व । अ॒स्मै॒ । \textbf{  17} \newline
                  \newline
                                \textbf{ TS 2.4.6.2} \newline
                  मि॒थु॒नम् । म॒द्ध्य॒तः । द॒धा॒ति॒ । पुष्ट्यै᳚ । प्र॒जन॑ना॒येति॑ प्र - जन॑नाय । सि॒नी॒वा॒ल्यै । च॒रुः । भ॒व॒ति॒ । वाक् । वै । सि॒नी॒वा॒ली । पुष्टिः॑ । खलु॑ । वै । वाक् । पुष्टि᳚म् । ए॒व । वाच᳚म् । उपेति॑ । ए॒ति॒ । ऐ॒न्द्रः । उ॒त्त॒म इत्यु॑त् - त॒मः । भ॒व॒ति॒ । तेन॑ । ए॒व । तत् । मि॒थु॒नम् । स॒प्त । ए॒तानि॑ । ह॒वीꣳषि॑ । भ॒व॒न्ति॒ । स॒प्त । ग्रा॒म्याः । प॒शवः॑ । स॒प्त । आ॒र॒ण्याः । स॒प्त । छन्दाꣳ॑सि । उ॒भय॑स्य । अव॑रुद्ध्या॒ इत्यव॑ - रु॒द्ध्यै॒ । अथ॑ । ए॒ताः । आहु॑ती॒रित्या - हु॒तीः॒ । जु॒हो॒ति॒ । ए॒ते । वै । दे॒वाः । पुष्टि॑पतय॒ इति॒ पुष्टि॑ - प॒त॒यः॒ । ते । ए॒व ( ) । अ॒स्मि॒न्न् । पुष्टि᳚म् । द॒ध॒ति॒ । पुष्य॑ति । प्र॒जयेति॑ प्र - जया᳚ । प॒शुभि॒रिति॑ प॒शु - भिः॒ । अथो॒ इति॑ । यत् । ए॒ताः । आहु॑ती॒रित्या - हु॒तीः॒ । जु॒होति॑ । प्रति॑ष्ठित्या॒ इति॒ प्रति॑ - स्थि॒त्यै॒ ॥ \textbf{  18} \newline
                  \newline
                      (अ॒स्मै॒ - त ए॒व - द्वाद॑श च)  \textbf{(A6)} \newline \newline
                                \textbf{ TS 2.4.7.1} \newline
                  मा॒रु॒तम् । अ॒सि॒ । म॒रुता᳚म् । ओजः॑ । अ॒पाम् । धारा᳚म् । भि॒न्धि॒ । र॒मय॑त । म॒रु॒तः॒ । श्ये॒नम् । आ॒यिन᳚म् । मनो॑जवस॒मिति॒ मनः॑ - ज॒व॒स॒म् । वृष॑णम् । सु॒वृ॒क्तिमिति॑ सु - वृ॒क्तिम् ॥ येन॑ । शर्द्धः॑ । उ॒ग्रम् । अव॑सृष्ट॒मित्यव॑ - सृ॒ष्ट॒म् । एति॑ । तत् । अ॒श्वि॒ना॒ । परीति॑ । ध॒त्त॒म् । स्व॒स्ति ॥ पु॒रो॒वा॒त इति॑ पुरः - वा॒तः । वर्.षन्न्॑ । जि॒न्वः । आ॒वृदित्या᳚ - वृत् । स्वाहा᳚ । वा॒ताव॒दिति॑ वा॒त - व॒त् । वर्.षन्न्॑ । उ॒ग्रः । आ॒वृदित्या᳚ - वृत् । स्वाहा᳚ । स्त॒नयन्न्॑ । वर्.षन्न्॑ । भी॒मः । आ॒वृदित्या᳚ - वृत् । स्वाहा᳚ । अ॒न॒श॒नि । अ॒व॒स्फूर्ज॒न्नित्य॑व-स्फूर्जन्न्॑ । दि॒द्युत् । वर्.षन्न्॑ । त्वे॒षः । आ॒वृदित्या᳚ - वृत् । स्वाहा᳚ । अ॒ति॒रा॒त्रमित्य॑ति - रा॒त्रम् । वर्.षन्न्॑ । पू॒र्तिः । आ॒वृदित्या᳚ - वृत् । \textbf{  19} \newline
                  \newline
                                \textbf{ TS 2.4.7.2} \newline
                  स्वाहा᳚ । ब॒हु । ह॒ । अ॒यम् । अ॒वृ॒षा॒त् । इति॑ । श्रु॒तः । आ॒वृदित्या᳚ - वृत् । स्वाहा᳚ । आ॒तप॒तीत्या᳚ - तप॑ति । वर्.षन्न्॑ । वि॒राडिति॑ वि - राट् । आ॒वृदित्या᳚ - वृत् । स्वाहा᳚ । अ॒व॒स्फूर्ज॒न्नित्य॑व - स्फूर्जन्न्॑ । दि॒द्युद् । वर्.षन्न्॑ । भू॒तः । आ॒वृदित्या᳚ - वृत् । स्वाहा᳚ । मान्दाः᳚ । वाशाः᳚ । शुन्ध्यूः᳚ । अजि॑राः ॥ ज्योति॑ष्मतीः । तम॑स्वरीः । उन्द॑तीः । सुफे॑ना॒ इति॒ सु - फे॒नाः॒ ॥ मित्र॑भृत॒ इति॒ मित्र॑ - भृ॒तः॒ । क्षत्र॑भृत॒ इति॒ क्षत्र॑ - भृ॒तः॒ । सुरा᳚ष्ट्रा॒ इति॒ सु - रा॒ष्ट्राः॒ । इ॒ह । मा॒ । अ॒व॒त॒ ॥ वृष्णः॑ । अश्व॑स्य । स॒दांन॒मिति॑ सं - दान᳚म् । अ॒सि॒ । वृष्ट्यै᳚ । त्वा॒ । उपेति॑ । न॒ह्या॒मि॒ ॥ \textbf{  20} \newline
                  \newline
                      (पू॒र्तिरा॒वृद् - द्विच॑त्वारिꣳशच्च)  \textbf{(A7)} \newline \newline
                                \textbf{ TS 2.4.8.1} \newline
                  देवाः᳚ । व॒स॒व्याः॒ । अग्ने᳚ । सो॒म॒ । सू॒र्य॒ ॥ देवाः᳚ । श॒र्म॒ण्याः॒ । मित्रा॑वरु॒णेति॒ मित्रा᳚ - व॒रु॒णा॒ । अ॒र्य॒म॒न्न् ॥ देवाः᳚ । स॒पी॒त॒य॒ इति॑ स - पी॒त॒यः॒ । अपा᳚म् । न॒पा॒त् । आ॒शु॒हे॒म॒न्नित्या॑शु - हे॒म॒न्न् ॥ उ॒द्नः । द॒त्त । उ॒द॒धिमित्यु॑द - धिम् । भि॒न्त॒ । दि॒वः ।   प॒र्जन्या᳚त् । अ॒न्तरि॑क्षात् । पृ॒थि॒व्याः । ततः॑ । नः॒ । वृष्ट्या᳚ । अ॒व॒त॒ ॥ दिवा᳚ । चि॒त् । तमः॑ । कृ॒ण्व॒न्ति॒ । प॒र्जन्ये॑न । उ॒द॒वा॒हेनेत्यु॑द - वा॒हेन॑ ॥ पृ॒थि॒वीम् । यत् । व्यु॒न्दन्तीति॑ वि-उ॒न्दन्ति॑ ॥ एति॑ । यम् । नरः॑ । सु॒दान॑व॒ इति॑ सु-दान॑वः । द॒दा॒शुषे᳚ । दि॒वः । कोश᳚म् । अचु॑च्यवुः ॥ वीति॑ । प॒र्जन्याः᳚ । सृ॒ज॒न्ति॒ । रोद॑सी॒ इति॑ । अन्विति॑ । धन्व॑ना । य॒न्ति॒ । \textbf{  21} \newline
                  \newline
                                \textbf{ TS 2.4.8.2} \newline
                  वृ॒ष्टयः॑ ॥ उदिति॑ । ई॒र॒य॒थ॒ । म॒रु॒तः॒ । स॒मु॒द्र॒तः । यू॒यम् । वृ॒ष्टिम् । व॒र्॒.ष॒य॒थ॒ । पु॒री॒षि॒णः॒ ॥ न । वः॒ । द॒स्राः॒ । उपेति॑ । द॒स्य॒न्ति॒ । धे॒नवः॑ ।   शुभ᳚म् । या॒ताम् । अन्विति॑ । रथाः᳚ । अ॒वृ॒थ्स॒त॒ ॥ सृ॒ज ।  वृ॒ष्टिम् । दि॒वः । एति॑ । अ॒द्भिरित्य॑त् - भिः । स॒मु॒द्रम् । पृ॒ण॒ ॥ अ॒ब्जा इत्य॑प् - जाः । अ॒सि॒ । प्र॒थ॒म॒जा इति॑ प्रथम - जाः । बल᳚म् । अ॒सि॒ । स॒मु॒द्रिय᳚म् ॥ उदिति॑ । न॒भं॒य॒ । पृ॒थि॒वीम् । भि॒न्धि । इ॒दम् । दि॒व्यम् । नभः॑ ॥ उ॒द्नः । दि॒व्यस्य॑ । नः॒ । दे॒हि॒ । ईशा॑नः । वीति॑ । सृ॒ज॒ । दृति᳚म् ॥ ये । दे॒वाः ( ) । दि॒विभा॑गा॒ इति॑ दि॒वि - भा॒गाः॒ । ये । अ॒न्तरि॑क्षभागा॒ इत्य॒न्तरि॑क्ष - भा॒गाः॒ । ये । पृ॒थि॒विभा॑गा॒ इति॑ पृथि॒वि - भा॒गाः॒ ॥ ते । इ॒मम् । य॒ज्ञ्म् । अ॒व॒न्तु॒ । ते । इ॒दम् । क्षेत्र᳚म् । एति॑ । वि॒श॒न्तु॒ । ते । इ॒दम् । क्षेत्र᳚म् । अनु॑ । वीति॑ । वि॒श॒न्तु॒ ॥ \textbf{  22} \newline
                  \newline
                      (य॒न्ति॒ - दे॒वा - विꣳ॑श॒तिश्च॑)  \textbf{(A8)} \newline \newline
                                \textbf{ TS 2.4.9.1} \newline
                  मा॒रु॒तम् । अ॒सि॒ । म॒रुता᳚म् । ओजः॑ । इति॑ । कृ॒ष्णम् । वासः॑ । कृ॒ष्णतू॑ष॒मिति॑ कृ॒ष्ण - तू॒ष॒म् । परीति॑ । ध॒त्ते॒ । ए॒तत् । वै । वृष्ट्यै᳚ । रू॒पम् । सरू॑प॒ इति॒ स - रू॒पः॒ ।  ए॒व । भू॒त्वा । प॒र्जन्य᳚म् । व॒र्.॒ष॒य॒ति॒ । र॒मय॑त । म॒रु॒तः॒ । श्ये॒नम् । आ॒यिन᳚म् । इति॑ । प॒श्चा॒द्वा॒तमिति॑ पश्चात् - वा॒तम् । प्रतीति॑ । मी॒व॒ति॒ । पु॒रो॒वा॒तमिति॑ पुरः - वा॒तम् । ए॒व । ज॒न॒य॒ति॒ । व॒र्॒.षस्य॑ । अव॑रुद्ध्या॒ इत्यव॑ - रु॒द्ध्यै॒ । वा॒त॒ना॒मानीति॑ वात - ना॒मानि॑ । जु॒हो॒ति॒ । वा॒युः । वै । वृष्ट्याः᳚ । ई॒शे॒ ।   वा॒युम् । ए॒व । स्वेन॑ । भा॒ग॒धेये॒नेति॑ भाग- धेये॑न । उपेति॑ । धा॒व॒ति॒ । सः । ए॒व । अ॒स्मै॒ । प॒र्जन्य᳚म् । व॒र्.॒ष॒य॒ति॒ । अ॒ष्टौ । \textbf{  23} \newline
                  \newline
                                \textbf{ TS 2.4.9.2} \newline
                  जु॒हो॒ति॒ । चत॑स्रः । वै । दिशः॑ । चत॑स्रः । अ॒वा॒न्त॒र॒दि॒शा इत्य॑वान्तर - दि॒शाः । दि॒ग्भ्य इति॑ दिक् - भ्यः । ए॒व । वृष्टि᳚म् । सम् । प्रेति॑ । च्या॒व॒य॒ति॒ । कृ॒ष्णा॒जि॒न इति॑ कृष्ण-अ॒जि॒ने । समिति॑ । यौ॒ति॒ । ह॒विः । ए॒व । अ॒कः॒ । अ॒न्त॒र्वे॒दीत्य॑न्तः - वे॒दि । समिति॑ । यौ॒ति॒ । अव॑रुद्ध्या॒ इत्यव॑ - रु॒द्ध्यै॒ । यती॑नाम् । अ॒द्यमा॑नानाम् । शी॒र्॒.षाणि॑ । परेति॑ । अ॒प॒त॒न्न् । ते । ख॒र्जूराः᳚ । अ॒भ॒व॒न्न् । तेषा᳚म् । रसः॑ । ऊ॒र्द्ध्वः । अ॒प॒त॒त् । तानि॑ । क॒रीरा॑णि । अ॒भ॒व॒न्न् ।  सौ॒म्यानि॑ । वै । क॒रीरा॑णि । सौ॒म्या । खलु॑ । वै । आहु॑ति॒रित्या - हु॒तिः॒ । दि॒वः । वृष्टि᳚म् । च्या॒व॒य॒ति॒ । यत् । क॒रीरा॑णि । भव॑न्ति । \textbf{  24} \newline
                  \newline
                                \textbf{ TS 2.4.9.3} \newline
                  सौ॒म्यया᳚ । ए॒व । आहु॒त्येत्या - हु॒त्या॒ । दि॒वः । वृष्टि᳚म् । अवेति॑ । रु॒न्धे॒ । मधु॑षा । समिति॑ । यौ॒ति॒ । अ॒पाम् । वै । ए॒षः । ओष॑धीनाम् । रसः॑ । यत् । मधु॑ । अ॒द्भ्य इत्य॑त् - भ्यः । ए॒व । ओष॑धीभ्य॒ इत्योष॑धि - भ्यः॒ । व॒र्.॒ष॒ति॒ । अथो॒ इति॑ । अ॒द्भ्य इत्य॑त् - भ्यः । ए॒व । ओष॑धीभ्य॒ इत्योष॑धि - भ्यः॒ । वृष्टि᳚म् । नीति॑ । न॒य॒ति॒ । मान्दाः᳚ । वाशाः᳚ । इति॑ । समिति॑ । यौ॒ति॒ । ना॒म॒धेयै॒रिति॑ नाम-धेयैः᳚ । ए॒व । ए॒नाः॒ । अच्छ॑ । ए॒ति॒ । अथो॒ इति॑ । यथा᳚ । ब्रू॒यात् । असौ᳚ ।   एति॑ । इ॒हि॒ । इति॑ । ए॒वम् । ए॒व । ए॒नाः॒ । ना॒म॒धेयै॒रिति॑ नाम-धेयैः᳚ । एति॑ । \textbf{  25} \newline
                  \newline
                                \textbf{ TS 2.4.9.4} \newline
                  च्या॒व॒य॒ति॒ । वृष्णः॑ । अश्व॑स्य । स॒दांन॒मिति॑ सं - दान᳚म् । अ॒सि॒ । वृष्ट्यै᳚ । त्वा॒ । उपेति॑ । न॒ह्या॒मि॒ । इति॑ । आ॒ह॒ । वृषा᳚ । वै । अश्वः॑ । वृषा᳚ । प॒र्जन्यः॑ । कृ॒ष्णः । इ॒व॒ । खलु॑ । वै । भू॒त्वा । व॒र्.॒ष॒ति॒ । रू॒पेण॑ । ए॒व । ए॒न॒म् । समिति॑ । अ॒र्द्ध॒य॒ति॒ । व॒र्॒.षस्य॑ । अव॑रुद्ध्य॒ इत्यव॑ - रु॒द्ध्यै॒ ॥ \textbf{  26} \newline
                  \newline
                      (अ॒ष्टौ - भव॑न्ति - नाम॒धेयै॒रै - का॒न्न त्रिꣳ॒॒शच्च॑)  \textbf{(A9)} \newline \newline
                                \textbf{ TS 2.4.10.1} \newline
                  देवाः᳚ । व॒स॒व्याः॒ । देवाः᳚ । श॒र्म॒ण्याः॒ । देवाः᳚ । स॒पी॒त॒य॒ इति॑ स - पी॒त॒यः॒ । इति॑ । एति॑ । ब॒द्ध्ना॒ति॒ । दे॒वता॑भिः । ए॒व । अ॒न्व॒हमित्य॑नु - अ॒हम् । वृष्टि᳚म् । इ॒च्छ॒ति॒ । यदि॑ । वर्.षे᳚त् । ताव॑ति । ए॒व । हो॒त॒व्य᳚म् । यदि॑ । न । वर्.षे᳚त् । श्वः । भू॒ते । ह॒विः । निरिति॑ । व॒पे॒त् । अ॒हो॒रा॒त्रे इत्य॑हः - रा॒त्रे । वै । मि॒त्रावरु॑णा॒विति॑ मि॒त्रा - वरु॑णौ । अ॒हो॒रा॒त्राभ्या॒मित्य॑हः - रा॒त्राभ्या᳚म् । खलु॑ । वै । प॒र्जन्यः॑ । व॒र्.॒ष॒ति॒ । नक्त᳚म् । वा॒ । हि । दिवा᳚ । वा॒ । वर्.ष॑ति । मि॒त्रावरु॑णा॒विति॑ मि॒त्रा - वरु॑णौ । ए॒व । स्वेन॑ । भा॒ग॒धेये॒नेति॑ भाग - धेये॑न । उपेति॑ । धा॒व॒ति॒ । तौ । ए॒व । अ॒स्मै॒ । \textbf{  27} \newline
                  \newline
                                \textbf{ TS 2.4.10.2} \newline
                  अ॒हो॒रा॒त्राभ्या॒मित्य॑हः-रा॒त्राभ्या᳚म् ।  प॒र्जन्य᳚म् । व॒र्.॒ष॒य॒तः॒ । अ॒ग्नये᳚ । धा॒म॒च्छद॒ इति॑ धाम - छदे᳚ । पु॒रो॒डाश᳚म् । अ॒ष्टाक॑पाल॒मित्य॒ष्टा - क॒पा॒ल॒म् । निरिति॑ । व॒पे॒त् । मा॒रु॒तम् । स॒प्तक॑पाल॒मिति॑ स॒प्त - क॒पा॒ल॒म् । सौ॒र्यम् । एक॑कपाल॒मित्येक॑ - क॒पा॒ल॒म् । अ॒ग्निः । वै । इ॒तः । वृष्टि᳚म् । उदिति॑ । ई॒र॒य॒ति॒ । म॒रुतः॑ । सृ॒ष्टाम् । न॒य॒न्ति॒ । य॒दा । खलु॑ । वै । अ॒सौ । आ॒दि॒त्यः । न्यङ्॑ । र॒श्मिभि॒रिति॑ र॒श्मि - भिः॒ । प॒र्या॒वर्त॑त॒ इति॑ परि - आ॒वर्त॑ते । अथ॑ । व॒र्.॒ष॒ति॒ । धा॒म॒च्छदिति॑ धाम - छत् । इ॒व॒ । खलु॑ । वै । भू॒त्वा ।   व॒र्.॒ष॒ति॒ । ए॒ताः । वै । दे॒वताः᳚ । वृष्ट्याः᳚ । ई॒श॒ते॒ । ताः । ए॒व । स्वेन॑ । भा॒ग॒धेये॒नेति॑ भाग - धेये॑न । उपेति॑ । धा॒व॒ति॒ । ताः । \textbf{  28} \newline
                  \newline
                                \textbf{ TS 2.4.10.3} \newline
                  ए॒व । अ॒स्मै॒ । प॒र्जन्य᳚म् । व॒र्.॒ष॒य॒न्ति॒ । उ॒त । अव॑र्.षिष्यन्न् । वर्.ष॑ति । ए॒व । सृ॒ज ।  वृ॒ष्टिम् । दि॒वः । एति॑ । अ॒द्भिरित्य॑त् - भिः । स॒मु॒द्रम् । पृ॒ण॒ । इति॑ । आ॒ह॒ । इ॒माः । च॒ । ए॒व । अ॒मूः । च॒ । अ॒पः । समिति॑ । अ॒र्द्ध॒य॒ति॒ । अथो॒ इति॑ । आ॒भिः । ए॒व । अ॒मूः । अच्छ॑ । ए॒ति॒ । अ॒ब्जा इत्य॑प् - जाः । अ॒सि॒ । प्र॒थ॒म॒जा इति॑ प्रथम - जाः । बल᳚म् । अ॒सि॒ । स॒मु॒द्रिय᳚म् । इति॑ । आ॒ह॒ । य॒था॒य॒जुरिति॑ यथा - य॒जुः । ए॒व । ए॒तत् । उदिति॑ । न॒भं॒य॒ । पृ॒थि॒वीम् । इति॑ । व॒र्.॒षा॒ह्वामिति॑ वर्.ष - ह्वाम् । जु॒हो॒ति॒ । ए॒षा । वै ( ) । ओष॑धीनाम् । वृ॒ष्टि॒वनि॒रिति॑ वृष्टि - वनिः॑ । तया᳚ । ए॒व । वृष्टि᳚म् । एति॑ । च्या॒व॒य॒ति॒ । ये । दे॒वाः । दि॒विभा॑गा॒ इति॑ दि॒वि - भा॒गाः॒ । इति॑ । कृ॒ष्णा॒जि॒नमिति॑ कृष्ण - अ॒जि॒नम् । अवेति॑ । धू॒नो॒ति॒ । इ॒मे । ए॒व । अ॒स्मै॒ । लो॒काः । प्री॒ताः । अ॒भीष्टा॒ इत्य॒भि - इ॒ष्टाः॒ । भ॒व॒न्ति॒ ॥ \textbf{  29} \newline
                  \newline
                      (अ॒स्मै॒ - धा॒व॒ति॒ ता - वा - एक॑विꣳशतिश्च )  \textbf{(A10)} \newline \newline
                                \textbf{ TS 2.4.11.1} \newline
                  सर्वा॑णि । छन्दाꣳ॑सि । ए॒तस्या᳚म् । इष्ट्या᳚म् । अ॒नूच्या॒नीत्य॑नु - उच्या॑नि । इति॑ । आ॒हुः॒ । त्रि॒ष्टुभः॑ । वै । ए॒तत् । वी॒र्य᳚म् । यत् । क॒कुत् । उ॒ष्णिहा᳚ । जग॑त्यै । यत् । उ॒ष्णि॒ह॒क॒कुभा॒वित्यु॑ष्णिह - क॒कुभौ᳚ । अ॒न्वाहेत्य॑नु - आह॑ । तेन॑ । ए॒व । सर्वा॑णि । छन्दाꣳ॑सि । अवेति॑ । रु॒न्धे॒ । गा॒य॒त्री । वै । ए॒षा । यत् । उ॒ष्णिहा᳚ । यानि॑ । च॒त्वारि॑ । अधीति॑ । अ॒क्षरा॑णि । चतु॑ष्पाद॒ इति॒ चतुः॑ - पा॒दः॒ । ए॒व । ते । प॒शवः॑ । यथा᳚ । पु॒रो॒डाशे᳚ । पु॒रो॒डाशः॑ । अधीति॑ । ए॒वम् । ए॒व । तत् । यत् । ऋ॒चि । अधीति॑ । अ॒क्षरा॑णि । यत् । जग॑त्या । \textbf{  30} \newline
                  \newline
                                \textbf{ TS 2.4.11.2} \newline
                  प॒रि॒द॒द्ध्यादिति॑ परि-द॒द्ध्यात् । अन्त᳚म् । य॒ज्ञ्म् । ग॒म॒ये॒त् । त्रि॒ष्टुभा᳚ । परीति॑ । द॒धा॒ति॒ । इ॒न्द्रि॒यम् । वै । वी॒र्य᳚म् । त्रि॒ष्टुक् । इ॒न्द्रि॒ये । ए॒व । वी॒र्ये᳚ । य॒ज्ञ्म् । प्रतीति॑ । स्था॒प॒य॒ति॒ । न । अन्त᳚म् । ग॒म॒य॒ति॒ । अग्ने᳚ । त्री । ते॒ । वाजि॑ना । त्री । स॒धस्थेति॑ स॒ध - स्था॒ । इति॑ । त्रिव॒त्येति॒ त्रि - व॒त्या॒ । परीति॑ । द॒धा॒ति॒ । स॒रू॒प॒त्वायेति॑ सरूप - त्वाय॑ । सर्वः॑ । वै । ए॒षः । य॒ज्ञ्ः । यत् । त्रै॒धा॒त॒वीय᳚म् । कामा॑य कामा॒येति॒ कामा॑य - का॒मा॒य॒ । प्रेति॑ । यु॒ज्य॒ते॒ । सर्वे᳚भ्यः । हि । कामे᳚भ्यः । य॒ज्ञ्ः । प्र॒यु॒ज्यत॒ इति॑ प्र - यु॒ज्यते᳚ । त्रै॒धा॒त॒वीये॑न । य॒जे॒त॒ । अ॒भि॒चर॒न्नित्य॑भि - चरन्न्॑ । सर्वः॑ । वै । 31(50) \textbf{  31} \newline
                  \newline
                                \textbf{ TS 2.4.11.3} \newline
                  ए॒षः । य॒ज्ञ्ः । यत् । त्रै॒धा॒त॒वीय᳚म् ।   सर्वे॑ण । ए॒व । ए॒न॒म् । य॒ज्ञेन॑ । अ॒भीति॑ । च॒र॒ति॒ । स्तृ॒णु॒ते । ए॒व । ए॒न॒म् । ए॒तया᳚ ।   ए॒व । य॒जे॒त॒ । अ॒भि॒च॒र्यमा॑ण॒ इत्य॑भि-च॒र्यमा॑णः । सर्वः॑ । वै । ए॒षः । य॒ज्ञ्ः । यत् । त्रै॒धा॒त॒वीय᳚म् । सर्वे॑ण । ए॒व । य॒ज्ञेन॑ । य॒ज॒ते॒ । न । ए॒न॒म् । अ॒भि॒चर॒न्नित्य॑भि - चरन्न्॑ । स्तृ॒णु॒ते॒ । ए॒तया᳚ । ए॒व । य॒जे॒त॒ । स॒हस्रे॑ण । य॒क्ष्यमा॑णः । प्रजा॑त॒मिति॒ प्र - जा॒त॒म् । ए॒व । ए॒न॒त् । द॒दा॒ति॒ । ए॒तया᳚ । ए॒व । य॒जे॒त॒ । स॒हस्रे॑ण । ई॒जा॒नः । अन्त᳚म् । वै । ए॒षः । प॒शू॒नाम् । ग॒च्छ॒ति॒ । \textbf{  32} \newline
                  \newline
                                \textbf{ TS 2.4.11.4} \newline
                  यः । स॒हस्रे॑ण । यज॑ते । प्र॒जाप॑ति॒रिति॑ प्र॒जा - प॒तिः॒ । खलु॑ । वै । प॒शून् । अ॒सृ॒ज॒त॒ । तान् । त्रै॒धा॒त॒वीये॑न । ए॒व । अ॒सृ॒ज॒त॒ । यः ।   ए॒वम् । वि॒द्वान् । त्रै॒धा॒त॒वीये॑न । प॒शुका॑म॒ इति॑ प॒शु - का॒मः॒ । यज॑ते । यस्मा᳚त् । ए॒व । योनेः᳚ । प्र॒जाप॑ति॒रिति॑ प्र॒जा-प॒तिः॒ । प॒शून् । असृ॑जत । तस्मा᳚त् । ए॒व । ए॒ना॒न् । सृ॒ज॒ते॒ । उपेति॑ । ए॒न॒म् । उत्त॑र॒मित्युत् - त॒र॒म् । स॒हस्र᳚म् । न॒म॒ति॒ । दे॒वता᳚भ्यः । वै । ए॒षः । एति॑ । वृ॒श्च्य॒ते॒ । यः । य॒क्ष्ये । इति॑ । उ॒क्त्वा । न । यज॑ते । त्रै॒धा॒त॒वीये॑न । य॒जे॒त॒ । सर्वः॑ । वै । ए॒षः । य॒ज्ञ्ः । \textbf{  33} \newline
                  \newline
                                \textbf{ TS 2.4.11.5} \newline
                  यत् । त्रै॒धा॒त॒वीय᳚म् । सर्वे॑ण । ए॒व । य॒ज्ञेन॑ । य॒ज॒ते॒ । न । दे॒वता᳚भ्यः । एति॑ । वृ॒श्च्य॒ते॒ । द्वाद॑शकपाल॒ इति॒ द्वाद॑श-क॒पा॒लः॒ । पु॒रो॒डाशः॑ । भ॒व॒ति॒ । ते । त्रयः॑ । चतु॑ष्कपाला॒ इति॒ चतुः॑-क॒पा॒लाः॒ । त्रि॒ष्ष॒मृ॒द्ध॒त्वायेति॑ त्रिष्षमृद्ध - त्वाय॑ । त्रयः॑ । पु॒रो॒डाशाः᳚ । भ॒व॒न्ति॒ ।   त्रयः॑ । इ॒मे । लो॒काः । ए॒षाम् । लो॒काना᳚म् ।  आप्त्यै᳚ । उत्त॑र उत्तर॒ इत्युत्त॑रः -उ॒त्त॒रः॒ ।  ज्यायान्॑ । भ॒व॒ति॒ । ए॒वम् । इ॒व॒ । हि । इ॒मे । लो॒काः । य॒व॒मय॒ इति॑ यव - मयः॑ । मद्ध्ये᳚ । ए॒तत् ।   वै । अ॒न्तरि॑क्षस्य । रू॒पम् । समृ॑द्ध्या॒ इति॒ सं - ऋ॒द्ध्यै॒ । सर्वे॑षाम् । अ॒भि॒ग॒मय॒न्नित्य॑भि - ग॒मयन्न्॑ । अवेति॑ । द्य॒ति॒ । अछ॑बंट्कार॒मित्यछ॑बंट् - का॒र॒म् । हिर॑ण्यम् । द॒दा॒ति॒ । तेजः॑ । ए॒व । \textbf{  34} \newline
                  \newline
                                \textbf{ TS 2.4.11.6} \newline
                  अवेति॑ । रु॒न्धे॒ । ता॒र्प्यम् । द॒दा॒ति॒ । प॒शून् । ए॒व । अवेति॑ । रु॒न्धे॒ । धे॒नुम् । द॒दा॒ति॒ । आ॒शिष॒ इत्या᳚ - शिषः॑ । ए॒व । अवेति॑ । रु॒न्धे॒ । साम्नः॑ । वै । ए॒षः । वर्णः॑ । यत् । हिर॑ण्यम् । यजु॑षाम् । ता॒र्प्यम् । उ॒क्था॒म॒दाना॒मित्यु॑क्थ - म॒दाना᳚म् । धे॒नुः । ए॒तान् । ए॒व । सर्वान्॑ । वर्णान्॑ । अवेति॑ । रु॒न्धे॒ ॥ \textbf{  35 } \newline
                  \newline
                      (जग॑त्या - ऽभि॒चर॒न्थ् सर्वो॒ वै - ग॑च्छति - य॒ज्ञ् - स्तेज॑ ए॒व - त्रिꣳ॒॒शच्च॑)  \textbf{(A11)} \newline \newline
                                \textbf{ TS 2.4.12.1} \newline
                  त्वष्टा᳚ । ह॒तपु॑त्र॒ इति॑ ह॒त - पु॒त्रः॒ । वीन्द्र॒मिति॒ वि - इ॒न्द्र॒म् । सोम᳚म् । एति॑ । अ॒ह॒र॒त् । तस्मिन्न्॑ । इन्द्रः॑ । उ॒प॒ह॒वमित्यु॑प-ह॒वम् । ऐ॒च्छ॒त॒ । तम् । न । उपेति॑ । अ॒ह्व॒य॒त॒ । पु॒त्रम् । मे॒ । अ॒व॒धीः॒ । इति॑ । सः । य॒ज्ञ्॒वे॒श॒समिति॑ यज्ञ् - वे॒श॒सम् । कृ॒त्वा । प्रा॒सहेति॑ प्र - सहा᳚ । सोम᳚म् । अ॒पि॒ब॒त् । तस्य॑ । यत् । अ॒त्यशि॑ष्य॒तेत्य॑ति - अशि॑ष्यत । तत् । त्वष्टा᳚ । आ॒ह॒व॒नीय॒मित्या᳚ - ह॒व॒नीय᳚म् । उप॑ । प्रेति॑ । अ॒व॒र्त॒य॒त् । स्वाहा᳚ । इन्द्र॑शत्रु॒रितीन्द्र॑ -  श॒त्रुः॒ । व॒र्द्ध॒स्व॒ । इति॑ । सः । याव॑त् । ऊ॒र्द्ध्वः । प॒रा॒विद्ध्य॒तीति॑ परा - विद्ध्य॑ति । ताव॑ति । स्व॒यम् । ए॒व । वीति॑ । अ॒र॒म॒त॒ । यदि॑ । वा॒ । ताव॑त् । प्र॒व॒णमिति॑ प्र - व॒नम् । \textbf{  36} \newline
                  \newline
                                \textbf{ TS 2.4.12.2} \newline
                  आसी᳚त् । यदि॑ । वा॒ । ताव॑त् । अधीति॑ । अ॒ग्नेः । आसी᳚त् ।   सः । सं॒भव॒न्निति॑ सं - भवन्न्॑ । अ॒ग्नीषोमा॒वित्य॒ग्नी - सोमौ᳚ । अ॒भि । समिति॑ । अ॒भ॒व॒त् । सः । इ॒षु॒मा॒त्रमि॑षुमात्र॒मिती॑षुमा॒त्रं - इ॒षु॒मा॒त्र॒म् । विष्वङ्॑ । अ॒व॒र्द्ध॒त॒ । सः । इ॒मान् । लो॒कान् । अ॒वृ॒णो॒त् । यत् । इ॒मान् । लो॒कान् । अवृ॑णोत् । तत् । वृ॒त्रस्य॑ । वृ॒त्र॒त्वमिति॑ वृत्र - त्वम् । तस्मा᳚त् । इन्द्रः॑ । अ॒बि॒भे॒त् । अपीति॑ । त्वष्टा᳚ । तस्मै᳚ । त्वष्टा᳚ । वज्र᳚म् । अ॒सि॒ञ्च॒त् । तपः॑ । वै । सः । वज्रः॑ । आ॒सी॒त् । तम् । उद्य॑न्तु॒मित्युत् -   य॒न्तु॒म् । न । अ॒श॒क्नो॒त् । अथ॑ । वै । तर्.हि॑ । विष्णुः॑ । \textbf{  37} \newline
                  \newline
                                \textbf{ TS 2.4.12.3} \newline
                  अ॒न्या । दे॒वता᳚ । आ॒सी॒त् । सः । अ॒ब्र॒वी॒त् । विष्णो᳚ । एति॑ । इ॒हि॒ । इ॒दम् । एति॑ । ह॒रि॒ष्या॒वः॒ । येन॑ । अ॒यम् । इ॒दम् । इति॑ । सः । विष्णुः॑ । त्रे॒धा । आ॒त्मान᳚म् । वि । नीति॑ । अ॒ध॒त्त॒ । पृ॒थि॒व्याम् । तृती॑यम् । अ॒न्तरि॑क्षे । तृती॑यम् । दि॒वि । तृती॑यम् । अ॒भि॒प॒र्या॒व॒र्तादित्य॑भि - प॒र्या॒व॒र्तात् । हि । अबि॑भेत् । यत् । पृ॒थि॒व्याम् । तृती॑यम् । आसी᳚त् । तेन॑ । इन्द्रः॑ । वज्र᳚म् । उदिति॑ । अ॒य॒च्छ॒त् । विष्ण्व॑नुस्थित॒ इति॒ विष्णु॑ - अ॒नु॒स्थि॒तः॒ । सः । अ॒ब्र॒वी॒त् । मा । मे॒ । प्रेति॑ । हाः॒ । अस्ति॑ । वै । इ॒दम् । \textbf{  38} \newline
                  \newline
                                \textbf{ TS 2.4.12.4} \newline
                  मयि॑ । वी॒र्य᳚म् । तत् । ते॒ । प्रेति॑ । दा॒स्या॒मि॒ । इति॑ ।   तत् । अ॒स्मै॒ । प्रेति॑ । अ॒य॒च्छ॒त् । तत् । प्रतीति॑ । अ॒गृ॒ह्णा॒त् । अधाः᳚ । मा॒ । इति॑ । तत् । विष्ण॑वे । अति॑ । प्रेति॑ । अ॒य॒च्छ॒त् । तत् । विष्णुः॑ । प्रतीति॑ । अ॒गृ॒ह्णा॒त् । अ॒स्मासु॑ । इन्द्रः॑ । इ॒न्द्रि॒यम् । द॒धा॒तु॒ । इति॑ । यत् । अ॒न्तरि॑क्षे । तृती॑यम् । आसी᳚त् । तेन॑ । इन्द्रः॑ । वज्र᳚म् । उदिति॑ । अ॒य॒च्छ॒त् । विष्ण्व॑नुस्थित॒ इति॒ विष्णु॑ - अ॒नु॒स्थि॒तः॒ । सः । अ॒ब्र॒वी॒त् । मा । मे॒ । प्रेति॑ । हाः॒ । अस्ति॑ । वै । इ॒दम् । \textbf{  39} \newline
                  \newline
                                \textbf{ TS 2.4.12.5} \newline
                  मयि॑ । वी॒र्य᳚म् । तत् । ते॒ । प्रेति॑ । दा॒स्या॒मि॒ । इति॑ । तत् । अ॒स्मै॒ । प्रेति॑ । अ॒य॒च्छ॒त् । तत् । प्रतीति॑ । अ॒गृ॒ह्णा॒त् । द्विः । मा॒ । अ॒धाः॒ । इति॑ । तत् । विष्ण॑वे । अति॑ । प्रेति॑ । अ॒य॒च्छ॒त् । तत् । विष्णुः॑ । प्रतीति॑ । अ॒गृ॒ह्णा॒त् । अ॒स्मासु॑ । इन्द्रः॑ । इ॒न्द्रि॒यम् । द॒धा॒तु॒ । इति॑ । यत् । दि॒वि । तृती॑यम् । आसी᳚त् । तेन॑ । इन्द्रः॑ । वज्र᳚म् । उदिति॑ । अ॒य॒च्छ॒त् । विष्ण्व॑नुस्थित॒ इति॒ विष्णु॑ - अ॒नु॒स्थि॒तः॒ । सः । अ॒ब्र॒वी॒त् । मा । मे॒ । प्रेति॑ । हाः॒ । येन॑ । अ॒हम् । \textbf{  40} \newline
                  \newline
                                \textbf{ TS 2.4.12.6} \newline
                  इ॒दम् । अस्मि॑ । तत् । ते॒ । प्रेति॑ । दा॒स्या॒मि॒ । इति॑ । त्वी(3) । इति॑ । अ॒ब्र॒वी॒त् । स॒धांमिति॑ सं - धाम् । तु । समिति॑ । द॒धा॒व॒है॒ । त्वाम् । ए॒व । प्रेति॑ । वि॒शा॒नि॒ । इति॑ । यत् । माम् । प्र॒वि॒शेरिति॑ प्र-वि॒शेः । किम् । मा॒ । भु॒ञ्ज्याः॒ । इति॑ । अ॒ब्र॒वी॒त् । त्वाम् । ए॒व । इ॒न्धी॒य॒ । तव॑ । भोगा॑य । त्वाम् । प्रेति॑ । वि॒शे॒य॒म् । इति॑ । अ॒ब्र॒वी॒त् । तम् । वृ॒त्रः । प्रेति॑ । अ॒वि॒श॒त् । उ॒दर᳚म् । वै । वृ॒त्रः । क्षुत् । खलु॑ । वै । म॒नु॒ष्य॑स्य । भ्रातृ॑व्यः । यः । \textbf{  41} \newline
                  \newline
                                \textbf{ TS 2.4.12.7} \newline
                  ए॒वम् । वेद॑ । हन्ति॑ । क्षुध᳚म् । भ्रातृ॑व्यम् । तत् । अ॒स्मै॒ । प्रेति॑ । अ॒य॒च्छ॒त् । तत् । प्रतीति॑ । अ॒गृ॒ह्णा॒त् । त्रिः । मा॒ । अ॒धाः॒ । इति॑ । तत् । विष्ण॑वे । अति॑ । प्रेति॑ । अ॒य॒च्छ॒त् । तत् । विष्णुः॑ । प्रतीति॑ । अ॒गृ॒ह्णा॒त् । अ॒स्मासु॑ । इन्द्रः॑ । इ॒न्द्रि॒यम् । द॒धा॒तु॒ । इति॑ । यत् । त्रिः । प्रेति॑ । अय॑च्छत् । त्रिः । प्र॒त्यगृ॑ह्णा॒दिति॑ प्रति – अगृ॑ह्णात् । तत् । त्रि॒धातो॒रिति॑ त्रि - धातोः᳚ । त्रि॒धा॒तु॒त्वमिति॑ त्रिधातु - त्वम् । यत् । विष्णुः॑ । अ॒न्वति॑ष्ठ॒तेत्य॑नु - अति॑ष्ठत । विष्ण॑वे । अति॑ । प्रेति॑ । अय॑च्छत् । तस्मा᳚त् । ऐ॒न्द्रा॒वै॒ष्ण॒वमित्यै᳚न्द्रा - वै॒ष्ण॒वम् । ह॒विः । भ॒व॒ति॒ ( ) । यत् । वै । इ॒दम् । किम् । च॒ । तत् । अ॒स्मै॒ । तत् । प्रेति॑ । अ॒य॒च्छ॒त् । ऋचः॑ । सामा॑नि । यजूꣳ॑षि । स॒हस्र᳚म् । वै । अ॒स्मै॒ । तत् । प्रेति॑ । अ॒य॒च्छ॒त् । तस्मा᳚त् । स॒हस्र॑दक्षिण॒मिति॑ स॒हस्र॑ - द॒क्षि॒ण॒म् ॥ \textbf{  42} \newline
                  \newline
                      (प्र॒व॒णं - ॅविष्णु॒- र्वा इ॒द- मि॒द - म॒हं - ॅयो - भ॑व॒ - त्येक॑ विꣳशतिश्च)  \textbf{(A12)} \newline \newline
                                \textbf{ TS 2.4.13.1} \newline
                  दे॒वाः । वै । रा॒ज॒न्या᳚त् । जाय॑मानात् ।   अ॒बि॒भ॒युः॒ । तम् । अ॒न्तः । ए॒व । सन्त᳚म् ।   दाम्ना᳚ । अपेति॑ । औ॒भं॒न्न् । सः । वै । ए॒षः । अपो᳚ब्ध॒ इत्यप॑- उ॒ब्धः॒ । जा॒य॒ते॒ । यत् । रा॒ज॒न्यः॑ । यत् । वै । ए॒षः । अन॑पोब्ध॒ इत्यन॑प - उ॒ब्धः॒ । जाये॑त । वृ॒त्रान् । घ्नन्न् । च॒रे॒त् । यम् । का॒मये॑त । रा॒ज॒न्य᳚म् । अन॑पोब्ध॒ इत्यन॑प - उ॒ब्धः॒ । जा॒ये॒त॒ । वृ॒त्रान् । घ्नन्न् । च॒रे॒त् । इति॑ । तस्मै᳚ । ए॒तम् । ऐ॒न्द्रा॒बा॒र्.॒ह॒स्प॒त्यमित्यै᳚न्द्रा-बा॒र्.॒ह॒स्प॒त्यम् । च॒रुम् । निरिति॑ । व॒पे॒त् ।   ऐ॒न्द्रः । वै । रा॒ज॒न्यः॑ । ब्रह्म॑ । बृह॒स्पतिः॑ । ब्रह्म॑णा । ए॒व । ए॒न॒म् ( ) । दाम्नः॑ । अ॒प्ॐभ॑ना॒दित्य॑प-उभं॑नात् । मु॒ञ्च॒ति॒ । हि॒र॒ण्मय᳚म् । दाम॑ । दक्षि॑णा । सा॒क्षादिति॑ स-अ॒क्षात् । ए॒व । ए॒न॒म् । दाम्नः॑ । अ॒प्ॐभ॑ना॒दित्य॑प - उभं॑नात् । मु॒ञ्च॒ति॒ ॥ \textbf{  43} \newline
                  \newline
                      (ए॒नं॒ - द्वाद॑श च)  \textbf{(A13)} \newline \newline
                                \textbf{ TS 2.4.14.1} \newline
                  नवो॑नव॒ इति॒ नवः॑ - न॒वः॒ । भ॒व॒ति॒ । जाय॑मानः । अह्ना᳚म् । के॒तुः । उ॒षसा᳚म् । ए॒ति॒ । अग्रे᳚ ॥ भा॒गम् । दे॒वेभ्यः॑ । वीति॑ । द॒धा॒ति॒ । आ॒यन्नित्या᳚ - यन्न् । प्रेति॑ । च॒न्द्रमाः᳚ । ति॒र॒ति॒ । दी॒र्घम् । आयुः॑ ॥ यम् । आ॒दि॒त्याः । अꣳ॒॒शुम् । आ॒प्या॒यय॒न्तीत्या᳚ - प्या॒यय॑न्ति । यम् । अक्षि॑तम् । अक्षि॑तयः । पिब॑न्ति ॥ तेन॑ । नः॒ । राजा᳚ । वरु॑णः । बृह॒स्पतिः॑ । एति॑ । प्या॒य॒य॒न्तु॒ । भुव॑नस्य । गो॒पा इति॑ गो - पाः ॥ प्राच्या᳚म् । दि॒शि । त्वम् । इ॒न्द्र॒ । अ॒सि॒ । राजा᳚ । उ॒त । उदी᳚च्याम् । वृ॒त्र॒ह॒न्निति॑ वृत्र - ह॒न्न् । वृ॒त्र॒हेति॑ वृत्र - हा । अ॒सि॒ ॥ यत्र॑ । यन्ति॑ । स्रो॒त्याः । तत् । \textbf{  44} \newline
                  \newline
                                \textbf{ TS 2.4.14.2} \newline
                  जि॒तम् । ते॒ । द॒क्षि॒ण॒तः । वृ॒ष॒भः । ए॒धि॒ । हव्यः॑ ॥ इन्द्रः॑ । ज॒या॒ति॒ । न । परेति॑ । ज॒या॒तै॒ । अ॒धि॒रा॒ज इत्य॑धि - रा॒जः । राज॒स्विति॒ राज॑ - सु॒ । रा॒ज॒या॒ति॒ ॥ विश्वाः᳚ । हि । भू॒याः । पृत॑नाः । अ॒भि॒ष्टीः । उ॒प॒सद्य॒ इत्यु॑प - सद्यः॑ । न॒म॒स्यः॑ । यथा᳚ । अस॑त् ॥ अ॒स्य । इत् । ए॒व । प्रेति॑ । रि॒रि॒चे॒ । म॒हि॒त्वमिति॑ महि - त्वम् । दि॒वः । पृ॒थि॒व्याः । परीति॑ । अ॒न्तरि॑क्षात् ॥ स्व॒राडिति॑ स्व - राट् । इन्द्रः॑ । दमे᳚ । एति॑ । वि॒श्वगू᳚र्त॒ इति॑ वि॒श्व-गू॒र्तः॒ । स्व॒रिः । अम॑त्रः । व॒व॒क्षे॒ ।    रणा॑य ॥ अ॒भीति॑ । त्वा॒ । शू॒र॒ । नो॒नु॒मः॒ । अदु॑ग्धाः । इ॒व॒ । धे॒नवः॑ ॥ ईशा॑नम् । \textbf{  45} \newline
                  \newline
                                \textbf{ TS 2.4.14.3} \newline
                  अ॒स्य । जग॑तः । सु॒व॒र्दृश॒मिति॑ सुवः - दृश᳚म् । ईशा॑नम् । इ॒न्द्र॒ । त॒स्थुषः॑ ॥ त्वाम् । इत् । हि । हवा॑महे । सा॒ता । वाज॑स्य । का॒रवः॑ ॥ त्वाम् । वृ॒त्रेषु॑ । इ॒न्द्र॒ । सत्प॑ति॒मिति॒ सत् - प॒ति॒म् । नरः॑ । त्वाम् । काष्ठा॑सु । अर्व॑तः ॥ यत् । द्यावः॑ । इ॒न्द्र॒ । ते॒ । श॒तम् । श॒तम् । भूमीः᳚ । उ॒त । स्युः ॥ न । त्वा॒ । व॒ज्रि॒न्न् । स॒हस्र᳚म् । सूर्याः᳚ । अन्विति॑ ।   न । जा॒तम् । अ॒ष्ट॒ । रोद॑सी॒ इति॑ ॥ पिब॑ । सोम᳚म् । इ॒न्द्र॒ । मन्द॑तु । त्वा॒ । यम् । ते॒ । सु॒षाव॑ । ह॒र्य॒श्वेति॑ हरि - अ॒श्व॒ । अद्रिः॑ ॥ \textbf{  46} \newline
                  \newline
                                \textbf{ TS 2.4.14.4} \newline
                  सो॒तुः । बा॒हुभ्या॒मिति॑ बा॒हु - भ्या॒म् । सुय॑त॒ इति॒ सु - य॒तः॒ । न । अर्वा᳚ ॥ रे॒वतीः᳚ । नः॒ । स॒ध॒माद॒ इति॑ सध - मादः॑ । इन्द्रे᳚ । स॒न्तु॒ । तु॒विवा॑जा॒ इति॑ तु॒वि - वा॒जाः॒ ॥ क्षु॒मन्तः॑ । याभिः॑ । मदे॑म ॥ उदिति॑ । अ॒ग्ने॒ । शुच॑यः । तव॑ । वीति॑ । ज्योति॑षा । उदिति॑ । उ॒ । त्यम् । जा॒तवे॑दस॒मिति॑ जा॒त - वे॒द॒स॒म् । स॒प्त । त्वा॒ । ह॒रितः॑ । रथे᳚ । वह॑न्ति । दे॒व॒ । सू॒र्य॒ ॥ शो॒चिष्के॑श॒मिति॑ शो॒चिः - के॒श॒म् । वि॒च॒क्ष॒णेति॑ वि -  च॒क्ष॒ण॒ ॥ चि॒त्रम् । दे॒वाना᳚म् । उदिति॑ । अ॒गा॒त् । अनी॑कम् । चक्षुः॑ । मि॒त्रस्य॑ । वरु॑णस्य ।   अ॒ग्नेः ॥ एति॑ । अ॒प्राः॒ । द्यावा॑पृथि॒वी इति॒ द्यावा᳚ - पृ॒थि॒वी । अ॒न्तरि॑क्षम् । सूर्यः॑ । आ॒त्मा ।   जग॑तः । त॒स्थुषः॑ । \textbf{  47} \newline
                  \newline
                                \textbf{ TS 2.4.14.5} \newline
                  च॒ ॥ विश्वे᳚ । दे॒वाः । ऋ॒ता॒वृध॒ इत्यृ॑त - वृधः॑ । ऋ॒तुभि॒रित्यृ॒तु-भिः॒ । ह॒व॒न॒श्रुत॒ इति॑ हवन - श्रुतः॑ ॥ जु॒षन्ता᳚म् । युज्य᳚म् ।    पयः॑ ॥ विश्वे᳚ । दे॒वाः॒ । शृ॒णु॒त । इ॒मम् । हव᳚म् । मे॒ । ये । अ॒न्तरि॑क्षे । ये । उपेति॑ । द्यवि॑ । स्थ ॥ ये । अ॒ग्नि॒जि॒ह्वा इत्य॑ग्नि - जि॒ह्वाः । उ॒त । वा॒ । यज॑त्राः । आ॒सद्येत्या᳚ - सद्य॑ । अ॒स्मिन्न् । ब॒र्॒.हिषि॑ । मा॒द॒य॒द्ध्व॒म् ॥ \textbf{  48} \newline
                  \newline
                      (त - दीशा॑न॒ - मद्रि॑ - स्त॒स्थुष॑ - स्त्रिꣳ॒॒शच्च॑)  \textbf{(A14)} \newline \newline
\textbf{praSna korvai with starting padams of 1 to 14 Anuvaakams :-} \newline
(दे॒वा म॑नु॒ष्या॑ - देवसु॒रा अ॑ब्रुवन् - देवासु॒रास्तेषां᳚ गाय॒त्री - प्र॒जाप॑ति॒स्ता यत्राऽ - ग्ने॒ गोभिः॑ - चि॒त्रया॑ - मारु॒तं - देवा॑ वसव्या॒ अग्ने॑ - मारु॒तमिति॒ - देवा॑ वसव्या॒ देवाः᳚ शर्मण्याः॒ - सर्वा॑णि॒ - त्वष्टा॑ ह॒तपु॑त्रो - दे॒वा वै रा॑ज॒न्या᳚न् - नवो॑नव॒ - श्चतु॑र्दश ) \newline

\textbf{korvai with starting padams of1, 11, 21 series of pa~jcAtis :-} \newline
(दे॒वा म॑नु॒ष्याः᳚ - प्र॒जां प॒शुन् - देवा॑ वसव्याः - परिद॒ध्यदि॒द- मस्‌म्य॒ - ष्टा च॑त्वारिꣳशत् ) \newline

\textbf{first and last padam of fourth praSnam of kANDam 2:-} \newline
(दे॒वा म॑नु॒ष्या॑ - मादयध्वं) \newline 


॥ हरिः॑ ॐ ॥॥ कृष्ण यजुर्वेदीय तैत्तिरीय संहितायां द्वितीयकाण्डे चतुर्थः प्रश्नः समाप्तः ॥ \newline
\pagebreak
2.4.1   Appendix\\2.4.14.4 - उद॑ग्ने शुच॑य॒स्तव॒>1\\उद॑ग्ने॒ शुच॑य॒स्तव॑ शु॒क्रा भ्राज॑न्त ईरते । तव॒ ज्योतीꣳ॑ष्य॒र्चयः॑ ॥ \\(appearing in TS-1.3.14.8 )\\\\2.4.14.4- विज्योति॒षो>2 ,\\वि ज्योति॑षा बृह॒ता भा᳚त्य॒ग्निरा॒वि र्विश्वा॑नि कृणुते महि॒त्वा ।\\प्रादे॑वी र्मा॒याः स॑हते-दु॒रेवाः॒ शिशी॑ते॒ शृङ्गे॒ रक्ष॑से वि॒निक्षे᳚ ॥\\(Appearing in TS-1.2.14.7 )\\\\2.4.14.4 - दु॒त्यं जा॒तवे॑दसꣳ>3\\उदु॒ त्यं जा॒तवे॑दसं दे॒वं ॅव॑हन्ति के॒तवः॑ । \\दृ॒शे विश्वा॑य॒ सूर्यं᳚ ॥ (आप्पॆअरिन्ग् इन्- ट्श् 1.4.43.1 )\\\\=============================\\
\pagebreak
        


\end{document}
