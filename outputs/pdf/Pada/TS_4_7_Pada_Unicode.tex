\documentclass[17pt]{extarticle}
\usepackage{babel}
\usepackage{fontspec}
\usepackage{polyglossia}
\usepackage{extsizes}



\setmainlanguage{sanskrit}
\setotherlanguages{english} %% or other languages
\setlength{\parindent}{0pt}
\pagestyle{myheadings}
\newfontfamily\devanagarifont[Script=Devanagari]{AdishilaVedic}


\newcommand{\VAR}[1]{}
\newcommand{\BLOCK}[1]{}




\begin{document}
\begin{titlepage}
    \begin{center}
 
\begin{sanskrit}
    { \Large
    ॐ नमः परमात्मने, श्री महागणपतये नमः, 
श्री गुरुभ्यो नमः, ह॒रिः॒ ॐ 
    }
    \\
    \vspace{2.5cm}
    \mbox{ \Huge
    4.7     चतुर्थकाण्डे सप्तमः प्रश्नः - वसोर्धारादिशिष्ट संस्काराभिधानं   }
\end{sanskrit}
\end{center}

\end{titlepage}
\tableofcontents

ॐ नमः परमात्मने , श्री महागणपतये नमः , श्री गुरुभ्यो नमः , ह॒रिः॒ ॐ \newline
4.7     चतुर्थकाण्डे सप्तमः प्रश्नः - वसोर्धारादिशिष्ट संस्काराभिधानं \newline

\addcontentsline{toc}{section}{ 4.7     चतुर्थकाण्डे सप्तमः प्रश्नः - वसोर्धारादिशिष्ट संस्काराभिधानं}
\markright{ 4.7     चतुर्थकाण्डे सप्तमः प्रश्नः - वसोर्धारादिशिष्ट संस्काराभिधानं \hfill https://www.vedavms.in \hfill}
\section*{ 4.7     चतुर्थकाण्डे सप्तमः प्रश्नः - वसोर्धारादिशिष्ट संस्काराभिधानं }
                                \textbf{ TS 4.7.1.1} \newline
                  अग्ना॑विष्णू॒ इत्यग्ना᳚ - वि॒ष्णू॒ । स॒जोष॒सेति॑ स - जोष॑सा । इ॒माः । व॒द्‌र्ध॒न्तु॒ । वा॒म् । गिरः॑ ॥ ध्यु॒म्नैः । वाजे॑भिः । एति॑ । ग॒त॒म् ॥ वाजः॑ । च॒ । मे॒ । प्र॒स॒व इति॑ प्र-स॒वः । च॒ । मे॒ । प्रय॑ति॒रिति॒ प्र-य॒तिः॒ । च॒ । मे॒ । प्रसि॑ति॒रिति॒ प्र-सि॒तिः॒ । च॒ । मे॒ । धी॒तिः । च॒ । मे॒ । क्रतुः॑ । च॒ । मे॒ । स्वरः॑ । च॒ । मे॒ । श्लोकः॑ । च॒ । मे॒ । श्रा॒वः । च॒ । मे॒ । श्रुतिः॑ । च॒ । मे॒ । ज्योतिः॑ । च॒ । मे॒ । सुवः॑ । च॒ । मे॒ । प्रा॒ण इति॑ प्र - अ॒नः । च॒ । मे॒ । अ॒पा॒न इत्य॑प - अ॒नः । \textbf{  1} \newline
                  \newline
                                \textbf{ TS 4.7.1.2} \newline
                  च॒ । मे॒ । व्या॒न इति॑ वि-अ॒नः । च॒ । मे॒ । असुः॑ । च॒ । मे॒ । चि॒त्तम् । च॒ । मे॒ । आधी॑त॒मित्या - धी॒त॒म् । च॒ । मे॒ । वाक् । च॒ । मे॒ । मनः॑ । च॒ । मे॒ । चक्षुः॑ । च॒ । मे॒ । श्रोत्र᳚म् । च॒ । मे॒ । दक्षः॑ । च॒ । मे॒ । बल᳚म् । च॒ । मे॒ । ओजः॑ । च॒ । मे॒ । सहः॑ । च॒ । मे॒ । आयुः॑ । च॒ । मे॒ । ज॒रा । च॒ । मे॒ । आ॒त्मा । च॒ । मे॒ । त॒नूः । च॒ । मे॒ ( ) । शर्म॑ । च॒ । मे॒ । वर्म॑ । च॒ । मे॒ । अङ्गा॑नि । च॒ । मे॒ । अ॒स्थानि॑ । च॒ । मे॒ । परूꣳ॑षि । च॒ । मे॒ । शरी॑राणि । च॒ । मे॒ ॥ \textbf{  2} \newline
                  \newline
                      (अ॒पा॒न - स्त॒नूश्च॑ मे॒ - ऽष्टाद॑श च)  \textbf{(A1)} \newline \newline
                                \textbf{ TS 4.7.2.1} \newline
                  ज्यैष्ठ्य᳚म् । च॒ । मे॒ । आधि॑पत्य॒मित्याधि॑-प॒त्य॒म् । च॒ । मे॒ । म॒न्युः । च॒ । मे॒ । भामः॑ । च॒ । मे॒ । अमः॑ । च॒ । मे॒ । अभंः॑ । च॒ । मे॒ । जे॒मा । च॒ । मे॒ । म॒हि॒मा । च॒ । मे॒ । व॒रि॒मा । च॒ । मे॒ । प्र॒थि॒मा । च॒ । मे॒ । व॒र्ष्मा । च॒ । मे॒ । द्रा॒घु॒या । च॒ । मे॒ । वृ॒द्धम् । च॒ । मे॒ । वृद्धिः॑ । च॒ । मे॒ । स॒त्यम् । च॒ । मे॒ । श्र॒द्धेति॑ श्रत् - धा । च॒ । मे॒ । जग॑त् । च॒ । \textbf{  3} \newline
                  \newline
                                \textbf{ TS 4.7.2.2} \newline
                  मे॒ । धन᳚म् । च॒ । मे॒ । वशः॑ । च॒ । मे॒ । त्विषिः॑ । च॒ । मे॒ । क्री॒डा । च॒ । मे॒ । मोदः॑ । च॒ । मे॒ । जा॒तम् । च॒ । मे॒ । ज॒नि॒ष्यमा॑णम् । च॒ । मे॒ । सू॒क्तमिति॑ सु - उ॒क्तम् । च॒ । मे॒ । सु॒कृ॒तमिति॑ सु - कृ॒तम् । च॒ । मे॒ । वि॒त्तम् । च॒ । मे॒ । वेद्य᳚म् । च॒ । मे॒ । भू॒तम् । च॒ । मे॒ । भ॒वि॒ष्यत् । च॒ । मे॒ । सु॒गमिति॑ सु - गम् । च॒ । मे॒ । सु॒पथ॒मिति॑ सु - पथ᳚म् । च॒ । मे॒ । ऋ॒द्धम् । च॒ । मे॒ । ऋद्धिः॑ ( ) । च॒ । मे॒ । क्लृ॒प्तम् । च॒ । मे॒ । क्लृप्तिः॑ । च॒ । मे॒ । म॒तिः । च॒ । मे॒ । सु॒म॒तिरिति॑ सु - म॒तिः । च॒ । मे॒ ॥ \textbf{  4} \newline
                  \newline
                      (जग॒च्च - र्द्धि॒ - श्चतु॑र्दश च)  \textbf{(A2)} \newline \newline
                                \textbf{ TS 4.7.3.1} \newline
                  शम् । च॒ । मे॒ । मयः॑ । च॒ । मे॒ । प्रि॒यम् । च॒ । मे॒ । अ॒नु॒का॒म इत्य॑नु - का॒मः । च॒ । मे॒ । कामः॑ । च॒ । मे॒ । सौ॒म॒न॒सः । च॒ । मे॒ । भ॒द्रम् । च॒ । मे॒ । श्रेयः॑ । च॒ । मे॒ । वस्यः॑ । च॒ । मे॒ । यशः॑ । च॒ । मे॒ । भगः॑ । च॒ । मे॒ । द्रवि॑णम् । च॒ । मे॒ । य॒न्ता । च॒ । मे॒ । ध॒र्ता । च॒ । मे॒ । क्षेमः॑ । च॒ । मे॒ । धृतिः॑ । च॒ । मे॒ । विश्व᳚म् । च॒ । \textbf{  5} \newline
                  \newline
                                \textbf{ TS 4.7.3.2} \newline
                  मे॒ । महः॑ । च॒ । मे॒ । सं॒ॅविदिति॑ सम्-वित् । च॒ । मे॒ । ज्ञात्र᳚म् । च॒ । मे॒ । सूः । च॒ । मे॒ । प्र॒सूरिति॑ प्र - सूः । च॒ । मे॒ । सीरं᳚ । च॒ । मे॒ । ल॒यः । च॒ । मे॒ । ऋ॒तम् । च॒ । मे॒ । अ॒मृत᳚म् । च॒ । मे॒ । अ॒य॒क्ष्मम् । च॒ । मे॒ । अना॑मयत् । च॒ । मे॒ । जी॒वातुः॑ । च॒ । मे॒ । दी॒र्घा॒यु॒त्वमिति॑ दीर्घायु - त्वम् । च॒ । मे॒ । अ॒न॒मि॒त्रम् । च॒ । मे॒ । अभ॑यम् । च॒ । मे॒ । सु॒गमिति॑ सु-गम् । च॒ । मे॒ । शय॑नम् ( ) । च॒ । मे॒ । सू॒षेति॑ सु - उ॒षा । च॒ । मे॒ । सु॒दिन॒मिति॑ सु - दिन᳚म् । च॒ । मे॒ ॥ \textbf{  6 } \newline
                  \newline
                      ( विश्वं॑ च॒ - शय॑न - म॒ष्टौ च॑ )  \textbf{(A3)} \newline \newline
                                \textbf{ TS 4.7.4.1} \newline
                  ऊर्क् । च॒ । मे॒ । सू॒नृता᳚ । च॒ । मे॒ । पयः॑ । च॒ । मे॒ । रसः॑ । च॒ । मे॒ । घृ॒तम् । च॒ । मे॒ । मधु॑ । च॒ । मे॒ । सग्धिः॑ । च॒ । मे॒ । सपी॑ति॒रिति॒ स - पी॒तिः॒ । च॒ । मे॒ । कृ॒षिः । च॒ । मे॒ । वृष्टिः॑ । च॒ । मे॒ । जैत्र᳚म् । च॒ । मे॒ । औद्भि॑द्य॒मित्यौत्-भि॒द्य॒म् । च॒ । मे॒ । र॒यिः । च॒ । मे॒ । रायः॑ । च॒ । मे॒ । पु॒ष्टम् । च॒ । मे॒ । पुष्टिः॑ । च॒ । मे॒ । वि॒भ्विति॑ वि - भु । च॒ । \textbf{  7} \newline
                  \newline
                                \textbf{ TS 4.7.4.2} \newline
                  मे॒ । प्र॒भ्विति॑ प्र - भु । च॒ । मे॒ । ब॒हु । च॒ । मे॒ । भूयः॑ । च॒ । मे॒ । पू॒र्णम् । च॒ । मे॒ । पू॒र्णत॑र॒मिति॑ पू॒र्ण - त॒र॒म् । च॒ । मे॒ । अक्षि॑तिः । च॒ । मे॒ । कूय॑वाः । च॒ । मे॒ । अन्न᳚म् । च॒ । मे॒ । अक्षु॑त् । च॒ । मे॒ । व्री॒हयः॑ । च॒ । मे॒ । यवाः᳚ । च॒ । मे॒ । माषाः᳚ । च॒ । मे॒ । तिलाः᳚ । च॒ । मे॒ । मु॒द्गाः । च॒ । मे॒ । ख॒ल्वाः᳚ । च॒ । मे॒ । गो॒धूमाः᳚ । च॒ । मे॒ । म॒सुराः᳚ ( ) । च॒ । मे॒ । प्रि॒यंग॑वः । च॒ । मे॒ । अण॑वः । च॒ । मे॒ । श्या॒माकाः᳚ । च॒ । मे॒ । नी॒वाराः᳚ । च॒ । मे॒ ॥ \textbf{  8 } \newline
                  \newline
                      (वि॒भु च॑ - म॒सुरा॒ - श्चतु॑र्दश च)  \textbf{(A4)} \newline \newline
                                \textbf{ TS 4.7.5.1} \newline
                  अश्मा᳚ । च॒ । मे॒ । मृत्ति॑का । च॒ । मे॒ । गि॒रयः॑ । च॒ । मे॒ । पर्व॑ताः । च॒ । मे॒ । सिक॑ताः । च॒ । मे॒ । वन॒स्पत॑यः । च॒ । मे॒ । हिर॑ण्यम् । च॒ । मे॒ । अयः॑ । च॒ । मे॒ । सीस᳚म् । च॒ । मे॒ । त्रपु॑ । च॒ । मे॒ । श्या॒मम् । च॒ । मे॒ । लो॒हम् । च॒ । मे॒ । अ॒ग्निः । च॒ । मे॒ । आपः॑ । च॒ । मे॒ । वी॒रुधः॑ । च॒ । मे॒ । ओष॑धयः । च॒ । मे॒ । कृ॒ष्ट॒प॒च्यमिति॑ कृष्ट - प॒च्यम् । च॒ । \textbf{  9} \newline
                  \newline
                                \textbf{ TS 4.7.5.2} \newline
                  मे॒ । अ॒कृ॒ष्ट॒प॒च्यमित्य॑कृष्ट - प॒च्यम् । च॒ । मे॒ । ग्रा॒म्याः । च॒ । मे॒ । प॒शवः॑ । आ॒र॒ण्याः । च॒ । य॒ज्ञेन॑ । क॒ल्प॒न्ता॒म् । वि॒त्तम् । च॒ । मे॒ । वित्तिः॑ । च॒ । मे॒ । भू॒तम् । च॒ । मे॒ । भूतिः॑ । च॒ । मे॒ । वसु॑ । च॒ । मे॒ । व॒स॒तिः । च॒ । मे॒ । कर्म॑ । च॒ । मे॒ । शक्तिः॑ । च॒ । मे॒ । अर्थः॑ । च॒ । मे॒ । एमः॑ । च॒ । मे॒ । इतिः॑ । च॒ । मे॒ । गतिः॑ । च॒ । मे॒ ॥ \textbf{  10} \newline
                  \newline
                      (कृ॒ष्ट॒प॒च्यं चा॒ - ऽष्टाच॑त्वारिꣳशच्च)  \textbf{(A5)} \newline \newline
                                \textbf{ TS 4.7.6.1} \newline
                  अ॒ग्निः । च॒ । मे॒ । इन्द्रः॑ । च॒ । मे॒ । सोमः॑ । च॒ । मे॒ । इन्द्रः॑ । च॒ । मे॒ । स॒वि॒ता । च॒ । मे॒ । इन्द्रः॑ । च॒ । मे॒ । सर॑स्वती । च॒ । मे॒ । इन्द्रः॑ । च॒ । मे॒ । पू॒षा । च॒ । मे॒ । इन्द्रः॑ । च॒ । मे॒ । बृह॒स्पतिः॑ । च॒ । मे॒ । इन्द्रः॑ । च॒ । मे॒ । मि॒त्रः । च॒ । मे॒ । इन्द्रः॑ । च॒ । मे॒ । वरु॑णः । च॒ । मे॒ । इन्द्रः॑ । च॒ । मे॒ । त्वष्टा᳚ । च॒ । \textbf{  11} \newline
                  \newline
                                \textbf{ TS 4.7.6.2} \newline
                  मे॒ । इन्द्रः॑ । च॒ । मे॒ । धा॒ता । च॒ । मे॒ । इन्द्रः॑ । च॒ । मे॒ । विष्णुः॑ । च॒ । मे॒ । इन्द्रः॑ । च॒ । मे॒ । अ॒श्विनौ᳚ । च॒ । मे॒ । इन्द्रः॑ । च॒ । मे॒ । म॒रुतः॑ । च॒ । मे॒ । इन्द्रः॑ । च॒ । मे॒ । विश्वे᳚ । च॒ । मे॒ । दे॒वाः । इन्द्रः॑ । च॒ । मे॒ । पृ॒थि॒वी । च॒ । मे॒ । इन्द्रः॑ । च॒ । मे॒ । अ॒न्तरि॑क्षम् । च॒ । मे॒ । इन्द्रः॑ । च॒ । मे॒ । द्यौः । च॒ । मे॒ ( ) । इन्द्रः॑ । च॒ । मे॒ । दिशः॑ । च॒ । मे॒ । इन्द्रः॑ । च॒ । मे॒ । मू॒र्द्धा । च॒ । मे॒ । इन्द्रः॑ । च॒ । मे॒ । प्र॒जाप॑ति॒रिति॑ प्र॒जा - प॒तिः॒ । च॒ । मे॒ । इन्द्रः॑ । च॒ । मे॒ ॥ \textbf{  12 } \newline
                  \newline
                      (त्वष्टा॑ च॒ - द्यौश्च॑ म॒ - एक॑विꣳशतिश्च)  \textbf{(A6)} \newline \newline
                                \textbf{ TS 4.7.7.1} \newline
                  अꣳ॒॒शुः । च॒ । मे॒ । र॒श्मिः । च॒ । मे॒ । अदा᳚भ्यः । च॒ । मे॒ । अधि॑पति॒रित्यधि॑ - प॒तिः॒ । च॒ । मे॒ । उ॒पाꣳ॒॒शुरित्यु॑प-अꣳ॒॒शुः । च॒ । मे॒ । अ॒न्त॒र्या॒म इत्य॑न्तः-या॒मः । च॒ । मे॒ । ऐ॒न्द्र॒वा॒य॒व इत्यै᳚न्द्र-वा॒य॒वः । च॒ । मे॒ । मै॒त्रा॒व॒रु॒ण इति॑ मैत्रा - व॒रु॒णः । च॒ । मे॒ । आ॒श्वि॒नः । च॒ । मे॒ । प्र॒ति॒प्र॒स्थान॒ इति॑ प्रति - प्र॒स्थानः॑ । च॒ । मे॒ । शु॒क्रः । च॒ । मे॒ । म॒न्थी । च॒ । मे॒ । आ॒ग्र॒य॒णः । च॒ । मे॒ । वै॒श्व॒दे॒व इति॑ वैश्व - दे॒वः । च॒ । मे॒ । ध्रु॒वः । च॒ । मे॒ । वै॒श्वा॒न॒रः । च॒ । मे॒ । ऋ॒तु॒ग्र॒हा इत्यृ॑तु - ग्र॒हाः । च॒ । \textbf{  13} \newline
                  \newline
                                \textbf{ TS 4.7.7.2} \newline
                  मे॒ । अ॒ति॒ग्रा॒ह्या᳚ इत्य॑ति-ग्रा॒ह्याः᳚ । च॒ । मे॒ । ऐ॒न्द्रा॒ग्न इत्यै᳚न्द्र-अ॒ग्नः । च॒ । मे॒ । वै॒श्व॒दे॒व इति॑ वैश्व - दे॒वः । च॒ । मे॒ । म॒रु॒त्व॒तीयाः᳚ । च॒ । मे॒ । मा॒हे॒न्द्र इति॑ माहा - इ॒न्द्रः । च॒ । मे॒ । आ॒दि॒त्यः । च॒ । मे॒ । सा॒वि॒त्रः । च॒ । मे॒ । सा॒र॒स्व॒तः । च॒ । मे॒ । पौ॒ष्णः । च॒ । मे॒ । पा॒त्नी॒व॒त इति॑ पात्नी-व॒तः । च॒ । मे॒ । हा॒रि॒यो॒ज॒न इति॑ हारि-यो॒ज॒नः । च॒ । मे॒ ॥ \textbf{  14} \newline
                  \newline
                      (ऋ॒तु॒ग्र॒हाश्च॒ - चतु॑स्त्रिꣳशच्च )  \textbf{(A7)} \newline \newline
                                \textbf{ TS 4.7.8.1} \newline
                  इ॒द्ध्मः । च॒ । मे॒ । ब॒र्॒.हिः । च॒ । मे॒ । वेदिः॑ । च॒ । मे॒ । धिष्णि॑याः । च॒ । मे॒ । स्रुचः॑ । च॒ । मे॒ । च॒म॒साः । च॒ । मे॒ । ग्रावा॑णः । च॒ । मे॒ । स्वर॑वः । च॒ । मे॒ । उ॒प॒र॒वा इत्यु॑प - र॒वाः । च॒ । मे॒ । अ॒धि॒षव॑णे॒ इत्य॑धि - सव॑ने । च॒ । मे॒ । द्रो॒ण॒क॒ल॒श इति॑ द्रोण-क॒ल॒शः । च॒ । मे॒ । वा॒य॒व्या॑नि । च॒ । मे॒ । पू॒त॒भृदिति॑ पूत - भृत् । च॒ । मे॒ । आ॒ध॒व॒नीय॒ इत्या᳚ - ध॒व॒नीयः॑ । च॒ । मे॒ । आग्नी᳚द्ध्र॒मित्याग्नि॑-इ॒द्ध्र॒म् । च॒ । मे॒ । ह॒वि॒द्‌र्धान॒मिति॑ हविः - धान᳚म् । च॒ । मे॒ । गृ॒हाः । च॒ ( ) । मे॒ । सदः॑ । च॒ । मे॒ । पु॒रो॒डाशाः᳚ । च॒ । मे॒ । प॒च॒ताः । च॒ । मे॒ । अ॒व॒भृ॒थ इत्य॑व - भृ॒थः । च॒ । मे॒ । स्व॒गा॒का॒र इति॑ स्वगा - का॒रः । च॒ । मे॒ ॥ \textbf{  15} \newline
                  \newline
                      (गृ॒हाश्च॒ - षोड॑श च)  \textbf{(A8)} \newline \newline
                                \textbf{ TS 4.7.9.1} \newline
                  अ॒ग्निः । च॒ । मे॒ । घ॒र्मः । च॒ । मे॒ । अ॒र्कः । च॒ । मे॒ । सूर्यः॑ । च॒ । मे॒ । प्रा॒ण इति॑ प्र - अ॒नः । च॒ । मे॒ । अ॒श्व॒मे॒ध इत्य॑श्व-मे॒धः । च॒ । मे॒ । पृ॒थि॒वी । च॒ । मे॒ । अदि॑तिः । च॒ । मे॒ । दितिः॑ । च॒ । मे॒ । द्यौः । च॒ । मे॒ । शक्व॑रीः । अ॒ङ्गुल॑यः । दिशः॑ । च॒ । मे॒ । य॒ज्ञेन॑ । क॒ल्प॒न्ता॒म् । ऋक् । च॒ । मे॒ । साम॑ । च॒ । मे॒ । स्तोमः॑ । च॒ । मे॒ । यजुः॑ । च॒ । मे॒ । दी॒क्षा ( ) । च॒ । मे॒ । तपः॑ । च॒ । मे॒ । ऋ॒तुः । च॒ । मे॒ । व्र॒तम् । च॒ । मे॒ । अ॒हो॒रा॒त्रयो॒रित्य॑हः - रा॒त्रयोः᳚ । वृ॒ष्ट्या । बृ॒ह॒द्र॒थ॒न्त॒रे इति॑ बृहत् - र॒थ॒न्त॒रे । च॒ । मे॒ । य॒ज्ञेन॑ । क॒ल्पे॒ता॒म् ॥ \textbf{  16 } \newline
                  \newline
                      (दी॒क्षाऽ - ष्टाद॑श च )  \textbf{(A9)} \newline \newline
                                \textbf{ TS 4.7.10.1} \newline
                  गर्भाः᳚ । च॒ । मे॒ । व॒थ्साः । च॒ । मे॒ । त्र्यवि॒रिति॑ त्रि - अविः॑ । च॒ । मे॒ । त्र्य॒वीति॑ त्रि - अ॒वी । च॒ । मे॒ । दि॒त्य॒वाडिति॑ दित्य - वाट् । च॒ । मे॒ । दि॒त्यौ॒ही । च॒ । मे॒ । पञ्चा॑वि॒रिति॒ पञ्च॑ - अ॒विः॒ । च॒ । मे॒ । प॒ञ्चा॒वीति॑ पञ्च-अ॒वी । च॒ । मे॒ । त्रि॒व॒थ्स इति॑ त्रि-व॒थ्सः । च॒ । मे॒ । त्रि॒व॒थ्सेति॑ त्रि - व॒थ्सा । च॒ । मे॒ । तु॒र्य॒वाडिति॑ तुर्य - वाट् । च॒ । मे॒ । तु॒र्यौ॒ही । च॒ । मे॒ । प॒ष्ठ॒वादिति॑ पष्ठ - वात् । च॒ । मे॒ । प॒ष्ठौ॒ही । च॒ । मे॒ । उ॒क्षा । च॒ । मे॒ । व॒शा । च॒ । मे॒ । ऋ॒ष॒भः । च॒ । \textbf{  17} \newline
                  \newline
                                \textbf{ TS 4.7.10.2} \newline
                  म॒ । वे॒हत् । च॒ । मे॒ । अ॒न॒ड्वान् । च॒ । मे॒ । धे॒नुः । च॒ । मे॒ । आयुः॑ । य॒ज्ञेन॑ । क॒ल्प॒ता॒म् । प्रा॒ण इति॑ प्र - अ॒नः । य॒ज्ञेन॑ । क॒ल्प॒ता॒म् । अ॒पा॒न इत्य॑प - अ॒नः । य॒ज्ञेन॑ । क॒ल्प॒ता॒म् । व्या॒न इति॑ वि - अ॒नः । य॒ज्ञेन॑ । क॒ल्प॒ता॒म् । चक्षुः॑ । य॒ज्ञेन॑ । क॒ल्प॒ता॒म् । श्रोत्र᳚म् । य॒ज्ञेन॑ । क॒ल्प॒ता॒म् । मनः॑ । य॒ज्ञेन॑ । क॒ल्प॒ता॒म् । वाक् । य॒ज्ञेन॑ । क॒ल्प॒ता॒म् । आ॒त्मा । य॒ज्ञेन॑ । क॒ल्प॒ता॒म् । य॒ज्ञ्ः । य॒ज्ञेन॑ । क॒ल्प॒ता॒म् ॥ \textbf{  18 } \newline
                  \newline
                      (ऋ॒ष॒भश्च॑ - चत्वारिꣳ॒॒शच्च॑)  \textbf{(A10)} \newline \newline
                                \textbf{ TS 4.7.11.1} \newline
                  एका᳚ । च॒ । मे॒ । ति॒स्रः । च॒ । मे॒ । पञ्च॑ । च॒ । मे॒ । स॒प्त । च॒ । मे॒ । नव॑ । च॒ । मे॒ । एका॑दश । च॒ । मे॒ । त्रयो॑द॒शेति॒ त्रयः॑ - द॒श॒ । च॒ । मे॒ । पञ्च॑द॒शेति॒ पञ्च॑-द॒श॒ । च॒ । मे॒ । स॒प्तद॒शेति॑ स॒प्त - द॒श॒ । च॒ । मे॒ । नव॑द॒शेति॒ नव॑ - द॒श॒ । च॒ । मे॒ । एक॑विꣳशति॒रित्येक॑ - विꣳ॒॒श॒तिः॒ । च॒ । मे॒ । त्रयो॑विꣳशति॒रिति॒ त्रयः॑-विꣳ॒॒श॒तिः॒ । च॒ । मे॒ । पञ्च॑विꣳशति॒रिति॒ पञ्च॑-विꣳ॒॒श॒तिः॒ । च॒ । मे॒ । स॒प्तविꣳ॑शति॒रिति॑ स॒प्त -विꣳ॒॒श॒तिः॒ । च॒ । मे॒ । नव॑विꣳशति॒रिति॒ नव॑ - विꣳ॒॒श॒तिः॒ । च॒ । मे॒ । एक॑त्रिꣳश॒दित्येक॑ - त्रिꣳ॒॒श॒त् । च॒ । मे॒ । त्रय॑स्त्रिꣳश॒दिति॒ त्रयः॑ - त्रिꣳ॒॒श॒त् । च॒ । \textbf{  19} \newline
                  \newline
                                \textbf{ TS 4.7.11.2} \newline
                  मे॒ । चत॑स्रः । च॒ । मे॒ । अ॒ष्टौ । च॒ । मे॒ । द्वाद॑श । च॒ । मे॒ । षोड॑श । च॒ । मे॒ । विꣳ॒श॒तिः । च॒ । मे॒ । चतु॑र्विꣳशति॒रिति॒ चतुः॑ - विꣳ॒॒श॒तिः॒ । च॒ । मे॒ । अ॒ष्टाविꣳ॑शति॒रित्य॒ष्टा - विꣳ॒॒श॒तिः॒ । च॒ । मे॒ । द्वात्रिꣳ॑शत् । च॒ । मे॒ । षट्त्रिꣳ॑श॒दिति॒ षट् - त्रिꣳ॒॒श॒त् । च॒ । मे॒ । च॒त्वा॒रिꣳ॒॒शत् । च॒ । मे॒ । चतु॑श्चत्वारिꣳश॒दिति॒ चतुः॑ - च॒त्वा॒रिꣳ॒॒श॒त् । च॒ । मे॒ । अ॒ष्टाच॑त्वारिꣳश॒दित्य॒ष्टा - च॒त्वा॒रिꣳ॒॒श॒त् । च॒ । मे॒ । वाजः॑ । च॒ । प्र॒स॒व इति॑ प्र - स॒वः । च॒ । अ॒पि॒ज इत्य॑पि-जः । च॒ । क्रतुः॑ । च॒ । सुवः॑ । च॒ । मू॒द्‌र्धा । च॒ । व्यश्नि॑य॒ इति॑ वि - अश्नि॑यः ( ) । च॒ । अ॒न्त्या॒य॒नः । च॒ । अन्त्यः॑ । च॒ । भौ॒व॒नः । च॒ । भुव॑नः । च॒ । अधि॑पति॒रित्यधि॑ - प॒तिः॒ । च॒ ॥ \textbf{  20 } \newline
                  \newline
                      (त्रय॑स्त्रिꣳशच्च॒ - व्यश्ञि॑य॒ - एका॑दश च )  \textbf{(A11)} \newline \newline
                                \textbf{ TS 4.7.12.1} \newline
                  वाजः॑ । नः॒ । स॒प्त । प्र॒दिश॒ इति॑ प्र - दिशः॑ । चत॑स्रः । वा॒ । प॒रा॒वत॒ इति॑ परा - वतः॑ ॥ वाजः॑ । नः॒ । विश्वैः᳚ । दे॒वैः । धन॑साता॒विति॒ धन॑ - सा॒तौ॒ । इ॒ह । अ॒व॒तु॒ ॥ विश्वे᳚ । अ॒द्य । म॒रुतः॑ । विश्वे᳚ । ऊ॒ती । विश्वे᳚ । भ॒व॒न्तु॒ । अ॒ग्नयः॑ । समि॑द्धा॒ इति॒ सं - इ॒द्धाः॒ ॥ विश्वे᳚ । नः॒ । दे॒वाः । अव॑सा । एति॑ । ग॒म॒न्तु॒ । विश्व᳚म् । अ॒स्तु॒ । द्रवि॑णम् । वाजः॑ । अ॒स्मे इति॑ ॥ वाज॑स्य । प्र॒स॒वमिति॑ प्र - स॒वम् । दे॒वाः॒ । रथैः᳚ । या॒त॒ । हि॒र॒ण्ययैः᳚ ॥ अ॒ग्निः । इन्द्रः॑ । बृह॒स्पतिः॑ । म॒रुतः॑ । सोम॑पीतय॒ इति॒ सोम॑ - पी॒त॒ये॒ ॥ वाजे॑वाज॒ इति॒ वाजे᳚ - वा॒जे॒ । अ॒व॒त॒ । वा॒जि॒नः॒ । नः॒ । धने॑षु । \textbf{  21} \newline
                  \newline
                                \textbf{ TS 4.7.12.2} \newline
                  वि॒प्राः॒ । अ॒मृ॒ताः॒ । ऋ॒त॒ज्ञा॒ इत्यृ॑त - ज्ञाः॒ ॥ अ॒स्य । मद्ध्वः॑ । पि॒ब॒त॒ । मा॒दय॑द्ध्वम् । तृ॒प्ताः । या॒त॒ । प॒थिभि॒रिति॑ प॒थि - भिः॒ । दे॒व॒यानै॒रिति॑ देव - यानैः᳚ ॥ वाजः॑ । पु॒रस्ता᳚त् । उ॒त । म॒द्ध्य॒तः । नः॒ । वाजः॑ । दे॒वान् । ऋ॒तुभि॒रित्यृ॒तु - भिः॒ । क॒ल्प॒या॒ति॒ ॥ वाज॑स्य । हि । प्र॒स॒व इति॑ प्र - स॒वः । नन्न॑मीति । विश्वाः᳚ । आशाः᳚ । वाज॑पति॒रिति॒ वाज॑ - प॒तिः॒ । भ॒वे॒य॒म् ॥ पयः॑ । पृ॒थि॒व्याम् । पयः॑ । ओष॑धीषु । पयः॑ । दि॒वि । अ॒न्तरि॑क्षे । पयः॑ । धा॒म् ॥ पय॑स्वतीः । प्र॒दिश॒ इति॑ प्र - दिशः॑ । स॒न्तु॒ । मह्य᳚म् ॥ समिति॑ । मा॒ । सृ॒जा॒मि॒ । पय॑सा । घृ॒तेन॑ । समिति॑ । मा॒ । सृ॒जा॒मि॒ । अ॒पः । \textbf{  22} \newline
                  \newline
                                \textbf{ TS 4.7.12.3} \newline
                  ओष॑धीभि॒रित्योष॑धि - भिः॒ ॥ सः । अ॒हम् । वाज᳚म् । स॒ने॒य॒म् । अ॒ग्ने॒ ॥ नक्तो॒षासा᳚ । सम॑न॒सेति॒ स-म॒न॒सा॒ । विरू॑पे॒ इति॒ वि-रू॒पे॒ । धा॒पये॑ते॒ इति॑ । शिशु᳚म् । एक᳚म् । स॒मी॒ची इति॑ ॥ द्यावा᳚ । क्षाम॑ । रु॒क्मः । अ॒न्तः । वीति॑ । भा॒ति॒ । दे॒वाः । अ॒ग्निम् । धा॒र॒य॒न्न् । द्र॒वि॒णो॒दा इति॑ द्रविणः - दाः ॥ स॒मु॒द्रः । अ॒सि॒ । नभ॑स्वान् । आ॒र्द्रदा॑नु॒रित्या॒र्द्र-दा॒नुः॒ । श॒भूंरिति॑ शं-भूः । म॒यो॒भूरिति॑ मयः-भूः । अ॒भीति॑ । मा॒ । वा॒हि॒ । स्वाहा᳚ । मा॒रु॒तः । अ॒सि॒ । म॒रुता᳚म् । ग॒णः । श॒भूंरिति॑ शं - भूः । म॒यो॒भूरिति॑ मयः - भूः । अ॒भीति॑ । मा॒ । वा॒हि॒ । स्वाहा᳚ । अ॒व॒स्युः । अ॒सि॒ । दुव॑स्वान् । श॒भूंरिति॑ शं - भूः । म॒यो॒भूरिति॑ मयः - भूः । अ॒भीति॑ । मा॒ ( ) । वा॒हि॒ । स्वाहा᳚ ॥ \textbf{  23} \newline
                  \newline
                      (धने᳚ - ष्व॒पो - दुव॑स्वाञ्छ॒भूंर्म॑यो॒भूर॒भ मा॒ -द्वे च॑ )  \textbf{(A12)} \newline \newline
                                \textbf{ TS 4.7.13.1} \newline
                  अ॒ग्निम् । यु॒न॒ज्मि॒ । शव॑सा । घृ॒तेन॑ । दि॒व्यम् । सु॒प॒र्णमिति॑ सु - प॒र्णम् । वय॑सा । बृ॒हन्त᳚म् ॥ तेन॑ । व॒यम् । प॒ते॒म॒ । ब्र॒द्ध्नस्य॑ । वि॒ष्टप᳚म् । सुवः॑ । रुहा॑णाः । अधीति॑ । नाके᳚ । उ॒त्त॒म इत्यु॑त् - त॒मे ॥ इ॒मौ । ते॒ । प॒क्षौ । अ॒जरौ᳚ । प॒त॒त्रिणः॑ । याभ्या᳚म् । रक्षाꣳ॑सि । अ॒प॒हꣳसीत्य॑प - हꣳसि॑ । अ॒ग्ने॒ ॥ ताभ्या᳚म् । प॒ते॒म॒ । सु॒कृता॒मिति॑ सु - कृता᳚म् । उ॒ । लो॒कम् । यत्र॑ । ऋष॑यः । प्र॒थ॒म॒जा इति॑ प्रथम - जाः । ये । पु॒रा॒णाः ॥ चित् । अ॒सि॒ । स॒मु॒द्रयो॑नि॒रिति॑ समु॒द्र - यो॒निः॒ । इन्दुः॑ । दक्षः॑ । श्ये॒नः । ऋ॒तावेत्यृ॒त - वा॒ ॥ हिर॑ण्यपक्ष॒ इति॒ हिर॑ण्य - प॒क्षः॒ । श॒कु॒नः । भु॒र॒ण्युः । म॒हान् । स॒धस्थ॒ इति॑ स॒ध - स्थे॒ । ध्रु॒वः । \textbf{  24} \newline
                  \newline
                                \textbf{ TS 4.7.13.2} \newline
                  एति॑ । निष॑त्त॒ इति॒ नि - स॒त्तः॒ ॥ नमः॑ । ते॒ । अ॒स्तु॒ । मा । मा॒ । हिꣳ॒॒सीः॒ । विश्व॑स्य । मू॒द्‌र्धन्न् । अधीति॑ । ति॒ष्ठ॒सि॒ । श्रि॒तः ॥ स॒मु॒द्रे । ते॒ । हृद॑यम् । अ॒न्तः । आयुः॑ । द्यावा॑पृथि॒वी इति॒ द्यावा᳚ - पृ॒थि॒वी । भुव॑नेषु । अर्पि॑ते॒ इति॑ ॥ उ॒द्नः । द॒त्त॒ । उ॒द॒धिमित्यु॑द - धिम् । भि॒न्त॒ । दि॒वः । प॒र्जन्या᳚त् । अ॒न्तरि॑क्षात् । पृ॒थि॒व्याः । ततः॑ । नः॒ । वृष्ट्या᳚ । अ॒व॒त॒ ॥ दि॒वः । मू॒द्‌र्धा । अ॒सि॒ । पृ॒थि॒व्याः । नाभिः॑ । ऊर्क् । अ॒पाम् । ओष॑धीनाम् ॥ वि॒श्वायु॒रिति॑ वि॒श्व - आ॒युः॒ । शर्म॑ । स॒प्रथा॒ इति॑ स - प्रथाः᳚ । नमः॑ । प॒थे ॥ येन॑ । ऋष॑यः । तप॑सा । स॒त्रम् । \textbf{  25} \newline
                  \newline
                                \textbf{ TS 4.7.13.3} \newline
                  आस॑त । इन्धा॑नाः । अ॒ग्निम् । सुवः॑ । आ॒भर॑न्त॒ इत्या᳚ - भर॑न्तः ॥ तस्मिन्न्॑ । अ॒हम् । नीति॑ । द॒धे॒ । नाके᳚ । अ॒ग्निम् । ए॒तम् । यम् । आ॒हुः । मन॑वः । स्ती॒र्णब॑र्.हिष॒मिति॑ स्ती॒र्ण - ब॒र्॒.हि॒ष॒म् ॥ तम् । पत्नी॑भिः । अन्विति॑ । ग॒च्छे॒म॒ । दे॒वाः॒ । पु॒त्रैः । भ्रातृ॑भि॒रिति॒ भ्रातृ॑ - भिः॒ । उ॒त । वा॒ । हिर॑ण्यैः ॥ नाक᳚म् । गृ॒ह्णा॒नाः । सु॒कृ॒तस्येति॑ सु - कृ॒तस्य॑ । लो॒के । तृ॒तीये᳚ । पृ॒ष्ठे । अधीति॑ । रो॒च॒ने । दि॒वः ॥ एति॑ । व॒चः । मद्ध्य᳚म् । अ॒रु॒ह॒त् । भु॒र॒ण्युः । अ॒यम् । अ॒ग्निः । सत्प॑ति॒रिति॒ सत् - प॒तिः॒ । चेकि॑तानः ॥ पृ॒ष्ठे । पृ॒थि॒व्याः । निहि॑त॒ इति॒ नि - हि॒तः॒ । दवि॑द्युतत् । अ॒ध॒स्प॒दमित्य॑धः - प॒दम् । कृ॒णु॒ते॒ । \textbf{  26} \newline
                  \newline
                                \textbf{ TS 4.7.13.4} \newline
                  ये । पृ॒त॒न्यवः॑ ॥ अ॒यम् । अ॒ग्निः । वी॒रत॑म॒ इति॑ वी॒र - त॒मः॒ । व॒यो॒धा इति॑ वयः - धाः । स॒ह॒स्रियः॑ । दी॒प्य॒ता॒म् । अप्र॑युच्छ॒न्नित्यप्र॑ - यु॒च्छ॒न्न् ॥ वि॒भ्राज॑मान॒ इति॑ वि - भ्राज॑मानः । स॒रि॒रस्य॑ । मद्ध्ये᳚ । उप॑ । प्रेति॑ । या॒त॒ । दि॒व्यानि॑ । धाम॑ ॥ सम् । प्रेति॑ । च्य॒व॒द्ध्व॒म् । अन्विति॑ । सम् । प्रेति॑ । या॒त॒ । अग्ने᳚ । प॒थः । दे॒व॒याना॒निति॑ दे॒व - यानान्॑ । कृ॒णु॒द्ध्व॒म् ॥ अ॒स्मिन्न् । स॒धस्थ॒ इति॑ स॒ध - स्थे॒ । अधीति॑ । उत्त॑रस्मि॒न्नित्युत्-त॒र॒स्मि॒न्न् । विश्वे᳚ । दे॒वाः॒ । यज॑मानः । च॒ । सी॒द॒त॒ ॥ येन॑ । स॒हस्र᳚म् । वह॑सि । येन॑ । अ॒ग्ने॒ । स॒र्व॒वे॒द॒समिति॑ सर्व - वे॒द॒सम् ॥ तेन॑ । इ॒मम् । य॒ज्ञ्म् । नः॒ । व॒ह॒ । दे॒व॒यान॒ इति॑ देव - यानः॑ । यः । \textbf{  27} \newline
                  \newline
                                \textbf{ TS 4.7.13.5} \newline
                  उ॒त्त॒म इत्यु॑त्-त॒मः ॥ उदिति॑ । बु॒द्ध्य॒स्व॒ । अ॒ग्ने॒ । प्रतीति॑ । जा॒गृ॒हि॒ । ए॒न॒म् । इ॒ष्टा॒पू॒र्ते इती᳚ष्टा - पू॒र्ते । समिति॑ । सृ॒जे॒था॒म् । अ॒यम् । च॒ ॥ पुनः॑ । कृ॒ण्वन्न् । त्वा॒ । पि॒तर᳚म् । युवा॑नम् । अ॒न्वाताꣳ॑सी॒दित्य॑नु-आताꣳ॑सीत् । त्वयि॑ । तन्तु᳚म् । ए॒तम् ॥ अ॒यम् । ते॒ । योनिः॑ । ऋ॒त्वियः॑ । यतः॑ । जा॒तः । अरो॑चथाः ॥ तम् । जा॒नन्न् । अ॒ग्ने॒ । एति॑ । रो॒ह॒ । अथ॑ । नः॒ । व॒द्‌र्ध॒य॒ । र॒यिम् ॥ \textbf{  28} \newline
                  \newline
                      (ध्रु॒वः - स॒त्रं - कृ॑णुते॒ - यः - स॒प्तत्रिꣳ॑शच्च )  \textbf{(A13)} \newline \newline
                                \textbf{ TS 4.7.14.1} \newline
                  मम॑ । अ॒ग्ने॒ । वर्चः॑ । वि॒ह॒वेष्विति॑ वि - ह॒वेषु॑ । अ॒स्तु॒ । व॒यम् । त्वा॒ । इन्धा॑नाः । त॒नुव᳚म् । पु॒षे॒म॒ ॥ मह्य᳚म् । न॒म॒न्ता॒म् । प्र॒दिश॒ इति॑ प्र - दिशः॑ । चत॑स्रः । त्वया᳚ । अद्ध्य॑क्षे॒णेत्यधि॑ - अ॒क्षे॒ण॒ । पृत॑नाः । ज॒ये॒म॒ ॥ मम॑ । दे॒वाः । वि॒ह॒व इति॑ वि - ह॒वे । स॒न्तु॒ । सर्वे᳚ । इन्द्रा॑वन्त॒ इतीन्द्र॑ - व॒न्तः॒ । म॒रुतः॑ । विष्णुः॑ । अ॒ग्निः ॥ मम॑ । अ॒न्तरि॑क्षम् । उ॒रु । गो॒पम् । अ॒स्तु॒ । मह्य᳚म् । वातः॑ । प॒व॒ता॒म् । कामे᳚ । अ॒स्मिन्न् ॥ मयि॑ । दे॒वाः । द्रवि॑णम् । एति॑ । य॒ज॒न्ता॒म् । मयि॑ । आ॒शीरित्या᳚ - शीः । अ॒स्तु॒ । मयि॑ । दे॒वहू॑ति॒रिति॑ दे॒व - हू॒तिः॒ ॥ दैव्या᳚ । होता॑रा । व॒नि॒ष॒न्त॒ । \textbf{  29} \newline
                  \newline
                                \textbf{ TS 4.7.14.2} \newline
                  पूर्वे᳚ । अरि॑ष्टाः । स्या॒म॒ । त॒नुवा᳚ । सु॒वीरा॒ इति॑ सु - वीराः᳚ ॥ मह्य᳚म् । य॒ज॒न्तु॒ । मम॑ । यानि॑ । ह॒व्या । आकू॑ति॒रित्या - कू॒तिः॒ । स॒त्या । मन॑सः । मे॒ । अ॒स्तु॒ ॥ एनः॑ । मा । नीति॑ । गा॒म् । क॒त॒मत् । च॒न । अ॒हम् । विश्वे᳚ । दे॒वा॒सः॒ । अधीति॑ । वो॒च॒त॒ । मे॒ ॥ देवीः᳚ । ष॒डु॒र्वी॒रिति॑ षट् - उ॒र्वीः॒ । उ॒रु । नः॒ । कृ॒णो॒त॒ । विश्वे᳚ । दे॒वा॒सः॒ । इ॒ह । वी॒र॒य॒द्ध्व॒म् ॥ मा । हा॒स्म॒हि॒ । प्र॒जयेति॑ प्र - जया᳚ । मा । त॒नूभिः॑ । मा । र॒धा॒म॒ । द्वि॒ष॒ते । सो॒म॒ । रा॒ज॒न्न् ॥ अ॒ग्निः । म॒न्युम् । प्र॒ति॒नु॒दन्निति॑ प्रति - नु॒दन्न् । पु॒रस्ता᳚त् । \textbf{  30} \newline
                  \newline
                                \textbf{ TS 4.7.14.3} \newline
                  अद॑ब्धः । गो॒पा इति॑ गो - पाः । परीति॑ । पा॒हि॒ । नः॒ । त्वम् ॥ प्र॒त्यञ्चः॑ । य॒न्तु॒ । नि॒गुत॒ इति॑ नि - गुतः॑ । पुनः॑ । ते । अ॒मा । ए॒षा॒म् । चि॒त्तम् । प्र॒बुधेति॑ प्र - बुधा᳚ । वीति॑ । ने॒श॒त् ॥ धा॒ता । धा॒तृ॒णाम् । भुव॑नस्य । यः । पतिः॑ । दे॒वम् । स॒वि॒तार᳚म् । अ॒भि॒मा॒ति॒षाह॒मित्य॑भिमाति - साह᳚म् ॥ इ॒मम् । य॒ज्ञ्म् । अ॒श्विना᳚ । उ॒भा । बृह॒स्पतिः॑ । दे॒वाः । पा॒न्तु॒ । यज॑मानम् । न्य॒र्थादिति॑ नि- अ॒र्थात् ॥ उ॒रु॒व्यचा॒ इत्यु॑रु-व्यचाः᳚ । नः॒ । म॒हि॒षः । शर्म॑ । यꣳ॒॒स॒त् । अ॒स्मिन्न् । हवे᳚ । पु॒रु॒हू॒त इति॑ पुरु - हू॒तः । पु॒रु॒क्षु ॥ सः । नः॒ । प्र॒जाय॒ इति॑ प्र - जायै᳚ । ह॒र्य॒श्वेति॑ हरि - अ॒श्व॒ । मृ॒ड॒य॒ । इन्द्र॑ । मा । \textbf{  31} \newline
                  \newline
                                \textbf{ TS 4.7.14.4} \newline
                  नः॒ । री॒रि॒षः॒ । मा । परेति॑ । दाः॒ ॥ ये । नः॒ । स॒पत्नाः᳚ । अपेति॑ । ते । भ॒व॒न्तु॒ । इ॒न्द्रा॒ग्निभ्या॒मिती᳚न्द्रा॒ग्नि-भ्या॒म् । अवेति॑ । बा॒धा॒म॒हे॒ । तान् ॥ वस॑वः । रु॒द्राः । आ॒दि॒त्याः । उ॒प॒रि॒स्पृश॒मित्यु॑परि - स्पृश᳚म् । मा॒ । उ॒ग्रम् । चेत्ता॑रम् । अ॒धि॒रा॒जमित्य॑धि - रा॒जम् । अ॒क्र॒न्न् ॥ अ॒र्वाञ्च᳚म् । इन्द्र᳚म् । अ॒मुतः॑ । ह॒वा॒म॒हे॒ । यः । गो॒जिदिति॑ गो - जित् । ध॒न॒जिदिति॑ धन - जित् । अ॒श्व॒जिदित्य॑श्व - जित् । यः ॥ इ॒मम् । नः॒ । य॒ज्ञ्म् । वि॒ह॒व इति॑ वि - ह॒वे । जु॒ष॒स्व॒ । अ॒स्य । कु॒र्मः॒ । ह॒रि॒व॒ इति॑ हरि - वः॒ । मे॒दिन᳚म् । त्वा॒ ॥ \textbf{  32 } \newline
                  \newline
                      (व॒नि॒ष॒न्त॒ - पु॒रस्ता॒न् - मा - त्रिच॑त्वारिꣳशच्च)  \textbf{(A14)} \newline \newline
                                \textbf{ TS 4.7.15.1} \newline
                  अ॒ग्नेः । म॒न्वे॒ । प्र॒थ॒मस्य॑ । प्रचे॑तस॒ इति॒ प्र - चे॒त॒सः॒ । यम् । पाञ्च॑जन्य॒मिति॒ पाञ्च॑ - ज॒न्य॒म् । ब॒हवः॑ । स॒मि॒न्धत॒ इति॑ सम् - इ॒न्धते᳚ ॥ विश्व॑स्याम् । वि॒शि । प्र॒वि॒वि॒शि॒वाꣳस॒मिति॑ प्र - वि॒वि॒शि॒वाꣳस᳚म् । ई॒म॒हे॒ । सः । नः॒ । मु॒ञ्च॒तु॒ । अꣳह॑सः ॥ यस्य॑ । इ॒दम् । प्रा॒णदिति॑ प्र - अ॒नत् । नि॒मि॒षदिति॑ नि - मि॒षत् । यत् । एज॑ति । यस्य॑ । जा॒तम् । जन॑मानम् । च॒ । केव॑लम् ॥ स्तौमि॑ । अ॒ग्निम् । ना॒थि॒तः । जो॒ह॒वी॒मि॒ । सः । नः॒ । मु॒ञ्च॒तु॒ । अꣳह॑सः ॥ इन्द्र॑स्य । म॒न्ये॒ । प्र॒थ॒मस्य॑ । प्रचे॑तस॒ इति॒ प्र - चे॒त॒सः॒ । वृ॒त्र॒घ्न इति॑ वृत्र-घ्नः । स्तोमाः᳚ । उपेति॑ । माम् । उ॒पागु॒रित्यु॑प - आगुः॑ ॥ यः । दा॒शुषः॑ । सु॒कृत॒ इति॑ सु - कृतः॑ । हव᳚म् । उपेति॑ । गन्ता᳚ । \textbf{  33} \newline
                  \newline
                                \textbf{ TS 4.7.15.2} \newline
                  सः । नः॒ । मु॒ञ्च॒तु॒ । अꣳह॑सः ॥ यः । स॒ग्रां॒ममिति॑ सम् - ग्रा॒मम् । नय॑ति । समिति॑ । व॒शी । यु॒धे । यः । पु॒ष्टानि॑ । सꣳ॒॒सृ॒जतीति॑ सं-सृ॒जति॑ । त्र॒याणि॑ ॥ स्तौमि॑ । इन्द्र᳚म् । ना॒थि॒तः । जो॒ह॒वी॒मि॒ । सः । नः॒ । मु॒ञ्च॒तु॒ । अꣳह॑सः ॥ म॒न्वे । वा॒म् । मि॒त्रा॒व॒रु॒णेति॑ मित्रा - व॒रु॒णा॒ । तस्य॑ । वि॒त्त॒म् । सत्यौ॑ज॒सेति॒ सत्य॑ - ओ॒ज॒सा॒ । दृꣳ॒॒ह॒णा॒ । यम् । नु॒देथे॒ इति॑ ॥ या । राजा॑नम् । स॒रथ॒मिति॑ स-रथ᳚म् । या॒थः । उ॒ग्रा॒ । ता । नः॒ । मु॒ञ्च॒त॒म् । आग॑सः ॥ यः । वा॒म् । रथः॑ । ऋ॒जुर॑श्मि॒रित्यृ॒जु - र॒श्मिः॒ । स॒त्यध॒र्मेति॑ स॒त्य - ध॒र्मा॒ । मिथु॑ । चर॑न्तम् । उ॒प॒यातीत्यु॑प - याति॑ । दू॒षयन्न्॑ ॥ स्तौमि॑ । \textbf{  34} \newline
                  \newline
                                \textbf{ TS 4.7.15.3} \newline
                  मि॒त्रावरु॒णेति॑ मि॒त्रा - वरु॑णा । ना॒थि॒तः । जो॒ह॒वी॒मि॒ । तौ । नः॒ । मु॒ञ्च॒त॒म् । आग॑सः ॥ वा॒योः । स॒वि॒तुः । वि॒दथा॑नि । म॒न्म॒हे॒ । यौ । आ॒त्म॒न्वदित्या᳚त्मन्न् - वत् । बि॒भृ॒तः । यौ । च॒ । रक्ष॑तः ॥ यौ । विश्व॑स्य । प॒रि॒भू इति॑ परि - भूः । ब॒भू॒वतुः॑ । तौ । नः॒ । मु॒ञ्च॒त॒म् । आग॑सः ॥ उपेति॑ । श्रेष्ठाः᳚ । नः॒ । आ॒शिष॒ इत्या᳚ - शिषः॑ । दे॒वयोः᳚ । धर्मे᳚ । अ॒स्थि॒र॒न्न् ॥ स्तौमि॑ । वा॒युम् । स॒वि॒तार᳚म् । ना॒थि॒तः । जो॒ह॒वी॒मि॒ । तौ । नः॒ । मु॒ञ्च॒त॒म् । आग॑सः ॥ र॒थीत॑मा॒विति॑ र॒थि - त॒मौ॒ । र॒थी॒नाम् । अ॒ह्वे॒ । ऊ॒तये᳚ । शुभं᳚ । गमि॑ष्ठौ । सु॒यमे॑भि॒रिति॑ सु - यमे॑भिः । अश्वैः᳚ ॥ ययोः᳚ । \textbf{  35} \newline
                  \newline
                                \textbf{ TS 4.7.15.4} \newline
                  वा॒म् । दे॒वौ॒ । दे॒वेषु॑ । अनि॑शित॒मित्यनि॑-शि॒त॒म् । ओजः॑ । तौ । नः॒ । मु॒ञ्च॒त॒म् । आग॑सः ॥ यत् । अया॑तम् । व॒ह॒तुम् । सू॒र्यायाः᳚ । त्रि॒च॒क्रेणेति॑ त्रि - च॒क्रेण॑ । सꣳ॒॒सद॒मिति॑ सम् - सद᳚म् । इ॒च्छमा॑नौ ॥ स्तौमि॑ । दे॒वौ । अ॒श्विनौ᳚ । ना॒थि॒तः । जो॒ह॒वी॒मि॒ । तौ । नः॒ । मु॒ञ्च॒त॒म् । आग॑सः ॥ म॒रुता᳚म् । म॒न्वे॒ । अधीति॑ । नः॒ । ब्रु॒व॒न्तु॒ । प्रेति॑ । इ॒माम् । वाच᳚म् । विश्वा᳚म् । अ॒व॒न्तु॒ । विश्वे᳚ ॥ आ॒शून् । हु॒वे॒ । सु॒यमा॒निति॑ सु - यमान्॑ । ऊ॒तये᳚ । ते । नः॒ । मु॒ञ्च॒न्तु॒ । एन॑सः ॥ ति॒ग्मम् । आयु॑धम् । वी॒डि॒तम् । सह॑स्वत् । दि॒व्यम् । शर्द्धः॑ । \textbf{  36} \newline
                  \newline
                                \textbf{ TS 4.7.15.5} \newline
                  पृत॑नासु । जि॒ष्णु ॥ स्तौमि॑ । दे॒वान् । म॒रुतः॑ । ना॒थि॒तः । जो॒ह॒वी॒मि॒ । ते । नः॒ । मु॒ञ्च॒न्तु॒ । एन॑सः ॥ दे॒वाना᳚म् । म॒न्वे॒ । अधीति॑ । नः॒ । ब्रु॒व॒न्तु॒ । प्रेति॑ । इ॒माम् । वाच᳚म् । विश्वा᳚म् । अ॒व॒न्तु॒ । विश्वे᳚ ॥ आ॒शून् । हु॒वे॒ । सु॒यमा॒निति॑ सु - यमान्॑ । ऊ॒तये᳚ । ते । नः॒ । मु॒ञ्च॒न्तु॒ । एन॑सः ॥ यत् । इ॒दम् । मा॒ । अ॒भि॒शोच॒तीत्य॑भि-शोच॑ति । पौरु॑षेयेण । दैव्ये॑न ॥ स्तौमि॑ । विश्वान्॑ । दे॒वान् । ना॒थि॒तः । जो॒ह॒वी॒मि॒ । ते । नः॒ । मु॒ञ्च॒न्तु॒ । एन॑सः ॥ अन्विति॑ । नः॒ । अ॒द्य । अनु॑मति॒रियनु॑ - म॒तिः॒ । अन्विति॑ । \textbf{  37} \newline
                  \newline
                                \textbf{ TS 4.7.15.6} \newline
                  इत् । अ॒नु॒म॒त॒ इत्य॑नु - म॒ते॒ । त्वम् । वै॒श्वा॒न॒रः । नः॒ । ऊ॒त्या । पृ॒ष्टः । दि॒वि ॥ ये इति॑ । अप्र॑थेताम् । अमि॑तेभिः । ओजो॑भि॒रित्योजः॑ - भिः॒ । ये इति॑ । प्र॒ति॒ष्ठे इति॑ प्रति - स्थे । अभ॑वताम् । वसू॑नाम् ॥ स्तौमि॑ । द्यावा॑पृथि॒वी इति॒ द्यावा᳚ - पृ॒थि॒वी । ना॒थि॒तः । जो॒ह॒वी॒मि॒ । ते इति॑ । नः॒ । मु॒ञ्च॒त॒म् । अꣳह॑सः ॥ उर्वी॒ इति॑ । रो॒द॒सी॒ इति॑ । वरि॑वः । कृ॒णो॒त॒म् । क्षेत्र॑स्य । प॒त्नी॒ इति॑ । अधीति॑ । नः॒ । ब्रू॒या॒त॒म् ॥ स्तौमि॑ । द्यावा॑पृथि॒वी इति॒ द्यावा᳚ - पृ॒थि॒वी । ना॒थि॒तः । जो॒ह॒वी॒मि॒ । ते इति॑ । नः॒ । मु॒ञ्च॒त॒म् । अꣳह॑सः ॥ यत् । ते॒ । व॒यम् । पु॒रु॒ष॒त्रेति॑ पुरुष-त्रा । य॒वि॒ष्ठ॒ । अवि॑द्वाꣳसः । च॒कृ॒म । कत् । च॒न । \textbf{  38} \newline
                  \newline
                                \textbf{ TS 4.7.15.7} \newline
                  आगः॑ ॥ कृ॒धि । स्विति॑ । अ॒स्मान् । अदि॑तेः । अना॑गाः । वीति॑ । एनाꣳ॑सि । शि॒श्र॒थः॒ । विष्व॑क् । अ॒ग्ने॒ ॥ यथा᳚ । ह॒ । तत् । व॒स॒वः॒ । गौ॒र्य᳚म् । चि॒त् । प॒दि । सि॒ताम् । अमु॑ञ्चत । य॒ज॒त्राः॒ ॥ ए॒वा । त्वम् । अ॒स्मत् । प्रेति॑ । मु॒ञ्च॒ । वीति॑ । अꣳहः॑ । प्रेति॑ । अ॒ता॒रि॒ । अ॒ग्ने॒ । प्र॒त॒रामिति॑ प्र - त॒राम् । नः॒ । आयुः॑ ॥(अ॒ग्नेर्म॑न्वे॒ - यस्ये॒द- मिन्द्र॑स्य॒ - यः सं॑ ग्रा॒ममिन्द्रꣳ॒॒ - स नो॑ मुञ्च॒त्वꣳ ह॑सः । म॒न्वे वा॒न्ता नो॑ मुञ्चत॒माग॑सः । यो वां᳚ - वा॒यो- रुप॑ - र॒थीत॑मौ॒ - यदया॑त-म॒श्विनौ॒ - तौ नो॑ मुञ्चत॒माग॑सः । म॒रुता᳚न्- ति॒ग्मं - म॒रुतो॑ - दे॒वानां॒ - ॅयदि॒दं ॅविश्वा॒न् - ते नो॑ मुञ्च॒न्त्वेन॑सः । अनु॑ न॒ - उर्वी॒ - द्यावा॑पृथि॒वी - ते नो॑ मुञ्चत॒मꣳह॑सो॒ यत्तै᳚ । च॒तुरꣳ ह॑सः॒ षाडाग॑सश्च॒तुरेन॑सो॒ द्विरꣳह॑सः । \textbf{  39 } \newline
                  \newline
                      (गन्ता॑ - दू॒षय॒न्थ् स्तौमि॒ - ययोः॒ - शर्द्धोऽ - नु॑मति॒रनु॑ - च॒न - चतु॑स्त्रिꣳशच्च)  \textbf{(A15)} \newline \newline
\textbf{praSna korvai with starting padams of 1 to 15 anuvAkams :-} \newline
अग्ना॑विष्णू॒ - ज्यैष्ठयꣳ॒॒ - शञ्चो - र्क्चा - ऽश्मा॑ चा॒ - ग्निश्चा॒- ऽꣳ॒शु - श्चे॒द्ध्मश्चा॒ -ऽग्निश्च॑ घ॒र्मा - गर्भा॒ - श्चैका॑ च॒ - वाजो॑ नो - अ॒ग्निं ॅयु॑नज्मि॒ - ममा᳚ऽग्ने - अ॒ग्नेर्म॑न्वे॒ - पञ्च॑दश । \newline

\textbf{korvai with starting padams of1, 11, 21 series of pa~jcAtis :-} \newline
(अग्ना॑विष्णू - अ॒ग्निश्च॒ - वाजो॑ नो॒ - अद॑ब्धो गो॒पा - नव॑त्रिꣳशत् ) \newline

\textbf{first and last padam of seventh praSnam of kANDam 4:-} \newline
(अग्ना॑विष्णू - प्रत॒रान्न॒ आयुः॑ ) \newline 


॥ हरिः॑ ॐ ॥
॥ कृष्ण यजुर्वेदीय तैत्तिरीय संहितायां चतुर्थकाण्डे सप्तमः प्रश्नः समाप्तः ॥

॥ इति चतुर्थं काण्डं ॥ \newline
\pagebreak
4.7.1   appendix\\4.7.15.5 - अनु॑नो॒ऽद्यानु॑मति॒ >1, रन्विद॑नुमते॒ त्वं >2\\अनु॑ नो॒ऽद्याऽनु॑मतिर्य॒ज्ञ्ं दे॒वेषु॑ \\मन्यतां । अ॒ग्निश्च॑ हव्य॒वाह॑नो॒ भव॑तां दा॒शुषे॒ मयः॑ ॥\\\\अन्विद॑नुमते॒ त्वं मन्या॑सै॒ शञ्च॑नः कृधि । \\क्रत्वे॒ दक्षा॑य नो हिनु॒ प्रण॒ आयूꣳ॑षि तारिषः ॥\\(appearing in TS 3.3.11.3 अन्द् 3.3.11.4)\\\\\\\\4.7.15.6 -वै᳚श्वान॒रो न॑ ऊ॒त्या>3, पृ॒ष्टो दि॒वि> 4\\वै॒श्वा॒न॒रो न॑ ऊ॒त्या ऽऽ प्र या॑तु परा॒वतः॑ । अ॒ग्निरु॒क्थेन॒ वाह॑सा ॥\\\\पृ॒ष्टो दि॒वि पृ॒ष्टो अ॒ग्निः पृ॑थि॒व्यां पृ॒ष्टो विश्वा॒ ओष॑धी॒रा वि॑वेश । \\वै॒श्वा॒न॒रः सह॑सा पृ॒ष्टो अ॒ग्निः सनो॒ दिवा॒ स रि॒षः पा॑तु॒ नक्तं᳚ ॥ \\(appearing in TS 1.5.11.1 अन्द् 1.5.11.2 )\\=============\\\\
\pagebreak
        


\end{document}
