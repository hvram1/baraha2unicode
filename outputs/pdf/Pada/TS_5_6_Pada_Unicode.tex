\documentclass[17pt]{extarticle}
\usepackage{babel}
\usepackage{fontspec}
\usepackage{polyglossia}
\usepackage{extsizes}



\setmainlanguage{sanskrit}
\setotherlanguages{english} %% or other languages
\setlength{\parindent}{0pt}
\pagestyle{myheadings}
\newfontfamily\devanagarifont[Script=Devanagari]{AdishilaVedic}


\newcommand{\VAR}[1]{}
\newcommand{\BLOCK}[1]{}




\begin{document}
\begin{titlepage}
    \begin{center}
 
\begin{sanskrit}
    { \Large
    ॐ नमः परमात्मने, श्री महागणपतये नमः, श्री गुरुभ्यो नमः ॥ ह॒रिः॒ ॐ 
    }
    \\
    \vspace{2.5cm}
    \mbox{ \Huge
    5.6      पञ्चमकाण्डे षष्ठः प्रश्नः - उपानुवाक्याभिधानं   }
\end{sanskrit}
\end{center}

\end{titlepage}
\tableofcontents

ॐ नमः परमात्मने, श्री महागणपतये नमः, श्री गुरुभ्यो नमः
ह॒रिः॒ ॐ \newline
5.6      पञ्चमकाण्डे षष्ठः प्रश्नः - उपानुवाक्याभिधानं \newline

\addcontentsline{toc}{section}{ 5.6      पञ्चमकाण्डे षष्ठः प्रश्नः - उपानुवाक्याभिधानं}
\markright{ 5.6      पञ्चमकाण्डे षष्ठः प्रश्नः - उपानुवाक्याभिधानं \hfill https://www.vedavms.in \hfill}
\section*{ 5.6      पञ्चमकाण्डे षष्ठः प्रश्नः - उपानुवाक्याभिधानं }
                                \textbf{ TS 5.6.1.1} \newline
                  हिर॑ण्यवर्णा॒ इति॒ हिर॑ण्य - व॒र्णाः॒ । शुच॑यः । पा॒व॒काः । यासु॑ । जा॒तः । क॒श्यपः॑ । यासु॑ । इन्द्रः॑ ॥ अ॒ग्निम् । याः । गर्भ᳚म् । द॒धि॒रे । विरू॑पा॒ इति॒ वि - रू॒पाः॒ । ताः । नः॒ । आपः॑ । शम् । स्यो॒नाः । भ॒व॒न्तु॒ ॥ यासा᳚म् । राजा᳚ । वरु॑णः । याति॑ । मद्ध्ये᳚ । स॒त्या॒नृ॒ते इति॑ सत्य - अ॒नृ॒ते । अ॒व॒पश्य॒न्नित्य॑व - पश्यन्न्॑ । जना॑नाम् ॥ म॒धु॒श्चुत॒ इति॑ मधु - श्चुतः॑ । शुच॑यः । याः । पा॒व॒काः । ताः । नः॒ । आपः॑ । शम् । स्यो॒नाः । भ॒व॒न्तु॒ ॥ यासा᳚म् । दे॒वाः । दि॒वि । कृ॒ण्वन्ति॑ । भ॒क्षम् । याः । अ॒न्तरि॑क्षे । ब॒हु॒धेति॑ बहु - धा । भव॑न्ति ॥ याः । पृ॒थि॒वीम् । पय॑सा । उ॒न्दन्ति॑ । \textbf{  1} \newline
                  \newline
                                \textbf{ TS 5.6.1.2} \newline
                  शु॒क्राः । ताः । नः॒ । आपः॑ । शम् । स्यो॒नाः । भ॒व॒न्तु॒ ॥ शि॒वेन॑ । मा॒ । चक्षु॑षा । प॒श्य॒त॒ । आ॒पः॒ । शि॒वया᳚ । त॒नुवा᳚ । उपेति॑ । स्पृ॒श॒त॒ । त्वच᳚म् । मे॒ ॥ सर्वान्॑ । अ॒ग्नीन् । अ॒फ्सु॒षद॒ इत्य॑फ्सु - सदः॑ । हु॒वे॒ । वः॒ । मयि॑ । वर्चः॑ । बल᳚म् । ओजः॑ । नीति॑ । ध॒त्त॒ ॥ यत् । अ॒दः । स॒प्रं॒य॒तीरिति॑ सं - प्र॒य॒तीः । अहौ᳚ । अन॑दत । ह॒ते ॥ तस्मा᳚त् । एति॑ । न॒द्यः॑ । नाम॑ । स्थ॒ । ता । वः॒ । नामा॑नि । सि॒न्ध॒वः॒ ॥ यत् । प्रेषि॑ता॒ इति॒ प्र - इ॒षि॒ताः॒ । वरु॑णेन । ताः । शीभ᳚म् । स॒मव॑ल्ग॒तेति॑ सं - अव॑ल्गत ॥ \textbf{  2} \newline
                  \newline
                                \textbf{ TS 5.6.1.3} \newline
                  तत् । आ॒प्नो॒त् । इन्द्रः॑ । वः॒ । य॒तीः । तस्मा᳚त् । आपः॑ । अन्विति॑ । स्थ॒न॒ ॥ अ॒प॒का॒ममित्य॑प - का॒मम् । स्यन्द॑मानाः । अवी॑वरत । वः॒ । हिक᳚म् ॥ इन्द्रः॑ । वः॒ । शक्ति॑भि॒रिति॒ शक्ति॑-भिः॒ । दे॒वीः॒ । तस्मा᳚त् । वाः । नाम॑ । वः॒ । हि॒तम् ॥ एकः॑ । दे॒वः । अपीति॑ । अ॒ति॒ष्ठ॒त् । स्यन्द॑मानाः । य॒था॒व॒शमिति॑ यथा - व॒शम् ॥ उदिति॑ । आ॒नि॒षुः॒ । म॒हीः । इति॑ । तस्मा᳚त् । उ॒द॒कम् । उ॒च्य॒ते॒ ॥ आपः॑ । भ॒द्राः । घृ॒तम् । इत् । आपः॑ । आ॒सुः॒ । अ॒ग्नीषोमा॒वित्य॒ग्नी - सोमौ᳚ । बि॒भ्र॒ति॒ । आपः॑ । इत् । ताः ॥ ती॒व्रः । रसः॑ । म॒धु॒पृचा॒मिति॑ मधु-पृचा᳚म् । \textbf{  3} \newline
                  \newline
                                \textbf{ TS 5.6.1.4} \newline
                  अ॒र॒ङ्ग॒म इत्य॑रं - ग॒मः । एति॑ । मा॒ । प्रा॒णेनेति॑ प्र - अ॒नेन॑ । स॒ह । वर्च॑सा । ग॒न्न् ॥ आत् । इत् । प॒श्या॒मि॒ । उ॒त । वा॒ । शृ॒णो॒मि॒ । एति॑ । मा॒ । घोषः॑ । ग॒च्छ॒ति॒ । वाक् । नः॒ । आ॒सा॒म् ॥ मन्ये᳚ । भे॒जा॒नः । अ॒मृत॑स्य । तर्.हि॑ । हिर॑ण्यवर्णा॒ इति॒ हिर॑ण्य - व॒र्णाः॒ । अतृ॑पम् । य॒दा । वः॒ ॥ आपः॑ । हि । स्थ । म॒यो॒भुव॒ इति॑ मयः - भुवः॑ । ताः । नः॒ । ऊ॒र्जे । द॒धा॒त॒न॒ ॥ म॒हे । रणा॑य । चक्ष॑से ॥ यः । वः॒ । शि॒वत॑म॒ इति॑ शि॒व - त॒मः॒ । रसः॑ । तस्य॑ । भा॒ज॒य॒त॒ । इ॒ह । नः॒ ॥ उ॒श॒तीः । इ॒व॒ । मा॒तरः॑ ( ) ॥ तस्मै᳚ । अर᳚म् । ग॒मा॒म॒ । वः॒ । यस्य॑ । क्षया॑य । जिन्व॑थ ॥ आपः॑ । ज॒नय॑थ । च॒ । नः॒ ॥ दि॒वि । श्र॒य॒स्व॒ । अ॒न्तरि॑क्षे । य॒त॒स्व॒ । पृ॒थि॒व्या । समिति॑ । भ॒व॒ । ब्र॒ह्म॒व॒र्च॒समिति॑ ब्रह्म - व॒र्च॒सम् । अ॒सि॒ । ब्र॒ह्म॒व॒र्च॒सायेति॑ ब्रह्म - व॒र्च॒साय॑ । त्वा॒ ॥ \textbf{ } \newline
                  \newline
                      दि॒वि श्र॑य स्वा॒न्तरि॑क्षे यतस्व पृथि॒व्या सं भ॑व ब्रह्मवर्च॒-सम॑सि ब्रह्मवर्च॒साय॑ त्वा ॥ 4 (उ॒दन्ति॑ - स॒मव॑ल्गत - मधु॒पृचां᳚ - मा॒तरो॒ - द्वाविꣳ॑शतिश्च)  \textbf{(A1)} \newline \newline
                                \textbf{ TS 5.6.2.1} \newline
                  अ॒पाम् । ग्रहान्॑ । गृ॒ह्णा॒ति॒ । ए॒तत् । वाव । रा॒ज॒सूय॒मिति॑ राज - सूय᳚म् । यत् । ए॒ते । ग्रहाः᳚ । स॒वः । अ॒ग्निः । व॒रु॒ण॒स॒व इति॑ वरुण - स॒वः । रा॒ज॒सूय॒मिति॑ राज - सूय᳚म् । अ॒ग्नि॒स॒व इत्य॑ग्नि - स॒वः । चित्यः॑ । ताभ्या᳚म् । ए॒व । सू॒य॒ते॒ । अथो॒ इति॑ । उ॒भौ । ए॒व । लो॒कौ । अ॒भीति॑ । ज॒य॒ति॒ । यः । च॒ । रा॒ज॒सूये॒नेति॑ राज - सूये॑न । ई॒जा॒नस्य॑ । यः । च॒ । अ॒ग्नि॒चित॒ इत्य॑ग्नि-चितः॑ । आपः॑ । भ॒व॒न्ति॒ । आपः॑ । वै । अ॒ग्नेः । भ्रातृ॑व्याः । यत् । अ॒पः । अ॒ग्नेः । अ॒धस्ता᳚त् । उ॒प॒दधा॒तीत्यु॑प - दधा॑ति । भ्रातृ॑व्याभिभूत्या॒ इति॒ भ्रातृ॑व्य - अ॒भि॒भू॒त्यै॒ । भव॑ति । आ॒त्मना᳚ । परेति॑ । अ॒स्य॒ । भ्रातृ॑व्यः । भ॒व॒ति॒ । अ॒मृत᳚म् । \textbf{  5} \newline
                  \newline
                                \textbf{ TS 5.6.2.2} \newline
                  वै । आपः॑ । तस्मा᳚त् । अ॒द्भिरित्य॑त्-भिः । अव॑तान्त॒मित्यव॑-ता॒न्त॒म् । अ॒भीति॑ । सि॒ञ्च॒न्ति॒ । न । आर्ति᳚म् । एति॑ । ऋ॒च्छ॒ति॒ । सर्व᳚म् । आयुः॑ । ए॒ति॒ । यस्य॑ । ए॒ताः । उ॒प॒धी॒यन्त॒ इत्यु॑प - धी॒यन्ते᳚ । यः । उ॒ । च॒ । ए॒नाः॒ । ए॒वम् । वेद॑ । अन्न᳚म् । वै । आपः॑ । प॒शवः॑ । आपः॑ । अन्न᳚म् । प॒शवः॑ । अ॒न्ना॒द इत्य॑न्न - अ॒दः । प॒शु॒मानिति॑ पशु - मान् । भ॒व॒ति॒ । यस्य॑ । ए॒ताः । उ॒प॒धी॒यन्त॒ इत्यु॑प-धी॒यन्ते᳚ । यः । उ॒ । च॒ । ए॒नाः॒ । ए॒वम् । वेद॑ । द्वाद॑श । भ॒व॒न्ति॒ । द्वाद॑श । मासाः᳚ । सं॒ॅव॒थ्स॒र इति॑ सं - व॒थ्स॒रः । सं॒ॅव॒थ्स॒रेणेति॑ सं - व॒थ्स॒रेण॑ । ए॒व । अ॒स्मै॒ । \textbf{  6} \newline
                  \newline
                                \textbf{ TS 5.6.2.3} \newline
                  अन्न᳚म् । अवेति॑ । रु॒न्धे॒ । पात्रा॑णि । भ॒व॒न्ति॒ । पात्रे᳚ । वै । अन्न᳚म् । अ॒द्य॒ते॒ । सयो॒नीति॒ स - यो॒नि॒ । ए॒व । अन्न᳚म् । अवेति॑ । रु॒न्धे॒ । एति॑ । द्वा॒द॒शात् । पुरु॑षात् । अन्न᳚म् । अ॒त्ति॒ । अथो॒ इति॑ । पात्रा᳚त् । न । छि॒द्य॒ते॒ । यस्य॑ । ए॒ताः । उ॒प॒धी॒यन्त॒ इत्यु॑प-धी॒यन्ते᳚ । यः । उ॒ । च॒ । ए॒नाः॒ । ए॒वम् । वेद॑ । कु॒भांः । च॒ । कु॒भींः । च॒ । मि॒थु॒नानि॑ । भ॒व॒न्ति॒ । मि॒थु॒नस्य॑ । प्रजा᳚त्या॒ इति॒ प्र - जा॒त्यै॒ । प्रेति॑ । प्र॒जयेति॑ प्र - जया᳚ । प॒शुभि॒रिति॑ प॒शु - भिः॒ । मि॒थु॒नैः । जा॒य॒ते॒ । यस्य॑ । ए॒ताः । उ॒प॒धी॒यन्त॒ इत्यु॑प - धी॒यन्ते᳚ । यः । उ॒ । \textbf{  7} \newline
                  \newline
                                \textbf{ TS 5.6.2.4} \newline
                  च॒ । ए॒नाः॒ । ए॒वम् । वेद॑ । शुक् । वै । अ॒ग्निः । सः । अ॒द्ध्व॒र्युम् । यज॑मानम् । प्र॒जा इति॑ प्र - जाः । शु॒चा । अ॒र्प॒य॒ति॒ । यत् । अ॒पः । उ॒प॒दधा॒तीत्यु॑प - दधा॑ति । शुच᳚म् । ए॒व । अ॒स्य॒ । श॒म॒य॒ति॒ । न । आर्ति᳚म् । एति॑ । ऋ॒च्छ॒ति॒ । अ॒द्ध्व॒र्युः । न । यज॑मानः । शाम्य॑न्ति । प्र॒जा इति॑ प्र - जाः । यत्र॑ । ए॒ताः । उ॒प॒धी॒यन्त॒ इत्यु॑प - धी॒यन्ते᳚ । अ॒पाम् । वै । ए॒तानि॑ । हृद॑यानि । यत् । ए॒ताः । आपः॑ । यत् । ए॒ताः । अ॒पः । उ॒प॒दधा॒तीत्यु॑प - दधा॑ति । दि॒व्याभिः॑ । ए॒व । ए॒नाः॒ । समिति॑ । सृ॒ज॒ति॒ । वर्.षु॑कः । प॒र्जन्यः॑ । \textbf{  8} \newline
                  \newline
                                \textbf{ TS 5.6.2.5} \newline
                  भ॒व॒ति॒ । यः । वै । ए॒तासा᳚म् । आ॒यत॑न॒मित्या᳚ - यत॑नम् । क्लृप्ति᳚म् । वेद॑ । आ॒यत॑नवा॒नित्या॒यत॑न - वा॒न् । भ॒व॒ति॒ । कल्प॑ते । अ॒स्मै॒ । अ॒नु॒सी॒तमित्य॑नु - सी॒तम् । उपेति॑ । द॒धा॒ति॒ । ए॒तत् । वै । आ॒सा॒म् । आ॒यत॑न॒मित्या᳚ - यत॑नम् । ए॒षा । क्लृप्तिः॑ । यः । ए॒वम् । वेद॑ । आ॒यत॑नवा॒नित्या॒यत॑न - वा॒न् । भ॒व॒ति॒ । कल्प॑ते । अ॒स्मै॒ । द्व॒द्वंमिति॑ द्वं - द्वम् । अ॒न्याः । उपेति॑ । द॒धा॒ति॒ । चत॑स्रः । मद्ध्ये᳚ । धृत्यै᳚ । अन्न᳚म् । वै । इष्ट॑काः । ए॒तत् । खलु॑ । वै । सा॒क्षादिति॑ स-अ॒क्षात् । अन्न᳚म् । यत् । ए॒षः । च॒रुः । यत् । ए॒तम् । च॒रुम् । उ॒प॒दधा॒तीत्यु॑प - दधा॑ति । सा॒क्षादिति॑ स - अ॒क्षात् । \textbf{  9} \newline
                  \newline
                                \textbf{ TS 5.6.2.6} \newline
                  ए॒व । अ॒स्मै॒ । अन्न᳚म् । अवेति॑ । रु॒न्धे॒ । म॒द्ध्य॒तः । उपेति॑ । द॒धा॒ति॒ । म॒द्ध्य॒तः । ए॒व । अ॒स्मै॒ । अन्न᳚म् । द॒धा॒ति॒ । तस्मा᳚त् । म॒द्ध्य॒तः । अन्न᳚म् । अ॒द्य॒ते॒ । बा॒र्.॒ह॒स्प॒त्यः । भ॒व॒ति॒ । ब्रह्म॑ । वै । दे॒वाना᳚म् । बृह॒स्पतिः॑ । ब्रह्म॑णा । ए॒व । अ॒स्मै॒ । अन्न᳚म् । अवेति॑ ।       रु॒न्धे॒ । ब्र॒ह्म॒व॒र्च॒समिति॑ ब्रह्म - व॒र्च॒सम् । अ॒सि॒ । ब्र॒ह्म॒व॒र्च॒सायेति॑ ब्रह्म - व॒र्च॒साय॑ । त्वा॒ । इति॑ । आ॒ह॒ । ते॒ज॒स्वी । ब्र॒ह्म॒व॒र्च॒सीति॑ ब्रह्म - व॒र्च॒सी । भ॒व॒ति॒ । यस्य॑ । ए॒षः । उ॒प॒धी॒यत॒ इत्यु॑प - धी॒यते᳚ । यः । उ॒ । च॒ । ए॒न॒म् । ए॒वम् । वेद॑ ॥ \textbf{  10} \newline
                  \newline
                      (अ॒मृत॑ - मस्मै - जायते॒ यस्यै॒ता - उ॑पधी॒यन्ते॒ य उ॑ - प॒र्जन्य॑ - उप॒दधा॑ति सा॒क्षाथ् - स॒प्तच॑त्वारिꣳशच्च)  \textbf{(A2)} \newline \newline
                                \textbf{ TS 5.6.3.1} \newline
                  भू॒ते॒ष्ट॒का इति॑ भूत - इ॒ष्ट॒काः । उपेति॑ । द॒धा॒ति॒ । अत्रा॒त्रेत्यत्र॑-अ॒त्र॒ । वै । मृ॒त्युः । जा॒य॒ते॒ । यत्र॑य॒त्रेति॒ यत्र॑-य॒त्र॒ । ए॒व । मृ॒त्युः । जाय॑ते । ततः॑ । ए॒व । ए॒न॒म् । अवेति॑ । य॒ज॒ते॒ । तस्मा᳚त् । अ॒ग्नि॒चिदित्य॑ग्नि - चित् । सर्व᳚म् । आयुः॑ । ए॒ति॒ । सर्वे᳚ । हि । अ॒स्य॒ । मृ॒त्यवः॑ । अवे᳚ष्टा॒ इत्यव॑ - इ॒ष्टाः॒ । तस्मा᳚त् । अ॒ग्नि॒चिदित्य॑ग्नि - चित् । न । अ॒भिच॑रित॒वा इत्य॒भि-च॒रि॒त॒वै । प्र॒त्यक् । ए॒न॒म् । अ॒भि॒चा॒र इत्य॑भि - चा॒रः । स्तृ॒णु॒ते॒ । सू॒यते᳚ । वै । ए॒षः । यः । अ॒ग्निम् । चि॒नु॒ते । दे॒व॒सु॒वामिति॑ देव - सु॒वाम् । ए॒तानि॑ । ह॒वीꣳषि॑ । भ॒व॒न्ति॒ । ए॒ताव॑न्तः । वै । दे॒वाना᳚म् । स॒वाः । ते । ए॒व । \textbf{  11} \newline
                  \newline
                                \textbf{ TS 5.6.3.2} \newline
                  अ॒स्मै॒ । स॒वान् । प्रेति॑ । य॒च्छ॒न्ति॒ । ते । ए॒न॒म् । सु॒व॒न्ते॒ । स॒वः । अ॒ग्निः । व॒रु॒ण॒स॒व इति॑ वरुण - स॒वः । रा॒ज॒सूय॒मिति॑ राज - सूय᳚म् । ब्र॒ह्म॒स॒व इति॑ ब्रह्म-स॒वः । चित्यः॑ । दे॒वस्य॑ । त्वा॒ । स॒वि॒तुः । प्र॒स॒व इति॑ प्र - स॒वे । इति॑ । आ॒ह॒ । स॒वि॒तृप्र॑सूत॒ इति॑ सवि॒तृ-प्र॒सू॒तः॒ । ए॒व । ए॒न॒म् । ब्रह्म॑णा । दे॒वता॑भिः । अ॒भीति॑ । सि॒ञ्च॒ति॒ । अन्न॑स्यान्न॒स्येत्यन्न॑स्य - अ॒न्न॒स्य॒ । अ॒भीति॑ । सि॒ञ्च॒ति॒ । अन्न॑स्यान्न॒स्येत्यन्न॑स्य - अ॒न्न॒स्य॒ । अव॑रुद्ध्या॒ इत्यव॑ - रु॒द्ध्यै॒ । पु॒रस्ता᳚त् । प्र॒त्यञ्च᳚म् । अ॒भीति॑ । सि॒ञ्च॒ति॒ । पु॒रस्ता᳚त् । हि । प्र॒ती॒चीन᳚म् । अन्न᳚म् । अ॒द्यते᳚ । शी॒र्.॒ष॒तः । अ॒भीति॑ । सि॒ञ्च॒ति॒ । शी॒र्.॒ष॒तः । हि । अन्न᳚म् । अ॒द्यते᳚ । एति॑ । मुखा᳚त् । अ॒न्वव॑स्रावय॒तीत्य॑नु - अव॑स्रावयति । \textbf{  12} \newline
                  \newline
                                \textbf{ TS 5.6.3.3} \newline
                  मु॒ख॒तः । ए॒व । अ॒स्मै॒ । अ॒न्नाद्य॒मित्य॑न्न -अद्य᳚म् । द॒धा॒ति॒ । अ॒ग्नेः । त्वा॒ । साम्रा᳚ज्ये॒नेति॒ सां - रा॒ज्ये॒न॒ । अ॒भीति॑ । सि॒ञ्चा॒मि॒ । इति॑ । आ॒ह॒ । ए॒षः । वै । अ॒ग्नेः । स॒वः । तेन॑ । ए॒व । ए॒न॒म् । अ॒भीति॑ । सि॒ञ्च॒ति॒ । बृह॒स्पतेः᳚ । त्वा॒ । साम्रा᳚ज्ये॒नेति॒ सां-रा॒ज्ये॒न॒ । अ॒भीति॑ । सि॒ञ्चा॒मि॒ । इति॑ । आ॒ह॒ । ब्रह्म॑ । वै । दे॒वाना᳚म् । बृह॒स्पतिः॑ । ब्रह्म॑णा । ए॒व । ए॒न॒म् । अ॒भीति॑ । सि॒ञ्च॒ति॒ । इन्द्र॑स्य । त्वा॒ । साम्रा᳚ज्ये॒नेति॒ सां - रा॒ज्ये॒न॒ । अ॒भीति॑ । सि॒ञ्चा॒मि॒ । इति॑ । आ॒ह॒ । इ॒न्द्रि॒यम् । ए॒व । अ॒स्मि॒न्न् । उ॒परि॑ष्टात् । द॒धा॒ति॒ । ए॒तत् । \textbf{  13} \newline
                  \newline
                                \textbf{ TS 5.6.3.4} \newline
                  वै । रा॒ज॒सूय॒स्येति॑ राज - सूय॑स्य । रू॒पम् । यः । ए॒वम् । वि॒द्वान् । अ॒ग्निम् । चि॒नु॒ते । उ॒भौ । ए॒व ।   लो॒कौ । अ॒भीति॑ । ज॒य॒ति॒ । यः । च॒ । रा॒ज॒सूये॒नेति॑ राज - सूये॑न । ई॒जा॒नस्य॑ । यः । च॒ । अ॒ग्नि॒चित॒ इत्य॑ग्नि - चितः॑ । इन्द्र॑स्य । सु॒षु॒वा॒णस्य॑ । द॒श॒धेति॑ दश - धा । इ॒न्द्रि॒यम् । वी॒र्य᳚म् । परेति॑ । अ॒प॒त॒त् । तत् । दे॒वाः । सौ॒त्रा॒म॒ण्या । समिति॑ । अ॒भ॒र॒न्न् । सू॒यते᳚ । वै । ए॒षः । यः । अ॒ग्निम् । चि॒नु॒ते । अ॒ग्निम् । चि॒त्वा । सौ॒त्रा॒म॒ण्या । य॒जे॒त॒ । इ॒न्द्रि॒यम् । ए॒व । वी॒र्य᳚म् । स॒भृंत्येति॑ सं - भृत्य॑ । आ॒त्मन्न् । ध॒त्ते॒ ॥ \textbf{  14} \newline
                  \newline
                      (त ए॒वा - न्वव॑स्रावयत्ये॒ - त - द॒ष्टाच॑त्वारिꣳशच्च)  \textbf{(A3)} \newline \newline
                                \textbf{ TS 5.6.4.1} \newline
                  स॒जूरिति॑ स - जूः । अब्दः॑ । अया॑वभि॒रित्यया॑व - भिः॒ । स॒जूरिति॑ स - जूः । उ॒षाः । अरु॑णीभिः । स॒जूरिति॑ स - जूः । सूर्यः॑ । एत॑शेन । स॒जोषा॒विति॑ स - जोषौ᳚ । अ॒श्विना᳚ । दꣳसो॑भि॒रिति॒ दꣳसः॑ - भिः॒ । स॒जूरिति॑ स-जूः । अ॒ग्निः । वै॒श्वा॒न॒रः । इडा॑भिः । घृ॒तेन॑ । स्वाहा᳚ । सं॒ॅव॒थ्स॒र इति॑ सं-व॒थ्स॒रः । वै । अब्दः॑ । मासाः᳚ । अया॑वाः । उ॒षाः । अरु॑णी । सूर्यः॑ । एत॑शः । इ॒मे इति॑ । अ॒श्विना᳚ । सं॒ॅव॒थ्स॒र इति॑ सं - व॒थ्स॒रः । अ॒ग्निः । वै॒श्वा॒न॒रः । प॒शवः॑ । इडा᳚ । प॒शवः॑ । घृ॒तम् । सं॒ॅव॒थ्स॒रमिति॑ सं - व॒थ्स॒रम् । प॒शवः॑ । अनु॑ । प्रेति॑ । जा॒य॒न्ते॒ । सं॒ॅव॒थ्स॒रेणेति॑ सं - व॒थ्स॒रेण॑ । ए॒व । अ॒स्मै॒ । प॒शून् । प्रेति॑ । ज॒न॒य॒ति॒ । द॒र्भ॒स्त॒बं इति॑ दर्भ - स्त॒बें । जु॒हो॒ति॒ । यत् । \textbf{  15} \newline
                  \newline
                                \textbf{ TS 5.6.4.2} \newline
                  वै । अ॒स्याः । अ॒मृत᳚म् । यत् । वी॒र्य᳚म् । तत् । द॒र्भाः । तस्मिन्न्॑ । जु॒हो॒ति॒ । प्रेति॑ । ए॒व । जा॒य॒ते॒ । अ॒न्ना॒द इत्य॑न्न - अ॒दः । भ॒व॒ति॒ । यस्य॑ । ए॒वम् । जुह्व॑ति । ए॒ताः । वै । दे॒वताः᳚ । अ॒ग्नेः । पु॒रस्ता᳚द्भागा॒ इति॑ पु॒रस्ता᳚त् - भा॒गाः॒ । ताः । ए॒व । प्री॒णा॒ति॒ । अथो॒ इति॑ । चक्षुः॑ । ए॒व । अ॒ग्नेः । पु॒रस्ता᳚त् । प्रतीति॑ । द॒धा॒ति॒ । अन॑न्धः । भ॒व॒ति॒ । यः । ए॒वम् । वेद॑ । आपः॑ । वै । इ॒दम् । अग्रे᳚ । स॒लि॒लम् । आ॒सी॒त् । सः । प्र॒जाप॑ति॒रिति॑ प्र॒जा - प॒तिः॒ । पु॒ष्क॒र॒प॒र्ण इति॑ पुष्कर - प॒र्णे । वातः॑ । भू॒तः । अ॒ले॒ला॒य॒त् । सः । \textbf{  16} \newline
                  \newline
                                \textbf{ TS 5.6.4.3} \newline
                  प्र॒ति॒ष्ठामिति॑ प्रति - स्थाम् । न । अ॒वि॒न्द॒त॒ । सः । ए॒तत् । अ॒पाम् । कु॒लाय᳚म् । अ॒प॒श्य॒त् । तस्मिन्न्॑ । अ॒ग्निम् । अ॒चि॒नु॒त॒ । तत् । इ॒यम् । अ॒भ॒व॒त् । ततः॑ । वै । सः । प्रतीति॑ । अ॒ति॒ष्ठ॒त् । याम् । पु॒रस्ता᳚त् । उ॒पाद॑धा॒दित्यु॑प - अद॑धात् । तत् । शिरः॑ । अ॒भ॒व॒त् । सा । प्राची᳚ । दिक् । याम् । द॒क्षि॒ण॒तः । उ॒पाद॑धा॒दित्यु॑प - अद॑धात् । सः । दक्षि॑णः । प॒क्षः । अ॒भ॒व॒त् । सा । द॒क्षि॒णा । दिक् । याम् । प॒श्चात् । उ॒पाद॑धा॒दित्यु॑प-अद॑धात् । तत् । पुच्छ᳚म् । अ॒भ॒व॒त् । सा । प्र॒तीची᳚ । दिक् । याम् । उ॒त्त॒र॒त इत्यु॑त् - त॒र॒तः । उ॒पाद॑धा॒दित्यु॑प - अद॑धात् । \textbf{  17 } \newline
                  \newline
                                \textbf{ TS 5.6.4.4} \newline
                  सः । उत्त॑र॒ इत्युत् - त॒रः॒ । प॒क्षः । अ॒भ॒व॒त् । सा । उदी॑ची । दिक् । याम् । उ॒परि॑ष्टात् । उ॒पाद॑धा॒दित्यु॑प - अद॑धात् । तत् । पृ॒ष्ठम् । अ॒भ॒व॒त् । सा । ऊ॒द्‌र्ध्वा । दिक् । इ॒यम् । वै । अ॒ग्निः । पञ्चे᳚ष्टक॒ इति॒ पञ्च॑-इ॒ष्ट॒कः॒ । तस्मा᳚त् । यत् । अ॒स्याम् । खन॑न्ति । अ॒भीति॑ । इष्ट॑काम् । तृ॒न्दन्ति॑ । अ॒भीति॑ । शर्क॑राम् । सर्वा᳚ । वै । इ॒यम् । वयो᳚भ्य॒ इति॒ वयः॑ - भ्यः॒ । नक्त᳚म् । दृ॒शे । दी॒प्य॒ते॒ । तस्मा᳚त् । इ॒माम् । वयाꣳ॑सि । नक्त᳚म् । न । अधीति॑ । आ॒स॒ते॒ । यः । ए॒वम् । वि॒द्वान् । अ॒ग्निम् । चि॒नु॒ते । प्रतीति॑ । ए॒व । \textbf{  18} \newline
                  \newline
                                \textbf{ TS 5.6.4.5} \newline
                  ति॒ष्ठ॒ति॒ । अ॒भीति॑ । दिशः॑ । ज॒य॒ति॒ । आ॒ग्ने॒यः । वै । ब्रा॒ह्म॒णः । तस्मा᳚त् । ब्रा॒ह्म॒णाय॑ । सर्वा॑सु । दि॒क्षु । अद्‌र्धु॑कम् । स्वाम् । ए॒व । तत् । दिश᳚म् । अन्विति॑ । ए॒ति॒ । अ॒पाम् । वै । अ॒ग्निः । कु॒लाय᳚म् । तस्मा᳚त् । आपः॑ । अ॒ग्निम् । हारु॑काः । स्वाम् । ए॒व । तत् । योनि᳚म् । प्रेति॑ । वि॒श॒न्ति॒ ॥ \textbf{  19} \newline
                  \newline
                      (यद॑- लेलाय॒थ् स-उ॑त्तर॒त उ॒पाद॑धा-दे॒व - द्वात्रिꣳ॑शच्च)  \textbf{(A4)} \newline \newline
                                \textbf{ TS 5.6.5.1} \newline
                  सं॒ॅव॒थ्स॒रमिति॑ सं - व॒थ्स॒रम् । उख्य᳚म् । भृ॒त्वा । द्वि॒तीये᳚ । सं॒ॅव॒थ्स॒र इति॑ सं - व॒थ्स॒रे । आ॒ग्ने॒यम् । अ॒ष्टाक॑पाल॒मित्य॒ष्टा - क॒पा॒ल॒म् । निरिति॑ । व॒पे॒त् । ऐ॒न्द्रम् । एका॑दशकपाल॒मित्येका॑दश - क॒पा॒ल॒म् । वै॒श्व॒दे॒वमिति॑ वैश्व - दे॒वम् । द्वाद॑शकपाल॒मिति॒ द्वाद॑श - क॒पा॒ल॒म् । बा॒र्.॒ह॒स्प॒त्यम् । च॒रुम् । वै॒ष्ण॒वम् । त्रि॒क॒पा॒लमिति॑ त्रि - क॒पा॒लम् । तृ॒तीये᳚ । सं॒ॅव॒थ्स॒र इति॑ सं - व॒थ्स॒रे । अ॒भि॒जितेत्य॑भि - जिता᳚ । य॒जे॒त॒ । यत् । अ॒ष्टाक॑पाल॒ इत्य॒ष्टा - क॒पा॒लः॒ । भव॑ति । अ॒ष्टाक्ष॒रेत्य॒ष्टा - अ॒क्ष॒रा॒ । गा॒य॒त्री । आ॒ग्ने॒यम् । गा॒य॒त्रम् । प्रा॒त॒स्स॒व॒नमिति॑ प्रातः - स॒व॒नम् । प्रा॒त॒स्स॒व॒नमिति॑ प्रातः-स॒व॒नम् । ए॒व । तेन॑ । दा॒धा॒र॒ । गा॒य॒त्रीम् । छन्दः॑ । यत् । एका॑दशकपाल॒ इत्येका॑दश - क॒पा॒लः॒ । भव॑ति । एका॑दशाक्ष॒रेत्येका॑दश - अ॒क्ष॒रा॒ । त्रि॒ष्टुक् । ऐ॒न्द्रम् । त्रैष्टु॑भम् । माद्ध्य॑न्दिनम् । सव॑नम् । माद्ध्य॑न्दिनम् । ए॒व । सव॑नम् । तेन॑ । दा॒धा॒र॒ । त्रि॒ष्टुभ᳚म् । \textbf{  20} \newline
                  \newline
                                \textbf{ TS 5.6.5.2} \newline
                  छन्दः॑ । यत् । द्वाद॑शकपाल॒ इति॒ द्वाद॑श - क॒पा॒लः॒ । भव॑ति । द्वाद॑शाक्ष॒रेति॒ द्वाद॑शा-अ॒क्ष॒रा॒ । जग॑ती । वै॒श्व॒दे॒वमिति॑ वैश्व - दे॒वम् । जाग॑तम् । तृ॒ती॒य॒स॒व॒नमिति॑ तृतीय - स॒व॒नम् । तृ॒ती॒य॒स॒व॒नमिति॑ तृतीय - स॒व॒नम् । ए॒व । तेन॑ । दा॒धा॒र॒ । जग॑तीम् । छन्दः॑ । यत् । बा॒र्.॒ह॒स्प॒त्यः । च॒रुः । भव॑ति । ब्रह्म॑ । वै । दे॒वाना᳚म् । बृह॒स्पतिः॑ । ब्रह्म॑ । ए॒व । तेन॑ । दा॒धा॒र॒ । यत् । वै॒ष्ण॒वः । त्रि॒क॒पा॒ल इति॑ त्रि - क॒पा॒लः । भव॑ति । य॒ज्ञ्ः । वै । विष्णुः॑ । य॒ज्ञ्म् । ए॒व । तेन॑ । दा॒धा॒र॒ । यत् । तृ॒तीये᳚ । सं॒ॅव॒थ्स॒र इति॑ सं - व॒थ्स॒रे । अ॒भि॒जितेत्य॑भि - जिता᳚ । यज॑ते । अ॒भिजि॑त्या॒ इत्य॒भि - जि॒त्यै॒ । यत् । सं॒ॅव॒थ्स॒रमिति॑ सं - व॒थ्स॒रम् । उख्य᳚म् । बि॒भर्ति॑ । इ॒मम् । ए॒व । \textbf{  21} \newline
                  \newline
                                \textbf{ TS 5.6.5.3} \newline
                  तेन॑ । लो॒कम् । स्पृ॒णो॒ति॒ । यत् । द्वि॒तीये᳚ । सं॒ॅव॒थ्स॒र इति॑ सं - व॒थ्स॒रे । अ॒ग्निम् । चि॒नु॒ते । अ॒न्तरि॑क्षम् । ए॒व । तेन॑ । स्पृ॒णो॒ति॒ । यत् । तृ॒तीये᳚ । सं॒ॅव॒थ्स॒र इति॑ सं - व॒थ्स॒रे । यज॑ते । अ॒मुम् । ए॒व । तेन॑ । लो॒कम् । स्पृ॒णो॒ति॒ । ए॒तम् । वै । परः॑ । आ॒ट्णा॒रः । क॒क्षीवा॒निति॑ क॒क्षी - वा॒न् । औ॒शि॒जः । वी॒तह॑व्य॒ इति॑ वी॒त - ह॒व्यः॒ । श्रा॒य॒सः । त्र॒सद॑स्युः । पौ॒रु॒कु॒थ्स्य इति॑ पौरु - कु॒थ्स्यः । प्र॒जाका॑मा॒ इति॑ प्र॒जा-का॒माः॒ । अ॒चि॒न्व॒त॒ । ततः॑ । वै । ते । स॒हस्रꣳ॑सहस्र॒मिति॑ स॒हस्रं᳚ - स॒ह॒स्र॒म् । पु॒त्रान् । अ॒वि॒न्द॒न्त॒ । प्रथ॑ते । प्र॒जयेति॑ प्र - जया᳚ । प॒शुभि॒रिति॑ प॒शु-भिः॒ । ताम् । मात्रा᳚म् । आ॒प्नो॒ति॒ । याम् । ते॒ । अग॑च्छन्न् । यः । ए॒वम् ( ) । वि॒द्वान् । ए॒तम् । अ॒ग्निम् । चि॒नु॒ते ॥ \textbf{  22} \newline
                  \newline
                      (दा॒धा॒र॒ त्रि॒ष्टुभ॑ - मि॒ममे॒वै - वं - च॒त्वारि॑ च)  \textbf{(A5)} \newline \newline
                                \textbf{ TS 5.6.6.1} \newline
                  प्र॒जाप॑ति॒रिति॑ प्र॒जा - प॒तिः॒ । अ॒ग्निम् । अ॒चि॒नु॒त॒ । सः । क्षु॒रप॑वि॒रिति॑ क्षु॒र - प॒विः॒ । भू॒त्वा । अ॒ति॒ष्ठ॒त् । तम् । दे॒वाः । बिभ्य॑तः । न । उपेति॑ । आ॒य॒न्न् । ते । छन्दो॑भि॒रिति॒ छन्दः॑ - भिः॒ । आ॒त्मान᳚म् । छा॒द॒यि॒त्वा । उपेति॑ । आ॒य॒न्न् । तत् । छन्द॑साम् । छ॒न्द॒स्त्वमिति॑ छन्दः - त्वम् । ब्रह्म॑ । वै । छन्दाꣳ॑सि । ब्रह्म॑णः । ए॒तत् । रू॒पम् । यत् । कृ॒ष्णा॒जि॒नमिति॑ कृष्ण - अ॒जि॒नम् । कार्ष्णी॒ इति॑ । उ॒पा॒नहौ᳚ । उपेति॑ । मु॒ञ्च॒ते॒ । छन्दो॑भि॒रिति॒ छन्दः॑ - भिः॒ । ए॒व । आ॒त्मान᳚म् । छा॒द॒यि॒त्वा । अ॒ग्निम् । उपेति॑ । च॒र॒ति॒ । आ॒त्मनः॑ । अहिꣳ॑सायै । दे॒व॒नि॒धिरिति॑ देव - नि॒धिः । वै । ए॒षः । नीति॑ । धी॒य॒ते॒ । यत् । अ॒ग्निः । \textbf{  23} \newline
                  \newline
                                \textbf{ TS 5.6.6.2} \newline
                  अ॒न्ये । वा॒ । वै । नि॒धिमिति॑ नि - धिम् । अगु॑प्तम् । वि॒न्दन्ति॑ । न । वा॒ । प्रति॑ । प्रेति॑ । जा॒ना॒ति॒ । उ॒खाम् । एति॑ । क्रा॒म॒ति॒ । आ॒त्मान᳚म् । ए॒व । अ॒धि॒पामित्य॑धि - पाम् । कु॒रु॒ते॒ । गुप्त्यै᳚ । अथो॒ इति॑ । खलु॑ । आ॒हुः॒ । न । आ॒क्रम्येत्या᳚ - क्रम्या᳚ । इति॑ । नै॒र्.॒ऋ॒तीति॑ नैः-ऋ॒ती । उ॒खा । यत् । आ॒क्रामे॒दित्या᳚ - क्रामे᳚त् । निर्.ऋ॑त्या॒ इति॒ निः-ऋ॒त्यै॒ । आ॒त्मान᳚म् । अपीति॑ । द॒द्ध्या॒त् । तस्मा᳚त् । न । आ॒क्रम्येत्या᳚-क्रम्या᳚ । पु॒रु॒ष॒शी॒र्॒.षमिति॑ पुरुष - शी॒र्॒.षम् । उपेति॑ । द॒धा॒ति॒ । गुप्त्यै᳚ । अथो॒ इति॑ । यथा᳚ । ब्रू॒यात् । ए॒तत् । मे॒ । गो॒पा॒य॒ । इति॑ । ता॒दृक् । ए॒व । तत् । \textbf{  24} \newline
                  \newline
                                \textbf{ TS 5.6.6.3} \newline
                  प्र॒जाप॑ति॒रिति॑ प्र॒जा - प॒तिः॒ । वै । अथ॑र्वा । अ॒ग्निः । ए॒व । द॒द्ध्यङ् । आ॒थ॒र्व॒णः । तस्य॑ । इष्ट॑काः । अ॒स्थानि॑ । ए॒तम् । ह॒ । वाव । तत् । ऋषिः॑ । अ॒भ्यनू॑वा॒चेत्य॑भि - अनू॑वाच । इन्द्रः॑ । द॒धी॒चः । अ॒स्थभि॒रित्य॒स्थ - भिः॒ । इति॑ । यत् । इष्ट॑काभिः । अ॒ग्निम् । चि॒नोति॑ । सात्मा॑न॒मिति॒ स - आ॒त्मा॒न॒म् । ए॒व । अ॒ग्निम् । चि॒नु॒ते॒ । सात्मेति॒ स - आ॒त्मा॒ । अ॒मुष्मिन्न्॑ । लो॒के । भ॒व॒ति॒ । यः । ए॒वम् । वेद॑ । शरी॑रम् । वै । ए॒तत् । अ॒ग्नेः । यत् । चित्यः॑ । आ॒त्मा । वै॒श्वा॒न॒रः । यत् । चि॒ते । वै॒श्वा॒न॒रम् । जु॒होति॑ । शरी॑रम् । ए॒व । सꣳ॒॒स्कृत्य॑ । \textbf{  25} \newline
                  \newline
                                \textbf{ TS 5.6.6.4} \newline
                  अ॒भ्यारो॑ह॒तीत्य॑भि - आरो॑हति । शरी॑रम् । वै । ए॒तत् । यज॑मानः । समिति॑ । कु॒रु॒ते॒ । यत् । अ॒ग्निम् । चि॒नु॒ते । यत् । चि॒ते । वै॒श्वा॒न॒रम् । जु॒होति॑ । शरी॑रम् । ए॒व । सꣳ॒॒स्कृत्य॑ । आ॒त्मना᳚ । अ॒भ्यारो॑ह॒तीत्य॑भि -आरो॑हति । तस्मा᳚त् । तस्य॑ । न । अवेति॑ । द्य॒न्ति॒ । जीवन्न्॑ । ए॒व । दे॒वान् । अपीति॑ । ए॒ति॒ । वै॒श्वा॒न॒र्या । ऋ॒चा । पुरी॑षम् । उपेति॑ । द॒धा॒ति॒ । इ॒यम् । वै । अ॒ग्निः । वै॒श्वा॒न॒रः । तस्य॑ । ए॒षा । चितिः॑ । यत् । पुरी॑षम् । अ॒ग्निम् । ए॒व । वै॒श्वा॒न॒रम् । चि॒नु॒ते॒ । ए॒षा । वै । अ॒ग्नेः ( ) । प्रि॒या । त॒नूः । यत् । वै॒श्वा॒न॒रः । प्रि॒याम् । ए॒व । अ॒स्य॒ । त॒नुव᳚म् । अवेति॑ । रु॒न्धे॒ ॥ \textbf{  26} \newline
                  \newline
                      (अ॒ग्नि - स्तथ् - सꣳ॒॒स्कृत्या॒ - ग्ने - र्दश॑ च)  \textbf{(A6)} \newline \newline
                                \textbf{ TS 5.6.7.1} \newline
                  अ॒ग्नेः । वै । दी॒क्षया᳚ । दे॒वाः । वि॒राज॒मिति॑ वि-राज᳚म् । आ॒प्नु॒व॒न्न् । ति॒स्रः । रात्रीः᳚ । दी॒क्षि॒तः । स्या॒त् । त्रि॒पदेति॑ त्रि - पदा᳚ । वि॒राडिति॑ वि - राट् । वि॒राज॒मिति॑ वि - राज᳚म् । आ॒प्नो॒ति॒ । षट् । रात्रीः᳚ । दी॒क्षि॒तः । स्या॒त् । षट् । वै । ऋ॒तवः॑ । सं॒ॅव॒थ्स॒र इति॑ सं - व॒थ्स॒रः । सं॒ॅव॒थ्स॒र इति॑ सं - व॒थ्स॒रः । वि॒राडिति॑ वि - राट् । वि॒राज॒मिति॑ वि - राज᳚म् । आ॒प्नो॒ति॒ । दश॑ । रात्रीः᳚ । दी॒क्षि॒तः । स्या॒त् । दशा᳚क्ष॒रेति॒ दश॑ - अ॒क्ष॒रा॒ । वि॒राडिति॑ वि - राट् । वि॒राज॒मिति॑ वि - राज᳚म् । आ॒प्नो॒ति॒ । द्वाद॑श । रात्रीः᳚ । दी॒क्षि॒तः । स्या॒त् । द्वाद॑श । मासाः᳚ । सं॒ॅव॒थ्स॒र इति॑ सं - व॒थ्स॒रः । सं॒ॅव॒थ्स॒र इति॑ सं - व॒थ्स॒रः । वि॒राडिति॑ वि - राट् । वि॒राज॒मिति॑ वि - राज᳚म् । आ॒प्नो॒ति॒ । त्रयो॑द॒शेति॒ त्रयः॑ - द॒श॒ । रात्रीः᳚ । दी॒क्षि॒तः । स्या॒त् । त्रयो॑द॒शेति॒ त्रयः॑ - द॒श॒ । \textbf{  27} \newline
                  \newline
                                \textbf{ TS 5.6.7.2} \newline
                  मासाः᳚ । सं॒ॅव॒थ्स॒र इति॑ सं-व॒थ्स॒रः । सं॒ॅव॒थ्स॒र इति॑ सं-व॒थ्स॒रः । वि॒राडिति॑ वि -   राट् । वि॒राज॒मिति॑ वि - राज᳚म् । आ॒प्नो॒ति॒ । पञ्च॑द॒शेति॒ पञ्च॑ - द॒श॒ । रात्रीः᳚ । दी॒क्षि॒तः । स्या॒त् । पञ्च॑द॒शेति॒ पञ्च॑ - द॒श॒ । वै । अ॒द्‌र्ध॒मा॒सस्येत्य॒॑द्‌र्ध - मा॒सस्य॑ ।   रात्र॑यः । अ॒द्‌र्ध॒मा॒स॒श इत्य॒॑द्‌र्धमास -   शः । सं॒ॅव॒थ्स॒र इति॑ सं - व॒थ्स॒रः । आ॒प्य॒ते॒ । सं॒ॅव॒थ्स॒र इति॑ सं - व॒थ्स॒रः । वि॒राडिति॑ वि - राट् । वि॒राज॒मिति॑ वि - राज᳚म् । आ॒प्नो॒ति॒ । स॒प्तद॒शेति॑ स॒प्त - द॒श॒ । रात्रीः᳚ । दी॒क्षि॒तः । स्या॒त् । द्वाद॑श । मासाः᳚ । पञ्च॑ । ऋ॒तवः॑ । सः । सं॒ॅव॒थ्स॒र इति॑ सं - व॒थ्स॒रः । सं॒ॅव॒थ्स॒र इति॑ सं - व॒थ्स॒रः । वि॒राडिति॑ वि - राट् । वि॒राज॒मिति॑ वि - राज᳚म् । आ॒प्नो॒ति॒ । चतु॑र्विꣳशति॒मिति॒ चतुः॑ - विꣳ॒॒श॒ति॒म् । रात्रीः᳚ । दी॒क्षि॒तः । स्या॒त् । चतु॑र्विꣳशति॒रिति॒ चतुः॑ - विꣳ॒॒श॒तिः॒ । अ॒द्‌र्ध॒मा॒सा इत्य॑द्‌र्ध-मा॒साः । सं॒ॅव॒थ्स॒र इति॑ सं - व॒थ्स॒रः । सं॒ॅव॒थ्स॒र इति॑ सं - व॒थ्स॒रः । वि॒राडिति॑ वि - राट् । वि॒राज॒मिति॑ वि - राज᳚म् । आ॒प्नो॒ति॒ । त्रिꣳ॒॒शत᳚म् । रात्रीः᳚ । दी॒क्षि॒तः । स्या॒त् । \textbf{  28} \newline
                  \newline
                                \textbf{ TS 5.6.7.3} \newline
                  त्रिꣳ॒॒शद॑क्ष॒रेति॑ त्रिꣳ॒॒शत् - अ॒क्ष॒रा॒ । वि॒राडिति॑ वि - राट् । वि॒राज॒मिति॑ वि - राज᳚म् । आ॒प्नो॒ति॒ । मास᳚म् । दी॒क्षि॒तः । स्या॒त् । यः । मासः॑ । सः । सं॒ॅव॒थ्स॒र इति॑ सं - व॒थ्स॒रः । सं॒ॅव॒थ्स॒र इति॑ सं - व॒थ्स॒रः । वि॒राडिति॑ वि - राट् । वि॒राज॒मिति॑ वि - राज᳚म् । आ॒प्नो॒ति॒ । च॒तुरः॑ । मा॒सः । दी॒क्षि॒तः । स्या॒त् । च॒तुरः॑ । वै । ए॒तम् । मा॒सः । वस॑वः । अ॒बि॒भ॒रुः॒ । ते । पृ॒थि॒वीम् । एति॑ । अ॒ज॒य॒न्न् । गा॒य॒त्रीम् । छन्दः॑ । अ॒ष्टौ । रु॒द्राः । ते । अ॒न्तरि॑क्षम् । एति॑ । अ॒ज॒य॒न्न् । त्रि॒ष्टुभ᳚म् । छन्दः॑ । द्वाद॑श । आ॒दि॒त्याः । ते । दिव᳚म् । एति॑ । अ॒ज॒य॒न्न् । जग॑तीम् । छन्दः॑ । ततः॑ । वै । ते ( ) । व्या॒वृत॒मिति॑ वि - आ॒वृत᳚म् । अ॒ग॒च्छ॒न्न् । श्रैष्ठ्य᳚म् । दे॒वाना᳚म् । तस्मा᳚त् । द्वाद॑श । मा॒सः । भृ॒त्वा । अ॒ग्निम् । चि॒न्वी॒त॒ । द्वाद॑श । मासाः᳚ । सं॒ॅव॒थ्स॒र इति॑ सं - व॒थ्स॒रः । सं॒ॅव॒थ्स॒र इति॑ सं - व॒थ्स॒रः । अ॒ग्निः । चित्यः॑ । तस्य॑ । अ॒हो॒रा॒त्राणीत्य॑हः - रा॒त्राणि॑ । इष्ट॑काः । आ॒प्तेष्ट॑क॒मित्या॒प्त-इ॒ष्ट॒क॒म् । ए॒न॒म् । चि॒नु॒ते॒ । अथो॒ इति॑ । व्या॒वृत॒मिति॑ वि - आ॒वृत᳚म् । ए॒व । ग॒च्छ॒ति॒ । श्रैष्ठ्य᳚म् । स॒मा॒नाना᳚म् ॥ \textbf{  29 } \newline
                  \newline
                      (स्या॒त् त्रयो॑दश - त्रिꣳ॒॒शतꣳ॒॒ रात्री᳚र्दीक्षि॒तः स्या॒द् - वै ते᳚ - ऽष्टाविꣳ॑शतिश्च)  \textbf{(A7)} \newline \newline
                                \textbf{ TS 5.6.8.1} \newline
                  सु॒व॒र्गायेति॑ सुवः - गाय॑ । वै । ए॒षः । लो॒काय॑ । ची॒य॒ते॒ । यत् । अ॒ग्निः । तम् । यत् । न । अ॒न्वा॒रोहे॒दित्य॑नु - आ॒रोहे᳚त् । सु॒व॒र्गादिति॑ सुवः-गात् । लो॒कात् । यज॑मानः । ही॒ये॒त॒ । पृ॒थि॒वीम् । एति॑ । अ॒क्र॒मि॒ष॒म् । प्रा॒ण इति॑ प्र - अ॒नः । मा॒ । मा । हा॒सी॒त् । अ॒न्तरि॑क्षम् । एति॑ । अ॒क्र॒मि॒ष॒म् । प्र॒जेति॑ प्र - जा । मा॒ । मा । हा॒सी॒त् । दिव᳚म् । एति॑ । अ॒क्र॒मि॒ष॒म् । सुवः॑ । अ॒ग॒न्म॒ । इति॑ । आ॒ह॒ । ए॒षः । वै । अ॒ग्नेः । अ॒न्वा॒रो॒ह इत्य॑नु-आ॒रो॒हः । तेन॑ । ए॒व । ए॒न॒म् । अ॒न्वारो॑ह॒तीत्य॑नु - आरो॑हति । सु॒व॒र्गस्येति॑ सुवः - गस्य॑ । लो॒कस्य॑ । सम॑ष्ट्या॒ इति॒ सं - अ॒ष्ट्यै॒ । यत् । प॒क्षस॑म्मिता॒मिति॑ प॒क्ष - स॒म्मि॒ता॒म् । मि॒नु॒यात् । \textbf{  30} \newline
                  \newline
                                \textbf{ TS 5.6.8.2} \newline
                  कनी॑याꣳसम् । य॒ज्ञ्॒क्र॒तुमिति॑ यज्ञ् - क्र॒तुम् । उपेति॑ । इ॒या॒त् । पापी॑यसी । अ॒स्य॒ । आ॒त्मनः॑ । प्र॒जेति॑ प्र - जा । स्या॒त् । वेदि॑सम्मिता॒मिति॒ वेदि॑ - स॒म्मि॒ता॒म् । मि॒नो॒ति॒ । ज्यायाꣳ॑सम् । ए॒व । य॒ज्ञ्॒क्र॒तुमिति॑ यज्ञ् - क्र॒तुम् । उपेति॑ । ए॒ति॒ । न । अ॒स्य॒ । आ॒त्मनः॑ । पापी॑यसी । प्र॒जेति॑ प्र - जा । भ॒व॒ति॒ । सा॒ह॒स्रम् । चि॒न्वी॒त॒ । प्र॒थ॒मम् । चि॒न्वा॒नः । स॒हस्र॑सम्मित॒ इति॑ स॒हस्र॑ - स॒म्मि॒तः॒ । वै । अ॒यम् । लो॒कः ।   इ॒मम् । ए॒व । लो॒कम् । अ॒भीति॑ । ज॒य॒ति॒ । द्विषा॑हस्र॒मिति॒ द्वि - सा॒ह॒स्र॒म् । चि॒न्वी॒त॒ । द्वि॒तीय᳚म् । चि॒न्वा॒नः । द्विषा॑हस्र॒मिति॒ द्वि - सा॒ह॒स्र॒म् । वै । अ॒न्तरि॑क्षम् । अ॒न्तरि॑क्षम् । ए॒व । अ॒भीति॑ । ज॒य॒ति॒ । त्रिषा॑हस्र॒मिति॒ त्रि - सा॒ह॒स्र॒म् । चि॒न्वी॒त॒ । तृ॒तीय᳚म् । चि॒न्वा॒नः । \textbf{  31} \newline
                  \newline
                                \textbf{ TS 5.6.8.3} \newline
                  त्रिषा॑हस्र॒ इति॒ त्रि - सा॒ह॒स्रः॒ । वै । अ॒सौ । लो॒कः । अ॒मुम् । ए॒व । लो॒कम् । अ॒भीति॑ । ज॒य॒ति॒ । जा॒नु॒द॒घ्नमिति॑ जानु - द॒घ्नम् । चि॒न्वी॒त॒ । प्र॒थ॒मम् । चि॒न्वा॒नः । गा॒य॒त्रि॒या । ए॒व । इ॒मम् । लो॒कम् । अ॒भ्यारो॑ह॒तीत्य॑भि - आरो॑हति । ना॒भि॒द॒घ्नमिति॑ नाभि - द॒घ्नम् । चि॒न्वी॒त॒ ।  द्वि॒तीय᳚म् । चि॒न्वा॒नः । त्रि॒ष्टुभा᳚ । ए॒व । अ॒न्तरि॑क्षम् । अ॒भ्यारो॑ह॒तीत्य॑भि - आरो॑हति । ग्री॒व॒द॒घ्नमिति॑ ग्रीव - द॒घ्नम् । चि॒न्वी॒त॒ । तृ॒तीय᳚म् । चि॒न्वा॒नः । जग॑त्या । ए॒व । अ॒मुम् ।  लो॒कम् । अ॒भ्यारो॑ह॒तीत्य॑भि - आरो॑हति । न । अ॒ग्निम् । चि॒त्वा । रा॒माम् । उपेति॑ । इ॒या॒त् । अ॒यो॒नौ । रेतः॑ । धा॒स्या॒मि॒ । इति॑ । न । द्वि॒तीय᳚म् । चि॒त्वा । अ॒न्यस्य॑ । स्त्रिय᳚म् । \textbf{  32} \newline
                  \newline
                                \textbf{ TS 5.6.8.4} \newline
                  उपेति॑ । इ॒या॒त् । न । तृ॒तीय᳚म् । चि॒त्वा । काम् । च॒न । उपेति॑ । इ॒या॒त् । रेतः॑ । वै । ए॒तत् । नीति॑ । ध॒त्ते॒ । यत् । अ॒ग्निम् । चि॒नु॒ते । यत् । उ॒पे॒यादित्यु॑प - इ॒यात् । रेत॑सा । वीति॑ । ऋ॒द्ध्ये॒त॒ । अथो॒ इति॑ । खलु॑ । आ॒हुः॒ । अ॒प्र॒ज॒स्यमित्य॑प्र-ज॒स्यम् । तत् । यत् । न । उ॒पे॒यादित्यु॑प - इ॒यात् । इति॑ । यत् । रे॒त॒स्सिचा॒विति॑ रेतः - सिचौ᳚ । उ॒प॒दधा॒तीत्यु॑प - दधा॑ति । ते इति॑ । ए॒व । यज॑मानस्य । रेतः॑ । बि॒भृ॒तः॒ । तस्मा᳚त् । उपेति॑ । इ॒या॒त् । रेत॑सः । अस्क॑न्दाय । त्रीणि॑ । वाव । रेताꣳ॑सि । पि॒ता । पु॒त्रः । पौत्रः॑ । \textbf{  33} \newline
                  \newline
                                \textbf{ TS 5.6.8.5} \newline
                  यत् । द्वे इति॑ । रे॒त॒स्सिचा॒विति॑ रेतः-सिचौ᳚ । उ॒प॒द॒द्ध्यादित्यु॑प-द॒ध्यात् । रेतः॑ । अ॒स्य॒ । वीति॑ । छि॒न्द्या॒त् । ति॒स्रः । उपेति॑ । द॒धा॒ति॒ । रेत॑सः । संत॑त्या॒ इति॒ सं - त॒त्यै॒ । इ॒यम् । वाव । प्र॒थ॒मा । रे॒त॒स्सिगिति॑ रेतः - सिक् । वाक् । वै । इ॒यम् । तस्मा᳚त् । पश्य॑न्ति । इ॒माम् । पश्य॑न्ति । वाच᳚म् । वद॑न्तीम् । अ॒न्तरि॑क्षम् । द्वि॒तीया᳚ । प्रा॒ण इति॑ प्र - अ॒नः । वै । अ॒न्तरि॑क्षम् । तस्मा᳚त् । न । अ॒न्तरि॑क्षम् । पश्य॑न्ति । न । प्रा॒णमिति॑ प्र - अ॒नम् । अ॒सौ । तृ॒तीया᳚ ।   चक्षुः॑ । वै । अ॒सौ । तस्मा᳚त् । पश्य॑न्ति । अ॒मूम् । पश्य॑न्ति । चक्षुः॑ । यजु॑षा । इ॒माम् । च॒ । \textbf{  34} \newline
                  \newline
                                \textbf{ TS 5.6.8.6} \newline
                  अ॒मूम् । च॒ । उपेति॑ । द॒धा॒ति॒ । मन॑सा । म॒द्ध्य॒माम् । ए॒षाम् । लो॒काना᳚म् । क्लृप्त्यै᳚ । अथो॒ इति॑ । प्रा॒णाना॒मिति॑ प्र - अ॒नाना᳚म् । इ॒ष्टः । य॒ज्ञ्ः । भृगु॑भि॒रिति॒ भृगु॑ - भिः॒ । आ॒शी॒र्दा इत्या॑शीः - दाः । वसु॑भि॒रिति॒ वसु॑ - भिः॒ । तस्य॑ । ते॒ । इ॒ष्टस्य॑ । वी॒तस्य॑ । द्रवि॑णा । इ॒ह । भ॒क्षी॒य॒ । इति॑ । आ॒ह॒ । स्तु॒त॒श॒स्त्रे इति॑ स्तुत - श॒स्त्रे । ए॒व । ए॒तेन॑ । दु॒हे॒ । पि॒ता । मा॒त॒रिश्वा᳚ । अच्छि॑द्रा । प॒दा । धाः॒ । अच्छि॑द्राः । उ॒शिजः॑ । प॒दा । अन्विति॑ । त॒क्षुः॒ । सोमः॑ । वि॒श्व॒विदिति॑ विश्व - वित् । ने॒ता । ने॒ष॒त् । बृह॒स्पतिः॑ । उ॒क्था॒म॒दानीत्यु॑क्थ - म॒दानि॑ । शꣳ॒॒सि॒ष॒त् । इति॑ । आ॒ह॒ । ए॒तत् । वै ( ) । अ॒ग्नेः । उ॒क्थम् । तेन॑ । ए॒व । ए॒न॒म् । अन्विति॑ । शꣳ॒॒स॒ति॒ ॥ \textbf{  35 } \newline
                  \newline
                      (मि॒नु॒यात् - तृ॒तीयं॑ चिन्वा॒नः - स्त्रियं॒ - पौत्र॑ - श्च॒ - वै - स॒प्त च॑)  \textbf{(A8)} \newline \newline
                                \textbf{ TS 5.6.9.1} \newline
                  सू॒यते᳚ । वै । ए॒षः । अ॒ग्नी॒नाम् । यः । उ॒खाया᳚म् । भ्रि॒यते᳚ । यत् । अ॒धः । सा॒दये᳚त् । गर्भाः᳚ । प्र॒पादु॑का॒ इति॑ प्र - पादु॑काः । स्युः॒ । अथो॒ इति॑ । यथा᳚ । स॒वात् । प्र॒त्य॒व॒रोह॒तीति॑ प्रति - अ॒व॒रोह॑ति । ता॒दृक् । ए॒व । तत् । आ॒स॒न्दी । सा॒द॒य॒ति॒ । गर्भा॑णाम् । धृत्यै᳚ । अप्र॑पादा॒येत्यप्र॑ - पा॒दा॒य॒ । अथो॒ इति॑ । स॒वम् । ए॒व । ए॒न॒म् । क॒रो॒ति॒ । गर्भः॑ । वै । ए॒षः । यत् । उख्यः॑ । योनिः॑ । शि॒क्य᳚म् । यत् । शि॒क्या᳚त् । उ॒खाम् । नि॒रूहे॒दिति॑ निः - ऊहे᳚त् । योनेः᳚ । गर्भ᳚म् । निरिति॑ । ह॒न्या॒त् । षडु॑द्याम॒मिति॒ षट्-उ॒द्या॒म॒म् । शि॒क्य᳚म् । भ॒व॒ति॒ । षो॒ढा॒वि॒हि॒त इति॑ षोढा - वि॒हि॒तः । वै । \textbf{  36} \newline
                  \newline
                                \textbf{ TS 5.6.9.2} \newline
                  पुरु॑षः । आ॒त्मा । च॒ । शिरः॑ । च॒ । च॒त्वारि॑ । अङ्गा॑नि । आ॒त्मन्न् । ए॒व । ए॒न॒म् । बि॒भ॒र्ति॒ । प्र॒जाप॑ति॒रिति॑ प्र॒जा - प॒तिः॒ । वै । ए॒षः । यत् । अ॒ग्निः । तस्य॑ । उ॒खा । च॒ । उ॒लूख॑लम् । च॒ । स्तनौ᳚ । तौ । अ॒स्य॒ । प्र॒जा इति॑ प्र - जाः । उपेति॑ । जी॒व॒न्ति॒ । यत् । उ॒खाम् । च॒ । उ॒लूख॑लम् । च॒ । उ॒प॒दधा॒तीत्यु॑प - दधा॑ति । ताभ्या᳚म् । ए॒व । यज॑मानः । अ॒मुष्मिन्न्॑ । लो॒के । अ॒ग्निम् । दु॒हे॒ । सं॒ॅव॒थ्स॒र इति॑ सं - व॒थ्स॒रः । वै । ए॒षः । यत् । अ॒ग्निः । तस्य॑ । त्रे॒धा॒वि॒हि॒ता इति॑ त्रेधा - वि॒हि॒ताः । इष्ट॑काः । प्रा॒जा॒प॒त्या इति॑ प्राजा - प॒त्याः॒ । वै॒ष्ण॒वीः । \textbf{  37} \newline
                  \newline
                                \textbf{ TS 5.6.9.3} \newline
                  वै॒श्व॒क॒र्म॒णीरिति॑ वैश्व - क॒र्म॒णीः । अ॒हो॒रा॒त्राणीत्य॑हः - रा॒त्राणि॑ । ए॒व । अ॒स्य॒ । प्रा॒जा॒प॒त्या इति॑ प्राजा - प॒त्याः । यत् । उख्य᳚म् । बि॒भर्ति॑ । प्रा॒जा॒प॒त्या इति॑ प्राजा-प॒त्याः । ए॒व । तत् । उपेति॑ । ध॒त्ते॒ । यत् । स॒मिध॒ इति॑ सं - इधः॑ । आ॒दधा॒तीत्या᳚ - दधा॑ति । वै॒ष्ण॒वाः । वै । वन॒स्पत॑यः । वै॒ष्ण॒वीः । ए॒व । तत् । उपेति॑ । ध॒त्ते॒ । यत् । इष्ट॑काभिः । अ॒ग्निम् । चि॒नोति॑ । इ॒यम् । वै । वि॒श्वक॒र्मेति॑ वि॒श्व - क॒र्मा॒ । वै॒श्व॒क॒र्म॒णीरिति॑ वैश्व - क॒र्म॒णीः । ए॒व । तत् । उपेति॑ । ध॒त्ते॒ । तस्मा᳚त् । आ॒हुः॒ । त्रि॒वृदिति॑ त्रि - वृत् । अ॒ग्निः । इति॑ । तम् । वै । ए॒तम् । यज॑मानः । ए॒व । चि॒न्वी॒त॒ । यत् । अ॒स्य॒ । अ॒न्यः ( ) । चि॒नु॒यात् । यत् । तम् । दक्षि॑णाभिः । न । रा॒धये᳚त् । अ॒ग्निम् । अ॒स्य॒ । वृ॒ञ्जी॒त॒ । यः । अ॒स्य॒ । अ॒ग्निम् । चि॒नु॒यात् । तम् । दक्षि॑णाभिः । रा॒ध॒ये॒त् । अ॒ग्निम् । ए॒व । तत् । स्पृ॒णो॒ति॒ ॥ \textbf{  38} \newline
                  \newline
                      (षो॒ढा॒वि॒हि॒तो वै - वै᳚ष्ण॒वी - र॒न्यो - विꣳ॑श॒तिश्च॑)  \textbf{(A9)} \newline \newline
                                \textbf{ TS 5.6.10.1} \newline
                  प्र॒जाप॑ति॒रिति॑ प्र॒जा - प॒तिः॒ । अ॒ग्निम् । अ॒चि॒नु॒त॒ । ऋ॒तुभि॒रित्यृ॒तु - भिः॒ । सं॒ॅव॒थ्स॒रमिति॑ सं - व॒थ्स॒रम् । व॒स॒न्तेन॑ । ए॒व । अ॒स्य॒ । पू॒र्वा॒द्‌र्धमिति॑ पूर्व - अ॒द्‌र्धम् । अ॒चि॒नु॒त॒ । ग्री॒ष्मेण॑ । दक्षि॑णम् । प॒क्षम् । व॒र्॒.षाभिः॑ । पुच्छ᳚म् । श॒रदा᳚ । उत्त॑र॒मित्युत् - त॒र॒म् । प॒क्षम् । हे॒म॒न्तेन॑ । मद्ध्य᳚म् । ब्रह्म॑णा । वै । अ॒स्य॒ । तत् । पू॒र्वा॒द्‌र्धमिति॑ पूर्व - अ॒द्‌र्धम् । अ॒चि॒नु॒त॒ । क्ष॒त्रेण॑ । दक्षि॑णम् । प॒क्षम् । प॒शुभि॒रिति॑ प॒शु - भिः॒ । पुच्छ᳚म् । वि॒शा । उत्त॑र॒मित्युत् - त॒र॒म् । प॒क्षम् । आ॒शया᳚ । मद्ध्य᳚म् । यः । ए॒वम् । वि॒द्वान् । अ॒ग्निम् । चि॒नु॒ते । ऋ॒तुभि॒रित्यृ॒तु - भिः॒ । ए॒व । ए॒न॒म् । चि॒नु॒ते॒ । अथो॒ इति॑ । ए॒तत् । ए॒व । सर्व᳚म् । अवेति॑ । \textbf{  39} \newline
                  \newline
                                \textbf{ TS 5.6.10.2} \newline
                  रु॒न्धे॒ । शृ॒ण्वन्ति॑ । ए॒न॒म् । अ॒ग्निम् । चि॒क्या॒नम् । अत्ति॑ । अन्न᳚म् । रोच॑ते । इ॒यम् । वाव । प्र॒थ॒मा । चितिः॑ । ओष॑धयः । वन॒स्पत॑यः । पुरी॑षम् । अ॒न्तरि॑क्षम् । द्वि॒तीया᳚ । वयाꣳ॑सि । पुरी॑षम् । अ॒सौ । तृ॒तीया᳚ । नक्ष॑त्राणि । पुरी॑षम् । य॒ज्ञ्ः । च॒तु॒र्थी । दक्षि॑णा । पुरी॑षम् । यज॑मानः । प॒ञ्च॒मी । प्र॒जेति॑ प्र - जा । पुरी॑षम् । यत् । त्रिचि॑तीक॒मिति॒ त्रि - चि॒ती॒क॒म् । चि॒न्वी॒त । य॒ज्ञ्म् । दक्षि॑णाम् । आ॒त्मान᳚म् । प्र॒जामिति॑ प्र - जाम् । अ॒न्तः । इ॒या॒त् । तस्मा᳚त् । पञ्च॑चितीक॒ इति॒ पञ्च॑ - चि॒ती॒कः॒ । चे॒त॒व्यः॑ । ए॒तत् । ए॒व । सर्व᳚म् । स्पृ॒णो॒ति॒ । यत् । ति॒स्रः । चित॑यः । \textbf{  40} \newline
                  \newline
                                \textbf{ TS 5.6.10.3} \newline
                  त्रि॒वृदिति॑ त्रि - वृत् । हि । अ॒ग्निः । यत् । द्वे इति॑ । द्वि॒पादिति॑ द्वि - पात् । यज॑मानः । प्रति॑ष्ठित्या॒ इति॒ प्रति॑ - स्थि॒त्यै॒ । पञ्च॑ । चित॑यः । भ॒व॒न्ति॒ । पाङ्क्तः॑ । पुरु॑षः । आ॒त्मान᳚म् । ए॒व । स्पृ॒णो॒ति॒ । पञ्च॑ । चित॑यः । भ॒व॒न्ति॒ । प॒ञ्चभि॒रिति॑ प॒ञ्च - भिः॒ । पुरी॑षैः । अ॒भीति॑ । ऊ॒ह॒ति॒ । दश॑ । समिति॑ । प॒द्य॒न्ते॒ । दशा᳚क्षर॒ इति॒ दश॑ - अ॒क्ष॒रः॒ । वै । पुरु॑षः । यावान्॑ । ए॒व । पुरु॑षः । तम् । स्पृ॒णो॒ति॒ । अथो॒ इति॑ । दशा᳚क्ष॒रेति॒ दश॑ - अ॒क्ष॒रा॒ । वि॒राडिति॑ वि - राट् । अन्न᳚म् । वि॒राडिति॑ वि - राट् । वि॒राजीति॑ वि - राजि॑ । ए॒व । अ॒न्नाद्य॒ इत्य॑न्न - अद्ये᳚ । प्रतीति॑ । ति॒ष्ठ॒ति॒ । सं॒ॅव॒थ्स॒र इति॑ सं - व॒थ्स॒रः । वै । ष॒ष्ठी । चितिः॑ । ऋ॒तवः॑ । पुरी॑षम् ( ) । षट् । चित॑यः । भ॒व॒न्ति॒ । षट् । पुरी॑षाणि । द्वाद॑श । समिति॑ । प॒द्य॒न्ते॒ । द्वाद॑श । मासाः᳚ । सं॒ॅव॒थ्स॒र इति॑ सं - व॒थ्स॒रः । सं॒ॅव॒थ्स॒र इति॑ सं - व॒थ्स॒रे । ए॒व । प्रतीति॑ । ति॒ष्ठ॒ति॒ ॥ \textbf{  41} \newline
                  \newline
                      (अव॒ - चित॑यः॒ - पुरी॑षं॒ - पञ्च॑दश च)  \textbf{(A10)} \newline \newline
                                \textbf{ TS 5.6.11.1} \newline
                  रोहि॑तः । धू॒म्ररो॑हित॒ इति॑ धू॒म्र - रो॒हि॒तः॒ । क॒र्कन्धु॑रोहित॒ इति॑ क॒र्कन्धु॑ - रो॒हि॒तः॒ । ते । प्रा॒जा॒प॒त्या इति॑ प्राजा - प॒त्याः । ब॒भ्रुः । अ॒रु॒णब॑भ्रु॒रित्य॑रु॒ण - ब॒भ्रुः॒ । शुक॑बभ्रु॒रिति॒ शुक॑ - ब॒भ्रुः॒ । ते । रौ॒द्राः । श्येतः॑ । श्ये॒ता॒क्ष इति॑ श्येत - अ॒क्षः । श्येत॑ग्रीव॒ इति॒ श्येत॑ - ग्री॒वः॒ । ते । पि॒तृ॒दे॒व॒त्या॑ इति॑ पितृ - दे॒व॒त्याः᳚ । ति॒स्रः । कृ॒ष्णाः । व॒शाः । वा॒रु॒ण्यः॑ । ति॒स्रः । श्वे॒ताः । व॒शाः । सौ॒र्यः॑ । मै॒त्रा॒बा॒र्.॒ह॒स्प॒त्या इति॑ मैत्रा - बा॒र्.॒ह॒स्प॒त्याः । धू॒म्रल॑लामा॒ इति॑ धू॒म्र - ल॒ला॒माः॒ । तू॒प॒राः ॥ \textbf{  42 } \newline
                  \newline
                      (रोहि॑तः॒-षड्वꣳ॑शतिः)  \textbf{(A11)} \newline \newline
                                \textbf{ TS 5.6.12.1} \newline
                  पृश्निः॑ । ति॒र॒श्चीन॑पृश्नि॒रिति॑ तिर॒श्चीन॑ - पृ॒श्निः॒ । ऊ॒द्‌र्ध्वपृ॑श्नि॒रित्यू॒द्‌र्ध्व - पृ॒श्निः॒ । ते । मा॒रु॒ताः । फ॒ल्गूः । लो॒हि॒तो॒र्णीति॑ लोहित - ऊ॒र्णीः । ब॒ल॒क्षी । ताः । सा॒र॒स्व॒त्यः॑ । पृष॑ती । स्थू॒लपृ॑ष॒तीति॑ स्थू॒ल - पृ॒ष॒ती॒ । क्षु॒द्रपृ॑ष॒तीति॑ क्षु॒द्र-पृ॒ष॒ती॒ । ताः । वै॒श्व॒दे॒व्य॑ इति॑ वैश्व - दे॒व्यः॑ । ति॒स्रः । श्या॒माः । व॒शाः । पौ॒ष्णियः॑ । ति॒स्रः । रोहि॑णीः । व॒शाः । मै॒त्रियः॑ । ऐ॒न्द्रा॒बा॒र्.॒ह॒स्प॒त्या इत्यै᳚न्द्रा - बा॒र्.॒ह॒स्प॒त्याः । अ॒रु॒णल॑लामा॒ इत्य॑रु॒ण - ल॒ला॒माः॒ । तू॒प॒राः ॥ \textbf{  43 } \newline
                  \newline
                      (पृश्निः॒ - षड्विꣳ॑शतिः)  \textbf{(A12)} \newline \newline
                                \textbf{ TS 5.6.13.1} \newline
                  शि॒ति॒बा॒हुरिति॑ शिति - बा॒हुः । अ॒न्यत॑श्शितिबाहु॒रित्य॒न्यतः॑ -शि॒ति॒बा॒हुः॒ । स॒म॒न्तशि॑तिबाहु॒रिति॑ सम॒न्त - शि॒ति॒बा॒हुः॒ । ते । ऐ॒न्द्र॒वा॒य॒वा इत्यै᳚न्द्र - वा॒य॒वाः । शि॒ति॒रन्ध्र॒ इति॑ शिति - रन्ध्रः॑ । अ॒न्यत॑श्शितिरन्ध्र॒ इत्य॒न्यतः॑ - शि॒ति॒र॒न्ध्रः॒ । स॒म॒न्तशि॑तिरन्ध्र॒ इति॑ सम॒न्त - शि॒ति॒र॒न्ध्रः॒ । ते । मै॒त्रा॒व॒रु॒णा इति॑ मैत्रा - व॒रु॒णाः । शु॒द्धवा॑ल॒ इति॑ शु॒द्ध-वा॒लः॒ । स॒र्वशु॑द्धवाल॒ इति॑ स॒र्व-शु॒द्ध॒वा॒लः॒ । म॒णिवा॑ल॒ इति॑ म॒णि - वा॒लः॒ । ते । आ॒श्वि॒नाः । ति॒स्रः । शि॒ल्पाः । व॒शाः । वै॒श्व॒दे॒व्य॑ इति॑ वैश्व - दे॒व्यः॑ । ति॒स्रः । श्येनीः᳚ । प॒र॒मे॒ष्ठिने᳚ । सो॒मा॒पौ॒ष्णा इति॑ सोमा - पौ॒ष्णाः । श्या॒मल॑लामा॒ इति॑ श्या॒म - ल॒ला॒माः॒ । तू॒प॒राः ॥ \textbf{  44 } \newline
                  \newline
                      (शि॒ति॒बा॒हुः पञ्च॑विꣳशतिः)  \textbf{(A13)} \newline \newline
                                \textbf{ TS 5.6.14.1} \newline
                  उ॒न्न॒त इत्यु॑त् - न॒तः । ऋ॒ष॒भः । वा॒म॒नः । ते । ऐ॒न्द्रा॒व॒रु॒णा इत्यै᳚न्द्रा - व॒रु॒णाः । शिति॑ककु॒दिति॒ शिति॑ - क॒कु॒त् । शि॒ति॒पृ॒ष्ठ इति॑ शिति - पृ॒ष्ठः । शिति॑भस॒दिति॒ शिति॑ - भ॒स॒त् । ते । ऐ॒न्द्रा॒बा॒र्.॒ह॒स्प॒त्या इत्यै᳚न्द्रा - बा॒र्.॒ह॒स्प॒त्याः । शि॒ति॒पादिति॑ शिति - पात् । शि॒त्योष्ठ॒ इति॑ शिति - ओष्ठः॑ । शि॒ति॒भ्रुरिति॑ शिति - भ्रुः । ते । ऐ॒न्द्रा॒वै॒ष्ण॒वा इत्यै᳚न्द्रा - वै॒ष्ण॒वाः । ति॒स्रः । सि॒द्ध्माः । व॒शाः । वै॒श्व॒क॒र्म॒ण्य॑ इति॑ वैश्व-क॒र्म॒ण्यः॑ । ति॒स्रः । धा॒त्रे । पृ॒षो॒द॒रा इति॑ पृष - उ॒द॒राः । ऐ॒न्द्रा॒पौ॒ष्णा इत्यै᳚न्द्रा - पौ॒ष्णाः । श्येत॑ललामा॒ इति॒ श्येत॑ - ल॒ला॒माः॒ । तू॒प॒राः ॥ \textbf{  45 } \newline
                  \newline
                      (उ॒न्न॒तः पञ्च॑विꣳशतिः)  \textbf{(A14)} \newline \newline
                                \textbf{ TS 5.6.15.1} \newline
                  क॒र्णाः । त्रयः॑ । या॒माः । सौ॒म्याः । त्रयः॑ । श्वि॒ति॒ङ्गाः । अ॒ग्नये᳚ । यवि॑ष्ठाय । त्रयः॑ । न॒कु॒लाः । ति॒स्रः । रोहि॑णीः । त्र्यव्य॒ इति॑ त्रि - अव्यः॑ । ताः । वसू॑नाम् । ति॒स्रः । अ॒रु॒णाः । दि॒त्यौ॒ह्यः॑ । ताः । रु॒द्राणा᳚म् । सो॒मै॒न्द्रा इति॑ सोम - ऐ॒न्द्राः । ब॒भ्रुल॑लामा॒ इति॑ ब॒भ्रु - ल॒ला॒माः॒ । तू॒प॒राः ॥ \textbf{  46} \newline
                  \newline
                      (क॒र्णास्त्रयो॑ - विꣳशतिः)  \textbf{(A15)} \newline \newline
                                \textbf{ TS 5.6.16.1} \newline
                  शु॒ण्ठाः । त्रयः॑ । वै॒ष्ण॒वाः । अ॒धी॒लो॒ध॒कर्णा॒ इत्य॑धीलोध - कर्णाः᳚ । त्रयः॑ । विष्ण॑वे । उ॒रु॒क्र॒मायेत्यु॑रु - क्र॒माय॑ । ल॒फ्सु॒दिनः॑ । त्रयः॑ । विष्ण॑वे । उ॒रु॒गा॒यायेत्यु॑रु - गा॒याय॑ । पञ्चा॑वी॒रिति॒ पञ्च॑ - अ॒वीः॒ । ति॒स्रः । आ॒दि॒त्याना᳚म् । त्रि॒व॒थ्सा इति॑ त्रि - व॒थ्साः । ति॒स्रः । अङ्गि॑रसाम् । ऐ॒न्द्रा॒वै॒ष्ण॒वा इत्यै᳚न्द्रा - वै॒ष्ण॒वाः । गौ॒रल॑लामा॒ इति॑ गौ॒र - ल॒ला॒माः॒ । तू॒प॒राः ॥ \textbf{  47} \newline
                  \newline
                      (शु॒ण्ठा - विꣳ॑श॒तिः)  \textbf{(A16)} \newline \newline
                                \textbf{ TS 5.6.17.1} \newline
                  इन्द्रा॑य । राज्ञे᳚ । त्रयः॑ । शि॒ति॒पृ॒ष्ठा इति॑ शिति - पृ॒ष्ठाः । इन्द्रा॑य । अ॒धि॒रा॒जायेत्य॑धि - रा॒जाय॑ । त्रयः॑ । शिति॑ककुद॒ इति॒ शिति॑ - क॒कु॒दः॒ । इन्द्रा॑य । स्व॒राज्ञ्॒ इति॑ स्व - राज्ञे᳚ । त्रयः॑ । शिति॑भसद॒ इति॒ शिति॑ - भ॒स॒दः॒ । ति॒स्रः । तु॒र्यौ॒ह्यः॑ । सा॒द्ध्याना᳚म् । ति॒स्रः । प॒ष्ठौ॒ह्यः॑ । विश्वे॑षाम् । दे॒वाना᳚म् । आ॒ग्ने॒न्द्राः । कृ॒ष्णल॑लामा॒ इति॑ कृ॒ष्ण - ल॒ला॒माः । तू॒प॒राः ॥ \textbf{  48} \newline
                  \newline
                      (इन्द्रा॑य॒ राज्ञे॒ - द्वाविꣳ॑शतिः)  \textbf{(A17)} \newline \newline
                                \textbf{ TS 5.6.18.1} \newline
                  अदि॑त्यै । त्रयः॑ । रो॒हि॒तै॒ता इति॑ रोहित - ए॒ताः । इ॒न्द्रा॒ण्यै । त्रयः॑ । कृ॒ष्णै॒ता इति॑ कृष्ण - ए॒ताः । कु॒ह्वै᳚ । त्रयः॑ । अ॒रु॒णै॒ता इत्य॑रुण - ए॒ताः । ति॒स्रः । धे॒नवः॑ । रा॒कायै᳚ । त्रयः॑ । अ॒न॒ड्वाहः॑ । सि॒नी॒वा॒ल्यै । आ॒ग्ना॒वै॒ष्ण॒वा इत्या᳚ग्ना - वै॒ष्ण॒वाः । रोहि॑तललामा॒ इति॒ रोहि॑त - ल॒ला॒माः॒ । तू॒प॒राः ॥ \textbf{  49 } \newline
                  \newline
                      (अदि॑त्या-अ॒ष्टाद॑श)  \textbf{(A18)} \newline \newline
                                \textbf{ TS 5.6.19.1} \newline
                  सौ॒म्याः । त्रयः॑ । पि॒शङ्गाः᳚ । सोमा॑य । राज्ञे᳚ । त्रयः॑ । सा॒रङ्गाः᳚ । पा॒र्ज॒न्याः । नभो॑रूपा॒ इति॒ नभः॑-रू॒पाः॒ । ति॒स्रः । अ॒जाः । म॒ल॒.हाः । इ॒न्द्रा॒ण्यै । ति॒स्रः । मे॒ष्यः॑ । आ॒दि॒त्याः । द्या॒वा॒पृ॒थि॒व्या॑ इति॑ द्यावा - पृ॒थि॒व्याः᳚ । मा॒लङ्गाः᳚ । तू॒प॒राः ॥ \textbf{  50} \newline
                  \newline
                      (सौ॒म्या - एका॒न्नविꣳ॑श॒तिः)  \textbf{(A19)} \newline \newline
                                \textbf{ TS 5.6.20.1} \newline
                  वा॒रु॒णाः । त्रयः॑ । कृ॒ष्णल॑लामा॒ इति॑ कृ॒ष्ण - ल॒ला॒माः॒ । वरु॑णाय । राज्ञे᳚ । त्रयः॑ । रोहि॑तललामा॒ इति॒ रोहि॑त - ल॒ला॒माः॒ । वरु॑णाय । रि॒शाद॑स॒ इति॑ रिश - अद॑से । त्रयः॑ । अ॒रु॒णल॑लामा॒ इत्य॑रु॒ण - ल॒ला॒माः॒ । शि॒ल्पाः । त्रयः॑ । वै॒श्व॒दे॒वा इति॑ वैश्व - दे॒वाः । त्रयः॑ । पृश्न॑यः । स॒र्व॒दे॒व॒त्या॑ इति॑ सर्व - दे॒व॒त्याः᳚ । ऐ॒न्द्रा॒सू॒रा इत्यै᳚न्द्रा - सू॒राः । श्येत॑ललामा॒ इति॒ श्येत॑ - ल॒ला॒माः॒ । तू॒प॒राः ॥ \textbf{  51} \newline
                  \newline
                      (वा॒रु॒णा - विꣳ॑श॒तिः)  \textbf{(A20)} \newline \newline
                                \textbf{ TS 5.6.21.1} \newline
                  सोमा॑य । स्व॒राज्ञ्॒ इति॑ स्व - राज्ञे᳚ । अ॒नो॒वा॒हावित्य॑नः - वा॒हौ । अ॒न॒ड्वाहौ᳚ । इ॒न्द्रा॒ग्निभ्या॒मिती᳚न्द्रा॒ग्नि - भ्या॒म् । ओ॒जो॒दाभ्या॒मित्यो॑जः-दाभ्या᳚म् । उष्टा॑रौ । इ॒न्द्रा॒ग्निभ्या॒मिती᳚न्द्रा॒ग्नि-भ्या॒म् । ब॒ल॒दाभ्या॒मिति॑ बल - दाभ्या᳚म् । सी॒र॒वा॒हाविति॑ सीर - वा॒हौ । अवी॒ इति॑ । द्वे इति॑ । धे॒नू इति॑ । भौ॒मी इति॑ । दि॒ग्भ्य इति॑ दिक् - भ्यः । वड॑बे॒ इति॑ । द्वे इति॑ । धे॒नू इति॑ । भौ॒मी इति॑ । वै॒रा॒जी इति॑ । पु॒रु॒षी इति॑ । द्वे इति॑ । धे॒नू इति॑ । भौ॒मी इति॑ । वा॒यवे᳚ । आ॒रो॒ह॒ण॒वा॒हावित्या॑रोहण - वा॒हौ । अ॒न॒ड्वाहौ᳚ । वा॒रु॒णी इति॑ । कृ॒ष्णे इति॑ । व॒शे इति॑ । अ॒रा॒ड्यौ᳚ । दि॒व्यौ । ऋ॒ष॒भौ । प॒रि॒म॒राविति॑ परि - म॒रौ ॥ \textbf{  52 } \newline
                  \newline
                      (सोमा॑य स्व॒राज्ञे॒ - चतु॑स्त्रिꣳशत्)  \textbf{(A21)} \newline \newline
                                \textbf{ TS 5.6.22.1} \newline
                  एका॑दश । प्रा॒तः । ग॒व्याः । प॒शवः॑ । एति॑ । ल॒भ्य॒न्ते॒ । छ॒ग॒लः । क॒ल्माषः॑ । कि॒कि॒दी॒विः । वि॒दी॒गयः॑ । ते । त्वा॒ष्ट्राः । सौ॒रीः । नव॑ । श्वे॒ताः । व॒शाः । अ॒नू॒ब॒न्ध्या॑ इत्य॑नु - ब॒न्ध्याः᳚ । भ॒व॒न्ति॒ । आ॒ग्ने॒यः । ऐ॒न्द्रा॒ग्न इत्यै᳚न्द्र - अ॒ग्नः । आ॒श्वि॒नः । ते । वि॒शा॒ल॒यू॒प इति॑ विशाल - यू॒पे । एति॑ । ल॒भ्य॒न्ते॒ ॥ \textbf{  53} \newline
                  \newline
                      (ऐका॑दश प्रा॒तः - पञ्च॑विꣳशतिः)  \textbf{(A22)} \newline \newline
                                \textbf{ TS 5.6.23.1} \newline
                  पि॒शङ्गाः᳚ । त्रयः॑ । वा॒स॒न्ताः । सा॒रङ्गाः᳚ । त्रयः॑ । ग्रैष्माः᳚ । पृष॑न्तः । त्रयः॑ । वार्.षि॑काः । पृश्न॑यः । त्रयः॑ । शा॒र॒दाः । पृ॒श्नि॒स॒क्था इति॑ पृश्नि - स॒क्थाः । त्रयः॑ । हैम॑न्तिकाः । अ॒व॒लि॒प्ता इत्य॑व-लि॒प्ताः । त्रयः॑ । शै॒शि॒राः । सं॒ॅव॒थ्स॒रायेति॑ सं - व॒थ्स॒राय॑ । निव॑क्षस॒ इति॒ नि - व॒क्ष॒सः॒ ॥रोहि॑तः कृ॒ष्णा धू॒म्रल॑लामाः॒ - पृश्निः॑ श्या॒मा अ॑रु॒णल॑लामाः -शितिबा॒हुः शि॒ल्पाः श्येनीः᳚ श्या॒मल॑लामा - उन्न॒तः सि॒द्ध्मा धा॒त्रे पौ॒ष्णाः श्येत॑ललामाः - क॒र्णा ब॒भ्रुल॑लामाः - शु॒ण्ठा गौ॒रल॑लामा॒ - इन्द्रा॑य कृ॒ष्णाल॑लामा॒ - अदि॑त्यै॒ रोहि॑त ललामः -सौ॒म्या मा॒लङ्गा॑ - वारु॒णाः सू॒राः श्येत॑ललामा॒ - दश॑ । \textbf{  20 } \newline
                  \newline
                      (पि॒शङ्गा॑ - विꣳश॒तिः)  \textbf{(A23)} \newline \newline
\textbf{praSna korvai with starting padams of 1 to 23 anuvAkams :-} \newline
(हिर॑ण्यवर्णा - अ॒पां ग्रहा᳚न् - भूतेष्ट॒काः - स॒जूः - सं॑ॅवथ्स॒रं - प्र॒जाप॑तिः॒ स क्षु॒रप॑वि - र॒वग्नेर्वै दी॒क्षया॑ - सुव॒र्गाय॒ तं ॅयन्न - सू॒यते᳚ - प्र॒जाप॑तिर् ऋ॒तुभी॒ - रोहि॑तः॒ - पृश्निः॑ - शितिबा॒हु - रु॑न्न॒तः - क॒र्णाः - शु॒ण्ठा - इन्द्रा॒या- दि॑त्यै - सौ॒म्या - वा॑रु॒णाः - सोमा॒यै - का॑दश - पि॒शङ्गा॒ - स्त्रयो॑विꣳशतिः) \newline

\textbf{korvai with starting padams of1, 11, 21 series of pa~jcAtis :-} \newline
(हिर॑ण्यवर्णा - भूतेष्ट॒काः - छन्दो॒ यत् - कनी॑याꣳसन्-त्रि॒वृद्ध्य॑ग्नि - र्वा॑रु॒णा - श्चतु॑ष्पञ्चा॒शत् ) \newline

\textbf{first and last padam of sixth praSnam of 5th kANDam} \newline
(हिर॑ण्यवर्णा॒ - निव॑क्षसः) \newline 


॥ हरिः॑ ॐ ॥
॥ कृष्ण यजुर्वेदीय तैत्तिरीय संहितायां पञ्चमकाण्डे षष्ठः प्रश्नः समाप्तः ॥ \newline
\pagebreak
\pagebreak
        


\end{document}
