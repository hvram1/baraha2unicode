\documentclass[17pt]{extarticle}
\usepackage{babel}
\usepackage{fontspec}
\usepackage{polyglossia}
\usepackage{extsizes}



\setmainlanguage{sanskrit}
\setotherlanguages{english} %% or other languages
\setlength{\parindent}{0pt}
\pagestyle{myheadings}
\newfontfamily\devanagarifont[Script=Devanagari]{AdishilaVedic}


\newcommand{\VAR}[1]{}
\newcommand{\BLOCK}[1]{}




\begin{document}
\begin{titlepage}
    \begin{center}
 
\begin{sanskrit}
    { \Large
    ॐ नमः परमात्मने, श्री महागणपतये नमः, 
श्री गुरुभ्यो नमः । ह॒रिः॒ ॐ ॥ 
    }
    \\
    \vspace{2.5cm}
    \mbox{ \Huge
    1.3     प्रथमकाण्डे तृतीयः प्रश्नः -(अग्निष्टोमे पशुः)   }
\end{sanskrit}
\end{center}

\end{titlepage}
\tableofcontents

ॐ नमः परमात्मने, श्री महागणपतये नमः, 
श्री गुरुभ्यो नमः । ह॒रिः॒ ॐ ॥ \newline
1.3     प्रथमकाण्डे तृतीयः प्रश्नः -(अग्निष्टोमे पशुः) \newline

\addcontentsline{toc}{section}{ 1.3     प्रथमकाण्डे तृतीयः प्रश्नः -(अग्निष्टोमे पशुः)}
\markright{ 1.3     प्रथमकाण्डे तृतीयः प्रश्नः -(अग्निष्टोमे पशुः) \hfill https://www.vedavms.in \hfill}
\section*{ 1.3     प्रथमकाण्डे तृतीयः प्रश्नः -(अग्निष्टोमे पशुः) }
                                \textbf{ TS 1.3.1.1} \newline
                  दे॒वस्य॑ । त्वा॒ । स॒वि॒तुः । प्र॒स॒व इति॑ प्र - स॒वे । अ॒श्विनोः᳚ । बा॒हुभ्या॒मिति॑ बा॒हु - भ्या॒म् । पू॒ष्णः । हस्ता᳚भ्याम् । एति॑ । द॒दे॒ । अभ्रिः॑ । अ॒सि॒ । नारिः॑ । अ॒सि॒ । परि॑लिखित॒मिति॒ परि॑ - लि॒खि॒त॒म् । रक्षः॑ । परि॑लिखिता॒ इति॒ परि॑ - लि॒खि॒ताः॒ । अरा॑तयः । इ॒दम् । अ॒हम् । रक्ष॑सः । ग्री॒वाः । अपीति॑ । कृ॒न्ता॒मि॒ । यः । अ॒स्मान् । द्वेष्टि॑ । यम् । च॒ । व॒यम् । द्वि॒ष्मः । इ॒दम् । अ॒स्य॒ । ग्री॒वाः । अपीति॑ । कृ॒न्ता॒मि॒ । दि॒वे । त्वा॒ । अ॒न्तरि॑क्षाय । त्वा॒ । पृ॒थि॒व्यै । त्वा॒ । शुन्ध॑ताम् । लो॒कः । पि॒तृ॒षद॑न॒ इति॑ पितृ - सद॑नः । यवः॑ । अ॒सि॒ । य॒वय॑ । अ॒स्मत् । द्वेषः॑ । \textbf{  1} \newline
                  \newline
                                \textbf{ TS 1.3.1.2} \newline
                  य॒वय॑ । अरा॑तीः । पि॒तृ॒णाम् । सद॑नम् । अ॒सि॒ । उदिति॑ । दिव᳚म् । स्त॒भा॒न॒ । एति॑ । अ॒न्तरि॑क्षम् । पृ॒ण॒ । पृ॒थि॒वीम् । दृꣳ॒॒ह॒ । द्यु॒ता॒नः । त्वा॒ । मा॒रु॒तः । मि॒नो॒तु॒ । मि॒॒त्रावरु॑णयो॒रिति॑ मि॒त्रा - वरु॑णयोः । ध्रु॒वेण॑ । धर्म॑णा । ब्र॒ह्म॒वनि॒मिति॑ ब्रह्म - वनि᳚म् । त्वा॒ । क्ष॒त्र॒वनि॒मिति॑ क्षत्र - वनि᳚म् । सु॒प्र॒जा॒वनि॒मिति॑ सुप्रजा - वनि᳚म् । रा॒य॒स्पो॒ष॒वनि॒मिति॑ रायस्योष - वनि᳚म् । परीति॑ । ऊ॒हा॒मि॒ । ब्रह्म॑ । दृꣳ॒॒ह॒ । क्ष॒त्रम् । दृꣳ॒॒ह॒ । प्र॒जामिति॑ प्र - जाम् । दृꣳ॒॒ह॒ । रा॒यः । पोष᳚म् । दृꣳ॒॒ह॒ । घृ॒तेन॑ । द्या॒वा॒ पृ॒थि॒वी॒ इति॑ द्यावा - पृ॒थि॒वी॒ । एति॑ । पृ॒णे॒था॒म् । इन्द्र॑स्य । सदः॑ । अ॒सि॒ । वि॒श्व॒ज॒नस्येति॑ विश्व - ज॒नस्य॑ । छा॒या । परीति॑ । त्वा॒ । गि॒र्व॒णः॒ । गिरः॑ । इ॒माः () । भ॒व॒न्तु॒ । वि॒श्वतः॑ । वृ॒द्धायु॒मिति॑ वृ॒द्ध - आ॒यु॒म् । अन्विति॑ । वृद्ध॑यः । जुष्टाः᳚ । भ॒व॒न्तु॒ । जुष्ट॑यः । इन्द्र॑स्य । स्यूः । अ॒सि॒ । इन्द्र॑स्य । ध्रु॒वम् । अ॒सि॒ । ऐ॒न्द्रम् । अ॒सि॒ । इन्द्रा॑य । त्वा॒ ॥ \textbf{  2 } \newline
                  \newline
                      (द्वेष॑ - इ॒मा - अ॒ष्टाद॑श च )  \textbf{(A1)} \newline \newline
                                \textbf{ TS 1.3.2.1} \newline
                  र॒क्षो॒हण॒ इति॑ रक्षः - हनः॑ । व॒ल॒ग॒हन॒ इति॑ वलग - हनः॑ । वै॒ष्ण॒वान् । ख॒ना॒मि॒ । इ॒दम् । अ॒हम् । तम् । व॒ल॒गमिति॑ वल - गम् । उदिति॑ । व॒पा॒मि॒ । यम् । नः॒ । स॒मा॒नः । यम् । अस॑मानः । नि॒च॒खानेति॑ नि - च॒खान॑ । इ॒दम् । ए॒न॒म् । अद्ध॑रम् । क॒रो॒मि॒ । यः । नः॒ । स॒मा॒नः । यः । अस॑मानः । अ॒रा॒ती॒यति॑ । गा॒य॒त्रेण॑ । छन्द॑सा । अव॑बाढ॒ इत्यव॑ - बा॒ढः॒ । व॒ल॒ग इति॑ वल - गः । किम् । अत्र॑ । भ॒द्रम् । तत् । नौ॒ । स॒ह । वि॒राडिति॑ वि - राट् । अ॒सि॒ । स॒प॒त्न॒हेति॑ सपत्न - हा । स॒म्राडिति॑ सम् - राट् । अ॒सि॒ । भ्रा॒तृ॒व्य॒हेति॑ भ्रातृव्य - हा । स्व॒राडिति॑ स्व - राट् । अ॒सि॒ । अ॒भि॒मा॒ति॒हेत्य॑भिमाति - हा । वि॒श्वा॒राडिति॑ विश्व - राट् । अ॒सि॒ । विश्वा॑साम् । ना॒ष्ट्राणा᳚म् । ह॒न्ता । \textbf{  3 } \newline
                  \newline
                                \textbf{ TS 1.3.2.2} \newline
                  र॒क्षो॒हण॒ इति॑ रक्षः - हनः॑ । व॒ल॒ग॒हन॒ इति॑ वलग - हनः॑ । प्रेति॑ । उ॒क्षा॒मि॒ । वै॒ष्ण॒वान् । र॒क्षो॒हण॒ इति॑ रक्षः - हनः॑ । व॒ल॒ग॒हन॒ इति॑ वलग - हनः॑ । अवेति॑ । न॒या॒मि॒ । वै॒ष्ण॒वान् । यवः॑ । अ॒सि॒ । य॒वय॑ । अ॒स्मत् । द्वेषः॑ । य॒वय॑ । अरा॑तीः । र॒क्षो॒हण॒ इति॑ रक्षः - हनः॑ । व॒ल॒ग॒हन॒ इति॑ वलग - हनः॑ । अवेति॑ । स्तृ॒णा॒मि॒ । वै॒ष्ण॒वान् । र॒क्षो॒हण॒ इति॑ रक्षः - हनः॑ । व॒ल॒ग॒हन॒ इति॑ वलग - हनः॑ । अ॒भीति॑ । जु॒हो॒मि॒ । वै॒ष्ण॒वान् । र॒क्षो॒हणा॒विति॑ रक्षः - हनौ᳚ । व॒ल॒ग॒हना॒विति॑ वलग - हनौ᳚ । उपेति॑ । द॒धा॒मि॒ । वै॒ष्ण॒वी इति॑ । र॒क्षो॒हणा॒विति॑ रक्षः - हनौ᳚ । व॒ल॒ग॒हना॒विति॑ वलग - हनौ᳚ । परीति॑ । ऊ॒हा॒मि॒ । वै॒ष्ण॒वी इति॑ । र॒क्षो॒हणा॒विति॑ रक्षः - हनौ᳚ । व॒ल॒ग॒हना॒विति॑ वलग - हनौ᳚ । परीति॑ । स्तृ॒णा॒मि॒ । वै॒ष्ण॒वी इति॑ । र॒क्षो॒हणा॒विति॑ रक्षः - हनौ᳚ । व॒ल॒ग॒हना॒विति॑ वलग - हनौ᳚ । वै॒ष्ण॒वी इति॑ । बृ॒हन्न् । अ॒सि॒ । बृ॒हद्ग्रा॒वेति॑ बृ॒हत् - ग्रा॒वा॒ । बृ॒ह॒तीम् । इन्द्रा॑य ( ) । वाच᳚म् । व॒द॒ ॥ \textbf{  4 } \newline
                  \newline
                      ( ह॒न्ते-न्द्रा॑य॒ द्वे च॑ )  \textbf{(A2)} \newline \newline
                                \textbf{ TS 1.3.3.1} \newline
                  वि॒भूरिति॑ वि - भूः । अ॒सि॒ । प्र॒वाह॑ण॒ इति॑ प्र - वाह॑नः । वह्निः॑ । अ॒सि॒ । ह॒व्य॒वाह॑न॒ इति॑ हव्य - वाह॑नः । श्वा॒त्रः । अ॒सि॒ । प्रचे॑ता॒ इति॒ प्र - चे॒ताः॒ । तु॒थः । अ॒सि॒ । वि॒श्ववे॑दा॒ इति॑ वि॒श्व - वे॒दाः॒ । उ॒शिक् । अ॒सि॒ । क॒विः । अङ्घा॑रिः । अ॒सि॒ । बंभा॑रिः । अ॒व॒स्युः । अ॒सि॒ । दुव॑स्वान् । शु॒न्ध्यूः । अ॒सि॒ । मा॒र्जा॒लीयः॑ । स॒म्राडिति॑ सं - राट् । अ॒सि॒ । कृ॒शानु॒रिति॑ कृ॒श - अ॒नुः॒ । प॒रि॒षद्य॒ इति॑ परि - सद्यः॑ । अ॒सि॒ । पव॑मानः । प्र॒तक्वेति॑ प्र - तक्वा᳚ । अ॒सि॒ । नभ॑स्वान् । अस॑मृंष्ट॒ इत्यसं᳚ - मृ॒ष्टः॒ । अ॒सि॒ । ह॒व्य॒सूद॒ इति॑ हव्य - सूदः॑ । ऋ॒तधा॒मेत्यृ॒त - धा॒मा॒ । अ॒सि॒ । सुव॑र्ज्योति॒रिति॒ सुवः॑ - ज्यो॒तिः॒ । ब्रह्म॑ज्योति॒रिति॒ ब्रह्म॑ - ज्यो॒तिः॒ । अ॒सि॒ । सुव॑र्धा॒मेति॒ सुवः॑ - धा॒मा॒ । अ॒जः । अ॒सि॒ । एक॑पा॒दित्येक॑ - पा॒त् । अहिः॑ । अ॒सि॒ । बु॒द्ध्नियः॑ । रौद्रे॑ण । अनी॑केन ( ) । पा॒हि । मा॒ । अ॒ग्ने॒ । पि॒पृ॒हि । मा॒ । मा । मा॒ । हिꣳ॒॒सीः॒ ॥ \textbf{  5 } \newline
                  \newline
                      (अनी॑केना॒-ष्टौ च॑)  \textbf{(A3)} \newline \newline
                                \textbf{ TS 1.3.4.1} \newline
                  त्वम् । सो॒म॒ । त॒नू॒कृद्भ्य॒ इति॑ तनू॒कृत् - भ्यः॒ । द्वेषो᳚भ्य॒ इति॒ द्वेषः॑ - भ्यः॒ । अ॒न्यकृ॑तेभ्य॒ इत्य॒न्य - कृ॒ते॒भ्यः॒ । उ॒रु । य॒न्ता । अ॒सि॒ । वरू॑थम् । स्वाहा᳚ । जु॒षा॒णः । अ॒प्तुः । आज्य॑स्य । वे॒तु॒ । स्वाहा᳚ । अ॒यं । नः॒ । अ॒ग्निः । वरि॑वः । कृ॒णो॒तु॒ । अ॒यम् । मृधः॑ । पु॒रः । ए॒तु॒ । प्र॒भि॒न्दन्निति॑ प्र - भि॒न्दन्न् ॥ अ॒यम् । शत्रून्॑ । ज॒य॒तु॒ । जर्.हृ॑षाणः । अ॒यम् । वाज᳚म् । ज॒य॒तु॒ । वाज॑साता॒विति॒ वाज॑ - सा॒तौ॒ ॥ उ॒रु । वि॒ष्णो॒ इति॑ । वीति॑ । क्र॒म॒स्व॒ । उ॒रु । क्षया॑य । नः॒ । कृ॒धि॒ ॥ घृ॒तम् । घृ॒त॒यो॒न॒ इति॑ घृत - यो॒ने॒ । पि॒ब॒ । प्रप्रेति॒ प्र - प्र॒ । य॒ज्ञ्प॑ति॒मिति॑ य॒ज्ञ् - प॒ति॒म् । ति॒र॒ ॥ सोमः॑ । जि॒गा॒ति॒ । गा॒तु॒विदिति॑ गातु - वित् । \textbf{  6} \newline
                  \newline
                                \textbf{ TS 1.3.4.2} \newline
                  दे॒वाना᳚म् । ए॒ति॒ । नि॒ष्कृ॒तमिति॑ निः - कृ॒तम् । ऋ॒तस्य॑ । योनि᳚म् । आ॒सद॒मित्या᳚ - सद᳚म् । अदि॑त्याः । सदः॑ । अ॒सि॒ । अदि॑त्याः । सदः॑ । एति॑ । सी॒द॒ । ए॒षः । वः॒ । दे॒व॒ । स॒वि॒तः॒ । सोमः॑ । तम् । र॒क्ष॒द्ध्व॒म् । मा । वः॒ । द॒भ॒॒त् । ए॒तत् । त्वम् । सो॒म॒ । दे॒वः । दे॒वान् । उपेति॑ । अ॒गाः॒ । इ॒दम् । अ॒हम् । म॒नु॒ष्यः॑ । म॒नु॒ष्यान्॑ । स॒ह । प्र॒जयेति॑ प्र - जया᳚ । स॒ह । रा॒यः । पोषे॑ण । नमः॑ । दे॒वेभ्यः॑ । स्व॒धेति॑ स्व - धा । पि॒तृभ्य॒ इति॑ पि॒तृ - भ्यः॒ । इ॒दम् । अ॒हम् । निरिति॑ । वरु॑णस्य । पाशा᳚त् । सुवः॑ । अ॒भि । \textbf{  7} \newline
                  \newline
                                \textbf{ TS 1.3.4.3} \newline
                  वीति॑ । ख्ये॒ष॒म् । वै॒श्वा॒न॒रम् । ज्योतिः॑ । अग्ने᳚ । व्र॒त॒प॒त॒ इति॑ व्रत - प॒ते॒ । त्वम् । व्र॒ताना᳚म् । व्र॒तप॑ति॒रिति॑ व्र॒त - प॒तिः॒ । अ॒सि॒ । या । मम॑ । त॒नूः । त्वयि॑ । अभू᳚त् । इ॒यम् । सा । मयि॑ । या । तव॑ । त॒नूः । मयि॑ । अभू᳚त् । ए॒षा । सा । त्वयि॑ । य॒था॒य॒थमिति॑ यथा - य॒थम् । नौ॒ । व्र॒त॒प॒त॒ इति॑ व्रत - प॒ते॒ । व्र॒तिनोः᳚ । व्र॒तानि॑ ॥ \textbf{  8} \newline
                  \newline
                      (गा॒तु॒विद॒-भ्ये-क॑त्रिꣳशच्च)  \textbf{(A4)} \newline \newline
                                \textbf{ TS 1.3.5.1} \newline
                  अतीति॑ । अ॒न्यान् । अगा᳚म् । न । अ॒न्यान् । उपेति॑ । अ॒गा॒म् । अ॒र्वाक् । त्वा॒ । परैः᳚ । अ॒वि॒द॒म् । प॒रः । अव॑रैः । तम् । त्वा॒ । जु॒षे॒ । वै॒ष्ण॒वम् । दे॒व॒य॒ज्याया॒ इति॑ देव - य॒ज्यायै᳚ । दे॒वः । त्वा॒ । स॒वि॒ता । मद्ध्वा᳚ । अ॒न॒क्तु॒ । ओष॑धे । त्राय॑स्व । ए॒न॒म् । स्वधि॑त॒ इति॒ स्व - धि॒ते॒ । मा । ए॒न॒म् । हिꣳ॒॒सीः॒ । दिव᳚म् । अग्रे॑ण । मा । ल॒खीः॒ । अ॒न्तरि॑क्षम् । मद्ध्ये॑न । मा । हिꣳ॒॒सीः॒ । पृ॒थि॒व्या । समिति॑ । भ॒व॒ । वन॑स्पते । श॒तव॑ल्.श॒ इति॑ श॒त - व॒ल्॒.शः॒ । वीति॑ । रो॒ह॒ । स॒हस्र॑वल्.शा॒ इति॑ स॒हस्र॑ - व॒ल्॒.शाः॒ । वीति॑ । व॒यम् । रु॒हे॒म॒ । यम् ( ) । त्वा॒ । अ॒यम् । स्वधि॑ति॒रिति॒ स्व - धि॒तिः॒ । तेति॑जानः । प्र॒णि॒नायेति॑ प्र - नि॒नाय॑ । म॒ह॒ते । सौभ॑गाय । अच्छि॑न्नः । रायः॑ । सु॒वीर॒ इति॑ सु - वीरः॑ ॥ \textbf{  9 } \newline
                  \newline
                      (यं-दश॑ च)  \textbf{(A5)} \newline \newline
                                \textbf{ TS 1.3.6.1} \newline
                  पृ॒थि॒व्यै । त्वा॒ । अ॒न्तरि॑क्षाय । त्वा॒ । दि॒वे । त्वा॒ । शुन्ध॑ताम् । लो॒कः । पि॒तृ॒षद॑न॒ इति॑ पितृ - सद॑नः । यवः॑ । अ॒सि॒ । य॒वय॑ । अ॒स्मत् । द्वेषः॑ । य॒वय॑ । अरा॑तीः । पि॒तृ॒णाम् । सद॑नम् । अ॒सि॒ । स्वा॒वे॒श इति॑ सु - आ॒वे॒शः । अ॒सि॒ । अ॒ग्रे॒गा इत्य॑ग्रे - गाः । ने॒तृ॒णाम् । वन॒स्पतिः॑ । अधीति॑ । त्वा॒ । स्था॒स्य॒ति॒ । तस्य॑ । वि॒त्ता॒त् । दे॒वः । त्वा॒ । स॒वि॒ता । मद्ध्वा᳚ । अ॒न॒क्तु॒ । सु॒पि॒प्प॒लाभ्य॒ इति॑ सु - पि॒प्प॒लाभ्यः॑ । त्वा॒ । ओष॑धीभ्य॒ इत्योष॑ध - भ्यः॒ । उदिति॑ । दिव᳚म् । स्त॒भा॒न॒ । एति॑ । अ॒न्तरि॑क्षम् । पृ॒ण॒ । पृ॒थि॒वीम् । उप॑रेण । दृꣳ॒॒ह॒ । ते । ते॒ । धामा॑नि । उ॒श्म॒सि॒ । \textbf{  10} \newline
                  \newline
                                \textbf{ TS 1.3.6.2} \newline
                  ग॒मद्ध्ये᳚ । गावः॑ । यत्र॑ । भूरि॑शृङ्गा॒ इति॒ भूरि॑ - शृ॒ङ्गाः॒ । अ॒यासः॑ ॥ अत्र॑ । अह॑ । तत् । उ॒रु॒गा॒यस्येत्यु॑रु - गा॒यस्य॑ । विष्णोः᳚ । प॒र॒मम् । प॒दम् । अवेति॑ । भा॒ति॒ । भूरेः᳚ ॥ विष्णोः᳚ । कर्मा॑णि । प॒श्य॒त॒ । यतः॑ । व्र॒तानि॑ । प॒स्प॒शे ॥ इन्द्र॑स्य । युज्यः॑ । सखा᳚ ॥ तत् । विष्णोः᳚ । प॒र॒मम् । प॒दम् । सदा᳚ । प॒श्य॒न्ति॒ । सू॒रयः॑ ॥ दि॒वि । इ॒व॒ । चक्षुः॑ । आत॑त॒मित्या - त॒त॒म् ॥ ब्र॒ह्म॒वनि॒मिति॑ ब्रह्म - वनि᳚म् । त्वा॒ । क्ष॒त्र॒वनि॒मिति॑ क्षत्र - वनि᳚म् । सु॒प्र॒जा॒वनि॒मिति॑ सुप्रजा - वनि᳚म् । रा॒य॒स्पो॒ष॒वनि॒मिति॑ रायस्पोष - वनि᳚म् । परीति॑ । ऊ॒हा॒मि॒ । ब्रह्म॑ । दृꣳ॒॒ह॒ । क्ष॒त्रम् । दृꣳ॒॒ह॒ । प्र॒जामिति॑ प्र - जाम् । दृꣳ॒॒ह॒ । रा॒यः । पोष᳚म् ( ) । दृꣳ॒॒ह॒ । प॒रि॒वीरिति॑ परि - वीः । अ॒सि॒ । परीति॑ । त्वा॒ । दैवीः᳚ । विशः॑ । व्य॒य॒न्ता॒म् । परीति॑ । इ॒मम् । रा॒यः । पोषः॑ । यज॑मानम् । म॒नु॒ष्याः᳚ । अ॒न्तरि॑क्षस्य । त्वा॒ । सानौ᳚ । अवेति॑ । गू॒हा॒मि॒ ॥ \textbf{  11 } \newline
                  \newline
                      (उ॒श्म॒सी॒-पोष॒मे-का॒न्नविꣳ॑श॒तिश्च॑)  \textbf{(A6)} \newline \newline
                                \textbf{ TS 1.3.7.1} \newline
                  इ॒षे । त्वा॒ । उ॒प॒वीरित्यु॑प - वीः । अ॒सि॒ । उपो॒ इति॑ । दे॒वान् । दैवीः᳚ । विशः॑ । प्रेति॑ । अ॒गुः॒ । वह्नीः᳚ । उ॒शिजः॑ । बृह॑स्पते । धा॒रय॑ । वसू॑नि । ह॒व्या । ते॒ । स्व॒द॒न्ता॒म् । देव॑ । त्व॒ष्टः॒ । वसु॑ । र॒ण्व॒ । रेव॑तीः । रम॑द्ध्वम् । अ॒ग्नेः । ज॒नित्र᳚म् । अ॒सि॒ । वृष॑णौ । स्थः॒ । उ॒र्वशी᳚ । अ॒सि॒ । आ॒युः । अ॒सि॒ । पु॒रू॒रवाः᳚ । घृ॒तेन॑ । अ॒क्ते इति॑ । वृष॑णम् । द॒धा॒था॒म् । गा॒य॒त्रम् । छन्दः॑ । अनु॑ । प्रेति॑ । जा॒य॒स्व॒ । त्रैष्टु॑भम् । जाग॑तम् । छन्दः॑ । अनु॑ । प्रेति॑ । जा॒य॒स्व॒ । भव॑तम् । \textbf{  12} \newline
                  \newline
                                \textbf{ TS 1.3.7.2} \newline
                  नः॒ । सम॑नसा॒विति॒ स - म॒न॒सौ॒ । समो॑कसा॒विति॒ सं - ओ॒क॒सौ॒ । अ॒रे॒पसौ᳚ ॥ मा । य॒ज्ञ्म् । हिꣳ॒॒सि॒ष्ट॒म् । मा । य॒ज्ञ्प॑ति॒मिति॑ य॒ज्ञ् - प॒ति॒म् । जा॒त॒वे॒द॒सा॒विति॑ जात - वे॒द॒सौ॒ । शि॒वौ । भ॒व॒त॒म् । अ॒द्य । नः॒ ॥ अ॒ग्नौ । अ॒ग्निः । च॒र॒ति॒ । प्रवि॑ष्ट॒ इति॒ प्र - वि॒ष्टः॒ । ऋषी॑णाम् । पु॒त्रः । अ॒धि॒रा॒ज इत्य॑धि - रा॒जः । ए॒षः ॥ स्वा॒हा॒कृत्येति॑ स्वाहा - कृत्य॑ । ब्रह्म॑णा । ते॒ । जु॒हो॒मि॒ । मा । दे॒वाना᳚म् । मि॒थु॒या । कः॒ । भा॒ग॒धेय॒मिति॑ भाग - धेय᳚म् ॥ \textbf{  13 } \newline
                  \newline
                      (भव॑त॒-मेक॑त्रिꣳशच्च)  \textbf{(A7)} \newline \newline
                                \textbf{ TS 1.3.8.1} \newline
                  एति॑ । द॒दे॒ । ऋ॒तस्य॑ । त्वा॒ । दे॒व॒ह॒वि॒रिति॑ देव - ह॒विः॒ । पाशे॑न । एति॑ । र॒भे॒ । धर्.ष॑ । मानु॑षान् । अ॒द्भ्य इत्य॑त् - भ्यः । त्वा॒ । ओष॑धीभ्य॒ इत्योष॑धि - भ्यः॒ । प्रेति॑ । उ॒क्षा॒मि॒ । अ॒पाम् । पे॒रुः । अ॒सि॒ । स्वा॒त्तम् । चि॒त् । सदे॑व॒मिति॒ स - दे॒व॒म् । ह॒व्यम् । आपः॑ । दे॒वीः॒ । स्वद॑त । ए॒न॒म् । समिति॑ । ते॒ । प्रा॒ण इति॑ प्र - अ॒नः । वा॒युना᳚ । ग॒च्छ॒ता॒म् । समिति॑ । यज॑त्रैः । अङ्गा॑नि । समिति॑ । य॒ज्ञ्प॑ति॒रिति॑ य॒ज्ञ् - प॒तिः॒ । आ॒शिषेत्या᳚ - शिषा᳚ । घृ॒तेन॑ । अ॒क्तौ । प॒शुम् । त्रा॒ये॒था॒म् । रेव॑तीः । य॒ज्ञ्प॑ति॒मिति॑ य॒ज्ञ् - प॒ति॒म् । प्रि॒य॒धेति॑ प्रिय - धा । एति॑ । वि॒श॒त॒ । उरो॒ इति॑ । अ॒न्त॒रि॒क्ष॒ । स॒जूरिति॑ स - जूः । दे॒वेन॑ । \textbf{  14} \newline
                  \newline
                                \textbf{ TS 1.3.8.2} \newline
                  वाते॑न । अ॒स्य । ह॒विषः॑ । त्मना᳚ । य॒ज॒ । समिति॑ । अ॒स्य॒ । त॒नुवा᳚ । भ॒व॒ । वर्.षी॑यः । वर्.षी॑यसि । य॒ज्ञे । य॒ज्ञ्प॑ति॒मिति॑ य॒ज्ञ् - प॒ति॒म् । धाः॒ । पृ॒थि॒व्याः । स॒म्पृच॒ इति॑ सम् - पृचः॑ । पा॒हि॒ । नमः॑ । ते॒ । आ॒ता॒नेत्या᳚ - ता॒न॒ । अ॒न॒र्वा । प्रेति॑ । इ॒हि॒ । घृ॒तस्य॑ । कु॒॒ल्याम् । अन्विति॑ । स॒ह । प्र॒जयेति॑ प्र - जया᳚ । स॒ह । रा॒यः । पोषे॑ण । आपः॑ । दे॒वीः॒ । शु॒द्धा॒यु॒व॒ इति॑ शुद्ध - यु॒वः॒ । शु॒द्धाः । यू॒यम् । दे॒वान् । ऊ॒ढ्व॒म् । शु॒द्धाः । व॒यम् । परि॑विष्टा॒ इति॒ परि॑ - वि॒ष्टाः॒ । प॒रि॒वे॒ष्टार॒ इति॑ परि - वे॒ष्टारः॑ । वः॒ । भू॒या॒स्म॒ ॥ \textbf{  15} \newline
                  \newline
                      (दे॒वन॒-चतु॑श्चत्वारिꣳशच्च)  \textbf{(A8)} \newline \newline
                                \textbf{ TS 1.3.9.1} \newline
                  वाक् । ते॒ । एति॑ । प्या॒य॒ता॒म् । प्रा॒ण इति॑ प्र - अ॒नः । ते॒ । एति॑ । प्या॒य॒ता॒म् । चक्षुः॑ । ते॒ । एति॑ । प्या॒य॒ता॒म् । श्रोत्र᳚म् । ते॒ । एति॑ । प्या॒य॒ता॒म् । या । ते॒ । प्रा॒णानिति॑ प्र - अ॒नान् । शुक् । ज॒गाम॑ । या । चक्षुः॑ । या । श्रोत्र᳚म् । यत् । ते॒ । क्रू॒रम् । यत् । आस्थि॑त॒मित्या - स्थि॒त॒म् । तत् । ते॒ । एति॑ । प्या॒य॒ता॒म् । तत् । ते॒ । ए॒तेन॑ । शु॒न्ध॒ता॒म् । नाभिः॑ । ते॒ । एति॑ । प्या॒य॒ता॒म् । पा॒युः । ते॒ । एति॑ । प्या॒य॒ता॒म् । शु॒द्धाः । च॒रित्राः᳚ । शम् । अ॒द्भ्य इत्य॑त् - भ्यः । \textbf{  16} \newline
                  \newline
                                \textbf{ TS 1.3.9.2} \newline
                  शम् । ओष॑धीभ्य॒ इत्योषा॑ध - भ्यः॒ । शम् । पृ॒थि॒व्यै । शम् । अहो᳚भ्या॒मित्यहः॑ - भ्या॒म् । ओष॑धे । त्राय॑स्व । ए॒न॒म् । स्वधि॑त॒ इति॒ स्व - धि॒ते॒ । मा । ए॒न॒म् । हिꣳ॒॒सीः॒ । रक्ष॑साम् । भा॒गः । अ॒सि॒ । इ॒दम् । अ॒हम् । रक्षः॑ । अ॒ध॒मम् । तमः॑ । न॒या॒मि॒ । यः । अ॒स्मान् । द्वेष्टि॑ । यम् । च॒ । व॒यम् । द्वि॒ष्मः । इ॒दम् । ए॒न॒म् । अ॒ध॒मम् । तमः॑ । न॒या॒मि॒ । इ॒षे । त्वा॒ । घृ॒तेन॑ । द्या॒वा॒पृ॒थि॒वी॒ इति॑ द्यावा - पृ॒थि॒वी॒ । प्रेति॑ । ऊ॒र्ण्वा॒था॒म् । अच्छि॑न्नः । रायः॑ । सु॒वीर॒ इति॑ सु - वीरः॑ । उ॒रु । अ॒न्तरि॑क्षम् । अन्विति॑ । इ॒हि॒ । वायो॒ इति॑ । वीति॑ । इ॒हि॒ ( ) । स्तो॒काना᳚म् । स्वाहा᳚ । ऊ॒र्द्ध्वन॑भस॒मित्यू॒र्द्ध्व - न॒भ॒स॒म् । मा॒रु॒तम् । ग॒च्छ॒त॒म् ॥ \textbf{  17} \newline
                  \newline
                      (अ॒द्भ्यो-वीहि॒-पञ्च॑ च)  \textbf{(A9)} \newline \newline
                                \textbf{ TS 1.3.10.1} \newline
                  समिति॑ । ते॒ । मन॑सा । मनः॑ । समिति॑ । प्रा॒णेनेति॑ प्र - अ॒नेन॑ । प्रा॒ण इति॑ प्र - अ॒नः । जुष्ट᳚म् । दे॒वेभ्यः॑ । ह॒व्यम् । घृ॒तव॒दिति॑ घृ॒त - व॒त् । स्वाहा᳚ । ऐ॒न्द्रः । प्रा॒ण इति॑ प्र - अ॒नः । अङ्गे॑अङ्ग॒ इत्यङ्गे᳚ - अ॒ङ्गे॒ । नीति॑ । दे॒द्ध्य॒त् । ऐ॒न्द्रः । अ॒पा॒न इत्य॑प - अ॒नः । अङ्गे॑अङ्ग॒ इत्यङ्गे᳚ - अ॒ङ्गे॒ । वीति॑ । बो॒भु॒व॒त् । देव॑ । त्व॒ष्टः॒ । भूरि॑ । ते॒ । सꣳस॒मिति॒ सं - स॒म् । ए॒तु॒ । विषु॑रूपा॒ इति॒ विषु॑ - रू॒पाः॒ । यत् । सल॑क्ष्माण॒ इति॒ स - ल॒क्ष्मा॒णः॒ । भव॑थ । दे॒व॒त्रेति॑ देव - त्रा । यन्त᳚म् । अव॑से । सखा॑यः । अन्विति॑ । त्वा॒ । मा॒ता । पि॒तरः॑ । म॒द॒न्तु॒ । श्रीः । अ॒सी॒ । अ॒ग्निः । त्वा॒ । श्री॒णा॒तु॒ । आपः॑ । समिति॑ । अ॒रि॒ण॒न्न् । वात॑स्य । \textbf{  18} \newline
                  \newline
                                \textbf{ TS 1.3.10.2} \newline
                  त्वा॒ । ध्रज्यै᳚ । पू॒ष्णः । रꣳह्यै᳚ । अ॒पाम् । ओष॑धीनाम् । रोहि॑ष्यै । घृ॒तम् । घृ॒त॒पा॒वा॒न॒ इति॑ घृत - पा॒वा॒नः॒ । पि॒ब॒त॒ । वसा᳚म् । व॒सा॒पा॒वा॒न॒ इति॑ वसा - पा॒वा॒नः॒ । पि॒ब॒त॒ । अ॒न्तरि॑क्षस्य । ह॒विः । अ॒सि॒ । स्वाहा᳚ । त्वा॒ । अ॒न्तरि॑क्षाय । दिशः॑ । प्र॒दिश॒ इति॑ प्र - दिशः॑ । आ॒दिश॒ इत्या᳚ - दिशः॑ । वि॒दिश॒ इति॑ वि - दिशः॑ । उ॒द्दिश॒ इत्यु॑त् - दिशः॑ । स्वाहा᳚ । दि॒ग्भ्य इति॑ दिक् - भ्यः । नमः॑ । दि॒ग्भ्य इति॑ दिक् - भ्यः ॥ \textbf{  19} \newline
                  \newline
                      (वा॑तस्या॒-ष्टाविꣳ॑शतिश्च)  \textbf{(A10)} \newline \newline
                                \textbf{ TS 1.3.11.1} \newline
                  स॒मु॒द्रम् । ग॒च्छ॒ । स्वाहा᳚ । अ॒न्तरि॑क्षम् । ग॒च्छ॒ । स्वाहा᳚ । दे॒वम् । स॒वि॒तार᳚म् । ग॒च्छ॒ । स्वाहा᳚ । अ॒हो॒रा॒त्रे इत्य॑हः - रा॒त्रे । ग॒च्छ॒ । स्वाहा᳚ । मि॒त्रावरु॑णा॒विति॑ मि॒त्रा - वरु॑णौ । ग॒च्छ॒ । स्वाहा᳚ । सोम᳚म् । ग॒च्छ॒ । स्वाहा᳚ । य॒ज्ञ्म् । ग॒च्छ॒ । स्वाहा᳚ । छन्दाꣳ॑सि । ग॒च्छ॒ । स्वाहा᳚ । द्यावा॑पृथि॒वी इति॒ द्यावा᳚ - पृ॒थि॒वी । ग॒च्छ॒ । स्वाहा᳚ । नभः॑ । दि॒व्यम् । ग॒च्छ॒ । स्वाहा᳚ । अ॒ग्निम् । वै॒श्वा॒न॒रम् । ग॒च्छ॒ । स्वाहा᳚ । अ॒द्भ्य इत्य॑त् - भ्यः । त्वा॒ । ओष॑धीभ्य॒ इत्योष॑धि - भ्यः॒ । मनः॑ । मे॒ । हार्दि॑ । य॒च्छ॒ । त॒नूम् । त्वच᳚म् । पु॒त्रम् । नप्ता॑रम् । अ॒शी॒य॒ । शुक् । अ॒सि॒ ( ) । तम् । अ॒भीति॑ । शो॒च॒ । यः । अ॒स्मान् । द्वेष्टि॑ । यम् । च॒ । व॒यम् । द्वि॒ष्मः । धाम्नो॑ धाम्न॒ इति॒ धाम्नः॑ - धा॒म्नः॒ । रा॒ज॒न्न् । इ॒तः । व॒रु॒ण॒ । नः॒ । मु॒ञ्च॒ । यत् । आपः॑ । अघ्नि॑याः । वरु॑ण । इति॑ । शपा॑महे । ततः॑ । व॒रु॒ण॒ । नः॒ । मु॒ञ्च॒ ॥ \textbf{  20 } \newline
                  \newline
                      (अ॒सि॒-षड्विꣳ॑शतिश्च )  \textbf{(A11)} \newline \newline
                                \textbf{ TS 1.3.12.1} \newline
                  ह॒विष्म॑तीः । इ॒माः । आपः॑ । ह॒विष्मान्॑ । दे॒वः । अ॒द्ध्व॒रः । ह॒विष्मान्॑ । एति॑ । वि॒वा॒स॒ति॒ । ह॒विष्मान्॑ । अ॒स्तु॒ । सूर्यः॑ ॥ अ॒ग्नेः । वः॒ । अप॑न्नगृह॒स्येत्यप॑न्न - गृ॒ह॒स्य॒ । सद॑सि । सा॒द॒या॒मि॒ । सु॒म्नाय॑ । सु॒म्नि॒नीः॒ । सु॒म्ने । मा॒ । ध॒त्त॒ । इ॒न्द्रा॒ग्नि॒योरिती᳚न्द्र - अ॒ग्नि॒योः । भा॒ग॒धेयी॒रिति॑ भाग - धेयीः᳚ । स्थ॒ । मि॒त्रावरु॑णयो॒रिति॑ मि॒त्रा - वरु॑णयोः । भा॒ग॒धेयी॒रिति॑ भाग - धेयीः᳚ । स्थ॒ । विश्वे॑षाम् । दे॒वाना᳚म् । भा॒ग॒धेयी॒रिति॑ भाग - धेयीः᳚ । स्थ॒ । य॒ज्ञे । जा॒गृ॒त॒ ॥ \textbf{  21 } \newline
                  \newline
                      (ह॒विष्म॑ती॒-श्चतु॑स्त्रिꣳशत्)  \textbf{(A12)} \newline \newline
                                \textbf{ TS 1.3.13.1} \newline
                  हृ॒दे । त्वा॒ । मन॑से । त्वा॒ । दि॒वे । त्वा॒ । सूर्या॑य । त्वा॒ । ऊ॒र्द्ध्वम् । इ॒मम् । अ॒द्ध्व॒रम् । कृ॒धि॒ । दि॒वि । दे॒वेषु॑ । होत्राः᳚ । य॒च्छ॒ । सोम॑ । रा॒ज॒न्न् । एति॑ । इ॒हि॒ । अवेति॑ । रो॒ह॒ । मा । भेः । मा । समिति॑ । वि॒क्थाः॒ । मा । त्वा॒ । हिꣳ॒॒सि॒ष॒म् । प्र॒जा इति॑ प्र - जाः । त्वम् । उ॒पाव॑रो॒हेत्यु॑प - अव॑रोह । प्र॒जा इति॑ प्र - जाः । त्वाम् । उ॒पाव॑रोह॒न्त्वित्यु॑प - अव॑रोहन्तु । शृ॒णोतु॑ । अ॒ग्निः । स॒मिधेति॑ सम् - इधा᳚ । हव᳚म् । मे॒ । शृ॒ण्वन्तु॑ । आपः॑ । धि॒षणाः᳚ । च॒ । दे॒वीः ॥ शृ॒णोत॑ । ग्रा॒वा॒णः॒ । वि॒दुषः॑ । नु । \textbf{  22} \newline
                  \newline
                                \textbf{ TS 1.3.13.2} \newline
                  य॒ज्ञ्म् । शृ॒णोतु॑ । दे॒वः । स॒वि॒ता । हव᳚म् । मे॒ ॥ देवीः᳚ । आ॒पः॒ । अ॒पा॒म् । न॒पा॒त् । यः । ऊ॒र्मिः । ह॒वि॒ष्यः॑ । इ॒न्द्रि॒यावा॒निती᳚न्द्रि॒य - वा॒न् । म॒दिन्त॑मः । तम् । दे॒वेभ्यः॑ । दे॒व॒त्रेति॑ देव - त्रा । ध॒त्त॒ । शु॒क्रम् । शु॒क्र॒पेभ्य॒ इति॑ शुक्र - पेभ्यः॑ । येषा᳚म् । भा॒गः । स्थ । स्वाहा᳚ । कार्.षिः॑ । अ॒सि॒ । अपेति॑ । अ॒पाम् । मृ॒द्ध्रम् । स॒मु॒द्रस्य॑ । वः॒ । अक्षि॑त्यै । उदिति॑ । न॒ये॒ ॥ यम् । अ॒ग्ने॒ । पृ॒थ्स्विति॑ पृ॒त् - सु । मर्त्य᳚म् । आवः॑ । वाजे॑षु । यम् । जु॒नाः ॥ सः । यन्ता᳚ । शश्व॑तीः । इषः॑ ॥ \textbf{  23} \newline
                  \newline
                      ( नु-स॒प्तच॑त्वारिꣳशच्च)  \textbf{(A13)} \newline \newline
                                \textbf{ TS 1.3.14.1} \newline
                  त्वम् । अ॒ग्ने॒ । रु॒द्रः । असु॑रः । म॒हः । दि॒वः । त्वम् । शर्धः॑ । मारु॑तम् । पृ॒क्षः । ई॒शि॒षे॒ ॥ त्वम् । वातैः᳚ । अ॒रु॒णैः । या॒सि॒ । श॒ङ्ग॒य इति॑ शं - ग॒यः । त्वम् । पू॒षा । वि॒ध॒त इति॑ वि - ध॒तः । पा॒सि॒ । नु । त्मना᳚ ॥ एति॑ । वः॒ । राजा॑नम् । अ॒द्ध्व॒रस्य॑ । रु॒द्रम् । होता॑रम् । स॒त्य॒यज॒मिति॑ सत्य - यज᳚म् । रोद॑स्योः ॥ अ॒ग्निम् । पु॒रा । त॒न॒यि॒त्नोः । अ॒चित्ता᳚त् । हिर॑ण्यरूप॒मिति॒ हिर॑ण्य - रू॒प॒म् । अव॑से । कृ॒णु॒द्ध्व॒म् ॥ अ॒ग्निः । होता᳚ । नीति॑ । स॒सा॒द॒ । यजी॑यान् । उ॒पस्थ॒ इत्यु॒प - स्थे॒ । मा॒तुः । सु॒र॒भौ । उ॒ । लो॒के ॥ युवा᳚ । क॒विः । पु॒रु॒नि॒ष्ठ इति॑ पुरु - नि॒ष्ठः । \textbf{  24} \newline
                  \newline
                                \textbf{ TS 1.3.14.2} \newline
                  ऋ॒तावेत्यृ॒ता - वा॒ । ध॒र्ता । कृ॒ष्टी॒नाम् । उ॒त । मद्ध्ये᳚ । इ॒द्धः ॥ सा॒द्ध्वीम् । अ॒कः॒ । दे॒ववी॑ति॒मिति॑ दे॒व - वी॒ति॒म् । नः॒ । अ॒द्य । य॒ज्ञ्स्य॑ । जि॒ह्वाम् । अ॒वि॒दा॒म॒ । गुह्या᳚म् ॥ सः । आयुः॑ । एति॑ । अ॒गा॒त् । सु॒र॒भिः । वसा॑नः । भ॒द्राम् । अ॒कः॒ । दे॒वहू॑ति॒मिति॑ दे॒व - हू॒ति॒म् । नः॒ । अ॒द्य ॥ अक्र॑न्दत् । अ॒ग्निः । स्त॒नयन्न्॑ । इ॒व॒ । द्यौः । क्षाम॑ । रेरि॑हत् । वी॒रुधः॑ । स॒म॒ञ्जन्निति॑ सं - अ॒ञ्जन्न् ॥ स॒द्यः । ज॒ज्ञा॒नः । वीति॑ । हि । ई॒म् । इ॒द्धः । अख्य॑त् । एति॑ । रोद॑सी॒ इति॑ । भा॒नुना᳚ । भा॒ति॒ । अ॒न्तः ॥ त्वे इति॑ । वसू॑नि । पु॒र्व॒णी॒केति॑ पुरु - अ॒नी॒क॒ । \textbf{  25} \newline
                  \newline
                                \textbf{ TS 1.3.14.3} \newline
                  हो॒तः॒ । दो॒षा । वस्तोः᳚ । एति॑ । ई॒रि॒रे॒ । य॒ज्ञिया॑सः ॥ क्षाम॑ । इ॒व॒ । विश्वा᳚ । भुव॑नानि । यस्मिन्न्॑ । समिति॑ । सौभ॑गानि । द॒धि॒रे । पा॒व॒के ॥ तुभ्य᳚म् । ताः । अ॒ङ्गि॒र॒स्त॒मेत्य॑ङ्गिरः - त॒म॒ । विश्वाः᳚ । सु॒क्षि॒तय॒ इति॑ सु - क्षि॒तयः॑ । पृथ॑क् ॥ अग्ने᳚ । कामा॑य । ये॒मि॒रे॒ ॥ अ॒श्याम॑ । तम् । काम᳚म् । अ॒ग्ने॒ । तव॑ । ऊ॒ती । अ॒श्याम॑ । र॒यिम् । र॒यि॒व॒ इति॑ रयि - वः॒ । सु॒वीर॒मिति॑ सु - वीर᳚म् ॥ अ॒श्याम॑ । वाज᳚म् । अ॒भीति॑ । वा॒जय॑न्तः । अ॒श्याम॑ । द्यु॒म्नम् । अ॒ज॒र॒ । अ॒जर᳚म् । ते॒ ॥ श्रेष्ठ᳚म् । य॒वि॒ष्ठ॒ । भा॒र॒त॒ । अग्ने᳚ । द्यु॒मन्त॒मिति॑ द्यु - मन्त᳚म् । एति॑ । भ॒र॒ ॥ \textbf{  26} \newline
                  \newline
                                \textbf{ TS 1.3.14.4} \newline
                  वसो॒ इति॑ । पु॒रु॒स्पृह॒मिति॑ पुरु - स्पृह᳚म् । र॒यिम् ॥ सः । श्वि॒ता॒नः । त॒न्य॒तुः । रो॒च॒न॒स्था इति॑ रोचन - स्थाः । अ॒जरे॑भिः । नान॑दद्भि॒रिति॒ नान॑दत् - भिः॒ । यवि॑ष्ठः ॥ यः । पा॒व॒कः । पु॒रु॒तम॒ इति॑ पुरु - तमः॑ । पु॒रूणि॑ । पृ॒थूनि॑ । अ॒ग्निः । अ॒नु॒यातीत्य॑नु - याति॑ । भर्वन्न्॑ ॥ आयुः॑ । ते॒ । वि॒श्वतः॑ । द॒ध॒त् । अ॒यम् । अ॒ग्निः । वर᳚ण्यः ॥ पुनः॑ । ते॒ । प्रा॒ण इति॑ प्र - अ॒नः । एति॑ । अ॒य॒ति॒ । परेति॑ । यक्ष्म᳚म् । सु॒वा॒मि॒ । ते॒ ॥ आ॒यु॒र्दा इत्या॑युः - दाः । अ॒ग्ने॒ । ह॒विषः॑ । जु॒षा॒णः । घृ॒तप्र॑तीक॒ इति॑ घृ॒त - प्र॒ती॒कः॒ । घृ॒तयो॑नि॒रिति॑ घृ॒त - यो॒निः॒ । ए॒धि॒ ॥ घृ॒तम् । पी॒त्वा । मधु॑ । चारु॑ । गव्य᳚म् । पि॒ता । इ॒व॒ । पु॒त्रम् । अ॒भीति॑ । \textbf{  27} \newline
                  \newline
                                \textbf{ TS 1.3.14.5} \newline
                  र॒क्ष॒ता॒त् । इ॒मम् ॥ तस्मै᳚ । ते॒ । प्र॒ति॒हर्य॑त॒ इति॑ प्रति - हर्य॑ते । जात॑वेद॒ इति॒ जात॑ - वे॒दः॒ । विच॑र्.षण॒ इति॒ वि - च॒र्॒.ष॒णे॒ ॥ अग्ने᳚ । जना॑मि । सु॒ष्टु॒तिमिति॑ सु - स्तु॒तिम् ॥ दि॒वः । परीति॑ । प्र॒थ॒मम् । ज॒ज्ञे॒ । अ॒ग्निः । अ॒स्मत् । द्वि॒तीय᳚म् । परीति॑॑ । जा॒तवे॑दा॒ इति॑ जा॒त - वे॒दाः॒ ॥ तृ॒तीय᳚म् । अ॒फ्स्वित्य॑प् - सु । नृ॒मणा॒ इति॑ नृ - मनाः᳚ । अज॑स्रम् । इन्धा॑नः । ए॒न॒म् । ज॒र॒ते॒ । स्वा॒धीरिति॑ स्व - धीः ॥ शुचिः॑ । पा॒व॒क॒ । वन्द्यः॑ । अग्ने᳚ । बृ॒हत् । वीति॑ । रो॒च॒से॒ ॥ त्वम् । घृ॒तेभिः॑ । आहु॑त॒ इत्या - हु॒तः॒ ॥ दृ॒शा॒नः । रु॒क्मः । उ॒र्व्या । वीति॑ । अ॒द्यौ॒त् । दु॒र्मर्.ष॒मिति॑ दुः - मर्.ष᳚म् । आयुः॑ । श्रि॒ये । रु॒चा॒नः ॥ अ॒ग्निः । अ॒मृतः॑ । अ॒भ॒व॒त् । वयो॑भि॒रिति॒ वयः॑ - भिः॒ । \textbf{  28} \newline
                  \newline
                                \textbf{ TS 1.3.14.6} \newline
                  यत् । ए॒न॒म् । द्यौः । अज॑नयत् । सु॒रेता॒ इति॑ सु - रेताः᳚ ॥ एति॑ । यत् । इ॒षे । नृ॒पति॒मिति॑ नृ - पति᳚म् । तेजः॑ । आन॑ट् । शुचि॑ । रेतः॑ । निषि॑क्त॒मिति॒ नि - सि॒क्त॒म् । द्यौः । अ॒भीके᳚ ॥ अ॒ग्निः । शर्ध᳚म् । अ॒न॒व॒द्यम् । युवा॑नम् । स्वा॒धिय॒मिति॑ स्व - धिय᳚म् । ज॒न॒य॒त् । सू॒दय॑त् । च॒ ॥ सः । तेजी॑यसा । मन॑सा । त्वोतः॑ । उ॒त । शि॒क्ष॒ । स्व॒प॒त्यस्येति॑ सु - अ॒प॒त्यस्य॑ । शि॒क्षोः ॥ अग्ने᳚ । रा॒यः । नृत॑म॒स्येति॒ नृ - त॒म॒स्य॒ । प्रभू॑ता॒विति॒ प्र - भू॒तौ॒ । भू॒याम॑ । ते॒ । सु॒ष्टु॒तय॒ इति॑ सु - स्तु॒तयः॑ । च॒ । वस्वः॑ ॥ अग्ने᳚ । सह॑न्तम् । एति॑ । भ॒र॒ । द्यु॒म्नस्य॑ । प्रा॒सहेति॑ प्र - सहा᳚ । र॒यिम् ॥ विश्वाः᳚ । यः । \textbf{  29} \newline
                  \newline
                                \textbf{ TS 1.3.14.7} \newline
                  च॒र्॒.ष॒णीः । अ॒भीति॑ । आ॒सा । वाजे॑षु । सा॒सह॑त् ॥ तम् । अ॒ग्ने॒ । पृ॒त॒ना॒सह॒मिति॑ पृतना - सह᳚म् । र॒यिम् । स॒ह॒स्वः॒ । एति॑ । भ॒र॒ ॥ त्वम् । हि । स॒त्यः । अद्भु॑तः । दा॒ता । वाज॑स्य । गोम॑त॒ इति॒ गो - म॒तः॒ ॥ उ॒क्षान्ना॒येत्यु॒क्ष - अ॒न्ना॒य॒ । व॒शान्ना॒येति॑ व॒शा - अ॒न्ना॒य॒ । सोम॑पृष्ठा॒येति॒ सोम॑ - पृ॒ष्ठा॒य॒ । वे॒धसे᳚ ॥ स्तोमैः᳚ । वि॒धे॒म॒ । अ॒ग्नये᳚ ॥ व॒द्मा । हि । सू॒नो॒ इति॑ । असि॑ । अ॒द्म॒सद्वेत्य॑द्म - सद्वा᳚ । च॒क्रे । अ॒ग्निः । ज॒नुषा᳚ । अज्म॑ । अन्न᳚म् ॥ सः । त्वम् । नः॒ । ऊ॒र्ज॒स॒न॒ इत्यु᳚र्ज - स॒ने॒ । ऊर्ज᳚म् । धाः॒ । राजा᳚ । इ॒व॒ । जेः॒ । अ॒वृ॒के । क्षे॒षि॒ । अ॒न्तः ॥ अग्ने᳚ । आयूꣳ॑षि । \textbf{  30} \newline
                  \newline
                                \textbf{ TS 1.3.14.8} \newline
                  प॒व॒से॒ । एति॑ । सु॒व॒ । ऊर्ज᳚म् । इष᳚म् । च॒ । नः॒ ॥ आ॒रे । बा॒ध॒स्व॒ । दु॒च्छुना᳚म् ॥ अग्ने᳚ । पव॑स्व । स्वपा॒ इति॑ सु - अपाः᳚ । अ॒स्मे इति॑ । वर्चः॑ । सु॒वीर्य॒मिति॑ सु - वीर्य᳚म् ॥ दध॑त् । पोष᳚म् । र॒यिम् । मयि॑ ॥ अग्ने᳚ । पा॒व॒क॒ । रो॒चिषा᳚ । म॒न्द्रया᳚ । दे॒व॒ । जि॒ह्वया᳚ ॥ एति॑ । दे॒वान् । व॒क्षि॒ । यक्षि॑ । च॒ ॥ सः । नः॒ । पा॒व॒क॒ । दी॒दि॒वः॒ । अग्ने᳚ । दे॒वान् । इ॒ह । एति॑ । व॒ह॒ ॥ उपेति॑ । य॒ज्ञ्म् । ह॒विः । च॒ । नः॒ ॥ अ॒ग्निः । शुचि॑व्रततम॒ इति॒ शुचि॑व्रत - त॒मः॒ । शुचिः॑ । विप्रः॑ । शुचिः॑ ( ) । क॒विः ॥ शुचिः॑ । रो॒च॒ते॒ । आहु॑त॒ इत्या - हु॒तः॒ ॥ उदिति॑ । अ॒ग्ने॒ । शुच॑यः । तव॑ । शु॒क्राः । भ्राज॑न्तः । ई॒र॒ते॒ ॥ तव॑ । ज्योतीꣳ॑षि । अ॒र्चयः॑ ॥ \textbf{  31} \newline
                  \newline
                      विप्रः॒ शुचि॒-श्चतु॑र्दश च)  \textbf{(A14)} \newline \newline
\textbf{praSna korvai with starting padams of 1 to 14 anuvAkams :-} \newline
(दे॒वस्य॑-रक्षो॒हणो॑ -वि॒भू-स्त्वꣳ सो॒मा- ऽत्य॒न्यानगां᳚ - पृथि॒व्या -इ॒षे त्वा - ऽऽद॑दे॒ - वाक्त॒-सन्ते॑- समु॒द्रꣳ- ह॒विष्म॑तीर्.-हृ॒दे- त्वम॑ग्ने रु॒द्र-श्चतु॑र्दश ) \newline

\textbf{korvai with starting padams of1, 11, 21 series of pa~jcAtis :-} \newline
(दे॒वस्य॑-ग॒मध्ये॑-ह॒विष्म॑तीः-पवस॒-एक॑त्रिꣳशत्) \newline

\textbf{first and last padam of third praSnam:-} \newline
(दे॒वस्या॒-र्चयः॑ ) \newline 


॥ हरिः॑ ॐ ॥
॥ कृष्ण यजुर्वेदीय तैत्तिरीय संहितायां प्रथमकाण्डे तृतीयः प्रश्नः समाप्तः ॥ \newline
\pagebreak
\pagebreak
        


\end{document}
