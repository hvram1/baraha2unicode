\documentclass[17pt]{extarticle}
\usepackage{babel}
\usepackage{fontspec}
\usepackage{polyglossia}
\usepackage{extsizes}



\setmainlanguage{sanskrit}
\setotherlanguages{english} %% or other languages
\setlength{\parindent}{0pt}
\pagestyle{myheadings}
\newfontfamily\devanagarifont[Script=Devanagari]{AdishilaVedic}


\newcommand{\VAR}[1]{}
\newcommand{\BLOCK}[1]{}




\begin{document}
\begin{titlepage}
    \begin{center}
 
\begin{sanskrit}
    { \Large
    ॐ नमः परमात्मने, श्री महागणपतये नमः, 
श्री गुरुभ्यो नमः, ह॒रिः॒ ॐ 
    }
    \\
    \vspace{2.5cm}
    \mbox{ \Huge
    4.4      चतुर्थकाण्डे चतुर्थः प्रश्नः - पञ्चमचितिशेषनिरूपणं   }
\end{sanskrit}
\end{center}

\end{titlepage}
\tableofcontents

ॐ नमः परमात्मने, श्री महागणपतये नमः, 
श्री गुरुभ्यो नमः , ह॒रिः॒ ॐ \newline
4.4      चतुर्थकाण्डे चतुर्थः प्रश्नः - पञ्चमचितिशेषनिरूपणं \newline

\addcontentsline{toc}{section}{ 4.4      चतुर्थकाण्डे चतुर्थः प्रश्नः - पञ्चमचितिशेषनिरूपणं}
\markright{ 4.4      चतुर्थकाण्डे चतुर्थः प्रश्नः - पञ्चमचितिशेषनिरूपणं \hfill https://www.vedavms.in \hfill}
\section*{ 4.4      चतुर्थकाण्डे चतुर्थः प्रश्नः - पञ्चमचितिशेषनिरूपणं }
                                \textbf{ TS 4.4.1.1} \newline
                  र॒श्मिः । अ॒सि॒ । क्षया॑य । त्वा॒ । क्षय᳚म् । जि॒न्व॒ । प्रेति॒रिति॒ प्र - इ॒तिः॒ । अ॒सि॒ । धर्मा॑य । त्वा॒ । धर्म᳚म् । जि॒न्व॒ । अन्वि॑ति॒रित्यनु॑ - इ॒तिः॒ । अ॒सि॒ । दि॒वे । त्वा॒ । दिव᳚म् । जि॒न्व॒ । स॒न्धिरिति॑ सं - धिः । अ॒सि॒ । अ॒न्तरि॑क्षाय । त्वा॒ । अ॒न्तरि॑क्षम् । जि॒न्व॒ । प्र॒ति॒धिरिति॑ प्रति-धिः । अ॒सि॒ । पृ॒थि॒व्यै । त्वा॒ । पृ॒थि॒वीम् । जि॒न्व॒ । वि॒ष्ट॒भं इति॑ वि-स्त॒भंः । अ॒सि॒ । वृष्ट्यै᳚ । त्वा॒ । वृष्टि᳚म् । जि॒न्व॒ । प्र॒वेति॑ प्र - वा । अ॒सि॒ । अह्ने᳚ । त्वा॒ । अहः॑ । जि॒न्व॒ । अ॒नु॒वेत्य॑नु - वा । अ॒सि॒ । रात्रि॑यै । त्वा॒ । रात्रि᳚म् । जि॒न्व॒ । उ॒शिक् । अ॒सि॒ । \textbf{  1} \newline
                  \newline
                                \textbf{ TS 4.4.1.2} \newline
                  वसु॑भ्य॒ इति॒ वसु॑ - भ्यः॒ । त्वा॒ । वसून्॑ । जि॒न्व॒ । प्र॒के॒त इति॑ प्र - के॒तः । अ॒सि॒ । रु॒द्रेभ्यः॑ । त्वा॒ । रु॒द्रान् । जि॒न्व॒ । सु॒दी॒तिरिति॑ सु- दी॒तिः । अ॒सि॒ । आ॒दि॒त्येभ्यः॑ । त्वा॒ । आ॒दि॒त्यान् । जि॒न्व॒ । ओजः॑ । अ॒सि॒ । पि॒तृभ्य॒ इति॑ पि॒तृ - भ्यः॒ । त्वा॒ । पि॒तॄन् । जि॒न्व॒ । तन्तुः॑ । अ॒सि॒ । प्र॒जाभ्य॒ इति॑ प्र - जाभ्यः॑ । त्वा॒ । प्र॒जा इति॑ प्र - जाः । जि॒न्व॒ । पृ॒त॒ना॒षाट् । अ॒सि॒ । प॒शुभ्य॒ इति॑ प॒शु - भ्यः॒ । त्वा॒ । प॒शून् । जि॒न्व॒ । रे॒वत् । अ॒सि॒ । ओष॑धीभ्य॒ इत्योष॑धि - भ्यः॒ । त्वा॒ । ओष॑धीः । जि॒न्व॒ । अ॒भि॒जिदित्य॑भि - जित् । अ॒सि॒ । यु॒क्तग्रा॒वेति॑ यु॒क्त - ग्रा॒वा॒ । इन्द्रा॑य । त्वा॒ । इन्द्र᳚म् । जि॒न्व॒ । अधि॑पति॒रित्यधि॑-प॒तिः॒ । अ॒सि॒ । प्रा॒णायेति॑ प्र-अ॒नाय॑ । \textbf{  2} \newline
                  \newline
                                \textbf{ TS 4.4.1.3} \newline
                  त्वा॒ । प्रा॒णमिति॑ प्र - अ॒नम् । जि॒न्व॒ । य॒न्ता । अ॒सि॒ । अ॒पा॒नायेत्य॑प - अ॒नाय॑ । त्वा॒ । अ॒पा॒नमित्य॑प - अ॒नम् । जि॒न्व॒ । सꣳ॒॒सर्प॒ इति॑ सं - सर्पः॑ । अ॒सि॒ । चक्षु॑षे । त्वा॒ । चक्षुः॑ । जि॒न्व॒ । व॒यो॒धा इति॑ वयो - धाः । अ॒सि॒ । श्रोत्रा॑य । त्वा॒ । श्रोत्र᳚म् । जि॒न्व॒ । त्रि॒वृदिति॑ त्रि - वृत् । अ॒सि॒ । प्र॒वृदिति॑ प्र - वृत् । अ॒सि॒ । सं॒ॅवृदिति॑ सं - वृत् । अ॒सि॒ । वि॒वृदिति वि - वृत् । अ॒सि॒ । सꣳ॒॒रो॒ह इति॑ सं - रो॒हः । अ॒सि॒ । नी॒रो॒ह इति॑ निः - रो॒हः । अ॒सि॒ । प्र॒रो॒ह इति॑ प्र - रो॒हः । अ॒सि॒ । अ॒नु॒रो॒ह इत्य॑नु - रो॒हः । अ॒सि॒ । व॒सु॒कः । अ॒सि॒ । वेष॑श्रि॒रिति॒ वेष॑ - श्रिः॒ । अ॒सि॒ । वस्य॑ष्टिः । अ॒सि॒ ॥ \textbf{  3} \newline
                  \newline
                      (उ॒शिग॑सि - प्रा॒णाय॒ - त्रिच॑त्वारिꣳशच्च)  \textbf{(A1)} \newline \newline
                                \textbf{ TS 4.4.2.1} \newline
                  राज्ञी᳚ । अ॒सि॒ । प्राची᳚ । दिक् । वस॑वः । ते॒ । दे॒वाः । अधि॑पतय॒ इत्यधि॑ - प॒त॒यः॒ । अ॒ग्निः । हे॒ती॒नाम् । प्र॒ति॒ध॒र्तेति॑ प्रति - ध॒र्ता । त्रि॒वृदिति॑ त्रि-वृत् । त्वा॒ । स्तोमः॑ । पृ॒थि॒व्याम् । श्र॒य॒तु॒ । आज्य᳚म् । उ॒क्थम् । अव्य॑थयत् । स्त॒भ्ना॒तु॒ । र॒थ॒न्त॒रमिति॑ रथं - त॒रम् । साम॑ । प्रति॑ष्ठित्या॒ इति॒ प्रति॑ - स्थि॒त्यै॒ । वि॒राडिति॑ वि - राट् । अ॒सि॒ । द॒क्षि॒णा । दिक् । रु॒द्राः । ते॒ । दे॒वाः । अधि॑पतय॒ इत्यधि॑ - प॒त॒यः॒ । इन्द्रः॑ । हे॒ती॒नाम् । प्र॒ति॒ध॒र्तेति॑ प्रति - ध॒र्ता । प॒ञ्च॒द॒श इति॑ पञ्च - द॒शः । त्वा॒ । स्तोमः॑ । पृ॒थि॒व्याम् । श्र॒य॒तु॒ । प्र उ॑गम् । उ॒क्थम् । अव्य॑थयत् । स्त॒भ्ना॒तु॒ । बृ॒हत् । साम॑ । प्रति॑ष्ठित्या॒ इति॒ प्रति॑ - स्थि॒त्यै॒ । स॒म्राडिति॑ सं - राट् । अ॒सि॒ । प्र॒तीची᳚ । दिक् । \textbf{  4} \newline
                  \newline
                                \textbf{ TS 4.4.2.2} \newline
                  आ॒दि॒त्याः । ते॒ । दे॒वाः । अधि॑पतय॒ इत्यधि॑ - प॒त॒यः॒ । सोमः॑ । हे॒ती॒नाम् । प्र॒ति॒ध॒र्तेति॑ प्रति - ध॒र्ता । स॒प्त॒द॒श इति॑ सप्त - द॒शः । त्वा॒ । स्तोमः॑ । पृ॒थि॒व्याम् । श्र॒य॒तु॒ । म॒रु॒त्व॒तीय᳚म् । उ॒क्थम् । अव्य॑थयत् । स्त॒भ्ना॒तु॒ । वै॒रू॒पम् । साम॑ । प्रति॑ष्ठित्या॒ इति॒ प्रति॑ - स्थि॒त्यै॒ । स्व॒राडिति॑ स्व - राट् । अ॒सि॒ । उदी॑ची । दिक् । विश्वे᳚ । ते॒ । दे॒वाः । अधि॑पतय॒ इत्यधि॑ - प॒त॒यः॒ । वरु॑णः । हे॒ती॒नाम् । प्र॒ति॒ध॒र्तेति॑ प्रति - ध॒र्ता । ए॒क॒विꣳ॒॒श इत्ये॑क-विꣳ॒॒शः । त्वा॒ । स्तोमः॑ । पृ॒थि॒व्याम् । श्र॒य॒तु॒ । निष्के॑वल्यम् । उ॒क्थम् । अव्य॑थयत् । स्त॒भ्ना॒तु॒ । वै॒रा॒जम् । साम॑ । प्रति॑ष्ठित्या॒ इति॒ प्रति॑ - स्थि॒त्यै॒ । अधि॑प॒त्नीत्यधि॑ - प॒त्नी॒ । अ॒सि॒ । बृ॒ह॒ती । दिक् । म॒रुतः॑ । ते॒ । दे॒वाः । अधि॑पतय॒ इत्यधि॑ - प॒त॒यः॒ । \textbf{  5} \newline
                  \newline
                                \textbf{ TS 4.4.2.3} \newline
                  बृह॒स्पतिः॑ । हे॒ती॒नाम् । प्र॒ति॒ध॒र्तेति॑ प्रति - ध॒र्ता । त्रि॒ण॒व॒त्र॒य॒स्त्रिꣳ॒॒शाविति॑ त्रिणव - त्र॒य॒स्त्रिꣳ॒॒शौ । त्वा॒ । स्तोमौ᳚ । पृ॒थि॒व्याम् । श्र॒य॒ता॒म् । वै॒श्व॒दे॒वा॒ग्नि॒मा॒रु॒ते इति॑ वैश्वदेव-अ॒ग्नि॒मा॒रु॒ते । उ॒क्थे इति॑ । अव्य॑थयन्ती॒ इति॑ । स्त॒भ्नी॒ता॒म् । शा॒क्व॒र॒रै॒व॒ते इति॑ शाक्वर - रै॒व॒ते । साम॑नी॒ इति॑ । प्रति॑ष्ठित्या॒ इति॒ प्रति॑ - स्थि॒त्यै॒ । अ॒न्तरि॑क्षाय । ऋष॑यः । त्वा॒ । प्र॒थ॒म॒जा इति॑ प्रथम - जाः । दे॒वेषु॑ । दि॒वः । मात्र॑या । व॒रि॒णा । प्र॒थ॒न्तु॒ । वि॒ध॒र्तेति॑ वि - ध॒र्ता । च॒ । अ॒यम् । अधि॑पति॒रित्यधि॑-प॒तिः॒ । च॒ । ते । त्वा॒ । सर्वे᳚ । सं॒ॅवि॒दा॒ना इति॑ सं - वि॒दा॒नाः । नाक॑स्य । पृ॒ष्ठे । सु॒व॒र्ग इति॑ सुवः - गे । लो॒के । यज॑मानम् । च॒ । सा॒द॒य॒न्तु॒ ॥ \textbf{  6 } \newline
                  \newline
                      (प्र॒तीची॒ दिङ् - म॒रुत॑स्ते दे॒वा अधि॑पतय - श्चत्वारिꣳ॒॒शच्च॑)  \textbf{(A2)} \newline \newline
                                \textbf{ TS 4.4.3.1} \newline
                  अ॒यम् । पु॒रः । हरि॑केश॒ इति॒ हरि॑ - के॒शः॒ । सूर्य॑रश्मि॒रिति॒ सूर्य॑ - र॒श्मिः॒ । तस्य॑ । र॒थ॒गृ॒थ्स इति॑ रथ - गृ॒थ्सः । च॒ । रथौ॑जा॒ इति॒ रथ॑ - ओ॒जाः॒ । च॒ । से॒ना॒नि॒ग्रा॒म॒ण्या॑विति॑ सेनानि - ग्रा॒म॒ण्यौ᳚ । पु॒ञ्जि॒क॒स्थ॒लेति॑ पुञ्जिक - स्थ॒ला । च॒ । कृ॒त॒स्थ॒लेति॑ कृत - स्थ॒ला । च॒ । अ॒फ्स॒रसौ᳚ । या॒तु॒धाना॒ इति॑ यातु - धानाः᳚ । हे॒तिः । रक्षाꣳ॑सि । प्रहे॑ति॒रिति॒ प्र-हे॒तिः॒ । अ॒यम् । द॒क्षि॒णा । वि॒श्वक॒र्मेति॑ वि॒श्व - क॒र्मा॒ । तस्य॑ । र॒थ॒स्व॒न इति॑ रथ - स्व॒नः । च॒ । रथे॑चित्र॒ इति॒ रथे᳚ - चि॒त्रः॒ । च॒ । से॒ना॒नि॒ग्रा॒म॒ण्या॑विति॑ सेनानि - ग्रा॒म॒ण्यौ᳚ । मे॒न॒का । च॒ । स॒ह॒ज॒न्येति॑ सह - ज॒न्या । च॒ । अ॒फ्स॒रसौ᳚ । द॒ङ्णवः॑ । प॒शवः॑ । हे॒तिः । पौरु॑षेयः । व॒धः । प्रहे॑ति॒रिति॒ प्र - हे॒तिः॒ । अ॒यम् । प॒श्चात् । वि॒श्वव्य॑चा॒ इति॑ वि॒श्व - व्य॒चाः॒ । तस्य॑ । रथ॑प्रोत॒ इति॒ रथ॑-प्रो॒तः॒ । च॒ । अस॑मरथ॒ इत्यस॑म - र॒थः॒ । च॒ । से॒ना॒नि॒ग्रा॒म॒ण्या॑विति॑ सेनानि - ग्रा॒म॒ण्यौ᳚ । प्र॒म्लोच॒न्तीति॑ प्र - म्लोच॑न्ती । च॒ । \textbf{  7} \newline
                  \newline
                                \textbf{ TS 4.4.3.2} \newline
                  अ॒नु॒म्लोच॒न्तीत्य॑नु - म्लोच॑न्ती । च॒ । अ॒फ्स॒रसौ᳚ । स॒र्पाः । हे॒तिः । व्या॒घ्राः । प्रहे॑ति॒रिति॒ प्र - हे॒तिः॒ । अ॒यम् । उ॒त्त॒रादित्यु॑त्-त॒रात् । सं॒ॅयद्व॑सु॒रिति॑ सं॒ॅयत् - व॒सुः॒ । तस्य॑ । से॒न॒जिदिति॑ सेन-जित् । च॒ । सु॒षेण॒ इति॑ सु - सेनः॑ । च॒ । से॒ना॒नि॒ग्रा॒म॒ण्या॑विति॑ सेनानि - ग्रा॒म॒ण्यौ᳚ । वि॒श्वाची᳚ । च॒ । घृ॒ताची᳚ । च॒ । अ॒फ्स॒रसौ᳚ । आपः॑ । हे॒ति । वातः॑ । प्रहे॑ति॒रिति॒ प्र - हे॒तिः॒ । अ॒यम् । उ॒परि॑ । अ॒र्वाग्व॑सु॒रित्य॒र्वाक् - व॒सुः॒ । तस्य॑ । तार्क्ष्यः॑ । च॒ । अरि॑ष्टनेमि॒रित्यरि॑ष्ट - ने॒मिः॒ । च॒ । से॒ना॒नि॒ग्रा॒म॒ण्या॑विति॑ सेनानि - ग्रा॒म॒ण्यौ᳚ । उ॒र्वशी᳚ । च॒ । पू॒र्वचि॑त्ति॒रिति॑ पू॒र्व - चि॒त्तिः॒ । च॒ । अ॒फ्स॒रसौ᳚ । वि॒द्युदिति॑ वित् - युत् । हे॒तिः । अ॒व॒स्फूर्ज॒न्नित्य॑व - स्फूर्जन्न्॑ । प्रहे॑ति॒रिति॒ प्र - हे॒तिः॒ । तेभ्यः॑ । नमः॑ । ते । नः॒ । मृ॒ड॒यन्तु॒ । ते । यम् । \textbf{  8} \newline
                  \newline
                                \textbf{ TS 4.4.3.3} \newline
                  द्वि॒ष्मः । यः । च॒ । नः॒ । द्वेष्टि॑ । तम् । वः॒ । जंभे᳚ । द॒धा॒मि॒ । आ॒योः । त्वा॒ । सद॑ने । सा॒द॒या॒मि॒ । अव॑तः । छा॒याया᳚म् । नमः॑ । स॒मु॒द्राय॑ । नमः॑ । स॒मु॒द्रस्य॑ । चक्ष॑से । प॒र॒मे॒ष्ठी । त्वा॒ । सा॒द॒य॒तु॒ । दि॒वः । पृ॒ष्ठे । व्यच॑स्वतीम् । प्रथ॑स्वतीम् । वि॒भूम॑ती॒मिति॑ वि - भूम॑तीम् । प्र॒भूम॑ती॒मिति॑ प्र - भूम॑तीम् । प॒रि॒भूम॑ती॒मिति॑ परि - भूम॑तीम् । दिव᳚म् । य॒च्छ॒ । दिव᳚म् । दृꣳ॒॒ह॒ । दिव᳚म् । मा । हिꣳ॒॒सीः॒ । विश्व॑स्मै । प्रा॒णायेति॑ प्र - अ॒नाय॑ । अ॒पा॒नायेत्य॑प - अ॒नाय॑ । व्या॒नायेति॑ वि - अ॒नाय॑ । उ॒दा॒नायेत्यु॑त् - अ॒नाय॑ । प्र॒ति॒ष्ठाया॒ इति॑ प्रति - स्थायै᳚ । च॒रित्रा॑य । सूर्यः॑ । त्वा॒ । अ॒भीति॑ । पा॒तु॒ । म॒ह्या । स्व॒स्त्या ( ) । छ॒र्दिषा᳚ । शन्त॑मे॒नेति॒ शं - त॒मे॒न॒ । तया᳚ । दे॒वत॑या । अ॒ङ्गि॒र॒स्वत् । ध्रु॒वा । सी॒द॒ ॥ प्रोथ॑त् । अश्वः॑ । न । यव॑से । अ॒वि॒ष्यन्न् । य॒दा । म॒हः । सं॒ॅवर॑णा॒दिति॑ सं - वर॑णात् । व्यस्था॒दिति॑ वि - अस्था᳚त् ॥ आत् । अ॒स्य॒ । वातः॑ । अन्विति॑ । वा॒ति॒ । शो॒चिः । अध॑ । स्म॒ । ते॒ । व्रज॑नम् । कृ॒ष्णम् । अ॒स्ति॒ ॥ \textbf{  9} \newline
                  \newline
                      (प्र॒म्लोच॑न्ती च॒ - यꣳ - स्व॒स्त्या - ऽष्टाविꣳ॑शतिश्च)  \textbf{(A3)} \newline \newline
                                \textbf{ TS 4.4.4.1} \newline
                  अ॒ग्निः । मू॒द्‌र्धा । दि॒वः । क॒कुत् । पतिः॑ । पृ॒थि॒व्याः । अ॒यम् ॥ अ॒पाम् । रेताꣳ॑सि । जि॒न्व॒ति॒ ॥ त्वाम् । अ॒ग्ने॒ । पुष्क॑रात् । अधीति॑ । अथ॑र्वा । निरिति॑ । अ॒म॒न्थ॒त॒ ॥ मू॒द्‌र्ध्नः । विश्व॑स्य । वा॒घतः॑ ॥ अ॒यम् । अ॒ग्निः । स॒ह॒स्रिणः॑ । वाज॑स्य । श॒तिनः॑ । पतिः॑ ॥ मू॒द्‌र्धा । क॒विः । र॒यी॒णाम् ॥ भुवः॑ । य॒ज्ञ्स्य॑ । रज॑सः । च॒ । ने॒ता । यत्र॑ । नि॒युद्भि॒रिति॑ नि॒युत् - भिः॒ । सच॑से । शि॒वाभिः॑ ॥ दि॒वि । मू॒द्‌र्धान᳚म् । द॒धि॒षे॒ । सु॒व॒र्॒.षामिति॑ सुवः - साम् । जि॒ह्वाम् । अ॒ग्ने॒ । च॒कृ॒षे॒ । ह॒व्य॒वाह॒मिति॑ हव्य - वाह᳚म् ॥ अबा॑धि । अ॒ग्निः । स॒मिधेति॑ सम् - इधा᳚ । जना॑नाम् । \textbf{  10} \newline
                  \newline
                                \textbf{ TS 4.4.4.2} \newline
                  प्रतीति॑ । धे॒नुम् । इ॒व । आ॒य॒तीमित्या᳚ - य॒तीम् । उ॒षास᳚म् ॥ य॒ह्वाः । इ॒व॒ । प्रेति॑ । व॒याम् । उ॒ज्जिहा॑ना॒ इत्यु॑त् - जिहा॑नाः । प्रेति॑ । भा॒नवः॑ । सि॒स्र॒ते॒ । नाक᳚म् । अच्छ॑ ॥ अवो॑चाम । क॒वये᳚ । मेद्ध्या॑य । वचः॑ । व॒न्दारु॑ । वृ॒ष॒भाय॑ । वृष्णे᳚ ॥ गवि॑ष्ठिरः । नम॑सा । स्तोम᳚म् । अ॒ग्नौ । दि॒वि । इ॒व॒ । रु॒क्मम् । उ॒र्व्यञ्च᳚म् । अ॒श्रे॒त् ॥ जन॑स्य । गो॒पा इति॑ गो - पाः । अ॒ज॒नि॒ष्ट॒। जागृ॑विः । अ॒ग्निः । सु॒दक्ष॒ इति॑ सु - दक्षः॑ । सु॒वि॒ताय॑ । नव्य॑से ॥ घृ॒तप्र॑तीक॒ इति॑ घृ॒त - प्र॒ती॒कः॒ । बृ॒ह॒ता । दि॒वि॒स्पृशेति॑ दिवि - स्पृशा᳚ । द्यु॒मदिति॑ द्यु - मत् । वीति॑ । भा॒ति॒ । भ॒र॒तेभ्यः॑ । शुचिः॑ ॥ त्वाम् । अ॒ग्ने॒ । अङ्गि॑रसः । \textbf{  11} \newline
                  \newline
                                \textbf{ TS 4.4.4.3} \newline
                  गुहा᳚ । हि॒तम् । अन्विति॑ । अ॒वि॒न्द॒न्न् । शि॒श्रि॒या॒णम् । वने॑वन॒ इति॒ वने᳚ - व॒ने॒ ॥ सः । जा॒य॒से॒ । म॒थ्यमा॑नः । सहः॑ । म॒हत् । त्वाम् । आ॒हुः॒ । सह॑सः । पु॒त्रम् । अ॒ङ्गि॒रः॒ ॥ य॒ज्ञ्स्य॑ । के॒तुम् । प्र॒थ॒मम् । पु॒रोहि॑त॒मिति॑ पु॒रः - हि॒त॒म् । अ॒ग्निम् । नरः॑ । त्रि॒ष॒ध॒स्थ इति॑ त्रि - स॒ध॒स्थे । समिति॑ । इ॒न्ध॒ते॒ ॥ इन्द्रे॑ण । दे॒वैः । स॒रथ॒मिति॑ स - रथ᳚म् । सः । ब॒र्॒.हिषि॑ । सीद॑त् । नीति॑ । होता᳚ । य॒जथा॑य । सु॒क्रतु॒रिति॑ सु - क्रतुः॑ ॥ त्वाम् । चि॒त्र॒श्र॒व॒स्त॒मेति॑ चित्रश्रवः-त॒म॒ । हव॑न्ते । वि॒क्षु । ज॒न्तवः॑ ॥ शो॒चिष्के॑श॒मिति॑ शो॒चिः - के॒श॒म् । पु॒रु॒प्रि॒येति॑ पुरु - प्रि॒य॒ । अग्ने᳚ । ह॒व्याय॑ । वोढ॑वे ॥ सखा॑यः । समिति॑ । वः॒ । स॒म्यञ्च᳚म् । इष᳚म् । \textbf{  12} \newline
                  \newline
                                \textbf{ TS 4.4.4.4} \newline
                  स्तोम᳚म् । च॒ । अ॒ग्नये᳚ ॥ वर्.षि॑ष्ठाय । क्षि॒ती॒नाम् । ऊ॒र्जः । नप्त्रे᳚ । सह॑स्वते ॥ सꣳस॒मिति॒ सम् - स॒म् । इत् । यु॒व॒से॒ । वृ॒ष॒न्न् । अग्ने᳚ । विश्वा॑नि । अ॒र्यः । आ ॥ इ॒डः । प॒दे । समिति॑ । इ॒द्ध्य॒से॒ । सः । नः॒ । वसू॑नि । एति॑ । भ॒र॒ ॥ ए॒ना । वः॒ । अ॒ग्निम् । नम॑सा । ऊ॒र्जः । नपा॑तम् । एति॑ । हु॒वे॒ ॥ प्रि॒यम् । चेति॑ष्ठम् । अ॒र॒तिम् । स्व॒द्ध्व॒रमिति॑ सु - अ॒द्ध्व॒रम् । विश्व॑स्य । दू॒तम् । अ॒मृत᳚म् ॥ सः । यो॒ज॒ते॒ । अ॒रु॒षः । वि॒श्वभो॑ज॒सेति॑ वि॒श्व - भो॒ज॒सा॒ । सः । दु॒द्र॒व॒त् । स्वा॑हुत॒ इति॒ सु - आ॒हु॒तः॒ ॥ सु॒ब्रह्मेति॑ सु - ब्रह्मा᳚ । य॒ज्ञ्ः । सु॒शमीति॑ सु - शमी᳚ । \textbf{  13} \newline
                  \newline
                                \textbf{ TS 4.4.4.5} \newline
                  वसू॑नाम् । दे॒वम् । राधः॑ । जना॑नाम् ॥ उदिति॑ । अ॒स्य॒ । शो॒चिः । अ॒स्था॒त् । आ॒जुह्वा॑न॒स्येत्या᳚ - जुह्वा॑नस्य । मी॒ढुषः॑ ॥ उदिति॑ । धू॒मासः॑ । अ॒रु॒षासः॑ । दि॒वि॒स्पृश॒ इति॑ दिवि - स्पृशः॑ । समिति॑ । अ॒ग्निम् । इ॒न्ध॒ते॒ । नरः॑ ॥ अग्ने᳚ । वाज॑स्य । गोम॑त॒ इति॒ गो-म॒तः॒ । ईशा॑नः । स॒ह॒सः॒ । य॒हो॒ इति॑ ॥ अ॒स्मे इति॑ । धे॒हि॒ । जा॒त॒वे॒द॒ इति॑ जात - वे॒दः॒ । महि॑ । श्रवः॑ ॥ सः । इ॒धा॒नः । वसुः॑ । क॒विः । अ॒ग्निः । ई॒डेन्यः॑ । गि॒रा ॥ रे॒वत् । अ॒स्मभ्य॒मित्य॒स्म - भ्य॒म् । पु॒र्व॒णी॒केति॑ पुरु-अ॒नी॒क॒ । दी॒दि॒हि॒ ॥ क्ष॒पः । रा॒ज॒न्न् । उ॒त । त्मना᳚ । अग्ने᳚ । वस्तोः᳚ । उ॒त । उ॒षसः॑ ॥ सः । ति॒ग्म॒ज॒भेंति॑ तिग्म - ज॒भं॒ । \textbf{  14} \newline
                  \newline
                                \textbf{ TS 4.4.4.6} \newline
                  र॒क्षसः॑ । द॒ह॒ । प्रति॑ ॥ एति॑ । ते॒ । अ॒ग्ने॒ । इ॒धी॒म॒हि॒ । द्यु॒मन्त॒मिति॑ द्यु - मन्त᳚म् । दे॒व॒ । अ॒जर᳚म् ॥ यत् । ह॒ । स्या । ते॒ । पनी॑यसी । स॒मिदिति॑ सं - इत् । दी॒दय॑ति । द्यवि॑ । इष᳚म् । स्तो॒तृभ्य॒ इति॑ स्तो॒तृ - भ्यः॒ । एति॑ । भ॒र॒ ॥ एति॑ । ते॒ । अ॒ग्ने॒ । ऋ॒चा । ह॒विः । शु॒क्रस्य॑ । ज्यो॒ति॒षः॒ । प॒ते॒ ॥ सुश्च॒न्द्रेति॒ सु-च॒न्द्र॒ । दस्म॑ । विश्प॑ते । हव्य॑वा॒डिति॒ हव्य॑ - वा॒ट् । तुभ्य᳚म् । हू॒य॒ते॒ । इष᳚म् । स्तो॒तृभ्य॒ इति॑ स्तो॒तृ - भ्यः॒ । एति॑ । भ॒र॒ ॥ उ॒भे इति॑ । सु॒श्च॒न्द्रेति॑ सु - च॒न्द्र॒ । स॒र्पिषः॑ । दर्वी॒ इति॑ । श्री॒णी॒षे॒ । आ॒सनि॑ ॥ उ॒तो इति॑ । नः॒ । उदिति॑ । पु॒पू॒र्याः॒ । \textbf{  15} \newline
                  \newline
                                \textbf{ TS 4.4.4.7} \newline
                  उ॒क्थेषु॑ । श॒व॒सः॒ । प॒ते॒ । इष᳚म् । स्तो॒तृभ्य॒ इति॑ स्तो॒तृ - भ्यः॒ । एति॑ । भ॒र॒ ॥ अग्ने᳚ । तम् । अ॒द्य । अश्व᳚म् । न । स्तोमैः᳚ । क्रतु᳚म् । न । भ॒द्रम् । हृ॒दि॒स्पृश॒मिति॑ हृदि - स्पृश᳚म् ॥ ऋ॒द्ध्याम॑ । ते॒ । ओहैः᳚ ॥ अध॑ । हि । अ॒ग्ने॒ । क्रतोः᳚ । भ॒द्रस्य॑ । दक्ष॑स्य । सा॒धोः ॥ र॒थीः । ऋ॒तस्य॑ । बृ॒ह॒तः । ब॒भूथ॑ ॥ आ॒भिः । ते॒ । अ॒द्य । गी॒र्भिः । गृ॒णन्तः॑ । अग्ने᳚ । दाशे॑म ॥ प्रेति॑ । ते॒ । दि॒वः । न । स्त॒न॒य॒न्ति॒ । शुष्माः᳚ ॥ ए॒भिः । नः॒ । अ॒र्कैः । भव॑ । नः॒ । अ॒र्वाङ् । \textbf{  16} \newline
                  \newline
                                \textbf{ TS 4.4.4.8} \newline
                  सुवः॑ । न । ज्योतिः॑ ॥ अग्ने᳚ । विश्वे॑भिः । सु॒मना॒ इति॑ सु - मनाः᳚ । अनी॑कैः ॥ अ॒ग्निम् । होता॑रम् । म॒न्ये॒ । दास्व॑न्तम् । वसोः᳚ । सू॒नुम् । सह॑सः । जा॒तवे॑दस॒मिति॑ जा॒त - वे॒द॒स॒म् ॥ विप्र᳚म् । न । जा॒तवे॑दस॒मिति॑ जा॒त - वे॒द॒स॒म् ॥ यः । ऊ॒द्‌र्ध्वया᳚ । स्व॒द्ध्व॒र इति॑ सु - अ॒द्ध्व॒रः । दे॒वः । दे॒वाच्या᳚ । कृ॒पा ॥ घृ॒तस्य॑ । विभ्रा᳚ष्टि॒मिति॒ वि - भ्रा॒ष्टि॒म् । अन्विति॑ । शु॒क्रशो॑चिष॒ इति॑ शु॒क्र - शो॒चि॒षः॒ । आ॒जुह्वा॑न॒स्येत्या᳚ - जुह्वा॑नस्य । स॒र्पिषः॑ ॥ अग्ने᳚ । त्वम् । नः॒ । अन्त॑मः ॥ उ॒त । त्रा॒ता । शि॒वः । भ॒व॒ । व॒रू॒थ्यः॑ ॥ तम् । त्वा॒ । शो॒चि॒ष्ठ॒ । दी॒दि॒वः॒ ॥ सु॒म्नाय॑ । नू॒नम् । ई॒म॒हे॒ । सखि॑भ्य॒ इति॒ सखि॑ - भ्यः॒ ॥ वसुः॑ । अ॒ग्निः । वसु॑श्रवा॒ इति॒ वसु॑ - श्र॒वाः॒ ( ) ॥ अच्छ॑ । न॒क्षि॒ । द्यु॒मत्त॑म॒ इति॑ द्यु॒मत् - त॒मः॒ । र॒यिम् । दाः॒ ॥ \textbf{  17} \newline
                  \newline
                      (जना॑ना॒ - मङ्गि॑रस॒ - इषꣳ॑ - सु॒शमी॑ - तिग्मजंभ - पुपूर्या - अ॒र्वाङ् - वसु॑श्रवाः॒ - पञ्च॑ च)  \textbf{(A4)} \newline \newline
                                \textbf{ TS 4.4.5.1} \newline
                  इ॒न्द्रा॒ग्निभ्या॒मिती᳚न्द्रा॒ग्नि - भ्या॒म् । त्वा॒ । स॒युजेति॑ स - युजा᳚ । यु॒जा । यु॒न॒ज्मि॒ । आ॒घा॒राभ्या॒मित्या᳚-घा॒राभ्या᳚म् । तेज॑सा । वर्च॑सा । उ॒क्थेभिः॑ । स्तोमे॑भिः । छन्दो॑भि॒रिति॒ छन्दः॑ - भिः॒ । र॒य्यै । पोषा॑य । स॒जा॒ताना॒मिति॑ स-जा॒ताना᳚म् । म॒द्ध्य॒म॒स्थेया॒येति॑ मद्ध्यम-स्थेया॑य । मया᳚ । त्वा॒ । स॒युजेति॑ स-युजा᳚ । यु॒जा । यु॒न॒ज्मि॒ । अ॒बां । दु॒ला । नि॒त॒त्निरिति॑ नि - त॒त्निः । अ॒भ्रय॑न्ती । मे॒घय॑न्ती । व॒र्॒.षय॑न्ती । चु॒पु॒णीका᳚ । नाम॑ । अ॒सि॒ । प्र॒जाप॑ति॒नेति॑ प्र॒जा - प॒ति॒ना॒ । त्वा॒ । विश्वा॑भिः । धी॒भिः । उपेति॑ । द॒धा॒मि॒ । पृ॒थि॒वी । उ॒द॒पु॒रमित्यु॑द - पु॒रम् । अन्ने॑न । वि॒ष्टा । म॒नु॒ष्याः᳚ । ते॒ । गो॒प्तारः॑ । अ॒ग्निः । विय॑त्त॒ इति॑ वि - य॒त्तः॒ । अ॒स्या॒म् । ताम् । अ॒हम् । प्रेति॑ । प॒द्ये॒ । सा । \textbf{  18} \newline
                  \newline
                                \textbf{ TS 4.4.5.2} \newline
                  मे॒ । शर्म॑ । च॒ । वर्म॑ । च॒ । अ॒स्तु॒ । अधि॑द्यौ॒रित्यधि॑ - द्यौः॒ । अ॒न्तरि॑क्षम् । ब्रह्म॑णा । वि॒ष्टा । म॒रुतः॑ । ते॒ । गो॒प्तारः॑ । वा॒युः । विय॑त्त॒ इति॒ वि - य॒त्तः॒ । अ॒स्या॒म् । ताम् । अ॒हम् । प्रेति॑ । प॒द्ये॒ । सा । मे॒ । शर्म॑ । च॒ । वर्म॑ । च॒ । अ॒स्तु॒ । द्यौः । अप॑राजि॒तेत्यप॑रा - जि॒ता॒ । अ॒मृते॑न । वि॒ष्टा । आ॒दि॒त्याः । ते॒ । गो॒प्तारः॑ । सूर्यः॑ । विय॑त्त॒ इति॒ वि-य॒त्तः॒ । अ॒स्या॒म् । ताम् । अ॒हम् । प्रेति॑ । प॒द्ये॒ । सा । मे॒ । शर्म॑ । च॒ । वर्म॑ । च॒ । अ॒स्तु॒ ॥ \textbf{  19} \newline
                  \newline
                      (सा - ऽष्टाच॑त्वारिꣳशच्च)  \textbf{(A5)} \newline \newline
                                \textbf{ TS 4.4.6.1} \newline
                  बृह॒स्पतिः॑ । त्वा॒ । सा॒द॒य॒तु॒ । पृ॒थि॒व्याः । पृ॒ष्ठे । ज्योति॑ष्मतीम् । विश्व॑स्मै । प्रा॒णायेति॑ प्र-अ॒नाय॑ । अ॒पा॒नायेत्य॑प - अ॒नाय॑ । विश्व᳚म् । ज्योतिः॑ । य॒च्छ॒ । अ॒ग्निः । ते॒ । अधि॑पति॒रित्यधि॑ - प॒तिः॒ । वि॒श्वक॒र्मेति॑ वि॒श्व - क॒र्मा॒ । त्वा॒ । सा॒द॒य॒तु॒ । अ॒न्तरि॑क्षस्य । पृ॒ष्ठे । ज्योति॑ष्मतीम् । विश्व॑स्मै । प्रा॒णायेति॑ प्र - अ॒नाय॑ । अ॒पा॒नायेत्य॑प - अ॒नाय॑ । विश्व᳚म् । ज्योतिः॑ । य॒च्छ॒ । वा॒युः । ते॒ । अधि॑पति॒रित्यधि॑ - प॒तिः॒ । प्र॒जाप॑ति॒रिति॑ प्र॒जा - प॒तिः॒ । त्वा॒ । सा॒द॒य॒तु॒ । दि॒वः । पृ॒ष्ठे । ज्योति॑ष्मतीम् । विश्व॑स्मै । प्रा॒णायेति॑ प्र - अ॒नाय॑ । अ॒पा॒नायेत्य॑प - अ॒नाय॑ । विश्व᳚म् । ज्योतिः॑ । य॒च्छ॒ । प॒र॒मे॒ष्ठी । ते॒ । अधि॑पति॒रित्यधि॑ - प॒तिः॒ । पु॒रो॒वा॒त॒सनि॒रिति॑ पुरोवात - सनिः॑ । अ॒स्य॒ । अ॒भ्र॒सनि॒रित्य॑भ्र - सनिः॑ । अ॒सि॒ । वि॒द्यु॒थ्सनि॒रिति॑ विद्युत् - सनिः॑ । \textbf{  20} \newline
                  \newline
                                \textbf{ TS 4.4.6.2} \newline
                  अ॒सि॒ । स्त॒न॒यि॒त्नु॒सनि॒रिति॑ स्तनयित्नु - सनिः॑ । अ॒सि॒ । वृ॒ष्टि॒सनि॒रिति॑ वृष्टि - सनिः॑ । अ॒सि॒ । अ॒ग्नेः । यानी᳚ । अ॒सि॒ । दे॒वाना᳚म् । अ॒ग्ने॒यानीत्य॑ग्ने-यानी᳚ । अ॒सि॒ । वा॒योः । यानी᳚ । अ॒सि॒ । दे॒वाना᳚म् । वा॒यो॒यानीति॑ वायो - यानी᳚ । अ॒सि॒ । अ॒न्तरि॑क्षस्य । यानी᳚ । अ॒सि॒ । दे॒वाना᳚म् । अ॒न्त॒रि॒क्ष॒यानीत्य॑न्तरिक्ष - यानी᳚ । अ॒सि॒ । अ॒न्तरि॑क्षम् । अ॒सि॒ । अ॒न्तरि॑क्षाय । त्वा॒ । स॒लि॒लाय॑ । त्वा॒ । सर्णी॑काय । त्वा॒ । सती॑का॒येति॒ स - ती॒का॒य॒ । त्वा॒ । केता॑य । त्वा॒ । प्रचे॑तस॒ इति॒ प्र - चे॒त॒से॒ । त्वा॒ । विव॑स्वते । त्वा॒ । दि॒वः । त्वा॒ । ज्योति॑षे । आ॒दि॒त्येभ्यः॑ । त्वा॒ । ऋ॒चे । त्वा॒ । रु॒चे । त्वा॒ । द्यु॒ते । त्वा॒ ( ) । भा॒से । त्वा॒ । ज्योति॑षे । त्वा॒ । य॒शो॒दामिति॑ यशः-दाम् । त्वा॒ । यश॑सि । ते॒जो॒दामिति॑ तेजः - दाम् । त्वा॒ । तेज॑सि । प॒यो॒दामिति॑ पयः - दाम् । त्वा॒ ।पय॑सि । व॒र्चो॒दामिति॑ वर्चः-दाम् । त्वा॒ । वर्च॑सि । द्र॒वि॒णो॒दामिति॑ द्रविणः - दाम् । त्वा॒ । द्रवि॑णे । सा॒द॒या॒मि॒ । तेन॑ । ऋषि॑णा । तेन॑ । ब्रह्म॑णा । तया᳚ । दे॒वत॑या । अ॒ङ्गि॒र॒स्वत् । ध्रु॒वा । सी॒द॒ ॥ \textbf{  21} \newline
                  \newline
                      (वि॒द्यु॒थ्सनि॑ - र्द्यु॒ते त्वै - का॒न्न त्रिꣳ॒॒शच्च॑)  \textbf{(A6)} \newline \newline
                                \textbf{ TS 4.4.7.1} \newline
                  भू॒य॒स्कृदिति॑ भूयः - कृत् । अ॒सि॒ । व॒रि॒व॒स्कृदिति॑ वरिवः - कृत् । अ॒सि॒ । प्राची᳚ । अ॒सि॒ । ऊ॒द्‌र्ध्वा । अ॒सि॒ । अ॒न्त॒रि॒क्ष॒सदित्य॑न्तरिक्ष - सत् । अ॒सि॒ । अ॒न्तरि॑क्षे । सी॒द॒ । अ॒फ्सु॒षदित्य॑फ्सु - सत् । अ॒सि॒ । श्ये॒न॒सदिति॑ श्येन-सत् । अ॒सि॒ । गृ॒द्ध्र॒सदिति॑ गृद्ध्र-सत् । अ॒सि॒ । सु॒प॒र्ण॒सदिति॑ सुपर्ण - सत् । अ॒सि॒ । ना॒क॒सदिति॑ नाक - सत् । अ॒सि॒ । पृ॒थि॒व्याः । त्वा॒ । द्रवि॑णे । सा॒द॒या॒मि॒ । अ॒न्तरि॑क्षस्य । त्वा॒ । द्रवि॑णे । सा॒द॒या॒मि॒ । दि॒वः । त्वा॒ । द्रवि॑णे । सा॒द॒या॒मि॒ । दि॒शाम् । त्वा॒ । द्रवि॑णे । सा॒द॒या॒मि॒ । द्र॒वि॒णो॒दामिति॑ द्रविणः - दाम् । त्वा॒ । द्रवि॑णे । सा॒द॒या॒मि॒ । प्रा॒णमिति॑ प्र - अ॒नम् । मे॒ । पा॒हि॒ । आ॒पा॒नमित्य॑प - अ॒नम् । मे॒ । पा॒हि॒ । व्या॒नमिति॑ वि - अ॒नम् । मे॒ । \textbf{  22} \newline
                  \newline
                                \textbf{ TS 4.4.7.2} \newline
                  पा॒हि॒ । आयुः॑ । मे॒ । पा॒हि॒ । वि॒श्वायु॒रिति॑ वि॒श्व-आ॒युः॒ । मे॒ । पा॒हि॒ । स॒र्वायु॒रिति॑ स॒र्व - आ॒युः॒ । मे॒ । पा॒हि॒ । अग्ने᳚ । यत् । ते॒ । पर᳚म् । हृत् । नाम॑ । तौ । एति॑ । इ॒हि॒ । समिति॑ । र॒भा॒व॒है॒ । पाञ्च॑जन्ये॒ष्विति॒ पाञ्च॑ - ज॒न्ये॒षु॒ । अपीति॑ । ए॒धि॒ । अ॒ग्ने॒ । यावाः᳚ । अया॑वाः । एवाः᳚ । ऊमाः᳚ । सब्दः॑ । सग॑रः । सु॒मेक॒ इति॑ सु - मेकः॑ ॥ \textbf{  23 } \newline
                  \newline
                      (व्या॒नं मे॒-द्वात्रिꣳ॑शच्च)  \textbf{(A7)} \newline \newline
                                \textbf{ TS 4.4.8.1} \newline
                  अ॒ग्निना᳚ । वि॒श्वा॒षाट् । सूर्ये॑ण । स्व॒राडिति॑ स्व - राट् । क्रत्वा᳚ । शची॒पतिः॑ । ऋ॒ष॒भेण॑ । त्वष्टा᳚ । य॒ज्ञेन॑ । म॒घवा॒निति॑ म॒घ-वा॒न् । दक्षि॑णया । सु॒व॒र्ग इति॑ सुवः - गः । म॒न्युना᳚ । वृ॒त्र॒हेति॑ वृत्र - हा । सौहा᳚र्द्येन । त॒नू॒धा इति॑ तनू - धाः । अन्ने॑न । गयः॑ । पृ॒थि॒व्या । अ॒स॒नो॒त् । ऋ॒ग्भिरित्यृ॑क् - भिः । अ॒न्ना॒द इत्य॑न्न - अ॒दः । व॒ष॒ट्का॒रेणेति॑ वषट् - का॒रेण॑ । ऋ॒द्धः । साम्ना᳚ । त॒नू॒पा इति॑ तनू - पाः । वि॒राजेति॑ वि - राजा᳚ । ज्योति॑ष्मान् । ब्रह्म॑णा । सो॒म॒पा इति॑ सोम - पाः । गोभिः॑ । य॒ज्ञ्म् । दा॒धा॒र॒ । क्ष॒त्रेण॑ । म॒नु॒ष्यान्॑ । अश्वे॑न । च॒ । रथे॑न । च॒ । व॒ज्री । ऋ॒तुभि॒रित्यृ॒तु - भिः॒ । प्र॒भुरिति॑ प्र - भुः । सं॒ॅव॒थ्स॒रेणेति॑ सं - व॒थ्स॒रेण॑ । प॒रि॒भूरिति॑ परि - भूः । तप॑सा । अना॑धृष्ट॒ इत्यना᳚ - धृ॒ष्टः॒ । सूर्यः॑ । सन्न् । त॒नूभिः॑ ॥ \textbf{  24} \newline
                  \newline
                      (अ॒ग्नि - रैका॒न्न प॑ञ्चा॒शत्)  \textbf{(A8)} \newline \newline
                                \textbf{ TS 4.4.9.1} \newline
                  प्र॒जाप॑ति॒रिति॑ प्र॒जा - प॒तिः॒ । मन॑सा । अन्धः॑ । अच्छे॑त॒ इत्यच्छ॑ - इ॒तः॒ । धा॒ता । दी॒क्षाया᳚म् । स॒वि॒ता । भृ॒त्याम् । पू॒षा । सो॒म॒क्रय॑ण्या॒मिति॑ सोम - क्रय॑ण्याम् । वरु॑णः । उप॑नद्ध॒ इत्युप॑ - न॒द्धः॒ । असु॑रः । क्री॒यमा॑णः । मि॒त्रः । क्री॒तः । शि॒पि॒वि॒ष्ट इति॑ शिपि - वि॒ष्टः । आसा॑दित॒ इत्या - सा॒दि॒तः॒ । न॒रंधि॑षः । प्रो॒ह्यमा॑ण॒ इति॑ प्र - उ॒ह्यमा॑णः । अधि॑पति॒रित्यधि॑-प॒तिः॒ । आग॑त॒ इत्या - ग॒तः॒ । प्र॒जाप॑ति॒रिति॑ प्र॒जा - प॒तिः॒ । प्र॒णी॒यमा॑न॒ इति॑ प्र-नी॒यमा॑नः । अ॒ग्निः । आग्नी᳚द्ध्र॒ इत्याग्नि॑-इ॒द्ध्रे॒ । बृह॒स्पतिः॑ । आग्नी᳚द्ध्रा॒दित्याग्नि॑ - इ॒द्ध्रा॒त् । प्र॒णी॒यमा॑न॒ इति॑ प्र - नी॒यमा॑नः । इन्द्रः॑ । ह॒वि॒द्‌र्धान॒ इति॑ हविः - धाने᳚ । अदि॑तिः । आसा॑दित॒ इत्या - सा॒दि॒तः॒ । विष्णुः॑ । उ॒पा॒व॒ह्रि॒यमा॑ण॒ इत्यु॑प - अ॒व॒ह्रि॒यमा॑णः । अथ॑र्वा । उपो᳚त्त॒ इत्युप॑ - उ॒त्तः॒ । य॒मः । अ॒भिषु॑त॒ इत्य॒भि - सु॒तः॒ । अ॒पू॒त॒पा इत्य॑पूत - पाः । आ॒धू॒यमा॑न॒ इत्या᳚ - धू॒यमा॑नः । वा॒युः । पू॒यमा॑नः । मि॒त्रः । क्षी॒र॒श्रीरिति॑ क्षीर-श्रीः । म॒न्थी । स॒क्तु॒श्रीरिति॑ सक्तु - श्रीः । वै॒श्व॒दे॒व इति॑ वैश्व - दे॒वः । उन्नी॑त॒ इत्युत् - नी॒तः॒ । रु॒द्रः ( ) । आहु॑त॒ इत्या - हु॒तः॒ । वा॒युः । आवृ॑त्त॒ इत्या - वृ॒त्तः॒ । नृ॒चक्षा॒ इति॑ नृ - चक्षाः᳚ । प्रति॑ख्यात॒ इति॒ प्रति॑ - ख्या॒तः॒ । भ॒क्षः । आग॑त॒ इत्या - ग॒तः॒ । पि॒तृ॒णाम् । ना॒रा॒शꣳ॒॒सः । असुः॑ । आत्तः॑ । सिन्धुः॑ । अ॒व॒भृ॒थमित्य॑व - भृ॒थम् । अ॒व॒प्र॒यन्नित्य॑व - प्र॒यन्न् । स॒मु॒द्रः । अव॑गत॒ इत्यव॑ - ग॒तः॒ । स॒लि॒लः । प्रप्लु॑त॒ इति॒ प्र - प्लु॒तः॒ । सुवः॑ । उ॒दृच॒मित्यु॑त् - ऋच᳚म् । ग॒तः ॥ \textbf{  25 } \newline
                  \newline
                      (रु॒द्र - एक॑विꣳशतिश्च)  \textbf{(A9)} \newline \newline
                                \textbf{ TS 4.4.10.1} \newline
                  कृत्ति॑काः । नक्ष॑त्रम् । अ॒ग्निः । दे॒वता᳚ । अ॒ग्नेः । रुचः॑ । स्थ॒ । प्र॒जाप॑ते॒रिति॑ प्र॒जा - प॒तेः॒ । धा॒तुः । सोम॑स्य । ऋ॒चे । त्वा॒ । रु॒चे । त्वा॒ । द्यु॒ते । त्वा॒ । भा॒से । त्वा॒ । ज्योति॑षे । त्वा॒ । रो॒हि॒णी । नक्ष॑त्रम् । प्र॒जाप॑ति॒रिति॑ प्र॒जा - प॒तिः॒ । दे॒वता᳚ । मृ॒ग॒शी॒र्॒.षमिति॑ मृग - शी॒र्॒.षम् । नक्ष॑त्रम् । सोमः॑ । दे॒वता᳚ । आ॒र्द्रा । नक्ष॑त्रम् । रु॒द्रः । दे॒वता᳚ । पुन॑र्वसू॒ इति॒ पुनः॑ - व॒सू॒ । नक्ष॑त्रम् । अदि॑तिः । दे॒वता᳚ । ति॒ष्यः॑ । नक्ष॑त्रम् । बृह॒स्पतिः॑ । दे॒वता᳚ । आ॒श्रे॒षा इत्या᳚- श्रे॒षाः । नक्ष॑त्रम् । स॒र्पाः । दे॒वता᳚ । म॒घाः । नक्ष॑त्रम् । पि॒तरः॑ । दे॒वता᳚ । फल्गु॑नी॒ इति॑ । नक्ष॑त्रम् । \textbf{  26} \newline
                  \newline
                                \textbf{ TS 4.4.10.2} \newline
                  अ॒र्य॒मा । दे॒वता᳚ । फल्गु॑नी॒ इति॑ । नक्ष॑त्रम् । भगः॑ । दे॒वता᳚ । हस्तः॑ । नक्ष॑त्रम् । स॒वि॒ता । दे॒वता᳚ । चि॒त्रा । नक्ष॑त्रम् । इन्द्रः॑ । दे॒वता᳚ । स्वा॒ती । नक्ष॑त्रम् । वा॒युः । दे॒वता᳚ । विशा॑खे॒ इति॒ वि - शा॒खे॒ । नक्ष॑त्रम् । इ॑006छ्;॒द्रा॒ग्नी इती᳚न्द्र - अ॒ग्नी । दे॒वता॑ । अ॒नू॒रा॒धा इत्य॑नु - रा॒धाः । नक्ष॑त्रम् । मि॒त्रः । दे॒वता᳚ । रो॒हि॒णी । नक्ष॑त्रम् । इन्द्रः॑ । दे॒वता᳚ । वि॒चृता॒विति॑ वि - चृतौ᳚ । नक्ष॑त्रम् । पि॒तरः॑ । दे॒वता᳚ । अ॒षा॒ढाः । नक्ष॑त्रम् । आपः॑ । दे॒वता᳚ । अ॒षा॒ढाः । नक्ष॑त्रम् । विश्वे᳚ । दे॒वाः । दे॒वता᳚ । श्रो॒णा । नक्ष॑त्रम् । विष्णुः॑ । दे॒वता᳚ । श्रवि॑ष्ठाः । नक्ष॑त्रम् । वस॑वः । \textbf{  27} \newline
                  \newline
                                \textbf{ TS 4.4.10.3} \newline
                  दे॒वता᳚ । श॒तभि॑ष॒गिति॑ श॒त - भि॒ष॒क् । नक्ष॑त्रम् । इन्द्रः॑ । दे॒वता᳚ । प्रो॒ष्ठ॒प॒दा इति॑ प्रोष्ठ - प॒दाः । नक्ष॑त्रम् । अ॒जः । एक॑पा॒दित्येक॑ - पा॒त् । दे॒वता᳚ । प्रो॒ष्ठ॒प॒दा इति॑ प्रोष्ठ - प॒दाः । नक्ष॑त्रम् । अहिः॑ । बु॒द्ध्नियः॑ । दे॒वता᳚ । रे॒वती᳚ । नक्ष॑त्रम् । पू॒षा । दे॒वता᳚ । अ॒श्व॒युजा॒वित्य॑श्व - युजौ᳚ । नक्ष॑त्रम् । अ॒श्विनौ᳚ । दे॒वता᳚ । अ॒प॒भर॑णी॒रित्य॑प - भर॑णीः । नक्ष॑त्रम् । य॒मः । दे॒वता᳚ । पू॒र्णा । प॒श्चात् । यत् । ते॒ । दे॒वाः । अद॑धुः ॥ \textbf{  28} \newline
                  \newline
                      (फल्गु॑नी॒ नक्ष॑त्रं॒ - ॅवस॑व॒ - स्त्रय॑स्त्रिꣳशच्च)  \textbf{(A10)} \newline \newline
                                \textbf{ TS 4.4.11.1} \newline
                  मधुः॑ । च॒ । माध॑वः । च॒ । वास॑न्तिकौ । ऋ॒तू इति॑ । शु॒क्रः । च॒ । शुचिः॑ । च॒ । ग्रैष्मौ᳚ । ऋ॒तू इति॑ । नभः॑ । च॒ । न॒भ॒स्यः॑ । च॒ । वार्.षि॑कौ । ऋ॒तू इति॑ । इ॒षः । च॒ । ऊ॒र्जः । च॒ । शा॒र॒दौ । ऋ॒तू इति॑ । सहः॑ । च॒ । स॒ह॒स्यः॑ । च॒ । हैम॑न्तिकौ । ऋ॒तू इति॑ । तपः॑ । च॒ । त॒प॒स्यः॑ । च॒ । शै॒शि॒रौ । ऋ॒तू इति॑ । अ॒ग्नेः । अ॒न्तः॒श्ले॒ष इत्य॑न्तः - श्ले॒षः । अ॒सि॒ । कल्पे॑ताम् । द्यावा॑पृथि॒वी इति॒ द्यावा᳚ - पृ॒थि॒वी । कल्प॑न्ताम् । आपः॑ । ओष॑धीः । कल्प॑न्ताम् । अ॒ग्नयः॑ । पृथ॑क् । मम॑ । ज्यैष्ठ्य॑य । सव्र॑ता॒ इति॒ स-व्र॒ताः॒ । \textbf{  29} \newline
                  \newline
                                \textbf{ TS 4.4.11.2} \newline
                  ये । अ॒ग्नयः॑ । सम॑नस॒ इति॒ स - म॒न॒सः॒ । अ॒न्त॒रा । द्यावा॑पृथि॒वी इति॒ द्यावा᳚ - पृ॒थि॒वी । शै॒शि॒रौ । ऋ॒तू इति॑ । अ॒भीति॑ । कल्प॑मानाः । इन्द्र᳚म् । इ॒व॒ । दे॒वाः । अ॒भि । समिति॑ । वि॒श॒न्तु॒ । सं॒ॅयदिति॑ सं - यत् । च॒ । प्रचे॑ता॒ इति॒ प्र - चे॒ताः॒ । च॒ । अ॒ग्नेः । सोम॑स्य । सूर्य॑स्य । उ॒ग्रा । च॒ । भी॒मा । च॒ । पि॒तृ॒णाम् । य॒मस्य॑ । इन्द्र॑स्य । ध्रु॒वा । च॒ । पृ॒थि॒वी । च॒ । दे॒वस्य॑ । स॒वि॒तुः । म॒रुता᳚म् । वरु॑णस्य । ध॒र्त्री । च॒ । धरि॑त्री । च॒ । मि॒त्रावरु॑णयो॒रिति॑ मि॒त्रा - वरु॑णयोः । मि॒त्रस्य॑ । धा॒तुः । प्राची᳚ । च॒ । प्र॒तीची᳚ । च॒ । वसू॑नाम् । रु॒द्राणां᳚ । \textbf{  30} \newline
                  \newline
                                \textbf{ TS 4.4.11.3} \newline
                  आ॒दि॒त्याना᳚म् । ते । ते॒ । अधि॑पतय॒ इत्यधि॑-प॒त॒यः॒ । तेभ्यः॑ । नमः॑ । ते । नः॒ । मृ॒ड॒य॒न्तु॒ । ते । यम् । द्वि॒ष्मः । यः । च॒ । नः॒ । द्वेष्टि॑ । तम् । वः॒ । जंभे᳚ । द॒धा॒मि॒ । स॒हस्र॑स्य । प्र॒मेति॑ प्र - मा । अ॒सि॒ । स॒हस्र॑स्य । प्र॒ति॒मेति॑ प्रति - मा । अ॒सि॒ । स॒हस्र॑स्य । वि॒मेति॑ वि - मा । अ॒सि॒ । स॒हस्र॑स्य । उ॒न्मेत्यु॑त् - मा । अ॒सि॒ । सा॒ह॒स्रः । अ॒सि॒ । स॒हस्रा॑य । त्वा॒ । इ॒माः । मे॒ । अ॒ग्ने॒ । इष्ट॑काः । धे॒नवः॑ । स॒न्तु॒ । एका᳚ । च॒ । श॒तम् । च॒ । स॒हस्र᳚म् । च॒ । अ॒युत᳚म् । च॒ । \textbf{  31} \newline
                  \newline
                                \textbf{ TS 4.4.11.4} \newline
                  नि॒युत॒मिति॑ नि - युत᳚म् । च॒ । प्र॒युत॒मिति॑ प्र-युत᳚म् । च॒ । अर्बु॑दम् । च॒ । न्य॑र्बुद॒मिति॒ नि - अ॒र्बु॒द॒म् । च॒ । स॒मु॒द्रः । च॒ । मद्ध्य᳚म् । च॒ । अन्तः॑ । च॒ । प॒रा॒द्‌र्ध इति॑ पर- अ॒द्‌र्धः । च॒ । इ॒माः । मे॒ । अ॒ग्ने॒ । इष्ट॑काः । धे॒नवः॑ । स॒न्तु॒ । ष॒ष्टिः । स॒हस्र᳚म् । अ॒युत᳚म् । अक्षी॑यमाणाः । ऋ॒त॒स्था इत्यृ॑त - स्थाः । स्थ॒ । ऋ॒ता॒वृध॒ इत्यृ॑त - वृधः॑ । घृ॒त॒श्चुत॒ इति॑ घृत - श्चुतः॑ । म॒धु॒श्चुत॒ इति॑ मधु - श्चुतः॑ । ऊर्ज॑स्वतीः । स्व॒धा॒विनी॒रिति॑ स्वधा - विनीः᳚ । ताः । मे॒ । अ॒ग्ने॒ । इष्ट॑काः । धे॒नवः॑ । स॒न्तु॒ । वि॒राज॒ इति॑ वि - राजः॑ । नाम॑ । का॒म॒दुघा॒ इति॑ काम - दुघाः᳚ । अ॒मुत्र॑ । अ॒मुष्मिन्न्॑ । लो॒के ॥ \textbf{  32} \newline
                  \newline
                      (सव्र॑ता - रु॒द्राणा॑ - म॒युत॑ञ्च॒ - पञ्च॑चत्वारिꣳशच्च)  \textbf{(A11)} \newline \newline
                                \textbf{ TS 4.4.12.1} \newline
                  स॒मिदिति॑ सम् - इत् । दि॒शाम् । आ॒शया᳚ । नः॒ । सु॒व॒र्विदिति॑ सुवः - वित् । मधोः᳚ । अतः॑ । माध॑वः । पा॒तु॒ । अ॒स्मान् ॥ अ॒ग्निः । दे॒वः । दु॒ष्टरी॑तुः । अदा᳚भ्यः । इ॒दम् । क्ष॒त्रम् । र॒क्ष॒तु॒ । पातु॑ । अ॒स्मान् ॥ र॒थ॒न्त॒रमिति॑ रथं - त॒रम् । साम॑भि॒रिति॒ साम॑ - भिः॒ । पा॒तु॒ । अ॒स्मान् । गा॒य॒त्री । छन्द॑साम् । वि॒श्वरू॒पेति॑ वि॒श्व - रू॒पा॒ ॥ त्रि॒वृदिति॑ त्रि-वृत् । नः॒ । वि॒ष्ठयेति॑ वि - स्थया᳚ । स्तोमः॑ । अह्ना᳚म् । स॒मु॒द्रः । वातः॑ । इ॒दम् । ओजः॑ । पि॒प॒र्तु॒ ॥ उ॒ग्रा । दि॒शाम् । अ॒भिभू॑ति॒रित्य॒भि - भू॒तिः॒ । व॒यो॒धा इति॑ वयः- धाः । शुचिः॑ । शु॒क्रे । अह॑नि । ओ॒ज॒सीना᳚ ॥ इन्द्र॑ । अधि॑पति॒रित्यधि॑ - प॒तिः॒ । पि॒पृ॒ता॒त् । अतः॑ । नः॒ । महि॑ । \textbf{  33} \newline
                  \newline
                                \textbf{ TS 4.4.12.2} \newline
                  क्ष॒त्रम् । वि॒श्वतः॑ । धा॒र॒य॒ । इ॒दम् ॥ बृ॒हत् । साम॑ । क्ष॒त्र॒भृदिति॑ क्षत्र - भृत् । वृ॒द्धवृ॑ष्णिय॒मिति॑ वृ॒द्ध-वृ॒ष्णि॒य॒म् । त्रि॒ष्टुभा᳚ । ओजः॑ । शु॒भि॒तम् । उ॒ग्रवी॑र॒मित्यु॒ग्र - वी॒र॒म् ॥ इन्द्र॑ । स्तोमे॑न । प॒ञ्च॒द॒शेनेति॑ पञ्च - द॒शेन॑ । मद्ध्य᳚म् । इ॒दम् । वाते॑न । सग॑रेण । र॒क्ष॒ ॥ प्राची᳚ । दि॒शाम् । स॒हय॑शा॒ इति॑ स॒ह - य॒शाः॒ । यश॑स्वती । विश्वे᳚ । दे॒वाः॒ । प्रा॒वृषा᳚ । अह्ना᳚म् । सुव॑र्व॒तीति॒ सुवः॑ - व॒ती॒ ॥ इ॒दम् । क्ष॒त्रम् । दु॒ष्टर᳚म् । अ॒स्तु॒ । ओजः॑ । अना॑धृष्ट॒मित्यना᳚ - धृ॒ष्ट॒म् । स॒ह॒स्रिय᳚म् । सह॑स्वत् ॥ वै॒रू॒पे । सामन्न्॑ । इ॒ह । तत् । श॒के॒म॒ । जग॑त्या । ए॒न॒म् । वि॒क्षु । एति॑ । वे॒श॒या॒मः॒ ॥ विश्वे᳚ । दे॒वाः॒ । स॒प्त॒द॒शेनेति॑ सप्त - द॒शेन॑ । \textbf{  34} \newline
                  \newline
                                \textbf{ TS 4.4.12.3} \newline
                  वर्चः॑ । इ॒दम् । क्ष॒त्रम् । स॒लि॒लवा॑त॒मिति॑ सलि॒ल - वा॒त॒म् । उ॒ग्रम् ॥ ध॒र्त्री । दि॒शाम् । क्ष॒त्रम् । इ॒दम् । दा॒धा॒र॒ । उ॒प॒स्थेत्यु॑प - स्था । आशा॑नाम् । मि॒त्रव॒दिति॑ मि॒त्र - व॒त् । अ॒स्तु॒ । ओजः॑ ॥ मित्रा॑वरु॒णेति॒ मित्रा᳚ - व॒रु॒णा॒ । श॒रदा᳚ । अह्ना᳚म् । चि॒कि॒त्नू॒ इति॑ । अ॒स्मै । रा॒ष्ट्राय॑ । महि॑ । शर्म॑ । य॒च्छ॒त॒म् ॥ वै॒रा॒जे । सामन्न्॑ । अधीति॑ । मे॒ । म॒नी॒षा । अ॒नु॒ष्टुभेत्य॑नु - स्तुभा᳚ । संभृ॑त॒मिति॑ सं - भृ॒त॒म् । वी॒र्य᳚म् । सहः॑ ॥ इ॒दम् । क्ष॒त्रम् । मि॒त्रव॒दिति॑ मि॒त्र - व॒त् । आ॒र्द्रदा॒न्वित्या॒र्द्र - दा॒नु॒ । मित्रा॑वरु॒णेति॒ मित्रा᳚ - व॒रु॒णा॒ । रक्ष॑तम् । आधि॑पत्यै॒रित्याधि॑-प॒त्यैः॒ ॥ स॒म्राडिति॑ सम् - राट् । दि॒शाम् । स॒हसा॒म्नीति॑ स॒ह - सा॒म्नी॒ । सह॑स्वती । ऋ॒तुः । हे॒म॒न्तः । वि॒ष्ठयेति॑ वि - स्थया᳚ । नः॒ । पि॒प॒र्तु॒ ॥ अ॒व॒स्युवा॑ता॒ इत्य॑व॒स्यु - वा॒ताः॒ । \textbf{  35} \newline
                  \newline
                                \textbf{ TS 4.4.12.4} \newline
                  बृ॒ह॒तीः । नु । शक्व॑रीः । इ॒मम् । य॒ज्ञ्म् । अ॒व॒न्तु॒ । नः॒ । घृ॒ताचीः᳚ । सुव॑र्व॒तीति॒ सुवः॑ - व॒ती॒ । सु॒दुघेति॑ सु - दुघा᳚ । नः॒ । पय॑स्वती । दि॒शाम् । दे॒वी । अ॒व॒तु॒ । नः॒ । घृ॒ताची᳚ ॥ त्वम् । गो॒पा इति॑ गो - पाः । पु॒र॒ ए॒तेति॑ पुरः - ए॒ता । उ॒त । प॒श्चात् । बृह॑स्पते । याम्या᳚म् । यु॒ङ्ग्धि॒ । वाच᳚म् ॥ ऊ॒द्‌र्ध्वा । दि॒शाम् । रन्तिः॑ । आशा᳚ । ओष॑धीनाम् । सं॒ॅव॒थ्स॒रेणेति॑ सं-व॒थ्स॒रेण॑ । स॒वि॒ता । नः॒ । अह्ना᳚म् ॥ रे॒वत् । साम॑ । अति॑॑च्छन्दा॒ इत्याति॑ - छ॒न्दाः॒ । उ॒ । छन्दः॑ । अजा॑तशत्रु॒रित्यजा॑त - श॒त्रुः॒ । स्यो॒ना । नः॒ । अ॒स्तु॒ ॥ स्तोम॑त्रयस्त्रिꣳश॒ इति॒ स्तोम॑ - त्र॒य॒स्त्रिꣳ॒॒शे॒ । भुव॑नस्य । प॒त्नि॒ । विव॑स्वद्वात॒ इति॒ विव॑स्वत् - वा॒ते॒ । अ॒भीति॑ । नः॒ । \textbf{  36} \newline
                  \newline
                                \textbf{ TS 4.4.12.5} \newline
                  गृ॒णा॒हि॒ ॥ घृ॒तव॒तीति॑ घृ॒त - व॒ती॒ । स॒वि॒तः॒ । आधि॑पत्यै॒रित्याधि॑-प॒त्यैः॒ । पय॑स्वती । रन्तिः॑ । आशा᳚ । नः॒ । अ॒स्तु॒ ॥ ध्रु॒वा । दि॒शाम् । विष्णु॑प॒त्नीति॒ विष्णु॑ - प॒त्नी॒ । अघो॑रा । अ॒स्य । ईशा॑ना । सह॑सः । या । म॒नोता᳚ ॥ बृह॒स्पतिः॑ । मा॒त॒रिश्वा᳚ । उ॒त । वा॒युः । स॒न्ध॒वा॒ना इति॑ सम् - धु॒वा॒नाः । वाताः᳚ । अ॒भीति॑ । नः॒ । गृ॒ण॒न्तु॒ ॥ वि॒ष्ट॒भं इति॑ वि-स्त॒भंः । दि॒वः । ध॒रुणः॑ । पृ॒थि॒व्याः । अ॒स्य । ईशा॑ना । जग॑तः । विष्णु॑प॒त्नीति॒ विष्णु॑ - प॒त्नी॒ ॥ वि॒श्वव्य॑चा॒ इति॑ वि॒श्व - व्य॒चाः॒ । इ॒षय॑न्ती । सुभू॑ति॒रिति॒ सु-भू॒तिः॒ । शि॒वा । नः॒ । अ॒स्तु॒ । अदि॑तिः । उ॒पस्थ॒ इत्यु॒प - स्थे॒ ॥ वै॒श्वा॒न॒रः । नः॒ । ऊ॒त्या । पृ॒ष्टः । दि॒वि । अन्विति॑ । नः॒ ( ) । अ॒द्य । अनु॑मति॒रित्यनु॑ - म॒तिः॒ । अन्विति॑ । इत् । अ॒नु॒म॒त॒ इत्य॑नु - म॒ते॒ । त्वम् । कया᳚ । नः॒ । चि॒त्रः । एति॑ । भु॒व॒त् । कः । अ॒द्य । यु॒ङ्क्ते॒ ॥ \textbf{  37} \newline
                  \newline
                      (महि॑ - सप्तद॒शेना॑ - ऽव॒स्युवा॑ता - अ॒भि नो - ऽनु॑ न॒ - श्चतु॑र्दश च)  \textbf{(A12)} \newline \newline
\textbf{praSna korvai with starting padams of 1 to 12 anuvAkams :-} \newline
(र॒श्मिर॑सि॒ - राज्ञ्य॑स्य॒ - यं पु॒रो हरि॑केशो॒ - ऽग्निर्मू॒र्द्ध - न्द्रा॒ग्निभ्यां॒ - बृह॒स्पति॑ - र्भूय॒स्कृद॑ - स्य॒ग्निना॑ विश्वा॒षाट् - प्र॒जाप॑ति॒र्मन॑सा॒ - कृत्ति॑का॒ - मधु॑श्च - स॒मिद्दि॒शां - द्वाद॑श ) \newline

\textbf{korvai with starting padams of1, 11, 21 series of pa~jcAtis :-} \newline
(र॒श्मिर॑सि॒ - प्रति॑ धे॒नु- म॑सि स्तनयित्नु॒सनि॑र - स्यादि॒त्यानाꣳ॑ - स॒प्तत्रिꣳ॑शत् ) \newline

\textbf{first and last padam of fourth praSnam of kANDam 4 :-} \newline
(र॒श्मिर॑सि॒ - को अ॒द्य यु॑ङ्क्ते) \newline 


॥ हरिः॑ ॐ ॥॥ कृष्ण यजुर्वेदीय तैत्तिरीय संहितायां चतुर्थ काण्डे चतुर्थः प्रश्नः समाप्तः ॥ \newline
\pagebreak
4.4.1   appendix\\4.4.10.3 - पू॒र्णा प॒श्चाद्>1 यत् ते॑ दे॒वा अद॑धुः >2\\पू॒र्णा प॒श्चादु॒त पू॒र्णा पु॒रस्ता॒दुन्म॑ध्य॒तः पौ᳚र्णमा॒सी जि॑गाय । \\तस्या᳚न् दे॒वा अधि॑ सं॒ॅवस॑न्त उत्त॒मे नाक॑ इ॒ह मा॑दयन्तां ॥\\\\यत्ते॑ दे॒वा अद॑धु र्भाग॒धेय॒ममा॑वास्ये सं॒ॅवस॑न्तो महि॒त्वा । \\सानो॑ य॒ज्ञ्ं पि॑पृहि विश्ववारे र॒यिं नो॑ धेहि सुभगे सु॒वीरं᳚ ॥\\(appearing in TS 3.5.1.1)\\\\4.4.12.5 - वै॒श्वा॒न॒रो न॑ ऊ॒त्या>3, पृ॒ष्टो दि॒व्य>4\\वै॒श्वा॒न॒रो न॑ ऊ॒त्याऽऽ प्र या॑तु परा॒वतः॑ । अ॒ग्निरु॒क्थेन॒ वाह॑सा ॥\\\\पृ॒ष्टो दि॒वि पृ॒ष्टो अ॒ग्निः पृ॑थि॒व्यां पृ॒ष्टो विश्वा॒ ओष॑धी॒रा वि॑वेश । \\वै॒श्वा॒न॒रः सह॑सा पृ॒ष्टो अ॒ग्निः सनो॒ दिवा॒ स रि॒षः पा॑तु॒ नक्तं᳚ ॥\\(appearing in TS 1.5.11.1)\\\\4.4.12.5 - नु॑ नो॒ द्यानु॑मति॒>5, रन्विद॑नुमते॒ त्वं >6\\अनु॑ नो॒ऽद्याऽनु॑मतिर्य॒ज्ञ्ं दे॒वेषु॑ \\मन्यतां । अ॒ग्निश्च॑ हव्य॒वाह॑नो॒ भव॑तां दा॒शुषे॒ मयः॑ ॥ \\\\अन्विद॑नुमते॒ त्वं मन्या॑सै॒ शञ्च॑नः कृधि । \\क्रत्वे॒ दक्षा॑य नो हिनु॒ प्रण॒ आयूꣳ॑षि तारिषः ॥ \\(appearing in TS 3.3.11.4)\\\\4.4.12.5 - कया॑ नश्चि॒त्र आभु॑व॒त्>7, को अ॒द्य यु॑ङ्क्त >8\\कया॑ नश्चि॒त्र आभु॑वदू॒ती स॒दा वृ॑धः॒ सखा᳚ । कया॒ शचि॑ष्ठया वृ॒ता ॥ \\\\को अ॒द्य यु॑ङ्क्ते धु॒रि गा ऋ॒तस्य॒ शिमी॑वतो भा॒मिनो॑ दुर्.हृणा॒यून् । \\आ॒सन्नि॑षून्. हृ॒थ्स्वसो॑ मयो॒भून्. य ए॑षां भृ॒त्यामृ॒णध॒थ् स जी॑वात् ॥ \\(appearing in TS 4.2.11.3)\\===========================\\
\pagebreak
        


\end{document}
