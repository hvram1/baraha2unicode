\documentclass[17pt]{extarticle}
\usepackage{babel}
\usepackage{fontspec}
\usepackage{polyglossia}
\usepackage{extsizes}



\setmainlanguage{sanskrit}
\setotherlanguages{english} %% or other languages
\setlength{\parindent}{0pt}
\pagestyle{myheadings}
\newfontfamily\devanagarifont[Script=Devanagari]{AdishilaVedic}


\newcommand{\VAR}[1]{}
\newcommand{\BLOCK}[1]{}




\begin{document}
\begin{titlepage}
    \begin{center}
 
\begin{sanskrit}
    { \Large
    ॐ नमः परमात्मने, श्री महागणपतये नमः, श्री गुरुभ्यो नमः
ह॒रिः॒ ॐ 
    }
    \\
    \vspace{2.5cm}
    \mbox{ \Huge
    7.4      सप्तमकाण्डे चतुर्थः प्रश्नः - सत्रकर्मनिरूपणं   }
\end{sanskrit}
\end{center}

\end{titlepage}
\tableofcontents

ॐ नमः परमात्मने, श्री महागणपतये नमः, श्री गुरुभ्यो नमः
ह॒रिः॒ ॐ \newline
7.4      सप्तमकाण्डे चतुर्थः प्रश्नः - सत्रकर्मनिरूपणं \newline

\addcontentsline{toc}{section}{ 7.4      सप्तमकाण्डे चतुर्थः प्रश्नः - सत्रकर्मनिरूपणं}
\markright{ 7.4      सप्तमकाण्डे चतुर्थः प्रश्नः - सत्रकर्मनिरूपणं \hfill https://www.vedavms.in \hfill}
\section*{ 7.4      सप्तमकाण्डे चतुर्थः प्रश्नः - सत्रकर्मनिरूपणं }
                                \textbf{ TS 7.4.1.1} \newline
                  बृह॒स्पतिः॑ । अ॒का॒म॒य॒त॒ । श्रत् । मे॒ । दे॒वाः । दधी॑रन्न् । गच्छे॑यम् । पु॒रो॒धामिति॑ पुरः - धाम् । इति॑ । सः । ए॒तम् । च॒तु॒र्विꣳ॒॒श॒ति॒रा॒त्रमिति॑ चतुर्विꣳशति - रा॒त्रम् । अ॒प॒श्य॒त् । तम् । एति॑ । अ॒ह॒र॒त् । तेन॑ । अ॒य॒ज॒त॒ । ततः॑ । वै । तस्मै᳚ । श्रत् । दे॒वाः । अद॑धत । अग॑च्छत् । पु॒रो॒धामिति॑ पुरः - धाम् । ये । ए॒वम् । वि॒द्वाꣳसः॑ । च॒तु॒र्विꣳ॒॒श॒ति॒रा॒त्रमिति॑ चतुर्विꣳशति - रा॒त्रम् । आस॑ते । श्रत् । ए॒भ्यः॒ । म॒नु॒ष्याः᳚ । द॒ध॒ते॒ । गच्छ॑न्ति । पु॒रो॒धामिति॑ पुरः-धाम् । ज्योतिः॑ । गौः । आयुः॑ । इति॑ । त्र्य॒हा इति॑ त्रि - अ॒हाः । भ॒व॒न्ति॒ । इ॒यम् । वाव । ज्योतिः॑ । अ॒न्तरि॑क्षम् । गौः । अ॒सौ । आयुः॑ । \textbf{  1} \newline
                  \newline
                                \textbf{ TS 7.4.1.2} \newline
                  इ॒मान् । ए॒व । लो॒कान् । अ॒भ्यारो॑ह॒न्तीत्य॑भि - आरो॑हन्ति । अ॒भि॒पू॒र्वमित्य॑भि - पू॒र्वम् । त्र्य॒हा इति॑ त्रि - अ॒हाः । भ॒व॒न्ति॒ । अ॒भि॒पू॒र्वमित्य॑भि - पू॒र्वम् । ए॒व । सु॒व॒र्गमिति॑ सुवः - गम् । लो॒कम् । अ॒भ्यारो॑ह॒न्तीत्य॑भि - आरो॑हन्ति । अस॑त्रम् । वै । ए॒तत् । यत् । अ॒छ॒न्दो॒ममित्य॑छन्दः-मम् । यत् । छ॒न्दो॒मा इति॑ छन्दः - माः । भव॑न्ति । तेन॑ । स॒त्रम् । दे॒वताः᳚ । ए॒व । पृ॒ष्ठैः । अवेति॑ । रु॒न्ध॒ते॒ । प॒शून् । छ॒न्दो॒मैरिति॑ छन्दः-मैः । ओजः॑ । वै । वी॒र्य᳚म् । पृ॒ष्ठानि॑ । प॒शवः॑ । छ॒न्दो॒मा इति॑ छन्दः - माः । ओज॑सि । ए॒व । वी॒र्ये᳚ । प॒शुषु॑ । प्रतीति॑ । ति॒ष्ठ॒न्ति॒ । बृ॒ह॒द्र॒थ॒न्त॒राभ्या॒मिति॑ बृहत् - र॒थ॒न्त॒राभ्या᳚म् । य॒न्ति॒ । इ॒यम् । वाव । र॒थ॒न्त॒रमिति॑ रथं - त॒रम् । अ॒सौ । बृ॒हत् । आ॒भ्याम् । ए॒व । \textbf{  2} \newline
                  \newline
                                \textbf{ TS 7.4.1.3} \newline
                  य॒न्ति॒ । अथो॒ इति॑ । अ॒नयोः᳚ । ए॒व । प्रतीति॑ । ति॒ष्ठ॒न्ति॒ । ए॒ते इति॑ । वै । य॒ज्ञ्स्य॑ । अ॒ञ्ज॒साय॑नी॒ इत्य॑ञ्जसा - अय॑नी । स्रु॒ती इति॑ । ताभ्या᳚म् । ए॒व । सु॒व॒र्गमिति॑ सुवः - गम् । लो॒कम् । य॒न्ति॒ । च॒तु॒र्विꣳ॒॒श॒ति॒रा॒त्र इति॑ चतुर्विꣳशति - रा॒त्रः । भ॒व॒ति॒ । चतु॑र्विꣳशति॒रिति॒ चतुः॑ - विꣳ॒॒श॒तिः॒ । अ॒द्‌र्ध॒मा॒सा इत्य॑द्‌र्ध - मा॒साः । सं॒ॅव॒थ्स॒र इति॑ सं - व॒थ्स॒रः । सं॒ॅव॒थ्स॒र इति॑ सं - व॒थ्स॒रः । सु॒व॒र्ग इति॑ सुवः - गः । लो॒कः । सं॒ॅव॒थ्स॒र इति॑ सं - व॒थ्स॒रे । ए॒व । सु॒व॒र्ग इति॑ सुवः - गे । लो॒के । प्रतीति॑ । ति॒ष्ठ॒न्ति॒ । अथो॒ इति॑ । चतु॑र्विꣳशत्यक्ष॒रेति॒ चतु॑र्विꣳशति - अ॒क्ष॒रा॒ । गा॒य॒त्री । गा॒य॒त्री । ब्र॒ह्म॒व॒र्च॒समिति॑ ब्रह्म - व॒र्च॒सम् । गा॒य॒त्रि॒या । ए॒व । ब्र॒ह्म॒व॒र्च॒समिति॑ ब्रह्म - व॒र्च॒सम् । अवेति॑ । रु॒न्ध॒ते॒ । अ॒ति॒रा॒त्रावित्य॑ति - रा॒त्रौ । अ॒भितः॑ । भ॒व॒तः॒ । ब्र॒ह्म॒व॒र्च॒सस्येति॑ ब्रह्म - व॒र्च॒सस्य॑ । परि॑गृहीत्या॒ इति॒ परि॑ - गृ॒ही॒त्यै॒ ॥ \textbf{  3} \newline
                  \newline
                      (अ॒सावायु॑ - रा॒भ्यामे॒व - पञ्च॑चत्वारिꣳशच्च)  \textbf{(A1)} \newline \newline
                                \textbf{ TS 7.4.2.1} \newline
                  यथा᳚ । वै । म॒नु॒ष्याः᳚ । ए॒वम् । दे॒वाः । अग्रे᳚ । आ॒स॒न्न् । ते । अ॒का॒म॒य॒न्त॒ । अव॑र्तिम् । पा॒प्मान᳚म् । मृ॒त्युम् । अ॒प॒हत्येत्य॑प-हत्य॑ । दैवी᳚म् । सꣳ॒॒सद॒मिति॑ सं - सद᳚म् । ग॒च्छे॒म॒ । इति॑ । ते । ए॒तम् । च॒तु॒र्विꣳ॒॒श॒ति॒रा॒त्रमिति॑ चतुर्विꣳशति - रा॒त्रम् । अ॒प॒श्य॒न्न् । तम् । एति॑ । अ॒ह॒र॒न्न् । तेन॑ । अ॒य॒ज॒न्त॒ । ततः॑ । वै । ते । अव॑र्तिम् । पा॒प्मान᳚म् । मृ॒त्युम् । अ॒प॒हत्येत्य॑प- हत्य॑ । दैवी᳚म् । सꣳ॒॒सद॒मिति॑ सं - सद᳚म् । अ॒ग॒च्छ॒न्न् । ये । ए॒वम् । वि॒द्वाꣳसः॑ । च॒तु॒र्विꣳ॒॒श॒ति॒रा॒त्रमिति॑ चतुर्विꣳशति - रा॒त्रम् । आस॑ते । अव॑र्तिम् । ए॒व । पा॒प्मान᳚म् । अ॒प॒हत्येत्य॑प- हत्य॑ । श्रिय᳚म् । ग॒च्छ॒न्ति॒ । श्रीः । हि । म॒नु॒ष्य॑स्य । \textbf{  4} \newline
                  \newline
                                \textbf{ TS 7.4.2.2} \newline
                  दैवी᳚ । सꣳ॒॒सदिति॑ सं - सत् । ज्योतिः॑ । अ॒ति॒रा॒त्र इत्य॑ति - रा॒त्रः । भ॒व॒ति॒ । सु॒व॒र्गस्येति॑ सुवः - गस्य॑ । लो॒कस्य॑ । अनु॑ख्यात्या॒ इत्यनु॑ -ख्या॒त्यै॒ । पृष्ठ्यः॑ । ष॒ड॒ह इति॑ षट् - अ॒हः । भ॒व॒ति॒ । षट् । वै । ऋ॒तवः॑ । सं॒ॅव॒थ्स॒र इति॑ सं - व॒थ्स॒रः । तम् । मासाः᳚ । अ॒द्‌र्ध॒मा॒सा इत्य॑द्‌र्ध - मा॒साः । ऋ॒तवः॑ । प्र॒विश्येति॑ प्र - विश्य॑ । दैवी᳚म् । सꣳ॒॒सद॒मिति॑ सं - सद᳚म् । अ॒ग॒च्छ॒न्न् । ये । ए॒वम् । वि॒द्वाꣳसः॑ । च॒तु॒र्विꣳ॒॒श॒ति॒रा॒त्रमिति॑ चतुर्विꣳशति - रा॒त्रम् । आस॑ते । सं॒ॅव॒थ्स॒रमिति॑ सं - व॒थ्स॒रम् । ए॒व । प्र॒विश्येति॑ प्र - विश्य॑ । वस्य॑सीम् । सꣳ॒॒सद॒मिति॑ सं - सद᳚म् । ग॒च्छ॒न्ति॒ । त्रयः॑ । त्र॒य॒स्त्रिꣳ॒॒शा इति॑ त्रयः - त्रिꣳ॒॒शाः । अ॒वस्ता᳚त् । भ॒व॒न्ति॒ । त्रयः॑ । त्र॒य॒स्त्रिꣳ॒॒शा इति॑ त्रय - त्रिꣳ॒॒शाः । प॒रस्ता᳚त् । त्र॒य॒स्त्रिꣳ॒॒शैरिति॑ त्रयः - त्रिꣳ॒॒शैः । ए॒व । उ॒भ॒यतः॑ । अव॑र्तिम् । पा॒प्मान᳚म् । अ॒प॒हत्येत्य॑प - हत्य॑ । दैवी᳚म् । सꣳ॒॒सद॒मिति॑ सं - सद᳚म् । म॒द्ध्य॒तः । \textbf{  5} \newline
                  \newline
                                \textbf{ TS 7.4.2.3} \newline
                  ग॒च्छ॒न्ति॒ । पृ॒ष्ठानि॑ । हि । दैवी᳚ । सꣳ॒॒सदिति॑ सं - सत् । जा॒मि । वै । ए॒तत् । कु॒र्व॒न्ति॒ । यत् । त्रयः॑ । त्र॒य॒स्त्रिꣳ॒॒शा इति॑ त्रयः-स्त्रिꣳ॒॒शाः । अ॒न्वञ्चः॑ । मद्ध्ये᳚ । अनि॑रुक्त॒ इत्यनिः॑ - उ॒क्तः॒ । भ॒व॒ति॒ । तेन॑ । अजा॑मि । ऊ॒द्‌र्ध्वानि॑ । पृ॒ष्ठानि॑ । भ॒व॒न्ति॒ । ऊ॒द्‌र्ध्वाः । छ॒न्दो॒मा इति॑ छन्दः - माः । उ॒भाभ्या᳚म् । रू॒पाभ्या᳚म् । सु॒व॒र्गमिति॑ सुवः - गम् । लो॒कम् । य॒न्ति॒ । अस॑त्रम् । वै । ए॒तत् । यत् । अ॒छ॒न्दो॒ममित्य॑छन्दः-मम् । यत् । छ॒न्दो॒मा इति॑ छन्दः- माः । भव॑न्ति । तेन॑ । स॒त्रम् । दे॒वताः᳚ । ए॒व । पृ॒ष्ठैः । अवेति॑ । रु॒न्ध॒ते॒ । प॒शून् । छ॒न्दो॒मैरिति॑ छन्दः - मैः । ओजः॑ । वै । वी॒र्य᳚म् । पृ॒ष्ठानि॑ । प॒शवः॑ । \textbf{  6} \newline
                  \newline
                                \textbf{ TS 7.4.2.4} \newline
                  छ॒न्दो॒मा इति॑ छन्दः - माः । ओज॑सि । ए॒व । वी॒र्ये᳚ । प॒शुषु॑ । प्रतीति॑ । ति॒ष्ठ॒न्ति॒ । त्रयः॑ । त्र॒य॒स्त्रिꣳ॒॒शा इति॑ त्रयः - त्रिꣳ॒॒शाः । अ॒वस्ता᳚त् । भ॒व॒न्ति॒ । त्रयः॑ । त्र॒य॒स्त्रिꣳ॒॒शा इति॑ त्रयः - त्रिꣳ॒॒शाः । प॒रस्ता᳚त् । मद्ध्ये᳚ । पृ॒ष्ठानि॑ । उरः॑ । वै । त्र॒य॒स्त्रिꣳ॒॒शा इति॑ त्रयः - त्रिꣳ॒॒शाः । आ॒त्मा । पृ॒ष्ठानि॑ । आ॒त्मने᳚ । ए॒व । तत् । यज॑मानाः । शर्म॑ । न॒ह्य॒न्ते॒ । अना᳚र्त्यै । बृ॒ह॒द्र॒थ॒न्त॒राभ्या॒मिति॑ बृहत् - र॒थ॒न्त॒राभ्या᳚म् । य॒न्ती॒ । इ॒यम् । वाव । र॒थ॒न्त॒रमिति॑ रथं - त॒रम् । अ॒सौ । बृ॒हत् । आ॒भ्याम् । ए॒व । य॒न्ति॒ । अथो॒ इति॑ । अ॒नयोः᳚ । ए॒व । प्रतीति॑ । ति॒ष्ठ॒न्ति॒ । ए॒ते इति॑ । वै । य॒ज्ञ्स्य॑ । अ॒ञ्ज॒साय॑नी॒ इत्य॑ञ्जसा - अय॑नी । स्रु॒ती इति॑ । ताभ्या᳚म् । ए॒व । \textbf{  7} \newline
                  \newline
                                \textbf{ TS 7.4.2.5} \newline
                  सु॒व॒र्गमिति॑ सुवः - गम् । लो॒कम् । य॒न्ति॒ । परा᳚ञ्चः । वै । ए॒ते । सु॒व॒र्गमिति॑ सुवः-गम् । लो॒कम् । अ॒भ्यारो॑ह॒न्तीत्य॑भि - आरो॑हन्ति । ये । प॒रा॒चीना॑नि । पृ॒ष्ठानि॑ । उ॒प॒यन्तीत्यु॑प - यन्ति॑ । प्र॒त्यङ् । ष॒ड॒ह इति॑ षट् - अ॒हः । भ॒व॒ति॒ । प्र॒त्यव॑रूढ्या॒ इति॑ प्रति - अव॑रूढ्यै । अथो॒ इति॑ । प्रति॑ष्ठित्या॒ इति॒ प्रति॑ - स्थि॒त्यै॒ । उ॒भयोः᳚ । लो॒कयोः᳚ । ऋ॒द्ध्वा । उदिति॑ । ति॒ष्ठ॒न्ति॒ । त्रि॒वृत॒ इति॑ त्रि - वृतः॑ । अधीति॑ । त्रि॒वृत॒मिति॑ त्रि - वृत᳚म् । उपेति॑ । य॒न्ति॒ । स्तोमा॑नाम् । संप॑त्त्या॒ इति॒ सं - प॒त्त्यै॒ । प्र॒भ॒वायेति॑ प्र - भ॒वाय॑ । ज्योतिः॑ । अ॒ग्नि॒ष्टो॒म इत्य॑ग्नि-स्तो॒मः । भ॒व॒ति॒ । अ॒यम् । वाव । सः । क्षयः॑ । अ॒स्मात् । ए॒व । तेन॑ । क्षया᳚त् । न । य॒न्ति॒ । च॒तु॒र्विꣳ॒॒श॒ति॒रा॒त्र इति॑ चतुर्विꣳशति-रा॒त्रः । भ॒व॒ति॒ । चतु॑र्विꣳशति॒रिति॒ चतुः॑-विꣳ॒॒श॒तिः॒ । अ॒द्‌र्ध॒मा॒सा इत्य॑द्‌र्ध - मा॒साः । सं॒ॅव॒थ्स॒र इति॑ सं - व॒थ्स॒रः ( ) । सं॒ॅव॒थ्स॒र इति॑ सं - व॒थ्स॒रः । सु॒व॒र्ग इति॑ सुवः - गः । लो॒कः । सं॒ॅव॒थ्स॒र इति॑ सं-व॒थ्स॒रे । ए॒व । सु॒व॒र्ग इति॑ सुवः - गे । लो॒के । प्रतीति॑ । ति॒ष्ठ॒न्ति॒ । अथो॒ इति॑ । चतु॑र्विꣳशत्यक्ष॒रेति॒ चत॑र्विꣳशति- अ॒क्ष॒रा॒ । गा॒य॒त्री । गा॒य॒त्री । ब्र॒ह्म॒व॒र्च॒समिति॑ ब्रह्म - व॒र्च॒सम् । गा॒य॒त्रि॒या । ए॒व । ब्र॒ह्म॒व॒र्च॒समिति॑ ब्रह्म - व॒र्च॒सम् । अवेति॑ । रु॒न्ध॒ते॒ । अ॒ति॒रा॒त्रावित्य॑ति - रा॒त्रौ । अ॒भितः॑ । भ॒व॒तः॒ । ब्र॒ह्म॒व॒र्च॒स्येति॑ ब्रह्म - व॒र्च॒सस्य॑ । परि॑गृहीत्या॒ इति॒ परि॑ - गृ॒ही॒त्यै॒ ॥ \textbf{  8} \newline
                  \newline
                      (म॒नु॒ष्य॑स्य - मध्य॒तः - प॒शव॒ - स्ताभ्या॑मे॒व - सं॑ॅवथ्स॒र - श्चतु॑र्विꣳशतिश्च)  \textbf{(A2)} \newline \newline
                                \textbf{ TS 7.4.3.1} \newline
                  ऋ॒क्षा । वा । इ॒यम् । अ॒लो॒मका᳚ । आ॒सी॒त् । सा । अ॒का॒म॒य॒त॒ । ओष॑धीभि॒रित्योष॑धि-भिः॒ । वन॒स्पति॑भि॒रिति॒ वन॒स्पति॑-भिः॒ । प्रेति॑ । जा॒ये॒य॒ । इति॑ । सा । ए॒ताः । त्रिꣳ॒॒शत᳚म् । रात्रीः᳚ । अ॒प॒श्य॒त् । ततः॑ । वै । इ॒यम् । ओष॑धीभि॒रित्योष॑धि - भिः॒ । वन॒स्पति॑भि॒रिति॒ वन॒स्पति॑ -भिः॒ । प्रेति॑ । अ॒जा॒य॒त॒ । ये । प्र॒जाका॑मा॒ इति॑ प्र॒जा - का॒माः॒ । प॒शुका॑मा॒ इति॑ प॒शु - का॒माः॒ । स्युः । ते । ए॒ताः । आ॒सी॒र॒न्न् । प्रेति॑ । ए॒व । जा॒य॒न्ते॒ । प्र॒जयेति॑ प्र-जया᳚ । प॒शुभि॒रिति॑ प॒शु-भिः॒ । इ॒यम् । वै । अ॒क्षु॒द्ध्य॒त् । सा । ए॒ताम् । वि॒राज॒मिति॑ वि - राज᳚म् । अ॒प॒श्य॒त् । ताम् । आ॒त्मन्न् । धि॒त्वा । अ॒न्नाद्य॒मित्य॑न्न - अद्य᳚म् । अवेति॑ । अ॒रु॒न्ध॒ । ओष॑धीः । \textbf{  9} \newline
                  \newline
                                \textbf{ TS 7.4.3.2} \newline
                  वन॒स्पतीन्॑ । प्र॒जामिति॑ प्र - जाम् । प॒शून् । तेन॑ । अ॒व॒द्‌र्ध॒त॒ । सा । जे॒मान᳚म् । म॒हि॒मान᳚म् । अ॒ग॒च्छ॒त् । ये । ए॒वम् । वि॒द्वाꣳसः॑ । ए॒ताः । आस॑ते । वि॒राज॒मिति॑ वि-राज᳚म् । ए॒व । आ॒त्मन्न् । धि॒त्वा । अ॒न्नाद्य॒मित्य॑न्न - अद्य᳚म् । अवेति॑ । रु॒न्ध॒ते॒ । वद्‌र्ध॑न्ते । प्र॒जयेति॑ प्र - जया᳚ । प॒शुभि॒रिति॑ प॒शु - भिः॒ । जे॒मान᳚म् । म॒हि॒मान᳚म् । ग॒च्छ॒न्ति॒ । ज्योतिः॑ । अ॒ति॒रा॒त्र इत्य॑ति - रा॒त्रः । भ॒व॒ति॒ । सु॒व॒र्गस्येति॑ सुवः - गस्य॑ । लो॒कस्य॑ । अनु॑ख्यात्या॒ इत्यनु॑ - ख्या॒त्यै॒ । पृष्ठ्यः॑ । ष॒ड॒ह इति॑ षट् - अ॒हः । भ॒व॒ति॒ । षट् । वै । ऋ॒तवः॑ । षट् । पृ॒ष्ठानि॑ । पृ॒ष्ठैः । ए॒व । ऋ॒तून् । अ॒न्वारो॑ह॒न्तीत्य॑नु - आरो॑हन्ति । ऋ॒तुभि॒रित्यृ॒तु - भिः॒ । सं॒ॅव॒थ्स॒रमिति॑ सं - व॒थ्स॒रम् । ते । सं॒ॅव॒थ्स॒र इति॑ सं - व॒थ्स॒रे । ए॒व । \textbf{  10} \newline
                  \newline
                                \textbf{ TS 7.4.3.3} \newline
                  प्रतीति॑ । ति॒ष्ठ॒न्ति॒ । त्र॒य॒स्त्रिꣳ॒॒शादिति॑ त्रयः - त्रिꣳ॒॒शात् । त्र॒य॒स्त्रिꣳ॒॒शमिति॑ त्रयः - त्रिꣳ॒॒शम् । उपेति॑ । य॒न्ति॒ । य॒ज्ञ्स्य॑ । संत॑त्या॒ इति॒ सं - त॒त्यै॒ । अथो॒ इति॑ । प्र॒जाप॑ति॒रिति॑ प्र॒जा-प॒तिः॒ । वै । त्र॒य॒स्त्रिꣳ॒॒श इति॑ त्रयः-त्रिꣳ॒॒शः । प्र॒जाप॑ति॒मिति॑ प्र॒जा-प॒ति॒म् । ए॒व । एति॑ । र॒भ॒न्ते॒ । प्रति॑ष्ठित्या॒ इति॒ प्रति॑ - स्थि॒त्यै॒ । त्रि॒ण॒व इति॑ त्रि - न॒वः । भ॒व॒ति॒ । विजि॑त्या॒ इति॒ वि - जि॒त्यै॒ । ए॒क॒विꣳ॒॒श इत्ये॑क - विꣳ॒॒शः । भ॒व॒ति॒ । प्रति॑ष्ठित्या॒ इति॒ प्रति॑ - स्थि॒त्यै॒ । अथो॒ इति॑ । रुच᳚म् । ए॒व । आ॒त्मन्न् । द॒ध॒ते॒ । त्रि॒वृदिति॑ त्रि - वृत् । अ॒ग्नि॒ष्टुदित्य॑ग्नि - स्तुत् । भ॒व॒ति॒ । पा॒प्मान᳚म् । ए॒व । तेन॑ । निरिति॑ । द॒ह॒न्ते॒ । अथो॒ इति॑ । तेजः॑ । वै । त्रि॒वृदिति॑ त्रि-वृत् । तेजः॑ । ए॒व । आ॒त्मन्न् । द॒ध॒ते॒ । प॒ञ्च॒द॒श इति॑ पञ्च - द॒शः । इ॒न्द्र॒स्तो॒म इती᳚न्द्र - स्तो॒मः । भ॒व॒ति॒ । इ॒न्द्रि॒यम् । ए॒व । अवेति॑ । \textbf{  11} \newline
                  \newline
                                \textbf{ TS 7.4.3.4} \newline
                  रु॒न्ध॒ते॒ । स॒प्त॒द॒श इति॑ सप्त - द॒शः । भ॒व॒ति॒ । अ॒न्नाद्य॒स्येत्य॑न्न- अद्य॑स्य । अव॑रुद्ध्या॒ इत्यव॑-रु॒द्ध्यै॒ । अथो॒ इति॑ । प्रेति॑ । ए॒व । तेन॑ । जा॒य॒न्ते॒ । ए॒क॒विꣳ॒॒श इत्ये॑क - विꣳ॒॒शः । भ॒व॒ति॒ । प्रति॑ष्ठित्या॒ इति॒ प्रति॑ - स्थि॒त्यै॒ । अथो॒ इति॑ । रुच᳚म् । ए॒व । आ॒त्मन्न् । द॒ध॒ते॒ । च॒तु॒र्विꣳ॒॒श इति॑ चतुः - विꣳ॒॒शः । भ॒व॒ति॒ । चतु॑र्विꣳशति॒रिति॒ चतुः॑ - विꣳ॒॒श॒तिः॒ । अ॒द्‌र्ध॒मा॒सा इत्य॑द्‌र्ध - मा॒साः । सं॒ॅव॒थ्स॒र इति॑ सं - व॒थ्स॒रः । सं॒ॅव॒थ्स॒र इति॑ सं - व॒थ्स॒रः । सु॒व॒र्ग इति॑ सुवः - गः । लो॒कः । सं॒ॅव॒थ्स॒र इति॑ सं - व॒थ्स॒रे । ए॒व । सु॒व॒र्ग इति॑ सुवः - गे । लो॒के । प्रतीति॑ । ति॒ष्ठ॒न्ति॒ । अथो॒ इति॑ । ए॒षः । वै । वि॒षू॒वानिति॑ विषु - वान् । वि॒षू॒वन्त॒ इति॑ विषु - वन्तः॑ । भ॒व॒न्ति॒ । ये । ए॒वम् । वि॒द्वाꣳसः॑ । ए॒ताः । आस॑ते । च॒तु॒र्विꣳ॒॒शादिति॑ चतुः - विꣳ॒॒शात् । पृ॒ष्ठानि॑ । उपेति॑ । य॒न्ति॒ । सं॒ॅव॒थ्स॒र इति॑ सं - व॒थ्स॒रे । ए॒व । प्र॒ति॒ष्ठायेति॑ प्रति-स्थाय॑ । \textbf{  12} \newline
                  \newline
                                \textbf{ TS 7.4.3.5} \newline
                  दे॒वताः᳚ । अ॒भ्यारो॑ह॒न्तीत्य॑भि - आरो॑हन्ति । त्र॒य॒स्त्रिꣳ॒॒शादिति॑ त्रयः - त्रिꣳ॒॒शात् । त्र॒य॒स्त्रिꣳ॒॒शमिति॑ त्रय - त्रिꣳ॒॒शम् । उपेति॑ । य॒न्ति॒ । त्रय॑स्त्रिꣳश॒दिति॒ त्रयः॑ - त्रिꣳ॒॒श॒त् । वै । दे॒वताः᳚ । दे॒वता॑सु । ए॒व । प्रतीति॑ । ति॒ष्ठ॒न्ति॒ । त्रि॒ण॒व इति॑ त्रि - न॒वः । भ॒व॒ति॒ । इ॒मे । वै । लो॒काः । त्रि॒ण॒व इति॑ त्रि - न॒वः । ए॒षु । ए॒व । लो॒केषु॑ । प्रतीति॑ । ति॒ष्ठ॒न्ति॒ । द्वौ । ए॒क॒विꣳ॒॒शावित्ये॑क - विꣳ॒॒शौ । भ॒व॒तः॒ । प्रति॑ष्ठित्या॒ इति॒ प्रति॑ - स्थि॒त्यै॒ । अथो॒ इति॑ । रुच᳚म् । ए॒व । आ॒त्मन्न् । द॒ध॒ते॒ । ब॒हवः॑ । षो॒ड॒शिनः॑ । भ॒व॒न्ति॒ । तस्मा᳚त् । ब॒हवः॑ । प्र॒जास्विति॑ प्र - जासु॑ । वृषा॑णः । यत् । ए॒ते । स्तोमाः᳚ । व्यति॑षक्ता॒ इति॑ वि - अति॑षक्ताः । भव॑न्ति । तस्मा᳚त् । इ॒यम् । ओष॑धीभि॒रित्योष॑धि - भिः॒ । वन॒स्पति॑भि॒रिति॒ वन॒स्पति॑ - भिः॒ । व्यति॑ष॒क्तेति॑ वि - अति॑षक्ता । \textbf{  13} \newline
                  \newline
                                \textbf{ TS 7.4.3.6} \newline
                  व्यति॑षज्यन्त॒ इति॑ वि - अति॑षज्यन्ते । प्र॒जयेति॑ प्र - जया᳚ । प॒शुभि॒रिति॑ प॒शु - भिः॒ । ये । ए॒वम् । वि॒द्वाꣳसः॑ । ए॒ताः । आस॑ते । अक्लृ॑प्ताः । वै । ए॒ते । सु॒व॒र्गमिति॑ सुवः - गम् । लो॒कम् । य॒न्ति॒ । उ॒च्चा॒व॒चान् । हि । स्तोमान्॑ । उ॒प॒यन्तीत्यु॑प - यन्ति॑ । यत् । ए॒ते । ऊ॒द्‌र्ध्वाः । क्लृ॒प्ताः । स्तोमाः᳚ । भव॑न्ति । क्लृ॒प्ताः । ए॒व । सु॒व॒र्गमिति॑ सुवः - गम् । लो॒कम् । य॒न्ति॒ । उ॒भयोः᳚ । ए॒भ्यः॒ । लो॒कयोः᳚ । क॒ल्प॒ते॒ । त्रिꣳ॒॒शत् । ए॒ताः । त्रिꣳ॒॒शद॑क्ष॒रेति॑ त्रिꣳ॒॒शत् - अ॒क्ष॒रा॒ । वि॒राडिति॑ वि - राट् । अन्न᳚म् । वि॒राडिति॑ वि - राट् । वि॒राजेति॑ वि - राजा᳚ । ए॒व । अ॒न्नाद्य॒मित्य॑न्न-अद्य᳚म् । अवेति॑ । रु॒न्ध॒ते॒ । अ॒ति॒रा॒त्रावित्य॑ति - रा॒त्रौ । अ॒भितः॑ । भ॒व॒तः॒ । अ॒न्नाद्य॒स्येत्य॑न्न - अद्य॑स्य । परि॑गृहीत्या॒ इति॒ परि॑ - गृ॒ही॒त्यै॒ ॥ \textbf{  14 } \newline
                  \newline
                      (ओष॑धीः - संॅवथ्स॒र ए॒वा - ऽव॑ - प्रति॒ष्ठाय॒ - व्यति॑ष॒क्त्यै - का॒न्नप॑ञ्चा॒शच्च॑)  \textbf{(A3)} \newline \newline
                                \textbf{ TS 7.4.4.1} \newline
                  प्र॒जाप॑ति॒रिति॑ प्र॒जा - प॒तिः॒ । सु॒व॒र्गमिति॑ सुवः - गम् । लो॒कम् । ऐ॒त् । तम् । दे॒वाः । येन॑ये॒नेति॒ येन॑ - ये॒न॒ । छन्द॑सा । अनु॑ । प्रेति॑ । अयु॑ञ्जत । तेन॑ । न । आ॒प्नु॒व॒न्न् । ते । ए॒ताः । द्वात्रिꣳ॑शतम् । रात्रीः᳚ । अ॒प॒श्य॒न्न् । द्वात्रिꣳ॑शदक्ष॒रेति॒ द्वात्रिꣳ॑शत् - अ॒क्ष॒रा॒ । अ॒नु॒ष्टुगित्य॑नु - स्तुक् । आनु॑ष्टुभ॒ इत्यानु॑ - स्तु॒भः॒ । प्र॒जाप॑ति॒रिति॑ प्र॒जा-प॒तिः॒ । स्वेन॑ । ए॒व । छन्द॑सा । प्र॒जाप॑ति॒मिति॑ प्र॒जा - प॒ति॒म् । आ॒प्त्वा । अ॒भ्या॒रुह्येत्य॑भि - आ॒रुह्य॑ । सु॒व॒र्गमिति॑ सुवः - गम् । लो॒कम् । आ॒य॒न्न् । ये । ए॒वम् । वि॒द्वाꣳसः॑ । ए॒ताः । आस॑ते । द्वात्रिꣳ॑शत् । ए॒ताः । द्वात्रिꣳ॑शदक्ष॒रेति॒ द्वात्रिꣳ॑शत् - अ॒क्ष॒रा॒ । अ॒नु॒ष्टुगित्य॑नु - स्तुक् । आनु॑ष्टुभ॒ इत्यानु॑ - स्तु॒भः॒ । प्र॒जाप॑ति॒रिति॑ प्र॒जा-प॒तिः॒ । स्वेन॑ । ए॒व । छन्द॑सा । प्र॒जाप॑ति॒मिति॑ प्र॒जा -प॒ति॒म् । आ॒प्त्वा । श्रिय᳚म् । ग॒च्छ॒न्ति॒ । \textbf{  15} \newline
                  \newline
                                \textbf{ TS 7.4.4.2} \newline
                  श्रीः । हि । म॒नु॒ष्य॑स्य । सु॒व॒र्ग इति॑ सुवः-गः । लो॒कः । द्वात्रिꣳ॑शत् । ए॒ताः । द्वात्रिꣳ॑शदक्ष॒रेति॒ द्वात्रिꣳ॑शत् - अ॒क्ष॒रा॒ । अ॒नु॒ष्टुगित्य॑नु - स्तुक् । वाक् । अ॒नु॒ष्टुबित्य॑नु - स्तुप् । सर्वा᳚म् । ए॒व । वाच᳚म् । आ॒प्नु॒व॒न्ति॒ । सर्वे᳚ । वा॒चः । व॒दि॒तारः॑ । भ॒व॒न्ति॒ । सर्वे᳚ । हि । श्रिय᳚म् । गच्छ॑न्ति । ज्योतिः॑ । गौः । आयुः॑ । इति॑ । त्र्य॒हा इति॑ त्रि - अ॒हाः । भ॒व॒न्ति॒ । इ॒यम् । वाव । ज्योतिः॑ । अ॒न्तरि॑क्षम् । गौः । अ॒सौ । आयुः॑ । इ॒मान् । ए॒व । लो॒कान् । अ॒भ्यारो॑ह॒न्तीत्य॑भि - आरो॑हन्ति । अ॒भि॒पू॒र्वमित्य॑भि - पू॒र्वम् । त्र्य॒हा इति॑ त्रि-अ॒हाः । भ॒व॒न्ति॒ । अ॒भि॒पू॒र्वमित्य॑भि - पू॒र्वम् । ए॒व । सु॒व॒र्गमिति॑ सुवः - गम् । लो॒कम् । अ॒भ्यारो॑ह॒न्तीत्य॑भि-आरो॑हन्ति । बृ॒ह॒द्र॒थ॒न्त॒राभ्या॒मिति॑ बृहत् - र॒थ॒न्त॒राभ्या᳚म् । य॒न्ति॒ । \textbf{  16} \newline
                  \newline
                                \textbf{ TS 7.4.4.3} \newline
                  इ॒यम् । वाव । र॒थ॒न्त॒रमिति॑ रथं - त॒रम् । अ॒सौ । बृ॒हत् । आ॒भ्याम् । ए॒व । य॒न्ति॒ । अथो॒ इति॑ । अ॒नयोः᳚ । ए॒व । प्रतीति॑ । ति॒ष्ठ॒न्ति॒ । ए॒ते इति॑ । वै । य॒ज्ञ्स्य॑ । अ॒ञ्ज॒साय॑नी॒ इत्य॑ञ्जसा - अय॑नी । स्रु॒ती इति॑ । ताभ्या᳚म् । ए॒व । सु॒व॒र्गमिति॑ सुवः - गम् । लो॒कम् । य॒न्ति॒ । परा᳚ञ्चः । वै । ए॒ते । सु॒व॒र्गमिति॑ सुवः - गम् । लो॒कम् । अ॒भ्यारो॑ह॒न्तीत्य॑भि - आरो॑हन्ति । ये । परा॑चः । त्र्य॒हानिति॑ त्रि - अ॒हान् । उ॒प॒यन्तीत्यु॑प - यन्ति॑ । प्र॒त्यङ् । त्र्य॒हा इति॑ त्रि - अ॒हः । भ॒व॒ति॒ । प्र॒त्यव॑रूढ्या॒ इति॑ प्रति - अव॑रूढ्यै । अथो॒ इति॑ । प्रति॑ष्ठित्या॒ इति॒ प्रति॑ - स्थि॒त्यै॒ । उ॒भयोः᳚ । लो॒कयोः᳚ । ऋ॒द्ध्वा । उदिति॑ । ति॒ष्ठ॒न्ति॒ । द्वात्रिꣳ॑शत् । ए॒ताः । तासा᳚म् । याः । त्रिꣳ॒॒शत् । त्रिꣳ॒॒शद॑क्ष॒रेति॑ त्रिꣳ॒॒शत् - अ॒क्ष॒रा॒ ( ) । वि॒राडिति॑ वि - राट् । अन्न᳚म् । वि॒राडिति॑ वि - राट् । वि॒राजेति॑ वि - राजा᳚ । ए॒व । अ॒न्नाद्य॒मित्य॑न्न - अद्य᳚म् । अवेति॑ । रु॒न्ध॒ते॒ । ये इति॑ । द्वे इति॑ । अ॒हो॒रा॒त्रे इत्य॑हः - रा॒त्रे । ए॒व । ते इति॑ । उ॒भाभ्या᳚म् । रू॒पाभ्या᳚म् । सु॒व॒र्गमिति॑ सुवः - गम् । लो॒कम् । य॒न्ति॒ । अ॒ति॒रा॒त्रावित्य॑ति - रा॒त्रौ । अ॒भितः॑ । भ॒व॒तः॒ । परि॑गृहीत्या॒ इति॒ परि॑ - गृ॒ही॒त्यै॒ ॥ \textbf{  17} \newline
                  \newline
                      (ग॒च्छ॒न्ति॒ - य॒न्ति॒ - त्रिꣳ॒॒शद॑क्षरा॒ - द्वाविꣳ॑शतिश्च)  \textbf{(A4)} \newline \newline
                                \textbf{ TS 7.4.5.1} \newline
                  द्वे इति॑ । वाव । दे॒व॒स॒त्रे इति॑ देव - स॒त्रे । द्वा॒द॒शा॒ह इति॑ द्वादश-अ॒हः । च॒ । ए॒व । त्र॒य॒स्त्रिꣳ॒॒श॒द॒ह इति॑ त्रयस्त्रिꣳशत्-अ॒हः । च॒ । ये । ए॒वम् । वि॒द्वाꣳसः॑ । त्र॒य॒स्त्रिꣳ॒॒श॒द॒हमिति॑ त्रयस्त्रिꣳशत्-अ॒हम् । आस॑ते । सा॒क्षादिति॑ स - आ॒क्षात् । ए॒व । दे॒वताः᳚ । अ॒भ्यारो॑ह॒न्तीत्य॑भि - आरो॑हन्ति । यथा᳚ । खलु॑ । वै । श्रेयान्॑ । अ॒भ्यारू॑ढ॒ इत्य॑भि - आरू॑ढः । का॒मय॑ते । तथा᳚ । क॒रो॒ति॒ । यदि॑ । अ॒व॒विद्ध्य॒तीत्य॑व - विद्ध्य॑ति । पापी॑यान् । भ॒व॒ति॒ । यदि॑ । न । अ॒व॒विद्ध्य॒तीत्य॑व - विद्ध्य॑ति । स॒दृङ्ङिति॑ स - दृङ् । ये । ए॒वम् । वि॒द्वाꣳसः॑ । त्र॒य॒स्त्रिꣳ॒॒श॒द॒हमिति॑ त्रयस्त्रिꣳशत् - अ॒हम् । आस॑ते । वीति॑ । पा॒प्मना᳚ । भ्रातृ॑व्येण । एति॑ । व॒र्त॒न्ते॒ । अ॒ह॒र्भाज॒ इत्य॑हः - भाजः॑ । वै । ए॒ताः । दे॒वाः । अग्रे᳚ । एति॑ । अ॒ह॒र॒न्न् । \textbf{  18} \newline
                  \newline
                                \textbf{ TS 7.4.5.2} \newline
                  अहः॑ । एकः॑ । अभ॑जत । अहः॑ । एकः॑ । ताभिः॑ । वै । ते । प्र॒बाहु॒गिति॑ प्र - बाहु॑क् । आ॒द्‌र्ध्नु॒व॒न्न् । ये । ए॒वम् । वि॒द्वाꣳसः॑ । त्र॒य॒स्त्रिꣳ॒॒श॒द॒हमिति॑ त्रयस्त्रिꣳशत् - अ॒हम् । आस॑ते । सर्वे᳚ । ए॒व । प्र॒बाहु॒गिति॑ प्र - बाहु॑क् । ऋ॒द्ध्नु॒व॒न्ति॒ । सर्वे᳚ । ग्राम॑णीय॒मिति॒ ग्राम॑ - नी॒य॒म् । प्रेति॑ । आ॒प्नु॒व॒न्ति॒ । प॒ञ्चा॒हा इति॑ पञ्च-अ॒हाः । भ॒व॒न्ति॒ । पञ्च॑ । वै । ऋ॒तवः॑ । सं॒ॅव॒थ्स॒र इति॑ सं - व॒थ्स॒रः । ऋ॒तुषु॑ । ए॒व । सं॒ॅव॒थ्स॒र इति॑ सं-व॒थ्स॒रे । प्रतीति॑ । ति॒ष्ठ॒न्ति॒ । अथो॒ इति॑ । पञ्चा᳚क्ष॒रेति॒ पञ्च॑-अ॒क्ष॒रा॒ । प॒ङ्क्तिः । पाङ्क्तः॑ । य॒ज्ञ्ः । य॒ज्ञ्म् । ए॒व । अवेति॑ । रु॒न्ध॒ते॒ । त्रीणि॑ । आ॒श्वि॒नानि॑ । भ॒व॒न्ति॒ । त्रयः॑ । इ॒मे । लो॒काः । ए॒षु । \textbf{  19} \newline
                  \newline
                                \textbf{ TS 7.4.5.3} \newline
                  ए॒व । लो॒केषु॑ । प्रतीति॑ । ति॒ष्ठ॒न्ति॒ । अथो॒ इति॑ । त्रीणि॑ । वै । य॒ज्ञ्स्य॑ । इ॒न्द्रि॒याणि॑ । तानि॑ । ए॒व । अवेति॑ । रु॒न्ध॒ते॒ । वि॒श्व॒जिदिति॑ विश्व - जित् । भ॒व॒ति॒ । अ॒न्नाद्य॒स्येत्य॑न्न - अद्य॑स्य । अव॑रुद्ध्या॒ इत्यव॑ - रु॒द्ध्यै॒ । सर्व॑पृष्ठ॒ इति॒ सर्व॑ - पृ॒ष्ठः॒ । भ॒व॒ति॒ । सर्व॑स्य । अ॒भिजि॑त्या॒ इत्य॒भि - जि॒त्यै॒ । वाक् । वै । द्वा॒द॒शा॒ह इति॑ द्वादश - अ॒हः । यत् । पु॒रस्ता᳚त् । द्वा॒द॒शा॒हमिति द्वादश - अ॒हम् । उ॒पे॒युरित्यु॑प-इ॒युः । अना᳚प्ताम् । वाच᳚म् । उपेति॑ । इ॒युः॒ । उ॒प॒दासु॒केत्यु॑प - दासु॑का । ए॒षा॒म् । वाक् । स्या॒त् । उ॒परि॑ष्टात् । द्वा॒द॒शा॒हमिति॑ द्वादश - अ॒हम् । उपेति॑ । य॒न्ति॒ । आ॒प्ताम् । ए॒व । वाच᳚म् । उपेति॑ । य॒न्ति॒ । तस्मा᳚त् । उ॒परि॑ष्टात् । वा॒चा । व॒दा॒मः॒ । अ॒वा॒न्त॒रमित्य॑व - अ॒न्त॒रम् । \textbf{  20} \newline
                  \newline
                                \textbf{ TS 7.4.5.4} \newline
                  वै । द॒श॒रा॒त्रेणेति॑ दश - रा॒त्रेण॑ । प्र॒जाप॑ति॒रिति॑ प्र॒जा - प॒तिः॒ । प्र॒जा इति॑ प्र - जाः । अ॒सृ॒ज॒त॒ । यत् । द॒श॒रा॒त्र इति॑ दश - रा॒त्रः । भव॑ति । प्र॒जा इति॑ प्र - जाः । ए॒व । तत् । यज॑मानाः । सृ॒ज॒न्ते॒ । ए॒ताम् । ह॒ । वै । उ॒द॒ङ्कः । शौ॒ल्बा॒य॒नः । स॒त्रस्य॑ । ऋद्धि᳚म् । उ॒वा॒च॒ । यत् । द॒श॒रा॒त्र इति॑ दश - रा॒त्रः । यत् । द॒श॒रा॒त्र इति॑ दश - रा॒त्रः । भव॑ति । स॒त्रस्य॑ । ऋद्ध्यै᳚ । अथो॒ इति॑ । यत् । ए॒व । पूर्वे॑षु । अह॒स्स्वित्यहः॑ - सु॒ । विलो॒मेति॒ वि - लो॒म॒ । क्रि॒यते᳚ । तस्य॑ । ए॒व । ए॒षा । शान्तिः॑ । द्व्य॒नी॒का इति॑ द्वि - अ॒नी॒काः । वै । ए॒ताः । रात्र॑यः । यज॑मानाः । वि॒श्व॒जिदिति॑ विश्व - जित् । स॒ह । अ॒ति॒रा॒त्रेणेत्य॑ति - रा॒त्रेण॑ । पूर्वाः᳚ । षोड॑श । स॒ह ( ) । अ॒ति॒रा॒त्रेणेत्य॑ति - रा॒त्रेण॑ । उत्त॑रा॒ इत्युत् - त॒राः॒ । षोड॑श । ये । ए॒वम् । वि॒द्वाꣳसः॑ । त्र॒य॒स्त्रिꣳ॒॒श॒द॒हमिति॑ त्रयस्त्रिꣳशत् - अ॒हम् । आस॑ते । एति॑ । ए॒षा॒म् । द्व्य॒नी॒केति॑ द्वि - अ॒नी॒का । प्र॒जेति॑ प्र - जा । जा॒य॒ते॒ । अ॒ति॒रा॒त्रावित्य॑ति - रा॒त्रौ । अ॒भितः॑ । भ॒व॒तः॒ । परि॑गृहीत्या॒ इति॒ परि॑ - गृ॒ही॒त्यै॒ ॥ \textbf{  21} \newline
                  \newline
                      (अ॒ह॒र॒- न्ने॒ष्व॑ - वान्त॒रꣳ - षोड॑श स॒ह - स॒प्तद॑श च)  \textbf{(A5)} \newline \newline
                                \textbf{ TS 7.4.6.1} \newline
                  आ॒दि॒त्याः । अ॒का॒म॒य॒न्त॒ । सु॒व॒र्गमिति॑ सुवः - गम् । लो॒कम् । इ॒या॒म॒ । इति॑ । ते । सु॒व॒र्गमिति॑ सुवः - गम् । लो॒कम् । न । प्रेति॑ । अ॒जा॒न॒न्न् । न । सु॒व॒र्गमिति॑ सुवः-गम् । लो॒कम् । आ॒य॒न्न् । ते । ए॒तम् । ष॒ट्त्रिꣳ॒॒श॒द्रा॒त्रमिति॑ षट्त्रिꣳशत् - रा॒त्रम् । अ॒प॒श्य॒न्न् । तम् । एति॑ । अ॒ह॒र॒न्न् । तेन॑ । अ॒य॒ज॒न्त॒ । ततः॑ । वै । ते । सु॒व॒र्गमिति॑ सुवः - गम् । लो॒कम् । प्रेति॑ । अ॒जा॒न॒न्न् । सु॒व॒र्गमिति॑ सुवः-गम् । लो॒कम् । आ॒य॒न्न् । ये । ए॒वम् । वि॒द्वाꣳसः॑ । ष॒ट्त्रिꣳ॒॒श॒द्रा॒त्रमिति॑ षट्त्रिꣳशत् - रा॒त्रम् । आस॑ते । सु॒व॒र्गमिति॑ सुवः - गम् । ए॒व । लो॒कम् । प्रेति॑ । जा॒न॒न्ति॒ । सु॒व॒र्गमिति॑ सुवः - गम् । लो॒कम् । य॒न्ति॒ । ज्योतिः॑ । अ॒ति॒रा॒त्र इत्य॑ति - रा॒त्रः । \textbf{  22} \newline
                  \newline
                                \textbf{ TS 7.4.6.2} \newline
                  भ॒व॒ति॒ । ज्योतिः॑ । ए॒व । पु॒रस्ता᳚त् । द॒ध॒ते॒ । सु॒व॒र्गस्येति॑ सुवः - गस्य॑ । लो॒कस्य॑ । अनु॑ख्यात्या॒ इत्यनु॑ - ख्या॒त्यै॒ । ष॒ड॒हा इति॑ षट् - अ॒हाः । भ॒व॒न्ति॒ । षट् । वै । ऋ॒तवः॑ । ऋ॒तुषु॑ । ए॒व । प्रतीति॑ । ति॒ष्ठ॒न्ति॒ । च॒त्वारः॑ । भ॒व॒न्ति॒ । चत॑स्रः । दिशः॑ । दि॒क्षु । ए॒व । प्रतीति॑ । ति॒ष्ठ॒न्ति॒ । अस॑त्रम् । वै । ए॒तत् । यत् । अ॒छ॒न्दो॒ममित्य॑छन्दः - मम् । यत् । छ॒न्दो॒मा इति॑ छन्दः - माः । भव॑न्ति । तेन॑ । स॒त्रम् । दे॒वताः᳚ । ए॒व । पृ॒ष्ठैः । अवेति॑ । रु॒न्ध॒ते॒ । प॒शून् । छ॒न्दो॒मैरिति॑ छन्दः - मैः । ओजः॑ । वै । वी॒र्य᳚म् । पृ॒ष्ठानि॑ । प॒शवः॑ । छ॒न्दो॒मा इति॑ छन्दः - माः । ओज॑सि । ए॒व । \textbf{  23} \newline
                  \newline
                                \textbf{ TS 7.4.6.3} \newline
                  वी॒र्ये᳚ । प॒शुषु॑ । प्रतीति॑ । ति॒ष्ठ॒न्ति॒ । ष॒ट्त्रिꣳ॒॒श॒द्रा॒त्र इति॑ षट्त्रिꣳशत्-रा॒त्रः । भ॒व॒ति॒ । षट्त्रिꣳ॑शदक्ष॒रेति॒ षट्त्रिꣳ॑शत्-अ॒क्ष॒रा॒ । बृ॒ह॒ती । बार्.ह॑ताः । प॒शवः॑ । बृ॒ह॒त्या । ए॒व । प॒शून् । अवेति॑ । रु॒न्ध॒ते॒ । बृ॒ह॒ती । छन्द॑साम् । स्वारा᳚ज्य॒मिति॒ स्व - रा॒ज्य॒म् । आ॒श्नु॒त॒ । अ॒श्नु॒वते᳚ । स्वारा᳚ज्य॒मिति॒ स्व - रा॒ज्य॒म् । ये । ए॒वम् । वि॒द्वाꣳसः॑ । ष॒ट्त्रिꣳ॒॒श॒द्रा॒त्रमिति॑ षट्त्रिꣳशत् - रा॒त्रम् । आस॑ते । सु॒व॒र्गमिति॑ सुवः - गम् । ए॒व । लो॒कम् । य॒न्ति॒ । अ॒ति॒रा॒त्रावित्य॑ति - रा॒त्रौ । अ॒भितः॑ । भ॒व॒तः॒ । सु॒व॒र्गस्येति॑ सुवः - गस्य॑ । लो॒कस्य॑ । परि॑गृहीत्या॒ इति॒ परि॑ - गृ॒ही॒त्यै॒ ॥ \textbf{  24} \newline
                  \newline
                      (अ॒ति॒रा॒त्र - ओज॑स्ये॒व - षट्त्रिꣳ॑शच्च)  \textbf{(A6)} \newline \newline
                                \textbf{ TS 7.4.7.1} \newline
                  वसि॑ष्ठः । ह॒तपु॑त्र॒ इति॑ ह॒त - पु॒त्रः॒ । अ॒का॒म॒य॒त॒ । वि॒न्देय॑ । प्र॒जामिति॑ प्र-जाम् । अ॒भीति॑ । सौ॒दा॒सान् । भ॒वे॒य॒म् । इति॑ । सः । ए॒तम् । ए॒क॒स्मा॒न्‌न॒प॒ञ्चा॒शमित्ये॑कस्मात् - न॒प॒ञ्चा॒शम् । अ॒प॒श्य॒त् । तम् । एति॑ । अ॒ह॒र॒त् । तेन॑ । अ॒य॒ज॒त॒ । ततः॑ । वै । सः । अवि॑न्दत । प्र॒जामिति॑ प्र-जाम् । अ॒भीति॑ । सौ॒दा॒सान् । अ॒भ॒व॒त् । ये । ए॒वम् । वि॒द्वाꣳसः॑ । ए॒क॒स्मा॒न्न॒प॒ञ्चा॒शमित्ये॑कस्मात्-न॒प॒ञ्चा॒शम् । आस॑ते । वि॒न्दन्ते᳚ । प्र॒जामिति॑ प्र - जाम् । अ॒भीति॑ । भ्रातृ॑व्यान् । भ॒व॒न्ति॒ । त्रयः॑ । त्रि॒वृत॒ इति॑ त्रि - वृतः॑ । अ॒ग्नि॒ष्टो॒मा इत्य॑ग्नि - स्तो॒माः । भ॒व॒न्ति॒ । वज्र॑स्य । ए॒व । मुख᳚म् । समिति॑ । श्य॒न्ति॒ । दश॑ । प॒ञ्च॒द॒शा इति॑ पञ्च - द॒शाः । भ॒व॒न्ति॒ । प॒ञ्च॒द॒श इति॑ पञ्च - द॒शः । वज्रः॑ । \textbf{  25} \newline
                  \newline
                                \textbf{ TS 7.4.7.2} \newline
                  वज्र᳚म् । ए॒व । भ्रातृ॑व्येभ्यः । प्रेति॑ । ह॒र॒न्ति॒ । षो॒ड॒शि॒मदिति॑ षोडशि - मत् । द॒श॒मम् । अहः॑ । भ॒व॒ति॒ । वज्रे᳚ । ए॒व । वी॒र्य᳚म् । द॒ध॒ति॒ । द्वाद॑श । स॒प्त॒द॒शा इति॑ सप्त - द॒शाः । भ॒व॒न्ति॒ । अ॒न्नाद्य॒स्येत्य॑न्न- अद्य॑स्य । अव॑रुद्ध्या॒ इत्यव॑-रु॒द्ध्यै॒ । अथो॒ इति॑ । प्रेति॑ । ए॒व । तैः । जा॒य॒न्ते॒ । पृष्ठ्यः॑ । ष॒ड॒ह इति॑ षट् - अ॒हः । भ॒व॒ति॒ । षट् । वै । ऋ॒तवः॑ । षट् । पृ॒ष्ठानि॑ । पृ॒ष्ठैः । ए॒व । ऋ॒तून् । अ॒न्वारो॑ह॒न्तीत्य॑नु-आरो॑हन्ति । ऋ॒तुभि॒रित्यृ॒तु - भिः॒ । सं॒ॅव॒थ्स॒रमिति॑ सं - व॒थ्स॒रम् । ते । सं॒ॅव॒थ्स॒र इति॑ सं - व॒थ्स॒रे । ए॒व । प्रतीति॑ । ति॒ष्ठ॒न्ति॒ । द्वाद॑श । ए॒क॒विꣳ॒॒शा इत्ये॑क - विꣳ॒॒शाः । भ॒व॒न्ति॒ । प्रति॑ष्ठित्या॒ इति॒ प्रति॑ - स्थि॒त्यै॒ । अथो॒ इति॑ । रुच᳚म् । ए॒व । आ॒त्मन्न् । \textbf{  26} \newline
                  \newline
                                \textbf{ TS 7.4.7.3} \newline
                  द॒ध॒ते॒ । ब॒हवः॑ । षो॒ड॒शिनः॑ । भ॒व॒न्ति॒ । विजि॑त्या॒ इति॒ वि - जि॒त्यै॒ । षट् । आ॒श्वि॒नानि॑ । भ॒व॒न्ति॒ । षट् । वै । ऋ॒तवः॑ । ऋ॒तुषु॑ । ए॒व । प्रतीति॑ । ति॒ष्ठ॒न्ति॒ । ऊ॒ना॒ति॒रि॒क्ता इत्यू॑न - अ॒ति॒रि॒क्ताः । वै । ए॒ताः । रात्र॑यः । ऊ॒नाः । तत् । यत् । एक॑स्यै । न । प॒ञ्चा॒शत् । अति॑रिक्ता॒ इत्यति॑ - रि॒क्ताः॒ । तत् । यत् । भूय॑सीः । अ॒ष्टाच॑त्वारिꣳशत॒ इत्य॒ष्टा-च॒त्वा॒रिꣳ॒॒श॒तः॒ । ऊ॒नात् । च॒ । खलु॑ । वै । अति॑रिक्ता॒दित्यति॑-रि॒क्ता॒त् । च॒ । प्र॒जाप॑ति॒रिति॑ प्र॒जा - प॒तिः॒ । प्रेति॑ । अ॒जा॒य॒त॒ । ये । प्र॒जाका॑मा॒ इति॑ प्र॒जा - का॒माः॒ । प॒शुका॑मा॒ इति॑ प॒शु - का॒माः॒ । स्युः । ते । ए॒ताः । आ॒सी॒र॒न्न् । प्रेति॑ । ए॒व । जा॒य॒न्ते॒ । प्र॒जयेति॑ प्र - जया᳚ ( ) । प॒शुभि॒रिति॑ प॒शु - भिः॒ । वै॒रा॒जः । वै । ए॒षः । य॒ज्ञ्ः । यत् । ए॒क॒स्मा॒न्न॒प॒ञ्चा॒श इत्ये॑कस्मात् - न॒प॒ञ्चा॒शः । ये । ए॒वम् । वि॒द्वाꣳसः॑ । ए॒क॒स्मा॒न्न॒प॒ञ्चा॒शमित्ये॑कस्मात् - न॒प॒ञ्चा॒शम् । आस॑ते । वि॒राज॒मिति॑ वि - राज᳚म् । ए॒व । ग॒च्छ॒न्ति॒ । अ॒न्ना॒दा इत्य॑न्न-अ॒दाः । भ॒व॒न्ति॒ । अ॒ति॒रा॒त्रावित्य॑ति - रा॒त्रौ । अ॒भितः॑ । भ॒व॒तः॒ । अ॒न्नाद्य॒स्येत्य॑न्न-अद्य॑स्य । परि॑गृहीत्या॒ इति॒ परि॑-गृ॒ही॒त्यै॒ ॥ \textbf{  27} \newline
                  \newline
                      (वज्र॑ - आ॒त्मन् - प्र॒जया॒ - द्वाविꣳ॑शतिश्च)  \textbf{(A7)} \newline \newline
                                \textbf{ TS 7.4.8.1} \newline
                  सं॒ॅव॒थ्स॒रायेति॑ सं - व॒थ्स॒राय॑ । दी॒क्षि॒ष्यमा॑णाः । ए॒का॒ष्ट॒काया॒मित्ये॑क - अ॒ष्ट॒काया᳚म् । दी॒क्षे॒र॒न्न् । ए॒षा । वै । सं॒ॅव॒थ्स॒रस्येति॑ सं - व॒थ्स॒रस्य॑ । पत्नी᳚ । यत् । ए॒का॒ष्ट॒केत्ये॑क-अ॒ष्ट॒का । ए॒तस्या᳚म् । वै । ए॒षः । ए॒ताम् । रात्रि᳚म् । व॒स॒ति॒ । सा॒क्षादिति॑ स - अ॒क्षात् । ए॒व । सं॒ॅव॒थ्स॒रमिति॑ सं -  व॒थ्स॒रम् । आ॒रभ्येत्या᳚ - रभ्य॑ । दी॒क्ष॒न्ते॒ । आर्त᳚म् । वै । ए॒ते । सं॒ॅव॒थ्स॒रस्येति॑ सं - व॒थ्स॒रस्य॑ । अ॒भीति॑ । दी॒क्ष॒न्ते॒ । ये । ए॒का॒ष्ट॒काया॒मित्ये॑क - अ॒ष्ट॒काया᳚म् । दीक्ष॑न्ते । अन्त॑नामाना॒वित्यन्त॑- ना॒मा॒नौ॒ । ऋ॒तू इति॑ । भ॒व॒तः॒ । व्य॑स्त॒मिति॒ वि-अ॒स्त॒म् । वै । ए॒ते । सं॒ॅव॒थ्स॒रस्येति॑ सं - व॒थ्स॒रस्य॑ । अ॒भीति॑ । दी॒क्ष॒न्ते॒ । ये । ए॒का॒ष्ट॒काया॒मित्ये॑क - अ॒ष्ट॒काया᳚म् । दीक्ष॑न्ते । अन्त॑नामाना॒वित्यन्त॑- ना॒मा॒नौ॒ । ऋ॒तू इति॑ । भ॒व॒तः॒ । फ॒ल्गु॒नी॒पू॒र्ण॒मा॒स इति॑ फल्गुनी- पू॒र्ण॒मा॒से । दी॒क्षे॒र॒न्न् । मुख᳚म् । वै । ए॒तत् । \textbf{  28} \newline
                  \newline
                                \textbf{ TS 7.4.8.2} \newline
                  सं॒ॅव॒थ्स॒रस्येति॑ सं - व॒थ्स॒रस्य॑ । यत् । फ॒ल्गु॒नी॒पू॒र्ण॒मा॒स इति॑ फल्गुनी-पू॒र्ण॒मा॒सः । मु॒ख॒तः । ए॒व । सं॒ॅव॒थ्स॒रमिति॑ सं-व॒थ्स॒रम् । आ॒रभ्येत्या᳚ - रभ्य॑ । दी॒क्ष॒न्ते॒ । तस्य॑ । एका᳚ । ए॒व । नि॒र्येति॑ निः - या । यथ् । सांमे᳚घ्य॒ इति॒ सां - मे॒घ्ये॒ । वि॒षू॒वानिति॑ विषु - वान् । स॒पंद्य॑त॒ इति॑ सं - पद्य॑ते । चि॒त्रा॒पू॒र्ण॒मा॒स इति॑ चित्रा - पू॒र्ण॒मा॒से । दी॒क्षे॒र॒न्न् । मुख᳚म् । वै । ए॒तत् । सं॒ॅव॒थ्स॒रस्येति॑ सं - व॒थ्स॒रस्य॑ । यत् । चि॒त्रा॒पू॒र्ण॒मा॒स इति॑ चित्रा - पू॒र्ण॒मा॒सः । मु॒ख॒तः । ए॒व । सं॒ॅव॒थ्स॒रमिति॑ सं - व॒थ्स॒रम् । आ॒रभ्येत्या᳚ - रभ्य॑ । दी॒क्ष॒न्ते॒ । तस्य॑ । न । का । च॒न । नि॒र्येति॑ निः - या । भ॒व॒ति॒ । च॒तु॒र॒ह इति॑ चतुः-अ॒हे । पु॒रस्ता᳚त् । पौ॒र्ण॒मा॒स्या इति॑ पौर्ण-मा॒स्यै । दी॒क्षे॒र॒न्न् । तेषा᳚म् । ए॒का॒ष्ट॒काया॒मित्ये॑क - अ॒ष्ट॒काया᳚म् । क्र॒यः । समिति॑ । प॒द्य॒ते॒ । तेन॑ । ए॒का॒ष्ट॒कामित्ये॑क-अ॒ष्ट॒काम् । न । छ॒म्बट् । कु॒र्व॒न्ति॒ । तेषा᳚म् । \textbf{  29} \newline
                  \newline
                                \textbf{ TS 7.4.8.3} \newline
                  पू॒र्व॒प॒क्ष इति॑ पूर्व - प॒क्षे । सु॒त्या । समिति॑ । प॒द्य॒ते॒ । पू॒र्व॒प॒क्षमिति॑ पूर्व - प॒क्षम् । मासाः᳚ । अ॒भि । समिति॑ । प॒द्य॒न्ते॒ । ते । पू॒र्व॒प॒क्ष इति॑ पूर्व-प॒क्षे । उदिति॑ । ति॒ष्ठ॒न्ति॒ । तान् । उ॒त्तिष्ठ॑त॒ इत्यु॑त्-तिष्ठ॑तः । ओष॑धयः । वन॒स्पत॑यः । अनु॑ । उदिति॑ । ति॒ष्ठ॒न्ति॒ । तान् । क॒ल्या॒णी । की॒र्तिः । अनु॑ । उदिति॑ । ति॒ष्ठ॒ति॒ । अरा᳚थ्सुः । इ॒मे । यज॑मानाः । इति॑ । तत् । अन्विति॑ । सर्वे᳚ । रा॒द्ध्नु॒व॒न्ति॒ ॥ \textbf{  30} \newline
                  \newline
                      (ए॒त - च्छ॒बंट्कु॑र्वन्ति॒ तेषां॒ - चतु॑स्त्रिꣳशच्च)  \textbf{(A8)} \newline \newline
                                \textbf{ TS 7.4.9.1} \newline
                  सु॒व॒र्गमिति॑ सुवः - गम् । वै । ए॒ते । लो॒कम् । य॒न्ति॒ । ये । स॒त्रम् । उ॒प॒यन्तीत्यु॑प - यन्ति॑ । अ॒भीति॑ । इ॒न्ध॒ते॒ । ए॒व । दी॒क्षाभिः॑ । आ॒त्मान᳚म् । श्र॒प॒य॒न्ते॒ । उ॒प॒सद्भि॒रित्यु॑प॒सत् - भिः॒ । द्वाभ्या᳚म् । लोम॑ । अवेति॑ । द्य॒न्ति॒ । द्वाभ्या᳚म् । त्वच᳚म् । द्वाभ्या᳚म् । असृ॑त् । द्वाभ्या᳚म् । माꣳ॒॒सम् । द्वाभ्या᳚म् । अस्थि॑ । द्वाभ्या᳚म् । म॒ज्जान᳚म् । आ॒त्मद॑क्षिण॒मित्या॒त्म - द॒क्षि॒ण॒म् । वै । स॒त्रम् । आ॒त्मान᳚म् । ए॒व । दक्षि॑णाम् । नी॒त्वा । सु॒व॒र्गमिति॑ सुवः - गम् । लो॒कम् । य॒न्ति॒ । शिखा᳚म् । अनु॑ । प्रेति॑ । व॒प॒न्ते॒ । ऋद्ध्यै᳚ । अथो॒ इति॑ । रघी॑याꣳसः । सु॒व॒र्गमिति॑ सुवः - गम् । लो॒कम् । अ॒या॒म॒ । इति॑ ( ) ॥ \textbf{  31 } \newline
                  \newline
                      (सु॒व॒र्गं - प॑ञ्चा॒शत्)  \textbf{(A9)} \newline \newline
                                \textbf{ TS 7.4.10.1} \newline
                  ब्र॒ह्म॒वा॒दिन॒ इति॑ ब्रह्म - वा॒दिनः॑ । व॒द॒न्ति॒ । अ॒ति॒रा॒त्र इत्य॑ति-रा॒त्रः । प॒र॒मः । य॒ज्ञ्॒क्र॒तू॒नामिति॑ यज्ञ्-क्र॒तू॒नाम् । कस्मा᳚त् । तम् । प्र॒थ॒मम् । उपेति॑ । य॒न्ति॒ । इति॑ । ए॒तत् । वै । अ॒ग्नि॒ष्टो॒ममित्य॑ग्नि - स्तो॒मम् । प्र॒थ॒मम् । उपेति॑ । य॒न्ति॒ । अथ॑ । उ॒क्थ्य᳚म् । अथ॑ । षो॒ड॒शिन᳚म् । अथ॑ । अ॒ति॒रा॒त्रमित्य॑ति - रा॒त्रम् । अ॒नु॒पू॒र्वमित्य॑नु - पू॒र्वम् । ए॒व । ए॒तत् । य॒ज्ञ्॒क्र॒तूनिति॑ यज्ञ् - क्र॒तून् । उ॒पेत्येत्यु॑प - इत्य॑ । तान् । आ॒लभ्येत्या᳚ - लभ्य॑ । प॒रि॒गृह्येति॑ परि - गृह्य॑ । सोम᳚म् । ए॒व । ए॒तत् । पिब॑न्तः । आ॒स॒ते॒ । ज्योति॑ष्टोम॒मिति॒ ज्योतिः॑ - स्तो॒म॒म् । प्र॒थ॒मम् । उपेति॑ । य॒न्ति॒ । ज्योति॑ष्टोम॒ इति॒ ज्योतिः॑ - स्तो॒मः॒ । वै । स्तोमा॑नाम् । मुख᳚म् । मु॒ख॒तः । ए॒व । स्तोमान्॑ । प्रेति॑ । यु॒ञ्ज॒ते॒ । ते । \textbf{  32} \newline
                  \newline
                                \textbf{ TS 7.4.10.2} \newline
                  सꣳस्तु॑ता॒ इति॒ सं - स्तु॒ताः॒ । वि॒राज॒मिति॑ वि - राज᳚म् । अ॒भि । समिति॑ । प॒द्य॒न्ते॒ । द्वे इति॑ । च॒ । ऋचौ᳚ । अतीति॑ । रि॒च्ये॒ते॒ इति॑ । एक॑या । गौः । अति॑रिक्त॒ इत्यति॑ - रि॒क्तः॒ । एक॑या । आयुः॑ । ऊ॒नः । सु॒व॒र्ग इति॑ सुवः - गः । वै । लो॒कः । ज्योतिः॑ । ऊर्क् । वि॒राडिति॑ वि - राट् । सु॒व॒र्गमिति॑ सुवः-गम् । ए॒व । तेन॑ । लो॒कम् । य॒न्ति॒ । र॒थ॒न्त॒रमिति॑ रथं - त॒रम् । दिवा᳚ । भव॑ति । र॒थ॒न्त॒रमिति॑ रथं-त॒रम् । नक्त᳚म् । इति॑ । आ॒हुः॒ । ब्र॒ह्म॒वा॒दिन॒ इति॑ ब्रह्म-वा॒दिनः॑ । केन॑ । तत् । अजा॑मि । इति॑ । सौ॒भ॒रम् । तृ॒ती॒य॒स॒व॒न इति॑ तृतीय - स॒व॒ने । ब्र॒ह्म॒सा॒ममिति॑ ब्रह्म - सा॒मम् । बृ॒हत् । तन् । म॒द्ध्य॒तः । द॒ध॒ति॒ । विधृ॑त्या॒ इति॒ वि - धृ॒त्यै॒ । तेन॑ । अजा॑मि ॥ \textbf{  33} \newline
                  \newline
                      (त - एका॒न्नप॑ञ्चा॒शच्च॑)  \textbf{(A10)} \newline \newline
                                \textbf{ TS 7.4.11.1} \newline
                  ज्योति॑ष्टोम॒मिति॒ ज्योतिः॑ - स्तो॒म॒म् । प्र॒थ॒मम् । उपेति॑ । य॒न्ति॒ । अ॒स्मिन्न् । ए॒व । तेन॑ । लो॒के । प्रतीति॑ । ति॒ष्ठ॒न्ति॒ । गोष्टो॑म॒मिति॒ गो - स्तो॒म॒म् । द्वि॒तीय᳚म् । उपेति॑ । य॒न्ति॒ । अ॒न्तरि॑क्षे । ए॒व । तेन॑ । प्रतीति॑ । ति॒ष्ठ॒न्ति॒ । आयु॑ष्टोम॒मित्यायुः॑ - स्ता॒म॒म् । तृ॒तीय᳚म् । उपेति॑ । य॒न्ति॒ । अ॒मुष्मिन्न्॑ । ए॒व । तेन॑ । लो॒के । प्रतीति॑ । ति॒ष्ठ॒न्ति॒ । इ॒यम् । वाव । ज्योतिः॑ । अ॒न्तरि॑क्षम् । गौः । अ॒सौ । आयुः॑ । यत् । ए॒तान् । स्तोमान्॑ । उ॒प॒यन्तीत्यु॑प - यन्ति॑ । ए॒षु । ए॒व । तत् । लो॒केषु॑ । स॒त्रिणः॑ । प्र॒ति॒तिष्ठ॑न्त॒ इति॑ प्रति - तिष्ठ॑न्तः । य॒न्ति॒ । ते । सꣳस्तु॑ता॒ इति॒ सं - स्तु॒ताः॒ । वि॒राज॒मिति॑ वि - राज᳚म् । \textbf{  34} \newline
                  \newline
                                \textbf{ TS 7.4.11.2} \newline
                  अ॒भि । समिति॑ । प॒द्य॒न्ते॒ । द्वे इति॑ । च॒ । ऋचौ᳚ । अतीति॑ । रि॒च्ये॒ते॒ इति॑ । एक॑या । गौः । अति॑रिक्त॒ इत्यति॑ - रि॒क्तः॒ । एक॑या । आयुः॑ । ऊ॒नः । सु॒व॒र्ग इति॑ सुवः - गः । वै । लो॒कः । ज्योतिः॑ । ऊर्क् । वि॒राडिति॑ वि - राट् । ऊर्ज᳚म् । ए॒व । अवेति॑ । रु॒न्ध॒ते॒ । ते । न । क्षु॒धा । आर्ति᳚म् । एति॑ । ऋ॒च्छ॒न्ति॒ । अक्षो॑धुकाः । भ॒व॒न्ति॒ । क्षुथ्स॑म्बाधा॒ इति॒ क्षुत् - स॒म्बा॒धाः॒ । इ॒व॒ । हि । स॒त्रिणः॑ । अ॒ग्नि॒ष्टो॒मावित्य॑ग्नि - स्तो॒मौ । अ॒भितः॑ । प्र॒धी इति॑ प्र -धी । तौ । उ॒क्थ्याः᳚ । मद्ध्ये᳚ । नभ्य᳚म् । तत् । तत् । ए॒तत् । प॒रि॒यदिति॑ परि - यत् । दे॒व॒च॒क्रमिति॑ देव - च॒क्रम् । यत् । ए॒तेन॑ । \textbf{  35} \newline
                  \newline
                                \textbf{ TS 7.4.11.3} \newline
                  ष॒ड॒हेनेति॑ षट् - अ॒हेन॑ । यन्ति॑ । दे॒व॒च॒क्रमिति॑ देव-च॒क्रम् । ए॒व । स॒मारो॑ह॒न्तीति॑ सं - आरो॑हन्ति । अरि॑ष्ट्यै । ते । स्व॒स्ति । समिति॑ । अ॒श्नु॒व॒ते॒ । ष॒ड॒हेनेति॑ षट् - अ॒हेन॑ । य॒न्ति॒ । षट् । वै । ऋ॒तवः॑ । ऋ॒तुषु॑ । ए॒व । प्रतीति॑ । ति॒ष्ठ॒न्ति॒ । उ॒भ॒यता᳚ज्योति॒षेत्यु॑भ॒यतः॑-ज्यो॒ति॒षा॒ । य॒न्ति॒ । उ॒भ॒यतः॑ । ए॒व । सु॒व॒र्ग इति॑ सुवः - गे । लो॒के । प्र॒ति॒तिष्ठ॑न्त॒ इति॑ प्रति-तिष्ठ॑न्तः । य॒न्ति॒ । द्वौ । ष॒ड॒हाविति॑ षट्-अ॒हौ । भ॒व॒तः॒ । तानि॑ । द्वाद॑श । अहा॑नि । समिति॑ । प॒द्य॒न्ते॒ । द्वा॒द॒शः । वै । पुरु॑षः । द्वे इति॑ । स॒क्थ्यौ᳚ । द्वौ । बा॒हू इति॑ । आ॒त्मा । च॒ । शिरः॑ । च॒ । च॒त्वारि॑ । अङ्गा॑नि । स्तनौ᳚ । द्वा॒द॒शौ । \textbf{  36} \newline
                  \newline
                                \textbf{ TS 7.4.11.4} \newline
                  तत् । पुरु॑षम् । अन्विति॑ । प॒र्याव॑र्तन्त॒ इति॑ परि - आव॑र्तन्ते । त्रयः॑ । ष॒ड॒हा इति॑ षट् - अ॒हाः । भ॒व॒न्ति॒ । तानि॑ । अ॒ष्टाद॒शेत्य॒ष्टा - द॒श॒ । अहा॑नि । समिति॑ । प॒द्य॒न्ते॒ । नव॑ । अ॒न्यानि॑ । नव॑ । अ॒न्यानि॑ । नव॑ । वै । पुरु॑षे । प्रा॒णा इति॑ प्र - अ॒नाः । तत् । प्रा॒णानिति॑ प्र-अ॒नान् । अन्विति॑ । प॒र्याव॑र्तन्त॒ इति॑ परि - आव॑र्तन्ते । च॒त्वारः॑ । ष॒ड॒हा इति॑ षट्- अ॒हाः । भ॒व॒न्ति॒ । तानि॑ । चतु॑र्विꣳशति॒रिति॒ चतुः॑-विꣳ॒॒श॒तिः॒ । अहा॑नि । समिति॑ । प॒द्य॒न्ते॒ । चतु॑र्विꣳशति॒रिति॒ चतुः॑ - विꣳ॒॒श॒तिः॒ । अ॒द्‌र्ध॒मा॒सा इत्यद्‌र्ध - मा॒साः । सं॒ॅव॒थ्स॒र इति॑ सं - व॒थ्स॒रः । तत् । सं॒ॅव॒थ्स॒रमिति॑ सं - व॒थ्स॒रम् । अन्विति॑ । प॒र्याव॑र्तन्त॒ इति॑ परि - आव॑र्तन्ते । अप्र॑तिष्ठित॒ इत्यप्र॑ति - स्थि॒तः॒ । सं॒ॅव॒थ्स॒र इति॑ सं - व॒थ्स॒रः । इति॑ । खलु॑ । वै । आ॒हुः॒ । वर्.षी॑यान् । प्र॒ति॒ष्ठाया॒ इति॑ प्रति - स्थायाः᳚ । इति॑ । ए॒ताव॑त् । वै ( ) । सं॒ॅव॒थ्स॒रस्येति॑ सं - व॒थ्स॒रस्य॑ । ब्राह्म॑णम् । याव॑त् । मा॒सः । मा॒सिमा॒सीति॑ मा॒सि - मा॒सि॒ । ए॒व । प्र॒ति॒तिष्ठ॑न्त॒ इति॑ प्रति - तिष्ठ॑न्तः । य॒न्ति॒ ॥ \textbf{  37} \newline
                  \newline
                      (वि॒राज॑ - मे॒तेन॑ - द्वाद॒शा - वे॒ताव॒द्वा - अ॒ष्टौ च॑)  \textbf{(A11)} \newline \newline
                                \textbf{ TS 7.4.12.1} \newline
                  मे॒षः । त्वा॒ । प॒च॒तैः । अ॒व॒तु॒ । लोहि॑तग्रीव॒ इति॒ लोहि॑त - ग्री॒वः॒ । छागैः᳚ । श॒ल्म॒लिः । वृद्ध्या᳚ । प॒र्णः । ब्रह्म॑णा । प्ल॒क्षः । मेधे॑न । न्य॒ग्रोधः॑ । च॒म॒सैः । उ॒दु॒म्बरः॑ । ऊ॒र्जा । गा॒य॒त्री । छन्दो॑भि॒रिति॒ छन्दः॑ - भिः॒ । त्रि॒वृदिति॑ त्रि - वृत् । स्तोमैः᳚ । अव॑न्तीः । स्थ॒ । अव॑न्तीः । त्वा॒ । अ॒व॒न्तु॒ । प्रि॒यम् । त्वा॒ । प्रि॒याणा᳚म् । वर्.षि॑ष्ठम् । आप्या॑नाम् । नि॒धी॒नामिति॑ नि - धी॒नाम् । त्वा॒ । नि॒धि॒पति॒मिति॑ निधि - पति᳚म् । ह॒वा॒म॒हे॒ । व॒सो॒ इति॑ । म॒म॒ ॥ \textbf{  38} \newline
                  \newline
                      (मे॒षः - षट् त्रिꣳ॑शत्)  \textbf{(A12)} \newline \newline
                                \textbf{ TS 7.4.13.1} \newline
                  कूप्या᳚भ्यः । स्वाहा᳚ । कूल्या᳚भ्यः । स्वाहा᳚ । वि॒क॒र्या᳚भ्य॒ इति॑ वि - क॒र्या᳚भ्यः । स्वाहा᳚ । अ॒व॒ट्या᳚भ्यः । स्वाहा᳚ । खन्या᳚भ्यः । स्वाहा᳚ । ह्रद्या᳚भ्यः । स्वाहा᳚ । सूद्या᳚भ्यः । स्वाहा᳚ । स॒र॒स्या᳚भ्यः । स्वाहा᳚ । वै॒श॒न्तीभ्यः॑ । स्वाहा᳚ । प॒ल्व॒ल्या᳚भ्यः । स्वाहा᳚ । वर्ष्या᳚भ्यः । स्वाहा᳚ । अ॒व॒र्ष्याभ्यः॑ । स्वाहा᳚ । ह्रा॒दुनी᳚भ्य॒ इति॑ ह्रा॒दुनि॑ - भ्यः॒ । स्वाहा᳚ । पृष्वा᳚भ्यः । स्वाहा᳚ । स्यन्द॑मानाभ्यः । स्वाहा᳚ । स्था॒व॒राभ्यः॑ । स्वाहा᳚ । ना॒दे॒यीभ्यः॑ । स्वाहा᳚ । सै॒न्ध॒वीभ्यः॑ । स्वाहा᳚ । स॒मु॒द्रिया᳚भ्यः । स्वाहा᳚ । सर्वा᳚भ्यः । स्वाहा᳚ ॥ \textbf{  39} \newline
                  \newline
                      (कूप्या᳚भ्य - श्चत्वारिꣳ॒॒शत्)  \textbf{(A13)} \newline \newline
                                \textbf{ TS 7.4.14.1} \newline
                  अ॒द्भ्य इत्य॑त् - भ्यः । स्वाहा᳚ । वह॑न्तीभ्यः । स्वाहा᳚ । प॒रि॒वह॑न्तीभ्य॒ इति॑ परि - वह॑न्तीभ्यः । स्वाहा᳚ । स॒म॒न्तमिति॑ सं-अ॒न्तम् । वह॑न्तीभ्यः । स्वाहा᳚ । शीघ्र᳚म् । वह॑न्तीभ्यः । स्वाहा᳚ । शीभ᳚म् । वह॑न्तीभ्यः । स्वाहा᳚ । उ॒ग्रम् । वह॑न्तीभ्यः । स्वाहा᳚ । भी॒मम् । वह॑न्तीभ्यः । स्वाहा᳚ । अम्भो᳚भ्य॒ इत्यम्भः॑-भ्यः॒ । स्वाहा᳚ । नभो᳚भ्य॒ इति॒ नभः॑ - भ्यः॒ । स्वाहा᳚ । महो᳚भ्य॒ इति॒ महः॑ - भ्यः॒ । स्वाहा᳚ । सर्व॑स्मै । स्वाहा᳚ ॥ \textbf{  40} \newline
                  \newline
                      (अ॒द्भ्य - एका॒न्नत्रिꣳ॒॒शत्)  \textbf{(A14)} \newline \newline
                                \textbf{ TS 7.4.15.1} \newline
                  यः । अर्व॑न्तम् । जिघाꣳ॑सति । तम् । अ॒भीति॑ । अ॒मी॒ति॒ । वरु॑णः ॥ प॒रः । मर्तः॑ । प॒रः । श्वा ॥ अ॒हम् । च॒ । त्वम् । च॒ । वृ॒त्र॒ह॒न्निति॑ वृत्र - ह॒न्न् । समिति॑ । ब॒भू॒व॒ । स॒निभ्य॒ इति॑ स॒नि - भ्यः॒ । आ ॥ अ॒रा॒ती॒वा । चि॒त् । अ॒द्रि॒व॒ इत्य॑द्रि - वः॒ । अन्विति॑ । नौ॒ । शू॒र॒ । मꣳ॒॒स॒तै॒ । भ॒द्राः । इन्द्र॑स्य । रा॒तयः॑ ॥ अ॒भीति॑ । क्रत्वा᳚ । इ॒न्द्र॒ । भूः॒ । अध॑ । ज्मन्न् । न । ते॒ । वि॒व्य॒क् । म॒हि॒मान᳚म् । रजाꣳ॑सि ॥ स्वेन॑ । हि । वृ॒त्रम् । शव॑सा । ज॒घन्थ॑ । न । शत्रुः॑ । अन्त᳚म् । वि॒वि॒द॒त् ( ) । यु॒धा । ते॒ ॥ \textbf{  41} \newline
                  \newline
                      (वि॒वि॒द॒द् - द्वे च॑)  \textbf{(A15)} \newline \newline
                                \textbf{ TS 7.4.16.1} \newline
                  नमः॑ । राज्ञे᳚ । नमः॑ । वरु॑णाय । नमः॑ । अश्वा॑य । नमः॑ । प्र॒जाप॑तय॒ इति॑ प्र॒जा - प॒त॒ये॒ । नमः॑ । अधि॑पतय॒ इत्यधि॑ - प॒त॒ये॒ । अधि॑पति॒रित्यधि॑ - प॒तिः॒ । अ॒सि॒ । अधि॑पति॒मित्यधि॑ - प॒ति॒म् । मा॒ । कु॒रु॒ । अधि॑पति॒रित्यधि॑ - प॒तिः॒ । अ॒हम् । प्र॒जाना॒मिति॑ प्र - जाना᳚म् । भू॒या॒स॒म् । माम् । धे॒हि॒ । मयि॑ । धे॒हि॒ । उ॒पाकृ॑ता॒येत्यु॑प - आकृ॑ताय । स्वाहा᳚ । आल॑ब्धा॒येत्या - ल॒ब्धा॒य॒ । स्वाहा᳚ । हु॒ताय॑ । स्वाहा᳚ ॥ \textbf{  42} \newline
                  \newline
                      (नम॒ - एका॒न्न त्रिꣳ॒॒शत्)  \textbf{(A16)} \newline \newline
                                \textbf{ TS 7.4.17.1} \newline
                  म॒यो॒भूरिति॑ मयः - भूः । वातः॑ । अ॒भीति॑ । वा॒तु॒ । उ॒स्राः । ऊर्ज॑स्वतीः । ओष॑धीः । एति॑ । रि॒श॒न्ता॒म् ॥ पीव॑स्वतीः । जी॒वध॑न्या॒ इति॑ जी॒व - ध॒न्याः॒ । पि॒ब॒न्तु॒ । अ॒व॒साय॑ । प॒द्वत॒ इति॑ पत् - वते᳚ । रु॒द्र॒ । मृ॒ड॒ ॥ याः । सरू॑पा॒ इति॒ स - रू॒पाः॒ । विरू॑पा॒ इति॒ वि - रू॒पाः॒ । एक॑रूपा॒ इत्येक॑-रू॒पाः॒ । यासा᳚म् । अ॒ग्निः । इष्ट्या᳚ । नामा॑नि । वेद॑ ॥ याः । अङ्गि॑रसः । तप॑सा । इ॒ह । च॒क्रुः । ताभ्यः॑ । प॒र्ज॒न्य॒ । महि॑ । शर्म॑ । य॒च्छ॒ ॥ याः । दे॒वेषु॑ । त॒नुव᳚म् । ऐर॑यन्त । यासा᳚म् । सोमः॑ । विश्वा᳚ । रू॒पाणि॑ । वेद॑ ॥ ताः । अ॒स्मभ्य॒मित्य॒स्म - भ्य॒म् । पय॑सा । पिन्व॑मानाः । प्र॒जाव॑ती॒रिति॑ प्र॒जा - व॒तीः॒ । इ॒न्द्र॒ । \textbf{  43} \newline
                  \newline
                                \textbf{ TS 7.4.17.2} \newline
                  गो॒ष्ठ इति॑ गो - स्थे । रि॒री॒हि॒ ॥ प्र॒जाप॑ति॒रिति॑ प्र॒जा-प॒तिः॒ । मह्य᳚म् । ए॒ताः । ररा॑णः । विश्वैः᳚ । दे॒वैः । पि॒तृभि॒रिति॑ पि॒तृ - भिः॒ । सं॒ॅवि॒दा॒न इति॑ सं - वि॒दा॒नः ॥ शि॒वाः । स॒तीः । उपेति॑ । नः॒ । गो॒ष्ठमिति॑ गो - स्थम् । एति॑ । अ॒कः॒ । तासा᳚म् । व॒यम् । प्र॒जयेति॑ प्र-जया᳚ । समिति॑ । स॒दे॒म॒ ॥ इ॒ह । धृतिः॑ । स्वाहा᳚ । इ॒ह । विधृ॑ति॒रिति॒ वि - धृ॒तिः॒ । स्वाहा᳚ । इ॒ह । रन्तिः॑ । स्वाहा᳚ । इ॒ह । रम॑तिः । स्वाहा᳚ । म॒हीम् । उ॒ । स्विति॑ । सु॒त्रामा॑ण॒मिति॑ सु - त्रामा॑णम् ॥ \textbf{  44 } \newline
                  \newline
                      (इ॒न्द्रा॒ - ष्टात्रिꣳ॑शच्च)  \textbf{(A17)} \newline \newline
                                \textbf{ TS 7.4.18.1} \newline
                  किम् । स्वि॒त् । आ॒सी॒त् । पू॒र्वचि॑त्ति॒रिति॑ पू॒र्व - चि॒त्तिः॒ । किम् । स्वि॒त् । आ॒सी॒त् । बृ॒हत् । वयः॑ ॥ किम् । स्वि॒त् । आ॒सी॒त् । पि॒श॒ङ्गि॒ला । किम् । स्वि॒त् । आ॒सी॒त् । पि॒लि॒प्पि॒ला ॥ द्यौः । आ॒सी॒त् । पू॒र्वचि॑त्ति॒रिति॑ पू॒र्व - चि॒त्तिः॒ । अश्वः॑ । आ॒सी॒त् । बृ॒हत् । वयः॑ ॥ रात्रिः॑ । आ॒सी॒त् । पि॒श॒ङ्गि॒ला । अविः॑ । आ॒सी॒त् । पि॒लि॒प्पि॒ला ॥ कः । स्वि॒त् । ए॒का॒की । च॒र॒ति॒ । कः । उ॒ । स्वि॒त् । जा॒य॒ते॒ । पुनः॑ ॥ किम् । स्वि॒त् । हि॒मस्य॑ । भे॒ष॒जम् । किम् । स्वि॒त् । आ॒वप॑न॒मित्या᳚ - वप॑नम् । म॒हत् ॥ सूर्यः॑ । ए॒का॒की । च॒र॒ति॒ । \textbf{  45} \newline
                  \newline
                                \textbf{ TS 7.4.18.2} \newline
                  च॒न्द्रमाः᳚ । जा॒य॒ते॒ । पुनः॑ ॥ अ॒ग्निः । हि॒मस्य॑ । भे॒ष॒जम् । भूमिः॑ । आ॒वप॑न॒मित्या᳚ - वप॑नम् । म॒हत् ॥ पृ॒च्छामि॑ । त्वा॒ । पर᳚म् । अन्त᳚म् । पृ॒थि॒व्याः । पृ॒च्छामि॑ । त्वा॒ । भुव॑नस्य । नाभि᳚म् ॥ पृ॒च्छामि॑ । त्वा॒ । वृष्णः॑ । अश्व॑स्य । रेतः॑ । पृ॒च्छामि॑ । वा॒चः । प॒र॒मम् । व्यो॑मेति॒ वि - ओ॒म॒ ॥ वेदि᳚म् । आ॒हुः॒ । पर᳚म् । अन्त᳚म् । पृ॒थि॒व्याः । य॒ज्ञ्म् । आ॒हुः॒ । भुव॑नस्य । नाभि᳚म् ॥ सोम᳚म् । आ॒हुः॒ । वृष्णः॑ । अश्व॑स्य । रेतः॑ । ब्रह्म॑ । ए॒व । वा॒चः । प॒र॒मम् । व्यो॑मेति॒ वि - ओ॒म॒ ॥ \textbf{  46} \newline
                  \newline
                      (सूर्य॑ एका॒की च॑रति॒ - षट्च॑त्वारिꣳशच्च)  \textbf{(A18)} \newline \newline
                                \textbf{ TS 7.4.19.1} \newline
                  अबें᳚ । अम्बा॑लि । अम्बि॑के । न । मा॒ । न॒य॒ति॒ । कः । च॒न ॥ स॒सस्ति॑ । अ॒श्व॒कः ॥ सुभ॑ग॒ इति॒ सु - भ॒गे॒ । काम्पी॑लवासि॒नीति॒ काम्पी॑ल - वा॒सि॒नि॒ । सु॒व॒र्ग इति॑ सुवः-गे । लो॒के । सम् । प्रेति॑ । ऊ॒र्ण्वा॒था॒म् ॥ एति॑ । अ॒हम् । अ॒जा॒नि॒ । ग॒र्भ॒धमिति॑ गर्भ - धम् । एति॑ । त्वम् । अ॒जा॒सि॒ । ग॒र्भ॒धमिति॑ गर्भ - धम् ॥ तौ । स॒ह । च॒तुरः॑ । प॒दः । सम् । प्रेति॑ । सा॒र॒या॒व॒है॒ ॥ वृषा᳚ । वा॒म् । रे॒तो॒धा इति॑ रेतः - धाः । रेतः॑ । द॒धा॒तु॒ । उदिति॑ । स॒क्थ्योः᳚ । गृ॒दम् । धे॒हि॒ । अ॒ञ्जिम् । उद॑ञ्जि॒मित्युत्-अ॒ञ्जि॒म् । अन्विति॑ । अ॒ज॒ ॥ यः । स्त्री॒णाम् । जी॒व॒भोज॑न॒ इति॑ जीव - भोज॑नः । यः । आ॒सा॒म् । \textbf{  47} \newline
                  \newline
                                \textbf{ TS 7.4.19.2} \newline
                  बि॒ल॒धाव॑न॒ इति॑ बिल - धाव॑नः ॥ प्रि॒यः । स्त्री॒णाम् । अ॒पी॒च्यः॑ ॥ यः । आ॒सा॒म् । कृ॒ष्णे । लक्ष्म॑णि । सर्दि॑गृदिम् । प॒राव॑धी॒दिति॑ परा - अव॑धीत् ॥ अम्बे᳚ । अम्बा॑लि । अम्बि॑के । न । मा॒ । य॒भ॒ति॒ । कः । च॒न ॥ स॒सस्ति॑ । अ॒श्व॒कः ॥ ऊ॒द्‌र्ध्वाम् । ए॒ना॒म् । उदिति॑ । श्र॒य॒ता॒त् । वे॒णु॒भा॒रमिति॑ वेणु-भा॒रम् । गि॒रौ । इ॒व॒ ॥ अथ॑ । अ॒स्याः॒ । मद्ध्य᳚म् । ए॒ध॒ता॒म् । शी॒ते । वाते᳚ । पु॒नन्न् । इ॒व॒ ॥ अम्बे᳚ । अम्बा॑लि । अम्बि॑के । न । मा॒ । य॒भ॒ति॒ । कः । च॒न ॥ स॒सस्ति॑ । अ॒श्व॒कः ॥ यत् । ह॒रि॒णी । यव᳚म् । अत्ति॑ । न । \textbf{  48} \newline
                  \newline
                                \textbf{ TS 7.4.19.3} \newline
                  पु॒ष्टम् । प॒शु । म॒न्य॒ते॒ ॥ शू॒द्रा । यत् । अर्य॑जा॒रेत्यर्य॑ - जा॒रा॒ । न । पोषा॑य । ध॒ना॒य॒ति॒ ॥ अम्बे᳚ । अम्बा॑लि । अम्बि॑के । न । मा॒ । य॒भ॒ति॒ । कः । च॒न ॥ स॒सस्ति॑ । अ॒श्व॒कः ॥ इ॒यम् । य॒का । श॒कु॒न्ति॒का । आ॒हल॒मित्या᳚-हल᳚म् । इति॑ । सर्प॑ति ॥ आह॑त॒मित्या-ह॒त॒म् । ग॒भे । पसः॑ । नीति॑ । ज॒ल्गु॒ली॒ति॒ । धाणि॑का ॥ अम्बे᳚ । अम्बा॑लि । अम्बि॑के । न । मा॒ । य॒भ॒ति॒ । कः । च॒न ॥ स॒सस्ति॑ । अ॒श्व॒कः ॥ मा॒ता । च॒ । ते॒ । पि॒ता । च॒ । ते॒ । अग्र᳚म् । वृ॒क्षस्य॑ । रो॒ह॒तः॒ ( ) ॥ \textbf{  49} \newline
                  \newline
                                \textbf{ TS 7.4.19.4} \newline
                  प्रेति॑ । सु॒ला॒मि॒ । इति॑ । ते॒ । पि॒ता । ग॒भे । मु॒ष्टिम् । अ॒तꣳ॒॒स॒य॒त् ॥ द॒धि॒क्राव्.ण्ण॒ इति॑ दधि-क्राव्.ण्णः॑ । अ॒का॒रि॒ष॒म् । जि॒ष्णोः । अश्व॑स्य । वा॒जिनः॑ ॥ सु॒र॒भि । नः॒ । मुखा᳚ । क॒र॒त् । प्रेति॑ । नः॒ । आयूꣳ॑षि । ता॒रि॒ष॒त् ॥ आपः॑ । हि । स्थ । म॒यो॒भुव॒ इति॑ मयः - भुवः॑ । ताः । नः॒ । ऊ॒र्जे । द॒धा॒त॒न॒ ॥ म॒हे । रणा॑य । चक्ष॑से ॥ यः । वः॒ । शि॒वत॑म॒ इति॑ शि॒व - त॒मः॒ । रसः॑ । तस्य॑ । भा॒ज॒य॒त॒ । इ॒ह । नः॒ ॥ उ॒श॒तीः । इ॒व॒ । मा॒तरः॑ ॥ तस्मै᳚ । अर᳚म् । ग॒मा॒म॒ । वः॒ । यस्य॑ । क्षया॑य । जिन्व॑थ ( ) ॥ आपः॑ । ज॒नय॑थ । च॒ । नः॒ ॥ \textbf{  50} \newline
                  \newline
                      (आ॒सा॒ - मत्ति॒ न - रो॑हतो॒ - जिन्व॑थ - च॒त्वारि॑ च)  \textbf{(A19)} \newline \newline
                                \textbf{ TS 7.4.20.1} \newline
                  भूः । भुवः॑ । सुवः॑ । वस॑वः । त्वा॒ । अ॒ञ्ज॒न्तु॒ । गा॒य॒त्रेण॑ । छन्द॑सा । रु॒द्राः । त्वा॒ । अ॒ञ्ज॒न्तु॒ । त्रैष्टु॑भेन । छन्द॑सा । आ॒दि॒त्याः । त्वा॒ । अ॒ञ्ज॒न्तु॒ । जाग॑तेन । छन्द॑सा । यत् । वातः॑ । अ॒पः । अग॑मत् । इन्द्र॑स्य । त॒नुव᳚म् । प्रि॒याम् ॥ ए॒तम् । स्तो॒तः॒ । ए॒तेन॑ । प॒था । पुनः॑ । अश्व᳚म् । एति॑ । व॒र्त॒या॒सि॒ । नः॒ ॥ लाजी(3)न् । शाची(3)न् । यशः॑ । म॒मा(4)ॅम् ॥ य॒व्यायै᳚ । ग॒व्यायै᳚ । ए॒तत् । दे॒वाः॒ । अन्न᳚म् । अ॒त्त॒ । ए॒तत् । अन्न᳚म् । अ॒द्धि॒ । प्र॒जा॒प॒त॒ इति॑ प्रजा - प॒ते॒ ॥ यु॒ञ्जन्ति॑ । ब्र॒द्ध्नम् ( ) । अ॒रु॒षम् । चर॑न्तम् । परीति॑ । त॒स्थुषः॑ ॥ रोच॑न्ते । रो॒च॒ना । दि॒वि ॥ यु॒ञ्जन्ति॑ । अ॒स्य॒ । काम्या᳚ । हरी॒ इति॑ । विप॑क्ष॒सेति॒ वि - प॒क्ष॒सा॒ । रथे᳚ ॥ शोणा᳚ । धृ॒ष्णू इति॑ । नृ॒वाह॒सेति॑ नृ-वाह॑सा ॥ के॒तुम् । कृ॒ण्वन्न् । अ॒के॒तवे᳚ । पेशः॑ । म॒र्याः॒ । अ॒पे॒शसे᳚ ॥ समिति॑ । उ॒षद्भि॒रित्यु॒षत् - भिः॒ । अ॒जा॒य॒थाः॒ ॥ \textbf{  51} \newline
                  \newline
                      (ब्र॒द्ध्नं - पञ्च॑विꣳशतिश्च)  \textbf{(A20)} \newline \newline
                                \textbf{ TS 7.4.21.1} \newline
                  प्रा॒णायेति॑ प्र - अ॒नाय॑ । स्वाहा᳚ । व्या॒नायेति॑ वि - अ॒नाय॑ । स्वाहा᳚ । अ॒पा॒नायेत्य॑प - अ॒नाय॑ । स्वाहा᳚ । स्नाव॑भ्य॒ इति॒ स्नाव॑ - भ्यः॒ । स्वाहा᳚ । स॒न्ता॒नेभ्य॒ इति॑ सं - ता॒नेभ्यः॑ । स्वाहा᳚ । परि॑सन्तानेभ्य॒ इति॒ परि॑ - स॒न्ता॒ने॒भ्यः॒ । स्वाहा᳚ । पर्व॑भ्य॒ इति॒ पर्व॑-भ्यः॒ । स्वाहा᳚ । स॒धांने᳚भ्य॒ इति॑ सं - धाने᳚भ्यः । स्वाहा᳚ । शरी॑रेभ्यः । स्वाहा᳚ । य॒ज्ञाय॑ । स्वाहा᳚ । दक्षि॑णाभ्यः । स्वाहा᳚ । सु॒व॒र्गायेति॑ सुवः - गाय॑ । स्वाहा᳚ । लो॒काय॑ । स्वाहा᳚ । सर्व॑स्मै । स्वाहा᳚ ॥ \textbf{  52} \newline
                  \newline
                      (प्रा॒णाया॒ - ष्टाविꣳ॑शतिः)  \textbf{(A21)} \newline \newline
                                \textbf{ TS 7.4.22.1} \newline
                  सि॒ताय॑ । स्वाहा᳚ । असि॑ताय । स्वाहा᳚ । अ॒भिहि॑ता॒येत्य॒भि - हि॒ता॒य॒ । स्वाहा᳚ । अन॑भिहिता॒येत्यन॑भि - हि॒ता॒य॒ । स्वाहा᳚ । यु॒क्ताय॑ । स्वाहा᳚ । अयु॑क्ताय । स्वाहा᳚ । सुयु॑क्ता॒येति॒ सु - यु॒क्ता॒य॒ । स्वाहा᳚ । उद्यु॑क्ता॒येत्युत् - यु॒क्ता॒य॒ । स्वाहा᳚ । विमु॑क्ता॒येति॒ वि - मु॒क्ता॒य॒ । स्वाहा᳚ । प्रमु॑क्ता॒येति॒ प्र - मु॒क्ता॒य॒ । स्वाहा᳚ । वञ्च॑ते । स्वाहा᳚ । प॒रि॒वञ्च॑त॒ इति॑ परि - वञ्च॑ते । स्वाहा᳚ । सं॒ॅवञ्च॑त॒ इति॑ सं - वञ्च॑ते । स्वाहा᳚ । अ॒नु॒वञ्च॑त॒ इत्य॑नु - वञ्च॑ते । स्वाहा᳚ । उ॒द्वञ्च॑त॒ इत्यु॑त्-वञ्च॑ते । स्वाहा᳚ । य॒ते । स्वाहा᳚ । धाव॑ते । स्वाहा᳚ । तिष्ठ॑ते । स्वाहा᳚ । सर्व॑स्मै । स्वाहा᳚ ॥ \textbf{  53} \newline
                  \newline
                      (सि॒ताया॒ - ष्टात्रिꣳ॑शत्)  \textbf{(A22)} \newline \newline
\textbf{praSna korvai with starting padams of 1 to 22 anuvAkams :-} \newline
(बृह॒स्पतिः॒ श्रद् - य॒था वा - ऋ॒क्षा वै - प्र॒जाप॑ति॒र्येन॑येन॒ - द्वे वाव दे॑वस॒त्रे - आ॑दि॒त्या अ॑कामयन्त सुव॒र्गं - ॅवसि॑ष्ठः - संॅवथ्स॒राय॑ -सुर्व॒र्गं ॅये स॒त्रं - ब्र॑ह्मवा॒दिनो॑ऽतिरा॒त्रो - ज्योति॑ष्टोमं - मे॒षः - कूप्या᳚भ्यो॒ - ऽद्भ्यो - यो - नमो॑ - मयो॒भूः - किꣳ स्वि॒द - म्बे॒ - भूः - प्रा॒णाय॑ - सि॒ताय॒ - द्वाविꣳ॑शतिः) \newline

\textbf{korvai with starting padams of1, 11, 21 series of pa~jcAtis :-} \newline
(बृह॒स्पतिः॒ - प्रति॑ तिष्ठन्ति॒ - वै द॑शरा॒त्रेण॑ - सुव॒र्गं - ॅयो अर्व॑न्तं॒ - भू - स्त्रिप॑ञ्चा॒शत्) \newline

\textbf{first and last padam of fourth praSnam of 7th kANDam} \newline
(बृह॒स्पतिः॒ - सर्व॑स्मै॒ स्वाहा᳚) \newline 


॥ हरिः॑ ॐ ॥
॥ कृष्ण यजुर्वेदीय तैत्तिरीय संहितायां सप्तमकाण्डे चतुर्त्थः प्रश्नः समाप्तः ॥ \newline
\pagebreak
7.4.1   AppEndix\\7.4.17.2 - म॒हीमू॒षु>1 सु॒त्रामा॑णं> 2 \\म॒हीमू॒षु मा॒तरꣳ॑ सुव्र॒ताना॑मृ॒तस्य॒ पत्नी॒मव॑से हुवेम । \\तु॒वि॒क्ष॒त्राम॒जर॑न्तीमुरू॒चीꣳ सु॒शर्मा॑ण॒मदि॑तिꣳ सु॒प्रणी॑तिं ॥ \\\\सु॒त्रामा॑णं पृथि॒वीं द्याम॑ने॒हसꣳ॑ सु॒शर्मा॑ण॒ मदि॑तिꣳ सु॒प्रणी॑तिं । \\दैवीं॒ नावꣳ॑ स्वरि॒त्रामना॑गस॒मस्र॑वन्ती॒मा रु॑हेमा स्व॒स्तये᳚ ॥ \\(आप्पेअरिन्ग् इन् ट्.श्.1.5.11.5)\\==============================\\
\pagebreak
        


\end{document}
