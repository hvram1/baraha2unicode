\documentclass[17pt]{extarticle}
\usepackage{babel}
\usepackage{fontspec}
\usepackage{polyglossia}
\usepackage{extsizes}



\setmainlanguage{sanskrit}
\setotherlanguages{english} %% or other languages
\setlength{\parindent}{0pt}
\pagestyle{myheadings}
\newfontfamily\devanagarifont[Script=Devanagari]{AdishilaVedic}


\newcommand{\VAR}[1]{}
\newcommand{\BLOCK}[1]{}




\begin{document}
\begin{titlepage}
    \begin{center}
 
\begin{sanskrit}
    { \Large
    ॐ नमः परमात्मने, श्री महागणपतये नमः, श्री गुरुभ्यो नमः
ह॒रिः॒ ॐ
========================================= 
    }
    \\
    \vspace{2.5cm}
    \mbox{ \Huge
    3.1     तृतीयकाण्डे प्रथमः प्रश्नः - न्यूनकर्माभिधानं   }
\end{sanskrit}
\end{center}

\end{titlepage}
\tableofcontents

ॐ नमः परमात्मने, श्री महागणपतये नमः, श्री गुरुभ्यो नमः
ह॒रिः॒ ॐ
========================================= \newline
3.1     तृतीयकाण्डे प्रथमः प्रश्नः - न्यूनकर्माभिधानं \newline

\addcontentsline{toc}{section}{ 3.1     तृतीयकाण्डे प्रथमः प्रश्नः - न्यूनकर्माभिधानं}
\markright{ 3.1     तृतीयकाण्डे प्रथमः प्रश्नः - न्यूनकर्माभिधानं \hfill https://www.vedavms.in \hfill}
\section*{ 3.1     तृतीयकाण्डे प्रथमः प्रश्नः - न्यूनकर्माभिधानं }
                                \textbf{ TS 3.1.1.1} \newline
                  प्र॒जाप॑ति॒रिति॑ प्र॒जा - प॒तिः॒ । अ॒का॒म॒य॒त॒ । प्र॒जा इति॑ प्र - जाः । सृ॒जे॒य॒ । इति॑ । सः । तपः॑ । अ॒त॒प्य॒त॒ । सः । स॒र्पान् । अ॒सृ॒ज॒त॒ । सः । अ॒का॒म॒य॒त॒ । प्र॒जा इति॑ प्र - जाः । सृ॒जे॒य॒ । इति॑ । सः । द्वि॒तीय᳚म् । अ॒त॒प्य॒त॒ । सः । वयाꣳ॑सि । अ॒सृ॒ज॒त॒ । सः । अ॒का॒म॒य॒त॒ । प्र॒जा इति॑ प्र-जाः । सृ॒जे॒य॒ । इति॑ । सः । तृ॒तीय᳚म् । अ॒त॒प्य॒त॒ । सः । ए॒तम् । दी॒क्षि॒त॒वा॒दमिति॑ दीक्षित - वा॒दम् । अ॒प॒श्य॒त् । तम् । अ॒व॒द॒त् । ततः॑ । वै । सः । प्र॒जा इति॑ प्र-जाः । अ॒सृ॒ज॒त॒ । यत् । तपः॑ । त॒प्त्वा । दी॒क्षि॒त॒वा॒दमिति॑ दीक्षित - वा॒दम् । वद॑ति । प्र॒जा इति॑ प्र - जाः । ए॒व । तत् । यज॑मानः । \textbf{  1} \newline
                  \newline
                                \textbf{ TS 3.1.1.2} \newline
                  सृ॒ज॒ते॒ । यत् । वै । दी॒क्षि॒तः । अ॒मे॒द्ध्यम् । पश्य॑ति । अपेति॑ । अ॒स्मा॒त् । दी॒क्षा । क्रा॒म॒ति॒ । नील᳚म् । अ॒स्य॒ । हरः॑ । वीति॑ । ए॒ति॒ । अब॑द्धम् । मनः॑ । द॒रिद्र᳚म् । चक्षुः॑ । सूर्यः॑ । ज्योति॑षाम् । श्रेष्ठः॑ । दीक्षे᳚ । मा । मा॒ । हा॒सीः॒ । इति॑ । आ॒ह॒ । न । अ॒स्मा॒त् । दी॒क्षा । अपेति॑ । क्रा॒म॒ति॒ । न । अ॒स्य॒ । नील᳚म् । न । हरः॑ । वीति॑ । ए॒ति॒ । यत् । वै । दी॒क्षि॒तम् । अ॒भि॒वर्.ष॒तीत्य॑भि-वर्.ष॑ति । दि॒व्याः । आपः॑ । अशा᳚न्ताः । ओजः॑ । बल᳚म् । दी॒क्षाम् । \textbf{  2} \newline
                  \newline
                                \textbf{ TS 3.1.1.3} \newline
                  तपः॑ । अ॒स्य॒ । निरिति॑ । घ्न॒न्ति॒ । उ॒न्द॒तीः । बल᳚म् । ध॒त्त॒ । ओजः॑ । ध॒त्त॒ । बल᳚म् । ध॒त्त॒ । मा । मे॒ । दी॒क्षाम् । मा । तपः॑ । निरिति॑ । व॒धि॒ष्ट॒ । इति॑ । आ॒ह॒ । ए॒तत् । ए॒व । सर्व᳚म् । आ॒त्मन्न् । ध॒त्ते॒ । न । अ॒स्य॒ । ओजः॑ । बल᳚म् । न । दी॒क्षाम् । न । तपः॑ । निरिति॑ । घ्न॒न्ति॒ । अ॒ग्निः । वै । दी॒क्षि॒तस्य॑ । दे॒वता᳚ । सः । अ॒स्मा॒त् । ए॒तर्.हि॑ । ति॒रः । इ॒व॒ । यर्.हि॑ । याति॑ । तम् । ई॒श्व॒रम् । रक्षाꣳ॑सि । हन्तोः᳚ । \textbf{  3} \newline
                  \newline
                                \textbf{ TS 3.1.1.4} \newline
                  भ॒द्रात् । अ॒भीति॑ । श्रेयः॑ । प्रेति॑ । इ॒हि॒ । बृह॒स्पतिः॑ । पु॒र॒ ए॒तेति॑ पुरः - ए॒ता । ते॒ । अ॒स्तु॒ । इति॑ । आ॒ह॒ । ब्रह्म॑ । वै । दे॒वाना᳚म् । बृह॒स्पतिः॑ । तम् । ए॒व । अ॒न्वार॑भत॒ इत्य॑नु - आर॑भते । सः । ए॒न॒म् । समिति॑ । पा॒र॒य॒ति॒ । एति॑ । इ॒दम् । अ॒ग॒न्म॒ । दे॒व॒यज॑न॒मिति॑ देव - यज॑नम् । पृ॒थि॒व्याः । इति॑ । आ॒ह॒ । दे॒व॒यज॑न॒मिति॑ देव - यज॑नम् । हि । ए॒षः । पृ॒थि॒व्याः । आ॒गच्छ॒तीत्या᳚ - गच्छ॑ति । यः । यज॑ते । विश्वे᳚ । दे॒वाः । यत् । अजु॑षन्त । पूर्वे᳚ । इति॑ । आ॒ह॒ । विश्वे᳚ । हि । ए॒तत् । दे॒वाः । जो॒षय॑न्ते । यत् । ब्रा॒ह्म॒णाः ( ) । ऋ॒ख्सा॒माभ्या॒मित्यृ॑ख्सा॒मा -भ्या॒म् । यजु॑षा । स॒न्तर॑न्त॒ इति॑ सं - तर॑न्तः । इति॑ । आ॒ह॒ । ऋ॒ख्सा॒माभ्या॒मित्यृ॑ख्सा॒मा - भ्या॒म् । हि । ए॒षः । यजु॑षा । स॒न्तर॒तीति॑ सं - तर॑ति । यः । यज॑ते । रा॒यः । पोषे॑ण । समिति॑ । इ॒षा । म॒दे॒म॒ । इति॑ । आ॒ह॒ । आ॒शिष॒मित्या᳚ - शिष᳚म् । ए॒व । ए॒ताम् । एति॑ । शा॒स्ते॒ ॥ \textbf{  4} \newline
                  \newline
                      (यज॑मानो - दी॒क्षाꣳ - हन्तो᳚ - र्ब्राह्म॒णा -श्चतु॑र्विꣳशतिश्च)  \textbf{(A1)} \newline \newline
                                \textbf{ TS 3.1.2.1} \newline
                  ए॒षः । ते॒ । गा॒य॒त्रः । भा॒गः । इति॑ । मे॒ । सोमा॑य । ब्रू॒ता॒त् । ए॒षः । ते॒ । त्रैष्टु॑भः । जाग॑तः । भा॒गः । इति॑ । मे॒ । सोमा॑य । ब्रू॒ता॒त् । छ॒न्दो॒माना॒मिति॑ छन्दः - माना᳚म् । साम्रा᳚ज्य॒मिति॒ साम् - रा॒ज्य॒म् । ग॒च्छ॒ । इति॑ । मे॒ । सोमा॑य । ब्रू॒ता॒त् । यः । वै । सोम᳚म् । राजा॑नम् । साम्रा᳚ज्य॒मिति॒ साम् - रा॒ज्य॒म् । लो॒कम् । ग॒म॒यि॒त्वा । क्री॒णाति॑ । गच्छ॑ति । स्वाना᳚म् । साम्रा᳚ज्य॒मिति॒ साम् - रा॒ज्य॒म् । छन्दाꣳ॑सि । खलु॑ । वै । सोम॑स्य । राज्ञ्ः॑ । साम्रा᳚ज्य॒ इति॒ साम् - रा॒ज्यः॒ । लो॒कः । पु॒रस्ता᳚त् । सोम॑स्य । क्र॒यात् । ए॒वम् । अ॒भीति॑ । म॒न्त्र॒ये॒त॒ । साम्रा᳚ज्य॒मिति॒ साम् - रा॒ज्य॒म् । ए॒व । \textbf{  5} \newline
                  \newline
                                \textbf{ TS 3.1.2.2} \newline
                  ए॒न॒म् । लो॒कम् । ग॒म॒यि॒त्वा । क्री॒णा॒ति॒ । गच्छ॑ति । स्वाना᳚म् । साम्रा᳚ज्य॒मिति॒ साम् - रा॒ज्य॒म् । यः । वै । ता॒नू॒न॒प्त्रस्येति॑ तानू - न॒प्त्रस्य॑ । प्र॒ति॒ष्ठामिति॑ प्रति - स्थाम् । वेद॑ । प्रतीति॑ । ए॒व । ति॒ष्ठ॒ति॒ । ब्र॒ह्म॒वा॒दिन॒ इति॑ ब्रह्म - वा॒दिनः॑ । व॒द॒न्ति॒ । न । प्रा॒श्नन्तीति॑ प्र - अ॒श्नन्ति॑ । न । जु॒ह्व॒ति॒ । अथ॑ । क्व॑ । ता॒नू॒न॒प्त्रमिति॑ तानू - न॒प्त्रम् । प्रतीति॑ । ति॒ष्ठ॒ति॒ । इति॑ । प्र॒जाप॑ता॒विति॑ प्र॒जा-प॒तौ॒ । मन॑सि । इति॑ । ब्रू॒या॒त् । त्रिः । अवेति॑ । जि॒घ्रे॒त् । प्र॒जाप॑ता॒विति॑ प्र॒जा - प॒तौ॒ । त्वा॒ । मन॑सि । जु॒हो॒मि॒ । इति॑ । ए॒षा । वै । ता॒नू॒न॒प्त्रस्येति॑ तानू - न॒प्त्रस्य॑ । प्र॒ति॒ष्ठेति॑ प्रति - स्था । यः । ए॒वम् । वेद॑ । प्रतीति॑ । ए॒व । ति॒ष्ठ॒ति॒ । यः । \textbf{  6} \newline
                  \newline
                                \textbf{ TS 3.1.2.3} \newline
                  वै । अ॒द्ध्व॒र्योः । प्र॒ति॒ष्ठामिति॑ प्रति - स्थाम् । वेद॑ । प्रतीति॑ । ए॒व । ति॒ष्ठ॒ति॒ । यतः॑ । मन्ये॑त । अन॑भिक्र॒म्येत्यन॑भि - क्र॒म्य॒ । हो॒ष्या॒मि॒ । इति॑ । तत् । तिष्ठन्न्॑ । एति॑ । श्रा॒व॒ये॒त् । ए॒षा । वै । अ॒द्ध्व॒र्योः । प्र॒ति॒ष्ठेति॑ प्रति - स्था । यः । ए॒वम् । वेद॑ । प्रतीति॑ । ए॒व । ति॒ष्ठ॒ति॒ । यत् । अ॒भि॒क्रम्येत्य॑भि - क्रम्य॑ । जु॒हु॒यात् । प्र॒ति॒ष्ठाया॒ इति॑ प्रति-स्थायाः᳚ । इ॒या॒त् । तस्मा᳚त् । स॒मा॒नत्र॑ । तिष्ठ॑ता । हो॒त॒व्य᳚म् । प्रति॑ष्ठित्या॒ इति॒ प्रति॑ - स्थि॒त्यै॒ । यः । वै । अ॒द्ध्व॒र्योः । स्वम् । वेद॑ । स्ववा॒निति॒ स्व - वा॒न् । ए॒व । भ॒व॒ति॒ । स्रुक् । वै । अ॒स्य॒ । स्वम् । वा॒य॒व्य᳚म् । अ॒स्य॒ । \textbf{  7} \newline
                  \newline
                                \textbf{ TS 3.1.2.4} \newline
                  स्वम् । च॒म॒सः । अ॒स्य॒ । स्वम् । यत् । वा॒य॒व्य᳚म् । वा॒ । च॒म॒सम् । वा॒ । अन॑न्वार॒भ्येत्यन॑नु - आ॒र॒भ्य॒ । आ॒श्रा॒वये॒दित्या᳚ - श्रा॒वये᳚त् । स्वात् । इ॒या॒त् । तस्मा᳚त् । अ॒न्वा॒रभ्येत्य॑नु - आ॒रभ्य॑ । आ॒श्राव्य॒मित्या᳚ - श्राव्य᳚म् । स्वात् । ए॒व । न । ए॒ति॒ । यः । वै । सोम᳚म् । अप्र॑तिष्ठा॒प्येत्यप्र॑ति - स्था॒प्य॒ । स्तो॒त्रम् । उ॒पा॒क॒रोतीत्यु॑प - आ॒क॒रोति॑ । अप्र॑तिष्ठित॒ इत्यप्र॑ति - स्थि॒तः॒ । सोमः॑ । भव॑ति । अप्र॑तिष्ठित॒ इत्यप्र॑ति - स्थि॒तः॒ । स्तोमः॑ । अप्र॑तिष्ठिता॒नीत्यप्र॑ति - स्थि॒ता॒नि॒ । उ॒क्थानि॑ । अप्र॑तिष्ठित॒ इत्यप्र॑ति - स्थि॒तः॒ । यज॑मानः । अप्र॑तिष्ठित॒ इत्यप्र॑ति - स्थि॒तः॒ । अ॒ध्व॒र्युः । वा॒य॒व्य᳚म् । वै । सोम॑स्य । प्र॒ति॒ष्ठेति॑ प्रति - स्था । च॒म॒सः । अ॒स्य॒ । प्र॒ति॒ष्ठेति॑ प्रति - स्था । सोमः॑ । स्तोम॑स्य । स्तोमः॑ । उ॒क्थाना᳚म् । ग्रह᳚म् । वा॒ ( ) । गृ॒ही॒त्वा । च॒म॒सम् । वा॒ । उ॒न्नीयेत्यु॑त् - नीय॑ । स्तो॒त्रम् । उ॒पाकु॑र्या॒दित्यु॑प - आकु॑र्यात् । प्रतीति॑ । ए॒व । सोम᳚म् । स्था॒पय॑ति । प्रतीति॑ । स्तोम᳚म् । प्रतीति॑ । उ॒क्थानि॑ । प्रतीति॑ । यज॑मानः । तिष्ठ॑ति । प्रतीति॑ । अ॒द्ध्व॒र्युः ॥ \textbf{  8 } \newline
                  \newline
                      (ए॒व - ति॑ष्ठति॒ यो - वा॑य॒व्य॑मस्य॒ - ग्रहं॒ ॅवै - का॒न्न - विꣳ॑श॒तिश्च॑)  \textbf{(A2)} \newline \newline
                                \textbf{ TS 3.1.3.1} \newline
                  य॒ज्ञ्म् । वै । ए॒तत् । समिति॑ । भ॒र॒न्ति॒ । यत् । सो॒म॒क्रय॑ण्या॒ इति॑ सोम - क्रय॑ण्यै । प॒दम् । य॒ज्ञ्॒मु॒खमिति॑ यज्ञ् - मु॒खम् । ह॒वि॒द्‌र्धाने॒ इति॑ हविः - धाने᳚ । यर्.हि॑ । ह॒वि॒द्‌र्धाने॒ इति॑ हविः - धाने᳚ । प्राची॒ इति॑ । प्र॒व॒र्तये॑यु॒रिति॑ प्र - व॒र्तये॑युः । तर्.हि॑ । तेन॑ । अक्ष᳚म् । उपेति॑ । अ॒ञ्ज्या॒त् । य॒ज्ञ्॒मु॒ख इति॑ यज्ञ्-मु॒खे । ए॒व । य॒ज्ञ्म् । अनु॑ । समिति॑ । त॒नो॒ति॒ । प्राञ्च᳚म् । अ॒ग्निम् । प्रेति॑ । ह॒र॒न्ति॒ । उदिति॑ । पत्नी᳚म् । एति॑ । न॒य॒न्ति॒ । अन्विति॑ । अनाꣳ॑सि । प्रेति॑ । व॒र्त॒य॒न्ति॒ । अथ॑ । वै । अ॒स्य॒ । ए॒षः । धिष्णि॑यः । ही॒य॒ते॒ । सः । अन्विति॑ । ध्या॒य॒ति॒ । सः । ई॒श्व॒रः । रु॒द्रः । भू॒त्वा । \textbf{  9} \newline
                  \newline
                                \textbf{ TS 3.1.3.2} \newline
                  प्र॒जामिति॑ प्र - जाम् । प॒शून् । यज॑मानस्य । शम॑यितोः । यर्.हि॑ । प॒शुम् । आप्री॑त॒मित्या-प्री॒त॒म् । उद॑ञ्चम् । नय॑न्ति । तर्.हि॑ । तस्य॑ । प॒शु॒श्रप॑ण॒मिति॑ पशु - श्रप॑णम् । ह॒रे॒त् । तेन॑ । ए॒व । ए॒न॒म् । भा॒गिन᳚म् । क॒रो॒ति॒ । यज॑मानः । वै । आ॒ह॒व॒नीय॒ इत्या᳚ - ह॒व॒नीयः॑ । यज॑मानम् । वै । ए॒तत् । वीति॑ । क॒र्॒.ष॒न्ते॒ । यत् । आ॒ह॒व॒नीया॒दित्या᳚ - ह॒व॒नीया᳚त् । प॒शु॒श्रप॑ण॒मिति॑ पशु - श्रप॑णम् । हर॑न्ति । सः । वा॒ । ए॒व । स्यात् । नि॒र्म॒न्थ्य॑मिति॑ निः - म॒न्थ्य᳚म् । वा॒ । कु॒र्या॒त् । यज॑मानस्य । सा॒त्म॒त्वायेति॑ सात्म - त्वाय॑ । यदि॑ । प॒शोः । अ॒व॒दान॒मित्य॑व - दान᳚म् । नश्ये᳚त् । आज्य॑स्य । प्र॒त्या॒ख्याय॒मिति॑ प्रति - आ॒ख्याय᳚म् । अवेति॑ । द्ये॒त् । सा । ए॒व । ततः॑ ( ) । प्राय॑श्चित्तिः । ये । प॒शुम् । वि॒म॒थ्नी॒रन्निति॑ वि - म॒थ्नी॒रन्न् । यः । तान् । का॒मये॑त । आर्ति᳚म् । एति॑ । ऋ॒च्छे॒युः॒ । इति॑ । कु॒वित् । अ॒ङ्ग । इति॑ । नमो॑वृक्तिव॒त्येति॒ नमो॑वृक्ति - व॒त्या॒ । ऋ॒चा । आग्नी᳚द्ध्र॒ इत्याग्नि॑ - इ॒ध्रे॒ । जु॒हु॒या॒त् । नमो॑वृक्ति॒मिति॒ नमः॑ - वृ॒क्ति॒म् । ए॒व । ए॒षा॒म् । वृ॒ङ्क्ते॒ । ता॒जक् । आर्ति᳚म् । एति॑ । ऋ॒च्छ॒न्ति॒ ॥ \textbf{  10} \newline
                  \newline
                      (भू॒त्वा - ततः॒ - षड्विꣳ॑शतिश्च)  \textbf{(A3)} \newline \newline
                                \textbf{ TS 3.1.4.1} \newline
                  प्र॒जाप॑ते॒रिति॑ प्र॒जा - प॒तेः॒ । जाय॑मानाः । प्र॒जा इति॑ प्र - जाः । जा॒ताः । च॒ । याः । इ॒माः ॥ तस्मै᳚ । प्रति॑ । प्रेति॑ । वे॒द॒य॒ । चि॒कि॒त्वान् । अन्विति॑ । म॒न्य॒ता॒म् ॥ इ॒मम् । प॒शुम् । प॒शु॒प॒त॒ इति॑ पशु - प॒ते॒ । ते॒ । अ॒द्य । ब॒द्ध्नामि॑ । अ॒ग्ने॒ । सु॒कृ॒तस्येति॑ सु - कृ॒तस्य॑ । मद्ध्ये᳚ ॥ अन्विति॑ । म॒न्य॒स्व॒ । सु॒यजेति॑ सु - यजा᳚ । य॒जा॒म॒ । जुष्ट᳚म् । दे॒वाना᳚म् । इ॒दम् । अ॒स्तु॒ । ह॒व्यम् ॥ प्र॒जा॒नन्त॒ इति॑ प्र - जा॒नन्तः॑ । प्रतीति॑ । गृ॒ह्ण॒न्ति॒ । पूर्वे᳚ । प्रा॒णमिति॑ प्र - अ॒नम् । अङ्गे᳚भ्यः । परीति॑ । आ॒चर॑न्त॒मित्या᳚ - चर॑न्तम् ॥ सु॒व॒र्गमिति॑ सुवः - गम् । या॒हि॒ । प॒थिभि॒रिति॑ प॒थि - भिः॒ । दे॒व॒यानै॒रिति॑ देव - यानैः᳚ । ओष॑धीषु । प्रतीति॑ । ति॒ष्ठ॒ । शरी॑रैः ॥ येषा᳚म् । ईशे᳚ । \textbf{  11} \newline
                  \newline
                                \textbf{ TS 3.1.4.2} \newline
                  प॒श॒पति॒रिति॑ पशु-पतिः॑ । प॒शू॒नाम् । चतु॑ष्पदा॒मिति॒ चतुः॑ - प॒दा॒म् । उ॒त । च॒ । द्वि॒पदा॒मिति॑ द्वि - पदा᳚म् ॥ निष्क्री॑त॒ इति॒ निः - क्री॒तः॒ । अ॒यम् । य॒ज्ञिय᳚म् । भा॒गम् । ए॒तु॒ । रा॒यः । पोषाः᳚ । यज॑मानस्य । स॒न्तु॒ ॥ ये । ब॒द्ध्यमा॑नम् । अन्विति॑ । ब॒द्ध्यमा॑नाः । अ॒भ्यैक्ष॒न्तेत्य॑भि - ऐक्ष॑न्त । मन॑सा । चक्षु॑षा । च॒ ॥ अ॒ग्निः । तान् । अग्रे᳚ । प्रेति॑ । मु॒मो॒क्तु॒ । दे॒वः । प्र॒जाप॑ति॒रिति॑ प्र॒जा - प॒तिः॒ । प्र॒जयेति॑ प्र - जया᳚ । सं॒ॅवि॒दा॒न इति॑ सं-वि॒दा॒नः ॥ ये । आ॒र॒ण्याः । प॒शवः॑ । वि॒श्वरू॑पा॒ इति॑ वि॒श्व - रू॒पाः॒ । विरू॑पा॒ इति॒ वि-रू॒पाः॒ । सन्तः॑ । ब॒हु॒धेति॑ बहु - धा । एक॑रूपा॒ इत्येक॑ - रू॒पाः॒ ॥ वा॒युः । तान् । अग्रे᳚ । प्रेति॑ । मु॒मो॒क्तु॒ । दे॒वः । प्र॒जाप॑ति॒रिति॑ प्र॒जा - प॒तिः॒ । प्र॒जयेति॑ प्र - जया᳚ । सं॒ॅवि॒दा॒न इति॑ सं - वि॒दा॒नः ॥ प्र॒मु॒ञ्चमा॑ना॒ इति॑ प्र - मु॒ञ्चमा॑नाः । \textbf{  12} \newline
                  \newline
                                \textbf{ TS 3.1.4.3} \newline
                  भुव॑नस्य । रेतः॑ । गा॒तुम् । ध॒त्त॒ । यज॑मानाय । दे॒वाः॒ ॥ उ॒पाकृ॑त॒मित्यु॑प-आकृ॑तम् । श॒श॒मा॒नम् । यत् । अस्था᳚त् । जी॒वम् । दे॒वाना᳚म् । अपीति॑ । ए॒तु॒ । पाथः॑ ॥ नाना᳚ । प्रा॒ण इति॑ प्र - अ॒नः । यज॑मानस्य । प॒शुना᳚ । य॒ज्ञ्ः । दे॒वेभिः॑ । स॒ह । दे॒व॒यान॒ इति॑ देव - यानः॑ ॥ जी॒वम् । दे॒वाना᳚म् । अपीति॑ । ए॒तु॒ । पाथः॑ । स॒त्याः । स॒न्तु॒ । यज॑मानस्य । कामाः᳚ ॥ यत् । प॒शुः । मा॒युम् । अकृ॑त । उरः॑ । वा॒ । प॒द्भिरिति॑ पत् - भिः । आ॒ह॒त इत्या᳚ - ह॒ते ॥ अ॒ग्निः । मा॒ । तस्मा᳚त् । एन॑सः । विश्वा᳚त् । मु॒ञ्च॒तु॒ । अꣳह॑सः ॥ शमि॑तारः । उ॒पेत॒नेत्यु॑प - एत॑न । य॒ज्ञ्म् । \textbf{  13} \newline
                  \newline
                                \textbf{ TS 3.1.4.4} \newline
                  दे॒वेभिः॑ । इ॒न्वि॒तम् ॥ पाशा᳚त् । प॒शुम् । प्रेति॑ । मु॒ञ्च॒त॒ । ब॒न्धात् । य॒ज्ञ्प॑ति॒मिति॑ य॒ज्ञ् - प॒ति॒म् । परि॑ ॥ अदि॑तिः । पाश᳚म् । प्रेति॑ । मु॒मो॒क्तु॒ । ए॒तम् । नमः॑ । प॒शुभ्य॒ इति॑ प॒शु - भ्यः॒ । प॒शु॒पत॑य॒ इति॑ पशु - पत॑ये । क॒रो॒मि॒ ॥ अ॒रा॒ती॒यन्त᳚म् । अध॑रम् । कृ॒णो॒मि॒ । यम् । द्वि॒ष्मः । तस्मिन्न्॑ । प्रतीति॑ । मु॒ञ्चा॒मि॒ । पाश᳚म् ॥ त्वाम् । उ॒ । ते । द॒धि॒रे॒ । ह॒व्य॒वाह॒मिति॑ हव्य - वाह᳚म् । शृ॒त॒कं॒र्तार॒मिति॑ शृतं - क॒र्तार᳚म् । उ॒त । य॒ज्ञिय᳚म् । च॒ ॥ अग्ने᳚ । सद॑क्ष॒ इति॒ स - द॒क्षः॒ । सत॑नु॒रिति॒ स-त॒नुः॒ । हि । भू॒त्वा । अथ॑ । ह॒व्या । जा॒त॒वे॒द॒ इति॑ जात - वे॒दः॒ । जु॒ष॒स्व॒ ॥ जात॑वेद॒ इति॒ जात॑ - वे॒दः॒ । व॒पया᳚ । ग॒च्छ॒ । दे॒वान् । त्वम् ( ) । हि । होता᳚ । प्र॒थ॒मः । ब॒भूथ॑ ॥ घृ॒तेन॑ । त्वम् । त॒नुवः॑ । व॒द्‌र्ध॒य॒स्व॒ । स्वाहा॑कृत॒मिति॒ स्वाहा᳚ - कृ॒त॒म् । ह॒विः । अ॒द॒न्तु॒ । दे॒वाः ॥ स्वाहा᳚ । दे॒वेभ्यः॑ । दे॒वेभ्यः॑ । स्वाहा᳚ ॥ \textbf{  14} \newline
                  \newline
                      (ईशे᳚ - प्रमु॒ञ्चमा॑ना - य॒ज्ञ्ं - त्वꣳ - षोड॑श च)  \textbf{(A4)} \newline \newline
                                \textbf{ TS 3.1.5.1} \newline
                  प्रा॒जा॒प॒त्या इति॑ प्राजा - प॒त्याः । वै । प॒शवः॑ । तेषा᳚म् । रु॒द्रः । अधि॑पति॒रित्यधि॑ - प॒तिः॒ । यत् । ए॒ताभ्या᳚म् । उ॒पा॒क॒रोतीत्यु॑प - आ॒क॒रोति॑ । ताभ्या᳚म् । ए॒व । ए॒न॒म् । प्र॒ति॒प्रोच्येति॑ प्रति - प्रोच्य॑ । एति॑ । ल॒भ॒ते॒ । आ॒त्मनः॑ । अना᳚व्रस्का॒येत्यना᳚ - व्र॒स्का॒य॒ । द्वाभ्या᳚म् । उ॒पाक॑रा॒तीत्यु॑प - आक॑रोति । द्वि॒पादिति॑ द्वि - पात् । यज॑मानः । प्रति॑ष्ठित्या॒ इति॒ प्रति॑ - स्थि॒त्यै॒ । उ॒पा॒कृत्येत्यु॑प - आ॒कृत्य॑ । पञ्च॑ । जु॒हो॒ति॒ । पाङ्क्ताः᳚ । प॒शवः॑ । प॒शून् । ए॒व । अवेति॑ । रु॒न्धे॒ । मृ॒त्यवे᳚ । वै । ए॒षः । नी॒य॒ते॒ । यत् । प॒शुः । तम् । यत् । अ॒न्वा॒रभे॒तेत्य॑नु - आ॒रभे॑त । प्र॒मायु॑क॒ इति॑ प्र - मायु॑कः । यज॑मानः । स्या॒त् । नाना᳚ । प्रा॒ण इति॑ प्र - अ॒नः । यज॑मानस्य । प॒शुना᳚ । इति॑ । आ॒ह॒ । व्यावृ॑त्त्या॒ इति॑ वि - आवृ॑त्त्यै । \textbf{  15} \newline
                  \newline
                                \textbf{ TS 3.1.5.2} \newline
                  यत् । प॒शुः । मा॒युम् । अकृ॑त । इति॑ । जु॒हो॒ति॒ । शान्त्यै᳚ । शमि॑तारः । उ॒पेत॒नेत्यु॑प - एत॑न । इति॑ । आ॒ह॒ । य॒था॒य॒जुरिति॑ यथा - य॒जुः । ए॒व । ए॒तत् । व॒पाया᳚म् । वै । आ॒ह्रि॒यमा॑णाया॒मित्या᳚ - ह्रि॒यमा॑णायाम् । अ॒ग्नेः । मेधः॑ । अपेति॑ । क्रा॒म॒ति॒ । त्वाम् । उ॒ । ते । द॒धि॒रे॒ । ह॒व्य॒वाह॒मिति॑ हव्य - वाह᳚म् । इति॑ । व॒पाम् । अ॒भीति॑ । जु॒हो॒ति॒ । अ॒ग्नेः । ए॒व । मेध᳚म् । अवेति॑ । रु॒न्धे॒ । अथो॒ इति॑ । शृ॒त॒त्वायेति॑ शृत - त्वाय॑ । पु॒रस्ता᳚थ्स्वाहाकृतय॒ इति॑ पु॒रस्ता᳚त् - स्वा॒हा॒कृ॒त॒यः॒ । वै । अ॒न्ये । दे॒वाः । उ॒परि॑ष्टाथ्स्वाहाकृतय॒ इत्यु॒परि॑ष्टात् - स्वा॒हा॒कृ॒त॒यः॒ । अ॒न्ये । स्वाहा᳚ । दे॒वेभ्यः॑ । दे॒वेभ्यः॑ । स्वाहा᳚ । इति॑ । अ॒भितः॑ । व॒पाम् ( ) । जु॒हो॒ति॒ । तान् । ए॒व । उ॒भयान्॑ । प्री॒णा॒ति॒ ॥ \textbf{  16 } \newline
                  \newline
                      (व्यावृ॑त्त्या - अ॒भितो॑ व॒पां - पञ्च॑ च)  \textbf{(A5)} \newline \newline
                                \textbf{ TS 3.1.6.1} \newline
                  यः । वै । अय॑थादेवत॒मित्यय॑था - दे॒व॒त॒म् । य॒ज्ञ्म् । उ॒प॒चर॒तीत्यु॑प - चर॑ति । एति॑ । दे॒वता᳚भ्यः । वृ॒श्च्य॒ते॒ । पापी॑यान् । भ॒व॒ति॒ । यः । य॒था॒दे॒व॒तमिति॑ यथा-दे॒व॒तम् । न । दे॒वता᳚भ्यः । एति॑ । वृ॒श्च्य॒ते॒ । वसी॑यान् । भ॒व॒ति॒ । आ॒ग्ने॒य्या । ऋ॒चा । आग्नी᳚द्ध्र॒मित्याग्नि॑ - इ॒द्ध्र॒म् । अ॒भीति॑ । मृ॒शे॒त् । वै॒ष्ण॒व्या । ह॒वि॒द्‌र्धान॒मिति॑ हविः - धान᳚म् । आ॒ग्ने॒य्या । स्रुचः॑ । वा॒य॒व्य॑या । वा॒य॒व्या॑नि । ऐ॒न्द्रि॒या । सदः॑ । य॒था॒दे॒व॒तमिति॑ यथा-दे॒व॒तम् । ए॒व । य॒ज्ञ्म् । उपेति॑ । च॒र॒ति॒ । न । दे॒वता᳚भ्यः । एति॑ । वृ॒श्च्य॒ते॒ । वसी॑यान् । भ॒व॒ति॒ । यु॒नज्मि॑ । ते॒ । पृ॒थि॒वीम् । ज्योति॑षा । स॒ह । यु॒नज्मि॑ । वा॒युम् । अ॒न्तरि॑क्षेण । \textbf{  17} \newline
                  \newline
                                \textbf{ TS 3.1.6.2} \newline
                  ते॒ । स॒ह । यु॒नज्मि॑ । वाच᳚म् । स॒ह । सूर्ये॑ण । ते॒ । यु॒नज्मि॑ । ति॒स्रः । वि॒पृच॒ इति॑ वि - पृचः॑ । सूर्य॑स्य । ते॒ ॥ अ॒ग्निः । दे॒वता᳚ । गा॒य॒त्री । छन्दः॑ । उ॒पाꣳ॒॒शोरित्यु॑प - अꣳ॒॒शोः । पात्र᳚म् । अ॒सि॒ । सोमः॑ । दे॒वता᳚ । त्रि॒ष्टुप् । छन्दः॑ । अ॒न्त॒र्या॒मस्येत्य॑न्तः - या॒मस्य॑ । पात्र᳚म् । अ॒सि॒ । इन्द्रः॑ । दे॒वता᳚ । जग॑ती । छन्दः॑ । इ॒न्द्र॒वा॒यु॒वोरिती᳚न्द्र-वा॒यु॒वोः । पात्र᳚म् । अ॒सि॒ । बृह॒स्पतिः॑ । दे॒वता᳚ । अ॒नु॒ष्टुबित्य॑नु - स्तुप् । छन्दः॑ । मि॒त्रावरु॑णयो॒रिति॑ मि॒त्रा-वरु॑णयोः । पात्र᳚म् । अ॒सि॒ । अ॒श्विनौ᳚ । दे॒वता᳚ । प॒ङ्क्तिः । छन्दः॑ । अ॒श्विनोः᳚ । पात्र᳚म् । अ॒सि॒ । सूर्यः॑ । दे॒वता᳚ । बृ॒ह॒ती । \textbf{  18} \newline
                  \newline
                                \textbf{ TS 3.1.6.3} \newline
                  छन्दः॑ । शु॒क्रस्य॑ । पात्र᳚म् । अ॒सि॒ । च॒न्द्रमाः᳚ । दे॒वता᳚ । स॒तोबृ॑ह॒तीति॑ स॒तः - बृ॒ह॒ती॒ । छन्दः॑ । म॒न्थिनः॑ । पात्र᳚म् । अ॒सि॒ । विश्वे᳚ । दे॒वाः । दे॒वता᳚ । उ॒ष्णिहा᳚ । छन्दः॑ । आ॒ग्र॒य॒णस्य॑ । पात्र᳚म् । अ॒सि॒ । इन्द्रः॑ । दे॒वता᳚ । क॒कुत् । छन्दः॑ । उ॒क्थाना᳚म् । पात्र᳚म् । अ॒सि॒ । पृ॒थि॒वी । दे॒वता᳚ । वि॒राडिति॑ वि - राट् । छन्दः॑ । ध्रु॒वस्य॑ । पात्र᳚म् । अ॒सि॒ ॥ \textbf{  19} \newline
                  \newline
                      (अ॒न्तरि॑क्षेण - बृह॒ती - त्रय॑स्त्रिꣳशच्च)  \textbf{(A6)} \newline \newline
                                \textbf{ TS 3.1.7.1} \newline
                  इ॒ष्टर्गः॑ । वै । अ॒द्ध्व॒र्युः । यज॑मानस्य । इ॒ष्टर्गः॑ । खलु॑ । वै । पूर्वः॑ । अ॒र्ष्टुः । क्षी॒य॒ते॒ । आ॒स॒न्या᳚त् । मा॒ । मन्त्रा᳚त् । पा॒हि॒ । कस्याः᳚ । चि॒त् । अ॒भिश॑स्त्या॒ इत्य॒भि - श॒स्त्याः॒ । इति॑ । पु॒रा । प्रा॒त॒र॒नु॒वा॒कादिति॑ प्रातः - अ॒नु॒वा॒कात् । जु॒हु॒या॒त् । आ॒त्मने᳚ । ए॒व । तत् । अ॒द्ध्व॒र्युः । पु॒रस्ता᳚त् । शर्म॑ । न॒ह्य॒ते॒ । अना᳚र्त्यै । सं॒ॅवे॒शायेति॑ सं - वे॒शाय॑ । त्वा॒ । उ॒प॒वे॒शायेतु॑प - वे॒शाय॑ । त्वा॒ । गा॒य॒त्रि॒याः । त्रि॒ष्टुभः॑ । जग॑त्याः । अ॒भिभू᳚त्या॒ इत्य॒भि - भू॒त्यै॒ । स्वाहा᳚ । प्राणा॑पाना॒विति॒ प्राण॑ - अ॒पा॒नौ॒ । मृ॒त्योः । मा॒ । पा॒त॒म् । प्राणा॑पाना॒विति॒ प्राण॑ - अ॒पा॒नौ॒ । मा । मा॒ । हा॒सि॒ष्ट॒म् । दे॒वता॑सु । वै । ए॒ते । प्रा॒णा॒पा॒नयो॒रिति॑ प्राण - अ॒पा॒नयोः᳚ । \textbf{  20} \newline
                  \newline
                                \textbf{ TS 3.1.7.2} \newline
                  व्याय॑च्छन्त॒ इति॑ वि - आय॑च्छन्ते । येषा᳚म् । सोमः॑ । स॒मृ॒च्छत॒ इति॑ सं - ऋ॒च्छते᳚ । सं॒ॅवे॒शायेति॑ सं - वे॒शाय॑ । त्वा॒ । उ॒प॒वे॒शायेत्यु॑प - वे॒शाय॑ । त्वा॒ । इति॑ । आ॒ह॒ । छन्दाꣳ॑सि । वै । सं॒ॅवे॒श इति॑ सं - वे॒शः । उ॒प॒वे॒श इत्यु॑प - वे॒शः । छन्दो॑भि॒रिति॒ छन्दः॑ - भिः॒ । ए॒व । अ॒स्य॒ । छन्दाꣳ॑सि । वृ॒ङ्क्ते॒ । प्रेति॑व॒न्तीति॒ प्रेति॑ - व॒न्ति॒ । आज्या॑नि । भ॒व॒न्ति॒ । अ॒भिजि॑त्या॒ इत्य॒भि - जि॒त्यै॒ । म॒रुत्व॑तीः । प्र॒ति॒पद॒ इति॑ प्रति - पदः॑ । विजि॑त्या॒ इति॒ वि - जि॒त्यै॒ । उ॒भे इति॑ । बृ॒ह॒द्र॒थ॒न्त॒रे इति॑ बृहत् - र॒थ॒न्त॒रे । भ॒व॒तः॒ । इ॒यम् । वाव । र॒थ॒न्त॒रमिति॑ रथम्-त॒रम् । अ॒सौ । बृ॒हत् । आ॒भ्याम् । ए॒व । ए॒न॒म् । अ॒न्तः । ए॒ति॒ । अ॒द्य । वाव । र॒थ॒न्त॒रमिति॑ रथम् - त॒रम् । श्वः । बृ॒हत् । अ॒द्या॒श्वादित्य॑द्य - श्वात् । ए॒व । ए॒न॒म् । अ॒न्तः । ए॒ति॒ । भू॒तम् । \textbf{  21} \newline
                  \newline
                                \textbf{ TS 3.1.7.3} \newline
                  वाव । र॒थ॒न्त॒रमिति॑ रथम् - त॒रम् । भ॒वि॒ष्यत् । बृ॒हत् । भू॒तात् । च॒ । ए॒व । ए॒न॒म् । भ॒वि॒ष्य॒तः । च॒ । अ॒न्तः । ए॒ति॒ । परि॑मित॒मिति॒ परि॑ - मि॒त॒म् । वाव । र॒थ॒न्त॒रमिति॑ रथम् - त॒रम् । अप॑रिमित॒मित्यप॑रि - मि॒त॒म् । बृ॒हत् । परि॑मिता॒दिति॒ परि॑ - मि॒ता॒त् । च॒ । ए॒व । ए॒न॒म् । अप॑रिमिता॒दित्यप॑रि - मि॒ता॒त् । च॒ । अ॒न्तः । ए॒ति॒ । वि॒श्वा॒मि॒त्र॒ज॒म॒द॒ग्नी इति॑ विश्वामित्र - ज॒म॒द॒ग्नी । वसि॑ष्ठेन । अ॒स्प॒द्‌र्धे॒ता॒म् । सः । ए॒तत् । ज॒मद॑ग्निः । वि॒ह॒व्य॑मिति॑ वि - ह॒व्य᳚म् । अ॒प॒श्य॒त् । तेन॑ । वै । सः । वसि॑ष्ठस्य । इ॒न्द्रि॒यम् । वी॒र्य᳚म् । अ॒वृ॒ङ्क्त॒ । यत् । वि॒ह॒व्य॑मिति॑ वि - ह॒व्य᳚म् । श॒स्यते᳚ । इ॒न्द्रि॒यम् । ए॒व । तत् । वी॒र्य᳚म् । यज॑मानः । भ्रातृ॑व्यस्य । वृ॒ङ्क्ते॒ ( ) । यस्य॑ । भूयाꣳ॑सः । य॒ज्ञ्॒क्र॒तव॒ इति॑ यज्ञ् - क्र॒तवः॑ । इति॑ । आ॒हुः॒ । सः । दे॒वताः᳚ । वृ॒ङ्क्ते॒ । इति॑ । यदि॑ । अ॒ग्नि॒ष्टो॒म इत्य॑ग्नि-स्तो॒मः । सोमः॑ । प॒रस्ता᳚त् । स्यात् । उ॒क्थ्य᳚म् । कु॒र्वी॒त॒ । यदि॑ । उ॒क्थ्यः॑ । स्यात् । अ॒ति॒रा॒त्रमित्य॑ति - रा॒त्रम् । कु॒र्वी॒त॒ । य॒ज्ञ्॒क्र॒तुभि॒रिति॑ यज्ञ्क्र॒तु - भिः॒ । ए॒व । अ॒स्य॒ । दे॒वताः᳚ । वृ॒ङ्क्ते॒ । वसी॑यान् । भ॒व॒ति॒ ॥ \textbf{  22 } \newline
                  \newline
                      (प्रा॒णा॒पा॒नयो᳚ - र्भू॒तं - ॅवृ॑ङ्क्ते॒ - ऽष्टाविꣳ॑शतिश्च)  \textbf{(A7)} \newline \newline
                                \textbf{ TS 3.1.8.1} \newline
                  नि॒ग्रा॒भ्या॑ इति॑ नि - ग्रा॒भ्याः᳚ । स्थ॒ । दे॒व॒श्रुत॒ इति॑ देव - श्रुतः॑ । आयुः॑ । मे॒ । त॒र्प॒य॒त॒ । प्रा॒णमिति॑ प्र- अ॒नम् । म॒ । त॒र्प॒य॒त॒ । अ॒पा॒नमित्य॑प - अ॒नम् । मे॒ । त॒र्प॒य॒त॒ । व्या॒नमिति॑ वि - अ॒नम् । मे॒ । त॒र्प॒य॒त॒ । चक्षुः॑ । मे॒ । त॒र्प॒य॒त॒ । श्रोत्र᳚म् । मे॒ । त॒र्प॒य॒त॒ । मनः॑ । मे॒ । त॒र्प॒य॒त॒। वाच᳚म् । मे॒ । त॒र्प॒य॒त॒ । आ॒त्मान᳚म् । मे॒ । त॒र्प॒य॒त॒ । अङ्गा॑नि । म॒ । त॒र्प॒य॒त॒ । प्र॒जामिति॑ प्र - जाम् । मे॒ । त॒र्प॒य॒त॒ । प॒शून् । मे॒ । त॒र्प॒य॒त॒ । गृ॒हान् । मे॒ । त॒र्प॒य॒त॒ । ग॒णान् । मे॒ । त॒र्प॒य॒त॒ । स॒र्वग॑ण॒मिति॑ स॒र्व - ग॒ण॒म् । मा॒ । त॒र्प॒य॒त॒ । त॒र्पय॑त । मा॒ । \textbf{  23} \newline
                  \newline
                                \textbf{ TS 3.1.8.2} \newline
                  ग॒णाः । मे॒ । मा । वीति॑ । तृ॒ष॒न्न् । ओष॑धयः । वै । सोम॑स्य । विशः॑ । विशः॑ । खलु॑ । वै । राज्ञ्ः॑ । प्रदा॑तो॒रिति॒ प्र - दा॒तोः॒ । ई॒श्व॒राः । ऐ॒न्द्रः । सोमः॑ । अवी॑वृधम् । वः॒ । मन॑सा । सु॒जा॒ता॒ इति॑ सु - जा॒ताः॒ । ऋत॑प्रजाता॒ इत्यृत॑ - प्र॒जा॒ताः॒ । भगे᳚ । इत् । वः॒ । स्या॒म॒ ॥ इन्द्रे॑ण । दे॒वीः । वी॒रुधः॑ । सं॒ॅवि॒दा॒ना इति॑ सं - वि॒दा॒नाः । अन्विति॑ । म॒न्य॒न्ता॒म् । सव॑नाय । सोम᳚म् । इति॑ । आ॒ह॒ । ओष॑धीभ्य॒ इत्योष॑धि - भ्यः॒ । ए॒व । ए॒न॒म् । स्वायै᳚ । वि॒शः । स्वायै᳚ । दे॒वता॑यै । नि॒र्याच्येति॑ निः - याच्य॑ । अ॒भीति॑ । सु॒नो॒ति॒ । यः । वै । सोम॑स्य । अ॒भि॒षू॒यमा॑ण॒स्येत्य॑भि - सू॒यमा॑णस्य । \textbf{  24} \newline
                  \newline
                                \textbf{ TS 3.1.8.3} \newline
                  प्र॒थ॒मः । अꣳ॒॒शुः । स्कन्द॑ति । सः । ई॒श्व॒रः । इ॒न्द्रि॒यम् । वी॒र्य᳚म् । प्र॒जामिति॑ प्र - जाम् । प॒शून् । यज॑मानस्य । निर्.ह॑न्तो॒रिति॒ निः - ह॒न्तोः॒ । तम् । अ॒भीति॑ । म॒न्त्र॒ये॒त॒ । एति॑ । मा॒ । अ॒स्का॒न् । स॒ह । प्र॒जयेति॑ प्र - जया᳚ । स॒ह । रा॒यः । पोषे॑ण । इ॒न्द्रि॒यम् । मे॒ । वी॒र्य᳚म् । मा । निरिति॑ । व॒धीः॒ । इति॑ । आ॒शिष॒मित्या᳚-शिष᳚म् । ए॒व । ए॒ताम् । एति॑ । शा॒स्ते॒ । इ॒न्द्रि॒यस्य॑ । वी॒र्य॑स्य । प्र॒जाया॒ इति॑ प्र - जायै᳚ । प॒शू॒नाम् । अनि॑र्घाता॒येत्यनिः॑ - घा॒ता॒य॒ । द्र॒फ्सः । च॒स्क॒न्द॒ । पृ॒थि॒वीम् । अन्विति॑ । द्याम् । इ॒मम् । च॒ । योनि᳚म् । अन्विति॑ । यः । च॒ ( ) । पूर्वः॑ । तृ॒तीय᳚म् । योनि᳚म् । अन्विति॑ । स॒ञ्चर॑न्त॒मिति॑ सं - चर॑न्तम् । द्र॒फ्सम् । जु॒हो॒मि॒ । अन्विति॑ । स॒प्त । होत्राः᳚ ॥ \textbf{  25} \newline
                  \newline
                      (त॒र्पय॑त मा - ऽभिषू॒यमा॑णस्य॒ - यश्च॒ - दश॑ च)  \textbf{(A8)} \newline \newline
                                \textbf{ TS 3.1.9.1} \newline
                  यः । वै । दे॒वान् । दे॒व॒य॒श॒सेनेति॑ देव - य॒श॒सेन॑ । अ॒र्पय॑ति । म॒नु॒ष्यान्॑ । म॒नु॒ष्य॒य॒श॒सेनेति॑ मनुष्य - य॒श॒सेन॑ । दे॒व॒य॒श॒सीति॑ देव - य॒श॒सी । ए॒व । दे॒वेषु॑ । भव॑ति । म॒नु॒ष्य॒य॒श॒सीति॑ मनुष्य-य॒श॒सी । म॒नु॒ष्ये॑षु । यान् । प्रा॒चीन᳚म् । आ॒ग्र॒य॒णात् । ग्रहान्॑ । गृ॒ह्णी॒यात् । तान् । उ॒पाꣳ॒॒श्वित्यु॑प - अꣳ॒॒शु । गृ॒ह्णी॒या॒त् । यान् । ऊ॒द्‌र्ध्वान् । तान् । उ॒प॒ब्दि॒मत॒ इत्यु॑पब्दि - मतः॑ । दे॒वान् । ए॒व । तत् । दे॒व॒य॒श॒सेनेति॑ देव - य॒श॒सेन॑ । अ॒र्प॒य॒ति॒ । म॒नु॒ष्यान्॑ । म॒नु॒ष्य॒य॒श॒सेनेति॑ मनुष्य - य॒श॒सेन॑ । दे॒व॒य॒श॒सीति॑ देव - य॒श॒सी । ए॒व । दे॒वेषु॑ । भ॒व॒ति॒ । म॒नु॒ष्य॒य॒श॒सीति॑ मनुष्य - य॒श॒सी । म॒नु॒ष्ये॑षु । अ॒ग्निः । प्रा॒त॒स्स॒व॒न इति॑ प्रातः - स॒व॒ने । पा॒तु॒ ।अ॒स्मान् । वै॒श्वा॒न॒रः । म॒हि॒ना । वि॒श्वश॑म्भू॒रिति॑ वि॒श्व - श॒म्भूः॒ ॥ सः । नः॒ । पा॒व॒कः । द्रवि॑णम् । द॒धा॒तु॒ । \textbf{  26} \newline
                  \newline
                                \textbf{ TS 3.1.9.2} \newline
                  आयु॑ष्मन्तः । स॒हभ॑क्षा॒ इति॑ स॒ह-भ॒क्षाः॒ । स्या॒म॒ ॥ विश्वे᳚ । दे॒वाः । म॒रुतः॑ । इन्द्रः॑ । अ॒स्मान् । अ॒स्मिन्न् । द्वि॒तीये᳚ । सव॑ने । न । ज॒ह्युः॒ ॥ आयु॑ष्मन्तः । प्रि॒यम् । ए॒षा॒म् । वद॑न्तः । व॒यम् । दे॒वाना᳚म् । सु॒म॒ताविति॑ सु - म॒तौ । स्या॒म॒ ॥ इ॒दम् । तृ॒तीय᳚म् । सव॑नम् । क॒वी॒नाम् । ऋ॒तेन॑ । ये । च॒म॒सम् । ऐर॑यन्त ॥ ते । सौ॒ध॒न्व॒नाः । सुवः॑ । आ॒न॒शा॒नाः । स्वि॑ष्टि॒मिति॒ सु - इ॒ष्टि॒म् । नः॒ । अ॒भीति॑ । वसी॑यः । न॒य॒न्तु॒ ॥ आ॒यत॑नवती॒रित्या॒यत॑न - व॒तीः॒ । वै । अ॒न्याः । आहु॑तय॒ इत्या - हु॒त॒यः॒ । हू॒यन्ते᳚ । अ॒ना॒य॒त॒ना इत्य॑ना - य॒त॒नाः । अ॒न्याः । याः । आ॒घा॒रव॑ती॒रित्या॑घा॒र - व॒तीः॒ । ताः । आ॒यत॑नवती॒रित्या॒यत॑न - व॒तीः॒ । याः । \textbf{  27} \newline
                  \newline
                                \textbf{ TS 3.1.9.3} \newline
                  सौ॒म्याः । ताः । अ॒ना॒य॒त॒ना इत्य॑ना - य॒त॒नाः । ऐ॒न्द्र॒वा॒य॒वमित्यै᳚न्द्र - वा॒य॒वम् । आ॒दायेत्या᳚ - दाय॑ । आ॒घा॒रमित्या᳚ - घा॒रम् । एति॑ । घा॒र॒ये॒त् । अ॒द्ध्व॒रः । य॒ज्ञ्ः । अ॒यम् । अ॒स्तु॒ । दे॒वाः॒ । ओष॑धीभ्य॒ इत्योष॑धि-भ्यः॒ । प॒शवे᳚ । नः॒ । जना॑य । विश्व॑स्मै । भू॒ताय॑ । अ॒ध्व॒रः । अ॒सि॒ । सः । पि॒न्व॒स्व॒ । घृ॒तव॒दिति॑ घृ॒त - व॒त् । दे॒व॒ । सो॒म॒ । इति॑ । सौ॒म्याः । ए॒व । तत् । आहु॑ती॒रित्या - हु॒तीः॒ । आ॒यत॑नवती॒रित्या॒यत॑न - व॒तीः॒ । क॒रो॒ति॒ । आ॒यत॑नवा॒नित्या॒यत॑न - वा॒न् । भ॒व॒ति॒ । यः । ए॒वम् । वेद॑ । अथो॒ इति॑ । द्यावा॑पृथि॒वी इति॒ द्यावा᳚-पृ॒थि॒वी । ए॒व । घृ॒तेन॑ । वीति॑ । उ॒न॒त्ति॒ । ते इति॑ । व्यु॑त्ते॒ इति॒ वि - उ॒त्ते॒ । उ॒प॒जी॒व॒नीये॒ इत्यु॑प - जी॒व॒नीये᳚ । भ॒व॒तः॒ । उ॒प॒जी॒व॒नीय॒ इत्यु॑प - जी॒व॒नीयः॑ । भ॒व॒ति॒ । \textbf{  28} \newline
                  \newline
                                \textbf{ TS 3.1.9.4} \newline
                  यः । ए॒वम् । वेद॑ । ए॒षः । ते॒ । रु॒द्र॒ । भा॒गः । यम् । नि॒रया॑चथा॒ इति॑ निः-अया॑चथाः । तम् । जु॒ष॒स्व॒ । वि॒देः । गौ॒प॒त्यम् । रा॒यः । पोष᳚म् । सु॒वीर्य॒मिति॑ सु - वीर्य᳚म् । सं॒ॅव॒थ्स॒रीणा॒मिति॑ सं - व॒थ्स॒रीणा᳚म् । स्व॒स्तिम् ॥ मनुः॑ । पु॒त्रेभ्यः॑ । दा॒यम् । वीति॑ । अ॒भ॒ज॒त् । सः । नाभा॒नेदि॑ष्ठम् । ब्र॒ह्म॒चर्य॒मिति॑ ब्रह्म-चर्य᳚म् । वस॑न्तम् । निरिति॑ । अ॒भ॒ज॒त् । सः । एति॑ । अ॒ग॒च्छ॒त् । सः । अ॒ब्र॒वी॒त् । क॒था । मा॒ । निरिति॑ । अ॒भा॒क् । इति॑ । न । त्वा॒ । निरिति॑ । अ॒भा॒क्ष॒म् । इति॑ । अ॒ब्र॒वी॒त् । अङ्गि॑रसः । इ॒मे । स॒त्रम् । आ॒स॒ते॒ । ते । \textbf{  29} \newline
                  \newline
                                \textbf{ TS 3.1.9.5} \newline
                  सु॒व॒र्गमिति॑ सुवः - गम् । लो॒कम् । न । प्रेति॑ । जा॒न॒न्ति॒ । तेभ्यः॑ । इ॒दम् । ब्राह्म॑णम् । ब्रू॒हि॒ । ते । सु॒व॒र्गमिति॑ सुवः - गम् । लो॒कम् । यन्तः॑ । ये । ए॒षा॒म् । प॒शवः॑ । तान् । ते॒ । दा॒स्य॒न्ति॒ । इति॑ । तत् । ए॒भ्यः॒ । अ॒ब्र॒वी॒त् । ते । सु॒व॒र्गमिति॑ सुवः - गम् । लो॒कम् । यन्तः॑ । ये । ए॒षा॒म् । प॒शवः॑ । आसन्न्॑ । तान् । अ॒स्मै॒ । अ॒द॒दुः॒ । तम् । प॒शुभि॒रिति॑ प॒शु - भिः॒ । चर॑न्तम् । य॒ज्ञ्॒वा॒स्ताविति॑ यज्ञ्-वा॒स्तौ । रु॒द्रः । एति॑ । अ॒ग॒च्छ॒त् । सः । अ॒ब्र॒वी॒त् । मम॑ । वै । इ॒मे । प॒शवः॑ । इति॑ । अदुः॑ । वै । \textbf{  30} \newline
                  \newline
                                \textbf{ TS 3.1.9.6} \newline
                  मह्य᳚म् । इ॒मान् । इति॑ । अ॒ब्र॒वी॒त् । न । वै । तस्य॑ । ते । ई॒श॒ते॒ । इति॑ । अ॒ब्र॒वी॒त् । यत् । य॒ज्ञ्॒वा॒स्ताविति॑ यज्ञ्-वा॒स्तौ । हीय॑ते । मम॑ । वै । तत् । इति॑ । तस्मा᳚त् । य॒ज्ञ्॒वा॒स्त्विति॑ यज्ञ् - वा॒स्तु । न । अ॒भ्य॒वेत्य॒मित्य॑भि - अ॒वेत्य᳚म् । सः । अ॒ब्र॒वी॒त् । य॒ज्ञे । मा॒ । एति॑ । भ॒ज॒ । अथ॑ । ते॒ । प॒शून् । न । अ॒भीति॑ । मꣳ॒॒स्ये॒ । इति॑ । तस्मै᳚ । ए॒तम् । म॒न्थिनः॑ । सꣳ॒॒स्रा॒वमिति॑ सं - स्रा॒वम् । अ॒जु॒हो॒त् । ततः॑ । वै । तस्य॑ । रु॒द्रः । प॒शून् । न । अ॒भीति॑ । अ॒म॒न्य॒त॒ । यत्र॑ । ए॒तम् ( ) । ए॒वम् । वि॒द्वान् । म॒न्थिनः॑ । सꣳ॒॒स्रा॒वमिति॑ सं - स्रा॒वम् । जु॒होति॑ । न । तत्र॑ । रु॒द्रः । प॒शून् । अ॒भीति॑ । म॒न्य॒त॒ ॥ \textbf{  31} \newline
                  \newline
                      (द॒धा॒त्वा॒ - यत॑नवती॒र्या - उ॑पजीव॒नीयो॑ भवति॒ - तेऽ - दु॒र्वै - यत्रै॒त - मेका॑दश च)  \textbf{(A9)} \newline \newline
                                \textbf{ TS 3.1.10.1} \newline
                  जुष्टः॑ । वा॒चः । भू॒या॒स॒म् । जुष्टः॑ । वा॒चः । पत॑ये । देवि॑ । वा॒क् ॥ यत् । वा॒चः । मधु॑म॒दिति॒ मधु॑-म॒त् । तस्मिन्न्॑ । मा॒ । धाः॒ । स्वाहा᳚ । सर॑स्वत्यै ॥ ऋ॒चा । स्तोम᳚म् । समिति॑ । अ॒द्‌र्ध॒य॒ । गा॒य॒त्रेण॑ । र॒थ॒न्त॒रमिति॑ रथं-त॒रम् ॥ बृ॒हत् । गा॒य॒त्रव॑र्त॒नीति॑ गाय॒त्र-व॒र्त॒नि॒ ॥ यः । ते॒ । द्र॒फ्सः । स्कन्द॑ति । यः । ते॒ । अꣳ॒॒शुः । बा॒हुच्यु॑त॒ इति॑ बा॒हु - च्यु॒तः॒ । धि॒षण॑योः । उ॒पस्था॒दित्यु॒प - स्था॒त् ॥ अ॒द्ध्व॒र्योः । वा॒ । परीति॑ । यः । ते॒ । प॒वित्रा᳚त् । स्वाहा॑कृत॒मिति॒ स्वाहा᳚ -कृ॒त॒म् । इन्द्रा॑य । तम् । जु॒हो॒मि॒ ॥ यः । द्र॒फ्सः । अꣳ॒॒शुः । प॒ति॒तः । पृ॒थि॒व्याम् । प॒रि॒वा॒पादिति॑ परि - वा॒पात् । \textbf{  32} \newline
                  \newline
                                \textbf{ TS 3.1.10.2} \newline
                  पु॒रो॒डाशा᳚त् । क॒र॒म्भात् ॥ धा॒ना॒सो॒मादिति॑ धाना-सो॒मात् । म॒न्थिनः॑ । इ॒न्द्र॒ । शु॒क्रात् । स्वाहा॑कृत॒मिति॒ स्वाहा᳚ - कृ॒त॒म् । इन्द्रा॑य । तम् । जु॒हो॒मि॒ ॥ यः । ते॒ । द्र॒फ्सः । मधु॑मा॒निति॒ मधु॑ - मा॒न् । इ॒न्द्रि॒यावा॒निती᳚न्द्रि॒य - वा॒न् । स्वाहा॑कृत॒ इति स्वाहा᳚-कृ॒तः॒ । पुनः॑ । अ॒प्येतीत्य॑पि - एति॑ । दे॒वान् ॥ दि॒वः । पृ॒थि॒व्याः । परीति॑ । अ॒न्तरि॑क्षात् । स्वाहा॑कृत॒मिति॒ स्वाहा᳚ - कृ॒त॒म् । इन्द्रा॑य । तम् । जु॒हो॒मि॒ ॥ अ॒द्ध्व॒र्युः । वै । ऋ॒त्विजा᳚म् । प्र॒थ॒मः । यु॒ज्य॒ते॒ । तेन॑ । स्तोमः॑ । यो॒क्त॒व्यः॑ । इति॑ । आ॒हुः॒ । वाक् । अ॒ग्रे॒गा इत्य॑ग्रे - गाः । अग्र᳚ । ए॒तु॒ । ऋ॒जु॒गा इत्यृ॑जु - गाः । दे॒वेभ्यः॑ । यशः॑ । मयि॑ । दध॑ती । प्रा॒णानिति॑ प्र - अ॒नान् । प॒शुषु॑ । प्र॒जामिति॑ प्र - जाम् । मयि॑ । \textbf{  33} \newline
                  \newline
                                \textbf{ TS 3.1.10.3} \newline
                  च॒ । यज॑माने । च॒ । इति॑ । आ॒ह॒ । वाच᳚म् । ए॒व । तत् । य॒ज्ञ्॒मु॒ख इति॑ यज्ञ् - मु॒खे । यु॒न॒क्ति॒ । वास्तु॑ । वै । ए॒तत् । य॒ज्ञ्स्य॑ । क्रि॒य॒ते॒ । यत् । ग्रहान्॑ । गृ॒ही॒त्वा । ब॒हि॒ष्प॒व॒मा॒नमिति॑ बहिः - प॒व॒मा॒नम् । सर्प॑न्ति । परा᳚ञ्चः । हि । यन्ति॑ । परा॑चीभिः । स्तु॒वते᳚ । वै॒ष्ण॒व्या । ऋ॒चा । पुनः॑ । एत्येत्या᳚ - इत्य॑ । उपेति॑ । ति॒ष्ठ॒ते॒ । य॒ज्ञ्ः । वै । विष्णुः॑ । य॒ज्ञ्म् । ए॒व । अ॒कः॒ । विष्णो॒ इति॑ । त्वम् । नः॒ । अन्त॑मः । शर्म॑ । य॒च्छ॒ । स॒ह॒न्त्य॒ ॥ प्रेति॑ । ते॒ । धाराः᳚ । म॒धु॒श्चुत॒ इति॑ मधु-श्चुतः॑ । उथ्स᳚म् । दु॒ह्र॒ते॒ ( ) । अक्षि॑तम् । इति॑ । आ॒ह॒ । यत् । ए॒व । अ॒स्य॒ । शया॑नस्य । उ॒प॒शुष्य॒तीत्यु॑प - शुष्य॑ति । तत् । ए॒व । अ॒स्य॒ । ए॒तेन॑ । एति॑ । प्या॒य॒य॒ति॒ ॥ \textbf{  34} \newline
                  \newline
                      (प॒रि॒वा॒पात् - प्र॒जां मयि॑ - दुह्रते॒ - चतु॑र्दश च)  \textbf{(A10)} \newline \newline
                                \textbf{ TS 3.1.11.1} \newline
                  अ॒ग्निना᳚ । र॒यिम् । अ॒श्न॒व॒त् । पोष᳚म् । ए॒व । दि॒वेदि॑व॒ इति॑ दि॒वे - दि॒व॒ ॥ य॒शस᳚म् । वी॒रव॑त्तम॒मिति॑ वी॒रव॑त् - त॒म॒म् ॥ गोमा॒निति॒ गो - मा॒न् । अ॒ग्ने॒ । अवि॑मा॒नित्यवि॑ - मा॒न् । अ॒श्वी । य॒ज्ञ्ः । नृ॒वथ्स॒खेति॑ नृ॒वत् - स॒खा॒ । सद᳚म् । इत् । अ॒प्र॒मृ॒ष्य इत्य॑प्र - मृ॒ष्यः ॥ इडा॑वा॒नितीडा᳚ - वा॒न् । ए॒षः । अ॒सु॒र॒ । प्र॒जावा॒निति॑ प्र॒जा - वा॒न् । दी॒र्घः । र॒यिः । पृ॒थु॒बु॒द्ध्न इति॑ पृथु - बु॒द्ध्नः । स॒भावा॒निति॑ स॒भा - वा॒न् ॥ एति॑ । प्या॒य॒स्व॒ । समिति॑ । ते॒ ॥ इ॒ह । त्वष्टा॑रम् । अ॒ग्रि॒यम् । वि॒श्वरू॑प॒मिति॑ वि॒श्व - रू॒प॒म् । उपेति॑ । ह्व॒ये॒ ॥ अ॒स्माक᳚म् । अ॒स्तु॒ । केव॑लः ॥ तत् । नः॒ । तु॒रीप᳚म् । अध॑ । पो॒ष॒यि॒त्नु । देव॑ । त्व॒ष्टः॒ । वीति॑ । र॒रा॒णः । स्य॒स्व॒ ॥ यतः॑ । वी॒रः । \textbf{  35} \newline
                  \newline
                                \textbf{ TS 3.1.11.2} \newline
                  क॒र्म॒ण्यः॑ । सु॒दक्ष॒ इति॑ सु - दक्षः॑ । यु॒क्तग्रा॒वेति॑ यु॒क्त - ग्रा॒वा॒ । जाय॑ते । दे॒वका॑म॒ इति॑ दे॒व - का॒मः॒ ॥ शि॒वः । त्व॒ष्टः॒ । इ॒ह । एति॑ । ग॒हि॒ । वि॒भुरिति॑ वि - भुः । पोषे᳚ । उ॒त । त्मना᳚ ॥ य॒ज्ञे य॑ज्ञ्॒ इति॑ य॒ज्ञे - य॒ज्ञे॒ । नः॒ । उदिति॑ । अ॒व॒ ॥ पि॒शङ्ग॑रूप॒ इति॑ पि॒शङ्ग॑-रू॒पः॒ । सु॒भर॒ इति॑ सु - भरः॑ । व॒यो॒धा इति॑ वयः - धाः । श्रु॒ष्टी । वी॒रः । जा॒य॒ते॒ । दे॒वका॑म॒ इति॑ दे॒व - का॒मः॒ ॥ प्र॒जामिति॑ प्र - जाम् । त्वष्टा᳚ । वीति॑ । स्य॒तु॒ । नाभि᳚म् । अ॒स्मे इति॑ । अथ॑ । दे॒वाना᳚म् । अपीति॑ । ए॒तु॒ । पाथः॑ ॥ प्रेति॑ । नः॒ । दे॒वी । एति॑ । नः॒ । दि॒वः ॥ पी॒पि॒वाꣳस᳚म् । सर॑स्वतः । स्तन᳚म् । यः । वि॒श्वद॑र्.शत॒ इति॑ वि॒श्व - द॒र्॒.श॒तः॒ ॥ धु॒क्षी॒महि॑ । प्र॒जामिति॑ प्र - जाम् । इष᳚म् ॥ \textbf{  36} \newline
                  \newline
                                \textbf{ TS 3.1.11.3} \newline
                  ये । ते॒ । स॒र॒स्वः॒ । ऊ॒र्मयः॑ । मधु॑मन्त॒ इति॒ मधु॑ - म॒न्तः॒ । घृ॒त॒श्चुत॒ इति॑ घृत - श्चुतः॑ ॥ तेषा᳚म् । ते॒ । सु॒म्नम् । ई॒म॒हे॒ ॥ यस्य॑ । व्र॒तम् । प॒शवः॑ । यन्ति॑ । सर्वे᳚ । यस्य॑ । व्र॒तम् । उ॒प॒तिष्ठ॑न्त॒ इत्यु॑प -तिष्ठ॑न्ते । आपः॑ ॥ यस्य॑ । व्र॒ते । पु॒ष्टि॒पति॒रिति॑ पुष्टि-पतिः॑ । निवि॑ष्ट॒ इति॒ नि - वि॒ष्टः॒ । तम् । सर॑स्वन्तम् । अव॑से । ह॒वे॒म॒ ॥ दि॒व्यम् । सु॒प॒र्णमिति॑ सु - प॒र्णम् । व॒य॒सम् । बृ॒हन्त᳚म् । अ॒पाम् । गर्भ᳚म् । वृ॒ष॒भम् । ओष॑धीनाम् ॥ अ॒भी॒प॒तः । वृ॒ष्ट्या । त॒र्पय॑न्तम् । तम् । सर॑स्वन्तम् । अव॑से । हु॒वे॒म॒ ॥ सिनी॑वालि । पृथु॑ष्टुक॒ इति॒ पृथु॑ - स्तु॒के॒ । या । दे॒वाना᳚म् । असि॑ । स्वसा᳚ ॥ जु॒षस्व॑ । ह॒व्यम् । \textbf{  37} \newline
                  \newline
                                \textbf{ TS 3.1.11.4} \newline
                  आहु॑त॒मित्या - हु॒त॒म् । प्र॒जामिति॑ प्र-जाम् । दे॒वि॒ । दि॒दि॒ड्ढि॒ । नः॒ ॥ या । सु॒पा॒णिरिति॑ सु - पा॒णिः । स्व॒ङ्गु॒रिरिति॑ सु - अ॒ङ्गु॒रिः । सु॒षूमेति॑ सु - सूमा᳚ । ब॒हु॒सूव॒रीति॑ बहु - सूव॑री ॥ तस्यै᳚ । वि॒श्पत्नि॑यै । ह॒विः । सि॒नी॒वा॒ल्यै । जु॒हो॒त॒न॒ ॥ इन्द्र᳚म् । वः॒ । वि॒श्वतः॑ । परीति॑ । इन्द्र᳚म् । नरः॑ ॥ असि॑तवर्णा॒ इत्यसि॑त - व॒र्णाः॒ । हर॑यः । सु॒प॒र्णा इति॑ सु - प॒र्णाः । मिहः॑ । वसा॑नाः । दिव᳚म् । उदिति॑ । प॒त॒न्ति॒ ॥ ते । एति॑ । अ॒व॒वृ॒त्र॒न्न् । सद॑नानि । कृ॒त्वा । आत् । इत् । पृ॒थि॒वी । घृ॒तैः । वीति॑ । उ॒द्य॒ते॒ ॥ हिर॑ण्यकेश॒ इति॒ हिर॑ण्य - के॒शः॒ । रज॑सः । वि॒सा॒र इति॑ वि - सा॒रे । अहिः॑ । धुनिः॑ । वातः॑ । इ॒व॒ । ध्रजी॑मान् ॥ शुचि॑भ्राजा॒ इति॒ शुचि॑ - भ्रा॒जाः॒ । उ॒षसः॑ । \textbf{  38} \newline
                  \newline
                                \textbf{ TS 3.1.11.5} \newline
                  नवे॑दाः । यश॑स्वतीः । अ॒प॒स्युवः॑ । न । स॒त्याः ॥ एति॑ । ते॒ । सु॒प॒र्णा इति॑ सु - प॒र्णाः । अ॒मि॒न॒न्त॒ । एवैः᳚ । कृ॒ष्णः । नो॒ना॒व॒ । वृ॒ष॒भः । यदि॑ । इ॒दम् ॥ शि॒वाभिः॑ । न । स्मय॑मानाभिः । एति॑ । अ॒गा॒त् । पत॑न्ति । मिहः॑ । स्त॒नय॑न्ति । अ॒भ्रा ॥ वा॒श्रा । इ॒व॒ । वि॒द्युदिति॑ वि-द्युत् । मि॒मा॒ति॒ । व॒थ्सम् । न । मा॒ता । सि॒ष॒क्ति॒ ॥ यत् । ए॒षा॒म् । वृ॒ष्टिः । अस॑र्जि ॥ पर्व॑तः । चि॒त् । महि॑ । वृ॒द्धः । बि॒भा॒य॒ । दि॒वः । चि॒त् । सानु॑ । रे॒ज॒त॒ । स्व॒ने । वः॒ ॥ यत् । क्रीड॑थ । म॒रु॒तः॒ । \textbf{  39} \newline
                  \newline
                                \textbf{ TS 3.1.11.6} \newline
                  ऋ॒ष्टि॒मन्त॒ इत्यृ॑ष्टि-मन्तः॑ । आपः॑ । इ॒व॒ । स॒द्ध्रिय॑ञ्चः । ध॒व॒द्ध्वे॒ ॥ अ॒भीति॑ । क्र॒न्द॒ । स्त॒नय॑ । गर्भ᳚म् । एति॑ । धाः॒ । उ॒द॒न्वतेत्यु॑दन्न् - वता᳚ । परीति॑ । दी॒य॒ । रथे॑न ॥ दृति᳚म् । स्विति॑ । क॒र्.॒ष॒ । विषि॑त॒मिति॒ वि - सि॒त॒म् । न्य॑ञ्चम् । स॒माः । भ॒व॒न्तु॒ । उ॒द्वतेत्य॑त् - वता᳚ । नि॒पा॒दा इति॑ नि - पा॒दाः ॥ त्वम् । त्या । चि॒त् । अच्यु॑ता । अग्ने᳚ । प॒शुः । न । यव॑से ॥ धाम॑ । ह॒ । यत् । ते॒ । अ॒ज॒र॒ । वना᳚ । वृ॒श्चन्ति॑ । शिक्व॑सः ॥ अग्ने᳚ । भूरी॑णि । तव॑ । जा॒त॒वे॒द॒ इति॑ जात - वे॒दः॒ । देव॑ । स्व॒धा॒व॒ इति॑ स्वधा - वः॒ । अ॒मृत॑स्य । धाम॑ ॥ याः । च॒ । \textbf{  40} \newline
                  \newline
                                \textbf{ TS 3.1.11.7} \newline
                  मा॒याः । मा॒यिना᳚म् । वि॒श्व॒मि॒न्वेति॑ विश्वम्-इ॒न्व॒ । त्वे इति॑ । पू॒र्वीः । स॒न्द॒धुरिति॑ सं - द॒धुः । पृ॒ष्ट॒ब॒न्धो॒ इति॑ पृष्ट - ब॒न्धो॒ ॥ दि॒वः । नः॒ । वृ॒ष्टिम् । म॒रु॒तः॒ । र॒री॒द्ध्व॒म् । प्रेति॑ । पि॒न्व॒त॒ । वृष्णः॑ । अश्व॑स्य । धाराः᳚ ॥ अ॒र्वाङ् । ए॒तेन॑ । स्त॒न॒यि॒त्नुना᳚ । एति॑ । इ॒हि॒ । अ॒पः । नि॒षि॒ञ्चन्निति॑ नि -सि॒ञ्चन्न् । असु॑रः । पि॒ता । नः॒ ॥ पिन्व॑न्ति । अ॒पः । म॒रुतः॑ । सु॒दान॑व॒ इति॑ सु - दान॑वः । पयः॑ । घृ॒तव॒दिति॑ घृ॒त - व॒त् । वि॒दथे॑षु । आ॒भुव॒ इत्या᳚ - भुवः॑ ॥ अत्य᳚म् । न । मि॒हे । वीति॑ । न॒य॒न्ति॒ । वा॒जिन᳚म् । उथ्स᳚म् । दु॒ह॒न्ति॒ । स्त॒नय॑न्तम् । अक्षि॑तम् ॥ उ॒द॒प्रुत॒ इत्यु॑द - प्रुतः॑ । म॒रु॒तः॒ । तान् । इ॒य॒र्त॒ । वृष्टि᳚म् । \textbf{  41} \newline
                  \newline
                                \textbf{ TS 3.1.11.8} \newline
                  ये । विश्वे᳚ । म॒रुतः॑ । जु॒नन्ति॑ ॥ क्रोशा॑ति । गर्दा᳚ । क॒न्या᳚ । इ॒व॒ । तु॒न्ना । पेरु᳚म् । तु॒ञ्जा॒ना । पत्या᳚ । इ॒व॒ । जा॒या ॥ घृ॒तेन॑ । द्यावा॑पृथि॒वी इति॒ द्यावा᳚ - पृ॒थि॒वी । मधु॑ना । समिति॑ । उ॒क्ष॒त॒ । पय॑स्वतीः । कृ॒णु॒त॒ । आपः॑ । ओष॑धीः ॥ ऊर्ज᳚म् । च॒ । तत्र॑ । सु॒म॒तिमिति॑ सु - म॒तिम् । च॒ । पि॒न्व॒थ॒ । यत्र॑ । न॒रः॒ । म॒रु॒तः॒ । सि॒ञ्चथ॑ । मधु॑ ॥ उदिति॑ । उ॒ । त्यम् । चि॒त्रम् ॥ औ॒र्व॒भृ॒गु॒वदित्यौ᳚वभृगु - वत् । शुचि᳚म् । अ॒प्न॒वा॒न॒वदित्य॑प्नवान - वत् । एति॑ । हु॒वे॒ ॥ अ॒ग्निम् । स॒मु॒द्रवा॑सस॒मिति॑ समु॒द्र - वा॒स॒स॒म् ॥ एति॑ । स॒वम् । स॒वि॒तुः । य॒था॒ । भग॑स्य ( ) । इ॒व॒ । भु॒जिम् । हु॒वे॒ ॥ अ॒ग्निम् । स॒मु॒द्रवा॑सस॒मिति॑ समु॒द्र - वा॒स॒स॒म् ॥ हु॒वे । वात॑स्वन॒मिति॒ वात॑ - स्व॒न॒म् । क॒विम् । प॒र्जन्य॑क्रन्द्य॒मिति॑ प॒र्जन्य॑ - क्र॒न्द्य॒म् । सहः॑ ॥ अ॒ग्निम् । स॒मु॒द्रवा॑सस॒मिति॑ समु॒द्र - वा॒स॒स॒म् ॥ \textbf{  42 } \newline
                  \newline
                      (वी॒र - इषꣳ॑ - ह॒व्य - मु॒षसो॑ - मरुत - श्च॒ - वृष्टिं॒ - भग॑स्य॒ - द्वाद॑श च)  \textbf{(A11)} \newline \newline
\textbf{praSna korvai with starting padams of 1 to 11 anuvAkams :-} \newline
(प्र॒जाप॑तिरकामयतै॒ - ष ते॑ गाय॒त्रो - य॒ज्ञ्ं ॅवै - प्र॒जाप॑ते॒र्जाय॑मानाः - प्राजाप॒त्या - यो वा अय॑थादेवत - मि॒ष्टर्गो॑ - निग्रा॒भ्याः᳚ स्थ॒ - यो वै दे॒वा - ञ्जुष्टो॒ - ऽग्निना॑ र॒यि - मेका॑दश ) \newline

\textbf{korvai with starting padams of1, 11, 21 series of pa~jcAtis :-} \newline
(प्र॒जाप॑तिरकामयत - प्र॒जाप॑ते॒र्जाय॑माना॒ - व्याय॑च्छन्ते॒ - मह्य॑मि॒मा - न्मा॒या मा॒यिना॒न् - द्विच॑त्वारिꣳशत्) \newline

\textbf{first and last padam of first praSnam of kANDam 3:-} \newline
(प्र॒जाप॑तिरकामयता॒ -ऽग्निꣳ स॑मु॒द्रवा॑ससं ) \newline 


॥ हरिः॑ ॐ ॥
॥ कृष्ण यजुर्वेदीय तैत्तिरीय संहितायां तृतीयकाण्डे प्रथमः प्रश्नः समाप्तः ॥ \newline
\pagebreak
3.1.1   appendix\\1.      3.1.11.1 - आ प्या॑यस्व॒ >1\\आप्या॑स्व॒ समे॑तु ते वि॒श्वतः॑ सोम॒ वृष्णि॑यं ।\\भवा॒ वाज॑स्य संग॒थे ॥ (Appearing in TS 4-2-7-4)\\\\2.      3.1.11.1 - सन्ते॑ >2\\सं ते॒ पयाꣳ॑सि॒ समु॑ यन्तु॒ वाजाः॒ सं ॅवृष्णि॑यान्. यभिमाति॒षाहः॑ ।\\आ॒प्याय॑मानो अ॒मृता॑य सोम दि॒वि श्रवाꣳ॑स्युत्त॒मानि॑ धिष्व ॥ \\(Appearing in TS 4-2-7-4)\\\\3 ऽ 4. 3.1.11.2 - प्रणो॑ दे॒व्या>3, नो॑ दि॒वः>4\\प्रणो॑ दे॒वी सर॑स्वती॒ वाजे॑भिर्वा॒जिनी॑वती । धी॒नाम॑वि॒त्र्य॑वतु ॥\\\\आ नो॑ दि॒वो बृ॑ह॒तः पर्व॑ता॒दा सर॑स्वती यज॒ता ग॑न्तु य॒ज्ञ्ं । \\हवं॑ दे॒वी जु॑जुषा॒णा घृ॒ताची॑ श॒ग्मां नो॒ वाच॑मुश॒ती शृ॑णोतु ।\\(Appearing in TS 1.8.22.1)\\\\5 ऽ 6. 3.1.11.4 - इन्द्रं॑ ॅवो वि॒श्वत॒स्परी>5 ,न्द्रं॒ नरः॑>6\\इन्द्रं॑ ॅवो वि॒श्वत॒स्परि॒ हवा॑महे॒ जने᳚भ्यः । अ॒स्माक॑मस्तु॒ केव॑लः ॥ \\\\इन्द्रं॒ नरो॑ ने॒मधि॑ता हवन्ते॒ यत्पार्या॑ यु॒नज॑ते॒ धिय॒स्ताः ।\\शूरो॒ नृषा॑ता॒ शव॑सश्चका॒न आ गोम॑ति व्र॒जे भ॑जा॒ त्वन्नः॑ ॥\\(Appearing in TS 1.6.12.1)\\\\7 ऽ 8. 3.1.11.8 - उदु॒त्यं>7, चि॒त्रं>8 ।\\उदु॒ त्यं जा॒तवे॑दसं दे॒वं ॅव॑हन्ति के॒तवः॑ । दृ॒शे विश्वा॑य॒ सूर्यं᳚ ॥\\\\चि॒त्रं दे॒वाना॒मुद॑गा॒दनी॑कं॒ चक्षु॑ र्मि॒त्रस्य॒ वरु॑णस्या॒ऽग्नेः ।\\आऽ प्रा॒ द्यावा॑पृथि॒वी अ॒न्तरि॑क्षꣳ॒॒ सूर्य॑ आ॒त्मा जग॑तस्त॒स्थुष॑श्च ॥ \\(Appearing in TS 1.4.43.1)\\
\pagebreak
        


\end{document}
