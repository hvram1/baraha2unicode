\documentclass[17pt]{extarticle}
\usepackage{babel}
\usepackage{fontspec}
\usepackage{polyglossia}
\usepackage{extsizes}



\setmainlanguage{sanskrit}
\setotherlanguages{english} %% or other languages
\setlength{\parindent}{0pt}
\pagestyle{myheadings}
\newfontfamily\devanagarifont[Script=Devanagari]{AdishilaVedic}


\newcommand{\VAR}[1]{}
\newcommand{\BLOCK}[1]{}




\begin{document}
\begin{titlepage}
    \begin{center}
 
\begin{sanskrit}
    { \Large
    ॐ नमः परमात्मने, श्री महागणपतये नमः, श्री गुरुभ्यो नमः
ह॒रिः॒ ॐ 
    }
    \\
    \vspace{2.5cm}
    \mbox{ \Huge
    2.6      द्वितीयकाण्डे षष्ठः प्रश्नः - अवशिष्टकर्माभिधानं   }
\end{sanskrit}
\end{center}

\end{titlepage}
\tableofcontents

ॐ नमः परमात्मने, श्री महागणपतये नमः, 
श्री गुरुभ्यो नमः, ह॒रिः॒ ॐ \newline
2.6      द्वितीयकाण्डे षष्ठः प्रश्नः - अवशिष्टकर्माभिधानं \newline

\addcontentsline{toc}{section}{ 2.6      द्वितीयकाण्डे षष्ठः प्रश्नः - अवशिष्टकर्माभिधानं}
\markright{ 2.6      द्वितीयकाण्डे षष्ठः प्रश्नः - अवशिष्टकर्माभिधानं \hfill https://www.vedavms.in \hfill}
\section*{ 2.6      द्वितीयकाण्डे षष्ठः प्रश्नः - अवशिष्टकर्माभिधानं }
                                \textbf{ TS 2.6.1.1} \newline
                  स॒मिध॒ इति॑ सं - इधः॑ । य॒ज॒ति॒ । व॒स॒न्तम् । ए॒व । ऋ॒तू॒नाम् । अवेति॑ । रु॒न्धे॒ । तनू॒नपा॑त॒मिति॒ तनू᳚ - नपा॑तम् । य॒ज॒ति॒ । ग्री॒ष्मम् । ए॒व । अवेति॑ । रु॒न्धे॒ । इ॒डः । य॒ज॒ति॒ । व॒र्॒.षाः । ए॒व । अवेति॑ । रु॒न्धे॒ । ब॒र्॒.हिः । य॒ज॒ति॒ । श॒रद᳚म् । ए॒व । अवेति॑ । रु॒न्धे॒ । स्वा॒हा॒का॒रमिति॑ स्वाहा - का॒रम् । य॒ज॒ति॒ । हे॒म॒न्तम् । ए॒व । अवेति॑ । रु॒न्धे॒ । तस्मा᳚त् । स्वाहा॑कृता॒ इति॒ स्वाहा᳚-कृ॒ताः॒ । हेमन्न्॑ । प॒शवः॑ । अवेति॑ । सी॒द॒न्ति॒ । स॒मिध॒ इति॑ सं - इधः॑ । य॒ज॒ति॒ । उ॒षसः॑ । ए॒व । दे॒वता॑नाम् । अवेति॑ । रु॒न्धे॒ । तनू॒नपा॑त॒मिति॒ तनू᳚ - नपा॑तम् । य॒ज॒ति॒ । य॒ज्ञ्म् । ए॒व । अवेति॑ । रु॒न्धे॒ । \textbf{  1} \newline
                  \newline
                                \textbf{ TS 2.6.1.2} \newline
                  इ॒डः । य॒ज॒ति॒ । प॒शून् । ए॒व । अवेति॑ । रु॒न्धे॒ । ब॒र्॒.हिः । य॒ज॒ति॒ । प्र॒जामिति॑ प्र - जाम् । ए॒व । अवेति॑ । रु॒न्धे॒ । स॒मान॑यत॒ इति॑ सं - आन॑यते । उ॒प॒भृत॒ इत्यु॑प - भृतः॑ । तेजः॑ । वै । आज्य᳚म् । प्र॒जा इति॑ प्र - जाः । ब॒र्॒.हिः । प्र॒जास्विति॑ प्र - जासु॑ । ए॒व । तेजः॑ । द॒धा॒ति॒ । स्वा॒हा॒का॒रमिति॑ स्वाहा-का॒रम् । य॒ज॒ति॒ । वाच᳚म् । ए॒व । अवेति॑ । रु॒न्धे॒ । दश॑ । समिति॑ । प॒द्य॒न्ते॒ । दशा᳚क्ष॒रेति॒ दश॑ - अ॒क्ष॒रा॒ । वि॒राडिति॑ वि - राट् । अन्न᳚म् । वि॒राडिति॑ वि - राट् । वि॒राजेति॑ वि - राजा᳚ । ए॒व । अ॒न्नाद्य॒मित्य॑न्न - अद्य᳚म् । अवेति॑ । रु॒न्धे॒ । स॒मिध॒ इति॑ सं - इधः॑ । य॒ज॒ति॒ । अ॒स्मिन्न् । ए॒व । लो॒के । प्रतीति॑ । ति॒ष्ठ॒ति॒ । तनू॒नपा॑त॒मिति॒ तनू᳚ - नपा॑तम् । य॒ज॒ति॒ । \textbf{  2} \newline
                  \newline
                                \textbf{ TS 2.6.1.3} \newline
                  य॒ज्ञे । ए॒व । अ॒न्तरि॑क्षे । प्रतीति॑ । ति॒ष्ठ॒ति॒ । इ॒डः । य॒ज॒ति॒ । प॒शुषु॑ । ए॒व । प्रतीति॑ । ति॒ष्ठ॒ति॒ । ब॒र्॒.हिः । य॒ज॒ति॒ । ये । ए॒व । दे॒व॒याना॒ इति॑ देव - यानाः᳚ । पन्था॑नः । तेषु॑ । ए॒व । प्रतीति॑ । ति॒ष्ठ॒ति॒ । स्वा॒हा॒का॒रमिति॑ स्वाहा - का॒रम् । य॒ज॒ति॒ । सु॒व॒र्ग इति॑ सुवः - गे । ए॒व । लो॒के । प्रतीति॑ । ति॒ष्ठ॒ति॒ । ए॒ताव॑न्तः । वै । दे॒व॒लो॒का इति॑ देव - लो॒काः । तेषु॑ । ए॒व । य॒था॒पू॒र्वमिति॑ यथा - पू॒र्वम् । प्रतीति॑ । ति॒ष्ठ॒ति॒ । दे॒वा॒सु॒रा इति॑ देव-अ॒सु॒राः । ए॒षु । लो॒केषु॑ । अ॒स्प॒र्द्ध॒न्त॒ । ते । दे॒वाः । प्र॒या॒जैरिति॑ प्र-या॒जैः । ए॒भ्यः । लो॒केभ्यः॑ । असु॑रान् । प्रेति॑ । अ॒नु॒द॒न्त॒ । तत् । प्र॒या॒जाना॒मिति॑ प्र - या॒जाना᳚म् । \textbf{  3} \newline
                  \newline
                                \textbf{ TS 2.6.1.4} \newline
                  प्र॒या॒ज॒त्वमिति॑ प्रयाज - त्वम् । यस्य॑ । ए॒वम् । वि॒दुषः॑ । प्र॒या॒जा इति॑ प्र-या॒जाः । इ॒ज्यन्ते᳚ । प्रेति॑ । ए॒भ्यः । लो॒केभ्यः॑ । भ्रातृ॑व्यान् । नु॒द॒ते॒ । अ॒भि॒क्राम॒मित्य॑भि - क्राम᳚म् । जु॒हो॒ति॒ । अ॒भिजि॑त्या॒ इत्य॒भि - जि॒त्यै॒ । यः । वै । प्र॒या॒जाना॒मिति॑ प्र - या॒जाना᳚म् । मि॒थु॒नम् । वेद॑ । प्रेति॑ । प्र॒जयेति॑ प्र - जया᳚ । प॒शुभि॒रिति॑ प॒शु - भिः॒ । मि॒थु॒नैः । जा॒य॒ते॒ । स॒मिध॒ इति॑ सं - इधः॑ । ब॒ह्वीः । इ॒व॒ । य॒ज॒ति॒ । तनू॒नपा॑त॒मिति॒ तनू᳚ - नपा॑तम् । एक᳚म् । इ॒व॒ । मि॒थु॒नम् । तत् । इ॒डः । ब॒ह्वीः । इ॒व॒ । य॒ज॒ति॒ । ब॒र्॒.हिः । एक᳚म् । इ॒व॒ । मि॒थु॒नम् । तत् । ए॒तत् । वै । प्र॒या॒जाना॒मिति॑ प्र - या॒जाना᳚म् । मि॒थु॒नम् । यः । ए॒वम् । वेद॑ । प्रेति॑ । \textbf{  4} \newline
                  \newline
                                \textbf{ TS 2.6.1.5} \newline
                  प्र॒जयेति॑ प्र - जया᳚ । प॒शुभि॒रिति॑ प॒शु - भिः॒ । मि॒थु॒नैः । जा॒य॒ते॒ । दे॒वाना᳚म् । वै । अनि॑ष्टाः । दे॒वताः᳚ । आसन्न्॑ । अथ॑ । असु॑राः । य॒ज्ञ्म् । अ॒जि॒घाꣳ॒॒स॒न्न् । ते । दे॒वाः । गा॒य॒त्रीम् । वीति॑ । औ॒ह॒न्न् । पञ्च॑ । अ॒क्षरा॑णि । प्रा॒चीना॑नि । त्रीणि॑ । प्र॒ती॒चीना॑नि । ततः॑ । वर्म॑ । य॒ज्ञाय॑ । अभ॑वत् । वर्म॑ । यज॑मानाय । यत् । प्र॒या॒जा॒नू॒या॒जा इति॑ प्रयाज - अ॒नू॒या॒जाः । इ॒ज्यन्ते᳚ । वर्म॑ । ए॒व । तत् । य॒ज्ञाय॑ । क्रि॒य॒ते॒ । वर्म॑ । यज॑मानाय । भ्रातृ॑व्याभिभूत्या॒ इति॒ भ्रातृ॑व्य - अ॒भि॒भू॒त्यै॒ । तस्मा᳚त् । वरू॑थम् । पु॒रस्ता᳚त् । वर्.षी॑यः । प॒श्चात् । ह्रसी॑यः । दे॒वाः । वै । पु॒रा । रक्षो᳚भ्य॒ इति॒ रक्षः॑-भ्यः॒ । \textbf{  5} \newline
                  \newline
                                \textbf{ TS 2.6.1.6} \newline
                  इति॑ । स्वा॒हा॒का॒रेणेति॑ स्वाहा - का॒रेण॑ । प्र॒या॒जेष्विति॑ प्र - या॒जेषु॑ । य॒ज्ञ्म् । सꣳ॒॒स्थाप्य॒मिति॑ सं - स्थाप्य᳚म् । अ॒प॒श्य॒न्न् । तम् । स्वा॒हा॒का॒रेणेति॑ स्वाहा - का॒रेण॑ । प्र॒या॒जेष्विति॑ प्र - या॒जेषु॑ । समिति॑ । अ॒स्था॒प॒य॒न्न् । वीति॑ । वै । ए॒तत् । य॒ज्ञ्म् । छि॒न्द॒न्ति॒ । यत् । स्वा॒हा॒का॒रेणेति॑ स्वाहा - का॒रेण॑ । प्र॒या॒जेष्विति॑ प्र- या॒जेषु॑ । सꣳ॒॒स्था॒पय॒न्तीति॑ सं - स्था॒पय॑न्ति । प्र॒या॒जानिति॑ प्र - या॒जान् । इ॒ष्ट्वा । ह॒वीꣳषि॑ । अ॒भीति॑ । घा॒र॒य॒ति॒ । य॒ज्ञ्स्य॑ । संत॑त्या॒ इति॒ सं - त॒त्यै॒ । अथो॒ इति॑ । ह॒विः । ए॒व । अ॒कः॒ । अथो॒ इति॑ । य॒था॒पू॒र्वमिति॑ यथा - पू॒र्वम् । उपेति॑ । ए॒ति॒ । पि॒ता । वै । प्र॒या॒जा इति॑ प्र - या॒जाः । प्र॒जेति॑ प्र - जा । अ॒नू॒या॒जा इत्य॑नु - या॒जाः । यत् । प्र॒या॒जानिति॑ प्र - या॒जान् । इ॒ष्ट्वा । ह॒वीꣳषि॑ । अ॒भि॒घा॒रय॒तीत्य॑भि - घा॒रय॑ति । पि॒ता । ए॒व । तत् । पु॒त्रेण॑ । साधा॑रणम् । \textbf{  6} \newline
                  \newline
                                \textbf{ TS 2.6.1.7} \newline
                  कु॒रु॒ते॒ । तस्मा᳚त् । आ॒हुः॒ । यः । च॒ । ए॒वम् । वेद॑ । यः । च॒ । न । क॒था । पु॒त्रस्य॑ । केव॑लम् । क॒था । साधा॑रणम् । पि॒तुः । इति॑ । अस्क॑न्नम् । ए॒व । तत् । यत् । प्र॒या॒जेष्विति॑ प्र - या॒जेषु॑ । इ॒ष्टेषु॑ । स्कन्द॑ति । गा॒य॒त्री । ए॒व । तेन॑ । गर्भ᳚म् । ध॒त्ते॒ । सा । प्र॒जामिति॑ प्र - जाम् । प॒शून् । यज॑मानाय । प्रेति॑ । ज॒न॒य॒ति॒ ॥ \textbf{  7} \newline
                  \newline
                      (य॒ज॒ति॒ य॒ज्ञामे॒वाव॑ रुन्धे॒ - तनू॒नपा॑तं ॅयजति - प्रया॒जाना॑ ट्ट मे॒वं ॅवेद॒ प्र - रक्षो᳚भ्यः॒ - साधा॑रणं॒ - पञ्च॑त्रिꣳशच्च )  \textbf{(A1)} \newline \newline
                                \textbf{ TS 2.6.2.1} \newline
                  चक्षु॑षी॒ इति॑ । वै । ए॒ते इति॑ । य॒ज्ञ्स्य॑ । यत् । आज्य॑भाग॒वित्याज्य॑ - भा॒गौ॒ । यत् । आज्य॑भागा॒वित्याज्य॑-भा॒गौ॒ । यज॑ति । चक्षु॑षी॒ इति॑ । ए॒व । तत् । य॒ज्ञ्स्य॑ । प्रतीति॑ । द॒धा॒ति॒ । पू॒र्वा॒र्द्ध इति॑ पूर्व - अ॒र्द्धे । जु॒हो॒ति॒ । तस्मा᳚त् । पू॒र्वा॒र्द्ध इति॑ पूर्व - अ॒र्द्धे । चक्षु॑षी॒ इति॑ । प्र॒बाहु॒गिति॑ प्र - बाहु॑क् । जु॒हो॒ति॒ । तस्मा᳚त् । प्र॒बाहु॒गिति॑ प्र - बाहु॑क् । चक्षु॑षी॒ इति॑ । दे॒व॒लो॒कमिति॑ देव-लो॒कम् । वै । अ॒ग्निना᳚ । यज॑मानः । अन्विति॑ । प॒श्य॒ति॒ । पि॒तृ॒लो॒कमिति॑ पितृ-लो॒कम् । सोमे॑न । उ॒त्त॒रा॒र्द्ध इत्यु॑त्तर-अ॒र्द्धे । अ॒ग्नये᳚ । जु॒हो॒ति॒ । द॒क्षि॒णा॒र्द्ध इति॑ दक्षिण-अ॒र्द्धे । सोमा॑य । ए॒वम् । इ॒व॒ । हि । इ॒मौ । लो॒कौ । अ॒नयोः᳚ । लो॒कयोः᳚ । अनु॑ख्यात्या॒ इत्यनु॑ - ख्या॒त्यै॒ । राजा॑नौ । वै । ए॒तौ । दे॒वता॑नाम् । \textbf{  8} \newline
                  \newline
                                \textbf{ TS 2.6.2.2} \newline
                  यत् । अ॒ग्नीषोमा॒वित्य॒ग्नी-सोमौ᳚ । अ॒न्त॒रा । दे॒वताः᳚ । इ॒ज्ये॒ते॒ इति॑ । दे॒वता॑नाम् । विधृ॑त्या॒ इति॒ वि - धृ॒त्यै॒ । तस्मा᳚त् । राज्ञा᳚ । म॒नु॒ष्याः᳚ । विधृ॑ता॒ इति॒ वि - धृ॒ताः॒ । ब्र॒ह्म॒वा॒दिन॒ इति॑ ब्रह्म-वा॒दिनः॑ । व॒द॒न्ति॒ । किम् । तत् । य॒ज्ञे । यज॑मानः । कु॒रु॒ते॒ । येन॑ । अ॒न्यतो॑दत॒ इत्य॒न्यतः॑ - द॒तः॒ । च॒ । प॒शून् । दा॒धार॑ । उ॒भ॒यतो॑दत॒ इत्यु॑भ॒यतः॑-द॒तः॒ । च॒ । इति॑ । ऋच᳚म् । अ॒नूच्येत्य॑नु - उच्य॑ । आज्य॑भाग॒स्येत्याज्य॑ - भा॒ग॒स्य॒ । जु॒षा॒णेन॑ । य॒ज॒ति॒ । तेन॑ । अ॒न्यतो॑दत॒ इत्य॒न्यतः॑ - द॒तः॒ । दा॒धा॒र॒ । ऋच᳚म् । अ॒नूच्येत्य॑नु - उच्य॑ । ह॒विषः॑ । ऋ॒चा । य॒ज॒ति॒ । तेन॑ । उ॒भ॒यतो॑दत॒ इत्यु॑भ॒यतः॑ - द॒तः॒ । दा॒धा॒र॒ । मू॒र्द्ध॒न्वतीति॑ मूर्द्धन्न् - वती᳚ । पु॒रो॒नु॒वा॒क्येति॑ पुरः-अ॒नु॒वा॒क्या᳚ । भ॒व॒ति॒ । मू॒र्द्धान᳚म् । ए॒व । ए॒न॒म् । स॒मा॒नाना᳚म् । क॒रो॒ति॒ । \textbf{  9} \newline
                  \newline
                                \textbf{ TS 2.6.2.3} \newline
                  नि॒युत्व॒त्येति॑ नि - युत्व॑त्या । य॒ज॒ति॒ । भ्रातृ॑व्यस्य । ए॒व । प॒शून् । नीति॑ । यु॒व॒ते॒ । के॒शिन᳚म् । ह॒ । दा॒र्भ्यम् । के॒शी । सात्य॑कामि॒रिति॒ सात्य॑ - का॒मिः॒ । उ॒वा॒च॒ । स॒प्तप॑दा॒मिति॑ स॒प्त - प॒दा॒म् । ते॒ । शक्व॑रीम् । श्वः । य॒ज्ञे । प्र॒यो॒क्तास॒ इति॑ प्र-यो॒क्तासे᳚ । यस्यै᳚ । वी॒र्ये॑ण । प्रेति॑ । जा॒तान् । भ्रातृ॑व्यान् । नु॒दते᳚ । प्रतीति॑ । ज॒नि॒ष्यमा॑णान् । यस्यै᳚ । वी॒र्ये॑ण । उ॒भयोः᳚ । लो॒कयोः᳚ । ज्योतिः॑ । ध॒त्ते । यस्यै᳚ । वी॒र्ये॑ण । पू॒र्वा॒र्द्धेनेति॑ पूर्व - अ॒र्द्धेन॑ । अ॒न॒ड्वान् । भु॒नक्ति॑ । ज॒घ॒ना॒र्द्धेनेति॑ जघन - अ॒र्द्धेन॑ । धे॒नुः । इति॑ । पु॒रस्ता᳚ल्ल॒क्ष्मेति॑ पु॒रस्ता᳚त् - ल॒क्ष्मा॒ । पु॒रो॒नु॒वा॒क्येति॑ पुरः - अ॒नु॒वा॒क्या᳚ । भ॒व॒ति॒ । जा॒तान् । ए॒व । भ्रातृ॑व्यान् । प्रेति॑ । नु॒द॒ते॒ । उ॒परि॑ष्टाल्ल॒क्ष्मेत्यु॒परि॑ष्टात् - ल॒क्ष्मा॒ । \textbf{  10} \newline
                  \newline
                                \textbf{ TS 2.6.2.4} \newline
                  या॒ज्या᳚ । ज॒नि॒ष्यमा॑णान् । ए॒व । प्रतीति॑ । नु॒द॒ते॒ । पु॒रस्ता᳚ल्ल॒क्ष्मेति॑ पु॒रस्ता᳚त् - ल॒क्ष्मा॒ । पु॒रो॒नु॒वा॒क्येति॑ पुरः - अ॒नु॒वा॒क्या᳚ । भ॒व॒ति॒ । अ॒स्मिन्न् । ए॒व । लो॒के । ज्योतिः॑ । ध॒त्ते॒ । उ॒परि॑ष्टाल्ल॒क्ष्मेत्यु॒परि॑ष्टात् - ल॒क्ष्मा॒ । या॒ज्या᳚ । अ॒मुष्मिन्न्॑ । ए॒व । लो॒के । ज्योतिः॑ । ध॒त्ते॒ । ज्योति॑ष्मन्तौ । अ॒स्मै॒ । इ॒मौ । लो॒कौ । भ॒व॒तः॒ । यः । ए॒वम् । वेद॑ । पु॒रस्ता᳚ल्ल॒क्ष्मेति॑ पु॒रस्ता᳚त् - ल॒क्ष्मा॒ । पु॒रो॒नु॒वा॒क्येति॑ पुरः - अ॒नु॒वा॒क्या᳚ । भ॒व॒ति॒ । तस्मा᳚त् । पू॒र्वा॒र्द्धेनेति॑ पूर्व - अ॒र्द्धेन॑ । अ॒न॒ड्वान् । भु॒न॒क्ति॒ । उ॒परि॑ष्टाल्ल॒क्ष्मेत्यु॒परि॑ष्टात् - ल॒क्ष्मा॒ । या॒ज्या᳚ । तस्मा᳚त् । ज॒घ॒ना॒र्द्धेनेति॑ जघन - अ॒र्द्धेन॑ । धे॒नुः । यः । ए॒वम् । वेद॑ । भु॒ङ्क्तः । ए॒न॒म् । ए॒तौ । वज्रः॑ । आज्य᳚म् । वज्रः॑ । आज्य॑भागा॒वित्याज्य॑ - भा॒गौ॒ । \textbf{  11} \newline
                  \newline
                                \textbf{ TS 2.6.2.5} \newline
                  वज्रः॑ । व॒ष॒ट्का॒र इति॑ वषट् - का॒रः । त्रि॒वृत॒मिति॑ त्रि - वृत᳚म् । ए॒व । वज्र᳚म् । स॒भृंत्येति॑ सं - भृत्य॑ । भ्रातृ॑व्याय । प्रेति॑ । ह॒र॒ति॒ । अछ॑बंट्कार॒मित्यछ॑बंट् - का॒र॒म् । अ॒प॒गूर्येत्य॑प - गूर्य॑ । वष॑ट् । क॒रो॒ति॒ । स्तृत्यै᳚ । गा॒य॒त्री । पु॒रो॒नु॒वा॒क्येति॑ पुरः - अ॒नु॒वा॒क्या᳚ । भ॒व॒ति॒ । त्रि॒ष्टुक् । या॒ज्या᳚ । ब्रह्मन्न्॑ । ए॒व । क्ष॒त्रम् । अ॒न्वार॑भंय॒तीत्य॑नु - आर॑भंयति । तस्मा᳚त् । ब्रा॒ह्म॒णः । मुख्यः॑ । मुख्यः॑ । भ॒व॒ति॒ । यः । ए॒वम् । वेद॑ । प्रेति॑ । ए॒व । ए॒न॒म् । पु॒रो॒नु॒वा॒क्य॑येति॑ पुरः - अ॒नु॒वा॒क्य॑या । आ॒ह॒ । प्रेति॑ । न॒य॒ति॒ । या॒ज्य॑या । ग॒मय॑ति । व॒ष॒ट्का॒रेणेति॑ वषट् - का॒रेण॑ । एति॑ । ए॒व । ए॒न॒म् । पु॒रो॒नु॒वा॒क्य॑येति॑ पुरः-अ॒नु॒वा॒क्य॑या । द॒त्ते॒ । प्रेति॑ । य॒च्छ॒ति॒ । या॒ज्य॑या । प्रतीति॑ । \textbf{  12} \newline
                  \newline
                                \textbf{ TS 2.6.2.6} \newline
                  व॒ष॒ट्का॒रेणेति॑ वषट् - का॒रेण॑ । स्था॒प॒य॒ति॒ । त्रि॒पदेति॑ त्रि - पदा᳚ । पु॒रो॒नु॒वा॒क्येति॑ पुरः - अ॒नु॒वा॒क्या᳚ । भ॒व॒ति॒ । त्रयः॑ । इ॒मे । लो॒काः । ए॒षु । ए॒व । लो॒केषु॑ । प्रतीति॑ । ति॒ष्ठ॒ति॒ । चतु॑ष्प॒देति॒ चतुः॑ - प॒दा॒ । या॒ज्या᳚ । चतु॑ष्पद॒ इति॒ चतुः॑-प॒दः॒ । ए॒व । प॒शून् । अवेति॑ । रु॒न्धे॒ । द्व्य॒क्ष॒र इति॑ द्वि - अ॒क्ष॒रः॒ । व॒ष॒ट्का॒र इति॑ वषट् - का॒रः । द्वि॒पादिति॑ द्वि-पात् । यज॑मानः । प॒शुषु॑ । ए॒व । उ॒परि॑ष्टात् । प्रतीति॑ । ति॒ष्ठ॒ति॒ । गा॒य॒त्री । पु॒रो॒नु॒वा॒क्येति॑ पुरः - अ॒नु॒वा॒क्या᳚ । भ॒व॒ति॒ । त्रि॒ष्टुक् । या॒ज्या᳚ । ए॒षा । वै । स॒प्तप॒देति॑ स॒प्त - प॒दा॒ । शक्व॑री । यत् । वै । ए॒तया᳚ । दे॒वाः । अशि॑क्षन्न् । तत् । अ॒श॒क्नु॒व॒न्न् । यः । ए॒वम् । वेद॑ । श॒क्नोति॑ । ए॒व ( ) । यत् । शिक्ष॑ति ॥ \textbf{  13} \newline
                  \newline
                      (दे॒वता॑नां - करोत्यु॒ - परि॑ष्टाल्ल॒क्ष्मा - ऽऽज्य॑भागौ॒ - प्रति॑ - श॒क्रोत्ये॒व - द्वे च॑ )  \textbf{(A2)} \newline \newline
                                \textbf{ TS 2.6.3.1} \newline
                  प्र॒जाप॑ति॒रिति॑ प्र॒जा - प॒तिः॒ । दे॒वेभ्यः॑ । य॒ज्ञान् । व्यादि॑श॒दिति॑ वि - आदि॑शत् । सः । आ॒त्मन्न् । आज्य᳚म् । अ॒ध॒त्त॒ । तम् । दे॒वाः । अ॒ब्रु॒व॒न्न् । ए॒षः । वाव । य॒ज्ञ्ः । यत् । आज्य᳚म् । अपीति॑ । ए॒व । नः॒ । अत्र॑ । अ॒स्तु॒ । इति॑ । सः । अ॒ब्र॒वी॒त् । यजान्॑ । वः॒ । आज्य॑भागा॒वित्याज्य॑ - भा॒गौ॒ । उपेति॑ । स्तृ॒णा॒न् । अ॒भीति॑ । घा॒र॒या॒न् । इति॑ । तस्मा᳚त् । यज॑न्ति । आज्य॑भागा॒वित्याज्य॑ - भा॒गौ॒ । उपेति॑ । स्तृ॒ण॒न्ति॒ । अ॒भीति॑ । घा॒र॒य॒न्ति॒ । ब्र॒ह्म॒वा॒दिन॒ इति॑ ब्रह्म - वा॒दिनः॑ । व॒द॒न्ति॒ । कस्मा᳚त् । स॒त्यात् । या॒तया॑मा॒नीति॑ या॒त - या॒मा॒नि॒ । अ॒न्यानि॑ । ह॒वीꣳषि॑ । अया॑तयाम॒मित्यया॑त-या॒म॒म् । आज्य᳚म् । इति॑ । प्रा॒जा॒प॒त्यमिति॑ प्राजा - प॒त्यम् । \textbf{  14} \newline
                  \newline
                                \textbf{ TS 2.6.3.2} \newline
                  इति॑ । ब्रू॒या॒त् । अया॑तया॒मेत्यया॑त - या॒मा॒ । हि । दे॒वाना᳚म् । प्र॒जाप॑ति॒रिति॑ प्र॒जा - प॒तिः॒ । इति॑ । छन्दाꣳ॑सि । दे॒वेभ्यः॑ । अपेति॑ । अ॒क्रा॒म॒न्न् । न । वः॒ । अ॒भा॒गानि॑ । ह॒व्यम् । व॒क्ष्या॒मः॒ । इति॑ । तेभ्यः॑ । ए॒तत् । च॒तु॒र॒व॒त्तमिति॑ चतुः - अ॒व॒त्तम् । अ॒धा॒र॒य॒न्न् । पु॒रो॒नु॒वा॒क्या॑या॒ इति॑ पुरः - अ॒नु॒वा॒क्या॑यै । या॒ज्या॑यै । दे॒वता॑यै । व॒ष॒ट्का॒रायेति॑ वषट् - का॒राय॑ । यत् । च॒तु॒र॒व॒त्तमिति॑ चतुः - अ॒व॒त्तम् । जु॒होति॑ । छन्दाꣳ॑सि । ए॒व । तत् । प्री॒णा॒ति॒ । तानि॑ । अ॒स्य॒ । प्री॒तानि॑ । दे॒वेभ्यः॑ । ह॒व्यम् । व॒ह॒न्ति॒ । अङ्गि॑रसः । वै । इ॒तः । उ॒त्त॒मा इत्यु॑त् - त॒माः । सु॒व॒र्गमिति॑ सुवः - गम् । लो॒कम् । आ॒य॒न्न् । तत् । ऋष॑यः । य॒ज्ञ्॒वा॒स्त्विति॑ यज्ञ् - वा॒स्तु । अ॒भ्य॒वाय॒न्नित्य॑भि - अ॒वायन्न्॑ । ते । \textbf{  15} \newline
                  \newline
                                \textbf{ TS 2.6.3.3} \newline
                  अ॒प॒श्य॒न्न् । पु॒रो॒डाश᳚म् । कू॒र्मम् । भू॒तम् । सर्प॑न्तम् । तम् । अ॒ब्रु॒व॒न्न् । इन्द्रा॑य । ध्रि॒य॒स्व॒ । बृह॒स्पत॑ये । ध्रि॒य॒स्व॒ । विश्वे᳚भ्यः । दे॒वेभ्यः॑ । ध्रि॒य॒स्व॒ । इति॑ । सः । न । अ॒द्ध्रि॒य॒त॒ । तम् । अ॒ब्रु॒व॒न्न् । अ॒ग्नये᳚ । ध्रि॒य॒स्व॒ । इति॑ । सः । अ॒ग्नये᳚ । अ॒द्ध्रि॒य॒त॒ । यत् । आ॒ग्ने॒यः । अ॒ष्टाक॑पाल॒ इत्य॒ष्टा - क॒पा॒लः॒ । अ॒मा॒वा॒स्या॑या॒मित्य॑मा - वा॒स्या॑याम् । च॒ । पौ॒र्ण॒मा॒स्यामिति॑ पौर्ण - मा॒स्याम् । च॒ । अ॒च्यु॒तः । भव॑ति । सु॒व॒र्गस्येति॑ सुवः - गस्य॑ । लो॒कस्य॑ । अ॒भिजि॑त्या॒ इत्य॒भि - जि॒त्यै॒ । तम् । अ॒ब्रु॒व॒न्न् । क॒था । अ॒हा॒स्थाः॒ । इति॑ । अनु॑पाक्त॒ इत्यनु॑प - अ॒क्तः॒ । अ॒भू॒व॒म् । इति॑ । अ॒ब्र॒वी॒त् । यथा᳚ । अक्षः॑ । अनु॑पाक्त॒ इत्यनु॑प - अ॒क्तः॒ । \textbf{  16} \newline
                  \newline
                                \textbf{ TS 2.6.3.4} \newline
                  अ॒वार्च्छ॒तीत्य॑व - ऋच्छ॑ति । ए॒वम् । अवेति॑ । आ॒र॒म् । इति॑ । उ॒परि॑ष्टात् । अ॒भ्यज्येत्य॑भि - अज्य॑ । अ॒धस्ता᳚त् । उपेति॑ । अ॒न॒क्ति॒ । सु॒व॒र्गस्येति॑ सुवः - गस्य॑ । लो॒कस्य॑ । सम॑ष्ट्या॒ इति॒ सं-अ॒ष्ट्यै॒ । सर्वा॑णि । क॒पाला॑नि । अ॒भीति॑ । प्र॒थ॒य॒ति॒ । ताव॑तः । पु॒रो॒डाशान्॑ । अ॒मुष्मिन्न्॑ । लो॒के । अ॒भीति॑ । ज॒य॒ति॒ । यः । विद॑ग्ध॒ इति॒ वि - द॒ग्धः॒ । सः । नै॒र्॒.ऋ॒त इति॑ नैः-ऋ॒तः । यः । अशृ॑तः । सः । रौ॒द्रः । यः । शृ॒तः । सः । सदे॑व॒ इति॒ स - दे॒वः॒ । तस्मा᳚त् । अवि॑दह॒तेत्यवि॑ - द॒ह॒ता॒ । शृ॒त॒कृंत्य॒ इति॑ शृतं - कृत्यः॑ । स॒दे॒व॒त्वायेति॑ सदेव - त्वाय॑ । भस्म॑ना । अ॒भीति॑ । वा॒स॒य॒ति॒ । तस्मा᳚त् । माꣳ॒॒सेन॑ । अस्थि॑ । छ॒न्नम् । वे॒देन॑ । अ॒भीति॑ । वा॒स॒य॒ति॒ । तस्मा᳚त् । \textbf{  17} \newline
                  \newline
                                \textbf{ TS 2.6.3.5} \newline
                  केशैः᳚ । शिरः॑ । छ॒न्नम् । प्रच्यु॑त॒मिति॒ प्र - च्यु॒त॒म् । वै । ए॒तत् । अ॒स्मात् । लो॒कात् । अग॑तम् । दे॒व॒लो॒कमिति॑ देव - लो॒कम् । यत् । शृ॒तम् । ह॒विः । अन॑भिघारित॒मित्यन॑भि - घा॒रि॒त॒म् । अ॒भि॒घार्येत्य॑भि - घार्य॑ । उदिति॑ । वा॒स॒य॒ति॒ । दे॒व॒त्रेति॑ देव - त्रा । ए॒व । ए॒न॒त् । ग॒म॒य॒ति॒ । यदि॑ । एक᳚म् । क॒पाल᳚म् । नश्ये᳚त् । एकः॑ । मासः॑ । सं॒ॅव॒थ्स॒रस्येति॑ सं - व॒थ्स॒रस्य॑ । अन॑वेत॒ इत्यन॑व - इ॒तः॒ । स्यात् । अथ॑ । यज॑मानः । प्रेति॑ । मी॒ये॒त॒ । यत् । द्वे इति॑ । नश्ये॑ताम् । द्वौ । मासौ᳚ । सं॒ॅव॒थ्स॒रस्येति॑ सं - व॒थ्स॒रस्य॑ । अन॑वेता॒वित्यन॑व - इ॒तौ॒ । स्याता᳚म् । अथ॑ । यज॑मानः । प्रेति॑ । मी॒ये॒त॒ । स॒ख्यांयेति॑ सं - ख्याय॑ । उदिति॑ । वा॒स॒य॒ति॒ । यज॑मानस्य । \textbf{  18} \newline
                  \newline
                                \textbf{ TS 2.6.3.6} \newline
                  गो॒पी॒थाय॑ । यदि॑ । नश्ये᳚त् । आ॒श्वि॒नम् । द्वि॒क॒पा॒लमिति॑ द्वि - क॒पा॒लम् । निरिति॑ । व॒पे॒त् । द्या॒वा॒पृ॒थि॒व्य॑मिति॑ द्यावा - पृ॒थि॒व्य᳚म् । एक॑कपाल॒मित्येक॑-क॒पा॒ल॒म् । अ॒श्विनौ᳚ । वै । दे॒वाना᳚म् । भि॒षजौ᳚ । ताभ्या᳚म् । ए॒व । अ॒स्मै॒ । भे॒ष॒जम् । क॒रो॒ति॒ । द्या॒वा॒पृ॒थि॒व्य॑ इति॑ द्यावा - पृ॒थि॒व्यः॑ । एक॑कपाल॒ इत्येक॑-क॒पा॒लः॒ । भ॒व॒ति॒ । अ॒नयोः᳚ । वै । ए॒तत् । न॒श्य॒ति॒ । यत् । नश्य॑ति । अ॒नयोः᳚ । ए॒व । ए॒न॒त् । वि॒न्द॒ति॒ । प्रति॑ष्ठित्या॒ इति॒ प्रति॑ - स्थि॒त्यै॒ ॥ \textbf{  19} \newline
                  \newline
                      (प्र॒जा॒प॒त्यं - ते - ऽक्षोऽनु॑पाक्तो - वे॒देना॒भि वा॑सयति॒ तस्मा॒द् - यज॑मानस्य॒ - द्वात्रिꣳ॑शच्च)  \textbf{(A3)} \newline \newline
                                \textbf{ TS 2.6.4.1} \newline
                  दे॒वस्य॑ । त्वा॒ । स॒वि॒तुः । प्र॒स॒व इति॑ प्र - स॒वे । इति॑ । स्फ्यम् । एति॑ । द॒त्ते॒ । प्रसू᳚त्या॒ इति॒ प्र - सू॒त्यै॒ । अ॒श्विनोः᳚ । बा॒हुभ्या॒मिति॑ बा॒हु-भ्या॒म् । इति॑ । आ॒ह॒ । अ॒श्विनौ᳚ । हि । दे॒वाना᳚म् । अ॒द्ध्व॒र्यू इति॑ । आस्ता᳚म् । पू॒ष्णः । हस्ता᳚भ्याम् । इति॑ । आ॒ह॒ । यत्यै᳚ । श॒तभृ॑ष्टि॒रिति॑ श॒त - भृ॒ष्टिः॒ । अ॒सि॒ । वा॒न॒स्प॒त्यः । द्वि॒ष॒तः । व॒धः । इति॑ । आ॒ह॒ । वज्र᳚म् । ए॒व । तत् । समिति॑ । श्य॒ति॒ । भ्रातृ॑व्याय । प्र॒ह॒रि॒ष्यन्निति॑ प्र - ह॒रि॒ष्यन्न् । स्त॒बं॒य॒जुरिति॑ स्तंब - य॒जुः । ह॒र॒ति॒ । ए॒ताव॑ती । वै । पृ॒थि॒वी । याव॑ती । वेदिः॑ । तस्याः᳚ । ए॒ताव॑तः । ए॒व । भ्रातृ॑व्यम् । निरिति॑ । भ॒ज॒ति॒ । \textbf{  20} \newline
                  \newline
                                \textbf{ TS 2.6.4.2} \newline
                  तस्मा᳚त् । न । अ॒भा॒गम् । निरिति॑ । भ॒ज॒न्ति॒ । त्रिः । ह॒र॒ति॒ । त्रयः॑ । इ॒मे । लो॒काः । ए॒भ्यः । ए॒व । ए॒न॒म् । लो॒केभ्यः॑ । निरिति॑ । भ॒ज॒ति॒ । तू॒ष्णीम् । च॒तु॒र्थम् । ह॒र॒ति॒ । अप॑रिमिता॒दित्यप॑रि-मि॒ता॒त् । ए॒व । ए॒न॒म् । निरिति॑ । भ॒ज॒ति॒ । उदिति॑ । ह॒न्ति॒ । यत् । ए॒व । अ॒स्याः॒ । अ॒मे॒द्ध्यम् । तत् । अपेति॑ । ह॒न्ति॒ । उदिति॑ । ह॒न्ति॒ । तस्मा᳚त् । ओष॑धयः । परेति॑ । भ॒व॒न्ति॒ । मूल᳚म् । छि॒न॒त्ति॒ । भ्रातृ॑व्यस्य । ए॒व । मूल᳚म् । छि॒न॒त्ति॒ । पि॒तृ॒दे॒व॒त्येति॑ पितृ - दे॒व॒त्या᳚ । अति॑खा॒तेत्यति॑ - खा॒ता॒ । इय॑तीम् । ख॒न॒ति॒ । प्र॒जाप॑ति॒नेति॑ प्र॒जा - प॒ति॒ना॒ । \textbf{  21} \newline
                  \newline
                                \textbf{ TS 2.6.4.3} \newline
                  य॒ज्ञ्॒मु॒खेनेति॑ यज्ञ् - मु॒खेन॑ । संमि॑ता॒मिति॒ सं - मि॒ता॒म् । एति॑ । प्र॒ति॒ष्ठाया॒ इति॑ प्रति - स्थायै᳚ । ख॒न॒ति॒ । यज॑मानम् । ए॒व । प्र॒ति॒ष्ठामिति॑ प्रति - स्थाम् । ग॒म॒य॒ति॒ । द॒क्षि॒ण॒तः । वर्.षी॑यसीम् । क॒रो॒ति॒ । दे॒व॒यज॑न॒स्येति॑ देव - यज॑नस्य । ए॒व । रू॒पम् । अ॒कः॒ । पुरी॑षवती॒मिति॒ पुरी॑ष - व॒ती॒म् । क॒रो॒ति॒ । प्र॒जेति॑ प्र - जा । वै । प॒शवः॑ । पुरी॑षम् । प्र॒जयेति॑ प्र - जया᳚ । ए॒व । ए॒न॒म् । प॒शुभि॒रिति॑ प॒शु - भिः॒ । पुरी॑षवन्त॒मिति॒ पुरी॑ष - व॒न्त॒म् । क॒रो॒ति॒ । उत्त॑र॒मित्युत् - त॒र॒म् । प॒रि॒ग्रा॒हमिति॑ परि-ग्रा॒हम् । परीति॑ । गृ॒ह्णा॒ति॒ । ए॒ताव॑ती । वै । पृ॒थि॒वी । याव॑ती । वेदिः॑ । तस्याः᳚ । ए॒ताव॑तः । ए॒व । भ्रातृ॑व्यम् । नि॒र्भज्येति॑ निः - भज्य॑ । आ॒त्मने᳚ । उत्त॑र॒मित्युत् - त॒र॒म् । प॒रि॒ग्रा॒हमिति॑ परि-ग्रा॒हम् । परीति॑ । गृ॒ह्णा॒ति॒ । क्रू॒रम् । इ॒व॒ । वै । \textbf{  22} \newline
                  \newline
                                \textbf{ TS 2.6.4.4} \newline
                  ए॒तत् । क॒रो॒ति॒ । यत् । वेदि᳚म् । क॒रोति॑ । धाः । अ॒सि॒ । स्व॒धेति॑ स्व - धा । अ॒सि॒ । इति॑ । यो॒यु॒प्य॒ते॒ । शान्त्यै᳚ । प्रोक्ष॑णी॒रिति॑ प्र - उक्ष॑णीः । एति॑ । सा॒द॒य॒ति॒ । आपः॑ । वै । र॒क्षो॒घ्नीरिति॑ रक्षः - घ्नीः । रक्ष॑साम् । अप॑हत्या॒ इत्यप॑ - ह॒त्यै॒ । स्फ्यस्य॑ । वर्त्मन्न्॑ । सा॒द॒य॒ति॒ । य॒ज्ञ्स्य॑ । संत॑त्या॒ इति॒ सं - त॒त्यै॒ । यम् । द्वि॒ष्यात् । तम् । ध्या॒ये॒त् । शु॒चा । ए॒व । ए॒न॒म् । अ॒र्प॒य॒ति॒ ॥ \textbf{  23} \newline
                  \newline
                      (भ॒ज॒ति॒ - प्र॒जाप॑तिने- व॒ वै - त्रय॑स्त्रिꣳशच्च)  \textbf{(A4)} \newline \newline
                                \textbf{ TS 2.6.5.1} \newline
                  ब्र॒ह्म॒वा॒दिन॒ इति॑ ब्रह्म - वा॒दिनः॑ । व॒द॒न्ति॒ । अ॒द्भिरित्य॑त् - भिः । ह॒वीꣳषि॑ । प्रेति॑ । औ॒क्षीः॒ । केन॑ । अ॒पः । इति॑ । ब्रह्म॑णा । इति॑ । ब्रू॒या॒त् । अ॒द्भिरित्य॑त् - भिः । हि । ए॒व । ह॒वीꣳषि॑ । प्रो॒क्षतीति॑ प्र - उ॒क्षति॑ । ब्रह्म॑णा । अ॒पः । इ॒द्ध्माब॒र्॒.हिरिती॒द्ध्मा - ब॒र्॒.हिः । प्रेति॑ । उ॒क्ष॒ति॒ । मेद्ध्य᳚म् । ए॒व । ए॒न॒त् । क॒रो॒ति॒ । वेदि᳚म् । प्रेति॑ । उ॒क्ष॒ति॒ । ऋ॒क्षा । वै । ए॒षा । अ॒लो॒मका᳚ । अ॒मे॒द्ध्या । यत् । वेदिः॑ । मेद्ध्या᳚म् । ए॒व । ए॒ना॒म् । क॒रो॒ति॒ । दि॒वे । त्वा॒ । अ॒न्तरि॑क्षाय । त्वा॒ । पृ॒थि॒व्यै । त्वा॒ । इति॑ । ब॒र्॒.हिः । आ॒साद्येत्या᳚ - साद्य॑ । प्रेति॑ । \textbf{  24} \newline
                  \newline
                                \textbf{ TS 2.6.5.2} \newline
                  उ॒क्ष॒ति॒ । ए॒भ्यः । ए॒व । ए॒न॒त् । लो॒केभ्यः॑ । प्रेति॑ । उ॒क्ष॒ति॒ । क्रू॒रम् । इ॒व॒ । वै । ए॒तत् । क॒रो॒ति॒ । यत् । खन॑ति । अ॒पः । नीति॑ । न॒य॒ति॒ । शान्त्यै᳚ । पु॒रस्ता᳚त् । प्र॒स्त॒रमिति॑ प्र - स्त॒रम् । गृ॒ह्णा॒ति॒ । मुख्य᳚म् । ए॒व । ए॒न॒म् । क॒रो॒ति॒ । इय॑न्तम् । गृ॒ह्णा॒ति॒ । प्र॒जाप॑ति॒नेति॑ प्र॒जा - प॒ति॒ना॒ । य॒ज्ञ्॒मु॒खेनेति॑ यज्ञ् - मु॒खेन॑ । संमि॑त॒मिति॒ सं - मि॒त॒म् । ब॒र्॒.हिः । स्तृ॒णा॒ति॒ । प्र॒जा इति॑ प्र - जाः । वै । ब॒र्॒.हिः । पृ॒थि॒वी । वेदिः॑ । प्र॒जा इति॑ प्र - जाः । ए॒व । पृ॒थि॒व्याम् । प्रतीति॑ । स्थ॒प॒य॒ति॒ । अन॑तिदृश्न॒मित्यन॑ति - दृ॒श्न॒म् । स्तृ॒णा॒ति॒ । प्र॒जयेति॑ प्र - जया᳚ । ए॒व । ए॒न॒म् । प॒शुभि॒रिति॑ प॒शु - भिः॒ । अन॑तिदृश्न॒मित्यन॑ति - दृ॒श्न॒म् । क॒रो॒ति॒ । \textbf{  25} \newline
                  \newline
                                \textbf{ TS 2.6.5.3} \newline
                  उत्त॑र॒मित्युत् - त॒र॒म् । ब॒र॒.हिषः॑ । प्र॒स्त॒रमिति॑ प्र - स्त॒रम् । सा॒द॒य॒ति॒ । प्र॒जा इति॑ प्र - जाः । वै । ब॒र्॒.हिः । यज॑मानः । प्र॒स्त॒र इति॑ प्र - स्त॒रः । यज॑मानम् । ए॒व । अय॑जमानात् । उत्त॑र॒मित्युत् - त॒र॒म् । क॒रो॒ति॒ । तस्मा᳚त् । यज॑मानः । अय॑जमानात् । उत्त॑र॒ इत्युत् - त॒रः॒ । अ॒न्तः । द॒धा॒ति॒ । व्यावृ॑त्त्या॒ इति॑ वि-आवृ॑त्त्यै । अ॒नक्ति॑ । ह॒विष्कृ॑त॒मिति॑ ह॒विः - कृ॒त॒म् । ए॒व । ए॒न॒म् । सु॒व॒र्गमिति॑ सुवः - गम् । लो॒कम् । ग॒म॒य॒ति॒ । त्रे॒धा । अ॒न॒क्ति॒ । त्रयः॑ । इ॒मे । लो॒काः । ए॒भ्यः । ए॒व । ए॒न॒म् । लो॒केभ्यः॑ । अ॒न॒क्ति॒ । न । प्रतीति॑ । शृ॒णा॒ति॒ । यत् । प्र॒ति॒शृ॒णी॒यादिति॑ प्रति - शृ॒णी॒यात् । अनू᳚र्द्ध्वं भावुक॒मित्यनू᳚र्द्ध्वं-भा॒वु॒क॒म् । यज॑मानस्य । स्या॒त् । उ॒परि॑ । इ॒व॒ । प्रेति॑ । ह॒र॒ति॒ । \textbf{  26} \newline
                  \newline
                                \textbf{ TS 2.6.5.4} \newline
                  उ॒परि॑ । इ॒व॒ । हि । सु॒व॒र्ग इति॑ सुवः - गः । लो॒कः । नीति॑ । य॒च्छ॒ति॒ । वृष्टि᳚म् । ए॒व । अ॒स्मै॒ । नीति॑ । य॒च्छ॒ति॒ । न । अत्य॑ग्र॒मित्यति॑ - अ॒ग्र॒म् । प्रेति॑ । ह॒रे॒त् । यत् । अत्य॑ग्र॒मित्यति॑ - अ॒ग्र॒म् । प्र॒हरे॒दिति॑ प्र - हरे᳚त् । अ॒त्या॒सा॒रिणीत्य॑ति-आ॒सा॒रिणी᳚ । अ॒द्ध्व॒र्योः । नाशु॑का । स्या॒त् । न । पु॒रस्ता᳚त् । प्रतीति॑ । अ॒स्ये॒त् । यत् । पु॒रस्ता᳚त् । प्र॒त्यस्ये॒दिति॑ प्रति - अस्ये᳚त् । सु॒व॒र्गादिति॑ सुवः - गात् । लो॒कात् । यज॑मानम् । प्रतीति॑ । नु॒दे॒त् । प्राञ्च᳚म् । प्रेति॑ । ह॒र॒ति॒ । यज॑मानम् । ए॒व । सु॒व॒र्गमिति॑ सुवः - गम् । लो॒कम् । ग॒म॒य॒ति॒ । न । विष्व॑ञ्चम् । वीति॑ । यु॒या॒त् । यत् । विष्व॑ञ्चम् । वि॒यु॒यादिति॑ वि-यु॒यात् । \textbf{  27} \newline
                  \newline
                                \textbf{ TS 2.6.5.5} \newline
                  स्त्री । अ॒स्य॒ । जा॒ये॒त॒ । ऊ॒र्द्ध्वम् । उदिति॑ । यौ॒ति॒ । ऊ॒र्द्ध्वम् । इ॒व॒ । हि । पुꣳ॒॒सः । पुमान्॑ । ए॒व । अ॒स्य॒ । जा॒य॒ते॒ । यत् । स्फ्येन॑ । वा॒ । उ॒प॒वे॒षेणेत्यु॑प-वे॒षेण॑ । वा॒ । यो॒यु॒प्येत॑ । स्तृतिः॑ । ए॒व । अ॒स्य॒ । सा । हस्ते॑न । यो॒यु॒प्य॒ते॒ । यज॑मानस्य । गो॒पी॒थाय॑ । ब्र॒ह्म॒वा॒दिन॒ इति॑ ब्रह्म - वा॒दिनः॑ । व॒द॒न्ति॒ । किम् । य॒ज्ञ्स्य॑ । यज॑मानः । इति॑ । प्र॒स्त॒र इति॑ प्र-स्त॒रः । इति॑ । तस्य॑ । क्व॑ । सु॒व॒र्ग इति॑ सुवः- गः । लो॒कः । इति॑ । आ॒ह॒व॒नीय॒ इत्या᳚ - ह॒व॒नीयः॑ । इति॑ । ब्रू॒या॒त् । यत् । प्र॒स्त॒रमिति॑ प्र - स्त॒रम् । आ॒ह॒व॒नीय॒ इत्या᳚ - ह॒व॒नीये᳚ । प्र॒हर॒तीति॑ प्र - हर॑ति । यज॑मानम् । ए॒व । \textbf{  28} \newline
                  \newline
                                \textbf{ TS 2.6.5.6} \newline
                  सु॒व॒र्गमिति॑ सुवः - गम् । लो॒कम् । ग॒म॒य॒ति॒ । वीति॑ । वै । ए॒तत् । यज॑मानः । लि॒श॒ते॒ । यत् । प्र॒स्त॒रमिति॑ प्र - स्त॒रम् । यो॒यु॒प्यन्ते᳚ । ब॒र्॒.हिः । अनु॑ । प्रेति॑ । ह॒र॒ति॒ । शान्त्यै᳚ । अ॒ना॒र॒भं॒ण इत्य॑ना - र॒भं॒णः । इ॒व॒ । वै । ए॒तर्.हि॑ । अ॒द्ध्व॒र्युः । सः । ई॒श्व॒रः । वे॒प॒नः । भवि॑तोः । ध्रु॒वा । अ॒सि॒ । इति॑ । इ॒माम् । अ॒भीति॑ । मृ॒श॒ति॒ । इ॒यम् । वै । ध्रु॒वा । अ॒स्याम् । ए॒व । प्रतीति॑ । ति॒ष्ठ॒ति॒ । न । वे॒प॒नः । भ॒व॒ति॒ । अगा(3)न् । अ॒ग्नी॒दित्य॑ग्नि - इ॒त् । इति॑ । आ॒ह॒ । यत् । ब्रू॒या॒त् । अगन्न्॑ । अ॒ग्निः । इति॑ ( ) । अ॒ग्नौ । अ॒ग्निम् । ग॒म॒ये॒त् । निरिति॑ । यज॑मानम् । सु॒व॒र्गादिति॑ सुवः-गात् । लो॒कात् । भ॒जे॒त् । अगन्न्॑ । इति॑ । ए॒व । ब्रू॒या॒त् । यज॑मानम् । ए॒व । सु॒व॒र्गमिति॑ सुवः - गम् । लो॒कम् । ग॒म॒य॒ति॒ ॥ \textbf{  29} \newline
                  \newline
                      (आ॒साद्य॒ प्रा - न॑तिदृश्नं करोति - हरति - वियु॒याद् - यज॑मानमे॒वा-ऽग्निरिति॑ - स॒प्तद॑श च )  \textbf{(A5)} \newline \newline
                                \textbf{ TS 2.6.6.1} \newline
                  अ॒ग्नेः । त्रयः॑ । ज्यायाꣳ॑सः । भ्रात॑रः । आ॒स॒न्न् । ते । दे॒वेभ्यः॑ । ह॒व्यम् । वह॑न्तः । प्रेति॑ । अ॒मी॒य॒न्त॒ । सः । अ॒ग्निः । अ॒बि॒भे॒त् । इ॒त्थम् । वाव । स्यः । आर्ति᳚म् । एति॑ । अ॒रि॒ष्य॒ति॒ । इति॑ । सः । निला॑यत । सः । अ॒पः । प्रेति॑ । अ॒वि॒श॒त् । तम् । दे॒वताः᳚ । प्रैष॒मिति॑ प्र - एष᳚म् । ऐ॒च्छ॒न्न् । तम् । मथ्स्यः॑ । प्रेति॑ । अ॒ब्र॒वी॒त् । तम् । अ॒श॒प॒त् । धि॒याधि॒येति॑ धि॒या-धि॒या॒ । त्वा॒ । व॒द्ध्या॒सुः॒ । यः । मा॒ । प्रेति॑ । अवो॑चः । इति॑ । तस्मा᳚त् । मथ्स्य᳚म् । धि॒याधि॒येति॑ धि॒या - धि॒या॒ । घ्न॒न्ति॒ । श॒प्तः । \textbf{  30} \newline
                  \newline
                                \textbf{ TS 2.6.6.2} \newline
                  हि । तम् । अन्विति॑ । अ॒वि॒न्द॒न्न् । तम् । अ॒ब्रु॒व॒न्न् । उपेति॑ । नः॒ । एति॑ । व॒र्त॒स्व॒ । ह॒व्यम् । नः॒ । व॒ह॒ । इति॑ । सः । अ॒ब्र॒वी॒त् । वर᳚म् । वृ॒णै॒ । यत् । ए॒व । गृ॒ही॒तस्य॑ । अहु॑तस्य । ब॒हिः॒ प॒रि॒धीति॑ बहिः - प॒रि॒धि । स्कन्दा᳚त् । तत् । मे॒ । भ्रातृ॑णाम् । भा॒ग॒धेय॒मिति॑ भाग - धेय᳚म् । अ॒स॒त् । इति॑ । तस्मा᳚त् । यत् । गृ॒ही॒तस्य॑ । अहु॑तस्य । ब॒हिः॒ प॒रि॒धीति॑ बहिः - प॒रि॒धि । स्कन्द॑ति । तेषा᳚म् । तत् । भा॒ग॒धेय॒मिति॑ भाग - धेय᳚म् । तान् । ए॒व । तेन॑ । प्री॒णा॒ति॒ । प॒रि॒धीनिति॑ परि - धीन् । परीति॑ । द॒धा॒ति॒ । रक्ष॑साम् । अप॑हत्या॒ इत्यप॑ - ह॒त्यै॒ । समिति॑ । स्प॒र्॒.श॒य॒ति॒ । \textbf{  31} \newline
                  \newline
                                \textbf{ TS 2.6.6.3} \newline
                  रक्ष॑साम् । अन॑न्ववचारा॒येत्यन॑नु - अ॒व॒चा॒रा॒य॒ । न । पु॒रस्ता᳚त् । परीति॑ । द॒धा॒ति॒ । आ॒दि॒त्यः । हि । ए॒व । उ॒द्यन्नित्यु॑त् - यन्न् । पु॒रस्ता᳚त् । रक्षाꣳ॑सि । अ॒प॒हन्तीत्य॑प - हन्ति॑ । ऊ॒र्द्ध्वे इति॑ । स॒मिधा॒विति॑ सं - इधौ᳚ । एति॑ । द॒धा॒ति॒ । उ॒परि॑ष्टात् । ए॒व । रक्षाꣳ॑सि । अपेति॑ । ह॒न्ति॒ । यजु॑षा । अ॒न्याम् । तू॒ष्णीम् । अ॒न्याम् । मि॒थु॒न॒त्वायेति॑ मिथुन - त्वाय॑ । द्वे इति॑ । एति॑ । द॒धा॒ति॒ । द्वि॒पादिति॑ द्वि - पात् । यज॑मानः । प्रति॑ष्ठित्या॒ इति॒ प्रति॑ - स्थि॒त्यै॒ । ब्र॒ह्म॒वा॒दिन॒ इति॑ ब्रह्म - वा॒दिनः॑ । व॒द॒न्ति॒ । सः । तु । वै । य॒जे॒त॒ । यः । य॒ज्ञ्स्य॑ । आर्त्या᳚ । वसी॑यान् । स्यात् । इति॑ । भूप॑तय॒ इति॒ भू - प॒त॒ये॒ । स्वाहा᳚ । भुव॑नपतय॒ इति॒ भुव॑न - प॒त॒ये॒ । स्वाहा᳚ । भू॒ताना᳚म् । \textbf{  32} \newline
                  \newline
                                \textbf{ TS 2.6.6.4} \newline
                  पत॑ये । स्वाहा᳚ । इति॑ । स्क॒न्नम् । अन्विति॑ ।   म॒न्त्र॒ये॒त॒ । य॒ज्ञ्स्य॑ । ए॒व । तत् । आर्त्या᳚ । यज॑मानः । वसी॑यान् । भ॒व॒ति॒ । भूय॑सीः । हि । दे॒वताः᳚ । प्री॒णाति॑ । जा॒मि । वै । ए॒तत् । य॒ज्ञ्स्य॑ । क्रि॒य॒ते॒ । यत् । अ॒न्वञ्चौ᳚ । पु॒रो॒डाशौ᳚ । उ॒पाꣳ॒॒शु॒या॒जमित्यु॑पाꣳशु-या॒जम् । अ॒न्त॒रा । य॒ज॒ति॒ । अजा॑मित्वा॒येत्यजा॑मि - त्वा॒य॒ । अथो॒ इति॑ । मि॒थु॒न॒त्वायेति॑ मिथुन - त्वाय॑ । अ॒ग्निः । अ॒मुष्मिन्न्॑ । लो॒के । आसी᳚त् । य॒मः । अ॒स्मिन्न् । ते । दे॒वाः । अ॒ब्रु॒व॒न्न् । एति॑ । इ॒त॒ । इ॒मौ । वि । परीति॑ । ऊ॒हा॒म॒ । इति॑ । अ॒न्नाद्ये॒नेत्य॑न्न - अद्ये॑न ।   दे॒वाः । अ॒ग्निम् । \textbf{  33} \newline
                  \newline
                                \textbf{ TS 2.6.6.5} \newline
                  उ॒पाम॑न्त्रय॒न्तेत्यु॑प-अम॑न्त्रयन्त । रा॒ज्येन॑ । पि॒तरः॑ । य॒मम् । तस्मा᳚त् । अ॒ग्निः । दे॒वाना᳚म् । अ॒न्ना॒द इत्य॑न्न - अ॒दः । य॒मः । पि॒तृ॒णाम् । राजा᳚ । यः । ए॒वम् । वेद॑ । प्रेति॑ । रा॒ज्यम् । अ॒न्नाद्य॒मित्य॑न्न-अद्य᳚म् । आ॒प्नो॒ति॒ । तस्मै᳚ । ए॒तत् । भा॒ग॒धेय॒मिति॑ भाग - धेय᳚म् । प्रेति॑ । अ॒य॒च्छ॒न्न् । यत् । अ॒ग्नये᳚ । स्वि॒ष्ट॒कृत॒ इति॑ स्विष्ट - कृते᳚ । अ॒व॒द्यन्तीत्य॑व - द्यन्ति॑ । यत् । अ॒ग्नये᳚ । स्वि॒ष्ट॒कृत॒ इति॑ स्विष्ट - कृते᳚ । अ॒व॒द्यतीत्य॑व - द्यति॑ । भा॒ग॒धेये॒नेति॑ भाग-धेये॑न । ए॒व । तत् । रु॒द्रम् । समिति॑ । अ॒र्द्ध॒य॒ति॒ । स॒कृथ्स॑कृ॒दिति॑ स॒कृत् - स॒कृ॒त् । अवेति॑ । द्य॒ति॒ । स॒कृत् । इ॒व॒ । हि । रु॒द्रः । उ॒त्त॒रा॒र्द्धादित्यु॑त्तर - अ॒र्द्धात् । अवेति॑ । द्य॒ति॒ । ए॒षा । वै । रु॒द्रस्य॑ । \textbf{  34} \newline
                  \newline
                                \textbf{ TS 2.6.6.6} \newline
                  दिक् । स्वाया᳚म् । ए॒व । दि॒शि । रु॒द्रम् । नि॒रव॑दयत॒ इति॑ निः-अव॑दयते । द्विः । अ॒भीति॑ । घा॒र॒य॒ति॒ । च॒तु॒र॒व॒त्तस्येति॑ चतुः - अ॒व॒त्तस्य॑ । आप्त्यै᳚ । प॒शवः॑ । वै । पूर्वाः᳚ । आहु॑तय॒ इत्या - हु॒त॒यः॒ । ए॒षः । रु॒द्रः । यत् । अ॒ग्निः । यत् । पूर्वाः᳚ । आहु॑ती॒रित्या - हु॒तीः॒ । अ॒भीति॑ । जु॒हु॒यात् । रु॒द्राय॑ । प॒शून् । अपीति॑ । द॒द्ध्या॒त् । अ॒प॒शुः । यज॑मानः । स्या॒त् । अ॒ति॒हायेत्य॑ति - हाय॑ । पूर्वाः᳚ । आहु॑ती॒रित्या - हु॒तीः॒ । जु॒हो॒ति॒ । प॒शू॒नाम् । गो॒पी॒थाय॑ ॥ \textbf{  35} \newline
                  \newline
                      (श॒प्तः - स्प॑र्.शयति - भू॒ताना॑ - म॒ग्निꣳ - रु॒द्रस्य॑ - स॒प्तत्रिꣳ॑शच्च )  \textbf{(A6)} \newline \newline
                                \textbf{ TS 2.6.7.1} \newline
                  मनुः॑ । पृ॒थि॒व्याः । य॒ज्ञिय᳚म् । ऐ॒च्छ॒त् । सः । घृ॒तम् । निषि॑क्त॒मिति॒ नि - सि॒क्त॒म् । अ॒वि॒न्द॒त् । सः । अ॒ब्र॒वी॒त् । कः । अ॒स्य । ई॒श्व॒रः । य॒ज्ञे । अपीति॑ । कर्तोः᳚ । इति॑ । तौ । अ॒ब्रू॒ता॒म् । मि॒त्रावरु॑णा॒विति॑ मि॒त्रा - वरु॑णौ । गोः । ए॒व । आ॒वम् । ई॒श्व॒रौ । कर्तोः᳚ । स्वः॒ । इति॑ । तौ । ततः॑ । गाम् । समिति॑ । ऐ॒र॒य॒ता॒म् । सा । यत्र॑य॒त्रेति॒ यत्र॑ - य॒त्र॒ । न्यक्रा॑म॒दिति॑ नि - अक्रा॑मत् । ततः॑ । घृ॒तम् । अ॒पी॒ड्य॒त॒ । तस्मा᳚त् । घृ॒तप॒दीति॑ घृ॒त - प॒दी॒ । उ॒च्य॒ते॒ । तत् । अ॒स्यै॒ । जन्म॑ । उप॑हूत॒मित्युप॑ - हू॒त॒म् । र॒थ॒न्त॒रमिति॑ रथं - त॒रम् । स॒ह । पृ॒थि॒व्या । इति॑ । आ॒ह॒ । \textbf{  36} \newline
                  \newline
                                \textbf{ TS 2.6.7.2} \newline
                  इ॒यम् । वै । र॒थ॒न्त॒रमिति॑ रथं - त॒रम् । इ॒माम् । ए॒व । स॒ह । अ॒न्नाद्ये॒नेत्य॑न्न - अद्ये॑न । उपेति॑ । ह्व॒य॒ते॒ । उप॑हूत॒मित्युप॑ - हू॒त॒म् । वा॒म॒दे॒व्यमिति॑ वाम - दे॒व्यम् । स॒ह । अ॒न्तरि॑क्षेण । इति॑ । आ॒ह॒ । प॒शवः॑ । वै । वा॒म॒दे॒व्यमिति॑ वाम - दे॒व्यम् । प॒शून् । ए॒व । स॒ह । अ॒न्तरि॑क्षेण । उपेति॑ । ह्व॒य॒ते॒ । उप॑हूत॒मित्युप॑-हू॒त॒म् । बृ॒हत् । स॒ह । दि॒वा । इति॑ । आ॒ह॒ । ऐ॒रम् । वै । बृ॒हत् । इरा᳚म् । ए॒व । स॒ह । दि॒वा । उपेति॑ । ह्व॒य॒ते॒ । उप॑हूता॒ इत्युप॑ - हू॒ताः॒ । स॒प्त । होत्राः᳚ । इति॑ । आ॒ह॒ । होत्राः᳚ । ए॒व । उपेति॑ । ह्व॒य॒ते॒ । उप॑हू॒तेत्युप॑-हू॒ता॒ । धे॒नुः । \textbf{  37} \newline
                  \newline
                                \textbf{ TS 2.6.7.3} \newline
                  स॒हर्.ष॒भेति॑ स॒ह-ऋ॒ष॒भा॒ । इति॑ । आ॒ह॒ । मि॒थु॒नम् । ए॒व । उपेति॑ । ह्व॒य॒ते॒ । उप॑हूत॒ इत्युप॑ - हू॒तः॒ । भ॒क्षः । सखा᳚ । इति॑ । आ॒ह॒ । सो॒म॒पी॒थमिति॑ सोम - पी॒थम् । ए॒व । उपेति॑ । ह्व॒य॒ते॒ । उप॑हू॒ताॅ(4)इत्युप॑ - हू॒ता(3)ॅ । हो इति॑ । इति॑ । आ॒ह॒ । आ॒त्मान᳚म् । ए॒व । उपेति॑ । ह्व॒य॒ते॒ । आ॒त्मा । हि । उप॑हूताना॒मित्युप॑ - हू॒ता॒ना॒म् । वसि॑ष्ठः । इडा᳚म् । उपेति॑ । ह्व॒य॒ते॒ । प॒शवः॑ । वै । इडा᳚ । प॒शून् । ए॒व । उपेति॑ । ह्व॒य॒ते॒ । च॒तुः । उपेति॑ । ह्व॒य॒ते॒ । चतु॑ष्पाद॒ इति॒ चतुः॑-पा॒दः॒ । हि । प॒शवः॑ । मा॒न॒वी । इति॑ । आ॒ह॒ । मनुः॑ । हि । ए॒ताम् । \textbf{  38} \newline
                  \newline
                                \textbf{ TS 2.6.7.4} \newline
                  अग्रे᳚ । अप॑श्यत् । घृ॒तप॒दीति॑ घृ॒त-प॒दी॒ । इति॑ । आ॒ह॒ । यत् । ए॒व । अ॒स्यै॒ । प॒दात् । घृ॒तम् । अपी᳚ड्यत । तस्मा᳚त् । ए॒वम् । आ॒ह॒ । मै॒त्रा॒व॒रु॒णीति॑ मैत्रा - व॒रु॒णी । इति॑ । आ॒ह॒ । मि॒त्रावरु॑णा॒विति॑ मि॒त्रा - वरु॑णौ । हि । ए॒ना॒म् । स॒मैर॑यता॒मिति॑ सं - ऐर॑यताम् । ब्रह्म॑ । दे॒वकृ॑त॒मिति॑ दे॒व - कृ॒त॒म् । उप॑हूत॒मित्युप॑ - हू॒त॒म् । इति॑ । आ॒ह॒ । ब्रह्म॑ । ए॒व । उपेति॑ । ह्व॒य॒ते॒ । दैव्याः᳚ । अ॒द्ध्व॒र्यवः॑ । उप॑हूता॒ इत्युप॑ - हू॒ताः॒ । उप॑हूता॒ इत्युप॑ - हू॒ताः॒ । म॒नु॒ष्याः᳚ । इति॑ । आ॒ह॒ । दे॒व॒म॒नु॒ष्यानिति॑ देव-म॒नु॒ष्यान् । ए॒व । उपेति॑ । ह्व॒य॒ते॒ । ये । इ॒मम् । य॒ज्ञ्म् । अवान्॑ । ये । य॒ज्ञ्प॑ति॒मिति॑ य॒ज्ञ् - प॒ति॒म् । वर्द्धान्॑ । इति॑ । आ॒ह॒ । \textbf{  39} \newline
                  \newline
                                \textbf{ TS 2.6.7.5} \newline
                  य॒ज्ञाय॑ । च॒ । ए॒व । यज॑मानाय । च॒ । आ॒शिष॒मित्या᳚-शिष᳚म् । एति॑ । शा॒स्ते॒ । उप॑हूते॒ इत्युप॑ - हू॒ते॒ । द्यावा॑पृथि॒वी इति॒ द्यावा᳚ - पृ॒थि॒वी । इति॑ । आ॒ह॒ । द्यावा॑पृथि॒वी इति॒ द्यावा᳚ - पृ॒थि॒वी । ए॒व । उपेति॑ । ह्व॒य॒ते॒ । पू॒र्व॒जे इति॑ पूर्व - जे । ऋ॒ताव॑री॒ इत्यृ॒त - व॒री॒ । इति॑ । आ॒ह॒ । पू॒र्व॒जे इति॑ पूर्व - जे । हि । ए॒ते इति॑ । ऋ॒ताव॑री॒ इत्यृ॒त - व॒री॒ । दे॒वी इति॑ । दे॒वपु॑त्रे॒ इति॑ दे॒व - पु॒त्रे॒ । इति॑ । आ॒ह॒ । दे॒वी इति॑ । हि । ए॒ते इति॑ । दे॒वपु॑त्रे॒ इति॑ दे॒व - पु॒त्रे॒ । उप॑हूत॒ इत्युप॑-हू॒तः॒ । अ॒यम् । यज॑मानः । इति॑ । आ॒ह॒ । यज॑मानम् । ए॒व । उपेति॑ । ह्व॒य॒ते॒ । उत्त॑रस्या॒मित्युत् - त॒र॒स्या॒म् । दे॒व॒य॒ज्याया॒मिति॑ देव - य॒ज्याया᳚म् । उप॑हूत॒ इत्युप॑ - हू॒तः॒ । भूय॑सि । ह॒वि॒ष्कर॑ण॒ इति॑ हविः - कर॑णे । उप॑हूत॒ इत्युप॑ - हू॒तः॒ । दि॒व्ये । धामन्न्॑ । उप॑हूत॒ इत्युप॑ - हू॒तः॒ । \textbf{  40} \newline
                  \newline
                                \textbf{ TS 2.6.7.6} \newline
                  इति॑ । आ॒ह॒ । प्र॒जेति॑ प्र - जा । वै । उत्त॒रेत्युत् - त॒रा॒ । दे॒व॒य॒ज्येति॑ देव - य॒ज्या । प॒शवः॑ । भूयः॑ । ह॒वि॒ष्कर॑ण॒मिति॑ हविः - कर॑णम् । सु॒व॒र्ग इति॑ सुवः-गः । लो॒कः । दि॒व्यम् । धाम॑ । इ॒दम् । अ॒सि॒ । इ॒दम् । अ॒सि॒ । इति॑ । ए॒व । य॒ज्ञ्स्य॑ । प्रि॒यम् । धाम॑ । उपेति॑ । ह्व॒य॒ते॒ । विश्व᳚म् । अ॒स्य॒ । प्रि॒यम् । उप॑हूत॒मित्युप॑ - हू॒त॒म् । इति॑ । आ॒ह॒ । अछ॑बंट्कार॒मित्यछ॑बंट् - का॒र॒म् । ए॒व । उपेति॑ । ह्व॒य॒ते॒ ॥ \textbf{  41} \newline
                  \newline
                      (आ॒ह॒ - धे॒नु- रे॒तां - ॅवर्द्धा॒नित्या॑ह॒ - धाम॒न्नुप॑हूत॒ - श्चतु॑स्त्रिꣳशच्च )  \textbf{(A7)} \newline \newline
                                \textbf{ TS 2.6.8.1} \newline
                  प॒शवः॑ । वै । इडा᳚ । स्व॒यम् । एति॑ । द॒त्ते॒ । काम᳚म् । ए॒व । आ॒त्मना᳚ । प॒शू॒नाम् । एति॑ । द॒त्ते॒ । न । हि । अ॒न्यः । काम᳚म् । प॒शू॒नाम् । प्र॒यच्छ॒तीति॑ प्र - यच्छ॑ति । वा॒चः । पत॑ये । त्वा॒ । हु॒तम् । प्रेति॑ । अ॒श्ना॒मि॒ । इति॑ । आ॒ह॒ । वाच᳚म् । ए॒व । भा॒ग॒धेये॒नेति॑ भाग - धेये॑न । प्री॒णा॒ति॒ । सद॑सः । पत॑ये । त्वा॒ । हु॒तम् । प्रेति॑ । अ॒श्ना॒मि॒ । इति॑ । आ॒ह॒ । स्व॒गाकृ॑त्या॒ इति॑ स्व॒गा - कृ॒त्यै॒ । च॒तु॒र॒व॒त्तमिति॑ चतुः - अ॒व॒त्तम् । भ॒व॒ति॒ । ह॒विः । वै । च॒तु॒र॒व॒त्तमिति॑ चतुः - अ॒व॒त्तम् । प॒शवः॑ । च॒तु॒र॒व॒त्तमिति॑ चतुः - अ॒व॒त्तम् । यत् । होता᳚ । प्रा॒श्नी॒यादिति॑ प्र - अ॒श्नी॒यात् । होता᳚ । \textbf{  42} \newline
                  \newline
                                \textbf{ TS 2.6.8.2} \newline
                  आर्ति᳚म् । एति॑ । ऋ॒च्छे॒त् । यत् । अ॒ग्नौ । जु॒हु॒यात् । रु॒द्राय॑ । प॒शून् । अपीति॑ । द॒द्ध्या॒त् । अ॒प॒शुः । यज॑मानः । स्या॒त् । वा॒चः । पत॑ये । त्वा॒ । हु॒तम् । प्रेति॑ । अ॒श्ना॒मि॒ । इति॑ । आ॒ह॒ । प॒रोक्ष॒मिति॑ परः - अक्ष᳚म् । ए॒व । ए॒न॒त् । जु॒हो॒ति॒ । सद॑सः । पत॑ये । त्वा॒ । हु॒तम् । प्रेति॑ । अ॒श्ना॒मि॒ । इति॑ । आ॒ह॒ । स्व॒गाकृ॑त्या॒ इति॑ स्व॒गा - कृ॒त्यै॒ । प्रेति॑ । अ॒श्न॒न्ति॒ । ती॒र्त्थे । ए॒व । प्रेति॑ । अ॒श्न॒न्ति॒ । दक्षि॑णाम् । द॒दा॒ति॒ । ती॒र्त्थे । ए॒व । दक्षि॑णाम् । द॒दा॒ति॒ । वीति॑ । वै । ए॒तत् । य॒ज्ञ्म् । \textbf{  43} \newline
                  \newline
                                \textbf{ TS 2.6.8.3} \newline
                  छि॒न्द॒न्ति॒ । यत् । म॒द्ध्य॒तः । प्रा॒श्नन्तीति॑ प्र - अ॒श्नन्ति॑ । अ॒द्भिरित्य॑त् - भिः । मा॒र्ज॒य॒न्ते॒ । आपः॑ । वै । सर्वाः᳚ । दे॒वताः᳚ । दे॒वता॑भिः । ए॒व । य॒ज्ञ्म् । समिति॑ । त॒न्व॒न्ति॒ । दे॒वाः । वै । य॒ज्ञात् । रु॒द्रम् । अ॒न्तः । आ॒य॒न्न् । सः । य॒ज्ञ्म् । अ॒वि॒द्ध्य॒त् । तम् । दे॒वाः । अ॒भि । समिति॑ । अ॒ग॒च्छ॒न्त॒ । कल्प॑ताम् । नः॒ । इ॒दम् । इति॑ । ते । अ॒ब्रु॒व॒न्न् । स्वि॑ष्ट॒मिति॒ सु - इ॒ष्ट॒म् । वै । नः॒ । इ॒दम् । भ॒वि॒ष्य॒ति॒ । यत् । इ॒मम् । रा॒ध॒यि॒ष्यामः॑ । इति॑ । तत् । स्वि॒ष्ट॒कृत॒ इति॑ स्विष्ट - कृतः॑ । स्वि॒ष्ट॒कृ॒त्त्वमिति॑ स्विष्टकृत् - त्वम् । तस्य॑ । आवि॑द्ध॒मित्या - वि॒द्ध॒म् । निरिति॑ । \textbf{  44} \newline
                  \newline
                                \textbf{ TS 2.6.8.4} \newline
                  अ॒कृ॒न्त॒न्न् । यवे॑न । संमि॑त॒मिति॒ सं - मि॒त॒म् । तस्मा᳚त् । य॒व॒मा॒त्रमिति॑ यव - मा॒त्रम् । अवेति॑ । द्ये॒त् । यत् । ज्यायः॑ । अ॒व॒द्येदित्य॑व - द्येत् । रो॒पये᳚त् । तत् । य॒ज्ञ्स्य॑ । यत् । उपेति॑ । च॒ । स्तृ॒णी॒यात् । अ॒भीति॑ । च॒ । घा॒रये᳚त् । उ॒भ॒य॒तः॒, सꣳ॒॒श्वा॒यीत्यु॑भयतः - सꣳ॒॒श्वा॒यि । कु॒र्या॒त् । अ॒व॒दायेत्य॑व - दाय॑ । अ॒भीति॑ । घा॒र॒य॒ति॒ । द्विः । समिति॑ । प॒द्य॒ते॒ । द्वि॒पादिति॑ द्वि - पात् । यज॑मानः । प्रति॑ष्ठित्या॒ इति॒ प्रति॑ - स्थि॒त्यै॒ । यत् । ति॒र॒श्चीन᳚म् । अ॒ति॒हरे॒दित्य॑ति - हरे᳚त् । अन॑भिविद्ध॒मित्यन॑भि - वि॒द्ध॒म् । य॒ज्ञ्स्य॑ । अ॒भीति॑ । वि॒द्ध्ये॒त् । अग्रे॑ण । परीति॑ । ह॒र॒ति॒ । ती॒र्त्थेन॑ । ए॒व । परीति॑ । ह॒र॒ति॒ । तत् । पू॒ष्णे । परीति॑ । अ॒ह॒र॒न्न् । तत् । \textbf{  45} \newline
                  \newline
                                \textbf{ TS 2.6.8.5} \newline
                  पू॒षा । प्राश्येति॑ प्र - अश्य॑ । द॒तः । अ॒रु॒ण॒त् । तस्मा᳚त् । पू॒षा । प्र॒पि॒ष्टभा॑ग॒ इति॑ प्रपि॒ष्ट - भा॒गः॒ । अ॒द॒न्तकः॑ । हि । तम् । दे॒वाः । अ॒ब्रु॒व॒न्न् । वीति॑ । वै । अ॒यम् । आ॒र्द्धि॒ । अ॒प्रा॒शि॒त्रि॒य इत्य॑प्र - अ॒शि॒त्रि॒यः । वै । अ॒यम् । अ॒भू॒त् । इति॑ । तत् । बृह॒स्पत॑ये । परीति॑ । अ॒ह॒र॒न्न् । सः । अ॒बि॒भे॒त् । बृह॒स्पतिः॑ । इ॒त्थम् । वाव । स्यः । आर्ति᳚म् । एति॑ । अ॒रि॒ष्य॒ति॒ । इति॑ । सः । ए॒तम् । मन्त्र᳚म् । अ॒प॒श्य॒त् । सूर्य॑स्य । त्वा॒ । चक्षु॑षा । प्रतीति॑ । प॒श्या॒मि॒ । इति॑ । अ॒ब्र॒वी॒त् । न । हि । सूर्य॑स्य । चक्षुः॑ । \textbf{  46} \newline
                  \newline
                                \textbf{ TS 2.6.8.6} \newline
                  किम् । च॒न । हि॒नस्ति॑ । सः । अ॒बि॒भे॒त् । प्र॒ति॒गृ॒ह्णन्त॒मिति॑ प्रति - गृ॒ह्णन्त᳚म् । मा॒ । हिꣳ॒॒सि॒ष्य॒ति॒ । इति॑ । दे॒वस्य॑ । त्वा॒ । स॒वि॒तुः । प्र॒स॒व इति॑ प्र - स॒वे । अ॒श्विनोः᳚ । बा॒हुभ्या॒मिति॑ बा॒हु - भ्या॒म् । पू॒ष्णः । हस्ता᳚भ्याम् । प्रतीति॑ । गृ॒ह्णा॒मि॒ । इति॑ । अ॒ब्र॒वी॒त् । स॒वि॒तृप्र॑सूत॒ इति॑ सवि॒तृ - प्र॒सू॒तः॒ । ए॒व । ए॒न॒त् । ब्रह्म॑णा । दे॒वता॑भिः । प्रतीति॑ । अ॒गृ॒ह्णा॒त् । सः । अ॒बि॒भे॒त् । प्रा॒श्नन्त॒मिति॑ प्र - अ॒श्नन्त᳚म् । मा॒ । हिꣳ॒॒सि॒ष्य॒ति॒ । इति॑ । अ॒ग्नेः । त्वा॒ । आ॒स्ये॑न । प्रेति॑ । अ॒श्ना॒मि॒ । इति॑ । अ॒ब्र॒वी॒त् । न । हि । अ॒ग्नेः । आ॒स्य᳚म् । किम् । च॒न । हि॒नस्ति॑ । सः । अ॒बि॒भे॒त् । \textbf{  47} \newline
                  \newline
                                \textbf{ TS 2.6.8.7} \newline
                  प्राशि॑त॒मिति॒ प्र-अ॒शि॒त॒म् । मा॒ । हिꣳ॒॒सि॒ष्य॒ति॒ । इति॑ । ब्रा॒ह्म॒णस्य॑ । उ॒दरे॑ण । इति॑ । अ॒ब्र॒वी॒त् । न । हि । ब्रा॒ह्म॒णस्य॑ । उ॒दर᳚म् । किम् । च॒न । हि॒नस्ति॑ । बृह॒स्पतेः᳚ । ब्रह्म॑णा । इति॑ । सः । हि । ब्रह्मि॑ष्ठः । अपेति॑ । वै । ए॒तस्मा᳚त् । प्रा॒णा इति॑ प्र - अ॒नाः । क्रा॒म॒न्ति॒ । यः । प्रा॒शि॒त्रमिति॑ प्र - अ॒शि॒त्रम् । प्रा॒श्नातीति॑ प्र - अ॒श्नाति॑ । अ॒द्भिरित्य॑त् - भिः । मा॒र्ज॒यि॒त्वा । प्रा॒णानिति॑ प्र-अ॒नान् । समिति॑ । मृ॒श॒ते॒ । अ॒मृत᳚म् । वै । प्रा॒णा इति॑ प्र - अ॒नाः । अ॒मृत᳚म् । आपः॑ । प्रा॒णानिति॑ प्र - अ॒नान् । ए॒व । य॒था॒स्था॒नमिति॑ यथा - स्था॒नम् । उपेति॑ । ह्व॒य॒ते॒ ॥ \textbf{  48} \newline
                  \newline
                      (प्रा॒श्नी॒याद्धोता॑ - य॒ज्ञ्ं - नि - र॑हर॒न्त - च्चक्षु॑ - रा॒स्यं॑ किञ्च॒न हि॒नस्ति॒ सो॑ऽबिभे॒ - च्चतु॑श्चत्वारिꣳशच्च )  \textbf{(A8)} \newline \newline
                                \textbf{ TS 2.6.9.1} \newline
                  अ॒ग्नीध॒ इत्य॑ग्नि - इधे᳚ । एति॑ । द॒धा॒ति॒ । अ॒ग्निमु॑खा॒नित्य॒ग्नि - मु॒खा॒न् । ए॒व । ऋ॒तून् । प्री॒णा॒ति॒ । स॒मिध॒मिति॑ सं - इध᳚म् । एति॑ । द॒धा॒ति॒ । उत्त॑रासा॒मित्युत् - त॒रा॒सा॒म् । आहु॑तीना॒मित्या - हु॒ती॒ना॒म् । प्रति॑ष्ठित्या॒ इति॒ प्रति॑ - स्थि॒त्यै॒ । अथो॒ इति॑ । स॒मिद्व॒तीति॑ स॒मित् - व॒ति॒ । ए॒व । जु॒हो॒ति॒ । प॒रि॒धीनिति॑ परि - धीन् । समिति॑ । मा॒र्ष्टि॒ । पु॒नाति॑ । ए॒व । ए॒ना॒न् । स॒कृथ्स॑कृ॒दिति॑ स॒कृत् - स॒कृ॒त् । समिति॑ । मा॒र्ष्टि॒ । पराङ्॑ । इ॒व॒ । हि । ए॒तर्.हि॑ । य॒ज्ञ्ः । च॒तुः । समिति॑ । प॒द्य॒ते॒ । चतु॑ष्पाद॒ इति॒ चतुः॑ - पा॒दः॒ । प॒शवः॑ । प॒शून् । ए॒व । अवेति॑ । रु॒न्धे॒ । ब्रह्मन्न्॑ । प्रेति॑ । स्था॒स्या॒मः॒ । इति॑ । आ॒ह॒ । अत्र॑ । वै । ए॒तर्.हि॑ । य॒ज्ञ्ः । श्रि॒तः । \textbf{  49} \newline
                  \newline
                                \textbf{ TS 2.6.9.2} \newline
                  यत्र॑ । ब्र॒ह्मा । यत्र॑ । ए॒व । य॒ज्ञ्ः । श्रि॒तः । ततः॑ । ए॒व । ए॒न॒म् । एति॑ । र॒भ॒ते॒ । यत् । हस्ते॑न । प्र॒मीवे॒दिति॑ प्र - मीवे᳚त् । वे॒प॒नः । स्या॒त् । यत् । शी॒र्ष्णा । शी॒र्॒.ष॒क्ति॒मानिति॑ शीर्.षक्ति-मान् । स्या॒त् । यत् । तू॒ष्णीम् । आसी॑त । अस॑प्रंत्त॒ इत्यसं᳚-प्र॒त्तः॒ । य॒ज्ञ्ः । स्या॒त् । प्रति॑ । ति॒ष्ठ॒ । इति॑ । ए॒व । ब्रू॒या॒त् । वा॒चि । वै । य॒ज्ञ्ः । श्रि॒तः । यत्र॑ । ए॒व । य॒ज्ञ्ः । श्रि॒तः । ततः॑ । ए॒व । ए॒न॒म् । सम् । प्रेति॑ । य॒च्छ॒ति॒ । देव॑ । स॒वि॒तः॒ । ए॒तत् । ते॒ । प्रेति॑ । \textbf{  50} \newline
                  \newline
                                \textbf{ TS 2.6.9.3} \newline
                  आ॒ह॒ । इति॑ । आ॒ह॒ । प्रसू᳚त्या॒ इति॒ प्र - सू॒त्यै॒ । बृह॒स्पतिः॑ । ब्र॒ह्मा । इति॑ । आ॒ह॒ । सः । हि । ब्रह्मि॑ष्ठः । सः । य॒ज्ञ्म् । पा॒हि॒ । सः । य॒ज्ञ्प॑ति॒मिति॑ य॒ज्ञ् - प॒ति॒म् । पा॒हि॒ । सः । माम् । पा॒हि॒ । इति॑ । आ॒ह॒ । य॒ज्ञाय॑ । यज॑मानाय । आ॒त्मने᳚ । तेभ्यः॑ । ए॒व । आ॒शिष॒मित्या᳚ - शिष᳚म् । एति॑ । शा॒स्ते॒ । अना᳚र्त्यै । आ॒श्राव्येत्या᳚ - श्राव्य॑ । आ॒ह॒ । दे॒वान् । य॒ज॒ । इति॑ । ब्र॒ह्म॒वा॒दिन॒ इति॑ ब्रह्म - वा॒दिनः॑ । व॒द॒न्ति॒ । इ॒ष्टाः । दे॒वताः᳚ । अथ॑ । क॒त॒मे । ए॒ते । दे॒वाः । इति॑ । छन्दाꣳ॑सि । इति॑ । ब्रू॒या॒त् । गा॒य॒त्रीम् । त्रि॒ष्टुभ᳚म् । \textbf{  51} \newline
                  \newline
                                \textbf{ TS 2.6.9.4} \newline
                  जग॑तीम् । इति॑ । अथो॒ इति॑ । खलु॑ । आ॒हुः॒ । ब्रा॒ह्म॒णाः । वै । छन्दाꣳ॑सि । इति॑ । तान् । ए॒व । तत् । य॒ज॒ति॒ । दे॒वाना᳚म् । वै । इ॒ष्टाः । दे॒वताः᳚ । आसन्न्॑ । अथ॑ । अ॒ग्निः । न । उदिति॑ । अ॒ज्व॒ल॒त् । तम् । दे॒वाः । आहु॑तीभि॒रित्याहु॑ति - भिः॒ । अ॒नू॒या॒जेष्वित्य॑नु - या॒जेषु॑ । अन्विति॑ । अ॒वि॒न्द॒न्न् । यत् । अ॒नू॒या॒जानित्य॑नु-या॒जान् । यज॑ति । अ॒ग्निम् । ए॒व । तत् । समिति॑ । इ॒न्धे॒ । ए॒तदुः॑ । वै । नाम॑ । आ॒सु॒रः । आ॒सी॒त् । सः । ए॒तर्.हि॑ । य॒ज्ञ्स्य॑ । आ॒शिष॒मित्या᳚ - शिष᳚म् । अ॒वृ॒ङ्क्त॒ । यत् । ब्रू॒यात् । ए॒तत् । \textbf{  52} \newline
                  \newline
                                \textbf{ TS 2.6.9.5} \newline
                  उ॒ । द्या॒वा॒पृ॒थि॒वी॒ इति॑ द्यावा - पृ॒थि॒वी॒ । भ॒द्रम् । अ॒भू॒त् । इति॑ । ए॒तदु᳚म् । ए॒व । आ॒सु॒रम् । य॒ज्ञ्स्य॑ । आ॒शिष॒मित्या᳚ - शिष᳚म् । ग॒म॒ये॒त् । इ॒दम् । द्या॒वा॒पृ॒थि॒वी॒ इति॑ द्यावा - पृ॒थि॒वी॒ । भ॒द्रम् । अ॒भू॒त् । इति॑ । ए॒व । ब्रू॒या॒त् । यज॑मानम् । ए॒व । य॒ज्ञ्स्य॑ । आ॒शिष॒मित्या᳚- शिष᳚म् । ग॒म॒य॒ति॒ । आर्द्ध्म॑ । सू॒क्त॒वा॒कमिति॑ सूक्त - वा॒कम् । उ॒त । न॒मो॒वा॒कमिति॑ नमः-वा॒कम् । इति॑ । आ॒ह॒ । इ॒दम् । अ॒रा॒थ्स्म॒ । इति॑ । वाव । ए॒तत् । आ॒ह॒ । उप॑श्रित॒ इत्युप॑ - श्रि॒तः॒ । दि॒वः । पृ॒थि॒व्योः । इति॑ । आ॒ह॒ । द्यावा॑पृथि॒व्योरिति॒ द्यावा᳚ - पृ॒थि॒व्योः । हि । य॒ज्ञ्ः । उप॑श्रित॒ इत्युप॑ - श्रि॒तः॒ । ओम॑न्वती॒ इत्योमन्न्॑ - व॒ती॒ । ते॒ । अ॒स्मिन्न् । य॒ज्ञे । य॒ज॒मा॒न॒ । द्यावा॑पृथि॒वी इति॒ द्यावा᳚ - पृ॒थि॒वी । \textbf{  53} \newline
                  \newline
                                \textbf{ TS 2.6.9.6} \newline
                  स्ता॒म् । इति॑ । आ॒ह॒ । आ॒शिष॒मित्या᳚-शिष᳚म् । ए॒व । ए॒ताम् । एति॑ । शा॒स्ते॒ । यत् । ब्रू॒यात् । सू॒पा॒व॒सा॒नेति॑ सु - उ॒पा॒व॒सा॒ना । च॒ । स्व॒द्ध्य॒व॒सा॒नेति॑ सु - अ॒द्ध्य॒व॒सा॒ना । च॒ । इति॑ । प्र॒मायु॑क॒ इति॑ प्र - मायु॑कः । यज॑मानः । स्या॒त् । य॒दा । हि । प्र॒मीय॑त॒ इति॑ प्र - मीय॑ते । अथ॑ । इ॒माम् । उ॒पा॒व॒स्यतीत्यु॑प - अ॒व॒स्यति॑ । सू॒प॒च॒र॒णेति॑ सु - उ॒प॒च॒र॒णा । च॒ । स्व॒धि॒च॒र॒णेति॑ सु - अ॒धि॒च॒र॒णा । च॒ । इति॑ । ए॒व । ब्रू॒या॒त् । वरी॑यसीम् । ए॒व । अ॒स्मै॒ । गव्यू॑तिम् । एति॑ । शा॒स्ते॒ । न । प्र॒मायु॑क॒ इति॑ प्र - मायु॑कः । भ॒व॒ति॒ । तयोः᳚ । आ॒विदीत्या᳚-विदि॑ । अ॒ग्निः । इ॒दम् । ह॒विः । अ॒जु॒ष॒त॒ । इति॑ । आ॒ह॒ । याः । अया᳚क्ष्म । \textbf{  54} \newline
                  \newline
                                \textbf{ TS 2.6.9.7} \newline
                  दे॒वताः᳚ । ताः । अ॒री॒र॒धा॒म॒ । इति॑ । वाव । ए॒तत् । आ॒ह॒ । यत् । न । नि॒र्दि॒शेदिति॑ निः - दि॒शेत् । प्रति॑वेश॒मिति॒ प्रति॑ - वे॒श॒म् । य॒ज्ञ्स्य॑ । आ॒शीरित्या᳚ - शीः । ग॒च्छे॒त् । एति॑ । शा॒स्ते॒ । अ॒यम् । यज॑मानः । अ॒सौ । इति॑ । आ॒ह॒ । नि॒र्दिश्येति॑ निः - दिश्य॑ । ए॒व । ए॒न॒म् । सु॒व॒र्गमिति॑ सुवः-गम् । लो॒कम् । ग॒म॒य॒ति॒ । आयुः॑ । एति॑ । शा॒स्ते॒ । सु॒प्र॒जा॒स्त्वमिति॑ सुप्रजाः - त्वम् । एति॑ । शा॒स्ते॒ । इति॑ । आ॒ह॒ । आ॒शिष॒मित्या᳚-शिष᳚म् । ए॒व । ए॒ताम् । एति॑ । शा॒स्ते॒ । स॒जा॒त॒व॒न॒स्यामिति॑ सजात - व॒न॒स्याम् । एति॑ । शा॒स्ते॒ । इति॑ । आ॒ह॒ । प्रा॒णा इति॑ प्र - अ॒नाः । वै । स॒जा॒ता इति॑ स - जा॒ताः । प्रा॒णानिति॑ प्र - अ॒नान् । ए॒व । \textbf{  55} \newline
                  \newline
                                \textbf{ TS 2.6.9.8} \newline
                  न । अ॒न्तः । ए॒ति॒ । तत् । अ॒ग्निः । दे॒वः । दे॒वेभ्यः॑ । वन॑ते । व॒यम् । अ॒ग्नेः । मानु॑षाः । इति॑ । आ॒ह॒ । अ॒ग्निः । दे॒वेभ्यः॑ । व॒नु॒ते । व॒यम् । म॒नु॒ष्ये᳚भ्यः । इति॑ । वाव । ए॒तत् । आ॒ह॒ । इ॒ह । गतिः॑ । वा॒मस्य॑ । इ॒दम् । च॒ । नमः॑ । दे॒वेभ्यः॑ । इति॑ । आ॒ह॒ । याः । च॒ । ए॒व । दे॒वताः᳚ । यज॑ति । याः । च॒ । न । ताभ्यः॑ । ए॒व । उ॒भयी᳚भ्यः । नमः॑ । क॒रो॒ति॒ । आ॒त्मनः॑ । अना᳚र्त्यै ॥ \textbf{  56} \newline
                  \newline
                      (श्रि॒तः - ते॒ प्र - त्रि॒ष्टुभ॑ - मे॒तद् - द्यावा॑पृथि॒वी - या अया᳚क्ष्म- प्रा॒णाने॒व - षट्च॑त्वारिꣳशच्च ) \textbf{(A9)} \newline \newline
                                \textbf{ TS 2.6.10.1} \newline
                  दे॒वाः । वै । य॒ज्ञ्स्य॑ । स्व॒गा॒क॒र्तार॒मिति॑ स्वगा - क॒र्तार᳚म् । न । अ॒वि॒न्द॒न्न् । ते । शं॒ॅयुमिति॑ शं - युम् । बा॒र्॒.ह॒स्प॒त्य॒म् । अ॒ब्रु॒व॒न्न् । इ॒मम् । नः॒ । य॒ज्ञ्म् । स्व॒गेति॑ स्व - गा । कु॒रु॒ । इति॑ । सः । अ॒ब्र॒वी॒त् । वर᳚म् । वृ॒णै॒ । यत् । ए॒व । अब्रा᳚ह्मणोक्त॒ इत्यब्रा᳚ह्मण - उ॒क्तः॒ । अश्र॑द्दधान॒ इत्यश्र॑त् - द॒धा॒नः॒ । यजा॑तै । सा । मे॒ । य॒ज्ञ्स्य॑ । आ॒शीरित्या᳚ - शीः । अ॒स॒त् । इति॑ । तस्मा᳚त् । यत् । अब्रा᳚ह्मणोक्त॒ इत्यब्रा᳚ह्मण - उ॒क्तः॒ । अश्र॑द्दधान॒ इत्यश्र॑त् - द॒धा॒नः॒ । यज॑ते । शं॒ॅयुमिति॑ शं-युम् । ए॒व । तस्य॑ । बा॒र्॒.ह॒स्प॒त्यम् । य॒ज्ञ्स्य॑ । आ॒शीरित्या᳚ - शीः । ग॒च्छ॒ति॒ । ए॒तत् । मम॑ । इति॑ । अ॒ब्र॒वी॒त् । किम् । मे॒ । प्र॒जाया॒ इति॑ प्र - जायाः᳚ । \textbf{  57} \newline
                  \newline
                                \textbf{ TS 2.6.10.2} \newline
                  इति॑ । यः । अ॒प॒गु॒राता॒ इत्य॑प - गु॒रातै᳚ । श॒तेन॑ । या॒त॒या॒त् । यः । नि॒हन॒दिति॑ नि - हन॑त् । स॒हस्रे॑ण । या॒त॒या॒त् । यः । लोहि॑तम् । क॒रव॑त् । याव॑तः । प्र॒स्कद्येति॑ प्र - स्कद्य॑ । पाꣳ॒॒सून् । सं॒गृ॒ह्णादिति॑ सं - गृ॒ह्णात् । ताव॑तः । सं॒ॅव॒थ्स॒रानिति॑ सं - व॒थ्स॒रान् । पि॒तृ॒लो॒कमिति॑ पितृ-लो॒कम् । न । प्रेति॑ । जा॒ना॒त् । इति॑ । तस्मा᳚त् । ब्रा॒ह्म॒णाय॑ । न । अपेति॑ । गु॒रे॒त॒ । न । नीति॑ । ह॒न्या॒त् । न । लोहि॑तम् । कु॒र्या॒त् । ए॒ताव॑ता । ह॒ । एन॑सा । भ॒व॒ति॒ । तत् । शं॒ॅयोरिति॑ शं-योः । एति॑ । वृ॒णी॒म॒हे॒ । इति॑ । आ॒ह॒ । य॒ज्ञ्म् । ए॒व । तत् । स्व॒गेति॑ स्व - गा । क॒रो॒ति॒ । तत् । \textbf{  58} \newline
                  \newline
                                \textbf{ TS 2.6.10.3} \newline
                  शं॒ॅयोरिति॑ शं - योः । एति॑ । वृ॒णी॒म॒हे॒ । इति॑ । आ॒ह॒ । शं॒ॅयुमिति॑ शं - युम् । ए॒व । ब॒र्.॒ह॒स्प॒त्यम् । भा॒ग॒धेये॒नेति॑ भाग - धेये॑न । समिति॑ । अ॒र्द्ध॒य॒ति॒ । गा॒तुम् । य॒ज्ञाय॑ । गा॒तुम् । य॒ज्ञ्प॑तय॒ इति॑ य॒ज्ञ्-प॒त॒ये॒ । इति॑ । आ॒ह॒ । आ॒शिष॒मित्या᳚ - शिष᳚म् । ए॒व । ए॒ताम् । एति॑ । शा॒स्ते॒ । सोम᳚म् । य॒ज॒ति॒ । रेतः॑ । ए॒व । तत् । द॒धा॒ति॒ । त्वष्टा॑रम् । य॒ज॒ति॒ । रेतः॑ । ए॒व । हि॒तम् । त्वष्टा᳚ । रू॒पाणि॑ । वीति॑ । क॒रो॒ति॒ । दे॒वाना᳚म् । पत्नीः᳚ । य॒ज॒ति॒ । मि॒थु॒न॒त्वायेति॑ मिथुन-त्वाय॑ । अ॒ग्निम् । गृ॒हप॑ति॒मिति॑ गृ॒ह - प॒ति॒म् । य॒ज॒ति॒ । प्रति॑ष्ठित्या॒ इति॒ प्रति॑ - स्थि॒त्यै॒ । जा॒मि । वै । ए॒तत् । य॒ज्ञ्स्य॑ । क्रि॒य॒ते॒ । \textbf{  59} \newline
                  \newline
                                \textbf{ TS 2.6.10.4} \newline
                  यत् । आज्ये॑न । प्र॒या॒जा इति॑ प्र - या॒जाः । इ॒ज्यन्ते᳚ । आज्ये॑न । प॒त्नी॒सं॒ॅया॒जा इति॑ पत्नी - सं॒ॅया॒जाः । ऋच᳚म् । अ॒नूच्येत्य॑नु - उच्य॑ । प॒त्नी॒सं॒ॅया॒जाना॒मिति॑ पत्नी - सं॒ॅया॒जाना᳚म् । ऋ॒चा । य॒ज॒ति॒ । अजा॑मित्वा॒येत्यजा॑मि - त्वा॒य॒ । अथो॒ इति॑ । मि॒थु॒न॒त्वायेति॑ मिथुन - त्वाय॑ । प॒ङ्क्तिप्रा॑यण॒ इति॑ प॒ङ्क्ति - प्रा॒य॒णः॒ । वै । य॒ज्ञ्ः । प॒ङ्क्त्यु॑दयन॒ इति॑ प॒ङ्क्ति - उ॒द॒य॒नः॒ । पञ्च॑ । प्र॒या॒जा इति॑ प्र - या॒जाः । इ॒ज्य॒न्ते॒ । च॒त्वारः॑ । प॒त्नी॒सं॒ॅया॒जा इति॑ पत्नी - सं॒ॅया॒जाः । स॒मि॒ष्ट॒य॒जुरिति॑ समिष्ट - य॒जुः । प॒ञ्च॒मम् । प॒ङ्क्तिम् । ए॒व । अन्विति॑ । प्र॒यन्तीति॑ प्र - यन्ति॑ । प॒ङ्क्तिम् । अनु॑ । उदिति॑ । य॒न्ति॒ ॥ \textbf{  60} \newline
                  \newline
                      (प्र॒जायाः᳚ - करोति॒ तत् - क्रि॑यते॒ - त्रय॑स्त्रिꣳशच्च )  \textbf{(A10)} \newline \newline
                                \textbf{ TS 2.6.11.1} \newline
                  यु॒क्ष्व । हि । दे॒व॒हूत॑मा॒निति॑ देव-हूत॑मान् । अश्वान्॑ । अ॒ग्ने॒ । र॒थीः । इ॒व॒ ॥ नीति॑ । होता᳚ । पू॒र्व्यः । स॒दः॒ ॥ उ॒त । नः॒ । दे॒व॒ । दे॒वान् । अच्छ॑ । वो॒चः॒ । वि॒दुष्ट॑र॒ इति॑ वि॒दुः - त॒रः॒ ॥ श्रत् । विश्वा᳚ । वार्या᳚ । कृ॒धि॒ ॥ त्वम् । ह॒ । यत् । य॒वि॒ष्ठ्य॒ । सह॑सः । सू॒नो॒ । आ॒हु॒तेत्या᳚ - हु॒त॒ ॥ ऋ॒तावेत्यृ॒त - वा॒ । य॒ज्ञियः॑ । भुवः॑ ॥ अ॒यम् । अ॒ग्निः । स॒ह॒स्रिणः॑ । वाज॑स्य । श॒तिनः॑ । पतिः॑ ॥ मू॒र्द्धा । क॒विः । र॒यी॒णाम् ॥ तम् । ने॒मिम् । ऋ॒भवः॑ । य॒था॒ । एति॑ । न॒म॒स्व॒ । सहू॑तिभि॒रिति॒ सहू॑ति - भिः॒ ॥ नेदी॑यः । य॒ज्ञ्म् । \textbf{  61} \newline
                  \newline
                                \textbf{ TS 2.6.11.2} \newline
                  अ॒ङ्गि॒रः॒ ॥ तस्मै᳚ । नू॒नम् । अ॒भिद्य॑व॒ इत्य॒भि - द्य॒वे॒ । वा॒चा । वि॒रू॒पेति॑ वि - रू॒प॒ । नित्य॑या ॥ वृष्णे᳚ । चो॒द॒स्व॒ । सु॒ष्टु॒तिमिति॑ सु - स्तु॒तिम् ॥ कम् । उ॒ । स्वि॒त् । अ॒स्य॒ । सेन॑या । अ॒ग्नेः । अपा॑कचक्षस॒ इत्यपा॑क-च॒क्ष॒सः॒ ॥ प॒णिम् । गोषु॑ । स्त॒रा॒म॒हे॒ ॥ मा । नः॒ । दे॒वाना᳚म् । विशः॑ । प्र॒स्ना॒तीरिति॑ प्र - स्ना॒तीः । इ॒व॒ । उ॒स्राः ॥ कृ॒शम् । न । हा॒सुः॒ । अघ्नि॑याः ॥ मा । नः॒ । स॒म॒स्य॒ । दू॒ढ्यः॑ । परि॑द्वेषस॒ इति॒ परि॑ - द्वे॒ष॒सः॒ । अꣳ॒॒ह॒तिः ॥ ऊ॒र्मिः । न । नाव᳚म् । एति॑ । व॒धी॒त् ॥ नमः॑ । ते॒ । अ॒ग्ने॒ । ओज॑से । गृ॒णन्ति॑ । दे॒व॒ । कृ॒ष्टयः॑ ॥ अमैः᳚ । \textbf{  62} \newline
                  \newline
                                \textbf{ TS 2.6.11.3} \newline
                  अ॒मित्र᳚म् । अ॒र्द॒य॒ ॥ कु॒वित् । स्विति॑ । नः॒ । गवि॑ष्टय॒ इति॒ गो-इ॒ष्ट॒ये॒ । अग्ने᳚ । सं॒ॅवेषि॑ष॒ इति॑ सं - वेषि॑षः । र॒यिम् ॥ उरु॑कृ॒दित्युरु॑-कृ॒त् । उ॒रु । नः॒ ॥ कृ॒धि॒ । मा । नः॒ । अ॒स्मिन्न् । म॒हा॒ध॒न इति॑ महा-ध॒ने । परेति॑ । व॒र्क् । भा॒र॒भृदिति॑ भार - भृत् । य॒था॒ ॥ सं॒ॅवर्ग॒मिति॑ सं - वर्ग᳚म् । समिति॑ । र॒यिम् । ज॒य॒ ॥ अ॒न्यम् । अ॒स्मत् । भि॒यै । इ॒यम् । अग्ने᳚ । सिष॑क्तु । दु॒च्छुना᳚ ॥ वर्द्ध॑ । नः॒ । अम॑व॒दित्यम॑-व॒त् । शवः॑ ॥ यस्य॑ । अजु॑षत् । न॒म॒स्विनः॑ । शमी᳚म् । अदु॑र्मख॒स्येत्यदुः॑-म॒ख॒स्य॒ । वा॒ ॥ तम् । घ॒ । इत् । अ॒ग्निः । वृ॒धा । अ॒व॒ति॒ ॥ पर॑स्याः । अधीति॑ । \textbf{  63} \newline
                  \newline
                                \textbf{ TS 2.6.11.4} \newline
                  सं॒ॅवत॒ इति॑ सं-वतः॑ । अव॑रान् । अ॒भि । एति॑ । त॒र॒ ॥ यत्र॑ । अ॒हम् । अस्मि॑ । तान् । अ॒व॒ ॥ वि॒द्म । हि । ते॒ । पु॒रा । व॒यम् । अग्ने᳚ । पि॒तुः । यथा᳚ । अव॑सः ॥ अध॑ । ते॒ । सु॒म्नम् । ई॒म॒हे॒ ॥ यः । उ॒ग्रः । इ॒व॒ । श॒र्य॒हेति॑ शर्य - हा । ति॒ग्मशृ॑ङ्ग॒ इति॑ ति॒ग्म - शृ॒ङ्गः॒ । न । वꣳस॑गः ॥ अग्ने᳚ । पुरः॑ । रु॒रोजि॑थ ॥ सखा॑यः । समिति॑ । वः॒ । स॒म्यञ्च᳚म् । इष᳚म् । स्तोम᳚म् । च॒ । अ॒ग्नये᳚ ॥ वर्.षि॑ष्ठाय । क्षि॒ती॒नाम् । ऊ॒र्जः । नप्त्रे᳚ । सह॑स्वते ॥ सꣳस॒मिति॒ सं-स॒म् । इत् । यु॒व॒से॒ । वृ॒ष॒न्न् ( ) । अग्ने᳚ । विश्वा॑नि । अ॒र्यः । आ ॥ इ॒डः । प॒दे । समिति॑ । इ॒द्ध्य॒से॒ । सः । नः॒ । वसू॑नि । एति॑ । भ॒र॒ ॥ प्रजा॑पत॒ इति॒ प्रजा᳚ - प॒ते॒ । सः । वे॒द॒ । सोमा॑पूष॒णेति॒ सोमा᳚ - पू॒ष॒णा॒ । इ॒मौ । दे॒वौ ॥ \textbf{  64} \newline
                  \newline
                      (य॒ज्ञ् - ममै॒ - रधि॑ - वृष॒ - न्नेका॒न्न विꣳ॑श॒तिश्च॑ )  \textbf{(A11)} \newline \newline
                                \textbf{ TS 2.6.12.1} \newline
                  उ॒शन्तः॑ । त्वा॒ । ह॒वा॒म॒हे॒ । उ॒शन्तः॑ । समिति॑ । इ॒धी॒म॒हि॒ ॥ उ॒शन्न् । उ॒श॒तः । एति॑ । व॒ह॒ । पि॒तॄन् । ह॒विषे᳚ । अत्त॑वे ॥ त्वम् । सो॒म॒ । प्रचि॑कित॒ इति॒ प्र - चि॒कि॒तः॒ । म॒नी॒षा । त्वम् । रजि॑ष्ठम् । अन्विति॑ । ने॒षि॒ । पन्था᳚म् ॥ तव॑ । प्रणी॒तीति॒ प्र - नी॒ती॒ । पि॒तरः॑ । नः॒ । इ॒न्दो॒ इति॑ । दे॒वेषु॑ । रत्न᳚म् । अ॒भ॒ज॒न्त॒ । धीराः᳚ ॥ त्वया᳚ । हि । नः॒ । पि॒तरः॑ । सो॒म॒ । पूर्वे᳚ । कर्मा॑णि । च॒क्रुः । प॒व॒मा॒न॒ । धीराः᳚ ॥ व॒न्वन्न् । अवा॑तः । प॒रि॒धीनिति॑ परि - धीन् । अपेति॑ । ऊ॒र्णु॒ । वी॒रेभिः॑ । अश्वैः᳚ । म॒घवेति॑ म॒घ - वा॒ । भ॒व॒ । \textbf{  65} \newline
                  \newline
                                \textbf{ TS 2.6.12.2} \newline
                  नः॒ ॥ त्वम् । सो॒म॒ । पि॒तृभि॒रिति॑ पि॒तृ - भिः॒ । सं॒ॅवि॒दा॒न इति॑ सं - वि॒दा॒नः । अन्विति॑ । द्यावा॑पृथि॒वी इति॒ द्यावा᳚-पृ॒थि॒वी । एति॑ । त॒त॒न्थ॒ ॥ तस्मै᳚ । ते॒ । इ॒न्दो॒ इति॑ । ह॒विषा᳚ । वि॒धे॒म॒ । व॒यम् । स्या॒म॒ । पत॑यः । र॒यी॒णाम् ॥ अग्नि॑ष्वात्ता॒ इत्यग्नि॑- स्वा॒त्ताः॒ । पि॒त॒रः॒ । एति॑ । इ॒ह । ग॒च्छ॒त॒ । सदः॑ सद॒ इति॒ सदः॑ - स॒दः॒ । स॒द॒त॒ । सु॒प्र॒णी॒त॒य॒ इति॑ सु - प्र॒णी॒त॒यः॒ ॥ अ॒त्त । ह॒वीꣳषि॑ । प्रय॑ता॒नीति॒ प्र - य॒ता॒नि॒ । ब॒र्॒.हिषि॑ । अथ॑ । र॒यिम् । सर्व॑वीर॒मिति॒ सर्व॑ - वी॒र॒म् । द॒धा॒त॒न॒ ॥ बर्.हि॑षद॒ इति॒ बर्.हि॑ - स॒दः॒ । पि॒त॒रः॒ । ऊ॒ती । अ॒र्वाक् । इ॒मा । वः॒ । ह॒व्या । च॒कृ॒म॒ । जु॒षद्ध्व᳚म् ॥ ते । एति॑ । ग॒त॒ । अव॑सा । शंत॑मे॒नेति॒ शं - त॒मे॒न॒ । अथ॑ । अ॒स्मभ्य॒मित्य॒स्म - भ्य॒म् । \textbf{  66} \newline
                  \newline
                                \textbf{ TS 2.6.12.3} \newline
                  शम् । योः । अ॒र॒पः । द॒धा॒त॒ ॥ एति॑ । अ॒हम् । पि॒तॄन् । सु॒वि॒दत्रा॒निति॑ सु - वि॒दत्रान्॑ । अ॒वि॒थ्सि॒ । नपा॑तम् । च॒ । वि॒क्रम॑ण॒मिति॑ वि - क्रम॑णम् । च॒ । विष्णोः᳚ ॥ ब॒र्॒.हि॒षद॒ इति॑ बर्.हि - सदः॑ । ये । स्व॒धयेति॑ स्व - धया᳚ । सु॒तस्य॑ । भज॑न्त । पि॒त्वः । ते । इ॒ह । आग॑मिष्ठा॒ इत्या - ग॒मि॒ष्ठाः॒ ॥ उप॑हूता॒ इत्युप॑ - हू॒ताः॒ । पि॒तरः॑ । सो॒म्यासः॑ । ब॒र्॒.हि॒ष्ये॑षु । नि॒धिष्विति॑ नि-धिषु॑ । प्रि॒येषु॑ ॥ ते । एति॑ । ग॒म॒न्तु॒ । ते । इ॒ह । श्रु॒व॒न्तु॒ । अधीति॑ । ब्रु॒व॒न्तु॒ । ते । अ॒व॒न्तु॒ । अ॒स्मान् ॥ उदिति॑ । ई॒र॒ता॒म् । अव॑रे । उदिति॑ । परा॑सः । उदिति॑ । म॒द्ध्य॒माः । पि॒तरः॑ । सो॒म्यासः॑ ॥ असु᳚म् । \textbf{  67} \newline
                  \newline
                                \textbf{ TS 2.6.12.4} \newline
                  ये । ई॒युः । अ॒वृ॒काः । ऋ॒त॒ज्ञा इत्यृ॑त - ज्ञाः । ते । नः॒ । अ॒व॒न्तु॒ । पि॒तरः॑ । हवे॑षु ॥ इ॒दम् । पि॒तृभ्य॒ इति॑ पि॒तृ - भ्यः॒ । नमः॑ । अ॒स्तु॒ । अ॒द्य । ये । पूर्वा॑सः । ये । उप॑रासः । ई॒युः ॥ ये । पार्थि॑वे । रज॑सि । एति॑ । निष॑त्ता॒ इति॒ नि - स॒त्ताः॒ । ये । वा॒ । नू॒नम् । सु॒वृ॒जना॒स्विति॑ सु-वृ॒जना॑सु । वि॒क्षु ॥ अध॑ । यथा᳚ । नः॒ । पि॒तरः॑ । परा॑सः । प्र॒त्नासः॑ । अ॒ग्ने॒ । ऋ॒तम् । आ॒शु॒षा॒णाः ॥ शुचि॑ । इत् । अ॒य॒न्न् । दीधि॑तिम् । उ॒क्थ॒शास॒ इत्यु॑क्थ - शासः॑ । क्षाम॑ । भि॒न्दन्तः॑ । अ॒रु॒णीः । अपेति॑ । व्र॒न्न् ॥ यत् । अ॒ग्ने॒ । \textbf{  68} \newline
                  \newline
                                \textbf{ TS 2.6.12.5} \newline
                  क॒व्य॒वा॒ह॒नेति॑ कव्य - वा॒ह॒न॒ । पि॒तॄन् । यक्षि॑ । ऋ॒ता॒वृध॒ इत्यृ॑त - वृधः॑ ॥ प्रेति॑ । च॒ । ह॒व्यानि॑ । व॒क्ष्य॒सि॒ । दे॒वेभ्यः॑ । च॒ । पि॒तृभ्य॒ इति॑ पि॒तृ - भ्यः॒ । आ ॥ त्वम् । अ॒ग्ने॒ । ई॒डि॒तः । जा॒त॒वे॒द॒ इति॑ जात - वे॒दः॒ । अवा᳚ट् । ह॒व्यानि॑ । सु॒र॒भीणि॑ । कृ॒त्वा ॥ प्रेति॑ । अ॒दाः॒ । पि॒तृभ्य॒ इति॑ पि॒तृ - भ्यः॒ । स्व॒धयेति॑ स्व - धया᳚ । ते । अ॒क्ष॒न्न् । अ॒द्धि । त्वम् । दे॒व॒ । प्रय॒तेति॒ प्र - य॒ता॒ । ह॒वीꣳषि॑ ॥ मात॑ली । क॒व्यैः । य॒मः । अङ्गि॑रोभि॒रित्यङ्गि॑रः - भिः॒ । बृह॒स्पतिः॑ । ऋक्व॑भि॒रित्यृक्व॑-भिः॒ । वा॒वृ॒धा॒नः ॥ यान् । च॒ । दे॒वाः । वा॒वृ॒धुः । ये । च॒ । दे॒वान् । स्वाहा᳚ । अ॒न्ये । स्व॒धयेति॑ स्व-धया᳚ । अ॒न्ये । म॒द॒न्ति॒ ॥ \textbf{  69} \newline
                  \newline
                                \textbf{ TS 2.6.12.6} \newline
                  इ॒मम् । य॒म॒ । प्र॒स्त॒रमिति॑ प्र - स्त॒रम् । एति॑ । हि । सीद॑ । अङ्गि॑रोभि॒रित्यङ्गि॑रः - भिः॒ । पि॒तृभि॒रिति॑ पि॒तृ - भिः॒ । सं॒ॅवि॒दा॒न इति॑ सं - वि॒दा॒नः ॥ एति॑ । त्वा॒ । मन्त्राः᳚ । क॒वि॒श॒स्ता इति॑ कवि - श॒स्ताः । व॒ह॒न्तु॒ । ए॒ना । रा॒ज॒न्न् । ह॒विषा᳚ । मा॒द॒य॒स्व॒ ॥ अङ्गि॑रोभि॒रित्यङ्गि॑रः - भिः॒ । एति॑ । ग॒हि॒ । य॒ज्ञिये॑भिः । यम॑ । वै॒रू॒पैः । इ॒ह । मा॒द॒य॒स्व॒ ॥ विव॑स्वन्तम् । हु॒वे॒ । यः । पि॒ता । ते॒ । अ॒स्मिन्न् । य॒ज्ञे । ब॒र्॒.हिषि॑ । एति॑ । नि॒षद्येति॑ नि - सद्य॑ ॥ अङ्गि॑रसः । नः॒ । पि॒तरः॑ । नव॑ग्वाः । अथ॑र्वाणः । भृग॑वः । सो॒म्यासः॑ ॥ तेषा᳚म् । व॒यम् । सु॒म॒ताविति॑ सु - म॒तौ । य॒ज्ञिया॑नाम् । अपीति॑ । भ॒द्रे । सौ॒म॒न॒से ( ) । स्या॒म॒ ॥ \textbf{  70} \newline
                  \newline
                      (भ॒वा॒ - ऽस्मभ्य॒ - मसुं॒ - ॅयद॑ग्ने - मदन्ति - सौमन॒स - एकं॑ च )  \textbf{(A12)} \newline \newline
\textbf{praSna korvai with starting padams of 1 to 12 Anuvaakams :-} \newline
(स॒मिध॒ - श्चक्षु॑षी - प्र॒जाप॑ति॒राज्यं॑ - दे॒वस्य॒ स्फ्यं - ब्र॑ह्मवा॒दिनो॒ ऽद्भि - र॒ग्नेस्त्रयो॒ - मनुः॑ पृथि॒व्याः - प॒शवो॒ - ऽग्नीधे॑ - दे॒वा वै य॒ज्ञ्स्य॑ - यु॒क्ष्वो - शन्त॑स्त्वा॒ - द्वाद॑श ) \newline

\textbf{korvai with starting padams of1, 11, 21 series of pa~jcAtis :-} \newline
(स॒मिधो॑ - या॒ज्या॑ - तस्मा॒न्नभा॒ऽगꣳ - हि तमन्वि- त्या॑ह प्र॒जा वा - आ॒हेत्या॑ह - यु॒क्ष्वा हि - स॑प्त॒तिः ) \newline

\textbf{first and last padam of Sixth praSnam of kANDam 2 :-} \newline
(स॒मिधः॑ - सौमन॒से स्या॑म ) \newline 


॥ हरिः॑ ॐ ॥॥ कृष्ण यजुर्वेदीय तैत्तिरीय संहितायां द्वितीयकाण्डे षष्टः प्रश्नः समाप्तः ॥
॥ इति द्वीतीयं काण्डं ॥ \newline
\pagebreak
2.6.1   Appendix\\2.6.11.4- प्रजा॑पते॒>1\\प्रजा॑पते॒ न त्वदे॒तान्य॒न्यो विश्वा॑ जा॒तानि॒ परि॒ ता ब॑भूव । \\यत्का॑मास्ते जुहु॒मस्तन्नो॑ अस्तु व॒यꣳ स्या॑म॒ पत॑यो रयी॒णाम् ॥\\(Appearing in TS-1.8.14.2)\\\\2.6.11.4- स वे॑द॒>2\\स वे॑द पु॒त्रः पि॒तरꣳ॒॒ स मा॒तरꣳ॒॒ स सू॒नुभ॑र्व॒थ् स भु॑व॒त् पुन॑र्मघः । \\स द्यामौर्णो॑द॒न्तरि॑क्षꣳ॒॒ स सु॒वः स विश्वा॒ भुवो॑ अभव॒थ् स \\आ*ऽभ॑वत् ॥\\(Appearing in TS - 2.2.12.1)\\\\2.6.11.4 - सोमा॑ पूषणे॒>3 , मौ दे॒वौ>4\\सोमा॑पूषणा॒ जन॑ना रयी॒णां जन॑ना दि॒वो जन॑ना पृथि॒व्याः ।\\जा॒तौ विश्व॑स्य॒ भुव॑नस्य गो॒पौ दे॒वा अ॑कृण्वन्न॒मृत॑स्य॒ नाभिं᳚ ॥\\\\इ॒मौ दे॒वौ जाय॑मानौ जुषन्ते॒मौ तमाꣳ॑सि गूहता॒मजु॑ष्टा । \\आ॒भ्यामिन्द्रः॑ प॒क्वमा॒मास्व॒न्तः सो॑मापू॒षभ्यां᳚ जनदु॒स्रिया॑सु ॥ \\(Appearing in TS- 1.8.22.5)\\\\\\\\
\pagebreak
        


\end{document}
