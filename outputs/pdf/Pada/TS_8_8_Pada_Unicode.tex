\documentclass[17pt]{extarticle}
\usepackage{babel}
\usepackage{fontspec}
\usepackage{polyglossia}
\usepackage{extsizes}



\setmainlanguage{sanskrit}
\setotherlanguages{english} %% or other languages
\setlength{\parindent}{0pt}
\pagestyle{myheadings}
\newfontfamily\devanagarifont[Script=Devanagari]{AdishilaVedic}


\newcommand{\VAR}[1]{}
\newcommand{\BLOCK}[1]{}




\begin{document}
\begin{titlepage}
    \begin{center}
 
\begin{sanskrit}
    { \Large
    ॐ नमः परमात्मने, श्री महागणपतये नमः, 
श्री गुरुभ्यो नमः । ह॒रिः॒ ॐ ॥ 
    }
    \\
    \vspace{2.5cm}
    \mbox{ \Huge
    1.8     प्रथमकाण्डे अषमः प्रश्नः - ( राजसूयः )   }
\end{sanskrit}
\end{center}

\end{titlepage}
\tableofcontents

ॐ नमः परमात्मने, श्री महागणपतये नमः,
श्री गुरुभ्यो नमः । ह॒रिः॒ ॐ ॥ \newline
1.8     प्रथमकाण्डे अषमः प्रश्नः - ( राजसूयः ) \newline

\addcontentsline{toc}{section}{ 1.8     प्रथमकाण्डे अषमः प्रश्नः - ( राजसूयः )}
\markright{ 1.8     प्रथमकाण्डे अषमः प्रश्नः - ( राजसूयः ) \hfill https://www.vedavms.in \hfill}
\section*{ 1.8     प्रथमकाण्डे अषमः प्रश्नः - ( राजसूयः ) }
                                \textbf{ TS 1.8.1.1} \newline
                  अनु॑मत्या॒ इत्यनु॑-म॒त्यै॒ । पु॒रो॒डाश᳚म् । अ॒ष्टाक॑पाल॒मित्य॒ष्टा-क॒पा॒ल॒म् । निरिति॑ । व॒प॒ति॒ । धे॒नुः । दक्षि॑णा । ये । प्र॒त्यञ्चः॑ । शम्या॑याः । अ॒व॒शीय॑न्त॒ इत्य॑व - शीय॑न्ते । तम् । नै॒र्.॒ऋ॒तमिति॑ नैः - ऋ॒तम् । एक॑कपाल॒मित्येक॑ - क॒पा॒ल॒म् । कृ॒ष्णम् । वासः॑ । कृ॒ष्णतू॑ष॒मिति॑ कृ॒ष्ण-तू॒ष॒म् । दक्षि॑णा । वीति॑ । इ॒हि॒ । स्वाहा᳚ । आहु॑ति॒मित्या-हु॒ति॒म् । जु॒षा॒णः । ए॒षः । ते॒ । नि॒र्.॒ऋ॒त॒ इति॑ निः-ऋ॒ते॒ । भा॒गः । भूते᳚ । ह॒विष्म॑ती । अ॒सि॒ । मु॒ञ्च । इ॒मम् । अꣳह॑सः । स्वाहा᳚ । नमः॑ । यः । इ॒दम् । च॒कार॑ । आ॒दि॒त्यम् । च॒रुम् । निरिति॑ । व॒प॒ति॒ । वरः॑ । दक्षि॑णा । आ॒ग्ना॒वै॒ष्ण॒वमित्या᳚ग्ना - वै॒ष्ण॒वम् । एका॑दशकपाल॒मित्येका॑दश - क॒पा॒ल॒म् । वा॒म॒नः । व॒ही । दक्षि॑णा । अ॒ग्नी॒षो॒मीय॒मित्य॑ग्नी - सो॒मीय᳚म् । \textbf{  1} \newline
                  \newline
                                \textbf{ TS 1.8.1.2} \newline
                  एका॑दशकपाल॒मित्येका॑दश-क॒पा॒ल॒म् । हिर॑ण्यम् । दक्षि॑णा । ऐ॒न्द्रम् । एका॑दशकपाल॒मित्येका॑दश - क॒पा॒ल॒म् । ऋ॒ष॒भः । व॒ही । दक्षि॑णा । आ॒ग्ने॒यम् । अ॒ष्टाक॑पाल॒मित्य॒ष्टा-क॒पा॒ल॒म् । ऐ॒न्द्रम् । दधि॑ । ऋ॒ष॒भः । व॒ही । दक्षि॑णा । ऐ॒न्द्रा॒ग्नमित्यै᳚न्द्र -अ॒ग्नम् । द्वाद॑शकपाल॒मिति॒ द्वाद॑श -क॒पा॒ल॒म् । वै॒श्व॒दे॒वमिति॑ वैश्व - दे॒वम् । च॒रुम् । प्र॒थ॒म॒ज इति॑ प्रथम - जः । व॒थ्सः । दक्षि॑णा । सौ॒म्यम् । श्या॒मा॒कम् । च॒रुम् । वासः॑ । दक्षि॑णा । सर॑स्वत्यै । च॒रुम् । सर॑स्वते । च॒रुम् । मि॒थु॒नौ । गावौ᳚ । दक्षि॑णा ॥ \textbf{  2} \newline
                  \newline
                      (अ॒ग्नी॒षो॒मीयं॒-चतु॑स्त्रिꣳशच्च)  \textbf{(A1)} \newline \newline
                                \textbf{ TS 1.8.2.1} \newline
                  आ॒ग्ने॒यम् । अ॒ष्टाक॑पाल॒मित्य॒ष्टा-क॒पा॒ल॒म् । निरिति॑ । व॒प॒ति॒ । सौ॒म्यम् । च॒रुम् । सा॒वि॒त्रम् । द्वाद॑शकपाल॒मिति॒ द्वाद॑श-क॒पा॒ल॒म् । सा॒र॒स्व॒तम् । च॒रुम् । पौ॒ष्णम् । च॒रुम् । मा॒रु॒तम् । स॒प्तक॑पाल॒मिति॑ स॒प्त -क॒पा॒ल॒म् । वै॒श्व॒दे॒वीमिति॑ वैश्व - दे॒वीम् । आ॒मिक्षा᳚म् । द्या॒वा॒पृ॒थि॒व्य॑मिति॑ द्यावा-पृ॒थि॒व्य᳚म् । एक॑कपाल॒मित्येक॑-क॒पा॒ल॒म् ॥ \textbf{  3} \newline
                  \newline
                      (आ॒ग्ने॒यꣳ सौ॒म्यं मा॑रु॒त-म॒ष्टाद॑श)  \textbf{(A2)} \newline \newline
                                \textbf{ TS 1.8.3.1} \newline
                  ऐ॒न्द्रा॒ग्नमित्यै᳚न्द्र - अ॒ग्नम् । एका॑दशकपाल॒मित्येका॑दश-क॒पा॒ल॒म् । मा॒रु॒तीम् । आ॒मिक्षा᳚म् । वा॒रु॒णीम् । आ॒मिक्षा᳚म् । का॒यम् । एक॑कपाल॒मित्येक॑-क॒पा॒लं॒ । प्र॒घा॒स्या॑निति॑ प्र-घा॒स्यान्॑ । ह॒वा॒म॒हे॒ । म॒रुतः॑ । य॒ज्ञ्वा॑हस॒ इति॑ य॒ज्ञ् - वा॒ह॒सः॒ । क॒र॒भेंण॑ । स॒जोष॑स॒ इति॑ स - जोष॑सः ॥ मो इति॑ । स्विति॑ । नः॒ । इ॒न्द्र॒ । पृ॒थ्स्विति॑ पृत्-सु । दे॒व॒ । अस्तु॑ । स्म॒ । ते॒ । शु॒ष्मि॒न्न् । अ॒व॒या ॥ म॒ही । हि । अ॒स्य॒ । मी॒ढुषः॑ । य॒व्या । ह॒विष्म॑तः । म॒रुतः॑ । वन्द॑ते । गीः ॥ यत् । ग्रामे᳚ । यत् । अर॑ण्ये । यत् । स॒भाया᳚म् । यत् । इ॒न्द्रि॒ये ॥ यत् । शू॒द्रे । यत् । अ॒र्ये᳚ । एनः॑ । च॒कृ॒म । व॒यम् ॥ यत् ( ) । एक॑स्य । अधीति॑ । धर्म॑णि । तस्य॑ । अ॒व॒यज॑न॒मित्य॑व-यज॑नम् । अ॒सि॒ । स्वाहा᳚ ॥ अक्रन्न्॑ । कर्म॑ । क॒र्म॒कृत॒ इति॑ कर्म-कृतः॑ । स॒ह । वा॒चा । म॒यो॒भु॒वेति॑ मयः-भु॒वा ॥ दे॒वेभ्यः॑ । कर्म॑ । कृ॒त्वा । अस्त᳚म् । प्रेति॑ । इ॒त॒ । सु॒दा॒न॒व॒ इति॑ सु-दा॒न॒वः॒ ॥ \textbf{  4 } \newline
                  \newline
                      (व॒यंॅयद्-विꣳ॑श॒तिश्च॑)  \textbf{(A3)} \newline \newline
                                \textbf{ TS 1.8.4.1} \newline
                  अ॒ग्नये᳚ । अनी॑कवत॒ इत्यनी॑क - व॒ते॒ । पु॒रो॒डाश᳚म् । अ॒ष्टाक॑पाल॒मित्य॒ष्टा-क॒पा॒ल॒म् । निरिति॑ । व॒प॒ति॒ । सा॒कम् । सूर्ये॑ण । उ॒द्य॒तेत्यु॑त् - य॒ता । म॒रुद्भ्य॒ इति॑ म॒रुत् - भ्यः॒ । सा॒न्त॒प॒नेभ्य॒ इति॑ सां - त॒प॒नेभ्यः॑ । म॒द्ध्यन्दि॑ने । च॒रुम् । म॒रुद्भ्य॒ इति॑ म॒रुत्-भ्यः॒ । गृ॒ह॒मे॒धिभ्य॒ इति॑ गृहमे॒धि-भ्यः॒ । सर्वा॑साम् । दु॒ग्धे । सा॒यम् । च॒रुम् । पू॒र्णा । द॒र्वि॒ । परेति॑ । प॒त॒ । सुपू॒र्णेति॒ सु - पू॒र्णा॒ । पुनः॑ । एति॑ । प॒त॒ ॥ व॒स्ना । इ॒व॒ । वीति॑ । क्री॒णा॒व॒है॒ । इष᳚म् । ऊर्ज᳚म् । श॒त॒क्र॒तो॒ इति॑ शत-क्र॒तो॒ ॥ दे॒हि । मे॒ । ददा॑मि । ते॒ । नीति॑ । मे॒ । धे॒हि॒ । नीति॑ । ते॒ । द॒धे॒ ॥ नि॒हार॒मिति॑ नि-हार᳚म् । इत् । नीति॑ । मे॒ । ह॒र॒ । नि॒हार॒मिति॑ नि-हार᳚म् । \textbf{  5} \newline
                  \newline
                                \textbf{ TS 1.8.4.2} \newline
                  नीति॑ । ह॒रा॒मि॒ । ते॒ ॥ म॒रुद्भ्य॒ इति॑ म॒रुत् - भ्यः॒ । क्री॒डिभ्य॒ इति॑ क्री॒डि-भ्यः॒ । पु॒रो॒डाश᳚म् । स॒प्तक॑पाल॒मिति॑ स॒प्त-क॒पा॒ल॒म् । निरिति॑ । व॒प॒ति॒ । सा॒कम् । सूर्ये॑ण । उ॒द्य॒तेत्यु॑त् - य॒ता । आ॒ग्ने॒यम् । अ॒ष्टाक॑पाल॒मित्य॒ष्टा-क॒पा॒ल॒म् । निरिति॑ । व॒प॒ति॒ । सौ॒म्यम् । च॒रुम् । सा॒वि॒त्रम् । द्वाद॑शकपाल॒मिति॒ द्वाद॑श-क॒पा॒ल॒म् । सा॒र॒स्व॒तम् । च॒रुम् । पौ॒ष्णम् । च॒रुम् । ऐ॒न्द्रा॒ग्नमित्यै᳚न्द्र - अ॒ग्नम् । एका॑दशकपाल॒मित्येका॑दश - क॒पा॒ल॒म् । ऐ॒न्द्रम् । च॒रुम् । वै॒श्व॒क॒र्म॒णमिति॑ वैश्व - क॒म॒र्णम् ॥ एक॑कपाल॒मित्येक॑-क॒पा॒ल॒म् । 6(30) \textbf{  6 } \newline
                  \newline
                      (ह॒रा॒ नि॒हारं॑-त्रिꣳ॒॒शच्च॑)  \textbf{(A4)} \newline \newline
                                \textbf{ TS 1.8.5.1} \newline
                  सोमा॑य । पि॒तृ॒मत॒ इति॑ पितृ - मते᳚ । पु॒रो॒डाश᳚म् । षट्क॑पाल॒मिति॒ षट् - क॒पा॒ल॒म् । निरिति॑ । व॒प॒ति॒ । पि॒तृभ्य॒ इति॑ पि॒तृ - भ्यः॒ । ब॒र्.॒हि॒षद्भ्य॒ इति॑ बर्.हि॒षद्-भ्यः॒ । धा॒नाः । पि॒तृभ्य॒ इति॑ पि॒तृ-भ्यः॒ । अ॒ग्नि॒ष्वा॒त्तेभ्य॒ इत्य॑ग्नि-स्वा॒त्तेभ्यः॑ । अ॒भि॒वा॒न्या॑या॒ इत्य॑भि-वा॒न्या॑यै । दु॒ग्धे । म॒न्थम् । ए॒तत् । ते॒ । त॒त॒ । ये । च॒ । त्वाम् । अन्विति॑ । ए॒तत् । ते॒ । पि॒ता॒म॒ह॒ । प्र॒पि॒ता॒म॒हेति॑ प्र - पि॒ता॒म॒ह॒ । ये । च॒ । त्वाम् । अन्विति॑ । अत्र॑ । पि॒त॒रः॒ । य॒था॒भा॒गमिति॑ यथा - भा॒गम् । म॒न्द॒द्ध्व॒म् । सु॒स॒दृंश॒मिति॑ सु - स॒दृंश᳚म् । त्वा॒ । व॒यम् । मघ॑व॒न्निति॒ मघ॑-व॒न्न् । म॒न्दि॒षी॒महि॑ ॥ प्रेति॑ । नू॒नम् । पू॒र्णव॑न्धुर॒ इति॑ पू॒र्ण - व॒न्धु॒रः॒ । स्तु॒तः । या॒सि॒ । वशान्॑ । अनु॑ ॥ योजा᳚ । नु । इ॒न्द्र॒ । ते॒ । हरी॒ इति॑ ॥ \textbf{  7} \newline
                  \newline
                                \textbf{ TS 1.8.5.2} \newline
                  अक्षन्न्॑ । अमी॑मदन्त । हि । अवेति॑ । प्रि॒याः । अ॒धू॒ष॒त॒ ॥ अस्तो॑षत । स्वभा॑नव॒ इति॒ स्व-भा॒न॒वः॒ । विप्राः᳚ । नवि॑ष्ठया । म॒ती ॥ योजा᳚ । नु । इ॒न्द्र॒ । ते॒ । हरी॒ इति॑ ॥ अक्षन्न्॑ । पि॒तरः॑ । अमी॑मदन्त । पि॒तरः॑ ॥ अती॑तृपन्त । पि॒तरः॑ । अमी॑मृजन्त । पि॒तरः॑ । परेति॑ । इ॒त॒ । पि॒त॒रः॒ । सो॒म्याः॒ । ग॒भीं॒रैः । प॒थिभि॒रिति॑ प॒थि-भिः॒ । पू॒र्व्यैः ॥ अथ॑ । पि॒तॄन् । सु॒वि॒दत्रा॒निति॑ सु - वि॒दत्रान्॑ । अपीति॑ । इ॒त॒ । य॒मेन॑ । ये । स॒ध॒माद॒मिति॑ सध - माद᳚म् । मद॑न्ति ॥ मनः॑ । नु । एति॑ । हु॒वा॒म॒हे॒ । ना॒रा॒शꣳ॒॒सेन॑ । स्तोमे॑न । पि॒तृ॒णाम् । च॒ । मन्म॑भि॒रिति॒ मन्म॑ - भिः॒ ॥ एति॑ । \textbf{  8} \newline
                  \newline
                                \textbf{ TS 1.8.5.3} \newline
                  न॒ । ए॒तु॒ । मनः॑ । पुनः॑ । क्रत्वे᳚ । दक्षा॑य । जी॒वसे᳚ ॥ ज्योक् । च॒ । सूर्य᳚म् । दृ॒शे ॥ पुनः॑ । नः॒ । पि॒तरः॑ । मनः॑ । ददा॑तु । दैव्यः॑ । जनः॑ ॥ जी॒वम् । व्रात᳚म् । स॒चे॒म॒हि॒ ॥ यत् । अ॒न्तरि॑क्षम् । पृ॒थि॒वीम् । उ॒त । द्याम् । यत् । मा॒तर᳚म् । पि॒तर᳚म् । वा॒ । जि॒हिꣳ॒॒सि॒म ॥ अ॒ग्निः । मा॒ । तस्मा᳚त् । एन॑सः । गार्.ह॑पत्य॒ इति॒ गार्.ह॑-प॒त्यः॒ । प्रेति॑ । मु॒ञ्च॒तु॒ । दु॒रि॒तेति॑ दुः - इ॒ता । यानि॑ । च॒कृ॒म । क॒रोतु॑ । माम् । अ॒ने॒नस᳚म् ॥ \textbf{  9} \newline
                  \newline
                       (हरी॒-मन्म॑भि॒रा-चतु॑श्चत्वारिꣳशच्च)  \textbf{(A5)} \newline \newline
                                \textbf{ TS 1.8.6.1} \newline
                  प्र॒ति॒पू॒रु॒षमिति॑ प्रति-पू॒रु॒षम् । एक॑कपाला॒नित्येक॑-क॒पा॒ला॒न् । निरिति॑ । व॒प॒ति॒ । एक᳚म् । अति॑रिक्त॒मित्यति॑-रि॒क्त॒म् । याव॑न्तः । गृ॒ह्याः᳚ । स्मः । तेभ्यः॑ । कम् । अ॒क॒र॒म् । प॒शू॒नाम् । शर्म॑ । अ॒सि॒ । शर्म॑ । यज॑मानस्य । शर्म॑ । मे॒ । य॒च्छ॒ । एकः॑ । ए॒व । रु॒द्रः । न । द्वि॒तीया॑य । त॒स्थे॒ । आ॒खुः । ते॒ । रु॒द्र॒ । प॒शुः । तम् । जु॒ष॒स्व॒ । ए॒षः । ते॒ । रु॒द्र॒ । भा॒गः । स॒ह । स्वस्रा᳚ । अबिं॑कया । तम् । जु॒ष॒स्व॒ । भे॒ष॒जम् । गवे᳚ । अश्वा॑य । पुरु॑षाय । भे॒ष॒जम् । अथो॒ इति॑ । अ॒स्मभ्य॒मित्य॒स्म-भ्य॒म् । भे॒ष॒जम् । सुभे॑षज॒मिति॒ सु - भे॒ष॒ज॒म् । \textbf{  10} \newline
                  \newline
                                \textbf{ TS 1.8.6.2} \newline
                  यथा᳚ । अस॑ति ॥ सु॒गमिति॑ सु-गम् । मे॒षाय॑ । मे॒ष्यै᳚ । अवेति॑ । अ॒बं॒ । रु॒द्रम् । अ॒दि॒म॒हि॒ । अवेति॑ । दे॒वम् । त्र्य॑बंक॒मिति॒ त्रि-अ॒बं॒क॒म् ॥ यथा᳚ । नः॒ । श्रेय॑सः । कर॑त् । यथा᳚ । नः॒ । वस्य॑सः । कर॑त् । यथा᳚ । नः॒ । प॒शु॒मत॒ इति॑ पशु - मतः॑ । कर॑त् । यथा᳚ । नः॒ । व्य॒व॒सा॒यया॒दिति॑ वि - अ॒व॒सा॒यया᳚त् ॥ त्र्य॑बंक॒मिति॒ त्रि - अ॒बं॒क॒म् । य॒जा॒म॒हे॒ । सु॒ग॒न्धिमिति॑ सु-ग॒न्धिम् । पु॒ष्टि॒वद्‌र्ध॑न॒मिति॑ पुष्टि-वद्‌र्ध॑नम् ॥ उ॒र्वा॒रु॒कम् । इ॒व॒ । बन्ध॑नात् । मृ॒त्योः । मु॒क्षी॒य॒ । मा । अ॒मृता᳚त् ॥ ए॒षः । ते॒ । रु॒द्र॒ । भा॒गः । तम् । जु॒ष॒स्व॒ । तेन॑ । अ॒व॒सेन॑ । प॒रः । मूज॑वत॒ इति॒ मूज॑ - व॒तः॒ । अतीति॑ । इ॒हि॒ ( ) । अव॑ततध॒न्वेत्यव॑तत - ध॒न्वा॒ । पिना॑कहस्त॒ इति॒ पिना॑क - ह॒स्तः॒ । कृत्ति॑वासा॒ इति॒ कृत्ति॑ - वा॒साः॒ ॥ \textbf{  11 } \newline
                  \newline
                      (सुभे॑षज-मिहि॒ त्रीणि॑ च)  \textbf{(A6)} \newline \newline
                                \textbf{ TS 1.8.7.1} \newline
                  ऐ॒न्द्रा॒ग्नमित्यै᳚न्द्र - अ॒ग्नम् । द्वाद॑शकपाल॒मिति॒ द्वाद॑श - क॒पा॒ल॒म् । वै॒श्व॒दे॒वमिति॑ वैश्व-दे॒वम् । च॒रुम् । इन्द्रा॑य । शुना॒सीरा॑य । पु॒रो॒डाश᳚म् । द्वाद॑शकपाल॒मिति॒ द्वाद॑श - क॒पा॒ल॒म् । वा॒य॒व्य᳚म् । पयः॑ । सौ॒र्यम् । एक॑कपाल॒मित्येक॑ - क॒पा॒ल॒म् । द्वा॒द॒श॒ग॒वमिति॑ द्वादश - ग॒वम् । सीर᳚म् । दक्षि॑णा । आ॒ग्ने॒यम् । अ॒ष्टाक॑पाल॒मित्य॒ष्टा - क॒पा॒ल॒म् । निरिति॑ । व॒प॒ति॒ । रौ॒द्रम् । गा॒वी॒धु॒कम् । च॒रुम् । ऐ॒न्द्रम् । दधि॑ । वा॒रु॒णम् । य॒व॒मय॒मिति॑ यव-मय᳚म् । च॒रुम् । व॒हिनी᳚ । धे॒नुः । दक्षि॑णा । ये । दे॒वाः । पु॒रः॒ सद॒ इति॑ पुरः - सदः॑ । अ॒ग्निने᳚त्रा॒ इत्य॒ग्नि - ने॒त्राः॒ । द॒क्षि॒ण॒सद॒ इति॑ दक्षिण - सदः॑ । य॒मने᳚त्रा॒ इति॑ य॒म - ने॒त्राः॒ । प॒श्चा॒थ्सद॒ इति॑ पश्चात् - सदः॑ । स॒वि॒तृने᳚त्रा॒ इति॑ सवि॒तृ - ने॒त्राः॒ । उ॒त्त॒र॒सद॒ इत्यु॑त्तर - सदः॑ । वरु॑णनेत्रा॒ इति॒ वरु॑ण - ने॒त्राः॒ । उ॒प॒रि॒षद॒ इत्यु॑परि - सदः॑ । बृह॒स्पति॑नेत्रा॒ इति॒ बृह॒स्पति॑ - ने॒त्राः॒ । र॒क्षो॒हण॒ इति॑ रक्षः - हनः॑ । ते । नः॒ । पा॒न्तु॒ । ते । नः॒ । अ॒व॒न्तु॒ । तेभ्यः॑ । \textbf{  12} \newline
                  \newline
                                \textbf{ TS 1.8.7.2} \newline
                  नमः॑ । तेभ्यः॑ । स्वाहा᳚ । समू॑ढ॒मिति॒ सं - ऊ॒ढ॒म् । रक्षः॑ । संद॑ग्ध॒मिति॒ सं-द॒ग्ध॒म् । रक्षः॑ । इ॒दम् । अ॒हम् । रक्षः॑ । अ॒भि । समिति॑ । द॒हा॒मि॒ । अ॒ग्नये᳚ । र॒क्षो॒घ्न इति॑ रक्षः - घ्ने । स्वाहा᳚ । य॒माय॑ । स॒वि॒त्रे । वरु॑णाय । बृह॒स्पत॑ये । दुव॑स्वते । र॒क्षो॒घ्न इति॑ रक्षः-घ्ने । स्वाहा᳚ । प्र॒ष्टि॒वा॒हीति॑ प्रष्टि-वा॒ही । रथः॑ । दक्षि॑णा । दे॒वस्य॑ । त्वा॒ । स॒वि॒तुः । प्र॒स॒व इति॑ प्र - स॒वे । अ॒श्विनोः᳚ । बा॒हुभ्या॒मिति॑ बा॒हु-भ्या॒म् । पू॒ष्णः । हस्ता᳚भ्याम् । रक्ष॑सः । व॒धम् । जु॒हो॒मि॒ । ह॒तम् । रक्षः॑ । अव॑धिष्म । रक्षः॑ । यत् । वस्ते᳚ । तत् । दक्षि॑णा ॥ \textbf{  13 } \newline
                  \newline
                      (तेभ्यः॒-पञ्च॑चत्वारिꣳशच्च)  \textbf{(A7)} \newline \newline
                                \textbf{ TS 1.8.8.1} \newline
                  धा॒त्रे । पु॒रो॒डाश᳚म् । द्वाद॑शकपाल॒मिति॒ द्वाद॑श - क॒पा॒ल॒म् । निरिति॑ । व॒प॒ति॒ । अनु॑मत्या॒ इत्यनु॑ - म॒त्यै॒ । च॒रुम् । रा॒कायै᳚ । च॒रुम् । सि॒नी॒वा॒ल्यै । च॒रुम् । कु॒ह्वै᳚ । च॒रुम् । मि॒थु॒नौ । गावौ᳚ । दक्षि॑णा । आ॒ग्ना॒वै॒ष्ण॒वमित्या᳚ग्ना -वै॒ष्ण॒वम् । एका॑दशकपाल॒मित्येका॑दश - क॒पा॒ल॒म् । निरिति॑ । व॒प॒ति॒ । ऐ॒न्द्रा॒वै॒ष्ण॒वमित्यै᳚न्द्रा - वै॒ष्ण॒वम् । एका॑दशकपाल॒मित्येका॑दश - क॒पा॒ल॒म् । वै॒ष्ण॒वम् । त्रि॒क॒पा॒लमिति॑ त्रि - क॒पा॒लम् । वा॒म॒नः । व॒ही । दक्षि॑णा । अ॒ग्नी॒षो॒मीय॒मित्य॑ग्नी - सो॒मीय᳚म् । एका॑दशकपाल॒मित्येका॑दश - क॒पा॒ल॒म् । निरिति॑ । व॒प॒ति॒ । इ॒न्द्रा॒सो॒मीय॒मिती᳚न्द्रा - सो॒मीय᳚म् । एका॑दशकपाल॒मित्येका॑दश - क॒पा॒ल॒म् । सौ॒म्यम् । च॒रुम् । ब॒भ्रुः । दक्षि॑णा । सो॒मा॒पौ॒ष्णमिति॑ सोमा - पौ॒ष्णम् । च॒रुम् । निरिति॑ । व॒प॒ति॒ । ऐ॒न्द्रा॒पौ॒ष्णमित्यै᳚न्द्रा - पौ॒ष्णम् । च॒रुम् । पौ॒ष्णम् । च॒रुम् । श्या॒मः । दक्षि॑णा । वै॒श्वा॒न॒रम् । द्वाद॑शकपाल॒मिति॒ द्वाद॑श-क॒पा॒ल॒म् । निरिति॑ ( ) । व॒प॒ति॒ । हिर॑ण्यम् । दक्षि॑णा । वा॒रु॒णम् । य॒व॒मय॒मिति॑ यव - मय᳚म् । च॒रुम् । अश्वः॑ । दक्षि॑णा ॥ \textbf{  14} \newline
                  \newline
                      (वै॒श्वा॒न॒रं द्वाद॑शकपालं॒ नि॒-रष्टौ च॑)  \textbf{(A8)} \newline \newline
                                \textbf{ TS 1.8.9.1} \newline
                  बा॒र्.॒ह॒स्प॒त्यम् । च॒रुम् । निरिति॑ । व॒प॒ति॒ । ब्र॒ह्मणः॑ । गृ॒हे । शि॒ति॒पृ॒ष्ठ इति॑ शिति - पृ॒ष्ठः । दक्षि॑णा । ऐ॒न्द्रम् । एका॑दशकपाल॒मित्येका॑दश-क॒पा॒ल॒म् । रा॒ज॒न्य॑स्य । गृ॒हे । ऋ॒ष॒भः । दक्षि॑णा । आ॒दि॒त्यम् । च॒रुम् । महि॑ष्यै । गृ॒हे । धे॒नुः । दक्षि॑णा । नै॒र॒.ऋ॒तमिति॑ नैः-ऋ॒तम् । च॒रुम् । प॒रि॒वृ॒क्त्या॑ इति॑ परि-वृ॒क्त्यै᳚ । गृ॒हे । कृ॒ष्णाना᳚म् । व्री॒ही॒णाम् । न॒खनि॑र्भिन्न॒मिति॑ न॒ख-नि॒र्भि॒न्न॒म् । कृ॒ष्णा । कू॒टा । दक्षि॑णा । आ॒ग्ने॒यम् । अ॒ष्टाक॑पाल॒मित्य॒ष्टा - क॒पा॒ल॒म् । से॒ना॒न्य॑ इति॑ सेना - न्यः॑ । गृ॒हे । हिर॑ण्यम् । दक्षि॑णा । वा॒रु॒णम् । दश॑कपाल॒मिति॒ दश॑-क॒पा॒ल॒म् । सू॒तस्य॑ । गृ॒हे । म॒हानि॑रष्ट॒ इति॑ म॒हा-नि॒र॒ष्टः॒ । दक्षि॑णा । मा॒रु॒तम् । स॒प्तक॑पाल॒मिति॑ स॒प्त - क॒पा॒ल॒म् । ग्रा॒म॒ण्य॑ इति॑ ग्राम - न्यः॑ । गृ॒हे । पृश्निः॑ । दक्षि॑णा । सा॒वि॒त्रम् । द्वाद॑शकपाल॒मिति॒ द्वाद॑श - क॒पा॒ल॒म् । \textbf{  15} \newline
                  \newline
                                \textbf{ TS 1.8.9.2} \newline
                  क्ष॒त्तुः । गृ॒हे । उ॒प॒द्ध्व॒स्त इत्यु॑प - ध्व॒स्तः । दक्षि॑णा । आ॒श्वि॒नम् । द्वि॒क॒पा॒लमिति॑ द्वि - क॒पा॒लम् । स॒ङ्ग्र॒ही॒तुरिति॑ सं-ग्र॒ही॒तुः । गृ॒हे । स॒वा॒त्या॑विति॑ स - वा॒त्यौ᳚ । दक्षि॑णा । पौ॒ष्णम् । च॒रुम् । भा॒ग॒दु॒घस्येति॑ भाग - दु॒घस्य॑ । गृ॒हे । श्या॒मः । दक्षि॑णा । रौ॒द्रम् । गा॒वी॒धु॒कम् । च॒रुम् । अ॒क्षा॒वा॒पस्येत्य॑क्ष - आ॒वा॒पस्य॑ । गृ॒हे । श॒बलः॑ । उद्वा॑र॒ इत्युत् - वा॒रः॒ । दक्षि॑णा । इन्द्रा॑य । सु॒त्रांण॒ इति॑ सु - त्रांणे᳚ । पु॒रो॒डाश᳚म् । एका॑दशकपाल॒मित्येका॑दश - क॒पा॒ल॒म् । प्रति॑ । निरिति॑ । व॒प॒ति॒ । इन्द्रा॑य । अꣳ॒॒हो॒मुच॒ इत्यꣳ॑हः-मुचे᳚ । अ॒यम् । नः॒ । राजा᳚ । वृ॒त्र॒हेति॑ वृत्र-हा । राजा᳚ । भू॒त्वा । वृ॒त्रम् । व॒द्ध्या॒त् । मै॒त्रा॒बा॒र्.॒ह॒स्प॒त्यमिति॑ मैत्रा-बा॒र्.॒ह॒स्प॒त्यम् । भ॒व॒ति॒ । श्वे॒तायै᳚ । श्वे॒तव॑थ्साया॒ इति॑ श्वे॒त-व॒थ्सा॒यै॒ । दु॒ग्धे । स्व॒य॒मूं॒र्त इति॑ स्वयम्-मू॒र्ते । स्व॒य॒मं॒थि॒त इति॑ स्वयं - म॒थि॒ते । आज्ये᳚ । आश्व॑त्थे । \textbf{  16} \newline
                  \newline
                                \textbf{ TS 1.8.9.3} \newline
                  पात्रे᳚ । चतुः॑स्रक्ता॒विति॒ चतुः॑ - स्र॒क्तौ॒ । स्व॒य॒म॒व॒प॒न्नाया॒ इति॑ स्वयं-अ॒व॒प॒न्नायै᳚ । शाखा॑यै । क॒र्णान् । च॒ । अक॑र्णान् । च॒ । त॒ण्डु॒लान् । वीति॑ । चि॒नु॒या॒त् । ये । क॒र्णाः । सः । पय॑सि । बा॒र्.॒ह॒स्प॒त्यः । ये । अक॑र्णाः । सः । आज्ये᳚ । मै॒त्रः । स्व॒य॒कृं॒तेति॑ स्वयम् - कृ॒ता । वेदिः॑ । भ॒व॒ति॒ । स्व॒य॒दिं॒नमिति॑ स्वयं - दि॒नम् । ब॒र्॒.हिः । स्व॒य॒कृं॒त इति॑ स्वयं - कृ॒तः । इ॒द्ध्मः । सा । ए॒व । श्वे॒ता । श्वे॒तव॒थ्सेति॑ श्वे॒त - व॒थ्सा॒ । दक्षि॑णा ॥ \textbf{  17} \newline
                  \newline
                      (सावि॒त्रं द्वाद॑शकपाल॒-माश्व॑त्थे॒ त्रय॑स्त्रिꣳशच्च)  \textbf{(A9)} \newline \newline
                                \textbf{ TS 1.8.10.1} \newline
                  अ॒ग्नये᳚ । गृ॒हप॑तय॒ इति॑ गृ॒ह - प॒त॒ये॒ । पु॒रो॒डाश᳚म् । अ॒ष्टाक॑पाल॒मित्य॒ष्टा - क॒पा॒ल॒म् । निरिति॑ । व॒प॒ति॒ । कृ॒ष्णाना᳚म् । व्री॒ही॒णाम् । सोमा॑य । वन॒स्पत॑ये । श्या॒मा॒कम् । च॒रुम् । स॒वि॒त्रे । स॒त्यप्र॑सवा॒येति॑ स॒त्य - प्र॒स॒वा॒य॒ । पु॒रो॒डाश᳚म् । द्वाद॑शकपाल॒मिति॒ द्वाद॑श - क॒पा॒ल॒म् । आ॒शू॒नाम् । व्री॒ही॒णाम् । रु॒द्राय॑ । प॒शु॒पत॑य॒ इति॑ पशु - पत॑ये । गा॒वी॒धु॒कम् । च॒रुम् । बृह॒स्पत॑ये । वा॒चः । पत॑ये । नै॒वा॒रम् । च॒रुम् । इन्द्रा॑य । ज्ये॒ष्ठाय॑ । पु॒रो॒डाश᳚म् । एका॑दशकपाल॒मित्येका॑दश - क॒पा॒ल॒म् । म॒हाव्री॑हीणा॒मिति॑ म॒हा-व्री॒ही॒णा॒म् । मि॒त्राय॑ । स॒त्याय॑ । आ॒बांना᳚म् । च॒रुम् । वरु॑णाय । धर्म॑पतय॒ इति॒ धर्म - प॒त॒ये॒ । य॒व॒मय॒मिति॑ यव - मय᳚म् । च॒रुम् । स॒वि॒ता । त्वा॒ । प्र॒स॒वाना॒मिति॑ प्र - स॒वाना᳚म् । सु॒व॒ता॒म् । अ॒ग्निः । गृ॒हप॑तीना॒मिति॑ गृ॒ह - प॒ती॒ना॒म् । सोमः॑ । वन॒स्पती॑नाम् । रु॒द्रः । प॒शू॒नाम् । \textbf{  18} \newline
                  \newline
                                \textbf{ TS 1.8.10.2} \newline
                  बृह॒स्पतिः॑ । वा॒चम् । इन्द्रः॑ । ज्ये॒ष्ठाना᳚म् । मि॒त्रः । स॒त्याना᳚म् । वरु॑णः । धर्म॑पतीना॒मिति॒ धर्म॑ - प॒ती॒ना॒म् । ये । दे॒वाः॒ । दे॒व॒सुव॒ इति॑ देव - सुवः॑ । स्थ । ते । इ॒मम् । आ॒मु॒ष्या॒य॒णम् । अ॒न॒मि॒त्राय॑ । सु॒व॒द्ध्व॒म् । म॒ह॒ते । क्ष॒त्राय॑ । म॒ह॒ते । आधि॑पत्या॒येत्याधि॑-प॒त्या॒य॒ । म॒ह॒ते । जान॑राज्या॒येति॒ जान॑-रा॒ज्या॒य॒ । ए॒षः । वः॒ । भ॒र॒ताः॒ । राजा᳚ । सोमः॑ । अ॒स्माक᳚म् । ब्रा॒ह्म॒णाना᳚म् । राजा᳚ । प्रतीति॑ । त्यत् । नाम॑ । रा॒ज्यम् । अ॒धा॒यि॒ । स्वाम् । त॒नुव᳚म् । वरु॑णः । अ॒शि॒श्रे॒त् । शुचेः᳚ । मि॒त्रस्य॑ । व्रत्याः᳚ । अ॒भू॒म॒ । अम॑न्महि । म॒ह॒तः । ऋ॒तस्य॑ । नाम॑ । सर्वे᳚ । व्राताः᳚ ( ) । वरु॑णस्य । अ॒भू॒व॒न्न् । वीति॑ । मि॒त्रः । एवैः᳚ । अरा॑तिम् । अ॒ता॒री॒त् । असू॑षुदन्त । य॒ज्ञियाः᳚ । ऋ॒तेन॑ । वीति॑ । उ॒ । त्रि॒तः । ज॒रि॒माण᳚म् । नः॒ । आ॒न॒ट् । विष्णोः᳚ । क्रमः॑ । अ॒सि॒ । विष्णोः᳚ । क्रा॒न्तम् । अ॒सि॒ । विष्णोः᳚ । विक्रा᳚न्त॒मिति॒ वि - क्रा॒न्त॒म् । अ॒सि॒ ॥(प॒शू॒नां-ॅव्राताः॒-पञ्च॑विꣳशतिश्च) (आ10 ) \textbf{  19} \newline
                  \newline
                      (प॒शू॒नां-ॅव्राताः॒-पञ्च॑विꣳशतिश्च)  \textbf{(A10)} \newline \newline
                                \textbf{ TS 1.8.11.1} \newline
                  अ॒र्थेत॒ इत्य॑र्थ-इतः॑ । स्थ॒ । अ॒पाम् । पतिः॑ । अ॒सि॒ । वृषा᳚ । अ॒सि॒ । ऊ॒र्मिः । वृ॒ष॒से॒न इति॑ वृष-से॒नः । अ॒सि॒ । व्र॒ज॒क्षित॒ इति॑ व्रज-क्षितः॑ । स्थ॒ । म॒रुता᳚म् । ओजः॑ । स्थ॒ । सूर्य॑वर्चस॒ इति॒ सूर्य॑ - व॒र्च॒सः॒ । स्थ॒ । सूर्य॑त्वचस॒ इति॒ सूर्य॑ - त्व॒च॒सः॒ । स्थ॒ । मान्दाः᳚ । स्थ॒ । वाशाः᳚ । स्थ॒ । शक्व॑रीः । स्थ॒ । वि॒श्व॒भृत॒ इति॑ विश्व - भृतः॑ । स्थ॒ । ज॒न॒भृत॒ इति॑ जन- भृतः॑ । स्थ॒ । अ॒ग्नेः । ते॒ज॒स्याः᳚ । स्थ॒ । अ॒पाम् । ओष॑धीनाम् । रसः॑ । स्थ॒ । अ॒पः । दे॒वीः । मधु॑मती॒रिति॒ मधु॑ - म॒तीः॒ । अ॒गृ॒ह्ण॒न्न् । ऊर्ज॑स्वतीः । रा॒ज॒सूया॒येति॑ राज - सूया॑य । चिता॑नाः ॥ याभिः॑ । मि॒त्रावरु॑णा॒विति॑ मि॒त्रा-वरु॑णौ । अ॒भ्यषि॑ञ्च॒न्नित्य॑भि-असि॑ञ्चन्न् । याभिः॑ । इन्द्र᳚म् । अन॑यन्न् । अतीति॑ ( ) । अरा॑तीः ॥ रा॒ष्ट्र॒दा इति॑ राष्ट्र-दाः । स्थ॒ । रा॒ष्ट्रम् । द॒त्त॒ । स्वाहा᳚ । रा॒ष्ट्र॒दा इति॑ राष्ट्र - दाः । स्थ॒ । रा॒ष्ट्रम् । अ॒मुष्मै᳚ । द॒त्त॒ ॥ \textbf{  20 } \newline
                  \newline
                      (अत्ये-का॑दश च)  \textbf{(A11)} \newline \newline
                                \textbf{ TS 1.8.12.1} \newline
                  देवीः᳚ । आ॒पः॒ । समिति॑ । मधु॑मती॒रिति॒ मधु॑- म॒तीः॒ । मधु॑मतीभि॒रिति॒ मधु॑-म॒ती॒भिः॒ । सृ॒ज्य॒द्ध्व॒म् । महि॑ । वर्चः॑ । क्ष॒त्रिया॑य । व॒न्वा॒नाः । अना॑धृष्टा॒ इत्यना᳚-धृ॒ष्टाः॒ । सी॒द॒त॒ । ऊर्ज॑स्वतीः । महि॑ । वर्चः॑ । क्ष॒त्रिया॑य । दध॑तीः । अनि॑भृष्ट॒मित्यनि॑-भृ॒ष्ट॒म् । अ॒सि॒ । वा॒चः । बन्धुः॑ । त॒पो॒जा इति॑ तपः - जाः । सोम॑स्य । दा॒त्रम् । अ॒सि॒ । शु॒क्राः । वः॒ । शु॒क्रेण॑ । उदिति॑ । पु॒ना॒मि॒ । च॒न्द्राः । च॒न्द्रेण॑ । अ॒मृताः᳚ । अ॒मृते॑न । स्वाहा᳚ । रा॒ज॒सूया॒येति॑ राज - सूया॑य । चिता॑नाः ॥ स॒ध॒माद॒ इति॑ सध-मादः॑ । द्यु॒म्निनीः᳚ । ऊर्जः॑ । ए॒ताः । अनि॑भृष्टा॒ इत्यनि॑-भृ॒ष्टाः॒ । अ॒प॒स्युवः॑ । वसा॑नः ॥ प॒स्त्या॑सु । च॒क्रे॒ । वरु॑णः । स॒धस्थ॒मिति॑ स॒ध - स्थ॒म् । अ॒पाम् । शिशुः॑ । \textbf{  21} \newline
                  \newline
                                \textbf{ TS 1.8.12.2} \newline
                  मा॒तृत॑मा॒स्विति॑ मा॒तृ - त॒मा॒सु॒ । अ॒न्तः ॥ क्ष॒त्रस्य॑ । उल्ब᳚म् । अ॒सि॒ । क्ष॒त्रस्य॑ । योनिः॑ । अ॒सि॒ । आवि॑न्नः । अ॒ग्निः । गृ॒हप॑ति॒रिति॑ गृ॒ह-प॒तिः॒ । आवि॑न्नः । इन्द्रः॑ । वृ॒द्धश्र॑वा॒ इति॑ वृ॒द्ध-श्र॒वाः॒ । आवि॑न्नः । पू॒षा । वि॒श्ववे॑दा॒ इति॑ वि॒श्व - वे॒दाः॒ । आवि॑न्नौ । मि॒त्रावरु॑णा॒विति॑ मि॒त्रा - वरु॑णौ । ऋ॒ता॒वृधा॒वित्यृ॑त-वृधौ᳚ । आवि॑न्ने॒ इति॑ । द्यावा॑पृथि॒वी इति॒ द्यावा᳚ - पृ॒थि॒वी । धृ॒तव्र॑ते॒ इति॑ धृ॒त - व्र॒ते॒ । आवि॑न्ना । दे॒वी । अदि॑तिः । वि॒श्व॒रू॒पीति॑ विश्व-रू॒पी । आवि॑न्नः । अ॒यम् । अ॒सौ । आ॒मु॒ष्या॒य॒णः । अ॒स्याम् । वि॒शि । अ॒स्मिन्न् । रा॒ष्ट्रे । म॒ह॒ते । क्ष॒त्राय॑ । म॒ह॒ते । आधि॑पत्या॒येत्याधि॑-प॒त्या॒य॒ । म॒ह॒ते । जान॑राज्या॒येति॒ जान॑-रा॒ज्या॒य॒ । ए॒षः । वः॒ । भ॒र॒ताः॒ । राजा᳚ । सोमः॑ । अ॒स्माक᳚म् । ब्रा॒ह्म॒णाना᳚म् । राजा᳚ । इन्द्र॑स्य । \textbf{  22} \newline
                  \newline
                                \textbf{ TS 1.8.12.3} \newline
                  वज्रः॑ । अ॒सि॒ । वार्त्र॑घ्न॒ इति॒ वार्त्र॑ - घ्नः॒ । त्वया᳚ । अ॒यम् । वृ॒त्रम् । व॒द्ध्या॒त् । श॒त्रु॒बाध॑ना॒ इति॑ शत्रु-बाध॑नाः । स्थ॒ । पा॒त । मा॒ । प्र॒त्यञ्च᳚म् । पा॒त । मा॒ । ति॒र्यञ्च᳚म् । अ॒न्वञ्च᳚म् । मा॒ । पा॒त॒ । दि॒ग्भ्य इति॑ दिक्-भ्यः । मा॒ । पा॒त॒ । विश्वा᳚भ्यः । मा॒ । ना॒ष्ट्राभ्यः॑ । पा॒त॒ । हिर॑ण्यवर्णा॒विति॒ हिर॑ण्य - व॒र्णौ॒ । उ॒षसा᳚म् । वि॒रो॒क इति॑ वि - रो॒के । अयः॑ स्थूणा॒विययः॑ - स्थू॒णौ॒ । उदि॑ता॒वियुत् - इ॒तौ॒ । सूर्य॑स्य । एति॑ । रो॒ह॒त॒म् । व॒रु॒ण॒ । मि॒त्र॒ । गर्त᳚म् । ततः॑ । च॒क्षा॒था॒म् । अदि॑तिम् । दिति᳚म् । च॒ ॥ \textbf{  23} \newline
                  \newline
                      (शिशु॒-रिन्द्र॒स्यै-क॑चत्वारिꣳशच्च)  \textbf{(A12)} \newline \newline
                                \textbf{ TS 1.8.13.1} \newline
                  स॒मिध॒मिति॑ सं-इध᳚म् । एति॑ । ति॒ष्ठ॒ । गा॒य॒त्री । त्वा॒ । छन्द॑साम् । अ॒व॒तु॒ । त्रि॒वृदिति॑ त्रि-वृत् । स्तोमः॑ । र॒थ॒न्त॒रमिति॑ रथं-त॒रम् । साम॑ । अ॒ग्निः । दे॒वता᳚ । ब्रह्म॑ । द्रवि॑णम् । उ॒ग्राम् । एति॑ । ति॒ष्ठ॒ । त्रि॒ष्टुप् । त्वा॒ । छन्द॑साम् । अ॒व॒तु॒ । प॒ञ्च॒द॒श इति॑ पञ्च-द॒शः । स्तोमः॑ । बृ॒हत् । साम॑ । इन्द्रः॑ । दे॒वता᳚ । क्ष॒त्रम् । द्रवि॑णम् । वि॒राज॒मिति॑ वि-राज᳚म् । एति॑ । ति॒ष्ठ॒ । जग॑ती । त्वा॒ । छन्द॑साम् । अ॒व॒तु॒ । स॒प्त॒द॒श इति॑ सप्त-द॒शः । स्तोमः॑ । वै॒रू॒पम् । साम॑ । म॒रुतः॑ । दे॒वता᳚ । विट् । द्रवि॑णम् । उदी॑चीम् । एति॑ । ति॒ष्ठ॒ । अ॒नु॒ष्टुबित्य॑नु - स्तुप् । त्वा॒ । \textbf{  24} \newline
                  \newline
                                \textbf{ TS 1.8.13.2} \newline
                  छन्द॑साम् । अ॒व॒तु॒ । ए॒क॒विꣳ॒॒श इत्ये॑क-विꣳ॒॒शः । स्तोमः॑ । वै॒रा॒जम् । साम॑ । मि॒त्रावरु॑णा॒विति॑ मि॒त्रा - वरु॑णौ । दे॒वता᳚ । बल᳚म् । द्रवि॑णम् । ऊ॒द्‌र्ध्वाम् । एति॑ । ति॒ष्ठ॒ । प॒ङ्क्तिः । त्वा॒ । छन्द॑साम् । अ॒व॒तु॒ । त्रि॒ण॒व॒त्र॒य॒स्त्रिꣳ॒॒शाविति॑ त्रिणव - त्र॒य॒स्त्रिꣳ॒॒शौ । स्तोमौ᳚ । शा॒क्व॒र॒रै॒व॒ते इति॑ शाक्वर-रै॒व॒ते । साम॑नी॒ इति॑ । बृह॒स्पतिः॑ । दे॒वता᳚ । वर्चः॑ । द्रवि॑णम् । ई॒दृङ् । च॒ । अ॒न्या॒दृङ् । च॒ । ए॒ता॒दृङ् । च॒ । प्र॒ति॒दृङिति॑ प्रति - दृङ् । च॒ । मि॒तः । च॒ । संमि॑त॒ इति॒ सं - मि॒तः॒ । च॒ । सभ॑रा॒ इति॒ स - भ॒राः॒ ॥ शु॒क्रज्यो॑ति॒रिति॑ शु॒क्र - ज्यो॒तिः॒ । च॒ । चि॒त्रज्यो॑ति॒रिति॑ चि॒त्र - ज्यो॒तिः॒ । च॒ । स॒त्यज्यो॑ति॒रिति॑ स॒त्य - ज्यो॒तिः॒ । च॒ । ज्योति॑ष्मान् । च॒ । स॒त्यः । च॒ । ऋ॒त॒पा इत्यृ॑त - पाः । च॒ । \textbf{  25} \newline
                  \newline
                                \textbf{ TS 1.8.13.3} \newline
                  अत्यꣳ॑हा॒ इत्यति॑-अꣳ॒॒हाः॒ ॥ अ॒ग्नये᳚ । स्वाहा᳚ । सोमा॑य । स्वाहा᳚ । स॒वि॒त्रे । स्वाहा᳚ । सर॑स्वत्यै । स्वाहा᳚ । पू॒ष्णे । स्वाहा᳚ । बृह॒स्पत॑ये । स्वाहा᳚ । इन्द्रा॑य । स्वाहा᳚ । घोषा॑य । स्वाहा᳚ । श्लोका॑य । स्वाहा᳚ । अꣳशा॑य । स्वाहा᳚ । भगा॑य । स्वाहा᳚ । क्षेत्र॑स्य । पत॑ये । स्वाहा᳚ । पृ॒थि॒व्यै । स्वाहा᳚ । अ॒न्तरि॑क्षाय । स्वाहा᳚ । दि॒वे । स्वाहा᳚ । सूर्या॑य । स्वाहा᳚ । च॒न्द्रम॑से । स्वाहा᳚ । नक्ष॑त्रेभ्यः । स्वाहा᳚ । अ॒द्भ्य इत्य॑त्-भ्यः । स्वाहा᳚ । ओष॑धीभ्य॒ इत्योष॑धि-भ्यः॒ । स्वाहा᳚ । वन॒स्पति॑भ्य॒ इति॒ वन॒स्पति॑-भ्यः॒ । स्वाहा᳚ । च॒रा॒च॒रेभ्यः॑ । स्वाहा᳚ । प॒रि॒प्ल॒वेभ्य॒ इति॑ परि - प्ल॒वेभ्यः॑ । स्वाहा᳚ । स॒री॒सृ॒पेभ्यः॑ । स्वाहा᳚ ( ) ॥ \textbf{  26} \newline
                  \newline
                      (अ॒नु॒ष्टुप्त्व॑र्-त॒पाश्च॑ - सरीसृ॒पेभ्यः॒ स्वाहा᳚)  \textbf{(A13)} \newline \newline
                                \textbf{ TS 1.8.14.1} \newline
                  सोम॑स्य । त्विषिः॑ । अ॒सि॒ । तव॑ । इ॒व॒ । मे॒ । त्विषिः॑ । भू॒या॒त् । अ॒मृत᳚म् । अ॒सि॒ । मृ॒त्योः । मा॒ । पा॒हि॒ । दि॒द्योत् । मा॒ । पा॒हि॒ । अवे᳚ष्टा॒ इत्यव॑-इ॒ष्टाः॒ । द॒न्द॒शूकाः᳚ । निर॑स्त॒मिति॒ निः-अ॒स्त॒म् । नमु॑चेः । शिरः॑ ॥ सोमः॑ । राजा᳚ । वरु॑णः । दे॒वाः । ध॒र्म॒सुव॒ इति॑ धर्म -सुवः॑ । च॒ । ये ॥ ते । ते॒ । वाच᳚म् । सु॒व॒न्ता॒म् । ते । ते॒ । प्रा॒णमिति॑ प्र-अ॒नम् । सु॒व॒न्ता॒म् । ते । ते॒ । चक्षुः॑ । सु॒व॒न्ता॒म् । ते । ते॒ । श्रोत्र᳚म् । सु॒व॒न्ता॒म् । सोम॑स्य । त्वा॒ । द्यु॒म्नेन॑ । अ॒भीति॑ । सि॒ञ्चा॒मि॒ । अ॒ग्नेः । \textbf{  27} \newline
                  \newline
                                \textbf{ TS 1.8.14.2} \newline
                  तेज॑सा । सूर्य॑स्य । वर्च॑सा । इन्द्र॑स्य । इ॒न्द्रि॒येण॑ । मि॒त्रावरु॑णयो॒रिति॑ मि॒त्रा - वरु॑णयोः । वी॒र्ये॑ण । म॒रुता᳚म् । ओज॑सा । क्ष॒त्राणा᳚म् । क्ष॒त्रप॑ति॒रिति॑ क्ष॒त्र - प॒तिः॒ । अ॒सि॒ । अतीति॑ । दि॒वः । पा॒हि॒ । स॒माव॑वृत्र॒न्निति॑ सं - आव॑वृत्रन्न् । अ॒ध॒राक् । उदी॑चीः । अहि᳚म् । बु॒द्ध्निय᳚म् । अन्विति॑॑ । स॒ञ्चर॑न्ती॒रिति॑ सं-चर॑न्तीः । ताः । पर्व॑तस्य । वृ॒ष॒भस्य॑ । पृ॒ष्ठे । नावः॑ । च॒र॒न्ति॒ । स्व॒सिच॒ इति॑ स्व-सिचः॑ । इ॒या॒नाः ॥ रुद्र॑ । यत् । ते॒ । क्रयि॑ । पर᳚म् । नाम॑ । तस्मै᳚ । हु॒तम् । अ॒सि॒ । य॒मेष्ट॒मिति॑ य॒म - इ॒ष्ट॒म् । अ॒सि॒ ॥ प्रजा॑पत॒ इति॒ प्रजा᳚-प॒ते॒ । न । त्वत् । ए॒तानि॑ । अ॒न्यः । विश्वा᳚ । जा॒तानि॑ । परीति॑ । ता ( ) । ब॒भू॒व॒ ॥ यत्का॑मा॒ इति॒ यत् - का॒माः॒ । ते॒ । जु॒हु॒मः । तत् । नः॒ । अ॒स्तु॒ । व॒यम् । स्या॒म॒ । पत॑यः । र॒यी॒णाम् ॥ \textbf{  28 } \newline
                  \newline
                      (अ॒ग्ने-स्तै-का॑दश च)  \textbf{(A14)} \newline \newline
                                \textbf{ TS 1.8.15.1} \newline
                  इन्द्र॑स्य । वज्रः॑ । अ॒सि॒ । वार्त्र॑घ्न॒ इति॒ वार्त्र॑-घ्नः॒ । त्वया᳚ । अ॒यम् । वृ॒त्रम् । व॒द्ध्या॒त् । मि॒त्रावरु॑णयो॒रिति॑ मि॒त्रा - वरु॑णयोः । त्वा॒ । प्र॒शा॒स्त्रोरिति॑ प्र-शा॒स्त्रोः । प्र॒शिषेति॑ प्र -शिषा᳚ । यु॒न॒ज्मि॒ । य॒ज्ञ्स्य॑ । योगे॑न । विष्णोः᳚ । क्रमः॑ । अ॒सि॒ । विष्णोः᳚ । क्रा॒न्तम् । अ॒सि॒ । विष्णोः᳚ । विक्रा᳚न्त॒मिति॒ वि - क्रा॒न्त॒म् । अ॒सि॒ । म॒रुता᳚म् । प्र॒स॒व इति॑ प्र - स॒वे । जे॒ष॒म् । आ॒प्तम् । मनः॑ । समिति॑ । अ॒हम् । इ॒न्द्रि॒येण॑ । वी॒र्ये॑ण । प॒शू॒नाम् । म॒न्युः । अ॒सि॒ । तव॑ । इ॒व॒ । मे॒ । म॒न्युः । भू॒या॒त् । नमः॑ । मा॒त्रे । पृ॒थि॒व्यै । मा । अ॒हम् । मा॒तर᳚म् । पृ॒थि॒वीम् । हिꣳ॒॒सि॒ष॒म् । मा । \textbf{  29} \newline
                  \newline
                                \textbf{ TS 1.8.15.2} \newline
                  माम् । मा॒ता । पृ॒थि॒वी । हिꣳ॒॒सी॒त् । इय॑त् । अ॒सि॒ । आयुः॑ । अ॒सि॒ । आयुः॑ । मे॒ । धे॒हि॒ । ऊर्क् । अ॒सि॒ । ऊर्ज᳚म् । मे॒ । धे॒हि॒ । युङ् । अ॒सि॒ । वर्चः॑ । अ॒सि॒ । वर्चः॑ । मयि॑ । धे॒हि॒ । अ॒ग्नये᳚ । गृ॒हप॑तय॒ इति॑ गृ॒ह-प॒त॒ये॒ । स्वाहा᳚ । सोमा॑य । वन॒स्पत॑ये । स्वाहा᳚ । इन्द्र॑स्य । बला॑य । स्वाहा᳚ । म॒रुता᳚म् । ओज॑से । स्वाहा᳚ । हꣳ॒॒सः । शु॒चि॒षदिति॑ शुचि - सत् । वसुः॑ । अ॒न्त॒रि॒क्ष॒दित्य॑न्तरिक्ष - सत् । होता᳚ । वे॒दि॒षदिति॑ वेदि - सत् । अति॑थिः । दु॒रो॒ण॒सदिति॑ दुरोण - सत् ॥ नृ॒षदिति॑ नृ - सत् । व॒र॒सदिति॑ वर - सत् । ऋ॒त॒सदित्यृ॑त -सत् । व्यो॒म॒सदिति॑ व्योम - सत् । अ॒ब्जा इत्य॑प् - जाः । गो॒जा इति॑ गो - जाः । ऋ॒त॒जा इत्यृ॑त-जाः ( ) । अ॒द्रि॒जा इत्य॑द्रि - जाः । ऋ॒तं । बृ॒हत् ॥ \textbf{  30} \newline
                  \newline
                      (हिꣳ॒॒सि॒षं॒ मर्-त॒जा-स्त्रीणि॑ च)  \textbf{(A15)} \newline \newline
                                \textbf{ TS 1.8.16.1} \newline
                  मि॒त्रः । अ॒सि॒ । वरु॑णः । अ॒सि॒ । समिति॑ । अ॒हम् । विश्वैः᳚ । दे॒वैः । क्ष॒त्रस्य॑ । नाभिः॑ । अ॒सि॒ । क्ष॒त्रस्य॑ । योनिः॑ । अ॒सि॒ । स्यो॒नाम् । एति॑ । सी॒द॒ । सु॒षदा॒मिति॑ सु - सदा᳚म् । एति॑ । सी॒द॒ । मा । त्वा॒ । हिꣳ॒॒सी॒त् । मा । मा॒ । हिꣳ॒॒सी॒त् । नीति॑ । स॒सा॒द॒ । धृ॒तव्र॑त॒ इति॑ धृ॒त - व्र॒तः॒ । वरु॑णः । प॒स्त्या॑सु । एति॑ । साम्रा᳚ज्या॒येति॒ सां-रा॒ज्या॒य॒ । सु॒क्रतु॒रिति॑ सु- क्रतुः॑ । ब्रह्मा(3)न् । त्वम् । रा॒ज॒न्न् । ब्र॒ह्मा । अ॒सि॒ । स॒वि॒ता । अ॒सि॒ । स॒त्यस॑व॒ इति॑ स॒त्य - स॒वः॒ । ब्रह्मा(3)न् । त्वम् । रा॒ज॒न्न् । ब्र॒ह्मा । अ॒सि॒ । इन्द्रः॑ । अ॒सि॒ । स॒त्यौजा॒ इति॑ स॒त्य - ओ॒जाः॒ । \textbf{  31} \newline
                  \newline
                                \textbf{ TS 1.8.16.2} \newline
                  ब्रह्मा(3)न् । त्वम् । रा॒ज॒न्न् । ब्र॒ह्मा । अ॒सि॒ । मि॒त्रः । अ॒सि॒ । सु॒शेव॒ इति॑ सु - शेवः॑ । ब्रह्मा(3)न् । त्वम् । रा॒ज॒न्न् । ब्र॒ह्मा । अ॒सि॒ । वरु॑णः । अ॒सि॒ । स॒त्यध॒र्मेति॑ स॒त्य - ध॒र्मा॒ । इन्द्र॑स्य । वज्रः॑ । अ॒सि॒ । वार्त्र॑घ्न॒ इति॒ वार्त्र॑ - घ्नः॒ । तेन॑ । मे॒ । र॒द्ध्य॒ । दिशः॑ । अ॒भीति॑ । अ॒यम् । राजा᳚ । अ॒भू॒त् । सुश्लो॒काॅ(4) इति॒ सु - श्लो॒काॅ(4) । सुम॑ङ्ग॒लाॅ(4) इति॒ सु - म॒ङ्ग॒लाॅ(4) । सत्य॑रा॒जा(3)निति॒ सत्य॑ - रा॒जा(3)न् ॥ अ॒पाम् । नप्त्रे᳚ । स्वाहा᳚ । ऊ॒र्जः । नप्त्रे᳚ । स्वाहा᳚ । अ॒ग्नये᳚ । गृ॒हप॑तय॒ इति॑ गृ॒ह - प॒त॒ये॒ । स्वाहा᳚ ॥ 32 \textbf{  32 } \newline
                  \newline
                      (स॒त्यौजा᳚-श्चत्वारिꣳ॒॒शच्च॑)  \textbf{(A16)} \newline \newline
                                \textbf{ TS 1.8.17.1} \newline
                  आ॒ग्ने॒यम् । अ॒ष्टाक॑पाल॒मित्य॒ष्टा-क॒पा॒ल॒म् । निरिति॑ । व॒प॒ति॒ । हिर॑ण्यम् । दक्षि॑णा । सा॒र॒स्व॒तम् । च॒रुम् । व॒थ्स॒त॒री । दक्षि॑णा । सा॒वि॒त्रम् । द्वाद॑शकपाल॒मिति॒ द्वाद॑श - क॒पा॒ल॒म् । उ॒प॒द्ध्व॒स्त इत्यु॑प - ध्व॒स्तः । दक्षि॑णा । पौ॒ष्णम् । च॒रुम् । श्या॒मः । दक्षि॑णा । बा॒र्.॒ह॒स्प॒त्यम् । च॒रुम् । शि॒ति॒पृ॒ष्ठ इति॑ शिति - पृ॒ष्ठः । दक्षि॑णा । ऐ॒न्द्रम् । एका॑दशकपाल॒मित्येका॑दश-क॒पा॒ल॒म् । ऋ॒ष॒भः । दक्षि॑णा । वा॒रु॒णम् । दश॑कपाल॒मिति॒ दश॑ - क॒पा॒ल॒म् । म॒हानि॑रष्ट॒ इति॑ म॒हा-नि॒र॒ष्टः॒ । दक्षि॑णा । सौ॒म्यम् । च॒रुम् । ब॒भ्रुः । दक्षि॑णा । त्वा॒ष्ट्रम् । अ॒ष्टाक॑पाल॒मित्य॒ष्टा - क॒पा॒ल॒म् । शु॒ण्ठः । दक्षि॑णा । वै॒ष्ण॒वम् । त्रि॒क॒पा॒लमिति॑ त्रि - क॒पा॒ल॒म् । वा॒म॒नः । दक्षि॑णा ॥ \textbf{  33} \newline
                  \newline
                      (आ॒ग्ने॒यꣳ हिर॑ण्यꣳ सारस्व॒तं-द्विच॑त्वारिꣳशत् )  \textbf{(A17)} \newline \newline
                                \textbf{ TS 1.8.18.1} \newline
                  स॒द्यः । दी॒क्ष॒य॒न्ति॒ । स॒द्यः । सोम᳚म् । क्री॒ण॒न्ति॒ । पु॒ण्ड॒रि॒स्र॒जाम् । प्रेति॑ । य॒च्छ॒ति॒ । द॒शभि॒रिति॑ द॒श-भिः॒ । व॒थ्स॒त॒रैः । सोम᳚म् । क्री॒णा॒ति॒ । द॒श॒पेय॒ इति॑ दश-पेयः॑ । भ॒व॒ति॒ । श॒तम् । ब्रा॒ह्म॒णाः । पि॒ब॒न्ति॒ । स॒प्त॒द॒शमिति॑ सप्त - द॒शम् । स्तो॒त्रम् । भ॒व॒ति॒ । प्रा॒का॒शौ । अ॒द्ध्व॒र्यवे᳚ । द॒दा॒ति॒ । स्रज᳚म् । उ॒द्गा॒त्र इत्यु॑त्-गा॒त्रे । रु॒क्मम् । होत्रे᳚ । अश्व᳚म् । प्र॒स्तो॒तृ॒प्र॒ति॒ह॒र्तृभ्या॒मिति॑ प्रस्तोतृप्रतिह॒र्तृ-भ्यां॒ । द्वाद॑श । प॒ष्ठौ॒हीः । ब्र॒ह्मणे᳚ । व॒शाम् । मै॒त्रा॒व॒रु॒णायेति॑ मैत्रा - व॒रु॒णाय॑ । ऋ॒ष॒भम् । ब्रा॒ह्म॒णा॒च्छꣳ॒॒सिने᳚ । वास॑सी॒ इति॑ । ने॒ष्टा॒पो॒तृभ्या॒मिति॑ नेष्टापो॒तृ - भ्या॒म् । स्थूरि॑ । य॒वा॒चि॒तमिति॑ यव - आ॒चि॒तम् । अ॒च्छा॒वा॒काय॑ । अ॒न॒ड्वाह᳚म् । अ॒ग्नीध॒ इत्य॑ग्नि - इधे᳚ । भा॒र्ग॒वः । होता᳚ । भ॒व॒ति॒ । श्रा॒य॒न्तीय᳚म् । ब्र॒ह्म॒सा॒ममिति॑ ब्रह्म - सा॒मम् । भ॒व॒ति॒ । वा॒र॒व॒न्तीय॒मिति॑ वार - व॒न्तीय᳚म् ( ) । अ॒ग्नि॒ष्टो॒म॒सा॒ममित्य॑ग्निष्टोम - सा॒मम् । सा॒र॒स्व॒तीः । अ॒पः । गृ॒ह्णा॒ति॒ ॥ \textbf{  34 } \newline
                  \newline
                      (वा॒॒र॒व॒न्तीयं॑ च॒त्वारि॑ च) \textbf{(A18)} \newline \newline
                                \textbf{ TS 1.8.19.1} \newline
                  आ॒ग्ने॒यम् । अ॒ष्टाक॑पाल॒मित्य॒ष्टा-क॒पा॒ल॒म् । निरिति॑ । व॒प॒ति॒ । हिर॑ण्यम् । दक्षि॑णा । ऐ॒न्द्रम् । एका॑दशकपाल॒मित्येका॑दश-क॒पा॒ल॒म् । ऋ॒ष॒भः । दक्षि॑णा । वै॒श्व॒दे॒वमिति॑ वैश्व-दे॒वम् । च॒रुम् । पि॒शङ्गी᳚ । प॒ष्टौ॒ही । दक्षि॑णा । मै॒त्रा॒व॒रु॒णीमिति॑ मैत्रा-व॒रु॒णीम् । आ॒मिक्षा᳚म् । व॒शा । दक्षि॑णा । बा॒र्.॒ह॒स्प॒त्यम् । च॒रुम् । शि॒ति॒पृ॒ष्ठ इति॑ शिति-पृ॒ष्ठः । दक्षि॑णा । आ॒दि॒त्याम् । म॒ल्॒.हाम् । ग॒र्भिणी᳚म् । एति॑ । ल॒भ॒ते॒ । मा॒रु॒तीम् । पृश्नि᳚म् । प॒ष्ठौ॒हीम् । अ॒श्विभ्या॒मित्य॒श्वि-भ्या॒म् । पू॒ष्णे । पु॒रो॒डाश᳚म् । द्वाद॑शकपाल॒मिति॒ द्वाद॑श - क॒पा॒ल॒म् । निरिति॑ । व॒प॒ति॒ । सर॑स्वते । स॒त्य॒वाच॒ इति॑ सत्य - वाचे᳚ । च॒रुम् । स॒वि॒त्रे । स॒त्यप्र॑सवा॒येति॑ स॒त्य - प्र॒स॒वा॒य॒ । पु॒रो॒डाश᳚म् । द्वाद॑शकपाल॒मिति॒ द्वाद॑श - क॒पा॒ल॒म् । ति॒सृ॒ध॒न्वमिति॑ तिसृ - ध॒न्वम् । शु॒ष्क॒दृ॒तिरिति॑ शुष्क - दृ॒तिः । दक्षि॑णा ॥ \textbf{  35} \newline
                  \newline
                      (अ॒ग्ने॒यꣳ हिर॑ण्यमै॒द्रमृ॑ष॒भो वै᳚श्वदे॒वं पि॒शङ्गी॑ बार्.हस्प॒त्यꣳ-स॒प्तच॑त्वारिꣳशत्)  \textbf{(A19)} \newline \newline
                                \textbf{ TS 1.8.20.1} \newline
                  आ॒ग्ने॒यम् । अ॒ष्टाक॑पाल॒मित्य॒ष्टा-क॒पा॒ल॒म् । निरिति॑ । व॒प॒ति॒ । सौ॒म्यम् । च॒रुम् । सा॒वि॒त्रम् । द्वाद॑शकपाल॒मिति॒ द्वाद॑श-क॒पा॒ल॒म् । बा॒र्.॒ह॒स्प॒त्यम् । च॒रुम् । त्वा॒ष्ट्रम् । अ॒ष्टाक॑पाल॒मित्य॒ष्टा - क॒पा॒ल॒म् । वै॒श्वा॒न॒रम् । द्वाद॑शकपाल॒मिति॒ द्वाद॑श - क॒पा॒ल॒म् । दक्षि॑णः । र॒थ॒वा॒ह॒न॒वा॒ह इति॑ रथवाहन-वा॒हः । दक्षि॑णा । सा॒र॒स्व॒तम् । च॒रुम् । निरिति॑ । व॒प॒ति॒ । पौ॒ष्णम् । च॒रुम् । मै॒त्रम् । च॒रुम् । वा॒रु॒णम् । च॒रुम् । क्षै॒त्र॒प॒त्यमिति॑ क्षैत्र - प॒त्यम् । च॒रुम् । आ॒दि॒त्यम् । च॒रुम् । उत्त॑र॒ इत्युत् - त॒रः॒ । र॒थ॒वा॒ह॒न॒वा॒ह इति॑ रथवाहन - वा॒हः । दक्षि॑णा ॥ \textbf{  36} \newline
                  \newline
                      (आ॒ग्ने॒यꣳ सौ॒म्यं बा॑र्.हस्प॒त्यं-चतु॑स्त्रिꣳशत्)  \textbf{(A20)} \newline \newline
                                \textbf{ TS 1.8.21.1} \newline
                  स्वा॒द्वीम् । त्वा॒ । स्वा॒दुना᳚ । ती॒व्राम् । ती॒व्रेण॑ । अ॒मृता᳚म् । अ॒मृते॑न । सृ॒जामि॑ । समिति॑ । सोमे॑न । सोमः॑ । अ॒सि॒ । अ॒श्विभ्या॒मित्य॒श्वि-भ्या॒म् । प॒च्य॒स्व॒ । सर॑स्वत्यै । प॒च्य॒स्व॒ । इन्द्रा॑य । सु॒त्रांण॒ इति॑ सु - त्रांणे᳚ । प॒च्य॒स्व॒ । पु॒नातु॑ । ते॒ । प॒रि॒स्रुत॒मिति॑ परि - स्रुत᳚म् । सोम᳚म् । सूर्य॑स्य । दु॒हि॒ता ॥ वारे॑ण । शश्व॑ता । तना᳚ ॥ वा॒युः । पू॒तः । प॒वित्रे॑ण । प्र॒त्यङ् । सोमः॑ । अति॑द्रुत॒ इत्यति॑-द्रु॒तः॒ ॥ इन्द्र॑स्य । युज्यः॑ । सखा᳚ ॥ कु॒वित् । अ॒ङ्ग । यव॑मन्त॒ इति॒ यव॑-म॒न्तः॒ । यव᳚म् । चि॒त् । यथा᳚ । दान्ति॑ । अ॒नु॒पू॒र्वमित्य॑नु - पू॒र्वम् । वि॒यूयेति॑ वि - यूय॑ ॥ इ॒हेहेती॒ह - इ॒ह॒ । ए॒षा॒म् । कृ॒णु॒त॒ । भोज॑नानि ( ) । ये । ब॒र्॒.हिषः॑ । नमो॑वृक्ति॒मिति॒ नमः॑ - वृ॒क्ति॒म् । न । ज॒ग्मुः ॥ आ॒श्वि॒नम् । धू॒म्रम् । एति॑ । ल॒भ॒ते॒ । सा॒र॒स्व॒तम् । मे॒षम् । ऐ॒न्द्रम् । ऋ॒ष॒भम् । ऐ॒न्द्रम् । एका॑दशकपाल॒मित्येका॑दश-क॒पा॒ल॒म् । निरिति॑ । व॒प॒ति॒ । सा॒वि॒त्रम् । द्वाद॑शकपाल॒मिति॒ द्वाद॑श - क॒पा॒ल॒म् । वा॒रु॒णम् । दश॑कपाल॒मिति॒ दश॑ - क॒पा॒ल॒म् । सोम॑प्रतीका॒ इति॒ सोम॑ - प्र॒ती॒काः॒ । पि॒त॒रः॒ । तृ॒प्णु॒त॒ । वड॑बा । दक्षि॑णा ॥ \textbf{  37 } \newline
                  \newline
                      (भोज॑नानि॒-षड्विꣳ॑शतिश्च)  \textbf{(A21)} \newline \newline
                                \textbf{ TS 1.8.22.1} \newline
                  अग्ना॑विष्णू॒ इत्यग्ना᳚ - वि॒ष्णू॒ । महि॑ । तत् । वा॒म् । म॒हि॒त्वमिति॑ महि - त्वम् । वी॒तम् । घृ॒तस्य॑ । गुह्या॑नि । नाम॑ ॥ दमे॑दम॒ इति॒ दमे᳚-द॒मे॒ । स॒प्त । रत्ना᳚ । दधा॑ना । प्रतीति॑ । वा॒म् । जि॒ह्वा । घृ॒तम् । एति॑ । च॒र॒ण्ये॒त् ॥ अग्ना॑विष्णू॒ इत्यग्ना᳚-वि॒ष्णू॒ । महि॑ । धाम॑ । प्रि॒यम् । वा॒म् । वी॒थः । घृ॒तस्य॑ । गुह्या᳚ । जु॒षा॒णा ॥ दमे॑दम॒ इति॒ दमे᳚ - द॒मे॒ । सु॒ष्टु॒तीरिति॑ सु-स्तु॒तीः । वा॒वृ॒धा॒ना । प्रतीति॑ । वा॒म् । जि॒ह्वा । घृ॒तम् । उदिति॑ । च॒र॒ण्ये॒त् ॥ प्रेति॑ । नः॒ । दे॒वी । सर॑स्वती । वाजे॑भिः । वा॒जिनी॑व॒तीति॑ वा॒जिनी᳚ - व॒ती॒ ॥ धी॒नाम् । अ॒वि॒त्री । अ॒व॒तु॒ ॥ एति॑ । नः॒ । दि॒वः । बृ॒ह॒तः । \textbf{  38} \newline
                  \newline
                                \textbf{ TS 1.8.22.2} \newline
                  पर्व॑तात् । एति॑ । सर॑स्वती । य॒ज॒ता । ग॒न्तु॒ । य॒ज्ञ्म् ॥ हव᳚म् । दे॒वी । जु॒जु॒षा॒णा । घृ॒ताची᳚ । श॒ग्माम् । नः॒ । वाच᳚म् । उ॒श॒ती । शृ॒णो॒तु॒ ॥ बृह॑स्पते । जु॒षस्व॑ । नः॒ । ह॒व्यानि॑ । वि॒श्व॒दे॒व्येति॑ विश्व-दे॒व्य॒ ॥ रास्व॑ । रत्ना॑नि । दा॒शुषे᳚ ॥ ए॒वा । पि॒त्रे । वि॒श्वदे॑वा॒येति॑ वि॒श्व-दे॒वा॒य॒ । वृष्णे᳚ । य॒ज्ञिः । वि॒धे॒म॒ । नम॑सा । ह॒विर्भि॒रिति॑ ह॒विः - भिः॒ ॥ बृह॑स्पते । सु॒प्र॒जा इति॑ सु - प्र॒जाः । वी॒रव॑न्त॒ इति॑ वी॒र - व॒न्तः॒ । व॒यम् । स्या॒म॒ । पत॑यः । र॒यी॒णाम् ॥ बृह॑स्पते । अतीति॑ । यत् । अ॒र्यः । अर्.हा᳚त् । द्यु॒मदिति॑ द्यु - मत् । वि॒भातीति॑ वि - भाति॑ । क्रतु॑म॒दिति॒ क्रतु॑ - म॒त् । जने॑षु ॥ यत् । दी॒दय॑त् । शव॑सा । \textbf{  39} \newline
                  \newline
                                \textbf{ TS 1.8.22.3} \newline
                  ऋ॒त॒प्र॒जा॒तेत्यृ॑त-प्र॒जा॒त॒ । तत् । अ॒स्मासु॑ । द्रवि॑णम् । धे॒हि॒ । चि॒त्रम् ॥ एति॑ । नः॒ । मि॒त्रा॒व॒रु॒णेति॑ मित्रा-व॒रु॒णा॒ । घृ॒तैः । गव्यू॑तिम् । उ॒क्ष॒त॒म् ॥ मद्ध्वा᳚ । रजाꣳ॑सि । सु॒क्र॒तू॒ इति॑ सु-क्र॒तू॒ ॥ प्रेति॑ । बा॒हवा᳚ । सि॒सृ॒त॒म् । जी॒वसे᳚ । नः॒ । एति॑ । नः॒ । गव्यू॑तिम् । उ॒क्ष॒त॒म् । घृ॒तेन॑ ॥ एति॑ । नः॒ । जने᳚ । श्र॒व॒य॒त॒म् । यु॒वा॒ना॒ । श्रु॒तम् । मे॒ । मि॒त्रा॒व॒रु॒णेति॑ मित्रा - व॒रु॒णा॒ । हवा᳚ । इ॒मा ॥ अ॒ग्निम् । वः॒ । पू॒र्व्यम् । गि॒रा । दे॒वम् । ई॒डे॒ । वसू॑नाम् ॥ स॒प॒र्यन्तः॑ । पु॒रु॒प्रि॒यमिति॑ पुरु-प्रि॒यम् । मि॒त्रम् । न । क्षे॒त्र॒साध॑स॒मिति॑ क्षेत्र - साध॑सम् ॥ म॒क्षु । दे॒वव॑त॒ इति॑ दे॒व - व॒तः॒ । रथः॑ । \textbf{  40} \newline
                  \newline
                                \textbf{ TS 1.8.22.4} \newline
                  शूरः॑ । वा॒ । पृ॒थ्स्विति॑ पृत्-सु । कासु॑ । चि॒त् ॥ दे॒वाना᳚म् । यः । इत् । मनः॑ । यज॑मानः । इय॑क्षति । अ॒भीति॑ । इत् । अय॑ज्वनः । भु॒व॒त् ॥ न । य॒ज॒मा॒न॒ । रि॒ष्य॒सि॒ । न । सु॒न्वा॒न॒ । न । दे॒व॒यो॒ इति॑ देव-यो॒ ॥ अस॑त् । अत्र॑ । सु॒वीर्य॒मिति॑ सु - वीर्य᳚म् । उ॒त । त्यत् । आ॒श्वश्वि॑य॒मित्या॑शु-अश्वि॑यम् ॥ नकिः॑ । तम् । कर्म॑णा । न॒श॒त् । न । प्रेति॑ । यो॒ष॒त् । न । यो॒ष॒ति॒ ॥ उपेति॑ । क्ष॒र॒न्ति॒ । सिन्ध॑वः । म॒यो॒भुव॒ इति॑ मयः - भुवः॑ । ई॒जा॒नम् । च॒ । य॒क्ष्यमा॑णम् । च॒ । धे॒नवः॑ ॥ पृ॒णन्त᳚म् । च॒ । पपु॑रिम् । च॒ । \textbf{  41} \newline
                  \newline
                                \textbf{ TS 1.8.22.5} \newline
                  श्र॒व॒स्यवः॑ । घृ॒तस्य॑ । धाराः᳚ । उपेति॑ । य॒न्ति॒ । वि॒श्वतः॑ ॥ सोमा॑रु॒द्रेति॒ सोमा᳚-रु॒द्रा॒ । वीति॑ । वृ॒ह॒त॒म् । विषू॑चीम् । अमी॑वा । या । नः॒ । गय᳚म् । आ॒वि॒वेशेत्या᳚ - वि॒वेश॑ ॥ आ॒रे । बा॒धे॒था॒म् । निर्.ऋ॑ति॒मिति॒ निः - ऋ॒ति॒म् । प॒रा॒चैः । कृ॒तम् । चि॒त् । एनः॑ । प्रेति॑ । मु॒मु॒क्त॒म् । अ॒स्मत् ॥ सोमा॑रु॒द्रेति॒ सोमा᳚-रु॒द्रा॒ । यु॒वम् । ए॒तानि॑ । अ॒स्मे इति॑ । विश्वा᳚ । त॒नूषु॑ । भे॒ष॒जानि॑ । ध॒त्त॒म् । अवेति॑ । स्य॒त॒म् । मु॒ञ्चत᳚म् । यत् । नः॒ । अस्ति॑ । त॒नूषु॑ । ब॒द्धम् । कृ॒तम् । एनः॑ । अ॒स्मत् । सोमा॑पूष॒णेति॒ सोमा᳚ - पू॒ष॒णा॒ । जन॑ना । र॒यी॒णाम् । जन॑ना । दि॒वः । जन॑ना ( ) । पृ॒थि॒व्याः ॥ जा॒तौ । विश्व॑स्य । भुव॑नस्य । गो॒पौ । दे॒वाः । अ॒कृ॒ण्व॒न्न् । अ॒मृत॑स्य । नाभि᳚म् ॥ इ॒मौ । दे॒वौ । जाय॑मानौ । जु॒ष॒न्त॒ । इ॒मौ । तमाꣳ॑सि । गू॒ह॒ता॒म् । अजु॑ष्टा ॥ आ॒भ्याम् । इन्द्रः॑ । प॒क्वम् । आ॒मासु॑ । अ॒न्तः । सो॒मा॒पू॒षभ्या॒मिति॑ सोमापू॒ष - भ्या॒म् । ज॒न॒त् । उ॒स्रिया॑सु ॥ \textbf{  42} \newline
                  \newline
                      (बृ॒ह॒तः-शव॑सा॒-रथः॒-पपु॑रिं च-दि॒वो जन॑ना॒-पञ्च॑विꣳशतिश्च)  \textbf{(A22)} \newline \newline
\textbf{praSna korvai with starting padams of 1 to 22 anuvAkams :-} \newline
(अनु॑मत्या-आग्ने॒य-मै᳚न्द्रा॒ग्नम॒ग्नये॒-सोमा॑य-प्रतिपू॒रुष-मै᳚न्द्राग्नं-धा॒त्रे बा॑र्.हस्प॒त्य-म॒ग्नये॒र्-ऽथतो॒-देवीः᳚-स॒मिधꣳ॒॒-सोम॒स्ये-न्द्र॑स्य -मि॒त्र-आ᳚ग्ने॒यꣳ-स॒द्य-आ᳚ग्ने॒यꣳ-मा᳚ग्ने॒यꣳ-स्वा॒द्वीं त्वा-ऽग्ना॑विष्णू॒-द्वाविꣳ॑शतिः । ) \newline

\textbf{korvai with starting padams of1, 11, 21 series of pa~jcAtis :-} \newline
(अनु॑मत्यै॒-यथाऽस॑ति॒-देवी॑रापो-मि॒त्रो॑ऽसि॒-शूरो॑ वा॒-द्विच॑त्वारिꣳशत् । ) \newline

\textbf{first and last padam of Eighth praSnam :-} \newline
(अनु॑मत्या-उ॒स्रिया॑सु ।) \newline 


॥ हरिः॑ ॐ ॥
॥ कृष्ण यजुर्वेदीय तैत्तिरीय संहितायां 
प्रथमकाण्डे अष्तमः प्रश्नः समाप्तः ॥

॥ इति प्रथमं काण्डं ॥
======================================= \newline
\pagebreak
\pagebreak
        


\end{document}
