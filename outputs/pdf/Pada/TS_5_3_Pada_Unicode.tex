\documentclass[17pt]{extarticle}
\usepackage{babel}
\usepackage{fontspec}
\usepackage{polyglossia}
\usepackage{extsizes}



\setmainlanguage{sanskrit}
\setotherlanguages{english} %% or other languages
\setlength{\parindent}{0pt}
\pagestyle{myheadings}
\newfontfamily\devanagarifont[Script=Devanagari]{AdishilaVedic}


\newcommand{\VAR}[1]{}
\newcommand{\BLOCK}[1]{}




\begin{document}
\begin{titlepage}
    \begin{center}
 
\begin{sanskrit}
    { \Large
    ॐ नमः परमात्मने, श्री महागणपतये नमः, श्री गुरुभ्यो नमः ॥ ह॒रिः॒ ॐ 
    }
    \\
    \vspace{2.5cm}
    \mbox{ \Huge
    5.3      पञ्चमकाण्डे तृतीयः प्रश्नः - चितीनां निरूपणं   }
\end{sanskrit}
\end{center}

\end{titlepage}
\tableofcontents

ॐ नमः परमात्मने, श्री महागणपतये नमः, श्री गुरुभ्यो नमः
ह॒रिः॒ ॐ \newline
5.3      पञ्चमकाण्डे तृतीयः प्रश्नः - चितीनां निरूपणं \newline

\addcontentsline{toc}{section}{ 5.3      पञ्चमकाण्डे तृतीयः प्रश्नः - चितीनां निरूपणं}
\markright{ 5.3      पञ्चमकाण्डे तृतीयः प्रश्नः - चितीनां निरूपणं \hfill https://www.vedavms.in \hfill}
\section*{ 5.3      पञ्चमकाण्डे तृतीयः प्रश्नः - चितीनां निरूपणं }
                                \textbf{ TS 5.3.1.1} \newline
                  उ॒थ्स॒न्न॒य॒ज्ञ् इत्यु॑थ्सन्न - य॒ज्ञ्ः । वै । ए॒षः । यत् । अ॒ग्निः । किम् । वा॒ । अह॑ । ए॒तस्य॑ । क्रि॒यते᳚ । किम् । वा॒ । न । यत् । वै । य॒ज्ञ्स्य॑ । क्रि॒यमा॑णस्य । अ॒न्त॒र्यन्तीत्य॑न्तः - यन्ति॑ । पूय॑ति । वै । अ॒स्य॒ । तत् । आ॒श्वि॒नीः । उपेति॑ । द॒धा॒ति॒ । अ॒श्विनौ᳚ । वै । दे॒वाना᳚म् । भि॒षजौ᳚ । ताभ्या᳚म् । ए॒व । अ॒स्मै॒ । भे॒ष॒जम् । क॒रो॒ति॒ । पञ्च॑ । उपेति॑ । द॒धा॒ति॒ । पाङ्क्तः॑ । य॒ज्ञ्ः । यावान्॑ । ए॒व । य॒ज्ञ्ः । तस्मै᳚ । भे॒ष॒जम् । क॒रो॒ति॒ । ऋ॒त॒व्याः᳚ । उपेति॑ । द॒धा॒ति॒ । ऋ॒तू॒नाम् । क्लृप्त्यै᳚ । \textbf{  1} \newline
                  \newline
                                \textbf{ TS 5.3.1.2} \newline
                  पञ्च॑ । उपेति॑ । द॒धा॒ति॒ । पञ्च॑ । वै । ऋ॒तवः॑ । याव॑न्तः । ए॒व । ऋ॒तवः॑ । तान् । क॒ल्प॒य॒ति॒ । स॒मा॒नप्र॑भृतय॒ इति॑ समा॒न - प्र॒भृ॒त॒यः॒ । भ॒व॒न्ति॒ । स॒मा॒नोद॑र्का॒ इति॑ समा॒न - उ॒द॒र्काः॒ । तस्मा᳚त् । स॒मा॒नाः । ऋ॒तवः॑ । एके॑न । प॒देन॑ । व्याव॑र्तन्त॒ इति॑ वि - आव॑र्तन्ते । तस्मा᳚त् । ऋ॒तवः॑ । व्याव॑र्तन्त॒ इति॑ वि-आव॑र्तन्ते । प्रा॒ण॒भृत॒ इति॑ प्राण-भृतः॑ । उपेति॑ । द॒धा॒ति॒ । ऋ॒तुषु॑ । ए॒व । प्रा॒णानिति॑ प्र - अ॒नान् । द॒धा॒ति॒ । तस्मा᳚त् । स॒मा॒नाः । सन्तः॑ । ऋ॒तवः॑ । न । जी॒र्य॒न्ति॒ । अथो॒ इति॑ । प्रेति॑ । ज॒न॒य॒ति॒ । ए॒व । ए॒ना॒न् । ए॒षः । वै । वा॒युः । यत् । प्रा॒ण इति॑ प्र - अ॒नः । यत् । ऋ॒त॒व्याः᳚ । उ॒प॒धायेत्यु॑प - धाय॑ । प्रा॒ण॒भृत॒ इति॑ प्राण - भृतः॑ । \textbf{  2} \newline
                  \newline
                                \textbf{ TS 5.3.1.3} \newline
                  उ॒प॒दधा॒तीत्यु॑प - दधा॑ति । तस्मा᳚त् । सर्वान्॑ । ऋ॒तून् । अन्विति॑ । वा॒युः । एति॑ । व॒री॒व॒र्ति॒ । वृ॒ष्टि॒सनी॒रिति॑ वृष्टि - सनीः᳚ । उपेति॑ । द॒धा॒ति॒ । वृष्टि᳚म् । ए॒व । अवेति॑ । रु॒न्धे॒ । यत् । ए॒क॒धेत्ये॑क - धा । उ॒प॒द॒द्ध्यादित्यु॑प - द॒द्ध्यात् । एक᳚म् । ऋ॒तुम् । व॒र्.॒षे॒त् । अ॒नु॒प॒रि॒हार॒मित्य॑नु-प॒रि॒हार᳚म् । सा॒द॒य॒ति॒ । तस्मा᳚त् । सर्वान्॑ । ऋ॒तून् । व॒र्.॒ष॒ति॒ । यत् । प्रा॒ण॒भृत॒ इति॑ प्राण - भृतः॑ । उ॒प॒धायेत्यु॑प - धाय॑ । वृ॒ष्टि॒सनी॒रिति॑ वृष्टि - सनीः᳚ । उ॒प॒दधा॒तीत्यु॑प - दधा॑ति । तस्मा᳚त् । वा॒युप्र॑च्यु॒तेति॑ वा॒यु - प्र॒च्यु॒ता॒ । दि॒वः । वृष्टिः॑ । ई॒र्ते॒ । प॒शवः॑ । वै । व॒य॒स्याः᳚ । नाना॑मनस॒ इति॒ नाना᳚ - म॒न॒सः॒ । खलु॑ । वै । प॒शवः॑ । नाना᳚व्रता॒ इति॒ नाना᳚ - व्र॒ताः॒ । ते । अ॒पः । ए॒व । अ॒भीति॑ । सम॑नस॒ इति॒ स - म॒न॒सः॒ । \textbf{  3} \newline
                  \newline
                                \textbf{ TS 5.3.1.4} \newline
                  यम् । का॒मये॑त । अ॒प॒शुः । स्या॒त् । इति॑ । व॒य॒स्याः᳚ । तस्य॑ । उ॒प॒धायेत्यु॑प - धाय॑ । अ॒प॒स्याः᳚ । उपेति॑ । द॒द्ध्या॒त् । अस᳚ज्ञांन॒मित्यसं᳚ - ज्ञा॒न॒म् । ए॒व । अ॒स्मै॒ । प॒शुभि॒रिति॑ प॒शु-भिः॒ । क॒रो॒ति॒ । अ॒प॒शुः । ए॒व । भ॒व॒ति॒ । यम् । का॒मये॑त । प॒शु॒मानिति॑ पशु - मान् । स्या॒त् । इति॑ । अ॒प॒स्याः᳚ । तस्य॑ । उ॒प॒धायेत्यु॑प - धाय॑ । व॒य॒स्याः᳚ । उपेति॑ । द॒द्ध्या॒त् । स॒ज्ञांन॒मिति॑ सं - ज्ञान᳚म् । ए॒व । अ॒स्मै॒ । प॒शुभि॒रिति॑ प॒शु - भिः॒ । क॒रो॒ति॒ । प॒शु॒मानिति॑ पशु - मान् । ए॒व । भ॒व॒ति॒ । चत॑स्रः । पु॒रस्ता᳚त् । उपेति॑ । द॒धा॒ति॒ । तस्मा᳚त् । च॒त्वारि॑ । चक्षु॑षः । रू॒पाणि॑ । द्वे इति॑ । शु॒क्ले इति॑ । द्वे इति॑ । कृ॒ष्णे इति॑ । \textbf{  4} \newline
                  \newline
                                \textbf{ TS 5.3.1.5} \newline
                  मू॒द्‌र्ध॒न्वती॒रिति॑ मूद्‌र्धन्न्-वतीः᳚ । भ॒व॒न्ति॒ । तस्मा᳚त् । पु॒रस्ता᳚त् । मू॒द्‌र्धा । पञ्च॑ । दक्षि॑णायाम् । श्रोण्या᳚म् । उपेति॑ । द॒धा॒ति॒ । पञ्च॑ । उत्त॑रस्या॒मित्युत्-त॒र॒स्या॒म् । तस्मा᳚त् । प॒श्चात् । वर्.षी॑यान् । पु॒रस्ता᳚त् प्रवण॒ इति॑ पु॒रस्ता᳚त् - प्र॒व॒णः॒ । प॒शुः । ब॒स्तः । वयः॑ । इति॑ । दक्षि॑णे । अꣳसे᳚ । उपेति॑ । द॒धा॒ति॒ । वृ॒ष्णिः । वयः॑ । इति॑ । उत्त॑र॒ इत्युत् - त॒रे॒ । अꣳसौ᳚ । ए॒व । प्रतीति॑ । द॒धा॒ति॒ । व्या॒घ्रः । वयः॑ । इति॑ । दक्षि॑णे । प॒क्षे । उपेति॑ । द॒धा॒ति॒ । सिꣳ॒॒हः । वयः॑ । इति॑ । उत्त॑र॒ इत्युत् - त॒रे॒ । प॒क्षयोः᳚ । ए॒व । वी॒र्य᳚म् । द॒धा॒ति॒ । पुरु॑षः । वयः॑ । इति॑ ( ) । मद्ध्ये᳚ । तस्मा᳚त् । पुरु॑षः । प॒शू॒नाम् । अधि॑पति॒रित्यधि॑ - प॒तिः॒ ॥ \textbf{  5 } \newline
                  \newline
                      (क्लृप्त्या॑ - उप॒धाय॑ प्राण॒भृतः॒-सम॑नसः-कृ॒ष्णे-पुरु॑षो॒ वय॒ इति॒ - पञ्च॑ च)  \textbf{(A1)} \newline \newline
                                \textbf{ TS 5.3.2.1} \newline
                  इन्द्रा᳚ग्नी॒ इतीन्द्र॑ - अ॒ग्नी॒ । अव्य॑थमानाम् । इति॑ । स्व॒य॒मा॒तृ॒ण्णामिति॑ स्वयं - आ॒तृ॒ण्णाम् । उपेति॑ । द॒धा॒ति॒ । इ॒न्द्रा॒ग्निभ्या॒मिती᳚न्द्रा॒ग्नि - भ्या॒म् । वै । इ॒मौ । लो॒कौ । विधृ॑ता॒विति॒ वि - धृ॒तौ॒ । अ॒नयोः᳚ । लो॒कयोः᳚ । विधृ॑त्या॒ इति॒ वि - धृ॒त्यै॒ । अधृ॑ता । इ॒व॒ । वै । ए॒षा । यत् । म॒द्ध्य॒मा । चितिः॑ । अ॒न्तरि॑क्षम् । इ॒व॒ । वै । ए॒षा । इन्द्रा᳚ग्नी॒ इतीन्द्र॑ - अ॒ग्नी॒ । इति॑ । आ॒ह॒ । इ॒न्द्रा॒ग्नी इती᳚न्द्र - अ॒ग्नी । वै । दे॒वाना᳚म् । ओ॒जो॒भृता॒वित्यो॑जः - भृतौ᳚ । ओज॑सा । ए॒व । ए॒ना॒म् । अ॒न्तरि॑क्षे । चि॒नु॒ते॒ । धृत्यै᳚ । स्व॒य॒मा॒तृ॒ण्णामिति॑ स्वयं - आ॒तृ॒ण्णाम् । उपेति॑ । द॒धा॒ति॒ । अ॒न्तरि॑क्षम् । वै । स्व॒य॒मा॒तृ॒ण्णेति॑ स्वयं-आ॒तृ॒ण्णा । अ॒न्तरि॑क्षम् । ए॒व । उपेति॑ । ध॒त्ते॒ । अश्व᳚म् । उपेति॑ । \textbf{  6} \newline
                  \newline
                                \textbf{ TS 5.3.2.2} \newline
                  घ्रा॒प॒य॒ति॒ । प्रा॒णमिति॑ प्र - अ॒नम् । ए॒व । अ॒स्या॒म् । द॒धा॒ति॒ । अथो॒ इति॑ । प्रा॒जा॒प॒त्य इति॑ प्राजा - प॒त्यः । वै । अश्वः॑ । प्र॒जाप॑ति॒नेति॑ प्र॒जा - प॒ति॒ना॒ । ए॒व । अ॒ग्निम् । चि॒नु॒ते॒ । स्व॒य॒मा॒तृ॒ण्णेति॑ स्वयं - आ॒तृ॒ण्णा । भ॒व॒ति॒ । प्रा॒णाना॒मिति॑ प्र - अ॒नाना᳚म् । उथ्सृ॑ष्ट्य॒ इत्युत् - सृ॒ष्ट्यै॒ । अथो॒ इति॑ । सु॒व॒र्गस्येति॑ सुवः - गस्य॑ । लो॒कस्य॑ । अनु॑ख्यात्या॒ इत्यनु॑ - ख्या॒त्यै॒ । दे॒वाना᳚म् । वै । सु॒व॒र्गमिति॑ सुवः - गम् । लो॒कम् । य॒ताम् । दिशः॑ । समिति॑ । अ॒व्ली॒य॒न्त॒ । ते । ए॒ताः । दिश्याः᳚ । अ॒प॒श्य॒न्न् । ताः । उपेति॑ । अ॒द॒ध॒त॒ । ताभिः॑ । वै । ते । दिशः॑ । अ॒दृꣳ॒॒ह॒न्न् । यत् । दिश्याः᳚ । उ॒प॒दधा॒तीत्यु॑प - दधा॑ति । दि॒शाम् । विधृ॑त्या॒ इति॒ वि - धृ॒त्यै॒ । दश॑ । प्रा॒ण॒भृत॒ इति॑ प्राण - भृतः॑ । पु॒रस्ता᳚त् । उपेति॑ । \textbf{  7} \newline
                  \newline
                                \textbf{ TS 5.3.2.3} \newline
                  द॒धा॒ति॒ । नव॑ । वै । पुरु॑षे । प्रा॒णा इति॑ प्र - अ॒नाः । नाभिः॑ । द॒श॒मी । प्रा॒णानिति॑ प्र - अ॒नान् । ए॒व । पु॒रस्ता᳚त् । ध॒त्ते॒ । तस्मा᳚त् । पु॒रस्ता᳚त् । प्रा॒णा इति॑ प्र - अ॒नाः । ज्योति॑ष्मतीम् । उ॒त्त॒मामित्यु॑त् - त॒माम् । उपेति॑ । द॒धा॒ति॒ । तस्मा᳚त् । प्रा॒णाना॒मिति॑ प्र - अ॒नाना᳚म् । वाक् । ज्योतिः॑ । उ॒त्त॒मेत्यु॑त् - त॒मा । दश॑ । उपेति॑ । द॒धा॒ति॒ । दशा᳚क्ष॒रेति॒ दश॑ - अ॒क्ष॒रा॒ । वि॒राडिति॑ वि - राट् । वि॒राडिति॑ वि - राट् । छन्द॑साम् । ज्योतिः॑ । ज्योतिः॑ । ए॒व । पु॒रस्ता᳚त् । ध॒त्ते॒ । तस्मा᳚त् । पु॒रस्ता᳚त् । ज्योतिः॑ । उपेति॑ । आ॒स्म॒हे॒ । छन्दाꣳ॑सि । प॒शुषु॑ । आ॒जिम् । अ॒युः॒ । तान् । बृ॒ह॒ती । उदिति॑ । अ॒ज॒य॒त् । तस्मा᳚त् । बार्.ह॑ताः । \textbf{  8} \newline
                  \newline
                                \textbf{ TS 5.3.2.4} \newline
                  प॒शवः॑ । उ॒च्य॒न्ते॒ । मा । छन्दः॑ । इति॑ । द॒क्षि॒ण॒तः । उपेति॑ । द॒धा॒ति॒ । तस्मा᳚त् । द॒क्षि॒णावृ॑त॒ इति॑ दक्षि॒णा - आ॒वृ॒तः॒ । मासाः᳚ । पृ॒थि॒वी । छन्दः॑ । इति॑ । प॒श्चात् । प्रति॑ष्ठित्या॒ इति॒ प्रति॑ - स्थि॒त्यै॒ । अ॒ग्निः । दे॒वता᳚ । इति॑ । उ॒त्त॒र॒त इत्यु॑त् - त॒र॒तः । ओजः॑ । वै । अ॒ग्निः । ओजः॑ । ए॒व । उ॒त्त॒र॒त इत्यु॑त्-त॒र॒तः । ध॒त्ते॒ । तस्मा᳚त् । उ॒त्त॒र॒तो॒ऽभि॒प्र॒या॒यीत्यु॑त्तरतः - अ॒भि॒प्र॒या॒यी । ज॒य॒ति॒ । षट्त्रिꣳ॑श॒दिति॒ षट् - त्रिꣳ॒॒श॒त् । समिति॑ । प॒द्य॒न्ते॒ । षट्त्रिꣳ॑शदक्ष॒रेति॒ षट्त्रिꣳ॑शत् -   अ॒क्ष॒रा॒ । बृ॒ह॒ती । बार्.ह॑ताः । प॒शवः॑ । बृ॒ह॒त्या । ए॒व । अ॒स्मै॒ । प॒शून् । अवेति॑ । रु॒न्धे॒ । बृ॒ह॒ती । छन्द॑साम् । स्वारा᳚ज्य॒मिति॒ स्व - रा॒ज्य॒म् । परीति॑ । इ॒या॒य॒ । यस्य॑ । ए॒ताः । \textbf{  9} \newline
                  \newline
                                \textbf{ TS 5.3.2.5} \newline
                  उ॒प॒धी॒यन्त॒ इत्यु॑प-धी॒यन्ते᳚ । गच्छ॑ति । स्वारा᳚ज्य॒मिति॒ स्व-रा॒ज्य॒म् । स॒प्त । वाल॑खिल्या॒ इति॒ वाल॑ - खि॒ल्याः॒ । पु॒रस्ता᳚त् । उपेति॑ । द॒धा॒ति॒ । स॒प्त । प॒श्चात् । स॒प्त । वै । शी॒र्.॒ष॒ण्याः᳚ । प्रा॒णा इति॑ प्र - अ॒नाः । द्वौ । अवा᳚ञ्चौ । प्रा॒णाना॒मिति॑ प्र - अ॒नाना᳚म् । स॒वी॒र्य॒त्वायेति॑ सवीर्य - त्वाय॑ । मू॒द्‌र्धा । अ॒सि॒ । राट् । इति॑ । पु॒रस्ता᳚त् । उपेति॑ । द॒धा॒ति॒ । यन्त्री᳚ । राट् । इति॑ । प॒श्चात् । प्रा॒णानिति॑ प्र - अ॒नान् । ए॒व । अ॒स्मै॒ । स॒मीचः॑ । द॒धा॒ति॒ ॥ \textbf{  10 } \newline
                  \newline
                      (अश्व॒मुप॑-पु॒रस्ता॒दुप॒-बार्.ह॑ता-ए॒ता-श्चतु॑स्त्रिꣳशच्च)  \textbf{(A2)} \newline \newline
                                \textbf{ TS 5.3.3.1} \newline
                  दे॒वाः । वै । यत् । य॒ज्ञे । अकु॑र्वत । तत् । असु॑राः । अ॒कु॒र्व॒त॒ । ते । दे॒वाः । ए॒ताः । अ॒क्ष्ण॒या॒स्तो॒मीया॒ इत्य॑क्ष्णया-स्तो॒मीयाः᳚ । अ॒प॒श्य॒न्न् । ताः । अ॒न्यथा᳚ । अ॒नूच्येत्य॑नु - उच्य॑ । अ॒न्यथा᳚ । उपेति॑ । अ॒द॒ध॒त॒ । तत् । असु॑राः । न । अ॒न्ववा॑य॒न्नित्य॑नु - अवा॑यन्न् । ततः॑ । दे॒वाः । अभ॑वन्न् । परेति॑ । असु॑राः । यत् । अ॒क्ष्ण॒या॒स्तो॒मीया॒ इत्य॑क्ष्णया - स्तो॒मीयाः᳚ । अ॒न्यथा᳚ । अ॒नूच्येत्य॑नु - उच्य॑ । अ॒न्यथा᳚ । उ॒प॒दधा॒तीत्यु॑प-दधा॑ति । भ्रातृ॑व्याभिभूत्या॒ इति॒ भ्रातृ॑व्य - अ॒भि॒भू॒त्यै॒ । भव॑ति । आ॒त्मना᳚ । परेति॑ । अ॒स्य॒ । भ्रातृ॑व्यः । भ॒व॒ति॒ । आ॒शुः । त्रि॒वृदिति॑ त्रि - वृत् । इति॑ । पु॒रस्ता᳚त् । उपेति॑ । द॒धा॒ति॒ । य॒ज्ञ्॒मु॒खमिति॑ यज्ञ् - मु॒खम् । वै । त्रि॒वृदिति॑ त्रि - वृत् । \textbf{  11} \newline
                  \newline
                                \textbf{ TS 5.3.3.2} \newline
                  य॒ज्ञ्॒मु॒खमिति॑ यज्ञ् - मु॒खम् । ए॒व । पु॒रस्ता᳚त् । वीति॑ । या॒त॒य॒ति॒ । व्यो॑मेति॒ वि - ओ॒म॒ । स॒प्त॒द॒श इति॑ सप्त - द॒शः । इति॑ । द॒क्षि॒ण॒तः । अन्न᳚म् । वै । व्यो॑मेति॒ वि - ओ॒म॒ । अन्न᳚म् । स॒प्त॒द॒श इति॑ सप्त - द॒शः । अन्न᳚म् । ए॒व । द॒क्षि॒ण॒तः । ध॒त्ते॒ । तस्मा᳚त् । दक्षि॑णेन । अन्न᳚म् । अ॒द्य॒ते॒ । ध॒रुणः॑ । ए॒क॒विꣳ॒॒श इत्ये॑क-विꣳ॒॒शः । इति॑ । प॒श्चात् । प्र॒ति॒ष्ठेति॑ प्रति - स्था । वै । ए॒क॒विꣳ॒॒श इत्ये॑क - विꣳ॒॒शः । प्रति॑ष्ठित्या॒ इति॒ प्रति॑ - स्थि॒त्यै॒ । भा॒न्तः । प॒ञ्च॒द॒श इति॑ पञ्च - द॒शः । इति॑ । उ॒त्त॒र॒त इत्यु॑त् - त॒र॒तः । ओजः॑ । वै । भा॒न्तः । ओजः॑ । प॒ञ्च॒द॒श इति॑ पञ्च - द॒शः । ओजः॑ । ए॒व । उ॒त्त॒र॒त इत्यु॑त् - त॒र॒तः । ध॒त्ते॒ । तस्मा᳚त् । उ॒त्त॒र॒तो॒ऽभि॒प्र॒या॒यीत्यु॑त्तरतः - अ॒भि॒प्र॒या॒यी । ज॒य॒ति॒ । प्रतू᳚र्ति॒रिति॒ प्र - तू॒र्तिः॒ । अ॒ष्टा॒द॒श इत्य॑ष्टा - द॒शः । इति॑ । पु॒रस्ता᳚त् । \textbf{  12} \newline
                  \newline
                                \textbf{ TS 5.3.3.3} \newline
                  उपेति॑ । द॒धा॒ति॒ । द्वौ । त्रि॒वृता॒विति॑ त्रि - वृतौ᳚ । अ॒भि॒पू॒र्वमित्य॑भि - पू॒र्वम् । य॒ज्ञ्॒मु॒ख इति॑ यज्ञ् - मु॒खे । वीति॑ । या॒त॒य॒ति॒ । अ॒भि॒व॒र्त इत्य॑भि - व॒र्तः । स॒विꣳ॒॒श इति॑ स-विꣳ॒॒शः । इति॑ । द॒क्षि॒ण॒तः । अन्न᳚म् । वै । अ॒भि॒व॒र्त इत्य॑भि - व॒र्तः । अन्न᳚म् । स॒विꣳ॒॒श इति॑ स - विꣳ॒॒शः । अन्न᳚म् । ए॒व । द॒क्षि॒ण॒तः । ध॒त्ते॒ । तस्मा᳚त् । दक्षि॑णेन । अन्न᳚म् । अ॒द्य॒ते॒ । वर्चः॑ । द्वा॒विꣳ॒॒शः । इति॑ । प॒श्चात् । यत् । विꣳ॒॒श॒तिः । द्वे इति॑ । तेन॑ । वि॒राजा॒विति॑ वि-राजौ᳚ । यत् । द्वे इति॑ । प्र॒ति॒ष्ठेति॑ प्रति - स्था । तेन॑ । वि॒राजो॒रिति॑ वि - राजोः᳚ । ए॒व । अ॒भि॒पू॒र्वमित्य॑भि - पू॒र्वम् । अ॒न्नाद्य॒ इत्य॑न्न - अद्ये᳚ । प्रतीति॑ । ति॒ष्ठ॒ति॒ । तपः॑ । न॒व॒द॒श इति॑ नव - द॒शः । इति॑ । उ॒त्त॒र॒त इत्यु॑त्- त॒र॒तः । तस्मा᳚त् । स॒व्यः । \textbf{  13} \newline
                  \newline
                                \textbf{ TS 5.3.3.4} \newline
                  हस्त॑योः । त॒प॒स्वित॑र॒ इति॑ तप॒स्वि - त॒रः॒ । योनिः॑ । च॒तु॒र्विꣳ॒॒श इति॑ चतुः - विꣳ॒॒शः । इति॑ । पु॒रस्ता᳚त् । उपेति॑ । द॒धा॒ति॒ । चतु॑र्विꣳशत्यक्ष॒रेति॒ चतु॑र्विꣳशति - अ॒क्ष॒रा॒ । गा॒य॒त्री । गा॒य॒त्री । य॒ज्ञ्॒मु॒खमिति॑ यज्ञ् - मु॒खम् । य॒ज्ञ्॒मु॒खमिति॑ यज्ञ् - मु॒खम् । ए॒व । पु॒रस्ता᳚त् । वीति॑ । या॒त॒य॒ति॒ । गर्भाः᳚ । प॒ञ्च॒विꣳ॒॒श इति॑ पञ्च - विꣳ॒॒शः । इति॑ । द॒क्षि॒ण॒तः । अन्न᳚म् । वै । गर्भाः᳚ । अन्न᳚म् । प॒ञ्च॒विꣳ॒॒श इति॑ पञ्च - विꣳ॒॒शः । अन्न᳚म् । ए॒व । द॒क्षि॒ण॒तः । ध॒त्ते॒ । तस्मा᳚त् । दक्षि॑णेन । अन्न᳚म् । अ॒द्य॒ते॒ । ओजः॑ । त्रि॒ण॒व इति॑ त्रि - न॒वः । इति॑ । प॒श्चात् । इ॒मे । वै । लो॒काः । त्रि॒ण॒व इति॑ त्रि - न॒वः । ए॒षु । ए॒व । लो॒केषु॑ । प्रतीति॑ । ति॒ष्ठ॒ति॒ । स॒भंर॑ण॒ इति॑ सं - भर॑णः । त्र॒यो॒विꣳ॒॒श इति॑ त्रयः - विꣳ॒॒शः । इति॑ । \textbf{  14} \newline
                  \newline
                                \textbf{ TS 5.3.3.5} \newline
                  उ॒त्त॒र॒त इत्यु॑त् - त॒र॒तः । तस्मा᳚त् । स॒व्यः । हस्त॑योः । स॒भां॒र्य॑तर॒ इति॑ संभा॒र्य॑ - त॒रः॒ । क्रतुः॑ । ए॒क॒त्रिꣳ॒॒श इत्ये॑क - त्रिꣳ॒॒शः । इति॑ । पु॒रस्ता᳚त् । उपेति॑ । द॒धा॒ति॒ । वाक् । वै । क्रतुः॑ । य॒ज्ञ्॒मु॒खमिति॑ यज्ञ् - मु॒खम् । वाक् । य॒ज्ञ्॒मु॒खमिति॑ यज्ञ् - मु॒खम् । ए॒व । पु॒रस्ता᳚त् । वीति॑ । या॒त॒य॒ति॒ । ब्र॒द्ध्नस्य॑ । वि॒ष्टप᳚म् । च॒तु॒स्त्रिꣳ॒॒श इति॑ चतुः - त्रिꣳ॒॒शः । इति॑ । द॒क्षि॒ण॒तः । अ॒सौ । वै । आ॒दि॒त्यः । ब्र॒द्ध्नस्य॑ । वि॒ष्टप᳚म् । ब्र॒ह्म॒व॒र्च॒समिति॑ ब्रह्म - व॒र्च॒सम् । ए॒व । द॒क्षि॒ण॒तः । ध॒त्ते॒ । तस्मा᳚त् । दक्षि॑णः । अद्‌र्धः॑ । ब्र॒ह्म॒व॒र्च॒सित॑र॒ इति॑ ब्रह्मवर्च॒सि - त॒रः॒ । प्र॒ति॒ष्ठेति॑ प्रति - स्था । त्र॒य॒स्त्रिꣳ॒॒श इति॑ त्रयः-त्रिꣳ॒॒शः । इति॑ । प॒श्चात् । प्रति॑ष्ठित्या॒ इति॒ प्रति॑-स्थि॒त्यै॒ । नाकः॑ । ष॒ट्त्रिꣳ॒॒श इति॑ षट्-त्रिꣳ॒॒शः । इति॑ । उ॒त्त॒र॒त इत्यु॑त्-त॒र॒तः । सु॒व॒र्ग इति॑ सुवः - गः । वै ( ) । लो॒कः । नाकः॑ । सु॒व॒र्गस्येति॑ सुवः - गस्य॑ । लो॒कस्य॑ । सम॑ष्ट्या॒ इति॒ सं - अ॒ष्ट्यै॒ ॥आ॒शु - र्व्यो॑म - ध॒रुणो॑ - भा॒न्तः - प्रतू᳚र्तिर -भिव॒र्तो - वर्च॒ - स्तपो॒ - योनि॒ - र्गर्भा॒ - ओजः॑ - स॒भंर॑णः॒ - क्रतु॑ - र्ब्र॒द्ध्रस्य॑ - प्रति॒ष्ठा - नाकः॒ - षोड॑श) \textbf{  3} \newline
                  \newline
                      (वै त्रि॒वृ - दिति॑ पु॒रस्ता᳚थ् - स॒व्य - स्त्र॑योविꣳ॒॒श इति॑ - सुव॒र्गो वै - पञ्च॑ च)  \textbf{(A3)} \newline \newline
                                \textbf{ TS 5.3.4.1} \newline
                  अ॒ग्नेः । भा॒गः । अ॒सि॒ । इति॑ । पु॒रस्ता᳚त् । उपेति॑ । द॒धा॒ति॒ । य॒ज्ञ्॒मु॒खमिति॑ यज्ञ् - मु॒खम् । वै । अ॒ग्निः । य॒ज्ञ्॒मु॒खमिति॑ यज्ञ्-मु॒खम् । दी॒क्षा । य॒ज्ञ्॒मु॒खमिति॑ यज्ञ् - मु॒खम् । ब्रह्म॑ । य॒ज्ञ्॒मु॒खमिति॑ यज्ञ् - मु॒खम् । त्रि॒वृदिति॑ त्रि - वृत् । य॒ज्ञ्॒मु॒खमिति॑ यज्ञ्-मु॒खम् । ए॒व । पु॒रस्ता᳚त् । वीति॑ । या॒त॒य॒ति॒ । नृ॒चक्ष॑सा॒मिति॑ नृ - चक्ष॑साम् । भा॒गः । अ॒सि॒ । इति॑ । द॒क्षि॒ण॒तः । शु॒श्रु॒वाꣳसः॑ । वै । नृ॒चक्ष॑स॒ इति॑ नृ - चक्ष॑सः । अन्न᳚म् । धा॒ता । जा॒ताय॑ । ए॒व । अ॒स्मै॒ । अन्न᳚म् । अपीति॑ । द॒धा॒ति॒ । तस्मा᳚त् । जा॒तः । अन्न᳚म् । अ॒त्ति॒ । ज॒नित्र᳚म् । स्पृ॒तम् । स॒प्त॒द॒श इति॑ सप्त - द॒शः । स्तोमः॑ । इति॑ । आ॒ह॒ । अन्न᳚म् । वै । ज॒नित्र᳚म् । \textbf{  16} \newline
                  \newline
                                \textbf{ TS 5.3.4.2} \newline
                  अन्न᳚म् । स॒प्त॒द॒श इति॑ सप्त - द॒शः । अन्न᳚म् । ए॒व । द॒क्षि॒ण॒तः । ध॒त्ते॒ । तस्मा᳚त् । दक्षि॑णेन । अन्न᳚म् । अ॒द्य॒ते॒ । मि॒त्रस्य॑ । भा॒गः । अ॒सि॒ । इति॑ । प॒श्चात् । प्रा॒ण इति॑ प्र - अ॒नः । वै । मि॒त्रः । अ॒पा॒न इत्य॑प - अ॒नः । वरु॑णः । प्रा॒णा॒पा॒नाविति॑ प्राण - अ॒पा॒नौ । ए॒व । अ॒स्मि॒न्न् । द॒धा॒ति॒ । दि॒वः । वृ॒ष्टिः । वाताः᳚ । स्पृ॒ताः । ए॒क॒विꣳ॒॒श इत्ये॑क - विꣳ॒॒शः । स्तोमः॑ । इति॑ । आ॒ह॒ । प्र॒ति॒ष्ठेति॑ प्रति - स्था । वै । ए॒क॒विꣳ॒॒श इत्ये॑क - विꣳ॒॒शः । प्रति॑ष्ठित्या॒ इति॒ प्रति॑-स्थि॒त्यै॒ । इन्द्र॑स्य । भा॒गः । अ॒सि॒ । इति॑ । उ॒त्त॒र॒त इत्यु॑त् - त॒र॒तः । ओजः॑ । वै । इन्द्रः॑ । ओजः॑ । विष्णुः॑ । ओजः॑ । क्ष॒त्रम् । ओजः॑ । प॒ञ्च॒द॒श इति॑ पञ्च - द॒शः । \textbf{  17} \newline
                  \newline
                                \textbf{ TS 5.3.4.3} \newline
                  ओजः॑ । ए॒व । उ॒त्त॒र॒त इत्यु॑त् - त॒र॒तः । ध॒त्ते॒ । तस्मा᳚त् । उ॒त्त॒र॒तो॒ऽभि॒प्र॒या॒यीत्यु॑त्तरतः-अ॒भि॒प्र॒या॒यी । ज॒य॒ति॒ । वसू॑नाम् । भा॒गः । अ॒सि॒ । इति॑ । पु॒रस्ता᳚त् । उपेति॑ । द॒धा॒ति॒ । य॒ज्ञ्॒मु॒खमिति॑ यज्ञ्-मु॒खम् । वै । वस॑वः । य॒ज्ञ्॒मु॒खमिति॑ यज्ञ् - मु॒खम् । रु॒द्राः । य॒ज्ञ्॒मु॒खमिति॑ यज्ञ् - मु॒खम् । च॒तु॒र्विꣳ॒॒श इति॑ चतुः - विꣳ॒॒शः । य॒ज्ञ्॒मु॒खमिति॑ यज्ञ् - मु॒खम् । ए॒व । पु॒रस्ता᳚त् । वीति॑ । या॒त॒य॒ति॒ । आ॒दि॒त्याना᳚म् । भा॒गः । अ॒सि॒ । इति॑ । द॒क्षि॒ण॒तः । अन्न᳚म् । वै । आ॒दि॒त्याः । अन्न᳚म् । म॒रुतः॑ । अन्न᳚म् । गर्भाः᳚ । अन्न᳚म् । प॒ञ्च॒विꣳ॒॒श इति॑ पञ्च - विꣳ॒॒शः । अन्न᳚म् । ए॒व । द॒क्षि॒ण॒तः । ध॒त्ते॒ । तस्मा᳚त् । दक्षि॑णेन । अन्न᳚म् । अ॒द्य॒ते॒ । अदि॑त्यै । भा॒गः । \textbf{  18} \newline
                  \newline
                                \textbf{ TS 5.3.4.4} \newline
                  अ॒सि॒ । इति॑ । प॒श्चात् । प्र॒ति॒ष्ठेति॑ प्रति - स्था । वै । अदि॑तिः । प्र॒ति॒ष्ठेति॑ प्रति - स्था । पू॒षा । प्र॒ति॒ष्ठेति॑ प्रति - स्था । त्रि॒ण॒व इति॑ त्रि - न॒वः । प्रति॑ष्ठित्या॒ इति॒ प्रति॑-स्थि॒त्यै॒ । दे॒वस्य॑ । स॒वि॒तुः । भा॒गः । अ॒सि॒ । इति॑ । उ॒त्त॒र॒त इत्यु॑त् - त॒र॒तः । ब्रह्म॑ । वै । दे॒वः । स॒वि॒ता । ब्रह्म॑ । बृह॒स्पतिः॑ । ब्रह्म॑ । च॒तु॒ष्टो॒म इति॑ चतुः - स्तो॒मः । ब्र॒ह्म॒व॒र्च॒समिति॑ ब्रह्म - व॒र्च॒सम् । ए॒व । उ॒त्त॒र॒त इत्यु॑त् - त॒र॒तः । ध॒त्ते॒ । तस्मा᳚त् । उत्त॑र॒ इत्युत् - त॒रः॒ । अद्‌र्धः॑ । ब्र॒ह्म॒व॒र्च॒सित॑र॒ इति॑ ब्रह्मवर्च॒सि - त॒रः॒ । सा॒वि॒त्रव॒तीति॑ सावि॒त्र - व॒ती॒ । भ॒व॒ति॒ । प्रसू᳚त्या॒ इति॒ प्र - सू॒त्यै॒ । तस्मा᳚त् । ब्रा॒ह्म॒णाना᳚म् । उदी॑ची । स॒निः । प्रसू॒तेति॒ प्र - सू॒ता॒ । ध॒र्त्रः । च॒तु॒ष्टो॒म इति॑ चतुः - स्तो॒मः । इति॑ । पु॒रस्ता᳚त् । उपेति॑ । द॒धा॒ति॒ । य॒ज्ञ्॒मु॒खमिति॑ यज्ञ् - मु॒खम् । वै । ध॒र्त्रः । \textbf{  19} \newline
                  \newline
                                \textbf{ TS 5.3.4.5} \newline
                  य॒ज्ञ्॒मु॒खमिति॑ यज्ञ् - मु॒खम् । च॒तु॒ष्टो॒म इति॑ चतुः - स्तो॒मः । य॒ज्ञ्॒मु॒खमिति॑ यज्ञ् - मु॒खम् । ए॒व । पु॒रस्ता᳚त् । वीति॑ । या॒त॒य॒ति॒ । यावा॑नाम् । भा॒गः । अ॒सि॒ । इति॑ । द॒क्षि॒ण॒तः । मासाः᳚ । वै । यावाः᳚ । अ॒द्‌र्ध॒मा॒सा इत्य॑द्‌र्ध - मा॒साः । अया॑वाः । तस्मा᳚त् । द॒क्षि॒णावृ॑त॒ इति॑ दक्षि॒णा - आ॒वृ॒तः॒ । मासाः᳚ । अन्न᳚म् । वै । यावाः᳚ । अन्न᳚म् । प्र॒जा इति॑ प्र - जाः । अन्न᳚म् । ए॒व । द॒क्षि॒ण॒तः । ध॒त्ते॒ । तस्मा᳚त् । दक्षि॑णेन । अन्न᳚म् । अ॒द्य॒ते॒ । ऋ॒भू॒णाम् । भा॒गः । अ॒सि॒ । इति॑ । प॒श्चात् । प्रति॑ष्ठित्या॒ इति॒ प्रति॑ - स्थि॒त्यै॒ । वि॒व॒र्त इति॑ वि - व॒र्तः । अ॒ष्टा॒च॒त्वा॒रिꣳ॒॒श इत्य॑ष्टा-च॒त्वा॒रिꣳ॒॒शः । इति॑ । उ॒त्त॒र॒त इत्यु॑त्-त॒र॒तः । अ॒नयोः᳚ । लो॒कयोः᳚ । स॒वी॒र्य॒त्वायेति॑ सवीर्य - त्वाय॑ । तस्मा᳚त् । इ॒मौ । लो॒कौ । स॒माव॑द्वीर्या॒विति॑ स॒माव॑त् - वी॒र्यौ॒ । \textbf{  20} \newline
                  \newline
                                \textbf{ TS 5.3.4.6} \newline
                  यस्य॑ । मुख्य॑वती॒रिति॒ मुख्य॑ - व॒तीः॒ । पु॒रस्ता᳚त् । उ॒प॒धी॒यन्त॒ इत्यु॑प - धी॒यन्ते᳚ । मुख्यः॑ । ए॒व । भ॒व॒ति॒ । एति॑ । अ॒स्य॒ । मुख्यः॑ । जा॒य॒ते॒ । यस्य॑ । अन्न॑वती॒रित्यन्न॑ - व॒तीः॒ । द॒क्षि॒ण॒तः । अत्ति॑ । अन्न᳚म् । एति॑ । अ॒स्य॒ । अ॒न्ना॒द इत्य॑न्न - अ॒दः । जा॒य॒ते॒ । यस्य॑ । प्र॒ति॒ष्ठाव॑ती॒रिति॑ प्रति॒ष्ठा - व॒तीः॒ । प॒श्चात् । प्रतीति॑ । ए॒व । ति॒ष्ठ॒ति॒ । यस्य॑ । ओज॑स्वतीः । उ॒त्त॒र॒त इत्यु॑त् - त॒र॒तः । ओ॒ज॒स्वी । ए॒व । भ॒व॒ति॒ । एति॑ । अ॒स्य॒ । ओ॒ज॒स्वी । जा॒य॒ते॒ । अ॒र्कः । वै । ए॒षः । यत् । अ॒ग्निः । तस्य॑ । ए॒तत् । ए॒व । स्तो॒त्रम् । ए॒तत् । श॒स्त्रम् । यत् । ए॒षा । वि॒धेति॑ वि - धा । \textbf{  21} \newline
                  \newline
                                \textbf{ TS 5.3.4.7} \newline
                  वि॒धी॒यत॒ इति॑ वि - धी॒यते᳚ । अ॒र्के । ए॒व । तत् । अ॒र्क्य᳚म् । अनु॑ । वीति॑ । धी॒य॒ते॒ । अत्ति॑ । अन्न᳚म् । एति॑ । अ॒स्य॒ । अ॒न्ना॒द इत्य॑न्न - अ॒दः । जा॒य॒ते॒ । यस्य॑ । ए॒षा । वि॒धेति॑ वि - धा । वि॒धी॒यत॒ इति॑ वि - धी॒यते᳚ । यः । उ॒ । च॒ । ए॒ना॒म् । ए॒वम् । वेद॑ । सृष्टीः᳚ । उपेति॑ । द॒धा॒ति॒ । य॒था॒सृ॒ष्टमिति॑ यथा - सृ॒ष्टम् । ए॒व । अवेति॑ । रु॒न्धे॒ । न । वै । इ॒दम् । दिवा᳚ । न । नक्त᳚म् । आ॒सी॒त् । अव्या॑वृत्त॒मित्यवि॑ - आ॒वृ॒त्त॒म् । ते । दे॒वाः । ए॒ताः । व्यु॑ष्टी॒रिति॒ वि - उ॒ष्टीः॒ । अ॒प॒श्य॒न्न् । ताः । उपेति॑ । अ॒द॒ध॒त॒ । ततः॑ । वै । इ॒दम् ( ) । वीति॑ । औ॒च्छ॒त् । यस्य॑ । ए॒ताः । उ॒प॒धी॒यन्त॒ इत्यु॑प - धी॒यन्ते᳚ । वीति॑ । ए॒व । अ॒स्मै॒ । उ॒च्छ॒ति॒ । अथो॒ इति॑ । तमः॑ । ए॒व । अपेति॑ । ह॒ते॒ ॥(अ॒ग्ने - र्नृ॒चक्ष॑सां - ज॒नित्रं॑ - मि॒त्र - स्येन्द्र॑स्य॒ -वसू॑ना - मादि॒त्याना॒ - मदि॑त्यै - दे॒वस्य॑ सवि॒तुः - सा॑वि॒त्रव॑ती - ध॒र्त्रो - यावा॑ना-मृभू॒णां - ॅवि॑व॒र्त - श्चतु॑र्दश) \textbf{  4} \newline
                  \newline
                      (वै ज॒नित्रं॑ - पञ्चद॒शो - ऽदि॑त्यै भा॒गो - वै ध॒र्त्रः - स॒माव॑द्वीर्यै-वि॒धा-ततो॒ वा इ॒दं - चतु॑र्दश च )  \textbf{(A4)} \newline \newline
                                \textbf{ TS 5.3.5.1} \newline
                  अग्ने᳚ । जा॒तान् । प्रेति॑ । नु॒द॒ । नः॒ । स॒पत्नान्॑ । इति॑ । पु॒रस्ता᳚त् । उपेति॑ । द॒धा॒ति॒ । जा॒तान् । ए॒व । भ्रातृ॑व्यान् । प्रेति॑ । नु॒द॒ते॒ । सह॑सा । जा॒तान् । इति॑ । प॒श्चात् । ज॒नि॒ष्यमा॑णान् । ए॒व । प्रतीति॑ । नु॒द॒ते॒ । च॒तु॒श्च॒त्वा॒रिꣳ॒॒श इति॑ चतुः - च॒त्वा॒रिꣳ॒॒शः । स्तोमः॑ । इति॑ । द॒क्षि॒ण॒तः । ब्र॒ह्म॒व॒र्च॒समिति॑ ब्रह्म - व॒र्च॒सम् । वै । च॒तु॒श्च॒त्वा॒रिꣳ॒॒श इति॑ चतुः - च॒त्वा॒रिꣳ॒॒शः । ब्र॒ह्म॒व॒र्च॒समिति॑ ब्रह्म - व॒र्च॒सम् । ए॒व । द॒क्षि॒ण॒तः । ध॒त्ते॒ । तस्मा᳚त् । दक्षि॑णः । अद्‌र्धः॑ । ब्र॒ह्म॒व॒र्च॒सित॑र॒ इति॑ ब्रह्मवर्च॒सि - त॒रः॒ । षो॒ड॒शः । स्तोमः॑ । इति॑ । उ॒त्त॒र॒त इत्यु॑त् - त॒र॒तः । ओजः॑ । वै । षो॒ड॒शः । ओजः॑ । ए॒व । उ॒त्त॒र॒त इत्यु॑त् - त॒र॒तः । ध॒त्ते॒ । तस्मा᳚त् । \textbf{  23} \newline
                  \newline
                                \textbf{ TS 5.3.5.2} \newline
                  उ॒त्त॒र॒तो॒ऽभि॒प्र॒या॒यीत्यु॑त्तरतः - अ॒भि॒प्र॒या॒यी । ज॒य॒ति॒ । वज्रः॑ । वै । च॒तु॒श्च॒त्वा॒रिꣳ॒॒श इति॑ चतुः-च॒त्वा॒रिꣳ॒॒शः । वज्रः॑ । षो॒ड॒शः । यत् । ए॒ते इति॑ । इष्ट॑के॒ इति॑ । उ॒प॒दधा॒तीत्यु॑प - दधा॑ति । जा॒तान् । च॒ । ए॒व । ज॒नि॒ष्यमा॑णान् । च॒ । भ्रातृ॑व्यान् । प्र॒णुद्येति॑ प्र-नुद्य॑ । वज्र᳚म् । अनु॑ । प्रेति॑ । ह॒र॒ति॒ । स्तृत्यै᳚ । पुरी॑षवती॒मिति॒ पुरी॑ष-व॒ती॒म् । मद्ध्ये᳚ । उपेति॑ । द॒धा॒ति॒ । पुरी॑षम् । वै । मद्ध्य᳚म् । आ॒त्मनः॑ । सात्मा॑न॒मिति॒ स - आ॒त्मा॒न॒म् । ए॒व । अ॒ग्निम् । चि॒नु॒ते॒ । सात्मेति॒ स॒ - आ॒त्मा॒ । अ॒मुष्मिन्न्॑ । लो॒के । भ॒व॒ति॒ । यः । ए॒वम् । वेद॑ । ए॒ताः । वै । अ॒स॒प॒त्नाः । नाम॑ । इष्ट॑काः । यस्य॑ । ए॒ताः । उ॒प॒धी॒यन्त॒ इत्यु॑प - धी॒यन्ते᳚ । \textbf{  24} \newline
                  \newline
                                \textbf{ TS 5.3.5.3} \newline
                  न । अ॒स्य॒ । स॒पत्नः॑ । भ॒व॒ति॒ । प॒शुः । वै । ए॒षः । यत् । अ॒ग्निः । वि॒राज॒ इति॑ वि - राजः॑ । उ॒त्त॒माया॒मित्यु॑त् - त॒माया᳚म् । चित्या᳚म् । उपेति॑ । द॒धा॒ति॒ । वि॒राज॒मिति॑ वि - राज᳚म् । ए॒व । उ॒त्त॒मामित्यु॑त् - त॒माम् । प॒शुषु॑ । द॒धा॒ति॒ । तस्मा᳚त् । प॒शु॒मानिति॑ पशु - मान् । उ॒त्त॒मामित्यु॑त् - त॒माम् । वाच᳚म् । व॒द॒ति॒ । दश॑द॒शेति॒ दश॑ - द॒श॒ । उपेति॑ । द॒धा॒ति॒ । स॒वी॒र्य॒त्वायेति॑ सवीर्य-त्वाय॑ । अ॒क्ष्ण॒या । उपेति॑ । द॒धा॒ति॒ । तस्मा᳚त् । अ॒क्ष्ण॒या । प॒शवः॑ । अङ्गा॑नि । प्रेति॑ । ह॒र॒न्ति॒ । प्रति॑ष्ठित्या॒ इति॒ प्रति॑ - स्थि॒त्यै॒ । यानि॑ । वै । छन्दाꣳ॑सि । सु॒व॒र्ग्या॑णीति॑ सुवः - ग्या॑नि । आसन्न्॑ । तैः । दे॒वाः । सु॒व॒र्गमिति॑ सुवः - गम् । लो॒कम् । आ॒य॒न्न् । तेन॑ । ऋष॑यः । \textbf{  25} \newline
                  \newline
                                \textbf{ TS 5.3.5.4} \newline
                  अ॒श्रा॒म्य॒न्न् । ते । तपः॑ । अ॒त॒प्य॒न्त॒ । तानि॑ । तप॑सा । अ॒प॒श्य॒न्न् । तेभ्यः॑ । ए॒ताः । इष्ट॑काः । निरिति॑ । अ॒मि॒म॒त॒ । एवः॑ । छन्दः॑ । वरि॑वः । छन्दः॑ । इति॑ । ताः । उपेति॑ । अ॒द॒ध॒त॒ । ताभिः॑ । वै । ते । सु॒व॒र्गमिति॑ सुवः-गम् । लो॒कम् । आ॒य॒न्न् । यत् । ए॒ताः । इष्ट॑काः । उ॒प॒दधा॒तीत्यु॑प - दधा॑ति । यानि॑ । ए॒व । छन्दाꣳ॑सि । सु॒व॒र्ग्या॑णीति॑ सुवः-ग्या॑नि । तैः । ए॒व । यज॑मानः । सु॒व॒र्गमिति॑ सुवः - गम् । लो॒कम् । ए॒ति॒ । य॒ज्ञेन॑ । वै । प्र॒जाप॑ति॒रिति॑ प्र॒जा - प॒तिः॒ । प्र॒जा इति॑ प्र - जाः । अ॒सृ॒ज॒त॒ । ताः । स्तोम॑भागै॒रिति॒ स्तोम॑ - भा॒गैः॒ । ए॒व । अ॒सृ॒ज॒त॒ । यत् । \textbf{  26} \newline
                  \newline
                                \textbf{ TS 5.3.5.5} \newline
                  स्तोम॑भागा॒ इति॒ स्तोम॑ - भा॒गाः॒ । उ॒प॒दधा॒तीत्यु॑प - दधा॑ति । प्र॒जा इति॑ प्र - जाः । ए॒व । तत् । यज॑मानः । सृ॒ज॒ते॒ । बृह॒स्पतिः॑ । वै । ए॒तत् । य॒ज्ञ्स्य॑ । तेजः॑ । समिति॑ । अ॒भ॒र॒त् । यत् । स्तोम॑भागा॒ इति॒ स्तोम॑ - भा॒गाः॒ । यत् । स्तोम॑भागा॒ इति॒ स्तोम॑ - भा॒गाः॒ । उ॒प॒दधा॒तीत्यु॑प - दधा॑ति । सते॑जस॒मिति॒ स - ते॒ज॒स॒म् । ए॒व । अ॒ग्निम् । चि॒नु॒ते॒ । बृह॒स्पतिः॑ । वै । ए॒ताम् । य॒ज्ञ्स्य॑ । प्र॒ति॒ष्ठामिति॑ प्रति - स्थाम् । अ॒प॒श्य॒त् । यत् । स्तोम॑भागा॒ इति॒ स्तोम॑ - भा॒गाः॒ । यत् । स्तोम॑भागा॒ इति॒ स्तोम॑ - भा॒गाः॒ । उ॒प॒दधा॒तीत्यु॑प - दधा॑ति । य॒ज्ञ्स्य॑ । प्रति॑ष्ठित्या॒ इति॒ प्रति॑-स्थि॒त्यै॒ । स॒प्तस॒प्तेति॑ स॒प्त - स॒प्त॒ । उपेति॑ । द॒धा॒ति॒ । स॒वी॒र्य॒त्वायेति॑ सवीर्य - त्वाय॑ । ति॒स्रः । मद्ध्ये᳚ । प्रति॑ष्ठित्या॒ इति॒ प्रति॑ - स्थि॒त्यै॒ ॥ \textbf{  27 } \newline
                  \newline
                      ( उ॒त्त॒र॒तो ध॑त्ते॒ तस्मा॑ - दुपधी॒यन्त॒ - ऋष॑यो - ऽसृजत॒ यत् - त्रिच॑त्वारिꣳशच्च)  \textbf{(A5)} \newline \newline
                                \textbf{ TS 5.3.6.1} \newline
                  र॒श्मिः । इति॑ । ए॒व । आ॒दि॒त्यम् । अ॒सृ॒ज॒त॒ । प्रेति॒रिति॒ प्र - इ॒तिः॒ । इति॑ । धर्म᳚म् । अन्वि॑ति॒रित्यनु॑ - इ॒तिः॒ । इति॑ । दिव᳚म् । स॒न्धिरिति॑ सं - धिः । इति॑ । अ॒न्तरि॑क्षम् । प्र॒ति॒धिरिति॑ प्रति - धिः । इति॑ । पृ॒थि॒वीम् । वि॒ष्ट॒भं इति॑ वि - स्त॒भंः । इति॑ । वृष्टि᳚म् । प्र॒वेति॑ प्र - वा । इति॑ । अहः॑ । अ॒नु॒वेत्य॑नु-वा । इति॑ । रात्रि᳚म् । उ॒शिक् । इति॑ । वसून्॑ । प्र॒के॒त इति॑ प्र - के॒तः । इति॑ । रु॒द्रान् । सु॒दी॒तिरिति॑ सु-दी॒तिः । इति॑ । आ॒दि॒त्यान् । ओजः॑ । इति॑ । पि॒तॄन् । तन्तुः॑ । इति॑ । प्र॒जा इति॑ प्र - जाः । पृ॒त॒ना॒षाट् । इति॑ । प॒शून् । रे॒वत् । इति॑ । ओष॑धीः । अ॒भि॒जिदित्य॑भि - जित् । अ॒सि॒ । यु॒क्तग्रा॒वेति॑ यु॒क्त - ग्रा॒वा॒ । \textbf{  28} \newline
                  \newline
                                \textbf{ TS 5.3.6.2} \newline
                  इन्द्रा॑य । त्वा॒ । इन्द्र᳚म् । जि॒न्व॒ । इति॑ । ए॒व । द॒क्षि॒ण॒तः । वज्र᳚म् । परीति॑ । औ॒ह॒त् । अ॒भिजि॑त्या॒ इत्य॒भि - जि॒त्यै॒ । ताः । प्र॒जा इति॑ प्र - जाः । अप॑प्राणा॒ इत्यप॑ - प्रा॒णाः॒ । अ॒सृ॒ज॒त॒ । तासु॑ । अधि॑पति॒रित्यधि॑-प॒तिः॒ । अ॒सि॒ । इति॑ । ए॒व । प्रा॒णमिति॑ प्र-अ॒नम् । अ॒द॒धा॒त् । य॒न्ता । इति॑ । अ॒पा॒नमित्य॑प - अ॒नम् । सꣳ॒॒सर्प॒ इति॑ सं - सर्पः॑ । इति॑ । चक्षुः॑ । व॒यो॒धा इति॑ वयः - धाः । इति॑ । श्रोत्र᳚म् । ताः । प्र॒जा इति॑ प्र - जाः । प्रा॒ण॒तीरिति॑ प्र - अ॒न॒तीः । अ॒पा॒न॒तीरित्य॑प - अ॒न॒तीः । पश्य॑न्तीः । शृ॒ण्व॒तीः । न । मि॒थु॒नी । अ॒भ॒व॒न्न् । तासु॑ । त्रि॒वृदिति॑ त्रि - वृत् । अ॒सि॒ । इति॑ । ए॒व । मि॒थु॒नम् । अ॒द॒धा॒त् । ताः । प्र॒जा इति॑ प्र-जाः । मि॒थु॒नी । \textbf{  29} \newline
                  \newline
                                \textbf{ TS 5.3.6.3} \newline
                  भव॑न्तीः । न । प्रेति॑ । अ॒जा॒य॒न्त॒ । ताः । सꣳ॒॒रो॒ह इति॑ सं - रो॒हः । अ॒सि॒ । नी॒रो॒ह इति॑ निः - रो॒हः । अ॒सि॒ । इति॑ । ए॒व । प्रेति॑ । अ॒ज॒न॒य॒त् । ताः । प्र॒जा इति॑ प्र - जाः । प्रजा॑ता॒ इति॒ प्र-जा॒ताः॒ । न । प्रतीति॑ । अ॒ति॒ष्ठ॒न्न् । ताः । व॒सु॒कः । अ॒सि॒ । वेष॑श्रि॒रिति॒ वेष॑-श्रिः॒ । अ॒सि॒ । वस्य॑ष्टिः । अ॒सि॒ । इति॑ । ए॒व । ए॒षु । लो॒केषु॑ । प्रतीति॑ । अ॒स्था॒प॒य॒त् । यत् । आह॑ । व॒सु॒कः । अ॒सि॒ । वेष॑श्रि॒रिति॒ वेष॑ - श्रिः॒ । अ॒सि॒ । वस्य॑ष्टिः । अ॒सि॒ । इति॑ । प्र॒जा इति॑ प्र - जाः । ए॒व । प्रजा॑ता॒ इति॒ प्र - जा॒ताः॒ । ए॒षु । लो॒केषु॑ । प्रतीति॑ । स्था॒प॒य॒ति॒ । सात्मेति॒ स - आ॒त्मा॒ । अ॒न्तरि॑क्षम् ( ) । रो॒ह॒ति॒ । सप्रा॑ण॒ इति॒ स - प्रा॒णः॒ । अ॒मुष्मिन्न्॑ । लो॒के । प्रतीति॑ । ति॒ष्ठ॒ति॒ । अव्य॑द्‌र्धुक॒ इत्यवि॑ - अ॒द्‌र्धु॒कः॒ । प्रा॒णा॒पा॒नाभ्या॒मिति॑ प्राण - अ॒पा॒नाभ्या᳚म् । भ॒व॒ति॒ । यः । ए॒वम् । वेद॑ ॥ \textbf{  30} \newline
                  \newline
                      (यु॒क्तग्रा॑वा - प्र॒जा मि॑थु॒न्य॑ - न्तरि॑क्षं॒ - द्वाद॑श च)  \textbf{(A6)} \newline \newline
                                \textbf{ TS 5.3.7.1} \newline
                  ना॒क॒सद्भि॒रिति॑ नाक॒सत् - भिः॒ । वै । दे॒वाः । सु॒व॒र्गमिति॑ सुवः - गम् । लो॒कम् । आ॒य॒न्न् । तत् । ना॒क॒सदा॒मिति॑ नाक - सदा᳚म् । ना॒क॒स॒त्त्वमिति॑ नाकसत् - त्वम् । यत् । ना॒क॒सद॒ इति॑ नाक - सदः॑ । उ॒प॒दधा॒तीत्यु॑प - दधा॑ति । ना॒क॒सद्भि॒रिति॑ नाक॒सत् - भिः॒ । ए॒व । तत् । यज॑मानः । सु॒व॒र्गमिति॑ सुवः-गम् । लो॒कम् । ए॒ति॒ । सु॒व॒र्ग इति॑ सुवः-गः । वै । लो॒कः । नाकः॑ । यस्य॑ । ए॒ताः । उ॒प॒धी॒यन्त॒ इत्यु॑प - धी॒यन्ते᳚ । न । अ॒स्मै॒ । अक᳚म् । भ॒व॒ति॒ । य॒ज॒मा॒ना॒य॒त॒नमिति॑ यजमान - आ॒य॒त॒नम् । वै । ना॒क॒सद॒ इति॑ नाक - सदः॑ । यत् । ना॒क॒सद॒ इति॑ नाक - सदः॑ । उ॒प॒दधा॒तीत्यु॑प - दधा॑ति । आ॒यत॑न॒मित्या᳚ - यत॑नम् । ए॒व । तत् । यज॑मानः । कु॒रु॒ते॒ । पृ॒ष्ठाना᳚म् । वै । ए॒तत् । तेजः॑ । संभृ॑त॒मिति॒ सं - भृ॒त॒म् । यत् । ना॒क॒सद॒ इति॑ नाक - सदः॑ । यत् । ना॒क॒सद॒ इति॑ नाक - सदः॑ । \textbf{  31} \newline
                  \newline
                                \textbf{ TS 5.3.7.2} \newline
                  उ॒प॒दधा॒तीत्यु॑प - दधा॑ति । पृ॒ष्ठाना᳚म् । ए॒व । तेजः॑ । अवेति॑ । रु॒न्धे॒ । प॒ञ्च॒चोडा॒ इति॑ पञ्च - चोडाः᳚ । उपेति॑ । द॒धा॒ति॒ । अ॒फ्स॒रसः॑ । ए॒व । ए॒न॒म् । ए॒ताः । भू॒ताः । अ॒मुष्मिन्न्॑ । लो॒के । उपेति॑ । शे॒रे॒ । अथो॒ इति॑ । त॒नू॒पानी॒रिति॑ तनू - पानीः᳚ । ए॒व । ए॒ताः । यज॑मानस्य । यम् । द्वि॒ष्यात् । तम् । उ॒प॒दध॒दित्यु॑प - दध॑त् । ध्या॒ये॒त् । ए॒ताभ्यः॑ । ए॒व । ए॒न॒म् । दे॒वता᳚भ्यः । एति॑ । वृ॒श्च॒ति॒ । ता॒जक् । आर्ति᳚म् । एति॑ । ऋ॒च्छ॒ति॒ । उत्त॑रा॒ इत्युत् - त॒राः । ना॒क॒सद्भ्य॒ इति॑ नाक॒सत्-भ्यः॒ । उपेति॑ । द॒धा॒ति॒ । यथा᳚ । जा॒याम् । आ॒नीयेत्या᳚ - नीय॑ । गृ॒हेषु॑ । नि॒षा॒दय॒तीति॑ नि - सा॒दय॑ति । ता॒दृक् । ए॒व । तत् । \textbf{  32} \newline
                  \newline
                                \textbf{ TS 5.3.7.3} \newline
                  प॒श्चात् । प्राची᳚म् । उ॒त्त॒मामित्यु॑त् - त॒माम् । उपेति॑ । द॒धा॒ति॒ । तस्मा᳚त् । प॒श्चात् । प्राची᳚ । पत्नी᳚ । अन्विति॑ । आ॒स्ते॒ । स्व॒य॒मा॒तृ॒ण्णामिति॑ स्वयं - आ॒तृ॒ण्णाम् । च॒ । वि॒क॒र्णीमिति॑ वि - क॒र्णीम् । च॒ । उ॒त्त॒मे इत्यु॑त् - त॒मे । उपेति॑ । द॒धा॒ति॒ । प्रा॒ण इति॑ प्र - अ॒नः । वै । स्व॒य॒मा॒तृ॒ण्णेति॑ स्वयं - आ॒तृ॒ण्णा । आयुः॑ । वि॒क॒र्णीति॑ वि - क॒र्णी । प्रा॒णमिति॑ प्र - अ॒नम् । च॒ । ए॒व । आयुः॑ । च॒ । प्रा॒णाना॒मिति॑ प्र - अ॒नाना᳚म् । उ॒त्त॒मावित्यु॑त् - त॒मौ । ध॒त्ते॒ । तस्मा᳚त् । प्रा॒ण इति॑ प्र - अ॒नः । च॒ । आयुः॑ । च॒ । प्रा॒णाना॒मिति॑ प्र - अ॒नाना᳚म् । उ॒त्त॒मावित्यु॑त् - त॒मौ । न । अ॒न्याम् । उत्त॑रा॒मित्युत् - त॒रा॒म् । इष्ट॑काम् । उपेति॑ । द॒द्ध्या॒त् । यत् । अ॒न्याम् । उत्त॑रा॒मित्युत् - त॒रा॒म् । इष्ट॑काम् । उ॒प॒द॒द्ध्यादित्यु॑प - द॒द्ध्यात् । प॒शू॒नाम् । \textbf{  33} \newline
                  \newline
                                \textbf{ TS 5.3.7.4} \newline
                  च॒ । यज॑मानस्य । च॒ । प्रा॒णमिति॑ प्र - अ॒नम् । च॒ । आयुः॑ । च॒ । अपीति॑ । द॒द्ध्या॒त् । तस्मा᳚त् । न । अ॒न्या । उत्त॒रेत्युत्-त॒रा॒ । इष्ट॑का । उ॒प॒धेयेत्यु॑प - धेया᳚ । स्व॒य॒मा॒तृ॒ण्णामिति॑ स्वयं - आ॒तृ॒ण्णाम् । उपेति॑ । द॒धा॒ति॒ । अ॒सौ । वै । स्व॒य॒मा॒तृ॒ण्णेति॑ स्वयं - आ॒तृ॒ण्णा । अ॒मूम् । ए॒व । उपेति॑ । ध॒त्ते॒ । अश्व᳚म् । उपेति॑ । घ्रा॒प॒य॒ति॒ । प्रा॒णमिति॑ प्र - अ॒नम् । ए॒व । अ॒स्या॒म् । द॒धा॒ति॒ । अथो॒ इति॑ । प्रा॒जा॒प॒त्य इति॑ प्राजा - प॒त्यः । वै । अश्वः॑ । प्र॒जाप॑ति॒नेति॑ प्र॒जा - प॒ति॒ना॒ । ए॒व । अ॒ग्निम् । चि॒नु॒ते॒ । स्व॒य॒मा॒तृ॒ण्णेति॑ स्वयं - आ॒तृ॒ण्णा । भ॒व॒ति॒ । प्रा॒णाना॒मिति॑ प्र - अ॒नाना᳚म् । उथ्सृ॑ष्ट्या॒ इत्युत् - सृ॒ष्ट्यै॒ । अथो॒ इति॑ । सु॒व॒र्गस्येति॑ सुवः - गस्य॑ । लो॒कस्य॑ । अनु॑ख्यात्या॒ इत्यनु॑ - ख्या॒त्यै॒ । ए॒षा । वै ( ) । दे॒वाना᳚म् । विक्रा᳚न्ति॒रिति॒ वि - क्रा॒न्तिः॒ । यत् । वि॒क॒र्णीति॑ वि - क॒र्णी । यत् । वि॒क॒र्णीमिति॑ वि - क॒र्णीम् । उ॒प॒दधा॒तीत्यु॑प - दधा॑ति । दे॒वाना᳚म् । ए॒व । विक्रा᳚न्ति॒मिति॒ वि- क्रा॒न्ति॒म् । अनु॑ । वीति॑ । क्र॒म॒ते॒ । उ॒त्त॒र॒त इत्यु॑त् - त॒र॒तः । उपेति॑ । द॒धा॒ति॒ । तस्मा᳚त् । उ॒त्त॒र॒त उ॑पचार॒ इत्यु॑त्तर॒तः - उ॒प॒चा॒रः॒ । अ॒ग्निः । वा॒यु॒मतीति॑ वायु - मती᳚ । भ॒व॒ति॒ । समि॑द्ध्या॒ इति॒ सं - इ॒द्ध्यै॒ ॥ \textbf{  34} \newline
                  \newline
                      (संभृ॑तं॒ ॅयन्ना॑क॒सदो॒ यन्ना॑क॒सद॒ - स्तत् - प॑शू॒ना-मे॒षा वै-द्वाविꣳ॑शतिश्च)  \textbf{(A7)} \newline \newline
                                \textbf{ TS 5.3.8.1} \newline
                  छन्दाꣳ॑सि । उपेति॑ । द॒धा॒ति॒ । प॒शवः॑ । वै । छन्दाꣳ॑सि । प॒शून् । ए॒व । अवेति॑ । रु॒न्धे॒ । छन्दाꣳ॑सि । वै । दे॒वाना᳚म् । वा॒मम् । प॒शवः॑ । वा॒मम् । ए॒व । प॒शून् । अवेति॑ । रु॒न्धे॒ । ए॒ताम् । ह॒ । वै । य॒ज्ञ्से॑न॒ इति॑ य॒ज्ञ् - से॒नः॒ । चै॒त्रि॒या॒य॒णः । चिति᳚म् । वि॒दाम् । च॒का॒र॒ । तया᳚ । वै । सः । प॒शून् । अवेति॑ । अ॒रु॒न्ध॒ । यत् । ए॒ताम् । उ॒प॒दधा॒तीत्यु॑प - द॒धा॑ति । प॒शून् । ए॒व । अवेति॑ । रु॒न्धे॒ । गा॒य॒त्रीः । पु॒रस्ता᳚त् । उपेति॑ । द॒धा॒ति॒ । तेजः॑ । वै । गा॒य॒त्री । तेजः॑ । ए॒व । \textbf{  35} \newline
                  \newline
                                \textbf{ TS 5.3.8.2} \newline
                  मु॒ख॒तः । ध॒त्ते॒ । मू॒द्‌र्ध॒न्वती॒रिति॑ मूर्धन्न् - वतीः᳚ । भ॒व॒न्ति॒ । मू॒द्‌र्धान᳚म् । ए॒व । ए॒न॒म् । स॒मा॒नाना᳚म् । क॒रो॒ति॒ । त्रि॒ष्टुभः॑ । उपेति॑ । द॒धा॒ति॒ । इ॒न्द्रि॒यम् । वै । त्रि॒ष्टुक् । इ॒न्द्रि॒यम् । ए॒व । म॒द्ध्य॒तः । ध॒त्ते॒ । जग॑तीः । उपेति॑ । द॒धा॒ति॒ । जाग॑ताः । वै । प॒शवः॑ । प॒शून् । ए॒व । अवेति॑ । रु॒न्धे॒ । अ॒नु॒ष्टुभ॒ इत्य॑नु - स्तुभः॑ । उपेति॑ । द॒धा॒ति॒ । प्रा॒णा इति॑ प्र - अ॒नाः । वै । अ॒नु॒ष्टुबित्य॑नु - स्तुप् । प्रा॒णाना॒मिति॑ प्र -  अ॒नाना᳚म् । उथ्सृ॑ष्ट्या॒ इत्युत् - सृ॒ष्ट्यै॒ । बृ॒ह॒तीः । उ॒ष्णिहाः᳚ । प॒ङ्क्तीः । अ॒क्षर॑पङ्क्ती॒रित्य॒क्षर॑ - प॒ङ्क्तीः॒ । इति॑ । विषु॑रूपा॒णीति॒ विषु॑ - रू॒पा॒णि॒ । छन्दाꣳ॑सि । उपेति॑ । द॒धा॒ति॒ । विषु॑रूपा॒ इति॒ विषु॑ - रू॒पाः॒ । वै । प॒शवः॑ । प॒शवः॑ । \textbf{  36} \newline
                  \newline
                                \textbf{ TS 5.3.8.3} \newline
                  छन्दाꣳ॑सि । विषु॑रूपा॒निति॒ विषु॑ - रू॒पा॒न् । ए॒व । प॒शून् । अवेति॑ । रु॒न्धे॒ । विषु॑रूप॒मिति॒ विषु॑-रू॒प॒म् । अ॒स्य॒ । गृ॒हे । दृ॒श्य॒ते॒ । यस्य॑ । ए॒ताः । उ॒प॒धी॒यन्त॒ इत्यु॑प - धी॒यन्ते᳚ । यः । उ॒ । च॒ । ए॒नाः॒ । ए॒वम् । वेद॑ । अति॑च्छन्दस॒मित्यति॑ - छ॒न्द॒स॒म् । उपेति॑ । द॒धा॒ति॒ । अति॑च्छन्दा॒ इत्यति॑ - छ॒न्दाः॒ । वै । सर्वा॑णि । छन्दाꣳ॑सि । सर्वे॑भिः । ए॒व । ए॒न॒म् । छन्दो॑भि॒रिति॒ छन्दः॑-भिः॒ । चि॒नु॒ते॒ । वर्ष्म॑ । वै । ए॒षा । छन्द॑साम् । यत् । अति॑च्छन्दा॒ इत्यति॑ - छ॒न्दाः॒ । यत् । अति॑च्छन्दस॒मित्यति॑ - छ॒न्द॒स॒म् । उ॒प॒दधा॒तीत्यु॑प - दधा॑ति । वर्ष्म॑ । ए॒व । ए॒न॒म् । स॒मा॒नाना᳚म् । क॒रो॒ति॒ । द्वि॒पदा॒ इति॑ द्वि - पदाः᳚ । उपेति॑ । द॒धा॒ति॒ । द्वि॒पादिति॑ द्वि - पात् । यज॑मानः ( ) । प्रति॑ष्ठित्या॒ इति॒ प्रति॑ - स्थि॒त्यै॒ ॥ \textbf{  37 } \newline
                  \newline
                      (तेज॑ ए॒व - प॒शवः॑ प॒शवो॒ - यज॑मान॒ - एक॑ञ्च)  \textbf{(A8)} \newline \newline
                                \textbf{ TS 5.3.9.1} \newline
                  सर्वा᳚भ्यः । वै । दे॒वता᳚भ्यः । अ॒ग्निः । ची॒य॒ते॒ । यत् । स॒युज॒ इति॑ स - युजः॑ । न । उ॒प॒द॒द्ध्यादित्यु॑प - द॒ध्यात् । दे॒वताः᳚ । अ॒स्य॒ । अ॒ग्निम् । वृ॒ञ्जी॒र॒न्न् । यत् । स॒युज॒ इति॑ स - युजः॑ । उ॒प॒दधा॒तीत्यु॑प - दधा॑ति । आ॒त्मना᳚ । ए॒व । ए॒न॒म् । स॒युज॒मिति॑ स-युज᳚म् । चि॒नु॒ते॒ । न । अ॒ग्निना᳚ । वीति॑ । ऋ॒द्ध्य॒ते॒ । अथो॒ इति॑ । यथा᳚ । पुरु॑षः । स्नाव॑भि॒रिति॒ स्नाव॑ - भिः॒ । संत॑त॒ इति॒ सं-त॒तः॒ । ए॒वम् । ए॒व । ए॒ताभिः॑ । अ॒ग्निः । संत॑त॒ इति॒ सं-त॒तः॒ । अ॒ग्निना᳚ । वै । दे॒वाः । सु॒व॒र्गमिति॑ सुवः - गम् । लो॒कम् । आ॒य॒न्न् । ताः । अ॒मूः । कृत्ति॑काः । अ॒भ॒व॒न्न् । यस्य॑ । ए॒ताः । उ॒प॒धी॒यन्त॒ इत्यु॑प - धी॒यन्ते᳚ । सु॒व॒र्गमिति॑ सुवः - गम् । ए॒व । \textbf{  38} \newline
                  \newline
                                \textbf{ TS 5.3.9.2} \newline
                  लो॒कम् । ए॒ति॒ । गच्छ॑ति । प्र॒का॒शमिति॑ प्र-का॒शम् । चि॒त्रम् । ए॒व । भ॒व॒ति॒ । म॒ण्ड॒ले॒ष्ट॒का इति॑ मण्डल - इ॒ष्ट॒काः । उपेति॑ । द॒धा॒ति॒ । इ॒मे । वै । लो॒काः । म॒ण्ड॒ले॒ष्ट॒का इति॑ मण्डल - इ॒ष्ट॒काः । इ॒मे । खलु॑ । वै । लो॒काः । दे॒व॒पु॒रा इति॑ देव-पु॒राः । दे॒व॒पु॒रा इति॑-पु॒राः । ए॒व । प्रेति॑ । वि॒श॒ति॒ । न । आर्ति᳚म् । एति॑ । ऋ॒च्छ॒ति॒ । अ॒ग्निम् । चि॒क्या॒नः । वि॒श्वज्यो॑तिष॒ इति॑ वि॒श्व - ज्यो॒ति॒षः॒ । उपेति॑ । द॒धा॒ति॒ । इ॒मान् । ए॒व । एताभिः॑ । लो॒कान् । ज्योति॑ष्मतः । कु॒रु॒ते॒ । अथो॒ इति॑ । प्रा॒णानिति॑ प्र - अ॒नान् । ए॒व । ए॒ताः । यज॑मानस्य । दा॒द्ध्र॒ति॒ । ए॒ताः । वै । दे॒वताः᳚ । सु॒व॒र्ग्या॑ इति॑ सुवः - ग्याः᳚ । ताः । ए॒व ( ) । अ॒न्वा॒रभ्येत्य॑नु - आ॒रभ्य॑ । सु॒व॒र्गमिति॑ सुवः - गम् । लो॒कम् । ए॒ति॒ ॥ \textbf{  39 } \newline
                  \newline
                      (सु॒व॒र्गमे॒व - ता ए॒व - च॒त्वारि॑ च)  \textbf{(A9)} \newline \newline
                                \textbf{ TS 5.3.10.1} \newline
                  वृ॒ष्टि॒सनी॒रिति॑ वृष्टि - सनीः᳚ । उपेति॑ । द॒धा॒ति॒ । वृष्टि᳚म् । ए॒व । अवेति॑ । रु॒न्धे॒ । यत् । ए॒क॒धेत्ये॑क - धा । उ॒प॒द॒द्ध्यादित्यु॑प - द॒द्ध्यात् । एक᳚म् । ऋ॒तुम् । व॒र्.॒षे॒त् । अ॒नु॒प॒रि॒हार॒मित्य॑नु - प॒रि॒हार᳚म् । सा॒द॒य॒ति॒ । तस्मा᳚त् । सर्वान्॑ । ऋ॒तून् । व॒र्.॒ष॒ति॒ । पु॒रो॒वा॒त॒सनि॒रिति॑ पुरोवात - सनिः॑ । अ॒सि॒ । इति॑ । आ॒ह॒ । ए॒तत् । वै । वृष्ट्यै᳚ । रू॒पम् । रू॒पेण॑ । ए॒व । वृष्टि᳚म् । अवेति॑ । रु॒न्धे॒ । सं॒ॅयानी॑भि॒रिति॑ सं - यानी॑भिः । वै । दे॒वाः । इ॒मान् । लो॒कान् । समिति॑ । अ॒युः॒ । तत् । सं॒ॅयानी॑ना॒मिति॑ सं - यानी॑नाम् । सं॒ॅया॒नि॒त्वमिति॑ संॅयानि - त्वम् । यत् । सं॒ॅयानी॒रिति॑ सं - यानीः᳚ । उ॒प॒दधा॒तीत्यु॑प - दधा॑ति । यथा᳚ । अ॒फ्स्वित्य॑प् - सु । ना॒वा । सं॒ॅयातीति॑ सं - याति॑ । ए॒वम् । \textbf{  40} \newline
                  \newline
                                \textbf{ TS 5.3.10.2} \newline
                  ए॒व । ए॒ताभिः॑ । यज॑मानः । इ॒मान् । लो॒कान् । समिति॑ । या॒ति॒ । प्ल॒वः । वै । ए॒षः । अ॒ग्नेः । यत् । सं॒ॅयानी॒रिति॑ सं - यानीः᳚ । यत् । सं॒ॅयानी॒रिति॑ सं - यानीः᳚ । उ॒प॒दधा॒तीत्यु॑प-दधा॑ति । प्ल॒वम् । ए॒व । ए॒तम् । अ॒ग्नये᳚ । उपेति॑ । द॒धा॒ति॒ । उ॒त । यस्य॑ । ए॒तासु॑ । उप॑हिता॒स्वित्युप॑ - हि॒ता॒सु॒ । आपः॑ । अ॒ग्निम् । हर॑न्ति । अहृ॑तः । ए॒व । अ॒स्य॒ । अ॒ग्निः । आ॒दि॒त्ये॒ष्ट॒का इत्या॑दित्य - इ॒ष्ट॒काः । उपेति॑ । द॒धा॒ति॒ । आ॒दि॒त्याः । वै । ए॒तम् । भूत्यै᳚ । प्रतीति॑ । नु॒द॒न्ते॒ । यः । अल᳚म् । भूत्यै᳚ । सन्न् । भूति᳚म् । न । प्रा॒प्नोतीति॑ प्र - आ॒प्नोति॑ । आ॒दि॒त्याः । \textbf{  41} \newline
                  \newline
                                \textbf{ TS 5.3.10.3} \newline
                  ए॒व । ए॒न॒म् । भूति᳚म् । ग॒म॒य॒न्ति॒ । अ॒सौ । वै । ए॒तस्य॑ । आ॒दि॒त्यः । रुच᳚म् । एति॑ । द॒त्ते॒ । यः । अ॒ग्निम् । चि॒त्वा । न । रोच॑ते । यत् । आ॒दि॒त्ये॒ष्ट॒का इत्या॑दित्य - इ॒ष्ट॒काः । उ॒प॒दधा॒तीत्यु॑प - दधा॑ति । अ॒सौ । ए॒व । अ॒स्मि॒न्न् । आ॒दि॒त्यः । रुच᳚म् । द॒धा॒ति॒ । यथा᳚ । अ॒सौ । दे॒वाना᳚म् । रोच॑ते । ए॒वम् । ए॒व । ए॒षः । म॒नु॒ष्या॑णाम् । रो॒च॒ते॒ । घृ॒ते॒ष्ट॒का इति॑ घृत - इ॒ष्ट॒काः । उपेति॑ । द॒धा॒ति॒ । ए॒तत् । वै । अ॒ग्नेः । प्रि॒यम् । धाम॑ । यत् । घृ॒तम् । प्रि॒येण॑ । ए॒व । ए॒न॒म् । धाम्ना᳚ । समिति॑ । अ॒द्‌र्ध॒य॒ति॒ । \textbf{  42} \newline
                  \newline
                                \textbf{ TS 5.3.10.4} \newline
                  अथो॒ इति॑ । तेज॑सा । अ॒नु॒प॒रि॒हार॒मित्य॑नु - प॒रि॒हार᳚म् । सा॒द॒य॒ति॒ । अप॑रिवर्ग॒मित्यप॑रि - व॒र्ग॒म् । ए॒व । अ॒स्मि॒न्न् । तेजः॑ । द॒धा॒ति॒ । प्र॒जाप॑ति॒रिति॑ प्र॒जा - प॒तिः॒ । अ॒ग्निम् । अ॒चि॒नु॒त॒ । सः । यश॑सा । वीति॑ । आ॒द्‌र्ध्य॒त॒ । सः । ए॒ताः । य॒शो॒दा इति॑ यशः - दाः । अ॒प॒श्य॒त् । ताः । उपेति॑ । अ॒ध॒त्त॒ । ताभिः॑ । वै । सः । यशः॑ । आ॒त्मन्न् । अ॒ध॒त्त॒ । यत् । य॒शो॒दा इति॑ यशः - दाः । उ॒प॒दधा॒तीत्यु॑प - दधा॑ति । यशः॑ । ए॒व । ताभिः॑ । यज॑मानः । आ॒त्मन्न् । ध॒त्ते॒ । पञ्च॑ । उपेति॑ । द॒धा॒ति॒ । पाङ्क्तः॑ । पुरु॑षः । यावान्॑ । ए॒व । पुरु॑षः । तस्मिन्न्॑ । यशः॑ । द॒धा॒ति॒ ॥ \textbf{  43} \newline
                  \newline
                      (ए॒वं - प्रा॒प्रोत्या॑दि॒त्या - अ॑र्धय॒त्ये - का॒न्न प॑ञ्चा॒शच्च॑)  \textbf{(A10)} \newline \newline
                                \textbf{ TS 5.3.11.1} \newline
                  दे॒वा॒सु॒रा इति॑ देव-अ॒सु॒राः । संॅय॑त्ता॒ इति॒ सं - य॒त्ताः॒ । आ॒स॒न्न् । कनी॑याꣳसः । दे॒वाः । आसन्न्॑ । भूयाꣳ॑सः । असु॑राः । ते । दे॒वाः । ए॒ताः । इष्ट॑काः । अ॒प॒श्य॒न्न् । ताः । उपेति॑ । अ॒द॒ध॒त॒ । भू॒य॒स्कृदिति॑ भूयः - कृत् । अ॒सि॒ । इति॑ । ए॒व । भूयाꣳ॑सः । अ॒भ॒व॒न्न् । वन॒स्पति॑भि॒रिति॒ वन॒स्पति॑ - भिः॒ । ओष॑धीभि॒रित्योष॑धि - भिः॒ । व॒रि॒व॒स्कृदिति॑ वरिवः - कृत् । अ॒सि॒ । इति॑ । इ॒माम् । अ॒ज॒य॒न्न् । प्राची᳚ । अ॒सि॒ । इति॑ । प्राची᳚म् । दिश᳚म् । अ॒ज॒य॒न्न् । ऊ॒द्‌र्ध्वा । अ॒सि॒ । इति॑ । अ॒मूम् । अ॒ज॒य॒न्न् । अ॒न्त॒रि॒क्ष॒सदित्य॑न्तरिक्ष - सत् । अ॒सि॒ । अ॒न्तरि॑क्षे । सी॒द॒ । इति॑ । अ॒न्तरि॑क्षम् । अ॒ज॒य॒न्न् । ततः॑ । दे॒वाः । अभ॑वन्न् । \textbf{  44} \newline
                  \newline
                                \textbf{ TS 5.3.11.2} \newline
                  परेति॑ । असु॑राः । यस्य॑ । ए॒ताः । उ॒प॒धी॒यन्त॒ इत्यु॑प - धी॒यन्ते᳚ । भूयान्॑ । ए॒व । भ॒व॒ति॒ । अ॒भीति॑ । इ॒मान् । लो॒कान् । ज॒य॒ति॒ । भव॑ति । आ॒त्मना᳚ । परेति॑ । अ॒स्य॒ । भ्रातृ॑व्यः । भ॒व॒ति॒ । अ॒फ्सु॒षदित्य॑फ्सु - सत् । अ॒सि॒ । श्ये॒न॒सदिति॑ श्येन - सत् । अ॒सि॒ । इति॑ । आ॒ह॒ । ए॒तत् । वै । अ॒ग्नेः । रू॒पम् । रू॒पेण॑ । ए॒व । अ॒ग्निम् । अवेति॑ । रु॒न्धे॒ । पृ॒थि॒व्याः । त्वा॒ । द्रवि॑णे । सा॒द॒या॒मि॒ । इति॑ । आ॒ह॒ । इ॒मान् । ए॒व । ए॒ताभिः॑ । लो॒कान् । द्रवि॑णावत॒ इति॒ द्रवि॑ण - व॒तः॒ । कु॒रु॒ते॒ । आ॒यु॒ष्याः᳚ । उपेति॑ । द॒धा॒ति॒ । आयुः॑ । ए॒व । \textbf{  45} \newline
                  \newline
                                \textbf{ TS 5.3.11.3} \newline
                  अ॒स्मि॒न्न् । द॒धा॒ति॒ । अग्ने᳚ । यत् । ते॒ । पर᳚म् । हृत् । नाम॑ । इति॑ । आ॒ह॒ । ए॒तत् । वै । अ॒ग्नेः । प्रि॒यम् । धाम॑ । प्रि॒यम् । ए॒व । अ॒स्य॒ । धाम॑ । उपेति॑ । आ॒प्नो॒ति॒ । तौ । एति॑ । इ॒हि॒ । समिति॑ । र॒भा॒व॒है॒ । इति॑ । आ॒ह॒ । वीति॑ । ए॒व । ए॒ने॒न॒ । परीति॑ । ध॒त्ते॒ । पाञ्च॑जन्ये॒ष्विति॒ पाञ्च॑ - ज॒न्ये॒षु॒ । अपीति॑ । ए॒धि॒ । अ॒ग्ने॒ । इति॑ । आ॒ह॒ । ए॒षः । वै । अ॒ग्निः । पाञ्च॑जन्य॒ इति॒ पाञ्च॑ - ज॒न्यः॒ । यः । पञ्च॑चितीक॒ इति॒ पञ्च॑-चि॒ती॒कः॒ । तस्मा᳚त् । ए॒वम् । आ॒ह॒ । ऋ॒त॒व्याः᳚ । उपेति॑ ( ) । द॒धा॒ति॒ । ए॒तत् । वै । ऋ॒तू॒नाम् । प्रि॒यम् । धाम॑ । यत् । ऋ॒त॒व्याः᳚ । ऋ॒तू॒नाम् । ए॒व । प्रि॒यम् । धाम॑ । अवेति॑ । रु॒न्धे॒ । सु॒मेक॒ इति॑ सु - मेकः॑ । इति॑ । आ॒ह॒ । सं॒ॅव॒थ्स॒र इति॑ सं - व॒थ्स॒रः । वै । सु॒मेक॒ इति॑ सु - मेकः॑ । सं॒ॅव॒थ्स॒रस्येति॑ सं - व॒थ्स॒रस्य॑ । ए॒व । प्रि॒यम् । धाम॑ । उपेति॑ । आ॒प्नो॒ति॒ ॥ \textbf{  46} \newline
                  \newline
                      (अभ॑व॒ - न्नायु॑रे॒॒व - र्त॒व्या॑ उप॒ - षड्विꣳ॑शतिश्च)  \textbf{(A11)} \newline \newline
                                \textbf{ TS 5.3.12.1} \newline
                  प्र॒जाप॑ते॒रिति॑ प्र॒जा - प॒तेः॒ । अक्षि॑ । अ॒श्व॒य॒त् । तत् । परेति॑ । अ॒प॒त॒त् । तत् । अश्वः॑ । अ॒भ॒व॒त् । यत् । अश्व॑यत् । तत् । अश्व॑स्य । अ॒श्व॒त्वमित्य॑श्व- त्वम् । तत् । दे॒वाः । अ॒श्व॒मे॒धेनेत्य॑श्व - मे॒धेन॑ । ए॒व । प्रतीति॑ । अ॒द॒धुः॒ । ए॒षः । वै । प्र॒जाप॑ति॒मिति॑ प्र॒जा - प॒ति॒म् । सर्व᳚म् । क॒रो॒ति॒ । यः । अ॒श्व॒मे॒धेनेत्य॑श्व - मे॒धेन॑ । यज॑ते । सर्वः॑ । ए॒व । भ॒व॒ति॒ । सर्व॑स्य । वै । ए॒षा । प्राय॑श्चित्तिः । सर्व॑स्य । भे॒ष॒जम् । सर्व᳚म् । वै । ए॒तेन॑ । पा॒प्मान᳚म् । दे॒वाः । अ॒त॒र॒न्न् । अपीति॑ । वै । ए॒तेन॑ । ब्र॒ह्म॒ह॒त्यामिति॑ ब्रह्म - ह॒त्याम् । अ॒त॒र॒न्न् । सर्व᳚म् । पा॒प्मान᳚म् । \textbf{  47} \newline
                  \newline
                                \textbf{ TS 5.3.12.2} \newline
                  त॒र॒ति॒ । तर॑ति । ब्र॒ह्म॒ह॒त्यामिति॑ ब्रह्म - ह॒त्याम् । यः । अ॒श्व॒मे॒धेनेत्य॑श्व - मे॒धेन॑ । यज॑ते । यः । उ॒ । च॒ । ए॒न॒म् । ए॒वम् । वेद॑ । उत्त॑र॒मित्युत् - त॒र॒म् । वै । तत् । प्र॒जाप॑ते॒रिति॑ प्र॒जा - प॒तेः॒ । अक्षि॑ । अ॒श्व॒य॒त् । तस्मा᳚त् । अश्व॑स्य । उ॒त्त॒र॒त इत्यु॑त् - त॒र॒तः । अवेति॑ । द्य॒न्ति॒ । द॒क्षि॒ण॒तः । अ॒न्येषा᳚म् । प॒शू॒नाम् । वै॒त॒सः । कटः॑ । भ॒व॒ति॒ । अ॒फ्सुयो॑नि॒रित्य॒फ्सु - यो॒निः॒ । वै । अश्वः॑ । अ॒फ्सु॒ज इत्य॑फ्सु - जः । वे॒त॒सः । स्वे । ए॒व । ए॒न॒म् । योनौ᳚ । प्रतीति॑ । स्था॒प॒य॒ति॒ । च॒तु॒ष्टो॒म इति॑ चतुः - स्तो॒मः । स्तोमः॑ । भ॒व॒ति॒ । स॒रट् । ह॒ । वै । अश्व॑स्य । सक्थि॑ । एति॑ । अ॒वृ॒ह॒त् ( ) । तत् । दे॒वाः । च॒तु॒ष्टो॒मेनेति॑ चतुः-स्तो॒मेन॑ । ए॒व । प्रतीति॑ । अ॒द॒धुः॒ । यत् । च॒तु॒ष्टो॒म इति॑ चतुः - स्तो॒मः । स्तोमः॑ । भव॑ति । अश्व॑स्य । स॒र्व॒त्वायेति॑ सर्व - त्वाय॑ ॥ \textbf{  48 } \newline
                  \newline
                      (सर्व॑म पा॒प्मान॑ - मवृह॒द् - द्वाद॑श च)  \textbf{(A12)} \newline \newline
\textbf{praSna korvai with starting padams of 1 to 12 anuvAkams :-} \newline
(उ॒थ्स॒न्न॒य॒ज्ञ् - इन्द्रा᳚ग्नी - दे॒वा वा अ॑क्ष्णयास्तो॒मीया॑ - अ॒ग्नेर्भा॒गो᳚ - ऽस्यग्ने॑ जा॒तान् - र॒श्मिरिति॑ - नाक॒सद्भिः॒ -छन्दाꣳ॑सि॒ - सर्वा᳚भ्यो - वृष्टि॒सनी᳚ - र्देवासु॒राः कनी॑याꣳसः - प्र॒जाप॑ते॒रक्षि॒ - द्वाद॑श ) \newline

\textbf{korvai with starting padams of1, 11, 21 series of pa~jcAtis :-} \newline
(उ॒थ्स॒न्न॒य॒ज्ञो - दे॒वा वै - यस्य॒ मुख्य॑वती - र्नाक॒सद्भि॑रे॒ - वै ताभि॑र॒ - ष्टाच॑त्वारिꣳशत्) \newline

\textbf{first and last padam of third praSnam of 5th kANDam} \newline
(उ॒थ्स॒न्न॒य॒ज्ञ्ः - स॑र्व॒त्वाय॑) \newline 


॥ हरिः॑ ॐ ॥
॥ कृष्ण यजुर्वेदीय तैत्तिरीय संहितायां पञ्चमकाण्डे तृतीयः प्रश्नः समाप्तः ॥
------------------------------------ \newline
\pagebreak
\pagebreak
        


\end{document}
