\documentclass[17pt]{extarticle}
\usepackage{babel}
\usepackage{fontspec}
\usepackage{polyglossia}
\usepackage{extsizes}



\setmainlanguage{sanskrit}
\setotherlanguages{english} %% or other languages
\setlength{\parindent}{0pt}
\pagestyle{myheadings}
\newfontfamily\devanagarifont[Script=Devanagari]{AdishilaVedic}


\newcommand{\VAR}[1]{}
\newcommand{\BLOCK}[1]{}




\begin{document}
\begin{titlepage}
    \begin{center}
 
\begin{sanskrit}
    { \Large
    ॐ नमः परमात्मने, श्री महागणपतये नमः, श्री गुरुभ्यो नमः
ह॒रिः॒ ॐ 
    }
    \\
    \vspace{2.5cm}
    \mbox{ \Huge
    6.2       षष्ठकाण्डे द्वितीयः प्रश्नः - सोममन्त्रब्राह्मणनिरूपणं   }
\end{sanskrit}
\end{center}

\end{titlepage}
\tableofcontents

ॐ नमः परमात्मने, श्री महागणपतये नमः, श्री गुरुभ्यो नमः
ह॒रिः॒ ॐ \newline
6.2       षष्ठकाण्डे द्वितीयः प्रश्नः - सोममन्त्रब्राह्मणनिरूपणं \newline

\addcontentsline{toc}{section}{ 6.2       षष्ठकाण्डे द्वितीयः प्रश्नः - सोममन्त्रब्राह्मणनिरूपणं}
\markright{ 6.2       षष्ठकाण्डे द्वितीयः प्रश्नः - सोममन्त्रब्राह्मणनिरूपणं \hfill https://www.vedavms.in \hfill}
\section*{ 6.2       षष्ठकाण्डे द्वितीयः प्रश्नः - सोममन्त्रब्राह्मणनिरूपणं }
                                \textbf{ TS 6.2.1.1} \newline
                  यत् । उ॒भौ । वि॒मुच्येति॑ वि-मुच्य॑ । आ॒ति॒थ्यम् । गृ॒ह्णी॒यात् । य॒ज्ञ्म् । वीति॑ । छि॒न्द्या॒त् । यत् । उ॒भौ । अवि॑मु॒च्येत्यवि॑ - मु॒च्य॒ । यथा᳚ । अना॑गता॒येत्यना᳚-ग॒ता॒य॒ । आ॒ति॒थ्यम् । क्रि॒यते᳚ । ता॒दृक् । ए॒व । तत् । विमु॑क्त॒ इति॒ वि - मु॒क्तः॒ । अ॒न्यः । अ॒न॒ड्वान् । भव॑ति । अवि॑मुक्त॒ इत्यवि॑ - मु॒क्तः॒ । अ॒न्यः । अथ॑ । आ॒ति॒थ्यम् । गृ॒ह्णा॒ति॒ । य॒ज्ञ्स्य॑ । सन्त॑त्या॒ इति॒ सं - त॒त्यै॒ । पत्नी᳚ । अ॒न्वार॑भत॒ इत्य॑नु - आर॑भते । पत्नी᳚ । हि । पारी॑णह्य॒स्येति॒ पारि॑ - न॒ह्य॒स्य॒ । ईशे᳚ । पत्नि॑या । ए॒व । अनु॑मत॒मित्यनु॑ - म॒त॒म् । निरिति॑ । व॒प॒ति॒ । यत् । वै । पत्नी᳚ । य॒ज्ञ्स्य॑ । क॒रोति॑ । मि॒थु॒नम् । तत् । अथो॒ इति॑ । पत्नि॑याः । ए॒व । \textbf{  1} \newline
                  \newline
                                \textbf{ TS 6.2.1.2} \newline
                  ए॒षः । य॒ज्ञ्स्य॑ । अ॒न्वा॒र॒म्भ इत्य॑नु -  आ॒र॒म्भः । अन॑वच्छित्त्या॒ इत्यन॑व - छि॒त्त्यै॒ । याव॑द्भि॒रिति॒ याव॑त् - भिः॒ । वै । राजा᳚ । अ॒नु॒च॒रैरित्य॑नु - च॒रैः । आ॒गच्छ॒तीत्या᳚ - गच्छ॑ति । सर्वे᳚भ्यः । वै । तेभ्यः॑ । आ॒ति॒थ्यम् । क्रि॒य॒ते॒ । छन्दाꣳ॑सि । खलु॑ । वै । सोम॑स्य । राज्ञ्ः॑ । अ॒नु॒च॒राणीत्य॑नु - च॒राणि॑ । अ॒ग्नेः । आ॒ति॒थ्यम् । अ॒सि॒ । विष्ण॑वे । त्वा॒ । इति॑ । आ॒ह॒ । गा॒य॒त्रि॒यै । ए॒व । ए॒तेन॑ । क॒रो॒ति॒ । सोम॑स्य । आ॒ति॒थ्यम् । अ॒सि॒ । विष्ण॑वे । त्वा॒ । इति॑ । आ॒ह॒ । त्रि॒ष्टुभे᳚ । ए॒व । ए॒तेन॑ । क॒रो॒ति॒ । अति॑थेः । आ॒ति॒थ्यम् । अ॒सि॒ । विष्ण॑वे । त्वा॒ । इति॑ । आ॒ह॒ । जग॑त्यै । \textbf{  2} \newline
                  \newline
                                \textbf{ TS 6.2.1.3} \newline
                  ए॒व । ए॒तेन॑ । क॒रो॒ति॒ । अ॒ग्नये᳚ । त्वा॒ । रा॒य॒स्पो॒ष॒दाव्न्न॒ इति॑ रायस्पोष - दाव्न्ने᳚ । विष्ण॑वे । त्वा॒ । इति॑ । आ॒ह॒ । अ॒नु॒ष्टुभ॒ इत्य॑नु - स्तुभे᳚ । ए॒व । ए॒तेन॑ । क॒रो॒ति॒ । श्ये॒नाय॑ । त्वा॒ । सो॒म॒भृत॒ इति॑ सोम - भृते᳚ । विष्ण॑वे । त्वा॒ । इति॑ । आ॒ह॒ । गा॒य॒त्रि॒यै । ए॒व । ए॒तेन॑ । क॒रो॒ति॒ । पञ्च॑ । कृत्वः॑ । गृ॒ह्णा॒ति॒ । पञ्चा᳚क्ष॒रेति॒ पञ्च॑ - अ॒क्ष॒रा॒ । प॒ङ्क्तिः । पाङ्क्तः॑ । य॒ज्ञ्ः । य॒ज्ञ्म् । ए॒व । अवेति॑ । रु॒न्धे॒ । ब्र॒ह्म॒वा॒दिन॒ इति॑ ब्रह्म - वा॒दिनः॑ । व॒द॒न्ति॒ । कस्मा᳚त् । स॒त्यात् । गा॒य॒त्रि॒यै । उ॒भ॒यतः॑ । आ॒ति॒थ्यस्य॑ । क्रि॒य॒ते॒ । इति॑ । यत् । ए॒व । अ॒दः । सोम᳚म् । एति॑ । \textbf{  3} \newline
                  \newline
                                \textbf{ TS 6.2.1.4} \newline
                  अह॑रत् । तस्मा᳚त् । गा॒य॒त्रि॒यै । उ॒भ॒यतः॑ । आ॒ति॒थ्यस्य॑ । क्रि॒य॒ते॒ । पु॒रस्ता᳚त् । च॒ । उ॒परि॑ष्टात् । च॒ । शिरः॑ । वै । ए॒तत् । य॒ज्ञ्स्य॑ । यत् । आ॒ति॒थ्यम् । नव॑कपाल॒ इति॒ नव॑ - क॒पा॒लः॒ । पु॒रो॒डाशः॑ । भ॒व॒ति॒ । तस्मा᳚त् । न॒व॒धेति॑ नव - धा । शिरः॑ । विष्यू॑त॒मिति॒ वि - स्यू॒त॒म् । नव॑कपाल॒ इति॒ नव॑ - क॒पा॒लः॒ । पु॒रो॒डाशः॑ । भ॒व॒ति॒ । ते । त्रयः॑ । त्रि॒क॒पा॒ला इति॑ त्रि - क॒पा॒लाः । त्रि॒वृतेति॑ त्रि - वृता᳚ । स्तोमे॑न । संमि॑ता॒ इति॒ सं - मि॒ताः॒ । तेजः॑ । त्रि॒वृदिति॑ त्रि - वृत् । तेजः॑ । ए॒व । य॒ज्ञ्स्य॑ । शी॒र्॒.षन्न् । द॒धा॒ति॒ । नव॑कपाल॒ इति॒ नव॑-क॒पा॒लः॒ । पु॒रो॒डाशः॑ । भ॒व॒ति॒ । ते । त्रयः॑ । त्रि॒क॒पा॒ला इति॑ त्रि - क॒पा॒लाः । त्रि॒वृतेति॑ त्रि-वृता᳚ । प्रा॒णेनेति॑ प्र-अ॒नेन॑ । संमि॑ता॒ इति॒ सं-मि॒ताः॒ । त्रि॒वृदिति॑ त्रि - वृत् । वै । \textbf{  4} \newline
                  \newline
                                \textbf{ TS 6.2.1.5} \newline
                  प्रा॒ण इति॑ प्र -अ॒नः । त्रि॒वृत॒मिति॑ त्रि - वृत᳚म् । ए॒व । प्रा॒णमिति॑ प्र -अ॒नम् । अ॒भि॒पू॒र्वमित्य॑भि - पू॒र्वम् । य॒ज्ञ्स्य॑ । शी॒र्॒.षन्न् । द॒धा॒ति॒ । प्र॒जाप॑ते॒रिति॑ प्र॒जा - प॒तेः॒ । वै । ए॒तानि॑ । पक्ष्मा॑णि । यत् । अ॒श्व॒वा॒ला इत्य॑श्व - वा॒लाः । ऐ॒क्ष॒वी इति॑ । ति॒रश्ची॒ इति॑ । यत् । आश्व॑वाल॒ इत्याश्व॑ - वा॒लः॒ । प्र॒स्त॒र इति॑ प्र - स्त॒रः । भव॑ति । ऐ॒क्ष॒वी इति॑ । ति॒रश्ची॒ इति॑ । प्र॒जाप॑ते॒रिति॑ प्र॒जा -प॒तेः॒ । ए॒व । तत् । चक्षुः॑ । समिति॑ । भ॒र॒ति॒ । दे॒वाः । वै । याः । आहु॑ती॒रित्या - हु॒तीः॒ । अजु॑हवुः । ताः । असु॑राः । नि॒ष्काव᳚म् । आ॒द॒न्न् । ते । दे॒वाः । का॒र्ष्म॒र्य᳚म् । अ॒प॒श्य॒न्न् । क॒र्म॒ण्यः॑ । वै । कर्म॑ । ए॒ने॒न॒ । कु॒र्वी॒त॒ । इति॑ । ते । का॒र्ष्म॒र्य॒मया॒निति॑ कार्ष्मर्य - मयान्॑ । प॒रि॒धीनिति॑ परि - धीन् । \textbf{  5} \newline
                  \newline
                                \textbf{ TS 6.2.1.6} \newline
                  अ॒कु॒र्व॒त॒ । तैः । वै । ते । रक्षाꣳ॑सि । अपेति॑ । अ॒घ्न॒त॒ । यत् । का॒र्ष्म॒र्य॒मया॒ इति॑ कार्ष्मर्य - मयाः᳚ । प॒रि॒धय॒ इति॑ परि - धयः॑ । भव॑न्ति । रक्ष॑साम् । अप॑हत्या॒ इत्यप॑ - ह॒त्यै॒ । समिति॑ । स्प॒र्॒.श॒य॒ति॒ । रक्ष॑साम् । अन॑न्ववचारा॒येत्यन॑नु - अ॒व॒चा॒रा॒य॒ । न । पु॒रस्ता᳚त् । परीति॑ । द॒धा॒ति॒ । आ॒दि॒त्यः । हि । ए॒व । उ॒द्यन्नित्यु॑त्- यन्न् । पु॒रस्ता᳚त् । रक्षाꣳ॑सि । अ॒प॒हन्तीत्य॑प - हन्ति॑ । ऊ॒द्‌र्ध्वे इति॑ । स॒मिधा॒विति॑ सं - इधौ᳚ । एति॑ । द॒धा॒ति॒ । उ॒परि॑ष्टात् । ए॒व । रक्षाꣳ॑सि । अपेति॑ । ह॒न्ति॒ । यजु॑षा । अ॒न्याम् । तू॒ष्णीम् । अ॒न्याम् । मि॒थु॒न॒त्वायेति॑ मिथुन - त्वाय॑ । द्वे इति॑ । एति॑ । द॒धा॒ति॒ । द्वि॒पादिति॑ द्वि - पात् । यज॑मानः । प्रति॑ष्ठित्या॒ इति॒ प्रति॑ - स्थि॒त्यै॒ । ब्र॒ह्म॒वा॒दिन॒ इति॑ ब्रह्म - वा॒दिनः॑ । व॒द॒न्ति॒ । \textbf{  6} \newline
                  \newline
                                \textbf{ TS 6.2.1.7} \newline
                  अ॒ग्निः । च॒ । वै । ए॒तौ । सोमः॑ । च॒ । क॒था । सोमा॑य । आ॒ति॒थ्यम् । क्रि॒यते᳚ । न । अ॒ग्नये᳚ । इति॑ । यत् । अ॒ग्नौ । अ॒ग्निम् । म॒थि॒त्वा । प्र॒हर॒तीति॑ प्र - हर॑ति । तेन॑ । ए॒व । अ॒ग्नये᳚ । आ॒ति॒थ्यम् । क्रि॒य॒ते॒ । अथो॒ इति॑ । खलु॑ । आ॒हुः॒ । अ॒ग्निः । सर्वाः᳚ । दे॒वताः᳚ । इति॑ । यत् । ह॒विः । आ॒साद्येत्या᳚ - साद्य॑ । अ॒ग्निम् । मन्थ॑ति । ह॒व्याय॑ । ए॒व । आस॑न्ना॒येत्या - स॒न्ना॒य॒ । सर्वाः᳚ । दे॒वताः᳚ । ज॒न॒य॒ति॒ ॥ \textbf{  7 } \newline
                  \newline
                      (पत्नि॑या ए॒व - जग॑त्या॒ - आ - त्रि॒वृद्वै - प॑रि॒धीन् - व॑द॒न्त्ये - क॑चत्वारिꣳशच्च)  \textbf{(A1)} \newline \newline
                                \textbf{ TS 6.2.2.1} \newline
                  दे॒वा॒सु॒रा इति॑ देव - अ॒सु॒राः । संॅय॑त्ता॒ इति॒ सं-य॒त्ताः॒ । आ॒स॒न्न् । ते । दे॒वाः । मि॒थः । विप्रि॑या॒ इति॒ वि - प्रि॒याः॒ । आ॒स॒न्न् । ते । अ॒न्यः । अ॒न्यस्मै᳚ । ज्यैष्ठ्या॑य । अति॑ष्ठमानाः । प॒ञ्च॒धेति॑ पञ्च-धा । वीति॑ । अ॒क्रा॒म॒न्न् । अ॒ग्निः । वसु॑भि॒रिति॒ वसु॑ - भिः॒ । सोमः॑ । रु॒द्रैः । इन्द्रः॑ । म॒रुद्भि॒रिति॑ म॒रुत् - भिः॒ । वरु॑णः । आ॒दि॒त्यैः । बृह॒स्पतिः॑ । विश्वैः᳚ । दे॒वैः । ते । अ॒म॒न्य॒न्त॒ । असु॑रेभ्यः । वै । इ॒दम् । भ्रातृ॑व्येभ्यः । र॒द्ध्या॒मः॒ । यत् । मि॒थः । विप्रि॑या॒ इति॒ वि - प्रि॒याः॒ । स्मः । याः । नः॒ । इ॒माः । प्रि॒याः । त॒नुवः॑ । ताः । स॒मव॑द्यामहा॒ इति॑ सं - अव॑द्यामहै । ताभ्यः॑ । सः । निरिति॑ । ऋ॒च्छा॒त् । यः । \textbf{  8} \newline
                  \newline
                                \textbf{ TS 6.2.2.2} \newline
                  नः॒ । प्र॒थ॒मः । अ॒न्यः । अ॒न्यस्मै᳚ । द्रुह्या᳚त् । इति॑ । तस्मा᳚त् । यः । सता॑नूनप्त्रिणा॒मिति॒ स - ता॒नू॒न॒प्त्रि॒णा॒म् । प्र॒थ॒मः । द्रुह्य॑ति । सः । आर्ति᳚म् । एति॑ । ऋ॒च्छ॒ति॒ । यत् । ता॒नू॒न॒प्त्रमिति॑ तानू - न॒प्त्रम् । स॒म॒व॒द्यतीति॑ सं-अ॒व॒द्यति॑ । भ्रातृ॑व्याभिभूत्या॒ इति॒ भ्रातृ॑व्य-अ॒भि॒भू॒त्यै॒ । भव॑ति । आ॒त्मना᳚ । परेति॑ । अ॒स्य॒ । भ्रातृ॑व्यः । भ॒व॒ति॒ । पञ्च॑ । कृत्वः॑ । अवेति॑ । द्य॒ति॒ । प॒ञ्च॒धेति॑ पञ्च - धा । हि । ते । तत् । स॒म॒वाद्य॒न्तेति॑ सं - अ॒वाद्य॑न्त । अथो॒ इति॑ । पञ्चा᳚क्ष॒रेति॒ पञ्च॑ - अ॒क्ष॒रा॒ । प॒ङ्क्तिः । पाङ्क्तः॑ । य॒ज्ञ्ः । य॒ज्ञ्म् । ए॒व । अवेति॑ । रु॒न्धे॒ । आप॑तय॒ इत्या - प॒त॒ये॒ । त्वा॒ । गृ॒ह्णा॒मि॒ । इति॑ । आ॒ह॒ । प्रा॒ण इति॑ प्र - अ॒नः । वै । \textbf{  9} \newline
                  \newline
                                \textbf{ TS 6.2.2.3} \newline
                  आप॑ति॒रित्या - प॒तिः॒ । प्रा॒णमिति॑ प्र - अ॒नम् । ए॒व । प्री॒णा॒ति॒ । परि॑पतय॒ इति॒ परि॑ - प॒त॒ये॒ । इति॑ । आ॒ह॒ । मनः॑ । वै । परि॑पति॒रिति॒ परि॑ - प॒तिः॒ । मनः॑ । ए॒व । प्री॒णा॒ति॒ । तनू॒नप्त्र॒ इति॒ तनू᳚ - नप्त्रे᳚ । इति॑ । आ॒ह॒ । त॒नुवः॑ । हि । ते । ताः । स॒म॒वाद्य॒न्तेति॑ सं- अ॒वाद्य॑न्त । शा॒क्व॒राय॑ । इति॑ । आ॒ह॒ । शक्त्यै᳚ । हि । ते । ताः । स॒म॒वाद्य॒न्तेति॑ सं-अ॒वाद्य॑न्त । शक्मन्न्॑ । ओजि॑ष्ठाय । इति॑ । आ॒ह॒ । ओजि॑ष्ठम् । हि । ते । तत् । आ॒त्मनः॑ । स॒म॒वाद्य॒न्तेति॑ सं-अ॒वाद्य॑न्त । अना॑धृष्ट॒मित्यना᳚ - धृ॒ष्ट॒म् । अ॒सि॒ । अ॒ना॒धृ॒ष्यमित्य॑ना - धृ॒ष्यम् । इति॑ । आ॒ह॒ । अना॑धृष्ट॒मित्यना᳚ - धृ॒ष्ट॒म् । हि । ए॒तत् । अ॒ना॒धृ॒ष्यमित्य॑ना - धृ॒ष्यम् । दे॒वाना᳚म् । ओजः॑ । \textbf{  10} \newline
                  \newline
                                \textbf{ TS 6.2.2.4} \newline
                  इति॑ । आ॒ह॒ । दे॒वाना᳚म् । हि । ए॒तत् । ओजः॑ । अ॒भि॒श॒स्ति॒पा इत्य॑भिशस्ति-पाः । अ॒न॒भि॒श॒स्ते॒न्यमित्य॑नभि - श॒स्ते॒न्यम् । इति॑ । आ॒ह॒ । अ॒भि॒श॒स्ति॒पा इत्य॑भिशस्ति - पाः । हि । ए॒तत् । अ॒न॒भि॒श॒स्ते॒न्यमित्य॑नभि - श॒स्ते॒न्यम् । अन्विति॑ । मे॒ । दी॒क्षाम् । दी॒क्षाप॑ति॒रिति॑ दी॒क्षा - प॒तिः॒ । म॒न्य॒ता॒म् । इति॑ । आ॒ह॒ । य॒था॒य॒जुरिति॑ यथा - य॒जुः । ए॒व । ए॒तत् । घृ॒तम् । वै । दे॒वाः । वज्र᳚म् । कृ॒त्वा । सोम᳚म् । अ॒घ्न॒न्न् । अ॒न्ति॒कम् । इ॒व॒ । खलु॑ । वै । अ॒स्य॒ । ए॒तत् । च॒र॒न्ति॒ । यत् । ता॒नू॒न॒प्त्रेणेति॑ तानू - न॒प्त्रेण॑ । प्र॒चर॒न्तीति॑ प्र - चर॑न्ति । अꣳ॒॒शुरꣳ॑शु॒रित्यꣳ॒॒शुः-अꣳ॒॒शुः॒ । ते॒ । दे॒व॒ । सो॒म॒ । एति॑ । प्या॒य॒ता॒म् । इति॑ । आ॒ह॒ । यत् । \textbf{  11} \newline
                  \newline
                                \textbf{ TS 6.2.2.5} \newline
                  ए॒व । अ॒स्य॒ । अ॒पु॒वा॒यते᳚ । यत् । मीय॑ते । तत् । ए॒व । अ॒स्य॒ । ए॒तेन॑ । एति॑ । प्या॒य॒य॒ति॒ । एति॑ । तुभ्य᳚म् । इन्द्रः॑ । प्या॒य॒ता॒म् । एति॑ । त्वम् । इन्द्रा॑य । प्या॒य॒स्व॒ । इति॑ । आ॒ह॒ । उ॒भौ । ए॒व । इन्द्र᳚म् । च॒ । सोम᳚म् । च॒ । एति॑ । प्या॒य॒य॒ति॒ । एति॑ । प्या॒य॒य॒ । सखीन्॑ । स॒न्या । मे॒धया᳚ । इति॑ । आ॒ह॒ । ऋ॒त्विजः॑ । वै । अ॒स्य॒ । सखा॑यः । तान् । ए॒व । एति॑ । प्या॒य॒य॒ति॒ । स्व॒स्ति । ते॒ । दे॒व॒ । सो॒म॒ । सु॒त्याम् । अ॒शी॒य॒ । \textbf{  12} \newline
                  \newline
                                \textbf{ TS 6.2.2.6} \newline
                  इति॑ । आ॒ह॒ । आ॒शिष॒मित्या᳚-शिष᳚म् । ए॒व । ए॒ताम् । एति॑ । शा॒स्ते॒ । प्रेति॑ । वै । ए॒ते । अ॒स्मात् । लो॒कात् । च्य॒व॒न्ते॒ । ये । सोम᳚म् । आ॒प्या॒यय॒न्तीत्या᳚-प्या॒यय॑न्ति । अ॒न्त॒रि॒क्ष॒दे॒व॒त्य॑ इत्य॑न्तरिक्ष-दे॒व॒त्यः॑ । हि । सोमः॑ । आप्या॑यित॒ इत्या - प्या॒यि॒तः॒ । एष्टः॑ । रायः॑ । प्रेति॑ । इ॒षे । भगा॑य । इति॑ । आ॒ह॒ । द्यावा॑पृथि॒वीभ्या॒मिति॒ द्यावा᳚ - पृ॒थि॒वीभ्या᳚म् । ए॒व । न॒म॒स्कृत्येति॑ नमः - कृत्य॑ । अ॒स्मिन्न् । लो॒के । प्रतीति॑ । ति॒ष्ठ॒न्ति॒ । दे॒वा॒सु॒रा इति॑ देव - अ॒सु॒राः । संॅय॑त्ता॒ इति॒ सं - य॒त्ताः॒ । आ॒स॒न्न् । ते । दे॒वाः । बिभ्य॑तः । अ॒ग्निम् । प्रेति॑ । अ॒वि॒श॒न्न् । तस्मा᳚त् । आ॒हुः॒ । अ॒ग्निः । सर्वाः᳚ । दे॒वताः᳚ । इति॑ । ते । \textbf{  13} \newline
                  \newline
                                \textbf{ TS 6.2.2.7} \newline
                  अ॒ग्निम् । ए॒व । वरू॑थम् । कृ॒त्वा । असु॑रान् । अ॒भीति॑ । अ॒भ॒व॒न्न् । अ॒ग्निम् । इ॒व॒ । खलु॑ । वै । ए॒षः । प्रेति॑ । वि॒श॒ति॒ । यः । अ॒वा॒न्त॒र॒दी॒क्षामित्य॑वान्तर - दी॒क्षाम् । उ॒पैतीत्यु॑प - एति॑ । भ्रातृ॑व्याभिभूत्या॒ इति॒ भ्रातृ॑व्य - अ॒भि॒भू॒त्यै॒ । भव॑ति । आ॒त्मना᳚ । परेति॑ । अ॒स्य॒ । भ्रातृ॑व्यः । भ॒व॒ति॒ । आ॒त्मान᳚म् । ए॒व । दी॒क्षया᳚ । पा॒ति॒ । प्र॒जामिति॑ प्र - जाम् । अ॒वा॒न्त॒र॒दी॒क्षयेत्य॑वान्तर - दी॒क्षया᳚ । स॒न्त॒रामिति॑ सं - त॒राम् । मेख॑लाम् । स॒माय॑च्छत॒ इति॑ सं - आय॑च्छते । प्रजेति॑ प्र - जा । हि । आ॒त्मनः॑ । अन्त॑रत॒रेत्यन्त॑र - त॒रा॒ । त॒प्तव्र॑त॒ इति॑ त॒प्त - व्र॒तः॒ । भ॒व॒ति॒ । मद॑न्तीभिः । मा॒र्ज॒य॒ते॒ । निरिति॑ । हि । अ॒ग्निः । शी॒तेन॑ । वाय॑ति । समि॑द्ध्या॒ इति॒ सं-इ॒द्ध्यै॒ । या । ते॒ । अ॒ग्ने॒ ( ) । रुद्रि॑या । त॒नूः । इति॑ । आ॒ह॒ । स्वया᳚ । ए॒व । ए॒न॒त् । दे॒वत॑या । व्र॒त॒य॒ति॒ । स॒यो॒नि॒त्वायेति॑ सयोनि - त्वाय॑ । शान्त्यै᳚ ॥ \textbf{  14 } \newline
                  \newline
                      (यो - वा - ओज॑ - आह॒ य - द॑शी॒ये - ति॒ ते᳚ - ऽग्न॒ - एका॑दश च)  \textbf{(A2)} \newline \newline
                                \textbf{ TS 6.2.3.1} \newline
                  तेषा᳚म् । असु॑राणाम् । ति॒स्रः । पुरः॑ । आ॒स॒न्न् । अ॒य॒स्मयी᳚ । अ॒व॒मा । अथ॑ । र॒ज॒ता । अथ॑ । हरि॑णी । ताः । दे॒वाः । जेतु᳚म् । न । अ॒श॒क्नु॒व॒न्न् । ताः । उ॒प॒सदेत्यु॑प - सदा᳚ । ए॒व । अ॒जि॒गी॒ष॒न्न् । तस्मा᳚त् । आ॒हुः॒ । यः । च॒ । ए॒वम् । वेद॑ । यः । च॒ । न । उ॒प॒सदेत्यु॑प - सदा᳚ । वै । म॒हा॒पु॒रमिति॑ महा - पु॒रम् । ज॒य॒न्ति॒ । इति॑ । ते । इषु᳚म् । समिति॑ । अ॒कु॒र्व॒त॒ । अ॒ग्निम् । अनी॑कम् । सोम᳚म् । श॒ल्यम् । विष्णु᳚म् । तेज॑नम् । ते । अ॒ब्रु॒व॒न्न् । कः । इ॒माम् । अ॒सि॒ष्य॒ति॒ । इति॑ । \textbf{  15} \newline
                  \newline
                                \textbf{ TS 6.2.3.2} \newline
                  रु॒द्रः । इति॑ । अ॒ब्रु॒व॒न्न् । रु॒द्रः । वै । क्रू॒रः । सः । अ॒स्य॒तु॒ । इति॑ । सः । अ॒ब्र॒वी॒त् । वर᳚म् । वृ॒णै॒ । अ॒हम् । ए॒व । प॒शू॒नाम् । अधि॑पति॒रित्यधि॑ - प॒तिः॒ । अ॒सा॒नि॒ । इति॑ । तस्मा᳚त् । रु॒द्रः । प॒शू॒नाम् । अधि॑पति॒रित्यधि॑ - प॒तिः॒ । ताम् । रु॒द्रः । अवेति॑ । अ॒सृ॒ज॒त् । सः । ति॒स्रः । पुरः॑ । भि॒त्वा । ए॒भ्यः । लो॒केभ्यः॑ । असु॑रान् । प्रेति॑ । अ॒नु॒द॒त॒ । यत् । उ॒प॒सद॒ इत्यु॑प - सदः॑ । उ॒प॒स॒द्यन्त॒ इत्यु॑प-स॒द्यन्ते᳚ । भ्रातृ॑व्यपराणुत्या॒ इति॒ भ्रातृ॑व्य-प॒रा॒णु॒त्यै॒ । न । अ॒न्याम् । आहु॑ति॒मित्या - हु॒ति॒म् । पु॒रस्ता᳚त् । जु॒हु॒या॒त् । यत् । अ॒न्याम् । आहु॑ति॒मित्या - हु॒ति॒म् । पु॒रस्ता᳚त् । जु॒हु॒यात् । \textbf{  16} \newline
                  \newline
                                \textbf{ TS 6.2.3.3} \newline
                  अ॒न्यत् । मुख᳚म् । कु॒र्या॒त् । स्रु॒वेण॑ । आ॒घा॒रमित्या᳚ - घा॒रम् । एति॑ । घा॒र॒य॒ति॒ । य॒ज्ञ्स्य॑ । प्रज्ञा᳚त्या॒ इति॒ प्र - ज्ञा॒त्यै॒ । पराङ्॑ । अ॒ति॒क्रम्येत्य॑ति - क्रम्य॑ । जु॒हो॒ति॒ । परा॑चः । ए॒व । ए॒भ्यः । लो॒केभ्यः॑ । यज॑मानः । भ्रातृ॑व्यान् । प्रेति॑ । नु॒द॒ते॒ । पुनः॑ । अ॒त्या॒क्रम्येत्य॑ति - आ॒क्रम्य॑ । उ॒प॒सद॒मित्यु॑प - सद᳚म् । जु॒हो॒ति॒ । प्र॒णुद्येति॑ प्र - नुद्य॑ । ए॒व । ए॒भ्यः । लो॒केभ्यः॑ । भ्रातृ॑व्यान् । जि॒त्वा । भ्रा॒तृ॒व्य॒लो॒कमिति॑ भ्रातृव्य - लो॒कम् । अ॒भ्यारो॑ह॒तीत्य॑भि - आरो॑हति । दे॒वाः । वै । याः । प्रा॒तः । उ॒प॒सद॒ इत्यु॑प - सदः॑ । उ॒पासी॑द॒न्नित्यु॑प - असी॑दन्न् । अह्नः॑ । ताभिः॑ । असु॑रान् । प्रेति॑ । अ॒नु॒द॒न्त॒ । याः । सा॒यम् । रात्रि॑यै । ताभिः॑ । यत् । सा॒यम्प्रा॑त॒रिति॑ सा॒यम् - प्रा॒तः॒ । उ॒प॒सद॒ इत्यु॑प - सदः॑ । \textbf{  17} \newline
                  \newline
                                \textbf{ TS 6.2.3.4} \newline
                  उ॒प॒स॒द्यन्त॒ इत्यु॑प - स॒द्यन्ते᳚ । अ॒हो॒रा॒त्राभ्या॒मित्य॑हः - रा॒त्राभ्या᳚म् । ए॒व । तत् । यज॑मानः । भ्रातृ॑व्यान् । प्रेति॑ । नु॒द॒ते॒ । याः । प्रा॒तः । या॒ज्याः᳚ । स्युः । ताः । सा॒यम् । पु॒रो॒नु॒वा॒क्या॑ इति॑ पुरः - अ॒नु॒वा॒क्याः᳚ । कु॒र्या॒त् । अया॑तयामत्वा॒येत्यया॑तयाम - त्वा॒य॒ । ति॒स्रः । उ॒प॒सद॒ इत्यु॑प - सदः॑ । उपेति॑ । ए॒ति॒ । त्रयः॑ । इ॒मे । लो॒काः । इ॒मान् । ए॒व । लो॒कान् । प्री॒णा॒ति॒ । षट् । समिति॑ । प॒द्य॒न्ते॒ । षट् । वै । ऋ॒तवः॑ । ऋ॒तून् । ए॒व । प्री॒णा॒ति॒ । द्वाद॑श । अ॒हीने᳚ । सोमे᳚ । उपेति॑ । ए॒ति॒ । द्वाद॑श । मासाः᳚ । सं॒ॅव॒थ्स॒र इति॑ सं - व॒थ्स॒रः । सं॒ॅव॒थ्स॒रमिति॑ सं - व॒थ्स॒रम् । ए॒व । प्री॒णा॒ति॒ । चतु॑र्विꣳशति॒रिति॒ चतुः॑ - विꣳ॒॒श॒तिः॒ । समिति॑ । \textbf{  18} \newline
                  \newline
                                \textbf{ TS 6.2.3.5} \newline
                  प॒द्य॒न्ते॒ । चतु॑र्विꣳशति॒रिति॒ चतुः॑ - विꣳ॒॒श॒तिः॒ । अ॒द्‌र्ध॒मा॒सा इत्य॑द्‌र्ध - मा॒साः । अ॒द्‌र्ध॒मा॒सानित्य॑र्ध - मा॒सान् । ए॒व । प्री॒णा॒ति॒ । आरा᳚ग्रा॒मित्यारा᳚ - अ॒ग्रा॒म् । अ॒वा॒न्त॒र॒दी॒क्षामित्य॑वान्तर - दी॒क्षाम् । उपेति॑ । इ॒या॒त् । यः । का॒मये॑त । अ॒स्मिन्न् । मे॒ । लो॒के । अद्‌र्धु॑कम् । स्या॒त् । इति॑ । एक᳚म् । अग्रे᳚ । अथ॑ । द्वौ । अथ॑ । त्रीन् । अथ॑ । च॒तुरः॑ । ए॒षा । वै । आरा॒ग्रेत्यारा᳚ - अ॒ग्रा॒ । अ॒वा॒न्त॒र॒दी॒क्षेत्य॑वान्तर - दी॒क्षा । अ॒स्मिन्न् । ए॒व । अ॒स्मै॒ । लो॒के । अद्‌र्धु॑कम् । भ॒व॒ति॒ । प॒रोव॑रीयसी॒मिति॑ प॒रः - व॒री॒य॒सी॒म् । अ॒वा॒न्त॒र॒दी॒क्षामित्य॑वान्तर - दी॒क्षाम् । उपेति॑ । इ॒या॒त् । यः । का॒मये॑त । अ॒मुष्मिन्न्॑ । मे॒ । लो॒के । अद्‌र्धु॑कम् । स्या॒त् । इति॑ । च॒तुरः॑ । अग्रे᳚ ( ) । अथ॑ । त्रीन् । अथ॑ । द्वौ । अथ॑ । एक᳚म् । ए॒षा । वै । प॒रोव॑रीय॒सीति॑ प॒रः - व॒री॒य॒सि॒ । अ॒वा॒न्त॒र॒दी॒क्षेत्य॑वान्तर - दी॒क्षा । अ॒मुष्मिन्न्॑ । ए॒व । अ॒स्मै॒ । लो॒के । अद्‌र्धु॑कम् । भ॒व॒ति॒ ॥ \textbf{  19} \newline
                  \newline
                      (अ॒सि॒ष्य॒तीति॑ - जुहु॒याथ् - सा॒यं प्रा॑तरुप॒सद॒ - श्चतु॑र्विꣳशतिः॒ सं - च॒तुरोऽग्रे॒ - षोड॑श च)  \textbf{(A3)} \newline \newline
                                \textbf{ TS 6.2.4.1} \newline
                  सु॒व॒र्गमिति॑ सुवः - गम् । वै । ए॒ते । लो॒कम् । य॒न्ति॒ । ये । उ॒प॒सद॒ इत्यु॑प - सदः॑ । उ॒प॒यन्तीयु॑प - यन्ति॑ । तेषा᳚म् । यः । उ॒न्नय॑त॒ इत्यु॑त् - नय॑ते । हीय॑ते । ए॒व । सः । न । उदिति॑ । अ॒ने॒षि॒ । इति॑ । सू᳚न्नीय॒मिति॒ सु - उ॒न्नी॒य॒म् । इ॒व॒ । यः । वै । स्वा॒र्थेता॒मिति॑ स्वार्थ-इता᳚म् । य॒ताम् । श्रा॒न्तः । हीय॑ते । उ॒त । सः । नि॒ष्ट्याय॑ । स॒ह । व॒स॒ति॒ । तस्मा᳚त् । स॒कृत् । उ॒न्नीयेत्यु॑त्-नीय॑ । न । अप॑रम् । उदिति॑ । न॒ये॒त॒ । द॒द्ध्ना । उदिति॑ । न॒ये॒त॒ । ए॒तत् । वै । प॒शू॒नाम् । रू॒पम् । रू॒पेण॑ । ए॒व । प॒शून् । अवेति॑ । रु॒न्धे॒ । \textbf{  20} \newline
                  \newline
                                \textbf{ TS 6.2.4.2} \newline
                  य॒ज्ञ्ः । दे॒वेभ्यः॑ । निला॑यत । विष्णुः॑ । रू॒पम् । कृ॒त्वा । सः । पृ॒थि॒वीम् । प्रेति॑ । अ॒वि॒श॒त् । तम् । दे॒वाः । हस्तान्॑ । सꣳ॒॒रभ्येति॑ सं - रभ्य॑ । ऐ॒च्छ॒न्न् । तम् । इन्द्रः॑ । उ॒पर्यु॑प॒रीत्यु॒परि॑ - उ॒प॒रि॒ । अतीति॑ । अ॒क्रा॒म॒त् । सः । अ॒ब्र॒वी॒त् । कः । मा॒ । अ॒यम् । उ॒पर्यु॑प॒रीत्यु॒परि॑ - उ॒प॒रि॒ । अतीति॑ । अ॒क्र॒मी॒त् । इति॑ । अ॒हम् । दु॒र्ग इति॑ दुः - गे । हन्ता᳚ । इति॑ । अथ॑ । कः । त्वम् । इति॑ । अ॒हम् । दु॒र्गादिति॑ दुः - गात् । आह॒र्तेत्या - ह॒र्ता॒ । इति॑ । सः । अ॒ब्र॒वी॒त् । दु॒र्ग इति॑ दुः - गे । वै । हन्ता᳚ । अ॒वो॒च॒थाः॒ । व॒रा॒हः । अ॒यम् । वा॒म॒मो॒ष इति॑ वाम - मो॒षः । \textbf{  21} \newline
                  \newline
                                \textbf{ TS 6.2.4.3} \newline
                  स॒प्ता॒नाम् । गि॒री॒णाम् । प॒रस्ता᳚त् । वि॒त्तम् । वेद्य᳚म् । असु॑राणाम् । बि॒भ॒र्ति॒ । तम् । ज॒हि॒ । यदि॑ । दु॒र्ग इति॑ दुः - गे । हन्ता᳚ । असि॑ । इति॑ । सः । द॒र्भ॒पु॒ञ्जी॒लमिति॑ दर्भ - पु॒ञ्जी॒लम् । उ॒द्वृह्येत्यु॑त् - वृह्य॑ । स॒प्त । गि॒रीन् । भि॒त्वा । तम् । अ॒ह॒न्न् । सः । अ॒ब्र॒वी॒त् । दु॒र्गादिति॑ दुः - गात् । वै । आह॒र्तेत्या - ह॒र्ता॒ । अ॒वो॒च॒थाः॒ । ए॒तम् । एति॑ । ह॒र॒ । इति॑ । तम् । ए॒भ्यः॒ । य॒ज्ञ्ः । ए॒व । य॒ज्ञ्म् । एति॑ । अ॒ह॒र॒त् । यत् । तत् । वि॒त्तम् । वेद्य᳚म् । असु॑राणाम् । अवि॑न्दन्त । तत् । एक᳚म् । वेद्यै᳚ । वे॒दि॒त्वमिति॑ वेदि - त्वम् । असु॑राणाम् । \textbf{  22} \newline
                  \newline
                                \textbf{ TS 6.2.4.4} \newline
                  वै । इ॒यम् । अग्रे᳚ । आ॒सी॒त् । याव॑त् । आसी॑नः । प॒रा॒पश्य॒तीति॑ परा - पश्य॑ति । ताव॑त् । दे॒वाना᳚म् । ते । दे॒वाः । अ॒ब्रु॒व॒न्न् । अस्तु॑ । ए॒व । नः॒ । अ॒स्याम् । अपीति॑ । इति॑ । किय॑त् । वः॒ । दा॒स्या॒मः॒ । इति॑ । याव॑त् । इ॒यम् । स॒ला॒वृ॒की । त्रिः । प॒रि॒क्राम॒तीति॑ परि-क्राम॑ति । ताव॑त् । नः॒ । द॒त्त॒ । इति॑ । सः । इन्द्रः॑ । स॒ला॒वृ॒की । रू॒पम् । कृ॒त्वा । इ॒माम् । त्रिः । स॒र्वतः॑ । परीति॑ । अ॒क्रा॒म॒त् । तत् । इ॒माम् । अ॒वि॒न्द॒न्त॒ । यत् । इ॒माम् । अवि॑न्दन्त । तत् । वेद्यै᳚ । वे॒दि॒त्वमिति॑ वेदि - त्वम् । \textbf{  23} \newline
                  \newline
                                \textbf{ TS 6.2.4.5} \newline
                  सा । वै । इ॒यम् । सर्वा᳚ । ए॒व । वेदिः॑ । इय॑ति । श॒क्ष्या॒मि॒ । इति॑ । तु । वै । अ॒व॒मायेत्य॑व - माय॑ । य॒ज॒न्ते॒ । त्रिꣳ॒॒शत् । प॒दानि॑ । प॒श्चात् । ति॒रश्ची᳚ । भ॒व॒ति॒ । षट्त्रिꣳ॑श॒दिति॒ षट् - त्रिꣳ॒॒श॒त् । प्राची᳚ । चतु॑र्विꣳशति॒रिति॒ चतुः॑ - विꣳ॒॒श॒तिः॒ । पु॒रस्ता᳚त् । ति॒रश्ची᳚ । दश॑द॒शेति॒ दश॑-द॒श॒ । समिति॑ । प॒द्य॒न्ते॒ । दशा᳚क्ष॒रेति॒ दश॑ - अ॒क्ष॒रा॒ । वि॒राडिति॑ वि - राट् । अन्न᳚म् । वि॒राडिति॑ वि - राट् । वि॒राजेति॑ वि - राजा᳚ । ए॒व । अ॒न्नाद्य॒मित्य॑न्न - अद्य᳚म् । अवेति॑ । रु॒न्धे॒ । उदिति॑ । ह॒न्ति॒ । यत् । ए॒व । अ॒स्याः॒ । अ॒मे॒द्ध्यम् । तत् । अपेति॑ । ह॒न्ति॒ । उदिति॑ । ह॒न्ति॒ । तस्मा᳚त् । ओष॑धयः । परेति॑ । भ॒व॒न्ति॒ ( ) । ब॒र्॒.हिः । स्तृ॒णा॒ति॒ । तस्मा᳚त् । ओष॑धयः । पुनः॑ । एति॑ । भ॒व॒न्ति॒ । उत्त॑र॒मित्युत् - त॒र॒म् । ब॒र्॒.हिषः॑ । उ॒त्त॒र॒ब॒र्॒.हिरित्यु॑त्तर - ब॒र्॒.हिः । स्तृ॒णा॒ति॒ । प्र॒जा इति॑ प्र - जाः । वै । ब॒र्॒.हिः । यज॑मानः । उ॒त्त॒र॒ब॒र॒.हिरित्यु॑त्तर - ब॒र॒.हिः । यज॑मानम् । ए॒व । अय॑जमानात् । उत्त॑र॒मित्युत् - त॒र॒म् । क॒रो॒ति॒ । तस्मा᳚त् । यज॑मानः । अय॑जमानात् । उत्त॑र॒ इत्युत् - त॒रः॒ ॥ \textbf{  24} \newline
                  \newline
                      (रु॒न्धे॒ - वा॒म॒मो॒षो - वे॑दि॒त्वमसु॑राणां - ॅवेदि॒त्वं - भ॑वन्ति॒ - पञ्च॑विꣳशतिश्च)  \textbf{(A4)} \newline \newline
                                \textbf{ TS 6.2.5.1} \newline
                  यत् । वै । अनी॑शानः । भा॒रम् । आ॒द॒त्त इत्या᳚ - द॒त्ते । वीति॑ । वै । सः । लि॒श॒ते॒ । यत् । द्वाद॑श । सा॒ह्नस्येति॑ स - अ॒ह्नस्य॑ । उ॒प॒सद॒ इत्यु॑प - सदः॑ । स्युः । ति॒स्त्रः । अ॒हीन॑स्य । य॒ज्ञ्स्य॑ । विलो॒मेति॒ वि - लो॒म॒ । क्रि॒ये॒त॒ । ति॒स्रः । ए॒व । सा॒ह्नस्येति॑ स - अ॒ह्नस्य॑ । उ॒प॒सद॒ इत्यु॑प - सदः॑ । द्वाद॑श । अ॒हीन॑स्य । य॒ज्ञ्स्य॑ । स॒वी॒र्य॒त्वायेति॑ सवीर्य - त्वाय॑ । अथो॒ इति॑ । सलो॒मेति॒ स-लो॒म॒ । क्रि॒य॒ते॒ । व॒थ्सस्य॑ । एकः॑ । स्तनः॑ । भा॒गी । हि । सः । अथ॑ । एक᳚म् । स्तन᳚म् । व्र॒तम् । उपेति॑ । ए॒ति॒ । अथ॑ । द्वौ । अथ॑ । त्रीन् । अथ॑ । च॒तुरः॑ । ए॒तत् । वै । \textbf{  25} \newline
                  \newline
                                \textbf{ TS 6.2.5.2} \newline
                  क्षु॒रप॒वीति॑ क्षु॒र - प॒वि॒ । नाम॑ । व्र॒तम् । येन॑ । प्रेति॑ । जा॒तान् । भ्रातृ॑व्यान् । नु॒दते᳚ । प्रतीति॑ । ज॒नि॒ष्यमा॑णान् । अथो॒ इति॑ । कनी॑यसा । ए॒व । भूयः॑ । उपेति॑ । ए॒ति॒ । च॒तुरः॑ । अग्रे᳚ । स्तनान्॑ । व्र॒तम् । उपेति॑ । ए॒ति॒ । अथ॑ । त्रीन् । अथ॑ । द्वौ । अथ॑ । एक᳚म् । ए॒तत् । वै । सु॒ज॒घ॒नमिति॑ सु - ज॒घ॒नम् । नाम॑ । व्र॒तम् । त॒प॒स्य᳚म् । सु॒व॒र्ग्य॑मिति॑ सुवः - ग्य᳚म् । अथो॒ इति॑ । प्रेति॑ । ए॒व । जा॒य॒ते॒ । प्र॒जयेति॑ प्र - जया᳚ । प॒शुभि॒रिति॑ प॒शु - भिः॒ । य॒वा॒गूः । रा॒ज॒न्य॑स्य । व्र॒तम् । क्रू॒रा । इ॒व॒ । वै । य॒वा॒गूः । क्रू॒रः । इ॒व॒ । \textbf{  26} \newline
                  \newline
                                \textbf{ TS 6.2.5.3} \newline
                  रा॒ज॒न्यः॑ । वज्र॑स्य । रू॒पम् । समृ॑द्ध्या॒ इति॒ सं-ऋ॒द्ध्यै॒ । आ॒मिक्षा᳚ । वैश्य॑स्य । पा॒क॒य॒ज्ञ्स्येति॑ पाक - य॒ज्ञ्स्य॑ । रू॒पम् । पुष्ट्यै᳚ । पयः॑ । ब्रा॒ह्म॒णस्य॑ । तेजः॑ । वै । ब्रा॒ह्म॒णः । तेजः॑ । पयः॑ । तेज॑सा । ए॒व । तेजः॑ । पयः॑ । आ॒त्मन्न् । ध॒त्ते॒ । अथो॒ इति॑ । पय॑सा । वै । गर्भाः᳚ । व॒द्‌र्ध॒न्ते॒ । गर्भः॑ । इ॒व॒ । खलु॑ । वै । ए॒षः । यत् । दी॒क्षि॒तः । यत् । अ॒स्य॒ । पयः॑ । व्र॒तम् । भव॑ति । आ॒त्मान᳚म् । ए॒व । तत् । व॒द्‌र्ध॒य॒ति॒ । त्रिव्र॑त॒ इति॒ त्रि - व्र॒तः॒ । वै । मनुः॑ । आ॒सी॒त् । द्विव्र॑ता॒ इति॒ द्वि - व्र॒ताः॒ । असु॑राः । एक॑व्रता॒ इत्येक॑ - व्र॒ताः॒ । \textbf{  27} \newline
                  \newline
                                \textbf{ TS 6.2.5.4} \newline
                  दे॒वाः । प्रा॒तः । म॒द्ध्यन्दि॑ने । सा॒यम् । तत् । मनोः᳚ । व्र॒तम् । आ॒सी॒त् । पा॒क॒य॒ज्ञ्स्येति॑ पाक - य॒ज्ञ्स्य॑ । रू॒पम् । पुष्ट्यै᳚ । प्रा॒तः । च॒ । सा॒यम् । च॒ । असु॑राणाम् । नि॒र्म॒द्ध्यमिति॑ निः-म॒द्ध्यम् । क्षु॒धः । रू॒पम् । ततः॑ । ते । परेति॑ । अ॒भ॒व॒न्न् । म॒द्ध्यन्दि॑ने । म॒द्ध्य॒रा॒त्र इति॑ मद्ध्य - रा॒त्रे । दे॒वाना᳚म् । ततः॑ । ते । अ॒भ॒व॒न्न् । सु॒व॒र्गमिति॑ सुवः-गम् । लो॒कम् । आ॒य॒न्न् । यत् । अ॒स्य॒ । म॒द्ध्यन्दि॑ने । म॒द्ध्य॒रा॒त्र इति॑ मद्ध्य-रा॒त्रे । व्र॒तम् । भव॑ति । म॒द्ध्य॒तः । वै । अन्ने॑न । भु॒ञ्ज॒ते॒ । म॒द्ध्य॒तः । ए॒व । तत् । ऊर्ज᳚म् । ध॒त्ते॒ । भ्रातृ॑व्याभिभूत्या॒ इति॒ भ्रातृ॑व्य - अ॒भि॒भू॒त्यै॒ । भव॑ति । आ॒त्मना᳚ । \textbf{  28} \newline
                  \newline
                                \textbf{ TS 6.2.5.5} \newline
                  परेति॑ । अ॒स्य॒ । भ्रातृ॑व्यः । भ॒व॒ति॒ । गर्भः॑ । वै । ए॒षः । यत् । दी॒क्षि॒तः । योनिः॑ । दी॒क्षि॒त॒वि॒मि॒तमिति॑ दीक्षित - वि॒मि॒तम् । यत् । दी॒क्षि॒तः । दी॒क्षि॒त॒वि॒मि॒तादिति॑ दीक्षित - वि॒मि॒तात् । प्र॒वसे॒दिति॑ प्र - वसे᳚त् । यथा᳚ । योनेः᳚ । गर्भः॑ । स्कन्द॑ति । ता॒दृक् । ए॒व । तत् । न । प्र॒व॒स्त॒व्य॑मिति॑ प्र - व॒स्त॒व्य᳚म् । आ॒त्मनः॑ । गो॒पी॒थाय॑ । ए॒षः । वै । व्या॒घ्रः । कु॒ल॒गो॒प इति॑ कुल - गो॒पः । यत् । अ॒ग्निः । तस्मा᳚त् । यत् । दी॒क्षि॒तः । प्र॒वसे॒दिति॑ प्र - वसे᳚त् । सः । ए॒न॒म् । ई॒श्व॒रः । अ॒नू॒त्थायेत्य॑नु - उ॒त्थाय॑ । हन्तोः᳚ । न । प्र॒व॒स्त॒व्य॑मिति॑ प्र - व॒स्त॒व्य᳚म् । आ॒त्मनः॑ । गुप्त्यै᳚ । द॒क्षि॒ण॒तः । श॒ये॒ । ए॒तत् । वै । यज॑मानस्य ( ) । आ॒यत॑न॒मित्या᳚ - यत॑नम् । स्वे । ए॒व । आ॒यत॑न॒ इत्या᳚-यत॑ने । श॒ये॒ । अ॒ग्निम् । अ॒भ्या॒वृत्येत्य॑भि-आ॒वृत्य॑ । श॒ये॒ । दे॒वताः᳚ । ए॒व । य॒ज्ञ्म् । अ॒भ्या॒वृत्येत्य॑भि -आ॒वृत्य॑ । श॒ये॒ ॥ \textbf{  29 } \newline
                  \newline
                      (ए॒तद्वै-क्रू॒र इ॒वै-क॑व्रता-आ॒त्मना॒-यज॑मानस्य॒-त्रयो॑दश च)  \textbf{(A5)} \newline \newline
                                \textbf{ TS 6.2.6.1} \newline
                  पु॒रोह॑वि॒षीति॑ पु॒रः-ह॒वि॒षि॒ । दे॒व॒यज॑न॒ इति॑ देव-यज॑ने । या॒ज॒ये॒त् । यम् । का॒मये॑त । उपेति॑ । ए॒न॒म् । उत्त॑र॒ इत्युत् - त॒रः॒ । य॒ज्ञ्ः । न॒मे॒त् । अ॒भीति॑ । सु॒व॒र्गमिति॑ सुवः-गम् । लो॒कम् । ज॒ये॒त् । इति॑ । ए॒तत् । वै । पु॒रोह॑वि॒रिति॑ पु॒रः-ह॒विः॒ । दे॒व॒यज॑न॒मिति॑ देव-यज॑नम् । यस्य॑ । होता᳚ । प्रा॒त॒र॒नु॒वा॒कमिति॑ प्रातः-अ॒नु॒वा॒कम् । अ॒नु॒ब्रु॒वन्नित्य॑नु - ब्रु॒वन्न् । अ॒ग्निम् । अ॒पः । आ॒दि॒त्यम् । अ॒भीति॑ । वि॒पश्य॒तीति॑ वि-पश्य॑ति । उपेति॑ । ए॒न॒म् । उत्त॑र॒ इत्युत् - त॒रः॒ । य॒ज्ञ्ः । न॒म॒ति॒ । अ॒भीति॑ । सु॒व॒र्गमिति॑ सुवः - गम् । लो॒कम् । ज॒य॒ति॒ । आ॒प्ते । दे॒व॒यज॑न॒ इति॑ देव-यज॑ने । या॒ज॒ये॒त् । भ्रातृ॑व्यवन्त॒मिति॒ भ्रातृ॑व्य - व॒न्त॒म् । पन्था᳚म् । वा॒ । अ॒धि॒स्प॒र्॒.॒शये॒दित्य॑धि-स्प॒र्॒.शये᳚त् । क॒र्तम् । वा॒ । याव॑त् । न । अन॑से । यात॒वै । \textbf{  30} \newline
                  \newline
                                \textbf{ TS 6.2.6.2} \newline
                  न । रथा॑य । ए॒तत् । वै । आ॒प्तम् । दे॒व॒यज॑न॒मिति॑ देव - यज॑नम् । आ॒प्नोति॑ । ए॒व । भ्रातृ॑व्यम् । न । ए॒न॒म् । भ्रातृ॑व्यः । आ॒प्नो॒ति॒ । एको᳚न्नत॒ इत्येक॑-उ॒न्न॒ते॒ । दे॒व॒यज॑न॒ इति॑ देव - यज॑ने । या॒ज॒ये॒त् । प॒शुका॑म॒मिति॑ प॒शु - का॒म॒म् । एको᳚न्नता॒दित्येक॑ - उ॒न्न॒ता॒त् । वै । दे॒व॒यज॑ना॒दिति॑ देव - यज॑नात् । अङ्गि॑रसः । प॒शून् । अ॒सृ॒ज॒न्त॒ । अ॒न्त॒रा । स॒दो॒ह॒वि॒द्‌र्धा॒ने इति॑ सदः-ह॒वि॒द्‌र्धा॒ने । उ॒न्न॒तमित्यु॑त् - न॒तम् । स्या॒त् । ए॒तत् । वै । एको᳚न्नत॒मित्येक॑ - उ॒न्न॒त॒म् । दे॒व॒यज॑न॒मिति॑ देव - यज॑नम् । प॒शु॒मानिति॑ पशु - मान् । ए॒व । भ॒व॒ति॒ । त्र्यु॑न्नत॒ इति॒ त्रि - उ॒न्न॒ते॒ । दे॒व॒यज॑न॒ इति॑ देव - यज॑ने । या॒ज॒ये॒त् । सु॒व॒र्गका॑म॒मिति॑ सुव॒र्ग - का॒म॒म् । त्र्यु॑न्नता॒दिति॒ त्रि - उ॒न्न॒ता॒त् । वै । दे॒व॒यज॑ना॒दिति॑ देव - यज॑नात् । अङ्गि॑रसः । सु॒व॒र्गमिति॑ सुवः - गम् । लो॒कम् । आ॒य॒न्न् । अ॒न्त॒रा । आ॒ह॒व॒नीय॒मित्या᳚-ह॒व॒नीय᳚म् । च॒ । ह॒वि॒द्‌र्धान॒मिति॑ हविः - धान᳚म् । च॒ । \textbf{  31} \newline
                  \newline
                                \textbf{ TS 6.2.6.3} \newline
                  उ॒न्न॒तमित्यु॑त् - न॒तम् । स्या॒त् । अ॒न्त॒रा । ह॒वि॒द्‌र्धान॒मिति॑ हविः - धान᳚म् । च॒ । सदः॑ । च॒ । अ॒न्त॒रा । सदः॑ । च॒ । गार्.ह॑पत्य॒मिति॒ गार्.ह॑ - प॒त्य॒म् । च॒ । ए॒तत् । वै । त्र्यु॑न्नत॒मिति॒ त्रि - उ॒न्न॒त॒म् । दे॒व॒यज॑न॒मिति॑ देव - यज॑नम् । सु॒व॒र्गमिति॑ सुवः - गम् । ए॒व । लो॒कम् । ए॒ति॒ । प्रति॑ष्ठित॒ इति॒ प्रति॑ - स्थि॒ते॒ । दे॒व॒यज॑न॒ इति॑ देव - यज॑ने । या॒ज॒ये॒त् । प्र॒ति॒ष्ठाका॑म॒मिति॑ प्रति॒ष्ठा - का॒म॒म् । ए॒तत् । वै । प्रति॑ष्ठित॒मिति॒ प्रति॑ - स्थि॒तम्॒ । दे॒व॒यज॑न॒मिति॑ देव - यज॑नम् । यत् । स॒र्वतः॑ । स॒मम् । प्रतीति॑ । ए॒व । ति॒ष्ठ॒ति॒ । यत्र॑ । अ॒न्या‌अ॑न्या॒ इत्य॒न्याः - अ॒न्याः॒ । ओष॑धयः । व्यति॑षक्ता॒ इति॑ वि - अति॑षक्ताः । स्युः । तत् । या॒ज॒ये॒त् । प॒शुका॑म॒मिति॑ प॒शु - का॒म॒म् । ए॒तत् । वै । प॒शू॒नाम् । रू॒पम् । रू॒पेण॑ । ए॒व । अ॒स्मै॒ । प॒शून् । \textbf{  32} \newline
                  \newline
                                \textbf{ TS 6.2.6.4} \newline
                  अवेति॑ । रु॒न्धे॒ । प॒शु॒मानिति॑ पशु - मान् । ए॒व । भ॒व॒ति॒ । निर्.ऋ॑तिगृहीत॒ इति॒ निर्.ऋ॑ति - गृ॒ही॒ते॒ । दे॒व॒यज॑न॒ इति॑ देव-यज॑ने । या॒ज॒ये॒त् । यम् । का॒मये॑त । निर्.ऋ॒त्येति॒ निः - ऋ॒त्या॒ । अ॒स्य॒ । य॒ज्ञ्म् । ग्रा॒ह॒ये॒य॒म् । इति॑ । ए॒तत् । वै । निर्.ऋ॑तिगृहीत॒मिति॒ निर्.ऋ॑ति - गृ॒ही॒त॒म् । दे॒व॒यज॑न॒मिति॑ देव - यज॑नम् । यत् । स॒दृश्यै᳚ । स॒त्याः᳚ । ऋ॒क्षम् । निर्.ऋ॒त्येति॒ निः - ऋ॒त्या॒ । ए॒व । अ॒स्य॒ । य॒ज्ञ्म् । ग्रा॒ह॒य॒ति॒ । व्यावृ॑त्त॒ इति॑ वि-आवृ॑त्ते । दे॒व॒यज॑न॒ इति॑ देव - यज॑ने । या॒ज॒ये॒त् । व्या॒वृत्का॑म॒मिति॑ व्या॒वृत्-का॒म॒म् । यम् । पात्रे᳚ । वा॒ । तल्पे᳚ । वा॒ । मीमाꣳ॑सेरन्न् । प्रा॒चीन᳚म् । आ॒ह॒व॒नीया॒दित्या᳚ -ह॒व॒नीया᳚त् । प्र॒व॒णमिति॑ प्र - व॒नम् । स्या॒त् । प्र॒ती॒चीन᳚म् । गार्.ह॑पत्या॒दिति॒ गार्.ह॑ - प॒त्या॒त् । ए॒तत् । वै । व्यावृ॑त्त॒मिति॑ वि - आवृ॑त्तम् । दे॒व॒यज॑न॒मिति॑ देव - यज॑नम् । वीति॑ । पा॒प्मना᳚ ( ) । भ्रातृ॑व्येण । एति॑ । व॒र्त॒ते॒ । न । ए॒न॒म् । पात्रे᳚ । न । तल्पे᳚ । मी॒माꣳ॒॒स॒न्ते॒ । का॒र्ये᳚ । दे॒व॒यज॑न॒ इति॑ देव-यज॑ने । या॒ज॒ये॒त् । भूति॑काम॒मिति॒ भूति॑ - का॒म॒म् । का॒र्यः॑ । वै । पुरु॑षः । भव॑ति । ए॒व ॥ \textbf{  33} \newline
                  \newline
                      (यात॒वै - ह॑वि॒र्धानं॑ च - प॒शून् - पा॒प्मना॒ - ऽष्टाद॑श च)  \textbf{(A6)} \newline \newline
                                \textbf{ TS 6.2.7.1} \newline
                  तेभ्यः॑ । उ॒त्त॒र॒वे॒दिरित्यु॑त्तर - वे॒दिः । सिꣳ॒॒हीः । रू॒पम् । कृ॒त्वा । उ॒भयान्॑ । अ॒न्त॒रा । अ॒प॒क्रम्येत्य॑प - क्रम्य॑ । अ॒ति॒ष्ठ॒त् । ते । दे॒वाः । अ॒म॒न्य॒न्त॒ । य॒त॒रान् । वै । इ॒यम् । उ॒पा॒व॒र्थ्स्यतीत्यु॑प-आ॒व॒र्थ्स्यति॑ । ते । इ॒दम् । भ॒वि॒ष्य॒न्ति॒ । इति॑ । ताम् । उपेति॑ । अ॒म॒न्त्र॒य॒न्त॒ । सा । अ॒ब्र॒वी॒त् । वर᳚म् । वृ॒णै॒ । सर्वान्॑ । मया᳚ । कामान्॑ । वीति॑ । अ॒श्न॒व॒थ॒ । पूर्वा᳚म् । तु । मा॒ । अ॒ग्नेः । आहु॑ति॒रित्या - हु॒तिः॒ । अ॒श्न॒व॒तै॒ । इति॑ । तस्मा᳚त् । उ॒त्त॒र॒वे॒दिमित्यु॑त्तर - वे॒दिम् । पूर्वा᳚म् । अ॒ग्नेः । व्याघा॑रय॒न्तीति॑ वि-आघा॑रयन्ति । वारे॑वृत॒मिति॒ वारे᳚-वृ॒त॒म् । हि । अ॒स्यै॒ । शम्य॑या । परीति॑ । मि॒मी॒ते॒ । \textbf{  34} \newline
                  \newline
                                \textbf{ TS 6.2.7.2} \newline
                  मात्रा᳚ । ए॒व । अ॒स्यै॒ । सा । अथो॒ इति॑ । यु॒क्तेन॑ । ए॒व । यु॒क्तम् । अवेति॑ । रु॒न्धे॒ । वि॒त्ताय॒नीति॑ वित्त - अय॑नी । मे॒ । अ॒सि॒ । इति॑ । आ॒ह॒ । वि॒त्ता । हि । ए॒ना॒न् । आव॑त् । ति॒क्ताय॒नीति॑ तिक्त - अय॑नी । मे॒ । अ॒सि॒ । इति॑ । आ॒ह॒ । ति॒क्तान् । हि । ए॒ना॒न् । आव॑त् । अव॑तात् । मा॒ । ना॒थि॒तम् । इति॑ । आ॒ह॒ । ना॒थि॒तान् । हि । ए॒ना॒न् । आव॑त् । अव॑तात् । मा॒ । व्य॒थि॒तम् । इति॑ । आ॒ह॒ । व्य॒थि॒तान् । हि । ए॒ना॒न् । आव॑त् । वि॒देः । अ॒ग्निः । नभः॑ । नाम॑ । \textbf{  35} \newline
                  \newline
                                \textbf{ TS 6.2.7.3} \newline
                  अग्ने᳚ । अ॒ङ्गि॒रः॒ । इति॑ । त्रिः । ह॒र॒ति॒ । ये । ए॒व । ए॒षु । लो॒केषु॑ । अ॒ग्नयः॑ । तान् । ए॒व । अवेति॑ । रु॒न्धे॒ । तू॒ष्णीम् । च॒तु॒र्थम् । ह॒र॒ति॒ । अनि॑रुक्त॒मित्यनिः॑ - उ॒क्त॒म् । ए॒व । अवेति॑ । रु॒न्धे॒ । सिꣳ॒॒हीः । अ॒सि॒ । म॒हि॒षीः । अ॒सि॒ । इति॑ । आ॒ह॒ । सिꣳ॒॒हीः । हि । ए॒षा । रू॒पम् । कृ॒त्वा । उ॒भयान्॑ । अ॒न्त॒रा । अ॒प॒क्रम्येत्य॑प - क्रम्य॑ । अति॑ष्ठत् । उ॒रु । प्र॒थ॒स्व॒ । उ॒रु । ते॒ । य॒ज्ञ्प॑ति॒रिति॑ य॒ज्ञ् - प॒तिः॒ । प्र॒थ॒ता॒म् । इति॑ । आ॒ह॒ । यज॑मानम् । ए॒व । प्र॒जयेति॑ प्र - जया᳚ । प॒शुभि॒रिति॑ प॒शु - भिः॒ । प्र॒थ॒य॒ति॒ । ध्रु॒वा । \textbf{  36} \newline
                  \newline
                                \textbf{ TS 6.2.7.4} \newline
                  अ॒सि॒ । इति॑ । समिति॑ । ह॒न्ति॒ । धृत्यै᳚ । दे॒वेभ्यः॑ । शु॒न्ध॒स्व॒ । दे॒वेभ्यः॑ । शु॒म्भ॒स्व॒ । इति॑ । अवेति॑ । च॒ । उ॒क्षति॑ । प्रेति॑ । च॒ । कि॒र॒ति॒ । शुद्ध्यै᳚ । इ॒न्द्र॒घो॒ष इती᳚न्द्र - घो॒षः । त्वा॒ । वसु॑भि॒रिति॒ वसु॑ - भिः॒ । पु॒रस्ता᳚त् । पा॒तु॒ । इति॑ । आ॒ह॒ । दि॒ग्भ्य इति॑ दिक् - भ्यः । ए॒व । ए॒ना॒म् । प्रेति॑ । उ॒क्ष॒ति॒ । दे॒वान् । च॒ । इत् । उ॒त्त॒र॒वे॒दिरित्यु॑त्तर - वे॒दिः । उ॒पाव॑व॒र्तीत्यु॑प - आव॑वर्ति । इ॒ह । ए॒व । वीति॑ । ज॒या॒म॒है॒ । इति॑ । असु॑राः । वज्र᳚म् । उ॒द्यत्येत्यु॑त् - यत्य॑ । दे॒वान् । अ॒भीति॑ । आ॒य॒न्त॒ । तान् । इ॒न्द्र॒घो॒ष इती᳚न्द्र - घो॒षः । वसु॑भि॒रिति॒ वसु॑ - भिः॒ । पु॒रस्ता᳚त् । अपेति॑ । \textbf{  37} \newline
                  \newline
                                \textbf{ TS 6.2.7.5} \newline
                  अ॒नु॒द॒त॒ । मनो॑जवा॒ इति॒ मनः॑ - ज॒वाः॒ । पि॒तृभि॒रिति॑ पि॒तृ - भिः॒ । द॒क्षि॒ण॒तः । प्रचे॑ता॒ इति॒ प्र - चे॒ताः॒ । रु॒द्रैः । प॒श्चात् । वि॒श्वक॒र्मेति॑ वि॒श्व - क॒र्मा॒ । आ॒दि॒त्यैः । उ॒त्त॒र॒त इत्यु॑त् - त॒र॒तः । यत् । ए॒वम् । उ॒त्त॒र॒वे॒दिमित्यु॑त्तर - वे॒दिम् । प्रो॒क्षतीति॑ प्र - उ॒क्षति॑ । दि॒ग्भ्य इति॑ दिक् - भ्यः । ए॒व । तत् । यज॑मानः । भ्रातृ॑व्यान् । प्रेति॑ । नु॒द॒ते॒ । इन्द्रः॑ । यतीन्॑ । सा॒ला॒वृ॒केभ्यः॑ । प्रेति॑ । अ॒य॒च्छ॒त् । तान् । द॒क्षि॒ण॒तः । उ॒त्त॒र॒वे॒द्या इत्यु॑त्तर - वे॒द्याः । आ॒द॒न्न् । यत् । प्रोक्ष॑णीना॒मिति॑ प्र - उक्ष॑णीनाम् । उ॒च्छिष्ये॒तेत्यु॑त् - शिष्ये॑त । तत् । द॒क्षि॒ण॒तः । उ॒त्त॒र॒वे॒द्या इत्यु॑त्तर - वे॒द्यै । नीति॑ । न॒ये॒त् । यत् । ए॒व । तत्र॑ । क्रू॒रम् । तत् । तेन॑ । श॒म॒य॒ति॒ । यम् । द्वि॒ष्यात् । तम् । ध्या॒ये॒त् । शु॒चा ( ) । ए॒व । ए॒न॒म् । अ॒र्प॒य॒ति॒ ॥(मि॒मी॒ते॒ - नाम॑ - ध्रु॒वा - ऽप॑ - शु॒चा - त्रीणि॑ च) ( आ7) \textbf{  38} \newline
                  \newline
                       \textbf{} \newline \newline
                                \textbf{ TS 6.2.8.1} \newline
                  सा । उ॒त्त॒र॒वे॒दिरित्यु॑त्तर-वे॒दिः । अ॒ब्र॒वी॒त् । सर्वान्॑ । मया᳚ । कामान्॑ । वीति॑ । अ॒श्न॒व॒थ॒ । इति॑ । ते । दे॒वाः । अ॒का॒म॒य॒न्त॒ । असु॑रान् । भ्रातृ॑व्यान् । अ॒भीति॑ । भ॒वे॒म॒ । इति॑ । ते । अ॒जु॒ह॒वुः॒ । सिꣳ॒॒हीः । अ॒सि॒ । स॒प॒त्न॒सा॒हीति॑ सपत्न - सा॒ही । स्वाहा᳚ । इति॑ । ते । असु॑रान् । भ्रातृ॑व्यान् । अ॒भीति॑ । अ॒भ॒व॒न्न् । ते । असु॑रान् । भ्रातृ॑व्यान् । अ॒भि॒भूयेत्य॑भि - भूय॑ । अ॒का॒म॒य॒न्त॒ । प्र॒जामिति॑ प्र-जाम् । वि॒न्दे॒म॒हि॒ । इति॑ । ते । अ॒जु॒ह॒वुः॒ । सिꣳ॒॒हीः । अ॒सि॒ । सु॒प्र॒जा॒वनि॒रिति॑ सुप्रजा - वनिः॑ । स्वाहा᳚ । इति॑ । ते । प्र॒जामिति॑ प्र-जाम् । अ॒वि॒न्द॒न्त॒ । ते । प्र॒जामिति॑ प्र - जाम् । वि॒त्त्वा । \textbf{  39} \newline
                  \newline
                                \textbf{ TS 6.2.8.2} \newline
                  अ॒का॒म॒य॒न्त॒ । प॒शून् । वि॒न्दे॒म॒हि॒ । इति॑ । ते । अ॒जु॒ह॒वुः॒ । सिꣳ॒॒हीः । अ॒सि॒ । रा॒य॒स्पो॒ष॒वनि॒रिति॑ रायस्पोष - वनिः॑ । स्वाहा᳚ । इति॑ । ते । प॒शून् । अ॒वि॒न्द॒न्त॒ । ते । प॒शून् । वि॒त्त्वा । अ॒का॒म॒य॒न्त॒ । प्र॒ति॒ष्ठामिति॑ प्रति - स्थाम् । वि॒न्दे॒म॒हि॒ । इति॑ । ते । अ॒जु॒ह॒वुः॒ । सिꣳ॒॒हीः । अ॒सि॒ । आ॒दि॒त्य॒वनि॒रित्या॑दित्य - वनिः॑ । स्वाहा᳚ । इति॑ । ते । इ॒माम् । प्र॒ति॒ष्ठामिति॑ प्रति - स्थाम् । अ॒वि॒न्द॒न्त॒ । ते । इ॒माम् । प्र॒ति॒ष्ठामिति॑ प्रति - स्थाम् । वि॒त्त्वा । अ॒का॒म॒य॒न्त॒ । दे॒वताः᳚ । आ॒शिष॒ इत्या᳚ - शिषः॑ । उपेति॑ । इ॒या॒म॒ । इति॑ । ते । अ॒जु॒ह॒वुः॒ । सिꣳ॒॒हीः । अ॒सि॒ । एति॑ । व॒ह॒ । दे॒वान् । दे॒व॒य॒त इति॑ देव - य॒ते । \textbf{  40} \newline
                  \newline
                                \textbf{ TS 6.2.8.3} \newline
                  यज॑मानाय । स्वाहा᳚ । इति॑ । ते । दे॒वताः᳚ । आ॒शिष॒ इत्या᳚ - शिषः॑ । उपेति॑ । आ॒य॒न्न् । पञ्च॑ । कृत्वः॑ । व्याघा॑रय॒तीति॑ वि - आघा॑रयति । पञ्चा᳚क्ष॒रेति॒ पञ्च॑-अ॒क्ष॒रा॒ । प॒ङ्क्तिः । पाङ्क्तः॑ । य॒ज्ञ्ः । य॒ज्ञ्म् । ए॒व । अवेति॑ । रु॒न्धे॒ । अ॒क्ष्ण॒या । व्याघा॑रय॒तीति॑ वि-आघा॑रयति । तस्मा᳚त् । अ॒क्ष्ण॒या । प॒शवः॑ । अङ्गा॑नि । प्रेति॑ । ह॒र॒न्ति॒ । प्रति॑ष्ठित्या॒ इति॒ प्रति॑ - स्थि॒त्यै॒ । भू॒तेभ्यः॑ । त्वा॒ । इति॑ । स्रुच᳚म् । उदिति॑ । गृ॒ह्णा॒ति॒ । ये । ए॒व । दे॒वाः । भू॒ताः । तेषा᳚म् । तत् । भा॒ग॒धेय॒मिति॑ भाग-धेय᳚म् । तान् । ए॒व । तेन॑ । प्री॒णा॒ति॒ । पौतु॑द्रवान् । प॒रि॒धीनिति॑ परि - धीन् । परीति॑ । द॒धा॒ति॒ । ए॒षाम् । \textbf{  41} \newline
                  \newline
                                \textbf{ TS 6.2.8.4} \newline
                  लो॒काना᳚म् । विधृ॑त्या॒ इति॒ वि-धृ॒त्यै॒ । अ॒ग्नेः । त्रयः॑ । ज्यायाꣳ॑सः । भ्रात॑रः । आ॒स॒न्न् । ते । दे॒वेभ्यः॑ । ह॒व्यम् । वह॑न्तः । प्रेति॑ । अ॒मी॒य॒न्त॒ । सः । अ॒ग्निः । अ॒बि॒भे॒त् । इ॒त्थम् । वाव । स्यः । आर्ति᳚म् । एति॑ । अ॒रि॒ष्य॒ति॒ । इति॑ । सः । निला॑यत । सः । याम् । वन॒स्पति॑षु । अव॑सत् । ताम् । पूतु॑द्रौ । याम् । ओष॑धीषु । ताम् । सु॒ग॒न्धि॒तेज॑न॒ इति॑ सुगन्धि - तेज॑ने । याम् । प॒शुषु॑ । ताम् । पेत्व॑स्य । अ॒न्त॒रा । शृङ्गे॒ इति॑ । तम् । दे॒वताः᳚ । प्रैष॒मिति॑ प्र -एष᳚म् । ऐ॒च्छ॒न्न् । तम् । अन्विति॑ । अ॒वि॒न्द॒न्न् । तम् । अ॒ब्रु॒व॒न्न् । \textbf{  42} \newline
                  \newline
                                \textbf{ TS 6.2.8.5} \newline
                  उपेति॑ । नः॒ । एति॑ । व॒र्त॒स्व॒ । ह॒व्यम् । नः॒ । व॒ह॒ । इति॑ । सः । अ॒ब्र॒वी॒त् । वर᳚म् । वृ॒णै॒ । यत् । ए॒व । गृ॒ही॒तस्य॑ । अहु॑तस्य । ब॒हिः॒प॒रि॒धीति॑ बहिः - प॒रि॒धि । स्कन्दा᳚त् । तत् । मे॒ । भ्रातृ॑णाम् । भा॒ग॒धेय॒मिति॑ भाग - धेय᳚म् । अ॒स॒त् । इति॑ । तस्मा᳚त् । यत् । गृ॒ही॒तस्य॑ । अहु॑तस्य । ब॒हिः॒प॒रि॒धीति॑ बहिः - प॒रि॒धि । स्कन्द॑ति । तेषा᳚म् । तत् । भा॒ग॒धेय॒मिति॑ भाग - धेय᳚म् । तान् । ए॒व । तेन॑ । प्री॒णा॒ति॒ । सः । अ॒म॒न्य॒त॒ । अ॒स्थ॒न्वन्त॒ इत्य॑स्थन्न् - वन्तः॑ । मे॒ । पूर्वे᳚ । भ्रात॑रः । प्रेति॑ । अ॒मे॒ष॒त॒ । अ॒स्थानि॑ । शा॒त॒यै॒ । इति॑ । सः । यानि॑ । \textbf{  43} \newline
                  \newline
                                \textbf{ TS 6.2.8.6} \newline
                  अ॒स्थानि॑ । अशा॑तयत । तत् । पूतु॑द्रु । अ॒भ॒व॒त् । यत् । माꣳ॒॒सम् । उप॑मृत॒मित्युप॑ - मृ॒त॒म् । तत् । गुल्गु॑लु । यत् । ए॒तान् । स॒भां॒रानिति॑ सं - भा॒रान् । स॒म्भर॒तीति॑ सं-भर॑ति । अ॒ग्निम् । ए॒व । तत् । समिति॑ । भ॒र॒ति॒ । अ॒ग्नेः । पुरी॑षम् । अ॒सि॒ । इति॑ । आ॒ह॒ । अ॒ग्नेः । हि । ए॒तत् । पुरी॑षम् । यत् । स॒भां॒रा इति॑ सं - भा॒राः । अथो॒ इति॑ । खलु॑ । आ॒हुः॒ । ए॒ते । वाव । ए॒न॒म् । ते । भ्रात॑रः । परीति॑ । शे॒रे॒ । यत् । पौतु॑द्रवाः । प॒रि॒धय॒ इति॑ परि-धयः॑ । इति॑ ॥ \textbf{  44 } \newline
                  \newline
                      (वि॒त्त्वा - दे॑वय॒त - ए॒षा - म॑ब्रुव॒न् - यानि॒ - चतु॑श्चत्वारिꣳशच्च)  \textbf{(A8)} \newline \newline
                                \textbf{ TS 6.2.9.1} \newline
                  ब॒द्धम् । अवेति॑ । स्य॒ति॒ । व॒रु॒ण॒पा॒शादिति॑ वरुण - पा॒शात् । ए॒व । ए॒ने॒ इति॑ । मु॒ञ्च॒ति॒ । प्रेति॑ । ने॒ने॒क्ति॒ । मेद्ध्ये॒ इति॑ । ए॒व । ए॒ने॒ इति॑ । क॒रो॒ति॒ । सा॒वि॒त्रि॒या । ऋ॒चा । हु॒त्वा । ह॒वि॒द्‌र्धाने॒ इति॑ हविः-धाने᳚ । प्रेति॑ । व॒र्त॒य॒ति॒ । स॒वि॒तृप्र॑सूत॒ इति॑ सवि॒तृ - प्र॒सू॒तः॒ । ए॒व । ए॒ने॒ इति॑ । प्रेति॑ । व॒र्त॒य॒ति॒ । वरु॑णः । वै । ए॒षः । दु॒र्वागिति॑ दुः - वाक् । उ॒भ॒यतः॑ । ब॒द्धः । यत् । अक्षः॑ । सः । यत् । उ॒थ्सर्जे॒दित्यु॑त् - सर्जे᳚त् । यज॑मानस्य । गृ॒हान् । अ॒भ्युथ्स॑र्जे॒दित्य॑भि - उथ्स॑र्जेत् । सु॒वागिति॑ सु - वाक् । दे॒व॒ । दुर्यान्॑ । एति॑ । व॒द॒ । इति॑ । आ॒ह॒ । गृ॒हाः । वै । दुर्याः᳚ । शान्त्यै᳚ । पत्नी᳚ । \textbf{  45} \newline
                  \newline
                                \textbf{ TS 6.2.9.2} \newline
                  उपेति॑ । अ॒न॒क्ति॒ । पत्नी᳚ । हि । सर्व॑स्य । मि॒त्रम् । मि॒त्र॒त्वायेति॑ मित्र - त्वाय॑ । यत् । वै । पत्नी᳚ । य॒ज्ञ्स्य॑ । क॒रोति॑ । मि॒थु॒नम् । तत् । अथो॒ इति॑ । पत्नि॑याः । ए॒व । ए॒षः । य॒ज्ञ्स्य॑ । अ॒न्वा॒र॒भं इत्य॑नु - आ॒र॒भंः । अन॑वच्छित्त्या॒ इत्यन॑व - छि॒त्त्यै॒ । वर्त्म॑ना । वै । अ॒न्वित्येत्य॑नु - इत्य॑ । य॒ज्ञ्म् । रक्षाꣳ॑सि । जि॒घाꣳ॒॒स॒न्ति॒ । वै॒ष्ण॒वीभ्या᳚म् । ऋ॒ग्भ्यामित्यृ॑क्-भ्याम् । वर्त्म॑नोः । जु॒हो॒ति॒ । य॒ज्ञ्ः । वै । विष्णुः॑ । य॒ज्ञात् । ए॒व । रक्षाꣳ॑सि । अपेति॑ । ह॒न्ति॒ । यत् । अ॒द्ध्व॒र्युः । अ॒न॒ग्नौ । आहु॑ति॒मित्या - हु॒ति॒म् । जु॒हु॒यात् । अ॒न्धः । अ॒द्ध्व॒र्युः । स्या॒त् । रक्षाꣳ॑सि । य॒ज्ञ्म् । ह॒न्युः॒ । \textbf{  46} \newline
                  \newline
                                \textbf{ TS 6.2.9.3} \newline
                  हिर॑ण्यम् । उ॒पास्येत्यु॑प - अस्य॑ । जु॒हो॒ति॒ । अ॒ग्नि॒वतीत्य॑ग्नि-वति॑ । ए॒व । जु॒हो॒ति॒ । न । अ॒न्धः । अ॒द्ध्व॒र्युः । भव॑ति । न । य॒ज्ञ्म् । रक्षाꣳ॑सि । घ्न॒न्ति॒ । प्राची॒ इति॑ । प्रेति॑ । इ॒त॒म् । अ॒द्ध्व॒रम् । क॒ल्प॑यन्ती॒ इति॑ । इति॑ । आ॒ह॒ । सु॒व॒र्गमिति॑ सुवः - गम् । ए॒व । ए॒ने॒ इति॑ । लो॒कम् । ग॒म॒य॒ति॒ । अत्र॑ । र॒मे॒था॒म् । वर्ष्मन्न्॑ । पृ॒थि॒व्याः । इति॑ । आ॒ह॒ । वर्ष्म॑ । हि । ए॒तत् । पृ॒थि॒व्याः । यत् । दे॒व॒यज॑न॒मिति॑ देव - यज॑नम् । शिरः॑ । वै । ए॒तत् । य॒ज्ञ्स्य॑ । यत् । ह॒वि॒द्‌र्धान॒मिति॑ हविः-धान᳚म् । दि॒वः । वा॒ । वि॒ष्णो॒ । उ॒त । वा॒ । पृ॒थि॒व्याः । \textbf{  47} \newline
                  \newline
                                \textbf{ TS 6.2.9.4} \newline
                  इति॑ । आ॒शीर्प॑द॒येत्या॒शीः - प॒द॒या॒ । ऋ॒चा । दक्षि॑णस्य । ह॒वि॒द्‌र्धान॒स्येति॑ हविः - धान॑स्य । मे॒थीम् । नीति॑ । ह॒न्ति॒ । शी॒र्॒.ष॒तः । ए॒व । य॒ज्ञ्स्य॑ । यज॑मानः । आ॒शिष॒ इत्या᳚ - शिषः॑ । अवेति॑ । रु॒न्धे॒ । द॒ण्डः । वै । औ॒प॒रः । तृ॒तीय॑स्य । ह॒वि॒द्‌र्धान॒स्येति॑ हविः - धान॑स्य । व॒ष॒ट्का॒रेणेति॑ वषट् - का॒रेण॑ । अक्ष᳚म् । अ॒च्छि॒न॒त् । यत् । तृ॒तीय᳚म् । छ॒दिः । ह॒वि॒द्‌र्धान॒योरिति॑ हविः-धान॑योः । उ॒दा॒ह्रि॒यत॒ इत्यु॑त् - आ॒ह्रि॒यते᳚ । तृ॒तीय॑स्य । ह॒वि॒द्‌र्धान॒स्येति॑ हविः - धान॑स्य । अव॑रुद्ध्या॒ इत्यव॑ - रु॒द्ध्यै॒ । शिरः॑ । वै । ए॒तत् । य॒ज्ञ्स्य॑ । यत् । ह॒वि॒द्‌र्धान॒मिति॑ हविः-धान᳚म् । विष्णोः᳚ । र॒राट᳚म् । अ॒सि॒ । विष्णोः᳚ । पृ॒ष्ठम् । अ॒सि॒ । इति॑ । आ॒ह॒ । तस्मा᳚त् । ए॒ता॒व॒द्धेत्ये॑तावत् - धा । शिरः॑ । विष्यू॑त॒मिति॒ वि - स्यू॒त॒म् । विष्णोः᳚ ( ) । स्यूः । अ॒सि॒ । विष्णोः᳚ । ध्रु॒वम् । अ॒सि॒ । इति॑ । आ॒ह॒ । वै॒ष्ण॒वम् । हि । दे॒वत॑या । ह॒वि॒द्‌र्धान॒मिति॑ हविः - धान᳚म् । यम् । प्र॒थ॒मम् । ग्र॒न्थिम् । ग्र॒थ्नी॒यात् । यत् । तम् । न । वि॒स्रꣳ॒॒सये॒दिति॑ वि - स्रꣳ॒॒सये᳚त् । अमे॑हेन । अ॒द्ध्व॒र्युः । प्रेति॑ । मी॒ये॒त॒ । तस्मा᳚त् । सः । वि॒स्रस्य॒ इति॑ वि-स्रस्यः॑ ॥ \textbf{  48 } \newline
                  \newline
                      (पत्नी॑ -हन्यु-र्वा पृथि॒व्या-विष्यू॑तं॒ ॅविष्णोः॒-षड्विꣳ॑शतिश्च)  \textbf{(A9)} \newline \newline
                                \textbf{ TS 6.2.10.1} \newline
                  दे॒वस्य॑ । त्वा॒ । स॒वि॒तुः । प्र॒स॒व इति॑ प्र -स॒वे । इति॑ । अभ्रि᳚म् । एति॑ । द॒त्ते॒ । प्रसू᳚त्या॒ इति॒ प्र - सू॒त्यै॒ । अ॒श्विनोः᳚ । बा॒हुभ्या॒मिति॑ बा॒हु-भ्या॒म् । इति॑ । आ॒ह॒ । अ॒श्विनौ᳚ । हि । दे॒वाना᳚म् । अ॒द्ध्व॒र्यू इति॑ । आस्ता᳚म् । पू॒ष्णः । हस्ता᳚भ्याम् । इति॑ । आ॒ह॒ । यत्यै᳚ । वज्रः॑ । इ॒व॒ । वै । ए॒षा । यत् । अभ्रिः॑ । अभ्रिः॑ । अ॒सि॒ । नारिः॑ । अ॒सि॒ । इति॑ । आ॒ह॒ । शान्त्यै᳚ । काण्डे॑काण्ड॒ इति॒ काण्डे᳚ - का॒ण्डे॒ । वै । क्रि॒यमा॑णे । य॒ज्ञ्म् । रक्षाꣳ॑सि । जि॒घाꣳ॒॒स॒न्ति॒ । परि॑लिखित॒मिति॒ परि॑ - लि॒खि॒त॒म् । रक्षः॑ । परि॑लिखिता॒ इति॒ परि॑ - लि॒खि॒ताः॒ । अरा॑तयः । इति॑ । आ॒ह॒ । रक्ष॑साम् । अप॑हत्या॒ इत्यप॑ - ह॒त्यै॒ । \textbf{  49} \newline
                  \newline
                                \textbf{ TS 6.2.10.2} \newline
                  इ॒दम् । अ॒हम् । रक्ष॑सः । ग्री॒वाः । अपीति॑ । कृ॒न्ता॒मि॒ । यः । अ॒स्मान् । द्वेष्टि॑ । यम् । च॒ । व॒यम् । द्वि॒ष्मः । इति॑ । आ॒ह॒ । द्वौ । वाव । पुरु॑षौ । यम् । च॒ । ए॒व । द्वेष्टि॑ । यः । च॒ । ए॒न॒म् । द्वेष्टि॑ । तयोः᳚ । ए॒व । अन॑न्तराय॒मित्यन॑न्तः - आ॒य॒म् । ग्री॒वाः । कृ॒न्त॒ति॒ । दि॒वे । त्वा॒ । अ॒न्तरि॑क्षाय । त्वा॒ । पृ॒थि॒व्यै । त्वा॒ । इति॑ । आ॒ह॒ । ए॒भ्यः । ए॒व । ए॒ना॒म् । लो॒केभ्यः॑ । प्रेति॑ । उ॒क्ष॒ति॒ । प॒रस्ता᳚त् । अ॒र्वाची᳚म् । प्रेति॑ । उ॒क्ष॒ति॒ । तस्मा᳚त् । \textbf{  50} \newline
                  \newline
                                \textbf{ TS 6.2.10.3} \newline
                  प॒रस्ता᳚त् । अ॒र्वाची᳚म् । म॒नु॒ष्याः᳚ । ऊर्ज᳚म् । उपेति॑ । जी॒व॒न्ति॒ । क्रू॒रम् । इ॒व॒ । वै । ए॒तत् । क॒रो॒ति॒ । यत् । खन॑ति । अ॒पः । अवेति॑ । न॒य॒ति॒ । शान्त्यै᳚ । यव॑मती॒रिति॒ यव॑ - म॒तीः॒ । अवेति॑ । न॒य॒ति॒ । ऊर्क् । वै । यवः॑ । ऊर्क् । उ॒दु॒म्बरः॑ । ऊ॒र्जा । ए॒व । ऊर्ज᳚म् । समिति॑ । अ॒द्‌र्ध॒य॒ति॒ । यज॑मानेन । सम्मि॒तेति॒ सं - मि॒ता॒ । औदु॑बंरी । भ॒व॒ति॒ । यावान्॑ । ए॒व । यज॑मानः । ताव॑तीम् । ए॒व । अ॒स्मि॒न्न् । ऊर्ज᳚म् । द॒धा॒ति॒ । पि॒तृ॒णाम् । सद॑नम् । अ॒सि॒ । इति॑ । ब॒र्॒.हिः । अवेति॑ । स्तृ॒णा॒ति॒ । पि॒तृ॒दे॒व॒त्य॑मिति॑ पितृ - दे॒व॒त्य᳚म् । \textbf{  51} \newline
                  \newline
                                \textbf{ TS 6.2.10.4} \newline
                  हि । ए॒तत् । यत् । निखा॑त॒मिति॒ नि - खा॒त॒म् । यत् । ब॒र्॒.हिः । अन॑वस्ती॒र्येत्यन॑व-स्ती॒र्य॒ । मि॒नु॒यात् । पि॒तृ॒दे॒व॒त्येति॑ पितृ - दे॒व॒त्या᳚ । निखा॒तेति॒ नि - खा॒ता॒ । स्या॒त् । ब॒र॒.हिः । अ॒व॒स्तीर्येत्य॑व- स्तीर्य॑ । मि॒नो॒ति॒ । अ॒स्याम् । ए॒व । ए॒ना॒म् । मि॒नो॒ति॒ । अथो॒ इति॑ । स्वा॒रुह॒मिति॑ स्व-रुह᳚म् । ए॒व । ए॒ना॒म् । क॒रो॒ति॒ । उदिति॑ । दिव᳚म् । स्त॒भा॒न॒ । एति॑ । अ॒न्तरि॑क्षम् । पृ॒ण॒ । इति॑ । आ॒ह॒ । ए॒षाम् । लो॒काना᳚म् । विधृ॑त्या॒ इति॒ वि - धृ॒त्यै॒ । द्यु॒ता॒नः । त्वा॒ । मा॒रु॒तः । मि॒नो॒तु॒ । इति॑ । आ॒ह॒ । द्यु॒ता॒नः । ह॒ । स्म॒ । वै । मा॒रु॒तः । दे॒वाना᳚म् । औदु॑बंरीम् । मि॒नो॒ति॒ । तेन॑ । ए॒व । \textbf{  52} \newline
                  \newline
                                \textbf{ TS 6.2.10.5} \newline
                  ए॒ना॒म् । मि॒नो॒ति॒ । ब्र॒ह्म॒वनि॒मिति॑ ब्रह्म - वनि᳚म् । त्वा॒ । क्ष॒त्र॒वनि॒मिति॑ क्षत्र - वनि᳚म् । इति॑ । आ॒ह॒ । य॒था॒य॒जु॒रिति॑ यथा - य॒जुः । ए॒व । ए॒तत् । घृ॒तेन॑ । द्या॒वा॒पृ॒थि॒वी॒ इति॑ द्यावा-पृ॒थि॒वी॒ । एति॑ । पृ॒णे॒था॒म् । इति॑ । औदु॑बंर्याम् । जु॒हो॒ति॒ । द्यावा॑पृथि॒वी इति॒ द्यावा᳚ - पृ॒थि॒वी । ए॒व । रसे॑न । अ॒न॒क्ति॒ । आ॒न्तमित्या᳚ - अ॒न्तम् । अ॒न्वव॑स्रावय॒तीय॑नु -अव॑स्रावयति । आ॒न्तमित्या᳚-अ॒न्तम् । ए॒व । यज॑मानम् । तेज॑सा । अ॒न॒क्ति॒ । ऐ॒न्द्रम् । अ॒सि॒ । इति॑ । छ॒दिः । अधि॑ । नीति॑ । द॒धा॒ति॒ । ऐ॒न्द्रम् । हि । दे॒वत॑या । सदः॑ । वि॒श्व॒ज॒नस्येति॑ विश्व - ज॒नस्य॑ । छा॒या । इति॑ । आ॒ह॒ । वि॒श्व॒ज॒नस्येति॑ विश्व - ज॒नस्य॑ । हि । ए॒षा । छा॒या । यत् । सदः॑ । नव॑छ॒दीति॒ नव॑ - छ॒दि॒ । \textbf{  53} \newline
                  \newline
                                \textbf{ TS 6.2.10.6} \newline
                  तेज॑स्काम॒स्येति॒ तेजः॑ - का॒म॒स्य॒ । मि॒नु॒या॒त् । त्रि॒वृतेति॑ त्रि-वृता᳚ । स्तोमे॑न । सम्मि॑त॒मिति॒ सं - मि॒त॒म् । तेजः॑ । त्रि॒वृदिति॑ त्रि -वृत् । ते॒ज॒स्वी । ए॒व । भ॒व॒ति॒ । एका॑दशछ॒दीत्येका॑दश - छ॒दि॒ । इ॒न्द्रि॒यका॑म॒स्येती᳚न्द्रि॒य - का॒म॒स्य॒ । एका॑दशाक्ष॒रेत्येका॑दश-अ॒क्ष॒रा॒ । त्रि॒ष्टुक् । इ॒न्द्रि॒यम् । त्रि॒ष्टुक् । इ॒न्द्रि॒या॒वी । ए॒व । भ॒व॒ति॒ । पञ्च॑दशछ॒दीति॒ पञ्च॑दश-छ॒दि॒ । भ्रातृ॑व्यवत॒ इति॒ भ्रातृ॑व्य - व॒तः॒ । प॒ञ्च॒द॒श इति॑ पञ्च - द॒शः । वज्रः॑ । भ्रातृ॑व्याभिभूत्या॒ इति॒ भ्रातृ॑व्य - अ॒भि॒भू॒त्यै॒ । स॒प्तद॑शछ॒दीति॑ स॒प्तद॑श - छ॒दि॒ । प्र॒जाका॑म॒स्येति॑ प्र॒जा - का॒म॒स्य॒ । स॒प्त॒द॒श इति॑ सप्त - द॒शः । प्र॒जाप॑ति॒रिति॑ प्र॒जा - प॒तिः॒ । प्र॒जाप॑ते॒रिति॑ प्र॒जा - प॒तेः॒ । आप्त्यै᳚ । एक॑विꣳशतिछ॒दीत्येक॑विꣳशति - छ॒दि॒ । प्र॒ति॒ष्ठाका॑म॒स्येति॑ प्रति॒ष्ठा - का॒म॒स्य॒ । ए॒क॒विꣳ॒॒श इत्ये॑क-विꣳ॒॒शः । स्तोमा॑नां । प्र॒ति॒ष्ठेति॑ प्रति - स्था । प्रति॑ष्ठित्या॒ इति॒ प्रति॑ - स्थि॒त्यै॒ । उ॒दर᳚म् । वै । सदः॑ । ऊर्क् । उ॒दु॒बंरः॑ । म॒द्ध्य॒तः । औदु॑बंरीम् । मि॒नो॒ति॒ । म॒द्ध्य॒तः । ए॒व । प्र॒जाना॒मिति॑ प्र - जाना᳚म् । ऊर्ज᳚म् । द॒धा॒ति॒ । तस्मा᳚त् । \textbf{  54} \newline
                  \newline
                                \textbf{ TS 6.2.10.7} \newline
                  म॒द्ध्य॒तः । ऊ॒र्जा । भु॒ञ्ज॒ते॒ । य॒ज॒मा॒न॒लो॒क इति॑ यजमान - लो॒के । वै । दक्षि॑णानि । छ॒दीꣳषि॑ । भ्रा॒तृ॒व्य॒लो॒क इति॑ भ्रातृव्य - लो॒के । उत्त॑रा॒णीत्युत् - त॒रा॒णि॒ । दक्षि॑णानि । उत्त॑रा॒णीत्युत् - त॒रा॒णि॒ । क॒रो॒ति॒ । यज॑मानम् । ए॒व । अय॑जमानात् । उत्त॑र॒मित्युत् - त॒र॒म् । क॒रो॒ति॒ । तस्मा᳚त् । यज॑मानः । अय॑जमानात् । उत्त॑र॒ इत्युत् - त॒रः॒ । अ॒न्त॒र्व॒र्तानित्य॑न्तः - व॒र्तान् । क॒रो॒ति॒ । व्यावृ॑त्त्या॒ इति॑ वि-आवृ॑त्त्यै । तस्मा᳚त् । अर॑ण्यम् । प्र॒जा इति॑ प्र - जाः । उपेति॑ । जी॒व॒न्ति॒ । परीति॑ । त्वा॒ । गि॒र्व॒णः॒ । गिरः॑ । इति॑ । आ॒ह॒ । य॒था॒य॒जुरिति॑ यथा - य॒जुः । ए॒व । ए॒तत् । इन्द्र॑स्य । स्यूः । अ॒सि॒ । इन्द्र॑स्य । ध्रु॒वम् । अ॒सि॒ । इति॑ । आ॒ह॒ । ऐ॒न्द्रम् । हि । दे॒वत॑या । सदः॑ ( ) । यम् । प्र॒थ॒मम् । ग्र॒न्थिम् । ग्र॒थ्नी॒यात् । यत् । तम् । न । वि॒स्रꣳ॒॒सये॒दिति॑ वि - स्रꣳ॒॒सये᳚त् । अमे॑हेन् । अ॒द्ध्व॒र्युः । प्रेति॑ । मी॒ये॒त॒ । तस्मा᳚त् । सः । वि॒स्रस्य॒ इति॑ वि - स्रस्यः॑ ॥ \textbf{  55} \newline
                  \newline
                      (अप॑हत्यै॒ - तस्मा᳚त् - पितृदेव॒त्यं॑ - तेनै॒व - नव॑छदि॒ - तस्मा॒थ् - सदः॒ - पञ्च॑दश च)  \textbf{(A10)} \newline \newline
                                \textbf{ TS 6.2.11.1} \newline
                  शिरः॑ । वै । ए॒तत् । य॒ज्ञ्स्य॑ । यत् । ह॒वि॒द्‌र्धान॒मिति॑ हविः-धान᳚म् । प्रा॒णा इति॑ प्र - अ॒नाः । उ॒प॒र॒वा इत्यु॑प - र॒वाः । ह॒वि॒द्‌र्धान॒ इति॑ हविः-धाने᳚ । खा॒य॒न्ते॒ । तस्मा᳚त् । शी॒र्॒.षन्न् । प्रा॒णा इति॑ प्र-अ॒नाः । अ॒धस्ता᳚त् । खा॒य॒न्ते॒ । तस्मा᳚त् । अ॒धस्ता᳚त् । शी॒र्ष्णः । प्रा॒णा इति॑ प्र-अ॒नाः । र॒क्षो॒हण॒ इति॑ रक्षः-हनः॑ । व॒ल॒ग॒हन॒ इति॑ वलग-हनः॑ । वै॒ष्ण॒वान् । ख॒ना॒मि॒ । इति॑ । आ॒ह॒ । वै॒ष्ण॒वाः । हि । दे॒वत॑या । उ॒प॒र॒वा इत्यु॑प - र॒वाः । असु॑राः । वै । नि॒र्यन्त॒ इति॑ निः - यन्तः॑ । दे॒वाना᳚म् । प्रा॒णेष्विति॑ प्र - अ॒नेषु॑ । व॒ल॒गानिति॑ वल-गान् । नीति॑ । अ॒ख॒न॒न्न् । तान् । बा॒हु॒मा॒त्र इति॑ बाहु-मा॒त्रे । अन्विति॑ । अ॒वि॒न्द॒न्न् । तस्मा᳚त् । बा॒हु॒मा॒त्रा इति॑ बाहु - मा॒त्राः । खा॒य॒न्ते॒ । इ॒दम् । अ॒हम् । तम् । व॒ल॒गमिति॑ वल - गम् । उदिति॑ । व॒पा॒मि॒ । \textbf{  56} \newline
                  \newline
                                \textbf{ TS 6.2.11.2} \newline
                  यम् । नः॒ । स॒मा॒नः । यम् । अस॑मानः । नि॒च॒खानिति॑ नि-च॒खान॑ । इति॑ । आ॒ह॒ । द्वौ । वाव । पुरु॑षौ । यः । च॒ । ए॒व । स॒मा॒नः । यः । च॒ । अस॑मानः । यम् । ए॒व । अ॒स्मै॒ । तौ । व॒ल॒गमिति॑ वल - गम् । नि॒खन॑त॒ इति॑ नि - खन॑तः । तम् । ए॒व । उदिति॑ । व॒प॒ति॒ । समिति॑ । तृ॒ण॒त्ति॒ । तस्मा᳚त् । संतृ॑ण्णा॒ इति॒ सं - तृ॒ण्णाः॒ । अ॒न्त॒र॒तः । प्रा॒णा इति॑ प्र - अ॒नाः । न । समिति॑ । भि॒न॒त्ति॒ । तस्मा᳚त् । अस॑भिंन्ना॒ इत्यसं᳚ - भि॒न्नाः॒ । प्रा॒णा इति॑ प्र - अ॒नाः । अ॒पः । अवेति॑ । न॒य॒ति॒ । तस्मा᳚त् । आ॒र्द्राः । अ॒न्त॒र॒तः । प्रा॒णा इति॑ प्र - अ॒नाः । यव॑मती॒रिति॒ यव॑ - म॒तीः॒ । अवेति॑ । न॒य॒ति॒ । \textbf{  57} \newline
                  \newline
                                \textbf{ TS 6.2.11.3} \newline
                  ऊर्क् । वै । यवः॑ । प्रा॒णा इति॑ प्र - अ॒नाः । उ॒प॒र॒वा इत्यु॑प-र॒वाः । प्रा॒णेष्विति॑ प्र - अ॒नेषु॑ । ए॒व । ऊर्ज᳚म् । द॒धा॒ति॒ । ब॒र्॒.हिः । अवेति॑ । स्तृ॒णा॒ति॒ । तस्मा᳚त् । लो॒म॒शाः । अ॒न्त॒र॒तः । प्रा॒णा इति॑ प्र - अ॒नाः । आज्ये॑न । व्याघा॑रय॒तीति॑ वि - आघा॑रयति । तेजः॑ । वै । आज्य᳚म् । प्रा॒णा इति॑ प्र - अ॒नाः । उ॒प॒र॒वा इत्यु॑प - र॒वाः । प्रा॒णेष्विति॑ प्र - अ॒नेषु॑ । ए॒व । तेजः॑ । द॒धा॒ति॒ । हनू॒ इति॑ । वै । ए॒ते इति॑ । य॒ज्ञ्स्य॑ । यत् । अ॒धि॒षव॑णे॒ इत्य॑धि-सव॑ने । न । समिति॑ । तृ॒ण॒त्ति॒ । अस॑तृंण्णे॒ इत्यसं᳚ - तृ॒ण्णे॒ । हि । हनू॒ इति॑ । अथो॒ इति॑ । खलु॑ । दी॒र्घ॒सो॒म इति॑ दीर्घ - सो॒मे । स॒तृंद्ये॒ इति॑ सं-तृद्ये᳚ । धृत्यै᳚ । शिरः॑ । वै । ए॒तत् । य॒ज्ञ्स्य॑ । यत् । ह॒वि॒द्‌र्धान॒मिति॑ हविः - धान᳚म् । \textbf{  58} \newline
                  \newline
                                \textbf{ TS 6.2.11.4} \newline
                  प्रा॒णा इति॑ प्र - अ॒नाः । उ॒प॒र॒वा इत्यु॑प - र॒वाः । हनू॒ इति॑ । अ॒धि॒षव॑णे॒ इत्य॑धि - सव॑ने । जि॒ह्वा । चर्म॑ । ग्रावा॑णः । दन्ताः᳚ । मुख᳚म् । आ॒ह॒व॒नीय॒ इत्या᳚ - ह॒व॒नीयः॑ । नासि॑का । उ॒त्त॒र॒वे॒दिरित्यु॑त्तर - वे॒दिः । उ॒दर᳚म् । सदः॑ । य॒दा । खलु॑ । वै । जि॒ह्वया᳚ । द॒थ्स्विति॑ दत् -सु । अधीति॑ । खाद॑ति । अथ॑ । मुख᳚म् । ग॒च्छ॒ति॒ । य॒दा । मुख᳚म् । गच्छ॑ति । अथ॑ । उ॒दर᳚म् । ग॒च्छ॒ति॒ । तस्मा᳚त् । ह॒वि॒द्‌र्धान॒ इति॑ हविः - धाने᳚ । चर्मन्न्॑ । अधीति॑ । ग्राव॑भि॒रिति॒ ग्राव॑ - भिः॒ । अ॒भि॒षुत्येत्य॑भि - सुत्य॑ । आ॒ह॒व॒नीय॒ इत्या᳚ - ह॒व॒नीये᳚ । हु॒त्वा । प्र॒त्यञ्चः॑ । प॒रेत्येति॑ परा - इत्य॑ । सद॑सि । भ॒क्ष॒य॒न्ति॒ । यः । वै । वि॒राज॒ इति॑ वि - राजः॑ । य॒ज्ञ्॒मु॒ख इति॑ यज्ञ् - मु॒खे । दोह᳚म् । वेद॑ । दु॒हे । ए॒व ( ) । ए॒ना॒म् । इ॒यम् । वै । वि॒राडिति॑ वि - राट् । तस्यै᳚ । त्वक् । चर्म॑ । ऊधः॑ । अ॒धि॒षव॑णे॒ इत्य॑धि-सव॑ने । स्तनाः᳚ । उ॒प॒र॒वा इत्यु॑प - र॒वाः । ग्रावा॑णः । व॒थ्साः । ऋ॒त्विजः॑ । दु॒ह॒न्ति॒ । सोमः॑ । पयः॑ । यः । ए॒वम् । वेद॑ । दु॒हे । ए॒व । ए॒ना॒म् ॥ \textbf{  59} \newline
                  \newline
                      (व॒पा॒मि॒-यव॑मती॒रव॑ नयति-हवि॒र्द्धान॑-मे॒व-त्रयो॑विꣳशतिश्च)  \textbf{(A11)} \newline \newline
\textbf{praSna korvai with starting padams of 1 to 11 anuvAkams :-} \newline
(यदु॒भौ - दे॑वासु॒राः मि॒थ - स्तेषाꣳ॑ - सुव॒र्गं - ॅयद्वा अनी॑शानः - पु॒रोह॑विषि॒ - तेभ्यः॒ - सोत्त॑रवे॒दि - र्ब॒द्धन् - दे॒वस्याऽभ्रिं॒ ॅवज्रः - शिरो॒ वा - एका॑दश ) \newline

\textbf{korvai with starting padams of1, 11, 21 series of pa~jcAtis :-} \newline
(यदु॒भा - वित्या॑ह दे॒वानां᳚ - ॅय॒ज्ञो दे॒वेभ्यो॒ - न रथा॑य॒ - यज॑मानाय - प॒रस्ता॑द॒र्वाची॒ - न्नव॑ पञ्चा॒शत्) \newline

\textbf{first and last padam of second praSnam Of 6th KANDam} \newline
(यदु॒भौ - दु॒ह ए॒वैनां᳚) \newline 


॥ हरिः॑ ॐ ॥॥ कृष्ण यजुर्वेदीय तैत्तिरीय संहितायां षष्ठकाण्डे द्वितीयः प्रश्नः समाप्तः ॥ \newline
\pagebreak
\pagebreak
        


\end{document}
