\documentclass[17pt]{extarticle}
\usepackage{babel}
\usepackage{fontspec}
\usepackage{polyglossia}
\usepackage{extsizes}



\setmainlanguage{sanskrit}
\setotherlanguages{english} %% or other languages
\setlength{\parindent}{0pt}
\pagestyle{myheadings}
\newfontfamily\devanagarifont[Script=Devanagari]{AdishilaVedic}


\newcommand{\VAR}[1]{}
\newcommand{\BLOCK}[1]{}




\begin{document}
\begin{titlepage}
    \begin{center}
 
\begin{sanskrit}
    { \Large
    ॐ नमः परमात्मने, श्री महागणपतये नमः, श्री गुरुभ्यो नमः
ह॒रिः॒ ॐ 
    }
    \\
    \vspace{2.5cm}
    \mbox{ \Huge
    2.3     द्वितीयकाण्डे तृतीयः प्रश्नः - इष्टिविधानं   }
\end{sanskrit}
\end{center}

\end{titlepage}
\tableofcontents

ॐ नमः परमात्मने, श्री महागणपतये नमः, श्री गुरुभ्यो नमः
ह॒रिः॒ ॐ \newline
2.3     द्वितीयकाण्डे तृतीयः प्रश्नः - इष्टिविधानं \newline

\addcontentsline{toc}{section}{ 2.3     द्वितीयकाण्डे तृतीयः प्रश्नः - इष्टिविधानं}
\markright{ 2.3     द्वितीयकाण्डे तृतीयः प्रश्नः - इष्टिविधानं \hfill https://www.vedavms.in \hfill}
\section*{ 2.3     द्वितीयकाण्डे तृतीयः प्रश्नः - इष्टिविधानं }
                                \textbf{ TS 2.3.1.1} \newline
                  आ॒दि॒त्येभ्यः॑ । भुव॑द्वद्भ्य॒ इति॒ भुव॑द्वत् - भ्यः॒ । च॒रुम् । निरिति॑ । व॒पे॒त् । भूति॑काम॒ इति॒ भूति॑ - का॒मः॒ । आ॒दि॒त्याः । वै । ए॒तम् । भूत्यै᳚ । प्रतीति॑ । नु॒द॒न्ते॒ । यः । अल᳚म् । भूत्यै᳚ । सन्न् । भूति᳚म् । न । प्रा॒प्नोतीति॑ प्र - आ॒प्नोति॑ । आ॒दि॒त्यान् । ए॒व । भुव॑द्वत॒ इति॒ भुव॑त् - व॒तः॒ । स्वेन॑ । भा॒ग॒धेये॒नेति॑ भाग - धेये॑न । उपेति॑ । धा॒व॒ति॒ । ते । ए॒व । ए॒न॒म् । भूति᳚म् । ग॒म॒य॒न्ति॒ । भव॑ति । ए॒व । आ॒दि॒त्येभ्यः॑ । धा॒रय॑द्वद्भ्य॒ इति॑ धा॒रय॑द्वत् - भ्यः॒ । च॒रुम् । निरिति॑ । व॒पे॒त् । अप॑रुद्ध॒ इत्यप॑ - रु॒द्धः॒ । वा॒ । अ॒प॒रु॒द्ध्यमा॑न॒ इत्य॑प - रु॒ध्यमा॑नः । वा॒ । आ॒दि॒त्याः । वै । अ॒प॒रो॒द्धार॒ इत्य॑प - रो॒द्धारः॑ । आ॒दि॒त्याः । अ॒व॒ग॒म॒यि॒तार॒ इत्य॑व - ग॒म॒यि॒तारः॑ । आ॒दि॒त्यान् । ए॒व । धा॒रय॑द्वत॒ इति॑ धा॒रय॑त् - व॒तः॒ । \textbf{  1} \newline
                  \newline
                                \textbf{ TS 2.3.1.2} \newline
                  स्वेन॑ । भा॒ग॒धेये॒नेति॑ भाग - धेये॑न । उपेति॑ । धा॒व॒ति॒ । ते । ए॒व । ए॒न॒म् । वि॒शि । दा॒द्ध्र॒ति॒ । अ॒न॒प॒रु॒द्ध्य इत्य॑नप - रु॒द्ध्यः । भ॒व॒ति॒ । अदि॑ते । अन्विति॑ । म॒न्य॒स्व॒ । इति॑ । अ॒प॒रु॒द्ध्यमा॑न॒ इत्य॑प - रु॒द्ध्यमा॑नः । अ॒स्य॒ । प॒दम् । एति॑ । द॒दी॒त॒ । इ॒यम् । वै । अदि॑तिः । इ॒यम् । ए॒व । अ॒स्मै॒ । रा॒ज्यम् । अन्विति॑ । म॒न्य॒ते॒ । स॒त्या । आ॒शीरित्या᳚ - शीः । इति॑ । आ॒ह॒ । स॒त्याम् । ए॒व । आ॒शिष॒मित्या᳚ - शिष᳚म् । कु॒रु॒ते॒ । इ॒ह । मनः॑ । इति॑ । आ॒ह॒ । प्र॒जा इति॑ प्र - जाः । ए॒व । अ॒स्मै॒ । सम॑नस॒ इति॒ स - म॒न॒सः॒ । क॒रो॒ति॒ । उप॑ । प्रेति॑ । इ॒त॒ । म॒रु॒तः॒ । \textbf{  2} \newline
                  \newline
                                \textbf{ TS 2.3.1.3} \newline
                  सु॒दा॒न॒व॒ इति॑ सु - दा॒न॒वः॒ । ए॒ना । वि॒श्पति॑ना । अ॒भीति॑ । अ॒मुम् । राजा॑नम् । इति॑ । आ॒ह॒ । मा॒रु॒ती । वै । विट् । ज्ये॒ष्ठः । वि॒श्पतिः॑ । वि॒शा । ए॒व । ए॒न॒म् । रा॒ष्ट्रेण॑ । समिति॑ । अ॒र्द्ध॒य॒ति॒ । यः । प॒रस्ता᳚त् । ग्रा॒म्य॒वा॒दीति॑ ग्राम्य - वा॒दी । स्यात् । तस्य॑ । गृ॒हात् । व्री॒हीन् । एति॑ । ह॒रे॒त् । शु॒क्लान् । च॒ । कृ॒ष्णान् । च॒ । वीति॑ । चि॒नु॒या॒त् । ये । शु॒क्लाः । स्युः । तम् । आ॒दि॒त्यम् । च॒रुम् । निरीति॑ । व॒पे॒त् । आ॒दि॒त्या । वै । दे॒वत॑या । विट् । विश᳚म् । ए॒व । अवेति॑ । ग॒च्छ॒ति॒ । \textbf{  3} \newline
                  \newline
                                \textbf{ TS 2.3.1.4} \newline
                  अव॑ग॒तेत्यव॑ - ग॒ता॒ । अ॒स्य॒ । विट् । अन॑वगत॒मित्यन॑व - ग॒त॒म् । रा॒ष्ट्रम् । इति॑ । आ॒हुः॒ । ये । कृ॒ष्णाः । स्युः । तम् । वा॒रु॒णम् । च॒रुम् । निरिति॑ । व॒पे॒त् । वा॒रु॒णम् । वै । रा॒ष्ट्रम् । उ॒भे इति॑ । ए॒व । विश᳚म् । च॒ । रा॒ष्ट्रम् । च॒ । अवेति॑ । ग॒च्छ॒ति॒ । यदि॑ । न । अ॒व॒गच्छे॒दित्य॑व-गच्छे᳚त् । इ॒मम् । अ॒हम् । आ॒दि॒त्येभ्यः॑ । भा॒गम् । निरिति॑ । व॒पा॒मि॒ । एति॑ । अ॒मुष्मा᳚त् । अ॒मुष्यै᳚ । वि॒शः । अव॑गन्तो॒रित्यव॑-ग॒न्तोः॒ । इति॑ । निरिति॑ । व॒पे॒त् । आ॒दि॒त्याः । ए॒व । ए॒न॒म् । भा॒ग॒धेय॒मिति॑ भाग - धेय᳚म् । प्रे॒फ्सन्त॒ इति॑ प्र-ई॒फ्सन्तः॑ । विश᳚म् । अवेति॑ । \textbf{  4} \newline
                  \newline
                                \textbf{ TS 2.3.1.5} \newline
                  ग॒म॒य॒न्ति॒ । यदि॑ । न । अ॒व॒गच्छे॒दित्य॑व - गच्छे᳚त् । आश्व॑त्थान् । म॒यूखान्॑ । स॒प्त । म॒द्ध्य॒मे॒षाया॒मिति॑ म॒द्ध्यम - ई॒षाया᳚म् । उपेति॑ । ह॒न्या॒त् । इ॒दम् । अ॒हम् । आ॒दि॒त्यान् । ब॒ध्ना॒मि॒ । एति॑ । अ॒मुष्मा᳚त् । अ॒मुष्यै᳚ । वि॒शः । अव॑गन्तो॒रित्यव॑-ग॒न्तोः॒ । इति॑ । आ॒दि॒त्याः । ए॒व । ए॒न॒म् । ब॒द्धवी॑रा॒ इति॑ ब॒द्ध -वी॒राः॒ । विश᳚म् । अवेति॑ । ग॒म॒य॒न्ति॒ । यदि॑ । न । अ॒व॒गच्छे॒दित्य॑व - गच्छे᳚त् । ए॒तम् । ए॒व । आ॒दि॒त्यम् । च॒रुम् । निरिति॑ । व॒पे॒त् । इ॒द्ध्मे । अपीति॑ । म॒यूखान्॑ । समिति॑ । न॒ह्ये॒त् । अ॒न॒प॒रु॒द्ध्यमित्य॑नप - रु॒द्ध्यम् । ए॒व । अवेति॑ । ग॒च्छ॒ति॒ । आश्व॑त्थाः । भ॒व॒न्ति॒ । म॒रुता᳚म् । वै । ए॒तत् ( ) । ओजः॑ । यत् । अ॒श्व॒त्थः । ओज॑सा । ए॒व । विश᳚म् । अवेति॑ । ग॒च्छ॒ति॒ । स॒प्त । भ॒व॒न्ति॒ । स॒प्तग॑णा॒ इति॑ स॒प्त - ग॒णाः॒ । वै । म॒रुतः॑ । ग॒ण॒श - इति॑ गण - शः । ए॒व । विश᳚म् । अवेति॑ । ग॒च्छ॒ति॒ ॥ \textbf{  5 } \newline
                  \newline
                      (धा॒रय॑द्वतो - मरुतो - गच्छति॒ - विश॒मवै॒ - त - द॒ष्टाद॑श च)  \textbf{(A1)} \newline \newline
                                \textbf{ TS 2.3.2.1} \newline
                  दे॒वाः । वै । मृ॒त्योः । अ॒बि॒भ॒युः॒ । ते । प्र॒जाप॑ति॒मिति॑ प्र॒जा-प॒ति॒म् । उपेति॑ । अ॒धा॒व॒न्न् । तेभ्यः॑ । ए॒ताम् । प्रा॒जा॒प॒त्यामिति॑ प्राजा - प॒त्याम् । श॒तकृ॑ष्णला॒मिति॑ श॒त-कृ॒ष्ण॒ला॒म् । निरिति॑ । अ॒व॒प॒त् । तया᳚ । ए॒व । ए॒षु॒ । अ॒मृत᳚म् । अ॒द॒धा॒त् । यः । मृ॒त्योः । बि॒भी॒यात् । तस्मै᳚ । ए॒ताम् । प्रा॒जा॒प॒त्यामिति॑ प्राजा - प॒त्याम् । श॒तकृ॑ष्णला॒मिति॑ श॒त - कृ॒ष्ण॒ला॒म् । निरिति॑ । व॒पे॒त् । प्र॒जाप॑ति॒मिति॑ प्र॒जा-प॒ति॒म् । ए॒व । स्वेन॑ । भा॒ग॒धेये॒नेति॑ भाग - धेये॑न । उपेति॑ । धा॒व॒ति॒ । सः । ए॒व । अ॒स्मि॒न्न् । आयुः॑ । द॒धा॒ति॒ । सर्व᳚म् । आयुः॑ । ए॒ति॒ । श॒तकृ॑ष्ण॒लेति॑ श॒त - कृ॒ष्ण॒ला॒ । भ॒व॒ति॒ । श॒तायु॒रिति॑ श॒त-आ॒युः॒ । पुरु॑षः । श॒तेन्द्रि॑य॒ इति॑ श॒त - इ॒न्द्रि॒यः॒ । आयु॑षि । ए॒व । इ॒न्द्रि॒ये । \textbf{  6} \newline
                  \newline
                                \textbf{ TS 2.3.2.2} \newline
                  प्रतीति॑ । ति॒ष्ठ॒ति॒ । घृ॒ते । भ॒व॒ति॒ । आयुः॑ । वै । घृ॒तम् । अ॒मृत᳚म् । हिर॑ण्यम् । आयुः॑ । च॒ । ए॒व । अ॒स्मै॒ । अ॒मृत᳚म् । च॒ । स॒मीची॒ इति॑ । द॒धा॒ति॒ । च॒त्वारि॑चत्वा॒रीति॑ च॒त्वारि॑ - च॒त्वा॒रि॒ । कृ॒ष्णला॑नि । अवेति॑ । द्य॒ति॒ । च॒तु॒र॒व॒त्तस्येति॑ चतुः - अ॒व॒त्तस्य॑ । आप्त्यै᳚ । ए॒क॒धेत्ये॑क - धा । ब्र॒ह्मणे᳚ । उपेति॑ । ह॒र॒ति॒ । ए॒क॒धेत्ये॑क - धा । ए॒व । यज॑माने । आयुः॑ । द॒धा॒ति॒ । अ॒सौ । आ॒दि॒त्यः । न । वीति॑ । अ॒रो॒च॒त॒ । तस्मै᳚ । दे॒वाः । प्राय॑श्चित्तिम् । ऐ॒च्छ॒न्न् । तस्मै᳚ । ए॒तम् । सौ॒र्यम् । च॒रुम् । निरिति॑ । अ॒व॒प॒न्न् । तेन॑ । ए॒व । अ॒स्मि॒न्न् । \textbf{  7} \newline
                  \newline
                                \textbf{ TS 2.3.2.3} \newline
                  रुच᳚म् । अ॒द॒धुः॒ । यः । ब्र॒ह्म॒व॒र्च॒सका॑म॒ इति॑ ब्रह्मवर्च॒स - का॒मः॒ । स्यात् । तस्मै᳚ । ए॒तम् । सौ॒र्यम् । च॒रुम् । निरिति॑ । व॒पे॒त् । अ॒मुम् । ए॒व । आ॒दि॒त्यम् । स्वेन॑ । भा॒ग॒धेये॒नेति॑ भाग - धेये॑न । उपेति॑ । धा॒व॒ति॒ । सः । ए॒व । अ॒स्मि॒न्न् । ब्र॒ह्म॒व॒र्च॒समिति॑ ब्रह्म - व॒र्च॒सम् । द॒धा॒ति॒ । ब्र॒ह्म॒व॒र्च॒सीति॑ ब्रह्म - व॒र्च॒सी । ए॒व । भ॒व॒ति॒ । उ॒भ॒यतः॑ । रु॒क्मौ । भ॒व॒तः॒ । उ॒भ॒यतः॑ । ए॒व । अ॒स्मि॒न्न् । रुच᳚म् । द॒धा॒ति॒ । प्र॒या॒जेप्र॑याज॒ इति॑ प्रया॒जे - प्र॒या॒जे॒ । कृ॒ष्णल᳚म् । जु॒हो॒ति॒ । दि॒ग्‌भ्य इति॑ दिक् - भ्यः । ए॒व । अ॒स्मै॒ । ब्र॒ह्म॒व॒र्च॒समिति॑ ब्रह्म - व॒र्च॒सम् । अवेति॑ । रु॒न्धे॒ । आ॒ग्ने॒यम् । अ॒ष्टाक॑पाल॒मित्य॒ष्टा - क॒पा॒ल॒म् । निरिति॑ । व॒पे॒त् । सा॒वि॒त्रम् । द्वाद॑शकपाल॒मिति॒ द्वाद॑श - क॒पा॒ल॒म् । भूम्यै᳚ । \textbf{  8} \newline
                  \newline
                                \textbf{ TS 2.3.2.4} \newline
                  च॒रुम् । यः । का॒मये॑त । हिर॑ण्यम् । वि॒न्दे॒य॒ । हिर॑ण्यम् । मा॒ । उपेति॑ । न॒मे॒त् । इति॑ । यत् । आ॒ग्ने॒यः । भव॑ति । आ॒ग्ने॒यम् । वै । हिर॑ण्यम् । यस्य॑ । ए॒व । हिर॑ण्यम् । तेन॑ । ए॒व । ए॒न॒त् । वि॒न्द॒ते॒ । सा॒वि॒त्रः । भ॒व॒ति॒ । स॒वि॒तृप्र॑सूत॒ इति॑ सवि॒तृ-प्र॒सू॒तः॒ । ए॒व । ए॒न॒त् । वि॒न्द॒ते॒ । भूम्यै᳚ । च॒रुः । भ॒व॒ति॒ । अ॒स्याम् । ए॒व । ए॒न॒त् । वि॒न्द॒ते॒ । उपेति॑ । ए॒न॒म् । हिर॑ण्यम् । न॒म॒ति॒ । वीति॑ । वै । ए॒षः । इ॒न्द्रि॒येण॑ । वी॒र्ये॑ण । ऋ॒द्ध्य॒ते॒ । यः । हिर॑ण्यम् । वि॒न्दते᳚ । ए॒ताम् । \textbf{  9} \newline
                  \newline
                                \textbf{ TS 2.3.2.5} \newline
                  ए॒व । निरिति॑ । व॒पे॒त् । हिर॑ण्यम् । वि॒त्त्वा । न । इ॒न्द्रि॒येण॑ । वी॒र्ये॑ण । वीति॑ । ऋ॒द्ध्य॒ते॒ । ए॒ताम् । ए॒व । निरिति॑ । व॒पे॒त् । यस्य॑ । हिर॑ण्यम् । नश्ये᳚त् । यत् । आ॒ग्ने॒यः । भव॑ति । आ॒ग्ने॒यम् । वै । हिर॑ण्यम् । यस्य॑ । ए॒व । हिर॑ण्यम् । तेन॑ । ए॒व । ए॒न॒त् । वि॒न्द॒ति॒ । सा॒वि॒त्रः । भ॒व॒ति॒ । स॒वि॒तृप्र॑सूत॒ इति॑ सवि॒तृ - प्र॒सू॒तः॒ । ए॒व । ए॒न॒त् । वि॒न्द॒ति॒ । भूम्यै᳚ । च॒रुः । भ॒व॒ति॒ । अ॒स्याम् । वै । ए॒तत् । न॒श्य॒ति॒ । यत् । नश्य॑ति । अ॒स्याम् । ए॒व । ए॒न॒त् । वि॒न्द॒ति॒ । इन्द्रः॑ । \textbf{  10} \newline
                  \newline
                                \textbf{ TS 2.3.2.6} \newline
                  त्वष्टुः॑ । सोम᳚म् । अ॒भी॒षहेत्य॑भि - सहा᳚ । अ॒पि॒ब॒त् । सः । विष्वङ्॑ । वीति॑ । आ॒र्च्छ॒त् । सः । इ॒न्द्रि॒येण॑ । सो॒म॒पी॒थेनेति॑ सोम - पी॒थेन॑ । वीति॑ । आ॒र्द्ध्य॒त॒ । सः । यत् । ऊ॒र्द्ध्वम् । उ॒दव॑मी॒दित्यु॑त् - अव॑मीत् । ते । श्या॒माकाः᳚ । अ॒भ॒व॒न्न् । सः । प्र॒जाप॑ति॒मिति॑ प्र॒जा - प॒ति॒म् । उपेति॑ । अ॒धा॒व॒त् । तस्मै᳚ । ए॒तम् । सो॒मे॒न्द्रम् । श्या॒मा॒कम् । च॒रुम् । निरिति॑ । अ॒व॒प॒त् । तेन॑ । ए॒व । अ॒स्मि॒न्न् । इ॒न्द्रि॒यम् । सो॒म॒पी॒थमिति॑ सोम - पी॒थम् । अ॒द॒धा॒त् । वीति॑ । वै । ए॒षः । इ॒न्द्रि॒येण॑ । सो॒म॒पी॒थेनेति॑ सोम - पी॒थेन॑ । ऋ॒द्ध्य॒ते॒ । यः । सोम᳚म् । वमि॑ति । यः । सो॒म॒वा॒मीति॑ सोम - वा॒मी । स्यात् । तस्मै᳚ । \textbf{  11} \newline
                  \newline
                                \textbf{ TS 2.3.2.7} \newline
                  ए॒तम् । सो॒मे॒न्द्रम् । श्या॒मा॒कम् । च॒रुम् । निरिति॑ । व॒पे॒त् । सोम᳚म् । च॒ । ए॒व । इन्द्र᳚म् । च॒ । स्वेन॑ । भा॒ग॒धेये॒नेति॑ भाग-धेये॑न । उपेति॑ । धा॒व॒ति॒ । तौ । ए॒व । अ॒स्मि॒न्न् । इ॒न्द्रि॒यम् । सो॒म॒पी॒थमिति॑ सोम - पी॒थम् । ध॒त्तः॒ । न । इ॒न्द्रि॒येण॑ । सो॒म॒पी॒थेनेति॑ सोम - पी॒थेन॑ । वीति॑ । ऋ॒द्ध्य॒ते॒ । यत् । सौ॒म्यः । भव॑ति । सो॒म॒पी॒थमिति॑ सोम - पी॒थम् । ए॒व । अवेति॑ । रु॒न्धे॒ । यत् । ऐ॒न्द्रः । भव॑ति । इ॒न्द्रि॒यम् । वै । सो॒म॒पी॒थ इति॑ सोम - पी॒थः । इ॒न्द्रि॒यम् । ए॒व । सो॒म॒पी॒थमिति॑ सोम - पी॒थम् । अवेति॑ । रु॒न्धे॒ । श्या॒मा॒कः । भ॒व॒ति॒ । ए॒षः । वाव । सः । सोमः॑ । \textbf{  12} \newline
                  \newline
                                \textbf{ TS 2.3.2.8} \newline
                  सा॒क्षादिति॑ स - अ॒क्षात् । ए॒व । सो॒म॒पी॒थमिति॑ सोम - पी॒थम् । अवेति॑ । रु॒न्धे॒ । अ॒ग्नये᳚ । दा॒त्रे । पु॒रो॒डाश᳚म् । अ॒ष्टाक॑पाल॒मित्य॒ष्टा - क॒पा॒ल॒म् । निरिति॑ । व॒पे॒त् । इन्द्रा॑य । प्र॒दा॒त्र इति॑ प्र - दा॒त्रे । पु॒रो॒डाश᳚म् । एका॑दशकपाल॒मित्येका॑दश - क॒पा॒ल॒म् । प॒शुका॑म॒ इति॑ प॒शु - का॒मः॒ । अ॒ग्निः । ए॒व । अ॒स्मै॒ । प॒शून् । प्र॒ज॒नय॒तीति॑ प्र - ज॒नय॑ति । वृ॒द्धान् । इन्द्रः॑ । प्रेति॑ । य॒च्छ॒ति॒ । दधि॑ । मधु॑ । घृ॒तम् । आपः॑ । धा॒नाः । भ॒व॒न्ति॒ । ए॒तत् । वै । प॒शू॒नाम् । रू॒पम् । रू॒पेण॑ । ए॒व । प॒शून् । अवेति॑ । रु॒न्धे॒ । प॒ञ्च॒गृ॒ही॒तमिति॑ पञ्च-गृ॒ही॒तम् । भ॒व॒ति॒ । पाङ्क्ताः᳚ । हि । प॒शवः॑ । ब॒हु॒रू॒पमिति॑ बहु - रू॒पम् । भ॒व॒ति॒ । ब॒हु॒रू॒पा इति॑ बहु - रू॒पाः । हि । प॒शवः॑ । \textbf{  13} \newline
                  \newline
                                \textbf{ TS 2.3.2.9} \newline
                  समृ॑द्ध्या॒ इति॒ सं - ऋ॒द्ध्यै॒ । प्रा॒जा॒प॒त्यमिति॑ प्राजा - प॒त्यम् । भ॒व॒ति॒ । प्रा॒जा॒प॒त्या इति॑ प्राजा - प॒त्याः । वै । प॒शवः॑ । प्र॒जाप॑ति॒रिति॑ प्र॒जा-प॒तिः॒ । ए॒व । अ॒स्मै॒ । प॒शून् । प्रेति॑ । ज॒न॒य॒ति॒ । आ॒त्मा । वै । पुरु॑षस्य । मधु॑ । यत् । मधु॑ । अ॒ग्नौ । जु॒होति॑ । आ॒त्मान᳚म् । ए॒व । तत् । यज॑मानः । अ॒ग्नौ । प्रेति॑ । द॒धा॒ति॒ । प॒ङ्क्त्यौ᳚ । या॒ज्या॒नु॒वा॒क्ये॑ इति॑ याज्या - अ॒नु॒वा॒क्ये᳚ । भ॒व॒तः॒ । पाङ्क्तः॑ । पुरु॑षः । पाङ्क्ताः᳚ । प॒शवः॑ । आ॒त्मान᳚म् । ए॒व । मृ॒त्योः । नि॒ष्क्रीयेति॑ निः-क्रीय॑ । प॒शून् । अवेति॑ । रु॒न्धे॒ ॥ \textbf{  14} \newline
                  \newline
                      (इ॒न्द्रि॒ये᳚ - ऽस्मि॒न् - भूम्या॑ - ए॒ता - मिन्द्रः॒ - स्यात् तस्मै॒ - सोमो॑ - बहु रू॒पा हि प॒शव॒ - एक॑चत्वारिꣳशच्च )  \textbf{(A2)} \newline \newline
                                \textbf{ TS 2.3.3.1} \newline
                  दे॒वाः । वै । स॒त्रम् । आ॒स॒त॒ । ऋद्धि॑परिमित॒मित्यृद्धि॑ - प॒रि॒मि॒त॒म् । यश॑स्कामा॒ इति॒ यशः॑ - का॒माः॒ । तेषा᳚म् । सोम᳚म् । राजा॑नम् । यशः॑ । आ॒र्च्छ॒त् । सः । गि॒रिम् । उदिति॑ । ऐ॒त् । तम् । अ॒ग्निः । अनु॑ । उदिति॑ । ऐ॒त् । तौ । अ॒ग्नीषोमा॒वित्य॒ग्नी- सोमौ᳚ । समिति॑ । अ॒भ॒व॒ता॒म् । तौ । इन्द्रः॑ । य॒ज्ञ्वि॑भ्रष्ट॒ इति॑ य॒ज्ञ् - वि॒भ्र॒ष्टः॒ । अनु॑ । परेति॑ । ऐ॒त् । तौ । अ॒ब्र॒वी॒त् । या॒जय॑तम् । मा॒ । इति॑ । तस्मै᳚ । ए॒ताम् । इष्टि᳚म् । निरिति॑ । अ॒व॒प॒ता॒म् । आ॒ग्ने॒यम् । अ॒ष्टाक॑पाल॒मित्य॒ष्टा - क॒पा॒ल॒म् । ऐ॒न्द्रम् । एका॑दशकपाल॒मित्येका॑दश - क॒पा॒ल॒म् । सौ॒म्यम् । च॒रुम् । तया᳚ । ए॒व । अ॒स्मि॒न्न् । तेजः॑ । \textbf{  15} \newline
                  \newline
                                \textbf{ TS 2.3.3.2} \newline
                  इ॒न्द्रि॒यम् । ब्र॒ह्म॒व॒र्च॒समिति॑ ब्रह्म - व॒र्च॒सम् । अ॒ध॒त्ता॒म् । यः । य॒ज्ञ्वि॑भ्रष्ट॒ इति॑ य॒ज्ञ् - वि॒भ्र॒ष्टः॒ । स्यात् । तस्मै᳚ । ए॒ताम् । इष्टि᳚म् । निरिति॑ । व॒पे॒त् । आ॒ग्ने॒यम् । अ॒ष्टाक॑पाल॒मित्य॒ष्टा - क॒पा॒ल॒म् । ऐ॒न्द्रम् । एका॑दशकपाल॒मित्येका॑दश - क॒पा॒ल॒म् । सौ॒म्यम् । च॒रुम् । यत् । आ॒ग्ने॒यः । भव॑ति । तेजः॑ । ए॒व । अ॒स्मि॒न्न् । तेन॑ । द॒धा॒ति॒ । यत् । ऐ॒न्द्रः । भव॑ति । इ॒न्द्रि॒यम् । ए॒व । अ॒स्मि॒न्न् । तेन॑ । द॒धा॒ति॒ । यत् । सौ॒म्यः । ब्र॒ह्म॒व॒र्च॒समिति॑ ब्रह्म-व॒र्च॒सम् । तेन॑ । आ॒ग्ने॒यस्य॑ । च॒ । सौ॒म्यस्य॑ । च॒ । ऐ॒न्द्रे । स॒माश्ले॑षये॒दिति॑ सं - आश्ले॑षयेत् । तेजः॑ । च॒ । ए॒व । अ॒स्मि॒न्न् । ब्र॒ह्म॒व॒र्च॒समिति॑ ब्रह्म - व॒र्च॒सम् । च॒ । स॒मीची॒ इति॑ । \textbf{  16} \newline
                  \newline
                                \textbf{ TS 2.3.3.3} \newline
                  द॒धा॒ति॒ । अ॒ग्नी॒षो॒मीय॒मित्य॑ग्नी - सो॒मीय᳚म् । एका॑दशकपाल॒मित्येका॑दश - क॒पा॒ल॒म् । निरिति॑ । व॒पे॒त् । यम् । कामः॑ । न । उ॒प॒नमे॒दित्यु॑प - नमे᳚त् । आ॒ग्ने॒यः । वै । ब्रा॒ह्म॒णः । सः । सोम᳚म् । पि॒ब॒ति॒ । स्वाम् । ए॒व । दे॒वता᳚म् । स्वेन॑ । भा॒ग॒धेये॒नेति॑ भाग - धेये॑न । उपेति॑ । धा॒व॒ति॒ । सा । ए॒व । ए॒न॒म् । कामे॑न । समिति॑ । अ॒र्द्ध॒य॒ति॒ । उपेति॑ । ए॒न॒म् । कामः॑ । न॒म॒ति॒ । अ॒ग्नी॒षो॒मीय॒मित्य॑ग्नी - सो॒मीय᳚म् । अ॒ष्टाक॑पाल॒मित्य॒ष्टा - क॒पा॒ल॒म् । निरिति॑ । व॒पे॒त् । ब्र॒ह्म॒व॒र्च॒सका॑म॒ इति॑ ब्रह्मवर्च॒स - का॒मः॒ । अ॒ग्नीषोमा॒वित्य॒ग्नी - सोमौ᳚ । ए॒व । स्वेन॑ । भा॒ग॒धेये॒नेति॑ भाग-धेये॑न । उपेति॑ । धा॒व॒ति॒ । तौ । ए॒व । अ॒स्मि॒न्न् । ब्र॒ह्म॒व॒र्च॒समिति॑ ब्रह्म - व॒र्च॒सम् । ध॒त्तः॒ । ब्र॒ह्म॒व॒र्च॒सीति॑ ब्रह्म - व॒र्च॒सी । ए॒व । 17 (50) \textbf{  17} \newline
                  \newline
                                \textbf{ TS 2.3.3.4} \newline
                  भ॒व॒ति॒ । यत् । अ॒ष्टाक॑पाल॒ इत्य॒ष्टा - क॒पा॒लः॒ । तेन॑ । आ॒ग्ने॒यः । यत् । श्या॒मा॒कः । तेन॑ । सौ॒म्यः । समृ॑द्ध्या॒ इति॒ सं - ऋ॒द्ध्यै॒ । सोमा॑य । वा॒जिने᳚ । श्या॒मा॒कम् । च॒रुम् । निरिति॑ । व॒पे॒त् । यः । क्लैब्या᳚त् । बि॒भी॒यात् । रेतः॑ । हि । वै । ए॒तस्मा᳚त् । वाजि॑नम् । अ॒प॒क्राम॒तीत्य॑प - क्राम॑ति । अथ॑ । ए॒षः । क्लैब्या᳚त् । बि॒भा॒य॒ । सोम᳚म् । ए॒व । वा॒जिन᳚म् । स्वेन॑ । भा॒ग॒धेये॒नेति॑ भाग - धेये॑न । उपेति॑ । धा॒व॒ति॒ । सः । ए॒व । अ॒स्मि॒न्न् । रेतः॑ । वाजि॑नम् । द॒धा॒ति॒ । न । क्ली॒बः । भ॒व॒ति॒ । ब्रा॒ह्म॒ण॒स्प॒त्यमिति॑ ब्राह्मणः - प॒त्यम् । एका॑दशकपाल॒मित्येका॑दश-क॒पा॒ल॒म् । निरिति॑ । व॒पे॒त् । ग्राम॑काम॒ इति॒ ग्राम॑ - का॒मः॒ । \textbf{  18} \newline
                  \newline
                                \textbf{ TS 2.3.3.5} \newline
                  ब्रह्म॑णः । पति᳚म् । ए॒व । स्वेन॑ । भा॒ग॒धेये॒नेति॑ भाग-धेये॑न । उपेति॑ । धा॒व॒ति॒ । सः । ए॒व । अ॒स्मै॒ । स॒जा॒तानिति॑ स - जा॒तान् । प्रेति॑ । य॒च्छ॒ति॒ । ग्रा॒मी । ए॒व । भ॒व॒ति॒ । ग॒णव॑ती॒ इति॑ ग॒ण - व॒ती॒ । या॒ज्या॒नु॒वा॒क्ये॑ इति॑ याज्या - अ॒नु॒वा॒क्ये᳚ । भ॒व॒तः॒ । स॒जा॒तैरिति॑ स - जा॒तैः । ए॒व । ए॒न॒म् । ग॒णव॑न्त॒मिति॑ ग॒ण - व॒न्त॒म् । क॒रो॒ति॒ । ए॒ताम् । ए॒व । निरिति॑ । व॒पे॒त् । यः । का॒मये॑त । ब्रह्मन्न्॑ । विश᳚म् । वीति॑ । ना॒श॒ये॒य॒म् । इति॑ । मा॒रु॒ती इति॑ । या॒ज्या॒नु॒वा॒क्ये॑ इति॑ याज्या - अ॒नु॒वा॒क्ये᳚ । कु॒र्या॒त् । ब्रह्मन्न्॑ । ए॒व । विश᳚म् । वीति॑ । ना॒श॒य॒ति॒ ॥ \textbf{ } \newline
                  \newline
                      विशं॒ ॅवि ना॑शयेय॒मिति॑ मारु॒ती या᳚ज्यानुवा॒क्ये॑ कुर्या॒द् ब्रह्म॑न्ने॒व विशं॒ ॅवि ना॑शयति ॥ 19 (तेजः॑ - स॒मीची᳚ - ब्रह्मवर्च॒स्ये॑व - ग्राम॑काम॒ - स्त्रिच॑त्वारिꣳशच्च )  \textbf{(A3)} \newline \newline
                                \textbf{ TS 2.3.4.1} \newline
                  अ॒र्य॒म्णे । च॒रुम् । निरिति॑ । व॒पे॒त् । सु॒व॒र्गका॑म॒ इति॑ सुव॒र्ग-का॒मः॒ । अ॒सौ । वै । आ॒दि॒त्यः । अ॒र्य॒मा । अ॒र्य॒मण᳚म् । ए॒व । स्वेन॑ । भा॒ग॒धेये॒नेति॑ भाग - धेये॑न । उपेति॑ । धा॒व॒ति॒ । सः । ए॒व । ए॒न॒म् । सु॒व॒र्गमिति॑ सुवः - गम् । लो॒कम् । ग॒म॒य॒ति॒ । अ॒र्य॒म्णे । च॒रुम् । निरिति॑ । व॒पे॒त् । यः । का॒मये॑त । दान॑कामा॒ इति॒ दान॑-का॒माः॒ । मे॒ । प्र॒जा इति॑ प्र-जाः । स्युः॒ । इति॑ । अ॒सौ । वै । आ॒दि॒त्यः । अ॒र्य॒मा । यः । खलु॑ । वै । ददा॑ति । सः । अ॒र्य॒मा । अ॒र्य॒मण᳚म् । ए॒व । स्वेन॑ । भा॒ग॒धेये॒नेति॑ भाग - धेये॑न । उपेति॑ । धा॒व॒ति॒ । सः । ए॒व । \textbf{  20} \newline
                  \newline
                                \textbf{ TS 2.3.4.2} \newline
                  अ॒स्मै॒ । दान॑कामा॒ इति॒ दान॑ - का॒माः॒ । प्र॒जा इति॑ प्र - जाः । क॒रो॒ति॒ । दान॑कामा॒ इति॒ दान॑ - का॒माः॒ । अ॒स्मै॒ । प्र॒जा इति॑ प्र - जाः । भ॒व॒न्ति॒ । अ॒र्य॒म्णे । च॒रुम् । निरिति॑ । व॒पे॒त् । यः । का॒मये॑त । स्व॒स्ति । ज॒नता᳚म् । इ॒या॒म् । इति॑ । अ॒सौ । वै । आ॒दि॒त्यः । अ॒र्य॒मा । अ॒र्य॒मण᳚म् । ए॒व । स्वेन॑ । भा॒ग॒धेये॒नेति॑ भाग-धेये॑न । उपेति॑ । धा॒व॒ति॒ । सः । ए॒व । ए॒न॒म् । तत् । ग॒म॒य॒ति॒ । यत्र॑ । जिग॑मिषति । इन्द्रः॑ । वै । दे॒वाना᳚म् । आ॒नु॒जा॒व॒र इत्या॑नु - जा॒व॒रः । आ॒सी॒त् । सः । प्र॒जाप॑ति॒मिति॑ प्र॒जा - प॒ति॒म् । उपेति॑ । अ॒धा॒व॒त् । तस्मै᳚ । ए॒तम् । ऐ॒न्द्रम् । आ॒नु॒षू॒कमित्या॑नु - सू॒कम् । एका॑दशकपाल॒मित्येका॑दश-क॒पा॒ल॒म् । निरिति॑ । 21 (50 \textbf{  21} \newline
                  \newline
                                \textbf{ TS 2.3.4.3} \newline
                  अ॒व॒प॒त् । तेन॑ । ए॒व । ए॒न॒म् । अग्र᳚म् । दे॒वता॑नाम् । परीति॑ । अ॒न॒य॒त् । बु॒द्ध्नव॑ती॒ इति॑ बु॒द्ध्न - व॒ती॒ । अग्र॑वती॒ इत्यग्र॑ - व॒ती॒ । या॒ज्या॒नु॒वा॒क्ये॑ इति॑ याज्या-अ॒नु॒वा॒क्ये᳚ । अ॒क॒रो॒त् । बु॒द्ध्नात् । ए॒व । ए॒न॒म् । अग्र᳚म् । परीति॑ । अ॒न॒य॒त् । यः । रा॒ज॒न्यः॑ । आ॒नु॒जा॒व॒र इत्या॑नु - जा॒व॒रः । स्यात् । तस्मै᳚ । ए॒तम् । ऐ॒न्द्रम् । आ॒नु॒षू॒कमित्या॑नु - सू॒कम् । एका॑दशकपाल॒मित्येका॑दश-क॒पा॒ल॒म् । निरिति॑ । व॒पे॒त् । इन्द्र᳚म् । ए॒व । स्वेन॑ । भा॒ग॒धेये॒नेति॑ भाग-धेये॑न । उपेति॑ । धा॒व॒ति॒ । सः । ए॒व । ए॒न॒म् । अग्र᳚म् । स॒मा॒नाना᳚म् । परीति॑ । न॒य॒ति॒ । बु॒द्ध्नव॑ती॒ इति॑ बु॒द्ध्न - व॒ती॒ । अग्र॑वती॒ इत्यग्र॑ - व॒ती॒ । या॒ज्या॒नु॒वा॒क्ये॑ इति॑ याज्या - अ॒नु॒वा॒क्ये᳚ । भ॒व॒तः॒ । बु॒द्ध्नात् । ए॒व । ए॒न॒म् । अग्र᳚म् । \textbf{  22} \newline
                  \newline
                                \textbf{ TS 2.3.4.4} \newline
                  परीति॑ । न॒य॒ति॒ । आ॒नु॒षू॒क इत्या॑नु - सू॒कः । भ॒व॒ति॒ । ए॒षा । हि । ए॒तस्य॑ । दे॒वता᳚ । यः । आ॒नु॒जा॒व॒र इत्या॑नु - जा॒व॒रः । समृ॑द्ध्या॒ इति॒ सं- ऋ॒द्ध्यै॒ । यः । ब्रा॒ह्म॒णः । आ॒नु॒जा॒व॒र इत्या॑नु - जा॒व॒रः । स्यात् । तस्मै᳚ । ए॒तम् । बा॒र्.॒ह॒स्प॒त्यम् । आ॒नु॒षू॒कमित्या॑नु-सू॒कम् । च॒रुम् । निरिति॑ । व॒पे॒त् । बृह॒स्पति᳚म् । ए॒व । स्वेन॑ । भा॒ग॒धेये॒नेति॑ भाग - धेये॑न । उपेति॑ । धा॒व॒ति॒ । सः । ए॒व । ए॒न॒म् । अग्र᳚म् । स॒मा॒नाना᳚म् । परीति॑ । न॒य॒ति॒ । बु॒द्ध्नव॑ती॒ इति॑ बु॒द्ध्न-व॒ती॒ । अग्र॑वती॒ इत्यग्र॑ - व॒ती॒ । या॒ज्या॒नु॒वा॒क्ये॑ इति॑ याज्या - अ॒नु॒वा॒क्ये᳚ । भ॒व॒तः॒ । बु॒द्ध्नात् । ए॒व । ए॒न॒म् । अग्र᳚म् । परीति॑ । न॒य॒ति॒ । आ॒नु॒षू॒क इत्या॑नु - सू॒कः । भ॒व॒ति॒ । ए॒षा । हि । ए॒तस्य॑ ( ) । दे॒वता᳚ । यः । आ॒नु॒जा॒व॒र इत्या॑नु - जा॒व॒रः । समृ॑द्ध्या॒ इति॒ सं - ऋ॒द्ध्यै॒ ॥ \textbf{  23} \newline
                  \newline
                      (ए॒व - निर - ग्र॑-मे॒तस्य॑ - च॒त्वारि॑ च)  \textbf{(A4)} \newline \newline
                                \textbf{ TS 2.3.5.1} \newline
                  प्र॒जाप॑ते॒रिति॑ प्र॒जा - प॒तेः॒ । त्रय॑स्त्रिꣳश॒दिति॒ त्रयः॑ - त्रिꣳ॒॒श॒त् । दु॒हि॒तरः॑ । आ॒स॒न्न् । ताः । सोमा॑य । राज्ञे᳚ । अ॒द॒दा॒त् । तासा᳚म् । रो॒हि॒णीम् । उपेति॑ । ऐ॒त् । ताः । ईर्ष्य॑न्तीः । पुनः॑ । अ॒ग॒च्छ॒न्न् । ताः । अन्विति॑ । ऐ॒त् । ताः । पुनः॑ । अ॒या॒च॒त॒ । ताः । अ॒स्मै॒ । न । पुनः॑ । अ॒द॒दा॒त् । सः । अ॒ब्र॒वी॒त् । ऋ॒तम् । अ॒मी॒ष्व॒ । यथा᳚ । स॒मा॒व॒च्छ इति॑ समावत् - शः । उ॒पै॒ष्यामीत्यु॑प - ए॒ष्यामि॑ । अथ॑ । ते॒ । पुनः॑ । दा॒स्या॒मि॒ । इति॑ । सः । ऋ॒तम् । आ॒मी॒त् । ताः । अ॒स्मै॒ । पुनः॑ । अ॒द॒दा॒त् । तासा᳚म् । रो॒हि॒णीम् । ए॒व । उपेति॑ । \textbf{  24} \newline
                  \newline
                                \textbf{ TS 2.3.5.2} \newline
                  ऐ॒त् । तम् । यक्ष्मः॑ । आ॒र्च्छ॒त् । राजा॑नम् । यक्ष्मः॑ । आ॒र॒त् । इति॑ । तत् । रा॒ज॒य॒क्ष्मस्येति॑ राज -य॒क्ष्मस्य॑ । जन्म॑ । यत् । पापी॑यान् । अभ॑वत् । तत् । पा॒प॒य॒क्ष्मस्येति॑ पाप - य॒क्ष्मस्य॑ । यत् । जा॒याभ्यः॑ । अवि॑न्दत् । तत् । जा॒येन्य॑स्य । यः । ए॒वम् । ए॒तेषा᳚म् । यक्ष्मा॑णाम् । जन्म॑ । वेद॑ । न ।  ए॒न॒म् । ए॒ते । यक्ष्माः᳚ । वि॒न्द॒न्ति॒ । सः । ए॒ताः । ए॒व । न॒म॒स्यन्न् । उपेति॑ । अ॒धा॒व॒त् । ताः । अ॒ब्रु॒व॒न्न् । वर᳚म् । वृ॒णा॒म॒है॒ । स॒मा॒व॒च्छ इति॑ समावत् - शः । ए॒व । नः॒ । उपेति॑ । अ॒यः॒ । इति॑ । तस्मै᳚ । ए॒तम् । \textbf{  25} \newline
                  \newline
                                \textbf{ TS 2.3.5.3} \newline
                  आ॒दि॒त्यम् । च॒रुम् । निरिति॑ । अ॒व॒प॒न्न् । तेन॑ । ए॒व । ए॒न॒म् । पा॒पात् । स्रामा᳚त् । अ॒मु॒ञ्च॒न्न् । यः । पा॒प॒य॒क्ष्मगृ॑हीत॒ इति॑ पापय॒क्ष्म - गृ॒ही॒तः॒ । स्यात् । तस्मै᳚ । ए॒तम् । आ॒दि॒त्यम् । च॒रुम् । निरिति॑ । व॒पे॒त् । आ॒दि॒त्यान् । ए॒व । स्वेन॑ । भा॒ग॒धेये॒नेति॑ भाग - धेये॑न । उपेति॑ । ध॒व॒ति॒ । ते । ए॒व । ए॒न॒म् । पा॒पात् । स्रामा᳚त् । मु॒ञ्च॒न्ति॒ । अ॒मा॒वा॒स्या॑या॒मित्य॑मा - वा॒स्या॑याम् । निरिति॑ । व॒पे॒त् । अ॒मुम् । ए॒व । ए॒न॒म् । आ॒प्याय॑मान॒मित्या᳚ - प्याय॑मानम् । अनु॑ । एति॑ । प्या॒य॒य॒ति॒ । नवो॑नव॒ इति॒ नवः॑ - न॒वः॒ । भ॒व॒ति॒ । जाय॑मानः । इति॑ । पु॒रो॒नु॒वा॒क्येति॑ पुरः - अ॒नु॒वा॒क्या᳚ । भ॒व॒ति॒ । आयुः॑ । ए॒व । अ॒स्मि॒न्न् ( ) । तया᳚ । द॒धा॒ति॒ । यम् । आ॒दि॒त्याः । अꣳ॒॒शुम् । आ॒प्या॒यय॒न्तीत्या᳚ - प्या॒यय॑न्ति । इति॑ । या॒ज्या᳚ । एति॑ । ए॒व । ए॒न॒म् ।  ए॒तया᳚ । प्या॒य॒य॒ति॒ ॥ \textbf{  26} \newline
                  \newline
                      (ए॒वोपै॒ -त- म॑स्मि॒न् - त्रयो॑दशच)  \textbf{(A5)} \newline \newline
                                \textbf{ TS 2.3.6.1} \newline
                  प्र॒जाप॑ति॒रिति॑ प्र॒जा - प॒तिः॒ । दे॒वेभ्यः॑ । अ॒न्नाद्य॒मित्य॑न्न - अद्य᳚म् । व्यादि॑श॒दिति॑ वि - आदि॑शत् । सः । अ॒ब्र॒वी॒त् । यत् । इ॒मान् । लो॒कान् । अ॒भीति॑ । अ॒ति॒रिच्या॑ता॒ इत्य॑ति - रिच्या॑तै । तत् । मम॑ । अ॒स॒त् । इति॑ । तत् । इ॒मान् । लो॒कान् । अ॒भि । अतीति॑ । अ॒रि॒च्य॒त । इन्द्र᳚म् । राजा॑नम् । इन्द्र᳚म् । अ॒धि॒रा॒जमित्य॑धि-रा॒जम् ।   इन्द्र᳚म् । स्व॒राजा॑न॒मिति॑ स्व - राजा॑नम् । ततः॑ । वै । सः । इ॒मान् । लो॒कान् । त्रे॒धा । अ॒दु॒ह॒त् । तत् । त्रि॒धातो॒रिति॑ त्रि - धातोः᳚ । त्रि॒धा॒तु॒त्वमिति॑ त्रिधातु - त्वम् । यम् । का॒मये॑त । अ॒न्ना॒द इत्य॑न्न - अ॒दः । स्या॒त् । इति॑ । तस्मै᳚ । ए॒तम् । त्रि॒धातु॒मिति॑ त्रि - धातु᳚म् । निरिति॑ । व॒पे॒त् । इन्द्रा॑य । राज्ञे᳚ । पु॒रो॒डाश᳚म् । \textbf{  27} \newline
                  \newline
                                \textbf{ TS 2.3.6.2} \newline
                  एका॑दशकपाल॒मित्येका॑दश - क॒पा॒ल॒म् । इन्द्रा॑य । अ॒धि॒रा॒जायेत्य॑धि - रा॒जाय॑ । इन्द्रा॑य । स्व॒राज्ञ्॒ इति॑ स्व - राज्ञे᳚ । अ॒यम् । वै । इन्द्रः॑ । राजा᳚ । अ॒यम् । इन्द्रः॑ । अ॒धि॒रा॒ज इत्य॑धि - रा॒जः । अ॒सौ । इन्द्रः॑ । स्व॒राडिति॑ स्व - राट् । इ॒मान् । ए॒व । लो॒कान् । स्वेन॑ । भा॒ग॒धेये॒नेति॑ भाग - धेये॑न । उपेति॑ । धा॒व॒ति॒ । ते । ए॒व । अ॒स्मै॒ । अन्न᳚म् । प्रेति॑ । य॒च्छ॒न्ति॒ । अ॒न्ना॒द इत्य॑न्न - अ॒दः । ए॒व । भ॒व॒ति॒ । यथा᳚ । व॒थ्सेन॑ । प्रत्ता᳚म् । गाम् । दु॒हे । ए॒वम् । ए॒व ।  इ॒मान् । लो॒कान् । प्रत्तान्॑ । काम᳚म् । अ॒न्नाद्य॒मित्य॑न्न - अद्य᳚म् । दु॒हे॒ । उ॒त्ता॒नेष्वित्यु॑त्-ता॒नेषु॑ । क॒पाले॑षु । अधीति॑ । श्र॒य॒ति॒ । अया॑तयामत्वा॒येत्यया॑तयाम - त्वा॒य॒ । त्रयः॑ ( ) । पु॒रो॒डाशाः᳚ । भ॒व॒न्ति॒ । त्रयः॑ । इ॒मे । लो॒काः । ए॒षाम् । लो॒काना᳚म् । आप्त्यै᳚ । उत्त॑र उत्तर॒ इत्युत्त॑रः - उ॒त्त॒रः॒ । ज्यायान्॑ । भ॒व॒ति॒ । ए॒वम् । इ॒व॒ । हि । इ॒मे । लो॒काः । समृ॑द्ध्या॒ इति॒ सं-ऋ॒द्ध्यै॒ ।  सर्वे॑षाम् । अ॒भि॒ग॒मय॒न्नित्य॑भि - ग॒मयन्न्॑ । अवेति॑ । द्य॒ति॒ । अछ॑बंट्कार॒मित्यछ॑बंट् - का॒र॒म् । व्य॒त्यास॒मिति॑ वि - अ॒त्यास᳚म् । अन्विति॑ । आ॒ह॒ । अनि॑र्दाहा॒येत्यनिः॑ - दा॒हा॒य॒ ॥ \textbf{  28} \newline
                  \newline
                      (पु॒रो॒डाशं॒ - त्रयः॒ - षड्विꣳ॑शतिश्च)  \textbf{(A6)} \newline \newline
                                \textbf{ TS 2.3.7.1} \newline
                  दे॒वा॒सु॒रा इति॑ देव - अ॒सु॒राः । संॅय॑त्ता॒ इति॒ सं-य॒त्ताः॒ । आ॒स॒न्न् । तान् । दे॒वान् । असु॑राः । अ॒ज॒य॒न्न् । ते । दे॒वाः । प॒रा॒जि॒ग्या॒ना इति॑ परा - जि॒ग्या॒नाः । असु॑राणाम् । वैश्य᳚म् । उपेति॑ । आ॒य॒न्न् । तेभ्यः॑ । इ॒न्द्रि॒यम् । वी॒र्य᳚म् । अपेति॑ । अ॒क्रा॒म॒त् । तत् । इन्द्रः॑ । अ॒चा॒य॒त् । तत् । अनु॑ । अपेति॑ । अ॒क्रा॒म॒त् । तत् । अ॒व॒रुध॒मित्य॑व - रुध᳚म् । न । अ॒श॒क्नो॒त् । तत् । अ॒स्मा॒त् । अ॒भ्य॒र्द्ध इत्य॑भि - अ॒र्द्धः । अ॒च॒र॒त् । सः । प्र॒जाप॑ति॒मिति॑ प्र॒जा - प॒ति॒म् । उपेति॑ । अ॒धा॒व॒त् । तम् । ए॒तया᳚ । सर्व॑पृष्ठ॒येति॒ सर्व॑ - पृ॒ष्ठ॒या॒ । अ॒या॒ज॒य॒त् । तया᳚ । ए॒व । अ॒स्मि॒न्न् । इ॒न्द्रि॒यम् । वी॒र्य᳚म् । अ॒द॒धा॒त् । यः । इ॒न्द्रि॒यका॑म॒ इती᳚न्द्रि॒य - का॒मः॒ । \textbf{  29} \newline
                  \newline
                                \textbf{ TS 2.3.7.2} \newline
                  वी॒र्य॑काम॒ इति॑ वी॒र्य॑ - का॒मः॒ । स्यात् । तम् । ए॒तया᳚ । सर्व॑पृष्ठ॒येति॒ सर्व॑-पृ॒ष्ठ॒या॒ । या॒ज॒ये॒त् । ए॒ताः । ए॒व । दे॒वताः᳚ । स्वेन॑ । भा॒ग॒धेये॒नेति॑ भाग-धेये॑न । उपेति॑ । धा॒व॒ति॒ । ताः । ए॒व । अ॒स्मि॒न्न् । इ॒न्द्रि॒यम् । वी॒र्य᳚म् । द॒ध॒ति॒ । यत् । इन्द्रा॑य । राथ॑न्तरा॒येति॒ राथ᳚म् - त॒रा॒य॒ । नि॒र्वप॒तीति॑ निः - वप॑ति । यत् । ए॒व । अ॒ग्नेः । तेजः॑ ।   तत् । ए॒व । अवेति॑ । रु॒न्धे॒ । यत् । इन्द्रा॑य । बार्.ह॑ताय । यत् । ए॒व । इन्द्र॑स्य । तेजः॑ । तत् । ए॒व । अवेति॑ ।   रु॒न्धे॒ । यत् । इन्द्रा॑य । वै॒रू॒पाय॑ । यत् । ए॒व । स॒वि॒तुः । तेजः॑ । तत् । \textbf{  30} \newline
                  \newline
                                \textbf{ TS 2.3.7.3} \newline
                  ए॒व । अवेति॑ । रु॒न्धे॒ । यत् । इन्द्रा॑य । वै॒रा॒जाय॑ । यत् । ए॒व । धा॒तुः । तेजः॑ । तत् । ए॒व । अवेति॑ । रु॒न्धे॒ । यत् । इन्द्रा॑य । शा॒क्व॒राय॑ । यत् । ए॒व । म॒रुता᳚म् । तेजः॑ । तत् । ए॒व । अवेति॑ । रु॒न्धे॒ । यत् । इन्द्रा॑य । रै॒व॒ताय॑ । यत् । ए॒व । बृह॒स्पतेः᳚ । तेजः॑ । तत् । ए॒व । अवेति॑ । रु॒न्धे॒ । ए॒ताव॑न्ति । वै । तेजाꣳ॑सि । तानि॑ । ए॒व । अवेति॑ । रु॒न्धे॒ । उ॒त्ता॒नेष्वित्यु॑त् - ता॒नेषु॑ । क॒पाले॑षु । अधीति॑ । श्र॒य॒ति॒ । अया॑तयामत्वा॒येत्यया॑तयाम - त्वा॒य॒ । द्वाद॑शकपाल॒ इति॒ द्वाद॑श - क॒पा॒लः॒ । पु॒रो॒डाशः॑ । \textbf{  31} \newline
                  \newline
                                \textbf{ TS 2.3.7.4} \newline
                  भ॒व॒ति॒ । वै॒श्व॒दे॒व॒त्वायेति॑ वैश्वदेव - त्वाय॑ । स॒म॒न्तमिति॑ सं - अ॒न्तम् । प॒र्यव॑द्य॒तीति॑ परि - अव॑द्यति । स॒म॒न्तमिति॑ सं - अ॒न्तम् । ए॒व ।  इ॒न्द्रि॒यम् । वी॒र्य᳚म् । यज॑माने । द॒धा॒ति॒ । व्य॒त्यास॒मिति॑ वि - अ॒त्यास᳚म् । अन्विति॑ । आ॒ह॒ । अनि॑र्दाहा॒येत्यनिः॑ - दा॒हा॒य॒ । अश्वः॑ ।   ऋ॒ष॒भः । वृ॒ष्णिः । ब॒स्तः । सा । दक्षि॑णा । वृ॒ष॒त्वायेति॑ वृष - त्वाय॑ । ए॒तया᳚ । ए॒व । य॒जे॒त॒ । अ॒भि॒श॒स्यमा॑न॒ इत्य॑भि - श॒स्यमा॑नः । ए॒ताः । च॒ । इत् । वै । अ॒स्य॒ । दे॒वताः᳚ । अन्न᳚म् । अ॒दन्ति॑ । अ॒दन्ति॑ । उ॒ । ए॒व । अ॒स्य॒ । म॒नु॒ष्याः᳚ ॥ \textbf{  32 } \newline
                  \newline
                      (इ॒न्द्रि॒य॒का॑मः-सवि॒तुस्तेज॒स्तत् - पु॑रो॒डाशो॒ -ऽष्टात्रिꣳ॑शच्च)  \textbf{(A7)} \newline \newline
                                \textbf{ TS 2.3.8.1} \newline
                  रज॑नः । वै । कौ॒णे॒यः । क्र॒तु॒जित॒मिति॑ क्रतु - जित᳚म् । जान॑किम् । च॒क्षु॒र्वन्य॒मिति॑ चक्षुः - वन्य᳚म् । अ॒या॒त् । तस्मै᳚ । ए॒ताम् । इष्टि᳚म् । निरिति॑ । अ॒व॒प॒त् । अ॒ग्नये᳚ । भ्राज॑स्वते । पु॒रो॒डाश᳚म् । अ॒ष्टाक॑पाल॒मित्य॒ष्टा - क॒पा॒ल॒म् । सौ॒र्यम् । च॒रुम् । अ॒ग्नये᳚ । भ्राज॑स्वते । पु॒रो॒डाश᳚म् । अ॒ष्टाक॑पाल॒मित्य॒ष्टा - क॒पा॒ल॒म् । तया᳚ । ए॒व । अ॒स्मि॒न्न् । चक्षुः॑ । अ॒द॒धा॒त् । यः । चक्षु॑ष्काम॒ इति॒ चक्षुः॑ - का॒मः॒ । स्यात् । तस्मै᳚ । ए॒ताम् । इष्टि᳚म् । निरिति॑ । व॒पे॒त् । अ॒ग्नये᳚ । भ्राज॑स्वते । पु॒रो॒डाश᳚म् । अ॒ष्टाक॑पाल॒मित्य॒ष्टा - क॒पा॒ल॒म् । सौ॒र्यम् । च॒रुम् । अ॒ग्नये᳚ । भ्राज॑स्वते । पु॒रो॒डाश᳚म् । अ॒ष्टाक॑पाल॒मित्य॒ष्टा - क॒पा॒ल॒म् । अ॒ग्नेः । वै । चक्षु॑षा । म॒नु॒ष्याः᳚ । वीति॑ । \textbf{  33} \newline
                  \newline
                                \textbf{ TS 2.3.8.2} \newline
                  प॒श्य॒न्ति॒ । सूर्य॑स्य । दे॒वाः । अ॒ग्निम् । च॒ । ए॒व । सूर्य᳚म् । च॒ । स्वेन॑ । भा॒ग॒धेये॒नेति॑ भाग - धेये॑न । उपेति॑ । धा॒व॒ति॒ । तौ । ए॒व । अ॒स्मि॒न्न् । चक्षुः॑ । ध॒त्तः॒ । चक्षु॑ष्मान् । ए॒व । भ॒व॒ति॒ । यत् । आ॒ग्ने॒यौ । भव॑तः । चक्षु॑षी॒ इति॑ । ए॒व । अ॒स्मि॒न्न् । तत् । प्रतीति॑ । द॒धा॒ति॒ । यत् । सौ॒र्यः । नासि॑काम् । तेन॑ । अ॒भितः॑ । सौ॒र्यम् । आ॒ग्ने॒यौ । भ॒व॒तः॒ । तस्मा᳚त् । अ॒भितः॑ । नासि॑काम् । चक्षु॑षी॒ इति॑ । तस्मा᳚त् । नासि॑कया । चक्षु॑षी॒ इति॑ । विधृ॑ते॒ इति॒ वि - धृ॒ते॒ । स॒मा॒नी इति॑ । या॒ज्या॒नु॒वा॒क्ये॑ इति॑ याज्या - अ॒नु॒वा॒क्ये᳚ । भ॒व॒तः॒ । स॒मा॒नम् । हि ( ) । चक्षुः॑ । समृ॑द्ध्या॒ इति॒ सं-ऋ॒द्ध्यै॒ । उदिति॑ ।   उ॒ । त्यम् । जा॒तवे॑दस॒मिति॑ जा॒त - वे॒द॒स॒म् । स॒प्त । त्वा॒ । ह॒रितः॑ । रथे᳚ । चि॒त्रम् । दे॒वाना᳚म् । उदिति॑ । अ॒गा॒त् । अनी॑कम् । इति॑ । पिण्डान्॑ । प्रेति॑ । य॒च्छ॒ति॒ । चक्षुः॑ । ए॒व । अ॒स्मै॒ । प्रेति॑ । य॒च्छ॒ति॒ । यत् । ए॒व । तस्य॑ । तत् ॥ \textbf{  34 } \newline
                  \newline
                      (वि - ह्य॑ - ष्टाविꣳ॑शतिश्च)  \textbf{(A8)} \newline \newline
                                \textbf{ TS 2.3.9.1} \newline
                  ध्रु॒वः । अ॒सि॒ । ध्रु॒वः । अ॒हम् । स॒जा॒तेष्विति॑ स - जा॒तेषु॑ । भू॒या॒स॒म् । धीरः॑ । चेत्ता᳚ । व॒सु॒विदिति॑ वसु - वित् । ध्रु॒वः । अ॒सि॒ । ध्रु॒वः । अ॒हम् । स॒जा॒तेष्विति॑ स - जा॒तेषु॑ । भू॒या॒स॒म् । उ॒ग्रः । चेत्ता᳚ । व॒सु॒विदिति॑ वसु - वित् । ध्रु॒वः । अ॒सि॒ ।  ध्रु॒वः । अ॒हम् । स॒जा॒तेष्विति॑ स - जा॒तेषु॑ । भू॒या॒स॒म् । अ॒भि॒भूरित्य॑भि - भूः । चेत्ता᳚ । व॒सु॒विदिति॑ वसु - वित् । आम॑न॒मित्या - म॒न॒म् । अ॒सि॒ । आम॑न॒स्येत्या - म॒न॒स्य॒ । दे॒वाः॒ । ये ।  स॒जा॒ता इति॑ स - जा॒ताः । कु॒मा॒राः । सम॑नस॒ इति॒ स - म॒न॒सः॒ । तान् । अ॒हम् । का॒म॒ये॒ । हृ॒दा । ते । माम् । का॒म॒य॒न्ता॒म् । हृ॒दा । तान् । मे॒ । आम॑नस॒ इत्या - म॒न॒सः॒ । कृ॒धि॒ । स्वाहा᳚ । आम॑न॒मित्या - म॒न॒म् । अ॒सि॒ । \textbf{  35} \newline
                  \newline
                                \textbf{ TS 2.3.9.2} \newline
                  आम॑न॒स्येत्या - म॒न॒स्य॒ । दे॒वाः॒ । याः । स्त्रियः॑ । सम॑नस॒ इति॒ स - म॒न॒सः॒ । ताः । अ॒हम् । का॒म॒ये॒ । हृ॒दा । ताः । माम् । का॒म॒य॒न्ता॒म् । हृ॒दा । ताः । मे॒ । आम॑नस॒ इत्या - म॒न॒सः॒ । कृ॒धि॒ । स्वाहा᳚ । वै॒श्व॒दे॒वीमिति॑ वैश्व - दे॒वीम् । सा॒ग्रं॒ह॒णीमिति॑ सां - ग्र॒ह॒णीम् । निरिति॑ । व॒पे॒त् । ग्राम॑काम॒ इति॒ ग्राम॑ - का॒मः॒ । वै॒श्व॒दे॒वा इति॑ वैश्व - दे॒वाः । वै । स॒जा॒ता इति॑ स - जा॒ताः । विश्वान्॑ । ए॒व । दे॒वान् । स्वेन॑ । भा॒ग॒धेये॒नेति॑ भाग - धेये॑न । उपेति॑ । धा॒व॒ति॒ । ते । ए॒व । अ॒स्मै॒ । स॒जा॒तानिति॑ स-जा॒तान् । प्रेति॑ । य॒च्छ॒न्ति॒ । ग्रा॒मी ।  ए॒व । भ॒व॒ति॒ । सा॒ग्रं॒ह॒णीति॑ सां - ग्र॒ह॒णी । भ॒व॒ति॒ । म॒नो॒ग्रह॑ण॒मिति॑ मनः - ग्रह॑णम् । वै । स॒ग्रंह॑ण॒मिति॑ सं - ग्रह॑णम् । मनः॑ ।  ए॒व । स॒जा॒ताना॒मिति॑ स - जा॒ताना᳚म् । \textbf{  36} \newline
                  \newline
                                \textbf{ TS 2.3.9.3} \newline
                  गृ॒ह्णा॒ति॒ । ध्रु॒वः । अ॒सि॒ । ध्रु॒वः । अ॒हम् । स॒जा॒तेष्विति॑ स-जा॒तेषु॑ । भू॒या॒स॒म् । इति॑ । प॒रि॒धीनिति॑ परि - धीन् । परीति॑ । द॒धा॒ति॒ । आ॒शिष॒मित्या᳚ - शिष᳚म् । ए॒व । ए॒ताम् । एति॑ । शा॒स्ते॒ । अथो॒ इति॑ । ए॒तत् । ए॒व । सर्व᳚म् । स॒जा॒तेष्विति॑ स-जा॒तेषु॑ । अधीति॑ । भ॒व॒ति॒ । यस्य॑ । ए॒वम् । वि॒दुषः॑ । ए॒ते । प॒रि॒धय॒ इति॑ परि - धयः॑ । प॒रि॒धी॒यन्त॒ इति॑ परि - धी॒यन्ते᳚ । आम॑न॒मित्या - म॒न॒म् । अ॒सि॒ । आम॑न॒स्येत्या - म॒न॒स्य॒ । दे॒वाः॒ । इति॑ । ति॒स्रः । आहु॑ती॒रित्या - हु॒तीः॒ । जु॒हो॒ति॒ । ए॒ताव॑न्तः । वै । स॒जा॒ता इति॑ स-जा॒ताः । ये । म॒हान्तः॑ । ये । क्षु॒ल्ल॒काः । याः । स्त्रियः॑ । तान् ।  ए॒व । अवेति॑ । रु॒न्धे॒ ( ) । ते । ए॒न॒म् । अव॑रुद्धा॒ इत्यव॑ - रु॒द्धाः॒ । उपेति॑ । ति॒ष्ठ॒न्ते॒ ॥ \textbf{  37} \newline
                  \newline
                      (स्वाहा ऽऽम॑नमसि - सजा॒तानाꣳ॑ - रुन्धे॒ - पञ्च॑ च )  \textbf{(A9)} \newline \newline
                                \textbf{ TS 2.3.10.1} \newline
                  यत् । नव᳚म् । ऐत् । तत् । नव॑नीत॒मिति॒ नव॑ - नी॒त॒म् । अ॒भ॒व॒त् । यत् । अस॑र्पत् । तत् । स॒र्पिः । अ॒भ॒व॒त् । यत् । अद्ध्रि॑यत । तत् । घृ॒तम् । अ॒भ॒व॒त् । अ॒श्विनोः᳚ । प्रा॒ण इति॑ प्र-अ॒नः । अ॒सि॒ । तस्य॑ । ते॒ । द॒त्ता॒म् । ययोः᳚ । प्रा॒ण इति॑ प्र - अ॒नः । असि॑ । स्वाहा᳚ । इन्द्र॑स्य । प्रा॒ण इति॑ प्र-अ॒नः । अ॒सि॒ । तस्य॑ । ते॒ । द॒दा॒तु॒ । यस्य॑ । प्रा॒ण इति॑ प्र - अ॒नः । असि॑ । स्वाहा᳚ । मि॒त्रावरु॑णयो॒रिति॑ मि॒त्रा - वरु॑णयोः । प्रा॒ण इति॑ प्र - अ॒नः । अ॒सि॒ । तस्य॑ । ते॒ । द॒त्ता॒म् । ययोः᳚ । प्रा॒ण इति॑ प्र - अ॒नः । असि॑ । स्वाहा᳚ । विश्वे॑षाम् । दे॒वाना᳚म् । प्रा॒ण इति॑ प्र - अ॒नः । अ॒सि॒ । \textbf{  38} \newline
                  \newline
                                \textbf{ TS 2.3.10.2} \newline
                  तस्य॑ । ते॒ । द॒द॒तु॒ । येषा᳚म् । प्रा॒ण इति॑ प्र-अ॒नः । असि॑ । स्वाहा᳚ । घृ॒तस्य॑ । धारा᳚म् । अ॒मृत॑स्य । पन्था᳚म् । इन्द्रे॑ण । द॒त्ताम् । प्रय॑ता॒मिति॒ प्र - य॒ता॒म् । म॒रुद्भि॒रिति॑ म॒रुत् - भिः॒ ॥ तत् । त्वा॒ । विष्णुः॑ । परीति॑ । अ॒प॒श्य॒त् । तत् । त्वा॒ । इडा᳚ । गवि॑ । ऐर॑यत् ॥ पा॒व॒मा॒नेन॑ । त्वा॒ । स्तोमे॑न । गा॒य॒त्रस्य॑ । व॒र्त॒न्या । उ॒पाꣳ॒॒शोरित्यु॑प-अꣳ॒॒शोः । वी॒र्ये॑ण । दे॒वः । त्वा॒ । स॒वि॒ता । उदिति॑ । सृ॒ज॒तु॒ । जी॒वात॑वे । जी॒व॒न॒स्यायै᳚ । बृ॒ह॒द्र॒थ॒न्त॒रयो॒रिति॑ बृहत् - र॒थ॒न्त॒रयोः᳚ । त्वा॒ । स्तोमे॑न । त्रि॒ष्टुभः॑ । व॒र्त॒न्या । शु॒क्रस्य॑ । वी॒र्ये॑ण । दे॒वः । त्वा॒ । स॒वि॒ता । उदिति॑ । \textbf{  39} \newline
                  \newline
                                \textbf{ TS 2.3.10.3} \newline
                  सृ॒ज॒तु॒ । जी॒वात॑वे । जी॒व॒न॒स्यायै᳚ । अ॒ग्नेः । त्वा॒ । मात्र॑या । जग॑त्यै । व॒र्त॒न्या । आ॒ग्र॒य॒णस्य॑ । वी॒र्ये॑ण । दे॒वः । त्वा॒ । स॒वि॒ता । उदिति॑ । सृ॒ज॒तु॒ । जी॒वात॑वे । जी॒व॒न॒स्यायै᳚ । इ॒मम् । अ॒ग्ने॒ । आयु॑षे । वर्च॑से । कृ॒धि॒ । प्रि॒यम् । रेतः॑ । व॒रु॒ण॒ । सो॒म॒ । रा॒ज॒न्न् ॥ मा॒ता । इ॒व॒ । अ॒स्मै॒ । अ॒दि॒ते॒ । शर्म॑ । य॒च्छ॒ । विश्वे᳚ । दे॒वाः॒ । जर॑दष्टि॒रिति॒ जर॑त् - अ॒ष्टिः॒ । यथा᳚ । अस॑त् ॥ अ॒ग्निः । आयु॑ष्मान् । सः । वन॒स्पति॑भि॒रिति॒ वन॒स्पति॑-भिः॒ । आयु॑ष्मान् । तेन॑ । त्वा॒ । आयु॑षा । आयु॑ष्मन्तम् । क॒रो॒मि॒ । सोमः॑ । आयु॑ष्मान् ( ) । सः । ओष॑धीभि॒रित्योष॑धि - भिः॒ । य॒ज्ञ्ः । आयु॑ष्मान् । सः । दक्षि॑णाभिः । ब्रह्म॑ । आयु॑ष्मत् । तत् । ब्रा॒ह्म॒णैः । आयु॑ष्मत् । दे॒वाः । आयु॑ष्मन्तः । ते । अ॒मृते॑न । पि॒तरः॑ । आयु॑ष्मन्तः । ते । स्व॒धयेति॑ स्व - धया᳚ । आयु॑ष्मन्तः । तेन॑ । त्वा॒ । आयु॑षा । आयु॑ष्मन्तम् । क॒रो॒मि॒ ॥ \textbf{  40} \newline
                  \newline
                      (विश्वे॑षां दे॒वानां᳚ प्रा॒णो॑ऽसि - त्रि॒ष्टुभो॑ वर्त॒न्या शु॒क्रस्य॑ वी॒र्ये॑ण दे॒वस्त्वा॑ सवि॒तोथ् - सोम॒ आयु॑ष्मा॒न् - पञ्च॑विꣳशतिश्च)  \textbf{(A10)} \newline \newline
                                \textbf{ TS 2.3.11.1} \newline
                  अ॒ग्निम् । वै । ए॒तस्य॑ । शरी॑रम् । ग॒च्छ॒ति॒ । सोम᳚म् । रसः॑ । वरु॑णः । ए॒न॒म् । व॒रु॒ण॒पा॒शेनेति॑ वरुण - पा॒शेन॑ । गृ॒ह्णा॒ति॒ । सर॑स्वतीम् । वाक् । अ॒ग्नाविष्णू॒ इत्य॒ग्ना - विष्णू᳚ । आ॒त्मा । यस्य॑ । ज्योक् । आ॒मय॑ति । यः । ज्योगा॑मया॒वीति॒ ज्योक् - आ॒म॒या॒वी॒ । स्यात् । यः । वा॒ । का॒मये॑त । सर्व᳚म् । आयुः॑ । इ॒या॒म् । इति॑ । तस्मै᳚ । ए॒ताम् । इष्टि᳚म् । निरिति॑ । व॒पे॒त् । आ॒ग्ने॒यम् । अ॒ष्टाक॑पाल॒मित्य॒ष्टा - क॒पा॒ल॒म् । सौ॒म्यम् । च॒रुम् । वा॒रु॒णम् । दश॑कपाल॒मिति॒ दश॑ - क॒पा॒ल॒म् । सा॒र॒स्व॒तम् । च॒रुम् । आ॒ग्ना॒वै॒ष्ण॒वमित्या᳚ग्ना - वै॒ष्ण॒वम् । एका॑दशकपाल॒मित्येका॑दश - क॒पा॒ल॒म् । अ॒ग्नेः । ए॒व । अ॒स्य॒ । शरी॑रम् । नि॒ष्क्री॒णातीति॑ निः - क्री॒णाति॑ । सोमा᳚त् । रस᳚म् । \textbf{  41} \newline
                  \newline
                                \textbf{ TS 2.3.11.2} \newline
                  वा॒रु॒णेन॑ । ए॒व । ए॒न॒म् । व॒रु॒ण॒पा॒शादिति॑ वरुण - पा॒शात् । मु॒ञ्च॒ति॒ । सा॒र॒स्व॒तेन॑ । वाच᳚म् । द॒धा॒ति॒ । अ॒ग्निः । सर्वाः᳚ । दे॒वताः᳚ । विष्णुः॑ । य॒ज्ञ्ः । दे॒वता॑भिः । च॒ । ए॒व । ए॒न॒म् । य॒ज्ञेन॑ । च॒ । भि॒ष॒ज्य॒ति॒ । उ॒त । यदि॑ । इ॒तासु॒रिती॒त - अ॒सुः॒ । भव॑ति । जीव॑ति । ए॒व । यत् । नव᳚म् । ऐत् । तत् । नव॑नीत॒मिति॒ नव॑-नी॒त॒म् । अ॒भ॒व॒त् । इति॑ । आज्य᳚म् । अवेति॑ । ई॒क्ष॒ते॒ । रू॒पम् । ए॒व । अ॒स्य॒ । ए॒तत् । म॒हि॒मान᳚म् । व्याच॑ष्ट॒ इति॑ वि-आच॑ष्टे । अ॒श्विनोः᳚ । प्रा॒ण इति॑ प्र - अ॒नः । अ॒सि॒ । इति॑ । आ॒ह॒ । अ॒श्विनौ᳚ । वै । दे॒वाना᳚म् । \textbf{  42} \newline
                  \newline
                                \textbf{ TS 2.3.11.3} \newline
                  भि॒षजौ᳚ । ताभ्या᳚म् । ए॒व । अ॒स्मै॒ । भे॒ष॒जम् । क॒रो॒ति॒ । इन्द्र॑स्य । प्रा॒ण इति॑ प्र - अ॒नः । अ॒सि॒ । इति॑ । आ॒ह॒ । इ॒न्द्रि॒यम् । ए॒व । अ॒स्मि॒न्न् । ए॒तेन॑ । द॒धा॒ति॒ । मि॒त्रावरु॑णया॒रिति॑ मि॒त्रा - वरु॑णयोः । प्रा॒ण इति॑ प्र - अ॒नः । अ॒सि॒ । इति॑ । आ॒ह॒ । प्रा॒णा॒पा॒नाविति॑ प्राण - अ॒पा॒नौ । ए॒व । अ॒स्मि॒न्न् । ए॒तेन॑ । द॒धा॒ति॒ । विश्वे॑षाम् । दे॒वाना᳚म् । प्रा॒ण इति॑ प्र - अ॒नः । अ॒सि॒ । इति॑ । आ॒ह॒ । वी॒र्य᳚म् । ए॒व । अ॒स्मि॒न्न् । ए॒तेन॑ । द॒धा॒ति॒ । घृ॒तस्य॑ । धारा᳚म् । अ॒मृत॑स्य । पन्था᳚म् । इति॑ । आ॒ह॒ । य॒था॒य॒जुरिति॑ यथा - य॒जुः । ए॒व । ए॒तत् । पा॒व॒मा॒नेन॑ । त्वा॒ । स्तोमे॑न । इति॑ । \textbf{  43} \newline
                  \newline
                                \textbf{ TS 2.3.11.4} \newline
                  आ॒ह॒ । प्रा॒णमिति॑ प्र - अ॒नम् । ए॒व । अ॒स्मि॒न्न् । ए॒तेन॑ । द॒धा॒ति॒ । बृ॒ह॒द्र॒थ॒न्त॒रयो॒रिति॑ बृहत्-र॒थ॒न्त॒रयोः᳚ । त्वा॒ । स्तोमे॑न । इति॑ । आ॒ह॒ । ओजः॑ । ए॒व । अ॒स्मि॒न्न् । ए॒तेन॑ । द॒धा॒ति॒ । अ॒ग्नेः । त्वा॒ । मात्र॑या । इति॑ । आ॒ह॒ । आ॒त्मान᳚म् । ए॒व । अ॒स्मि॒न्न् । ए॒तेन॑ । द॒धा॒ति॒ । ऋ॒त्विजः॑ । परीति॑ । आ॒हुः॒ । याव॑न्तः । ए॒व । ऋ॒त्विजः॑ । ते । ए॒न॒म् । भि॒ष॒ज्य॒न्ति॒ । ब्र॒ह्मणः॑ । हस्त᳚म् । अ॒न्वा॒रभ्येत्य॑नु - आ॒रभ्य॑ । परीति॑ । आ॒हुः॒ । ए॒क॒धेत्ये॑क - धा । ए॒व । यज॑माने । आयुः॑ । द॒ध॒ति॒ । यत् । ए॒व । तस्य॑ । तत् । हिर॑ण्यात् । \textbf{  44} \newline
                  \newline
                                \textbf{ TS 2.3.11.5} \newline
                  घृ॒तम् । निरिति॑ । पि॒ब॒ति॒ । आयुः॑ । वै । घृ॒तम् । अ॒मृत᳚म् । हिर॑ण्यम् । अ॒मृता᳚त् । ए॒व । आयुः॑ । निरिति॑ । पि॒ब॒ति॒ । श॒तमा॑न॒मिति॑ श॒त - मा॒न॒म् । भ॒व॒ति॒ । श॒तायु॒रिति॑ श॒त - आ॒युः॒ । पुरु॑षः । श॒तेन्द्रि॑य॒ इति॑ श॒त - इ॒न्द्रि॒यः॒ । आयु॑षि । ए॒व । इ॒न्द्रि॒ये । प्रतीति॑ । ति॒ष्ठ॒ति॒ । अथो॒ इति॑ । खलु॑ । याव॑तीः । समाः᳚ । ए॒ष्यन्न् । मन्ये॑त । ताव॑न्मान॒मिति॒ ताव॑त् - मा॒न॒म् । स्या॒त् । समृ॑द्ध्या॒ इति॒ सम् - ऋ॒द्ध्यै॒ । इ॒मम् । अ॒ग्ने॒ । आयु॑षे । वर्च॑से । कृ॒धि॒ । इति॑ । आ॒ह॒ । आयुः॑ । ए॒व । अ॒स्मि॒न्न् । वर्चः॑ । द॒धा॒ति॒ । विश्वे᳚ । दे॒वाः॒ । जर॑दष्टि॒रिति॒ जर॑त् - अ॒ष्टिः॒ । यथा᳚ । अस॑त् । इति॑ ( ) । आ॒ह॒ । जर॑दष्टि॒मिति॒ जर॑त् - अ॒ष्टि॒म् । ए॒व । ए॒न॒म् । क॒रो॒ति॒ । अ॒ग्निः । आयु॑ष्मान् । इति॑ । हस्त᳚म् । गृ॒ह्णा॒ति॒ । ए॒ते । वै । दे॒वाः । आयु॑ष्मन्तः । ते । ए॒व । अ॒स्मि॒न्न् । आयुः॑ । द॒ध॒ति॒ । सर्व᳚म् । आयुः॑ । ए॒ति॒ ॥ \textbf{  45} \newline
                  \newline
                      (रसं॑-दे॒वानाꣳ॒॒-स्तोमे॒नेति॒-हिर॑ण्या॒-दस॒दिति॒-द्वाविꣳ॑शतिश्च) \textbf{(A11)} \newline \newline
                                \textbf{ TS 2.3.12.1} \newline
                  प्र॒जाप॑ति॒रिति॑ प्र॒जा - प॒तिः॒ । वरु॑णाय । अश्व᳚म् । अ॒न॒य॒त् । सः । स्वाम् । दे॒वता᳚म् । आ॒र्च्छ॒त् । सः । परीति॑ । अ॒दी॒र्य॒त॒ । सः । ए॒तम् । वा॒रु॒णम् । चतु॑ष्कपाल॒मिति॒ चतुः॑ - क॒पा॒ल॒म् । अ॒प॒श्य॒त् । तम् । निरिति॑ । अ॒व॒प॒त् । ततः॑ । वै । सः । व॒रु॒ण॒पा॒शादिति॑ वरुण - पा॒शात् । अ॒मु॒च्य॒त॒ । वरु॑णः । वै । ए॒तम् । गृ॒ह्णा॒ति॒ । यः । अश्व᳚म् । प्र॒ति॒गृ॒ह्णातीति॑ प्रति - गृ॒ह्णाति॑ । याव॑तः । अश्वान्॑ । प्र॒ति॒गृ॒ह्णी॒यादिति॑ प्रति - गृ॒ह्णी॒यात् । ताव॑तः । वा॒रु॒णान् । चतु॑ष्कपाला॒निति॒ चतुः॑ - क॒पा॒ला॒न् । निरिति॑ । व॒पे॒त् । वरु॑णम् । ए॒व । स्वेन॑ । भा॒ग॒धेये॒नेति॑ भाग - धेये॑न । उपेति॑ । धा॒व॒ति॒ । सः । ए॒व । ए॒न॒म् । व॒रु॒ण॒पा॒शादिति॑ वरुण - पा॒शात् । मु॒ञ्च॒ति॒ । \textbf{  46} \newline
                  \newline
                                \textbf{ TS 2.3.12.2} \newline
                  चतु॑ष्कपाला॒ इति॒ चतुः॑ - क॒पा॒लाः॒ । भ॒व॒न्ति॒ । चतु॑ष्पा॒दिति॒ चतुः॑ - पा॒त् । हि । अश्वः॑ । समृ॑द्ध्या॒ इति॒ सं - ऋ॒द्ध्यै॒ । एक᳚म् । अति॑रिक्त॒मित्यति॑ - रि॒क्त॒म् । निरिति॑ । व॒पे॒त् । यम् । ए॒व । प्र॒ति॒ग्रा॒हीति॑ प्रति - ग्रा॒ही । भव॑ति । यम् । वा॒ । न । अ॒द्ध्येतीत्य॑धि - एति॑ । तस्मा᳚त् । ए॒व । व॒रु॒ण॒पा॒शादिति॑ वरुण - पा॒शात् ।   मु॒च्य॒ते॒ । यदि॑ । अप॑रम् । प्र॒ति॒ग्रा॒हीति॑ प्रति -  ग्रा॒ही । स्यात् । सौ॒र्यम् । एक॑कपाल॒मित्येक॑ - क॒पा॒ल॒म् । अनु॑ । निरिति॑ । व॒पे॒त् । अ॒मुम् । ए॒व । आ॒दि॒त्यम् । उ॒च्चा॒रमित्यु॑त् - चा॒रम् । कु॒रु॒ते॒ । अ॒पः । अ॒व॒भृ॒थमित्य॑व -भृ॒थम् । अवेति॑ । ए॒ति॒ । अ॒फ्स्वित्य॑प् - सु । वै । वरु॑णः । सा॒क्षादिति॑ स-अ॒क्षात् । ए॒व । वरु॑णम् । अवेति॑ । य॒ज॒ते॒ । अ॒पो॒न॒प्त्रीय॒मित्य॑पः - न॒प्त्रीय᳚म् । च॒रुम् ( ) । पुनः॑ । एत्येत्या᳚ - इत्य॑ । निरिति॑ । व॒पे॒त् । अ॒फ्सुयो॑नि॒रित्य॒फ्सु - यो॒निः॒ । वै । अश्वः॑ । स्वाम् । ए॒व । ए॒न॒म् । योनि᳚म् । ग॒म॒य॒ति॒ । सः । ए॒न॒म् । शा॒न्तः । उपेति॑ । ति॒ष्ठ॒ते॒ ॥ \textbf{  47 } \newline
                  \newline
                      (मु॒ञ्च॒ति॒ - च॒रुꣳ - स॒प्तद॑श च)  \textbf{(A12)} \newline \newline
                                \textbf{ TS 2.3.13.1} \newline
                  या । वा॒म् । इ॒न्द्रा॒व॒रु॒णेती᳚न्द्रा - व॒रु॒णा॒ । य॒त॒व्या᳚ । त॒नूः । तया᳚ । इ॒मम् । अꣳह॑सः । मु॒ञ्च॒त॒म् । या । वा॒म् । इ॒न्द्रा॒व॒रु॒णेती᳚न्द्रा - व॒रु॒णा॒ । स॒ह॒स्या᳚ । र॒क्ष॒स्या᳚ । ते॒ज॒स्या᳚ । त॒नूः । तया᳚ । इ॒मम् । अꣳह॑सः । मु॒ञ्च॒त॒म् ।  यः । वा॒म् । इ॒न्द्रा॒व॒रु॒णा॒विती᳚न्द्रा - व॒रु॒णौ॒ । अ॒ग्नौ । स्रामः॑ । तम् । वा॒म् । ए॒तेन॑ । अवेति॑ । य॒जे॒ । यः । वा॒म् । इ॒न्द्रा॒व॒रु॒णेती᳚न्द्रा - व॒रु॒णा॒ । द्वि॒पाथ्स्विति॑ द्वि॒पात् - सु॒ । प॒शुषु॑ । चतु॑ष्पा॒थ्स्विति॒ चतु॑ष्पात् - सु॒ । गो॒ष्ठ इति॑ गो- स्थे । गृ॒हेषु॑ । अ॒फ्स्वित्य॑प् - सु । ओष॑धीषु । वन॒स्पति॑षु । स्रामः॑ । तम् । वा॒म् । ए॒तेन॑ । अवेति॑ । य॒जे॒ । इन्द्रः॑ । वै । ए॒तस्य॑ । \textbf{  48} \newline
                  \newline
                                \textbf{ TS 2.3.13.2} \newline
                  इ॒न्द्रि॒येण॑ । अपेति॑ । क्रा॒म॒ति॒ । वरु॑णः । ए॒न॒म् । व॒रु॒ण॒पा॒शेनेति॑ वरुण - पा॒शेन॑ । गृ॒ह्णा॒ति॒ । यः । पा॒प्मना᳚ । गृ॒ही॒तः । भव॑ति । यः ।   पा॒प्मना᳚ । गृ॒ही॒तः । स्यात् । तस्मै᳚ । ए॒ताम् । ऐ॒न्द्रा॒व॒रु॒णीमित्यै᳚न्द्रा - व॒रु॒णीम् । प॒य॒स्या᳚म् । निरिति॑ । व॒पे॒त् । इन्द्रः॑ । ए॒व । अ॒स्मि॒न्न् । इ॒न्द्रि॒यम् । द॒धा॒ति॒ । वरु॑णः । ए॒न॒म् । व॒रु॒ण॒पा॒शादिति॑ वरुण - पा॒शात् । मु॒ञ्च॒ति॒ । प॒य॒स्या᳚ । भ॒व॒ति॒ । पयः॑ । हि । वै । ए॒तस्मा᳚त् । अ॒प॒क्राम॒तीत्य॑प - क्राम॑ति । अथ॑ । ए॒षः । पा॒प्मना᳚ । गृ॒ही॒तः । यत् । प॒य॒स्या᳚ । भव॑ति । पयः॑ । ए॒व । अ॒स्मि॒न्न् । तया᳚ । द॒धा॒ति॒ । प॒य॒स्या॑याम् । \textbf{  49} \newline
                  \newline
                                \textbf{ TS 2.3.13.3} \newline
                  पु॒रो॒डाश᳚म् । अवेति॑ । द॒धा॒ति॒ । आ॒त्म॒न्वन्त॒मित्या᳚त्मन्न् - वन्त᳚म् । ए॒व । ए॒न॒म् । क॒रो॒ति॒ । अथो॒ इति॑ । आ॒यत॑नवन्त॒मित्या॒यत॑न - व॒न्त॒म् । ए॒व । च॒तु॒र्द्धेति॑ चतुः - धा । वीति॑ । ऊ॒ह॒ति॒ । दि॒क्षु । ए॒व । प्रतीति॑ । ति॒ष्ठ॒ति॒ । पुनः॑ । समिति॑ । ऊ॒ह॒ति॒ । दि॒ग्भ्य इति॑ दिक् - भ्यः । ए॒व ।   अ॒स्मै॒ । भे॒ष॒जम् । क॒रो॒ति॒ । स॒मूह्येति॑ सं - ऊह्य॑ । अवेति॑ । द्य॒ति॒ । यथा᳚ । आवि॑द्ध॒मित्या - वि॒द्ध॒म् । नि॒ष्कृ॒न्ततीति॑ निः - कृ॒न्तति॑ । ता॒दृक् । ए॒व । तत् । यः । वा॒म् । इ॒न्द्रा॒व॒रु॒णा॒विती᳚न्द्रा - व॒रु॒णौ॒ । अ॒ग्नौ । स्रामः॑ । तम् । वा॒म् । ए॒तेन॑ । अवेति॑ । य॒जे॒ । इति॑ । आ॒ह॒ । दुरि॑ष्ट्या॒ इति॒ दुः - इ॒ष्ट्याः॒ । ए॒व । ए॒न॒म् । पा॒ति॒ ( ) । यः । वा॒म् । इ॒न्द्रा॒व॒रु॒णेती᳚न्द्रा - व॒रु॒णा॒ । द्वि॒पाथ्स्विति॑ द्वि॒पात् - सु॒ । प॒शुषु॑ । स्रामः॑ । तम् । वा॒म् ।  ए॒तेन॑ । अवेति॑ । य॒जे॒ । इति॑ । आ॒ह॒ । ए॒ताव॑तीः । वै । आपः॑ । ओष॑धयः । वन॒स्पत॑यः । प्र॒जा इति॑ प्र - जाः । प॒शवः॑ । उ॒प॒जी॒व॒नीया॒ इत्यु॑प - जी॒व॒नीयाः᳚ । ताः । ए॒व । अ॒स्मै॒ । व॒रु॒ण॒पा॒शादिति॑ वरुण - पा॒शात् । मु॒ञ्च॒ति॒ ॥ \textbf{  50} \newline
                  \newline
                      (ए॒तस्य॑ - पय॒स्या॑यां - पाति॒ - षड्विꣳ॑शतिश्च )  \textbf{(A13)} \newline \newline
                                \textbf{ TS 2.3.14.1} \newline
                  सः । प्र॒त्न॒वदिति॑ प्रत्न - वत् । नीति॑ । काव्या᳚ । इन्द्र᳚म् । वः॒ । वि॒श्वतः॑ । परीति॑ । इन्द्र᳚म् । नरः॑ ॥ त्वम् । नः॒ । सो॒म॒ । वि॒श्वतः॑ । रक्ष॑ । रा॒ज॒न्न् । अ॒घा॒य॒त इत्य॑घ - य॒तः ॥ न । रि॒ष्ये॒त् । त्वाव॑त॒ इति॒ त्व - व॒तः॒ । सखा᳚ ॥ या । ते॒ । धामा॑नि । दि॒वि । या । पृ॒थि॒व्याम् । या । पर्व॑तेषु । ओष॑धीषु । अ॒फ्सित्य॑प् - सु ॥ तेभिः॑ । नः॒ । विश्वैः᳚ । सु॒मना॒ इति॑ सु-मनाः᳚ । अहे॑डन्न् । राजन्न्॑ । सो॒म॒ । प्रतीति॑ । ह॒व्या । गृ॒भा॒य॒ ॥ अग्नी॑षो॒मेत्यग्नी᳚ - सो॒मा॒ । सवे॑द॒सेति॒ स - वे॒द॒सा॒ । सहू॑ती॒ इति॒ स - हू॒ती॒ । व॒न॒त॒म् । गिरः॑ ॥ समिति॑ । दे॒व॒त्रेति॑ देव - त्रा । ब॒भू॒व॒थुः॒ ॥ यु॒वम् । \textbf{  51} \newline
                  \newline
                                \textbf{ TS 2.3.14.2} \newline
                  ए॒तानि॑ । दि॒वि । रो॒च॒नानि॑ । अ॒ग्निः । च॒ । सो॒म॒ । सक्र॑तू॒ इति॒ स - क्र॒तू॒ । अ॒ध॒त्त॒म् ॥ यु॒वम् । सिन्धून्॑ । अ॒भिश॑स्ते॒रित्य॒भि - श॒स्तेः॒ । अ॒व॒द्यात् । अग्नी॑षोमा॒वित्यग्नी᳚ - सो॒मौ॒ । अमु॑ञ्चतम् । गृ॒भी॒तान् ॥ अग्नी॑षोमा॒वित्यग्नी᳚ - सो॒मौ॒ । इ॒मम् । स्विति॑ । मे॒ । शृ॒णु॒तम् । वृ॒ष॒णा॒ । हव᳚म् । प्रतीति॑ ॥ सू॒क्ता॒नीति॑ सु - उ॒क्तानि॑ । ह॒र्य॒त॒म् । भव॑तम् । दा॒शुषे᳚ । मयः॑ ॥ एति॑ । अ॒न्यम् । दि॒वः । मा॒त॒रिश्वा᳚ । ज॒भा॒र॒ । अम॑थ्नात् । अ॒न्यम् । परीति॑ । श्ये॒नः । अद्रेः᳚ ॥ अग्नी॑षो॒मेत्यग्नी᳚ - सो॒मा॒ । ब्रह्म॑णा । वा॒वृ॒धा॒ना । उ॒रुम् । य॒ज्ञाय॑ । च॒क्र॒थुः॒ । उ॒ । लो॒कम् ॥ अग्नी॑षो॒मेत्यग्नी᳚ - सो॒मा॒ । ह॒विषः॑ । प्रस्थि॑त॒स्येति॒ प्र - स्थि॒त॒स्य॒ । वी॒तम् । \textbf{  52} \newline
                  \newline
                                \textbf{ TS 2.3.14.3} \newline
                  हर्य॑तम् । वृ॒ष॒णा॒ । जु॒षेथा᳚म् ॥ सु॒शर्मा॒णेति॑ सु - शर्मा॑णा । स्वव॒सेति॑ सु - अव॑सा । हि । भू॒तम् । अथ॑ । ध॒त्त॒म् । यज॑मानाय । शम् । योः ॥ एति॑ । प्या॒य॒स्व॒ । समिति॑ । ते॒ ॥ ग॒णाना᳚म् । त्वा॒ । ग॒णप॑ति॒मिति॑ ग॒ण - प॒ति॒म् । ह॒वा॒म॒हे॒ । क॒विम् । क॒वी॒नाम् । उ॒प॒मश्र॑वस्तम॒मित्यु॑प॒मश्र॑वः-त॒म॒म् ॥ ज्ये॒ष्ठ॒राज॒मिति॑ ज्येष्ठ-राज᳚म् । ब्रह्म॑णाम् । ब्र॒ह्म॒णः॒ । प॒ते॒ । एति॑ । नः॒ । शृ॒ण्वन्न् । ऊ॒तिभि॒रित्यू॒ति - भिः॒ । सी॒द॒ । साद॑नम् ॥ सः । इत् । जने॑न । सः । वि॒शा । सः । जन्म॑ना । सः । पु॒त्रैः । वाज᳚म् । भ॒र॒ते॒ । धना᳚ । नृभि॒रिति॒ नृ - भिः॒ ॥ दे॒वाना᳚म् । यः । पि॒तर᳚म् । आ॒विवा॑स॒तीत्या᳚ - विवा॑सति । \textbf{  53} \newline
                  \newline
                                \textbf{ TS 2.3.14.4} \newline
                  श्र॒द्धाम॑ना॒ इति॑ श्र॒द्धा - म॒नाः॒ । ह॒विषा᳚ । ब्रह्म॑णः । पति᳚म् ॥ सः । सु॒ष्टुभेति॑ सु - स्तुभा᳚ । सः । ऋक्व॑ता । ग॒णेन॑ । व॒लम् । रु॒रो॒ज॒ । फ॒लि॒गम् । रवे॑ण ॥ बृह॒स्पतिः॑ । उ॒स्त्रियाः᳚ । ह॒व्य॒सूद॒ इति॑ हव्य - सूदः॑ । कनि॑क्रदत् । वाव॑शतीः । उदिति॑ । आ॒ज॒त् ॥ मरु॑तः । यत् । ह॒ । वः॒ । दि॒वः । या । वः॒ । शर्म॑ ॥ अ॒र्य॒मा । एति॑ । या॒ति॒ । वृ॒ष॒भः । तुवि॑ष्मान् । दा॒ता । वसू॑नाम् । पु॒रु॒हू॒त इति॑ पुरु - हू॒तः । अर्.हन्न्॑ ॥ स॒ह॒स्रा॒क्ष इति॑ सहस्र - अ॒क्षः । गो॒त्र॒भिदिति॑ गोत्र-भित् । वज्र॑बाहु॒रिति॒ वज्र॑ - बा॒हुः॒ । अ॒स्मासु॑ । दे॒वः । द्रवि॑णम् । द॒धा॒तु॒ ॥ ये । ते॒ । अ॒र्य॒म॒न्न् । ब॒हवः॑ । दे॒व॒याना॒ इति॑ देव - यानाः᳚ । पन्था॑नः । \textbf{  54} \newline
                  \newline
                                \textbf{ TS 2.3.14.5} \newline
                  रा॒ज॒न्न् । दि॒वः । आ॒चर॒न्तीत्या᳚ - चर॑न्ति ॥ तेभिः॑ । नः॒ । दे॒व॒ । महि॑ । शर्म॑ । य॒च्छ॒ । शम् । नः॒ । ए॒धि॒ । द्वि॒पद॒ इति॑ द्वि - पदे᳚ । शम् । चतु॑ष्पद॒ इति॒ चतुः॑ - प॒दे॒ ॥ बु॒द्ध्नात् । अग्र᳚म् । अङ्गि॑रोभि॒रित्यङ्गि॑रः-भिः॒ । गृ॒णा॒नः । वीति॑ । पर्व॑तस्य । दृꣳ॒॒हि॒तानि॑ । ऐ॒र॒त् ॥ रु॒जत् । रोधाꣳ॑सि । कृ॒त्रिमा॑णि । ए॒षा॒म् । सोम॑स्य । ता । मदे᳚ । इन्द्रः॑ । च॒का॒र॒ ॥ बु॒द्ध्नात् । अग्रे॑ण । वीति॑ । मि॒मा॒य॒ । मानैः᳚ । वज्रे॑ण । खानि॑ । अ॒तृ॒ण॒त् । न॒दीना᳚म् ॥ वृथा᳚ । अ॒सृ॒ज॒त् । प॒थिभि॒रिति॑ प॒थि - भिः॒ । दी॒र्घ॒या॒थैरिति॑ दीर्घ - या॒थैः । सोम॑स्य । ता । मदे᳚ । इन्द्रः॑ । च॒का॒र॒ ॥ \textbf{  55} \newline
                  \newline
                                \textbf{ TS 2.3.14.6} \newline
                  प्रेति॑ । यः । ज॒ज्ञे । वि॒द्वान् । अ॒स्य । बन्धु᳚म् । विश्वा॑नि । दे॒वः । जनि॑मा । वि॒व॒क्ति॒ ॥ ब्रह्म॑ । ब्रह्म॑णः । उदिति॑ । ज॒भा॒र॒ । मद्ध्या᳚त् । नी॒चा । उ॒च्चा । स्व॒धयेति॑ स्व - धया᳚ । अ॒भि । प्रेति॑ । त॒स्थौ॒ ॥ म॒हान् । म॒ही इति॑ । अ॒स्त॒भा॒य॒त् । वीति॑ । जा॒तः । द्याम् । सद्म॑ । पार्त्थि॑वम् । च॒ । रजः॑ ॥ सः । बु॒द्ध्नात् । आ॒ष्ट॒ । ज॒नुषा᳚ । अ॒भीति॑ । अग्र᳚म् । बृह॒स्पतिः॑ । दे॒वता᳚ । यस्य॑ । स॒म्राडिति॑ सं-राट् ॥ बु॒द्ध्नात् । यः । अग्र᳚म् । अ॒भ्यर्तीत्य॑भि-अर्ति॑ । ओज॑सा । बृह॒स्पति᳚म् । एति॑ । वि॒वा॒स॒न्ति॒ । दे॒वाः ( ) ॥ भि॒नत् । व॒लम् । वीति॑ । पुरः॑ । द॒र्द॒री॒ति॒ । कनि॑क्रदत् । सुवः॑ । अ॒पः । जि॒गा॒य॒ ॥ \textbf{  56} \newline
                  \newline
                      (यु॒वं - ॅवी॒तमा॒ - विवा॑सति॒ - पन्था॑नो - दीर्घया॒थैः सोम॑स्य॒ ता मद॒ इन्द्र॑श्चकार - दे॒वा - नव॑ च)  \textbf{(A14)} \newline \newline
\textbf{praSna korvai with starting padams of 1 to 14 Anuvaakams :-} \newline
(आ॒दि॒त्येभ्यो॑ - दे॒वा वै मृ॒त्यो - र्दे॒वा वै - स॒त्रम॑ - र्य॒म्णे -प्र॒जाप॑ते॒स्त्रय॑स्त्रिꣳशत् - प्र॒जाप॑ति र्दे॒वेभ्यो॒ऽन्नाद्यं॑ -देवासु॒रास्तान् - रज॑नो - ध्रु॒वो॑ऽसि॒ - यन्नव॑ - म॒ग्निं ॅवै - प्र॒जाप॑ति॒ र्वरु॑णाय॒ - या वा॑मिन्द्रा वरुणा॒ - स प्र॑त्न॒व -च्चतु॑र्दश) \newline

\textbf{korvai with starting padams of1, 11, 21 series of pa~jcAtis :-} \newline
(आ॒दि॒त्येभ्य॒ - स्त्वष्टु॑ - रस्मै॒ दान॑कामा - ए॒वाव॑ रुन्धे॒ - ऽग्निं ॅवै - स प्र॑त्न॒वथ् - षट्प॑ञ्चा॒शत् ) \newline

\textbf{first and last padam of third praSnam of kANDam 2:-} \newline
(आ॒दि॒त्येभ्यः॒ - सुव॑र॒पो जि॑गाय ) \newline 


॥ हरिः॑ ॐ ॥॥ कृष्ण यजुर्वेदीय तैत्तिरीय संहितायां द्वितीयकाण्डे तृतीयः प्रश्नः समाप्तः ॥ \newline
\pagebreak
2.3.1   Appendix\\2.3.14.1 - स प्र॑त्न॒व>1\\स प्र॑त्न॒वन्नवी॑य॒-साऽ*ग्ने᳚ द्यु॒म्नेन॑ सं॒ॅयता᳚ । \\बृ॒हत् त॑तन्थ भा॒नुना᳚ ॥ (appearing in TS2.2.12.1)\\\\2.3.14.1 - न्नि काव्ये>2\\नि काव्या॑ वे॒धसः॒ शश्व॑तस्क॒र्.हस्ते॒ दधा॑नो॒ नर्यां॑ पु॒रूणि॑ ।\\अ॒ग्निर्भु॑वद्रयि॒पती॑ रयी॒णाꣳ स॒त्रा च॑क्रा॒णो अ॒मृता॑नि॒ विश्वा᳚ ॥\\(Appearing in TS2.2.12.2)\\\\2.3.14.1- न्द्रं॑ ॅवो वि॒श्वत॒स्परी>3\\इन्द्रं॑ ॅवो वि॒श्वत॒स्परि॒ हवा॑महे॒ जने᳚भ्यः । अ॒स्माक॑मस्तु॒ केव॑लः ॥\\(Appearing in TS1.6.12.1)\\\\2.3.14.1 - न्द्रं॒ नरः॑>4\\इन्द्रं॒ नरो॑ ने॒मधि॑ता हवन्ते॒ यत्पार्या॑ यु॒नज॑ते॒ धिय॒स्ताः ।\\शूरो॒ नृषा॑ता॒ शव॑सश्चका॒न आ गोम॑ति व्र॒जे भ॑जा॒ त्वन्नः॑ ॥\\(Appearing in TS 1.6.12.1)\\\\2.3.14.3 - आप्या॑यस्व॒>5\\आप्या॑यस्व॒ समे॑तु ते वि॒श्वतः॑ सोम॒ वृष्णि॑यं ।भवा॒ वाज॑स्य सङ्ग॒थे ॥\\(Appearing in TS 4.2.7.4)\\\\2.3.14.3- सं ते᳚>6\\सं ते॒ पयाꣳ॑सि॒ समु॑ यन्तु॒ वाजाः॒ सं ॅवृष्णि॑यान्. यभिमाति॒षाहः॑ ।\\आ॒प्याय॑मानो अ॒मृता॑य सोम दि॒वि श्रवाꣳ॑स्युत्त॒मानि॑ धिष्व ॥\\(Appearing in TS 4.2.7.4)\\\\2.3.14.4 -मरु॑तो॒ यद्ध॑वो दि॒वो>7\\मरु॑तो॒ यद्ध॑ वो दि॒वः सु॑म्ना॒ यन्तो॒ हवा॑महे । \\आ तू न॒ उप॑ गन्तन ॥ (Appearing in TS1.5.11.4 )\\\\2.3.14.4 -या वः॒ श॑र्म>8\\या वः॒ शर्म॑ शशमा॒नाय॒ सन्ति॑ त्रि॒धातू॑नि दा॒शुषे॑ यच्छ॒ताधि॑ । \\अ॒स्मभ्यं॒ तानि॑ मरुतो॒ वि य॑न्त र॒यिं नो॑ धत्त वृषणः सु॒वीरं᳚ ॥\\(Appearing in TS1.5.11.5 ) \\
\pagebreak
        


\end{document}
