\documentclass[17pt]{extarticle}
\usepackage{babel}
\usepackage{fontspec}
\usepackage{polyglossia}
\usepackage{extsizes}



\setmainlanguage{sanskrit}
\setotherlanguages{english} %% or other languages
\setlength{\parindent}{0pt}
\pagestyle{myheadings}
\newfontfamily\devanagarifont[Script=Devanagari]{AdishilaVedic}


\newcommand{\VAR}[1]{}
\newcommand{\BLOCK}[1]{}




\begin{document}
\begin{titlepage}
    \begin{center}
 
\begin{sanskrit}
    { \Large
    ॐ नमः परमात्मने, श्री महागणपतये नमः, श्री गुरुभ्यो नमः
ह॒रिः॒ ॐ
========================================= 
    }
    \\
    \vspace{2.5cm}
    \mbox{ \Huge
    3.5      तृतीयकाण्डे पञ्चमः प्रश्नः - इष्टिशेषाभिधानं   }
\end{sanskrit}
\end{center}

\end{titlepage}
\tableofcontents

ॐ नमः परमात्मने, श्री महागणपतये नमः, श्री गुरुभ्यो नमः
ह॒रिः॒ ॐ
========================================= \newline
3.5      तृतीयकाण्डे पञ्चमः प्रश्नः - इष्टिशेषाभिधानं \newline

\addcontentsline{toc}{section}{ 3.5      तृतीयकाण्डे पञ्चमः प्रश्नः - इष्टिशेषाभिधानं}
\markright{ 3.5      तृतीयकाण्डे पञ्चमः प्रश्नः - इष्टिशेषाभिधानं \hfill https://www.vedavms.in \hfill}
\section*{ 3.5      तृतीयकाण्डे पञ्चमः प्रश्नः - इष्टिशेषाभिधानं }
                                \textbf{ TS 3.5.1.1} \newline
                  पू॒र्णा । प॒श्चात् । उ॒त । पू॒र्णा । पु॒रस्ता᳚त् । उदिति॑ । म॒द्ध्य॒तः । पौ॒र्ण॒मा॒सीति॑ पौर्ण - मा॒सी । जि॒गा॒य॒ ॥ तस्या᳚म् । दे॒वाः । अधीति॑ । सं॒ॅवस॑न्त॒ इति॑ सं - वस॑न्तः । उ॒त्त॒म इत्यु॑त् - त॒मे । नाके᳚ । इ॒ह । मा॒द॒य॒न्ता॒म् ॥ यत् । ते॒ । दे॒वाः । अद॑धुः । भा॒ग॒धेय॒मिति॑ भाग - धेय᳚म् । अमा॑वास्य॒ इत्यमा᳚ - वा॒स्ये॒ । सं॒ॅवस॑न्त॒ इति॑ सं - वस॑न्तः । म॒हि॒त्वेति॑ महि-त्वा ॥ सा । नः॒ । य॒ज्ञ्म् । पि॒पृ॒हि॒ । वि॒श्व॒वा॒र॒ इति॑ विश्व - वा॒रे॒ । र॒यिम् । नः॒ । धे॒हि॒ । सु॒भ॒ग॒ इति॑ सु - भ॒गे॒ । सु॒वीर॒मिति॑ सु - वीर᳚म् ॥ नि॒वेश॒नीति॑ नि - वेश॑नी । स॒गंम॒नीति॑ सं - गम॑नी । वसू॑नाम् । विश्वा᳚ । रू॒पाणि॑ । वसू॑नि । आ॒वे॒शय॒न्तीत्या᳚ - वे॒शय॑न्ती ॥ स॒ह॒स्र॒पो॒षमिति॑ सहस्र - पो॒षम् । सु॒भगेति॑ सु-भगा᳚ । ररा॑णा । सा । नः॒ । एति॑ । ग॒न्न् । वर्च॑सा । \textbf{  1} \newline
                  \newline
                                \textbf{ TS 3.5.1.2} \newline
                  सं॒ॅवि॒दा॒नेति॑ सं - वि॒दा॒ना ॥ अग्नी॑षोमा॒वित्यग्नी᳚ - सो॒मौ॒ । प्र॒थ॒मौ । वी॒र्ये॑ण । वसून्॑ । रु॒द्रान् । आ॒दि॒त्यान् । इ॒ह । जि॒न्व॒त॒म् ॥ मा॒द्ध्यम् । हि । पौ॒र्ण॒मा॒समिति॑ पौर्ण - मा॒सम् । जु॒षेथा᳚म् । ब्रह्म॑णा । वृ॒द्धौ । सु॒कृ॒तेनेति॑ सु - कृ॒तेन॑ । सा॒तौ । अथ॑ । अ॒स्मभ्य॒मित्य॒स्म-भ्य॒म् । स॒हवी॑रा॒मिति॑ स॒ह-वी॒रा॒म् । र॒यिम् । नीति॑ । य॒च्छ॒त॒म् ॥ आ॒दि॒त्याः । च॒ । अङ्गि॑रसः । च॒ । अ॒ग्नीन् । एति॑ । अ॒द॒ध॒त॒ । ते । द॒र्॒.श॒पू॒र्ण॒मा॒साविति॑ दर्.श - पू॒र्ण॒मा॒सौ । प्रेति॑ । ऐ॒फ्स॒न्न् । तेषा᳚म् । अङ्गि॑रसाम् । निरु॑प्त॒मिति॒ निः - उ॒प्त॒म् । ह॒विः । आसी᳚त् । अथ॑ । आ॒दि॒त्याः । ए॒तौ । होमौ᳚ । अ॒प॒श्य॒न्न् । तौ । अ॒जु॒ह॒वुः॒ । ततः॑ । वै । ते । द॒र्॒.श॒पू॒र्ण॒मा॒साविति॑ दर्.श - पू॒र्ण॒मा॒सौ । \textbf{  2} \newline
                  \newline
                                \textbf{ TS 3.5.1.3} \newline
                  पूर्वे᳚ । एति॑ । अ॒ल॒भ॒न्त॒ । द॒र्॒.श॒पू॒र्ण॒मा॒साविति॑ दर्.श - पू॒र्ण॒मा॒सौ । आ॒लभ॑मान॒ इत्या᳚-लभ॑मानः । ए॒तौ । होमौ᳚ । पु॒रस्ता᳚त् । जु॒हु॒या॒त् । सा॒क्षादिति॑ स - अ॒क्षात् । ए॒व । द॒र्॒.श॒पू॒र्ण॒मा॒साविति॑ दर्.श - पू॒र्ण॒मा॒सौ । एति॑ । ल॒भे॒ते॒ । ब्र॒ह्म॒वा॒दिन॒ इति॑ ब्रह्म - वा॒दिनः॑ । व॒द॒न्ति॒ । सः । तु । वै । द॒र्॒.श॒पू॒र्ण॒मा॒साविति॑ दर्.श - पू॒र्ण॒मा॒सौ । एति॑ । ल॒भे॒त॒ । यः । ए॒न॒योः॒ । अ॒नु॒लो॒ममित्य॑नु - लो॒मम् । च॒ । प्र॒ति॒लो॒ममिति॑ प्रति - लो॒मम् । च॒ । वि॒द्यात् । इति॑ । अ॒मा॒वा॒स्या॑या॒ इत्य॑मा - वा॒स्या॑याः । ऊ॒र्द्ध्वम् । तत् । अ॒नु॒लो॒ममित्य॑नु - लो॒मम् । पौ॒र्ण॒मा॒स्या इति॑ पौर्ण - मा॒स्यै । प्र॒ती॒चीन᳚म् । तत् । प्र॒ति॒लो॒ममिति॑ प्रति - लो॒मम् । यत् । पौ॒र्ण॒मा॒सीमिति॑ पौर्ण - मा॒सीम् । पूर्वा᳚म् । आ॒लभे॒तेत्या᳚ - लभे॑त । प्र॒ति॒लो॒ममिति॑ प्रति-लो॒मम् । ए॒नौ॒ । एति॑ । ल॒भे॒त॒ । अ॒मुम् । अ॒प॒क्षीय॑माण॒मित्य॑प - क्षीय॑माणम् । अनु॑ । अपेति॑ । \textbf{  3} \newline
                  \newline
                                \textbf{ TS 3.5.1.4} \newline
                  क्षी॒ये॒ते॒ । सा॒र॒स्व॒तौ । होमौ᳚ । पु॒रस्ता᳚त् । जु॒हु॒या॒त् । अ॒मा॒वा॒स्येत्य॑मा-वा॒स्या᳚ । वै । सर॑स्वती । अ॒नु॒लो॒ममित्य॑नु-लो॒मम् । ए॒व । ए॒नौ॒ । एति॑ । ल॒भे॒ते॒ । अ॒मुम् । आ॒प्याय॑मान॒मित्या᳚ - प्याय॑मानम् । अनु॑ । एति॑ । प्या॒य॒ते॒ । आ॒ग्ना॒वै॒ष्ण॒वमित्या᳚ग्ना - वै॒ष्ण॒वम् । एका॑दशकपाल॒मित्येका॑दश-क॒पा॒ल॒म् । पु॒रस्ता᳚त् । निरिति॑ । व॒पे॒त् । सर॑स्वत्यै । च॒रुम् । सर॑स्वते । द्वाद॑शकपाल॒मिति॒ द्वाद॑श-क॒पा॒ल॒म् । यत् । आ॒ग्ने॒यः । भव॑ति । अ॒ग्निः । वै । य॒ज्ञ्॒मु॒खमिति॑ यज्ञ्-मु॒खम् । य॒ज्ञ्॒मु॒खमिति॑ यज्ञ्-मु॒खम् । ए॒व । ऋद्धि᳚म् । पु॒रस्ता᳚त् । ध॒त्ते॒ । यत् । वै॒ष्ण॒वः । भव॑ति । य॒ज्ञ्ः । वै । विष्णुः॑ । य॒ज्ञ्म् । ए॒व । आ॒रभ्येत्या᳚ - रभ्य॑ । प्रेति॑ । त॒नु॒ते॒ । सर॑स्वत्यै ( ) । च॒रुः । भ॒व॒ति॒ । सर॑स्वते । द्वाद॑शकपाल॒ इति॒ द्वाद॑श-क॒पा॒लः॒ । अ॒मा॒वा॒स्येत्य॑मा - वा॒स्या᳚ । वै । सर॑स्वती । पू॒र्णमा॑स॒ इति॑ पू॒र्ण - मा॒सः॒ । सर॑स्वान् । तौ । ए॒व । सा॒क्षादिति॑ स - अ॒क्षात् । एति॑ । र॒भ॒ते॒ । ऋ॒द्ध्नोति॑ । आ॒भ्या॒म् । द्वाद॑शकपाल॒ इति॒ द्वाद॑श - क॒पा॒लः॒ । सर॑स्वते । भ॒व॒ति॒ । मि॒थु॒न॒त्वायेति॑ मिथुन - त्वाय॑ । प्रजा᳚त्या॒ इति॒ प्र - जा॒त्यै॒ । मि॒थु॒नौ । गावौ᳚ । दक्षि॑णा । समृ॑द्ध्या॒ इति॒ सं - ऋ॒द्ध्यै॒ ॥ \textbf{  4} \newline
                  \newline
                      (वर्च॑सा॒ - वै ते द॑र्.शपूर्णमा॒सा - वप॑ - तनुते॒ सर॑स्वत्यै॒ - पञ्च॑विꣳशतिश्च)  \textbf{(A1)} \newline \newline
                                \textbf{ TS 3.5.2.1} \newline
                  ऋष॑यः । वै । इन्द्र᳚म् । प्र॒त्यक्ष॒मिति॑ प्रति - अक्ष᳚म् । न । अ॒प॒श्य॒न्न् । तम् । वसि॑ष्ठः । प्र॒त्यक्ष॒मिति॑ प्रति - अक्ष᳚म् । अ॒प॒श्य॒त् । सः । अ॒ब्र॒वी॒त् । ब्राह्म॑णम् । ते॒ । व॒क्ष्या॒मि॒ । यथा᳚ । त्वत्पु॑रोहिता॒ इति॒ त्वत् - पु॒रो॒हि॒ताः॒ । प्र॒जा इति॑ प्र - जाः । प्र॒ज॒नि॒ष्यन्त॒ इति॑ प्र - ज॒नि॒ष्यन्ते᳚ । अथ॑ । मा॒ । इत॑रेभ्यः । ऋषि॑भ्य॒ इत्यृषि॑ - भ्यः॒ । मा । प्रेति॑ । वो॒चः॒ । इति॑ । तस्मै᳚ । ए॒तान् । स्तोम॑भागा॒निति॒ स्तोम॑ - भा॒गा॒न् । अ॒ब्र॒वी॒त् । ततः॑ । वसि॑ष्ठपुरोहिता॒ इति॒ वसि॑ष्ठ - पु॒रो॒हि॒ताः॒ । प्र॒जा इति॑ प्र - जाः । प्रेति॑ । अ॒जा॒य॒न्त॒ । तस्मा᳚त् । वा॒सि॒ष्ठः । ब्र॒ह्मा । का॒र्यः॑ । प्रेति॑ । ए॒व । जा॒य॒ते॒ । र॒श्मिः । अ॒सि॒ । क्षया॑य । त्वा॒ । क्षय᳚म् । जि॒न्व॒ । इति॑ । \textbf{  5} \newline
                  \newline
                                \textbf{ TS 3.5.2.2} \newline
                  आ॒ह॒ । दे॒वाः । वै । क्षयः॑ । दे॒वेभ्यः॑ । ए॒व । य॒ज्ञ्म् । प्रेति॑ । आ॒ह॒ । प्रेति॒रिति॒ प्र - इ॒तिः॒ । अ॒सि॒ । धर्मा॑य । त्वा॒ । धर्म᳚म् । जि॒न्व॒ । इति॑ । आ॒ह॒ । म॒नु॒ष्याः᳚ । वै । धर्मः॑ । म॒नु॒ष्ये᳚भ्यः । ए॒व । य॒ज्ञ्म् । प्रेति॑ । आ॒ह॒ । अन्वि॑ति॒रित्यनु॑-इ॒तिः॒ । अ॒सि॒ । दि॒वे । त्वा॒ । दिव᳚म् । जि॒न्व॒ । इति॑ । आ॒ह॒ । ए॒भ्यः । ए॒व । लो॒केभ्यः॑ । य॒ज्ञ्म् । प्रेति॑ । आ॒ह॒ । वि॒ष्ट॒भं इति॑ वि - स्त॒भंः । अ॒सि॒ । वृष्ट्यै᳚ । त्वा॒ । वृष्टि᳚म् । जि॒न्व॒ । इति॑ । आ॒ह॒ । वृष्टि᳚म् । ए॒व । अवेति॑ । \textbf{  6} \newline
                  \newline
                                \textbf{ TS 3.5.2.3} \newline
                  रु॒न्धे॒ । प्र॒वेति॑ प्र - वा । अ॒सि॒ । अ॒नु॒वेत्य॑नु - वा । अ॒सि॒ । इति॑ । आ॒ह॒ । मि॒थु॒न॒त्वायेति॑ मिथुन - त्वाय॑ । उ॒शिक् । अ॒सि॒ । वसु॑भ्य॒ इति॒ वसु॑ - भ्यः॒ । त्वा॒ । वसून्॑ । जि॒न्व॒ । इति॑ । आ॒ह॒ । अ॒ष्टौ । वस॑वः । एका॑दश । रु॒द्राः । द्वाद॑श । आ॒दि॒त्याः । ए॒ताव॑न्तः । वै । दे॒वाः । तेभ्यः॑ । ए॒व । य॒ज्ञ्म् । प्रेति॑ । आ॒ह॒ । ओजः॑ । अ॒सि॒ । पि॒तृभ्य॒ इति॑ पि॒तृ - भ्यः॒ । त्वा॒ । पि॒तॄन् । जि॒न्व॒ । इति॑ । आ॒ह॒ । दे॒वान् । ए॒व । पि॒तॄन् । अनु॑ । समिति॑ । त॒नो॒ति॒ । तन्तुः॑ । अ॒सि॒ । प्र॒जाभ्य॒ इति॑ प्र - जाभ्यः॑ । त्वा॒ । प्र॒जा इति॑ प्र - जाः । जि॒न्व॒ । \textbf{  7} \newline
                  \newline
                                \textbf{ TS 3.5.2.4} \newline
                  इति॑ । आ॒ह॒ । पि॒तॄन् । ए॒व । प्र॒जा इति॑ प्र - जाः । अनु॑ । समिति॑ । त॒नो॒ति॒ । पृ॒त॒ना॒षाट् । अ॒सि॒ । प॒शुभ्य॒ इति॑ प॒शु - भ्यः॒ । त्वा॒ । प॒शून् । जि॒न्व॒ । इति॑ । आ॒ह॒ । प्र॒जा इति॑ प्र - जाः । ए॒व । प॒शून् । अनु॑ । समिति॑ । त॒नो॒ति॒ । रे॒वत् । अ॒सि॒ । ओष॑धीभ्य॒ इत्योष॑धि - भ्यः॒ । त्वा॒ । ओष॑धीः । जि॒न्व॒ । इति॑ । आ॒ह॒ । ओष॑धीषु । ए॒व । प॒शून् । प्रतीति॑ । स्था॒प॒य॒ति॒ । अ॒भि॒जिदित्य॑भि - जित् । अ॒सि॒ । यु॒क्तग्रा॒वेति॑ यु॒क्त - ग्रा॒वा॒ । इन्द्रा॑य । त्वा॒ । इन्द्र᳚म् । जि॒न्व॒ । इति॑ । आ॒ह॒ । अ॒भिजि॑त्या॒ इत्य॒भि - जि॒त्यै॒ । अधि॑पति॒रित्यधि॑ - प॒तिः॒ । अ॒सि॒ । प्रा॒णायेति॑ प्र - अ॒नाय॑ । त्वा॒ । प्रा॒णमिति॑ प्र-अ॒नम् । \textbf{  8} \newline
                  \newline
                                \textbf{ TS 3.5.2.5} \newline
                  जि॒न्व॒ । इति॑ । आ॒ह॒ । प्र॒जास्विति॑ प्र - जासु॑ । ए॒व । प्रा॒णानिति॑ प्र - अ॒नान् । द॒धा॒ति॒ । त्रि॒वृदिति॑ त्रि - वृत् । अ॒सि॒ । प्र॒वृदिति॑ प्र - वृत् । अ॒सि॒ । इति॑ । आ॒ह॒ । मि॒थु॒न॒त्वायेति॑ मिथुन - त्वाय॑ । सꣳ॒॒रो॒ह इति॑ सं - रो॒हः । अ॒सि॒ । नी॒रो॒ह इति॑ निः - रो॒हः । अ॒सि॒ । इति॑ । आ॒ह॒ । प्रजा᳚त्या॒ इति॒ प्र - जा॒त्यै॒ । व॒सु॒कः । अ॒सि॒ । वेष॑श्रि॒रिति॒ वेष॑- श्रिः॒ । अ॒सि॒ । वस्य॑ष्टिः । अ॒सि॒ । इति॑ । आ॒ह॒ । प्रति॑ष्ठित्या॒ इति॒ प्रति॑ - स्थि॒त्यै॒ ॥ \textbf{  9} \newline
                  \newline
                      (जि॒न्वेत्य - व॑ - प्र॒जा जि॑न्व - प्रा॒णन् - त्रिꣳ॒॒शच्च॑) \textbf{(A2)} \newline \newline
                                \textbf{ TS 3.5.3.1} \newline
                  अ॒ग्निना᳚ । दे॒वेन॑ । पृत॑नाः । ज॒या॒मि॒ । गा॒य॒त्रेण॑ । छन्द॑सा । त्रि॒वृतेति॑ त्रि - वृता᳚ । स्तोमे॑न । र॒थ॒न्त॒रेणेति॑ रथं - त॒रेण॑ । साम्ना᳚ । व॒ष॒ट्का॒रेणेति॑ वषट् - का॒रेण॑ । वज्रे॑ण । पू॒र्व॒जानिति॑ पूर्व - जान् । भ्रातृ॑व्यान् । अध॑रान् । पा॒द॒या॒मि॒ । अवेति॑ । ए॒ना॒न् । बा॒धे॒ । प्रतीति॑ । ए॒ना॒न् । नु॒दे॒ । अ॒स्मिन्न् । क्षये᳚ । अ॒स्मिन्न् । भू॒मि॒लो॒क इति॑ भूमि - लो॒के । यः । अ॒स्मान् । द्वेष्टि॑ । यम् । च॒ । व॒यम् । द्वि॒ष्मः । विष्णोः᳚ । क्रमे॑ण । अतीति॑ । ए॒ना॒न् । क्रा॒मा॒मि॒ । इन्द्रे॑ण । दे॒वेन॑ । पृत॑नाः । ज॒या॒मि॒ । त्रैष्टु॑भेन । छन्द॑सा । प॒ञ्च॒द॒शेनेति॑ पञ्च - द॒शेन॑ । स्तोमे॑न । बृ॒ह॒ता । साम्ना᳚ । व॒ष॒ट्का॒रेणेति॑ वषट् - का॒रेण॑ । वज्रे॑ण । \textbf{  10} \newline
                  \newline
                                \textbf{ TS 3.5.3.2} \newline
                  स॒ह॒जानिति॑ सह - जान् । विश्वे॑भिः । दे॒वेभिः॑ । पृत॑नाः । ज॒या॒मि॒ । जाग॑तेन । छन्द॑सा । स॒प्त॒द॒शेनेति॑ सप्त - द॒शेन॑ । स्तोमे॑न । वा॒म॒दे॒व्येनेति॑ वाम-दे॒व्येन॑ । साम्ना᳚ । व॒ष॒ट्का॒रेणेति॑ वषट् - का॒रेण॑ । वज्रे॑ण । अ॒प॒र॒जानित्य॑पर - जान् । इन्द्रे॑ण । स॒युज॒ इति॑ स-युजः॑ । व॒यम् । सा॒स॒ह्याम॑ । पृ॒त॒न्य॒तः ॥ घ्नन्तः॑ । वृ॒त्राणि॑ । अ॒प्र॒ति ॥ यत् । ते॒ । अ॒ग्ने॒ । तेजः॑ । तेन॑ । अ॒हम् । ते॒ज॒स्वी । भू॒या॒स॒म् । यत् । ते॒ । अ॒ग्ने॒ । वर्चः॑ । तेन॑ । अ॒हम् । व॒र्च॒स्वी । भू॒या॒स॒म् । यत् । ते॒ । अ॒ग्ने॒ । हरः॑ । तेन॑ । अ॒हम् । ह॒र॒स्वी । भू॒या॒स॒म् ॥ \textbf{  11} \newline
                  \newline
                      (बृ॒ह॒ता साम्ना॑ वषट्का॒रेण॒ वज्रे॑ण॒ - षट्च॑त्वारिꣳशच्च)  \textbf{(A3)} \newline \newline
                                \textbf{ TS 3.5.4.1} \newline
                  ये । दे॒वाः । य॒ज्ञ्॒हन॒ इति॑ यज्ञ् - हनः॑ । य॒ज्ञ्॒मुष॒ इति॑ यज्ञ् - मुषः॑ । पृ॒थि॒व्याम् । अधीति॑ । आस॑ते ॥ अ॒ग्निः । मा॒ । तेभ्यः॑ । र॒क्ष॒तु॒ । गच्छे॑म । सु॒कृत॒ इति॑ सु - कृतः॑ । व॒यम् ॥ एति॑ । अ॒ग॒न्म॒ । मि॒त्रा॒व॒रु॒णेति॑ मित्रा-व॒रु॒णा॒ । व॒रे॒ण्या॒ । रात्री॑णाम् । भा॒गः । यु॒वयोः᳚ । यः । अस्ति॑ ॥ नाक᳚म् । गृ॒ह्णा॒नाः । सु॒कृ॒तस्येति॑ सु - कृ॒तस्य॑ । लो॒के । तृ॒तीये᳚ । पृ॒ष्ठे । अधीति॑ । रो॒च॒ने । दि॒वः ॥ ये । दे॒वाः । य॒ज्ञ्॒हन॒ इति॑ यज्ञ् - हनः॑ । य॒ज्ञ्॒मुष॒ इति॑ यज्ञ् - मुषः॑ । अ॒न्तरि॑क्षे । अधीति॑ । आस॑ते ॥ वा॒युः । मा॒ । तेभ्यः॑ । र॒क्ष॒तु॒ । गच्छे॑म । सु॒कृत॒ इति॑ सु - कृतः॑ । व॒यम् ॥ याः । ते॒ । रात्रीः᳚ । स॒वि॒तः॒ । \textbf{  12} \newline
                  \newline
                                \textbf{ TS 3.5.4.2} \newline
                  दे॒व॒यानी॒रिति॑ देव - यानीः᳚ । अ॒न्त॒रा । द्यावा॑पृथि॒वी इति॒ द्यावा᳚ - पृ॒थि॒वी । वि॒यन्तीति॑ वि - यन्ति॑ ॥ गृ॒हैः । च॒ । सर्वैः᳚ । प्र॒जयेति॑ प्र - जया᳚ । नु । अग्रे᳚ । सुवः॑ । रुहा॑णाः । त॒र॒त॒ । रजाꣳ॑सि ॥ ये । दे॒वाः । य॒ज्ञ्॒हन॒ इति॑ यज्ञ् - हनः॑ । य॒ज्ञ्॒मुष॒ इति॑ यज्ञ् - मुषः॑ । दि॒वि । अधीति॑ । आस॑ते ॥ सूर्यः॑ । मा॒ । तेभ्यः॑ । र॒क्ष॒तु॒ । गच्छे॑म । सु॒कृत॒ इति॑ सु - कृतः॑ । व॒यम् ॥ येन॑ । इन्द्रा॑य । स॒मभ॑र॒ इति॑ सं - अभ॑रः । पयाꣳ॑सि । उ॒त्त॒मेनेत्यु॑त् - त॒मेन॑ । ह॒विषा᳚ । जा॒त॒वे॒द॒ इति॑ जात - वे॒दः॒ ॥ तेन॑ । अ॒ग्ने॒ । त्वम् । उ॒त । व॒र्द्ध॒य॒ । इ॒मम् । स॒जा॒ताना॒मिति॑ स - जा॒ताना᳚म् । श्रैष्ठ्ये᳚ । एति॑ । धे॒हि॒ । ए॒न॒म् ॥ य॒ज्ञ्॒हन॒ इति॑ यज्ञ् - हनः॑ । वै । दे॒वाः । य॒ज्ञ्॒मुष॒ इति॑ यज्ञ् - मुषः॑ । \textbf{  13} \newline
                  \newline
                                \textbf{ TS 3.5.4.3} \newline
                  स॒न्ति॒ । ते । ए॒षु । लो॒केषु॑ । आ॒स॒ते॒ । आ॒ददा॑ना॒ इत्या᳚ - ददा॑नाः । वि॒म॒थ्ना॒ना इति॑ वि-म॒थ्ना॒नाः । यः । ददा॑ति । यः । यज॑ते । तस्य॑ ॥ ये । दे॒वाः । य॒ज्ञ्॒हन॒ इति॑ यज्ञ् - हनः॑ । पृ॒थि॒व्याम् । अधीति॑ । आस॑ते । ये । अ॒न्तरि॑क्षे । ये । दि॒वि । इति॑ । आ॒ह॒ । इ॒मान् । ए॒व । लो॒कान् । ती॒र्त्वा । सगृ॑ह॒ इति॒ स - गृ॒हः॒ । सप॑शु॒रिति॒ स - प॒शुः॒ । सु॒व॒र्गमिति॑ सुवः - गम् । लो॒कम् । ए॒ति॒ । अपेति॑ । वै । सोमे॑न । ई॒जा॒नात् । दे॒वताः᳚ । च॒ । य॒ज्ञ्ः । च॒ । क्रा॒म॒न्ति॒ । आ॒ग्ने॒यम् । पञ्च॑कपाल॒मिति॒ पञ्च॑ - क॒पा॒ल॒म् । उ॒द॒व॒सा॒नीय॒मित्यु॑त् - अ॒व॒सा॒नीय᳚म् । निरिति॑ । व॒पे॒त् । अ॒ग्निः । सर्वाः᳚ । दे॒वताः᳚ । \textbf{  14} \newline
                  \newline
                                \textbf{ TS 3.5.4.4} \newline
                  पाङ्क्तः॑ । य॒ज्ञ्ः । दे॒वताः᳚ । च॒ । ए॒व । य॒ज्ञ्म् । च॒ । अवेति॑ । रु॒न्धे॒ । गा॒य॒त्रः । वै । अ॒ग्निः । गा॒य॒त्रछ॑न्दा॒ इति॑ गाय॒त्र - छ॒न्दाः॒ । तम् । छन्द॑सा । वीति॑ । अ॒र्द्ध॒य॒ति॒ । यत् । पञ्च॑कपाल॒मिति॒ पञ्च॑ - क॒पा॒ल॒म् । क॒रोति॑ । अ॒ष्टाक॑पाल॒ इत्य॒ष्टा - क॒पा॒लः॒ । का॒र्यः॑ । अ॒ष्टाक्ष॒रेत्य॒ष्टा - अ॒क्ष॒रा॒ । गा॒य॒त्री । गा॒य॒त्रः । अ॒ग्निः । गा॒य॒त्रछ॑न्दा॒ इति॑ गाय॒त्र - छ॒न्दाः॒ । स्वेन॑ । ए॒व । ए॒न॒म् । छन्द॑सा । समिति॑ । अ॒र्द्ध॒य॒ति॒ । प॒ङ्क्त्यौ᳚ । या॒ज्या॒नु॒वा॒क्ये॑ इति॑ याज्या - अ॒नु॒वा॒क्ये᳚ । भ॒व॒तः॒ । पाङ्क्तः॑ । य॒ज्ञ्ः । तेन॑ । ए॒व । य॒ज्ञात् । न । ए॒ति॒ ॥ \textbf{  15 } \newline
                  \newline
                      (स॒वि॒त॒-र्दे॒वा य॑ज्ञ्॒मुषः॒ - सर्वा॑ दे॒वता॒ - स्त्रिच॑त्वारिꣳशच्च)  \textbf{(A4)} \newline \newline
                                \textbf{ TS 3.5.5.1} \newline
                  सूर्यः॑ । मा॒ । दे॒वः । दे॒वेभ्यः॑ । पा॒तु॒ । वा॒युः । अ॒न्तरि॑क्षात् । यज॑मानः । अ॒ग्निः । मा॒ । पा॒तु॒ । चक्षु॑षः ॥ सक्ष॑ । शूष॑ । सवि॑तः । विश्व॑चर्.षण॒ इति॒ विश्व॑ - च॒र्॒.ष॒णे॒ । ए॒तेभिः॑ । सो॒म॒ । नाम॑भि॒रिति॒ नाम॑ - भिः॒ । वि॒धे॒म॒ । ते॒ । तेभिः॑ । सो॒म॒ । नाम॑भि॒रिति॒ नाम॑-भिः॒ । वि॒धे॒म॒ । ते॒ ॥ अ॒हम् । प॒रस्ता᳚त् । अ॒हम् । अ॒वस्ता᳚त् । अ॒हम् । ज्योति॑षा । वीति॑ । तमः॑ । व॒वा॒र॒ ॥ यत् । अ॒न्तरि॑क्षम् । तत् । उ॒ । मे॒ । पि॒ता । अ॒भू॒त् । अ॒हम् । सूर्य᳚म् । उ॒भ॒यतः॑ । द॒द॒र्॒.श॒ । अ॒हम् । भू॒या॒स॒म् । उ॒त्त॒म इत्यु॑त् - त॒मः । स॒मा॒नाना᳚म् । \textbf{  16} \newline
                  \newline
                                \textbf{ TS 3.5.5.2} \newline
                  एति॑ । स॒मु॒द्रात् । एति॑ । अ॒न्तरि॑क्षात् । प्र॒जाप॑ति॒रिति॑ प्र॒जा - प॒तिः॒ । उ॒द॒धिमित्यु॑द - धिम् । च्या॒व॒या॒ति॒ । इन्द्रः॑ । प्रेति॑ । स्नौ॒तु॒ । म॒रुतः॑ । व॒र्॒.ष॒य॒न्तु॒ । उदिति॑ । न॒भं॒य॒ । पृ॒थि॒वीम् । भि॒न्धि । इ॒दम् । दि॒व्यम् । नभः॑ ॥ उ॒द्रः । दि॒व्यस्य॑ । नः॒ । दे॒हि॒ । ईशा॑नः । वीति॑ । सृ॒ज॒ । दृति᳚म् ॥ प॒शवः॑ । वै । ए॒ते । यत् । आ॒दि॒त्यः । ए॒षः । रु॒द्रः । यत् । अ॒ग्निः । ओष॑धीः । प्रास्येति॑ प्र - अस्य॑ । अ॒ग्नौ । आ॒दि॒त्यम् । जु॒हो॒ति॒ । रु॒द्रात् । ए॒व । प॒शून् । अ॒न्तः । द॒धा॒ति॒ । अथो॒ इति॑ । ओष॑धीषु । ए॒व । प॒शून् । \textbf{  17} \newline
                  \newline
                                \textbf{ TS 3.5.5.3} \newline
                  प्रतीति॑ । स्था॒प॒य॒ति॒ । क॒विः । य॒ज्ञ्स्य॑ । वीति॑ । त॒नो॒ति॒ । पन्था᳚म् । नाक॑स्य । पृ॒ष्ठे । अधीति॑ । रो॒च॒ने । दि॒वः ॥ येन॑ । ह॒व्यम् । वह॑सि । यासि॑ । दू॒तः । इ॒तः । प्रचे॑ता॒ इति॒ प्र-चे॒ताः॒ । अ॒मुतः॑ । सनी॑यान् ॥ याः । ते॒ । विश्वाः᳚ । स॒मिध॒ इति॑ सं - इधः॑ । सन्ति॑ । अ॒ग्ने॒ । याः । पृ॒थि॒व्याम् । ब॒र्॒.हिषि॑ । सूर्ये᳚ । याः ॥ ताः । ते॒ । ग॒च्छ॒न्तु॒ । आहु॑ति॒मित्या - हु॒ति॒म् । घृ॒तस्य॑ । दे॒वा॒य॒त इति॑ देव - य॒ते । यज॑मानाय । शर्म॑ ॥ आ॒शासा॑न॒ इत्या᳚ - शासा॑नः । सु॒वीर्य॒मिति॑ सु - वीर्य᳚म् । रा॒यः । पोष᳚म् । स्वश्वि॑य॒मिति॑ सु - अश्वि॑यम् ॥ बृह॒स्पति॑ना । रा॒या । स्व॒गाकृ॑त॒ इति॑ स्व॒गा - कृ॒तः॒ । मह्य᳚म् । यज॑मानाय ( ) । ति॒ष्ठ॒ ॥ \textbf{  18} \newline
                  \newline
                      (स॒मा॒नाना॒-मोष॑धीष्वे॒व प॒शून् - मह्यं॒ ॅयज॑माना॒ - यैक॑ञ्च)  \textbf{(A5)} \newline \newline
                                \textbf{ TS 3.5.6.1} \newline
                  समिति॑ । त्वा॒ । न॒ह्या॒मि॒ । पय॑सा । घृ॒तेन॑ । समिति॑ । त्वा॒ । न॒ह्या॒मि॒ । अ॒पः । ओष॑धीभि॒रित्योष॑धि - भिः॒ ॥ समिति॑ । त्वा॒ । न॒ह्या॒मि॒ । प्र॒जयेति॑ प्र - जया᳚ । अ॒हम् । अ॒द्य । सा । दी॒क्षि॒ता । स॒न॒वः॒ । वाज᳚म् । अ॒स्मे इति॑ ॥ प्रेति॑ । ए॒तु॒ । ब्रह्म॑णः । पत्नी᳚ । वेदि᳚म् । वर्णे॑न । सी॒द॒तु॒ ॥ अथ॑ । अ॒हम् । अ॒नु॒का॒मिनीत्य॑नु-का॒मिनी᳚ । स्वे । लो॒के । वि॒शै । इ॒ह ॥ सु॒प्र॒जस॒ इति॑ सु - प्र॒जसः॑ । त्वा॒ । व॒यम् । सु॒पत्नी॒रिति॑ सु - पत्नीः᳚ । उपेति॑ । से॒दि॒म॒ ॥ अग्ने᳚ । स॒प॒त्न॒दंभ॑न॒मिति॑ सपत्न - दंभ॑नम् । अद॑ब्धासः । अदा᳚भ्यम् ॥ इ॒मम् । वीति॑ । स्या॒मि॒ । वरु॑णस्य । पाश᳚म् । \textbf{  19} \newline
                  \newline
                                \textbf{ TS 3.5.6.2} \newline
                  यम् । अब॑द्ध्नीत । स॒वि॒ता । सु॒केत॒ इति॑ सु-केतः॑ ॥ धा॒तुः । च॒ । योनौ᳚ । सु॒कृ॒तस्येति॑ सु - कृ॒तस्य॑ । लो॒के । स्यो॒नम् । मे॒ । स॒ह । पत्या᳚ । क॒रो॒मि॒ ॥ प्रेति॑ । इ॒हि॒ । उ॒देहीत्यु॑त्-एहि॑ । ऋ॒तस्य॑ । वा॒मीः । अन्विति॑ । अ॒ग्निः । ते॒ । अग्र᳚म् । न॒य॒तु॒ । अदि॑तिः । मद्ध्य᳚म् । द॒द॒ता॒म् । रु॒द्राव॑सृ॒ष्टेति॑ रु॒द्र-अ॒व॒सृ॒ष्टा॒ । अ॒सि॒ । यु॒वा । नाम॑ । मा । मा॒ । हिꣳ॒॒सीः॒ । वसु॑भ्य॒ इति॒ वसु॑-भ्यः॒ । रु॒द्रेभ्यः॑ । आ॒दि॒त्येभ्यः॑ । विश्वे᳚भ्यः । वः॒ । दे॒वेभ्यः॑ । प॒न्नेज॑नी॒रिति॑ पत् - नेज॑नीः । गृ॒ह्णा॒मि॒ । य॒ज्ञाय॑ । वः॒ । प॒न्नेज॑नी॒रिति॑ पत् - नेज॑नीः । सा॒द॒या॒मि॒ । विश्व॑स्य । ते॒ । विश्वा॑वत॒ इति॒ विश्व॑-व॒तः॒ । वृष्णि॑यावत॒ इति॒ वृष्णि॑य-व॒तः॒ । \textbf{  20} \newline
                  \newline
                                \textbf{ TS 3.5.6.3} \newline
                  तव॑ । अ॒ग्ने॒ । वा॒मीः । अन्विति॑ । स॒दृंशीति॑ सं - दृशि॑ । विश्वा᳚ । रेताꣳ॑सि । धि॒षी॒य॒ । अगन्न्॑ । दे॒वान् । य॒ज्ञ्ः । नीति॑ । दे॒वीः । दे॒वेभ्यः॑ । य॒ज्ञ्म् । अ॒शि॒ष॒न्न् । अ॒स्मिन्न् । सु॒न्व॒ति । यज॑माने । आ॒शिष॒ इत्या᳚ - शिषः॑ । स्वाहा॑कृता॒ इति॒ स्वाहा᳚ - कृ॒ताः॒ । स॒मु॒द्रे॒ष्ठा इति॑ समुद्रे - स्थाः । ग॒न्ध॒र्वम् । एति॑ । ति॒ष्ठ॒त॒ । अनु॑ ॥ वात॑स्य । पत्मन्न्॑ । इ॒डः । ई॒डि॒ताः ॥ \textbf{  21 } \newline
                  \newline
                      (पाशं॒ - ॅवृष्णि॑यावत - स्त्रिꣳ॒॒शच्च॑)  \textbf{(A6)} \newline \newline
                                \textbf{ TS 3.5.7.1} \newline
                  व॒ष॒ट्का॒र इति॑ वषट् - का॒रः । वै । गा॒य॒त्रि॒यै । शिरः॑ । अ॒च्छि॒न॒त् । तस्यै᳚ । रसः॑ । परेति॑ । अ॒प॒त॒त् । सः । पृ॒थि॒वीम् । प्रेति॑ । अ॒वि॒श॒त् । सः । ख॒दि॒रः । अ॒भ॒व॒त् । यस्य॑ । खा॒दि॒रः । स्रु॒वः । भव॑ति । छन्द॑साम् । ए॒व । रसे॑न । अवेति॑ । द्य॒ति॒ । सर॑सा॒ इति॒ स - र॒साः॒ । अ॒स्य॒ । आहु॑तय॒ इत्या - हु॒त॒यः॒ । भ॒व॒न्ति॒ । तृ॒तीय॑स्याम् । इ॒तः । दि॒वि । सोमः॑ । आ॒सी॒त् । तम् । गा॒य॒त्री । एति॑ । अ॒ह॒र॒त् । तस्य॑ । प॒र्णम् । अ॒च्छि॒द्य॒त॒ । तत् । प॒र्णः । अ॒भ॒व॒त् । तत् । प॒र्णस्य॑ । प॒र्ण॒त्वमिति॑ पर्ण - त्वम् । यस्य॑ । प॒र्ण॒मयीति॑ पर्ण - मयी᳚ । जु॒हूः । \textbf{  22} \newline
                  \newline
                                \textbf{ TS 3.5.7.2} \newline
                  भव॑ति । सौ॒म्याः । अ॒स्य॒ । आहु॑तय॒ इत्या - हु॒त॒यः॒ । भ॒व॒न्ति॒ । जु॒षन्ते᳚ । अ॒स्य॒ । दे॒वाः । आहु॑ती॒रित्या - हु॒तीः॒ । दे॒वाः । वै । ब्रह्मन्न्॑ । अ॒व॒द॒न्त॒ । तत् । प॒र्णः । उपेति॑ । अ॒शृ॒णो॒त् । सु॒श्रवा॒ इति॑ सु - श्रवाः᳚ । वै । नाम॑ । यस्य॑ । प॒र्ण॒मयीति॑ पर्ण - मयी᳚ । जु॒हूः । भव॑ति । न । पा॒पम् । श्लोक᳚म् । शृ॒णो॒ति॒ । ब्रह्म॑ । वै । प॒र्णः । विट् । म॒रुतः॑ । अन्न᳚म् । विट् । मा॒रु॒तः । अ॒श्व॒त्थः । यस्य॑ । प॒र्ण॒मयीति॑ पर्ण-मयी᳚ । जु॒हूः । भव॑ति । आश्व॑त्थी । उ॒प॒भृतित्यु॑प - भृत् । ब्रह्म॑णा । ए॒व । अन्न᳚म् । अवेति॑ । रु॒न्धे॒ । अथो॒ इति॑ । ब्रह्म॑ । \textbf{  23} \newline
                  \newline
                                \textbf{ TS 3.5.7.3} \newline
                  ए॒व । वि॒शि । अधीति॑ । ऊ॒ह॒ति॒ । रा॒ष्ट्रम् । वै । प॒र्णः । विट् । अ॒श्व॒त्थः । यत् । प॒र्ण॒मयीति॑ पर्ण - मयी᳚ । जु॒हूः । भव॑ति । आश्व॑त्थी । उ॒प॒भृतित्यु॑प-भृत् । रा॒ष्ट्रम् । ए॒व । वि॒शि । अधीति॑ । ऊ॒ह॒ति॒ । प्र॒जाप॑ति॒रिति॑ प्र॒जा - प॒तिः॒ । वै । अ॒ज॒हो॒त् । सा । यत्र॑ । आहु॑ति॒रित्या - हु॒तः॒ । प्र॒त्यति॑ष्ठ॒दिति॑ प्रति - अति॑ष्ठत् । ततः॑ । विक॑ङ्कत॒ इति॒ वि - क॒ङ्क॒तः॒ । उदिति॑ । अ॒ति॒ष्ठ॒त् । ततः॑ । प्र॒जा इति॑ प्र - जाः । अ॒सृ॒ज॒त॒ । यस्य॑ । वैक॑ङ्कती । ध्रु॒वा । भव॑ति । प्रतीति॑ । ए॒व । अ॒स्य॒ । आहु॑तय॒ इत्या-हु॒त॒यः॒ । ति॒ष्ठ॒न्ति॒ । अथो॒ इति॑ । प्रेति॑ । ए॒व । जा॒य॒ते॒ । ए॒तत् । वै । स्रु॒चाम् ( ) । रू॒पम् । यस्य॑ । ए॒वꣳरू॑पा॒ इत्ये॒वं - रू॒पाः॒ । स्रुचः॑ । भव॑न्ति । सर्वा॑णि । ए॒व । ए॒न॒म् । रू॒पाणि॑ । प॒शू॒नाम् । उपेति॑ । ति॒ष्ठ॒न्ते॒ । न । अ॒स्य॒ । अप॑रूप॒मित्यप॑ - रू॒प॒म् । आ॒त्मन्न् । जा॒य॒ते॒ ॥ \textbf{  24} \newline
                  \newline
                      (जु॒हू - रथो॒ ब्रह्म॑ - स्रु॒चाꣳ - स॒प्तद॑श च)  \textbf{(A7)} \newline \newline
                                \textbf{ TS 3.5.8.1} \newline
                  उ॒प॒या॒मगृ॑हीत॒ इत्यु॑पया॒म - गृ॒ही॒तः॒ । अ॒सि॒ । प्र॒जाप॑तय॒ इति॑ प्र॒जा-प॒त॒ये॒ । त्वा॒ । ज्योति॑ष्मते । ज्योति॑ष्मन्तम् । गृ॒ह्णा॒मि॒ । दक्षा॑य । द॒क्ष॒वृध॒ इति॑ दक्ष - वृधे᳚ । रा॒तम् । दे॒वेभ्यः॑ । अ॒ग्नि॒जि॒ह्वेभ्य॒ इत्य॑ग्नि - जि॒ह्वेभ्यः॑ । त्वा॒ । ऋ॒ता॒युभ्य॒ इत्यृ॑ता॒यु - भ्यः॒ । इन्द्र॑ज्येष्ठेभ्य॒ इतीन्द्र॑-ज्ये॒ष्ठे॒भ्यः॒ । वरु॑णराजभ्य॒ इति॒ वरु॑णराज-भ्यः॒ । वाता॑पिभ्य॒ इति॒ वाता॑पि-भ्यः॒ । प॒र्जन्या᳚त्मभ्य॒ इति॑ प॒र्जन्या᳚त्म-भ्यः॒ । दि॒वे । त्वा॒ । अ॒न्तरि॑क्षाय । त्वा॒ । पृ॒थि॒व्यै । त्वा॒ । अपेति॑ । इ॒न्द्र॒ । द्वि॒ष॒तः । मनः॑ । अपेति॑ । जिज्या॑सतः । ज॒हि॒ । अपेति॑ । यः । नः॒ । अ॒रा॒ती॒यति॑ । तम् । ज॒हि॒ । प्रा॒णायेति॑ प्रा - अ॒नाय॑ । त्वा॒ । अ॒पा॒नायेत्य॑प-अ॒नाय॑ । त्वा॒ । व्या॒नायेति॑ वि - अ॒नाय॑ । त्वा॒ । स॒ते । त्वा॒ । अस॑ते । त्वा॒ । अ॒द्भ्य इत्य॑त् - भ्यः । त्वा॒ । ओष॑धीभ्य॒ इत्योष॑धि-भ्यः॒ ( ) । विश्वे᳚भ्यः । त्वा॒ । भू॒तेभ्यः॑ । यतः॑ । प्र॒जा इति॑ प्र - जाः । अक्खि॑द्राः । अजा॑यन्त । तस्मै᳚ । त्वा॒ । प्र॒जाप॑तय॒ इति॑ प्र॒जा - प॒त॒ये॒ । वि॒भू॒दाव्.न्न॒ इति॑ विभु - दाव्.न्ने᳚ । ज्योति॑ष्मते । ज्योति॑ष्मन्तम् । जु॒हो॒मि॒ ॥ \textbf{  25 } \newline
                  \newline
                      (ओष॑धीभ्य॒ - श्चतु॑र्दश च)  \textbf{(A8)} \newline \newline
                                \textbf{ TS 3.5.9.1} \newline
                  याम् । वै । अ॒द्ध्व॒र्युः । च॒ । यज॑मानः । च॒ । दे॒वता᳚म् । अ॒न्त॒रि॒त इत्य॑न्तः - इ॒तः । तस्यै᳚ । एति॑ । वृ॒श्च्ये॒ते॒ इति॑ । प्रा॒जा॒प॒त्यमिति॑ प्राजा - प॒त्यम् । द॒धि॒ग्र॒हमिति॑ दधि - ग्र॒हम् । गृ॒ह्णी॒या॒त् । प्र॒जाप॑ति॒रिति॑ प्र॒जा - प॒तिः॒ । सर्वाः᳚ । दे॒वताः᳚ । दे॒वता᳚भ्यः । ए॒व । नीति॑ । ह्नु॒वा॒ते॒ इति॑ । ज्ये॒ष्ठः । वै । ए॒षः । ग्रहा॑णाम् । यस्य॑ । ए॒षः । गृ॒ह्यते᳚ । ज्यैष्ट्य᳚म् । ए॒व । ग॒च्छ॒ति॒ । सर्वा॑साम् । वै । ए॒तत् । दे॒वता॑नाम् । रू॒पम् । यत् । ए॒षः । ग्रहः॑ । यस्य॑ । ए॒षः । गृ॒ह्यते᳚ । सर्वा॑णि । ए॒व । ए॒न॒म् । रू॒पाणि॑ । प॒शू॒नाम् । उपेति॑ । ति॒ष्ठ॒न्ते॒ । उ॒प॒या॒मगृ॑हीत॒ इत्यु॑पया॒म - गृ॒ही॒तः॒ । \textbf{  26} \newline
                  \newline
                                \textbf{ TS 3.5.9.2} \newline
                  अ॒सि॒ । प्र॒जाप॑तय॒ इति॑ प्र॒जा - प॒त॒ये॒ । त्वा॒ । ज्योति॑ष्मते । ज्योति॑ष्मन्तम् । गृ॒ह्णा॒मि॒ । इति॑ । आ॒ह॒ । ज्योतिः॑ । ए॒व । ए॒न॒म् । स॒मा॒नाना᳚म् । क॒रो॒ति॒ । अ॒ग्नि॒जि॒ह्वेभ्य॒ इत्य॑ग्नि - जि॒ह्वेभ्यः॑ । त्वा॒ । ऋ॒ता॒युभ्य॒ इत्यृ॑ता॒यु - भ्यः॒ । इति॑ । आ॒ह॒ । ए॒ताव॑तीः । वै । दे॒वताः᳚ । ताभ्यः॑ । ए॒व । ए॒न॒म् । सर्वा᳚भ्यः । गृ॒ह्णा॒ति॒ । अपेति॑ । इ॒न्द्र॒ । द्वि॒ष॒तः । मनः॑ । इति॑ । आ॒ह॒ । भ्रातृ॑व्यापनुत्त्या॒ इति॒ भ्रातृ॑व्य - अ॒प॒नु॒त्त्यै॒ । प्रा॒णायेति॑ प्र - अ॒नाय॑ । त्वा॒ । अ॒पा॒नायेत्य॑प - अ॒नाय॑ । त्वा॒ । इति॑ । आ॒ह॒ । प्रा॒णानिति॑ प्र - अ॒नान् । ए॒व । यज॑माने । द॒धा॒ति॒ । तस्मै᳚ । त्वा॒ । प्र॒जाप॑तय॒ इति॑ प्र॒जा - प॒त॒ये॒ । वि॒भू॒दाव्.न्न॒ इति॑ विभु - दाव्.न्ने᳚ । ज्योति॑ष्मते । ज्योति॑ष्मन्तम् । जु॒हो॒मि॒ । \textbf{  27} \newline
                  \newline
                                \textbf{ TS 3.5.9.3} \newline
                  इति॑ । आ॒ह॒ । प्र॒जाप॑ति॒रिति॑ प्र॒जा - प॒तिः॒ । सर्वाः᳚ । दे॒वताः᳚ । सर्वा᳚भ्यः । ए॒व । ए॒न॒म् । दे॒वता᳚भ्यः । जु॒हो॒ति॒ । आ॒ज्य॒ग्र॒हमित्या᳚ज्य - ग्र॒हम् । गृ॒ह्णी॒या॒त् । तेज॑स्काम॒स्येति॒ तेजः॑ - का॒म॒स्य॒ । तेजः॑ । वै । आज्य᳚म् । ते॒ज॒स्वी । ए॒व । भ॒व॒ति॒ । सो॒म॒ग्र॒हमिति॑ सोम - ग्र॒हम् । गृ॒ह्णी॒या॒त् । ब्र॒ह्म॒व॒र्च॒सका॑म॒स्येति॑ ब्रह्मवर्च॒स - का॒म॒स्य॒ । ब्र॒ह्म॒व॒र्च॒समिति॑ ब्रह्म - व॒र्च॒सम् । वै । सोमः॑ । ब्र॒ह्म॒व॒र्च॒सीति॑ ब्रह्म - व॒र्च॒सी । ए॒व । भ॒व॒ति॒ । द॒धि॒ग्र॒हमिति॑ दधि - ग्र॒हम् । गृ॒ह्णी॒या॒त् । प॒शुका॑म॒स्येति॑ प॒शु - का॒म॒स्य॒ । ऊर्क् । वै । दधि॑ । ऊर्क् । प॒शवः॑ । ऊ॒र्जा । ए॒व । अ॒स्मै॒ । ऊर्ज᳚म् । प॒शून् । अवेति॑ । रु॒न्धे॒ ॥ \textbf{  28} \newline
                  \newline
                      (उ॒प॒या॒मगृ॑हीतो - जुहोमि॒ - त्रिच॑त्वारिꣳशच्च)  \textbf{(A9)} \newline \newline
                                \textbf{ TS 3.5.10.1} \newline
                  त्वे इति॑ । क्रतु᳚म् । अपीति॑ । वृ॒ञ्ज॒न्ति॒ । विश्वे᳚ । द्विः । यत् । ए॒ते । त्रिः । भव॑न्ति । ऊमाः᳚ ॥ स्वा॒दोः । स्वादी॑यः । स्वा॒दुना᳚ । सृ॒ज॒ । समिति॑ । अतः॑ । उ॒ । स्विति॑ । मधु॑ । मधु॑ना । अ॒भीति॑ । यो॒धि॒ ॥ उ॒प॒या॒मगृ॑हीत॒ इत्यु॑पया॒म - गृ॒ही॒तः॒ । अ॒सि॒ । प्र॒जाप॑तय॒ इति॑ प्र॒जा - प॒त॒ये॒ । त्वा॒ । जुष्ट᳚म् । गृ॒ह्णा॒मि॒ । ए॒षः । ते॒ । योनिः॑ । प्र॒जाप॑तय॒ इति॑ प्र॒जा - प॒त॒ये॒ । त्वा॒ ॥ प्रा॒ण॒ग्र॒हानिति॑ प्राण-ग्र॒हान् । गृ॒ह्णा॒ति॒ । ए॒ताव॑त् । वै । अ॒स्ति॒ । याव॑त् । ए॒ते । ग्रहाः᳚ । स्तोमाः᳚ । छन्दाꣳ॑सि । पृ॒ष्ठानि॑ । दिशः॑ । याव॑त् । ए॒व । अस्ति॑ । तत् । \textbf{  29} \newline
                  \newline
                                \textbf{ TS 3.5.10.2} \newline
                  अवेति॑ । रु॒न्धे॒ । ज्ये॒ष्ठाः । वै । ए॒तान् । ब्रा॒ह्म॒णाः । पु॒रा । वि॒दाम् । अ॒क्र॒न्न् । तस्मा᳚त् । तेषा᳚म् । सर्वाः᳚ । दिशः॑ । अ॒भिजि॑ता॒ इत्य॒भि - जि॒ताः॒ । अ॒भू॒व॒न्न् । यस्य॑ । ए॒ते । गृ॒ह्यन्ते᳚ । ज्यैष्ठ्य᳚म् । ए॒व । ग॒च्छ॒ति॒ । अ॒भीति॑ । दिशः॑ । ज॒य॒ति॒ । पञ्च॑ । गृ॒ह्य॒न्ते॒ । पञ्च॑ । दिशः॑ । सर्वा॑सु । ए॒व । दि॒क्षु । ऋ॒द्ध्नु॒व॒न्ति॒ । नव॑न॒वेति॒ नव॑ - न॒व॒ । गृ॒ह्य॒न्ते॒ । नव॑ । वै । पुरु॑षे । प्रा॒णा इति॑ प्र - अ॒नाः । प्रा॒णानिति॑ प्र - अ॒नान् । ए॒व । यज॑मानेषु । द॒ध॒ति॒ । प्रा॒य॒णीय॒ इति॑ प्र - अ॒य॒नीये᳚ । च॒ । उ॒द॒य॒नीय॒ इत्यु॑त् - अ॒य॒नीये᳚ । च॒ । गृ॒ह्य॒न्ते॒ । प्रा॒णा इति॑ प्र - अ॒नाः । वै । प्रा॒ण॒ग्र॒हा इति॑ प्राण - ग्र॒हाः । \textbf{  30} \newline
                  \newline
                                \textbf{ TS 3.5.10.3} \newline
                  प्रा॒णैरिति॑ प्र - अ॒नैः । ए॒व । प्र॒यन्तीति॑ प्र - यन्ति॑ । प्रा॒णैरिति॑ प्र - अ॒नैः । उदिति॑ । य॒न्ति॒ । द॒श॒मे । अहन्न्॑ । गृ॒ह्य॒न्ते॒ । प्रा॒णा इति॑ प्र - अ॒नाः । वै । प्रा॒ण॒ग्र॒हा इति॑ प्राण - ग्र॒हाः । प्रा॒णेभ्य॒ इति॑ प्र - अ॒नेभ्यः॑ । खलु॑ । वै । ए॒तत् । प्र॒जा इति॑ प्र - जाः । य॒न्ति॒ । यत् । वा॒म॒दे॒व्यमिति॑ वाम - दे॒व्यम् । योनेः᳚ । च्यव॑ते । द॒श॒मे । अहन्न्॑ । वा॒म॒दे॒व्यमिति॑ वाम - दे॒व्यम् । योनेः᳚ । च्य॒व॒ते॒ । यत् । द॒श॒मे । अहन्न्॑ । गृ॒ह्यन्ते᳚ । प्रा॒णेभ्य॒ इति॑ प्र - अ॒नेभ्यः॑ । ए॒व । तत् । प्र॒जा इति॑ प्र - जाः । न । य॒न्ति॒ ॥ \textbf{  31} \newline
                  \newline
                      (तत् - प्रा॑णग्र॒हाः - स॒प्तविꣳ॑शच्च)  \textbf{(A10)} \newline \newline
                                \textbf{ TS 3.5.11.1} \newline
                  प्रेति॑ । दे॒वम् । दे॒व्या । धि॒या । भर॑त । जा॒तवे॑दस॒मिति॑ जा॒त - वे॒द॒स॒म् ॥ ह॒व्या । नः॒ । व॒क्ष॒त् । आ॒नु॒षक् ॥ अ॒यम् । उ॒ । स्यः । प्रेति॑ । दे॒व॒युरिति॑ देव-युः । होता᳚ । य॒ज्ञाय॑ । नी॒य॒ते॒ ॥ रथः॑ । न । योः । अ॒भीवृ॑त॒ इत्य॒भि - वृ॒तः॒ । घृणी॑वान् । चे॒त॒ति॒ । त्मना᳚ ॥ अ॒यम् । अ॒ग्निः । उ॒रु॒ष्य॒ति॒ । अ॒मृता᳚त् । इ॒व॒ । जन्म॑नः ॥ सह॑सः । चि॒त् । सही॑यान् । दे॒वः । जी॒वात॑वे । कृ॒तः ॥ इडा॑याः । त्वा॒ । प॒दे । व॒यम् । नाभा᳚ । पृ॒थि॒व्याः । अधि॑ ॥ जात॑वेद॒ इति॒ जात॑ - वे॒दः॒ । नीति॑ । धी॒म॒हि॒ । अग्ने᳚ । ह॒व्याय॑ । वोढ॑वे ॥ \textbf{  32} \newline
                  \newline
                                \textbf{ TS 3.5.11.2} \newline
                  अग्ने᳚ । विश्वे॑भिः । स्व॒नी॒केति॑ सु - अ॒नी॒क॒ । दे॒वैः । ऊर्णा॑वन्त॒मित्यूर्णा᳚-व॒न्त॒म् । प्र॒थ॒मः । सी॒द॒ । योनि᳚म् ॥ कु॒ला॒यिन᳚म् । घृ॒तव॑न्त॒मिति॑ घृ॒त - व॒न्त॒म् । स॒वि॒त्रे । य॒ज्ञ्म् । न॒य॒ । यज॑मानाय । सा॒धु ॥ सीद॑ । हो॒तः॒ । स्वे । उ॒ । लो॒के । चि॒कि॒त्वान् । सा॒दय॑ । य॒ज्ञ्म् । सु॒कृ॒तस्येति॑ सु - कृ॒तस्य॑ । योनौ᳚ ॥ दे॒वा॒वीरिति॑ देव - अ॒वीः । दे॒वान् । ह॒विषा᳚ । य॒जा॒सि॒ । अग्ने᳚ । बृ॒हत् । यज॑माने । वयः॑ । धाः॒ ॥ नीति॑ । होता᳚ । हो॒तृ॒षद॑न॒ इति॑ होतृ-सद॑ने । विदा॑नः । त्वे॒षः । दी॒दि॒वान् । अ॒स॒द॒त् । सु॒दक्ष॒ इति॑ सु - दक्षः॑ ॥ अद॑ब्धव्रतप्रमति॒रित्यद॑ब्धव्रत - प्र॒म॒तिः॒ । वसि॑ष्ठः । स॒ह॒स्र॒भं॒र इति॑ सहस्रं - भ॒रः । शुचि॑जिह्व॒ इति॒ शुचि॑ - जि॒ह्वः॒ । अ॒ग्निः ॥ त्वम् । दू॒तः । त्वम् । \textbf{  33} \newline
                  \newline
                                \textbf{ TS 3.5.11.3} \newline
                  उ॒ । नः॒ । प॒र॒स्पा इति॑ परः - पाः । त्वम् । वस्यः॑ । एति॑ । वृ॒ष॒भ॒ । प्र॒णे॒तेति॑ प्र - ने॒ता ॥ अग्ने᳚ । तो॒कस्य॑ । नः॒ । तने᳚ । त॒नूना᳚म् । अप्र॑युच्छ॒नित्यप्र॑ - यु॒च्छ॒न्न् । दीद्य॑त् । बो॒धि॒ । गो॒पा इति॑ गो-पाः ॥ अ॒भीति॑ । त्वा॒ । दे॒व॒ । स॒वि॒तः॒ । ईशा॑नम् । वार्या॑णाम् ॥ सदा᳚ । अ॒व॒न्न् । भा॒गम् । ई॒म॒हे॒ ॥ म॒ही । द्यौः । पृ॒थि॒वी । च॒ । नः॒ । इ॒मम् । य॒ज्ञ्म् । मि॒मि॒क्ष॒ता॒म् ॥ पि॒पृ॒ताम् । नः॒ । भरी॑मभि॒रिति॒ भरी॑म-भिः॒ ॥ त्वाम् । अ॒ग्ने॒ । पुष्क॑रात् । अधीति॑ । अथ॑र्वा । निरिति॑ । अ॒म॒न्थ॒त॒ ॥ मू॒र्द्ध्नः । विश्व॑स्य । वा॒घतः॑ ॥ तम् । उ॒ । \textbf{  34} \newline
                  \newline
                                \textbf{ TS 3.5.11.4} \newline
                  त्वा॒ । द॒द्ध्यङ् । ऋषिः॑ । पु॒त्रः । ई॒धे॒ । अथ॑र्वणः ॥ वृ॒त्र॒हण॒मिति॑ वृत्र - हन᳚म् । पु॒र॒न्द॒रमिति॑ पुरं - द॒रम् ॥ तम् । उ॒ । त्वा॒ । पा॒थ्यः । वृषा᳚ । समिति॑ । ई॒धे॒ । द॒स्यु॒हन्त॑म॒मिति॑ दस्यु - हन्त॑मम् ॥ ध॒न॒ञ्ज॒यमिति॑ धनं - ज॒यम् । रणे॑रण॒ इति॒ रणे᳚ - र॒णे॒ ॥ उ॒त । ब्रु॒व॒न्तु॒ । ज॒न्तवः॑ । उदिति॑ । अ॒ग्निः । वृ॒त्र॒हेति॑ वृत्र - हा । अ॒ज॒नि॒ ॥ ध॒न॒ञ्ज॒य इति॑ धनं - ज॒यः । रणे॑रण॒ इति॒ रणे᳚-र॒णे॒ ॥ एति॑ । यम् । हस्ते᳚ । न । खा॒दिन᳚म् । शिशु᳚म् । जा॒तम् । न । बिभ्र॑ति ॥ वि॒शाम् । अ॒ग्निम् । स्व॒द्ध्व॒रमिति॑ सु - अ॒ध्व॒रम् ॥ प्रेति॑ । दे॒वम् । दे॒ववी॑तय॒ इति॑ दे॒व - वी॒त॒ये॒ । भर॑त । व॒सु॒वित्त॑म॒मिति॑ वसु॒वित्-त॒म॒म् ॥ एति॑ । स्वे । योनौ᳚ । नीति॑ । सी॒द॒तु॒ ॥ एति॑ । \textbf{  35} \newline
                  \newline
                                \textbf{ TS 3.5.11.5} \newline
                  जा॒तम् । जा॒तवे॑द॒सीति॑ जा॒त - वे॒द॒सि॒ । प्रि॒यम् । शि॒शी॒त॒ । अति॑थिम् ॥ स्यो॒ने । एति॑ । गृ॒हप॑ति॒मिति॑ गृ॒ह - प॒ति॒म् ॥ अ॒ग्निना᳚ । अ॒ग्निः । समिति॑ । इ॒द्ध्य॒ते॒ । क॒विः । गृ॒हप॑ति॒रिति॑ गृ॒ह - प॒तिः॒ । युवा᳚ ॥ ह॒व्य॒वाडिति॑ हव्य - वाट् । जु॒ह्वा᳚स्य॒ इति॑ जु॒हु - आ॒स्यः॒ ॥ त्वम् । हि । अ॒ग्ने॒ । अ॒ग्निना᳚ । विप्रः॑ । विप्रे॑ण । सन्न् । स॒ता ॥ सखा᳚ । सख्या᳚ । स॒मि॒द्ध्यस॒ इति॑ सं - इ॒ध्यसे᳚ ॥ तम् । म॒र्ज॒य॒न्त॒ । सु॒क्रतु॒मिति॑ सु - क्रतु᳚म् । पु॒रो॒यावा॑न॒मिति॑ पुरः - यावा॑नम् । आ॒जिषु॑ ॥ स्वेषु॑ । क्षये॑षु । वा॒जिन᳚म् ॥ य॒ज्ञेन॑ । य॒ज्ञ्म् । अ॒य॒ज॒न्त॒ । दे॒वाः । तानि॑ । धर्मा॑णि । प्र॒थ॒मानि॑ । आ॒स॒न्न् ॥ ते । ह॒ । नाक᳚म् । म॒हि॒मानः॑ । स॒च॒न्ते॒ । यत्र॑ ( ) । पूर्वे᳚ । सा॒द्ध्याः । सन्ति॑ । दे॒वाः ॥ \textbf{  36} \newline
                  \newline
                      (वोढ॑वे- दू॒तस्त्वं - तमु॑ - सीद॒त्वा - यत्र॑ - च॒त्वारि॑ च)  \textbf{(A11)} \newline \newline
\textbf{praSna korvai with starting padams of 1 to 11 anuvAkams :-} \newline
(पू॒र्ण - र्.ष॑यो॒ - ऽग्निना॒ - ये दे॒वाः - सूर्यो॑ मा॒ - सन्त्वा॑ नह्यामि - वषट्का॒रः स ख॑दि॒र - उ॑पया॒मगृ॑हीतोऽसि॒ - यां ॅवै - त्वे क्रतुं॒ - प्रदे॒व - मेका॑दश ) \newline

\textbf{korvai with starting padams of1, 11, 21 series of pa~jcAtis :-} \newline
(पू॒र्णा - स॑ह॒जान् - तवा᳚ऽग्ने - प्रा॒णैरे॒व - षट्त्रिꣳ॑शत्) \newline

\textbf{first and last padam of Fifth praSnam of kANDam 3:-} \newline
(पू॒र्णा - सन्ति॑ दे॒वाः) \newline 


॥ हरिः॑ ॐ ॥॥ कृष्ण यजुर्वेदीय तैत्तिरीय संहितायां तृतीयकाण्डे पञ्चमः प्रश्नः समाप्तः ॥

॥ इति तृतीयं काण्डं ॥ \newline
\pagebreak
\pagebreak
        


\end{document}
