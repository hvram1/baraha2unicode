\documentclass[17pt]{extarticle}
\usepackage{babel}
\usepackage{fontspec}
\usepackage{polyglossia}
\usepackage{extsizes}



\setmainlanguage{sanskrit}
\setotherlanguages{english} %% or other languages
\setlength{\parindent}{0pt}
\pagestyle{myheadings}
\newfontfamily\devanagarifont[Script=Devanagari]{AdishilaVedic}


\newcommand{\VAR}[1]{}
\newcommand{\BLOCK}[1]{}




\begin{document}
\begin{titlepage}
    \begin{center}
 
\begin{sanskrit}
    { \Large
    ॐ नमः परमात्मने, श्री महागणपतये नमः, श्री गुरुभ्यो नमः
ह॒रिः॒ ॐ
========================================= 
    }
    \\
    \vspace{2.5cm}
    \mbox{ \Huge
    3.4      तृतीयकाण्डे चतुर्थः प्रश्नः - इष्टिहोमाभिधानं   }
\end{sanskrit}
\end{center}

\end{titlepage}
\tableofcontents

ॐ नमः परमात्मने, श्री महागणपतये नमः, श्री गुरुभ्यो नमः
ह॒रिः॒ ॐ
========================================= \newline
3.4      तृतीयकाण्डे चतुर्थः प्रश्नः - इष्टिहोमाभिधानं \newline

\addcontentsline{toc}{section}{ 3.4      तृतीयकाण्डे चतुर्थः प्रश्नः - इष्टिहोमाभिधानं}
\markright{ 3.4      तृतीयकाण्डे चतुर्थः प्रश्नः - इष्टिहोमाभिधानं \hfill https://www.vedavms.in \hfill}
\section*{ 3.4      तृतीयकाण्डे चतुर्थः प्रश्नः - इष्टिहोमाभिधानं }
                                \textbf{ TS 3.4.1.1} \newline
                  वीति॑ । वै । ए॒तस्य॑ । य॒ज्ञ्ः । ऋ॒द्ध्य॒ते॒ । यस्य॑ । ह॒विः । अ॒ति॒रिच्य॑त॒ इत्य॑ति - रिच्य॑ते । सूर्यः॑ । दे॒वः । दि॒वि॒षद्भ्य॒ इति॑ दिवि॒षत् - भ्यः॒ । इति॑ । आ॒ह॒ । बृह॒स्पति॑ना । च॒ । ए॒व । अ॒स्य॒ । प्र॒जाप॑ति॒नेति॑ प्र॒जा - प॒ति॒ना॒ । च॒ । य॒ज्ञ्स्य॑ । व्यृ॑द्ध॒मिति॒ वि - ऋ॒द्ध॒म् । अपीति॑ । व॒प॒ति॒ । रक्षाꣳ॑सि । वै । ए॒तत् । प॒शुम् । स॒च॒न्ते॒ । यत् । ए॒क॒दे॒व॒त्य॑ इत्ये॑क - दे॒व॒त्यः॑ । आल॑ब्ध॒ इत्या - ल॒ब्धः॒ । भूयान्॑ । भव॑ति । यस्याः᳚ । ते॒ । हरि॑तः । गर्भः॑ । इति॑ । आ॒ह॒ । दे॒व॒त्रेति॑ देव - त्रा । ए॒व । ए॒ना॒म् । ग॒म॒य॒ति॒ । रक्ष॑साम् । अप॑हत्या॒ इत्यप॑ - ह॒त्यै॒ । एति॑ । व॒र्त॒न॒ । व॒र्त॒य॒ । इति॑ । आ॒ह॒ । \textbf{  1} \newline
                  \newline
                                \textbf{ TS 3.4.1.2} \newline
                  ब्रह्म॑णा । ए॒व । ए॒न॒म् । एति॑ । व॒र्त॒य॒ति॒ । वीति॑ । ते॒ । भि॒न॒द्मि॒ । त॒क॒रीम् । इति॑ । आ॒ह॒ । य॒था॒य॒जुरिति॑ यथा - य॒जुः । ए॒व । ए॒तत् । उ॒रु॒द्र॒फ्स इत्यु॑रु - द्र॒फ्सः । वि॒श्वरू॑प॒ इति॑ वि॒श्व - रू॒पः॒ । इन्दुः॑ । इति॑ । आ॒ह॒ । प्र॒जेति॑ प्र - जा । वै । प॒शवः॑ । इन्दुः॑ । प्र॒जयेति॑ प्र - जया᳚ । ए॒व । ए॒न॒म् । प॒शुभि॒रिति॑ प॒शु - भिः॒ । समिति॑ । अ॒र्द्ध॒य॒ति॒ । दिव᳚म् । वै । य॒ज्ञ्स्य॑ । व्यृ॑द्ध॒मिति॒ वि-ऋ॒द्ध॒म् । ग॒च्छ॒ति॒ । पृ॒थि॒वीम् । अति॑रिक्त॒मित्यति॑ - रि॒क्त॒म् । तत् । यत् । न । श॒मये᳚त् । आर्ति᳚म् । एति॑ । ऋ॒च्छे॒त् । यज॑मानः । म॒ही । द्यौः । पृ॒थि॒वी । च॒ । नः॒ । इति॑ । \textbf{  2} \newline
                  \newline
                                \textbf{ TS 3.4.1.3} \newline
                  आ॒ह॒ । द्यावा॑पृथि॒वीभ्या॒मिति॒ द्यावा᳚ - पृ॒थि॒वीभ्या᳚म् । ए॒व । य॒ज्ञ्स्य॑ । व्यृ॑द्ध॒मिति॒ वि - ऋ॒द्ध॒म् । च॒ । अति॑रिक्त॒मित्यति॑ - रि॒क्त॒म् । च॒ । श॒म॒य॒ति॒ । न । आर्ति᳚म् । एति॑ । ऋ॒च्छ॒ति॒ । यज॑मानः । भस्म॑ना । अ॒भि । समिति॑ । ऊ॒ह॒ति॒ । स्व॒गाकृ॑त्या॒ इति॑ स्व॒गा - कृ॒त्यै॒ । अथो॒ इति॑ । अ॒नयोः᳚ । वै । ए॒षः । गर्भः॑ । अ॒नयोः᳚ । ए॒व । ए॒न॒म् । द॒धा॒ति॒ । यत् । अ॒व॒द्येदित्य॑व-द्येत् । अतीति॑ । तत् । रे॒च॒ये॒त् । यत् । न । अ॒व॒द्येदित्य॑व - द्येत् । प॒शोः । आल॑ब्ध॒स्येत्या-ल॒ब्ध॒स्य॒ । न । अवेति॑ । द्ये॒त् । पु॒रस्ता᳚त् । नाभ्याः᳚ । अ॒न्यत् । अ॒व॒द्येदित्य॑व - द्येत् । उ॒परि॑ष्टात् । अ॒न्यत् । पु॒रस्ता᳚त् । वै । नाभ्यै᳚ । \textbf{  3} \newline
                  \newline
                                \textbf{ TS 3.4.1.4} \newline
                  प्रा॒ण इति॑ प्र - अ॒नः । उ॒परि॑ष्टात् । अ॒पा॒न इत्य॑प-अ॒नः । यावान्॑ । ए॒व । प॒शुः । तस्य॑ । अवेति॑ । द्य॒ति॒ । विष्ण॑वे । शि॒पि॒वि॒ष्टायेति॑ शिपि - वि॒ष्टाय॑ । जु॒हो॒ति॒ । यत् । वै । य॒ज्ञ्स्य॑ । अ॒ति॒रिच्य॑त॒ इत्य॑ति - रिच्य॑ते । यः । प॒शोः । भू॒मा । या । पुष्टिः॑ । तत् । विष्णुः॑ । शि॒पि॒वि॒ष्ट इति॑ शिपि - वि॒ष्टः । अति॑रिक्त॒ इत्यति॑ - रि॒क्ते॒ । ए॒व । अति॑रिक्त॒मित्यति॑ - रि॒क्त॒म् । द॒धा॒ति॒ । अति॑रिक्त॒स्येत्यति॑ - रि॒क्त॒स्य॒ । शान्त्यै᳚ । अ॒ष्टाप्रू॒डित्य॒ष्टा - प्रू॒ट् । हिर॑ण्यम् । दक्षि॑णा । अ॒ष्टाप॒दीत्य॒ष्टा - प॒दी॒ । हि । ए॒षा । आ॒त्मा । न॒व॒मः । प॒शोः । आप्त्यै᳚ । अ॒न्त॒र॒को॒श इत्य॑न्तर - को॒शे । उ॒ष्णीषे॑ण । आवि॑ष्टित॒मित्या - वि॒ष्टि॒त॒म् । भ॒व॒ति॒ । ए॒वम् । इ॒व॒ । हि । प॒शुः । उल्ब᳚म् । इ॒व॒ ( ) । चर्म॑ । इ॒व॒ । माꣳ॒॒सम् । इ॒व॒ । अस्थि॑ । इ॒व॒ । यावान्॑ । ए॒व । प॒शुः । तम् । आ॒प्त्वा । अवेति॑ । रु॒न्धे॒ । यस्य॑ । ए॒षा । य॒ज्ञे । प्राय॑श्चित्तिः । क्रि॒यते᳚ । इ॒ष्ट्वा । वसी॑यान् । भ॒व॒ति॒ ॥ \textbf{  4} \newline
                  \newline
                      (व॒र्त॒यत्या॑ह-न॒ इति॒-वै नाभ्या॒-उल्ब॑मि॒वै-क॑विꣳशतिश्च)  \textbf{(A1)} \newline \newline
                                \textbf{ TS 3.4.2.1} \newline
                  एति॑ । वा॒यो॒ इति॑ । भू॒ष॒ । शु॒चि॒पा॒ इति॑ शुचि - पाः॒ । उपेति॑ । नः॒ । स॒हस्र᳚म् । ते॒ । नि॒युत॒ इति॑ नि - युतः॑ । वि॒श्व॒वा॒रेति॑ विश्व - वा॒र॒ ॥ उपो॒ इति॑ । ते॒ । अन्धः॑ । मद्य᳚म् । अ॒या॒मि॒ । यस्य॑ । दे॒व॒ । द॒धि॒षे । पू॒र्व॒पेय॒मिति॑ पूर्व - पेय᳚म् ॥ आकू᳚त्या॒ इत्या-कू॒त्यै॒ । त्वा॒ । कामा॑य । त्वा॒ । स॒मृध॒ इति॑ सं - ऋधे᳚ । त्वा॒ । कि॒क्कि॒टा । ते॒ । मनः॑ । प्र॒जाप॑तय॒ इति॑ प्र॒जा - प॒त॒ये॒ । स्वाहा᳚ । कि॒क्कि॒टा । ते॒ । प्रा॒णमिति॑ प्र - अ॒नम् । वा॒यवे᳚ । स्वाहा᳚ । कि॒क्कि॒टा । ते॒ । चक्षुः॑ । सूर्या॑य । स्वाहा᳚ । कि॒क्कि॒टा । ते॒ । श्रोत्र᳚म् । द्यावा॑पृथि॒वीभ्या॒मिति॒ द्यावा᳚-पृ॒थि॒वीभ्या᳚म् । स्वाहा᳚ । कि॒क्कि॒टा । ते॒ । वाच᳚म् । सर॑स्वत्यै । स्वाहा᳚ । \textbf{  5} \newline
                  \newline
                                \textbf{ TS 3.4.2.2} \newline
                  त्वम् । तु॒रीया᳚ । व॒शिनी᳚ । व॒शा । अ॒सि॒ । स॒कृत् । यत् । त्वा॒ । मन॑सा । गर्भः॑ । एति॑ । अश॑यत् ॥ व॒शा । त्वम् । व॒शिनी᳚ । ग॒च्छ॒ । दे॒वान् । स॒त्याः । स॒न्तु॒ । यज॑मानस्य । कामाः᳚ ॥ अ॒जा । अ॒सि॒ । र॒यि॒ष्ठेति॑ रयि - स्था । पृ॒थि॒व्याम् । सी॒द॒ । ऊ॒र्द्ध्वा । अ॒न्तरि॑क्षम् । उपेति॑ । ति॒ष्ठ॒स्व॒ । दि॒वि । ते॒ । बृ॒हत् । भाः ॥ तन्तु᳚म् । त॒न्वन्न् । रज॑सः । भा॒नुम् । अन्विति॑ । इ॒हि॒ । ज्योति॑ष्मतः । प॒थः । र॒क्ष॒ । धि॒या । कृ॒तान् ॥ अ॒नु॒ल्ब॒णम् । व॒य॒त॒ । जोगु॑वाम् । अपः॑ । मनुः॑( ) । भ॒व॒ । ज॒नय॑ । दैव्य᳚म् । जन᳚म् ॥ मन॑सः । ह॒विः । अ॒सि॒ । प्र॒जाप॑ते॒रिति॑ प्र॒जा - प॒तेः॒ । वर्णः॑ । गात्रा॑णाम् । ते॒ । गा॒त्र॒भाज॒ इति॑ गात्र - भाजः॑ । भू॒या॒स्म॒ ॥ \textbf{  6} \newline
                  \newline
                      (सर॑स्वत्यै॒ स्वाहा॒ - मनु॒ - स्त्रयो॑दश च)  \textbf{(A2)} \newline \newline
                                \textbf{ TS 3.4.3.1} \newline
                  इ॒मे इति॑ । वै । स॒ह । आ॒स्ता॒म् । ते इति॑ । वा॒युः । वीति॑ । अ॒वा॒त् । ते इति॑ । गर्भ᳚म् । अ॒द॒धा॒ता॒म् । तम् । सोमः॑ । प्रेति॑ । अ॒ज॒न॒य॒त् । अ॒ग्निः । अ॒ग्र॒स॒त॒ । सः । ए॒तम् । प्र॒जाप॑ति॒रिति॑ प्र॒जा - प॒तिः॒ । आ॒ग्ने॒यम् । अ॒ष्टाक॑पाल॒मित्य॒ष्टा - क॒पा॒ल॒म् । अ॒प॒श्य॒त् । तम् । निरिति॑ । अ॒व॒प॒त् । तेन॑ । ए॒व । ए॒ना॒म् । अ॒ग्नेः । अधि॑ । निरिति॑ । अ॒क्री॒णा॒त् । तस्मा᳚त् । अपीति॑ । अ॒न्य॒दे॒व॒त्या॑मित्य॑न्य - दे॒व॒त्या᳚म् । आ॒लभ॑मान॒ इत्या᳚ - लभ॑मानः । आ॒ग्ने॒यम् । अ॒ष्टाक॑पाल॒मित्य॒ष्टा-क॒पा॒ल॒म् । पु॒रस्ता᳚त् । निरिति॑ । व॒पे॒त् । अ॒ग्नेः । ए॒व । ए॒ना॒म् । अधीति॑ । नि॒ष्क्रीयेति॑ निः - क्रीय॑ । एति॑ । ल॒भ॒ते॒ । यत् । \textbf{  7} \newline
                  \newline
                                \textbf{ TS 3.4.3.2} \newline
                  वा॒युः । व्यवा॒दिति॑ वि - अवा᳚त् । तस्मा᳚त् । वा॒य॒व्या᳚ । यत् । इ॒मे इति॑ । गर्भ᳚म् । अद॑धाताम् । तस्मा᳚त् । द्या॒वा॒पृ॒थि॒व्येति॑ द्यावा - पृ॒थि॒व्या᳚ । यत् । सोमः॑ । प्रेति॑ । अज॑नयत् । अ॒ग्निः । अग्र॑सत । तस्मा᳚त् । अ॒ग्नी॒षो॒मीयेत्य॑ग्नी-सो॒मीया᳚ । यत् । अ॒नयोः᳚ । वि॒य॒त्योरिति॑ वि - य॒त्योः । वाक् । अव॑दत् । तस्मा᳚त् । सा॒र॒स्व॒ती । यत् । प्र॒जाप॑ति॒रिति॑ प्र॒जा - प॒तिः॒ । अ॒ग्नेः । अधीति॑ । नि॒रक्री॑णा॒दिति॑ निः - अक्री॑णात् । तस्मा᳚त् । प्रा॒जा॒प॒त्येति॑ प्राजा - प॒त्या । सा । वै । ए॒षा । स॒र्व॒दे॒व॒त्येति॑ सर्व - दे॒व॒त्या᳚ । यत् । अ॒जा । व॒शा । वा॒य॒व्या᳚म् । एति॑ । ल॒भे॒त॒ । भूति॑काम॒ इति॒ भूति॑ - का॒मः॒ । वा॒युः । वै । क्षेपि॑ष्ठा । दे॒वता᳚ । वा॒युम् । ए॒व । स्वेन॑ । \textbf{  8} \newline
                  \newline
                                \textbf{ TS 3.4.3.3} \newline
                  भा॒ग॒धेये॒नेति॑ भाग - धेये॑न । उपेति॑ । धा॒व॒ति॒ । सः । ए॒व । ए॒न॒म् । भूति᳚म् । ग॒म॒य॒ति॒ । द्या॒वा॒पृ॒थि॒व्या॑मिति॑ द्यावा - पृ॒थि॒व्या᳚म् । एति॑ । ल॒भे॒त॒ । कृ॒षमा॑णः । प्र॒ति॒ष्ठाका॑म॒ इति॑ प्रति॒ष्ठा-का॒मः॒ । दि॒वः । ए॒व । अ॒स्मै॒ । प॒र्जन्यः॑ । व॒र्॒.ष॒ति॒ । वीति॑ । अ॒स्याम् । ओष॑धयः । रो॒ह॒न्ति॒ । स॒मर्द्धु॑क॒मिति॑ सं - अर्द्धु॑कम् । अ॒स्य॒ । स॒स्यम् । भ॒व॒ति॒ । अ॒ग्नी॒षो॒मीया॒मित्य॑ग्नी-सो॒मीया᳚म् । एति॑ । ल॒भे॒त॒ । यः । का॒मये॑त । अन्न॑वा॒न्नित्यन्न॑ - वा॒न् । अ॒न्ना॒द इत्य॑न्न - अ॒दः । स्या॒म् । इति॑ । अ॒ग्निना᳚ । ए॒व । अन्न᳚म् । अवेति॑ । रु॒न्धे॒ । सोमे॑न । अ॒न्नाद्य॒मित्य॑न्न - अद्य᳚म् । अन्न॑वा॒न्नित्यन्न॑ - वा॒न् । ए॒व । अ॒न्ना॒द इत्य॑न्न - अ॒दः । भ॒व॒ति॒ । सा॒र॒स्व॒तीम् । एति॑ । ल॒भे॒त॒ । यः । \textbf{  9} \newline
                  \newline
                                \textbf{ TS 3.4.3.4} \newline
                  ई॒श्व॒रः । वा॒चः । वदि॑तोः । सन्न् । वाच᳚म् । न । वदे᳚त् । वाक् । वै । सर॑स्वती । सर॑स्वतीम् । ए॒व । स्वेन॑ । भा॒ग॒धेये॒नेति॑ भाग - धेये॑न । उपेति॑ । धा॒व॒ति॒ । सा । ए॒व । अ॒स्मि॒न्न् । वाच᳚म् । द॒धा॒ति॒ । प्रा॒जा॒प॒त्यामिति॑ प्राजा - प॒त्याम् । एति॑ । ल॒भे॒त॒ । यः । का॒मये॑त । अन॑भिजित॒मित्यन॑भि - जि॒त॒म् । अ॒भीति॑ । ज॒ये॒य॒म् । इति॑ । प्र॒जाप॑ति॒रिति॑ प्र॒जा - प॒तिः॒ । सर्वाः᳚ । दे॒वताः᳚ । दे॒वता॑भिः । ए॒व । अन॑भिजित॒मित्यन॑भि - जि॒त॒म् । अ॒भीति॑ । ज॒य॒ति॒ । वा॒य॒व्य॑या । उ॒पाक॑रो॒तीयु॑प - आक॑रोति । वा॒योः । ए॒व । ए॒ना॒म् । अ॒व॒रुद्ध्येत्य॑व - रुद्ध्य॑ । एति॑ । ल॒भ॒ते॒ । आकू᳚त्या॒ इत्या - कू॒त्यै॒ । त्वा॒ । कामा॑य । त्वा॒ । \textbf{  10} \newline
                  \newline
                                \textbf{ TS 3.4.3.5} \newline
                  इति॑ । आ॒ह॒ । य॒था॒य॒जुरिति॑ यथा - य॒जुः । ए॒व । ए॒तत् । कि॒क्कि॒टा॒कार॒मिति॑ किक्किटा - कार᳚म् । जु॒हो॒ति॒ । कि॒क्कि॒टा॒का॒रेणेति॑ किक्किटा - का॒रेण॑ । वै । ग्रा॒म्याः । प॒शवः॑ । र॒म॒न्ते॒ । प्रेति॑ । आ॒र॒ण्याः । प॒त॒न्ति॒ । यत् । कि॒क्कि॒टा॒कार॒मिति॑ किक्किटा - कार᳚म् । जु॒होति॑ । ग्रा॒म्याणा᳚म् । प॒शू॒नाम् । धृत्यै᳚ । पर्य॑ग्ना॒विति॒ परि॑-अ॒ग्नौ॒ । क्रि॒यमा॑णे । जु॒हो॒ति॒ । जीव॑न्तीम् । ए॒व । ए॒ना॒म् । सु॒व॒र्गमिति॑ सुवः - गम् । लो॒कम् । ग॒म॒य॒ति॒ । त्वम् । तु॒रीया᳚ । व॒शिनी᳚ । व॒शा । अ॒सि॒ । इति॑ । आ॒ह॒ । दे॒व॒त्रेति॑ देव - त्रा । ए॒व । ए॒ना॒म् । ग॒म॒य॒ति॒ । स॒त्याः । स॒न्तु॒ । यज॑मानस्य । कामाः᳚ । इति॑ । आ॒ह॒ । ए॒षः । वै । कामः॑ । \textbf{  11} \newline
                  \newline
                                \textbf{ TS 3.4.3.6} \newline
                  यज॑मानस्य । यत् । अना᳚र्तः । उ॒दृच॒मित्यु॑त् - ऋच᳚म् । गच्छ॑ति । तस्मा᳚त् । ए॒वम् । आ॒ह॒ । अ॒जा । अ॒सि॒ । र॒यि॒ष्ठेति॑ रयि - स्था । इति॑ । आ॒ह॒ । ए॒षु । ए॒व । ए॒ना॒म् । लो॒केषु॑ । प्रतीति॑ । स्था॒प॒य॒ति॒ । दि॒वि । ते॒ । बृ॒हत् । भाः । इति॑ । आ॒ह॒ । सु॒व॒र्ग इति॑ सुवः- गे । ए॒व । अ॒स्मै॒ । लो॒के । ज्योतिः॑ । द॒धा॒ति॒ । तन्तु᳚म् । त॒न्वन्न् । रज॑सः । भा॒नुम् । अन्विति॑ । इ॒हि॒ । इति॑ । आ॒ह॒ । इ॒मान् । ए॒व । अ॒स्मै॒ । लो॒कान् । ज्योति॑ष्मतः । क॒रो॒ति॒ । अ॒नु॒ल्ब॒णम् । व॒य॒त॒ । जोगु॑वाम् । अपः॑ । इति॑ । \textbf{  12} \newline
                  \newline
                                \textbf{ TS 3.4.3.7} \newline
                  आ॒ह॒ । यत् । ए॒व । य॒ज्ञे । उ॒ल्बण᳚म् । क्रि॒यते᳚ । तस्य॑ । ए॒व । ए॒षा । शान्तिः॑ । मनुः॑ । भ॒व॒ । ज॒नय॑ । दैव्य᳚म् । जन᳚म् । इति॑ । आ॒ह॒ । मा॒न॒व्यः॑ । वै । प्र॒जा इति॑ प्र - जाः । ताः । ए॒व । आ॒द्याः᳚ । कु॒रु॒ते॒ । मन॑सः । ह॒विः । अ॒सि॒ । इति॑ । आ॒ह॒ । स्व॒गाकृ॑त्या॒ इति॑ स्व॒गा - कृ॒त्यै॒ । गात्रा॑णाम् । ते॒ । गा॒त्र॒भाज॒ इति॑ गात्र - भाजः॑ । भू॒या॒स्म॒ । इति॑ । आ॒ह॒ । आ॒शिष॒मित्या᳚ - शिष᳚म् । ए॒व । ए॒ताम् । एति॑ । शा॒स्ते॒ । तस्यै᳚ । वै । ए॒तस्याः᳚ । एक᳚म् । ए॒व । अदे॑वयजन॒मित्यदे॑व - य॒ज॒न॒म् । यत् । आल॑ब्धाया॒मित्या - ल॒ब्धा॒या॒म् । अ॒भ्रः । \textbf{  13} \newline
                  \newline
                                \textbf{ TS 3.4.3.8} \newline
                  भव॑ति । यत् । आल॑ब्धाया॒मित्या - ल॒ब्धा॒या॒म् । अ॒भ्रः । स्यात् । अ॒फ्स्वित्य॑प्-सु । वा॒ । प्र॒वे॒शये॒दिति॑ प्र - वे॒शये᳚त् । सर्वा᳚म् । वा॒ । प्रेति॑ । अ॒श्नी॒या॒त् । यत् । अ॒फ्स्वित्य॑प् - सु । प्र॒वे॒शये॒दिति॑ प्र - वे॒शये᳚त् । य॒ज्ञ्॒वे॒श॒समिति॑ यज्ञ् - वे॒श॒सम् । कु॒र्या॒त् । सर्वा᳚म् । ए॒व । प्रेति॑ । अ॒श्नी॒या॒त् । इ॒न्द्रि॒यम् । ए॒व । आ॒त्मन् । ध॒त्ते॒ । सा । वै । ए॒षा । त्र॒या॒णाम् । ए॒व । अव॑रु॒द्धेत्यव॑ - रु॒द्धा॒ । सं॒ॅव॒थ्स॒र॒सद॒ इति॑ संॅवथ्सर - सदः॑ । स॒ह॒स्र॒या॒जिन॒ इति॑ सहस्र - या॒जिनः॑ । गृ॒ह॒मे॒धिन॒ इति॑ गृह - मे॒धिनः॑ । ते । ए॒व । ए॒तया᳚ । य॒जे॒र॒न्न् । तेषा᳚म् । ए॒व । ए॒षा । आ॒प्ता ॥ \textbf{  14} \newline
                  \newline
                      (यथ् - स्वेन॑ - सारस्व॒तीमा ल॑भेत॒ यः - कामा॑य त्वा॒ - कामो - ऽप॒ इत्य॒ - भ्रो - द्विच॑त्वारिꣳशच्च)  \textbf{(A3)} \newline \newline
                                \textbf{ TS 3.4.4.1} \newline
                  चि॒त्तम् । च॒ । चित्तिः॑ । च॒ । आकू॑त॒मित्या - कू॒त॒म् । च॒ । आकू॑ति॒रित्या - कू॒तिः॒ । च॒ । विज्ञा॑त॒मिति॒ वि - ज्ञा॒त॒म् । च॒ । वि॒ज्ञान॒मिति॑ वि-ज्ञान᳚म् । च॒ । मनः॑ । च॒ । शक्व॑रीः । च॒ । दर्.शः॑ । च॒ । पू॒र्णमा॑स॒ इति॑ पू॒र्ण - मा॒सः॒ । च॒ । बृ॒हत् । च॒ । र॒थ॒न्त॒रमिति॑ रथं - त॒रम् । च॒ । प्र॒जाप॑ति॒रिति॑ प्र॒जा - प॒तिः॒ । जयान्॑ । इन्द्रा॑य । वृष्णे᳚ । प्रेति॑ । अ॒य॒च्छ॒त् । उ॒ग्रः । पृ॒त॒नाज्ये॑षु । तस्मै᳚ । विशः॑ । समिति॑ । अ॒न॒म॒न्त॒ । सर्वाः᳚ । सः । उ॒ग्रः । सः । हि । हव्यः॑ । ब॒भूव॑ । दे॒वा॒सु॒रा इति॑ देव - अ॒सु॒राः । संॅय॑त्ता॒ इति॒ सं - य॒त्ताः॒ । आ॒स॒न्न् । सः । इन्द्रः॑ । प्र॒जाप॑ति॒मिति॑ प्र॒जा - प॒ति॒म् । उपेति॑ ( ) । अ॒धा॒व॒त् । तस्मै᳚ । ए॒तान् । जयान्॑ । प्रेति॑ । अ॒य॒च्छ॒त् । तान् । अ॒जु॒हो॒त् । ततः॑ । वै । दे॒वाः । असु॑रान् । अ॒ज॒य॒न्न् । यत् । अज॑यन्न् । तत् । जया॑नाम् । ज॒य॒त्वमिति॑ जय - त्वम् । स्पर्द्ध॑मानेन । ए॒ते । हो॒त॒व्याः᳚ । जय॑ति । ए॒व । ताम् । पृत॑नाम् ॥ \textbf{  15} \newline
                  \newline
                      (उप॒ - पञ्च॑विꣳशतिश्च)  \textbf{(A4)} \newline \newline
                                \textbf{ TS 3.4.5.1} \newline
                  अ॒ग्निः । भू॒ताना᳚म् । अधि॑पति॒रित्यधि॑- प॒तिः॒ । सः । मा॒ । अ॒व॒तु॒ । इन्द्रः॑ । ज्ये॒ष्ठाना᳚म् । य॒मः । पृ॒थि॒व्याः । वा॒युः । अ॒न्तरि॑क्षस्य । सूर्यः॑ । दि॒वः । च॒न्द्रमाः᳚ । नक्ष॑त्राणाम् । बृह॒स्पतिः॑ । ब्रह्म॑णः । मि॒त्रः । स॒त्याना᳚म् । वरु॑णः । अ॒पाम् । स॒मु॒द्रः । स्रो॒त्याना᳚म् । अन्न᳚म् । साम्रा᳚ज्याना॒मिति॒ सां - रा॒ज्या॒ना॒म् । अधि॑प॒तीत्यधि॑-प॒ति॒ । तत् । मा॒ । अ॒व॒तु॒ । सोमः॑ । ओष॑धीनाम् । स॒वि॒ता । प्र॒स॒वाना॒मिति॑ प्र - स॒वाना᳚म् । रु॒द्रः । प॒शू॒नाम् । त्वष्टा᳚ । रू॒पाणा᳚म् । विष्णुः॑ । पर्व॑तानाम् । म॒रुतः॑ । ग॒णाना᳚म् । अधि॑पतय॒ इत्यधि॑ - प॒त॒यः॒ । ते । मा॒ । अ॒व॒न्तु॒ । पित॑रः । पि॒ता॒म॒हाः॒ । प॒रे॒ । अ॒व॒रे॒ ( ) । तताः᳚ । त॒ता॒म॒हाः॒ । इ॒ह । मा॒ । अ॒व॒त॒ ॥ अ॒स्मिन्न् । ब्रह्मन्न्॑ । अ॒स्मिन्न् । क्ष॒त्रे । अ॒स्याम् । आ॒शिषीत्या᳚ - शिषि॑ । अ॒स्याम् । पु॒रो॒धाया॒मिति॑ पुरः - धाया᳚म् । अ॒स्मिन्न् । कर्मन्न्॑ । अ॒स्याम् । दे॒वहू᳚त्या॒मिति॑ दे॒व - हू॒त्या॒म् ॥ \textbf{  16} \newline
                  \newline
                      (अ॒व॒रे॒ - स॒प्तद॑श च)  \textbf{(A5)} \newline \newline
                                \textbf{ TS 3.4.6.1} \newline
                  दे॒वाः । वै । यत् । य॒ज्ञे । अकु॑र्वत । तत् । असु॑राः । अ॒कु॒र्व॒त॒ । ते । दे॒वाः । ए॒तान् । अ॒भ्या॒ता॒नानित्य॑भि - आ॒ता॒नान् । अ॒प॒श्य॒न्न् । तान् । अ॒भ्यात॑न्व॒तेत्य॑भि - आत॑न्वत । यत् । दे॒वाना᳚म् । कर्म॑ । आसी᳚त् । आर्द्ध्य॑त । तत् । यत् । असु॑राणाम् । न । तत् । आ॒र्द्ध्य॒त॒ । येन॑ । कर्म॑णा । ईर्थ्से᳚त् । तत्र॑ । हो॒त॒व्याः᳚ । ऋ॒द्ध्नोति॑ । ए॒व । तेन॑ । कर्म॑णा । यत् । विश्वे᳚ । दे॒वाः । स॒मभ॑र॒न्निति॑ सं - अभ॑रन्न् । तस्मा᳚त् । अ॒भ्या॒ता॒ना इत्य॑भि - आ॒ता॒नाः । वै॒श्व॒दे॒वा इति॑ वैश्व - दे॒वाः । यत् । प्र॒जाप॑ति॒रिति॑ प्र॒जा - प॒तिः॒ । जयान्॑ । प्रेति॑ । अय॑च्छत् । तस्मा᳚त् । जयाः᳚ । प्रा॒जा॒प॒त्या इति॑ प्राजा-प॒त्याः । \textbf{  17} \newline
                  \newline
                                \textbf{ TS 3.4.6.2} \newline
                  यत् । रा॒ष्ट्र॒भृद्भि॒रिति॑ राष्ट्र॒भृत् - भिः॒ । रा॒ष्ट्रम् । एति॑ । अद॑दत । तत् । रा॒ष्ट्र॒भृता॒मिति॑ राष्ट्र - भृता᳚म् । रा॒ष्ट्र॒भृ॒त्त्वमिति॑ राष्ट्रभृत् - त्वम् । ते । दे॒वाः । अ॒भ्या॒ता॒नैरित्य॑भि - आ॒ता॒नैः । असु॑रान् । अ॒भ्यात॑न्व॒तेत्य॑भि - आत॑न्वत । जयैः᳚ । अ॒ज॒य॒न् । रा॒ष्ट्र॒भृद्भि॒रिति॑ राष्ट्र॒भृत् - भिः॒ । रा॒ष्ट्रम् । एति॑ । अ॒द॒द॒त॒ । यत् । दे॒वाः । अ॒भ्या॒ता॒नैरित्य॑भि - आ॒ता॒नैः । असु॑रान् । अ॒भ्यात॑न्व॒तेत्य॑भि - आत॑न्वत । तत् । अ॒भ्या॒ता॒नाना॒मित्य॑भि - आ॒ता॒नाना᳚म् । अ॒भ्या॒ता॒न॒त्वमित्य॑भ्यातान - त्वम् । यत् । जयैः᳚ । अज॑यन्न् । तत् । जया॑नाम् । ज॒य॒त्वमिति॑ जय - त्वम् । यत् । रा॒ष्ट्र॒भृद्भि॒रिति॑ राष्ट्र॒भृत् - भिः॒ । रा॒ष्ट्रम् । एति॑ । अद॑दत । तत् । रा॒ष्ट्र॒भृता॒मिति॑ राष्ट्र - भृता᳚म् । रा॒ष्ट्र॒भृ॒त्त्वमिति॑ राष्ट्रभृत् - त्वम् । ततः॑ । दे॒वाः । अभ॑वन्न् । परेति॑ । असु॑राः । यः । भ्रातृ॑व्यवा॒निति॒ भ्रातृ॑व्य - वा॒न् । स्यात् । सः ( ) । ए॒तान् । जु॒हु॒या॒त् । अ॒भ्या॒ता॒नैरित्य॑भि-आ॒ता॒नैः । ए॒व । भ्रातृ॑व्यान् । अ॒भ्यात॑नुत॒ इत्य॑भि - आत॑नुते । जयैः᳚ । ज॒य॒ति॒ । रा॒ष्ट्र॒भृद्भि॒रिति॑ राष्ट्र॒भृत् - भिः॒ । रा॒ष्ट्रम् । एति॑ । द॒त्ते॒ । भव॑ति । आ॒त्मना᳚ । परेति॑ । अ॒स्य॒ । भ्रातृ॑व्यः । भ॒व॒ति॒ ॥ \textbf{  18 } \newline
                  \newline
                      (प्रा॒जा॒प॒त्याः-सो᳚ऽ-ष्टा द॑श च)  \textbf{(A6)} \newline \newline
                                \textbf{ TS 3.4.7.1} \newline
                  ऋ॒ता॒षाट् । ऋ॒तधा॒मेत्यृ॒त - धा॒मा॒ । अ॒ग्निः । ग॒न्ध॒र्वः । तस्य॑ । ओष॑धयः । अ॒फ्स॒रसः॑ । ऊर्जः॑ । नाम॑ । सः । इ॒दम् । ब्रह्म॑ । क्ष॒त्रम् । पा॒तु॒ । ताः । इ॒दम् । ब्रह्म॑ । क्ष॒त्रम् । पा॒न्तु॒ । तस्मै᳚ । स्वाहा᳚ । ताभ्यः॑ । स्वाहा᳚ । सꣳ॒॒हि॒त इति॑ सं - हि॒तः । वि॒श्वसा॒मेति॑ वि॒श्व - सा॒मा॒ । सूर्यः॑ । ग॒न्ध॒र्वः । तस्य॑ । मरी॑चयः । अ॒फ्स॒रसः॑ । आ॒युव॒ इत्या᳚ - युवः॑ । सु॒षु॒म्न इति॑ सु - सु॒म्नः । सूर्य॑रश्मि॒रिति॒ सूर्य॑-र॒श्मिः॒ । च॒न्द्रमाः᳚ । ग॒न्ध॒र्वः । तस्य॑ । नक्ष॑त्राणि । अ॒फ्स॒रसः॑ । बे॒कुर॑यः । भु॒ज्युः । सु॒प॒र्ण इति॑ सु - प॒र्णः । य॒ज्ञ्ः । ग॒न्ध॒र्वः । तस्य॑ । दक्षि॑णाः । अ॒फ्स॒रसः॑ । स्त॒वाः । प्र॒जाप॑ति॒रिति॑ प्र॒जा-प॒तिः॒ । वि॒श्वक॒र्मेति॑ वि॒श्व - क॒र्मा॒ । मनः॑ । \textbf{  19} \newline
                  \newline
                                \textbf{ TS 3.4.7.2} \newline
                  ग॒न्ध॒र्वः । तस्य॑ । ऋ॒ख्सा॒मानीत्यृ॑क्-सा॒मानि॑ । अ॒फ्स॒रसः॑ । वह्न॑यः । इ॒षि॒रः । वि॒श्वव्य॑चा॒ इति॑ वि॒श्व - व्य॒चाः॒ । वातः॑ । ग॒न्ध॒र्वः । तस्य॑ । आपः॑ । अ॒फ्स॒रसः॑ । मु॒दाः । भुव॑नस्य । प॒ते॒ । यस्य॑ । ते॒ । उ॒परि॑ । गृ॒हाः । इ॒ह । च॒ ॥ सः । नः॒ । रा॒स्व॒ । अज्या॑निम् । रा॒यः । पोष᳚म् । सु॒वीर्य॒मिति॑ सु - वीर्य᳚म् । सं॒ॅव॒थ्स॒रीणा॒मिति॑ सं - व॒थ्स॒रीणा᳚म् । स्व॒स्तिम् ॥ प॒र॒मे॒ष्ठी । अधि॑प॒तिरित्यधि॑ - प॒तिः॒ । मृ॒त्युः । ग॒न्ध॒र्वः । तस्य॑ । विश्व᳚म् । अ॒फ्स॒रसः॑ । भुवः॑ । सु॒क्षि॒तिरिति॑ सु - क्षि॒तिः । सुभू॑ति॒रिति॒ सु - भू॒तिः॒ । भ॒द्र॒कृदिति॑ भद्र - कृत् । सुव॑र्वा॒निति॒ सुवः॑ - वा॒न् । प॒र्जन्यः॑ । ग॒न्ध॒र्वः । तस्य॑ । वि॒द्युत॒ इति॑ वि-द्युतः॑ । अ॒फ्स॒रसः॑ । रुचः॑ । दू॒रेहे॑ति॒रिति॑ दू॒रे - हे॒तिः॒ । अ॒मृ॒ड॒यः । \textbf{  20} \newline
                  \newline
                                \textbf{ TS 3.4.7.3} \newline
                  मृ॒त्युः । ग॒न्ध॒र्वः । तस्य॑ । प्र॒जा इति॑ प्र - जाः । अ॒फ्स॒रसः॑ । भी॒रुवः॑ । चारुः॑ । कृ॒प॒ण॒का॒शीति॑ कृपण-का॒शी । कामः॑ । ग॒न्ध॒र्वः । तस्य॑ । आ॒धय॒ इत्या᳚ - धयः॑ । अ॒फ्स॒रसः॑ । शो॒चय॑न्तीः । नाम॑ । सः । इ॒दम् । ब्रह्म॑ । क्ष॒त्रम् । पा॒तु॒ । ताः । इ॒दम् । ब्रह्म॑ । क्ष॒त्रम् । पा॒न्तु॒ । तस्मै᳚ । स्वाहा᳚ । ताभ्यः॑ । स्वाहा᳚ । सः । नः॒ । भु॒व॒न॒स्य॒ । प॒ते॒ । यस्य॑ । ते॒ । उ॒परि॑ । गृ॒हाः । इ॒ह । च॒ ॥ उ॒रु । ब्रह्म॑णे । अ॒स्मै । क्ष॒त्राय॑ । महि॑ । शर्म॑ । य॒च्छ॒ ॥ \textbf{  21} \newline
                  \newline
                      (मनो॑ - ऽमृड॒यः - षट्च॑त्वारिꣳशच्च)  \textbf{(A7)} \newline \newline
                                \textbf{ TS 3.4.8.1} \newline
                  रा॒ष्ट्रका॑मा॒येति॑ रा॒ष्ट्र - का॒मा॒य॒ । हो॒त॒व्याः᳚ । रा॒ष्ट्रम् । वै । रा॒ष्ट्र॒भृत॒ इति॑ राष्ट्र - भृतः॑ । रा॒ष्ट्रेण॑ । ए॒व । अ॒स्मै॒ । रा॒ष्ट्रम् । अवेति॑ । रु॒न्धे॒ । रा॒ष्ट्रम् । ए॒व । भ॒व॒ति॒ । आ॒त्मने᳚ । हो॒त॒व्याः᳚ । रा॒ष्ट्रम् । वै । रा॒ष्ट्र॒भृत॒ इति॑ राष्ट्र - भृतः॑ । रा॒ष्ट्रम् । प्र॒जेति॑ प्र - जा । रा॒ष्ट्रम् । प॒शवः॑ । रा॒ष्ट्रम् । यत् । श्रेष्ठः॑ । भव॑ति । रा॒ष्ट्रेण॑ । ए॒व । रा॒ष्ट्रम् । अवेति॑ । रु॒न्धे॒ । वसि॑ष्ठः । स॒मा॒नाना᳚म् । भ॒व॒ति॒ । ग्राम॑कामा॒येति॒ ग्राम॑ - का॒मा॒य॒ । हो॒त॒व्याः᳚ । रा॒ष्ट्रम् । वै । रा॒ष्ट्र॒भृत॒ इति॑ राष्ट्र-भृतः॑ । रा॒ष्ट्रम् । स॒जा॒ता इति॑ स - जा॒ताः । रा॒ष्ट्रेण॑ । ए॒व । अ॒स्मै॒ । रा॒ष्ट्रम् । स॒जा॒तानिति॑ स - जा॒तान् । अवेति॑ । रु॒न्धे॒ । ग्रा॒मी । \textbf{  22} \newline
                  \newline
                                \textbf{ TS 3.4.8.2} \newline
                  ए॒व । भ॒व॒ति॒ । अ॒धि॒देव॑न॒ इत्य॑धि - देव॑ने । जु॒हो॒ति॒ । अ॒धि॒देव॑न॒ इत्य॑धि - देव॑ने । ए॒व । अ॒स्मै॒ । स॒जा॒तानिति॑ स-जा॒तान् । अवेति॑ । रु॒न्धे॒ । ते । ए॒न॒म् । अव॑रुद्धा॒ इत्यव॑ - रु॒द्धाः॒ । उपेति॑ । ति॒ष्ठ॒न्ते॒ । र॒थ॒मु॒ख इति॑ रथ - मु॒खे । ओज॑स्काम॒स्येत्योजः॑ - का॒म॒स्य॒ । हो॒त॒व्याः᳚ । ओजः॑ । वै । रा॒ष्ट्र॒भृत॒ इति॑ राष्ट्र-भृतः॑ । ओजः॑ । रथः॑ । ओज॑सा । ए॒व । अ॒स्मै॒ । ओजः॑ । अवेति॑ । रु॒न्धे॒ । ओ॒ज॒स्वी । ए॒व । भ॒व॒ति॒ । यः । रा॒ष्ट्रात् । अप॑भूत॒ इत्यप॑ - भू॒तः॒ । स्यात् । तस्मै᳚ । हो॒त॒व्याः᳚ । याव॑न्तः । अ॒स्य॒ । रथाः᳚ । स्युः । तान् । ब्रू॒या॒त् । यु॒ङ्ध्वम् । इति॑ । रा॒ष्ट्रम् । ए॒व । अ॒स्मै॒ । यु॒न॒क्ति॒ । \textbf{  23} \newline
                  \newline
                                \textbf{ TS 3.4.8.3} \newline
                  आहु॑तय॒ इत्या - हु॒त॒यः॒ । वै । ए॒तस्य॑ । अक्लृ॑प्ताः । यस्य॑ । रा॒ष्ट्रम् । न । कल्प॑ते । स्व॒र॒थस्येति॑ स्व - र॒थस्य॑ । दक्षि॑णम् । च॒क्रम् । प्र॒वृह्येति॑ प्र - वृह्य॑ । ना॒डीम् । अ॒भीति॑ । जु॒हु॒या॒त् । आहु॑ती॒रित्या - हु॒तीः॒ । ए॒व । अ॒स्य॒ । क॒ल्प॒य॒ति॒ । ताः । अ॒स्य॒ । कल्प॑मानाः । रा॒ष्ट्रम् । अन्विति॑ । क॒ल्प॒ते॒ । स॒ग्रां॒म इति॑ सं - ग्रा॒मे । संॅय॑त्त॒ इति॒ सं - य॒त्ते॒ । हो॒त॒व्याः᳚ । रा॒ष्ट्रम् । वै । रा॒ष्ट्र॒भृत॒ इति॑ राष्ट्र - भृतः॑ । रा॒ष्ट्रे । खलु॑ । वै । ए॒ते । व्याय॑च्छन्त॒ इति॑ वि - आय॑च्छन्ते । ये । स॒ग्रां॒ममिति॑ सं - ग्रा॒मम् । सं॒ॅयन्तीति॑ सं-यन्ति॑ । यस्य॑ । पूर्व॑स्य । जुह्व॑ति । सः । ए॒व । भ॒व॒ति॒ । जय॑ति । तम् । स॒ग्रां॒ममिति॑ सं - ग्रा॒मम् । मा॒न्धु॒कः । इ॒द्ध्मः । \textbf{  24} \newline
                  \newline
                                \textbf{ TS 3.4.8.4} \newline
                  भ॒व॒ति॒ । अङ्गा॑राः । ए॒व । प्र॒ति॒वेष्ट॑माना॒ इति॑ प्रति - वेष्ट॑मानाः । अ॒मित्रा॑णाम् । अ॒स्य॒ । सेना᳚म् । प्रतीति॑ । वे॒ष्ट॒य॒न्ति॒ । यः । उ॒न्माद्ये॒दित्यु॑त् - माद्ये᳚त् । तस्मै᳚ । हो॒त॒व्याः᳚ । ग॒न्ध॒र्वा॒फ्स॒रस॒ इति॑ गन्धर्व - अ॒फ्स॒रसः॑ । वै । ए॒तम् । उदिति॑ । मा॒द॒य॒न्ति॒ । यः । उ॒न्माद्य॒तीत्यु॑त् - माद्य॑ति । ए॒ते । खलु॑ । वै । ग॒न्ध॒र्वा॒फ्स॒रस॒ इति॑ गन्धर्व - अ॒फ्स॒रसः॑ । यत् । रा॒ष्ट्र॒भृत॒ इति॑ राष्ट्र - भृतः॑ । तस्मै᳚ । स्वाहा᳚ । ताभ्यः॑ । स्वाहा᳚ । इति॑ । जु॒हो॒ति॒ । तेन॑ । ए॒व । ए॒ना॒न् । श॒म॒य॒ति॒ । नैय॑ग्रोधः । औदु॑बंरः । आश्व॑त्थः । प्लाक्षः॑ । इति॑ । इ॒द्ध्मः । भ॒व॒ति॒ । ए॒ते । वै । ग॒न्ध॒र्वा॒फ्स॒रसा॒मिति॑ गन्धर्व - अ॒फ्स॒रसा᳚म् । गृ॒हाः । स्वे । ए॒व । ए॒ना॒न् । \textbf{  25} \newline
                  \newline
                                \textbf{ TS 3.4.8.5} \newline
                  आ॒यत॑न॒ इत्या᳚ - यत॑ने । श॒म॒य॒ति॒ । अ॒भि॒चर॒तेत्य॑भि - चर॑ता । प्र॒ति॒लो॒ममिति॑ प्रति - लो॒मम् । हो॒त॒व्याः᳚ । प्रा॒णानिति॑ प्र - अ॒नान् । ए॒व । अ॒स्य॒ । प्र॒तीचः॑ । प्रतीति॑ । यौ॒ति॒ । तम् । ततः॑ । येन॑ । केन॑ । च॒ । स्तृ॒णु॒ते॒ । स्वकृ॑त॒ इति॒ स्व - कृ॒ते॒ । इरि॑णे । जु॒हो॒ति॒ । प्र॒द॒र इति॑ प्र - द॒रे । वा॒ । ए॒तत् । वै । अ॒स्यै । निर्.ऋ॑तिगृहीत॒मिति॒ निर्.ऋ॑ति - गृ॒ही॒त॒म् । निर्.ऋ॑तिगृहीत॒ इति॒ निर्.ऋ॑ति-गृ॒ही॒ते॒ । ए॒व । ए॒न॒म् । निर्.ऋ॒त्येति॒ निः-ऋ॒त्या॒ । ग्रा॒ह॒य॒ति॒ । यत् । वा॒चः । क्रू॒रम् । तेन॑ । वष॑ट् । क॒रो॒ति॒ । वा॒चः । ए॒व । ए॒न॒म् । क्रू॒रेण॑ । प्रेति॑ । वृ॒श्च॒ति॒ । ता॒जक् । आर्ति᳚म् । एति॑ । ऋ॒च्छ॒ति॒ । यस्य॑ । का॒मये॑त । अ॒न्नाद्य॒मित्य॑न्न - अद्य᳚म् । \textbf{  26} \newline
                  \newline
                                \textbf{ TS 3.4.8.6} \newline
                  एति॑ । द॒दी॒य॒ । इति॑ । तस्य॑ । स॒भाया᳚म् । उ॒त्ता॒न इत्यु॑त् - ता॒नः । नि॒पद्येति॑ नि - पद्य॑ । भुव॑नस्य । प॒ते॒ । इति॑ । तृणा॑नि । समिति॑ । गृ॒ह्णी॒या॒त् । प्र॒जाप॑ति॒रिति॑ प्र॒जा - प॒तिः॒ । वै । भुव॑नस्य । पतिः॑ । प्र॒जाप॑ति॒नेति॑ प्र॒जा - प॒ति॒ना॒ । ए॒व । अ॒स्य॒ । अ॒न्नाद्य॒मित्य॑न्न - अद्य᳚म् । एति॑ । द॒त्ते॒ । इ॒दम् । अ॒हम् । अ॒मुष्य॑ । आ॒मु॒ष्या॒य॒णस्य॑ । अ॒न्नाद्य॒मित्य॑न्न - अद्य᳚म् । ह॒रा॒मि॒ । इति॑ । आ॒ह॒ । अ॒न्नाद्य॒मित्य॑न्न - अद्य᳚म् । ए॒व । अ॒स्य॒ । ह॒र॒ति॒ । ष॒ड्भिरिति॑ षट् - भिः । ह॒र॒ति॒ । षट् । वै । ऋ॒तवः॑ । प्र॒जाप॑ति॒नेति॑ प्र॒जा-प॒ति॒ना॒ । ए॒व । अ॒स्य॒ । अ॒न्नाद्य॒मित्य॑न्न -अद्य᳚म् । आ॒दायेत्या᳚-दाय॑ । ऋ॒तवः॑ । अ॒स्मै॒ । अनु॑ । प्रेति॑ । य॒च्छ॒न्ति॒ । \textbf{  27} \newline
                  \newline
                                \textbf{ TS 3.4.8.7} \newline
                  यः । ज्ये॒ष्ठब॑न्धु॒रिति॑ ज्ये॒ष्ठ - ब॒न्धुः॒ । अप॑भूत॒ इत्यप॑ - भू॒तः॒ । स्यात् । तम् । स्थले᳚ । अ॒व॒साय्येत्य॑व - साय्य॑ । ब्र॒ह्मौ॒द॒नमिति॑ ब्रह्म - ओ॒द॒नम् । चतुः॑ शराव॒मिति॒ चतुः॑ - श॒रा॒व॒म् । प॒क्त्वा । तस्मै᳚ । हो॒त॒व्याः᳚ । वर्ष्म॑ । वै । रा॒ष्ट्र॒भृत॒ इति॑ राष्ट्र - भृतः॑ । वर्ष्म॑ । स्थल᳚म् । वर्ष्म॑णा । ए॒व । ए॒न॒म् । वर्ष्म॑ । स॒मा॒नाना᳚म् । ग॒म॒य॒ति॒ । चतुः॑ शराव॒ इति॒ चतुः॑ - श॒रा॒वः॒ । भ॒व॒ति॒ । दि॒क्षु । ए॒व । प्रतीति॑ । ति॒ष्ठ॒ति॒ । क्षी॒रे । भ॒व॒ति॒ । रुच᳚म् । ए॒व । अ॒स्मि॒न्न् । द॒धा॒ति॒ । उदिति॑ । ह॒र॒ति॒ । शृ॒त॒त्वायेति॑ शृत - त्वाय॑ । स॒र्पिष्वान्॑ । भ॒व॒ति॒ । मे॒द्ध्य॒त्वायेति॑ मेद्ध्य-त्वाय॑ । च॒त्वारः॑ । आ॒र्॒.षे॒याः । प्रेति॑ । अ॒श्न॒न्ति॒ । दि॒शाम् । ए॒व । ज्योति॑षि । जु॒हो॒ति॒ ॥ \textbf{  28} \newline
                  \newline
                      (ग्रा॒मी - यु॑नक्ती॒ - ध्मः - स्व ए॒वैना॑ - न॒न्नाद्यं॑ - ॅयच्छ॒न्त्ये - का॒न्न प॑ञ्चा॒शच्च॑)  \textbf{(A8)} \newline \newline
                                \textbf{ TS 3.4.9.1} \newline
                  देवि॑काः । निरिति॑ । व॒पे॒त् । प्र॒जाका॑म॒ इति॑ प्र॒जा - का॒मः॒ । छन्दाꣳ॑सि । वै । देवि॑काः । छन्दाꣳ॑सि । इ॒व॒ । खलु॑ । वै । प्र॒जा इति॑ प्र-जाः । छन्दो॑भि॒रिति॒ छन्दः॑ - भिः॒ । ए॒व । अ॒स्मै॒ । प्र॒जा इति॑ प्र-जाः । प्रेति॑ । ज॒न॒य॒ति॒ । प्र॒थ॒मम् । धा॒तार᳚म् । क॒रो॒ति॒ । मि॒थु॒नी । ए॒व । तेन॑ । क॒रो॒ति॒ । अन्विति॑ । ए॒व । अ॒स्मै॒ । अनु॑मति॒रित्य॑नु - म॒तिः॒ । म॒न्य॒ते॒ । रा॒ते । रा॒का । प्रेति॑ । सि॒नी॒वा॒ली । ज॒न॒य॒ति॒ । प्र॒जास्विति॑ प्र - जासु॑ । ए॒व । प्रजा॑ता॒स्विति॒ प्र - जा॒ता॒सु॒ । कु॒ह्वा᳚ । वाच᳚म् । द॒धा॒ति॒ । ए॒ताः । ए॒व । निरिति॑ । व॒पे॒त् । प॒शुका॑म॒ इति॑ प॒शु - का॒मः॒ । छन्दाꣳ॑सि । वै । देवि॑काः । छन्दाꣳ॑सि । \textbf{  29} \newline
                  \newline
                                \textbf{ TS 3.4.9.2} \newline
                  इ॒व॒ । खलु॑ । वै । प॒शवः॑ । छन्दो॑भि॒रिति॒ छन्दः॑ - भिः॒ । ए॒व । अ॒स्मै॒ । प॒शून् । प्रेति॑ । ज॒न॒य॒ति॒ । प्र॒थ॒मम् । धा॒तार᳚म् । क॒रो॒ति॒ । प्रेति॑ । ए॒व । तेन॑ । वा॒प॒य॒ति॒ । अन्विति॑ । ए॒व । अ॒स्मै॒ । अनु॑मति॒रित्य॑नु - म॒तिः॒ । म॒न्य॒ते॒ । रा॒ते । रा॒का । प्रेति॑ । सि॒नी॒वा॒ली । ज॒न॒य॒ति॒ । प॒शून् । ए॒व । प्रजा॑ता॒निति॒ प्र-जा॒ता॒न् । कु॒ह्वा᳚ । प्रतीति॑ । स्था॒प॒य॒ति॒ । ए॒ताः । ए॒व । निरिति॑ । व॒पे॒त् । ग्राम॑काम॒ इति॒ ग्राम॑ - का॒मः॒ । छन्दाꣳ॑सि । वै । देवि॑काः । छन्दाꣳ॑सि । इ॒व॒ । खलु॑ । वै । ग्रामः॑ । छन्दो॑भि॒रिति॒ छन्दः॑ - भिः॒ । ए॒व । अ॒स्मै॒ । ग्राम᳚म् । \textbf{  30} \newline
                  \newline
                                \textbf{ TS 3.4.9.3} \newline
                  अवेति॑ । रु॒न्धे॒ । म॒द्ध्य॒तः । धा॒तार᳚म् । क॒रो॒ति॒ । म॒द्ध्य॒तः । ए॒व । ए॒न॒म् । ग्राम॑स्य । द॒धा॒ति॒ । ए॒ताः । ए॒व । निरिति॑ । व॒पे॒त् । ज्योगा॑मया॒वीति॒ ज्योक् - आ॒म॒या॒वी॒ । छन्दाꣳ॑सि । वै । देवि॑काः । छन्दाꣳ॑सि । खलु॑ । वै । ए॒तम् । अ॒भीति॑ । म॒न्य॒न्ते॒ । यस्य॑ । ज्योक् । आ॒मय॑ति । छन्दो॑भि॒रिति॒ छन्दः॑ - भिः॒ । ए॒व । ए॒न॒म् । अ॒ग॒दम् । क॒रो॒ति॒ । म॒द्ध्य॒तः । धा॒तार᳚म् । क॒रो॒ति॒ । म॒द्ध्य॒तः । वै । ए॒तस्य॑ । अक्लृ॑प्तम् । यस्य॑ । ज्योक् । आ॒मय॑ति । म॒द्ध्य॒तः । ए॒व । अ॒स्य॒ । तेन॑ । क॒ल्प॒य॒ति॒ । ए॒ताः । ए॒व । निरिति॑ । \textbf{  31} \newline
                  \newline
                                \textbf{ TS 3.4.9.4} \newline
                  व॒पे॒त् । यम् । य॒ज्ञ्ः । न । उ॒प॒नमे॒दित्यु॑प - नमे᳚त् । छन्दाꣳ॑सि । वै । देवि॑काः । छन्दाꣳ॑सि । खलु॑ । वै । ए॒तम् । न । उपेति॑ । न॒म॒न्ति॒ । यम् । य॒ज्ञ्ः । न । उ॒प॒नम॒तीत्यु॑प - नम॑ति । प्र॒थ॒मम् । धा॒तार᳚म् । क॒रो॒ति॒ । मु॒ख॒तः । ए॒व । अ॒स्मै॒ । छन्दाꣳ॑सि । द॒धा॒ति॒ । उपेति॑ । ए॒न॒म् । य॒ज्ञ्ः । न॒म॒ति॒ । ए॒ताः । ए॒व । निरिति॑ । व॒पे॒त् । ई॒जा॒नः । छन्दाꣳ॑सि । वै । देवि॑काः । या॒तया॑मा॒नीति॑ या॒त - या॒मा॒नि॒ । इ॒व॒ । खलु॑ । वै । ए॒तस्य॑ । छन्दाꣳ॑सि । यः । ई॒जा॒नः । उ॒त्त॒ममित्यु॑त् - त॒मम् । धा॒तार᳚म् । क॒रो॒ति॒ । \textbf{  32} \newline
                  \newline
                                \textbf{ TS 3.4.9.5} \newline
                  उ॒परि॑ष्टात् । ए॒व । अ॒स्मै॒ । छन्दाꣳ॑सि । अया॑तयामा॒नीत्यया॑त - या॒मा॒नि॒ । अवेति॑ । रु॒न्धे॒ । उपेति॑ । ए॒न॒म् । उत्त॑र॒ इत्युत् - त॒रः॒ । य॒ज्ञ्ः । न॒म॒ति॒ । ए॒ताः । ए॒व । निरिति॑ । व॒पे॒त् । यम् । मे॒धा । न । उ॒प॒नमे॒दित्यु॑प - नमे᳚त् । छन्दाꣳ॑सि । वै । देवि॑काः । छन्दाꣳ॑सि । खलु॑ । वै । ए॒तम् । न । उपेति॑ । न॒म॒न्ति॒ । यम् । मे॒धा । न । उ॒प॒नम॒तीत्यु॑प - नम॑ति । प्र॒थ॒मम् । धा॒तार᳚म् । क॒रो॒ति॒ । मु॒ख॒तः । ए॒व । अ॒स्मै॒ । छन्दाꣳ॑सि । द॒धा॒ति॒ । उपेति॑ । ए॒न॒म् । मे॒धा । न॒म॒ति॒ । ए॒ताः । ए॒व । निरिति॑ । व॒पे॒त् । \textbf{  33} \newline
                  \newline
                                \textbf{ TS 3.4.9.6} \newline
                  रुक्का॑म॒ इति॒ रुक्-का॒मः॒ । छन्दाꣳ॑सि । वै । देवि॑काः । छन्दाꣳ॑सि । इ॒व॒ । खलु॑ । वै । रुक् । छन्दो॑भि॒रिति॒ छन्दः॑-भिः॒ । ए॒व । अ॒स्मि॒न्न् । रुच᳚म् । द॒धा॒ति॒ । क्षी॒रे । भ॒व॒न्ति॒ । रुच᳚म् । ए॒व । अ॒स्मि॒न्न् । द॒ध॒ति॒ । म॒द्ध्य॒तः । धा॒तार᳚म् । क॒रो॒ति॒ । म॒द्ध्य॒तः । ए॒व । ए॒न॒म् । रु॒चः । द॒धा॒ति॒ । गा॒य॒त्री । वै । अनु॑मति॒रित्य॑नु - म॒तिः॒ । त्रि॒ष्टुक् । रा॒का । जग॑ती । सि॒नी॒वा॒ली । अ॒नु॒ष्टुबित्य॑नु - स्तुप् । कु॒हूः । धा॒ता । व॒ष॒ट्का॒र इति॑ वषट् - का॒रः । पू॒र्व॒प॒क्ष इति॑ पूर्व - प॒क्षः । रा॒का । अ॒प॒र॒प॒क्ष इत्य॑पर - प॒क्षः । कु॒हूः । अ॒मा॒वा॒स्येत्य॑मा - वा॒स्या᳚ । सि॒नी॒वा॒ली । पौ॒र्ण॒मा॒सीति॑ पौर्ण - मा॒सी । अनु॑मति॒रित्य॑नु - म॒तिः॒ । च॒न्द्रमाः᳚ । धा॒ता । अ॒ष्टौ । \textbf{  34} \newline
                  \newline
                                \textbf{ TS 3.4.9.7} \newline
                  वस॑वः । अ॒ष्टाक्ष॒रेत्य॒ष्टा - अ॒क्ष॒रा॒ । गा॒य॒त्री । एका॑दश । रु॒द्राः । एका॑दशाक्ष॒रेत्येका॑दश - अ॒क्ष॒रा॒ । त्रि॒ष्टुप् । द्वाद॑श । आ॒दि॒त्याः । द्वाद॑शाक्ष॒रेति॒ द्वाद॑श-अ॒क्ष॒रा॒ । जग॑ती । प्र॒जाप॑ति॒रिति॑ प्र॒जा - प॒तिः॒ । अ॒नु॒ष्टुबित्य॑नु-स्तुप् । धा॒ता । व॒ष॒ट्का॒र इति॑ वषट् - का॒रः । ए॒तत् । वै । देवि॑काः । सर्वा॑णि । च॒ । छन्दाꣳ॑सि । सर्वाः᳚ । च॒ । दे॒वताः᳚ । व॒ष॒ट्का॒र इति॑ वषट् - का॒रः । ताः । यत् । स॒ह । सर्वाः᳚ । नि॒र्वपे॒दिति॑ निः - वपे᳚त् । ई॒श्व॒राः । ए॒न॒म् । प्र॒दह॒ इति॑ प्र - दहः॑ । द्वे इति॑ । प्र॒थ॒मे इति॑ । नि॒रुप्येति॑ निः - उप्य॑ । धा॒तुः । तृ॒तीय᳚म् । निरिति॑ । व॒पे॒त् । तथो॒ इति॑ । ए॒व । उत्त॑रे॒ इत्युत् - त॒रे॒ । निरिति॑ । व॒पे॒त् । तथा᳚ । ए॒न॒म् । न । प्रेति॑ । द॒ह॒न्ति॒ ( ) । अथो॒ इति॑ । यस्मै᳚ । कामा॑य । नि॒रु॒प्यन्त॒ इति॑ निः - उ॒प्यन्ते᳚ । तम् । ए॒व । आ॒भिः॒ । उपेति॑ । आ॒प्नो॒ति॒ ॥ \textbf{  35} \newline
                  \newline
                      (प॒शुका॑म॒श्छन्दाꣳ॑सि॒ वै देवि॑का॒श्छन्दाꣳ॑सि॒-ग्रामं॑-कल्पयत्ये॒ता ए॒व नि-रु॑त्त॒मन्धा॒तारं॑ करोति - मे॒धा न॑मत्ये॒ता ए॒व निर्व॑पे - द॒ष्टौ - द॑हन्ति॒ - नव॑ च)  \textbf{(A9)} \newline \newline
                                \textbf{ TS 3.4.10.1} \newline
                  वास्तोः᳚ । प॒ते॒ । प्रतीति॑ । जा॒नी॒हि॒ । अ॒स्मान् । स्वा॒वे॒श इति॑ सु-आ॒वे॒शः । अ॒न॒मी॒वः । भ॒व॒ । नः॒ ॥ यत् । त्वा॒ । ईम॑हे । प्रतीति॑ । तत् । नः॒ । जु॒ष॒स्व॒ । शम् । नः॒ । ए॒धि॒ । द्वि॒पद॒ इति॑ द्वि - पदे᳚ । शम् । चतु॑ष्पद॒ इति॒ चतुः॑ - प॒दे॒ ॥ वास्तोः᳚ । प॒ते॒ । श॒ग्मया᳚ । सꣳ॒॒सदेति॑ सं - सदा᳚ । ते॒ । स॒क्षी॒महि॑ । र॒ण्वया᳚ । गा॒तु॒मत्येति॑ गातु - मत्या᳚ ॥ आवः॑ । क्षेमे᳚ । उ॒त । योगे᳚ । वर᳚म् । नः॒ । यू॒यम् । पा॒त॒ । स्व॒स्तिभि॒रिति॑ स्व॒स्ति - भिः॒ । सदा᳚ । नः॒ ॥ यत् । सा॒यंप्रा॑त॒रिति॑ सा॒यं - प्रा॒तः॒ । अ॒ग्नि॒हो॒त्रमित्य॑ग्नि - हो॒त्रम् । जु॒होति॑ । आ॒हु॒ती॒ष्ट॒का इत्या॑हुति - इ॒ष्ट॒काः । ए॒व । ताः । उपेति॑ । ध॒त्ते॒ । \textbf{  36} \newline
                  \newline
                                \textbf{ TS 3.4.10.2} \newline
                  यज॑मानः । अ॒हो॒रा॒त्राणीत्य॑हः-रा॒त्राणि॑ । वै । ए॒तस्य॑ । इष्ट॑काः । यः । आहि॑ताग्नि॒रित्याहि॑त - अ॒ग्निः॒ । यत् । सा॒यंप्रा॑त॒रिति॑ सा॒यं-प्रा॒तः॒ । जु॒होति॑ । अ॒हो॒रा॒त्राणीत्य॑हः - रा॒त्राणि॑ । ए॒व । आ॒प्त्वा । इष्ट॑काः । कृ॒त्वा । उपेति॑ । ध॒त्ते॒ । दश॑ । स॒मा॒नत्र॑ । जु॒हो॒ति॒ । दशा᳚क्ष॒रेति॒ दश॑ - अ॒क्ष॒रा॒ । वि॒राडिति॑ वि - राट् । वि॒राज॒मिति॑ वि - राज᳚म् । ए॒व । आ॒प्त्वा । इष्ट॑काम् । कृ॒त्वा । उपेति॑ । ध॒त्ते॒ । अथो॒ इति॑ । वि॒राजीति॑ वि - राजि॑ । ए॒व । य॒ज्ञ्म् । आ॒प्नो॒ति॒ । चित्य॑श्चित्य॒ इति॒ चित्यः॑ - चि॒त्यः॒ । अ॒स्य॒ । भ॒व॒ति॒ । तस्मा᳚त् । यत्र॑ । दश॑ । उ॒षि॒त्वा । प्र॒यातीति॑ प्र - याति॑ । तत् । य॒ज्ञ्॒वा॒स्त्विति॑ यज्ञ्-वा॒स्तु । अवा᳚स्तु । ए॒व । तत् । यत् । ततः॑ । अ॒र्वा॒चीन᳚म् । \textbf{  37} \newline
                  \newline
                                \textbf{ TS 3.4.10.3} \newline
                  रु॒द्रः । खलु॑ । वै । वा॒स्तो॒ष्प॒तिरिति॑ वास्तोः-प॒तिः । यत् । अहु॑त्वा । वा॒स्तो॒ष्प॒तीय॒मिति॑ वास्तोः - प॒तीय᳚म् । प्र॒या॒यादिति॑ प्र - या॒यात् । रु॒द्रः । ए॒न॒म् । भू॒त्वा । अ॒ग्निः । अ॒नू॒त्थायेत्य॑नु - उ॒त्थाय॑ । ह॒न्या॒त् । वा॒स्तो॒ष्प॒तीय॒मिति॑ वास्तोः - प॒तीय᳚म् । जु॒हो॒ति॒ । भा॒ग॒धेये॒नेति॑ भाग - धेये॑न । ए॒व । ए॒न॒म् । श॒म॒य॒ति॒ । न । आर्ति᳚म् । एति॑ । ऋ॒च्छ॒ति॒ । यज॑मानः । यत् । यु॒क्ते । जु॒हु॒यात् । यथा᳚ । प्रया॑त॒ इति॒ प्र - या॒ते॒ । वास्तौ᳚ । आहु॑ति॒मित्या-हु॒ति॒म् । जु॒होति॑ । ता॒दृक् । ए॒व । तत् । यत् । अयु॑क्ते । जु॒हु॒यात् । यथा᳚ । क्षेमे᳚ । आहु॑ति॒मित्या - हु॒ति॒म् । जु॒होति॑ । ता॒दृक् । ए॒व । तत् । अहु॑तम् । अ॒स्य॒ । वा॒स्तो॒ष्प॒तीय॒मिति॑ वास्तोः - प॒तीय᳚म् । स्यात् । \textbf{  38} \newline
                  \newline
                                \textbf{ TS 3.4.10.4} \newline
                  दक्षि॑णः । यु॒क्तः । भव॑ति । स॒व्यः । अयु॑क्तः । अथ॑ । वा॒स्तो॒ष्प॒तीय॒मिति॑ वास्तोः -प॒तीय᳚म् । जु॒हो॒ति॒ । उ॒भय᳚म् । ए॒व । अ॒कः॒ । अप॑रिवर्ग॒मित्यप॑रि - व॒र्ग॒म् । ए॒व । ए॒न॒म् । श॒म॒य॒ति॒ । यत् । एक॑या । जु॒हु॒यात् । द॒र्वि॒हो॒ममिति॑ दर्वि - हो॒मम् । कु॒र्या॒त् । पु॒रो॒नु॒वा॒क्या॑मिति॑ पुरः - अ॒नु॒वा॒क्या᳚म् । अ॒नूच्येत्य॑नु - उच्य॑ । या॒ज्य॑या । जु॒हो॒ति॒ । स॒दे॒व॒त्वायेति॑ सदेव - त्वाय॑ । यत् । हु॒ते । आ॒द॒द्ध्यादित्या᳚ - द॒द्ध्यात् । रु॒द्रम् । गृ॒हान् । अ॒न्वारो॑हये॒दित्य॑नु - आरो॑हयेत् । यत् । अ॒व॒क्षाणा॒नीत्य॑व - क्षाणा॑नि । अस॑प्रंक्षा॒प्येत्यसं᳚ - प्र॒क्षा॒प्य॒ । प्र॒या॒यादिति॑ प्र - या॒यात् । यथा᳚ । य॒ज्ञ्॒वे॒श॒समिति॑ यज्ञ् - वे॒श॒सम् । वा॒ । आ॒दह॑न॒मित्या᳚ - दह॑नम् । वा॒ । ता॒दृक् । ए॒व । तत् । अ॒यम् । ते॒ । योनिः॑ । ऋ॒त्वियः॑ । इति॑ । अ॒रण्योः᳚ । स॒मारो॑हय॒तीति॑ सं - आरो॑हयति । \textbf{  39} \newline
                  \newline
                                \textbf{ TS 3.4.10.5} \newline
                  ए॒षः । वै । अ॒ग्नेः । योनिः॑ । स्वे । ए॒व । ए॒न॒म् । योनौ᳚ । स॒मारो॑हय॒तीति॑ सं-आरो॑हयति । अथो॒ इति॑ । खलु॑ । आ॒हुः॒ । यत् । अ॒रण्योः᳚ । स॒मारू॑ढ॒ इति॑ सं-आरू॑ढः । नश्ये᳚त् । उदिति॑ । अ॒स्य॒ । अ॒ग्निः । सी॒दे॒त् । पु॒न॒रा॒धेय॒ इति॑ पुनः - आ॒धेयः॑ । स्या॒त् । इति॑ । या । ते॒ । अ॒ग्न॒ । य॒ज्ञिया᳚ । त॒नूः । तया᳚ । एति॑ । इ॒हि॒ । एति॑ । रो॒ह॒ । इति॑ । आ॒त्मन्न् । स॒मारो॑हयत॒ इति॑ सं-आरो॑हयते । यज॑मानः । वै । अ॒ग्नेः । योनिः॑ । स्वाया᳚म् । ए॒व । ए॒न॒म् । योन्या᳚म् । स॒मारो॑हयत॒ इति॑ सं - आरो॑हयते ॥ \textbf{  40} \newline
                  \newline
                      (ध॒त्ते॒-ऽर्वा॒चीनꣳ॑ -स्याथ्-स॒मारो॑हयति॒ -पञ्च॑चत्वारिꣳशच्च)  \textbf{(A10)} \newline \newline
                                \textbf{ TS 3.4.11.1} \newline
                  त्वम् । अ॒ग्ने॒ । बृ॒हत् । वयः॑ । दधा॑सि । दे॒व॒ । दा॒शुषे᳚ ॥ क॒विः । गृ॒हप॑ति॒रिति॑ गृ॒ह - प॒तिः॒ । युवा᳚ ॥ ह॒व्य॒वाडिति॑ हव्य - वाट् । अ॒ग्निः । अ॒जरः॑ । पि॒ता । नः॒ । वि॒भुरिति॑ वि - भुः । वि॒भावेति॑ वि - भावा᳚ । सु॒दृशी॑क॒ इति॑ सु - दृशी॑कः । अ॒स्मे इति॑ ॥ सु॒गा॒र्॒.ह॒प॒त्या इति॑ सु - गा॒र्॒.ह॒प॒त्याः । समिति॑ । इषः॑ । दि॒दी॒हि॒ । ॒स्म॒द्रिय॒गित्य॑स्म - द्रिय॑क् । समिति॑ । मि॒मी॒हि॒ । श्रवाꣳ॑सि ॥ त्वम् । च॒ । सो॒म॒ । नः॒ । वशः॑ । जी॒वातु᳚म् । न । म॒रा॒म॒हे॒ ॥ प्रि॒यस्तो᳚त्र॒ इति॑ प्रि॒य - स्तो॒त्रः॒ । वन॒स्पतिः॑ ॥ ब्र॒ह्मा । दे॒वाना᳚म् । प॒द॒वीरिति॑ पद - वीः । क॒वी॒नाम् । ऋषिः॑ । विप्रा॑णाम् । म॒हि॒षः । मृ॒गाणा᳚म् ॥ श्ये॒नः । गृद्ध्रा॑णाम् । स्वधि॑ति॒रिति॒ स्व - धि॒तिः॒ । वना॑नाम् । सोमः॑ । \textbf{  41} \newline
                  \newline
                                \textbf{ TS 3.4.11.2} \newline
                  प॒वित्र᳚म् । अतीति॑ । ए॒ति॒ । रेभन्न्॑ ॥ एति॑ । वि॒श्वदे॑व॒मिति॑ वि॒श्व - दे॒व॒म् । सत्प॑ति॒मिति॒ सत् - प॒ति॒म् । सू॒क्तैरिति॑ सु-उ॒क्तैः । अ॒द्य । वृ॒णी॒म॒हे॒ ॥ स॒त्यस॑व॒मिति॑ स॒त्य-स॒व॒म् । स॒वि॒तार᳚म् ॥ एति॑ । स॒त्येन॑ । रज॑सा । वर्त॑मानः । नि॒वे॒शय॒न्निति॑ नि-वे॒शयन्न्॑ । अ॒मृत᳚म् । मर्त्य᳚म् । च॒ ॥ हि॒र॒ण्यये॑न । स॒वि॒ता । रथे॑न । एति॑ । दे॒वः । या॒ति॒ । भुव॑ना । वि॒पश्य॒न्निति॑ वि-पश्यन्न्॑ ॥ यथा᳚ । नः॒ । अदि॑तिः । कर॑त् । पश्वे᳚ । नृभ्य॒ इति॒ नृ - भ्यः॒ । यथा᳚ । गवे᳚ ॥ यथा᳚ । तो॒काय॑ । रु॒द्रिय᳚म् ॥ मा । नः॒ । तो॒के । तन॑ये । मा । नः॒ । आयु॑षि । मा । नः॒ । गोषु॑ । मा । \textbf{  42} \newline
                  \newline
                                \textbf{ TS 3.4.11.3} \newline
                  नः॒ । अश्वे॑षु । री॒रि॒षः॒ ॥ वी॒रान् । मा । नः॒ । रु॒द्र॒ । भा॒मि॒तः । व॒धीः॒ । ह॒विष्म॑न्तः । नम॑सा । वि॒धे॒म॒ । ते॒ ॥ उ॒द॒प्रुत॒ इत्यु॑द - प्रुतः॑ । न । वयः॑ । रक्ष॑माणाः । वाव॑दतः । अ॒भ्रिय॑स्य । इ॒व॒ । घोषाः᳚ ॥ गि॒रि॒भ्रज॒ इति॑ गिरि - भ्रजः॑ । न । ऊ॒र्मयः॑ । मद॑न्तः । बृह॒स्पति᳚म् । अ॒भीति॑ । अ॒र्काः । अ॒ना॒व॒न्न् ॥ हꣳ॒॒सैः । इ॒व॒ । सखि॑भि॒रिति॒ सखि॑ - भिः॒ । वाव॑दद्भि॒रिति॒ वाव॑दत्- भिः॒ । अ॒श्म॒न्मया॒नीत्य॑श्मन्न् - मया॑नि । नह॑ना । व्यस्य॒न्निति॑ वि-अस्यन्न्॑ ॥ बृह॒स्पतिः॑ । अ॒भि॒कनि॑क्रद॒दित्य॑भि - कनि॑क्रदत् । गाः । उ॒त । प्रेति॑ । अ॒स्तौ॒त् । उदिति॑ । च॒ । वि॒द्वान् । अ॒गा॒य॒त् ॥ एति॑ । इ॒न्द्र॒ । सा॒न॒सिम् । र॒यिम् । \textbf{  43} \newline
                  \newline
                                \textbf{ TS 3.4.11.4} \newline
                  स॒जित्वा॑न॒मिति॑ स - जित्वा॑नम् । स॒दा॒सह॒मिति॑ सदा - सह᳚म् ॥ वर्.षि॑ष्ठम् । ऊ॒तये᳚ । भ॒र॒ ॥ प्रेति॑ । स॒सा॒हि॒षे॒ । पु॒रु॒हू॒तेति॑ पुरु-हू॒त॒ । शत्रून्॑ । ज्येष्ठः॑ । ते॒ । शुष्मः॑ । इ॒ह । रा॒तिः । अ॒स्तु॒ ॥ इन्द्र॑ । एति॑ । भ॒र॒ । दक्षि॑णेन । वसू॑नि । पतिः॑ । सिन्धू॑नाम् । अ॒सि॒ । रे॒वती॑नाम् ॥ त्वम् । सु॒तस्य॑ । पी॒तये᳚ । स॒द्यः । वृ॒द्धः । अ॒जा॒य॒थाः॒ । इन्द्र॑ । ज्यैष्ठ्या॑य । सु॒क्र॒तो॒ इति॑ सु-क्र॒तो॒ ॥ भुवः॑ । त्वम् । इ॒न्द्र॒ । ब्रह्म॑णा । म॒हान् । भुवः॑ । विश्वे॑षु । सव॑नेषु । य॒ज्ञियः॑ ॥ भुवः॑ । नॄन् । च्यौ॒त्नः । विश्व॑स्मिन्न् । भरे᳚ । ज्येष्ठः॑ । च॒ । मन्त्रः॑ । \textbf{  44} \newline
                  \newline
                                \textbf{ TS 3.4.11.5} \newline
                  वि॒श्व॒च॒र्॒.ष॒ण॒ इति॑ विश्व - च॒र्॒.ष॒णे॒ ॥ मि॒त्रस्य॑ । च॒र्॒.ष॒णी॒धृत॒ इति॑ चर्.षणी - धृतः॑ । श्रवः॑ । दे॒वस्य॑ । सा॒न॒सिम् ॥ स॒त्यम् । चि॒त्रश्र॑वस्तम॒मिति॑ चि॒त्रश्र॑वः - त॒म॒म् ॥ मि॒त्रः । जनान्॑ । या॒त॒य॒ति॒ । प्र॒जा॒नन्निति॑ प्र - जा॒नन् । मि॒त्रः । दा॒धा॒र॒ । पृ॒थि॒वीम् । उ॒त । द्याम् ॥ मि॒त्रः । कृ॒ष्टीः । अनि॑मि॒षेत्यनि॑ - मि॒षा॒ । अ॒भीति॑ । च॒ष्टे॒ । स॒त्याय॑ । ह॒व्यम् । घृ॒तव॒दिति॑ घृ॒त-व॒त् । वि॒धे॒म॒ ॥ प्रेति॑ । सः । मि॒त्र॒ । मर्तः॑ । अ॒स्तु॒ । प्रय॑स्वान् । यः । ते॒ । आ॒दि॒त्य॒ । शिक्ष॑ति । व्र॒तेन॑ ॥ न । ह॒न्य॒ते॒ । न । जी॒य॒ते॒ । त्वोतः॑ । न । ए॒न॒म् । अꣳहः॑ । अ॒श्नो॒ति॒ । अन्ति॑तः । न । दू॒रात् ॥ यत् । \textbf{  45} \newline
                  \newline
                                \textbf{ TS 3.4.11.6} \newline
                  चि॒त् । हि । ते॒ । विशः॑ । य॒था॒ । प्रेति॑ । दे॒व॒ । व॒रु॒ण॒ । व्र॒तम् ॥ मि॒नी॒मसि॑ । द्यवि॑द्य॒वीति॒ द्यवि॑ - द्य॒वि॒ ॥ यत् । किम् । च॒ । इ॒दम् । व॒रु॒ण॒ । दैव्ये᳚ । जने᳚ । अ॒भि॒द्रो॒हमित्य॑भि - द्रो॒हम् । म॒नु॒ष्याः᳚ । चरा॑मसि ॥ अचि॑त्ती । यत् । तव॑ । धर्मा᳚ । यु॒यो॒पि॒म । मा । नः॒ । तस्मा᳚त् । एन॑सः । दे॒व॒ । री॒रि॒षः॒ ॥ कि॒त॒वासः॑ । यत् । रि॒रि॒पुः । न । दी॒वि । यत् । वा॒ । घ॒ । स॒त्यम् । उ॒त । यत् । न । वि॒द्म ॥ सर्वा᳚ । ता । वीति॑ । स्य॒ । शि॒थि॒रा ( ) । इ॒व॒ । दे॒व॒ । अथ॑ । ते॒ । स्या॒म॒ । व॒रु॒ण॒ । प्रि॒यासः॑ ॥ \textbf{  46} \newline
                  \newline
                      (सोमो॒-गोषु॒ मा- र॒यिं - मन्त्रो॒ -य-च्छि॑थि॒रा-स॒प्त च॑ )  \textbf{(A11)} \newline \newline
\textbf{praSna korvai with starting padams of 1 to 11 anuvAkams :-} \newline
(वि वा ए॒तस्या - ऽऽवा॑यो - इ॒मे वै - चि॒त्तञ्चा॒ - ऽग्निर्भू॒तानां᳚ - दे॒वा वा अ॑भ्याता॒ना - नृ॑ता॒षाड् - रा॒ष्ट्रका॑माय॒ - देवि॑का॒ - वास्तो᳚ष्पते॒ - त्वम॑ग्ने बृ॒ह - देका॑दश ) \newline

\textbf{korvai with starting padams of1, 11, 21 series of pa~jcAtis :-} \newline
(वि वा ए॒तस्ये - त्या॑ह - मृ॒त्युर्ग॑न्ध॒र्वो - ऽव॑ रुन्धे मद्ध्य॒त - स्त्वम॑ग्ने बृ॒हथ् - षट्च॑त्वारिꣳशत्) \newline

\textbf{first and last padam of Fourth praSnam of kANDam 3 :-} \newline
(वि वा ए॒तस्य॑ - प्रि॒यासः॑ ) \newline 


॥ हरिः॑ ॐ ॥॥ कृष्ण यजुर्वेदीय तैत्तिरीय संहितायां तृतीयकाण्डे चतुर्थः प्रश्नः समाप्तः ॥ \newline
\pagebreak
\pagebreak
        


\end{document}
