\documentclass[17pt]{extarticle}
\usepackage{babel}
\usepackage{fontspec}
\usepackage{polyglossia}
\usepackage{extsizes}



\setmainlanguage{sanskrit}
\setotherlanguages{english} %% or other languages
\setlength{\parindent}{0pt}
\pagestyle{myheadings}
\newfontfamily\devanagarifont[Script=Devanagari]{AdishilaVedic}


\newcommand{\VAR}[1]{}
\newcommand{\BLOCK}[1]{}




\begin{document}
\begin{titlepage}
    \begin{center}
 
\begin{sanskrit}
    { \Large
    ॐ नमः परमात्मने, श्री महागणपतये नमः, श्री गुरुभ्यो नमः ॥ ह॒रिः॒ ॐ 
    }
    \\
    \vspace{2.5cm}
    \mbox{ \Huge
    5.7      पञ्चमकाण्डे सप्तमः प्रश्नः-उपानुवाक्यावशिष्टकर्मनिरूपणं   }
\end{sanskrit}
\end{center}

\end{titlepage}
\tableofcontents

ॐ नमः परमात्मने, श्री महागणपतये नमः, श्री गुरुभ्यो नमः
ह॒रिः॒ ॐ \newline
5.7      पञ्चमकाण्डे सप्तमः प्रश्नः-उपानुवाक्यावशिष्टकर्मनिरूपणं \newline

\addcontentsline{toc}{section}{ 5.7      पञ्चमकाण्डे सप्तमः प्रश्नः-उपानुवाक्यावशिष्टकर्मनिरूपणं}
\markright{ 5.7      पञ्चमकाण्डे सप्तमः प्रश्नः-उपानुवाक्यावशिष्टकर्मनिरूपणं \hfill https://www.vedavms.in \hfill}
\section*{ 5.7      पञ्चमकाण्डे सप्तमः प्रश्नः-उपानुवाक्यावशिष्टकर्मनिरूपणं }
                                \textbf{ TS 5.7.1.1} \newline
                  यः । वै । अय॑थादेवत॒मित्यय॑था-दे॒व॒त॒म् । अ॒ग्निम् । चि॒नु॒ते । एति॑ । दे॒वता᳚भ्यः । वृ॒श्च्य॒ते॒ । पापी॑यान् । भ॒व॒ति॒ । यः । य॒था॒दे॒व॒तमिति॑ यथा - दे॒व॒तम् । न । दे॒वता᳚भ्यः । एति॑ । वृ॒श्च्य॒ते॒ । वसी॑यान् । भ॒व॒ति॒ । आ॒ग्ने॒य्या । गा॒य॒त्रि॒या । प्र॒थ॒माम् । चिति᳚म् । अ॒भीति॑ । मृ॒शे॒त् । त्रि॒ष्टुभाः᳚ । द्वि॒तीया᳚म् । जग॑त्या । तृ॒तीया᳚म् । अ॒नु॒ष्टुभेत्य॑नु - स्तुभा᳚ । च॒तु॒र्थीम् । प॒ङ्क्त्या । प॒ञ्च॒मीम् । य॒था॒दे॒व॒तमिति॑ यथा - दे॒व॒तम् । ए॒व । अ॒ग्निम् । चि॒नु॒ते॒ । न । दे॒वता᳚भ्यः । एति॑ । वृ॒श्च्य॒ते॒ । वसी॑यान् । भ॒व॒ति॒ । इडा॑यै । वै । ए॒षा । विभ॑क्ति॒रिति॒ वि - भ॒क्तिः॒ । प॒शवः॑ । इडा᳚ । प॒शुभि॒रिति॑ प॒शु-भिः॒ । ए॒न॒म् । \textbf{  1} \newline
                  \newline
                                \textbf{ TS 5.7.1.2} \newline
                  चि॒नु॒ते॒ । यः । वै । प्र॒जाप॑तय॒ इति॑ प्र॒जा -  प॒त॒ये॒ । प्र॒ति॒प्रोच्येति॑ प्रति - प्रोच्य॑ । अ॒ग्निम् । चि॒नोति॑ । न । आर्ति᳚म् । एति॑ । ऋ॒च्छ॒ति॒ । अश्वौ᳚ । अ॒भितः॑ । ति॒ष्ठे॒ता॒म् । कृ॒ष्णः । उ॒त्त॒र॒त इत्यु॑त् -   त॒र॒तः । श्वे॒तः । दक्षि॑णः । तौ । आ॒लभ्येत्या᳚ - लभ्य॑ । इष्ट॑काः । उपेति॑ । द॒द्ध्या॒त् । ए॒तत् । वै । प्र॒जाप॑ते॒रिति॑ प्र॒जा - प॒तेः॒ । रू॒पम् । प्रा॒जा॒प॒त्य इति॑ प्राजा - प॒त्यः । अश्वः॑ । सा॒क्षादिति॑ स - अ॒क्षात् । ए॒व । प्र॒जाप॑तय॒ इति॑ प्र॒जा - प॒त॒ये॒ । प्र॒ति॒प्रोच्येति॑ प्रति - प्रोच्य॑ । अ॒ग्निम् । चि॒नो॒ति॒ । न । आर्ति᳚म् । एति॑ । ऋ॒च्छ॒ति॒ । ए॒तत् । वै । अह्नः॑ । रू॒पम् । यत् । श्वे॒तः । अश्वः॑ । रात्रि॑यै । कृ॒ष्णः । ए॒तत् । अह्नः॑ । \textbf{  2} \newline
                  \newline
                                \textbf{ TS 5.7.1.3} \newline
                  रू॒पम् । यत् । इष्ट॑काः । रात्रि॑यै । पुरी॑षम् । इष्ट॑काः । उ॒प॒धा॒स्यन्नित्यु॑प - धा॒स्यन्न् । श्वे॒तम् । अश्व᳚म् । अ॒भीति॑ । मृ॒शे॒त् । पुरी॑षम् । उ॒प॒धा॒स्यन्नित्यु॑प - धा॒स्यन्न् । कृ॒ष्णम् । अ॒हो॒रा॒त्राभ्या॒मित्य॑हः - रा॒त्राभ्या᳚म् । ए॒व । ए॒न॒म् । चि॒नु॒ते॒ । हि॒र॒ण्य॒पा॒त्रमिति॑ हिरण्य-पा॒त्रम् । मधोः᳚ । पू॒र्णम् । द॒दा॒ति॒ । म॒ध॒व्यः॑ । अ॒सा॒नि॒ । इति॑ । सौ॒र्या । चि॒त्रव॒त्येति॑ चि॒त्र-व॒त्या॒ । अवेति॑ । ई॒क्ष॒ते॒ । चि॒त्रम् । ए॒व । भ॒व॒ति॒ । म॒द्ध्यन्दि॑ने । अश्व᳚म् । अवेति॑ । घ्रा॒प॒य॒ति॒ । अ॒सौ । वै । आ॒दि॒त्यः । इन्द्रः॑ । ए॒षः । प्र॒जाप॑ति॒रिति॑ प्र॒जा-प॒तिः॒ । प्रा॒जा॒प॒त्य इति॑ प्राजा - प॒त्यः । अश्वः॑ । तम् । ए॒व । सा॒क्षादिति॑ स - अ॒क्षात् । ऋ॒द्ध्नो॒ति॒ ॥ \textbf{  3} \newline
                  \newline
                      (ए॒न॒ - मे॒तदह्नो॒ - ऽष्टाच॑त्वारिꣳशच्च)  \textbf{(A1)} \newline \newline
                                \textbf{ TS 5.7.2.1} \newline
                  त्वाम् । अ॒ग्ने॒ । वृ॒ष॒भम् । चेकि॑तानम् । पुनः॑ । युवा॑नम् । ज॒नयन्न्॑ । उ॒पागा॒मित्यु॑प- आगा᳚म् ॥ अ॒स्थू॒रि । नः॒ । गार्.ह॑पत्या॒नीति॒ गार्.ह॑ - प॒त्या॒नि॒ । स॒न्तु॒ । ति॒ग्मेन॑ । नः॒ । ब्रह्म॑णा । समिति॑ । शि॒शा॒धि॒ ॥ प॒शवः॑ । वै । ए॒ते । यत् । इष्ट॑काः । चित्या᳚चिंत्या॒मिति॒ चित्यां᳚ - चि॒त्या॒म् । ऋ॒ष॒भम् । उपेति॑ । द॒धा॒ति॒ । मि॒थु॒नम् । ए॒व । अ॒स्य॒ । तत् । य॒ज्ञे । क॒रो॒ति॒ । प्र॒जन॑ना॒येति॑ प्र - जन॑नाय । तस्मा᳚त् । यू॒थेयू॑थ॒ इति॑ यू॒थे - यू॒थे॒ । ऋ॒ष॒भः ॥ सं॒ॅव॒थ्स॒रस्येति॑ सं - व॒थ्स॒रस्य॑ । प्र॒ति॒मामिति॑ प्रति - माम् । याम् । त्वा॒ । रा॒त्रि॒ । उ॒पास॑त॒ इत्यु॑प - आस॑ते ॥ प्र॒जामिति॑ प्र- जाम् । सु॒वीरा॒मिति॑ सु - वीरा᳚म् । कृ॒त्वा । विश्व᳚म् । आयुः॑ । वीति॑ । अ॒श्न॒व॒त् ॥ प्रा॒जा॒प॒त्यामिति॑ प्राजा - प॒त्याम् । \textbf{  4} \newline
                  \newline
                                \textbf{ TS 5.7.2.2} \newline
                  ए॒ताम् । उपेति॑ । द॒धा॒ति॒ । इ॒यम् । वाव । ए॒षा । ए॒का॒ष्ट॒केत्ये॑क - अ॒ष्ट॒का । यत् । ए॒व । ए॒का॒ष्ट॒काया॒मित्ये॑क - अ॒ष्ट॒काया᳚म् । अन्न᳚म् । क्रि॒यते᳚ । तत् । ए॒व । ए॒तया᳚ । अवेति॑ । रु॒न्धे॒ । ए॒षा । वै । प्र॒जाप॑ते॒रिति॑ प्र॒जा - प॒तेः॒ । का॒म॒दुघेति॑ काम - दुघा᳚ । तया᳚ । ए॒व । यज॑मानः । अ॒मुष्मिन्न्॑ । लो॒के । अ॒ग्निम् । दु॒हे॒ । येन॑ । दे॒वाः । ज्योति॑षा । ऊ॒द्‌र्ध्वाः । उ॒दाय॒न्नित्यु॑त् - आयन्न्॑ । येन॑ । आ॒दि॒त्याः । वस॑वः । येन॑ । रु॒द्राः ॥ येन॑ । अङ्गि॑रसः । म॒हि॒मान᳚म् । आ॒न॒शुः । तेन॑ । ए॒तु॒ । यज॑मानः । स्व॒स्ति ॥ सु॒व॒र्गायेति॑ सुवः - गाय॑ । वै । ए॒षः । लो॒काय॑ । \textbf{  5} \newline
                  \newline
                                \textbf{ TS 5.7.2.3} \newline
                  ची॒य॒ते॒ । यत् । अ॒ग्निः । येन॑ । दे॒वाः । ज्योति॑षा । उ॒द्‌र्ध्वाः । उ॒दाय॒न्नित्यु॑त्-आयन्न्॑ । इति॑ । उख्य᳚म् । समिति॑ । इ॒न्धे॒ । इष्ट॑काः । ए॒व । ए॒ताः । उपेति॑ । ध॒त्ते॒ । वा॒न॒स्प॒त्याः । सु॒व॒र्गस्येति॑ सुवः - गस्य॑ । लो॒कस्य॑ । सम॑ष्ट्या॒ इति॒ सं - अ॒ष्ट्यै॒ । श॒तायु॑धा॒येति॑ श॒त - आ॒यु॒धा॒य॒ । श॒तवी᳚र्या॒येति॑ श॒त - वी॒र्या॒य॒ । श॒तोत॑य॒ इति॑ श॒त - ऊ॒त॒ये॒ । अ॒भि॒मा॒ति॒षाह॒ इत्य॑भिमाति -   साहे᳚ ॥ श॒तम् । यः । नः॒ । श॒रदः॑ । अजी॑तान् । इन्द्रः॑ । ने॒ष॒त् । अतीति॑ । दु॒रि॒तानीति॑ दुः - इ॒तानि॑ । विश्वा᳚ ॥ ये । च॒त्वारः॑ । प॒थयः॑ । दे॒व॒याना॒ इति॑ देव - यानाः᳚ । अ॒न्त॒रा । द्यावा॑पृथि॒वी इति॒ द्यावा᳚ - पृ॒थि॒वी । वि॒यन्तीति॑ वि - यन्ति॑ ॥ तेषा᳚म् । यः । अज्या॑निम् । अजी॑तिम् । आ॒वहा॒दित्या᳚ - वहा᳚त् । तस्मै᳚ । नः॒ । दे॒वाः॒ । \textbf{  6} \newline
                  \newline
                                \textbf{ TS 5.7.2.4} \newline
                  परीति॑ । द॒त्त॒ । इ॒ह । सर्वे᳚ ॥ ग्री॒ष्मः । हे॒म॒न्तः । उ॒त । नः॒ । व॒स॒न्तः । श॒रत् । व॒र्.॒षाः । सु॒वि॒तम् । नः॒ । अ॒स्तु॒ ॥ तेषा᳚म् । ऋ॒तू॒नाम् । श॒तशा॑रदाना॒मिति॑ श॒त - शा॒र॒दा॒ना॒म् । नि॒वा॒त इति॑ नि - वा॒ते । ए॒षा॒म् । अभ॑ये । स्या॒म॒ ॥ इ॒दु॒व॒थ्स॒रायेती॑दु - व॒थ्स॒राय॑ । प॒रि॒व॒थ्स॒रायेति॑ परि - व॒थ्स॒राय॑ । सं॒ॅव॒थ्स॒रायेति॑ सं - व॒थ्स॒राय॑ । कृ॒णु॒त॒ । बृ॒हत् । नमः॑ ॥ तेषा᳚म् । व॒यम् । सु॒म॒ताविति॑ सु - म॒तौ । य॒ज्ञिया॑नाम् । ज्योक् । अजी॑ताः । अह॑ताः । स्या॒म॒ ॥ भ॒द्रात् । नः॒ । श्रेयः॑ । समिति॑ । अ॒नै॒ष्ट॒ । दे॒वाः॒ । त्वया᳚ । अ॒व॒सेन॑ । समिति॑ । अ॒शी॒म॒हि॒ । त्वा॒ ॥ सः । नः॒ । म॒यो॒भूरिति॑ मयः-भूः । पि॒तो॒ इति॑ । \textbf{  7} \newline
                  \newline
                                \textbf{ TS 5.7.2.5} \newline
                  एति॑ । वि॒श॒स्व॒ । शम् । तो॒काय॑ । त॒नुवे᳚ । स्यो॒नः ॥ अज्या॑नीः । ए॒ताः । उपेति॑ । द॒धा॒ति॒ । ए॒ताः । वै । दे॒वताः᳚ । अप॑राजिता॒ इत्यप॑रा - जि॒ताः॒ । ताः । ए॒व । प्रेति॑ । वि॒श॒ति॒ । न । ए॒व । जी॒य॒ते॒ । ब्र॒ह्म॒वा॒दिन॒ इति॑ ब्रह्म - वा॒दिनः॑ । व॒द॒न्ति॒ । यत् । अ॒द्‌र्ध॒मा॒सा इत्य॑द्‌र्ध - मा॒साः । मासाः᳚ । ऋ॒तवः॑ । सं॒ॅव॒थ्स॒र इति॑ सं - व॒थ्स॒रः । ओष॑धीः । पच॑न्ति । अथ॑ । कस्मा᳚त् । अ॒न्याभ्यः॑ । दे॒वता᳚भ्यः । आ॒ग्र॒य॒णम् । निरिति॑ । उ॒प्य॒ते॒ । इति॑ । ए॒ताः । हि । तत् । दे॒वताः᳚ । उ॒दज॑य॒न्नित्यु॑त् - अज॑यन्न् । यत् । ऋ॒तुभ्य॒ इत्यृ॒तु-भ्यः॒ । नि॒र्वपे॒दिति॑ निः - वपे᳚त् । दे॒वता᳚भ्यः । स॒मद॒मिति॑ स - मद᳚म् । द॒द्ध्या॒त् । आ॒ग्र॒य॒णम् ( ) । नि॒रुप्येति॑ निः - उप्य॑ । ए॒ताः । आहु॑ती॒रित्या- हु॒तीः॒ । जु॒हो॒ति॒ । अ॒द्‌र्ध॒मा॒सानित्य॑र्ध - मा॒सान् । ए॒व । मासान्॑ । ऋ॒तून् । सं॒ॅव॒थ्स॒रमिति॑ सं - व॒थ्स॒रम् । प्री॒णा॒ति॒ । न । दे॒वता᳚भ्यः । स॒मद॒मिति॑ स - मद᳚म् । द॒धा॒ति॒ । भ॒द्रात् । नः॒ । श्रेयः॑ । समिति॑ । अ॒नै॒ष्ट॒ । दे॒वाः॒ । इति॑ । आ॒ह॒ । हु॒ताद्या॒येति॑ हुत - अद्या॑य । यज॑मानस्य । अप॑राभावा॒येत्यप॑रा - भा॒वा॒य॒ ॥ \textbf{  8} \newline
                  \newline
                      (प्र॒जा॒प॒त्यां - ॅलो॒काय॑ - देवाः - पितो - दध्यादाग्रय॒णं - पञ्च॑विꣳशतिश्च)  \textbf{(A2)} \newline \newline
                                \textbf{ TS 5.7.3.1} \newline
                  इन्द्र॑स्य । वज्रः॑ । अ॒सि॒ । वार्त्र॑घ्न॒ इति॒ वार्त्र॑ - घ्नः॒ । त॒नू॒पा इति॑ तनू - पाः । नः॒ । प्र॒ति॒स्प॒श इति॑ प्रति - स्प॒शः ॥ यः । नः॒ । पु॒रस्ता᳚त् । द॒क्षि॒ण॒तः । प॒श्चात् । उ॒त्त॒र॒त इत्यु॑त् - त॒र॒तः । अ॒घा॒युरित्य॑घ - युः । अ॒भि॒दास॒तीत्य॑भि - दास॑ति । ए॒तम् । सः । अश्मा॑नम् । ऋ॒च्छ॒तु॒ ॥ दे॒वा॒सु॒रा इति॑ देव - अ॒सु॒राः । संॅय॑त्ता॒ इति॒ सं - य॒त्ताः॒ । आ॒स॒न्न् । ते । असु॑राः । दि॒ग्भ्य इति॑ दिक्-भ्यः । एति॑ । अ॒बा॒ध॒न्त॒ । तान् । दे॒वाः । इष्वा᳚ । च॒ । वज्रे॑ण । च॒ । अपेति॑ । अ॒नु॒द॒न्त॒ । यत् । व॒ज्रिणीः᳚ । उ॒प॒दधा॒तीत्यु॑प-दधा॑ति । इष्वा᳚ । च॒ । ए॒व । तत् । वज्रे॑ण । च॒ । यज॑मानः । भ्रातृ॑व्यान् । अपेति॑ । नु॒द॒ते॒ । दि॒क्षु । उपेति॑ । \textbf{  9} \newline
                  \newline
                                \textbf{ TS 5.7.3.2} \newline
                  द॒धा॒ति॒ । दे॒व॒पु॒रा इति॑ देव - पु॒राः । ए॒व । ए॒ताः । त॒नू॒पानी॒रिति॑ तनू - पानीः᳚ । परीति॑ । ऊ॒ह॒ते॒ । अग्ना॑विष्णू॒ इत्यग्ना᳚ - वि॒ष्णू॒ । स॒जोष॒सेति॑ स - जोष॑सा । इ॒माः । व॒द्‌र्ध॒न्तु॒ । वा॒म् । गिरः॑ ॥ द्यु॒म्नैः । वाजे॑भिः । एति॑ । ग॒त॒म् ॥ ब्र॒ह्म॒वा॒दिन॒ इति॑ ब्रह्म - वा॒दिनः॑ । व॒द॒न्ति॒ । यत् । न । दे॒वता॑यै । जुह्व॑ति । अथ॑ । कि॒दें॒व॒त्येति॑ किं - दे॒व॒त्या᳚ । वसोः᳚ । धारा᳚ । इति॑ । अ॒ग्निः । वसुः॑ । तस्य॑ । ए॒षा । धारा᳚ । विष्णुः॑ । वसुः॑ । तस्य॑ । ए॒षा । धारा᳚ । आ॒ग्ना॒वै॒ष्ण॒व्येत्या᳚ग्ना - वै॒ष्ण॒व्या । ऋ॒चा । वसोः᳚ । धारा᳚म् । जु॒हो॒ति॒ । भा॒ग॒धेये॒नेति॑ भाग - धेये॑न । ए॒व । ए॒नौ॒ । समिति॑ । अ॒द्‌र्ध॒य॒ति॒ । अथो॒ इति॑ । ए॒ताम् । \textbf{  10} \newline
                  \newline
                                \textbf{ TS 5.7.3.3} \newline
                  ए॒व । आहु॑ति॒मित्या - हु॒ति॒म् । आ॒यत॑नवती॒मित्या॒यत॑न - व॒ती॒म् । क॒रो॒ति॒ । यत्का॑म॒ इति॒ यत् - का॒मः॒ । ए॒ना॒म् । जु॒होति॑ । तत् । ए॒व । अवेति॑ । रु॒न्धे॒ । रु॒द्रः । वै । ए॒षः । यत् । अ॒ग्निः । तस्य॑ । ए॒ते इति॑ । त॒नुवौ᳚ । घो॒रा । अ॒न्या । शि॒वा । अ॒न्या । यत् । श॒त॒रु॒द्रीय॒मिति॑ शत-रु॒द्रीय᳚म् । जु॒होति॑ । या । ए॒व । अ॒स्य॒ । घो॒रा । त॒नूः । ताम् । तेन॑ । श॒म॒य॒ति॒ । यत् । वसोः᳚ । धारा᳚म् । जु॒होति॑ । या । ए॒व । अ॒स्य॒ । शि॒वा । त॒नूः । ताम् । तेन॑ । प्री॒णा॒ति॒ । यः । वै । वसोः᳚ । धारा॑यै । \textbf{  11} \newline
                  \newline
                                \textbf{ TS 5.7.3.4} \newline
                  प्र॒ति॒ष्ठामिति॑ प्रति - स्थाम् । वेद॑ । प्रतीति॑ । ए॒व । ति॒ष्ठ॒ति॒ । यत् । आज्य᳚म् । उ॒च्छिष्ये॒तेत्यु॑त् - शिष्ये॑त । तस्मिन्न्॑ । ब्र॒ह्मौ॒द॒नमिति॑ ब्रह्म - ओ॒द॒नम् । प॒चे॒त् । तम् । ब्रा॒ह्म॒णाः । च॒त्वारः॑ । प्रेति॑ । अ॒श्नी॒युः॒ । ए॒षः । वै । अ॒ग्निः । वै॒श्वा॒न॒रः । यत् । ब्रा॒ह्म॒णः । ए॒षा । खलु॑ । वै । अ॒ग्नेः । प्रि॒या । त॒नूः । यत् । वै॒श्वा॒न॒रः । प्रि॒याया᳚म् । ए॒व । ए॒ना॒म् । त॒नुवा᳚म् । प्रतीति॑ । स्था॒प॒य॒ति॒ । चत॑स्रः । धे॒नूः । द॒द्या॒त् । ताभिः॑ । ए॒व । यज॑मानः । अ॒मुष्मिन्न्॑ । लो॒के । अ॒ग्निम् । दु॒हे॒ ॥ \textbf{  12} \newline
                  \newline
                      (उपै॒ - तां - धारा॑यै॒ - षट्च॑त्वारिꣳशच्च)  \textbf{(A3)} \newline \newline
                                \textbf{ TS 5.7.4.1} \newline
                  चित्ति᳚म् । जु॒हो॒मि॒ । मन॑सा । घृ॒तेन॑ । इति॑ । आ॒ह॒ । अदा᳚भ्या । वै । नाम॑ । ए॒षा । आहु॑ति॒रित्या-हु॒तिः॒ । वै॒श्व॒कर्म॒णीति॑ वैश्व - क॒र्म॒णी । न । ए॒न॒म् । चि॒क्या॒नम् । भ्रातृ॑व्यः । द॒भ्नो॒ति॒ । अथो॒ इति॑ । दे॒वताः᳚ । ए॒व । अवेति॑ । रु॒न्धे॒ । अग्ने᳚ । तम् । अ॒द्य । इति॑ । प॒ङ्क्त्या । जु॒हो॒ति॒ । प॒ङ्क्त्या । आहु॒त्येत्या - हु॒त्या॒ । य॒ज्ञ्॒मु॒खमिति॑ यज्ञ् - मु॒खम् । एति॑ । र॒भ॒ते॒ । स॒प्त । ते॒ । अ॒ग्ने॒ । स॒मिध॒ इति॑ सं - इधः॑ । स॒प्त । जि॒ह्वाः । इति॑ । आ॒ह॒ । होत्राः᳚ । ए॒व । अवेति॑ । रु॒न्धे॒ । अ॒ग्निः । दे॒वेभ्यः॑ । अपेति॑ । अ॒क्रा॒म॒त् । भा॒ग॒धेय॒मिति॑ भाग - धेय᳚म् । \textbf{  13} \newline
                  \newline
                                \textbf{ TS 5.7.4.2} \newline
                  इ॒च्छमा॑नः । तस्मै᳚ । ए॒तत् । भा॒ग॒धेय॒मिति॑ भाग - धेय᳚म् । प्रेति॑ । अ॒य॒च्छ॒न्न् । ए॒तत् । वै । अ॒ग्नेः । अ॒ग्नि॒हो॒त्रमित्य॑ग्नि - हो॒त्रम् । ए॒तर्.हि॑ । खलु॑ । वै । ए॒षः । जा॒तः । यर्.हि॑ । सर्वः॑ । चि॒तः । जा॒ताय॑ । ए॒व । अ॒स्मै॒ । अन्न᳚म् । अपीति॑ । द॒धा॒ति॒ । सः । ए॒न॒म् । प्री॒तः । प्री॒णा॒ति॒ । वसी॑यान् । भ॒व॒ति॒ । ब्र॒ह्म॒वा॒दिन॒ इति॑ ब्रह्म-वा॒दिनः॑ । व॒द॒न्ति॒ । यत् । ए॒षः । गार्.ह॑पत्य॒ इति॒ गार्.ह॑-प॒त्यः॒ । ची॒यते᳚ । अथ॑ । क्व॑ । अ॒स्य॒ । आ॒ह॒व॒नीय॒ इत्या᳚ -   ह॒व॒नीयः॑ । इति॑ । अ॒सौ । आ॒दि॒त्यः । इति॑ । ब्रू॒या॒त् । ए॒तस्मिन्न्॑ । हि । सर्वा᳚भ्यः । दे॒वता᳚भ्यः । जुह्व॑ति । \textbf{  14} \newline
                  \newline
                                \textbf{ TS 5.7.4.3} \newline
                  यः । ए॒वम् । वि॒द्वान् । अ॒ग्निम् । चि॒नु॒ते । सा॒क्षादिति॑ स - अ॒क्षात् । ए॒व । दे॒वताः᳚ । ऋ॒द्ध्नो॒ति॒ । अग्ने᳚ । य॒श॒स्वि॒न्न् । यश॑सा । इ॒मम् । अ॒र्प॒य॒ । इन्द्रा॑वती॒मितीन्द्र॑ - व॒ती॒म् । अप॑चिती॒मित्यप॑ - चि॒ती॒म् । इ॒ह । एति॑ । व॒ह॒ ॥ अ॒यम् । मू॒द्‌र्धा । प॒र॒मे॒ष्ठी । सु॒वर्चा॒ इति॑ सु - वर्चाः᳚ । स॒मा॒नाना᳚म् । उ॒त्त॒मश्लो॑क॒ इत्यु॑त्त॒म - श्लो॒कः॒ । अ॒स्तु॒ ॥ भ॒द्रम् । पश्य॑न्तः । उपेति॑ । से॒दुः॒ । अग्रे᳚ । तपः॑ । दी॒क्षाम् । ऋष॑यः । सु॒व॒र्विद॒ इति॑ सुवः - विदः॑ ॥ ततः॑ । क्ष॒त्रम् । बल᳚म् । ओजः॑ । च॒ । जा॒तम् । तत् । अ॒स्मै । दे॒वाः । अ॒भि । समिति॑ । न॒म॒न्तु॒ ॥ धा॒ता । वि॒धा॒तेति॑ वि - धा॒ता । प॒र॒मा । \textbf{  15} \newline
                  \newline
                                \textbf{ TS 5.7.4.4} \newline
                  उ॒त । स॒दृंगिति॑ सं - दृक् । प्र॒जाप॑ति॒रिति॑ प्र॒जा - प॒तिः॒ । प॒र॒मे॒ष्ठी । वि॒राजेति॑ वि - राजा᳚ ॥ स्तोमाः᳚ । छन्दाꣳ॑सि । नि॒विद॒ इति॑ नि - विदः॑ । मे॒ । आ॒हुः॒ । ए॒तस्मै᳚ । रा॒ष्ट्रम् । अ॒भि । समिति॑ । न॒मा॒म॒ ॥ अ॒भ्याव॑र्तद्ध्व॒मित्य॑भि - आव॑र्तद्ध्वम् । उपेति॑ । मा॒ । एति॑ । इ॒त॒ । सा॒कम् । अ॒यम् । शा॒स्ता । अधि॑पति॒रित्यधि॑ - प॒तिः॒ । वः॒ । अ॒स्तु॒ ॥ अ॒स्य । वि॒ज्ञान॒मिति॑ वि - ज्ञान᳚म् । अनु॑ । समिति॑ । र॒भ॒द्ध्व॒म् । इ॒मम् । प॒श्चात् । अन्विति॑ । जी॒वा॒थ॒ । सर्वे᳚ ॥ रा॒ष्ट्र॒भृत॒ इति॑ राष्ट्र - भृतः॑ । ए॒ताः । उपेति॑ । द॒धा॒ति॒ । ए॒षा । वै । अ॒ग्नेः । चितिः॑ । रा॒ष्ट्र॒भृदिति॑ राष्ट्र - भृत् । तया᳚ । ए॒व । अ॒स्मि॒न्न् । रा॒ष्ट्रम् । द॒धा॒ति॒ ( ) । रा॒ष्ट्रम् । ए॒व । भ॒व॒ति॒ । न । अ॒स्मा॒त् । रा॒ष्ट्रम् । भ्रꣳ॒॒श॒ते॒ ॥ \textbf{  16} \newline
                  \newline
                      (भा॒ग॒धेयं॒ - जुह्व॑ति - पर॒मा - रा॒ष्ट्रं द॑धाति - स॒प्त च॑)  \textbf{(A4)} \newline \newline
                                \textbf{ TS 5.7.5.1} \newline
                  यथा᳚ । वै । पु॒त्रः । जा॒तः । म्रि॒यते᳚ । ए॒वम् । वै । ए॒षः । म्रि॒य॒ते॒ । यस्य॑ । अ॒ग्निः । उख्यः॑ । उ॒द्वाय॒तीयु॑त् - वाय॑ति । यत् । नि॒र्म॒न्थ्य॑मिति॑ निः - म॒न्थ्य᳚म् । कु॒र्यात् । वीति॑ । छि॒न्द्या॒त् । भ्रातृ॑व्यम् । अ॒स्मै॒ । ज॒न॒ये॒त् । सः । ए॒व । पुनः॑ । प॒रीद्ध्य॒ इति॑ परि-इद्ध्यः॑ । स्वात् । ए॒व । ए॒न॒म् । योनेः᳚ । ज॒न॒य॒ति॒ । न । अ॒स्मै॒ । भ्रातृ॑व्यम् । ज॒न॒य॒ति॒ । तमः॑ । वै । ए॒तम् । गृ॒ह्णा॒ति॒ । यस्य॑ । अ॒ग्निः । उख्यः॑ । उ॒द्वाय॒तीत्यु॑त् - वाय॑ति । मृ॒त्युः । तमः॑ । कृ॒ष्णम् । वासः॑ । कृ॒ष्णा । धे॒नुः । दक्षि॑णा । तम॑सा । \textbf{  17} \newline
                  \newline
                                \textbf{ TS 5.7.5.2} \newline
                  ए॒व । तमः॑ । मृ॒त्युम् । अपेति॑ । ह॒ते॒ । हिर॑ण्यम् । द॒दा॒ति॒ । ज्योतिः॑ । वै । हिर॑ण्यम् । ज्योति॑षा । ए॒व । तमः॑ । अपेति॑ । ह॒ते॒ । अथो॒ इति॑ । तेजः॑ । वै । हिर॑ण्यम् । तेजः॑ । ए॒व । आ॒त्मन्न् । ध॒त्ते॒ । सुवः॑ । न । घ॒र्मः । स्वाहा᳚ । सुवः॑ । न । अ॒र्कः । स्वाहा᳚ । सुवः॑ । न । शु॒क्रः । स्वाहा᳚ । सुवः॑ । न । ज्योतिः॑ । स्वाहा᳚ । सुवः॑ । न । सूर्यः॑ । स्वाहा᳚ । अ॒र्कः । वै । ए॒षः । यत् । अ॒ग्निः । अ॒सौ । आ॒दि॒त्यः । \textbf{  18} \newline
                  \newline
                                \textbf{ TS 5.7.5.3} \newline
                  अ॒श्व॒मे॒ध इत्य॑श्व - मे॒धः । यत् । ए॒ताः । आहु॑ती॒रित्या - हु॒तीः॒ । जु॒होति॑ । अ॒र्का॒श्व॒मे॒धयो॒रित्य॑र्क - अ॒श्व॒मे॒धयोः᳚ । ए॒व । ज्योतीꣳ॑षि । समिति॑ । द॒धा॒ति॒ । ए॒षः । ह॒ । तु । वै । अ॒र्का॒श्व॒मे॒धीत्य॑र्क-अ॒श्व॒मे॒धी । यस्य॑ । ए॒तत् । अ॒ग्नौ । क्रि॒यते᳚ । आपः॑ । वै । इ॒दम् । अग्रे᳚ । स॒लि॒लम् । आ॒सी॒त् । सः । ए॒ताम् । प्र॒जाप॑ति॒रिति॑ प्र॒जा - प॒तिः॒ । प्र॒थ॒माम् । चिति᳚म् । अ॒प॒श्य॒त् । ताम् । उपेति॑ । अ॒ध॒त्त॒ । तत् । इ॒यम् । अ॒भ॒व॒त् । तम् । वि॒श्वक॒र्मेति॑ वि॒श्व - क॒र्मा॒ । अ॒ब्र॒वी॒त् । उपेति॑ । त्वा॒ । एति॑ । अ॒या॒नि॒ । इति॑ । न । इ॒ह । लो॒कः । अ॒स्ति॒ । इति॑ । \textbf{  19} \newline
                  \newline
                                \textbf{ TS 5.7.5.4} \newline
                  अ॒ब्र॒वी॒त् । सः । ए॒ताम् । द्वि॒तीया᳚म् । चिति᳚म् । अ॒प॒श्य॒त् । ताम् । उपेति॑ । अ॒ध॒त्त॒ । तत् । अ॒न्तरि॑क्षम् । अ॒भ॒व॒त् । सः । य॒ज्ञ्ः । प्र॒जाप॑ति॒मिति॑ प्र॒जा - प॒ति॒म् । अ॒ब्र॒वी॒त् । उपेति॑ । त्वा॒ । एति॑ । अ॒या॒नि॒ । इति॑ । न । इ॒ह । लो॒कः । अ॒स्ति॒ । इति॑ । अ॒ब्र॒वी॒त् । सः । वि॒श्वक॑र्माण॒मिति॑ वि॒श्व-क॒र्मा॒ण॒म् । अ॒ब्र॒वी॒त् । उपेति॑ । त्वा॒ । एति॑ । अ॒या॒नि॒ । इति॑ । केन॑ । मा॒ । उ॒पैष्य॒सीत्यु॑प - ऐष्य॑सि । इति॑ । दिश्या॑भिः । इति॑ । अ॒ब्र॒वी॒त् । तम् । दिश्या॑भिः । उ॒पैदित्यु॑प - ऐत् । ताः । उपेति॑ । अ॒ध॒त्त॒ । ताः । दिशः॑ । \textbf{  20} \newline
                  \newline
                                \textbf{ TS 5.7.5.5} \newline
                  अ॒भ॒व॒न्न् । सः । प॒र॒मे॒ष्ठी । प्र॒जाप॑ति॒मिति॑ प्र॒जा-प॒ति॒म् । अ॒ब्र॒वी॒त् । उपेति॑ । त्वा॒ । एति॑ । अ॒या॒नि॒ । इति॑ । न । इ॒ह । लो॒कः । अ॒स्ति॒ । इति॑ । अ॒ब्र॒वी॒त् । सः । वि॒श्वक॑र्माण॒मिति॑ वि॒श्व - क॒र्मा॒ण॒म् । च॒ । य॒ज्ञ्म् । च॒ । अ॒ब्र॒वी॒त् । उपेति॑ । वा॒म् । एति॑ । अ॒या॒नि॒ । इति॑ । न । इ॒ह । लो॒कः । अ॒स्ति॒ । इति॑ । अ॒ब्रू॒ता॒म् । सः । ए॒ताम् । तृ॒तीया᳚म् । चिति᳚म् । अ॒प॒श्य॒त् । ताम् । उपेति॑ । अ॒ध॒त्त॒ । तत् । अ॒सौ । अ॒भ॒व॒त् । सः । आ॒दि॒त्यः । प्र॒जाप॑ति॒मिति॑ प्र॒जा-प॒ति॒म् । अ॒ब्र॒वी॒त् । उपेति॑ । त्वा॒ । \textbf{  21} \newline
                  \newline
                                \textbf{ TS 5.7.5.6} \newline
                  एति॑ । अ॒या॒नि॒ । इति॑ । न । इ॒ह । लो॒कः । अ॒स्ति॒ । इति॑ । अ॒ब्र॒वी॒त् । सः । वि॒श्वक॑र्माण॒मिति॑ वि॒श्व - क॒र्मा॒ण॒म् । च॒ । य॒ज्ञ्म् । च॒ । अ॒ब्र॒वी॒त् । उपेति॑ । वा॒म् । एति॑ । अ॒या॒नि॒ । इति॑ । न । इ॒ह । लो॒कः । अ॒स्ति॒ । इति॑ । अ॒ब्रू॒ता॒म् । सः । प॒र॒मे॒ष्ठिन᳚म् । अ॒ब्र॒वी॒त् । उपेति॑ । त्वा॒ । एति॑ । अ॒या॒नि॒ । इति॑ । केन॑ । मा॒ । उ॒पैष्य॒सीत्यु॑प - ऐष्य॑सि । इति॑ । लो॒कं॒पृ॒णयेति॑ लोकं - पृ॒णया᳚ । इति॑ । अ॒ब्र॒वी॒त् । तम् । लो॒क॒पृं॒णयेति॑ लोकं - पृ॒णया᳚ । उ॒पैदित्यु॑प - ऐत् । तस्मा᳚त् । अया॑तया॒म्नीत्यया॑त - या॒म्नी॒ । लो॒क॒पृं॒णेति॑ लोकं - पृ॒णा । अया॑तया॒मेत्यया॑त - या॒मा॒ । हि । अ॒सौ । \textbf{  22} \newline
                  \newline
                                \textbf{ TS 5.7.5.7} \newline
                  आ॒दि॒त्यः । तान् । ऋष॑यः । अ॒ब्रु॒व॒न्न् । उपेति॑ । वः॒ । एति॑ । अ॒या॒म॒ । इति॑ । केन॑ । नः॒ । उ॒पैष्य॒थेत्यु॑प - ऐष्य॑थ । इति॑ । भू॒म्ना । इति॑ । अ॒ब्रु॒व॒न्न् । तान् । द्वाभ्या᳚म् । चिती᳚भ्या॒मिति॑ चिति॑ - भ्या॒म् । उ॒पाय॒न्नित्यु॑प - आयन्न्॑ । सः । पञ्च॑चितीक॒ इति॒ पञ्च॑-चि॒ती॒कः॒ । समिति॑ । अ॒प॒द्य॒त॒ । यः । ए॒वम् । वि॒द्वान् । अ॒ग्निम् । चि॒नु॒ते । भूयान्॑ । ए॒व । भ॒व॒ति॒ । अ॒भीति॑ । इ॒मान् । लो॒कान् । ज॒य॒ति॒ । वि॒दुः । ए॒न॒म् । दे॒वाः । अथो॒ इति॑ । ए॒तासा᳚म् । ए॒व । दे॒वता॑नाम् । सायु॑ज्यम् । ग॒च्छ॒ति॒ ॥ \textbf{  23 } \newline
                  \newline
                       (तम॑सा - ऽऽदि॒त्यो᳚ - ऽस्तीति॒ - दिश॑ - आदि॒त्यः प्र॒जाप॑तिमब्रवी॒दुप॑ त्वा॒ - ऽसौ - पञ्च॑चत्वारिꣳशच्च)  \textbf{(A5)} \newline \newline
                                \textbf{ TS 5.7.6.1} \newline
                  वयः॑ । वै । अ॒ग्निः । यत् । अ॒ग्नि॒चिदित्य॑ग्नि - चित् । प॒क्षिणः॑ । अ॒श्नी॒यात् । तम् । ए॒व । अ॒ग्निम् । अ॒द्या॒त् । आर्ति᳚म् । एति॑ । ऋ॒च्छे॒त् । सं॒ॅव॒थ्स॒रमिति॑ सं - व॒थ्स॒रम् । व्र॒तम् । च॒रे॒त् । सं॒ॅव॒थ्स॒रमिति॑ सं - व॒थ्स॒रम् । हि । व्र॒तम् । न । अतीति॑ । प॒शुः । वै । ए॒षः । यत् । अ॒ग्निः । हि॒नस्ति॑ । खलु॑ । वै । तम् । प॒शुः । यः । ए॒न॒म् । पु॒रस्ता᳚त् । प्र॒त्यञ्च᳚म् । उ॒प॒चर॒तीत्यु॑प - चर॑ति । तस्मा᳚त् । प॒श्चात् । प्राङ् । उ॒प॒चर्य॒ इत्यु॑प - चर्यः॑ । आ॒त्मनः॑ । अहिꣳ॑सायै । तेजः॑ । अ॒सि॒ । तेजः॑ । मे॒ । य॒च्छ॒ । पृ॒थि॒वीम् । य॒च्छ॒ । \textbf{  24} \newline
                  \newline
                                \textbf{ TS 5.7.6.2} \newline
                  पृ॒थि॒व्यै । मा॒ । पा॒हि॒ । ज्योतिः॑ । अ॒सि॒ । ज्योतिः॑ । मे॒ । य॒च्छ॒ । अ॒न्तरि॑क्षम् । य॒च्छ॒ । अ॒न्तरि॑क्षात् । मा॒ । पा॒हि॒ । सुवः॑ । अ॒सि॒ । सुवः॑ । मे॒ । य॒च्छ॒ । दिव᳚म् । य॒च्छ॒ । दि॒वः । मा॒ । पा॒हि॒ । इति॑ । आ॒ह॒ । ए॒ताभिः॑ । वै । इ॒मे । लो॒काः । विधृ॑ता॒ इति॒ वि - धृ॒ताः॒ । यत् । ए॒ताः । उ॒प॒दधा॒तीत्यु॑प - दधा॑ति । ए॒षाम् । लो॒काना᳚म् । विधृ॑त्या॒ इति॒ वि - धृ॒त्यै॒ । स्व॒य॒मा॒तृ॒ण्णा इति॑ स्वयं - आ॒तृ॒ण्णाः । उ॒प॒धायेत्यु॑प - धाय॑ । हि॒र॒ण्ये॒ष्ट॒का इति॑ हिरण्य - इ॒ष्ट॒काः । उपेति॑ । द॒धा॒ति॒ । इ॒मे । वै । लो॒काः । स्व॒य॒मा॒तृ॒ण्णा इति॑ स्वयं-आ॒तृ॒ण्णाः । ज्योतिः॑ । हिर॑ण्यम् । यत् । स्व॒य॒मा॒तृ॒ण्णा इति॑ स्वयं - आ॒तृ॒ण्णाः । उ॒प॒धायेत्यु॑प - धाय॑ । \textbf{  25} \newline
                  \newline
                                \textbf{ TS 5.7.6.3} \newline
                  हि॒र॒ण्ये॒ष्ट॒का इति॑ हिरण्य - इ॒ष्ट॒काः । उ॒प॒दधा॒तीत्यु॑प - दधा॑ति । इ॒मान् । ए॒व । ए॒ताभिः॑ । लो॒कान् । ज्योति॑ष्मतः । कु॒रु॒ते॒ । अथो॒ इति॑ । ए॒ताभिः॑ । ए॒व । अ॒स्मै॒ । इ॒मे । लो॒काः । प्रेति॑ । भा॒न्ति॒ । याः । ते॒ । अ॒ग्ने॒ । सूर्ये᳚ । रुचः॑ । उ॒द्य॒त इत्यु॑त् - य॒तः । दिव᳚म् । आ॒त॒न्वन्तीत्या᳚ - त॒न्वन्ति॑ । र॒श्मिभि॒रिति॑ र॒श्मि - भिः॒ ॥ ताभिः॑ । सर्वा॑भिः । रु॒चे । जना॑य । नः॒ । कृ॒धि॒ ॥ याः । वः॒ । दे॒वाः॒ । सूर्ये᳚ । रुचः॑ । गोषु॑ । अश्वे॑षु । याः । रुचः॑ ॥ इन्द्रा᳚ग्नी॒ इतीन्द्र॑-अ॒ग्नी॒ । ताभिः॑ । सर्वा॑भिः । रुच᳚म् । नः॒ । ध॒त्त॒ । बृ॒ह॒स्प॒ते॒ ॥ रुच᳚म् । नः॒ । धे॒हि॒ । \textbf{  26} \newline
                  \newline
                                \textbf{ TS 5.7.6.4} \newline
                  ब्रा॒ह्म॒णेषु॑ । रुच᳚म् । राज॒स्विति॒ राज॑ - सु॒ । नः॒ । कृ॒धि॒ ॥ रुच᳚म् । वि॒श्ये॑षु । शू॒द्रेषु॑ । मयि॑ । धे॒हि॒ । रु॒चा । रुच᳚म् ॥ द्वे॒धा । वै । अ॒ग्निम् । चि॒क्या॒नस्य॑ । यशः॑ । इ॒न्द्रि॒यम् । ग॒च्छ॒ति॒ । अ॒ग्निम् । वा॒ । चि॒तम् । ई॒जा॒नम् । वा॒ । यत् । ए॒ताः । आहु॑ती॒रित्या - हु॒तीः॒ । जु॒होति॑ । आ॒त्मन्न् । ए॒व । यशः॑ । इ॒न्द्रि॒यम् । ध॒त्ते॒ । ई॒श्व॒रः । वै । ए॒षः । आर्ति᳚म् । आर्तो॒रित्या - अ॒र्तोः॒ । यः । अ॒ग्निम् । चि॒न्वन्न् । अ॒धि॒क्राम॒तीत्य॑धि - क्राम॑ति । तत् । त्वा॒ । या॒मि॒ । ब्रह्म॑णा । वन्द॑मानः । इति॑ । वा॒रु॒ण्या । ऋ॒चा । \textbf{  27} \newline
                  \newline
                                \textbf{ TS 5.7.6.5} \newline
                  जु॒हु॒या॒त् । शान्तिः॑ । ए॒व । ए॒षा । अ॒ग्नेः । गुप्तिः॑ । आ॒त्मनः॑ । ह॒विष्कृ॑त॒ इति॑ ह॒विः - कृ॒तः॒ । वै । ए॒षः । यः । अ॒ग्निम् । चि॒नु॒ते । यथा᳚ । वै । ह॒विः । स्कन्द॑ति । ए॒वम् । वै । ए॒षः । स्क॒न्द॒ति॒ । यः । अ॒ग्निम् । चि॒त्वा । स्त्रिय᳚म् । उ॒पैतीत्यु॑प - एति॑ । मै॒त्रा॒व॒रु॒ण्येति॑ मैत्रा - व॒रु॒ण्या । आ॒मिक्ष॑या । य॒जे॒त॒ । मै॒त्रा॒व॒रु॒णता॒मिति॑ मैत्रा - व॒रु॒णता᳚म् । ए॒व । उपेति॑ । ए॒ति॒ । आ॒त्मनः॑ । अस्क॑न्दाय । यः । वै । अ॒ग्निम् । ऋ॒तु॒स्थामित्यृ॑तु - स्थाम् । वेद॑ । ऋ॒तुर्.ऋ॑तु॒रित्यृ॒तुः - ऋ॒तुः॒ । अ॒स्मै॒ । कल्प॑मानः । ए॒ति॒ । प्रतीति॑ । ए॒व । ति॒ष्ठ॒ति॒ । सं॒ॅव॒थ्स॒र इति॑ सं - व॒थ्स॒रः । वै । अ॒ग्निः । \textbf{  28} \newline
                  \newline
                                \textbf{ TS 5.7.6.6} \newline
                  ऋ॒तु॒स्था इत्यृ॑तु - स्थाः । तस्य॑ । व॒स॒न्तः । शिरः॑ । ग्री॒ष्मः । दक्षि॑णः । प॒क्षः । व॒र्॒.षाः । पुच्छ᳚म् । श॒रत् । उत्त॑र॒ इत्युत् - त॒रः॒ । प॒क्षः । हे॒म॒न्तः । मद्ध्य᳚म् । पू॒र्व॒प॒क्षा इति॑ पूर्व - प॒क्षाः । चित॑यः । अ॒प॒र॒प॒क्षा इत्य॑पर - प॒क्षाः । पुरी॑षम् । अ॒हो॒रा॒त्राणीत्य॑हः-रा॒त्राणि॑ । इष्ट॑काः । ए॒षः । वै । अ॒ग्निः । ऋ॒तु॒स्था इत्यृ॑तु - स्थाः । यः । ए॒वम् । वेद॑ । ऋ॒तुर्.ऋ॑तु॒रित्यृ॒तुः - ऋ॒तुः॒ । अ॒स्मै॒ । कल्प॑मानः । ए॒ति॒ । प्रतीति॑ । ए॒व । ति॒ष्ठ॒ति॒ । प्र॒जाप॑ति॒रिति॑ प्र॒जा - प॒तिः॒ । वै । ए॒तम् । ज्यैष्ठ्य॑काम॒ इति॒ ज्यैष्ठ्य॑ - का॒मः॒ । नीति॑ । अ॒ध॒त्त॒ । ततः॑ । वै । सः । ज्यैष्ठ्य᳚म् । अ॒ग॒च्छ॒त् । यः । ए॒वम् । वि॒द्वान् । अ॒ग्निम् । चि॒नु॒ते ( ) । ज्यैष्ठ्य᳚म् । ए॒व । ग॒च्छ॒ति॒ ॥ \textbf{  29} \newline
                  \newline
                      (पृ॒थि॒वीं ॅय॑च्छ॒ - यथ् स्व॑यमातृ॒ण्णा उ॑प॒धाय॑ - धेह्यृ॒ - चा - ग्नि - श्चि॑नु॒ते - त्रीणि॑ च)  \textbf{(A6)} \newline \newline
                                \textbf{ TS 5.7.7.1} \newline
                  यत् । आकू॑ता॒दित्या - कू॒ता॒त् । स॒मसु॑स्रो॒दिति॑ सं - असु॑स्रोत् । हृ॒दः । वा॒ । मन॑सः । वा॒ । संभृ॑त॒मिति॒ सं - भृ॒त॒म् । चक्षु॑षः । वा॒ ॥ तम् । अनु॑ । प्रेति॑ । इ॒हि॒ । सु॒कृ॒तस्येति॑ सु - कृ॒तस्य॑ । लो॒कम् । यत्र॑ । ऋष॑यः । प्र॒थ॒म॒जा इति॑ प्रथम-जाः । ये । पु॒रा॒णाः ॥ ए॒तम् । स॒ध॒स्थेति॑ सध - स्थ॒ । परीति॑ । ते॒ । द॒दा॒मि॒ । यम् । आ॒वहा॒दित्या᳚ - वहा᳚त् । शे॒व॒धिमिति॑ शेव - धिम् । जा॒तवे॑दा॒ इति॑ जा॒त - वे॒दाः॒ ॥ अ॒न्वा॒ग॒न्तेत्य॑नु-आ॒ग॒न्ता । य॒ज्ञ्प॑ति॒रिति॑ य॒ज्ञ्-प॒तिः॒ । वः॒ । अत्र॑ । तम् । स्म॒ । जा॒नी॒त॒ । प॒र॒मे । व्यो॑म॒न्निति॒ वि-ओ॒म॒न्न् ॥ जा॒नी॒तात् । ए॒न॒म् । प॒र॒मे । व्यो॑म॒न्निति॒ वि - ओ॒म॒न्न् । देवाः᳚ । स॒ध॒स्था॒ इति॑ सध - स्थाः॒ । वि॒द । रू॒पम् । अ॒स्य॒ ॥ यत् । आ॒गच्छा॒दित्या᳚ - गच्छा᳚त् । \textbf{  30} \newline
                  \newline
                                \textbf{ TS 5.7.7.2} \newline
                  प॒थिभि॒रिति॑ प॒थि - भिः॒ । दे॒व॒यानै॒रिति॑ देव - यानैः᳚ । इ॒ष्टा॒पू॒र्ते इती᳚ष्टा-पू॒र्ते । कृ॒णु॒ता॒त् । आ॒विः । अ॒स्मै॒ ॥ सम् । प्रेति॑ । च्य॒व॒द्ध्व॒म् । अन्विति॑ । सम् । प्रेति॑ । या॒त॒ । अग्ने᳚ । प॒थः । दे॒व॒याना॒निति॑ देव - यानान्॑ । कृ॒णु॒द्ध्व॒म् ॥ अ॒स्मिन्न् । स॒धस्थ॒ इति॑ स॒ध - स्थे॒ । अधीति॑ । उत्त॑रस्मि॒न्नित्युत् - त॒र॒स्मि॒न्न् । विश्वे᳚ । दे॒वाः॒ । यज॑मानः । च॒ । सी॒द॒त॒ ॥ प्र॒स्त॒रेणेति॑ प्र - स्त॒रेण॑ । प॒रि॒धिनेति॑ परि - धिना᳚ । स्रु॒चा । वेद्या᳚ । च॒ । ब॒र्॒.हिषा᳚ ॥ ऋ॒चा । इ॒मम् । य॒ज्ञ्म् । नः॒ । व॒ह॒ । सुवः॑ । दे॒वेषु॑ । गन्त॑वे ॥ यत् । इ॒ष्टम् । यत् । प॒रा॒दान॒मिति॑ परा - दान᳚म् । यत् । द॒त्तम् । या । च॒ । दक्षि॑णा ॥ तत् । \textbf{  31} \newline
                  \newline
                                \textbf{ TS 5.7.7.3} \newline
                  अ॒ग्निः । वै॒श्व॒क॒र्म॒ण इति॑ वैश्व - क॒र्म॒णः । सुवः॑ । दे॒वेषु॑ । नः॒ । द॒ध॒त् ॥ येन॑ । स॒हस्र᳚म् । वह॑सि । येन॑ । अ॒ग्ने॒ । स॒र्व॒वे॒द॒समिति॑ सर्व - वे॒द॒सम् ॥ तेन॑ । इ॒मम् । य॒ज्ञ्म् । नः॒ । व॒ह॒ । सुवः॑ । दे॒वेषु॑ । गन्त॑वे ॥ येन॑ । अ॒ग्ने॒ । दक्षि॑णाः । यु॒क्ताः । य॒ज्ञ्म् । वह॑न्ति । ऋ॒त्विजः॑ ॥ तेन॑ । इ॒मम् । य॒ज्ञ्म् । नः॒ । व॒ह॒ । सुवः॑ । दे॒वेषु॑ । गन्त॑वे ॥ येन॑ । अ॒ग्ने॒ । सु॒कृत॒ इति॑ सु - कृतः॑ । प॒था । मधोः᳚ । धाराः᳚ । व्या॒न॒शुरिति॑ वि - आ॒न॒शुः ॥ तेन॑ । इ॒मम् । य॒ज्ञ्म् । नः॒ । व॒ह॒ । सुवः॑ । दे॒वेषु॑ । गन्त॑वे ( ) ॥ यत्र॑ । धाराः᳚ । अन॑पेता॒ इत्यन॑प - इ॒ताः॒ । मधोः᳚ । घृ॒तस्य॑ । च॒ । याः ॥ तत् । अ॒ग्निः । वै॒श्व॒क॒र्म॒ण इति॑ वैश्व - क॒र्म॒णः । सुवः॑ । दे॒वेषु॑ । नः॒ । द॒ध॒त् ॥ \textbf{  32} \newline
                  \newline
                      (आ॒गच्छा॒त् - त - द्वया॑न॒शु स्तेने॒मं ॅय॒ज्ञ्ं नो॑ वह॒ सुव॑र्दे॒वेषु॒ गन्त॑वे॒ - चतु॑र्दश च)  \textbf{(A7)} \newline \newline
                                \textbf{ TS 5.7.8.1} \newline
                  याः । ते॒ । अ॒ग्ने॒ । स॒मिध॒ इति॑ सं - इधः॑ । यानि॑ । धाम॑ । या । जि॒ह्वा । जा॒त॒वे॒द॒ इति॑ जात-वे॒दः॒ । यः । अ॒र्चिः ॥ ये । ते॒ । अ॒ग्ने॒ । मे॒डयः॑ । ये । इन्द॑वः । तेभिः॑ । आ॒त्मान᳚म् । चि॒नु॒हि॒ । प्र॒जा॒नन्निति॑ प्र - जा॒नन्न् ॥ उ॒थ्स॒न्न॒य॒ज्ञ् इत्यु॑थ्सन्न - य॒ज्ञ्ः । वै । ए॒षः । यत् । अ॒ग्निः । किम् । वा॒ । अह॑ । ए॒तस्य॑ । क्रि॒यते᳚ । किम् । वा॒ । न । यत् । वै । अ॒द्ध्व॒र्युः । अ॒ग्नेः । चि॒न्वन्न् । अ॒न्त॒रेतीय॑न्तः - एति॑ । आ॒त्मनः॑ । वै । तत् । अ॒न्तः । ए॒ति॒ । याः । ते॒ । अ॒ग्ने॒ । स॒मिध॒ इति॑ सं - इधः॑ । यानि॑ । \textbf{  33} \newline
                  \newline
                                \textbf{ TS 5.7.8.2} \newline
                  धाम॑ । इति॑ । आ॒ह॒ । ए॒षा । वै । अ॒ग्नेः । स्व॒यं॒चि॒तिरिति॑ स्वयं - चि॒तिः । अ॒ग्निः । ए॒व । तत् । अ॒ग्निम् । चि॒नो॒ति । न । अ॒द्ध्व॒र्युः । आ॒त्मनः॑ । अ॒न्तः । ए॒ति॒ । चत॑स्रः । आशाः᳚ । प्रेति॑ । च॒र॒न्तु॒ । अ॒ग्नयः॑ । इ॒मम् । नः॒ । य॒ज्ञ्म् । न॒य॒तु॒ । प्र॒जा॒नन्निति॑ प्र - जा॒नन्न् ॥ घृ॒तम् । पिन्वन्न्॑ । अ॒जर᳚म् । सु॒वीर॒मिति॑ सु-वीर᳚म् । ब्रह्म॑ । स॒मिदिति॑ सं - इत् । भ॒व॒ति॒ । आहु॑तीना॒मित्या - हु॒ती॒ना॒म् ॥ सु॒व॒र्गायेति॑ सुवः - गाय॑ । वै । ए॒षः । लो॒काय॑ । उपेति॑ । धी॒य॒ते॒ । यत् । कू॒र्मः । चत॑स्रः । आशाः᳚ । प्रेति॑ । च॒र॒न्तु॒ । अ॒ग्नयः॑ । इति॑ । आ॒ह॒ । \textbf{  34} \newline
                  \newline
                                \textbf{ TS 5.7.8.3} \newline
                  दिशः॑ । ए॒व । ए॒तेन॑ । प्रेति॑ । जा॒ना॒ति॒ । इ॒मम् । नः । य॒ज्ञ्म् । न॒य॒तु॒ । प्र॒जा॒नन्निति॑ प्र - जा॒नन्न् । इति॑ । आ॒ह॒ । सु॒व॒र्गस्येति॑ सुवः - गस्य॑ । लो॒कस्य॑ । अ॒भिनी᳚त्या॒ इत्य॒भि-नी॒त्यै॒ । ब्रह्म॑ । स॒मिदिति॑ सं - इत् । भ॒व॒ति॒ । आहु॑तीना॒मित्या - हु॒ती॒ना॒म् । इति॑ । आ॒ह॒ । ब्रह्म॑णा । वै । दे॒वाः । सु॒व॒र्गमिति॑ सुवः - गम् । लो॒कम् । आ॒य॒न्न् । यत् । ब्रह्म॑ण्व॒त्येति॒ ब्रह्मण्॑ - व॒त्या॒ । उ॒प॒दधा॒तीत्यु॑प - दधा॑ति । ब्रह्म॑णा । ए॒व । तत् । यज॑मानः । सु॒व॒र्गमिति॑ सुवः - गम् । लो॒कम् । ए॒ति॒ । प्र॒जाप॑ति॒रिति॑ प्र॒जा - प॒तिः॒ । वै । ए॒षः । यत् । अ॒ग्निः । तस्य॑ । प्र॒जा इति॑ प्र - जाः । प॒शवः॑ । छन्दाꣳ॑सि । रू॒पम् । सर्वान्॑ । वर्णान्॑ । इष्ट॑कानाम् ( ) । कु॒र्या॒त् । रू॒पेण॑ । ए॒व । प्र॒जामिति॑ प्र - जाम् । प॒शून् । छन्दाꣳ॑सि । अवेति॑ । रु॒न्धे॒ । अथो॒ इति॑ । प्र॒जाभ्य॒ इति॑ प्र - जाभ्यः॑ । ए॒व । ए॒न॒म् । प॒शुभ्य॒ इति॑ प॒शु-भ्यः॒ । छन्दो᳚भ्य॒ इति॒ छन्दः॑ - भ्यः॒ । अ॒व॒रुद्ध्येत्य॑व - रुद्ध्य॑ । चि॒नु॒ते॒ ॥ \textbf{  35} \newline
                  \newline
                      (यान्य॒ - ग्नय॒ इत्या॒हे - ष्ट॑कानाꣳ॒॒ - षोड॑श च)  \textbf{(A8)} \newline \newline
                                \textbf{ TS 5.7.9.1} \newline
                  मयि॑ । गृ॒ह्णा॒मि॒ । अग्रे᳚ । अ॒ग्निम् । रा॒यः । पोषा॑य । सु॒प्र॒जा॒स्त्वायेति॑ सुप्रजाः - त्वाय॑ । सु॒वीर्या॒येति॑ सु - वीर्या॑य ॥ मयि॑ । प्र॒जामिति॑ प्र - जाम् । मयि॑ । वर्चः॑ । द॒धा॒मि॒ । अरि॑ष्टाः । स्या॒म॒ । त॒नुवा᳚ । सु॒वीरा॒ इति॑ सु - वीराः᳚ ॥ यः । नः॒ । अ॒ग्निः । पि॒त॒रः॒ । हृ॒थ्स्विति॑ हृत् - सु । अ॒न्तः । अम॑र्त्यः । मर्त्यान्॑ । आ॒वि॒वेशेत्या᳚ - वि॒वेश॑ ॥ तम् । आ॒त्मन्न् । परीति॑ । गृ॒ह्णी॒म॒हे॒ । व॒यम् । मा । सः । अ॒स्मान् । अ॒व॒हायेत्य॑व - हाय॑ । परेति॑ । गा॒त् ॥ यत् । अ॒द्ध्व॒र्युः । आ॒त्मन्न् । अ॒ग्निम् । अगृ॑हीत्वा । अ॒ग्निम् । चि॒नु॒यात् । यः । अ॒स्य॒ । स्वः । अ॒ग्निः । तम् । अपीति॑ । \textbf{  36} \newline
                  \newline
                                \textbf{ TS 5.7.9.2} \newline
                  यज॑मानाय । चि॒नु॒या॒त् । अ॒ग्निम् । खलु॑ । वै । प॒शवः॑ । अनु॑ । उपेति॑ । ति॒ष्ठ॒न्ते॒ । अ॒प॒क्रामु॑का॒ इत्य॑प - क्रामु॑काः । अ॒स्मा॒त् । प॒शवः॑ । स्युः॒ । मयि॑ । गृ॒ह्णा॒मि॒ । अग्रे᳚ । अ॒ग्निम् । इति॑ । आ॒ह॒ । आ॒त्मन्न् । ए॒व । स्वम् । अ॒ग्निम् । दा॒धा॒र॒ । न । अ॒स्मा॒त् । प॒शवः॑ । अपेति॑ । क्रा॒म॒न्ति॒ । ब्र॒ह्म॒वा॒दिन॒ इति॑ ब्रह्म-वा॒दिनः॑ । व॒द॒न्ति॒ । यत् । मृत् । च॒ । आपः॑ । च॒ । अ॒ग्नेः । अ॒ना॒द्यम् । अथ॑ । कस्मा᳚त् । मृ॒दा । च॒ । अ॒द्भिरित्य॑त्-भिः । च॒ । अ॒ग्निः । ची॒य॒ते॒ । इति॑ । यत् । अ॒द्भिरित्य॑त् - भिः । सं॒ॅयौतीति॑ सं - यौति॑ । \textbf{  37} \newline
                  \newline
                                \textbf{ TS 5.7.9.3} \newline
                  आपः॑ । वै । सर्वाः᳚ । दे॒वताः᳚ । दे॒वता॑भिः । ए॒व । ए॒न॒म् । समिति॑ । सृ॒ज॒ति॒ । यत् । मृ॒दा । चि॒नोति॑ । इ॒यम् । वै । अ॒ग्निः । वै॒श्वा॒न॒रः । अ॒ग्निना᳚ । ए॒व । तत् । अ॒ग्निम् । चि॒नो॒ति॒ । ब्र॒ह्म॒वा॒दिन॒ इति॑ ब्रह्म - वा॒दिनः॑ । व॒द॒न्ति॒ । यत् । मृ॒दा । च॒ । अ॒द्भिरित्य॑त् - भिः । च॒ । अ॒ग्निः । ची॒यते᳚ । अथ॑ । कस्मा᳚त् । अ॒ग्निः । उ॒च्य॒ते॒ । इति॑ । यत् । छन्दो॑भि॒रिति॒ छन्दः॑ - भिः॒ । चि॒नोति॑ । अ॒ग्नयः॑ । वै । छन्दाꣳ॑सि । तस्मा᳚त् । अ॒ग्निः । उ॒च्य॒ते॒ । अथो॒ इति॑ । इ॒यम् । वै । अ॒ग्निः । वै॒श्वा॒न॒रः॒ । यत् । \textbf{  38} \newline
                  \newline
                                \textbf{ TS 5.7.9.4} \newline
                  मृ॒दा । चि॒नोति॑ । तस्मा᳚त् । अ॒ग्निः । उ॒च्य॒ते॒ । हि॒र॒ण्ये॒ष्ट॒का इति॑ हिरण्य - इ॒ष्ट॒काः । उपेति॑ । द॒धा॒ति॒ । ज्योतिः॑ । वै । हिर॑ण्यम् । ज्योतिः॑ । ए॒व । अ॒स्मि॒न्न् । द॒धा॒ति॒ । अथो॒ इति॑ । तेजः॑ । वै । हिर॑ण्यम् । तेजः॑ । ए॒व । आ॒त्मन्न् । ध॒त्ते॒ । यः । वै । अ॒ग्निम् । स॒र्वतो॑मुख॒मिति॑ स॒र्वतः॑ - मु॒ख॒म् । चि॒नु॒ते । सर्वा॑सु । प्र॒जास्विति॑ प्र - जासु॑ । अन्न᳚म् । अ॒त्ति॒ । सर्वाः᳚ । दिशः॑ । अ॒भीति॑ । ज॒य॒ति॒ । गा॒य॒त्रीम् । पु॒रस्ता᳚त् । उपेति॑ । द॒धा॒ति॒ । त्रि॒ष्टुभ᳚म् । द॒क्षि॒ण॒तः । जग॑तीम् । प॒श्चात् । अ॒नु॒ष्टुभ॒मित्य॑नु-स्तुभ᳚म् । उ॒त्त॒र॒त इत्यु॑त्-त॒र॒तः । प॒ङ्क्तिम् । मद्ध्य᳚ । ए॒षः । वै ( ) । अ॒ग्निः । स॒र्वतो॑मुख॒ इति॑ स॒र्वतः॑ - मु॒खः॒ । तम् । यः । ए॒वम् । वि॒द्वान् । चि॒नु॒ते । सर्वा॑सु । प्र॒जास्विति॑ प्र-जासु॑ । अन्न᳚म् । अ॒त्ति॒ । सर्वाः᳚ । दिशः॑ । अ॒भीति॑ । ज॒य॒ति॒ । अथो॒ इति॑ । दि॒शि । ए॒व । दिश᳚म् । प्रेति॑ । व॒य॒ति॒ । तस्मा᳚त् । दि॒शि । दिक् । प्रोतेति॒ प्र - उ॒ता॒ ॥ \textbf{  39 } \newline
                  \newline
                      (अपि॑-सं॒ॅयौति॑-वैश्वान॒रो य-दे॒ष वै-पञ्च॑विꣳशतिश्च)  \textbf{(A9)} \newline \newline
                                \textbf{ TS 5.7.10.1} \newline
                  प्र॒जाप॑ति॒रिति॑ प्र॒जा - प॒तिः॒ । अ॒ग्निम् । अ॒सृ॒ज॒त॒ । सः । अ॒स्मा॒त् । सृ॒ष्टः । प्राङ् । प्रेति॑ । अ॒द्र॒व॒त् । तस्मै᳚ । अश्व᳚म् । प्रतीति॑ । आ॒स्य॒त् । सः । द॒क्षि॒णा । एति॑ । अ॒व॒र्त॒त॒ । तस्मै᳚ । वृ॒ष्णिम् । प्रतीति॑ । आ॒स्य॒त् । सः । प्र॒त्यङ् । एति॑ । अ॒व॒र्त॒त॒ । तस्मै᳚ । ऋ॒ष॒भम् । प्रतीति॑ । आ॒स्य॒त् । सः । उदङ्॑ । एति॑ । अ॒व॒र्त॒त॒ । तस्मै᳚ । ब॒स्तम् । प्रतीति॑ । आ॒स्य॒त् । सः । ऊ॒द्‌र्ध्वः । अ॒द्र॒व॒त् । तस्मै᳚ । पुरु॑षम् । प्रतीति॑ । आ॒स्य॒त् । यत् । प॒शु॒शी॒र्॒.षाणीति॑ पशु - शी॒र्॒.षाणि॑ । उ॒प॒दधा॒तीत्यु॑प - दधा॑ति । स॒र्वतः॑ । ए॒व । ए॒न॒म् । \textbf{  40} \newline
                  \newline
                                \textbf{ TS 5.7.10.2} \newline
                  अ॒व॒रुद्ध्येत्य॑व - रुद्ध्य॑ । चि॒नु॒ते॒ । ए॒ताः । वै । प्रा॒ण॒भृत॒ इति॑ प्राण - भृतः॑ । चक्षु॑ष्मतीः । इष्ट॑काः । यत् । प॒शु॒शी॒र्.॒षाणीति॑ पशु - शी॒र्.॒षाणि॑ । यत् । प॒शु॒शी॒र्.॒षाणीति॑ पशु - शी॒र्.॒षाणि॑ । उ॒प॒दधा॒तीत्यु॑प - दधा॑ति । ताभिः॑ । ए॒व । यज॑मानः । अ॒मुष्मिन्न्॑ । लो॒के । प्रेति॑ । अ॒नि॒ति॒ । अथो॒ इति॑ । ताभिः॑ । ए॒व । अ॒स्मै॒ । इ॒मे । लो॒काः । प्रेति॑ । भा॒न्ति॒ । मृ॒दा । अ॒भि॒लिप्येत्य॑भि - लिप्य॑ । उपेति॑ । द॒धा॒ति॒ । मे॒द्ध्य॒त्वायेति॑ मेद्ध्य-त्वाय॑ । प॒शुः । वै । ए॒षः । यत् । अ॒ग्निः । अन्न᳚म् । प॒शवः॑ । ए॒षः । खलु॑ । वै । अ॒ग्निः । यत् । प॒शु॒शी॒र्.॒षाणीति॑ पशु-शी॒र्.॒षाणि॑ । यम् । का॒मये॑त । कनी॑यः । अ॒स्य॒ । अन्न᳚म् । \textbf{  41} \newline
                  \newline
                                \textbf{ TS 5.7.10.3} \newline
                  स्या॒त् । इति॑ । स॒तं॒रामिति॑ सं - त॒राम् । तस्य॑ । प॒शु॒शी॒र्.॒षाणीति॑ पशु - शी॒र्.॒षाणि॑ । उपेति॑ । द॒द्ध्या॒त् । कनी॑यः । ए॒व । अ॒स्य॒ । अन्न᳚म् । भ॒व॒ति॒ । यम् । का॒मये॑त । स॒माव॑त् । अ॒स्य॒ । अन्न᳚म् । स्या॒त् । इति॑ । म॒द्ध्य॒तः । तस्य॑ । उपेति॑ । द॒द्ध्या॒त् । स॒माव॑त् । ए॒व । अ॒स्य॒ । अन्न᳚म् । भ॒व॒ति॒ । यम् । का॒मये॑त । भूयः॑ । अ॒स्य॒ । अन्न᳚म् । स्या॒त् । इति॑ । अन्ते॑षु । तस्य॑ । व्यु॒दूह्येति॑ वि - उ॒दूह्य॑ । उपेति॑ । द॒द्ध्या॒त् । अ॒न्त॒तः । ए॒व । अ॒स्मै॒ । अन्न᳚म् । अवेति॑ । रु॒न्धे॒ । भूयः॑ । अ॒स्य॒ । अन्न᳚म् । भ॒व॒ति॒ ( ) ॥ \textbf{  42 } \newline
                  \newline
                      (ए॒न॒- म॒स्यान्नं॒ - भूयो॒ऽस्याऽन्नं॑ भवति)  \textbf{(A10)} \newline \newline
                                \textbf{ TS 5.7.11.1} \newline
                  स्ते॒गान् । दꣳष्ट्रा᳚भ्याम् । म॒ण्डूकान्॑ । जंभ्ये॑भिः । आद॑काम् । खा॒देन॑ । ऊर्ज᳚म् । सꣳ॒॒सू॒देनेति॑ सं-सू॒देन॑ । अर॑ण्यम् । जांबी॑लेन । मृद᳚म् । ब॒र्स्वे॑भिः । शर्क॑राभिः । अव॑काम् । अव॑काभिः । शर्क॑राम् । उ॒थ्सा॒देनेत्यु॑त् - सा॒देन॑ । जि॒ह्वाम् । अ॒व॒क्र॒न्देनेत्य॑व - क्र॒न्देन॑ । तालु᳚म् । सर॑स्वतीम् । जि॒ह्वा॒ग्रेणेति॑ जिह्वा - अ॒ग्रेण॑ ॥ \textbf{  43 } \newline
                  \newline
                      (स्ते॒गान् - द्वाविꣳ॑शतिः)  \textbf{(A11)} \newline \newline
                                \textbf{ TS 5.7.12.1} \newline
                  वाज᳚म् । हनू᳚भ्या॒मिति॒ हनु॑ - भ्या॒म् । अ॒पः । आ॒स्ये॑न । आ॒दि॒त्यान् । श्मश्रु॑भि॒रिति॒ श्मश्रु॑ - भिः॒ । उ॒प॒या॒ममित्यु॑प - या॒मम् । अध॑रेण । ओष्ठे॑न । सत् । उत्त॑रे॒णेत्युत् - त॒रे॒ण॒ । अन्त॑रेण । अ॒नू॒का॒शमित्य॑नु - का॒शम् । प्र॒का॒शेनेति॑ प्र - का॒शेन॑ । बाह्य᳚म् । स्त॒न॒यि॒त्नुम् । नि॒र्बा॒धेनेति॑ निः-बा॒धेन॑ । सू॒र्या॒ग्नी इति॑ सूर्य-अ॒ग्नी । चक्षु॑र्भ्या॒मिति॒ चक्षुः॑ - भ्या॒म् । वि॒द्युता॒विति॑ वि - द्युतौ᳚ । क॒नान॑काभ्याम् । अ॒शनि᳚म् । म॒स्तिष्के॑ण । बल᳚म् । म॒ज्जभि॒रिति॑ म॒ज्ज - भिः॒ ॥ \textbf{  44 } \newline
                  \newline
                      (वाजं॒ पञ्च॑विꣳशतिः)  \textbf{(A12)} \newline \newline
                                \textbf{ TS 5.7.13.1} \newline
                  कू॒र्मान् । श॒फैः । अ॒च्छला॑भिः । क॒पिञ्ज॑लान् । साम॑ । कुष्ठि॑काभिः । ज॒वम् । जङ्घा॑भिः । अ॒ग॒दम् । जानु॑भ्या॒मिति॒ जानु॑-भ्या॒म् । वी॒र्य᳚म् । कु॒हाभ्या᳚म् । भ॒यम् । प्र॒चा॒लाभ्या॒मिति॑ प्र - चा॒लाभ्या᳚म् । गुहा᳚ । उ॒प॒प॒क्षाभ्या॒मित्यु॑प - प॒क्षाभ्या᳚म् । अ॒श्विनौ᳚ । अꣳसा᳚भ्याम् । अदि॑तिम् । शी॒र्ष्णा । निर्.ऋ॑ति॒मिति॒ निः - ऋ॒ति॒म् । निर्जा᳚ल्मके॒नेति॒ निः-जा॒ल्म॒के॒न॒ । शी॒र्ष्णा ॥ \textbf{  45 } \newline
                  \newline
                      (कू॒र्मान्-त्रयो॑विꣳशतिः)  \textbf{(A13)} \newline \newline
                                \textbf{ TS 5.7.14.1} \newline
                  योक्त्र᳚म् । गृद्ध्रा॑भिः । यु॒गम् । आन॑ते॒नेत्या - न॒ते॒न॒ । चि॒त्तम् । मन्या॑भिः । स॒क्र्ॐ॒शानिति॑ सं - क्रो॒शान् । प्रा॒णैरिति॑ प्र - अ॒नैः । प्र॒का॒शेनेति॑ प्र - का॒शेन॑ । त्वच᳚म् । प॒रा॒का॒शेनेति॑ परा - का॒शेन॑ । अन्त॑राम् । म॒शकान्॑ । केशैः᳚ । इन्द्र᳚म् । स्वप॒सेति॑ सु - अप॑सा । वहे॑न । बृह॒स्पति᳚म् । श॒कु॒नि॒सा॒देनेति॑ शकुनि - सा॒देन॑ । रथ᳚म् । उ॒ष्णिहा॑भिः ॥ \textbf{  46} \newline
                  \newline
                      (योक्त्र॒ - मेक॑विꣳशतिः)  \textbf{(A14)} \newline \newline
                                \textbf{ TS 5.7.15.1} \newline
                  मि॒त्रावरु॑णा॒विति॑ मि॒त्रा - वरु॑णौ । श्रोणी᳚भ्या॒मिति॒ श्रोणि॑ - भ्या॒म् । इ॒न्द्रा॒ग्नी इती᳚न्द्र - अ॒ग्नी । शि॒ख॒ण्डाभ्या᳚म् । इन्द्रा॒बृह॒स्पती॒ इतीन्द्रा᳚ - बृह॒स्पती᳚ । ऊ॒रुभ्या॒मित्यू॒रु - भ्या॒म् । इन्द्रा॒विष्णू॒ इतीन्द्रा᳚ - विष्णू᳚ । अ॒ष्ठी॒वद्भ्या॒मित्य॑ष्ठी॒वत् - भ्या॒म् । स॒वि॒तार᳚म् । पुच्छे॑न । ग॒न्ध॒र्वान् । शेपे॑न । अ॒फ्स॒रसः॑ । मु॒ष्काभ्या᳚म् । पव॑मानम् । पा॒युना᳚ । प॒वित्र᳚म् । पोत्रा᳚भ्याम् । आ॒क्रम॑ण॒मित्या᳚ - क्रम॑णम् । स्थू॒राभ्या᳚म् । प्र॒ति॒क्रम॑ण॒मिति॑ प्रति - क्रम॑णम् । कुष्ठा᳚भ्याम् ॥ \textbf{  47} \newline
                  \newline
                      (मि॒त्रावरु॑णौ॒ - द्वाविꣳ॑शतिः)  \textbf{(A15)} \newline \newline
                                \textbf{ TS 5.7.16.1} \newline
                  इन्द्र॑स्य । क्रो॒डः । अदि॑त्यै । पा॒ज॒स्य᳚म् । दि॒शाम् । ज॒त्रवः॑ । जी॒मूतान्॑ । हृ॒द॒यौ॒प॒शाभ्या॒मिति॑ हृदय - औ॒प॒शाभ्या᳚म् । अ॒न्तरि॑क्षम् । पु॒रि॒तता᳚ । नभः॑ । उ॒द॒र्ये॑ण । इ॒न्द्रा॒णीम् । प्ली॒ह्ना । व॒ल्मीकान्॑ । क्लो॒म्ना । गि॒रीन् । प्ला॒शिभि॒रिति॑ प्ला॒शि-भिः॒ । स॒मु॒द्रम् । उ॒दरे॑ण । वै॒श्वा॒न॒रम् । भस्म॑ना ॥ \textbf{  48} \newline
                  \newline
                      (इन्द्र॑स्य॒ - द्वावि॑शतिः॒)  \textbf{(A16)} \newline \newline
                                \textbf{ TS 5.7.17.1} \newline
                  पू॒ष्णः । व॒नि॒ष्ठुः । अ॒न्धा॒हेरित्य॑न्ध-अ॒हेः । स्थू॒र॒गु॒देति॑ स्थूर - गु॒दा । स॒र्पान् । गुदा॑भिः । ऋ॒तून् । पृ॒ष्टीभि॒रिति॑ पृ॒ष्टि - भिः॒ । दिव᳚म् । पृ॒ष्ठेन॑ । वसू॑नाम् । प्र॒थ॒मा । कीक॑सा । रु॒द्राणा᳚म् । द्वि॒तीया᳚ । आ॒दि॒त्याना᳚म् । तृ॒तीया᳚ । अङ्गि॑रसाम् । च॒तु॒र्थी । सा॒द्ध्याना᳚म् । प॒ञ्च॒मी । विश्वे॑षाम् । दे॒वाना᳚म् । ष॒ष्ठी ॥ \textbf{  49} \newline
                  \newline
                      (पू॒ष्ण - श्चतु॑र्विꣳशतिः)  \textbf{(A17)} \newline \newline
                                \textbf{ TS 5.7.18.1} \newline
                  ओजः॑ । ग्री॒वाभिः॑ । निर्.ऋ॑ति॒मिति॒ निः-ऋ॒ति॒म् । अ॒स्थभि॒रित्य॒स्थ-भिः॒ । इन्द्र᳚म् । स्वप॒सेति॑ सु - अप॑सा । वहे॑न । रु॒द्रस्य॑ । वि॒च॒ल इति॑ वि - च॒लः । स्क॒न्धः । अ॒हो॒रा॒त्रयो॒रित्य॑हः - रा॒त्रयोः᳚ । द्वि॒तीयः॑ । अ॒द्‌र्ध॒मा॒साना॒मित्य॑द्‌र्ध - मा॒साना᳚म् । तृ॒तीयः॑ । मा॒साम् । च॒तु॒र्थः । ऋ॒तू॒नाम् । प॒ञ्च॒मः । सं॒ॅव॒थ्स॒रस्येति॑ सं - व॒थ्स॒रस्य॑ । ष॒ष्ठः ॥ \textbf{  50} \newline
                  \newline
                      (ओजो॑ - विꣳश॒तिः)  \textbf{(A18)} \newline \newline
                                \textbf{ TS 5.7.19.1} \newline
                  आ॒न॒न्दमित्या᳚ - न॒न्दम् । न॒न्दथु॑ना । काम᳚म् । प्र॒त्या॒साभ्या॒मिति॑ प्रति - आ॒साभ्या᳚म् । भ॒यम् । शि॒ती॒मभ्या॒मिति॑ शिती॒म - भ्या॒म् । प्र॒शिष॒मिति॑ प्र - शिष᳚म् । प्र॒शा॒साभ्या॒मिति॑ प्र - शा॒साभ्या᳚म् । सू॒र्या॒च॒न्द्र॒मसा॒विति॑ सूर्या - च॒न्द्र॒मसौ᳚ । वृक्या᳚भ्याम् । श्या॒म॒श॒ब॒लाविति॑ श्याम - श॒ब॒लौ । मत॑स्नाभ्याम् । व्यु॑ष्टि॒मिति॒ वि-उ॒ष्टि॒म् । रू॒पेण॑ । निम्रु॑क्ति॒मिति॒ नि-म्रु॒क्ति॒म् । अरू॑पेण ॥ \textbf{  51 } \newline
                  \newline
                      (आ॒न॒न्दꣳ - षोड॑श)  \textbf{(A19)} \newline \newline
                                \textbf{ TS 5.7.20.1} \newline
                  अहः॑ । माꣳ॒॒सेन॑ । रात्रि᳚म् । पीव॑सा । अ॒पः । यू॒षेण॑ । घृ॒तम् । रसे॑न । श्याम् । वस॑या । दू॒षीका॑भिः । ह्रा॒दुनि᳚म् । अश्रु॑भि॒रित्यश्रु॑ - भिः॒ । पृष्वा᳚म् । दिव᳚म् । रू॒पेण॑ । नक्ष॑त्राणि । प्रति॑रूपे॒णेति॒ प्रति॑ - रू॒पे॒ण॒ । पृ॒थि॒वीम् । चर्म॑णा । छ॒वीम् । छ॒व्या᳚ । उ॒पाकृ॑ता॒येत्यु॑प - आकृ॑ताय । स्वाहा᳚ । आल॑ब्धा॒येत्या - ल॒ब्धा॒य॒ । स्वाहा᳚ । हु॒ताय॑ । स्वाहा᳚ ॥ \textbf{  52 } \newline
                  \newline
                      (अह॑र॒ - ष्टाविꣳ॑शतिः)  \textbf{(A20)} \newline \newline
                                \textbf{ TS 5.7.21.1} \newline
                  अ॒ग्नेः । प॒क्ष॒तिः । सर॑स्वत्यै । निप॑क्षति॒रिति॒ नि-प॒क्ष॒तिः॒ । सोम॑स्य । तृ॒तीया᳚ । अ॒पाम् । च॒तु॒र्थी । ओष॑धीनाम् । प॒ञ्च॒मी । सं॒ॅव॒थ्स॒रस्येति॑ सं - व॒थ्स॒रस्य॑ । ष॒ष्ठी । म॒रुता᳚म् । स॒प्त॒मी । बृह॒स्पतेः᳚ । अ॒ष्ट॒मी । मि॒त्रस्य॑ । न॒व॒मी । वरु॑णस्य । द॒श॒मी । इन्द्र॑स्य । ए॒का॒द॒शी । विश्वे॑षाम् । दे॒वाना᳚म् । द्वा॒द॒शी । द्यावा॑पृथि॒व्योरिति॒ द्यावा᳚-पृ॒थि॒व्योः । पा॒र्श्वम् । य॒मस्य॑ । पा॒टू॒रः ॥ \textbf{  53 } \newline
                  \newline
                      (अ॒ग्ने-रेका॒न्न त्रिꣳ॒॒शत्)  \textbf{(A21)} \newline \newline
                                \textbf{ TS 5.7.22.1} \newline
                  वा॒योः । प॒क्ष॒तिः । सर॑स्वतः । निप॑क्षति॒रिति॒ नि - प॒क्ष॒तिः॒ । च॒न्द्रम॑सः । तृ॒तीया᳚ । नक्ष॑त्राणाम् । च॒तु॒र्थी । स॒वि॒तुः । प॒ञ्च॒मी । रु॒द्रस्य॑ । ष॒ष्ठी । स॒र्पाणा᳚म् । स॒प्त॒मी । अ॒र्य॒म्णः । अ॒ष्ट॒मी । त्वष्टुः॑ । न॒व॒मी । धा॒तुः । द॒श॒मी । इ॒न्द्रा॒ण्याः । ए॒का॒द॒शी । अदि॑त्यै । द्वा॒द॒शी । द्यावा॑पृथि॒व्योरिति॒ द्यावा᳚ - पृ॒थि॒व्योः । पा॒र्श्वम् । य॒म्यै᳚ । पा॒टू॒रः ॥ \textbf{  54 } \newline
                  \newline
                      (वा॒यो - र॒ष्टाविꣳ॑शतिः)  \textbf{(A22)} \newline \newline
                                \textbf{ TS 5.7.23.1} \newline
                  पन्था᳚म् । अ॒नू॒वृग्भ्या॒मित्य॑नू॒वृक् - भ्या॒म् । संत॑ति॒मिति॒ सं-त॒ति॒म् । स्ना॒व॒न्या᳚भ्याम् । शुकान्॑ । पि॒त्तेन॑ । ह॒रि॒माण᳚म् । य॒क्ना । हली᳚क्ष्णान् । पा॒प॒वा॒तेनेति॑ पाप - वा॒तेन॑ । कू॒श्मान् । शक॑भि॒रिति॒ शक॑ - भिः॒ । श॒व॒र्तान् । ऊव॑द्ध्येन । शुनः॑ । वि॒शस॑ने॒नेति॑ वि - शस॑नेन । स॒र्पान् । लो॒हि॒त॒ग॒न्धेनेति॑ लोहित - ग॒न्धेन॑ । वयाꣳ॑सि । प॒क्व॒ग॒न्धेनेति॑ पक्व - ग॒न्धेन॑ । पि॒पीलि॑काः । प्र॒शा॒देनेति॑ प्र - शा॒देन॑ ॥ \textbf{  55 } \newline
                  \newline
                      (पन्थां॒ - द्वाविꣳ॑शतिः)  \textbf{(A23)} \newline \newline
                                \textbf{ TS 5.7.24.1} \newline
                  क्रमैः᳚ । अतीति॑ । अ॒क्र॒मी॒त् । वा॒जी । विश्वैः᳚ । दे॒वैः । य॒ज्ञियैः᳚ । सं॒ॅवि॒दा॒न इति॑ सं - वि॒दा॒नः ॥ सः । नः॒ । न॒य॒ । सु॒कृ॒तस्येति॑ सु - कृ॒तस्य॑ । लो॒कम् । तस्य॑ । ते॒ । व॒यम् । स्व॒धयेति॑ स्व-धया᳚ । म॒दे॒म॒ ॥ \textbf{  56} \newline
                  \newline
                      (क्रमै॑ - र॒ष्टाद॑श)  \textbf{(A24)} \newline \newline
                                \textbf{ TS 5.7.25.1} \newline
                  द्यौः । ते॒ । पृ॒ष्ठम् । पृ॒थि॒वी । स॒धस्थ॒मिति॑ स॒ध - स्थ॒म् । आ॒त्मा । अ॒न्तरि॑क्षम् । स॒मु॒द्रः । योनिः॑ । सूर्यः॑ । ते॒ । चक्षुः॑ । वातः॑ । प्रा॒ण इति॑ प्र - अ॒नः । च॒न्द्रमाः᳚ । श्रोत्र᳚म् । मासाः᳚ । च॒ । अ॒द्‌र्ध॒मा॒सा इत्य॑द्‌र्ध - मा॒साः । च॒ । पर्वा॑णि । ऋ॒तवः॑ । अङ्गा॑नि । सं॒ॅव॒थ्स॒र इति॑ सं - व॒थ्स॒रः । म॒हि॒मा ॥ \textbf{  57} \newline
                  \newline
                      (द्यौ - पञ्च॑विꣳशतिः)  \textbf{(A25)} \newline \newline
                                \textbf{ TS 5.7.26.1} \newline
                  अ॒ग्निः । प॒शुः । आ॒सी॒त् । तेन॑ । अ॒य॒ज॒न्त॒ । सः । ए॒तम् । लो॒कम् । अ॒ज॒य॒त् । यस्मिन्न्॑ । अ॒ग्निः । सः । ते॒ । लो॒कः । तम् । जे॒ष्य॒सि॒ । अथ॑ । अवेति॑ । जि॒घ्र॒ । वा॒युः । प॒शुः । आ॒सी॒त् । तेन॑ । अ॒य॒ज॒न्त॒ । सः । ए॒तम् । लो॒कम् । अ॒ज॒य॒त् । यस्मिन्न्॑ । वा॒युः । सः । ते॒ । लो॒कः । तस्मा᳚त् । त्वा॒ । अ॒न्तः । ए॒ष्या॒मि॒ । यदि॑ । न । अ॒व॒जिघ्र॒सीत्य॑व-जिघ्र॑सि । आ॒दि॒त्यः । प॒शुः । आ॒सी॒त् । तेन॑ । अ॒य॒ज॒न्त॒ । सः । ए॒तम् । लो॒कम् । अ॒ज॒य॒त् । यस्मिन्न्॑ ( ) । आ॒दि॒त्यः । सः । ते॒ । लो॒कः । तम् । जे॒ष्य॒सि॒ । यदि॑ । अ॒व॒जिघ्र॒सीत्य॑व - जिघ्र॑सि ॥ \textbf{  58 } \newline
                  \newline
                      (यस्मि॑ - न्न॒ष्टौ च॑)  \textbf{(A26)} \newline \newline
\textbf{praSna korvai with starting padams of 1 to 26 anuvAkams :-} \newline
(यो वा अय॑थादेवत॒ - न्त्वाम॑ग्न॒ - इन्द्र॑स्य॒ - चित्तिं॒ - ॅयथा॒ वै - वयो॒ वै - यदाकू॑ता॒द् - यास्ते॑ अग्ने॒ - मयि॑ गृह्णामि - प्र॒जाप॑तिः॒ सो᳚ऽस्माथ् - स्ते॒गान् - वाजं॑ - कू॒र्मान् - योक्त्रं॑ - मि॒त्रावरु॑णा॒ - विन्द्र॑स्य - पू॒ष्ण - ओज॑ - आन॒न्द - मह॑ - र॒ग्ने - र्वा॒योः - पन्थां॒ - क्रमै॒ - र्द्यौस्ते॒ - ऽग्निः प॒शुरा॑सी॒थ् - षड्विꣳ॑शतिः) \newline

\textbf{korvai with starting padams of1, 11, 21 series of pa~jcAtis :-} \newline
(यो वा - ए॒वाऽऽहु॑ति - मभवन् - प॒थिभि॑ - रव॒रुध्या॑ - ऽऽन॒न्द - म॒ष्टौ प॑ञ्च॒शत् ) \newline

\textbf{first and last padam of seventh praSnam of 5th kANDam :-} \newline
(यो वा अय॑थादेवतं॒ - ॅयद्य॑व॒जिघ्र॑सि ) \newline 


॥ हरिः॑ ॐ ॥
॥ कृष्ण यजुर्वेदीय तैत्तिरीय संहितायां पञ्चमकाण्डे सप्तमः प्रश्नः समाप्तः ॥

॥ इति पञ्चमं काण्डं ॥
=========================== \newline
\pagebreak
\pagebreak
        


\end{document}
