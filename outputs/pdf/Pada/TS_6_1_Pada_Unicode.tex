\documentclass[17pt]{extarticle}
\usepackage{babel}
\usepackage{fontspec}
\usepackage{polyglossia}
\usepackage{extsizes}



\setmainlanguage{sanskrit}
\setotherlanguages{english} %% or other languages
\setlength{\parindent}{0pt}
\pagestyle{myheadings}
\newfontfamily\devanagarifont[Script=Devanagari]{AdishilaVedic}


\newcommand{\VAR}[1]{}
\newcommand{\BLOCK}[1]{}




\begin{document}
\begin{titlepage}
    \begin{center}
 
\begin{sanskrit}
    { \Large
    ॐ नमः परमात्मने, श्री महागणपतये नमः, श्री गुरुभ्यो नमः
ह॒रिः॒ ॐ 
    }
    \\
    \vspace{2.5cm}
    \mbox{ \Huge
    6.1      षष्ठकाण्डे प्रथमः प्रश्नः - सोममन्त्रब्राह्मणनिरूपणं   }
\end{sanskrit}
\end{center}

\end{titlepage}
\tableofcontents

ॐ नमः परमात्मने, श्री महागणपतये नमः, श्री गुरुभ्यो नमः
ह॒रिः॒ ॐ \newline
6.1      षष्ठकाण्डे प्रथमः प्रश्नः - सोममन्त्रब्राह्मणनिरूपणं \newline

\addcontentsline{toc}{section}{ 6.1      षष्ठकाण्डे प्रथमः प्रश्नः - सोममन्त्रब्राह्मणनिरूपणं}
\markright{ 6.1      षष्ठकाण्डे प्रथमः प्रश्नः - सोममन्त्रब्राह्मणनिरूपणं \hfill https://www.vedavms.in \hfill}
\section*{ 6.1      षष्ठकाण्डे प्रथमः प्रश्नः - सोममन्त्रब्राह्मणनिरूपणं }
                                \textbf{ TS 6.1.1.1} \newline
                  प्रा॒चीन॑वꣳश॒मिति॑ प्रा॒चीन॑ - वꣳ॒॒श॒म् । क॒रो॒ति॒ । दे॒व॒म॒नु॒ष्या इति॑ देव - म॒नु॒ष्याः । दिशः॑ । वीति॑ । अ॒भ॒ज॒न्त॒ । प्राची᳚म् । दे॒वाः । द॒क्षि॒णा । पि॒तरः॑ । प्र॒तीची᳚म् । म॒नु॒ष्याः᳚ । उदी॑चीम् । रु॒द्राः । यत् । प्रा॒चीन॑वꣳश॒मिति॑ प्रा॒चीन॑ - वꣳ॒॒श॒म् । क॒रोति॑ । दे॒व॒लो॒कमिति॑ देव - लो॒कम् । ए॒व । तत् । यज॑मानः । उ॒पाव॑र्तत॒ इत्यु॑प - आव॑र्तते । परीति॑ । श्र॒य॒ति॒ । अ॒न्तर्.हि॑त॒ इत्य॒न्तः - हि॒तः॒ । हि । दे॒व॒लो॒क इति॑ देव - लो॒कः । म॒नु॒ष्य॒लो॒कादिति॑ मनुष्य - लो॒कात् । न । अ॒स्मात् । लो॒कात् । स्वे॑तव्य॒मिति॒ सु - ए॒त॒व्य॒म् । इ॒व॒ । इति॑ । आ॒हुः॒ । कः । हि । तत् । वेद॑ । यदि॑ । अ॒मुष्मिन्न्॑ । लो॒के । अस्ति॑ । वा॒ । न । वा॒ । इति॑ । दि॒क्षु । अ॒ती॒का॒शान् । क॒रो॒ति॒ । \textbf{  1} \newline
                  \newline
                                \textbf{ TS 6.1.1.2} \newline
                  उ॒भयोः᳚ । लो॒कयोः᳚ । अ॒भिजि॑त्या॒ इत्य॒भि - जि॒त्यै॒ । के॒श॒श्म॒श्र्विति॑ केश-श्म॒श्रु । व॒प॒ते॒ । न॒खानि॑ । नीति॑ । कृ॒न्त॒ते॒ । मृ॒ता । वै । ए॒षा । त्वक् । अ॒मे॒द्ध्या । यत् । के॒श॒श्म॒श्र्विति॑ केश - श्म॒श्रु । मृ॒ताम् । ए॒व । त्वच᳚म् । अ॒मे॒द्ध्याम् । अ॒प॒हत्येत्य॑प-हत्य॑ । य॒ज्ञियः॑ । भू॒त्वा । मेध᳚म् । उपेति॑ । ए॒ति॒ । अङ्गि॑रसः । सु॒व॒र्गमिति॑ सुवः - गम् । लो॒कम् । यन्तः॑ । अ॒फ्स्वित्य॑प्-सु । दी॒क्षा॒त॒पसी॒ इति॑ दीक्षा-त॒पसी᳚ । प्रेति॑ । अ॒वे॒श॒य॒न्न् । अ॒फ्स्वित्य॑प्- सु । स्ना॒ति॒ । सा॒क्षादिति॑ स - अ॒क्षात् । ए॒व । दी॒क्षा॒त॒पसी॒ इति॑ दीक्षा - त॒पसी᳚ । अवेति॑ । रु॒न्धे॒ । ती॒र्थे । स्ना॒ति॒ । ती॒र्थे । हि । ते । ताम् । प्रेति॑ । अवे॑शयन्न् । ती॒र्थे । स्ना॒ति॒ । \textbf{  2} \newline
                  \newline
                                \textbf{ TS 6.1.1.3} \newline
                  ती॒र्थम् । ए॒व । स॒मा॒नाना᳚म् । भ॒व॒ति॒ । अ॒पः । अ॒श्ना॒ति॒ । अ॒न्त॒र॒तः । ए॒व । मेद्ध्यः॑ । भ॒व॒ति॒ । वास॑सा । दी॒क्ष॒य॒ति॒ । सौ॒म्यम् । वै । क्षौम᳚म् । दे॒वत॑या । सोम᳚म् । ए॒षः । दे॒वता᳚म् । उपेति॑ । ए॒ति॒ । यः । दीक्ष॑ते । सोम॑स्य । त॒नूः । अ॒सि॒ । त॒नुव᳚म् । मे॒ । पा॒हि॒ । इति॑ । आ॒ह॒ । स्वाम् । ए॒व । दे॒वता᳚म् । उपेति॑ । ए॒ति॒ । अथो॒ इति॑ । आ॒शिष॒मित्या᳚ - शिष᳚म् । ए॒व । ए॒ताम् । एति॑ । शा॒स्ते॒ । अ॒ग्नेः । तू॒षा॒धान॒मिति॑ तूष - आ॒धान᳚म् । वा॒योः । वा॒त॒पान॒मिति॑ वात-पान᳚म् । पि॒तृ॒णाम् । नी॒विः । ओष॑धीनाम् । प्र॒घा॒त इति॑ प्र - घा॒तः । \textbf{  3} \newline
                  \newline
                                \textbf{ TS 6.1.1.4} \newline
                  आ॒दि॒त्याना᳚म् । प्रा॒ची॒न॒ता॒न इति॑ प्राचीन - ता॒नः । विश्वे॑षाम् । दे॒वाना᳚म् । ओतुः॑ । नक्ष॑त्राणाम् । अ॒ती॒का॒शाः । तत् । वै । ए॒तत् । स॒र्व॒दे॒व॒त्य॑मिति॑ सर्व - दे॒व॒त्य᳚म् । यत् । वासः॑ । यत् । वास॑सा । दी॒क्षय॑ति । सर्वा॑भिः । ए॒व । ए॒न॒म् । दे॒वता॑भिः । दी॒क्ष॒य॒ति॒ । ब॒हिःप्रा॑ण॒ इति॑ ब॒हिः - प्रा॒णः॒ । वै । म॒नु॒ष्यः॑ । तस्य॑ । अश॑नम् । प्रा॒ण इति॑ प्र -अ॒नः । अ॒श्नाति॑ । सप्रा॑ण॒ इति॒ स - प्रा॒णः॒ । ए॒व । दी॒क्ष॒ते॒ । आशि॑तः । भ॒व॒ति॒ । यावान्॑ । ए॒व । अ॒स्य॒ । प्रा॒ण इति॑ प्र -अ॒नः । तेन॑ । स॒ह । मेध᳚म् । उपेति॑ । ए॒ति॒ । घृ॒तम् । दे॒वाना᳚म् । मस्तु॑ । पि॒तृ॒णाम् । निष्प॑क्व॒मिति॒ निः - प॒क्व॒म् । म॒नु॒ष्या॑णाम् । तत् । वै । \textbf{  4} \newline
                  \newline
                                \textbf{ TS 6.1.1.5} \newline
                  ए॒तत् । स॒र्व॒दे॒व॒त्य॑मिति॑ सर्व - दे॒व॒त्य᳚म् । यत् । नव॑नीत॒मिति॒ नव॑ - नी॒त॒म् । यत् । नव॑नीते॒नेति॒ नव॑ - नी॒ते॒न॒ । अ॒भ्य॒ङ्क्त इत्य॑भि - अ॒ङ्क्ते । सर्वाः᳚ । ए॒व । दे॒वताः᳚ । प्री॒णा॒ति॒ । प्रच्यु॑त॒ इति॒ प्र - च्यु॒तः॒ । वै । ए॒षः । अ॒स्मात् । लो॒कात् । अग॑तः । दे॒व॒लो॒कमिति॑ देव - लो॒कम् । यः । दी॒क्षि॒तः । अ॒न्त॒रा । इ॒व॒ । नव॑नीत॒मिति॒ नव॑ - नी॒त॒म् । तस्मा᳚त् । नव॑नीते॒नेति॒ नव॑ - नी॒ते॒न॒ । अ॒भीति॑ । अ॒ङ्क्ते॒ । अ॒नु॒लो॒ममित्य॑नु - लो॒मम् । यजु॑षा । व्यावृ॑त्त्या॒ इति॑ वि - आवृ॑त्त्यै । इन्द्रः॑ । वृ॒त्रम् । अ॒ह॒न्न् । तस्य॑ । क॒नीनि॑का । परेति॑ । अ॒प॒त॒त् । तत् । आञ्ज॑न॒मित्या᳚-अञ्ज॑नम् । अ॒भ॒व॒त् । यत् । आ॒ङ्क्त इत्या᳚ - अ॒ङ्क्ते । चक्षुः॑ । ए॒व । भ्रातृ॑व्यस्य । वृ॒ङ्क्ते॒ । दक्षि॑णम् । पूर्व᳚म् । एति॑ । अ॒ङ्क्ते॒ । \textbf{  5} \newline
                  \newline
                                \textbf{ TS 6.1.1.6} \newline
                  स॒व्यम् । हि । पूर्व᳚म् । म॒नु॒ष्याः᳚ । आ॒ञ्जत॒ इत्या᳚ - अ॒ञ्जते᳚ । न । नीति॑ । धा॒व॒ते॒ । नीति॑ । इ॒व॒ । हि । म॒नु॒ष्याः᳚ । धाव॑न्ते । पञ्च॑ । कृत्वः॑ । एति॑ । अ॒ङ्क्ते॒ । पञ्चा᳚क्ष॒रेति॒ पञ्च॑-अ॒क्ष॒रा॒ । प॒ङ्क्तिः । पाङ्क्तः॑ । य॒ज्ञ्ः । य॒ज्ञ्म् । ए॒व । अवेति॑ । रु॒न्धे॒ । परि॑मित॒मिति॒ परि॑ - मि॒त॒म् । एति॑ । अ॒ङ्क्ते॒ । अप॑रिमित॒मित्यप॑रि - मि॒त॒म् । हि । म॒नु॒ष्याः᳚ । आ॒ञ्जत॒ इत्या᳚ - अ॒ञ्जते᳚ । सतू॑ल॒येति॒ स-तू॒ल॒या॒ । एति॑ । अ॒ङ्क्ते॒ । अप॑तूल॒येत्यप॑-तू॒ल॒या॒ । हि । म॒नु॒ष्याः᳚ । आ॒ञ्जत॒ इत्या᳚ - अ॒ञ्जते᳚ । व्यावृ॑त्त्या॒ इति॑ वि - आवृ॑त्त्यै । यत् । अप॑तूल॒येत्यप॑ - तू॒ल॒या॒ । आ॒ञ्जी॒तेत्या᳚ - अ॒ञ्जी॒त । वज्रः॑ । इ॒व॒ । स्या॒त् । सतू॑ल॒येति॒ स - तू॒ल॒या॒ । एति॑ । अ॒ङ्क्ते॒ । मि॒त्र॒त्वायेति॑ मित्र - त्वाय॑ । \textbf{  6} \newline
                  \newline
                                \textbf{ TS 6.1.1.7} \newline
                  इन्द्रः॑ । वृ॒त्रम् । अ॒ह॒न्न् । सः । अ॒पः । अ॒भीति॑ । अ॒म्रि॒य॒त॒ । तासा᳚म् । यत् । मेद्ध्य᳚म् । य॒ज्ञिय᳚म् । सदे॑व॒मिति॒ स - दे॒व॒म् । आसी᳚त् । तत् । अप॑ । उदिति॑ । अ॒क्रा॒म॒त् । ते । द॒र्भाः । अ॒भ॒व॒न्न् । यत् । द॒र्भ॒पु॒ञ्जी॒लैरिति॑ दर्भ - पु॒ञ्जी॒लैः । प॒वय॑ति । याः । ए॒व । मेद्ध्याः᳚ । य॒ज्ञियाः᳚ । सदे॑वा॒ इति॒ स - दे॒वाः॒ । आपः॑ । ताभिः॑ । ए॒व । ए॒न॒म् । प॒व॒य॒ति॒ । द्वाभ्या᳚म् । प॒व॒य॒ति॒ । अ॒हो॒रा॒त्राभ्या॒मित्य॑हः-रा॒त्राभ्या᳚म् । ए॒व । ए॒न॒म् । प॒व॒य॒ति॒ । त्रि॒भिरिति॑ त्रि - भिः । प॒व॒य॒ति॒ । त्रयः॑ । इ॒मे । लो॒काः । ए॒भिः । ए॒व । ए॒न॒म् । लो॒कैः । प॒व॒य॒ति॒ । प॒ञ्चभि॒रिति॑ प॒ञ्च - भिः॒ । \textbf{  7} \newline
                  \newline
                                \textbf{ TS 6.1.1.8} \newline
                  प॒व॒य॒ति॒ । पञ्चा᳚क्ष॒रेति॒ पञ्च॑ - अ॒क्ष॒रा॒ । प॒ङ्क्तिः । पाङ्क्तः॑ । य॒ज्ञ्ः । य॒ज्ञाय॑ । ए॒व । ए॒न॒म् । प॒व॒य॒ति॒ । ष॒ड्भिरिति॑ षट् - भिः । प॒व॒य॒ति॒ । षट् । वै । ऋ॒तवः॑ । ऋ॒तुभि॒रित्यृ॒तु - भिः॒ । ए॒व । ए॒न॒म् । प॒व॒य॒ति॒ । स॒प्तभि॒रिति॑ स॒प्त - भिः॒ । प॒व॒य॒ति॒ । स॒प्त । छन्दाꣳ॑सि । छन्दो॑भि॒रिति॒ छन्दः॑ - भिः॒ । ए॒व । ए॒न॒म् । प॒व॒य॒ति॒ । न॒वभि॒रिति॑ न॒व - भिः॒ । प॒व॒य॒ति॒ । नव॑ । वै । पुरु॑षे । प्रा॒णा इति॑ प्र - अ॒नाः । सप्रा॑ण॒मिति॒ स-प्रा॒ण॒म् । ए॒व । ए॒न॒म् । प॒व॒य॒ति॒ । एक॑विꣳश॒त्येत्येक॑-विꣳ॒॒श॒त्या॒ । प॒व॒य॒ति॒ । दश॑ । हस्त्याः᳚ । अ॒ङ्गुल॑यः । दश॑ । पद्याः᳚ । आ॒त्मा । ए॒क॒विꣳ॒॒श इत्येक॑ - विꣳ॒॒शः । यावान्॑ । ए॒व । पुरु॑षः । तम् । अप॑रिवर्ग॒मित्यप॑रि - व॒र्ग॒म् । \textbf{  8} \newline
                  \newline
                                \textbf{ TS 6.1.1.9} \newline
                  प॒व॒य॒ति॒ । चि॒त्पति॒रिति॑ चित् - पतिः॑ । त्वा॒ । पु॒ना॒तु॒ । इति॑ । आ॒ह॒ । मनः॑ । वै । चि॒त्पति॒रिति॑ चित् - पतिः॑ । मन॑सा । ए॒व । ए॒न॒म् । प॒व॒य॒ति॒ । वा॒क्पति॒रिति॑ वाक् - पतिः॑ । त्वा॒ । पु॒ना॒तु॒ । इति॑ । आ॒ह॒ । वा॒चा । ए॒व । ए॒न॒म् । प॒व॒य॒ति॒ । दे॒वः । त्वा॒ । स॒वि॒ता । पु॒ना॒तु॒ । इति॑ । आ॒ह॒ । स॒वि॒तृप्र॑सूत॒ इति॑ सवि॒तृ - प्र॒सू॒तः॒ । ए॒व । ए॒न॒म् । प॒व॒य॒ति॒ । तस्य॑ । ते॒ । प॒वि॒त्र॒प॒त॒ इति॑ पवित्र - प॒ते॒ । प॒वित्रे॑ण । यस्मै᳚ । कम् । पु॒ने । तत् । श॒के॒य॒म् । इति॑ । आ॒ह॒ । आ॒शिष॒मित्या᳚ - शिष᳚म् । ए॒व । ए॒ताम् । एति॑ । शा॒स्ते॒ ॥ \textbf{  9} \newline
                  \newline
                      (अ॒ती॒का॒शान् क॑रो॒त्य - वे॑शयन् ति॒र्त्थ स्ना॑ति - प्रघा॒तो - म॑नु॒ष्या॑णां॒ तद्वा - आऽङ्क्ते॑ - मित्र॒त्वाय॑ - प॒ञ्चभि॒ - रप॑रिवर्ग - म॒ष्टाच॑त्वारिꣳशच्च)  \textbf{(A1)} \newline \newline
                                \textbf{ TS 6.1.2.1} \newline
                  याव॑न्तः । वै । दे॒वाः । य॒ज्ञाय॑ । अपु॑नत । ते । ए॒व । अ॒भ॒व॒न्न् । यः । ए॒वम् । वि॒द्वान् । य॒ज्ञाय॑ । पु॒नी॒ते । भव॑ति । ए॒व । ब॒हिः । प॒व॒यि॒त्वा । अ॒न्तः । प्रेति॑ । पा॒द॒य॒ति॒ । म॒नु॒ष्य॒लो॒क इति॑ मनुष्य - लो॒के । ए॒व । ए॒न॒म् । प॒व॒यि॒त्वा । पू॒तम् । दे॒व॒लो॒कमिति॑ देव - लो॒कम् । प्रेति॑ । न॒य॒ति॒ । अदी᳚क्षितः । एक॑या । आहु॒त्येत्या - हु॒त्या॒ । इति॑ । आ॒हुः॒ । स्रु॒वेण॑ । चत॑स्रः । जु॒हो॒ति॒ । दी॒क्षि॒त॒त्वायेति॑ दीक्षित-त्वाय॑ । स्रु॒चा । प॒ञ्च॒मीम् । पञ्चा᳚क्ष॒रेति॒ पञ्च॑ - अ॒क्ष॒रा॒ । प॒ङ्क्तिः । पाङ्क्तः॑ । य॒ज्ञ्ः । य॒ज्ञ्म् । ए॒व । अवेति॑ । रु॒न्धे॒ । आकू᳚त्या॒ इत्या - कू॒त्यै॒ । प्र॒युज॒ इति॑ प्र - युजे᳚ । अ॒ग्नये᳚ । \textbf{  10} \newline
                  \newline
                                \textbf{ TS 6.1.2.2} \newline
                  स्वाहा᳚ । इति॑ । आ॒ह॒ । आकू॒त्येत्या - कू॒त्या॒ । हि । पुरु॑षः । य॒ज्ञ्म् । अ॒भीति॑ । प्र॒यु॒ङ्क्त इति॑ प्र-यु॒ङ्क्ते । यजे॑य । इति॑ । मे॒धायै᳚ । मन॑से । अ॒ग्नये᳚ । स्वाहा᳚ । इति॑ । आ॒ह॒ । मे॒धया᳚ । हि । मन॑सा । पुरु॑षः । य॒ज्ञ्म् । अ॒भि॒गच्छ॒तीत्य॑भि - गच्छ॑ति । सर॑स्वत्यै । पू॒ष्णे । अ॒ग्नये᳚ । स्वाहा᳚ । इति॑ । आ॒ह॒ । वाक् । वै । सर॑स्वती । पृ॒थि॒वी । पू॒षा । वा॒चा । ए॒व । पृ॒थि॒व्या । य॒ज्ञ्म् । प्रेति॑ । यु॒ङ्क्ते॒ । आपः॑ । दे॒वीः॒ । बृ॒ह॒तीः॒ । वि॒श्व॒श॒भुं॒व॒ इति॑ विश्व-श॒भुं॒वः॒ । इति॑ । आ॒ह॒ । याः । वै । वर्ष्याः᳚ । ताः । \textbf{  11} \newline
                  \newline
                                \textbf{ TS 6.1.2.3} \newline
                  आपः॑ । दे॒वीः । बृ॒ह॒तीः । वि॒श्वश॑भुंव॒ इति॑ वि॒श्व - श॒भुं॒वः॒ । यत् । ए॒तत् । यजुः॑ । न । ब्रू॒यात् । दि॒व्याः । आपः॑ । अशा᳚न्ताः । इ॒मम् । लो॒कम् । एति॑ । ग॒च्छे॒युः॒ । आपः॑ । दे॒वीः॒ । बृ॒ह॒तीः॒ । वि॒श्व॒श॒भुं॒व॒ इति॑ विश्व-श॒भुं॒वः॒ । इति॑ । आ॒ह॒ । अ॒स्मै । ए॒व । ए॒नाः॒ । लो॒काय॑ । श॒म॒य॒ति॒ । तस्मा᳚त् । शा॒न्ताः । इ॒मम् । लो॒कम् । एति॑ । ग॒च्छ॒न्ति॒ । द्यावा॑पृथि॒वी इति॒ द्यावा᳚ -पृ॒थि॒वी । इति॑ । आ॒ह॒ । द्यावा॑पृथि॒व्योरिति॒ द्यावा᳚-पृ॒थि॒व्योः । हि । य॒ज्ञ्ः । उ॒रु । अ॒न्तरि॑क्षम् । इति॑ । आ॒ह॒ । अ॒न्तरि॑क्षे । हि । य॒ज्ञ्ः । बृह॒स्पतिः॑ । नः॒ । ह॒विषा᳚ । वृ॒धा॒तु॒ । \textbf{  12} \newline
                  \newline
                                \textbf{ TS 6.1.2.4} \newline
                  इति॑ । आ॒ह॒ । ब्रह्म॑ । वै । दे॒वाना᳚म् । बृह॒स्पतिः॑ । ब्रह्म॑णा । ए॒व । अ॒स्मै॒ । य॒ज्ञ्म् । अवेति॑ । रु॒न्धे॒ । यत् । ब्रू॒यात् । वि॒धेः॒ । इति॑ । य॒ज्ञ्॒स्था॒णुमिति॑ यज्ञ् - स्था॒णुम् । ऋ॒च्छे॒त् । वृ॒धा॒तु॒ । इति॑ । आ॒ह॒ । य॒ज्ञ्॒स्था॒णुमिति॑ यज्ञ् - स्था॒णुम् । ए॒व । परीति॑ । वृ॒ण॒क्ति॒ । प्र॒जाप॑ति॒रिति॑ प्र॒जा - प॒तिः॒ । य॒ज्ञ्म् । अ॒सृ॒ज॒त॒ । सः । अ॒स्मा॒त् । सृ॒ष्टः । पराङ्॑ । ऐ॒त् । सः । प्रेति॑ । यजुः॑ । अव्ली॑नात् । प्रेति॑ । साम॑ । तम् । ऋक् । उदिति॑ । अ॒य॒च्छ॒त् । यत् । ऋक् । उ॒दय॑च्छ॒दित्यु॑त् - अय॑च्छत् । तत् । औ॒द्ग्र॒ह॒णस्येत्यौ᳚त्-ग्र॒ह॒णस्य॑ । औ॒द्ग्र॒ह॒ण॒त्वमित्यौ᳚द्ग्रहण - त्वम् । ऋ॒चा । \textbf{  13} \newline
                  \newline
                                \textbf{ TS 6.1.2.5} \newline
                  जु॒हो॒ति॒ । य॒ज्ञ्स्य॑ । उद्य॑त्या॒ इत्युत् - य॒त्यै॒ । अ॒नु॒ष्टुबित्य॑नु - स्तुप् । छन्द॑साम् । उदिति॑ । अ॒य॒च्छ॒त् । इति॑ । आ॒हुः॒ । तस्मा᳚त् । अ॒नु॒ष्टुभेत्य॑नु - स्तुभा᳚ । जु॒हो॒ति॒ । य॒ज्ञ्स्य॑ । उद्य॑त्या॒ इत्युत् - य॒त्यै॒ । द्वाद॑श । वा॒थ्स॒ब॒न्धानीति॑ वाथ्स - ब॒न्धानि॑ । उदिति॑ । अ॒य॒च्छ॒न्न् । इति॑ । आ॒हुः॒ । तस्मा᳚त् । द्वा॒द॒शभि॒रिति॑ द्वाद॒श-भिः॒ । आ॒थ्स॒ब॒न्ध॒विद॒ इति॑ वाथ्सबन्ध - विदः॑ । दी॒क्ष॒य॒न्ति॒ । सा । वै । ए॒षा । ऋक् । अ॒नु॒ष्टुगित्य॑नु - स्तुक् । वाक् । अ॒नु॒ष्टुगित्य॑नु - स्तुक् । यत् । ए॒तया᳚ । ऋ॒चा । दी॒क्षय॑ति । वा॒चा । ए॒व । ए॒न॒म् । सर्व॑या । दी॒क्ष॒य॒ति॒ । विश्वे᳚ । दे॒वस्य॑ । ने॒तुः । इति॑ । आ॒ह॒ । सा॒वि॒त्री । ए॒तेन॑ । मर्तः॑ । वृ॒णी॒त॒ । स॒ख्यम् । \textbf{  14} \newline
                  \newline
                                \textbf{ TS 6.1.2.6} \newline
                  इति॑ । आ॒ह॒ । पि॒तृ॒दे॒व॒त्येति॑ पितृ - दे॒व॒त्या᳚ । ए॒तेन॑ । विश्वे᳚ । रा॒यः । इ॒षु॒द्ध्य॒सि॒ । इति॑ । आ॒ह॒ । वै॒श्व॒दे॒वीति॑ वैश्व-दे॒वी । ए॒तेन॑ । द्यु॒म्नम् । वृ॒णी॒त॒ । पु॒ष्यसे᳚ । इति॑ । आ॒ह॒ । पौ॒ष्णी । ए॒तेन॑ । सा । वै । ए॒षा । ऋक् । स॒र्व॒दे॒व॒त्येति॑ सर्व - दे॒व॒त्या᳚ । यत् । ए॒तया᳚ । ऋ॒चा । दी॒क्षय॑ति । सर्वा॑भिः । ए॒व । ए॒न॒म् । दे॒वता॑भिः । दी॒क्ष॒य॒ति॒ । स॒प्ताक्ष॑र॒मिति॑ स॒प्त - अ॒क्ष॒र॒म् । प्र॒थ॒मम् । प॒दम् । अ॒ष्टाक्ष॑रा॒णीत्य॒ष्टा -अ॒क्ष॒रा॒णि॒ । त्रीणि॑ । यानि॑ । त्रीणि॑ । तानि॑ । अ॒ष्टौ । उपेति॑ । य॒न्ति॒ । यानि॑ । च॒त्वारि॑ । तानि॑ । अ॒ष्टौ । यत् । अ॒ष्टाक्ष॒रेत्य॒ष्टा - अ॒क्ष॒रा॒ । तेन॑ । \textbf{  15} \newline
                  \newline
                                \textbf{ TS 6.1.2.7} \newline
                  गा॒य॒त्री । यत् । एका॑दशाक्ष॒रेत्येका॑दश - अ॒क्ष॒रा॒ । तेन॑ । त्रि॒ष्टुक् । यत् । द्वाद॑शाक्ष॒रेति॒ द्वाद॑श - अ॒क्ष॒रा॒ । तेन॑ । जग॑ती । सा । वै । ए॒षा । ऋक् । सर्वा॑णि । छन्दाꣳ॑सि । यत् । ए॒तया᳚ । ऋ॒चा । दी॒क्षय॑ति । सर्वे॑भिः । ए॒व । ए॒न॒म् । छन्दो॑भि॒रिति॒ छन्दः॑ - भिः॒ । दी॒क्ष॒य॒ति॒ । स॒प्ताक्ष॑र॒मिति॑ स॒प्त - अ॒क्ष॒रम् । प्र॒थ॒मम् । प॒दम् । स॒प्तप॒देति॑ स॒प्त - प॒दा॒ । शक्व॑री । प॒शवः॑ । शक्व॑री । प॒शून् । ए॒व । अवेति॑ । रु॒न्धे॒ । एक॑स्मात् । अ॒क्षरा᳚त् । अना᳚प्तम् । प्र॒थ॒मम् । प॒दम् । तस्मा᳚त् । यत् । वा॒चः । अना᳚प्तम् । तत् । म॒नु॒ष्याः᳚ । उपेति॑ । जी॒व॒न्ति॒ । पू॒र्णया᳚ । जु॒हो॒ति॒ ( ) । पू॒र्णः । इ॒व॒ । हि । प्र॒जाप॑ति॒रिति॑ प्र॒जा-प॒तिः॒ । प्र॒जाप॑ते॒रिति॑ प्र॒जा-प॒तेः॒ । आप्त्यै᳚ । न्यू॑न॒येति॒ नि-ऊ॒न॒या॒ । जु॒हो॒ति॒ । न्यू॑ना॒दिति॒ नि - ऊ॒ना॒त् । हि । प्र॒जाप॑ति॒रिति॑ प्र॒जा - प॒तिः॒ । प्र॒जा इति॑ प्र - जाः । असृ॑जत । प्र॒जाना॒मिति॑ प्र-जाना᳚म् । सृष्ट्यै᳚ ॥ \textbf{  16} \newline
                  \newline
                      (अ॒ग्नये॒ - ता - वृ॑धात्वृ॒ - चा - स॒ख्यं - तेन॑ - जुहोति॒ - पञ्च॑दश च)  \textbf{(A2)} \newline \newline
                                \textbf{ TS 6.1.3.1} \newline
                  ऋ॒ख्सा॒मे इत्यृ॑क्-सा॒मे । वै । दे॒वेभ्यः॑ । य॒ज्ञाय॑ । अति॑ष्ठमाने॒ इति॑ । कृष्णः॑ । रू॒पम् । कृ॒त्वा । अ॒प॒क्रम्येत्य॑प -क्रम्य॑ । अ॒ति॒ष्ठ॒ता॒म् । ते । अ॒म॒न्य॒न्त॒ । यम् । वै । इ॒मे इति॑ । उ॒पा॒व॒र्थ्स्यत॒ इत्यु॑प-आ॒व॒र्थ्स्यतः॑ । सः । इ॒दम् । भ॒वि॒ष्य॒ति॒ । इति॑ । ते इति॑ । उपेति॑ । अ॒म॒न्त्र॒य॒न्त॒ । ते इति॑ । अ॒हो॒रा॒त्रयो॒रित्य॑हः - रा॒त्रयोः᳚ । म॒हि॒मान᳚म् । अ॒प॒नि॒धायेत्य॑प - नि॒धाय॑ । दे॒वान् । उ॒पाव॑र्तेता॒मित्यु॑प-आव॑र्तेताम् । ए॒षः । वै । ऋ॒चः । वर्णः॑ । यत् । शु॒क्लम् । कृ॒ष्णा॒जि॒नस्येति॑ कृष्ण - अ॒जि॒नस्य॑ । ए॒षः । साम्नः॑ । यत् । कृ॒ष्णम् । ऋ॒ख्सा॒मयो॒रित्यृ॑क् - सा॒मयोः᳚ । शिल्पे॒ इति॑ । स्थः॒ । इति॑ । आ॒ह॒ । ऋ॒ख्सा॒मे इत्यृ॑क् - सा॒मे । ए॒व । अवेति॑ । रु॒न्धे॒ । ए॒षः । \textbf{  17} \newline
                  \newline
                                \textbf{ TS 6.1.3.2} \newline
                  वै । अह्नः॑ । वर्णः॑ । यत् । शु॒क्लम् । कृ॒ष्णा॒जि॒नस्येति॑ कृष्ण - अ॒जि॒नस्य॑ । ए॒षः । रात्रि॑याः । यत् । कृ॒ष्णम् । यत् । ए॒व । ए॒न॒योः॒ । तत्र॑ । न्य॑क्त॒मिति॒ नि-अ॒क्त॒म् । तत् । ए॒व । अवेति॑ । रु॒न्धे॒ । कृ॒ष्णा॒जि॒नेनेति॑ कृष्ण-अ॒जि॒नेन॑ । दी॒क्ष॒य॒ति॒ । ब्रह्म॑णः । वै । ए॒तत् । रू॒पम् । यत् । कृ॒ष्णा॒जि॒नमिति॑ कृष्ण - अ॒जि॒नम् । ब्रह्म॑णा । ए॒व । ए॒न॒म् । दी॒क्ष॒य॒ति॒ । इ॒माम् । धिय᳚म् । शिक्ष॑माणस्य । दे॒व॒ । इति॑ । आ॒ह॒ । य॒था॒य॒जुरिति॑ यथा - य॒जुः । ए॒व । ए॒तत् । गर्भः॑ । वै । ए॒षः । यत् । दी॒क्षि॒तः । उल्ब᳚म् । वासः॑ । प्रेति॑ । ऊ॒र्णु॒ते॒ । तस्मा᳚त् । \textbf{  18} \newline
                  \newline
                                \textbf{ TS 6.1.3.3} \newline
                  गर्भाः᳚ । प्रावृ॑ताः । जा॒य॒न्ते॒ । न । पु॒रा । सोम॑स्य । क्र॒यात् । अपेति॑ । ऊ॒र्ण्वी॒त॒ । यत् । पु॒रा । सोम॑स्य । क्र॒यात् । अ॒पो॒र्ण्वी॒तेत्य॑प-ऊ॒र्ण्वी॒त । गर्भाः᳚ । प्र॒जाना॒मिति॑ प्र - जाना᳚म् । प॒रा॒पातु॑का॒ इति॑ परा-पातु॑काः । स्युः॒ । क्री॒ते । सोमे᳚ । अपेति॑ । ऊ॒र्णु॒ते॒ । जाय॑ते । ए॒व । तत् । अथो॒ इति॑ । यथा᳚ । वसी॑याꣳसम् । प्र॒त्य॒पो॒र्णु॒त इति॑ प्रति -अ॒पो॒र्णु॒ते । ता॒दृक् । ए॒व । तत् । अङ्गि॑रसः । सु॒व॒र्गमिति॑ सुवः - गम् । लो॒कम् । यन्तः॑ । ऊर्ज᳚म् । वीति॑ । अ॒भ॒ज॒न्त॒ । ततः॑ । यत् । अ॒त्यशि॑ष्य॒तेत्य॑ति - अशि॑ष्यत । ते । श॒राः । अ॒भ॒व॒न्न् । ऊर्क् । वै । श॒राः । यत् । श॒र॒मयीति॑ शर - मयी᳚ । \textbf{  19} \newline
                  \newline
                                \textbf{ TS 6.1.3.4} \newline
                  मेख॑ला । भव॑ति । ऊर्ज᳚म् । ए॒व । अवेति॑ । रु॒न्धे॒ । म॒द्ध्य॒तः । समिति॑ । न॒ह्य॒ति॒ । म॒द्ध्य॒तः । ए॒व । अ॒स्मै॒ । ऊर्ज᳚म् । द॒धा॒ति॒ । तस्मा᳚त् । म॒द्ध्य॒तः । ऊ॒र्जा । भु॒ञ्ज॒ते॒ । ऊ॒द्‌र्ध्वम् । वै । पुरु॑षस्य । नाभ्यै᳚ । मेद्ध्य᳚म् । अ॒वा॒चीन᳚म् । अ॒मे॒द्ध्यम् । यत् । म॒द्ध्य॒तः । स॒न्नह्य॒तीति॑ सं - नह्य॑ति । मेद्ध्य᳚म् । च॒ । ए॒व । अ॒स्य॒ । अ॒मे॒द्ध्यम् । च॒ । व्याव॑र्तय॒तीति॑ वि-आव॑र्तयति । इन्द्रः॑ । वृ॒त्राय॑ । वज्र᳚म् । प्रेति॑ । अ॒ह॒र॒त् । सः । त्रे॒धा । वीति॑ । अ॒भ॒व॒त् । स्फ्यः । तृती॑यम् । रथः॑ । तृती॑यम् । यूपः॑ । तृती॑यम् । \textbf{  20} \newline
                  \newline
                                \textbf{ TS 6.1.3.5} \newline
                  ये । अ॒न्त॒श्श॒रा इत्य॑न्तः-श॒राः । अशी᳚र्यन्त । ते । श॒राः । अ॒भ॒व॒न्न् । तत् । श॒राणा᳚म् । श॒र॒त्वमिति॑ शर-त्वम् । वज्रः॑ । वै । श॒राः । क्षुत् । खलु॑ । वै । म॒नु॒ष्य॑स्य । भ्रातृ॑व्यः । यत् । श॒र॒मयीति॑ शर - मयी᳚ । मेख॑ला । भव॑ति । वज्रे॑ण । ए॒व । सा॒क्षादिति॑ स-अ॒क्षात् । क्षुध᳚म् । भ्रातृ॑व्यम् । म॒द्ध्य॒तः । अपेति॑ । ह॒ते॒ । त्रि॒वृदिति॑ त्रि - वृत् । भ॒व॒ति॒ । त्रि॒वृदिति॑ त्रि - वृत् । वै । प्रा॒ण इति॑ प्र -अ॒नः । त्रि॒वृत॒मिति॑ त्रि - वृत᳚म् । ए॒व । प्रा॒णमिति॑ प्र - अ॒नम् । म॒द्ध्य॒तः । यज॑माने । द॒धा॒ति॒ । पृ॒थ्वी । भ॒व॒ति॒ । रज्जू॑नाम् । व्यावृ॑त्या॒ इति॑ वि - आवृ॑त्यै । मेख॑लया । यज॑मानम् । दी॒क्ष॒य॒ति॒ । योक्त्रे॑ण । पत्नी᳚म् । मि॒थु॒न॒त्वायेति॑ मिथुन - त्वाय॑ । \textbf{  21} \newline
                  \newline
                                \textbf{ TS 6.1.3.6} \newline
                  य॒ज्ञ्ः । दक्षि॑णाम् । अ॒भीति॑ । अ॒द्ध्या॒य॒त् । ताम् । समिति॑ । अ॒भ॒व॒त् । तत् । इन्द्रः॑ । अ॒चा॒य॒त् । सः । अ॒म॒न्य॒त॒ । यः । वै । इ॒तः । ज॒नि॒ष्यते᳚ । सः । इ॒दम् । भ॒वि॒ष्य॒ति॒ । इति॑ । ताम् । प्रेति॑ । अ॒वि॒श॒त् । तस्याः᳚ । इन्द्रः॑ । ए॒व । अ॒जा॒य॒त॒ । सः । अ॒म॒न्य॒त॒ । यः । वै । मत् । इ॒तः । अप॑रः । ज॒नि॒ष्यते᳚ । सः । इ॒दम् । भ॒वि॒ष्य॒ति॒ । इति॑ । तस्याः᳚ । अ॒नु॒मृश्येत्य॑नु - मृश्य॑ । योनि᳚म् । एति॑ । अ॒च्छि॒न॒त् । सा । सू॒तव॒शेति॑ सू॒त - व॒शा॒ । अ॒भ॒व॒त् । तत् । सू॒तव॑शाया॒ इति॑ सू॒त - व॒शा॒यै॒ । जन्म॑ । \textbf{  22} \newline
                  \newline
                                \textbf{ TS 6.1.3.7} \newline
                  ताम् । हस्ते᳚ । नीति॑ । अ॒वे॒ष्ट॒य॒त॒ । ताम् । मृ॒गेषु॑ । नीति॑ । अ॒द॒धा॒त् । सा । कृ॒ष्ण॒वि॒षा॒णेति॑ कृष्ण - वि॒षा॒णा । अ॒भ॒व॒त् । इन्द्र॑स्य । योनिः॑ । अ॒सि॒ । मा । मा॒ । हिꣳ॒॒सीः॒ । इति॑ । कृ॒ष्ण॒वि॒षा॒णामिति॑ कृष्ण - वि॒षा॒णाम् । प्रेति॑ । य॒च्छ॒ति॒ । सयो॑नि॒मिति॒ स - यो॒नि॒म् । ए॒व । य॒ज्ञ्म् । क॒रो॒ति॒ । सयो॑नि॒मिति॒ स - यो॒नि॒म् । दक्षि॑णाम् । सयो॑नि॒मिति॒ स - यो॒नि॒म् । इन्द्र᳚म् । स॒यो॒नि॒त्वायेति॑ सयोनि-त्वाय॑ । कृ॒ष्यै । त्वा॒ । सु॒स॒स्याया॒ इति॑ सु - स॒स्यायै᳚ । इति॑ । आ॒ह॒ । तस्मा᳚त् । अ॒कृ॒ष्ट॒प॒च्या इत्य॑कृष्ट - प॒च्याः । ओष॑धयः । प॒च्य॒न्ते॒ । सु॒पि॒प्प॒लाभ्य॒ इति॑ सु - पि॒प्प॒लाभ्यः॑ । त्वा॒ । ओष॑धीभ्य॒ इत्योष॑धि - भ्यः॒ । इति॑ । आ॒ह॒ । तस्मा᳚त् । ओष॑धयः । फल᳚म् । गृ॒ह्ण॒न्ति॒ । यत् । हस्ते॑न । \textbf{  23} \newline
                  \newline
                                \textbf{ TS 6.1.3.8} \newline
                  क॒ण्डू॒येत॑ । पा॒म॒न॒म्भावु॑का॒ इति॑ पामनम् - भावु॑काः । प्र॒जा इति॑ प्र - जाः । स्युः॒ । यत् । स्मये॑त । न॒ग्न॒म्भावु॑का॒ इति॑ नग्नम् - भावु॑काः । कृ॒ष्ण॒वि॒षा॒णयेति॑ कृष्ण-वि॒षा॒णया᳚ । क॒ण्डू॒य॒ते॒ । अ॒पि॒गृह्येत्य॑पि - गृह्य॑ । स्म॒य॒ते॒ । प्र॒जाना॒मिति॑ प्र - जाना᳚म् । गो॒पी॒थाय॑ । न । पु॒रा । दक्षि॑णाभ्यः । नेतोः᳚ । कृ॒ष्ण॒वि॒षा॒णामिति॑ कृष्ण - वि॒षा॒णाम् । अवेति॑ । चृ॒ते॒त् । यत् । पु॒रा । दक्षि॑णाभ्यः । नेतोः᳚ । कृ॒ष्ण॒वि॒षा॒णामिति॑ कृष्ण-वि॒षा॒णाम् । अ॒व॒चृ॒तेदित्य॑व-चृ॒तेत् । योनिः॑ । प्र॒जाना॒मिति॑ प्र - जाना᳚म् । प॒रा॒पातु॒केति॑ परा - पातु॑का । स्या॒त् । नी॒तासु॑ । दक्षि॑णासु । चात्वा॑ले । कृ॒ष्ण॒वि॒षा॒णामिति॑ कृष्ण - वि॒षा॒णाम् । प्रेति॑ । अ॒स्य॒ति॒ । योनिः॑ । वै । य॒ज्ञ्स्य॑ । चात्वा॑लम् । योनिः॑ । कृ॒ष्ण॒वि॒षा॒णेति॑ कृष्ण - वि॒षा॒णा । योनौ᳚ । ए॒व । योनि᳚म् । द॒धा॒ति॒ । य॒ज्ञ्स्य॑ । स॒यो॒नि॒त्वायेति॑ सयोनि - त्वाय॑ ॥ \textbf{  24} \newline
                  \newline
                      (रु॒न्ध॒ ए॒ष - तस्मा᳚ - च्छर॒मयी॒ - यूप॒स्तृती॑यं - मिथुन॒त्वाय॒ - जन्म॒ - हस्ते॑ना॒ - ऽष्टाच॑त्वारिꣳशच्च)  \textbf{(A3)} \newline \newline
                                \textbf{ TS 6.1.4.1} \newline
                  वाक् । वै । दे॒वेभ्यः॑ । अपेति॑ । अ॒क्रा॒म॒त् । य॒ज्ञाय॑ । अति॑ष्ठमाना । सा । वन॒स्पतीन्॑ । प्रेति॑ । अ॒वि॒श॒त् । सा । ए॒षा । वाक् । वन॒स्पति॑षु । व॒द॒ति॒ । या । दु॒न्दु॒भौ । या । तूण॑वे । या । वीणा॑याम् । यत् । दी॒क्षि॒त॒द॒ण्डमिति॑ दीक्षित-द॒ण्डम् । प्र॒यच्छ॒तीति॑ प्र-यच्छ॑ति । वाच᳚म् । ए॒व । अवेति॑ । रु॒न्धे॒ । औदु॑म्बरः । भ॒व॒ति॒ । ऊर्क् । वै । उ॒दु॒बंरः॑ । ऊर्ज᳚म् । ए॒व । अवेति॑ । रु॒न्धे॒ । मुखे॑न । सम्मि॑त॒ इति॒ सं - मि॒तः॒ । भ॒व॒ति॒ । मु॒ख॒तः । ए॒व । अ॒स्मै॒ । ऊर्ज᳚म् । द॒धा॒ति॒ । तस्मा᳚त् । मु॒ख॒तः । ऊ॒र्जा । भु॒ञ्ज॒ते॒ । \textbf{  25} \newline
                  \newline
                                \textbf{ TS 6.1.4.2} \newline
                  क्री॒ते । सोमे᳚ । मै॒त्रा॒व॒रु॒णायेति॑ मैत्रा - व॒रु॒णाय॑ । द॒ण्डम् । प्रेति॑ । य॒च्छ॒ति॒ । मै॒त्रा॒व॒रु॒ण इति॑ मैत्रा - व॒रु॒णः । हि । पु॒रस्ता᳚त् । ऋ॒त्विग्भ्य॒ इत्यृ॒त्विक् - भ्यः॒ । वाच᳚म् । वि॒भज॒तीति॑ वि - भज॑ति । ताम् । ऋ॒त्विजः॑ । यज॑माने । प्रतीति॑ । स्था॒प॒य॒न्ति॒ । स्वाहा᳚ । य॒ज्ञ्म् । मन॑सा । इति॑ । आ॒ह॒ । मन॑सा । हि । पुरु॑षः । य॒ज्ञ्म् । अ॒भि॒गच्छ॒तीत्य॑भि - गच्छ॑ति । स्वाहा᳚ । द्यावा॑पृथि॒वीभ्या॒मिति॒ द्यावा᳚ - पृ॒थि॒वीभ्या᳚म् । इति॑ । आ॒ह॒ । द्यावा॑पृथि॒व्योरिति॒ द्यावा᳚-पृ॒थि॒व्योः । हि । य॒ज्ञ्ः । स्वाहा᳚ । उ॒रोः । अ॒न्तरि॑क्षात् । इति॑ । आ॒ह॒ । अ॒न्तरि॑क्षे । हि । य॒ज्ञ्ः । स्वाहा᳚ । य॒ज्ञ्म् । वाता᳚त् । एति॑ । र॒भे॒ । इति॑ । आ॒ह॒ । अ॒यम् । \textbf{  26} \newline
                  \newline
                                \textbf{ TS 6.1.4.3} \newline
                  वाव । यः । पव॑ते । सः । य॒ज्ञ्ः । तम् । ए॒व । सा॒क्षादिति॑ स-अ॒क्षात् । एति॑ । र॒भ॒ते॒ । मु॒ष्टी इति॑ । क॒रो॒ति॒ । वाच᳚म् । य॒च्छ॒ति॒ । य॒ज्ञ्स्य॑ । धृत्यै᳚ । अदी᳚क्षिष्ट । अ॒यम् । ब्रा॒ह्म॒णः । इति॑ । त्रिः । उ॒पाꣳ॒॒श्वित्यु॑प - अꣳ॒॒शु । आ॒ह॒ । दे॒वेभ्यः॑ । ए॒व । ए॒न॒म् । प्रेति॑ । आ॒ह॒ । त्रिः । उ॒च्चैः । उ॒भये᳚भ्यः । ए॒व । ए॒न॒म् । दे॒व॒म॒नु॒ष्येभ्य॒ इति॑ देव - म॒नु॒ष्येभ्यः॑ । प्रेति॑ । आ॒ह॒ । न । पु॒रा । नक्ष॑त्रेभ्यः । वाच᳚म् । वीति॑ । सृ॒जे॒त् । यत् । पु॒रा । नक्ष॑त्रेभ्यः । वाच᳚म् । वि॒सृ॒जेदिति॑ वि - सृ॒जेत् । य॒ज्ञ्म् । वीति॑ । छि॒न्द्या॒त् । \textbf{  27} \newline
                  \newline
                                \textbf{ TS 6.1.4.4} \newline
                  उदि॑ते॒ष्वित्युत् - इ॒ते॒षु॒ । नक्ष॑त्रेषु । व्र॒तम् । कृ॒णु॒त॒ । इति॑ । वाच᳚म् । वीति॑ । सृ॒ज॒ति॒ । य॒ज्ञ्व्र॑त॒ इति॑ य॒ज्ञ्-व्र॒तः॒ । वै । दी॒क्षि॒तः । य॒ज्ञ्म् । ए॒व । अ॒भीति॑ । वाच᳚म् । वीति॑ । सृ॒ज॒ति॒ । यदि॑ । वि॒सृ॒जेदिति॑ वि - सृ॒जेत् । वै॒ष्ण॒वीम् । ऋच᳚म् । अन्विति॑ । ब्रू॒या॒त् । य॒ज्ञ्ः । वै । विष्णुः॑ । य॒ज्ञेन॑ । ए॒व । य॒ज्ञ्म् । समिति॑ । त॒नो॒ति॒ । दैवी᳚म् । धिय᳚म् । म॒ना॒म॒हे॒ । इति॑ । आ॒ह॒ । य॒ज्ञ्म् । ए॒व । तत् । म्र॒द॒य॒ति॒ । सु॒पा॒रेति॑ सु - पा॒रा । नः॒ । अ॒स॒त् । वशे᳚ । इति॑ । आ॒ह॒ । व्यु॑ष्टि॒मिति॒ वि - उ॒ष्टि॒म् । ए॒व । अवेति॑ । रु॒न्धे॒ । \textbf{  28} \newline
                  \newline
                                \textbf{ TS 6.1.4.5} \newline
                  ब्र॒ह्म॒वा॒दिन॒ इति॑ ब्रह्म - वा॒दिनः॑ । व॒द॒न्ति॒ । हो॒त॒व्य᳚म् । दी॒क्षि॒तस्य॑ । गृ॒हा(3) इ । न । हो॒त॒व्या(3)म् । इति॑ । ह॒विः । वै । दी॒क्षि॒तः । यत् । जु॒हु॒यात् । यज॑मानस्य । अ॒व॒दायेत्य॑व - दाय॑ । जु॒हु॒या॒त् । यत् । न । जु॒हु॒यात् । य॒ज्ञ्॒प॒रुरिति॑ यज्ञ् - प॒रुः । अ॒न्तः । इ॒या॒त् । ये । दे॒वाः । मनो॑जाता॒ इति॒ मनः॑ - जा॒ताः॒ । म॒नो॒युज॒ इति॑ मनः-युजः॑ । इति॑ । आ॒ह॒ । प्रा॒णा इति॑ प्र - अ॒नाः । वै । दे॒वाः । मनो॑जाता॒ इति॒ मनः॑ - जा॒ताः॒ । म॒नो॒युज॒ इति॑ मनः - युजः॑ । तेषु॑ । ए॒व । प॒रोक्ष॒मिति॑ परः - अक्ष᳚म् । जु॒हो॒ति॒ । तत् । न । इ॒व॒ । हु॒तम् । न । इ॒व॒ । अहु॑तम् । स्व॒पन्त᳚म् । वै । दी॒क्षि॒तम् । रक्षाꣳ॑सि । जि॒घाꣳ॒॒स॒न्ति॒ । अ॒ग्निः । \textbf{  29} \newline
                  \newline
                                \textbf{ TS 6.1.4.6} \newline
                  खलु॑ । वै । र॒क्षो॒हेति॑ रक्षः - हा । अग्ने᳚ । त्वम् । स्विति॑ । जा॒गृ॒हि॒ । व॒यम् । स्विति॑ । म॒न्दि॒षी॒म॒हि॒ । इति॑ । आ॒ह॒ । अ॒ग्निम् । ए॒व । अ॒धि॒पामित्य॑धि - पाम् । कृ॒त्वा । स्व॒पि॒ति॒ । रक्ष॑साम् । अप॑हत्या॒ इत्यप॑ - ह॒त्यै॒ । अ॒व्र॒त्यम् । इ॒व॒ । वै । ए॒षः । क॒रो॒ति॒ । यः । दी॒क्षि॒तः । स्वपि॑ति । त्वम् । अ॒ग्ने॒ । व्र॒त॒पा इति॑ व्रत-पाः । अ॒सि॒ । इति॑ । आ॒ह॒ । अ॒ग्निः । वै । दे॒वाना᳚म् । व्र॒तप॑ति॒रिति॑ व्र॒त - प॒तिः॒ । सः । ए॒व । ए॒न॒म् । व्र॒तम् । एति॑ । ल॒म्भ॒य॒ति॒ । दे॒वः । एति॑ । मर्त्ये॑षु । एति॑ । इति॑ । आ॒ह॒ । दे॒वः । \textbf{  30} \newline
                  \newline
                                \textbf{ TS 6.1.4.7} \newline
                  हि । ए॒षः । सन्न् । मर्त्ये॑षु । त्वम् । य॒ज्ञेषु॑ । ईड्यः॑ । इति॑ । आ॒ह॒ । ए॒तम् । हि । य॒ज्ञेषु॑ । ईड॑ते । अपेति॑ । वै । दी॒क्षि॒तात् । सु॒षु॒पुषः॑ । इ॒न्द्रि॒यम् । दे॒वताः᳚ । क्रा॒म॒न्ति॒ । विश्वे᳚ । दे॒वाः । अ॒भीति॑ । माम् । एति॑ । अ॒व॒वृ॒त्र॒न्न् । इति॑ । आ॒ह॒ । इ॒न्द्रि॒येण॑ । ए॒व । ए॒न॒म् । दे॒वता॑भिः । समिति॑ । न॒य॒ति॒ । यत् । ए॒तत् । यजुः॑ । न । ब्रू॒यात् । याव॑तः । ए॒व । प॒शून् । अ॒भीति॑ । दीक्षे॑त । ताव॑न्तः । अ॒स्य॒ । प॒शवः॑ । स्युः॒ । रास्व॑ । इय॑त् । \textbf{  31} \newline
                  \newline
                                \textbf{ TS 6.1.4.8} \newline
                  सो॒म॒ । एति॑ । भूयः॑ । भ॒र॒ । इति॑ । आ॒ह॒ । अप॑रिमिता॒नित्यप॑रि-मि॒ता॒न् । ए॒व । प॒शून् । अवेति॑ । रु॒न्धे॒ । च॒न्द्रम् । अ॒सि॒ । मम॑ । भोगा॑य । भ॒व॒ । इति॑ । आ॒ह॒ । य॒था॒दे॒व॒तमिति॑ यथा - दे॒व॒तम् । ए॒व । ए॒नाः॒ । प्रतीति॑ । गृ॒ह्णा॒ति॒ । वा॒यवे᳚ । त्वा॒ । वरु॑णाय । त्वा॒ । इति॑ । यत् । ए॒वम् । ए॒ताः । न । अ॒नु॒दि॒शेदित्य॑नु - दि॒शेत् । अय॑थादेवत॒मित्यय॑था - दे॒व॒त॒म् । दक्षि॑णाः । ग॒म॒ये॒त् । एति॑ । दे॒वता᳚भ्यः । वृ॒श्च्ये॒त॒ । यत् । ए॒वम् । ए॒ताः । अ॒नु॒दि॒शतीत्य॑नु - दि॒शति॑ । य॒था॒दे॒व॒तमिति॑ यथा - दे॒व॒तम् । ए॒व । दक्षि॑णाः । ग॒म॒य॒ति॒ । न । दे॒वता᳚भ्यः । एति॑ । \textbf{  32 } \newline
                  \newline
                                \textbf{ TS 6.1.4.9} \newline
                  वृ॒श्च्य॒ते॒ । देवीः᳚ । आ॒पः॒ । अ॒पा॒म् । न॒पा॒त् । इति॑ । आ॒ह॒ । यत् । वः॒ । मेद्ध्य᳚म् । य॒ज्ञिय᳚म् । सदे॑व॒मिति॒ स - दे॒व॒म् । तत् । वः॒ । मा । अवेति॑ । क्र॒मि॒ष॒म् । इति॑ । वाव । ए॒तत् । आ॒ह॒ । अच्छि॑न्नम् । तन्तु᳚म् । पृ॒थि॒व्याः । अन्विति॑ । गे॒ष॒म् । इति॑ । आ॒ह॒ । सेतु᳚म् । ए॒व । कृ॒त्वा । अतीति॑ । ए॒ति॒ ॥ \textbf{  33 } \newline
                  \newline
                      (भु॒ञ्ज॒ते॒ - ऽयं - छि॑न्द्याद् - रुन्धे॒ - ऽग्नि - रा॑ह दे॒व - इय॑द् - दे॒वता᳚भ्य॒ आ - त्रय॑स्त्रिꣳशच्च)  \textbf{(A4)} \newline \newline
                                \textbf{ TS 6.1.5.1} \newline
                  दे॒वाः । वै । दे॒व॒यज॑न॒मिति॑ देव - यज॑नम् । अ॒द्ध्य॒व॒सायेत्य॑धि - अ॒व॒साय॑ । दिशः॑ । न । प्रेति॑ । अ॒जा॒न॒न्न् । ते । अ॒न्यः । अ॒न्यम् । उपेति॑ । अ॒धा॒व॒न्न् । त्वया᳚ । प्रेति॑ । जा॒ना॒म॒ । त्वया᳚ । इति॑ । ते । अदि॑त्याम् । समिति॑ । अ॒द्ध्र॒य॒न्त॒ । त्वया᳚ । प्रेति॑ । जा॒ना॒म॒ । इति॑ । सा । अ॒ब्र॒वी॒त् । वर᳚म् । वृ॒णै॒ । मत्प्रा॑यणा॒ इति॒ मत् - प्रा॒य॒णाः॒ । ए॒व । वः॒ । य॒ज्ञाः । मदु॑दयना॒ इति॒ मत्-उ॒द॒य॒नाः॒ । अ॒स॒न्न् । इति॑ । तस्मा᳚त् । आ॒दि॒त्यः । प्रा॒य॒णीय॒ इति॑ प्र-अ॒य॒नीयः॑ । य॒ज्ञाना᳚म् । आ॒दि॒त्यः । उ॒द॒य॒नीय॒ इत्यु॑त् - अ॒य॒नीयः॑ । पञ्च॑ । दे॒वताः᳚ । य॒ज॒ति॒ । पञ्च॑ । दिशः॑ । दि॒शाम् । प्रज्ञा᳚त्या॒ इति॒ प्र - ज्ञा॒त्यै॒ । \textbf{  34} \newline
                  \newline
                                \textbf{ TS 6.1.5.2} \newline
                  अथो॒ इति॑ । पञ्चा᳚क्ष॒रेति॒ पञ्च॑ - अ॒क्ष॒रा॒ । प॒ङ्क्तिः । पाङ्क्तः॑ । य॒ज्ञ्ः । य॒ज्ञ्म् । ए॒व । अवेति॑ । रु॒न्धे॒ । पथ्या᳚म् । स्व॒स्तिम् । अ॒य॒ज॒न्न् । प्राची᳚म् । ए॒व । तया᳚ । दिश᳚म् । प्रेति॑ । अ॒जा॒न॒न्न् । अ॒ग्निना᳚ । द॒क्षि॒णा । सोमे॑न । प्र॒तीची᳚म् । स॒वि॒त्रा । उदी॑चीम् । अदि॑त्या । ऊ॒द्‌र्ध्वाम् । पथ्या᳚म् । स्व॒स्तिम् । य॒ज॒ति॒ । प्राची᳚म् । ए॒व । तया᳚ । दिश᳚म् । प्रेति॑ । जा॒ना॒ति॒ । पथ्या᳚म् । स्व॒स्तिम् । इ॒ष्ट्वा । अ॒ग्नीषोमा॒वित्य॒ग्नी-सोमौ᳚ । य॒ज॒ति॒ । चक्षु॑षी॒ इति॑ । वै । ए॒ते इति॑ । य॒ज्ञ्स्य॑ । यत् । अ॒ग्नीषोमा॒वित्य॒ग्नी - सोमौ᳚ । ताभ्या᳚म् । ए॒व । अन्विति॑ । प॒श्य॒ति॒ । \textbf{  35} \newline
                  \newline
                                \textbf{ TS 6.1.5.3} \newline
                  अ॒ग्नीषोमा॒वित्य॒ग्नी - सोमौ᳚ । इ॒ष्ट्वा । स॒वि॒तार᳚म् । य॒ज॒ति॒ । स॒वि॒तृप्र॑सूत॒ इति॑ सवि॒तृ - प्र॒सू॒तः॒ । ए॒व । अन्विति॑ । प॒श्य॒ति॒ । स॒वि॒तार᳚म् । इ॒ष्ट्वा । अदि॑तिम् । य॒ज॒ति॒ । इ॒यम् । वै । अदि॑तिः । अ॒स्याम् । ए॒व । प्र॒ति॒ष्ठायेति॑ प्रति - स्थाय॑ । अन्विति॑ । प॒श्य॒ति॒ । अदि॑तिम् । इ॒ष्ट्वा । मा॒रु॒तीम् । ऋच᳚म् । अन्विति॑ । आ॒ह॒ । म॒रुतः॑ । वै । दे॒वाना᳚म् । विशः॑ । दे॒व॒वि॒शमिति॑ देव - वि॒शम् । खलु॑ । वै । कल्प॑मानम् । म॒नु॒ष्य॒वि॒शमिति॑ मनुष्य-वि॒शम् । अन्विति॑ । क॒ल्प॒ते॒ । यत् । मा॒रु॒तीम् । ऋच᳚म् । अ॒न्वाहेत्य॑नु-आह॑ । वि॒शाम् । क्लृप्त्यै᳚ । ब्र॒ह्म॒वा॒दिन॒ इति॑ ब्रह्म - वा॒दिनः॑ । व॒द॒न्ति॒ । प्र॒या॒जव॒दिति॑ प्रया॒ज - व॒त् । अ॒न॒नू॒या॒जमित्य॑ननु - या॒जम् । प्रा॒य॒णीय॒मिति॑ प्र - अ॒य॒णीय᳚म् । का॒र्य᳚म् । अ॒नू॒या॒जव॒दित्य॑नूया॒ज - व॒त् । \textbf{  36} \newline
                  \newline
                                \textbf{ TS 6.1.5.4} \newline
                  अ॒प्र॒या॒जमित्य॑प्र - या॒जम् । उ॒द॒य॒नीय॒मित्यु॑त् - अ॒य॒नीय᳚म् । इति॑ । इ॒मे । वै । प्र॒या॒जा इति॑ प्र - या॒जाः । अ॒मी इति॑ । अ॒नू॒या॒जा इत्य॑नु - या॒जाः । सा । ए॒व । सा । य॒ज्ञ्स्य॑ । सन्त॑ति॒रिति॒ सं - त॒तिः॒ । तत् । तथा᳚ । न । का॒र्य᳚म् । आ॒त्मा । वै । प्र॒या॒जा इति॑ प्र - या॒जाः । प्र॒जेति॑ प्र - जा । अ॒नू॒या॒जा इत्य॑नु-या॒जाः । यत् । प्र॒या॒जानिति॑ प्र-या॒जान् । अ॒न्त॒रि॒यादित्य॑न्तः - इ॒यात् । आ॒त्मान᳚म् । अ॒न्तः । इ॒या॒त् । यत् । अ॒नू॒या॒जानित्य॑नु - या॒जान् । अ॒न्त॒रि॒यादित्य॑न्तः - इ॒यात् । प्र॒जामिति॑ प्र-जाम् । अ॒न्तः । इ॒या॒त् । यतः॑ । खलु॑ । वै । य॒ज्ञ्स्य॑ । वित॑त॒स्येति॒ वि-त॒त॒स्य॒ । न । क्रि॒यते᳚ । तत् । अन्विति॑ । य॒ज्ञ्ः । परेति॑ । भ॒व॒ति॒ । य॒ज्ञ्म् । प॒रा॒भव॑न्त॒मिति॑ परा - भव॑न्तम् । यज॑मानः । अनु॑ । \textbf{  37} \newline
                  \newline
                                \textbf{ TS 6.1.5.5} \newline
                  परेति॑ । भ॒व॒ति॒ । प्र॒या॒जव॒दिति॑ प्रया॒ज - व॒त् । ए॒व । अ॒नू॒या॒जव॒दित्य॑नूया॒ज - व॒त् । प्रा॒य॒णीय॒मिति॑ प्र - अ॒य॒नीय᳚म् । का॒र्य᳚म् । प्र॒या॒जव॒दिति॑ प्रया॒ज - व॒त् । अ॒नू॒या॒जव॒दित्य॑नूया॒ज-व॒त् । उ॒द॒य॒नीय॒मित्यु॑त्-अ॒य॒नीय᳚म् । न । आ॒त्मान᳚म् । अ॒न्त॒रेतीत्य॑न्तः-एति॑ । न । प्र॒जामिति॑ प्र - जाम् । न । य॒ज्ञ्ः । प॒रा॒भव॒तीति॑ परा - भव॑ति । न । यज॑मानः । प्रा॒य॒णीय॒स्येति॑ प्र - अ॒य॒नीय॑स्य । नि॒ष्का॒से । उ॒द॒य॒नीय॒मित्यु॑त् - अ॒य॒नीय᳚म् । अ॒भि । निरिति॑ । व॒प॒ति॒ । सा । ए॒व । सा । य॒ज्ञ्स्य॑ । सन्त॑ति॒रिति॒ सं - त॒तिः॒ । याः । प्रा॒य॒णीय॒स्येति॑ प्र - अ॒य॒नीय॑स्य । या॒ज्याः᳚ । यत् । ताः । उ॒द॒य॒नीय॒स्येत्यु॑त्-अ॒य॒नीय॑स्य । या॒ज्याः᳚ । कु॒र्यात् । पराङ्॑ । अ॒मुम् । लो॒कम् । एति॑ । रो॒हे॒त् । प्र॒मायु॑क॒ इति॑ प्र - मायु॑कः । स्या॒त् । याः । प्रा॒य॒णीय॒स्येति॑ प्र - अ॒य॒नीय॑स्य । पु॒रो॒नु॒वा॒क्या॑ इति॑ पुरः - अ॒नु॒वा॒क्याः᳚ । ताः ( ) । उ॒द॒य॒नीय॒स्येत्यु॑त् - अ॒य॒नीय॑स्य । या॒ज्याः᳚ । क॒रो॒ति॒ । अ॒स्मिन्न् । ए॒व । लो॒के । प्रतीति॑ । ति॒ष्ठ॒ति॒ ॥ \textbf{  38} \newline
                  \newline
                      (प्रज्ञा᳚त्यै - पश्यत्य - नूया॒जव॒ - द्यज॑मा॒नोऽनु॑ - पुरोनुवा॒क्या᳚स्ता - अ॒ष्टौ च॑)  \textbf{(A5)} \newline \newline
                                \textbf{ TS 6.1.6.1} \newline
                  क॒द्रूः । च॒ । वै । सु॒प॒र्णीति॑ सु - प॒र्णी । च॒ । आ॒त्म॒रू॒पयो॒रित्या᳚त्म - रू॒पयोः᳚ । अ॒स्प॒द्‌र्धे॒ता॒म् । सा । क॒द्रूः । सु॒प॒र्णीमिति॑ सु-प॒र्णीम् । अ॒ज॒य॒त् । सा । अ॒ब्र॒वी॒त् । तृ॒तीय॑स्याम् । इ॒तः । दि॒वि । सोमः॑ । तम् । एति॑ । ह॒र॒ । तेन॑ । आ॒त्मान᳚म् । निरिति॑ । क्री॒णी॒ष्व॒ । इति॑ । इ॒यम् । वै । क॒द्रूः । अ॒सौ । सु॒प॒र्णीति॑ सु-प॒र्णी । छन्दाꣳ॑सि । सौ॒प॒र्णे॒याः । सा । अ॒ब्र॒वी॒त् । अ॒स्मै । वै । पि॒तरौ᳚ । पु॒त्रान् । बि॒भृ॒तः॒ । तृ॒तीय॑स्याम् । इ॒तः । दि॒वि । सोमः॑ । तम् । एति॑ । ह॒र॒ । तेन॑ । आ॒त्मान᳚म् । निरिति॑ । क्री॒णी॒ष्व॒ । \textbf{  39} \newline
                  \newline
                                \textbf{ TS 6.1.6.2} \newline
                  इति॑ । मा॒ । क॒द्रूः । अ॒वो॒च॒त् । इति॑ । जग॑ती । उदिति॑ । अ॒प॒त॒त् । चतु॑र्दशाक्ष॒रेति॒ चतु॑र्दश -   अ॒क्ष॒रा॒ । स॒ती । सा । अप्रा॒प्येत्यप्र॑ - आ॒प्य॒ । नीति॑ । अ॒व॒र्त॒त॒ । तस्यै᳚ । द्वे इति॑ । अ॒क्षरे॒ इति॑ । अ॒मी॒ये॒ता॒म् । सा । प॒शुभि॒रिति॑ प॒शु - भिः॒ । च॒ । दी॒क्षया᳚ । च॒ । एति॑ । अ॒ग॒च्छ॒त् । तस्मा᳚त् । जग॑ती । छन्द॑साम् । प॒श॒व्य॑त॒मेति॑ पश॒व्य॑ - त॒मा॒ । तस्मा᳚त् । प॒शु॒मन्त॒मिति॑ पशु - मन्त᳚म् । दी॒क्षा । उपेति॑ । न॒म॒ति॒ । त्रि॒ष्टुक् । उदिति॑ । अ॒प॒त॒त् । त्रयो॑दशाक्ष॒रेति॒ त्रयो॑दश - अ॒क्ष॒रा॒ । स॒ती । सा । अप्रा॒प्येत्यप्र॑ - आ॒प्य॒ । नीति॑ । अ॒व॒र्त॒त॒ । तस्यै᳚ । द्वे इति॑ । अ॒क्षरे॒ इति॑ । अ॒मी॒ये॒ता॒म् । सा । दक्षि॑णाभिः । च॒ । \textbf{  40} \newline
                  \newline
                                \textbf{ TS 6.1.6.3} \newline
                  तप॑सा । च॒ । एति॑ । अ॒ग॒च्छ॒त् । तस्मा᳚त् । त्रि॒ष्टुभः॑ । लो॒के । माद्ध्य॑न्दिने । सव॑ने । दक्षि॑णाः । नी॒य॒न्ते॒ । ए॒तत् । खलु॑ । वाव । तपः॑ । इति॑ । आ॒हुः॒ । यः । स्वम् । ददा॑ति । इति॑ । गा॒य॒त्री । उदिति॑ । अ॒प॒त॒त् । चतु॑रक्ष॒रेति॒ चतुः॑ - अ॒क्ष॒रा॒ । स॒ती । अ॒जया᳚ । ज्योति॑षा । तम् । अ॒स्यै॒ । अ॒जा । अ॒भीति॑ । अ॒रु॒न्ध॒ । तत् । अ॒जायाः᳚ । अ॒ज॒त्वमित्य॑ज-त्वम् । सा । सोम᳚म् । च॒ । एति॑ । अह॑रत् । च॒त्वारि॑ । च॒ । अ॒क्षरा॑णि । सा । अ॒ष्टाक्ष॒रेत्य॒ष्टा - अ॒क्ष॒रा॒ । समिति॑ । अ॒प॒द्य॒त॒ । ब्र॒ह्म॒वा॒दिन॒ इति॑ ब्रह्म - वा॒दिनः॑ । व॒द॒न्ति॒ । \textbf{  41} \newline
                  \newline
                                \textbf{ TS 6.1.6.4} \newline
                  कस्मा᳚त् । स॒त्यात् । गा॒य॒त्री । कनि॑ष्ठा । छन्द॑साम् । स॒ती । य॒ज्ञ्॒मु॒खमिति॑ यज्ञ् - मु॒खम् । परीति॑ । इ॒या॒य॒ । इति॑ । यत् । ए॒व । अ॒दः । सोम᳚म् । एति॑ । अह॑रत् । तस्मा᳚त् । य॒ज्ञ्॒मु॒खमिति॑ यज्ञ् - मु॒खम् । परीति॑ । ऐ॒त् । तस्मा᳚त् । ते॒ज॒स्विनी॑त॒मेति॑ तेज॒स्विनी᳚-त॒मा॒ । प॒द्भ्यामिति॑ पत्-भ्याम् । द्वे इति॑ । सव॑ने॒ इति॑ । स॒मगृ॑ह्णा॒दिति॑ सं - अगृ॑ह्णात् । मुखे॑न । एक᳚म् । यत् । मुखे॑न । स॒मगृ॑ह्णा॒दिति॑ सं - अगृ॑ह्णात् । तत् । अ॒ध॒य॒त् । तस्मा᳚त् । द्वे इति॑ । सव॑ने॒ इति॑ । शु॒क्रव॑ती॒ इति॑ शु॒क्र - व॒ती॒ । प्रा॒त॒स्स॒व॒नमिति॑ प्रातः - स॒व॒नम् । च॒ । माद्ध्य॑न्दिनम् । च॒ । तस्मा᳚त् । तृ॒ती॒य॒स॒व॒न इति॑ तृतीय - स॒व॒ने । ऋ॒जी॒षम् । अ॒भीति॑ । सु॒न्व॒न्ति॒ । धी॒तम् । इ॒व॒ । हि । मन्य॑न्ते । \textbf{  42} \newline
                  \newline
                                \textbf{ TS 6.1.6.5} \newline
                  आ॒शिर᳚म् । अवेति॑ । न॒य॒ति॒ । स॒शु॒क्र॒त्वायेति॑ सशुक्र - त्वाय॑ । अथो॒ इति॑ । समिति॑ । भ॒र॒ति॒ । ए॒व । ए॒न॒त् । तम् । सोम᳚म् । आ॒ह्रि॒यमा॑ण॒मित्या᳚ - ह्रि॒यमा॑णम् । ग॒न्ध॒र्वः । वि॒श्वाव॑सु॒रिति॑ वि॒श्व - व॒सुः॒ । परीति॑ । अ॒मु॒ष्णा॒त् । सः । ति॒स्रः । रात्रीः᳚ । परि॑मुषित॒ इति॒ परि॑ - मु॒षि॒तः॒ । अ॒व॒स॒त् । तस्मा᳚त् । ति॒स्रः । रात्रीः᳚ । क्री॒तः । सोमः॑ । व॒स॒ति॒ । ते । दे॒वाः । अ॒ब्रु॒व॒न्न् । स्त्रीका॑मा॒ इति॒ स्त्री - का॒माः॒ । वै । ग॒न्ध॒र्वाः । स्त्रि॒या । निरिति॑ । क्री॒णा॒म॒ । इति॑ । ते । वाच᳚म् । स्त्रिय᳚म् । एक॑हायनी॒मित्येक॑ - हा॒य॒नी॒म् । कृ॒त्वा । तया᳚ । निरिति॑ । अ॒क्री॒ण॒न्न् । सा । रो॒हित् । रू॒पम् । कृ॒त्वा । ग॒न्ध॒र्वेभ्यः॑ । \textbf{  43} \newline
                  \newline
                                \textbf{ TS 6.1.6.6} \newline
                  अ॒प॒क्रम्येत्य॑प - क्रम्य॑ । अ॒ति॒ष्ठ॒त् । तत् । रो॒हितः॑ । जन्म॑ । ते । दे॒वाः । अ॒ब्रु॒व॒न्न् । अपेति॑ । यु॒ष्मत् । अक्र॑मीत् । न । अ॒स्मान् । उ॒पाव॑र्तत॒ इत्यु॑प - आव॑र्तते । वीति॑ । ह्व॒या॒म॒है॒ । इति॑ । ब्रह्म॑ । ग॒न्ध॒र्वाः । अव॑दन्न् । अगा॑यन्न् । दे॒वाः । सा । दे॒वान् । गाय॑तः । उ॒पाव॑र्त॒तेत्यु॑प-आव॑र्तत । तस्मा᳚त् । गाय॑न्तम् । स्त्रियः॑ । का॒म॒य॒न्ते॒ । कामु॑काः । ए॒न॒म् । स्त्रियः॑ । भ॒व॒न्ति॒ । यः । ए॒वम् । वेद॑ । अथो॒ इति॑ । यः । ए॒वम् । वि॒द्वान् । अपीति॑ । जन्ये॑षु । भव॑ति । तेभ्यः॑ । ए॒व । द॒द॒ति॒ । उ॒त । यत् । ब॒हुत॑या॒ इति॑ ब॒हु - त॒याः॒ । \textbf{  44} \newline
                  \newline
                                \textbf{ TS 6.1.6.7} \newline
                  भव॑न्ति । एक॑हाय॒न्येत्येक॑ - हा॒य॒न्या॒ । क्री॒णा॒ति॒ । वा॒चा । ए॒व । ए॒न॒म् । सर्व॑या । क्री॒णा॒ति॒ । तस्मा᳚त् । एक॑हायना॒ इत्येक॑-हा॒य॒नाः॒ । म॒नु॒ष्याः᳚ । वाच᳚म् । व॒द॒न्ति॒ । अकू॑टया । अक॑र्णया । अका॑णया । अश्लो॑णया । अस॑प्तशफ॒येत्यस॑प्त - श॒फ॒या॒ । क्री॒णा॒ति॒ । सर्व॑या । ए॒व । ए॒न॒म् । क्री॒णा॒ति॒ । यत् । श्वे॒तया᳚ । क्री॒णी॒यात् । दु॒श्चर्मेति॑ दुः - चर्मा᳚ । यज॑मानः । स्या॒त् । यत् । कृ॒ष्णया᳚ । अ॒नु॒स्तर॒णीत्य॑नु - स्तर॑णी । स्या॒त् । प्र॒मायु॑क॒ इति॑ प्र - मायु॑कः । यज॑मानः । स्या॒त् । यत् । द्वि॒रू॒पयेति॑ द्वि - रू॒पया᳚ । वात्र॒घ्नीति॒ वात्र॑ - घ्नी॒ । स्या॒त् । सः । वा॒ । अ॒न्यम् । जि॒नी॒यात् । तम् । वा॒ । अ॒न्यः । जि॒नी॒या॒त् । अ॒रु॒णया᳚ । पि॒ङ्गा॒क्ष्येति॑ पिङ्ग - अ॒क्ष्या ( ) । क्री॒णा॒ति॒ । ए॒तत् । वै । सोम॑स्य । रू॒पम् । स्वया᳚ । ए॒व । ए॒न॒म् । दे॒वत॑या । क्री॒णा॒ति॒ ॥ \textbf{  45 } \newline
                  \newline
                      (निष्क्री॑णीष्व॒ - दक्षि॑णाभिश्च - वदन्ति॒ - मन्य॑न्ते-गन्ध॒र्वेभ्यो॑-ब॒हुत॑याः-पिङ्गा॒क्ष्या-दश॑ च )  \textbf{(A6)} \newline \newline
                                \textbf{ TS 6.1.7.1} \newline
                  तत् । हिर॑ण्यम् । अ॒भ॒व॒त् । तस्मा᳚त् । अ॒द्भ्य इत्य॑त् - भ्यः । हिर॑ण्यम् । पु॒न॒न्ति॒ । ब्र॒ह्म॒वा॒दिन॒ इति॑ ब्रह्म - वा॒दिनः॑ । व॒द॒न्ति॒ । कस्मा᳚त् । स॒त्यात् । अ॒न॒स्थिके॑न । प्र॒जा इति॑ प्र - जाः । प्र॒वीय॑न्त॒ इति॑ प्र - वीय॑न्ते । अ॒स्थ॒न्वती॒रित्य॑स्थन्न् - वतीः᳚ । जा॒य॒न्ते॒ । इति॑ । यत् । हिर॑ण्यम् । घृ॒ते । अ॒व॒धायेत्य॑व - धाय॑ । जु॒होति॑ । तस्मा᳚त् । अ॒न॒स्थिके॑न । प्र॒जा इति॑ प्र - जाः । प्रेति॑ । वी॒य॒न्ते॒ । अ॒स्थ॒न्वती॒रित्य॑स्थन्न् - वतीः᳚ । जा॒य॒न्ते॒ । ए॒तत् । वै । अ॒ग्नेः । प्रि॒यम् । धाम॑ । यत् । घृ॒तम् । तेजः॑ । हिर॑ण्यम् । इ॒यम् । ते॒ । शु॒क्र॒ । त॒नूः । इ॒दम् । वर्चः॑ । इति॑ । आ॒ह॒ । सते॑जस॒मिति॒ स - ते॒ज॒स॒म् । ए॒व । ए॒न॒म् । सत॑नु॒मिति॒ स - त॒नु॒म् । \textbf{  46} \newline
                  \newline
                                \textbf{ TS 6.1.7.2} \newline
                  क॒रो॒ति॒ । अथो॒ इति॑ । समिति॑ । भ॒र॒ति॒ । ए॒व । ए॒न॒म् । यत् । अब॑द्धम् । अ॒व॒द॒द्ध्यादित्य॑व -द॒द्ध्यात् । गर्भाः᳚ । प्र॒जाना॒मिति॑ प्र - जाना᳚म् । प॒रा॒पातु॑का॒ इति॑ परा-पातु॑काः । स्युः॒ । ब॒द्धम् । अवेति॑ । द॒धा॒ति॒ । गर्भा॑णाम् । धृत्यै᳚ । नि॒ष्ट॒र्क्य᳚म् । ब॒द्ध्ना॒ति॒ । प्र॒जाना॒मिति॑ प्र - जाना᳚म् । प्र॒जन॑ना॒येति॑ प्र-जन॑नाय । वाक् । वै । ए॒षा । यत् । सो॒म॒क्रय॒णीति॑ सोम - क्रय॑णी । जूः । अ॒सि॒ । इति॑ । आ॒ह॒ । यत् । हि । मन॑सा । जव॑ते । तत् । वा॒चा । वद॑ति । धृ॒ता । मन॑सा । इति॑ । आ॒ह॒ । मन॑सा । हि । वाक् । धृ॒ता । जुष्टा᳚ । विष्ण॑वे । इति॑ । आ॒ह॒ । \textbf{  47} \newline
                  \newline
                                \textbf{ TS 6.1.7.3} \newline
                  य॒ज्ञ्ः । वै । विष्णुः॑ । य॒ज्ञाय॑ । ए॒व । ए॒ना॒म् । जुष्टा᳚म् । क॒रो॒ति॒ । तस्याः᳚ । ते॒ । स॒त्यस॑वस॒ इति॑ स॒त्य - स॒व॒सः॒ । प्र॒स॒व इति॑ प्र - स॒वे । इति॑ । आ॒ह॒ । स॒वि॒तृप्र॑सूता॒मिति॑ सवि॒तृ - प्र॒सू॒ता॒म् । ए॒व । वाच᳚म् । अवेति॑ । रु॒न्धे॒ । काण्डे॑काण्ड॒ इति॒ काण्डे᳚-का॒ण्डे॒ । वै । क्रि॒यमा॑णे । य॒ज्ञ्म् । रक्षाꣳ॑सि । जि॒घाꣳ॒॒स॒न्ति॒ । ए॒षः । खलु॑ । वै । अर॑क्षोहत॒ इत्यर॑क्षः - ह॒तः॒ । पन्थाः᳚ । यः । अ॒ग्नेः । च॒ । सूर्य॑स्य । च॒ । सूर्य॑स्य । चक्षुः॑ । एति॑ । अ॒रु॒ह॒म् । अ॒ग्नेः । अ॒क्ष्णः । क॒नीनि॑काम् । इति॑ । आ॒ह॒ । यः । ए॒व । अर॑क्षोहत॒ इत्यर॑क्षः-ह॒तः॒ । पन्थाः᳚ । तम् । स॒मारो॑ह॒तीति॑ सं - आरो॑हति । \textbf{  48} \newline
                  \newline
                                \textbf{ TS 6.1.7.4} \newline
                  वाक् । वै । ए॒षा । यत् । सो॒म॒क्रय॒णीति॑ सोम - क्रय॑णी । चित् । अ॒सि॒ । म॒ना । अ॒सि॒ । इति॑ । आ॒ह॒ । शास्ति॑ । ए॒व । ए॒ना॒म् । ए॒तत् । तस्मा᳚त् । शि॒ष्टाः । प्र॒जा इति॑ प्र - जाः । जा॒य॒न्ते॒ । चित् । अ॒सि॒ । इति॑ । आ॒ह॒ । यत् । हि । मन॑सा । चे॒तय॑ते । तत् । वा॒चा । वद॑ति । म॒ना । अ॒सि॒ । इति॑ । आ॒ह॒ । यत् । हि । मन॑सा । अ॒भि॒गच्छ॒तीत्य॑भि - गच्छ॑ति । तत् । क॒रोति॑ । धीः । अ॒सि॒ । इति॑ । आ॒ह॒ । यत् । हि । मन॑सा । ध्याय॑ति । तत् । वा॒चा । \textbf{  49} \newline
                  \newline
                                \textbf{ TS 6.1.7.5} \newline
                  वद॑ति । दक्षि॑णा । अ॒सि॒ । इति॑ । आ॒ह॒ । दक्षि॑णा । हि । ए॒षा । य॒ज्ञिया᳚ । अ॒सि॒ । इति॑ । आ॒ह॒ । य॒ज्ञिया᳚म् । ए॒व । ए॒ना॒म् । क॒रो॒ति॒ । क्ष॒त्रिया᳚ । अ॒सि॒ । इति॑ । आ॒ह॒ । क्ष॒त्रिया᳚ । हि । ए॒षा । अदि॑तिः । अ॒सि॒ । उ॒भ॒यत॑श्शी॒र्ष्णीत्यु॑भ॒यतः॑ - शी॒र्ष्णी॒ । इति॑ । आ॒ह॒ । यत् । ए॒व । आ॒दि॒त्यः । प्रा॒य॒णीय॒ इति॑ प्र - अ॒य॒णीयः॑ । य॒ज्ञाना᳚म् । आ॒दि॒त्यः । उ॒द॒य॒नीय॒ इत्यु॑त् - अ॒य॒नीयः॑ । तस्मा᳚त् । ए॒वम् । आ॒ह॒ । यत् । अब॑द्धा । स्यात् । अय॑ता । स्या॒त् । यत् । प॒दि॒ब॒द्धेति॑ पदि - ब॒द्धा । अ॒नु॒स्तर॒णीत्य॑नु - स्तर॑णी । स्या॒त् । प्र॒मायु॑क॒ इति॑ प्र - मायु॑कः । यज॑मानः । स्या॒त् । \textbf{  50} \newline
                  \newline
                                \textbf{ TS 6.1.7.6} \newline
                  यत् । क॒र्ण॒गृ॒ही॒तेति॑ कर्ण-गृ॒ही॒ता । वार्त्र॒घ्नीति॒ वार्त्र॑ - घ्नी॒ । स्या॒त् । सः । वा॒ । अ॒न्यम् । जि॒नी॒यात् । तम् । वा॒ । अ॒न्यः । जि॒नी॒या॒त् । मि॒त्रः । त्वा॒ । प॒दि । ब॒द्ध्ना॒तु॒ । इति॑ । आ॒ह॒ । मि॒त्रः । वै । शि॒वः । दे॒वाना᳚म् । तेन॑ । ए॒व । ए॒ना॒म् । प॒दि । ब॒द्ध्ना॒ति॒ । पू॒षा । अद्ध्व॑नः । पा॒तु॒ । इति॑ । आ॒ह॒ । इ॒यम् । वै । पू॒षा । इ॒माम् । ए॒व । अ॒स्याः॒ । अ॒धि॒पामित्य॑धि - पाम् । अ॒कः॒ । सम॑ष्ट्या॒ इति॒ सं - अ॒ष्ट्यै॒ । इन्द्रा॑य । अद्ध्य॑क्षा॒येत्यधि॑ - अ॒क्षा॒य॒ । इति॑ । आ॒ह॒ । इन्द्र᳚म् । ए॒व । अ॒स्याः॒ । अद्ध्य॑क्ष॒मित्यधि॑ - अ॒क्ष॒म् । क॒रो॒ति॒ । \textbf{  51} \newline
                  \newline
                                \textbf{ TS 6.1.7.7} \newline
                  अन्विति॑ । त्वा॒ । मा॒ता । म॒न्य॒ता॒म् । अन्विति॑ । पि॒ता । इति॑ । आ॒ह॒ । अनु॑मत॒येत्यनु॑ - म॒त॒या॒ । ए॒व । ए॒न॒या॒ । क्री॒णा॒ति॒ । सा । दे॒वि॒ । दे॒वम् । अच्छ॑ । इ॒हि॒ । इति॑ । आ॒ह॒ । दे॒वी । हि । ए॒षा । दे॒वः । सोमः॑ । इन्द्रा॑य । सोम᳚म् । इति॑ । आ॒ह॒ । इन्द्रा॑य । हि । सोमः॑ । आ॒ह्रि॒यत॒ इत्या᳚ - ह्रि॒यते᳚ । यत् । ए॒तत् । यजुः॑ । न । ब्रू॒यात् । परा॑ची । ए॒व । सो॒म॒क्रय॒णीति॑ सोम - क्रय॑णी । इ॒या॒त् । रु॒द्रः । त्वा॒ । एति॑ । व॒र्त॒य॒तु॒ । इति॑ । आ॒ह॒ । रु॒द्रः । वै । क्रू॒रः । \textbf{  52} \newline
                  \newline
                                \textbf{ TS 6.1.7.8} \newline
                  दे॒वाना᳚म् । तम् । ए॒व । अ॒स्यै॒ । प॒रस्ता᳚त् । द॒धा॒ति॒ । आवृ॑त्त्या॒ इत्या - वृ॒त्त्यै॒ । क्रू॒रम् । इ॒व॒ । वै । ए॒तत् । क॒रो॒ति॒ । यत् । रु॒द्रस्य॑ । की॒र्तय॑ति । मि॒त्रस्य॑ । प॒था । इति॑ । आ॒ह॒ । शान्त्यै᳚ । वा॒चा । वै । ए॒षः । वीति॑ । क्री॒णी॒ते॒ । यः । सो॒म॒क्रय॒ण्येति॑ सोम - क्रय॑ण्या । स्व॒स्ति । सोम॑स॒खेति॒ सोम॑ - स॒खा॒ । पुनः॑ । एति॑ । इ॒हि॒ । स॒ह । र॒य्या । इति॑ । आ॒ह॒ । वा॒चा । ए॒व । वि॒क्रीयेति॑ वि - क्रीय॑ । पुनः॑ । आ॒त्मन् । वाच᳚म् । ध॒त्ते॒ । अनु॑पदासु॒केत्यनु॑प - दा॒सु॒का॒ । अ॒स्य॒ । वाक् । भ॒व॒ति॒ । यः । ए॒वम् । वेद॑ ( ) ॥ \textbf{  53 } \newline
                  \newline
                      (सत॑नुं॒ - ॅविष्ण॑व॒ इत्या॑ह - स॒मारो॑हति॒ - ध्याय॑ति॒ तद् वा॒चा - यज॑मानः स्यात् - करोति - क्रू॒रो - वेद॑)  \textbf{(A7)} \newline \newline
                                \textbf{ TS 6.1.8.1} \newline
                  षट् । प॒दानि॑ । अनु॑ । नीति॑ । क्रा॒म॒ति॒ । ष॒ड॒हमिति॑ षट् - अ॒हम् । वाक् । न । अतीति॑ । व॒द॒ति॒ । उ॒त । सं॒ॅव॒थ्स॒रस्येति॑ सं-व॒थ्स॒रस्य॑ । अय॑ने । याव॑ती । ए॒व । वाक् । ताम् । अवेति॑ । रु॒न्धे॒ । स॒प्त॒मे । प॒दे । जु॒हो॒ति॒ । स॒प्तप॒देति॑ स॒प्त - प॒दा॒ । शक्व॑री । प॒शवः॑ । शक्व॑री । प॒शून् । ए॒व । अवेति॑ । रु॒न्धे॒ । स॒प्त । ग्रा॒म्याः । प॒शवः॑ । स॒प्त । आ॒र॒ण्याः । स॒प्त । छन्दाꣳ॑सि । उ॒भय॑स्य । अव॑रुद्ध्या॒ इत्यव॑ - रु॒द्ध्यै॒ । वस्वी᳚ । अ॒सि॒ । रु॒द्रा । अ॒सि॒ । इति॑ । आ॒ह॒ । रू॒पम् । ए॒व । अ॒स्याः॒ । ए॒तत् । म॒हि॒मान᳚म् । \textbf{  54} \newline
                  \newline
                                \textbf{ TS 6.1.8.2} \newline
                  व्याच॑ष्ट॒ इति॑ वि-आच॑ष्टे । बृह॒स्पतिः॑ । त्वा॒ । सु॒म्ने । र॒ण्व॒तु॒ । इति॑ । आ॒ह॒ । ब्रह्म॑ । वै । दे॒वाना᳚म् । बृह॒स्पतिः॑ । ब्रह्म॑णा । ए॒व । अ॒स्मै॒ । प॒शून् । अवेति॑ । रु॒न्धे॒ । रु॒द्रः । वसु॑भि॒रिति॒ वसु॑ - भिः॒ । एति॑ । चि॒के॒तु॒ । इति॑ । आ॒ह॒ । आवृ॑त्त्या॒ इत्या - वृ॒त्त्यै॒ । पृ॒थि॒व्याः । त्वा॒ । मू॒द्‌र्धन्न् । एति॑ । जि॒घ॒र्मि॒ । दे॒व॒यज॑न॒ इति॑ देव-यज॑ने । इति॑ । आ॒ह॒ । पृ॒थि॒व्याः । हि । ए॒षः । मू॒द्‌र्धा । यत् । दे॒व॒यज॑न॒मिति॑ देव - यज॑नम् । इडा॑याः । प॒दे । इति॑ । आ॒ह॒ । इडा॑यै । हि । ए॒तत् । प॒दम् । यत् । सो॒म॒क्रय॑ण्या॒ इति॑ सोम - क्रय॑ण्यै । घृ॒तव॒तीति॑ घृ॒त-व॒ति॒ । स्वाहा᳚ । \textbf{  55} \newline
                  \newline
                                \textbf{ TS 6.1.8.3} \newline
                  इति॑ । आ॒ह॒ । यत् । ए॒व । अ॒स्यै॒ । प॒दात् । घृ॒तम् । अपी᳚ड्यत । तस्मा᳚त् । ए॒वम् । आ॒ह॒ । यत् । अ॒द्ध्व॒र्युः । अ॒न॒ग्नौ । आहु॑ति॒मित्या - हु॒ति॒म् । जु॒हु॒यात् । अ॒न्धः । अ॒द्ध्व॒र्युः । स्या॒त् । रक्षाꣳ॑सि । य॒ज्ञ्म् । ह॒न्युः॒ । हिर॑ण्यम् । उ॒पास्येत्यु॑प - अस्य॑ । जु॒हो॒ति॒ । अ॒ग्नि॒वतीत्य॑ग्नि - वति॑ । ए॒व । जु॒हो॒ति॒ । न । अ॒न्धः । अ॒द्ध्व॒र्युः । भव॑ति । न । य॒ज्ञ्म् । रक्षाꣳ॑सि । घ्न॒न्ति॒ । काण्डे॑काण्ड॒ इति॒ काण्डे᳚ - का॒ण्डे॒ । वै । क्रि॒यमा॑णे । य॒ज्ञ्म् । रक्षाꣳ॑सि । जि॒घाꣳ॒॒स॒न्ति॒ । परि॑लिखित॒मिति॒ परि॑ - लि॒खि॒त॒म् । रक्षः॑ । परि॑लिखिता॒ इति॒ परि॑ - लि॒खि॒ताः॒ । अरा॑तयः । इति॑ । आ॒ह॒ । रक्ष॑साम् । अप॑हत्या॒ इत्यप॑ - ह॒त्यै॒ । \textbf{  56} \newline
                  \newline
                                \textbf{ TS 6.1.8.4} \newline
                  इ॒दम् । अ॒हम् । रक्ष॑सः । ग्री॒वाः । अपीति॑ । कृ॒न्ता॒मि॒ । यः । अ॒स्मान् । द्वेष्टि॑ । यम् । च॒ । व॒यम् । द्वि॒ष्मः । इति॑ । आ॒ह॒ । द्वौ । वाव । पुरु॑षौ । यम् । च॒ । ए॒व । द्वेष्टि॑ । यः । च॒ । ए॒न॒म् । द्वेष्टि॑ । तयोः᳚ । ए॒व । अन॑न्तराय॒मित्यन॑न्तः - आ॒य॒म् । ग्री॒वाः । कृ॒न्त॒ति॒ । प॒शवः॑ । वै । सो॒म॒क्रय॑ण्या॒ इति॑ सोम - क्रय॑ण्यै । प॒दम् । या॒वत्त्मू॒तमिति॑ यावत् - त्मू॒तम् । समिति॑ । व॒प॒ति॒ । प॒शून् । ए॒व । अवेति॑ । रु॒न्धे॒ । अ॒स्मे इति॑ । रायः॑ । इति॑ । समिति॑ । व॒प॒ति॒ । आ॒त्मान᳚म् । ए॒व । अ॒द्ध्व॒र्युः । \textbf{  57} \newline
                  \newline
                                \textbf{ TS 6.1.8.5} \newline
                  प॒शुभ्य॒ इति॑ प॒शु - भ्यः॒ । न । अ॒न्तः । ए॒ति॒ । त्वे इति॑ । रायः॑ । इति॑ । यज॑मानाय । प्रेति॑ । य॒च्छ॒ति॒ । यज॑माने । ए॒व । र॒यिम् । द॒धा॒ति॒ । तोते᳚ । रायः॑ । इति॑ । पत्नि॑यै । अ॒द्‌र्धः । वै । ए॒षः । आ॒त्मनः॑ । यत् । पत्नी᳚ । यथा᳚ । गृ॒हेषु॑ । नि॒ध॒त्त इति॑ नि - ध॒त्ते । ता॒दृक् । ए॒व । तत् । त्वष्टी॑मती । ते॒ । स॒पे॒य॒ । इति॑ । आ॒ह॒ । त्वष्टा᳚ । वै । प॒शू॒नाम् । मि॒थु॒नाना᳚म् । रू॒प॒कृदिति॑ रूप - कृत् । रू॒पम् । ए॒व । प॒शुषु॑ । द॒धा॒ति॒ । अ॒स्मै । वै । लो॒काय॑ । गार्.ह॑पत्य॒ इति॒ गार्.ह॑ - प॒त्यः॒ । एति॑ । धी॒य॒ते॒ ( ) । अ॒मुष्मै᳚ । आ॒ह॒व॒नीय॒ इत्या᳚ - ह॒व॒नीयः॑ । यत् । गार्.ह॑पत्य॒ इति॒ गार्.ह॑- प॒त्ये॒ । उ॒प॒वपे॒दित्यु॑प - वपे᳚त् । अ॒स्मिन्न् । लो॒के । प॒शु॒मानिति॑ पशु - मान् । स्या॒त् । यत् । आ॒ह॒व॒नीय॒ इत्या᳚ - ह॒व॒नीये᳚ । अ॒मुष्मिन्न्॑ । लो॒के । प॒शु॒मानिति॑ पशु - मान् । स्या॒त् । उ॒भयोः᳚ । उपेति॑ । व॒प॒ति॒ । उ॒भयोः᳚ । ए॒व । ए॒न॒म् । लो॒कयोः᳚ । प॒शु॒मन्त॒मिति॑ पशु - मन्त᳚म् । क॒रो॒ति॒ ॥ \textbf{  58} \newline
                  \newline
                      (म॒हि॒मानꣳ॒॒ - स्वाहा - ऽप॑हत्या - अध्व॒र्यु - धी॑यते॒ - चतु॑र्विꣳशतिश्च)  \textbf{(A8)} \newline \newline
                                \textbf{ TS 6.1.9.1} \newline
                  ब्र॒ह्म॒वा॒दिन॒ इति॑ ब्रह्म - वा॒दिनः॑ । व॒द॒न्ति॒ । वि॒चित्य॒ इति॑ वि-चित्यः॑ । सोमा(3)ः । न । वि॒चित्या(3) इति वि - चित्या(3)ः । इति॑ । सोमः॑ । वै । ओष॑धीनाम् । राजा᳚ । तस्मिन्न्॑ । यत् । आप॑न्न॒मित्या - प॒न्न॒म् । ग्र॒सि॒तम् । ए॒व । अ॒स्य॒ । तत् । यत् । वि॒चि॒नु॒यादिति॑ वि - चि॒नु॒यात् । यथा᳚ । आ॒स्या᳚त् । ग्र॒सि॒तम् । नि॒ष्खि॒दतीति॑ निः -   खि॒दति॑ । ता॒दृक् । ए॒व । तत् । यत् । न । वि॒चि॒नु॒यादिति॑ वि - चि॒नु॒यात् । यथा᳚ । अ॒क्षन्न् । आप॑न्न॒मित्या - प॒न्न॒म् । वि॒धाव॒तीति॑ वि - धाव॑ति । ता॒दृक् । ए॒व । तत् । क्षोधु॑कः । अ॒द्ध्व॒र्युः । स्यात् । क्षोधु॑कः । यज॑मानः । सोम॑विक्रयि॒न्निति॒ सोम॑ -वि॒क्र॒यि॒न्न् । सोम᳚म् । शो॒ध॒य॒ । इति॑ । ए॒व । ब्रू॒या॒त् । यदि॑ । इत॑रम् । \textbf{  59} \newline
                  \newline
                                \textbf{ TS 6.1.9.2} \newline
                  यदि॑ । इत॑रम् । उ॒भये॑न । ए॒व । सो॒म॒वि॒क्र॒यिण॒मिति॑ सोम - वि॒क्र॒यिण᳚म् । अ॒र्प॒य॒ति॒ । तस्मा᳚त् । सो॒म॒वि॒क्र॒यीति॑ सोम - वि॒क्र॒यी । क्षोधु॑कः । अ॒रु॒णः । ह॒ । स्म॒ । आ॒ह॒ । औप॑वेशि॒रित्यौप॑-वे॒शिः॒ । सो॒म॒क्रय॑ण॒ इति॑ सोम - क्रय॑णे । ए॒व । अ॒हम् । तृ॒ती॒य॒स॒व॒नमिति॑ तृतीय - स॒व॒नम् । अवेति॑ । रु॒न्धे॒ । इति॑ । प॒शू॒नाम् । चर्मन्न्॑ । मि॒मी॒ते॒ । प॒शून् । ए॒व । अवेति॑ । रु॒न्धे॒ । प॒शवः॑ । हि । तृ॒तीय᳚म् । सव॑नम् । यम् । का॒मये॑त । अ॒प॒शुः । स्या॒त् । इति॑ । ऋ॒क्ष॒तः । तस्य॑ । मि॒मी॒त॒ । ऋ॒क्षम् । वै । अ॒प॒श॒व्यम् । अ॒प॒शुः । ए॒व । भ॒व॒ति॒ । यम् । का॒मये॑त । प॒शु॒मानिति॑ पशु - मान् । स्या॒त् । \textbf{  60} \newline
                  \newline
                                \textbf{ TS 6.1.9.3} \newline
                  इति॑ । लो॒म॒तः । तस्य॑ । मि॒मी॒त॒ । ए॒तत् । वै । प॒शू॒नाम् । रू॒पम् । रू॒पेण॑ । ए॒व । अ॒स्मै॒ । प॒शून् । अवेति॑ । रु॒न्धे॒ । प॒शु॒मानिति॑ पशु - मान् । ए॒व । भ॒व॒ति॒ । अ॒पाम् । अन्ते᳚ । क्री॒णा॒ति॒ । सर॑स॒मिति॒ स-र॒स॒म् । ए॒व । ए॒न॒म् । क्री॒णा॒ति॒ । अ॒मात्यः॑ । अ॒सि॒ । इति॑ । आ॒ह॒ । अ॒मा । ए॒व । ए॒न॒म् । कु॒रु॒ते॒ । शु॒क्रः । ते॒ । ग्रहः॑ । इति॑ । आ॒ह॒ । शु॒क्रः । हि । अ॒स्य॒ । ग्रहः॑ । अन॑सा । अच्छ॑ । या॒ति॒ । म॒हि॒मान᳚म् । ए॒व । अ॒स्य॒ । अच्छ॑ । या॒ति॒ । अन॑सा । \textbf{  61} \newline
                  \newline
                                \textbf{ TS 6.1.9.4} \newline
                  अच्छ॑ । या॒ति॒ । तस्मा᳚त् । अ॒नो॒वा॒ह्य॑मित्य॑नः - वा॒ह्य᳚म् । स॒मे । जीव॑नम् । यत्र॑ । खलु॑ । वै । ए॒तम् । शी॒र्ष्णा । हर॑न्ति । तस्मा᳚त् । शी॒र्॒.ष॒हा॒र्य॑मिति॑ शीर्.ष-हा॒र्य᳚म् । गि॒रौ । जीव॑नम् । अ॒भीति॑ । त्यम् । दे॒वम् । स॒वि॒तार᳚म् । इति॑ । अति॑च्छन्द॒सेत्यति॑ - छ॒न्द॒सा॒ । ऋ॒चा । मि॒मी॒त॒ । अति॑च्छन्दा॒ इत्यति॑ - छ॒न्दाः॒ । वै । सर्वा॑णि । छन्दाꣳ॑सि । सर्वे॑भिः । ए॒व । ए॒न॒म् । छन्दो॑भि॒रिति॒ छन्दः॑-भिः॒ । मि॒मी॒ते॒ । वर्ष्म॑ । वै । ए॒षा । छन्द॑साम् । यत् । अति॑च्छन्दा॒ इत्यति॑ - छ॒न्दाः॒ । यत् । अति॑च्छन्द॒सेत्यति॑-छ॒न्द॒सा॒ । ऋ॒चा । मिमी॑ते । वर्ष्म॑ । ए॒व । ए॒न॒म् । स॒मा॒नाना᳚म् । क॒रो॒ति॒ । एक॑यैक॒येत्येक॑या - ए॒क॒या॒ । उ॒थ्सर्ग॒मित्यु॑त् - सर्ग᳚म् । \textbf{  62} \newline
                  \newline
                                \textbf{ TS 6.1.9.5} \newline
                  मि॒मी॒ते॒ । अया॑तयाम्नियायातयाम्नि॒येत्यया॑तयाम्निया-अ॒या॒त॒या॒म्नि॒या॒ । ए॒व । ए॒न॒म् । मि॒मी॒ते॒ । तस्मा᳚त् । नाना॑वीर्या॒ इति॒ नाना᳚ - वी॒र्याः॒ । अ॒ङ्गुल॑यः । सर्वा॑सु । अ॒ङ्गु॒ष्ठम् । उप॑ । नीति॑ । गृ॒ह्णा॒ति॒ । तस्मा᳚त् । स॒माव॑द्वीर्य॒ इति॑ स॒माव॑त् - वी॒र्यः॒ । अ॒न्याभिः॑ । अ॒ङ्गुलि॑भि॒रित्य॒ङ्गुलि॑ - भिः॒ । तस्मा᳚त् । सर्वाः᳚ । अनु॑ । समिति॑ । च॒र॒ति॒ । यत् । स॒ह । सर्वा॑भिः । मिमी॑त । सꣳश्लि॑ष्टा॒ इति॒ सं - श्लि॒ष्टाः॒ । अ॒ङ्गुल॑यः । जा॒ये॒र॒न्न् । एक॑यैक॒येत्येक॑या-ए॒क॒या॒ । उ॒थ्सर्ग॒मित्यु॑त् - सर्ग᳚म् । मि॒मी॒ते॒ । तस्मा᳚त् । विभ॑क्ता॒ इति॒ वि - भ॒क्ताः॒ । जा॒य॒न्ते॒ । पञ्च॑ । कृत्वः॑ । यजु॑षा । मि॒मी॒ते॒ । पञ्चा᳚क्ष॒रेति॒ पञ्च॑ - अ॒क्ष॒रा॒ । प॒ङ्क्तिः । पाङ्क्तः॑ । य॒ज्ञ्ः । य॒ज्ञ्म् । ए॒व । अवेति॑ । रु॒न्धे॒ । पञ्च॑ । कृत्वः॑ । तू॒ष्णीम् । \textbf{  63} \newline
                  \newline
                                \textbf{ TS 6.1.9.6} \newline
                  दश॑ । समिति॑ । प॒द्य॒न्ते॒ । दशा᳚क्ष॒रेति॒ दश॑ - अ॒क्ष॒रा॒ । वि॒राडिति॑ वि - राट् । अन्न᳚म् । वि॒राडिति॑ वि - राट् । वि॒राजेति॑ वि - राजा᳚ । ए॒व । अ॒न्नाद्य॒मित्य॑न्न -अद्य᳚म् । अवेति॑ । रु॒न्धे॒ । यत् । यजु॑षा । मिमी॑ते । भू॒तम् । ए॒व । अवेति॑ । रु॒न्धे॒ । यत् । तू॒ष्णीम् । भ॒वि॒ष्यत् । यत् । वै । तावान्॑ । ए॒व । सोमः॑ । स्यात् । याव॑न्तम् । मिमी॑ते । यज॑मानस्य । ए॒व । स्या॒त् । न । अपीति॑ । स॒द॒स्या॑नाम् । प्र॒जाभ्य॒ इति॑ प्र - जाभ्यः॑ । त्वा॒ । इति॑ । उप॑ । समिति॑ । ऊ॒ह॒ति॒ । स॒द॒स्यान्॑ । ए॒व । अ॒न्वाभ॑ज॒तीत्य॑नु-आभ॑जति । वास॑सा । उपेति॑ । न॒ह्य॒ति॒ । स॒र्व॒दे॒व॒त्य॑मिति॑ सर्व - दे॒व॒त्य᳚म् । वै । \textbf{  64} \newline
                  \newline
                                \textbf{ TS 6.1.9.7} \newline
                  वासः॑ । सर्वा॑भिः । ए॒व । ए॒न॒म् । दे॒वता॑भिः । समिति॑ । अ॒द्‌र्ध॒य॒ति॒ । प॒शवः॑ । वै । सोमः॑ । प्रा॒णायेति॑ प्र - अ॒नाय॑ । त्वा॒ । इति॑ । उपेति॑ । न॒ह्य॒ति॒ । प्रा॒णमिति॑ प्र - अ॒नम् । ए॒व । प॒शुषु॑ । द॒धा॒ति॒ । व्या॒नायेति॑ वि - अ॒नाय॑ । त्वा॒ । इति॑ । अन्विति॑ । शृ॒न्थ॒ति॒ । व्या॒नमिति॑ वि - अ॒नम् । ए॒व । प॒शुषु॑ । द॒धा॒ति॒ । तस्मा᳚त् । स्व॒पन्त᳚म् । प्रा॒णा इति॑ प्र - अ॒नाः । न । ज॒ह॒ति॒ ॥ \textbf{  65} \newline
                  \newline
                      (इत॑रं - पशु॒मान्थ् स्या᳚द् - या॒त्यन॑सो॒ - थ्सर्गं॑ - तू॒ष्णीꣳ - स॑र्वदेव॒त्यं॑ ॅवै - त्रय॑स्त्रिꣳशच्च)  \textbf{(A9)} \newline \newline
                                \textbf{ TS 6.1.10.1} \newline
                  यत् । क॒लया᳚ । ते॒ । श॒फेन॑ । ते॒ । क्री॒णा॒नि॒ । इति॑ । पणे॑त । अगो॑अर्घ॒मित्यगो᳚ - अ॒र्घ॒म् । सोम᳚म् । कु॒र्यात् । अगो॑अर्घ॒मित्यगो᳚ - अ॒र्घ॒म् । यज॑मानम् । अगो॑अर्घ॒मित्यगो᳚-अ॒र्घ॒म् । अ॒द्ध्व॒र्युम् । गोः । तु । म॒हि॒मान᳚म् । न । अवेति॑ । ति॒रे॒त् । गवा᳚ । ते॒ । क्री॒णा॒नि॒ । इति॑ । ए॒व । ब्रू॒या॒त् । गो॒अ॒र्घमिति॑ गो - अ॒र्घम् । ए॒व । सोम᳚म् । क॒रोति॑ । गो॒अ॒र्घमिति॑ गो - अ॒र्घम् । यज॑मानम् । गो॒अ॒र्घमिति॑ गो - अ॒र्घम् । अ॒द्ध्व॒र्युम् । न । गोः । म॒हि॒मान᳚म् । अवेति॑ । ति॒र॒ति॒ । अ॒जया᳚ । क्री॒णा॒ति॒ । सत॑पस॒मिति॒ स - त॒प॒स॒म् । ए॒व । ए॒न॒म् । क्री॒णा॒ति॒ । हिर॑ण्येन । क्री॒णा॒ति॒ । सशु॑क्र॒मिति॒ स - शु॒क्र॒म् । ए॒व । \textbf{  66} \newline
                  \newline
                                \textbf{ TS 6.1.10.2} \newline
                  ए॒न॒म् । क्री॒णा॒ति॒ । धे॒न्वा । क्री॒णा॒ति॒ । साशि॑र॒मिति॒ स - आ॒शि॒र॒म् । ए॒व । ए॒न॒म् । क्री॒णा॒ति॒ । ऋ॒ष॒भेण॑ । क्री॒णा॒ति॒ । सेन्द्र॒मिति॒ स-इ॒न्द्र॒म् । ए॒व । ए॒न॒म् । क्री॒णा॒ति॒ । अ॒न॒डुहा᳚ । क्री॒णा॒ति॒ । वह्निः॑ । वै । अ॒न॒ड्वान् । वह्नि॑ना । ए॒व । वह्नि॑ । य॒ज्ञ्स्य॑ । क्री॒णा॒ति॒ । मि॒थु॒नाभ्या᳚म् । क्री॒णा॒ति॒ । मि॒थु॒नस्य॑ । अव॑रुद्ध्या॒ इत्यव॑ - रु॒द्ध्यै॒ । वास॑सा । क्री॒णा॒ति॒ । स॒र्व॒दे॒व॒त्य॑मिति॑ सर्व - दे॒व॒त्य᳚म् । वै । वासः॑ । सर्वा᳚भ्यः । ए॒व । ए॒न॒म् । दे॒वता᳚भ्यः । क्री॒णा॒ति॒ । दश॑ । समिति॑ । प॒द्य॒न्ते॒ । दशा᳚क्ष॒रेति॒ दश॑ - अ॒क्ष॒रा॒ । वि॒राडिति॑ वि - राट् । अन्न᳚म् । वि॒राडिति॑ वि - राट् । वि॒राजेति॑ वि - राजा᳚ । ए॒व । अ॒न्नाद्य॒मित्य॑न्न -अद्य᳚म् । अवेति॑ । रु॒न्धे॒ । \textbf{  67} \newline
                  \newline
                                \textbf{ TS 6.1.10.3} \newline
                  तपसः॑ । त॒नूः । अ॒सि॒ । प्र॒जाप॑ते॒रिति॑ प्र॒जा - प॒तेः॒ । वर्णः॑ । इति॑ । आ॒ह॒ । प॒शुभ्य॒ इति॑ प॒शु - भ्यः॒ । ए॒व । तत् । अ॒द्ध्व॒र्युः । नीति॑ । ह्नु॒ते॒ । आ॒त्मनः॑ । अना᳚व्रस्का॒येत्यना᳚ - व्र॒स्का॒य॒ । गच्छ॑ति । श्रिय᳚म् । प्रेति॑ । प॒शून् । आ॒प्नो॒ति॒ । यः । ए॒वम् । वेद॑ । शु॒क्रम् । ते॒ । शु॒क्रेण॑ । क्री॒णा॒मि॒ । इति॑ । आ॒ह॒ । य॒था॒य॒जुरिति॑ यथा-यजुः । ए॒व । ए॒तत् । दे॒वाः । वै । येन॑ । हिर॑ण्येन । सोम᳚म् । अक्री॑णन्न् । तत् । अ॒भी॒षहेत्य॑भि-सहा᳚ । पुनः॑ । एति॑ । अ॒द॒द॒त॒ । कः । हि । तेज॑सा । वि॒क्रे॒ष्यत॒ इति॑ वि - क्रे॒ष्यते᳚ । इति॑ । येन॑ । हिर॑ण्येन । \textbf{  68} \newline
                  \newline
                                \textbf{ TS 6.1.10.4} \newline
                  सोम᳚म् । क्री॒णी॒यात् । तत् । अ॒भी॒षहेत्य॑भ - सहा᳚ । पुनः॑ । एति॑ । द॒दी॒त॒ । तेजः॑ । ए॒व । आ॒त्मन्न् । ध॒त्ते॒ । अ॒स्मे इति॑ । ज्योतिः॑ । सो॒म॒वि॒क्र॒यिणीति॑ सोम-वि॒क्र॒यिणि॑ । तमः॑ । इति॑ । आ॒ह॒ । ज्योतिः॑ । ए॒व । यज॑माने । द॒धा॒ति॒ । तम॑सा । सो॒म॒वि॒क्र॒यिण॒मिति॑ सोम - वि॒क्र॒यिण᳚म् । अ॒र्प॒य॒ति॒ । यत् । अनु॑पग्र॒थ्येत्यनु॑प - ग्र॒थ्य॒ । ह॒न्यात् । द॒न्द॒शूकाः᳚ । ताम् । समा᳚म् । स॒र्पाः । स्युः॒ । इ॒दम् । अ॒हम् । स॒र्पाणा᳚म् । द॒न्द॒शूका॑नाम् । ग्री॒वाः । उपेति॑ । ग्र॒थ्ना॒मि॒ । इति॑ । आ॒ह॒ । अद॑न्दशूकाः । ताम् । समा᳚म् । स॒र्पाः । भ॒व॒न्ति॒ । तम॑सा । सो॒म॒वि॒क्र॒यिण॒मिति॑ सोम - वि॒क्र॒यिण᳚म् । वि॒द्ध्य॒ति॒ । स्वान॑ । \textbf{  69} \newline
                  \newline
                                \textbf{ TS 6.1.10.5} \newline
                  भ्राज॑ । इति॑ । आ॒ह॒ । ए॒ते । वै । अ॒मुष्मिन्न्॑ । लो॒के । सोम᳚म् । अ॒र॒क्ष॒न्न् । तेभ्यः॑ । अधीति॑ । सोम᳚म् । एति॑ । अ॒ह॒र॒न्न् । यत् । ए॒तेभ्यः॑ । सो॒म॒क्रय॑णा॒निति॑ सोम - क्रय॑णान् । न । अ॒नु॒दि॒शेदित्य॑नु - दि॒शेत् । अक्री॑तः । अ॒स्य॒ । सोमः॑ । स्या॒त् । न । अ॒स्य॒ । ए॒ते । अ॒मुष्मिन्न्॑ । लो॒के । सोम᳚म् । र॒क्षे॒युः॒ । यत् । ए॒तेभ्यः॑ । सो॒म॒क्रय॑णा॒निति॑ सोम - क्रय॑णान् । अ॒नु॒दि॒शतीत्य॑नु - दि॒शति॑ । क्री॒तः । अ॒स्य॒ । सोमः॑ । भ॒व॒ति॒ । ए॒ते । अ॒स्य॒ । अ॒मुष्मिन्न्॑ । लो॒के । सोम᳚म् । र॒क्ष॒न्ति॒ ॥ \textbf{  70} \newline
                  \newline
                      (सशु॑क्रमे॒व - रु॑न्ध॒ - इति॒ येन॒ हिर॑ण्येन॒ - स्वान॒ - चतु॑श्चत्वारिꣳशच्च)  \textbf{(A10)} \newline \newline
                                \textbf{ TS 6.1.11.1} \newline
                  वा॒रु॒णः । वै । क्री॒तः । सोमः॑ । उप॑नद्ध॒ इत्युप॑-न॒द्धः॒ । मि॒त्रः । नः॒ । एति॑ । इ॒हि॒ । सुमि॑त्रधा॒ इति॒ सुमि॑त्र - धाः॒ । इति॑ । आ॒ह॒ । शान्त्यै᳚ । इन्द्र॑स्य । ऊ॒रुम् । एति॑ । वि॒श॒ । दक्षि॑णम् । इति॑ । आ॒ह॒ । दे॒वाः । वै । यम् । सोम᳚म् । अक्री॑णन्न् । तम् । इन्द्र॑स्य । ऊ॒रौ । दक्षि॑णे । एति॑ । अ॒सा॒द॒य॒न्न् । ए॒षः । खलु॑ । वै । ए॒तर्.हि॑ । इन्द्रः॑ । यः । यज॑ते । तस्मा᳚त् । ए॒वम् । आ॒ह॒ । उदिति॑ । आयु॑षा । स्वा॒युषेति॑ सु - आ॒युषा᳚ । इति॑ । आ॒ह॒ । दे॒वताः᳚ । ए॒व । अ॒न्वा॒रभ्येत्य॑नु - आ॒रभ्य॑ । उदिति॑ । \textbf{  71} \newline
                  \newline
                                \textbf{ TS 6.1.11.2} \newline
                  ति॒ष्ठ॒ति॒ । उ॒रु । अ॒न्तरि॑क्षम् । अन्विति॑ । इ॒हि॒ । इति॑ । आ॒ह॒ । अ॒न्त॒रि॒क्ष॒दे॒व॒त्य॑ इत्य॑न्तरिक्ष - दे॒व॒त्यः॑ । हि । ए॒तर्.हि॑ । सोमः॑ । अदि॑त्याः । सदः॑ । अ॒सि॒ । अदि॑त्याः । सदः॑ । एति॑ । सी॒द॒ । इति॑ । आ॒ह॒ । य॒था॒य॒जुरिति॑ यथा-य॒जुः । ए॒व । ए॒तत् । वीति॑ । वै । ए॒न॒म् । ए॒तत् । अ॒द्‌र्ध॒य॒ति॒ । यत् । वा॒रु॒णम् । सन्त᳚म् । मै॒त्रम् । क॒रोति॑ । वा॒रु॒ण्या । ऋ॒चा । एति॑ । सा॒द॒य॒ति॒ । स्वया᳚ । ए॒व । ए॒न॒म् । दे॒वत॑या । समिति॑ । अ॒द्‌र्ध॒य॒ति॒ । वास॑सा । प॒र्यान॑ह्य॒तीति॑ परि - आन॑ह्यति । स॒र्व॒दे॒व॒त्य॑मिति॑ सर्व -दे॒व॒त्य᳚म् । वै । वासः॑ । सर्वा॑भिः । ए॒व । \textbf{  72} \newline
                  \newline
                                \textbf{ TS 6.1.11.3} \newline
                  ए॒न॒म् । दे॒वता॑भिः । समिति॑ । अ॒द्‌र्ध॒य॒ति॒ । अथो॒ इति॑ । रक्ष॑साम् । अप॑हत्या॒ इत्यप॑-ह॒त्यै॒ । वने॑षु । वीति॑ । अ॒न्तरि॑क्षम् । त॒ता॒न॒ । इति॑ । आ॒ह॒ । वने॑षु । हि । वीति॑ । अ॒न्तरि॑क्षम् । त॒तान॑ । वाज᳚म् । अर्व॒थ्स्वित्यर्व॑त् - सु॒ । इति॑ । आ॒ह॒ । वाज᳚म् । हि । अर्व॒थ्स्वित्यर्व॑त् - सु॒ । पयः॑ । अ॒घ्नि॒यासु॑ । इति॑ । आ॒ह॒ । पयः॑ । हि । अ॒घ्नि॒यासु॑ । हृ॒थ्स्विति॑ हृत् - सु । क्रतु᳚म् । इति॑ । आ॒ह॒ । हृ॒थ्स्विति॑ हृत्-सु । हि । क्रतु᳚म् । वरु॑णः । वि॒क्षु । अ॒ग्निम् । इति॑ । आ॒ह॒ । वरु॑णः । हि । वि॒क्षु । अ॒ग्निम् । दि॒वि । सूर्य᳚म् । \textbf{  73} \newline
                  \newline
                                \textbf{ TS 6.1.11.4} \newline
                  इति॑ । आ॒ह॒ । दि॒वि । हि । सूर्य᳚म् । सोम᳚म् । अद्रौ᳚ । इति॑ । आ॒ह॒ । ग्रावा॑णः । वै । अद्र॑यः । तेषु॑ । वै । ए॒षः । सोम᳚म् । द॒धा॒ति॒ । यः । यज॑ते । तस्मा᳚त् । ए॒वम् । आ॒ह॒ । उदिति॑ । उ॒ । त्यम् । जा॒तवे॑दस॒मिति॑ जा॒त - वे॒द॒स॒म् । इति॑ । सौ॒र्या । ऋ॒चा । कृ॒ष्णा॒जि॒नमिति॑ कृष्ण - अ॒जि॒नम् । प्र॒त्यान॑ह्य॒तीति॑ प्रति-आन॑ह्यति । रक्ष॑साम् । अप॑हत्या॒ इत्यप॑ - ह॒त्यै॒ । उस्रौ᳚ । एति॑ । इ॒त॒म् । धू॒र्॒.षा॒हा॒विति॑ धूः - सा॒हौ॒ । इति॑ । आ॒ह॒ । य॒था॒य॒जुरिति॑ यथा - य॒जुः । ए॒व । ए॒तत् । प्रेति॑ । च्य॒व॒स्व॒ । भु॒वः॒ । प॒ते॒ । इति॑ । आ॒ह॒ । भू॒ताना᳚म् । हि । \textbf{  74} \newline
                  \newline
                                \textbf{ TS 6.1.11.5} \newline
                  ए॒षः । पतिः॑ । विश्वा॑नि । अ॒भीति॑ । धामा॑नि । इति॑ । आ॒ह॒ । विश्वा॑नि । हि । ए॒षः । अ॒भीति॑ । धामा॑नि । प्र॒च्यव॑त॒ इति॑ प्र - च्यव॑ते । मा । त्वा॒ । प॒रि॒प॒रीति॑ परि - प॒री । वि॒द॒त् । इति॑ । आ॒ह॒ । यत् । ए॒व । अ॒दः । सोम᳚म् । आ॒ह्रि॒यमा॑ण॒मित्या᳚ - ह्रि॒यमा॑णम् । ग॒न्ध॒र्वः । वि॒श्वाव॑सु॒रिति॑ वि॒श्व - व॒सुः॒ । प॒र्यमु॑ष्णा॒दिति॑ परि - अमु॑ष्णात् । तस्मा᳚त् । ए॒वम् । आ॒ह॒ । अप॑रिमोषा॒येत्यप॑रि - मो॒षा॒य॒ । यज॑मानस्य । स्व॒स्त्यय॒नीति॑ स्वस्ति - अय॑नी । अ॒सि॒ । इति॑ । आ॒ह॒ । यज॑मानस्य । ए॒व । ए॒षः । य॒ज्ञ्स्य॑ । अ॒न्वा॒र॒म्भ इत्य॑नु - आ॒र॒म्भः । अन॑वच्छित्त्या॒ इत्यन॑व - छि॒त्यै॒ । वरु॑णः । वै । ए॒षः । यज॑मानम् । अ॒भि । एति॑ । ए॒ति॒ । यत् । \textbf{  75} \newline
                  \newline
                                \textbf{ TS 6.1.11.6} \newline
                  क्री॒तः । सोमः॑ । उप॑नद्ध॒ इत्युप॑ - न॒द्धः॒ । नमः॑ । मि॒त्रस्य॑ । वरु॑णस्य । चक्ष॑से । इति॑ । आ॒ह॒ । शान्त्यै᳚ । एति॑ । सोम᳚म् । वह॑न्ति । अ॒ग्निना᳚ । प्रतीति॑ । ति॒ष्ठ॒ते॒ । तौ । स॒भंव॑न्ता॒विति॑ सं - भव॑न्तौ । यज॑मानम् । अ॒भि । समिति॑ । भ॒व॒तः॒ । पु॒रा । खलु॑ । वाव । ए॒षः । मेधा॑य । आ॒त्मान᳚म् । आ॒रभ्येत्या᳚-रभ्य॑ । च॒र॒ति॒ । यः । दी॒क्षि॒तः । यत् । अ॒ग्नी॒षो॒मीय॒मित्य॑ग्नी - सो॒मीय᳚म् । प॒शुम् । आ॒लभ॑त॒ इत्या᳚ - लभ॑ते । आ॒त्म॒नि॒ष्क्रय॑ण॒ इत्या᳚त्म - नि॒ष्क्रय॑णः । ए॒व । अ॒स्य॒ । सः । तस्मा᳚त् । तस्य॑ । न । आ॒श्य᳚म् । पु॒रु॒ष॒नि॒ष्क्रय॑ण॒ इति॑ पुरुष - नि॒ष्क्रय॑णः । इ॒व॒ । हि । अथो॒ इति॑ । खलु॑ । आ॒हुः॒ ( ) । अ॒ग्नीषोमा᳚भ्या॒मित्य॒ग्नी - सोमा᳚भ्याम् । वै । इन्द्रः॑ । वृ॒त्रम् । अ॒ह॒न्न् । इति॑ । यत् । अ॒ग्नी॒षो॒मीय॒मित्य॑ग्नी - सो॒मीय᳚म् । प॒शुम् । आ॒लभ॑त॒ इत्या᳚-लभ॑ते । वार्त्र॑घ्न॒ इति॒ वार्त्र॑ - घ्नः॒ । ए॒व । अ॒स्य॒ । सः । तस्मा᳚त् । उ॒ । आ॒श्य᳚म् । वा॒रु॒ण्या । ऋ॒चा । परीति॑ । च॒र॒ति॒ । स्वया᳚ । ए॒व । ए॒न॒म् । दे॒वत॑या । परीति॑ । च॒र॒ति॒ ॥ \textbf{  76} \newline
                  \newline
                      (अ॒न्वा॒रभ्योथ् - सर्वा॑भिरे॒व - सूर्यं॑ - भू॒तानाꣳ॒॒ ह्ये॑ - ति॒ य - दा॑हुः - स॒प्तविꣳ॑शतिश्च)  \textbf{(A11)} \newline \newline
\textbf{praSna korvai with starting padams of 1 to 11 anuvAkams :-} \newline
(प्रा॒चीन॑वꣳशं॒ -ॅयाव॑न्त - ऋख्सा॒मे - वाग्वै दे॒वेभ्यो॑ - दे॒वा वै दे॑व॒यज॑नं - क॒द्रूश्च॒ - तद्धिर॑ण्यꣳ॒॒ - षट् प॒दानि॑ - ब्रह्मवा॒दिनो॑ वि॒चित्यो॒ - यत् क॒लया॑ ते - वारु॒णो वै क्री॒तः सोम॒ - एका॑दश) \newline

\textbf{korvai with starting padams of1, 11, 21 series of pa~jcAtis :-} \newline
(प्रा॒चीन॑वꣳशꣳ॒॒ - स्वाहेत्या॑ह॒ - ये᳚ऽन्तः श॒रा - ह्ये॑ष सं - तप॑सा च॒ - यत्क॑र्णगृही॒ - तेति॑ लोम॒तो - वा॑रु॒णः - षट्थ् स॑प्ततिः ) \newline

\textbf{first and last padam of first praSnam Of 6th KANDam} \newline
(प्रा॒चीन॑वꣳ शं॒ - परि॑ चरति) \newline 


॥ हरिः॑ ॐ ॥
॥ कृष्ण यजुर्वेदीय तैत्तिरीय संहितायां षष्ठकाण्डे प्रथमः प्रश्नः समाप्तः ॥
------------------------------------ \newline
\pagebreak
\pagebreak
        


\end{document}
