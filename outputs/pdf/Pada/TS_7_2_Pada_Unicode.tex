\documentclass[17pt]{extarticle}
\usepackage{babel}
\usepackage{fontspec}
\usepackage{polyglossia}
\usepackage{extsizes}



\setmainlanguage{sanskrit}
\setotherlanguages{english} %% or other languages
\setlength{\parindent}{0pt}
\pagestyle{myheadings}
\newfontfamily\devanagarifont[Script=Devanagari]{AdishilaVedic}


\newcommand{\VAR}[1]{}
\newcommand{\BLOCK}[1]{}




\begin{document}
\begin{titlepage}
    \begin{center}
 
\begin{sanskrit}
    { \Large
    ॐ नमः परमात्मने, श्री महागणपतये नमः, श्री गुरुभ्यो नमः
ह॒रिः॒ ॐ 
    }
    \\
    \vspace{2.5cm}
    \mbox{ \Huge
    7.2      सप्तमकाण्डे द्वितीयः प्रश्नः - षड् रात्राद्यानां निरूपणं   }
\end{sanskrit}
\end{center}

\end{titlepage}
\tableofcontents

ॐ नमः परमात्मने, श्री महागणपतये नमः, श्री गुरुभ्यो नमः
ह॒रिः॒ ॐ \newline
7.2      सप्तमकाण्डे द्वितीयः प्रश्नः - षड् रात्राद्यानां निरूपणं \newline

\addcontentsline{toc}{section}{ 7.2      सप्तमकाण्डे द्वितीयः प्रश्नः - षड् रात्राद्यानां निरूपणं}
\markright{ 7.2      सप्तमकाण्डे द्वितीयः प्रश्नः - षड् रात्राद्यानां निरूपणं \hfill https://www.vedavms.in \hfill}
\section*{ 7.2      सप्तमकाण्डे द्वितीयः प्रश्नः - षड् रात्राद्यानां निरूपणं }
                                \textbf{ TS 7.2.1.1} \newline
                  सा॒द्ध्याः । वै । दे॒वाः । सु॒व॒र्गका॑मा॒ इति॑ सुव॒र्ग - का॒माः॒ । ए॒तम् । ष॒ड्रा॒त्रमिति॑ षट् - रा॒त्रम् । अ॒प॒श्य॒न्न् । तम् । एति॑ । अ॒ह॒र॒न्न् । तेन॑ । अ॒य॒ज॒न्त॒ । ततः॑ । वै । ते । सु॒व॒र्गमिति॑ सुवः-गम् । लो॒कम् । आ॒य॒न्न् । ये । ए॒वम् । वि॒द्वाꣳसः॑ । ष॒ड्रा॒त्रमिति॑ षट् - रा॒त्रम् । आस॑ते । सु॒व॒र्गमिति॑ सुवः - गम् । ए॒व । लो॒कम् । य॒न्ति॒ । दे॒व॒स॒त्रमिति॑ देव - स॒त्रम् । वै । ष॒ड्रा॒त्र इति॑ षट् - रा॒त्रः । प्र॒त्यक्ष॒मिति॑ प्रति - अक्ष᳚म् । हि । ए॒तानि॑ । पृ॒ष्ठानि॑ । ये । ए॒वम् । वि॒द्वाꣳसः॑ । ष॒ड्रा॒त्रमिति॑ षट् - रा॒त्रम् । आस॑ते । सा॒क्षादिति॑ स - अ॒क्षात् । ए॒व । दे॒वताः᳚ । अ॒भ्यारो॑ह॒न्तीत्य॑भि - आरो॑हन्ति । ष॒ड्रा॒त्र इति॑ षट् - रा॒त्रः । भ॒व॒ति॒ । षट् । वै । ऋ॒तवः॑ । षट् । पृ॒ष्ठानि॑ । \textbf{  1} \newline
                  \newline
                                \textbf{ TS 7.2.1.2} \newline
                  पृ॒ष्ठैः । ए॒व । ऋ॒तून् । अ॒न्वारो॑ह॒न्तीत्य॑नु - आरो॑हन्ति । ऋ॒तुभि॒रित्यृ॒तु -भिः॒ । सं॒ॅव॒थ्स॒रमिति॑ सं - व॒थ्स॒रम् । ते । सं॒ॅव॒थ्स॒र इति॑ सं - व॒थ्स॒रे । ए॒व । प्रतीति॑ । ति॒ष्ठ॒न्ति॒ । बृ॒ह॒द्र॒थ॒न्त॒राभ्या॒मिति॑ बृहत् - र॒थ॒न्त॒राभ्या᳚म् । य॒न्ति॒ । इ॒यम् । वाव । र॒थ॒न्त॒रमिति॑ रथं - त॒रम् । अ॒सौ । बृ॒हत् । आ॒भ्याम् । ए॒व । य॒न्ति॒ । अथो॒ इति॑ । अ॒नयोः᳚ । ए॒व । प्रतीति॑ । ति॒ष्ठ॒न्ति॒ । ए॒ते इति॑ । वै । य॒ज्ञ्स्य॑ । अ॒ञ्ज॒साय॑नी॒ इत्य॑ञ्जसा - अय॑नी । स्रु॒ती इति॑ । ताभ्या᳚म् । ए॒व । सु॒व॒र्गमिति॑ सुवः - गम् । लो॒कम् । य॒न्ति॒ । त्रि॒वृदिति॑ त्रि - वृत् । अ॒ग्नि॒ष्टो॒म इत्य॑ग्नि - स्तो॒मः । भ॒व॒ति॒ । तेजः॑ । ए॒व । अवेति॑ । रु॒न्ध॒ते॒ । प॒ञ्च॒द॒श इति॑ पञ्च - द॒शः । भ॒व॒ति॒ । इ॒न्द्रि॒यम् । ए॒व । अवेति॑ । रु॒न्ध॒ते॒ । स॒प्त॒द॒श इति॑ सप्त - द॒शः । \textbf{  2} \newline
                  \newline
                                \textbf{ TS 7.2.1.3} \newline
                  भ॒व॒ति॒ । अ॒न्नाद्य॒स्येत्य॑न्न - अद्य॑स्य । अव॑रुद्ध्या॒ इत्यव॑ - रु॒द्ध्यै॒ । अथो॒ इति॑ । प्रेति॑ । ए॒व । तेन॑ । जा॒य॒न्ते॒ । ए॒क॒विꣳ॒॒श इत्ये॑क - विꣳ॒॒शः । भ॒व॒ति॒ । प्रति॑ष्ठित्या॒ इति॒ प्रति॑ - स्थि॒त्यै॒ । अथो॒ इति॑ । रुच᳚म् । ए॒व । आ॒त्मन्न् । द॒ध॒ते॒ । त्रि॒ण॒व इति॑ त्रि - न॒वः । भ॒व॒ति॒ । विजि॑त्या॒ इति॒ वि - जि॒त्यै॒ । त्र॒य॒स्त्रिꣳ॒॒श इति॑ त्रयः - त्रिꣳ॒॒शः । भ॒व॒ति॒ । प्रति॑ष्ठित्या॒ इति॒ प्रति॑ - स्थि॒त्यै॒ । स॒दो॒ह॒वि॒द्‌र्धा॒निन॒ इति॑ सदः - ह॒वि॒द्‌र्धा॒निनः॑ । ए॒तेन॑ । ष॒ड्रा॒त्रेणेति॑ षट् - रा॒त्रेण॑ । य॒जे॒र॒न्न् । आश्व॑त्थी॒ इति॑ । ह॒वि॒द्‌र्धान॒मिति॑ हविः - धान᳚म् । च॒ । आग्नी᳚द्ध्र॒मित्याग्नि॑-इ॒द्ध्र॒म् । च॒ । भ॒व॒तः॒ । तत् । हि । सु॒व॒र्ग्य॑मिति॑ सुवः - ग्य᳚म् । च॒क्रीव॑ती॒ इति॑ । भ॒व॒तः॒ । सु॒व॒र्गस्येति॑ सुवः - गस्य॑ । लो॒कस्य॑ । सम॑ष्ट्या॒ इति॒ सं-अ॒ष्ट्यै॒ । उ॒लूख॑लबुद्ध्न॒ इत्यु॒लूख॑ल - बु॒द्ध्नः॒ । यूपः॑ । भ॒व॒ति॒ । प्रति॑ष्ठित्या॒ इति॒ प्रति॑ - स्थि॒त्यै॒ । प्राञ्चः॑ । या॒न्ति॒ । प्राङ् । इ॒व॒ । हि । सु॒व॒र्ग इति॑ सुवः - गः । \textbf{  3} \newline
                  \newline
                                \textbf{ TS 7.2.1.4} \newline
                  लो॒कः । सर॑स्वत्या । या॒न्ति॒ । ए॒षः । वै । दे॒व॒यान॒ इति॑ देव-यानः॑ । पन्थाः᳚ । तम् । ए॒व । अ॒न्वारो॑ह॒न्तीत्य॑नु - आरो॑हन्ति । आ॒क्रोश॑न्त॒ इत्या᳚ - क्रोश॑न्तः । या॒न्ति॒ । अव॑र्तिम् । ए॒व । अ॒न्यस्मिन्न्॑ । प्र॒ति॒षज्येति॑ प्रति - सज्य॑ । प्र॒ति॒ष्ठामिति॑ प्रति-स्थाम् । ग॒च्छ॒न्ति॒ । य॒दा । दश॑ । श॒तम् । कु॒र्वन्ति॑ । अथ॑ । एक᳚म् । उ॒त्थान॒मित्यु॑त्-स्थान᳚म् । श॒तायु॒रिति॑ श॒त - आ॒युः॒ । पुरु॑षः । श॒तेन्द्रि॑य॒ इति॑ श॒त-इ॒न्द्रि॒यः॒ । आयु॑षि । ए॒व । इ॒न्द्रि॒ये । प्रतीति॑ । ति॒ष्ठ॒न्ति॒ । य॒दा । श॒तम् । स॒हस्र᳚म् । कु॒र्वन्ति॑ । अथ॑ । एक᳚म् । उ॒त्थान॒मित्यु॑त्- स्थान᳚म् । स॒हस्र॑संमित॒ इति॑ स॒हस्र॑-स॒म्मि॒तः॒ । वै । अ॒सौ । लो॒कः । अ॒मुम् । ए॒व । लो॒कम् । अ॒भीति॑ । ज॒य॒न्ति॒ । य॒दा ( ) । ए॒षा॒म् । प्र॒मीये॒तेति॑ प्र - मीये॑त । य॒दा । वा॒ । जीये॑रन्न् । अथ॑ । एक᳚म् । उ॒त्थान॒मित्यु॑त् - स्थान᳚म् । तत् । हि । ती॒र्थम् ॥ \textbf{  4 } \newline
                  \newline
                      (पृ॒ष्ठानि॑-सप्तद॒शः-सु॑व॒र्गो-ज॑यन्ति य॒दै - का॑दश च)  \textbf{(A1)} \newline \newline
                                \textbf{ TS 7.2.2.1} \newline
                  कु॒सु॒रु॒बिन्दः॑ । औद्दा॑लकि॒रित्यौत् - दा॒ल॒किः॒ । अ॒का॒म॒य॒त॒ । प॒शु॒मानिति॑ पशु - मान् । स्या॒म् । इति॑ । सः । ए॒तम् । स॒प्त॒रा॒त्रमिति॑ सप्त-रा॒त्रम् । एति॑ । अ॒ह॒र॒त् । तेन॑ । अ॒य॒ज॒त॒ । तेन॑ । वै । सः । याव॑न्तः । ग्रा॒म्याः । प॒शवः॑ । तान् । अवेति॑ । अ॒रु॒न्ध॒ । यः । ए॒वम् । वि॒द्वान् । स॒प्त॒रा॒त्रेणेति॑ सप्त - रा॒त्रेण॑ । यज॑ते । याव॑न्तः । ए॒व । ग्रा॒म्याः । प॒शवः॑ । तान् । ए॒व । अवेति॑ । रु॒न्धे॒ । स॒प्त॒रा॒त्र इति॑ सप्त-रा॒त्रः । भ॒व॒ति॒ । स॒प्त । ग्रा॒म्याः । प॒शवः॑ । स॒प्त । आ॒र॒ण्याः । स॒प्त । छन्दाꣳ॑सि । उ॒भय॑स्य । अव॑रुद्ध्या॒ इत्यव॑-रु॒द्ध्यै॒ । त्रि॒वृदिति॑ त्रि - वृत् । अ॒ग्नि॒ष्टो॒म इत्य॑ग्नि-स्तो॒मः । भ॒व॒ति॒ । तेजः॑ । \textbf{  5} \newline
                  \newline
                                \textbf{ TS 7.2.2.2} \newline
                  ए॒व । अवेति॑ । रु॒न्धे॒ । प॒ञ्च॒द॒श इति॑ पञ्च - द॒शः । भ॒व॒ति॒ । इ॒न्द्रि॒यम् । ए॒व । अवेति॑ । रु॒न्धे॒ । स॒प्त॒द॒श इति॑ सप्त - द॒शः । भ॒व॒ति॒ । अ॒न्नाद्य॒स्येत्य॑न्न - अद्य॑स्य । अव॑रुद्ध्या॒ इत्यव॑-रु॒द्ध्यै॒ । अथो॒ इति॑ । प्रेति॑ । ए॒व । तेन॑ । जा॒य॒ते॒ । ए॒क॒विꣳ॒॒श इत्ये॑क - विꣳ॒॒शः । भ॒व॒ति॒ । प्रति॑ष्ठित्या॒ इति॒ प्रति॑ - स्थि॒त्यै॒ । अथो॒ इति॑ । रुच᳚म् । ए॒व । आ॒त्मन्न् । ध॒त्ते॒ । त्रि॒ण॒व इति॑ त्रि-न॒वः । भ॒व॒ति॒ । विजि॑त्या॒ इति॒ वि - जि॒त्यै॒ । प॒ञ्च॒विꣳ॒॒श इति॑ पञ्च - विꣳ॒॒शः । अ॒ग्नि॒ष्टो॒म इत्य॑ग्नि - स्तो॒मः । भ॒व॒ति॒ । प्र॒जाप॑ते॒रिति॑ प्र॒जा-प॒तेः॒ । आप्त्यै᳚ । म॒हा॒व्र॒तवा॒निति॑ महाव्र॒त-वा॒न् । अ॒न्नाद्य॒स्येत्य॑न्न-अद्य॑स्य । अव॑रुद्ध्या॒ इत्यव॑ - रु॒द्ध्यै॒ । वि॒श्व॒जिदिति॑ विश्व - जित् । सर्व॑पृष्ठ॒ इति॒ सर्व॑ - पृ॒ष्ठः॒ । अ॒ति॒रा॒त्र इत्य॑ति - रा॒त्रः । भ॒व॒ति॒ । सर्व॑स्य । अ॒भिजि॑त्या॒ इत्य॒भि - जि॒त्यै॒ । यत् । प्र॒त्यक्ष॒मिति॑ प्रति - अक्ष᳚म् । पूर्वे॑षु । अह॒स्स्वित्यहः॑ - सु॒ । पृ॒ष्ठानि॑ । उ॒पे॒युरित्यु॑प - इ॒युः । प्र॒त्यक्ष॒मिति॑ प्रति - अक्ष᳚म् । \textbf{  6} \newline
                  \newline
                                \textbf{ TS 7.2.2.3} \newline
                  वि॒श्व॒जितीति॑ विश्व - जिति॑ । यथा᳚ । दु॒ग्धाम् । उ॒प॒सीद॒तीत्यु॑प- सीद॑ति । ए॒वम् । उ॒त्त॒ममित्यु॑त् - त॒मम् । अहः॑ । स्या॒त् । न । ए॒क॒रा॒त्र इत्ये॑क - रा॒त्रः । च॒न । स्या॒त् । बृ॒ह॒द्र॒थ॒न्त॒रे इति॑ बृहत्-र॒थ॒न्त॒रे । पूर्वे॑षु । अह॒स्स्वित्यहः॑ - सु॒ । उपेति॑ । य॒न्ति॒ । इ॒यम् । वाव । र॒थ॒न्त॒रमिति॑ रथं - त॒रम् । अ॒सौ । बृ॒हत् । आ॒भ्याम् । ए॒व । न । य॒न्ति॒ । अथो॒ इति॑ । अ॒नयोः᳚ । ए॒व । प्रतीति॑ । ति॒ष्ठ॒न्ति॒ । यत् । प्र॒त्यक्ष॒मिति॑ प्रति - अक्ष᳚म् । वि॒श्व॒जितीति॑ विश्व - जिति॑ । पृ॒ष्ठानि॑ । उ॒प॒यन्तीत्यु॑प - यन्ति॑ । यथा᳚ । प्रत्ता᳚म् । दु॒हे । ता॒दृक् । ए॒व । तत् ॥ \textbf{  7} \newline
                  \newline
                      (तेज॑ - उपे॒युः प्र॒त्यक्षं॒ - द्विच॑त्वारिꣳशच्च)  \textbf{(A2)} \newline \newline
                                \textbf{ TS 7.2.3.1} \newline
                  बृह॒स्पतिः॑ । अ॒का॒म॒य॒त॒ । ब्र॒ह्म॒व॒र्च॒सीति॑ ब्रह्म - व॒र्च॒सी । स्या॒म् । इति॑ । सः । ए॒तम् । अ॒ष्ट॒रा॒त्रमित्य॑ष्ट-रा॒त्रम् । अ॒प॒श्य॒त् । तम् । एति॑ । अ॒ह॒र॒त् । तेन॑ । अ॒य॒ज॒त॒ । ततः॑ । वै । सः । ब्र॒ह्म॒व॒र्च॒सीति॑ ब्रह्म-व॒र्च॒सी । अ॒भ॒व॒त् । यः । ए॒वम् । वि॒द्वान् । अ॒ष्ट॒रा॒त्रेणेत्य॑ष्ट - रा॒त्रेण॑ । यज॑ते । ब्र॒ह्म॒व॒र्च॒सीति॑ ब्रह्म - व॒र्च॒सी । ए॒व । भ॒व॒ति॒ । अ॒ष्ट॒रा॒त्र इत्य॑ष्ट - रा॒त्रः । भ॒व॒ति॒ । अ॒ष्टाक्ष॒रेत्य॒ष्टा - अ॒क्ष॒रा॒ । गा॒य॒त्री । गा॒य॒त्री । ब्र॒ह्म॒व॒र्च॒समिति॑ ब्रह्म - व॒र्च॒सम् । गा॒य॒त्रि॒या । ए॒व । ब्र॒ह्म॒व॒र्च॒समिति॑ ब्रह्म - व॒र्च॒सम् । अवेति॑ । रु॒न्धे॒ । अ॒ष्ट॒रा॒त्र इत्य॑ष्ट-रा॒त्रः । भ॒व॒ति॒ । चत॑स्रः । वै । दिशः॑ । चत॑स्रः । अ॒वा॒न्त॒र॒दि॒शा इत्य॑वान्तर-दि॒शाः । दि॒ग्भ्य इति॑ दिक् - भ्यः । ए॒व । ब्र॒ह्म॒व॒र्च॒समिति॑ ब्रह्म - व॒र्च॒सम् । अवेति॑ । रु॒न्धे॒ । \textbf{  8} \newline
                  \newline
                                \textbf{ TS 7.2.3.2} \newline
                  त्रि॒वृदिति॑ त्रि - वृत् । अ॒ग्नि॒ष्टो॒म इत्य॑ग्नि-स्तो॒मः । भ॒व॒ति॒ । तेजः॑ । ए॒व । अवेति॑ । रु॒न्धे॒ । प॒ञ्च॒द॒श इति॑ पञ्च - द॒शः । भ॒व॒ति॒ । इ॒न्द्रि॒यम् । ए॒व । अवेति॑ । रु॒न्धे॒ । स॒प्त॒द॒श इति॑ सप्त - द॒शः । भ॒व॒ति॒ । अ॒न्नाद्य॒स्येत्य॑न्न - अद्य॑स्य । अव॑रुद्ध्या॒ इत्यव॑ - रु॒द्ध्यै॒ । अथो॒ इति॑ । प्रेति॑ । ए॒व । तेन॑ । जा॒य॒ते॒ । ए॒क॒विꣳ॒॒श इत्ये॑क - विꣳ॒॒शः । भ॒व॒ति॒ । प्रति॑ष्ठित्या॒ इति॒ प्रति॑ - स्थि॒त्यै॒ । अथो॒ इति॑ । रुच᳚म् । ए॒व । आ॒त्मन्न् । ध॒त्ते॒ । त्रि॒ण॒व इति॑ त्रि-न॒वः । भ॒व॒ति॒ । विजि॑त्या॒ इति॒ वि - जि॒त्यै॒ । त्र॒य॒स्त्रिꣳ॒॒श इति॑ त्रयः - त्रिꣳ॒॒शः । भ॒व॒ति॒ । प्रति॑ष्ठित्या॒ इति॒ प्रति॑ - स्थि॒त्यै॒ । प॒ञ्च॒विꣳ॒॒श इति॑ पञ्च - विꣳ॒॒शः । अ॒ग्नि॒ष्टो॒म इत्य॑ग्नि - स्तो॒मः । भ॒व॒ति॒ । प्र॒जाप॑ते॒रिति॑ प्र॒जा-प॒तेः॒ । आप्त्यै᳚ । म॒हा॒व्र॒तवा॒निति॑ महाव्र॒त-वा॒न् । अ॒न्नाद्य॒स्येत्य॑न्न - अद्य॑स्य । अव॑रुद्ध्या॒ इत्यव॑ - रु॒द्ध्यै॒ । वि॒श्व॒जिदिति॑ विश्व - जित् । सर्व॑पृष्ठ॒ इति॒ सर्व॑ - पृ॒ष्ठः॒ । अ॒ति॒रा॒त्र इत्य॑ति - रा॒त्रः । भ॒व॒ति॒ । सर्व॑स्य । अ॒भिजि॑त्या॒ इत्य॒भि - जि॒त्यै॒ ( ) ॥ \textbf{  9} \newline
                  \newline
                      (दि॒ग्भ्य ए॒व ब्र॑ह्मवर्च॒समव॑ रुन्धे॒ - ऽभिजि॑त्यै)  \textbf{(A3)} \newline \newline
                                \textbf{ TS 7.2.4.1} \newline
                  प्र॒जाप॑ति॒रिति॑ प्र॒जा - प॒तिः॒ । प्र॒जा इति॑ प्र-जाः । अ॒सृ॒ज॒त॒ । ताः । सृ॒ष्टाः । क्षुध᳚म् । नीति॑ । आ॒य॒न्न् । सः । ए॒तम् । न॒व॒रा॒त्रमिति॑ नव - रा॒त्रम् । अ॒प॒श्य॒त् । तम् । एति॑ । अ॒ह॒र॒त् । तेन॑ । अ॒य॒ज॒त॒ । ततः॑ । वै । प्र॒जाभ्य॒ इति॑ प्र - जाभ्यः॑ । अ॒क॒ल्प॒त॒ । यर्.हि॑ । प्र॒जा इति॑ प्र - जाः । क्षुध᳚म् । नि॒गच्छे॑यु॒रिति॑ नि - गच्छे॑युः । तर्.हि॑ । न॒व॒रा॒त्रेणेति॑ नव - रा॒त्रेण॑ । य॒जे॒त॒ । इ॒मे । हि । वै । ए॒तासा᳚म् । लो॒काः । अक्लृ॑प्ताः । अथ॑ । ए॒ताः । क्षुध᳚म् । नीति॑ । ग॒च्छ॒न्ति॒ । इ॒मान् । ए॒व । आ॒भ्यः॒ । लो॒कान् । क॒ल्प॒य॒ति॒ । तान् । कल्प॑मानान् । प्र॒जाभ्य॒ इति॑ प्र - जाभ्यः॑ । अन्विति॑ । क॒ल्प॒ते॒ । कल्प॑न्ते । \textbf{  10} \newline
                  \newline
                                \textbf{ TS 7.2.4.2} \newline
                  अ॒स्मै॒ । इ॒मे । लो॒काः । ऊर्ज᳚म् । प्र॒जास्विति॑ प्र - जासु॑ । द॒धा॒ति॒ । त्रि॒रा॒त्रेणेति॑ त्रि - रा॒त्रेण॑ । ए॒व । इ॒मम् । लो॒कम् । क॒ल्प॒य॒ति॒ । त्रि॒रा॒त्रेणेति॑ त्रि - रा॒त्रेण॑ । अ॒न्तरि॑क्षम् । त्रि॒रा॒त्रेणेति॑ त्रि - रा॒त्रेण॑ । अ॒मुम् । लो॒कम् । यथा᳚ । गु॒णे । गु॒णम् । अ॒न्वस्य॒तीत्य॑नु-अस्य॑ति । ए॒वम् । ए॒व । तत् । लो॒के । लो॒कम् । अन्विति॑ । अ॒स्य॒ति॒ । धृत्यै᳚ । अशि॑थिलंभावा॒येत्यशि॑थिलं - भा॒वा॒य॒ । ज्योतिः॑ । गौः । आयुः॑ । इति॑ । ज्ञा॒ताः । स्तोमाः᳚ । भ॒व॒न्ति॒ । इ॒यम् । वाव । ज्योतिः॑ । अ॒न्तरि॑क्षम् । गौः । अ॒सौ । आयुः॑ । ए॒षु । ए॒व । लो॒केषु॑ । प्रतीति॑ । ति॒ष्ठ॒न्ति॒ । ज्ञात्र᳚म् । प्र॒जाना॒मिति॑ प्र - जाना᳚म् । \textbf{  11} \newline
                  \newline
                                \textbf{ TS 7.2.4.3} \newline
                  ग॒च्छ॒ति॒ । न॒व॒रा॒त्र इति॑ नव - रा॒त्रः । भ॒व॒ति॒ । अ॒भि॒पू॒र्वमित्य॑भि-पू॒र्वम् । ए॒व । अ॒स्मि॒न्न् । तेजः॑ । द॒धा॒ति॒ । यः । ज्योगा॑मया॒वीति॒ ज्योक् - आ॒म॒या॒वी॒ । स्यात् । सः । न॒व॒रा॒त्रेणेति॑ नव - रा॒त्रेण॑ । य॒जे॒त॒ । प्रा॒णा इति॑ प्र-अ॒नाः । हि । वै । ए॒तस्य॑ । अधृ॑ताः । अथ॑ । ए॒तस्य॑ । ज्योक् । आ॒म॒य॒ति॒ । प्रा॒णानिति॑ प्र - अ॒नान् । ए॒व । अ॒स्मि॒न्न् । दा॒धा॒र॒ । उ॒त । यदि॑ । इ॒तासु॒रिती॒त - अ॒सुः॒ । भव॑ति । जीव॑ति । ए॒व ॥ \textbf{  12} \newline
                  \newline
                      (कल्प॑न्ते-प्र॒जनां॒ - त्रय॑स्त्रिꣳशच्च)  \textbf{(A4)} \newline \newline
                                \textbf{ TS 7.2.5.1} \newline
                  प्र॒जाप॑ति॒रिति॑ प्र॒जा - प॒तिः॒ । अ॒का॒म॒य॒त॒ । प्रेति॑ । जा॒ये॒य॒ । इति॑ । सः । ए॒तम् । दश॑होतार॒मिति॒ दश॑ - हो॒ता॒र॒म् । अ॒प॒श्य॒त् । तम् । अ॒जु॒हो॒त् । तेन॑ । द॒श॒रा॒त्रमिति॑ दश - रा॒त्रम् । अ॒सृ॒ज॒त॒ । तेन॑ । द॒श॒रा॒त्रेणेति॑ दश - रा॒त्रेण॑ । प्रेति॑ । अ॒जा॒य॒त॒ । द॒श॒रा॒त्रायेति॑ दश-रा॒त्राय॑ । दी॒क्षि॒ष्यमा॑णः । दश॑होतार॒मिति॒ दश॑ - हो॒ता॒र॒म् । जु॒हु॒या॒त् । दश॑हो॒त्रेति॒ दश॑-हो॒त्रा॒ । ए॒व । द॒श॒रा॒त्रमिति॑ दश-रा॒त्रम् । सृ॒ज॒ते॒ । तेन॑ । द॒श॒रा॒त्रेणेति॑ दश-रा॒त्रेण॑ । प्रेति॑ । जा॒य॒ते॒ । वै॒रा॒जः । वै । ए॒षः । य॒ज्ञ्ः । यत् । द॒श॒रा॒त्र इति॑ दश - रा॒त्रः । यः । ए॒वम् । वि॒द्वान् । द॒श॒रा॒त्रेणेति॑ दश - रा॒त्रेण॑ । यज॑ते । वि॒राज॒मिति॑ वि - राज᳚म् । ए॒व । ग॒च्छ॒ति॒ । प्रा॒जा॒प॒त्य इति॑ प्राजा - प॒त्यः । वै । ए॒षः । य॒ज्ञ्ः । यत् । द॒श॒रा॒त्र इति॑ दश - रा॒त्रः । \textbf{  13} \newline
                  \newline
                                \textbf{ TS 7.2.5.2} \newline
                  यः । ए॒वम् । वि॒द्वान् । द॒श॒रा॒त्रेणेति॑ दश - रा॒त्रेण॑ । यज॑ते । प्रेति॑ । ए॒व । जा॒य॒ते॒ । इन्द्रः॑ । वै । स॒दृङ्ङिति॑ स - दृङ् । दे॒वता॑भिः । आ॒सी॒त् । सः । न । व्या॒वृत॒मिति॑ वि - आ॒वृत᳚म् । अ॒ग॒च्छ॒त् । सः । प्र॒जाप॑ति॒मिति॑ प्र॒जा - प॒ति॒म् । उपेति॑ । अ॒धा॒व॒त् । तस्मै᳚ । ए॒तम् । द॒श॒रा॒त्रमिति॑ दश-रा॒त्रम् । प्रेति॑ । अ॒य॒च्छ॒त् । तम् । एति॑ । अ॒ह॒र॒त् । तेन॑ । अ॒य॒ज॒त॒ । ततः॑ । वै । सः । अ॒न्याभिः॑ । दे॒वता॑भिः । व्या॒वृत॒मिति॑ वि - आ॒वृत᳚म् । अ॒ग॒च्छ॒त् । यः । ए॒वम् । वि॒द्वान् । द॒श॒रा॒त्रेणेति॑ दश - रा॒त्रेण॑ । यज॑ते । व्या॒वृत॒मिति॑ वि - आ॒वृत᳚म् । ए॒व । पा॒प्मना᳚ । भ्रातृ॑व्येण । ग॒च्छ॒ति॒ । त्रि॒क॒कुदिति॑ त्रि - क॒कुत् । वै । \textbf{  14} \newline
                  \newline
                                \textbf{ TS 7.2.5.3} \newline
                  ए॒षः । य॒ज्ञ्ः । यत् । द॒श॒रा॒त्र इति॑ दश - रा॒त्रः । क॒कुत् । प॒ञ्च॒द॒श इति॑ पञ्च - द॒शः । क॒कुत् । ए॒क॒विꣳ॒॒श इत्ये॑क - विꣳ॒॒शः । क॒कुत् । त्र॒य॒स्त्रिꣳ॒॒श इति॑ त्रयः - त्रिꣳ॒॒शः । यः । ए॒वम् । वि॒द्वान् । द॒श॒रा॒त्रेणेति॑ दश - रा॒त्रेण॑ । यज॑ते । त्रि॒क॒कुदिति॑ त्रि - क॒कुत् । ए॒व । स॒मा॒नाना᳚म् । भ॒व॒ति॒ । यज॑मानः । प॒ञ्च॒द॒श इति॑ पञ्च -द॒शः । यज॑मानः । ए॒क॒विꣳ॒॒श इत्ये॑क - विꣳ॒॒शः । यज॑मानः । त्र॒य॒स्त्रिꣳ॒॒श इति॑ त्रयः - त्रिꣳ॒॒शः । पुरः॑ । इत॑राः । अ॒भि॒च॒र्यमा॑ण॒ इत्य॑भि - च॒र्यमा॑णः । द॒श॒रा॒त्रेणेति॑ दश - रा॒त्रेण॑ । य॒जे॒त॒ । दे॒व॒पु॒रा इति॑ देव-पु॒राः । ए॒व । परीति॑ । ऊ॒ह॒ते॒ । तस्य॑ । न । कुतः॑ । च॒न । उ॒पा॒व्या॒ध इत्यु॑प - आ॒व्या॒धः । भ॒व॒ति॒ । न । ए॒न॒म् । अ॒भि॒चर॒न्नित्य॑भि - चरन्न्॑ । स्तृ॒णु॒ते॒ । दे॒वा॒सु॒रा इति॑ देव-अ॒सु॒राः । संॅय॑त्ता॒ इति॒ सं - य॒त्ताः॒ । आ॒स॒न्न् । ते । दे॒वाः । ए॒ताः । \textbf{  15} \newline
                  \newline
                                \textbf{ TS 7.2.5.4} \newline
                  दे॒व॒पु॒रा इति॑ देव-पु॒राः । अ॒प॒श्य॒न्न् । यत् । द॒श॒रा॒त्र इति॑ दश-रा॒त्रः । ताः । परीति॑ । औ॒ह॒न्त॒ । तेषा᳚म् । न । कुतः॑ । च॒न । उ॒पा॒व्या॒ध इत्यु॑प - आ॒व्या॒धः । अ॒भ॒व॒त् । ततः॑ । दे॒वाः । अभ॑वन्न् । परेति॑ । असु॑राः । यः । भ्रातृ॑व्यवा॒निति॒ भ्रातृ॑व्य - वा॒न् । स्यात् । सः । द॒श॒रा॒त्रेणेति॑ दश - रा॒त्रेण॑ । य॒जे॒त॒ । दे॒व॒पु॒रा इति॑ देव - पु॒राः । ए॒व । परीति॑ । ऊ॒ह॒ते॒ । तस्य॑ । न । कुतः॑ । च॒न । उ॒पा॒व्या॒ध इत्यु॑प- आ॒व्या॒धः । भ॒व॒ति॒ । भव॑ति । आ॒त्मना᳚ । परेति॑ । अ॒स्य॒ । भ्रातृ॑व्यः । भ॒व॒ति॒ । स्तोमः॑ । स्तोम॑स्य । उप॑स्तिः । भ॒व॒ति॒ । भ्रातृ॑व्यम् । ए॒व । उप॑स्तिम् । कु॒रु॒ते॒ । जा॒मि । वै । \textbf{  16} \newline
                  \newline
                                \textbf{ TS 7.2.5.5} \newline
                  ए॒तत् । कु॒र्व॒न्ति॒ । यत् । ज्यायाꣳ॑सम् । स्तोम᳚म् । उ॒पेत्येत्यु॑प-इत्य॑ । कनी॑याꣳसम् । उ॒प॒यन्तीत्यु॑प - यन्ति॑ । यत् । अ॒ग्नि॒ष्टो॒म॒सा॒मानीत्य॑ग्निष्टोम-सा॒मानि॑ । अ॒वस्ता᳚त् । च॒ । प॒रस्ता᳚त् । च॒ । भव॑न्ति । अजा॑मित्वा॒येत्यजा॑मि - त्वा॒य॒ । त्रि॒वृदिति॑ त्रि - वृत् । अ॒ग्नि॒ष्टो॒म इत्य॑ग्नि - स्तो॒मः । अ॒ग्नि॒ष्टुदित्य॑ग्नि-स्तुत् । आ॒ग्ने॒यीषु॑ । भ॒व॒ति॒ । तेजः॑ । ए॒व । अवेति॑ । रु॒न्धे॒ । प॒ञ्च॒द॒श इति॑ पञ्च-द॒शः । उ॒क्थ्यः॑ । ऐ॒न्द्रीषु॑ । इ॒न्द्रि॒यम् । ए॒व । अवेति॑ । रु॒न्धे॒ । त्रि॒व॒दिति॑ त्रि - वृत् । अ॒ग्नि॒ष्टो॒म इत्य॑ग्नि - स्तो॒मः । वै॒श्व॒दे॒वीष्विति॑ वैश्व - दे॒वीषु॑ । पुष्टि᳚म् । ए॒व । अवेति॑ । रु॒न्धे॒ । स॒प्त॒द॒श इति॑ सप्त - द॒शः । अ॒ग्नि॒ष्टो॒म इत्य॑ग्नि - स्तो॒मः । प्रा॒जा॒प॒त्यास्विति॑ प्राजा - प॒त्यासु॑ । ती॒व्र॒सो॒म इति॑ तीव्र-सो॒मः । अ॒न्नाद्य॒स्येत्य॑न्न - अद्य॑स्य । अव॑रुद्ध्या॒ इत्यव॑-रुद्॒ध्यै॒ । अथो॒ इति॑ । प्रेति॑ । ए॒व । तेन॑ । जा॒य॒ते॒ । \textbf{  17} \newline
                  \newline
                                \textbf{ TS 7.2.5.6} \newline
                  ए॒क॒विꣳ॒॒श इत्ये॑क - विꣳ॒॒शः । उ॒क्थ्यः॑ । सौ॒रीषु॑ । प्रति॑ष्ठित्या॒ इति॒ प्रति॑ - स्थि॒त्यै॒ । अथो॒ इति॑ । रुच᳚म् । ए॒व । आ॒त्मन्न् । ध॒त्ते॒ । स॒प्त॒द॒श इति॑ सप्त - द॒शः । अ॒ग्नि॒ष्टो॒म इत्य॑ग्नि - स्तो॒मः । प्रा॒जा॒प॒त्यास्विति॑ प्राजा - प॒त्यासु॑ । उ॒प॒ह॒व्य॑ इत्यु॑प - ह॒व्यः॑ । उ॒प॒ह॒वमित्यु॑प - ह॒वम् । ए॒व । ग॒च्छ॒ति॒ । त्रि॒ण॒वाविति॑ त्रि - न॒वौ । अ॒ग्नि॒ष्टो॒मावित्य॑ग्नि - स्तो॒मौ । अ॒भितः॑ । ऐ॒न्द्रीषु॑ । विजि॑त्या॒ इति॒ वि - जि॒त्यै॒ । त्र॒य॒स्त्रिꣳ॒॒श इति॑ त्रयः - त्रिꣳ॒॒शः । उ॒क्थ्यः॑ । वै॒श्व॒दे॒वीष्विति॑ वैश्व - दे॒वीषु॑ । प्रति॑ष्ठित्या॒ इति॒ प्रति॑ - स्थि॒त्यै॒ । वि॒श्व॒जिदिति॑ विश्व - जित् । सर्व॑पृष्ठ॒ इति॒ सर्व॑ - पृ॒ष्ठः॒ । अ॒ति॒रा॒त्र इत्य॑ति - रा॒त्रः । भ॒व॒ति॒ । सर्व॑स्य । अ॒भिजि॑त्या॒ इत्य॒भि - जि॒त्यै॒ ॥ \textbf{  18} \newline
                  \newline
                      (प्र॒जा॒प॒त्यो वा ए॒ष य॒ज्ञो यद् द॑शरा॒त्र - स्त्रि॑क॒कुद्धा - ए॒ता - वै - जा॑यत॒ - एक॑त्रिꣳशच्च)  \textbf{(A5)} \newline \newline
                                \textbf{ TS 7.2.6.1} \newline
                  ऋ॒तवः॑ । वै । प्र॒जाका॑मा॒ इति॑ प्र॒जा - का॒माः॒ । प्र॒जामिति॑ प्र - जाम् । न । अ॒वि॒न्द॒न्त॒ । ते । अ॒का॒म॒य॒न्त॒ । प्र॒जामिति॑ प्र - जाम् । सृ॒जे॒म॒हि॒ । प्र॒जामिति॑ प्र-जाम् । अवेति॑ । रु॒न्धी॒म॒हि॒ । प्र॒जामिति॑ प्र - जाम् । वि॒न्दे॒म॒हि॒ । प्र॒जाव॑न्त॒ इति॑ प्र॒जा - व॒न्तः॒ । स्या॒म॒ । इति॑ । ते । ए॒तम् । ए॒का॒द॒श॒रा॒त्रमित्ये॑कादश - रा॒त्रम् । अ॒प॒श्य॒न्न् । तम् । एति॑ । अ॒ह॒र॒न्न् । तेन॑ । अ॒य॒ज॒न्त॒ । ततः॑ । वै । ते । प्र॒जामिति॑ प्र - जाम् । अ॒सृ॒ज॒न्त॒ । प्र॒जामिति॑ प्र - जाम् । अवेति॑ । अ॒रु॒न्ध॒त॒ । प्र॒जामिति॑ प्र - जाम् । अ॒वि॒न्द॒न्त॒ । प्र॒जाव॑न्त॒ इति॑ प्र॒जा - व॒न्तः॒ । अ॒भ॒व॒न्न् । ते । ऋ॒तवः॑ । अ॒भ॒व॒न्न् । तत् । आ॒र्त॒वाना᳚म् । आ॒र्त॒व॒त्वमित्या᳚र्तव - त्वम् । ऋ॒तू॒नाम् । वै । ए॒ते । पु॒त्राः । तस्मा᳚त् । \textbf{  19} \newline
                  \newline
                                \textbf{ TS 7.2.6.2} \newline
                  आ॒र्त॒वाः । उ॒च्य॒न्ते॒ । ये । ए॒वम् । वि॒द्वाꣳसः॑ । ए॒का॒द॒श॒रा॒त्रमित्ये॑कादश - रा॒त्रम् । आस॑ते । प्र॒जामिति॑ प्र - जाम् । ए॒व । सृ॒ज॒न्ते॒ । प्र॒जामिति॑ प्र - जाम् । अवेति॑ । रु॒न्ध॒ते॒ । प्र॒जामिति॑ प्र - जाम् । वि॒न्द॒न्ते॒ । प्र॒जाव॑न्त॒ इति॑ प्र॒जा - व॒न्तः॒ । भ॒व॒न्ति॒ । ज्योतिः॑ । अ॒ति॒रा॒त्र इत्य॑ति - रा॒त्रः । भ॒व॒ति॒ । ज्योतिः॑ । ए॒व । पु॒रस्ता᳚त् । द॒ध॒ते॒ । सु॒व॒र्गस्येति॑ सुवः - गस्य॑ । लो॒कस्य॑ । अनु॑ख्यात्या॒ इत्यनु॑ - ख्या॒त्यै॒ । पृष्ठ्यः॑ । ष॒ड॒ह इति॑ षट् - अ॒हः । भ॒व॒ति॒ । षट् । वै । ऋ॒तवः॑ । षट् । पृ॒ष्ठानि॑ । पृ॒ष्ठैः । ए॒व । ऋ॒तून् । अ॒न्वारो॑ह॒न्तीत्य॑नु - आरो॑हन्ति । ऋ॒तुभि॒रित्यृ॒तु - भिः॒ । सं॒ॅव॒थ्स॒रमिति॑ सं - व॒थ्स॒रम् । ते । सं॒ॅव॒थ्स॒र इति॑ सं - व॒थ्स॒रे । ए॒व । प्रतीति॑ । ति॒ष्ठ॒न्ति॒ । च॒तु॒र्विꣳ॒॒श इति॑ चतुः-विꣳ॒॒शः । भ॒व॒ति॒ । चतु॑र्विꣳशत्यक्ष॒रेति॒ चतु॑विꣳशति - अ॒क्ष॒रा॒ । गा॒य॒त्री । \textbf{  20} \newline
                  \newline
                                \textbf{ TS 7.2.6.3} \newline
                  गा॒य॒त्रम् । ब्र॒ह्म॒व॒र्च॒समिति॑ ब्रह्म - व॒र्च॒सम् । गा॒य॒त्रि॒याम् । ए॒व । ब्र॒ह्म॒व॒र्च॒स इति॑ ब्रह्म - व॒र्च॒से । प्रतीति॑ । ति॒ष्ठ॒न्ति॒ । च॒तु॒श्च॒त्वा॒रिꣳ॒॒श इति॑ चतुः - च॒त्वा॒रिꣳ॒॒शः । भ॒व॒ति॒ । चतु॑श्चत्वारिꣳशदक्ष॒रेति॒ चतु॑श्चत्वारिꣳशत् - अ॒क्ष॒रा॒ । त्रि॒ष्टुक् । इ॒न्द्रि॒यम् । त्रि॒ष्टुप् । त्रि॒ष्टुभि॑ । ए॒व । इ॒न्द्रि॒ये । प्रतीति॑ । ति॒ष्ठ॒न्ति॒ । अ॒ष्टा॒च॒त्वा॒रिꣳ॒॒श इत्य॑ष्टा - च॒त्वा॒रिꣳ॒॒शः । भ॒व॒ति॒ । अ॒ष्टाच॑त्वारिꣳशदक्ष॒रेत्य॒ष्टाच॑त्वारिꣳशत् - अ॒क्ष॒रा॒ । जग॑ती । जाग॑ताः । प॒शवः॑ । जग॑त्याम् । ए॒व । प॒शुषु॑ । प्रतीति॑ । ति॒ष्ठ॒न्ति॒ । ए॒का॒द॒श॒रा॒त्र इत्ये॑कादश-रा॒त्रः । भ॒व॒ति॒ । पञ्च॑ । वै । ऋ॒तवः॑ । आ॒र्त॒वाः । पञ्च॑ । ऋ॒तुषु॑ । ए॒व । आ॒र्त॒वेषु॑ । सं॒ॅव॒थ्स॒र इति॑ सं - व॒थ्स॒रे । प्र॒ति॒ष्ठायेति॑ प्रति - स्थाय॑ । प्र॒जामिति॑ प्र - जाम् । अवेति॑ । रु॒न्ध॒ते॒ । अ॒ति॒रा॒त्रावित्य॑ति - रा॒त्रौ । अ॒भितः॑ । भ॒व॒तः॒ । प्र॒जाया॒ इति॑ प्र - जायै᳚ । परि॑गृहीत्या॒ इति॒ परि॑-गृ॒ही॒त्यै॒ ॥ \textbf{  21 } \newline
                  \newline
                      (तस्मा᳚द् - गाय॒त्र्ये - का॒न्नप॑ञ्चा॒शच्च॑)  \textbf{(A6)} \newline \newline
                                \textbf{ TS 7.2.7.1} \newline
                  ऐ॒न्द्र॒वा॒य॒वाग्रा॒नित्यै᳚न्द्रवाय॒व - अ॒ग्रा॒न् । गृ॒ह्णी॒या॒त् । यः । का॒मये॑त । य॒था॒पू॒र्वमिति॑ यथा - पू॒र्वम् । प्र॒जा इति॑ प्र - जाः । क॒ल्पे॒र॒न्न् । इति॑ । य॒ज्ञ्स्य॑ । वै । क्लृप्ति᳚म् । अन्विति॑ । प्र॒जा इति॑ प्र - जाः । क॒ल्प॒न्ते॒ । य॒ज्ञ्स्य॑ । अक्लृ॑प्तिम् । अन्विति॑ । न । क॒ल्प॒न्ते॒ । य॒था॒पू॒र्वमिति॑ यथा - पू॒र्वम् । ए॒व । प्र॒जा इति॑ प्र - जाः । क॒ल्प॒य॒ति॒ । न । ज्यायाꣳ॑सम् । कनी॑यान् । अतीति॑ । क्रा॒म॒ति॒ । ऐ॒न्द्र॒वा॒य॒वाग्रा॒नित्यै᳚न्द्रवाय॒व - अ॒ग्रा॒न् । गृ॒ह्णी॒या॒त् । आ॒म॒या॒विनः॑ । प्रा॒णेनेति॑ प्र - अ॒नेन॑ । वै । ए॒षः । वीति॑ । ऋ॒द्ध्य॒ते॒ । यस्य॑ । आ॒मय॑ति । प्रा॒ण इति॑ प्र - अ॒नः । ऐ॒न्द्र॒वा॒य॒व इत्यै᳚न्द्र - वा॒य॒वः । प्रा॒णेनेति॑ प्र - अ॒नेन॑ । ए॒व । ए॒न॒म् । समिति॑ । अ॒द्‌र्ध॒य॒ति॒ । मै॒त्रा॒व॒रु॒णाग्रा॒निति॑ मैत्रावरु॒ण - अ॒ग्रा॒न् । गृ॒ह्णी॒र॒न्न् । येषा᳚म् । दी॒क्षि॒ताना᳚म् । प्र॒मीये॒तेति॑ प्र - मीये॑त । \textbf{  22} \newline
                  \newline
                                \textbf{ TS 7.2.7.2} \newline
                  प्रा॒णा॒पा॒नाभ्या॒मिति॑ प्राण - अ॒पा॒नाभ्या᳚म् । वै । ए॒ते । वीति॑ । ऋ॒द्ध्य॒न्ते॒ । येषा᳚म् । दी॒क्षि॒ताना᳚म् । प्र॒मीय॑त॒ इति॑ प्र - मीय॑ते । प्रा॒णा॒पा॒नाविति॑ प्राण - अ॒पा॒नौ । मि॒त्रावरु॑णा॒विति॑ मि॒त्रा - वरु॑णौ । प्रा॒णा॒पा॒नाविति॑ प्राण - अ॒पा॒नौ । ए॒व । मु॒ख॒तः । परीति॑ । ह॒र॒न्ते॒ । आ॒श्वि॒नाग्रा॒नित्या᳚श्वि॒न-अ॒ग्रा॒न् । गृ॒ह्णी॒त॒ । आ॒नु॒जा॒व॒र इत्या॑नु-जा॒व॒रः । अ॒श्विनौ᳚ । वै । दे॒वाना᳚म् । आ॒नु॒जा॒व॒रावित्या॑नु - जा॒व॒रौ । प॒श्चा । इ॒व॒ । अग्र᳚म् । परीति॑ । ऐ॒ता॒म् । अ॒श्विनौ᳚ । ए॒तस्य॑ । दे॒वता᳚ । यः । आ॒नु॒जा॒व॒र इत्या॑नु - जा॒व॒रः । तौ । ए॒व । ए॒न॒म् । अग्र᳚म् । परीति॑ । न॒य॒तः॒ । शु॒क्राग्रा॒निति॑ शु॒क्र - अ॒ग्रा॒न् । गृ॒ह्णी॒त॒ । ग॒तश्री॒रिति॑ ग॒त - श्रीः॒ । प्र॒ति॒ष्ठाका॑म॒ इति॑ प्रति॒ष्ठा - का॒मः॒ । अ॒सौ । वै । आ॒दि॒त्यः । शु॒क्रः । ए॒षः । अन्तः॑ । अन्त᳚म् । म॒नु॒ष्यः॑ । \textbf{  23} \newline
                  \newline
                                \textbf{ TS 7.2.7.3} \newline
                  श्रि॒यै । ग॒त्वा । नीति॑ । व॒र्त॒ते॒ । अन्ता᳚त् । ए॒व । अन्त᳚म् । एति॑ । र॒भ॒ते॒ । न । ततः॑ । पापी॑यान् । भ॒व॒ति॒ । म॒न्थ्य॑ग्रा॒निति॑ म॒न्थि-अ॒ग्रा॒न् । गृ॒ह्णी॒त॒ । अ॒भि॒चर॒न्नित्य॑भि - चरन्न्॑ । आ॒र्त॒पा॒त्रमित्या᳚र्त - पा॒त्रम् । वै । ए॒तत् । यत् । म॒न्थि॒पा॒त्रमिति॑ मन्थि - पा॒त्रम् । मृ॒त्युना᳚ । ए॒व । ए॒न॒म् । ग्रा॒ह॒य॒ति॒ । ता॒जक् । आर्ति᳚म् । एति॑ । ऋ॒च्छ॒ति॒ । आ॒ग्र॒य॒णाग्रा॒नित्या᳚ग्रय॒ण-अ॒ग्रा॒न् । गृ॒ह्णी॒त॒ । यस्य॑ । पि॒ता । पि॒ता॒म॒हः । पुण्यः॑ । स्यात् । अथ॑ । तत् । न । प्रा॒प्नु॒यादिति॑ प्र - आ॒प्नु॒यात् । वा॒चा । वै । ए॒षः । इ॒न्द्रि॒येण॑ । वीति॑ । ऋ॒द्ध्य॒ते॒ । यस्य॑ । पि॒ता । पि॒ता॒म॒हः । पुण्यः॑ । \textbf{  24} \newline
                  \newline
                                \textbf{ TS 7.2.7.4} \newline
                  भव॑ति । अथ॑ । तत् । न । प्रा॒प्नोतीति॑ प्र - आ॒प्नोति॑ । उरः॑ । इ॒व॒ । ए॒तत् । य॒ज्ञ्स्य॑ । वाक् । इ॒व॒ । यत् । आ॒ग्र॒य॒णः । वा॒चा । ए॒व । ए॒न॒म् । इ॒न्द्रि॒येण॑ । समिति॑ । अ॒द्‌र्ध॒य॒ति॒ । न । ततः॑ । पापी॑यान् । भ॒व॒ति॒ । उ॒क्थ्या᳚ग्रा॒नित्यु॒क्थ्य॑ - अ॒ग्रा॒न् । गृ॒ह्णी॒त॒ । अ॒भि॒च॒र्यमा॑ण॒ इत्य॑भि - च॒र्यमा॑णः । सर्वे॑षाम् । वै । ए॒तत् । पात्रा॑णाम् । इ॒न्द्रि॒यम् । यत् । उ॒क्थ्य॒पा॒त्रमित्यु॑क्थ्य - पा॒त्रम् । सर्वे॑ण । ए॒व । ए॒न॒म् । इ॒न्द्रि॒येण॑ । अति॑ । प्रेति॑ । यु॒ङ्क्ते॒ । सर॑स्वति । अ॒भीति॑ । नः॒ । ने॒षि॒ । वस्यः॑ । इति॑ । पु॒रो॒रुच॒मिति॑ पुरः - रुच᳚म् । कु॒र्या॒त् । वाक् । वै । \textbf{  25} \newline
                  \newline
                                \textbf{ TS 7.2.7.5} \newline
                  सर॑स्वती । वा॒चा । ए॒व । ए॒न॒म् । अति॑ । प्रेति॑ । यु॒ङ्क्ते॒ । मा । त्वत् । क्षेत्रा॑णि । अर॑णानि । ग॒न्म॒ । इति॑ । आ॒ह॒ । मृ॒त्योः । वै । क्षेत्रा॑णि । अर॑णानि । तेन॑ । ए॒व । मृ॒त्योः । क्षेत्रा॑णि । न । ग॒च्छ॒ति॒ । पू॒र्णान् । ग्रहान्॑ । गृ॒ह्णी॒या॒त् । आ॒म॒या॒विनः॑ । प्रा॒णानिति॑ प्र - अ॒नान् । वै । ए॒तस्य॑ । शुक् । ऋ॒च्छ॒ति॒ । यस्य॑ । आ॒मय॑ति । प्रा॒णा इति॑ प्र - अ॒नाः । ग्रहाः᳚ । प्रा॒णानिति॑ प्र-अ॒नान् । ए॒व । अ॒स्य॒ । शु॒चः । मु॒ञ्च॒ति॒ । उ॒त । यदि॑ । इ॒तासु॒रिती॒त - अ॒सुः॒ । भव॑ति । जीव॑ति । ए॒व । पू॒र्णान् । ग्रहान्॑ ( ) । गृ॒ह्णी॒या॒त् । यर्.हि॑ । प॒र्जन्यः॑ । न । वर्.ष᳚त् । प्रा॒णानिति॑ प्र - अ॒नान् । वै । ए॒तर्.हि॑ । प्र॒जाना॒मिति॑ प्र - जाना᳚म् । शुक् । ऋ॒च्छ॒ति॒ । यर्.हि॑ । प॒र्जन्यः॑ । न । वर्.ष॑ति । प्रा॒णा इति॑ प्र - अ॒नाः । ग्रहाः᳚ । प्रा॒णानिति॑ प्र - अ॒नान् । ए॒व । प्र॒जाना॒मिति॑ प्र - जाना᳚म् । शु॒चः । मु॒ञ्च॒ति॒ । ता॒जक् । प्रेति॑ । व॒र्.ष॒ति॒ ॥ \textbf{  26} \newline
                  \newline
                      (प्र॒मीये॑त - मनु॒ष्य॑ - ऋद्ध्यते॒ यस्य॑ पि॒ता पि॑ताम॒हः पुण्यो॒-वाग्वा-ए॒व पू॒र्णान् ग्रहा॒न्-पञ्च॑विꣳशतिश्च)  \textbf{(A7)} \newline \newline
                                \textbf{ TS 7.2.8.1} \newline
                  गा॒य॒त्रः । वै । ऐ॒न्द्र॒वा॒य॒व इत्यै᳚न्द्र - वा॒य॒वः । गा॒य॒त्रम् । प्रा॒य॒णीय॒मिति॑ प्र - अ॒य॒नीय᳚म् । अहः॑ । तस्मा᳚त् । प्रा॒य॒णीय॒ इति॑ प्र - अ॒य॒नीये᳚ । अहन्न्॑ । ऐ॒न्द्र॒वा॒य॒व इत्यै᳚न्द्र - वा॒य॒वः । गृ॒ह्य॒ते॒ । स्वे । ए॒व । ए॒न॒म् । आ॒यत॑न॒ इत्या᳚ - यत॑ने । गृ॒ह्णा॒ति॒ । त्रैष्टु॑भः । वै । शु॒क्रः । त्रैष्टु॑भम् । द्वि॒तीय᳚म् । अहः॑ । तस्मा᳚त् । द्वि॒तीये᳚ । अहन्न्॑ । शु॒क्रः । गृ॒ह्य॒ते॒ । स्वे । ए॒व । ए॒न॒म् । आ॒यत॑न॒ इत्या᳚ - यत॑ने । गृ॒ह्णा॒ति॒ । जाग॑तः । वै । आ॒ग्र॒य॒णः । जाग॑तम् । तृ॒तीय᳚म् । अहः॑ । तस्मा᳚त् । तृ॒तीये᳚ । अहन्न्॑ । आ॒ग्र॒य॒णः । गृ॒ह्य॒ते॒ । स्वे । ए॒व । ए॒न॒म् । आ॒यत॑न॒ इत्या᳚ - यत॑ने । गृ॒ह्णा॒ति॒ । ए॒तत् । वै । \textbf{  27} \newline
                  \newline
                                \textbf{ TS 7.2.8.2} \newline
                  य॒ज्ञ्म् । आ॒प॒त् । यत् । छन्दाꣳ॑सि । आ॒प्नोति॑ । यत् । आ॒ग्र॒य॒णः । श्वः । गृ॒ह्यते᳚ । यत्र॑ । ए॒व । य॒ज्ञ्म् । अदृ॑शन्न् । ततः॑ । ए॒व । ए॒न॒म् । पुनः॑ । प्रेति॑ । यु॒ङ्क्ते॒ । जग॑न्मुख॒ इति॒ जग॑त् - मु॒खः॒ । वै । द्वि॒तीयः॑ । त्रि॒रा॒त्र इति॑ त्रि - रा॒त्रः । जाग॑तः । आ॒ग्र॒य॒णः । यत् । च॒तु॒र्थे । अहन्न्॑ । आ॒ग्र॒य॒णः । गृ॒ह्यते᳚ । स्वे । ए॒व । ए॒न॒म् । आ॒यत॑न॒ इत्या᳚ - यत॑ने । गृ॒ह्णा॒ति॒ । अथो॒ इति॑ । स्वम् । ए॒व । छन्दः॑ । अन्विति॑ । प॒र्याव॑र्तन्त॒ इति॑ परि-आव॑र्तन्ते । राथ॑न्तर॒ इति॒ राथं᳚-त॒रः॒ । वै । ऐ॒न्द्र॒वा॒य॒व इत्यै᳚न्द्र - वा॒य॒वः । राथ॑न्तर॒मिति॒ राथं᳚ - त॒र॒म् । प॒ञ्च॒मम् । अहः॑ । तस्मा᳚त् । प॒ञ्च॒मे । अहन्न्॑ । \textbf{  28} \newline
                  \newline
                                \textbf{ TS 7.2.8.3} \newline
                  ऐ॒न्द्र॒वा॒य॒व इत्यै᳚न्द्र - वा॒य॒वः । गृ॒ह्य॒ते॒ । स्वे । ए॒व । ए॒न॒म् । आ॒यत॑न॒ इत्या᳚ - यत॑ने । गृ॒ह्णा॒ति॒ । बार्.ह॑तः । वै । शु॒क्रः । बार्.ह॑तम् । ष॒ष्ठम् । अहः॑ । तस्मा᳚त् । ष॒ष्ठे । अहन्न्॑ । शु॒क्रः । गृ॒ह्य॒ते॒ । स्वे । ए॒व । ए॒न॒म् । आ॒यत॑न॒ इत्या᳚ - यत॑ने । गृ॒ह्णा॒ति॒ । ए॒तत् । वै । द्वि॒तीय᳚म् । य॒ज्ञ्म् । आ॒प॒त् । यत् । छन्दाꣳ॑सि । आ॒प्नोति॑ । यत् । शु॒क्रः । श्वः । गृ॒ह्यते᳚ । यत्र॑ । ए॒व । य॒ज्ञ्म् । अदृ॑शन्न् । ततः॑ । ए॒व । ए॒न॒म् । पुनः॑ । प्रेति॑ । यु॒ङ्क्ते॒ । त्रि॒ष्टुङ्मु॑ख॒ इति॑ त्रि॒ष्टुक् - मु॒खः॒ । वै । तृ॒तीयः॑ । त्रि॒रा॒त्र इति॑ त्रि - रा॒त्रः । त्रैष्टु॑भः । \textbf{  29} \newline
                  \newline
                                \textbf{ TS 7.2.8.4} \newline
                  शु॒क्रः । यत् । स॒प्त॒मे । अहन्न्॑ । शु॒क्रः । गृ॒ह्यते᳚ । स्वे । ए॒व । ए॒न॒म् । आ॒यत॑न॒ इत्या᳚ - यत॑ने । गृ॒ह्णा॒ति॒ । अथो॒ इति॑ । स्वम् । ए॒व । छन्दः॑ । अन्विति॑ । प॒र्याव॑र्तन्त॒ इति॑ परि - आव॑र्तन्ते । वाक् । वै । आ॒ग्र॒य॒णः । वाक् । अ॒ष्ट॒मम् । अहः॑ । तस्मा᳚त् । अ॒ष्ट॒मे । अहन्न्॑ । आ॒ग्र॒य॒णः । गृ॒ह्य॒ते॒ । स्वे । ए॒व । ए॒न॒म् । आ॒यत॑न॒ इत्या᳚ - यत॑ने । गृ॒ह्णा॒ति॒ । प्रा॒ण इति॑ प्र - अ॒नः । वै । ऐ॒न्द्र॒वा॒य॒व इत्यै᳚न्द्र - वा॒य॒वः । प्रा॒ण इति॑ प्र - अ॒नः । न॒व॒मम् । अहः॑ । तस्मा᳚त् । न॒व॒मे । अहन्न्॑ । ऐ॒न्द्र॒वा॒य॒व इत्यै᳚न्द्र - वा॒य॒वः । गृ॒ह्य॒ते॒ । स्वे । ए॒व । ए॒न॒म् । आ॒यत॑न॒ इत्या᳚ - यत॑ने । गृ॒ह्णा॒ति॒ । ए॒तत् । \textbf{  30} \newline
                  \newline
                                \textbf{ TS 7.2.8.5} \newline
                  वै । तृ॒तीय᳚म् । य॒ज्ञ्म् । आ॒प॒त् । यत् । छन्दाꣳ॑सि । आ॒प्नोति॑ । यत् । ऐ॒न्द्र॒वा॒य॒व इत्यै᳚न्द्र - वा॒य॒वः । श्वः । गृ॒ह्यते᳚ । यत्र॑ । ए॒व । य॒ज्ञ्म् । अदृ॑शन्न् । ततः॑ । ए॒व । ए॒न॒म् । पुनः॑ । प्रेति॑ । यु॒ङ्क्ते॒ । अथो॒ इति॑ । स्वम् । ए॒व । छन्दः॑ । अन्विति॑ । प॒र्याव॑र्तन्त॒ इति॑ परि - आव॑र्तन्ते । प॒थः । वै । ए॒ते । अधीति॑ । अप॑थेन । य॒न्ति॒ । ये । अ॒न्येन॑ । ऐ॒न्द्र॒वा॒य॒वादित्यै᳚न्द्र - वा॒य॒वात् । प्र॒ति॒पद्य॑न्त॒ इति॑ प्रति - पद्य॑न्ते । अन्तः॑ । खलु॑ । वै । ए॒षः । य॒ज्ञ्स्य॑ । यत् । द॒श॒मम् । अहः॑ । द॒श॒मे । अहन्न्॑ । ऐ॒न्द्र॒वा॒य॒व इत्यै᳚न्द्र - वा॒य॒वः । गृ॒ह्य॒ते॒ । य॒ज्ञ्स्य॑ । \textbf{  31} \newline
                  \newline
                                \textbf{ TS 7.2.8.6} \newline
                  ए॒व । अन्त᳚म् । ग॒त्वा । अप॑थात् । पन्था᳚म् । अपीति॑ । य॒न्ति॒ । अथो॒ इति॑ । यथा᳚ । वही॑यसा । प्र॒ति॒सार॒मिति॑ प्रति-सार᳚म् । वह॑न्ति । ता॒दृक् । ए॒व । तत् । छन्दाꣳ॑सि । अ॒न्यः । अ॒न्यस्य॑ । लो॒कम् । अ॒भीति॑ । अ॒द्ध्या॒य॒न्न् । तानि॑ । ए॒तेन॑ । ए॒व । दे॒वाः । वीति॑ । अ॒वा॒ह॒य॒न्न् । ऐ॒न्द्र॒वा॒य॒वस्येत्यै᳚न्द्र - वा॒य॒वस्य॑ । वै । ए॒तत् । आ॒यत॑न॒मित्या᳚ - यत॑नम् । यत् । च॒तु॒र्थम् । अहः॑ । तस्मिन्न्॑ । आ॒ग्र॒य॒णः । गृ॒ह्य॒ते॒ । तस्मा᳚त् । आ॒ग्र॒य॒णस्य॑ । आ॒यत॑न॒ इत्या᳚-यत॑ने । न॒व॒मे । अहन्न्॑ । ऐ॒न्द्र॒वा॒य॒व इत्यै᳚न्द्र - वा॒य॒वः । गृ॒ह्य॒ते॒ । शु॒क्रस्य॑ । वै । ए॒तत् । आ॒यत॑न॒मित्या᳚ - यत॑नम् । यत् । प॒ञ्च॒मम् । \textbf{  32} \newline
                  \newline
                                \textbf{ TS 7.2.8.7} \newline
                  अहः॑ । तस्मिन्न्॑ । ऐ॒न्द्र॒वा॒य॒व इत्यै᳚न्द्र - वा॒य॒वः । गृ॒ह्य॒ते॒ । तस्मा᳚त् । ऐ॒न्द्र॒वा॒य॒वस्येत्यै᳚न्द्र - वा॒य॒वस्य॑ । आ॒यत॑न॒ इत्या᳚-यत॑ने । स॒प्त॒मे । अहन्न्॑ । शु॒क्रः । गृ॒ह्य॒ते॒ । आ॒ग्र॒य॒णस्य॑ । वै । ए॒तत् । आ॒यत॑न॒मित्या᳚ - यत॑नम् । यत् । ष॒ष्ठम् । अहः॑ । तस्मिन्न्॑ । शु॒क्रः । गृ॒ह्य॒ते॒ । तस्मा᳚त् । शु॒क्रस्य॑ । आ॒यत॑न॒ इत्या᳚ - यत॑ने । अ॒ष्ट॒मे । अहन्न्॑ । आ॒ग्र॒य॒णः । गृ॒ह्य॒ते॒ । छन्दाꣳ॑सि । ए॒व । तत् । वीति॑ । वा॒ह॒य॒ति॒ । प्रेति॑ । वस्य॑सः । वि॒वा॒हमिति॑ वि-वा॒हम् । आ॒प्नो॒ति॒ । यः । ए॒वम् । वेद॑ । अथो॒ इति॑ । दे॒वता᳚भ्यः । ए॒व । य॒ज्ञे । सं॒ॅविद॒मिति॑ सं - विद᳚म् । द॒धा॒ति॒ । तस्मा᳚त् । इ॒दम् । अ॒न्यः । अ॒न्यस्मै᳚ ( ) । द॒दा॒ति॒ ॥ \textbf{  33 } \newline
                  \newline
                      (ए॒तद्वै - प॑ञ्च॒मेऽह॒न् - त्रैष्टु॑भ - ए॒तद् - गृ॑ह्यते य॒ज्ञ्स्य॑ - पञ्च॒म - म॒न्यस्मा॒ - एक॑ञ्च)  \textbf{(A8)} \newline \newline
                                \textbf{ TS 7.2.9.1} \newline
                  प्र॒जाप॑ति॒रिति॑ प्र॒जा - प॒तिः॒ । अ॒का॒म॒य॒त॒ । प्रेति॑ । जा॒ये॒य॒ । इति॑ । सः । ए॒तम् । द्वा॒द॒श॒रा॒त्रमिति॑ द्वादश - रा॒त्रम् । अ॒प॒श्य॒त् । तम् । एति॑ । अ॒ह॒र॒त् । तेन॑ । अ॒य॒ज॒त॒ । ततः॑ । वै । सः । प्रेति॑ । अ॒जा॒य॒त॒ । यः । का॒मये॑त । प्रेति॑ । जा॒ये॒य॒ । इति॑ । सः । द्वा॒द॒श॒रा॒त्रेणेति॑ द्वादश - रा॒त्रेण॑ । य॒जे॒त॒ । प्रेति॑ । ए॒व । जा॒य॒ते॒ । ब्र॒ह्म॒वा॒दिन॒ इति॑ ब्रह्म-वा॒दिनः॑ । व॒द॒न्ति॒ । अ॒ग्नि॒ष्टो॒मप्रा॑यणा॒ इत्य॑ग्निष्टो॒म - प्रा॒य॒णाः॒ । य॒ज्ञाः । अथ॑ । कस्मा᳚त् । अ॒ति॒रा॒त्र इत्य॑ति - रा॒त्रः । पूर्वः॑ । प्रेति॑ । यु॒ज्य॒ते॒ । इति॑ । चक्षु॑षी॒ इति॑ । वै । ए॒ते इति॑ । य॒ज्ञ्स्य॑ । यत् । अ॒ति॒रा॒त्रावित्य॑ति - रा॒त्रौ । क॒नीनि॑के॒ इति॑ । अ॒ग्नि॒ष्टो॒मावित्य॑ग्नि- स्तो॒मौ । यत् । \textbf{  34} \newline
                  \newline
                                \textbf{ TS 7.2.9.2} \newline
                  अ॒ग्नि॒ष्टो॒ममित्य॑ग्नि-स्तो॒मम् । पूर्व᳚म् । प्र॒यु॒ञ्जी॒रन्निति॑ प्र-यु॒ञ्जी॒रन्न् । ब॒हि॒द्‌र्धेति॑ बहिः - धा । क॒नीनि॑के॒ इति॑ । द॒द्ध्युः॒ । तस्मा᳚त् । अ॒ति॒रा॒त्र इत्य॑ति - रा॒त्रः । पूर्वः॑ । प्रेति॑ । यु॒ज्य॒ते॒ । चक्षु॑षी॒ इति॑ । ए॒व । य॒ज्ञे । धि॒त्वा । म॒द्ध्य॒तः । क॒नीनि॑के॒ इति॑ । प्रतीति॑ । द॒ध॒ति॒ । यः । वै । गा॒य॒त्रीम् । ज्योतिः॑पक्षा॒मिति॒ ज्योतिः॑ - प॒क्षा॒म् । वेद॑ । ज्योति॑षा । भा॒सा । सु॒व॒र्गमिति॑ सुवः - गम् । लो॒कम् । ए॒ति॒ । यौ । अ॒ग्नि॒ष्टो॒मावित्य॑ग्नि - स्तो॒मौ । तौ । प॒क्षौ । ये । अन्त॑रे । अ॒ष्टौ । उ॒क्थ्याः᳚ । सः । आ॒त्मा । ए॒षा । वै । गा॒य॒त्री । ज्योतिः॑प॒क्षेति॒ ज्योतिः॑ - प॒क्षा॒ । यः । ए॒वम् । वेद॑ । ज्योति॑षा । भा॒सा । सु॒व॒र्गमिति॑ सुवः - गम् । लो॒कम् । \textbf{  35} \newline
                  \newline
                                \textbf{ TS 7.2.9.3} \newline
                  ए॒ति॒ । प्र॒जाप॑ति॒रिति॑ प्र॒जा - प॒तिः॒ । वै । ए॒षः । द्वा॒द॒श॒धेति॑ द्वादश - धा । विहि॑त॒ इति॒ वि - हि॒तः॒ । यत् । द्वा॒द॒श॒रा॒त्र इति॑ द्वादश - रा॒त्रः । यौ । अ॒ति॒रा॒त्रावित्य॑ति - रा॒त्रौ । तौ । प॒क्षौ । ये । अन्त॑रे । अ॒ष्टौ । उ॒क्थ्याः᳚ । सः । आ॒त्मा । प्र॒जाप॑ति॒रिति॑ प्र॒जा-प॒तिः॒ । वाव । ए॒षः । सन्न् । सत् । ह॒ । वै । स॒त्रेण॑ । स्पृ॒णो॒ति॒ । प्रा॒णा इति॑ प्र - अ॒नाः । वै । सत् । प्रा॒णानिति॑ प्र - अ॒नान् । ए॒व । स्पृ॒णो॒ति॒ । सर्वा॑साम् । वै । ए॒ते । प्र॒जाना॒मिति॑ प्र - जाना᳚म् । प्रा॒णैरिति॑ प्र - अ॒नैः । आ॒स॒ते॒ । ये । स॒त्रम् । आस॑ते । तस्मा᳚त् । पृ॒च्छ॒न्ति॒ । किम् । ए॒ते । स॒त्रिणः॑ । इति॑ । प्रि॒यः । प्र॒जाना॒मिति॑ प्र-जाना᳚म् ( ) । उत्थि॑त॒ इत्युत् - स्थि॒तः॒ । भ॒व॒ति॒ । यः । ए॒वम् । वेद॑ ॥ \textbf{  36} \newline
                  \newline
                      (अ॒ग्नि॒ष्टो॒मौ यथ् - सु॑व॒र्गं ॅलो॒कं - प्रि॒यः प्र॒जानां॒ - पञ्च॑ च)  \textbf{(A9)} \newline \newline
                                \textbf{ TS 7.2.10.1} \newline
                  न । वै । ए॒षः । अ॒न्यतो॑वैश्वानर॒ इत्य॒न्यतः॑ - वै॒श्वा॒न॒रः॒ । सु॒व॒र्गायेति॑ सुवः - गाय॑ । लो॒काय॑ । प्रेति॑ । अ॒भ॒व॒त् । ऊ॒द्‌र्ध्वः । ह॒ । वै । ए॒षः । आत॑त॒ इत्या - त॒तः॒ । आ॒सी॒त् । ते । दे॒वाः । ए॒तम् । वै॒श्वा॒न॒रम् । परीति॑ । औ॒ह॒न्न् । सु॒व॒र्गस्येति॑ सुवः - गस्य॑ । लो॒कस्य॑ । प्रभू᳚त्या॒ इति॒ प्र-भू॒त्यै॒ । ऋ॒तवः॑ । वै । ए॒तेन॑ । प्र॒जाप॑ति॒मिति॑ प्र॒जा-प॒ति॒म् । अ॒या॒ज॒य॒न्न् । तेषु॑ । आ॒द्‌र्ध्नो॒त् । अधीति॑ । तत् । ऋ॒द्ध्नोति॑ । ह॒ । वै । ऋ॒त्विक्षु॑ । यः । ए॒वम् । वि॒द्वान् । द्वा॒द॒शा॒हेनेति॑ द्वादश - अ॒हेन॑ । यज॑ते । ते । अ॒स्मि॒न्न् । ऐ॒च्छ॒न्त॒ । सः । रस᳚म् । अह॑ । व॒स॒न्ताय॑ । प्रेति॑ । अय॑च्छत् । \textbf{  37} \newline
                  \newline
                                \textbf{ TS 7.2.10.2} \newline
                  यव᳚म् । ग्री॒ष्माय॑ । ओष॑धीः । व॒र्॒.षाभ्यः॑ । व्री॒हीन् । श॒रदे᳚ । मा॒ष॒ति॒लाविति॑ माष - ति॒लौ । हे॒म॒न्त॒शि॒शि॒राभ्या॒मिति॑ हेमन्त - शि॒शि॒राभ्या᳚म् । तेन॑ । इन्द्र᳚म् । प्र॒जाप॑ति॒रिति॑ प्र॒जा-प॒तिः॒ । अ॒या॒ज॒य॒त् । ततः॑ । वै । इन्द्रः॑ । इन्द्रः॑ । अ॒भ॒व॒त् । तस्मा᳚त् । आ॒हुः॒ । आ॒नु॒जा॒व॒रस्येत्या॑नु - जा॒व॒रस्य॑ । य॒ज्ञ्ः । इति॑ । सः । हि । ए॒तेन॑ । अग्रे᳚ । अय॑जत । ए॒षः । ह॒ । वै । कु॒णप᳚म् । अ॒त्ति॒ । यः । स॒त्रे । प्र॒ति॒गृ॒ह्णातीति॑ प्रति - गृ॒ह्णाति॑ । पु॒रु॒ष॒कु॒ण॒पमिति॑ पुरुष - कु॒ण॒पम् । अ॒श्व॒कु॒ण॒पमित्य॑श्व - कु॒ण॒पम् । गौः । वै । अन्न᳚म् । येन॑ । पात्रे॑ण । अन्न᳚म् । बिभ्र॑ति । यत् । तत् । न । नि॒र्णेनि॑ज॒तीति॑ निः - नेनि॑जति । ततः॑ । अधीति॑ । \textbf{  38} \newline
                  \newline
                                \textbf{ TS 7.2.10.3} \newline
                  मल᳚म् । जा॒य॒ते॒ । एकः॑ । ए॒व । य॒जे॒त॒ । एकः॑ । हि । प्र॒जाप॑ति॒रिति॑ प्र॒जा - प॒तिः॒ । आद्‌र्ध्नो᳚त् । द्वाद॑श । रात्रीः᳚ । दी॒क्षि॒तः । स्या॒त् । द्वाद॑श । मासाः᳚ । सं॒ॅव॒थ्स॒र इति॑ सं - व॒थ्स॒रः । सं॒ॅव॒थ्स॒र इति॑ सं - व॒थ्स॒रः । प्र॒जाप॑ति॒रिति॑ प्र॒जा - प॒तिः॒ । प्र॒जाप॑ति॒रिति॑ प्र॒जा - प॒तिः॒ । वाव । ए॒षः । ए॒षः । ह॒ । तु । वै । जा॒य॒ते॒ । यः । तप॑सः । अधीति॑ । जाय॑ते । च॒तु॒द्‌र्धेति॑ चतुः - धा । वै । ए॒ताः । ति॒स्रस्ति॑स्र॒ इति॑ ति॒स्रः-ति॒स्रः॒ । रात्र॑यः । यत् । द्वाद॑श । उ॒प॒सद॒ इत्यु॑प - सदः॑ । याः । प्र॒थ॒माः । य॒ज्ञ्म् । ताभिः॑ । समिति॑ । भ॒र॒ति॒ । याः । द्वि॒तीयाः᳚ । य॒ज्ञ्म् । ताभिः॑ । एति॑ । र॒भ॒ते॒ । \textbf{  39} \newline
                  \newline
                                \textbf{ TS 7.2.10.4} \newline
                  याः । तृ॒तीयाः᳚ । पात्रा॑णि । ताभिः॑ । निरिति॑ । ने॒नि॒क्ते॒ । याः । च॒तु॒र्थीः । अपीति॑ । ताभिः॑ । आ॒त्मान᳚म् । अ॒न्त॒र॒तः । शु॒न्ध॒ते॒ । यः । वै । अ॒स्य॒ । प॒शुम् । अत्ति॑ । माꣳ॒॒सम् । सः । अ॒त्ति॒ । यः । पु॒रो॒डाश᳚म् । म॒स्तिष्क᳚म् । सः । यः । प॒रि॒वा॒पमिति॑ परि - वा॒पम् । पुरी॑षम् । सः । यः । आज्य᳚म् । म॒ज्जान᳚म् । सः । यः । सोम᳚म् । स्वेद᳚म् । सः । अपीति॑ । ह॒ । वै । अ॒स्य॒ । शी॒र्.॒ष॒ण्याः᳚ । नि॒ष्पद॒ इति॑ निः - पदः॑ । प्रतीति॑ । गृ॒ह्णा॒ति॒ । यः । द्वा॒द॒शा॒ह इति॑ द्वादश-अ॒हे । प्र॒ति॒गृ॒ह्णातीति॑ प्रति - गृ॒ह्णाति॑ । तस्मा᳚त् । द्वा॒द॒शा॒हेनेति॑ द्वादश - अ॒हेन॑ ( ) । न । याज्य᳚म् । पा॒प्मनः॑ । व्यावृ॑त्त्या॒ इति॑ वि - आवृ॑त्त्यै ॥ \textbf{  40} \newline
                  \newline
                      (अय॑च्छ॒ - दधि॑ - रभते - द्वादशा॒हेन॑ - च॒त्वारि॑ च)  \textbf{(A10)} \newline \newline
                                \textbf{ TS 7.2.11.1} \newline
                  एक॑स्मै । स्वाहा᳚ । द्वाभ्या᳚म् । स्वाहा᳚ । त्रि॒भ्य इति॑ त्रि - भ्यः । स्वाहा᳚ । च॒तुर्भ्य॒ इति॑ च॒तुः - भ्यः॒ । स्वाहा᳚ । प॒ञ्चभ्य॒ इति॑ प॒ञ्च - भ्यः॒ । स्वाहा᳚ । ष॒ड्भ्य इति॑ षट्- भ्यः । स्वाहा᳚ । स॒प्तभ्य॒ इति॑ स॒प्त - भ्यः॒ । स्वाहा᳚ । अ॒ष्टा॒भ्यः । स्वाहा᳚ । न॒वभ्य॒ इति॑ न॒व - भ्यः॒ । स्वाहा᳚ । द॒शभ्य॒ इति॑ द॒श - भ्यः॒ । स्वाहा᳚ । ए॒का॒द॒शभ्य॒ इत्ये॑काद॒श - भ्यः॒ । स्वाहा᳚ । द्वा॒द॒शभ्य॒ इति॑ द्वाद॒श - भ्यः॒ । स्वाहा᳚ । त्र॒यो॒द॒शभ्य॒ इति॑ त्रयोद॒श-भ्यः॒ । स्वाहा᳚ । च॒तु॒र्द॒शभ्य॒ इति॑ चतुर्द॒श - भ्यः॒ । स्वाहा᳚ । प॒ञ्च॒द॒शभ्य॒ इति॑ पञ्चद॒श - भ्यः॒ । स्वाहा᳚ । षो॒ड॒शभ्य॒ इति॑ षोड॒श - भ्यः॒ । स्वाहा᳚ । स॒प्त॒द॒शभ्य॒ इति॑ सप्तद॒श - भ्यः॒ । स्वाहा᳚ । अ॒ष्टा॒द॒शभ्य॒ इत्य॑ष्टाद॒श - भ्यः॒ । स्वाहा᳚ । एका᳚त् । न । विꣳ॒॒श॒त्यै । स्वाहा᳚ । नव॑विꣳशत्या॒ इति॒ नव॑ - विꣳ॒॒श॒त्यै॒ । स्वाहा᳚ । एका᳚त् । न । च॒त्वा॒रिꣳ॒॒शते᳚ । स्वाहा᳚ । नव॑चत्वारिꣳशत॒ इति॒ नव॑ - च॒त्वा॒रिꣳ॒॒श॒ते॒ । स्वाहा᳚ । एका᳚त् । न ( ) । ष॒ष्ट्यै । स्वाहा᳚ । नव॑षष्ट्या॒ इति॒ नव॑ - ष॒ष्ट्यै॒ । स्वाहा᳚ । एका᳚त् । न । अ॒शी॒त्यै । स्वाहा᳚ । नवा॑शीत्या॒ इति॒ नव॑ - अ॒शी॒त्यै॒ । स्वाहा᳚ । एका᳚त् । न । श॒ताय॑ । स्वाहा᳚ । श॒ताय॑ । स्वाहा᳚ । द्वाभ्या᳚म् । श॒ताभ्या᳚म् । स्वाहा᳚ । सर्व॑स्मै । स्वाहा᳚ ॥ \textbf{  41} \newline
                  \newline
                      (नव॑चत्वारिꣳशते॒ स्वाहै-का॒न्नैक॑विꣳशतिश्च)  \textbf{(A11)} \newline \newline
                                \textbf{ TS 7.2.12.1} \newline
                  एक॑स्मै । स्वाहा᳚ । त्रि॒भ्य इति॑ त्रि - भ्यः । स्वाहा᳚ । प॒ञ्चभ्य॒ इति॑ प॒ञ्च - भ्यः॒ । स्वाहा᳚ । स॒प्तभ्य॒ इति॑ स॒प्त - भ्यः॒ । स्वाहा᳚ । न॒वभ्य॒ इति॑ न॒व - भ्यः॒ । स्वाहा᳚ । ए॒का॒द॒शभ्य॒ इत्ये॑काद॒श-भ्यः॒ । स्वाहा᳚ । त्र॒यो॒द॒शभ्य॒ इति॑ त्रयोद॒श - भ्यः॒ । स्वाहा᳚ । प॒ञ्च॒द॒शभ्य॒ इति॑ पञ्चद॒श - भ्यः॒ । स्वाहा᳚ । स॒प्त॒द॒शभ्य॒ इति॑ सप्तद॒श - भ्यः॒ । स्वाहा᳚ । एका᳚त् । न । विꣳ॒॒श॒त्यै । स्वाहा᳚ । नव॑विꣳशत्या॒ इति॒ नव॑ - विꣳ॒॒श॒त्यै॒ । स्वाहा᳚ । एका᳚त् । न । च॒त्वा॒रिꣳ॒॒शते᳚ । स्वाहा᳚ । नव॑चत्वारिꣳशत॒ इति॒ नव॑-च॒त्वा॒रिꣳ॒॒श॒ते॒ । स्वाहा᳚ । एका᳚त् । न । ष॒ष्ट्यै । स्वाहा᳚ । नव॑षष्ट्या॒ इति॒ नव॑-ष॒ष्ट्यै॒ । स्वाहा᳚ । एका᳚त् । न । अ॒शी॒त्यै । स्वाहा᳚ । नवा॑शीत्या॒ इति॒ नव॑ - अ॒शी॒त्यै॒ । स्वाहा᳚ । एका᳚त् । न । श॒ताय॑ । स्वाहा᳚ । श॒ताय॑ । स्वाहा᳚ । सर्व॑स्मै । स्वाहा᳚ ॥ \textbf{  42} \newline
                  \newline
                      (एक॑स्मै त्रि॒भ्यः - प॑ञ्चा॒शत्)  \textbf{(A12)} \newline \newline
                                \textbf{ TS 7.2.13.1} \newline
                  द्वाभ्या᳚म् । स्वाहा᳚ । च॒तुर्भ्य॒ इति॑ च॒तुः - भ्यः॒ । स्वाहा᳚ । ष॒ड्भ्य इति॑ षट्- भ्यः । स्वाहा᳚ । अ॒ष्टा॒भ्यः । स्वाहा᳚ । द॒शभ्य॒ इति॑ द॒श-भ्यः॒ । स्वाहा᳚ । द्वा॒द॒शभ्य॒ इति॑ द्वाद॒श - भ्यः॒ । स्वाहा᳚ । च॒तु॒र्द॒शभ्य॒ इति॑ चतुर्द॒श - भ्यः॒ । स्वाहा᳚ । षो॒ड॒शभ्य॒ इति॑ षोड॒श - भ्यः॒ । स्वाहा᳚ । अ॒ष्टा॒द॒शभ्य॒ इत्य॑ष्टाद॒श - भ्यः॒ । स्वाहा᳚ । विꣳ॒॒श॒त्यै । स्वाहा᳚ । अ॒ष्टान॑वत्या॒ इत्य॒ष्टा - न॒व॒त्यै॒ । स्वाहा᳚ । श॒ताय॑ । स्वाहा᳚ । सर्व॑स्मै । स्वाहा᳚ ॥ \textbf{  43} \newline
                  \newline
                      (द्वाभ्या॑म॒ष्टान॑वत्यै॒ - षड्विꣳ॑शतिः)  \textbf{(A13)} \newline \newline
                                \textbf{ TS 7.2.14.1} \newline
                  त्रि॒भ्य इति॑ त्रि - भ्यः । स्वाहा᳚ । प॒ञ्चभ्य॒ इति॑ प॒ञ्च - भ्यः॒ । स्वाहा᳚ । स॒प्तभ्य॒ इति॑ स॒प्त-भ्यः॒ । स्वाहा᳚ । न॒वभ्य॒ इति॑ न॒व-भ्यः॒ । स्वाहा᳚ । ए॒का॒द॒शभ्य॒ इत्ये॑काद॒श - भ्यः॒ । स्वाहा᳚ । त्र॒यो॒द॒शभ्य॒ इति॑ त्रयोद॒श - भ्यः॒ । स्वाहा᳚ । प॒ञ्च॒द॒शभ्य॒ इति॑ पञ्चद॒श - भ्यः॒ । स्वाहा᳚ । स॒प्त॒द॒शभ्य॒ इति॑ सप्तद॒श - भ्यः॒ । स्वाहा᳚ । एका᳚त् । न । विꣳ॒॒श॒त्यै । स्वाहा᳚ । नव॑विꣳशत्या॒ इति॒ नव॑ - विꣳ॒॒श॒त्यै॒ । स्वाहा᳚ । एका᳚त् । न । च॒त्वा॒रिꣳ॒॒शते᳚ । स्वाहा᳚ । नव॑चत्वारिꣳशत॒ इति॒ नव॑ - च॒त्वा॒रिꣳ॒॒श॒ते॒ । स्वाहा᳚ । एका᳚त् । न । ष॒ष्ट्यै । स्वाहा᳚ । नव॑षष्ट्या॒ इति॒ नव॑ - ष॒ष्ट्यै॒ । स्वाहा᳚ । एका᳚त् । न । अ॒शी॒त्यै । स्वाहा᳚ । नवा॑शीत्या॒ इति॒ नव॑ - अ॒शी॒त्यै॒ । स्वाहा᳚ । एका᳚त् । न । श॒ताय॑ । स्वाहा᳚ । श॒ताय॑ । स्वाहा᳚ । सर्व॑स्मै । स्वाहा᳚ ॥ \textbf{  44} \newline
                  \newline
                      (त्रि॒भ्यो᳚ - ऽष्टाचत्वारिꣳ॒॒शत्)  \textbf{(A14)} \newline \newline
                                \textbf{ TS 7.2.15.1} \newline
                  च॒तुर्भ्य॒ इति॑ च॒तुः - भ्यः॒ । स्वाहा᳚ । अ॒ष्टा॒भ्यः । स्वाहा᳚ । द्वा॒द॒शभ्य॒ इति॑ द्वाद॒श - भ्यः॒ । स्वाहा᳚ । षो॒ड॒शभ्य॒ इति॑ षोड॒श - भ्यः॒ । स्वाहा᳚ । विꣳ॒॒श॒त्यै । स्वाहा᳚ । षण्ण॑वत्या॒ इति॒ षट्-न॒व॒त्यै॒ । स्वाहा᳚ । श॒ताय॑ । स्वाहा᳚ । सर्व॑स्मै । स्वाहा᳚ ॥ \textbf{  45} \newline
                  \newline
                      (च॒तुर्भ्यः॒ षण्ण॑वत्यै॒ - षोड॑श)  \textbf{(A15)} \newline \newline
                                \textbf{ TS 7.2.16.1} \newline
                  प॒ञ्चभ्य॒ इति॑ प॒ञ्च - भ्यः॒ । स्वाहा᳚ । द॒शभ्य॒ इति॑ द॒श - भ्यः॒ । स्वाहा᳚ । प॒ञ्च॒द॒शभ्य॒ इति॑ पञ्चद॒श - भ्यः॒ । स्वाहा᳚ । विꣳ॒॒श॒त्यै । स्वाहा᳚ । पञ्च॑नवत्या॒ इति॒ पञ्च॑ - न॒व॒त्यै॒ । स्वाहा᳚ । श॒ताय॑ । स्वाहा᳚ । सर्व॑स्मै । स्वाहा᳚ ॥ \textbf{  46} \newline
                  \newline
                      (प॒ञ्चभ्यः॒ पञ्च॑नवत्यै॒ - चतु॑र्दश)  \textbf{(A16)} \newline \newline
                                \textbf{ TS 7.2.17.1} \newline
                  द॒शभ्य॒ इति॑ द॒श - भ्यः॒ । स्वाहा᳚ । विꣳ॒॒श॒त्यै । स्वाहा᳚ । त्रिꣳ॒॒शते᳚ । स्वाहा᳚ । च॒त्वा॒रिꣳ॒॒शते᳚ । स्वाहा᳚ । प॒ञ्चा॒शते᳚ । स्वाहा᳚ । ष॒ष्ट्यै । स्वाहा᳚ । स॒प्त॒त्यै । स्वाहा᳚ । अ॒शी॒त्यै । स्वाहा᳚ । न॒व॒त्यै । स्वाहा᳚ । श॒ताय॑ । स्वाहा᳚ । सर्व॑स्मै । स्वाहा᳚ ॥ \textbf{  47} \newline
                  \newline
                      (द॒शभ्यो॒ - द्वाविꣳ॑शतिः)  \textbf{(A17)} \newline \newline
                                \textbf{ TS 7.2.18.1} \newline
                  विꣳ॒॒श॒त्यै । स्वाहा᳚ । च॒त्वा॒रिꣳ॒॒शते᳚ । स्वाहा᳚ । ष॒ष्ट्यै । स्वाहा᳚ । अ॒शी॒त्यै । स्वाहा᳚ । श॒ताय॑ । स्वाहा᳚ । सर्व॑स्मै । स्वाहा᳚ ॥ \textbf{  48} \newline
                  \newline
                      (विꣳ॒॒श॒त्यै - द्वाद॑श)  \textbf{(A18)} \newline \newline
                                \textbf{ TS 7.2.19.1} \newline
                  प॒ञ्चा॒शते᳚ । स्वाहा᳚ । श॒ताय॑ । स्वाहा᳚ । द्वाभ्या᳚म् । श॒ताभ्या᳚म् । स्वाहा᳚ । त्रि॒भ्य इति॑ त्रि - भ्यः । श॒तेभ्यः॑ । स्वाहा᳚ । च॒तुर्भ्य॒ इति॑ च॒तुः - भ्यः॒ । श॒तेभ्यः॑ । स्वाहा᳚ । प॒ञ्चभ्य॒ इति॑ प॒ञ्च-भ्यः॒ । श॒तेभ्यः॑ । स्वाहा᳚ । ष॒ड्भ्य इति॑ षट्- भ्यः । श॒तेभ्यः॑ । स्वाहा᳚ । स॒प्तभ्य॒ इति॑ स॒प्त-भ्यः॒ । श॒तेभ्यः॑ । स्वाहा᳚ । अ॒ष्टा॒भ्यः । श॒तेभ्यः॑ । स्वाहा᳚ । न॒वभ्य॒ इति॑ न॒व - भ्यः॒ । श॒तेभ्यः॑ । स्वाहा᳚ । स॒हस्रा॑य । स्वाहा᳚ । सर्व॑स्मै । स्वाहा᳚ ॥ \textbf{  49 } \newline
                  \newline
                      (प॒ञ्चा॒शते॒ - द्वात्रिꣳ॑शत्)  \textbf{(A19)} \newline \newline
                                \textbf{ TS 7.2.20.1} \newline
                  श॒ताय॑ । स्वाहा᳚ । स॒हस्रा॑य । स्वाहा᳚ । अ॒युता॑य । स्वाहा᳚ । नि॒युता॒येति॑ नि - युता॑य । स्वाहा᳚ । प्र॒युता॒येति॑ प्र - युता॑य । स्वाहा᳚ । अर्बु॑दाय । स्वाहा᳚ । न्य॑र्बुदा॒येति॒ नि - अ॒र्बु॒दा॒य॒ । स्वाहा᳚ । स॒मु॒द्राय॑ । स्वाहा᳚ । मद्ध्या॑य । स्वाहा᳚ । अन्ता॑य । स्वाहा᳚ । प॒रा॒द्‌र्धायेति॑ पर-अ॒द्‌र्धाय॑ । स्वाहा᳚ । उ॒षसे᳚ । स्वाहा᳚ । व्यु॑ष्ट्या॒ इति॒ वि - उ॒ष्ट्यै॒ । स्वाहा᳚ । उ॒दे॒ष्य॒त इत्यु॑त् - ए॒ष्य॒ते । स्वाहा᳚ । उ॒द्य॒त इत्यु॑त् - य॒ते । स्वाहा᳚ । उदि॑ता॒येत्युत् - इ॒ता॒य॒ । स्वाहा᳚ । सु॒व॒र्गायेति॑ सुवः-गाय॑ । स्वाहा᳚ । लो॒काय॑ । स्वाहा᳚ । सर्व॑स्मै । स्वाहा᳚ ॥ \textbf{  50 } \newline
                  \newline
                      (श॒ताया॒-ऽष्टात्रिꣳ॑शत्)  \textbf{(A20)} \newline \newline
\textbf{praSna korvai with starting padams of 1 to 20 anuvAkams :-} \newline
(सा॒ध्याः ष॑ड् रा॒त्रं - कु॑सुरु॒बिन्दः॑ सप्तरा॒त्रं - बृह॒स्पति॑रष्टरा॒त्रं - प्र॒जाप॑ति॒स्ताः क्षुध॑न्नवरा॒त्रं - प्र॒जाप॑तिरकामयत॒ दश॑होतारात्र - मृ॒तव॑ - ऐन्द्रवाय॒वाग्रा᳚न् - गाय॒त्रो वै - प्र॒जाप॑तिः॒ स द्वा॑दशरा॒त्रं - न वा -एक॑स्मा॒ - एक॑स्मै॒ - द्वाभ्यां᳚ - त्रि॒भ्यः - च॒तुर्भ्यः॑ - प॒ञ्चभ्यो॑ - द॒शभ्यो॑ - विꣳश॒त्यै - प॑ञ्चा॒शते॑ - श॒ताय॑ - विꣳश॒तिः ) \newline

\textbf{korvai with starting padams of1, 11, 21 series of pa~jcAtis :-} \newline
(सा॒ध्या - अ॑स्मा इ॒मे लो॒का - गा॑य॒त्रं - ॅवै तृ॒तीय॒ - मेक॑स्मै - पञ्चा॒शत् ) \newline

\textbf{first and last padam of second praSnam of 7th kANDam} \newline
(सा॒ध्याः - सर्व॑स्मै॒ स्वाहा᳚ ) \newline 


॥ हरिः॑ ॐ ॥

॥ कृष्ण यजुर्वेदीय तैत्तिरीय संहितायां सप्तमकाण्डे द्वितीयः प्रश्नः समाप्तः ॥
================================== \newline
\pagebreak
\pagebreak
        


\end{document}
