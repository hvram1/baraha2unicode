\documentclass[17pt]{extarticle}
\usepackage{babel}
\usepackage{fontspec}
\usepackage{polyglossia}
\usepackage{extsizes}



\setmainlanguage{sanskrit}
\setotherlanguages{english} %% or other languages
\setlength{\parindent}{0pt}
\pagestyle{myheadings}
\newfontfamily\devanagarifont[Script=Devanagari]{AdishilaVedic}


\newcommand{\VAR}[1]{}
\newcommand{\BLOCK}[1]{}




\begin{document}
\begin{titlepage}
    \begin{center}
 
\begin{sanskrit}
    { \Large
    ॐ नमः परमात्मने, श्री महागणपतये नमः, श्री गुरुभ्यो नमः
ह॒रिः॒ ॐ 
    }
    \\
    \vspace{2.5cm}
    \mbox{ \Huge
    6.6       षष्ठकाण्डे षष्ठः प्रश्नः - सोममन्त्रब्राह्मणनिरूपणं   }
\end{sanskrit}
\end{center}

\end{titlepage}
\tableofcontents

ॐ नमः परमात्मने, श्री महागणपतये नमः, श्री गुरुभ्यो नमः
ह॒रिः॒ ॐ \newline
6.6       षष्ठकाण्डे षष्ठः प्रश्नः - सोममन्त्रब्राह्मणनिरूपणं \newline

\addcontentsline{toc}{section}{ 6.6       षष्ठकाण्डे षष्ठः प्रश्नः - सोममन्त्रब्राह्मणनिरूपणं}
\markright{ 6.6       षष्ठकाण्डे षष्ठः प्रश्नः - सोममन्त्रब्राह्मणनिरूपणं \hfill https://www.vedavms.in \hfill}
\section*{ 6.6       षष्ठकाण्डे षष्ठः प्रश्नः - सोममन्त्रब्राह्मणनिरूपणं }
                                \textbf{ TS 6.6.1.1} \newline
                  सु॒व॒र्गायेति॑ सुवः - गाय॑ । वै । ए॒तानि॑ । लो॒काय॑ । हू॒य॒न्ते॒ । यत् । दा॒क्षि॒णानि॑ । द्वाभ्या᳚म् । गार्.ह॑पत्य॒ इति॒ गार्.ह॑ - प॒त्ये॒ । जु॒हो॒ति॒ । द्वि॒पादिति॑ द्वि - पात् । यज॑मानः । प्रति॑ष्ठित्या॒ इति॒ प्रति॑ - स्थि॒त्यै॒ । आग्नी᳚द्ध्र॒ इत्याग्नि॑ - इ॒द्ध्रे॒ । जु॒हो॒ति॒ । अ॒न्तरि॑क्षे । ए॒व । एति॑ । क्र॒म॒ते॒ । सदः॑ । अ॒भि । एति॑ । ए॒ति॒ । सु॒व॒र्गमिति॑ सुवः - गम् । ए॒व । ए॒न॒म् । लो॒कम् । ग॒म॒य॒ति॒ । सौ॒रीभ्या᳚म् । ऋ॒ग्भ्यामित्यृ॑क् - भ्याम् । गार्.ह॑पत्य॒ इति॒ गार्.ह॑-प॒त्ये॒ । जु॒हो॒ति॒ । अ॒मुम् । ए॒व । ए॒न॒म् । लो॒कम् । स॒मारो॑हय॒तीति॑ सं - आरो॑हयति । नय॑व॒त्येति॒ नय॑ - व॒त्या॒ । ऋ॒चा । आग्नी᳚द्ध्र॒ इत्याग्नि॑ - इ॒द्ध्रे॒ । जु॒हो॒ति॒ । सु॒व॒र्गस्येति॑ सुवः - गस्य॑ । लो॒कस्य॑ । अ॒भिनी᳚त्य॒ इत्य॒भि -नी॒त्यै॒ । दिव᳚म् । ग॒च्छ॒ । सुवः॑ । प॒त॒ । इति॑ । हिर॑ण्यम् । \textbf{  1} \newline
                  \newline
                                \textbf{ TS 6.6.1.2} \newline
                  हु॒त्वा । उदिति॑ । गृ॒ह्णा॒ति॒ । सु॒व॒र्गमिति॑ सुवः - गम् । ए॒व । ए॒न॒म् । लो॒कम् । ग॒म॒य॒ति॒ । रू॒पेण॑ । वः॒ । रू॒पम् । अ॒भि । एति॑ । ए॒मि॒ । इति॑ । आ॒ह॒ । रू॒पेण॑ । हि । आ॒सा॒म् । रू॒पम् । अ॒भि । एति॑ । एति॑ । यत् । हिर॑ण्येन । तु॒थः । वः॒ । वि॒श्ववे॑दा॒ इति॑ वि॒श्व - वे॒दाः॒ । वीति॑ । भ॒ज॒तु॒ । इति॑ । आ॒ह॒ । तु॒थः । ह॒ । स्म॒ । वै । वि॒श्ववे॑दा॒ इति॑ वि॒श्व - वे॒दाः॒ । दे॒वाना᳚म् । दक्षि॑णाः । वीति॑ । भ॒ज॒ति॒ । तेन॑ । ए॒व । ए॒नाः॒ । वीति॑ । भ॒ज॒ति॒ । ए॒तत् । ते॒ । अ॒ग्ने॒ । राधः॑ । \textbf{  2} \newline
                  \newline
                                \textbf{ TS 6.6.1.3} \newline
                  एति॑ । ए॒ति॒ । सोम॑च्युत॒मिति॒ सोम॑ - च्यु॒त॒म् । इति॑ । आ॒ह॒ । सोम॑च्युत॒मिति॒ सोम॑ - च्यु॒त॒म् । हि । अ॒स्य॒ । राधः॑ । एति॑ । एति॑ । तत् । मि॒त्रस्य॑ । प॒था । न॒य॒ । इति॑ । आ॒ह॒ । शान्त्यै᳚ । ऋ॒तस्य॑ । प॒था । प्रेति॑ । इ॒त॒ । च॒न्द्रद॑क्षिणा॒ इति॑ च॒न्द्र - द॒क्षि॒णाः॒ । इति॑ । आ॒ह॒ । स॒त्यम् । वै । ऋ॒तम् । स॒त्येन॑ । ए॒व । ए॒नाः॒ । ऋ॒तेन॑ । वीति॑ । भ॒ज॒ति॒ । य॒ज्ञ्स्य॑ । प॒था । सु॒वि॒ता । नय॑न्तीः । इति॑ । आ॒ह॒ । य॒ज्ञ्स्य॑ । हि । ए॒ताः । प॒था । यन्ति॑ । यत् । दक्षि॑णाः । ब्रा॒ह्म॒णम् । अ॒द्य । रा॒द्ध्या॒स॒म् । \textbf{  3 } \newline
                  \newline
                                \textbf{ TS 6.6.1.4} \newline
                  ऋषि᳚म् । आ॒र्॒.षे॒यम् । इति॑ । आ॒ह॒ । ए॒षः । वै । ब्रा॒ह्म॒णः । ऋषिः॑ । आ॒र्॒.षे॒यः । यः । शु॒श्रु॒वान् । तस्मा᳚त् । ए॒वम् । आ॒ह॒ । वीति॑ । सुवः॑ । पश्य॑ । वीति॑ । अ॒न्तरि॑क्षम् । इति॑ । आ॒ह॒ । सु॒व॒र्गमिति॑ सुवः-गम् । ए॒व । ए॒न॒म् । लो॒कम् । ग॒म॒य॒ति॒ । यत॑स्व । स॒द॒स्यैः᳚ । इति॑ । आ॒ह॒ । मि॒त्र॒त्वायेति॑ मित्र - त्वाय॑ । अ॒स्मद्दा᳚त्रा॒ इत्य॒स्मत् - दा॒त्राः॒ । दे॒व॒त्रेति॑ देव - त्रा । ग॒च्छ॒त॒ । मधु॑मती॒रिति॒ मधु॑ - म॒तीः॒ । प्र॒दा॒तार॒मिति॑ प्र - दा॒तार᳚म् । एति॑ । वि॒श॒त॒ । इति॑ । आ॒ह॒ । व॒यम् । इ॒ह । प्र॒दा॒तार॒ इति॑ प्र - दा॒तारः॑ । स्मः । अ॒स्मान् । अ॒मुत्र॑ । मधु॑मती॒रिति॒ मधु॑ - म॒तीः॒ । एति॑ । वि॒श॒त॒ । इति॑ । \textbf{  4} \newline
                  \newline
                                \textbf{ TS 6.6.1.5} \newline
                  वाव । ए॒तत् । आ॒ह॒ । हिर॑ण्यम् । द॒दा॒ति॒ । ज्योतिः॑ । वै । हिर॑ण्यम् । ज्योतिः॑ । ए॒व । पु॒रस्ता᳚त् । ध॒त्ते॒ । सु॒व॒र्गस्येति॑ सुवः - गस्य॑ । लो॒कस्य॑ । अनु॑ख्यात्या॒ इत्यनु॑ - ख्या॒त्यै॒ । अ॒ग्नीध॒ इत्य॑ग्नि - इधे᳚ । द॒दा॒ति॒ । अ॒ग्निमु॑खा॒नित्य॒ग्नि - मु॒खा॒न् । ए॒व । ऋ॒तून् । प्री॒णा॒ति॒ । ब्र॒ह्मणे᳚ । द॒दा॒ति॒ । प्रसू᳚त्या॒ इति॒ प्र-सू॒त्यै॒ । होत्रे᳚ । द॒दा॒ति॒ । आ॒त्मा । वै । ए॒षः । य॒ज्ञ्स्य॑ । यत् । होता᳚ । आ॒त्मान᳚म् । ए॒व । य॒ज्ञ्स्य॑ । दक्षि॑णाभिः । समिति॑ । अ॒द्‌र्ध॒य॒ति॒ ॥ \textbf{  5} \newline
                  \newline
                      (हिर॑ण्यꣳ॒॒ - राधो॑ - राध्यास - म॒मुत्र॒ मधु॑मती॒रा वि॑श॒तेत्य॒ - ष्टात्रिꣳ॑शच्च)  \textbf{(A1)} \newline \newline
                                \textbf{ TS 6.6.2.1} \newline
                  स॒मि॒ष्ट॒य॒जूꣳषीति॑ समिष्ट - य॒जूꣳषि॑ । जु॒हो॒ति॒ । य॒ज्ञ्स्य॑ । समि॑ष्ट्या॒ इति॒ सं - इ॒ष्ट्यै॒ । यत् । वै । य॒ज्ञ्स्य॑ । क्रू॒रम् । यत् । विलि॑ष्ट॒मिति॒ वि - लि॒ष्ट॒म् । यत् । अ॒त्येतीत्य॑ति - एति॑ । यत् । न । अ॒त्येतीत्य॑ति - एति॑ । यत् । अ॒ति॒क॒रोतीत्य॑ति - क॒रोति॑ । यत् । न । अपीति॑ । क॒रोति॑ । तत् । ए॒व । तैः । प्री॒णा॒ति॒ । नव॑ । जु॒हो॒ति॒ । नव॑ । वै । पुरु॑षे । प्रा॒णा इति॑ प्र - अ॒नाः । पुरु॑षेण । य॒ज्ञ्ः । सम्मि॑त॒ इति॒ सं - मि॒तः॒ । यावान्॑ । ए॒व । य॒ज्ञ्ः । तम् । प्री॒णा॒ति॒ । षट् । ऋग्मि॑याणि । जु॒हो॒ति॒ । षट् । वै । ऋ॒तवः॑ । ऋ॒तून् । ए॒व । प्री॒णा॒ति॒ । त्रीणि॑ । यजूꣳ॑षि । \textbf{  6} \newline
                  \newline
                                \textbf{ TS 6.6.2.2} \newline
                  त्रयः॑ । इ॒मे । लो॒काः । इ॒मान् । ए॒व । लो॒कान् । प्री॒णा॒ति॒ । यज्ञ्॑ । य॒ज्ञ्म् । ग॒च्छ॒ । य॒ज्ञ्प॑ति॒मिति॑ य॒ज्ञ् - प॒ति॒म् । ग॒च्छ॒ । इति॑ । आ॒ह॒ । य॒ज्ञ्प॑ति॒मिति॑ य॒ज्ञ् - प॒ति॒म् । ए॒व । ए॒न॒म् । ग॒म॒य॒ति॒ । स्वाम् । योनि᳚म् । ग॒च्छ॒ । इति॑ । आ॒ह॒ । स्वाम् । ए॒व । ए॒न॒म् । योनि᳚म् । ग॒म॒य॒ति॒ । ए॒षः । ते॒ । य॒ज्ञ्ः । य॒ज्ञ्॒प॒त॒ इति॑ यज्ञ् - प॒ते॒ । स॒हसू᳚क्तवाक॒ इति॑ स॒हसू᳚क्त - वा॒कः॒ । सु॒वीर॒ इति॑ सु - वीरः॑ । इति॑ । आ॒ह॒ । यज॑माने । ए॒व । वी॒र्य᳚म् । द॒धा॒ति॒ । वा॒सि॒ष्ठः । ह॒ । सा॒त्य॒ह॒व्य इति॑ सात्य - ह॒व्यः । दे॒व॒भा॒गमिति॑ देव - भा॒गम् । प॒प्र॒च्छ॒ । यत् । सृञ्ज॑यान् । ब॒हु॒या॒जिन॒ इति॑ बहु - या॒जिनः॑ । अयी॑यजः । य॒ज्ञे । \textbf{  7} \newline
                  \newline
                                \textbf{ TS 6.6.2.3} \newline
                  य॒ज्ञ्म् । प्रतीति॑ । अ॒ति॒ष्ठि॒पा(3)ः । य॒ज्ञ्प॒ता(3)विति॑ य॒ज्ञ्-प॒ता(3)उ । इति॑ । सः । ह॒ । उ॒वा॒च॒ । य॒ज्ञ्प॑ता॒विति॑ य॒ज्ञ्-प॒तौ॒ । इति॑ । स॒त्यात् । वै । सृञ्ज॑याः । परेति॑ । ब॒भू॒वुः॒ । इति॑ । ह॒ । उ॒वा॒च॒ । य॒ज्ञे । वाव । य॒ज्ञ्ः । प्र॒ति॒ष्ठाप्य॒ इति॑ प्रति - स्थाप्यः॑ । आ॒सी॒त् । यज॑मानस्य । अप॑राभावा॒येत्यप॑रा - भा॒वा॒य॒ । इति॑ । देवाः᳚ । गा॒तु॒वि॒द॒ इति॑ गातु - वि॒दः॒ । गा॒तुम् । वि॒त्त्वा । गा॒तुम् । इ॒त॒ । इति॑ । आ॒ह॒ । य॒ज्ञे । ए॒व । य॒ज्ञ्म् । प्रतीति॑ । स्था॒प॒य॒ति॒ । यज॑मानस्य । अप॑राभावा॒येत्यप॑रा- भा॒वा॒य॒ ॥ \textbf{  8} \newline
                  \newline
                      (यजूꣳ॑षि - य॒ज्ञ् - एक॑चत्वारिꣳशच्च)  \textbf{(A2)} \newline \newline
                                \textbf{ TS 6.6.3.1} \newline
                  अ॒व॒भृ॒थ॒य॒जूꣳषीत्य॑वभृथ - य॒जूꣳषि॑ । जु॒हो॒ति॒ । यत् । ए॒व । अ॒र्वा॒चीन᳚म् । एक॑हायना॒दित्येक॑ - हा॒य॒ना॒त् । एनः॑ । क॒रोति॑ । तत् । ए॒व । तैः । अवेति॑ । य॒ज॒ते॒ । अ॒पः । अ॒व॒भृ॒थमित्य॑व - भृ॒थम् । अवेति॑ । ए॒ति॒ । अ॒फ्स्वित्य॑प् - सु । वै । वरु॑णः । सा॒क्षादिति॑ स - अ॒क्षात् । ए॒व । वरु॑णम् । अवेति॑ । य॒ज॒ते॒ । वर्त्म॑ना । वै । अ॒न्वित्येत्य॑नु - इत्य॑ । य॒ज्ञ्म् । रक्षाꣳ॑सि । जि॒घाꣳ॒॒स॒न्त॒ । साम्ना᳚ । प्र॒स्तो॒तेति॑ प्र - स्तो॒ता । अ॒न्ववै॒तीत्य॑नु - अवै॑ति । साम॑ । वै । र॒क्षो॒हेति॑ रक्षः - हा । रक्ष॑साम् । अप॑हत्या॒ इत्यप॑ - ह॒त्यै॒ । त्रिः । नि॒धन॒मिति॑ नि - धन᳚म् । उपेति॑ । ए॒ति॒ । त्रयः॑ । इ॒मे । लो॒काः । ए॒भ्यः । ए॒व । लो॒केभ्यः॑ । रक्षाꣳ॑सि । \textbf{  9} \newline
                  \newline
                                \textbf{ TS 6.6.3.2} \newline
                  अपेति॑ । ह॒न्ति॒ । पुरु॑षःपुरुष॒ इति॒ पुरु॑षः - पु॒रु॒षः॒ । नि॒धन॒मिति॑ नि - धन᳚म् । उपेति॑ । ए॒ति॒ । पुरु॑षःपुरुष॒ इति॒ पुरु॑षः - पु॒रु॒षः॒ । हि । र॒क्ष॒स्वी । रक्ष॑साम् । अप॑हत्या॒ इत्यप॑ - ह॒त्यै॒ । उ॒रुम् । हि । राजा᳚ । वरु॑णः । च॒कार॑ । इति॑ । आ॒ह॒ । प्रति॑ष्ठित्या॒ इति॒ प्रति॑-स्थि॒त्यै॒ । श॒तम् । ते॒ । रा॒ज॒न्न् । भि॒षजः॑ । स॒हस्र᳚म् । इति॑ । आ॒ह॒ । भे॒ष॒जम् । ए॒व । अ॒स्मै॒ । क॒रो॒ति॒ । अ॒भिष्ठि॑त॒ इत्य॒भि - स्थि॒तः॒ । वरु॑णस्य । पाशः॑ । इति॑ । आ॒ह॒ । व॒रु॒ण॒पा॒शमिति॑ वरुण - पा॒शम् । ए॒व । अ॒भीति॑ । ति॒ष्ठ॒ति॒ । ब॒र॒.हिः । अ॒भीति॑ । जु॒हो॒ति॒ । आहु॑तीना॒मित्या - हु॒ती॒ना॒म् । प्रति॑ष्ठित्या॒ इति॒ प्रति॑ - स्थि॒त्यै॒ । अथो॒ इति॑ । अ॒ग्नि॒वतीत्य॑ग्नि - वति॑ । ए॒व । जु॒हो॒ति॒ । अप॑बर्.हिष॒ इत्यप॑ - ब॒र॒.हि॒षः॒ । प्र॒या॒जानिति॑ प्र - या॒जान् । \textbf{  10} \newline
                  \newline
                                \textbf{ TS 6.6.3.3} \newline
                  य॒ज॒ति॒ । प्र॒जा इति॑ प्र - जाः । वै । ब॒र्॒.हिः । प्र॒जा इति॑ प्र-जाः । ए॒व । व॒रु॒ण॒पा॒शादिति॑ वरुण - पा॒शात् । मु॒ञ्च॒ति॒ । आज्य॑भागा॒वित्याज्य॑ - भा॒गौ॒ । य॒ज॒ति॒ । य॒ज्ञ्स्य॑ । ए॒व । चक्षु॑षी॒ इति॑ । न । अ॒न्तः । ए॒ति॒ । वरु॑णम् । य॒ज॒ति॒ । व॒रु॒ण॒पा॒शादिति॑ वरुण - पा॒शात् । ए॒व । ए॒न॒म् । मु॒ञ्च॒ति॒ । अ॒ग्नीवरु॑णा॒वित्य॒ग्नी - वरु॑णौ । य॒ज॒ति॒ । सा॒क्षादिति॑ स - अ॒क्षात् । ए॒व । ए॒न॒म् । व॒रु॒ण॒पा॒शादिति॑ वरुण - पा॒शात् । मु॒ञ्च॒ति॒ । अप॑बर्.हिषा॒वित्यप॑ - ब॒र॒.हि॒षौ॒ । अ॒नू॒या॒जावित्य॑नु - या॒जौ । य॒ज॒ति॒ । प्र॒जा इति॑ प्र - जाः । वै । ब॒र॒.हिः । प्र॒जा इति॑ प्र - जाः । ए॒व । व॒रु॒ण॒पा॒शादिति॑ वरुण - पा॒शात् । मु॒ञ्च॒ति॒ । च॒तुरः॑ । प्र॒या॒जानिति॑ प्र - या॒जान् । य॒ज॒ति॒ । द्वौ । अ॒नू॒या॒जावित्य॑नु - या॒जौ । षट् । समिति॑ । प॒द्य॒न्ते॒ । षट् । वै । ऋ॒तवः॑ । \textbf{  11} \newline
                  \newline
                                \textbf{ TS 6.6.3.4} \newline
                  ऋ॒तुषु॑ । ए॒व । प्रतीति॑ । ति॒ष्ठ॒ति॒ । अव॑भृ॒थेत्यव॑ - भृ॒थ॒ । नि॒च॒ङ्कु॒णेति॑ नि - च॒ङ्कु॒ण॒ । इति॑ । आ॒ह॒ । य॒थो॒दि॒तमिति॑ यथा - उ॒दि॒तम् । ए॒व । वरु॑णम् । अवेति॑ । य॒ज॒ते॒ । स॒मु॒द्रे । ते॒ । हृद॑यम् । अ॒फ्स्वित्य॑प्-सु । अ॒न्तः । इति॑ । आ॒ह॒ । स॒मु॒द्रे । हि । अ॒न्तः । वरु॑णः । समिति॑ । त्वा॒ । वि॒श॒न्तु॒ । ओष॑धीः । उ॒त । आपः॑ । इति॑ । आ॒ह॒ । अ॒द्भिरित्य॑त्-भिः । ए॒व । ए॒न॒म् । ओष॑धीभि॒रित्योष॑धि - भिः॒ । स॒म्यञ्च᳚म् । द॒धा॒ति॒ । देवीः᳚ । आ॒पः॒ । ए॒षः । वः॒ । गर्भः॑ । इति॑ । आ॒ह॒ । य॒था॒य॒जुरिति॑ यथा - य॒जुः । ए॒व । ए॒तत् । प॒शवः॑ । वै । \textbf{  12} \newline
                  \newline
                                \textbf{ TS 6.6.3.5} \newline
                  सोमः॑ । यत् । भि॒न्दू॒नाम् । भ॒क्षये᳚त् । प॒शु॒मानिति॑ पशु - मान् । स्या॒त् । वरु॑णः । तु । ए॒न॒म् । गृ॒ह्णी॒या॒त् । यत् । न । भ॒क्षये᳚त् । अ॒प॒शुः । स्या॒त् । न । ए॒न॒म् । वरु॑णः । गृ॒ह्णी॒या॒त् । उ॒प॒स्पृश्य॒मित्यु॑प - स्पृश्य᳚म् । ए॒व । प॒शु॒मानिति॑ पशु - मान् । भ॒व॒ति॒ । न । ए॒न॒म् । वरु॑णः । गृ॒ह्णा॒ति॒ । प्रति॑युत॒ इति॒ प्रति॑ - यु॒तः॒ । वरु॑णस्य । पाशः॑ । इति॑ । आ॒ह॒ । व॒रु॒ण॒पा॒शादिति॑ वरुण - पा॒शात् । ए॒व । निरिति॑ । मु॒च्य॒ते॒ । अप्र॑तीक्ष॒मित्यप्र॑ति - ई॒क्ष॒म् । एति॑ । य॒न्ति॒ । वरु॑णस्य । अ॒न्तर्.हि॑त्या॒ इत्य॒न्तः - हि॒त्यै॒ । एधः॑ । अ॒सि॒ । ए॒धि॒षी॒महि॑ । इति॑ । आ॒ह॒ । स॒मिधेति॑ सम् - इधा᳚ । ए॒व । अ॒ग्निम् । न॒म॒स्यन्तः॑ ( ) । उ॒पाय॒न्तीत्यु॑प - आय॑न्ति । तेजः॑ । अ॒सि॒ । तेजः॑ । मयि॑ । धे॒हि॒ । इति॑ । आ॒ह॒ । तेजः॑ । ए॒व । आ॒त्मन्न् । ध॒त्ते॒ ॥ \textbf{  13} \newline
                  \newline
                      (रक्षाꣳ॑सि - प्रया॒जा - नृ॒तवो॒ - वै - न॑म॒स्यन्तो॒ - द्वाद॑श च)  \textbf{(A3)} \newline \newline
                                \textbf{ TS 6.6.4.1} \newline
                  स्फ्येन॑ । वेदि᳚म् । उदिति॑ । ह॒न्ति॒ । र॒था॒क्षेणेति॑ रथ-अ॒क्षेण॑ । वीति॑ । मि॒मी॒ते॒ । यूप᳚म् । मि॒नो॒ति॒ । त्रि॒वृत॒मिति॑ त्रि - वृत᳚म् । ए॒व । वज्र᳚म् । स॒भृंत्येति॑ सं - भृत्य॑ । भ्रातृ॑व्याय । प्रेति॑ । ह॒र॒ति॒ । स्तृत्यै᳚ । यत् । अ॒न्त॒र्वे॒दीत्य॑न्तः - वे॒दि । मि॒नु॒यात् । दे॒व॒लो॒कमिति॑ देव - लो॒कम् । अ॒भीति॑ । ज॒ये॒त् । यत् । ब॒हि॒र्वे॒दीति॑ बहिः - वे॒दि । म॒नु॒ष्य॒लो॒कमिति॑ मनुष्य - लो॒कम् । वे॒द्य॒न्तस्येति॑ वेदि - अ॒न्तस्य॑ । स॒न्धाविति॑ सं - धौ । मि॒नो॒ति॒ । उ॒भयोः᳚ । लो॒कयोः᳚ । अ॒भिजि॑त्या॒ इत्य॒भि - जि॒त्यै॒ । उप॑रसम्मिता॒मित्युप॑र - स॒मिं॒ता॒म् । मि॒नु॒या॒त् । पि॒तृ॒लो॒कका॑म॒स्येति॑ पितृलो॒क - का॒म॒स्य॒ । र॒श॒नस॑म्मिता॒मिति॑ रश॒न - स॒मिं॒ता॒म् । म॒नु॒ष्य॒लो॒कका॑म॒स्येति॑ मनुष्यलो॒क - का॒म॒स्य॒ । च॒षाल॑संमिता॒मिति॑ च॒षाल॑ - स॒मिं॒ता॒म् । इ॒न्द्रि॒यका॑म॒स्येती᳚न्द्रि॒य - का॒म॒स्य॒ । सर्वान्॑ । स॒मान् । प्र॒ति॒ष्ठाका॑म॒स्येति॑ प्रति॒ष्ठा - का॒म॒स्य॒ । ये । त्रयः॑ । म॒द्ध्य॒माः । तान् । स॒मान् । प॒शुका॑म॒स्येति॑ प॒शु - का॒म॒स्य॒ । ए॒तान् । वै । \textbf{  14} \newline
                  \newline
                                \textbf{ TS 6.6.4.2} \newline
                  अन्विति॑ । प॒शवः॑ । उपेति॑ । ति॒ष्ठ॒न्ते॒ । प॒शु॒मानिति॑ पशु - मान् । ए॒व । भ॒व॒ति॒ । व्यति॑षजे॒दिति॑ वि - अति॑षजेत् । इत॑रान् । प्र॒जयेति॑ प्र - जया᳚ । ए॒व । ए॒न॒म् । प॒शुभि॒रिति॑ प॒शु - भिः॒ । व्यति॑षज॒तीति॑ वि - अति॑षजति । यम् । का॒मये॑त । प्र॒मायु॑क॒ इति॑ प्र - मायु॑कः । स्या॒त् । इति॑ । ग॒र्त॒मित॒मिति॑ गर्त - मित᳚म् । तस्य॑ । मि॒नु॒या॒त् । उ॒त्त॒रा॒द्‌र्ध्य॑मित्यु॑त्तर - अ॒द्‌र्ध्य᳚म् । वर्.षि॑ष्ठम् । अथ॑ । ह्रसी॑याꣳसम् । ए॒षा । वै । ग॒र्त॒मिदिति॑ गर्त - मित् । यस्य॑ । ए॒वम् । मि॒नोति॑ । ता॒जक् । प्रेति॑ । मी॒य॒ते॒ । द॒क्षि॒णा॒द्‌र्ध्य॑मिति॑ दक्षिण - अ॒द्‌र्ध्य᳚म् । वर्.षि॑ष्ठम् । मि॒नु॒या॒त् । सु॒व॒र्गका॑म॒स्येति॑ सुव॒र्ग - का॒म॒स्य॒ । अथ॑ । ह्रसी॑याꣳसम् । आ॒क्रम॑ण॒मित्या᳚ - क्रम॑णम् । ए॒व । तत् । सेतु᳚म् । यज॑मानः । कु॒रु॒ते॒ । सु॒व॒र्गस्येति॑ सुवः - गस्य॑ । लो॒कस्य॑ । सम॑ष्ट्य॒ इति॒ सं - अ॒ष्ट्यै॒ । \textbf{  15} \newline
                  \newline
                                \textbf{ TS 6.6.4.3} \newline
                  यत् । एक॑स्मिन्न् । यूपे᳚ । द्वे इति॑ । र॒श॒ने इति॑ । प॒रि॒व्यय॒तीति॑ परि - व्यय॑ति । तस्मा᳚त् । एकः॑ । द्वे इति॑ । जा॒ये इति॑ । वि॒न्द॒ते॒ । यत् । न । एका᳚म् । र॒श॒नाम् । द्वयोः᳚ । यूप॑योः । प॒रि॒व्यय॒तीति॑ परि - व्यय॑ति । तस्मा᳚त् । न । एका᳚ । द्वौ । पती॒ इति॑ । वि॒न्द॒ते॒ । यम् । का॒मये॑त । स्त्री । अ॒स्य॒ । जा॒ये॒त॒ । इति॑ । उ॒पा॒न्त इत्यु॑प - अ॒न्ते । तस्य॑ । व्यति॑षजे॒दिति॑ वि - अति॑षजेत् । स्त्री । ए॒व । अ॒स्य॒ । जा॒य॒ते॒ । यम् । का॒मये॑त । पुमान्॑ । अ॒स्य॒ । जा॒ये॒त॒ । इति॑ । आ॒न्तमित्या᳚ - अ॒न्तम् । तस्य॑ । प्रेति॑ । वे॒ष्ट॒ये॒त् । पुमान्॑ । ए॒व । अ॒स्य॒ । \textbf{  16} \newline
                  \newline
                                \textbf{ TS 6.6.4.4} \newline
                  जा॒य॒ते॒ । असु॑राः । वै । दे॒वान् । द॒क्षि॒ण॒तः । उपेति॑ । अ॒न॒य॒न्न् । तान् । दे॒वाः । उ॒प॒श॒येनेत्यु॑प - श॒येन॑ । ए॒व । अपेति॑ । अ॒नु॒द॒न्त॒ । तत् । उ॒प॒श॒यस्येत्यु॑प - श॒यस्य॑ । उ॒प॒श॒य॒त्वमित्यु॑पशय - त्वम् । यत् । द॒क्षि॒ण॒तः । उ॒प॒श॒य इत्यु॑प - श॒यः । उ॒प॒शय॒ इत्यु॑प - शये᳚ । भ्रातृ॑व्यापनुत्त्या॒ इति॒ भ्रातृ॑व्य-अ॒प॒नु॒त्त्यै॒ । सर्वे᳚ । वै । अ॒न्ये । यूपाः᳚ । प॒शु॒मन्त॒ इति॑ पशु - मन्तः॑ । अथ॑ । उ॒प॒श॒य इत्यु॑प - श॒यः । ए॒व । अ॒प॒शुः । तस्य॑ । यज॑मानः । प॒शुः । यत् । न । नि॒र्दि॒शेदिति॑ निः - दि॒शेत् । आर्ति᳚म् । एति॑ । ऋ॒च्छे॒त् । यज॑मानः । अ॒सौ । ते॒ । प॒शुः । इति॑ । निरिति॑ । दि॒शे॒त् । यम् । द्वि॒ष्यात् । यम् । ए॒व । \textbf{  17} \newline
                  \newline
                                \textbf{ TS 6.6.4.5} \newline
                  द्वेष्टि॑ । तम् । अ॒स्मै॒ । प॒शुम् । निरिति॑ । दि॒श॒ति॒ । यदि॑ । न । द्वि॒ष्यात् । आ॒खुः । ते॒ । प॒शुः । इति॑ । ब्रू॒या॒त् । न । ग्रा॒म्यान् । प॒शून् । हि॒नस्ति॑ । न । आ॒र॒ण्यान् । प्र॒जाप॑ति॒रिति॑ प्र॒जा - प॒तिः॒ । प्र॒जा इति॑ प्र - जाः । अ॒सृ॒ज॒त॒ । सः । अ॒न्नाद्ये॒नेत्य॑न्न - अद्ये॑न । वीति॑ । अ॒द्‌र्ध्य॒त॒ । सः । ए॒ताम् । ए॒का॒द॒शिनी᳚म् । अ॒प॒श्य॒त् । तया᳚ । वै । सः । अ॒न्नाद्य॒मित्य॑न्न - अद्य᳚म् । अवेति॑ । अ॒रु॒न्ध॒ । यत् । दश॑ । यूपाः᳚ । भव॑न्ति । दशा᳚क्ष॒रेति॒ दश॑ - अ॒क्ष॒रा॒ । वि॒राडिति॑ वि - राट् । अन्न᳚म् । वि॒राडिति॑ वि - राट् । वि॒राजेति॑ वि - राजा᳚ । ए॒व । अ॒न्नाद्य॒मित्य॑न्न - अद्य᳚म् । अवेति॑ । रु॒न्धे॒ । \textbf{  18} \newline
                  \newline
                                \textbf{ TS 6.6.4.6} \newline
                  यः । ए॒का॒द॒शः । स्तनः॑ । ए॒व । अ॒स्यै॒ । सः । दु॒हे । ए॒व । ए॒ना॒म् । तेन॑ । वज्रः॑ । वै । ए॒षा । समिति॑ । मी॒य॒ते॒ । यत् । ए॒का॒द॒शिनी᳚ । सा । ई॒श्व॒रा । पु॒रस्ता᳚त् । प्र॒त्यञ्च᳚म् । य॒ज्ञ्म् । संम॑र्दितो॒रिति॒ सं - म॒र्दि॒तोः॒ । यत् । पा॒त्नी॒व॒तमिति॑ पात्नी - व॒तम् । मि॒नोति॑ । य॒ज्ञ्स्य॑ । प्रतीति॑ । उत्त॑ब्ध्या॒ इत्युत् - स्त॒ब्ध्यै॒ । स॒य॒त्वायेति॑ सय - त्वाय॑ ॥ \textbf{  19 } \newline
                  \newline
                      (वै - सम॑ष्ट्यै॒ - पुमा॑ने॒वास्य॒ - यमे॒व - रु॑न्धे - त्रिꣳ॒॒शच्च॑)  \textbf{(A4)} \newline \newline
                                \textbf{ TS 6.6.5.1} \newline
                  प्र॒जाप॑ति॒रिति॑ प्र॒जा - प॒तिः॒ । प्र॒जा इति॑ प्र-जाः । अ॒सृ॒ज॒त॒ । सः । रि॒रि॒चा॒नः । अ॒म॒न्य॒त॒ । सः । ए॒ताम् । ए॒का॒द॒शिनी᳚म् । अ॒प॒श्य॒त् । तया᳚ । वै । सः । आयुः॑ । इ॒न्द्रि॒यम् । वी॒र्य᳚म् । आ॒त्मन्न् । अ॒ध॒त्त॒ । प्र॒जा इति॑ प्र - जाः । इ॒व॒ । खलु॑ । वै । ए॒षः । सृ॒ज॒ते॒ । यः । यज॑ते । सः । ए॒तर्.हि॑ । रि॒रि॒चा॒नः । इ॒व॒ । यत् । ए॒षा । ए॒का॒द॒शिनी᳚ । भव॑ति । आयुः॑ । ए॒व । तया᳚ । इ॒न्द्रि॒यम् । वी॒र्य᳚म् । यज॑मानः । आ॒त्मन्न् । ध॒त्ते॒ । प्रेति॑ । ए॒व । आ॒ग्ने॒येन॑ । वा॒प॒य॒ति॒ । मि॒थु॒नम् । सा॒र॒स्व॒त्या । क॒रो॒ति॒ । रेतः॑ । \textbf{  20} \newline
                  \newline
                                \textbf{ TS 6.6.5.2} \newline
                  सौ॒म्येन॑ । द॒धा॒ति॒ । प्रेति॑ । ज॒न॒य॒ति॒ । पौ॒ष्णेन॑ । बा॒र्॒.ह॒स्प॒त्यः । भ॒व॒ति॒ । ब्रह्म॑ । वै । दे॒वाना᳚म् । बृह॒स्पतिः॑ । ब्रह्म॑णा । ए॒व । अ॒स्मै॒ । प्र॒जा इति॑ प्र-जाः । प्रेति॑ । ज॒न॒य॒ति॒ । वै॒श्व॒दे॒व इति॑ वैश्व - दे॒वः । भ॒व॒ति॒ । वै॒श्व॒दे॒व्य॑ इति॑ वैश्व - दे॒व्यः॑ । वै । प्र॒जा इति॑ प्र - जाः । प्र॒जा इति॑ प्र - जाः । ए॒व । अ॒स्मै॒ । प्रेति॑ । ज॒न॒य॒ति॒ । इ॒न्द्रि॒यम् । ए॒व । ऐ॒न्द्रेण॑ । अवेति॑ । रु॒न्धे॒ । विश᳚म् । मा॒रु॒तेन॑ । ओजः॑ । बल᳚म् । ऐ॒न्द्रा॒ग्नेनेत्यै᳚न्द्र - अ॒ग्नेन॑ । प्र॒स॒वायेति॑ प्र - स॒वाय॑ । सा॒वि॒त्रः । नि॒र्व॒रु॒ण॒त्वायेति॑ निर्वरुण - त्वाय॑ । वा॒रु॒णः । म॒द्ध्य॒तः । ऐ॒न्द्रम् । एति॑ । ल॒भ॒ते॒ । म॒द्ध्य॒तः । ए॒व । इ॒न्द्रि॒यम् । यज॑माने । द॒धा॒ति॒ । \textbf{  21} \newline
                  \newline
                                \textbf{ TS 6.6.5.3} \newline
                  पु॒रस्ता᳚त् । ऐ॒न्द्रस्य॑ । वै॒श्व॒दे॒वमिति॑ वैश्व - दे॒वम् । एति॑ । ल॒भ॒ते॒ । वै॒श्व॒दे॒वमिति॑ वैश्व - दे॒वम् । वै । अन्न᳚म् । अन्न᳚म् । ए॒व । पु॒रस्ता᳚त् । ध॒त्ते॒ । तस्मा᳚त् । पु॒रस्ता᳚त् । अन्न᳚म् । अ॒द्य॒ते॒ । ऐ॒न्द्रम् । आ॒लभ्येत्या᳚-लभ्य॑ । मा॒रु॒तम् । एति॑ । ल॒भ॒ते॒ । विट् । वै । म॒रुतः॑ । विश᳚म् । ए॒व । अ॒स्मै॒ । अन्विति॑ । ब॒द्ध्ना॒ति॒ । यदि॑ । का॒मये॑त । यः । अव॑गत॒ इत्यव॑ - ग॒तः॒ । सः । अपेति॑ । रु॒द्ध्य॒ता॒म् । यः । अप॑रुद्ध॒ उत्यप॑- रु॒द्धः॒ । सः । अवेति॑ । ग॒च्छ॒तु॒ । इति॑ । ऐ॒न्द्रस्य॑ । लो॒के । वा॒रु॒णम् । एति॑ । ल॒भे॒त॒ । वा॒रु॒णस्य॑ । लो॒के । ऐ॒न्द्रम् । \textbf{  22} \newline
                  \newline
                                \textbf{ TS 6.6.5.4} \newline
                  यः । ए॒व । अव॑गत॒ इत्यव॑ - ग॒तः॒ । सः । अपेति॑ । रु॒द्ध्य॒ते॒ । यः । अप॑रुद्ध॒ उत्यप॑-रु॒द्धः॒ । सः । अवेति॑ । ग॒च्छ॒ति॒ । यदि॑ । का॒मये॑त । प्र॒जा इति॑ प्र - जाः । मु॒ह्ये॒युः॒ । इति॑ । प॒शून् । व्यति॑षजे॒दिति॑ वि - अति॑षजेत् । प्र॒जा इति॑ प्र - जाः । ए॒व । मो॒ह॒य॒ति॒ । यत् । अ॒भि॒वा॒ह॒त इत्य॑भि - वा॒ह॒तः । अ॒पाम् । वा॒रु॒णम् । आ॒लभे॒तेत्या᳚ - लभे॑त । प्र॒जा इति॑ प्र - जाः । वरु॑णः । गृ॒ह्णी॒या॒त् । द॒क्षि॒ण॒तः । उद॑ञ्चम् । एति॑ । ल॒भ॒ते॒ । अ॒प॒वा॒ह॒त इत्य॑प - वा॒ह॒तः । अ॒पाम् । प्र॒जाना॒मिति॑ प्र - जाना᳚म् । अव॑रुणग्राहा॒येत्यव॑रुण - ग्रा॒हा॒य॒ ॥ \textbf{  23 } \newline
                  \newline
                      (रेतो॒ - यज॑माने दधाति - लो॒क ऐ॒न्द्रꣳ - स॒प्तत्रिꣳ॑शच्च)  \textbf{(A5)} \newline \newline
                                \textbf{ TS 6.6.6.1} \newline
                  इन्द्रः॑ । पत्नि॑या । मनु᳚म् । अ॒या॒ज॒य॒त् । ताम् । पर्य॑ग्निकृता॒मिति॒ पर्य॑ग्नि - कृ॒ता॒म् । उदिति॑ । अ॒सृ॒ज॒त् । तया᳚ । मनुः॑ । आ॒द्‌र्ध्नो॒त् । यत् । पर्य॑ग्निकृत॒मिति॒ पर्य॑ग्नि-कृ॒त॒म् । पा॒त्नी॒व॒तमिति॑ पात्नी-व॒तम् । उ॒थ्सृ॒जतीत्यु॑त् - सृ॒जति॑ । याम् । ए॒व । मनुः॑ । ऋद्धि᳚म् । आद्‌र्ध्नो᳚त् । ताम् । ए॒व । यज॑मानः । ऋ॒द्ध्नो॒ति॒ । य॒ज्ञ्स्य॑ । वै । अप्र॑तिष्ठिता॒दित्यप्र॑ति - स्थि॒ता॒त् । य॒ज्ञ्ः । परेति॑ । भ॒व॒ति॒ । य॒ज्ञ्म् । प॒रा॒भव॑न्त॒मिति॑ परा - भव॑न्तम् । यज॑मानः । अनु॑ । परेति॑ । भ॒व॒ति॒ । यत् । आज्ये॑न । पा॒त्नी॒व॒तमिति॑ पात्नी - व॒तम् । सꣳ॒॒स्था॒पय॒तीति॑ सं - स्था॒पय॑ति । य॒ज्ञ्स्य॑ । प्रति॑ष्ठित्या॒ इति॒ प्रति॑-स्थि॒त्यै॒ । य॒ज्ञ्म् । प्र॒ति॒तिष्ठ॑न्त॒मिति॑ प्रति - तिष्ठ॑न्तम् । यज॑मानः । अनु॑ । प्रतीति॑ । ति॒ष्ठ॒ति॒ । इ॒ष्टम् । व॒पया᳚ । \textbf{  24} \newline
                  \newline
                                \textbf{ TS 6.6.6.2} \newline
                  भव॑ति । अनि॑ष्टम् । व॒शया᳚ । अथ॑ । पा॒त्नी॒व॒तेनेति॑ पात्नी - व॒तेन॑ । प्रेति॑ । च॒र॒ति॒ । ती॒र्त्थ । ए॒व । प्रेति॑ । च॒र॒ति॒ । अथो॒ इति॑ । ए॒तर्.हि॑ । ए॒व । अ॒स्य॒ । यामः॑ । त्वा॒ष्ट्रः । भ॒व॒ति॒ । त्वष्टा᳚ । वै । रेत॑सः । सि॒क्तस्य॑ । रू॒पाणि॑ । वीति॑ । क॒रो॒ति॒ । तम् । ए॒व । वृषा॑णम् । पत्नी॑षु । अपीति॑ । सृ॒ज॒ति॒ । सः । अ॒स्मै॒ । रू॒पाणि॑ । वीति॑ । क॒रो॒ति॒ ॥ \textbf{  25 } \newline
                  \newline
                      (व॒पया॒ - षट्त्रिꣳ॑शच्च)  \textbf{(A6)} \newline \newline
                                \textbf{ TS 6.6.7.1} \newline
                  घ्नन्ति॑ । वा । ए॒तत् । सोम᳚म् । यत् । अ॒भि॒षु॒ण्वन्तीत्य॑भि-सु॒न्वन्ति॑ । यत् । सौ॒म्यः । भव॑ति । यथा᳚ । मृ॒ताय॑ । अ॒नु॒स्तर॑णी॒मित्य॑नु - स्तर॑णीम् । घ्नन्ति॑ । ता॒दृक् । ए॒व । तत् । यत् । उ॒त्त॒रा॒द्‌र्ध इत्यु॑त्तर - अ॒द्‌र्धे । वा॒ । मद्ध्ये᳚ । वा॒ । जु॒हु॒यात् । दे॒वता᳚भ्यः । स॒मद॒मिति॑ स - मद᳚म् । द॒द्ध्या॒त् । द॒क्षि॒णा॒द्‌र्ध इति॑ दक्षिण - अ॒द्‌र्धे । जु॒हो॒ति॒ । ए॒षा । वै । पि॒तृ॒णाम् । दिक् । स्वाया᳚म् । ए॒व । दि॒शि । पि॒तॄन् । नि॒रव॑दयत॒ इति॑ निः - अव॑दयते । उ॒द्गा॒तृभ्य॒ इत्यु॑द्गा॒तृ - भ्यः॒ । ह॒र॒न्ति॒ । सा॒म॒दे॒व॒त्य॑ इति॑ साम - दे॒व॒त्यः॑ । वै । सौ॒म्यः । यत् । ए॒व । साम्नः॑ । छ॒बं॒ट्कु॒र्वन्तीति॑ छंबट् - कु॒र्वन्ति॑ । तस्य॑ । ए॒व । सः । शान्तिः॑ । अवेति॑ । \textbf{  26} \newline
                  \newline
                                \textbf{ TS 6.6.7.2} \newline
                  ई॒क्ष॒न्ते॒ । प॒वित्र᳚म् । वै । सौ॒म्यः । आ॒त्मान᳚म् । ए॒व । प॒व॒य॒न्ते॒ । यः । आ॒त्मान᳚म् । न । प॒रि॒पश्ये॒दिति॑ परि-पश्ये᳚त् । इ॒तासु॒रिती॒त-अ॒सुः॒ । स्या॒त् । अ॒भि॒द॒दिमित्य॑भि - द॒दिम् । कृ॒त्वा । अवेति॑ । ई॒क्षे॒त॒ । तस्मिन्न्॑ । हि । आ॒त्मान᳚म् । प॒रि॒पश्य॒तीति॑ परि - पश्य॑ति । अथो॒ इति॑ । आ॒त्मान᳚म् । ए॒व । प॒व॒य॒ते॒ । यः । ग॒तम॑ना॒ इति॑ ग॒त - म॒नाः॒ । स्यात् । सः । अवेति॑ । ई॒क्षे॒त॒ । यत् । मे॒ । मनः॑ । परा॑गत॒मिति॒ परा᳚-ग॒त॒म् । यत् । वा॒ । मे॒ । अप॑रागत॒मित्यप॑रा-ग॒त॒म् ॥ राज्ञा᳚ । सोमे॑न । तत् । व॒यम् । अ॒स्मासु॑ । धा॒र॒या॒म॒सि॒ । इति॑ । मनः॑ । ए॒व । आ॒त्मन्न् । दा॒धा॒र॒ । \textbf{  27} \newline
                  \newline
                                \textbf{ TS 6.6.7.3} \newline
                  न । ग॒तम॑ना॒ इति॑ ग॒त - म॒नाः॒ । भ॒व॒ति॒ । अपेति॑ । वै । तृ॒ती॒य॒स॒व॒न इति॑ तृतीय - स॒व॒ने । य॒ज्ञ्ः । क्रा॒म॒ति॒ । ई॒जा॒नात् । अनी॑जानम् । अ॒भीति॑ । आ॒ग्ना॒वै॒ष्ण॒व्येत्या᳚ग्ना - वै॒ष्ण॒व्या । ऋ॒चा । घृ॒तस्य॑ । य॒ज॒ति॒ । अ॒ग्निः । सर्वाः᳚ । दे॒वताः᳚ । विष्णुः॑ । य॒ज्ञ्ः । दे॒वताः᳚ । च॒ । ए॒व । य॒ज्ञ्म् । च॒ । दा॒धा॒र॒ । उ॒पाꣳ॒॒श्वित्यु॑प - अꣳ॒॒शु । य॒ज॒ति॒ । मि॒थु॒न॒त्वायेति॑ मिथुन - त्वाय॑ । ब्र॒ह्म॒वा॒दिन॒ इति॑ ब्रह्म - वा॒दिनः॑ । व॒द॒न्ति॒ । मि॒त्रः । य॒ज्ञ्स्य॑ । स्वि॑ष्ट॒मिति॒ सु-इ॒ष्ट॒म् । यु॒व॒ते॒ । वरु॑णः । दुरि॑ष्ट॒मिति॒ दुः - इ॒ष्ट॒म् । क्व॑ । तर्.हि॑ । य॒ज्ञ्ः । क्व॑ । यज॑मानः । भ॒व॒ति॒ । इति॑ । यत् । मै॒त्रा॒व॒रु॒णीमिति॑ मैत्रा - व॒रु॒णीम् । व॒शाम् । आ॒लभ॑त॒ इत्या᳚ - लभ॑ते । मि॒त्रेण॑ । ए॒व । \textbf{  28} \newline
                  \newline
                                \textbf{ TS 6.6.7.4} \newline
                  य॒ज्ञ्स्य॑ । स्वि॑ष्ट॒मिति॒ सु - इ॒ष्ट॒म् । श॒म॒य॒ति॒ । वरु॑णेन । दुरि॑ष्ट॒मिति॒ दुः - इ॒ष्ट॒म् । न । आर्ति᳚म् । एति॑ । ऋ॒च्छ॒ति॒ । यज॑मानः । यथा᳚ । वै । लाङ्ग॑लेन । उ॒र्वरा᳚म् । प्र॒भि॒न्दन्तीति॑ प्र - भि॒न्दन्ति॑ । ए॒वम् । ऋ॒ख्सा॒मे इत्यृ॑क् - सा॒मे । य॒ज्ञ्म् । प्रेति॑ । भि॒न्तः॒ । यत् । मै॒त्रा॒व॒रु॒णीमिति॑ मैत्रा - व॒रु॒णीम् । व॒शाम् । आ॒ल॑भत॒ इत्या᳚-लभ॑ते । य॒ज्ञाय॑ । ए॒व । प्रभि॑न्ना॒येति॒ प्र - भि॒न्ना॒य॒ । म॒त्य᳚म् । अ॒न्ववा᳚स्य॒तीत्य॑नु-अवा᳚स्यति । शान्त्यै᳚ । या॒तया॑मा॒नीति॑ या॒त-या॒मा॒नि॒ । वै । ए॒तस्य॑ । छन्दाꣳ॑सि । यः । ई॒जा॒नः । छन्द॑साम् । ए॒षः । रसः॑ । यत् । व॒शा । यत् । मै॒त्रा॒व॒रु॒णीमिति॑ मैत्रा - व॒रु॒णीम् । व॒शाम् । आ॒लभ॑त॒ इत्या᳚ - लभ॑ते । छन्दाꣳ॑सि । ए॒व । पुनः॑ । एति॑ । प्री॒णा॒ति॒ ( ) । अया॑तयामत्वा॒येत्यया॑तयाम - त्वा॒य॒ । अथो॒ इति॑ । छन्द॒स्स्विति॒ छन्दः॑ - सु॒ । ए॒व । रस᳚म् । द॒धा॒ति॒ ॥ \textbf{  29} \newline
                  \newline
                      (अव॑ - दाधार - मि॒त्रेणै॒व - प्री॑णाति॒ - षट्च॑)  \textbf{(A7)} \newline \newline
                                \textbf{ TS 6.6.8.1} \newline
                  दे॒वा । वै । इ॒न्द्रि॒यम् । वी॒र्या᳚म् । वीति॑ । अ॒भ॒ज॒न्त॒ । ततः॑ । यत् । अ॒त्यशि॑ष्य॒तेत्य॑ति - अशि॑ष्यत । तत् । अ॒ति॒ग्रा॒ह्या॑ इत्य॑ति-ग्रा॒ह्याः᳚ । अ॒भ॒व॒न्न् । तत् । अ॒ति॒ग्रा॒ह्या॑णा॒मित्य॑ति - ग्रा॒ह्या॑णाम् । अ॒ति॒ग्रा॒ह्य॒त्वमित्य॑तिग्राह्य - त्वम् । यत् । अ॒ति॒ग्रा॒ह्या॑ इत्य॑ति-ग्रा॒ह्याः᳚ । गृ॒ह्यन्ते᳚ । इ॒न्द्रि॒यम् । ए॒व । तत् । वी॒र्य᳚म् । यज॑मानः । आ॒त्मन्न् । ध॒त्ते॒ । तेजः॑ । आ॒ग्ने॒येन॑ । इ॒न्द्रि॒यम् । ऐ॒न्द्रेण॑ । ब्र॒ह्म॒व॒र्च॒समिति॑ ब्रह्म - व॒र्च॒सम् । सौ॒र्येण॑ । उ॒प॒स्तंभ॑न॒मित्यु॑प - स्तंभ॑नम् । वै । ए॒तत् । य॒ज्ञ्स्य॑ । यत् । अ॒ति॒ग्रा॒ह्या॑ इत्य॑ति - ग्रा॒ह्याः᳚ । च॒क्रे इति॑ । पृ॒ष्ठानि॑ । यत् । पृष्ठ्ये᳚ । न । गृ॒ह्णी॒यात् । प्राञ्च᳚म् । य॒ज्ञ्म् । पृ॒ष्ठानि॑ । समिति॑ । शृ॒णी॒युः॒ । यत् । उ॒क्थ्ये᳚ । \textbf{  30} \newline
                  \newline
                                \textbf{ TS 6.6.8.2} \newline
                  गृ॒ह्णी॒यात् । प्र॒त्यञ्च᳚म् । य॒ज्ञ्म् । अ॒ति॒ग्रा॒ह्या॑ इत्य॑ति - ग्रा॒ह्याः᳚ । समिति॑ । शृ॒णी॒युः॒ । वि॒श्व॒जितीति॑ विश्व - जिति॑ । सर्व॑पृष्ठ॒ इति॒ सर्व॑ - पृ॒ष्ठे॒ । ग्र॒ही॒त॒व्याः᳚ । य॒ज्ञ्स्य॑ । स॒वी॒र्य॒त्वायेति॑ सवीर्य-त्वाय॑ । प्र॒जाप॑ति॒रिति॑ प्र॒जा - प॒तिः॒ । दे॒वेभ्यः॑ । य॒ज्ञान् । व्यादि॑श॒दिति॑ वि - आदि॑शत् । सः । प्रि॒याः । त॒नूः । अप॑ । नीति॑ । अ॒ध॒त्त॒ । तत् । अ॒ति॒ग्रा॒ह्या॑ इत्य॑ति-ग्रा॒ह्याः᳚ । अ॒भ॒व॒न्न् । वित॑नु॒रिति॒ वि-त॒नुः॒ । तस्य॑ । य॒ज्ञ्ः । इति॑ । आ॒हुः॒ । यस्य॑ । अ॒ति॒ग्रा॒ह्या॑ इत्य॑ति - ग्रा॒ह्याः᳚ । न । गृ॒ह्यन्ते᳚ । इति॑ । अपीति॑ । अ॒ग्नि॒ष्टो॒म इत्य॑ग्नि - स्तो॒मे । ग्र॒ही॒त॒व्याः᳚ । य॒ज्ञ्स्य॑ । स॒त॒नु॒त्वायेति॑ सतनु - त्वाय॑ । दे॒वताः᳚ । वै । सर्वाः᳚ । स॒दृशीः᳚ । आ॒स॒न्न् । ताः । न । व्या॒वृत॒मिति॑ वि-आ॒वृत᳚म् । अ॒ग॒च्छ॒न्न् । ते । दे॒वाः । \textbf{  31} \newline
                  \newline
                                \textbf{ TS 6.6.8.3} \newline
                  ए॒ते । ए॒तान् । ग्रहान्॑ । अ॒प॒श्य॒न्न् । तान् । अ॒गृ॒ह्ण॒त॒ । आ॒ग्ने॒यम् । अ॒ग्निः । ऐ॒न्द्रम् । इन्द्रः॑ । सौ॒र्यम् । सूर्यः॑ । ततः॑ । वै । ते । अ॒न्याभिः॑ । दे॒वता॑भिः । व्या॒वृत॒मिति॑ वि - आ॒वृत᳚म् । अ॒ग॒च्छ॒न्न् । यस्य॑ । ए॒वम् । वि॒दुषः॑ । ए॒ते । ग्रहाः᳚ । गृ॒ह्यन्ते᳚ । व्या॒वृत॒मिति॑ वि-आ॒वृत᳚म् । ए॒व । पा॒प्मना᳚ । भ्रातृ॑व्येण । ग॒च्छ॒ति॒ । इ॒मे । लो॒काः । ज्योति॑ष्मन्तः । स॒माव॑द्वीर्या॒ इति॑ स॒माव॑त् - वी॒र्याः॒ । का॒र्याः᳚ । इति॑ । आ॒हुः॒ । आ॒ग्ने॒येन॑ । अ॒स्मिन्न् । लो॒के । ज्योतिः॑ । ध॒त्ते॒ । ऐ॒न्द्रेण॑ । अ॒न्तरि॑क्षे । इ॒न्द्र॒वा॒यू इती᳚न्द्र - वा॒यू । हि । स॒युजा॒विति॑ स-युजौ᳚ । सौ॒र्येण॑ । अ॒मुष्मिन्न्॑ । लो॒के । \textbf{  32} \newline
                  \newline
                                \textbf{ TS 6.6.8.4} \newline
                  ज्योतिः॑ । ध॒त्ते॒ । ज्योति॑ष्मन्तः । अ॒स्मै॒ । इ॒मे । लो॒काः । भ॒व॒न्ति॒ । स॒माव॑त्वीर्या॒निति॑ स॒माव॑त् - वी॒र्या॒न् । ए॒ना॒न् । कु॒रु॒ते॒ । ए॒तान् । वै । ग्रहान्॑ । ब॒बांवि॒श्वव॑यसा॒विति॑ ब॒बां - वि॒श्वव॑यसौ । अ॒वि॒त्ता॒म् । ताभ्या᳚म् । इ॒मे । लो॒काः । परा᳚ञ्चः । च॒ । अ॒र्वाञ्चः॑ । च॒ । प्रेति॑ । अ॒भुः॒ । यस्य॑ । ए॒वम् । वि॒दुषः॑ । ए॒ते । ग्रहाः᳚ । गृ॒ह्यन्ते᳚ । प्रेति॑ । अ॒स्मै॒ । इ॒मे । लो॒काः । परा᳚ञ्चः । च॒ । अ॒र्वाञ्चः॑ । च॒ । भा॒न्ति॒ ॥ \textbf{  33 } \newline
                  \newline
                      (उ॒क्थ्ये॑ - दे॒वा - अ॒मुष्मि॑न् ॅलो॒क - एका॒न्नच॑त्वारिꣳ॒॒शच्च॑)  \textbf{(A8)} \newline \newline
                                \textbf{ TS 6.6.9.1} \newline
                  दे॒वाः । वै । यत् । य॒ज्ञे । अकु॑र्वत । तत् । असु॑राः । अ॒कु॒र्व॒त॒ । ते । दे॒वाः । अदा᳚भ्ये । छन्दाꣳ॑सि । सव॑नानि । समिति॑ । अ॒स्था॒प॒य॒न्न् । ततः॑ । दे॒वाः । अभ॑वन्न् । परेति॑ । असु॑राः । यस्य॑ । ए॒वम् । वि॒दुषः॑ । अदा᳚भ्यः । गृ॒ह्यते᳚ । भव॑ति । आ॒त्मना᳚ । परेति॑ । अ॒स्य॒ । भ्रातृ॑व्यः । भ॒व॒ति॒ । यत् । वै । दे॒वाः । असु॑रान् । अदा᳚भ्येन । अद॑भ्नुवन्न् । तत् । अदा᳚भ्यस्य । अ॒दा॒भ्य॒त्वमित्य॑दाभ्य - त्वम् । यः । ए॒वम् । वेद॑ । द॒भ्नोति॑ । ए॒व । भ्रातृ॑व्यम् । न । ए॒न॒म् । भ्रातृ॑व्यः । द॒भ्नो॒ति॒ । \textbf{  34} \newline
                  \newline
                                \textbf{ TS 6.6.9.2} \newline
                  ए॒षा । वै । प्र॒जाप॑ते॒रिति॑ प्र॒जा-प॒तेः॒ । अ॒ति॒मो॒क्षिणीत्य॑ति-मो॒क्षिणी᳚ । नाम॑ । त॒नूः । यत् । अदा᳚भ्यः । उप॑नद्ध॒स्येत्युप॑-न॒द्ध॒स्य॒ । गृ॒ह्णा॒ति॒ । अति॑मुक्त्या॒ इत्यति॑ - मु॒क्त्यै॒ । अतीति॑ । पा॒प्मान᳚म् । भ्रातृ॑व्यम् । मु॒च्य॒ते॒ । यः । ए॒वम् । वेद॑ । घ्नन्ति॑ । वै । ए॒तत् । सोम᳚म् । यत् । अ॒भि॒षु॒ण्वन्तीत्य॑भि - सु॒न्वन्ति॑ । सोमे᳚ । ह॒न्यमा॑ने । य॒ज्ञ्ः । ह॒न्य॒ते॒ । य॒ज्ञे । यज॑मानः । ब्र॒ह्म॒वा॒दिन॒ इति॑ ब्रह्म-वा॒दिनः॑ । व॒द॒न्ति॒ । किम् । तत् । य॒ज्ञे । यज॑मानः । कु॒रु॒ते॒ । येन॑ । जीवन्न्॑ । सु॒व॒र्गमिति॑ सुवः - गम् । लो॒कम् । एति॑ । इति॑ । जी॒व॒ग्र॒ह इति॑ जीव - ग्र॒हः । वै । ए॒षः । यत् । अदा᳚भ्यः । अन॑भिषुत॒स्येत्यन॑भि - सु॒त॒स्य॒ । गृ॒ह्णा॒ति॒ ( ) । जीव॑न्तम् । ए॒व । ए॒न॒म् । सु॒व॒र्गमिति॑ सुवः - गम् । लो॒कम् । ग॒म॒य॒ति॒ । वीति॑ । वै । ए॒तत् । य॒ज्ञ्म् । छि॒न्द॒न्ति॒ । यत् । अदा᳚भ्ये । सꣳ॒॒स्था॒पय॒न्तीति॑ सं - स्था॒पय॑न्ति । अꣳ॒॒शून् । अपीति॑ । स॒ज॒ति॒ । य॒ज्ञ्स्य॑ । सन्त॑त्या॒ इति॒ सं - त॒त्यै॒ ॥ \textbf{  35} \newline
                  \newline
                      (द॒भ्नो॒त्य - न॑भिषुतस्य गृह्णा॒त्ये - का॒न्नविꣳ॑श॒तिश्च॑)  \textbf{(A9)} \newline \newline
                                \textbf{ TS 6.6.10.1} \newline
                  दे॒वाः । वै । प्र॒बाहु॒गिति॑ प्र - बाहु॑क् । ग्रहान्॑ । अ॒गृ॒ह्ण॒त॒ । सः । ए॒तम् । प्र॒जाप॑ति॒रिति॑ प्र॒जा - प॒तिः॒ । अꣳ॒॒शुम् । अ॒प॒श्य॒त् । तम् । अ॒गृ॒ह्णी॒त॒ । तेन॑ । वै । सः । आ॒द्‌र्ध्नो॒त् । यस्य॑ । ए॒वम् । वि॒दुषः॑ । अꣳ॒॒शुः । गृ॒ह्यते᳚ । ऋ॒द्ध्नोति॑ । ए॒व । स॒कृद॑भिषुत॒स्येति॑ स॒कृत् - अ॒भि॒षु॒त॒स्य॒ । गृ॒ह्णा॒ति॒ । स॒कृत् । हि । सः । तेन॑ । आद्‌र्ध्नो᳚त् । मन॑सा । गृ॒ह्णा॒ति॒ । मनः॑ । इ॒व॒ । हि । प्र॒जाप॑ति॒रिति॑ प्र॒जा - प॒तिः॒ । प्र॒जाप॑ते॒रिति॑ प्र॒जा - प॒तेः॒ । आप्त्यै᳚ । औदु॑बंरेण । गृ॒ह्णा॒ति॒ । ऊर्क् । वै । उ॒दु॒म्बरः॑ । ऊर्ज᳚म् । ए॒व । अवेति॑ । रु॒न्धे॒ । चतु॑स्स्र॒क्तीति॒ चतुः॑ - स्र॒क्ति॒ । भ॒व॒ति॒ । दि॒क्षु । \textbf{  36} \newline
                  \newline
                                \textbf{ TS 6.6.10.2} \newline
                  ए॒व । प्रतीति॑ । ति॒ष्ठ॒ति॒ । यः । वै । अꣳ॒॒शोः । आ॒यत॑न॒मित्या᳚-यत॑नम् । वेद॑ । आ॒यत॑नवा॒नित्या॒यत॑न - वा॒न् । भ॒व॒ति॒ । वा॒म॒दे॒व्यमिति॑ वाम - दे॒व्यम् । इति॑ । साम॑ । तत् । वै । अ॒स्य॒ । आ॒यत॑न॒मित्या᳚ - यत॑नम् । मन॑सा । गाय॑मानः । गृ॒ह्णा॒ति॒ । आ॒यत॑नवा॒नित्या॒यत॑न - वा॒न् । ए॒व । भ॒व॒ति॒ । यत् । अ॒द्ध्व॒र्युः । अꣳ॒॒शुम् । गृ॒ह्णन्न् । न । अ॒द्‌र्धये᳚त् । उ॒भाभ्या᳚म् । न । ऋ॒द्ध्ये॒त॒ । अ॒द्ध्व॒र्यवे᳚ । च॒ । यज॑मानाय । च॒ । यत् । अ॒द्‌र्धये᳚त् । उ॒भाभ्या᳚म् । ऋ॒द्ध्ये॒त॒ । अन॑वान॒मित्यन॑व - अ॒न॒म् । गृ॒ह्णा॒ति॒ । सा । ए॒व । अ॒स्य॒ । ऋद्धिः॑ । हिर॑ण्यम् । अ॒भि । वीति॑ । अ॒नि॒ति॒ ( ) । अ॒मृत᳚म् । वै । हिर॑ण्यम् । आयुः॑ । प्रा॒ण इति॑ प्र-अ॒नः । आयु॑षा । ए॒व । अ॒मृत᳚म् । अ॒भीति॑ । धि॒नो॒ति॒ । श॒तमा॑न॒मिति॑ श॒त - मा॒न॒म् । भ॒व॒ति॒ । श॒तायु॒रिति॑ श॒त - आ॒युः॒ । पुरु॑षः । श॒तेन्द्रि॑य॒ इति॑ श॒त-इ॒न्द्रि॒यः॒ । आयु॑षि । ए॒व । इ॒न्द्रि॒ये । प्रतीति॑ । ति॒ष्ठ॒ति॒ ॥ \textbf{  37} \newline
                  \newline
                      (दि॒क्ष्व॑ - निति - विꣳश॒तिश्च॑)  \textbf{(A10)} \newline \newline
                                \textbf{ TS 6.6.11.1} \newline
                  प्र॒जाप॑ति॒रिति॑ प्र॒जा - प॒तिः॒ । दे॒वेभ्यः॑ । य॒ज्ञान् । व्यादि॑श॒दिति॑ वि - आदि॑शत् । सः । रि॒रि॒चा॒नः । अ॒म॒न्य॒त॒ । सः । य॒ज्ञाना᳚म् । षो॒ड॒श॒धेति॑ षोडश - धा । इ॒न्द्रि॒यम् । वी॒र्य᳚म् । आ॒त्मान᳚म् । अ॒भि । समिति॑ । अ॒क्खि॒द॒त् । तत् । षो॒ड॒शी । अ॒भ॒व॒त् । न । वै । षो॒ड॒शी । नाम॑ । य॒ज्ञ्ः । अ॒स्ति॒ । यत् । वाव । षो॒ड॒शम् । स्तो॒त्रम् । षो॒ड॒शम् । श॒स्त्रम् । तेन॑ । षो॒ड॒शी । तत् । षो॒ड॒शिनः॑ । षो॒ड॒शि॒त्वमिति॑ षोडशि -त्वम् । यत् । षो॒ड॒शी । गृ॒ह्यते᳚ । इ॒न्द्रि॒यम् । ए॒व । तत् । वी॒र्य᳚म् । यज॑मानः । आ॒त्मन्न् । ध॒त्ते॒ । दे॒वेभ्यः॑ । वै । सु॒व॒र्ग इति॑ सुवः - गः । लो॒कः । \textbf{  38} \newline
                  \newline
                                \textbf{ TS 6.6.11.2} \newline
                  न । प्रेति॑ । अ॒भ॒व॒त् । ते । ए॒तम् । षो॒ड॒शिन᳚म् । अ॒प॒श्य॒न्न् । तम् । अ॒गृ॒ह्ण॒त॒ । ततः॑ । वै । तेभ्यः॑ । सु॒व॒र्ग इति॑ सुवः - गः । लो॒कः । प्रेति॑ । अ॒भ॒व॒त् । यत् । षो॒ड॒शी । गृ॒ह्यते᳚ । सु॒व॒र्गस्येति॑ सुवः-गस्य॑ । लो॒कस्य॑ । अ॒भिजि॑त्या॒ इत्य॒भि - जि॒त्यै॒ । इन्द्रः॑ । वै । दे॒वाना᳚म् । आ॒नु॒जा॒व॒र इत्या॑नु - जा॒व॒रः । आ॒सी॒त् । सः । प्र॒जाप॑ति॒मिति॑ प्र॒जा - प॒ति॒म् । उपेति॑ । अ॒धा॒व॒त् । तस्मै᳚ । ए॒तम् । षो॒ड॒शिन᳚म् । प्रेति॑ । अ॒य॒च्छ॒त् । तम् । अ॒गृ॒ह्णी॒त॒ । ततः॑ । वै । सः । अग्र᳚म् । दे॒वता॑नाम् । परीति॑ । ऐ॒त् । यस्य॑ । ए॒वम् । वि॒दुषः॑ । षो॒ड॒शी । गृ॒ह्यते᳚ । \textbf{  39} \newline
                  \newline
                                \textbf{ TS 6.6.11.3} \newline
                  अग्र᳚म् । ए॒व । स॒मा॒नाना᳚म् । परीति॑ । ए॒ति॒ । प्रा॒त॒स्स॒व॒न इति॑ प्रातः - स॒व॒ने । गृ॒ह्णा॒ति॒ । वज्रः॑ । वै । षो॒ड॒शी । वज्रः॑ । प्रा॒त॒स्स॒व॒नमिति॑ प्रातः - स॒व॒नम् । स्वात् । ए॒व । ए॒न॒म् । योनेः᳚ । निरिति॑ । गृ॒ह्णा॒ति॒ । सव॑नेसवन॒ इति॒ सव॑ने - स॒व॒ने॒ । अ॒भीति॑ । गृ॒ह्णा॒ति॒ । सव॑नाथ्सवना॒दिति॒ सव॑नात् - स॒व॒ना॒त् । ए॒व । ए॒न॒म् । प्रेति॑ । ज॒न॒य॒ति॒ । तृ॒ती॒य॒स॒व॒न इति॑ तृतीय - स॒व॒ने । प॒शुका॑म॒स्येति॑ प॒शु - का॒म॒स्य॒ । गृ॒ह्णी॒या॒त् । वज्रः॑ । वै । षो॒ड॒शी । प॒शवः॑ । तृ॒ती॒य॒स॒व॒नमिति॑ तृतीय - स॒व॒नम् । वज्रे॑ण । ए॒व । अ॒स्मै॒ । तृ॒ती॒य॒स॒व॒नादिति॑ तृतीय - स॒व॒नात् । प॒शून् । अवेति॑ । रु॒न्धे॒ । न । उ॒क्थ्ये᳚ । गृ॒ह्णी॒या॒त् । प्र॒जेति॑ प्र - जा । वै । प॒शवः॑ । उ॒क्थानि॑ । यत् । उ॒क्थ्ये᳚ । \textbf{  40} \newline
                  \newline
                                \textbf{ TS 6.6.11.4} \newline
                  गृ॒ह्णी॒यात् । प्र॒जामिति॑ प्र - जाम् । प॒शून् । अ॒स्य॒ । निरिति॑ । द॒हे॒त् । अ॒ति॒रा॒त्र इत्य॑ति - रा॒त्रे । प॒शुका॑म॒स्येति॑ प॒शु - का॒म॒स्य॒ । गृ॒ह्णी॒या॒त् । वज्रः॑ । वै । षो॒ड॒शी । वज्रे॑ण । ए॒व । अ॒स्मै॒ । प॒शून् । अ॒व॒रुद्ध्येत्य॑व - रुद्ध्य॑ । रात्रि॑या । उ॒परि॑ष्टात् । श॒म॒य॒ति॒ । अपीति॑ । अ॒ग्नि॒ष्टो॒म इत्य॑ग्नि - स्तो॒मे । रा॒ज॒न्य॑स्य । गृ॒ह्णी॒या॒त् । व्या॒वृत्का॑म॒ इति॑ व्या॒वृत् - का॒मः॒ । हि । रा॒ज॒न्यः॑ । यज॑ते । सा॒ह्न इति॑ स - अ॒ह्ने । ए॒व । अ॒स्मै॒ । वज्र᳚म् । गृ॒ह्णा॒ति॒ । सः । ए॒न॒म् । वज्रः॑ । भूत्यै᳚ । इ॒न्धे॒ । निरिति॑ । वा॒ । द॒ह॒ति॒ । ए॒क॒विꣳ॒॒शमित्ये॑क-विꣳ॒॒शम् । स्तो॒त्रम् । भ॒व॒ति॒ । प्रति॑ष्ठित्या॒ इति॒ प्रति॑ - स्थि॒त्यै॒ । हरि॑व॒दिति॒ हरि॑ - व॒त् । श॒स्य॒ते॒ । इन्द्र॑स्य । प्रि॒यम् । धाम॑ । \textbf{  41} \newline
                  \newline
                                \textbf{ TS 6.6.11.5} \newline
                  उपेति॑ । आ॒प्नो॒ति॒ । कनी॑याꣳसि । वै । दे॒वेषु॑ । छन्दाꣳ॑सि । आसन्न्॑ । ज्यायाꣳ॑सि ।   असु॑रेषु । ते । दे॒वाः । कनी॑यसा । छन्द॑सा । ज्यायः॑ । छन्दः॑ । अ॒भि । वीति॑ । अ॒शꣳ॒॒स॒न्न् । ततः॑ । वै । ते । असु॑राणाम् । लो॒कम् । अ॒वृ॒ञ्ज॒त॒ । यत् । कनी॑यसा । छन्द॑सा । ज्यायः॑ । छन्दः॑ । अ॒भीति॑ । वि॒शꣳस॒तीति॑ वि - शꣳस॑ति । भ्रातृ॑व्यस्य । ए॒व । तत् । लो॒कम् । वृ॒ङ्क्ते॒ । षट् । अ॒क्षरा॑णि । अतीति॑ । रे॒च॒य॒न्ति॒ । षट् । वै । ऋ॒तवः॑ । ऋ॒तून् । ए॒व । प्री॒णा॒ति॒ । च॒त्वारि॑ । पूर्वा॑णि । अवेति॑ । क॒ल्प॒य॒न्ति॒ । \textbf{  42} \newline
                  \newline
                                \textbf{ TS 6.6.11.6} \newline
                  चतु॑ष्पद॒ इति॒ चतुः॑-प॒दः॒ । ए॒व । प॒शून् । अवेति॑ । रु॒न्धे॒ । द्वे इति॑ । उत्त॑रे॒ इत्युत् - त॒रे॒ । द्वि॒पद॒ इति॑ द्वि - पदः॑ । ए॒व । अवेति॑ । रु॒न्धे॒ । अ॒नु॒ष्टुभ॒मित्य॑नु - स्तुभ᳚म् । अ॒भि । समिति॑ । पा॒द॒य॒न्ति॒ । वाक् । वै । अ॒नु॒ष्टुबित्य॑नु - स्तुप् । तस्मा᳚त् । प्रा॒णाना॒मिति॑ प्र - अ॒नाना᳚म् । वाक् । उ॒त्त॒मेत्यु॑त् - त॒मा । स॒म॒या॒वि॒षि॒त इति॑ समया - वि॒षि॒ते । सूर्ये᳚ । षो॒ड॒शिनः॑ । स्तो॒त्रम् । उ॒पाक॑रो॒तीत्यु॑प - आक॑रोति । ए॒तस्मिन्न्॑ । वै । लो॒के । इन्द्रः॑ । वृ॒त्रम् । अ॒ह॒न्न् । सा॒क्षादिति॑ स - अ॒क्षात् । ए॒व । वज्र᳚म् । भ्रातृ॑व्याय । प्रेति॑ । ह॒र॒ति॒ । अ॒रु॒ण॒पि॒श॒ङ्ग इत्य॑रुण - पि॒श॒ङ्गः । अश्वः॑ । दक्षि॑णा । ए॒तत् । वै । वज्र॑स्य । रू॒पम् । समृ॑द्ध्या॒ इति॒ सं - ऋ॒द्ध्यै॒ ॥ \textbf{  43} \newline
                  \newline
                      (लो॒को - वि॒दुषः॑ षोड॒शी गृ॒ह्यते॒ - यदु॒क्थ्ये॑ - धाम॑ - कल्पयन्ति - स॒प्तच॑त्वारिꣳशच्च)  \textbf{(A11)} \newline \newline
\textbf{praSna korvai with starting padams of 1 to 11 anuvAkams :-} \newline
(सु॒व॒र्गाय॒ यद् दा᳚क्षि॒णानि॑ - समिष्ट य॒जूꣳ - ष्य॑वभृथ य॒जूꣳषि॒ - स्फ्येन॑ - प्र॒जाप॑तिरेकाद॒शिनी॒ - मिन्द्रः॒ पत्नि॑या॒ - घ्नन्ति॑ - दे॒वा वा इ॑न्द्रि॒यं ॅवी॒र्यं॑ - दे॒वा वा अदा᳚भ्ये - दे॒वा वै प्र॒बाहु॑क् - प्र॒जाप॑तिर्दे॒वेभ्यः॒ स रि॑रिचा॒नः - षो॑डश॒धैका॑दश) \newline

\textbf{korvai with starting padams of1, 11, 21 series of pa~jcAtis :-} \newline
(सु॒व॒र्गाय॑ - यजति प्र॒जाः - सौ॒म्येन॑ - गृह्णी॒यात् प्र॒त्यञ्चं॑ - गृह्णी॒यात् प्र॒जां प॒शून् - त्रिच॑त्वारिꣳशत्) \newline

\textbf{first and last padam of sixth praSnam of 6th KANDam :-} \newline
(सु॒व॒र्गाय॒ - वज्र॑स्य रू॒पꣳ समृ॑द्ध्यै) \newline 


॥ हरिः॑ ॐ ॥
॥ कृष्ण यजुर्वेदीय तैत्तिरीय संहितायां षष्ठकाण्डे षष्ठः प्रश्नः समाप्तः ॥

॥ इति षष्ठं काण्डं ॥
======================== \newline
\pagebreak
\pagebreak
        


\end{document}
