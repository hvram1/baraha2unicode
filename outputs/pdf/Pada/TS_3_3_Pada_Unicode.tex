\documentclass[17pt]{extarticle}
\usepackage{babel}
\usepackage{fontspec}
\usepackage{polyglossia}
\usepackage{extsizes}



\setmainlanguage{sanskrit}
\setotherlanguages{english} %% or other languages
\setlength{\parindent}{0pt}
\pagestyle{myheadings}
\newfontfamily\devanagarifont[Script=Devanagari]{AdishilaVedic}


\newcommand{\VAR}[1]{}
\newcommand{\BLOCK}[1]{}




\begin{document}
\begin{titlepage}
    \begin{center}
 
\begin{sanskrit}
    { \Large
    ॐ नमः परमात्मने, श्री महागणपतये नमः, श्री गुरुभ्यो नमः
ह॒रिः॒ ॐ
========================================= 
    }
    \\
    \vspace{2.5cm}
    \mbox{ \Huge
    3.3     तृतीयकाण्डे तृतीयः प्रश्नः - वैकृतविधीनामभिधानं   }
\end{sanskrit}
\end{center}

\end{titlepage}
\tableofcontents

ॐ नमः परमात्मने, श्री महागणपतये नमः, श्री गुरुभ्यो नमः
ह॒रिः॒ ॐ
========================================= \newline
3.3     तृतीयकाण्डे तृतीयः प्रश्नः - वैकृतविधीनामभिधानं \newline

\addcontentsline{toc}{section}{ 3.3     तृतीयकाण्डे तृतीयः प्रश्नः - वैकृतविधीनामभिधानं}
\markright{ 3.3     तृतीयकाण्डे तृतीयः प्रश्नः - वैकृतविधीनामभिधानं \hfill https://www.vedavms.in \hfill}
\section*{ 3.3     तृतीयकाण्डे तृतीयः प्रश्नः - वैकृतविधीनामभिधानं }
                                \textbf{ TS 3.3.1.1} \newline
                  अग्ने᳚ । ते॒ज॒स्वि॒न्न् । ते॒ज॒स्वी । त्वम् । दे॒वेषु॑ । भू॒याः॒ । तेज॑स्वन्तम् । माम् । आयु॑ष्मन्तम् । वर्च॑स्वन्तम् । म॒नु॒ष्ये॑षु । कु॒रु॒ । दी॒क्षायै᳚ । च॒ । त्वा॒ । तप॑सः । च॒ । तेज॑से । जु॒हो॒मि॒ । ते॒जो॒विदिति॑ तेजः - वित् । अ॒सि॒ । तेजः॑ । मा॒ । मा । हा॒सी॒त् । मा । अ॒हम् । तेजः॑ । हा॒सि॒ष॒म् । मा । माम् । तेजः॑ । हा॒सी॒त् । इन्द्र॑ । ओ॒ज॒स्वि॒न्न् । ओ॒ज॒स्वी । त्वम् । दे॒वेषु॑ । भू॒याः॒ । ओज॑स्वन्तम् । माम् । आयु॑ष्मन्तम् । वर्च॑स्वन्तम् । म॒नु॒ष्ये॑षु । कु॒रु॒ । ब्रह्म॑णः । च॒ । त्वा॒ । क्ष॒त्रस्य॑ । च॒ । \textbf{  1} \newline
                  \newline
                                \textbf{ TS 3.3.1.2} \newline
                  ओज॑से । जु॒हो॒मि॒ । ओ॒जो॒विदित्यो॑जः - वित् । अ॒सि॒ । ओजः॑ । मा॒ । मा । हा॒सी॒त् । मा । अ॒हम् । ओजः॑ । हा॒सि॒ष॒म् । मा । माम् । ओजः॑ । हा॒सी॒त् । सूर्य॑ । भ्रा॒ज॒स्वि॒न्न् । भ्रा॒ज॒स्वी । त्वम् । दे॒वेषु॑ । भू॒याः॒ । भ्राज॑स्वन्तम् । माम् । आयु॑ष्मन्तम् । वर्च॑स्वन्तम् । म॒नु॒ष्ये॑षु । कु॒रु॒ । वा॒योः । च॒ । त्वा॒ । अ॒पाम् । च॒ । भ्राज॑से । जु॒हो॒मि॒ । सु॒व॒र्विदिति॑ सुवः - वित् । अ॒सि॒ । सुवः॑ । मा॒ । मा । हा॒सी॒त् । मा । अ॒हम् । सुवः॑ । हा॒सि॒ष॒म् । मा । माम् । सुवः॑ । हा॒सी॒त् । मयि॑ ( ) । मे॒धाम् । मयि॑ । प्र॒जामिति॑ प्र - जाम् । मयि॑ । अ॒ग्निः । तेजः॑ । द॒धा॒तु॒ । मयि॑ । मे॒धाम् । मयि॑ । प्र॒जामिति॑ प्र - जाम् । मयि॑ । इन्द्रः॑ । इ॒न्द्रि॒यम् । द॒धा॒तु॒ । मयि॑ । मे॒धाम् । मयि॑ । प्र॒जामिति॑ प्र - जाम् । मयि॑ । सूर्यः॑ । भ्राजः॑ । द॒धा॒तु॒ ॥ \textbf{  2} \newline
                  \newline
                      (क्ष॒त्रस्य॑ च॒ - मयि॒ - त्रयो॑विꣳशतिश्च)  \textbf{(A1)} \newline \newline
                                \textbf{ TS 3.3.2.1} \newline
                  वा॒युः । हि॒कं॒र्तेति॑ हिं - क॒र्ता । अ॒ग्निः । प्र॒स्तो॒तेति॑ प्र - स्तो॒ता । प्र॒जाप॑ति॒रिति॑ प्र॒जा - प॒तिः॒ । साम॑ । बृह॒स्पतिः॑ । उ॒द्गा॒तेत्यु॑त् - गा॒ता । विश्वे᳚ । दे॒वाः । उ॒प॒गा॒तार॒ इत्यु॑प - गा॒तारः॑ । म॒रुतः॑ । प्र॒ति॒ह॒र्तार॒ इति॑ प्रति - ह॒र्तारः॑ । इन्द्रः॑ । नि॒धन॒मिति॑ नि - धन᳚म् । ते । दे॒वाः । प्रा॒ण॒भृत॒ इति॑ प्राण - भृतः॑ । प्रा॒णमिति॑ प्र - अ॒नम् । मयि॑ । द॒ध॒तु॒ । ए॒तत् । वै । सर्व᳚म् । अ॒द्ध्व॒र्युः । उ॒पा॒कु॒र्वन्नित्यु॑प - आ॒कु॒र्वन्न् । उ॒द्गा॒तृभ्य॒ इत्यु॑द्गा॒तृ - भ्यः॒ । उ॒पाक॑रो॒तीत्यु॑प - आक॑रोति । ते । दे॒वाः । प्रा॒ण॒भृत॒ इति॑ प्राण - भृतः॑ । प्रा॒णमिति॑ प्र-अ॒नम् । मयि॑ । द॒ध॒तु॒ । इति॑ । आ॒ह॒ । ए॒तत् । ए॒व । सर्व᳚म् । आ॒त्मन्न् । ध॒त्ते॒ । इडा᳚ । दे॒व॒हूरिति॑ देव - हूः । मनुः॑ । य॒ज्ञ्॒नीरिति॑ यज्ञ् - नीः । बृह॒स्पतिः॑ । उ॒क्था॒म॒दानीत्यु॑क्थ - म॒दानि॑ । शꣳ॒॒सि॒ष॒त् । विश्वे᳚ । दे॒वाः । \textbf{  3} \newline
                  \newline
                                \textbf{ TS 3.3.2.2} \newline
                  सू॒क्त॒वाच॒ इति॑ सूक्त - वाचः॑ । पृथि॑वि । मा॒तः॒ । मा । मा॒ । हिꣳ॒॒सीः॒ । मधु॑ । म॒नि॒ष्ये॒ । मधु॑ । ज॒नि॒ष्ये॒ । मधु॑ । व॒क्ष्या॒मि॒ । मधु॑ । व॒दि॒ष्या॒मि॒ । मधु॑मती॒मिति॒ मधु॑ - म॒ती॒म् । दे॒वेभ्यः॑ । वाच᳚म् । उ॒द्या॒स॒म् । शु॒श्रू॒षेण्या᳚म् । म॒नु॒ष्ये᳚भ्यः । तम् । मा॒ । दे॒वाः । अ॒व॒न्तु॒ । शो॒भायै᳚ । पि॒तरः॑ । अन्विति॑ । म॒द॒न्तु॒ ॥ \textbf{  4} \newline
                  \newline
                      (शꣳ॒॒सि॒ष॒द् विश्वे॑ दे॒वा - अ॒ष्टाविꣳ॑शतिश्च)  \textbf{(A2)} \newline \newline
                                \textbf{ TS 3.3.3.1} \newline
                  वस॑वः । त्वा॒ । प्रेति॑ । वृ॒ह॒न्तु॒ । गा॒य॒त्रेण॑ । छन्द॑सा । अ॒ग्नेः । प्रि॒यम् । पाथः॑ । उपेति॑ । इ॒हि॒ । रु॒द्राः । त्वा॒ । प्रेति॑ । वृ॒ह॒न्तु॒ । त्रैष्टु॑भेन । छन्द॑सा । इन्द्र॑स्य । प्रि॒यम् । पाथः॑ । उपेति॑ । इ॒हि॒ । आ॒दि॒त्याः । त्वा॒ । प्रेति॑ । वृ॒ह॒न्तु॒ । जाग॑तेन । छन्द॑सा । विश्वे॑षाम् । दे॒वाना᳚म् । प्रि॒यम् । पाथः॑ । उपेति॑ । इ॒हि॒ । मान्दा॑सु । ते॒ । शु॒क्र॒ । शु॒क्रम् । एति॑ । धू॒नो॒मि॒ । भ॒न्दना॑सु । कोत॑नासु । नूत॑नासु । रेशी॑षु । मेषी॑षु । वाशी॑षु । वि॒श्व॒भृथ्स्विति॑ विश्व॒भृत् - सु॒ । माद्ध्वी॑षु । क॒कु॒हासु॑ । शक्व॑रीषु । \textbf{  5} \newline
                  \newline
                                \textbf{ TS 3.3.3.2} \newline
                  शु॒क्रासु॑ । ते॒ । शु॒क्र॒ । शु॒क्रम् । एति॑ । धू॒नो॒मि॒ । शु॒क्रम् । ते॒ । शु॒क्रेण॑ । गृ॒ह्णा॒मि॒ । अह्नः॑ । रू॒पेण॑ । सूर्य॑स्य । र॒श्मिभि॒रिति॑ र॒श्मि-भिः॒ ॥ एति॑ । अ॒स्मि॒न्न् । उ॒ग्राः । अ॒चु॒च्य॒वुः॒ । दि॒वः । धाराः᳚ । अ॒स॒श्च॒त॒ ॥ क॒कु॒हम् । रू॒पम् । वृ॒ष॒भस्य॑ । रो॒च॒ते॒ । बृ॒हत् । सोमः॑ । सोम॑स्य । पु॒रो॒गा इति॑ पुरः - गाः । शु॒क्रः । शु॒क्रस्य॑ । पु॒रो॒गा इति॑ पुरः - गाः ॥ यत् । ते॒ । सो॒म॒ । अदा᳚भ्यम् । नाम॑ । जागृ॑वि । तस्मै᳚ । ते॒ । सो॒म॒ । सोमा॑य । स्वाहा᳚ । उ॒शिक् । त्वम् । दे॒व॒ । सो॒म॒ । गा॒य॒त्रेण॑ । छन्द॑सा । अ॒ग्नेः । \textbf{  6} \newline
                  \newline
                                \textbf{ TS 3.3.3.3} \newline
                  प्रि॒यम् । पाथः॑ । अपीति॑ । इ॒हि॒ । व॒शी । त्वम् । दे॒व॒ । सो॒म॒ । त्रैष्टु॑भेन । छन्द॑सा । इन्द्र॑स्य । प्रि॒यम् । पाथः॑ । अपीति॑ । इ॒हि॒ । अ॒स्मथ्स॒खेत्य॒स्मत् - स॒खा॒ । त्वम् । दे॒व॒ । सो॒म॒ । जाग॑तेन । छन्द॑सा । विश्वे॑षाम् । दे॒वाना᳚म् । प्रि॒यम् । पाथः॑ । अपीति॑ । इ॒हि॒ । एति॑ । नः॒ । प्रा॒ण इति॑ प्र - अ॒नः । ए॒तु॒ । प॒रा॒वत॒ इति॑ परा-वतः॑ । एति॑ । अ॒न्तरि॑क्षात् । दि॒वः । परि॑ ॥ आयुः॑ । पृ॒थि॒व्याः । अधीति॑ । अ॒मृत᳚म् । अ॒सि॒ । प्रा॒णायेति॑ प्र - अ॒नाय॑ । त्वा॒ ॥ इ॒न्द्रा॒ग्नी इती᳚न्द्र-अ॒ग्नी । मे॒ । वर्चः॑ । कृ॒णु॒ता॒म् । वर्चः॑ । सोमः॑ । बृह॒स्पतिः॑ ( ) ॥ वर्चः॑ । मे॒ । विश्वे᳚ । दे॒वाः । वर्चः॑ । मे॒ । ध॒त्त॒म् । अ॒श्वि॒ना॒ ॥ द॒ध॒न्वे । वा॒ । यत् । ई॒म् । अन्विति॑ । वोच॑त् । ब्रह्मा॑णि । वेः । उ॒ । तत् ॥ परीति॑ । विश्वा॑नि । काव्या᳚ । ने॒मिः । च॒क्रम् । इ॒व॒ । अ॒भ॒व॒त् ॥ \textbf{  7} \newline
                  \newline
                      (शक्व॑रीष्व॒ - ग्ने - र्बृह॒स्पतिः॒ - पञ्च॑विꣳशतिश्च)  \textbf{(A3)} \newline \newline
                                \textbf{ TS 3.3.4.1} \newline
                  ए॒तत् । वै । अ॒पाम् । ना॒म॒धेय॒मिति॑ नाम - धेय᳚म् । गुह्य᳚म् । यत् । आ॒धा॒वा इत्या᳚ - धा॒वाः । मान्दा॑सु । ते॒ । शु॒क्र॒ । शु॒क्रम् । एति॑ । धू॒नो॒मि॒ । इति॑ । आ॒ह॒ । अ॒पाम् । ए॒व । ना॒म॒धेये॒नेति॑ नाम - धेये॑न । गुह्ये॑न । दि॒वः । वृष्टि᳚म् । अवेति॑ । रु॒न्धे॒ । शु॒क्रम् । ते॒ । शु॒क्रेण॑ । गृ॒ह्णा॒मि॒ । इति॑ । आ॒ह॒ । ए॒तत् । वै । अह्नः॑ । रू॒पम् । यत् । रात्रिः॑ । सूर्य॑स्य । र॒श्मयः॑ । वृष्ट्याः᳚ । ई॒श॒ते॒ । अह्नः॑ । ए॒व । रू॒पेण॑ । सूर्य॑स्य । र॒श्मिभि॒रिति॑ र॒श्मि - भिः॒ । दि॒वः । वृष्टि᳚म् । च्या॒व॒य॒ति॒ । एति॑ । अ॒स्मि॒न्न् । उ॒ग्राः । \textbf{  8} \newline
                  \newline
                                \textbf{ TS 3.3.4.2} \newline
                  अ॒चु॒च्य॒वुः॒ । इति॑ । आ॒ह॒ । य॒था॒य॒जुरिति॑ यथा-य॒जुः । ए॒व । ए॒तत् । क॒कु॒हम् । रू॒पम् । वृ॒ष॒भस्य॑ । रो॒च॒ते॒ । बृ॒हत् । इति॑ । आ॒ह॒ । ए॒तत् । वै । अ॒स्य॒ । क॒कु॒हम् । रू॒पम् । यत् । वृष्टिः॑ । रू॒पेण॑ । ए॒व । वृष्टि᳚म् । अवेति॑ । रु॒न्धे॒ । यत् । ते॒ । सो॒म॒ । अदा᳚भ्यम् । नाम॑ । जागृ॑वि । इति॑ । आ॒ह॒ । ए॒षः । ह॒ । वै । ह॒विषा᳚ । ह॒विः । य॒ज॒ति॒ । यः । अदा᳚भ्यम् । गृ॒ही॒त्वा । सोमा॑य । जु॒होति॑ । परेति॑ । वै । ए॒तस्य॑ । आयुः॑ । प्रा॒ण इति॑ प्र - अ॒नः । ए॒ति॒ । \textbf{  9} \newline
                  \newline
                                \textbf{ TS 3.3.4.3} \newline
                  यः । अꣳ॒॒शुम् । गृ॒ह्णाति॑ । एति॑ । नः॒ । प्रा॒ण इति॑ प्र - अ॒नः । ए॒तु॒ । प॒रा॒वत॒ इति॑ परा - वतः॑ । इति॑ । आ॒ह॒ । आयुः॑ । ए॒व । प्रा॒णमिति॑ प्र - अ॒नम् । आ॒त्मन्न् । ध॒त्ते॒ । अ॒मृत᳚म् । अ॒सि॒ । प्रा॒णायेति॑ प्र - अ॒नाय॑ । त्वा॒ । इति॑ । हिर॑ण्यम् । अ॒भि । वीति॑ । अ॒नि॒ति॒ । अ॒मृत᳚म् । वै । हिर॑ण्यम् । आयुः॑ । प्रा॒ण इति॑ प्र - अ॒नः । अ॒मृते॑न । ए॒व । आयुः॑ । आ॒त्मन्न् । ध॒त्ते॒ । श॒तमा॑न॒मिति॑ श॒त - मा॒न॒म् । भ॒व॒ति॒ । श॒तायु॒रिति॑ श॒त - आ॒युः॒ । पुरु॑षः । श॒तेन्द्रि॑य॒ इति॑ श॒त - इ॒न्द्रि॒यः॒ । आयु॑षि । ए॒व । इ॒न्द्रि॒ये । प्रतीति॑ । ति॒ष्ठ॒ति॒ । अ॒पः । उपेति॑ । स्पृ॒श॒ति॒ । भे॒ष॒जम् । वै । आपः॑ ( ) । भे॒ष॒जम् । ए॒व । कु॒रु॒ते॒ ॥ \textbf{  10 } \newline
                  \newline
                      (उ॒ग्रा - ए॒त्या - प॒ - स्त्रीणि॑ च)  \textbf{(A4)} \newline \newline
                                \textbf{ TS 3.3.5.1} \newline
                  वा॒युः । अ॒सि॒ । प्रा॒ण इति॑ प्र - अ॒नः । नाम॑ । स॒वि॒तुः । आधि॑पत्य॒ इत्याधि॑ - प॒त्ये॒ । अ॒पा॒नमित्य॑प - अ॒नम् । मे॒ । दाः॒ । चक्षुः॑ । अ॒सि॒ । श्रोत्र᳚म् । नाम॑ । धा॒तुः । आधि॑पत्य॒ इत्याधि॑-प॒त्ये॒ । आयुः॑ । मे॒ । दाः॒ । रू॒पम् । अ॒सि॒ । वर्णः॑ । नाम॑ । बृह॒स्पतेः᳚ । आधि॑पत्य॒ इत्याधि॑ - प॒त्ये॒ । प्र॒जामिति॑ प्र-जाम् । मे॒ । दाः॒ । ऋ॒तम् । अ॒सि॒ । स॒त्यम् । नाम॑ । इन्द्र॑स्य । आधि॑पत्य॒ इत्याधि॑ - प॒त्ये॒ । क्ष॒त्रम् । मे॒ । दाः॒ । भू॒तम् । अ॒सि॒ । भव्य᳚म् । नाम॑ । पि॒तृ॒णाम् । आधि॑पत्य॒ इत्याधि॑ - प॒त्ये॒ । अ॒पाम् । ओष॑धीनाम् । गर्भ᳚म् । धाः॒ । ऋ॒तस्य॑ । त्वा॒ । व्यो॑मन॒ इति॒ वि - ओ॒म॒ने॒ । ऋ॒तस्य॑ । \textbf{  11} \newline
                  \newline
                                \textbf{ TS 3.3.5.2} \newline
                  त्वा॒ । विभू॑मन॒ इति॒ वि - भू॒म॒ने॒ । ऋ॒तस्य॑ । त्वा॒ । विध॑र्मण॒ इति॒ वि - ध॒र्म॒णे॒ । ऋ॒तस्य॑ । त्वा॒ । स॒त्याय॑ । ऋ॒तस्य॑ । त्वा॒ । ज्योति॑षे । प्र॒जाप॑ति॒रिति॑ प्र॒जा - प॒तिः॒ । वि॒राज॒मिति॑ वि - राज᳚म् । अ॒प॒श्य॒त् । तया᳚ । भू॒तम् । च॒ । भव्य᳚म् । च॒ । अ॒सृ॒ज॒त॒ । ताम् । ऋषि॑भ्य॒ इत्यृषि॑ - भ्यः॒ । ति॒रः । अ॒द॒धा॒त् । ताम् । ज॒मद॑ग्निः । तप॑सा । अ॒प॒श्य॒त् । तया᳚ । वै । सः । पृश्नीन्॑ । कामान्॑ । अ॒सृ॒ज॒त॒ । तत् । पृश्नी॑नाम् । पृ॒श्नि॒त्वमिति॑ पृश्नि - त्वम् । यत् । पृश्न॑यः । गृ॒ह्यन्ते᳚ । पृश्नीन्॑ । ए॒व । तैः । कामान्॑ । यज॑मानः । अवेति॑ । रु॒न्धे॒ । वा॒युः । अ॒सि॒ । प्रा॒ण इति॑ प्र - अ॒नः । \textbf{  12} \newline
                  \newline
                                \textbf{ TS 3.3.5.3} \newline
                  नाम॑ । इति॑ । आ॒ह॒ । प्रा॒णा॒पा॒नाविति॑ प्राण-अ॒पा॒नौ । ए॒व । अवेति॑ । रु॒न्धे॒ । चक्षुः॑ । अ॒सि॒ । श्रोत्र᳚म् । नाम॑ । इति॑ । आ॒ह॒ । आयुः॑ । ए॒व । अवेति॑ । रु॒न्धे॒ । रू॒पम् । अ॒सि॒ । वर्णः॑ । नाम॑ । इति॑ । आ॒ह॒ । प्र॒जामिति॑ प्र - जाम् । ए॒व । अवेति॑ । रु॒न्धे॒ । ऋ॒तम् । अ॒सि॒ । स॒त्यम् । नाम॑ । इति॑ । आ॒ह॒ । क्ष॒त्रम् । ए॒व । अवेति॑ । रु॒न्धे॒ । भू॒तम् । अ॒सि॒ । भव्य᳚म् । नाम॑ । इति॑ । आ॒ह॒ । प॒शवः॑ । वै । अ॒पाम् । ओष॑धीनाम् । गर्भः॑ । प॒शून् । ए॒व । \textbf{  13} \newline
                  \newline
                                \textbf{ TS 3.3.5.4} \newline
                  अवेति॑ । रु॒न्धे॒ । ए॒ताव॑त् । वै । पुरु॑षम् । प॒रितः॑ । तत् । ए॒व । अवेति॑ । रु॒न्धे॒ । ऋ॒तस्य॑ । त्वा॒ । व्यो॑मन॒ इति॒ वि - ओ॒म॒ने॒ । इति॑ । आ॒ह॒ । इ॒यम् । वै । ऋ॒तस्य॑ । व्यो॑मेति॒ वि - ओ॒म॒ । इ॒माम् । ए॒व । अ॒भीति॑ । ज॒य॒ति॒ । ऋ॒तस्य॑ । त्वा॒ । विभू॑मन॒ इति॒ वि-भू॒म॒ने॒ । इति॑ । आ॒ह॒ । अ॒न्तरि॑क्षम् । वै । ऋ॒तस्य॑ । विभू॒मेति॒ वि - भू॒म॒ । अ॒न्तरि॑क्षम् । ए॒व । अ॒भीति॑ । ज॒य॒ति॒ । ऋ॒तस्य॑ । त्वा॒ । विध॑र्मण॒ इति॒ वि - ध॒र्म॒णे॒ । इति॑ । आ॒ह॒ । द्यौः । वै । ऋ॒तस्य॑ । विध॒र्मेति॒ वि -ध॒र्म॒ । दिव᳚म् । ए॒व । अ॒भीति॑ । ज॒य॒ति॒ । ऋ॒तस्य॑ । \textbf{  14} \newline
                  \newline
                                \textbf{ TS 3.3.5.5} \newline
                  त्वा॒ । स॒त्याय॑ । इति॑ । आ॒ह॒ । दिशः॑ । वै । ऋ॒तस्य॑ । स॒त्यम् । दिशः॑ । ए॒व । अ॒भीति॑ । ज॒य॒ति॒ । ऋ॒तस्य॑ । त्वा॒ । ज्योति॑षे । इति॑ । आ॒ह॒ । सु॒व॒र्ग इति॑ सुवः - गः । वै । लो॒कः । ऋ॒तस्य॑ । ज्योतिः॑ । सु॒व॒र्गमिति॑ सुवः - गम् । ए॒व । लो॒कम् । अ॒भीति॑ । ज॒य॒ति॒ । ए॒ताव॑न्तः । वै । दे॒व॒लो॒का इति॑ देव - लो॒काः । तान् । ए॒व । अ॒भीति॑ । ज॒य॒ति॒ । दश॑ । समिति॑ । प॒द्य॒न्ते॒ । दशा᳚क्ष॒रेति॒ दश॑ - अ॒क्ष॒रा॒ । वि॒राडिति॑ वि - राट् । अन्न᳚म् । वि॒राडिति॑ वि - राट् । वि॒राजीति॑ वि - राजि॑ । ए॒व । अ॒न्नाद्य॒ इत्य॑न्न - अद्ये᳚ । प्रतीति॑ । ति॒ष्ठ॒ति॒ ॥ \textbf{  15} \newline
                  \newline
                      (व्यो॑मन ऋ॒तस्य॑ - प्रा॒णः - प॒शुने॒व - विध॑र्म॒ दिव॑मे॒वाभि ज॑यत्यृ॒तस्य॒ -षट्च॑त्वारिꣳशच्च)  \textbf{(A5)} \newline \newline
                                \textbf{ TS 3.3.6.1} \newline
                  दे॒वाः । वै । यत् । य॒ज्ञेन॑ । न । अ॒वारु॑न्ध॒तेत्य॑व - अरु॑न्धत । तत् । परैः᳚ । अवेति॑ । अ॒रु॒न्ध॒त॒ । तत् । परा॑णाम् । प॒र॒त्वमिति॑ पर - त्वम् । यत् । परे᳚ । गृ॒ह्यन्ते᳚ । यत् । ए॒व । य॒ज्ञेन॑ । न । अ॒व॒रु॒न्ध इत्यव॑-रु॒न्धे । तस्य॑ । अव॑रुद्ध्या॒ इत्यव॑ -रु॒द्ध्यै॒ । यम् । प्र॒थ॒मम् । गृ॒ह्णाति॑ । इ॒मम् । ए॒व । तेन॑ । लो॒कम् । अ॒भीति॑ । ज॒य॒ति॒ । यम् । द्वि॒तीय᳚म् । अ॒न्तरि॑क्षम् । तेन॑ । यम् । तृ॒तीय᳚म् । अ॒मुम् । ए॒व । तेन॑ । लो॒कम् । अ॒भीति॑ । ज॒य॒ति॒ । यत् । ए॒ते । गृ॒ह्यन्ते᳚ । ए॒षाम् । लो॒काना᳚म् । अ॒भिजि॑त्या॒ इत्य॒भि - जि॒त्यै॒ । \textbf{  16} \newline
                  \newline
                                \textbf{ TS 3.3.6.2} \newline
                  उत्त॑रे॒ष्वित्युत् - त॒रे॒षु॒ । अहः॒ स्वित्यहः॑-सु॒ । अ॒मुतः॑ । अ॒र्वाञ्चः॑ । गृ॒ह्य॒न्ते॒ । अ॒भि॒जित्येत्य॑भि-जित्य॑ । ए॒व । इ॒मान् । लो॒कान् । पुनः॑ । इ॒मम् । लो॒कम् । प्र॒त्यव॑रोह॒न्तीति॑ प्रति - अव॑रोहन्ति । यत् । पूर्वे॑षु । अहः॒ स्वित्यहः॑ - सु॒ । इ॒तः । परा᳚ञ्चः । गृ॒ह्यन्ते᳚ । तस्मा᳚त् । इ॒तः । परा᳚ञ्चः । इ॒मे । लो॒काः । यत् । उत्त॑रे॒ष्वित्युत् - त॒रे॒षु॒ । अहः॒ स्वित्यहः॑ - सु॒ । अ॒मुतः॑ । अ॒र्वाञ्चः॑ । गृ॒ह्यन्ते᳚ । तस्मा᳚त् । अ॒मुतः॑ । अ॒र्वाञ्चः॑ । इ॒मे । लो॒काः । तस्मा᳚त् । अया॑तयाम्न॒ इत्यया॑त-या॒म्नः॒ । लो॒कान् । म॒नु॒ष्याः᳚ । उपेति॑ । जी॒व॒न्ति॒ । ब्र॒ह्म॒वा॒दिन॒ इति॑ ब्रह्म - वा॒दिनः॑ । व॒द॒न्ति॒ । कस्मा᳚त् । स॒त्यात् । अ॒द्भ्य इत्य॑त् - भ्यः । ओष॑धयः । समिति॑ । भ॒व॒न्ति॒ । ओष॑धयः । \textbf{  17} \newline
                  \newline
                                \textbf{ TS 3.3.6.3} \newline
                  म॒नु॒ष्या॑णाम् । अन्न᳚म् । प्र॒जाप॑ति॒मिति॑ प्र॒जा - प॒ति॒म् । प्र॒जा इति॑ प्र - जाः । अनु॑ । प्रेति॑ । जा॒य॒न्ते॒ । इति॑ । परान्॑ । अन्विति॑ । इति॑ । ब्रू॒या॒त् । यत् । गृ॒ह्णाति॑ । अ॒द्भ्य इत्य॑त् - भ्यः । त्वा॒ । ओष॑धीभ्य॒ इत्योष॑धि - भ्यः॒ । गृ॒ह्णा॒मि॒ । इति॑ । तस्मा᳚त् । अ॒द्भ्य इत्य॑त्-भ्यः । ओष॑धयः । समिति॑ । भ॒व॒न्ति॒ । यत् । गृ॒ह्णाति॑ । ओष॑धीभ्य॒ इत्योष॑धि - भ्यः॒ । त्वा॒ । प्र॒जाभ्य॒ इति॑ प्र - जाभ्यः॑ । गृ॒ह्णा॒मि॒ । इति॑ । तस्मा᳚त् । ओष॑धयः । म॒नु॒ष्या॑णाम् । अन्न᳚म् । यत् । गृ॒ह्णाति॑ । प्र॒जाभ्य॒ इति॑ प्र - जाभ्यः॑ । त्वा॒ । प्र॒जाप॑तय॒ इति॑ प्र॒जा - प॒त॒ये॒ । गृ॒ह्णा॒मि॒ । इति॑ । तस्मा᳚त् । प्र॒जाप॑ति॒मिति॑ प्र॒जा - प॒ति॒म् । प्र॒जा इति॑ प्र - जाः । अनु॑ । प्रेति॑ । जा॒य॒न्ते॒ ॥ \textbf{  18} \newline
                  \newline
                      (अ॒भिजि॑त्यै - भव॒न्त्योष॑धयो॒ - ऽष्टा च॑त्वारिꣳशच्च)  \textbf{(A6)} \newline \newline
                                \textbf{ TS 3.3.7.1} \newline
                  प्र॒जाप॑ति॒रिति॑ प्र॒जा - प॒तिः॒ । दे॒वा॒सु॒रानिति॑ देव - अ॒सु॒रान् । अ॒सृ॒ज॒त॒ । तत् । अन्विति॑ । य॒ज्ञ्ः । अ॒सृ॒ज्य॒त॒ । य॒ज्ञ्म् । छन्दाꣳ॑सि । ते । विष्व॑ञ्चः । वीति॑ । अ॒क्रा॒म॒न्न् । सः । असु॑रान् । अन्विति॑ । य॒ज्ञ्ः । अपेति॑ । अ॒क्रा॒म॒त् । य॒ज्ञ्म् । छन्दाꣳ॑सि । ते । दे॒वाः । अ॒म॒न्य॒न्त॒ । अ॒मी इति॑ । वै । इ॒दम् । अ॒भू॒व॒न्न् । यत् । व॒यम् । स्मः । इति॑ । ते । प्र॒जाप॑ति॒मिति॑ प्र॒जा-प॒ति॒म् । उपेति॑ । अ॒धा॒व॒न्न् । सः । अ॒ब्र॒वी॒त् । प्र॒जाप॑ति॒रिति॑ प्र॒जा - प॒तिः॒ । छन्द॑साम् । वी॒र्य᳚म् । आ॒दायेत्या᳚ - दाय॑ । तत् । वः॒ । प्रेति॑ । दा॒स्या॒मि॒ । इति॑ । सः । छन्द॑साम् । वी॒र्य᳚म् । \textbf{  19} \newline
                  \newline
                                \textbf{ TS 3.3.7.2} \newline
                  आ॒दायेत्या᳚-दाय॑ । तत् । ए॒भ्यः॒ । प्रेति॑ । अ॒य॒च्छ॒त् । तत् । अन्विति॑ । छन्दाꣳ॑सि । अपेति॑ । अ॒क्रा॒म॒न्न् । छन्दाꣳ॑सि । य॒ज्ञ्ः । ततः॑ । दे॒वाः । अभ॑वन्न् । परेति॑ । असु॑राः । यः । ए॒वम् । छन्द॑साम् । वी॒र्य᳚म् । वेद॑ । एति॑ । श्रा॒व॒य॒ । अस्तु॑ । श्रौष॑ट् । यज॑ । ये । यजा॑महे । व॒ष॒ट्का॒र इति॑ वषट् - का॒रः । भव॑ति । आ॒त्मना᳚ । परेति॑ । अ॒स्य॒ । भ्रातृ॑व्यः । भ॒व॒ति॒ । ब्र॒ह्म॒वा॒दिन॒ इति॑ ब्रह्म - वा॒दिनः॑ । व॒द॒न्ति॒ । कस्मै᳚ । कम् । अ॒द्ध्व॒र्युः । एति॑ । श्रा॒व॒य॒ति॒ । इति॑ । छन्द॑साम् । वी॒र्या॑य । इति॑ । ब्रू॒या॒त् । ए॒तत् । वै । \textbf{  20} \newline
                  \newline
                                \textbf{ TS 3.3.7.3} \newline
                  छन्द॑साम् । वी॒र्य᳚म् । एति॑ । श्रा॒व॒य॒ । अस्तु॑ । श्रौष॑ट् । यज॑ । ये । यजा॑महे । व॒ष॒ट्का॒र इति॑ वषट् - का॒रः । यः । ए॒वम् । वेद॑ । सवी᳚र्यै॒रिति॒ स - वी॒र्यैः॒ । ए॒व । छन्दो॑भि॒रिति॒ छन्दः॑ - भिः॒ । अ॒र्च॒ति॒ । यत् । किम् । च॒ । अर्च॑ति । यत् । इन्द्रः॑ । वृ॒त्रम् । अहन्न्॑ । अ॒मे॒द्ध्यम् । तत् । यत् । यतीन्॑ । अ॒पाव॑प॒दित्य॑प - अव॑पत् । अ॒मे॒द्ध्यम् । तत् । अथ॑ । कस्मा᳚त् । ऐ॒न्द्रः । य॒ज्ञ्ः । एति॑ । सꣳस्था॑तो॒रिति॒ सं - स्था॒तोः॒ । इति॑ । आ॒हुः॒ । इन्द्र॑स्य । वै । ए॒षा । य॒ज्ञिया᳚ । त॒नूः । यत् । य॒ज्ञ्ः । ताम् । ए॒व । तत् ( ) । य॒ज॒न्ति॒ । यः । ए॒वम् । वेद॑ । उपेति॑ । ए॒न॒म् । य॒ज्ञ्ः । न॒म॒ति॒ ॥ \textbf{  21} \newline
                  \newline
                      (छन्द॑सां ॅवी॒र्यं॑ - ॅवा - ए॒व त - द॒ष्टौ च॑)  \textbf{(A7)} \newline \newline
                                \textbf{ TS 3.3.8.1} \newline
                  आ॒यु॒र्दा इत्या॑युः - दाः । अ॒ग्ने॒ । ह॒विषः॑ । जु॒षा॒णः । घृ॒तप्र॑तीक॒ इति॑ घृ॒त - प्र॒ती॒कः॒ । घृ॒तयो॑नि॒रिति॑ घृ॒त-यो॒निः॒ । ए॒धि॒ ॥ घृ॒तम् । पी॒त्वा । मधु॑ । चारु॑ । गव्य᳚म् । पि॒ता । इ॒व॒ । पु॒त्रम् । अ॒भीति॑ । र॒क्ष॒ता॒त् । इ॒मम् ॥ एति॑ । वृ॒श्च्य॒ते॒ । वै । ए॒तत् । यज॑मानः । अ॒ग्निभ्या॒मित्य॒ग्नि - भ्या॒म् । यत् । ए॒न॒योः॒ । शृ॒त॒कृंत्येति॑ शृतं - कृत्य॑ । अथ॑ । अ॒न्यत्र॑ । अ॒व॒भृ॒थमित्य॑व - भृ॒थम् । अ॒वैतीत्य॑व - एति॑ । आ॒यु॒र्दा इत्या॑युः - दाः । अ॒ग्ने॒ । ह॒विषः॑ । जु॒षा॒णः । इति॑ । अ॒व॒भृ॒थमित्य॑व - भृ॒थम् । अ॒वै॒ष्यन्नित्य॑व-ए॒ष्यन्न् । जु॒हु॒या॒त् । आहु॒त्येत्या - हु॒त्या॒ । ए॒व । ए॒नौ॒ । श॒म॒य॒ति॒ । न । आर्ति᳚म् । एति॑ । ऋ॒च्छ॒ति॒ । यज॑मानः । यत् । कुसी॑दम् । \textbf{  22} \newline
                  \newline
                                \textbf{ TS 3.3.8.2} \newline
                  अप्र॑तीत्त॒मित्यप्र॑ति-इ॒त्त॒म् । मयि॑ । येन॑ । य॒मस्य॑ । ब॒लिना᳚ । चरा॑मि ॥ इ॒ह । ए॒व । सन्न् । नि॒रव॑दय॒ इति॑ निः-अव॑दये । तत् । ए॒तत् । तत् । अ॒ग्ने॒ । अ॒नृ॒णः । भ॒वा॒मि॒ ॥ विश्व॑लो॒पेति॒ विश्व॑ - लो॒प॒ । वि॒श्व॒दा॒वस्येति॑ विश्व - दा॒वस्य॑ । त्वा॒ । आ॒सन्न् । जु॒हो॒मि॒ । अ॒ग्धादित्य॑ग्ध - अत् । एकः॑ । अ॒हु॒तादित्य॑हुत - अत् । एकः॑ । स॒म॒स॒नादिति॑ समसन - अत् । एकः॑ ॥ ते । नः॒ । कृ॒ण्व॒न्तु॒ । भे॒ष॒जम् । सदः॑ । सहः॑ । वरे᳚ण्यम् ॥ अ॒यम् । नः॒ । नभ॑सा । पु॒रः । सꣳ॒॒स्फान॒ इति॑ सं - स्फानः॑ । अ॒भीति॑ । र॒क्ष॒तु॒ ॥ गृ॒हाणा᳚म् । अस॑मर्त्या॒ इत्यसं᳚ - ऋ॒त्यै॒ । ब॒हवः॑ । नः॒ । गृ॒हाः । अ॒स॒न्न् ॥ सः । त्वम् । नः॒ । \textbf{  23} \newline
                  \newline
                                \textbf{ TS 3.3.8.3} \newline
                  न॒भ॒सः॒ । प॒ते॒ । ऊर्ज᳚म् । नः॒ । धे॒हि॒ । भ॒द्रया᳚ ॥ पुनः॑ । नः॒ । न॒ष्टम् । एति॑ । कृ॒धि॒ । पुनः॑ । नः॒ । र॒यिम् । एति॑ । कृ॒धि॒ ॥ देव॑ । सꣳ॒॒स्फा॒नेति॑ सं - स्फा॒न॒ । स॒ह॒स्र॒पो॒षस्येति॑ सहस्र - पो॒षस्य॑ । ई॒शि॒षे॒ । सः । नः॒ । रा॒स्व॒ । अज्या॑निम् । रा॒यः । पोष᳚म् । सु॒वीर्य॒मिति॑ सु - वीर्य᳚म् । सं॒ॅव॒थ्स॒रीणा॒मिति॑ सं - व॒थ्स॒रीणा᳚म् । स्व॒स्तिम् ॥ अ॒ग्निः । वाव । य॒मः । इ॒यम् । य॒मी । कुसी॑दम् । वै । ए॒तत् । य॒मस्य॑ । यज॑मानः । एति॑ । द॒त्ते॒ । यत् । ओष॑धीभि॒रित्योष॑धि - भिः॒ । वेदि᳚म् । स्तृ॒णाति॑ । यत् । अनु॑पौ॒ष्येत्यनु॑प - ओ॒ष्य॒ । प्र॒या॒यादिति॑ प्र - या॒यात् । ग्री॒व॒ब॒द्धमिति॑ ग्रीव - ब॒द्धम् । ए॒न॒म् । \textbf{  24} \newline
                  \newline
                                \textbf{ TS 3.3.8.4} \newline
                  अ॒मुष्मिन्न्॑ । लो॒के । ने॒नी॒ये॒र॒न्न् । यत् । कुसी॑दम् । अप्र॑तीत्त॒मित्यप्र॑ति - इ॒त्त॒म् । मयि॑ । इति॑ । उपेति॑ । ओ॒ष॒ति॒ । इ॒ह । ए॒व । सन्न् । य॒मम् । कुसी॑दम् । नि॒र॒व॒दायेति॑ निः - अ॒व॒दाय॑ । अ॒नृ॒णः । सु॒व॒र्गमिति॑ सुवः-गम् । लो॒कम् । ए॒ति॒ । यदि॑ । मि॒श्रम् । इ॒व॒ । चरे᳚त् । अ॒ञ्ज॒लिना᳚ । सक्तून्॑ । प्र॒दा॒व्य॑ इति॑ प्र - दा॒व्ये᳚ । जु॒हु॒या॒त् । ए॒षः । वै । अ॒ग्निः । वै॒श्वा॒न॒रः । यत् । प्र॒दा॒व्य॑ इति॑ प्र - दा॒व्यः॑ । सः । ए॒व । ए॒न॒म् । स्व॒द॒य॒ति॒ । अह्ना᳚म् । वि॒धान्या॒मिति॑ वि - धान्या᳚म् । ऐ॒का॒ष्ट॒काया॒मित्ये॑क - अ॒ष्ट॒काया᳚म् । अ॒पू॒पम् । चतुः॑ शराव॒मिति॒ चतुः॑ - श॒रा॒व॒म् । प॒क्त्वा । प्रा॒तः । ए॒तेन॑ । कक्ष᳚म् । उपेति॑ । ओ॒षे॒त् । यदि॑ । \textbf{  25} \newline
                  \newline
                                \textbf{ TS 3.3.8.5} \newline
                  दह॑ति । पु॒ण्य॒सम॒मिति॑ पुण्य - सम᳚म् । भ॒व॒ति॒ । यदि॑ । न । दह॑ति । पा॒प॒सम॒मिति॑ पाप - सम᳚म् । ए॒तेन॑ । ह॒ । स्म॒ । वै । ऋष॑यः । पु॒रा । वि॒ज्ञाने॒नेति॑ वि - ज्ञाने॑न । दी॒र्घ॒स॒त्रमिति॑ दीर्घ - स॒त्रम् । उपेति॑ । य॒न्ति॒ । यः । वै । उ॒प॒द्र॒ष्टार॒मित्यु॑प - द्र॒ष्टार᳚म् । उ॒प॒श्रो॒तार॒मित्यु॑प - श्रो॒तार᳚म् । अ॒नु॒ख्या॒तार॒मित्य॑नु - ख्या॒तार᳚म् । वि॒द्वान् । यज॑ते । समिति॑ । अ॒मुष्मिन्न्॑ । लो॒के । इ॒ष्टा॒पू॒र्तेनेती᳚ष्ट - पू॒र्तेन॑ । ग॒च्छ॒ते॒ । अ॒ग्निः । वै । उ॒प॒द्र॒ष्टेत्यु॑प-द्र॒ष्टा । वा॒युः । उ॒प॒श्रो॒तेत्यु॑प - श्रो॒ता । आ॒दि॒त्यः । अ॒नु॒ख्या॒तेत्य॑नु-ख्या॒ता । तान् । यः । ए॒वम् । वि॒द्वान् । यज॑ते । समिति॑ । अ॒मुष्मिन्न्॑ । लो॒के । इ॒ष्टा॒पू॒र्तेनेती᳚ष्ट - पू॒र्तेन॑ । ग॒च्छ॒ते॒ । अ॒यम् । नः॒ । नभ॑सा । पु॒रः । \textbf{  26} \newline
                  \newline
                                \textbf{ TS 3.3.8.6} \newline
                  इति॑ । आ॒ह॒ । अ॒ग्निः । वै । नभ॑सा । पु॒रः । अ॒ग्निम् । ए॒व । तत् । आ॒ह॒ । ए॒तत् । मे॒ । गो॒पा॒य॒ । इति॑ । सः । त्वम् । नः॒ । न॒भ॒सः॒ । प॒ते॒ । इति॑ । आ॒ह॒ । वा॒युः । वै । नभ॑सः । पतिः॑ । वा॒युम् । ए॒व । तत् । आ॒ह॒ । ए॒तत् । मे॒ । गो॒पा॒य॒ । इति॑ । देव॑ । सꣳ॒॒स्फा॒नेति॑ सं - स्फा॒न॒ । इति॑ । आ॒ह॒ । अ॒सौ । वै । आ॒दि॒त्यः । दे॒वः । सꣳ॒॒स्फान॒ इति॑ सं - स्फानः॑ । आ॒दि॒त्यम् । ए॒व । तत् । आ॒ह॒ । ए॒तत् । मे॒ । गो॒पा॒य॒ । इति॑ ( ) ॥ \textbf{  27} \newline
                  \newline
                      (कुसी॑दं॒ - त्वं न॑ - एन - मोषे॒द्यदि॑ - पु॒र - आ॑दि॒त्यमे॒व तदा॑है॒तन्मे॑ गोपा॒येति॑)  \textbf{(A8)} \newline \newline
                                \textbf{ TS 3.3.9.1} \newline
                  ए॒तम् । युवा॑नम् । परीति॑ । वः॒ । द॒दा॒मि॒ । तेन॑ । क्रीड॑न्तीः । च॒र॒त॒ । प्रि॒येण॑ ॥ मा । नः॒ । शा॒प्त॒ । ज॒नुषा᳚ । सु॒भा॒गा॒ इति॑ सु - भा॒गाः॒ । रा॒यः । पोषे॑ण । समिति॑ । इ॒षा । म॒दे॒म॒ ॥ नमः॑ । म॒हि॒म्ने । उ॒त । चक्षु॑षे । ते॒ । मरु॑ताम् । पि॒तः॒ । तत् । अ॒हम् । गृ॒णा॒मि॒ ॥ अन्विति॑ । म॒न्य॒स्व॒ । सु॒यजेति॑ सु-यजा᳚ । य॒जा॒म॒ । जुष्ट᳚म् । दे॒वाना᳚म् । इ॒दम् । अ॒स्तु॒ । ह॒व्यम् ॥ दे॒वाना᳚म् । ए॒षः । उ॒प॒ना॒ह इत्यु॑प - ना॒हः । आ॒सी॒त् । अ॒पाम् । गर्भः॑ । ओष॑धीषु । न्य॑क्त॒ इति॒ नि - अ॒क्तः॒ ॥ सोम॑स्य । द्र॒फ्सम् । अ॒वृ॒णी॒त॒ । पू॒षा । \textbf{  28} \newline
                  \newline
                                \textbf{ TS 3.3.9.2} \newline
                  बृ॒हन्न् । अद्रिः॑ । अ॒भ॒व॒त् । तत् । ए॒षा॒म् ॥ पि॒ता । व॒थ्साना᳚म् । पतिः॑ । अ॒घ्नि॒याना᳚म् । अथो॒ इति॑ । पि॒ता । म॒ह॒ताम् । गर्ग॑राणाम् ॥ व॒थ्सः । ज॒रायु॑ । प्र॒ति॒धुगिति॑ प्रति - धुक् । पी॒यूषः॑ । आ॒मिक्षा᳚ । मस्तु॑ । घृ॒तम् । अ॒स्य॒ । रेतः॑ ॥ त्वाम् । गावः॑ । अ॒वृ॒ण॒त॒ । रा॒ज्याय॑ । त्वाम् । ह॒व॒न्त॒ । म॒रुतः॑ । स्व॒र्का इति॑ सु - अ॒र्काः ॥ वर्ष्मन्न्॑ । क्ष॒त्रस्य॑ । क॒कुभि॑ । शि॒श्रि॒या॒णः । ततः॑ । नः॒ । उ॒ग्रः । वीति॑ । भ॒ज॒ । वसू॑नि ॥ व्यृ॑द्धे॒नेति॒ वि - ऋ॒द्धे॒न॒ । वै । ए॒षः । प॒शुना᳚ । य॒ज॒ते॒ । यस्य॑ । ए॒तानि॑ । न । क्रि॒यन्ते᳚ । ए॒षः ( ) । ह॒ । तु । वै । समृ॑द्धे॒नेति॒ सं - ऋ॒द्धे॒न॒ । य॒ज॒ते॒ । यस्य॑ । ए॒तानि॑ । क्रि॒यन्ते᳚ ॥ \textbf{  29} \newline
                  \newline
                      (पू॒षा - क्रि॒यन्त॑ ए॒षो᳚ - ऽष्टौ च॑)  \textbf{(A9)} \newline \newline
                                \textbf{ TS 3.3.10.1} \newline
                  सूर्यः॑ । दे॒वः । दि॒वि॒षद्भ्य॒ इति॑ दिवि॒षत् - भ्यः॒ । धा॒ता । क्ष॒त्राय॑ । वा॒युः । प्र॒जाभ्य॒ इति॑ प्र - जाभ्यः॑ ॥ बृह॒स्पतिः॑ । त्वा॒ । प्र॒जाप॑तय॒ इति॑ प्र॒जा-प॒त॒ये॒ । ज्योति॑ष्मतीम् । जु॒हो॒तु॒ ॥ यस्याः᳚ । ते॒ । हरि॑तः । गर्भः॑ । अथो॒ इति॑ । योनिः॑ । हि॒र॒ण्ययी᳚ ॥ अङ्गा॑नि । अह्रु॑ता । यस्यै᳚ । ताम् । दे॒वैः । समिति॑ । अ॒जी॒ग॒म॒म् ॥ एति॑ । व॒र्त॒न॒ । व॒र्त॒य॒ । नीति॑ । नि॒व॒र्त॒नेति॑ नि - व॒र्त॒न॒ । व॒र्त॒य॒ । इन्द्र॑ । न॒र्द॒बु॒द॒ ॥ भूम्याः᳚ । चत॑स्रः । प्र॒दिश॒ इति॑ प्र - दिशः॑ । ताभिः॑ । एति॑ । व॒र्त॒य॒ । पुनः॑ ॥ वीति॑ । ते॒ । भि॒न॒द्मि॒ । त॒क॒रीम् । वीति॑ । योनि᳚म् । वीति॑ । ग॒वी॒न्यौ᳚ ॥ वीति॑ । \textbf{  30} \newline
                  \newline
                                \textbf{ TS 3.3.10.2} \newline
                  मा॒तर᳚म् । च॒ । पु॒त्रम् । च॒ । वीति॑ । गर्भ᳚म् । च॒ । ज॒रायु॑ । च॒ ॥ ब॒हिः । ते॒ । अ॒स्तु॒ । बाल् । इति॑ ॥ उ॒रु॒द्र॒फ्स इत्यु॑रु - द्र॒फ्सः । वि॒श्वरू॑प॒ इति॑ वि॒श्व - रू॒पः॒ । इन्दुः॑ । पव॑मानः । धीरः॑ । आ॒न॒ञ्ज॒ । गर्भ᳚म् ॥ एक॑प॒दीत्येक॑ - प॒दी॒ । द्वि॒पदीति॑ द्वि- पदी᳚ । त्रि॒पदीति॑ त्रि - पदी᳚ । चतु॑ष्प॒दीति॒ चतुः॑ - प॒दी॒ । पञ्च॑प॒दीति॒ पञ्च॑-प॒दी॒ । षट्प॒दीति॒ षट् - प॒दी॒ । स॒प्तप॒दीति॑ स॒प्त - प॒दी॒ । अ॒ष्टाप॒दीत्य॒ष्टा-प॒दी॒ । भुव॑ना । अन्विति॑ । प्र॒थ॒ता॒म् । स्वाहा᳚ ॥ म॒ही । द्यौः । पृ॒थि॒वी । च॒ । नः॒ । इ॒मम् । य॒ज्ञ्म् । मि॒मि॒क्ष॒ता॒म् ॥ पि॒पृ॒ताम् । नः॒ । भरी॑मभि॒रिति॒ भरी॑म - भिः॒ ॥ \textbf{  31} \newline
                  \newline
                      (ग॒वि॒न्यौ॑ वि - चतु॑श्चत्वारिꣳशच्च)  \textbf{(A10)} \newline \newline
                                \textbf{ TS 3.3.11.1} \newline
                  इ॒दम् । वा॒म् । आ॒स्ये᳚ । ह॒विः । प्रि॒यम् । इ॒न्द्रा॒बृ॒ह॒स्प॒ती॒ इती᳚न्द्रा-बृ॒ह॒स्प॒ती॒ ॥ उ॒क्थम् । मदः॑ । च॒ । श॒स्य॒ते॒ ॥ अ॒यम् । वा॒म् । परीति॑ । सि॒च्य॒ते॒ । सोमः॑ । इ॒न्द्रा॒बृ॒ह॒स्प॒ती॒ इती᳚न्द्रा-बृ॒ह॒स्प॒ती॒ ॥ चारुः॑ । मदा॑य । पी॒तये᳚ ॥ अ॒स्मे इति॑ । इ॒न्द्रा॒बृ॒ह॒स्प॒ती॒ इती᳚न्द्रा - बृ॒ह॒स्प॒ती॒ । र॒यिम् । ध॒त्त॒म् । श॒त॒ग्विन॒मिति॑ शत - ग्विन᳚म् ॥ अश्वा॑वन्त॒मित्यश्व॑ - व॒न्त॒म् । स॒ह॒स्रिण᳚म् ॥ बृह॒स्पतिः॑ । नः॒ । परीति॑ । पा॒तु॒ । प॒श्चात् । उ॒त । उत्त॑रस्मा॒दित्युत् - त॒र॒स्मा॒त् । अध॑रात् । अ॒घा॒योरित्य॑घ - योः ॥ इन्द्रः॑ । पु॒रस्ता᳚त् । उ॒त । म॒द्ध्य॒तः । नः॒ । सखा᳚ । सखि॑भ्य॒ इति॒ सखि॑ - भ्यः॒ । वरि॑वः । कृ॒णो॒तु॒ ॥ वीति॑ । ते॒ । विष्व॑क् । वात॑जूतास॒ इति॒ वात॑ - जू॒ता॒सः॒ । अ॒ग्ने॒ । भामा॑सः । \textbf{  32} \newline
                  \newline
                                \textbf{ TS 3.3.11.2} \newline
                  शु॒चे॒ । शुच॑यः । च॒र॒न्ति॒ ॥ तु॒वि॒म्र॒क्षास॒ इति॑ तुवि - म्र॒क्षासः॑ । दि॒व्याः । नव॑ग्वाः । वना᳚ । व॒न॒न्ति॒ । धृ॒ष॒ता । रु॒जन्तः॑ ॥ त्वाम् । अ॒ग्ने॒ । मानु॑षीः । ई॒ड॒ते॒ । विशः॑ । हो॒त्रा॒विद॒मिति॑ होत्रा - विद᳚म् । विवि॑चि॒मिति॒ वि-वि॒चि॒म् । र॒त्न॒धात॑म॒मिति॑ रत्न - धात॑मम् ॥ गुहा᳚ । सन्त᳚म् । सु॒भ॒गेति॑ सु - भ॒ग॒ । वि॒श्वद॑र्.शत॒मिति॑ वि॒श्व-द॒र॒.श॒त॒म् । तु॒वि॒ष्म॒णस᳚म् । सु॒यज॒मिति॑ सु - यज᳚म् । घृ॒त॒श्रिय॒मिति॑ घृत - श्रिय᳚म् ॥ धा॒ता । द॒दा॒तु॒ । नः॒ । र॒यिम् । ईशा॑नः । जग॑तः । पतिः॑ ॥ सः । नः॒ । पू॒र्णेन॑ । वा॒व॒न॒त् ॥ धा॒ता । प्र॒जाया॒ इति॑ प्र - जयाः᳚ । उ॒त । रा॒यः । ई॒शे॒ । धा॒ता । इ॒दम् । विश्व᳚म् । भुव॑नम् । ज॒जा॒न॒ ॥ धा॒ता । पु॒त्रम् । यज॑मानाय । दाता᳚ । \textbf{  33} \newline
                  \newline
                                \textbf{ TS 3.3.11.3} \newline
                  तस्मै᳚ । उ॒ । ह॒व्यम् । घृ॒तव॒दिति॑ घृ॒त-व॒त् । वि॒धे॒म॒ ॥ धा॒ता । द॒दा॒तु॒ । नः॒ । र॒यिम् । प्राची᳚म् । जी॒वातु᳚म् । अक्षि॑ताम् ॥ व॒यम् । दे॒वस्य॑ । धी॒म॒हि॒ । सु॒म॒तिमिति॑ सु - म॒तिम् । स॒त्यरा॑धस॒ इति॑ स॒त्य - रा॒ध॒सः॒ ॥ धा॒ता । द॒दा॒तु॒ । दा॒शुषे᳚ । वसू॑नि । प्र॒जाका॑मा॒येति॑ प्र॒जा - का॒मा॒य॒ । मी॒ढुषे᳚ । दु॒रो॒ण इति॑ दुः-ओ॒ने ॥ तस्मै᳚ । दे॒वाः । अ॒मृताः᳚ । समिति॑ । व्य॒य॒न्ता॒म् । विश्वे᳚ । दे॒वासः॑ । अदि॑तिः । स॒जोषा॒ इति॑ स - जोषाः᳚ ॥ अन्विति॑ । नः॒ । अ॒द्य । अनु॑मति॒रित्यनु॑ - म॒तिः॒ । य॒ज्ञ्म् । दे॒वेषु॑ । म॒न्य॒ता॒म् ॥ अ॒ग्निः । च॒ । ह॒व्य॒वाह॑न॒ इति॑ हव्य - वाह॑नः । भव॑ताम् । दा॒शुषे᳚ । मयः॑ ॥ अन्विति॑ । इत् । अ॒नु॒म॒त॒ इत्य॑नु - म॒ते॒ । त्वम् । \textbf{  34} \newline
                  \newline
                                \textbf{ TS 3.3.11.4} \newline
                  मन्या॑सै । शम् । च॒ । नः॒ । कृ॒धि॒ ॥ क्रत्वे᳚ । दक्षा॑य । नः॒ । हि॒नु॒ । प्रेति॑ । नः॒ । आयूꣳ॑षि । ता॒रि॒षः॒ ॥ अन्विति॑ । म॒न्य॒ता॒म् । अ॒नु॒मन्य॑मा॒नेत्य॑नु - मन्य॑माना । प्र॒जाव॑न्त॒मिति॑ प्र॒जा - व॒न्त॒म् । र॒यिम् । अक्षी॑यमाणम् ॥ तस्यै᳚ । व॒यम् । हेड॑सि । मा । अपीति॑ । भू॒म॒ । सा । नः॒ । दे॒वी । सु॒हवेति॑ सु - हवा᳚ । शर्म॑ । य॒च्छ॒तु॒ ॥ यस्या᳚म् । इ॒दम् । प्र॒दिशीति॑ प्र - दिशि॑ । यत् । वि॒रोच॑त॒ इति॑ वि - रोच॑ते । अनु॑मति॒मित्यनु॑ - म॒ति॒म् । प्रतीति॑ । भू॒ष॒न्ति॒ । आ॒यवः॑ ॥ यस्याः᳚ । उ॒पस्थ॒ इत्यु॒प - स्थः॒ । उ॒रु । अ॒न्तरि॑क्षम् । सा । नः॒ । दे॒वी । सु॒हवेति॑ सु - हवा᳚ । शर्म॑ । य॒च्छ॒तु॒ ॥ \textbf{  35} \newline
                  \newline
                                \textbf{ TS 3.3.11.5} \newline
                  रा॒काम् । अ॒हम् । सु॒हवा॒मिति॑ सु - हवा᳚म् । सु॒ष्टु॒तीति॑ सु-स्तु॒ती । हु॒वे॒ । शृ॒णोतु॑ । नः॒ । सु॒भगेति॑ सु - भगा᳚ । बोध॑तु । त्मना᳚ ॥ सीव्य॑तु । अपः॑ । सू॒च्या । अच्छि॑द्यमानया । ददा॑तु । वी॒रम् । श॒तदा॑य॒मिति॑ श॒त - दा॒य॒म् । उ॒क्थ्य᳚म् ॥ याः । ते॒ । रा॒के॒ । सु॒म॒तय॒ इति॑ सु - म॒तयः॑ । सु॒पेश॑स॒ इति॑ सु - पेश॑सः । याभिः॑ । ददा॑सि । दा॒शुषे᳚ । वसू॑नि ॥ ताभिः॑ । नः॒ । अ॒द्य । सु॒मना॒ इति॑ सु - मनाः᳚ । उ॒पाग॒हीत्यु॑प - आग॑हि । स॒ह॒स्र॒पो॒षमिति॑ सहस्र - पो॒षम् । सु॒भ॒ग॒ इति॑ सु - भ॒गे॒ । ररा॑णा ॥ सिनी॑वालि । या । सु॒पा॒णिरिति॑ सु - पा॒णिः ॥ कु॒हूम् । अ॒हम् । सु॒भगा॒मिति॑ सु - भगा᳚म् । वि॒द्म॒नाप॑स॒मिति॑ विद्म॒न - अ॒प॒स॒म् । अ॒स्मिन्न् । य॒ज्ञे । सु॒हवा॒मिति॑ सु - हवा᳚म् । जो॒ह॒वी॒मि॒ ॥ सा । नः॒ । द॒दा॒तु॒ । श्रव॑णम् ( ) । पि॒तृ॒णाम् । तस्याः᳚ । ते॒ । दे॒वि॒ । ह॒विषा᳚ । वि॒धे॒म॒ ॥ कु॒हूः । दे॒वाना᳚म् । अ॒मृत॑स्य । पत्नी᳚ । हव्या᳚ । नः॒ । अ॒स्य । ह॒विषः॑ । चि॒के॒तु॒ ॥ समिति॑ । दा॒शुषे᳚ । कि॒रतु॑ । भूरि॑ । वा॒मम् । रा॒यः । पोष᳚म् । चि॒कि॒तुषे᳚ । द॒धा॒तु॒ ॥ \textbf{  36} \newline
                  \newline
                      (भामा॑सो॒ - दाता॒ - त्व - म॒न्तरि॑क्षꣳ॒॒ सा नो॑ दे॒वी सु॒हवा॒ शर्म॑ यच्छतु॒ -श्रव॑णं॒ - चतु॑र्विꣳशतिश्च)  \textbf{(A11)} \newline \newline
\textbf{praSna korvai with starting padams of 1 to 11 anuvAkams :-} \newline
(अग्ने॑ तेजस्विन् - वा॒यु - र्वस॑वस्त् - वै॒तद्वा अ॒पां - ॅवा॒युर॑सि प्रा॒णो नाम॑ - दे॒वा वै यद्य॒ज्ञेन॒न - प्र॒जाप॑ति र्देवासु॒रा - ना॑यु॒र्दा - ए॒तं ॅयुवा॑नꣳ॒॒ - सूर्यो॑ दे॒व - इ॒दं ॅवा॒ - मेका॑दश) \newline

\textbf{korvai with starting padams of1, 11, 21 series of pa~jcAtis :-} \newline
(अग्ने॑ तेजस्विन् - वा॒युर॑सि॒ - छन्द॑सां ॅवी॒र्यं॑ - मा॒तर॑ञ्च॒ - षट्त्रिꣳ॑शत् ) \newline

\textbf{first and last padam of Third praSnam of kANDam 3 :-} \newline
(अग्ने॑ तेजस्विꣳ - श्चिकि॒तुषे॑ दधातु ) \newline 


॥ हरिः॑ ॐ ॥॥ कृष्ण यजुर्वेदीय तैत्तिरीय संहितायां तृतीयकाण्डे तृतीयः प्रश्नः समाप्तः ॥ \newline
\pagebreak
3.3.1   AppEndix\\3.3.11.5 - सिनी॑वालि॒ >1, या सु॑पा॒णिः >2\\सिनी॑वालि॒ पृथु॑ष्टुके॒ या दे॒वाना॒मसि॒ स्वसा᳚ । \\जु॒षस्व॑ ह॒व्य माहु॑तं प्र॒जां दे॑वि दिदिड्ढि नः ॥\\या सु॑पा॒णिः स्व॑ङ्गु॒रिः सु॒षूमा॑ बहु॒सूव॑री । \\तस्यै᳚ वि॒श्पत्नि॑यै ह॒विः सि॑नीवा॒ल्यै जु॑होतन ॥ \\(Appearing in TS- 3.1.11.4)\\
\pagebreak
        


\end{document}
