\documentclass[17pt]{extarticle}
\usepackage{babel}
\usepackage{fontspec}
\usepackage{polyglossia}
\usepackage{extsizes}



\setmainlanguage{sanskrit}
\setotherlanguages{english} %% or other languages
\setlength{\parindent}{0pt}
\pagestyle{myheadings}
\newfontfamily\devanagarifont[Script=Devanagari]{AdishilaVedic}


\newcommand{\VAR}[1]{}
\newcommand{\BLOCK}[1]{}




\begin{document}
\begin{titlepage}
    \begin{center}
 
\begin{sanskrit}
    { \Large
    ॐ नमः परमात्मने, श्री महागणपतये नमः, श्री गुरुभ्यो नमः
ह॒रिः॒ ॐ 
    }
    \\
    \vspace{2.5cm}
    \mbox{ \Huge
    7.3      सप्तमकाण्डे तृतीयः प्रश्नः - सत्रजातनिरूपणं   }
\end{sanskrit}
\end{center}

\end{titlepage}
\tableofcontents

ॐ नमः परमात्मने, श्री महागणपतये नमः, श्री गुरुभ्यो नमः
ह॒रिः॒ ॐ \newline
7.3      सप्तमकाण्डे तृतीयः प्रश्नः - सत्रजातनिरूपणं \newline

\addcontentsline{toc}{section}{ 7.3      सप्तमकाण्डे तृतीयः प्रश्नः - सत्रजातनिरूपणं}
\markright{ 7.3      सप्तमकाण्डे तृतीयः प्रश्नः - सत्रजातनिरूपणं \hfill https://www.vedavms.in \hfill}
\section*{ 7.3      सप्तमकाण्डे तृतीयः प्रश्नः - सत्रजातनिरूपणं }
                                \textbf{ TS 7.3.1.1} \newline
                  प्र॒जव॒मिति॑ प्र - जव᳚म् । वै । ए॒तेन॑ । य॒न्ति॒ । यत् । द॒श॒मम् । अहः॑ । पा॒पा॒व॒हीय॒मिति॑ पाप - अ॒व॒हीय᳚म् । वै । ए॒तेन॑ । भ॒व॒न्ति॒ । यत् । द॒श॒मम् । अहः॑ । यः । वै । प्र॒जव॒मिति॑ प्र - जव᳚म् । य॒ताम् । अप॑थेन । प्र॒ति॒पद्य॑त॒ इति॑ प्रति - पद्य॑ते । यः । स्था॒णुम् । हन्ति॑ । यः । भ्रेष᳚म् । न्येतीति॑ नि - एति॑ । सः । ही॒य॒ते॒ । सः । यः । वै । द॒श॒मे । अहन्न्॑ । अ॒वि॒वा॒क्य इत्य॑वि - वा॒क्ये । उ॒प॒ह॒न्यत॒ इत्यु॑प-ह॒न्यते᳚ । सः । ही॒य॒ते॒ । तस्मै᳚ । यः । उप॑हता॒येत्युप॑-ह॒ता॒य॒ । व्याहेति॑ वि - आह॑ । तम् । ए॒व । अ॒न्वा॒रभ्येत्य॑नु - आ॒रभ्य॑ । समिति॑ । अ॒श्नु॒ते॒ । अथ॑ । यः । व्याहेति॑ वि - आह॑ । सः । 1 ( 50) \textbf{  1} \newline
                  \newline
                                \textbf{ TS 7.3.1.2} \newline
                  ही॒य॒ते॒ । तस्मा᳚त् । द॒श॒मे । अहन्न्॑ । अ॒वि॒वा॒क्य इत्य॑वि - वा॒क्ये । उप॑हता॒येत्युप॑ - ह॒ता॒य॒ । न । व्युच्य॒मिति॑ वि-उच्य᳚म् । अथो॒ इति॑ । खलु॑ । आ॒हुः॒ । य॒ज्ञ्स्य॑ । वै । समृ॑द्धे॒नेति॒ सं - ऋ॒द्धे॒न॒ । दे॒वाः । सु॒व॒र्गमिति॑ सुवः - गम् । लो॒कम् । आ॒य॒न्न् । य॒ज्ञ्स्य॑ । व्यृ॑द्धे॒नेति॒ वि - ऋ॒द्धे॒न॒ । असु॑रान् । परेति॑ । अ॒भा॒व॒य॒न्न् । इति॑ । यत् । खलु॑ । वै । य॒ज्ञ्स्य॑ । समृ॑द्ध॒मिति॒ सं - ऋ॒द्ध॒म् । तत् । यज॑मानस्य । यत् । व्यृ॑द्ध॒मिति॒ वि-ऋ॒द्ध॒म् । तत् । भ्रातृ॑व्यस्य । सः । यः । वै । द॒श॒मे । अहन्न्॑ । अ॒वि॒वा॒क्य इत्य॑वि - वा॒क्ये । उ॒प॒ह॒न्यत॒ इत्यु॑प - ह॒न्यते᳚ । सः । ए॒व । अतीति॑ । रे॒च॒य॒ति॒ । ते । ये । बाह्याः᳚ । दृ॒शी॒कवः॑ । \textbf{  2} \newline
                  \newline
                                \textbf{ TS 7.3.1.3} \newline
                  स्युः । ते । वीति॑ । ब्रू॒युः॒ । यदि॑ । तत्र॑ । न । वि॒न्देयुः॑ । अ॒न्त॒स्स॒द॒सादित्य॑न्तः - स॒द॒सात् । व्युच्य॒मिति॑ वि - उच्य᳚म् । यदि॑ । तत्र॑ । न । वि॒न्देयुः॑ । गृ॒हप॑ति॒नेति॑ गृ॒ह - प॒ति॒ना॒ । व्युच्य॒मिति॑ वि- उच्य᳚म् । तत् । व्युच्य॒मिति॑ वि - उच्य᳚म् । ए॒व । अथ॑ । वै । ए॒तत् । स॒र्प॒रा॒ज्ञिया॒ इति॑ सर्प - रा॒ज्ञियाः᳚ । ऋ॒ग्भिरित्यृ॑क् - भिः । स्तु॒व॒न्ति॒ । इ॒यम् । वै । सर्प॑तः । राज्ञी᳚ । यत् । वै । अ॒स्याम् । किम् । च॒ । अर्च॑न्ति । यत् । आ॒नृ॒चुः । तेन॑ । इ॒यम् । स॒र्प॒रा॒ज्ञीति॑ सर्प-रा॒ज्ञी । ते । यत् । ए॒व । किम् । च॒ । वा॒चा । आ॒नृ॒चुः । यत् । अ॒तोधि॑ । अ॒र्चि॒तारः॑ । \textbf{  3} \newline
                  \newline
                                \textbf{ TS 7.3.1.4} \newline
                  तत् । उ॒भय᳚म् । आ॒प्त्वा । अ॒व॒रुद्ध्येत्य॑व - रुद्ध्य॑ । उदिति॑ । ति॒ष्ठा॒म॒ । इति॑ । ताभिः॑ । मन॑सा । स्तु॒व॒ते॒ । न । वै । इ॒माम् । अ॒श्व॒र॒थ इत्य॑श्व - र॒थः । न । अ॒श्व॒त॒री॒र॒थ इत्यश्व॑तरी - र॒थः । स॒द्यः । पर्या᳚प्तु॒मिति॒ परि॑-आ॒प्तु॒म् । अ॒र्.॒ह॒ति॒ । मनः॑ । वै । इ॒माम् । स॒द्यः । पर्या᳚प्तु॒मिति॒ परि॑ - आ॒प्तु॒म् । अ॒र्.॒ह॒ति॒ । मनः॑ । परि॑भवितु॒मिति॒ परि॑-भ॒वि॒तु॒म् । अथ॑ । ब्रह्म॑ । व॒द॒न्ति॒ । परि॑मिता॒ इति॒ परि॑-मि॒ताः॒ । वै । ऋचः॑ । परि॑मिता॒नीति॒ परि॑ - मि॒ता॒नि॒ । सामा॑नि । परि॑मिता॒नीति॒ परि॑ - मि॒ता॒नि॒ । यजूꣳ॑षि । अथ॑ । ए॒तस्य॑ । ए॒व । अन्तः॑ । न । अ॒स्ति॒ । यत् । ब्रह्म॑ । तत् । प्र॒ति॒गृ॒ण॒त इति॑ प्रति - गृ॒ण॒ते । एति॑ । च॒क्षी॒त॒ । सः ( ) । प्र॒ति॒ग॒र इति॑ प्रति - ग॒रः ॥ \textbf{  4} \newline
                  \newline
                      (व्याह॒ स - दृ॑शी॒कवो᳚ - ऽर्चि॒तारः॒ - स - एकं॑ च)  \textbf{(A1)} \newline \newline
                                \textbf{ TS 7.3.2.1} \newline
                  ब्र॒ह्म॒वा॒दिन॒ इति॑ ब्रह्म - वा॒दिनः॑ । व॒द॒न्ति॒ । किम् । द्वा॒द॒शा॒हस्येति॑ द्वादश - अ॒हस्य॑ । प्र॒थ॒मेन॑ । अह्ना᳚ । ऋ॒त्विजा᳚म् । यज॑मानः । वृ॒ङ्क्ते॒ । इति॑ । तेजः॑ । इ॒न्द्रि॒यम् । इति॑ । किम् । द्वि॒तीये॑न । इति॑ । प्रा॒णानिति॑ प्र - अ॒नान् । अ॒न्नाद्य॒मित्य॑न्न - अद्य᳚म् । इति॑ । किम् । तृ॒तीये॑न । इति॑ । त्रीन् । इ॒मान् । लो॒कान् । इति॑ । किम् । च॒तु॒र्त्थेन॑ । इति॑ । चतु॑ष्पद॒ इति॒ चतुः॑-प॒दः॒ । प॒शून् । इति॑ । किम् । प॒ञ्च॒मेन॑ । इति॑ । पञ्चा᳚क्षरा॒मिति॒ पञ्च॑ - अ॒क्ष॒रा॒म् । प॒ङ्क्तिम् । इति॑ । किम् । ष॒ष्ठेन॑ । इति॑ । षट् । ऋ॒तून् । इति॑ । किम् । स॒प्त॒मेन॑ । इति॑ । स॒प्तप॑दा॒मिति॑ स॒प्त - प॒दा॒म् । शक्व॑रीम् । इति॑ । \textbf{  5} \newline
                  \newline
                                \textbf{ TS 7.3.2.2} \newline
                  किम् । अ॒ष्ट॒मेन॑ । इति॑ । अ॒ष्टाक्ष॑रा॒मित्य॒ष्टा - अ॒क्ष॒रा॒म् । गा॒य॒त्रीम् । इति॑ । किम् । न॒व॒मेन॑ । इति॑ । त्रि॒वृत॒मिति॑ त्रि - वृत᳚म् । स्तोम᳚म् । इति॑ । किम् । द॒श॒मेन॑ । इति॑ । दशा᳚क्षरा॒मिति॒ दश॑ - अ॒क्ष॒रा॒म् । वि॒राज॒मिति॑ वि - राज᳚म् । इति॑ । किम् । ए॒का॒द॒शेन॑ । इति॑ । एका॑दशाक्षरा॒मित्येका॑दश - अ॒क्ष॒रा॒म् । त्रि॒ष्टुभ᳚म् । इति॑ । किम् । द्वा॒द॒शेन॑ । इति॑ । द्वाद॑शाक्षरा॒मिति॒ द्वाद॑श - अ॒क्ष॒रा॒म् । जग॑तीम् । इति॑ । ए॒ताव॑त् । वै । अ॒स्ति॒ । याव॑त् । ए॒तत् । याव॑त् । ए॒व । अस्ति॑ । तत् । ए॒षा॒म् । वृ॒ङ्क्ते॒ ॥ \textbf{  6} \newline
                  \newline
                      (शक्व॑री॒मित्ये - क॑चत्वारिꣳशच्च)  \textbf{(A2)} \newline \newline
                                \textbf{ TS 7.3.3.1} \newline
                  ए॒षः । वै । आ॒प्तः । द्वा॒द॒शा॒ह इति॑ द्वादश - अ॒हः । यत् । त्र॒यो॒द॒श॒रा॒त्र इति॑ त्रयोदश - रा॒त्रः । स॒मा॒नम् । हि । ए॒तत् । अहः॑ । यत् । प्रा॒य॒णीय॒ इति॑ प्र - अ॒य॒नीयः॑ । च॒ । उ॒द॒य॒नीय॒ इत्यु॑त् - अ॒य॒नीयः॑ । च॒ । त्र्य॑तिरात्र॒ इति॒ त्रि - अ॒ति॒रा॒त्रः॒ । भ॒व॒ति॒ । त्रयः॑ । इ॒मे । लो॒काः । ए॒षाम् । लो॒काना᳚म् । आप्त्यै᳚ । प्रा॒ण इति॑ प्र - अ॒नः । वै । प्र॒थ॒मः । अ॒ति॒रा॒त्र इत्य॑ति - रा॒त्रः । व्या॒न इति॑ वि - अ॒नः । द्वि॒तीयः॑ । अ॒पा॒न इत्य॑प-अ॒नः । तृ॒तीयः॑ । प्रा॒णा॒पा॒नो॒दा॒नेष्विति॑ प्राणापान - उ॒दा॒नेषु॑ । ए॒व । अ॒न्नाद्य॒ इत्य॑न्न - अद्ये᳚ । प्रतीति॑ । ति॒ष्ठ॒न्ति॒ । सर्व᳚म् । आयुः॑ । य॒न्ति॒ । ये । ए॒वम् । वि॒द्वाꣳसः॑ । त्र॒यो॒द॒श॒रा॒त्रमिति॑ त्रयोदश - रा॒त्रम् । आस॑ते । तत् । आ॒हुः॒ । वाक् । वै । ए॒षा । वित॒तेति॒ वि - त॒ता॒ । \textbf{  7} \newline
                  \newline
                                \textbf{ TS 7.3.3.2} \newline
                  यत् । द्वा॒द॒शा॒ह इति॑ द्वादश - अ॒हः । ताम् । वीति॑ । छि॒न्द्युः॒ । यत् । मद्ध्ये᳚ । अ॒ति॒रा॒त्रमित्य॑ति-रा॒त्रम् । कु॒र्युः । उ॒प॒दासु॒केत्यु॑प-दासु॑का । गृ॒हप॑ते॒रिति॑ गृ॒ह - प॒तेः॒ । वाक् । स्या॒त् । उ॒परि॑ष्टात् । छ॒न्दो॒माना॒मिति॑ छन्दः - माना᳚म् । म॒हा॒व्र॒तमिति॑ महा - व्र॒तम् । कु॒र्व॒न्ति॒ । संत॑ता॒मिति॒ सं-त॒ता॒म् । ए॒व । वाच᳚म् । अवेति॑ । रु॒न्ध॒ते॒ । अनु॑पदासु॒केत्यनु॑प - दा॒सु॒का॒ । गृ॒हप॑ते॒रिति॑ गृ॒ह - प॒तेः॒ । वाक् । भ॒व॒ति॒ । प॒शवः॑ । वै । छ॒न्दो॒मा इति॑ छन्दः - माः । अन्न᳚म् । म॒हा॒व्र॒तमिति॑ महा - व्र॒तम् । यत् । उ॒परि॑ष्टात् । छ॒न्दो॒माना॒मिति॑ छन्दः - माना᳚म् । म॒हा॒व्र॒तमिति॑ महा - व्र॒तम् । कु॒र्वन्ति॑ । प॒शुषु॑ । च॒ । ए॒व । अ॒न्नाद्य॒ इत्य॑न्न - अद्ये᳚ । च॒ । प्रतीति॑ । ति॒ष्ठ॒न्ति॒ ॥ \textbf{  8 } \newline
                  \newline
                      (वित॑ता॒ - त्रिच॑त्वारिꣳशच्च)  \textbf{(A3)} \newline \newline
                                \textbf{ TS 7.3.4.1} \newline
                  आ॒दि॒त्याः । अ॒का॒म॒य॒न्त॒ । उ॒भयोः᳚ । लो॒कयोः᳚ । ऋ॒द्ध्नु॒या॒म॒ । इति॑ । ते । ए॒तम् । च॒तु॒र्द॒श॒रा॒त्रमिति॑ चतुर्दश - रा॒त्रम् । अ॒प॒श्य॒न्न् । तम् । एति॑ । अ॒ह॒र॒न्न् । तेन॑ । अ॒य॒ज॒न्त॒ । ततः॑ । वै । ते । उ॒भयोः᳚ । लो॒कयोः᳚ । आ॒द्‌र्ध्नु॒व॒न्न् । अ॒स्मिन्न् । च॒ । अ॒मुष्मिन्न्॑ । च॒ । ये । ए॒वम् । वि॒द्वाꣳसः॑ । च॒तु॒र्द॒श॒रा॒त्रमिति॑ चतुर्दश - रा॒त्रम् । आस॑ते । उ॒भयोः᳚ । ए॒व । लो॒कयोः᳚ । ऋ॒द्ध्नु॒व॒न्ति॒ । अ॒स्मिन्न् । च॒ । अ॒मुष्मिन्न्॑ । च॒ । च॒तु॒र्द॒श॒रा॒त्र इति॑ चतुर्दश - रा॒त्रः । भ॒व॒ति॒ । स॒प्त । ग्रा॒म्याः । ओष॑धयः । स॒प्त । आ॒र॒ण्याः । उ॒भयी॑षाम् । अव॑रुद्ध्या॒ इत्यव॑-रु॒द्ध्यै॒ । यत् । प॒रा॒चीना॑नि । पृ॒ष्ठानि॑ । \textbf{  9} \newline
                  \newline
                                \textbf{ TS 7.3.4.2} \newline
                  भव॑न्ति । अ॒मुम् । ए॒व । तैः । लो॒कम् । अ॒भीति॑ । ज॒य॒न्ति॒ । यत् । प्र॒ती॒चीना॑नि । पृ॒ष्ठानि॑ । भव॑न्ति । इ॒मम् । ए॒व । तैः । लो॒कम् । अ॒भीति॑ । ज॒य॒न्ति॒ । त्र॒य॒स्त्रिꣳ॒॒शाविति॑ त्रयः - त्रिꣳ॒॒शौ । म॒द्ध्य॒तः । स्तोमौ᳚ । भ॒व॒तः॒ । साम्रा᳚ज्य॒मिति॒ सां - रा॒ज्य॒म् । ए॒व । ग॒च्छ॒न्ति॒ । अ॒धि॒रा॒जावित्य॑धि - रा॒जौ । भ॒व॒तः॒ । अ॒धि॒रा॒जा इत्य॑धि - रा॒जाः । ए॒व । स॒मा॒नाना᳚म् । भ॒व॒न्ति॒ । अ॒ति॒रा॒त्रावित्य॑ति - रा॒त्रौ । अ॒भितः॑ । भ॒व॒तः॒ । परि॑गृहीत्या॒ इति॒ परि॑ - गृ॒ही॒त्यै॒ ॥ \textbf{  10} \newline
                  \newline
                      (पृ॒ष्ठानि॒ - चतु॑स्त्रिꣳशच्च)  \textbf{(A4)} \newline \newline
                                \textbf{ TS 7.3.5.1} \newline
                  प्र॒जाप॑ति॒रिति॑ प्र॒जा - प॒तिः॒ । सु॒व॒र्गमिति॑ सुवः - गम् । लो॒कम् । ऐ॒त् । तम् । दे॒वाः । अन्विति॑ । आ॒य॒न्न् । तान् । आ॒दि॒त्याः । च॒ । प॒शवः॑ । च॒ । अन्विति॑ । आ॒य॒न्न् । ते । दे॒वाः । अ॒ब्रु॒व॒न्. । यान् । प॒शून् । उ॒पाजी॑वि॒ष्मेत्यु॑प - अजी॑विष्म । ते । इ॒मे । अ॒न्वाग्म॒न्नित्य॑नु-आग्मन्न्॑ । इति॑ । तेभ्यः॑ । ए॒तम् । च॒तु॒र्द॒श॒रा॒त्रमिति॑ चतुर्दश-रा॒त्रम् । प्रतीति॑ । औ॒ह॒न्न् । ते । आ॒दि॒त्याः । पृ॒ष्ठैः । सु॒व॒र्गमिति॑ सुवः - गम् । लो॒कम् । एति॑ । अ॒रो॒ह॒न्न् । त्र्य॒हाभ्या॒मिति॑ त्रि - अ॒हाभ्या᳚म् । अ॒स्मिन्न् । लो॒के । प॒शून् । प्रतीति॑ । औ॒ह॒न्न् । पृ॒ष्ठैः । आ॒दि॒त्याः । अ॒मुष्मिन्न्॑ । लो॒के । आद्‌र्ध्नु॑वन्न् । त्र्य॒हाभ्या॒मिति॑ त्रि - अ॒हाभ्या᳚म् । अ॒स्मिन्न् । \textbf{  11} \newline
                  \newline
                                \textbf{ TS 7.3.5.2} \newline
                  लो॒के । प॒शवः॑ । ये । ए॒वम् । वि॒द्वाꣳसः॑ । च॒तु॒र्द॒श॒रा॒त्रमिति॑ चतुर्दश - रा॒त्रम् । आस॑ते । उ॒भयोः᳚ । ए॒व । लो॒कयोः᳚ । ऋ॒द्ध्नु॒व॒न्ति॒ । अ॒स्मिन्न् । च॒ । अ॒मुष्मिन्न्॑ । च॒ । पृ॒ष्ठैः । ए॒व । अ॒मुष्मिन्न्॑ । लो॒के । ऋ॒द्ध्नु॒वन्ति॑ । त्र्य॒हाभ्या॒मिति॑ त्रि - अ॒हाभ्या᳚म् । अ॒स्मिन्न् । लो॒के । ज्योतिः॑ । गौः । आयुः॑ । इति॑ । त्र्य॒ह इति॑ त्रि-अ॒हः । भ॒व॒ति॒ । इ॒यम् । वाव । ज्योतिः॑ । अ॒न्तरि॑क्षम् । गौः । अ॒सौ । आयुः॑ । इ॒मान् । ए॒व । लो॒कान् । अ॒भ्यारो॑ह॒न्तीत्य॑भि - आरो॑हन्ति । यत् । अ॒न्यतः॑ । पृ॒ष्ठानि॑ । स्युः । विवि॑वध॒मिति॒ वि-वि॒व॒ध॒म् । स्या॒त् । मद्ध्ये᳚ । पृ॒ष्ठानि॑ । भ॒व॒न्ति॒ । स॒वि॒व॒ध॒त्वायेति॑ सविवध - त्वाय॑ । \textbf{  12} \newline
                  \newline
                                \textbf{ TS 7.3.5.3} \newline
                  ओजः॑ । वै । वी॒र्य᳚म् । पृ॒ष्ठानि॑ । ओजः॑ । ए॒व । वी॒र्य᳚म् । म॒द्ध्य॒तः । द॒ध॒ते॒ । बृ॒ह॒द्र॒थ॒न्त॒राभ्या॒मिति॑ बृहत् - र॒थ॒न्त॒राभ्या᳚म् । य॒न्ति॒ । इ॒यम् । वाव । र॒थ॒न्त॒रमिति॑ रथं - त॒रम् । अ॒सौ । बृ॒हत् । आ॒भ्याम् । ए॒व । य॒न्ति॒ । अथो॒ इति॑ । अ॒नयोः᳚ । ए॒व । प्रतीति॑ । ति॒ष्ठ॒न्ति॒ । ए॒ते इति॑ । वै । य॒ज्ञ्स्य॑ । अ॒ञ्ज॒साय॑नी॒ इत्य॑ञ्जसा - अय॑नी । स्रु॒ती इति॑ । ताभ्या᳚म् । ए॒व । सु॒व॒र्गमिति॑ सुवः - गम् । लो॒कम् । य॒न्ति॒ । परा᳚ञ्चः । वै । ए॒ते । सु॒व॒र्गमिति॑ सुवः - गम् । लो॒कम् । अ॒भ्यारो॑ह॒न्तीत्य॑भि - आरो॑हन्ति । ये । प॒रा॒चीना॑नि । पृ॒ष्ठानि॑ । उ॒प॒यन्तीत्यु॑प - यन्ति॑ । प्र॒त्यङ् । त्र्य॒ह इति॑ त्रि - अ॒हः । भ॒व॒ति॒ । प्र॒त्यव॑रूढ्या॒ इति॑ प्रति - अव॑रूढ्यै । अथो॒ इति॑ । प्रति॑ष्ठित्या॒ इति॒ प्रति॑-स्थि॒त्यै॒ ( ) । उ॒भयोः᳚ । लो॒कयोः᳚ । ऋ॒द्ध्वा । उदिति॑ । ति॒ष्ठ॒न्ति॒ । चतु॑र्द॒शेति॒ चतुः॑ - द॒श॒ । ए॒ताः । तासा᳚म् । याः । दश॑ । दशा᳚क्ष॒रेति॒ दश॑ - अ॒क्ष॒रा॒ । वि॒राडिति॑ वि - राट् । अन्न᳚म् । वि॒राडिति॑ वि - राट् । वि॒राजेति॑ वि - राजा᳚ । ए॒व । अ॒न्नाद्य॒मित्य॑न्न - अद्य᳚म् । अवेति॑ । रु॒न्ध॒ते॒ । याः । चत॑स्रः । चत॑स्रः । दिशः॑ । दि॒क्षु । ए॒व । प्रतीति॑ । ति॒ष्ठ॒न्ति॒ । अ॒ति॒रा॒त्रावित्य॑ति - रा॒त्रौ । अ॒भितः॑ । भ॒व॒तः॒ । परि॑गृहीत्या॒ इति॒ परि॑ - गृ॒ही॒त्यै॒ ॥ \textbf{  13} \newline
                  \newline
                      (आर्ध्नु॑वन् त्र्य॒हाभ्या॑म॒स्मिन्थ् - स॑विवध॒त्वाय॒ - प्रति॑ष्ठत्या॒ - एक॑त्रिꣳशच्च)  \textbf{(A5)} \newline \newline
                                \textbf{ TS 7.3.6.1} \newline
                  इन्द्रः॑ । वै । स॒दृङ्ङिति॑ स - दृङ् । दे॒वता॑भिः । आ॒सी॒त् । सः । न । व्या॒वृत॒मिति॑ वि - आ॒वृत᳚म् । अ॒ग॒च्छ॒त् । सः । प्र॒जाप॑ति॒मिति॑ प्र॒जा - प॒ति॒म् । उपेति॑ । अ॒धा॒व॒त् । तस्मै᳚ । ए॒तम् । प॒ञ्च॒द॒श॒रा॒त्रमिति॑ पञ्चदश-रा॒त्रम् । प्रेति॑ । अ॒य॒च्छ॒त् । तम् । एति॑ । अ॒ह॒र॒त् । तेन॑ । अ॒य॒ज॒त॒ । ततः॑ । वै । सः । अ॒न्याभिः॑ । दे॒वता॑भिः । व्या॒वृत॒मिति॑ वि - आ॒वृत᳚म् । अ॒ग॒च्छ॒त् । ये । ए॒वम् । वि॒द्वाꣳसः॑ । प॒ञ्च॒द॒श॒रा॒त्रमिति॑ पञ्चदश - रा॒त्रम् । आस॑ते । व्या॒वृत॒मिति॑ वि - आ॒वृत᳚म् । ए॒व । पा॒प्मना᳚ । भ्रातृ॑व्येण । ग॒च्छ॒न्ति॒ । ज्योतिः॑ । गौः । आयुः॑ । इति॑ । त्र्य॒ह इति॑ त्रि - अ॒हः । भ॒व॒ति॒ । इ॒यम् । वाव । ज्योतिः॑ । अ॒न्तरि॑क्षम् । \textbf{  14} \newline
                  \newline
                                \textbf{ TS 7.3.6.2} \newline
                  गौः । अ॒सौ । आयुः॑ । ए॒षु । ए॒व । लो॒केषु॑ । प्रतीति॑ । ति॒ष्ठ॒न्ति॒ । अस॑त्रम् । वै । ए॒तत् । यत् । अ॒छ॒न्दो॒ममित्य॑छन्दः - मम् । यत् । छ॒न्दो॒मा इति॑ छन्दः - माः । भव॑न्ति । तेन॑ । स॒त्रम् । दे॒वताः᳚ । ए॒व । पृ॒ष्ठैः । अवेति॑ । रु॒न्ध॒ते॒ । प॒शून् । छ॒न्दो॒मैरिति॑ छन्दः-मैः । ओजः॑ । वै । वी॒र्य᳚म् । पृ॒ष्ठानि॑ । प॒शवः॑ । छ॒न्दो॒मा इति॑ छन्दः - माः । ओज॑सि । ए॒व । वी॒र्ये᳚ । प॒शुषु॑ । प्रतीति॑ । ति॒ष्ठ॒न्ति॒ । प॒ञ्च॒द॒श॒रा॒त्र इति॑ पञ्चदश - रा॒त्रः । भ॒व॒ति॒ । प॒ञ्च॒द॒श इति॑ पञ्च - द॒शः । वज्रः॑ । वज्र᳚म् । ए॒व । भ्रातृ॑व्येभ्यः । प्रेति॑ । ह॒र॒न्ति॒ । अ॒ति॒रा॒त्रावित्य॑ति - रा॒त्रौ । अ॒भितः॑ । भ॒व॒तः॒ । इ॒न्द्रि॒यस्य॑ ( ) । परि॑गृहीत्या॒ इति॒ परि॑ - गृ॒ही॒त्यै॒ ॥ \textbf{  15 } \newline
                  \newline
                      (अ॒न्तरि॑क्ष-मिन्द्रि॒यस्यै-क॑ञ्च)  \textbf{(A6)} \newline \newline
                                \textbf{ TS 7.3.7.1} \newline
                  इन्द्रः॑ । वै । शि॒थि॒लः । इ॒व॒ । अप्र॑तिष्ठित॒ इत्यप्र॑ति - स्थि॒तः॒ । आ॒सी॒त् । सः । असु॑रेभ्यः । अ॒बि॒भे॒त् । सः । प्र॒जाप॑ति॒मिति॑ प्र॒जा - प॒ति॒म् । उपेति॑ । अ॒धा॒व॒त् । तस्मै᳚ । ए॒तम् । प॒ञ्च॒द॒श॒रा॒त्रमिति॑ पञ्चदश - रा॒त्रम् । वज्र᳚म् । प्रेति॑ । अ॒य॒च्छ॒त् । तेन॑ । असु॑रान् । प॒रा॒भाव्येति॑ परा - भाव्य॑ । वि॒जित्येति॑ वि-जित्य॑ । श्रिय᳚म् । अ॒ग॒च्छ॒त् । अ॒ग्नि॒ष्टुतेत्य॑ग्नि - स्तुता᳚ । पा॒प्मान᳚म् । निरिति॑ । अ॒द॒ह॒त॒ । प॒ञ्च॒द॒श॒रा॒त्रेणेति॑ पञ्चदश - रा॒त्रेण॑ । ओजः॑ । बल᳚म् । इ॒न्द्रि॒यम् । वी॒र्य᳚म् । आ॒त्मन्न् । अ॒ध॒त्त॒ । ये । ए॒वम् । वि॒द्वाꣳसः॑ । प॒ञ्च॒द॒श॒रा॒त्रमिति॑ पञ्चदश - रा॒त्रम् । आस॑ते । भ्रातृ॑व्यान् । ए॒व । प॒रा॒भाव्येति॑ परा - भाव्य॑ । वि॒जित्येति॑ वि - जित्य॑ । श्रिय᳚म् । ग॒च्छ॒न्ति॒ । अ॒ग्नि॒ष्टुतेत्य॑ग्नि - स्तुता᳚ । पा॒प्मान᳚म् । निरिति॑ । \textbf{  16} \newline
                  \newline
                                \textbf{ TS 7.3.7.2} \newline
                  द॒ह॒न्ते॒ । प॒ञ्च॒द॒श॒रा॒त्रेणेति॑ पञ्चदश - रा॒त्रेण॑ । ओजः॑ । बल᳚म् । इ॒न्द्रि॒यम् । वी॒र्य᳚म् । आ॒त्मन्न् । द॒ध॒ते॒ । ए॒ताः । ए॒व । प॒श॒व्याः᳚ । पञ्च॑द॒शेति॒ पञ्च॑ - द॒श॒ । वै । अ॒द्‌र्ध॒मा॒सस्येत्य॒॑द्‌र्ध - मा॒सस्य॑ । रात्र॑यः । अ॒द्‌र्ध॒मा॒स॒श इत्य॑द्‌र्धमास - शः । सं॒ॅव॒थ्स॒र इति॑ सं-व॒थ्स॒रः । आ॒प्य॒ते॒ । सं॒ॅव॒थ्स॒रमिति॑ सं-व॒थ्स॒रम् । प॒शवः॑ । अनु॑ । प्रेति॑ । जा॒य॒न्ते॒ । तस्मा᳚त् । प॒श॒व्याः᳚ । ए॒ताः । ए॒व । सु॒व॒र्ग्या॑ इति॑ सुवः - ग्याः᳚ । पञ्च॑द॒शेति॒ पञ्च॑ - द॒श॒ । वै । अ॒द्‌र्ध॒मा॒सस्येत्य॑द्‌र्ध -मा॒सस्य॑ । रात्र॑यः । अ॒द्‌र्ध॒मा॒स॒श इत्य॑द्‌र्धमास - शः । सं॒ॅव॒थ्स॒र इति॑ सं - व॒थ्स॒रः । आ॒प्य॒ते॒ । सं॒ॅव॒थ्स॒र इति॑ सं - व॒थ्स॒रः । सु॒व॒र्ग इति॑ सुवः - गः । लो॒कः । तस्मा᳚त् । सु॒व॒र्ग्या॑ इति॑ सुवः - ग्याः᳚ । ज्योतिः॑ । गौः । आयुः॑ । इति॑ । त्र्य॒ह इति॑ त्रि - अ॒हः । भ॒व॒ति॒ । इ॒यम् । वाव । ज्योतिः॑ । अ॒न्तरि॑क्षम् । \textbf{  17} \newline
                  \newline
                                \textbf{ TS 7.3.7.3} \newline
                  गौः । अ॒सौ । आयुः॑ । इ॒मान् । ए॒व । लो॒कान् । अ॒भ्यारो॑ह॒न्तीत्य॑भि - आरो॑हन्ति । यत् । अ॒न्यतः॑ । पृ॒ष्ठानि॑ । स्युः । विवि॑वध॒मिति॒ वि - वि॒व॒ध॒म् । स्या॒त् । मद्ध्ये᳚ । पृ॒ष्ठानि॑ । भ॒व॒न्ति॒ । स॒वि॒व॒ध॒त्वायेति॑ सविवध - त्वाय॑ । ओजः॑ । वै । वी॒र्य᳚म् । पृ॒ष्ठानि॑ । ओजः॑ । ए॒व । वी॒र्य᳚म् । म॒द्ध्य॒तः । द॒ध॒ते॒ । बृ॒ह॒द्र॒थ॒न्त॒राभ्या॒मिति॑ बृहत्-र॒थ॒न्त॒राभ्या᳚म् । य॒न्ति॒ । इ॒यम् । वाव । र॒थ॒न्त॒रमिति॑ रथं - त॒रम् । अ॒सौ । बृ॒हत् । आ॒भ्याम् । ए॒व । य॒न्ति॒ । अथो॒ इति॑ । अ॒नयोः᳚ । ए॒व । प्रतीति॑ । ति॒ष्ठ॒न्ति॒ । ए॒ते इति॑ । वै । य॒ज्ञ्स्य॑ । अ॒ञ्ज॒साय॑नी॒ इत्य॑ञ्जसा-अय॑नी । स्रु॒ती इति॑ । ताभ्या᳚म् । ए॒व । सु॒व॒र्गमिति॑ सुवः - गम् । लो॒कम् । \textbf{  18} \newline
                  \newline
                                \textbf{ TS 7.3.7.4} \newline
                  य॒न्ति॒ । परा᳚ञ्चः । वै । ए॒ते । सु॒व॒र्गमिति॑ सुवः - गम् । लो॒कम् । अ॒भ्यारो॑ह॒न्तीत्य॑भि - आरो॑हन्ति । ये । प॒रा॒चीना॑नि । पृ॒ष्ठानि॑ । उ॒प॒यन्तीत्यु॑प - यन्ति॑ । प्र॒त्यङ् । त्र्य॒ह इति॑ त्रि - अ॒हः । भ॒व॒ति॒ । प्र॒त्यव॑रूढ्या॒ इति॑ प्रति - अव॑रूढ्यै । अथो॒ इति॑ । प्रति॑ष्ठित्या॒ इति॒ प्रति॑ - स्थि॒त्यै॒ । उ॒भयोः᳚ । लो॒कयोः᳚ । ऋ॒द्ध्वा । उदिति॑ । ति॒ष्ठ॒न्ति॒ । पञ्च॑द॒शेति॒ पञ्च॑ - द॒श॒ । ए॒ताः । तासा᳚म् । याः । दश॑ । दशा᳚क्ष॒रेति॒ दश॑ - अ॒क्ष॒रा॒ । वि॒राडिति॑ वि - राट् । अन्न᳚म् । वि॒राडिति॑ वि - राट् । वि॒राजेति॑ वि - राजा᳚ । ए॒व । अ॒न्नाद्य॒मित्य॑न्न - अद्य᳚म् । अवेति॑ । रु॒न्ध॒ते॒ । याः । पञ्च॑ । पञ्च॑ । दिशः॑ । दि॒क्षु । ए॒व । प्रतीति॑ । ति॒ष्ठ॒न्ति॒ । अ॒ति॒रा॒त्रावित्य॑ति - रा॒त्रौ । अ॒भितः॑ । भ॒व॒तः॒ । इ॒न्द्रि॒यस्य॑ । वी॒र्य॑स्य । प्र॒जाया॒ इति॑ प्र - जायै᳚ ( ) । प॒शू॒नाम् । परि॑गृहीत्या॒ इति॒ परि॑ - गृ॒ही॒त्यै॒ ॥ \textbf{  19 } \newline
                  \newline
                      (ग॒च्छ॒न्त्य॒ग्नि॒ष्टुता॑ पा॒प्मानं॒ नि-र॒न्तरि॑क्षं - ॅलो॒कं - प्र॒जायै॒ - द्वे च॑)  \textbf{(A7)} \newline \newline
                                \textbf{ TS 7.3.8.1} \newline
                  प्र॒जाप॑ति॒रिति॑ प्र॒जा - प॒तिः॒ । अ॒का॒म॒य॒त॒ । अ॒न्ना॒द इत्य॑न्न - अ॒दः । स्या॒म् । इति॑ । सः । ए॒तम् । स॒प्त॒द॒श॒रा॒त्रमिति॑ सप्तदश - रा॒त्रम् । अ॒प॒श्य॒त् । तम् । एति॑ । अ॒ह॒र॒त् । तेन॑ । अ॒य॒ज॒त॒ । ततः॑ । वै । सः । अ॒न्ना॒द इत्य॑न्न - अ॒दः । अ॒भ॒व॒त् । ये । ए॒वम् । वि॒द्वाꣳसः॑ । स॒प्त॒द॒श॒रा॒त्रमिति॑ सप्तदश-रा॒त्रम् । आस॑ते । अ॒न्ना॒दा इत्य॑न्न-अ॒दाः । ए॒व । भ॒व॒न्ति॒ । प॒ञ्चा॒ह इति॑ पञ्च - अ॒हः । भ॒व॒ति॒ । पञ्च॑ । वै । ऋ॒तवः॑ । सं॒ॅव॒थ्स॒र इति॑ सं - व॒थ्स॒रः । ऋ॒तुषु॑ । ए॒व । सं॒ॅव॒थ्स॒र इति॑ सं - व॒थ्स॒रे । प्रतीति॑ । ति॒ष्ठ॒न्ति॒ । अथा॒ इति॑ । पञ्चा᳚क्ष॒रेति॒ पञ्च॑ - अ॒क्ष॒रा॒ । प॒ङ्क्तिः । पाङ्क्तः॑ । य॒ज्ञ्ः । य॒ज्ञ्म् । ए॒व । अवेति॑ । रु॒न्ध॒ते॒ । अस॑त्रम् । वै । ए॒तत् । \textbf{  20} \newline
                  \newline
                                \textbf{ TS 7.3.8.2} \newline
                  यत् । अ॒छ॒न्दो॒ममित्य॑छन्दः-मम् । यत् । छ॒न्दो॒मा इति॑ छन्दः - माः । भव॑न्ति । तेन॑ । स॒त्रम् । दे॒वताः᳚ । ए॒व । पृ॒ष्ठैः । अवेति॑ । रु॒न्ध॒ते॒ । प॒शून् । छ॒न्दो॒मैरिति॑ छन्दः - मैः । ओजः॑ । वै । वी॒र्य᳚म् । पृ॒ष्ठानि॑ । प॒शवः॑ । छ॒न्दो॒मा इति॑ छन्दः - माः । ओज॑सि । ए॒व । वी॒र्ये᳚ । प॒शुषु॑ । प्रतीति॑ । ति॒ष्ठ॒न्ति॒ । स॒प्त॒द॒श॒रा॒त्र इति॑ सप्तदश - रा॒त्रः । भ॒व॒ति॒ । स॒प्त॒द॒श इति॑ सप्त - द॒शः । प्र॒जाप॑ति॒रिति॑ प्र॒जा-प॒तिः॒ । प्र॒जाप॑ते॒रिति॑ प्र॒जा - प॒तेः॒ । आप्त्यै᳚ । अ॒ति॒रा॒त्रावित्य॑ति - रा॒त्रौ । अ॒भितः॑ । भ॒व॒तः॒ । अ॒न्नाद्य॒स्येत्य॑न्न - अद्य॑स्य । परि॑गृहीत्या॒ इति॒ परि॑ - गृ॒ही॒त्यै॒ ॥ \textbf{  21} \newline
                  \newline
                      (ए॒तथ् - स॒प्तत्रिꣳ॑शच्च)  \textbf{(A8)} \newline \newline
                                \textbf{ TS 7.3.9.1} \newline
                  सा । वि॒राडिति॑ वि - राट् । वि॒क्रम्येति॑ वि - क्रम्य॑ । अ॒ति॒ष्ठ॒त् । ब्रह्म॑णा । दे॒वेषु॑ । अन्ने॑न । असु॑रेषु । ते । दे॒वाः । अ॒का॒म॒य॒न्त॒ । उ॒भय᳚म् । समिति॑ । वृ॒ञ्जी॒म॒हि॒ । ब्रह्म॑ । च॒ । अन्न᳚म् । च॒ । इति॑ । ते । ए॒ताः । विꣳ॒॒श॒तिम् । रात्रीः᳚ । अ॒प॒श्य॒न्न् । ततः॑ । वै । ते । उ॒भय᳚म् । समिति॑ । अ॒वृ॒ञ्ज॒त॒ । ब्रह्म॑ । च॒ । अन्न᳚म् । च॒ । ब्र॒ह्म॒व॒र्च॒सिन॒ इति॑ ब्रह्म - व॒र्च॒सिनः॑ । अ॒न्ना॒दा इत्य॑न्न - अ॒दाः । अ॒भ॒व॒न्न् । ये । ए॒वम् । वि॒द्वाꣳसः॑ । ए॒ताः । आस॑ते । उ॒भय᳚म् । ए॒व । समिति॑ । वृ॒ञ्ज॒ते॒ । ब्रह्म॑ । च॒ । अन्न᳚म् । च॒ । \textbf{  22} \newline
                  \newline
                                \textbf{ TS 7.3.9.2} \newline
                  ब्र॒ह्म॒व॒र्च॒सिन॒ इति॑ ब्रह्म - व॒र्च॒सिनः॑ । अ॒न्ना॒दा इत्य॑न्न - अ॒दाः । भ॒व॒न्ति॒ । द्वे इति॑ । वै । ए॒ते इति॑ । वि॒राजा॒विति॑ वि-राजौ᳚ । तयोः᳚ । ए॒व । नाना᳚ । प्रतीति॑ । ति॒ष्ठ॒न्ति॒ । विꣳ॒॒शः । वै । पुरु॑षः । दश॑ । हस्त्याः᳚ । अ॒ङ्गुल॑यः । दश॑ । पद्याः᳚ । यावान्॑ । ए॒व । पुरु॑षः । तम् । आ॒प्त्वा । उदिति॑ । ति॒ष्ठ॒न्ति॒ । ज्योतिः॑ । गौः । आयुः॑ । इति॑ । त्र्य॒हा इति॑ त्रि - अ॒हाः । भ॒व॒न्ति॒ । इ॒यम् । वाव । ज्योतिः॑ । अ॒न्तरि॑क्षम् । गौः । अ॒सौ । आयुः॑ । इ॒मान् । ए॒व । लो॒कान् । अ॒भ्यारो॑ह॒न्तीत्य॑भि - आरो॑हन्ति । अ॒भि॒पू॒र्वमित्य॑भि - पू॒र्वम् । त्र्य॒हा इति॑ त्रि - अ॒हाः । भ॒व॒न्ति॒ । अ॒भि॒पू॒र्वमित्य॑भि - पू॒र्वम् । ए॒व । सु॒व॒र्गमिति॑ सुवः - गम् । \textbf{  23} \newline
                  \newline
                                \textbf{ TS 7.3.9.3} \newline
                  लो॒कम् । अ॒भ्यारो॑ह॒न्तीत्य॑भि-आरो॑हन्ति । यत् । अ॒न्यतः॑ । पृ॒ष्ठानि॑ । स्युः । विवि॑वध॒मिति॒ वि - वि॒व॒ध॒म् । स्या॒त् । मद्ध्ये᳚ । पृ॒ष्ठानि॑ । भ॒व॒न्ति॒ । स॒वि॒व॒ध॒त्वायेति॑ सविवध - त्वाय॑ । ओजः॑ । वै । वी॒र्य᳚म् । पृ॒ष्ठानि॑ । ओजः॑ । ए॒व । वी॒र्य᳚म् । म॒द्ध्य॒तः । द॒ध॒ते॒ । बृ॒ह॒द्र॒थ॒न्त॒राभ्या॒मिति॑ बृहत् - र॒थ॒न्त॒राभ्या᳚म् । य॒न्ति॒ । इ॒यम् । वाव । र॒थ॒न्त॒रमिति॑ रथं - त॒रम् । अ॒सौ । बृ॒हत् । आ॒भ्याम् । ए॒व । य॒न्ति॒ । अथो॒ इति॑ । अ॒नयोः᳚ । ए॒व । प्रतीति॑ । ति॒ष्ठ॒न्ति॒ । ए॒ते इति॑ । वै । य॒ज्ञ्स्य॑ । अ॒ञ्ज॒साय॑नी॒ इत्य॑ञ्जसा - अय॑नी । स्रु॒ती इति॑ । ताभ्या᳚म् । ए॒व । सु॒व॒र्गमिति॑ सुवः - गम् । लो॒कम् । य॒न्ति॒ । परा᳚ञ्चः । वै । ए॒ते । सु॒व॒र्गमिति॑ सुवः - गम् ( ) । लो॒कम् । अ॒भ्यारो॑ह॒न्तीत्य॑भि - आरो॑हन्ति । ये । प॒रा॒चीना॑नि । पृ॒ष्ठानि॑ । उ॒प॒यन्तीत्यु॑प - यन्ति॑ । प्र॒त्यङ् । त्र्य॒ह इति॑ त्रि - अ॒हः । भ॒व॒ति॒ । प्र॒त्यव॑रूढ्या॒ इति॑ प्रति - अव॑रूढ्यै । अथो॒ इति॑ । प्रति॑ष्ठित्या॒ इति॒ प्रति॑ - स्थि॒त्यै॒ । उ॒भयोः᳚ । लो॒कयोः᳚ । ऋ॒द्ध्वा । उदिति॑ । ति॒ष्ठ॒न्ति॒ । अ॒ति॒रा॒त्रावित्य॑ति - रा॒त्रौ । अ॒भितः॑ । भ॒व॒तः॒ । ब्र॒ह्म॒व॒र्च॒सस्येति॑ ब्रह्म - व॒र्च॒सस्य॑ । अ॒न्नाद्य॒स्येत्य॑न्न - अद्य॑स्य । परि॑गृहीत्या॒ इति॒ परि॑ - गृ॒ही॒त्यै॒ ॥ \textbf{  24 } \newline
                  \newline
                      (वृ॒ञ्ज॒ते॒ ब्रह्म॒ चां नं॑ च - सुव॒र्ग - मे॒ते सु॑व॒र्गं - त्रयो॑विꣳशतिश्च)  \textbf{(A9)} \newline \newline
                                \textbf{ TS 7.3.10.1} \newline
                  अ॒सौ । आ॒दि॒त्यः । अ॒स्मिन्न् । लो॒के । आ॒सी॒त् । तम् । दे॒वाः । पृ॒ष्ठैः । प॒रि॒गृह्येति॑ परि - गृह्य॑ । सु॒व॒र्गमिति॑ सुवः - गम् । लो॒कम् । अ॒ग॒म॒य॒न्न् । परैः᳚ । अ॒वस्ता᳚त् । परीति॑ । अ॒गृ॒ह्ण॒न्न् । दि॒वा॒की॒र्त्ये॑नेति॑ दिवा - की॒र्त्ये॑न । सु॒व॒र्ग इति॑ सुवः - गे । लो॒के । प्रतीति॑ । अ॒स्था॒प॒य॒न्न् । परैः᳚ । प॒रस्ता᳚त् । परीति॑ । अ॒गृ॒ह्ण॒न्न् । पृ॒ष्ठैः । उ॒पावा॑रोह॒न्नित्यु॑प - अवा॑रोहन्न् । सः । वै । अ॒सौ । आ॒दि॒त्यः । अ॒मुष्मिन्न्॑ । लो॒के । परैः᳚ । उ॒भ॒यतः॑ । परि॑गृहीत॒ इति॒ परि॑-गृ॒ही॒तः॒ । यत् । पृ॒ष्ठानि॑ । भव॑न्ति । सु॒व॒र्गमिति॑ सुवः - गम् । ए॒व । तैः । लो॒कम् । यज॑मानाः । य॒न्ति॒ । परैः᳚ । अ॒वस्ता᳚त् । परीति॑ । गृ॒ह्ण॒न्ति॒ । दि॒वा॒की॒र्त्ये॑नेति॑ दिवा - की॒र्त्ये॑न । \textbf{  25} \newline
                  \newline
                                \textbf{ TS 7.3.10.2} \newline
                  सु॒व॒र्ग इति॑ सुवः-गे । लो॒के । प्रतीति॑ । ति॒ष्ठ॒न्ति॒ । परैः᳚ । प॒रस्ता᳚त् । परीति॑ । गृ॒ह्ण॒न्ति॒ । पृ॒ष्ठैः । उ॒पाव॑रोह॒न्तीत्यु॑प - अव॑रोहन्ति । यत् । परे᳚ । प॒रस्ता᳚त् । न । स्युः । परा᳚ञ्चः । सु॒व॒र्गादिति॑ सुवः - गात् । लो॒कात् । निरिति॑ । प॒द्ये॒र॒न्न् । यत् । अ॒वस्ता᳚त् । न । स्युः । प्र॒जा इति॑ प्र - जाः । निरिति॑ । द॒हे॒युः॒ । अ॒भितः॑ । दि॒वा॒की॒र्त्य॑मिति॑ दिवा - की॒र्त्य᳚म् । पर॑स्सामान॒ इति॒ परः॑ - सा॒मा॒नः॒ । भ॒व॒न्ति॒ । सु॒व॒र्ग इति॑ सुवः - गे । ए॒व । ए॒ना॒न्न् । लो॒के । उ॒भ॒यतः॑ । परीति॑ । गृ॒ह्ण॒न्ति॒ । यज॑मानाः । वै । दि॒वा॒की᳚र्त्यमिति॑ दिवा - की॒र्त्य᳚म् । सं॒ॅव॒थ्स॒र इति॑ सं - व॒थ्स॒रः । पर॑स्सामान॒ इति॒ परः॑ - सा॒मा॒नः॒ । अ॒भितः॑ । दि॒वा॒की॒र्त्य॑मिति॑ दिवा - की॒र्त्य᳚म् । पर॑स्सामान॒ इति॒ परः॑ - सा॒मा॒नः॒ । भ॒व॒न्ति॒ । सं॒ॅव॒थ्स॒र इति॑ सं - व॒थ्स॒रे । ए॒व । उ॒भ॒यतः॑ । \textbf{  26} \newline
                  \newline
                                \textbf{ TS 7.3.10.3} \newline
                  प्रतीति॑ । ति॒ष्ठ॒न्ति॒ । पृ॒ष्ठम् । वै । दि॒वा॒की॒र्त्य॑मिति॑ दिवा - की॒र्त्य᳚म् । पा॒र्श्वे इति॑ । पर॑स्सामान॒ इति॒ परः॑ - सा॒मा॒नः॒ । अ॒भितः॑ । दि॒वा॒की॒र्त्य॑मिति॑ दिवा - की॒र्त्य᳚म् । पर॑स्सामान॒ इति॒ परः॑-सा॒मा॒नः॒ । भ॒व॒न्ति॒ । तस्मा᳚त् । अ॒भितः॑ । पृ॒ष्ठम् । पा॒र्श्वे इति॑ । भूयि॑ष्ठाः । ग्रहाः᳚ । गृ॒ह्य॒न्ते॒ । भूयि॑ष्ठम् । श॒स्य॒ते॒ । य॒ज्ञ्स्य॑ । ए॒व । तत् । म॒द्ध्य॒तः । ग्र॒न्थिम् । ग्र॒थ्न॒न्ति॒ । अवि॑स्रꣳसा॒येत्यवि॑ - स्रꣳ॒॒सा॒य॒ । स॒प्त । गृ॒ह्य॒न्ते॒ । स॒प्त । वै । शी॒र्.॒ष॒ण्याः᳚ । प्रा॒णा इति॑ प्र - अ॒नाः । प्रा॒णानिति॑ प्र - अ॒नान् । ए॒व । यज॑मानेषु । द॒ध॒ति॒ । यत् । प॒रा॒चीना॑नि । पृ॒ष्ठानि॑ । भव॑न्ति । अ॒मुम् । ए॒व । तैः । लो॒कम् । अ॒भ्यारो॑ह॒न्तीत्य॑भि-आरो॑हन्ति । यत् । इ॒मम् । लो॒कम् । न । \textbf{  27} \newline
                  \newline
                                \textbf{ TS 7.3.10.4} \newline
                  प्र॒त्य॒व॒रोहे॑यु॒रिति॑ प्रति - अ॒व॒रोहे॑युः । उदिति॑ । वा॒ । माद्ये॑युः । यज॑मानाः । प्रेति॑ । वा॒ । मी॒ये॒र॒न्न् । यत् । प्र॒ती॒चीना॑नि । पृ॒ष्ठानि॑ । भव॑न्ति । इ॒मम् । ए॒व । तैः । लो॒कम् । प्र॒त्यव॑रोह॒न्तीति॑ प्रति - अव॑रोहन्ति । अथो॒ इति॑ । अ॒स्मिन्न् । ए॒व । लो॒के । प्रतीति॑ । ति॒ष्ठ॒न्ति॒ । अनु॑न्मादा॒येत्यनु॑त् - मा॒दा॒य॒ । इन्द्रः॑ । वै । अप्र॑तिष्ठित॒ इत्यप्र॑ति - स्थि॒तः॒ । आ॒सी॒त् । सः । प्र॒जाप॑ति॒मिति॑ प्र॒जा-प॒ति॒म् । उपेति॑ । अ॒धा॒व॒त् । तस्मै᳚ । ए॒तम् । ए॒क॒विꣳ॒॒श॒ति॒रा॒त्रमित्ये॑कविꣳशति - रा॒त्रम् । प्रेति॑ । अ॒य॒च्छ॒त् । तम् । एति॑ । अ॒ह॒र॒त् । तेन॑ । अ॒य॒ज॒त॒ । ततः॑ । वै । सः । प्रतीति॑ । अ॒ति॒ष्ठ॒त् । ये । ब॒हु॒या॒जिन॒ इति॑ बहु - या॒जिनः॑ । अप्र॑तिष्ठिता॒ इत्यप्र॑ति - स्थि॒ताः॒ । \textbf{  28} \newline
                  \newline
                                \textbf{ TS 7.3.10.5} \newline
                  स्युः । ते । ए॒क॒विꣳ॒॒श॒ति॒रा॒त्रमित्ये॑कविꣳशति - रा॒त्रम् । आ॒सी॒र॒न्न् । द्वाद॑श । मासाः᳚ । पञ्च॑ । ऋ॒तवः॑ । त्रयः॑ । इ॒मे । लो॒काः । अ॒सौ । आ॒दि॒त्यः । ए॒क॒विꣳ॒॒श इत्ये॑क - विꣳ॒॒शः । ए॒ताव॑न्तः । वै । दे॒व॒लो॒का इति॑ देव - लो॒काः । तेषु॑ । ए॒व । य॒था॒पू॒र्वमिति॑ यथा - पू॒र्वम् । प्रतीति॑ । ति॒ष्ठ॒न्ति॒ । अ॒सौ । आ॒दि॒त्यः । न । वीति॑ । अ॒रो॒च॒त॒ । सः । प्र॒जाप॑ति॒मिति॑ प्र॒जा - प॒ति॒म् । उपेति॑ । अ॒धा॒व॒त् । तस्मै᳚ । ए॒तम् । ए॒क॒विꣳ॒॒श॒ति॒रा॒त्रमित्ये॑कविꣳशति - रा॒त्रम् । प्रेति॑ । अ॒य॒च्छ॒त् । तम् । एति॑ । अ॒ह॒र॒त् । तेन॑ । अ॒य॒ज॒त॒ । ततः॑ । वै । सः । अ॒रो॒च॒त॒ । ये । ए॒वम् । वि॒द्वाꣳसः॑ । ए॒क॒विꣳ॒॒श॒ति॒रा॒त्रमित्ये॑कविꣳशति - रा॒त्रम् । आस॑ते ( ) । रोच॑न्ते । ए॒व । ए॒क॒विꣳ॒॒श॒ति॒रा॒त्र इत्येक॑विꣳशति - रा॒त्रः । भ॒व॒ति॒ । रुक् । वै । ए॒क॒विꣳ॒॒श इत्ये॑क - विꣳ॒॒शः । रुच᳚म् । ए॒व । ग॒च्छ॒न्ति॒ । अथो॒ इति॑ । प्र॒ति॒ष्ठामिति॑ प्रति - स्थाम् । ए॒व । प्र॒ति॒ष्ठेति॑ प्रति - स्था । हि । ए॒क॒विꣳ॒॒श इत्ये॑क - विꣳ॒॒शः । अ॒ति॒रा॒त्रावित्य॑ति - रा॒त्रौ । अ॒भितः॑ । भ॒व॒तः॒ । ब्र॒ह्म॒व॒र्च॒सस्येति॑ ब्रह्म - व॒र्च॒सस्य॑ । परि॑गृहीत्या॒ इति॒ परि॑ - गृ॒ही॒त्यै॒ ॥ \textbf{  29} \newline
                  \newline
                      (गृ॒ह्ण॒न्ति॒ दि॒वा॒की॒र्त्ये॑नै॒ - वोभ॒यतो॒ - ना - प्र॑तिष्ठिता॒ - आस॑त॒ - एक॑विꣳशतिश्च)  \textbf{(A10)} \newline \newline
                                \textbf{ TS 7.3.11.1} \newline
                  अ॒र्वाङ् । य॒ज्ञ्ः । समिति॑ । क्रा॒म॒तु॒ । अ॒मुष्मा᳚त् । अधीति॑ । माम् । अ॒भि ॥ ऋषी॑णाम् । यः । पु॒रोहि॑त॒ इति॑ पु॒रः - हि॒तः॒ ॥ निर्दे॑व॒मिति॒ निः - दे॒व॒म् । निर्वी॑र॒मिति॒ निः - वी॒र॒म् । कृ॒त्वा । विष्क॑न्ध॒मिति॒ वि - स्क॒न्ध॒म् । तस्मिन्न्॑ । ही॒य॒ता॒म् । यः । अ॒स्मान् । द्वेष्टि॑ ॥ शरी॑रम् । य॒ज्ञ्॒श॒म॒लमिति॑ यज्ञ् - श॒म॒लम् । कुसी॑दम् । तस्मिन्न्॑ । सी॒द॒तु॒ । यः । अ॒स्मान् । द्वेष्टि॑ ॥ यज्ञ्॑ । य॒ज्ञ्स्य॑ । यत् । तेजः॑ । तेन॑ । समिति॑ । क्रा॒म॒ । माम् । अ॒भि ॥ ब्रा॒ह्म॒णान् । ऋ॒त्विजः॑ । दे॒वान् । य॒ज्ञ्स्य॑ । तप॑सा । ते॒ । स॒व॒ । अ॒हम् । एति॑ । हु॒वे॒ ॥ इ॒ष्टेन॑ । प॒क्वम् । उपेति॑ । \textbf{  30} \newline
                  \newline
                                \textbf{ TS 7.3.11.2} \newline
                  ते॒ । हु॒वे॒ । स॒व॒ । अ॒हम् ॥ समिति॑ । ते॒ । वृ॒ञ्जे॒ । सु॒कृ॒तमिति॑ सु - कृ॒तम् । समिति॑ । प्र॒जामिति॑ प्र - जाम् । प॒शून् ॥ प्रै॒षानिति॑ प्र - ए॒षान् । सा॒मि॒धे॒नीरिति॑ सां - इ॒धे॒नीः । आ॒घा॒रावित्या᳚ - घा॒रौ । आज्य॑भागा॒वित्याज्य॑ - भा॒गौ॒ । आश्रु॑त॒मित्या - श्रु॒त॒म् । प्र॒त्याश्रु॑त॒मिति॑ प्रति - आश्रु॑तम् । एति॑ । शृ॒णा॒मि॒ । ते॒ ॥ प्र॒या॒जा॒नू॒या॒जानिति॑ प्रयाज - अ॒नू॒या॒जान् । स्वि॒ष्ट॒कृत॒मिति॑ स्विष्ट - कृत᳚म् । इडा᳚म् । आ॒शिष॒ इत्या᳚ - शिषः॑ । एति॑ । वृ॒ञ्जे॒ । सुवः॑ ॥ अ॒ग्निना᳚ । इन्द्रे॑ण । सोमे॑न । सर॑स्वत्या । विष्णु॑ना । दे॒वता॑भिः ॥ या॒ज्या॒नु॒वा॒क्या᳚भ्या॒मिति॑ याज्या - अ॒नु॒वा॒क्या᳚भ्याम् । उपेति॑ । ते॒ । हु॒वे॒ । स॒व॒ । अ॒हम् । य॒ज्ञ्म् । एति॑ । द॒दे॒ । ते॒ । वष॑ट्कृत॒मिति॒ वष॑ट् - कृ॒त॒म् ॥ स्तु॒तम् । श॒स्त्रम् । प्र॒ति॒ग॒रमिति॑ प्रति - ग॒रम् । ग्रह᳚म् । इडा᳚म् । आ॒शिष॒ इत्या᳚ - शिषः॑ । \textbf{  31} \newline
                  \newline
                                \textbf{ TS 7.3.11.3} \newline
                  एति॑ । वृ॒ञ्जे॒ । सुवः॑ ॥ प॒त्नी॒सं॒ॅया॒जानिति॑ पत्नी-सं॒ॅया॒जान् । उपेति॑ । ते॒ । हु॒वे॒ । स॒व॒ । अ॒हम् । स॒मि॒ष्ट॒य॒जुरिति॑ समिष्ट - य॒जुः । एति॑ । द॒दे॒ । तव॑ ॥ प॒शून् । सु॒तम् । पु॒रो॒डाशान्॑ । सव॑नानि । एति॑ । उ॒त । य॒ज्ञ्म् ॥ दे॒वान् । सेन्द्रा॒निति॒ स-इ॒न्द्रा॒न् । उपेति॑ । ते॒ । हु॒वे॒ । स॒व॒ । अ॒हम् । अ॒ग्निमु॑खा॒नित्य॒ग्नि - मु॒खा॒न् । सोम॑वत॒ इति॒ सोम॑-व॒तः॒ । ये । च॒ । विश्वे᳚ ॥ \textbf{  32 } \newline
                  \newline
                      (उप॒ - ग्रह॒मिडा॑मा॒शिषो॒ - द्वात्रिꣳ॑शच्च)  \textbf{(A11)} \newline \newline
                                \textbf{ TS 7.3.12.1} \newline
                  भू॒तम् । भव्य᳚म् । भ॒वि॒ष्यत् । वष॑ट् । स्वाहा᳚ । नमः॑ । ऋक् । साम॑ । यजुः॑ । वष॑ट् । स्वाहा᳚ । नमः॑ । गा॒य॒त्री । त्रि॒ष्टुप् । जग॑ती । वष॑ट् । स्वाहा᳚ । नमः॑ । पृ॒थिवी । अ॒न्तरि॑क्षम् । द्यौः । वष॑ट् । स्वाहा᳚ । नमः॑ । अ॒ग्निः । वा॒युः । सूर्यः॑ । वष॑ट् । स्वाहा᳚ । नमः॑ । प्रा॒ण इति॑ प्र - अ॒नः । व्या॒न इति॑ वि - अ॒नः । अ॒पा॒न इत्य॑प-अ॒नः । वष॑ट् । स्वाहा᳚ । नमः॑ । अन्न᳚म् । कृ॒षिः । वृष्टिः॑ । वष॑ट् । स्वाहा᳚ । नमः॑ । पि॒ता । पु॒त्रः । पौत्रः॑ । वष॑ट् । स्वाहा᳚ । नमः॑ । भूः । भुवः॑ ( ) । सुवः॑ । वष॑ट् । स्वाहा᳚ । नमः॑ ॥ \textbf{  33 } \newline
                  \newline
                      (भुव॑ - श्च॒त्वारि॑ च)  \textbf{(A12)} \newline \newline
                                \textbf{ TS 7.3.13.1} \newline
                  एति॑ । मे॒ । गृ॒हाः । भ॒व॒न्तु॒ । एति॑ । प्र॒जेति॑ प्र - जा । मे॒ । एति॑ । मा॒ । य॒ज्ञ्ः । वि॒श॒तु॒ । वी॒र्या॑वा॒निति॑ वी॒र्य॑ - वा॒न् ॥ आपः॑ । दे॒वीः । य॒ज्ञियाः᳚ । मा॒ । एति॑ । वि॒श॒न्तु॒ । स॒हस्र॑स्य । मा॒ । भू॒मा । मा । प्रेति॑ । हा॒सी॒त् ॥ एति॑ । मे॒ । ग्रहः॑ । भ॒व॒तु॒ । एति॑ । पु॒रो॒रुगिति॑ पुरः - रुक् । स्तु॒त॒श॒स्त्रे इति॑ स्तुत - श॒स्त्रे । मा॒ । एति॑ । वि॒श॒ता॒म् । स॒मीची॒ इति॑ ॥ आ॒दि॒त्याः । रु॒द्राः । वस॑वः । मे॒ । स॒द॒स्याः᳚ । स॒हस्र॑स्य । मा॒ । भू॒मा । मा । प्रेति॑ । हा॒सी॒त् ॥ एति॑ । मा॒ । अ॒ग्नि॒ष्टो॒म इत्य॑ग्नि - स्तो॒मः । वि॒श॒तु॒ ( ) । उ॒क्थ्यः॑ । च॒ । अ॒ति॒रा॒त्र इत्य॑ति - रा॒त्रः । मा॒ । एति॑ । वि॒श॒तु॒ । आ॒पि॒श॒र्व॒र इत्या॑पि - श॒र्व॒रः ॥ ति॒रो‌अ॑ह्निया॒ इति॑ ति॒रः - अ॒ह्नि॒याः॒ । मा॒ । सुहु॑ता॒ इति॒ सु - हु॒ताः॒ । एति॑ । वि॒श॒न्तु॒ । स॒हस्र॑स्य । मा॒ । भू॒मा । मा । प्रेति॑ । हा॒सी॒त् ॥ \textbf{  34 } \newline
                  \newline
                      (अ॒ग्नि॒ष्टो॒मो वि॑शत्व॒ - ष्टाद॑श च)  \textbf{(A13)} \newline \newline
                                \textbf{ TS 7.3.14.1} \newline
                  अ॒ग्निना᳚ । तपः॑ । अन्विति॑ । अ॒भ॒व॒त् । वा॒चा । ब्रह्म॑ । म॒णिना᳚ । रू॒पाणि॑ । इन्द्रे॑ण । दे॒वान् । वाते॑न । प्रा॒णानिति॑ प्र - अ॒नान् । सूर्ये॑ण । द्याम् । च॒न्द्रम॑सा । नक्ष॑त्राणि । य॒मेन॑ । पि॒तॄन् । राज्ञा᳚ । म॒नु॒ष्यान्॑ । फ॒लेन॑ । ना॒दे॒यान् । अ॒ज॒ग॒रेण॑ । स॒र्पान् । व्या॒घ्रेण॑ । आ॒र॒ण्यान् । प॒शून् । श्ये॒नेन॑ । प॒त॒त्रिणः॑ । वृष्णा᳚ । अश्वान्॑ । ऋ॒ष॒भेण॑ । गाः । ब॒स्तेन॑ । अ॒जाः । वृ॒ष्णिना᳚ । अवीः᳚ । व्री॒हिणा᳚ । अन्ना॑नि । यवे॑न । ओष॑धीः । न्य॒ग्रोधे॑न । वन॒स्पतीन्॑ । उ॒दु॒म्बरे॑ण । ऊर्ज᳚म् । गा॒य॒त्रि॒या । छन्दाꣳ॑सि । त्रि॒वृतेति॑ त्रि - वृता᳚ । स्तोमान्॑ । ब्रा॒ह्म॒णेन॑ ( ) । वाच᳚म् ॥ \textbf{  35} \newline
                  \newline
                      (ब्रा॒ह्म॒णेनै - कं॑ च)  \textbf{(A14)} \newline \newline
                                \textbf{ TS 7.3.15.1} \newline
                  स्वाहा᳚ । आ॒धिमित्या᳚ - धिम् । आधी॑ता॒येत्या - धी॒ता॒य॒ । स्वाहा᳚ । स्वाहा᳚ । आधी॑त॒मित्या - धी॒त॒म् । मन॑से । स्वाहा᳚ । स्वाहा᳚ । मनः॑ । प्र॒जाप॑तय॒ इति॑ प्र॒जा - प॒त॒ये॒ । स्वाहा᳚ । काय॑ । स्वाहा᳚ । कस्मै᳚ । स्वाहा᳚ । क॒त॒मस्मै᳚ । स्वाहा᳚ । अदि॑त्यै । स्वाहा᳚ । अदि॑त्यै । म॒ह्यै᳚ । स्वाहा᳚ । अदि॑त्यै । सु॒मृ॒डी॒काया॒ इति॑ सु - मृ॒डी॒कायै᳚ । स्वाहा᳚ । सर॑स्वत्यै । स्वाहा᳚ । सर॑स्वत्यै । बृ॒ह॒त्यै᳚ । स्वाहा᳚ । सर॑स्वत्यै । पा॒व॒कायै᳚ । स्वाहा᳚ । पू॒ष्णे । स्वाहा᳚ । पू॒ष्णे । प्र॒प॒थ्या॑येति॑ प्र-प॒थ्या॑य । स्वाहा᳚ । पू॒ष्णे । न॒रन्धि॑षाय । स्वाहा᳚ । त्वष्ट्रे᳚ । स्वाहा᳚ । त्वष्ट्रे᳚ । तु॒रीपा॑य । स्वाहा᳚ । त्वष्ट्रे᳚ । पु॒रु॒रूपा॒येति॑ पुरु - रूपा॑य । स्वाहा᳚ ( ) । विष्ण॑वे । स्वाहा᳚ । विष्ण॑वे । नि॒खु॒र्य॒पायेति॑ निखुर्य - पाय॑ । स्वाहा᳚ । विष्ण॑वे । नि॒भू॒य॒पायेति॑ निभूय - पाय॑ । स्वाहा᳚ । सर्व॑स्मै । स्वाहा᳚ ॥ \textbf{  36 } \newline
                  \newline
                      (पु॒रु॒रूपा॑य॒ स्वाहा॒ - दश॑ च)  \textbf{(A15)} \newline \newline
                                \textbf{ TS 7.3.16.1} \newline
                  द॒द्भ्य इति॑ दत् - भ्यः । स्वाहा᳚ । हनू᳚भ्या॒मिति॒ हनु॑ - भ्या॒म् । स्वाहा᳚ । ओष्ठा᳚भ्याम् । स्वाहा᳚ । मुखा॑य । स्वाहा᳚ । नासि॑काभ्याम् । स्वाहा᳚ । अ॒क्षीभ्या᳚म् । स्वाहा᳚ । कर्णा᳚भ्याम् । स्वाहा᳚ । पा॒रे । इ॒क्षवः॑ । अ॒वा॒र्ये᳚भ्यः । पक्ष्म॑भ्य॒ इति॒ पक्ष्म॑ - भ्यः॒ । स्वाहा᳚ । अ॒वा॒रे । इ॒क्षवः॑ । पा॒र्ये᳚भ्यः । पक्ष्म॑भ्य॒ इति॒ पक्ष्म॑ - भ्यः॒ । स्वाहा᳚ । शी॒र्ष्णे । स्वाहा᳚ । भ्रू॒भ्याम् । स्वाहा᳚ । ल॒लाटा॑य । स्वाहा᳚ । मू॒द्‌र्ध्ने । स्वाहा᳚ । म॒स्तिष्का॑य । स्वाहा᳚ । केशे᳚भ्यः । स्वाहा᳚ । वहा॑य । स्वाहा᳚ । ग्री॒वाभ्यः॑ । स्वाहा᳚ । स्क॒न्धेभ्यः॑ । स्वाहा᳚ । कीक॑साभ्यः । स्वाहा᳚ । पृ॒ष्टीभ्य॒ इति॑ पृ॒ष्टि - भ्यः॒ । स्वाहा᳚ । पा॒ज॒स्या॑य । स्वाहा᳚ । पा॒र्श्वाभ्या᳚म् । स्वाहा᳚ । \textbf{  37} \newline
                  \newline
                                \textbf{ TS 7.3.16.2} \newline
                  अꣳसा᳚भ्याम् । स्वाहा᳚ । दो॒षभ्या॒मिति॑ दो॒ष - भ्या॒म् । स्वाहा᳚ । बा॒हुभ्या॒मिति॑ बा॒हु - भ्या॒म् । स्वाहा᳚ । जङ्घा᳚भ्याम् । स्वाहा᳚ । श्रोणी᳚भ्या॒मिति॒ श्राणि॑ - भ्या॒म् । स्वाहा᳚ । ऊ॒रुभ्या॒मित्यू॒रु - भ्या॒म् । स्वाहा᳚ । अ॒ष्ठी॒वद्भ्या॒मित्य॑ष्ठी॒वत् - भ्या॒म् । स्वाहा᳚ । जङ्घा᳚भ्याम् । स्वाहा᳚ । भ॒सदे᳚ । स्वाहा᳚ । शि॒ख॒ण्डेभ्यः॑ । स्वाहा᳚ । वा॒ल॒धाना॒येति॑ वाल - धाना॑य । स्वाहा᳚ । अ॒ण्डाभ्या᳚म् । स्वाहा᳚ । शेपा॑य । स्वाहा᳚ । रेत॑से । स्वाहा᳚ । प्र॒जाभ्य॒ इति॑ प्र - जाभ्यः॑ । स्वाहा᳚ । प्र॒जन॑ना॒येति॑ प्र - जन॑नाय । स्वाहा᳚ । प॒द्भ्य इति॑ पत् - भ्यः । स्वाहा᳚ । श॒फेभ्यः॑ । स्वाहा᳚ । लोम॑भ्य॒ इति॒ लोम॑ - भ्यः॒ । स्वाहा᳚ । त्व॒चे । स्वाहा᳚ । लोहि॑ताय । स्वाहा᳚ । माꣳ॒॒साय॑ । स्वाहा᳚ । स्नाव॑भ्य॒ इति॒ स्नाव॑ - भ्यः॒ । स्वाहा᳚ । अ॒स्थभ्य॒ इत्य॒स्थ - भ्यः॒ । स्वाहा᳚ । म॒ज्जभ्य॒ इति॑ म॒ज्ज - भ्यः॒ । स्वाहा᳚ ( ) । अङ्गे᳚भ्यः । स्वाहा᳚ । आ॒त्मने᳚ । स्वाहा᳚ । सर्व॑स्मै । स्वाहा᳚ ॥ \textbf{  38} \newline
                  \newline
                      (पा॒र्श्वाभ्याꣳ॒॒ स्वाहा॑ - म॒ज्जभ्यः॒ स्वाहा॒ - षट् च॑)  \textbf{(A16)} \newline \newline
                                \textbf{ TS 7.3.17.1} \newline
                  अ॒ञ्ज्ये॒तायेत्य॑ञ्जि - ए॒ताय॑ । स्वाहा᳚ । अ॒ञ्जि॒स॒क्थायेत्य॑ञ्जि-स॒क्थाय॑ । स्वाहा᳚ । शि॒ति॒पद॒ इति॑ शिति - पदे᳚ । स्वाहा᳚ । शिति॑ककुद॒ इति॒ शिति॑ - क॒कु॒दे॒ । स्वाहा᳚ । शि॒ति॒रन्ध्रा॒येति॑ शिति - रन्ध्रा॑य । स्वाहा᳚ । शि॒ति॒पृ॒ष्ठायेति॑ शिति - पृ॒ष्ठाय॑ । स्वाहा᳚ । शि॒त्यꣳसा॒येति॑ शिति - अꣳसा॑य । स्वाहा᳚ । पु॒ष्प॒कर्णा॒येति॑ पुष्प - कर्णा॑य । स्वाहा᳚ । शि॒त्योष्ठा॒येति॑ शिति - ओष्ठा॑य । स्वाहा᳚ । शि॒ति॒भ्रव॒ इति॑ शिति - भ्रवे᳚ । स्वाहा᳚ । शिति॑भसद॒ इति॒ शिति॑ - भ॒स॒दे॒ । स्वाहा᳚ । श्वे॒तानू॑काशा॒येति॑ श्वे॒त - अ॒नू॒का॒शा॒य॒ । स्वाहा᳚ । अ॒ञ्जये᳚ । स्वाहा᳚ । ल॒लामा॑य । स्वाहा᳚ । असि॑तज्ञ्व॒ इत्यसि॑त - ज्ञ्॒वे॒ । स्वाहा᳚ । कृ॒ष्णै॒तायेति॑ कृष्ण - ए॒ताय॑ । स्वाहा᳚ । रो॒हि॒तै॒तायेति॑ रोहित - ए॒ताय॑ । स्वाहा᳚ । अ॒रु॒णै॒तायेत्य॑रुण - ए॒ताय॑ । स्वाहा᳚ । ई॒दृशा॑य । स्वाहा᳚ । की॒दृशा॑य । स्वाहा᳚ । ता॒दृशा॑य । स्वाहा᳚ । स॒दृशा॑य । स्वाहा᳚ । विस॑दृशा॒येति॒ वि-स॒दृ॒शा॒य॒ । स्वाहा᳚ । सुस॑दृशा॒येति॒ सु - स॒दृ॒शा॒य॒ । स्वाहा᳚ । रू॒पाय॑ । स्वाहा᳚ ( ) । सर्व॑स्मै । स्वाहा᳚ ॥ \textbf{  39} \newline
                  \newline
                      (रू॒पाय॒ स्वाहा॒ - द्वे च॑ )  \textbf{(A17)} \newline \newline
                                \textbf{ TS 7.3.18.1} \newline
                  कृ॒ष्णाय॑ । स्वाहा᳚ । श्वे॒ताय॑ । स्वाहा᳚ । पि॒शङ्गा॑य । स्वाहा᳚ । सा॒रङ्गा॑य । स्वाहा᳚ । अ॒रु॒णाय॑ । स्वाहा᳚ । गौ॒राय॑ । स्वाहा᳚ । ब॒भ्रवे᳚ । स्वाहा᳚ । न॒कु॒लाय॑ । स्वाहा᳚ । रोहि॑ताय । स्वाहा᳚ । शोणा॑य । स्वाहा᳚ । श्या॒वाय॑ । स्वाहा᳚ । श्या॒माय॑ । स्वाहा᳚ । पा॒क॒लाय॑ । स्वाहा᳚ । सु॒रू॒पायेति॑ सु - रू॒पाय॑ । स्वाहा᳚ । अनु॑रूपा॒येत्यनु॑ - रू॒पा॒य॒ । स्वाहा᳚ । विरू॑पा॒येति॒ वि-रू॒पा॒य॒ । स्वाहा᳚ । सरू॑पा॒येति॒ स-रू॒पा॒य॒ । स्वाहा᳚ । प्रति॑रूपा॒येति॒ प्रति॑ - रू॒पा॒य॒ । स्वाहा᳚ । श॒बला॑य । स्वाहा᳚ । क॒म॒लाय॑ । स्वाहा᳚ । पृश्न॑ये । स्वाहा᳚ । पृ॒श्नि॒स॒क्थायेति॑ पृश्नि - स॒क्थाय॑ । स्वाहा᳚ । सर्व॑स्मै । स्वाहा᳚ ॥ \textbf{  40} \newline
                  \newline
                      (कृ॒ष्णाय॒ - षट्च॑त्वारिꣳशत्)  \textbf{(A18)} \newline \newline
                                \textbf{ TS 7.3.19.1} \newline
                  ओष॑धीभ्य॒ इत्योष॑धि-भ्यः॒ । स्वाहा᳚ । मूले᳚भ्यः । स्वाहा᳚ । तूले᳚भ्यः । स्वाहा᳚ । काण्डे᳚भ्यः । स्वाहा᳚ । वल्.शे᳚भ्यः । स्वाहा᳚ । पुष्पे᳚भ्यः । स्वाहा᳚ । फले᳚भ्यः । स्वाहा᳚ । गृ॒ही॒तेभ्यः॑ । स्वाहा᳚ । अगृ॑हीतेभ्यः । स्वाहा᳚ । अव॑पन्नेभ्य॒ इत्यव॑ - प॒न्ने॒भ्यः॒ । स्वाहा᳚ । शया॑नेभ्यः । स्वाहा᳚ । सर्व॑स्मै । स्वाहा᳚ ॥ \textbf{  41 } \newline
                  \newline
                      (ओष॑धीभ्य॒ - श्चतु॑र्विꣳशतिः)  \textbf{(A19)} \newline \newline
                                \textbf{ TS 7.3.20.1} \newline
                  वन॒स्पति॑भ्य॒ इति॒ वन॒स्पति॑ - भ्यः॒ । स्वाहा᳚ । मूले᳚भ्यः । स्वाहा᳚ । तूले᳚भ्यः । स्वाहा᳚ । स्कन्धो᳚भ्य॒ इति॒ स्कन्धः॑ - भ्यः॒ । स्वाहा᳚ । शाखा᳚भ्यः । स्वाहा᳚ । प॒र्णेभ्यः॑ । स्वाहा᳚ । पुष्पे᳚भ्यः । स्वाहा᳚ । फले᳚भ्यः । स्वाहा᳚ । गृ॒ही॒तेभ्यः॑ । स्वाहा᳚ । अगृ॑हीतेभ्यः । स्वाहा᳚ । अव॑पन्नेभ्य॒ इत्यव॑ - प॒न्ने॒भ्यः॒ । स्वाहा᳚ । शया॑नेभ्यः । स्वाहा᳚ । शि॒ष्टाय॑ । स्वाहा᳚ । अति॑शिष्टा॒येत्यति॑ - शि॒ष्टा॒य॒ । स्वाहा᳚ । परि॑शिष्टा॒येति॒ परि॑-शि॒ष्टा॒य॒ । स्वाहा᳚ । सꣳशि॑ष्टा॒येति॒ सं - शि॒ष्टा॒य॒ । स्वाहा᳚ । उच्छि॑ष्टा॒येत्युत् - शि॒ष्टा॒य॒ । स्वाहा᳚ । रि॒क्ताय॑ । स्वाहा᳚ । अरि॑क्ताय । स्वाहा᳚ । प्ररि॑क्ता॒येति॒ प्र - रि॒क्ता॒य॒ । स्वाहा᳚ । सꣳरि॑क्ता॒येति॒ सं - रि॒क्ता॒य॒ । स्वाहा᳚ । उद्रि॑क्ता॒येत्युत् - रि॒क्ता॒य॒ । स्वाहा᳚ । सर्व॑स्मै । स्वाहा᳚ ॥ \textbf{  42 } \newline
                  \newline
                      (वन॒स्पति॑भ्यः॒ - षट् च॑त्वारिꣳशत्)  \textbf{(A20)} \newline \newline
\textbf{praSna korvai with starting padams of 1 to 20 anuvAkams :-} \newline
(प्र॒जवं॑ - ब्रह्मवा॒दिनः॒ कि - मे॒ष वा आ॒प्त - आ॑दि॒त्या उ॒भयोः᳚ - प्र॒जाप॑ति॒रन्वा॑य॒ -न्निन्द्रो॒ वै स॒दृङ् - ङिन्द्रो॒ वै शि॑थि॒लः - प्र॒जाप॑तिरकामयता ऽन्ना॒दः - सा वि॒राड॒ - सावा॑दि॒त्यो᳚ - ऽर्वाङ् - भू॒त - मा मे॒ - ऽग्निना॒ -स्वाहा॒ऽऽधिन् - द॒द्भ्यो᳚ऽ - ञ्ज्ये॒ताय॑ - कृ॒ष्णा - यौष॑धीभ्यो॒ - वन॒स्पति॑भ्यो - विꣳश॒तिः) \newline

\textbf{korvai with starting padams of1, 11, 21 series of pa~jcAtis :-} \newline
(प्र॒जवं॑ - प्र॒जाप॑ति॒ - र्यद॑छन्दो॒मं - ते॑ हुवे सवा॒ऽह - मोष॑धीभ्यो॒ - द्विच॑त्वारिꣳशत्) \newline

\textbf{first and last padam of third praSnam of 7th kANDam} \newline
(प्र॒जवꣳ॒॒ - सर्व॑स्मै॒ स्वाहा᳚) \newline 


॥ हरिः॑ ॐ ॥
॥ कृष्ण यजुर्वेदीय तैत्तिरीय संहितायां सप्तमकाण्डे तृतीयः प्रश्नः समाप्तः ॥

================================= \newline
\pagebreak
\pagebreak
        


\end{document}
