\documentclass[17pt]{extarticle}
\usepackage{babel}
\usepackage{fontspec}
\usepackage{polyglossia}
\usepackage{extsizes}



\setmainlanguage{sanskrit}
\setotherlanguages{english} %% or other languages
\setlength{\parindent}{0pt}
\pagestyle{myheadings}
\newfontfamily\devanagarifont[Script=Devanagari]{AdishilaVedic}


\newcommand{\VAR}[1]{}
\newcommand{\BLOCK}[1]{}




\begin{document}
\begin{titlepage}
    \begin{center}
 
\begin{sanskrit}
    { \Large
    ॐ नमः परमात्मने, श्री महागणपतये नमः, श्री गुरुभ्यो नमः
ह॒रिः॒ ॐ 
    }
    \\
    \vspace{2.5cm}
    \mbox{ \Huge
    2.1      द्वितीयकाण्डे  प्रथमः प्रश्नः - पशुविधानं   }
\end{sanskrit}
\end{center}

\end{titlepage}
\tableofcontents

ॐ नमः परमात्मने, श्री महागणपतये नमः, श्री गुरुभ्यो नमः
ह॒रिः॒ ॐ \newline
2.1      द्वितीयकाण्डे  प्रथमः प्रश्नः - पशुविधानं \newline

\addcontentsline{toc}{section}{ 2.1      द्वितीयकाण्डे  प्रथमः प्रश्नः - पशुविधानं}
\markright{ 2.1      द्वितीयकाण्डे  प्रथमः प्रश्नः - पशुविधानं \hfill https://www.vedavms.in \hfill}
\section*{ 2.1      द्वितीयकाण्डे  प्रथमः प्रश्नः - पशुविधानं }
                                \textbf{ TS 2.1.1.1} \newline
                  वा॒य॒व्य᳚म् । श्वे॒तम् । एति॑ । ल॒भे॒त॒ । भूति॑काम॒ इति॒ भूति॑-का॒मः॒ । वा॒युः । वै । क्षेपि॑ष्ठा । दे॒वता᳚ । वा॒युम् । ए॒व । स्वेन॑ । भा॒ग॒धेये॒नेति॑ भाग - धेये॑न । उपेति॑ । धा॒व॒ति॒ । सः । ए॒व । ए॒न॒म् । भूति᳚म् । ग॒म॒य॒ति॒ । भव॑ति । ए॒व । अति॑क्षि॒प्रेत्यति॑ - क्षि॒प्रा॒ । दे॒वता᳚ । इति॑ । आ॒हुः॒ । सा । ए॒न॒म् । ई॒श्व॒रा । प्र॒दह॒ इति॑ प्र-दहः॑ । इति॑ । ए॒तम् । ए॒व । सन्त᳚म् । वा॒यवे᳚ । नि॒युत्व॑त॒ इति॑ नि-युत्व॑ते । एति॑ । ल॒भे॒त॒ । नि॒युदिति॑ नि - युत् । वै । अ॒स्य॒ । धृतिः॑ । धृ॒तः । ए॒व । भूति᳚म् । उपेति॑ । ए॒ति॒ । अप्र॑दाहा॒येत्यप्र॑ - दा॒हा॒य॒ । भव॑ति । ए॒व । \textbf{  1} \newline
                  \newline
                                \textbf{ TS 2.1.1.2} \newline
                  वा॒यवे᳚ । नि॒युत्व॑त॒ इति॑ नि - युत्व॑ते । एति॑ । ल॒भे॒त॒ । ग्राम॑काम॒ इति॒ ग्राम॑ - का॒मः॒ । वा॒युः । वै । इ॒माः । प्र॒जा इति॑ प्र - जाः । न॒स्यो॒ता इति॑ नसि - ओ॒ताः । ने॒नी॒य॒ते॒ । वा॒युम् । ए॒व । नि॒युत्व॑न्त॒मिति॑ नि-युत्व॑न्तम् । स्वेन॑ । भा॒ग॒धेये॒नेति॑ भाग - धेये॑न । उपेति॑ । धा॒व॒ति॒ । सः । ए॒व । अ॒स्मै॒ । प्र॒जा इति॑ प्र - जाः । न॒स्यो॒ता इति॑ नसि - ओ॒ताः । नीति॑ । य॒च्छ॒ति॒ । ग्रा॒मी । ए॒व । भ॒व॒ति॒ । नि॒युत्व॑त॒ इति॑ नि-युत्व॑ते । भ॒व॒ति॒ । ध्रु॒वाः । ए॒व । अ॒स्मै॒ । अन॑पगा॒ इत्यन॑प - गाः॒ । क॒रो॒ति॒ । वा॒यवे᳚ । नि॒युत्व॑त॒ इति॑ नि - युत्व॑ते । एति॑ । ल॒भे॒त॒ । प्र॒जाका॑म॒ इति॑ प्र॒जा - का॒मः॒ । प्रा॒ण इति॑ प्र- अ॒नः । वै । वा॒युः । अ॒पा॒न इत्य॑प - अ॒नः । नि॒युदिति॑ नि - युत् । प्रा॒णा॒पा॒नाविति॑ प्राण- अ॒पा॒नौ । खलु॑ । वै । ए॒तस्य॑ । प्र॒जाया॒ इति॑ प्र - जायाः᳚ । \textbf{  2} \newline
                  \newline
                                \textbf{ TS 2.1.1.3} \newline
                  अपेति॑ । क्रा॒म॒तः॒ । यः । अल᳚म् । प्र॒जाया॒ इति॑ प्र - जायै᳚ । सन्न् । प्र॒जामिति॑ प्र -जाम् । न । वि॒न्दते᳚ । वा॒युम् । ए॒व । नि॒युत्व॑न्त॒मिति॑ नि - युत्व॑न्तम् । स्वेन॑ । भा॒ग॒धेये॒नेति॑ भाग-धेये॑न । उपेति॑ । धा॒व॒ति॒ । सः । ए॒व । अ॒स्मै॒ । प्रा॒णा॒पा॒नाभ्या॒मिति॑ प्राण - अ॒पा॒नाभ्या᳚म् । प्र॒जामिति॑ प्र -जाम् । प्रेति॑ । ज॒न॒य॒ति॒ । वि॒न्दते᳚ । प्र॒जामिति॑ प्र -जाम् । वा॒यवे᳚ । नि॒युत्व॑त॒ इति॑ नि-युत्व॑ते । एति॑ । ल॒भे॒त॒ । ज्योगा॑मया॒वीति॒ ज्योक् - आ॒म॒या॒वी॒ । प्रा॒ण इति॑ प्र - अ॒नः । वै । वा॒युः । अ॒पा॒न इत्य॑प - अ॒नः । नि॒युदिति॑ नि - युत् । प्रा॒णा॒पा॒नाविति॑ प्राण-अ॒पा॒नौ । खलु॑ । वै । ए॒तस्मा᳚त् । अपेति॑ । क्रा॒म॒तः॒ । यस्य॑ । ज्योक् । आ॒मय॑ति । वा॒युम् । ए॒व । नि॒युत्व॑न्त॒मिति॑ नि - युत्व॑न्तम् । स्वेन॑ । भा॒ग॒धेये॒नेति॑ भाग-धेये॑न । उपेति॑ । \textbf{  3} \newline
                  \newline
                                \textbf{ TS 2.1.1.4} \newline
                  धा॒व॒ति॒ । सः । ए॒व । अ॒स्मि॒न्न् । प्रा॒णा॒पा॒नाविति॑ प्राण - अ॒पा॒नौ । द॒धा॒ति॒ । उ॒त । यदि॑ । इ॒तासु॒रिती॒त-अ॒सुः॒ । भव॑ति । जीव॑ति । ए॒व । प्र॒जाप॑ति॒रिति॑ प्र॒जा - प॒तिः॒ । वै । इ॒दम् । एकः॑ । आ॒सी॒त् । सः । अ॒का॒म॒य॒त॒ । प्र॒जा इति॑ प्र - जाः । प॒शून् । सृ॒जे॒य॒ । इति॑ । सः । आ॒त्मनः॑ । व॒पाम् । उदिति॑ । अ॒क्खि॒द॒त् । ताम् । अ॒ग्नौ । प्रेति॑ । अ॒गृ॒ह्णा॒त् । ततः॑ । अ॒जः । तू॒प॒रः । समिति॑ । अ॒भ॒व॒त् । तम् । स्वायै᳚ । दे॒वता॑यै । एति॑ । अ॒ल॒भ॒त॒ । ततः॑ । वै । सः । प्र॒जा इति॑ प्र - जाः । प॒शून् । अ॒सृ॒ज॒त॒ । यः । प्र॒जाका॑म॒ इति॑ प्र॒जा - का॒मः॒ । \textbf{  4} \newline
                  \newline
                                \textbf{ TS 2.1.1.5} \newline
                  प॒शुका॑म॒ इति॑ प॒शु - का॒मः॒ । स्यात् । सः । ए॒तम् । प्रा॒जा॒प॒त्यमिति॑ प्राजा - प॒त्यम् । अ॒जम् । तू॒प॒रम् । एति॑ । ल॒भे॒त॒ । प्र॒जाप॑ति॒मिति॑ प्र॒जा - प॒ति॒म् । ए॒व । स्वेन॑ । भा॒ग॒धेये॒नेति॑ भाग - धेये॑न । उपेति॑ । धा॒व॒ति॒ । सः । ए॒व । अ॒स्मै॒ । प्र॒जामिति॑ प्र-जाम् । प॒शून् । प्रेति॑ । ज॒न॒य॒ति॒ । यत् । श्म॒श्रु॒णः । तत् । पुरु॑षाणाम् । रू॒पम् । यत् । तू॒प॒रः । तत् । अश्वा॑नाम् । यत् । अ॒न्यतो॑द॒न्नित्य॒न्यतः॑ - द॒न्न् । तत् । गवा᳚म् । यत् । अव्याः᳚ । इ॒व॒ । श॒फाः । तत् । अवी॑नाम् । यत् । अ॒जः । तत् । अ॒जाना᳚म् । ए॒ताव॑न्तः । वै । ग्रा॒म्याः । प॒शवः॑ । तान् । \textbf{  5} \newline
                  \newline
                                \textbf{ TS 2.1.1.6} \newline
                  रू॒पेण॑ । ए॒व । अवेति॑ । रु॒न्धे॒ । सो॒मा॒पौ॒ष्णमिति॑ सोमा - पौ॒ष्णम् । त्रै॒तम् । एति॑ । ल॒भे॒त॒ । प॒शुका॑म॒ इति॑ प॒शु - का॒मः॒ । द्वौ । वै । अ॒जायै᳚ । स्तनौ᳚ । नाना᳚ । ए॒व । द्वौ । अ॒भीति॑ । जाये॑ते॒ इति॑ । ऊर्ज᳚म् । पुष्टि᳚म् । तृ॒तीयः॑ । सो॒मा॒पू॒षणा॒विति॑ सोमा - पू॒षणौ᳚ । ए॒व । स्वेन॑ । भा॒ग॒धेये॒नेति॑ भाग - धेये॑न । उपेति॑ । धा॒व॒ति॒ । तौ । ए॒व । अ॒स्मै॒ । प॒शून् । प्रेति॑ । ज॒न॒य॒तः॒ । सोमः॑ । वै । रे॒तो॒धा इति॑ रेतः - धाः । पू॒षा । प॒शू॒नाम् । प्र॒ज॒न॒यि॒तेति॑ प्र-ज॒न॒यि॒ता । सोमः॑ । ए॒व । अ॒स्मै॒ । रेतः॑ । दधा॑ति । पू॒षा । प॒शून् । प्रेति॑ । ज॒न॒य॒ति॒ । औदु॑म्बरः । यूपः॑ ( ) । भ॒व॒ति॒ । ऊर्क् । वै । उ॒दु॒म्बरः॑ । ऊर्क् । प॒शवः॑ । ऊ॒र्जा । ए॒व । अ॒स्मै॒ । ऊर्ज᳚म् । प॒शून् । अवेति॑ । रु॒न्धे॒ ॥ \textbf{  6 } \newline
                  \newline
                      (अप्र॑दाहाय॒ भव॑त्ये॒व - प्र॒जाया॑ - आ॒मय॑ति वा॒युमे॒व नि॒युत्व॑तꣳ॒॒ स्वेन॑ भाग॒धेये॒नोप॑ - प्र॒जाक॑म॒ - स्तान् - यूप॒ - स्त्रयो॑दश च )  \textbf{(A1)} \newline \newline
                                \textbf{ TS 2.1.2.1} \newline
                  प्र॒जाप॑ति॒रिति॑ प्र॒जा - प॒तिः॒ । प्र॒जा इति॑ प्र - जाः । अ॒सृ॒ज॒त॒ । ताः । अ॒स्मा॒त् । सृ॒ष्टाः । परा॑चीः । आ॒य॒न्न् । ताः । वरु॑णम् । अ॒ग॒च्छ॒न्न् । ताः । अन्विति॑ । ऐ॒त् । ताः । पुनः॑ । अ॒या॒च॒त॒ । ताः । अ॒स्मै॒ । न । पुनः॑ । अ॒द॒दा॒त् । सः । अ॒ब्र॒वी॒त् । वर᳚म् । वृ॒णी॒ष्व॒ । अथ॑ । मे॒ । पुनः॑ । दे॒हि॒ । इति॑ । तासा᳚म् । वर᳚म् । एति॑ । अ॒ल॒भ॒त॒ । सः । कृ॒ष्णः । एक॑शितिपा॒दित्येक॑ - शि॒ति॒पा॒त् । अ॒भ॒व॒त् । यः । वरु॑णगृहीत॒ इति॒ वरु॑ण - गृ॒ही॒तः॒ । स्यात् । सः । ए॒तम् । वा॒रु॒णम् । कृ॒ष्णम् । एक॑शितिपाद॒मित्येक॑ - शि॒ति॒पा॒द॒म् । एति॑ । ल॒भे॒त॒ । वरु॑णम् । \textbf{  7 } \newline
                  \newline
                                \textbf{ TS 2.1.2.2} \newline
                  ए॒व । स्वेन॑ । भा॒ग॒धेये॒नेति॑ भाग - धेये॑न । उपेति॑ । धा॒व॒ति॒ । सः । ए॒व । ए॒न॒म् । व॒रु॒ण॒पा॒शादिति॑ वरुण - पा॒शात् । मु॒ञ्च॒ति॒ । कृ॒ष्णः । एक॑शितिपा॒दित्येक॑ - शि॒ति॒पा॒त् । भ॒व॒ति॒ । वा॒रु॒णः । हि । ए॒षः । दे॒वत॑या । समृ॑द्ध्या॒ इति॒ सम् - ऋ॒द्धयै॒ । सुव॑र्भानु॒रिति॒ सुवः॑ - भा॒नुः॒ । आ॒सु॒रः । सूर्य᳚म् । तम॑सा । अ॒वि॒द्ध्य॒त् । तस्मै᳚ । दे॒वाः । प्राय॑श्चित्तिम् । ऐ॒च्छ॒न्न् । तस्य॑ । यत् । प्र॒थ॒मम् । तमः॑ । अ॒पाघ्न॒न्नित्य॑प - अघ्नन्न्॑ । सा । कृ॒ष्णा । अविः॑ । अ॒भ॒व॒त् । यत् । द्वि॒तीय᳚म् । सा । फल्गु॑नी । यत् । तृ॒तीय᳚म् । सा । ब॒ल॒क्षी । यत् । अ॒द्ध्य॒स्थादित्य॑धि - अ॒स्थात् । अ॒पाकृ॑न्त॒न्नित्य॑प - अकृ॑न्तन्न् । सा । अविः॑ । व॒शा । \textbf{  8} \newline
                  \newline
                                \textbf{ TS 2.1.2.3} \newline
                  समिति॑ । अ॒भ॒व॒त् । ते । दे॒वाः । अ॒ब्रु॒व॒न्न् । दे॒व॒प॒शुरिति॑ देव-प॒शुः । वै । अ॒यम् । समिति॑ । अ॒भू॒त् । कस्मै᳚ । इ॒मम् । एति॑ । ल॒फ्स्या॒म॒हे॒ । इति॑ । अथ॑ । वै । तर्.हि॑ । अल्पा᳚ । पृ॒थि॒वी । आसी᳚त् । अजा॑ताः । ओष॑धयः । ताम् । अवि᳚म् । व॒शाम् । आ॒दि॒त्येभ्यः॑ । कामा॑य । एति॑ । अ॒ल॒भ॒न्त॒ । ततः॑ । वै । अप्र॑थत । पृ॒थि॒वी । अजा॑यन्त । ओष॑धयः । यः । का॒मये॑त । प्रथे॑य । प॒शुभि॒रिति॑ प॒शु - भिः॒ । प्रेति॑ । प्र॒जयेति॑ प्र - जया᳚ । जा॒ये॒य॒ । इति॑ । सः । ए॒ताम् । अवि᳚म् । व॒शाम् । आ॒दि॒त्येभ्यः॑ । कामा॑य । \textbf{  9} \newline
                  \newline
                                \textbf{ TS 2.1.2.4} \newline
                  एति॑ । ल॒भे॒त॒ । आ॒दि॒त्यान् । ए॒व । काम᳚म् । स्वेन॑ । भा॒ग॒धेये॒नेति॑ भाग - धेये॑न । उपेति॑ । धा॒व॒ति॒ । ते । ए॒व । ए॒न॒म् । प्र॒थय॑न्ति । प॒शुभि॒रिति॑ प॒शु - भिः॒ । प्रेति॑ । प्र॒जयेति॑ प्र - जया᳚ । ज॒न॒य॒न्ति॒ । अ॒सौ । आ॒दि॒त्यः । न । वीति॑ । अ॒रो॒च॒त॒ । तस्मै᳚ । दे॒वाः । प्राय॑श्चित्तिम् । ऐ॒च्छ॒न्न् । तस्मै᳚ । ए॒ताः । म॒ल्॒.हाः । एति॑ । अ॒ल॒भ॒न्त॒ । आ॒ग्ने॒यीम् । कृ॒ष्ण॒ग्री॒वीमिति॑ कृष्ण - ग्री॒वीम् । सꣳ॒॒हि॒तामिति॑ सं - हि॒ताम् । ऐ॒न्द्रीम् । श्वे॒ताम् । बा॒र्.॒ह॒स्प॒त्याम् । ताभिः॑ । ए॒व । अ॒स्मि॒न्न् । रुच᳚म् । अ॒द॒धुः॒ । यः । ब्र॒ह्म॒व॒र्च॒सका॑म॒ इति॑ ब्रह्मवर्च॒स - का॒मः॒ । स्यात् । तस्मै᳚ । ए॒ताः । म॒ल्॒.हाः । एति॑ । ल॒भे॒त॒ । \textbf{  10} \newline
                  \newline
                                \textbf{ TS 2.1.2.5} \newline
                  आ॒ग्ने॒यीम् । कृ॒ष्ण॒ग्री॒वीमिति॑ कृष्ण - ग्री॒वीम् । सꣳ॒॒हि॒तामिति॑ सं - हि॒ताम् । ऐ॒न्द्रीम् । श्वे॒ताम् । बा॒र्.॒ह॒स्प॒त्याम् । ए॒ताः । ए॒व । दे॒वताः᳚ । स्वेन॑ । भा॒ग॒धेये॒नेति॑ भाग-धेये॑न । उपेति॑ । धा॒व॒ति॒ । ताः । ए॒व । अ॒स्मि॒न्न् । ब्र॒ह्म॒व॒र्च॒समिति॑ ब्रह्म - व॒र्च॒सम् । द॒ध॒ति॒ । ब्र॒ह्म॒व॒र्च॒सीति॑ ब्रह्म - व॒र्च॒सी । ए॒व । भ॒व॒ति॒ । व॒सन्ता᳚ । प्रा॒तः । आ॒ग्ने॒यीम् । कृ॒ष्ण॒ग्री॒वीमिति॑ कृष्ण - ग्री॒वीम् । एति॑ । ल॒भे॒त॒ । ग्री॒ष्मे । म॒द्ध्यन्दि॑ने । सꣳ॒॒हि॒तामिति॑ सं - हि॒ताम् । ऐ॒न्द्रीम् । श॒रदि॑ । अ॒प॒रा॒ह्ण इत्य॑पर - अ॒ह्ने । श्वे॒ताम् । बा॒र्.॒ह॒स्प॒त्याम् । त्रीणि॑ । वै । आ॒दि॒त्यस्य॑ । तेजाꣳ॑सि । व॒सन्ता᳚ । प्रा॒तः । ग्री॒ष्मे । म॒द्ध्यन्दि॑ने । श॒रदि॑ । अ॒प॒रा॒ह्ण इत्य॑पर - अ॒ह्ने । याव॑न्ति । ए॒व । तेजाꣳ॑सि । तानि॑ । ए॒व । \textbf{  11} \newline
                  \newline
                                \textbf{ TS 2.1.2.6} \newline
                  अवेति॑ । रु॒न्धे॒ । सं॒ॅव॒थ्स॒रमिति॑ सं - व॒थ्स॒रम् । प॒र्याल॑भ्यन्त॒ इति॑ परि - आल॑भ्यन्ते । सं॒ॅव॒थ्स॒र इति॑ सं-व॒थ्स॒रः । वै । ब्र॒ह्म॒व॒र्च॒सस्येति॑ ब्रह्म - व॒र्च॒सस्य॑ । प्र॒दा॒तेति॑ प्र - दा॒ता । सं॒ॅव॒थ्स॒र इति॑ सं - व॒थ्स॒रः । ए॒व । अ॒स्मै॒ । ब्र॒ह्म॒व॒र्च॒समिति॑ ब्रह्म - व॒र्च॒सम् । प्रेति॑ । य॒च्छ॒ति॒ । ब्र॒ह्म॒व॒र्च॒सीति॑ ब्रह्म - व॒र्च॒सी । ए॒व । भ॒व॒ति॒ । ग॒र्भिण॑यः । भ॒व॒न्ति॒ । इ॒न्द्रि॒यम् । वै । गर्भः॑ । इ॒न्द्रि॒यम् । ए॒व । अ॒स्मि॒न्न् । द॒ध॒ति॒ । सा॒र॒स्व॒तीम् । मे॒षीम् । एति॑ । ल॒भे॒त॒ । यः । ई॒श्व॒रः । वा॒चः । वदि॑तोः । सन्न् । वाच᳚म् । न । वदे᳚त् । वाक् । वै । सर॑स्वती । सर॑स्वतीम् । ए॒व । स्वेन॑ । भा॒ग॒धेये॒नेति॑ भाग - धेये॑न । उपेति॑ । धा॒व॒ति॒ । सा । ए॒व । अ॒स्मि॒न्न् । \textbf{  12} \newline
                  \newline
                                \textbf{ TS 2.1.2.7} \newline
                  वाच᳚म् । द॒धा॒ति॒ । प्र॒व॒दि॒तेति॑ प्र - व॒दि॒ता । वा॒चः । भ॒व॒ति॒ । अप॑न्नद॒तीत्यप॑न्न - द॒ती॒ । भ॒व॒ति॒ । तस्मा᳚त् । म॒नु॒ष्याः᳚ । सर्वा᳚म् । वाच᳚म् । व॒द॒न्ति॒ । आ॒ग्ने॒यम् । कृ॒ष्णग्री॑व॒मिति॑ कृ॒ष्ण-ग्री॒व॒म् । एति॑ । ल॒भे॒त॒ । सौ॒म्यम् । ब॒भ्रुम् । ज्योगा॑मया॒वीति॒ ज्योक् - आ॒म॒या॒वी॒ । अ॒ग्निम् । वै । ए॒तस्य॑ । शरी॑रम् । ग॒च्छ॒ति॒ । सोम᳚म् । रसः॑ । यस्य॑ । ज्योक् । आ॒मय॑ति । अ॒ग्नेः । ए॒व । अ॒स्य॒ । शरी॑रम् । नि॒ष्क्री॒णातीति॑ निः - क्री॒णाति॑ । सोमा᳚त् । रस᳚म् । उ॒त । यदि॑ । इ॒तासु॒रिती॒त - अ॒सुः॒ । भव॑ति । जीव॑ति । ए॒व । सौ॒म्यम् । ब॒भ्रुम् । एति॑ । ल॒भे॒त॒ । आ॒ग्ने॒यम् । कृ॒ष्णग्री॑व॒मिति॑ कृ॒ष्ण - ग्री॒व॒म् । प्र॒जाका॑म॒ इति॑ प्र॒जा - का॒मः॒ । सोमः॑ । \textbf{  13} \newline
                  \newline
                                \textbf{ TS 2.1.2.8} \newline
                  वै । रे॒तो॒धा इति॑ रेतः - धाः । अ॒ग्निः । प्र॒जाना॒मिति॑ प्र - जाना᳚म् । प्र॒ज॒न॒यि॒तेति॑ प्र-ज॒न॒यि॒ता । सोमः॑ । ए॒व । अ॒स्मै॒ । रेतः॑ । दधा॑ति । अ॒ग्निः । प्र॒जामिति॑ प्र-जाम् । प्रेति॑ । ज॒न॒य॒ति॒ । वि॒न्दते᳚ । प्र॒जामिति॑ प्र-जाम् । आ॒ग्ने॒यम् । कृ॒ष्णग्री॑व॒मिति॑ कृ॒ष्ण - ग्री॒व॒म् । एति॑ । ल॒भे॒त॒ । सौ॒म्यम् । ब॒भ्रुम् । यः । ब्रा॒ह्म॒णः । वि॒द्याम् । अ॒नूच्येत्य॑नु - उच्य॑ । न । वि॒रोचे॒तेति॑ वि - रोचे॑त । यत् । आ॒ग्ने॒यः । भव॑ति । तेजः॑ । ए॒व । अ॒स्मि॒न्न् । तेन॑ । द॒धा॒ति॒ । यत् । सौ॒म्यः । ब्र॒ह्म॒व॒र्च॒समिति॑ ब्रह्म - व॒र्च॒सम् । तेन॑ । कृ॒ष्णग्री॑व॒ इति॑ कृ॒ष्ण - ग्री॒वः॒ । आ॒ग्ने॒यः । भ॒व॒ति॒ । तमः॑ । ए॒व । अ॒स्मा॒त् । अपेति॑ । ह॒न्ति॒ । श्वे॒तः । भ॒व॒ति॒ । \textbf{  14} \newline
                  \newline
                                \textbf{ TS 2.1.2.9} \newline
                  रुच᳚म् । ए॒व । अ॒स्मि॒न्न् । द॒धा॒ति॒ । ब॒भ्रुः । सौ॒म्यः । भ॒व॒ति॒ । ब्र॒ह्म॒व॒र्च॒समिति॑ ब्रह्म - व॒र्च॒सम् । ए॒व । अ॒स्मि॒न्न् । त्विषि᳚म् । द॒धा॒ति॒ । आ॒ग्ने॒यम् । कृ॒ष्णग्री॑व॒मिति॑ कृ॒ष्ण - ग्री॒व॒म् । एति॑ । ल॒भे॒त॒ । सौ॒म्यम् । ब॒भ्रुम् । आ॒ग्ने॒यम् । कृ॒ष्णग्री॑व॒मिति॑ कृ॒ष्ण - ग्री॒व॒म् । पु॒रो॒धाया॒मिति॑ पुरः - धाया᳚म् । स्पर्द्ध॑मानः । आ॒ग्ने॒यः । वै । ब्रा॒ह्म॒णः । सौ॒म्यः । रा॒ज॒न्यः॑ । अ॒भितः॑ । सौ॒म्यम् । आ॒ग्ने॒यौ । भ॒व॒तः॒ । तेज॑सा । ए॒व । ब्रह्म॑णा । उ॒भ॒यतः॑ । रा॒ष्ट्रम् । परीति॑ । गृ॒ह्णा॒ति॒ । ए॒क॒धेत्ये॑क - धा । स॒मावृ॑ङ्क्त॒ इति॑ सं - आवृ॑ङ्क्ते । पु॒रः । ए॒न॒म् । द॒ध॒ते॒ ॥ \textbf{  15} \newline
                  \newline
                      (ल॒भे॒त॒ वरु॑णं - ॅव॒शै - तामविं॑ ॅव॒शामा॑दि॒त्येभ्यः॒ कामा॑य - म॒ल्हा आ ल॑भेत॒ - तान्ये॒व - सैवास्मि॒न्थ् - सोमः॑ - स्वे॒तो भ॑वति॒ - त्रिच॑त्वारिꣳशच्च )  \textbf{(A2)} \newline \newline
                                \textbf{ TS 2.1.3.1} \newline
                  दे॒वा॒सु॒रा इति॑ देव - अ॒सु॒राः । ए॒षु । लो॒केषु॑ । अ॒स्प॒र्द्ध॒न्त॒ । सः । ए॒तम् । विष्णुः॑ । वा॒म॒नम् । अ॒प॒श्य॒त् । तम् । स्वायै᳚ । दे॒वता॑यै । एति॑ । अ॒ल॒भ॒त॒ । ततः॑ । वै । सः । इ॒मान् । लो॒कान् । अ॒भीति॑ । अ॒ज॒य॒त् । वै॒ष्ण॒वम् । वा॒म॒नम् । एति॑ । ल॒भे॒त॒ । स्पर्द्ध॑मानः । विष्णुः॑ । ए॒व । भू॒त्वा । इ॒मान् । लो॒कान् । अ॒भीति॑ । ज॒य॒ति॒ । विष॑म॒ इति॒ वि - स॒मे॒ । एति॑ । ल॒भे॒त॒ । विष॑मा॒ इति॒ वि - स॒माः॒ । इ॒व॒ । हि । इ॒मे । लो॒काः । समृ॑द्ध्या॒ इति॒ सं - ऋ॒द्ध्यै॒ । इन्द्रा॑य । म॒न्यु॒मत॒ इति॑ मन्यु - मते᳚ । मन॑स्वते । ल॒लाम᳚म् । प्रा॒शृ॒ङ्गम् । एति॑ । ल॒भे॒त॒ । स॒ङ्ग्रा॒म इति॑ सं - ग्रा॒मे । \textbf{  16} \newline
                  \newline
                                \textbf{ TS 2.1.3.2} \newline
                  संॅय॑त्त॒ इति॒ सं - य॒त्ते॒ । इ॒न्द्रि॒येण॑ । वै । म॒न्युना᳚ । मन॑सा । स॒ङ्ग्रा॒ममिति॑ सं - ग्रा॒मम् । ज॒य॒ति॒ । इन्द्र᳚म् । ए॒व । म॒न्यु॒मन्त॒मिति॑ मन्यु - मन्त᳚म् । मन॑स्वन्तम् । स्वेन॑ । भा॒ग॒धेये॒नेति॑ भाग - धेये॑न । उपेति॑ । धा॒व॒ति॒ । सः । ए॒व । अ॒स्मि॒न्न् । इ॒न्द्रि॒यम् । म॒न्युम् । मनः॑ । द॒धा॒ति॒ । जय॑ति । तम् । स॒ङ्ग्रा॒ममिति॑ सं - ग्रा॒मम् । इन्द्रा॑य । म॒रुत्व॑ते । पृ॒श्नि॒स॒क्थमिति॑ पृश्नि - स॒क्थम् । एति॑ । ल॒भे॒त॒ । ग्राम॑काम॒ इति॒ ग्राम॑ - का॒मः॒ । इन्द्र᳚म् । ए॒व । म॒रुत्व॑न्तम् । स्वेन॑ । भा॒ग॒धेये॒नेति॑ भाग - धेये॑न । उपेति॑ । धा॒व॒ति॒ । सः । ए॒व । अ॒स्मै॒ । स॒जा॒तानिति॑ स - जा॒तान् । प्रेति॑ । य॒च्छ॒ति॒ । ग्रा॒मी । ए॒व । भ॒व॒ति॒ । यत् । ऋ॒ष॒भः । तेन॑ । \textbf{  17} \newline
                  \newline
                                \textbf{ TS 2.1.3.3} \newline
                  ऐ॒न्द्रः । यत् । पृश्निः॑ । तेन॑ । मा॒रु॒तः । समृ॑द्ध्या॒ इति॒ सं-ऋ॒द्ध्यै॒ । प॒श्चात् । पृ॒श्नि॒स॒क्थ इति॑ पृश्नि - स॒क्थः । भ॒व॒ति॒ । प॒श्चा॒द॒न्व॒व॒सा॒यिनी॒मिति॑ पश्चात् - अ॒न्व॒व॒सा॒यिनी᳚म् । ए॒व । अ॒स्मै॒ । विश᳚म् । क॒रो॒ति॒ । सौ॒म्यम् । ब॒भ्रुम् । एति॑ । ल॒भे॒त॒ । अन्न॑काम॒ इत्यन्न॑ - का॒मः॒ । सौ॒म्यम् । वै । अन्न᳚म् । सोम᳚म् । ए॒व । स्वेन॑ । भा॒ग॒धेये॒नेति॑ भाग - धेये॑न । उपेति॑ । धा॒व॒ति॒ । सः । ए॒व । अ॒स्मै॒ । अन्न᳚म् । प्रेति॑ । य॒च्छ॒ति॒ । अ॒न्ना॒द इत्य॑न्न - अ॒दः । ए॒व । भ॒व॒ति॒ । ब॒भ्रुः । भ॒व॒ति॒ । ए॒तत् । वै । अन्न॑स्य । रू॒पम् । समृ॑द्ध्या॒ इति॒ सं - ऋ॒द्ध्यै॒ । सौ॒म्यम् । ब॒भ्रुम् । एति॑ । ल॒भे॒त॒ । यम् । अल᳚म् । \textbf{  18} \newline
                  \newline
                                \textbf{ TS 2.1.3.4} \newline
                  रा॒ज्याय॑ । सन्त᳚म् । रा॒ज्यम् । न । उ॒प॒नमे॒दित्यु॑प-नमे᳚त् । सौ॒म्यम् । वै । रा॒ज्यम् । सोम᳚म् । ए॒व । स्वेन॑ । भा॒ग॒धेये॒नेति॑ भाग - धेये॑न । उपेति॑ । धा॒व॒ति॒ । सः । ए॒व । अ॒स्मै॒ । रा॒ज्यम् । प्रेति॑ । य॒च्छ॒ति॒ । उपेति॑ । ए॒न॒म् । रा॒ज्यम् । न॒म॒ति॒ । ब॒भ्रुः । भ॒व॒ति॒ । ए॒तत् । वै । सोम॑स्य । रू॒पम् । समृ॑द्ध्या॒ इति॒ सं - ऋ॒द्ध्यै॒ । इन्द्रा॑य । वृ॒त्र॒तुर॒ इति॑ वृत्र - तुरे᳚ । ल॒लाम᳚म् । प्रा॒शृ॒ङ्गम् । एति॑ । ल॒भे॒त॒ । ग॒तश्री॒रिति॑ ग॒त - श्रीः॒ । प्र॒ति॒ष्ठाका॑म॒ इति॑ प्रति॒ष्ठा - का॒मः॒ । पा॒प्मान᳚म् । ए॒व । वृ॒त्रम् । ती॒र्त्वा । प्र॒ति॒ष्ठामिति॑ प्रति - स्थाम् । ग॒च्छ॒ति॒ । इन्द्रा॑य । अ॒भि॒मा॒ति॒घ्न इत्य॑भिमाति - घ्ने । ल॒लाम᳚म् । प्रा॒शृ॒ङ्गम् । एति॑ । \textbf{  19} \newline
                  \newline
                                \textbf{ TS 2.1.3.5} \newline
                  ल॒भे॒त॒ । यः । पा॒प्मना᳚ । गृ॒ही॒तः । स्यात् । पा॒प्मा । वै । अ॒भिमा॑ति॒रित्य॒भि - मा॒तिः॒ । इन्द्र᳚म् । ए॒व । अ॒भि॒मा॒ति॒हन॒मित्य॑भिमाति - हन᳚म् । स्वेन॑ । भा॒ग॒धेये॒नेति॑ भाग - धेये॑न । उपेति॑ । धा॒व॒ति॒ । सः । ए॒व । अ॒स्मा॒त् । पा॒प्मान᳚म् । अ॒भिमा॑ति॒मित्य॒भि - मा॒ति॒म् । प्रेति॑ । नु॒द॒ते॒ । इन्द्रा॑य । व॒ज्रिणे᳚ । ल॒लाम᳚म् । प्रा॒शृ॒ङ्गम् । एति॑ । ल॒भे॒त॒ । यम् । अल᳚म् । रा॒ज्याय॑ । सन्त᳚म् । रा॒ज्यम् । न । उ॒प॒नमे॒दित्यु॑प - नमे᳚त् । इन्द्र᳚म् । ए॒व । व॒ज्रिण᳚म् । स्वेन॑ । भा॒ग॒धेये॒नेति॑ भाग - धेये॑न । उपेति॑ । धा॒व॒ति॒ । सः । ए॒व । अ॒स्मै॒ । वज्र᳚म् । प्रेति॑ । य॒च्छ॒ति॒ । सः । ए॒न॒म् ( ) । वज्रः॑ । भूत्यै᳚ । इ॒न्धे॒ । उपेति॑ । ए॒न॒म् । रा॒ज्यम् । न॒म॒ति॒ । ल॒लामः॑ । प्रा॒शृ॒ङ्गः । भ॒व॒ति॒ । ए॒तत् । वै । वज्र॑स्य । रू॒पम् । समृ॑द्ध्या॒ इति॒ सं - ऋ॒द्ध्यै॒ ॥ \textbf{  20 } \newline
                  \newline
                      (स॒ग्रां॒मे - तेना - ल॑ - मभिमाति॒घ्ने ल॒लामं॑ प्राशृ॒ङ्गमै - नं॒ - पञ्च॑दश च )  \textbf{(A3)} \newline \newline
                                \textbf{ TS 2.1.4.1} \newline
                  अ॒सौ । आ॒दि॒त्यः । न । वीति॑ । अ॒रो॒च॒त॒ । तस्मै᳚ । दे॒वाः । प्राय॑श्चित्तिम् । ऐ॒च्छ॒न्न् । तस्मै᳚ । ए॒ताम् । दश॑र्.षभा॒मिति॒ दश॑ - ऋ॒ष॒भा॒म् । एति॑ । अ॒ल॒भ॒न्त॒ । तया᳚ । ए॒व । अ॒स्मि॒न्न् । रुच᳚म् । अ॒द॒धुः॒ । यः । ब्र॒ह्म॒व॒र्च॒सका॑म॒ इति॑ ब्रह्मवर्च॒स - का॒मः॒ । स्यात् । तस्मै᳚ । ए॒ताम् । दश॑र्.षभा॒मिति॒ दश॑ - ऋ॒ष॒भा॒म् । एति॑ । ल॒भे॒त॒ । अ॒मुम् । ए॒व । आ॒दि॒त्यम् । स्वेन॑ । भा॒ग॒धेये॒नेति॑ भाग - धेये॑न । उपेति॑ । धा॒व॒ति॒ । सः । ए॒व । अ॒स्मि॒न्न् । ब्र॒ह्म॒व॒र्च॒समिति॑ ब्रह्म - व॒र्च॒सम् । द॒धा॒ति॒ । ब्र॒ह्म॒व॒र्च॒सीति॑ ब्रह्म - व॒र्च॒सी । ए॒व । भ॒व॒ति॒ । व॒सन्ता᳚ । प्रा॒तः । त्रीन् । ल॒लामान्॑ । एति॑ । ल॒भे॒त॒ । ग्री॒ष्मे । म॒द्ध्यन्दि॑ने । \textbf{  21} \newline
                  \newline
                                \textbf{ TS 2.1.4.2} \newline
                  त्रीन् । शि॒ति॒पृ॒ष्ठानिति॑ शिति - पृ॒ष्ठान् । श॒रदि॑ । अ॒प॒रा॒ह्ण इत्य॑पर-अ॒ह्ने । त्रीन् । शि॒ति॒वारा॒निति॑ शिति - वारान्॑ । त्रीणि॑ । वै । आ॒दि॒त्यस्य॑ । तेजाꣳ॑सि । व॒सन्ता᳚ । प्रा॒तः । ग्री॒ष्मे । म॒द्ध्यन्दि॑ने । श॒रदि॑ । अ॒प॒रा॒ह्ण इत्य॑पर-अ॒ह्ने । याव॑न्ति । ए॒व । तेजाꣳ॑सि । तानि॑ । ए॒व । अवेति॑ । रु॒न्धे॒ । त्रय॑स्त्रय॒ इति॒ त्रयः॑-त्र॒यः॒ । एति॑ । ल॒भ्य॒न्ते॒ । अ॒भि॒पू॒र्वमित्य॑भि - पू॒र्वम् । ए॒व । अ॒स्मि॒न्न् । तेजः॑ । द॒धा॒ति॒ । सं॒ॅव॒थ्स॒रमिति॑ सं - व॒थ्स॒रम् । प॒र्याल॑भ्यन्त॒ इति॑ परि-आल॑भ्यन्ते । सं॒ॅव॒थ्स॒र इति॑ सं - व॒थ्स॒रः । वै । ब्र॒ह्म॒व॒र्च॒सस्येति॑ ब्रह्म - व॒र्च॒सस्य॑ । प्र॒दा॒तेति॑ प्र - दा॒ता । सं॒ॅव॒थ्स॒र इति॑ सं - व॒थ्स॒रः । ए॒व । अ॒स्मै॒ । ब्र॒ह्म॒व॒र्च॒समिति॑ ब्रह्म - व॒र्च॒सम् । प्रेति॑ । य॒च्छ॒ति॒ । ब्र॒ह्म॒व॒र्च॒सीति॑ ब्रह्म - व॒र्च॒सी । ए॒व । भ॒व॒ति॒ । सं॒ॅव॒थ्स॒रस्येति॑ सं - व॒थ्स॒रस्य॑ । प॒रस्ता᳚त् । प्रा॒जा॒प॒त्यमिति॑ प्राजा - प॒त्यम् । कद्रु᳚म् । \textbf{  22} \newline
                  \newline
                                \textbf{ TS 2.1.4.3} \newline
                  एति॑ । ल॒भे॒त॒ । प्र॒जाप॑ति॒रिति॑ प्र॒जा - प॒तिः॒ । सर्वाः᳚ । दे॒वताः᳚ । दे॒वता॑सु । ए॒व । प्रतीति॑ । ति॒ष्ठ॒ति॒ । यदि॑ । बि॒भी॒यात् । दु॒श्चर्मेति॑ दुः-चर्मा᳚ । भ॒वि॒ष्या॒मि॒ । इति॑ । सो॒मा॒पौ॒ष्णमिति॑ सोमा - पौ॒ष्णम् । श्या॒मम् । एति॑ । ल॒भे॒त॒ । सौ॒म्यः । वै । दे॒वत॑या । पुरु॑षः । पौ॒ष्णाः । प॒शवः॑ । स्वया᳚ । ए॒व । अ॒स्मै॒ । दे॒वत॑या । प॒शुभि॒रिति॑ प॒शु - भिः॒ । त्वच᳚म् । क॒रो॒ति॒ । न । दु॒श्चर्मेति॑ दुः - चर्मा᳚ । भ॒व॒ति॒ । दे॒वाः । च॒ । वै । य॒मः । च॒ । अ॒स्मिन्न् । लो॒के । अ॒स्प॒र्ध॒न्त॒ । सः । य॒मः । दे॒वाना᳚म् । इ॒न्द्रि॒यम् । वी॒र्य᳚म् । अ॒यु॒व॒त॒ । तत् । य॒मस्य॑ । \textbf{  23} \newline
                  \newline
                                \textbf{ TS 2.1.4.4} \newline
                  य॒म॒त्वमिति॑ यम - त्वम् । ते । दे॒वाः । अ॒म॒न्य॒न्त॒ । य॒मः । वै । इ॒दम् । अ॒भू॒त् । यत् । व॒यम् । स्मः । इति॑ । ते । प्र॒जाप॑ति॒मिति॑ प्र॒जा - प॒ति॒म् । उपेति॑ । अ॒धा॒वन्न् । सः । ए॒तौ । प्र॒जाप॑ति॒रिति॑ प्र॒जा - प॒तिः॒ । आ॒त्मनः॑ । उ॒क्ष॒व॒शावित्यु॑क्ष - व॒शौ । निरिति॑ । अ॒मि॒मी॒त॒ । ते । दे॒वाः । वै॒ष्णा॒व॒रु॒णीमिति॑ वैष्णा-व॒रु॒णीम् । व॒शाम् । एति॑ । अ॒ल॒भ॒न्त॒ । ऐ॒न्द्रम् । उ॒क्षाण᳚म् । तम् । वरु॑णेन । ए॒व । ग्रा॒ह॒यि॒त्वा । विष्णु॑ना । य॒ज्ञेन॑ । प्रेति॑ । अ॒नु॒द॒न्त॒ । ऐ॒न्द्रेण॑ । ए॒व । अ॒स्य॒ । इ॒न्द्रि॒यम् । अ॒वृ॒ञ्ज॒त॒ । यः । भ्रातृ॑व्यवा॒निति॒ भ्रातृ॑व्य - वा॒न् । स्यात् । सः । स्पर्द्ध॑मानः । वै॒ष्णा॒व॒रु॒णीमिति॑ वैष्णा - व॒रु॒णीम् । \textbf{  24} \newline
                  \newline
                                \textbf{ TS 2.1.4.5} \newline
                  व॒शाम् । एति॑ । ल॒भे॒त॒ । ऐ॒न्द्रम् । उ॒क्षाण᳚म् । वरु॑णेन । ए॒व । भ्रातृ॑व्यम् । ग्रा॒ह॒यि॒त्वा । विष्णु॑ना । य॒ज्ञेन॑ । प्रेति॑ । नु॒द॒ते॒ । ऐ॒न्द्रेण॑ । ए॒व । अ॒स्य॒ । इ॒न्द्रि॒यम् । वृ॒ङ्क्ते॒ । भव॑ति । आ॒त्मना᳚ । परेति॑ । अ॒स्य॒ । भ्रातृ॑व्यः । भ॒व॒ति॒ । इन्द्रः॑ । वृ॒त्रम् । अ॒ह॒न्न् । तम् । वृ॒त्रः । ह॒तः । षो॒ड॒शभि॒रिति॑ षोड॒श - भिः॒ । भो॒गैः । अ॒सि॒ना॒त् । तस्य॑ । वृ॒त्रस्य॑ । शी॒र्.॒ष॒तः । गावः॑ । उदिति॑ । आ॒य॒न्न् । ताः । वै॒दे॒ह्यः॑ । अ॒भ॒व॒न्न् । तासा᳚म् । ऋ॒ष॒भः । ज॒घने᳚ । अनु॑ । उदिति॑ । ऐ॒त् । तम् । इन्द्रः॑ । \textbf{  25} \newline
                  \newline
                                \textbf{ TS 2.1.4.6} \newline
                  अ॒चा॒य॒त् । सः । अ॒म॒न्य॒त॒ । यः । वै । इ॒मम् । आ॒लभे॒तेत्या᳚ - लभे॑त । मुच्ये॑त । अ॒स्मात् । पा॒प्मनः॑ । इति॑ । सः । आ॒ग्ने॒यम् । कृ॒ष्णग्री॑व॒मिति॑ कृ॒ष्ण - ग्री॒व॒म् । एति॑ । अ॒ल॒भ॒त॒ । ऐ॒न्द्रम् । ऋ॒ष॒भम् । तस्य॑ । अ॒ग्निः । ए॒व । स्वेन॑ । भा॒ग॒धेये॒नेति॑ भाग - धेये॑न । उप॑सृत॒ इत्युप॑ - सृ॒तः॒ । षो॒ड॒श॒धेति॑ षोडश-धा । वृ॒त्रस्य॑ । भो॒गान् । अपीति॑ । अ॒द॒ह॒त् । ऐ॒न्द्रेण॑ । इ॒न्द्रि॒यम् । आ॒त्मन्न् । अ॒ध॒त्त॒ । यः । पा॒प्मना᳚ । गृ॒ही॒तः । स्यात् । सः । आ॒ग्ने॒यम् । कृ॒ष्णग्री॑व॒मिति॑ कृ॒ष्ण - ग्री॒व॒म् । एति॑ । ल॒भे॒त॒ । ऐ॒न्द्रम् । ऋ॒ष॒भम् । अ॒ग्निः । ए॒व । अ॒स्य॒ । स्वेन॑ । भा॒ग॒धेये॒नेति॑ भाग - धेये॑न । उप॑सृत॒ इत्युप॑ - सृ॒तः॒ । \textbf{  26} \newline
                  \newline
                                \textbf{ TS 2.1.4.7} \newline
                  पा॒प्मान᳚म् । अपीति॑ । द॒ह॒ति॒ । ऐ॒न्द्रेण॑ । इ॒न्द्रि॒यम् । आ॒त्मन्न् । ध॒त्ते॒ । मुच्य॑ते । पा॒प्मनः॑ । भव॑ति । ए॒व । द्या॒वा॒पृ॒थि॒व्या॑मिति॑ द्यावा - पृ॒थि॒व्या᳚म् । धे॒नुम् । एति॑ । ल॒भे॒त॒ । ज्योग॑परुद्ध॒ इति॒ ज्योक् - अ॒प॒रु॒द्धः॒ । अ॒नयोः᳚ । हि । वै । ए॒षः । अप्र॑तिष्ठित॒ इत्यप्र॑ति - स्थि॒तः॒ । अथ॑ । ए॒षः । ज्योक् । अप॑रुद्ध॒ इत्यप॑ - रु॒द्धः॒ । द्यावा॑पृथि॒वी इति॒ द्यावा᳚ - पृ॒थि॒वी । ए॒व । स्वेन॑ । भा॒ग॒धेये॒नेति॑ भाग - धेये॑न । उपेति॑ । धा॒व॒ति॒ । ते इति॑ । ए॒व । ए॒न॒म् । प्र॒ति॒ष्ठामिति॑ प्रति - स्थाम् । ग॒म॒य॒तः॒ । प्रतीति॑ । ए॒व । ति॒ष्ठ॒ति॒ । प॒र्या॒रिणी᳚ । भ॒व॒ति॒ । प॒र्या॒रि । इ॒व॒ । हि । ए॒तस्य॑ । रा॒ष्ट्रम् । यः । ज्योग॑परुद्ध॒ इति॒ ज्योक् - अ॒प॒रु॒द्धः॒ । समृ॑द्ध्या॒ इति॒ सं - ऋ॒द्ध्यै॒ । वा॒य॒व्य᳚म् । \textbf{  27} \newline
                  \newline
                                \textbf{ TS 2.1.4.8} \newline
                  व॒थ्सम् । एति॑ । ल॒भे॒त॒ । वा॒युः । वै । अ॒नयोः᳚ । व॒थ्सः । इ॒मे । वै । ए॒तस्मै᳚ । लो॒काः । अप॑शुष्का॒ इत्यप॑ - शु॒ष्काः॒ । विट् । अप॑शु॒ष्केत्यप॑ - शु॒ष्का॒ । अथ॑ । ए॒षः । ज्योक् । अप॑रुद्ध॒ इत्यप॑ - रु॒द्धः॒ । वा॒युम् । ए॒व । स्वेन॑ । भा॒ग॒धेये॒नेति॑ भाग - धेये॑न । उपेति॑ । धा॒व॒ति॒ । सः । ए॒व । अ॒स्मै॒ । इ॒मान् । लो॒कान् । विश᳚म् । प्रेति॑ । दा॒प॒य॒ति॒ । प्रेति॑ । अ॒स्मै॒ । इ॒मे । लो॒काः । स्नु॒व॒न्ति॒ । भु॒ञ्ज॒ती । ए॒न॒म् । विट् । उपेति॑ । ति॒ष्ठ॒ते॒ ॥ \textbf{  28} \newline
                  \newline
                      (म॒ध्यन्दि॑ने॒ - कद्रुं॑ - ॅय॒मस्य॒ - स्पर्द्ध॑मानो वैष्णावरु॒णीं -तमिन्द्रो᳚ - ऽस्य॒ स्वेन॑ भाग॒धेये॒नोप॑सृतो - वाय॒व्यं॑ - द्विच॑त्वारिꣳशच्च)  \textbf{(A4)} \newline \newline
                                \textbf{ TS 2.1.5.1} \newline
                  इन्द्रः॑ । व॒लस्य॑ । बिल᳚म् । अपेति॑ । औ॒र्णो॒त् । सः । यः । उ॒त्त॒म इत्यु॑त् - त॒मः । प॒शुः । आसी᳚त् । तम् । पृ॒ष्ठम् । प्रतीति॑ । स॒गृंह्येति॑ सं - गृह्य॑ । उदिति॑ । अ॒क्खि॒द॒त् । तम् । स॒हस्र᳚म् । प॒शवः॑ । अनु॑ । उदिति॑ । आ॒य॒न्न् । सः । उ॒न्न॒त इत्यु॑त् - न॒तः । अ॒भ॒व॒त् । यः । प॒शुका॑म॒ इति॑ प॒शु-का॒मः॒ । स्यात् । सः । ए॒तम् । ऐ॒न्द्रम् । उ॒न्न॒तमित्यु॑त्-न॒तम् । एति॑ । ल॒भे॒त॒ । इन्द्र᳚म् । ए॒व । स्वेन॑ । भा॒ग॒धेये॒नेति॑ भाग - धेये॑न । उपेति॑ । धा॒व॒ति॒ । सः । ए॒व । अ॒स्मै॒ । प॒शून् । प्रेति॑ । य॒च्छ॒ति॒ । प॒शु॒मानिति॑ पशु - मान् । ए॒व । भ॒व॒ति॒ । उ॒न्न॒त इत्यु॑त् - न॒तः । \textbf{  29} \newline
                  \newline
                                \textbf{ TS 2.1.5.2} \newline
                  भ॒व॒ति॒ । सा॒ह॒स्री । वै । ए॒षा । ल॒क्ष्मी । यत् । उ॒न्न॒त इत्यु॑त् - न॒तः । ल॒क्ष्मिया᳚ । ए॒व । प॒शून् । अवेति॑ । रु॒न्धे॒ । य॒दा । स॒हस्र᳚म् । प॒शून् । प्रा॒प्नु॒यादिति॑ प्र - आ॒प्नु॒यात् । अथ॑ । वै॒ष्ण॒वम् । वा॒म॒नम् । एति॑ । ल॒भे॒त॒ । ए॒तस्मिन्न्॑ । वै । तत् । स॒हस्र᳚म् । अधीति॑ । अ॒ति॒ष्ठ॒त् । तस्मा᳚त् । ए॒षः । वा॒म॒नः । समी॑षित॒ इति॒ सं - ई॒षि॒तः॒ । प॒शुभ्य॒ इति॑ प॒शु - भ्यः॒ । ए॒व । प्रजा॑तेभ्य॒ इति॒ प्र - जा॒ते॒भ्यः॒ । प्र॒ति॒ष्ठामिति॑ प्रति - स्थाम् । द॒धा॒ति॒ । कः । अ॒र्॒.ह॒ति॒ । स॒हस्र᳚म् । प॒शून् । प्राप्तु॒मिति॒ प्र - आ॒प्तु॒म् । इति॑ । आ॒हुः॒ । अ॒हो॒रा॒त्राणीत्य॑हः - रा॒त्राणि॑ । ए॒व । स॒हस्र᳚म् । स॒म्पाद्येति॑ सं - पाद्य॑ । एति॑ । ल॒भे॒त॒ । प॒शवः॑ । \textbf{  30} \newline
                  \newline
                                \textbf{ TS 2.1.5.3} \newline
                  वै । अ॒हो॒रा॒त्राणीइत्य॑हः - रा॒त्राणि॑ । प॒शून् । ए॒व । प्रजा॑ता॒निति॒ प्र - जा॒ता॒न् । प्र॒ति॒ष्ठामिति॑ प्रति - स्थाम् । ग॒म॒य॒ति॒ । ओष॑धीभ्य॒ इत्योष॑धि - भ्यः॒ । वे॒हत᳚म् । एति॑ । ल॒भे॒त॒ । प्र॒जाका॑म॒ इति॑ प्र॒जा - का॒मः॒ । ओष॑धयः । वै । ए॒तम् । प्र॒जाया॒ इति॑ प्र - जायै᳚ । परीति॑ । बा॒ध॒न्ते॒ । यः । अल᳚म् । प्र॒जाया॒ इति॑ प्र - जायै᳚ । सन्न् । प्र॒जामिति॑ प्र-जाम् । न । वि॒न्दते᳚ । ओष॑धयः । खलु॑ । वै । ए॒तस्यै᳚ । सूतु᳚म् । अपीति॑ । घ्न॒न्ति॒ । या । वे॒हत् । भव॑ति । ओष॑धीः । ए॒व । स्वेन॑ । भा॒ग॒धेये॒नेति॑ भाग - धेये॑न । उपेति॑ । धा॒व॒ति॒ । ताः । ए॒व । अ॒स्मै॒ । स्वात् । योनेः᳚ । प्र॒जामिति॑ प्र - जाम् । प्रेति॑ । ज॒न॒य॒न्ति॒ । वि॒न्दते᳚ । 31(50) \textbf{  31} \newline
                  \newline
                                \textbf{ TS 2.1.5.4} \newline
                  प्र॒जामिति॑ प्र - जाम् । आपः॑ । वै । ओष॑धयः । अस॑त् । पुरु॑षः । आपः॑ । ए॒व । अ॒स्मै॒ । अस॑तः । सत् । द॒द॒ति॒ । तस्मा᳚त् । आ॒हुः॒ । यः । च॒ । ए॒वम् । वेद॑ । यः । च॒ । न । आपः॑ । तु । वाव । अस॑तः । सत् । द॒द॒ति॒ । इति॑ । ऐ॒न्द्रीम् । सू॒तव॑शा॒मिति॑ सू॒त - व॒शा॒म् । एति॑ । ल॒भे॒त॒ । भूति॑काम॒ इति॒ भूति॑ - का॒मः॒ । अजा॑तः । वै । ए॒षः । यः । अल᳚म् । भूत्यै᳚ । सन्न् । भूति᳚म् । न । प्रा॒प्नोतीति॑ प्र - आ॒प्नोति॑ । इन्द्र᳚म् । खलु॑ । वै । ए॒षा । सू॒त्वा । व॒शा । अ॒भ॒व॒त् । \textbf{  32} \newline
                  \newline
                                \textbf{ TS 2.1.5.5} \newline
                  इन्द्र᳚म् । ए॒व । स्वेन॑ । भा॒ग॒धेये॒नेति॑ भाग - धेये॑न । उपेति॑ । धा॒व॒ति॒ । सः । ए॒व । ए॒न॒म् । भूति᳚म् । ग॒म॒य॒ति॒ । भव॑ति । ए॒व । यम् । सू॒त्वा । व॒शा । स्यात् । तम् । ऐ॒न्द्रम् । ए॒व । एति॑ । ल॒भे॒त॒ । ए॒तत् । वाव । तत् । इ॒न्द्रि॒यम् । सा॒क्षादिति॑ स - अ॒क्षात् । ए॒व । इ॒न्द्रि॒यम् । अवेति॑ । रु॒न्धे॒ । ऐ॒न्द्रा॒ग्नमित्यै᳚न्द्र - अ॒ग्नम् । पु॒न॒रु॒थ्सृ॒ष्टमिति॑ पुनः - उ॒थ्सृ॒ष्टम् । एति॑ । ल॒भे॒त॒ । यः । एति॑ । तृ॒तीया᳚त् । पुरु॑षात् । सोम᳚म् । न । पिबे᳚त् । विच्छि॑न्न॒ इति॒ वि - छि॒न्नः॒ । वै । ए॒तस्य॑ । सो॒म॒पी॒थ इति॑ सोम - पी॒थः । यः । ब्रा॒ह्म॒णः । सन्न् । एति॑ । \textbf{  33} \newline
                  \newline
                                \textbf{ TS 2.1.5.6} \newline
                  तृ॒तीया᳚त् । पुरु॑षात् । सोम᳚म् । न । पिब॑ति । इ॒न्द्रा॒ग्नी इती᳚न्द्र - अ॒ग्नी । ए॒व । स्वेन॑ । भा॒ग॒धेये॒नेति॑ भाग-धेये॑न । उपेति॑ । धा॒व॒ति॒ । तौ । ए॒व । अ॒स्मै॒ । सो॒म॒पी॒थमिति॑ सोम - पी॒थम् । प्रेति॑ । य॒च्छ॒तः॒ । उपेति॑ । ए॒न॒म् । सो॒म॒पी॒थ इति॑ सोम - पी॒थः । न॒म॒ति॒ । यत् । ऐ॒न्द्रः । भव॑ति । इ॒न्द्रि॒यम् । वै । सो॒म॒पी॒थ इति॑ सोम-पी॒थः । इ॒न्द्रि॒यम् । ए॒व । सो॒म॒पी॒थमिति॑ सोम - पी॒थम् । अवेति॑ । रु॒न्धे॒ । यत् । आ॒ग्ने॒यः । भव॑ति । आ॒ग्ने॒यः । वै । ब्रा॒ह्म॒णः । स्वाम् । ए॒व । दे॒वता᳚म् । अनु॑ । समिति॑ । त॒नो॒ति॒ । पु॒न॒रु॒थ्सृ॒ष्ट इति॑ पुनः-उ॒थ्सृ॒ष्टः । भ॒व॒ति॒ । पु॒न॒रु॒थ्सृ॒ष्ट इति॑ पुनः - उ॒थ्सृ॒ष्टः । इ॒व॒ । हि । ए॒तस्य॑ । \textbf{  34} \newline
                  \newline
                                \textbf{ TS 2.1.5.7} \newline
                  सो॒म॒पी॒थ इति॑ सोम - पी॒थः । समृ॑द्ध्या॒ इति॒ सं - ऋ॒द्ध्यै॒ । ब्रा॒ह्म॒ण॒स्प॒त्यमिति॑ ब्राह्मणः - प॒त्यम् । तू॒प॒रम् । एति॑ । ल॒भे॒त॒ । अ॒भि॒चर॒न्नित्य॑भि - चरन्न्॑ । ब्रह्म॑णः । पति᳚म् । ए॒व । स्वेन॑ । भा॒ग॒धेये॒नेति॑ भाग - धेये॑न । उपेति॑ । धा॒व॒ति॒ । तस्मै᳚ । ए॒व । ए॒न॒म् । एति॑ । वृ॒श्च॒ति॒ । ता॒जक् । आर्ति᳚म् । एति॑ । ऋ॒च्छ॒ति॒ । तू॒प॒रः । भ॒व॒ति॒ । क्षु॒रप॑वि॒रिति॑ क्षु॒र - प॒विः॒ । वै । ए॒षा । ल॒क्ष्मी । यत् । तू॒प॒रः । समृ॑द्ध्य॒ इति॒ सं - ऋ॒द्ध्यै॒ । स्फ्यः । यूपः॑ । भ॒व॒ति॒ । वज्रः॑ । वै । स्फ्यः । वज्र᳚म् । ए॒व । अ॒स्मै॒ । प्रेति॑ । ह॒र॒ति॒ । श॒र॒मय॒मिति॑ शर - मय᳚म् । ब॒र्॒.हिः । शृ॒णाति॑ । ए॒व । ए॒न॒म् । वैभी॑दकः । इ॒द्ध्मः ( ) । भि॒नत्ति॑ । ए॒व । ए॒न॒म् ॥ \textbf{  35} \newline
                  \newline
                      (भ॒व॒त्यु॒न्न॒तः - प॒शवो॑ - जनयन्ति वि॒न्दते॑ - ऽभव॒थ् - सन्नै - तस्ये॒ - ध्म - स्त्रीणि॑ च)  \textbf{(A5)} \newline \newline
                                \textbf{ TS 2.1.6.1} \newline
                  बा॒र्.॒ह॒स्प॒त्यम् । शि॒ति॒पृ॒ष्ठमिति॑ शिति - पृ॒ष्ठम् । एति॑ । ल॒भे॒त॒ । ग्राम॑काम॒ इति॒ ग्राम॑ - का॒मः॒ । यः । का॒मये॑त । पृ॒ष्ठम् । स॒मा॒नाना᳚म् । स्या॒म् । इति॑ । बृह॒स्पति᳚म् । ए॒व । स्वेन॑ । भा॒ग॒धेये॒नेति॑ भाग - धेये॑न । उपेति॑ । धा॒व॒ति॒ । सः । ए॒व । ए॒न॒म् । पृ॒ष्ठम् । स॒मा॒नाना᳚म् । क॒रो॒ति॒ । ग्रा॒मी । ए॒व । भ॒व॒ति॒ । शि॒ति॒पृ॒ष्ठ इति॑ शिति - पृ॒ष्ठः । भ॒व॒ति॒ । बा॒र्.॒ह॒स्प॒त्यः । हि । ए॒षः । दे॒वत॑या । समृ॑द्ध्या॒ इति॒ सं - ऋ॒द्ध्यै॒ । पौ॒ष्णम् । श्या॒मम् । एति॑ । ल॒भे॒त॒ । अन्न॑काम॒ इत्यन्न॑ - का॒मः॒ । अन्न᳚म् । वै । पू॒षा । पू॒षण᳚म् । ए॒व । स्वेन॑ । भा॒ग॒धेये॒नेति॑ भाग - धेये॑न । उपेति॑ । धा॒व॒ति॒ । सः । ए॒व । अ॒स्मै॒ । \textbf{  36} \newline
                  \newline
                                \textbf{ TS 2.1.6.2} \newline
                  अन्न᳚म् । प्रेति॑ । य॒च्छ॒ति॒ । अ॒न्ना॒द इत्य॑न्न - अ॒दः । ए॒व । भ॒व॒ति॒ । श्या॒मः । भ॒व॒ति॒ । ए॒तत् । वै । अन्न॑स्य । रू॒पम् । समृ॑द्ध्या॒ इति॒ सं - ऋ॒द्ध्यै॒ । मा॒रु॒तम् । पृश्नि᳚म् । एति॑ । ल॒भे॒त॒ । अन्न॑काम॒ इत्यन्न॑ - का॒मः॒ । अन्न᳚म् । वै । म॒रुतः॑ । म॒रुतः॑ । ए॒व । स्वेन॑ । भा॒ग॒धेये॒नेति॑ भाग - धेये॑न । उपेति॑ । धा॒व॒ति॒ । ते । ए॒व । अ॒स्मै॒ । अन्न᳚म् । प्रेति॑ । य॒च्छ॒न्ति॒ । अ॒न्ना॒द इत्य॑न्न - अ॒दः । ए॒व । भ॒व॒ति॒ । पृश्निः॑ । भ॒व॒ति॒ । ए॒तत् । वै । अन्न॑स्य । रू॒पम् । समृ॑द्ध्या॒ इति॒ सं - ऋ॒द्ध्यै॒ । ऐ॒न्द्रम् । अ॒रु॒णम् । एति॑ । ल॒भे॒त॒ । इ॒न्द्रि॒यका॑म॒ इती᳚न्द्रि॒य - का॒मः॒ । इन्द्र᳚म् । ए॒व । \textbf{  37} \newline
                  \newline
                                \textbf{ TS 2.1.6.3} \newline
                  स्वेन॑ । भा॒ग॒धेये॒नेति॑ भाग - धेये॑न । उपेति॑ । धा॒व॒ति॒ । सः । ए॒व । अ॒स्मि॒न्न् । इ॒न्द्रि॒यम् । द॒धा॒ति॒ । इ॒न्द्रि॒या॒वी । ए॒व । भ॒व॒ति॒ । अ॒रु॒णः । भ्रूमा॒निति॒ भ्रू - मा॒न् । भ॒व॒ति॒ । ऐ॒तत् । वै । इन्द्र॑स्य । रू॒पम् । समृ॑द्ध्या॒ इति॒ सं - ऋ॒द्ध्यै॒ । सा॒वि॒त्रम् । उ॒प॒द्ध्व॒स्तमित्यु॑प - ध्व॒स्तम् । एति॑ । ल॒भे॒त॒ । स॒निका॑म॒ इति॑ स॒नि - का॒मः॒ । स॒वि॒ता । वै । प्र॒स॒वाना॒मिति॑ प्र-स॒वाना᳚म् । ई॒शे॒ । स॒वि॒तार᳚म् । ए॒व । स्वेन॑ । भा॒ग॒धेये॒नेति॑ भाग - धेये॑न । उपेति॑ । धा॒व॒ति॒ । सः । ए॒व । अ॒स्मै॒ । स॒निम् । प्रेति॑ । सु॒व॒ति॒ । दान॑कामा॒ इति॒ दान॑-का॒माः॒ । अ॒स्मै॒ । प्र॒जा इति॑ प्र-जाः । भ॒व॒न्ति॒ । उ॒प॒द्ध्व॒स्त इत्यु॑प - ध्व॒स्तः । भ॒व॒ति॒ । सा॒वि॒त्रः । हि । ए॒षः । \textbf{  38} \newline
                  \newline
                                \textbf{ TS 2.1.6.4} \newline
                  दे॒वत॑या । समृ॑द्ध्या॒ इति॒ सं - ऋ॒द्ध्यै॒ । वै॒श्व॒दे॒वमिति॑ वैश्व-दे॒वम् । ब॒हु॒रू॒पमिति॑ बहु - रू॒पम् । एति॑ । ल॒भे॒त॒ । अन्न॑काम॒ इत्यन्न॑ - का॒मः॒ । वै॒श्व॒दे॒वमिति॑ वैश्व - दे॒वम् । वै । अन्न᳚म् । विश्वान्॑ । ए॒व । दे॒वान् । स्वेन॑ । भा॒ग॒धेये॒नेति॑ भाग-धेये॑न । उपेति॑ । धा॒व॒ति॒ । ते । ए॒व । अ॒स्मै॒ । अन्न᳚म् । प्रेति॑ । य॒च्छ॒न्ति॒ । अ॒न्ना॒द इत्य॑न्न - अ॒दः । ए॒व । भ॒व॒ति॒ । ब॒हु॒रू॒प इति॑ बहु-रू॒पः । भ॒व॒ति॒ । ब॒हु॒रू॒पमिति॑ बहु - रू॒पम् । हि । अन्न᳚म् । समृ॑द्ध्या॒ इति॒ सं - ऋ॒द्ध्यै॒ । वै॒श्व॒दे॒वमिति॑ वैश्व - दे॒वम् । ब॒हु॒रू॒पमिति॑ बहु - रू॒पम् । एति॑ । ल॒भे॒त॒ । ग्राम॑काम॒ इति॒ ग्राम॑ - का॒मः॒ । वै॒श्व॒दे॒वा इति॑ वैश्व - दे॒वाः । वै । स॒जा॒ता इति॑ स - जा॒ताः । विश्वान्॑ । ए॒व । दे॒वान् । स्वेन॑ । भा॒ग॒धेये॒नेति॑ भाग-धेये॑न । उपेति॑ । धा॒व॒ति॒ । ते । ए॒व । अ॒स्मै॒ । \textbf{  39} \newline
                  \newline
                                \textbf{ TS 2.1.6.5} \newline
                  स॒जा॒तानिति॑ स-जा॒तान् । प्रेति॑ । य॒च्छ॒न्ति॒ । ग्रा॒मी । ए॒व । भ॒व॒ति॒ । ब॒हु॒रू॒प इति॑ बहु - रू॒पः । भ॒व॒ति॒ । ब॒हु॒दे॒व॒त्य॑ इति॑ बहु-दे॒व॒त्यः॑ । हि । ए॒षः । समृ॑द्ध्या॒ इति॒ सं - ऋ॒द्ध्यै॒ । प्र॒जा॒प॒त्यमिति॑ प्राजा - प॒त्यम् । तू॒प॒रम् । एति॑ । ल॒भे॒त॒ । यस्य॑ । अना᳚ज्ञात॒मित्यना᳚ - ज्ञा॒त॒म् । इ॒व॒ । ज्योक् । आ॒मये᳚त् । प्रा॒जा॒प॒त्य इति॑ प्राजा - प॒त्यः । वै । पुरु॑षः । प्र॒जाप॑ति॒रिति॑ प्र॒जा - प॒तिः॒ । खलु॑ । वै । तस्य॑ । वे॒द॒ । यस्य॑ । अना᳚ज्ञात॒मित्यना᳚ - ज्ञा॒त॒म् । इ॒व॒ । ज्योक् । आ॒मय॑ति । प्र॒जाप॑ति॒मिति॑ प्र॒जा - प॒ति॒म् । ए॒व । स्वेन॑ । भा॒ग॒धेये॒नेति॑ भाग - धेये॑न । उपेति॑ । धा॒व॒ति॒ । सः । ए॒व । ए॒न॒म् । तस्मा᳚त् । स्रामा᳚त् । मु॒ञ्च॒ति॒ । तू॒प॒रः । भ॒व॒ति॒ । प्रा॒जा॒प॒त्य इति॑ प्राजा-प॒त्यः । हि ( ) । ए॒षः । दे॒वत॑या । समृ॑द्ध्या॒ इति॒ सं - ऋ॒द्ध्यै॒ ॥ \textbf{  40} \newline
                  \newline
                      (अ॒स्मा॒ - इन्द्र॑मे॒वै - ष - स॑जा॒ता विश्वा॑ने॒व दे॒वान्थ् स्वेन॑ भाग॒धेये॒नोप॑ धावति॒ त ए॒वास्मै᳚ - प्राजाप॒त्यो हि - त्रीणि॑ च) \textbf{(A6)} \newline \newline
                                \textbf{ TS 2.1.7.1} \newline
                  व॒ष॒ट्का॒र इति॑ वषट् - का॒रः । वै । गा॒य॒त्रि॒यै । शिरः॑ । अ॒च्छि॒न॒त् । तस्यै᳚ । रसः॑ । परेति॑ । अ॒प॒त॒त् । तम् । बृह॒स्पतिः॑ । उपेति॑ । अ॒गृ॒ह्णा॒त् । सा । शि॒ति॒पृ॒ष्ठेति॑ शिति - प॒ष्ठा । व॒शा । अ॒भ॒व॒त् । यः । द्वि॒तीयः॑ । प॒राप॑त॒दिति॑ परा - अप॑तत् । तम् । मि॒त्रावरु॑णा॒विति॑ मि॒त्रा - वरु॑णौ । उपेति॑ । अ॒गृ॒ह्णी॒ता॒म् । सा । द्वि॒रू॒पेति॑ द्वि - रू॒पा । व॒शा । अ॒भ॒व॒त् । यः । तृ॒तीयः॑ । प॒राप॑त॒दिति॑ परा - अप॑तत् । तम् । विश्वे᳚ । दे॒वाः । उपेति॑ । अ॒गृ॒ह्ण॒न्न् । सा । ब॒हु॒रू॒पेति॑ बहु - रू॒पा । व॒शा । अ॒भ॒व॒त् । यः । च॒तु॒र्थः । प॒राप॑त॒दिति॑ परा - अप॑तत् । सः । पृ॒थि॒वीम् । प्रेति॑ । अ॒वि॒श॒त् । तम् । बृह॒स्पतिः॑ । अ॒भीति॑ । \textbf{  41} \newline
                  \newline
                                \textbf{ TS 2.1.7.2} \newline
                  अ॒गृ॒ह्णा॒त् । अस्तु॑ । ए॒व । अ॒यम् । भोगा॑य । इति॑ । सः । उ॒क्ष॒व॒श इत्यु॑क्ष - व॒शः । समिति॑ । अ॒भ॒व॒त् । यत् । लोहि॑तम् । प॒राप॑त॒दिति॑ परा - अप॑तत् । तत् । रु॒द्रः । उपेति॑ । अ॒गृ॒ह्णा॒त् । सा । रौ॒द्री । रोहि॑णी । व॒शा । अ॒भ॒व॒त् । बा॒र्.॒ह॒स्प॒त्याम् । शि॒ति॒पृ॒ष्ठामिति॑ शिति - पृ॒ष्ठाम् । एति॑ । ल॒भे॒त॒ । ब्र॒ह्म॒व॒र्च॒सका॑म॒ इति॑ ब्रह्मवर्च॒स - का॒मः॒ । बृह॒स्पति᳚म् । ए॒व । स्वेन॑ । भा॒ग॒धेये॒नेति॑ भाग - धेये॑न । उपेति॑ । धा॒व॒ति॒ । सः । ए॒व । अ॒स्मि॒न्न् । ब्र॒ह्म॒व॒र्च॒समिति॑ ब्रह्म - व॒र्च॒सम् । द॒धा॒ति॒ । ब्र॒ह्म॒व॒र्च॒सीति॑ ब्रह्म - व॒र्च॒सी । ए॒व । भ॒व॒ति॒ । छन्द॑साम् । वै । ए॒षः । रसः॑ । यत् । व॒शा । रसः॑ । इ॒व॒ । खलु॑ । \textbf{  42} \newline
                  \newline
                                \textbf{ TS 2.1.7.3} \newline
                  वै । ब्र॒ह्म॒व॒र्च॒समिति॑ ब्रह्म - व॒र्च॒सम् । छन्द॑साम् । ए॒व । रसे॑न । रस᳚म् । ब्र॒ह्म॒व॒र्च॒समिति॑ ब्रह्म - व॒र्च॒सम् । अवेति॑ । रु॒न्धे॒ । मै॒त्रा॒व॒रु॒णीमिति॑ मैत्रा - व॒रु॒णीम् । द्वि॒रू॒पामिति॑ द्वि - रू॒पाम् । एति॑ । ल॒भे॒त॒ । वृष्टि॑काम॒ इति॒ वृष्टि॑ - का॒मः॒ । मै॒त्रम् । वै । अहः॑ । वा॒रु॒णी । रात्रिः॑ । अ॒हो॒रा॒त्राभ्या॒मित्य॑हः - रा॒त्राभ्या᳚म् । खलु॑ । वै । प॒र्जन्यः॑ । व॒र्.॒ष॒ति॒ । मि॒त्रावरु॑णा॒विति॑ मि॒त्रा - वरु॑णौ । ए॒व । स्वेन॑ । भा॒ग॒धेये॒नेति॑ भाग - धेये॑न । उपेति॑ । धा॒व॒ति॒ । तौ । ए॒व । अ॒स्मै॒ । अ॒हो॒रा॒त्राभ्या॒मित्य॑हः - रा॒त्राभ्या᳚म् । प॒र्जन्य᳚म् । व॒र्.॒ष॒य॒तः॒ । छन्द॑साम् । वै । ए॒षः । रसः॑ । यत् । व॒शा । रसः॑ । इ॒व॒ । खलु॑ । वै । वृष्टिः॑ । छन्द॑साम् । ए॒व । रसे॑न । \textbf{  43} \newline
                  \newline
                                \textbf{ TS 2.1.7.4} \newline
                  रस᳚म् । वृष्टि᳚म् । अवेति॑ । रु॒न्धे॒ । मै॒त्रा॒व॒रु॒णीमिति॑ मैत्रा - व॒रु॒णीम् । द्वि॒रू॒पामिति॑ द्वि - रू॒पाम् । एति॑ । ल॒भे॒त॒ । प्र॒जाका॑म॒ इति॑ प्र॒जा - का॒मः॒ । मै॒त्रम् । वै । अहः॑ । वा॒रु॒णी । रात्रिः॑ । अ॒हो॒रा॒त्राभ्या॒मित्य॑हः - रा॒त्राभ्या᳚म् । खलु॑ । वै । प्र॒जा इति॑ प्र - जाः । प्रेति॑ । जा॒य॒न्ते॒ । मि॒त्रावरु॑णा॒विति॑ मि॒त्रा-वरु॑णौ । ए॒व । स्वेन॑ । भा॒ग॒धेये॒नेति॑ भाग - धेये॑न । उपेति॑ । धा॒व॒ति॒ । तौ । ए॒व । अ॒स्मै॒ । अ॒हो॒रा॒त्राभ्या॒मित्य॑हः - रा॒त्राभ्या᳚म् । प्र॒जामिति॑ प्र -जाम् । प्रेति॑ । ज॒न॒य॒तः॒ । छन्द॑साम् । वै । ए॒षः । रसः॑ । यत् । व॒शा । रसः॑ । इ॒व॒ । खलु॑ । वै । प्र॒जेति॑ प्र-जा । छन्द॑साम् । ए॒व । रसे॑न । रस᳚म् । प्र॒जामिति॑ प्र -जाम् । अवेति॑ । \textbf{  44} \newline
                  \newline
                                \textbf{ TS 2.1.7.5} \newline
                  रु॒न्धे॒ । वै॒श्व॒दे॒वीमिति॑ वैश्व - दे॒वीम् । ब॒हु॒रू॒पामिति॑ बहु - रू॒पाम् । एति॑ । ल॒भे॒त॒ । अन्न॑काम॒ इत्यन्न॑ - का॒मः॒ । वै॒श्व॒दे॒वमिति॑ वैश्व - दे॒वम् । वै । अन्न᳚म् । विश्वान्॑ । ए॒व । दे॒वान् । स्वेन॑ । भा॒ग॒धेये॒नेति॑ भाग - धेये॑न । उपेति॑ । धा॒व॒ति॒ । ते । ए॒व । अ॒स्मै॒ । अन्न᳚म् । प्रेति॑ । य॒च्छ॒न्ति॒ । अ॒न्ना॒द इत्य॑न्न - अ॒दः । ए॒व । भ॒व॒ति॒ । छन्द॑साम् । वै । ए॒षः । रसः॑ । यत् । व॒शा । रसः॑ । इ॒व॒ । खलु॑ । वै । अन्न᳚म् । छन्द॑साम् । ए॒व । रसे॑न । रस᳚म् । अन्न᳚म् । अवेति॑ । रु॒न्धे॒ । वै॒श्व॒दे॒वीमिति॑ वैश्व - दे॒वीम् । ब॒हु॒रू॒पामिति॑ बहु-रू॒पाम् । एति॑ । ल॒भे॒त॒ । ग्राम॑काम॒ इति॒ ग्राम॑ - का॒मः॒ । वै॒श्व॒दे॒वा इति॑ वैश्व - दे॒वाः । वै । \textbf{  45} \newline
                  \newline
                                \textbf{ TS 2.1.7.6} \newline
                  स॒जा॒ता इति॑ स - जा॒ताः । विश्वान्॑ । ए॒व । दे॒वान् । स्वेन॑ । भा॒ग॒धेये॒नेति॑ भाग - धेये॑न । उपेति॑ । धा॒व॒ति॒ । ते । ए॒व । अ॒स्मै॒ । स॒जा॒तानिति॑ स-जा॒तान् । प्रेति॑ । य॒च्छ॒न्ति॒ । ग्रा॒मी । ए॒व । भ॒व॒ति॒ । छन्द॑साम् । वै । ए॒षः । रसः॑ । यत् । व॒शा । रसः॑ । इ॒व॒ । खलु॑ । वै । स॒जा॒ता इति॑ स - जा॒ताः । छन्द॑साम् । ए॒व । रसे॑न । रस᳚म् । स॒जा॒तानिति॑ स - जा॒तान् । अवेति॑ । रु॒न्धे॒ । बा॒र्.॒ह॒स्प॒त्यम् । उ॒क्ष॒व॒शमित्यु॑क्ष - व॒शम् । एति॑ । ल॒भे॒त॒ । ब्र॒ह्म॒व॒र्च॒सका॑म॒ इति॑ ब्रह्मवर्च॒स - का॒मः॒ । बृह॒स्पति᳚म् । ए॒व । स्वेन॑ । भा॒ग॒धेये॒नेति॑ भाग - धेये॑न । उपेति॑ । धा॒व॒ति॒ । सः । ए॒व । अ॒स्मि॒न्न् । ब्र॒ह्म॒व॒र्च॒समिति॑ ब्रह्म - व॒र्च॒सम् । \textbf{  46} \newline
                  \newline
                                \textbf{ TS 2.1.7.7} \newline
                  द॒धा॒ति॒ । ब्र॒ह्म॒व॒र्च॒सीति॑ ब्रह्म - व॒र्च॒सी । ए॒व । भ॒व॒ति॒ । वश᳚म् । वै । ए॒षः । च॒र॒ति॒ । यत् । उ॒क्षा । वशः॑ । इ॒व॒ । खलु॑ । वै । ब्र॒ह्म॒व॒र्च॒समिति॑ ब्रह्म - व॒र्च॒सम् । वशे॑न । ए॒व । वश᳚म् । ब्र॒ह्म॒व॒र्च॒समिति॑ ब्रह्म - व॒र्च॒सम् । अवेति॑ । रु॒न्धे॒ । रौ॒द्रीम् । रोहि॑णीम् । एति॑ । ल॒भे॒त॒ । अ॒भि॒चर॒न्नित्य॑भि-चरन्॑ । रु॒द्रम् । ए॒व । स्वेन॑ । भा॒ग॒धेये॒नेति॑ भाग - धेये॑न । उपेति॑ । धा॒व॒ति॒ । तस्मै᳚ । ए॒व । ए॒न॒म् । एति॑ । वृ॒श्च॒ति॒ । ता॒जक् । आर्ति᳚म् । एति॑ । ऋ॒च्छ॒ति॒ । रोहि॑णी । भ॒व॒ति॒ । रौ॒द्री । हि । ए॒षा । दे॒वत॑या । समृ॑द्ध्या॒ इति॒ सं - ऋ॒द्ध्यै॒ । स्फ्यः । यूपः॑ ( ) । भ॒व॒ति॒ । वज्रः॑ । वै । स्फ्यः । वज्र᳚म् । ए॒व । अ॒स्मै॒ । प्रेति॑ । ह॒र॒ति॒ । श॒र॒मय॒मिति॑ शर - मय᳚म् । ब॒र्॒.हिः । शृ॒णाति॑ । ए॒व । ए॒न॒म् । वैभी॑दकः । इ॒द्ध्मः । भि॒नत्ति॑ । ए॒व । ए॒न॒म् ॥ \textbf{  47} \newline
                  \newline
                      (अ॒भि - खलु॒ - वृष्टिः॒ छन्द॑सामे॒व रसे॑न - प्र॒जामव॑ - वैश्वदे॒वा वै - ब्र॑ह्मवर्च॒सं - ॅयूप॒ - एका॒न्नविꣳ॑श॒तिश्च॑)  \textbf{(A7)} \newline \newline
                                \textbf{ TS 2.1.8.1} \newline
                  अ॒सौ । आ॒दि॒त्यः । न । वीति॑ । अ॒रो॒च॒त॒ । तस्मै᳚ । दे॒वाः । प्राय॑श्चित्तिम् । ऐ॒च्छ॒न्न् । तस्मै᳚ । ए॒ताम् । सौ॒रीम् । श्वे॒ताम् । व॒शाम् । एति॑ । अ॒ल॒भ॒न्त॒ । तया᳚ । ए॒व । अ॒स्मि॒न्न् । रुच᳚म् । अ॒द॒धुः॒ । यः । ब्र॒ह्म॒व॒र्च॒सका॑म॒ इति॑ ब्रह्मवर्च॒स - का॒मः॒ । स्यात् । तस्मै᳚ । ए॒ताम् । सौ॒रीम् । श्वे॒ताम् । व॒शाम् । एति॑ । ल॒भे॒त॒ । अ॒मुम् । ए॒व । आ॒दि॒त्यम् । स्वेन॑ । भा॒ग॒धेये॒नेति॑ भाग - धेये॑न । उपेति॑ । धा॒व॒ति॒ । सः । ए॒व । अ॒स्मि॒न्न् । ब॒ह्म॒व॒र्च॒समिति॑ ब्रह्म - व॒र्च॒सम् । द॒धा॒ति॒ । ब्र॒ह्म॒व॒र्च॒सीति॑ ब्रह्म - व॒र्च॒सी । ए॒व । भ॒व॒ति॒ । बै॒ल्॒.वः । यूपः॑ । भ॒व॒ति॒ । अ॒सौ । \textbf{  48} \newline
                  \newline
                                \textbf{ TS 2.1.8.2} \newline
                  वै । आ॒दि॒त्यः । यतः॑ । अजा॑यत । ततः॑ । बि॒ल्वः॑ । उदिति॑ । अ॒ति॒ष्ठ॒त् । सयो॒नीति॒ स - यो॒नि॒ । ए॒व । ब्र॒ह्म॒व॒र्च॒समिति॑ ब्रह्म - व॒र्च॒सम् । अवेति॑ । रु॒न्धे॒ । ब्रा॒ह्म॒ण॒स्प॒त्यामिति॑ ब्राह्मणः - प॒त्याम् । ब॒भ्रु॒क॒र्णीमिति॑ बभ्रु - क॒र्णीम् । एति॑ । ल॒भे॒त॒ । अ॒भि॒चर॒न्नित्य॑भि - चरन्॑ । वा॒रु॒णम् । दश॑कपाल॒मिति॒ दश॑ - क॒पा॒ल॒म् । पु॒रस्ता᳚त् । निरिति॑ । व॒पे॒त् । वरु॑णेन । ए॒व । भ्रातृ॑व्यम् । ग्रा॒ह॒यि॒त्वा । ब्रह्म॑णा । स्तृ॒णु॒ते॒ । ब॒भ्रु॒क॒र्णीति॑ बभ्रु - क॒र्णी । भ॒व॒ति॒ । ए॒तत् । वै । ब्रह्म॑णः । रू॒पम् । समृ॑द्ध्या॒ इति॒ सं-ऋ॒द्ध्यै॒ । स्फ्यः । यूपः॑ । भ॒व॒ति॒ । वज्रः॑ । वै । स्फ्यः । वज्र᳚म् । ए॒व । अ॒स्मै॒ । प्रेति॑ । ह॒र॒ति॒ । श॒र॒मय॒मिति॑ शर - मय᳚म् । ब॒र्॒.हिः । शृ॒णाति॑ । \textbf{  49} \newline
                  \newline
                                \textbf{ TS 2.1.8.3} \newline
                  ए॒व । ए॒न॒म् । वैभी॑दकः । इ॒द्ध्मः । भि॒नत्ति॑ । ए॒व । ए॒न॒म् । वै॒ष्ण॒वम् । वा॒म॒नम् । एति॑ । ल॒भे॒त॒ । यम् । य॒ज्ञ्ः । न । उ॒प॒नमे॒दित्यु॑प-नमे᳚त् । विष्णुः॑ । वै । य॒ज्ञ्ः । विष्णु᳚म् । ए॒व । स्वेन॑ । भा॒ग॒धेये॒नेति॑ भाग - धेये॑न । उपेति॑ । धा॒व॒ति॒ । सः । ए॒व । अ॒स्मै॒ । य॒ज्ञ्म् । प्रेति॑ । य॒च्छ॒ति॒ । उपेति॑ । ए॒न॒म् । य॒ज्ञ्ः । न॒म॒ति॒ । वा॒म॒नः । भ॒व॒ति॒ । वै॒ष्ण॒वः । हि । ए॒षः । दे॒वत॑या । समृ॑द्ध्या॒ इति॒ सं - ऋ॒द्ध्यै॒ । त्वा॒ष्ट्रम् । व॒ड॒बम् । एति॑ । ल॒भे॒त॒ । प॒शुका॑म॒ इति॑ प॒शु - का॒मः॒ । त्वष्टा᳚ । वै । प॒शू॒नाम् । मि॒थु॒नाना᳚म् । \textbf{  50} \newline
                  \newline
                                \textbf{ TS 2.1.8.4} \newline
                  प्र॒ज॒न॒यि॒तेति॑ प्र - ज॒न॒यि॒ता । त्वष्टा॑रम् । ए॒व । स्वेन॑ । भा॒ग॒धेये॒नेति॑ भाग - धेये॑न । उपेति॑ । धा॒व॒ति॒ । सः । ए॒व । अ॒स्मै॒ । प॒शून् । मि॒थु॒नान् । प्रेति॑ । ज॒न॒य॒ति॒ । प्र॒जेति॑ प्र - जा । हि । वै । ए॒तस्मिन्न्॑ । प॒शवः॑ । प्रवि॑ष्टा॒ इति॒ प्र - वि॒ष्टाः॒ । अथ॑ । ए॒षः । पुमान्॑ । सन्न् । व॒ड॒बः । सा॒क्षादिति॑ स - अ॒क्षात् । ए॒व । प्र॒जामिति॑ प्र -जाम् । प॒शून् । अवेति॑ । रु॒न्धे॒ । मै॒त्रम् । श्वे॒तम् । एति॑ । ल॒भे॒त॒ । स॒ङ्ग्रा॒म इति॑ सं - ग्रा॒मे । संॅय॑त्त॒ इति॒ सं - य॒त्ते॒ । स॒म॒यका॑म॒ इति॑ सम॒य - का॒मः॒ । मि॒त्रम् । ए॒व । स्वेन॑ । भा॒ग॒धेये॒नेति॑ भाग - धेये॑न । उपेति॑ । धा॒व॒ति॒ । सः । ए॒व । ए॒न॒म् । मि॒त्रेण॑ । समिति॑ । न॒य॒ति॒ । \textbf{  51} \newline
                  \newline
                                \textbf{ TS 2.1.8.5} \newline
                  वि॒शा॒ल इति॑ वि - शा॒लः । भ॒व॒ति॒ । व्यव॑सायय॒तीति॑ वि - अव॑साययति । ए॒व । ए॒न॒म् । प्रा॒जा॒प॒त्यमिति॑ प्राजा - प॒त्यम् । कृ॒ष्णम् । एति॑ । ल॒भे॒त॒ । वृष्टि॑काम॒ इति॒ वृष्टि॑ - का॒मः॒ । प्र॒जाप॑ति॒रिति॑ प्र॒जा - प॒तिः॒ । वै । वृष्ट्याः᳚ । ई॒शे॒ । प्र॒जाप॑ति॒मिति॑ प्र॒जा - प॒ति॒म् । ए॒व । स्वेन॑ । भा॒ग॒धेये॒नेति॑ भाग-धेये॑न । उपेति॑ । धा॒व॒ति॒ । सः । ए॒व । अ॒स्मै॒ । प॒र्जन्य᳚म् । व॒र्॒.ष॒य॒ति॒ । कृ॒ष्णः । भ॒व॒ति॒ । ए॒तत् । वै । वृष्‌ट्यै᳚ । रू॒पम् । रू॒पेण॑ । ए॒व । वृष्टि᳚म् । अवेति॑ । रु॒न्धे॒ । श॒बलः॑ । भ॒व॒ति॒ । वि॒द्युत॒मिति॑ वि - द्युत᳚म् । ए॒व । अ॒स्मै॒ । ज॒न॒यि॒त्वा । व॒र्.॒ष॒य॒त॒ । अ॒वा॒शृ॒ङ्गः । भ॒व॒ति॒ । वृष्टि᳚म् । ए॒व । अ॒स्मै॒ । नीति॑ । य॒च्छ॒ति॒ ( ) ॥ \textbf{  52} \newline
                  \newline
                      (अ॒सौ - शृ॒णाति॑ - मिथु॒नानां᳚ - नयति - यच्छति)  \textbf{(A8)} \newline \newline
                                \textbf{ TS 2.1.9.1} \newline
                  वरु॑णम् । सु॒षु॒वा॒णम् । अ॒न्नाद्य॒मित्य॑न्न - अद्य᳚म् । न । उपेति॑ । अ॒न॒म॒त् । सः । ए॒ताम् । वा॒रु॒णीम् । कृ॒ष्णाम् । व॒शाम् । अ॒प॒श्य॒त् । ताम् । स्वायै᳚ । दे॒वता॑यै । एति॑ । अ॒ल॒भ॒त॒ । ततः॑ । वै । तम् । अ॒न्नाद्य॒मित्य॑न्न - अद्य᳚म् । उपेति॑ । अ॒न॒म॒त् । यम् । अल᳚म् । अ॒न्नाद्या॒येत्य॑न्न - अद्या॑य । सन्त᳚म् । अ॒न्नाद्य॒मित्य॑न्न - अद्य᳚म् । न । उ॒प॒नमे॒दित्यु॑प-नमे᳚त् । सः । ए॒ताम् । वा॒रु॒णीम् । कृ॒ष्णाम् । व॒शाम् । एति॑ । ल॒भे॒त॒ । वरु॑णम् । ए॒व । स्वेन॑ । भा॒ग॒धेये॒नेति॑ भाग - धेये॑न । उपेति॑ । धा॒व॒ति॒ । सः । ए॒व । अ॒स्मै॒ । अन्न᳚म् । प्रेति॑ । य॒च्छ॒ति॒ । अ॒न्ना॒द इत्य॑न्न - अ॒दः । \textbf{  53} \newline
                  \newline
                                \textbf{ TS 2.1.9.2} \newline
                  ए॒व । भ॒व॒ति॒ । कृ॒ष्णा । भ॒व॒ति॒ । वा॒रु॒णी । हि । ए॒षा । दे॒वत॑या । समृ॑द्ध्या॒ इति॒ सं - ऋ॒द्ध्यै॒ । मै॒त्रम् । श्वे॒तम् । एति॑ । ल॒भे॒त॒ । वा॒रु॒णम् । कृ॒ष्णम् । अ॒पाम् । च॒ । ओष॑धीनाम् । च॒ । स॒धांविति॑ सं - धौ । अन्न॑काम॒ इत्यन्न॑ - का॒मः॒ । मै॒त्रीः । वै । ओष॑धयः । वा॒रु॒णीः । आपः॑ । अ॒पाम् । च॒ । खलु॑ । वै । ओष॑धीनाम् । च॒ । रस᳚म् । उपेति॑ । जी॒वा॒मः॒ । मि॒त्रावरु॑णा॒विति॑ मि॒त्रा - वरु॑णौ । ए॒व । स्वेन॑ । भा॒ग॒धेये॒नेति॑ भाग - धेये॑न । उपेति॑ । धा॒व॒ति॒ । तौ । ए॒व । अ॒स्मै॒ । अन्न᳚म् । प्रेति॑ । य॒च्छ॒तः॒ । अ॒न्ना॒द इत्य॑न्न - अ॒दः । ए॒व । भ॒व॒ति॒ । \textbf{  54} \newline
                  \newline
                                \textbf{ TS 2.1.9.3} \newline
                  अ॒पाम् । च॒ । ओष॑धीनाम् । च॒ । स॒धांविति॑ सं-धौ । एति॑ । ल॒भ॒ते॒ । उ॒भय॑स्य । अव॑रुद्ध्या॒ इत्यव॑ - रु॒ध्यै॒ । विशा॑ख॒ इति॒ वि-शा॒खः॒ । यूपः॑ । भ॒व॒ति॒ । द्वे इति॑ । हि । ए॒ते इति॑ । दे॒वते॒ इति॑ । समृ॑द्ध्या॒ इति॒ सं - ऋ॒द्ध्यै॒ । मै॒त्रम् । श्वे॒तम् । एति॑ । ल॒भे॒त॒ । वा॒रु॒णम् । कृ॒ष्णम् । ज्योगा॑मया॒वीति॒ ज्योक् - आ॒म॒या॒वी॒ । यत् । मै॒त्रः । भव॑ति । मि॒त्रेण॑ । ए॒व । अ॒स्मै॒ । वरु॑णम् । श॒म॒य॒ति॒ । यत् । वा॒रु॒णः । सा॒क्षादिति॑ स - अ॒क्षात् । ए॒व । ए॒न॒म् । व॒रु॒ण॒पा॒शादिति॑ वरुण - पा॒शात् । मु॒ञ्च॒ति॒ । उ॒त । यदि॑ । इ॒तासु॒रिती॒त-अ॒सुः॒ । भव॑ति । जीव॑ति । ए॒व । दे॒वाः । वै । पुष्टि᳚म् । न । अ॒वि॒न्द॒न्न् । \textbf{  55} \newline
                  \newline
                                \textbf{ TS 2.1.9.4} \newline
                  ताम् । मि॒थु॒ने । अ॒प॒श्य॒न्न् । तस्या᳚म् । न । समिति॑ । अ॒रा॒ध॒य॒न्न् । तौ । अ॒श्विनौ᳚ । अ॒ब्रू॒ता॒म् । आ॒वयोः᳚ । वै । ए॒षा । मा । ए॒तस्या᳚म् । व॒द॒द्ध्व॒म् । इति॑ । सा । अ॒श्विनोः᳚ । ए॒व । अ॒भ॒व॒त् । यः । पुष्टि॑काम॒ इति॒ पुष्टि॑ - का॒मः॒ । स्यात् । सः । ए॒ताम् । आ॒श्वि॒नीम् । य॒मीम् । व॒शाम् । एति॑ । ल॒भे॒त॒ । अ॒श्विनौ᳚ । ए॒व । स्वेन॑ । भा॒ग॒धेये॒नेति॑ भाग - धेये॑न । उपेति॑ । धा॒व॒ति॒ । तौ । ए॒व । अ॒स्मि॒न्न् । पुष्टि᳚म् । ध॒त्तः॒ । पुष्य॑ति । प्र॒जयेति॑ प्र - जया᳚ । प॒शुभि॒रिति॑ प॒शु - भिः॒ ॥ \textbf{  56} \newline
                  \newline
                      (अ॒न्ना॒दो᳚ - ऽन्ना॒द ए॒व भ॑वत्य - विन्द॒न् - पञ्च॑चत्वारिꣳशच्च)  \textbf{(A9)} \newline \newline
                                \textbf{ TS 2.1.10.1} \newline
                  आ॒श्वि॒नम् । धू॒म्रल॑लाम॒मिति॑ धू॒म्र - ल॒ला॒म॒म् । एति॑ । ल॒भे॒त॒ । यः । दुर्ब्रा᳚ह्मण॒ इति॒ दुः - ब्रा॒ह्म॒णः॒ । सोम᳚म् । पिपा॑सेत् । अ॒श्विनौ᳚ । वै । दे॒वाना᳚म् । असो॑मपा॒वित्यसो॑म - पौ॒ । आ॒स्ता॒म् । तौ । प॒श्चा । सो॒म॒पी॒थमिति॑ सोम - पी॒थम् । प्रेति॑ । आ॒प्नु॒ता॒म् । अ॒श्विनौ᳚ । ए॒तस्य॑ । दे॒वता᳚ । यः । दुर्ब्रा᳚ह्मण॒ इति॒ दुः- ब्रा॒ह्म॒णः॒ । सोम᳚म् । पिपा॑सति । अ॒श्विनौ᳚ । ए॒व । स्वेन॑ । भा॒ग॒धेये॒नेति॑ भाग - धेये॑न । उपेति॑ । धा॒व॒ति॒ । तौ । ए॒व । अ॒स्मै॒ । सो॒म॒पी॒थमिति॑ सोम-पी॒थम् । प्रेति॑ । य॒च्छ॒तः॒ । उपेति॑ । ए॒न॒म् । सो॒म॒पी॒थ इति॑ सोम - पी॒थः । न॒म॒ति॒ । यत् । धू॒म्रः । भव॑ति । धू॒म्रि॒माण᳚म् । ए॒व । अ॒स्मा॒त् । अपेति॑ । ह॒न्ति॒ । ल॒लामः॑ । \textbf{  57} \newline
                  \newline
                                \textbf{ TS 2.1.10.2} \newline
                  भ॒व॒ति॒ । मु॒ख॒तः । ए॒व । अ॒स्मि॒न्न् । तेजः॑ । द॒धा॒ति॒ । वा॒य॒व्य᳚म् । गो॒मृ॒गमिति॑ गो - मृ॒गम् । एति॑ । ल॒भे॒त॒ । यम् । अज॑घ्निवाꣳसम् । अ॒भि॒शꣳसे॑यु॒रित्य॑भि - शꣳसे॑युः । अपू॑ता । वै । ए॒तम् । वाक् । ऋ॒च्छ॒ति॒ । यम् । अज॑घ्निवाꣳसम् । अ॒भि॒शꣳस॒न्तीत्य॑भि - शꣳस॑न्ति । न । ए॒षः । ग्रा॒म्यः । प॒शुः । न । आ॒र॒ण्यः । यत् । गो॒मृ॒ग इति॑ गो - मृ॒गः । न । इ॒व॒ । ए॒षः । ग्रामे᳚ । न । अर॑ण्ये । यम् । अज॑घ्निवाꣳसम् । अ॒भि॒शꣳस॒न्तीत्य॑भि-शꣳस॑न्ति । वा॒युः । वै । दे॒वाना᳚म् । प॒वित्र᳚म् । वा॒युम् । ए॒व । स्वेन॑ । भा॒ग॒धेये॒नेति॑ भाग-धेये॑न । उपेति॑ । धा॒व॒ति॒ । सः । ए॒व । \textbf{  58} \newline
                  \newline
                                \textbf{ TS 2.1.10.3} \newline
                  ए॒न॒म् । प॒व॒य॒ति॒ । परा॑ची । वै । ए॒तस्मै᳚ । व्यु॒च्छन्तीति॑ वि-उ॒च्छन्ती᳚ । वीति॑ । उ॒च्छ॒ति॒ । तमः॑ । पा॒प्मान᳚म् । प्रेति॑ । वि॒श॒ति॒ । यस्य॑ । आ॒श्वि॒ने । श॒स्यमा॑ने । सूर्यः॑ । न । आ॒विः । भव॑ति । सौ॒र्यम् । ब॒हु॒रू॒पमिति॑ बहु - रू॒पम् । एति॑ । ल॒भे॒त॒ । अ॒मुम् । ए॒व । आ॒दि॒त्यम् । स्वेन॑ । भा॒ग॒धेये॒नेति॑ भाग - धेये॑न । उपेति॑ । धा॒व॒ति॒ । सः । ए॒व । अ॒स्मा॒त् । तमः॑ । पा॒प्मान᳚म् । अपेति॑ । ह॒न्ति॒ । प्र॒तीची᳚ । अ॒स्मै॒ । व्यु॒च्छन्तीति॑ वि - उच्छन्ती᳚ । वीति॑ । उ॒च्छ॒ति॒ । अपेति॑ । तमः॑ । पा॒प्मान᳚म् । ह॒ते॒ ॥ \textbf{  59} \newline
                  \newline
                      (ल॒लामः॒ - स ए॒व - षट्च॑त्वारिꣳशच्च)  \textbf{(A10)} \newline \newline
                                \textbf{ TS 2.1.11.1} \newline
                  इन्द्र᳚म् । वः॒ । वि॒श्वतः॑ । परीति॑ । इन्द्र᳚म् । नरः॑ । मरु॑तः । यत् । ह॒ । वः॒ । दि॒वः । या । वः॒ । शर्म॑ ॥ भरे॑षु । इन्द्र᳚म् । सु॒हव॒मिति॑ सु - हव᳚म् । ह॒वा॒म॒हे॒ । अꣳ॒॒हो॒मुच॒मित्यꣳ॑हः - मुच᳚म् । सु॒कृत॒मिति॑ सु - कृत᳚म् । दैव्य᳚म् । जन᳚म् ॥ अ॒ग्निम् । मि॒त्रम् । वरु॑णम् । सा॒तये᳚ । भग᳚म् । द्यावा॑पृथि॒वी इति॒ द्यावा᳚ - पृ॒थि॒वी । म॒रुतः॑ । स्व॒स्तये᳚ ॥ म॒मत्तु॑ । नः॒ । परि॒ज्मेति॒ परि॑-ज्मा॒ । व॒स॒र्॒.हा । म॒मत्तु॑ । वातः॑ । अ॒पाम् । वृष॑ण्वा॒निति॒ वृषण॑ - वा॒न् ॥ शि॒शी॒तम् । इ॒न्द्रा॒प॒र्व॒तेती᳚न्द्रा - प॒र्व॒ता॒ । यु॒वम् । नः॒ । तत् । नः॒ । विश्वे᳚ । व॒रि॒व॒स्य॒न्तु॒ । दे॒वाः ॥ प्रि॒या । वः॒ । नाम॑ । \textbf{  60} \newline
                  \newline
                                \textbf{ TS 2.1.11.2} \newline
                  हु॒वे॒ । तु॒राणा᳚म् ॥ एति॑ । यत् । तृ॒पत् । म॒रु॒तः॒ । वा॒व॒शा॒नाः ॥ श्रि॒यसे᳚ । कम् । भा॒नुभि॒रिति॑ भा॒नु-भिः॒ । समिति॑ । मि॒मि॒क्षि॒रे॒ । ते । र॒श्मिभि॒रिति॑ र॒श्मि - भिः॒ । ते । ऋक्व॑भि॒रित्यृक्व॑ - भिः॒ । सु॒खा॒दय॒ इति॑ सु - खा॒दयः॑ ॥ ते । वाशी॑मन्त॒ इति॒ वाशि॑ - म॒न्तः॒ । इ॒ष्मिणः॑ । अभी॑रवः । वि॒द्रे । प्रि॒यस्य॑ । मारु॑तस्य । धाम्नः॑ ॥ अ॒ग्निः । प्र॒थ॒मः । वसु॑भि॒रिति॒ वसु॑-भिः॒ । नः॒ । अ॒व्या॒त् । सोमः॑ । रु॒द्रेभिः॑ । अ॒भीति॑ । र॒क्ष॒तु॒ । त्मना᳚ ॥ इन्द्रः॑ । म॒रुद्भि॒रिति॑ म॒रुत् - भिः॒ । ऋ॒तु॒धेत्यृ॑तु - धा । कृ॒णो॒तु॒ । आ॒दि॒त्यैः । नः॒ । वरु॑णः । समिति॑ । शि॒शा॒तु॒ ॥ समिति॑ । नः॒ । दे॒वः । वसु॑भि॒रिति॒ वसु॑ - भिः॒ । अ॒ग्निः । समिति॑ । \textbf{  61} \newline
                  \newline
                                \textbf{ TS 2.1.11.3} \newline
                  सोमः॑ । त॒नूभिः॑ । रु॒द्रिया॑भिः ॥ समिति॑ । इन्द्रः॑ । म॒रुद्भि॒रिति॑ म॒रुत् - भिः॒ । य॒ज्ञियैः᳚ । समिति॑ । आ॒दि॒त्यैः । नः॒ । वरु॑णः । अ॒जि॒ज्ञि॒प॒त् ॥ यथा᳚ । आ॒दि॒त्याः । वसु॑भि॒रिति॒ वसु॑ - भिः॒ । स॒म्ब॒भू॒वुरिति॑ सं - ब॒भू॒वुः । म॒रुद्भि॒रिति॑ म॒रुत् - भिः॒ । रु॒द्राः । स॒मजा॑न॒तेति॑ सं - अजा॑नत् । अ॒भि ॥ ए॒वा । त्रि॒णा॒म॒न्निति॑ त्रि - ना॒म॒न्न् । अहृ॑णीयमानाः । विश्वे᳚ । दे॒वाः । सम॑नस॒ इति॒ स - म॒न॒सः॒ । भ॒व॒न्तु॒ ॥ कुत्र॑ । चि॒त्॒ । यस्य॑ । समृ॑ता॒विति॒ सं-ऋ॒तौ॒ । र॒ण्वाः । नरः॑ । नृ॒षद॑न॒ इति॑ नृ-सद॑ने ॥ अर्.ह॑न्तः । चि॒त् । यम् । इ॒न्ध॒ते । स॒ञ्ज॒नय॒न्तीति॑ सं - ज॒नय॑न्ति । ज॒न्तवः॑ ॥ समिति॑ । यत् । इ॒षः । वना॑महे । समिति॑ । ह॒व्या । मानु॑षाणाम् ॥ उ॒त । द्यु॒म्नस्य॑ । शव॑सः । \textbf{  62} \newline
                  \newline
                                \textbf{ TS 2.1.11.4} \newline
                  ऋ॒तस्य॑ । र॒श्मिम् । एति॑ । द॒दे॒ ॥ य॒ज्ञ्ः । दे॒वाना᳚म् । प्रतीति॑ । ए॒ति॒ । सु॒म्नम् । आदि॑त्यासः । भव॑त । मृ॒ड॒यन्तः॑ ॥ एति॑ । वः॒ । अ॒र्वाची᳚ । सु॒म॒तिरिति॑ सु - म॒तिः । व॒वृ॒त्या॒त् । अꣳ॒॒होः । चि॒त् । या । व॒रि॒वो॒वित्त॒रेति॑ वरिवो॒वित् - त॒रा॒ । अस॑त् ॥ शुचिः॑ । अ॒पः । सू॒यव॑सा॒ इति॑ सु - यव॑साः । अद॑ब्धः । उपेति॑ । क्षे॒ति॒ । वृ॒द्धव॑या॒ इति॑ वृ॒द्ध - व॒याः॒ । सु॒वीर॒ इति॑ सु - वीरः॑ ॥ नकिः॑ । तम् । घ्न॒न्ति॒ । अन्ति॑तः । न । दू॒रात् । यः । आ॒दि॒त्याना᳚म् । भव॑ति । प्रणी॑त॒विति॒ प्र - नी॒तौ॒ ॥ धा॒रय॑न्तः । आ॒दि॒त्यासः॑ । जग॑त् । स्थाः । दे॒वाः । विश्व॑स्य । भुव॑नस्य । गो॒पा इति॑ गो-पाः ॥ दी॒र्घाधि॑य॒ इति॑ दी॒र्घ - धि॒यः॒ । रक्ष॑माणाः । \textbf{  63} \newline
                  \newline
                                \textbf{ TS 2.1.11.5} \newline
                  अ॒सु॒र्य᳚म् । ऋ॒तावा॑न॒ इत्यृ॒त - वा॒नः॒ । चय॑मानाः । ऋ॒णानि॑ ॥ ति॒स्रः । भूमीः᳚ । धा॒र॒य॒न्न् । त्रीन् । उ॒त । द्यून् । त्रीणि॑ । व्र॒ता । वि॒दथे᳚ । अ॒न्तः । ए॒षा॒म् ॥ ऋ॒तेन॑ । आ॒दि॒त्याः॒ । महि॑ । वः॒ । म॒हि॒त्वमिति॑ महि - त्वम् । तत् । अ॒र्य॒म॒न्न् । व॒रु॒ण॒ । मि॒त्र॒ । चारु॑ ॥ त्यान् । नु । क्ष॒त्रियान्॑ । अवः॑ । आ॒दि॒त्यान् । या॒चि॒षा॒म॒हे॒ ॥ सु॒मृ॒डी॒कानिति॑ सु - म॒डी॒कान् । अ॒भिष्ट॑ये ॥ न । द॒क्षि॒णा । वीति॑ । चि॒कि॒ते॒ । न । स॒व्या । न । प्रा॒चीन᳚म् । आ॒दि॒त्याः॒ । न । उ॒त । प॒श्चा ॥ पा॒क्या᳚ । चि॒त् । व॒स॒वः॒ । धी॒र्या᳚ । चि॒त् । \textbf{  64} \newline
                  \newline
                                \textbf{ TS 2.1.11.6} \newline
                  यु॒ष्मानी॑तः । अभ॑यम् । ज्योतिः॑ । अ॒श्या॒म् ॥ आ॒दि॒त्याना᳚म् । अव॑सा । नूत॑नेन । स॒क्षी॒महि॑ । शर्म॑णा । शन्त॑मे॒नेति॒ शं - त॒मे॒न॒ ॥ अ॒ना॒गा॒स्त्व इत्य॑नागाः-त्वे । अ॒दि॒ति॒त्व इत्य॑दिति - त्वे । तु॒रासः॑ । इ॒मम् । य॒ज्ञ्म् । द॒ध॒तु॒ । श्रोष॑माणाः ॥ इ॒मम् । मे॒ । व॒रु॒ण॒ । श्रु॒ध॒ । हव᳚म् । अ॒द्य । च॒ । मृ॒ड॒य॒ ॥ त्वाम् । अ॒व॒स्युः । एति॑ । च॒के॒ ॥ तत् । त्वा॒ । या॒मि॒ । ब्रह्म॑णा । वन्द॑मानः । तत् । एति॑ । शा॒स्ते॒ । यज॑मानः । ह॒विर्भि॒रिति॑ ह॒विः - भिः॒ ॥ अहे॑डमानः । व॒रु॒ण॒ । इ॒ह । बो॒धि॒ । उरु॑शꣳ॒॒सेत्युरु॑ - शꣳ॒॒स॒ । मा । नः॒ । आयुः॑ । प्रेति॑ । मो॒षीः॒ ॥ \textbf{  65 } \newline
                  \newline
                       (नामा॒ - ऽग्निः सꣳ - शव॑सो॒ - रक्ष॑माणा - धी॒र्यां॑ चि॒दे - का॒न्न प॑ञ्चा॒शच्च॑)  \textbf{(A11)} \newline \newline
\textbf{praSna korvai with starting padams of 1 to 11 Anuvaakams :-} \newline
(वा॒य॒व्यं॑ - प्रा॒जप॑ति॒स्ता वरु॑णं - देवासु॒रा ए॒ष्व॑ - सावा॑दि॒त्यो दश॑र्.षभा॒-मिन्द्रो॑ व॒लस्य॑ - बार्.हस्प॒त्यं - ॅव॑षट्का॒रो॑ - ऽसौसौ॒रीं॒ - ॅव॑रुण -माश्वि॒न - मिन्द्रं॑ ॅवो॒ नर॒ - एकाद॑श) \newline

\textbf{korvai with starting padams of1, 11, 21 series of pa~jcAtis :-} \newline
(वा॒य॒व्य॑ - माग्ने॒यीं कृ॑ष्णग्री॒वी - म॒सावा॑दि॒त्यो - वा अ॑होरा॒त्राणि॑ - वषट्का॒रः - प्र॑जनयि॒ता - हु॑वे तु॒राणां॒ - पञ्च॑षष्टिः ) \newline

\textbf{first and last padam of first praSnam of kANDam 2:-} \newline
(वा॒य॒व्यं॑ - प्रमो॑षीः) \newline 


॥ हरिः॑ ॐ ॥॥ कृष्ण यजुर्वेदीय तैत्तिरीय संहितायां द्वितीयकाण्डे प्रथमः प्रश्नः समाप्तः ॥ \newline
\pagebreak
2.1.1   Appendix\\2.1.11.1- इन्द्रं॑ ॅवो वि॒श्वत॒स्परी >1, \\इन्द्रं॑ ॅवो वि॒श्वत॒स्परि॒ हवा॑महे॒ जने᳚भ्यः । \\अ॒स्माक॑मस्तु॒ केव॑लः ॥ (appearing in TS 1.6.12.1)\\2.1.11.1- न्द्रं॒ नरो॒>2, \\इन्द्रं॒ नरो॑ ने॒मधि॑ता हवन्ते॒ यत्पार्या॑ यु॒नज॑ते॒ धिय॒स्ताः ।\\शूरो॒ नृषा॑ता॒ शव॑सश्चका॒न आ गोम॑ति व्र॒जे भ॑जा॒ त्वन्नः॑ ॥\\(अप्पॆअरिन्ग् इन्ट्श् 1.6.12.1)\\\\2.1.11.1 - मरु॑तो॒ यद्ध॑ वो दि॒वो>3, \\मरु॑तो॒ यद्ध॑ वो दि॒वः सु॑म्ना॒ यन्तो॒ हवा॑महे । \\आ तू न॒ उप॑ गन्तन ॥ (appearing in TS 1.5.11.4 )\\\\2.1.11.1 - या वः॒ शर्म॑>4\\या वः॒ शर्म॑ शशमा॒नाय॒ सन्ति॑ त्रि॒धातू॑नि दा॒शुषे॑ यच्छ॒ताधि॑ । \\अ॒स्मभ्यं॒ तानि॑ मरुतो॒ वि य॑न्त र॒यिं नो॑ धत्त वृषणः सु॒वीरं᳚ ॥\\(appearing in TS 1.5.11.5 ) \\
\pagebreak
        


\end{document}
