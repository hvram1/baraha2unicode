\documentclass[17pt]{extarticle}
\usepackage{babel}
\usepackage{fontspec}
\usepackage{polyglossia}
\usepackage{extsizes}



\setmainlanguage{sanskrit}
\setotherlanguages{english} %% or other languages
\setlength{\parindent}{0pt}
\pagestyle{myheadings}
\newfontfamily\devanagarifont[Script=Devanagari]{AdishilaVedic}


\newcommand{\VAR}[1]{}
\newcommand{\BLOCK}[1]{}




\begin{document}
\begin{titlepage}
    \begin{center}
 
\begin{sanskrit}
    { \Large
    ॐ नमः परमात्मने, श्री महागणपतये नमः, श्री गुरुभ्यो नमः
ह॒रिः॒ ॐ 
    }
    \\
    \vspace{2.5cm}
    \mbox{ \Huge
    2.5      द्वितीयकाण्डे पञ्चमः प्रश्नः - इष्टिविधानं   }
\end{sanskrit}
\end{center}

\end{titlepage}
\tableofcontents

ॐ नमः परमात्मने, श्री महागणपतये नमः, 
श्री गुरुभ्यो नमः, ह॒रिः॒ ॐ \newline
2.5      द्वितीयकाण्डे पञ्चमः प्रश्नः - इष्टिविधानं \newline

\addcontentsline{toc}{section}{ 2.5      द्वितीयकाण्डे पञ्चमः प्रश्नः - इष्टिविधानं}
\markright{ 2.5      द्वितीयकाण्डे पञ्चमः प्रश्नः - इष्टिविधानं \hfill https://www.vedavms.in \hfill}
\section*{ 2.5      द्वितीयकाण्डे पञ्चमः प्रश्नः - इष्टिविधानं }
                                \textbf{ TS 2.5.1.1} \newline
                  वि॒श्वरू॑प॒ इति॑ वि॒श्व - रू॒पः॒ । वै । त्वा॒ष्ट्रः । पु॒रोहि॑त॒ इति॑ पु॒रः-हि॒तः॒ । दे॒वाना᳚म् । आ॒सी॒त् । स्व॒स्रीयः॑ । असु॑राणाम् । तस्य॑ । त्रीणि॑ । शी॒र्.॒षाणि॑ । आ॒स॒न्न् । सो॒म॒पान॒मिति॑ सोम - पान᳚म् । सु॒रा॒पान॒मिति॑ सुरा - पान᳚म् । अ॒न्नाद॑न॒मित्य॑न्न - अद॑नम् । सः । प्र॒त्यक्ष॒मिति॑ प्रति - अक्ष᳚म् । दे॒वेभ्यः॑ । भा॒गम् । अ॒व॒द॒त् । प॒रोक्ष॒मिति॑ परः - अक्ष᳚म् । असु॑रेभ्यः । सर्व॑स्मै । वै । प्र॒त्यक्ष॒मिति॑ प्रति - अक्ष᳚म् । भा॒गम् । व॒द॒न्ति॒ । यस्मै᳚ । ए॒व । प॒रोक्ष॒मिति॑ परः - अक्ष᳚म् । वद॑न्ति । तस्य॑ । भा॒गः । उ॒दि॒तः । तस्मा᳚त् । इन्द्रः॑ । अ॒बि॒भे॒त् । ई॒दृङ् । वै । रा॒ष्ट्रम् । वीति॑ । प॒र्याव॑र्तय॒तीति॑ परि - आव॑र्तयति । इति॑ । तस्य॑ । वज्र᳚म् । आ॒दायेत्या᳚ - दाय॑ । शी॒र्.॒षाणि॑ । अ॒च्छि॒न॒त् । यत् । सो॒म॒पान॒मिति॑ सोम-पान᳚म् । \textbf{  1 } \newline
                  \newline
                                \textbf{ TS 2.5.1.2} \newline
                  आसी᳚त् । सः । क॒पिञ्ज॑लः । अ॒भ॒व॒त् । यत् । सु॒रा॒पान॒मिति॑ सुरा - पान᳚म् । सः । क॒ल॒विङ्कः॑ । यत् । अ॒न्नाद॑न॒मित्य॑न्न-अद॑नम् । सः । ति॒त्ति॒रिः । तस्य॑ । अ॒ञ्ज॒लिना᳚ । ब्र॒ह्म॒ह॒त्यामिति॑ ब्रह्म-ह॒त्याम् । उपेति॑ । अ॒गृ॒ह्णा॒त् । ताम् । सं॒ॅव॒थ्स॒रमिति॑ सं - व॒थ्स॒रम् । अ॒बि॒भः॒ । तम् । भू॒तानि॑ । अ॒भीति॑ । अ॒क्रो॒श॒न्न् । ब्रह्म॑ह॒न्निति॒ ब्रह्म॑ - ह॒न्न् । इति॑ । सः । पृ॒थि॒वीम् । उपेति॑ ।  अ॒सी॒द॒त् । अ॒स्यै । ब्र॒ह्म॒ह॒त्याया॒ इति॑ ब्रह्म - ह॒त्यायै᳚ । तृती॑यम् । प्रतीति॑ । गृ॒हा॒ण॒ । इति॑ । सा । अ॒ब्र॒वी॒त् । वर᳚म् । वृ॒णै॒ । खा॒तात् । प॒रा॒भ॒वि॒ष्यन्तीति॑ परा - भ॒वि॒ष्यन्ती᳚ । म॒न्ये॒ । ततः॑ । मा । परेति॑ । भू॒व॒म् । इति॑ । पु॒रा । ते॒ । \textbf{  2} \newline
                  \newline
                                \textbf{ TS 2.5.1.3} \newline
                  सं॒ॅव॒थ्स॒रादिति॑ सं - व॒थ्स॒रात् । अपीति॑ । रो॒हा॒त् । इति॑ । अ॒ब्र॒वी॒त् । तस्मा᳚त् । पु॒रा । सं॒ॅव॒थ्स॒रादिति॑ सं - व॒थ्स॒रात् । पृ॒थि॒व्यै । खा॒तम् । अपीति॑ ।  रो॒ह॒ति॒ । वारे॑वृत॒मिति॒ वारे᳚ - वृ॒त॒म् । हि ।  अ॒स्यै॒ । तृती॑यम् । ब्र॒ह्म॒ह॒त्याया॒ इति॑ ब्रह्म - ह॒त्यायै᳚ । प्रतीति॑ । अ॒गृ॒ह्णा॒त् । तत् । स्वकृ॑त॒मिति॒ स्व - कृ॒त॒म् । इरि॑णम् । अ॒भ॒व॒त् । तस्मा᳚त् । आहि॑ताग्नि॒रित्याहि॑त - अ॒ग्निः॒ । श्र॒द्धादे॑व॒ इति॑ श्र॒द्धा - दे॒वः॒ । स्वकृ॑त॒ इति॒ स्व - कृ॒ते॒ । इरि॑णे । न । अवेति॑ । स्ये॒त् । ब्र॒ह्म॒ह॒त्याया॒ इति॑ ब्रह्म - ह॒त्यायै᳚ । हि । ए॒षः । वर्णः॑ । सः । वन॒स्पतीन्॑ । उपेति॑ । अ॒सी॒द॒त् । अ॒स्यै । ब्र॒ह्म॒ह॒त्याया॒ इति॑ ब्रह्म - ह॒त्यायै᳚ । तृती॑यम् । प्रतीति॑ । गृ॒ह्णी॒त॒ । इति॑ । ते । अ॒ब्रु॒व॒न्न् । वर᳚म् । वृ॒णा॒म॒है॒ । वृ॒क्णात् । \textbf{  3} \newline
                  \newline
                                \textbf{ TS 2.5.1.4} \newline
                  प॒रा॒भ॒वि॒ष्यन्त॒ इति॑ परा-भ॒वि॒ष्यन्तः॑ । म॒न्या॒म॒हे॒ । ततः॑ । मा । परेति॑ । भू॒म॒ । इति॑ । आ॒व्रश्च॑ना॒दित्या᳚-व्रश्च॑नात् । वः॒ । भूयाꣳ॑सः । उदिति॑ । ति॒ष्ठा॒न् । इति॑ । अ॒ब्र॒वी॒त् । तस्मा᳚त् । आ॒व्रश्च॑ना॒दित्या᳚ - व्रश्च॑नात् । वृ॒क्षाणा᳚म् । भूयाꣳ॑सः । उदिति॑ । ति॒ष्ठ॒न्ति॒ । वारे॑वृत॒मिति॒ वारे᳚ - वृ॒त॒म् । हि । ए॒षा॒म् । तृती॑यम् । ब्र॒ह्म॒ह॒त्याया॒ इति॑ ब्रह्म - ह॒त्यायै᳚ । प्रतीति॑ । अ॒गृ॒ह्ण॒न्न् । सः । नि॒र्या॒स इति॑ निः - या॒सः । अ॒भ॒व॒त् । तस्मा᳚त् । नि॒र्या॒सस्येति॑ निः - या॒सस्य॑ । न । आ॒श्य᳚म् । ब्र॒ह्म॒ह॒त्याया॒ इति॑ ब्रह्म-ह॒त्यायै᳚ । हि । ए॒षः । वर्णः॑ । अथो॒ इति॑ । खलु॑ । यः । ए॒व । लोहि॑तः । यः । वा॒ । आ॒व्रश्च॑ना॒दित्या᳚ - व्रश्च॑नात् । नि॒र्येष॒तीति॑ निः - येष॑ति । तस्य॑ । न । आ॒श्य᳚म् । \textbf{  4} \newline
                  \newline
                                \textbf{ TS 2.5.1.5} \newline
                  काम᳚म् । अ॒न्यस्य॑ । सः । स्त्री॒षꣳ॒॒सा॒दमिति॑ स्त्री - सꣳ॒॒सा॒दम् । उपेति॑ । अ॒सी॒द॒त् । अ॒स्यै । ब्र॒ह्म॒ह॒त्याया॒ इति॑ ब्रह्म - ह॒त्यायै᳚ । तृती॑यम् । प्रतीति॑ । गृ॒ह्णी॒त॒ । इति॑ । ताः । अ॒ब्रु॒व॒न्न् । वर᳚म् । वृ॒णा॒म॒है॒ । ऋत्वि॑यात् । प्र॒जामिति॑ प्र - जाम् । वि॒न्दा॒म॒है॒ । काम᳚म् । एति॑ । विज॑नितो॒रिति॒ वि - ज॒नि॒तोः॒ । समिति॑ । भ॒वा॒म॒ । इति॑ । तस्मा᳚त् । ऋत्वि॑यात् । स्त्रियः॑ । प्र॒जामिति॑ प्र-जाम् । वि॒न्द॒न्ते॒ । काम᳚म् । एति॑ । विज॑नितो॒रिति॒ वि - ज॒नि॒तोः॒ । समिति॑ । भ॒व॒न्ति॒ । वारे॑वृत॒मिति॒ वारे᳚ - वृ॒त॒म् । हि । आ॒सा॒म् । तृती॑यम् । ब्र॒ह्म॒ह॒त्याया॒ इति॑ ब्रह्म - ह॒त्यायै᳚ । प्रतीति॑ । अ॒गृ॒ह्ण॒न्न् । सा । मल॑वद्वासा॒ इति॒ मल॑वत् - वा॒साः॒ । अ॒भ॒व॒त् । तस्मा᳚त् । मल॑वद्वास॒सेति॒ मल॑वत् - वा॒स॒सा॒ । न । समिति॑ । व॒दे॒त॒ । \textbf{  5} \newline
                  \newline
                                \textbf{ TS 2.5.1.6} \newline
                  न । स॒ह । आ॒सी॒त॒ । न । अ॒स्याः॒ । अन्न᳚म् । अ॒द्या॒त् । ब्र॒ह्म॒ह॒त्याया॒ इति॑ ब्रह्म - ह॒त्यायै᳚ । हि । ए॒षा । वर्ण᳚म् । प्र॒ति॒मुच्येति॑ प्रति-मुच्य॑ । आस्ते᳚ । अथो॒ इति॑ । खलु॑ । आ॒हुः॒ । अ॒भ्यञ्ज॑न॒मित्य॑भि - अञ्ज॑नम् । वाव । स्त्रि॒याः । अन्न᳚म् । अ॒भ्यञ्ज॑न॒मित्य॑भि - अञ्ज॑नम् । ए॒व । न । प्र॒ति॒गृह्य॒मिति॑ प्रति - गृह्य᳚म् । काम᳚म् । अ॒न्यत् । इति॑ । याम् । मल॑वद्वासस॒मिति॒ मल॑वत् - वा॒स॒स॒म् । स॒भंव॒न्तीति॑ सं - भव॑न्ति । यः । ततः॑ । जाय॑ते । सः । अ॒भि॒श॒स्त इत्य॑भि-श॒स्तः । याम् । अर॑ण्ये । तस्यै᳚ । स्ते॒नः । याम् । परा॑चीम् । तस्यै᳚ । ह्री॒त॒मु॒खीति॑ ह्रीत - मु॒खी । अ॒प॒ग॒ल्भ इत्य॑प - ग॒ल्भः । या । स्नाति॑ । तस्याः᳚ । अ॒फ्सित्य॑प् - सु । मारु॑कः । या । \textbf{  6} \newline
                  \newline
                                \textbf{ TS 2.5.1.7} \newline
                  अ॒भ्य॒ङ्क्त इत्य॑भि - अ॒ङ्क्ते । तस्यै᳚ । दु॒श्चर्मेति॑ दुः - चर्मा᳚ । या । प्र॒लि॒खत॒ इति॑ प्र - लि॒खते᳚ । तस्यै᳚ । ख॒ल॒तिः । अ॒प॒मा॒रीत्य॑प - मा॒री । या । आ॒ङ्क्त इत्या᳚ - अ॒ङ्क्ते । तस्यै᳚ । का॒णः । या । द॒तः । धाव॑ते । तस्यै᳚ । श्या॒वद॒न्निति॑ श्या॒व - द॒न्न् । या । न॒खानि॑ । नि॒कृ॒न्तत॒ इति॑ नि - कृ॒न्तते᳚ । तस्यै᳚ । कु॒न॒खी । या । कृ॒णत्ति॑ । तस्यै᳚ । क्ली॒बः । या । रज्जु᳚म् । सृ॒जति॑ । तस्याः᳚ । उ॒द्बन्धु॑क॒ इत्यु॑त् - बन्धु॑कः । या । प॒र्णेन॑ । पिब॑ति । तस्याः᳚ । उ॒न्मादु॑क॒ इत्यु॑त् - मादु॑कः । या । ख॒र्वेण॑ । पिब॑ति । तस्यै᳚ । ख॒र्वः । ति॒स्रः । रात्रीः᳚ । व्र॒तम् । च॒रे॒त् । अ॒ञ्ज॒लिना᳚ । वा॒ । पिबे᳚त् । अख॑र्वेण । वा॒ ( ) । पात्रे॑ण । प्र॒जाया॒ इति॑ प्र - जायै᳚ । गो॒पी॒थाय॑ ॥ \textbf{  7} \newline
                  \newline
                      (यथ् सो॑म॒पानं॑ - ते - वृ॒क्णात्- तस्य॒ नाऽऽश्यं॑ - ॅवदेत॒ -मारु॑को॒ या -ऽख॑र्वेण वा॒ - त्रीणि॑ च)  \textbf{(A1)} \newline \newline
                                \textbf{ TS 2.5.2.1} \newline
                  त्वष्टा᳚ । ह॒तपु॑त्र॒ इति॑ ह॒त-पु॒त्रः॒ । वीन्द्र॒मिति॒ वि - इ॒न्द्र॒म् । सोम᳚म् । एति॑ । अ॒ह॒र॒त् । तस्मिन्न्॑ । इन्द्रः॑ । उ॒प॒ह॒वमित्यु॑प-ह॒वम् । ऐ॒च्छ॒त॒ । तम् । न । उपेति॑ । अ॒ह्व॒य॒त॒ । पु॒त्रम् । मे॒ । अ॒व॒धीः॒ । इति॑ । सः । य॒ज्ञ्॒वे॒श॒समिति॑ यज्ञ् - वे॒श॒सम् । कृ॒त्वा । प्रा॒सहेति॑ प्र - सहा᳚ । सोम᳚म् । अ॒पि॒ब॒त् । तस्य॑ । यत् । अ॒त्यशि॑ष्य॒तेत्य॑ति - अशि॑ष्यत । तत् । त्वष्टा᳚ । आ॒ह॒व॒नीय॒मित्या᳚-ह॒व॒नीय᳚म् । उप॑ । प्रेति॑ । अ॒व॒र्त॒य॒त् । स्वाहा᳚ । इन्द्र॑शत्रु॒रितीन्द्र॑ - श॒त्रुः॒ । व॒र्द्ध॒स्व॒ । इति॑ । यत् । अव॑र्तयत् । तत् । वृ॒त्रस्य॑ । वृ॒त्र॒त्वमिति॑ वृत्र-त्वम् । यत् । अब्र॑वीत् । स्वाहा᳚ । इन्द्र॑शत्रु॒रितीन्द्र॑ - श॒त्रुः॒ । व॒र्द्ध॒स्व॒ । इति॑ । तस्मा᳚त् । अ॒स्य॒ । \textbf{  8} \newline
                  \newline
                                \textbf{ TS 2.5.2.2} \newline
                  इन्द्रः॑ । शत्रुः॑ । अ॒भ॒व॒त् । सः । स॒भंव॒न्निति॑ सं - भवन्न्॑ । अ॒ग्नीषोमा॒वित्य॒ग्नी - सोमौ᳚ । अ॒भि । समिति॑ । अ॒भ॒व॒त् । सः । इ॒षु॒मा॒त्रमि॑षुमात्र॒मिती॑षुमा॒त्रं - इ॒षु॒मा॒त्र॒म् । विष्वङ्॑ । अ॒व॒र्द्ध॒त॒ । सः । इ॒मान् । लो॒कान् । अ॒वृ॒णो॒त् । यत् । इ॒मान् । लो॒कान् । अवृ॑णोत् । तत् । वृ॒त्रस्य॑ । वृ॒त्र॒त्वमिति॑ वृत्र - त्वम् । तस्मा᳚त् । इन्द्रः॑ । अ॒बि॒भे॒त् । सः । प्र॒जाप॑ति॒मिति॑ प्र॒जा-प॒ति॒म् । उपेति॑ । अ॒धा॒व॒त् । शत्रुः॑ । मे॒ । अ॒ज॒नि॒ । इति॑ । तस्मै᳚ । वज्र᳚म् । सि॒क्त्वा । प्रेति॑ । अ॒य॒च्छ॒त् । ए॒तेन॑ । ज॒हि॒ । इति॑ । तेन॑ । अ॒भीति॑ । आ॒य॒त॒ । तौ । अ॒ब्रू॒ता॒म् । अ॒ग्नीषोमा॒वित्य॒ग्नी - सोमौ᳚ । मा । \textbf{  9} \newline
                  \newline
                                \textbf{ TS 2.5.2.3} \newline
                  प्रेति॑ । हाः॒ । आ॒वम् । अ॒न्तः । स्वः॒ । इति॑ । मम॑ । वै । यु॒वम् । स्थः॒ । इति॑ । अ॒ब्र॒वी॒त् । माम् । अ॒भि । एति॑ । इ॒त॒म् । इति॑ । तौ । भा॒ग॒धेय॒मिति॑ भाग - धेय᳚म् । ऐ॒च्छे॒ता॒म् । ताभ्या᳚म् । ए॒तम् । अ॒ग्नी॒षो॒मीय॒मित्य॑ग्नी - सो॒मीय᳚म् । एका॑दशकपाल॒मित्येका॑दश - क॒पा॒ल॒म् । पू॒र्णमा॑स॒ इति॑ पू॒र्ण - मा॒से॒ । प्रेति॑ । अ॒य॒च्छ॒त् । तौ । अ॒ब्रू॒ता॒म् । अ॒भीति॑ । संद॑ष्टा॒विति॒ सं - द॒ष्टौ॒ । वै । स्वः॒ । न । श॒क्नु॒वः॒ । ऐतु॒मित्या - ए॒तु॒म् । इति॑ । सः । इन्द्रः॑ । आ॒त्मनः॑ । शी॒त॒रू॒राविति॑ शीत - रू॒रौ । अ॒ज॒न॒य॒त् । तत् । शी॒त॒रू॒रयो॒रिति॑ शीत - रू॒रयोः᳚ । जन्म॑ । यः । ए॒वं । शी॒त॒रू॒रयो॒रिति॑ शीत - रू॒रयोः᳚ । जन्म॑ । वेद॑ । \textbf{  10} \newline
                  \newline
                                \textbf{ TS 2.5.2.4} \newline
                  न । ए॒न॒म् । शी॒त॒रू॒राविति॑ शीत - रू॒रौ । ह॒तः॒ । ताभ्या᳚म् । ए॒न॒म् । अ॒भीति॑ । अ॒न॒य॒त् । तस्मा᳚त् । ज॒ञ्ज॒भ्यमा॑नात् । अ॒ग्नीषोमा॒वित्य॒ग्नी - सोमौ᳚ । निरिति॑ । अ॒क्रा॒म॒ता॒म् । प्रा॒णा॒पा॒नाविति॑ प्राण - अ॒पा॒नौ । वै । ए॒न॒म् । तत् । अ॒ज॒हि॒ता॒म् । प्रा॒ण इति॑ प्र - अ॒नः । वै । दक्षः॑ । अ॒पा॒न इत्य॑प - अ॒नः । क्रतुः॑ । तस्मा᳚त् । ज॒ञ्ज॒भ्यमा॑नः । ब्रू॒या॒त् । मयि॑ । द॒क्ष॒क्र॒तू इति॑ दक्ष-क्र॒तू । इति॑ । प्रा॒णा॒पा॒नाविति॑ प्राण - अ॒पा॒नौ । ए॒व । आ॒त्मन्न् । ध॒त्ते॒ । सर्व᳚म् । आयुः॑ । ए॒ति॒ । सः । दे॒वताः᳚ । वृ॒त्रात् । नि॒र्॒.हूयेति॑ निः - हूय॑ । वार्त्र॑घ्न॒मिति॒ वार्त्र॑ - घ्न॒म् । ह॒विः । पू॒र्णमा॑स॒ इति॑ पू॒र्ण - मा॒से॒ । निरिति॑ । अ॒व॒प॒त् । घ्नन्ति॑ । वै । ए॒न॒म् । पू॒र्णमा॑स॒ इति॑ पू॒र्ण - मा॒से॒ । एति॑ । \textbf{  11} \newline
                  \newline
                                \textbf{ TS 2.5.2.5} \newline
                  अ॒मा॒वा॒स्या॑या॒मित्य॑मा - वा॒स्या॑याम् । प्या॒य॒य॒न्ति॒ । तस्मा᳚त् । वार्त्र॑घ्नी॒ इति॒ वार्त्र॑ - घ्नी॒ । पू॒र्णमा॑स॒ इति॑ पू॒र्ण - मा॒से॒ । अन्विति॑ । उ॒च्ये॒ते॒ इति॑ । वृध॑न्वती॒ इति॒ वृधन्न्॑ - व॒ती॒ । अ॒मा॒वा॒स्या॑या॒मित्य॑मा - वा॒स्या॑याम् । तत् । सꣳ॒॒स्थाप्येति॑ सं - स्थाप्य॑ । वार्त्र॑घ्न॒मिति॒ वार्त्र॑ - घ्न॒म् । ह॒विः । वज्र᳚म् । आ॒दायेत्या᳚ - दाय॑ । पुनः॑ । अ॒भीति॑ । आ॒य॒त॒ । ते इति॑ । अ॒ब्रू॒ता॒म् । द्यावा॑पृथि॒वी इति॒ द्यावा᳚ - पृ॒थि॒वी । मा । प्रेति॑ । हाः॒ । आ॒वयोः᳚ । वै । श्रि॒तः । इति॑ । ते इति॑ । अ॒ब्रू॒ता॒म् । वर᳚म् । वृ॒णा॒व॒है॒ । नक्ष॑त्रविहि॒तेति॒ नक्ष॑त्र - वि॒हि॒ता॒ । अ॒हम् । असा॑नि । इति॑ । अ॒सौ । अ॒ब्र॒वी॒त् । चि॒त्रवि॑हि॒तेति॑ चि॒त्र - वि॒हि॒ता॒ । अ॒हम् । इति॑ । इ॒यम् । तस्मा᳚त् । नक्ष॑त्रविहि॒तेति॒ नक्ष॑त्र - वि॒हि॒ता॒ । अ॒सौ । चि॒त्रवि॑हि॒तेति॑ चि॒त्र - वि॒हि॒ता॒ । इ॒यम् । यः । ए॒वम् । द्यावा॑पृथि॒व्योरिति॒ द्यावा᳚ - पृ॒थि॒व्योः । \textbf{  12} \newline
                  \newline
                                \textbf{ TS 2.5.2.6} \newline
                  वर᳚म् । वेद॑ । एति॑ । ए॒न॒म् । वरः॑ । ग॒च्छ॒ति॒ । सः । आ॒भ्याम् । ए॒व । प्रसू॑त॒ इति॒ प्र - सू॒तः॒ । इन्द्रः॑ । वृ॒त्रम् । अ॒ह॒न्न् । ते । दे॒वाः । वृ॒त्रम् । ह॒त्वा । अ॒ग्नीषोमा॒वित्य॒ग्नी - सोमौ᳚ । अ॒ब्रु॒व॒न्न् । ह॒व्यम् । नः॒ । व॒ह॒त॒म् । इति॑ । तौ । अ॒ब्रू॒ता॒म् । अप॑तेजसा॒वित्यप॑-ते॒ज॒सौ॒ । वै । त्यौ । वृ॒त्रे । वै । त्ययोः᳚ । तेजः॑ । इति॑ । ते । अ॒ब्रु॒व॒न्न् । कः । इ॒दम् । अच्छ॑ । ए॒ति॒ । इति॑ । गौः । इति॑ । अ॒ब्रु॒व॒न्न् । गौः । वाव । सर्व॑स्य । मि॒त्रम् । इति॑ । सा । अ॒ब्र॒वी॒त् । \textbf{  13} \newline
                  \newline
                                \textbf{ TS 2.5.2.7} \newline
                  वर᳚म् । वृ॒णै॒ । मयि॑ । ए॒व । स॒ता । उ॒भये॑न । भु॒न॒जा॒द्ध्वै॒ । इति॑ । तत् । गौः । एति॑ । अ॒ह॒र॒त् । तस्मा᳚त् । गवि॑ । स॒ता । उ॒भये॑न । भु॒ञ्ज॒ते॒ । ए॒तत् । वै । अ॒ग्नेः । तेजः॑ । यत् । घृ॒तम् । ए॒तत् । सोम॑स्य । यत् । पयः॑ । यः । ए॒वं । अ॒ग्नीषोम॑यो॒रित्य॒ग्नी-सोम॑योः । तेजः॑ । वेद॑ । ते॒ज॒स्वी । ए॒व । भ॒व॒ति॒ । ब्र॒ह्म॒वा॒दिन॒ इति॑ ब्रह्म - वा॒दिनः॑ । व॒द॒न्ति॒ । कि॒दें॒व॒त्य॑मिति॑ किं - दे॒व॒त्य᳚म् । पौ॒र्ण॒मा॒समिति॑ पौर्ण - मा॒सम् । इति॑ । प्रा॒जा॒प॒त्यमिति॑ प्रजा - प॒त्यम् । इति॑ । ब्रू॒या॒त् । तेन॑ । इन्द्र᳚म् । ज्ये॒ष्ठम् । पु॒त्रम् । नि॒रवा॑सायय॒दिति॑ निः-अवा॑साययत् । इति॑ । तस्मा᳚त् ( ) । ज्ये॒ष्ठम् । पु॒त्रम् । धने॑न । नि॒रव॑सायय॒न्तीति॑ निः - अव॑साययन्ति ॥ \textbf{  14} \newline
                  \newline
                      (अ॒स्य॒ - मा - वेदा - ऽऽ - द्यावा॑पृथि॒व्यो - र॑ब्रवी॒ - दिति॒ तस्मा᳚ - च्च॒त्वारि॑ च)  \textbf{(A2)} \newline \newline
                                \textbf{ TS 2.5.3.1} \newline
                  इन्द्र᳚म् । वृ॒त्रम् । ज॒घ्नि॒वाꣳस᳚म् । मृधः॑ । अ॒भि । प्रेति॑ । अ॒वे॒प॒न्त॒ । सः । ए॒तम् । वै॒मृ॒धम् । पू॒र्णमा॑स॒ इति॑ पू॒र्ण - मा॒से॒ । अ॒नु॒नि॒र्वा॒प्य॑मित्य॑नु - नि॒र्वा॒प्य᳚म् । अ॒प॒श्य॒त् । तम् । निरिति॑ । अ॒व॒प॒त् । तेन॑ । वै । सः । मृधः॑ । अपेति॑ । अ॒ह॒त॒ । यत् । वै॒मृ॒धः । पू॒र्णमा॑स॒ इति॑ पू॒र्ण - मा॒से॒ । अ॒नु॒नि॒र्वा॒प्य॑ इत्य॑नु - नि॒र्वा॒प्यः॑ । भव॑ति । मृधः॑ । ए॒व । तेन॑ । यज॑मानः । अपेति॑ । ह॒ते॒ । इन्द्रः॑ । वृ॒त्रम् । ह॒त्वा । दे॒वता॑भिः । च॒ । इ॒न्द्रि॒येण॑ । च॒ । वीति॑ । आ॒र्द्ध्य॒त॒ । सः । ए॒तम् । आ॒ग्ने॒यम् । अ॒ष्टाक॑पाल॒मित्य॒ष्टा - क॒पा॒ल॒म् । अ॒मा॒वा॒स्या॑या॒मित्य॑मा - वा॒स्या॑याम् । अ॒प॒श्य॒त् । ऐ॒न्द्रम् । दधि॑ । \textbf{  15} \newline
                  \newline
                                \textbf{ TS 2.5.3.2} \newline
                  तम् । निरिति॑ । अ॒व॒प॒त् । तेन॑ । वै । सः । दे॒वताः᳚ । च॒ । इ॒न्द्रि॒यम् । च॒ । अवेति॑ । अ॒रु॒न्ध॒ । यत् । आ॒ग्ने॒यः । अ॒ष्टाक॑पाल॒ इत्य॒ष्टा - क॒पा॒लः॒ । अ॒मा॒वा॒स्या॑या॒मित्य॑मा - वा॒स्या॑याम् । भव॑ति । ऐ॒न्द्रम् । दधि॑ । दे॒वताः᳚ । च॒ । ए॒व । तेन॑ । इ॒न्द्रि॒यम् । च॒ । यज॑मानः । अवेति॑ । रु॒न्धे॒ । इन्द्र॑स्य । वृ॒त्रम् । ज॒घ्नुषः॑ । इ॒न्द्रि॒यम् । वी॒र्य᳚म् । पृ॒थि॒वीम् । अनु॑ । वीति॑ । आ॒र्च्छ॒त् । तत् । ओष॑धयः । वी॒रुधः॑ । अ॒भ॒व॒न्न् । सः । प्र॒जाप॑ति॒मिति॑ प्र॒जा - प॒ति॒म् । उपेति॑ । अ॒धा॒व॒त् । वृ॒त्रं । मे॒ । ज॒घ्नुषः॑ । इ॒न्द्रि॒यम् । वी॒र्य᳚म् । \textbf{  16} \newline
                  \newline
                                \textbf{ TS 2.5.3.3} \newline
                  पृ॒थि॒वीम् । अनु॑ । वीति॑ । आ॒र॒त् । तत् । ओष॑धयः । वी॒रुधः॑ । अ॒भू॒व॒न्न् । इति॑ । सः । प्र॒जाप॑ति॒रिति॑ प्र॒जा - प॒तिः॒ । प॒शून् । अ॒ब्र॒वी॒त् । ए॒तत् । अ॒स्मै॒ । समिति॑ । न॒य॒त॒ । इति॑ । तत् । प॒शवः॑ । ओष॑धीभ्य॒ इत्योष॑धि - भ्यः॒ । अधीति॑ । आ॒त्मन्न् । समिति॑ । अ॒न॒य॒न्न् । तत् । प्रतीति॑ । अ॒दु॒ह॒न्न् । यत् । स॒मन॑य॒न्निति॑ सं - अन॑यन्न् । तत् । सा॒नां॒य्यस्येति॑ सां - ना॒य्यस्य॑ । सा॒नां॒य्य॒त्वमिति॑ सांनाय्य - त्वम् । यत् । प्र॒त्यदु॑ह॒न्निति॑ प्रति - अदु॑हन्न् । तत् । प्र॒ति॒धुष॒ इति॑ प्रति - धुषः॑ । प्र॒ति॒धु॒क्त्वमिति॑ प्रतिधुक् - त्वम् । समिति॑ । अ॒नै॒षुः॒ । प्रतीति॑ । अ॒धु॒क्ष॒न्न् । न । तु । मयि॑ । श्र॒य॒ते॒ । इति॑ । अ॒ब्र॒वी॒त् । ए॒तत् । अ॒स्मै॒ । \textbf{  17} \newline
                  \newline
                                \textbf{ TS 2.5.3.4} \newline
                  शृ॒तम् । कु॒रु॒त॒ । इति॑ । अ॒ब्र॒वी॒त् । तत् । अ॒स्मै॒ । शृ॒तम् । अ॒कु॒र्व॒न्न् । इ॒न्द्रि॒यम् । वाव । अ॒स्मि॒न्न् । वी॒र्य᳚म् । तत् । अ॒श्र॒य॒न्न् । तत् । शृ॒तस्य॑ । शृ॒त॒त्वमिति॑ शृत - त्वम् । समिति॑ । अ॒नै॒षुः॒ । प्रतीति॑ । अ॒धु॒क्ष॒न्न् । शृ॒तम् । अ॒क्र॒न्न् । न । तु । मा॒ । धि॒नो॒ति॒ । इति॑ । अ॒ब्र॒वी॒त् । ए॒तत् । अ॒स्मै॒ । दधि॑ । कु॒रु॒त॒ । इति॑ । अ॒ब्र॒वी॒त् । तत् । अ॒स्मै॒ । दधि॑ । अ॒कु॒र्व॒न्न् । तत् । ए॒न॒म् । अ॒धि॒नो॒त् । तत् । द॒द्ध्नः । द॒धि॒त्वमिति॑ दधि - त्वम् । ब्र॒ह्म॒वा॒दिन॒ इति॑ ब्रह्म-वा॒दिनः॑ । व॒द॒न्ति॒ । द॒द्ध्नः । पूर्व॑स्य । अ॒व॒देय॒मित्य॑व - देय᳚म् । \textbf{  18} \newline
                  \newline
                                \textbf{ TS 2.5.3.5} \newline
                  दधि॑ । हि । पूर्व᳚म् । क्रि॒यते᳚ । इति॑ । अना॑दृ॒त्येत्यना᳚ - दृ॒त्य॒ । तत् । शृ॒तस्य॑ । ए॒व । पूर्व॑स्य । अवेति॑ । द्ये॒त् । इ॒न्द्रि॒यम् । ए॒व । अ॒स्मि॒न्न् । वी॒र्य᳚म् । श्रि॒त्वा । द॒द्ध्ना । उ॒परि॑ष्टात् । धि॒नो॒ति॒ । य॒था॒पू॒र्वमिति॑ यथा - पू॒र्वम् । उपेति॑ । ए॒ति॒ । यत् । पू॒तीकैः᳚ । वा॒ । प॒र्ण॒व॒ल्कैरिति॑ पर्ण - व॒ल्कैः । वा॒ । आ॒त॒ञ्च्यादित्या᳚ - त॒ञ्च्यात् । सौ॒म्यम् । तत् । यत् । क्व॑लैः । रा॒क्ष॒सम् । तत् । यत् । त॒ण्डु॒लैः । वै॒श्व॒दे॒वमिति॑ वैश्व - दे॒वम् । तत् । यत् । आ॒तञ्च॑ने॒नेत्या᳚ - तञ्च॑नेन । मा॒नु॒षम् । तत् । यत् । द॒द्ध्ना । तत् । सेन्द्र॒मिति॒ स - इ॒न्द्र॒म् । द॒द्ध्ना । एति॑ । त॒न॒क्ति॒ । \textbf{  19} \newline
                  \newline
                                \textbf{ TS 2.5.3.6} \newline
                  से॒न्द्र॒त्वायेति॑ सेन्द्र - त्वाय॑ । अ॒ग्नि॒हो॒त्रो॒च्छे॒ष॒णमित्य॑ग्निहोत्र - उ॒च्छे॒ष॒णम् । अ॒भ्यात॑न॒क्तीत्य॑भि - आत॑नक्ति । य॒ज्ञ्स्य॑ । संत॑त्या॒ इति॒ सं-त॒त्यै॒ । इन्द्रः॑ । वृ॒त्रम् । ह॒त्वा । परा᳚म् । प॒रा॒वत॒मिति॑ परा - वत᳚म् । अ॒ग॒च्छ॒त् । अपेति॑ । अ॒रा॒ध॒म् । इति॑ । मन्य॑मानः । तम् । दे॒वताः᳚ । प्रैष॒मिति॑ प्र - एष᳚म् । ऐ॒च्छ॒न्न् । सः । अ॒ब्र॒वी॒त् । प्र॒जाप॑ति॒रिति॑ प्र॒जा - प॒तिः॒ । यः । प्र॒थ॒मः । अ॒नु॒वि॒न्दतीत्य॑नु - वि॒न्दति॑ । तस्य॑ । प्र॒थ॒मम् । भा॒ग॒धेय॒मिति॑ भाग - धेय᳚म् । इति॑ । तम् । पि॒तरः॑ । अन्विति॑ । अ॒वि॒न्द॒न्न् । तस्मा᳚त् । पि॒तृभ्य॒ इति॑ पि॒तृ - भ्यः॒ । पू॒र्वे॒द्युः । क्रि॒य॒ते॒ । सः । अ॒मा॒वा॒स्या॑मित्य॑मा - वा॒स्या᳚म् । प्रति॑ । एति॑ । अ॒ग॒च्छ॒त् । तम् । दे॒वाः । अ॒भि । समिति॑ । अ॒ग॒च्छ॒न्त॒ । अ॒मा । वै । नः॒ । \textbf{  20} \newline
                  \newline
                                \textbf{ TS 2.5.3.7} \newline
                  अ॒द्य । वसु॑ । व॒स॒ति॒ । इति॑ । इन्द्रः॑ । हि । दे॒वाना᳚म् । वसु॑ । तत् । अ॒मा॒वा॒स्या॑या॒ इत्य॑मा - वा॒स्या॑याः । अ॒मा॒वा॒स्य॒त्वमित्य॑मावास्य - त्वम् । ब्र॒ह्म॒वा॒दिन॒ इति॑ ब्रह्म-वा॒दिनः॑ । व॒द॒न्ति॒ । किं॒दे॒व॒त्य॑मिति॑ किं - दे॒व॒त्य᳚म् । सा॒नां॒य्यमिति॑ सां - ना॒य्यम् । इति॑ । वै॒श्व॒दे॒वमिति॑ वैश्व - दे॒वम् । इति॑ । ब्रू॒या॒त् । विश्वे᳚ । हि । तत् । दे॒वाः । भा॒ग॒धेय॒मिति॑ भाग - धेय᳚म् । अ॒भीति॑ । स॒मग॑च्छ॒न्तेति॑ सं - अग॑च्छन्त । इति॑ । अथो॒ इति॑ । खलु॑ । ऐ॒न्द्रम् । इति॑ । ए॒व । ब्रू॒या॒त् । इन्द्र᳚म् । वाव । ते । तत् । भि॒ष॒ज्यन्तः॑ । अ॒भि । समिति॑ । अ॒ग॒च्छ॒न्त॒ । इति॑ ॥ \textbf{  21} \newline
                  \newline
                      (दधि॑ - मे ज॒घ्नुष॑ इन्द्रि॒यं ॅवी॒र्य॑ - मित्य॑ब्रवीदे॒तद॑स्मा - अव॒देयं॑ - तनक्ति - नो॒ - द्विच॑त्वारिꣳशच्च)  \textbf{(A3)} \newline \newline
                                \textbf{ TS 2.5.4.1} \newline
                  ब्र॒ह्म॒वा॒दिन॒ इति॑ ब्रह्म - वा॒दिनः॑ । व॒द॒न्ति॒ । सः । तु । वै । द॒र्.॒श॒पू॒र्ण॒मा॒साविति॑ दर्.श - पू॒र्ण॒मा॒सौ । य॒जे॒त॒ । यः । ए॒नौ॒ । सेन्द्रा॒विति॒ स - इ॒न्द्रौ॒ । यजे॑त । इति॑ । वै॒मृ॒धः । पू॒र्णमा॑स॒ इति॑ पू॒र्ण - मा॒से॒ । अ॒नु॒नि॒र्वा॒प्य॑ इत्य॑नु - नि॒र्वा॒प्यः॑ । भ॒व॒ति॒ । तेन॑ । पू॒र्णमा॑स॒ इति॑ पू॒र्ण - मा॒सः॒ । सेन्द्र॒ इति॒ स - इ॒न्द्रः॒ । ऐ॒न्द्रम् । दधि॑ । अ॒मा॒वा॒स्या॑या॒मित्य॑मा - वा॒स्या॑याम् । तेन॑ । अ॒मा॒वा॒स्येत्य॑मा - वा॒स्या᳚ । सेन्द्रेति॒ स - इ॒न्द्रा॒ । यः । ए॒वम् । वि॒द्वान् । द॒र्.॒श॒पू॒र्ण॒मा॒साविति॑ दर्.श - पू॒र्ण॒मा॒सौ । यज॑ते । सेन्द्रा॒विति॒ स - इ॒न्द्रौ॒ । ए॒व । ए॒नौ॒ । य॒ज॒ते॒ । श्वः श्व॒ इति॒ श्वः - श्वः॒ । अ॒स्मै॒ । ई॒जा॒नाय॑ । वसी॑यः । भ॒व॒ति॒ । दे॒वाः । वै । यत् । य॒ज्ञे । अकु॑र्वत । तत् । असु॑राः । अ॒कु॒र्व॒त॒ । ते । दे॒वाः । ए॒ताम् । \textbf{  22} \newline
                  \newline
                                \textbf{ TS 2.5.4.2} \newline
                  इष्टि᳚म् । अ॒प॒श्य॒न्न् । आ॒ग्ना॒वै॒ष्ण॒वमित्या᳚ग्ना - वै॒ष्ण॒वम् । एका॑दशकपाल॒मित्येका॑दश - क॒पा॒ला॒म् । सर॑स्वत्यै । च॒रुम् । सर॑स्वते । च॒रुम् । ताम् । पौ॒र्ण॒मा॒समिति॑ पौर्ण - मा॒सम् । सꣳ॒॒स्थाप्येति॑ सं - स्थाप्य॑ । अनु॑ । निरिति॑ । अ॒व॒प॒न्न् । ततः॑ । दे॒वाः । अभ॑वन्न् । परेति॑ । असु॑राः । यः । भ्रातृ॑व्यवा॒निति॒ भ्रातृ॑व्य - वा॒न् । स्यात् । सः । पौ॒र्ण॒मा॒समिति॑ पौर्ण - मा॒सम् । सꣳ॒॒स्थाप्येति॑ सं-स्थाप्य॑ । ए॒ताम् । इष्टि᳚म् । अनु॑ । निरिति॑ । व॒पे॒त् । पौ॒र्ण॒मा॒सेनेति॑ पौर्ण - मा॒सेन॑ । ए॒व । वज्र᳚म् । भ्रातृ॑व्याय । प्र॒हृत्येति॑ प्र - हृत्य॑ । आ॒ग्ना॒वै॒ष्ण॒वेनेत्या᳚ग्ना-वै॒ष्ण॒वेन॑ । दे॒वताः᳚ । च॒ । य॒ज्ञ्म् । च॒ । भ्रातृ॑व्यस्य । वृ॒ङ्क्ते॒ । मि॒थु॒नान् । प॒शून् । सा॒र॒स्व॒ताभ्या᳚म् । याव॑त् । ए॒व । अ॒स्य॒ । अस्ति॑ । तत् । \textbf{  23} \newline
                  \newline
                                \textbf{ TS 2.5.4.3} \newline
                  सर्व᳚म् । वृ॒ङ्क्ते॒ । पौ॒र्ण॒मा॒सीमिति॑ पौर्ण - मा॒सीम् । ए॒व । य॒जे॒त॒ । भ्रातृ॑व्यवा॒निति॒ भ्रातृ॑व्य - वा॒न् । न । अ॒मा॒वा॒स्या॑मित्य॑मा-वा॒स्या᳚म् । ह॒त्वा । भ्रातृ॑व्यम् । न । एति॑ । प्या॒य॒य॒ति॒ । सा॒क॒प्रं॒स्था॒यीये॒नेति॑ साकं - प्र॒स्था॒यीये॑न । य॒जे॒त॒ । प॒शुका॑म॒ इति॑ प॒शु - का॒मः॒ । यस्मै᳚ । वै । अल्पे॑न । आ॒हर॒न्तीत्या᳚ - हर॑न्ति । न । आ॒त्मना᳚ । तृप्य॑ति । न । अ॒न्यस्मै᳚ । द॒दा॒ति॒ । यस्मै᳚ । म॒ह॒ता । तृप्य॑ति । आ॒त्मना᳚ । ददा॑ति । अ॒न्यस्मै᳚ । म॒ह॒ता । पू॒र्णम् । हो॒त॒व्य᳚म् । तृ॒प्तः । ए॒व । ए॒न॒म् । इन्द्रः॑ । प्र॒जयेति॑ प्र - जया᳚ । प॒शुभि॒रिति॑ प॒शु-भिः॒ । त॒र्प॒य॒ति॒ । दा॒रु॒पा॒त्रेणेति॑ दारु - पा॒त्रेण॑ । जु॒हो॒ति॒ । न । हि । मृ॒न्मय॒मिति॑ मृत् - मय᳚म् । आहु॑ति॒मित्या - हु॒ति॒म् । आ॒न॒शे । औदु॑बंरम् । \textbf{  24} \newline
                  \newline
                                \textbf{ TS 2.5.4.4} \newline
                  भ॒व॒ति॒ । ऊर्क् । वै । उ॒दु॒बंरः॑ । ऊर्क् । प॒शवः॑ । ऊ॒र्जा । ए॒व । अ॒स्मै॒ । ऊर्ज᳚म् । प॒शून् । अवेति॑ । रु॒न्धे॒ । न । अग॑तश्री॒रित्यग॑त - श्रीः॒ । म॒हे॒न्द्रमिति॑ महा - इ॒न्द्रम् । य॒जे॒त॒ । त्रयः॑ । वै । ग॒तश्रि॑य॒ इति॑ ग॒त - श्रि॒यः॒ । शु॒श्रु॒वान् । ग्रा॒म॒णीरिति॑ ग्राम - नीः । रा॒ज॒न्यः॑ । तेषा᳚म् । म॒हे॒न्द्र इति॑ महा-इ॒न्द्रः । दे॒वता᳚ । यः । वै । स्वाम् । दे॒वता᳚म् । अ॒ति॒यज॑त॒ इत्य॑ति - यज॑ते । प्रेति॑ । स्वायै᳚ । दे॒वता॑यै । च्य॒व॒ते॒ । न । परा᳚म् । प्रेति॑ । आ॒प्नो॒ति॒ । पापी॑यान् । भ॒व॒ति॒ । सं॒ॅव॒थ्स॒रमिति॑ सं-व॒थ्स॒रम् । इन्द्र᳚म् । य॒जे॒त॒ । सं॒ॅव॒थ्स॒रमिति॑ सं - व॒थ्स॒रम् । हि । व्र॒तम् । न । अतीति॑ । स्वा । \textbf{  25} \newline
                  \newline
                                \textbf{ TS 2.5.4.5} \newline
                  ए॒व । ए॒न॒म् । दे॒वता᳚ । इ॒ज्यमा॑ना । भूत्यै᳚ । इ॒न्धे॒ । वसी॑यान् । भ॒व॒ति॒ । सं॒ॅव॒थ्स॒रस्येति॑ सं - व॒थ्स॒रस्य॑ । प॒रस्ता᳚त् । अ॒ग्नये᳚ । व्र॒तप॑तय॒ इति॑ व्र॒त - प॒त॒ये॒ । पु॒रो॒डाश᳚म् । अ॒ष्टाक॑पाल॒मित्य॒ष्टा - क॒पा॒ल॒म् । निरिति॑ । व॒पे॒त् । सं॒ॅव॒थ्स॒रमिति॑ सं - व॒थ्स॒रम् । ए॒व । ए॒न॒म् । वृ॒त्रम् । ज॒घ्नि॒वाꣳस᳚म् । अ॒ग्निः । व्र॒तप॑ति॒रिति॑ व्र॒त-प॒तिः॒ । व्र॒तम् । एति॑ । ल॒भं॒य॒ति॒ । ततः॑ । अधीति॑ । काम᳚म् । य॒जे॒त॒ ॥(ए॒तां - त - दौदु॑म्बरꣳ॒॒ - स्वा - \textbf{  26 } \newline
                  \newline
                      (ए॒तां - त - दौदु॑म्बरꣳ॒॒ - स्वा - त्रिꣳ॒॒शच्च॑ ) \textbf{(A4)} \newline \newline
                                \textbf{ TS 2.5.5.1} \newline
                  न । असो॑मया॒जीत्यसो॑म - या॒जी॒ । समिति॑ । न॒ये॒त् । अना॑गत॒मित्यना᳚ - ग॒त॒म् । वै । ए॒तस्य॑ । पयः॑ । यः । असो॑मया॒जीत्यसो॑म - या॒जी॒ । यत् । असो॑मया॒जीत्यसो॑म-या॒जी॒ । स॒नंये॒दिति॑ सं - नये᳚त् । प॒रि॒मो॒ष इति॑ परि - मो॒षः । ए॒व । सः । अनृ॑तम् । क॒रो॒ति॒ । अथो॒ इति॑ । परेति॑ । ए॒व । सि॒च्य॒ते॒ । सो॒म॒या॒जीति॑ सोम - या॒जी । ए॒व । समिति॑ । न॒ये॒त् । पयः॑ । वै । सोमः॑ । पयः॑ । सा॒नां॒य्यमिति॑ सां - ना॒य्यम् । पय॑सा । ए॒व । पयः॑ । आ॒त्मन्न् । ध॒त्ते॒ । वीति॑ । वै । ए॒तम् । प्र॒जयेति॑ प्र - जया᳚ । प॒शुभि॒रिति॑ प॒शु - भिः॒ । अ॒र्द्ध॒य॒ति॒ । व॒र्द्धय॑ति । अ॒स्य॒ । भ्रातृ॑व्यम् । यस्य॑ । ह॒विः । निरु॑प्त॒मिति॒ निः - उ॒प्त॒म् । पु॒रस्ता᳚त् । च॒न्द्रमाः᳚ । \textbf{  27} \newline
                  \newline
                                \textbf{ TS 2.5.5.2} \newline
                  अ॒भीति॑ । उ॒देतीत्यु॑त् - एति॑ । त्रे॒धा । त॒ण्डु॒लान् । वीति॑ । भ॒जे॒त् । ये । म॒द्ध्य॒माः । स्युः । तान् । अ॒ग्नये᳚ । दा॒त्रे । पु॒रो॒डाश᳚म् । अ॒ष्टाक॑पाल॒मित्य॒ष्टा - क॒पा॒ल॒म् । कु॒र्या॒त् । ये । स्थवि॑ष्ठाः । तान् । इन्द्रा॑य । प्र॒दा॒त्र इति॑ प्र - दा॒त्रे । द॒धन्न् । च॒रुम् । ये । अणि॑ष्ठाः । तान् । विष्ण॑वे । शि॒पि॒वि॒ष्टायेति॑ शिपि - वि॒ष्टाय॑ । शृ॒ते । च॒रुम् । अ॒ग्निः । ए॒व । अ॒स्मै॒ । प्र॒जामिति॑ प्र - जाम् । प्र॒ज॒नय॒तीति॑ प्र - ज॒नय॑ति । वृ॒द्धाम् । इन्द्रः॑ । प्रेति॑ । य॒च्छ॒ति॒ । य॒ज्ञ्ः । वै । विष्णुः॑ । प॒शवः॑ । शिपिः॑ । य॒ज्ञे । ए॒व । प॒शुषु॑ । प्रतीति॑ । ति॒ष्ठ॒ति॒ । न । द्वे इति॑ । \textbf{  28} \newline
                  \newline
                                \textbf{ TS 2.5.5.3} \newline
                  य॒जे॒त॒ । यत् । पूर्व॑या । सं॒प्र॒तीति॑ सं - प्र॒ति । यजे॑त । उत्त॑र॒येत्युत् - त॒र॒या॒ । छ॒बंट् । कु॒र्या॒त् । यत् । उत्त॑र॒येत्युत् - त॒र॒या॒ । सं॒प्र॒तीति॑ सं-प्र॒ति । यजे॑त । पूर्व॑या । छ॒बंट् । कु॒र्या॒त् । न । इष्टिः॑ । भव॑ति । न । य॒ज्ञ्ः । तत् । अन्विति॑ । ह्री॒त॒मु॒खीति॑ ह्रीत - मु॒खी । अ॒प॒ग॒ल्भ इत्य॑प - ग॒ल्भः । जा॒य॒ते॒ । एका᳚म् । ए॒व । य॒जे॒त॒ । प्र॒ग॒ल्भ इति॑ प्र - ग॒ल्भः । अ॒स्य॒ । जा॒य॒ते॒ । अना॑दृ॒त्येत्यना᳚-दृ॒त्य॒ । तत् । द्वे इति॑ । ए॒व । य॒जे॒त॒ । य॒ज्ञ्॒मु॒खमिति॑ यज्ञ् - मु॒खम् । ए॒व । पूर्व॑या । आ॒लभ॑त॒ इत्या᳚ - लभ॑ते । यज॑ते । उत्त॑र॒येत्युत् - त॒र॒या॒ । दे॒वताः᳚ । ए॒व । पूर्व॑या । अ॒व॒रु॒न्ध इत्य॑व - रु॒न्धे । इ॒न्द्रि॒यम् । उत्त॑र॒येत्युत् - त॒र॒या॒ । दे॒व॒लो॒कमिति॑ देव - लो॒कम् । ए॒व । \textbf{  29} \newline
                  \newline
                                \textbf{ TS 2.5.5.4} \newline
                  पूर्व॑या । अ॒भि॒जय॒तीत्य॑भि - जय॑ति । म॒नु॒ष्य॒लो॒कमिति॑ मनुष्य - लो॒कम् । उत्त॑र॒येत्युत् - त॒र॒या॒ । भूय॑सः । य॒ज्ञ्॒क्र॒तूनिति॑ यज्ञ् - क्र॒तून् । उपेति॑ । ए॒ति॒ । ए॒षा । वै । सु॒मना॒ इति॑ सु - मनाः᳚ । नाम॑ । इष्टिः॑ । यम् । अ॒द्य । ई॒जा॒नम् । प॒श्चात् । च॒न्द्रमाः᳚ । अ॒भीति॑ । उ॒देतीत्यु॑त् - एति॑ । अ॒स्मिन्न् । ए॒व । अ॒स्मै॒ । लो॒के । अर्द्धु॑कम् । भ॒व॒ति॒ । दा॒क्षा॒य॒ण॒य॒ज्ञेनेति॑ दाक्षायण - य॒ज्ञेन॑ । सु॒व॒र्गका॑म॒ इति॑ सुव॒र्ग - का॒मः॒ । य॒जे॒त॒ । पू॒र्णमा॑स॒ इति॑ पू॒र्ण - मा॒से॒ । समिति॑ । न॒ये॒त् । मै॒त्रा॒व॒रु॒ण्येति॑ मैत्रा - व॒रु॒ण्या । आ॒मिक्ष॑या । अ॒मा॒वा॒स्या॑या॒मित्य॑मा - वा॒स्या॑याम् । य॒जे॒त॒ । पू॒र्णमा॑स॒ इति॑ पू॒र्ण - मा॒से॒ । वै । दे॒वाना᳚म् । सु॒तः । तेषा᳚म् । ए॒तम् । अ॒र्द्ध॒मा॒समित्य॑र्द्ध - मा॒सम् । प्रसु॑त॒ इति॒ प्र - सु॒तः॒ । तेषा᳚म् । मै॒त्रा॒वरु॒णीति॑ मैत्रा - व॒रु॒णी । व॒शा । अ॒मा॒वा॒स्या॑या॒मित्य॑मा - वा॒स्या॑याम् । अ॒नू॒ब॒न्ध्येत्य॑नु - ब॒न्ध्या᳚ । यत् । \textbf{  30} \newline
                  \newline
                                \textbf{ TS 2.5.5.5} \newline
                  पू॒र्वे॒द्युः । यज॑ते । वेदि᳚म् । ए॒व । तत् । क॒रो॒ति॒ । यत् । व॒थ्सान् । अ॒पा॒क॒रोतीत्य॑प-आ॒क॒रोति॑ । स॒दो॒ह॒वि॒र्द्धा॒ने इति॑ सदः-ह॒वि॒र्द्धा॒ने । ए॒व । समिति॑ । मि॒नो॒ति॒ । यत् । यज॑ते । दे॒वैः । ए॒व । सु॒त्याम् । समिति॑ । पा॒द॒य॒ति॒ । सः । ए॒तम् । अ॒र्द्ध॒मा॒समित्य॑र्द्ध - मा॒सम् । स॒ध॒माद॒मिति॑ सध - माद᳚म् । दे॒वैः । सोम᳚म् । पि॒ब॒ति॒ । यत् । मै॒त्रा॒व॒रु॒ण्येति॑ मैत्रा - व॒रु॒ण्या । आ॒मिक्ष॑या । अ॒मा॒वा॒स्या॑या॒मित्य॑मा - वा॒स्या॑याम् । यज॑ते । या । ए॒व । अ॒सौ । दे॒वाना᳚म् । व॒शा । अ॒नू॒ब॒न्ध्येत्य॑नु - ब॒न्ध्या᳚ । सो इति॑ । ए॒व । ए॒षा । ए॒तस्य॑ । सा॒क्षादिति॑ स - अ॒क्षात् । वै । ए॒षः । दे॒वान् । अ॒भ्यारो॑ह॒तीत्य॑भि - आरो॑हति । यः । ए॒षा॒म् । य॒ज्ञ्म् । \textbf{  31} \newline
                  \newline
                                \textbf{ TS 2.5.5.6} \newline
                  अ॒भ्या॒रोह॒तीत्य॑भि - आ॒रोह॑ति । यथा᳚ । खलु॑ । वै । श्रेयान्॑ । अ॒भ्यारू॑ढ॒ इत्य॑भि - आरू॑ढः । का॒मय॑ते । तथा᳚ । क॒रो॒ति॒ । यदि॑ । अ॒व॒विद्ध्य॒तीत्य॑व - विद्ध्य॑ति । पापी॑यान् । भ॒व॒ति॒ । यदि॑ । न । अ॒व॒विद्ध्य॒तीत्य॑व - विद्ध्य॑ति । स॒दृङ्ङिति॑ स - दृङ् । व्या॒वृत्का॑म॒ इति॑ व्या॒वृत् - का॒मः॒ । ए॒तेन॑ । य॒ज्ञेन॑ । य॒जे॒त॒ । क्षु॒रप॑वि॒रिति॑ क्षु॒र - प॒विः॒ । हि । ए॒षः । य॒ज्ञ्ः । ता॒जक् । पुण्यः॑ । वा॒ । भव॑ति । प्रेति॑ । वा॒ । मी॒य॒ते॒ । तस्य॑ । ए॒तत् । व्र॒तम् । न । अनृ॑तम् । व॒दे॒त् । न । माꣳ॒॒सम् । अ॒श्नी॒या॒त् । न । स्त्रिय᳚म् । उपेति॑ । इ॒या॒त् । न । अ॒स्य॒ । पल्पू॑लनेन । वासः॑ । प॒ल्पू॒ल॒ये॒युः॒ ( ) । ए॒तत् । हि । दे॒वाः । सर्व᳚म् । न । कु॒र्वन्ति॑ ॥ \textbf{  32} \newline
                  \newline
                      (च॒न्द्रमा॒ -द्वे -दे॑वलो॒कमे॒व - यद्- य॒ज्ञ्ं- प॑ल्पूलयेयुः॒ -षट् च॑)  \textbf{(A5)} \newline \newline
                                \textbf{ TS 2.5.6.1} \newline
                  ए॒षः । वै । दे॒व॒र॒थ इति॑ देव - र॒थः । यत् । द॒र्.॒श॒पू॒र्ण॒मा॒साविति॑ दर्.श - पू॒र्ण॒मा॒सौ । यः । द॒र्.॒श॒पू॒र्ण॒मा॒साविति॑ दर्.श - पू॒र्ण॒मा॒सौ । इ॒ष्ट्वा । सोमे॑न । यज॑ते । रथ॑स्पष्ट॒ इति॒ रथ॑ - स्प॒ष्टे॒ । ए॒व । अ॒व॒सान॒ इत्य॑व-साने᳚ । वरे᳚ । दे॒वाना᳚म् । अवेति॑ । स्य॒ति॒ । ए॒तानि॑ । वै । अङ्गा॒परूꣳ॒॒षीत्यङ्गा᳚ - परूꣳ॑षि । सं॒ॅव॒थ्स॒रस्येति॑ सं - व॒थ्स॒रस्य॑ । यत् । द॒र्.॒श॒पू॒र्ण॒मा॒साविति॑ दर्.श-पू॒र्ण॒मा॒सौ । यः । ए॒वम् । वि॒द्वान् । द॒र्.॒श॒पू॒र्ण॒मा॒साविति॑ दर्.श - पू॒र्ण॒मा॒सौ । यज॑ते । अङ्गा॒परूꣳ॒॒षीत्यङ्गा᳚ - परूꣳ॑षि । ए॒व । सं॒ॅव॒थ्स॒रस्येति॑ सं - व॒थ्स॒रस्य॑ । प्रतीति॑ । द॒धा॒ति॒ । ए॒ते इति॑ । वै । सं॒ॅव॒थ्स॒रस्येति॑ सं - व॒थ्स॒रस्य॑ । चक्षु॑षी॒ इति॑ । यत् । द॒र्.॒श॒पू॒र्ण॒मा॒साविति॑ दर्.श - पू॒र्ण॒मा॒सौ । यः । ए॒वम् । वि॒द्वान् । द॒र्.॒श॒पू॒र्ण॒मा॒साविति॑ दर्.श - पू॒र्ण॒मा॒सौ । यज॑ते । ताभ्या᳚म् । ए॒व । सु॒व॒र्गमिति॑ सुवः - गम् । लो॒कम् । अन्विति॑ । प॒श्य॒ति॒ । \textbf{  33} \newline
                  \newline
                                \textbf{ TS 2.5.6.2} \newline
                  ए॒षा । वै । दे॒वाना᳚म् । विक्रा᳚न्ति॒रिति॒ वि - क्रा॒न्तिः॒ । यत् । द॒र्.॒श॒पू॒र्ण॒मा॒साविति॑ दर्.श - पू॒र्ण॒मा॒सौ । यः । ए॒वम् । वि॒द्वान् । द॒र्.॒श॒पू॒र्ण॒मा॒साविति॑ दर्.श - पू॒र्ण॒मा॒सौ । यज॑ते । दे॒वाना᳚म् । ए॒व । विक्रा᳚न्ति॒मिति॒ वि - क्रा॒न्ति॒म् । अनु॑ । वीति॑ । क्र॒म॒ते॒ । ए॒षः । वै । दे॒व॒यान॒ इति॑ देव - यानः॑ । पन्थाः᳚ । यत् । द॒र्.॒श॒पू॒र्ण॒मा॒साविति॑ दर्.श - पू॒र्ण॒मा॒सौ । यः । ए॒वम् । वि॒द्वान् । द॒र्.॒श॒पू॒र्ण॒मा॒साविति॑ दर्.श - पू॒र्ण॒मा॒सौ । यज॑ते । यः । ए॒व । दे॒व॒यान॒ इति॑ देव-यानः॑ । पन्थाः᳚ । तम् । स॒मारो॑ह॒तीति॑ सं - आरो॑हति । ए॒तौ । वै । दे॒वाना᳚म् । हरी॒ इति॑ । यत् । द॒र्.॒श॒पू॒र्ण॒मा॒साविति॑ दर्.श - पू॒र्ण॒मा॒सौ । यः । ए॒वम् । वि॒द्वान् । द॒र्.॒श॒पू॒र्ण॒मा॒साविति॑ दर्.श - पू॒र्ण॒मा॒सौ । यज॑ते । यौ । ए॒व । दे॒वाना᳚म् । हरी॒ इति॑ । ताभ्या᳚म् । \textbf{  34} \newline
                  \newline
                                \textbf{ TS 2.5.6.3} \newline
                  ए॒व । ए॒भ्यः॒ । ह॒व्यम् । व॒ह॒ति॒ । ए॒तत् । वै । दे॒वाना᳚म् । आ॒स्य᳚म् । यत् । द॒र्.॒श॒पू॒र्ण॒मा॒साविति॑ दर्.श-पू॒र्ण॒मा॒सौ । यः । ए॒वम् । वि॒द्वान् । द॒र्.॒श॒पू॒र्ण॒मा॒साविति॑ दर्.श - पू॒र्ण॒मा॒सौ । यज॑ते । सा॒क्षादिति॑ स - अ॒क्षात् । ए॒व । दे॒वाना᳚म् । आ॒स्ये᳚ । जु॒हो॒ति॒ । ए॒षः । वै । ह॒वि॒र्द्धा॒नीति॑ हविः - धा॒नी । यः । द॒र्.॒श॒पू॒र्ण॒मा॒स॒या॒जीति॑ दर्.शपूर्णमास - या॒जी । सा॒यंप्रा॑त॒रिति॑ सा॒यं - प्रा॒तः॒ । अ॒ग्नि॒हो॒त्रमित्य॑ग्नि - हो॒त्रम् । जु॒हो॒ति॒ । यज॑ते । द॒र्.॒श॒पू॒र्ण॒मा॒साविति॑ दर्.श - पू॒र्ण॒मा॒सौ । अह॑रह॒रित्यहः॑ - अ॒हः॒ । ह॒वि॒र्द्धा॒निना॒मिति॑ हविः - धा॒निना᳚म् । सु॒तः । यः । ए॒वम् । वि॒द्वान् । द॒र्.॒श॒पू॒र्ण॒मा॒साविति॑ दर्.श - पू॒र्ण॒मा॒सौ । यज॑ते । ह॒वि॒र्द्धा॒नीति॑ हविः - धा॒नी । अ॒स्मि॒ । इति॑ । सर्व᳚म् । ए॒व । अ॒स्य॒ । ब॒र्॒.हि॒ष्य᳚म् । द॒त्तम् । भ॒व॒ति॒ । दे॒वाः । वै । अहः॑ । \textbf{  35} \newline
                  \newline
                                \textbf{ TS 2.5.6.4} \newline
                  य॒ज्ञिय᳚म् । न । अ॒वि॒न्द॒न्न् । ते । द॒र्.॒श॒पू॒र्ण॒मा॒साविति॑ दर्.श-पू॒र्ण॒मा॒सौ । अ॒पु॒न॒न्न् । तौ । वै । ए॒तौ । पू॒तौ । मेद्ध्यौ᳚ । यत् । द॒र्.॒श॒पू॒र्ण॒मा॒साविति॑ दर्.श - पू॒र्ण॒मा॒सौ । यः । ए॒वम् । वि॒द्वान् । द॒र्.॒श॒पू॒र्ण॒मा॒साविति॑ दर्.श-पू॒र्ण॒मा॒सौ । यज॑ते । पू॒तौ । ए॒व । ए॒नौ॒ । मेद्ध्यौ᳚ । य॒ज॒ते॒ । न । अ॒मा॒वा॒स्या॑या॒मित्य॑मा - वा॒स्या॑याम् । च॒ । पौ॒र्ण॒मा॒स्यामिति॑ पौर्ण - मा॒स्याम् । च॒ । स्त्रिय᳚म् । उपेति॑ । इ॒या॒त् । यत् । उ॒पे॒यादित्यु॑प - इ॒यात् । निरि॑न्द्रिय॒ इति॒ निः - इ॒न्द्रि॒यः॒ । स्या॒त् । सोम॑स्य । वै । राज्ञ्ः॑ । अ॒र्द्ध॒मा॒सस्येत्य॑र्द्ध - मा॒सस्य॑ । रात्र॑यः । पत्न॑यः । आ॒स॒न्न् । तासा᳚म् । अ॒मा॒वा॒स्या॑मित्य॑मा-वा॒स्या᳚म् । च॒ । पौ॒र्ण॒मा॒सीमिति॑ पौर्ण-मा॒सीम् । च॒ । न । उपेति॑ । ऐ॒त् । \textbf{  36} \newline
                  \newline
                                \textbf{ TS 2.5.6.5} \newline
                  ते इति॑ । ए॒न॒म् । अ॒भि । समिति॑ । अ॒न॒ह्ये॒ता॒म् । तम् । यक्ष्मः॑ । आ॒र्च्छ॒त् । राजा॑नम् । यक्ष्मः॑ । आ॒र॒त् । इति॑ । तत् । रा॒ज॒य॒क्ष्मस्येति॑ राज - य॒क्ष्मस्य॑ । जन्म॑ । यत् । पापी॑यान् । अभ॑वत् । तत् । पा॒प॒य॒क्ष्मस्येति॑ पाप - य॒क्ष्मस्य॑ । यत् । जा॒याभ्या᳚म् । अवि॑न्दत् । तत् । जा॒येन्य॑स्य । यः । ए॒वम् । ए॒तेषा᳚म् । यक्ष्मा॑णाम् । जन्म॑ । वेद॑ । न । ए॒न॒म् । ए॒ते । यक्ष्माः᳚ । वि॒न्द॒न्ति॒ । सः । ए॒ते इति॑ । ए॒व । न॒म॒स्यन्न् । उपेति॑ । अ॒धा॒व॒त् । ते इति॑ । अ॒ब्रू॒ता॒म् । वर᳚म् । वृ॒णा॒व॒है॒ । आ॒वम् । दे॒वाना᳚म् । भा॒ग॒धे इति॑ भाग - धे । अ॒सा॒व॒ । \textbf{  37} \newline
                  \newline
                                \textbf{ TS 2.5.6.6} \newline
                  आ॒वत् । अधीति॑ । दे॒वाः । इ॒ज्या॒न्तै॒ । इति॑ । तस्मा᳚त् । स॒दृशी॑नाम् । रात्री॑णाम् । अ॒मा॒वा॒स्या॑या॒मित्य॑मा - वा॒स्या॑याम् । च॒ । पौ॒र्ण॒मा॒स्यामिति॑ पौर्ण-मा॒स्याम् । च॒ । दे॒वाः । इ॒ज्य॒न्ते॒ । ए॒ते इति॑ । हि । दे॒वाना᳚म् । भा॒ग॒धे इति॑ भाग - धे । भा॒ग॒धा इति॑ भाग - धाः । अ॒स्मै॒ । म॒नु॒ष्याः᳚ । भ॒व॒न्ति॒ । यः । ए॒वम् । वेद॑ । भू॒तानि॑ । क्षुध᳚म् । अ॒घ्न॒न्न् । स॒द्यः । म॒नु॒ष्याः᳚ । अ॒र्द्ध॒मा॒स इत्य॑र्द्ध - मा॒से । दे॒वाः । मा॒सि । पि॒तरः॑ । सं॒ॅव॒थ्स॒र इति॑ सं - व॒थ्स॒रे । वन॒स्पत॑यः । तस्मा᳚त् । अह॑रह॒रित्यहः॑ - अ॒हः॒ । म॒नु॒ष्याः᳚ । अश॑नम् । इ॒च्छ॒न्ते॒ । अ॒र्द्ध॒मा॒स इत्य॑र्द्ध-मा॒से । दे॒वाः । इ॒ज्य॒न्ते॒ । मा॒सि । पि॒तृभ्य॒ इति॑ पि॒तृ - भ्यः॒ । क्रि॒य॒ते॒ । सं॒ॅव॒थ्स॒र इति॑ सं - व॒थ्स॒रे । वन॒स्पत॑यः । फल᳚म् ( ) । गृ॒ह्ण॒न्ति॒ । यः । ए॒वम् । वेद॑ । हन्ति॑ । क्षुध᳚म् । भ्रातृ॑व्यम् ॥ \textbf{  38 } \newline
                  \newline
                      (प॒श्य॒ति॒ - ताभ्या॒ -मह॑ - रै - दसाव॒ -फलꣳ॑ -स॒प्त च॑)  \textbf{(A6)} \newline \newline
                                \textbf{ TS 2.5.7.1} \newline
                  दे॒वाः । वै । न । ऋ॒चि । न । यजु॑षि । अ॒श्र॒य॒न्त॒ । ते । सामन्न्॑ । ए॒व । अ॒श्र॒य॒न्त॒ । हिम् । क॒रो॒ति॒ । साम॑ । ए॒व । अ॒कः॒ । हिम् । क॒रो॒ति॒ । यत्र॑ । ए॒व । दे॒वाः । अश्र॑यन्त । ततः॑ । ए॒व । ए॒ना॒न् । प्रेति॑ । यु॒ङ्क्ते॒ । हिम् । क॒रो॒ति॒ । वा॒चः । ए॒व । ए॒षः । योगः॑ । हिम् । क॒रो॒ति॒ । प्र॒जा इति॑ प्र - जाः । ए॒व । तत् । यज॑मानः । सृ॒ज॒ते॒ । त्रिः । प्र॒थ॒माम् । अन्विति॑ । आ॒ह॒ । त्रिः । उ॒त्त॒मामित्यु॑त् - त॒माम् । य॒ज्ञ्स्य॑ । ए॒व । तत् । ब॒र्॒.सम् । \textbf{  39} \newline
                  \newline
                                \textbf{ TS 2.5.7.2} \newline
                  न॒ह्य॒ति॒ । अप्र॑स्रꣳसा॒येत्यप्र॑ - स्रꣳ॒॒सा॒य॒ । संत॑त॒मिति॒ सं - त॒त॒म् । अन्विति॑ । आ॒ह॒ । प्रा॒णाना॒मिति॑ प्र - अ॒नाना᳚म् । अ॒न्नाद्य॒स्येत्य॑न्न - अद्य॑स्य । संत॑त्या॒ इति॒ सं - त॒त्यै॒ । अथो॒ इति॑ । रक्ष॑साम् । अप॑हत्या॒ इत्यप॑ - ह॒त्यै॒ । राथ॑न्तरी॒मिति॒ राथं᳚ - त॒री॒म् । प्र॒थ॒माम् । अन्विति॑ । आ॒ह॒ । राथ॑न्तर॒ इति॒ राथं᳚-त॒रः॒ । वै । अ॒यम् । लो॒कः । इ॒मम् । ए॒व । लो॒कम् । अ॒भीति॑ । ज॒य॒ति॒ । त्रिः । वीति॑ । गृ॒ह्णा॒ति॒ । त्रयः॑ । इ॒मे । लो॒काः । इ॒मान् । ए॒व । लो॒कान् । अ॒भीति॑ । ज॒य॒ति॒ । बार्.ह॑तीम् । उ॒त्त॒मामित्यु॑त् - त॒माम् । अन्विति॑ । आ॒ह॒ । बार्.ह॑तः । वै । अ॒सौ । लो॒कः । अ॒मुम् । ए॒व । लो॒कम् । अ॒भीति॑ । ज॒य॒ति॒ । प्रेति॑ । वः॒ । \textbf{  40} \newline
                  \newline
                                \textbf{ TS 2.5.7.3} \newline
                  वाजाः᳚ । इति॑ । अनि॑रुक्ता॒मित्यनिः॑ - उ॒क्ता॒म् । प्रा॒जा॒प॒त्यामिति॑ प्राजा - प॒त्याम् । अन्विति॑ । आ॒ह॒ । य॒ज्ञ्ः । वै । प्र॒जाप॑ति॒रिति॑ प्र॒जा - प॒तिः॒ । य॒ज्ञ्म् । ए॒व । प्र॒जाप॑ति॒मिति॑ प्र॒जा-प॒ति॒म् । एति॑ । र॒भ॒ते॒ । प्रेति॑ । वः॒ । वाजाः᳚ । इति॑ । अन्विति॑ । आ॒ह॒ । अन्न᳚म् । वै । वाजः॑ । अन्न᳚म् । ए॒व । अवेति॑ । रु॒न्धे॒ । प्रेति॑ । वः॒ । वाजाः᳚ । इति॑ । अन्विति॑ । आ॒ह॒ । तस्मा᳚त् । प्रा॒चीन᳚म् । रेतः॑ । धी॒य॒ते॒ । अग्ने᳚ । एति॑ । या॒हि॒ । वी॒तये᳚ । इति॑ । आ॒ह॒ । तस्मा᳚त् । प्र॒तीचीः᳚ । प्र॒जा इति॑ प्र - जाः । जा॒य॒न्ते॒ । प्रेति॑ । वः॒ । वाजाः᳚ । \textbf{  41} \newline
                  \newline
                                \textbf{ TS 2.5.7.4} \newline
                  इति॑ । अन्विति॑ । आ॒ह॒ । मासाः᳚ । वै । वाजाः᳚ । अ॒द्‌र्ध॒मा॒सा इत्य॑द्‌र्ध - मा॒साः । अ॒भिद्य॑व॒ इत्य॒भि - द्य॒वः॒ । दे॒वाः । ह॒विष्म॑न्तः । गौः । घृ॒ताची᳚ । य॒ज्ञ्ः । दे॒वान् । जि॒गा॒ति॒ । यज॑मानः । सु॒म्न॒युरिति॑ सुम्न - युः । इ॒दम् । अ॒सि॒ । इ॒दम् । अ॒सि॒ । इति॑ । ए॒व । य॒ज्ञ्स्य॑ । प्रि॒यम् । धाम॑ । अवेति॑ । रु॒न्धे॒ । यम् । का॒मये॑त । सर्व᳚म् । आयुः॑ । इ॒या॒त् । इति॑ । प्रेति॑ । वः॒ । वाजाः᳚ । इति॑ । तस्य॑ । अ॒नूच्येत्य॑नु - उच्य॑ । अग्ने᳚ । एति॑ । या॒हि॒ । वी॒तये᳚ । इति॑ । संत॑त॒मिति॒ सं - त॒त॒म् । उत्त॑र॒मित्युत् - त॒र॒म् । अ॒द्‌र्ध॒र्चमित्य॑द्‌र्ध - ऋ॒चम् । एति॑ । ल॒भे॒त॒ । \textbf{  42} \newline
                  \newline
                                \textbf{ TS 2.5.7.5} \newline
                  प्रा॒णेनेति॑ प्र - अ॒नेन॑ । ए॒व । अ॒स्य॒ । अ॒पा॒नमित्य॑प - अ॒नम् । दा॒धा॒र॒ । सर्व᳚म् । आयुः॑ । ए॒ति॒ । यः । वै । अ॒र॒त्निम् । सा॒मि॒धे॒नीना॒मिति॑ सां - इ॒धे॒नीना᳚म् । वेद॑ । अ॒र॒त्नौ । ए॒व । भ्रातृ॑व्यम् । कु॒रु॒ते॒ । अ॒द्‌र्ध॒र्चावित्य॑द्‌र्ध - ऋ॒चौ । समिति॑ । द॒धा॒ति॒ । ए॒षः । वै । अ॒र॒त्निः । सा॒मि॒धे॒नीना॒मिति॑ सां - इ॒धे॒नीना᳚म् । यः । ए॒वम् । वेद॑ । अ॒र॒त्नौ । ए॒व । भ्रातृ॑व्यम् । कु॒रु॒ते॒ । ऋषेर्॑. ऋषे॒रित्यृषेः᳚ - ऋ॒षेः॒ । वै । ए॒ताः । निर्मि॑ता॒ इति॒ निः-मि॒ताः॒ । यत् । सा॒मि॒धे॒न्य॑ इति॑ सां - इ॒धे॒न्यः॑ । ताः । यत् । असं॑ॅयुक्ता॒ इत्यसं᳚ - यु॒क्ताः॒ । स्युः । प्र॒जयेति॑ प्र - जया᳚ । प॒शुभि॒रिति॑ प॒शु - भिः॒ । यज॑मानस्य । वीति॑ । ति॒ष्ठे॒र॒न्न् । अ॒द्‌र्ध॒र्चावित्य॑द्‌र्ध - ऋ॒चौ । समिति॑ । द॒धा॒ति॒ । समिति॑ ( ) । यु॒न॒क्ति॒ । ए॒व । ए॒नाः॒ । ताः । अ॒स्मै॒ । संॅयु॑क्ता॒ इति॒ सं - यु॒क्ताः॒ । अव॑रुद्धा॒ इत्यव॑ - रु॒द्धाः॒ । सर्वा᳚म् । आ॒शिष॒मित्या᳚ - शिष᳚म् । दु॒ह्रे॒ ॥ \textbf{  43 } \newline
                  \newline
                      (ब॒र्.सं - ॅवो॑ - जायन्ते॒ प्रवो॒ वाजा॑ - लभेत - दधाति॒ सं - दश॑ च)  \textbf{(A7)} \newline \newline
                                \textbf{ TS 2.5.8.1} \newline
                  अय॑ज्ञ्ः । वै । ए॒षः । यः । अ॒सा॒मा । अग्ने᳚ । एति॑ । या॒हि॒ । वी॒तये᳚ । इति॑ । आ॒ह॒ । र॒थ॒न्त॒रस्येति॑ रथं-त॒रस्य॑ । ए॒षः । वर्णः॑ । तम् । त्वा॒ । स॒मिद्भि॒रिति॑ स॒मित् - भिः॒ । अ॒ङ्गि॒रः॒ । इति॑ । आ॒ह॒ । वा॒म॒दे॒व्यस्येति॑ वाम - दे॒व्यस्य॑ । ए॒षः । वर्णः॑ । बृ॒हत् । अ॒ग्ने॒ । सु॒वीर्य॒मिति॑ सु - वीर्य᳚म् । इति॑ । आ॒ह॒ । बृ॒ह॒तः । ए॒षः । वर्णः॑ । यत् । ए॒तम् । तृ॒चम् । अ॒न्वाहेत्य॑नु - आह॑ । य॒ज्ञ्म् । ए॒व । तत् । साम॑न्वन्त॒मिति॒ सामन्न्॑ - व॒न्त॒म् । क॒रो॒ति॒ । अ॒ग्निः । अ॒मुष्मिन्न्॑ । लो॒के । आसी᳚त् । आ॒दि॒त्यः । अ॒स्मिन्न् । तौ । इ॒मौ । लो॒कौ । अशा᳚न्तौ । \textbf{  44} \newline
                  \newline
                                \textbf{ TS 2.5.8.2} \newline
                  आ॒स्ता॒म् । ते । दे॒वाः । अ॒ब्रु॒व॒न्न् । एति॑ । इ॒त॒ । इ॒मौ । वि । परीति॑ । ऊ॒हा॒म॒ । इति॑ । अग्ने᳚ । एति॑ । या॒हि॒ । वी॒तये᳚ । इति॑ । अ॒स्मिन्न् । लो॒के । अ॒ग्निम् । अ॒द॒धुः॒ । बृ॒हत् । अ॒ग्ने॒ । सु॒वीर्य॒मिति॑ सु - वीर्य᳚म् । इति॑ । अ॒मुष्मिन्न्॑ । लो॒के । आ॒दि॒त्यम् । ततः॑ । वै । इ॒मौ । लो॒कौ । अ॒शा॒म्य॒ता॒म् । यत् । ए॒वम् । अ॒न्वाहेत्य॑नु - आह॑ । अ॒नयोः᳚ । लो॒कयोः᳚ । शान्त्यै᳚ । शाम्य॑तः । अ॒स्मै॒ । इ॒मौ । लो॒कौ । यः । ए॒वम् । वेद॑ । पञ्च॑द॒शेति॒ पञ्च॑ - द॒श॒ । सा॒मि॒धे॒नीरिति॑ सां - इ॒धे॒नीः । अन्विति॑ । आ॒ह॒ । पञ्च॑द॒शेति॒ पञ्च॑ - द॒श॒ । \textbf{  45} \newline
                  \newline
                                \textbf{ TS 2.5.8.3} \newline
                  वै । अ॒र्द्ध॒मा॒सस्येत्य॑र्द्ध - मा॒सस्य॑ । रात्र॑यः । अ॒र्द्ध॒मा॒स॒श इत्य॑र्द्धमास - शः । सं॒ॅव॒थ्स॒र इति॑ सं - व॒थ्स॒रः । आ॒प्य॒ते॒ । तासा᳚म् । त्रीणि॑ । च॒ । श॒तानि॑ । ष॒ष्टिः । च॒ । अ॒क्षरा॑णि । ताव॑तीः । सं॒ॅव॒थ्स॒रस्येति॑ सं - व॒थ्स॒रस्य॑ । रात्र॑यः । अ॒क्ष॒र॒श इत्य॑क्षर-शः । ए॒व । सं॒ॅव॒थ्स॒रमिति॑ सं - व॒थ्स॒रम् । आ॒प्नो॒ति॒ । नृ॒मेध॒ इति॑ नृ - मेधः॑ । च॒ । परु॑च्छेपः । च॒ । ब्र॒ह्म॒वाद्य॒मिति॑ ब्रह्म - वाद्य᳚म् । अ॒व॒दे॒ता॒म् । अ॒स्मिन्न् । दारौ᳚ । आ॒र्द्रे । अ॒ग्निम् । ज॒न॒या॒व॒ । य॒त॒रः । नौ॒ । ब्रह्मी॑यान् । इति॑ । नृ॒मेध॒ इति॑ नृ - मेधः॑ । अ॒भीति॑ । अ॒व॒द॒त् । सः । धू॒मम् । अ॒ज॒न॒य॒त् । परु॑च्छेपः । अ॒भीति॑ । अ॒व॒द॒त् । सः । अ॒ग्निम् । अ॒ज॒न॒य॒त् । ऋषे᳚ । इति॑ । अ॒ब्र॒वी॒त् । \textbf{  46} \newline
                  \newline
                                \textbf{ TS 2.5.8.4} \newline
                  यत् । स॒माव॑त् । वि॒द्व । क॒था । त्वम् । अ॒ग्निम् । अजी॑जनः । न । अ॒हम् । इति॑ । सा॒मि॒धे॒नीना॒मिति॑ सां - इ॒धे॒नीना᳚म् । ए॒व । अ॒हम् । वर्ण᳚म् । वे॒द॒ । इति॑ । अ॒ब्र॒वी॒त् । यत् । घृ॒तव॒दिति॑ घृ॒त-व॒त् । प॒दम् । अ॒नू॒च्यत॒ इत्य॑नु - उ॒च्यते᳚ । सः । आ॒सा॒म् । वर्णः॑ । तम् । त्वा॒ । स॒मिद्भि॒रिति॑ स॒मित् - भिः॒ । अ॒ङ्गि॒रः॒ । इति॑ । आ॒ह॒ । सा॒मि॒धे॒नीष्विति॑ सां - इ॒धे॒नीषु॑ । ए॒व । तत् । ज्योतिः॑ । ज॒न॒य॒ति॒ । स्त्रियः॑ । तेन॑ । यत् । ऋचः॑ । स्त्रियः॑ । तेन॑ । यत् । गा॒य॒त्रियः॑ । स्त्रियः॑ । तेन॑ । यत् । सा॒मि॒धे॒न्य॑ इति॑ सां - इ॒धे॒न्यः॑ । वृष॑ण्वती॒मिति॒ वृषण्॑ - व॒ती॒म् । अन्विति॑ । आ॒ह॒ । \textbf{  47} \newline
                  \newline
                                \textbf{ TS 2.5.8.5} \newline
                  तेन॑ । पुꣳस्व॑तीः । तेन॑ । सेन्द्रा॒ इति॒ स - इ॒न्द्राः॒ । तेन॑ । मि॒थु॒नाः । अ॒ग्निः । दे॒वाना᳚म् । दू॒तः । आसी᳚त् । उ॒शना᳚ । का॒व्यः । असु॑राणाम् । तौ । प्र॒जाप॑ति॒मिति॑ प्र॒जा - प॒ति॒म् । प्र॒श्नम् । ऐ॒ता॒म् । सः । प्र॒जाप॑ति॒रिति॑ प्र॒जा - प॒तिः॒ । अ॒ग्निम् । दू॒तम् । वृ॒णी॒म॒हे॒ । इति॑ । अ॒भीति॑ । प॒र्याव॑र्त॒तेति॑ परि - आव॑र्तत । ततः॑ । दे॒वाः । अभ॑वन्न् । परेति॑ । असु॑राः । यस्य॑ । ए॒वम् । वि॒दुषः॑ । अ॒ग्निम् । दू॒तम् । वृ॒णी॒म॒हे॒ । इति॑ । अ॒न्वाहेत्य॑नु - आह॑ । भव॑ति । आ॒त्मना᳚ । परेति॑ । अ॒स्य॒ । भ्रातृ॑व्यः । भ॒व॒ति॒ । अ॒द्ध्व॒रव॑ती॒मित्य॑द्ध्व॒र-व॒ती॒म् । अन्वति॑ । आ॒ह॒ । भ्रातृ॑व्यम् । ए॒व । ए॒तया᳚ । \textbf{  48} \newline
                  \newline
                                \textbf{ TS 2.5.8.6} \newline
                  ध्व॒र॒ति॒ । शो॒चिष्के॑श॒ इति॑ शो॒चिः - के॒शः॒ । तम् । ई॒म॒हे॒ । इति॑ । आ॒ह॒ । प॒वित्र᳚म् । ए॒व । ए॒तत् । यज॑मानम् । ए॒व । ए॒तया᳚ । प॒व॒य॒ति॒ । समि॑द्ध॒ इति॒ सं - इ॒द्धः॒ । अ॒ग्ने॒ । आ॒हु॒तेत्या᳚ - हु॒त॒ । इति॑ । आ॒ह॒ । प॒रि॒धिमिति॑ परि - धिम् । ए॒व । ए॒तम् । परीति॑ । द॒धा॒ति॒ । अस्क॑न्दाय । यत् । अतः॑ । ऊ॒द्‌र्ध्वम् । अ॒भ्या॒द॒द्ध्यादित्य॑भि - आ॒द॒द्ध्यात् । यथा᳚ । ब॒हिः॒ प॒रि॒धीति॑ बहिः - प॒रि॒धि । स्कन्द॑ति । ता॒दृक् । ए॒व । तत् । त्रयः॑ । वै । अ॒ग्नयः॑ । ह॒व्य॒वाह॑न॒ इति॑ हव्य - वाह॑नः । दे॒वाना᳚म् । क॒व्य॒वाह॑न॒ इति॑ कव्य - वाह॑नः । पि॒तृ॒णाम् । स॒हर॑क्षा॒ इति॑ स॒ह - र॒क्षाः॒ । असु॑राणाम् । ते । ए॒तर्.हि॑ । एति॑ । शꣳ॒॒स॒न्ते॒ । माम् । व॒रि॒ष्य॒ते॒ । माम् । \textbf{  49} \newline
                  \newline
                                \textbf{ TS 2.5.8.7} \newline
                  इति॑ । वृ॒णी॒द्ध्वम् । ह॒व्य॒वाह॑न॒मिति॑ हव्य - वाह॑नम् । इति॑ । आ॒ह॒ । यः । ए॒व । दे॒वाना᳚म् । तम् । वृ॒णी॒ते॒ । आ॒र॒.षे॒यम् । वृ॒णी॒ते॒ । बन्धोः᳚ । ए॒व । न । ए॒ति॒ । अथो॒ इति॑ । संत॑त्या॒ इति॒ सं - त॒त्यै॒ । प॒रस्ता᳚त् । अ॒र्वाचः॑ । वृ॒णी॒ते॒ । तस्मा᳚त् । प॒रस्ता᳚त् । अ॒र्वाञ्चः॑ । म॒नु॒ष्यान्॑ । पि॒तरः॑ । अनु॑ । प्रेति॑ । पि॒प॒ते॒ ॥ \textbf{  50} \newline
                  \newline
                      (अशा᳚न्ता - वाह॒ पञ्च॑दशा - ब्रवी॒ - दन्वा॑है॒ - तया॑ - वरिष्यते॒ मा - मेका॒न्नत्रिꣳ॒॒शच्च॑)  \textbf{(A8)} \newline \newline
                                \textbf{ TS 2.5.9.1} \newline
                  अग्ने᳚ । म॒हान् । अ॒सि॒ । इति॑ । आ॒ह॒ । म॒हान् । हि । ए॒षः । यत् । अ॒ग्निः । ब्रा॒ह्म॒ण॒ । इति॑ । आ॒ह॒ । ब्रा॒ह्म॒णः । हि । ए॒षः । भा॒र॒त॒ । इति॑ । आ॒ह॒ । ए॒षः । हि । दे॒वेभ्यः॑ । ह॒व्यम् । भर॑ति । दे॒वेद्ध॒ इति॑ दे॒व - इ॒द्धः॒ । इति॑ । आ॒ह॒ । दे॒वाः । हि । ए॒तम् । ऐन्ध॑त । मन्वि॑द्ध॒ इति॒ मनु॑ - इ॒द्धः॒ । इति॑ । आ॒ह॒ । मनुः॑ । हि । ए॒तम् । उत्त॑र॒ इत्युत् - त॒रः॒ । दे॒वेभ्यः॑ । ऐन्ध॑ । ऋषि॑ष्टुत॒ इत्यृषि॑ - स्तु॒तः॒ । इति॑ । आ॒ह॒ । ऋष॑यः । हि । ए॒तम् । अस्तु॑वन्न् । विप्रा॑नुमदित॒ इति॒ विप्र॑ - अ॒नु॒म॒दि॒तः॒ । इति॑ । आ॒ह॒ । \textbf{  51} \newline
                  \newline
                                \textbf{ TS 2.5.9.2} \newline
                  विप्राः᳚ । हि । ए॒ते । यत् । शु॒श्रु॒वाꣳसः॑ । क॒वि॒श॒स्त इति॑ कवि - श॒स्तः । इति॑ । आ॒ह॒ । क॒वयः॑ । हि । ए॒ते । यत् । शु॒श्रु॒वाꣳसः॑ । ब्रह्म॑सꣳशित॒ इति॒ ब्रह्म॑ - सꣳ॒॒शि॒तः॒ । इति॑ । आ॒ह॒ । ब्रह्म॑सꣳशित॒ इति॒ ब्रह्म॑ - सꣳ॒॒शि॒तः॒ । हि । ए॒षः । घृ॒ताह॑वन॒ इति॑ घृ॒त - आ॒ह॒व॒नः॒ । इति॑ । आ॒ह॒ । घृ॒ता॒हु॒तिरिति॑ घृत-आ॒हु॒तिः । हि । अ॒स्य॒ । प्रि॒यत॒मेति॑ प्रि॒य - त॒मा॒ । प्र॒णीरिति॑ प्र - नीः । य॒ज्ञाना᳚म् । इति॑ । आ॒ह॒ । प्र॒णीरिति॑ प्र-नीः । हि । ए॒षः । य॒ज्ञाना᳚म् । र॒थीः । अ॒द्ध्व॒राणा᳚म् । इति॑ । आ॒ह॒ । ए॒षः । हि । दे॒व॒र॒थ इति॑ देव - र॒थः । अ॒तूर्तः॑ । होता᳚ । इति॑ । आ॒ह॒ । न । हि । ए॒तम् । कः । च॒न । \textbf{  52} \newline
                  \newline
                                \textbf{ TS 2.5.9.3} \newline
                  तर॑ति । तूर्णिः॑ । ह॒व्य॒वाडिति॑ हव्य - वाट् । इति॑ । आ॒ह॒ । सर्व᳚म् । हि । ए॒षः । तर॑ति । आस्पात्र᳚म् । जु॒हूः । दे॒वाना᳚म् । इति॑ । आ॒ह॒ । जु॒हूः । हि । ए॒षः । दे॒वाना᳚म् । च॒म॒सः । दे॒व॒पान॒ इति॑ देव - पानः॑ । इति॑ । आ॒ह॒ । च॒म॒सः । हि । ए॒षः । दे॒व॒पान॒ इति॑ देव - पानः॑ । अ॒रान् । इ॒व॒ । अ॒ग्ने॒ । ने॒मिः । दे॒वान् । त्वम् । प॒रि॒भूरिति॑ परि-भूः । अ॒सि॒ । इति॑ । आ॒ह॒ । दे॒वान् । हि । ए॒षः । प॒रि॒भूरिति॑ परि - भूः । यत् । ब्रू॒यात् । एति॑ । व॒ह॒ । दे॒वान् । दे॒व॒य॒त इति॑ देव - य॒ते । यज॑मानाय । इति॑ । भ्रातृ॑व्यम् । अ॒स्मै॒ । \textbf{  53} \newline
                  \newline
                                \textbf{ TS 2.5.9.4} \newline
                  ज॒न॒ये॒त् । एति॑ । व॒ह॒ । दे॒वान् । यज॑मानाय । इति॑ । आ॒ह॒ । यज॑मानम् । ए॒व । ए॒तेन॑ । व॒द्‌र्ध॒य॒ति॒ । अ॒ग्निम् । अ॒ग्ने॒ । एति॑ । व॒ह॒ । सोम᳚म् । एति॑ । व॒ह॒ । इति॑ । आ॒ह॒ । दे॒वताः᳚ । ए॒व । तत् । य॒था॒पू॒र्वमिति॑ यथा - पू॒र्वम् । उपेति॑ । ह्व॒य॒ते॒ । एति॑ । च॒ । अ॒ग्ने॒ । दे॒वान् । वह॑ । सु॒यजेति॑ सु - यजा᳚ । च॒ । य॒ज॒ । जा॒त॒वे॒द॒ इति॑ जात-वे॒दः॒ । इति॑ । आ॒ह॒ । अ॒ग्निम् । ए॒व । तत् । समिति॑ । श्य॒ति॒ । सः । अ॒स्य॒ । सꣳशि॑त॒ इति॒ सं - शि॒तः॒ । दे॒वेभ्यः॑ । ह॒व्यम् । व॒ह॒ति॒ । अ॒ग्निः । होता᳚ । \textbf{  54} \newline
                  \newline
                                \textbf{ TS 2.5.9.5} \newline
                  इति॑ । आ॒ह॒ । अ॒ग्निः । वै । दे॒वाना᳚म् । होता᳚ । यः । ए॒व । दे॒वाना᳚म् । होता᳚ । तम् । वृ॒णी॒ते॒ । स्मः । व॒यम् । इति॑ । आ॒ह॒ । आ॒त्मान᳚म् । ए॒व । स॒त्त्वमिति॑ सत् - त्वम् । ग॒म॒य॒ति॒ । सा॒धु । ते॒ । य॒ज॒मा॒न॒ । दे॒वता᳚ । इति॑ । आ॒ह॒ । आ॒शिष॒मित्या᳚ - शिष᳚म् । ए॒व । ए॒ताम् । एति॑ । शा॒स्ते॒ । यत् । ब्रू॒यात् । यः । अ॒ग्निम् । होता॑रम् । अवृ॑थाः । इति॑ । अ॒ग्निना᳚ । उ॒भ॒यतः॑ । यज॑मानम् । परीति॑ । गृ॒ह्णी॒या॒त् । प्र॒मायु॑क॒ इति॑ प्र - मायु॑कः । स्या॒त् । य॒ज॒मा॒न॒दे॒व॒त्येति॑ यजमान - दे॒व॒त्या᳚ । वै । जु॒हूः । भ्रा॒तृ॒व्य॒दे॒व॒त्येति॑ भ्रातृव्य - दे॒व॒त्या᳚ । उ॒प॒भृदित्यु॑प - भृत् । \textbf{  55} \newline
                  \newline
                                \textbf{ TS 2.5.9.6} \newline
                  यत् । द्वे इति॑ । इ॒व॒ । ब्रू॒यात् । भ्रातृ॑व्यम् । अ॒स्मै॒ । ज॒न॒ये॒त् । घृ॒तव॑ती॒मिति॑ घृ॒त - व॒ती॒म् । अ॒द्ध्व॒र्यो॒ इति॑ । स्रुच᳚म् । एति॑ । अ॒स्य॒स्व॒ । इति॑ । आ॒ह॒ । यज॑मानम् । ए॒व । ए॒तेन॑ । व॒द्‌र्ध॒य॒ति॒ । दे॒वा॒युव॒मिति॑ देव - युव᳚म् । इति॑ । आ॒ह॒ । दे॒वान् । हि । ए॒षा । अव॑ति । वि॒श्ववा॑रा॒मिति॑ वि॒श्व-वा॒रा॒म् । इति॑ । आ॒ह॒ । विश्व᳚म् । हि । ए॒षा । अव॑ति । ईडा॑महै । दे॒वान् । ई॒डेन्यान्॑ । न॒म॒स्याम॑ । न॒म॒स्यान्॑ । यजा॑म । य॒ज्ञियान्॑ । इति॑ । आ॒ह॒ । म॒नु॒ष्याः᳚ । वै । ई॒डेन्याः᳚ । पि॒तरः॑ । न॒म॒स्याः᳚ । दे॒वाः । य॒ज्ञियाः᳚ । दे॒वताः᳚ । ए॒व ( ) । तत् । य॒था॒भा॒गमिति॑ यथा - भा॒गम् । य॒ज॒ति॒ ॥ \textbf{  56 } \newline
                  \newline
                      (विप्रा॑नुमदित॒ इत्या॑ह - च॒ना - ऽस्मै॒ - होतो॑ - प॒भृद् - दे॒वता॑ ए॒व - त्रीणि॑ च)  \textbf{(A9)} \newline \newline
                                \textbf{ TS 2.5.10.1} \newline
                  त्रीन् । तृ॒चान् । अन्विति॑ । ब्रू॒या॒त् । रा॒ज॒न्य॑स्य । त्रयः॑ । वै । अ॒न्ये । रा॒ज॒न्या᳚त् । पुरु॑षाः । ब्रा॒ह्म॒णः । वैश्यः॑ । शू॒द्रः । तान् । ए॒व । अ॒स्मै॒ । अनु॑का॒नित्यनु॑ - का॒न् । क॒रो॒ति॒ । पञ्च॑द॒शेति॒ पञ्च॑ - द॒श॒ । अन्विति॑ । ब्रू॒या॒त् । रा॒ज॒न्य॑स्य । प॒ञ्च॒द॒श इति॑ पञ्च-द॒शः । वै । रा॒ज॒न्यः॑ । स्वे । ए॒व । ए॒न॒म् । स्तोमे᳚ । प्रतीति॑ । स्था॒प॒य॒ति॒ । त्रि॒ष्टुभा᳚ । परीति॑ । द॒द्ध्या॒त् । इ॒न्द्रि॒यम् । वै । त्रि॒ष्टुक् । इ॒न्द्रि॒यका॑म॒ इती᳚न्द्रि॒य - का॒मः॒ । खलु॑ । वै । रा॒ज॒न्यः॑ । य॒ज॒ते॒ । त्रि॒ष्टुभा᳚ । ए॒व । अ॒स्मै॒ । इ॒न्द्रि॒यम् । परीति॑ । गृ॒ह्णा॒ति॒ । यदि॑ । का॒मये॑त । \textbf{  57} \newline
                  \newline
                                \textbf{ TS 2.5.10.2} \newline
                  ब्र॒ह्म॒व॒र्च॒समिति॑ ब्रह्म-व॒र्च॒सम् । अ॒स्तु॒ । इति॑ । गा॒य॒त्रि॒या । परीति॑ । द॒द्ध्या॒त् । ब्र॒ह्म॒व॒र्च॒समिति॑ ब्रह्म - व॒र्च॒सम् । वै । गा॒य॒त्री । ब्र॒ह्म॒व॒र्च॒समिति॑ ब्रह्म - व॒र्च॒सम् । ए॒व । भ॒व॒ति॒ । स॒प्तद॒शेति॑ स॒प्त - द॒श॒ । अन्विति॑ । ब्रू॒या॒त् । वैश्य॑स्य । स॒प्त॒द॒श इति॑ सप्त - द॒शः । वै । वैश्यः॑ । स्वे । ए॒व । ए॒न॒म् । स्तोमे᳚ । प्रतीति॑ । स्था॒प॒य॒ति॒ । जग॑त्या । परीति॑ । द॒द्ध्या॒त् । जाग॑ताः । वै । प॒शवः॑ । प॒शुका॑म॒ इति॑ प॒शु - का॒मः॒ । खलु॑ । वै । वैश्यः॑ । य॒ज॒ते॒ । जग॑त्या । ए॒व । अ॒स्मै॒ । प॒शून् । परीति॑ । गृ॒ह्णा॒ति । एक॑विꣳशति॒मित्येक॑ - विꣳ॒॒श॒ति॒म् । अन्विति॑ । ब्रू॒या॒त् । प्र॒ति॒ष्ठाका॑म॒स्येति॑ प्रति॒ष्ठा - का॒म॒स्य॒ । ए॒क॒विꣳ॒॒श इत्ये॑क-विꣳ॒॒शः । स्तोमा॑नाम् । प्र॒ति॒ष्ठेति॑ प्रति - स्था । प्रति॑ष्ठित्या॒ इति॒ प्रति॑ - स्थि॒त्यै॒ । \textbf{  58} \newline
                  \newline
                                \textbf{ TS 2.5.10.3} \newline
                  चतु॑र्विꣳशति॒मिति॒ चतुः॑ - विꣳ॒॒श॒ति॒म् । अन्विति॑ । ब्रू॒या॒त् । ब्र॒ह्म॒व॒र्च॒सका॑म॒स्येति॑ ब्रह्मवर्च॒स - का॒म॒स्य॒ । चतु॑र्विꣳशत्यक्ष॒रेति॒ चतु॑र्विꣳशति - अ॒क्ष॒रा॒ । गा॒य॒त्री । गा॒य॒त्री । ब्र॒ह्म॒व॒र्च॒समिति॑ ब्रह्म - व॒र्च॒सम् । गा॒य॒त्रि॒या । ए॒व । अ॒स्मै॒ । ब्र॒ह्म॒व॒र्च॒समिति॑ ब्रह्म - व॒र्च॒सम् । अवेति॑ । रु॒न्धे॒ । त्रिꣳ॒॒शत᳚म् । अन्विति॑ । ब्रू॒या॒त् । अन्न॑काम॒स्येत्यन्न॑ - का॒म॒स्य॒ । त्रिꣳ॒॒शद॑क्ष॒रेति॑ त्रिꣳ॒॒शत् - अ॒क्ष॒रा॒ । वि॒राडिति॑ वि - राट् । अन्न᳚म् । वि॒राडिति॑ वि - राट् । वि॒राजेति॑ वि - राजा᳚ । ए॒व । अ॒स्मै॒ । अ॒न्नाद्य॒मित्य॑न्न - अद्य᳚म् । अवेति॑ । रु॒न्धे॒ । द्वात्रिꣳ॑शतम् । अन्विति॑ । ब्रू॒या॒त् । प्र॒ति॒ष्ठाका॑म॒स्येति॑ प्रति॒ष्ठा - का॒म॒स्य॒ । द्वात्रिꣳ॑शदक्ष॒रेति॒ द्वात्रिꣳ॑शत् - अ॒क्ष॒रा॒ । अ॒नु॒ष्टुगित्य॑नु - स्तुक् । अ॒नु॒ष्टुबित्य॑नु - स्तुप् । छन्द॑साम् । प्र॒ति॒ष्ठेति॑ प्रति - स्था । प्रति॑ष्ठित्या॒ इति॒ प्रति॑ - स्थि॒त्यै॒ । षट्त्रिꣳ॑शत॒मिति॒ षट्-त्रिꣳ॒॒श॒त॒म् । अन्विति॑ । ब्रू॒या॒त् । प॒शुका॑म॒स्येति॑ प॒शु - का॒म॒स्य॒ । षट्त्रिꣳ॑शदक्ष॒रेति॒ षट्त्रिꣳ॑शत् - अ॒क्ष॒रा॒ । बृ॒ह॒ती । बार्.ह॑ताः । प॒शवः॑ । बृ॒ह॒त्या । ए॒व । अ॒स्मै॒ । प॒शून् । \textbf{  59} \newline
                  \newline
                                \textbf{ TS 2.5.10.4} \newline
                  अवेति॑ । रु॒न्धे॒ । चतु॑श्चत्वारिꣳशत॒मिति॒ चतुः॑ - च॒त्वा॒रिꣳ॒॒श॒त॒म् । अन्विति॑ । ब्रू॒या॒त् । इ॒न्द्रि॒यका॑म॒स्येती᳚न्द्रि॒य - का॒म॒स्य॒ । चतु॑श्चत्वारिꣳशदक्ष॒रेति॒ चतु॑श्चत्वारिꣳशत् - अ॒क्ष॒रा॒ । त्रि॒ष्टुक् । इ॒न्द्रि॒यम् । त्रि॒ष्टुप् । त्रि॒ष्टुभा᳚ । ए॒व । अ॒स्मै॒ । इ॒न्द्रि॒यम् । अवेति॑ । रु॒न्धे॒ । अ॒ष्टाच॑त्वारिꣳशत॒मित्य॒ष्टा - च॒त्वा॒रिꣳ॒॒श॒त॒म् । अन्विति॑ । ब्रू॒या॒त् । प॒शुका॑म॒स्येति॑ प॒शु - का॒म॒स्य॒ । अ॒ष्टाच॑त्वारिꣳशदक्ष॒रेत्य॒ष्टाच॑त्वारिꣳशत् - अ॒क्ष॒रा॒ । जग॑ती । जाग॑ताः । प॒शवः॑ । जग॑त्या । ए॒व । अ॒स्मै॒ । प॒शून् । अवेति॑ । रु॒न्धे॒ । सर्वा॑णि । छन्दाꣳ॑सि । अन्विति॑ । ब्रू॒या॒त् । ब॒हु॒या॒जिन॒ इति॑ बहु - या॒जिनः॑ । सर्वा॑णि । वै । ए॒तस्य॑ । छन्दाꣳ॑सि । अव॑रुद्धा॒नीत्यव॑ - रु॒द्धा॒नि॒ । यः । ब॒हु॒या॒जीति॑ बहु - या॒जी । अप॑रिमित॒मित्यप॑रि - मि॒त॒म् । अन्विति॑ । ब्रू॒या॒त् । अप॑रिमित॒स्येत्यप॑रि - मि॒त॒स्य॒ । अव॑रुद्ध्या॒ इत्यव॑ - रु॒द्ध्यै॒ ॥ \textbf{  60} \newline
                  \newline
                      (का॒मये॑त॒ - प्रति॑ष्ठित्यै - प॒शून्थ् - स॒प्तच॑त्वारिꣳशच्च)  \textbf{(A10)} \newline \newline
                                \textbf{ TS 2.5.11.1} \newline
                  निवी॑त॒मिति॒ नि - वी॒त॒म् । म॒नु॒ष्या॑णाम् । प्रा॒ची॒ना॒वी॒तमिति॑ प्राचीन - आ॒वी॒तम् । पि॒तृ॒णाम् । उप॑वीत॒मित्युप॑ - वी॒त॒म् । दे॒वाना᳚म् । उपेति॑ । व्य॒य॒ते॒ । दे॒व॒ल॒क्ष्ममिति॑ देव - ल॒क्ष्मम् । ए॒व । तत् । कु॒रु॒ते॒ । तिष्ठन्न्॑ । अन्विति॑ । आ॒ह॒ । तिष्ठन्न्॑ । हि । आश्रु॑ततर॒मित्याश्रु॑त - त॒र॒म् । वद॑ति । तिष्ठन्न्॑ । अन्विति॑ । आ॒ह॒ । सु॒व॒र्गस्येति॑ सुवः - गस्य॑ । लो॒कस्य॑ । अ॒भिजि॑त्या॒ इत्य॒भि - जि॒त्यै॒ । आसी॑नः । य॒ज॒ति॒ । अ॒स्मिन्न् । ए॒व । लो॒के । प्रतीति॑ । ति॒ष्ठ॒ति॒ । यत् । क्रौ॒ञ्चम् । अ॒न्वाहेत्य॑नु-आह॑ । आ॒सु॒रम् । तत् । यत् । म॒न्द्रम् । मा॒नु॒षम् । तत् । यत् । अ॒न्त॒रा । तत् । सदे॑व॒मिति॒ स - दे॒व॒म् । अ॒न्त॒रा । अ॒नूच्य॒मित्य॑नु - उच्य᳚म् । स॒दे॒व॒त्वायेति॑ सदेव - त्वाय॑ । वि॒द्वाꣳसः॑ । वै । \textbf{  61} \newline
                  \newline
                                \textbf{ TS 2.5.11.2} \newline
                  पु॒रा । होता॑रः । अ॒भू॒व॒न्न् । तस्मा᳚त् । विधृ॑ता॒ इति॒ वि - धृ॒ताः॒ । अद्ध्वा॑नः । अभू॑वन्न् । न । पन्था॑नः । समिति॑ । अ॒रु॒क्ष॒न्न् । अ॒न्त॒र्वे॒दीत्य॑न्तः - वे॒दि । अ॒न्यः । पादः॑ । भव॑ति । ब॒हि॒र्वे॒दीति॑ बहिः - वे॒दि । अ॒न्यः । अथ॑ । अन्विति॑ । आ॒ह॒ । अद्ध्व॑नाम् । विधृ॑त्या॒ इति॒ वि - धृ॒त्यै॒ । प॒थाम् । असꣳ॑रोहा॒येत्यसं᳚ - रो॒हा॒य॒ । अथो॒ इति॑ । भू॒तम् । च॒ । ए॒व । भ॒वि॒ष्यत् । च॒ । अवेति॑ । रु॒न्धे॒ । अथो॒ इति॑ । परि॑मित॒मिति॒ परि॑ - मि॒त॒म् । च॒ । ए॒व । अप॑रिमित॒मित्यप॑रि - मि॒त॒म् । च॒ । अवेति॑ । रु॒न्धे॒ । अथो॒ इति॑ । ग्रा॒म्यान् । च॒ । ए॒व । प॒शून् । आ॒र॒ण्यान् । च॒ । अवेति॑ । रु॒न्धे॒ । अथो॒ इति॑ । \textbf{  62} \newline
                  \newline
                                \textbf{ TS 2.5.11.3} \newline
                  दे॒व॒लो॒कमिति॑ देव - लो॒कम् । च॒ । ए॒व । म॒नु॒ष्य॒लो॒कमिति॑ मनुष्य - लो॒कम् । च॒ । अ॒भीति॑ । ज॒य॒ति॒ । दे॒वाः । वै । सा॒मि॒धे॒नीरिति॑ सां - इ॒धे॒नीः । अ॒नूच्येत्य॑नु - उच्य॑ । य॒ज्ञ्म् । न । अन्विति॑ । अ॒प॒श्य॒न्न् । सः । प्र॒जाप॑ति॒रिति॑ प्र॒जा - प॒तिः॒ । तू॒ष्णीम् । आ॒घा॒रमित्या᳚-घा॒रम् । एति॑ । अ॒घा॒रय॒त् । ततः॑ । वै । दे॒वाः । य॒ज्ञ्म् । अन्विति॑ । अ॒प॒श्य॒न्न् । यत् । तू॒ष्णीम् । आ॒घा॒रमित्या᳚ - घा॒रम् । आ॒घा॒रय॒तीत्या᳚ - घा॒रय॑ति । य॒ज्ञ्स्य॑ । अनु॑ख्यात्या॒ इत्यनु॑ - ख्या॒त्यै॒ । अथो॒ इति॑ । सा॒मि॒धे॒नीरिति॑ सां - इ॒धे॒नीः । ए॒व । अ॒भीति॑ । अ॒न॒क्ति॒ । अलू᳚क्षः । भ॒व॒ति॒ । यः । ए॒वम् । वेद॑ । अथो॒ इति॑ । त॒र्पय॑ति । ए॒व । ए॒नाः॒ । तृप्य॑ति । प्र॒जयेति॑ प्र - जया᳚ । प॒शुभि॒रिति॑ प॒शु - भिः॒ । \textbf{  63} \newline
                  \newline
                                \textbf{ TS 2.5.11.4} \newline
                  यः । ए॒वम् । वेद॑ । यत् । एक॑या । आ॒घा॒रये॒दित्या᳚ - घा॒रये᳚त् । एका᳚म् । प्री॒णी॒या॒त् । यत् । द्वाभ्या᳚म् । द्वे इति॑ । प्री॒णी॒या॒त् । यत् । ति॒सृभि॒रिति॑ ति॒सृ - भिः॒ । अतीति॑ । तत् । रे॒च॒ये॒त् । मन॑सा । एति॑ । घा॒र॒य॒ति॒ । मन॑सा । हि । अना᳚प्तम् । आ॒प्यते᳚ । ति॒र्यञ्च᳚म् । एति॑ । घा॒र॒य॒ति॒ । अछ॑बंट्कार॒मित्यछ॑बंट् - का॒र॒म् । वाक् । च॒ । मनः॑ । च॒ । आ॒र्ती॒ये॒ता॒म् । अ॒हम् । दे॒वेभ्यः॑ । ह॒व्यम् । व॒हा॒मि॒ । इति॑ । वाक् । अ॒ब्र॒वी॒त् । अ॒हम् । दे॒वेभ्यः॑ । इति॑ । मनः॑ । तौ । प्र॒जाप॑ति॒मिति॑ प्र॒जा - प॒ति॒म् । प्र॒श्नम् । ऐ॒ता॒म् । सः । अ॒ब्र॒वी॒त् । \textbf{  64} \newline
                  \newline
                                \textbf{ TS 2.5.11.5} \newline
                  प्र॒जाप॑ति॒रिति॑ प्र॒जा - प॒तिः॒ । दू॒तीः । ए॒व । त्वम् । मन॑सः । अ॒सि॒ । यत् । हि । मन॑सा । ध्याय॑ति । तत् । वा॒चा । वद॑ति । इति॑ । तत् । खलु॑ । तुभ्य᳚म् । न । वा॒चा । जु॒ह॒व॒न्न् । इति॑ । अ॒ब्र॒वी॒त् । तस्मा᳚त् । मन॑सा । प्र॒जाप॑तय॒ इति॑ प्र॒जा - प॒त॒ये॒ । जु॒ह्व॒ति॒ । मनः॑ । इ॒व॒ । हि । प्र॒जाप॑ति॒रिति॑ प्र॒जा - प॒तिः॒ । प्र॒जाप॑ते॒रिति॑ प्र॒जा - प॒तेः॒ । आप्त्यै᳚ । प॒रि॒धीनिति॑ परि - धीन् । समिति॑ । मा॒र्ष्टि॒ । पु॒नाति॑ । ए॒व । ए॒ना॒न् । त्रिः । म॒द्ध्य॒मम् । त्रयः॑ । वै । प्रा॒णा इति॑ प्र - अ॒नाः । प्रा॒णानिति॑ प्र - अ॒नान् । ए॒व । अ॒भिति॑ । ज॒य॒ति॒ । त्रिः । द॒क्षि॒णा॒द्‌र्ध्य॑मिति॑ दक्षिण - अ॒द्‌र्ध्य᳚म् । त्रयः॑ । \textbf{  65} \newline
                  \newline
                                \textbf{ TS 2.5.11.6} \newline
                  इ॒मे । लो॒काः । इ॒मान् । ए॒व । लो॒कान् । अ॒भीति॑ । ज॒य॒ति॒ । त्रिः । उ॒त्त॒रा॒द्‌र्ध्य॑मित्यु॑त्तर - अ॒द्‌र्ध्य᳚म् । त्रयः॑ । वै । दे॒व॒याना॒ इति॑ देव - यानाः᳚ । पन्था॑नः । तान् । ए॒व । अ॒भीति॑ । ज॒य॒ति॒ । त्रिः । उपेति॑ । वा॒ज॒य॒ति॒ । त्रयः॑ । वै । दे॒व॒लो॒का इति॑ देव - लो॒काः । दे॒व॒लो॒कानिति॑ देव - लो॒कान् । ए॒व । अ॒भीति॑ । ज॒य॒ति॒ । द्वाद॑श । समिति॑ । प॒द्य॒न्ते॒ । द्वाद॑श । मासाः᳚ । सं॒ॅव॒थ्स॒र इति॑ सं - व॒थ्स॒रः । सं॒ॅव॒थ्स॒रमिति॑ सं - व॒थ्स॒रम् । ए॒व । प्री॒णा॒ति॒ । अथो॒ इति॑ । सं॒ॅव॒थ्स॒रमिति॑ सं - व॒थ्स॒रम् । ए॒व । अ॒स्मै॒ । उपेति॑ । द॒धा॒ति॒ । सु॒व॒र्गस्येति॑ सुवः - गस्य॑ । लो॒कस्य॑ । सम॑ष्ट्या॒ इति॒ सं - अ॒ष्ट्यै॒ । आ॒घा॒रमित्या᳚ - घा॒रम् । एति॑ । घा॒र॒य॒ति॒ । ति॒रः । इ॒व॒ । \textbf{  66} \newline
                  \newline
                                \textbf{ TS 2.5.11.7} \newline
                  वै । सु॒व॒र्ग इति॑ सुवः - गः । लो॒कः । सु॒व॒र्गमिति॑ सुवः - गम् । ए॒व । अ॒स्मै॒ । लो॒कम् । प्रेति॑ । रो॒च॒य॒ति॒ । ऋ॒जुम् । एति॑ । घा॒र॒य॒ति॒ । ऋ॒जुः । इ॒व॒ । हि । प्रा॒ण इति॑ प्र - अ॒नः । संत॑त॒मिति॒ सं - त॒त॒म् । एति॑ । घा॒र॒य॒ति॒ । प्रा॒णाना॒मिति॑ प्र - अ॒नाना᳚म् । अ॒न्नाद्य॒स्येत्य॑न्न - अद्य॑स्य । संत॑त्या॒ इति॒ सं - त॒त्यै॒ । अथो॒ इति॑ । रक्ष॑साम् । अप॑हत्या॒ इत्यप॑ - ह॒त्यै॒ । यम् । का॒मये॑त । प्र॒मायु॑क॒ इति॑ प्र - मायु॑कः । स्या॒त् । इति॑ । जि॒ह्मम् । तस्य॑ । एति॑ । घा॒र॒ये॒त् । प्रा॒णमिति॑ प्र-अ॒नम् । ए॒व । अ॒स्मा॒त् । जि॒ह्मम् । न॒य॒ति॒ । ता॒जक् । प्रेति॑ । मी॒य॒ते॒ । शिरः॑ । वै । ए॒तत् । य॒ज्ञ्स्य॑ । यत् । आ॒घा॒र इत्या᳚ - घा॒रः । आ॒त्मा । ध्रु॒वा । \textbf{  67} \newline
                  \newline
                                \textbf{ TS 2.5.11.8} \newline
                  आ॒घा॒रमित्या᳚ - घा॒रम् । आ॒घार्येत्या᳚ - घार्य॑ । ध्रु॒वाम् । समिति॑ । अ॒न॒क्ति॒ । आ॒त्मन्न् । ए॒व । य॒ज्ञ्स्य॑ । शिरः॑ । प्रतीति॑ । द॒धा॒ति॒ । अ॒ग्निः । दे॒वाना᳚म् । दू॒तः । आसी᳚त् । दैव्यः॑ । असु॑राणाम् । तौ । प्र॒जाप॑ति॒मिति॑ प्र॒जा - प॒ति॒म् । प्र॒श्नम् । ऐ॒ता॒म् । सः । प्र॒जाप॑ति॒रिति॑ प्र॒जा - प॒तिः॒ । ब्रा॒ह्म॒णम् । अ॒ब्र॒वी॒त् । ए॒तत् । वीति॑ । ब्रू॒हि॒ । इति॑ । एति॑ । श्रा॒व॒य॒ । इति॑ । इ॒दम् । दे॒वाः॒ । शृ॒णु॒त॒ । इति॑ । वाव । तत् । अ॒ब्र॒वी॒त् । अ॒ग्निः । दे॒वः । होता᳚ । इति॑ । यः । ए॒व । दे॒वाना᳚म् । तम् । अ॒वृ॒णी॒त॒ । ततः॑ । दे॒वाः । \textbf{  68} \newline
                  \newline
                                \textbf{ TS 2.5.11.9} \newline
                  अभ॑वन्न् । परेति॑ । असु॑राः । यस्य॑ । ए॒वम् । वि॒दुषः॑ । प्र॒व॒रमिति॑ प्र - व॒रम् । प्र॒वृ॒णत॒ इति॑ प्र - वृ॒णते᳚ । भव॑ति । आ॒त्मना᳚ । परेति॑ । अ॒स्य॒ । भ्रातृ॑व्यः । भ॒व॒ति॒ । यत् । ब्रा॒ह्म॒णः । च॒ । अब्रा᳚ह्मणः । च॒ । प्र॒श्नम् । ए॒याता॒मित्या᳚ - इ॒याता᳚म् । ब्रा॒ह्म॒णाय॑ । अधीति॑ । ब्रू॒या॒त् । यत् । ब्रा॒ह्म॒णाय॑ । अ॒द्ध्याहेत्य॑धि-आह॑ । आ॒त्मने᳚ । अधीति॑ । आ॒ह॒ । यत् । ब्रा॒ह्म॒णम् । प॒राहेति॑ परा - आह॑ । आ॒त्मान᳚म् । परेति॑ । आ॒ह॒ । तस्मा᳚त् । ब्रा॒ह्म॒णः । न । प॒रोच्य॒ इति॑ परा-उच्यः॑ ॥ \textbf{  69} \newline
                  \newline
                      (वा - आ॑र॒ण्याꣳश्चाव॑ रु॒न्धेऽथो॑ - प॒शुभिः॒ - सो᳚ऽब्रवीद् - दक्षिणा॒र्द्ध्यं॑ त्रय॑ -इव - ध्रु॒वा - दे॒वा - श्च॑त्वारिꣳ॒॒शच्च॑ )  \textbf{(A11)} \newline \newline
                                \textbf{ TS 2.5.12.1} \newline
                  आयुः॑ । ते॒ । आ॒यु॒र्दा इत्या॑युः - दाः । अ॒ग्ने॒ । एति॑ । प्या॒य॒स्व॒ । समिति॑ । ते॒ । अवेति॑ । ते॒ । हेडः॑ । उदिति॑ । उ॒त्त॒ममित्यु॑त् - त॒मम् । प्रेति॑ । नः॒ । दे॒वी । एति॑ । नः॒ । दि॒वः । अग्ना॑विष्णू॒ इत्यग्ना᳚-वि॒ष्णू॒ । अग्ना॑विष्णू॒ इत्यग्ना᳚ - वि॒ष्णू॒ । इ॒मम् । मे॒ । व॒रु॒ण॒ । तत् । त्वा॒ । या॒मि॒ । उदिति॑ । उ॒ । त्यम् । चि॒त्रम् ॥ अ॒पाम् । नपा᳚त् । एति॑ । हि । अस्था᳚त् । उ॒पस्थ॒मित्यु॒प - स्थ॒म् । जि॒ह्माना᳚म् । ऊ॒द्‌र्ध्वः । वि॒द्युत॒मिति॑ वि - द्युत᳚म् । वसा॑नः ॥ तस्य॑ । ज्येष्ठ᳚म् । म॒हि॒मान᳚म् । वह॑न्तीः । हिर॑ण्यवर्णा॒ इति॒ हिर॑ण्य - व॒र्णाः॒ । परीति॑ । य॒न्ति॒ । य॒ह्वीः ॥ समिति॑ । \textbf{  70} \newline
                  \newline
                                \textbf{ TS 2.5.12.2} \newline
                  अ॒न्याः । यन्ति॑ । उपेति॑ । य॒न्ति॒ । अ॒न्याः । स॒मा॒नम् । ऊ॒र्वम् । न॒द्यः॑ । पृ॒ण॒न्ति॒ ॥ तम् । उ॒ । शुचि᳚म् । शुच॑यः । दी॒दि॒वाꣳस᳚म् । अ॒पाम् । नपा॑तम् । परीति॑ । त॒स्थुः॒ । आपः॑ ॥ तम् । अस्मे॑राः । यु॒व॒तयः॑ । युवा॑नम् । म॒र्मृ॒ज्यमा॑नाः । परीति॑ । य॒न्ति॒ । आपः॑ ॥ सः । शु॒क्रेण॑ । शिक्व॑ना । रे॒वत् । अ॒ग्निः । दी॒दाय॑ । अ॒नि॒द्ध्मः । घृ॒तनि॑र्णि॒गिति॑ घृ॒त - नि॒र्णि॒क् । अ॒फ्स्वित्य॑प् - सु ॥ इन्द्रा॒वरु॑णयो॒रितीन्द्रा᳚ - वरु॑णयोः । अ॒हम् । स॒म्राजो॒रिति॑ सं - राजोः᳚ । अवः॑ । एति॑ । वृ॒णे॒ ॥ ता । नः॒ । मृ॒डा॒तः॒ । ई॒दृशे᳚ ॥ इन्द्रा॑वरु॒णेतीन्द्रा᳚ - व॒रु॒णा॒ । यु॒वम् । अ॒द्ध्व॒राय॑ । नः॒ । \textbf{  71} \newline
                  \newline
                                \textbf{ TS 2.5.12.3} \newline
                  वि॒शे । जना॑य । महि॑ । शर्म॑ । य॒च्छ॒त॒म् ॥ दी॒र्घप्र॑यज्यु॒मिति॑ दी॒र्घ - प्र॒य॒ज्यु॒म् । अतीति॑ । यः । व॒नु॒ष्यति॑ । व॒यम् । ज॒ये॒म॒ । पृत॑नासु । दू॒ढ्यः॑ ॥ एति॑ । नः॒ । मि॒त्रा॒व॒रु॒णेति॑ मित्रा - व॒रु॒णा॒ । प्रेति॑ । बा॒हवा᳚ ॥ त्वम् । नः॒ । अ॒ग्ने॒ । वरु॑णस्य । वि॒द्वान् । दे॒वस्य॑ । हेडः॑ । अवेति॑ । या॒सि॒सी॒ष्ठाः॒ ॥ यजि॑ष्ठः । वह्नि॑तम॒ इति॒ वह्नि॑-त॒मः॒ । शोशु॑चानः । विश्वा᳚ । द्वेषाꣳ॑सि । प्रेति॑ । मु॒मु॒ग्धि॒ । अ॒स्मत् ॥ सः । त्वम् । नः॒ । अ॒ग्ने॒ । अ॒व॒मः । भ॒व॒ । ऊ॒ती । नेदि॑ष्ठः । अ॒स्याः । उ॒षसः॑ । व्यु॑ष्टा॒विति॒ वि - उ॒ष्टौ॒ ॥ अवेति॑ । य॒क्ष्व॒ । नः॒ । वरु॑णम् । \textbf{  72} \newline
                  \newline
                                \textbf{ TS 2.5.12.4} \newline
                  ररा॑णः । वी॒हि । मृ॒डी॒कम् । सु॒हव॒ इति॑ सु - हवः॑ । नः॒ । ए॒धि॒ ॥ प्रप्रेति॒ प्र - प्र॒ । अ॒यम् । अ॒ग्निः । भ॒र॒तस्य॑ । शृ॒ण्वे॒ । वीति॑ । यत् । सूर्यः॑ । न । रोच॑ते । बृ॒हत् । भाः ॥ अ॒भीति॑ । यः । पू॒रुम् । पृत॑नासु । त॒स्थौ । दी॒दाय॑ । दैव्यः॑ । अति॑थिः । शि॒वः । नः॒ ॥ प्रेति॑ । ते॒ । य॒क्षि॒ । प्रेति॑ । ते॒ । इ॒य॒र्मि॒ । मन्म॑ । भुवः॑ । यथा᳚ । वन्द्यः॑ । नः॒ । हवे॑षु ॥ धन्वन्न्॑ । इ॒व॒ । प्र॒पेति॑ प्र - पा । अ॒सि॒ । त्वम् । अ॒ग्ने॒ । इ॒य॒क्षवे᳚ । पू॒रवे᳚ । प्र॒त्न॒ । रा॒ज॒न्न् ॥ \textbf{  73} \newline
                  \newline
                                \textbf{ TS 2.5.12.5} \newline
                  वीति॑ । पाज॑सा । वीति॑ । ज्योति॑षा ॥ सः । त्वम् । अ॒ग्ने॒ । प्रती॑केन । प्रतीति॑ । ओ॒ष॒ । या॒तु॒धा॒न्य॑ इति॑ यातु - धा॒न्यः॑ ॥ उ॒रु॒क्षये॒ष्वित्यु॑रु - क्षये॑षु । दीद्य॑त् ॥ तम् । सु॒प्रती॑क॒मिति॑ सु - प्रती॑कम् । सु॒दृश॒मिति॑ सु - दृश᳚म् । स्वञ्च᳚म् । अवि॑द्वाꣳसः । वि॒दुष्ट॑र॒मिति॑ वि॒दुः - त॒र॒म् । स॒पे॒म॒ ॥ सः । य॒क्ष॒त् । विश्वा᳚ । व॒युना॑नि । वि॒द्वान् । प्रेति॑ । ह॒व्यम् । अ॒ग्निः । अ॒मृते॑षु । वो॒च॒त् ॥ अꣳ॒॒हो॒मुच॒ इत्यꣳ॑हः - मुचे᳚ । वि॒वेष॑ । यत् । मा॒ । वीति॑ । नः॒ । इ॒न्द्र॒ । इन्द्र॑ । क्ष॒त्रम् । इ॒न्द्रि॒याणि॑ । श॒त॒क्र॒तो॒ इति॑ शत-क्र॒तो॒ । अन्विति॑ । ते॒ । दा॒यि॒ ॥ \textbf{  74 } \newline
                  \newline
                      (य॒ह्वीः स - म॑ध्व॒राय॑ नो॒ - वरु॑णꣳ - राजꣳ॒॒ -तु॑श्चत्वारिꣳशच्च)  \textbf{(A12)} \newline \newline
\textbf{praSna korvai with starting padams of 1 to 12 Anuvaakams :-} \newline
(वि॒श्वरू॑प॒ - स्त्वष्टे - न्द्रं॑ ॅवृ॒त्रं - ब्र॑ह्मवा॒दिनः॒ स त्वै - नाऽसो॑मयाज्ये॒ - ष वै दे॑वर॒थो - दे॒वा वै नर्चि ना - य॒ज्ञो - ऽग्ने॑ म॒हान् - त्रीन् - निवी॑त॒ - मायु॑ष्टे॒ - द्वाद॑श) \newline

\textbf{korvai with starting padams of1, 11, 21 series of pa~jcAtis :-} \newline
(वि॒श्वरू॑पो॒ - नैनꣳ॑ शीतरू॒रा - व॒द्य वसु॑ - पूर्वे॒द्यु - र्वाजा॒ इत्य - ग्ने॑ म॒हान् - निवी॑त - म॒न्या यन्ति॒ - चतुः॑ सप्ततिः ) \newline

\textbf{first and last padam of fifth praSnam of kANDam 2:-} \newline
(वि॒श्वरूपो॒ - ऽनु॑ ते दायि) \newline 


॥ हरिः॑ ॐ ॥॥ कृष्ण यजुर्वेदीय तैत्तिरीय संहितायां द्वितीयकाण्डे पञ्चमः प्रश्नः समाप्तः ॥ \newline
\pagebreak
2.5.1   Appendix\\2.5.12.1 - आयु॑ष्ट >1, आयु॒र्दा अ॑ग्न॒ >2\\आयु॑ष्टे वि॒श्वतो॑ दधद॒यम॒ग्नि र्वरे᳚ण्यः । \\पुन॑स्ते प्रा॒ण आय॑ति॒ परा॒ यक्ष्मꣳ॑ सुवामि ते । \\\\आ॒यु॒र्दा अ॑ग्ने ह॒विषो॑ जुषा॒णो घृ॒तप्र॑तीको घृ॒तयो॑निरेधि । \\घृ॒तं पी॒त्वा मधु॒ चारु॒ गव्यं॑ पि॒तेव॑ पु॒त्रम॒भि र॑क्षतादि॒मं । \\(Appearing in TS -1.3.14.4 )\\\\2.5.12.1 - आ प्या॑यस्व॒ >3, सं ते >4\\आप्या॑स्व॒ समे॑तु ते वि॒श्वतः॑ सोम॒ वृष्णि॑यं ।\\भवा॒ वाज॑स्य संग॒थे । \\\\सं ते॒ पयाꣳ॑सि॒ समु॑ यन्तु॒ वाजाः॒ सं ॅवृष्णि॑यान्. यभिमाति॒षाहः॑ ।\\आ॒प्याय॑मानो अ॒मृता॑य सोम दि॒वि श्रवाꣳ॑स्युत्त॒मानि॑ धिष्व ॥ \\(Appearing in TS- 4-2-7-4)\\\\2.5.12.1 - अव॑ ते॒ हेड॒>5, उदु॑त्त॒मं>6\\अव॑ ते॒ हेडो॑ वरुण॒ नमो॑भि॒रव॑ य॒ज्ञेभि॑रीमहे ह॒विर्भिः॑ । क्षय॑न्न॒स्मभ्य॑मसुर प्रचेतो॒ राज॒न्नेनाꣳ॑सि शिश्रथः कृ॒तानि॑ ॥ \\\\उदु॑त्त॒मं ॅव॑रुण॒ पाश॑म॒स्मदवा॑ऽध॒मं ॅविम॑ध्य॒मꣳ श्र॑थाय । \\अथा॑ व॒यमा॑दित्य व्र॒ते तवाऽना॑गसो॒ अदि॑तये स्याम ॥\\(Appearing in TS- 1.5.11.3)\\\\2.5.12.1 - प्रणो॑ दे॒व्या>7, नो॑ दि॒वो>8\\प्रणो॑ दे॒वी सर॑स्वती॒ वाजे॑भिर्वा॒जिनी॑वती । धी॒नाम॑वि॒त्र्य॑वतु ॥\\\\आ नो॑ दि॒वो बृ॑ह॒तः पर्व॑ता॒दा सर॑स्वती यज॒ता ग॑न्तु य॒ज्ञ्ं । \\हवं॑ दे॒वी जु॑जुषा॒णा घृ॒ताची॑ श॒ग्मां नो॒ वाच॑मुश॒ती शृ॑णोतु ॥\\(Appearing in TS - 1.8.22.1)\\\\2.5.12.1 - अग्ना॑ विष्णू॒>9, अग्ना॑विष्णू>10\\अग्ना॑विष्णू॒ महि॒ तद्वां᳚ महि॒त्वं ॅवी॒तं घृ॒तस्य॒ गुह्या॑नि॒ नाम॑ । \\दमे॑दमे स॒प्त रत्ना॒ दधा॑ना॒ प्रति॑ ॅवां जि॒ह्वा घृ॒तमा च॑रण्येत् ॥\\\\अग्ना॑विष्णू॒ महि॒ धाम॑ प्रि॒यं ॅवां᳚ ॅवी॒थो घृ॒तस्य॒ गुह्या॑ जुषा॒णा । \\दमे॑दमे सुष्टु॒तीर्वा॑वृधा॒ना प्रति॑ वां जि॒ह्वा घृ॒तमुच्च॑रण्येत् ॥\\(Appearing in TS- 1.8.22.1)\\\\2.5.12.1 - इ॒मं मे॑ वरुण॒>11, तत् त्वा॑ या॒>12\\इ॒मं मे॑ वरुण श्रुधी॒ हव॑म॒द्या च॑ मृडय । त्वाम॑व॒स्युरा च॑के ॥\\\\तत्त्वा॑ यामि॒ ब्रह्म॑णा॒ वन्द॑मान॒स्तदा शा᳚स्ते॒ यज॑मानो ह॒विर्भिः॑ । \\अहे॑डमानो वरुणे॒ह बो॒ध्युरु॑शꣳस॒ मा न॒ आयुः॒ प्रमो॑षीः ॥\\(Appearing in TS- 2.1.11.6)\\\\2.5.12.1 - म्यु दु॒त्यं>13, चि॒त्रं >14\\उदु॒ त्यं जा॒तवे॑दसं दे॒वं ॅव॑हन्ति के॒तवः॑ । दृ॒शे विश्वा॑य॒ सूर्यं᳚ ॥\\\\चि॒त्रं दे॒वाना॒-मुद॑गा॒दनी॑कं॒ चक्षु॑ र्मि॒त्रस्य॒ वरु॑णस्या॒ऽग्नेः ।\\आऽप्रा॒ द्यावा॑पृथि॒वी अ॒न्तरि॑क्षꣳ॒॒ सूर्य॑ आ॒त्मा जग॑तस्तस्थुष॑श्च ॥\\(Appearing in TS- 1.4.43.1)\\\\2.5.12.3 - आ नो॑ मित्रावरुणा॒>15, प्र बा॒हवा᳚>16\\आ नो॑ मित्रावरुणा घृ॒तैर्गव्यू॑तिमुक्षतं । मध्वा॒ रजाꣳ॑सि सुक्रतू ॥\\\\प्र बा॒हवा॑ सिसृतं जी॒वसे॑ न॒ आ नो॒ गव्यू॑तिमुक्षतं घृ॒तेन॑ । \\आ नो॒ जने᳚ श्रवयतं ॅयुवाना श्रु॒तं मे॑ मित्रावरुणा॒ हवे॒मा ॥\\(Appearing in TS- 1.8.22.3)\\\\2.5.12.5 वि पाज॑सा॒>17\\वि पाज॑सापृ॒थुना॒ शोशु॑चानो॒ बाध॑स्व द्वि॒षो र॒क्षसो॒ अमी॑वाः ।\\सु॒शर्म॑णो बृह॒तः शर्म॑णि स्याम॒ग्नेर॒हꣳ सु॒हव॑स्य॒ प्रणी॑तौ ॥\\(Appearing in TS- 4.1.5.1)\\\\2.5.12.5 वि ज्योति॑षा>18\\विज्योति॑षा बृह॒ता भा᳚त्य॒ग्निरा॒वि-र्विश्वा॑नि कृणुते महि॒त्वा ।\\प्राऽदे॑वी-र्मा॒याः स॑हते-दु॒रेवाः॒ शिशी॑ते॒ शृङ्गे॒ रक्ष॑से वि॒निक्षे᳚ ॥\\(Appearing in TS- 1.2.14.7)\\\\2.5.12.5अꣳ॒॒हो॒मुचे॑ >19, वि॒वेष॒ यन्मा॒>20\\अꣳ॒॒हो॒मुचे॒ प्रभ॑रेमा मनी॒षा मो॑षिष्ठ॒-दाव्ने॑ सुम॒तिं गृ॑णा॒नाः । \\इ॒दमि॑न्द्र॒ प्रति॑ ह॒व्यं गृ॑भाय स॒त्याः स॑न्तु॒ यज॑मानस्य॒ कामाः᳚ ॥ \\वि॒वेष॒ यन्मा॑ धि॒षणा॑ ज॒जान॒ स्तवै॑ पु॒रा पार्या॒दिन्द्र॒ मह्नः॑ । \\अꣳह॑सो॒ यत्र॑ पी॒पर॒द्यथा॑ नो ना॒वेव॒ यान्त॑ मु॒भये॑ हवन्ते ॥ \\(Appearing in TS- 1.6.12.3)\\\\2.5.12.5 विन॑ इ॒न्द्रे>21, न्द्र॑ क्ष॒त्र>22\\वि न॑ इन्द्र॒ मृधो॑ जहि नी॒चा य॑च्छ पृतन्य॒तः । \\अ॒ध॒स्प॒दं तमीं᳚ कृधि॒ यो अ॒स्माꣳ अ॑भि॒दास॑ति ॥\\\\इन्द्र॑ क्ष॒त्रम॒भि वा॒ममोजो ऽजा॑यथा वृषभ चर्.षणी॒नां । \\अपा॑ऽनुदो॒ जन॑ममित्र॒ यन्त॑ मु॒रुं दे॒वेभ्यो॑ अकृणोरु लो॒कं ॥ \\(Appearing in TS- 1.6.12.4)\\\\2.5.12.5 इन्द्रि॒याणि॑ शतक्र॒तो>23, ऽनु॑ ते दायि>24\\इ॒न्द्रि॒याणि॑ शतक्रतो॒ या ते॒ जने॑षु प॒ञ्चसु॑ । \\इन्द्र॒ तानि॑ त॒ आ वृ॑णे ॥\\\\अनु॑ ते दायि म॒ह इ॑न्द्रि॒याय॑ स॒त्राते॒ विश्व॒मनु॑ वृत्र॒हत्ये᳚ । \\अनु॑ क्ष॒त्रमनु॒ सहो॑ यज॒त्रेन्द्र॑ दे॒वेभि॒रनु॑ ते नृ॒षह्ये᳚ ॥ \\(Appearing in TS- 1.6.12.1)\\================================\\
\pagebreak
        


\end{document}
