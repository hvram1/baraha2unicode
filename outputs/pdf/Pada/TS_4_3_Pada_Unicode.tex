\documentclass[17pt]{extarticle}
\usepackage{babel}
\usepackage{fontspec}
\usepackage{polyglossia}
\usepackage{extsizes}



\setmainlanguage{sanskrit}
\setotherlanguages{english} %% or other languages
\setlength{\parindent}{0pt}
\pagestyle{myheadings}
\newfontfamily\devanagarifont[Script=Devanagari]{AdishilaVedic}


\newcommand{\VAR}[1]{}
\newcommand{\BLOCK}[1]{}




\begin{document}
\begin{titlepage}
    \begin{center}
 
\begin{sanskrit}
    { \Large
    ॐ नमः परमात्मने, श्री महागणपतये नमः, 
श्री गुरुभ्यो नमः, ह॒रिः॒ ॐ 
    }
    \\
    \vspace{2.5cm}
    \mbox{ \Huge
    4.3      चतुर्थकाण्डे तृतीयः प्रश्नः - चितिवर्णनं   }
\end{sanskrit}
\end{center}

\end{titlepage}
\tableofcontents

ॐ नमः परमात्मने, श्री महागणपतये नमः, 
श्री गुरुभ्यो नमः , ह॒रिः॒ ॐ \newline
4.3      चतुर्थकाण्डे तृतीयः प्रश्नः - चितिवर्णनं \newline

\addcontentsline{toc}{section}{ 4.3      चतुर्थकाण्डे तृतीयः प्रश्नः - चितिवर्णनं}
\markright{ 4.3      चतुर्थकाण्डे तृतीयः प्रश्नः - चितिवर्णनं \hfill https://www.vedavms.in \hfill}
\section*{ 4.3      चतुर्थकाण्डे तृतीयः प्रश्नः - चितिवर्णनं }
                                \textbf{ TS 4.3.1.1} \newline
                  अ॒पाम् । त्वा॒ । एमन्न्॑ । सा॒द॒या॒मि॒ । अ॒पाम् । त्वा॒ । ओद्मन्न्॑ । सा॒द॒या॒मि॒ । अ॒पाम् । त्वा॒ । भस्मन्न्॑ । सा॒द॒या॒मि॒ । अ॒पाम् । त्वा॒ । ज्योति॑षि । सा॒द॒या॒मि॒ । अ॒पाम् । त्वा॒ । अय॑ने । सा॒द॒या॒मि॒ । अ॒र्ण॒वे । सद॑ने । सी॒द॒ । स॒मु॒द्रे । सद॑ने । सी॒द॒ । स॒लि॒ले । सद॑ने । सी॒द॒ । अ॒पाम् । क्षये᳚ । सी॒द॒ । अ॒पाम् । सधि॑षि । सी॒द॒ । अ॒पाम् । त्वा॒ । सद॑ने । सा॒द॒या॒मि॒ । अ॒पाम् । त्वा॒ । स॒धस्थ॒ इति॑ स॒ध - स्थे॒ । सा॒द॒या॒मि॒ । अ॒पाम् । त्वा॒ । पुरी॑षे । सा॒द॒या॒मि॒ । अ॒पाम् । त्वा॒ । योनौ᳚ ( ) । सा॒द॒या॒मि॒ । अ॒पाम् । त्वा॒ । पाथ॑सि । सा॒द॒या॒मि॒ । गा॒य॒त्री । छन्दः॑ । त्रि॒ष्टुप् । छन्दः॑ । जग॑ती । छन्दः॑ । अ॒नु॒ष्टुबित्य॑नु - स्तुप् । छन्दः॑ । प॒ङ्क्तिः । छन्दः॑ ॥ \textbf{  1 } \newline
                  \newline
                      (योनौ॒ - पञ्च॑दश च)  \textbf{(A1)} \newline \newline
                                \textbf{ TS 4.3.2.1} \newline
                  अ॒यम् । पु॒रः । भुवः॑ । तस्य॑ । प्रा॒ण इति॑ प्र - अ॒नः । भौ॒वा॒य॒नः । व॒स॒न्तः । प्रा॒णा॒य॒नः । गा॒य॒त्री । वा॒स॒न्ती । गा॒य॒त्रि॒यै । गा॒य॒त्रम् । गा॒य॒त्रात् । उ॒पाꣳ॒॒शुरित्यु॑प - अꣳ॒॒शुः । उ॒पाꣳ॒॒शोरित्यु॑प-अꣳ॒॒शोः । त्रि॒वृदिति॑ त्रि - वृत् । त्रि॒वृत॒ इति॑ त्रि - वृतः॑ । र॒थ॒न्त॒रमिति॑ रथं - त॒रम् । र॒थ॒न्त॒रादिति॑ रथं - त॒रात् । वसि॑ष्ठः । ऋषिः॑ । प्र॒जाप॑तिगृहीत॒येति॑ प्र॒जाप॑ति - गृ॒ही॒त॒या॒ । त्वया᳚ । प्रा॒णमिति॑ प्र - अ॒नम् । गृ॒ह्णा॒मि॒ । प्र॒जाभ्य॒ इति॑ प्र - जाभ्यः॑ । अ॒यम् । द॒क्षि॒णा । वि॒श्वक॒र्मेति॑ वि॒श्व - क॒र्मा॒ । तस्य॑ । मनः॑ । वै॒श्व॒क॒र्म॒णमिति॑ वैश्व - क॒र्म॒णम् । ग्री॒ष्मः । मा॒न॒सः । त्रि॒ष्टुक् । ग्रै॒ष्मी । त्रि॒ष्टुभः॑ । ऐ॒डम् । ऐ॒डात् । अ॒न्त॒र्या॒म इत्य॑न्तः - या॒मः । अ॒न्त॒र्या॒मादित्य॑न्तः - या॒मात् । प॒ञ्च॒द॒श इति॑ पञ्च - द॒शः । प॒ञ्च॒द॒शादिति॑ पञ्च-द॒शात् । बृ॒हत् । बृ॒ह॒तः । भ॒रद्वा॑जः । ऋषिः॑ । प्र॒जाप॑तिगृहीत॒येति॑ प्र॒जाप॑ति - गृ॒ही॒त॒या॒ । त्वया᳚ । मनः॑ । \textbf{  2} \newline
                  \newline
                                \textbf{ TS 4.3.2.2} \newline
                  गृ॒ह्णा॒मि॒ । प्र॒जाभ्य॒ इति॑ प्र - जाभ्यः॑ । अ॒यम् । प॒श्चात् । वि॒श्वव्य॑चा॒ इति॑ वि॒श्व - व्य॒चाः॒ । तस्य॑ । चक्षुः॑ । वै॒श्व॒व्य॒च॒समिति॑ वैश्व - व्य॒च॒सम् । व॒र्॒.षाणि॑ । चा॒क्षु॒षाणि॑ । जग॑ती । वा॒र्॒.षी । जग॑त्याः । ऋक्ष॑मम् । ऋक्ष॑मात् । शु॒क्रः । शु॒क्रात् । स॒प्त॒द॒श इति॑ सप्त - द॒शः । स॒प्त॒द॒शादिति॑ सप्त-द॒शात् । वै॒रू॒पम् । वै॒रू॒पात् । वि॒श्वामि॑त्र॒ इति॑ वि॒श्व - मि॒त्रः॒ । ऋषिः॑ । प्र॒जाप॑तिगृहीत॒येति॑ प्र॒जाप॑ति - गृ॒ही॒त॒या॒ । त्वया᳚ । चक्षुः॑ । गृ॒ह्णा॒मि॒ । प्र॒जाभ्य॒ इति॑ प्र - जाभ्यः॑ । इ॒दम् । उ॒त्त॒रादित्यु॑त् - त॒रात् । सुवः॑ । तस्य॑ । श्रोत्र᳚म् । सौ॒वम् । श॒रत् । श्रौ॒त्री । अ॒नु॒ष्टुबित्य॑नु - स्तुप् । शा॒र॒दी । अ॒नु॒ष्टुभ॒ इत्य॑नु - स्तुभः॑ । स्वा॒रम् । स्वा॒रात् । म॒न्थी । म॒न्थिनः॑ । ए॒क॒विꣳ॒॒श इत्ये॑क - विꣳ॒॒शः । ए॒क॒विꣳ॒॒शादित्ये॑क - विꣳ॒॒शात् । वै॒रा॒जम् । वै॒रा॒जात् । ज॒मद॑ग्निः । ऋषिः॑ । प्र॒जाप॑तिगृहीत॒येति॑ प्र॒जाप॑ति - गृ॒ही॒त॒या॒ । \textbf{  3} \newline
                  \newline
                                \textbf{ TS 4.3.2.3} \newline
                  त्वया᳚ । श्रोत्र᳚म् । गृ॒ह्णा॒मि॒ । प्र॒जाभ्य॒ इति॑ प्र-जाभ्यः॑ । इ॒यम् । उ॒परि॑ । म॒तिः । तस्यै᳚ । वाक् । मा॒ती । हे॒म॒न्तः । वा॒च्या॒य॒नः । प॒ङ्क्तिः । है॒म॒न्ती । प॒ङ्क्त्यै । नि॒धन॑व॒दिति॑ नि॒धन॑ - व॒त् । नि॒धन॑वत॒ इति॑ नि॒धन॑ - व॒तः॒ । आ॒ग्र॒य॒णः । आ॒ग्र॒य॒णात् । त्रि॒ण॒व॒त्र॒य॒स्त्रिꣳ॒॒शाविति॑ त्रिणव - त्र॒य॒स्त्रिꣳ॒॒शौ । त्रि॒ण॒व॒त्र॒य॒स्त्रिꣳ॒॒शाभ्या॒मिति॑ त्रिणव - त्र॒य॒स्त्रिꣳ॒॒शाभ्या᳚म् । शा॒क्व॒र॒रै॒व॒ते इति॑ शाक्वर - रै॒व॒ते । शा॒क्व॒र॒रै॒व॒ताभ्या॒मिति॑ शाक्वर - रै॒व॒ताभ्या᳚म् । वि॒श्वक॒र्मेति॑ वि॒श्व - क॒र्मा॒ । ऋषिः॑ । प्र॒जाप॑तिगृहीत॒येति॑ प्र॒जाप॑ति - गृ॒ही॒त॒या॒ । त्वया᳚ । वाच᳚म् । गृ॒ह्णा॒मि॒ । प्र॒जाभ्य॒ इति॑ प्र - जाभ्यः॑ ॥ \textbf{  4 } \newline
                  \newline
                      (त्वया॒ मनो॑-ज॒मद॑ग्नि॒र्॒.ऋषिः॑ प्र॒जाप॑तिगृहीतया-त्रिꣳ॒॒शच्च॑)  \textbf{(A2)} \newline \newline
                                \textbf{ TS 4.3.3.1} \newline
                  प्राची᳚ । दि॒शाम् । व॒स॒न्तः । ऋ॒तू॒नाम् । अ॒ग्निः । दे॒वता᳚ । ब्रह्म॑ । द्रवि॑णम् । त्रि॒वृदिति॑ त्रि - वृत् । स्तोमः॑ । सः । उ॒ । प॒ञ्च॒द॒शव॑र्तनि॒रिति॑ पञ्चद॒श - व॒र्त॒निः॒ । त्र्यवि॒रिति॑ त्रि - अविः॑ । वयः॑ । कृ॒तम् । अया॑नाम् । पु॒रो॒वा॒त इति॑ पुरः - वा॒तः । वातः॑ । सान॑गः । ऋषिः॑ । द॒क्षि॒णा । दि॒शाम् । ग्री॒ष्मः । ऋ॒तू॒नाम् । इन्द्रः॑ । दे॒वता᳚ । क्ष॒त्रम् । द्रवि॑णम् । प॒ञ्च॒द॒श इति॑ पञ्च - द॒शः । स्तोमः॑ । सः । उ॒ । स॒प्त॒द॒शव॑र्तनि॒रिति॑ सप्तद॒श - व॒र्त॒निः॒ । दि॒त्य॒वाडिति॑ दित्य - वाट् । वयः॑ । त्रेता᳚ । अया॑नाम् । द॒क्षि॒णा॒द्वा॒त इति॑ दक्षिणात् - वा॒तः । वातः॑ । स॒ना॒तन॒ इति॑ सना - तनः॑ । ऋषिः॑ । प्र॒तीची᳚ । दि॒शाम् । व॒र्॒.षाः । ऋ॒तू॒नाम् । विश्वे᳚ । दे॒वाः । दे॒वता᳚ । विट् । \textbf{  5} \newline
                  \newline
                                \textbf{ TS 4.3.3.2} \newline
                  द्रवि॑णम् । स॒प्त॒द॒श इति॑ सप्त-द॒शः । स्तोमः॑ । सः । उ॒ । ए॒क॒विꣳ॒॒शव॑र्तनि॒रित्ये॑कविꣳ॒॒श - व॒र्त॒निः॒ । त्रि॒व॒थ्स इति॑ त्रि-व॒थ्सः । वयः॑ । द्वा॒प॒रः । अया॑नाम् । प॒श्चा॒द्वा॒त इति॑ पश्चात्-वा॒तः । वातः॑ । अ॒ह॒भूनः॑ । ऋषिः॑ । उदी॑ची । दि॒शाम् । श॒रत् । ऋ॒तू॒नाम् । मि॒त्रावरु॑णा॒विति॑ मि॒त्रा - वरु॑णौ । दे॒वता᳚ । पु॒ष्टम् । द्रवि॑णम् । ए॒क॒विꣳ॒॒श इत्ये॑क - विꣳ॒॒शः । स्तोमः॑ । सः । उ॒ । त्रि॒ण॒वव॑र्तनि॒रिति॑ त्रिण॒व - व॒र्त॒निः॒ । तु॒र्य॒वाडिति॑ तुर्य - वाट् । वयः॑ । आ॒स्क॒न्द इत्या᳚ - स्क॒न्दः । अया॑नाम् । उ॒त्त॒रा॒द्वा॒त इत्यु॑त्तरात् - वा॒तः । वातः॑ । प्र॒त्नः । ऋषिः॑ । ऊ॒द्‌र्ध्वा । दि॒शाम् । हे॒म॒न्त॒शि॒शि॒राविति॑ हेमन्त - शि॒शि॒रौ । ऋ॒तू॒नाम् । बृह॒स्पतिः॑ । दे॒वता᳚ । वर्चः॑ । द्रवि॑णम् । त्रि॒ण॒व इति त्रि॑ - न॒वः । स्तोमः॑ । सः । उ॒ । त्र॒य॒स्त्रिꣳ॒॒शव॑र्तनि॒रिति॑ त्रयस्त्रिꣳ॒॒श - व॒र्त॒निः॒ । प॒ष्ठ॒वादिति॑ पष्ठ - वात् । वयः॑ ( ) । अ॒भि॒भूरित्य॑भि - भूः । अया॑नाम् । वि॒ष्व॒ग्वा॒त इति॑ विष्वक् - वा॒तः । वातः॑ । सु॒प॒र्ण इति॑ सु - प॒र्णः । ऋषिः॑ । पि॒तरः॑ । पि॒ता॒म॒हाः । परे᳚ । अव॑रे । ते । नः॒ । पा॒न्तु॒ । ते । नः॒ । अ॒व॒न्तु॒ । अ॒स्मिन्न् । ब्रह्मन्न्॑ । अ॒स्मिन्न् । क्ष॒त्रे । अ॒स्याम् । आ॒शिषीत्या᳚ - शिषि॑ । अ॒स्याम् । पु॒रो॒धाया॒मिति॑ पुरः - धाया᳚म् । अ॒स्मिन्न् । कर्मन्न्॑ । अ॒स्याम् । दे॒वहू᳚त्या॒मिति॑ दे॒व-हू॒त्या॒म् ॥ \textbf{  6} \newline
                  \newline
                      (विट् - प॑ष्ठ॒वाद् वयो॒ - ऽष्टाविꣳ॑शतिश्च)  \textbf{(A3)} \newline \newline
                                \textbf{ TS 4.3.4.1} \newline
                  ध्रु॒वक्षि॑ती॒रिति॑ ध्रु॒व - क्षि॒तिः॒ । ध्रु॒वयो॑नि॒रिति॑ ध्रु॒व-यो॒निः॒ । ध्रु॒वा । अ॒सि॒ । ध्रु॒वाम् । योनि᳚म् । एति॑ । सी॒द॒ । सा॒द्ध्या ॥ उख्य॑स्य । के॒तुम् । प्र॒थ॒मम् । पु॒रस्ता᳚त् । अ॒श्विना᳚ । अ॒द्ध्व॒र्यू इति॑ । सा॒द॒य॒ता॒म् । इ॒ह । त्वा॒ ॥ स्वे । दक्षे᳚ । दक्ष॑पि॒तेति॒ दक्ष॑ - पि॒ता॒ । इ॒ह । सी॒द॒ । दे॒व॒त्रेति॑ देव - त्रा । पृ॒थि॒वी । बृ॒ह॒ती । ररा॑णा ॥ स्वा॒स॒स्थेति॑ सु - आ॒स॒स्था । त॒नुवा᳚ । समिति॑ । वि॒श॒स्व॒ । पि॒ता । इ॒व॒ । ए॒धि॒ । सू॒नवे᳚ । एति॑ । सु॒शेवेति॑ सु - शेवा᳚ । अ॒श्विना᳚ । अ॒द्ध्व॒र्यू इति॑ । सा॒द॒य॒ता॒म् । इ॒ह । त्वा॒ ॥ कु॒ला॒यिनी᳚ । वसु॑म॒तीति॒ वसु॑ - म॒ती॒ । व॒यो॒धा इति॑ वयः - धाः । र॒यिम् । नः॒ । व॒द्‌र्ध॒ । ब॒हु॒लम् । सु॒वीर॒मिति॑ सु - वीर᳚म् ॥ \textbf{  7 } \newline
                  \newline
                                \textbf{ TS 4.3.4.2} \newline
                  अपेति॑ । अम॑तिम् । दु॒र्म॒तिमिति॑ दुः - म॒तिम् । बाध॑माना । रा॒यः । पोषे᳚ । य॒ज्ञ्प॑ति॒मिति॑ य॒ज्ञ् - प॒ति॒म् । आ॒भज॒न्तीत्या᳚ - भज॑न्ती । सुवः॑ । धे॒हि॒ । यज॑मानाय । पोष᳚म् । अ॒श्विना᳚ । अ॒द्ध्व॒र्यू इति॑ । सा॒द॒य॒ता॒म् । इ॒ह । त्वा॒ ॥ अ॒ग्नेः । पुरी॑षम् । अ॒सि॒ । दे॒व॒यानीति॑ देव - यानी᳚ । ताम् । त्वा॒ । विश्वे᳚ । अ॒भीति॑ । गृ॒ण॒न्तु॒ । दे॒वाः ॥ स्तोम॑पृ॒ष्ठेति॒ स्तोम॑ - पृ॒ष्ठा॒ । घृ॒तव॒तीति॑ घृ॒त - व॒ती॒ । इ॒ह । सी॒द॒ । प्र॒जाव॒दिति॑ प्र॒जा - व॒त् । अ॒स्मे इति॑ । द्रवि॑णा । एति॑ । य॒ज॒स्व॒ । आ॒श्विना᳚ । अ॒द्ध्व॒र्यू इति॑ । सा॒द॒य॒ता॒म् । इ॒ह । त्वा॒ ॥ दि॒वः । मू॒द्‌र्धा । अ॒सि॒ । पृ॒थि॒व्याः । नाभिः॑ । वि॒ष्टंभ॒नीति॑ वि - स्तंभ॑नी । दि॒शाम् । अधि॑प॒त्नीत्यधि॑ - प॒त्नी॒ । भुव॑नानाम् ॥ \textbf{  8} \newline
                  \newline
                                \textbf{ TS 4.3.4.3} \newline
                  ऊ॒र्मिः । द्र॒फ्सः । अ॒पाम् । अ॒सि॒ । वि॒श्वक॒र्मेति॑ वि॒श्व - क॒र्मा॒ । ते॒ । ऋषिः॑ । अ॒श्विना᳚ । अ॒द्ध्व॒र्यू इति॑ । सा॒द॒य॒ता॒म् । इ॒ह । त्वा॒ ॥ स॒जूरिति॑ स - जूः । ऋ॒तुभि॒रित्यृ॒तु - भिः॒ । स॒जूरिति॑ स - जूः । वि॒धाभि॒रिति॑ वि-धाभिः॑ । स॒जूरिति॑ स-जूः । वसु॑भि॒रिति॒ वसु॑-भिः॒ । स॒जूरिति॑ स - जूः । रु॒द्रैः । स॒जूरिति॑ स - जूः । आ॒दि॒त्यैः । स॒जूरिति॑ स - जूः । विश्वैः᳚ । दे॒वैः । स॒जूरिति॑ स - जूः । दे॒वैः । स॒जूरिति॑ स - जूः । दे॒वैः । व॒यो॒ना॒धैरिति॑ वयः - ना॒धैः । अ॒ग्नये᳚ । त्वा॒ । वै॒श्वा॒न॒राय॑ । अ॒श्विना᳚ । अ॒द्ध्व॒र्यू इति॑ । सा॒द॒य॒ता॒म् । इ॒ह । त्वा॒ ॥ प्रा॒णमिति॑ प्र - अ॒नम् । मे॒ । पा॒हि॒ । अ॒पा॒नमित्य॑प - अ॒नम् । मे॒ । पा॒हि॒ । व्या॒नमिति॑ वि-अ॒नम् । मे॒ । पा॒हि॒ । चक्षुः॑ । मे॒ । उ॒र्व्या ( ) । वीति॑ । भा॒हि॒ । श्रोत्र᳚म् । मे॒ । श्लो॒क॒य॒ । अ॒पः । पि॒न्व॒ । ओष॑धीः । जि॒न्व॒ । द्वि॒पादिति॑ द्वि - पात् । पा॒हि॒ । चतु॑ष्पा॒दिति॒ चतुः॑ - पा॒त् । अ॒व॒ । दि॒वः । वृष्टि᳚म् । एति॑ । ई॒र॒य॒ ॥ \textbf{  9 } \newline
                  \newline
                      (सु॒वीरं॒ - भुव॑नाना - मु॒र्व्या - स॒प्तद॑श च)  \textbf{(A4)} \newline \newline
                                \textbf{ TS 4.3.5.1} \newline
                  त्र्यवि॒रिति॑ त्रि - अविः॑ । वयः॑ । त्रि॒ष्टुप् । छन्दः॑ । दि॒त्य॒वाडिति॑ दित्य - वाट् । वयः॑ । वि॒राडिति॑ वि - राट् । छन्दः॑ । पञ्चा॑वि॒रिति॒ पञ्च॑-अ॒विः॒ । वयः॑ । गा॒य॒त्री । छन्दः॑ । त्रि॒व॒थ्स इति॑ त्रि - व॒थ्सः । वयः॑ । उ॒ष्णिहा᳚ । छन्दः॑ । तु॒र्य॒वाडिति॑ तुर्य - वाट् । वयः॑ । अ॒नु॒ष्टुबित्य॑नु - स्तुप् । छन्दः॑ । प॒ष्ठ॒वादिति॑ पष्ठ - वात् । वयः॑ । बृ॒ह॒ती । छन्दः॑ । उ॒क्षा । वयः॑ । स॒तोबृ॑ह॒तीति॑ स॒तः - बृ॒ह॒ती॒ । छन्दः॑ । ऋ॒ष॒भः । वयः॑ । क॒कुत् । छन्दः॑ । धे॒नुः । वयः॑ । जग॑ती । छन्दः॑ । अ॒न॒ड्वान् । वयः॑ । प॒ङ्क्तिः । छन्दः॑ । ब॒स्तः । वयः॑ । वि॒व॒लमिति॑ वि - व॒लम् । छन्दः॑ । वृ॒ष्णिः । वयः॑ । वि॒शा॒लमिति॑ वि-शा॒लम् । छन्दः॑ । पुरु॑षः । वयः॑ ( ) । त॒न्द्रम् । छन्दः॑ । व्या॒घ्रः । वयः॑ । अना॑धृष्ट॒मित्यना᳚ - धृ॒ष्ट॒म् । छन्दः॑ । सिꣳ॒॒हः । वयः॑ । छ॒दिः । छन्दः॑ । वि॒ष्ट॒भं इति॑ वि - स्त॒भंः । वयः॑ । अधि॑पति॒रित्यधि॑-प॒तिः॒ । छन्दः॑ । क्ष॒त्रम् । वयः॑ । मय॑न्दम् । छन्दः॑ । वि॒श्वक॒र्मेति॑ वि॒श्व - क॒र्मा॒ । वयः॑ । प॒र॒मे॒ष्ठी । छन्दः॑ । मू॒द्‌र्धा । वयः॑ । प्र॒जाप॑ति॒रिति॑ प्र॒जा - प॒तिः॒ । छन्दः॑ ॥ \textbf{  10 } \newline
                  \newline
                      (पुरु॑षो॒ वयः॒ - षड् विꣳ॑शतिश्च)  \textbf{(A5)} \newline \newline
                                \textbf{ TS 4.3.6.1} \newline
                  इन्द्रा᳚ग्नी॒ इतीन्द्र॑ - अ॒ग्नी॒ । अव्य॑थमानाम् । इष्ट॑काम् । दृꣳ॒॒ह॒त॒म् । यु॒वम् ॥ पृ॒ष्ठेन॑ । द्यावा॑पृथि॒वी इति॒ द्यावा᳚ -पृ॒थि॒वी । अ॒न्तरि॑क्षम् । च॒ । वीति॑ । बा॒ध॒ता॒म् ॥ वि॒श्वक॒र्मेति॑ वि॒श्व - क॒र्मा॒ । त्वा॒ । सा॒द॒य॒तु॒ । अ॒न्तरि॑क्षस्य । पृ॒ष्ठे । व्यच॑स्वतीम् । प्रथ॑स्वतीम् । भास्व॑तीम् । सू॒रि॒मती॒मिति॑ सूरि - मती᳚म् । एति॑ । या । द्याम् । भासि॑ । एति॑ । पृ॒थि॒वीम् । एति॑ । उ॒रु । अ॒न्तरि॑क्षम् । अ॒न्तरि॑क्षम् । य॒च्छ॒ । अ॒न्तरि॑क्षम् । दृꣳ॒॒ह॒ । अ॒न्तरि॑क्षम् । मा । हिꣳ॒॒सीः॒ । विश्व॑स्मै । प्रा॒णायेति॑ प्रा - अ॒नाय॑ । अ॒पा॒नायेत्य॑प - अ॒नाय॑ । व्या॒नायेति॑ वि - अ॒नाय॑ । उ॒दा॒नायेत्यु॑त् - अ॒नाय॑ । प्र॒ति॒ष्ठाया॒ इति॑ प्रति - स्थायै᳚ । च॒रित्रा॑य । वा॒युः । त्वा॒ । अ॒भीति॑ । पा॒तु॒ । म॒ह्या । स्व॒स्त्या । छ॒र्दिषा᳚ । \textbf{  11} \newline
                  \newline
                                \textbf{ TS 4.3.6.2} \newline
                  शन्त॑मे॒नेति॒ शं - त॒मे॒न॒ । तया᳚ । दे॒वत॑या । अ॒ङ्गि॒र॒स्वत् । ध्रु॒वा । सी॒द॒ ॥ राज्ञी᳚ । अ॒सि॒ । प्राची᳚ । दिक् । वि॒राडिति॑ वि-राट् । अ॒सि॒ । द॒क्षि॒णा । दिक् । स॒म्राडिति॑ सं - राट् । अ॒सि॒ । प्र॒तीची᳚ । दिक् । स्व॒राडिति॑ स्व - राट् । अ॒सि॒ । उदी॑ची । दिक् । अधि॑प॒त्नीत्यधि॑-प॒त्नी॒ । अ॒सि॒ । बृ॒ह॒ती । दिक् । आयुः॑ । मे॒ । पा॒हि॒ । प्रा॒णमिति॑ प्र - अ॒नम् । मे॒ । पा॒हि॒ । अ॒पा॒नमित्य॑प - अ॒नम् । मे॒ । पा॒हि॒ । व्या॒नमिति॑ वि - अ॒नम् । मे॒ । पा॒हि॒ । चक्षुः॑ । मे॒ । पा॒हि॒ । श्रोत्र᳚म् । मे॒ । पा॒हि॒ । मनः॑ । मे॒ । जि॒न्व॒ । वाच᳚म् । मे॒ । पि॒न्व॒ ( ) । आ॒त्मान᳚म् । मे॒ । पा॒हि॒ । ज्योतिः॑ । मे॒ । य॒च्छ॒ ॥ \textbf{  12 } \newline
                  \newline
                      (छ॒र्दिषा॑ - पिन्व॒ - षट्च॑)  \textbf{(A6)} \newline \newline
                                \textbf{ TS 4.3.7.1} \newline
                  मा । छन्दः॑ । प्र॒मेति॑ प्र - मा । छन्दः॑ । प्र॒ति॒मेति॑ प्रति-मा । छन्दः॑ । अ॒स्री॒विः । छन्दः॑ । प॒ङ्क्तिः । छन्दः॑ । उ॒ष्णिहा᳚ । छन्दः॑ । बृ॒ह॒ती । छन्दः॑ । अ॒नु॒ष्टुबित्य॑नु - स्तुप् । छन्दः॑ । वि॒राडिति॑ वि - राट् । छन्दः॑ । गा॒य॒त्री । छन्दः॑ । त्रि॒ष्टुप् । छन्दः॑ । जग॑ती । छन्दः॑ । पृ॒थि॒वी । छन्दः॑ । अ॒न्तरि॑क्षम् । छन्दः॑ । द्यौः । छन्दः॑ । समाः᳚ । छन्दः॑ । नक्ष॑त्राणि । छन्दः॑ । मनः॑ । छन्दः॑ । वाक् । छन्दः॑ । कृ॒षिः । छन्दः॑ । हिर॑ण्यम् । छन्दः॑ । गौः । छन्दः॑ । अ॒जा । छन्दः॑ । अश्वः॑ । छन्दः॑ ॥ अ॒ग्निः । दे॒वता᳚ । \textbf{  13} \newline
                  \newline
                                \textbf{ TS 4.3.7.2} \newline
                  वातः॑ । दे॒वता᳚ । सूर्यः॑ । दे॒वता᳚ । च॒न्द्रमाः᳚ । दे॒वता᳚ । वस॑वः । दे॒वता᳚ । रु॒द्राः । दे॒वता᳚ । आ॒दि॒त्याः । दे॒वता᳚ । विश्वे᳚ । दे॒वाः । दे॒वता᳚ । म॒रुतः॑ । दे॒वता᳚ । बृह॒स्पतिः॑ । दे॒वता᳚ । इन्द्रः॑ । दे॒वता᳚ । वरु॑णः । दे॒वता᳚ । मू॒द्‌र्धा । अ॒सि॒ । राट् । ध्रु॒वा । अ॒सि॒ । ध॒रुणा᳚ । य॒न्त्री । अ॒सि॒ । यमि॑त्री । इ॒षे । त्वा॒ । ऊ॒र्जे । त्वा॒ । कृ॒ष्यै । त्वा॒ । क्षेमा॑य । त्वा॒ । यन्त्री᳚ । राट् । ध्रु॒वा । अ॒सि॒ । धर॑णी । ध॒र्त्री । अ॒सि॒ । धरि॑त्री । आयु॑षे । त्वा॒ ( ) । वर्च॑से । त्वा॒ । ओज॑से । त्वा॒ । बला॑य । त्वा॒ ॥ 14 \textbf{  14} \newline
                  \newline
                      (दे॒वता - ऽऽयु॑षे त्वा॒ - षट् च॑ )  \textbf{(A7)} \newline \newline
                                \textbf{ TS 4.3.8.1} \newline
                  आ॒शुः । त्रि॒वृदिति॑ त्रि-वृत् । भा॒न्तः । प॒ञ्च॒द॒श इति॑ पञ्च-द॒शः । व्यो॑मेति॒ वि - ओ॒म॒ । स॒प्त॒द॒श इति॑ सप्त - द॒शः । प्रतू᳚र्ति॒रिति॒ प्र-तू॒र्तिः॒ । अ॒ष्टा॒द॒श इत्य॑ष्टा-द॒शः । तपः॑ । न॒व॒द॒श इति॑ नव-द॒शः । अ॒भि॒व॒र्त इत्य॑भि - व॒र्तः । स॒विꣳ॒॒श इति॑ स - विꣳ॒॒शः । ध॒रुणः॑ । ए॒क॒विꣳ॒॒श इत्ये॑क - विꣳ॒॒शः । वर्चः॑ । द्वा॒विꣳ॒॒शः । स॒भंर॑ण॒ इति॑ सं - भर॑णः । त्र॒यो॒विꣳ॒॒श इति॑ त्रयः - विꣳ॒॒शः । योनिः॑ । च॒तु॒र्विꣳ॒॒श इति॑ चतुः - विꣳ॒॒शः । गर्भाः᳚ । प॒ञ्च॒विꣳ॒॒श इति॑ पञ्च - विꣳ॒॒शः । ओजः॑ । त्रि॒ण॒व इति॑ त्रि - न॒वः । क्रतुः॑ । ए॒क॒त्रिꣳ॒॒श इत्ये॑क - त्रिꣳ॒॒शः । प्र॒ति॒ष्ठेति॑ प्रति - स्था । त्र॒य॒स्त्रिꣳ॒॒श इति॑ त्रयः - त्रिꣳ॒॒शः । ब्र॒द्ध्नस्य॑ । वि॒ष्टप᳚म् । च॒तु॒स्त्रिꣳ॒॒श इति॑ चतुः - त्रिꣳ॒॒शः । नाकः॑ । ष॒ट्त्रिꣳ॒॒श इति॑ षट् - त्रिꣳ॒॒शः । वि॒व॒र्त इति॑ वि - व॒र्तः । अ॒ष्टा॒च॒त्वा॒रिꣳ॒॒श इत्य॑ष्टा-च॒त्वा॒रिꣳ॒॒शः । ध॒र्त्रः । च॒तु॒ष्टो॒म इति॑ चतुः-स्तो॒मः ॥ \textbf{  15} \newline
                  \newline
                      (आ॒शुः - स॒प्तत्रिꣳ॑शत्)  \textbf{(A8)} \newline \newline
                                \textbf{ TS 4.3.9.1} \newline
                  अ॒ग्नेः । भा॒गः । अ॒सि॒ । दी॒क्षायाः᳚ । आधि॑पत्य॒मित्याधि॑ - प॒त्य॒म् । ब्रह्म॑ । स्पृ॒तम् । त्रि॒वृदिति॑ त्रि - वृत् । स्तोमः॑ । इन्द्र॑स्य । भा॒गः । अ॒सि॒ । विष्णोः᳚ । आधि॑पत्य॒मित्याधि॑ - प॒त्य॒म् । क्ष॒त्रम् । स्पृ॒तम् । प॒ञ्च॒द॒श इति॑ पञ्च - द॒शः । स्तोमः॑ । नृ॒चक्ष॑सा॒मिति॑ नृ - चक्ष॑साम् । भा॒गः । अ॒सि॒ । धा॒तुः । आधि॑पत्य॒मित्याधि॑ - प॒त्य॒म् । ज॒नित्र᳚म् । स्पृ॒तम् । स॒प्त॒द॒श इति॑ सप्त - द॒शः । स्तोमः॑ । मि॒त्रस्य॑ । भा॒गः । अ॒सि॒ । वरु॑णस्य । आधि॑पत्य॒मित्याधि॑ - प॒त्य॒म् । दि॒वः । वृ॒ष्टिः । वाताः᳚ । स्पृ॒ताः । ए॒क॒विꣳ॒॒श इत्ये॑क - विꣳ॒॒शः । स्तोमः॑ । अदि॑त्यै । भा॒गः । अ॒सि॒ । पू॒ष्णः । आधि॑पत्य॒मित्याधि॑ - प॒त्य॒म् । ओजः॑ । स्पृ॒तम् । त्रि॒ण॒व इति॑ त्रि - न॒वः । स्तोमः॑ । वसू॑नाम् । भा॒गः । अ॒सि॒ । \textbf{  16} \newline
                  \newline
                                \textbf{ TS 4.3.9.2} \newline
                  रु॒द्राणा᳚म् । आधि॑पत्य॒मित्याधि॑ - प॒त्य॒म् । चतु॑ष्पा॒दिति॒ चतुः॑-पा॒त् । स्पृ॒तम् । च॒तु॒र्विꣳ॒॒श इति॑ चतुः-विꣳ॒॒शः । स्तोमः॑ । आ॒दि॒त्याना᳚म् । भा॒गः । अ॒सि॒ । म॒रुता᳚म् । आधि॑पत्य॒मित्याधि॑ - प॒त्य॒म् । गर्भाः᳚ । स्पृ॒ताः । प॒ञ्च॒विꣳ॒॒श इति॑ पञ्च - विꣳ॒॒शः । स्तोमः॑ । दे॒वस्य॑ । स॒वि॒तुः । भा॒गः । अ॒सि॒ । बृह॒स्पतेः᳚ । आधि॑पत्य॒मित्याधि॑ - प॒त्य॒म् । स॒मीचीः᳚ । दिशः॑ । स्पृ॒ताः । च॒तु॒ष्टो॒म इति॑ चतुः-स्तो॒मः । स्तोमः॑ । यावा॑नाम् । भा॒गः । अ॒सि॒ । अया॑वानाम् । आधि॑पत्य॒मित्याधि॑ - प॒त्य॒म् । प्र॒जा इति॑ प्र - जाः । स्पृ॒ताः । च॒तु॒श्च॒त्वा॒रिꣳ॒॒श इति॑ चतुः - च॒त्वा॒रिꣳ॒॒शः । स्तोमः॑ । ऋ॒भू॒णाम् । भा॒गः । अ॒सि॒ । विश्वे॑षाम् । दे॒वाना᳚म् । आधि॑पत्य॒मित्याधि॑-प॒त्य॒म् । भू॒तम् । निशा᳚न्त॒मिति॒ नि - शा॒न्त॒म् । स्पृ॒तम् । त्र॒य॒स्त्रिꣳ॒॒श इति॑ त्रयः - त्रिꣳ॒॒शः । स्तोमः॑ ॥ \textbf{  17} \newline
                  \newline
                      (वसू॑नां भा॒गो॑ऽसि॒ - षट्च॑त्वारिꣳशच्च)  \textbf{(A9)} \newline \newline
                                \textbf{ TS 4.3.10.1} \newline
                  एक॑या । अ॒स्तु॒व॒त॒ । प्र॒जा इति॑ प्र - जाः । अ॒धी॒य॒न्त॒ । प्र॒जाप॑ति॒रिति॑ प्र॒जा - प॒तिः॒ । अधि॑पति॒रित्यधि॑ - प॒तिः॒ । आ॒सी॒त् । ति॒सृभि॒रिति॑ ति॒सृ - भिः॒ । अ॒स्तु॒व॒त॒ । ब्रह्म॑ । अ॒सृ॒ज्य॒त॒ । ब्रह्म॑णः । पतिः॑ । अधि॑पति॒रित्यधि॑-प॒तिः॒ । आ॒सी॒त् । प॒ञ्चभि॒रिति॑ प॒ञ्च-भिः॒ । अ॒स्तु॒व॒त॒ । भू॒तानि॑ । अ॒सृ॒ज्य॒न्त॒ । भू॒ताना᳚म् । पतिः॑ । अधि॑पति॒रित्यधि॑ - प॒तिः॒ । आ॒सी॒त् । स॒प्तभि॒रिति॑ स॒प्त - भिः॒ । अ॒स्तु॒व॒त॒ । स॒प्त॒र्॒.षय॒ इति॑ सप्त - ऋ॒षयः॑ । अ॒सृ॒ज्य॒न्त॒ । धा॒ता । अधि॑पति॒रित्यधि॑ - प॒तिः॒ । आ॒सी॒त् । न॒वभि॒रिति॑ न॒व - भिः॒ । अ॒स्तु॒व॒त॒ । पि॒तरः॑ । अ॒सृ॒ज्य॒न्त॒ । अदि॑तिः । अधि॑प॒त्नीत्यधि॑-प॒त्नी॒ । आ॒सी॒त् । ए॒का॒द॒शभि॒रित्ये॑काद॒श - भिः॒ । अ॒स्तु॒व॒त॒ । ऋ॒तवः॑ । अ॒सृ॒ज्य॒न्त॒ । आ॒र्त॒वः । अधि॑पति॒रित्यधि॑ - प॒तिः॒ । आ॒सी॒त् । त्र॒यो॒द॒शभि॒रिति॑ त्रयोद॒श - भिः॒ । अ॒स्तु॒व॒त॒ । मासाः᳚ । अ॒सृ॒ज्य॒न्त॒ । सं॒ॅव॒थ्स॒र इति॑ सं - व॒थ्स॒रः । अधि॑पति॒रित्यधि॑ - प॒तिः॒ । \textbf{  18} \newline
                  \newline
                                \textbf{ TS 4.3.10.2} \newline
                  आ॒सी॒त् । प॒ञ्च॒द॒शभि॒रिति॑ पञ्चद॒श -भिः॒ । अ॒स्तु॒व॒त॒ । क्ष॒त्रम् । अ॒सृ॒ज्य॒त॒ । इन्द्रः॑ । अधि॑पति॒रित्यधि॑ - प॒तिः॒ । आ॒सी॒त् । स॒प्त॒द॒शभि॒रिति॑ सप्तद॒श-भिः॒ । अ॒स्तु॒व॒त॒ । प॒शवः॑ । अ॒सृ॒ज्य॒न्त॒ । बृह॒स्पतिः॑ । अधि॑पति॒रित्यधि॑ - प॒तिः॒ । आ॒सी॒त् । न॒व॒द॒शभि॒रिति॑ नवद॒श -भिः॒ । अ॒स्तु॒व॒त॒ । शू॒द्रा॒र्याविति॑ शूद्र - अ॒र्यौ । अ॒सृ॒ज्ये॒ता॒म् । अ॒हो॒रा॒त्रे इत्य॑हः - रा॒त्रे । अधि॑पत्नी॒ इत्यधि॑- प॒त्नी॒ । आ॒स्ता॒म् । एक॑विꣳश॒त्येत्येक॑ - विꣳ॒॒श॒त्या॒ । अ॒स्तु॒व॒त । एक॑शफा॒ इत्येक॑ - श॒फाः॒ । प॒शवः॑ । अ॒सृ॒ज्य॒न्त॒ । वरु॑णः । अधि॑पति॒रित्यधि॑ - प॒तिः॒ । आ॒सी॒त् । त्रयो॑विꣳश॒त्येति॒ त्रयः॑-विꣳ॒॒श॒त्या॒ । अ॒स्तु॒व॒त॒ । क्षु॒द्राः । प॒शवः॑ । अ॒सृ॒ज्य॒न्त॒ । पू॒षा । अधि॑पति॒रित्यधि॑ - प॒तिः॒ । आ॒सी॒त् । पञ्च॑विꣳश॒त्येति॒ पञ्च॑ - विꣳ॒॒श॒त्या॒ । अ॒स्तु॒व॒त॒ । आ॒र॒ण्याः । प॒शवः॑ । अ॒सृ॒ज्य॒न्त॒ । वा॒युः । अधि॑पति॒रित्यधि॑ - प॒तिः॒ । आ॒सी॒त् । स॒प्तविꣳ॑श॒त्येति॑ स॒प्त - विꣳ॒॒श॒त्या॒ । अ॒स्तु॒व॒त॒ । द्यावा॑पृथि॒वी इति॒ द्यावा᳚ - पृ॒थि॒वी । वीति॑ । \textbf{  19} \newline
                  \newline
                                \textbf{ TS 4.3.10.3} \newline
                  ए॒ता॒म् । वस॑वः । रु॒द्राः । आ॒दि॒त्याः । अनु॑ । वीति॑ । आ॒य॒न्न् । तेषा᳚म् । आधि॑पत्य॒मित्याधि॑ - प॒त्य॒म् । आ॒सी॒त् । नव॑विꣳश॒त्येति॒ नव॑ - विꣳ॒॒श॒त्या॒ । अ॒स्तु॒व॒त॒ । वन॒स्पत॑यः । अ॒सृ॒ज्य॒न्त॒ । सोमः॑ । अधि॑पति॒रित्यधि॑ - प॒तिः॒ । आ॒सी॒त् । एक॑त्रिꣳश॒तेत्ये॑क-त्रिꣳ॒॒श॒ता॒ । अ॒स्तु॒व॒त॒ । प्र॒जा इति॑ प्र - जाः । अ॒सृ॒ज्य॒न्त॒ । यावा॑नाम् । च॒ । अया॑वानाम् । च॒ । आधि॑पत्य॒मित्याधि॑ - प॒त्य॒म् । आ॒सी॒त् । त्रय॑स्त्रिꣳश॒तेति॒ त्रयः॑-त्रिꣳ॒॒श॒ता॒ । अ॒स्तु॒व॒त॒ । भू॒तानि॑ । अ॒शा॒म्य॒न्न् । प्र॒जाप॑ति॒रिति॑ प्र॒जा - प॒तिः॒ । प॒र॒मे॒ष्ठी । अधि॑पति॒रित्यधि॑ - प॒तिः॒ । आ॒सी॒त् ॥ \textbf{  20} \newline
                  \newline
                      (सं॒ॅव॒थ्स॒रोऽधि॑पति॒- र्वि - पञ्च॑त्रिꣳशच्च)  \textbf{(A10)} \newline \newline
                                \textbf{ TS 4.3.11.1} \newline
                  इ॒यम् । ए॒व । सा । या । प्र॒थ॒मा । व्यौच्छ॒दिति॑ वि - औच्छ॑त् । अ॒न्तः । अ॒स्याम् । च॒र॒ति॒ । प्रवि॒ष्टेति॒ प्र - वि॒ष्टा॒ ॥ व॒धूः । ज॒जा॒न॒ । न॒व॒गदिति॑ नव - गत् । जनि॑त्री । त्रयः॑ । ए॒ना॒म् । म॒हि॒मानः॑ । स॒च॒न्ते॒ ॥ छन्द॑स्वती॒ इति॑ । उ॒षसा᳚ । पेपि॑शाने॒ इति॑ । स॒मा॒नम् । योनि᳚म् । अन्विति॑ । स॒ञ्चर॑न्ती॒ इति॑ सं - चर॑न्ती ॥ सूर्य॑पत्नी॒ इति॒ सूर्य॑ - प॒त्नी॒ । वीति॑ । च॒र॒तः॒ । प्र॒जा॒न॒ती इति॑ प्र-जा॒न॒ती । के॒तुम् । कृ॒ण्वा॒ने इति॑ । अ॒जरे॒ इति॑ । भूरि॑रे॒तसेति॒ भूरि॑ - रे॒त॒सा॒ ॥ ऋ॒तस्य॑ । पन्था᳚म् । अन्विति॑ । ति॒स्रः । एति॑ । अ॒गुः॒ । त्रयः॑ । घ॒र्मासः॑ । अन्विति॑ । ज्योति॑षा । एति॑ । अ॒गुः॒ ॥ प्र॒जामिति॑ प्र - जाम् । एका᳚ । रक्ष॑ति । ऊर्ज᳚म् । एका᳚ । \textbf{  21} \newline
                  \newline
                                \textbf{ TS 4.3.11.2} \newline
                  व्र॒तम् । एका᳚ । र॒क्ष॒ति॒ । दे॒व॒यू॒नामिति॑ देव - यू॒नाम् ॥ च॒तु॒ष्टो॒म इति॑ चतुः-स्तो॒मः । अ॒भ॒व॒त् । या । तु॒रीया᳚ । य॒ज्ञ्स्य॑ । प॒क्षौ । ऋ॒ष॒यः॒ । भव॑न्ती ॥ गा॒य॒त्रीम् । त्रि॒ष्टुभ᳚म् । जग॑तीम् । अ॒नु॒ष्टुभ॒मित्य॑नु-स्तुभ᳚म् । बृ॒हत् । अ॒र्कम् । यु॒ञ्जा॒नाः । सुवः॑ । एति॑ । अ॒भ॒र॒न्न् । इ॒दम् ॥ प॒ञ्चभि॒रिति॑ प॒ञ्च - भिः॒ । धा॒ता । वीति॑ । द॒धौ॒ । इ॒दम् । यत् । तासा᳚म् । स्वसॄः᳚ । अ॒ज॒न॒य॒त् । पञ्च॑प॒ञ्चेति॒ पञ्च॑ - प॒ञ्च॒ ॥ तासा᳚म् । उ॒ । य॒न्ति॒ । प्र॒य॒वेणेति॑ प्र - य॒वेन॑ । पञ्च॑ । नाना᳚ । रू॒पाणि॑ । क्रत॑वः । वसा॑नाः ॥ त्रिꣳ॒॒शत् । स्वसा॑रः । उपेति॑ । य॒न्ति॒ । नि॒ष्कृ॒तमिति॑ निः - कृ॒तम् । स॒मा॒नम् । के॒तुम् । प्र॒ति॒मु॒ञ्चमा॑ना॒ इति॑ प्रति - मु॒ञ्चमा॑नाः ॥ \textbf{  22} \newline
                  \newline
                                \textbf{ TS 4.3.11.3} \newline
                  ऋ॒तून् । त॒न्व॒ते॒ । क॒वयः॑ । प्र॒जा॒न॒तीरिति॑ प्र - जा॒न॒तीः । मद्ध्ये॑छन्दस॒ इति॒ मद्ध्ये᳚ - छ॒न्द॒सः॒ । परीति॑ । य॒न्ति॒ । भास्व॑तीः ॥ ज्योति॑ष्मती । प्रतीति॑ । मु॒ञ्च॒ते॒ । नभः॑ । रात्री᳚ । दे॒वी । सूर्य॑स्य । व्र॒तानि॑ ॥ वीति॑ । प॒श्य॒न्ति॒ । प॒शवः॑ । जाय॑मानाः । नाना॑रूपा॒ इति॒ नाना᳚ - रू॒पाः॒ । मा॒तुः । अ॒स्याः । उ॒पस्थ॒ इत्यु॒प - स्थे॒ ॥ ए॒का॒ष्ट॒केत्ये॑क- अ॒ष्ट॒का । तप॑सा । तप्य॑माना । ज॒जान॑ । गर्भ᳚म् । म॒हि॒मान᳚म् । इन्द्र᳚म् ॥ तेन॑ । दस्यून्॑ । वीति॑ । अ॒स॒ह॒न्त॒ । दे॒वाः । ह॒न्ता । असु॑राणाम् । अ॒भ॒व॒त् । शची॑भि॒रिति॒ शचि॑ - भिः॒ ॥ अना॑नुजा॒मित्यना॑नु - जा॒म् । अ॒नु॒जामित्य॑नु-जाम् । माम् । अ॒क॒र्त॒ । स॒त्यम् । वद॑न्ती । अन्विति॑ । इ॒च्छे॒ । ए॒तत् ॥ भू॒यास᳚म् । \textbf{  23} \newline
                  \newline
                                \textbf{ TS 4.3.11.4} \newline
                  अ॒स्य॒ । सु॒म॒ताविति॑ सु - म॒तौ । यथा᳚ । यू॒यम् । अ॒न्या । वः॒ । अ॒न्याम् । अतीति॑ । मा । प्रेति॑ । यु॒क्त॒ ॥ अभू᳚त् । मम॑ । सु॒म॒ताविति॑ सु - म॒तौ । वि॒श्ववे॑दा॒ इति॑ वि॒श्व - वे॒दाः॒ । आष्ट॑ । प्र॒ति॒ष्ठामिति॑ प्रति - स्थाम् । अवि॑दत् । हि । गा॒धम् ॥ भू॒यास᳚म् । अ॒स्य॒ । सु॒म॒ताविति॑ सु - म॒तौ । यथा᳚ । यू॒यम् । अ॒न्या । वः॒ । अ॒न्याम् । अतीति॑ । मा । प्रेति॑ । यु॒क्त॒ ॥ पञ्च॑ । व्यु॑ष्टी॒रिति॒ वि - उ॒ष्टीः॒ । अन्विति॑ । पञ्च॑ । दोहाः᳚ । गाम् । पञ्च॑नाम्नी॒मिति॒ पञ्च॑ - ना॒म्नी॒म् । ऋ॒तवः॑ । अन्विति॑ । पञ्च॑ ॥ पञ्च॑ । दिशः॑ । प॒ञ्च॒द॒शेनेति॑ पञ्च - द॒शेन॑ । क्लृ॒प्ताः । स॒मा॒नमू᳚द्‌र्ध्नी॒रिति॑ समा॒न - मू॒द्‌र्ध्नीः॒ । अ॒भीति॑ । लो॒कम् । एक᳚म् ॥ \textbf{  24} \newline
                  \newline
                                \textbf{ TS 4.3.11.5} \newline
                  ऋ॒तस्य॑ । गर्भः॑ । प्र॒थ॒मा । व्यू॒षुषीति॑ वि - ऊ॒षुषी᳚ । अ॒पाम् । एका᳚ । म॒हि॒मान᳚म् । बि॒भ॒र्ति॒ ॥ सूर्य॑स्य । एका᳚ । चर॑ति । नि॒ष्कृ॒तेष्विति॑ निः-कृ॒तेषु॑ । घ॒र्मस्य॑ । एका᳚ । स॒वि॒ता । एका᳚म् । नीति॑ । य॒च्छ॒ति॒ ॥ या । प्र॒थ॒मा । व्यौच्छ॒दिति॑ वि - औच्छ॑त् । सा । धे॒नुः । अ॒भ॒व॒त् । य॒मे ॥ सा । नः॒ । पय॑स्वती । धु॒क्ष्व॒ । उत्त॑रामुत्तरा॒मित्युत्त॑रां-उ॒त्त॒रा॒म् । समा᳚म् ॥ शु॒क्रर्.ष॒भेति॑ शु॒क्र - ऋ॒ष॒भा॒ । नभ॑सा । ज्योति॑षा । एति॑ । अ॒गा॒त् । वि॒श्वरू॒पेति॑ वि॒श्व - रू॒पा॒ । श॒ब॒लीः । अ॒ग्निके॑तु॒रित्य॒ग्नि - के॒तुः॒ ॥ स॒मा॒नम् । अर्थ᳚म् । स्व॒प॒स्यमा॒नेति॑ सु-अ॒प॒स्यमा॑ना । बिभ्र॑ती । ज॒राम् । अ॒ज॒रे॒ । उ॒षः॒ । एति॑ । अ॒गाः॒ ॥ ऋ॒तू॒नाम् । पत्नी᳚ ( ) । प्र॒थ॒मा । इ॒यम् । एति॑ । अ॒गा॒त् । अह्ना᳚म् । ने॒त्री । ज॒नि॒त्री । प्र॒जाना॒मिति॑ प्र - जाना᳚म् ॥ एका᳚ । स॒ती । ब॒हु॒धेति॑ बहु - धा । उ॒षः॒ । वीति॑ । उ॒च्छ॒सि॒ । अजी᳚र्णा । त्वम् । ज॒र॒य॒सि॒ । सर्व᳚म् । अ॒न्यत् ॥ \textbf{  25 } \newline
                  \newline
                      (ऊर्ज॒मेका᳚ - प्रतिमु॒ञ्चमा॑ना - भू॒यास॒ - मेकं॒ - पत्न्ये का॒न्न विꣳ॑श॒तिश्च॑)  \textbf{(A11)} \newline \newline
                                \textbf{ TS 4.3.12.1} \newline
                  अग्ने᳚ । जा॒तान् । प्रेति॑ । नु॒द॒ । नः॒ । स॒पत्नान्॑ । प्रतीति॑ । अजा॑तान् । जा॒त॒वे॒द॒ इति॑ जात - वे॒दः॒ । नु॒द॒स्व॒ ॥ अ॒स्मे इति॑ । दी॒दि॒हि॒ । सु॒मना॒ इति॑ सु - मनाः᳚ । अहे॑डन्न् । तव॑ । स्या॒म् । शर्मन्न्॑ । त्रि॒वरू॑थ॒ इति॑ त्रि-वरू॑थः । उ॒द्भिदित्यु॑त् - भित् ॥ सह॑सा । जा॒तान् । प्रेति॑ । नु॒द॒ । नः॒ । स॒पत्नान्॑ । प्रतीति॑ । अजा॑तान् । जा॒त॒वे॒द॒ इति॑ जात - वे॒दः॒ । नु॒द॒स्व॒ ॥ अधीति॑ । नः॒ । ब्रू॒हि॒ । सु॒म॒न॒स्यमा॑न॒ इति॑ सु - म॒न॒स्यमा॑नः । व॒यम् । स्या॒म॒ । प्रेति॑ । नु॒द॒ । नः॒ । स॒पत्नान्॑ ॥ च॒तु॒श्च॒त्वा॒रिꣳ॒॒श इति॑ चतुः - च॒त्वा॒रिꣳ॒॒शः । स्तोमः॑ । वर्चः॑ । द्रवि॑णम् । षो॒ड॒शः । स्तोमः॑ । ओजः॑ । द्रवि॑णम् । पृ॒थि॒व्याः । पुरी॑षम् । अ॒सि॒ । \textbf{  26} \newline
                  \newline
                                \textbf{ TS 4.3.12.2} \newline
                  अफ्सः॑ । नाम॑ ॥ एवः॑ । छन्दः॑ । वरि॑वः । छन्दः॑ । श॒भूंरिति॑ शं - भूः । छन्दः॑ । प॒रि॒भूरिति॑ परि - भूः । छन्दः॑ । आ॒च्छत् । छन्दः॑ । मनः॑ । छन्दः॑ । व्यचः॑ । छन्दः॑ । सिन्धुः॑ । छन्दः॑ । स॒मु॒द्रम् । छन्दः॑ । स॒लि॒लम् । छन्दः॑ । सं॒ॅयदिति॑ सं - यत् । छन्दः॑ । वि॒यदिति॑ वि - यत् । छन्दः॑ । बृ॒हत् । छन्दः॑ । र॒थ॒न्त॒रमिति॑ रथं - त॒रम् । छन्दः॑ । नि॒का॒य इति॑ नि - का॒यः । छन्दः॑ । वि॒व॒ध इति॑ वि-व॒धः । छन्दः॑ । गिरः॑ । छन्दः॑ । भ्रजः॑ । छन्दः॑ । स॒ष्टुबिति॑ स - स्तुप् । छन्दः॑ । अ॒नु॒ष्टुबित्य॑नु - स्तुप् । छन्दः॑ । क॒कुत् । छन्दः॑ । त्रि॒क॒कुदिति॑ त्रि - क॒कुत् । छन्दः॑ । का॒व्यम् । छन्दः॑ । अ॒ङ्कु॒पम् । छन्दः॑ । \textbf{  27} \newline
                  \newline
                                \textbf{ TS 4.3.12.3} \newline
                  प॒दप॑ङ्क्ति॒रिति॑ प॒द - प॒ङ्क्तिः॒ । छन्दः॑ । अ॒क्षर॑पङ्क्ति॒रित्य॒क्षर॑ - प॒ङ्क्तिः॒ । छन्दः॑ । वि॒ष्टा॒रप॑ङ्क्ति॒रिति॑ विष्टा॒र - प॒ङ्क्तिः॒ । छन्दः॑ । क्षु॒रः । भृज्वान्॑ । छन्दः॑ । प्र॒च्छत् । छन्दः॑ । प॒क्षः । छन्दः॑ । एवः॑ । छन्दः॑ । वरि॑वः । छन्दः॑ । वयः॑ । छन्दः॑ । व॒य॒स्कृदिति॑ वयः - कृत् । छन्दः॑ । वि॒शा॒लमिति॑ वि - शा॒लम् । छन्दः॑ । विष्प॑र्धा॒ इति॒ विः-स्प॒द्‌र्धाः॒ । छन्दः॑ । छ॒दिः । छन्दः॑ । दू॒रो॒ह॒णमिति॑ दूः - रो॒ह॒णम् । छन्दः॑ । त॒न्द्रम् । छन्दः॑ । अ॒ङ्का॒ङ्कम् । छन्दः॑ ॥ \textbf{  28 } \newline
                  \newline
                      (अ॒स्य॒ - ङ्कु॒पञ्छन्द॒ - स्त्रय॑स्त्रिꣳशच्च)  \textbf{(A12)} \newline \newline
                                \textbf{ TS 4.3.13.1} \newline
                  अ॒ग्निः । वृ॒त्राणि॑ । ज॒ङ्घ॒न॒त् । द्र॒वि॒ण॒स्युः । वि॒प॒न्ययेति॑ वि - प॒न्यया᳚ ॥ समि॑द्ध॒ इति॑ सम् - इ॒द्धः॒ । शु॒क्रः । आहु॑त॒ इत्या - हु॒तः॒ ॥ त्वम् । सो॒म॒ । अ॒सि॒ । सत्प॑ति॒रिति॒ सत् - प॒तिः॒ । त्वम् । राजा᳚ । उ॒त । वृ॒त्र॒हेति॑ वृत्र - हा ॥ त्वम् । भ॒द्रः । अ॒सि॒ । क्रतुः॑ ॥ भ॒द्रा । ते॒ । अ॒ग्ने॒ । स्व॒नी॒केति॑ सु - अ॒नी॒क॒ । स॒न्दृगेति॑ सं - दृक् । घो॒रस्य॑ । स॒तः । विषु॑णस्य । चारुः॑ ॥ न । यत् । ते॒ । शो॒चिः । तम॑सा । वर॑न्त । न । ध्व॒स्मानः॑ । त॒नुवि॑ । रेपः॑ । एति॑ । धुः॒ ॥ भ॒द्रम् । ते॒ । अ॒ग्ने॒ । स॒ह॒सि॒न्न् । अनी॑कम् । उ॒पा॒के । एति॑ । रो॒च॒ते॒ । सूर्य॑स्य ॥ \textbf{  29} \newline
                  \newline
                                \textbf{ TS 4.3.13.2} \newline
                  रुश॑त् । दृ॒शे । द॒दृ॒शे॒ । न॒क्त॒या । चि॒त् । अरू᳚क्षितम् । दृ॒शे । एति॑ । रू॒पे । अन्न᳚म् ॥ सः । ए॒ना । अनी॑केन । सु॒वि॒दत्र॒ इति॑ सु-वि॒दत्रः॑ । अ॒स्मे इति॑ । यष्टा᳚ । दे॒वान् । आय॑जिष्ठ॒ इत्या - य॒जि॒ष्ठः॒ । स्व॒स्ति ॥ अद॑ब्धः । गो॒पा इति गो-पाः । उ॒त । नः॒ । प॒र॒स्पा इति॑ परः-पाः । अग्ने᳚ । द्यु॒मदिति॑ द्यु - मत् । उ॒त । रे॒वत् । दि॒दी॒हि॒ ॥ स्व॒स्ति । नः॒ । दि॒वः । अ॒ग्ने॒ । पृ॒थि॒व्याः । वि॒श्वायु॒रिति॑ वि॒श्व - आ॒युः॒ । धे॒हि॒ । य॒जथा॑य । दे॒व॒ ॥ यत् । सी॒महि॑ । दि॒वि॒जा॒तेति॑ दिवि - जा॒त॒ । प्रश॑स्त॒मिति॒ प्र - श॒स्त॒म् । तत् । अ॒स्मासु॑ । द्रवि॑णम् । धे॒हि॒ । चि॒त्रम् ॥ यथा᳚ । हो॒तः॒ । मनु॑षः । \textbf{  30} \newline
                  \newline
                                \textbf{ TS 4.3.13.3} \newline
                  दे॒वता॒तेति॑ दे॒व - ता॒ता॒ । य॒ज्ञेभिः॑ । सू॒नो॒ इति॑ । स॒ह॒सः॒ । यजा॑सि ॥ ए॒वा । नः॒ । अ॒द्य । स॒म॒ना । स॒मा॒नान् । उ॒शन्न् । अ॒ग्ने॒ । उ॒श॒तः । य॒क्षि॒ । दे॒वान् ॥ अ॒ग्निम् । ई॒डे॒ । पु॒रोहि॑त॒मिति॑ पु॒रः - हि॒त॒म् । य॒ज्ञ्स्य॑ । दे॒वम् । ऋ॒त्विज᳚म् ॥ होता॑रम् । र॒त्न॒धात॑म॒मिति॑ रत्न - धात॑मम् ॥ वृषा᳚ । सो॒म॒ । द्यु॒मानिति॑ द्यु-मान् । अ॒सि॒ । वृषा᳚ । दे॒व॒ । वृष॑व्रत॒ इति॒ वृष॑ - व्र॒तः॒ ॥ वृषा᳚ । धर्मा॑णि । द॒धि॒षे॒ ॥ सान्त॑पना॒ इति॒ सां - त॒प॒नाः॒ । इ॒दम् । ह॒विः । मरु॑तः । तत् । जु॒जु॒ष्ट॒न॒ ॥ यु॒ष्माक॑ । ऊ॒ती । रि॒शा॒द॒स॒ इति॑ रिश - अ॒द॒सः॒ ॥ यः । नः॒ । मर्तः॑ । व॒स॒वः॒ । दु॒र्॒.हृ॒णा॒युरिति॑ दुः - हृ॒णा॒युः । ति॒रः । स॒त्यानि॑ । म॒रु॒तः॒ । \textbf{  31 } \newline
                  \newline
                                \textbf{ TS 4.3.13.4} \newline
                  जिघाꣳ॑सात् ॥ द्रु॒हः । पाश᳚म् । प्रतीति॑ । सः । मु॒ची॒ष्ट॒ । तपि॑ष्ठेन । तप॑सा । ह॒न्त॒न॒ । तम् ॥ सं॒ॅव॒थ्स॒रीणा॒ इति॑ सं - व॒थ्स॒रीणाः᳚ । म॒रुतः॑ । स्व॒र्का इति॑ सु - अ॒र्काः । उ॒रु॒क्षया॒ इत्यु॑रु - क्षयाः᳚ । सग॑णा॒ इति॒ स - ग॒णाः॒ । मानु॑षेषु ॥ ते । अ॒स्मत् । पाशान्॑ । प्रेति॑ । मु॒ञ्च॒न्तु॒ । अꣳह॑सः । सा॒न्त॒प॒ना इति॑ सां - त॒प॒नाः । म॒दि॒राः । मा॒द॒यि॒ष्णवः॑ ॥ पि॒प्री॒हि । दे॒वान् । उ॒श॒तः । य॒वि॒ष्ठ॒ । वि॒द्वान् । ऋ॒तून् । ऋ॒तु॒प॒त॒ इत्यृ॑तु-प॒ते॒ । य॒ज॒ । इ॒ह ॥ ये । दैव्याः᳚ । ऋ॒त्विजः॑ । तेभिः॑ । अ॒ग्ने॒ । त्वम् । होतॄ॑णाम् । अ॒सि॒ । आय॑जिष्ठ॒ इत्या - य॒जि॒ष्ठः॒ ॥ अग्ने᳚ । यत् । अ॒द्य । वि॒शः । अ॒द्ध्व॒र॒स्य॒ । हो॒तः॒ । पाव॑क । \textbf{  32} \newline
                  \newline
                                \textbf{ TS 4.3.13.5} \newline
                  शो॒चे॒ । वेः । त्वम् । हि । यज्वा᳚ ॥ ऋ॒ता । य॒जा॒सि॒ । म॒हि॒ना । वीति॑ । यत् । भूः । ह॒व्या । व॒ह॒ । य॒वि॒ष्ठ॒ । या । ते॒ । अ॒द्य ॥ अ॒ग्निना᳚ । र॒यिम् । अ॒श्न॒व॒त् । पोष᳚म् । ए॒व । दि॒वेदि॑व॒ इति॑ दि॒वे - दि॒वे॒ ॥ य॒शस᳚म् । वी॒रव॑त्तम॒मिति॑ वी॒रव॑त् - त॒म॒म् ॥ ग॒य॒स्फान॒ इति॑ गय - स्फानः॑ । अ॒मी॒व॒हेत्य॑मीव - हा । व॒सु॒विदिति॑ वसु - वित् । पु॒ष्टि॒वद्‌र्ध॑न॒ इति॑ पुष्टि - वर्ध॑नः ॥ सु॒मि॒त्र इति॑ सु - मि॒त्रः । सो॒म॒ । नः॒ । भ॒व॒ ॥ गृह॑मेधास॒ इति॒ गृह॑ - मे॒धा॒सः॒ । एति॑ । ग॒त॒ । मरु॑तः । मा । अपेति॑ । भू॒त॒न॒ ॥ प्र॒मु॒ञ्चन्त॒ इति॑ प्र - मु॒ञ्चन्तः॑ । नः॒ । अꣳह॑सः ॥ पू॒र्वीभिः॑ । हि । द॒दा॒शि॒म । श॒रद्भि॒रिति॑ श॒रत् - भिः॒ । म॒रु॒तः॒ । व॒यम् ॥ महो॑भि॒रिति॒ महः॑ - भिः॒ । \textbf{  33 } \newline
                  \newline
                                \textbf{ TS 4.3.13.6} \newline
                  च॒र्॒.ष॒णी॒नाम् ॥ प्रेति॑ । बु॒द्ध्निया᳚ । ई॒र॒ते॒ । वः॒ । महाꣳ॑सि । प्रेति॑ । नामा॑नि । प्र॒य॒ज्य॒व॒ इति॑ प्र - य॒ज्य॒वः॒ । ति॒र॒द्ध्व॒म् ॥ स॒ह॒स्रिय᳚म् । दम्य᳚म् । भा॒गम् । ए॒तम् । गृ॒ह॒मे॒धीय॒मिति॑ गृह - मे॒धीय᳚म् । म॒रु॒तः॒ । जु॒ष॒द्ध्व॒म् ॥ उपेति॑ । यम् । एति॑ । यु॒व॒तिः । सु॒दक्ष॒मिति॑ सु-दक्ष᳚म् । दो॒षा । वस्तोः᳚ । ह॒विष्म॑ती । घृ॒ताची᳚ ॥ उपेति॑ । स्व । ए॒न॒म् । अ॒रम॑तिः । व॒सू॒युरिति॑ वसू - युः ॥ इ॒मो इति॑ । अ॒ग्ने॒ । वी॒तत॑मा॒नीति॑ वी॒त - त॒मा॒नि॒ । ह॒व्या । अज॑स्रः । व॒क्षि॒ । दे॒वता॑ति॒मिति॑ दे॒व - ता॒ति॒म् । अच्छ॑ ॥ प्रतीति॑ । नः॒ । ई॒म् । सु॒र॒भीणि॑ । वि॒य॒न्तु॒ ॥ क्री॒डम् । वः॒ । शद्‌र्धः॑ । मारु॑तम् । अ॒न॒र्वाण᳚म् । र॒थे॒शुभ॒मिति॑ रथे - शुभ᳚म् ॥ \textbf{  34} \newline
                  \newline
                                \textbf{ TS 4.3.13.7} \newline
                  कण्वाः᳚ । अ॒भि । प्रेति॑ । गा॒य॒त॒ ॥ अत्या॑सः । न । ये । म॒रुतः॑ । स्वञ्चः॑ । य॒क्ष॒दृश॒ इति॑ यक्ष - दृशः॑ । न । शु॒भय॑न्त । मर्याः᳚ ॥ ते । ह॒र्म्ये॒ष्ठा इति॑ हर्म्ये-स्थाः । शिश॑वः । न । शु॒भ्राः । व॒थ्सासः॑ । न । प्र॒क्री॒डिन॒ इति॑ प्र - क्री॒डिनः॑ । प॒यो॒धा इति॑ पयः - धाः ॥ प्रेति॑ । ए॒षा॒म् । अज्मे॑षु । वि॒थु॒रा । इ॒व॒ । रे॒ज॒ते॒ । भूमिः॑ । यामे॑षु । यत् । ह॒ । यु॒ञ्जते᳚ । शु॒भे ॥ ते । क्री॒डयः॑ । धुन॑यः । भ्राज॑दृष्टय॒ इति॒ भ्राज॑त् - ऋ॒ष्ट॒यः॒ । स्व॒यम् । म॒हि॒त्वमिति॑ महि - त्वम् । प॒न॒य॒न्त॒ । धूत॑यः ॥ उ॒प॒ह्व॒रेष्वित्यु॑प-ह्व॒रेषु॑ । यत् । अचि॑द्ध्वम् । य॒यिम् । वयः॑ । इ॒व॒ । म॒रु॒तः॒ । केन॑ । \textbf{  35} \newline
                  \newline
                                \textbf{ TS 4.3.13.8} \newline
                  चि॒त् । प॒था ॥ श्चोत॑न्ति । कोशाः᳚ । उपेति॑ । वः॒ । रथे॑षु । एति॑ । घृ॒तम् । उ॒क्ष॒त॒ । मधु॑वर्ण॒मिति॒ मधु॑ - व॒र्ण॒म् । अर्च॑ते ॥ अ॒ग्निम॑ग्नि॒मित्य॒ग्निम् - अ॒ग्नि॒म् । हवी॑मभि॒रिति॒ हवी॑म - भिः॒ । सदा᳚ । ह॒व॒न्त॒ । वि॒श्पति᳚म् ॥ ह॒व्य॒वाह॒मिति॑ हव्य - वाह᳚म् । पु॒रु॒प्रि॒यमिति॑ पुरु - प्रि॒यम् ॥ तम् । हि । शश्व॑न्तः । ईड॑ते । स्रु॒चा । दे॒वम् । घृ॒त॒श्चुतेति॑ घृत - श्चुता᳚ ॥ अ॒ग्निम् । ह॒व्याय॑ । वोढ॑वे ॥ इन्द्रा᳚ग्नी॒ इतीन्द्र॑ - अ॒ग्नी॒ । रो॒च॒ना । दि॒वः । श्नथ॑त् । वृ॒त्रम् । इन्द्र᳚म् । वः॒ । वि॒श्वतः॑ । परीति॑ । इन्द्र᳚म् । नरः॑ । विश्व॑कर्म॒न्निति॒ विश्व॑ - क॒र्म॒न्न् । ह॒विषा᳚ । वा॒वृ॒धा॒नः । विश्व॑कर्म॒न्निति॒ विश्व॑ - क॒र्म॒न्न् । ह॒विषा᳚ । वद्‌र्ध॑नेन ॥ \textbf{  36 } \newline
                  \newline
                      (सूर्य॑स्य॒ - मनु॑षो - मरुतः॒ - पाव॑क॒ - महो॑भी - रथे॒शुभ॒ - ङ्केन॒ - षड्च॑त्वारिꣳशच्च)  \textbf{(A13)} \newline \newline
\textbf{praSna korvai with starting padams of 1 to 13 anuvAkams :-} \newline
(अ॒पान्त्वेम॑ - न्न॒यं पु॒रो भुवः॒ - प्राची᳚ - ध्रु॒वक्षि॑ति॒ - स्त्र्यवि॒ - रिन्द्रा᳚ग्नी॒ - मा छन्द॑ - आ॒शुस्त्रि॒वृ - द॒ग्नेर्भा॒गो᳚ - ऽस्येक॑ - ये॒यमे॒व सा या - ऽग्ने॑ जा॒ता - न॒ग्निर्वृ॒त्राणि॒ - त्रयो॑दश ) \newline

\textbf{korvai with starting padams of1, 11, 21 series of pa~jcAtis :-} \newline
(अ॒पान्त्वे - न्द्रा᳚ग्नी - इ॒यमे॒व - दे॒वता॑ता॒ - षट्त्रिꣳ॑शत् ) \newline

\textbf{first and last padam of third praSnam of kANDam 4 :-} \newline
(अ॒पान्त्वेम॑न् - ह॒विषा॒ वर्ध॑नेन) \newline 


॥ हरिः॑ ॐ ॥॥ कृष्ण यजुर्वेदीय तैत्तिरीय संहितायां चतुर्थ काण्डे तृतीयः प्रश्नः समाप्तः ॥ \newline
\pagebreak
4.3.1   appendix\\4.3.13.8-इन्द्रा᳚ग्नी रोच॒ना दि॒वः > 1, श्नथ॑द्-वृ॒त्र >2, \\इन्द्रा᳚ग्नी रोच॒ना दि॒वः परि॒ वाजे॑षु भूषथः । तद्वां᳚ चेति॒ प्रवी॒र्यं᳚ ॥\\\\श्नथ॑द्-वृ॒त्रमु॒त स॑नोति॒ वाज॒मिन्द्रा॒ यो अ॒ग्नी सहु॑री सप॒र्यात् । \\इ॒र॒ज्यन्ता॑ वस॒व्य॑स्य॒ भूरेः॒ सह॑स्तमा॒ सह॑सा वाज॒यन्ता᳚ ॥\\(appearing in TS 4.2.11.1) \\\\4.3.13.8 मिन्द्रं॑ ॅवो वि॒श्वत॒स्परी >3, न्द्रं॒ नरो॒ >4, \\इन्द्रं॑ ॅवो वि॒श्वत॒स्परि॒ हवा॑महे॒ जने᳚भ्यः । अ॒स्माक॑मस्तु॒ केव॑लः ॥ \\\\इन्द्रं॒ नरो॑ ने॒मधि॑ता हवन्ते॒ यत्पार्या॑ यु॒नज॑ते॒ धिय॒स्ताः ।\\शूरो॒ नृषा॑ता॒ शव॑सश्चका॒न आ गोम॑ति व्र॒जे भ॑जा॒ त्वन्नः॑ ॥\\(appearing in TS 1.6.12.1)\\\\\\4.3.13.8 विश्व॑कर्मन्. ह॒विषा॑ वावृधा॒नो>5, \\विश्व॑कर्मन्. ह॒विषा॒ वर्ध॑नेन > 6 \\विश्व॑कर्मन्. ह॒विषा॑ वावृधा॒नः स्व॒यं ॅय॑जस्व त॒नुवं॑ जुषा॒णः ।\\मुह्य॑न्त्व॒न्ये अ॒भितः॑ स॒पत्ना॑ इ॒हास्माकं॑ म॒घवा॑ सू॒रिर॑स्तु ॥\\\\विश्व॑कर्मन्. ह॒विषा॒ वर्ध॑नेन त्रा॒तार॒मिन्द्र॑मकृणोरव॒ध्यं ।\\तस्मै॒ विशः॒ सम॑नमन्त पू॒र्वीर॒यमु॒ग्रो वि॑ह॒व्यो॑ यथास॑त् ॥\\(appearing in TS 4.6.2.6)\\
\pagebreak
        


\end{document}
