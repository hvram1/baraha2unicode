\documentclass[17pt]{extarticle}
\usepackage{babel}
\usepackage{fontspec}
\usepackage{polyglossia}
\usepackage{extsizes}



\setmainlanguage{sanskrit}
\setotherlanguages{english} %% or other languages
\setlength{\parindent}{0pt}
\pagestyle{myheadings}
\newfontfamily\devanagarifont[Script=Devanagari]{AdishilaVedic}


\newcommand{\VAR}[1]{}
\newcommand{\BLOCK}[1]{}




\begin{document}
\begin{titlepage}
    \begin{center}
 
\begin{sanskrit}
    { \Large
    ॐ नमः परमात्मने, श्री महागणपतये नमः, 
श्री गुरुभ्यो नमः । ह॒रिः॒ ॐ ॥ 
    }
    \\
    \vspace{2.5cm}
    \mbox{ \Huge
    1.1     प्रथमः प्रश्नः - (दर्.शपूर्णमासौ)   }
\end{sanskrit}
\end{center}

\end{titlepage}
\tableofcontents

ॐ नमः परमात्मने, श्री महागणपतये नमः, 
श्री गुरुभ्यो नमः । ह॒रिः॒ ॐ ॥ \newline
1.1     प्रथमः प्रश्नः - (दर्.शपूर्णमासौ) \newline

\addcontentsline{toc}{section}{ 1.1     प्रथमः प्रश्नः - (दर्.शपूर्णमासौ)}
\markright{ 1.1     प्रथमः प्रश्नः - (दर्.शपूर्णमासौ) \hfill https://www.vedavms.in \hfill}
\section*{ 1.1     प्रथमः प्रश्नः - (दर्.शपूर्णमासौ) }
                                \textbf{ TS 1.1.1.1} \newline
                  इ॒षे । त्वा॒ । ऊ॒र्जे । त्वा॒ । वा॒यवः॑ । स्थ॒ । उ॒पा॒यव॒ इत्यु॑प - आ॒यवः॑ । स्थ॒ । दे॒वः । वः॒ । स॒वि॒ता । प्रेति॑ । अ॒र्प॒य॒तु॒ । श्रेष्ठ॑तमा॒येति॒ श्रेष्ठ॑ - त॒मा॒य॒ । कर्म॑णे । एति॑ । प्या॒य॒ध्व॒म् । अ॒घ्नि॒याः॒ । दे॒व॒भा॒गमिति॑ देव- भा॒गम् । ऊर्ज॑स्वतीः । पय॑स्वतीः । प्र॒जाव॑ती॒रिति॑ प्र॒जा-व॒तीः॒ । अ॒न॒मी॒वाः । अ॒य॒क्ष्माः । मा । वः॒ । स्ते॒नः । ई॒श॒त॒ । मा । अ॒घशꣳ॑स॒ इत्य॒घ - शꣳ॒॒सः॒ । रु॒द्रस्य॑ । हे॒तिः । परीति॑ । वः॒ । वृ॒ण॒क्तु॒ । ध्रु॒वाः । अ॒स्मिन्न् । गोप॑ता॒विति॒ गो -प॒तौ॒ । स्या॒त॒ । ब॒ह्वीः । यज॑मानस्य । प॒शून् । पा॒हि॒ ॥ \textbf{  1 } \newline
                  \newline
                      (इ॒षे - त्रिच॑त्वारिꣳशत् )  \textbf{(A1)} \newline \newline
                                \textbf{ TS 1.1.2.1} \newline
                  य॒ज्ञ्स्य॑ । घो॒षत् । अ॒सि॒ । प्रत्यु॑ष्ट॒मिति॒ प्रति॑ - उ॒ष्ट॒म् । रक्षः॑ । प्रत्यु॑ष्टा॒ इति॒ प्रति॑ - उ॒ष्टाः॒ । अरा॑तयः । प्रेति॑ । इ॒यम् । अ॒गा॒त् । धि॒षणा᳚ । ब॒र्.॒हिः । अच्छ॑ । मनु॑ना । कृ॒ता । स्व॒धयेति॑ स्व-धया᳚ । वित॒ष्टेति॒ वि - त॒ष्टा॒ । ते । एति॑ । व॒ह॒न्ति॒ । क॒वयः॑ । पु॒रस्ता᳚त् । दे॒वेभ्यः॑ । जुष्ट᳚म् । इ॒ह । ब॒र्.॒.हिः । आ॒सद॒ इत्या᳚ - सदे᳚ । दे॒वाना᳚म् । प॒रि॒षू॒तमिति॑ परि -सू॒तम् । अ॒सि॒ । व॒र्.॒षवृ॑द्ध॒मिति॑ व॒र्.॒ष -वृ॒द्ध॒म् । अ॒सि॒ । देव॑बर्.हि॒रिति॒ देव॑ - ब॒र्.॒हिः॒ । मा । त्वा॒ । अ॒न्वक् । मा । ति॒र्यक् । पर्व॑ । ते॒ । रा॒ध्या॒स॒म् । आ॒च्छे॒त्तेत्या᳚ - छे॒त्ता । ते॒ । मा । रि॒ष॒म् । देव॑बर्.हि॒रिति॒ देव॑ -ब॒र्.॒हिः॒ । श॒तव॑ल्.श॒मिति॑ श॒त - व॒ल्.श॒म् । वीति॑ । रो॒ह॒ । स॒हस्र॑वल्शा॒ इति॑ स॒हस्र॑ -व॒ल्शाः॒ । \textbf{  2 } \newline
                  \newline
                                \textbf{ TS 1.1.2.2} \newline
                  वीति॑ । व॒यम् । रु॒हे॒म॒ । पृ॒थि॒व्याः । सं॒पृच॒ इति॑ सं - पृचः॑ । पा॒हि॒ । सु॒सं॒भृतेति॑ सु - सं॒भृता᳚ । त्वा॒ । समिति॑ । भ॒रा॒मि॒ । अदि॑त्यै । रास्ना᳚ । अ॒सि॒ । इ॒न्द्रा॒ण्यै । सं॒नह॑न॒मिति॑ सं - नह॑नम् । पू॒षा । ते॒ । ग्र॒न्थिम् । ग्र॒थ्ना॒तु॒ । सः । ते॒ । मा । एति॑ । स्था॒त् । इन्द्र॑स्य । त्वा॒ । बा॒हुभ्या॒मिति॑ बा॒हु-भ्या॒म् । उदिति॑ । य॒च्छे॒ । बृह॒स्पतेः᳚ । मू॒र्ध्ना । ह॒रा॒मि॒ । उ॒रु । अ॒न्तरि॑क्षम् । अन्विति॑ । इ॒हि॒ । दे॒व॒ङ्ग॒ममिति॑ देवम् - ग॒मम् । अ॒सि॒ ॥ \textbf{  3 } \newline
                  \newline
                      (स॒हस्र॑वल्.शा - अ॒ष्टात्रिꣳ॑शच्च)  \textbf{(A2)} \newline \newline
                                \textbf{ TS 1.1.3.1} \newline
                  शुन्ध॑द्ध्वम् । दैव्या॑य । कर्म॑णे । दे॒व॒य॒ज्याया॒ इति॑ देव -य॒ज्यायै᳚ । मा॒त॒रिश्व॑नः । घ॒र्मः । अ॒सि॒ । द्यौः । अ॒सि॒ । पृ॒थि॒वी । अ॒सि॒ । वि॒श्वधा॑या॒ इति॑ वि॒श्व -धा॒याः॒ । अ॒सि॒ । प॒र॒मेण॑ । धाम्ना᳚ । दृꣳह॑स्व । मा । ह्वाः॒ । वसू॑नाम् । प॒वित्र᳚म् । अ॒सि॒ । श॒तधा॑र॒मिति॑ श॒त -धा॒र॒म् ।  वसू॑नाम् । प॒वित्र᳚म् । अ॒सि॒ । स॒हस्र॑धार॒मिति॑ स॒हस्र॑ - धा॒र॒म् । हु॒तः । स्तो॒कः । हु॒तः । द्र॒फ्सः । अ॒ग्नये᳚ । बृ॒ह॒ते । नाका॑य । स्वाहा᳚ । द्यावा॑ पृथि॒वीभ्या॒मिति॒ द्यावा᳚ - पृ॒थि॒वीभ्या᳚म् । सा । वि॒श्वायु॒रिति॑ वि॒श्व- आ॒युः॒ । सा । वि॒श्वव्य॑चा॒ इति॑ वि॒श्व -व्य॒चाः॒ । सा । वि॒श्वक॒र्मेति॑ वि॒श्व -क॒र्मा॒ । समिति॑ । पृ॒च्य॒द्ध्व॒म् । ऋ॒ता॒व॒री॒रित्यृ॑त - व॒रीः॒ । ऊ॒र्मिणीः᳚ । मधु॑मत्तमा॒ इति॒ मधु॑मत्-त॒माः॒ । म॒न्द्राः । धन॑स्य । सा॒तये᳚ । सोमे॑न ( ) । त्वा॒ । एति॑ । त॒न॒च्मि॒ । इन्द्रा॑य । दधि॑ । विष्णो॒ इति॑ । ह॒व्यम् । र॒क्ष॒स्व॒ ॥ \textbf{  4 } \newline
                  \newline
                      (सोमे॑ - ना॒ष्टौ च॑)  \textbf{(A3)} \newline \newline
                                \textbf{ TS 1.1.4.1} \newline
                  कर्म॑णे । वा॒म् । दे॒वेभ्यः॑ । श॒के॒य॒म् । वेषा॑य । त्वा॒ । प्रत्यु॑ष्ट॒मिति॒ प्रति॑ -उ॒ष्ट॒म् । रक्षः॑ । प्रत्यु॑ष्टा॒ इति॒ प्रति॑ -उ॒ष्टाः॒ । अरा॑तयः । धूः । अ॒सि॒ ।  धूर्व॑ । धूर्व॑न्तम् । धूर्व॑ । तम् । यः । अ॒स्मान् । धूर्व॑ति । तम् । धू॒र्व॒ । यम् । व॒यम् । धूर्वा॑मः । त्वम् । दे॒वाना᳚म् । अ॒सि॒ । सस्नि॑तम॒मिति॒ सस्नि॑ - त॒म॒म् । पप्रि॑तम॒मिति॒ पप्रि॑ -त॒म॒म् । जुष्ट॑तम॒मिति॒ जुष्ट॑ - त॒म॒म् । वह्नि॑तम॒मिति॒ वह्नि॑ - त॒म॒म् । दे॒व॒हूत॑म॒मिति॑ देव-हूत॑मम् । अह्रु॑तम् । अ॒सि॒ । ह॒वि॒र्धान॒मिति॑ हविः - धान᳚म् । दृꣳह॑स्व । मा । ह्वाः॒ । मि॒त्रस्य॑ । त्वा॒ । चक्षु॑षा । प्रेति॑ । ई॒क्षे॒ । मा । भेः । मा । समिति॑ । वि॒क्था॒ । मा । त्वा॒ । \textbf{  5} \newline
                  \newline
                                \textbf{ TS 1.1.4.2} \newline
                  हिꣳ॒॒सि॒ष॒म् । उ॒रु । वाता॑य । दे॒वस्य॑ । त्वा॒ । स॒वि॒तुः । प्र॒स॒व इति॑ प्र -स॒वे । अ॒श्विनोः᳚ । बा॒हुभ्या॒मिति॑ बा॒हु -भ्या॒म् । पू॒ष्णः । हस्ता᳚भ्याम् । अ॒ग्नये᳚ । जुष्ट᳚म् । निरिति॑ । व॒पा॒मि॒ । अ॒ग्नीषोमा᳚भ्या॒मित्य॒ग्नी - सोमा᳚भ्याम् । इ॒दम् । दे॒वाना᳚म् । इ॒दम् । उ॒ । नः॒ । स॒ह । स्फा॒त्यै । त्वा॒ । न । अरा᳚त्यै । सुवः॑ । अ॒भि । वीति॑ । ख्ये॒ष॒म् । वै॒श्वा॒न॒रम् । ज्योतिः॑ । दृꣳह॑न्ताम् । दुर्याः᳚ । द्यावा॑पृथि॒व्योरिति॒ द्यावा᳚ - पृ॒थि॒व्योः । उ॒रु । अ॒न्तरि॑क्षम् । अन्विति॑ । इ॒हि॒ । अदि॑त्याः । त्वा॒ । उ॒पस्थ॒ इत्यु॒प-स्थे॒ । सा॒द॒या॒मि॒ । अग्ने᳚ । ह॒व्यम् । र॒क्ष॒स्व॒ ॥ \textbf{  6 } \newline
                  \newline
                      ( मा त्वा॒ - षट्च॑त्वारिꣳशच्च )  \textbf{(A4)} \newline \newline
                                \textbf{ TS 1.1.5.1} \newline
                  दे॒वः । वः॒ । स॒वि॒ता । उदिति॑ । पु॒ना॒तु॒ । अच्छि॑द्रेण । प॒वित्रे॑ण । वसोः᳚ । सूर्य॑स्य । र॒श्मिभि॒रिति॑ र॒श्मि - भिः॒ । आपः॑ । दे॒वीः॒ । अ॒ग्रे॒पु॒व॒ इत्य॑ग्रे - पु॒वः॒ । अ॒ग्रे॒गु॒व॒ इत्य॑ग्रे - गु॒वः॒ । अग्रे᳚ । इ॒मम् । य॒ज्ञ्म् । न॒य॒त॒ । अग्रे᳚ । य॒ज्ञ्प॑ति॒मिति॑ य॒ज्ञ् - प॒ति॒म् । ध॒त्त॒ । यु॒ष्मान् । इन्द्रः॑ । अ॒वृ॒णी॒त॒ । वृ॒त्र॒तूर्य॒ इति॑ वृत्र -तूर्ये᳚ । यू॒यम् । इन्द्र᳚म् । अ॒वृ॒णी॒ध्व॒म् । वृ॒त्र॒तूर्य॒ इति॑ वृत्र -तूर्ये᳚ । प्रोक्षि॑ता॒ इति॒ प्र -उ॒क्षि॒ताः॒ । स्थ॒ । अ॒ग्नये᳚ । वः॒ । जुष्ट᳚म् । प्रेति॑ । उ॒क्षा॒मि॒ । अ॒ग्नीषोमा᳚भ्या॒मित्य॒ग्नी - सोमा᳚भ्याम् । शुन्ध॑ध्वम् । दैव्या॑य । कर्म॑णे । दे॒व॒य॒ज्याया॒ इति॑ देव-य॒ज्यायै᳚ । अव॑धूत॒मित्यव॑ - धू॒त॒म् । रक्षः॑ । अव॑धूता॒ इत्यव॑ - धू॒ताः॒ । अरा॑तयः । अदि॑त्याः । त्वक् । अ॒सि॒ । प्रतीति॑ । त्वा॒ । \textbf{  7} \newline
                  \newline
                                \textbf{ TS 1.1.5.2} \newline
                  पृ॒थि॒वी । वे॒त्तु॒ । अ॒धि॒षव॑ण॒मित्य॑धि - सव॑नम् । अ॒सि॒ । वा॒न॒स्प॒त्यम् । प्रतीति॑ । त्वा॒ । अदि॑त्याः । त्वक् । वे॒त्तु॒ । अ॒ग्नेः । त॒नूः । अ॒सि॒ । वा॒चः । वि॒सर्ज॑न॒मिति॑ वि-सर्ज॑नम् । दे॒ववी॑तय॒ इति॑ दे॒व- वी॒त॒ये॒ । त्वा॒ । गृ॒ह्णा॒मि॒ । अद्रिः॑ । अ॒सि॒ । वा॒न॒स्प॒त्यः । सः । इ॒दम् । दे॒वेभ्यः॑ । ह॒व्यम् । सु॒शमीति॑ सु- शमि॑ । श॒मि॒ष्व॒ । इष᳚म् । एति॑ । व॒द॒ । ऊर्ज᳚म् । एति॑ । व॒द॒ । द्यु॒मदिति॑ द्यु-मत् । व॒द॒त॒ । व॒यम् । स॒ङ्घा॒तमिति॑ सं - घा॒तम् । जे॒ष्म॒ । व॒र्.॒षवृ॑द्ध॒मिति॑ व॒र्.॒ष -वृ॒द्ध॒म् । अ॒सि॒ । प्रतीति॑ । त्वा॒ । व॒र्.॒षवृ॑द्ध॒मिति॑ व॒र्.॒ष -वृ॒द्ध॒म् । वे॒त्तु॒ । परा॑पूत॒मिति॒ परा᳚ - पू॒त॒म् । रक्षः॑ । परा॑पूता॒ इति॒ परा᳚ - पू॒ताः॒ । अरा॑तयः । रक्ष॑साम् । भा॒गः ( ) । अ॒सि॒ । वा॒युः । वः॒ । वीति॑ । वि॒न॒क्तु॒ । दे॒वः । वः॒ । स॒वि॒ता । हिर॑ण्यपाणि॒रिति॒ हिर॑ण्य - पा॒णिः॒ । प्रतीति॑ । गृ॒ह्णा॒तु॒ ॥ \textbf{  8 } \newline
                  \newline
                      ( त्वा॒ - भा॒ग - एका॑दश च )  \textbf{(A5)} \newline \newline
                                \textbf{ TS 1.1.6.1} \newline
                  अव॑धूत॒मित्यव॑ - धू॒त॒म् । रक्षः॑ । अव॑धूता॒ इत्यव॑ - धू॒ताः॒ । अरा॑तयः । अदि॑त्याः । त्वक् । अ॒सि॒ । प्रतीति॑ । त्वा॒ । पृ॒थि॒वी । वे॒त्तु॒ । दि॒वः । स्क॒म्भ॒निः । अ॒सि॒ । प्रतीति॑ । त्वा॒ । अदि॑त्याः । त्वक् । वे॒त्तु॒ । धि॒षणा᳚ । अ॒सि॒ । प॒र्व॒त्या । प्रतीति॑ । त्वा॒ । दि॒वः । स्क॒म्भ॒निः । वे॒त्तु॒ । धि॒षणा᳚ । अ॒सि॒ । पा॒र्व॒ते॒यी । प्रतीति॑ । त्वा॒ । प॒र्व॒तिः । वे॒त्तु॒ । दे॒वस्य॑ । त्वा॒ । स॒वि॒तुः । प्र॒स॒व इति॑ प्र -स॒वे । अ॒श्विनोः᳚ । बा॒हुभ्या॒मिति॑ बा॒हु-भ्या॒म् । पू॒ष्णः । हस्ता᳚भ्याम् । अधीति॑ । व॒पा॒मि॒ । धा॒न्य᳚म् । अ॒सि॒ । धि॒नु॒हि । दे॒वान् । प्रा॒णायेति॑ प्र-अ॒नाय॑ । त्वा॒ ( ) । अ॒पा॒नात्ये॑प -अ॒नाय॑ । त्वा॒ । व्या॒नायेति॑ वि- अ॒नाय॑ । त्वा॒ । दी॒र्घाम् । अन्विति॑ । प्रसि॑ति॒मिति॒ प्र - सि॒ति॒म् । आयु॑षे । धा॒म् । दे॒वः । वः॒ । स॒वि॒ता । हिर॑ण्यपाणि॒रिति॒ हिर॑ण्य - पा॒णिः॒ । प्रतीति॑ । गृ॒ह्णा॒तु॒ ॥ \textbf{  9 } \newline
                  \newline
                      (प्रा॒णाय॑ त्वा॒ - पञ्च॑दश च)  \textbf{(A6)} \newline \newline
                                \textbf{ TS 1.1.7.1} \newline
                  धृष्टिः॑ । अ॒सि॒ । ब्रह्म॑ । य॒च्छ॒ । अपेति॑ । अ॒ग्ने॒ । अ॒ग्निम् । आ॒माद॒मित्या॑म -अद᳚म् । ज॒हि॒ । निरिति॑ । क्र॒व्याद॒मिति॑ क्रव्य- अद᳚म् । से॒ध॒ । एति॑ । दे॒व॒यज॒मिति॑ देव-यज᳚म् । व॒ह॒ । निर्द॑ग्ध॒मिति॒ निः - द॒ग्ध॒म् । रक्षः॑ । निर्द॑ग्धा॒ इति॒ निः - द॒ग्धाः॒ । अरा॑तयः । ध्रु॒वम् । अ॒सि॒ । पृ॒थि॒वीम् । दृꣳ॒॒ह॒ । आयुः॑ । दृꣳ॒॒ह॒ । प्र॒जामिति॑ प्र -जाम् । दृꣳ॒॒ह॒ । स॒जा॒तानिति॑ स-जा॒तान् । अ॒स्मै । यज॑मानाय । परीति॑ । ऊ॒ह॒ । ध॒र्त्रम् । अ॒सि॒ । अ॒न्तरि॑क्षम् । दृꣳ॒॒ह॒ । प्रा॒णमिति॑ प्र- अ॒नम् । दृꣳ॒॒ह॒ । आ॒पा॒नमित्य॑प - अ॒नम् । दृꣳ॒॒ह॒ । स॒जा॒तानिति॑ स-जा॒तान् । अ॒स्मै । यज॑मानाय । परीति॑ । ऊ॒ह॒ । ध॒रुण᳚म् । अ॒सि॒ । दिव᳚म् । दृꣳ॒॒ह॒ । चक्षुः॑ । \textbf{  10 } \newline
                  \newline
                                \textbf{ TS 1.1.7.2} \newline
                  दृꣳ॒॒ह॒ । श्रोत्र᳚म् । दृꣳ॒॒ह॒ । स॒जा॒तानिति॑ स-जा॒तान् । अ॒स्मै । यज॑मानाय । परीति॑ । ऊ॒ह॒ । धर्म॑ । अ॒सि॒ । दिशः॑ । दृꣳ॒॒ह॒ । योनि᳚म् । दृꣳ॒॒ह॒ । प्र॒जामिति॑ प्र -जाम् । दृꣳ॒॒ह॒ । स॒जा॒तानिति॑ स-जा॒तान् । अ॒स्मै । यज॑मानाय । परीति॑ । ऊ॒ह॒ । चितः॑ । स्थ॒ । प्र॒जामिति॑ प्र -जाम् । अ॒स्मै । र॒यिम् । अ॒स्मै । स॒जा॒तानिति॑ स-जा॒तान् । अ॒स्मै । यज॑मानाय । परीति॑ । ऊ॒ह॒ । भृगू॑णाम् । अङ्गि॑रसाम् । तप॑सा । त॒प्य॒ध्व॒म् । यानि॑ । घ॒र्मे । क॒पाला॑नि । उ॒प॒चि॒न्वन्तीत्यु॑प - चि॒न्वन्ति॑ । वे॒धसः॑ ॥ पू॒ष्णः । तानि॑ । अपीति॑ । व्र॒ते । इ॒न्द्र॒वा॒यू इती᳚न्द्र-वायू । वीति॑ । मु॒ञ्च॒ता॒म् ॥ \textbf{  11} \newline
                  \newline
                      (चक्षु॑ - र॒ष्टाच॑त्वारिꣳशच्च)  \textbf{(A7)} \newline \newline
                                \textbf{ TS 1.1.8.1} \newline
                  समिति॑ । व॒पा॒मि॒ । समिति॑ । आपः॑ । अ॒द्भिरित्य॑त् - भिः । अ॒ग्म॒त॒ । समिति॑ । ओष॑धयः । रसे॑न । समिति॑ । रे॒वतीः᳚ । जग॑तीभिः । मधु॑मती॒रिति॒ मधु॑ - म॒तीः॒ । मधु॑मतीभि॒रिति॒ मधु॑ - म॒ती॒भिः॒ । सृ॒ज्य॒ध्व॒म् । अ॒द्भ्य इत्य॑त् - भ्यः । परीति॑ । प्रजा॑ता॒ इति॒ प्र -जा॒ताः॒ । स्थ॒ । समिति॑ । अ॒द्भिरित्य॑त् - भिः । पृ॒च्य॒ध्व॒म् । जन॑यत्यै । त्वा॒ । समिति॑ । यौ॒मि॒ । अ॒ग्नये᳚ । त्वा॒ । अ॒ग्नीषोमा᳚भ्या॒मित्य॒ग्नी - सोमा᳚भ्याम् । म॒खस्य॑ । शिरः॑ । अ॒सि॒ । घ॒र्मः । अ॒सि॒ । वि॒श्वायु॒रुरिति॑ वि॒श्व- आ॒युः॒ । उ॒रु । प्र॒थ॒स्व॒ । उ॒रु । ते॒ । य॒ज्ञ्प॑ति॒रिति॑ य॒ज्ञ् -प॒तिः॒ । प्र॒थ॒ता॒म् । त्वच᳚म् । गृ॒ह्णी॒ष्व॒ । अ॒न्तरि॑त॒मित्य॒न्तः - इ॒त॒म् । रक्षः॑ । अ॒न्तरि॑ता॒ इत्य॒न्तः - इ॒ताः॒ । अरा॑तयः । दे॒वः । त्वा॒ । स॒वि॒ता ( ) । श्र॒प॒य॒तु॒ । वर्.षि॑ष्ठे । अधीति॑ । नाके᳚ । अ॒ग्निः । ते॒ । त॒नुव᳚म् । मा । अतीति॑ । धा॒क् । अग्ने᳚ । ह॒व्यम् । र॒क्ष॒स्व॒ । समिति॑ । ब्रह्म॑णा । पृ॒च्य॒स्व॒ । ए॒क॒ताय॑ । स्वाहा᳚ । द्वि॒ताय॑ । स्वाहा᳚ । त्रि॒ताय॑ । स्वाहा᳚ ॥ \textbf{  12 } \newline
                  \newline
                      (स॒वि॒ता - द्वाविꣳ॑शतिश्च)  \textbf{(A8)} \newline \newline
                                \textbf{ TS 1.1.9.1} \newline
                  एति॑ । द॒दे॒ । इन्द्र॑स्य । बा॒हुः । अ॒सि॒ । दक्षि॑णः । स॒हस्र॑भृष्टि॒रिति॑ स॒हस्र॑ -भृ॒ष्टिः॒ । श॒तते॑जा॒ इति॑ श॒त -ते॒जाः॒ । वा॒युः । अ॒सि॒ । ति॒ग्मते॑जा॒ इति॑ ति॒ग्म - ते॒जाः॒ । पृथि॑वि । दे॒व॒य॒ज॒नीति॑ देव -य॒ज॒नि॒ । ओष॑ध्याः । ते॒ । मूल᳚म् । मा । हिꣳ॒॒सि॒ष॒म् । अप॑हत॒ इत्यप॑ -ह॒तः॒ । अ॒ररुः॑ । पृ॒थि॒व्यै । व्र॒जम् । ग॒च्छ॒ । गो॒स्थान॒मिति॑ गो-स्थान᳚म् । वर्.ष॑तु । ते॒ । द्यौः । ब॒धा॒न । दे॒व॒ । स॒वि॒तः॒ । प॒र॒मस्या᳚म् । प॒रा॒वतीति॑ परा - वति॑ । श॒तेन॑ । पाशैः᳚ । यः । अ॒स्मान् । द्वेष्टि॑ । यम् । च॒ । व॒यम् । द्वि॒ष्मः । तम् । अतः॑ । मा । मौ॒क् । अप॑हत॒ इत्यप॑ -ह॒तः॒ । अ॒ररुः॑ । पृ॒थि॒व्यै । दे॒व॒यज॑न्या॒ इति॑ देव -यज॑न्यै । व्र॒जम् । \textbf{  13} \newline
                  \newline
                                \textbf{ TS 1.1.9.2} \newline
                  ग॒च्छ॒ । गो॒स्थान॒मिति॑ गो-स्थान᳚म् । वर्.ष॑तु । ते॒ । द्यौः । ब॒धा॒न । दे॒व॒ । स॒वि॒तः॒ । पर॒मस्या᳚म् । प॒रा॒वतीति॑ परा - वति॑ । श॒तेन॑ । पाशैः᳚ । यः । अ॒स्मान् । द्वेष्टि॑ । यम् । च॒ । व॒यम् । द्वि॒ष्मः । तम् । अतः॑ । मा । मौ॒क् । अप॑हत॒ इत्यप॑ -ह॒तः॒ । अ॒ररुः॑ । पृ॒थि॒व्याः । अदे॑वयजन॒ इत्यदे॑व - य॒ज॒नः॒ । व्र॒जम् । ग॒च्छ॒ । गो॒स्थान॒मिति॑ गो-स्थान᳚म् । वर्.ष॑तु । ते॒ । द्यौः । ब॒धा॒न । दे॒व॒ । स॒वि॒तः॒ । प॒र॒मस्या᳚म् । प॒रा॒वतीति॑ परा - वति॑ । श॒तेन॑ । पाशैः᳚ । यः । अ॒स्मान् । द्वेष्टि॑ । यम् । च॒ । व॒यम् । द्वि॒ष्मः । तम् । अतः॑ । मा । \textbf{  14} \newline
                  \newline
                                \textbf{ TS 1.1.9.3} \newline
                  मौ॒क् । अ॒ररुः॑ । ते॒ । दिव᳚म् । मा । स्का॒न् । वस॑वः । त्वा॒ । परीति॑ । गृ॒ह्ण॒न्तु॒ । गा॒य॒त्रेण॑ । छन्द॑सा । रु॒द्राः । त्वा॒ । परीति॑ । गृ॒ह्ण॒न्तु॒ । त्रैष्टु॑भेन । छन्द॑सा । आ॒दि॒त्याः । त्वा॒ । परीति॑ । गृ॒ह्ण॒न्तु॒ । जाग॑तेन । छन्द॑सा । दे॒वस्य॑ । स॒वि॒तुः । स॒वे । कर्म॑ । कृ॒ण्व॒न्ति॒ । वे॒धसः॑ । ऋ॒तम् । अ॒सि॒ । ऋ॒त॒सद॑न॒मित्यृ॑त -सद॑नम् । अ॒सि॒ । ऋ॒त॒श्रीरित्यृ॑त - श्रीः । अ॒सि॒ । धाः । अ॒सि॒ । स्व॒धेति॑ स्व- धा । अ॒सि॒ । उ॒र्वी । च॒ । अ॒सि॑ । वस्वी᳚ । च॒ । अ॒सि॒ । पु॒रा । क्रू॒रस्य॑ । वि॒सृप॒ इति॑ वि- सृपः॑ । वि॒र॒फ्शि॒न्निति॑ वि- र॒फ्शि॒न्न् ( ) । उ॒दा॒दायेत्यु॑त् - आ॒दाय॑ । पृ॒थि॒वीम् । जी॒रदा॑नु॒रिति॑ जी॒र - दा॒नुः॒ । याम् । ऐर॑यन्न् । च॒न्द्रम॑सि । स्व॒धाभि॒रिति॑ स्व-धाभिः॑ । ताम् । धीरा॑सः । अ॒नु॒दृश्येत्य॑नु अनु - दृश्य॑ । य॒ज॒न्ते॒ ॥ \textbf{  15 } \newline
                  \newline
                      (दे॒व॒यज॑न्यै व्र॒जं - तमतो॒ मा - वि॑रफ्शि॒न् - नेका॑दश च)  \textbf{(A9)} \newline \newline
                                \textbf{ TS 1.1.10.1} \newline
                  प्रत्यु॑ष्ट॒मिति॒ प्रति॑ - उ॒ष्ट॒म् । रक्षः॑ । प्रत्यु॑ष्टा॒ इति॒ प्रति॑ -उ॒ष्टाः॒ । अरा॑तयः । अ॒ग्नेः । वः॒ । तेजि॑ष्ठेन । तेज॑सा । निरिति॑ । त॒पा॒मि॒ । गो॒ष्ठमिति॑ गो -स्थम् । मा । निरिति॑ । मृ॒क्ष॒म् । वा॒जिन᳚म् । त्वा॒ । स॒प॒त्न॒सा॒हमिति॑ सपत्न - सा॒हम् । समिति॑ । मा॒र्ज्मि॒ । वाच᳚म् । प्रा॒णमिति॑ प्रा- अ॒नम् । चक्षुः॑ । श्रोत्र᳚म् । प्र॒जामिति॑ प्र -जाम् । योनि᳚म् । मा । निरिति॑ । मृ॒क्ष॒म् । वा॒जिनी᳚म् । त्वा॒ । स॒प॒त्न॒सा॒हीमिति॑ सपत्न - सा॒हीम् । समिति॑ । मा॒र्ज्मि॒ । आ॒शासा॒नेत्या᳚ -शासा॑ना । सौ॒म॒न॒सम् । प्र॒जामिति॑ प्र -जाम् । सौभा᳚ग्यम् । त॒नूम् ॥ अ॒ग्नेः । अनु॑व्र॒तेत्यनु॑ -व्र॒ता॒ । भू॒त्वा । समिति॑ । न॒ह्ये॒ । सु॒कृ॒तायेति॑ सु -कृ॒ताय॑ । कम् ॥ सु॒प्र॒जस॒ इति॑ सु-प्र॒जसः॑ । त्वा॒ । व॒यम् । सु॒पत्नी॒रिति॑ सु-पत्नीः᳚ । उपेति॑ । \textbf{  16} \newline
                  \newline
                                \textbf{ TS 1.1.10.2} \newline
                  से॒दि॒म॒ ॥ अग्ने᳚ । स॒प॒त्न॒दम्भ॑न॒मिति॑ सपत्न-दम्भ॑नम् । अद॑ब्धासः । अदा᳚भ्यम् ॥ इ॒मम् । वीति॑ । स्या॒मि॒ । वरु॑णस्य । पाश᳚म् । यम् । अब॑ध्नीत । स॒वि॒ता । सु॒केत॒ इति॑ सु-केतः॑ ॥ धा॒तुः । च॒ । योनौ᳚ । सु॒कृ॒तस्येति॑ सु -कृ॒तस्य॑ । लो॒के । स्यो॒नम् । मे॒ । स॒ह । पत्या᳚ । क॒रो॒मि॒ ॥ समिति॑ । आयु॑षा । समिति॑ । प्र॒जयेति॑ प्र-जया᳚ । समिति॑ । अ॒ग्ने॒ । वर्च॑सा । पुनः॑ ॥ समिति॑ । पत्नी᳚ । पत्या᳚ । अ॒हम् । ग॒च्छे॒ । समिति॑ । आ॒त्मा । त॒नुवा᳚ । मम॑ ॥ म॒ही॒नाम् । पयः॑ । अ॒सि॒ । ओष॑धीनाम् । रसः॑ । तस्य॑ । ते॒ । अक्षी॑यमाणस्य । निरिति॑ । \textbf{  17 } \newline
                  \newline
                                \textbf{ TS 1.1.10.3} \newline
                  व॒पा॒मि॒ । म॒ही॒नाम् । पयः॑ । अ॒सि॒ । ओष॑धीनाम् । रसः॑ । अद॑ब्धेन । त्वा॒ । चक्षु॑षा । अवेति॑ । ई॒क्षे॒ । सु॒प्र॒जा॒स्त्वायेति॑ सुप्रजाः - त्वाय॑ । तेजः॑ । अ॒सि॒ । तेजः॑ । अनु॑ । प्रेति॑ । इ॒हि॒ । अ॒ग्निः । ते॒ । तेजः॑ । मा । वीति॑ । नै॒त् । अ॒ग्नेः । जि॒ह्वा । अ॒सि॒ । सु॒भूरिति॑ सु - भूः । दे॒वाना᳚म् । धाम्ने॑ धाम्न॒ इति॒ धाम्ने᳚ - धा॒म्ने॒ । दे॒वेभ्यः॑ । यजु॑षेयजुष॒ इति॒ यजु॑षे -य॒जु॒षे॒ । भ॒व॒ । शु॒क्रम् । अ॒सि॒ । ज्योतिः॑ । अ॒सि॒ । तेजः॑ । अ॒सि॒ । दे॒वः । वः॒ । स॒वि॒ता । उदिति॑ । पु॒ना॒तु॒ । अच्छि॑द्रेण । प॒वित्रे॑ण । वसोः᳚ । सूर्य॑स्य । र॒श्मिभि॒रिति॑ र॒श्मि-भिः॒ । शु॒क्रम् ( ) । त्वा॒ । शु॒क्राया᳚म् । धाम्ने॑ धाम्न॒ इति॒ धाम्ने᳚ - धा॒म्ने॒ । दे॒वेभ्यः॑ । यजु॑षेयजुष॒ इति॒ यजु॑षे - य॒जु॒षे॒ । गृ॒ह्णा॒मि॒ । ज्योतिः॑ । त्वा॒ । ज्योति॑षि । अ॒र्चिः । त्वा॒ । अ॒र्चिषि॑ । धाम्ने॑ धाम्न॒ इति॒ धाम्ने᳚ - धा॒म्ने॒ । दे॒वेभ्यः॑ । यजु॑षेयजुष॒ इति॒ यजु॑षे -य॒जु॒षे॒ । गृ॒ह्णा॒मि॒ ॥ \textbf{  18} \newline
                  \newline
                      (उप॒ - नी - र॒श्मिभिः॑ शु॒क्रꣳ - षोड॑श च)  \textbf{(A10)} \newline \newline
                                \textbf{ TS 1.1.11.1} \newline
                  कृष्णः॑ । अ॒सि॒ । आ॒ख॒रे॒ष्ठ इत्या॑खरे - स्थः । अ॒ग्नये᳚ । त्वा॒ । स्वाहा᳚ । वेदिः॑ । अ॒सि॒ । ब॒र्.॒हिषे᳚ । त्वा॒ । स्वाहा᳚ । ब॒र्.॒हिः । अ॒सि॒ । स्रु॒ग्भ्य इति॑ स्रुक् -भ्यः । त्वा॒ । स्वाहा᳚ । दि॒वे । त्वा॒ । अ॒न्तरि॑क्षाय । त्वा॒ । पृ॒थि॒व्यै । त्वा॒ । स्व॒धेति॑ स्व-धा । पि॒तृभ्य॒ इति॑ पि॒तृ -भ्यः॒ । ऊर्क् । भ॒व॒ । ब॒र्.॒हि॒षद्भ्य॒ इति॑ बर्.हि॒षत् - भ्यः॒ । ऊ॒र्जा । पृ॒थि॒वीम् । ग॒च्छ॒त॒ । विष्णोः᳚ । स्तूपः॑ । अ॒सि॒ । ऊर्णा᳚म्रदस॒मित्यूर्णा᳚ - म्र॒द॒स॒म् । त्वा॒ । स्तृ॒णा॒मि॒ । स्वा॒स॒स्थमिति॑ सु -आ॒स॒स्थम् । दे॒वेभ्यः॑ । ग॒न्ध॒र्वः॑ । अ॒सि॒ । वि॒श्वाव॑सु॒रिति॑ वि॒श्व - व॒सुः॒ । विश्व॑स्मात् । ईष॑तः । यज॑मानस्य । प॒रि॒धिरिति॑ परि - धिः । इ॒डः । ई॒डि॒तः । इन्द्र॑स्य । बा॒हुः । अ॒सि॒ । \textbf{  19} \newline
                  \newline
                                \textbf{ TS 1.1.11.2} \newline
                  दक्षि॑णः । यज॑मानस्य । प॒रि॒धिरिति॑ परि - धिः । इ॒डः । ई॒डि॒तः । मि॒त्रावरु॑णा॒विति॑ मि॒त्रा - वरु॑णौ । त्वा॒ । उ॒त्त॒र॒त इत्यु॑त् - त॒र॒तः । परीति॑ । ध॒त्ता॒म् । ध्रु॒वेण॑ । धर्म॑णा । यज॑मानस्य । प॒रि॒धिरिति॑ परि - धिः । इ॒डः । ई॒डि॒तः । सूर्यः॑ । त्वा॒ । पु॒रस्ता᳚त् । पा॒तु॒ । कस्याः᳚ । चि॒त् । अ॒भिश॑स्त्या॒ इत्य॒भि -श॒स्त्याः॒ । वी॒तिहो᳚त्र॒मिति॑ -वी॒ति - हो॒त्र॒म् । त्वा॒ । क॒वे॒ । द्यु॒मन्त॒मिति॑ द्यु - मन्त᳚म् । समिति॑ । इ॒धी॒म॒हि॒ । अग्ने᳚ । बृ॒हन्त᳚म् । अ॒ध्व॒रे । वि॒शः । य॒न्त्रे इति॑ । स्थः॒ । वसू॑नाम् । रु॒द्राणा᳚म् । आ॒दि॒त्याना᳚म् । सद॑सि । सी॒द॒ । जु॒हूः । उ॒प॒भृदित्यु॑प - भृत् । ध्रु॒वा । अ॒सि॒ । घृ॒ताची᳚ । नाम्ना᳚ । प्रि॒येण॑ । नाम्ना᳚ । प्रि॒ये । सद॑सि ( ) । सी॒द॒ । ए॒ताः । अ॒स॒द॒न्न् । सु॒कृ॒तस्येति॑ सु-कृ॒तस्य॑ । लो॒के । ताः । वि॒ष्णो॒ इति॑ । पा॒हि॒ । पा॒हि । य॒ज्ञ्म् । पा॒हि । य॒ज्ञ्प॑ति॒मिति॑ य॒ज्ञ्-प॒ति॒म् । पा॒हि । माम् । य॒ज्ञ्॒निय॒मिति॑ यज्ञ् - निय᳚म् ॥ \textbf{  20} \newline
                  \newline
                      (बा॒हुर॑सि - प्रि॒ये सद॑सि॒ - पञ्च॑दश च)  \textbf{(A11)} \newline \newline
                                \textbf{ TS 1.1.12.1} \newline
                  भुव॑नम् । अ॒सि॑ । वीति॑ । प्र॒थ॒स्व॒ । अग्ने᳚ । यष्टः॑ । इ॒दम् । नमः॑ ॥ जुहु॑ । एति॑ । इ॒हि॒ । अ॒ग्निः । त्वा॒ । ह्व॒य॒ति॒ । दे॒व॒य॒ज्याया॒ इति॑ देव -य॒ज्यायै᳚ । उप॑भृ॒दित्युप॑ -भृ॒त् । एति॑ । इ॒हि॒ । दे॒वः । त्वा॒ । स॒वि॒ता । ह्व॒य॒ति॒ । दे॒व॒य॒ज्याया॒ इति॑ देव -य॒ज्यायै᳚ । अग्ना॑विष्णू॒ इत्यग्ना᳚ -वि॒ष्णू॒ । मा । वा॒म् । अवेति॑ । क्र॒मि॒ष॒म् । वीति॑ । जि॒हा॒था॒म् । मा । मा॒ । समिति॑ । ता॒प्त॒म् । लो॒कम् । मे॒ । लो॒क॒कृ॒ता॒विति॑ लोक-कृ॒तौ॒ । कृ॒णु॒त॒म् । विष्णोः᳚ । स्थान᳚म् । अ॒सि॒ । इ॒तः । इन्द्रः॑ । अ॒कृ॒णो॒त् । वी॒र्या॑णि । स॒मा॒रभ्येति॑ सं-आ॒रभ्य॑ । ऊ॒र्द्ध्वः । अ॒द्ध्व॒रः । दि॒वि॒स्पृश॒मिति॑ दिवि -स्पृश᳚म् । अह्रु॑तः ( ) । य॒ज्ञ्ः । य॒ज्ञ्प॑ते॒रिति॑ य॒ज्ञ् - प॒तेः॒ । इन्द्रा॑वा॒नितीन्द्र॑ - वा॒न् । स्वाहा᳚ । बृ॒हत् । भाः । पा॒हि । मा॒ । अ॒ग्ने॒ । दुश्च॑रिता॒दिति॒ दुः - च॒रि॒ता॒त् । एति॑ । मा॒ । सुच॑रित॒ इति॒ सु - च॒रि॒ते॒ । भ॒ज॒ । म॒खस्य॑ । शिरः॑ । अ॒सि॒ । समिति॑ । ज्योति॑षा । ज्योतिः॑ । अ॒ङ्क्ता॒म् ॥ \textbf{  21 } \newline
                  \newline
                      (अह्रु॑त॒ - एक॑विꣳशतिश्च)  \textbf{(A12)} \newline \newline
                                \textbf{ TS 1.1.13.1} \newline
                  वाज॑स्य । मा॒ । प्र॒स॒वेनेति॑ प्र-स॒वेन॑ । उ॒द्ग्रा॒भेणेत्यु॑त् -ग्रा॒भेण॑ । उदिति॑ । अ॒ग्र॒भी॒त् ॥ अथ॑ । स॒पत्नान्॑ । इन्द्रः॑ । मे॒ । नि॒ग्रा॒भेणेति॑ नि-ग्रा॒भेण॑ । अध॑रान् । अ॒कः॒ ॥ उ॒द्ग्रा॒भमित्यु॑त् -ग्रा॒भम् । च॒ । नि॒ग्रा॒भमिति॑ नि -ग्रा॒भम् । च॒ । ब्रह्म॑ । दे॒वाः । अ॒वी॒वृ॒ध॒न्न् ॥ अथ॑ । स॒पत्नान्॑ । इ॒न्द्रा॒ग्नी इती᳚न्द्र-अ॒ग्नी । मे॒ । वि॒षू॒चीनान्॑ । वीति॑ । अ॒स्य॒ता॒म् ॥ वसु॑भ्य॒ इति॒ वसु॑-भ्यः॒ । त्वा॒ । रु॒द्रेभ्यः॑ । त्वा॒ । आ॒दि॒त्येभ्यः॑ । त्वा॒ । अ॒क्तम् । रिहा॑णाः । वि॒यन्तु॑ । वयः॑ ॥ प्र॒जामिति॑ प्र-जाम् । योनि᳚म् । मा । निरिति॑ । मृ॒क्ष॒म् । एति॑ । प्या॒य॒न्ता॒म् । आपः॑ । ओष॑धयः । म॒रुता᳚म् । पृष॑तयः । स्थ॒ । दिवं᳚ । \textbf{  22} \newline
                  \newline
                                \textbf{ TS 1.1.13.2} \newline
                  ग॒च्छ॒ । ततः॑ । नः॒ । वृष्टि᳚म् । एति॑ । ई॒र॒य॒ ॥ आ॒यु॒ष्पा इत्या॑युः- पाः । अ॒ग्ने॒ । अ॒सि॒ । आयुः॑ । मे॒ । पा॒हि॒ । च॒क्षु॒ष्पा इति॑ - चक्षुः - पाः । अ॒ग्ने॒ । अ॒सि॒ । चक्षुः॑ । मे॒ । पा॒हि॒ । ध्रु॒वा । अ॒सि॒ । यम् । प॒रि॒धिमिति॑ परि -धिम् । प॒र्यध॑त्था॒ इति॑ परि -अध॑त्थाः । अग्ने᳚ । दे॒व॒ । प॒णिभि॒रिति॑ प॒णि -भिः॒ । वी॒यमा॑णः ॥ तम् । ते॒ । ए॒तम् । अन्विति॑ । जोष᳚म् । भ॒रा॒मि॒ । न । इत् । ए॒षः । त्वत् । अ॒प॒चे॒तया॑ता॒ इत्य॑प - चे॒तया॑तै । य॒ज्ञ्स्य॑ । पाथः॑ । उप॑ । समिति॑ । इ॒त॒म् । सꣳ॒॒स्रा॒वभा॑गा॒ इति॑ सꣳस्रा॒व -भा॒गाः॒ । स्थ॒ । इ॒षाः । बृ॒हन्तः॑ । प्र॒स्त॒रे॒ष्ठा इति॑ प्रस्तरे- स्थाः । ब॒र्.॒हि॒षद॒ इति बर्.हि-सदः॑ । च॒ । \textbf{  23} \newline
                  \newline
                                \textbf{ TS 1.1.13.3} \newline
                  दे॒वाः । इ॒माम् । वाच᳚म् । अ॒भीति॑ । विश्वे᳚ । गृ॒णन्तः॑ । आ॒सद्येत्या᳚ - सद्य॑ । अ॒स्मिन्न् । ब॒र्.॒हिषि॑ । मा॒द॒य॒ध्व॒म् । अ॒ग्नेः । वा॒म् । अप॑न्नगृह॒स्येत्यप॑न्न - गृ॒ह॒स्य॒ । सद॑सि । सा॒द॒या॒मि॒ । सु॒म्नाय॑ । सु॒म्नि॒नी॒ इति॑ । सु॒म्ने । मा॒ । ध॒त्त॒म् । धु॒रि । धु॒र्यौ᳚ । पा॒त॒म् । अग्ने᳚ । अ॒द॒ब्धा॒यो॒ इत्य॑दब्ध-आ॒यो॒ । अ॒शी॒त॒त॒नो॒ इत्य॑शित -त॒नो॒ । पा॒हि । मा॒ । अ॒द्य । दि॒वः । पा॒हि । प्रसि॑त्या॒ इति॒ प्र-सि॒त्यै॒ । पा॒हि । दुरि॑ष्ट्या॒ इति॒ दुः -इ॒ष्ट्यै॒ । पा॒हि । दु॒र॒द्म॒न्या इति॑ दुः -अ॒द्म॒न्यै । पा॒हि । दुश्च॑रिता॒दिति॒ दुः - च॒रि॒ता॒त् । अवि॑षम् । नः॒ । पि॒तुम् । कृ॒णु॒ । सु॒षदेति॑ सु -सदा᳚ । योनि᳚म् । स्वाहा᳚ । देवाः᳚ । गा॒तु॒वि॒द॒ इति॑ गातु-वि॒दः॒ । गा॒तुम् । वि॒त्त्वा । गा॒तुम् ( ) । इ॒त॒ । मन॑सः । प॒ते॒ । इ॒मम् । नः॒ । दे॒व॒ । दे॒वेषु॑ । य॒ज्ञ्म् । स्वाहा᳚ । वा॒चि । स्वाहा᳚ । वाते᳚ । धाः॒ ॥ \textbf{  24} \newline
                  \newline
                      (दिवं॑ - च - वि॒त्त्वा गा॒तुं - त्रयो॑दश च)  \textbf{(A13)} \newline \newline
                                \textbf{ TS 1.1.14.1} \newline
                  उ॒भा । वा॒म् । इ॒न्द्रा॒ग्नी॒ इती᳚न्द्र - अ॒ग्नी॒ । आ॒हु॒वध्यै᳚ । उ॒भा । राध॑सः । स॒ह । मा॒द॒यध्यै᳚ ॥ उ॒भा । दा॒तारौ᳚ । इ॒षाम् । र॒यी॒णाम् । उ॒भा । वाज॑स्य । सा॒तये᳚ । हु॒वे॒ । वा॒म् । अश्र॑वम् । हि । भू॒रि॒दाव॑त्त॒रेति॑ भूरि॒दाव॑त् -त॒रा॒ । वा॒म् ॥ विजा॑मातु॒रिति॒ वि - जा॒मा॒तुः॒ । उ॒त । वा॒ । घ॒ । स्या॒लात् ॥ अथ॑ । सोम॑स्य । प्रय॒तीति॒ प्र-य॒ती॒ । यु॒वभ्या॒मिति॑ यु॒व - भ्या॒म् । इन्द्रा᳚ग्नी॒ इतीन्द्र॑ - अ॒ग्नी॒ । स्तोम᳚म् । ज॒न॒या॒मि॒ । नव्य᳚म् ॥ इन्द्रा᳚ग्नी॒ इतीन्द्र॑ - अ॒ग्नी॒ । न॒व॒तिम् । पुरः॑ । दा॒सप॑त्नी॒रिति॑ दा॒स - प॒त्नीः॒ । अ॒धू॒नु॒त॒म् ॥ सा॒कम् । एके॑न । कर्म॑णा ॥ शुचि᳚म् । नु । स्तोम᳚म् । नव॑जात॒मिति॒ नव॑- जा॒त॒म् । अ॒द्य । इन्द्रा᳚ग्नी॒ इतीन्द्र॑ - अ॒ग्नी॒ । वृ॒त्र॒ह॒णेति॑ वृत्र -ह॒ना॒ । जु॒षेथा᳚म् ॥ \textbf{  25} \newline
                  \newline
                                \textbf{ TS 1.1.14.2} \newline
                  उ॒भा । हि । वा॒म् । सु॒हवेति॑ सु-हवा᳚ । जोह॑वीमि । ता । वाज᳚म् । स॒द्यः । उ॒श॒ते । धेष्ठा᳚ ॥ व॒यम् । उ॒ । त्वा॒ । प॒थः॒ । प॒ते॒ । रथ᳚म् । न । वाज॑सातय॒ इति॒ वाज॑-सा॒त॒ये॒ ॥ धि॒ये । पू॒ष॒न्न् । अ॒यु॒ज्म॒हि॒ ॥ प॒थस्प॑थ॒ इति॑ प॒थः - प॒थः॒ । परि॑पति॒मिति॒ परि॑-प॒ति॒म् । व॒च॒स्या । कामे॑न । कृ॒तः । अ॒भीति॑ । आ॒न॒ट् । अ॒र्कम् ॥ सः । नः॒ । रा॒स॒त् । शु॒रुधः॑ । च॒न्द्राग्रा॒ इति॑ च॒न्द्र - अ॒ग्राः॒ । धिय॑न्धिय॒मिति॒ धियं᳚ - धि॒य॒म् । सी॒ष॒धा॒ति॒ । प्रेति॑ । पू॒षा ॥ क्षेत्र॑स्य । पति॑ना । व॒यम् । हि॒तेन॑ । इ॒व॒ । ज॒या॒म॒सि॒ ॥ गाम् । अश्व᳚म् । पो॒ष॒यि॒त्नु । एति॑ । सः । नः॒ । \textbf{  26} \newline
                  \newline
                                \textbf{ TS 1.1.14.3} \newline
                  मृ॒डा॒ति॒ । ई॒दृशे᳚ ॥ क्षेत्र॑स्य । प॒ते॒ । मधु॑मन्त॒मिति॒ मधु॑-म॒न्त॒म् । ऊ॒र्मिम् । धे॒नुः । इ॒व॒ । पयः॑ । अ॒स्मासु॑ । धु॒क्ष्व॒ ॥ म॒धु॒श्चुत॒मिति॑ मधु-श्चुत᳚म् । घृ॒तम् । इ॒व॒ । सुपू॑त॒मिति॒ सु-पू॒त॒म् । ऋ॒तस्य॑ । नः॒ । पत॑यः । मृ॒ड॒य॒न्तु॒ ॥ अग्ने᳚ । नय॑ । सु॒पथेति॑ सु-पथा᳚ । रा॒ये । अ॒स्मान् । विश्वा॑नि । दे॒व॒ । व॒युना॑नि । वि॒द्वान् ॥ यु॒यो॒धि । अ॒स्मत् । जु॒हु॒रा॒णम् । एनः॑ । भूयि॑ष्ठाम् । ते॒ । नम॑उक्ति॒मिति॒ नमः॑-उ॒क्ति॒म् । वि॒धे॒म॒ ॥ एति॑ । दे॒वाना᳚म् । अपीति॑ । पन्था᳚म् । अ॒ग॒न्म॒ । यत् । श॒क्नवा॑म । तत् । अन्विति॑ । प्रवो॑ढु॒मिति॒ प्र- वो॒ढु॒म् ॥ अ॒ग्निः । वि॒द्वान् । सः । य॒जा॒त् । \textbf{  27} \newline
                  \newline
                                \textbf{ TS 1.1.14.4} \newline
                  सः । इत् । उ॒ । होता᳚ । सः । अ॒ध्व॒रान् । सः । ऋ॒तून् । क॒ल्प॒या॒ति॒ ॥ यत् । वाहि॑ष्ठम् । तत् । अ॒ग्नये᳚ । बृ॒हत् । अ॒र्च॒ । वि॒भा॒व॒सो॒ इति॑ विभा-व॒सो॒ ॥ महि॑षी । इ॒व॒ । त्वत् । र॒यिः । त्वत् । वाजाः᳚ । उदिति॑ । ई॒र॒ते॒ ॥ अग्ने᳚ । त्वम् । पा॒र॒य॒ । नव्यः॑ । अ॒स्मान् । स्व॒स्तिभि॒रिति॑ स्व॒स्ति -भिः॒ । अतीति॑ । दु॒र्गाणीति॑ दुः -गानि॑ । विश्वा᳚ ॥ पूः । च॒ । पृ॒थ्वी । ब॒हु॒ला । नः॒ । उ॒र्वी । भव॑ । तो॒काय॑ । तन॑याय । शम् । योः ॥ त्वम् । अ॒ग्ने॒ । व्र॒त॒पा इति॑ व्रत- पाः । अ॒सि॒ । दे॒वः । एति॑ ( ) । मर्त्ये॑षु । आ ॥ त्वम् । य॒ज्ञेषु॑ । ईड्यः॑ ॥ यत् । वः॒ । व॒यम् । प्र॒मि॒नामेति॑ प्र-मि॒नाम॑ । व्र॒तानि॑ । वि॒दुषा᳚म् । दे॒वाः॒ । अवि॑दुष्टरास॒ इत्यवि॑दुः - त॒रा॒सः॒ ॥ अ॒ग्निः । तत् । विश्व᳚म् । एति॑ । पृ॒णा॒ति॒ । वि॒द्वान् । येभिः॑ । दे॒वान् । ऋ॒तुभि॒रित्यृ॒तु -भिः॒ । क॒ल्पया॑ति ॥ \textbf{  28} \newline
                  \newline
                      (जु॒षेथा॒मा - स नो॑ - यजा॒ - दा - त्रयो॑विꣳशतिश्च)  \textbf{(A14)} \newline \newline
\textbf{praSna korvai with starting padams of 1 to 14 anuvAkams :-} \newline
(इ॒षे त्वा॑ - य॒ज्ञ्स्य॒ - शुन्ध॑ध्वं॒ - कर्म॑णे वां - दे॒वो-ऽव॑धूतं॒ - धुष्टिः॒ - सं ॅव॑पा॒- म्या द॑दे॒ - प्रत्यु॑ष्टं॒ - कृष्णो॑ऽसि॒ - भुव॑नमसि॒ - वाज॑स्यो॒भा वां॒ - चतु॑र्दश ) \newline

\textbf{korvai with starting padams of1, 11, 21 series of pa~jcAtis :-} \newline
(इ॒षे - दृꣳ॑ह॒ - भुव॑न - म॒ष्टाविꣳ॑शतिः ) \newline

\textbf{first and last padam of First praSnam :-} \newline
(इ॒षे त्वा॑ - क॒ल्पया॑ति) \newline 


॥ हरिः॑ ॐ ॥
॥ कृष्ण यजुर्वेदीय तैत्तिरीय संहितायां प्रथमकाण्डे प्रथमः प्रश्नः समाप्तः ॥ \newline
\pagebreak
\pagebreak
        


\end{document}
