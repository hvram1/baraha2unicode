\documentclass[17pt]{extarticle}
\usepackage{babel}
\usepackage{fontspec}
\usepackage{polyglossia}
\usepackage{extsizes}



\setmainlanguage{sanskrit}
\setotherlanguages{english} %% or other languages
\setlength{\parindent}{0pt}
\pagestyle{myheadings}
\newfontfamily\devanagarifont[Script=Devanagari]{AdishilaVedic}


\newcommand{\VAR}[1]{}
\newcommand{\BLOCK}[1]{}




\begin{document}
\begin{titlepage}
    \begin{center}
 
\begin{sanskrit}
    { \Large
    ॐ नमः परमात्मने, श्री महागणपतये नमः, श्री गुरुभ्यो नमः
ह॒रिः॒ ॐ 
    }
    \\
    \vspace{2.5cm}
    \mbox{ \Huge
    2.2     द्वितीयकाण्डे द्वितीयः प्रश्नः - इष्टिविधानं   }
\end{sanskrit}
\end{center}

\end{titlepage}
\tableofcontents

ॐ नमः परमात्मने, श्री महागणपतये नमः, 
श्री गुरुभ्यो नमः, ह॒रिः॒ ॐ \newline
2.2     द्वितीयकाण्डे द्वितीयः प्रश्नः - इष्टिविधानं \newline

\addcontentsline{toc}{section}{ 2.2     द्वितीयकाण्डे द्वितीयः प्रश्नः - इष्टिविधानं}
\markright{ 2.2     द्वितीयकाण्डे द्वितीयः प्रश्नः - इष्टिविधानं \hfill https://www.vedavms.in \hfill}
\section*{ 2.2     द्वितीयकाण्डे द्वितीयः प्रश्नः - इष्टिविधानं }
                                \textbf{ TS 2.2.1.1} \newline
                  प्र॒जाप॑ति॒रिति॑ प्र॒जा-प॒तिः॒ । प्र॒जा इति॑ प्र - जाः । अ॒सृ॒ज॒त॒ । ताः । सृ॒ष्टाः । इ॒न्द्रा॒ग्नी इती᳚न्द्र - अ॒ग्नी । अपेति॑ । अ॒गू॒ह॒ता॒म् । सः । अ॒चा॒य॒त् । प्र॒जाप॑ति॒रिति॑ प्र॒जा - प॒तिः॒ । इ॒न्द्रा॒ग्नी इती᳚न्द्र-अ॒ग्नी । वै । मे॒ । प्र॒जा इति॑ प्र - जाः । अपेति॑ । अ॒घु॒क्ष॒ता॒म् । इति॑ । सः । ए॒तम् । ऐ॒न्द्रा॒ग्नमित्यै᳚न्द्र - अ॒ग्नम् । एका॑दशकपाल॒मित्येका॑दश - क॒पा॒ल॒म् । अ॒प॒श्य॒त् । तम् । निरिति॑ । अ॒व॒प॒त् । तौ । अ॒स्मै॒ । प्र॒जा इति॑ प्र-जाः । प्रेति॑ । अ॒सा॒ध॒य॒ता॒म् । इ॒न्द्रा॒ग्नी इती᳚न्द्र - अ॒ग्नी । वै । ए॒तस्य॑ । प्र॒जामिति॑ प्र - जाम् । अपेति॑ । गू॒ह॒तः॒ । यः । अल᳚म् । प्र॒जाया॒ इति॑ प्र - जायै᳚ । सन्न् । प्र॒जामिति॑ प्र - जाम् । न । वि॒न्दते᳚ । ऐ॒न्द्रा॒ग्नमित्यै᳚न्द्र - अ॒ग्नम् । एका॑दशकपाल॒मित्येका॑दश -क॒पा॒ल॒म् । निरिति॑ । व॒पे॒त् । प्र॒जाका॑म॒ इति॑ प्र॒जा - का॒मः॒ । इ॒न्द्रा॒ग्नी इती᳚न्द्र - अ॒ग्नी । \textbf{  1} \newline
                  \newline
                                \textbf{ TS 2.2.1.2} \newline
                  ए॒व । स्वेन॑ । भा॒ग॒धेये॒नेति॑ भाग - धेये॑न । उपेति॑ । धा॒व॒ति॒ । तौ । ए॒व । अ॒स्मै॒ । प्र॒जामिति॑ प्र - जाम् । प्रेति॑ । सा॒ध॒य॒तः॒ । वि॒न्दते᳚ । प्र॒जामिति॑ प्र - जाम् । ऐ॒न्द्रा॒ग्नमित्यै᳚न्द्र - अ॒ग्नम् । एका॑दशकपाल॒मित्येका॑दश - क॒पा॒ल॒म् । निरिति॑ । व॒पे॒त् । स्पर्द्ध॑मानः । क्षेत्रे᳚ । वा॒ । स॒जा॒तेष्विति॑ स - जा॒तेषु॑ । वा॒ । इ॒न्द्रा॒ग्नी इती᳚न्द्र - अ॒ग्नी । ए॒व । स्वेन॑ । भा॒ग॒धेये॒नेति॑ भाग-धेये॑न । उपेति॑ । धा॒व॒ति॒ । ताभ्या᳚म् । ए॒व । इ॒न्द्रि॒यम् । वी॒र्य᳚म् । भ्रातृ॑व्यस्य । वृ॒ङ्क्ते॒ । वीति॑ । पा॒प्मना᳚ । भ्रातृ॑व्येण । ज॒य॒ते॒ । अपेति॑ । वै । ए॒तस्मा᳚त् । इ॒न्द्रि॒यम् । वी॒र्य᳚म् । क्रा॒म॒ति॒ । यः । स॒ग्रां॒ममिति॑ सं - ग्रा॒मम् । उ॒प॒प्र॒यातीत्यु॑प - प्र॒याति॑ । ऐ॒न्द्रा॒ग्नमित्यै᳚न्द्र-अ॒ग्नम् । एका॑दशकपाल॒मित्येका॑दश - क॒पा॒ल॒म् । निरिति॑ । \textbf{  2} \newline
                  \newline
                                \textbf{ TS 2.2.1.3} \newline
                  व॒पे॒त् । स॒ग्रां॒ममिति॑ सं - ग्रा॒मम् । उ॒प॒प्र॒या॒स्यन्नित्यु॑प - प्र॒या॒स्यन्न् । इ॒न्द्रा॒ग्नी इती᳚न्द्र - अ॒ग्नी । ए॒व । स्वेन॑ । भा॒ग॒धेये॒नेति॑ भाग - धेये॑न । उपेति॑ । धा॒व॒ति॒ । तौ । ए॒व । अ॒स्मि॒न्न् । इ॒न्द्रि॒यम् । वी॒र्य᳚म् । ध॒त्तः॒ । स॒ह । इ॒न्द्रि॒येण॑ । वी॒र्ये॑ण । उप॑ । प्रेति॑ । या॒ति॒ । जय॑ति । तम् । स॒ग्रां॒ममिति॑ सं - ग्रा॒मम् । वीति॑ । वै । ए॒षः । इ॒न्द्रि॒येण॑ । वी॒र्ये॑ण । ऋ॒द्ध्य॒ते॒ । यः । सं॒ग्रा॒ममिति॑ सं - ग्रा॒मम् । जय॑ति । ऐ॒न्द्रा॒ग्नमित्यै᳚न्द्र - अ॒ग्नम् । एका॑दशकपाल॒मित्येका॑दश - क॒पा॒ल॒म् । निरिति॑ । व॒पे॒त् । स॒ग्रां॒ममिति॑ सं - ग्रा॒मम् । जि॒त्वा । इ॒न्द्रा॒ग्नी इती᳚न्द्र - अ॒ग्नी । ए॒व । स्वेन॑ । भा॒ग॒धेये॒नेति॑ भाग - धेये॑न । उपेति॑ । धा॒व॒ति॒ । तौ । ए॒व । अ॒स्मि॒न्न् । इ॒न्द्रि॒यम् । वी॒र्य᳚म् । \textbf{  3} \newline
                  \newline
                                \textbf{ TS 2.2.1.4} \newline
                  ध॒त्तः॒ । न । इ॒न्द्रि॒येण॑ । वी॒र्ये॑ण । वीति॑ । ऋ॒द्ध्य॒ते॒ । अपेति॑ । वै । ए॒तस्मा᳚त् । इ॒न्द्रि॒यम् । वी॒र्य᳚म् । क्रा॒म॒ति॒ । यः । एति॑ । ज॒नता᳚म् । ऐ॒न्द्रा॒ग्नमित्यै᳚न्द्र - अ॒ग्नम् । एका॑दशकपाल॒मित्येका॑दश-क॒पा॒ल॒म् । निरिति॑ । व॒पे॒त् । ज॒नता᳚म् । ए॒ष्यन्न् । इ॒न्द्रा॒ग्नी इती᳚न्द्र-अ॒ग्नी । ए॒व । स्वेन॑ । भा॒ग॒धेये॒नेति॑ भाग - धेये॑न । उपेति॑ । धा॒व॒ति॒ । तौ । ए॒व । अ॒स्मि॒न्न् । इ॒न्द्रि॒यम् । वी॒र्य᳚म् । ध॒त्तः॒ । स॒ह । इ॒न्द्रि॒येण॑ । वी॒र्ये॑ण । ज॒नता᳚म् । ए॒ति॒ । पौ॒ष्णम् । च॒रुम् । अनु॑ । निरिति॑ । व॒पे॒त् । पू॒षा । वै । इ॒न्द्रि॒यस्य॑ । वी॒र्य॑स्य । अ॒नु॒प्र॒दा॒तेत्य॑नु - प्र॒दा॒ता । पू॒षण᳚म् । ए॒व । \textbf{  4} \newline
                  \newline
                                \textbf{ TS 2.2.1.5} \newline
                  स्वेन॑ । भा॒ग॒धेये॒नेति॑ भाग - धेये॑न । उपेति॑ । धा॒व॒ति॒ । सः । ए॒व । अ॒स्मै॒ । इ॒न्द्रि॒यम् । वी॒र्य᳚म् । अनु॑ । प्रेति॑ । य॒च्छ॒ति॒ । क्षै॒त्र॒प॒त्यमिति॑ क्षैत्र - प॒त्यम् । च॒रुम् । निरिति॑ । व॒पे॒त् । ज॒नता᳚म् । आ॒गत्येत्या᳚ - गत्य॑ । इ॒यम् । वै । क्षेत्र॑स्य । पतिः॑ । अ॒स्याम् । ए॒व । प्रतीति॑ । ति॒ष्ठ॒ति॒ । ऐ॒न्द्रा॒ग्नमित्यै᳚न्द्र - अ॒ग्नम् । एका॑दशकपाल॒मित्येका॑दश-क॒पा॒ल॒म् । उ॒परि॑ष्टात् । निरिति॑ । व॒पे॒त् । अ॒स्याम् । ए॒व । प्र॒ति॒ष्ठायेति॑ प्रति - स्थाय॑ । इ॒न्द्रि॒यम् । वी॒र्य᳚म् । उ॒परि॑ष्टात् । आ॒त्मन्न् । ध॒त्ते॒ ॥ \textbf{  5} \newline
                  \newline
                      (प्र॒जाका॑म इन्द्रा॒ग्नी - उ॑पप्र॒यात्यै᳚न्द्रा॒ग्नमेका॑दशकपालं॒ नि- र्वी॒र्यं॑ - पू॒षण॑मे॒ वै - का॒न्नच॑त्वारिꣳ॒॒शच्च॑ )  \textbf{(A1)} \newline \newline
                                \textbf{ TS 2.2.2.1} \newline
                  अ॒ग्नये᳚ । प॒थि॒कृत॒ इति॑ पथि - कृते᳚ । पु॒रो॒डाश᳚म् । अ॒ष्टाक॑पाल॒मित्य॒ष्टा - क॒पा॒ल॒म् । निरिति॑ । व॒पे॒त् । यः । द॒र्॒.श॒पू॒र्ण॒मा॒स॒या॒जीति॑ दर्.शपूर्णमास - या॒जी । सन्न् । अ॒मा॒वा॒स्या॑मित्य॑मा - वा॒स्या᳚म् । वा॒ । पौ॒र्ण॒मा॒सीमिति॑ पौर्ण - मा॒सीम् । वा॒ । अ॒ति॒पा॒दये॒दित्य॑ति - पा॒दये᳚त् । प॒थः । वै । ए॒षः । अधीति॑ । अप॑थेन । ए॒ति॒ । यः । द॒र्.॒श॒पू॒र्ण॒मा॒स॒या॒जीति॑ दर्.शपूर्णमास - या॒जी । सन्न् । अ॒मा॒वा॒स्या॑मित्य॑मा - वा॒स्या᳚म् । वा॒ । पौ॒र्ण॒मा॒सीमिति॑ पौर्ण - मा॒सीम् । वा॒ । अ॒ति॒पा॒दय॒तीत्य॑ति - पा॒दय॑ति । अ॒ग्निम् । ए॒व । प॒थि॒कृत॒मिति॑ पथि - कृत᳚म् । स्वेन॑ । भा॒ग॒धेये॒नेति॑ भाग - धेये॑न । उपेति॑ । धा॒व॒ति॒ । सः । ए॒व । ए॒न॒म् । अप॑थात् । पन्था᳚म् । अपीति॑ । न॒य॒ति॒ । अ॒न॒ड्वान् । दक्षि॑णा । व॒ही । हि । ए॒षः । समृ॑द्ध्या॒ इति॒ सं - ऋ॒द्ध्यै॒ । अ॒ग्नये᳚ । व्र॒तप॑तय॒ इति॑ व्र॒त-प॒त॒ये॒ । \textbf{  6} \newline
                  \newline
                                \textbf{ TS 2.2.2.2} \newline
                  पु॒रो॒डाश᳚म् । अ॒ष्टाक॑पाल॒मित्य॒ष्टा - क॒पा॒ल॒म् । निरिति॑ । व॒पे॒त् । यः । आहि॑ताग्नि॒रित्याहि॑त-अ॒ग्निः॒ । सन्न् । अ॒व्र॒त्यम् । इ॒व॒ । चरे᳚त् । अ॒ग्निम् । ए॒व । व्र॒तप॑ति॒मिति॑ व्र॒त - प॒ति॒म् । स्वेन॑ । भा॒ग॒धेये॒नेति॑ भाग - धेये॑न । उपेति॑ । धा॒व॒ति॒ । सः । ए॒व । ए॒न॒म् । व्र॒तम् । एति॑ । ल॒भं॒य॒ति॒ । व्रत्यः॑ । भ॒व॒ति॒ । अ॒ग्नये᳚ । र॒क्षो॒घ्न इति॑ रक्षः - घ्ने । पु॒रो॒डाश᳚म् । अ॒ष्टाक॑पाल॒मित्य॒ष्टा - क॒पा॒ल॒म् । निरिति॑ । व॒पे॒त् । यम् । रक्षाꣳ॑सि । सचे॑रन्न् । अ॒ग्निम् । ए॒व । र॒क्षो॒हण॒मिति॑ रक्षः - हन᳚म् । स्वेन॑ । भा॒ग॒धेये॒नेति॑ भाग-धेये॑न । उपेति॑ । धा॒व॒ति॒ । सः । ए॒व । अ॒स्मा॒त् । रक्षाꣳ॑सि । अपेति॑ । ह॒न्ति॒ । निशि॑ताया॒मिति॒ नि - शि॒ता॒या॒म् । निरिति॑ । व॒पे॒त् । \textbf{  7} \newline
                  \newline
                                \textbf{ TS 2.2.2.3} \newline
                  निशि॑ताया॒मिति॒ नि - शि॒ता॒या॒म् । हि । रक्षाꣳ॑सि । प्रे॒रत॒ इति॑ प्र - ई॒रते᳚ । स॒प्रेंर्णा॒नीति॑ सं - प्रेर्णा॑नि । ए॒व । ए॒ना॒नि॒ । ह॒न्ति॒ । परि॑श्रित॒ इति॒ परि॑ - श्रि॒ते॒ । या॒ज॒ये॒त् । रक्ष॑साम् । अन॑न्ववचारा॒येत्यन॑नु - अ॒व॒चा॒रा॒य॒ । र॒क्षो॒घ्नी इति॑ रक्षः - घ्नी । या॒ज्या॒नु॒वा॒क्ये॑ इति॑ याज्या - अ॒नु॒वा॒क्ये᳚ । भ॒व॒तः॒ । रक्ष॑साम् । स्तृत्यै᳚ । अ॒ग्नये᳚ । रु॒द्रव॑त॒ इति॑ रु॒द्र - व॒ते॒ । पु॒रो॒डाश᳚म् । अ॒ष्टाक॑पाल॒मित्य॒ष्टा - क॒पा॒ल॒म् । निरिति॑ । व॒पे॒त् । अ॒भि॒चर॒न्नित्य॑भि - चरन्न्॑ । ए॒षा । वै । अ॒स्य॒ । घो॒रा । त॒नूः । यत् । रु॒द्रः । तस्मै᳚ । ए॒व । ए॒न॒म् । एति॑ । वृ॒श्च॒ति॒ । ता॒जक् । आर्ति᳚म् । एति॑ । ऋ॒च्छ॒ति॒ । अ॒ग्नये᳚ । सु॒र॒भि॒मत॒ इति॑ सुरभि - मते᳚ । पु॒रो॒डाश᳚म् । अ॒ष्टाक॑पाल॒मित्य॒ष्टा - क॒पा॒ल॒म् । निरिति॑ । व॒पे॒त् । यस्य॑ । गावः॑ । वा॒ । पुरु॑षाः । \textbf{  8} \newline
                  \newline
                                \textbf{ TS 2.2.2.4} \newline
                  वा॒ । प्र॒मीये॑र॒न्निति॑ प्र - मीये॑रन्न् । यः । वा॒ । बि॒भी॒यात् । ए॒षा । वै । अ॒स्य॒ । भे॒ष॒ज्या᳚ । त॒नूः । यत् । सु॒र॒भि॒मतीति॑ सुरभि-मती᳚ । तया᳚ । ए॒व । अ॒स्मै॒ । भे॒ष॒जम् । क॒रो॒ति॒ । सु॒र॒भि॒मत॒ इति॑ सुरभि - मते᳚ । भ॒व॒ति॒ । पू॒ती॒ग॒न्धस्येति॑ पूती - ग॒न्धस्य॑ । अप॑हत्या॒ इत्यप॑ - ह॒त्यै॒ । अ॒ग्नये᳚ । क्षाम॑वत॒ इति॒ क्षाम॑ - व॒ते॒ । पु॒रो॒डाश᳚म् । अ॒ष्टाक॑पाल॒मित्य॒ष्टा - क॒पा॒ल॒म् । निरिति॑ । व॒पे॒त् । स॒ग्रां॒म इति॑ सं - ग्रा॒मे । संॅय॑त्त॒ इति॒ सं - य॒त्ते॒ । भा॒ग॒धेये॒नेति॑ भाग - धेये॑न । ए॒व । ए॒न॒म् । श॒म॒यि॒त्वा । परान्॑ । अ॒भि । निरिति॑ । दि॒श॒ति॒ । यम् । अव॑रेषाम् । विद्ध्य॑न्ति । जीव॑ति । सः । यम् । परे॑षाम् । प्रेति॑ । सः । मी॒य॒ते॒ । जय॑ति । तम् । स॒ग्रां॒ममिति॑ सं - ग्रा॒मम् । \textbf{  9} \newline
                  \newline
                                \textbf{ TS 2.2.2.5} \newline
                  अ॒भीति॑ । वै । ए॒षः । ए॒तान् । उ॒च्य॒ति॒ । येषा᳚म् । पू॒र्वा॒प॒रा इति॑ पूर्व - अ॒प॒राः । अ॒न्वञ्चः॑ । प्र॒मीय॑न्त॒ इति॑ प्र - मीय॑न्ते । पु॒रु॒षा॒हु॒तिरिति॑ पुरुष - आ॒हु॒तिः । हि । अ॒स्य॒ । प्रि॒यत॒मेति॑ प्रि॒य - त॒मा॒ । अ॒ग्नये᳚ । क्षाम॑वत॒ इति॒ क्षाम॑ - व॒ते॒ । पु॒रो॒डाश᳚म् । अ॒ष्टाक॑पाल॒मित्य॒ष्टा - क॒पा॒ल॒म् । निरिति॑ । व॒पे॒त् । भा॒ग॒धेये॒नेति॑ भाग - धेये॑न । ए॒व । ए॒न॒म् । श॒म॒य॒ति॒ । न । ए॒षा॒म् । पु॒रा । आयु॑षः । अप॑रः । प्रेति॑ । मी॒य॒ते॒ । अ॒भीति॑ । वै । ए॒षः । ए॒तस्य॑ । गृ॒हान् । उ॒च्य॒ति॒ । यस्य॑ । गृ॒हान् । दह॑ति । अ॒ग्नये᳚ । क्षाम॑वत॒ इति॒ क्षाम॑ - व॒ते॒ । पु॒रो॒डाश᳚म् । अ॒ष्टाक॑पाल॒मित्य॒ष्टा - क॒पा॒ल॒म् । निरिति॑ । व॒पे॒त् । भा॒ग॒धेये॒नेति॑ भाग - धेये॑न । ए॒व । ए॒न॒म् । श॒य॒म॒ति॒ । न ( ) । अ॒स्य॒ । अप॑रम् । गृ॒हान् । द॒ह॒ति॒ ॥ \textbf{  10 } \newline
                  \newline
                      (व्र॒तप॑तये॒ - निशि॑तायां॒ निर्व॑पे॒त् - पुरु॑षाः - संग्रा॒मं - न - च॒त्वारि॑ च)  \textbf{(A2)} \newline \newline
                                \textbf{ TS 2.2.3.1} \newline
                  अ॒ग्नये᳚ । कामा॑य । पु॒रो॒डाश᳚म् । अ॒ष्टाक॑पाल॒मित्य॒ष्टा - क॒पा॒ल॒म् । निरिति॑ । व॒पे॒त् । यम् । कामः॑ । न । उ॒प॒नमे॒दित्यु॑प - नमे᳚त् । अ॒ग्निम् । ए॒व । काम᳚म् । स्वेन॑ । भा॒ग॒धेये॒नेति॑ भाग - धेये॑न । उपेति॑ । धा॒व॒ति॒ । सः । ए॒व । ए॒न॒म् । कामे॑न । समिति॑ । अ॒र्द्ध॒य॒ति॒ । उपेति॑ । ए॒न॒म् । कामः॑ । न॒म॒ति॒ । अ॒ग्नये᳚ । यवि॑ष्ठाय । पु॒रो॒डाश᳚म् । अ॒ष्टाक॑पाल॒मित्य॒ष्टा-क॒पा॒ल॒म् । निरिति॑ । व॒पे॒त् । स्पर्द्ध॑मानः । क्षेत्रे᳚ । वा॒ । स॒जा॒तेष्विति॑ स - जा॒तेषु॑ । वा॒ । अ॒ग्निम् । ए॒व । यवि॑ष्ठम् । स्वेन॑ । भा॒ग॒धेये॒नेति॑ भाग - धेये॑न । उपेति॑ । धा॒व॒ति॒ । तेन॑ । ए॒व । इ॒न्द्रि॒यम् । वी॒र्य᳚म् । भ्रातृ॑व्यस्य । \textbf{  11} \newline
                  \newline
                                \textbf{ TS 2.2.3.2} \newline
                  यु॒व॒ते॒ । वीति॑ । पा॒प्मना᳚ । भ्रातृ॑व्येण । ज॒य॒ते॒ । अ॒ग्नये᳚ । यवि॑ष्ठाय । पु॒रो॒डाश᳚म् । अ॒ष्टाक॑पाल॒मित्य॒ष्टा - क॒पा॒ल॒म् । निरिति॑ । व॒पे॒त् । अ॒भि॒च॒र्यमा॑ण॒ इत्य॑भि - च॒र्यमा॑णः । अ॒ग्निम् । ए॒व । यवि॑ष्ठम् । स्वेन॑ । भा॒ग॒धेये॒नेति॑ भाग - धेये॑न । उपेति॑ । धा॒व॒ति॒ । सः । ए॒व । अ॒स्मा॒त् । रक्षाꣳ॑सि । य॒व॒य॒ति॒ । न । ए॒न॒म् । अ॒भि॒चर॒न्नित्य॑भि - चरन्न्॑ । स्तृ॒णु॒ते॒ । अ॒ग्नये᳚ । आयु॑ष्मते । पु॒रो॒डाश᳚म् । अ॒ष्टाक॑पाल॒मित्य॒ष्टा-क॒पा॒ल॒म् । निरिति॑ । व॒पे॒त् । यः । का॒मये॑त । सर्व᳚म् । आयुः॑ । इ॒या॒म् । इति॑ । अ॒ग्निम् । ए॒व । आयु॑ष्मन्तम् । स्वेन॑ । भा॒ग॒धेये॒नेति॑ भाग - धेये॑न । उपेति॑ । धा॒व॒ति॒ । सः । ए॒व । अ॒स्मि॒न्न् । \textbf{  12} \newline
                  \newline
                                \textbf{ TS 2.2.3.3} \newline
                  आयुः॑ । द॒धा॒ति॒ । सर्व᳚म् । आयुः॑ । ए॒ति॒ । अ॒ग्नये᳚ । जा॒तवे॑दस॒ इति॑ जा॒त - वे॒द॒से॒ । पु॒रो॒डाश᳚म् । अ॒ष्टाक॑पाल॒मित्य॒ष्टा - क॒पा॒ल॒म् । निरिति॑ । व॒पे॒त् । भूति॑काम॒ इति॒ भूति॑ - का॒मः॒ । अ॒ग्निम् । ए॒व । जा॒तवे॑दस॒मिति॑ जा॒त - वे॒द॒स॒म् । स्वेन॑ । भा॒ग॒धेये॒नेति॑ भाग - धेये॑न । उपेति॑ । धा॒व॒ति॒ । सः । ए॒व । ए॒न॒म् । भूति᳚म् । ग॒म॒य॒ति॒ । भव॑ति । ए॒व । अ॒ग्नये᳚ । रुक्म॑ते । पु॒रो॒डाश᳚म् । अ॒ष्टाक॑पाल॒मित्य॒ष्टा-क॒पा॒ल॒म् । निरिति॑ । व॒पे॒त् । रुक्का॑म॒ इति॒ रुक् - का॒मः॒ । अ॒ग्निम् । ए॒व । रुक्म॑न्तम् । स्वेन॑ । भा॒ग॒धेये॒नेति॑ भाग - धेये॑न । उपेति॑ । धा॒व॒ति॒ । सः । ए॒व । अ॒स्मि॒न्न् । रुच᳚म् । द॒धा॒ति॒ । रोच॑ते । ए॒व । अ॒ग्नये᳚ । तेज॑स्वते । पु॒रो॒डाश᳚म् । \textbf{  13} \newline
                  \newline
                                \textbf{ TS 2.2.3.4} \newline
                  अ॒ष्टाक॑पाल॒मित्य॒ष्टा - क॒पा॒ल॒म् । निरिति॑ । व॒पे॒त् । तेज॑स्काम॒ इति॒ तेजः॑ - का॒मः॒ । अ॒ग्निम् । ए॒व । तेज॑स्वन्तम् । स्वेन॑ । भा॒ग॒धेये॒नेति॑ भाग-धेये॑न । उपेति॑ । धा॒व॒ति॒ । सः । ए॒व । अ॒स्मि॒न्न् । तेजः॑ । द॒धा॒ति॒ । ते॒ज॒स्वी । ए॒व । भ॒व॒ति॒ । अ॒ग्नये᳚ । सा॒ह॒न्त्याय॑ । पु॒रो॒डाश᳚म् । अ॒ष्टाक॑पाल॒मित्य॒ष्टा - क॒पा॒ल॒म् । निरिति॑ । व॒पे॒त् । सीक्ष॑माणः । अ॒ग्निम् । ए॒व । सा॒ह॒न्त्यम् । स्वेन॑ । भा॒ग॒धेये॒नेति॑ भाग - धेये॑न । उपेति॑ । धा॒व॒ति॒ । तेन॑ । ए॒व । स॒ह॒ते॒ । यम् । सीक्ष॑ते ॥ \textbf{  14 } \newline
                  \newline
                      (भ्रातृ॑व्यस्या -स्मि॒न् - तेज॑स्वते पुरो॒डश॑ - म॒ष्टात्रिꣳ॑शच्च)  \textbf{(A3)} \newline \newline
                                \textbf{ TS 2.2.4.1} \newline
                  अ॒ग्नये᳚ । अन्न॑वत॒ इत्यन्न॑ - व॒ते॒ । पु॒रो॒डाश᳚म् । अ॒ष्टाक॑पाल॒मित्य॒ष्टा - क॒पा॒ल॒म् । निरिति॑ । व॒पे॒त् । यः । का॒मये॑त । अन्न॑वा॒नित्यन्न॑ - वा॒न् । स्या॒म् । इति॑ । अ॒ग्निम् । ए॒व । अन्न॑वन्त॒मित्यन्न॑ - व॒न्त॒म् । स्वेन॑ । भा॒ग॒धेये॒नेति॑ भाग - धेये॑न । उपेति॑ । धा॒व॒ति॒ । सः । ए॒व । ए॒न॒म् । अन्न॑वन्त॒मित्यन्न॑ - व॒न्त॒म् । क॒रो॒ति॒ । अन्न॑वा॒नित्यन्न॑ - वा॒न् । ए॒व । भ॒व॒ति॒ । अ॒ग्नये᳚ । अ॒न्ना॒दायेत्य॑न्न - अ॒दाय॑ । पु॒रो॒डाश᳚म् । अ॒ष्टाक॑पाल॒मित्य॒ष्टा - क॒पा॒ल॒म् । निरिति॑ । व॒पे॒त् । यः । का॒मये॑त । अ॒न्ना॒द इत्य॑न्न - अ॒दः । स्या॒म् । इति॑ । अ॒ग्निम् । ए॒व । अ॒न्ना॒दमित्य॑न्न - अ॒दम् । स्वेन॑ । भा॒ग॒धेये॒नेति॑ भाग - धेये॑न । उपेति॑ । धा॒व॒ति॒ । सः । ए॒व । ए॒न॒म् । अ॒न्ना॒दमित्य॑न्न - अ॒दम् । क॒रो॒ति॒ । अ॒न्ना॒द इत्य॑न्न - अ॒दः । \textbf{  15} \newline
                  \newline
                                \textbf{ TS 2.2.4.2} \newline
                  ए॒व । भ॒व॒ति॒ । अ॒ग्नये᳚ । अन्न॑पतय॒ इत्यन्न॑ - प॒त॒ये॒ । पु॒रो॒डाश᳚म् । अ॒ष्टाक॑पाल॒मित्य॒ष्टा - क॒पा॒ल॒म् । निरिति॑ । व॒पे॒त् । यः । का॒मये॑त । अन्न॑पति॒रित्यन्न॑ - प॒तिः॒ । स्या॒म् । इति॑ । अ॒ग्निम् । ए॒व । अन्न॑पति॒मित्यन्न॑ - प॒ति॒म् । स्वेन॑ । भा॒ग॒धेये॒नेति॑ भाग - धेये॑न । उपेति॑ । धा॒व॒ति॒ । सः । ए॒व । ए॒न॒म् । अन्न॑पति॒मित्यन्न॑ - प॒ति॒म् । क॒रो॒ति॒ । अन्न॑पति॒रित्यन्न॑ - प॒तिः॒ । ए॒व । भ॒व॒ति॒ । अ॒ग्नये᳚ । पव॑मानाय । पु॒रो॒डाश᳚म् । अ॒ष्टाक॑पाल॒मित्य॒ष्टा - क॒पा॒ल॒म् । निरिति॑ । व॒पे॒त् । अ॒ग्नये᳚ । पा॒व॒काय॑ । अ॒ग्नये᳚ । शुच॑ये । ज्योगा॑मया॒वीति॒ ज्योक् - आ॒म॒या॒वी॒ । यत् । अ॒ग्नये᳚ । पव॑मानाय । नि॒र्वप॒तीति॑ निः - वप॑ति । प्रा॒णमिति॑ प्र - अ॒नम् । ए॒व । अ॒स्मि॒न्न् । तेन॑ । द॒धा॒ति॒ । यत् । अ॒ग्नये᳚ । \textbf{  16} \newline
                  \newline
                                \textbf{ TS 2.2.4.3} \newline
                  पा॒व॒काय॑ । वाच᳚म् । ए॒व । अ॒स्मि॒न्न् । तेन॑ । द॒धा॒ति॒ । यत् । अ॒ग्नये᳚ । शुच॑ये । आयुः॑ । ए॒व । अ॒स्मि॒न्न् । तेन॑ । द॒धा॒ति॒ । उ॒त । यदि॑ । इ॒तासु॒रिती॒त-अ॒सुः॒ । भव॑ति । जीव॑ति । ए॒व । ए॒ताम् । ए॒व । निरिति॑ । व॒पे॒त् । चक्षु॑ष्काम॒ इति॒ चक्षुः॑ - का॒मः॒ । यत् । अ॒ग्नये᳚ । पव॑मानाय । नि॒र्वप॒तीति॑ निः - वप॑ति । प्रा॒णमिति॑ प्र-अ॒नम् । ए॒व । अ॒स्मि॒न्न् । तेन॑ । द॒धा॒ति॒ । यत् । अ॒ग्नये᳚ । पा॒व॒काय॑ । वाच᳚म् । ए॒व । अ॒स्मि॒न्न् । तेन॑ । द॒धा॒ति॒ । यत् । अ॒ग्नये᳚ । शुच॑ये । चक्षुः॑ । ए॒व । अ॒स्मि॒न्न् । तेन॑ । द॒धा॒ति॒ । \textbf{  17} \newline
                  \newline
                                \textbf{ TS 2.2.4.4} \newline
                  उ॒त । यदि॑ । अ॒न्धः । भव॑ति । प्रेति॑ । ए॒व । प॒श्य॒ति॒ । अ॒ग्नये᳚ । पु॒त्रव॑त॒ इति॑ पु॒त्र - व॒ते॒ । पु॒रो॒डाश᳚म् । अ॒ष्टाक॑पाल॒मित्य॒ष्टा - क॒पा॒ल॒म् । निरिति॑ । व॒पे॒त् । इन्द्रा॑य । पु॒त्रिणे᳚ । पु॒रो॒डाश᳚म् । एका॑दशकपाल॒मित्येका॑दश - क॒पा॒ल॒म् । प्र॒जाका॑म॒ इति॑ प्र॒जा - का॒मः॒ । अ॒ग्निः । ए॒व । अ॒स्मै॒ । प्र॒जामिति॑ प्र-जाम् । प्र॒ज॒नय॒तीति॑ प्र - ज॒नय॑ति । वृ॒द्धाम् । इन्द्रः॑ । प्रेति॑ । य॒च्छ॒ति॒ । अ॒ग्नये᳚ । रस॑वत॒ इति॒ रस॑-व॒ते॒ । अ॒ज॒क्षी॒र इत्य॑ज - क्षी॒रे । च॒रुम् । निरिति॑ । व॒पे॒त् । यः । का॒मये॑त । रस॑वा॒निति॒ रस॑ - वा॒न् । स्या॒म् । इति॑ । अ॒ग्निम् । ए॒व । रस॑वन्त॒मिति॒ रस॑ - व॒न्त॒म् । स्वेन॑ । भा॒ग॒धेये॒नेति॑ भाग - धेये॑न । उपेति॑ । धा॒व॒ति॒ । सः । ए॒व । ए॒न॒म् । रस॑वन्त॒मिति॒ रस॑ - व॒न्त॒म् । क॒रो॒ति॒ । \textbf{  18} \newline
                  \newline
                                \textbf{ TS 2.2.4.5} \newline
                  रस॑वा॒निति॒ रस॑ - वा॒न् । ए॒व । भ॒व॒ति॒ । अ॒ज॒क्षी॒र इत्य॑ज - क्षी॒रे । भ॒व॒ति॒ । आ॒ग्ने॒यी । वै । ए॒षा । यत् । अ॒जा । सा॒क्षादिति॑ स - अ॒क्षात् । ए॒व । रस᳚म् । अवेति॑ । रु॒न्धे॒ । अ॒ग्नये᳚ । वसु॑मत॒ इति॒ वसु॑ - म॒ते॒ । पु॒रो॒डाश᳚म् । अ॒ष्टाक॑पाल॒मित्य॒ष्टा - क॒पा॒ल॒म् । निरिति॑ । व॒पे॒त् । यः । का॒मये॑त । वसु॑मा॒निति॒ वसु॑ - मा॒न् । स्या॒म् । इति॑ । अ॒ग्निम् । ए॒व । वसु॑मन्त॒मिति॒ वसु॑ - म॒न्त॒म् । स्वेन॑ । भा॒ग॒धेये॒नेति॑ भाग - धेये॑न । उपेति॑ । धा॒व॒ति॒ । सः । ए॒व । ए॒न॒म् । वसु॑मन्त॒मिति॒ वसु॑ - म॒न्त॒म् । क॒रो॒ति॒ । वसु॑मा॒निति॒॒ वसु॑ - मा॒न् । ए॒व । भ॒व॒ति॒ । अ॒ग्नये᳚ । वा॒ज॒सृत॒ इति॑ वाज - सृते᳚ । पु॒रो॒डाश᳚म् । अ॒ष्टाक॑पाल॒मित्य॒ष्टा - क॒पा॒ल॒म् । निरिति॑ । व॒पे॒त् । स॒ग्रां॒म इति॑ सं - ग्रा॒मे । संॅय॑त्त॒ इति॒ सं - य॒त्ते॒ । वाज᳚म् । \textbf{  19} \newline
                  \newline
                                \textbf{ TS 2.2.4.6} \newline
                  वै । ए॒षः । सि॒सी॒र्.॒ष॒ति॒ । यः । स॒ग्रां॒ममिति॑ सं - ग्रा॒मम् । जिगी॑षति । अ॒ग्निः । खलु॑ । वै । दे॒वाना᳚म् । वा॒ज॒सृदिति॑ वाज - सृत् । अ॒ग्निम् । ए॒व । वा॒ज॒सृत॒मिति॑ वाज-सृत᳚म् । स्वेन॑ । भा॒ग॒धेये॒नेति॑ भाग - धेये॑न । उपेति॑ । धा॒व॒ति॒ । धाव॑ति । वाज᳚म् । हन्ति॑ । वृ॒त्रम् । जय॑ति । तम् । स॒ग्रां॒ममिति॑ सं - ग्रा॒मम् । अथो॒ इति॑ । अ॒ग्निः । इ॒व॒ । न । प्र॒ति॒धृष॒ इति॑ प्रति - धृषे᳚ । भ॒व॒ति॒ । अ॒ग्नये᳚ । अ॒ग्नि॒वत॒ इत्य॑ग्नि - वते᳚ । पु॒रो॒डाश᳚म् । अ॒ष्टाक॑पाल॒मित्य॒ष्टा - क॒पा॒ल॒म् । निरिति॑ । व॒पे॒त् । यस्य॑ । अ॒ग्नौ । अ॒ग्निम् । अ॒भ्यु॒द्धरे॑यु॒रित्य॑भि - उ॒द्धरे॑युः । निर्दि॑ष्टभाग॒ इति॒ निर्दि॑ष्ट - भा॒गः॒ । वै । ए॒तयोः᳚ । अ॒न्यः । अनि॑र्दिष्टभाग॒ इत्यनि॑र्दिष्ट - भा॒गः॒ । अ॒न्यः । तौ । स॒भंव॑न्ता॒विति॑ सं - भव॑न्तौ । यज॑मानम् । \textbf{  20} \newline
                  \newline
                                \textbf{ TS 2.2.4.7} \newline
                  अ॒भि । समिति॑ । भ॒व॒तः॒ । सः । ई॒श्व॒रः । आर्ति᳚म् । आर्तो॒रित्या - अ॒र्तोः॒ । यत् । अ॒ग्नये᳚ । अ॒ग्नि॒वत॒ इत्य॑ग्नि - वते᳚ । नि॒र्वप॒तीति॑ निः-वप॑ति । भा॒ग॒धेये॒नेति॑ भाग - धेये॑न । ए॒व । ए॒नौ॒ । श॒म॒य॒ति॒ । न । आर्ति᳚म् । एति॑ । ऋ॒च्छ॒ति॒ । यज॑मानः । अ॒ग्नये᳚ । ज्योति॑ष्मते । पु॒रो॒डाश᳚म् । अ॒ष्टाक॑पाल॒मित्य॒ष्टा - क॒पा॒ल॒म् । निरिति॑ । व॒पे॒त् । यस्य॑ । अ॒ग्निः । उद्धृ॑त॒ इत्युत् - हृ॒तः॒ । अहु॑ते । अ॒ग्नि॒हो॒त्र इत्य॑ग्नि - हो॒त्रे । उ॒द्वाये॒दित्यु॑त् - वाये᳚त् । अप॑रः । आ॒दीप्येत्या᳚ - दीप्य॑ । अ॒नू॒द्धृत्य॒ इत्य॑नु - उ॒द्धृत्यः॑ । इति॑ । आ॒हुः॒ । तत् । तथा᳚ । न । का॒र्य᳚म् । यत् । भा॒ग॒धेय॒मिति॑ भाग - धेय᳚म् । अ॒भीति॑ । पूर्वः॑ । उ॒द्ध्रि॒यत॒ इत्यु॑त् - ह्रि॒यते᳚ । किम् । अप॑रः । अ॒भि । उदिति॑ । \textbf{  21} \newline
                  \newline
                                \textbf{ TS 2.2.4.8} \newline
                  ह्रि॒ये॒त॒ । इति॑ । तानि॑ । ए॒व । अ॒व॒क्षाणा॒नीत्य॑व - क्षाणा॑नि । स॒न्नि॒धायेति॑ सं - नि॒धाय॑ । म॒न्थे॒त् । इ॒तः । प्र॒थ॒मम् । ज॒ज्ञे॒ । अ॒ग्निः । स्वात् । योनेः᳚ । अधीति॑ । जा॒तवे॑दा॒ इति॑ जा॒त - वे॒दाः॒ ॥ सः । गा॒य॒त्रि॒या । त्रि॒ष्टुभा᳚ । जग॑त्या । दे॒वेभ्यः॑ । ह॒व्यम् । व॒ह॒तु॒ । प्र॒जा॒नन्निति॑ प्र - जा॒नन्न् । इति॑ । छन्दो॑भि॒रिति॒ छन्दः॑ - भिः॒ । ए॒व । ए॒न॒म् । स्वात् । योनेः᳚ । प्रेति॑ । ज॒न॒य॒ति॒ । ए॒षः । वाव । सः । अ॒ग्निः । इति॑ । आ॒हुः॒ । ज्योतिः॑ । तु । वै । अ॒स्य॒ । परा॑पतित॒मिति॒ परा᳚ - प॒ति॒त॒म् । इति॑ । यत् । अ॒ग्नये᳚ । ज्योति॑ष्मते । नि॒र्वप॒तीति॑ निः - वप॑ति । यत् । ए॒व । अ॒स्य॒ ( ) । ज्योतिः॑ । परा॑पतित॒मिति॒ परा᳚ - प॒ति॒त॒म् । तत् । ए॒व । अवेति॑ । रु॒न्धे॒ ॥ \textbf{  22} \newline
                  \newline
                      (क॒रो॒त्य॒न्ना॒दो - द॑धाति॒ यद॒ग्नये॒ - शुच॑ये॒ चक्षु॑रे॒वास्मि॒न् तेन॑ दधाति -करोति॒ - वाजं॒ -ॅयज॑मान॒ - मु - दे॒वास्य॒ - षट्च॑)  \textbf{(A4)} \newline \newline
                                \textbf{ TS 2.2.5.1} \newline
                  वै॒श्वा॒न॒रम् । द्वाद॑शकपाल॒मिति॒ द्वाद॑श - क॒पा॒ल॒म् । निरिति॑ । व॒पे॒त् । वा॒रु॒णम् । च॒रुम् । द॒धि॒क्राव्‌ण्ण॒ इति॑ दधि - क्राव्‌ण्णे᳚ । च॒रुम् । अ॒भि॒श॒स्यमा॑न॒ इत्य॑भि - श॒स्यमा॑नः । यत् । वै॒श्वा॒न॒रः । द्वाद॑शकपाल॒ इति॒ द्वाद॑श - क॒पा॒लः॒ । भव॑ति । सं॒ॅव॒थ्स॒र इति॑ सं - व॒थ्स॒रः । वै । अ॒ग्निः । वै॒श्वा॒न॒रः । सं॒ॅव॒थ्स॒रेणेति॑ सं - व॒थ्स॒रेण॑ । ए॒व । ए॒न॒म् । स्व॒द॒य॒ति॒ । अपेति॑ । पा॒पम् । वर्ण᳚म् । ह॒ते॒ । वा॒रु॒णेन॑ । ए॒व । ए॒न॒म् । व॒रु॒ण॒पा॒शादिति॑ वरुण - पा॒शात् । मु॒ञ्च॒ति॒ । द॒धि॒क्राव्‌ण्णेति॑ दधि -क्राव्‌ण्णा᳚ । पु॒ना॒ति॒ । हिर॑ण्यम् । दक्षि॑णा । प॒वित्र᳚म् । वै । हिर॑ण्यम् । पु॒नाति॑ । ए॒व । ए॒न॒म् । आ॒द्य᳚म् । अ॒स्य॒ । अन्न᳚म् । भ॒व॒ति॒ । ए॒ताम् । ए॒व । निरिति॑ । व॒पे॒त् । प्र॒जाका॑म॒ इति॑ प्र॒जा - का॒मः॒ । सं॒ॅव॒थ्स॒र इति॑ सं - व॒थ्स॒रः । \textbf{  23} \newline
                  \newline
                                \textbf{ TS 2.2.5.2} \newline
                  वै । ए॒तस्य॑ । अशा᳚न्तः । योनि᳚म् । प्र॒जाया॒ इति॑ प्र - जायै᳚ । प॒शू॒नाम् । निरिति॑ । द॒ह॒ति॒ । यः । अल᳚म् । प्र॒जाया॒ इति॑ प्र-जायै᳚ । सन्न् । प्र॒जामिति॑ प्र - जाम् । न । वि॒न्दते᳚ । यत् । वै॒श्वा॒न॒रः । द्वाद॑शकपाल॒ इति॒ द्वाद॑श - क॒पा॒लः॒ । भव॑ति । सं॒ॅव॒थ्स॒र इति॑ सं - व॒थ्स॒रः । वै । अ॒ग्निः । वै॒श्वा॒न॒रः । सं॒ॅव॒थ्स॒रमिति॑ सं - व॒थ्स॒रम् । ए॒व । भा॒ग॒धेये॒नेति॑ भाग - धेये॑न । श॒म॒य॒ति॒ । सः । अ॒स्मै॒ । शा॒न्तः । स्वात् । योनेः᳚ । प्र॒जामिति॑ प्र - जाम् । प्रेति॑ । ज॒न॒य॒ति॒ । वा॒रु॒णेन॑ । ए॒व । ए॒न॒म् । व॒रु॒ण॒पा॒शादिति॑ वरुण - पा॒शात् । मु॒ञ्च॒ति॒ । द॒धि॒क्राव्‌ण्णेति॑ दधि - क्राव्‌ण्णा᳚ । पु॒ना॒ति॒ । हिर॑ण्यम् । दक्षि॑णा । प॒वित्र᳚म् । वै । हिर॑ण्यम् । पु॒नाति॑ । ए॒व । ए॒न॒म् । \textbf{  24} \newline
                  \newline
                                \textbf{ TS 2.2.5.3} \newline
                  वि॒न्दते᳚ । प्र॒जामिति॑ प्र - जाम् । वै॒श्वा॒न॒रम् । द्वाद॑शकपाल॒मिति॒ द्वाद॑श - क॒पा॒ल॒म् । निरिति॑ । व॒पे॒त् । पु॒त्रे । जा॒ते । यत् । अ॒ष्टाक॑पाल॒ इत्य॒ष्टा - क॒पा॒लः॒ । भव॑ति । गा॒य॒त्रि॒या । ए॒व । ए॒न॒म् । ब्र॒ह्म॒व॒र्च॒सेनेति॑ ब्रह्म - व॒र्च॒सेन॑ । पु॒ना॒ति॒ । यत् । नव॑कपाल॒ इति॒ नव॑ - क॒पा॒लः॒ । त्रि॒वृतेति॑ त्रि - वृता᳚ । ए॒व । अ॒स्मि॒न्न् । तेजः॑ । द॒धा॒ति॒ । यत् । दश॑कपाल॒ इति॒ दश॑ - क॒पा॒लः॒ । वि॒राजेति॑ वि - राजा᳚ । ए॒व । अ॒स्मि॒न्न् । अ॒न्नाद्य॒मित्य॑न्न - अद्य᳚म् । द॒धा॒ति॒ । यत् । एका॑दशकपाल॒ इत्येका॑दश - क॒पा॒लः॒ । त्रि॒ष्टुभा᳚ । ए॒व । अ॒स्मि॒न्न् । इ॒न्द्रि॒यम् । द॒धा॒ति॒ । यत् । द्वाद॑शकपाल॒ इति॒ द्वाद॑श - क॒पा॒लः॒ । जग॑त्या । ए॒व । अ॒स्मि॒न्न् । प॒शून् । द॒धा॒ति॒ । यस्मिन्न्॑ । जा॒ते । ए॒ताम् । इष्टि᳚म् । नि॒र्वप॒तीति॑ निः - वप॑ति । पू॒तः । \textbf{  25} \newline
                  \newline
                                \textbf{ TS 2.2.5.4} \newline
                  ए॒व । ते॒ज॒स्वी । अ॒न्ना॒द इत्य॑न्न - अ॒दः । इ॒न्द्रि॒या॒वी । प॒शु॒मानिति॑ पशु - मान् । भ॒व॒ति॒ । अवेति॑ । वै । ए॒षः । सु॒व॒र्गादिति॑ सुवः - गात् । लो॒कात् । छि॒द्य॒ते॒ । यः । द॒र्॒.श॒पू॒र्ण॒मा॒स॒या॒जीति॑ दर्.शपूर्णमास - या॒जी । सन्न् । अ॒मा॒वा॒स्या॑मित्य॑मा - वा॒स्या᳚म् । वा॒ । पौ॒र्ण॒मा॒सीमिति॑ पौर्ण - मा॒सीम् । वा॒ । अ॒ति॒पा॒दय॒तीत्य॑ति - पा॒दय॑ति । सु॒व॒र्गायेति॑ सुवः - गाय॑ । हि । लो॒काय॑ । द॒र्॒.श॒पू॒र्ण॒मा॒साविति॑ दर्.श - पू॒र्ण॒मा॒सौ । इ॒ज्येते॒ इति॑ । वै॒श्वा॒न॒रम् । द्वाद॑शकपाल॒मिति॒ द्वाद॑श - क॒पा॒ल॒म् । निरिति॑ । व॒पे॒त् । अ॒मा॒वा॒स्या॑मित्य॑मा - वा॒स्या᳚म् । वा॒ । पौ॒र्ण॒मा॒सीमिति॑ पौर्ण - मा॒सीम् । वा॒ । अ॒ति॒पाद्येत्य॑ति - पाद्य॑ । सं॒ॅव॒थ्स॒र इति॑ सं - व॒थ्स॒रः । वै । अ॒ग्निः । वै॒श्वा॒न॒रः । सं॒ॅव॒थ्स॒रमिति॑ सं - व॒थ्स॒रम् । ए॒व । प्री॒णा॒ति॒ । अथो॒ इति॑ । सं॒ॅव॒थ्स॒रमिति॑ सं - व॒थ्स॒रम् । ए॒व । अ॒स्मै॒ । उपेति॑ । द॒धा॒ति॒ । सु॒व॒र्गस्येति॑ सुवः - गस्य॑ । लो॒कस्य॑ । सम॑ष्ट्या॒ इति॒ सं - अ॒ष्ट्यै॒ । \textbf{  26} \newline
                  \newline
                                \textbf{ TS 2.2.5.5} \newline
                  अथो॒ इति॑ । दे॒वताः᳚ । ए॒व । अ॒न्वा॒रभ्येत्य॑नु - आ॒रभ्य॑ । सु॒व॒र्गमिति॑ सुवः - गम् । लो॒कम् । ए॒ति॒ । वी॒र॒हेति॑ वीर-हा । वै । ए॒षः । दे॒वाना᳚म् । यः । अ॒ग्निम् । उ॒द्वा॒सय॑त॒ इत्यु॑त् - वा॒सय॑ते । न । वै । ए॒तस्य॑ । ब्रा॒ह्म॒णाः । ऋ॒ता॒यव॒ इत्यृ॑त - यवः॑ । पु॒रा । अन्न᳚म् । अ॒क्ष॒न्न् । आ॒ग्ने॒यम् । अ॒ष्टाक॑पाल॒मित्य॒ष्टा - क॒पा॒ल॒म् । निरिति॑ । व॒पे॒त् । वै॒श्वा॒न॒रम् । द्वाद॑शकपाल॒मिति॒ द्वाद॑श-क॒पा॒ल॒म् । अ॒ग्निम् । उ॒द्वा॒स॒यि॒ष्यन्नित्यु॑त् - वा॒स॒यि॒ष्यन्न् । यत् । अ॒ष्टाक॑पाल॒ इत्य॒ष्टा - क॒पा॒लः॒ । भव॑ति । अ॒ष्टाक्ष॒रेत्य॒ष्टा - अ॒क्ष॒रा॒ । गा॒य॒त्री । गा॒य॒त्रः । अ॒ग्निः । यावान्॑ । ए॒व । अ॒ग्निः । तस्मै᳚ । आ॒ति॒थ्यम् । क॒रो॒ति॒ । अथो॒ इति॑ । यथा᳚ । जन᳚म् । य॒ते । अ॒व॒सम् । क॒रोति॑ । ता॒दृक् । \textbf{  27} \newline
                  \newline
                                \textbf{ TS 2.2.5.6} \newline
                  ए॒व । तत् । द्वाद॑शकपाल॒ इति॒ द्वाद॑श - क॒पा॒लः॒ । वै॒श्वा॒न॒रः । भ॒व॒ति॒ । द्वाद॑श । मासाः᳚ । सं॒ॅव॒थ्स॒र इति॑ सं - व॒थ्स॒रः । सं॒ॅव॒थ्स॒र इति॑ सं - व॒थ्स॒रः । खलु॑ । वै । अ॒ग्नेः । योनिः॑ । स्वाम् । ए॒व । ए॒न॒म् । योनि᳚म् । ग॒म॒य॒ति॒ । आ॒द्य᳚म् । अ॒स्य॒ । अन्न᳚म् । भ॒व॒ति॒ । वै॒श्वा॒न॒रम् । द्वाद॑शकपाल॒मिति॒ द्वाद॑श - क॒पा॒ल॒म् । निरिति॑ । व॒पे॒त् । मा॒रु॒तम् । स॒प्तक॑पाल॒मिति॑ स॒प्त - क॒पा॒ल॒म् । ग्राम॑काम॒ इति॒ ग्राम॑ - का॒मः॒ । आ॒ह॒व॒नीय॒ इत्या᳚ -ह॒व॒नीये᳚ । वै॒श्वा॒न॒रम् । अधीति॑ । श्र॒य॒ति॒ । गार्.ह॑पत्य॒ इति॒ गार्.ह॑ - प॒त्ये॒ । मा॒रु॒तम् । पा॒प॒व॒स्य॒सस्येति॑ पाप - व॒स्य॒सस्य॑ । विधृ॑त्या॒ इति॒ वि - धृ॒त्यै॒ । द्वाद॑शकपाल॒ इति॒ द्वाद॑श - क॒पा॒लः॒ । वै॒श्वा॒न॒रः । भ॒व॒ति॒ । द्वाद॑श । मासाः᳚ । सं॒ॅव॒थ्स॒र इति॑ सं-व॒थ्स॒रः । सं॒ॅव॒थ्स॒रेणेति॑ सं - व॒थ्स॒रेण॑ । ए॒व । अ॒स्मै॒ । स॒जा॒तानिति॑ स - जा॒तान् । च्या॒व॒य॒ति॒ । मा॒रु॒तः । भ॒व॒ति॒ । \textbf{  28} \newline
                  \newline
                                \textbf{ TS 2.2.5.7} \newline
                  म॒रुतः॑ । वै । दे॒वाना᳚म् । विशः॑ । दे॒व॒वि॒शेनेति॑ देव - वि॒शेन॑ । ए॒व । अ॒स्मै॒ । म॒नु॒ष्य॒वि॒शमिति॑ मनुष्य - वि॒शम् । अवेति॑ । रु॒न्धे॒ । स॒प्तक॑पाल॒ इति॑ स॒प्त - क॒पा॒लः॒ । भ॒व॒ति॒ । स॒प्तग॑णा॒ इति॑ स॒प्त - ग॒णाः॒ । वै । म॒रुतः॑ । ग॒ण॒श इति॑ गण-शः । ए॒व । अ॒स्मै॒ । स॒जा॒तानिति॑ स - जा॒तान् । अवेति॑ । रु॒न्धे॒ । अ॒नू॒च्यमा॑न॒ इत्य॑नु - उ॒च्यमा॑ने । एति॑ । सा॒द॒य॒ति॒ । विश᳚म् । ए॒व । अ॒स्मै॒ । अनु॑वर्त्मान॒मित्यनु॑ - व॒र्त्मा॒न॒म् । क॒रो॒ति॒ ॥ \textbf{  29} \newline
                  \newline
                      (प्र॒जाका॑मः संॅवथ्स॒रः - पु॒नात्ये॒वैनं॑ - पू॒तः - सम॑ष्ट्यै -ता॒दृङ् - मा॑रु॒तो भ॑व॒ - त्येका॒न्न त्रिꣳ॒॒शच्च॑ )  \textbf{(A5)} \newline \newline
                                \textbf{ TS 2.2.6.1} \newline
                  आ॒दि॒त्यम् । च॒रुम् । निरिति॑ । व॒पे॒त् । स॒ग्रां॒ममिति॑ सं - ग्रा॒मम् । उ॒प॒प्र॒या॒स्यन्नित्यु॑प - प्र॒या॒स्यन्न् । इ॒यम् । वै । अदि॑तिः । अ॒स्याम् । ए॒व । पूर्वे᳚ । प्रतीति॑ । ति॒ष्ठ॒न्ति॒ । वै॒श्वा॒न॒रम् । द्वाद॑शकपाल॒मिति॒ द्वाद॑श - क॒पा॒ल॒म् । निरिति॑ । व॒पे॒त् । आ॒यत॑न॒मित्या᳚ - यत॑नम् । ग॒त्वा । सं॒ॅव॒थ्स॒र इति॑ सं - व॒थ्स॒रः । वै । अ॒ग्निः । वै॒श्वा॒न॒रः । सं॒ॅव॒थ्स॒र इति॑ सं-व॒थ्स॒रः । खलु॑ । वै । दे॒वाना᳚म् । आ॒यत॑न॒मित्या᳚ - यत॑नम् । ए॒तस्मा᳚त् । वै । आ॒यत॑ना॒दित्या᳚-यत॑नात् । दे॒वाः । असु॑रान् । अ॒ज॒य॒न्न् । यत् । वै॒श्वा॒न॒रम् । द्वाद॑शकपाल॒मिति॒ द्वाद॑श - क॒पा॒ल॒म् । नि॒र्वप॒तीति॑ निः - वप॑ति । दे॒वाना᳚म् । ए॒व । आ॒यत॑न॒ इत्या᳚ - यत॑ने । य॒त॒ते॒ । जय॑ति । तम् । स॒ग्रां॒ममिति॑ सं - ग्रा॒मम् । ए॒तस्मिन्न्॑ । वै । ए॒तौ । मृ॒जा॒ते॒ इति॑ । \textbf{  30} \newline
                  \newline
                                \textbf{ TS 2.2.6.2} \newline
                  यः । वि॒द्वि॒षा॒णयो॒रिति॑ वि-द्वि॒षा॒णयोः᳚ । अन्न᳚म् । अत्ति॑ । वै॒श्वा॒न॒रम् । द्वाद॑शकपाल॒मिति॒ द्वाद॑श - क॒पा॒ल॒म् । निरिति॑ । व॒पे॒त् । वि॒द्वि॒षा॒णयो॒रिति॑ वि - द्वि॒षा॒णयोः᳚ । अन्न᳚म् । ज॒ग्ध्वा । सं॒ॅव॒थ्स॒र इति॑ सं - व॒थ्स॒रः । वै । अ॒ग्निः । वै॒श्वा॒न॒रः । सं॒ॅव॒थ्स॒रस्व॑दित॒मिति॑ संॅवथ्स॒र - स्व॒दि॒त॒म् । ए॒व । अ॒त्ति॒ । न । अ॒स्मि॒न्न् । मृ॒जा॒ते॒ इति॑ । सं॒ॅव॒थ्स॒रायेति॑ सं - व॒थ्स॒राय॑ । वै । ए॒तौ । समिति॑ । अ॒मा॒ते॒ इति॑ । यौ । स॒म॒माते॒ इति॑ सं-अ॒माते᳚ । तयोः᳚ । यः । पूर्वः॑ । अ॒भि॒द्रुह्य॒तीत्य॑भि - द्रुह्य॑ति । तम् । वरु॑णः । गृ॒ह्णा॒ति॒ । वै॒श्वा॒न॒रम् । द्वाद॑शकपाल॒मिति॒ द्वाद॑श - क॒पा॒ल॒म् । निरिति॑ । व॒पे॒त् । स॒म॒मा॒नयो॒रिति॑ सं - अ॒मा॒नयोः᳚ । पूर्वः॑ । अ॒भि॒द्रुह्येत्य॑भि - द्रुह्य॑ । सं॒ॅव॒थ्स॒र इति॑ सं - व॒थ्स॒रः । वै । अ॒ग्निः । वै॒श्वा॒न॒रः । सं॒ॅव॒थ्स॒रमिति॑ सं - व॒थ्स॒रम् । ए॒व । आ॒प्त्वा । नि॒र्व॒रु॒णमिति॑ निः - व॒रु॒णम् । \textbf{  31} \newline
                  \newline
                                \textbf{ TS 2.2.6.3} \newline
                  प॒रस्ता᳚त् । अ॒भीति॑ । द्रु॒ह्य॒ति॒ । न । ए॒न॒म् । वरु॑णः । गृ॒ह्णा॒ति॒ । आ॒व्य᳚म् । वै । ए॒षः । प्रतीति॑ । गृ॒ह्णा॒ति॒ । यः । अवि᳚म् । प्र॒ति॒गृ॒ह्णातीति॑ प्रति - गृ॒ह्णाति॑ । वै॒श्वा॒न॒रम् । द्वाद॑शकपाल॒मिति॒ द्वाद॑श - क॒पा॒ल॒म् । निरिति॑ । व॒पे॒त् । अवि᳚म् । प्र॒ति॒गृह्येति॑ प्रति - गृह्य॑ । सं॒ॅव॒थ्स॒र इति॑ सं - व॒थ्स॒रः । वै । अ॒ग्निः । वै॒श्वा॒न॒रः । सं॒ॅव॒थ्स॒रस्व॑दिता॒मिति॑ संॅवथ्स॒र - स्व॒दि॒ता॒म् । ए॒व । प्रतीति॑ । गृ॒ह्णा॒ति॒ । न । आ॒व्य᳚म् । प्रतीति॑ । गृ॒ह्णा॒ति॒ । आ॒त्मनः॑ । वै । ए॒षः । मात्रा᳚म् । आ॒प्नो॒ति॒ । यः । उ॒भ॒याद॑त् । प्र॒ति॒गृ॒ह्णातीति॑ प्रति - गृ॒ह्णाति॑ । अश्व᳚म् । वा॒ । पुरु॑षम् । वा॒ । वै॒श्वा॒न॒रम् । द्वाद॑शकपाल॒मिति॒ द्वाद॑श - क॒पा॒ल॒म् । निरिति॑ । व॒पे॒त् । उ॒भ॒याद॑त् । \textbf{  32} \newline
                  \newline
                                \textbf{ TS 2.2.6.4} \newline
                  प्र॒ति॒गृह्येति॑ प्रति - गृह्य॑ । सं॒ॅव॒थ्स॒र इति॑ सं - व॒थ्स॒रः । वै । अ॒ग्निः । वै॒श्वा॒न॒रः । सं॒ॅव॒थ्स॒रस्व॑दित॒मिति॑ संॅवथ्स॒र - स्व॒दि॒त॒म् । ए॒व । प्रतीति॑ । गृ॒ह्णा॒ति॒ । न । आ॒त्मनः॑ । मात्रा᳚म् । आ॒प्नो॒ति॒ । वै॒श्वा॒न॒रम् । द्वाद॑शकपाल॒मिति॒ द्वाद॑श - क॒पा॒ल॒म् । निरिति॑ । व॒पे॒त् । स॒निम् । ए॒ष्यन्न् । सं॒ॅव॒थ्स॒र इति॑ सं - व॒थ्स॒रः । वै । अ॒ग्निः । वै॒श्वा॒न॒रः । य॒दा । खलु॑ । वै । सं॒ॅव॒थ्स॒रमिति॑ सं - व॒थ्स॒रम् । ज॒नता॑याम् । चर॑ति । अथ॑ । सः । ध॒ना॒र्घ इति॑ धन - अ॒र्घः । भ॒व॒ति॒॒ । यत् । वै॒श्वा॒न॒रम् । द्वाद॑शकपाल॒मिति॒ द्वाद॑श - क॒पा॒ल॒म् । नि॒र्वप॒तीति॑ निः - वप॑ति । सं॒ॅव॒थ्स॒रसा॑ता॒मिति॑ संॅवथ्स॒र-सा॒ता॒म् । ए॒व । स॒निम् । अ॒भि । प्रेति॑ । च्य॒व॒ते॒ । दान॑कामा॒ इति॒ दान॑ - का॒माः॒ । अ॒स्मै॒ । प्र॒जा इति॑ प्र - जाः । भ॒व॒न्ति॒ । यः । वै । सं॒ॅव॒थ्स॒रमिति॑ सं - व॒थ्स॒रम् । \textbf{  33} \newline
                  \newline
                                \textbf{ TS 2.2.6.5} \newline
                  प्र॒युज्येति॑ प्र - युज्य॑ । न । वि॒मु॒ञ्चतीति॑ वि - मु॒ञ्चति॑ । अ॒प्र॒ति॒ष्ठा॒न इत्य॑प्रति - स्था॒नः । वै । सः । भ॒व॒ति॒ । ए॒तम् । ए॒व । वै॒श्वा॒न॒रम् । पुनः॑ । आ॒गत्येत्या᳚ - गत्य॑ । निरिति॑ । व॒पे॒त् । यम् । ए॒व । प्र॒यु॒ङ्क्त इति॑ प्र - यु॒ङ्क्ते । तम् । भा॒ग॒धेये॒नेति॑ भाग - धेये॑न । वीति॑ । मु॒ञ्च॒ति॒ । प्रति॑ष्ठित्या॒ इति॒ प्रति॑ - स्थि॒त्यै॒ । यया᳚ । रज्वा᳚ । उ॒त्त॒मामित्यु॑त् - त॒माम् । गाम् । आ॒जेदित्या᳚ - अ॒जेत् । ताम् । भ्रातृ॑व्याय । प्रेति॑ । हि॒णु॒या॒त् । निर्.ऋ॑ति॒मिति॒ निः - ऋ॒ति॒म् । ए॒व । अ॒स्मै॒ । प्रेति॑ । हि॒णो॒ति॒ ॥ \textbf{  34 } \newline
                  \newline
                      (मृ॒जा॒ते॒ - नि॒र्व॒रु॒णं - ॅव॑पेदुभ॒याद॒द् - यो वै सं॑ॅवथ्स॒रꣳ - षट्त्रिꣳ॑शच्च)  \textbf{(A6)} \newline \newline
                                \textbf{ TS 2.2.7.1} \newline
                  ऐ॒न्द्रम् । च॒रुम् । निरिति॑ । व॒पे॒त् । प॒शुका॑म॒ इति॑ प॒शु - का॒मः॒ । ऐ॒न्द्राः । वै । प॒शवः॑ । इन्द्र᳚म् । ए॒व । स्वेन॑ । भा॒ग॒धेये॒नेति॑ भाग - धेये॑न । उपेति॑ । धा॒व॒ति॒ । सः । ए॒व । अ॒स्मै॒ । प॒शून् । प्रेति॑ । य॒च्छ॒ति॒ । प॒शु॒मानिति॑ पशु - मान् । ए॒व । भ॒व॒ति॒ । च॒रुः । भ॒व॒ति॒ । स्वात् । ए॒व । अ॒स्मै॒ । योनेः᳚ । प॒शून् । प्रेति॑ । ज॒न॒य॒ति॒ । इन्द्रा॑य । इ॒न्द्रि॒याव॑त॒ इती᳚न्द्रि॒य - व॒ते॒ । पु॒रो॒डाश᳚म् । एका॑दशकपाल॒मित्येका॑दश - क॒पा॒ल॒म् । निरिति॑ । व॒पे॒त् । प॒शुका॑म॒ इति॑ प॒शु - का॒मः॒ । इ॒न्द्रि॒यम् । वै । प॒शवः॑ । इन्द्र᳚म् । ए॒व । इ॒न्द्रि॒याव॑न्त॒मिती᳚न्द्रि॒य - व॒न्त॒म् । स्वेन॑ । भा॒ग॒धेये॒नेति॑ भाग - धेये॑न । उपेति॑ । धा॒व॒ति॒ । सः । \textbf{  35} \newline
                  \newline
                                \textbf{ TS 2.2.7.2} \newline
                  ए॒व । अ॒स्मै॒ । इ॒न्द्रि॒यम् । प॒शून् । प्रेति॑ । य॒च्छ॒ति॒ । प॒शु॒मानिति॑ पशु - मान् । ए॒व । भ॒व॒ति॒ । इन्द्रा॑य । घ॒र्मव॑त॒ इति॑ घ॒र्म - व॒ते॒ । पु॒रो॒डाश᳚म् । एका॑दशकपाल॒मित्येका॑दश - क॒पा॒ल॒म् । निरिति॑ । व॒पे॒त् । ब्र॒ह्म॒व॒र्च॒सका॑म॒ इति॑ ब्रह्मवर्च॒स - का॒मः॒ । ब्र॒ह्म॒व॒र्च॒समिति॑ ब्रह्म - व॒र्च॒सम् । वै । घ॒र्मः । इन्द्र᳚म् । ए॒व । घ॒र्मव॑न्त॒मिति॑ घ॒र्म - व॒न्त॒म् । स्वेन॑ । भा॒ग॒धेये॒नेति॑ भाग - धेये॑न । उपेति॑ । धा॒व॒ति॒ । सः । ए॒व । अ॒स्मि॒न्न् । ब्र॒ह्म॒व॒र्च॒समिति॑ ब्रह्म - व॒र्च॒सम् । द॒धा॒ति॒ । ब्र॒ह्म॒व॒र्च॒सीति॑ ब्रह्म - व॒र्च॒सी । ए॒व । भ॒व॒ति॒ । इन्द्रा॑य । अ॒र्कव॑त॒ इत्य॒र्क - व॒ते॒ । पु॒रो॒डाश᳚म् । एका॑दशकपाल॒मित्येका॑दश - क॒पा॒ल॒म् । निरिति॑ । व॒पे॒त् । अन्न॑काम॒ इत्यन्न॑ - का॒मः॒ । अ॒र्कः । वै । दे॒वाना᳚म् । अन्न᳚म् । इन्द्र᳚म् । ए॒व । अ॒र्कव॑न्त॒मित्य॒र्क - व॒न्त॒म् । स्वेन॑ । भा॒ग॒धेये॒नेति॑ भाग - धेये॑न । \textbf{  36} \newline
                  \newline
                                \textbf{ TS 2.2.7.3} \newline
                  उपेति॑ । धा॒व॒ति॒ । सः । ए॒व । अ॒स्मै॒ । अन्न᳚म् । प्रेति॑ । य॒च्छ॒ति॒ । अ॒न्ना॒द इत्य॑न्न - अ॒दः । ए॒व । भ॒व॒ति॒ । इन्द्रा॑य । घ॒र्मव॑त॒ इति॑ घ॒र्म - व॒ते॒ । पु॒रो॒डाश᳚म् । एका॑दशकपाल॒मित्येका॑दश - क॒पा॒ल॒म् । निरिति॑ । व॒पे॒त् । इन्द्रा॑य । इ॒न्द्रि॒याव॑त॒ इती᳚न्द्रि॒य - व॒ते॒ । इन्द्रा॑य । अ॒र्कव॑त॒ इत्य॒र्क - व॒ते॒ । भूति॑काम॒ इति॒ भूति॑ - का॒मः॒ । यत् । इन्द्रा॑य । घ॒र्मव॑त॒ इति॑ घ॒र्म - व॒ते॒ । नि॒र्वप॒तीति॑ निः-वप॑ति । शिरः॑ । ए॒व । अ॒स्य॒ । तेन॑ । क॒रो॒ति॒ । यत् । इन्द्रा॑य । इ॒न्द्रि॒याव॑त॒ इती᳚न्द्रि॒य - व॒ते॒ । आ॒त्मान᳚म् । ए॒व । अ॒स्य॒ । तेन॑ । क॒रो॒ति॒ । यत् । इन्द्रा॑य । अ॒र्कव॑त॒ इत्य॒र्क - व॒ते॒ । भू॒तः । ए॒व । अ॒न्नाद्य॒ इत्य॑न्न - अद्ये᳚ । प्रतीति॑ । ति॒ष्ठ॒ति॒ । भव॑ति । ए॒व । इन्द्रा॑य । \textbf{  37} \newline
                  \newline
                                \textbf{ TS 2.2.7.4} \newline
                  अꣳ॒॒हो॒मुच॒ इत्यꣳ॑हः - मुचे᳚ । पु॒रो॒डाश᳚म् । एका॑दशकपाल॒मित्येका॑दश - क॒पा॒ल॒म् । निरिति॑ । व॒पे॒त् । यः । पा॒प्मना᳚ । गृ॒ही॒तः । स्यात् । पा॒प्मा । वै । अꣳहः॑ । इन्द्र᳚म् । ए॒व । अꣳ॒॒हो॒मुच॒मित्यꣳ॑हः - मुच᳚म् । स्वेन॑ । भा॒ग॒धेये॒नेति॑ भाग-धेये॑न । उपेति॑ । धा॒व॒ति॒ । सः । ए॒व । ए॒न॒म् । पा॒प्मनः॑ । अꣳह॑सः । मु॒ञ्च॒ति॒ । इन्द्रा॑य । वै॒मृ॒धाय॑ । पु॒रो॒डाश᳚म् । एका॑दशकपाल॒मित्येका॑दश - क॒पा॒ल॒म् । निरिति॑ । व॒पे॒त् । यम् । मृधः॑ । अ॒भीति॑ । प्र॒वेपे॑र॒न्निति॑ प्र - वेपे॑रन्न् । रा॒ष्ट्राणि॑ । वा॒ । अ॒भीति॑ । स॒मि॒युरिति॑ सं - इ॒युः । इन्द्र᳚म् । ए॒व । वै॒मृ॒धम् । स्वेन॑ । भा॒ग॒धेये॒नेति॑ भाग - धेये॑न । उपेति॑ । धा॒व॒ति॒ । सः । ए॒व । अ॒स्मा॒त् । मृधः॑ । \textbf{  38} \newline
                  \newline
                                \textbf{ TS 2.2.7.5} \newline
                  अपेति॑ । ह॒न्ति॒ । इन्द्रा॑य । त्रा॒त्रे । पु॒रो॒डाश᳚म् । एका॑दशकपाल॒मित्येका॑दश - क॒पा॒ल॒म् । निरिति॑ । व॒पे॒त् । ब॒द्धः । वा॒ । परि॑यत्त॒ इति॒ परि॑ - य॒त्तः॒ । वा॒ । इन्द्र᳚म् । ए॒व । त्रा॒तार᳚म् । स्वेन॑ । भा॒ग॒धेये॒नेति॑ भाग - धेये॑न । उपेति॑ । धा॒व॒ति॒ । सः । ए॒व । ए॒न॒म् । त्रा॒य॒ते॒ । इन्द्रा॑य । अ॒र्का॒श्व॒मे॒धव॑त॒ इत्य॑र्काश्वमे॒ध-व॒ते॒ । पु॒रो॒डाश᳚म् । एका॑दशकपाल॒मित्येका॑दश - क॒पा॒ल॒म् । निरिति॑ । व॒पे॒त् । यम् । म॒हा॒य॒ज्ञ् इति॑ महा - य॒ज्ञ्ः । न । उ॒प॒नमे॒दित्यु॑प - नमे᳚त् । ए॒ते इति॑ । वै । म॒हा॒य॒ज्ञ्स्येति॑ महा - य॒ज्ञ्स्य॑ । अन्त्ये॒ इति॑ । त॒नू इति॑ । यत् । अ॒र्का॒श्व॒मे॒धावित्य॑र्क - अ॒श्व॒मे॒धौ । इन्द्र᳚म् । ए॒व । अ॒र्का॒श्व॒मे॒धव॑न्त॒मित्य॑र्काश्वमे॒ध - व॒न्त॒म् । स्वेन॑ । भा॒ग॒धेये॒नेति॑ भाग - धेये॑न । उपेति॑ । धा॒व॒ति॒ । सः । ए॒व । अ॒स्मै॒ ( ) । अ॒न्त॒तः । म॒हा॒य॒ज्ञ्मिति॑ महा - य॒ज्ञ्म् । च्या॒व॒य॒ति॒ । उपेति॑ । ए॒न॒म् । म॒हा॒य॒ज्ञ् इति॑ महा - य॒ज्ञ्ः । न॒म॒ति॒ ॥ \textbf{  39} \newline
                  \newline
                      (इ॒न्द्रि॒याव॑न्तꣳ॒॒ स्वेन॑ भाग॒धेये॒नोप॑ धावति॒ सो᳚ - ऽर्कव॑न्तꣳ॒॒ स्वेन॑ भाग॒धेये॑नै॒ - वेन्द्रा॑या - स्मा॒न् मृधो᳚ - ऽस्मै - स॒प्त च॑ )  \textbf{(A7)} \newline \newline
                                \textbf{ TS 2.2.8.1} \newline
                  इ॒न्द्रा॑य । अन्वृ॑जव॒ इत्यनु॑ - ऋ॒ज॒वे॒ । पु॒रो॒डाश᳚म् । एका॑दशकपाल॒मित्येका॑दश - क॒पा॒ल॒म् । निरिति॑ । व॒पे॒त् । ग्राम॑काम॒ इति॒ ग्राम॑ - का॒मः॒ । इन्द्र᳚म् । ए॒व । अन्वृ॑जु॒मित्यनु॑-ऋ॒जु॒म् । स्वेन॑ । भा॒ग॒धेये॒नेति॑ भाग-धेये॑न । उपेति॑ । धा॒व॒ति॒ । सः । ए॒व । अ॒स्मै॒ । स॒जा॒तानिति॑ स - जा॒तान् । अनु॑का॒नित्यनु॑ - का॒न् । क॒रो॒ति॒ । ग्रा॒मी । ए॒व । भ॒व॒ति॒ । इ॒न्द्रा॒ण्यै । च॒रुम् । निरिति॑ । व॒पे॒त् । यस्य॑ । सेना᳚ । असꣳ॑शि॒तेत्यसं᳚ - शि॒ता॒ । इ॒व॒ । स्यात् । इ॒न्द्रा॒णी । वै । सेना॑यै । दे॒वता᳚ । इ॒न्द्रा॒णीम् । ए॒व । स्वेन॑ । भा॒ग॒धेये॒नेति॑ भाग - धेये॑न । उपेति॑ । धा॒व॒ति॒ । सा । ए॒व । अ॒स्य॒ । सेना᳚म् । समिति॑ । श्य॒ति॒ । बल्ब॑जान् । अपीति॑ । \textbf{  40} \newline
                  \newline
                                \textbf{ TS 2.2.8.2} \newline
                  इ॒द्ध्मे । समिति॑ । न॒ह्ये॒त् । गौः । यत्र॑ । अधि॑ष्क॒न्नेत्यधि॑ - स्क॒न्ना॒ । न्यमे॑ह॒दिति॑ नि - अमे॑हत् । ततः॑ । बल्ब॑जाः । उदिति॑ । अ॒ति॒ष्ठ॒न्न् । गवा᳚म् । ए॒व । ए॒न॒म् । न्या॒यमिति॑ नि - आ॒यम् । अ॒पि॒नीयेत्य॑पि -नीय॑ । गाः । वे॒द॒य॒ति॒ । इन्द्रा॑य । म॒न्यु॒मत॒ इति॑ मन्यु - मते᳚ । मन॑स्वते । पु॒रो॒डाश᳚म् । एका॑दशकपाल॒मित्येका॑दश - क॒पा॒ल॒म् । निरिति॑ । व॒पे॒त् । स॒ग्रां॒म इति॑ सं - ग्रा॒मे । संॅय॑त्त॒ इति॒ सं - य॒त्ते॒ । इ॒न्द्रि॒येण॑ । वै । म॒न्युना᳚ । मन॑सा । स॒ग्रां॒ममिति॑ सं - ग्रा॒मम् । ज॒य॒ति॒ । इन्द्र᳚म् । ए॒व । म॒न्यु॒मन्त॒मिति॑ मन्यु - मन्त᳚म् । मन॑स्वन्तम् । स्वेन॑ । भा॒ग॒धेये॒नेति॑ भाग-धेये॑न । उपेति॑ । धा॒व॒ति॒ । सः । ए॒व । अ॒स्मि॒न्न् । इ॒न्द्रि॒यम् । म॒न्युम् । मनः॑ । द॒धा॒ति॒ । जय॑ति । तम् । \textbf{  41} \newline
                  \newline
                                \textbf{ TS 2.2.8.3} \newline
                  स॒ग्रां॒ममिति॑ सं -ग्रा॒मम् । ए॒ताम् । ए॒व । निरिति॑ । व॒पे॒त् । यः । ह॒तम॑ना॒ इति॑ ह॒त - म॒नाः॒ । स्व॒यंपा॑प॒ इति॑ स्व॒यं - पा॒पः॒ । इ॒व॒ । स्यात् । ए॒तानि॑ । हि । वै । ए॒तस्मा᳚त् । अप॑क्रान्ता॒नीत्यप॑ - क्रा॒न्ता॒नि॒ । अथ॑ । ए॒षः । ह॒तम॑ना॒ इति॑ ह॒त - म॒नाः॒ । स्व॒यंपा॑प॒ इति॑ स्व॒यं - पा॒पः॒ । इन्द्र᳚म् । ए॒व । म॒न्यु॒मन्त॒मिति॑ मन्यु - मन्त᳚म् । मन॑स्वन्तम् । स्वेन॑ । भा॒ग॒धेये॒नेति॑ भाग-धेये॑न । उपेति॑ । धा॒व॒ति॒ । सः । ए॒व । अ॒स्मि॒न्न् । इ॒न्द्रि॒यम् । म॒न्युम् । मनः॑ । द॒धा॒ति॒ । न । ह॒तम॑ना॒ इति॑ ह॒त-म॒नाः॒ । स्व॒यंपा॑प॒ इति॑ स्व॒यं-पा॒पः॒ । भ॒व॒ति॒ । इन्द्रा॑य । दा॒त्रे । पु॒रो॒डाश᳚म् । एका॑दशकपाल॒मित्येका॑दश - क॒पा॒ल॒म् । निरिति॑ । व॒पे॒त् । यः । का॒मये॑त । दान॑कामा॒ इति॒ दान॑ - का॒माः॒ । मे॒ । प्र॒जा इति॑ प्र - जाः । स्युः॒ । \textbf{  42} \newline
                  \newline
                                \textbf{ TS 2.2.8.4} \newline
                  इति॑ । इन्द्र᳚म् । ए॒व । दा॒तार᳚म् । स्वेन॑ । भा॒ग॒धेये॒नेति॑ भाग - धेये॑न । उपेति॑ । धा॒व॒ति॒ । सः । ए॒व । अ॒स्मै॒ । दान॑कामा॒ इति॒ दान॑-का॒माः॒ । प्र॒जा इति॑ प्र - जाः । क॒रो॒ति॒ । दान॑कामा॒ इति॒ दान॑ - का॒माः॒ । अ॒स्मै॒ । प्र॒जा इति॑ प्र - जाः । भ॒व॒न्ति॒ । इन्द्रा॑य । प्र॒दा॒त्र इति॑ प्र - दा॒त्रे । पु॒रो॒डाश᳚म् । एका॑दशकपाल॒मित्येका॑दश - क॒पा॒ल॒म् । निरिति॑ । व॒पे॒त् । यस्मै᳚ । प्रत्त᳚म् । इ॒व॒ । सत् । न । प्र॒दी॒येतेति॑ प्र - दी॒येत॑ । इन्द्र᳚म् । ए॒व । प्र॒दा॒तार॒मिति॑ प्र - दा॒तार᳚म् । स्वेन॑ । भा॒ग॒धेये॒नेति॑ भाग - धेये॑न । उपेति॑ । धा॒व॒ति॒ । सः । ए॒व । अ॒स्मै॒ । प्रेति॑ । दा॒प॒य॒ति॒ । इन्द्रा॑य । सु॒त्रांण॒ इति॑ सु - त्रांणे᳚ । पु॒रो॒डाश᳚म् । एका॑दशकपाल॒मित्येका॑दश - क॒पा॒ल॒म् । निरिति॑ । व॒पे॒त् । अप॑रुद्ध॒ इत्यप॑ - रु॒द्धः॒ । वा॒ । \textbf{  43} \newline
                  \newline
                                \textbf{ TS 2.2.8.5} \newline
                  अ॒प॒रु॒द्ध्यमा॑न॒ इत्य॑प - रु॒द्ध्यमा॑नः । वा॒ । इन्द्र᳚म् । ए॒व । सु॒त्रामा॑ण॒मिति॑ सु - त्रामा॑णम् । स्वेन॑ । भा॒ग॒धेये॒नेति॑ भाग-धेये॑न । उपेति॑ । धा॒व॒ति॒ । सः । ए॒व । ए॒न॒म् । त्रा॒य॒ते॒ । अ॒न॒प॒रु॒द्ध्य इत्य॑नप - रु॒द्ध्यः । भ॒व॒ति॒ । इन्द्रः॑ । वै । स॒दृङ्ङिति॑ स - दृङ् । दे॒वता॑भिः । आ॒सी॒त् । सः । न । व्या॒वृत॒मिति॑ वि - आ॒वृत᳚म् । अ॒ग॒च्छ॒त् । सः । प्र॒जाप॑ति॒मिति॑ प्र॒जा - प॒ति॒म् । उपेति॑ । अ॒धा॒व॒त् । तस्मै᳚ । ए॒तम् । ऐ॒न्द्रम् । एका॑दशकपाल॒मित्येका॑दश - क॒पा॒ल॒म् । निरिति॑ । अ॒व॒प॒त् । तेन॑ । ए॒व । अ॒स्मि॒न्न् । इ॒न्द्रि॒यम् । अ॒द॒धा॒त् । शक्व॑री॒ इति॑ । या॒ज्या॒नु॒वा॒क्ये॑ इति॑ याज्या - अ॒नु॒वा॒क्ये᳚ । अ॒क॒रो॒त् । वज्रः॑ । वै । शक्व॑री । सः । ए॒न॒म् । वज्रः॑ । भूत्यै᳚ । ऐ॒न्ध॒ । \textbf{  44} \newline
                  \newline
                                \textbf{ TS 2.2.8.6} \newline
                  सः । अ॒भ॒व॒त् । सः । अ॒बि॒भे॒त् । भू॒तः । प्रेति॑ । मा॒ । ध॒क्ष्य॒ति॒ । इति॑ । सः । प्र॒जाप॑ति॒मिति॑ प्र॒जा-प॒ति॒म् । पुनः॑ । उपेति॑ । अ॒धा॒व॒त् । सः । प्र॒जाप॑ति॒रिति॑ प्र॒जा - प॒तिः॒ । शक्व॑र्याः । अधीति॑ । रे॒वती᳚म् । निरिति॑ । अ॒मि॒मी॒त॒ । शान्त्यै᳚ । अप्र॑दाहा॒येत्यप्र॑ - दा॒हा॒य॒ । यः । अल᳚म् । श्रि॒यै । सन्न् । स॒दृङ्ङिति॑ स - दृङ्ङ् । स॒मा॒नैः । स्यात् । तस्मै᳚ । ए॒तम् । ऐ॒न्द्रम् । एका॑दशकपाल॒मित्येका॑दश - क॒पा॒ल॒म् । निरिति॑ । व॒पे॒त् । इन्द्र᳚म् । ए॒व । स्वेन॑ । भा॒ग॒धेये॒नेति॑ भाग-धेये॑न । उपेति॑ । धा॒व॒ति॒ । सः । ए॒व । अ॒स्मि॒न्न् । इ॒न्द्रि॒यम् । द॒धा॒ति॒ । रे॒वती᳚ । पु॒रो॒नु॒वा॒क्येति॑ पुरः - अ॒नु॒वा॒क्या᳚ । भ॒व॒ति॒ ( ) । शान्त्यै᳚ । अप्र॑दाहा॒येत्यप्र॑ - दा॒हा॒य॒ । शक्व॑री । या॒ज्या᳚ । वज्रः॑ । वै । शक्व॑री । सः । ए॒न॒म् । वज्रः॑ । भूत्यै᳚ । इ॒न्धे॒ । भव॑ति । ए॒व ॥ \textbf{  45 } \newline
                  \newline
                      (अपि॒ - तꣳ - स्यु॑ - र्वै - न्ध - भवति॒ - चतु॑र्दश च )  \textbf{(A8)} \newline \newline
                                \textbf{ TS 2.2.9.1} \newline
                  आ॒ग्ना॒वै॒ष्ण॒वमित्या᳚ग्ना - वै॒ष्ण॒वम् । एका॑दशकपाल॒मित्येका॑दश - क॒पा॒ल॒म् । निरिति॑ । व॒पे॒त् । अ॒भि॒चर॒न्नित्य॑भि - चर॒न्न्॑ । सर॑स्वती । आज्य॑भा॒गेत्याज्य॑ - भा॒गा॒ । स्यात् । बा॒र्.॒ह॒स्प॒त्यः । च॒रुः । यत् । आ॒ग्ना॒वै॒ष्ण॒व इत्या᳚ग्ना - वै॒ष्ण॒वः । एका॑दशकपाल॒ इत्येका॑दश-क॒पा॒लः॒ । भव॑ति । अ॒ग्निः । सर्वाः᳚ । दे॒वताः᳚ । विष्णुः॑ । य॒ज्ञ्ः । दे॒वता॑भिः । च॒ । ए॒व । ए॒न॒म् । य॒ज्ञेन॑ । च॒ । अ॒भीति॑ । च॒र॒ति॒ । सर॑स्वती । आज्य॑भा॒गेत्याज्य॑-भा॒गा॒ । भ॒व॒ति॒ । वाक् । वै । सर॑स्वती । वा॒चा । ए॒व । ए॒न॒म् । अ॒भीति॑ । च॒र॒ति॒ । बा॒र्.॒ह॒स्प॒त्यः । च॒रुः । भ॒व॒ति॒ । ब्रह्म॑ । वै । दे॒वाना᳚म् । बृह॒स्पतिः॑ । ब्रह्म॑णा । ए॒व । ए॒न॒म् । अ॒भीति॑ । च॒र॒ति॒ । \textbf{  46} \newline
                  \newline
                                \textbf{ TS 2.2.9.2} \newline
                  प्रतीति॑ । वै । प॒रस्ता᳚त् । अ॒भि॒चर॑न्त॒मित्य॑भि - चर॑न्तम् । अ॒भीति॑ । च॒र॒न्ति॒ । द्वेद्वे॒ इति॒ द्वे - द्वे॒ । पु॒रो॒नु॒वा॒क्य॑ इति॑ पुरः - अ॒नु॒वा॒क्ये᳚ । कु॒र्या॒त् । अतीति॑ । प्रयु॑क्त्या॒ इति॒ प्र - यु॒क्त्यै॒ । ए॒तया᳚ । ए॒व । य॒जे॒त॒ । अ॒भि॒च॒र्यमा॑ण॒ इत्य॑भि - च॒र्यमा॑णः । दे॒वता॑भिः । ए॒व । दे॒वताः᳚ । प्र॒ति॒चर॒तीति॑ प्रति - चर॑ति । य॒ज्ञेन॑ । य॒ज्ञ्म् । वा॒चा । वाच᳚म् । ब्रह्म॑णा । ब्रह्म॑ । सः । दे॒वताः᳚ । च॒ । ए॒व । य॒ज्ञ्म् । च॒ । म॒द्ध्य॒तः । व्यव॑सर्प॒तीति॑ वि - अव॑सर्पति । तस्य॑ । न । कुतः॑ । च॒न । उ॒पा॒व्या॒ध इत्यु॑प - आ॒व्या॒धः । भ॒व॒ति॒ । न । ए॒न॒म् । अ॒भि॒चर॒न्नित्य॑भि - चरन्न्॑ । स्तृ॒णु॒ते॒ । आ॒ग्ना॒वै॒ष्ण॒वमित्या᳚ग्ना - वै॒ष्ण॒वम् । एका॑दशकपाल॒मित्येका॑दश - क॒पा॒ल॒म् । निरिति॑ । व॒पे॒त् । यम् । य॒ज्ञ्ः । न । \textbf{  47} \newline
                  \newline
                                \textbf{ TS 2.2.9.3} \newline
                  उ॒प॒नमे॒दित्यु॑प - नमे᳚त् । अ॒ग्निः । सर्वाः᳚ । दे॒वताः᳚ । विष्णुः॑ । य॒ज्ञ्ः । अ॒ग्निम् । च॒ । ए॒व । विष्णु᳚म् । च॒ । स्वेन॑ । भा॒ग॒धेये॒नेति॑ भाग-धेये॑न । उपेति॑ । धा॒व॒ति॒ । तौ । ए॒व । अ॒स्मै॒ । य॒ज्ञ्म् । प्रेति॑ । य॒च्छ॒तः॒ । उपेति॑ । ए॒न॒म् । य॒ज्ञ्ः । न॒म॒ति॒ । आ॒ग्ना॒वै॒ष्ण॒वमित्या᳚ग्ना - वै॒ष्ण॒वम् । घृ॒ते । च॒रुम् । निरिति॑ । व॒पे॒त् । चक्षु॑ष्काम॒ इति॒ चक्षुः॑ - का॒मः॒ । अ॒ग्नेः । वै । चक्षु॑षा । म॒नु॒ष्याः᳚ । वीति॑ । प॒श्य॒न्ति॒ । य॒ज्ञ्स्य॑ । दे॒वाः । अ॒ग्निम् । च॒ । ए॒व । विष्णु᳚म् । च॒ । स्वेन॑ । भा॒ग॒धेये॒नेति॑ भाग - धेये॑न । उपेति॑ । धा॒व॒ति॒ । तौ । ए॒व । \textbf{  48} \newline
                  \newline
                                \textbf{ TS 2.2.9.4} \newline
                  अ॒स्मि॒न्न् । चक्षुः॑ । ध॒त्तः॒ । चक्षु॑ष्मान् । ए॒व । भ॒व॒ति॒ । धे॒न्वै । वै । ए॒तत् । रेतः॑ । यत् । आज्य᳚म् । अ॒न॒डुहः॑ । त॒ण्डु॒लाः । मि॒थु॒नात् । ए॒व । अ॒स्मै॒ । चक्षुः॑ । प्रेति॑ । ज॒न॒य॒ति॒ । घृ॒ते । भ॒व॒ति॒ । तेजः॑ । वै । घृ॒तम् । तेजः॑ । चक्षुः॑ । तेज॑सा । ए॒व । अ॒स्मै॒ । तेजः॑ । चक्षुः॑ । अवेति॑ । रु॒न्धे॒ । इ॒न्द्रि॒यम् । वै । वी॒र्य᳚म् । वृ॒ङ्क्ते॒ । भ्रातृ॑व्यः । यज॑मानः । अय॑जमानस्य । अ॒द्ध्व॒रक॑ल्पा॒मित्य॑द्ध्व॒र-क॒ल्पा॒म् । प्रति॑ । निरिति॑ । व॒पे॒त् । भ्रातृ॑व्ये । यज॑माने । न । अ॒स्य॒ । इ॒न्द्रि॒यम् । \textbf{  49} \newline
                  \newline
                                \textbf{ TS 2.2.9.5} \newline
                  वी॒र्य᳚म् । वृ॒ङ्क्ते॒ । पु॒रा । वा॒चः । प्रव॑दितो॒रिति॒ प्र - व॒दि॒तोः॒ । निरिति॑ । व॒पे॒त् । याव॑ती । ए॒व । वाक् । ताम् । अप्रो॑दिता॒मित्यप्र॑ - उ॒दि॒ता॒म् । भ्रातृ॑व्यस्य । वृ॒ङ्क्ते॒ । ताम् । अ॒स्य॒ । वाच᳚म् । प्र॒वद॑न्ती॒मिति॑ प्र - वद॑न्तीम् । अ॒न्याः । वाचः॑ । अनु॑ । प्रेति॑ । व॒द॒न्ति॒ । ताः । इ॒न्द्रि॒यम् । वी॒र्य᳚म् । यज॑माने । द॒ध॒ति॒ । आ॒ग्ना॒वै॒ष्ण॒वमित्या᳚ग्ना - वै॒ष्ण॒वम् । अ॒ष्टाक॑पाल॒मित्य॒ष्टा-क॒पा॒ल॒म् । निरिति॑ । व॒पे॒त् । प्रा॒तः॒ स॒व॒नस्येति॑ प्रातः - स॒व॒नस्य॑ । आ॒का॒ल इत्या᳚ - का॒ले । सर॑स्वती । आज्य॑भा॒गेत्याज्य॑ - भा॒गा॒ । स्यात् । बा॒र्.॒ह॒स्प॒त्यः । च॒रुः । यत् । अ॒ष्टाक॑पाल॒ इत्य॒ष्टा - क॒पा॒लः॒ । भव॑ति । अ॒ष्टाक्ष॒रेत्य॒ष्टा- अ॒क्ष॒रा॒ । गा॒य॒त्री । गा॒य॒त्रम् । प्रा॒तः॒ स॒व॒नमिति॑ प्रातः - स॒व॒नम् । प्रा॒तः॒ स॒व॒नमिति॑ प्रातः - स॒व॒नम् । ए॒व । तेन॑ । आ॒प्नो॒ति॒ । \textbf{  50} \newline
                  \newline
                                \textbf{ TS 2.2.9.6} \newline
                  आ॒ग्ना॒वै॒ष्ण॒वमित्या᳚ग्ना - वै॒ष्ण॒वम् । एका॑दशकपाल॒मित्येका॑दश - क॒पा॒ल॒म् । निरिति॑ । व॒पे॒त् । माद्ध्य॑न्दिनस्य । सव॑नस्य । आ॒का॒ल इत्या᳚ - का॒ले । सर॑स्वती । आज्य॑भा॒गेत्याज्य॑ - भा॒गा॒ । स्यात् । बा॒र्.॒ह॒स्प॒त्यः । च॒रुः । यत् । एका॑दशकपाल॒ इत्येका॑दश - क॒पा॒लः॒ । भव॑ति । एका॑दशाक्ष॒रेत्येका॑दश - अ॒क्ष॒रा॒ । त्रि॒ष्टुप् । त्रैष्टु॑भम् । माद्ध्य॑न्दिनम् । सव॑नम् । माद्ध्य॑न्दिनम् । ए॒व । सव॑नम् । तेन॑ । आ॒प्नो॒ति॒ । आ॒ग्ना॒वै॒ष्ण॒वमित्या᳚ग्ना - वै॒ष्ण॒वम् । द्वाद॑शकपाल॒मिति॒ द्वाद॑श - क॒पा॒ल॒म् । निरिति॑ । व॒पे॒त् । तृ॒ती॒य॒स॒व॒नस्येति॑ तृतीय- स॒व॒नस्य॑ । आ॒का॒ल इत्या᳚ - का॒ले । सर॑स्वती । आज्य॑भा॒गेत्याज्य॑ - भा॒गा॒ । स्यात् । बा॒र्.॒ह॒स्प॒त्यः । च॒रुः । यत् । द्वाद॑शकपाल॒ इति॒ द्वाद॑श - क॒पा॒लः॒ । भव॑ति । द्वाद॑शाक्ष॒रेति॒ द्वाद॑श - अ॒क्ष॒रा॒ । जग॑ती । जाग॑तम् । तृ॒ती॒य॒स॒व॒नमिति॑ तृतीय - स॒व॒नम् । तृ॒ती॒य॒स॒व॒नमिति॑ तृतीय -स॒व॒नम् । ए॒व । तेन॑ । आ॒प्नो॒ति॒ । दे॒वता॑भिः । ए॒व । दे॒॒वताः᳚ । \textbf{  51} \newline
                  \newline
                                \textbf{ TS 2.2.9.7} \newline
                  प्र॒ति॒चर॒तीति॑ प्रति - चर॑ति । य॒ज्ञेन॑ । य॒ज्ञ्म् । वा॒चा । वाच᳚म् । ब्रह्म॑णा । ब्रह्म॑ । क॒पालैः᳚ । ए॒व । छन्दाꣳ॑सि । आ॒प्नोति॑ । पु॒रो॒डाशैः᳚ । सव॑नानि । मै॒त्रा॒व॒रु॒णमिति॑ मैत्रा - व॒रु॒णम् । एक॑कपाल॒मित्येक॑ - क॒पा॒ल॒म् । निरिति॑ । व॒पे॒त् । व॒शायै᳚ । का॒ले । या । ए॒व । अ॒सौ । भ्रातृ॑व्यस्य । व॒शा । अ॒नू॒ब॒न्ध्येत्य॑नु - ब॒न्ध्या᳚ । सो इति॑ । ए॒व । ए॒षा । ए॒तस्य॑ । एक॑कपाल॒ इत्येक॑-क॒पा॒लः॒ । भ॒व॒ति॒ । न । हि । क॒पालैः᳚ । प॒शुम् । अर्.ह॑ति । आप्तु᳚म् ॥(ब्रह्म॑णै॒वैन॑म॒भि च॑रति - य॒ज्ञो न - तावे॒वा - ऽस्ये᳚न्द्रि॒य - मा᳚प्नोति -दे॒वताः᳚ - \textbf{  52} \newline
                  \newline
                      (ब्रह्म॑णै॒वैन॑म॒भि च॑रति - य॒ज्ञो न - तावे॒वा - ऽस्ये᳚न्द्रि॒य - मा᳚प्नोति -दे॒वताः᳚ - स॒प्तत्रिꣳ॑शच्च )  \textbf{(A9)} \newline \newline
                                \textbf{ TS 2.2.10.1} \newline
                  अ॒सौ । आ॒दि॒त्यः । न । वीति॑ । अ॒रो॒च॒त॒ । तस्मै᳚ । दे॒वाः । प्राय॑श्चित्तिम् । ऐ॒च्छ॒न्न् । तस्मै᳚ । ए॒तम् । सो॒मा॒रौ॒द्रमिति॑ सोमा-रौ॒द्रम् । च॒रुम् । निरिति॑ । अ॒व॒प॒न्न् । तेन॑ । ए॒व । अ॒स्मि॒न्न् । रुच᳚म् । अ॒द॒धुः॒ । यः । ब्र॒ह्म॒व॒र्च॒सका॑म॒ इति॑ ब्रह्मवर्च॒स - का॒मः॒ । स्यात् । तस्मै᳚ । ए॒तम् । सो॒मा॒रौ॒द्रमिति॑ सोमा - रौ॒द्रम् । च॒रुम् । निरिति॑ । व॒पे॒त् । सोम᳚म् । च॒ । ए॒व । रु॒द्रम् । च॒ । स्वेन॑ । भा॒ग॒धेये॒नेति॑ भाग-धेये॑न । उपेति॑ । धा॒व॒ति॒ । तौ । ए॒व । अ॒स्मि॒न्न् । ब्र॒ह्म॒व॒र्च॒समिति॑ ब्रह्म - व॒र्च॒सम् । ध॒त्तः॒ । ब्र॒ह्म॒व॒र्च॒सीति॑ ब्रह्म - व॒र्च॒सी । ए॒व । भ॒व॒ति॒ । ति॒ष्या॒पू॒र्ण॒मा॒स इति॑ तिष्या - पू॒र्ण॒मा॒से । निरिति॑ । व॒पे॒त् । रु॒द्रः । \textbf{  53} \newline
                  \newline
                                \textbf{ TS 2.2.10.2} \newline
                  वै । ति॒ष्यः॑ । सोमः॑ । पू॒र्णमा॑स॒ इति॑ पू॒र्ण - मा॒सः॒ । सा॒क्षादिति॑ स - अ॒क्षात् । ए॒व । ब्र॒ह्म॒व॒र्च॒समिति॑ ब्रह्म - व॒र्च॒सम् । अवेति॑ । रु॒न्धे॒ । परि॑श्रित॒ इति॒ परि॑ - श्रि॒ते॒ । या॒ज॒य॒ति॒ । ब्र॒ह्म॒व॒र्च॒सस्येति॑ ब्रह्म - व॒र्च॒सस्य॑ । परि॑गृहीत्या॒ इति॒ परि॑ - गृ॒ही॒त्यै॒ । श्वे॒तायै᳚ । श्वे॒तव॑थ्साया॒ इति॑ श्वे॒त - व॒थ्सा॒यै॒ । दु॒ग्धम् । म॒थि॒तम् । आज्य᳚म् । भ॒व॒ति॒ । आज्य᳚म् । प्रोक्ष॑ण॒मिति॑ प्र - उक्ष॑णम् । आज्ये॑न । मा॒र्ज॒य॒न्ते॒ । याव॑त् । ए॒व । ब्र॒ह्म॒व॒र्च॒समिति॑ ब्रह्म - व॒र्च॒सम् । तत् । सर्व᳚म् । क॒रो॒ति॒ । अतीति॑ । ब्र॒ह्म॒व॒र्च॒समिति॑ ब्रह्म - व॒र्च॒सम् । क्रि॒य॒ते॒ । इति॑ । आ॒हुः॒ । ई॒श्व॒रः । दु॒श्चर्मेति॑ दुः - चर्मा᳚ । भवि॑तोः । इति॑ । मा॒न॒वी इति॑ । ऋचौ᳚ । धा॒य्ये॑ इति॑ । कु॒र्या॒त् । यत् । वै । किम् । च॒ । मनुः॑ । अव॑दत् । तत् । भे॒ष॒जम् । \textbf{  54} \newline
                  \newline
                                \textbf{ TS 2.2.10.3} \newline
                  भे॒ष॒जम् । ए॒व । अ॒स्मै॒ । क॒रो॒ति॒ । यदि॑ । बि॒भी॒यात् । दु॒श्चर्मेति॑ दुः - चर्मा᳚ । भ॒वि॒ष्या॒मि॒ । इति॑ । सो॒मा॒पौ॒ष्णमिति॑ सोमा - पौ॒ष्णम् । च॒रुम् । निरिति॑ । व॒पे॒त् । सौ॒म्यः । वै । दे॒वत॑या । पुरु॑षः । पौ॒ष्णाः । प॒शवः॑ । स्वया᳚ । ए॒व । अ॒स्मै॒ । दे॒वत॑या । प॒शुभि॒रिति॑ प॒शु - भिः॒ । त्वच᳚म् । क॒रो॒ति॒ । न । दु॒श्चर्मेति॑ दुः - चर्मा᳚ । भ॒व॒ति॒ । सो॒मा॒रौ॒द्रमिति॑ सोमा - रौ॒द्रम् । च॒रुम् । निरिति॑ । व॒पे॒त् । प्र॒जाका॑म॒ इति॑ प्र॒जा - का॒मः॒ । सोमः॑ । वै । रे॒तो॒धा इति॑ रेतः - धाः । अ॒ग्निः । प्र॒जाना॒मिति॑ प्र - जाना᳚म् । प्र॒ज॒न॒यि॒तेति॑ प्र - ज॒न॒यि॒ता । सोमः॑ । ए॒व । अ॒स्मै॒ । रेतः॑ । दधा॑ति । अ॒ग्निः । प्र॒जामिति॑ प्र - जाम् । प्रेति॑ । ज॒न॒य॒ति॒ । वि॒न्दते᳚ । \textbf{  55} \newline
                  \newline
                                \textbf{ TS 2.2.10.4} \newline
                  प्र॒जामिति॑ प्र -जाम् । सो॒मा॒रौ॒द्रमिति॑ सोमा- रौ॒द्रम् । च॒रुम् । निरिति॑ । व॒पे॒त् । अ॒भि॒चर॒न्नित्य॑भि-चरन्न्॑ । सौ॒म्यः । वै । दे॒वत॑या । पुरु॑षः । ए॒षः । रु॒द्रः । यत् । अ॒ग्निः । स्वायाः᳚ । ए॒व । ए॒न॒म् । दे॒वता॑यै । नि॒ष्क्रीयेति॑ निः - क्रीय॑ । रु॒द्राय॑ । अपीति॑ । द॒धा॒ति॒ । ता॒जक् । आर्ति᳚म् । एति॑ । ऋ॒च्छ॒ति॒ । सो॒मा॒रौ॒द्रमिति॑ सोमा-रौ॒द्रम् । च॒रुम् । निरिति॑ । व॒पे॒त् । ज्योगा॑मया॒वीति॒ ज्योक् - आ॒म॒या॒वी॒ । सोम᳚म् । वै । ए॒तस्य॑ । रसः॑ । ग॒च्छ॒ति॒ । अ॒ग्निम् । शरी॑रम् । यस्य॑ । ज्योक् । आ॒मय॑ति । सोमा᳚त् । ए॒व । अ॒स्य॒ । रस᳚म् । नि॒ष्क्री॒णातीति॑ निः - क्री॒णाति॑ । अ॒ग्नेः । शरी॑रम् । उ॒त । यदि॑ । \textbf{  56} \newline
                  \newline
                                \textbf{ TS 2.2.10.5} \newline
                  इ॒तासु॒रिती॒त - अ॒सुः॒ । भव॑ति । जीव॑ति । ए॒व । सो॒मा॒रु॒द्रयो॒रिति॑ सोमा - रु॒द्रयोः᳚ । वै । ए॒तम् । ग्र॒सि॒तम् । होता᳚ । निरिति॑ । खि॒द॒ति॒ । सः । ई॒श्व॒रः । आर्ति᳚म् । आता॒र्रिया - अ॒र्तोः॒ । अ॒न॒ड्वान् । होत्रा᳚ । देयः॑ । वह्निः॑ । वै । अ॒न॒ड्वान् । वह्निः॑ । होता᳚ । वह्नि॑ना । ए॒व । वह्नि᳚म् । आ॒त्मान᳚म् । स्पृ॒णो॒ति॒ । सो॒मा॒रौ॒द्रमिति॑ सोमा - रौ॒द्रम् । च॒रुम् । निरिति॑ । व॒पे॒त् । यः । का॒मये॑त । स्वे । अ॒स्मै॒ । आ॒यत॑न॒ इत्या᳚ - यत॑ने । भ्रातृ॑व्यम् । ज॒न॒ये॒य॒म् । इति॑ । वेदि᳚म् । प॒रि॒गृह्येति॑ परि - गृह्य॑ । अ॒र्द्धम् । उ॒द्ध॒न्यादित्यु॑त्-ह॒न्यात् । अ॒र्द्धम् । न । अ॒र्द्धम् । ब॒र्॒.हिषः॑ । स्तृ॒णी॒यात् । अ॒र्द्धम् ( ) । न । अ॒र्द्धम् । इ॒॒द्ध्मस्य॑ । अ॒भ्या॒द॒द्ध्यादित्य॑भि - आ॒द॒द्ध्यात् । अ॒र्द्धम् । न । स्वे । ए॒व । अ॒स्मै॒ । आ॒यत॑न॒ इत्या᳚ - यत॑ने । भ्रातृ॑व्यम् । ज॒न॒य॒ति॒ ॥ \textbf{ } \newline
                  \newline
                      -तासु॒ र्भव॑ति॒ जीव॑त्ये॒व सो॑मारु॒द्रयो॒र्वा ए॒तं ग्र॑सि॒तꣳ होता॒ निष्खि॑दति॒ स ई᳚श्व॒र आर्ति॒मार्तो॑-रन॒ड्वान्. होत्रा॒ देयो॒ वह्नि॒र्वा अ॑न॒ड्वान्. वह्नि॒र्॒.होता॒ वह्नि॑नै॒व वह्नि॑मा॒त्मानꣳ॑ स्पृणोति सोमारौ॒द्रं च॒रुं निर्व॑पे॒द्यः का॒मये॑त॒ स्वे᳚ऽस्मा आ॒यत॑ने॒ भ्रातृ॑व्यं जनयेय॒मिति॒ वेदिं॑ परि॒गृह्या॒र्द्धमु॑द्ध॒न्याद॒र्द्धं नार्द्धं ब॒र्॒.हिषः॑ स्तृणी॒याद॒र्द्धं ( ) नार्द्ध-मि॒द्ध्मस्या᳚भ्या-द॒द्ध्या-दद्॒र्द्धं न स्व ए॒वास्मा॑ आ॒यत॑ने॒ भ्रातृ॑व्यं जनयति ॥ 57 (रु॒द्रो - भे॑ष॒जं - वि॒न्दते॒- यदि॑ - स्तृणी॒याद॒र्द्धं - द्वाद॑श च)  \textbf{(A10)} \newline \newline
                                \textbf{ TS 2.2.11.1} \newline
                  ऐ॒न्द्रम् । एका॑दशकपाल॒मित्येका॑दश - क॒पा॒ल॒म् । निरिति॑ । व॒पे॒त् । मा॒रु॒तम् । स॒प्तक॑पाल॒मिति॑ स॒प्त - क॒पा॒ल॒म् । ग्राम॑काम॒ इति॒ ग्राम॑ - का॒मः॒ । इन्द्र᳚म् । च॒ । ए॒व । म॒रुतः॑ । च॒ । स्वेन॑ । भा॒ग॒धेये॒नेति॑ भाग - धेये॑न । उपेति॑ । धा॒व॒ति॒ । ते । ए॒व । अ॒स्मै॒ । स॒जा॒तानिति॑ स-जा॒तान् । प्रेति॑ । य॒च्छ॒न्ति॒ । ग्रा॒मी । ए॒व । भ॒व॒ति॒ । आ॒ह॒व॒नीय॒ इत्या᳚ - ह॒व॒नीये᳚ । ऐ॒न्द्रम् । अधीति॑ । श्र॒य॒ति॒ । गार्.ह॑पत्य॒ इति॒ गार्.ह॑ - प॒त्ये॒ । मा॒रु॒तम् । पा॒प॒व॒स्य॒सस्येति॑ पाप - व॒स्य॒सस्य॑ । विधृ॑त्या॒ इति॒ वि - धृ॒त्यै॒ । स॒प्तक॑पाल॒ इति॑ स॒प्त - क॒पा॒लः॒ । मा॒रु॒तः । भ॒व॒ति॒ । स॒प्तग॑णा॒ इति॑ स॒प्त - ग॒णाः॒ । वै । म॒रुतः॑ । ग॒ण॒श इति॑ गण-शः । ए॒व । अ॒स्मै॒ । स॒जा॒तानिति॑ स - जा॒तान् । अवेति॑ । रु॒न्धे॒ । अ॒नू॒च्यमा॑न॒ इत्य॑नु - उ॒च्यमा॑ने । एति॑ । सा॒द॒य॒ति॒ । विश᳚म् । ए॒व । \textbf{  58} \newline
                  \newline
                                \textbf{ TS 2.2.11.2} \newline
                  अ॒स्मै॒ । अनु॑वर्त्मान॒मित्यनु॑ - व॒र्त्मा॒न॒म् । क॒रो॒ति॒ । ए॒ताम् । ए॒व । निरिति॑ । व॒पे॒त् । यः । का॒मये॑त । क्ष॒त्राय॑ । च॒ । वि॒शे । च॒ । स॒मद॒मिति॑ स - मद᳚म् । द॒द्ध्या॒म् । इति॑ । ऐ॒न्द्रस्य॑ । अ॒व॒द्यन्नित्य॑व - द्यन्न् । ब्रू॒या॒त् । इन्द्रा॑य । अन्विति॑॑ । ब्रू॒हि॒ । इति॑ । आ॒श्राव्येत्या᳚ - श्राव्य॑ । ब्रू॒या॒त् । म॒रुतः॑ । य॒ज॒ । इति॑ । मा॒रु॒तस्य॑ । अ॒व॒द्यन्नित्य॑व - द्यन्न् । ब्रू॒या॒त् । म॒रुद्भ्य॒ इति॑ म॒रुत् - भ्यः॒ । अन्विति॑ । ब्रू॒हि॒ । इति॑ । आ॒श्राव्येत्या᳚ - श्राव्य॑ । ब्रू॒या॒त् । इन्द्र᳚म् । य॒ज॒ । इति॑ । स्वे । ए॒व । ए॒भ्यः॒ । भा॒ग॒धेय॒ इति॑ भाग - धेये᳚ । स॒मद॒मिति॑ स - मद᳚म् । द॒धा॒ति॒ । वि॒तृꣳ॒॒हा॒णा इति॑ वि-तृꣳ॒॒हा॒णाः । ति॒ष्ठ॒न्ति॒ । ए॒ताम् । ए॒व । \textbf{  59} \newline
                  \newline
                                \textbf{ TS 2.2.11.3} \newline
                  निरिति॑ । व॒पे॒त् । यः । का॒मये॑त । कल्पे॑रन्न् । इति॑ । य॒था॒द॒व॒तमिति॑ यथा - दे॒व॒तम् । अ॒व॒दायेत्य॑व - दाय॑ । य॒था॒दे॒व॒तमिति॑ यथा - दे॒व॒तम् । य॒जे॒त् । भा॒ग॒धेये॒नेति॑ भाग-धेये॑न । ए॒व । ए॒ना॒न् । य॒था॒य॒थमिति॑ यथा - य॒थम् । क॒ल्प॒य॒ति॒ । कल्प॑न्ते । ए॒व । ऐ॒न्द्रम् । एका॑दशकपाल॒मित्येका॑दश - क॒पा॒ल॒म् । निरिति॑ । व॒पे॒त् । वै॒श्व॒दे॒वमिति॑ वैश्व - दे॒वम् । द्वाद॑शकपाल॒मिति॒ द्वाद॑श - क॒पा॒ल॒म् । ग्राम॑काम॒ इति॒ ग्राम॑ - का॒मः॒ । इन्द्र᳚म् । च॒ । ए॒व । विश्वान्॑ । च॒ । दे॒वान् । स्वेन॑ । भा॒ग॒धेये॒नेति॑ भाग - धेये॑न । उपेति॑ । धा॒व॒ति॒ । ते । ए॒व । अ॒स्मै॒ । स॒जा॒तानिति॑ स - जा॒तान् । प्रेति॑ । य॒च्छ॒न्ति॒ । ग्रा॒मी । ए॒व । भ॒व॒ति॒ । ऐ॒न्द्रस्य॑ । अ॒व॒दायेत्य॑व - दाय॑ । वै॒श्व॒दे॒वस्येति॑ वैश्व-दे॒वस्य॑ । अवेति॑ । द्ये॒त् । अथ॑ । ऐ॒न्द्रस्यः॑ । \textbf{  60} \newline
                  \newline
                                \textbf{ TS 2.2.11.4} \newline
                  उ॒परि॑ष्टात् । इ॒न्द्रि॒येण॑ । ए॒व । अ॒स्मै॒ । उ॒भ॒यतः॑ । स॒जा॒तानिति॑ स - जा॒तान् । परीति॑ । गृ॒ह्णा॒ति॒ । उ॒पा॒धा॒य्य॑ पूर्वय॒मित्यु॑पाधा॒य्य॑ - पू॒र्व॒य॒म् । वासः॑ । दक्षि॑णा । स॒जा॒ताना॒मिति॑ स - जा॒ताना᳚म् । उप॑हित्या॒ इत्युप॑ - हि॒त्यै॒ । पृश्नि॑यै । दु॒ग्धे । प्रैय॑ङ्गवम् । च॒रुम् । निरिति॑ । व॒पे॒त् । म॒रुद्भ्य॒ इति॑ म॒रुत् - भ्यः॒ । ग्राम॑काम॒ इति॒ ग्राम॑ - का॒मः॒ । पृश्नि॑यै । वै । पय॑सः । म॒रुतः॑ । जा॒ताः । पृश्नि॑यै । प्रि॒यङ्ग॑वः । मा॒रु॒ताः । खलु॑ । वै । दे॒वत॑या । स॒जा॒ता इति॑ स - जा॒ताः । म॒रुतः॑ । ए॒व । स्वेन॑ । भा॒ग॒धेये॒नेति॑ भाग - धेये॑न । उपेति॑ । धा॒व॒ति॒ । ते । ए॒व । अ॒स्मै॒ । स॒जा॒तानिति॑ स - जा॒तान् । प्रेति॑ । य॒च्छ॒न्ति॒ । ग्रा॒मी । ए॒व । भ॒व॒ति॒ । प्रि॒यव॑ती॒ इति॑ प्रि॒य - व॒ती॒ । या॒ज्या॒नु॒वा॒क्ये॑ इति॑ याज्या - अ॒नु॒वा॒क्ये᳚ । \textbf{  61} \newline
                  \newline
                                \textbf{ TS 2.2.11.5} \newline
                  भ॒व॒तः॒ । प्रि॒यम् । ए॒व । ए॒न॒म् । स॒मा॒नाना᳚म् । क॒रो॒ति॒ । द्वि॒पदेति॑ द्वि - पदा᳚ । पु॒रो॒नु॒वा॒क्येति॑ पुरः - अ॒नु॒वा॒क्या᳚ । भ॒व॒ति॒ । द्वि॒पद॒ इति॑ द्वि - पदः॑ । ए॒व । अवेति॑ । रु॒न्धे॒ । चतु॑ष्प॒देति॒ चतुः॑ - प॒दा॒ । या॒ज्या᳚ । चतु॑ष्पद॒ इति॒ चतुः॑ - प॒दः॒ । ए॒व । प॒शून् । अवेति॑ । रु॒न्धे॒ । दे॒वा॒सु॒रा इति॑ देव - अ॒सु॒राः । संॅय॑त्ता॒ इति॒ सं - य॒त्ताः॒ । आ॒स॒न्न् । ते । दे॒वाः । मि॒थः । विप्रि॑या॒ इति॒ वि - प्रि॒याः॒ । आ॒स॒न्न् । ते । अ॒न्यः । अ॒न्यस्मै᳚ । ज्यैष्ठ्या॑य । अति॑ष्ठमानाः । च॒तु॒र्द्धेति॑ चतुः - धा । वीति॑ । अ॒क्रा॒म॒न्न् । अ॒ग्निः । वसु॑भि॒रिति॒ वसु॑ - भिः॒ । सोमः॑ । रु॒द्रैः । इन्द्रः॑ । म॒रुद्भि॒रिति॑ म॒रुत् - भिः॒ । वरु॑णः । आ॒दि॒त्यैः । सः । इन्द्रः॑ । प्र॒जाप॑ति॒मिति॑ प्र॒जा - प॒ति॒म् । उपेति॑ । अ॒धा॒व॒त् । तम् । \textbf{  62} \newline
                  \newline
                                \textbf{ TS 2.2.11.6} \newline
                  ए॒तया᳚ । स॒ज्ञांन्येति॑ सं - ज्ञान्या᳚ । अ॒या॒ज॒य॒त् । अ॒ग्नये᳚ । वसु॑मत॒ इति॒ वसु॑ - म॒ते॒ । पु॒रो॒डाश᳚म् । अ॒ष्टाक॑पाल॒मित्य॒ष्टा - क॒पा॒ल॒म् । निरिति॑ । अ॒व॒प॒त् । सोमा॑य । रु॒द्रव॑त॒ इति॑ रु॒द्र - व॒ते॒ । च॒रुम् । इन्द्रा॑य । म॒रुत्व॑ते । पु॒रो॒डाश᳚म् । एका॑दशकपाल॒मित्येका॑दश - क॒पा॒ल॒म् । वरु॑णाय । आ॒दि॒त्यव॑त॒ इत्या॑दि॒त्य - व॒ते॒ । च॒रुम् । ततः॑ । वै । इन्द्र᳚म् । दे॒वाः । ज्यैष्ठ्या॑य । अ॒भि । समिति॑ । अ॒जा॒न॒त॒ । यः । स॒मा॒नैः । मि॒थः । विप्रि॑य॒ इति॒ वि - प्रि॒यः॒ । स्यात् । तम् । ए॒तया᳚ । स॒ज्ञांन्येति॑ सं - ज्ञान्या᳚ । या॒ज॒ये॒त् । अ॒ग्नये᳚ । वसु॑मत॒ इति॒ वसु॑ - म॒ते॒ । पु॒रो॒डाश᳚म् । अ॒ष्टाक॑पाल॒मित्य॒ष्टा - क॒पा॒ल॒म् । निरिति॑ । व॒पे॒त् । सोमा॑य । रु॒द्रव॑त॒ इति॑ रु॒द्र - व॒ते॒ । च॒रुम् । इन्द्रा॑य । म॒रुत्व॑ते । पु॒रो॒डाश᳚म् । एका॑दशकपाल॒मित्येका॑दश - क॒पा॒ल॒म् । वरु॑णाय ( ) । आ॒दि॒त्यव॑त॒ इत्या॑दि॒त्य - व॒ते॒ । च॒रुम् । इन्द्र᳚म् । ए॒व । ए॒न॒म् । भू॒तम् । ज्यैष्ठ्या॑य । स॒मा॒नाः । अ॒भि । समिति॑ । जा॒न॒ते॒ । वसि॑ष्ठः । स॒मा॒नाना᳚म् । भ॒व॒ति॒ ॥ \textbf{  63} \newline
                  \newline
                      (विश॑मे॒व - ति॑ष्ठन्त्ये॒तामे॒ - वाथै॒न्द्रस्य॑ - याज्यानुवा॒क्ये॑ - तं - ॅवरु॑णाय॒ -चतु॑र्दश च)  \textbf{(A11)} \newline \newline
                                \textbf{ TS 2.2.12.1} \newline
                  हि॒र॒ण्य॒ग॒र्भ इति॑ हिरण्य - ग॒र्भः । आपः॑ । ह॒ । यत् । प्रजा॑पत॒ इति॒ प्रजा᳚ - प॒ते॒ ॥ सः । वे॒द॒ । पु॒त्रः । पि॒तर᳚म् । सः । मा॒तर᳚म् । सः । सू॒नुः । भु॒व॒त् । सः । भु॒व॒त् । पुन॑र्मघ॒ इति॒ पुनः॑ - म॒घः॒ ॥ सः । द्याम् । और्णो᳚त् । अ॒न्तरि॑क्षम् । सः । सुवः॑ । सः । विश्वाः᳚ । भुवः॑ । अ॒भ॒व॒त् । सः । एति॑ । अ॒भ॒व॒त् ॥ उदिति॑ । उ॒ । त्यम् । चि॒त्रम् ॥ सः । प्र॒त्न॒वदिति॑ प्रत्न-वत् । नवी॑यसा । अग्ने᳚ । द्यु॒म्नेन॑ । सं॒ॅयतेति॑ सं - यता᳚ ॥ बृ॒हत् । त॒त॒न्थ॒ । भा॒नुना᳚ ॥ नीति॑ । काव्या᳚ । वे॒धसः॑ । शश्व॑तः । कः॒ । हस्ते᳚ । दधा॑नः । \textbf{  64} \newline
                  \newline
                                \textbf{ TS 2.2.12.2} \newline
                  नर्या᳚ । पु॒रूणि॑ ॥ अ॒ग्निः । भु॒व॒त् । र॒यि॒पति॒रिति॑ रयि - पतिः॑ । र॒यी॒णाम् । स॒त्रा । च॒क्रा॒णः । अ॒मृता॑नि । विश्वा᳚ ॥ हिर॑ण्यपाणि॒मिति॒ हिर॑ण्य - पा॒णि॒म् । ऊ॒तये᳚ । स॒वि॒तार᳚म् । उपेति॑ । ह्व॒ये॒ ॥ सः । चेत्ता᳚ । दे॒वता᳚ । प॒दम् ॥ वा॒मम् । अ॒द्य । स॒वि॒तः॒ । वा॒मम् । उ॒ । श्वः । दि॒वेदि॑व॒ इति॑ दि॒वे - दि॒वे॒ । वा॒मम् । अ॒स्मभ्य॒मित्य॒स्म - भ्य॒म् । सा॒वीः॒ ॥ वा॒मस्य॑ । हि । क्षय॑स्य । दे॒व॒ । भूरेः᳚ । अ॒या । धि॒या । वा॒म॒भाज॒ इति॑ वाम - भाजः॑ । स्या॒म॒ ॥ बट् । इ॒त्था । पर्व॑तानाम् । खि॒द्रम् । बि॒भ॒र्॒.षि॒ । पृ॒थि॒वि॒ ॥ प्रेति॑ । या । भू॒मि॒ । प्र॒व॒त्व॒ति॒ । म॒ह्ना । जि॒नोषि॑ । \textbf{  65} \newline
                  \newline
                                \textbf{ TS 2.2.12.3} \newline
                  म॒हि॒नि॒ ॥ स्तोमा॑सः । त्वा॒ । वि॒चा॒रि॒णीति॑ वि - चा॒रि॒णि॒ । प्रतीति॑ । स्तो॒भ॒न्ति॒ । अ॒क्तुभि॒रित्य॒क्तु - भिः॒ ॥ प्रेति॑ । या । वाज᳚म् । न । हेष॑न्तम् । पे॒रुम् । अस्य॑सि । अ॒र्जु॒नि॒ ॥ ऋ॒दू॒दरे॑ण । सख्या᳚ । स॒चे॒य॒ । यः । मा॒ । न । रिष्ये᳚त् । ह॒र्य॒श्वेति॑ हरि-अ॒श्व॒ । पी॒तः ॥ अ॒यम् । यः । सोमः॑ । न्यधा॒यीति॑ नि - अधा॑यि । अ॒स्मे इति॑ । तस्मै᳚ । इन्द्र᳚म् । प्र॒तिर॒मिति॑ प्र - तिर᳚म् । ए॒मि॒ । अच्छ॑ ॥ आपा᳚न्तमन्यु॒रित्यापा᳚न्त - म॒न्युः॒ । तृ॒पल॑प्रभ॒र्मेति॑ तृ॒पल॑ - प्र॒भ॒र्मा॒ । धुनिः॑ । शिमी॑वान् । शरु॑मा॒निति॒ शरु॑ - मा॒न् । ऋ॒जी॒षी ॥ सोमः॑ । विश्वा॑नि । अ॒त॒सा । वना॑नि । न । अ॒र्वाक् । इन्द्र᳚म् । प्र॒ति॒माना॒नीति॑ प्रति - माना॑नि । दे॒भुः॒ ॥ प्रेति॑ । \textbf{  66} \newline
                  \newline
                                \textbf{ TS 2.2.12.4} \newline
                  सु॒वा॒नः । सोमः॑ । ऋ॒त॒युरित्यृ॑त - युः । चि॒के॒त॒ । इन्द्रा॑य । ब्रह्म॑ । ज॒मद॑ग्निः । अर्चन्न्॑ ॥ वृषा᳚ । य॒न्ता । अ॒सि॒ । शव॑सः । तु॒रस्य॑ । अ॒न्तः । य॒च्छ॒ । गृ॒ण॒ते । ध॒र्त्रम् । दृꣳ॒॒ह॒ ॥ स॒बाध॒ इति॑ स - बाधः॑ । ते॒ । मद᳚म् । च॒ । शु॒ष्म॒यम् । च॒ । ब्रह्म॑ । नरः॑ । ब्र॒ह्म॒कृत॒ इति॑ ब्रह्म - कृतः॑ । स॒प॒र्य॒न्न् ॥ अ॒र्कः । वा॒ । यत् । तु॒रते᳚ । सोम॑चक्षा॒ इति॒ सोम॑ - च॒क्षाः॒ । तत्र॑ । इत् । इन्द्रः॑ । द॒ध॒ते॒ । पृ॒थ्स्विति॑ पृत् - सु । तु॒र्याम् ॥ वष॑ट् । ते॒ । वि॒ष्णो॒ । आ॒सः । एति॑ । कृ॒णो॒मि॒ । तत् । मे॒ । जु॒ष॒स्व॒ । शि॒पि॒वि॒ष्टेति॑ शिपि - वि॒ष्ट॒ । ह॒व्यम् ॥ \textbf{  67} \newline
                  \newline
                                \textbf{ TS 2.2.12.5} \newline
                  वर्द्ध॑न्तु । त्वा॒ । सु॒ष्टु॒तय॒ इति॑ सु - स्तु॒तयः॑ । गिरः॑ । मे॒ । यू॒यम् । पा॒त॒ । स्व॒स्तिभि॒रिति॑ स्व॒स्ति - भिः॒ । सदा᳚ । नः॒ ॥ प्रेति॑ । तत् । ते॒ । अ॒द्य । शि॒पि॒वि॒ष्टेति॑ शिपि - वि॒ष्ट॒ । नाम॑ । अ॒र्यः । शꣳ॒॒सा॒मि॒ । व॒युना॑नि । वि॒द्वान् ॥ तम् । त्वा॒ । गृ॒णा॒मि॒ । त॒वस᳚म् । अत॑वीयान् । क्षय॑न्तम् । अ॒स्य । रज॑सः । प॒रा॒के ॥ किम् । इत् । ते॒ । वि॒ष्णो॒ इति॑ । प॒रि॒चक्ष्य॒मिति॑ परि - चक्ष्य᳚म् । भू॒त् । प्रेति॑ । यत् । व॒व॒क्षे । शि॒पि॒वि॒ष्ट इति॑ शिपि - वि॒ष्टः । अ॒स्मि॒ ॥ मा । वर्पः॑ । अ॒स्मत् । अपेति॑ । गू॒हः॒ । ए॒तत् । यत् । अ॒न्यरू॑प॒ इत्य॒न्य - रू॒पः॒ । स॒मि॒थ इति॑ सं - इ॒थे । ब॒भूथ॑ ॥ \textbf{  68} \newline
                  \newline
                                \textbf{ TS 2.2.12.6} \newline
                  अग्ने᳚ । दाः । दा॒शुषे᳚ । र॒यिम् । वी॒रव॑न्त॒मिति॑ वी॒र - व॒न्त॒म् । परी॑णस॒मिति॒ परि॑ - न॒स॒म् । शि॒शी॒हि । नः॒ । सू॒नु॒मत॒ इति॑ सूनु-मतः॑ ॥ दाः । नः॒ । अ॒ग्ने॒ । श॒तिनः॑ । दाः । स॒ह॒स्रिणः॑ । दु॒रः । न । वाज᳚म् । श्रुत्यै᳚ । अपेति॑ । वृ॒धि॒ ॥ प्राची॒ इति॑ । द्यावा॑पृथि॒वी इति॒ द्यावा᳚ - पृ॒थि॒वी । ब्रह्म॑णा । कृ॒धि॒ । सुवः॑ । न । शु॒क्रम् । उ॒षसः॑ । वीति॑ । दि॒द्यु॒तुः॒ ॥ अ॒ग्निः । दाः॒ । द्रवि॑णम् । वी॒रपे॑शा॒ इति॑ वी॒र - पे॒शाः॒ । अ॒ग्निः । ऋषि᳚म् । यः । स॒हस्रा᳚ । स॒नोति॑ ॥ अ॒ग्निः । दि॒वि । ह॒व्यम् । एति॑ । त॒ता॒न॒ । अ॒ग्नेः । धामा॑नि । विभृ॒तेति॒ वि - भृ॒ता॒ । पु॒रु॒त्रेति॑ पुरु - त्रा ॥ मा । \textbf{  69} \newline
                  \newline
                                \textbf{ TS 2.2.12.7} \newline
                  नः॒ । म॒र्द्धीः॒ । एति॑ । तु । भ॒र॒ ॥ घृ॒तम् । न । पू॒तम् । त॒नूः । अ॒रे॒पाः । शुचि॑ । हिर॑ण्यम् ॥ तत् । ते॒ । रु॒क्मः । न । रो॒च॒त॒ । स्व॒धा॒व॒ इति॑ स्वधा - वः॒ ॥ उ॒भे इति॑ । सु॒श्च॒न्द्रेति॑ सु - च॒न्द्र॒ । स॒र्पिषः॑ । दर्वी॒ इति॑ । श्री॒णी॒षे॒ । आ॒सनि॑ ॥ उ॒तो इति॑ । नः॒ । उदिति॑ । पु॒पू॒र्याः॒ । उ॒क्थेषु॑ । श॒व॒सः॒ । प॒ते॒ । इष᳚म् । स्तो॒तृभ्य॒ इति॑ स्तो॒तृ-भ्यः॒ । एति॑ । भ॒र॒ ॥ वायो॒ इति॑ । श॒तम् । हरी॑णाम् । यु॒वस्व॑ । पोष्या॑णाम् ॥ उ॒त । वा॒ । ते॒ । स॒ह॒स्रिणः॑ । रथः॑ । एति॑ । या॒तु॒ । पाज॑सा ॥ प्रेति॑ । याभिः॑ । \textbf{  70} \newline
                  \newline
                                \textbf{ TS 2.2.12.8} \newline
                  यासि॑ । दा॒श्वाꣳस᳚म् । अच्छ॑ । नि॒युद्भि॒रिति॑ नि॒युत् - भिः॒ । वा॒यो॒ । इ॒ष्टये᳚ । दु॒रो॒ण इति॑ दुः - ओ॒ने ॥ नीति॑ । नः॒ । र॒यिम् । सु॒भोज॑स॒मिति॑ सु - भोज॑सम् । यु॒व॒ । इ॒ह । नीति॑ । वी॒रव॒दिति॑ वी॒र - व॒त् । गव्य᳚म् । अश्वि॑यम् । च॒ । राधः॑ ॥ रे॒वतीः᳚ । नः॒ । स॒ध॒माद॒ इति॑ सध - मादः॑ । इन्द्रे᳚ । स॒न्तु॒ । तु॒विवा॑जा॒ इति॑ तु॒वि - वा॒जाः॒ ॥ क्षु॒मन्तः॑ । याभिः॑ । मदे॑म ॥ रे॒वान् । इत् । रे॒वतः॑ । स्तो॒ता । स्यात् । त्वाव॑त॒ इति॒ त्व - व॒तः॒ । म॒घोनः॑ ॥ प्रेति॑ । इत् । उ॒ । ह॒रि॒व॒ इति॑ हरि - वः॒ । श्रु॒तस्य॑ ॥ \textbf{  71} \newline
                  \newline
                      (दधा॑नो - जि॒नोषि॑ - देभुः॒ प्र - ह॒व्यं - ब॒भूथ॒ - मा - याभि॑ - श्चत्वारिꣳ॒॒शच्च॑ )  \textbf{(A12)} \newline \newline
\textbf{praSna korvai with starting padams of 1 to 12 Anuvaakams :-} \newline
(प्र॒जाप॑ति॒स्ताः सृ॒ष्टा - अ॒ग्नये॑ पथि॒कृते॒ - ग्नये॒ कामा॑या॒ - ग्नयेन्न॑वते -वैश्वान॒र -मा॑दि॒त्यं च॒रु - मै॒न्द्रं च॒रु - मिन्द्रा॒यान्वृ॑जव - आग्नावैष्ण॒व -म॒सौ सो॑मारौ॒द्र - मै॒न्द्रम॒का॑दशकपालꣳ- हिरण्यग॒र्भो - द्वाद॑श ) \newline

\textbf{korvai with starting padams of1, 11, 21 series of pa~jcAtis :-} \newline
(प्र॒जाप॑ति - र॒ग्नये॒ कामा॑या॒ - ऽभि सं भ॑वतो॒ - यो वि॑द्विषा॒णयो॑ -रि॒ध्मे सन्न॑ ह्ये - दाग्नावैष्ण॒वमु॒ - परि॑ष्टा॒ - द्यासि॑ दा॒श्वाꣳस॒ - मेक॑सप्ततिः ) \newline

\textbf{first and last padam of second praSnam of kANDam 2 :-} \newline
(प्र॒जाप॑तिः॒ - प्रेदु॑ हरिवः श्रु॒तस्य॑) \newline 


॥ हरिः॑ ॐ ॥॥ कृष्ण यजुर्वेदीय तैत्तिरीय संहितायां द्वितीयकाण्डे द्वितीयः प्रश्नः समाप्तः ॥ \newline
\pagebreak
2.2.1    AppEndix\\2.2.12.1 - हि॒र॒ण्य॒ग॒र्भ >1,\\हि॒र॒ण्य॒ग॒र्भः सम॑वर्त॒ताग्रे॑ भू॒तस्य॑ जा॒तः पति॒रेक॑ आसीत् ।\\स दा॑धार पृथि॒वीं द्यामु॒तेमां कस्मै॑ दे॒वाय॑ ह॒विषा॑ विदेम ॥\\(Appearing in TS 4-1-8-3)\\\\2.2.12.1 - आपो॑ ह॒ यत्>2, \\आपो॑ ह॒ यन् म॑ह॒ती र्विश्व॒माय॒न् दक्षं॒ दधा॑ना ज॒नय॑न्तीर॒ग्निं ।\\ततो॑ दे॒वानां॒ निर॑वर्त॒तासु॒रेकः॒ कस्मै॑ दे॒वाय॑ ह॒विषा॑ विधेम ॥\\(Appearing in TS 4-1-8-3)\\\\2.2.12.1 - प्रजा॑पते>3\\प्रजा॑पते॒ न त्वदे॒तान्य॒न्यो विश्वा॑ जा॒तानि॒ परि॒ ता ब॑भूव । \\यत्का॑मास्ते जुहु॒मस्तन्नो॑ अस्तु व॒यꣳ स्या॑म॒ पत॑यो रयी॒णां ॥\\(Appearing in TS 1-8-14-2)\\\\2.2.12.1 - उदु॒त्यं>4\\उदु॒ त्यंजा॒तवे॑दसं दे॒वं ॅव॑हन्ति के॒तवः॑ । \\दृ॒शे विश्वा॑य॒ सूर्यं᳚ ॥(appearing in TS 1-4-43-1)\\\\2.2.12.1 - चि॒त्रं>5\\चि॒त्रं दे॒वाना॒-मुद॑गा॒दनी॑कं॒ चक्षु॑ र्मि॒त्रस्य॒ वरु॑णस्या॒ऽग्नेः । \\आऽ प्रा॒ द्यावा॑पृथि॒वी अ॒न्तरि॑क्षꣳ॒॒ सूर्य॑ आ॒त्मा जग॑तस्त॒स्थुष॑श्च ॥\\(Appearing in TS 1-4-43-1)\\\\2.2.12.7- नो॑ मर्धी॒>6\\मा नो॑ मर्धी॒रा भ॑रा द॒द्धि तन्नः॒ प्र दा॒शुषे॒ दात॑वे॒ भूरि॒ यत् ते᳚ । \\नव्ये॑ दे॒ष्णे श॒स्ते अ॒स्मिन् त॑ उ॒क्थे प्र ब्र॑वाम व॒यमि॑न्द्र स्तु॒वन्तः॑ ॥\\(Appearing in TS 1-7-13-3)\\\\2.2.12.7 - रा तू भ॑र>7\\आ तू भ॑र॒ माकि॑रे॒तत् परि॑ष्ठाद्वि॒द्मा हि त्वा॒ वसु॑पतिं॒ ॅवसू॑नां । \\इन्द्र॒ यत् ते॒ माहि॑नं॒ दत्र॒मस्त्य॒स्मभ्यं॒ तद्ध॑र्यश्व॒ प्रय॑न्धि ॥\\(Appearing in TS 1-7-13-3)\\==================================\\
\pagebreak
        


\end{document}
