\documentclass[17pt]{extarticle}
\usepackage{babel}
\usepackage{fontspec}
\usepackage{polyglossia}
\usepackage{extsizes}



\setmainlanguage{sanskrit}
\setotherlanguages{english} %% or other languages
\setlength{\parindent}{0pt}
\pagestyle{myheadings}
\newfontfamily\devanagarifont[Script=Devanagari]{AdishilaVedic}


\newcommand{\VAR}[1]{}
\newcommand{\BLOCK}[1]{}




\begin{document}
\begin{titlepage}
    \begin{center}
 
\begin{sanskrit}
    { \Large
    ॐ नमः परमात्मने, श्री महागणपतये नमः, श्री गुरुभ्यो नमः
ह॒रिः॒ ॐ 
    }
    \\
    \vspace{2.5cm}
    \mbox{ \Huge
    6.5      षष्ठकाण्डे पञ्चमः प्रश्नः - सोममन्त्रब्राह्मणनिरूपणं   }
\end{sanskrit}
\end{center}

\end{titlepage}
\tableofcontents

ॐ नमः परमात्मने, श्री महागणपतये नमः, श्री गुरुभ्यो नमः
ह॒रिः॒ ॐ \newline
6.5      षष्ठकाण्डे पञ्चमः प्रश्नः - सोममन्त्रब्राह्मणनिरूपणं \newline

\addcontentsline{toc}{section}{ 6.5      षष्ठकाण्डे पञ्चमः प्रश्नः - सोममन्त्रब्राह्मणनिरूपणं}
\markright{ 6.5      षष्ठकाण्डे पञ्चमः प्रश्नः - सोममन्त्रब्राह्मणनिरूपणं \hfill https://www.vedavms.in \hfill}
\section*{ 6.5      षष्ठकाण्डे पञ्चमः प्रश्नः - सोममन्त्रब्राह्मणनिरूपणं }
                                \textbf{ TS 6.5.1.1} \newline
                  इन्द्रः॑ । वृ॒त्राय॑ । वज्र᳚म् । उदिति॑ । अ॒य॒च्छ॒त् । सः । वृ॒त्रः । वज्रा᳚त् । उद्य॑ता॒दित्युत्-य॒ता॒त् । अ॒बि॒भे॒त् । सः । अ॒ब्र॒वी॒त् । मा । मे॒ । प्रेति॑ । हाः॒ । अस्ति॑ । वै । इ॒दम् । मयि॑ । वी॒र्य᳚म् । तत् । ते॒ । प्रेति॑ । दा॒स्या॒मि॒ । इति॑ । तस्मै᳚ । उ॒क्थ्य᳚म् । प्रेति॑ । अ॒य॒च्छ॒त् । तस्मै᳚ । द्वि॒तीय᳚म् । उदिति॑ । अ॒य॒च्छ॒त् । सः । अ॒ब्र॒वी॒त् । मा । मे॒ । प्रेति॑ । हाः॒ । अस्ति॑ । वै । इ॒दम् । मयि॑ । वी॒र्य᳚म् । तत् । ते॒ । प्रेति॑ । दा॒स्या॒मि॒ । इति॑ । \textbf{  1 } \newline
                  \newline
                                \textbf{ TS 6.5.1.2} \newline
                  तस्मै᳚ । उ॒क्थ्य᳚म् । ए॒व । प्रेति॑ । अ॒य॒च्छ॒त् । तस्मै᳚ । तृ॒तीय᳚म् । उदिति॑ । अ॒य॒च्छ॒त् । तम् । विष्णुः॑ । अन्विति॑ । अ॒ति॒ष्ठ॒त॒ । ज॒हि । इति॑ । सः । अ॒ब्र॒वी॒त् । मा । मे॒ । प्रेति॑ । हाः॒ । अस्ति॑ । वै । इ॒दम् । मयि॑ । वी॒र्य᳚म् । तत् । ते॒ । प्रेति॑ । दा॒स्या॒मि॒ । इति॑ । तस्मै᳚ । उ॒क्थ्य᳚म् । ए॒व । प्रेति॑ । अ॒य॒च्छ॒त् । तम् । निर्मा॑य॒मिति॒ निः-मा॒य॒म् । भू॒तम् । अ॒ह॒न्न् । य॒ज्ञ्ः । हि । तस्य॑ । मा॒या । आसी᳚त् । यत् । उ॒क्थ्यः॑ । गृ॒ह्यते᳚ । इ॒न्द्रि॒यम् । ए॒व । \textbf{  2} \newline
                  \newline
                                \textbf{ TS 6.5.1.3} \newline
                  तत् । वी॒र्य᳚म् । यज॑मानः । भ्रातृ॑व्यस्य । वृ॒ङ्क्ते॒ । इन्द्रा॑य । त्वा॒ । बृ॒हद्व॑त॒ इति॑ बृ॒हत् - व॒ते॒ । वय॑स्वते । इति॑ । आ॒ह॒ । इन्द्रा॑य । हि । सः । तम् । प्रेति॑ । अय॑च्छत् । तस्मै᳚ । त्वा॒ । विष्ण॑वे । त्वा॒ । इति॑ । आ॒ह॒ । यत् । ए॒व । विष्णुः॑ । अ॒न्वति॑ष्ठ॒तेत्य॑नु - अति॑ष्ठत । ज॒हि । इति॑ । तस्मा᳚त् । विष्णु᳚म् । अ॒न्वाभ॑ज॒तीत्य॑नु - आभ॑जति । त्रिः । निरिति॑ । गृ॒ह्णा॒ति॒ । त्रिः । हि । सः । तम् । तस्मै᳚ । प्रेति॑ । अय॑च्छत् । ए॒षः । ते॒ । योनिः॑ । पुन॑र्.हवि॒रिति॒ पुनः॑ - ह॒विः॒ । अ॒सि॒ । इति॑ । आ॒ह॒ । पुनः॑पुन॒रिति॒ पुनः॑ - पु॒नः॒ । \textbf{  3} \newline
                  \newline
                                \textbf{ TS 6.5.1.4} \newline
                  हि । अ॒स्मा॒त् । नि॒र्गृ॒ह्णातीति॑ निः - गृ॒ह्णाति॑ । चक्षुः॑ । वै । ए॒तत् । य॒ज्ञ्स्य॑ । यत् । उ॒क्थ्यः॑ । तस्मा᳚त् । उ॒क्थ्य᳚म् । हु॒तम् । सोमाः᳚ । अ॒न्वाय॒न्तीत्य॑नु - आय॑न्ति । तस्मा᳚त् । आ॒त्मा । चक्षुः॑ । अन्विति॑ । ए॒ति॒ । तस्मा᳚त् । एक᳚म् । यन्त᳚म् । ब॒हवः॑ । अन्विति॑ । य॒न्ति॒ । तस्मा᳚त् । एकः॑ । ब॒हू॒नाम् । भ॒द्रः । भ॒व॒ति॒ । तस्मा᳚त् । एकः॑ । ब॒ह्वीः । जा॒याः । वि॒न्द॒ते॒ । यदि॑ । का॒मये॑त । अ॒द्ध्व॒र्युः । आ॒त्मान᳚म् । य॒ज्ञ्॒य॒श॒सेनेति॑ यज्ञ् - य॒श॒सेन॑ । अ॒र्प॒ये॒य॒म् । इति॑ । अ॒न्त॒रा । आ॒ह॒व॒नीय॒मित्या᳚ - ह॒व॒नीय᳚म् । च॒ । ह॒वि॒द्‌र्धान॒मिति॑ हविः - धान᳚म् । च॒ । तिष्ठन्न्॑ । अवेति॑ । न॒ये॒त् । \textbf{  4} \newline
                  \newline
                                \textbf{ TS 6.5.1.5} \newline
                  आ॒त्मान᳚म् । ए॒व । य॒ज्ञ्॒य॒श॒सेनेति॑ यज्ञ्-य॒श॒सेन॑ । अ॒र्प॒य॒ति॒ । यदि॑ । का॒मये॑त । यज॑मानम् । य॒ज्ञ्॒य॒श॒सेनेति॑ यज्ञ् - य॒श॒सेन॑ । अ॒र्प॒ये॒य॒म् । इति॑ । अ॒न्त॒रा । स॒दो॒ह॒वि॒द्‌र्धा॒ने इति॑ सदः - ह॒वि॒द्‌र्धा॒ने । तिष्ठन्न्॑ । अवेति॑ । न॒ये॒त् । यज॑मानम् । ए॒व । य॒ज्ञ्॒य॒श॒सेनेति॑ यज्ञ् - य॒श॒सेन॑ । अ॒र्प॒य॒ति॒ । यदि॑ । का॒मये॑त । स॒द॒स्यान्॑ । य॒ज्ञ्॒य॒श॒सेनेति॑ यज्ञ् - य॒श॒सेन॑ । अ॒र्प॒ये॒य॒म् । इति॑ । सदः॑ । आ॒लभ्येत्या᳚ - लभ्य॑ । अवेति॑ । न॒ये॒त् । स॒द॒स्यान्॑ । ए॒व । य॒ज्ञ्॒य॒श॒सेनेति॑ यज्ञ् - य॒श॒सेन॑ । अ॒र्प॒य॒ति॒ ॥ \textbf{  5} \newline
                  \newline
                      (इती᳚ - न्द्रि॒यमे॒व - पुनः॑ पुन - र्नये॒त् - त्रय॑स्त्रिꣳशच्च)  \textbf{(A1)} \newline \newline
                                \textbf{ TS 6.5.2.1} \newline
                  आयुः॑ । वै । ए॒तत् । य॒ज्ञ्स्य॑ । यत् । ध्रु॒वः । उ॒त्त॒म इत्यु॑त् - त॒मः । ग्रहा॑णाम् । गृ॒ह्य॒ते॒ । तस्मा᳚त् । आयुः॑ । प्रा॒णाना॒मिति॑ प्र - अ॒नाना᳚म् । उ॒त्त॒ममित्यु॑त्-त॒मम् । मू॒द्‌र्धान᳚म् । दि॒वः । अ॒र॒तिम् । पृ॒थि॒व्याः । इति॑ । आ॒ह॒ । मू॒द्‌र्धान᳚म् । ए॒व । ए॒न॒म् । स॒मा॒नाना᳚म् । क॒रो॒ति॒ । वै॒श्वा॒न॒रम् । ऋ॒ताय॑ । जा॒तम् । अ॒ग्निम् । इति॑ । आ॒ह॒ । वै॒श्वा॒न॒रम् । हि । दे॒वत॑या । आयुः॑ । उ॒भ॒यतो॑वैश्वानर॒ इत्यु॑भ॒यतः॑ - वै॒श्वा॒न॒रः॒ । गृ॒ह्य॒ते॒ । तस्मा᳚त् । उ॒भ॒यतः॑ । प्रा॒णा इति॑ प्र - अ॒नाः । अ॒धस्ता᳚त् । च॒ । उ॒परि॑ष्टात् । च॒ । अ॒र्द्धिनः॑ । अ॒न्ये । ग्रहाः᳚ । गृ॒ह्यन्ते᳚ । अ॒द्‌र्धी । ध्रु॒वः । तस्मा᳚त् । \textbf{  6} \newline
                  \newline
                                \textbf{ TS 6.5.2.2} \newline
                  अ॒र्द्धि । अवाङ्॑ । प्रा॒ण इति॑ प्र - अ॒नः । अ॒न्येषा᳚म् । प्रा॒णाना॒मिति॑ प्र - अ॒नाना᳚म् । उपो᳚प्त॒ इत्युप॑ - उ॒प्ते॒ । अ॒न्ये । ग्रहाः᳚ । सा॒द्यन्ते᳚ । अनु॑पोप्त॒ इत्यनु॑प - उ॒प्ते॒ । ध्रु॒वः । तस्मा᳚त् । अ॒स्थ्ना । अ॒न्याः । प्र॒जा इति॑ प्र - जाः । प्र॒ति॒तिष्ठ॒न्तीति॑ प्रति - तिष्ठ॑न्ति । माꣳ॒॒सेन॑ । अ॒न्याः । असु॑राः । वै । उ॒त्त॒र॒त इत्यु॑त् - त॒र॒तः । पृ॒थि॒वीम् । प॒र्याचि॑कीर्.ष॒न्निति॑ परि - आचि॑कीर्.षन्न् । ताम् । दे॒वाः । ध्रु॒वेण॑ । अ॒दृ॒ह॒न्न् । तत् । ध्रु॒वस्य॑ । ध्रु॒व॒त्वमिति॑ ध्रुव - त्वम् । यत् । ध्रु॒वः । उ॒त्त॒र॒त इत्यु॑त् - त॒र॒तः । सा॒द्यते᳚ । धृत्यै᳚ । आयुः॑ । वै । ए॒तत् । य॒ज्ञ्स्य॑ । यत् । ध्रु॒वः । आ॒त्मा । होता᳚ । यत् । हो॒तृ॒च॒म॒स इति॑ होतृ - च॒म॒से । ध्रु॒वम् । अ॒व॒नय॒तीत्य॑व - नय॑ति । आ॒त्मन्न् । ए॒व । य॒ज्ञ्स्य॑ । \textbf{  7} \newline
                  \newline
                                \textbf{ TS 6.5.2.3} \newline
                  आयुः॑ । द॒धा॒ति॒ । पु॒रस्ता᳚त् । उ॒क्थस्य॑ । अ॒व॒नीय॒ इत्य॑व - नीयः॑ । इति॑ । आ॒हुः॒ । पु॒रस्ता᳚त् । हि । आयु॑षः । भु॒ङ्क्ते । म॒द्ध्य॒तः । अ॒व॒नीय॒ इत्य॑व - नीयः॑ । इति॑ । आ॒हुः॒ । म॒द्ध्य॒मेन॑ । हि । आयु॑षः । भु॒ङ्क्ते । उ॒त्त॒रा॒द्‌र्ध इत्यु॑त्तर - अ॒द्‌र्धे । अ॒व॒नीय॒ इत्य॑व - नीयः॑ । इति॑ । आ॒हुः॒ । उ॒त्त॒मेनेत्यु॑त् - त॒मेन॑ । हि । आयु॑षः । भु॒ङ्क्ते । वै॒श्व॒दे॒व्यामिति॑ वैश्व - दे॒व्याम् । ऋ॒चि । श॒स्यमा॑नायाम् । अवेति॑ । न॒य॒ति॒ । वै॒श्व॒दे॒व्य॑ इति॑ वैश्व - दे॒व्यः॑ । वै । प्र॒जा इति॑ प्र - जाः । प्र॒जास्विति॑ प्र - जासु॑ । ए॒व । आयुः॑ । द॒धा॒ति॒ ॥ \textbf{  8} \newline
                  \newline
                      (ध्रु॒वस्तस्मा॑ - दे॒व य॒ज्ञ्स्यै - का॒न्नच॑त्वारिꣳ॒॒शच्च॑)  \textbf{(A2)} \newline \newline
                                \textbf{ TS 6.5.3.1} \newline
                  य॒ज्ञेन॑ । वै । दे॒वाः । सु॒व॒र्गमिति॑ सुवः - गम् । लो॒कम् । आ॒य॒न्न् । ते । अ॒म॒न्य॒न्त॒ । म॒नु॒ष्याः᳚ । नः॒ । अ॒न्वाभ॑विष्य॒न्तीत्य॑नु-आभ॑विष्यन्ति । इति॑ । ते । सं॒ॅव॒थ्स॒रे॒णेति॑ सं - व॒थ्स॒रेण॑ । यो॒प॒यि॒त्वा । सु॒व॒र्गमिति॑ सुवः - गम् । लो॒कम् । आ॒य॒न्न् । तम् । ऋष॑यः । ऋ॒तु॒ग्र॒हैरित्यृ॑तु - ग्र॒हैः । ए॒व । अनु॑ । प्रेति॑ । अ॒जा॒न॒न्न् । यत् । ऋ॒तु॒ग्र॒हा इत्यृ॑तु - ग्र॒हाः । गृ॒ह्यन्ते᳚ । सु॒व॒र्गसेति॑ सुवः - गस्य॑ । लो॒कस्य॑ । प्रज्ञा᳚त्या॒ इति॒ प्र - ज्ञा॒त्यै॒ । द्वाद॑श । गृ॒ह्य॒न्ते॒ । द्वाद॑श । मासाः᳚ । सं॒ॅव॒थ्स॒र इति॑ सं - व॒थ्स॒रः । सं॒ॅव॒थ्स॒रस्येति॑ सं - व॒थ्स॒रस्य॑ । प्रज्ञा᳚त्या॒ इति॒ प्र - ज्ञा॒त्यै॒ । स॒ह । प्र॒थ॒मौ । गृ॒ह्ये॒ते॒ इति॑ । स॒ह । उ॒त्त॒मावितु॑त् - त॒मौ । तस्मा᳚त् । द्वौद्वा॒विति॒ द्वौ - द्वौ॒ । ऋ॒तू इति॑ । उ॒भ॒यतो॑मुख॒मित्यु॑भ॒यतः॑ - मु॒ख॒म् । ऋ॒तु॒पा॒त्रमित्यृ॑तु - पा॒त्रम् । भ॒व॒ति॒ । कः । \textbf{  9} \newline
                  \newline
                                \textbf{ TS 6.5.3.2} \newline
                  हि । तत् । वेद॑ । यतः॑ । ऋ॒तू॒नाम् । मुख᳚म् । ऋ॒तुना᳚ । प्रेति॑ । इ॒ष्य॒ । इति॑ । षट् । कृत्वः॑ । आ॒ह॒ । षट् । वै । ऋ॒तवः॑ । ऋ॒तून् । ए॒व । प्री॒णा॒ति॒ । ऋ॒तुभि॒रित्यृ॒तु - भिः॒ । इति॑ । च॒तुः । चतु॑ष्पद॒ इति॒ चतुः॑ - प॒दः॒ । ए॒व । प॒शून् । प्री॒णा॒ति॒ । द्विः । पुनः॑ । ऋ॒तुना᳚ । आ॒ह॒ । द्वि॒पद॒ इति॑ द्वि - पदः॑ । ए॒व । प्री॒णा॒ति॒ । ऋ॒तुना᳚ । प्रेति॑ । इ॒ष्य॒ । इति॑ । षट् । कृत्वः॑ । आ॒ह॒ । ऋ॒तुभि॒रित्यृ॒तु - भिः॒ । इति॑ । च॒तुः । तस्मा᳚त् । चतु॑ष्पाद॒ इति॒ चतुः॑ - पा॒दः॒ । प॒शवः॑ । ऋ॒तून् । उपेति॑ । जी॒व॒न्ति॒ । द्विः । \textbf{  10} \newline
                  \newline
                                \textbf{ TS 6.5.3.3} \newline
                  पुनः॑ । ऋ॒तुना᳚ । आ॒ह॒ । तस्मा᳚त् । द्वि॒पाद॒ इति॑ द्वि - पादः॑ । चतु॑ष्पद॒ इति॒ चतुः॑ - प॒दः॒ । प॒शून् । उपेति॑ । जी॒व॒न्ति॒ । ऋ॒तुना᳚ । प्रेति॑ । इ॒ष्य॒ । इति॑ । षट् । कृत्वः॑ । आ॒ह॒ । ऋ॒तुभि॒रित्यृ॒तु - भिः॒ । इति॑ । च॒तुः । द्विः । पुनः॑ । ऋ॒तुना᳚ । आ॒ह॒ । आ॒क्रम॑ण॒मित्या᳚ - क्रम॑णम् । ए॒व । तत् । सेतु᳚म् । यज॑मानः । कु॒रु॒ते॒ । सु॒व॒र्गसेति॑ सुवः - गस्य॑ । लो॒कस्य॑ । सम॑ष्ट्या॒ इति॒ सं - अ॒ष्ट्यै॒ । न । अ॒न्यः । अ॒न्यम् । अनु॑ । प्रेति॑ । प॒द्ये॒त॒ । यत् । अ॒न्यः । अ॒न्यम् । अ॒नु॒प्र॒पद्ये॒तेत्य॑नु - प्र॒पद्ये॑त । ऋ॒तुः । ऋ॒तुम् । अनु॑ । प्रेति॑ । प॒द्ये॒त॒ । ऋ॒तवः॑ । मोहु॑काः । स्युः॒ । \textbf{  11} \newline
                  \newline
                                \textbf{ TS 6.5.3.4} \newline
                  प्रसि॑द्ध॒मिति॒ प्र - सि॒द्ध॒म् । ए॒व । अ॒द्ध्व॒र्युः । दक्षि॑णेन । प्रेति॑ । प॒द्य॒ते॒ । प्रसि॑द्ध॒मिति॒ प्र - सि॒द्ध॒म् । प्र॒ति॒प्र॒स्था॒तेति॑ प्रति-प्र॒स्था॒ता । उत्त॑रे॒णेत्युत्-त॒रे॒ण॒ । तस्मा᳚त् । आ॒दि॒त्यः । षट् । मा॒सः । दक्षि॑णेन । ए॒ति॒ । षट् । उत्त॑रे॒णेत्युत्-त॒रे॒ण॒ । उ॒प॒या॒मगृ॑हीत॒ इत्यु॑पया॒म-गृ॒ही॒तः॒ । अ॒सि॒ । सꣳ॒॒सर्प॒ इति॑ सं - सर्पः॑ । अ॒सि॒ । अꣳ॒॒ह॒स्प॒त्यायेत्यꣳ॑हः - प॒त्याय॑ । त्वा॒ । इति॑ । आ॒ह॒ । अस्ति॑ । त्र॒यो॒द॒श इति॑ त्रयः - द॒शः । मासः॑ । इति॑ । आ॒हुः॒ । तम् । ए॒व । तत् । प्री॒णा॒ति॒ ॥ \textbf{  12} \newline
                  \newline
                      (को - जी॑वन्ति॒ द्विः - स्यु॒ - श्चतु॑स्त्रिꣳशच्च)  \textbf{(A3)} \newline \newline
                                \textbf{ TS 6.5.4.1} \newline
                  सु॒व॒र्गायेति॑ सुवः - गाय॑ । वै । ए॒ते । लो॒काय॑ । गृ॒ह्य॒न्ते॒ । यत् । ऋ॒तु॒ग्र॒हा इत्यृ॑तु - ग्र॒हाः । ज्योतिः॑ । इ॒न्द्रा॒ग्नी इती᳚न्द्र - अ॒ग्नी । यत् । ऐ॒न्द्रा॒ग्नमित्यै᳚न्द्र - अ॒ग्नम् । ऋ॒तु॒पा॒त्रेणेत्यृ॑तु - पा॒त्रेण॑ । गृ॒ह्णाति॑ । ज्योतिः॑ । ए॒व । अ॒स्मै॒ । उ॒परि॑ष्टात् । द॒धा॒ति॒ । सु॒व॒र्गसेति॑ सुवः-गस्य॑ । लो॒कस्य॑ । अनु॑ख्यात्या॒ इत्यनु॑ - ख्या॒त्यै॒ । ओ॒जो॒भृता॒वित्यो॑जः - भृतौ᳚ । वै । ए॒तौ । दे॒वाना᳚म् । यत् । इ॒न्द्रा॒ग्नी इती᳚न्द्र - अ॒ग्नी । यत् । ऐ॒न्द्रा॒ग्न इत्यै᳚न्द्र-अ॒ग्नः । गृ॒ह्यते᳚ । ओजः॑ । ए॒व । अवेति॑ । रु॒न्धे॒ । वै॒श्व॒दे॒वमिति॑ वैश्व - दे॒वम् । शु॒क्र॒पा॒त्रेणेति॑ शुक्र - पा॒त्रेण॑ । गृ॒ह्णा॒ति॒ । वै॒श्व॒दे॒व्य॑ इति॑ वैश्व - दे॒व्यः॑ । वै । प्र॒जा इति॑ प्र - जाः । अ॒सौ । आ॒दि॒त्यः । शु॒क्रः । यत् । वै॒श्व॒दे॒वमिति॑ वैश्व - दे॒वम् । शु॒क्र॒पा॒त्रेणेति॑ शुक्र - पा॒त्रेण॑ । गृ॒ह्णाति॑ । तस्मा᳚त् । अ॒सौ । आ॒दि॒त्यः । \textbf{  13} \newline
                  \newline
                                \textbf{ TS 6.5.4.2} \newline
                  सर्वाः᳚ । प्र॒जा इति॑ प्र - जाः । प्र॒त्यङ् । उदिति॑ । ए॒ति॒ । तस्मा᳚त् । सर्वः॑ । ए॒व । म॒न्य॒ते॒ । माम् । प्रति॑ । उदिति॑ । अ॒गा॒त् । इति॑ । वै॒श्व॒दे॒वमिति॑ वैश्व - दे॒वम् । शु॒क्र॒पा॒त्रेणेति॑ शुक्र-पा॒त्रेण॑ । गृ॒ह्णा॒ति॒ । वै॒श्व॒दे॒व्य॑ इति॑ वैश्व - दे॒व्यः॑ । वै । प्र॒जा इति॑ प्र - जाः । तेजः॑ । शु॒क्रः । यत् । वै॒श्व॒दे॒वमिति॑ वैश्व - दे॒वम् । शु॒क्र॒पा॒त्रेणेति॑ शुक्र - पा॒त्रेण॑ । गृ॒ह्णाति॑ । प्र॒जास्विति॑ प्र - जासु॑ । ए॒व । तेजः॑ । द॒धा॒ति॒ ॥ \textbf{  14} \newline
                  \newline
                      (तस्मा॑द॒सावा॑दि॒त्य - स्त्रिꣳ॒॒शच्च॑)  \textbf{(A4)} \newline \newline
                                \textbf{ TS 6.5.5.1} \newline
                  इन्द्रः॑ । म॒रुद्भि॒रिति॑ म॒रुत् - भिः॒ । सांॅवि॑द्ये॒नेति॒ सां - वि॒द्ये॒न॒ । माद्ध्य॑न्दिने । सव॑ने । वृ॒त्रम् । अ॒ह॒न्न् । यत् । माद्ध्य॑न्दिने । सव॑ने । म॒रु॒त्व॒तीयाः᳚ । गृ॒ह्यन्ते᳚ । वार्त्र॑घ्ना॒ इति॒ वार्त्र॑ - घ्नाः॒ । ए॒व । ते । यज॑मानस्य । गृ॒ह्य॒न्ते॒ । तस्य॑ । वृ॒त्रम् । ज॒घ्नुषः॑ । ऋ॒तवः॑ । अ॒मु॒ह्य॒न्न् । सः । ऋ॒तु॒पा॒त्रेणेत्यृ॑तु - पा॒त्रेण॑ । म॒रु॒त्व॒तीयान्॑ । अ॒गृ॒ह्णा॒त् । ततः॑ । वै । सः । ऋ॒तून् । प्रेति॑ । अ॒जा॒ना॒त् । यत् । ऋ॒तु॒पा॒त्रेणेत्यृ॑तु-पा॒त्रेण॑ । म॒रु॒त्व॒तीयाः᳚ । गृ॒ह्यन्ते᳚ । ऋ॒तू॒नाम् । प्रज्ञा᳚त्या॒ इति॒ प्र-ज्ञा॒त्यै॒ । वज्र᳚म् । वै । ए॒तम् । यज॑मानः । भ्रातृ॑व्याय । प्रेति॑ । ह॒र॒ति॒ । यत् । म॒रु॒त्व॒तीयाः᳚ । उदिति॑ । ए॒व । प्र॒थ॒मेन॑ । \textbf{  15} \newline
                  \newline
                                \textbf{ TS 6.5.5.2} \newline
                  य॒च्छ॒ति॒ । प्रेति॑ । ह॒र॒ति॒ । द्वि॒तीये॑न । स्तृ॒णु॒ते । तृ॒तीये॑न । आयु॑धम् । वै । ए॒तत् । यज॑मानः । समिति॑ । कु॒रु॒ते॒ । यत् । म॒रु॒त्व॒तीयाः᳚ । धनुः॑ । ए॒व । प्र॒थ॒मः । ज्या । द्वि॒तीयः॑ । इषुः॑ । तृ॒तीयः॑ । प्रतीति॑ । ए॒व । प्र॒थ॒मेन॑ । ध॒त्ते॒ । वीति॑ । सृ॒ज॒ति॒ । द्वि॒तीये॑न । विद्ध्य॑ति । तृ॒तीये॑न । इन्द्रः॑ । वृ॒त्रम् । ह॒त्वा । परा᳚म् । प॒रा॒वत॒मिति॑ परा - वत᳚म् । अ॒ग॒च्छ॒त् । अपेति॑ । अ॒रा॒ध॒म् । इति॑ । मन्य॑मानः । सः । हरि॑तः । अ॒भ॒व॒त् । सः । ए॒तान् । म॒रु॒त्व॒तीयान्॑ । आ॒त्म॒स्पर॑णा॒नित्या᳚त्म - स्पर॑णान् । अ॒प॒श्य॒त् । तान् । अ॒गृ॒ह्णी॒त॒ । \textbf{  16} \newline
                  \newline
                                \textbf{ TS 6.5.5.3} \newline
                  प्रा॒णमिति॑ प्र - अ॒नम् । ए॒व । प्र॒थ॒मेन॑ । अ॒स्पृ॒णु॒त॒ । अ॒पा॒नमित्य॑प - अ॒नम् । द्वि॒तीये॑न । आ॒त्मान᳚म् । तृ॒तीये॑न । आ॒त्म॒स्पर॑णा॒ इत्या᳚त्म - स्पर॑णाः । वै । ए॒ते । यज॑मानस्य । गृ॒ह्य॒न्ते॒ । यत् । म॒रु॒त्व॒तीयाः᳚ । प्रा॒णमिति॑ प्र - अ॒नम् । ए॒व । प्र॒थ॒मेन॑ । स्पृ॒णु॒ते॒ । अ॒पा॒नमित्य॑प-अ॒नम् । द्वि॒तीये॑न । आ॒त्मान᳚म् । तृ॒तीये॑न । इन्द्रः॑ । वृ॒त्रम् । अ॒ह॒न्न् । तम् । दे॒वाः । अ॒ब्रु॒व॒न्न् । म॒हान् । वै । अ॒यम् । अ॒भू॒त् । यः । वृ॒त्रम् । अव॑धीत् । इति॑ । तत् । म॒हे॒न्द्रस्येति॑ महा - इ॒न्द्रस्य॑ । म॒हे॒न्द्र॒त्वमिति॑ महेन्द्र - त्वम् । सः । ए॒तम् । मा॒हे॒न्द्रमिति॑ माहा - इ॒न्द्रम् । उ॒द्धा॒रमित्यु॑त् - हा॒रम् । उदिति॑ । अ॒ह॒र॒त॒ । वृ॒त्रम् । ह॒त्वा । अ॒न्यासु॑ । दे॒वता॑सु ( ) । अधीति॑ । यत् । मा॒हे॒न्द्र इति॑ माहा - इ॒न्द्रः । गृ॒ह्यते᳚ । उ॒द्धा॒रमित्यु॑त् - हा॒रम् । ए॒व । तम् । यज॑मानः । उदिति॑ । ह॒र॒ते॒ । अ॒न्यासु॑ । प्र॒जास्विति॑ प्र-जासु॑ । अधीति॑ । शु॒क्र॒पा॒त्रेणेति॑ शुक्र - पा॒त्रेण॑ । गृ॒ह्णा॒ति॒ । य॒ज॒मा॒न॒दे॒व॒त्य॑ इति॑ यजमान - दे॒व॒त्यः॑ । वै । मा॒हे॒न्द्र इति॑ माहा - इ॒न्द्रः । तेजः॑ । शु॒क्रः । यत् । मा॒हे॒न्द्रमिति॑ माहा - इ॒न्द्रम् । शु॒क्र॒पा॒त्रेणेति॑ शुक्र - पा॒त्रेण॑ । गृ॒ह्णाति॑ । यज॑माने । ए॒व । तेजः॑ । द॒धा॒ति॒ ॥ \textbf{  17} \newline
                  \newline
                      (प्र॒थ॒मेना॑ - गृह्णीत - दे॒वता᳚स्व॒ - ष्टाविꣳ॑शतिश्च)  \textbf{(A5)} \newline \newline
                                \textbf{ TS 6.5.6.1} \newline
                  अदि॑तिः । पु॒त्रका॒मेति॑ पु॒त्र - का॒मा॒ । सा॒द्ध्येभ्यः॑ । दे॒वेभ्यः॑ । ब्र॒ह्मौ॒द॒नमिति॑ ब्रह्म - ओ॒द॒नम् । अ॒प॒च॒त् । तस्यै᳚ । उ॒च्छेष॑ण॒मित्यु॑त्-शेष॑णम् । अ॒द॒दुः॒ । तत् । प्रेति॑ । अ॒श्ना॒त् । सा । रेतः॑ । अ॒ध॒त्त॒ । तस्यै᳚ । च॒त्वारः॑ । आ॒दि॒त्याः । अ॒जा॒य॒न्त॒ । सा । द्वि॒तीय᳚म् । अ॒प॒च॒त् । सा । अ॒म॒न्य॒त॒ । उ॒च्छेष॑णा॒दित्यु॑त् - शेष॑णात् । मे॒ । इ॒मे । अ॒ज्ञ्॒त॒ । यत् । अग्रे᳚ । प्रा॒शि॒ष्यामीति॑ प्र - अ॒शि॒ष्यामि॑ । इ॒तः । मे॒ । वसी॑याꣳसः । ज॒नि॒ष्य॒न्ते॒ । इति॑ । सा । अग्रे᳚ । प्रेति॑ । अ॒श्ना॒त् । सा । रेतः॑ । अ॒ध॒त्त॒ । तस्यै᳚ । व्यृ॑द्ध॒मिति॒ वि - ऋ॒द्ध॒म् । आ॒ण्डम् । अ॒जा॒य॒त॒ । सा । आ॒दि॒त्येभ्यः॑ । ए॒व । \textbf{  18} \newline
                  \newline
                                \textbf{ TS 6.5.6.2} \newline
                  तृ॒तीय᳚म् । अ॒प॒च॒त् । भोगा॑य । मे॒ । इ॒दम् । श्रा॒न्तम् । अ॒स्तु॒ । इति॑ । ते । अ॒ब्रु॒व॒न्न् । वर᳚म् । वृ॒णा॒म॒है॒ । यः । अतः॑ । जाया॑तै । अ॒स्माक᳚म् । सः । एकः॑ । अ॒स॒त् । यः । अ॒स्य॒ । प्र॒जाया॒मिति॑ प्र - जाया᳚म् । ऋद्ध्या॑तै । अ॒स्माक᳚म् । भोगा॑य । भ॒वा॒त् । इति॑ । ततः॑ । विव॑स्वान् । आ॒दि॒त्यः । अ॒जा॒य॒त॒ । तस्य॑ । वै । इ॒यम् । प्र॒जेति॑ प्र - जा । यत् । म॒नु॒ष्याः᳚ । तासु॑ । एकः॑ । ए॒व । ऋ॒द्धः । यः । यज॑ते । सः । दे॒वाना᳚म् । भोगा॑य । भ॒व॒ति॒ । दे॒वाः । वै । य॒ज्ञात् । \textbf{  19} \newline
                  \newline
                                \textbf{ TS 6.5.6.3} \newline
                  रु॒द्रम् । अ॒न्तः । आ॒य॒न्न् । सः । आ॒दि॒त्यान् । अ॒न्वाक्र॑म॒तेत्य॑नु - आक्र॑मत । ते । द्वि॒दे॒व॒त्या॑निति॑ द्वि - दे॒व॒त्यान्॑ । प्रेति॑ । अ॒प॒द्य॒न्त॒ । तान् । न । प्रति॑ । प्रेति॑ । अ॒य॒च्छ॒न्न् । तस्मा᳚त् । अपीति॑ । वद्ध्य᳚म् । प्रप॑न्न॒मिति॒ प्र - प॒न्न॒म् । न । प्रति॑ । प्रेति॑ । य॒च्छ॒न्ति॒ । तस्मा᳚त् । द्वि॒दे॒व॒त्ये᳚भ्य॒ इति॑ द्वि - दे॒व॒त्ये᳚भ्यः । आ॒दि॒त्यः । निरिति॑ । गृ॒ह्य॒ते॒ । यत् । उ॒च्छेष॑णा॒दित्यु॑त् - शेष॑णात् । अजा॑यन्त । तस्मा᳚त् । उ॒च्छेष॑णा॒दित्यु॑त् - शेष॑णात् । गृ॒ह्य॒ते॒ । ति॒सृभि॒रिति॑ ति॒सृ - भिः॒ । ऋ॒ग्भिरित्यृ॑क् - भिः । गृ॒ह्णा॒ति॒ । मा॒ता । पि॒ता । पु॒त्रः । तत् । ए॒व । तत् । मि॒थु॒नम् । उल्ब᳚म् । गर्भः॑ । ज॒रायु॑ । तत् । ए॒व । तत् । \textbf{  20} \newline
                  \newline
                                \textbf{ TS 6.5.6.4} \newline
                  मि॒थु॒नम् । प॒शवः॑ । वै । ए॒ते । यत् । आ॒दि॒त्यः । ऊर्क् । दधि॑ । द॒द्ध्ना । म॒द्ध्य॒तः । श्री॒णा॒ति॒ । ऊर्ज᳚म् । ए॒व । प॒शू॒नाम् । म॒द्ध्य॒तः । द॒धा॒ति॒ । शृ॒ता॒त॒ङ्क्ये॑नेति॑ शृत - आ॒त॒ङ्क्ये॑न । मे॒द्ध्य॒त्वायेति॑ मेद्ध्य - त्वाय॑ । तस्मा᳚त् । आ॒मा । प॒क्वम् । दु॒हे॒ । प॒शवः॑ । वै । ए॒ते । यत् । आ॒दि॒त्यः । प॒रि॒श्रित्येति॑ परि - श्रित्य॑ । गृ॒ह्णा॒ति॒ । प्र॒ति॒रुद्ध्येति॑ प्रति - रुद्ध्य॑ । ए॒व । अ॒स्मै॒ । प॒शून् । गृ॒ह्णा॒ति॒ । प॒शवः॑ । वै । ए॒ते । यत् । आ॒दि॒त्यः । ए॒षः । रु॒द्रः । यत् । अ॒ग्निः । प॒रि॒श्रित्येति॑ परि - श्रित्य॑ । गृ॒ह्णा॒ति॒ । रु॒द्रात् । ए॒व । प॒शून् । अ॒न्तः । द॒धा॒ति॒ । \textbf{  21} \newline
                  \newline
                                \textbf{ TS 6.5.6.5} \newline
                  ए॒षः । वै । विव॑स्वान् । आ॒दि॒त्यः । यत् । उ॒पाꣳ॒॒शु॒सव॑न॒ इत्यु॑पाꣳशु - सव॑नः । सः । ए॒तम् । ए॒व । सो॒म॒पी॒थमिति॑ सोम - पी॒थम् । परीति॑ । श॒ये॒ । एति॑ । तृ॒ती॒य॒स॒व॒नादिति॑ तृतीय-स॒व॒नात् । विव॑स्वः । आ॒दि॒त्य॒ । ए॒षः । ते॒ । सो॒म॒पी॒थ इति॑ सोम - पी॒थः । इति॑ । आ॒ह॒ । विव॑स्वन्तम् । ए॒व । आ॒दि॒त्यम् । सो॒म॒पी॒थेनेति॑ सोम - पी॒थेन॑ । समिति॑ । अ॒द्‌र्ध॒य॒ति॒ । या । दि॒व्या । वृष्टिः॑ । तया᳚ । त्वा॒ । श्री॒णा॒मि॒ । इति॑ । वृष्टि॑काम॒स्येति॒ वृष्टि॑ - का॒म॒स्य॒ । श्री॒णी॒या॒त् । वृष्टि᳚म् । ए॒व । अवेति॑ । रु॒न्धे॒ । यदि॑ । ता॒जक् । प्र॒स्कन्दे॒दिति॑ प्र - स्कन्दे᳚त् । वर्.षु॑कः । प॒र्जन्यः॑ । स्या॒त् । यदि॑ । चि॒रम् । अव॑र्.षुकः । न ( ) । सा॒द॒य॒ति॒ । अस॑न्नात् । हि । प्र॒जा इति॑ प्र - जाः । प्र॒जाय॑न्त॒ इति॑ प्र - जाय॑न्ते । न । अन्विति॑ । वष॑ट् । क॒रो॒ति॒ । यत् । अ॒नु॒व॒ष॒ट्कु॒र्यादित्य॑नु - व॒ष॒ट्कु॒र्यात् । रु॒द्रम् । प्र॒जा इति॑ प्र - जाः । अ॒न्वव॑सृजे॒दित्य॑नु - अव॑सृजेत् । न । हु॒त्वा । अन्विति॑ । ई॒क्षे॒त॒ । यत् । अ॒न्वीक्षे॒तेत्य॑नु - ईक्षे॑त । चक्षुः॑ । अ॒स्य॒ । प्र॒मायु॑क॒मिति॑ प्र - मायु॑कम् । स्या॒त् । तस्मा᳚त् । न । अ॒न्वीक्ष्य॒ इत्य॑नु - ईक्ष्यः॑ ॥ \textbf{  22} \newline
                  \newline
                      (ए॒व - य॒ज्ञा - ज्ज॒रायु॒ तदे॒व तद॒ - न्तर्द॑धाति॒ - न - स॒प्तविꣳ॑शतिश्च)  \textbf{(A6)} \newline \newline
                                \textbf{ TS 6.5.7.1} \newline
                  अ॒न्त॒र्या॒म॒पा॒त्रेणेत्य॑न्तर्याम - पा॒त्रेण॑ । सा॒वि॒त्रम् । आ॒ग्र॒य॒णात् । गृ॒ह्णा॒ति॒ । प्र॒जाप॑ति॒रिति॑ प्र॒जा-प॒तिः॒ । वै । ए॒षः । यत् । आ॒ग्र॒य॒णः । प्र॒जाना॒मिति॑ प्र - जाना᳚म् । प्र॒जन॑ना॒येति॑ प्र - जन॑नाय । न । सा॒द॒य॒ति॒ । अस॑न्नात् । हि । प्र॒जा इति॑ प्र - जाः । प्र॒जाय॑न्त॒ इति॑ प्र - जाय॑न्ते । न । अन्विति॑ । वष॑ट् । क॒रो॒ति॒ । यत् । अ॒नु॒व॒ष॒ट्कु॒र्यादित्य॑नु - व॒ष॒ट्कु॒र्यात् । रु॒द्रम् । प्र॒जा इति॑ प्र - जाः । अ॒न्वव॑सृजे॒दित्य॑नु - अव॑सृजेत् । ए॒षः । वै । गा॒य॒त्रः । दे॒वाना᳚म् । यत् । स॒वि॒ता । ए॒षः । गा॒य॒त्रि॒यै । लो॒के । गृ॒ह्य॒ते॒ । यत् । आ॒ग्र॒य॒णः । यत् । अ॒न्त॒र्या॒म॒पा॒त्रेणेत्य॑न्तर्याम - पा॒त्रेण॑ । सा॒वि॒त्रम् । आ॒ग्र॒य॒णात् । गृ॒ह्णाति॑ । स्वात् । ए॒व । ए॒न॒म् । योनेः᳚ । निरिति॑ । गृ॒ह्णा॒ति॒ । विश्वे᳚ । \textbf{  23} \newline
                  \newline
                                \textbf{ TS 6.5.7.2} \newline
                  दे॒वाः । तृ॒तीय᳚म् । सव॑नम् । न । उदिति॑ । अ॒य॒च्छ॒न् । ते । स॒वि॒तार᳚म् । प्रा॒त॒स्स॒व॒नभा॑ग॒मिति॑ प्रातस्सव॒न - भा॒ग॒म् । सन्त᳚म् । तृ॒ती॒य॒स॒व॒नमिति॑ तृतीय-स॒व॒नम् । अ॒भि । परीति॑ । अ॒न॒य॒न्न् । ततः॑ । वै । ते । तृ॒तीय᳚म् । सव॑नम् । उदिति॑ । अ॒य॒च्छ॒न्न् । यत् । तृ॒ती॒य॒स॒व॒न इति॑ तृतीय - स॒व॒ने । सा॒वि॒त्रः । गृ॒ह्यते᳚ । तृ॒तीय॑स्य । सव॑नस्य । उद्य॑त्या॒ इत्युत् - य॒त्यै॒ । स॒वि॒तृ॒पा॒त्रेणेति॑ सवितृ-पा॒त्रेण॑ । वै॒श्व॒दे॒वमिति॑ वैश्व - दे॒वम् । क॒लशा᳚त् । गृ॒ह्णा॒ति॒ । वै॒श्व॒दे॒व्य॑ इति॑ वैश्व - दे॒व्यः॑ । वै । प्र॒जा इति॑ प्र-जाः । वै॒श्व॒दे॒व इति॑ वैश्व-दे॒वः । क॒लशः॑ । स॒वि॒ता । प्र॒स॒वाना॒मिति॑ प्र - स॒वाना᳚म् । ई॒शे॒ । यत् । स॒वि॒तृ॒पा॒त्रेणेति॑ सवितृ - पा॒त्रेण॑ । वै॒श्व॒दे॒वमिति॑ वैश्व - दे॒वम् । क॒लशा᳚त् । गृ॒ह्णाति॑ । स॒वि॒तृप्र॑सूत॒ इति॑ सवि॒तृ - प्र॒सू॒तः॒ । ए॒व । अ॒स्मै॒ । प्र॒जा इति॑ प्र - जाः । प्रेति॑ । \textbf{  24} \newline
                  \newline
                                \textbf{ TS 6.5.7.3} \newline
                  ज॒न॒य॒ति॒ । सोमे᳚ । सोम᳚म् । अ॒भीति॑ । गृ॒ह्णा॒ति॒ । रेतः॑ । ए॒व । तत् । द॒धा॒ति॒ । सु॒शर्मेति॑ सु - शर्मा᳚ । अ॒सि॒ । सु॒प्र॒ति॒ष्ठा॒न इति॑ सु - प्र॒ति॒ष्ठा॒नः । इति॑ । आ॒ह॒ । सोमे᳚ । हि । सोम᳚म् । अ॒भि॒गृ॒ह्णातीत्य॑भि - गृ॒ह्णाति॑ । प्रति॑ष्ठित्या॒ इति॒ प्रति॑ - स्थि॒त्यै॒ । ए॒तस्मिन्न्॑ । वै । अपीति॑ । ग्रहे᳚ । म॒नु॒ष्ये᳚भ्यः । दे॒वेभ्यः॑ । पि॒तृभ्य॒ इति॑ पि॒तृ - भ्यः॒ । क्रि॒य॒ते॒ । सु॒शर्मेति॑ सु - शर्मा᳚ । अ॒सि॒ । सु॒प्र॒ति॒ष्ठा॒न इति॑ सु - प्र॒ति॒ष्ठा॒नः । इति॑ । आ॒ह॒ । म॒नु॒ष्ये᳚भ्यः । ए॒व । ए॒तेन॑ । क॒रो॒ति॒ । बृ॒हत् । इति॑ । आ॒ह॒ । दे॒वेभ्यः॑ । ए॒व । ए॒तेन॑ । क॒रो॒ति॒ । नमः॑ । इति॑ । आ॒ह॒ । पि॒तृभ्य॒ इति॑ पि॒तृ - भ्यः॒ । ए॒व । ए॒तेन॑ । क॒रो॒ति॒ ( ) । ए॒ताव॑तीः । वै । दे॒वताः᳚ । ताभ्यः॑ । ए॒व । ए॒न॒म् । सर्वा᳚भ्यः । गृ॒ह्णा॒ति॒ । ए॒षः । ते॒ । योनिः॑ । विश्वे᳚भ्यः । त्वा॒ । दे॒वेभ्यः॑ । इति॑ । आ॒ह॒ । वै॒श्व॒दे॒व इति॑ वैश्व - दे॒वः । हि । ए॒षः ॥ \textbf{  25 } \newline
                  \newline
                      (विश्वे॒ - प्र - पि॒तृभ्य॑ ए॒वैतेन॑ करो॒त्ये - का॒न्नविꣳ॑श॒तिश्च॑)  \textbf{(A7)} \newline \newline
                                \textbf{ TS 6.5.8.1} \newline
                  प्रा॒ण इति॑ प्र - अ॒नः । वै । ए॒षः । यत् । उ॒पाꣳ॒॒शुरित्यु॑प-अꣳ॒॒शुः । यत् । उ॒पाꣳ॒॒शु॒पा॒त्रेणेत्यु॑पाꣳशु - पा॒त्रेण॑ । प्र॒थ॒मः । च॒ । उ॒त्त॒म इत्यु॑त् - त॒मः । च॒ । ग्रहौ᳚ । गृ॒ह्येते॒ इति॑ । प्रा॒णमिति॑ प्र - अ॒नम् । ए॒व । अन्विति॑ । प्र॒यन्तीति॑ प्र - यन्ति॑ । प्रा॒णमिति॑ प्र - अ॒नम् । अनु॑ । उदिति॑ । य॒न्ति॒ । प्र॒जाप॑ति॒रिति॑ प्र॒जा - प॒तिः॒ । वै । ए॒षः । यत् । आ॒ग्र॒य॒णः । प्रा॒ण इति॑ प्र - अ॒नः । उ॒पाꣳ॒॒शुरित्यु॑प-अꣳ॒॒शुः । पत्नीः᳚ । प्र॒जा इति॑ प्र - जाः । प्रेति॑ । ज॒न॒य॒न्ति॒ । यत् । उ॒पाꣳ॒॒शु॒पा॒त्रेणेत्यु॑पाꣳशु - पा॒त्रेण॑ । पा॒त्नी॒व॒तमिति॑ पात्नी - व॒तम् । आ॒ग्र॒य॒णात् । गृ॒ह्णाति॑ । प्र॒जाना॒मिति॑ प्र - जाना᳚म् । प्र॒जन॑ना॒येति॑ प्र - जन॑नाय । तस्मा᳚त् । प्रा॒णमिति॑ प्र - अ॒नम् । प्र॒जा इति॑ प्र-जाः । अनु॑ । प्रेति॑ । जा॒य॒न्ते॒ । दे॒वाः । वै । इ॒त इ॑त॒ इती॒तः - इ॒तः॒ । पत्नीः᳚ । सु॒व॒र्गमिति॑ सुवः-गम् । \textbf{  26} \newline
                  \newline
                                \textbf{ TS 6.5.8.2} \newline
                  लो॒कम् । अ॒जि॒गाꣳ॒॒स॒न्न् । ते । सु॒व॒र्गमिति॑ सुवः - गम् । लो॒कम् । न । प्रेति॑ । अ॒जा॒न॒न्न् । ते । ए॒तम् । पा॒त्नी॒व॒तमिति॑ पात्नी - व॒तम् । अ॒प॒श्य॒न्न् । तम् । अ॒गृ॒ह्ण॒त । ततः॑ । वै । ते । सु॒व॒र्गमिति॑ सुवः - गम् । लो॒कम् । प्रेति॑ । अ॒जा॒न॒न्न् । यत् । पा॒त्नी॒व॒त इति॑ पात्नी - व॒तः । गृ॒ह्यते᳚ । सु॒व॒र्गसेति॑ सुवः - गस्य॑ । लो॒कस्य॑ । प्रज्ञा᳚त्या॒ इति॒ प्र - ज्ञा॒त्यै॒ । सः । सोमः॑ । न । अ॒ति॒ष्ठ॒त॒ । स्त्री॒भ्यः । गृ॒ह्यमा॑णः । तम् । घृ॒तम् । वज्र᳚म् । कृ॒त्वा । अ॒घ्न॒न्न् । तम् । निरि॑न्द्रिय॒मिति॒ निः - इ॒न्द्रि॒य॒म् । भू॒तम् । अ॒गृ॒ह्ण॒न्न् । तस्मा᳚त् । स्त्रियः॑ । निरि॑न्द्रिया॒ इति॒ निः-इ॒न्द्रि॒याः॒ । अदा॑यादी॒रित्यदा॑य-अ॒दीः॒ । अपीति॑ । पा॒पात् । पुꣳ॒॒सः । उप॑स्तितर॒मित्युप॑स्ति - त॒र॒म् । \textbf{  27} \newline
                  \newline
                                \textbf{ TS 6.5.8.3} \newline
                  व॒द॒न्ति॒ । यत् । घृ॒तेन॑ । पा॒त्नी॒व॒तमिति॑ पात्नी - व॒तम् । श्री॒णाति॑ । वज्रे॑ण । ए॒व । ए॒न॒म् । वशे᳚ । कृ॒त्वा । गृ॒ह्णा॒ति॒ । उ॒प॒या॒मगृ॑हीत॒ इत्यु॑पया॒म - गृ॒ही॒तः॒ । अ॒सि॒ । इति॑ । आ॒ह॒ । इ॒यम् । वै । उ॒प॒या॒म इत्यु॑प - या॒मः । तस्मा᳚त् । इ॒माम् । प्र॒जा इति॑ प्र - जाः । अनु॑ । प्रेति॑ । जा॒य॒न्ते॒ । बृह॒स्पति॑सुत॒स्येति॒ बृह॒स्पति॑-सु॒त॒स्य॒ । ते॒ । इति॑ । आ॒ह॒ । ब्रह्म॑ । वै । दे॒वाना᳚म् । बृह॒स्पतिः॑ । ब्रह्म॑णा । ए॒व । अ॒स्मै॒ । प्र॒जा इति॑ प्र - जाः । प्रेति॑ । ज॒न॒य॒ति॒ । इ॒न्दो॒ इति॑ । इति॑ । आ॒ह॒ । रेतः॑ । वै । इन्दुः॑ । रेतः॑ । ए॒व । तत् । द॒धा॒ति॒ । इ॒न्द्रि॒या॒व॒ इती᳚न्द्रिय - वः॒ । इति॑ । \textbf{  28} \newline
                  \newline
                                \textbf{ TS 6.5.8.4} \newline
                  आ॒ह॒ । प्र॒जा इति॑ प्र - जाः । वै । इ॒न्द्रि॒यम् । प्र॒जा इति॑ प्र-जाः । ए॒व । अ॒स्मै॒ । प्रेति॑ । ज॒न॒य॒ति॒ । अग्ना(3) इ । इति॑ । आ॒ह॒ । अ॒ग्निः । वै । रे॒तो॒धा इति॑ रेतः - धाः । पत्नी॑व॒ इति॒ पत्नी᳚ - वः॒ । इति॑ । आ॒ह॒ । मि॒थु॒न॒त्वायेति॑ मिथुन-त्वाय॑ । स॒जूरिति॑ स - जूः । दे॒वेन॑ । त्वष्ट्रा᳚ । सोम᳚म् । पि॒ब॒ । इति॑ । आ॒ह॒ । त्वष्टा᳚ । वै । प॒शू॒नाम् । मि॒थु॒नाना᳚म् । रू॒प॒कृदिति॑ रूप - कृत् । रू॒पम् । ए॒व । प॒शुषु॑ । द॒धा॒ति॒ । दे॒वाः । वै । त्वष्टा॑रम् । अ॒जि॒घाꣳ॒॒स॒न्न् । सः । पत्नीः᳚ । प्रेति॑ । अ॒प॒द्य॒त॒ । तम् । न । प्रति॑ । प्रेति॑ । अ॒य॒च्छ॒न्न् । तस्मा᳚त् । अपीति॑ । \textbf{  29} \newline
                  \newline
                                \textbf{ TS 6.5.8.5} \newline
                  वद्ध्य᳚म् । प्रप॑न्न॒मिति॒ प्र - प॒न्न॒म् । न । प्रति॑ । प्रेति॑ । य॒च्छ॒न्ति॒ । तस्मा᳚त् । पा॒त्नी॒व॒त इति॑ पात्नी - व॒ते । त्वष्ट्रे᳚ । अपीति॑ । गृ॒ह्य॒ते॒ । न । सा॒द॒य॒ति॒ । अस॑न्नात् । हि । प्र॒जा इति॑ प्र - जाः । प्र॒जाय॑न्त॒ इति॑ प्र - जाय॑न्ते । न । अन्विति॑ । वष॑ट् । क॒रो॒ति॒ । यत् । अ॒नु॒व॒ष॒ट्कु॒र्यादित्य॑नु - व॒ष॒ट्कु॒र्यात् । रु॒द्रम् । प्र॒जा इति॑ प्र - जाः । अ॒न्वव॑सृजे॒दित्य॑नु - अव॑सृजेत् । यत् । न । अ॒नु॒व॒ष॒ट्कु॒र्यादित्य॑नु-व॒ष॒ट्कु॒र्यात् । अशा᳚न्तम् । अ॒ग्नीदित्य॑ग्नि-इत् । सोम᳚म् । भ॒क्ष॒ये॒त् ।  उ॒पाꣳ॒॒श्वित्यु॑प - अꣳ॒॒शु । अन्विति॑ । वष॑ट् । क॒रो॒ति॒ । न । रु॒द्रम् । प्र॒जा इति॑ प्र - जाः । अ॒न्व॒व॒सृ॒जतीत्य॑नु - अ॒व॒सृ॒जति॑ । शा॒न्तम् । अ॒ग्नीदित्य॑ग्नि - इत् । सोम᳚म् । भ॒क्ष॒य॒ति॒ । अग्नी॒दित्यग्नि॑ - इ॒त् । नेष्टुः॑ । उ॒पस्थ॒मित्यु॒प - स्थ॒म् । एति॑ । सी॒द॒ । \textbf{  30} \newline
                  \newline
                                \textbf{ TS 6.5.8.6} \newline
                  नेष्टः॑ । पत्नी᳚म् । उ॒दान॒येत्यु॑त् - आन॑य । इति॑ । आ॒ह॒ । अ॒ग्नीदित्य॑ग्नि-इत् । ए॒व । नेष्ट॑रि । रेतः॑ । दधा॑ति । नेष्टा᳚ । पत्नि॑याम् । उ॒द्गा॒त्रेत्यु॑त्-गा॒त्रा । समिति॑ । ख्या॒प॒य॒ति॒ । प्र॒जाप॑ति॒रिति॑ प्र॒जा-प॒तिः॒ । वै । ए॒षः । यत् । उ॒द्गा॒तेत्यु॑त् - गा॒ता । प्र॒जाना॒मिति॑ प्र - जाना᳚म् । प्र॒जन॑ना॒येति॑ प्र - जन॑नाय । अ॒पः । उप॑ । प्रेति॑ । व॒र्त॒य॒ति॒ । रेतः॑ । ए॒व । तत् । सि॒ञ्च॒ति॒ । ऊ॒रुणा᳚ । उप॑ । प्रेति॑ । व॒र्त॒य॒ति॒ । ऊ॒रुणा᳚ । हि । रेतः॑ । सि॒च्यते᳚ । न॒ग्न॒ङ्कृत्येति॑ नग्नम् - कृत्य॑ । ऊ॒रुम् । उप॑ । प्रेति॑ । व॒र्त॒य॒ति॒ । य॒दा । हि । न॒ग्नः । ऊ॒रुः । भव॑ति । अथ॑ । मि॒थु॒नी ( ) । भ॒व॒तः॒ । अथ॑ । रेतः॑ । सि॒च्य॒ते॒ । अथ॑ । प्र॒जा इति॑ प्र - जाः । प्रेति॑ । जा॒य॒न्ते॒ ॥ \textbf{  31} \newline
                  \newline
                      (पत्नीः᳚ सुव॒र्ग - मुप॑स्तितर - मिन्द्रियाव॒ इत्य - पि॑ - सीद - मिथु॒न्य॑ - ष्टौ च॑)  \textbf{(A8)} \newline \newline
                                \textbf{ TS 6.5.9.1} \newline
                  इन्द्रः॑ । वृ॒त्रम् । अ॒ह॒न्न् । तस्य॑ । शी॒र्॒.ष॒क॒पा॒लमिति॑ शीर्.ष-क॒पा॒लम् । उदिति॑ । औ॒ब्ज॒त् । सः । द्रो॒ण॒क॒ल॒श इति॑ द्रोण-क॒ल॒शः । अ॒भ॒व॒त् । तस्मा᳚त् । सोमः॑ । समिति॑ । अ॒स्र॒व॒त् । सः । हा॒रि॒यो॒ज॒न इति॑ हारि - यो॒ज॒नः । अ॒भ॒व॒त् । तम् । वीति॑ । अ॒चि॒कि॒थ्स॒त् । जु॒हवा॒नी(3) । मा । हौ॒षा(3)म् । इति॑ । सः । अ॒म॒न्य॒त॒ । यत् । हो॒ष्यामि॑ । आ॒मम् । हो॒ष्या॒मि॒ । यत् । न । हो॒ष्यामि॑ । य॒ज्ञ्॒वे॒श॒समिति॑ यज्ञ् - वे॒श॒सम् । क॒रि॒ष्या॒मि॒ । इति॑ । तम् । अ॒द्ध्रि॒य॒त॒ । होतु᳚म् । सः । अ॒ग्निः । अ॒ब्र॒वी॒त् । न । मयि॑ । आ॒मम् । हो॒ष्य॒सि॒ । इति॑ । तम् । धा॒नाभिः॑ । अ॒श्री॒णा॒त् । \textbf{  32} \newline
                  \newline
                                \textbf{ TS 6.5.9.2} \newline
                  तम् । शृ॒तम् । भू॒तम् । अ॒जु॒हो॒त् । यत् । धा॒नाभिः॑ । हा॒रि॒यो॒ज॒नमिति॑ हारि - यो॒ज॒नम् । श्री॒णाति॑ । शृ॒त॒त्वायेति॑ शृत - त्वाय॑ । शृ॒तम् । ए॒व । ए॒न॒म् । भू॒तम् । जु॒हो॒ति॒ । ब॒ह्वीभिः॑ । श्री॒णा॒ति॒ । ए॒ताव॑तीः । ए॒व । अ॒स्य॒ । अ॒मुष्मिन्न्॑ । लो॒के । का॒म॒दुघा॒ इति॑ काम - दुघाः᳚ । भ॒व॒न्ति॒ । अथो॒ इति॑ । खलु॑ । आ॒हुः॒ । ए॒ताः । वै । इन्द्र॑स्य । पृश्न॑यः । का॒म॒दुघा॒ इति॑ काम - दुघाः᳚ । यत् । हा॒रि॒यो॒ज॒नीरिति॑ हारि - यो॒ज॒नीः । इति॑ । तस्मा᳚त् । ब॒ह्वीभिः॑ । श्री॒णी॒या॒त् । ऋ॒ख्सा॒मे इत्यृ॑क् - सा॒मे । वै । इन्द्र॑स्य । हरी॒ इति॑ । सो॒म॒पाना॒विति॑ सोम - पानौ᳚ । तयोः᳚ । प॒रि॒धय॒ इति॑ परि - धयः॑ । आ॒धान॒मित्या᳚ - धान᳚म् । यत् । अप्र॑हृ॒त्येत्यप्र॑ - हृ॒त्य॒ । प॒रि॒धीनिति॑ परि - धीन् । जु॒हु॒यात् । अ॒न्तरा॑धानाभ्या॒मित्य॒न्तः-आ॒धा॒ना॒भ्या॒म् । \textbf{  33} \newline
                  \newline
                                \textbf{ TS 6.5.9.3} \newline
                  घा॒सम् । प्रेति॑ । य॒च्छे॒त् । प्र॒हृत्येति॑ प्र - हृत्य॑ । प॒रि॒धीनिति॑ परि - धीन् । जु॒हो॒ति॒ । निरा॑धानाभ्या॒मिति॒ निः - आ॒धा॒ना॒भ्या॒म् । ए॒व । घा॒सम् । प्रेति॑ । य॒च्छ॒ति॒ । उ॒न्ने॒तेत्यु॑त् - ने॒ता । जु॒हो॒ति॒ । या॒तया॒मेति॑ या॒त - या॒मा॒ । इ॒व॒ । हि । ए॒तर्.हि॑ । अ॒द्ध्व॒र्युः । स्व॒गाकृ॑त॒ इति॑ स्व॒गा-कृ॒तः॒ । यत् । अ॒द्ध्व॒र्युः । जु॒हु॒यात् । यथा᳚ । विमु॑क्त॒मित्॒ वि - मु॒क्त॒म् । पुनः॑ । यु॒नक्ति॑ । ता॒दृक् । ए॒व । तत् । शी॒र्॒.षन्न् । अ॒धि॒नि॒धायेत्य॑धि - नि॒धाय॑ । जु॒हो॒ति॒ । शी॒र्॒.ष॒तः । हि । सः । स॒मभ॑व॒दिति॑ सं - अभ॑वत् । वि॒क्रम्येति॑ वि-क्रम्य॑ । जु॒हो॒ति॒ । वि॒क्रम्येति॑ वि - क्रम्य॑ । हि । इन्द्रः॑ । वृ॒त्रम् । अहन्न्॑ । समृ॑द्ध्या॒ इति॒ सम्-ऋ॒द्ध्यै॒ । प॒शवः॑ । वै । हा॒रि॒यो॒ज॒नीरिति॑ हारि-यो॒ज॒नीः । यत् । स॒भिं॒न्द्यादिति॑ सं - भि॒न्द्यात् । अल्पाः᳚ । \textbf{  34} \newline
                  \newline
                                \textbf{ TS 6.5.9.4} \newline
                  ए॒न॒म् । प॒शवः॑ । भु॒ञ्जन्तः॑ । उपेति॑ । ति॒ष्ठे॒र॒न्न् । यत् । न । स॒भिं॒न्द्यादिति॑ सं - भि॒न्द्यात् । ब॒हवः॑ । ए॒न॒म् । प॒शवः॑ । अभु॑ञ्जन्तः । उपेति॑ । ति॒ष्ठे॒र॒न्न् । मन॑सा । समिति॑ । बा॒ध॒ते॒ । उ॒भय᳚म् । क॒रो॒ति॒ । ब॒हवः॑ । ए॒व । ए॒न॒म् । प॒शवः॑ । भु॒ञ्जन्तः॑ । उपेति॑ । ति॒ष्ठ॒न्ते॒ । उ॒न्ने॒तरीत्यु॑त् - ने॒तरि॑ । उ॒प॒ह॒वमित्यु॑प - ह॒वम् । इ॒च्छ॒न्ते॒ । यः । ए॒व । तत्र॑ । सो॒म॒पी॒थ इति॑ सोम - पी॒थः । तम् । ए॒व । अवेति॑ । रु॒न्ध॒ते॒ । उ॒त्त॒र॒वे॒द्यामित्यु॑त्तर - वे॒द्याम् । नीति॑ । व॒प॒ति॒ । प॒शवः॑ । वै । उ॒त्त॒र॒वे॒दिरित्यु॑त्तर - वे॒दिः । प॒शवः॑ । हा॒रि॒यो॒ज॒नीरिति॑ हारि - यो॒ज॒नीः । प॒शुषु॑ । ए॒व । प॒शून् । प्रतीति॑ । स्था॒प॒य॒न्ति॒ ( ) ॥ \textbf{  35} \newline
                  \newline
                      (अ॒श्री॒णा॒ - द॒न्तरा॑धानाभ्या॒ - मल्पाः᳚ - स्थापयन्ति)  \textbf{(A9)} \newline \newline
                                \textbf{ TS 6.5.10.1} \newline
                  ग्रहान्॑ । वै । अन्विति॑ । प्र॒जा इति॑ प्र - जाः । प॒शवः॑ । प्रेति॑ । जा॒य॒न्ते॒ । उ॒पाꣳ॒॒श्व॒न्त॒र्या॒मावित्यु॑पाꣳशु - अ॒न्त॒र्या॒मौ । अ॒जा॒वय॒ इत्य॑जा - अ॒वयः॑ । शु॒क्राम॒न्थिना॒विति॑ शु॒क्रा - म॒न्थिनौ᳚ । पुरु॑षाः । ऋ॒तु॒ग्र॒हानित्यृ॑तु - ग्र॒हान् । एक॑शफा॒ इत्येक॑ - श॒फाः॒ । आ॒दि॒त्य॒ग्र॒हमित्या॑दित्य - ग्र॒हम् । गावः॑ । आ॒दि॒त्य॒ग्र॒ह इत्या॑दित्य - ग्र॒हः । भूयि॑ष्ठाभिः । ऋ॒ग्भिरित्यृ॑क्-भिः । गृ॒ह्य॒ते॒ । तस्मा᳚त् । गावः॑ । प॒शू॒नाम् । भूयि॑ष्ठाः । यत् । त्रिः । उ॒पाꣳ॒॒शुमित्यु॑प - अ॒शुम् । हस्ते॑न । वि॒गृ॒ह्णातीति॑ वि - गृ॒ह्णाति॑ । तस्मा᳚त् । द्वौ । त्रीन् । अ॒जा । ज॒नय॑ति । अथ॑ । अव॑यः । भूय॑सीः । पि॒ता । वै । ए॒षः । यत् । आ॒ग्र॒य॒णः । पु॒त्रः । क॒लशः॑ । यत् । आ॒ग्र॒य॒णः । उ॒प॒दस्ये॒दित्यु॑प - दस्ये᳚त् । क॒लशा᳚त् । गृ॒ह्णी॒या॒त् । यथा᳚ । पि॒ता । \textbf{  36} \newline
                  \newline
                                \textbf{ TS 6.5.10.2} \newline
                  पु॒त्रम् । क्षि॒तः । उ॒प॒धाव॒तीत्यु॑प-धाव॑ति । ता॒दृक् । ए॒व । तत् । यत् । क॒लशः॑ । उ॒प॒दस्ये॒दित्यु॑प-दस्ये᳚त् । आ॒ग्र॒य॒णात् । गृ॒ह्णी॒या॒त् । यथा᳚ । पु॒त्रः । पि॒तर᳚म् । क्षि॒तः । उ॒प॒धाव॒तीत्यु॑प - धाव॑ति । ता॒दृक् । ए॒व । तत् । आ॒त्मा । वै । ए॒षः । य॒ज्ञ्स्य॑ । यत् । आ॒ग्र॒य॒णः । यत् । ग्रहः॑ । वा॒ । क॒लशः॑ । वा॒ । उ॒प॒दस्ये॒दित्यु॑प - दस्ये᳚त् । आ॒ग्र॒य॒णात् । गृ॒ह्णी॒या॒त् । आ॒त्मनः॑ । ए॒व । अधीति॑ । य॒ज्ञ्म् । निरिति॑ । क॒रो॒ति॒ । अवि॑ज्ञात॒ इत्यवि॑ - ज्ञा॒तः॒ । वै । ए॒षः । गृ॒ह्य॒ते॒ । यत् । आ॒ग्र॒य॒णः । स्था॒ल्या । गृ॒ह्णाति॑ । वा॒य॒व्ये॑न । जु॒हो॒ति॒ । तस्मा᳚त् । \textbf{  37} \newline
                  \newline
                                \textbf{ TS 6.5.10.3} \newline
                  गभे॑र्ण । अवि॑ज्ञाते॒नेत्यवि॑ - ज्ञा॒ते॒न॒ । ब्र॒ह्म॒हेति॑ ब्रह्म - हा । अ॒व॒भृ॒थमित्यव॑ - भृ॒थम् । अवेति॑ । य॒न्ति॒ । परेति॑ । स्था॒लीः । अस्य॑न्ति । उदिति॑ । वा॒य॒व्या॑नि । ह॒र॒न्ति॒ । तस्मा᳚त् । स्त्रिय᳚म् । जा॒ताम् । परेति॑ । अ॒स्य॒न्ति॒ । उदिति॑ । पुमाꣳ॑सम् । ह॒र॒न्ति॒ । यत् । पु॒रो॒रुच॒मिति॑ पुरः - रुच᳚म् । आह॑ । यथा᳚ । वस्य॑से । आ॒हर॒तीत्या᳚-हर॑ति । ता॒दृक् । ए॒व । तत् । यत् । ग्रह᳚म् । गृ॒ह्णाति॑ । यथा᳚ । वस्य॑से । आ॒हृत्येत्या᳚ - हृत्य॑ । प्रेति॑ । आह॑ । ता॒दृक् । ए॒व । तत् । यत् । सा॒दय॑ति । यथा᳚ । वस्य॑से । उ॒प॒नि॒धायेत्यु॑प - नि॒धाय॑ । अ॒प॒क्राम॒तीत्य॑प - क्राम॑ति । ता॒दृक् । ए॒व । तत् । यत् ( ) । वै । य॒ज्ञ्स्य॑ । साम्ना᳚ । यजु॑षा । क्रि॒यते᳚ । शि॒थि॒लम् । तत् । यत् । ऋ॒चा । तत् । दृ॒ढम् । पु॒रस्ता॑दुपयामा॒ इति॑ पु॒रस्ता᳚त् - उ॒प॒या॒माः॒ । यजु॑षा । गृ॒ह्य॒न्ते॒ । उ॒परि॑ष्टादुपयामा॒ इत्यु॒परि॑ष्टात् - उ॒प॒या॒माः॒ । ऋ॒चा । य॒ज्ञ्स्य॑ । धृत्यै᳚ ॥ \textbf{  38 } \newline
                  \newline
                      (यथा॑ पि॒ता-तस्मा॑-दप॒क्राम॑ति ता॒दृगे॒व तद् य-द॒ष्टा द॑श च)  \textbf{(A10)} \newline \newline
                                \textbf{ TS 6.5.11.1} \newline
                  प्रेति॑ । अ॒न्यानि॑ । पात्रा॑णि । यु॒ज्यन्ते᳚ । न । अ॒न्यानि॑ । यानि॑ । प॒रा॒चीना॑नि । प्र॒यु॒ज्यन्त॒ इति॑ प्र - यु॒ज्यन्ते᳚ । अ॒मुम् । ए॒व । तैः । लो॒कम् । अ॒भीति॑ । ज॒य॒ति॒ । पराङ्॑ । इ॒व॒ । हि । अ॒सौ । लो॒कः । यानि॑ । पुनः॑ । प्र॒यु॒ज्यन्त॒ इति॑ प्र - यु॒ज्यन्ते᳚ । इ॒मम् । ए॒व । तैः । लो॒कम् । अ॒भीति॑ । ज॒य॒ति॒ । पुनः॑पुन॒रिति॒ पुनः॑ - पु॒नः॒ । इ॒व॒ । हि । अ॒यम् । लो॒कः । प्रेति॑ । अ॒न्यानि॑ । पात्रा॑णि । यु॒ज्यन्ते᳚ । न । अ॒न्यानि॑ । यानि॑ । प॒रा॒चीना॑नि । प्र॒यु॒ज्यन्त॒ इति॑ प्र - यु॒ज्यन्ते᳚ । तानि॑ । अन्विति॑ । ओष॑धयः । परेति॑ । भ॒व॒न्ति॒ । यानि॑ । पुनः॑ । \textbf{  39} \newline
                  \newline
                                \textbf{ TS 6.5.11.2} \newline
                  प्र॒यु॒ज्यन्त॒ इति॑ प्र - यु॒ज्यन्ते᳚ । तानि॑ । अन्विति॑ । ओष॑धयः । पुनः॑ । एति॑ । भ॒व॒न्ति॒ । प्रेति॑ । अ॒न्यानि॑ । पात्रा॑णि । यु॒ज्यन्ते᳚ । न । अ॒न्यानि॑ । यानि॑ । प॒रा॒चीना॑नि । प्र॒यु॒ज्यन्त॒ इति॑ प्र-यु॒ज्यन्ते᳚ । तानि॑ । अन्विति॑ । आ॒र॒ण्याः । प॒शवः॑ । अर॑ण्यम् । अपेति॑ । य॒न्ति॒ । यानि॑ । पुनः॑ । प्र॒यु॒ज्यन्त॒ इति॑ प्र - यु॒ज्यन्ते᳚ । तानि॑ । अन्विति॑ । ग्रा॒म्याः । प॒शवः॑ । ग्राम᳚म् । उ॒पाव॑य॒न्तीत्यु॑प-अव॑यन्ति । यः । वै । ग्रहा॑णाम् । नि॒दान॒मिति॑ नि - दान᳚म् । वेद॑ । नि॒दान॑वा॒निति॑ नि॒दान॑ - वा॒न् । भ॒व॒ति॒ । आज्य᳚म् । इति॑ । उ॒क्थम् । तत् । वै । ग्रहा॑णाम् । नि॒दान॒मिति॑ नि - दान᳚म् । यत् । उ॒पाꣳ॒॒श्वित्यु॑प - अꣳ॒॒शु । शꣳस॑ति । तत् । \textbf{  40} \newline
                  \newline
                                \textbf{ TS 6.5.11.3} \newline
                  उ॒पाꣳ॒॒श्व॒न्त॒र्या॒मयो॒रित्यु॑पाꣳशु - अ॒न्त॒र्या॒मयोः᳚ । यत् । उ॒च्चैः । तत् । इत॑रेषाम् । ग्रहा॑णाम् । ए॒तत् । वै । ग्रहा॑णाम् । नि॒दान॒मिति॑ नि - दान᳚म् । यः । ए॒वम् । वेद॑ । नि॒दान॑वा॒निति॑ नि॒दान॑ - वा॒न् । भ॒व॒ति॒ । यः । वै । ग्रहा॑णाम् । मि॒थु॒नम् । वेद॑ । प्रेति॑ । प्र॒जयेति॑ प्र - जया᳚ । प॒शुभि॒रिति॑ प॒शु - भिः॒ । मि॒थु॒नैः । जा॒य॒ते॒ । स्था॒लीभिः॑ । अ॒न्ये । ग्रहाः᳚ । गृ॒ह्यन्ते᳚ । वा॒य॒व्यैः᳚ । अ॒न्ये । ए॒तत् । वै । ग्रहा॑णाम् । मि॒थु॒नम् । यः । ए॒वम् । वेद॑ । प्रेति॑ । प्र॒जयेति॑ प्र - जया᳚ । प॒शुभि॒रिति॑ प॒शु - भिः॒ । मि॒थु॒नैः । जा॒य॒ते॒ । इन्द्रः॑ । त्वष्टुः॑ । सोम᳚म् । अ॒भी॒षहेत्य॑भि - सहा᳚ । अ॒पि॒ब॒त् । सः । विष्वङ्॑ । \textbf{  41} \newline
                  \newline
                                \textbf{ TS 6.5.11.4} \newline
                  वीति॑ । आ॒र्च्छ॒त् । सः । आ॒त्मन्न् । आ॒रम॑ण॒मित्या᳚ - रम॑णम् । न । अ॒वि॒न्द॒त् । सः । ए॒तान् । अ॒नु॒स॒व॒नमित्य॑नु-स॒व॒नम् । पु॒रो॒डाशान्॑ । अ॒प॒श्य॒त् । तान् । निरिति॑ । अ॒व॒प॒त् । तैः । वै । सः । आ॒त्मन्न् । आ॒रम॑ण॒मित्या᳚ - रम॑णम् । अ॒कु॒रु॒त॒ । तस्मा᳚त् । अ॒नु॒स॒व॒नमित्य॑नु - स॒व॒नम् । पु॒रो॒डाशाः᳚ । निरिति॑ । उ॒प्य॒न्ते॒ । तस्मा᳚त् । अ॒नु॒स॒व॒नमित्य॑नु - स॒व॒नम् । पु॒रो॒डाशा॑नाम् । प्रेति॑ । अ॒श्नी॒या॒त् । आ॒त्मन्न् । ए॒व । आ॒रम॑ण॒मित्या᳚ - रम॑णम् । कु॒रु॒ते॒ । न । ए॒न॒म् । सोमः॑ । अतीति॑ । प॒व॒ते॒ । ब्र॒ह्म॒वा॒दिन॒ इति॑ ब्रह्म - वा॒दिनः॑ । व॒द॒न्ति॒ । न । ऋ॒चा । न । यजु॑षा । प॒ङ्क्तिः । आ॒प्य॒ते॒ । अथ॑ । किम् ( ) । य॒ज्ञ्स्य॑ । पा॒ङ्क्त॒त्वमिति॑ पाङ्क्त - त्वम् । इति॑ । धा॒नाः । क॒र॒म्भः । प॒रि॒वा॒प इति॑ परि - वा॒पः । पु॒रो॒डाशः॑ । प॒य॒स्या᳚ । तेन॑ । प॒ङ्क्तिः । आ॒प्य॒ते॒ । तत् । य॒ज्ञ्स्य॑ । पा॒ङ्क्त॒त्वमिति॑ पाङ्क्त - त्वम् ॥ \textbf{  42 } \newline
                  \newline
                      (भ॒व॒न्ति॒ यानि॒ पुनः॒ - शꣳस॑ति॒ त - द्विष्व॒ङ् - किं - चतु॑र्दश च)  \textbf{(A11)} \newline \newline
\textbf{praSna korvai with starting padams of 1 to 11 anuvAkams :-} \newline
(इन्द्रो॑ वृ॒त्राय- ऽऽयु॒र्वे - य॒ज्ञेन॑ - सुव॒र्गा - येन्द्रो॑ म॒रुद्भि॒ - रदि॑ति - रन्तर्यामपा॒त्रेण॑ - प्रा॒ण उ॑पाꣳशु पा॒त्रे - णेन्द्रो॑ वृ॒त्रम॑ह॒॒न् तस्य॒ - ग्रहा॒न् - प्रान्या - न्येका॑दश) \newline

\textbf{korvai with starting padams of1, 11, 21 series of pa~jcAtis :-} \newline
(इन्द्रो॑ वृ॒त्राय॒ - पुन॑र्. ऋ॒तुना॑ऽऽह - मिथु॒नं प॒शवो॒ - नेष्टः॒ पत्नी॑ - मुपाꣳश्वन्तर्या॒मयो॒ - द्विच॑त्वारिꣳशत्) \newline

\textbf{first and last padam of fifth praSnam of 6th KANDam} \newline
(इन्द्रो॑ वृ॒त्राय॑ - पाङ्क्त॒त्वं) \newline 


॥ हरिः॑ ॐ ॥
॥ कृष्ण यजुर्वेदीय तैत्तिरीय संहितायां षष्ठकाण्डे पञ्चमः प्रश्नः समाप्तः ॥ \newline
\pagebreak
\pagebreak
        


\end{document}
