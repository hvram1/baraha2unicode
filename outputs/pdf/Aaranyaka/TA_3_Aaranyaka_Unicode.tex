\documentclass[17pt]{extarticle}
\usepackage{babel}
\usepackage{fontspec}
\usepackage{polyglossia}
\usepackage{extsizes}



\setmainlanguage{sanskrit}
\setotherlanguages{english} %% or other languages
\setlength{\parindent}{0pt}
\pagestyle{myheadings}
\newfontfamily\devanagarifont[Script=Devanagari]{AdishilaVedic}


\newcommand{\VAR}[1]{}
\newcommand{\BLOCK}[1]{}




\begin{document}
\begin{titlepage}
    \begin{center}
 
\begin{sanskrit}
    { \Large
    ॐ नमः परमात्मने, श्री महागणपतये नमः
श्री गुरुभ्यो नमः, ह॒रिः॒ ॐ 
    }
    \\
    \vspace{2.5cm}
    \mbox{ \Huge
    कृष्ण यजुर्वेदीय तैत्तिरीय आरण्यके तृतीयः प्रपाठकः   }
\end{sanskrit}
\end{center}

\end{titlepage}
\tableofcontents

ॐ नमः परमात्मने, श्री महागणपतये नमः
श्री गुरुभ्यो नमः, ह॒रिः॒ ॐ \newline
3.1     तृतीयः प्रपाठकः - चातुर्होत्रचयनम् \newline

\addcontentsline{toc}{section}{ 3.1     तृतीयः प्रपाठकः - चातुर्होत्रचयनम्}
\markright{ 3.1     तृतीयः प्रपाठकः - चातुर्होत्रचयनम् \hfill https://www.vedavms.in \hfill}
\section*{ 3.1     तृतीयः प्रपाठकः - चातुर्होत्रचयनम् }
                                \textbf{ T.A.3.1.1} \newline
                  चित्तिः॒ स्रुक् । चि॒त्तमाज्य᳚म् । वाग्वेदिः॑ । आधी॑तं ब॒र्.॒हिः । केतो॑ अ॒ग्निः । विज्ञा॑तम॒ग्निः । वाक्प॑ति॒र्.॒होता᳚ । मन॑ उपव॒क्ता । प्रा॒णो ह॒विः । सामा᳚द्ध्व॒र्युः ( ) ॥ वाच॑स्पते विधे नामन्न् । वि॒धेम॑ ते॒ नाम॑ । वि॒धेस्त्व-म॒स्माकं॒ नाम॑ । वा॒चस्पतिः॒ सोमं॑ पिबतु । आऽस्मासु॑ नृ॒म्णं धा॒थ्स्वाहा᳚ । \textbf{ 1} \newline
                  \newline
                                                        (अ॒द्ध्व॒र्युः पञ्च च) \textbf{1} \newline \newline
                                \textbf{ T.A.3.2.1} \newline
                  पृ॒थि॒वी होता᳚ । द्यौर॑द्ध्व॒र्युः । रु॒द्रो᳚ऽग्नीत् । बृह॒स्पति॑रुपव॒क्ता ॥ वाच॑स्पते वा॒चो वी॒र्ये॑ण । संभृ॑तत-मे॒नाय॑क्ष्यसे । यज॑मानाय॒ वार्य᳚म् ।  आ सुव॒स्कर॑स्मै । वा॒चस्पतिः॒ सोमं॑ पिबति ।  ज॒जन॒दिन्द्र॑मिन्द्रि॒याय॒ स्वाहा᳚ । \textbf{ 2} \newline
                  \newline
                                                        (पृ॒थि॒वी होता॒ दश॑) \textbf{2} \newline \newline
                                \textbf{ T.A.3.3.1} \newline
                  अ॒ग्निर्. होता᳚ । अ॒श्विना᳚ऽद्ध्व॒र्यू । त्वष्टा॒ऽग्नीत् । मि॒त्र उ॑पव॒क्ता ॥ सोमः॒ सोम॑स्य पुरो॒गाः । शु॒क्रः शु॒क्रस्य॑ पुरो॒गाः । श्रा॒तास्त॑ इन्द्र॒ सोमाः᳚ । वाता॑पेर्. हवन॒श्रुतः॒ स्वाहा᳚ । \textbf{ 3} \newline
                  \newline
                                                        (अ॒ग्निर्. होता॒ऽष्टौ) \textbf{3} \newline \newline
                                \textbf{ T.A.3.4.1} \newline
                  सूर्यं॑ ते॒ चक्षुः॑ । वातं॑ प्रा॒णः ।  द्यां पृ॒ष्ठम् । अ॒न्तरि॑क्षमा॒त्मा ।  अङ्गै᳚र् य॒ज्ञ्म् । पृ॒थि॒वीꣳ शरी॑रैः ॥ वाच॑स्प॒तेऽच्छि॑द्रया वा॒चा । अच्छि॑द्रया जु॒ह्वा᳚ । दि॒वि दे॑वा॒वृधꣳ॒॒ होत्रा॒ मेर॑यस्व॒ स्वाहा᳚ । \textbf{ 4} \newline
                  \newline
                                                        (सूर्यं॑ ते॒ नव॑) \textbf{4} \newline \newline
                                \textbf{ T.A.3.5.1} \newline
                  म॒हाह॑वि॒र्॒. होता᳚ । स॒त्यह॑विरद्ध्व॒र्युः । अच्यु॑तपाजा अ॒ग्नीत् । अच्यु॑तमना उपव॒क्ता । अ॒ना॒धृ॒ष्यश्चा᳚ प्रतिधृ॒ष्यश्च॑ य॒ज्ञ्स्या॑भिग॒रौ । अ॒यास्य॑ उद्गा॒ता ॥ वाच॑स्पते हृद्विधे नामन्न् । वि॒धेम॑ ते॒ नाम॑ ।  वि॒धेस्त्वम॒स्माकं॒ नाम॑ । वा॒चस्पतिः॒ सोम॑मपात् ( ) । मा दैव्य॒स्तन्तुः॒ छेदि॒ मा म॑नु॒ष्यः॑ । नमो॑ दि॒वे । नमः॑ पृथि॒व्यै स्वाहा᳚ । \textbf{ 5} \newline
                  \newline
                                                        (अ॒पा॒त् त्रीणि॑ च) \textbf{5} \newline \newline
                                \textbf{ T.A.3.6.1} \newline
                  वाग्घोता᳚ । दी॒क्षा पत्नी᳚ । वातो᳚ऽद्ध्व॒र्युः । आपो॑ऽभिग॒रः । मनो॑ ह॒विः । तप॑सि जुहोमि ॥  भूर्भुव॒स्सुवः॑ । ब्रह्म॑ स्वय॒म्भु ।  ब्रह्म॑णे स्वय॒म्भुवे॒ स्वाहा᳚ । \textbf{ 6} \newline
                  \newline
                                                        (वाग्घोता॒ नव॑) \textbf{6} \newline \newline
                                \textbf{ T.A.3.7.1} \newline
                  ब्रा॒ह्म॒ण एक॑होता । स य॒ज्ञ्ः । स मे॑ ददातु प्र॒जां प॒शून् पुष्टिं॒ ॅयशः॑ । य॒ज्ञ्श्च॑ मे भूयात् ॥ अ॒ग्निर् द्विहो॑ता । स भ॒र्ता । स मे॑ ददातु प्र॒जां प॒शून् पुष्टिं॒ ॅयशः॑ । भ॒र्ता च॑ मे भूयात् ॥  पृ॒थि॒वी त्रिहो॑ता । स प्र॑ति॒ष्ठा \textbf{ 7} \newline
                  \newline
                                                                  \textbf{ T.A.3.7.2} \newline
                  स मे॑ ददातु प्र॒जां प॒शून् पुष्टिं॒ ॅयशः॑ । प्र॒ति॒ष्ठा च॑ मे भूयात् ॥ अ॒न्तरि॑क्षं॒ चतु॑र्.होता । स वि॒ष्ठाः । स मे॑ ददातु प्र॒जां प॒शून् पुष्टिं॒ ॅयशः॑ । वि॒ष्ठाश्च॑ मे भूयात् ॥  वा॒युः पञ्च॑होता । स प्रा॒णः ।  स मे॑ ददातु प्र॒जां प॒शून् पुष्टिं॒ ॅयशः॑ । प्रा॒णश्च॑ मे भूयात् । \textbf{ 8} \newline
                  \newline
                                                                  \textbf{ T.A.3.7.3} \newline
                  च॒न्द्रमाः॒ षड्ढो॑ता । स ऋ॒तून् क॑ल्पयाति ।  स मे॑ ददातु प्र॒जां प॒शून् पुष्टिं॒ ॅयशः॑ । ऋ॒तव॑श्च मे कल्पन्ताम् ॥ अन्नꣳ॑ स॒प्तहो॑ता । स प्रा॒णस्य॑ प्रा॒णः । स मे॑ ददातु प्र॒जां प॒शून् पुष्टिं॒ ॅयशः॑ । प्रा॒णस्य॑ च मे प्रा॒णो भू॑यात् ॥ द्यौर॒ष्टहो॑ता । सो॑ऽनाधृ॒ष्यः \textbf{ 9} \newline
                  \newline
                                                                  \textbf{ T.A.3.7.4} \newline
                  स मे॑ ददातु प्र॒जां प॒शून् पुष्टिं॒ ॅयशः॑ । अ॒ना॒धृ॒ष्यश्च॑ भूयासम् ॥  आ॒दि॒त्यो नव॑होता । स ते॑ज॒स्वी ।  स मे॑ ददातु प्र॒जां प॒शून् पुष्टिं॒ ॅयशः॑ । ते॒ज॒स्वी च॑ भूयासम् ॥ प्र॒जाप॑ति॒र् दश॑होता । स इ॒दꣳ सर्व᳚म् । स मे॑ ददातु प्र॒जां प॒शून् पुष्टिं॒ ॅयशः॑ । सर्वं॑ च मे भूयात् ( ) । \textbf{ 10} \newline
                  \newline
                                                  
                (ब्रा॒ह्म॒णो य॒ज्ञो᳚ऽग्निर्भ॒र्ता पृ॑थि॒वी प्र॑ति॒ष्ठाऽन्तरि॑क्षं ॅवि॒ष्ठा वा॒युः  प्रा॒णश्च॒न्द्रमा॑ ऋ॒तूनन्नꣳ॒॒ स प्रा॒णस्य॑ प्रा॒णो द्यौर॑नाधृ॒ष्य आ॑दि॒त्यः  स ते॑ज॒स्वी प्र॒जाप॑तिः॒ स इ॒दꣳ सर्वꣳ॒॒ सर्वं॑ च मे भूयात्) \newline
                                      (प्र॒ति॒ष्ठा-प्रा॒णश्च॑ मे भूया-दनाधृ॒ष्यः-सर्वं॑ च मे भूयात्) \textbf{7} \newline \newline
                                \textbf{ T.A.3.8.1} \newline
                  अ॒ग्निर् यजु॑र्भिः । स॒वि॒ता स्तोमैः᳚ । इन्द्र॑ उक्थाम॒दैः ।  मि॒त्रावरु॑णावा॒शिषा᳚ । अङ्गि॑रसो॒ धिष्णि॑यैर॒ग्निभिः॑ । म॒रुतः॑ सदोहविर्द्धा॒नाभ्या᳚म् । आपः॒ प्रोक्ष॑णीभिः ।  ओष॑धयो ब॒र्॒.हिषा᳚ । अदि॑ति॒र्वेद्या᳚ । सोमो॑ दी॒क्षया᳚ \textbf{ 11} \newline
                  \newline
                                                                  \textbf{ T.A.3.8.2} \newline
                  त्वष्टे॒द्ध्मेन॑ । विष्णु॑र् य॒ज्ञेन॑ । वस॑व॒ आज्ये॑न । आ॒दि॒त्या दक्षि॑णाभिः । विश्वे॑ दे॒वा ऊ॒र्जा ।  पू॒षा स्व॑गाका॒रेण॑ । बृह॒स्पतिः॑ पुरो॒धया᳚ ।  प्र॒जाप॑तिरुद्गी॒थेन॑ । अ॒न्तरि॑क्षं प॒वित्रे॑ण । वा॒युः पात्रैः᳚ ( ) । अ॒हꣳ श्र॒द्धया᳚ । \textbf{ 12} \newline
                  \newline
                                                        (दी॒क्षया॒ - पात्रै॒रेकं॑ च) \textbf{8} \newline \newline
                                \textbf{ T.A.3.9.1} \newline
                  सेनेन्द्र॑स्य । धेना॒ बृह॒स्पतेः᳚ । प॒त्थ्या॑ पू॒ष्णः । वाग्वा॒योः । दी॒क्षा सोम॑स्य । पृ॒थि॒व्य॑ग्नेः । वसू॑नां गाय॒त्री । रु॒द्राणां᳚ त्रि॒ष्टुक् । आ॒दि॒त्यानां॒ जग॑ती । विष्णो॑रनु॒ष्टुक् \textbf{ 13} \newline
                  \newline
                                                                  \textbf{ T.A.3.9.2} \newline
                  वरु॑णस्य वि॒राट् । य॒ज्ञ्स्य॑ प॒ङ्क्तिः । प्र॒जाप॑ते॒रनु॑मतिः । मि॒त्रस्य॑ श्र॒द्धा । स॒वि॒तुः प्रसू॑तिः । सूर्य॑स्य॒ मरी॑चिः । च॒न्द्रम॑सो रोहि॒णी । ऋषी॑णामरुन्ध॒ती । प॒र्जन्य॑स्य वि॒द्युत् ।  चत॑स्रो॒ दिशः॑ ( ) । चत॑स्रोऽवान्तरदि॒शाः । अह॑श्च॒ रात्रि॑श्च । कृ॒षिश्च॒ वृष्टि॑श्च ।  त्विषि॒श्चाप॑चितिश्च । आप॒श्चौष॑धयश्च ।  ऊर्क्च॑ सू॒नृता॑ च दे॒वानां॒ पत्न॑यः । \textbf{ 14} \newline
                  \newline
                                                        (अ॒नु॒ष्टुग् - दिशः॒ षट्च॑) \textbf{9} \newline \newline
                                \textbf{ T.A.3.10.1} \newline
                  दे॒वस्य॑ त्वा सवि॒तुः प्र॑स॒वे । अ॒श्विनो᳚र् बा॒हुभ्या᳚म् ।  पू॒ष्णो हस्ता᳚भ्यां॒ प्रति॑गृह्णामि ।  राजा᳚ त्वा॒ वरु॑णो नयतु देवि दक्षिणे॒ऽग्नये॒ हिर॑ण्यम् । तेना॑मृत॒त्वम॑श्याम् । वयो॑ दा॒त्रे । मयो॒ मह्य॑मस्तु प्रतिग्रही॒त्रे । क इ॒दं कस्मा॑ अदात् । कामः॒ कामा॑य । कामो॑ दा॒ता \textbf{ 15} \newline
                  \newline
                                                                  \textbf{ T.A.3.10.2} \newline
                  कामः॑ प्रतिग्रही॒ता । कामꣳ॑ समु॒द्रमावि॑श । कामे॑न त्वा॒ प्रति॑गृह्णामि ।  कामै॒तत्ते᳚ । ए॒षा ते॑ काम॒ दक्षि॑णा । उ॒त्ता॒नस्त्वा᳚ऽऽङ्गीर॒सः प्रति॑गृह्णातु ।  सोमा॑य॒ वासः॑ । रु॒द्राय॒ गाम् । वरु॑णा॒याश्व᳚म् ।  प्र॒जाप॑तये॒ पुरु॑षम् \textbf{ 16} \newline
                  \newline
                                                                  \textbf{ T.A.3.10.3} \newline
                  मन॑वे॒ तल्प᳚म् । त्वष्ट्रे॒ऽजाम् । पू॒ष्णेऽवि᳚म् । निर्.ऋ॑त्या अश्वतरगर्द॒भौ । हि॒मव॑तो ह॒स्तिन᳚म् । ग॒न्ध॒र्वा॒-फ्स॒राभ्यः॑-स्रगलङ्कर॒णे ।  विश्वे᳚भ्यो दे॒वेभ्यो॑ धा॒न्यम् । वा॒चेऽन्न᳚म् । ब्रह्म॑ण ओद॒नम् । स॒मु॒द्रायापः॑ \textbf{ 17} \newline
                  \newline
                                                                  \textbf{ T.A.3.10.4} \newline
                  उ॒त्ता॒नाया᳚ङ्गीर॒सायानः॑ । वै॒श्वा॒न॒राय॒ रथ᳚म् ॥  वै॒श्वा॒न॒रः प्र॒त्नथा॒ नाक॒मारु॑हत् ।  दि॒वः पृ॒ष्ठं भन्द॑मानः सु॒मन्म॑भिः ।  स पू᳚र्व॒वज्ज॒नय॑ज्ज॒न्तवे॒ धन᳚म् । स॒मा॒नम॑ज्मा॒ परि॑याति॒ जागृ॑विः ॥  राजा᳚ त्वा॒ वरु॑णो नयतु देवि दक्षिणे वैश्वान॒राय॒ रथ᳚म् । तेना॑मृत॒त्वम॑श्याम् । वयो॑ दा॒त्रे ।  मयो॒ मह्य॑मस्तु प्रतिग्रही॒त्रे ( ) \textbf{ 18} \newline
                  \newline
                                                                  \textbf{ T.A.3.10.5} \newline
                  क इ॒दं कस्मा॑ अदात् । कामः॒ कामा॑य । कामो॑ दा॒ता ।  कामः॑ प्रतिग्रही॒ता । कामꣳ॑ समु॒द्रमावि॑श । कामे॑न त्वा॒ प्रति॑गृह्णामि । कामै॒तत्ते᳚ । ए॒षा ते॑ काम॒ दक्षि॑णा । उ॒त्ता॒नस्त्वा᳚ऽऽङ्गीर॒सः प्रति॑गृह्णातु । \textbf{ 19} \newline
                  \newline
                                                        (दा॒ता - पुरु॑ष॒ - मापः॑ - प्रतिग्रही॒त्रे - +नव॑ च) \textbf{10} \newline \newline
                                \textbf{ T.A.3.11.1} \newline
                  सु॒वर्णं॑ घ॒र्मं परि॑वेद वे॒नम् । इन्द्र॑स्या॒त्मानं॑ दश॒धा चर॑न्तम् ।  अ॒न्तः स॑मु॒द्रे मन॑सा॒ चर॑न्तम् । ब्रह्माऽन्व॑विन्द॒द् दश॑होतार॒मर्णे᳚ ।  अ॒न्तः प्रवि॑ष्टः शा॒स्ता जना॑नाम् । एकः॒ सन्ब॑हु॒धा वि॑चारः । श॒तꣳ शु॒क्राणि॒ यत्रैकं॒ भव॑न्ति । सर्वे॒ वेदा॒ यत्रैकं॒ भव॑न्ति । सर्वे॒ होता॑रो॒ यत्रैकं॒ भव॑न्ति । स॒ मान॑सीन आ॒त्मा जना॑नाम् \textbf{ 20} \newline
                  \newline
                                                                  \textbf{ T.A.3.11.2} \newline
                  अ॒न्तः प्रवि॑ष्टः शा॒स्ता जना॑नाꣳ॒॒ सर्वा᳚त्मा ।  सर्वाः᳚ प्र॒जा यत्रैकं॒ भव॑न्ति । चतु॑र्.होतारो॒ यत्र॑ स॒म्पदं॒ गच्छ॑न्ति दे॒वैः । स॒ मान॑सीन आ॒त्मा जना॑नाम् ॥ ब्रह्मेन्द्र॑म॒ग्निं जग॑तः प्रति॒ष्ठाम् ।  दि॒व आ॒त्मानꣳ॑ सवि॒तारं॒ बृह॒स्पति᳚म् ।  चतु॑र्.होतारं प्र॒दिशोऽनु॑ क्ल॒प्तम् । वा॒चो वी॒र्यं॑ तप॒साऽन्व॑विन्दत् ॥ अ॒न्तः प्रवि॑ष्टं क॒र्तार॑मे॒तम् ।  त्वष्टा॑रꣳ रू॒पाणि॑ विकु॒र्वन्तं॑ ॅविप॒श्चिम् \textbf{ 21} \newline
                  \newline
                                                                  \textbf{ T.A.3.11.3} \newline
                  अ॒मृत॑स्य प्रा॒णं ॅय॒ज्ञ्मे॒तम् ।  चतु॑र्.होतृणामा॒त्मानं॑ क॒वयो॒ निचि॑क्युः ॥  अ॒न्तः प्रवि॑ष्टं क॒र्तार॑मे॒तम् ।  दे॒वानां॒ बन्धु॒ निहि॑तं॒ गुहा॑सु ।  अ॒मृते॑न क्ल॒प्तं ॅय॒ज्ञ्मे॒तम् ।  चतु॑र्.होतृणामा॒त्मानं॑ क॒वयो॒ निचि॑क्युः ॥  श॒तं नि॒युतः॒ परि॑वेद॒ विश्वा॑ वि॒श्ववा॑रः । विश्व॑मि॒दं ॅवृ॑णाति ।  इन्द्र॑स्या॒त्मा निहि॑तः॒ पञ्च॑होता ।  अ॒मृतं॑ दे॒वाना॒मायुः॑ प्र॒जाना᳚म् । \textbf{ 22} \newline
                  \newline
                                                                  \textbf{ T.A.3.11.4} \newline
                  इन्द्रꣳ॒॒ राजा॑नꣳ सवि॒तार॑मे॒तम् । वा॒योरा॒त्मानं॑ क॒वयो॒ निचि॑क्युः ।  र॒श्मिꣳ र॑श्मी॒नां मद्ध्ये॒ तप॑न्तम् । ऋ॒तस्य॑ प॒दे क॒वयो॒ निपा᳚न्ति ॥ य आ᳚ण्डको॒शे भुव॑नं बि॒भर्ति॑ । अनि॑र्भिण्णः॒ सन्नथ॑ लो॒कान्. वि॒चष्टे᳚ ।  यस्या᳚ण्डको॒शꣳ शुष्म॑मा॒हुः प्रा॒णमुल्ब᳚म् ।  तेन॑ क्लृ॒प्तो॑ऽमृते॑ना॒हम॑स्मि ॥ सु॒वर्णं॒ कोशꣳ॒॒ रज॑सा॒ परी॑वृतम् ।  दे॒वानां᳚ ॅवसु॒धानीं᳚ ॅवि॒राज᳚म् \textbf{ 23} \newline
                  \newline
                                                                  \textbf{ T.A.3.11.5} \newline
                  अ॒मृत॑स्य पू॒र्णान्तामु॑ क॒लां ॅविच॑क्षते ।  पादꣳ॒॒ षड्ढो॑तु॒र्न किला॑वि विथ्से ॥ येन॒र्तवः॑ पञ्च॒धोत क्ल॒॒प्ताः । उ॒त वा॑ ष॒ड्धा मन॒सोत क्लृ॒॒प्ताः । तꣳ षड्ढो॑तारमृ॒तुभिः॒ कल्प॑मानम् । ऋ॒तस्य॑ प॒दे क॒वयो॒ निपा᳚न्ति ॥ अ॒न्तः प्रवि॑ष्टं क॒र्तार॑मे॒तम् । अ॒न्तश्च॒न्द्रम॑सि॒ मन॑सा॒ चर॑न्तम् । स॒हैव सन्तं॒ न विजा॑नन्ति दे॒वाः ।  इन्द्र॑स्या॒त्मानꣳ॑ शत॒धा चर॑न्तम् । \textbf{ 24} \newline
                  \newline
                                                                  \textbf{ T.A.3.11.6} \newline
                  इन्द्रो॒ राजा॒ जग॑तो॒ य ईशे᳚ । स॒प्तहो॑ता सप्त॒धा विक्लृ॑प्तः ॥ परे॑ण॒ तन्तुं॑ परिषि॒च्यमा॑नम् । अ॒न्तरा॑दि॒त्ये मन॑सा॒ चर॑न्तम् । दे॒वानाꣳ॒॒ हृद॑यं॒ ब्रह्माऽन्व॑विन्दत् ॥ ब्रह्मै॒तद्ब्रह्म॑ण॒ उज्ज॑भार । अ॒र्कꣳ श्चोत॑न्तꣳ सरि॒रस्य॒ मद्ध्ये᳚ ॥ आ यस्मिन्᳚थ् स॒प्त पेर॑वः । मेह॑न्ति बहु॒लाꣳ श्रिय᳚म् । ब॒ह्व॒श्वामि॑न्द्र॒ गोम॑तीम् । \textbf{ 25} \newline
                  \newline
                                                                  \textbf{ T.A.3.11.7} \newline
                  अच्यु॑तां बहु॒लाꣳ श्रिय᳚म् । स हरि॑र्वसु॒वित्त॑मः ।  पे॒रुरिन्द्रा॑य पिन्वते ॥ ब॒ह्व॒श्वामि॑न्द्र॒ गोम॑तीम् । अच्यु॑तां बहु॒लाꣳ श्रिय᳚म् । मह्य॒मिन्द्रो॒ निय॑च्छतु ॥ श॒तꣳ श॒ता अ॑स्य यु॒क्ता हरी॑णाम् ।  अ॒र्वाङा या॑तु॒ वसु॑भी र॒श्मिरिन्द्रः॑ । प्रमꣳह॑माणो बहु॒लाꣳ श्रिय᳚म् ।  र॒श्मिरिन्द्रः॑ सवि॒ता मे॒ निय॑च्छतु । \textbf{ 26} \newline
                  \newline
                                                                  \textbf{ T.A.3.11.8} \newline
                  घृ॒तं तेजो॒ मधु॑मदिन्द्रि॒यम् । मय्य॒यम॒ग्निर्द॑धातु ॥ हरिः॑ पत॒ङ्गः प॑ट॒री सु॑प॒र्णः । दि॒वि॒क्षयो॒ नभ॑सा॒ य एति॑ ।  स न॒ इन्द्रः॑ कामव॒रं द॑दातु ॥ पञ्चा॑रं च॒क्रं परि॑वर्तते पृ॒थु । हिर॑ण्यज्योतिः सरि॒रस्य॒ मद्ध्ये᳚ । अज॑स्रं॒ ज्योति॒र्नभ॑सा॒ सर्प॑देति ।  स न॒ इन्द्रः॑ कामव॒रं द॑दातु ॥  स॒प्त यु॑ञ्जन्ति॒ रथ॒मेक॑चक्रम् \textbf{ 27} \newline
                  \newline
                                                                  \textbf{ T.A.3.11.9} \newline
                  एको॒ अश्वो॑ वहति सप्तना॒मा । त्रि॒नाभि॑ च॒क्रम॒जर॒मन॑र्वम् । येने॒मा विश्वा॒ भुव॑नानि तस्थुः ॥ भ॒द्रं पश्य॑न्त॒ उप॑सेदु॒रग्रे᳚ ।  तपो॑ दी॒क्षामृष॑यः सुव॒र्विदः॑ । ततः॑ क्ष॒त्रं बल॒मोज॑श्च जा॒तम् । तद॒स्मै दे॒वा अ॒भि सं न॑मन्तु ॥ श्वे॒तꣳ र॒श्मिं बो॑भु॒ज्यमा॑नम् ।  अ॒पां ने॒तारं॒ भुव॑नस्य गो॒पाम् ।  इन्द्रं॒ निचि॑क्युः पर॒मे व्यो॑मन्न् । \textbf{ 28} \newline
                  \newline
                                                                  \textbf{ T.A.3.11.10} \newline
                  रोहि॑णीः पिङ्ग॒ला एक॑रूपाः । क्षर॑न्तीः पिङ्ग॒ला एक॑रूपाः । श॒तꣳ स॒हस्रा॑णि प्र॒युता॑नि॒ नाव्या॑नाम् ॥ अ॒यं ॅयश्श्वे॒तो र॒श्मिः । परि॒ सर्व॑मि॒दं जग॑त् । प्र॒जां प॒शून् धना॑नि । अ॒स्माकं॑ ददातु ॥  श्वे॒तो र॒श्मिः परि॒ सर्वं॑ बभूव । सुव॒न्मह्यं॑ प॒शून्. वि॒श्वरू॑पान् ॥ प॒त॒ङ्गम॒क्तमसु॑रस्य मा॒यया᳚ \textbf{ 29} \newline
                  \newline
                                                                  \textbf{ T.A.3.11.11} \newline
                  हृ॒दा प॑श्यन्ति॒ मन॑सा मनी॒षिणः॑ । स॒मु॒द्रे अ॒न्तः क॒वयो॒ विच॑क्षते । मरी॑चीनां प॒दमि॑च्छन्ति वे॒धसः॑ ॥ प॒त॒ङ्गो वाचं॒ मन॑सा बिभर्ति ।  तां ग॑न्ध॒र्वो॑ऽवद॒द् गर्भे॑ अ॒न्तः । तां द्योत॑मानाꣳ स्व॒र्यं॑ मनी॒षाम् ।  ऋ॒तस्य॑ प॒दे क॒वयो॒ निपा᳚न्ति ॥ ये ग्रा॒म्याः प॒शवो॑ वि॒श्वरू॑पाः । विरू॑पाः॒ सन्तो॑ बहु॒धैक॑रूपाः । अ॒ग्निस्ताꣳ अग्रे॒ प्रमु॑मोक्तु दे॒वः \textbf{ 30} \newline
                  \newline
                                                                  \textbf{ T.A.3.11.12} \newline
                  प्र॒जाप॑तिः प्र॒जया॑ सम्ॅविदा॒नः ॥ वी॒तꣳ स्तु॑केस्तुके । यु॒वम॒स्मासु॒ निय॑च्छतम् । प्रप्र॑ य॒ज्ञ्प॑तिं तिर ॥ ये ग्रा॒म्याः प॒शवो॑ वि॒श्वरू॑पाः । विरू॑पाः॒ सन्तो॑ बहु॒धैक॑रूपाः ।  तेषाꣳ॑ सप्ता॒नामि॒ह रन्ति॑रस्तु ।  रा॒यस्पोषा॑य सुप्रजा॒स्त्वाय॑ सु॒वीर्या॑य ॥  य आ॑र॒ण्याः प॒शवो॑ वि॒श्वरू॑पाः ।  विरू॑पाः॒ सन्तो॑ बहु॒धैक॑रूपाः ( ) \textbf{ 31} \newline
                  \newline
                                                                  \textbf{ T.A.3.11.13} \newline
                  वा॒युस्ताꣳ अग्रे॒ प्रमु॑मोक्तु दे॒वः । प्र॒जाप॑तिः प्र॒जया॑ सम्ॅविदा॒नः ॥ इडा॑यै सृ॒प्तं घृ॒तव॑च्चराच॒रम् । दे॒वा अन्व॑विन्द॒न्गुहा॑ हि॒तम् ॥  य आ॑र॒ण्याः प॒शवो॑ वि॒श्वरू॑पाः । विरू॑पाः॒ सन्तो॑ बहु॒धैक॑रूपाः । तेषाꣳ॑ सप्ता॒नामि॒ह रन्ति॑रस्तु ।  रा॒यस्पोषा॑य सुप्रजा॒स्त्वाय॑ सु॒वीर्या॑य । \textbf{ 32} \newline
                  \newline
                                                        (आ॒त्मा जना॑नां - ॅविकु॒र्वन्तं॑ ॅविप॒श्विं - प्र॒जानां᳚ - ॅवसु॒धानीं᳚  ॅवि॒राजं॒ - चर॑न्तं॒ - गोम॑तीं - मे॒ निय॑च्छ॒ - त्वेक॑चक्रं॒ - ॅव्यो॑मन् - मा॒यया॑ - दे॒व - एक॑रूपा - +अ॒ष्टौ च॑) \textbf{11} \newline \newline
                                \textbf{ T.A.3.12.1} \newline
                  स॒हस्र॑शीर्.षा॒ पुरु॑षः । स॒ह॒स्रा॒क्षः स॒हस्र॑पात् । स भूमिं॑ ॅवि॒श्वतो॑ वृ॒त्वा । अत्य॑तिष्ठद्-दशाङ्गु॒लम् ॥ पुरु॑ष ए॒वेदꣳ सर्व᳚म् । यद्-भू॒तं ॅयच्च॒ भव्य᳚म् ।  उ॒तामृ॑त॒त्वस्येशा॑नः । यदन्ने॑नाति॒रोह॑ति ॥ ए॒तावा॑नस्य महि॒मा । अतो॒ ज्यायाꣳ॑श्च॒ पूरु॑षः \textbf{ 33} \newline
                  \newline
                                                                  \textbf{ T.A.3.12.2} \newline
                  पादो᳚ऽस्य॒ विश्वा॑ भू॒तानि॑ । त्रि॒पाद॑स्या॒मृतं॑ दि॒वि ॥  त्रि॒पादू॒र्द्ध्व उदै॒त्पुरु॑षः । पादो᳚ऽस्ये॒हाभ॑वा॒त् पुनः॑ । ततो॒ विष्व॒ङ् व्य॑क्रामत् । सा॒श॒ना॒न॒श॒ने अ॒भि ॥ तस्मा᳚द् वि॒राड॑जायत । वि॒राजो॒ अधि॒ पूरु॑षः ।  स जा॒तो अत्य॑रिच्यत । प॒श्चाद् भूमि॒मथो॑ पु॒रः । \textbf{ 34} \newline
                  \newline
                                                                  \textbf{ T.A.3.12.3} \newline
                  यत्पुरु॑षेण ह॒विषा᳚ । दे॒वा य॒ज्ञ्मत॑न्वत । व॒स॒न्तो अ॑स्यासी॒दाज्य᳚म् । ग्री॒ष्म इ॒द्ध्मः श॒रद्ध॒विः ॥ स॒प्तास्या॑सन् परि॒धयः॑ ।  त्रिः स॒प्त स॒मिधः॑ कृ॒ताः । दे॒वा यद्-य॒ज्ञ्ं त॑न्वा॒नाः । अब॑द्ध्न॒न् पुरु॑षं प॒शुम् ॥ तं ॅय॒ज्ञ्ं ब॒र्॒.हिषि॒ प्रौक्षन्न्॑ ।  पुरु॑षं जा॒तम॑ग्र॒तः \textbf{ 35} \newline
                  \newline
                                                                  \textbf{ T.A.3.12.4} \newline
                  तेन॑ दे॒वा अय॑जन्त । सा॒द्ध्या ऋष॑यश्च॒ ये ॥  तस्मा᳚द् य॒ज्ञाथ् स॑र्व॒हुतः॑ । सम्भृ॑तं पृषदा॒ज्यम् । प॒शूꣳस्ताꣳश्च॑क्रे वाय॒व्यान्॑ ।  आ॒र॒ण्यान् ग्रा॒म्याश्च॒ ये ॥ तस्मा᳚द् य॒ज्ञाथ् स॑र्व॒हुतः॑ । ऋचः॒ सामा॑नि जज्ञिरे । छन्दाꣳ॑सि जज्ञिरे॒ तस्मा᳚त् । यजु॒-स्तस्मा॑दजायत । \textbf{ 36} \newline
                  \newline
                                                                  \textbf{ T.A.3.12.5} \newline
                  तस्मा॒दश्वा॑ अजायन्त । ये के चो॑भ॒याद॑तः । गावो॑ ह जज्ञिरे॒ तस्मा᳚त् । तस्मा᳚ ज्जा॒ता अ॑जा॒वयः॑ ॥ यत् पुरु॑षं॒ ॅव्य॑दधुः । क॒ति॒धा व्य॑कल्पयन्न् । मुखं॒ किम॑स्य॒ कौ बा॒हू । कावू॒रू पादा॑वुच्येते ॥  ब्रा॒ह्म॒णो᳚ऽस्य॒ मुख॑मासीत् । बा॒हू रा॑ज॒न्यः॑ कृ॒तः \textbf{ 37} \newline
                  \newline
                                                                  \textbf{ T.A.3.12.6} \newline
                  ऊ॒रू तद॑स्य॒ यद्-वैश्यः॑ । प॒द्भ्याꣳ शू॒द्रो अ॑जायत ॥ च॒न्द्रमा॒ मन॑सो जा॒तः । चक्षोः॒ सूर्यो॑ अजायत ।  मुखा॒दिन्द्र॑श्चा॒ग्निश्च॑ । प्रा॒णाद् वा॒युर॑जायत ॥ नाभ्या॑ आसीद॒न्तरि॑क्षम् । शी॒र्ष्णो द्यौः सम॑वर्तत । प॒द्भ्यां भूमि॒र् दिशः॒ श्रोत्रा᳚त् ।  तथा॑ लो॒काꣳ अ॑कल्पयन्न् । \textbf{ 38} \newline
                  \newline
                                                                  \textbf{ T.A.3.12.7} \newline
                  वेदा॒हमे॒तं पुरु॑षं म॒हान्त᳚म् । आ॒दि॒त्यव॑र्णं॒ तम॑स॒स्तु पा॒रे । सर्वा॑णि रू॒पाणि॑ वि॒चित्य॒ धीरः॑ । नामा॑नि कृ॒त्वाऽभि॒वद॒न्॒. यदास्ते᳚ ॥  धा॒ता पु॒रस्ता॒द्यमु॑दाज॒हार॑ । श॒क्रः प्रवि॒द्वान् प्र॒दिश॒श्चत॑स्रः । तमे॒वं ॅवि॒द्वान॒मृत॑ इ॒ह भ॑वति । नान्यः पन्था॒ अय॑नाय विद्यते ॥  य॒ज्ञेन॑ य॒ज्ञ्म॑यजन्त दे॒वाः । तानि॒ धर्मा॑णि प्रथ॒मान्या॑सन्न् ( ) । ते ह॒ नाकं॑ महि॒मानः॑ सचन्ते । यत्र॒ पूर्वे॑ सा॒द्ध्याः सन्ति॑ दे॒वाः । \textbf{ 39} \newline
                  \newline
                                                  
                (ज्याया॒नधि॒ पूरु॑षः) (अ॒न्यत्र॒ पुरु॑षः ) \newline
                                      (पुरु॑षः - पु॒रो᳚ - ऽग्र॒तो॑ - ऽजायत - कृ॒तो॑ - ऽकल्पयन् - नास॒न् द्वे च॑) \textbf{12} \newline \newline
                                \textbf{ T.A.3.13.1} \newline
                  अ॒द्भ्यः सम्भू॑तः पृथि॒व्यै रसा᳚च्च । वि॒श्वक॑र्मणः॒ सम॑वर्त॒ताधि॑ ।  तस्य॒ त्वष्टा॑ वि॒दध॑द्-रू॒पमे॑ति । तत् पुरु॑षस्य॒ विश्व॒माजा॑न॒मग्रे᳚ ॥ वेदा॒हमे॒तं पुरु॑षं म॒हान्त᳚म् । आ॒दि॒त्यव॑र्णं॒ तम॑सः॒ पर॑स्तात् ।  तमे॒वं ॅवि॒द्वान॒मृत॑ इ॒ह भ॑वति । नान्यः पन्था॑ विद्य॒तेऽय॑नाय ॥ प्र॒जाप॑तिश्चरति॒ गर्भे॑ अ॒न्तः । अ॒जाय॑मानो बहु॒धा विजा॑यते \textbf{ 40} \newline
                  \newline
                                                                  \textbf{ T.A.3.13.2} \newline
                  तस्य॒ धीराः॒ परि॑जानन्ति॒ योनि᳚म् । मरी॑चीनां प॒दमि॑च्छन्ति वे॒धसः॑ ॥ यो दे॒वेभ्य॒ आत॑पति । यो दे॒वानां᳚ पु॒रोहि॑तः । पूर्वो॒ यो दे॒वेभ्यो॑ जा॒तः । नमो॑ रु॒चाय॒ ब्राह्म॑ये ॥ रुचं॑ ब्रा॒ह्मं ज॒नय॑न्तः । दे॒वा अग्रे॒ तद॑ब्रुवन्न् ।  यस्त्वै॒वं ब्रा᳚ह्म॒णो वि॒द्यात् । तस्य॑ दे॒वा अस॒न्वशे᳚ ( ) ॥ ह्रीश्च॑ ते ल॒क्ष्मीश्च॒ पत्न्यौ᳚ । अ॒हो॒रा॒त्रे पा॒र्श्वे । नक्ष॑त्राणि रू॒पम् । अ॒श्विनौ॒ व्यात्त᳚म् । इ॒ष्टं म॑निषाण । अ॒मुं म॑निषाण । सर्वं॑ मनिषाण । \textbf{ 41} \newline
                  \newline
                                                        (जा॒य॒ते॒ - वशे॑ स॒प्त च॑) \textbf{13} \newline \newline
                                \textbf{ T.A.3.14.1} \newline
                  भ॒र्ता सन् भ्रि॒यमा॑णो बिभर्ति । एको॑ दे॒वो ब॑हु॒धा निवि॑ष्टः । य॒दा भा॒रं त॒न्द्रय॑ते॒ स भर्तु᳚म् । नि॒धाय॑ भा॒रं पुन॒रस्त॑मेति ॥  तमे॒व मृ॒त्युम॒मृतं॒ तमा॑हुः । तं भ॒र्तारं॒ तमु॑ गो॒प्तार॑माहुः । स भृ॒तो भ्रि॒यमा॑णो बिभर्ति । य ए॑नं॒ ॅवेद॑ स॒त्येन॒ भर्तु᳚म् ॥ स॒द्यो जा॒तमु॒त ज॑हात्ये॒षः । उ॒तो जर॑न्तं॒ न ज॑हा॒त्येक᳚म् \textbf{ 42} \newline
                  \newline
                                                                  \textbf{ T.A.3.14.2} \newline
                  उ॒तो ब॒हूनेक॒मह॑र्जहार । अत॑न्द्रो दे॒वः सद॑मे॒व प्रार्थः॑ ॥ यस्तद्वेद॒ यत॑ आब॒भूव॑ । स॒न्धां च॒ याꣳ स॑न्द॒धे ब्रह्म॑णै॒षः ।  रम॑ते॒ तस्मि॑न्नु॒त जी॒र्णे शया॑ने । नैनं॑ जहा॒त्यह॑स्सु पू॒र्व्येषु॑ ॥ त्वामापो॒ अनु॒ सर्वा᳚श्चरन्ति जान॒तीः । व॒थ्सं पय॑सा पुना॒नाः ।  त्वम॒ग्निꣳ ह॑व्य॒वाहꣳ॒॒ समि᳚न्थ्से ।  त्वं भ॒र्ता मा॑त॒रिश्वा᳚ प्र॒जाना᳚म् । \textbf{ 43} \newline
                  \newline
                                                                  \textbf{ T.A.3.14.3} \newline
                  त्वं ॅय॒ज्ञ्स्त्वमु॑ वे॒वासि॒ सोमः॑ । तव॑ दे॒वा हव॒माय॑न्ति॒ सर्वे᳚ । त्वमेको॑ऽसि ब॒हूननु॒प्रवि॑ष्टः । नम॑स्ते अस्तु सु॒हवो॑ म एधि ॥  नमो॑ वामस्तु शृणु॒तꣳ हवं॑ मे । प्राणा॑पानावजि॒रꣳ स॒चंर॑न्तौ ।  ह्वया॑मि वां॒ ब्रह्म॑णा तू॒र्तमेत᳚म् । यो मां द्वेष्टि॒ तं ज॑हि तं ॅयुवाना ॥  प्राणा॑पानौ सम्ॅविदा॒नौ ज॑हितम् ।  अ॒मुष्यासु॑ना॒ मा संग॑साथां ( ) \textbf{ 44} \newline
                  \newline
                                                                  \textbf{ T.A.3.14.4} \newline
                  तं मे॑ देवा॒ ब्रह्म॑णा सम्ॅविदा॒नौ ।  व॒धाय॑ दत्तं॒ तम॒हꣳ ह॑नामि ॥ अस॑ज्जजान स॒त आब॑भूव । यं ॅयं॑ ज॒जान॒ स उ॑ गो॒पो अ॑स्य ।  य॒दा भा॒रं त॒न्द्रय॑ते॒ स भर्तु᳚म् । प॒रास्य॑ भा॒रं पुन॒रस्त॑मेति ॥  तद्वै त्वं प्रा॒णो अ॑भवः । म॒हान्भोगः॑ प्र॒जाप॑तेः । भुजः॑ करि॒ष्यमा॑णः । यद्दे॒वान्प्राण॑यो॒ नव॑ ( ) । \textbf{ 45} \newline
                  \newline
                                                        (एकं॑ - प्र॒जानां᳚ - गसाथां॒ - नव॑ ) \textbf{14} \newline \newline
                                \textbf{ T.A.3.15.1} \newline
                  हरिꣳ॒॒ हर॑न्त॒मनु॑यन्ति दे॒वाः । विश्व॒स्येशा॑नं ॅवृष॒भं म॑ती॒नाम् । ब्रह्म॒ सरू॑प॒मनु॑ मे॒दमागा᳚त् । अय॑नं॒ मा विव॑धी॒र् विक्र॑मस्व ॥  माच्छि॑दो मृत्यो॒ मा व॑धीः । मा मे॒ बलं॒ ॅविवृ॑हो॒ मा प्रमो॑षीः । प्र॒जां मा मे॑ रीरिष॒ आयु॑रुग्र । नृ॒चक्ष॑सं त्वा ह॒विषा॑ विधेम ॥  स॒द्यश्च॑कमा॒नाय॑ । प्र॒वे॒पा॒नाय॑ मृ॒त्यवे᳚ \textbf{ 46} \newline
                  \newline
                                                                  \textbf{ T.A.3.15.2} \newline
                  प्रास्मा॒ आशा॑ अशृण्वन्न् । कामे॑नाजनय॒न् पुनः॑ ॥  कामे॑न मे॒ काम॒ आगा᳚त् । हृद॑या॒द्द्धृद॑यं मृ॒त्योः । यद॒मीषा॑म॒दः प्रि॒यम् । तदैतूप॒ माम॒भि ॥ परं॑ मृत्यो॒ अनु॒ परे॑हि॒ पन्था᳚म् । यस्ते॒ स्व इत॑रो देव॒याना᳚त् । चक्षु॑ष्मते शृण्व॒ते ते᳚ ब्रवीमि । मा नः॑ प्र॒जाꣳ री॑रिषो॒ मोत वी॒रान् ( ) ॥  प्र पू॒र्व्यं मन॑सा॒ वन्द॑मानः । नाध॑मानो वृष॒भं च॑र्.षणी॒नाम् । यः प्र॒जाना॑मेक॒राण्मानु॑षीणाम् ।  मृ॒त्युं ॅय॑जे प्रथम॒जामृ॒तस्य॑ । \textbf{ 47} \newline
                  \newline
                                                        (मृ॒त्यवे॑ - वी॒राꣳश्च॒त्वारि॑ च) \textbf{15} \newline \newline
                                \textbf{ T.A.3.16.1} \newline
                  त॒रणि॑र् वि॒श्वद॑र्.शतो ज्योति॒ष्कृद॑सि सूर्य । विश्व॒माभा॑सि रोच॒नम् ।  उ॒प॒या॒मगृ॑हीतोऽसि॒ सूर्या॑य त्वा॒ भ्राज॑स्वत ए॒ष ते॒ योनिः॒ सूर्या॑य त्वा॒ भ्राज॑स्वते । \textbf{ 48} \newline
                  \newline
                                                        (णॊ खॊर्वै fऒर् थिस् आनुवाकम्) \textbf{16} \newline \newline
                                \textbf{ T.A.3.17.1} \newline
                  आ प्या॑यस्व मदिन्तम॒ सोम॒ विश्वा॑भिरू॒तिभिः॑ ।  भवा॑ नः स॒प्रथ॑स्तमः । \textbf{ 49} \newline
                  \newline
                                                        (णॊ खॊर्वै fऒर् थिस् आनुवाकम्) \textbf{17} \newline \newline
                                \textbf{ T.A.3.18.1} \newline
                  ई॒युष्टे ये पूर्व॑तरा॒मप॑श्यन् व्यु॒च्छन्ती॑मु॒षसं॒ मर्त्या॑सः ।  अ॒स्माभि॑रू॒ नु प्र॑ति॒चक्ष्या॑ऽभू॒दो ते य॑न्ति॒ ये अ॑प॒रीषु॒ पश्यान्॑ । \textbf{ 50} \newline
                  \newline
                                                        (णॊ खॊर्वै fऒर् थिस् आनुवाकम्) \textbf{18} \newline \newline
                                \textbf{ T.A.3.19.1} \newline
                  ज्योति॑ष्मतीं त्वा सादयामि ज्योति॒ष्कृतं॑ त्वा सादयामि ज्योति॒र्विदं॑ त्वा सादयामि॒ भास्व॑तीं त्वा सादयामि॒ ज्वल॑न्तीं त्वा सादयामि मल्मला॒भव॑न्तीं त्वा सादयामि॒ दीप्य॑मानां त्वा सादयामि॒ रोच॑मानां  त्वा सादया॒म्यज॑स्रां त्वा सादयामि बृ॒हज्ज्यो॑तिषं त्वा सादयामि बो॒धय॑न्तीं त्वा सादयामि॒ जाग्र॑तीं त्वा सादयामि । \textbf{ 51} \newline
                  \newline
                                                        (णॊ खॊर्वै fऒर् थिस् आनुवाकम्) \textbf{19} \newline \newline
                                \textbf{ T.A.3.20.1} \newline
                  प्र॒या॒साय॒ स्वाहा॑ऽऽया॒साय॒ स्वाहा॑ विया॒साय॒ स्वाहा॑ सम्ॅया॒साय॒ स्वाहो᳚द्या॒साय॒ स्वाहा॑ऽवया॒साय॒ स्वाहा॑ शु॒चे स्वाहा॒ शोका॑य॒ स्वाहा॑ तप्य॒त्वै स्वाहा॒ तप॑ते॒ स्वाहा᳚ ब्रह्मह॒त्यायै॒ स्वाहा॒ सर्व॑स्मै॒ स्वाहा᳚ । \textbf{ 52} \newline
                  \newline
                                                        (णॊ खॊर्वै fऒर् थिस् आनुवाकम्) \textbf{20} \newline \newline
                                \textbf{ T.A.3.21.1} \newline
                  चि॒त्तꣳ स॑न्ता॒नेन॑ भवं ॅय॒क्ना रु॒द्रं तनि॑म्ना पशु॒पतिꣳ॑ स्थूलहृद॒येना॒ग्निꣳ हृद॑येन रु॒द्रं ॅलोहि॑तेन श॒र्वं मत॑स्नाभ्यां  महादे॒वम॒न्तः पा᳚र्श्वेनौषिष्ठ॒हनꣳ॑ शिङ्गीनिको॒श्या᳚भ्याम् । \textbf{ 53} \newline
                  \newline
                                                        (णॊ खॊर्वै fऒर् थिस् आनुवाकम्) \textbf{21} \newline \newline
\textbf{Prapaataka Korvai with starting Padams of 1 to21 Anuvaakams :-} \newline
(चित्तिः॑ - पृथि॒ - व्य॑ग्निः॒ - सूर्यं॑ ते॒ चक्षु॑र् - म॒हाह॑वि॒र्.॒ होता॒ - वाग्घोता᳚ - ब्राह्म॒ण एक॑होता॒ - ऽग्निर्यजु॑र्भिः॒ - सेनेन्द्र॑स्य - दे॒वस्य॑ - सु॒वर्णं॑ घ॒र्मꣳ - स॒हस्र॑शीर्.षा॒- ऽद्भ्यो - भ॒र्ता - हरिं॑ - त॒रणि॒ - राप्या॑य - स्वे॒युष्टे ये - ज्योति॑ष्मतीं - प्रया॒साय॑ - चि॒त्त मेक॑विꣳशतिः) \newline

\textbf{korvai with starting padams of1, 11, 21 Series of Dasinis :-} \newline
(चित्ति॑ - र॒ग्निर्यजु॑र्भि - र॒न्तः प्रवि॑ष्टः - प्र॒जाप॑ति॒ - स्तस्य॒ धीरा॒ - ज्योति॑ष्मतीं॒ त्रिप॑ञ्चा॒शत् ) \newline

\textbf{first and last padam in TA, 3rd Prapaatakam :-} \newline
( चित्तिः॑ - शिङ्गीनिको॒श्या᳚भ्यां ) \newline 


॥ कृष्ण यजुर्वेदीय तैत्तिरीय ब्राह्मणे आरण्यके तृतीयः प्रपाठकः समाप्तः ॥ \newline
\pagebreak
\pagebreak
        


\end{document}
