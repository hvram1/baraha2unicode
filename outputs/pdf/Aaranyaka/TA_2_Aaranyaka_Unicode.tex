\documentclass[17pt]{extarticle}
\usepackage{babel}
\usepackage{fontspec}
\usepackage{polyglossia}
\usepackage{extsizes}



\setmainlanguage{sanskrit}
\setotherlanguages{english} %% or other languages
\setlength{\parindent}{0pt}
\pagestyle{myheadings}
\newfontfamily\devanagarifont[Script=Devanagari]{AdishilaVedic}


\newcommand{\VAR}[1]{}
\newcommand{\BLOCK}[1]{}




\begin{document}
\begin{titlepage}
    \begin{center}
 
\begin{sanskrit}
    { \Large
    ॐ नमः परमात्मने, श्री महागणपतये नमः
श्री गुरुभ्यो नमः, ह॒रिः॒ ॐ 
    }
    \\
    \vspace{2.5cm}
    \mbox{ \Huge
    कृष्ण यजुर्वेदीय तैत्तिरीय आरण्यके द्वितीयः प्रपाठकः   }
\end{sanskrit}
\end{center}

\end{titlepage}
\tableofcontents

ॐ नमः परमात्मने, श्री महागणपतये नमः
श्री गुरुभ्यो नमः, ह॒रिः॒ ॐ \newline
2.1     द्वितीयः प्रपाठकः - स्वाध्यायब्राह्मणं \newline

\addcontentsline{toc}{section}{ 2.1     द्वितीयः प्रपाठकः - स्वाध्यायब्राह्मणं}
\markright{ 2.1     द्वितीयः प्रपाठकः - स्वाध्यायब्राह्मणं \hfill https://www.vedavms.in \hfill}
\section*{ 2.1     द्वितीयः प्रपाठकः - स्वाध्यायब्राह्मणं }
                                \textbf{ T.A.2.1.1} \newline
                  सह॒ वै दे॒वानां॒ चासु॑राणां च य॒ज्ञु प्रत॑तावास्तां ॅव॒यꣳ स्व॒र्गं ॅलो॒कमे᳚ष्यामो व॒यमे᳚ष्याम॒ इति॒ तेऽसु॑राः स॒न्नह्य॒ सह॑सै॒वाच॑रन् ब्रह्म॒चर्ये॑ण॒ तप॑सैव दे॒वास्तेऽसु॑रा अमुह्यꣳ॒॒स्ते न प्राजा॑नꣳ॒॒स्ते परा॑भव॒न्ते न स्व॒र्गं ॅलो॒कमा॑य॒न्, प्रसृ॑तेन॒ वै य॒ज्ञेन॑ दे॒वाः स्व॒र्गं ॅलो॒कमा॑य॒न् न प्रसृ॑ते॒नासु॑रा॒न् परा॑ऽभावय॒न् ,प्रसृ॑तो ह॒ वै य॑ज्ञोपवी॒तिनो॑ य॒ज्ञोऽप्र॑सृ॒तोऽनु॑पवी॒तिनो॒ यत्किञ्च॑ ब्राह्म॒णो य॑ज्ञोपवी॒त्यधी॑ते॒ यज॑त ए॒व तत् तस्मा᳚द् यज्ञोपवी॒त्ये॑वाधी॑यीत या॒जये॒द् यजे॑त वा य॒ज्ञ्स्य॒ प्रसृ॑त्या॒ , अजि॑नं॒ ॅवासो॑ वा दक्षिण॒त उ॑प॒वीय॒ दक्षि॑णं बा॒हुमुद्ध॑र॒तेऽव॑ धत्ते स॒व्यमिति॑ यज्ञोपवी॒तमे॒तदे॒व विप॑रीतं प्राचीनावी॒तꣳ स॒म्ॅवीतं॑ मानु॒षम् । \textbf{ 1} \newline
                  \newline
                                                        (णो खॊर्वै fऒर् थिस् आनुवाकम्) \textbf{1} \newline \newline
                                \textbf{ T.A.2.2.1} \newline
                  रक्षाꣳ॑सि॒ हवा॑ पुरोनुवा॒के तपोऽग्र॑मतिष्ठन्त॒ तान् प्र॒जाप॑तिर् व॒रेणो॒पाम॑न्त्रयत॒ तानि॒ वर॑मवृणीतादि॒त्यो नो॒ योद्धा॒ इति॒ तान् प्र॒जाप॑ति-रब्रवी॒द्-योध॑य॒द्ध्वमिति॒ तस्मा॒दुत्ति॑ष्ठन्तꣳ॒॒ हवा॒ तानि॒ रक्षाꣳ॑स्यादि॒त्यं ॅयोध॑यन्ति॒ याव॑दस्त॒मन्व॑गा॒त्  तानि॑ ह॒ वा ए॒तानि॒ रक्षाꣳ॑सि गायत्रि॒या ऽभि॑मन्त्रिते॒नांभ॑सा शाम्यन्ति॒ तदु॑ ह॒ वा ए॒ते ब्र॑ह्मवा॒दिनः॑ पू॒र्वाभि॑मु॒खाः स॒न्ध्यायां᳚ गायत्रि॒याऽभि॑मन्त्रिता॒ आप॑ ऊ॒र्द्ध्वं ॅविक्षि॑पन्ति॒ ता ए॒ता आपो॑ व॒ज्रीभू॒त्वा तानि॒ रक्षाꣳ॑सि  म॒न्देहाऽरु॑णे द्वी॒पे प्रक्षि॑पन्ति॒ यत् प्र॑दक्षि॒णं प्रक्र॑मन्ति॒ तेन॑ पा॒प्मान॒मव॑ धून्वन्त्यु॒द्यन्त॑मस्तं॒ ॅयन्त॑मादि॒त्य-म॑भिद्ध्या॒यन् कु॒र्वन् ब्रा᳚ह्म॒णो वि॒द्वान्थ् स॒कलं॑ भ॒द्रम॑श्नुते॒ ऽसावा॑दि॒त्यो ब्र॒ह्मेति॒ ब्रह्मै॒व सन् ब्रह्मा॒प्येति॒ य ए॒वं ॅवेद॑ । \textbf{ 2} \newline
                  \newline
                                                        (णो खॊर्वै fऒर् थिस् आनुवाकम्) \textbf{2} \newline \newline
                                \textbf{ T.A.2.3.1} \newline
                  यद्-दे॑वा देव॒हेड॑नं॒ (देव॒हेळ॑नं॒) देवा॑सश्चकृ॒मा व॒यम् ।  आदि॑त्या॒स्तस्मा᳚न्मा मुञ्चत॒र्तस्य॒र्तेन॒ मामि॒त ॥  देवा॑ जीवनका॒म्या यद्-वा॒चाऽनृ॑तमूदि॒म ।  तस्मा᳚न्न इ॒ह मु॑ञ्चत॒ विश्वे॑ देवाः स॒जोष॑सः ॥  ऋ॒तेन॑ द्यावापृथिवी ऋ॒तेन॒ त्वꣳ स॑रस्वति ।  कृ॒तान्नः॑ पा॒ह्येन॑सो॒ यत् किञ्चानृ॑तमूदि॒म ॥  इ॒न्द्रा॒ग्नी मि॒त्रावरु॑णौ॒ सोमो॑ धा॒ता बृह॒स्पतिः॑ ।  ते नो॑ मुञ्च॒न्त्वेन॑सो॒ यद॒न्यकृ॑तमारि॒म ॥  स॒जा॒त॒शꣳ॒॒सादु॒त जा॑मिशꣳ॒॒ साज्ज्याय॑सः॒ शꣳसा॑दु॒त वा॒ कनी॑यसः ।  अना॑धृष्टं दे॒वकृ॑तं॒ ॅयदेन॒स्तस्मा॒त् त्वम॒स्माञ् जा॑तवेदो मुमुग्धि ( ) ॥  यद्वा॒चा यन् मन॑सा बा॒हुभ्या॑-मू॒रुभ्या॑मष्ठी॒वद्भ्या॑ शि॒श्नैर्यदनृ॑तं चकृ॒मा व॒यम् ।  अ॒ग्निर्मा॒ तस्मा॒देन॑सा॒ गार्.ह॑पत्यः॒ प्रमु॑ञ्चतु चकृ॒म यानि॑ दुष्कृ॒ता ॥  येन॑ त्रि॒तो अ॑र्ण॒वान् नि॑र्ब॒भूव॒ येन॒ सूर्यं॒ तम॑सो निर्मु॒मोच॑ ।  येनेन्द्रो॒ विश्वा॒ अज॑हा॒दरा॑ती॒स्तेना॒हं ज्योति॑षा॒ ज्योति॑रानशा॒न आ᳚क्षि ॥  यत् कुसी॑द॒मप्र॑तीत्तं॒ मये॒ह येन॑ य॒मस्य॑ नि॒धिना॒ चरा॑मि ।  ए॒तत्तद॑ग्ने अनृ॒णो भ॑वामि॒ जीव॑न्ने॒व प्रति॒ तत्ते॑ दधामि ॥  "यन्मयि॑ मा॒ता{17}" "यदा॑ पि॒पेष॒{18}" "यद॒न्तरि॑क्षं॒{19}"  " ॅयदा॒शसा{20} "ऽति॑क्रामामि{21}" "त्रि॒ते दे॒वा{22}"  "दि॒वि जा॒ता{23}" "यदाप॑{24}" "इ॒मं मे॑ वरुण॒{25}"  "तत्त्वा॑ यामि॒{26}" "त्वं नो॑ अग्ने॒{27}" "सत्वं नो॑ अग्ने॒{28}" "त्वम॑ग्ने अ॒यासि॑{29}"। \textbf{ 3} \newline
                  \newline
                                                        (मु॒मु॒ग्धि॒ स॒प्त च॑) \textbf{3} \newline \newline
                                \textbf{ T.A.2.4.1} \newline
                  यददी᳚व्यन्नृ॒णम॒हं ब॒भूवादि॑थ् सन्वा संज॒गर॒ जने᳚भ्यः । अ॒ग्निर्मा॒ तस्मा॒दिन्द्र॑श्च सम्ॅविदा॒नौ प्रमु॑ञ्चताम् ॥ यद्धस्ता᳚भ्यां च॒कर॒ किल्बि॑षाण्य॒क्षाणां᳚ ॅव॒ग्नुमु॑प॒जिघ्न॑मानः ।  उ॒ग्र॒म्प॒श्या च॑ राष्ट्र॒भृच्च॒ तान्य॑फ्स॒रसा॒ वनु॑दत्ता-मृ॒णानि॑ ॥ उग्र॑पंश्ये॒ राष्ट्र॑भृ॒त् किल्बि॑षाणि॒ यद॒क्षवृ॑त्त॒मनु॑दत्तमे॒तत् ।  नेन्न॑ ऋ॒णानृ॒णव॒ इथ्स॑मानो य॒मस्य॑ लो॒के अधि॑ रज्जु॒राय॑ ॥ "अव॑ ते॒ हेड॒ (हेळ॒)  {30}" "उदु॑त्त॒म{31}" "मि॒मं मे॑ वरुण॒{32}"  "तत्त्वा॑ यामि॒{33}" "त्वं नो॑ अग्ने॒{34}" "सत्वं नो॑ अग्ने{36} ॥ संकु॑सुको॒ विकु॑सुको निर्.ऋ॒थो यश्च॑ निस्व॒नः । तेऽ(1॒)स्मद्-यक्ष्म॒मना॑गसो दू॒राद्-दू॒रम॑चीचतम् ॥ निर्य॑क्ष्ममचीचते कृ॒त्यां निर्.ऋ॑तिं च ( ) । तेन॒ योऽ(1॒)स्मथ्समृ॑च्छातै॒ तम॑स्मै॒ प्रसु॑वामसि ॥ दु॒श्शꣳ॒॒सा॒नु॒शꣳ॒॒साभ्यां᳚ घ॒णेना॑नुघ॒णेन॑ च । तेना॒न्योऽ(1)॒स्मथ् समृ॑च्छातै॒ तम॑स्मै॒ प्रसु॑वामसि ॥  सम्ॅवर्च॑सा॒ पय॑सा॒ सन्त॒नूभि॒रग॑न्महि॒ मन॑सा॒ सꣳ शि॒वेन॑ । त्वष्टा॑ नो॒ अत्र॒ विद॑धातु रा॒योऽनु॑मार्ष्टु त॒न्वो(1॒) यद्-विलि॑ष्टम् । \textbf{ 4} \newline
                  \newline
                                                        (कृ॒त्यां निर्.ऋ॑तिं च॒ पञ्च॑ च) \textbf{4} \newline \newline
                                \textbf{ T.A.2.5.1} \newline
                  आयु॑ष्टे वि॒श्वतो॑ दधद॒यम॒ग्निर्वरे᳚ण्यः ।  पुन॑स्ते प्रा॒ण आया॑ति॒ परा॒ यक्ष्मꣳ॑ सुवामि ते ॥  आ॒यु॒र्दा अ॑ग्ने ह॒विषो॑ जुषा॒णो घृ॒तप्र॑तीको घृ॒तयो॑निरेधि । घृ॒तं पी॒त्वा मधु॒ चारु॒ गव्यं॑ पि॒तेव॑ पु॒त्रम॒भि र॑क्षतादि॒मम् ॥  इ॒मम॑ग्न॒ आयु॑षे॒ वर्च॑से कृधि ति॒ग्ममोजो॑ वरुण॒ सꣳशि॑शाधि । मा॒तेवा᳚स्मा अदिते॒ शर्म॑ यच्छ॒ विश्वे॑ देवा॒ जर॑दष्टि॒र् यथाऽस॑त् ॥  अग्न॒ आयूꣳ॑षि पवस॒ आसु॒वोर्ज॒मिषं॑ च नः ।  आ॒रे बा॑धस्व दु॒च्छुना᳚म् ॥ अग्ने॒ पव॑स्व॒ स्वपा॑ अ॒स्मे वर्चः॑ सु॒वीर्य᳚म् । दध॑द् र॒यिं मयि॒ पोष᳚म् । \textbf{ 1-3} \newline
                  \newline
                                                        (पोषं॑ - दद्ध्मसि - पु॒रोहि॑तश्च॒त्वारि॑ च) \textbf{5} \newline \newline
                                \textbf{ T.A.2.6.1} \newline
                  वै॒श्वा॒न॒राय॒ प्रति॑वेदयामो॒ यदी॑नृ॒णꣳ स॑ङ्ग॒रो दे॒वता॑सु । स ए॒तान् पाशा᳚न् प्र॒मुच॒न् प्रवे॑द॒ स नो॑ मुञ्चातु दुरि॒तादव॒द्यात् ॥  वै॒श्वा॒न॒रः पव॑यान्नः प॒वित्रै॒र्यथ् स॑ङ्ग॒र-म॒भिधावा᳚म्या॒शाम् । अना॑जान॒न् मन॑सा॒ याच॑मानो॒ यदत्रैनो॒ अव॒ तथ् सु॑वामि ॥  अ॒मी ये सु॒भगे॑ दि॒वि वि॒चृतौ॒ नाम॒ तार॑के ।  प्रेहामृत॑स्य यच्छता-मे॒तद्-ब॑द्धक॒मोच॑नम् ॥  विजि॑हीर्ष्व लो॒कान् कृ॑धि ब॒न्धान् मु॑ञ्चासि॒ बद्ध॑कम् । योने॑रिव॒ प्रच्यु॑तो॒ गर्भः॒ सर्वा᳚न् प॒थो अ॑नुष्व ॥  स प्र॑जा॒नन् प्रति॑गृभ्णीत वि॒द्वान् प्र॒जाप॑तिः प्रथम॒जा ऋ॒तस्य॑ । अ॒स्माभि॑र्द॒त्तं ज॒रसः॑ प॒रस्ता॒दच्छि॑न्नं॒ तन्तु॑मनु॒ संच॑रेम । \textbf{ 1-2} \newline
                  \newline
                                                        (च॒रे॒म॒ - पु॒त्रꣳ षट्च॑) \textbf{6} \newline \newline
                                \textbf{ T.A.2.7.1} \newline
                  वात॑रशना ह॒ वा ऋष॑यः श्रम॒णा ऊ॒र्द्ध्वम॑न्थि॒नो ब॑भूवु॒स्ता-नृष॑यो॒ऽर्थ-मा॑यꣳ॒॒स्ते नि॒लाय॑-मचरꣳ॒॒स्ते-ऽनु॑प्रविशुः कूश्मा॒ण्डानि॒ ताꣳ-स्तेष्वन्व॑विन्द-ञ्छ्र॒द्धया॑ च॒ तप॑सा च॒ , तानृष॑योऽब्रुवन् क॒था नि॒लायं॑ चर॒थेति॒ त ऋषी॑नब्रुव॒न्नमो॑ वोऽस्तु भगवन्तो॒ऽस्मिन् धा᳚म्नि॒ केन॑ वः सपर्या॒मेति॒ तानृष॑योऽब्रुवन् प॒वित्रं॑ नो ब्रूत॒ येना॑रे॒पसः॑ स्या॒मेति॒ त ए॒तानि॑ सू॒क्तान्य॑पश्य॒न्॒. , यद्-दे॑वा देव॒हेड॑नं॒ (देव॒हेळ॑नं॒) ॅयददी᳚व्यन्नृ॒णम॒हं ब॒भूवायु॑ष्टे वि॒श्वतो॑  दध॒दित्ये॒तैराज्यं॑ जुहुत वैश्वान॒राय॒ प्रति॑वेदयाम॒ इत्युप॑तिष्ठत॒ यद॑र्वा॒चीन॒मेनो᳚ भ्रूणह॒त्याया॒स्तस्मा᳚न् मोक्ष्यद्ध्व॒ इति॒ ,त ए॒तैर॑जुहवु॒स्तेऽरे॒पसो॑ऽभवन् कर्मा॒दिष्वे॒तैर्-जु॑हुयात् पू॒तो दे॑वलो॒कान् थ्सम॑श्नुते । \textbf{ 7} \newline
                  \newline
                                                        (णो खॊर्वै fऒर् थिस् आनुवाकम्) \textbf{7} \newline \newline
                                \textbf{ T.A.2.8.1} \newline
                  कू॒श्मा॒ण्डैर् जु॑हुया॒द्-योऽपू॑त इव॒ मन्ये॑त॒ यथा᳚ स्ते॒नो यथा᳚ भ्रूण॒हैवमे॒ष भ॑वति॒ योऽयोनौ॒ रेतः॑ सि॒ञ्चति॒, यद॑र्वा॒चीन॒मेनो᳚ भ्रूणह॒त्याया॒स्तस्मा᳚न् मुच्यते॒ याव॒देनो॑ दी॒क्षामुपै॑ति दीक्षि॒त ए॒तैः स॑त॒ति जु॑होति सम्ॅवथ्स॒रं दी᳚क्षि॒तो भ॑वति सम्ॅवथ्स॒रादे॒वात्मानं॑ पुनीते॒ मासं॑ दीक्षि॒तो भ॑वति॒ यो मासः॒ स सम्ॅवथ्स॒रः स॑म्ॅवथ्स॒रादे॒वात्मानं॑ पुनीते॒ चतु॑र्विꣳशतिꣳ॒॒ रात्री᳚र्दीक्षि॒तो भ॑वति॒ चतु॑र्विꣳशति-रर्द्धमा॒साः स॑म्ॅवथ्स॒रः स॑म्ॅवथ्स॒रादे॒वात्मानं॑ पुनीते॒ द्वाद॑श॒ रात्री᳚र्दीक्षि॒तो भ॑वति॒ द्वाद॑श॒ मासाः᳚ सम्ॅवथ्स॒रः स॑म्ॅवथ्स॒रादे॒वात्मानं॑ पुनीते॒ षड्रात्री᳚र्दीक्षि॒तो भ॑वति॒ षड्वा ऋ॒तवः॑ सम्ॅवथ्स॒रः स॑म्ॅवथ्स॒रादे॒वात्मानं॑ पुनीते ति॒स्रो रात्री᳚र्-दीक्षि॒तो भ॑वति त्रि॒पदा॑ गाय॒त्री गा॑यत्रि॒या ए॒वात्मानं॑ पुनीते॒ न माꣳ॒॒सम॑श्नीया॒न्न स्त्रिय॒मुपे॑या॒न्नोपर्या॑सीत॒ जुगु॑फ्से॒तानृ॑ता॒त्पयो᳚ ब्राह्म॒णस्य॑ व्र॒तं ॅय॑वा॒गू रा॑ज॒न्य॑स्या॒मिक्षा॒ वैश्य॒स्याथो॑ सौ॒म्येऽप्य॑द्ध्व॒र ए॒तद्व्र॒तं ब्रू॑या॒द् यदि॒ मन्ये॑-तोप॒दस्या॒मीत्यो॑द॒नं धा॒नाः सक्तूं᳚ घृ॒तमित्यनु॑व्रतयेदा॒त्मनोऽनु॑पदासाय । \textbf{ 8} \newline
                  \newline
                                                        (णो खॊर्वै fऒर् थिस् आनुवाकम्) \textbf{8} \newline \newline
                                \textbf{ T.A.2.9.1} \newline
                  अ॒जान्. ह॒ वै पृश्नीꣳ॑स्तप॒स्यमा॑ना॒न् ब्रह्म॑ स्वयं॒भ्व॑भ्यान॑र्.ष॒त्त ऋष॑योऽभव॒न् तदृषी॑णा-मृषि॒त्वं तां दे॒वता॒मुपा॑तिष्ठन्त य॒ज्ञ्का॑मा॒स्त ए॒तं ब्र॑ह्मय॒ज्ञ्-म॑पश्य॒न्-तमाऽह॑र॒न्-तेना॑यजन्त॒ यदृ॒चोऽद्ध्यगी॑षत॒ ताः पय॑ आहुतयो दे॒वाना॑मभव॒न्॒. यद्-यजूꣳ॑षि घृ॒ताहु॑तयो॒ यथ् सामा॑नि॒ सोमा॑हुतयो॒ यदथ॑र्वाङ्गि॒रसो॒ मद्ध्वा॑हुतयो॒ यद्ब्रा᳚ह्म॒णानी॑तिहा॒सान्-पु॑रा॒णानि॒ कल्पा॒न् गाथा॑ नाराशꣳ॒॒सीर्मे॑दाहु॒तयो॑ दे॒वाना॑-मभव॒न्ताभिः॒ क्षुधं॑ पा॒प्मान॒-मपा᳚घ्न॒न्-नप॑हतपाप्मानो दे॒वाः स्व॒र्गं ॅलो॒कमा॑य॒न् ब्रह्म॑णः॒ सायु॑ज्य॒-मृष॑योऽगच्छन्न् । \textbf{ 9} \newline
                  \newline
                                                        (णो खॊर्वै fऒर् थिस् आनुवाकम्) \textbf{9} \newline \newline
                                \textbf{ T.A.2.10.1} \newline
                  पञ्च॒ वा ए॒ते म॑हाय॒ज्ञाः स॑त॒ति प्रता॑यन्ते सत॒ति सन्ति॑ष्ठन्ते देवय॒ज्ञ्ः पि॑तृय॒ज्ञो भू॑तय॒ज्ञो म॑नुष्यय॒ज्ञो ब्र॑ह्मय॒ज्ञ् इति॒ यद॒ग्नौ जु॒होत्य॒पि स॒मिधं॒ तद्-दे॑वय॒ज्ञ्ः सन्ति॑ष्ठते॒ यत् पि॒तृभ्यः॑ स्व॒धा क॒रोत्यप्य॒पस्तत् पि॑तृय॒ज्ञ्ः सन्ति॑ष्ठते॒यद्-भू॒तेभ्यो॑ ब॒लिꣳ हर॑ति॒ तद्-भू॑तय॒ज्ञ्ः सन्ति॑ष्ठते॒ यद्ब्रा᳚ह्म॒णेभ्योऽन्नं॒ ददा॑ति॒ तन्म॑नुष्यय॒ज्ञ्ः सन्ति॑ष्ठते॒ यथ्स्वा᳚द्ध्या॒य-मधी॑यी॒-तैका॑मप्यृ॒चं ॅयजुः॒-साम॑ वा॒ तद्ब्र॑ह्मय॒ज्ञ्ः सन्ति॑ष्ठते॒ यदृ॒चोऽधी॑ते॒ पय॑सः॒ कूल्या॑ अस्य पि॒तॄन् थ्स्व॒धा अ॒भिव॑हन्ति॒ यद्-यजूꣳ॑षि घृ॒तस्य॑ कूल्या॒ यथ् सामा॑नि॒ सोम॑ एभ्यः पवते॒ यदथ॑र्वाङ्गि॒रसो॒ मधोः᳚ कूल्या॒ यद्ब्रा᳚ह्म॒णा-नी॑तिहा॒सान्-पु॑रा॒णानि॒ कल्पा॒न् गाथा॑ नाराशꣳ॒॒सीर्-मेद॑सः॒ कूल्या॑ अस्य पि॒तॄन्थ् स्व॒धा अ॒भिव॑हन्ति॒ यदृ॒चोऽधी॑ते॒ पय॑ आहुतिभिरे॒व तद्-दे॒वाꣳस्त॑र्पयति॒ यद्-यजूꣳ॑षि घृ॒ताहु॑तिभि॒र्यथ् सामा॑नि॒ सोमा॑हुतिभि॒र्-यदथ॑र्वाङ्गि॒रसो॒ मद्ध्वा॑हुतिभि॒र्-यद्ब्रा᳚ह्म॒णा-नी॑तिहा॒सान्-पु॑रा॒णानि॒ कल्पा॒न् गाथा॑ नाराशꣳ॒॒सीर्-मे॑दाहु॒तिभि॑रे॒व तद्-दे॒वाꣳस्त॑र्पयति॒ त ए॑नन् तृ॒प्ता आयु॑षा॒ तेज॑सा॒ वर्च॑सा श्रि॒या यश॑सा ब्रह्मवर्च॒सेना॒न्नाद्ये॑न च तर्पयन्ति । \textbf{ 10} \newline
                  \newline
                                                        (णो खॊर्वै fऒर् थिस् आनुवाकम्) \textbf{10} \newline \newline
                                \textbf{ T.A.2.11.1} \newline
                  ब्र॒ह्म॒य॒ज्ञेन॑ य॒क्ष्यमा॑णः॒ प्राच्यां᳚ दि॒शि ग्रामा॒दछ॑दिर्द॒र्॒.श उदी᳚च्यां प्रागुदी॒च्यां ॅवो॒दित॑ आदि॒त्ये द॑क्षिण॒त उ॑प॒वीयो॑प॒विश्य॒ हस्ता॑वव॒निज्य॒ त्रिराचा॑मे॒द्द्विः प॑रि॒मृज्य॑ स॒कृदु॑प॒स्पृश्य॒ शिर॒श्चक्षु॑षी॒ नासि॑के॒ श्रोत्रे॒ हृद॑यमा॒लभ्य॒ यत् त्रिरा॒चाम॑ति॒ तेन॒ ऋचः॑ प्रीणाति॒ यद्द्विः प॑रि॒मृज॑ति॒ तेन यजूꣳ॑षि॒ यथ् स॒कृदु॑प॒स्पृश॑ति॒ तेन॒ सामा॑नि॒ (यथ् स॒व्यं पा॒णि पा॒दौ प्रो॒क्षति॒) यच्छिर॒श्चक्षु॑षी॒ नासि॑के॒ श्रोत्रे॒ हृद॑यमा॒लभ॑ते॒ तेनाथ॑र्वाङ्गि॒रसो᳚ ब्राह्म॒णा-नी॑तिहा॒सान् पु॑रा॒णानि॒ कल्पा॒न् गाथा॑ नाराशꣳ॒॒सीः प्री॑णाति॒ दर्भा॑णां म॒हदु॑प॒स्तीर्यो॒पस्थं॑ कृ॒त्वा प्राङासी॑नः स्वाद्ध्या॒यमधी॑यीता॒पां ॅवा ए॒ष ओष॑धीनाꣳ॒॒ रसो॒ यद्-द॒र्भाः सर॑समॆ॒व ब्रह्म॑ कुरुते दक्षिणोत्त॒रौ पा॒णी (पा॒दौ) कृ॒त्वा सप॒वित्रा॒वोमिति॒ प्रति॑पद्यत ए॒तद्वै यजु॑स्त्रयीं ॅवि॒द्यां प्रत्ये॒षा वागे॒तत्प॑र॒मम॒क्षरं॒ तदे॒तदृ॒चाऽभ्यु॑क्तमृ॒चो अ॒क्षरे॑ पर॒मे व्यो॑म॒न्॒. यस्मि॑न् दे॒वा अधि॒ विश्वे॑ निषे॒दुर्यस्तन्न वेद॒ किमृ॒चा क॑रिष्यति॒ य इत्तद्वि॒दुस्त इ॒मे समा॑सत॒ इति॒ त्रीने॒व प्रायु॑ङ्क्त॒ भूर्भुवः॒स्व॑रित्या॑है॒तद्वै वा॒चः स॒त्यं ॅयदे॒व वा॒चः स॒त्यं तत् प्रायु॒ङ्क्ताथ॑ सावि॒त्रीं गा॑य॒त्रीं त्रिरन्वा॑ह प॒च्छो᳚ऽर्द्धर्च॑शोऽनवा॒नꣳ स॑वि॒ता श्रियः॑ प्रसवि॒ता श्रिय॑मे॒वाप्नो॒त् यथो᳚ प्र॒ज्ञात॑यै॒व प्र॑ति॒पदा॒ छन्दाꣳ॑सि॒ प्रति॑पद्यते । \textbf{ 11} \newline
                  \newline
                                                        (णो खॊर्वै fऒर् थिस् आनुवाकम्) \textbf{11} \newline \newline
                                \textbf{ T.A.2.12.1} \newline
                  ग्रामे॒ मन॑सा स्वाद्ध्या॒यमधी॑यीत॒ दिवा॒ नक्तं॑ ॅवे॒ति ह॑ स्मा॒ह शौ॒च आ᳚ह्ने॒य उ॒तार॑ण्ये॒ऽबल॑ उ॒त वा॒चोत तिष्ठ॑न्नु॒त व्रज॑न्नु॒तासी॑न उ॒त शया॑नो॒ऽधीयी॑तै॒व स्वा᳚द्ध्या॒यं तप॑स्वी॒ पुण्यो॑ भवति॒ य ए॒वं ॅवि॒द्वान्थ् स्वा᳚द्ध्या॒यमधी॑ते॒ नमो॒ ब्रह्म॑णे॒ नमो॑ अस्त्व॒ग्नये॒ नमः॑ पृथि॒व्यै नम॒ ओष॑धीभ्यः ।  नमो॑ वा॒चे नमो॑ वा॒चस्पत॑ये॒ नमो॒ विष्ण॑वे बृह॒ते क॑रोमि । \textbf{ 12} \newline
                  \newline
                                                        (णो खॊर्वै fऒर् थिस् आनुवाकम्) \textbf{12} \newline \newline
                                \textbf{ T.A.2.13.1} \newline
                  म॒द्ध्यन्दि॑ने प्र॒बल॒मधी॑यीता॒सौ खलु॒ वावैष आ॑दि॒त्यो यद्ब्रा᳚ह्म॒णस्तस्मा॒त्तर्.हि॒ तेक्ष्णि॑ष्ठं तपति॒ तदे॒षाऽभ्यु॑क्ता ॥  चि॒त्रं दे॒वाना॒मुद॑गा॒दनी॑कं॒ चक्षु॑र्मि॒त्रस्य॒ वरु॑णस्या॒ग्नेः ।  आऽप्रा॒ द्यावा॑पृथि॒वी अ॒न्तरि॑क्षꣳ॒॒ सूर्य॑ आ॒त्मा जग॑तस्त॒स्थुष॒श्चेति॒ स वा ए॒ष य॒ज्ञ्ः स॒द्यः प्रता॑यते स॒द्यः सन्ति॑ष्ठते॒ तस्य॒ प्राख्सा॒यम॑वभृ॒थो नमो॒ ब्रह्म॑ण॒ इति॑ परिधा॒नीयां॒ त्रिरन्वा॑हा॒प उ॑प॒स्पृश्य॑ गृ॒हाने॑ति॒ ततो॒ यत्किंच॒ ददा॑ति॒ सा दक्षि॑णा । \textbf{ 13} \newline
                  \newline
                                                        (णो खॊर्वै fऒर् थिस् आनुवाकम्) \textbf{13} \newline \newline
                                \textbf{ T.A.2.14.1} \newline
                  तस्य॒ वा ए॒तस्य॑ य॒ज्ञ्स्य॒ मेघो॑ हवि॒र्द्धानं॑ ॅवि॒द्युद॒ग्निर्व॒र्.॒षꣳ ह॒विः स्त॑नयि॒त्नुर्-व॑षट्का॒रो यद॑व॒स्फूर्ज॑ति॒ सोऽनु॑वषट्का॒रो वा॒युरा॒त्माऽमा॑वा॒स्या᳚ स्विष्ट॒कृद्य ए॒वं ॅवि॒द्वान्मे॒घे व॒र्.॒षति॑ वि॒द्योत॑माने स्त॒नय॑त्यव॒स्फूर्ज॑ति॒ पव॑माने वा॒याव॑मावा॒स्या॑याꣳ स्वाद्ध्या॒यमधी॑ते॒  तप॑ ए॒व तत्त॑प्यते॒ तपो॑ हि स्वाद्ध्या॒य इत्यु॑त्त॒मं नाकꣳ॑ रोहत्युत्त॒मः स॑मा॒नानां᳚ भवति॒ याव॑न्तꣳ ह॒ वा इ॒मां ॅवि॒त्तस्य॑ पू॒र्णां दद॑थ्स्व॒र्गं ॅलो॒कं ज॑यति॒ ताव॑न्तं ॅलो॒कं ज॑यति॒ भूयाꣳ॑सं चाक्ष॒य्यं चाप॑ पुनर्मृ॒त्युं ज॑यति॒ ब्रह्म॑णः॒ सायु॑ज्यं गच्छति । \textbf{ 14} \newline
                  \newline
                                                        (णॊ खॊर्वै fऒर् थिस् आनुवाकम्) \textbf{14} \newline \newline
                                \textbf{ T.A.2.15.1} \newline
                  तस्य॒ वा ए॒तस्य॑ य॒ज्ञ्स्य॒ द्वाव॑नद्ध्या॒यौ यदा॒ऽऽत्माऽशुचि॒र्-यद्-दे॒शः समृ॑द्धिर्दैव॒तानि॒ य ए॒वं ॅवि॒द्वान् म॑हारा॒त्र उ॒षस्युदि॑ते॒ व्रजꣳ॒॒स्तिष्ठ॒न्-नासी॑नः॒ शया॑नो॒ऽरण्ये᳚ ग्रामे॒ वा याव॑त्त॒रसꣳ॑ स्वाद्ध्या॒यमधी॑ते॒ सर्वा᳚न् ॅलो॒काञ्ज॑यति॒ सर्वा᳚न् ॅलो॒कान॑नृ॒णोऽनु॒ संच॑रति॒ तदे॒षाऽभ्यु॑क्ता ॥ अ॒नृ॒णा अ॒स्मिन्न॑नृ॒णाः पर॑स्मिꣳस्तृ॒तीये॑ लो॒के अ॑नृ॒णाः स्या॑म । ये दे॑व॒याना॑ उ॒त पि॑तृ॒याणाः॒ सर्वा᳚न् प॒थो अ॑नृ॒णा आक्षी॑ये॒मेत्य॒ग्निं ॅवै जा॒तं पा॒प्मा ज॑ग्राह॒ तंदे॒वा आहु॑तीभिः पा॒प्मान॒मपा᳚घ्न॒न्-नाहु॑तीनां ॅय॒ज्ञेन॑ य॒ज्ञ्स्य॒ दक्षि॑णाभि॒र्-दक्षि॑णानां ब्राह्म॒णेन॑ ब्राह्म॒णस्य॒ छन्दो॑भिः॒ छन्द॑साꣳ स्वाद्ध्या॒येनाप॑हतपाप्मा स्वाद्ध्या॒यो॑ दे॒वप॑वित्रं॒ ॅवा ए॒तत्तं ॅयोऽनू᳚थ् सृ॒जत्यभा॑गो वा॒चि भ॑व॒त्यभा॑गो ना॒के तदे॒षाऽभ्यु॑क्ता ॥ यस्ति॒त्याज॑ सखि॒विदꣳ॒॒ सखा॑यं॒ न तस्य॑ वा॒च्यपि॑ भा॒गो अ॑स्ति । यदीꣳ॑ शृ॒णोत्य॒लकꣳ॑ शृणोति॒ न हि प्र॒वेद॑ सुकृ॒तस्य॒ पन्था॒मिति॒ तस्मा᳚थ् स्वाद्ध्या॒योऽद्ध्ये॑त॒व्यो॑ यं ॅयं॑ क्र॒तुमधी॑ते॒ तेन॑ तेनास्ये॒ष्टं भ॑वत्य॒ग्नेर्-वा॒यो-रा॑दि॒त्यस्य॒ सायु॑ज्यं गच्छति॒ तदे॒षाऽभ्यु॑क्ता ॥  ये अ॒र्वाङु॒त वा॑ पुरा॒णे वे॒दं ॅवि॒द्वाꣳस॑म॒भितो॑ वदन्त्यादि॒त्यमे॒व ते परि॑वदन्ति॒ सर्वे॑ अ॒ग्निं द्वि॒तीयं॑ तृ॒तीयं॑ च हꣳ॒॒समिति॒ याव॑ती॒र्वै दे॒वता॒स्ताः सर्वा॑ वेद॒विदि॑ ब्राह्म॒णे व॑सन्ति॒ तस्मा᳚द् ब्राह्म॒णेभ्यो॑ वेद॒विद्भ्यो॑ दि॒वेदि॑वे॒ नम॑स्कुर्या॒न्नाश्ली॒लं की᳚र्तयेदे॒ता ए॒व दे॒वताः᳚ प्रीणाति । \textbf{ 15} \newline
                  \newline
                                                        (णो खॊर्वै fऒर् थिस् आनुवाकम्) \textbf{15} \newline \newline
                                \textbf{ T.A.2.16.1} \newline
                  रिच्य॑त इव॒ वा ए॒ष प्रेव रि॑च्यते॒ यो या॒जय॑ति॒ प्रति॑ वा गृ॒ह्णाति॑ या॒जयि॑त्वा प्रतिगृ॒ह्य वाऽन॑श्न॒न्त्रिः स्वा᳚द्ध्या॒यं ॅवे॒दमधी॑यीत त्रिरा॒त्रं ॅवा॑ सावि॒त्रीं गा॑य॒त्री-म॒न्वाति॑रेचयति॒ वरो॒ दक्षि॑णा॒ वरे॑णै॒व वरꣳ॑ स्पृणोत्या॒त्मा हि वरः॑ । \textbf{ 16} \newline
                  \newline
                                                        (णो खॊर्वै fऒर् थिस् आनुवाकम्) \textbf{16} \newline \newline
                                \textbf{ T.A.2.17.1} \newline
                  दु॒हे ह॒ वा ए॒ष छन्दाꣳ॑सि॒ यो या॒जय॑ति॒ स येन॑ यज्ञ्क्र॒तुना॑ या॒जये॒थ् सोऽर॑ण्यं प॒रेत्य॑ शुचौ दे॒शे स्वा᳚द्ध्या॒य-मे॒वैन॒-मधी॑यन्नासी॒त तस्या॒नश॑नं दी॒क्षा स्था॒नमु॑प॒सद॒ आस॑नꣳ सु॒त्या वाग्जु॒हूर्मन॑ उप॒भृद्धृ॒तिर्द्ध्रु॒वा प्रा॒णो ह॒विः सामा᳚द्ध्व॒र्युः स वा ए॒ष य॒ज्ञ्ः  प्रा॒णद॑क्षि॒णोऽन॑न्तदक्षिणः॒ समृ॑द्धतरः । \textbf{ 17} \newline
                  \newline
                                                        (णॊ खॊर्वै fऒर् थिस् आनुवाकम्) \textbf{17} \newline \newline
                                \textbf{ T.A.2.18.1} \newline
                  क॒ति॒धाऽव॑कीर्णी प्रवि॒शति॑ चतु॒र्द्धेत्या॑हुर्-ब्रह्मवा॒दिनो॑ म॒रुतः॑ प्रा॒णैरिन्द्रं॒ बले॑न॒ बृह॒स्पतिं॑ ब्रह्मवर्च॒सेना॒ग्निमे॒वेत॑रेण॒ सर्वे॑ण॒ तस्यै॒तां प्राय॑श्चित्तिं ॅवि॒दाञ्च॑कार सुदे॒वः का᳚श्य॒पो यो  ब्र॑ह्मचा॒र्य॑व॒किरे॑दमावा॒स्या॑याꣳ॒॒ रात्र्या॑म॒ग्निं प्र॒णीयो॑पसमा॒धाय॒ द्विराज्य॑स्योप॒घातं॑ जुहोति॒ कामाव॑कीर्णो॒ऽस्म्यव॑कीर्णोऽस्मि॒ काम॒ कामा॑य॒ स्वाहा॒ कामाभि॑द्रुग्धो॒ऽस्म्यभि॑द्रुग्धोऽस्मि॒ काम॒ कामा॑य॒ स्वाहेत्य॒मृतं॒ ॅवा आज्य॑म॒मृत॑मे॒वात्मन् ध॑त्ते हु॒त्वा प्रय॑ताञ्ज॒लिः कवा॑तिर्यङ्ङ॒ग्निम॒भि म॑न्त्रयेत॒ सं मा॑ सिञ्चन्तु म॒रुतः॒ समिन्द्रः॒ संबृह॒स्पतिः॑ ।  सं मा॒ऽयम॒ग्निः सि॑ञ्च॒त्वायु॑षा च॒ बले॑न॒ चायु॑ष्मन्तं करोत॒ मेति॒ प्रति॑ हास्मै म॒रुतः॑ प्रा॒णान् द॑धति॒ प्रतीन्द्रो॒ बलं॒ प्रति॒ बृह॒स्पति॑र् ब्रह्मवर्च॒सं प्रत्य॒ग्निरि॒तर॒थ्सर्वꣳ॒॒ सर्व॑तनुर्भू॒त्वा सर्व॒मायु॑रेति॒ त्रिर॒भिम॑न्त्रयेत॒ त्रिष॑त्या॒ हि दे॒वा योऽपू॑त इव॒ मन्ये॑त॒ स इ॒त्थं जु॑हुयादि॒त्थ-म॒भिम॑न्त्रयेत॒ पुनी॑त ए॒वात्मान॒-मायु॑रे॒वात्मन् ध॑त्ते॒ वरो॒ दक्षि॑णा॒ वरे॑णै॒व वरꣳ॑ स्पृणोत्या॒त्मा हि वरः॑ । \textbf{ 18} \newline
                  \newline
                                                        (णो खॊर्वै fऒर् थिस् आनुवाकम्) \textbf{18} \newline \newline
                                \textbf{ T.A.2.19.1} \newline
                  भूः प्रप॑द्ये॒ भुवः॒ प्रप॑द्ये॒ स्वः॑ प्रप॑द्ये॒ भूर्भुवः॒स्वः॑ प्रप॑द्ये॒ ब्रह्म॒ प्रप॑द्ये ब्रह्मको॒शं प्रप॑द्ये॒ऽमृतं॒ प्रप॑द्येऽमृतको॒शं प्रप॑द्ये चतुर्जा॒लं ब्र॑ह्मको॒शं ॅयं मृ॒त्युर्नाव॒पश्य॑ति॒ तं प्रप॑द्ये दे॒वान् प्रप॑द्ये देवपु॒रं प्रप॑द्ये॒ परी॑वृतो॒ वरी॑वृतो॒ ब्रह्म॑णा॒ वर्म॑णा॒ऽहं तेज॑सा॒ कश्य॑पस्य॒ यस्मै॒ नम॒स्तच्छिरो॒ धर्मो॑ मू॒र्द्धानं॑ ब्र॒ह्मोत्त॑रा॒ हनु॑र्य॒ज्ञोऽध॑रा॒ विष्णु॒र्॒. हृद॑यꣳ सम्ॅवथ्स॒रः प्र॒जन॑नम॒श्विनौ॑ पूर्व॒पादा॑व॒त्रिर्मद्ध्यं॑ मि॒त्रावरु॑णा वपर॒पादा॑व॒ग्निः पुच्छ॑स्य प्रथ॒मं काण्डं॒ तत॒ इन्द्र॒स्ततः॑ प्र॒जाप॑ति॒रभ॑यं चतु॒र्थꣳ स वा ए॒ष दि॒व्यः शा᳚क्व॒रः शिशु॑मार॒स्तꣳ ह॒ य ए॒वं ॅवेदाप॑ पुनर्मृ॒त्युं ज॑यति॒ जय॑ति स्व॒र्गं ॅलो॒कं नाद्ध्वनि॒ प्रमी॑यते॒ नाफ्सु प्रमी॑यते॒ नाग्नौ प्रमी॑यते॒ नान॒पत्यः॑ प्रमी॒यते॑ ल॒घ्वान्नो॑ भवति ध्रु॒वस्त्वम॑सि ध्रु॒वस्य क्षि॑तमसि॒ त्वं भू॒ताना॒मधि॑पतिरसि॒ त्वं भू॒तानाꣳ॒॒ श्रेष्ठो॑ऽसि॒ त्वां भू॒तान्युप॑प॒र्याव॑र्तन्ते॒ नम॑स्ते॒ नमः॒ सर्वं॑ ते॒ **नमो॒ नमः॑ । \textbf{ 19} \newline
                  \newline
                                                        (णॊ खॊर्वै fऒर् थिस् आनुवाकम्) \textbf{19} \newline \newline
                                \textbf{ T.A.2.20.1} \newline
                  नमः॒ प्राच्यै॑ दि॒शे याश्च॑ दे॒वता॑ ए॒तस्यां॒ प्रति॑वसन्त्ये॒ ताभ्य॑श्च॒ नमो॒ नमो दक्षि॑णायै दि॒शे याश्च॑ दे॒वता॑ ए॒तस्यां॒ प्रति॑वसन्त्ये॒ ताभ्य॑श्च॒ नमो॒ नमः॒ प्रती᳚च्यै दि॒शे याश्च॑ दे॒वता॑ ए॒तस्यां॒ प्रति॑वसन्त्ये॒ ताभ्य॑श्च॒ नमो॒ नम॒ उदी᳚च्यै दि॒शे याश्च॑ दे॒वता॑ ए॒तस्यां॒ प्रति॑वसन्त्ये॒ ताभ्य॑श्च॒ नमो॒ नम॑ ऊ॒र्द्ध्वायै॑ दि॒शे याश्च॑ दे॒वता॑ ए॒तस्यां॒ प्रति॑वसन्त्ये॒ ताभ्य॑श्च॒ नमो॒ नमोऽध॑रायै दि॒शे याश्च॑ दे॒वता॑ ए॒तस्यां॒ प्रति॑वसन्त्ये॒ ताभ्य॑श्च॒ नमो॒ नमो॑ऽवान्त॒रायै॑ दि॒शे याश्च॑ दे॒वता॑ ए॒तस्यां॒ प्रति॑वसन्त्ये॒ ताभ्य॑श्च॒ नमो॒ नमो गङ्गायमुनयोर्मद्ध्ये ये॑ वस॒न्ति॒ ते मे प्रसन्नात्मानस्चिरं जीवितं ॅव॑र्द्धय॒न्ति॒ नमो गङ्गायमुनयोर्मुनि॑भ्यश्च॒ नमो॒ नमो गङ्गायमुनयोर्मुनि॑भ्यश्च॒ नमः । \textbf{ 20} \newline
                  \newline
                                                        (णॊ खॊर्वै fऒर् थिस् आनुवाकम्) \textbf{20} \newline \newline
\textbf{Prapaataka Korvai with starting Padams of 1 to20 Anuvaakams :-} \newline
(सह॒ - रक्षाꣳ॑सि॒ - यद् दे॑वाः स॒प्तद॑श॒ - यददी᳚व्य॒न् पञ्च॑दश॒ - यु॑ष्टे॒ चतु॑स्त्रिꣳशद् - वैश्वान॒राय॒ षड्विꣳ॑शति॒र्-वात॑रशना ह - कुश्मा॒ण्डै- र॒जान्. ह॒ - पञ्च॑ - ब्रह्मय॒ज्ञेन॒ - ग्रामे॑ - म॒द्ध्यन्दि॑ने॒ - तस्य॒ वै मेघ॒ - स्तस्य॒ वै द्वौ - रिच्य॑ते - दु॒हे ह॑ - कति॒धाऽव॑कीर्णी॒ - भूर् - नमः॒ प्राच्यै॑ विꣳश॒तिः ) \newline

\textbf{korvai with starting padams of1, 11, 21 Series of Dasinis :-} \newline
छन्नॊत् बॆ दॆरिवॆद् अस् थॆ वक्यम्स् इन् सॊमॆ ऒf थे डसिनिस् दॊ नॊत् चॊन्fऒर्म् तॊ थॆ स्तन्दर्द् ऒf 10 वक्यम्स् पॆर् डसिनि. ठॆ समॆ रॆअसॊन् इस् अप्पिलिचब्लॆ fऒर् नॊन् अवैलबित्य् ऒf आनुवाक खॊर्वै इन् सॊमॆ ऒf थॆ अबॊवॆ आनुवाकम्स्. \newline

\textbf{first and last padam in TA, 2nd Prapaatakam :-} \newline
(सह॑ वै - मुनि॑भ्यश्च॒ नमः ) \newline 


॥ कृष्ण यजुर्वेदीय तैत्तिरीय ब्राह्मणे आरण्यके द्वितीयः प्रपाठकः समाप्तः ॥

Appendix (of Expansions)
ट्.आ. 2.3.1 - "यन्मयि॑ मा॒ता{17}" 
यन्मयि॑ मा॒ता गर्भे॑ स॒ति । एन॑श्च॒कार॒ यत्पि॒ता । 
अ॒ग्निर्मा॒ तस्मा॒देन॑सः ॥ {17} 
({17} appearing in T.B.3.7.12.3) 

ट्.आ. 2.3.1 - "यदा॑ पि॒पेष॒{18}" "यद॒न्तरि॑क्षं॒{19}" 
"ॅयदा॒शसा{20} "ऽति॑क्रामामि{21}" "त्रि॒ते दे॒वा{22}" 
यदा॑ पि॒पेष॑ मा॒तरं॑ पि॒तरं᳚ । पु॒त्रः प्रमु॑दितो॒ धयन्न्॑ । अहिꣳ॑सितौ पि॒तरौ॒ मया॒ तत् । तद॑ग्ने अनृ॒णो भ॑वामि ॥ {18}
({18} appearing in T.B.3.7.12.4)

यद॒न्तरि॑क्षं पृथि॒वीमु॒त द्यां । यन्मा॒तरं॑ पि॒तरं॑ ॅवा जिहिꣳसि॒म । 
अ॒ग्निर्मा॒ तस्मा॒देन॑सः ॥ {19}
({19} appearing in T.B.3.7.12.4)

यदा॒शसा॑ नि॒शसा॒ यत्प॑रा॒शसा᳚ । यदेन॑श्च कृ॒मा नूत॑नं॒ ॅयत्पु॑रा॒णं । 
अ॒ग्निर्मा॒ तस्मा॒देन॑सः ॥ {20} 
({20} appearing in T.B.3.7.12.4)


अति॑क्रामामि दुरि॒तं ॅयदेनः॑ । जहा॑मि रि॒प्रं प॑र॒मे स॒धस्थे᳚ । 
यत्र॒ यन्ति॑ सु॒कृतो॒ नापि॑ दु॒ष्कृतः॑ । तमारो॑हामि सु॒कृतां॒ नु लो॒कं ॥ {21} 
({21} appearing in T.B.3.7.12.5)
त्रि॒ते दे॒वा अ॑मृजतै॒-तदेनः॑ । त्रि॒त ए॒तन्म॑नु॒ष्ये॑षु मामृजे । 
ततो॑ मा॒ यदि॒ किंचि॑दान॒शे । 
अ॒ग्निर्मा॒ तस्मा॒देन॑सः ॥ {22}
({22} appearing in T.B.3.7.12.5)

ट्.आ. 2.3.1 - "दि॒वि जा॒ता{23}" "यदाप॑{24}" 
दि॒वि जा॒ता अ॒फ्सु जा॒ताः । या जा॒ता ओष॑धीभ्यः । 
अथो॒ या अ॑ग्नि॒जा आपः॑ । ता नः॑ शुन्धन्तु॒ शुन्ध॑नीः ॥ {23}

यदापो॒ नक्तं॑ दुरि॒तं चरा॑म । यद्वा॒ दिवा॒ नूत॑नं॒ ॅयत्पु॑रा॒णं । 
हिर॑ण्य-वर्णा॒स्तत॒ उत्पु॑नीत नः ॥ {24}
(Both {23} and {24] appearing in T.B.3.7.12.6)

ट्.आ. 2.3.1 -"इ॒मं मे॑ वरुण॒{25}" "तत्त्वा॑ यामि॒{26}" 
इ॒मं मे॑ वरुण श्रुधी॒ हव॑म॒द्या च॑ मृडय । त्वाम॑व॒स्युरा च॑के ॥ {25}

तत्त्वा॑ यामि॒ ब्रह्म॑णा॒ वन्द॑मान॒-स्तदा शा᳚स्ते॒ यज॑मानो ह॒विर्भिः॑ । 
अहे॑डमानो वरुणे॒ह बो॒द्ध्युरु॑शस॒ मा न॒ आयुः॒ प्रमो॑षीः ॥ {26} 
( Both {25} and {26] appearing in T.S.2.1.11.6)
ट्.आ. 2.3.1 -"त्वं नो॑ अग्ने॒{27}" "सत्वं नो॑ अग्ने॒{28}"
त्वं नो॑ अग्ने॒ वरु॑णस्य वि॒द्वान् दे॒वस्य॒ हेडोऽव॑ यासि सीष्ठाः । 
यजि॑ष्ठो॒ वह्नि॑ तमः॒ शोशु॑चानो॒ विश्वा॒ द्वेषा॑सि॒ प्रमु॑मुग्ध्य॒स्मत् ॥ {27}

स त्वंनो॑ अग्नेऽव॒मो भ॑वो॒ती नेदि॑ष्ठो अ॒स्या उ॒षसो॒ व्यु॑ष्टौ । 
अव॑ यक्ष्व नो॒ वरु॑ण॒॒ ररा॑णो वी॒हि मृ॑डी॒क सु॒हवो॑ न एधि ॥ {28}
( Both {27} and {28] appearing in T.S.2.5.12.3)

ट्.आ. 2.3.1 -"त्वम॑ग्ने अ॒यासि॑{29}" 
त्वम॑ग्ने अ॒याऽसि॑ । अ॒या सन्मन॑सा हि॒तः । अ॒या सन्.ह॒व्यमू॑हिषे । 
अ॒या नो॑ धेहि भेष॒जम् । इ॒ष्टो अ॒ग्निराहु॑तः । स्वाहा॑कृतः पिपर्तु नः । 
स्व॒गा दे॒वेभ्य॑ इ॒दं नमः॑ ॥ {29}
({29} appearing in T.B.2.4.1.9 )

ट्.आ. 2.4.1 -"अव॑ ते॒ हेड॒{30}" "उदु॑त्त॒म{31}"
अव॑ ते॒ हेडो॑ वरुण॒ नमो॑भि॒रव॑ य॒ज्ञेभि॑रीमहे ह॒विर्भिः॑ । 
क्षय॑न्न॒स्मभ्य॑मसुर प्रचेतो॒ राज॒न्नेनाꣳ॑सि शिश्रथः कृ॒तानि॑ ॥ {30}

उदु॑त्त॒मं ॅव॑रुण॒ पाश॑म॒स्मदवा॑ऽध॒मं ॅविम॑द्ध्य॒मꣳ श्र॑थाय । 
अथा॑ व॒यमा॑दित्य व्र॒ते तवाऽना॑गसो॒ अदि॑तये स्याम ॥ {31}
(Both {30} and {31} appearing in T.S.1.5.11.3)

ट्.आ. 2.4.1 -"मि॒मं मे॑ वरुण॒{32}
इ॒मं मे॑ वरुण श्रुधी॒ हव॑म॒द्या च॑ मृडय । त्वाम॑व॒स्युरा च॑के ॥ {32}
({32] appearing in T.S.2.1.11.6)

ट्.आ. 2.4.1 -"तत्त्वा॑ यामि॒{33}" 
{33} appearing is same as {26} above

ट्.आ. 2.4.1 -"त्वं नो॑ अग्ने॒{34}" 
{34} appearing is same as {27} above

ट्.आ. 2.4.1 -"सत्वं नो॑ अग्ने{35}
{35} appearing is same as {28} above

[**फाट भ्हेदम् - नमो॒ नमः॑ शिशुकुमाराय॒ नमः॑ ] \newline
\pagebreak
\pagebreak
        


\end{document}
