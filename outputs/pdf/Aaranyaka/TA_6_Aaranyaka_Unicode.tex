\documentclass[17pt]{extarticle}
\usepackage{babel}
\usepackage{fontspec}
\usepackage{polyglossia}
\usepackage{extsizes}



\setmainlanguage{sanskrit}
\setotherlanguages{english} %% or other languages
\setlength{\parindent}{0pt}
\pagestyle{myheadings}
\newfontfamily\devanagarifont[Script=Devanagari]{AdishilaVedic}


\newcommand{\VAR}[1]{}
\newcommand{\BLOCK}[1]{}




\begin{document}
\begin{titlepage}
    \begin{center}
 
\begin{sanskrit}
    { \Large
    ॐ नमः परमात्मने, श्री महागणपतये नमः
श्री गुरुभ्यो नमः, ह॒रिः॒ ॐ 
    }
    \\
    \vspace{2.5cm}
    \mbox{ \Huge
    कृष्ण यजुर्वेदीय तैत्तिरीय आरण्यके षष्ठः प्रपाठकः   }
\end{sanskrit}
\end{center}

\end{titlepage}
\tableofcontents

ॐ नमः परमात्मने, श्री महागणपतये नमः
श्री गुरुभ्यो नमः, ह॒रिः॒ ॐ \newline
6.1     षष्ठः प्रपाठकः - महानारायणोपनिषत् \newline

\addcontentsline{toc}{section}{ 6.1     षष्ठः प्रपाठकः - महानारायणोपनिषत्}
\markright{ 6.1     षष्ठः प्रपाठकः - महानारायणोपनिषत् \hfill https://www.vedavms.in \hfill}
\section*{ 6.1     षष्ठः प्रपाठकः - महानारायणोपनिषत् }
                                \textbf{ T.A.6.1.1} \newline
                  अंभ॑स्य पा॒रे भुव॑नस्य॒ मद्ध्ये॒ नाक॑स्य पृ॒ष्ठे म॑ह॒तो मही॑यान् ।  शु॒क्रेण॒ ज्योतीꣳ॑षि समनु॒प्रवि॑ष्टः प्र॒जाप॑तिश्चरति॒ गर्भे॑ अ॒न्तः ॥  यस्मि॑न्नि॒दꣳ सञ्च॒ विचैति॒ सर्वं॒ ॅयस्मि॑न् दे॒वा अधि॒ विश्वे॑ निषे॒दुः ।  तदे॒व भू॒॒तं तदु॒ भव्य॑मा इ॒दं तद॒क्षरे॑ पर॒मे व्यो॑मन्न् ॥ येना॑ वृ॒तं खञ्च॒ दिवं॑ म॒हीञ्च॒ येना॑दि॒त्य-स्तप॑ति॒ तेज॑सा॒ भ्राज॑सा च ।  यम॒न्तः स॑मु॒द्रे क॒वयो॒ वय॑न्ति॒ यद॒क्षरे॑ पर॒मे प्र॒जाः ॥ यतः॑ प्रसू॒ता ज॒गतः॑ प्रसूती॒ तोये॑न जी॒वान् व्यच॑सर्ज॒ (व्यस॑सर्ज॒) भूम्यां᳚ ।  यदोष॑धीभिः पु॒रुषा᳚न् प॒शूꣳश्च॒ विवे॑श भू॒तानि॑ चराच॒राणि॑ ॥ अतः॑ परं॒ नान्य॒-दणी॑यसहि॒ परा᳚त् परं॒ ॅयन् मह॑तो म॒हान्तं᳚ ।  यदे॑क-म॒व्यक्त॒-मन॑न्तरूपं॒ ॅविश्वं॑ पुरा॒णं तम॑सः॒ पर॑स्तात् । \textbf{ 1} \newline
                  \newline
                                                                  \textbf{ T.A.6.1.2} \newline
                  तदे॒वर्त्तं तदु॑ स॒त्यमा॑हु॒-स्तदे॒व ब्रह्म॑ पर॒मं क॑वी॒नां ।  इ॒ष्टा॒पू॒र्त्तं ब॑हु॒धा जा॒तं जाय॑मानं ॅवि॒श्वं बि॑भर्त्ति॒ भुव॑नस्य॒ नाभिः॑ ॥ तदे॒वाग्नि-स्तद्वा॒यु-स्तथ्सूर्य॒स्तदु॑ च॒न्द्रमाः᳚ ।  तदे॒व शु॒क्रम॒मृतं॒ तद्ब्रह्म॒ तदापः॒ स प्र॒जाप॑तिः ॥ सर्वे॑ निमे॒षा ज॒ज्ञिरे॑ वि॒द्युतः॒ पुरु॑षा॒दधि॑ ।  क॒ला मु॑हू॒र्त्ताः काष्ठा᳚श्चाहो-रा॒त्राश्च॑ सर्व॒शः ॥ अ॒र्द्ध॒मा॒सा मासा॑ ऋ॒तवः॑ सम्ॅवथ्स॒रश्च॑ कल्पन्तां । स आपः॑ प्रदु॒घे उ॒भे इ॒मे अ॒न्तरि॑क्ष॒-मथो॒ सुवः॑ ॥ नैन॑-मू॒र्द्ध्वं न ति॒र्यं च॒ न मद्ध्ये॒ परि॑जग्रभत् । न तस्ये॑शे॒ कश्च॒न तस्य॑ नाम म॒हद्यशः॑ । \textbf{ 2} \newline
                  \newline
                                                                  \textbf{ T.A.6.1.3} \newline
                  न स॒दृंशे॑ तिष्ठति॒ रूप॑मस्य॒ न चक्षु॑षा पश्यति॒ कश्च॒नैनं᳚ ।  हृ॒दा म॑नी॒षा मन॑सा॒ऽभि क्लृ॑प्तो॒ य ए॑नं ॅवि॒दु-रमृ॑ता॒स्ते भ॑वन्ति ॥  ए꣡पन्सिऒन् ऒf अ॒द्भ्यः संभू॑तः (आप्पॆअरिन्ग् इन् ट्.आ.3.13.1)  अ॒द्भ्यः संभू॑तः पृथि॒व्यै रसा᳚च्च । वि॒श्वक॑र्मणः॒ सम॑वर्त्त॒ताधि॑ ।  तस्य॒ त्वष्टा॑ वि॒दध॑द्रू॒पमे॑ति ॥  तत्पुरु॑षस्य॒ विश्व॒माजा॑न॒मग्रे᳚ ।  वेदा॒हमे॒तं पुरु॑षं म॒हान्तं᳚ ।  आ॒दि॒त्य व॑र्णं॒ तम॑सः॒ पर॑स्तात् ।  तमे॒वं ॅवि॒द्वान॒मृत॑ इ॒ह भ॑वति । नान्यः पन्था॑ विद्य॒तेऽय॑नाय  ॥  प्र॒जाप॑तिश्चरति॒ गर्भे॑ अ॒न्तः । अ॒जाय॑मानो बहु॒धा विजा॑यते । तस्य॒ धीराः॒ परि॑जानन्ति॒ योनिं᳚ । मरी॑चीनां प॒दमि॑च्छन्ति वे॒धसः॑ ॥ यो दे॒वेभ्य॒ आत॑पति । यो दे॒वानां᳚ पु॒रोहि॑तः । पूर्वो॒ यो दे॒वेभ्यो॑ जा॒तः । नमो॑ रु॒चाय॒ ब्राह्म॑ये  ॥ रुचं॑ ब्रा॒ह्मं ज॒नय॑न्तः । दे॒वा अग्रे॒ तद॑ब्रुवन् ।  यस्त्वै॒वं ब्रा᳚ह्म॒णो वि॒द्यात् । तस्य॑ दे॒वा अस॒न् वशे᳚  ॥ ह्रीश्च॑ते ल॒क्ष्मीश्च॒ पत्न्यौ᳚ । अ॒हो॒रा॒त्रे पा॒र्श्वे । नक्ष॑त्राणि रू॒पं ।  अ॒श्विनौ॒ व्यात्तं᳚ । इ॒ष्टं म॑निषाण । अ॒मुं म॑निषाण । सर्वं॑ मनिषाण ॥    ए꣡पन्सिऒन् ऒf हिरण्यग॒र्भ ः (आप्पॆअरिन्ग् इन् ट्.श्.4.1.8.3) हि॒र॒ण्य॒ग॒र्भः सम॑वर्त्त॒ताग्रे॑ भू॒तस्य॑ जा॒तः पति॒रेक॑ आसीत् ।  स दा॑धार पृथि॒वीं द्यामु॒तेमां कस्मै॑ दे॒वाय॑ ह॒विषा॑ विधेम ॥   यः प्रा॑ण॒तो नि॑मिष॒तो म॑हि॒त्वैक॒ इद्राजा॒ जग॑तो ब॒भूव॑ ।  य ईशे॑ अ॒स्य द्वि॒पद॒-श्चतु॑ष्प॒दः कस्मै॑ दे॒वाय॑ ह॒विषा॑ विधेम ॥   य आ᳚त्म॒दा ब॑ल॒दा यस्य॒ विश्व॑ उपा॒स॑ते प्र॒शिषं॒ ॅयस्य॑ दे॒वाः ।  यस्य॑ छा॒यामृतं॒ ॅयस्य॑ मृ॒त्युः कस्मै॑ दे॒वाय॑ ह॒विषा॑ विधेम ॥   ‘यस्ये॒मे हि॒मव॑न्तो महि॒त्वा यस्य॑ समु॒द्रꣳ र॒सया॑ स॒हाऽऽहुः ।  यस्ये॒माः प्र॒दिशो॒ यस्य॑ बा॒हू कस्मै॑ दे॒वाय॑ ह॒विषा॑ विधेम ॥   यं क्रन्द॑सी॒ अव॑सा तस्तभा॒ने अ॒भ्यैक्षे॑तां॒ मन॑सा॒ रेज॑माने ।  यत्राधि॒सूर॒ उदि॑तौ॒ व्येति॒ कस्मै॑ दे॒वाय॑ ह॒विषा॑ विधेम ॥   येन॒ द्यौरु॒ग्रा पृ॑थि॒वी च॑ दृ॒ढे येन॒ सुव॑स् स्तभि॒तं ॅयेन॒ नाक॑: ।  यो अ॒न्तरि॑क्षे॒ रज॑सो वि॒मानः॒ कस्मै॑ दे॒वाय॑ ह॒विषा॑ विधेम ॥   आपो॑ ह॒ यन्म॑ह॒ती विश्व॒माय॒न्-दक्ष॒न्दधा॑ना ज॒नय॑न्ती-र॒ग्निं ।  ततो॑ दे॒वाना॒-न्निर॑वर्त्त॒तासु॒रेकः॒ कस्मै॑ दे॒वाय॑ ह॒विषा॑ विधेम ॥   यश्चि॒दापो॑ महि॒ना प॒र्यप॑श्य॒द्-द्क्ष॒न्दधा॑ना ज॒नय॑न्ती-र॒ग्निं ।  यो दे॒वेष्वधि॑ दे॒व येक॒ आसी॒त् कस्मै॑ दे॒वाय॑ ह॒विषा॑ विधेम ॥   अ॒द्भ्यः संभू॑तो हिरण्यग॒र्भ इत्य॒ष्टौ ॥ ए॒ष हि दे॒वः प्र॒दिशोऽनु॒ सर्वाः॒ पूर्वो॑ हि जा॒तः स उ॒ गर्भे॑ अ॒न्तः ।  स वि॒जाय॑मानः सजनि॒ष्यमा॑णः प्र॒त्यङ्-मुखा᳚ स्तिष्ठति वि॒श्वतो॑मुखः ॥ वि॒श्वत॑श्च-क्षुरु॒त वि॒श्वतो॑ मुखो वि॒श्वतो॑ हस्त उ॒त वि॒श्वत॑स्पात् ।  सं बा॒हुभ्यां॒ नम॑ति॒ सं पत॑त्रै॒र्-द्यावा॑ पृथि॒वी ज॒नय॑न् दे॒व एकः॑ ॥ वे॒नस्तत् पश्य॒न्. विश्वा॒ भुव॑नानि वि॒द्वान्. यत्र॒ विश्वं॒ भव॒त्येक॑-नीळं ।  यस्मि॑न्नि॒दꣳ सञ्च॒ विचैकꣳ॒॒ स ओतः॒ प्रोत॑श्च वि॒भुः प्र॒जासु॑ ॥ प्रतद्वो॑चे अ॒मृत॒न्नु वि॒द्वान् ग॑न्ध॒र्वो नाम॒ निहि॑तं॒ गुहा॑सु \textbf{ 3} \newline
                  \newline
                                                                  \textbf{ T.A.6.1.4} \newline
                  त्रीणि॑ प॒दा निहि॑ता॒ गुहा॑सु॒ यस्तद्वेद॑ सवि॒तुः पि॒ताऽस॑त् ॥ स नो॒ बन्धु॑र्-जनि॒ता स वि॑धा॒ता धामा॑नि॒ वेद॒ भुव॑नानि॒ विश्वा᳚ । यत्र॑ दे॒वा अ॒मृत॑मान-शा॒नास्तृ॒तीये॒ धामा᳚न्य॒-भ्यैर॑यन्त ॥ परि॒ द्यावा॑पृथि॒वी य॑न्ति स॒द्यः परि॑ लो॒कान् परि॒ दिशः॒ परि॒ सुवः॑ । ऋ॒तस्य॒ तन्तुं॑ ॅविततं ॅवि॒चृत्य॒ तद॑पश्य॒त् तद॑भवत् प्र॒जासु॑ ॥ प॒रीत्य॑ लो॒कान् प॒रीत्य॑ भू॒तानि॑ प॒रीत्य॒ सर्वाः᳚ प्र॒दिशो॒ दिश॑श्च । प्र॒जाप॑तिः प्रथम॒जा ऋ॒तस्या॒त्मना॒-ऽऽत्मान॑-म॒भि-संब॑भूव ॥ सद॑स॒स्पति॒-मद्भु॑तं प्रि॒यमिन्द्र॑स्य॒ काम्यं᳚ । सनिं॑ मे॒धा म॑यासिषं ॥ उद्दी᳚प्यस्व जातवेदो ऽप॒घ्नन्निऋ॑र्.तिं॒ मम॑ \textbf{ 4} \newline
                  \newline
                                                                  \textbf{ T.A.6.1.5} \newline
                  प॒शूꣳश्च॒ मह्य॒माव॑ह॒ जीव॑नञ्च॒ दिशो॑ दिश ॥ मानो॑ हिꣳसी ज्जातवेदो॒ गामश्वं॒ पुरु॑षं॒ जग॑त् ।  अबि॑भ्र॒दग्न॒ आग॑हि श्रि॒या मा॒ परि॑पातय ॥ पुरु॑षस्य विद्म सहस्रा॒क्षस्य॑ महादे॒वस्य॑ धीमहि । तन्नो॑ रुद्रः प्रचो॒दया᳚त् ॥  तत्पुरु॑षाय वि॒द्महे॑ महादे॒वाय॑ धीमहि ।  तन्नो॑ रुद्रः प्रचो॒दया᳚त् ॥  तत्पुरु॑षाय वि॒द्महे॑ वक्रतु॒ण्डाय॑ धीमहि ।  तन्नो॑ दन्तिः प्रचो॒दया᳚त् ॥ तत्पुरु॑षाय वि॒द्महे॑ चक्रतु॒ण्डाय॑ धीमहि \textbf{ 5} \newline
                  \newline
                                                                  \textbf{ T.A.6.1.6} \newline
                  तन्नो॑ नन्दिः प्रचो॒दया᳚त् ॥  तत्पुरु॑षाय वि॒द्महे॑ महासे॒नाय॑ धीमहि ।  तन्नः॑ षण्मुखः प्रचो॒दया᳚त् ॥ तत्पुरु॑षाय वि॒द्महे॑ सुवर्णप॒क्षाय॑ धीमहि । तन्नो॑ गरुडः प्रचो॒दया᳚त् ॥ वे॒दा॒त्म॒नाय॑ वि॒द्महे॑ हिरण्यग॒र्भाय॑ धीमहि । तन्नो᳚ ब्रह्म प्रचो॒दया᳚त् ॥ ना॒रा॒य॒णाय॑ वि॒द्महे॑ वासुदे॒वाय॑ धीमहि ।  तन्नो॑ विष्णुः प्रचो॒दया᳚त् ॥ व॒ज्र॒न॒खाय॑ वि॒द्महे॑ तीक्ष्ण-दꣳ॒॒ष्ट्राय॑ धीमहि \textbf{ 6} \newline
                  \newline
                                                                  \textbf{ T.A.6.1.7} \newline
                  तन्नो॑ नारसिꣳहः प्रचो॒दया᳚त् ॥  भा॒स्क॒राय॑ वि॒द्महे॑ महद्द्युतिक॒राय॑ धीमहि ।  तन्नो॑ आदित्यः प्रचो॒दया᳚त् ॥  वै॒श्वा॒न॒राय॑ वि॒द्महे॑ लाली॒लाय॑ धीमहि ।  तन्नो॑ अग्निः प्रचो॒दया᳚त् ॥  का॒त्या॒य॒नाय॑ वि॒द्महे॑ कन्यकु॒मारि॑ धीमहि ।  तन्नो॑ दुर्गिः प्रचो॒दया᳚त् ॥  स॒ह॒स्र॒पर॑मा दे॒वी॒ श॒तमू॑ला श॒ताङ्कु॑रा ।  सर्वꣳ॑ हरतु॑ मे पा॒पं॒ दू॒र्वा दुः॑स्वप्न॒ नाश॑नी ॥  काण्डा᳚त् काण्डात् प्र॒रोह॑न्ती॒ परु॑षः परुषः॒ परि॑ \textbf{ 7} \newline
                  \newline
                                                                  \textbf{ T.A.6.1.8} \newline
                  ए॒वा नो॑ दूर्वे॒ प्रत॑नु स॒हस्रे॑ण श॒तेन॑ च ॥  या श॒तेन॑ प्रत॒नोषि॑ स॒हस्रे॑ण वि॒रोह॑सि ।  तस्या᳚स्ते देवीष्टके वि॒धेम॑ ह॒विषा॑ व॒यं ॥  अश्व॑क्रा॒न्ते र॑थक्रा॒न्ते॒ वि॒ष्णुक्रा᳚न्ते व॒सुन्ध॑रा ।  शिरसा॑ धार॑यिष्या॒मि॒ र॒क्ष॒स्व मां᳚ पदे॒ पदे ॥  भूमिर्-धेनुर् धरणी लो॑कधा॒रिणी ।  उ॒धृता॑ऽसि व॑राहे॒ण॒ कृ॒ष्णे॒न श॑त बा॒हुना ॥  मृ॒त्तिके॑ हन॑ मे पा॒पं॒ ॅय॒न्म॒या दु॑ष्कृतं॒ कृतं । मृ॒त्तिके᳚ ब्रह्म॑दत्ता॒ऽसि॒ का॒श्यपे॑नाभि॒मन्त्रि॑ता । मृ॒त्तिके॑ देहि॑ मे पु॒ष्टिं॒ त्व॒यि स॑र्वं प्र॒तिष्ठि॑तं । \textbf{ 8} \newline
                  \newline
                                                                  \textbf{ T.A.6.1.9} \newline
                  मृ॒त्तिके᳚ प्रतिष्ठि॑ते स॒र्वं॒ त॒न्मे नि॑र्णुद॒ मृत्ति॑के । तया॑ ह॒तेन॑ पापे॒न॒ ग॒च्छा॒मि प॑रमां॒ गतिं ॥ यत॑ इन्द्र॒ भया॑महे॒ ततो॑ नो॒ अभ॑यं कृधि ।  मघ॑वन् छ॒ग्धि तव॒ तन्न॑ ऊ॒तये॒ विद्विषो॒ विमृधो॑ जहि ॥  स्व॒स्ति॒दा वि॒शस्पति॑र्-वृत्र॒हा विमृधो॑ व॒शी ।  वृषेन्द्रः॑ पु॒र ए॑तु नः स्वस्ति॒दा अ॑भयङ्क॒रः ॥  स्व॒स्ति न॒ इन्द्रो॑ वृ॒द्धश्र॑वाः स्व॒स्ति नः॑ पू॒षा वि॒श्ववे॑दाः । स्व॒स्ति न॒स्तार्क्ष्यो॒ अरि॑ष्टनेमिः स्व॒स्ति नो॒ बृह॒स्पति॑र् दधातु ॥  आपा᳚न्त-मन्युस्तृ॒पल॑प्रभर्मा॒ धुनिः॒ शिमी॑वा॒ञ्-छरु॑माꣳ ऋजी॒षी । सोमो॒ विश्वा᳚न्यत॒सा वना॑नि॒ नार्वागिन्द्रं॑ प्रति॒माना॑नि देभुः । \textbf{ 9} \newline
                  \newline
                                                                  \textbf{ T.A.6.1.10} \newline
                  ब्रह्म॑ जज्ञा॒नं प्र॑थ॒मं पु॒रस्ता॒द्-विसी॑म॒तः सु॒रुचो॑ वे॒न आ॑वः । स बु॒ध्निया॑ उप॒मा अ॑स्य वि॒ष्ठाः स॒तश्च॒ योनि॒-मस॑तश्च॒ विवः॑ ॥  स्यो॒ना पृ॑थिवि॒ भवा॑ नृक्ष॒रा नि॒वेश॑नी । यच्छा॑ नः॒ शर्म॑ स॒प्रथाः᳚ ॥  ग॒न्ध॒द्वा॒रां दु॑राध॒र्.षां॒ नि॒त्यपु॑ष्टां करी॒षिणीं᳚ । ई॒श्वरीꣳ॑ सर्व॑भूता॒नां॒ तामि॒होप॑ह्वये॒ श्रियं ॥  श्री᳚र्मे भ॒जतु । अलक्ष्मी᳚र्मे न॒श्यतु ।  विष्णु॑मुखा॒ वै दे॒वाः छन्दो॑-भिरि॒मान् ॅलो॒का-न॑नपज॒य्य-म॒भ्य॑जयन्न् । म॒हाꣳ इन्द्रो॒ वज्र॑बाहुः षोड॒शी शर्म॑ यच्छतु । \textbf{ 10} \newline
                  \newline
                                                                  \textbf{ T.A.6.1.11} \newline
                  स्व॒स्ति नो॑ म॒घवा॑ करोतु॒ हन्तु॑ पा॒प्मानं॒ ॅयो᳚ऽस्मान् द्वेष्टि॑ ॥  सो॒मानꣳ॒॒ स्वर॑णं कृणु॒हि ब्र॑ह्मणस्पते । क॒क्षीव॑न्तं॒ ॅय औ॑शि॒जं ।  शरी॑रं ॅयज्ञ्शम॒लं कुसी॑दं॒ तस्मि᳚न् थ्सीदतु॒ यो᳚ऽस्मान् द्वेष्टि॑ ॥  चर॑णं प॒वित्रं॒ ॅवित॑तं पुरा॒णं ॅयेन॑ पू॒त-स्तर॑ति दुष्कृ॒तानि॑ ।  तेन॑ प॒वित्रे॑ण शु॒द्धेन॑ पू॒ता अति॑ पा॒प्मान॒-मरा॑तिं तरेम ॥  स॒जोषा॑ इन्द्र॒ सग॑णो म॒रुद्भिः॒ सोमं॑ पिब वृत्रहञ्छूर वि॒द्वान् ।  ज॒हि शत्रूꣳ॒॒ रप॒ मृधो॑ नुद॒स्वाथाभ॑यं कृणुहि वि॒श्वतो॑ नः ॥  सु॒मि॒त्रा न॒ आप॒ ओष॑धयः सन्तु दुर्मि॒त्रास्तस्मै॑  भूयासु॒र् या᳚ऽस्मान् द्वेष्टि॒ यञ्च॑ व॒यं द्वि॒ष्मः ॥  आपो॒ हिष्ठा म॑यो॒ भुव॒स्ता न॑ ऊ॒र्जे द॑धातन \textbf{ 11} \newline
                  \newline
                                                                  \textbf{ T.A.6.1.12} \newline
                  म॒हेरणा॑य॒ चक्ष॑से । यो वः॑ शि॒वत॑मो॒ रस॒-स्तस्य॑ भाजयते॒ ह नः॑ । उ॒श॒ती-रि॑व मा॒तरः॑ । तस्मा॒ अर॑ङ्गमामवो॒ यस्य॒ क्षया॑य॒ जिन्व॑थ ।  आपो॑ ज॒नय॑था च नः ॥ हिर॑ण्यशृङ्गं॒ ॅवरु॑णं॒ प्रप॑द्ये ती॒र्थं मे॑ देहि॒ याचि॑तः ।  य॒न्मया॑ भु॒क्त-म॒साधू॑नां पा॒पेभ्य॑श्च प्र॒तिग्र॑हः ॥ यन्मे॒ मन॑सा वा॒चा॒ क॒र्म॒णा वा दु॑ष्कृतं॒ कृतं । तन्न॒ इन्द्रो॒ वरु॑णो॒ बृह॒स्पतिः॑ सवि॒ता च॑ पुनन्तु॒ पुनः॑ पुनः ॥  नमो॒ऽग्नये᳚-ऽफ्सु॒मते॒ नम॒ इन्द्रा॑य॒ नमो॒ वरु॑णाय॒ नमो वारुण्यै॑ नमो॒ऽद्भ्यः । \textbf{ 12} \newline
                  \newline
                                                                  \textbf{ T.A.6.1.13} \newline
                  यद॒पां क्रू॒रं ॅयद॑मे॒द्ध्यं ॅयद॑शा॒न्तं तदप॑गच्छतात् ॥ अ॒त्या॒श॒ना-द॑तीपा॒ना॒-द्य॒च्च उ॒ग्रात् प्र॑ति॒ग्रहा᳚त् । तन्नो॒ वरु॑णो रा॒जा॒ पा॒णिना᳚ ह्यव॒मर्.श॑तु ॥ सो॑ऽहम॑पा॒पो वि॒रजो॒ निर्मु॒क्तो मु॑क्तकि॒ल्बिषः । नाक॑स्य पृ॒ष्ठमारु॑ह्य॒ गच्छे॒द् ब्रह्म॑सलो॒कतां ॥ यश्चा॒फ्सु वरु॑णः॒ स पु॒नात्व॑घमर्.ष॒णः ॥ इ॒मं मे॑ गंगे यमुने सरस्वति॒ शुतु॑द्रि॒ स्तोमꣳ॑ सचता॒ परु॒ष्णिया ।  अ॒सि॒क्नि॒या म॑रुद्वृधे वि॒तस्त॒या-ऽऽर्जी॑कीये शृणु॒ह्या सु॒षोम॑या ॥  ऋ॒तञ्च॑ स॒त्यञ्चा॒-भी᳚द्धा॒ त्तप॒सोऽद्ध्य॑जायत ।  ततो॒ रात्रि॑-रजायत॒ ततः॑ समु॒द्रो अ॑र्ण॒वः \textbf{ 13} \newline
                  \newline
                                                                  \textbf{ T.A.6.1.14} \newline
                  स॒मु॒द्रा-द॑र्ण॒वा-दधि॑ सम्ॅवथ्स॒रो अ॑जायत ।  अ॒हो॒रा॒त्राणि॑ वि॒दध॒द् (मि॒दध॒द्) विश्व॑स्य मिष॒तो व॒शी ॥ सू॒र्या॒च॒न्द्र॒मसौ॑ धा॒ता य॑था पू॒र्व म॑कल्पयत् । दिव॑ञ्च पृथि॒वीं चा॒न्तरि॑क्ष॒ मथो॒ सुवः॑ ॥ यत् पृ॑थि॒व्याꣳ रज॑स्स्व॒ मान्तरि॑क्षे वि॒रोद॑सी । इ॒माꣳ स्तदा॒पो व॑रुणः पु॒नात्व॑घमर्.ष॒णः ॥ पु॒नन्तु॒ वस॑वः पु॒नातु॒ वरु॑णः पु॒नात्व॑घमर्.ष॒णः ।  ए॒ष भू॒तस्य॑ म॒द्ध्ये भुव॑नस्य गो॒प्ता ॥  ए॒ष पु॒ण्यकृ॑तां ॅलो॒का॒ने॒ष मृ॒त्योर्. हि॑र॒ण्मयं᳚ । द्यावा॑पृथि॒व्योर्. हि॑र॒ण्मयꣳ॒॒ सꣳश्रि॑तꣳ॒॒ सुवः॑ \textbf{ 14} \newline
                  \newline
                                                                  \textbf{ T.A.6.1.15} \newline
                  स नः॒ सुवः॒ सꣳ शि॑शाधि ॥  आर्द्रं॒ ज्वल॑ति॒ ज्योति॑-र॒हम॑स्मि । ज्योति॒र्-ज्वल॑ति॒ ब्रह्मा॒हम॑स्मि । यो॑ऽहम॑स्मि॒ ब्रह्मा॒हम॑स्मि । अ॒हम॑स्मि॒ ब्रह्मा॒हम॑स्मि । अ॒हमे॒वाहं मां जु॑होमि॒ स्वाहा᳚ ॥ अ॒का॒र्य॒-का॒र्य॑व की॒र्णी स्ते॒नो भ्रू॑ण॒हा गु॑रुत॒ल्पगः । वरु॑णो॒-ऽपाम॑घमर्.ष॒ण-स्तस्मा᳚त् पा॒पात् प्रमु॑च्यते ॥  र॒जोभूमि॑स्त्व॒माꣳ रोद॑यस्व॒ प्रव॑दन्ति॒ धीराः᳚ ॥  आक्रा᳚न्थ्-समु॒द्रः प्र॑थ॒मे विध॑र्म-ञ्ज॒नय॑न् प्र॒जा भुव॑नस्य॒ राजा᳚ ( ) ।  वृषा॑ प॒वित्रे॒ अधि॒सानो॒ अव्ये॑ बृ॒हथ् सोमो॑  वावृधे सुवा॒न इन्दुः॑ । \textbf{ 15} \newline
                  \newline
                                                  
                रु॒द्रो॒ रु॒द्रश्च॒ दन्ति॒श्च॒ न॒न्दिः॒ ष॒ण्मु॒ख॒ ए॒व च॑ । ग॒रु॒डो॒ ब्र॒ह्म॒ वि॒ष्णु॒श्च॒ ना॒र॒सिꣳ॒॒ह॒स्त॒थै॒व च॑ । आ॒दि॒त्यो॒ऽग्नि॒श्च॒ दु॒र्गि॒श्च॒ क्र॒मे॒ण॒ द्वाद॒शाम्भ॑सि ।    म॒ म॒ व॒ च॒ म॒ सु॒ वे॒ ना॒ व॒ भा॒ वै॒ का॒त्या॒य॒नाय॑ ।   अनुवाकं - 2 \newline
                                      (पुर॑स्ता॒द् - यशो॒ - गुहा॑सु॒ - मम॑ - चक्रतु॒ण्डाय॑ धीमहि - तीक्षदꣳ॒॒ष्ठ्राय॑ धीमहि॒ - परि॑ - प्र॒तिष्ठि॑तं - देभुर् - यच्छतु - दधातना॒- द्भ्यो᳚ - ऽर्ण॒वः - सुवो॒ - राजैकं॑ च) \textbf{1} \newline \newline
                                \textbf{ T.A.6.2.1} \newline
                  जा॒तवे॑दसे सुनवाम॒ सोम॑-मरातीय॒तो निद॑हाति॒ वेदः॑ । स नः॑ पर्.ष॒दति॑ दु॒र्गाणि॒ विश्वा॑ ना॒वेव॒ सिन्धुं॑ दुरि॒ताऽत्य॒ग्निः ॥  ताम॒ग्निव॑र्णां॒ तप॑सा ज्वल॒न्तीं ॅवै॑रोच॒नीं क॑र्मफ॒लेषु॒ जुष्टां᳚ । दु॒र्गां दे॒वीꣳ शर॑णम॒हं प्रप॑द्ये सु॒तर॑सि तरसे॒ नमः॑ ॥  अग्ने॒ त्वं पा॑रया॒ नव्यो॑ अ॒स्मान्थ् स्व॒स्ति-भि॒रति॑ दु॒र्गाणि॒ विश्वा᳚ । पूश्च॑ पृ॒थ्वी ब॑हु॒ला न॑ उ॒र्वी भवा॑ तो॒काय॒ तन॑याय॒ शम्ॅयोः ॥  विश्वा॑नि नो दु॒र्गहा॑ जातवेदः॒ सिन्धुं॒ न ना॒वा दु॑रि॒ताऽति॑पर्.षि । अग्ने॑ अत्रि॒वन् मन॑सा गृणा॒नो᳚ऽस्माकं॑ बोद्ध्यवि॒ता त॒नूनां᳚ ॥  पृ॒त॒ना॒जितꣳ॒॒ सह॑मान-मु॒ग्रम॒ग्निꣳ हु॑वेम पर॒माथ् स॒धस्था᳚त् । स नः॑ पर्.ष॒दति॑ दु॒र्गाणि॒ विश्वा॒ क्षाम॑द्दे॒वो अति॑ दुरि॒ताऽत्य॒ग्निः ( ) ॥  प्र॒त्नोषि॑-क॒मीड्यो॑ अद्ध्व॒रेषु॑ स॒नाच्च॒ होता॒ नव्य॑श्च॒ सथ्सि॑ । स्वाञ्चा᳚ग्ने त॒नुवं॑ पि॒प्रय॑स्वा॒स्मभ्य॑ञ्च॒ सौभ॑ग॒माय॑जस्व ॥  गोभि॒र्-जुष्ट॑म॒युजो॒ निषि॑क्तं॒ तवे᳚न्द्र विष्णो॒-रनु॒सञ्च॑रेम ।  नाक॑स्य पृ॒ष्ठम॒भि स॒म्ॅवसा॑नो॒ वैष्ण॑वीं ॅलो॒क इ॒ह मा॑दयन्तां । \textbf{ 16} \newline
                  \newline
                                                        (दु॒रि॒ताऽत्य॒ग्निश्च॒त्वारि॑ च) \textbf{2} \newline \newline
                                \textbf{ T.A.6.3.1} \newline
                  भू-रन्न॑-म॒ग्नये॑ पृथि॒व्यै स्वाहा॒ ,  भुवोऽन्नं॑ ॅवा॒यवे॒ऽन्तरि॑क्षाय॒ स्वाहा॒,  सुव॒रन्न॑-मादि॒त्याय॑ दि॒वे स्वाहा॒ ,   भूर्भुव॒स्सुव॒-रन्नं॑ च॒न्द्रम॑से दि॒ग्भ्यः स्वाहा॒ , नमो॑ दे॒वेभ्यः॑ स्व॒धा पि॒तृभ्यो॒ भूर्भुव॒स्सुव॒-रन्न॒म्ॐ । \textbf{ 17} \newline
                  \newline
                                                         \textbf{} \newline \newline
                                \textbf{ T.A.6.4.1} \newline
                  भूर॒ग्नये॑ पृथि॒व्यै स्वाहा॒,  भुवो॑ वा॒यवे॒ऽन्तरि॑क्षाय॒ स्वाहा॒ , सुव॑रादि॒त्याय॑ दि॒वे स्वाहा॒ ,  भूर्भुव॒स्सुव॑-श्च॒न्द्रम॑से दि॒ग्भ्यः स्वाहा॒,  नमो॑ दे॒वेभ्यः॑ स्व॒धा पि॒तृभ्यो॒ भूर्भुव॒स्सुव॒-रग्न॒ ॐ । \textbf{ 18} \newline
                  \newline
                                                         \textbf{} \newline \newline
                                \textbf{ T.A.6.5.1} \newline
                  भूर॒ग्नये॑ च पृथि॒व्यै च॑ मह॒ते च॒ स्वाहा॒ , भुवो॑ वा॒यवे॑ चा॒न्तरि॑क्षाय च मह॒ते च॒ स्वाहा॒ , सुव॑रादि॒त्याय॑ च दि॒वे च॑ मह॒ते च॒ स्वाहा॒ ,  भू-र्भुव॒स्सुव॑-श्च॒न्द्रम॑से च॒ नक्ष॑त्रेभ्यश्च दि॒ग्भ्यश्च॑ मह॒ते च॒ स्वाहा॒,  नमो॑ दे॒वेभ्यः॑ स्व॒धा पि॒तृभ्यो॒ भूर्भुव॒स्सुव॒र् मह॒र्ॐ । \textbf{ 19} \newline
                  \newline
                                                         \textbf{} \newline \newline
                                \textbf{ T.A.6.6.1} \newline
                  पाहि नो अग्न एन॑से स्वा॒हा । पाहि नो विश्ववेद॑से स्वा॒हा । यज्ञ्ं पाहि विभाव॑सो स्वा॒हा । सर्वं पाहि शतक्र॑तो स्वा॒हा । \textbf{ 20} \newline
                  \newline
                                                         \textbf{} \newline \newline
                                \textbf{ T.A.6.7.1} \newline
                  पा॒हि नो॑ अग्न॒ एक॑या । पा॒ह्यु॑त द्वि॒तीय॑या । पा॒ह्यूर्जं॑ तृ॒तीय॑या ।  पा॒हि गी॒र्भि-श्च॑त॒सृभि॑र् वसो॒ स्वाहा᳚ । \textbf{ 21} \newline
                  \newline
                                                         \textbf{} \newline \newline
                                \textbf{ T.A.6.8.1} \newline
                  यश्छन्द॑सा-मृष॒भो वि॒श्वरू॑प॒-श्छन्दो᳚भ्य॒ श्छन्दाꣳ॑स्या वि॒वेश॑ । सचाꣳ शिक्यः पुरो वाचो॑पनि॒ष-दिन्द्रो᳚ ज्ये॒ष्ठ इ॑न्द्रि॒याय॒ ऋषि॑भ्यो॒  नमो॑ दे॒वेभ्यः॑ स्व॒धा पि॒तृभ्यो॒ भूर्भुव॒स्सुव॒ श्छन्द॒ ॐ । \textbf{ 22} \newline
                  \newline
                                                         \textbf{} \newline \newline
                                \textbf{ T.A.6.9.1} \newline
                  नमो॒ ब्रह्म॑णे धा॒रणं॑ मे अ॒स्त्व-नि॑राकरणं-धा॒रयि॑ता भूयासं॒ कर्ण॑योः श्रु॒तं माच्यो᳚ढ्वं॒ ममा॒मुष्य॒ ॐ । \textbf{ 23} \newline
                  \newline
                                                         \textbf{} \newline \newline
                                \textbf{ T.A.6.10.1} \newline
                  ऋ॒तं तपः॑ स॒त्यं तपः॑ श्रु॒तं तपः॑ शा॒न्तं तपो॒ दम॒ स्तपः॒ शम॒स्तपो॒ दानं॒ तपो॒ यज्ञ्ं॒ तपो॒ भूर्भुव॒स्सुव॒र्-ब्रह्मै॒-तदुपा᳚स्यै॒-तत्तपः॑ । \textbf{ 24} \newline
                  \newline
                                                         \textbf{} \newline \newline
                                \textbf{ T.A.6.11.1} \newline
                  यथा॑ वृ॒क्षस्य॑ स॒पुंष्पि॑तस्य दू॒राद् ग॒न्धो वा᳚त्ये॒वं पुण्य॑स्य क॒र्मणो॑ दू॒राद् ग॒न्धो वा॑ति॒ यथा॑ऽसिधा॒रां क॒र्त्तेऽव॑हिता-मव॒क्रामे॒ यद्युवे॒ युवे॒ हवा॑ वि॒ह्वयि॑ष्यामि क॒र्त्तं प॑तिष्या॒मीत्ये॒व-म॒मृता॑-दा॒त्मानं॑ जु॒गुफ्से᳚त् । \textbf{ 25} \newline
                  \newline
                                                         \textbf{} \newline \newline
                                \textbf{ T.A.6.12.1} \newline
                  अ॒णो-रणी॑यान् मह॒तो मही॑या-ना॒त्मा गुहा॑यां॒ निहि॑तोऽस्य ज॒न्तोः । तम॑क्रतुं पश्यति वीतशो॒को धा॒तुः प्र॒सादा᳚न्-महि॒मान॑मीशं ॥  स॒प्त प्रा॒णाः प्र॒भव॑न्ति॒ तस्मा᳚थ् स॒प्तार्चिषः॑ स॒मिधः॑ स॒प्त जि॒ह्वाः । स॒प्त इ॒मे लो॒का येषु॒ चर॑न्ति प्रा॒णा गु॒हाश॑या॒-न्निहि॑ताः स॒प्त स॑प्त ॥  अतः॑ समु॒द्रा गि॒रय॑श्च॒ सर्वे॒ऽस्माथ् स्यन्द॑न्ते॒ सिन्ध॑वः॒ सर्व॑रूपाः । अत॑श्च॒ विश्वा॒ ओष॑धयो॒ रसा᳚च्च॒ येनै॑ष भू॒त-स्ति॑ष्ठत्यन्तरा॒त्मा ॥  ब्र॒ह्मा दे॒वानां᳚ पद॒वीः क॑वी॒ना-मृषि॒र्-विप्रा॑णां महि॒षो मृ॒गाणां᳚ ।  श्ये॒नो गृध्रा॑णाꣳ॒॒ स्वधि॑ति॒र्-वना॑नाꣳ॒॒ सोमः॑ प॒वित्र॒ मत्ये॑ति॒ रेभन्न्॑ ॥  अ॒जा मेकां॒ ॅलोहि॑त-शुक्ल-कृ॒ष्णां ब॒ह्वीं प्र॒जां ज॒नय॑न्तीꣳ॒॒ सरू॑पां । अ॒जो ह्येको॑ जु॒षमा॑णोऽनु॒शेते॒ जहा᳚त्येनां भु॒क्त-भो॑गा॒मजो᳚ऽन्यः । \textbf{ 26} \newline
                  \newline
                                                                  \textbf{ T.A.6.12.2} \newline
                  हꣳ॒॒सः शु॑चि॒षद् वसु॑-रन्तरिक्ष॒-सद्धोता॑ वेदि॒ष-दति॑थिर्-दुरोण॒सत् ।  नृ॒षद्व॑र॒-सदृ॑त॒-सद् व्यो॑म॒-सद॒ब्जा गो॒जा ऋ॑त॒जा अ॑द्रि॒जा ऋ॒तं बृ॒हत् ॥  घृ॒तं मि॑मिक्षिरे घृ॒तम॑स्य॒ योनि॑र्-घृ॒ते श्रि॒तो घृ॒तमु॑वस्य॒ धाम॑ ।  अ॒नु॒ष्व॒धमाव॑ह मा॒दय॑स्व॒ स्वाहा॑ कृतं ॅवृषभ वक्षि ह॒व्यं ॥  स॒मु॒द्रा दू॒र्मिर्-मधु॑माꣳ॒॒ उदा॑र-दुपाꣳ॒॒शुना॒ सम॑मृत॒त्व मा॑नट् ।  घृ॒तस्य॒ नाम॒ गुह्यं॒ ॅयदस्ति॑ जि॒ह्वा दे॒वाना॑-म॒मृत॑स्य॒ नाभिः॑ ॥  व॒यं नाम॒ प्रब्र॑वामा घृ॒तेना॒स्मिन्. य॒ज्ञे धा॑रयामा॒ नमो॑भिः ।  उप॑ ब्र॒ह्मा शृ॑णवच्छ॒स्यमा॑नं॒ चतुः॑ शृङ्गो ऽवमीद् गौ॒र ए॒तत् ॥  च॒त्वारि॒ शृङ्गा॒ त्रयो॑ अस्य॒ पादा॒ द्वे शी॒र॒.षे स॒प्त हस्ता॑सो अ॒स्य ।  त्रिधा॑ ब॒द्धो वृ॑ष॒भो रो॑रवीति म॒हो दे॒वो मर्त्याꣳ॒॒ आवि॑वेश । \textbf{ 27} \newline
                  \newline
                                                                  \textbf{ T.A.6.12.3} \newline
                  त्रिधा॑ हि॒तं प॒णिभि॑र् गु॒ह्यमा॑नं॒ गवि॑-दे॒वासो॑ घृ॒तमन्व॑विन्दन्न् । इन्द्र॒ एकꣳ॒॒ सूर्य॒ एकं॑ जजान वे॒ना देकꣳ॑ स्व॒धया॒ निष्ट॑तक्षुः ॥  यो दे॒वानां᳚ प्रथ॒मं पु॒रस्ता॒द् विश्वा॒धियो॑ रु॒द्रो म॒हर्.षिः॑ । हि॒र॒ण्य॒ग॒र्भं प॑श्यत॒ जाय॑मानꣳ॒॒ सनो॑ दे॒वः शु॒भया॒ स्मृत्या॒ सम्ॅयु॑नक्तु ॥ यस्मा॒त्परं॒ नाप॑र॒ मस्ति॒ किञ्चि॒द्यस्मा॒न् नाणी॑यो॒ न ज्यायो᳚ऽस्ति॒ कश्चि॑त् ।  वृ॒क्ष इ॑व स्तब्धो दि॒वि ति॑ष्ठ॒-त्येक॒स्तेने॒दं पू॒र्णं पुरु॑षेण॒ सर्वं᳚ ॥  न कर्म॑णा न प्र॒जया॒ धने॑न॒ त्यागे॑नैके अमृत॒त्व-मा॑न॒शुः ।  परे॑ण॒ नाकं॒ निहि॑तं॒ गुहा॑यां ॅवि॒भ्राज॑दे॒तद् यत॑यो वि॒शन्ति॑ ॥ वे॒दा॒न्त॒ वि॒ज्ञान॒-सुनि॑श्चिता॒र्थाः सन्या॑स यो॒गाद्यत॑यः शुद्ध॒ सत्त्वाः᳚ ।  ते ब्र॑ह्मलो॒के तु॒ परा᳚न्तकाले॒ परा॑मृता॒त् परि॑मुच्यन्ति॒ सर्वे᳚ ( ) ॥ द॒ह्रं॒ ॅवि॒पा॒पं प॒रमे᳚श्म भूतं॒ ॅयत् पु॑ण्डरी॒कं पु॒रम॑द्ध्य सꣳ॒॒स्थं । त॒त्रा॒पि॒ द॒ह्रं ग॒गनं॑ ॅविशोक॒-स्तस्मि॑न्. यद॒न्तस्त-दुपा॑सित॒व्यं ॥  यो वेदादौ स्व॑रः प्रो॒क्तो॒ वे॒दान्ते॑ च  प्र॒तिष्ठि॑तः । तस्य॑ प्र॒कृति॑-लीन॒स्य॒ यः॒ परः॑ स॒ म॒हेश्व॑रः । \textbf{ 28} \newline
                  \newline
                                                        (अजो᳚ऽन्य॒ - आवि॑वेश॒ - सर्वे॑ च॒त्वारि॑ च) \textbf{12} \newline \newline
                                \textbf{ T.A.6.13.1} \newline
                  स॒ह॒स्र॒शीर्.षं॑ दे॒वं॒ ॅवि॒श्वाक्षं॑ ॅवि॒श्व शं॑ भुवं । विश्वं॑ ना॒राय॑णं दे॒व॒म॒क्षरं॑ पर॒मं प॒दं ॥  वि॒श्वतः॒ पर॑मान्नि॒त्यं॒ ॅवि॒श्वं ना॑राय॒णꣳ ह॑रिं । विश्व॑मे॒वेदं पुरु॑ष॒-स्तद् विश्व॒मुप॑जीवति ॥  पतिं॒ ॅविश्व॑स्या॒त्मेश्व॑रꣳ॒॒ शाश्व॑तꣳ शि॒वम॑च्युतं । ना॒राय॒णं म॑हाज्ञे॒यं॒ ॅवि॒श्वात्मा॑नं प॒राय॑णं ॥  ना॒राय॒ण प॑रो ज्यो॒ति॒रा॒त्मा ना॑रय॒णः प॑रः ।  ना॒राय॒ण प॑रं ब्र॒ह्म॒ त॒त्त्वं ना॑राय॒णः प॑रः । ना॒राय॒ण प॑रो ध्या॒ता॒ ध्या॒नं ना॑राय॒णः प॑रः ॥ यच्च॑ कि॒ञ्चिज्-ज॑गथ् स॒र्वं॒ दृ॒श्यते᳚ श्रूय॒तेऽपि॑ वा \textbf{ 29} \newline
                  \newline
                                                                  \textbf{ T.A.6.13.2} \newline
                  अन्त॑र् ब॒हिश्च॑ तथ् स॒र्वं॒ ॅव्या॒प्य ना॑राय॒णः स्थि॑तः ॥  अन॑न्त॒ मव्य॑यं क॒विꣳ स॑मु॒द्रेऽन्तं॑ ॅवि॒श्व श॑भुंवं । प॒द्म॒को॒श-प्र॑तीका॒शꣳ॒॒ हृ॒दयं॑ चाप्य॒धोमु॑खं ॥  अधो॑ नि॒ष्ट्या वि॑तस्त्या॒न्ते॒ ना॒भ्यामु॑परि॒ तिष्ठ॑ति ।  ज्वा॒ल॒मा॒ला कु॑लं भा॒ती॒ वि॒श्वस्या॑यत॒नं म॑हत् ॥  सन्त॑तꣳ शि॒लाभि॑स्तु॒ लंब॑त्या कोश॒सन्नि॑भं ।  तस्यान्ते॑ सुषि॒रꣳ सू॒क्ष्मं तस्मि᳚न्थ् स॒र्वं प्रति॑ष्ठितं ॥  तस्य॒ मद्ध्ये॑ म॒हान॑ग्निर् वि॒श्वार्चि॑र् वि॒श्वतो॑ मुखः । सोऽग्र॑भु॒ग् विभ॑जन् ति॒ष्ठ॒न्-नाहा॑र-मज॒रः क॒विः । ति॒र्य॒गू॒र्ध्व म॑धः शा॒यी॒ र॒श्मय॑स्तस्य॒ सन्त॑ता ( ) ॥ स॒न्ता॒पय॑ति स्वं दे॒हमापा॑दतल॒ मस्त॑कः । तस्य॒ मद्ध्ये॒ वह्नि॑शिखा अ॒णीयो᳚र्द्ध्वा व्य॒वस्थि॑तः ॥  नी॒लतो॑ यद॑ मद्ध्य॒स्था॒द् वि॒द्युल्ले॑खेव॒ भास्व॑रा ।  नी॒वार॒ शूक॑वत्त॒न्वी॒ पी॒ता भा᳚स्वत्य॒णूप॑मा ॥  तस्याः᳚ शिखा॒या म॑द्ध्ये प॒रमा᳚त्मा व्य॒वस्थि॑तः ।  स ब्रह्म॒ स शिवः॒ स हरिः॒ सेन्द्रः॒ सोऽक्ष॑रः पर॒मः स्व॒राट् । \textbf{ 30} \newline
                  \newline
                                                        (अपि॑ वा॒ - सन्त॑ता॒ षट् च॑) \textbf{13} \newline \newline
                                \textbf{ T.A.6.14.1} \newline
                  आ॒दि॒त्यो वा ए॒ष ए॒तन् म॒ण्डलं॒ तप॑ति॒ तत्र॒ ता ऋच॒स्तदृ॒चा म॑ण्डलꣳ॒॒ स ऋ॒चां ॅलो॒कोऽथ॒य ए॒ष ए॒तस्मि॑न् म॒ण्डले॒ऽर्चिर् दी॒प्यते॒ तानि॒ सामा॑नि॒ स सा॒म्नां ॅलो॒कोऽथ॒ य ए॒ष ए॒तस्मि॑न् म॒ण्डले॒ऽर्चिषि॒ पुरु॑ष॒स्तानि॒ यजूꣳ॑षि॒ स यजु॑षा मण्डलꣳ॒॒ स यजु॑षां ॅलो॒कः सैषा त्र॒य्येव॑ वि॒द्या त॑पति॒ य ए॒षो᳚ऽन्त-रा॑दि॒त्ये हि॑र॒ण्मयः॒ पुरु॑षः । \textbf{ 31} \newline
                  \newline
                                                         \textbf{} \newline \newline
                                \textbf{ T.A.6.15.1} \newline
                  आ॒दि॒त्यो वै तेज॒ ओजो॒ बलं॒ ॅयश॒-श्चक्षुः॒ श्रोत्र॑मा॒त्मा मनो॑ म॒न्युर्-मनु॑र्-मृ॒त्युः स॒त्यो मि॒त्रो वा॒युरा॑का॒शः प्रा॒णो लो॑कपा॒लः कः किं कं तथ् स॒त्यमन्न॑-म॒मृता॑ जी॒वो विश्वः॑ कत॒मः स्व॑य॒भुं ब्रह्मै॒ तदमृ॑त ए॒ष पुरु॑ष ए॒ष भू॒ताना॒-मधि॑पति॒र्-ब्रह्म॑णः॒ सायु॑ज्यꣳ सलो॒कता॑-माप्नो-त्ये॒तासा॑मे॒व दे॒वता॑नाꣳ॒॒ सायु॑ज्यꣳ सा॒र्.ष्टिताꣳ॑ समान लो॒कता॑-माप्नोति॒ य ए॒वं ॅवेदे᳚त्युप॒निषत् । \textbf{ 32} \newline
                  \newline
                                                         \textbf{} \newline \newline
                                \textbf{ T.A.6.16.1} \newline
                  निध॑नपतये॒ नमः । निध॑नपतान्तिकाय॒ नमः ।  ऊर्द्ध्वाय॒ नमः । ऊर्द्ध्वलिङ्गाय॒ नमः ।  हिरण्याय॒ नमः । हिरण्यलिङ्गाय॒ नमः ।  सुवर्णाय॒ नमः । सुवर्णलिङ्गाय॒ नमः ।  दिव्याय॒ नमः । दिव्यलिङ्गाय॒ नमः \textbf{ 33} \newline
                  \newline
                                                                  \textbf{ T.A.6.16.2} \newline
                  भवाय॒ नमः । भवलिङ्गाय॒ नमः ।  शर्वाय॒ नमः । शर्वलिङ्गाय॒ नमः ।  शिवाय॒ नमः । शिवलिङ्गाय॒ नमः ।  ज्वलाय॒ नमः । ज्वललिङ्गाय॒ नमः ।  आत्माय॒ नमः । आत्मलिङ्गाय॒ नमः ।  परमाय॒ नमः । परमलिङ्गाय॒ नमः ।  एतथ्सोमस्य॑ सूर्य॒स्य॒ सर्वलिङ्गꣳ॑ स्थाप॒य॒ति॒ पाणिमन्त्रं॑ पवि॒त्रं । \textbf{ 34} \newline
                  \newline
                                                         \textbf{} \newline \newline
                                \textbf{ T.A.6.17.1} \newline
                  स॒द्योजा॒तं प्र॑पद्या॒मि॒ स॒द्योजा॒ताय॒ वै नमो॒ नमः॑ । भ॒वे भ॑वे॒ नाति॑भवे भवस्व॒ मां । भ॒वोद्भ॑वाय॒ नमः॑ । \textbf{ 35} \newline
                  \newline
                                                         \textbf{} \newline \newline
                                \textbf{ T.A.6.18.1} \newline
                  वा॒म॒दे॒वाय॒ नमो᳚ ज्ये॒ष्ठाय॒ नमः॑ श्रे॒ष्ठाय॒ नमो॑ रु॒द्राय॒ नमः॒ काला॑य॒  नमः॒ कल॑विकरणाय॒ नमो॒ बल॑विकरणाय॒ नमो॒ बला॑य॒ नमो॒ बल॑प्रमथनाय॒ नमः॒ सर्व॑भूतदमनाय॒ नमो॑ म॒नोन्म॑नाय॒ नमः॑ । \textbf{ 36} \newline
                  \newline
                                                         \textbf{} \newline \newline
                                \textbf{ T.A.6.19.1} \newline
                  अ॒घोरे᳚भ्योऽथ॒ घोरे᳚भ्यो॒ घोर॒घोर॑तरेभ्यः ।  सर्वे᳚भ्यः सर्व॒ शर्वे᳚भ्यो॒ नम॑स्ते अस्तु रु॒द्ररू॑पेभ्यः । \textbf{ 37} \newline
                  \newline
                                                         \textbf{} \newline \newline
                                \textbf{ T.A.6.20.1} \newline
                  तत्पुरु॑षाय वि॒द्महे॑ महादे॒वाय॑ धीमहि ।  तन्नो॑ रुद्रः प्रचो॒दया᳚त् । \textbf{ 38} \newline
                  \newline
                                                         \textbf{} \newline \newline
                                \textbf{ T.A.6.21.1} \newline
                  ईशानः सर्व॑विद्या॒ना॒- मीश्वरः सर्व॑भूता॒नां॒ ब्रह्माधि॑पति॒र् ब्रह्म॒णोऽधि॑पति॒र् ब्रह्मा॑ शि॒वो मे॑ अस्तु सदाशि॒व्ॐ । \textbf{ 39} \newline
                  \newline
                                                         \textbf{} \newline \newline
                                \textbf{ T.A.6.22.1} \newline
                  नमो हिरण्यबाहवे हिरण्यवर्णाय हिरण्यरूपाय हिरण्यपतये ऽबिंकापतय उमापतये पशुपतये॑ नमो॒ नमः । \textbf{ 40} \newline
                  \newline
                                                         \textbf{} \newline \newline
                                \textbf{ T.A.6.23.1} \newline
                  ऋ॒तꣳ स॒त्यं प॑रं ब्र॒ह्म॒ पु॒रुषं॑ कृष्ण॒पिङ्ग॑लं । ऊ॒र्द्ध्वरे॑तं ॅवि॑रूपा॒क्षं॒ ॅवि॒श्वरू॑पाय॒ वै नमो॒ नमः॑ । \textbf{ 41} \newline
                  \newline
                                                         \textbf{} \newline \newline
                                \textbf{ T.A.6.24.1} \newline
                  सर्वो॒ वै रु॒द्रस्तस्मै॑ रु॒द्राय॒ नमो॑ अस्तु ।  पुरु॑षो॒ वै रु॒द्रः सन्म॒हो नमो॒ नमः॑ ।  विश्वं॑ भू॒तं भुव॑नं चि॒त्रं ब॑हु॒धा जा॒तं जाय॑मानं च॒ यत् । सर्वो॒ ह्ये॑ष रु॒द्रस्तस्मै॑ रु॒द्राय॒ नमो॑ अस्तु । \textbf{ 42} \newline
                  \newline
                                                         \textbf{} \newline \newline
                                \textbf{ T.A.6.25.1} \newline
                  कद्रु॒द्राय॒ प्रचे॑तसे मी॒ढुष्ट॑माय॒ तव्य॑से । वो॒चेम॒ शन्त॑मꣳ हृ॒दे ॥ सर्वो॒ह्ये॑ष रु॒द्रस्तस्मै॑ रु॒द्राय॒ नमो॑ अस्तु । \textbf{ 43} \newline
                  \newline
                                                         \textbf{} \newline \newline
                                \textbf{ T.A.6.26.1} \newline
                  यस्य॒ वै क॑ङ्कत्यग्नि-होत्र॒हव॑णी भवति॒ प्रत्ये॒वा-स्याहु॑तय-स्तिष्ठ॒न्त्यथो॒ प्रति॑ष्ठित्यै । \textbf{ 44} \newline
                  \newline
                                                         \textbf{} \newline \newline
                                \textbf{ T.A.6.27.1} \newline
                  (ठिस् ए꣡पन्सिऒन् इस् अप्पॆअरिन्ग् इन् ट्.श्.1.2.14.1 fऒर् कृ॒णु॒ष्व पाज॒ इति॒ पञ्च॑ । )  कृ॒णु॒ष्व पाजः॒ प्रसि॑ति॒न्न पृ॒थ्वीं ॅया॒हि राजे॒ वा॑मवाꣳ॒॒ इभे॑न । तृ॒ष्वीमनु॒ प्रसि॑तिं द्रूणा॒नोऽस्ता॑ऽसि॒ विद्ध्य॑ र॒क्ष स॒स्तपि॑ष्ठैः ॥ 1  तव॑ भ्र॒मास॑ आशु॒या प॑त॒न्त्यनु॑ स्पृश धृष॒ता शोशु॑चानः । तपूꣳ॑ष्यग्ने जु॒ह्वा॑ पत॒गांन स॑न्दितो॒ विसृ॑ज॒ विष्व॑ गु॒ल्काः ॥ 2   प्रति॒स्पशो॒ विसृ॑ज॒ तूर्णि॑ तमो॒ भवा॑ पा॒युर् वि॒शो अ॒स्या अद॑ब्धः । यो नो॑ दू॒रे अ॒घशꣳ॑सो॒ यो अन्त्य॑ग्ने॒ माकि॑ष्टे॒ व्यथि॒रा  द॑धर्.षीत् ॥ 3   उद॑ग्ने तिष्ठ॒ प्रत्या त॑नुष्व॒न्य॑ मित्राꣳ॑ ओषतात् तिग्महेते । यो नो॒ अरा॑तिꣳ समिधान च॒क्रे नी॒चातं ध॑क्ष्यत॒ सन्न शुष्कं᳚ ॥ 4  ऊ॒र्द्ध्वो भ॑व॒ प्रति॑विद्॒ध्या-ध्य॒स्मदा॒ विष्कृ॑णुष्व॒ दैव्या᳚न्यग्ने । अव॑स्थि॒रा त॑नुहि यातु॒ जूनां᳚ जा॒मिमजा॑मिं॒ प्रमृ॑णीहि॒ शत्रून्॑ ॥ 5  कृ॒णु॒ष्व पाज॒ इति॒ पञ्च॑ । \textbf{ 45} \newline
                  \newline
                                                         \textbf{} \newline \newline
                                \textbf{ T.A.6.28.1} \newline
                  अदि॑तिर्-दे॒वा ग॑न्ध॒र्वा म॑नु॒ष्याः᳚ पि॒तरो-ऽसु॑रा॒-स्तेषाꣳ॑ सर्व भू॒तानां᳚ मा॒ता मे॒दिनी॑ मह॒ता म॒ही सा॑वि॒त्री गा॑य॒त्री जग॑त्यु॒र्वी पृ॒थ्वी ब॑हु॒ला विश्वा॑ भू॒ता क॑त॒मा काया सा स॒त्ये-त्य॒मृतेति॑ वसि॒ष्ठः । \textbf{ 46} \newline
                  \newline
                                                         \textbf{} \newline \newline
                                \textbf{ T.A.6.29.1} \newline
                  आपो॒ वा इ॒दꣳ सर्वं॒ ॅविश्वा॑ भू॒तान्यापः॑ प्रा॒णा वा आपः॑ प॒शव॒ आपोऽन्न॒मापो -ऽमृ॑त॒मापः॑ स॒म्राडापो॑ वि॒राडापः॑ स्व॒राडापः॒ छन्दाꣳ॒॒स्यापो॒ ज्योतीꣳ॒॒ष्यापो॒ यजूꣳ॒॒ष्यापः॑ स॒त्यमापः॒ सर्वा॑ दे॒वता॒ आपो॒ भूर्भुव॒स्सुव॒राप॒ ॐ । \textbf{ 47} \newline
                  \newline
                                                         \textbf{} \newline \newline
                                \textbf{ T.A.6.30.1} \newline
                  आपः॑ पुनन्तु पृथि॒वीं पृ॑थि॒वी पू॒ता पु॑नातु॒ मां ।  पु॒नन्तु॒ ब्रह्म॑ण॒स्पति॒र् ब्रह्म॑ पू॒ता पु॑नातु॒  मां ॥   यदुच्छि॑ष्ट॒-मभो᳚ज्यं॒ ॅयद्वा॑ दु॒श्चरि॑तं॒ मम॑ । सर्वं॑ पुनन्तु॒ मामापो॑-ऽस॒ताञ्च॑ प्रति॒ग्रहꣳ॒॒ स्वाहा᳚ । \textbf{ 48} \newline
                  \newline
                                                         \textbf{} \newline \newline
                                \textbf{ T.A.6.31.1} \newline
                  अग्निश्च मा मन्युश्च मन्युपतयश्च मन्यु॑कृते॒भ्यः ।  पापेभ्यो॑ रक्ष॒न्तां । यदह्ना पाप॑मका॒र्॒.षं ।  मनसा वाचा॑ हस्ता॒भ्यां । पद्भ्या-मुदरे॑ण शि॒श्ना ।  अह॒स्तद॑वलु॒पंतु । यत्किञ्च॑ दुरि॒तं मयि॑ ।  इदमह-माममृ॑त यो॒नौ ।  सत्ये ज्योतिषि जुहो॑मि स्वा॒हा । \textbf{ 49} \newline
                  \newline
                                                         \textbf{} \newline \newline
                                \textbf{ T.A.6.32.1} \newline
                  सूर्यश्च मा मन्युश्च मन्युपतयश्च मन्यु॑कृते॒भ्यः । पापेभ्यो॑ रक्ष॒न्तां । यद्रात्रिया पाप॑मका॒र्॒.षं ।  मनसा वाचा॑ हस्ता॒भ्यां । पद्भ्या-मुदरे॑ण शि॒श्ना ।  रात्रि॒-स्तद॑वलु॒पंतु । यत्किञ्च॑ दुरि॒तं मयि॑ ।  इदमह-माममृ॑त यो॒नौ । सूर्ये ज्योतिषि जुहो॑मि स्वा॒हा । \textbf{ 50} \newline
                  \newline
                                                         \textbf{} \newline \newline
                                \textbf{ T.A.6.33.1} \newline
                  ओमित्येकाक्ष॑रं ब्र॒ह्म । अग्निर्देवता ब्रह्म॑ इत्या॒र्.षं । गायत्रं छन्दं परमात्मं॑ सरू॒पं । सायुज्यं ॅवि॑नियो॒गं । \textbf{ 51} \newline
                  \newline
                                                         \textbf{} \newline \newline
                                \textbf{ T.A.6.34.1} \newline
                  आया॑तु॒ वर॑दा दे॒वी॒ अ॒क्षरं॑ ब्रह्म॒ संमि॑तं ।  गा॒य॒त्रीं᳚ छन्द॑सां मा॒तेदं ब्र॑ह्म जु॒षस्व॑ मे ।  यदह्ना᳚त् कुरु॑ते पा॒पं॒ तदह्ना᳚त् प्रति॒मुच्य॑ते ।  यद् रात्रिया᳚त् कुरु॑ते पा॒पं॒ तद् रात्रिया᳚त् प्रति॒मुच्य॑ते ।  सर्व॑ व॒र्णे म॑हादे॒वि॒ स॒न्ध्या वि॑द्ये स॒रस्व॑ति । \textbf{ 52} \newline
                  \newline
                                                         \textbf{} \newline \newline
                                \textbf{ T.A.6.35.1} \newline
                  ओजो॑ऽसि॒ सहो॑ऽसि॒ बल॑मसि॒ भ्राजो॑ऽसि दे॒वानां॒ धाम॒नामा॑॑ऽसि॒ विश्व॑मसि वि॒श्वायुः॒ सर्व॑मसि स॒र्वायु-रभिभूर्ॐ-गायत्री-मावा॑हया॒मि॒  सावित्री-मावा॑हया॒मि॒ सरस्वती-मावा॑हया॒मि॒ छन्दर्.षी-नावा॑हया॒मि॒ श्रिय-मावा॑हया॒मि॒ गायत्रिया गायत्री छन्दो विश्वामित्र ऋषिः सविता देवताऽग्निर्मुखं ब्रह्मा शिरो विष्णुर्.हृदयꣳ रुद्रः  शिखा पृथिवीयोनिः प्राणापान-व्यानोदान-समाना सप्राणा श्वेतवर्णा साङ्ख्यायन-सगोत्रा गायत्री चतुर्विꣳशत्यक्षरा त्रिपदा॑ षट्कु॒क्षिः॒ पञ्च शीर्.षोपनयने वि॑नियो॒ग॒,  ॐ भूः । ॐ भुवः । ओꣳ सुवः । ॐ महः । ॐ जनः ।  ॐ तपः । ओꣳ स॒त्यं । ॐ तथ् स॑वि॒तुर्वरे᳚ण्यं॒ भर्गो॑ दे॒वस्य॑ धीमहि ।  धियो॒ यो नः॑ प्रचो॒दया᳚त् ।  ओमापो॒ ज्योती॒ रसो॒ऽमृतं॒ ब्रह्म॒ भूर्भुव॒स्सुव॒र्ॐ । \textbf{ 53} \newline
                  \newline
                                                         \textbf{} \newline \newline
                                \textbf{ T.A.6.36.1} \newline
                  उ॒त्तमे॑ शिख॑रे जा॒ते॒ भू॒म्यां प॑र्वत॒ मूर्द्ध॑नि ।  ब्रा॒ह्मणे᳚भ्यो-ऽभ्य॑नुज्ञा॒ता॒ ग॒च्छ दे॑वि य॒थासु॑खं ॥ स्तुतो मया वरदा वे॑दमा॒ता॒ प्रचोदयन्ती पवने᳚ द्विजा॒ता । आयुः पृथिव्यां-द्रविणं ब्र॑ह्मव॒र्च॒सं॒ मह्यं दत्वा प्रजातुं  ब्र॑ह्मलो॒कं । \textbf{ 54} \newline
                  \newline
                                                         \textbf{} \newline \newline
                                \textbf{ T.A.6.37.1} \newline
                  घृणिः॒ सूर्य॑ आदि॒त्यो न प्रभा॑-वा॒त्यक्ष॑रं । मधु॑क्षरन्ति॒ तद् र॑सं । स॒त्यं ॅवै तद् रस॒-मापो॒ ज्योती॒रसो॒ऽमृतं॒ ब्रह्म॒  भूर्भुव॒स्सुव॒र्ॐ । \textbf{ 55} \newline
                  \newline
                                                         \textbf{} \newline \newline
                                \textbf{ T.A.6.38.1} \newline
                  ब्रह्म॑ मेतु॒॒ मां । मधु॑ मेतु॒ मां । ब्रह्म॑-मे॒व मधु॑ मेतु॒ मां ॥ यास्ते॑ सोम प्र॒जाव॒थ्सो-ऽभि॒सो अ॒हं । दुःष्व॑प्न॒हन् दु॑रुष्वह ।  यास्ते॑ सोम प्रा॒णाꣳस्तां जु॑होमि ॥  त्रिसु॑पर्ण॒ मया॑चितं ब्राह्म॒णाय॑ दद्यात् । ब्र॒ह्म॒ह॒त्यां ॅवा ए॒ते घ्न॑न्ति ।  ये ब्रा᳚ह्म॒णा-स्त्रिसु॑पर्णं॒ पठ॑न्ति । ते सोमं॒ प्राप्नु॑वन्ति । आ॒स॒ह॒स्रात् प॒क्तिं पुन॑न्ति । ॐ । \textbf{ 56} \newline
                  \newline
                                                         \textbf{} \newline \newline
                                \textbf{ T.A.6.39.1} \newline
                  ब्रह्म॑ मे॒धया᳚ । मधु॑ मे॒धया᳚ । ब्रह्म॑मे॒व मधु॑ मे॒धया᳚ ॥ अ॒द्या नो॑ देव सवितः प्र॒जाव॑थ्सावीः॒ सौभ॑गं । परा॑ दुः॒ष्वप्नि॑यꣳ सुव । विश्वा॑नि देव सवितर्-दुरि॒तानि॒ परा॑सुव । यद् भ॒द्रं तन्म॒ आसु॑व ॥ मधु॒वाता॑ ऋताय॒ते मधु॑क्षरन्ति॒ सिन्ध॑वः । माद्ध्वी᳚र्नः स॒न्त्वोष॑धीः । मधु॒नक्त॑ मु॒तोषसि॒ मधु॑म॒त् पार्थि॑वꣳ॒॒ रजः॑ । मधु॒द्यौर॑स्तु नः पि॒ता । मधु॑मान्नो॒ वन॒स्पति॒र्-मधु॑माꣳ अस्तु॒ सूर्यः॑ ।  माद्ध्वी॒ र्गावो॑ भवन्तु नः ॥  य इ॒मं त्रिसु॑पर्ण॒-मया॑चितं ब्राह्म॒णाय॑ दद्यात् ।  भ्रू॒ण॒ह॒त्यां ॅवा ए॒ते घ्न॑न्ति । ये ब्रा᳚ह्म॒णा-स्त्रिसु॑पर्णं॒ पठ॑न्ति । ते सोमं॒ प्राप्नु॑वन्ति । आ॒स॒ह॒स्रात् प॒क्तिं पुन॑न्ति । ॐ । \textbf{ 57} \newline
                  \newline
                                                         \textbf{} \newline \newline
                                \textbf{ T.A.6.40.1} \newline
                  ब्रह्म॑ मे॒धवा᳚ । मधु॑ मे॒धवा᳚ । ब्रह्म॑मे॒व मधु॑ मे॒धवा᳚ ॥ ब्र॒ह्मा दे॒वानां᳚ पद॒वीः क॑वी॒ना-मृषि॒र् विप्रा॑णां महि॒षो मृ॒गाणां᳚ ।  श्ये॒नो गृद्ध्रा॑णाꣳ॒॒ स्वधि॑ति॒र्-वना॑नाꣳ॒॒ सोमः॑ प॒वित्र॒॒-मत्ये॑ति॒ रेभन्न्॑ ॥  हꣳ॒॒सः शु॑चि॒षद् वसु॑रन्तरिक्ष॒ सद्धोता॑- वेदि॒ष-दति॑थिर्-दुरोण॒सत् ।  नृ॒षद्व॑र॒-सदृ॑त॒-सद् व्यो॑म॒-सद॒ब्जा- गो॒जा ऋ॑त॒जा अ॑द्रि॒जा ऋ॒तं बृ॒हत् ॥ ऋ॒चेत्वा॑ रु॒चेत्वा॒ समिथ् स्र॑वन्ति स॒रितो॒ न धेना᳚: ।  अ॒न्तर्. हृ॒दा मन॑सा पू॒यमा॑नाः ।  घृ॒तस्य॒ धारा॑ अ॒भिचा॑कशीमि ।  हि॒र॒ण्ययो॑ वेत॒सो मद्ध्य॑ आसां ।  तस्मि᳚न्थ् सुप॒र्णो म॑धु॒कृत् कु॑ला॒यी भज॑न्नास्ते॒ मधु॑ दे॒वता᳚भ्यः ।  तस्या॑ सते॒ हर॑यः स॒प्ततीरे᳚ स्व॒धां दुहा॑ना अ॒मृत॑स्य॒ धारां᳚ । य इ॒दं त्रिसु॑पर्ण॒-मया॑चितं ब्राह्म॒णाय॑ दद्यात् । वी॒र॒ह॒त्यां ॅवा ए॒ते घ्न॑न्ति । ये ब्रा᳚ह्म॒णा-स्त्रिसु॑पर्णं॒ पठ॑न्ति ।  ते सोमं॒ प्राप्नु॑वन्ति । आ॒स॒ह॒स्रात् पं॒क्तिं पुन॑न्ति । ॐ । \textbf{ 58} \newline
                  \newline
                                                         \textbf{} \newline \newline
                                \textbf{ T.A.6.41.1} \newline
                  मे॒धादे॒वी जु॒षमा॑णा न॒ आगा᳚द् वि॒श्वाची॑ भ॒द्रा सु॑मन॒स्य मा॑ना । त्वया॒ जुष्टा॑ नु॒दमा॑ना दु॒रुक्ता᳚न् बृ॒हद्व॑देम वि॒दथे॑ सु॒वीराः᳚ ॥  त्वया॒ जुष्ट॑ ऋ॒षिर् भ॑वति देवि॒ त्वया॒ ब्रह्मा॑ऽऽग॒तश्री॑ रु॒त त्वया᳚ । त्वया॒ जुष्ट॑श्चि॒त्रं ॅवि॑न्दते वसु॒ सा नो॑ जुषस्व॒ द्रवि॑णो नमेधे । \textbf{ 59} \newline
                  \newline
                                                         \textbf{} \newline \newline
                                \textbf{ T.A.6.42.1} \newline
                  मे॒धां म॒ इन्द्रो॑ ददातु मे॒धां दे॒वी सर॑स्वती ।  मे॒धां मे॑ अ॒श्विना॑-वु॒भावाध॑त्तां॒ पुष्क॑रस्रजा ॥ अ॒फ्स॒रासु॑ च॒ या मे॒धा ग॑न्ध॒र्वेषु॑ च॒ यन्मनः॑ ।  दैवीं᳚ मे॒धा सर॑स्वती॒ सा मां᳚ मे॒धा सु॒रभि॑र् जुषताꣳ॒॒ स्वाहा᳚ । \textbf{ 60} \newline
                  \newline
                                                         \textbf{} \newline \newline
                                \textbf{ T.A.6.43.1} \newline
                  आमां᳚ मे॒धा सु॒रभि॑र् वि॒श्वरू॑पा॒ हिर॑ण्यवर्णा॒ जग॑ती जग॒म्या । ऊर्ज॑स्वती॒ पय॑सा॒ पिन्व॑माना॒ सा मां᳚ मे॒धा सु॒प्रती॑का जुषन्तां । \textbf{ 61} \newline
                  \newline
                                                         \textbf{} \newline \newline
                                \textbf{ T.A.6.44.1} \newline
                  मयि॑ मे॒धां मयि॑ प्र॒जां मय्य॒ग्निस्तेजो॑ दधातु॒ मयि॑ मे॒धां मयि॑ प्र॒जां मयीन्द्र॑ इन्द्रि॒यं द॑धातु॒ मयि॑ मे॒धां मयि॑ प्र॒जां मयि॒ सूर्यो॒ भ्राजो॑ दधातु । \textbf{ 62} \newline
                  \newline
                                                         \textbf{} \newline \newline
                                \textbf{ T.A.6.45.1} \newline
                  अपै॑तु मृ॒त्यु-र॒मृत॑न्न॒ आग॑न् वैवस्व॒तो नो॒ अभ॑यङ्कृणोतु । प॒र्णं ॅवन॒स्पते॑ रिवा॒भिनः॑ शीयताꣳ र॒यिः सच॑तान्नः॒ शची॒पतिः॑ । \textbf{ 63} \newline
                  \newline
                                                         \textbf{} \newline \newline
                                \textbf{ T.A.6.46.1} \newline
                  परं॑ मृत्यो॒ अनु॒ परे॑हि॒ पन्थां॒ ॅयस्ते॒स्व इत॑रो देव॒याना᳚त् । चक्षु॑ष्मते शृण्व॒ते ते᳚ ब्रवीमि॒ मानः॑ प्र॒जाꣳ री॑रिषो॒ मोत वी॒रान् । \textbf{ 64} \newline
                  \newline
                                                         \textbf{} \newline \newline
                                \textbf{ T.A.6.47.1} \newline
                  वातं॑ प्रा॒णं मन॑सा॒ न्वार॑भामहे प्र॒जाप॑तिं॒ ॅयो भुव॑नस्य गो॒पाः । सनो॑ मृ॒त्यो स्त्रा॑यतां॒ पात्वꣳह॑सो॒ ज्योग् जी॒वा ज॒राम॑शीमहि । \textbf{ 65} \newline
                  \newline
                                                         \textbf{} \newline \newline
                                \textbf{ T.A.6.48.1} \newline
                  अ॒मु॒त्र॒ भूया॒दध॒ यद्य॒मस्य॒ बृह॑स्पते अ॒भिश॑स्ते॒र मु॑ञ्चः । प्रत्यौ॑हता म॒श्विना॑ मृ॒त्यु म॑स्माद् दे॒वाना॑मग्ने भि॒षजा॒ शची॑भिः । \textbf{ 66} \newline
                  \newline
                                                         \textbf{} \newline \newline
                                \textbf{ T.A.6.49.1} \newline
                  हरिꣳ॒॒ हर॑न्त॒- मनु॑यन्ति दे॒वा विश्व॒स्येशा॑नं ॅवृष॒भं म॑ती॒नां । ब्रह्म॒ सरू॑प॒-मनु॑मे॒दमा॑गा॒-दय॑नं॒ मा विव॑धी॒र् विक्र॑मस्व । \textbf{ 67} \newline
                  \newline
                                                         \textbf{} \newline \newline
                                \textbf{ T.A.6.50.1} \newline
                  शल्कै॑र॒ग्नि-मि॑न्धा॒न उ॒भौ लो॒कौ स॑नेम॒हं ।  उ॒भयो᳚ र्लो॒कया॑र्. ऋ॒ध्द्वाऽति॑ मृ॒त्युं त॑राम्य॒हं । \textbf{ 68} \newline
                  \newline
                                                         \textbf{} \newline \newline
                                \textbf{ T.A.6.51.1} \newline
                  मा छि॑दो मृत्यो॒ मा व॑धी॒र्मा मे॒ बलं॒ ॅविवृ॑हो॒ मा प्रमो॑षीः । प्र॒जां मा मे॑ रीरिष॒ आयु॑रुग्र नृ॒चक्ष॑सं त्वा ह॒विषा॑ विधेम । \textbf{ 69} \newline
                  \newline
                                                         \textbf{} \newline \newline
                                \textbf{ T.A.6.52.1} \newline
                  मा नो॑ म॒हान्त॑मु॒त मा नो॑ अर्भ॒कं मा न॒ उक्ष॑न्तमु॒त मा न॑ उक्षि॒तं ।  मा नो॑ वधीः पि॒तरं॒ मोत मा॒तरं॑ प्रि॒या मा न॑स्त॒नुवो॑ रुद्र रीरिषः । \textbf{ 70} \newline
                  \newline
                                                         \textbf{} \newline \newline
                                \textbf{ T.A.6.53.1} \newline
                  मा न॑स्तो॒के तन॑ये॒ मा न॒ आयु॑षि॒ मा नो॒ गोषु॒ मा नो॒ अश्वे॑षु रीरिषः । वी॒रान्मा नो॑ रुद्र भामि॒तोव॑धीर्.-ह॒विष्म॑न्तो॒ नम॑सा विधेम ते । \textbf{ 71} \newline
                  \newline
                                                         \textbf{} \newline \newline
                                \textbf{ T.A.6.54.1} \newline
                  प्रजा॑पते॒ न त्वदे॒ता-न्य॒न्यो विश्वा॑ जा॒तानि॒ परि॒ता ब॑भूव । यत् का॑मास्ते जुहु॒मस्तन्नो॑ अस्तु व॒यꣳ स्या॑म॒ पत॑यो रयी॒णां । \textbf{ 72} \newline
                  \newline
                                                         \textbf{} \newline \newline
                                \textbf{ T.A.6.55.1} \newline
                  स्व॒स्ति॒दा वि॒शस्पति॑र् वृत्र॒हा विमृधो॑ व॒शी ।  वृषेन्द्रः॑ पु॒र ए॑तु नस्स्वस्ति॒दा अ॑भयं क॒रः । \textbf{ 73} \newline
                  \newline
                                                         \textbf{} \newline \newline
                                \textbf{ T.A.6.56.1} \newline
                  त्र्यं॑बकं ॅयजामहे सुग॒न्धिं पु॑ष्टि॒वर्द्ध॑नं । उ॒र्वा॒रु॒कमि॑व॒ बन्ध॑नान्-मृ॒त्योर् मु॑क्षीय॒ माऽमृता᳚त् । \textbf{ 74} \newline
                  \newline
                                                         \textbf{} \newline \newline
                                \textbf{ T.A.6.57.1} \newline
                  ये ते॑ स॒हस्र॑म॒युतं॒ पाशा॒ मृत्यो॒ मर्त्या॑य॒ हन्त॑वे । तान्. य॒ज्ञ्स्य॑ मा॒यया॒ सर्वा॒नव॑ यजामहे । \textbf{ 75} \newline
                  \newline
                                                         \textbf{} \newline \newline
                                \textbf{ T.A.6.58.1} \newline
                  मृ॒त्यवे॒ स्वाहा॑ मृ॒त्यवे॒ स्वाहा᳚ । \textbf{ 76} \newline
                  \newline
                                                         \textbf{} \newline \newline
                                \textbf{ T.A.6.59.1} \newline
                  दे॒वकृ॑त॒स्यैन॑सो-ऽव॒यज॑नमसि॒ स्वाहा᳚ ।  म॒नु॒ष्य॑कृत॒स्यैन॑सो ऽव॒यज॑नमसि॒ स्वाहा᳚ ।  पि॒तृकृ॑त॒स्यैन॑सो ऽव॒यज॑नमसि॒ स्वाहा᳚ ।  आ॒त्मकृ॑त॒स्यैन॑सो ऽव॒यज॑नमसि॒ स्वाहा᳚ ।  अ॒न्यकृ॑त॒स्यैन॑सो ऽव॒यज॑नमसि॒ स्वाहा᳚ ।  अ॒स्मत्कृ॑त॒स्यैन॑सो ऽव॒यज॑नमसि॒ स्वाहा᳚ ।  यद्दि॒वा च॒ नक्तं॒ चैन॑श्चकृ॒म तस्या॑ व॒यज॑नमसि॒ स्वाहा᳚ ।  यथ् स्व॒पन्त॑श्च॒ जाग्र॑त॒-श्चैन॑श्च-कृ॒म तस्या॑ व॒यज॑नमसि॒ स्वाहा᳚ ।  यथ् सु॒षुप्त॑श्च॒ जाग्र॑त॒-श्चैन॑श्च-कृ॒म तस्या॑ व॒यज॑नमसि॒ स्वाहा᳚ ।  यद् वि॒द्वाꣳस॒श्चा वि॑द्वाꣳस॒श्चैन॑श्च-कृ॒म तस्या॑ व॒यज॑नमसि॒ स्वाहा᳚ ।  एनस एनसो वयजनम॑सि स्वा॒हा । \textbf{ 77} \newline
                  \newline
                                                         \textbf{} \newline \newline
                                \textbf{ T.A.6.60.1} \newline
                  यद्वो॑ देवाश्चकृ॒म जि॒ह्वया॑ गु॒रुमन॑सो वा॒ प्रयु॑ती देव॒ हेड॑नं ।  अरा॑वा॒यो नो॑ अ॒भिदु॑च्छुना॒यते॒ तस्मि॒न् तदेनो॑ वसवो॒ निधे॑तन॒ स्वाहा᳚ । \textbf{ 78} \newline
                  \newline
                                                         \textbf{} \newline \newline
                                \textbf{ T.A.6.61.1} \newline
                  कामोऽकार्.षी᳚न् नमो॒ नमः ।  कामोऽकार्.षीत् कामः करोति नाहं करोमि  कामः कर्त्ता नाहं कर्त्ता कामः॑ कार॒यिता नाहं॑ कार॒यिता एष ते काम कामा॑य स्वा॒हा । \textbf{ 79} \newline
                  \newline
                                                         \textbf{} \newline \newline
                                \textbf{ T.A.6.62.1} \newline
                  मन्युरकार्.षी᳚न् नमो॒ नमः ।  मन्युरकार्.षीन् मन्युः करोति नाहं करोमि मन्युः कर्त्ता नाहं कर्त्ता मन्युः॑ कार॒यिता नाहं॑ कार॒यिता एष ते मन्यो मन्य॑वे स्वा॒हा । \textbf{ 80} \newline
                  \newline
                                                         \textbf{} \newline \newline
                                \textbf{ T.A.6.63.1} \newline
                  तिलाञ्जुहोमि सरसाꣳ सपिष्टान् गन्धार मम चित्ते रम॑न्तु स्वा॒हा ॥ गावो हिरण्यं धनमन्नपानꣳ सर्वेषाꣳ श्रि॑यै स्वा॒हा ॥ श्रियञ्च लक्ष्मिञ्च पुष्टिञ्च कीर्त्तिं॑ चा नृ॒ण्यतां ।  ब्रह्मण्यं ब॑हुपु॒त्रतां । श्रद्धामेधे प्रजाः सन्ददा॑तु स्वा॒हा । \textbf{ 81} \newline
                  \newline
                                                         \textbf{} \newline \newline
                                \textbf{ T.A.6.64.1} \newline
                  तिलाः कृष्णा-स्ति॑लाः श्वे॒ता॒-स्तिलाः सौम्या व॑शानु॒गाः ।  तिलाः पुनन्तु॑ मे पा॒पं॒ ॅयत्किञ्चिद् दुरितं म॑यि स्वा॒हा ॥ चोर॒स्यान्नं न॑वश्रा॒द्धं॒ ब्र॒ह्म॒हा गु॑रुत॒ल्पगः । गोस्तेयꣳ स॑रापा॒नं॒ भ्रूणहत्या तिला शान्तिꣳ शमय॑न्तु स्वा॒हा ॥  श्रीश्च लक्ष्मीश्च पुष्टीश्च कीर्त्तिं॑ चा नृ॒ण्यतां ।  ब्रह्मण्यं ब॑हुपु॒त्रतां ।  श्रद्धामेधे प्रज्ञातु जातवेदः संददा॑तु स्वा॒हा । \textbf{ 82} \newline
                  \newline
                                                         \textbf{} \newline \newline
                                \textbf{ T.A.6.65.1} \newline
                  प्राणापान-व्यानोदान-समाना मे॑ शुद्ध्य॒न्तां॒  ज्योति॑ र॒हं ॅवि॒रजा॑ विपा॒प्मा भू॑यासꣳ॒॒ स्वाहा᳚ ॥  वाङ्-मन-श्चक्षुः-श्रोत्र-जिह्वा-घ्राण-रेतो-बुद्ध्याकूतिः सङ्कल्पा मे॑ शुद्ध्य॒न्तां॒ ज्योति॑ र॒हं ॅवि॒रजा॑ विपा॒प्मा भू॑यासꣳ॒॒ स्वाहा᳚ ॥  त्वक्-चर्म-माꣳस-रुधिर-मेदो-मज्जा-स्नायवो-ऽस्थीनि  मे॑ शुद्ध्य॒न्तां॒ ज्योति॑ र॒हं ॅवि॒रजा॑ विपा॒प्मा भू॑यासꣳ॒॒ स्वाहा᳚ ॥  शिरः पाणि पाद पार्श्व पृष्ठो-रूदर-जङ्घ-शिश्र्नोपस्थ पायवो मे॑ शुद्ध्य॒न्तां॒ ज्योति॑ र॒हं ॅवि॒रजा॑ विपा॒प्मा भू॑यासꣳ॒॒ स्वाहा᳚ ॥  उत्तिष्ठ पुरुष हरित-पिङ्गल लोहिताक्षि देहि देहि ददापयिता मे॑ शुद्ध्य॒न्तां॒ ज्योति॑ र॒हं ॅवि॒रजा॑ विपा॒प्मा भू॑यासꣳ॒॒ स्वाहा᳚ । \textbf{ 83} \newline
                  \newline
                                                         \textbf{} \newline \newline
                                \textbf{ T.A.6.66.1} \newline
                  पृथिव्याप स्तेजो वायु-राकाशा मे॑ शुद्ध्य॒न्तां॒  ज्योति॑ र॒हं ॅवि॒रजा॑ विपा॒प्मा भू॑यासꣳ॒॒ स्वाहा᳚ ॥  शब्द-स्पर्.श-रूपरस-गन्धा मे॑ शुद्ध्य॒न्तां॒  ज्योति॑ र॒हं ॅवि॒रजा॑ विपा॒प्मा भू॑यासꣳ॒॒ स्वाहा᳚ ॥  मनो-वाक्-काय-कर्माणि मे॑ शुद्ध्य॒न्तां॒  ज्योति॑ र॒हं ॅवि॒रजा॑ विपा॒प्मा भू॑यासꣳ॒॒ स्वाहा᳚ ॥  अव्यक्तभावै-र॑हङ्का॒र॒र् ज्योति॑ र॒हं ॅवि॒रजा॑ विपा॒प्मा भू॑यासꣳ॒॒ स्वाहा᳚ ॥  आत्मा मे॑ शुद्ध्य॒न्तां॒ ज्योति॑ र॒हं ॅवि॒रजा॑ विपा॒प्मा भू॑यासꣳ॒॒ स्वाहा᳚ ॥  अन्तरात्मा मे॑ शुद्ध्य॒न्तां॒ ज्योति॑ र॒हं ॅवि॒रजा॑ विपा॒प्मा भू॑यासꣳ॒॒ स्वाहा᳚ ॥  परमात्मा मे॑ शुद्ध्य॒न्तां॒ ज्योति॑ र॒हं ॅवि॒रजा॑ विपा॒प्मा भू॑यासꣳ॒॒ स्वाहा᳚ ॥  क्षु॒धे स्वाहा᳚ ॥ क्षुत्पि॑पासाय॒ स्वाहा᳚ ॥  विवि॑ट्यै॒ स्वाहा᳚ ॥ ऋग्वि॑धानाय॒ स्वाहा᳚ ॥ क॒षो᳚त्काय॒ स्वाहा᳚ ॥  क्षु॒त्पि॒पा॒साम॑लं ज्ये॒ष्ठा॒म॒ल॒क्ष्मीर् ना॑शया॒म्यहं ।  अभू॑ति॒-मस॑मृद्धि॒ञ्च॒ सर्वां (सर्वा) निर्णुद मे पाप्मा॑नꣳ स्वा॒हा ॥  अन्नमय-प्राणमय-मनोमय-विज्ञानमय-मानन्दमय-मात्मा मे॑ शुद्ध्य॒न्तां॒ ज्योति॑ र॒हं ॅवि॒रजा॑ विपा॒प्मा भू॑यासꣳ॒॒ स्वाहा᳚ । \textbf{ 84} \newline
                  \newline
                                                         \textbf{} \newline \newline
                                \textbf{ T.A.6.67.1} \newline
                  अ॒ग्नये॒ स्वाहा᳚ । विश्वे᳚भ्यो दे॒वेभ्यः॒ स्वाहा᳚ । ध्रु॒वाय॑ भू॒माय॒ स्वाहा᳚ । ध्रु॒व॒क्षित॑ये॒ स्वाहा᳚ । अ॒च्यु॒त॒क्षित॑ये॒ स्वाहा᳚ । अ॒ग्नये᳚ स्विष्ट॒कृते॒ स्वाहा᳚ ॥ धर्मा॑य॒ स्वाहा᳚ । अध॑र्माय॒ स्वाहा᳚ । अ॒द्भ्यः स्वाहा᳚ । ओ॒ष॒धि॒व॒न॒स्प॒तिभ्यः॒ स्वाहा᳚ \textbf{ 85} \newline
                  \newline
                                                                  \textbf{ T.A.6.67.2} \newline
                  र॒क्षो॒दे॒व॒ज॒नेभ्यः॒ स्वाहा᳚ ।  गृह्या᳚भ्यः॒ स्वाहा᳚ । अ॒व॒साने᳚भ्यः॒ स्वाहा᳚ । अ॒व॒सान॑पतिभ्यः॒ स्वाहा᳚ ।  स॒र्व॒भू॒तेभ्यः॒ स्वाहा᳚ । कामा॑य॒ स्वाहा᳚ । अ॒न्तरि॑क्षाय॒ स्वाहा᳚ । यदेज॑ति॒ जग॑ति॒ यच्च॒ चेष्ट॑ति॒ नाम्नो॑ भा॒गोऽयं नाम्ने॒ स्वाहा᳚ ।  पृ॒थि॒व्यै स्वाहा᳚ । अ॒न्तरि॑क्षाय॒ स्वाहा᳚ \textbf{ 86} \newline
                  \newline
                                                                  \textbf{ T.A.6.67.3} \newline
                  दि॒वे स्वाहा᳚ । सूर्या॑य॒ स्वाहा᳚ । च॒न्द्रम॑से॒ स्वाहा᳚ । नक्ष॑त्रेभ्यः॒ स्वाहा᳚ । इन्द्रा॑य॒ स्वाहा᳚ । बृह॒स्पत॑ये॒ स्वाहा᳚ । प्र॒जाप॑तये॒ स्वाहा᳚ ।  ब्रह्म॑णे॒ स्वाहा᳚ । स्व॒धा पि॒तृभ्यः॒ स्वाहा᳚ ।  नमो॑ रु॒द्राय॑ पशु॒पत॑ये॒ स्वाहा᳚ \textbf{ 87} \newline
                  \newline
                                                                  \textbf{ T.A.6.67.4} \newline
                  दे॒वेभ्यः॒ स्वाहा᳚ । पि॒तृभ्यः॑ स्व॒धाऽस्तु॑ । भू॒तेभ्यो॒ नमः॑ । म॒नु॒ष्ये᳚भ्यो॒ हन्ता᳚ । प्र॒जाप॑तये॒ स्वाहा᳚ । प॒र॒मे॒ष्ठिने॒ स्वाहा᳚ ॥  यथा कू॑पः श॒तधा॑रः स॒हस्र॑धारो॒ अक्षि॑तः ।  ए॒वा मे॑ अस्तु धा॒न्यꣳ स॒हस्र॑धार॒-मक्षि॑तं । धन॑धान्यै॒ स्वाहा᳚ ॥  ये भू॒ताः प्र॒चर॑न्ति॒ दिवा॒नक्तं॒ बलि॑-मि॒च्छन्तो॑ वि॒तुद॑स्य॒ प्रेष्याः᳚ ( ) । तेभ्यो॑ ब॒लिं पु॑ष्टि॒कामो॑ हरामि॒ मयि॒ पुष्टिं॒ पुष्टि॑पतिर् दधातु॒ स्वाहा᳚ । \textbf{ 88} \newline
                  \newline
                                                        (ओ॒ष॒धि॒व॒न॒स्प॒तिभ्यः॒ स्वाहा॒ - ऽन्तरि॑क्षाय॒ स्वाहा॒ - नमो॑ रु॒द्राय॑ पशु॒पत॑ये॒ स्वाहा॑ - वि॒तुद॑स्य॒ प्रेष्या॒ एकं॑ च) \textbf{67} \newline \newline
                                \textbf{ T.A.6.68.1} \newline
                  ओं᳚ तद् ब्र॒ह्म । ओं᳚ तद् वा॒युः । ओं᳚ तदा॒त्मा । ओं᳚ तथ् स॒त्यं । ओं᳚ तथ् सर्वं᳚ । ओं᳚ तत् पुरो॒र् नमः ।  अन्तश्चरति॑ भूते॒षु॒ गुहायां ॅवि॑श्व मू॒र्त्तिषु ।  त्वं ॅयज्ञ्स्त्वं ॅवषट्कारस्त्व-मिद्रस्त्वꣳ रुद्रस्त्वं-ॅविष्णुस्त्वं ब्रह्मत्वं॑ प्रजा॒पतिः ।  त्वं त॑दाप॒ आपो॒ ज्योती॒ रसो॒ऽमृतं॒ ब्रह्म॒ भूर्भुव॒स्सुव॒र्ॐ । \textbf{ 89} \newline
                  \newline
                                                         \textbf{} \newline \newline
                                \textbf{ T.A.6.69.1} \newline
                  श्र॒द्धायां᳚ प्रा॒णे निवि॑ष्टो॒ऽमृतं॑ जुहोमि ।  श्र॒द्धाया॑मपा॒ने निवि॑ष्टो॒ऽमृतं॑ जुहोमि ।  श्र॒द्धायां᳚ ॅव्या॒ने निवि॑ष्टो॒ऽमृतं॑ जुहोमि ।  श्र॒द्धाया॑मुदा॒ने निवि॑ष्टो॒ऽमृतं॑ जुहोमि ।  श्र॒द्धायाꣳ॑ समा॒ने निवि॑ष्टो॒ऽमृतं॑ जुहोमि ।  ब्रह्म॑णि म आ॒त्माऽमृ॑त॒त्वाय॑ ॥ अ॒मृ॒तो॒प॒स्तर॑णमसि ॥ श्र॒द्धायां᳚ प्रा॒णे निवि॑ष्टो॒ऽमृतं॑ जुहोमि ।  शि॒वो मा॑ वि॒शाप्र॑दाहाय । प्रा॒णाय॒ स्वाहा᳚ ।  श्र॒द्धाया॑मपा॒ने निवि॑ष्टो॒ऽमृतं॑ जुहोमि ।  शि॒वो मा॑ वि॒शाप्र॑दाहाय । अ॒पा॒नाय॒ स्वाहा᳚ ।  श्र॒द्धायां᳚ ॅव्या॒ने निवि॑ष्टो॒ऽमृतं॑ जुहोमि । शि॒वो मा॑ वि॒शाप्र॑दाहाय । व्या॒नाय॒ स्वाहा᳚ ।  श्र॒द्धाया॑-मुदा॒ने निवि॑ष्टो॒ऽमृतं॑ जुहोमि ।  शि॒वो मा॑ वि॒शाप्र॑दाहाय । उ॒दा॒नाय॒ स्वाहा᳚ ।  श्र॒द्धायाꣳ॑ समा॒ने निवि॑ष्टो॒ऽमृतं॑ जुहोमि । शि॒वो मा॑ वि॒शाप्र॑दाहाय । स॒मा॒नाय॒ स्वाहा᳚ ।  ब्रह्म॑णि म आ॒त्माऽमृ॑त॒त्वाय॑ ॥ अ॒मृ॒ता॒पि॒धा॒नम॑सि । \textbf{ 90} \newline
                  \newline
                                                         \textbf{} \newline \newline
                                \textbf{ T.A.6.70.1} \newline
                  श्र॒द्धायां᳚ प्रा॒णे निवि॑श्या॒ऽमृतꣳ॑ हु॒तं । प्रा॒ण मन्ने॑नाप्यायस्व ।  श्र॒द्धाया॑मपा॒ने निवि॑श्या॒ऽमृतꣳ॑ हु॒तं । अ॒पा॒न मन्ने॑नाप्यायस्व ।  श्र॒द्धायां᳚ ॅव्या॒ने निवि॑श्या॒ऽमृतꣳ॑ हु॒तं । व्या॒न मन्ने॑नाप्यायस्व ।  श्र॒द्धाया॑-मुदा॒ने निवि॑श्या॒ऽमृतꣳ॑ हु॒तं । उ॒दा॒न मन्ने॑नाप्यायस्व ।  श्र॒द्धायाꣳ॑ समा॒ने निवि॑श्या॒ऽमृतꣳ॑ हु॒तं ।  स॒मान॒ मन्ने॑नाप्यायस्व । \textbf{ 91} \newline
                  \newline
                                                         \textbf{} \newline \newline
                                \textbf{ T.A.6.71.1} \newline
                  अङ्गुष्ठमात्रः पुरुषोऽङ्गुष्ठञ्च॑ समा॒श्रितः ।  ईशः सर्वस्य जगतः प्रभुः प्रीणाति॑ विश्व॒भुक् । \textbf{ 92} \newline
                  \newline
                                                         \textbf{} \newline \newline
                                \textbf{ T.A.6.72.1} \newline
                  वाङ्म॑ आ॒सन्न् । न॒सोः प्रा॒णः । अ॒क्ष्यो-श्चक्षुः॑ । कर्ण॑योः॒ श्रोत्रं᳚ । बा॒हु॒वोर् बलं᳚ । ऊ॒रु॒वो रोजः॑ । अरि॑ष्टा॒ विश्वा॒न्यङ्गा॑नि त॒नूः ।  त॒नुवा॑ मे स॒ह नम॑स्ते अस्तु॒ मा मा॑ हिꣳसीः । \textbf{ 93} \newline
                  \newline
                                                         \textbf{} \newline \newline
                                \textbf{ T.A.6.73.1} \newline
                  वयः॑ सुप॒र्णा उप॑सेदु॒रिन्द्रं॑ प्रि॒य मे॑धा॒ ऋष॑यो॒ नाध॑मानाः । अप॑द्ध्वा॒न्त मू᳚र्णु॒हि पू॒र्धि चक्षु॑र्-मुमु॒ग्ध्य॑स्मान् नि॒धये॑ ऽवब॒द्धान् । \textbf{ 94} \newline
                  \newline
                                                         \textbf{} \newline \newline
                                \textbf{ T.A.6.74.1} \newline
                  प्राणानां ग्रन्थिरसि रुद्रो मा॑ विशा॒न्तकः ।  तेनान्नेना᳚-प्याय॒स्व । \textbf{ 95} \newline
                  \newline
                                                         \textbf{} \newline \newline
                                \textbf{ T.A.6.75.1} \newline
                  नमो रुद्राय विष्णवे मृत्यु॑र्मे पा॒हि । \textbf{ 96} \newline
                  \newline
                                                         \textbf{} \newline \newline
                                \textbf{ T.A.6.76.1} \newline
                  त्वम॑ग्ने॒ द्युभि॒-स्त्वमा॑शु-शु॒क्षणि॒-स्त्वम॒द्भ्य-स्त्वमश्म॑न॒स्परि॑ । त्वं ॅवने᳚भ्य॒-स्त्वमोष॑धीभ्य॒-स्त्वं नृ॒णां नृ॑पते जायसे॒ शुचिः॑ । \textbf{ 97} \newline
                  \newline
                                                         \textbf{} \newline \newline
                                \textbf{ T.A.6.77.1} \newline
                  शि॒वेन॑ मे॒ सन्ति॑ष्ठस्व स्यो॒नेन॑ मे॒ सन्ति॑ष्ठस्व सुभू॒तेन॑ मे॒ सन्ति॑ष्ठस्व ब्रह्मवर्च॒सेन॑ मे॒ सन्ति॑ष्ठस्व य॒ज्ञ्स्यर्द्धि॒ मनु॒ सन्ति॑ष्ठ॒ स्वोप॑ ते यज्ञ्॒ नम॒ उप॑ ते॒ नम॒ उप॑ ते॒ नम॑: । \textbf{ 98} \newline
                  \newline
                                                         \textbf{} \newline \newline
                                \textbf{ T.A.6.78.1} \newline
                  स॒त्यं परं॒ परꣳ॑ स॒त्यꣳ स॒त्येन॒ न सु॑व॒र्गा-ल्लो॒काच्च्य॑वन्ते क॒दाच॒न स॒ताꣳ हि स॒त्यं तस्मा᳚थ् स॒त्ये र॑मन्ते॒, \textbf{ 1-11} \newline
                  \newline
                                                         \textbf{} \newline \newline
                                \textbf{ T.A.6.79.1} \newline
                  प्रा॒जा॒प॒त्यो हारु॑णिः सुप॒र्णेयः॑ प्र॒जाप॑तिं पि॒तर॒-मुप॑ससार॒ किं भ॑गव॒न्तः प॑र॒मं ॅव॑द॒न्तीति॒ तस्मै॒ प्रो॑वाच, \textbf{ 1-20} \newline
                  \newline
                                                         \textbf{} \newline \newline
                                \textbf{ T.A.6.80.1} \newline
                  तस्यै॒वं ॅवि॒दुषो॑ य॒ज्ञ्स्या॒त्मा यज॑मानः-श्र॒ध्दापत्नी॒ शरी॑र-मि॒द्ध्ममुरो॒ वेदि॒र्-लोमा॑नि ब॒र॒.हिर्-वे॒दः-शिखा॒ हृद॑यं॒ ॅयूपः॒ काम॒ आज्यं॑ म॒न्युः प॒शु-स्तपो॒ऽग्निर्-दमः॑ शमयि॒ता दक्षि॑णा॒-वाग्घोता᳚ प्रा॒ण उ॑द्ग॒ता चक्षु॑रध्व॒र्युर्-मनो॒ ब्रह्मा॒ श्रोत्र॑म॒ग्नी-  ध्याव॒ध्द्रिय॑ते॒ सा दी॒क्षा यदश्र्ना॑ति॒ तध्दवि॒र्-यत्पिब॑ति॒ तद॑स्य सोमपा॒नं ॅयद्रम॑ते॒ तदु॑प॒सदो॒ यथ् सं॒चर॑-त्युप॒विश॑-त्यु॒त्तिष्ठ॑ते च॒ सप्र॑व॒र्ग्यो॑ यन्मुखं॒ तदा॑हव॒नीयो॒ या व्याहृ॑ति-राहु॒तिर्-यद॑स्य वि॒ज्ञानं॒ तज्जु॒होति॒ यथ्सा॒यं प्रा॒तर॑त्ति॒ तथ्स॒मिधं॒ ॅयत्प्रा॒तर् म॒द्ध्यन्दि॑नꣳ सा॒यं च॒ तानि॒ सव॑नानि॒    ये अ॑होरा॒त्रे ते द॑र्.शपूर्णमा॒सौ ये᳚ऽर्द्धमा॒साश्च॒ मासा᳚श्च॒ ते चा॑तुर्मा॒स्यानि॒ य ऋ॒तव॒स्ते प॑शुब॒न्धा ये सं॑ॅवथ्स॒राश्च॑ परिवथ्स॒राश्च॒ तेऽह॑र्ग॒णाः स॑र्व वेद॒सं ॅवा ए॒तथ् स॒त्रं ॅयन्मर॑णं॒ तद॑व॒भृथ॑   ए॒तद्वै ज॑रामर्य-मग्निहो॒त्रꣳ स॒त्रं ॅय ए॒वं ॅवि॒द्वा-नु॑द॒गय॑ने प्र॒मीय॑ते दे॒वाना॑मे॒व म॑हि॒मानं॑ ग॒त्वाऽऽदि॒त्यस्य॒ सायु॑ज्यं गच्छ॒त्यथ॒ यो द॑क्षि॒णे प्र॒मीय॑ते पितृ॒णा-मे॒व म॑हि॒मानं॑ ग॒त्वा च॒न्द्रम॑सः॒ सायु॑ज्यꣳ सलो॒कता॑-माप्नोत्ये॒तौ वै सू᳚र्या चन्द्र॒मसौ᳚र्-महि॒मानौ᳚ ब्राह्म॒णो वि॒द्वा-न॒भिज॑यति॒ तस्मा᳚द् ब्र॒ह्मणो॑ महि॒मान॑माप्नोति॒ तस्मा᳚द् ब्र॒ह्मणो॑ महि॒मान॑-मित्युप॒निषत् । \textbf{ 101} \newline
                  \newline
                                                         \textbf{} \newline \newline
\textbf{Prapaataka Korvai with starting Padams of 1 to 80 Anuvaakams :-} \newline
(अम्भ॒स्यैक॑पञ्चा॒शच्छ॒तं - जा॒तवे॑दसे॒ चतु॑र्दश॒ - भूरन्नं॒ - भूर॒ग्नये॒ - भूर॒ग्नये॒ चैक॑मेकं - पाहि - पा॒हि च॒त्वारि॑ चत्वारि॒ - यश्छन्द॑सां॒ द्वे - नमो॒ ब्रह्म॑णे - ऋ॒तं तपो॒ - यथा॑ वृ॒क्षस्यैक॑ मेक - म॒णोरणी॑याꣳ॒॒ श्चतु॑स्त्रिꣳशथ् - सहस्र॒शी॑षꣳ॒॒ षट्विꣳ॑शति - रादि॒त्यो वा ए॒ष - आ॑दि॒त्यो वै तेज॒ एक॑मेकं॒ - निध॑नपतये॒ त्रयो॑विꣳशतिः - स॒द्योजा॒तं त्रीणि॑ - वामदे॒वायैक॑ - म॒घोरे᳚भ्य॒ - स्तत्पुरु॑षाय॒ द्वे द्वे॒ - ईशानो - नमो हिरण्यबाहव॒ एक॑मेक - मृ॒तꣳ स॒त्यं द्वे - सर्वो॒ वै च॒त्वारि॒ - कद्रु॒द्राय॒ त्रीणि॒ - यस्य॒ वै कङ्क॑ती - कृणु॒ष्व पाजो - ऽदि॑ति॒ - रापो॒ वा इ॒दꣳ सर्व॒ मेक॑मेक॒ - मापः॑ पुनन्तु च॒त्वा - र्यग्निश्च - सूर्यश्च नव॑ - न॒वोमिति॑ च॒त्वा - र्याया॑तु॒ पचौ - जो॑ऽसि॒ दशो॒ - त्तमे॑ च॒त्वारि॒ - घृणि॒स्त्रीणि॒ - ब्रह्म॑मेतु॒ मां यास्ते᳚ ब्रह्मह॒त्यां द्वाद॑श॒ - ब्रह्म॑ मे॒धया॒ऽद्या न॑ इ॒मं भ्रू॑षह॒त्यां - ब्रह्म॑ मे॒धवा᳚ ब्र॒ह्मा दे॒वाना॑मि॒दं ॅवी॑रह॒त्यामेका॒न्न विꣳ॑शति॒ रेका॒न्नविꣳ॑शतिर् - मे॒धा दे॒वी - मे॒धां म॒ इन्द्र॑श्च॒त्वारि॑ चत्वा॒र्या - मां᳚ मे॒धा द्वे - मयि॑ मे॒धा मेक॒- मपै॑तु॒ - परं॒ - ॅवातं॑ प्रा॒ण - म॑मुत्र॒भूया॒द् - द्धरिꣳ॒॒ - शल्कै॑र॒ग्निं - मा छि॑दो मृत्यो॒ - मा नो॑ म॒हान्तं॒ - मान॑स्तो॒के - प्रजा॑पते - स्वस्ति॒दा - त्र्य॑म्बकं॒ - ॅये ते॑ स॒हस्रं॒ द्वे द्वे - मृ॒त्यवे॒ स्वाहैकं॑ - दे॒वकृ॑त॒स्यैका॑दश॒ - यद्वो॑ देवाः॒ - कामोऽकार्.षी॒न् - मन्युरकार्.षी॒द् द्वे द्वे॒ - तिलाञ्जुहोमि गावः श्रियं प्र॑जाः पञ्च॒ - तिलाः कृण्षाश्चोर॑स्य॒ श्रीः प्रज्ञातु जातवे॑दः स॒प्त - प्राण वाक् त्वक् छिर उत्तिष्ठ पुरुष॑ पञ्च॒ - पृथिवी शब्द मनो वाग् व्यक्ताऽऽत्माऽन्तरात्मा परमात्मा मे᳚ क्षु॒धेऽन्नमय॒ पञ्च॑दशा॒ - ग्नये॒ स्वाहैक॑चत्वारिꣳ॒॒श - र्दो᳚ न्तद्ब्र॒ह्म नव॑ - श्र॒द्धायां᳚ प्रा॒णे निविष्ट॒ श्चतु॑र्विꣳशतिः - श्र॒द्धायां॒ दशा - ङ्गुष्ठ मात्रः पुरुषो द्वे - वाङ्म॑ आ॒सन्न॒ष्टौ - वयः॑ सुप॒र्षाः - प्राणानां ग्रन्थिरसि द्वे द्वे - नमो रुद्रायैकं॒ - त्वम॑ग्ने॒ द्युभिर्॒ द्वे - शि॒वेन॑ मे॒ सन्ति॑ष्ठस्व - स॒त्यं - प्रा॑जाप॒त्य - स्तस्यै॒व मेक॑ मेक॒ मशतिः) \newline

\textbf{korvai with starting padams of1, 11, 21 Series Of Dasinis :-} \newline
(अम्भ॑स्यपा॒रे - स्व॒स्ति नः॑ - पा॒हि नो॑ अग्न॒ एक॑या - ऽऽदि॒त्यो वा ए॒ष - ऋ॒तꣳ स॒त्य - मो मित्या - मां᳚ मे॒धा - मा न॑स्तो॒के - तिलाजुहोमि - श्र॒द्धायां᳚ प्रा॒णे निवि॑श्य॒ - तस्यै॒व मेकोत्त॑रश॒तम्) \newline

\textbf{first and last padam in TA, 6th Prapaatakam :-} \newline
एन्द् ऒf fइर्स्त्ळस्त् फदम् \newline 


कृष्ण यजुर्वेदीय तैत्तिरीय आरण्यके षष्ठः प्रपाठकः समाप्तः ॥ \newline
\pagebreak
\pagebreak
        


\end{document}
