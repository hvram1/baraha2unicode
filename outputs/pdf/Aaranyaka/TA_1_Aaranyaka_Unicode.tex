\documentclass[17pt]{extarticle}
\usepackage{babel}
\usepackage{fontspec}
\usepackage{polyglossia}
\usepackage{extsizes}



\setmainlanguage{sanskrit}
\setotherlanguages{english} %% or other languages
\setlength{\parindent}{0pt}
\pagestyle{myheadings}
\newfontfamily\devanagarifont[Script=Devanagari]{AdishilaVedic}


\newcommand{\VAR}[1]{}
\newcommand{\BLOCK}[1]{}




\begin{document}
\begin{titlepage}
    \begin{center}
 
\begin{sanskrit}
    { \Large
    ॐ नमः परमात्मने, श्री महागणपतये नमः
श्री गुरुभ्यो नमः, ह॒रिः॒ ॐ 
    }
    \\
    \vspace{2.5cm}
    \mbox{ \Huge
    कृष्ण यजुर्वेदीय तैत्तिरीय आरण्यके प्रथमः प्रपाठकः   }
\end{sanskrit}
\end{center}

\end{titlepage}
\tableofcontents

ॐ नमः परमात्मने, श्री महागणपतये नमः
श्री गुरुभ्यो नमः, ह॒रिः॒ ॐ \newline
1.1     प्रथमः प्रपाठकः- अरुणकेतुकचयनं \newline

\addcontentsline{toc}{section}{ 1.1     प्रथमः प्रपाठकः- अरुणकेतुकचयनं}
\markright{ 1.1     प्रथमः प्रपाठकः- अरुणकेतुकचयनं \hfill https://www.vedavms.in \hfill}
\section*{ 1.1     प्रथमः प्रपाठकः- अरुणकेतुकचयनं }
                                \textbf{ T.A.1.1.1} \newline
                  भ॒द्रं कर्णे॑भिः शृणु॒याम॑ देवाः ।  भ॒द्रं प॑श्येमा॒क्षभि॒र् यज॑त्राः ।  स्थि॒रैरङ्गै᳚स्तुष्टु॒वाꣳ स॑स्त॒नूभिः॑ । व्यशे॑म दे॒वहि॑तं॒ ॅयदायुः॑ ॥  स्व॒स्ति न॒ इन्द्रो॑ वृ॒द्धश्र॑वाः ।  स्व॒स्ति नः॑ पू॒षा वि॒श्ववे॑दाः ।  स्व॒स्ति न॒स्तार्क्ष्यो॒ अरि॑ष्टनेमिः ।  स्व॒स्ति नो॒ बृह॒स्पति॑र्-दधातु ॥  आप॑मापाम॒पः सर्वाः᳚ ।  अ॒स्मा-द॒स्मा-दि॒तोऽमुतः॑ \textbf{ 1} \newline
                  \newline
                                                                  \textbf{ T.A.1.1.2} \newline
                  अ॒ग्निर्वा॒युश्च॒ सूर्य॑श्च । स॒ह स॑ञ्च-स्क॒रर्द्धि॑या ॥ वा॒य्वश्वा॑ रश्मि॒पत॑यः । मरी᳚च्यात्मानो॒ अद्रु॑हः । दे॒वीर् भु॑वन॒ सूव॑रीः । पु॒त्र॒व॒त्त्वाय॑ मे सुत ॥ महानाम्नीर्-म॑हामा॒नाः । म॒ह॒सो म॑हसः॒ स्वः॑ ।  दे॒वीः प॑र्जन्य॒ सूव॑रीः । पु॒त्र॒व॒त्त्वाय॑ मे सुत । \textbf{ 2} \newline
                  \newline
                                                                  \textbf{ T.A.1.1.3} \newline
                  अ॒पाश्न्यु॑ष्णि-म॒पा रक्षः॑ । अ॒पाश्न्यु॑ष्णि-म॒पा रघं᳚ । अपा᳚घ्रा॒मप॑ चा॒वर्तिं᳚ । अप॑ दे॒वीरि॒तो हि॑त ॥ वज्रं॑ दे॒वीरजी॑ताꣳश्च ।  भुव॑नं देव॒सूव॑रीः । आ॒दि॒त्यानदि॑तिं दे॒वीं । योनि॑नोर्द्ध्व-मु॒दीष॑त ॥ शि॒वा नः॒ शन्त॑मा भवन्तु । दि॒व्या आप॒ ओष॑धयः( ) ।  सु॒मृ॒डी॒का सर॑स्वति । मा ते॒ व्यो॑म स॒न्दृशि॑ । \textbf{ 3} \newline
                  \newline
                                                        (अ॒मुतः॑ - सु॒ - तौष॑धयो॒ द्वे च॑ ) \textbf{1} \newline \newline
                                \textbf{ T.A.1.2.1} \newline
                  स्मृतिः॑ प्र॒त्यक्ष॑-मैति॒ह्यं᳚ । अनु॑मान-श्चतुष्ट॒यं । ए॒तैरादि॑त्य मण्डलं ।  सर्वै॑रेव॒ विधा᳚स्यते ॥ सूर्यो॒ मरी॑चि॒माद॑त्ते । सर्वस्मा᳚द् भुव॑नाद॒धि । तस्याः पाक वि॑शेषे॒ण । स्मृ॒तं का॑ल वि॒शेष॑णं ॥  न॒दीव॒ प्रभ॑वात् का॒चित् । अ॒क्षय्या᳚थ् स्यन्द॒ते य॑था \textbf{ 4} \newline
                  \newline
                                                                  \textbf{ T.A.1.2.2} \newline
                  तान्नद्योऽभि स॑माय॒न्ति । सो॒रुः सती॑ न नि॒वर्त॑ते ॥  ए॒वन्ना॒ना स॑मुत्था॒नाः । का॒लाः स॑म्ॅवथ्स॒रꣳ श्रि॑ताः । अणुशश्च म॑हश॒श्च । सर्वे॑ समव॒यन्त्रि॑ तं ॥ स तैः᳚ स॒र्वैः स॑मावि॒ष्टः । ऊ॒रुः स॑न्न नि॒वर्त॑ते । अधि सम्ॅवथ्स॑रं ॅवि॒द्यात् ।  तदेव॑ लक्ष॒णे । \textbf{ 5} \newline
                  \newline
                                                                  \textbf{ T.A.1.2.3} \newline
                  अणुभिश्च म॑हद्भि॒श्च । स॒मारू॑ढः प्र॒दृश्य॑ते । सम्ॅवथ्सरः प्र॑त्यक्षे॒ण ।  ना॒धिस॑त्त्वः प्र॒दृश्य॑ते ॥ प॒टरो॑ विक्लि॑धः पि॒ङ्गः । ए॒तद् व॑रुण॒ लक्ष॑णं । यत्रैत॑-दुप॒दृश्य॑ते । स॒हस्रं॑ तत्र॒ नीय॑ते ॥  एकꣳ हि शिरो ना॑ना मु॒खे । कृ॒थ्स्नं त॑दृत॒ लक्ष॑णं \textbf{ 6} \newline
                  \newline
                                                                  \textbf{ T.A.1.2.4} \newline
                  उभयतः सप्ते᳚न्द्रिया॒णि । ज॒ल्पितं॑ त्वेव॒ दिह्य॑ते ॥  शुक्लकृष्णे सम्ॅव॑थ्सर॒स्य । दक्षिण वाम॑योः पा॒र्श्वयोः । तस्यै॒षा भव॑ति ॥ शु॒क्रं ते॑ अ॒न्यद्य॑ज॒तं ते॑ अ॒न्यत् ।  विषु॑रूपे॒ अह॑नी॒ द्यौरि॑वासि । विश्वा॒ हि मा॒या अव॑सि स्वधावः । भ॒द्रा ते॑ पूषन्नि॒ह रा॒तिर॒स्त्विति॑ ॥  नात्र॒ भुव॑नं ( ) । न पू॒षा । न प॒शवः॑ ।  नादित्यः सम्ॅवथ्सर एव प्रत्यक्षेण प्रियत॑मं ॅवि॒द्यात् । एतद्वै सम्ॅवथ्सरस्य प्रियत॑मꣳ रू॒पं । योऽस्य महानर्थ उत्पथ्स्यमा॑नो भ॒वति । इदं पुण्यं कु॑रुष्वे॒ति । तमाहर॑णं द॒द्यात् । \textbf{ 7} \newline
                  \newline
                                                        (य॒था॒ - ल॒क्ष॒ण - ऋ॑तु॒लक्ष॑णं॒ - भुव॑नꣳ स॒प्त च॑) \textbf{2} \newline \newline
                                \textbf{ T.A.1.3.1} \newline
                  सा॒क॒ञ्जानाꣳ॑ स॒प्तथ॑माहु-रेक॒जं । षडु॑द्य॒मा ऋष॑यो देव॒जा इति॑ । तेषा॑मि॒ष्टानि॒ विहि॑तानि धाम॒शः । स्था॒त्रे रे॑जन्ते॒ विकृ॑तानि रूप॒शः ॥ कोनु॑ मर्या॒ अमि॑थितः । सखा॒ सखा॑यमब्रवीत् । जहा॑को अ॒स्मदी॑षते ॥ यस्ति॒त्याज॑ सखि॒विदꣳ॒॒ सखा॑यं ।  न तस्य॑ वा॒च्यपि॑ भा॒गो अ॑स्ति ।  यदीꣳ॑ शृ॒णोत्य॒लकꣳ॑ शृणोति \textbf{ 8} \newline
                  \newline
                                                                  \textbf{ T.A.1.3.2} \newline
                  न हि प्र॒वेद॑ सुकृ॒तस्य॒ पन्था॒मिति॑ ॥  ऋ॒तुर्. ऋ॑तुना नु॒द्यमा॑नः । विन॑नादा॒भिधा॑वः । षष्टिश्च त्रिꣳश॑का व॒ल्गाः ।  शु॒क्लकृ॑ष्णौ च॒ षाष्टि॑कौ ॥  सा॒रा॒ग॒व॒स्त्रैर्-ज॒रद॑क्षः ।  व॒स॒न्तो वसु॑भिः स॒ह । स॒म्ॅव॒थ्स॒रस्य॑ सवि॒तुः । प्रै॒ष॒कृत् प्र॑थ॒मः स्मृ॑तः ॥ अ॒मूना॒दय॑-तेत्य॒न्यान् \textbf{ 9} \newline
                  \newline
                                                                  \textbf{ T.A.1.3.3} \newline
                  अ॒मूꣳश्च॑ परि॒रक्ष॑तः । ए॒ता वा॒चः प्र॑युज्य॒न्ते । यत्रै त॑दुप॒दृश्य॑ते ॥ ए॒तदे॒व वि॑जानी॒यात् । प्र॒माणं॑ काल॒पर्य॑ये । वि॒शे॒ष॒णं तु॑ वक्ष्या॒मः । ऋ॒तूनां᳚ तन्नि॒बोध॑त ॥ शुक्लवासा॑ रुद्र॒गणः । ग्री॒ष्मेणा॑वर्त॒ते स॑ह ।  नि॒जह॑न् पृथि॑वीꣳ स॒र्वां \textbf{ 10} \newline
                  \newline
                                                                  \textbf{ T.A.1.3.4} \newline
                  ज्यो॒तिषा᳚ ऽप्रति॒ख्येन॑ सः ॥ वि॒श्व॒रू॒पाणि॑ वासाꣳ॒॒सि । आ॒दि॒त्यानां᳚ नि॒बोध॑त । सम्ॅवथ्सरीणं॑ कर्म॒फलं । वर्.षाभिर् द॑दताꣳ॒॒ सह ॥ अदुःखो॑ दुःख च॑क्षुरि॒व । तद्मा॑ पीत इव॒ दृश्य॑ते । शीतेना᳚ व्यथ॑यन्नि॒व । रु॒रुद॑क्ष इव॒ दृश्य॑ते ॥ ह्लादयते᳚ ज्वल॑तश्चै॒व ( ) ।  शा॒म्यत॑श्चास्य॒ चक्षु॑षी । यावै प्रजा भ्रꣳ॑श्य॒न्ते । सम्ॅवथ्सरात्ता भ्रꣳ॑श्य॒न्ते ॥ याः॒ प्रति॑तिष्ठ॒न्ति । संॅवथ्सरे ताः प्रति॑तिष्ठ॒न्ति । व॒॒र्॒.षाभ्य॑ इत्य॒र्थः । \textbf{ 11} \newline
                  \newline
                                                        (शृ॒णो॒ - त्य॒न्यान्थ् - स॒र्वा - मे॒व षट्च॑) \textbf{3} \newline \newline
                                \textbf{ T.A.1.4.1} \newline
                  अक्षि॑दुः॒खोत्थि॑तस्यै॒व । वि॒प्रस॑न्ने क॒नीनि॑के । आङ्क्ते चाद्ग॑णं ना॒स्ति । ऋ॒भूणां᳚ तन्नि॒बोध॑त ॥ क॒न॒का॒भानि॑ वासाꣳ॒॒सि । अ॒हता॑नि नि॒बोध॑त ।  अन्नमश्र्नीत॑ मृज्मी॒त । अ॒हं ॅवो॑ जीव॒नप्र॑दः ॥ ए॒ता वा॒चः प्र॑युज्य॒न्ते ।  श॒रद्य॑त्रोप॒ दृश्य॑ते \textbf{ 12} \newline
                  \newline
                                                                  \textbf{ T.A.1.4.2} \newline
                  अभिधून्वन्तो-ऽभिघ्न॑न्त इ॒व । वा॒तव॑न्तो म॒रुद्ग॑णाः ॥ अमुतो जेतुमिषुमु॑खमि॒व । सन्नद्धाः सह द॑दृशे॒ ह । अपद्ध्वस्तैर्-वस्तिव॑र्णैरि॒व । वि॒शि॒खासः॑ कप॒र्दिनः ॥ अक्रुद्धस्य योथ्स्य॑मान॒स्य । कृ॒द्धस्ये॑व॒ लोहि॑नी । हेमतश्चक्षु॑षी वि॒द्यात् । अ॒क्ष्णयोः᳚ क्षिप॒णोरि॑व । \textbf{ 13} \newline
                  \newline
                                                                  \textbf{ T.A.1.4.3} \newline
                  दुर्भिक्षं देव॑लोके॒षु । म॒नूना॑मुद॒कं गृ॑हे । ए॒ता वा॒चः प्र॑वद॒न्तीः । वै॒द्युतो॑ यान्ति॒ शैशि॑रीः ॥ ता अ॒ग्निः पव॑माना॒ अन्वै᳚क्षत । इ॒ह जी॑वि॒काम-प॑रिपश्यन्न् । तस्यै॒षा भव॑ति ॥ इ॒हेह वः॑ स्वत॒पसः । मरु॑तः॒ सूर्य॑त्वचः । शर्म॑ स॒प्रथा॒ आवृ॑णे ( ) । \textbf{ 14} \newline
                  \newline
                                                        (दृश्य॑त - इ॒वा - वृ॑णे) \textbf{4} \newline \newline
                                \textbf{ T.A.1.5.1} \newline
                  अति॑ ता॒म्राणि॑ वासाꣳ॒॒सि । अ॒ष्टिव॑ज्रि श॒तघ्नि॑ च ।  विश्वे देवा विप्र॑हर॒न्ति । अ॒ग्निजि॑ह्व अ॒सश्च॑त ॥ नैव देवो॑ न म॒र्त्यः । न राजा व॑रुणो॒ विभुः । नाग्निर् नेन्द्रो न प॑वमा॒नः । मा॒तृक्क॑च्च न॒ विद्य॑ते ॥ दि॒व्यस्यैका॒ धनु॑रार्त्निः ।  पृ॒थि॒व्यामप॑रा श्रि॒ता \textbf{ 15} \newline
                  \newline
                                                                  \textbf{ T.A.1.5.2} \newline
                  तस्येन्द्रो वम्रि॑रूपे॒ण । ध॒नुर्ज्या॑-मछि॒नथ्स्व॑यं ॥ तदि॑न्द्र॒धनु॑रित्य॒ज्यं । अ॒भ्रव॑र्णेषु॒ चक्ष॑ते । एतदेव शम्ॅयोर्-बार्.ह॑स्पत्य॒स्य । ए॒तद् रु॑द्रस्य॒ धनुः ॥ रु॒द्रस्य॑ त्वेव॒ धनु॑रार्त्निः । शिर॒ उत्पि॑पेष ।  स प्र॑व॒र्ग्यो॑ऽभवत् । तस्मा॒द् यः सप्र॑व॒र्ग्येण॑ य॒ज्ञेन॒ यज॑ते ( ) ।  रु॒द्रस्य॒ स शिरः॒ प्रति॑दधाति । नैनꣳ॑ रु॒द्र आरु॑को भवति । य ए॒वं ॅवेद॑ । \textbf{ 16} \newline
                  \newline
                                                        (श्रि॒ता - यज॑ते॒ त्रीणि॑ च) \textbf{5} \newline \newline
                                \textbf{ T.A.1.6.1} \newline
                  अ॒त्यू॒र्द्ध्वा॒क्षोऽति॑रश्चात् । शिशि॑रः प्र॒दृश्य॑ते ।  नैव रूपं न॑ वासाꣳ॒॒सि । न चक्षुः॑ प्रति॒दृश्य॑ते ॥ अ॒न्योन्यं॒ तु न॑ हिꣳस्रा॒तः । स॒त स्त॑द् देव॒लक्ष॑णं । लोहितोऽक्ष्णि शा॑रशी॒र्ष्णिः ।  सू॒र्यस्यो॑दय॒नं प्र॑ति ॥ त्वं करोषि॑ न्यञ्ज॒लिकां । त्वं॒ करो॑षि नि॒जानु॑कां \textbf{ 17} \newline
                  \newline
                                                                  \textbf{ T.A.1.6.2} \newline
                  निजानुकामे᳚ न्यञ्ज॒लिका । अमी वाच-मुपास॑तामि॒ति ॥  तस्मै सर्व ऋतवो॑ नम॒न्ते । मर्यादा करत्वात् प्र॑पुरो॒धां । ब्राह्मण॑ आप्नो॒ति । य ए॑वं ॅवे॒द ।  स खलु सम्ॅवथ्सर एतैः सेनानी॑भिः स॒ह ।  इन्द्राय सर्वान्-कामान॑भिव॒हति । स द्र॒॒फ्सः ।  तस्यै॒षा भव॑ति । \textbf{ 18} \newline
                  \newline
                                                                  \textbf{ T.A.1.6.3} \newline
                  अव॑ द्र॒फ्सो अꣳ॑श॒मती॑मतिष्ठत् । इ॒या॒नः कृ॒ष्णो द॒शभिः॑ स॒हस्रैः᳚ ।  आव॒र्त-मिन्द्रः॒ शच्या॒ धम॑न्तं । उपस्नुहि तं नृमणा-मथ॑द्रामि॒ति ॥ एतयै वेन्द्रः सला वृ॑क्या स॒ह । असुरान् प॑रिवृ॒श्चति । पृथि॑व्यꣳ॒॒ शुम॑ती । ताम॒न्व-व॑स्थितः सम्ॅवथ्स॒रो दि॒वञ्च॑ ।  नैवं ॅविदुषा-ऽऽचार्या᳚न् तेवा॒सिनौ । अन्योन्यस्मै᳚ द्रुह्या॒तां ( ) । यो द्रु॒ह्यति । भ्रश्यते स्व॑र्गाल् लो॒कात् । इत्यतु म॑ण्डला॒नि । सूर्य मण्डला᳚ न्याख्या॒यिकाः । अत ऊर्द्ध्वꣳस॑निर्व॒चनाः । \textbf{ 19} \newline
                  \newline
                                                        (नि॒जानु॑कां॒ - भव॑ति - द्रुह्या॒तां पञ्च॑ च) \textbf{6} \newline \newline
                                \textbf{ T.A.1.7.1} \newline
                  आरोगो भ्राजः पटरः॑ पत॒ङ्गः । स्वर्णरो ज्योतिषीमान्॑. विभा॒सः । ते अस्मै सर्वे दिवमा॑तप॒न्ति । ऊर्जं दुहाना अनपस्फुर॑न्त इ॒ति ॥ कश्य॑पोऽष्ट॒मः । स महामेरुं न॑ जहा॒ति । तस्यै॒षा भव॑ति ॥ यत्ते॒ शिल्पं॑ कश्यप रोच॒नाव॑त् । इ॒न्द्रि॒याव॑त् पुष्क॒लं चि॒त्रभा॑नु ।  यस्मि॒न् थ्सूर्या॒ अर्पि॑ताः स॒प्त सा॒कं \textbf{ 20} \newline
                  \newline
                                                                  \textbf{ T.A.1.7.2} \newline
                  तस्मिन् राजान-मधिविश्रये॑ममि॒ति ॥ ते अस्मै सर्वे कश्यपा-ज्ज्योति॑र् -लभ॒न्ते । तान्थ्सोमः कश्यपादधि॑ निर्द्ध॒मति । भ्रस्ता कर्म कृ॑दिवै॒वं ॥ प्राणो जीवानीन्द्रिय॑ जीवा॒नि । सप्त शीर्.ष॑ण्याः प्रा॒णाः ।  सूर्या इ॑त्याचा॒र्याः ॥ अपश्यमह मेतान्थ् सप्त सू᳚र्यानि॒ति । पञ्चकर्णो॑ वाथ्स्या॒यनः । सप्तकर्ण॑श्च प्ला॒क्षिः \textbf{ 21} \newline
                  \newline
                                                                  \textbf{ T.A.1.7.3} \newline
                  आनुश्रविक एव नौ कश्य॑प इ॒ति । उभौ॑ वेद॒यिते ।  न हि शेकुमिव महामे॑रुं ग॒न्तुं ॥  अपश्यमहमेतथ् सूर्यमण्डलं परिव॑र्तमा॒नं ।  गा॒र्ग्यः प्रा॑णत्रा॒तः । गच्छन्त म॑हामे॒रुं । एक॑ञ्चाज॒हतं ॥ भ्राजपटर पत॑ङ्गा नि॒हने । तिष्ठन्ना॑तप॒न्ति । तस्मा॑दि॒ह तप्त्रि॑ तपाः \textbf{ 22} \newline
                  \newline
                                                                  \textbf{ T.A.1.7.4} \newline
                  अ॒मुत्रे॒तरे । तस्मा॑दि॒हा तप्त्रि॑ तपाः ॥ तेषा॑मेषा॒ भव॑ति ॥ स॒प्त सूर्या॒ दिव॒-मनु॒ प्रवि॑ष्टाः । तान॒न्वेति॑ प॒थिभि॑र् दक्षि॒णावान्॑ । ते अस्मै सर्वे घृतमा॑तप॒न्ति । ऊर्जं दुहाना अनपस्फुर॑न्त इ॒ति ॥  सप्तर्त्विजः सूर्या इ॑त्याचा॒र्याः ॥ तेषा॑मेषा॒ भव॑ति ॥  स॒प्त दिशो॒ नाना॑ सूर्याः \textbf{ 23} \newline
                  \newline
                                                                  \textbf{ T.A.1.7.5} \newline
                  स॒प्त होता॑र ऋ॒त्विजः॑ । देवा आदित्या॑ ये स॒प्त ।  तेभिः सोमाभी रक्ष॑ण इ॒ति ॥ तद॑प्याम्ना॒यः । दिग्भ्राज ऋतू᳚न् करो॒ति ॥ एत॑यैवा॒वृता ऽऽसहस्रसूर्यताया इति वै॑शम्पा॒यनः ॥ तस्यै॒षा भव॑ति ॥  यद्द्याव॑ इन्द्र ते श॒तꣳ श॒तं भूमीः᳚ । उ॒त स्युः ।  न त्वा॑ वज्रिन्थ् स॒हस्रꣳ॒॒ सूर्याः᳚ ( ) \textbf{ 24} \newline
                  \newline
                                                                  \textbf{ T.A.1.7.6} \newline
                  अनु न जातमष्ट रोद॑सी इ॒ति ॥ नाना लिङ्गत्वा-दृतूनां नाना॑ सूर्य॒त्वं ॥ अष्टौ तु व्यवसि॑ता इ॒ति ॥ सूर्यमण्डला-न्यष्टा॑त ऊ॒र्द्ध्वं ॥  तेषा॑मेषा॒ भव॑ति ॥ चि॒त्रं दे॒वाना॒-मुद॑गा॒दनी॑कं । चक्षु॑र् मि॒त्रस्य॒ वरु॑णस्या॒ग्नेः । आऽप्रा॒ द्यावा॑ पृथि॒वी अ॒न्तरि॑क्षं ।  सूर्य आत्मा जगतस्तस्थु॑षश्चे॒ति । \textbf{ 25} \newline
                  \newline
                                                        (सा॒कं - प्ला॒क्षि - स्तप्त्रि॑तपा॒ - नाना॑सूर्याः॒ - सूर्या॒ - +नव॑ च) \textbf{7} \newline \newline
                                \textbf{ T.A.1.8.1} \newline
                  क्वेदमभ्रं॑ निवि॒शते । क्वायꣳ॑ सम्ॅवथ्स॒रो मि॑थः । क्वाहः क्वेयन् दे॑व रा॒त्री । क्व मासा ऋ॑तवः॒ श्रिताः ॥ अर्द्धमासा॑ मुहू॒र्ताः । निमेषास्तु॑टिभिः॒ (निमेषास्त्र॑टिभिः॒) सह । क्वेमा आपो नि॑विश॒न्ते । य॒दीतो॑ यान्ति॒ संप्र॑ति ॥ काला अफ्सु नि॑विश॒न्ते । आ॒पः सूर्ये॑ स॒माहि॑ताः \textbf{ 26} \newline
                  \newline
                                                                  \textbf{ T.A.1.8.2} \newline
                  अभ्रा᳚ण्य॒पः प्र॑पद्य॒न्ते । वि॒द्युथ्सूर्ये॑ स॒माहि॑ता ॥ अनवर्णे इ॑मे भू॒मी । इ॒यञ्चा॑सौ च॒ रोद॑सी ॥ किꣳ स्विदत्रान्त॑रा भू॒तं । ये॒नेमे वि॑धृते॒ उभे ।  वि॒ष्णुना॑ विधृ॑ते भू॒मी । इ॒ति व॑थ्सस्य॒ वेद॑ना ॥  इरा॑वती धेनु॒मती॒ हि भू॒तं । सू॒य॒व॒सिनी॒ मनु॑षे दश॒स्ये᳚ \textbf{ 27} \newline
                  \newline
                                                                  \textbf{ T.A.1.8.3} \newline
                  व्य॑ष्टभ्ना॒-द्रोद॑सी॒ विष्ण॑वे॒ते । दा॒धर्थ॑ पृथि॒वी-म॒भितो॑ म॒यूखैः᳚ ॥  किन्त-द्विष्णो र्बल॑मा॒हुः । का॒ दीप्तिः॑ किं प॒राय॑णं ।  एको॑ य॒द्धा-र॑य द्दे॒वः । रे॒जती॑ रोद॒सी उ॑भे ॥ वाताद्विष्णोर् ब॑ल मा॒हुः । अ॒क्षरा᳚द् दीप्ति॒ रुच्य॑ते । त्रि॒पदा॒द्धार॑यद् दे॒वः । यद्विष्णो॑रेक॒-मुत्त॑मं । \textbf{ 28} \newline
                  \newline
                                                                  \textbf{ T.A.1.8.4} \newline
                  अ॒ग्नयो॑ वाय॑वश्चै॒व । ए॒तद॑स्य प॒राय॑णं ॥ पृच्छामि त्वा प॑रं मृ॒त्युं । अ॒वमं॑ मद्ध्य॒मञ्च॑तुं । लो॒कञ्च॒ पुण्य॑पापा॒नां ।  ए॒तत् पृ॑च्छामि॒ संप्र॑ति ॥ अ॒मुमा॑हुः प॑रं मृ॒त्युं । प॒वमा॑नं तु॒ मद्ध्य॑मं । अ॒ग्निरे॒वाव॑मो मृ॒त्युः । च॒न्द्रमा᳚-श्चतु॒रुच्य॑ते । \textbf{ 29} \newline
                  \newline
                                                                  \textbf{ T.A.1.8.5} \newline
                  अ॒ना॒भो॒गाः प॑रं मृ॒त्युं । पा॒पाः स॑म्ॅयन्ति॒ सर्व॑दा । आभोगास्त्वेव॑ सम्ॅय॒न्ति । य॒त्र पु॑ण्यकृ॒तो ज॑नाः ॥ ततो॑ म॒द्ध्यम॑माय॒न्ति । च॒तुम॑ग्निञ्च॒ संप्र॑ति ॥ पृच्छामि त्वा॑ पाप॒कृतः । य॒त्र या॑तय॒ते य॑मः । त्वन्नस्तद् -ब्रह्म॑न् प्रब्रू॒हि । य॒दि वे᳚त्थाऽस॒तो गृ॑हान् । \textbf{ 30} \newline
                  \newline
                                                                  \textbf{ T.A.1.8.6} \newline
                  क॒श्यपा॑ दुदि॑ताः सू॒र्याः । पा॒पान्नि॑र्घ्नन्ति॒ सर्व॑दा ।  रोदस्योरन्त॑र् देशे॒षु । तत्र न्यस्यन्ते॑ वास॒वैः ॥  ते ऽशरीराः प्र॑पद्य॒न्ते ।  य॒था ऽपु॑ण्यस्य॒ कर्म॑णः । अपा᳚ण्य॒पाद॑ केशा॒सः । त॒त्र ते॑ऽयोनि॒जा ज॑नाः ॥ मृत्वा पुनर्मृत्यु-मा॑पद्य॒न्ते । अ॒द्यमा॑नाः स्व॒कर्म॑भिः \textbf{ 31} \newline
                  \newline
                                                                  \textbf{ T.A.1.8.7} \newline
                  आशातिकाः क्रिम॑य इ॒व । ततः पूयन्ते॑ वास॒वैः ॥  अपै॑तं मृ॒त्युं ज॑यति । य ए॒वं ॅवेद॑ । स खल्वैवं॑ ॅविद्ब्रा॒ह्मणः । दी॒र्घश्रु॑त्तमो॒ भव॑ति । कश्य॑प॒स्याति॑थिः॒ सिद्धग॑मनः॒ सिद्धाग॑मनः ॥  तस्यै॒षा भव॑ति ॥ आयस्मिन᳚-थ्स॒प्त वा॑स॒वाः ।  रोह॑न्ति पू॒र्व्या॑ रुहः॑ \textbf{ 32} \newline
                  \newline
                                                                  \textbf{ T.A.1.8.8} \newline
                  ऋषि॑र्.ह दीर्घ॒श्रुत्त॑मः । इन्द्रस्य घर्मो अति॑थिरि॒ति ॥  कश्यपः पश्य॑को भ॒वति । यथ्सर्वं परिपश्यती॑ति सौ॒क्ष्म्यात् ॥ अथाग्ने॑रष्टपु॑रुष॒स्य । तस्यै॒षा भव॑ति ॥ अग्ने॒ नय॑ सु॒पथा॑ रा॒ये अ॒स्मान् ।  विश्वा॑नि देव व॒युना॑नि वि॒द्वान् । यु॒यो॒द्ध्य॑स्म-ज्जु॑हुरा॒णमेनः॑ । भूयिष्ठान्ते नम उक्तिं ॅवि॑धेमे॒ति ( ) । \textbf{ 33} \newline
                  \newline
                                                        (स॒माहि॑ता - दश॒स्ये॑ - उत्त॑म॒-मुच्य॑ते-गृहान्-थ्स्व॒कर्म॑भिः-पू॒र्व्या॑ रुह॑-इ॒ति) \textbf{8} \newline \newline
                                \textbf{ T.A.1.9.1} \newline
                  अग्निश्च जात॑वेदा॒श्च । सहोजा अ॑जिरा॒प्रभुः । वैश्वानरो न॑र्यापा॒श्च । प॒ङ्क्तिरा॑धाश्च॒ सप्त॑मः ।  विसर्पे वाष्ट॑मोऽग्नी॒नां । एतेऽष्टौ वसवः क्षि॑ता इ॒ति ॥ यथर्त्वे-वाग्ने-रर्चिर्वर्ण॑ विशे॒षाः ।  नीलार्चिश्च पीतका᳚र्चिश्चे॒ति ॥ अथ वायो-रेकादश-पुरुषस्यैकादश॑स्त्रीक॒स्य ॥  प्रभ्राजमाना व्य॑वदा॒ताः \textbf{ 34} \newline
                  \newline
                                                                  \textbf{ T.A.1.9.2} \newline
                  याश्च वासु॑कि वै॒द्युताः । रजताः परु॑षाः श्या॒माः ।  कपिला अ॑तिलो॒हिताः । ऊर्द्ध्वा अवप॑तन्ता॒श्च । वैद्युत इ॑त्येका॒दश ॥ नैनं ॅवैद्युतो॑ हिन॒स्ति । य ए॑वं ॅवे॒द ॥ स होवाच व्यासः पा॑राश॒र्यः ।  विद्युद्वधमेवाहं मृत्युमै᳚च्छमि॒ति ॥  न त्वका॑मꣳ ह॒न्ति \textbf{ 35} \newline
                  \newline
                                                                  \textbf{ T.A.1.9.3} \newline
                  य ए॑वं ॅवे॒द ॥ अथ ग॑न्धर्व॒गणाः । स्वान॒ भ्राट् । अङ्घा॑रि॒र् बंभा॑रिः । हस्तः॒ सुह॑स्तः । कृशा॑नुर् वि॒श्वाव॑सुः । मूर्द्धन्वान्थ्-सू᳚र्यव॒र्चाः ।  कृतिरित्येकादश ग॑न्धर्व॒गणाः ॥ देवाश्च म॑हादे॒वाः । रश्मयश्च देवा॑ गर॒गिरः । \textbf{ 36} \newline
                  \newline
                                                                  \textbf{ T.A.1.9.4} \newline
                  नैनं गरो॑ हिन॒स्ति । य ए॑वं ॅवे॒द ॥ गौ॒री मि॑माय सलि॒लानि॒ तक्ष॑ती । एक॑पदी द्वि॒पदी॒ सा चतु॑ष्पदी । अ॒ष्टाप॑दी॒ नव॑पदी बभू॒वुषी᳚ । सहस्राक्षरा परमे व्यो॑मन्नि॒ति ॥ वाचो॑ विशे॒षणं ॥  अथ निगद॑व्याख्या॒ताः । ताननुक्र॑मिष्या॒मः ॥  व॒राहवः॑-स्वत॒पसः \textbf{ 37} \newline
                  \newline
                                                                  \textbf{ T.A.1.9.5} \newline
                  वि॒द्युन् म॑हसो॒ धूप॑यः । श्वापयो गृहमेधा᳚श्चेत्ये॒ते । ये॒ चेमेऽशि॑मिवि॒द्विषः ॥ पर्जन्याः सप्त पृथिवीमभिव॑र्.ष॒न्ति ।  वृष्टि॑भिरि॒ति । एतयैव विभक्ति वि॑परी॒ताः । स॒प्तभि॒र्वातै॑ रुदी॒रिताः । अमूं ॅलोका-नभिव॑र्.ष॒न्ति । तेषा॑मेषा॒ भव॑ति ॥ स॒मा॒न-मे॒तदुद॑कं \textbf{ 38} \newline
                  \newline
                                                                  \textbf{ T.A.1.9.6} \newline
                  उ॒च्चैत्य॑व॒ चाह॑भिः । भूमिं॑ प॒र्जन्या॒ जिन्व॑न्ति ।  दिवं जिन्वन्-त्यग्न॑य इ॒ति ॥ यदक्ष॑रं भू॒तकृ॑तं ।  विश्वे॑ देवा उ॒पास॑ते । म॒हर्.षि॑मस्य गो॒प्तारं᳚ । ज॒मद॑ग्नि॒-मकु॑र्वत ॥ ज॒मद॑ग्नि॒-राप्या॑यते । छन्दो॑भि-श्चतुरुत्त॒रैः । राज्ञ्ः॒ सोम॑स्य तृ॒प्तासः॑ \textbf{ 39} \newline
                  \newline
                                                                  \textbf{ T.A.1.9.7} \newline
                  ब्रह्म॑णा वी॒र्या॑वता । शि॒वा नः॑ प्र॒दिशो॒ दिशः॑ ॥ तच्छ॒ॅय्योरा वृ॑णीमहे । गा॒तुं ॅय॒ज्ञाय॑ । गा॒तुं ॅय॒ज्ञ्प॑तये । दैवी᳚ स्व॒स्तिर॑स्तु नः ।  स्व॒स्तिर् मानु॑षेभ्यः । ऊ॒र्द्ध्वं जि॑गातु भेष॒जं ।  शन्नो॑ अस्तु द्वि॒पदे᳚ । शञ्चतु॑ष्पदे ( ) ॥  सोमपा(3) असोमपा(3) इति निगद॑व्याख्या॒ताः । \textbf{ 40} \newline
                  \newline
                                                        (व्य॒व॒दा॒ता - ह॒न्ति - ग॑र॒गिर - स्त॒पस - उद॑कं - तृप्तास॒ - श्वतु॑ष्पद॒ एकं॑ च) \textbf{9} \newline \newline
                                \textbf{ T.A.1.10.1} \newline
                  स॒ह॒स्र॒वृदि॑यं भू॒मिः । प॒रं ॅव्यो॑म स॒हस्र॑वृत् । अ॒श्विना॑ भुज्यू॑ नास॒त्या । वि॒श्वस्य॑ जग॒तस्प॑ती ॥ जाया भूमिः प॑तिर्व्यो॒म । मि॒थुन॑न्ता अ॒तुर्य॑थुः । पुत्रो बृहस्प॑ती रु॒द्रः । स॒रमा॑ इति॑ स्त्रीपु॒मं ॥ शु॒क्रं ॅवा॑म॒न्यद्य॑ज॒तं ॅवा॑म॒न्यत् । विषु॑रूपे॒ अह॑नी॒ द्यौरि॑व स्थः \textbf{ 41} \newline
                  \newline
                                                                  \textbf{ T.A.1.10.2} \newline
                  विश्वा॒ हि मा॒या अव॑थः स्वधावन्तौ । भ॒द्रा वां᳚ पूषणावि॒ह रा॒तिर॑स्तु ॥ वासा᳚त्यौ चि॒त्रौ जग॑तो नि॒धानौ᳚ । द्यावा॑भूमी च॒रथः॑ सꣳ॒॒ सखा॑यौ ।  ताव॒श्विना॑ रा॒सभा᳚श्वा॒ हवं॑ मे । शु॒भ॒स्प॒ती॒ आ॒गतꣳ॑ सू॒र्यया॑ स॒ह ॥ त्युग्रो॑ ह भु॒ज्यु-म॑श्विनोद मे॒घे । र॒यिन्न कश्चि॑न् ममृ॒वाॅ(2) अवा॑हाः । तमू॑हथुर्-नौ॒भिरा᳚त्म॒न्-वती॑भिः ।  अ॒न्त॒रि॒क्ष॒ प्रुड्भि॒र-पो॑दकाभिः । \textbf{ 42} \newline
                  \newline
                                                                  \textbf{ T.A.1.10.3} \newline
                  ति॒स्रः क्षप॒स्त्रिरहा॑ ऽति॒व्रज॑द्भिः । नास॑त्या भु॒ज्युमू॑हथुः पत॒ङ्गैः । स॒मु॒द्रस्य॒ धन्व॑न्ना॒र्द्रस्य॑ पा॒रे । त्रि॒भी रथैः᳚ श॒तप॑द्भिः॒ षड॑श्वैः ॥ स॒वि॒तारं॒ ॅवित॑न्वन्तं । अनु॑बद्ध्नाति शांब॒रः । आपपूर्.षं-ब॑रश्चै॒व । स॒विता॑ ऽरेप॒सो॑ ऽभवत् ॥ त्यꣳ सुतृप्तं ॅवि॑दित्वै॒व । ब॒हुसो॑म गि॒रं ॅव॑शी \textbf{ 43} \newline
                  \newline
                                                                  \textbf{ T.A.1.10.4} \newline
                  अन्वेति तुग्रो व॑क्रिया॒न्तं । आयसूयान्थ् सोम॑तृफ्सु॒षु ॥  स सङ्ग्राम-स्तमो᳚द्योऽत्यो॒तः । वाचो गाः पि॑पाति॒ तत् ।  स तद्गोभिः स्तवा᳚ ऽत्येत्य॒न्ये । र॒क्षसा॑ ऽनन्वि॒ताश्च॑ ये ॥ अ॒न्वेति॒ परि॑वृत्त्या॒ऽस्तः । ए॒वमे॒तौ स्थो॑ अश्विना । ते ए॒ते द्युः॑ पृथि॒व्योः । अह॑रह॒र्-गर्भ॑न्दधाथे । \textbf{ 44} \newline
                  \newline
                                                                  \textbf{ T.A.1.10.5} \newline
                  तयो॑ रे॒तौ व॒थ्सा व॑होरा॒त्रे । पृ॒थि॒व्या अहः॑ । दि॒वो रात्रिः॑ । ता अवि॑सृष्टौ । दंप॑ती ए॒व भ॑वतः ॥ तयो॑ रे॒तौ व॒थ्सौ । अ॒ग्निश्चा॑-दि॒त्यश्च॑ । रा॒त्रेर्व॒थ्सः । श्वे॒त आ॑दि॒त्यः । अह्नो॒ऽग्निः \textbf{ 45} \newline
                  \newline
                                                                  \textbf{ T.A.1.10.6} \newline
                  ता॒म्रो अ॑रु॒णः । ता अवि॑सृष्टौ । दम्प॑ती ए॒व भ॑वतः ॥ तयो॑ रे॒तौ व॒थ्सौ । वृ॒त्रश्च॑ वैद्यु॒तश्च॑ । अ॒ग्नेर्वृ॒त्रः । वै॒द्युत॑ आदि॒त्यस्य॑ ।  ता अवि॑सृष्टौ । दम्प॑ती ए॒व भ॑वतः ॥ तयो॑ रे॒तौ व॒थ्सौ \textbf{ 46} \newline
                  \newline
                                                                  \textbf{ T.A.1.10.7} \newline
                  उ॒ष्मा च॑ नीहा॒रश्च॑ । वृ॒त्रस्यो॒ष्मा । वै॒द्यु॒तस्य॑ नीहा॒रः । तौ तावे॒व प्रति॑पद्येते ॥ सेयꣳ रात्री॑ ग॒र्भिणी॑ पु॒त्रेण॒ सम्ॅव॑सति ।  तस्या॒ वा ए॒तदु॒ल्बणं᳚ । यद्रात्रौ॑ र॒श्मयः॑ । यथा॒ गोर्ग॒र्भिण्या॑ उ॒ल्बणं᳚ । ए॒वमे॒तस्या॑ उ॒ल्बणं᳚ ॥ प्रजयिष्णुः प्रजया च पशुभि॑श्च भ॒वति ( ) ।  य ए॑वं ॅवे॒द । एतमुद्यन्त-मपिय॑न्तञ्चे॒ति । आदित्यः पुण्य॑स्य व॒थ्सः । अथ पवि॑त्राङ्गि॒रसः । \textbf{ 47} \newline
                  \newline
                                                        (स्थो - ऽपो॑दकाभिर् - वशी - दधाथे - अ॒ग्नि - स्तयो॑ रे॒तौ व॒थ्सौ - भ॒वति च॒त्वारि॑ च) \textbf{10} \newline \newline
                                \textbf{ T.A.1.11.1} \newline
                  प॒वित्र॑वन्तः॒ परि॒वाज॒मास॑ते । पि॒तैषां᳚ प्र॒त्नो अ॒भिर॑क्षति व्र॒तं । म॒हस्स॑मु॒द्रं ॅवरु॑ण स्ति॒रोद॑धे । धीरा॑ इच्छेकु॒र्द्धरु॑णेष्वा॒रभं᳚ ॥  प॒वित्रं॑ ते॒ वित॑तं॒ ब्रह्म॑ण॒स्पते᳚ । प्रभु॒र्गात्रा॑णि॒ पर्ये॑षि वि॒श्वतः॑ ।  अत॑प्ततनू॒र् न तदा॒मो अ॑श्नुते । श्रृ॒तास॒ इद्वह॑न्-त॒स्तथ् समा॑शत ॥ “ब्र॒ह्मा दे॒वानां᳚{1}” ॥ अस॑तः स॒द्ये तत॑क्षुः \textbf{ 48} \newline
                  \newline
                                                                  \textbf{ T.A.1.11.2} \newline
                  ऋष॑यः स॒प्तात्रि॑श्च॒ यत् । सर्वेऽत्रयो अ॑गस्त्य॒श्च ।  नक्ष॑त्रैः॒ शङ्कृ॑तो ऽवसन्न् ॥ अथ॑ सवितुः॒ श्यावाश्व॒स्या ऽवर्त्ति॑कामस्य ॥ अ॒मी य ऋक्षा॒ निहि॑ता स उ॒च्चा । नक्तं॒ ददृ॑श्रे॒ कुह॑चि॒द्दिवे॑युः ।  अद॑ब्धानि॒ वरु॑णस्य व्र॒तानि॑ । वि॒चा॒कश॑-च्च॒न्द्रमा॒ नक्ष॑त्रमेति ॥  तथ् स॑वि॒तुर् वरे᳚ण्यं । भर्गो॑ दे॒वस्य॑ धीमहि \textbf{ 49} \newline
                  \newline
                                                                  \textbf{ T.A.1.11.3} \newline
                  धियो॒ यो नः॑ प्रचो॒दया᳚त् ॥ तथ्स॑वि॒तुर् वृ॑णीमहे ।  व॒यं दे॒वस्य॒ भोज॑नं । श्रेष्ठꣳ॑ सर्व॒धात॑मं । तुरं॒ भग॑स्य धीमहि ॥  अपा॑गूहत सविता॒ तृभीन्॑ । सर्वा᳚न् दि॒वो अन्ध॑सः ।  नक्त॑006छ्;॒तान्य॑भवन् दृ॒शे । अस्थ्य॒स्थ्ना संभ॑विष्यामः ॥ नाम॒ नामै॒व ना॒म मे᳚ \textbf{ 50} \newline
                  \newline
                                                                  \textbf{ T.A.1.11.4} \newline
                  नपुꣳस॑कं॒ पुमाꣳ॒॒स्त्र्य॑स्मि । स्थाव॑रोऽस्म्यथ॒ जङ्ग॑मः ।  य॒जेऽयक्षि॒ यष्टा॒हे च॑ ॥ मया॑ भू॒तान्य॑यक्षत । प॒शवो॑ मम॑ भूता॒नि । अनूबन्ध्यो ऽस्म्य॑हं ॅवि॒भुः ॥ स्त्रियः॑ स॒तीः । ता उ॑ मे पुꣳ॒॒स आ॑हुः ।  पश्य॑दक्ष॒ण्वान्न-विचे॑तद॒न्धः । क॒विर्यः पु॒त्रः स इ॒मा चि॑केत \textbf{ 51} \newline
                  \newline
                                                                  \textbf{ T.A.1.11.5} \newline
                  यस्ता वि॑जा॒नाथ् स॑वि॒तुः पि॒ताऽस॑त् ॥ अ॒न्धो मणिम॑विन्दत् । तम॑नङ्गुलि॒-राव॑यत् । अ॒ग्री॒वः प्रत्य॑मुञ्चत् । तमजि॑ह्व अ॒सश्च॑त ॥  ऊर्द्ध्वमूल-म॑वाक्छा॒खं । वृ॒क्षं ॅया॑ वेद॒ संप्र॑ति । न स जातु जनः॑ श्रद्द॒द्ध्यात् । मृ॒त्युर्मा॑ मार॒यादि॑तिः ॥ हसितꣳ रुदि॑तङ्गी॒तं \textbf{ 52} \newline
                  \newline
                                                                  \textbf{ T.A.1.11.6} \newline
                  वीणा॑ पण व॒लासि॑तं । मृ॒तञ्जी॒वञ्च॑ यत्कि॒ञ्चित् ।  अ॒ङ्गानि॑ स्नेव॒ विद्धि॑ तत् ॥ अतृ॑ष्यꣳ॒॒स्तृष्य॑ ध्यायत् । अ॒स्माज्जा॒ता मे॑ मिथू॒ चरन्न्॑ । पुत्रो निर्.ऋत्या॑ वैदे॒हः । अ॒चेता॑ यश्च॒ चेत॑नः ॥ स॒ तं मणिम॑विन्दत् । सो॑ऽनङ्गुलि॒राव॑यत् ।  सो॒ऽग्री॒वः प्रत्य॑मुन्चत् \textbf{ 53} \newline
                  \newline
                                                                  \textbf{ T.A.1.11.7} \newline
                  सोऽजि॑ह्वो अ॒सश्च॑त ॥ नैतमृषिं ॅविदित्वा नग॑रं प्र॒विशेत् । य॑दि प्र॒विशेत् । मि॒थौ चरि॑त्वा प्र॒विशेत् । तथ्संभव॑स्य व्र॒तं ॥ आ॒ तम॑ग्ने र॒थन्ति॑ष्ठ । एका᳚श्वमेक॒ योज॑नं । एकचक्र॑-मेक॒धुरं ।  वा॒त ध्रा॑जि ग॒तिं ॅवि॑भो ॥ न॒ रि॒ष्यति॑ न व्य॒थते ( ) \textbf{ 54} \newline
                  \newline
                                                                  \textbf{ T.A.1.11.8} \newline
                  ना॒स्याक्षो॑ यातु॒ सज्ज॑ति । यच्छ्वेता᳚न् रोहि॑ताꣳश्चा॒ग्नेः । र॒थे यु॑क्त्वाऽधि॒तिष्ठ॑ति ॥ एकया च दशभिश्च॑ स्वभू॒ते । द्वाभ्या मिष्टये विꣳ॑शत्या॒ च । तिसृभिश्च वहसे त्रिꣳ॑शता॒ च ।  नियुद्भिर्-वायविह ता॑ विमु॒ञ्च । \textbf{ 55} \newline
                  \newline
                                                        (तत॑क्षुर् - धीमहि - ना॒म मे॑ - चिकेत - गी॒तं - प्रत्य॑मुञ्चद् - व्य॒थते - +स॒प्त च॑) \textbf{11} \newline \newline
                                \textbf{ T.A.1.12.1} \newline
                  आत॑नुष्व॒ प्रत॑नुष्व । उ॒द्धमाध॑म॒ सन्ध॑म । आदित्ये चन्द्र॑वर्णा॒नां । गर्भ॒ मा धे॑हि॒ यः पुमान्॑ ॥ इ॒तः सि॒क्तꣳ सूर्य॑गतं ।  च॒न्द्रम॑से॒ रस॑ङ्कृधि ।  वारादं जन॑या-ग्रे॒ऽग्निं । य एको॑ रुद्र॒ उच्य॑ते ॥  अ॒सं॒ ख्या॒ताः स॑हस्रा॒णि ।  स्म॒र्यते॑ न च॒ दृश्य॑ते \textbf{ 56} \newline
                  \newline
                                                                  \textbf{ T.A.1.12.2} \newline
                  ए॒वमे॒तन्नि॑बोधत ॥ आ म॒न्द्रै-रि॑न्द्र॒ हरि॑भिः । या॒ हि म॒यूर॑-रोमभिः । मात्वा केचिन्निये मुरि॑न्न पा॒शिनः । द॒ध॒न्वेव॒ ता इ॑हि ॥ मा म॒न्द्रै-रि॑न्द्र॒ हरि॑भिः । या॒मि म॒यूर॑ रोमभिः । मा मा केचिन्निये मुरि॑न्न पा॒शिनः । नि॒ध॒न्वेव॒ ताॅ(2) इ॑मि ॥  अणुभिश्च म॑हद्भि॒श्च \textbf{ 57} \newline
                  \newline
                                                                  \textbf{ T.A.1.12.3} \newline
                  नि॒घृष्वै॑ रस॒मायु॑तैः । कालैर्. हरित्व॑माप॒न्नैः । इन्द्राया॑हि स॒हस्र॑ युक् ॥ अ॒ग्निर् वि॒भ्राष्टि॑ वसनः । वा॒युः श्वेत॑सिकद्रु॒कः । स॒म्ॅव॒थ्स॒रो वि॑षू॒ वर्णैः᳚ । नित्या॒स्ते ऽनुच॑रास्त॒व ॥  सुब्रह्मण्योꣳ सुब्रह्मण्योꣳ सु॑ब्रह्म॒ण्य्ॐ ।  इन्द्रागच्छ हरिव आगच्छ मे॑धाति॒थेः । मेष वृषणश्व॑स्य मे॒ने \textbf{ 58} \newline
                  \newline
                                                                  \textbf{ T.A.1.12.4} \newline
                  गौरावस्कन्दिन्न-हल्या॑यै जा॒र । कौशिक-ब्राह्मण गौतम॑ब्रुवा॒ण ॥ अ॒रु॒णाश्वा॑ इ॒हाग॑ताः । वस॑वः पृथिवि॒ क्षितः॑ । अ॒ष्टौ दि॒ग्वास॑सो॒ ऽग्नयः॑ । अग्निश्च जातवेदा᳚श्चेत्ये॒ते ॥ ताम्राश्वा᳚-स्ताम्र॒रथाः । ताम्रवर्णा᳚ स्तथा॒ऽसिताः । दण्डहस्ताः᳚ खाद॒ग्दतः । इतो रुद्राः᳚ परा॒ङ्गताः \textbf{ 59} \newline
                  \newline
                                                                  \textbf{ T.A.1.12.5} \newline
                  उक्तꣳ स्थानं प्रमाणञ्च॑ पुर॒ इत ॥ बृह॒स्पति॑श्च सवि॒ता च॑ । वि॒श्वरू॑पै-रि॒हाग॑तां । रथे॑नोदक॒वर्त्म॑ना । अ॒फ्सुषा॑ इति॒ तद्द्व॑योः ॥  उक्तो वेषो॑ वासाꣳ॒॒सि च । कालावयवाना-मितः॑ प्रती॒ज्या । वासात्या॑ इत्य॒श्विनोः । कोऽन्तरिक्षे शब्दङ्क॑रोती॒ति । वासिष्ठो रौहिणो मीमाꣳ॑सां च॒क्रे ( ) । तस्यै॒षा भव॑ति ॥  “वा॒श्रेव॑ वि॒द्यु{2}” दिति॑ ॥ ब्रह्म॑ण उ॒दर॑णमसि । ब्रह्म॑ण उदी॒रण॑मसि । ब्रह्म॑ण आ॒स्तर॑णमसि । ब्रह्म॑ण उप॒स्तर॑णमसि । \textbf{ 60} \newline
                  \newline
                                                        (दृश्य॑ते॒ - च - मे॒ने - प॑रा॒ं॒ गता - श्च॒क्रे षट् च॑) \textbf{12} \newline \newline
                                \textbf{ T.A.1.13.1} \newline
                  अ॒ष्टयो॑नी-म॒ष्टपु॑त्रां । अ॒ष्टप॑त्नी-मि॒मां महीं᳚ । अ॒हं ॅवेद॒ न मे॑ मृत्युः ।  न चामृ॑त्युर॒घाह॑रत् ॥ अ॒ष्टयो᳚न्य॒ष्ट पु॑त्रं । अ॒ष्टप॑दि॒द-म॒न्तरि॑क्षं । अ॒हं ॅवेद॒ न मे॑ मृत्युः । न चामृ॑त्युर॒घाह॑रत् । अ॒ष्टयो॑नी-म॒ष्टपु॑त्रां । अ॒ष्टप॑त्नी-म॒मून्दिवं᳚ \textbf{ 61} \newline
                  \newline
                                                                  \textbf{ T.A.1.13.2} \newline
                  अ॒हं ॅवेद॒ न मे॑ मृत्युः ।  न चामृ॑त्युर॒घाह॑रत् ॥ “सु॒त्रामा॑णं{3}” “म॒हीमू॒षु”{4} ॥ अदि॑ति॒र्द्यौ-रदि॑ति-र॒न्तरि॑क्षं ।  अदि॑ति र्मा॒ता स पि॒ता स पु॒त्रः । विश्वे॑ दे॒वा अदि॑तिः॒ पञ्च॒ जनाः᳚ । अदि॑तिर्-जा॒त-मदि॑ति॒र्-जनि॑त्वं ॥ अ॒ष्टौ पु॒त्रासो॒ अदि॑तेः । ये जा॒ता स्त॒न्वः॑ परि॑ । दे॒वाॅ (2) उप॑प्रैथ् स॒प्तभिः॑ \textbf{ 62} \newline
                  \newline
                                                                  \textbf{ T.A.1.13.3} \newline
                  प॒रा॒ मा॒र्ता॒ण्डमास्य॑त् ॥ स॒प्तभिः॑ पु॒त्रै-रदि॑तिः । उप॒ प्रैत् पू॒र्व्यं॑ ॅयुगं᳚ । प्र॒जायै॑ मृ॒त्यवे त॑त् । प॒रा॒ मा॒र्ता॒ण्ड-माभ॑र॒दिति॑ ॥ ताननुक्र॑मिष्या॒मः ॥ मि॒त्रश्च॒ वरु॑णश्च । धा॒ता चा᳚र्य॒मा च॑ । अꣳश॑श्च॒ भग॑श्च । इन्द्रश्च विवस्वाꣳ॑श्चेत्ये॒ते ( ) ॥ "हि॒र॒ण्य॒ग॒र्भो{5}” “हꣳ॒॒सःशु॑चि॒षत्{6}” ।  “ब्रह्म॑ जज्ञा॒नं{7}” “तदित् प॒द{8}” मिति॑ ॥  ग॒र्भः प्रा॑जाप॒त्यः । अथ॒ पुरु॑षः स॒प्तपुरु॑षः । \textbf{ 63} \newline
                  \newline
                                                        (अ॒मूं दिवꣳ॑ - स॒प्तभि॑ - रे॒ते च॒त्वारि॑ च) \textbf{13} \newline \newline
                                \textbf{ T.A.1.14.1} \newline
                  योऽसौ॑ त॒पन्नु॒देति॑ । स सर्वे॑षां भू॒तानां᳚ प्रा॒णाना॒दायो॒देति॑ । मा मे᳚ प्र॒जाया॒ मा प॑शू॒नां । मा मम॑ प्रा॒णाना॒दायोद॑गाः ॥  अ॒सौ यो᳚ ऽस्त॒मेति॑ । स सर्वे॑षां भू॒तानां᳚ प्रा॒णाना॒दाया॒स्तमेति॑ । मा मे᳚ प्र॒जाया॒ मा प॑शू॒नां । मा मम॑ प्रा॒णाना॒दाया स्त॑ङ्गाः ॥  अ॒सौ य आ॒पूर्य॑ति । स सर्वे॑षां भू॒तानां᳚ प्रा॒णै रा॒पूर्य॑ति \textbf{ 64} \newline
                  \newline
                                                                  \textbf{ T.A.1.14.2} \newline
                  मा मे᳚ प्र॒जाया॒ मा प॑शू॒नां । मा मम॑ प्रा॒णै-रा॒पूरि॑ष्ठाः ॥ अ॒सौ यो॑ऽप॒क्षीय॑ति । स सर्वे॑षां भू॒तानां᳚ प्रा॒णै-रप॑क्षीयति । मा मे᳚ प्र॒जाया॒ मा प॑शू॒नां । मा मम॑ प्रा॒णै-रप॑क्षेष्ठाः ॥ अ॒मूनि॒ नक्ष॑त्राणि । सर्वे॑षां भू॒तानां᳚ प्रा॒णैरप॑ प्रसर्पन्ति॒ चोथ्स॑र्पन्ति च ।  मा मे᳚ प्र॒जाया॒ मा प॑शू॒नां ।  मा मम॑ प्रा॒णैरप॑ प्रसृपत॒ मोथ्सृ॑पत । \textbf{ 65} \newline
                  \newline
                                                                  \textbf{ T.A.1.14.3} \newline
                  इ॒मे मासा᳚-श्चार्द्धमा॒साश्च॑ ।  सर्वे॑षां भू॒तानां᳚ प्रा॒णैरप॑ प्रसर्पन्ति॒ चोथ्स॑र्पन्ति च । मा मे᳚ प्र॒जाया॒ मा प॑शू॒नां ।  मा मम॑ प्रा॒णैरप॑ प्रसृपत॒ मोथ्सृ॑पत । इ॒म ऋ॒तवः॑ । सर्वे॑षां भू॒तानां᳚ प्रा॒णैरप॑ प्रसर्पन्ति॒ चोथ्स॑र्पन्ति च ।  मा मे᳚ प्र॒जाया॒ मा प॑शू॒नां ।  मा मम॑ प्रा॒णैरप॑ प्रसृपत॒ मोथ्सृ॑पत ॥ अ॒यꣳ स॑म्ॅवथ्स॒रः ।  सर्वे॑षां भू॒तानां᳚ प्रा॒णैरप॑ प्रसर्पति॒ चोथ्स॑र्पति च \textbf{ 66} \newline
                  \newline
                                                                  \textbf{ T.A.1.14.4} \newline
                  मा मे᳚ प्र॒जाया॒ मा प॑शू॒नां । मा मम॑ प्रा॒णैरप॑ प्रसृप॒ मोथ्सृ॑प ॥  इ॒दमहः॑ । सर्वे॑षां भू॒तानां᳚ प्रा॒णै-रप॑ प्रसर्पति॒ चोथ्स॑र्पति च ।  मा मे᳚ प्र॒जाया॒ मा प॑शू॒नां । मा मम॑ प्रा॒णैरप॑ प्रसृप॒ मोथ्सृ॑प ।  इ॒यꣳ रात्रिः॑ । सर्वे॑षां भू॒तानां᳚ प्रा॒णै-रप॑ प्रसर्पति॒ चोथ्स॑र्पति च ।  मा मे᳚ प्र॒जाया॒ मा प॑शू॒नां । मा मम॑ प्रा॒णैरप॑ प्रसृप॒ मोथ्सृ॑प ( ) ॥  ओउम् भूर्भुव॒स्स्वः॑ ॥ एतद्वो मिथुनं मा नो मिथु॑नꣳ री॒ढ्वं । \textbf{ 67} \newline
                  \newline
                                                  
                (योसौ॒ षोड॑शा॒मूनि॒ द्वाद॑शा॒यं चतु॑र्दश)  (उ॒देत्य॑स्त॒मेत्या॒पूर्य॑त्यप॒क्षीय॑त्य॒मूनि॒ नक्ष॑त्राणी॒ मे मासा॑ इ॒म ऋ॒तवो॒ऽयꣳ स॑म्ॅवथ्स॒र इ॒दमह॑रि॒यꣳ रात्रि॒र्दश॑) \newline
                                      (प्रा॒णैरा॒पूर्य॑ति॒-मोथ्सृ॑पत॒-चोथ्स॑र्पति च॒ - मोथ्सृ॑प॒ द्वे च॑) \textbf{14} \newline \newline
                                \textbf{ T.A.1.15.1} \newline
                  अथादित्यस्याष्ट पु॑रुष॒स्य ॥  वसूना मादित्यानाꣳ स्थाने स्वतेज॑सा भा॒नि ॥  रुद्राणा-मादित्यानाꣳ स्थाने स्वतेज॑सा भा॒नि ॥  आदित्याना-मादित्यानाꣳ स्थाने स्वतेज॑सा भा॒नि ॥ सताꣳ॑सत्या॒नां । आदित्यानाꣳ स्थाने स्वतेज॑सा भा॒नि ॥ अभिधून्वता॑-मभि॒घ्नतां । वातव॑तां म॒रुतां । आदित्यानाꣳ स्थाने स्वतेज॑सा भा॒नि ॥  ऋभूणा-मादित्यानाꣳ स्थाने स्वतेज॑सा भा॒नि ( ) ॥  विश्वेषां᳚ देवा॒नां । आदित्यानाꣳ स्थाने स्वतेज॑सा भा॒नि ॥ सम्ॅवथ्सर॑स्य स॒वितुः । आदित्यस्य स्थाने स्वतेज॑सा भा॒नि ॥ ओउम् भूर्भुव॒स्स्वः॑ ।  रश्मयो वो मिथुनं मा नो मिथु॑नꣳ री॒ढ्वं । \textbf{ 68} \newline
                  \newline
                                                        (ऋभूणामादित्यानाꣳ स्थाने स्वतेज॑सा भा॒नि षट्च॑) \textbf{15} \newline \newline
                                \textbf{ T.A.1.16.1} \newline
                  आरोगस्य स्थाने स्वतेज॑सा भा॒नि । भ्राजस्य स्थाने स्वतेज॑सा भा॒नि । पटरस्य स्थाने स्वतेज॑सा भा॒नि । पतङ्गस्य स्थाने स्वतेज॑सा भा॒नि ।  स्वर्णरस्य स्थाने स्वतेज॑सा भा॒नि ।  ज्योतिषीमतस्य स्थाने स्वतेज॑सा भा॒नि ।  विभासस्य स्थाने स्वतेज॑सा भा॒नि ।  कश्यपस्य स्थाने स्वतेज॑सा भा॒नि । ओउम् भूर्भुव॒स्स्वः॑ । आपो वो मिथुनं मा नो मिथु॑नꣳ री॒ढ्वं । \textbf{ 69} \newline
                  \newline
                                                        (आरोगस्य दश॑) \textbf{16} \newline \newline
                                \textbf{ T.A.1.17.1} \newline
                  अथ वायो-रेकादश-पुरुषस्यैकादश॑-स्त्रीक॒स्य ॥ प्रभ्राजमानानाꣳ रुद्राणाꣳ स्थाने स्वतेज॑सा भा॒नि ।  व्यवदातानाꣳ रुद्राणाꣳ स्थाने स्वतेज॑सा भा॒नि ।  वासुकिवैद्युतानाꣳ रुद्राणाꣳ स्थाने स्वतेज॑सा भा॒नि ।  रजतानाꣳ रुद्राणाꣳ स्थाने स्वतेज॑सा भा॒नि ।  परुषाणाꣳ रुद्राणाꣳ स्थाने स्वतेज॑सा भा॒नि ।  श्यामानाꣳ रुद्राणाꣳ स्थाने स्वतेज॑सा भा॒नि ।  कपिलानाꣳ रुद्राणाꣳ स्थाने स्वतेज॑सा भा॒नि ।  अतिलोहितानाꣳ रुद्राणाꣳ स्थाने स्वतेज॑सा भा॒नि ।  ऊर्द्ध्वानाꣳ रुद्राणाꣳ स्थाने स्वतेज॑सा भा॒नि \textbf{ 70} \newline
                  \newline
                                                                  \textbf{ T.A.1.17.2} \newline
                  अवपतन्तानाꣳ रुद्राणाꣳ स्थाने स्वतेज॑सा भा॒नि ।  वैद्युतानाꣳ रुद्राणाꣳ स्थाने स्वतेज॑सा भा॒नि ।  प्रभ्राजमानीनाꣳ रुद्राणीनाꣳ स्थाने स्वतेज॑सा भा॒नि ।  व्यवदातीनाꣳ रुद्राणीनाꣳ स्थाने स्वतेज॑सा भा॒नि ।  वासुकिवैद्युतीनाꣳ रुद्राणीनाꣳ स्थाने स्वतेज॑सा भा॒नि ।  रजतानाꣳ रुद्राणीनाꣳ स्थाने स्वतेज॑सा भा॒नि ।  परुषाणाꣳ रुद्राणीनाꣳ स्थाने स्वतेज॑सा भा॒नि ।  श्यामानाꣳ रुद्राणीनाꣳ स्थाने स्वतेज॑सा भा॒नि ।  कपिलानाꣳ रुद्राणीनाꣳ स्थाने स्वतेज॑सा भा॒नि ।  अतिलोहितीनाꣳ रुद्राणीनाꣳ स्थाने स्वतेज॑सा भा॒नि ( ) ।  ऊर्द्ध्वानाꣳ रुद्राणीनाꣳ स्थाने स्वतेज॑सा भा॒नि ।  अवपतन्तीनाꣳ रुद्राणीनाꣳ स्थाने स्वतेज॑सा भा॒नि ।  वैद्युतीनाꣳ रुद्राणीनाꣳ स्थाने स्वतेज॑सा भा॒नि ।  ओउम् भूर्भुव॒स्स्वः॑ ।  रूपाणि वो मिथुनं मा नो मिथु॑नꣳ री॒ढ्वं । \textbf{ 71} \newline
                  \newline
                                                        ꣳ रुद्राणाꣳ स्थाने स्वतेज॑सा भा॒ - न्यतिलोहितीनाꣳ रुद्राणीनाꣳ स्थाने स्वतेज॑सा भा॒नि (पञ्च॑ च) \textbf{17} \newline \newline
                                \textbf{ T.A.1.18.1} \newline
                  अथाग्ने॑रष्ट पु॑रुष॒स्य ॥ अग्नेः पूर्व-दिश्यस्य स्थाने स्वतेज॑सा भा॒नि ।  जातवेदस उपदिश्यस्य स्थाने स्वतेज॑सा भा॒नि ।  सहोजसो दक्षिण-दिश्यस्य स्थाने स्वतेज॑सा भा॒नि ।  अजिराप्रभव उपदिश्यस्य स्थाने स्वतेज॑सा भा॒नि ।  वैश्वानरस्यापरदिश्यस्य स्थाने स्वतेज॑सा भा॒नि ।  नर्यापस उपदिश्यस्य स्थाने स्वतेज॑सा भा॒नि ।  पङ्क्तिराधस उदग्दिश्यस्य स्थाने स्वतेज॑सा भा॒नि ।  विसर्पिण उपदिश्यस्य स्थाने स्वतेज॑सा भा॒नि ।  ओउम् भूर्भुव॒स्स्वः॑( ) । दिशो वो मिथुनं मा नो मिथु॑नꣳ री॒ढ्वं । \textbf{ 72} \newline
                  \newline
                                                  
                (एतद् रश्मय आपो रूपाणि दिशः पञ्च॑) \newline
                                      (स्व॑रेक॑म् च) \textbf{18} \newline \newline
                                \textbf{ T.A.1.19.1} \newline
                  दक्षिणपूर्व-स्यान्दिशि विस॑र्पी न॒रकः । तस्मान्नः प॑रिपा॒हि ॥  दक्षिणा-परस्या-न्दिश्य विस॑र्पी न॒रकः ।  तस्मान्नः प॑रिपा॒हि ॥  उत्तर-पूर्वस्या-न्दिशि विषा॑दी न॒रकः ।  तस्मान्नः प॑रिपा॒हि ॥  उत्तरा-परस्या-न्दिश्य विषा॑दी न॒रकः ।  तस्मान्नः प॑रिपा॒हि ॥  ”आ यस्मिन्थ्सप्त वासवा{9}” “इन्द्रियाणि शतक्रत॑{10}” वित्ये॒ते । \textbf{ 73} \newline
                  \newline
                                                        (दक्षिणपूर्वस्याम् नव॑) \textbf{19} \newline \newline
                                \textbf{ T.A.1.20.1} \newline
                  इ॒न्द्र॒ घो॒षा वो॒ वसु॑भिः पु॒रस्ता॒-दुप॑दधतां ।  मनो॑जवसो वः पि॒तृभि॑र् दक्षिण॒त उप॑दधतां ।  प्रचे॑ता वो रु॒द्रैः प॒श्चा-दुप॑दधतां ।  वि॒श्वक॑र्मा व आदि॒त्यै-रु॑त्तर॒त उप॑दधतां । त्वष्टा॑ वो रू॒पै-रु॒परि॑ष्टा॒-दुप॑दधतां । संज्ञानं ॅवः प॑श्चादि॒ति ॥ आ॒दि॒त्यः सर्वो॒ऽग्निः पृ॑थि॒व्यां । वा॒युर॒न्तरि॑क्षे । सूर्यो॑ दि॒वि ।  च॒न्द्रमा॑ दि॒क्षु ( ) । नक्ष॑त्राणि॒ स्वलो॒के ॥ ए॒वा ह्ये॑व । ए॒वा ह्य॑ग्ने । ए॒वा हि वा॑यो । ए॒वा ही᳚न्द्र । ए॒वा हि पू॑षन्न् । ए॒वा हि दे॑वाः । \textbf{ 74} \newline
                  \newline
                                                        (दि॒क्षु स॒प्त च॑) \textbf{20} \newline \newline
                                \textbf{ T.A.1.21.1} \newline
                  आप॑मापाम॒पः सर्वाः᳚ । अ॒स्मा-द॒स्मादि॒तोऽमुतः॑ ।  अ॒ग्निर्वा॒युश्च॒ सूर्य॑श्च । स॒ह स॑ञ्चस्क॒रर्द्धि॑या ॥  वा॒य्वश्वा॑ रश्मि॒ पत॑यः । मरी᳚च्यात्मानो॒ अद्रु॑हः ।  दे॒वी र्भु॑वन॒ सूव॑रीः । पु॒त्र॒व॒त्त्वाय॑ मे सुत ॥  महानाम्नीर्-म॑हामा॒नाः । म॒ह॒सो म॑हस॒स्स्वः॑ \textbf{ 75} \newline
                  \newline
                                                                  \textbf{ T.A.1.21.2} \newline
                  दे॒वीः प॑र्जन्य॒ सूव॑रीः । पु॒त्र॒व॒त्त्वाय॑ मे सुत ॥  अ॒पाश्न्यु॑ष्णि-म॒पा रक्षः॑ । अ॒पाश्न्यु॑ष्णि-म॒पा रघं᳚ । अपा᳚घ्रा॒मप॑ चा॒वर्तिं᳚ । अप॑ दे॒वीरि॒तो हि॑त ॥  वज्रं॑ दे॒वीरजी॑ताꣳश्च । भुव॑नं देव॒ सूव॑रीः । आ॒दि॒त्यानदि॑तिं दे॒वीं । योनि॑नोर्द्ध्व-मु॒दीष॑त । \textbf{ 76} \newline
                  \newline
                                                                  \textbf{ } \newline
                   \textbf{ 0} \newline
                  \newline
                                                        ꣳ स॑स्त॒नूभिः॑ । व्यशे॑म दे॒वहि॑तं॒ ॅयदायुः॑ ॥ स्व॒स्ति न॒ इन्द्रो॑ वृ॒द्धश्र॑वाः । स्व॒स्ति नः॑ पू॒षा वि॒श्ववे॑दाः । स्व॒स्ति न॒स्तार्क्ष्यो॒ अरि॑ष्टनेमिः । स्व॒स्ति नो॒ बृह॒स्पति॑र्-दधातु ॥ के॒तवो॒ अरु॑णासश्च । ऋ॒ष॒यो वात॑रश॒नाः( ) । प्र॒ति॒ष्ठाꣳ श॒तधा॑ हि । स॒माहि॑तासो सहस्र॒धाय॑सं ॥ शि॒वा नः॒ शन्त॑मा भवन्तु । दि॒व्या आप॒ ओष॑धयः ॥ सु॒मृ॒डी॒का सर॑स्वति । माते॒ व्यो॑म स॒न्दृशि॑ ॥ 77 (16) (स्व॑ - रु॒दीष॑त॒ - वात॑रश॒नाः षट्च॑) \textbf{21} \newline \newline
                                \textbf{ T.A.1.22.1} \newline
                  यो॑ऽपां पुष्पं॒ ॅवेद॑ । पुष्प॑वान् प्र॒जावा᳚न् पशु॒मान् भ॑वति । च॒न्द्रमा॒ वा अ॒पां पुष्पं᳚ । पुष्प॑वान् प्र॒जावा᳚न् पशु॒मान् भ॑वति ।  य ए॒वं ॅवेद॑ ॥ यो॑ऽपामा॒यत॑नं॒ ॅवेद॑ । आ॒यत॑नवान् भवति । अ॒ग्निर्वा अ॒पामा॒यत॑नं । आ॒यत॑नवान् भवति । यो᳚ऽग्नेरा॒यत॑नं॒ ॅवेद॑ \textbf{ 78} \newline
                  \newline
                                                                  \textbf{ T.A.1.22.2} \newline
                  आ॒यत॑नवान् भवति । आपो॒ वा अ॒ग्नेरा॒यत॑नं । आ॒यत॑नवान् भवति । य ए॒वं ॅवेद॑ ॥ यो॑ऽपामा॒॒यत॑नं॒ ॅवेद॑ । आ॒यत॑नवान् भवति ।  वा॒युर्वा अ॒पामा॒यत॑नं । आ॒यत॑नवान् भवति ।  यो वा॒योरा॒यत॑नं॒ ॅवेद॑ । आ॒यत॑नवान् भवति \textbf{ 79} \newline
                  \newline
                                                                  \textbf{ T.A.1.22.3} \newline
                  आपो॒ वै वा॒योरा॒यत॑नं । आ॒यत॑नवान् भवति । य ए॒वं ॅवेद॑ ॥ यो॑ऽपामा॒यत॑नं॒ ॅवेद॑ । आ॒यत॑नवान् भवति । अ॒सौ वै तप॑न्न॒पा-मा॒यत॑नं । आ॒यत॑नवान् भवति । यो॑ऽमुष्य॒-तप॑त आ॒यत॑नं॒ ॅवेद॑ । आ॒यत॑नवान् भवति । आपो॒वा अ॒मुष्य॒-तप॑त आ॒यत॑नं \textbf{ 80} \newline
                  \newline
                                                                  \textbf{ T.A.1.22.4} \newline
                  आ॒यत॑नवान् भवति । य ए॒वं ॅवेद॑ ॥ यो॑ऽपामा॒यत॑नं॒ ॅवेद॑ । आ॒यत॑नवान् भवति । च॒न्द्रमा॒ वा अ॒पामा॒यत॑नं । आ॒यत॑नवान् भवति ।  यश्च॒न्द्रम॑स आ॒यत॑नं॒ ॅवेद॑ । आ॒यत॑नवान् भवति । आपो॒ वै च॒न्द्रम॑स आ॒यत॑नं । आ॒यत॑नवान् भवति \textbf{ 81} \newline
                  \newline
                                                                  \textbf{ T.A.1.22.5} \newline
                  य ए॒वं ॅवेद॑ ॥ यो॑ऽपामा॒यत॑नं॒ ॅवेद॑ । आ॒यत॑नवान् भवति । नक्ष॑त्राणि॒ वा अ॒पामा॒यत॑नं । आ॒यत॑नवान् भवति । यो नक्ष॑त्राणा-मा॒यत॑नं॒ ॅवेद॑ । आ॒यत॑नवान् भवति । आपो॒ वै नक्ष॑त्राणा-मा॒यत॑नं । आ॒यत॑नवान् भवति ।  य ए॒वं ॅवेद॑ । \textbf{ 82} \newline
                  \newline
                                                                  \textbf{ T.A.1.22.6} \newline
                  यो॑ऽपामा॒यत॑नं॒ ॅवेद॑ । आ॒यत॑नवान् भवति । प॒र्जन्यो॒ वा अ॒पामा॒यत॑नं । आ॒यत॑नवान् भवति ।  य: प॒र्जन्य॑-स्या॒यत॑नं॒ ॅवेद॑ ।  आ॒यत॑नवान् भवति । आपो॒ वै प॒र्जन्य॑-स्या॒यत॑नं ।  आ॒यत॑नवान् भवति । य ए॒वं ॅवेद॑ ॥ यो॑ऽपामा॒यत॑नं॒ ॅवेद॑ । \textbf{ 83} \newline
                  \newline
                                                                  \textbf{ T.A.1.22.7} \newline
                  आ॒यत॑नवान् भवति । स॒म्ॅव॒थ्स॒रो वा अ॒पामा॒यत॑नं । आ॒यत॑नवान् भवति । यः स॑म्ॅवथ्स॒र-स्या॒यत॑नं॒ ॅवेद॑ ।  आ॒यत॑नवान् भवति । आपो॒ वै स॑म्ॅवथ्स॒र-स्या॒यत॑नं । आ॒यत॑नवान् भवति । य ए॒वं ॅवेद॑ ॥  यो᳚ऽफ्सु नावं॒ प्रति॑ष्ठितां॒ ॅवेद॑ । प्रत्ये॒व ति॑ष्ठति \textbf{ 84} \newline
                  \newline
                                                                  \textbf{ T.A.1.22.8} \newline
                  इ॒मे वै लो॒का अ॒फ्सु प्रति॑ष्ठिताः । तदे॒षाऽभ्यनू᳚क्ता ॥  अ॒पाꣳ रस॒मुद॑यꣳसन्न् । सूर्ये॑ शु॒क्रꣳ स॒माभृ॑तं । अ॒पाꣳ रस॑स्य॒ यो रसः॑ । तं ॅवो॑ गृह्णा-म्युत्त॒ममिति॑ ॥ इ॒मे वै लो॒का अ॒पाꣳ रसः॑ । ते॑ऽमुष्मि॑न्-नादि॒त्ये स॒माभ॑ताः ॥  जा॒नु॒द॒घ्नी-मु॑त्तर-वे॒दीङ्खा॒त्वा । अ॒पां पू॑रयि॒त्वा गु॑ल्फद॒घ्नं \textbf{ 85} \newline
                  \newline
                                                                  \textbf{ T.A.1.22.9} \newline
                  पुष्करपर्णैः पुष्करदण्डैः पुष्करैश्च॑ सꣳस्ती॒र्य । तस्मि॑न् विहा॒यसे । अ॒ग्निं प्र॒णीयो॑प-समा॒धाय॑ ॥ ब्र॒ह्म॒वा॒दिनो॑ वदन्ति ।  कस्मा᳚त् प्रणी॒तेऽय-म॒ग्निश्ची॒यते᳚ । साः प्र॑णी॒तेऽयम॒फ्सु ह्यय॑ञ्ची॒यते᳚ ।  अ॒सौ भुव॑ने॒ऽप्य-ना॑हिताग्नि-रे॒ताः ।  तम॒भित॑ ए॒ता अ॒भीष्ट॑का॒ उप॑दधाति ॥  अ॒ग्नि॒हो॒त्रे द॑र्.शपूर्ण-मा॒सयोः᳚ ।  प॒शु॒ब॒न्धे चा॑तुर्मा॒स्येषु॑ \textbf{ 86} \newline
                  \newline
                                                                  \textbf{ T.A.1.22.10} \newline
                  अथो॑ आहुः । सर्वे॑षु यज्ञ्क्र॒तुष्विति॑ ॥  ए॒तद्ध॑ स्म॒ वा आ॑हुः शण्डि॒लाः । कम॒ग्निञ्चि॑नुते ॥ स॒त्रि॒य-म॒ग्निञ्चि॑न्वा॒नः ।  स॒म्ॅव॒थ्स॒रं प्र॒त्यक्षे॑ण ॥ कम॒ग्निञ्चि॑नुते । सा॒वि॒त्र-म॒ग्निञ्चि॑न्वा॒नः । अ॒मुमा॑दि॒त्यं प्र॒त्यक्षे॑ण ॥ कम॒ग्निञ्चि॑नुते \textbf{ 87} \newline
                  \newline
                                                                  \textbf{ T.A.1.22.11} \newline
                  ना॒चि॒के॒त-म॒ग्निञ्चि॑न्वा॒नः । प्रा॒णान् प्र॒त्यक्षे॑ण ॥ कम॒ग्निञ्चि॑नुते । चा॒तु॒र्॒.हो॒त्रि॒य-म॒ग्निञ्चि॑न्वा॒नः । ब्रह्म॑ प्र॒त्यक्षे॑ण ॥ कम॒ग्निञ्चि॑नुते ।  वै॒श्व॒सृ॒ज-म॒ग्निञ्चि॑न्वा॒नः । शरी॑रं प्र॒त्यक्षे॑ण ॥ कम॒ग्निञ्चि॑नुते ।  उ॒पा॒नु॒वा॒क्य॑मा॒शु-म॒ग्निञ्चि॑न्वा॒नः \textbf{ 88} \newline
                  \newline
                                                                  \textbf{ T.A.1.22.12} \newline
                  इ॒मान् ॅलो॒कान् प्र॒त्यक्षे॑ण ॥ कम॒ग्निञ्चि॑नुते ।  इ॒ममा॑रुण-केतुक-म॒ग्निञ्चि॑न्वा॒न इति॑ । य ए॒वासौ ।  इ॒तश्चा॒-मुत॑श्चा-व्यतीपा॒ती । तमिति॑ ॥ यो᳚ऽग्नेर्मि॑थू॒या वेद॑ । मि॒थु॒न॒वान् भ॑वति । आपो॒ वा अ॒ग्नेर्मि॑थू॒याः ।  मि॒थु॒न॒वान् भ॑वति ( ) । य ए॒वं ॅवेद॑ । \textbf{ 89} \newline
                  \newline
                                                  
                (पुष्प॑म॒ग्निर्वा॒युर॒सौ वै तप॑ञ्च॒न्द्रमा॒ नक्ष॑त्राणि प॒र्जन्यः॑ सम्ॅवथ्स॒रो यो᳚ऽफ्सु नाव॑मे॒तद्ध॑ स्म॒ वै स॑त्रि॒यꣳ स॑म्ॅवथ्स॒रꣳ सा॑वि॒त्रम॒मुं ना॑चिके॒तं प्रा॒णाꣳश्चा॑तुर्.होत्रि॒यं ब्रह्म॑ वैश्वसृ॒जꣳ शरी॑रमुपानुवा॒क्य॑मा॒शुमि॒मान् ॅलो॒कानि॒ममा॑रुणकेतुकं॒ ॅय ए॒वासौ) \newline
                                      (वेद॑ - भव - त्या॒यत॑नं - भवति॒ - वेद॒ - वेद॑ - तिष्ठति - गुल्फद॒घ्नं - चा॑तुर्मा॒स्ये - ष्व॒मुमा॑दि॒त्यं प्र॒त्यक्षे॑ण॒ कम॒ग्निं चि॑नुत - उपानुवा॒क्य॑मा॒शुम॒ग्निं चि॑न्वा॒नो - मि॑थू॒या मि॑थुन॒वान् भ॑व॒त्येकं॑ च) \textbf{22} \newline \newline
                                \textbf{ T.A.1.23.1} \newline
                  आपो॒ वा इ॒दमा॑सन्थ् सलि॒लमे॒व ।  स प्र॒जाप॑ति॒रेकः॑ पुष्करप॒र्णे सम॑भवत् ।  तस्यान्त॒र् मन॑सि कामः॒ सम॑वर्तत । इ॒दꣳ सृ॑जेय॒मिति॑ । तस्मा॒द्यत् पुरु॑षो॒ मन॑साऽभि॒गच्छ॑ति । तद्वा॒चा व॑दति ।  तत्कर्म॑णा करोति । तदे॒षा ऽभ्यनू᳚क्ता ॥ काम॒स्तदग्रे॒ सम॑वर्त॒ताधि॑ । मन॑सो॒ रेतः॑ प्रथ॒मं ॅयदासी᳚त् \textbf{ 90} \newline
                  \newline
                                                                  \textbf{ T.A.1.23.2} \newline
                  स॒तो बन्धु॒मस॑ति॒ निर॑विन्दन्न् । हृ॒दि प्र॒तीष्या॑ क॒वयो॑ मनी॒षेति॑ ॥ उपै॑न॒न्तदुप॑नमति । यत् का॑मो॒ भव॑ति । य ए॒वं ॅवेद॑ ॥ स तपो॑ऽतप्यत ।  स तप॑स्त॒प्त्वा । शरी॑रमधूनुत । तस्य॒ यन् माꣳ॒॒समासी᳚त् । ततो॑ऽरु॒णाः के॒तवो॒ वात॑रश॒ना ऋष॑य॒ उद॑तिष्ठन्न् \textbf{ 91} \newline
                  \newline
                                                                  \textbf{ T.A.1.23.3} \newline
                  ये नखाः᳚ । ते वै॑खान॒साः । ये वालाः᳚ । ते वा॑लखि॒ल्याः । यो रसः॑ । सो॑ऽपां ॥ अ॒न्त॒र॒तः कू॒र्मं भू॒तꣳ सर्प॑न्तं । तम॑ब्रवीत् । मम॒ वै त्वङ्-माꣳ॒॒सा । सम॑भूत् \textbf{ 92} \newline
                  \newline
                                                                  \textbf{ T.A.1.23.4} \newline
                  नेत्य॑ब्रवीत् । पूर्व॑मे॒वाह-मि॒हास॒मिति॑ । तत्पुरु॑षस्य पुरुष॒त्वं । स स॒हस्र॑शीर्.षा॒ पुरु॑षः । स॒ह॒स्रा॒क्षः स॒हस्र॑पात् । भू॒त्वोद॑तिष्ठत् । तम॑ब्रवीत् । त्वं ॅवै पूर्वꣳ॑ सम॑भूः ।  त्वमि॒दं पूर्वः॑ कुरु॒ष्वेति॑ ॥ स इ॒त आ॒दायापः॑ \textbf{ 93} \newline
                  \newline
                                                                  \textbf{ T.A.1.23.5} \newline
                  अ॒ञ्ज॒लिना॑ पु॒रस्ता॑-दु॒पाद॑धात् । ए॒वा ह्ये॒वेति॑ ।  तत॑ आदि॒त्य उद॑तिष्ठत् । सा प्राची॒ दिक् ॥  अथा॑रु॒णः के॒तुर्-द॑क्षिण॒त उ॒पाद॑धात् । ए॒वा ह्यग्न॒ इति॑ । ततो॒ वा अ॒ग्निरुद॑तिष्ठत् । सा द॑क्षि॒णा दिक् । अथा॑रु॒णः के॒तुः प॒श्चादु॒पाद॑धात् । ए॒वा हि वायो॒ इति॑ \textbf{ 94} \newline
                  \newline
                                                                  \textbf{ T.A.1.23.6} \newline
                  ततो॑ वा॒युरुद॑तिष्ठत् । सा प्र॒तीची॒ दिक् ।  अथा॑रु॒णः के॒तु-रु॑त्तर॒त उ॒पाद॑धात् । ए॒वा हीन्द्रेति॑ ।  ततो॒ वा इन्द्र॒ उद॑तिष्ठत् । सोदी॑ची॒ दिक् ।  अथा॑रु॒णः के॒तुर्-मद्ध्य॑ उ॒पाद॑धात् । ए॒वा हि पूष॒न्निति॑ । ततो॒ वै पू॒षोद॑तिष्ठत् । सेयन्दिक् \textbf{ 95} \newline
                  \newline
                                                                  \textbf{ T.A.1.23.7} \newline
                  अथा॑रु॒णः के॒तुरु॒परि॑ष्टा-दु॒पाद॑धात् । ए॒वा हि देवा॒ इति॑ । ततो॑ देव मनु॒ष्याः पि॒तरः॑ । ग॒न्ध॒र्वा॒-फ्स॒रस॒ श्चोद॑-तिष्ठन्न् ।  सोर्द्ध्वा दिक् ॥ या वि॒प्रुषो॑ वि॒परा॑पतन्न् ।  ताभ्योऽसु॑रा॒ रक्षाꣳ॑सि पिशा॒चाश्चो-द॑तिष्ठन्न् । तस्मा॒त्ते परा॑भवन्न् । वि॒प्रुड्भ्यो॒ हि ते सम॑भवन्न् ॥ तदे॒षाऽभ्यनू᳚क्ता । \textbf{ 96} \newline
                  \newline
                                                                  \textbf{ T.A.1.23.8} \newline
                  आपो॑ ह॒ यद् बृ॑ह॒तीर् गर्भ॒मायन्न्॑ । दक्षं॒ दधा॑ना ज॒नय॑न्तीः स्वय॒म्भुं । तत॑ इ॒मेऽद्ध्य-सृ॑ज्यन्त॒ सर्गाः᳚ । अद्भ्यो॒ वा इ॒दꣳ सम॑भूत् ।  तस्मा॑दि॒दꣳ सर्वं॒ ब्रह्म॑ स्वय॒भ्विंति॑ ॥  तस्मा॑दि॒दꣳ सर्वꣳ॒॒ शिथि॑ल-मि॒वा ध्रुव॑-मिवाभवत् ॥ प्र॒जाप॑ति॒र् वाव तत् । आ॒त्मना॒ऽऽत्मानं॑ ॅवि॒धाय॑ ।  तदे॒वानु॒ प्रावि॑शत् ॥ तदे॒षाऽभ्यनू᳚क्ता ( ) । \textbf{ 97} \newline
                  \newline
                                                                  \textbf{ T.A.1.23.9} \newline
                  वि॒धाय॑ लो॒कान्. वि॒धाय॑ भू॒तानि॑ । वि॒धाय॒ सर्वाः᳚ प्र॒दिशो॒ दिश॑श्च । प्र॒जाप॑तिः प्रथम॒जा ऋ॒तस्य॑ । आ॒त्मना॒ऽऽत्मा-न॑म॒भि-सम्ॅवि॑वे॒शेति॑ ॥  सर्व॑मे॒वेदमा॒प्त्वा । सर्व॑-मव॒रुद्ध्य॑ । तदे॒वानु॒ प्रवि॑शति । य ए॒वं ॅवेद॑ । \textbf{ 98} \newline
                  \newline
                                                        (आसी॑ - दतिष्ठन् - नभू॒ - दपो॒ - वायो॒ इति - सेयं दिग॒ - भ्यनू᳚क्ता॒ - ऽभ्यनू᳚क्ता॒ -+ऽष्टौ च॑) \textbf{23} \newline \newline
                                \textbf{ T.A.1.24.1} \newline
                  चतु॑ष्टय्य॒ आपो॑ गृह्णाति । च॒त्वारि॒ वा अ॒पाꣳ रू॒पाणि॑ । मेघो॑ वि॒द्युत् । स्त॒न॒यि॒त्नुर्-वृ॒ष्टिः । तान्ये॒वा व॑रुन्धे ॥ आ॒तप॑ति॒ वर्ष्या॑ गृह्णाति ।  ताः पु॒रस्ता॒-दुप॑दधाति । ए॒ता वै ब्र॑ह्मवर्च॒स्या आपः॑ । मु॒ख॒त ए॒व ब्र॑ह्मवर्च॒स-मव॑रुन्धे ।  तस्मा᳚न्-मुख॒तो ब्र॑ह्मवर्च॒सित॑रः । \textbf{ 99} \newline
                  \newline
                                                                  \textbf{ T.A.1.24.2} \newline
                  कूप्या॑ गृह्णाति । ता द॑क्षिण॒त उप॑दधाति । ए॒ता वै ते॑ज॒स्विनी॒रापः॑ । तेज॑ ए॒वास्य॑ दक्षिण॒तो द॑धाति । तस्मा॒द् दक्षि॒णोर्द्ध॑ ऽस्तेज॒स्वित॑रः ॥  स्था॒व॒रा गृ॑ह्णाति । ताः प॒श्चादुप॑दधाति । प्रति॑ष्ठिता॒ वै स्था॑व॒राः ।  प॒श्चादे॒व प्रति॑तिष्ठति ॥ वह॑न्तीर् गृह्णाति \textbf{ 100} \newline
                  \newline
                                                                  \textbf{ T.A.1.24.3} \newline
                  ता उ॑त्तर॒त उप॑दधाति । ओज॑सा॒ वा ए॒ता वह॑न्तीरि॒वोद्-ग॑तीरि॒व आकूज॑तीरि॒व धाव॑न्तीः । ओज॑ ए॒वास्यो᳚त्तर॒तो द॑धाति । तस्मा॒दुत्त॒रोऽद्ध॑ ओज॒स्वित॑रः ॥ स॒भां॒र्या गृ॑ह्णाति ।  ता मद्ध्य॒ उप॑दधाति । इ॒यं ॅवै स॑भां॒र्याः । अ॒स्यामे॒व प्रति॑तिष्ठति ॥ प॒ल्व॒ल्या गृ॑ह्णाति । ता उ॒परि॑ष्टा-दु॒पाद॑धाति \textbf{ 101} \newline
                  \newline
                                                                  \textbf{ T.A.1.24.4} \newline
                  अ॒सौ वै प॑ल्व॒ल्याः । अ॒मुष्या॑मे॒व प्रति॑तिष्ठति ॥ दि॒क्षूप॑दधाति । दि॒क्षु वा आपः॑ । अन्नं॒ ॅवा आपः॑ । अ॒द्भ्यो वा अन्न॑ञ्जायते । यदे॒वाद्भ्यो-ऽन्नं॒ जाय॑ते । तदव॑रुन्धे ॥ तं ॅवा ए॒तम॑रु॒णाः के॒तवो॒ वात॑रश॒ना ऋष॑यो-ऽचिन्वन्न् । तस्मा॑-दारुण के॒तुकः॑ ( ) ॥ तदे॒षाऽभ्यनू᳚क्ता ॥ के॒तवो॒ अरु॑णासश्च । ऋ॒ष॒यो वात॑रश॒नाः ।  प्र॒ति॒ष्ठाꣳ श॒तधा॑ हि । स॒माहि॑तासो सहस्र॒धाय॑स॒मिति॑ ॥ श॒तश॑श्चै॒व ऽस॒हस्र॑शश्च॒ प्रति॑तिष्ठति । य ए॒तम॒ग्निञ्चि॑नु॒ते ।  य उ॑चैनमे॒वं ॅवेद॑ । \textbf{ 102} \newline
                  \newline
                                                        (ब्र॒ह्म॒व॒र्च॒सित॑रो॒ - वह॑न्तीर् गृह्णाति॒ - ता उ॒परि॑ष्टादु॒पाद॑धा - त्यारुणके॒तुको॒ऽष्टौ च॑) \textbf{24} \newline \newline
                                \textbf{ T.A.1.25.1} \newline
                  जा॒नु॒द॒घ्नी-मु॑त्तरवे॒दीङ्खा॒त्वा । अ॒पां पू॑रयति । अ॒पाꣳ स॑र्व॒त्वाय॑ । पु॒ष्क॒र॒प॒र्णꣳ रु॒क्मं पुरु॑ष॒-मित्युप॑दधाति । तपो॒ वै पु॑ष्करप॒र्णं ।  स॒त्यꣳ रु॒क्मः । अ॒मृतं॒ पुरु॑षः । ए॒ताव॒द्वा वा᳚स्ति । याव॑दे॒तत् । याव॑दे॒वास्ति॑ \textbf{ 103} \newline
                  \newline
                                                                  \textbf{ T.A.1.25.2} \newline
                  तदव॑रुन्धे ॥ कू॒र्ममुप॑दधाति । अ॒पामे॒व मेध॒मव॑रुन्धे । अथो᳚ स्व॒र्गस्य॑ लो॒कस्य॒ सम॑ष्ट्यै ॥ आप॑मापाम॒पः सर्वाः᳚ ।  अ॒स्माद॒स्मा दि॒तोऽमुतः॑ । अ॒ग्निर्वा॒युश्च॒ सूर्य॑श्च ।  स॒ह स॑ञ्चस्क॒रर्द्धि॑या॒ इति॑ । “वा॒य्वश्वा॑ रश्मि॒पत॑यः{11}” ॥ लो॒कं पृ॑णच्छि॒द्रं पृ॑ण \textbf{ 104} \newline
                  \newline
                                                                  \textbf{ T.A.1.25.3} \newline
                  यास्ति॒स्रः प॑रम॒जाः ॥ “इ॒न्द्र॒घो॒षा वो॒ वसु॑भि{12}” “ रे॒वा ह्ये॒वे{13}” ति॑ ॥ पञ्च॒ चित॑य॒ उप॑दधाति । पाङ्क्त॒ऽग्निः । यावा॑ने॒वाग्निः । तञ्चि॑नुते ॥  लो॒कं पृ॑णया द्वि॒तीया॒-मुप॑दधाति ।  पञ्च॑पदा॒ वै वि॒राट् । तस्या॒ वा इ॒यं पादः॑ । अ॒न्तरि॑क्षं॒ पादः॑ ( ) । द्यौः पादः॑ । दिशः॒ पादः॑ । प॒रोर॑जाः॒ पादः॑ ॥ वि॒राज्ये॒व प्रति॑तिष्ठति ।  य ए॒तम॒ग्निञ्चि॑नु॒ते । य उ॑चैनमे॒वं ॅवेद॑ । \textbf{ 105} \newline
                  \newline
                                                        (अस्ति॑ - पृणा॒ - न्तरि॑क्ष॒म् पादः॒ षट्च॑) \textbf{25} \newline \newline
                                \textbf{ T.A.1.26.1} \newline
                  अ॒ग्निं प्र॒णीयो॑प-समा॒धाय॑ । तम॒भित॑ ए॒ता अ॒भीष्ट॑का॒ उप॑दधाति । अ॒ग्नि॒हो॒त्रे द॑र्.शपूर्ण-मा॒सयोः᳚ । प॒शु॒ब॒न्धे चा॑तुर्मा॒स्येषु॑ ।  अथो॑ आहुः । सर्वे॑षु यज्ञ्क्र॒तुष्विति॑ ॥  अथ॑ हस्मा हारु॒णः स्वा॑य॒म्भुवः॑ ।  सा॒वि॒त्रः सर्वो॒ऽग्नि-रित्यन॑नुषङ्गं मन्यामहे ।  नाना॒ वा ए॒तेषां᳚ ॅवी॒र्या॑णि ॥ कम॒ग्निञ्चि॑नुते \textbf{ 106} \newline
                  \newline
                                                                  \textbf{ T.A.1.26.2} \newline
                  स॒त्रि॒य म॒ग्निञ्चि॑न्वा॒नः । कम॒ग्निञ्चि॑नुते ।  सा॒वि॒त्र म॒ग्निञ्चि॑न्वा॒नः । कम॒ग्निञ्चि॑नुते । ना॒चि॒के॒त म॒ग्निञ्चि॑न्वा॒नः । कम॒ग्निञ्चि॑नुते ।  चा॒तु॒र्.॒ हो॒त्रि॒य-म॒ग्निञ्चि॑न्वा॒नः । कम॒ग्निञ्चि॑नुते ।  वै॒श्व॒सृ॒ज म॒ग्निञ्चि॑न्वा॒नः । कम॒ग्निञ्चि॑नुते \textbf{ 107} \newline
                  \newline
                                                                  \textbf{ T.A.1.26.3} \newline
                  उ॒पा॒नु॒वा॒क्य॑-मा॒शु म॒ग्निञ्चि॑न्वा॒नः । कम॒ग्निञ्चि॑नुते । इ॒ममा॑रुण-केतुक म॒ग्निञ्चि॑न्वा॒न इति॑ ॥ वृषा॒ वा अ॒ग्निः । वृषा॑णौ॒ सꣳस्फा॑लयेत् । ह॒न्येता᳚स्य य॒ज्ञ्ः । तस्मा॒न्नानु॒षज्यः॑ ॥  सोत्त॑रवे॒दिषु॑ क्र॒तुषु॑ चिन्वीत । उ॒त्त॒र॒वे॒द्याꣳ ह्य॑ग्निश्ची॒यते᳚ ॥ प्र॒जाका॑मश्चिन्वीत \textbf{ 108} \newline
                  \newline
                                                                  \textbf{ T.A.1.26.4} \newline
                  प्रा॒जा॒प॒त्यो वा ए॒षो᳚ऽग्निः । प्रा॒जा॒प॒त्याः प्र॒जाः । प्र॒जावा᳚न् भवति । य ए॒वं ॅवेद॑ ॥ प॒शुका॑मश्चिन्वीत । सं॒ज्ञानं॒ ॅवा ए॒तत् प॑शू॒नां । यदापः॑ ।  प॒शू॒नामे॒व स॒ज्ञांने॒ ऽग्निञ्चि॑नुते । प॒शु॒मान् भ॑वति । य ए॒वं ॅवेद॑ । \textbf{ 109} \newline
                  \newline
                                                                  \textbf{ T.A.1.26.5} \newline
                  वृष्टि॑कामश्चिन्वीत । आपो॒ वै वृष्टिः॑ । प॒र्जन्यो॒ वर्.षु॑को भवति । य ए॒वं ॅवेद॑ ॥ आ॒म॒या॒वी चि॑न्वीत । आपो॒ वै भे॑ष॒जं । भे॒ष॒ज-मे॒वास्मै॑ करोति । सर्व॒मायु॑रेति ॥ अ॒भि॒चरꣳ॑ श्चिन्वीत । वज्रो॒ वा आपः॑ \textbf{ 110} \newline
                  \newline
                                                                  \textbf{ T.A.1.26.6} \newline
                  वज्र॑मे॒व भ्रातृ॑व्येभ्यः॒ प्रह॑रति । स्तृ॒णु॒त ए॑नं ॥ तेज॑स्कामो॒ यश॑स्कामः ।  ब्र॒ह्म॒व॒र्च॒सका॑मः स्व॒र्गका॑मश्चिन्वीत । ए॒ता व॒द्वा वा᳚स्ति । याव॑दे॒तत् । याव॑दे॒वास्ति॑ । तदव॑रुन्धे ॥ तस्यै॒ तद्व्र॒तं । वर्.ष॑ति॒ न धा॑वेत् \textbf{ 111} \newline
                  \newline
                                                                  \textbf{ T.A.1.26.7} \newline
                  अ॒मृतं॒ ॅवा आपः॑ । अ॒मृत॒स्या-न॑न्तरित्यै ॥  नाफ्सु-मूत्र॑पुरी॒षङ्कु॑र्यात् । न निष्ठी॑वेत् । न वि॒वस॑नः स्नायात् । गुह्यो॒ वा ए॒षो᳚ऽग्निः ।  ए॒तस्या॒ग्ने रन॑ति दाहाय ॥ न पु॑ष्करप॒र्णानि॒ हिर॑ण्यं॒ ॅवाऽधि॒तिष्ठे᳚त् । ए॒तस्या॒ग्ने-रन॑भ्यारोहाय ॥ न कूर्म॒स्याश्नी॑यात् ( ) । नोद॒कस्या॒-घातु॑का॒न्येन॑-मोद॒कानि॑ भवन्ति । अ॒घातु॑का॒ आपः॑ । य ए॒तम॒ग्निञ्चि॑नु॒ते । य उ॑चैनमे॒वं ॅवेद॑ । \textbf{ 112} \newline
                  \newline
                                                        (चि॒नु॒ते॒ - चि॒नु॒ते॒ - प्र॒जाका॑मश्चिन्वीत॒ - य ए॒वं ॅवेदा - पो॑ - धावे॒ - दश्नी॑याच्च॒त्वारि॑ च) \textbf{26} \newline \newline
                                \textbf{ T.A.1.27.1} \newline
                  इ॒मा नु॑कं॒ भुव॑ना सीषधेम । इन्द्र॑श्च॒ विश्वे॑च दे॒वाः ॥ य॒ज्ञ्ञ्च॑ नस्त॒न्वञ्च॑ प्र॒जाञ्च॑ । आ॒दि॒त्यैरिन्दः॑ स॒ह सी॑षधातु ॥  आ॒दि॒त्यैरिन्द्रः॒ सग॑णो-म॒रुद्भिः॑ । अ॒स्माकं॑ भूत्ववि॒ता त॒नूनां᳚ ॥ आप्ल॑वस्व॒ प्रप्ल॑वस्व । आ॒ण्डी भ॑व ज॒ मा मु॒हुः । सुखादीन्दुः॑ खनि॒धनां । प्रति॑मुञ्चस्व॒ स्वां पु॒रं । \textbf{ 113} \newline
                  \newline
                                                                  \textbf{ T.A.1.27.2} \newline
                  मरी॑चयः स्वायंभु॒वाः । ये श॑री॒राण्य॑ कल्पयन्न् ।  ते ते॑ दे॒हङ्क॑ल्पयन्तु । मा च॑ ते॒ ख्या स्म॑ तीरिषत् ॥ उत्ति॑ष्ठत॒ मा स्व॑प्त । अ॒ग्नि-मि॑च्छद्ध्वं॒ भार॑ताः । राज्ञ्ः॒ सोम॑स्य तृ॒प्तासः॑ । सूर्ये॑ण स॒युजो॑षसः ॥ “युवा॑ सु॒वासाः᳚{14}” ॥  अ॒ष्टाच॑क्रा॒ नव॑द्वारा \textbf{ 114} \newline
                  \newline
                                                                  \textbf{ T.A.1.27.3} \newline
                  दे॒वानां॒ पूर॑यो॒द्ध्या । तस्याꣳ॑ हिरण्म॑यः को॒शः । स्व॒र्गो लो॒को ज्योति॒षा ऽऽवृ॑तः ॥ यो वै तां᳚ ब्रह्म॑णो वे॒द । अ॒मृते॑नावृ॒तां पु॑रीं । तस्मै᳚ ब्रह्म च॑ ब्रह्मा॒ च । आ॒युः कीर्तिं॑ प्र॒जान्द॑दुः ॥ वि॒भ्राज॑मानाꣳ॒॒ हरि॑णीं । य॒शसा॑ संप॒रीवृ॑तां । पुरꣳ॑ हिरण्म॑यीं ब्र॒ह्मा \textbf{ 115} \newline
                  \newline
                                                                  \textbf{ T.A.1.27.4} \newline
                  वि॒वेशा॑प॒राजि॑ता ॥ पराङ्गेत्य॑ (पराङत्य॑) ज्याम॒यी ।  पराङ्गेत्य॑ (पराङत्य॑) नाश॒की । इ॒ह चा॑मुत्र॑ चान्वे॒ति । वि॒द्वान् दे॑वासु॒रानु॑भ॒यान् ॥  यत्कु॑मा॒री म॒न्द्रय॑ते । य॒द्यो॒षिद्य-त्प॑ति॒व्रता᳚ । अरि॑ष्टं॒ ॅयत्किञ्च॑ क्रि॒यते᳚ ।  अ॒ग्नि-स्तदनु॑ वेधति ॥ अ॒शृता॑सः शृ॑तास॒श्च \textbf{ 116} \newline
                  \newline
                                                                  \textbf{ T.A.1.27.5} \newline
                  य॒ज्वानो॒ येऽप्य॑य॒ज्वनः॑ । स्व॑र्यन्तो॒ नापे᳚क्षन्ते ।  इन्द्र॑-म॒ग्निञ्च॑ ये वि॒दुः ॥ सिक॑ता इव स॒म्ॅयन्ति॑ ।  र॒श्मिभिः॑-समु॒दीरि॑ताः । अ॒स्मा-ल्लो॒काद॑-मुष्मा॒च्च । ऋ॒षिभि॑-रदात्-पृ॒श्निभिः॑ ॥ अपे॑त॒ वीत॒ वि च॑ सर्प॒तातः॑ । येऽत्र॒ स्थ पु॑रा॒णा ये च॒ नूत॑नाः ।  अहो॑भि-र॒द्भि-र॒क्तुभि॒-र्व्य॑क्तं \textbf{ 117} \newline
                  \newline
                                                                  \textbf{ T.A.1.27.6} \newline
                  य॒मो द॑दात्व-व॒सान॑मस्मै ॥ नृ मु॑णन्तु नृ पा॒त्वर्यः॑ । अ॒कृ॒ष्टा ये च॒ कृष्ट॑जाः । कु॒मारी॑षु क॒नीनी॑षु । जा॒रिणी॑षु च॒ ये हि॒ताः ॥ रेतः॑ पीता॒ आण्ड॑पीताः । अङ्गा॑रेषु च॒ ये हु॒ताः । उ॒भया᳚न् पुत्र॑ पौत्र॒कान् । यु॒वे॒ऽहं ॅय॒मराज॑गान् ॥ “ श॒तमिन्नु श॒रदः॑{15}” ( ) ॥  अदो॒ यद्ब्रह्म॑ विल॒बं । पि॒तृ॒णाञ्च॑ य॒मस्य॑ च ।  वरु॑ण॒-स्याश्वि॑नो-र॒ग्नेः । म॒रुता᳚ञ्च वि॒हाय॑सां ॥  का॒म॒प्र॒यव॑णं मे अस्तु । स ह्ये॑वास्मि॑ स॒नात॑नः । इति नाको ब्रह्मिश्रवो॑ रायो॒ धनं । पु॒त्रानापो॑ दे॒वीरि॒हाहि॑ता । \textbf{ 118} \newline
                  \newline
                                                        (पु॒रं - नव॑द्वारा - ब्र॒ह्मा - च - व्य॑क्तꣳ - श॒रदो॒ऽष्टौ च॑) \textbf{27} \newline \newline
                                \textbf{ T.A.1.28.1} \newline
                  विशी᳚र्ष्णीं॒ गृद्ध्र॑-शीर्ष्णीञ्च । अपेतो॑ निर्.ऋ॒तिꣳ ह॑थः । परिबाधꣳ श्वे॑तकु॒क्षं । नि॒जङ्घꣳ॑ शब॒लोद॑रं ॥ स॒ तान्. वा॒च्याय॑या स॒ह । अग्ने॒ नाश॑य स॒न्दृशः॑ । ई॒र्ष्या॒सू॒ये बु॑भु॒क्षां । म॒न्युं कृ॒त्यां च॑ दीधिरे । रथे॑न किꣳशु॒काव॑ता । अग्ने॒ नाश॑य स॒न्दृशः॑ । \textbf{ 119} \newline
                  \newline
                                                        (विशी᳚र्ष्णीं॒ दश॑) \textbf{28} \newline \newline
                                \textbf{ T.A.1.29.1} \newline
                  प॒र्जन्या॑य॒ प्रगा॑यत । दि॒वस्पु॒त्राय॑ मी॒ढुषे᳚ । स नो॑ य॒वस॑मिच्छतु ॥ इ॒दं ॅवचः॑ प॒र्जन्या॑य स्व॒राजे᳚ । हृ॒दो अ॒स्त्वन्त॑र॒न्त-द्यु॑योत ।  म॒यो॒भूर्वातो॑ वि॒श्वकृ॑ष्टयः सन्त्व॒स्मे । सु॒पि॒प्प॒ला ओष॑धीर् दे॒वगो॑पाः ॥  यो गर्भ॒-मोष॑धीनां । गवा᳚ङ्कृ॒णोत्यर्व॑तां । प॒र्जन्यः॑ पुरु॒षीणां᳚ । \textbf{ 120} \newline
                  \newline
                                                        (प॒र्जन्या॑य॒ दश॑) \textbf{29} \newline \newline
                                \textbf{ T.A.1.30.1} \newline
                  पुन॑र्मामैत्विन्द्रि॒यं । पुन॒रायुः॒ पुन॒र्भगः॑ । पुन॒र् ब्राह्म॑ण-मैतु मा । पुन॒र् द्रवि॑ण मैतु मा ॥ यन्मे॒ऽद्य रेतः॑ पृथि॒वीमस्कान्॑ ।  यदोष॑धीर॒प्यस॑र॒द्-यदापः॑ । इ॒दन्तत् पुन॒राद॑दे । दी॒र्घा॒यु॒त्वाय॒ वर्च॑से ॥ यन्मे॒ रेतः॒ प्रसि॑च्यते । यन्म॒ आजा॑यते॒ पुनः॑( ) । तेन॑ माम॒मृतं॑ कुरु । तेन॑ सुप्र॒जस॑ङ्कुरु । \textbf{ 121} \newline
                  \newline
                                                        (पुन॒र्द्वे च॑) \textbf{30} \newline \newline
                                \textbf{ T.A.1.31.1} \newline
                  अ॒द्भय-स्तिरो॒धा जा॑यत । तव॑ वैश्रव॒णः स॑दा । तिरो॑ धेहि सप॒त्नान्नः॑ । ये अपो॒ऽश्नन्ति॑ केच॒न ॥ त्वा॒ष्ट्रीं मा॒यां ॅव᳚श्रव॒णः ।  रथꣳ॑ सहस्र॒ वन्धु॑रं । पु॒रु॒श्च॒क्रꣳ सह॑स्राश्वं । आस्था॒ याया॑हि नो ब॒लिं ॥ यस्मै॑ भू॒तानि॑ ब॒लिमाव॑हन्ति । धन॒ङ्गावो॒ हस्ति॒ हिर॑ण्य॒मश्वान्॑ \textbf{ 122} \newline
                  \newline
                                                                  \textbf{ T.A.1.31.2} \newline
                  असा॑म सुम॒तौ य॒ज्ञिय॑स्य । श्रियं॒ बिभ्र॒तो ऽन्न॑मुखीं ॅवि॒राजं᳚ ॥ सु॒द॒र्॒.श॒ने च॑ क्रौ॒ञ्चे च॑ । मै॒ना॒गे च॑ म॒हागि॑रौ । श॒तद्वा॒ट्टर॑गम॒न्ता (स॒तद्वा॒ट्टर॑गम॒न्ता) ।  सꣳ॒॒हार्यं॒ नग॑रं॒ तव॑ ॥  इति मन्त्राः᳚ । कल्पो॑ऽत ऊ॒र्द्ध्वं ॥  यदि॒ बलिꣳ॒॒ हरे᳚त् । हि॒र॒ण्य॒ना॒भये॑ वितु॒दये॑ कौबे॒राया॒यं ब॑लिः \textbf{ 123} \newline
                  \newline
                                                                  \textbf{ T.A.1.31.3} \newline
                  सर्वभूताधिपतये न॑म इ॒ति । अथ बलिꣳ हृत्वोप॑तिष्ठे॒त ॥ क्ष॒त्रं क्ष॒त्रं ॅवै᳚श्रव॒णः । ब्राह्मणा॑ वयꣳ॒॒ स्मः ।  नम॑स्ते अस्तु॒ मा मा॑ हिꣳसीः । अस्मात् प्रविश्यान्न॑मद्धी॒ति ॥ अथ तमग्नि-मा॑दधी॒त । यस्मिन्ने तत्कर्म प्र॑युञ्जी॒त ॥  ति॒रोधा॒ भूः । ति॒रोधा॒ भुवः॑ \textbf{ 124} \newline
                  \newline
                                                                  \textbf{ T.A.1.31.4} \newline
                  ति॒रोधाः॒ स्वः॑ । ति॒रोधा॒ भूर्भुव॒स्स्वः॑ ।  सर्वेषां ॅलोकाना-माधिपत्ये॑ सीदे॒ति ॥ अथ तमग्नि॑-मिन्धी॒त । यस्मिन्ने तत्कर्म प्र॑युञ्जी॒त ॥ ति॒रोधा॒ भूः स्वाहा᳚ । ति॒रोधा॒ भुवः॒ स्वाहा᳚ । ति॒रोधाः॒ स्वः॑ स्वाहा᳚ । ति॒रोधा॒ भूर्भुव॒स्स्व॑स्स्वाहा᳚ ॥  यस्मिन्नस्य काले सर्वा आहुतीर्. हुता॑ भवे॒युः \textbf{ 125} \newline
                  \newline
                                                                  \textbf{ T.A.1.31.5} \newline
                  अपि ब्राह्मण॑मुखी॒नाः । तस्मिन्नह्नः काले प्र॑युञ्जी॒त ।  परः॑ सु॒प्तज॑नाद्वे॒पि ॥ मा स्म प्रमाद्यन्त॑ माद्ध्या॒पयेत् । सर्वार्थाः᳚ सिद्ध्य॒न्ते । य ए॑वं ॅवे॒द । क्षुद्ध्य-न्निद॑मजा॒नतां ।  सर्वार्था न॑ सिद्ध्य॒न्ते ॥ यस्ते॑ वि॒घातु॑को भ्रा॒ता ।  ममान्तर्. हृ॑दये॒ श्रितः \textbf{ 126} \newline
                  \newline
                                                                  \textbf{ T.A.1.31.6} \newline
                  तस्मा॑ इ॒ममग्र॒ पिण्ड॑ञ्जुहोमि । स मे᳚ऽर्था॒न् मा विव॑धीत् । मयि॒ स्वाहा᳚ ॥ रा॒जा॒धि॒रा॒जाय॑ प्रसह्य सा॒हिने᳚ । नमो॑ व॒यं ॅवै᳚श्रव॒णाय॑ कुर्महे ।  स मे॒ कामा॒न् काम॒ कामा॑य॒ मह्यं᳚ । का॒मे॒श्व॒रो वै᳚श्रव॒णो द॑दातु । कु॒बे॒राय॑ वैश्रव॒णाय॑ । म॒हा॒रा॒जाय॒ नमः॑ । के॒तवो॒ अरु॑णासश्च ( ) ।  ऋ॒ष॒यो वात॑रश॒नाः । प्र॒ति॒ष्ठाꣳ श॒तधा॑ हि ।  स॒माहि॑तासो सहस्र॒धाय॑सं । शि॒वा नः॒ शन्त॑मा भवन्तु । दि॒व्या आप॒ ओष॑धयः । सु॒मृ॒डी॒का सर॑स्वति । मा ते॒ व्यो॑म स॒न्दृशि॑ । \textbf{ 127} \newline
                  \newline
                                                        (अश्वा᳚न्-बलि॒र्-भुवो॑ - भवे॒युः - श्रित - श्च॑ स॒प्त च॑) \textbf{31} \newline \newline
                                \textbf{ T.A.1.32.1} \newline
                  सम्ॅवथ्सरमेत॑द् व्रत॒ञ्चरेत् । द्वौ॑ वा मा॒सौ ॥ नियमः स॑मासे॒न ॥ तस्मिन् नियम॑ विशे॒षाः । त्रिषवण-मुदको॑पस्प॒र्॒.शी । चतुर्थ कालपान॑भक्तः॒ स्यात् । अहरहर्वा भैक्ष॑मश्न॒यात् ।  औदुम्बरीभिः समिद्भि-रग्निं॑ परि॒चरेत् ।  पुनर्मा मैत्विन्द्रिय-मित्येतेना-ऽनु॑वाके॒न ।  उद्धृत परिपूताभिरद्भिः कार्यं॑ कुर्वी॒त \textbf{ 128} \newline
                  \newline
                                                                  \textbf{ T.A.1.32.2} \newline
                  अ॑सञ्च॒यवान् । अग्नये वायवे॑ सूर्या॒य । ब्रह्मणे प्र॑जाप॒तये । चन्द्रमसे न॑क्षत्रे॒भ्यः । ऋतुभ्यः सम्ॅव॑थ्सरा॒य । वरुणा-यारुणायेति व्र॑तहो॒माः । प्र॒व॒र्ग्यव॑दादे॒शः । अरुणाः का᳚ण्ड ऋ॒षयः ॥ अरण्ये॑ऽधीयी॒रन्न् । भद्रंकर्णेभिरिति द्वे॑ जपि॒त्वा \textbf{ 129} \newline
                  \newline
                                                                  \textbf{ T.A.1.32.3} \newline
                  महानाम्नीभि-रुदकꣳ सꣳ॑स्प॒र्श्य । तमाचा᳚र्यो द॒द्यात् । शिवानः शन्तमे-त्योषधी॑राल॒भते । सुमृडीके॑ति भू॒मिं । एवम॑पव॒र्गे ।  धे॑नुर् द॒क्षिणा । कꣳ सम्ॅवास॑श्च क्षौ॒मं । अन्य॑द्वा शु॒क्लं । य॑था श॒क्ति वा । एवꣳ स्वाद्ध्याय॑ धर्मे॒ण ( ) । अरण्ये॑ऽधीयी॒त । तपस्वी पुण्यो भवति तपस्वी पु॑ण्यो भ॒वति । \textbf{ 130} \newline
                  \newline
                                                        (कु॒र्वी॒त - ज॑पि॒त्वा - स्वाद्ध्याय॑धर्मे॒ण द्वे च॑) \textbf{32} \newline \newline
\textbf{Prapaataka Korvai with starting Padams of 1 to 32 Anuvaakams :-} \newline
(भ॒द्रꣳ - स्मृतिः॑ - साक॒जांना॒ - मक्ष्य- ति॑ता॒म्रा - ण्य॑त्युर्द्ध्वा॒क्ष - आरोगः - क्वेद - मग्निश्च - स॑हस्र॒वृत् - प॒वित्र॑वन्त॒ -आत॑नुष्वा॒ -ष्टयो॑नीं॒ - ॅयोऽसा॒ - वथादित्य - स्यारोग-स्याथ वायो- रथाग्ने॒र् - दक्षिणपूर्वस्या - मि॑न्द्रघो॒षा व॒-आप॑मापां॒-ॅयो॑ऽपा - मापो॒ वै - चतु॑ष्टय्यो - जानुद॒घ्नी - म॒ग्निं प्र॒णीये॒ - मा नु॑ कं॒ - ॅविशी᳚र्ष्णीं - प॒र्जन्या॑य॒ - पुन॑ - र॒द्भ्यः -सम्ॅवथ्सरं द्वात्रिꣳ॑शत्) \newline

\textbf{korvai with starting padams of1, 11, 21 Series of Dasinis :-} \newline
(भ॒द्रं - ज्यो॒तिषा॒ - तस्मिन् राजानं - क॒श्यपा᳚थ्-सहस्र॒वृदि॑यं॒- नपुꣳस॑क - म॒ष्टयो॑नी॒ - मवपतन्ताना - मा॒यत॑नवान् भवति - स॒तो बन्धुं॒ - ता उ॑त्तर॒तो - वज्र॑मे॒व - पुन॑र्मामैतु त्रिꣳ॒॒शदु॑त्तरश॒तम्) \newline

\textbf{first and last padam in TA, 1st Prapaatakam :-} \newline
(भद्रं - तपस्वी पुण्यो भवति तपस्वी पु॑ण्यो भ॒वति) \newline 


॥ कृष्ण यजुर्वेदीय तैत्तिरीय आरण्यके प्रथमः प्रपाठकः समाप्तः ॥

Appendix (Of Expansions)
ट्.आ.1.11.1 - ब्र॒ह्मा दे॒वानां᳚{1}
ब्र॒ह्मा दे॒वानां᳚ पद॒वीः क॑वी॒नामृषि॒र्विप्रा॑णां महि॒षो मृ॒गाणां᳚ । 
श्ये॒नो गृध्रा॑णाꣳ॒॒ स्वधि॑ति॒र् वना॑नाꣳ॒॒ सोमः॑ प॒वित्र॒मत्ये॑ति॒ रेभन्न्॑ ॥ {1}
(Appearing in T.S.3.4.11.1)

ट्.आ.1.12.5 - वा॒श्रेव॑ वि॒द्यु{2} 
वा॒श्रेव॑ वि॒द्युन्मि॑माति व॒थ्सं न मा॒ता सि॑षक्ति । 
यदे॑षां ॅवृ॒ष्टिरस॑र्जि ॥ {2} 
(Appearing in T.S.3.1.11.5)

ट्.आ.1.13.2 - सु॒त्रामा॑णं{3} म॒हीमू॒षु{4}
सु॒त्रामा॑णं पृथि॒वीं द्याम॑ने॒हसꣳ॑ सु॒शर्मा॑ण॒ मदि॑तिꣳ सु॒प्रणी॑तिं । 
दैवीं॒ नावꣳ॑ स्वरि॒त्रामना॑गस॒मस्र॑वन्ती॒मा रु॑हेमा स्व॒स्तये᳚ ॥ {3}

म॒हीमू॒ षु मा॒तरꣳ॑ सुव्र॒ताना॑मृ॒तस्य॒ पत्नी॒मव॑से हुवेम । 
तु॒वि॒क्ष॒त्राम॒जर॑न्तीमुरू॒चीꣳ सु॒शर्मा॑ण॒मदि॑तिꣳ सु॒प्रणी॑तिं ॥ {4} 
(Both {3} and {4} appearing in T.S.1.5.11.5)

ट्.आ.1.13.3 - "हि॒र॒ण्य॒ग॒र्भो{5} 
हि॒र॒ण्य॒ग॒र्भः सम॑वर्त॒ताग्रे॑ भू॒तस्य॑ जा॒तः पति॒रेक॑ आसीत् । 
स दा॑धार पृथि॒वीं द्या मु॒तेमां कस्मै॑ दे॒वाय॑ ह॒विषा॑ विधेम ॥ {5}
(Appearing in T.S.4.1.8.3)

ट्.आ.1.13.3 - हꣳ॒॒सः शु॑चि॒षत्{6} 
हꣳ॒॒सः शु॑चि॒षद् वसु॑रन्तरिक्ष॒-सद्धोता॑ वेदि॒षदति॑थि र्दुरोण॒सत् । 
नृ॒षद् व॑र॒सदृ॑त॒सद् व्यो॑म॒सद॒ब्जा गो॒जा ऋ॑त॒जा अ॑द्रि॒जा ऋ॒तं-बृ॒हत् ॥ {6} 
(Appearing in T.S.1.8.15.2)

ट्.आ.1.13.3 -ब्रह्म॑ जज्ञा॒नं{7} 
ब्रह्म॑ जज्ञा॒नं प्र॑थ॒मं पु॒रस्ता॒द्वि सी॑म॒तः सु॒रुचो॑ वे॒न आ॑वः । 
स बु॒ध्निया॑ उप॒मा अ॑स्य वि॒ष्ठाः स॒तश्च॒ योनि॒मस॑तश्च॒ विवः॑ ॥ {7}
(Appearing in T.S.4.2.8.2)

ट्.आ.1.13.3 -तदित् प॒द{8}
तदित्प॒दं न विचि॑केत वि॒द्वान् । यन्मृ॒तः पुन॑र॒प्येति॑ जी॒वान् । 
त्रि॒वृद्यद् भुव॑नस्य रथ॒वृत् । जी॒वो गर्भो॒ न मृ॒तः स जी॑वात् ॥ {8}
(Appearing in T.B.3.7.10.6)

ट्.आ.1.19.1 - आ यस्मिन्थ्सप्त वासवा{9} इन्द्रियाणि शतक्रत॑{10} 
आयस्मिन्᳚ थ्स॒प्तवा॑स॒वा स्तिष्ठ॑न्ति स्वा॒रुहो॑ यथा ।
ऋषि॑र्.ह दीर्घ॒श्रुत्त॑म॒ इन्द्र॑स्य घ॒र्मो अति॑थिः ॥ {9}
(Appearing in T.S.1.6.12.2)

इ॒न्द्रि॒याणि॑ शतक्रतो॒ या ते॒ जने॑षु प॒ञ्चसु॑ । 
इन्द्र॒ तानि॑ त॒ आ वृ॑णे ॥ {10} 
(Appearing in T.S.1.6.12.1)

ट्.आ.1.25.2 - वा॒य्वश्चा॑ रश्मि॒पत॑यः{11} 
वा॒य्वश्वा॑ रश्मि॒पत॑यः । मरी᳚च्यात्मानो॒ अद्रु॑हः । दे॒वीर् भु॑वन॒ सूव॑रीः । 
पु॒त्र॒व॒त्वाय॑ मे सुत ॥ {11} 
(Appearing in T.A.1.1.2)


ट्.आ.1.25.3 - इ॒न्द्र॒घो॒षा वो॒ वसु॑भि{12} रे॒वा ह्ये॒वे{13} इ॒न्द्र॒ घो॒षा वो॒ वसु॑भिः पु॒रस्ता॒-दुप॑दधतां ॥ {12}
ए॒वा ह्ये॑व ॥ {13} 
(shOrt and lOngform arE thE same in {13}) (Both appearing in T.A.1.20.1)

ट्.आ.1.27.2 - युवा॑ सु॒वासाः᳚{14} 
युवा॑ सु॒वासाः॒ परि॑वीत॒ आगा᳚त् । स उ॒ श्रेया᳚न् भवति॒ जाय॑मानः ।
तं धीरा॑सः क॒वय॒ उन्न॑यन्ति । स्वा॒धियो॒ मन॑सा देव॒यन्तः॑ ॥ {14}
(Appearing in T.B.3.6.1.3 )

ट्.आ.1.27.6 - श॒तमिन्नु श॒रदः॑{15} 
श॒तमिन्नु श॒रदो॒ अन्ति॑ देवा॒ यत्रा॑ नश्च॒क्रा ज॒रसं॑ त॒नूनां᳚ । पु॒त्रासो॒ यत्र॑ पि॒तरो॒ भव॑न्ति॒ मा नो॑ म॒ध्या री॑रिष॒तायु॒र्गन्तोः᳚ ॥ {15}
(This Expansion is appearing in "Apastambeeya Mantra Bagam (APMB)", also known as "EkagnikAdam", in 2nd Prasnam, 4th Section as 3rd Mantra.) 
(This is an odd Expansion as the expanded Mantra is not appearing in any of conventional Taittriya texts likE "Samhita", "Brahamanam" "Aranyakam") \newline
\pagebreak
\pagebreak
        


\end{document}
