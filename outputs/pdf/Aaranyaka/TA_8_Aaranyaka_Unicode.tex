\documentclass[17pt]{extarticle}
\usepackage{babel}
\usepackage{fontspec}
\usepackage{polyglossia}
\usepackage{extsizes}



\setmainlanguage{sanskrit}
\setotherlanguages{english} %% or other languages
\setlength{\parindent}{0pt}
\pagestyle{myheadings}
\newfontfamily\devanagarifont[Script=Devanagari]{AdishilaVedic}


\newcommand{\VAR}[1]{}
\newcommand{\BLOCK}[1]{}




\begin{document}
\begin{titlepage}
    \begin{center}
 
\begin{sanskrit}
    { \Large
    ॐ नमः परमात्मने, श्री महागणपतये नमः
श्री गुरुभ्यो नमः, ह॒रिः॒ ॐ 
    }
    \\
    \vspace{2.5cm}
    \mbox{ \Huge
    कृष्ण यजुर्वेदीय तैत्तिरीय आरण्यके अष्टमः प्रपाठकः   }
\end{sanskrit}
\end{center}

\end{titlepage}
\tableofcontents

ॐ नमः परमात्मने, श्री महागणपतये नमः
श्री गुरुभ्यो नमः, ह॒रिः॒ ॐ \newline
8.1     अष्टमः प्रपाठकः \newline

\addcontentsline{toc}{section}{ 8.1     अष्टमः प्रपाठकः}
\markright{ 8.1     अष्टमः प्रपाठकः \hfill https://www.vedavms.in \hfill}
\section*{ 8.1     अष्टमः प्रपाठकः }
                                \textbf{ T.A.8.1.1} \newline
                  दे॒वा वै स॒त्रमा॑सत । ऋद्धि॑परिमित॒म् ॅयश॑स्कामाः । ते᳚ऽब्रुवन्न् । यन्नः॑ प्रथ॒मं ॅयश॑ ऋ॒च्छात् । सर्वे॑षा-न्न॒स्तथ्-स॒हास॒दिति॑ ।  तेषा᳚म् कुरुक्षे॒त्रं ॅवेदि॑रासीत् । तस्यै॑ खाण्ड॒वो द॑क्षिणा॒र्द्ध आ॑सीत् । तूर्घ्न॑मुत्तरा॒र्द्धः । प॒री॒णज्ज॑घना॒र्द्धः । म॒रव॑ उत्क॒रः \textbf{ 1} \newline
                  \newline
                                                                  \textbf{ T.A.8.1.2} \newline
                  तेषा᳚म् म॒खं ॅवै᳚ष्ण॒वं ॅयश॑ आर्च्छत् । तन्न्य॑कामयत । तेनापा᳚क्रामत् । तम् दे॒वा अन्वा॑यन्न् । यशो॑ऽव॒रुरु॑थ्समानाः । तस्या॒न्वाग॑तस्य ।  स॒व्याद्धनु॒रजा॑यत । दक्षि॑णा॒दिष॑वः । तस्मा॑दिषुध॒न्वम् पुण्य॑जन्म ।  य॒ज्ञ्ज॑न्मा॒ हि \textbf{ 2} \newline
                  \newline
                                                                  \textbf{ T.A.8.1.3} \newline
                  तमेकꣳ॒॒ सन्त᳚म् । ब॒हवो॒ नाभ्य॑धृष्णुवन्न् । तस्मा॒देक॑मिषुध॒न्विन᳚म् । ब॒हवो॑ऽनिषुध॒न्वा नाभिधृ॑ष्णुवन्ति । सो᳚ऽस्मयत । एक॑म् मा॒ सन्त॑म् ब॒हवो॒ नाभ्य॑धर्.षिषु॒रिति॑ ।  तस्य॑ सिष्मिया॒णस्य॒ तेजोऽपा᳚क्रामत् । तद्-दे॒वा ओष॑धीषु॒ न्य॑मृजुः । ते श्या॒माका॑ अभवन्न् । स्म॒याका॒ वै नामै॒ते \textbf{ 3} \newline
                  \newline
                                                                  \textbf{ T.A.8.1.4} \newline
                  तथ् स्म॒याका॑नाꣳ स्मयाक॒त्वम् । तस्मा᳚द्-दीक्षि॒तेना॑पि॒गृह्य॑ स्मेत॒व्य᳚म् । तेज॑सो॒ धृत्यै᳚ । स धनुः॑ प्रति॒ष्कभ्या॑तिष्ठत् । ता उ॑प॒दीका॑ अब्रुव॒न्-वर॑म् ॅवृणामहै । अथ॑ व इ॒मꣳ र॑न्धयाम ।  यत्र॒ क्व॑ च॒ खना॑म । तद॒पो॑ऽभितृ॑णदा॒मेति॑ ।  तस्मा॑दुप॒दीका॒ यत्र॒ क्व॑ च॒ खन॑न्ति । तद॒पो॑ऽभितृ॑न्दन्ति \textbf{ 4} \newline
                  \newline
                                                                  \textbf{ T.A.8.1.5} \newline
                  वारे॑वृतꣳ॒॒ ह्या॑साम् । तस्य॒ ज्यामप्या॑दन्न् ।  तस्य॒ धनु॑र्वि॒प्रव॑माणꣳ॒॒ शिर॒ उद॑वर्तयत् ।  तद् द्यावा॑पृथि॒वी अनु॒प्राव॑र्तत । यत् प्राव॑र्तत ।  तत् प्र॑व॒र्ग्य॑स्य प्रवर्ग्य॒त्वम् । यद्घ्राॅ(4) इत्यप॑तत् ।  तद्ध॒र्मस्य॑ घर्म॒त्वम् ।  म॒ह॒तो वी॒र्य॑मपप्त॒दिति॑ । तन् म॑हावी॒रस्य॑ महावीर॒त्वम् \textbf{ 5} \newline
                  \newline
                                                                  \textbf{ T.A.8.1.6} \newline
                  यद॒स्याः स॒मभ॑रन्न् । तथ् स॒म्राज्ञ्ः॑ सम्रा॒ट्त्वम् ।  तꣳ स्तृ॒तम् दे॒वता᳚स्त्रे॒धा व्य॑गृह्णत । अ॒ग्निः प्रा॑तस्सव॒नम् । इन्द्रो॒ माद्ध्य॑न्दिनꣳ॒॒ सव॑नम् । विश्वे॑ दे॒वास्तृ॑तीयसव॒नम् । तेनाप॑शीर्ष्णा य॒ज्ञेन॒ यज॑मानाः । नाशिषो॒ऽवारु॑न्धत ।  न सु॑व॒र्गं ॅलो॒कम॒भ्य॑जयन्न् ।  ते दे॒वा अ॒श्विना॑वब्रुवन्न् \textbf{ 6} \newline
                  \newline
                                                                  \textbf{ T.A.8.1.7} \newline
                  भि॒षजौ॒ वै स्थः॑ । इ॒दं ॅय॒ज्ञ्स्य॒ शिरः॒ प्रति॑धत्त॒मिति॑ । ताव॑ब्रूतां॒ ॅवर॑म् ॅवृणावहै । ग्रह॑ ए॒व ना॒वत्रापि॑ गृह्यता॒मिति॑ ।  ताभ्या॑-मे॒तमा᳚श्वि॒न-म॑गृह्णन्न् ।  तावे॒तद्-य॒ज्ञ्स्य॒ शिरः॒ प्रत्य॑धत्ताम् ।  यत् प्र॑व॒र्ग्यः॑ । तेन॒ सशी᳚र्ष्णा य॒ज्ञेन॒ यज॑मानाः । अवा॒शिषोऽरु॑न्धत ।  अ॒भि सु॑व॒र्गं ॅलो॒कम॑जयन्न् ( ) । यत् प्र॑व॒र्ग्य॑म् प्रवृ॒णक्ति॑ । य॒ज्ञ्स्यै॒व तच्छिरः॒ प्रति॑दधाति ।  तेन॒ सशी᳚र्ष्णा य॒ज्ञेन॒ यज॑मानः ।  अवा॒शिषो॑ रु॒न्धे । अ॒भि सु॑व॒र्गं ॅलो॒कम् ज॑यति ।  तस्मा॑दे॒ष आ᳚श्वि॒नप्र॑वया इव । यत् प्र॑व॒र्ग्यः॑ । \textbf{ 7} \newline
                  \newline
                                                        (उ॒त्क॒रो - ह्ये॑ - ते - तृ॑न्दन्ति - महावीर॒त्व - म॑ब्रुवन् - नजयन्थ् स॒प्त च॑) \textbf{1} \newline \newline
                                \textbf{ T.A.8.2.1} \newline
                  सा॒वि॒त्रम् जु॑होति॒ प्रसू᳚त्यै । च॒तु॒र्गृ॒ही॒तेन॑ जुहोति ।  चतु॑ष्पादः प॒शवः॑ । प॒शूने॒वाव॑रुन्धे । चत॑स्रो॒ दिशः॑ । दि॒क्ष्वे॑व प्रति॑तिष्ठति ।  छन्दाꣳ॑सि दे॒वेभ्योऽपा᳚क्रामन्न् ।  न वो॑ऽभा॒गानि॑ ह॒व्यं ॅव॑क्ष्याम॒ इति॑ ।  तेभ्य॑ ए॒त-च्च॑तुर्गृही॒त-म॑धारयन्न् । पु॒रो॒नु॒वा॒क्या॑यै या॒ज्या॑यै \textbf{ 8} \newline
                  \newline
                                                                  \textbf{ T.A.8.2.2} \newline
                  दे॒वता॑यै वषट्का॒राय॑ । यच्च॑तुर्गृही॒तम् जु॒होति॑ ।  छन्दाꣳ॑स्ये॒व तत् प्री॑णाति । तान्य॑स्य प्री॒तानि॑ दे॒वेभ्यो॑ ह॒व्यं ॅव॑हन्ति । ब्र॒ह्म॒वा॒दिनो॑ वदन्ति । हो॒त॒व्य॑म् दीक्षि॒तस्य॑ गृ॒हा(3) इ न हो॑त॒व्या(3) मिति॑ । ह॒विर्वै दी᳚क्षि॒तः । यज्जु॑हु॒यात् । ह॒विष्कृ॑तं॒ ॅयज॑मानम॒ग्नौ प्रद॑द्ध्यात् ।  यन्न जु॑हु॒यात् \textbf{ 9} \newline
                  \newline
                                                                  \textbf{ T.A.8.2.3} \newline
                  य॒ज्ञ्॒प॒रुर॒न्तरि॑यात् । यजु॑रे॒व व॑देत् ।  न ह॒विष्कृ॑तं॒ ॅयज॑मानम॒ग्नौ प्र॒दधा॑ति । न य॑ज्ञ्प॒रुर॒न्तरे॑ति । गा॒य॒त्री छन्दाꣳ॒॒स्यत्य॑मन्यत ।  तस्यै॑ वषट्का॒रो᳚ऽभ्यय्य॒ शिरो᳚ऽच्छिनत् ।  तस्यै᳚ द्वे॒धा रसः॒ परा॑ऽपतत् । पृ॒थि॒वीम॒र्द्धः प्रावि॑शत् । प॒शून॒र्द्धः । यः पृ॑थि॒वीम् प्रावि॑शत् \textbf{ 10} \newline
                  \newline
                                                                  \textbf{ T.A.8.2.4} \newline
                  स ख॑दि॒रो॑ऽभवत् । यः प॒शून् । सो॑ऽजाम् । यत्खा॑दि॒र्यभ्रि॒र् भव॑ति । छन्द॑सामे॒व रसे॑न य॒ज्ञ्स्य॒ शिरः॒ सम्भ॑रति । यदौदु॑म्बरी ।  ऊर्ग्वा उ॑दु॒म्बरः॑ । ऊ॒र्जैव य॒ज्ञ्स्य॒ शिरः॒ सम्भ॑रति । यद्-वै॑ण॒वी । तेजो॒ वै वेणुः॑ \textbf{ 11} \newline
                  \newline
                                                                  \textbf{ T.A.8.2.5} \newline
                  तेज॑सै॒व य॒ज्ञ्स्य॒ शिरः॒ सम्भ॑रति । यद्-वैक॑ङ्कती ।  भा ए॒वाव॑रुन्धे । दे॒वस्य॑ त्वा सवि॒तुः प्र॑स॒व इत्यभ्रि॒माद॑त्ते॒ प्रसू᳚त्यै । अ॒श्विनो᳚र्-बा॒हुभ्या॒मित्या॑ह । अ॒श्विनौ॒ हि दे॒वाना॑मद्ध्व॒र्यू आस्ता᳚म् ।  पू॒ष्णो हस्ता᳚भ्या॒मित्याह॒ यत्यै᳚ । वज्र॑ इव॒ वा ए॒षा । यदभ्रिः॑ । अभ्रि॑रसि॒ नारि॑र॒सीत्या॑ह॒ शान्त्यै᳚ \textbf{ 12} \newline
                  \newline
                                                                  \textbf{ T.A.8.2.6} \newline
                  अ॒द्ध्व॒र॒कृद्-दे॒वेभ्य॒ इत्या॑ह । य॒ज्ञो वा अ॑द्ध्व॒रः । य॒ज्ञ्॒कृद्-दे॒वेभ्य॒ इति॒ वावैतदा॑ह । उत्ति॑ष्ठ ब्रह्मणस्पत॒ इत्या॑ह ।  ब्रह्म॑णै॒व य॒ज्ञ्स्य॒ शिरोऽच्छै॑ति ।  प्रैतु॒ ब्रह्म॑ण॒स्पति॒रित्या॑ह । प्रेत्यै॒व य॒ज्ञ्स्य॒ शिरोऽच्छै॑ति । प्र दे॒व्ये॑तु सू॒नृतेत्या॑ह ।  य॒ज्ञो वै सू॒नृता᳚ । अच्छा॑ वी॒रन्नर्य॑म् प॒ङ्क्तिरा॑धस॒मित्या॑ह \textbf{ 13} \newline
                  \newline
                                                                  \textbf{ T.A.8.2.7} \newline
                  पाङ्क्तो॒ हि य॒ज्ञ्ः । दे॒वा य॒ज्ञ्न्न॑यन्तु न॒ इत्या॑ह ।  दे॒वाने॒व य॑ज्ञ्॒नियः॑ कुरुते । देवी᳚ द्यावापृथिवी॒ अनु॑ मे मꣳसाथा॒मित्या॑ह । आ॒भ्यामे॒वानु॑मतो य॒ज्ञ्स्य॒ शिरः॒ सम्भ॑रति । ऋ॒द्ध्यास॑म॒द्य म॒खस्य॒ शिर॒ इत्या॑ह । य॒ज्ञो वै म॒खः ।  ऋ॒द्ध्यास॑म॒द्य य॒ज्ञ्स्य॒ शिर॒ इति॒ वावैतदा॑ह ।  म॒खाय॑ त्वा म॒खस्य॑ त्वा शी॒र्ष्ण इत्या॑ह ।  नि॒र्दिश्यै॒वैन॑द्धरति \textbf{ 14} \newline
                  \newline
                                                                  \textbf{ T.A.8.2.8} \newline
                  त्रिर्.ह॑रति । त्रय॑ इ॒मे लो॒काः । ए॒भ्य ए॒व लो॒केभ्यो॑ य॒ज्ञ्स्य॒ शिरः॒ सम्भ॑रति । तू॒ष्णीम् च॑तु॒र्थꣳ ह॑रति । अप॑रिमितादे॒व य॒ज्ञ्स्य॒ शिरः॒ सम्भ॑रति ।  मृ॒त्ख॒नादग्रे॑ हरति । तस्मा᳚न् मृत्ख॒नः क॑रु॒ण्य॑तमः । इय॒त्यग्र॑ आसी॒रित्या॑ह । अ॒स्यामे॒वाछ॑म्बट्कारं ॅय॒ज्ञ्स्य॒ शिरः॒ सम्भ॑रति । ऊर्ज॒म् ॅवा ए॒तꣳ रस॑म् पृथि॒व्या उ॑प॒दीका॒ उद्दि॑हन्ति \textbf{ 15} \newline
                  \newline
                                                                  \textbf{ T.A.8.2.9} \newline
                  यद्-व॒ल्मीक᳚म् । यद्-व॑ल्मीकव॒पा स॑म्भा॒रो भव॑ति ।  ऊर्ज॑मे॒व रस॑म् पृथि॒व्या अव॑रुन्धे । अथो॒ श्रोत्र॑मे॒व । श्रोत्रꣳ॒॒ ह्ये॑तत् पृ॑थि॒व्याः । यद्-व॒ल्मीकः॑ । अब॑धिरो भवति । य ए॒वं ॅवेद॑ । इन्द्रो॑ वृ॒त्राय॒ वज्र॒मुद॑यच्छत् ।  स यत्र॑यत्र प॒राक्र॑मत \textbf{ 16} \newline
                  \newline
                                                                  \textbf{ T.A.8.2.10} \newline
                  तन्नाद्ध्रि॑यत । स पू॑तीकस्त॒म्बे परा᳚क्रमत । सो᳚ऽद्ध्रियत । सो᳚ऽब्रवीत् । ऊ॒तिं ॅवै मे॑ धा॒ इति॑ । तदू॒तीका॑ना-मूतीक॒त्वम् । यदू॒तीका॒ भव॑न्ति ।  य॒ज्ञायै॒वोतिम् द॑धति । अ॒ग्नि॒जा अ॑सि प्र॒जाप॑ते॒ रेत॒ इत्या॑ह । य ए॒व रसः॑ प॒शून् प्रावि॑शत् \textbf{ 17} \newline
                  \newline
                                                                  \textbf{ T.A.8.2.11} \newline
                  तमे॒वाव॑रुन्धे । पञ्चै॒ते स॑म्भा॒रा भ॑वन्ति । पाङ्क्तो॑ य॒ज्ञ्ः । यावा॑ने॒व य॒ज्ञ्ः । तस्य॒ शिरः॒ सम्भ॑रति । यद्ग्रा॒म्याणा᳚म् पशू॒नाञ्चर्म॑णा स॒म्भरे᳚त् । ग्रा॒म्यान् प॒शूञ्छु॒चाऽर्प॑येत् । कृ॒ष्णा॒जि॒नेन॒ सम्भ॑रति । आ॒र॒ण्याने॒व प॒शूञ्छु॒चाऽर्प॑यति । तस्मा᳚थ्-स॒माव॑त्-पशू॒नाम् प्र॒जाय॑मानानाम् \textbf{ 18} \newline
                  \newline
                                                                  \textbf{ T.A.8.2.12} \newline
                  आ॒र॒ण्याः प॒शवः॒ कनी॑याꣳसः । शु॒चा ह्यृ॑ताः ।  लो॒म॒तः सम्भ॑रति । अतो॒ ह्य॑स्य॒ मेद्ध्य᳚म् । प॒रि॒गृह्या य॑न्ति । रक्ष॑सा॒मप॑हत्यै ।  ब॒हवो॑ हरन्ति । अप॑चितिमे॒वास्मि॑न् दधति ।  उद्ध॑ते॒ सिक॑तोपोप्ते॒ परि॑श्रिते॒ निद॑धति॒ शान्त्यै᳚ ।  मद॑न्तीभि॒रुप॑ सृजति \textbf{ 19} \newline
                  \newline
                                                                  \textbf{ T.A.8.2.13} \newline
                  तेज॑ ए॒वास्मि॑न् दधाति । मधु॑ त्वा मधु॒ला क॑रो॒त्वित्या॑ह । ब्रह्म॑णै॒वास्मि॒न् -तेजो॑ दधाति ।  यद्ग्रा॒म्याणा॒म् पात्रा॑णाम् क॒पालैः᳚ सꣳसृ॒जेत् ।  ग्रा॒म्याणि॒ पात्रा॑णि शु॒चाऽर्प॑येत् । अ॒र्म॒क॒पा॒लैः सꣳसृ॑जति । ए॒तानि॒ वा अ॑नुपजीवनी॒यानि॑ । तान्ये॒व शु॒चाऽर्प॑यति ।  शर्क॑राभिः॒ सꣳसृ॑जति॒ धृत्यै᳚ । अथो॑ श॒न्त्वाय॑ ( ) । अ॒ज॒लो॒मैः सꣳसृ॑जति । ए॒षा वा अ॒ग्नेः प्रि॒या त॒नूः । यद॒जा । प्रि॒ययै॒वैन॑म् त॒नुवा॒ सꣳसृ॑जति । अथो॒ तेज॑सा । कृ॒ष्णा॒जि॒नस्य॒ लोम॑भिः॒ सꣳसृ॑जति । य॒ज्ञो वै कृ॑ष्णाजि॒नम् ।  य॒ज्ञेनै॒व य॒ज्ञ्ꣳ सꣳसृ॑जति । \textbf{ 20} \newline
                  \newline
                                                        (या॒जा॑यै॒ - न जु॑हु॒या - दवि॑श॒द् - वेणः॒ - शान्त्यै॑ - प॒ङ्क्तिरा॑धस॒मित्या॑ह - हरति - दिहन्ति - प॒राक्र॑म॒ - तावि॑शत् - प्र॒जाय॑मानानाꣳ - सृजति - श॒न्त्वाया॒ ष्टौ च॑) \textbf{2} \newline \newline
                                \textbf{ T.A.8.3.1} \newline
                  परि॑श्रिते करोति । ब्र॒ह्म॒व॒र्च॒सस्य॒ परि॑गृहीत्यै । न कु॒र्वन्न॒भि प्रा᳚ण्यात् । यत् कु॒र्वन्न॑भि प्रा॒ण्यात् । प्रा॒णाञ्छु॒चाऽर्प॑येत् । अ॒प॒हाय॒ प्राणि॑ति ।  प्रा॒णाना᳚म् गोपी॒थाय॑ । न प्र॑व॒र्ग्य॑म्-चादि॒त्यम्-चा॒न्तरे॑यात् । यद॑न्तरे॒यात् । दु॒श्चर्मा᳚ स्यात् \textbf{ 21} \newline
                  \newline
                                                                  \textbf{ T.A.8.3.2} \newline
                  तस्मा॒न्नान्त॒राय्य᳚म् । आ॒त्मनो॑ गोपी॒थाय॑ । वेणु॑ना करोति । तेजो॒ वै वेणुः॑ । तेजः॑ प्रव॒र्ग्यः॑ । तेज॑सै॒व तेजः॒ सम॑र्द्धयति । म॒खस्य॒ शिरो॒ऽसीत्या॑ह । य॒ज्ञो वै म॒खः । तस्यै॒तच्छिरः॑ । यत् प्र॑व॒र्ग्यः॑ \textbf{ 22} \newline
                  \newline
                                                                  \textbf{ T.A.8.3.3} \newline
                  तस्मा॑दे॒वमा॑ह । य॒ज्ञ्स्य॑ प॒दे स्थ॒ इत्या॑ह । य॒ज्ञ्स्य॒ ह्ये॑ते प॒दे । अथो॒ प्रति॑ष्ठित्यै । गा॒य॒त्रेण॑ त्वा॒ छन्द॑सा करो॒मीत्या॑ह ।  छन्दो॑भिरे॒वैन॑म् करोति । त्र्यु॑द्धिम् करोति । त्रय॑ इ॒मे लो॒काः । ए॒षाम् ॅलो॒काना॒माप्त्यै᳚ । छन्दो॑भिः करोति \textbf{ 23} \newline
                  \newline
                                                                  \textbf{ T.A.8.3.4} \newline
                  वी॒र्य॑म् ॅवै छन्दाꣳ॑सि । वी॒र्ये॑णै॒वैन॑म् करोति ।  यजु॑षा॒ बिल॑म् करोति॒ व्यावृ॑यै । इय॑न्तम् करोति ।  प्र॒जाप॑तिना यज्ञ्मु॒खेन॒ सम्मि॑तम् । इय॑न्तम् करोति ।  य॒ज्ञ्॒प॒रुषा॒ सम्मि॑तम् । इय॑न्तम् करोति ।  ए॒ताव॒द्वै पुरु॑षे वी॒र्य᳚म् । वी॒र्य॑सम्मितम् \textbf{ 24} \newline
                  \newline
                                                                  \textbf{ T.A.8.3.5} \newline
                  अप॑रिमितम् करोति । अप॑रिमित॒स्याव॑रुद्ध्यै । प॒रि॒ग्री॒वम् क॑रोति॒ धृत्यै᳚ । सूर्य॑स्य॒ हर॑सा श्रा॒येत्या॑ह । य॒था॒य॒जुरे॒वैतत् । अ॒श्व॒श॒केन॑ धूपयति । प्रा॒जा॒प॒त्यो वा अश्वः॑ सयोनि॒त्वाय॑ । वृष्णो॒ अश्व॑स्य नि॒ष्पद॒सीत्या॑ह ।  अ॒सौ वा आ॑दि॒त्यो वृषाऽश्वः॑ । तस्य॒ छन्दाꣳ॑सि नि॒ष्पत् \textbf{ 25} \newline
                  \newline
                                                                  \textbf{ T.A.8.3.6} \newline
                  छन्दो॑भिरे॒वैन॑म् धूपयति । अ॒र्चिषे᳚ त्वा शो॒चिषे॒ त्वेत्या॑ह । तेज॑ ए॒वास्मि॑न् दधाति । वा॒रु॒णो॑ऽभीद्धः॑ । मै॒त्रियोपै॑ति॒ शान्त्यै᳚ । सिद्ध्यै॒ त्वेत्या॑ह । य॒था॒य॒जुरे॒वैतत् । दे॒वस्त्वा॑ सवि॒तोद्व॑प॒त्वित्या॑ह ।  स॒वि॒तृप्र॑सूत ए॒वैन॒म् ब्रह्म॑णा दे॒वता॑भि॒रुद्व॑पति । अप॑द्यमानः पृथि॒व्यामाशा॒ दिश॒ आपृ॒णेत्या॑ह \textbf{ 26} \newline
                  \newline
                                                                  \textbf{ T.A.8.3.7} \newline
                  तस्मा॑द॒ग्निः सर्वा॒ दिशोऽनु॒ विभा॑ति ।  उत्ति॑ष्ठ बृ॒हन्-भ॑वो॒र्द्ध्वस्ति॑ष्ठ ध्रु॒वस्त्वमित्या॑ह॒ प्रति॑ष्ठित्यै ।  ई॒श्व॒रो वा ए॒षो᳚ऽन्धो भवि॑तोः ।  यः प्र॑व॒र्ग्य॑म॒न्वीक्ष॑ते । सूर्य॑स्य त्वा॒ चक्षु॒षाऽन्वी᳚क्ष॒ इत्या॑ह ।  चक्षु॑षो गोपी॒थाय॑ । ऋ॒जवे᳚ त्वा सा॒धवे᳚ त्वा सुक्षि॒त्यै त्वा॒ भूत्यै॒ त्वेत्या॑ह । इ॒यं ॅवा ऋ॒जुः । अ॒न्तरि॑क्षꣳ सा॒धु । अ॒सौ सु॑क्षि॒तिः \textbf{ 27} \newline
                  \newline
                                                                  \textbf{ T.A.8.3.8} \newline
                  दिशो॒ भूतिः॑ । इ॒माने॒वास्मै॑ लो॒कान् क॑ल्पयति । अथो॒ प्रति॑ष्ठित्यै । इ॒द-म॒ह-म॒मुमा॑मुष्याय॒णम् ॅवि॒शा प॒शुभि॑र्-ब्रह्मवर्च॒सेन॒ पर्यू॑हा॒मीत्या॑ह ।  वि॒शैवैन॑म् प॒शुभि॑र्-ब्रह्मवर्च॒सेन॒ पर्यू॑हति ।  वि॒शेति॑ राज॒न्य॑स्य ब्रूयात् । वि॒शैवैन॒म् पर्यू॑हति । प॒शुभि॒रिति॒ वैश्य॑स्य । प॒शुभि॑रे॒वैन॒म् पर्यू॑हति । अ॒सु॒र्य॑म् पात्र॒मना᳚च्छृण्णम् \textbf{ 28} \newline
                  \newline
                                                                  \textbf{ T.A.8.3.9} \newline
                  आच्छृ॑णत्ति । दे॒व॒त्राऽकः॑ । अ॒ज॒क्षी॒रेणाच्छृ॑णत्ति ।  प॒र॒मं ॅवा ए॒तत् पयः॑ । यद॑जक्षी॒रम् । प॒र॒मेणै॒वैन॒म् पय॒साऽऽच्छृ॑णत्ति ।  यजु॑षा॒ व्यावृ॑त्त्यै । छन्दो॑भि॒राच्छृ॑णत्ति । छन्दो॑भि॒र्वा ए॒ष क्रि॑यते । छन्दो॑भिरे॒व छन्दाꣳ॒॒स्याच्छृ॑णत्ति ( ) । छृ॒न्धि वाच॒मित्या॑ह ।  वाच॑मे॒वाव॑रुन्धे । छृ॒न्ध्यूर्ज॒मित्या॑ह । ऊर्ज॑मे॒वाव॑रुन्धे । छृ॒न्धि ह॒विरित्या॑ह । ह॒विरे॒वाकः॑ । देव॑ पुरश्चर स॒घ्यास॒न्त्वेत्या॑ह ।  य॒था॒य॒जुरे॒वैतत् । \textbf{ 29} \newline
                  \newline
                                                        (स्या॒द्- यत् प्र॑व॒र्ग्यः॑ - छन्दो॑भिः करोति - वी॒र्य॑ सम्मित॒म् - छन्दाꣳ॑सि नि॒ष्पत् - पृ॒णेत्या॑ह - सुक्षि॒ति - रना᳚च्छृण्ण॒म् - छन्दाꣳ॒॒स्याच्छृ॑ण त्त्य॒ष्टौ च॑) \textbf{3} \newline \newline
                                \textbf{ T.A.8.4.1} \newline
                  ब्रह्म॒न् प्रच॑रिष्यामो॒ होत॑र्-घ॒र्मम॒भिष्टु॒हीत्या॑ह ।  ए॒ष वा ए॒तर्.हि॒ बृह॒स्पतिः॑ । यद्ब्र॒ह्मा ।  तस्मा॑ ए॒व प्र॑ति॒प्रोच्य॒ प्रच॑रति ।  आ॒त्मनोऽना᳚र्त्यै । य॒माय॑ त्वा म॒खाय॒ त्वेत्या॑ह ।  ए॒ता वा ए॒तस्य॑ दे॒वताः᳚ । ताभि॑रे॒वैनꣳ॒॒ सम॑र्द्धयति । मद॑न्तीभिः॒ प्रोक्ष॑ति । तेज॑ ए॒वास्मि॑न् दधाति \textbf{ 30} \newline
                  \newline
                                                                  \textbf{ T.A.8.4.2} \newline
                  अ॒भि॒पू॒र्वम् प्रोक्ष॑ति । अ॒भि॒पू॒र्व-मे॒वास्मि॒न्-तेजो॑ दधाति । त्रिः प्रोक्ष॑ति । त्र्या॑वृ॒द्धि य॒ज्ञ्ः । अथो॑ मेद्ध्य॒त्वाय॑ । होताऽन्वा॑ह । रक्ष॑सा॒मप॑हत्यै । अन॑वानम् । प्रा॒णानाꣳ॒॒ सन्त॑त्यै ।  त्रि॒ष्टुभः॑ स॒तीर्-गा॑य॒त्रीरि॒वान्वा॑ह \textbf{ 31} \newline
                  \newline
                                                                  \textbf{ T.A.8.4.3} \newline
                  गा॒य॒त्रो हि प्रा॒णः । प्रा॒णमे॒व यज॑माने दधाति । सन्त॑त॒मन्वा॑ह । प्रा॒णाना॑म॒न्नाद्य॑स्य॒ सन्त॑त्यै । अथो॒ रक्ष॑सा॒मप॑हत्यै । यत् परि॑मिता अनुब्रू॒यात् । परि॑मित॒मव॑रुन्धीत । अप॑रिमिता॒ अन्वा॑ह ।  अप॑रिमित॒स्याव॑रुद्ध्यै । शिरो॒ वा ए॒तद्-य॒ज्ञ्स्य॑ \textbf{ 32} \newline
                  \newline
                                                                  \textbf{ T.A.8.4.4} \newline
                  यत् प्र॑व॒र्ग्यः॑ । ऊङ्र्मुञ्जाः᳚ । यन् मौ॒ञ्जो वे॒दो भव॑ति । ऊ॒र्जैव य॒ज्ञ्स्य॒ शिरः॒ सम॑र्द्धयति । प्रा॒णा॒हु॒तीर्-जु॑होति । प्रा॒णाने॒व यज॑माने दधाति । स॒प्त जु॑होति । स॒प्त वै शी॑र्.ष॒ण्याः᳚ प्रा॒णाः । प्रा॒णाने॒वास्मि॑न् दधाति । दे॒वस्त्वा॑ सवि॒ता मद्ध्वा॑ऽन॒क्त्वित्या॑ह \textbf{ 33} \newline
                  \newline
                                                                  \textbf{ T.A.8.4.5} \newline
                  तेज॑सै॒वैन॑मनक्ति । पृ॒थि॒वीम् तप॑सस्त्राय॒स्वेति॒ हिर॑ण्य॒मुपा᳚स्यति । अ॒स्या अन॑तिदाहाय । शिरो॒ वा ए॒तद्-य॒ज्ञ्स्य॑ । यत् प्र॑व॒र्ग्यः॑ । अ॒ग्निः सर्वा॑ दे॒वताः᳚ । प्र॒ल॒वाना॒दीप्योपा᳚स्यति । दे॒वता᳚स्वे॒व य॒ज्ञ्स्य॒ शिरः॒ प्रति॑दधाति । अप्र॑तिशीर्णाग्रम् भवति ।  ए॒तद्-ब॑र्हि॒र्ह्ये॑षः \textbf{ 34} \newline
                  \newline
                                                                  \textbf{ T.A.8.4.6} \newline
                  अ॒र्चिर॑सि शो॒चिर॒सीत्या॑ह । तेज॑ ए॒वास्मि॑न् ब्रह्मवर्च॒सम् द॑धाति । सꣳसी॑दस्व म॒हाꣳ अ॒सीत्या॑ह । म॒हान्. ह्ये॑षः । ब्र॒ह्म॒वा॒दिनो॑ वदन्ति ।  ए॒ते वाव त ऋ॒त्विजः॑ । ये द॑र्.शपूर्णमा॒सयोः᳚ ।  अथ॑ क॒था होता॒ यज॑मानाया॒शिषो॒ नाशा᳚स्त॒ इति॑ ।  पु॒रस्ता॑दाशीः॒ खलु॒ वा अ॒न्यो य॒ज्ञ्ः । उ॒परि॑ष्टादाशीर॒न्यः \textbf{ 35} \newline
                  \newline
                                                                  \textbf{ T.A.8.4.7} \newline
                  अ॒ना॒धृ॒ष्या पु॒रस्ता॒दिति॒ यदे॒तानि॒ यजूꣳ॒॒ष्याह॑ । शी॒र्॒.ष॒त ए॒व य॒ज्ञ्स्य॒ यज॑मान आ॒शिषोऽव॑रुन्धे । आयुः॑ पु॒रस्ता॑दाह । प्र॒जाम् द॑क्षिण॒तः । प्रा॒णम् प॒श्चात् ।  श्रोत्र॑मुत्तर॒तः । विधृ॑तिमु॒परि॑ष्टात् । प्रा॒णाने॒वास्मै॑ स॒मीचो॑ दधाति ।  ई॒श्व॒रो वा ए॒ष दिशोऽनून्म॑दितोः ।  यन्दिशोऽनु॑ व्यास्था॒पय॑न्ति \textbf{ 36} \newline
                  \newline
                                                                  \textbf{ T.A.8.4.8} \newline
                  मनो॒रश्वा॑ऽसि॒ भूरि॑पु॒त्रेती॒मा-म॒भिमृ॑शति ।  इ॒यं ॅवै मनो॒रश्वा॒ भूरि॑पुत्रा । अ॒स्यामे॒व प्रति॑तिष्ठ॒त्यनु॑न्मादाय । सू॒प॒सदा॑ मे भूया॒ मा मा॑ हिꣳसी॒रित्या॒हाहिꣳ॑सायै ।  चितः॑ स्थ परि॒चित॒ इत्या॑ह । अप॑चितिमे॒वास्मि॑न् दधाति । शिरो॒ वा ए॒तद्-य॒ज्ञ्स्य॑ । यत् प्र॑व॒र्ग्यः॑ ।  अ॒सौ खलु॒ वा आ॑दि॒त्यः प्र॑व॒र्ग्यः॑ । तस्य॑ म॒रुतो॑ र॒श्मयः॑ \textbf{ 37} \newline
                  \newline
                                                                  \textbf{ T.A.8.4.9} \newline
                  स्वाहा॑ म॒रुद्भिः॒ परि॑श्रय॒स्वेत्या॑ह । अ॒मुमे॒वादि॒त्यꣳ र॒श्मिभिः॒ पर्यू॑हति ।  तस्मा॑-द॒सावा॑दि॒त्यो॑ऽमुष्मि॑न् ॅलो॒के र॒श्मिभिः॒ पर्यू॑ढः । तस्मा॒द्-राजा॑ वि॒शा पर्यू॑ढः । तस्मा᳚द्-ग्राम॒णीः स॑जा॒तैः पर्यू॑ढः । अ॒ग्नेः सृ॒ष्टस्य॑ य॒तः । विक॑ङ्कत॒म् भा आ᳚र्च्छत् ।  यद् वैक॑ङ्कताः परि॒धयो॒ भव॑न्ति ।  भा ए॒वाव॑रुन्धे । द्वाद॑श भवन्ति \textbf{ 38} \newline
                  \newline
                                                                  \textbf{ T.A.8.4.10} \newline
                  द्वाद॑श॒ मासाः᳚ सम्ॅवथ्स॒रः । स॒म्ॅव॒थ्स॒रमे॒वाव॑रुन्धे ।  अस्ति॑ त्रयोद॒शो मास॒ इत्या॑हुः । यत् त्र॑योद॒शः प॑रि॒धिर्भव॑ति । तेनै॒व त्र॑योद॒शम् मास॒मव॑रुन्धे । अ॒न्तरि॑क्षस्यान्त॒र्-द्धिर॒सीत्या॑ह॒ व्यावृ॑त्त्यै । दिव॒म् तप॑सस्त्राय॒स्वे-त्यु॒परि॑ष्टा॒-द्धिर॑ण्य॒मधि॒ निद॑धाति । अ॒मुष्या॒ अन॑तिदाहाय । अथो॑ आ॒भ्यामे॒वैन॑मुभ॒यतः॒ परि॑गृह्णाति ।  अर्.ह॑न् बिभर्.षि॒ साय॑कानि॒ धन्वेत्या॑ह \textbf{ 39} \newline
                  \newline
                                                                  \textbf{ T.A.8.4.11} \newline
                  स्तौत्ये॒वैन॑मे॒तत् । गा॒य॒त्रम॑सि॒ त्रैष्टु॑भमसि॒ जाग॑तम॒सीति॑ ध॒वित्रा॒ण्याद॑त्ते । छन्दो॑भिरे॒वैना॒न्याद॑त्ते । मधु॒ मद्ध्विति॑ धूनोति । प्रा॒णो वै मधु॑ । प्रा॒णमे॒व यज॑माने दधाति । त्रिः परि॑यन्ति । त्रि॒वृद्धि प्रा॒णः ।  त्रिः परि॑यन्ति । त्र्या॑वृ॒द्धि य॒ज्ञ्ः \textbf{ 40} \newline
                  \newline
                                                                  \textbf{ T.A.8.4.12} \newline
                  अथो॒ रक्ष॑सा॒मप॑हत्यै । त्रिः पुनः॒ परि॑यन्ति ।  षट्थ् सम्प॑द्यन्ते । षड्वा ऋ॒तवः॑ । ऋ॒तुष्वे॒व प्रति॑तिष्ठन्ति । यो वै घ॒र्मस्य॑ प्रि॒यान्-त॒नुव॑मा॒क्राम॑ति । दु॒श्चर्मा॒ वै स भ॑वति ।  ए॒ष ह॒ वा अ॑स्य प्रि॒यान् त॒नुव॒माक्रा॑मति ।  य स्त्रिः प॒रीत्य॑ चतु॒र्थम् पर्ये॑ति ।  ए॒ताꣳ ह॒ वा अ॑स्यो॒ग्रदे॑वो॒ राज॑नि॒राच॑क्राम \textbf{ 41} \newline
                  \newline
                                                                  \textbf{ T.A.8.4.13} \newline
                  ततो॒ वै स दु॒श्चर्मा॑ऽभवत् । तस्मा॒त् त्रिः प॒रीत्य॒ न च॑तु॒र्थम् परी॑यात् । आ॒त्मनो॑ गोपी॒थाय॑ । प्रा॒णा वै ध॒वित्रा॑णि । अव्य॑तिषङ्गम् धून्वन्ति ।  प्रा॒णाना॒-मव्य॑तिषङ्गाय॒ क्लृप्त्यै᳚ । वि॒नि॒षद्य॑ धून्वन्ति । दि॒क्ष्वे॑व प्रति॑तिष्ठन्ति । ऊ॒र्द्ध्वन्धू᳚न्वन्ति । सु॒व॒र्गस्य॑ लो॒कस्य॒ सम॑ष्ट्यै ( ) । स॒र्वतो॑ धून्वन्ति ।  तस्मा॑द॒यꣳ स॒र्वतः॑ पवते । \textbf{ 42} \newline
                  \newline
                                                        (द॒धा॒- ती॒वान्वा॑ह - य॒ज्ञ् - स्या॑है॒ - ष - उ॒परि॑ष्टादाशीर॒न्यो - व्या᳚स्था॒पय॑न्ति - र॒श्मयो॑ - भवन्ति॒ - धन्वेत्या॑ह - य॒ज्ञ् - श्च॑क्राम॒ - सम॑ष्ट्यै॒ द्वे च॑) \textbf{4} \newline \newline
                                \textbf{ T.A.8.5.1} \newline
                  अ॒ग्निष्ट्वा॒ वसु॑भिः पु॒रस्ता᳚द्-रोचयतु गाय॒त्रेण॒ छन्द॒सेत्या॑ह । अ॒ग्निरे॒वैनं॒ ॅवसु॑भिः पु॒रस्ता᳚द्-रोचयति गाय॒त्रेण॒ छन्द॑सा । स मा॑ रुचि॒तो रो॑च॒येत्या॑ह । आ॒शिष॑मे॒वैतामाशा᳚स्ते ।  इन्द्र॑स्त्वा रु॒द्रैर्-द॑क्षिण॒तो रो॑चयतु॒ त्रैष्टु॑भेन॒  छन्द॒सेत्या॑ह । इन्द्र॑ ए॒वैनꣳ॑ रु॒द्रैर्-द॑क्षिण॒तो रो॑चयति॒ त्रैष्टु॑भेन॒ छन्द॑सा । स मा॑ रुचि॒तो रो॑च॒येत्या॑ह । आ॒शिष॑मे॒वैतामाशा᳚स्ते ।  वरु॑णस्त्वाऽऽदि॒त्यैः प॒श्चाद्-रो॑चयतु॒ जाग॑तेन॒ छन्द॒सेत्या॑ह । वरु॑ण ए॒वैन॑मादि॒त्यैः प॒श्चाद्-रो॑चयति॒ जाग॑तेन॒ छन्द॑सा \textbf{ 43} \newline
                  \newline
                                                                  \textbf{ T.A.8.5.2} \newline
                  स मा॑ रुचि॒तो रो॑च॒येत्या॑ह । आ॒शिष॑मे॒वैतामाशा᳚स्ते ।  द्यु॒ता॒नस्त्वा॑ मारु॒तो म॒रुद्भि॑रुत्तर॒तो रो॑चय॒त्वानु॑ष्टुभेन॒ छन्द॒सेत्या॑ह ।  द्यु॒ता॒न ए॒वैन॑म् मारु॒तो म॒रुद्भि॑रुत्तर॒तो रो॑चय॒त्यानु॑ष्टुभेन॒ छन्द॑सा ।  स मा॑ रुचि॒तो रो॑च॒येत्या॑ह । आ॒शिष॑मे॒वैतामाशा᳚स्ते ।  बृह॒स्पति॑स्त्वा॒ विश्वै᳚र्-दे॒वैरु॒परि॑ष्टाद्-रोचयतु॒ पाङ्क्ते॑न॒ छन्द॒सेत्या॑ह ।  बृह॒स्पति॑रे॒वैन॒म् ॅविश्वै᳚र्-दे॒वैरु॒परि॑ष्टाद्-रोचयति॒ पाङ्क्ते॑न॒ छन्द॑सा ।  स मा॑ रुचि॒तो रो॑च॒येत्या॑ह । आ॒शिष॑मे॒वैतामाशा᳚स्ते \textbf{ 44} \newline
                  \newline
                                                                  \textbf{ T.A.8.5.3} \newline
                  रो॒चि॒तस्त्वम् दे॑व घर्म दे॒वेष्वसीत्या॑ह । रो॒चि॒तो ह्ये॑ष दे॒वेषु॑ । रो॒चि॒षी॒याहम् म॑नु॒ष्ये᳚ष्वित्या॑ह । रोच॑त ए॒वैष म॑नु॒ष्ये॑षु । सम्रा᳚ड् घर्म रुचि॒तस्त्वन् दे॒वेष्वायु॑ष्माꣳ स्तेज॒स्वी ब्र॑ह्मवर्च॒स्य॑सीत्या॑ह ।  रु॒चि॒तो ह्ये॑ष दे॒वेष्वायु॑ष्माꣳ स्तेज॒स्वी ब्र॑ह्मवर्च॒सी ।  रु॒चि॒तो॑ऽहम् म॑नु॒ष्ये᳚ष्वायु॑ष्माꣳ स्तेज॒स्वी ब्र॑ह्मवर्च॒सी भू॑यास॒मित्या॑ह ।  रु॒चि॒त ए॒वैष म॑नु॒ष्ये᳚ष्वायु॑ष्माꣳ स्तेज॒स्वी ब्र॑ह्मवर्च॒सी भ॑वति । रुग॑सि॒ रुच॒म् मयि॑ धेहि॒ मयि॒ रुगित्या॑ह । आ॒शिष॑मे॒वैतामाशा᳚स्ते ( ) ।  तं ॅयदे॒तैर्-यजु॑र्भि॒ररो॑चयित्वा । रु॒चि॒तो घ॒र्म इति॑ प्रब्रू॒यात् । अरो॑चुकोऽद्ध्व॒र्युः स्यात् ।  अरो॑चुको॒ यज॑मानः । अथ॒ यदे॑नमे॒तैर्यजु॑र्भी रोचयि॒त्वा । रु॒चि॒तो घर्म॒ इति॒ प्राह॑ । रोचु॑कोऽद्ध्व॒र्युर् भव॑ति । रोचु॑को॒ यज॑मानः । \textbf{ 45} \newline
                  \newline
                                                        (प॒श्चाद् रो॑चयति॒ जग॑तेन॒ छन्द॑सा॒ - पाङ्क्ते॑न॒ छन्द॑सा॒ स मा॑ रुचि॒तो रो॑च॒येत्या॑हा॒शिष॑मे॒वैतामाशा᳚स्ते - शास्ते॒ ऽष्टौ च॑) \textbf{5} \newline \newline
                                \textbf{ T.A.8.6.1} \newline
                  शिरो॒ वा ए॒तद्-य॒ज्ञ्स्य॑ । यत् प्र॑व॒र्ग्यः॑ । ग्री॒वा उ॑प॒सदः॑ । पु॒रस्ता॑दुप॒सदा᳚म् प्रव॒र्ग्य॑म् प्रवृ॑णक्ति । ग्री॒वास्वे॒व य॒ज्ञ्स्य॒ शिरः॒ प्रति॑दधाति । त्रिः प्रवृ॑णक्ति । त्रय॑ इ॒मे लो॒काः । ए॒भ्य ए॒व लो॒केभ्यो॑ य॒ज्ञ्स्य॒ शिरोऽव॑रुन्धे ।  षट्थ् सम्प॑द्यन्ते । षड्वा ऋ॒तवः॑ \textbf{ 46} \newline
                  \newline
                                                                  \textbf{ T.A.8.6.2} \newline
                  ऋ॒तुभ्य॑ ए॒व य॒ज्ञ्स्य॒ शिरोऽव॑रुन्धे । द्वाद॑श॒ कृत्वः॒ प्रवृ॑णक्ति । द्वाद॑श॒ मासाः᳚ सम्ॅवथ्स॒रः । स॒म्ॅव॒थ्स॒रादे॒व य॒ज्ञ्स्य॒ शिरोऽव॑रुन्धे । चतु॑र्विꣳशतिः॒ सम्प॑द्यन्ते । चतु॑र्विꣳशतिरर्द्धमा॒साः ।  अ॒र्द्ध॒मा॒सेभ्य॑ ए॒व य॒ज्ञ्स्य॒ शिरोऽव॑रुन्धे । अथो॒ खलु॑ । स॒कृदे॒व प्र॒वृज्यः॑ । एकꣳ॒॒ हि शिरः॑ \textbf{ 47} \newline
                  \newline
                                                                  \textbf{ T.A.8.6.3} \newline
                  अ॒ग्नि॒ष्टो॒मे प्रवृ॑णक्ति । ए॒तावा॒न्॒. वै य॒ज्ञ्ः । यावा॑नग्निष्टो॒मः । यावा॑ने॒व य॒ज्ञ्ः । तस्य॒ शिरः॒ प्रति॑दधाति । नोक्थ्ये᳚ प्रवृ॑ञ्ज्यात् ।  प्र॒जा वै प॒शव॑ उ॒क्थानि॑ । यदु॒क्थ्ये᳚ प्रवृ॒ञ्ज्यात् ।  प्र॒जाम् प॒शून॑स्य॒ निर्द॑हेत् । वि॒श्व॒जिति॒ सर्व॑पृष्ठे॒ प्रवृ॑णक्ति \textbf{ 48} \newline
                  \newline
                                                                  \textbf{ T.A.8.6.4} \newline
                  पृ॒ष्ठानि॒ वा अच्यु॑तञ्च्यावयन्ति ।  पृ॒ष्ठैरे॒वास्मा॒ अच्यु॑तञ्च्यावयि॒त्वाऽव॑रुन्धे । अप॑श्यम् गो॒पामित्या॑ह । प्रा॒णो वै गो॒पाः । प्रा॒णमे॒व प्र॒जासु॒ विया॑तयति ।  अप॑श्यम् गो॒पामित्या॑ह । अ॒सौ वा आ॑दि॒त्यो गो॒पाः । स हीमाः प्र॒जा गो॑पा॒यति॑ । तमे॒व प्र॒जाना᳚म् गो॒प्तार॑म् कुरुते । अनि॑पद्यमान॒मित्या॑ह \textbf{ 49} \newline
                  \newline
                                                                  \textbf{ T.A.8.6.5} \newline
                  न ह्ये॑ष नि॒पद्य॑ते । आ च॒ परा॑ च प॒थिभि॒श्चर॑न्त॒मित्या॑ह । आ च॒ ह्ये॑ष परा॑ च प॒थिभि॒श्चर॑ति । स स॒द्ध्रीचीः॒ स विषू॑ची॒र्वसा॑न॒ इत्या॑ह । स॒द्ध्रीची᳚श्च॒ ह्ये॑ष विषू॑चीश्च॒ वसा॑नः प्र॒जा अ॒भि वि॒पश्य॑ति । आ व॑रीवर्ति॒ भुव॑नेष्व॒न्तरित्या॑ह । आ ह्ये॑ष व॑री॒वर्ति॒ भुव॑नेष्व॒न्तः । अत्र॑ प्रा॒वीर्मधु॒ माद्ध्वी᳚भ्या॒म् मधु॒ माधू॑चीभ्या॒मित्या॑ह ।  वास॑न्तिकावे॒वास्मा॑ ऋ॒तू क॑ल्पयति ।  सम॒ग्निर॒ग्निना॑ ग॒तेत्या॑ह \textbf{ 50} \newline
                  \newline
                                                                  \textbf{ T.A.8.6.6} \newline
                  ग्रैष्मा॑वे॒वास्मा॑ ऋ॒तू क॑ल्पयति । सम॒ग्निर॒ग्निना॑ ग॒तेत्या॑ह । अ॒ग्निर्ह्ये॑वैषो᳚ऽग्निना॑ स॒ङ्गच्छ॑ते । स्वाहा॒ सम॒ग्निस्तप॑सा ग॒तेत्या॑ह ।  पूर्व॑मे॒वोदि॒तम् । उत्त॑रेणा॒भिगृ॑णाति ।  ध॒र्ता दि॒वो विभा॑सि॒ रज॑सः पृथि॒व्या इत्या॑ह ।  वार्.षि॑कावे॒वास्मा॑ ऋ॒तू क॑ल्पयति ।  हृ॒दे त्वा॒ मन॑से॒ त्वेत्या॑ह । शा॒र॒दावे॒वास्मा॑ ऋ॒तू क॑ल्पयति \textbf{ 51} \newline
                  \newline
                                                                  \textbf{ T.A.8.6.7} \newline
                  दि॒वि दे॒वेषु॒ होत्रा॑ य॒च्छेत्या॑ह । होत्रा॑भिरे॒वेमान् ॅलो॒कान्थ् सम् द॑धाति । विश्वा॑साम् भुवाम् पत॒ इत्या॑ह । हैम॑न्तिकावे॒वास्मा॑ ऋ॒तू क॑ल्पयति ।  दे॒व॒श्रूस्त्वम् दे॑व घर्म दे॒वान् पा॒हीत्या॑ह ।  शै॒शि॒रावे॒वास्मा॑ ऋ॒तू क॑ल्पयति ।  त॒पो॒जां ॅवाच॑म॒स्मे निय॑च्छ देवा॒युव॒मित्या॑ह ।  या वै मेद्ध्या॒ वाक् । सा त॑पो॒जाः । तामे॒वाव॑रुन्धे \textbf{ 52} \newline
                  \newline
                                                                  \textbf{ T.A.8.6.8} \newline
                  गर्भो॑ दे॒वाना॒मित्या॑ह । गर्भो॒ ह्ये॑ष दे॒वाना᳚म् । पि॒ता म॑ती॒नामित्या॑ह । प्र॒जा वै म॒तयः॑ । तासा॑मे॒ष ए॒व पि॒ता । यत् प्र॑व॒र्ग्यः॑ ।  तस्मा॑दे॒वमा॑ह । पतिः॑ प्र॒जाना॒मित्या॑ह । पति॒र्ह्ये॑ष प्र॒जाना᳚म् । मतिः॑ कवी॒नामित्या॑ह \textbf{ 53} \newline
                  \newline
                                                                  \textbf{ T.A.8.6.9} \newline
                  मति॒र्ह्ये॑ष क॑वी॒नाम् । सम् दे॒वो दे॒वेन॑ सवि॒त्रा ऽय॑तिष्ट॒ सꣳ सूर्ये॑णारु॒क्तेत्या॑ह ।  अ॒मुञ्चै॒वादि॒त्यम् प्र॑व॒र्ग्य॑ञ्च॒ सꣳशा᳚स्ति ।  आ॒यु॒र्दास्त्वम॒स्मभ्य॑म् घर्म वर्चो॒दा अ॒सीत्या॑ह ।  आ॒शिष॑मे॒वैतामाशा᳚स्ते । पि॒ता नो॑ऽसि पि॒ता नो॑ बो॒धेत्या॑ह । बो॒धय॑त्ये॒वैन᳚म् । न वै॒ ते॑ऽवका॒शा भ॑वन्ति । पत्नि॑यै दश॒मः ।  नव॒ वै पुरु॑षे प्रा॒णाः \textbf{ 54} \newline
                  \newline
                                                                  \textbf{ T.A.8.6.10} \newline
                  नाभि॑र्दश॒मी । प्रा॒णाने॒व यज॑माने दधाति । अथो॒ दशा᳚क्षरा वि॒राट् । अन्नं॑ ॅवि॒राट् । वि॒राजै॒वान्नाद्य॒मव॑रुन्धे । य॒ज्ञ्स्य॒ शिरो᳚ऽच्छिद्यत । तद्-दे॒वा होत्रा॑भिः॒ प्रत्य॑दधुः । ऋ॒त्विजोऽवे᳚क्षन्ते । ए॒ता वै होत्राः᳚ । होत्रा॑भिरे॒व य॒ज्ञ्स्य॒ शिरः॒ प्रति॑दधाति \textbf{ 55} \newline
                  \newline
                                                                  \textbf{ T.A.8.6.11} \newline
                  रु॒चि॒तमवे᳚क्षन्ते । रु॒चि॒ताद्वै प्र॒जाप॑तिः प्र॒जा अ॑सृजत । प्र॒जानाꣳ॒॒ सृष्ट्यै᳚ । रु॒चि॒तमवे᳚क्षन्ते । रु॒चि॒ताद्वै प॒र्जन्यो॑ वर्.षति । वर्.षु॑कः प॒र्जन्यो॑ भवति । सम् प्र॒जा ए॑धन्ते । रु॒चि॒तमवे᳚क्षन्ते । रु॒चि॒तम् ॅवै ब्र॑ह्मवर्च॒सम् । ब्र॒ह्म॒व॒र्च॒सिनो॑ भवन्ति \textbf{ 56} \newline
                  \newline
                                                                  \textbf{ T.A.8.6.12} \newline
                  अ॒धी॒यन्तोऽवे᳚क्षन्ते । सर्व॒मायु॑र्यन्ति । न पत्न्यवे᳚क्षेत । यत् पन्त्य॒वेक्षे॑त । प्रजा॑येत । प्र॒जान्त्व॑स्यै॒ निर्द॑हेत् । यन्नावेक्षे॑त । न प्रजा॑येत ।  नास्यै᳚ प्र॒जान्निर्द॑हेत् । ति॒र॒स्कृत्य॒ यजु॑र् वाचयति ( ) । प्रजा॑यते । नास्यै᳚ प्र॒जान्निर्द॑हति । त्वष्टी॑मती ते सपे॒येत्या॑ह । सपा॒द्धि प्र॒जाः प्र॒जाय॑न्ते । \textbf{ 57} \newline
                  \newline
                                                        (ऋ॒तवो॒ - हि शिरः॒ - सर्व॑पृष्ठे॒ प्रवृ॑ण॒ - क्त्यनि॑पद्यमान॒मित्या॑ह - ग॒तेत्या॑ह - शार॒दावे॒वास्मा॑ ऋ॒तू क॑ल्पयति - रुन्धे - कवि॒नामित्या॑ह - प्रा॒णाः - प्रति॑दधाति - भवन्ति - वाचयति च॒त्वारि॑ च) \textbf{6} \newline \newline
                                \textbf{ T.A.8.7.1} \newline
                  दे॒वस्य॑ त्वा सवि॒तुः प्र॑स॒व इति॑ रश॒नामाद॑त्ते॒ प्रसू᳚त्यै । अ॒श्विनो᳚र्-बा॒हुभ्या॒मित्या॑ह । अ॒श्विनौ॒ हि दे॒वाना॑मद्ध्व॒र्यू आस्ता᳚म् । पू॒ष्णो हस्ता᳚भ्या॒मित्या॑ह॒ यत्यै᳚ । आद॒देऽदि॑त्यै॒ रास्ना॒ऽसीत्या॑ह॒ यजु॑ष्कृत्यै । इड॒ एह्यदि॑त॒ एहि॒ सर॑स्व॒त्येहीत्या॑ह । ए॒तानि॒ वा अ॑स्यै देवना॒मानि॑ ।  दे॒व॒ना॒मैरे॒वैना॒-माह्व॑यति । असा॒वेह्य-सा॒वेह्य-सा॒वेहीत्या॑ह । ए॒तानि॒ वा अ॑स्यै मनुष्यना॒मानि॑ \textbf{ 58} \newline
                  \newline
                                                                  \textbf{ T.A.8.7.2} \newline
                  म॒नु॒ष्य॒ना॒मैरे॒वैना॒माह्व॑यति ।  षट्थ् सम्प॑द्यन्ते । षड्वा ऋ॒तवः॑ ।  ऋ॒तुभि॑रे॒वैना॒माह्व॑यति । अदि॑त्या उ॒ष्णीष॑म॒सीत्या॑ह । य॒था॒य॒जुरे॒वैतत् ।  वा॒युर॑स्यै॒ड इत्या॑ह । वा॒यु॒दे॒व॒त्यो॑ वै व॒थ्सः ।  पू॒षा त्वो॒पाव॑सृज॒त्वित्या॑ह । पौ॒ष्णा वै दे॒वत॑या प॒शवः॑ \textbf{ 59} \newline
                  \newline
                                                                  \textbf{ T.A.8.7.3} \newline
                  स्वयै॒वैन॑म् दे॒वत॑यो॒पाव॑सृजति । अ॒श्विभ्या॒म् प्रदा॑प॒येत्या॑ह । अ॒श्विनौ॒ वै दे॒वाना᳚म् भि॒षजौ᳚ । ताभ्या॑मे॒वास्मै॑ भेष॒जम् क॑रोति । यस्ते॒ स्तनः॑ शश॒य इत्या॑ह । स्तौत्ये॒वैना᳚म् । उस्र॑ घ॒र्मꣳ शिꣳ॒॒षोस्र॑ घ॒र्मम् पा॑हि घ॒र्माय॑ शिꣳ॒॒षेत्या॑ह । यथा᳚ ब्रू॒याद॒मुष्मै॑ दे॒हीति॑ । ता॒दृगे॒व तत् । बृह॒स्पति॒स्त्वोप॑सीद॒त्वित्या॑ह \textbf{ 60} \newline
                  \newline
                                                                  \textbf{ T.A.8.7.4} \newline
                  ब्रह्म॒ वै दे॒वाना॒म् बृह॒स्पतिः॑ । ब्रह्म॑णै॒वैना॒मुप॑सीदति ।  दान॑वः स्थ॒ पेर॑व॒ इत्या॑ह । मेद्ध्या॑ने॒वैना᳚न्करोति ।  वि॒ष्व॒ग्वृतो॒ लोहि॑ते॒नेत्या॑ह॒ व्यावृ॑त्त्यै ।  अ॒श्विभ्या᳚म् पिन्वस्व॒ सर॑स्वत्यै पिन्वस्व पू॒ष्णे पि॑न्वस्व॒ बृह॒स्पत॑ये पिन्व॒स्वेत्या॑ह ।  ए॒ताभ्यो॒ ह्ये॑षा दे॒वता᳚भ्यः॒ पिन्व॑ते ।  इन्द्रा॑य पिन्व॒स्वेन्द्रा॑य पिन्व॒स्वेत्या॑ह ।  इन्द्र॑मे॒व भा॑ग॒धेये॑न॒ सम॑र्द्धयति । द्विरिन्द्रा॒येत्या॑ह \textbf{ 61} \newline
                  \newline
                                                                  \textbf{ T.A.8.7.5} \newline
                  तस्मा॒दिन्द्रो॑ दे॒वता॑नाम् भूयिष्ठ॒भाक्त॑मः ।  गा॒य॒त्रो॑ऽसि॒ त्रैष्टु॑भोऽसि॒ जाग॑तम॒सीति॑ शफोपय॒मानाद॑त्ते ।  छन्दो॑भिरे॒वैना॒नाद॑त्ते । स॒होर्जो भा॒गेनोप॒मेहीत्या॑ह ।  ऊ॒र्ज ए॒वैन॑म् भा॒गम॑कः ।  अ॒श्विनौ॒ वा ए॒तद्-य॒ज्ञ्स्य॒ शिरः॑ प्रति॒दध॑तावब्रूताम् ।  आ॒वाभ्या॑मे॒व पूर्वा᳚भ्यां॒ ॅवष॑ट्क्रियाता॒ इति॑ ।  इन्द्रा᳚श्विना॒ मधु॑नः सार॒घस्येत्या॑ह ।  अ॒श्विभ्या॑मे॒व पूर्वा᳚भ्यां॒ ॅवष॑ट्करोति ।  अथो॑ अ॒श्विना॑वे॒व भा॑ग॒धेये॑न॒ सम॑र्द्धयति \textbf{ 62} \newline
                  \newline
                                                                  \textbf{ T.A.8.7.6} \newline
                  घ॒र्मम् पा॑त वसवो॒ यज॑ता॒ वडित्या॑ह ।  वसू॑ने॒व भा॑ग॒धेये॑न॒ सम॑र्द्धयति । यद्-व॑षट्कु॒र्यात् । या॒तया॑माऽस्य वषट्का॒रः स्या᳚त् ।  यन्न व॑षट्कु॒र्यात् । रक्षाꣳ॑सि य॒ज्ञ्ꣳ ह॑न्युः । वडित्या॑ह । प॒रोक्ष॑मे॒व वष॑ट्करोति । नास्य॑ या॒तया॑मा वषट्का॒रो भव॑ति ।  न य॒ज्ञ्ꣳ रक्षाꣳ॑सि घ्नन्ति \textbf{ 63} \newline
                  \newline
                                                                  \textbf{ T.A.8.7.7} \newline
                  स्वाहा᳚ त्वा॒ सूर्य॑स्य र॒श्मये॑ वृष्टि॒वन॑ये जुहो॒मीत्या॑ह । यो वा अ॑स्य॒ पुण्यो॑ र॒श्मिः । स वृ॑ष्टि॒वनिः॑ । तस्मा॑ ए॒वैन॑म् जुहोति ।  मधु॑ ह॒विर॒सीत्या॑ह । स्व॒दय॑त्ये॒वैन᳚म् । सूर्य॑स्य॒ तप॑स्त॒पेत्या॑ह ।  य॒था॒य॒जुरे॒वैतत् । द्यावा॑पृथि॒वीभ्या᳚म् त्वा॒ परि॑गृह्णा॒मीत्या॑ह ।  द्यावा॑पृथि॒वीभ्या॑-मे॒वैन॒म् परि॑गृह्णाति \textbf{ 64} \newline
                  \newline
                                                                  \textbf{ T.A.8.7.8} \newline
                  अ॒न्तरि॑क्षेण॒ त्वोप॑यच्छा॒मीत्या॑ह । अ॒न्तरि॑क्षेणै॒वैन॒-मुप॑यच्छति ।  न वा ए॒तम् म॑नु॒ष्यो॑ भर्तु॑मर्.हति ।  दे॒वाना᳚म् त्वा पितृ॒णामनु॑मतो॒ भर्तुꣳ॑ शकेय॒मित्या॑ह । दे॒वैरे॒वैन॑म् पि॒तृभि॒रनु॑मत॒ आद॑त्ते । वि वा ए॑नमे॒तद॑र्द्धयन्ति । यत् प॒श्चात् प्र॒वृज्य॑ पु॒रो जुह्व॑ति । तेजो॑ऽसि॒ तेजोऽनु॒ प्रेहीत्या॑ह । तेज॑ ए॒वास्मि॑न् दधाति । दि॒वि॒स्पृङ्मा मा॑ हिꣳसी-रन्तरिक्ष॒स्पृङ्मा मा॑ हिꣳसीः  पृथिवि॒स्पृङ्मा मा॑ हिꣳसी॒रित्या॒हाहिꣳ॑सायै \textbf{ 65} \newline
                  \newline
                                                                  \textbf{ T.A.8.7.9} \newline
                  सुव॑रसि॒ सुव॑र्मे यच्छ॒ दिवं॑ ॅयच्छ दि॒वो मा॑ पा॒हीत्या॑ह । आ॒शिष॑मे॒वैतामाशा᳚स्ते । शिरो॒ वा ए॒तद्-य॒ज्ञ्स्य॑ । यत् प्र॑व॒र्ग्यः॑ ।  आ॒त्मा वा॒युः । उ॒द्यत्य॑ वातना॒मान्या॑ह ।  आ॒त्मन्ने॒व य॒ज्ञ्स्य॒ शिरः॒ प्रति॑दधाति ।  अन॑वानम् । प्रा॒णानाꣳ॒॒ सन्त॑त्यै । पञ्चा॑ह \textbf{ 66} \newline
                  \newline
                                                                  \textbf{ T.A.8.7.10} \newline
                  पाङ्क्तो॑ य॒ज्ञ्ः । यावा॑ने॒व य॒ज्ञ्ः । तस्य॒ शिरः॒ प्रति॑दधाति । अ॒ग्नये᳚ त्वा॒ वसु॑मते॒ स्वाहेत्या॑ह ।  अ॒सौ वा आ॑दि॒त्यो᳚ऽग्निर्-वसु॑मान् । तस्मा॑ ए॒वैन॑ञ्जुहोति । सोमा॑य त्वा रु॒द्रव॑ते॒ स्वाहेत्या॑ह ।  च॒न्द्रमा॒ वै सोमो॑ रु॒द्रवान्॑ । तस्मा॑ ए॒वैन॑ञ्जुहोति । वरु॑णाय त्वाऽऽदि॒त्यव॑ते॒ स्वाहेत्या॑ह \textbf{ 67} \newline
                  \newline
                                                                  \textbf{ T.A.8.7.11} \newline
                  अ॒फ्सु वै वरु॑ण आदि॒त्यवान्॑ । तस्मा॑ ए॒वैन॑ञ्जुहोति ।  बृह॒स्पत॑ये त्वा वि॒श्वदे᳚व्यावते॒ स्वाहेत्या॑ह ।  ब्रह्म॒ वै दे॒वाना॒म् बृह॒स्पतिः॑ ।  ब्रह्म॑ण ए॒वैन॑ञ्जुहोति ।  स॒वि॒त्रे त्व॑र्भु॒मते॑ विभु॒मते᳚ प्रभु॒मते॒ वाज॑वते॒ स्वाहेत्या॑ह । स॒म्ॅव॒थ्स॒रो वै स॑वि॒तर्भु॒मान्. वि॑भु॒मान् प्र॑भु॒मान्. वाज॑वान् । तस्मा॑ ए॒वैन॑ञ्जुहोति । य॒माय॒ त्वाऽङ्गि॑रस्वते पितृ॒मते॒ स्वाहेत्या॑ह ।  प्रा॒णो वै य॒मोऽङ्गि॑रस्वान् पितृ॒मान् \textbf{ 68} \newline
                  \newline
                                                                  \textbf{ T.A.8.7.12} \newline
                  तस्मा॑ ए॒वैन॑ञ्जुहोति । ए॒ताभ्य॑ ए॒वैन॑म् दे॒वता᳚भ्यो जुहोति । दश॒ सम्प॑द्यन्ते । दशा᳚क्षरा वि॒राट् । अन्नं॑ ॅवि॒राट् । वि॒राजै॒वान्नाद्य॒मव॑रुन्धे । रौ॒हि॒णाभ्यां॒ ॅवै दे॒वाः सु॑व॒र्गं ॅलो॒कमा॑यन्न् । तद्-रौ॑हि॒णयो॑ रौहिण॒त्वम् । यद्-रौ॑हि॒णौ भव॑तः । रौ॒हि॒णाभ्या॑मे॒व तद्-यज॑मानः सुव॒र्गं ॅलो॒कमे॑ति ( ) ।  अह॒र्ज्योतिः॑ के॒तुना॑ जुषताꣳ सुज्यो॒तिर्ज्योति॑षाꣳ॒॒ स्वाहा॒ रात्रि॒र्ज्योतिः॑ के॒तुना॑ जुषताꣳ सुज्यो॒तिर्ज्योति॑षाꣳ॒॒ स्वाहेत्या॑ह ।  आ॒दि॒त्यमे॒व तद॒मुष्मि॑न् ॅलो॒केऽह्ना॑ प॒रस्ता᳚द्-दाधार । रात्रि॑या अ॒वस्ता᳚त् ।  तस्मा॑द॒सावा॑दि॒त्यो॑ ऽमुष्मि॑न् ॅलो॒के॑ऽहोरा॒त्राभ्या᳚म् धृ॒तः । \textbf{ 69} \newline
                  \newline
                                                        (म॒नु॒ष्य॒ना॒मानि॑ - प॒शवः॑ - सीद॒त्वित्या॒ - हेन्द्रा॒येत्या॑हा - र्द्धयति - घ्नन्ति - गृह्णा॒ - त्यहिꣳ॑सायै॒ - पञ्चा॑हा - दि॒त्यव॑ते॒ स्वाहेत्या॑ह - पितृ॒मा - ने॑ति च॒त्वारि॑ च) \textbf{7} \newline \newline
                                \textbf{ T.A.8.8.1} \newline
                  विश्वा॒ आशा॑ दक्षिण॒सदित्या॑ह । विश्वा॑ने॒व दे॒वान् प्री॑णाति । अथो॒ दुरि॑ष्ट्या ए॒वैन॑म् पाति । विश्वा᳚न् दे॒वान॑याडि॒हेत्या॑ह । विश्वा॑ने॒व दे॒वान् भा॑ग॒धेये॑न॒ सम॑र्द्धयति । स्वाहा॑कृतस्य घ॒र्मस्य॒ मधोः᳚ पिबतमश्वि॒नेत्या॑ह ।  अ॒श्विना॑वे॒व भा॑ग॒धेये॑न॒ सम॑र्द्धयति ।  स्वाहा॒ऽग्नये॑ य॒ज्ञिया॑य॒ शंॅयजु॑र्भि॒रित्या॑ह । अ॒भ्ये॑वैन॑म् घारयति । अथो॑ ह॒विरे॒वाकः॑ \textbf{ 70} \newline
                  \newline
                                                                  \textbf{ T.A.8.8.2} \newline
                  अश्वि॑ना घ॒र्मम् पा॑तꣳ हार्दिवा॒न-मह॑र्दि॒वाभि॑-रू॒तिभि॒रित्या॑ह । अ॒श्विना॑वे॒व भा॑ग॒धेये॑न॒ सम॑र्द्धयति । अनु॑ वा॒न् द्यावा॑पृथि॒वी मꣳ॑साता॒मित्या॒हानु॑मत्यै ।  स्वाहेन्द्रा॑य॒ स्वाहेन्द्रा॒ वडित्या॑ह ।  इन्द्रा॑य॒ हि पु॒रो हू॒यते᳚ । आ॒श्राव्या॑ह घ॒र्मस्य॑ य॒जेति॑ । वष॑ट्कृते जुहोति । रक्ष॑सा॒मप॑हत्यै । अनु॑यजति स्व॒गाकृ॑त्यै ।  घ॒र्मम॑पातमश्वि॒नेत्या॑ह \textbf{ 71} \newline
                  \newline
                                                                  \textbf{ T.A.8.8.3} \newline
                  पूर्व॑मे॒वोदि॒तम् । उत्त॑रेणा॒भिगृ॑णाति ।  अनु॑ वा॒न्-द्यावा॑पृथि॒वी अ॑मꣳसाता॒मित्या॒हानु॑मत्यै ।  तम् प्रा॒व्य॑म् ॅयथा॒वण्णमो॑ दि॒वे नमः॑ पृथि॒व्या इत्या॑ह । य॒था॒य॒जुरे॒वैतत् । दि॒वि धा॑ इ॒मं ॅय॒ज्ञ्ं ॅय॒ज्ञ्मि॒मम् दि॒वि धा॒ इत्या॑ह ।  सु॒व॒र्गमे॒वैन॑म् ॅलो॒कम् ग॑मयति ।  दिव॑म् गच्छा॒न्तरि॑क्षम् गच्छ पृथि॒वीम् ग॒च्छेत्या॑ह ।  ए॒ष्वे॑वैन॑म् ॅलो॒केषु॒ प्रति॑ष्ठापयति । पञ्च॑ प्र॒दिशो॑ ग॒च्छेत्या॑ह \textbf{ 72} \newline
                  \newline
                                                                  \textbf{ T.A.8.8.4} \newline
                  दि॒क्ष्वे॑वैन॒म् प्रति॑ष्ठापयति ।  दे॒वान् घ॑र्म॒पान् ग॑च्छ पि॒तॄन् घ॑र्म॒पान् ग॒च्छेत्या॑ह । उ॒भये᳚ष्वे॒वैन॒म् प्रति॑ष्ठापयति । यत् पिन्व॑ते । वर्.षु॑कः प॒र्जन्यो॑ भवति । तस्मा॒त् पिन्व॑मानः॒ पुण्यः॑ । यत् प्राङ्पिन्व॑ते । तद्-दे॒वाना᳚म् । यद्-द॑क्षि॒णा । तत्-पि॑तृ॒णाम् \textbf{ 73} \newline
                  \newline
                                                                  \textbf{ T.A.8.8.5} \newline
                  यत् प्र॒त्यक् । तन्-म॑नु॒ष्या॑णाम् । यदुदङ्ङ्॑ । तद्-रु॒द्राणा᳚म् । प्राञ्च॒मुद॑ञ्चम् पिन्वयति । दे॒व॒त्राऽकः॑ । अथो॒ खलु॑ । सर्वा॒ अनु॒ दिशः॑ पिन्वयति । सर्वा॒ दिशः॒ समे॑धन्ते । अ॒न्तः॒प॒रि॒धि पि॑न्वयति \textbf{ 74} \newline
                  \newline
                                                                  \textbf{ T.A.8.8.6} \newline
                  तेज॒सोऽस्क॑न्दाय । इ॒षे पी॑पिह्यू॒र्जे पी॑पि॒हीत्या॑ह ।  इष॑मे॒वोर्जं॒ ॅयज॑माने दधाति । यज॑मानाय पीपि॒हीत्या॑ह । यज॑मानायै॒वैता-मा॒शिष॒माशा᳚स्ते ।  मह्य॒ञ्ज्यैष्ठ्या॑य पीपि॒हीत्या॑ह ।  आ॒त्मन॑ ए॒वैतामा॒शिष॒माशा᳚स्ते ।  त्विष्यै᳚ त्वा द्यु॒म्नाय॑ त्वेन्द्रि॒याय॑ त्वा॒ भूत्यै॒ त्वेत्या॑ह । य॒था॒य॒जुरे॒वैतत् । धर्मा॑ऽसि सु॒धर्मा मे᳚ न्य॒स्मे ब्रह्मा॑णि धार॒येत्या॑ह \textbf{ 75} \newline
                  \newline
                                                                  \textbf{ T.A.8.8.7} \newline
                  ब्रह्म॑न्ने॒वैन॒म् प्रति॑ष्ठापयति । नेत्त्वा॒ वातः॑ स्क॒न्दया॒दिति॒ यद्य॑भि॒चरे᳚त् । अ॒मुष्य॑ त्वा प्रा॒णे सा॑दयाम्य॒मुना॑ स॒ह नि॑र॒र्थम् ग॒च्छेति॑ ब्रूया॒द्-यम् द्वि॒ष्यात् । यमे॒व द्वेष्टि॑ । तेनै॑नꣳ स॒ह नि॑र॒र्थम् ग॑मयति । पू॒ष्णे शर॑से॒ स्वाहेत्या॑ह । या ए॒व दे॒वता॑ हु॒तभा॑गाः । ताभ्य॑ ए॒वैन॑ञ्जुहोति । ग्राव॑भ्यः॒ स्वाहेत्या॑ह । या ए॒वान्तरि॑क्षे॒ वाचः॑ \textbf{ 76} \newline
                  \newline
                                                                  \textbf{ T.A.8.8.8} \newline
                  ताभ्य॑ ए॒वैन॑ञ्जुहोति । प्र॒ति॒रेभ्यः॒ स्वाहेत्या॑ह । प्रा॒णा वै दे॒वाः प्र॑ति॒राः । तेभ्य॑ ए॒वैन॑ञ्जुहोति । द्यावा॑पृथि॒वीभ्याꣳ॒॒ स्वाहेत्या॑ह ।  द्यावा॑पृथि॒वीभ्या॑-मे॒वैन॑ञ्जुहोति । पि॒तृभ्यो॑ घर्म॒पेभ्यः॒ स्वाहेत्या॑ह । ये वै यज्वा॑नः । ते पि॒तरो॑ घर्म॒पाः । तेभ्य॑ ए॒वैन॑ञ्जुहोति \textbf{ 77} \newline
                  \newline
                                                                  \textbf{ T.A.8.8.9} \newline
                  रु॒द्राय॑ रु॒द्रहो᳚त्रे॒ स्वाहेत्या॑ह । रु॒द्रमे॒व भा॑ग॒धेये॑न॒ सम॑र्द्धयति । स॒र्वतः॒ सम॑नक्ति । स॒र्वत॑ ए॒व रु॒द्रन्नि॒रव॑दयते । उद॑ञ्च॒न्निर॑स्यति ।  ए॒षा वै रु॒द्रस्य॒ दिक् । स्वाया॑मे॒व दि॒शि रु॒द्रन्नि॒रव॑दयते । अ॒प उप॑स्पृशति मेद्ध्य॒त्वाय॑ । नान्वी᳚क्षेत । यद॒न्वीक्षे॑त \textbf{ 78} \newline
                  \newline
                                                                  \textbf{ T.A.8.8.10} \newline
                  चक्षु॑रस्य प्र॒मायु॑कꣳ स्यात् । तस्मा॒न्नान्वीक्ष्यः॑ ।  अपी॑परो॒ माऽह्नो॒ रात्रि॑यै मा पाह्ये॒षा ते॑ अग्ने स॒मित्तया॒ समि॑द्ध्य॒स्वायु॑र्मे दा॒ वर्च॑सा माऽञ्जी॒रित्या॑ह । आयु॑रे॒वास्मि॒न् वर्चो॑ दधाति ।  अपी॑परो मा॒ रात्रि॑या॒ अह्नो॑ मा पाह्ये॒षा ते॑ अग्ने स॒मित्तया॒ समि॑द्ध्य॒स्वायु॑र्मे दा॒ वर्च॑सा माऽञ्जी॒रित्या॑ह । आयु॑रे॒वास्मि॒न् वर्चो॑ दधाति ।  अ॒ग्निर्-ज्योति॒र्-ज्योति॑र॒ग्निः स्वाहा॒ सूर्यो॒ ज्योति॒र्ज्योतिः॒ सूर्यः॒ स्वाहेत्या॑ह । य॒था॒य॒जुरे॒वैतत् । ब्र॒ह्म॒वा॒दिनो॑ वदन्ति । हो॒त॒व्य॑मग्निहो॒त्रा(3) न्न हो॑त॒व्या(3) मिति॑ \textbf{ 79} \newline
                  \newline
                                                                  \textbf{ T.A.8.8.11} \newline
                  यद्-यजु॑षा जुहु॒यात् । अय॑थापूर्व॒माहु॑ती जुहुयात् । यन्न जु॑हु॒यात् । अ॒ग्निः परा॑भवेत् । भूः स्वाहेत्ये॒व हो॑त॒व्य᳚म् । य॒था॒पू॒र्वमाहु॑ती जु॒होति॑ । नाग्निः परा॑भवति । हु॒तꣳ ह॒विर्मधु॑ ह॒विरित्या॑ह ।  स्व॒दय॑त्ये॒वैन᳚म् । इन्द्र॑तमे॒ऽग्नावित्या॑ह \textbf{ 80} \newline
                  \newline
                                                                  \textbf{ T.A.8.8.12} \newline
                  प्रा॒णो वा इन्द्र॑तमो॒ऽग्निः । प्रा॒ण ए॒वैन॒मिन्द्र॑तमे॒ऽग्नौ जु॑होति ।  पि॒ता नो॑ऽसि॒ मा मा॑ हिꣳसी॒रित्या॒हाहिꣳ॑सायै ।  अ॒श्याम॑ ते देव घर्म॒ मधु॑मतो॒ वाज॑वतः पितु॒मत॒ इत्या॑ह । आ॒शिष॑मे॒वैतामाशा᳚स्ते ।  स्व॒धा॒विनो॑ऽशी॒महि॑ त्वा॒ मा मा॑ हिꣳसी॒-रित्या॒हाहिꣳ॑सायै । तेज॑सा॒ वा ए॒ते व्यृ॑द्ध्यन्ते । ये प्र॑व॒र्ग्ये॑ण॒ चर॑न्ति । प्राश्ञ॑न्ति ।  तेज॑ ए॒वात्मन्-द॑धते \textbf{ 81} \newline
                  \newline
                                                                  \textbf{ T.A.8.8.13} \newline
                  स॒म्ॅव॒थ्स॒रन्न माꣳ॒॒सम॑श्ञीयात् । न रा॒मामुपे॑यात् । न मृ॒न्मये॑न पिबेत् । नास्य॑ रा॒म उच्छि॑ष्टम् पिबेत् । तेज॑ ए॒व तथ्सꣳश्य॑ति ।  दे॒वा॒सु॒राः सम्ॅय॑त्ता आसन्न् । ते दे॒वा वि॑ज॒यमु॑प॒यन्तः॑ । वि॒भ्राजि॑ सौ॒र्ये ब्रह्म॒ सन्न्य॑दधत । यत् किञ्च॑ दिवाकी॒र्त्य᳚म् ।  तदे॒तेनै॒व व्र॒तेना॑गोपायत् ( ) । तस्मा॑दे॒तद्-व्र॒तम् चा॒र्य᳚म् । तेज॑सो गोपी॒थाय॑ । तस्मा॑दे॒तानि॒ यजूꣳ॑षि वि॒भ्राजः॑ सौ॒र्यस्येत्या॑हुः ।  स्वाहा᳚ त्वा॒ सूर्य॑स्य र॒श्मिभ्य॒ इति॑ प्रा॒तः सꣳसा॑दयति । स्वाहा᳚ त्वा॒ नक्ष॑त्रेभ्य॒ इति॑ सा॒यम् ।  ए॒ता वा ए॒तस्य॑ दे॒वताः᳚ । ताभि॑रे॒वैनꣳ॒॒ सम॑र्द्धयति । \textbf{ 82} \newline
                  \newline
                                                        (अ॒क॒ - र॒श्वि॒नेत्या॑ह - प्र॒दिशो॑ ग॒च्छेत्या॑ह - पितृ॒णा - म॑न्तः परि॒धि पि॑न्वयति - धार॒येत्या॑ह॒ - वाचो॑ - धर्म॒पा स्तेभ्य॑ ए॒वैन॑म् जुहो - त्य॒न्वीक्षे॑त - होत॒व्या (3) मि - त्य॒ग्रावित्या॑ह - दधते - ऽगोपायथ् स॒प्त च॑) \textbf{8} \newline \newline
                                \textbf{ T.A.8.9.1} \newline
                  घर्म॒ या ते॑ दि॒वि शुगिति॑ ति॒स्र आहु॑तीर्-जुहोति ।  छन्दो॑भिरे॒वास्यै॒भ्यो लो॒केभ्यः॒ शुच॒मव॑ यजते । इय॒त्यग्रे॑ जुहोति । अथेय॒त्यथेय॑ति । त्रय॑ इ॒मे लो॒काः ।  ए॒भ्य ए॒व लो॒केभ्यः॒ शुच॒मव॑ यजते ।  अनु॑ नो॒ऽद्यानु॑मति॒रित्या॒हानु॑मत्यै । दि॒वस्त्वा॑ पर॒स्पाया॒ इत्या॑ह । दि॒व ए॒वेमान् ॅलो॒कान् दा॑धार ।  ब्रह्म॑णस्त्वा पर॒स्पाया॒ इत्या॑ह \textbf{ 83} \newline
                  \newline
                                                                  \textbf{ T.A.8.9.2} \newline
                  ए॒ष्वे॑व लो॒केषु॑ प्र॒जा दा॑धार । प्रा॒णस्य॑ त्वा पर॒स्पाया॒ इत्या॑ह । प्र॒जास्वे॒व प्रा॒णान् दा॑धार । शिरो॒ वा ए॒तद्-य॒ज्ञ्स्य॑ । यत् प्र॑व॒र्ग्यः॑ । अ॒सौ खलु॒ वा आ॑दि॒त्यः प्र॑व॒र्ग्यः॑ । तं ॅयद्-द॑क्षि॒णा प्र॒त्यञ्च॒-मुद॑ञ्च-मुद्वा॒सये᳚त् ।  जि॒ह्मं ॅय॒ज्ञ्स्य॒ शिरो॑ हरेत् ।  प्राञ्च॒मुद्वा॑सयति । पु॒रस्ता॑दे॒व य॒ज्ञ्स्य॒ शिरः॒ प्रति॑दधाति \textbf{ 84} \newline
                  \newline
                                                                  \textbf{ T.A.8.9.3} \newline
                  प्राञ्च॒मुद्वा॑सयति । तस्मा॑द॒सावा॑दि॒त्यः पु॒रस्ता॒दुदे॑ति । श॒फो॒प॒य॒मान्ध॒वित्रा॑णि॒ धृष्टी॒ इत्य॒न्वव॑हरन्ति ।  सात्मा॑नमे॒वैनꣳ॒॒ सत॑नुम् करोति । सात्मा॒ऽमुष्मि॑न् ॅलो॒के भ॑वति ।  य ए॒वं ॅवेद॑ । औदु॑म्बराणि भवन्ति । ऊर्ग्वा उ॑दु॒म्बरः॑ । ऊर्ज॑मे॒वाव॑रुन्धे । वर्त्म॑ना॒ वा अ॒न्वित्य॑ \textbf{ 85} \newline
                  \newline
                                                                  \textbf{ T.A.8.9.4} \newline
                  य॒ज्ञ्ꣳ रक्षाꣳ॑सि जिघाꣳसन्ति । साम्ना᳚ प्रस्तो॒ताऽन्ववै॑ति । साम॒ वै र॑क्षो॒हा । रक्ष॑सा॒मप॑हत्यै । त्रिर्नि॒धन॒मुपै॑ति । त्रय॑ इ॒मे लो॒काः । ए॒भ्य ए॒व लो॒केभ्यो॒ रक्षाꣳ॒॒स्यप॑हन्ति । पुरु॑षःपुरुषो नि॒धन॒मुपै॑ति । पुरु॑षःपुरुषो॒ हि र॑क्ष॒स्वी । रक्ष॑सा॒मप॑हत्यै \textbf{ 86} \newline
                  \newline
                                                                  \textbf{ T.A.8.9.5} \newline
                  यत् पृ॑थि॒व्यामु॑द्वा॒सये᳚त् । पृ॒थि॒वीꣳ शु॒चाऽर्प॑येत् । यद॒फ्सु । अ॒पः शु॒चार्प॑येत् । यदोष॑धीषु । ओष॑धीः शु॒चाऽर्प॑येत् ।  यद्-वन॒स्पति॑षु । वन॒स्पती᳚ञ्छु॒चाऽर्प॑येत् । हिर॑ण्यन्नि॒धायोद्वा॑सयति ।  अ॒मृतं॒ ॅवै हिर॑ण्यम् \textbf{ 87} \newline
                  \newline
                                                                  \textbf{ T.A.8.9.6} \newline
                  अ॒मृत॑ ए॒वैन॒म् प्रति॑ष्ठापयति ।  व॒ल्गुर॑सि शं॒ ॅयुधा॑या॒ इति॒ त्रिः प॑रिषि॒ञ्चन्पर्ये॑ति । त्रि॒वृद्वा अ॒ग्निः । यावा॑ने॒वाग्निः । तस्य॒ शुचꣳ॑ शमयति । त्रिः पुनः॒ पर्ये॑ति । षट्थ् सम्प॑द्यन्ते । षड्वा ऋ॒तवः॑ । ऋ॒तुभि॑रे॒वास्य॒ शुचꣳ॑ शमयति । चतुः॑ स्रक्ति॒र्नाभि॑र्. ऋ॒तस्येत्या॑ह \textbf{ 88} \newline
                  \newline
                                                                  \textbf{ T.A.8.9.7} \newline
                  इ॒यं ॅवा ऋ॒तम् । तस्या॑ ए॒ष ए॒व नाभिः॑ । यत् प्र॑व॒र्ग्यः॑ । तस्मा॑दे॒वमा॑ह । सदो॑ वि॒श्वायु॒रित्या॑ह । सदो॒ हीयम् । अप॒ द्वेषो॒ अप॒ ह्वर॒ इत्या॑ह॒ भ्रातृ॑व्यापनुत्त्यै । घर्मै॒तत्तेऽन्न॑मे॒तत् पुरी॑ष॒मिति॑ द॒द्ध्ना म॑धुमि॒श्रेण॑ पूरयति ।  ऊर्ग्वा अ॒न्नाद्य॒न्दधि॑ । ऊ॒र्जैवैन॑म॒न्नाद्ये॑न॒ सम॑र्द्धयति \textbf{ 89} \newline
                  \newline
                                                                  \textbf{ T.A.8.9.8} \newline
                  अन॑शनायुको भवति । य ए॒वं ॅवेद॑ ।  रन्ति॒र्नामा॑सि दि॒व्यो ग॑न्ध॒र्व इत्या॑ह ।  रू॒पमे॒वास्यै॒-तन्-म॑हि॒मानꣳ॒॒ रन्ति॑म् ब॒न्धुतां॒ ॅव्याच॑ष्टे । सम॒हमायु॑षा॒ सम् प्रा॒णेनेत्या॑ह । आ॒शिष॑मे॒वैतामाशा᳚स्ते ।  व्य॑सौ यो᳚ऽस्मान् द्वेष्टि॒ यञ्च॑ व॒यम् द्वि॒ष्म इत्या॑ह । अ॒भि॒चा॒र ए॒वास्यै॒षः । अचि॑क्रद॒द्-वृषा॒ हरि॒रित्या॑ह । वृषा॒ ह्ये॑षः \textbf{ 90} \newline
                  \newline
                                                                  \textbf{ T.A.8.9.9} \newline
                  वृषा॒ हरिः॑ । म॒हान्-मि॒त्रो न द॑र्.श॒त इत्या॑ह । स्तौत्ये॒वैन॑मे॒तत् । चिद॑सि समु॒द्रयो॑नि॒रित्या॑ह । स्वामे॒वैनं॒ ॅयोनि॑ङ्गमयति ।  नम॑स्ते अस्तु॒ मा मा॑ हिꣳसी॒रित्या॒हाहिꣳ॑सायै ।  वि॒श्वाव॑सुꣳ सोमगन्ध॒र्वमित्या॑ह । यदे॒वास्य॑ क्रि॒यमा॑णस्यान्त॒र्यन्ति॑ । तदे॒वास्यै॒तेना प्या॑ययति ।  वि॒श्वाव॑सुर॒भि तन्नो॑ गृणा॒त्वित्या॑ह \textbf{ 91} \newline
                  \newline
                                                                  \textbf{ T.A.8.9.10} \newline
                  पूर्व॑मे॒वोदि॒तम् । उत्त॑रेणा॒भिगृ॑णाति ।  धियो॑ हिन्वा॒नो धिय॒ इन्नो॑ अव्या॒दित्या॑ह । ऋ॒तूने॒वास्मै॑ कल्पयति । प्रासा᳚म् गन्ध॒र्वो अ॒मृता॑नि वोच॒दित्या॑ह । प्रा॒णा वा अ॒मृताः᳚ ।  प्रा॒णाने॒वास्मै॑ कल्पयति । ए॒तत्त्वन्दे॑व घर्म दे॒वो दे॒वानुपा॑गा॒ इत्या॑ह । दे॒वो ह्ये॑ष सन्दे॒वानु॒पैति॑ । इ॒दम॒हम् म॑नु॒ष्यो॑ मनु॒ष्या॑नित्या॑ह \textbf{ 92} \newline
                  \newline
                                                                  \textbf{ T.A.8.9.11} \newline
                  म॒नु॒ष्यो॑ हि । ए॒ष सन्-म॑नु॒ष्या॑नु॒पैति॑ ।  ई॒श्व॒रो वै प्र॑व॒र्ग्य॑मुद्वा॒सयन्न्॑ । प्र॒जाम् प॒शून्थ् सो॑मपी॒थम॑नू॒द्वासः॒ सोम॑पी॒थानु॒मेहि॑ । स॒ह प्र॒जया॑ स॒ह रा॒यस्पोषे॒णेत्या॑ह । प्र॒जामे॒व प॒शून्थ् सो॑मपी॒थमा॒त्मन् ध॑त्ते । सु॒मि॒त्रा न॒ आप॒ ओष॑धयः स॒न्त्वित्या॑ह । आ॒शिष॑मे॒वैतामाशा᳚स्ते ।  दु॒र्मि॒त्रास्तस्मै॑ भूयासु॒र्-यो᳚ऽस्मान्-द्वेष्टि॒ यञ्च॑ व॒यम् द्वि॒ष्म इत्या॑ह । अ॒भि॒चा॒र ए॒वास्यै॒षः ( ) । प्र वा ए॒षो᳚ऽस्माल् लो॒काच्च्य॑वते ।  यः प्र॑व॒र्ग्य॑मुद्वा॒सय॑ति । उदु॒ त्यम् चि॒त्रमिति॑ सौ॒रीभ्या॑मृ॒ग्भ्याम् पुन॒रेत्य॒ गार्.ह॑पत्ये जुहोति ।  अ॒यम् ॅवै लो॒को गार्.ह॑पत्यः । अ॒स्मिन्ने॒व लो॒के प्रति॑तिष्ठति ।  अ॒सौ खलु॒ वा आ॑दि॒त्यः सु॑व॒र्गो लो॒कः ।  यथ् सौ॒री भव॑तः । तेनै॒व सु॑व॒र्गाल् लो॒कान्नैति॑ । \textbf{ 93} \newline
                  \newline
                                                        (ब्रह्म॑णस्त्वा पर॒स्पाया॒ इत्या॑ह - दधा - त्य॒न्वित्य॑ - रक्ष॒स्वी रक्ष॑सा॒मप॑हत्यै॒ - वै हिर॑ण्य - माहा - र्द्धयति॒ - ह्ये॑ष - गृ॑णा॒त्वित्या॑ह - मनु॒ष्या॑नित्या॑हा - स्यै॒षो᳚ ऽष्टौ च॑) \textbf{9} \newline \newline
                                \textbf{ T.A.8.10.1} \newline
                  प्र॒जाप॑तिं॒ ॅवै दे॒वाः शु॒क्रम् पयो॑ऽदुह्रन्न् । तदे᳚भ्यो॒ न व्य॑भवत् । तद॒ग्निर्व्य॑करोत् । तानि॒ शुक्रि॑याणि॒ सामा᳚न्यभवन्न् । तेषां॒ ॅयो रसो॒ऽत्यक्ष॑रत् । तानि॑ शुक्रय॒जूꣳष्य॑भवन्न् ।  शुक्रि॑याणां॒ ॅवा ए॒तानि॒ शुक्रि॑याणि । सा॒म॒प॒य॒सम्ॅवा ए॒तयो॑र॒न्यत् ।  दे॒वाना॑म॒न्यत्पयः॑ । यद्गोः पयः॑ \textbf{ 94} \newline
                  \newline
                                                                  \textbf{ T.A.8.10.2} \newline
                  तथ् साम्नः॒ पयः॑ । यद॒जायै॒ पयः॑ । तद्-दे॒वाना॒म् पयः॑ । तस्मा॒द्-यत्रै॒तैर्-यजु॑र्भि॒श्चर॑न्ति । तत् पय॑सा चरन्ति । प्र॒जाप॑तिमे॒व तद् दे॒वान्पय॑सा॒ऽन्नाद्ये॑न॒ सम॑र्द्धयन्ति । ए॒ष ह॒ त्वै सा॒क्षात् प्र॑व॒र्ग्य॑म् भक्षयति । यस्यै॒वं ॅवि॒दुषः॑ प्रव॒र्ग्यः॑ प्रवृ॒ज्यते᳚ । उ॒त्त॒र॒वे॒द्या-मुद्वा॑सये॒त्-तेज॑स्कामस्य । तेजो॒ वा उ॑त्तरवे॒दिः \textbf{ 95} \newline
                  \newline
                                                                  \textbf{ T.A.8.10.3} \newline
                  तेजः॑ प्रव॒र्ग्यः॑ । तेज॑सै॒व तेजः॒ सम॑र्द्धयति ।  उ॒त्त॒र॒वे॒द्या-मुद्वा॑सये॒-दन्न॑कामस्य । शिरो॒ वा ए॒तद्-य॒ज्ञ्स्य॑ । यत् प्र॑व॒र्ग्यः॑ । मुख॑मुत्तरवे॒दिः ।  शी॒र्ष्णैव मुखꣳ॒॒ सन्द॑धात्य॒न्नाद्या॑य । अ॒न्ना॒द ए॒व भ॑वति । यत्र॒ खलु॒ वा ए॒तमुद्वा॑सितं॒ ॅवयाꣳ॑सि प॒र्यास॑ते । परि॒ वै ताꣳ समा᳚म् प्र॒जा वयाꣳ॑स्यासते \textbf{ 96} \newline
                  \newline
                                                                  \textbf{ T.A.8.10.4} \newline
                  तस्मा॑-दुत्तरवे॒द्या-मे॒वोद्वा॑सयेत् । प्र॒जाना᳚म् गोपी॒थाय॑ । पु॒रो वा॑ प॒श्चाद्वोद्वा॑सयेत् । पु॒रस्ता॒द्वा ए॒तज्ज्योति॒रुदे॑ति । तत् प॒श्चान्निम्रो॑चति । स्वामे॒वैन॒म् ॅयोनि॒मनूद्वा॑सयति ।  अ॒पाम् मद्ध्य॒ उद्वा॑सयेत् । अ॒पां ॅवा ए॒तन्मद्ध्या॒-ज्ज्योति॑रजायत । ज्योतिः॑ प्रव॒र्ग्यः॑ । स्व ए॒॒वैनं॒ ॅयोनौ॒ प्रति॑ष्ठापयति \textbf{ 97} \newline
                  \newline
                                                                  \textbf{ T.A.8.10.5} \newline
                  यम् द्वि॒ष्यात् । यत्र॒ स स्यात् । तस्या᳚म् दि॒श्युद्वा॑सयेत् ।  ए॒ष वा अ॒ग्निर्-वै᳚श्वान॒रः । यत्-प्र॑व॒र्ग्यः॑ ।  अ॒ग्निनै॒वैनं॑ ॅवैश्वान॒रेणा॒भि प्रव॑र्तयति ।  औदु॑म्बर्याꣳ॒॒ शाखा॑या॒मुद्वा॑सयेत् ।  ऊर्ग्वा उ॑दु॒म्बरः॑ । अन्न॑म् प्रा॒णः । शुग्घ॒र्मः \textbf{ 98} \newline
                  \newline
                                                                  \textbf{ T.A.8.10.6} \newline
                  इ॒दम॒ह-म॒मुष्या॑मुष्याय॒णस्य॑ शु॒चा प्रा॒णमपि॑ दहा॒मीत्या॑ह । शु॒चैवास्य॑ प्रा॒णमपि॑ दहति । ता॒जगार्ति॒मार्च्छ॑ति । यत्र॑ द॒र्भा उ॑प॒दीक॑सन्तताः॒ स्युः । तदुद्वा॑सये॒द्-वृष्टि॑कामस्य । ए॒ता वा अ॒पाम॑नू॒ज्झाव॑र्यो॒ नाम॑ ।  यद्-द॒र्भाः । अ॒सौ खलु॒ वा आ॑दि॒त्य इ॒तो वृष्टि॒मुदी॑रयति । अ॒सावे॒वास्मा॑ आदि॒त्यो वृष्टि॒न्निय॑च्छति । ता आपो॒ निय॑ता॒ धन्व॑ना यन्ति ( ) । \textbf{ 99} \newline
                  \newline
                                                        (गोः पय॑ - उत्तरवे॒दि - रा॑सते - स्थापयति - घ॒र्मो - य॑न्ति) \textbf{10} \newline \newline
                                \textbf{ T.A.8.11.1} \newline
                  प्र॒जाप॑तिः सम्भ्रि॒यमा॑णः । स॒म्राट्थ् सम्भृ॑तः । घ॒र्मः प्रवृ॑क्तः । म॒हा॒वी॒र उद्वा॑सितः । अ॒सौ खलु॒ वावैष आ॑दि॒त्यः । यत् प्र॑व॒र्ग्यः॑ ।  स ए॒तानि॒ नामा᳚न्यकुरुत । य ए॒वं ॅवेद॑ । वि॒दुरे॑न॒न्नाम्ना᳚ । ब्र॒ह्म॒वा॒दिनो॑ वदन्ति \textbf{ 100} \newline
                  \newline
                                                                  \textbf{ T.A.8.11.2} \newline
                  यो वै वसी॑याꣳ सम्ॅयथाना॒ममु॑प॒चर॑ति ।  पुण्या᳚र्तिं॒ ॅवै स तस्मै॑ कामयते । पुण्या᳚र्तिमस्मै कामयन्ते । य ए॒वं ॅवेद॑ । तस्मा॑दे॒वं ॅवि॒द्वान् । घ॒र्म इति॒ दिवाऽऽच॑क्षीत । स॒म्राडिति॒ नक्त᳚म् । ए॒ते वा ए॒तस्य॑ प्रि॒ये त॒नुवौ᳚ । ए॒ते अ॑स्य प्रि॒ये नाम॑नी । प्रि॒ययै॒वैन॑म् त॒नुवा᳚ \textbf{ 101} \newline
                  \newline
                                                                  \textbf{ T.A.8.11.3} \newline
                  प्रि॒येण॒ नाम्ना॒ सम॑र्द्धयति । की॒र्तिर॑स्य॒ पूर्वा ग॑च्छति ज॒नता॑माय॒तः । गा॒य॒त्री दे॒वेभ्योऽपा᳚क्रामत् । तान्दे॒वाः प्र॑व॒र्ग्ये॑णै॒वानु॒ व्य॑भवन्न् ।  प्र॒व॒र्ग्ये॑णाप्नुवन्न् । यच्च॑तुर्विꣳशतिः॒ कृत्वः॑ प्रव॒र्ग्य॑म् प्रवृ॒णक्ति॑ । गा॒य॒त्रीमे॒व तदनु॒ विभ॑वति । गा॒य॒त्रीमा᳚प्नोति । पूर्वा᳚ऽस्य॒ जनं॑ ॅय॒तः की॒र्तिर्ग॑च्छति । वै॒श्व॒दे॒वः सꣳस॑न्नः \textbf{ 102} \newline
                  \newline
                                                                  \textbf{ T.A.8.11.4} \newline
                  वस॑वः॒ प्रवृ॑क्तः । सोमो॑ऽभिकी॒र्यमा॑णः । आ॒श्वि॒नः पय॑स्यानी॒यमा॑ने ।  मारु॒तः क्वथन्न्॑ । पौ॒ष्ण उद॑न्तः । सा॒र॒स्व॒तो वि॒ष्यन्द॑मानः । मै॒त्रः शरो॑ गृहीतः । तेज॒ उद्य॑तो वा॒युः । ह्रि॒यमा॑णः प्र॒जाप॑तिः ।  हू॒यमा॑नो॒ वाग्घु॒तः \textbf{ 103} \newline
                  \newline
                                                                  \textbf{ T.A.8.11.5} \newline
                  अ॒सौ खलु॒ वावैष आ॑दि॒त्यः । यत् प्र॑व॒र्ग्यः॑ ।  स ए॒तानि॒ नामा᳚न्यकुरुत । य ए॒वं ॅवेद॑ । वि॒दुरे॑न॒न्नाम्ना᳚ । ब्र॒ह्म॒वा॒दिनो॑ वदन्ति ।  यन्मृ॒न्मय॒-माहु॑ति॒न्नाश्ञु॒तेऽथ॑ । कस्मा॑दे॒षो᳚ऽश्ञुत॒ इति॑ । वागे॒ष इति॑ ब्रूयात् । वा॒च्ये॑व वाच॑म् दधाति \textbf{ 104} \newline
                  \newline
                                                                  \textbf{ T.A.8.11.6} \newline
                  तस्मा॑दश्ञुते । प्र॒जाप॑ति॒र्वा ए॒ष द्वा॑दश॒धा विहि॑तः । यत् प्र॑व॒र्ग्यः॑ । यत् प्राग॑वका॒शेभ्यः॑ । तेन॑ प्र॒जा अ॑सृजत । अ॒व॒का॒शैर्-दे॑वासु॒रान॑सृजत । यदू॒र्द्ध्वम॑वका॒शेभ्यः॑ । तेनान्न॑मसृजत । अन्न॑म् प्र॒जाप॑तिः । प्र॒जाप॑ति॒र्वावैषः ( ) । \textbf{ 105} \newline
                  \newline
                                                        (व॒द॒न्ति॒ - त॒नुवा॒ - सꣳस॑न्नो - हू॒यमा॑नो॒ वाग्घु॒तो - द॑धा - त्ये॒षः ) \textbf{11} \newline \newline
                                \textbf{ T.A.8.12.1} \newline
                  स॒वि॒ता भू॒त्वा प्र॑थ॒मेऽह॒न् प्रवृ॑ज्यते । तेन॒ कामाꣳ॑ एति । यद्-द्वि॒तीयेऽह॑न्-प्रवृ॒ज्यते᳚ । अ॒ग्निर्-भू॒त्वा दे॒वाने॑ति । यत्-तृ॒तीयेऽह॑न्-प्रवृ॒ज्यते᳚ । वा॒युर्-भू॒त्वा प्रा॒णाने॑ति । यच्च॑तु॒र्थेऽह॑न्-प्रवृ॒ज्यते᳚ । आ॒दि॒त्यो भू॒त्वा र॒श्मीने॑ति । यत् प॑ञ्च॒मेऽह॑न्-प्रवृ॒ज्यते᳚ । च॒न्द्रमा॑ भू॒त्वा नक्ष॑त्राण्येति \textbf{ 106} \newline
                  \newline
                                                                  \textbf{ T.A.8.12.2} \newline
                  यथ्-ष॒ष्ठेऽह॑न्-प्रवृ॒ज्यते᳚ । ऋ॒तुर्-भू॒त्वा स॑म्ॅवथ्स॒रमे॑ति । यथ्-स॑प्त॒मेऽह॑न्-प्रवृ॒ज्यते᳚ । धा॒ता भू॒त्वा शक्व॑रीमेति । यद॑ष्ट॒मेऽह॑न्-प्रवृ॒ज्यते᳚ । बृह॒स्पति॑र्-भू॒त्वा गा॑य॒त्रीमे॑ति ।  यन्-न॑व॒मेऽह॑न्-प्रवृ॒ज्यते᳚ । मि॒त्रो भू॒त्वा त्रि॒वृत॑ इ॒मान् ॅलो॒काने॑ति । यद्-द॑श॒मेऽह॑न्-प्रवृ॒ज्यते᳚ । वरु॑णो भू॒त्वा वि॒राज॑मेति \textbf{ 107} \newline
                  \newline
                                                                  \textbf{ T.A.8.12.3} \newline
                  यदे॑काद॒शेऽह॑न्-प्रवृ॒ज्यते᳚ । इन्द्रो॑ भू॒त्वा त्रि॒ष्टुभ॑मेति । यद्-द्वा॑द॒शेऽह॑न्-प्रवृ॒ज्यते᳚ । सोमो॑ भू॒त्वा सु॒त्यामे॑ति । यत्-पु॒रस्ता॑दुप॒सदा᳚म् प्रवृ॒ज्यते᳚ । तस्मा॑दि॒तः परा॑ङ॒मूं ॅलो॒काꣳस्तप॑न्नेति । यदु॒परि॑ष्टा-दुप॒सदा᳚म् प्रवृ॒ज्यते᳚ । तस्मा॑द॒मुतो॒ऽर्वाङि॒मां ॅलो॒काꣳस्तप॑न्नेति । य ए॒वं ॅवेद॑ ।  ऐव त॑पति ( ) । \textbf{ 108} \newline
                  \newline
                                                        (नक्ष॑त्राण्येति - वि॒राज॑मेति - तपति) \textbf{12} \newline \newline
\textbf{Prapaataka Korvai with starting Padams of 1 to 12 Anuvaakams :-} \newline
दे॒वा वै स॒त्रꣳ - सा॑वि॒त्रम् - परि॑श्रिते॒ - ब्रह्म॒न् प्रच॑रिष्यामो॒ - ऽग्निष्ट्वा॒ - शिरो᳚ ग्री॒वा - दे॒वस्य॑ रश॒नां - ॅविश्वा॒ आशा॒ - धर्म॒ या त᳚ - प्र॒जाप॑तिꣳ शु॒क्रम् - प्र॒जाप॑तिः सम्भ्रि॒यमा॑णः - सवि॒ता भू॒त्वा द्वाद॑श) \newline

\textbf{korvai with starting padams of1, 11, 21 Series of Dasinis :-} \newline
(दे॒वा वै स॒त्रꣳ - स ख॑दि॒रः - परि॑श्रिते - ऽभिपू॒र्व - मथो॒ रक्ष॑सा॒म् - ग्रैष्मा॑वे॒वास्मै॒ - ब्रह्म॒ वै दे॒वाना॒ - मश्वि॑ना घ॒र्मम् पा॑तम् - प्रा॒णो वै - वृषा॒ हरि॒र् - यो वै वसी॑याꣳ सम्ॅयथाना॒म म॒ष्टोत्त॑रश॒तम्) \newline

\textbf{first and last padam in TA, 8th Prapaatakam :-} \newline
(दे॒वा वै स॒त्र - मैव त॑पति) \newline 


====================================== \newline
\pagebreak
\pagebreak
        


\end{document}
