\documentclass[17pt]{extarticle}
\usepackage{babel}
\usepackage{fontspec}
\usepackage{polyglossia}
\usepackage{extsizes}



\setmainlanguage{sanskrit}
\setotherlanguages{english} %% or other languages
\setlength{\parindent}{0pt}
\pagestyle{myheadings}
\newfontfamily\devanagarifont[Script=Devanagari]{AdishilaVedic}


\newcommand{\VAR}[1]{}
\newcommand{\BLOCK}[1]{}




\begin{document}
\begin{titlepage}
    \begin{center}
 
\begin{sanskrit}
    { \Large
    ॐ नमः परमात्मने, श्री महागणपतये नमः
श्री गुरुभ्यो नमः, ह॒रिः॒ ॐ 
    }
    \\
    \vspace{2.5cm}
    \mbox{ \Huge
    कृष्ण यजुर्वेदीय तैत्तिरीय आरण्यके - पञ्चमः प्रपाठकः   }
\end{sanskrit}
\end{center}

\end{titlepage}
\tableofcontents

ॐ नमः परमात्मने, श्री महागणपतये नमः
श्री गुरुभ्यो नमः, ह॒रिः॒ ॐ \newline
5.1     पञ्चमः प्रपाठकः - शीक्षा वल्ली \newline

\addcontentsline{toc}{section}{ 5.1     पञ्चमः प्रपाठकः - शीक्षा वल्ली}
\markright{ 5.1     पञ्चमः प्रपाठकः - शीक्षा वल्ली \hfill https://www.vedavms.in \hfill}
\section*{ 5.1     पञ्चमः प्रपाठकः - शीक्षा वल्ली }
                                \textbf{ T.A.5.1.1} \newline
                  शन्नो॑ मि॒त्रः शं ॅवरु॑णः । शन्नो॑ भवत्वर्य॒मा । शन्न॒ इन्द्रो॒ बृह॒स्पतिः॑ । शन्नो॒ विष्णु॑-रुरुक्र॒मः । नमो॒ ब्रह्म॑णे । नम॑स्ते वायो । त्वमे॒व प्र॒त्यक्षं॒ ब्रह्मा॑सि । त्वमे॒व प्र॒त्यक्षं॒ ब्रह्म॑ वदिष्यामि । ऋ॒तं ॅव॑दिष्यामि । स॒त्यं ॅव॑दिष्यामि ( ) । तन्माम॑वतु । तद्व॒क्तार॑मवतु । अव॑तु॒ मां । अव॑तु व॒क्तारं᳚ ॥ ॐ शान्तिः॒ शान्तिः॒ शान्तिः॑ । \textbf{ 1} \newline
                  \newline
                                                        (स॒त्त्यम् ॅव॑दिष्यामि॒ पञ्च॑ च) \textbf{1} \newline \newline
                                \textbf{ T.A.5.2.1} \newline
                  शीक्षां ॅव्या᳚ख्यास्या॒मः । वर्णः॒ स्वरः । मात्रा॒ बलं । साम॑ सन्ता॒नः । इत्युक्तः शी᳚क्षाद्ध्या॒यः । \textbf{ 2} \newline
                  \newline
                                                        (शीक्षाम् पञ्च॑) \textbf{2} \newline \newline
                                \textbf{ T.A.5.3.1} \newline
                  स॒ह नौ॒ यशः । स॒ह नौ ब्र॑ह्मव॒र्चसं ॥  अथातः सꣳहिताया उपनिषदं ॅव्या᳚ख्यास्या॒मः । पञ्चस्वधि-क॑रणे॒षु । अधिलोकमधि-ज्यौतिष-मधिविद्य-मधिप्रज॑-मद्ध्या॒त्मं । ता महासꣳहिता इ॑त्याच॒क्षते ॥ अथा॑धि लो॒कं । पृथिवी पू᳚र्वरू॒पं । द्यौरुत्त॑ररू॒पं । आका॑शः स॒न्धिः \textbf{ 3} \newline
                  \newline
                                                                  \textbf{ T.A.5.3.2} \newline
                  वायुः॑ सन्धा॒नं । इत्य॑धि लो॒कं ॥  अथा॑धि ज्यौ॒तिषं । अग्निः पू᳚र्व रू॒पं । आदित्य उत्त॑ररू॒पं । आ॑पः स॒न्धिः । वैद्युतः॑ सन्धा॒नं । इत्य॑धि ज्यौ॒तिषं ॥ अथा॑धि वि॒द्यं । आचार्यः पू᳚र्व रू॒पं \textbf{ 4} \newline
                  \newline
                                                                  \textbf{ T.A.5.3.3} \newline
                  अन्तेवास्युत्त॑ररू॒पं । वि॑द्या स॒न्धिः ।  प्रवचनꣳ॑ सन्धा॒नं । इत्य॑धि वि॒द्यं ॥  अथाधि॒प्रजं । माता पू᳚र्वरू॒पं । पितोत्त॑ररू॒पं । प्र॑जा स॒न्धिः । प्रजननꣳ॑ सन्धा॒नं । इत्यधि॒प्रजं । \textbf{ 5} \newline
                  \newline
                                                                  \textbf{ T.A.5.3.4} \newline
                  अथाद्ध्या॒त्मं । अधराहनुः पू᳚र्व रू॒पं । उत्तराहनु-रुत्त॑ररू॒पं । वाख् स॒न्धिः । जिह्वा॑ सन्धा॒नं । इत्यद्ध्या॒त्मं ॥ इती माम॑हा सꣳ॒॒हिताः ॥ य एवमेता महासꣳहिता व्याख्या॑ता वे॒द । सन्धीयते प्रज॑या प॒शुभिः । ब्रह्मवर्च-सेनान्नाद्येन सुवर्ग्येण॑ (सुवर्गेण॑) लोके॒न ( ) । \textbf{ 6} \newline
                  \newline
                                                        (स॒न्धि - राचार्यः पू᳚र्वरू॒प - मित्यधि॒प्रजम् - ॅलो॑के॒न) \textbf{3} \newline \newline
                                \textbf{ T.A.5.4.1} \newline
                  य श्छन्द॑सा-मृष॒भो वि॒श्वरू॑पः । छन्दो॒भ्यो-ऽद्ध्य॒मृता᳚थ् संब॒भूव॑ । स मेन्द्रो॑ मे॒धया᳚ स्पृणोतु । अ॒मृत॑स्य देव॒ धार॑णो भूयासं ॥  शरी॑रं मे॒ विच॑र्.षणं । जि॒ह्वा मे॒ मधु॑मत्तमा । कर्णा᳚भ्यां॒ भूरि॒ विश्रु॑वं । ब्रह्म॑णः को॒शो॑ऽसि मे॒धया ऽपि॑हितः । श्रु॒तं मे॑ गोपाय ॥ आ॒वह॑न्ती वितन्वा॒ना \textbf{ 7} \newline
                  \newline
                                                                  \textbf{ T.A.5.4.2} \newline
                  कु॒र्वा॒णा चीर॑-मा॒त्मनः॑ । वासाꣳ॑सि॒ मम॒ गाव॑श्च । अ॒न्न॒पा॒ने च॑ सर्व॒दा । ततो॑ मे॒ श्रिय॒-माव॑ह । लो॒म॒शां प॒शुभिः॑ स॒ह स्वाहा᳚ ॥ आमा॑यन्तु ब्रह्मचा॒रिणः॒ स्वाहा᳚ ।  विमा॑ऽऽयन्तु ब्रह्मचा॒रिणः॒ स्वाहा᳚ । प्रमा॑ऽऽयन्तु ब्रह्मचा॒रिणः॒ स्वाहा᳚ । दमा॑यन्तु ब्रह्मचा॒रिणः॒ स्वाहा᳚ । शमा॑यन्तु ब्रह्मचा॒रिणः॒ स्वाहा᳚ । \textbf{ 8} \newline
                  \newline
                                                                  \textbf{ T.A.5.4.3} \newline
                  यशो॒ जने॑ऽसानि॒ स्वाहा᳚ । श्रेया॒न॒. वस्य॑सोऽसानि॒ स्वाहा᳚ ॥ तं त्वा॑ भग॒ प्रवि॑शानि॒ स्वाहा᳚ । स मा॑ भग॒ प्रवि॑श॒ स्वाहा᳚ । तस्मि᳚नथ् स॒हस्र॑ शाखे । निभ॑गा॒हं त्वयि॑ मृजे॒ स्वाहा᳚ ॥ यथाऽऽपः॒ प्रव॑ता॒ऽऽयन्ति॑ । यथा॒ मासा॑ अहर्ज॒रं ।  ए॒वं मां ब्र॑ह्मचा॒रिणः॑ ।  धात॒राय॑न्तु स॒र्वतः॒ स्वाहा᳚ ( ) ॥  प्र॒ति॒वे॒शो॑ऽसि॒ प्र मा॑ भाहि॒ प्र मा॑ पद्यस्व । \textbf{ 9} \newline
                  \newline
                                                        (वि॒त॒न्वा॒ना - शमा॑यन्तु ब्रह्मचा॒रिणः॒ स्वाहा॒-धात॒राय॑न्तु स॒र्वतः॒ स्वाहैक॑म् च) \textbf{4} \newline \newline
                                \textbf{ T.A.5.5.1} \newline
                  भू-र्भुव॒-स्सुव॒-रिति॒ वा ए॒ता स्ति॒स्रो व्याहृ॑तयः ॥  तासा॑मुहस्मै॒ तां च॑तु॒र्त्थीं । माहा॑चमस्यः॒ प्रवे॑दयते । मह॒ इति॑ ॥ तद् ब्रह्म॑ । स आ॒त्मा । अङ्गा᳚न्य॒न्या दे॒वताः᳚ ॥ भूरिति॒ वा अ॒यं ॅलो॒कः । भुव॒ इत्य॒न्तरि॑क्षं । सुव॒-रित्य॒सौ लो॒कः \textbf{ 10} \newline
                  \newline
                                                                  \textbf{ T.A.5.5.2} \newline
                  मह॒ इत्या॑दि॒त्यः । आ॒दि॒त्येन॒ वाव सर्वे॑ लो॒का मही॑यन्ते ॥ भूरिति॒ वा अ॒ग्निः । भुव॒ इति॑ वा॒युः । सुव॒रित्या॑दि॒त्यः । मह॒ इति॑ च॒न्द्रमाः᳚ । च॒न्द्रम॑सा॒ वाव सर्वा॑णि॒ ज्योतीꣳ॑षि॒ मही॑यन्ते ॥ भू-रिति॒ वा ऋचः॑ । भुव॒ इति॒ सामा॑नि ।  सुव॒ रिति॒ यजूꣳ॑षि \textbf{ 11} \newline
                  \newline
                                                                  \textbf{ T.A.5.5.3} \newline
                  मह॒ इति॒ ब्रह्म॑ । ब्रह्म॑णा॒ वाव सर्वे॑ वे॒दा मही॑यन्ते ॥ भूरिति॒ वै प्रा॒णः । भुव॒ इत्य॑पा॒नः । सुव॒रिति॑ व्या॒नः । मह॒ इत्यन्नं᳚ । अन्ने॑न॒ वाव सर्वे᳚ प्रा॒णा मही॑यन्ते ॥ ता वा ए॒ता-श्चत॑स्र श्चतु॒र्द्धा । चत॑स्र-श्चतस्रो॒ व्याहृ॑तयः । ता यो वेद॑ ( ) ।  स वे॑द॒ ब्रह्म॑ । सर्वे᳚ऽस्मै दे॒वा ब॒लिमाव॑हन्ति । \textbf{ 12} \newline
                  \newline
                                                        (अ॒सौ लो॒को - यजूꣳ॑षि॒ - वेद॒ द्वे च॑) \textbf{5} \newline \newline
                                \textbf{ T.A.5.6.1} \newline
                  स य ए॒षो᳚ऽन्तर्.-हृ॑दय आका॒शः । तस्मि॑न्न॒यं पुरु॑षो मनो॒मयः॑ । अमृ॑तो हिर॒ण्मयः॑ ॥ अन्त॑रेण॒ तालु॑के । य ए॒ष स्तन॑ इवाव॒लंब॑ते ।  से᳚न्द्र यो॒निः । यत्रा॒सौ के॑शा॒न्तो वि॒वर्त॑ते । व्य॒पोह्य॑ शीर्.षकपा॒ले ॥ भूरित्य॒ग्नौ प्रति॑तिष्ठति । भुव॒ इति॑ वा॒यौ \textbf{ 13} \newline
                  \newline
                                                                  \textbf{ T.A.5.6.2} \newline
                  सुव॒रित्या॑दि॒त्ये । मह॒ इति॒ ब्रह्म॑णि । आ॒प्नोति॒ स्वारा᳚ज्यं । आ॒प्नोति॒ मन॑स॒स्पतिं᳚ । वाक्प॑ति॒ श्चक्षु॑ष्पतिः । श्रोत्र॑पति र्वि॒ज्ञान॑पतिः ।  ए॒तत् ततो॑ भवति । आ॒का॒श श॑रीरं॒ ब्रह्म॑ ।  स॒त्यात्म॑ प्रा॒णारा॑मं॒ मन॑ आनन्दं । शान्ति॑ समृद्ध-म॒मृतं᳚ ( ) ॥  इति॑ प्राचीन यो॒ग्योपा᳚स्व । \textbf{ 14} \newline
                  \newline
                                                        (वा॒या-व॒मृत॒ मेक॑म् च) \textbf{6} \newline \newline
                                \textbf{ T.A.5.7.1} \newline
                  पृ॒थि॒व्य॑न्तरि॑क्षं॒ द्यौर्-दिशो॑ऽवान्तरदि॒शाः । अ॒ग्निर्-वा॒यु-रा॑दि॒त्य-श्च॒न्द्रमा॒ नक्ष॑त्राणि ।  आप॒ ओष॑धयो॒ वन॒स्पत॑य आका॒श आ॒त्मा । इत्य॑धिभू॒तं ॥ अथाद्ध्या॒त्मं । प्रा॒णो व्या॒नो॑ऽपा॒न उ॑दा॒नः स॑मा॒नः । चक्षुः॒ श्रोत्रं॒ मनो॒ वाक् त्वक् ।  चर्म॑ माꣳ॒॒ सꣳस्नावास्थि॑ म॒ज्जा ॥  ए॒त द॑धिवि॒धाय॒र्॒.षि॒ रवो॑चत् ।  पाङ्क्तं॒ ॅवा इ॒दꣳ सर्वं᳚ ( ) ।  पाङ्क्त॑नै॒व पाङ्क्तꣳ॑ स्पृणो॒तीति॑ । \textbf{ 15} \newline
                  \newline
                                                        (सर्व॒मेक॑म् च) \textbf{7} \newline \newline
                                \textbf{ T.A.5.8.1} \newline
                  ओ-मिति॒ ब्रह्म॑ ॥ ओ-मिती॒दꣳ सर्वं᳚ ॥  ओमित्ये॒त-द॑नुकृति हस्म॒ वा अ॒प्यो श्रा॑व॒येत्या-श्रा॑वयन्ति । ओ-मिति॒ सामा॑नि गायन्ति । ओꣳ शोमिति॑ श॒स्त्राणि॑ शꣳसन्ति । ओमित्य॑द्ध्व॒र्युः प्र॑तिग॒रं प्रति॑गृणाति । ओमिति॒ ब्रह्मा॒ प्रसौ॑ति ।  ओमित्य॑ग्निहो॒त्र-मनु॑जानाति । ओमिति॑ ब्राह्म॒णः प्र॑व॒क्ष्यन्ना॑ह॒ ब्रह्मोपा᳚प्नवा॒नीति॑ ।  ब्रह्मै॒वो-पा᳚प्नोति । \textbf{ 16} \newline
                  \newline
                                                        (ओम् दश॑) \textbf{8} \newline \newline
                                \textbf{ T.A.5.9.1} \newline
                  ऋतं च स्वाद्ध्याय प्रव॑चने॒ च ।  सत्यं च स्वाद्ध्याय प्रव॑चने॒ च ।  तपश्च स्वाद्ध्याय प्रव॑चने॒ च ।  दमश्च स्वाद्ध्याय प्रव॑चने॒ च ।  शमश्च स्वाद्ध्याय प्रव॑चने॒ च ।  अग्नयश्च स्वाद्ध्याय प्रव॑चने॒ च ।  अग्निहोत्रं च स्वाद्ध्याय प्रव॑चने॒ च ।  अतिथयश्च स्वाद्ध्याय प्रव॑चने॒ च ।  मानुषं च स्वाद्ध्याय प्रव॑चने॒ च ।  प्रजा च स्वाद्ध्याय प्रव॑चने॒ च ( ) ।  प्रजनश्च स्वाद्ध्याय प्रव॑चने॒ च ।  प्रजातिश्च स्वाद्ध्याय प्रव॑चने॒ च ॥  सत्यमिति सत्यवचा॑ राथी॒ तरः ।  तप इति तपोनित्यः पौ॑रुशि॒ष्टिः ।  स्वाद्ध्याय प्रवचने एवेति नाको॑ मौद्ग॒ल्यः ।  तद्धि तप॑स्तद्धि॒ तपः । \textbf{ 17} \newline
                  \newline
                                                        (प्रजा च स्वाद्ध्यायप्रव॑चने॒ च षट्च॑) \textbf{9} \newline \newline
                                \textbf{ T.A.5.10.1} \newline
                  अ॒हं ॅवृ॒क्षस्य॒ रेरि॑वा ।  की॒र्तिः पृ॒ष्ठं गि॒रेरि॑व ।  ऊ॒र्द्ध्व प॑वित्रो वा॒जिनी॑व स्व॒मृत॑मस्मि । द्रवि॑णꣳ॒॒ सव॑र्चसं । सुमेधा अ॑मृतो॒क्षितः । इति त्रिशङ्कोर्-वेदा॑नुव॒चनं । \textbf{ 18} \newline
                  \newline
                                                        (अ॒हꣳ षट्) \textbf{10} \newline \newline
                                \textbf{ T.A.5.11.1} \newline
                  वेदमनूच्या- ऽचार्योन्तेवासिन-म॑नुशा॒स्ति ॥  सत्यं॒ ॅवद । धर्मं॒ चर ॥ स्वाद्ध्याया᳚न्मा प्र॒मदः । आचार्याय प्रियं धनमाहृत्य प्रजातन्तुं मा व्य॑वच्छे॒थ्सीः । सत्यान्न प्रम॑दित॒व्यं । धर्मान्न प्रम॑दित॒व्यं । कुशलान्न प्रम॑दित॒व्यं ।  भूत्यै न प्रम॑दित॒व्यं । स्वाद्ध्याय प्रवचनाभ्यां न प्रम॑दित॒व्यं \textbf{ 19} \newline
                  \newline
                                                                  \textbf{ T.A.5.11.2} \newline
                  देवपितृकार्याभ्यां न प्रम॑दित॒व्यं ॥  मातृ॑देवो॒ भव । पितृ॑देवो॒ भव ।  आचार्य॑ देवो॒ भव । अतिथि॑देवो॒ भव ॥  यान्यनवद्यानि॑ कर्मा॒णि । तानि सेवि॑तव्या॒नि ।  नो इ॑तरा॒णि ॥ यान्यस्माकꣳ सुच॑रिता॒नि । तानि त्वयो॑पास्या॒नि \textbf{ 20} \newline
                  \newline
                                                                  \textbf{ T.A.5.11.3} \newline
                  नो इ॑तरा॒णि ॥ ये के चास्म-च्छ्रेयाꣳ॑सो ब्रा॒ह्मणाः । तेषां त्वयाऽऽसने न प्रश्व॑सित॒व्यं ॥ श्रद्ध॑या दे॒यं । अश्रद्ध॑याऽदे॒यं । श्रि॑या दे॒यं । ह्रि॑या दे॒यं । भि॑या दे॒यं । सम्ॅवि॑दा दे॒यं ॥  अथ यदि ते कर्मविचिकिथ्सा वा वृत्त विचिकि॑थ्सा वा॒ स्यात् \textbf{ 21} \newline
                  \newline
                                                                  \textbf{ T.A.5.11.4} \newline
                  ये तत्र ब्राह्मणाः᳚ सम्म॒र॒.शिनः । युक्ता॑ आयु॒क्ताः ।  अलूक्षा॑ धर्म॑कामाः॒ स्युः । यथा ते॑ तत्र॑ वर्ते॒रन्न् । तथा तत्र॑ वर्ते॒थाः ॥ अथाभ्या᳚ख्याते॒षु । ये तत्र ब्राह्मणाः᳚ सम्म॒र्॒.शिनः । युक्ता॑ आयु॒क्ताः ।  अलूक्षा॑ धर्म॑ कामाः॒ स्युः । यथा ते॑ तेषु॑ वर्ते॒रन्न् ( ) \textbf{ 22} \newline
                  \newline
                                                                  \textbf{ T.A.5.11.5} \newline
                  तथा तेषु॑ वर्ते॒थाः ॥ एष॑ आदे॒शः ।  एष उ॑पदे॒शः । एषा वे॑दोप॒निषत् । एतद॑नुशा॒सनं । एव मुपा॑सित॒व्यं । एव मुचैत॑दुपा॒स्यं । \textbf{ 23} \newline
                  \newline
                                                        (स्वाद्ध्यायप्रवचनाभ्याम् न प्रम॑दित॒व्यम् - तानित्वयो॑पास्या॒नि - स्यात् - तेषु॑ वर्ते॒रन्थ् - +स॒प्त च॑) \textbf{11} \newline \newline
                                \textbf{ T.A.5.12.1} \newline
                  शन्नो॑ मि॒त्रः शं ॅवरु॑णः । शन्नो॑ भवत्वर्य॒मा ।  शन्न॒ इन्द्रो॒ बृह॒स्पतिः॑ । शन्नो॒ विष्णु॑ रुरुक्र॒मः । नमो॒ ब्रह्म॑णे । नम॑स्ते वायो ।  त्वमे॒व प्र॒त्यक्षं॒ ब्रह्मा॑सि ।  त्वामे॒व प्र॒त्यक्षं॒ ब्रह्मा वा॑दिषं ।  ऋ॒तम॑वादिषं । स॒त्यम॑वादिषं ( ) । तन्मामा॑वीत् । तद्व॒क्तार॑मावीत् । आवी॒न्मां ।  आवी᳚द्व॒क्तारं᳚ ॥ ॐ शान्तिः॒ शान्तिः॒ शान्तिः॑ । \textbf{ 24} \newline
                  \newline
                                                        (स॒त्यम॑वादिष॒म् पञ्च॑ च) \textbf{12} \newline \newline
                                \textbf{ T.A.5.13.1} \newline
                  स॒ह ना॑ ववतु । स॒ह नौ॑ भुनक्तु । स॒ह वी॒र्यं॑ करवावहै । ते॒ज॒स्विना॒ वधी॑तमस्तु॒ मा वि॑द्विषा॒वहै᳚ ॥ ॐ शान्तिः॒ शान्तिः॒ शान्तिः॑ । \textbf{ 25} \newline
                  \newline
                                                        (स॒ह पञ्च॑) \textbf{13} \newline \newline
                                \textbf{ T.A.5.14.1} \newline
                  ब्र॒ह्म॒विदा᳚प्नोति॒ परं᳚ ॥  तदे॒षाऽभ्यु॑क्ता । स॒त्यं ज्ञा॒न-म॑न॒न्तं ब्रह्म॑ ।  यो वेद॒ निहि॑तं॒ गुहा॑यां पर॒मे व्यो॑मन्न् ।  सो᳚ऽश्नुते॒ सर्वा॒न् कामा᳚न्थ् स॒ह । ब्रह्म॑णा विप॒श्चितेति॑ ॥  तस्मा॒द्वा ए॒तस्मा॑दा॒त्मन॑ आका॒शः संभू॑तः । आ॒का॒शाद् वा॒युः । वा॒योर॒ग्निः । अ॒ग्नेरापः॑ । अ॒द्भ्यः पृ॑थि॒वी । पृ॒थि॒व्या ओष॑धयः ।  ओष॑धी॒भ्योऽन्नं᳚ । अन्ना॒त् पुरु॑षः ॥  स वा एष पुरुषोऽन्न॑रस॒मयः ॥ तस्येद॑मेव॒ शिरः । अयं दक्षि॑णः प॒क्षः । अयमुत्त॑रः प॒क्षः । अयमात्मा᳚ । इदं पुच्छं॑ प्रति॒ष्ठा ॥ तदप्येष श्लो॑को भ॒वति । \textbf{ 26} \newline
                  \newline
                                                                  \textbf{ T.A.5.14.2} \newline
                  अन्ना॒द्वै प्र॒जाः प्र॒जाय॑न्ते । याः काश्च॑ पृथि॒वीꣳ श्रि॒ताः । अथो॒ अन्ने॑नै॒व जी॑वन्ति । अथै॑न॒दपि॑यन्त्यन्त॒तः । अन्नꣳ॒॒ हि भू॒तानां॒ ज्येष्ठं᳚ । तस्मा᳚थ् सर्वौष॒ध मु॑च्यते ।  सर्वं॒ ॅवै तेऽन्न॑माप्नुवन्ति । येऽन्नं॒ ब्रह्मो॒पास॑ते । अन्नꣳ॒॒ हि भू॒तानां॒ ज्येष्ठं᳚ । तस्मा᳚थ् सर्वौष॒ध मु॑च्यते । अन्ना᳚द् भू॒तानि॒ जाय॑न्ते । जाता॒न्यन्ने॑न वर्द्धन्ते ।  अद्यतेऽत्ति च॑ भूता॒नि । तस्मादन्नं तदुच्य॑त इ॒ति ॥  तस्माद्वा एतस्मा-दन्न॑रस॒मयात् ।  अन्योऽन्तर आत्मा᳚ प्राण॒ मयः । तेनै॑ष पू॒र्णः ॥  स वा एष पुरुषवि॑ध ए॒व । तस्य पुरु॑ष वि॒धतां ।  अन्वयं॑ पुरुष॒ विधः । तस्य प्राण॑ एव॒ शिरः ॥  व्यानो दक्षि॑णः प॒क्षः । अपान उत्त॑रः प॒क्षः ।  आका॑श आ॒त्मा । पृथिवी पुच्छं॑ प्रति॒ष्ठा ॥  तदप्येष श्लो॑को भ॒वति । \textbf{ 27} \newline
                  \newline
                                                                  \textbf{ T.A.5.14.3} \newline
                  प्रा॒णं दे॒वा अनु॒प्राण॑न्ति । म॒नु॒ष्याः᳚ प॒शव॑श्च॒ ये । प्रा॒णो हि भू॒ताना॒मायुः॑ । तस्मा᳚थ् सर्वायु॒ष मु॑च्यते । सर्व॑मे॒व त॒ आयु॑र्यन्ति ।  ये प्रा॒णं ब्रह्मो॒पास॑ते ।  प्राणो हि भूता॑नामा॒युः । तस्माथ् सर्वायुष-मुच्य॑त इ॒ति ॥ तस्यैष एव शारी॑र आ॒त्मा । यः॑ पूर्व॒स्य ॥  तस्माद्वा एतस्मा᳚त् प्राण॒मयात् । अन्योऽन्तर आत्मा॑ मनो॒मयः । तेनै॑ष पू॒र्णः ॥ स वा एष पुरुषवि॑ध ए॒व । तस्य पुरु॑षवि॒धतां ।  अन्वयं॑ पुरुष॒विधः । तस्य यजु॑रेव॒ शिरः । ऋग् दक्षि॑णः प॒क्षः । सामोत्त॑रः प॒क्षः । आदे॑श आ॒त्मा । अथर्वाङ्गिरसः पुच्छं॑ प्रति॒ष्ठा ॥  तदप्येष श्लो॑को भ॒वति । \textbf{ 28} \newline
                  \newline
                                                                  \textbf{ T.A.5.14.4} \newline
                  यतो॒ वाचो॒ निव॑र्तन्ते । अप्रा᳚प्य॒ मन॑सा स॒ह । आनन्दं ब्रह्म॑णो वि॒द्वान् । न बिभेति कदा॑चने॒ति ॥ तस्यैष एव शारी॑र आ॒त्मा । यः॑ पूर्व॒स्य ॥  तस्माद्वा एतस्मा᳚न् मनो॒मयात् । अन्योऽन्तर आत्मा वि॑ज्ञान॒मयः । तेनै॑ष पू॒र्णः ॥ स वा एष पुरुषवि॑ध ए॒व । तस्य पुरु॑ष वि॒धतां ।  अन्वयं॑ पुरुष॒ विधः । तस्य श्र॑द्धैव॒ शिरः । ऋतं दक्षि॑णः प॒क्षः । सत्यमुत्त॑रः प॒क्षः । यो॑ग आ॒त्मा । महः पुच्छं॑ प्रति॒ष्ठा ॥ तदप्येष श्लो॑को भ॒वति । \textbf{ 29} \newline
                  \newline
                                                                  \textbf{ T.A.5.14.5} \newline
                  वि॒ज्ञानं॑ ॅय॒ज्ञ्ं त॑नुते । कर्मा॑णि तनु॒तेऽपि॑ च । वि॒ज्ञानं॑ दे॒वाः सर्वे᳚ । ब्रह्म॒ ज्येष्ठ॒-मुपा॑सते । वि॒ज्ञानं॒ ब्रह्म॒ चेद् वेद॑ । तस्मा॒च्चेन्न प्र॒माद्य॑ति । श॒रीरे॑ पाप्म॑नो हि॒त्वा । सर्वान्-कामान्थ्-समश्नु॑त इ॒ति ॥ तस्यैष एव शारी॑र आ॒त्मा । यः॑ पूर्व॒स्य ॥ तस्माद्वा एतस्माद्-वि॑ज्ञान॒मयात् । अन्योऽन्तर आत्मा॑ऽऽनन्द॒मयः । तेनै॑ष पू॒र्णः ॥ स वा एष पुरुषवि॑ध ए॒व । तस्य पुरु॑षवि॒धतां ।  अन्वयं॑ पुरुष॒विधः । तस्य प्रिय॑-मेव॒ शिरः ।  मोदो दक्षि॑णः प॒क्षः ।  प्रमोद उत्त॑रः प॒क्षः । आन॑न्द आ॒त्मा ।  ब्रह्म पुच्छं॑ प्रति॒ष्ठा ॥  तदप्येष श्लो॑को भ॒वति । \textbf{ 30} \newline
                  \newline
                                                                  \textbf{ T.A.5.14.6} \newline
                  अस॑न्ने॒व स॑ भवति । अस॒द् ब्रह्मेति॒ वेद॒ चेत् ।  अस्ति ब्रह्मेति॑ चेद् वे॒द । सन्तमेनं ततो वि॑दुरि॒ति ॥ तस्यैष एव शारी॑र आ॒त्मा । यः॑ पूर्व॒स्य ॥ अथातो॑ऽनु प्र॒श्नाः ॥ उ॒ता वि॒द्वा न॒मुं ॅलो॒कं प्रेत्य॑ । कश्च॒न ग॑च्छ॒ती(3) । आहो॑ वि॒द्वान॒मुं ॅलो॒कं प्रेत्य॑ । कश्चि॒त् सम॑श्नु॒ता(3) उ॒ ॥ सो॑ऽकामयत ।  ब॒हुस्यां॒ प्रजा॑ये॒येति॑ । स तपो॑ऽतप्यत । स तप॑स्त॒प्त्वा । इ॒दꣳ सर्व॑-मसृजत ॥ यदि॒दं किञ्च॑ । तथ्सृ॒ष्ट्वा । तदे॒वानु॒ प्रावि॑शत् ॥ तद॑नुप्र॒विश्य॑ । सच्च॒ त्यच्चा॑भवत् । नि॒रुक्तं॒ चा नि॑रुक्तं च ।  नि॒लय॑नं॒ चा नि॑लयनं च । वि॒ज्ञानं॒ चा वि॑ज्ञानं च । सत्यं चानृतं च स॑त्य-म॒भवत् ॥ यदि॑दं कि॒ञ्च ।  तथ्सत्य-मि॑त्या च॒क्षते ॥ तदप्येष श्लो॑को भ॒वति । \textbf{ 31} \newline
                  \newline
                                                                  \textbf{ T.A.5.14.7} \newline
                  अस॒द्वा इ॒दमग्र॑ आसीत् । ततो॒ वै सद॑जायत ।  तदात्मानꣳ स्वय॑-मकु॒रुत । तस्मात् तथ् सुकृतमुच्य॑त इ॒ति ॥ यद्वै॑ तथ् सु॒कृतं । र॑सो वै॒ सः । रसꣳ ह्येवायं ॅलब्ध्वाऽऽन॑न्दी भ॒वति ॥ को ह्येवान्या᳚त् कः प्रा॒ण्यात् । यदेष आकाश आन॑न्दो न॒ स्यात् । एष ह्येवान॑न्दया॒ति ॥ य॒दा ह्ये॑वैष॒ एतस्मिन्-नदृश्ये ऽनात्म्ये ऽनिरुक्ते ऽनिलयने ऽभयं  प्रति॑ष्ठां ॅवि॒न्दते । अथ सोऽभयं ग॑तो भ॒वति ॥  य॒दा ह्ये॑वैष॒ एतस्मिन्-नुदर-मन्त॑रं कु॒रुते ।  अथ तस्य भ॑यं भ॒वति ॥ तत्त्वेव भयं ॅविदुषो-ऽम॑न्वान॒स्य ॥  तदप्येष श्लो॑को भ॒वति । \textbf{ 32} \newline
                  \newline
                                                                  \textbf{ T.A.5.14.8} \newline
                  भी॒षाऽस्मा॒-द्वातः॑ पवते । भी॒षोदे॑ति॒ सूर्यः॑ ।  भीषाऽस्मा-दग्नि॑-श्चेन्द्र॒श्च । मृत्युर् धावति पञ्च॑म इ॒ति ॥ सैषाऽऽनन्दस्य मीमाꣳ॑सा भ॒वति ॥ युवा स्याथ् साधु यु॑वाऽद्ध्या॒यकः । आशिष्ठो दृढिष्ठो॑ बलि॒ष्ठः । तस्येयं पृथिवी सर्वा वित्तस्य॑ पूर्णा॒ स्यात् ।  स एको मानुष॑ आन॒न्दः ॥ ते ये शतं मानुषा॑ आन॒न्दाः \textbf{ 8-4} \newline
                  \newline
                                                                  \textbf{ T.A.5.14.9} \newline
                  यतो॒ वाचो॒ निव॑र्तन्ते । अप्रा᳚प्य॒ मन॑सा स॒ह । आनन्दं ब्रह्म॑णो वि॒द्वान् । न बिभेति कुत॑श्चने॒ति ॥ एतꣳ ह वा व॑ न त॒पति । किमहꣳ साधु॑ नाक॒रवं । किमहं पाप-मकर॑व-मि॒ति ॥  स य एवं ॅविद्वानेते आत्मा॑नꣳ स्पृ॒णुते ॥  उ॒भे ह्ये॑वैष॒ एते आत्मा॑नꣳ स्पृ॒णुते ।  य ए॒वं ॅवेद॑ ॥ इत्यु॑प॒निष॑त् । \textbf{ 34} \newline
                  \newline
                                                        (ब्र॒ह्म॒विदिदमयमिदमेक॑विꣳशतिः ।  अन्ना॒ दन्न॑रस॒मयात् प्राणो॒ व्यानोऽपान आका॑शः॒ पृथिवी पुच्छꣳ॒॒ षड्विꣳ॑शतिः ।  प्रा॒णम् प्रा॑ण॒मयान् म॑नो॒ यजु॒र्.॒ ऋख् सामादे॒शोऽथर्वाङ्गिरसः पुच्छ॒म् द्वाविꣳ॑शतिः ।  यतः॑ श्र॒द्धर्तꣳ सत्यम् ॅयो॑गो॒ महो॑ऽष्टाद॑श । वि॒ज्ञान॒म् प्रिय॒म् मोदः प्रमोद आन॑न्दो॒ ब्रह्म पुच्छ॒म् द्वाविꣳ॑शतिः ।  अस॑न्ने॒वाथाष्टावि ꣳ॑शतिः । अस॒थ् षोड॑श ।  भी॒षाऽस्मा॒न् मानुषो॒ मनुष्यगन्धर्वाणा॒म् देवगन्धर्वाणा॒म् पितृणाम्चिरलोकलोकाना॒ माजानजानाम् कर्मदेवानाम् ॅये कर्मणा देवाना॒मिन्द्र॑स्य॒ बृहस्पतेः॒ प्रजापते॒र् ब्रह्मणः॒ स यश्व॑ संक्रा॒मत्येक॑पञ्चा॒शत् ।  यतः॒ कुत॑श्च॒ नैतमेका॑दश॒ नव॑) \textbf{14} \newline \newline
                                \textbf{ T.A.5.15.1} \newline
                  भृगु॒र्वै वा॑रु॒णिः । वरु॑णं॒ पित॑र॒-मुप॑ससार । अधी॑हि भगवो॒ ब्रह्मेति॑ ॥ तस्मा॑ ए॒तत् प्रो॑वाच । अन्नं॑ प्रा॒णं चक्षुः॒ श्रोत्रं॒ मनो॒ वाच॒मिति॑ ॥  तꣳ हो॑वाच । यतो॒ वा इ॒मानि॒ भूता॑नि॒ जाय॑न्ते ।  येन॒ जाता॑नि॒ जीव॑न्ति । यत् प्रय॑न्त्य॒भि सम्ॅवि॑शन्ति ।  तद् विजि॑ज्ञासस्व । तद् ब्रह्मेति॑ ॥  स तपो॑ऽतप्यत । स तप॑-स्त॒प्त्वा । \textbf{ 35} \newline
                  \newline
                                                                  \textbf{ T.A.5.15.2} \newline
                  अन्नं॒ ब्रह्मेति॒ व्य॑जानात् । अ॒न्नाद्ध्ये॑व खल्वि॒मानि॒ भूता॑नि॒ जाय॑न्ते । अन्ने॑न॒ जाता॑नि॒ जीव॑न्ति । अन्नं॒ प्रय॑न्त्य॒भि-सम्ॅवि॑श॒न्तीति॑ ॥ तद् वि॒ज्ञाय॑ । पुन॑रे॒व वरु॑णं॒ पित॑र॒-मुप॑ससार । अधी॑हि भगवो॒ ब्रह्मेति॑ । तꣳ हो॑वाच । तप॑सा॒ ब्रह्म॒ विजि॑ज्ञासस्व ।  तपो॒ ब्रह्मेति॑ । स तपो॑ऽतप्यत । स तप॑-स्त॒प्त्वा । \textbf{ 36} \newline
                  \newline
                                                                  \textbf{ T.A.5.15.3} \newline
                  प्रा॒णो ब्र॒ह्मेति॒ व्य॑जानात् । प्रा॒णाद्ध्ये॑व खल्वि॒मानि॒ भूता॑नि॒ जाय॑न्ते । प्रा॒णेन॒ जाता॑नि॒ जीव॑न्ति ।  प्रा॒णं प्रय॑न्त्य॒भि-सम्ॅवि॑श॒न्तीति॑ ॥ तद् वि॒ज्ञाय॑ ।  पुन॑रे॒व वरु॑णं॒ पित॑र॒-मुप॑ससार । अधी॑हि भगवो॒ ब्रह्मेति॑ । तꣳ हो॑वाच । तप॑सा॒ ब्रह्म॒ विजि॑ज्ञासस्व । तपो॒ ब्रह्मेति॑ । स तपो॑ऽतप्यत । स तप॑-स्त॒प्त्वा । \textbf{ 37} \newline
                  \newline
                                                                  \textbf{ T.A.5.15.4} \newline
                  मनो॒ ब्रह्मेति॒ व्य॑जानात् । मन॑सो॒ ह्ये॑व खल्वि॒मानि॒ भूता॑नि॒ जाय॑न्ते । मन॑सा॒ जाता॑नि॒ जीव॑न्ति । मनः॒ प्रय॑न्त्य॒भि-सम्ॅवि॑श॒न्तीति॑ ॥ तद् वि॒ज्ञाय॑ । पुन॑रे॒व वरु॑णं॒ पित॑र॒-मुप॑ससार । अधी॑हि भगवो॒ ब्रह्मेति॑ । तꣳ हो॑वाच । तप॑सा॒ ब्रह्म॒ विजि॑ज्ञासस्व ।  तपो॒ ब्रह्मेति॑ । स तपो॑ऽतप्यत । स तप॑-स्त॒प्त्वा । \textbf{ 38} \newline
                  \newline
                                                                  \textbf{ T.A.5.15.5} \newline
                  वि॒ज्ञानं॒ ब्रह्मेति॒ व्य॑जानात् ।  वि॒ज्ञाना॒-द्ध्ये॑व खल्वि॒मानि॒ भूता॑नि॒ जाय॑न्ते ।  वि॒ज्ञाने॑न॒ जाता॑नि॒ जीव॑न्ति । वि॒ज्ञानं॒ प्रय॑न्त्य॒भि-सम्ॅवि॑श॒न्तीति॑ ॥  तद् वि॒ज्ञाय॑ । पुन॑रे॒व वरु॑णं॒ पित॑र॒-मुप॑ससार । अधी॑हि भगवो॒ ब्रह्मेति॑ । तꣳ हो॑वाच । तप॑सा॒ ब्रह्म॒ विजि॑ज्ञासस्व । तपो॒ ब्रह्मेति॑ । स तपो॑ऽतप्यत । स तप॑-स्त॒प्त्वा । \textbf{ 39} \newline
                  \newline
                                                                  \textbf{ T.A.5.15.6} \newline
                  आ॒न॒न्दो ब्र॒ह्मेति॒ व्य॑जानात् । आ॒नन्दा॒-द्ध्ये॑व-खल्वि॒मानि॒ भूता॑नि॒ जाय॑न्ते ।  आ॒न॒न्देन॒ जाता॑नि॒ जीव॑न्ति । आ॒न॒न्दं प्रय॑न्त्य॒भि-सम्ॅवि॑श॒न्तीति॑ ॥ सैषा भा᳚र्ग॒वी वा॑रु॒णी वि॒द्या । प॒र॒मे व्यो॑म॒न् प्रति॑ष्ठिता ॥ य ए॒वं ॅवेद॒ प्रति॑तिष्ठति । अन्न॑वानन्ना॒दो भ॑वति ।  म॒हान् भ॑वति प्र॒जया॑ प॒शुभि॑र् ब्रह्मवर्च॒सेन॑ ।  म॒हान् की॒र्त्या । \textbf{ 40} \newline
                  \newline
                                                                  \textbf{ T.A.5.15.7} \newline
                  अन्नं॒ न नि॑न्द्यात् । तद् व्र॒तं ॥ प्रा॒णो वा अन्नं᳚ । शरी॑रमन्ना॒दं । प्रा॒णे शरी॑रं॒ प्रति॑ष्ठितं । शरी॑रे प्रा॒णः प्रति॑ष्ठितः ।  तदे॒त-दन्न॒-मन्ने॒ प्रति॑ष्ठितं ॥  स य ए॒त-दन्न॒-मन्ने॒ प्रति॑ष्ठितं॒ ॅवेद॒ प्रति॑तिष्ठति । अन्न॑वा-नन्ना॒दो भ॑वति ।  म॒हान् भ॑वति प्र॒जया॑ प॒शुभि॑र्-ब्रह्मवर्च॒सेन॑ ।  म॒हान् की॒र्त्या । \textbf{ 41} \newline
                  \newline
                                                                  \textbf{ T.A.5.15.8} \newline
                  अन्नं॒ न परि॑चक्षीत । तद् व्र॒तं । आपो॒ वा अन्नं᳚ । ज्योति॑रन्ना॒दं । अ॒फ्सु ज्योतिः॒ प्रति॑ष्ठितं । ज्योति॒ष्यापः॒ प्रति॑ष्ठिताः ।  तदे॒त-दन्न॒-मन्ने॒ प्रति॑ष्ठितं ॥  स य ए॒त-दन्न॒-मन्ने॒ प्रति॑ष्ठितं॒ ॅवेद॒ प्रति॑तिष्ठति । अन्न॑वा-नन्ना॒दो भ॑वति । म॒हान् भ॑वति प्र॒जया॑ प॒शुभि॑र् ब्रह्मवर्च॒सेन॑ ।  म॒हान् कीर्त्या । \textbf{ 42} \newline
                  \newline
                                                                  \textbf{ T.A.5.15.9} \newline
                  अन्नं॑ ब॒हु कु॑र्वीत । तद् व्र॒तं ॥ पृ॒थि॒वी वा अन्नं᳚ । आ॒का॒शो᳚ऽन्ना॒दः । पृ॒थि॒व्या-मा॑का॒शः प्रति॑ष्ठितः । आ॒का॒शे पृ॑थि॒वी प्रति॑ष्ठिता ।  तदे॒त-दन्न॒-मन्ने॒ प्रति॑ष्ठितं ॥  स य ए॒त-दन्न॒-मन्ने॒ प्रति॑ष्ठितं॒ ॅवेद॒ प्रति॑तिष्ठति । अन्न॑वानन्ना॒दो भ॑वति । म॒हान् भ॑वति प्र॒जया॑ प॒शुभि॑र् ब्रह्मवर्च॒सेन॑ ।  म॒हान् की॒र्त्या । \textbf{ 43} \newline
                  \newline
                                                                  \textbf{ } \newline
                   \textbf{ 0} \newline
                  \newline
                                                        (भृगु॒स्त्रयो॑द॒शा - न्न॑म् - प्रा॒णो - मनो॑ - वि॒ज्ञान॒म् द्वाद॑श द्वादशा - न॒न्दो दशा - न्न॒म् न नि॑न्द्या॒ - दन्न॒म् न परि॑चक्षी॒ता -न्न॑म् ब॒हुकु॑र्वि॒ तैका॑दशैकादश॒ - न कंच नैक॑षष्टि॒र्दश॑ ) \textbf{15} \newline \newline
\textbf{Prapaataka Korvai with starting Padams of 1 to 15 Anuvaakams :-} \newline
(शम् नः॒ - शीक्षाꣳ - स॒ह नौ॒ - य ॑007आ;छन्द॑सा॒म् - भूः - स यः - पृ॑थि॒ - व्योमि - त्यृतम् चा॒ - हम् - ॅवेदमनुच्य - शम् नः॑ - स॒ह ना॑ववतु - ब्रह्म॒विद् - भृगुः॒ पञ्च॑दश) \newline

\textbf{korvai with starting padams of1, 11, 21 Series of Dasinis :-} \newline
(शम् नो॒ - मह॒ इत्या॑दि॒त्यो - नो इ॑तरा॒ण्य - स॑न्ने॒वा - न्न॒म् न नि॑न्द्या॒च्चतु॑श्चत्वारिꣳशत् ) \newline

\textbf{first and last padam in TA, 5th Prapaatakam :-} \newline
(शन्न॒ - इत्यु॑प॒निष॑त् ) \newline 


॥ कृष्ण यजुर्वेदीय तैत्तिरीय ब्राह्मणे आरण्यके पञ्चमः प्रपाठकः समाप्तः ॥
========================== \newline
\pagebreak
\pagebreak
        


\end{document}
