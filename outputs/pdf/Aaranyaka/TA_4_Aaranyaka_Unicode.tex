\documentclass[17pt]{extarticle}
\usepackage{babel}
\usepackage{fontspec}
\usepackage{polyglossia}
\usepackage{extsizes}



\setmainlanguage{sanskrit}
\setotherlanguages{english} %% or other languages
\setlength{\parindent}{0pt}
\pagestyle{myheadings}
\newfontfamily\devanagarifont[Script=Devanagari]{AdishilaVedic}


\newcommand{\VAR}[1]{}
\newcommand{\BLOCK}[1]{}




\begin{document}
\begin{titlepage}
    \begin{center}
 
\begin{sanskrit}
    { \Large
    ॐ नमः परमात्मने, श्री महागणपतये नमः
श्री गुरुभ्यो नमः, ह॒रिः॒ ॐ 
    }
    \\
    \vspace{2.5cm}
    \mbox{ \Huge
    कृष्ण यजुर्वेदीय तैत्तिरीय आरण्यके प्रथमः प्रपाठकः   }
\end{sanskrit}
\end{center}

\end{titlepage}
\tableofcontents

ॐ नमः परमात्मने, श्री महागणपतये नमः
श्री गुरुभ्यो नमः, ह॒रिः॒ ॐ \newline
4.1     चतुर्थः प्रपाठकः - पितृमेधमन्त्राः \newline

\addcontentsline{toc}{section}{ 4.1     चतुर्थः प्रपाठकः - पितृमेधमन्त्राः}
\markright{ 4.1     चतुर्थः प्रपाठकः - पितृमेधमन्त्राः \hfill https://www.vedavms.in \hfill}
\section*{ 4.1     चतुर्थः प्रपाठकः - पितृमेधमन्त्राः }
                                \textbf{ T.A.4.1.1} \newline
                  प॒रे॒यु॒वाꣳस॑म् प्र॒वतो॑ म॒हीरनु॑ ब॒हुभ्यः॒ पन्था॑मनपस्पशा॒नम् । वै॒व॒स्व॒तꣳ स॒ङ्गम॑न॒म् जना॑नाम् ॅय॒मꣳ राजा॑नꣳ ह॒विषा॑ दुवस्यत ॥ इ॒दम् त्वा॒ वस्त्र॑म् प्रथ॒मं न्वाग॒न्नपै॒तदू॑ह॒ यदि॒हाबि॑भः पु॒रा ।  इ॒ष्टा॒पू॒र्तमनु॒ संप॑श्य॒ दक्षि॑णा॒म् ॅयथा॑ ते द॒त्तम् ब॑हु॒धा वि ब॑न्धुषु ॥  इ॒मौ यु॑नज्मि ते व॒ह्नी असु॑नीथाय वो॒ढवे᳚ ।  याभ्या᳚म् ॅय॒मस्य॒ साद॑नꣳ सु॒कृता॒म् चापि॑ गच्छतात् ॥ पू॒षा त्वे॒तश्च्या॑वयतु॒ प्रवि॒द्वा-नन॑ष्टपशु॒र्-भुव॑नस्य गो॒पाः ।  स त्वै॒तेभ्यः॒ परि॑ददात् पि॒तृभ्यो॒ऽग्निर्दे॒वेभ्यः॑ सुवि॒दत्रे᳚भ्यः ॥ पू॒षेमा आशा॒ अनु॑वेद॒ सर्वाः॒ सो अ॒स्माꣳ अभ॑यतमेन नेषत् ।  स्व॒स्ति॒दा अघृ॑णिः॒ सर्व॑वी॒रो-ऽप्र॑युच्छन्पु॒र ए॑तु॒ प्रवि॒द्वान् । \textbf{ 1} \newline
                  \newline
                                                                  \textbf{ T.A.4.1.2} \newline
                  आयु॑र्-वि॒श्वायुः॒ परि॑पासति त्वा पू॒षा त्वा॑ पातु॒ प्रप॑थे पु॒रस्ता᳚त् । यत्रास॑ते सु॒कृतो॒ यत्र॒ ते य॒युस्तत्र॑ त्वा दे॒वः स॑वि॒ता द॑धातु ॥  भुव॑नस्य पत इ॒दꣳ ह॒विः ॥ अ॒ग्नये॑ रयि॒मते॒ स्वाहा᳚ ॥ पुरु॑षस्य सयाव॒र्यपेद॒घानि॑ मृज्महे ।  यथा॑ नो॒ अत्र॒ नाप॑रः पु॒रा ज॒रस॒ आय॑ति ॥  पुरु॑षस्य सयावरि॒ वि ते᳚ प्रा॒णम॑सि स्रसम् ।  शरी॑रेण म॒हीमिहि॑ स्व॒धयेहि॑ पि॒तॄनुप॑ प्र॒जया॒ऽस्मानि॒हाव॑ह ॥ मैव॑म् माꣳ॒॒स्ता प्रि॑ये॒ऽहम् दे॒वी स॒ती पि॑तृलो॒कम् ॅयदैषि॑ ।  वि॒श्ववा॑रा॒ नभ॑सा॒ सम्ॅव्य॑यन्त्यु॒भौ नो॑ लो॒कौ  पय॑सा॒ऽभ्याव॑वृथ्स्व । \textbf{ 2} \newline
                  \newline
                                                                  \textbf{ T.A.4.1.3} \newline
                  इ॒यम् नारी॑ पतिलो॒कम् ॅवृ॑णा॒ना निप॑द्यत॒ उप॑ त्वा मर्त्य॒ प्रेत᳚म् । विश्व॑म् पुरा॒णमनु॑ पा॒लय॑न्ती॒ तस्यै᳚ प्र॒जाम् द्रवि॑णम् चे॒ह धे॑हि ॥ उदी᳚र्ष्व नार्य॒भि जी॑वलो॒कमि॒तासु॑मे॒तमुप॑शेष॒ एहि॑ ।  ह॒स्त॒ग्रा॒भस्य॑ दिधि॒षोस्त्वमे॒तत् पत्यु॑र्जनि॒त्वम॒भि संब॑भूव ॥  सु॒वर्णꣳ॒॒ हस्ता॑दा॒ददा॑ना मृ॒तस्य॑ श्रि॒यै ब्रह्म॑णे॒ तेज॑से॒ बला॑य । अत्रै॒व त्वमि॒ह व॒यꣳ सु॒शेवा॒ विश्वाः॒ स्पृधो॑ अ॒भिमा॑तीर्जयेम ॥  धनु॒र्॒. हस्ता॑दा॒ददा॑ना मृ॒तस्य॑ श्रि॒यै क्ष॒त्रायौज॑से॒ बला॑य । अत्रै॒व त्वमि॒ह व॒यꣳ सु॒शेवा॒ विश्वाः॒ स्पृधो॑ अ॒भिमा॑तीर्जयेम ।  मणिꣳ॒॒ हस्ता॑दा॒ददा॑ना मृ॒तस्य॑ श्रि॒यै वि॒शे पुष्ट्यै॒ बला॑य । अत्रै॒व त्वमि॒ह व॒यꣳ सु॒शेवा॒ विश्वाः॒ स्पृधो॑ अ॒भिमा॑तीर्जयेम । \textbf{ 3} \newline
                  \newline
                                                                  \textbf{ T.A.4.1.4} \newline
                  इ॒मम॑ग्ने चम॒सम् मा विजी᳚ह्वरः प्रि॒यो दे॒वाना॑मु॒त सो॒म्याना᳚म् । ए॒ष यश्च॑म॒सो दे॑व॒पान॒स्तस्मि॑न् दे॒वा अ॒मृता॑ मादयन्ताम् ॥  अ॒ग्नेर्वर्म॒ परि॒ गोभि॑र्व्ययस्व॒ संप्रोर्णु॑ष्व॒ मेद॑सा॒ पीव॑सा च । नेत्त्वा॑ धृ॒ष्णुर्. हर॑सा॒ जर्.हृ॑षाणो॒ दध॑द्-विध॒क्ष्यन्-पर्य॒ङ्खया॑तै ॥  मैन॑मग्ने॒ विद॑हो॒ माऽभिशो॑चो॒ माऽस्य॒ त्वच॑म् चिक्षिपो॒ मा शरी॑रम् । य॒दा शृ॒तम् क॒रवो॑ जातवे॒दोऽथे॑मेन॒म् प्रहि॑णुतात् पि॒तृभ्यः॑ ॥  शृ॒तम् ॅय॒दा क॒रसि॑ जातवे॒दोऽथे॑मेन॒म् परि॑दत्तात् पि॒तृभ्यः॑ । य॒दा गच्छा॒त्यसु॑नीतिमे॒तामथा॑ दे॒वाना᳚म् ॅवश॒नीर्भ॑वाति ॥ सूर्य॑म् ते॒ चक्षु॑र्-गच्छतु॒ वात॑मा॒त्मा द्याम् च॒ गच्छ॑ पृथि॒वीम् च॒ धर्म॑णा ।  अ॒पो वा॑ गच्छ॒ यदि॒ तत्र॑ ते हि॒तमोष॑धीषु॒ प्रति॑तिष्ठा॒ शरी॑रैः ( ) ॥ अ॒जो भा॒गस्तप॑सा॒ तम् त॑पस्व॒ तम् ते॑ शो॒चिस्त॑पतु॒ तम् ते॑ अ॒र्चिः ।  यास्ते॑ शि॒वास्त॒नुवो॑ जातवेद॒स्ताभि॑र्-वहे॒मꣳ सु॒कृता॒म् ॅयत्र॑ लो॒काः ॥ अ॒यम् ॅवै त्वम॒स्मादधि॒ त्वमे॒तद॒यम् ॅवै तद॑स्य॒ योनि॑रसि । वै॒श्वा॒न॒रः पु॒त्रः पि॒त्रे लो॑क॒कृ-ज्जा॑तवेदो॒ वहे॑मꣳ सु॒कृता॒म् ॅयत्र॑ लो॒काः । \textbf{ 4} \newline
                  \newline
                                                        (वि॒द्वा-न॒भ्याव॑वृथ्स्वा॒ - भिमा॑तीर्जयेम॒ - शरी᳚रैश्च॒त्वारि॑ च ) \textbf{1} \newline \newline
                                \textbf{ T.A.4.2.1} \newline
                  य ए॒तस्य॑ प॒थो गो॒प्तार॒स्तेभ्यः॒ स्वाहा॒ य ए॒तस्य॑ प॒थो र॑क्षि॒तार॒स्तेभ्यः॒ स्वाहा॒ य ए॒तस्य॑ प॒थो॑ऽभिर॑क्षि॒तार॒स्तेभ्यः॒ स्वाहा᳚ऽऽख्या॒त्रे स्वाहा॑ऽपाख्या॒त्रे स्वाहा॑ऽभि॒लाल॑पते॒ स्वाहा॑ऽप॒लाल॑पते॒ स्वाहा॒ऽग्नये॑ कर्म॒कृते॒ स्वाहा॒ यमत्र॒ नाधी॒मस्तस्मै॒ स्वाहा᳚ ॥  यस्त॑ इ॒द्ध्मम् ज॒भर॑थ् सिष्विदा॒नो मू॒र्द्धान॑म् ॅवा त॒तप॑ते त्वा॒या । दिवो॒ विश्व॑स्माथ् सीमघाय॒त उ॑रुष्यः ॥ अ॒स्मात्त्वमधि॑ जा॒तो॑ऽसि॒ त्वद॒यम् जा॑यता॒म् पुनः॑ । अ॒ग्नये॑ वैश्वान॒राय॑ सुव॒र्गाय॑ लो॒काय॒ स्वाहा᳚ । \textbf{ 5} \newline
                  \newline
                                                        (य ए॒तस्य॒ त्वत् पञ्च॑ ) \textbf{2} \newline \newline
                                \textbf{ T.A.4.3.1} \newline
                  प्र के॒तुना॑ बृह॒ता भा᳚त्य॒ग्निरा॒विर्-विश्वा॑नि वृष॒भो रो॑रवीति । दि॒वश्चि॒दन्ता॒दुप॒ मामु॒दान॑ड॒पामु॒पस्थे॑ महि॒षो व॑वर्द्ध ॥  इ॒दंत॒ एक॑म् प॒र ऊ॑त॒ एक॑म् तृ॒तीये॑न॒ ज्योति॑षा॒ सम्ॅवि॑शस्व । स॒म्ॅवेश॑नस्त॒नुवै॒ चारु॑रेधि प्रि॒यो दे॒वाना᳚म् पर॒मे स॒धस्थे᳚ ॥  नाके॑ सुप॒र्णमुप॒ यत् पत॑न्तꣳ हृ॒दा वेन॑न्तो अ॒भ्यच॑क्षत त्वा । हिर॑ण्यपक्ष॒म् ॅवरु॑णस्य दू॒तम् ॅय॒मस्य॒ योनौ॑ शकु॒नम् भु॑र॒ण्युम् ॥  अति॑द्रव सारमे॒यौ श्वानौ॑ चतुर॒क्षौ श॒बलौ॑ सा॒धुना॑ प॒था । अथा॑ पि॒तॄन्थ् सु॑वि॒दत्राꣳ॒॒ अपी॑हि य॒मेन॒ ये स॑ध॒माद॒म् मद॑न्ति ॥  यौ ते॒ श्वानौ॑ यम रक्षि॒तारौ॑ चतुर॒क्षौ प॑थि॒रक्षी॑ नृ॒चक्ष॑सा ।  ताभ्याꣳ॑ राज॒न् परि॑ देह्येनꣳ स्व॒स्ति चा᳚स्मा अनमी॒वम् च॑ धेहि । \textbf{ 6} \newline
                  \newline
                                                                  \textbf{ T.A.4.3.2} \newline
                  उ॒रु॒ण॒सा-व॑सु॒तृपा॑-वुलुम्ब॒लौ य॒मस्य॑ दू॒तौ च॑रतो॒ वशाꣳ॒॒ अनु॑ । ताव॒स्मभ्य॑म् दृ॒शये॒ सूर्या॑य॒ पुन॑र्दत् ता॒वसु॑म॒द्येह भ॒द्रम् ॥  सोम॒ एके᳚भ्यः पवते घृ॒तमेक॒ उपा॑सते ।  येभ्यो॒ मधु॑ प्र॒धाव॑ति॒ ताꣳ-श्चि॑दे॒वापि॑गच्छतात् ॥  ये युद्ध्य॑न्ते प्र॒धने॑षु॒ शूरा॑सो॒ ये त॑नु॒त्यजः॑ । ये वा॑ स॒हस्र॑दक्षिणा॒स्ताꣳ श्चि॑दे॒वापि॑गच्छतात् ॥  तप॑सा॒ ये अ॑नाधृ॒ष्या-स्तप॑सा॒ ये सुव॑र्ग॒ताः ।  तपो॒ ये च॑क्रि॒रे म॒हत् ताꣳ-श्चि॑दे॒वापि॑गच्छतात् ॥  अश्म॑न्वती रेवतीः॒ सꣳर॑भद्ध्व॒-मुत्ति॑ष्ठत॒ प्रत॑रता सखायः । अत्रा॑ जहाम॒ ये अस॒न्नशे॑वाः शि॒वान्. व॒यम॒भि वाजा॒नुत्त॑रेम । \textbf{ 7} \newline
                  \newline
                                                                  \textbf{ T.A.4.3.3} \newline
                  यद्वै दे॒वस्य॑ सवि॒तुः प॒वित्रꣳ॑ स॒हस्र॑धार॒म् ॅवित॑तम॒न्तरि॑क्षे । येनापु॑ना॒दिन्द्र॒मना᳚र्त॒-मार्त्यै॒ तेना॒हम् माꣳ स॒र्वत॑नुम् पुनामि ॥  या रा॒ष्ट्रात् प॒न्नादप॒ यन्ति॒ शाखा॑ अ॒भिमृ॑ता नृ॒पति॑मि॒च्छमा॑नाः । धा॒तुस्ताः सर्वाः॒ पव॑नेन पू॒ताः प्र॒जया॒ऽस्मान्-र॒य्या वर्च॑सा॒ सꣳसृ॑जाथ ॥  उद्व॒यम् तम॑स॒स्परि॒ पश्य॑न्तो॒ ज्योति॒रुत्त॑रम् ।  दे॒वम् दे॑व॒त्रा सूर्य॒मग॑न्म॒ ज्योति॑रुत्त॒मम् ॥  धा॒ता पु॑नातु सवि॒ता पु॑नातु । अ॒ग्नेस्तेज॑सा॒ सूर्य॑स्य॒ वर्च॑सा । \textbf{ 8} \newline
                  \newline
                                                        (धे॒ - ह्युत्त॑रे - मा॒ष्टौ च॑) \textbf{3} \newline \newline
                                \textbf{ T.A.4.4.1} \newline
                  यम् ते॑ अ॒ग्निमम॑न्थाम वृष॒भाये॑व॒ पक्त॑वे ।  इ॒मम् तꣳ श॑मयामसि क्षी॒रेण॑ चोद॒केन॑ च ॥  यन्त्वम॑ग्ने स॒मद॑ह॒स्त्वमु॒ निर्वा॑पया॒ पुनः॑ ।  क्या॒म्बूरत्र॑ जायताम् पाकदू॒र्वा व्य॑ल्कशा ॥  शीति॑के॒ शीति॑कावति॒ ह्लादु॑के॒ ह्लादु॑कावति ।  म॒ण्डू॒क्या॑ सुसङ्ग॒मये॒मꣳ स्व॑ग्निꣳ श॒मय॑ ॥ शम् ते॑ धन्व॒न्या आपः॒ शमु॑ ते सन्त्वनू॒क्याः᳚ । शम् ते॑ समु॒द्रिया॒ आपः॒ शमु॑ ते सन्तु॒ वर्ष्याः᳚ ॥  शम् ते॒ स्रव॑न्ती-स्त॒नुवे॒ शमु॑ ते सन्तु॒ कूप्याः᳚ ।  शम् ते॑ नीहा॒रो व॑र्.षतु॒ शमु॒ पृष्वाऽव॑शीयताम् । \textbf{ 9} \newline
                  \newline
                                                                  \textbf{ T.A.4.4.2} \newline
                  अव॑ सृज॒ पुन॑रग्ने पि॒तृभ्यो॒ यस्त॒ आहु॑त॒श्चर॑ति स्व॒धाभिः॑ । आयु॒र्वसा॑न॒ उप॑ यातु॒ शेषꣳ॒॒ सङ्ग॑च्छताम् त॒नुवा॑ जातवेदः ॥  सङ्ग॑च्छस्व पि॒तृभिः॒ सꣳ स्व॒धाभिः॒ समि॑ष्टापू॒र्तेन॑ पर॒मे व्यो॑मन्न् ।  यत्र॒ भूम्यै॑ वृ॒णसे॒ तत्र॑ गच्छ॒ तत्र॑ त्वा दे॒वः स॑वि॒ता द॑धातु ॥ यत्ते॑ कृ॒ष्णः श॑कु॒न आ॑तु॒तोद॑ पिपी॒लः स॒र्प उ॒त वा॒ श्वाप॑दः ।  अ॒ग्निष्टद्-विश्वा॑दनृ॒णम् कृ॑णोतु॒ सोम॑श्च॒ यो ब्रा᳚ह्म॒णमा॑वि॒वेश॑ ॥ उत्ति॒ष्ठात॑-स्त॒नुवꣳ॒॒ सम्भ॑रस्व॒ मेह गात्र॒मव॑हा॒ मा शरी॑रम् ।  यत्र॒ भूम्यै॑ वृ॒णसे॒ तत्र॑ गच्छ॒ तत्र॑ त्वा दे॒वः स॑वि॒ता द॑धातु ॥ इ॒दम् त॒ एक॑म् प॒र ऊ॑त॒ एक॑म् तृ॒तीये॑न॒ ज्योति॑षा॒ सम्ॅवि॑शस्व ।  स॒म्ॅवेश॑न-स्त॒नुवै॒ चारु॑रेधि प्रि॒यो दे॒वाना᳚म् पर॒मे स॒धस्थे᳚ ( ) ॥ उत्ति॑ष्ठ॒ प्रेहि॒ प्रद्र॒वौकः॑ कृणुष्व पर॒मे व्यो॑मन्न् । य॒मेन॒ त्वम् ॅय॒म्या॑ सम्ॅविदा॒नोत्-त॒मम् नाक॒मधि॑ रोहे॒मम् ॥  "अश्म॑न्वती रेवती॒र्{36}" "यद्वै दे॒वस्य॑ सवि॒तुः प॒वित्र॒म्{37}" "ॅया रा॒ष्ट्रात् प॒न्ना{38}" "दुद्व॒यम् तम॑स॒स्परि॑{39}" "धा॒ता पु॑नातु{40}" ॥  अ॒स्मात्त्वमधि॑ जा॒तो᳚ऽस्य॒यम् त्वदधि॑जायताम् ।  अ॒ग्नये॑ वैश्वान॒राय॑ सुव॒र्गाय॑ लो॒काय॒ स्वाहा᳚ । \textbf{ 10} \newline
                  \newline
                                                        (अव॑शीयताꣳ - स॒धस्थे॒ पञ्च॑ च) \textbf{4} \newline \newline
                                \textbf{ T.A.4.5.1} \newline
                  आया॑तु दे॒वः सु॒मना॑भि-रू॒तिभि॑र्-य॒मो ह॑वे॒ह प्रय॑ताभिर॒क्ता । आसी॑दताꣳ सुप्र॒यते॑ह ब॒र्॒.हिष्यूर्जा॑य जा॒त्यै मम॑ शत्रु॒हत्यै᳚ ॥  य॒मे इ॑व॒ यत॑माने॒ यदैत॒म् प्रवा᳚म्भर॒न् मानु॑षा देव॒यन्तः॑ । आसी॑दतꣳ॒॒ स्वमु॑ लो॒कम् ॅविदा॑ने स्वास॒स्थे भ॑वत॒मिन्द॑वे नः ॥  य॒माय॒ सोमꣳ॑ सुनुत य॒माय॑ जुहुता ह॒विः ।  य॒मꣳ ह॑ य॒ज्ञो ग॑च्छत्य॒ग्निदू॑तो॒ अर॑ङ्कृतः ॥  य॒माय॑ घृ॒तव॑द्ध॒विर्जु॒होत॒ प्र च॑ तिष्ठत ।  स नो॑ दे॒वेष्वाय॑मद्-दी॒र्घमायुः॒ प्र जी॒वसे᳚ ॥  य॒माय॒ मधु॑मत्तमꣳ॒॒ राज्ञे॑ ह॒व्यम् जु॑होतन । इ॒दम् नम॒ ऋषि॑भ्यः पूर्व॒जेभ्यः॒ पूर्वे᳚भ्यः पथि॒कृद्भ्यः॑ । \textbf{ 11} \newline
                  \newline
                                                                  \textbf{ T.A.4.5.2} \newline
                  योऽस्य॒ कौष्ठ्य॒ जग॑तः॒ पार्थि॑व॒स्यैक॑ इद्व॒शी ।  य॒मम् भ॑ङ्ग्यश्र॒वो गा॑य॒ यो राजा॑ऽनप॒रोद्ध्यः॑ ॥ य॒मम् गाय॑ भङ्ग्य॒श्रवो॒ यो राजा॑ऽनप॒रोद्ध्यः॑ । येना॒पो न॒द्यो॑ धन्वा॑नि॒ येन॒ द्यौः पृ॑थि॒वी दृ॒ढा ॥ हि॒र॒ण्य॒क॒क्ष्यान्-थ्सु॒धुरान्॑. हिरण्या॒क्षा-न॑यश्श॒फान् । अश्वा॑-न॒नश्य॑तो दा॒न॒म् ॅय॒मो रा॑जाऽभि॒तिष्ठ॑ति ॥ य॒मो दा॑धार पृथि॒वीं ॅय॒मो विश्व॑मि॒दम् जग॑त् । य॒माय॒ सर्व॒मित्र॑स्थे॒ यत् प्रा॒णद्-वा॒युर॑क्षि॒तम् ॥ यथा॒ पञ्च॒ यथा॒ षड् य॒था पञ्च॑ द॒शर्.ष॑यः । य॒मम् ॅयो वि॑द्या॒थ्स ब्रू॑याद्य॒थैक ऋषि॑र्विजान॒ते । \textbf{ 12} \newline
                  \newline
                                                                  \textbf{ T.A.4.5.3} \newline
                  त्रिक॑द्रुकेभिः॒ पत॑ति॒ षडु॒र्वी-रेक॒मिद्-बृ॒हत् ।  गा॒य॒त्री त्रि॒ष्टुप्छन्दाꣳ॑सि॒ सर्वा॒ ता य॒म आहि॑ता ॥  अह॑रह॒र्नय॑मानो॒ गामश्व॒म् पुरु॑ष॒म् जग॑त् ।  वैव॑स्वतो॒ न तृ॑प्यति॒ पञ्च॑भि॒र्मान॑वैर्य॒मः ॥  वैव॑स्वते॒ विवि॑च्यन्ते॒ यमे॒ राज॑नि ते ज॒नाः ।  ये चे॒ह स॒त्येनेच्छ॑न्ते॒ य उ॒ चानृ॑तवादि॒नः ॥  ते रा॑जन्नि॒ह विवि॑च्यन्ते॒ऽथा य॑न्ति त्वा॒मुप॑ ।  दे॒वाꣳश्च॒ ये न॑म॒स्यन्ति॒ ब्राह्म॑णाꣳश्चाप॒चित्य॑ति ॥  यस्मि॑न् वृ॒क्षे सु॑पला॒शे दे॒वैः स॒पिंब॑ते य॒मः ।  अत्रा॑ नो वि॒श्पतिः॑ पि॒ता पु॑रा॒णा अनु॑वेनति ( ) । \textbf{ 13} \newline
                  \newline
                                                        (प॒थि॒कृद्भ्यो॑ - विजान॒ते - ऽनु॑वेनति) \textbf{5} \newline \newline
                                \textbf{ T.A.4.6.1} \newline
                  वै॒श्वा॒न॒रे ह॒विरि॒दम् जु॑होमि साह॒स्र-मुथ्सꣳ॑ श॒तधा॑रमे॒तम् । तस्मि॑न्ने॒ष पि॒तर॑म् पिताम॒हम् प्रपि॑तामहम् बिभर॒त् पिन्व॑माने ॥  द्र॒फ्सश्च॑स्कन्द पृथि॒वीमनु॒ द्यामि॒मञ्च॒ योनि॒मनु॒ यश्च॒ पूर्वः॑ ।  तृ॒तीय॒म् ॅयोनि॒मनु॑ स॒ञ्चर॑न्तम् द्र॒फ्सम् जु॑हो॒म्यनु॑ स॒प्त होत्राः᳚ ॥  इ॒मꣳ स॑मु॒द्रꣳ श॒तधा॑र॒मुथ्स॑म् ॅव्य॒च्यमा॑न॒म् भुव॑नस्य॒ मद्ध्ये᳚ ।  घृ॒तम् दुहा॑ना॒मदि॑ति॒म् जना॒याग्ने॒ मा हिꣳ॑सीः पर॒मे व्यो॑मन्न् ॥ अपे॑त॒ वीत॒ वि च॑ सर्प॒तातो॒ येऽत्र॒ स्थ पु॑रा॒णा ये च॒ नूत॑नाः ।  अहो॑भिर॒द्भि-र॒क्तुभि॒र्-व्य॑क्तम् ॅय॒मो द॑दा-त्वव॒सान॑मस्मै ॥ स॒वि॒तैतानि॒ शरी॑राणि पृथि॒व्यै मा॒तुरु॒पस्थ॒ आद॑धे ।  तेभि॑र्-युज्यन्ता-मघ्नि॒याः । \textbf{ 14} \newline
                  \newline
                                                                  \textbf{ T.A.4.6.2} \newline
                  शु॒नम् ॅवा॒हाः शु॒नम् ना॒राः शु॒नम् कृ॑षतु॒ लाङ्ग॑लम् ।  शु॒नम् ॅव॑र॒त्रा ब॑द्ध्यन्ताꣳ शु॒न-मष्ट्रा॒-मुदि॑ङ्गय॒ शुना॑सीरा शु॒नम॒स्मासु॑ धत्तम् ॥  शुना॑सीरावि॒माम् ॅवाच॒म् ॅयद्-दि॒वि च॑क्र॒थुः पयः॑ । तेने॒मा-मुप॑सिञ्चतम् ॥  सीते॒ वन्दा॑महे त्वा॒ऽर्वाची॑ सुभगे भव ।  यथा॑ नः सु॒भगा स॑सि॒ यथा॑ नः सु॒फला स॑सि ॥  स॒वि॒तैतानि॒ शरी॑राणि पृथि॒व्यै मा॒तुरु॒पस्थ॒ आद॑धे । तेभि॑रदिते॒ शम् भ॑व ॥  विमु॑च्यद्ध्व-मघ्नि॒या दे॑व॒याना॒ अता॑रिष्म॒ तम॑सस्पा॒रम॒स्य ।  ज्योति॑रापाम॒ सुव॑रगन्म ( ) । \textbf{ 15} \newline
                  \newline
                                                                  \textbf{ T.A.4.6.3} \newline
                  प्र वाता॒ वान्ति॑ प॒तय॑न्ति वि॒द्युत॒ उदोष॑धीर्जिहते॒ पिन्व॑ते॒ सुवः॑ । इरा॒ विश्व॑स्मै॒ भुव॑नाय जायते॒ यत् प॒र्जन्यः॑ पृथि॒वीꣳ रेत॒साऽव॑ति ॥  यथा॑ य॒माय॑ हा॒र्म्य-मव॑प॒न् पञ्च॑मान॒वाः ।  ए॒वम् ॅव॑पामि हा॒र्म्यम् ॅयथाऽसा॑म जीवलो॒के भूर॑यः ॥ चितः॑ स्थ परि॒चित॑ ऊर्द्ध्व॒चितः॑ श्रयद्ध्वम् पि॒तरो॑ दे॒वता᳚ ।  प्र॒जाप॑तिर्वः सादयतु॒ तया॑ दे॒वत॑या ॥  "आ प्या॑यस्व॒{41}" "सम् ते᳚{42}" । \textbf{ 16} \newline
                  \newline
                                                        (अ॒घ्नि॒या - अ॑गन्म - +स॒प्त च॑) \textbf{6} \newline \newline
                                \textbf{ T.A.4.7.1} \newline
                  उत्ते॑ तभ्नोमि पृथि॒वीम् त्वत्परी॒मम् ॅलो॒कम् नि॒दध॒न्मो अ॒हꣳ रि॑षम् । ए॒ताꣳ स्थूणा᳚म् पि॒तरो॑ धारयन्तु॒ तेऽत्रा॑ य॒मः साद॑नात्ते मिनोतु ॥  उप॑सर्प मा॒तर॒म् भूमि॑मे॒ता-मु॑रु॒व्यच॑सम् पृथि॒वीꣳ सु॒शेवा᳚म् । ऊर्ण॑म्रदा युव॒तिर् दक्षि॑णावत्ये॒षा त्वा॑ पातु॒ निर्.ऋ॑त्या उ॒पस्थे᳚ ॥  उच्छ्म॑ञ्चस्व पृथिवि॒ मा विबा॑धिथाः सूपाय॒नाऽस्मै॑ भव सूपवञ्च॒ना । मा॒ता पु॒त्रम् ॅयथा॑ सि॒चाऽभ्ये॑नम् भूमि वृणु ॥ उ॒च्छ्मञ्च॑माना पृथि॒वी हि तिष्ठ॑सि स॒हस्र॒म्मित॒ उप॒ हि श्रय॑न्ताम् ।  ते गृ॒हासो॑ मधु॒श्चुतो॒ विश्वाहा᳚स्मै शर॒णाः स॒न्त्वत्र॑ ॥ एणी᳚र्द्धा॒ना हरि॑णी॒रर्जु॑नीः सन्तु धे॒नवः॑ । तिल॑वथ्सा॒ ऊर्ज॑मस्मै॒ दुहा॑ना॒ विश्वाहा॑ स॒न्त्वन॑पस्फुरन्तीः । \textbf{ 17} \newline
                  \newline
                                                                  \textbf{ T.A.4.7.2} \newline
                  ए॒षा ते॑ यम॒साद॑ने स्व॒धा निधी॑यते गृ॒हे । अक्षि॑ति॒र्नाम॑ ते असौ ॥ इ॒दम् पि॒तृभ्यः॒ प्रभ॑रेम ब॒र्॒.हिर् दे॒वेभ्यो॒ जीव॑न्त॒ उत्त॑रम् भरेम ।  तत्त्व॑मारो॒हासो॒ मेघ्यो॒ भव॑म् ॅय॒मेन॒ त्वम् ॅय॒म्या॑ सम्ॅविदा॒नः ॥ मा त्वा॑ वृ॒क्षौ संबा॑धिष्टा॒म् मा मा॒ता पृ॑थिवि॒ त्वम् ।  पि॒तॄन्. हि यत्र॒ गच्छा॒स्येधा॑सम् ॅयम॒राज्ये᳚ ॥  मा त्वा॑ वृ॒क्षौ संबा॑धेथा॒म् मा मा॒ता पृ॑थि॒वी म॒ही । वै॒व॒स्व॒तꣳ हि गच्छा॑सि यम॒राज्ये॒ विरा॑जसि ॥  न॒ळम् प्ल॒वमारो॑है॒तम् न॒ळेन॑ प॒थोऽन्वि॑हि ।  स त्व॑म् न॒ळप्ल॑वो भू॒त्वा॒ सन्त॑र॒ प्रत॒रोत्त॑र । \textbf{ 18} \newline
                  \newline
                                                                  \textbf{ T.A.4.7.3} \newline
                  स॒वि॒तैतानि॒ शरी॑राणि पृथि॒व्यै मा॒तुरु॒पस्थ॒ आद॑धे । तेभ्यः॑ पृथिवि॒ शम् भ॑व ॥  षड्ढो॑ता॒ सूर्य॑म् ते॒ चक्षु॑र् गच्छतु॒ वात॑मा॒त्मा द्याम् च॒ गच्छ॑ पृथि॒वीम् च॒ धर्म॑णा ।  अ॒पो वा॑ गच्छ॒ यदि॒ तत्र॑ ते हि॒तमोष॑धीषु॒ प्रति॑तिष्ठा॒ शरी॑रैः ॥ पर॑म् मृत्यो॒ अनु॒ परे॑हि॒ पन्था॒म् ॅयस्ते॒ स्व इत॑रो देव॒याना᳚त् ।  चक्षु॑ष्मते शृण्व॒ते ते᳚ ब्रवीमि॒ मा नः॑ प्र॒जाꣳ री॑रिषो॒ मोत वी॒रान् ॥  शम् ॅवातः॒ शꣳ हि ते॒ घृणिः॒ शमु॑ ते स॒न्त्वोष॑धीः । कल्प॑न्ताम् मे॒ दिशः॑ श॒ग्माः ॥  पृ॒थि॒व्यास्त्वा॑ लो॒के सा॑दयाम्य॒मुष्य॒ शर्मा॑सि पि॒तरो॑ दे॒वाता᳚ ( ) । प्र॒जाप॑तिस्त्वा सादयतु॒ तया॑ दे॒वत॑या ॥ अ॒न्तरि॑क्षस्य त्वा दि॒वस्त्वा॑ दि॒शाम् त्वा॒ नाक॑स्य त्वा पृ॒ष्ठे ब्र॒द्ध्नस्य॑ त्वा वि॒ष्टपे॑ सादयाम्य॒मुष्य॒ शर्मा॑सि पि॒तरो॑ दे॒वता᳚ ।  प्र॒जाप॑तिस्त्वा सादयतु॒ तया॑ दे॒वत॑या । \textbf{ 19} \newline
                  \newline
                                                        (अन॑पस्फुरन्ती॒ - रुत्त॑र - दे॒वत॑या॒ द्वे च॑) \textbf{7} \newline \newline
                                \textbf{ T.A.4.8.1} \newline
                  अ॒पू॒पवा᳚न् घृ॒तवाꣳ॑श्च॒रुरेह सी॑दतूत्तभ्नु॒वन् पृ॑थि॒वीम् द्यामु॒तोपरि॑ । यो॒नि॒कृतः॑ पथि॒कृतः॑ सपर्यत॒ ये दे॒वाना᳚म् घृ॒तभा॑गा इ॒ह स्थ ।  ए॒षा ते यम॒साद॑ने स्व॒धा निधी॑यते गृ॒हे॑ऽसौ ।  दशा᳚क्षरा॒ ताꣳ र॑क्षस्व॒ ताम् गो॑पायस्व॒ ताम् ते॒ परि॑ ददामि॒ तस्या᳚म् त्वा॒ मा द॑भन् पि॒तरो॑ दे॒वता᳚ ।  प्र॒जाप॑तिस्त्वा सादयतु॒ तया॑ दे॒वत॑या ॥  अ॒पू॒पवा᳚-ञ्छृ॒तवा᳚न्-क्षी॒रवा॒न्-दधि॑वा॒न्-मधु॑माꣳ-श्च॒रुरेह सी॑दतूत्तभ्नु॒वन्-पृ॑थि॒वीम् द्यामु॒तोपरि॑ । यो॒नि॒कृतः॑ पथि॒कृतः॑ सपर्यत॒ ये दे॒वानाꣳ॑ शृ॒तभा॑गाः क्षी॒रभा॑गा॒ दधि॑भागा॒ मधु॑भागा इ॒ह स्थ । ए॒षा ते॑ यम॒साद॑ने स्व॒धा निधी॑यते गृ॒हे॑ऽसौ । श॒ताक्ष॑रा स॒हस्रा᳚क्षरा॒-ऽयुता᳚क्ष॒रा-ऽच्यु॑ताक्षरा॒ ताꣳ र॑क्षस्व॒ ताम् गो॑पायस्व॒ ताम् ते॒ परि॑ददामि॒ तस्या᳚म् त्वा॒ मा द॑भन् पि॒तरो॑ दे॒वता᳚ । प्र॒जाप॑तिस्त्वा सादयतु॒ तया॑ दे॒वत॑या । \textbf{ 20} \newline
                  \newline
                                                        (अ॒पू॒पवा॑नसौ॒ दश॑) \textbf{8} \newline \newline
                                \textbf{ T.A.4.9.1} \newline
                  ए॒तास्ते᳚ स्व॒धा अ॒मृताः᳚ करोमि॒ यास्ते॑ धा॒नाः प॑रि॒किरा॒म्यत्र॑ । तास्ते॑ य॒मः पि॒तृभिः॑ सम्ॅविदा॒नोऽत्र॑ धे॒नूः का॑म॒दुघाः᳚ करोतु ॥  त्वामर्जु॒नौष॑धीना॒म् पयो᳚ ब्र॒ह्माण॒ इद्वि॑दुः ।  तासा᳚म् त्वा॒ मद्ध्या॒दाद॑दे च॒रुभ्यो॒ अपि॑धातवे ॥  दू॒र्वाणाꣳ॑ स्त॒म्बमाह॑रै॒ताम् प्रि॒यत॑मा॒म् मम॑ । इ॒माम् दिश॑म् मनु॒ष्या॑णा॒म् भूयि॒ष्ठाऽनु॒ वि रो॑हतु ॥ काशा॑नाꣳ स्त॒म्बमाह॑र॒ रक्ष॑सा॒मप॑हत्यै ।  य ए॒तस्यै॑ दि॒शः प॒राभ॑वन्नघा॒यवो॒ यथा॒ ते नाभ॑वा॒न् पुनः॑ ॥ द॒र्भाणाꣳ॑ स्त॒म्बमाह॑र पितृ॒णामोष॑धीम् प्रि॒याम् । अन्वस्यै॒ मूल॑म् जीवा॒दनु॒ काण्ड॒मथो॒ फल᳚म् । \textbf{ 21} \newline
                  \newline
                                                                  \textbf{ T.A.4.9.2} \newline
                  "लो॒कम् पृ॑ण॒{43}" "ता अ॑स्य॒ सूद॑दोहसः{44}" ॥  शम् ॅवातः॒ शꣳहि ते॒ घृणिः॒ शमु॑ ते स॒न्त्वोष॑धीः । कल्प॑न्ताम् ते॒ दिशः॒ सर्वाः᳚ ॥ इ॒दमे॒व मेतोऽप॑रा॒मार्ति॑माराम॒ काञ्च॒न ।  तथा॒ तद॒श्विभ्या᳚म् कृ॒तम् मि॒त्रेण॒ वरु॑णेन च ॥  व॒र॒णो वा॑रयादि॒दम् दे॒वो वन॒स्पतिः॑ ।  आर्त्यै॒ निर्.ऋ॑त्यै॒ द्वेषा᳚च्च॒ वन॒स्पतिः॑ ॥  विधृ॑तिरसि॒ विधा॑रया॒स्मद॒घा द्वेषाꣳ॑सि श॒मि श॒मया॒स्मद॒घा द्वेषाꣳ॑सि य॒व य॒वया॒स्मद॒घा द्वेषाꣳ॑सि ॥  पृ॒थि॒वीम् ग॑च्छा॒न्तरि॑क्षम् गच्छ॒ दिव॑म् गच्छ॒ दिशो॑ गच्छ॒ सुव॑र्गच्छ॒ सुव॑र्गच्छ॒ दिशो॑ गच्छ॒ दिव॑म् गच्छा॒न्तरि॑क्षम् गच्छ पृथि॒वीम् ग॑च्छा॒पो वा॑ गच्छ॒ यदि॒ तत्र॑ ते हि॒तमोष॑धीषु॒ प्रति॑तिष्ठा॒ शरी॑रैः ॥ "अश्म॑न्वती रेवती॒र्{43}" "यद्वै दे॒वस्य॑ सवि॒तुः प॒वित्र॒म्{44}"  "ॅया रा॒ष्ट्रात्प॒न्ना{45}" "दुद्व॒यम् तम॑स॒स्परि॑{46}"  धा॒ता पु॑नातु{47}"( ) । \textbf{ 22} \newline
                  \newline
                                                        (अथो॒ फल॑म् - धा॒ता पु॑नातु) \textbf{9} \newline \newline
                                \textbf{ T.A.4.10.1} \newline
                  आरो॑ह॒तायु॑र्-ज॒रस॑म् गृणा॒ना अ॑नुपू॒र्वम् ॅयत॑माना॒ यति॒ष्ट । इ॒ह त्वष्टा॑ सु॒जनि॑मा सु॒रत्नो॑ दी॒र्घमायुः॑ करतु जी॒वसे॑ वः ॥  यथाऽहा᳚न्यनुपू॒र्वम् भव॑न्ति॒ यथ॒र्तव॑ ऋ॒तुभि॒र्-यन्ति॑ क्लृ॒प्ताः । यथा॒ न पूर्व॒मप॑रो॒ जहा᳚त्ये॒वा धा॑त॒रायूꣳ॑षि कल्पयैषाम् ॥  न हि॑ ते अग्ने त॒नुवै᳚ क्रू॒रम् च॒कार॒ मर्त्यः॑ ।  क॒पिर्-ब॑भस्ति॒ तेज॑न॒म् पुन॑र्ज॒रायु॒ गौरि॑व ।  अप॑ नः॒ शोशु॑चद॒घमग्ने॑ शुशु॒द्ध्या र॒यिम् ।  अप॑ नः॒ शोशु॑चद॒घम् मृ॒त्यवे॒ स्वाहा᳚ ॥  अ॒न॒ड्वाह॑-म॒न्वार॑भामहे स्व॒स्तये᳚ ।  स न॒ इन्द्र॑ इव दे॒वेभ्यो॒ वह्निः॑ स॒पांर॑णो भव । \textbf{ 23} \newline
                  \newline
                                                                  \textbf{ T.A.4.10.2} \newline
                  इ॒मे जी॒वा वि॑मृ॒तैराव॑वर्-ति॒न्नभू᳚द्-भ॒द्रा दे॒वहू॑तिम् नो अ॒द्य । प्राञ्जो॑ऽगामा नृ॒तये॒ हसा॑य॒ द्राघी॑य॒ आयुः॑ प्रत॒राम् दधा॑नाः ॥  मृ॒त्योः प॒दम् ॅयोपय॑न्तो॒ यदैम॒ द्राघी॑य॒ आयुः॑ प्रत॒राम् दधा॑नाः । आ॒प्याय॑मानाः प्र॒जया॒ धने॑न शु॒द्धाः पू॒ता भ॑वथ यज्ञियासः ॥  इ॒मम् जी॒वेभ्यः॑ परि॒धिम् द॑धामि॒ मा नो नु॑ गा॒दप॑रो॒ अर्द्ध॑मे॒तम् ।  श॒तम् जी॑वन्तु श॒रदः॑ पुरू॒चीस्ति॒रो मृ॒त्युम् द॑द्महे॒ पर्व॑तेन ॥ इ॒मा नारी॑रविध॒वाः सु॒पत्नी॒राञ्ज॑नेन स॒र्पिषा॒ संमृ॑शन्ताम् ।  अ॒न॒श्रवो॑ अनमी॒वाः सु॒शेवा॒ आरो॑हन्तु॒ जन॑यो॒ योनि॒मग्रे᳚ ॥ यदाञ्ज॑नम् त्रैककु॒दम् जा॒तꣳ हि॒मव॑त॒स्परि॑ ।  तेना॒मृत॑स्य॒ मूले॒नारा॑तीर्-जम्भयामसि ( ) ॥  यथा॒ त्व-मु॑द्भि॒नथ्-स्यो॑षधे पृथि॒व्या अधि॑ ।  ए॒वमि॒म उद्भि॑न्दन्तु की॒र्त्या यश॑सा ब्रह्मवर्च॒सेन॑ ॥ अ॒जो᳚ऽस्यजा॒स्मद॒घा द्वेषाꣳ॑सि य॒वो॑ऽसि य॒वया॒स्मद॒घा द्वेषाꣳ॑सि । \textbf{ 24} \newline
                  \newline
                                                        (भ॒व॒ - ज॒म्भ॒या॒म॒सि॒ त्रीणि॑ च) \textbf{10} \newline \newline
                                \textbf{ T.A.4.11.1} \newline
                  अप॑ नः॒ शोशु॑चद॒घमग्ने॑ शुशु॒द्ध्या र॒यिम् । अप॑ नः॒ शोशु॑चद॒घम् ॥ सु॒क्षे॒त्रि॒या सु॑गातु॒या व॑सू॒या च॑ यजामहे । अप॑ नः॒ शोशु॑चद॒घम् ॥  प्र यद्-भन्दि॑ष्ठ एषा॒म् प्रास्माका॑सश्च सू॒रयः॑ । अप॑ नः॒ शोशु॑चद॒घम् ॥ प्रयद॒ग्नेः सह॑स्वतो वि॒श्वतो॒ यन्ति॑ सू॒रयः॑ । अप॑ नः॒ शोशु॑चद॒घम् ॥  प्रयत्ते॑ अग्ने सू॒रयो॒ जाये॑महि॒ प्र ते॑ व॒यम् ।  अप॑ नः॒ शोशु॑चद॒घम् । \textbf{ 25} \newline
                  \newline
                                                                  \textbf{ T.A.4.11.2} \newline
                  त्वꣳ हि वि॑श्वतोमुख वि॒श्वतः॑ परि॒भूरसि॑ । अप॑ नः॒ शोशु॑चद॒घम् ॥ द्विषो॑ नो विश्वतोमु॒खाति॑ ना॒वेव॑ पारय । अप॑ नः॒ शोशु॑चद॒घम् ॥ स नः॒ सिन्धु॑मिव ना॒वयाऽति॑ पर्.षा स्व॒स्तये᳚ । अप॑ नः॒ शोशु॑चद॒घम् ॥ आपः॑ प्रव॒णादि॑व य॒ती-रपा॒स्मथ्-स्य॑न्दताम॒घम् । अप॑ नः॒ शोशु॑चद॒घम् ॥ उ॒द्व॒ना-दु॑द॒कानी॒वापा॒स्मथ्-स्य॑न्दताम॒घम् ।  अप॑ नः॒ शोशु॑चद॒घम् ( ) ॥ आ॒न॒न्दाय॑ प्रमो॒दाय॒ पुन॒रागाꣳ॒॒ स्वान्गृ॒हान् । अप॑ नः॒ शोशु॑चद॒घम् ॥ न वै तत्र॒ प्रमी॑यते॒ गौरश्वः॒ पुरु॑षः प॒शुः ।  यत्रे॒दम् ब्रह्म॑ क्रि॒यते॑ परि॒धिर्जीव॑नाय॒ कमप॑ नः॒ शोशु॑चद॒घम् । \textbf{ 26} \newline
                  \newline
                                                        (अ॒घ - म॒घम् च॒त्वारि॑ च) \textbf{11} \newline \newline
                                \textbf{ T.A.4.12.1} \newline
                  अप॑श्याम युव॒ति-मा॒चर॑न्तीम् मृ॒ताय॑ जी॒वाम् प॑रिणी॒यमा॑नाम् । अ॒न्धेन॒ या तम॑सा॒ प्रावृ॑ताऽसि॒ प्राची॒मवा॑ची॒-मव॒यन्-नरि॑ष्ट्यै ॥  मयै॒ताम् माꣳ॒॒स्ताम् भ्रि॒यमा॑णा दे॒वी स॒ती पि॑तृलो॒कम् ॅयदैषि॑ । वि॒श्ववा॑रा॒ नभ॑सा॒ सम्ॅव्य॑यन्त्यु॒भौ नो॑ लो॒कौ पय॒साऽऽवृ॑णीहि ॥  रयि॑ष्ठाम॒ग्निम् मधु॑मन्तमू॒र्मिण॒मूर्जः॑ सन्तम् त्वा॒ पय॒सोप॒ सꣳस॑देम ।  सꣳ र॒य्या समु॒ वर्च॑सा॒ सच॑स्वा नः स्व॒स्तये᳚ ॥  ये जी॒वा ये च॑ मृ॒ता ये जा॒ता ये च॒ जन्त्याः᳚ ।  तेभ्यो॑ घृ॒तस्य॑ धारयितु॒म् मधु॑धारा व्युन्द॒ती ॥  मा॒ता रु॒द्राणा᳚म् दुहि॒ता वसू॑नाꣳ॒॒ स्वसा॑ऽऽदि॒त्याना॑म॒-मृत॑स्य॒ नाभिः॑ ।  प्रणु॒वोच॑म् चिकि॒तुषे॒ जना॑य॒ मा गामना॑गा॒मदि॑तिम् ॅवधिष्ट ( ) ॥ पिब॑तूद॒कम् तृणा᳚न्यत्तु । ओमुथ् सृ॒जत । \textbf{ 27} \newline
                  \newline
                                                        (व॒धि॒ष्ट॒ द्वे च॑) \textbf{12} \newline \newline
\textbf{Prapaataka Korvai with starting Padams of 1 to12 Anuvaakams :-} \newline
(प॒रे॒यु॒वाꣳस॒म् प्र वि॒द्वान् भुव॑नस्या॒भ्याव॑ वृथ्स्वा॒ जो भा॒गो॑ऽयम् ॅवै चतुश्चत्वारिꣳशत् । य ए॒तस्य॒ त्वत् पञ्च॑ । प्रके॒तुने॒दन्ते॒ नाके॑ सुप॒र्णम् ॅयौ ते॒ ये युद्ध्य॑न्ते॒ तप॒साऽश्म॑न्वती रेवतीः॒ सꣳर॑भद्ध्वम॒ष्टाविꣳ॑शतिः । यं ते॒ यत्त॒ उत्तिष्ठा॒त॑ इ॒दन्त॒ उत्ति॑ष्ठ॒ प्रेह्यश्म॒न्॒. यद्वा उद्व॒यम॒यम् पञ्च॑विꣳशतिः । आया॑तु त्रिꣳ॒॒शत् । वै॒श्वा॒न॒रे तस्मि॑न् द्र॒फ्स इ॒ममपे॒ताहो॑भिर् युज्यन्तामघ्नि॒या अ॑दिते पा॒रम् ॅव॒ आ प्या॑यस्व स॒प्तविꣳ॑शतिः । उत्ते॑ गृहेऽखाषि॑ति॒स्तेभ्यः॑ पृ॑थि॒व्या अ॒न्तरि॑क्षस्य॒ द्वात्रिꣳ॑शत् । अ॒पू॒पवा॑नसौ॒ दश॑ श॒त द॑श । ए॒तास्ते॑ ते॒ दिशः॒ सर्वा॑ इ॒दमश्म॑न् विꣳश॒तिः । ओ॑रोहत त॒नुवै᳚ क्रू॒रम् च॒कार॒ पुन॑र्मृ॒त्यवे॒ मा नो नु॑ गाद् दद्मह इ॒मा नारीः॒ परि॒ त्रयो॑विꣳशतिः । अप॑ नः सुक्षेत्रि॒या प्र यद्भन् दि॑ष्ठः॒ प्र यद॒ग्नेः प्र यत्ते॑ अग्ने॒ त्वꣳ हि द्विषः॒ स नः॒ सिन्धु॒मापः॑ प्रव॒णादु॑द् व॒नादा॑न॒न्दाय॒ न वै तत्र॒ यत्रे॒दम् चतु॑र्विꣳशतिः॒ । अप॑श्या॒मावृ॑णीहि॒ द्वाद॑श द्वादश ।) \newline

\textbf{korvai with starting padams of1, 11, 21 Series of Dasinis :-} \newline
(प॒रे॒यु॒वाꣳस॒ - माया᳚ - त्वे॒तास्ते॑ स॒प्तविꣳ॑शतिः ) \newline

\textbf{first and last padam in TA, 4th Prapaatakam :-} \newline
(प॒रे॒यु॒वाꣳस॒ - मोमुथ्सृ॒जत ) \newline 


॥ कृष्ण यजुर्वेदीय तैत्तिरीय ब्राह्मणे आरण्यके चतुर्थः प्रपाठकः समाप्तः ॥
==================
Appendix (of Expansions)
ट्.आ.4.4.2 - अश्म॑न्वती रेवती॒र्{36}" "यद्वै दे॒वस्य॑ सवि॒तुः प॒वित्र॒म्{37}" "ॅया रा॒ष्ट्रात् प॒न्ना{38}" "दुद्व॒यम् तम॑स॒स्परि॑{39}" "धा॒ता पु॑नातु{40}" 
अश्म॑न्वती रेवतीः॒ सꣳ र॑भद्ध्व॒ मुत्ति॑ष्ठत॒ प्रत॑रता सखायः । अत्रा॑ जहाम॒ ये अस॒न्नशे॑वाः शि॒वान् व॒यम॒भि वाजा॒नुत्त॑रेम ॥ {36}
(Appearing in T.A.4.3.2)


यद्वै दे॒वस्य॑ सवि॒तुः प॒वित्रꣳ॑ स॒हस्र॑धार॒म् ॅवित॑तम॒न्तरि॑क्षे । येनापु॑ना॒दिन्द्र॒मना॑र्त॒-मार्त्यै॒ तेना॒हम् माꣳ स॒र्वत॑नुम् पुनामि ॥ {37}
(Appearing in T.A.4.3.3)

या रा॒ष्ट्रात् प॒न्नादप॒ यन्ति॒ शाखा॑ अ॒भिमृ॑ता नृ॒पति॑मि॒च्छमा॑नाः । धा॒तुस्ताः सर्वाः॒ पव॑नेन पू॒ताः प्र॒जया॒ऽस्मान्-र॒य्या वर्च॑सा॒ सꣳसृ॑जाथ ॥ {38} 
(Appearing in T.A.4.3.3)

उद्व॒यम् तम॑स॒स्परि॒ पश्य॑न्तो॒ ज्योति॒रुत्त॑रम् । दे॒वम् दे॑व॒त्रा सूर्य॒मग॑न्म॒ ज्योति॑रुत्त॒मम् ॥ {39} 
(Appearing in T.A.4.3.3)

धा॒ता पु॑नातु सवि॒ता पु॑नातु । अ॒ग्नेस्तेज॑सा॒ सूर्य॑स्य॒ वर्च॑सा ॥ {40}
(Appearing in T.A.4.3.3) \newline
\pagebreak
\pagebreak
        


\end{document}
