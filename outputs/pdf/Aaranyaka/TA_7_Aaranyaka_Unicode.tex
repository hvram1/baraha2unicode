\documentclass[17pt]{extarticle}
\usepackage{babel}
\usepackage{fontspec}
\usepackage{polyglossia}
\usepackage{extsizes}



\setmainlanguage{sanskrit}
\setotherlanguages{english} %% or other languages
\setlength{\parindent}{0pt}
\pagestyle{myheadings}
\newfontfamily\devanagarifont[Script=Devanagari]{AdishilaVedic}


\newcommand{\VAR}[1]{}
\newcommand{\BLOCK}[1]{}




\begin{document}
\begin{titlepage}
    \begin{center}
 
\begin{sanskrit}
    { \Large
    ॐ नमः परमात्मने, श्री महागणपतये नमः
श्री गुरुभ्यो नमः, ह॒रिः॒ ॐ 
    }
    \\
    \vspace{2.5cm}
    \mbox{ \Huge
    कृष्ण यजुर्वेदीय तैत्तिरीय आरण्यके प्रथमः प्रपाठकः   }
\end{sanskrit}
\end{center}

\end{titlepage}
\tableofcontents

ॐ नमः परमात्मने, श्री महागणपतये नमः
श्री गुरुभ्यो नमः, ह॒रिः॒ ॐ \newline
7.1     सप्तमः प्रपाठकः \newline

\addcontentsline{toc}{section}{ 7.1     सप्तमः प्रपाठकः}
\markright{ 7.1     सप्तमः प्रपाठकः \hfill https://www.vedavms.in \hfill}
\section*{ 7.1     सप्तमः प्रपाठकः }
                                \textbf{ T.A.7.1.1} \newline
                  नमो॑ वा॒चे या चो॑दि॒ता या चानु॑दिता॒ तस्यै॑ वा॒चे नमो॒ नमो॑ वा॒चे नमो॑ वा॒चस्पत॑ये॒ नम॒ ऋषि॑भ्यो मन्त्र॒कृद्भ्यो॒ मन्त्र॑पतिभ्यो॒ मा मामृष॑यो मन्त्र॒कृतो॑ मन्त्र॒पत॑यः॒ परा॑दु॒र्माऽहमृषी᳚न् मन्त्र॒कृतो॑ मन्त्र॒पती॒न् परा॑दां ॅवैश्वदे॒वीं ॅवाच॑मुद्यासꣳ शि॒वामद॑स्ता॒म् जुष्टा᳚म् दे॒वेभ्यः॒ शर्म॑ मे॒ द्यौः शर्म॑ पृथि॒वी शर्म॒ विश्व॑मि॒दम् जग॑त् ।  शर्म॑ च॒न्द्रश्च॒ सूर्य॑श्च॒ शर्म॑ ब्रह्मप्रजाप॒ती ॥  भू॒तं ॅव॑दिष्ये॒ भुव॑नं ॅवदिष्ये॒ तेजो॑ वदिष्ये॒ यशो॑ वदिष्ये॒ तपो॑ वदिष्ये॒ ब्रह्म॑ वदिष्ये स॒त्यं ॅव॑दिष्ये॒ तस्मा॑ अ॒हमि॒द-मु॑प॒स्तर॑ण॒-मुप॑स्तृण उप॒स्तर॑णम् मे प्र॒जायै॑ पशू॒नाम् भू॑या-दुप॒स्तर॑णम॒हम् प्र॒जायै॑ पशू॒नाम् भू॑यास॒म् प्राणा॑पानौ मृ॒त्योर्मा॑ पात॒म् प्राणा॑पानौ॒ मा मा॑ हासिष्ट॒म् मधु॑ मनिष्ये॒ मधु॑ जनिष्ये॒ मधु॑ वक्ष्यामि॒ मधु॑ वदिष्यामि॒ मधु॑मतीम् दे॒वेभ्यो॒ वाच॑मुद्यासꣳ शुश्रू॒षेण्या᳚म् मनु॒ष्ये᳚भ्य॒स्तम् मा॑ दे॒वा अ॑वन्तु शो॒भायै॑ पि॒तरोऽनु॑मदन्तु ॥  ओम् शान्तिः॒ शान्तिः॒ शान्तिः॑ । \textbf{ 1} \newline
                  \newline
                                                         \textbf{} \newline \newline
                                \textbf{ T.A.7.2.1} \newline
                  यु॒ञ्जते॒ मन॑ उ॒त यु॑ञ्जते॒ धियः॑ । विप्रा॒ विप्र॑स्य बृह॒तो वि॑प॒श्चितः॑ । वि होत्रा॑ दधे वयुना॒विदेक॒ इत् । म॒ही दे॒वस्य॑ सवि॒तुः परि॑ष्टुतिः ॥ दे॒वस्य॑ त्वा सवि॒तुः प्र॑स॒वे । अ॒श्विनो᳚र् बा॒हुभ्या᳚म् ।  पू॒ष्णो हस्ता᳚भ्या॒माद॑दे ॥ अभ्रि॑रसि॒ नारि॑रसि ।  अ॒द्ध्व॒र॒कृद्-दे॒वेभ्यः॑ । उत्ति॑ष्ठ ब्रह्मणस्पते \textbf{ 2} \newline
                  \newline
                                                                  \textbf{ T.A.7.2.2} \newline
                  दे॒व॒यन्त॑स्त्वेमहे । उप॒ प्रय॑न्तु म॒रुतः॑ सु॒दान॑वः । इन्द्र॑ प्रा॒शूर्भ॑वा॒ सचा᳚ ॥ प्रैतु॒ ब्रह्म॑ण॒स्पतिः॑ । प्र दे॒व्ये॑तु सू॒नृता᳚ ।  अच्छा॑ वी॒रन्नर्य॑म् प॒ङ्क्तिरा॑धसम् । दे॒वा य॒ज्ञ्न्न॑यन्तु नः ॥ देवी᳚ द्यावापृथिवी॒ अनु॑ मे मꣳसाथाम् । ऋ॒द्ध्यास॑म॒द्य । म॒खस्य॒ शिरः॑ । \textbf{ 3} \newline
                  \newline
                                                                  \textbf{ T.A.7.2.3} \newline
                  म॒खाय॑ त्वा । म॒खस्य॑ त्वा शी॒र्ष्णे ॥ इय॒त्यग्र॑ आसीः ।  ऋ॒द्ध्यास॑म॒द्य । म॒खस्य॒ शिरः॑ । म॒खाय॑ त्वा । म॒खस्य॑ त्वा शी॒र्ष्णे ॥ देवी᳚र् वम्रीर॒स्य भू॒तस्य॑ प्रथमजा ऋतावरीः ॥ ऋ॒द्ध्यास॑म॒द्य । म॒खस्य॒ शिरः॑ \textbf{ 4} \newline
                  \newline
                                                                  \textbf{ T.A.7.2.4} \newline
                  म॒खाय॑ त्वा । म॒खस्य॑ त्वा शी॒र्ष्णे ॥ इन्द्र॒स्यौजो॑ऽसि ।  ऋ॒द्ध्यास॑म॒द्य । म॒खस्य॒ शिरः॑ । म॒खाय॑ त्वा । म॒खस्य॑ त्वा शी॒र्ष्णे ॥ अ॒ग्नि॒जा अ॑सि प्र॒जाप॑ते॒ रेतः॑ । ऋ॒द्ध्यास॑म॒द्य । म॒खस्य॒ शिरः॑ \textbf{ 5} \newline
                  \newline
                                                                  \textbf{ T.A.7.2.5} \newline
                  म॒खाय॑ त्वा । म॒खस्य॑ त्वा शी॒र्ष्णे ॥ आयु॑र् धेहि प्रा॒णम् धे॑हि । अ॒पा॒नम् धे॑हि व्या॒नम् धे॑हि । चक्षु॑र् धेहि॒ श्रोत्रं॑ धेहि । मनो॑ धेहि॒ वाचं॑ धेहि । आ॒त्मानं॑ धेहि प्रति॒ष्ठां धे॑हि ।  मां धे॑हि॒ मयि॑ धेहि ॥ मधु॑ त्वा मधु॒ला क॑रोतु ।  म॒खस्य॒ शिरो॑ऽसि ( ) \textbf{ 6} \newline
                  \newline
                                                                  \textbf{ T.A.7.2.6} \newline
                  य॒ज्ञ्स्य॑ प॒दे स्थः॑ । गा॒य॒त्रेण॑ त्वा॒ छन्द॑सा करोमि । त्रैष्टु॑भेन त्वा॒ छन्द॑सा करोमि । जाग॑तेन त्वा॒ छन्द॑सा करोमि ॥ म॒खस्य॒ रास्ना॑ऽसि । अदि॑तिस्ते॒ बिल॑ङ्गृह्णातु । पाङ्क्ते॑न॒ छन्द॑सा ।  सूर्य॑स्य॒ हर॑सा श्राय । म॒खो॑ऽसि । \textbf{ 7} \newline
                  \newline
                                                  
                (इय॑ति॒ देवी॒रिन्द्र॒स्यौजो᳚ऽस्यग्नि॒जा अ॒स्यायु॑र्द्धेहि प्रा॒णम् पञ्च॑)  अनुवाकं-3 \newline
                                      (प॒ते॒ - शिर॑ - ऋतावरीर्. ऋद्ध्यास॑म॒द्य म॒खस्य॒ शिरः॒ - शिरः॒ - शिरो॑ऽसि॒ - +नव॑ च) \textbf{2} \newline \newline
                                \textbf{ T.A.7.3.1} \newline
                  वृष्णो॒ अश्व॑स्य नि॒ष्पद॑सि । वरु॑णस्त्वा धृ॒तव्र॑त॒ आधू॑पयतु । मि॒त्रावरु॑णयोर्  ध्रु॒वेण॒ धर्म॑णा ॥ अ॒र्चिषे᳚ त्वा । शो॒चिषे᳚ त्वा ।  ज्योति॑षे त्वा । तप॑से त्वा ॥ अ॒भीमम् म॑हि॒ना दिव᳚म् ।  मि॒त्रो ब॑भूव स॒प्रथाः᳚ । उ॒त श्रव॑सा पृथि॒वीम् \textbf{ 8} \newline
                  \newline
                                                                  \textbf{ T.A.7.3.2} \newline
                  मि॒त्रस्य॑ चर्.षणी॒धृतः॑ ।  श्रवो॑ दे॒वस्य॑ सान॒सिम् । द्यु॒म्नञ्चि॒त्रश्र॑वस्तमम् ॥ सिद्ध्यै᳚ त्वा । दे॒वस्त्वा॑ सवि॒तोद्व॑पतु ।  सु॒पा॒णिः स्व॑ङ्गु॒रिः । सु॒बा॒हुरु॒त शक्त्या᳚ ॥ अप॑द्यमानः पृथि॒व्याम् । आशा॒ दिश॒ आ पृ॑ण । उत्ति॑ष्ठ बृ॒हन् भ॑व \textbf{ 9} \newline
                  \newline
                                                                  \textbf{ T.A.7.3.3} \newline
                  ऊ॒र्द्ध्वस्ति॑ष्ठ ध्रु॒वस्त्वम् ॥ सूर्य॑स्य त्वा॒ चक्षु॒षाऽन्वी᳚क्षे । ऋ॒जवे᳚ त्वा । सा॒धवे᳚ त्वा । सु॒क्षि॒त्यै त्वा॒ भूत्यै᳚ त्वा ॥  इ॒दम॒हम॒मुमा॑मुष्याय॒णं ॅवि॒शा प॒शुभि॑र् ब्रह्मवर्च॒सेन॒ पर्यू॑हामि ।  गा॒य॒त्रेण॑ त्वा॒ छन्द॒सा ऽऽच्छृ॑णद्मि । त्रैष्टु॑भेन त्वा॒ छन्द॒सा ऽऽच्छृ॑णद्मि ।  जाग॑तेन त्वा॒ छन्द॒सा ऽऽच्छृ॑णद्मि । छृ॒णत्तु॑ त्वा॒ वाक् ( ) । छृ॒णत्तु॒ त्वोर्क् । छृ॒णत्तु॑ त्वा ह॒विः ।  छृ॒न्धि वाच᳚म् । छृ॒न्ध्यूर्ज᳚म् । छृ॒न्धि ह॒विः ॥  देव॑ पुरश्चर स॒घ्यास॑न्त्वा । \textbf{ 10} \newline
                  \newline
                                                        (पृ॒थि॒वीम् - भ॑व॒ - वाख् षट्च॑) \textbf{3} \newline \newline
                                \textbf{ T.A.7.4.1} \newline
                  ब्रह्म॑न् प्रव॒र्ग्ये॑ण॒ प्रच॑रिष्यामः ।  होत॑र् घ॒र्मम॒भिष्टु॑हि । अग्नी॒द्रौहि॑णौ पुरो॒डाशा॒वधि॑श्रय । प्रति॑प्रस्थात॒र्विह॑र ।  प्रस्तो॑तः॒ सामा॑नि गाय ॥ यजु॑र्युक्तꣳ॒॒ साम॑भि॒राक्त॑खन्त्वा । विश्वै᳚र् दे॒वैरनु॑मतम् म॒रुद्भिः॑ । दक्षि॑णाभिः॒ प्रत॑तम् पारयि॒ष्णुम् ।  स्तुभो॑ वहन्तु सुमन॒स्यमा॑नम् । स नो॒ रुच॑न्धे॒ह्यहृ॑णीयमानः ( ) । भूर्भुव॒स्सुवः॑ ।  ओमिन्द्र॑वन्तः॒ प्रच॑रत । \textbf{ 11} \newline
                  \newline
                                                        (अहृ॑णीयमानो॒ द्वे च॑) \textbf{4} \newline \newline
                                \textbf{ T.A.7.5.1} \newline
                  ब्रह्म॒न् प्रच॑रिष्यामः । होत॑र् घ॒र्मम॒भिष्टु॑हि ॥ य॒माय॑ त्वा म॒खाय॑ त्वा । सूर्य॑स्य॒ हर॑से त्वा ॥ प्रा॒णाय॒ स्वाहा᳚ व्या॒नाय॒ स्वाहा॑ ऽपा॒नाय॒ स्वाहा᳚ ।  चक्षु॑षे॒ स्वाहा॒ श्रोत्रा॑य॒ स्वाहा᳚ । मन॑से॒ स्वाहा॑ वा॒चे सर॑स्वत्यै॒ स्वाहा᳚ । दक्षा॑य॒ स्वाहा॒ क्रत॑वे॒ स्वाहा᳚ । ओज॑से॒ स्वाहा॒ बला॑य॒ स्वाहा᳚ ॥  दे॒वस्त्वा॑ सवि॒ता मद्ध्वा॑ऽनक्तु \textbf{ 12} \newline
                  \newline
                                                                  \textbf{ T.A.7.5.2} \newline
                  पृ॒थि॒वीं तप॑सस्त्रायस्व । अ॒र्चिर॑सि शो॒चिर॑सि॒ ज्योति॑रसि॒ तपो॑ऽसि । सꣳसी॑दस्व म॒हाꣳ अ॑सि । शोच॑स्व देव॒वीत॑मः । वि धू॒मम॑ग्ने अरु॒षम् मि॑येद्ध्य । सृ॒ज प्र॑शस्त दर्.श॒तम् ॥  अ॒ञ्जन्ति॒ यम् प्र॒थय॑न्तो॒ न विप्राः᳚ । व॒पाव॑न्त॒न्नाग्निना॒ तप॑न्तः ।  पि॒तुर्न पु॒त्र उप॑सि॒ प्रेष्ठः॑ । आ घ॒र्मो अ॒ग्निमृ॒तय॑न्नसादीत् । \textbf{ 13} \newline
                  \newline
                                                                  \textbf{ T.A.7.5.3} \newline
                  अ॒ना॒धृ॒ष्या पु॒रस्ता᳚त् । अ॒ग्नेराधि॑पत्ये ।  आयु॑र्मे दाः । पु॒त्रव॑ती दक्षिण॒तः । इन्द्र॒स्याधि॑पत्ये । प्र॒जाम् मे॑ दाः । सु॒षदा॑ प॒श्चात् । दे॒वस्य॑ सवि॒तुराधि॑पत्ये । प्रा॒णम् मे॑ दाः । आश्रु॑तिरुत्तर॒तः \textbf{ 14} \newline
                  \newline
                                                                  \textbf{ T.A.7.5.4} \newline
                  मि॒त्रावरु॑णयो॒राधि॑पत्ये । श्रोत्र॑म् मे दाः । विधृ॑तिरु॒परि॑ष्टात् । बृह॒स्पते॒राधि॑पत्ये ।  ब्रह्म॑ मे दाः क्ष॒त्रम् मे॑ दाः । तेजो॑ मे धा॒ वर्चो॑ मे धाः । यशो॑ मे धा॒ स्तपो॑ मे धाः ।  मनो॑ मे धाः ॥ मनो॒रश्वा॑ऽसि॒ भूरि॑पुत्रा ।  विश्वा᳚भ्यो मा ना॒ष्ट्राभ्यः॑ पाहि \textbf{ 15} \newline
                  \newline
                                                                  \textbf{ T.A.7.5.5} \newline
                  सू॒प॒सदा॑ मे भूया॒ मा मा॑ हिꣳसीः ॥ तपो॒ ष्व॑ग्ने॒ अन्त॑राꣳ अ॒मित्रान्॑ ।  तपा॒शꣳस॑मर॒रुषः॒ पर॑स्य । तपा॑ वसो चिकिता॒नो अ॒चित्तान्॑ । वि ते॑ तिष्ठन्ताम॒जरा॑ अ॒यासः॑ ॥ चितः॑ स्थ परि॒चितः॑ । स्वाहा॑ म॒रुद्भिः॒ परि॑श्रयस्व । मा अ॑सि । प्र॒मा अ॑सि । प्र॒ति॒मा अ॑सि \textbf{ 16} \newline
                  \newline
                                                                  \textbf{ T.A.7.5.6} \newline
                  स॒म्मा अ॑सि । वि॒मा अ॑सि । उ॒न्मा अ॑सि । अ॒न्तरि॑क्षस्यान्त॒र्द्धिर॑सि ॥  दिव॒न्तप॑सस्त्रायस्व । आ॒भिर् गी॒र्भिर् यदतो॑ न ऊ॒नम् ।  आप्या॑यय हरिवो॒ वर्द्ध॑मानः । य॒दा स्तो॒तृभ्यो॒ महि॑ गो॒त्रा रु॒जासि॑ । भू॒यि॒ष्ठ॒भाजो॒ अध॑ ते स्याम ॥ शु॒क्रन्ते॑ अ॒न्यद् य॑ज॒तन्ते॑ अ॒न्यत् \textbf{ 17} \newline
                  \newline
                                                                  \textbf{ T.A.7.5.7} \newline
                  विषु॑रूपे॒ अह॑नी॒ द्यौरि॑वासि । विश्वा॒ हि मा॒या अव॑सि स्वधावः । भ॒द्रा ते॑ पूषन्नि॒ह रा॒तिर॑स्तु ॥ अर्.ह॑न् बिभर्.षि॒ साय॑कानि॒ धन्व॑ । अर्.ह॑न्नि॒ष्कं ॅय॑ज॒तं ॅवि॒श्वरू॑पम् । अर्.ह॑न्नि॒दं द॑यसे॒ विश्व॒-मब्भु॑वम् । न वा ओजी॑यो रुद्र॒ त्वद॑स्ति ॥ गा॒य॒त्रम॑सि । त्रैष्टु॑भमसि । जाग॑तमसि ( ) । मधु॒ मधु॒ मधु॑ । \textbf{ 18} \newline
                  \newline
                                                        (अ॒न॒ - क्त्व॒सा॒दी॒ - दु॒त्त॒र॒तः - पा॑हि - प्रति॒मा अ॑सि - यज॒तन्ते॑ अ॒न्यज् - जाग॑तम॒स्येकं॑ च) \textbf{5} \newline \newline
                                \textbf{ T.A.7.6.1} \newline
                  दश॒ प्राची॒र्दश॑ भासि दक्षि॒णा । दश॑ प्र॒तीची॒र्दश॑ भा॒स्युदी॑चीः । दशो॒र्द्ध्वा भा॑सि सुमन॒स्यमा॑नः । स नो॒ रुच॑न्धे॒ह्यहृ॑णीयमानः ॥  अ॒ग्निष्ट्वा॒ वसु॑भिः पु॒रस्ता᳚द्-रोचयतु गाय॒त्रेण॒ छन्द॑सा । स मा॑ रुचि॒तो रो॑चय ॥ इन्द्र॑स्त्वा रु॒द्रैर्-द॑क्षिण॒तो रो॑चयतु॒ त्रैष्टु॑भेन॒ छन्द॑सा ।  स मा॑ रुचि॒तो रो॑चय । वरु॑णस्त्वाऽऽदि॒त्यैः प॒श्चाद्-रो॑चयतु॒ जाग॑तेन॒ छन्द॑सा ।  स मा॑ रुचि॒तो रो॑चय \textbf{ 19} \newline
                  \newline
                                                                  \textbf{ T.A.7.6.2} \newline
                  द्यु॒ता॒नस्त्वा॑ मारु॒तो म॒रुद्भि॑रुत्तर॒तो रो॑चय॒त्वानु॑ष्टुभेन॒ छन्द॑सा । स मा॑ रुचि॒तो रो॑चय । बृह॒स्पति॑स्त्वा॒ विश्वै᳚र् दे॒वैरु॒परि॑ष्टाद्-रोचयतु॒ पाङ्क्ते॑न॒ छन्द॑सा । स मा॑ रुचि॒तो रो॑चय ॥ रो॒चि॒तस्त्वम् दे॑व घर्म दे॒वेष्वसि॑ ।  रो॒चि॒षी॒याहम् म॑नु॒ष्ये॑षु ॥  सम्रा᳚ड् घर्म रुचि॒तस्त्वं ए॒वेष्वायु॑ष्माꣳस्तेज॒स्वी ब्र॑ह्मवर्च॒स्य॑सि । रु॒चि॒तो॑ऽहम् म॑नु॒ष्ये᳚ष्वायु॑ष्माꣳ-स्तेज॒स्वी ब्र॑ह्मवर्च॒सी भू॑यासम् ॥ रुग॑सि । रुच॒म् मयि॑ धेहि ( ) \textbf{ 20} \newline
                  \newline
                                                                  \textbf{ T.A.7.6.3} \newline
                  मयि॒ रुक् ॥ दश॑ पु॒रस्ता᳚द्-रोचसे । दश॑ दक्षि॒णा । दश॑ प्र॒त्यङ्ङ् । दशोदङ्ङ्॑ । दशो॒र्द्ध्वो भा॑सि सुमन॒स्यमा॑नः । स नः॑ सम्रा॒डिष॒मूर्ज॑म् धेहि । वा॒जी वा॒जिने॑ पवस्व । रो॒चि॒तो घ॒र्मो रु॑ची॒य । \textbf{ 21} \newline
                  \newline
                                                        (रो॒च॒य॒ - धे॒हि॒ - +नव॑ च) \textbf{6} \newline \newline
                                \textbf{ T.A.7.7.1} \newline
                  अप॑श्यङ्गो॒पामनि॑पद्यमानम् । आ च॒ परा॑ च प॒थिभि॒श्चर॑न्तम् । स स॒द्ध्रीचीः॒ स विषू॑ची॒र्वसा॑नः । आ व॑रीवर्ति॒ भुव॑नेष्व॒न्तः ॥ अत्र॑ प्रा॒वीः । मधु॒ माद्ध्वी᳚भ्या॒म् मधु॒ माधू॑चीभ्याम् । अनु॑ वाम् दे॒ववी॑तये ॥ सम॒ग्निर॒ग्निना॑ गत । सम् दे॒वेन॑ सवि॒त्रा । सꣳ सूर्ये॑ण रोचते । \textbf{ 22} \newline
                  \newline
                                                                  \textbf{ T.A.7.7.2} \newline
                  स्वाहा॒ सम॒ग्निस्तप॑सा गत । सम् दे॒वेन॑ सवि॒त्रा ।  सꣳ सूर्ये॑णारोचिष्ट ॥ ध॒र्ता दि॒वो विभा॑सि॒ रज॑सः । पृ॒थि॒व्या ध॒र्ता । उ॒रोर॒न्तरि॑क्षस्य ध॒र्ता । ध॒र्ता दे॒वो दे॒वाना᳚म् । अम॑र्त्यस्तपो॒जाः ॥ हृ॒दे त्वा॒ मन॑से त्वा । दि॒वे त्वा॒ सूर्या॑य त्वा \textbf{ 23} \newline
                  \newline
                                                                  \textbf{ T.A.7.7.3} \newline
                  ऊ॒र्द्ध्वमि॒म-म॑द्ध्व॒रङ्कृ॑धि । दि॒वि दे॒वेषु॒ होत्रा॑ यच्छ ॥ विश्वा॑साम् भुवाम् पते । विश्व॑स्य भुवनस्पते । विश्व॑स्य मनसस्पते । विश्व॑स्य वचसस्पते । विश्व॑स्य तपसस्पते । विश्व॑स्य ब्रह्मणस्पते । दे॒व॒श्रूस्त्वन्दे॑व घर्म दे॒वान् पा॑हि ॥ त॒पो॒जां ॅवाच॑म॒स्मे निय॑च्छ देवा॒युव᳚म् । \textbf{ 24} \newline
                  \newline
                                                                  \textbf{ T.A.7.7.4} \newline
                  गर्भो॑ दे॒वाना᳚म् ॥ पि॒ता म॑ती॒नाम् ॥  पतिः॑ प्र॒जाना᳚म् ॥ मतिः॑ कवी॒नाम् ॥  सन्दे॒वो दे॒वेन॑ सवि॒त्राऽय॑तिष्ट । सꣳ सूर्ये॑णारुक्त ॥ आ॒यु॒र्दास्त्वम॒स्मभ्यं॑ घर्म वर्चो॒दा अ॑सि ॥ पि॒ता नो॑ऽसि पि॒ता नो॑ बोध ॥ आ॒यु॒र्द्धास्त॑नू॒धाः प॑यो॒धाः ।  व॒र्चो॒दा व॑रिवो॒दा द्र॑विणो॒दाः \textbf{ 25} \newline
                  \newline
                                                                  \textbf{ T.A.7.7.5} \newline
                  अ॒न्त॒रि॒क्ष॒प्र॒ उ॒रोर्वरी॑यान् । अ॒शी॒महि॑ त्वा॒ मा मा॑ हिꣳसीः ॥ त्वम॑ग्ने गृ॒हप॑तिर्-वि॒शाम॑सि । विश्वा॑सा॒म् मानु॑षीणाम् । श॒तम् पू॒र्भिर्य॑विष्ठ पा॒ह्यꣳह॑सः । स॒मे॒द्धारꣳ॑ श॒तꣳ हिमाः᳚ ।  तं॒ द्रा॒विणꣳ॑ हार्दिवा॒नम् । इ॒हैव रा॒तयः॑ सन्तु ॥ त्वष्टी॑मती ते सपेय । सु॒रेता॒ रेतो॒ दधा॑ना ( ) ।  वी॒रं ॅवि॑देय॒ तव॑ स॒न्दृशि॑ ।  माऽहꣳ रा॒यस्पोषे॑ण॒ वि यो॑षम् । \textbf{ 26} \newline
                  \newline
                                                        (रो॒च॒ते॒-सूर्य॑य त्वा - देवा॒युवं॑ - द्रविणो॒दा - दधा॑ना॒ द्वे च॑) \textbf{7} \newline \newline
                                \textbf{ T.A.7.8.1} \newline
                  दे॒वस्य॑ त्वा सवि॒तुः प्र॑स॒वे । अ॒श्विनो᳚र् बा॒हुभ्या᳚म् ।  पू॒ष्णो हस्ता᳚भ्या॒मा द॑दे ॥ अदि॑त्यै॒ रास्ना॑ऽसि । इड॒ एहि॑ । अदि॑त॒ एहि॑ । सर॑स्व॒त्येहि॑ । असा॒वेहि॑ ।  असा॒वेहि॑ । असा॒वेहि॑ \textbf{ 27} \newline
                  \newline
                                                                  \textbf{ T.A.7.8.2} \newline
                  अदि॑त्या उ॒ष्णीष॑मसि । वा॒युर॑स्यै॒डः । पू॒षा त्वो॒पाव॑सृजतु । अ॒श्विभ्या॒म् प्र दा॑पय । यस्ते॒ स्तनः॑ शश॒यो यो म॑यो॒भूः । येन॒ विश्वा॒ पुष्य॑सि॒ वार्या॑णि । यो र॑त्न॒धा व॑सु॒विद्यः सु॒दत्रः॑ ।  सर॑स्वति॒ तमि॒ह धात॑वेऽकः ॥ उस्र॑ घ॒र्मꣳ शिꣳ॑ष । उस्र॑ घ॒र्मम् पा॑हि \textbf{ 28} \newline
                  \newline
                                                                  \textbf{ T.A.7.8.3} \newline
                  घ॒र्माय॑ शिꣳष । बृह॒स्पति॒स्त्वोप॑सीदतु । दान॑वः स्थ॒ पेर॑वः । वि॒ष्व॒ग्वृतो॒ लोहि॑तेन । अ॒श्विभ्या᳚म् पिन्वस्व । सर॑स्वत्यै पिन्वस्व ।  पू॒ष्णे पि॑न्वस्व । बृह॒स्पत॑ये पिन्वस्व । इन्द्रा॑य पिन्वस्व । इन्द्रा॑य पिन्वस्व \textbf{ 29} \newline
                  \newline
                                                                  \textbf{ T.A.7.8.4} \newline
                  गा॒य॒त्रो॑ऽसि । त्रैष्टु॑भोऽसि । जाग॑तमसि । स॒होर्जो भा॒गेनोप॒ मेहि॑ । इन्द्रा᳚श्विना॒ मधु॑नः सार॒घस्य॑ । घ॒र्मम् पा॑त वसवो॒ यज॑ता॒ वट् । स्वाहा᳚ त्वा॒ सूर्य॑स्य र॒श्मये॑ वृष्टि॒वन॑ये जुहोमि । मधु॑ ह॒विर॑सि । सूर्य॑स्य॒ तप॑स्तप । द्यावा॑पृथि॒वीभ्यां᳚ त्वा॒ परि॑गृह्णामि ( ) \textbf{ 30} \newline
                  \newline
                                                                  \textbf{ T.A.7.8.5} \newline
                  अ॒न्तरि॑क्षेण॒ त्वोप॑यच्छामि । दे॒वानां᳚ त्वा पितृ॒णामनु॑मतो॒ भर्तुꣳ॑ शकेयम् । तेजो॑ऽसि ।  तेजोऽनु॒प्रेहि॑ । दि॒वि॒स्पृङ्मा मा॑ हिꣳसीः ।  अ॒न्त॒रि॒क्ष॒स्पृङ्मा मा॑ हिꣳसीः । पृ॒थि॒वि॒स्पृङ्मा मा॑ हिꣳसीः ।  सुव॑रसि॒ सुव॑र्मे यच्छ ।  दिवं॑ ॅयच्छ दि॒वो मा॑ पाहि । \textbf{ 31} \newline
                  \newline
                                                        (एहि॑ - पाहि - पिन्वस्व - गृह्णामि॒ - +नव॑ च) \textbf{8} \newline \newline
                                \textbf{ T.A.7.9.1} \newline
                  स॒मु॒द्राय॑ त्वा॒ वाता॑य॒ स्वाहा᳚ । स॒लि॒लाय॑ त्वा॒ वाता॑य॒ स्वाहा᳚ । अ॒ना॒धृ॒ष्याय॑ त्वा॒ वाता॑य॒ स्वाहा᳚ । अ॒प्र॒ति॒धृ॒ष्याय॑ त्वा॒ वाता॑य॒ स्वाहा᳚ ।  अ॒व॒स्यवे᳚ त्वा॒ वाता॑य॒ स्वाहा᳚ । दुव॑स्वते त्वा॒ वाता॑य॒ स्वाहा᳚ । शिमि॑द्वते त्वा॒ वाता॑य॒ स्वाहा᳚ । अ॒ग्नये᳚ त्वा॒ वसु॑मते॒ स्वाहा᳚ ।  सोमा॑य त्वा रु॒द्रव॑ते॒ स्वाहा᳚ ।  वरु॑णाय त्वाऽऽदि॒त्यव॑ते॒ स्वाहा᳚ \textbf{ 32} \newline
                  \newline
                                                                  \textbf{ T.A.7.9.2} \newline
                  बृह॒स्पत॑ये त्वा वि॒श्वदे᳚व्यावते॒ स्वाहा᳚ ।  स॒वि॒त्रे त्व॑र्भु॒मते॑ विभु॒मते᳚ प्रभु॒मते॒ वाज॑वते॒ स्वाहा᳚ । य॒माय॒ त्वाऽङ्गि॑रस्वते पितृ॒मते॒ स्वाहा᳚ ॥ विश्वा॒ आशा॑ दक्षिण॒सत् । विश्वा᳚न् दे॒वान॑याडि॒ह । स्वाहा॑कृतस्य घ॒र्मस्य॑ । मधोः᳚ पिबतमश्विना ।  स्वाहा॒ऽग्नये॑ य॒ज्ञिया॑य । शं ॅयजु॑र्भिः ॥  अश्वि॑ना घ॒र्मम् पा॑तꣳ हार्दिवा॒नम् \textbf{ 33} \newline
                  \newline
                                                                  \textbf{ T.A.7.9.3} \newline
                  अह॑र्-दि॒वाभि॑-रू॒तिभिः॑ । अनु॑ वा॒न् द्यावा॑पृथि॒वी मꣳ॑साताम् । स्वाहेन्द्रा॑य ॥ स्वाहेन्द्रा॒वट् । घ॒र्मम॑पातमश्विना हार्दिवा॒नम् ।  अह॑र्-दि॒वाभि॑रू॒तिभिः॑ । अनु॑ वा॒न् द्यावा॑पृथि॒वी अ॑मꣳसाताम् । तम् प्रा॒व्यं॑ ॅयथा॒ वट् । नमो॑ दि॒वे । नमः॑ पृथि॒व्यै ( ) \textbf{ 34} \newline
                  \newline
                                                                  \textbf{ T.A.7.9.4} \newline
                  दि॒वि धा॑ इ॒मं ॅय॒ज्ञ्म् । य॒ज्ञ्मि॒मं दि॒वि धाः᳚ । दिव॑ङ्गच्छ । अ॒न्तरि॑क्षङ्गच्छ । पृ॒थि॒वीङ्ग॑च्छ । पञ्च॑ प्र॒दिशो॑ गच्छ । दे॒वान् घ॑र्म॒पान् ग॑च्छ । पि॒तॄन् घ॑र्म॒पान् ग॑च्छ । \textbf{ 35} \newline
                  \newline
                                                        (आ॒दि॒त्यव॑ते॒ स्वाहा॑-हार्दिवा॒नम्-पृ॑थि॒व्या-+अ॒ष्टौ च॑) \textbf{9} \newline \newline
                                \textbf{ T.A.7.10.1} \newline
                  इ॒षे पी॑पिहि । ऊ॒र्जे पी॑पिहि । ब्रह्म॑णे पीपिहि । क्ष॒त्राय॑ पीपिहि । अ॒द्भ्यः पी॑पिहि । ओष॑धीभ्यः पीपिहि । वन॒स्पति॑भ्यः पीपिहि । द्यावा॑पृथि॒वीभ्या᳚म् पीपिहि । सु॒भू॒ताय॑ पीपिहि । ब्र॒ह्म॒व॒र्च॒साय॑ पीपिहि \textbf{ 36} \newline
                  \newline
                                                                  \textbf{ T.A.7.10.2} \newline
                  यज॑मानाय पीपिहि । मह्य॒ञ्ज्यैष्ठ्या॑य पीपिहि ॥ त्विष्यै᳚ त्वा । द्यु॒म्नाय॑ त्वा । इ॒न्द्रि॒याय॑ त्वा॒ भूत्यै᳚ त्वा । धर्मा॑ऽसि सु॒धर्मा मे᳚ न्य॒स्मे । ब्रह्मा॑णि धारय । क्ष॒त्राणि॑ धारय । विश॑न्धारय ।  नेत्त्वा॒ वातः॑ स्क॒न्दया᳚त् \textbf{ 37} \newline
                  \newline
                                                                  \textbf{ T.A.7.10.3} \newline
                  अ॒मुष्य॑ त्वा प्रा॒णे सा॑दयामि । अ॒मुना॑ स॒ह नि॑र॒र्त्थङ्ग॑च्छ । यो᳚ऽस्मान् द्वेष्टि॑ ।  यञ्च॑ व॒यं द्वि॒ष्मः ॥ पू॒ष्णे शर॑से॒ स्वाहा᳚ ।  ग्राव॑भ्यः॒ स्वाहा᳚ । प्र॒ति॒रेभ्यः॒ स्वाहा᳚ । द्यावा॑पृथि॒वीभ्याꣳ॒॒ स्वाहा᳚ ।  पि॒तृभ्यो॑ घर्म॒पेभ्यः॒ स्वाहा᳚ ॥  रु॒द्राय॑ रु॒द्रहो᳚त्रे॒ स्वाहा᳚ \textbf{ 38} \newline
                  \newline
                                                                  \textbf{ T.A.7.10.4} \newline
                  अह॒र् ज्योतिः॑ के॒तुना॑ जुषताम् । सु॒ज्यो॒तिर् ज्योति॑षाꣳ॒॒ स्वाहा᳚ । रात्रि॒र्ज्योतिः॑ के॒तुना॑ जुषताम् । सु॒ज्यो॒तिर्-ज्योति॑षाꣳ॒॒ स्वाहा᳚ । अपी॑परो॒ माऽह्नो॒ रात्रि॑यै मा पाहि । ए॒षा ते॑ अग्ने स॒मित् । तया॒ समि॑द्ध्यस्व । आयु॑र्मे दाः । वर्च॑सा माऽञ्जीः । अपी॑परो मा॒ रात्रि॑या॒ अह्नो॑ मा पाहि \textbf{ 39} \newline
                  \newline
                                                                  \textbf{ T.A.7.10.5} \newline
                  ए॒षा ते॑ अग्ने स॒मित् । तया॒ समि॑द्ध्यस्व । आयु॑र्मे दाः । वर्च॑सा माऽञ्जीः । अ॒ग्निर्-ज्योति॒र्-ज्योति॑र॒ग्निः स्वाहा᳚ ।  सूर्यो॒ ज्योति॒र्-ज्योतिः॒ सूर्यः॒ स्वाहा᳚ । भूः स्वाहा᳚ । हु॒तꣳ ह॒विः । मधु॑ ह॒विः । इन्द्र॑तमे॒ऽग्नौ ( ) \textbf{ 40} \newline
                  \newline
                                                                  \textbf{ T.A.7.10.6} \newline
                  पि॒ता नो॑ऽसि॒ मा मा॑ हिꣳसीः । अ॒श्याम॑ ते देवघर्म ।  मधु॑मतो॒ वाज॑वतः पितु॒मतः॑ । अङ्गि॑रस्वतः स्वधा॒विनः॑ । अ॒शी॒महि॑ त्वा॒ मा मा॑ हिꣳसीः ॥ स्वाहा᳚ त्वा॒ सूर्य॑स्य र॒श्मिभ्यः॑ । स्वाहा᳚ त्वा॒ नक्ष॑त्रेभ्यः । \textbf{ 41} \newline
                  \newline
                                                        (ब्र॒ह्म॒व॒र्च॒साय॑ पीपिहि - स्क॒न्दया᳚द् - रु॒द्राय॑ रु॒द्रहो᳚त्रे॒ स्वाहा - ऽह्नो॑ मा पाह् - य॒ग्नौ - +स॒प्त च॑) \textbf{10} \newline \newline
                                \textbf{ T.A.7.11.1} \newline
                  घर्म॒ या ते॑ दि॒वि शुक् । या गा॑य॒त्रे छन्द॑सि । या ब्रा᳚ह्म॒णे । या ह॑वि॒र्द्धाने᳚ । तान्त॑ ए॒तेनाव॑यजे॒ स्वाहा᳚ ॥ घर्म॒ या ते॒ऽन्तरि॑क्षे॒ शुक् ।  या त्रैष्टु॑भे॒ छन्द॑सि । या रा॑ज॒न्ये᳚ । याऽऽग्नी᳚द्ध्रे ।  तान्त॑ ए॒तेनाव॑यजे॒ स्वाहा᳚ \textbf{ 42} \newline
                  \newline
                                                                  \textbf{ T.A.7.11.2} \newline
                  घर्म॒ या ते॑ पृथि॒व्याꣳ शुक् । या जाग॑ते॒ छन्द॑सि । या वैश्ये᳚ । या सद॑सि । तान्त॑ ए॒तेनाव॑यजे॒ स्वाहा᳚ ॥ अनु॑नो॒ऽद्यानु॑मतिः{1} । अन्विद॑नुमते॒ त्वम् {2} । दि॒वस्त्वा॑ पर॒स्पायाः᳚ । अ॒न्तरि॑क्षस्य त॒नुवः॑ पाहि । पृ॒थि॒व्यास्त्वा॒ धर्म॑णा \textbf{ 43} \newline
                  \newline
                                                                  \textbf{ T.A.7.11.3} \newline
                  व॒यमनु॑क्रामाम सुवि॒ताय॒ नव्य॑से ॥ ब्रह्म॑णस्त्वा पर॒स्पायाः᳚ । क्ष॒त्रस्य॑ त॒नुवः॑ पाहि । वि॒शस्त्वा॒ धर्म॑णा । व॒यमनु॑क्रामाम सुवि॒ताय॒ नव्य॑से ॥ प्रा॒णस्य॑ त्वा पर॒स्पायै᳚ । चक्षु॑षस्त॒नुवः॑ पाहि । श्रोत्र॑स्य त्वा॒ धर्म॑णा । व॒यमनु॑क्रामाम सुवि॒ताय॒ नव्य॑से ॥ व॒ल्गुर॑सि शं॒ ॅयुधा॑याः \textbf{ 44} \newline
                  \newline
                                                                  \textbf{ T.A.7.11.4} \newline
                  शिशु॒र्-जन॑धायाः ॥ शञ्च॒ वक्षि॒ परि॑ च॒ वक्षि॑ ।  चतुः॑ स्रक्ति॒र्नाभि॑र्. ऋ॒तस्य॑ । सदो॑ वि॒श्वायुः॒ शर्म॑ स॒प्रथाः᳚ । अप॒ द्वेषो॒ अप॒ ह्वरः॑ । अ॒न्यद्व्र॑तस्य सश्चिम । घर्मै॒तत् तेऽन्न॑मे॒तत् पुरी॑षम् । तेन॒ वर्द्ध॑स्व॒ चा च॑ प्यायस्व । व॒र्द्धि॒षी॒महि॑ च व॒यम् । आ च॑ प्यासिषी॒महि॑ । \textbf{ 45} \newline
                  \newline
                                                                  \textbf{ T.A.7.11.5} \newline
                  रन्ति॒र्-नामा॑सि दि॒व्यो ग॑न्ध॒र्वः । तस्य॑ ते प॒द्वद्ध॑वि॒र्द्धान᳚म् । अ॒ग्निरद्ध्य॑क्षाः । रु॒द्रोऽधि॑पतिः ॥ सम॒हमायु॑षा । सम् प्रा॒णेन॑ ।  सं ॅवर्च॑सा । सम् पय॑सा । सङ्गौ॑प॒त्येन॑ । सꣳ रा॒यस्पोषे॑ण । \textbf{ 46} \newline
                  \newline
                                                                  \textbf{ T.A.7.11.6} \newline
                  व्य॑सौ । यो᳚ऽस्मान् द्वेष्टि॑ । यञ्च॑ व॒यं द्वि॒ष्मः ॥  अचि॑क्रद॒द्वृषा॒ हरिः॑ । म॒हान्-मि॒त्रो न द॑र्.श॒तः । सꣳ सूर्ये॑ण रोचते ॥ चिद॑सि समु॒द्रयो॑निः । इन्दु॒र् दक्षः॑ श्ये॒न ऋ॒तावा᳚ । हिर॑ण्यपक्षः शकु॒नो भु॑र॒ण्युः । म॒हान्थ् स॒धस्थे᳚ ध्रु॒व आनिष॑त्तः \textbf{ 47} \newline
                  \newline
                                                                  \textbf{ T.A.7.11.7} \newline
                  नम॑स्ते अस्तु॒ मा मा॑ हिꣳसीः ॥ वि॒श्वाव॑सुꣳ सोम गन्ध॒र्वम् । आपो॑ ददृ॒शुषीः᳚ । तदृ॒तेना॒ व्या॑यन्न् । तद॒न्ववै᳚त् । इन्द्रो॑ रारहा॒ण आ॑साम् । परि॒ सूर्य॑स्य परि॒धीꣳ र॑पश्यत् ॥ वि॒श्वाव॑सुर॒भि तन्नो॑ गृणातु । दि॒व्यो ग॑न्ध॒र्वो रज॑सो वि॒मानः॑ ।  यद्वा॑ घा स॒त्यमु॒त यन्न वि॒द्म \textbf{ 48} \newline
                  \newline
                                                                  \textbf{ T.A.7.11.8} \newline
                  धियो॑ हिन्वा॒नो धिय॒ इन्नो॑ अव्यात् ॥ सस्नि॑मविन्द॒च्चर॑णे न॒दीना᳚म् ।  अपा॑वृणो॒द्-दुरो॒ अश्म॑व्रजानाम् । प्रासा᳚न् गन्ध॒र्वो अ॒मृता॑नि वोचत् । इन्द्रो॒ दक्ष॒म् परि॑जानाद॒हीन᳚म् ॥ ए॒तत्त्वम् दे॑व घर्म दे॒वो दे॒वानुपा॑गाः ।  इ॒दम॒हम् म॑नु॒ष्यो॑ मनु॒ष्यान्॑ । सोम॑पी॒थानु॒ मेहि॑ । स॒ह प्र॒जया॑ स॒ह रा॒यस्पोषे॑ण ॥  सु॒मि॒त्रा न॒ आप॒ ओष॑धयः सन्तु ( ) \textbf{ 49} \newline
                  \newline
                                                                  \textbf{ T.A.7.11.9} \newline
                  दु॒र्मि॒त्रास्तस्मै॑ भूयासुः । यो᳚ऽस्मान् द्वेष्टि॑ ।  यञ्च॑ व॒यं द्वि॒ष्मः ॥ उद्व॒यन्तम॑स॒स्परि॑ {3} । उदु॒त्यं चि॒त्रम् {4} । इ॒ममू॒ षु त्यम॒स्मभ्यꣳ॑ स॒निम् । गा॒य॒त्रन्नवी॑याꣳसम् ।  अग्ने॑ दे॒वेषु॒ प्रवो॑चः । \textbf{ 50} \newline
                  \newline
                                                        (याऽऽग्नी᳚द्ध्रे॒ तान्ता॑ ए॒तेनाव॑यजे॒ स्वाहा॒ - धर्म॑णा - शं॒ ॅयुधा॑याः - प्यासिषी॒महि॒-पोषे॑ण॒-निष॑त्तो-वि॒द्म-स॑न्-+त्व॒ष्टौ च॑) \textbf{11} \newline \newline
                                \textbf{ T.A.7.12.1} \newline
                  म॒ही॒नाम् पयो॑ऽसि॒ विहि॑तं देव॒त्रा ।  ज्यो॒ति॒र्भा अ॑सि॒ वन॒स्पती॑ना॒मोष॑धीनाꣳ॒॒ रसः॑ ।  वा॒जिन॑म् त्वा वा॒जिनोऽव॑ नयामः ।  ऊ॒र्द्ध्वम् मनः॑ सुव॒र्गम् । \textbf{ 51} \newline
                  \newline
                                                        (णॊ कॊर्वै fऒर् थिस् अनुवक्कम् ) \textbf{12} \newline \newline
                                \textbf{ T.A.7.13.1} \newline
                  अस्का॒न् द्यौः पृ॑थि॒वीम् । अस्का॑नृष॒भो युवा॒ गाः ।  स्क॒न्नेमा विश्वा॒ भुव॑ना । स्क॒न्नो य॒ज्ञ्ः प्रज॑नयतु ॥ अस्का॒नज॑नि॒ प्राज॑नि । आ स्क॒न्नाज्जा॑यते॒ वृषा᳚ ।  स्क॒न्नात् प्रज॑निषीमहि । \textbf{ 52} \newline
                  \newline
                                                        (णॊ कॊर्वै fऒर् थिस् अनुवक्कम् ) \textbf{13} \newline \newline
                                \textbf{ T.A.7.14.1} \newline
                  या पु॒रस्ता᳚द् वि॒द्युदाप॑तत् । तान्त॑ ए॒तेनाव॑ यजे॒ स्वाहा᳚ । या द॑क्षिण॒तः । या प॒श्चात् । योत्त॑र॒तः । योपरि॑ष्टाद् वि॒द्युदाप॑तत् । तान्त॑ ए॒तेनाव॑ यजे॒ स्वाहा᳚ । \textbf{ 53} \newline
                  \newline
                                                        (णॊ कॊर्वै fऒर् थिस् अनुवक्कम् ) \textbf{14} \newline \newline
                                \textbf{ T.A.7.15.1} \newline
                  प्रा॒णाय॒ स्वाहा᳚ व्या॒नाय॒ स्वाहा॑ ऽपा॒नाय॒ स्वाहा᳚ ।  चक्षु॑षे॒ स्वाहा॒ श्रोत्रा॑य॒ स्वाहा᳚ ।  मन॑से॒ स्वाहा॑ वा॒चे सर॑स्वत्यै॒ स्वाहा᳚ । \textbf{ 54} \newline
                  \newline
                                                        (णॊ कॊर्वै fऒर् थिस् अनुवक्कम् ) \textbf{15} \newline \newline
                                \textbf{ T.A.7.16.1} \newline
                  पू॒ष्णे स्वाहा॑ पू॒ष्णे शर॑से॒ स्वाहा᳚ ।  पू॒ष्णे प्र॑प॒त्थ्या॑य॒ स्वाहा॑ पू॒ष्णे न॒रन्धि॑षाय॒ स्वाहा᳚ । पू॒ष्णेऽङ्घृ॑णये॒ स्वाहा॑ पू॒ष्णे न॒रुणा॑य॒ स्वाहा᳚ ।  पू॒ष्णे सा॑के॒ताय॒ स्वाहा᳚ । \textbf{ 55} \newline
                  \newline
                                                        (णॊ कॊर्वै fऒर् थिस् अनुवक्कम् ) \textbf{16} \newline \newline
                                \textbf{ T.A.7.17.1} \newline
                  उद॑स्य॒ शुष्मा᳚द् भा॒नुर्नार्त॒ बिभ॑र्ति । भा॒रम् पृ॑थि॒वी न भूम॑ । प्र शु॒क्रैतु॑ दे॒वी म॑नी॒षा । अ॒स्मथ् सुत॑ष्टो॒ रथो॒ न वा॒जी । अर्च॑न्त॒ एके॒ महि॒ साम॑ मन्वत । तेन॒ सूर्य॑मधारयन्न् ।  तेन॒ सूर्य॑मरोचयन्न् । घ॒र्मः शिर॒स्तद॒यम॒ग्निः ।  पुरी॑षमसि॒ संप्रि॑यम् प्र॒जया॑ प॒शुभि॑र्भुवत् । प्र॒जापति॑स्त्वा सादयतु । तया॑ दे॒वत॑या ऽङ्गिर॒स्वद् ध्रु॒वा सी॑द । \textbf{ 56} \newline
                  \newline
                                                        (णॊ कॊर्वै fऒर् थिस् अनुवक्कम्) \textbf{17} \newline \newline
                                \textbf{ T.A.7.18.1} \newline
                  यास्ते॑ अग्न आ॒र्द्रा योन॑यो॒ याः कु॑ला॒यिनीः᳚ ।  ये ते॑ अग्न॒ इन्द॑वो॒ या उ॒ नाभ॑यः । यास्ते॑ अग्ने त॒नुव॒ ऊर्जो॒ नाम॑ । ताभि॒स्त्वमु॒भयी॑भिः सम्ॅविदा॒नः । प्र॒जाभि॑रग्ने॒ द्रवि॑णे॒ह सी॑द ।  प्र॒जाप॑तिस्त्वा सादयतु । तया॑ दे॒वत॑याऽङ्गिर॒स्वद् ध्रु॒वा सी॑द । \textbf{ 57} \newline
                  \newline
                                                        (णॊ कॊर्वै fऒर् थिस् अनुवक्कम्) \textbf{18} \newline \newline
                                \textbf{ T.A.7.19.1} \newline
                  अ॒ग्निर॑सि वैश्वान॒रो॑ऽसि । स॒म्ॅव॒थ्स॒रो॑ऽसि परिवथ्स॒रो॑ऽसि । इ॒दा॒व॒थ्स॒रो॑ऽसीदुवथ्स॒रो॑ऽसि । इ॒द्व॒थ्स॒रो॑ऽसि वथ्स॒रो॑ऽसि । तस्य॑ ते वस॒न्तः शिरः॑ । ग्री॒ष्मो दक्षि॑णः प॒क्षः । व॒र्॒.षाः पुच्छ᳚म् ।  श॒रदुत्त॑रः प॒क्षः । हे॒म॒न्तो मद्ध्य᳚म् । पू॒र्व॒प॒क्षाश्चित॑यः ( ) । अ॒प॒र॒प॒क्षाः पुरी॑षम् । अ॒हो॒रा॒त्राणीष्ट॑काः । तस्य॑ ते॒ मासा᳚श्चार्द्धमा॒साश्च॑ कल्पन्ताम् ।  ऋ॒तव॑स्ते कल्पन्ताम् । स॒म्ॅव॒थ्स॒रस्ते॑ कल्पताम् ।  अ॒हो॒रा॒त्राणि॑ ते कल्पन्ताम् । एति॒ प्रेति॒ वीति॒ समित्युदिति॑ ।  प्र॒जाप॑तिस्त्वा सादयतु । तया॑ दे॒वत॑याऽङ्गिर॒स्वद् ध्रु॒वा सी॑द । \textbf{ 58} \newline
                  \newline
                                                        (चित॑यो॒ नव॑ च) \textbf{19} \newline \newline
                                \textbf{ T.A.7.20.1} \newline
                  भूर्भुव॒स्सुवः॑ । ऊ॒र्द्ध्व ऊ॒ षुण॑ ऊ॒तये᳚ । ऊ॒र्द्ध्वो नः॑ पा॒ह्यꣳह॑सः ॥  वि॒धुन्द॑द्रा॒णꣳ सम॑ने बहू॒नाम् । युवा॑नꣳ॒॒ सन्त॑म् पलि॒तो ज॑गार । दे॒वस्य॑ पश्य॒ काव्य॑म् महि॒त्वाऽद्या म॒मार॑ । स ह्यः॒ समा॑न ॥  यदृ॒ते चि॑दभि॒श्रिषः॑ । पु॒रा ज॒र्तृभ्य॑ आ॒तृदः॑ ।  सन्धा॑ता स॒न्धिम् म॒घवा॑ पुरो॒वसुः॑ \textbf{ 59} \newline
                  \newline
                                                                  \textbf{ T.A.7.20.2} \newline
                  निष्क॑र्ता॒ विह्रु॑त॒म् पुनः॑ ॥ पुन॑रू॒र्जा{5}, स॒ह र॒य्या {6} । मा नो॑ घर्म व्यथि॒तो वि॑व्यथो नः । मा नः॒ पर॒मध॑र॒म्मा रजो॑ऽनैः । मोष्व॑स्माꣳ स्तम॑स्यन्त॒रा धाः᳚ । मा रु॒द्रिया॑सो अ॒भिगु॑र्वृ॒धानः॑ ॥ मा नः॒ क्रतु॑भिर्. हीडि॒तेभि॑र॒स्मान् । द्विषा॑ सुनीते॒ मा परा॑दाः ।  मा नो॑ रु॒द्रो निर्.ऋ॑ति॒र्मा नो॒ अस्ता᳚ ।  मा द्यावा॑पृथि॒वी ही॑डिषाताम् । \textbf{ 60} \newline
                  \newline
                                                                  \textbf{ T.A.7.20.3} \newline
                  उप॑ नो मित्रावरुणावि॒हाव॑तम् । अ॒न्वादी᳚द्ध्याथामि॒ह नः॑ सखाया । आ॒दि॒त्याना॒म् प्रसि॑तिर्.हे॒तिः । उ॒ग्रा श॒तापा᳚ष्ठा घ॒ विषा॒ परि॑णो वृणक्तु ॥  इ॒मम् मे॑ वरुण॒{7}, तत्त्वा॑ यामि{8} । त्वन्नो॑ अग्ने॒{9},  स त्वन्नो॑ अग्ने{10} । त्वम॑ग्ने अ॒यासि॑{11) ॥  उद्व॒यन्तम॑स॒स्परि॑ {12} । उदु॒त्यं{13}, चि॒त्रम् {14} ।  वयः॑ सुप॒र्णाः {15} ( ) । \textbf{ 61} \newline
                  \newline
                                                        (पु॒रो॒वसु॑र्.हीडिषाताꣳ सुप॒र्णाः) \textbf{20} \newline \newline
                                \textbf{ T.A.7.21.1} \newline
                  भूर्भुव॒स्सुवः॑ । मयि॒ त्यदि॑न्द्रि॒यम् म॒हत् । मयि॒ दक्षो॒ मयि॒ क्रतुः॑ । मयि॑ धायि सु॒वीर्य᳚म् । त्रिशु॑ग्घ॒र्मो विभा॑तु मे । आकू᳚त्या॒ मन॑सा स॒ह ।  वि॒राजा॒ ज्योति॑षा स॒ह । य॒ज्ञेन॒ पय॑सा स॒ह । ब्रह्म॑णा॒ तेज॑सा स॒ह । क्ष॒त्रेण॒ यश॑सा स॒ह ( ) । स॒त्येन॒ तप॑सा स॒ह । तस्य॒ दोह॑मशीमहि ।  तस्य॑ सु॒म्नम॑शीमहि । तस्य॑ भ॒क्षम॑शीमहि ।  तस्य॑ त॒ इन्द्रे॑ण पी॒तस्य॒ मधु॑मतः ।  उप॑हूत॒स्योप॑हूतो भक्षयामि । \textbf{ 62} \newline
                  \newline
                                                        (यश॑सा स॒ह षट्च॑) \textbf{21} \newline \newline
                                \textbf{ T.A.7.22.1} \newline
                  यास्ते॑ अग्ने घो॒रास्त॒नुवः॑ । क्षुच्च॒ तृष्णा॑ च । अस्नु॒क्चाना॑हुतिश्च । अ॒श॒न॒या च॑ पिपा॒सा च॑ । से॒दिश्चाम॑तिश्च । ए॒तास्ते॑ अग्ने घो॒रास्त॒नुवः॑ ।  ताभि॑र॒मुङ्ग॑च्छ । यो᳚ऽस्मान् द्वेष्टि॑ ।  यञ्च॑ व॒यं द्वि॒ष्मः । \textbf{ 63} \newline
                  \newline
                                                        (णॊ कॊर्वै fऒर् थिस् अनुवक्कम् ) \textbf{22} \newline \newline
                                \textbf{ T.A.7.23.1} \newline
                  स्निक्च॒ स्नीहि॑तिश्च॒ स्निहि॑तिश्च । उ॒ष्णा च॑ शी॒ता च॑ ।  उ॒ग्रा च॑ भी॒मा च॑ । स॒दाम्नी॑ से॒दिरनि॑रा । ए॒तास्ते॑ अग्ने घो॒रास्त॒नुवः॑ । ताभि॑र॒मुङ्ग॑च्छ । यो᳚ऽस्मान् द्वेष्टि॑ । यञ्च॑ व॒यं द्वि॒ष्मः । \textbf{ 64} \newline
                  \newline
                                                        (णॊ कॊर्वै fऒर् थिस् अनुवक्कम् ) \textbf{23} \newline \newline
                                \textbf{ T.A.7.24.1} \newline
                  धुनि॑श्च ध्वा॒न्तश्च॑ ध्व॒नश्च॑ ध्व॒नयꣳ॑श्च ।  नि॒लि॒म्पश्च॑ विलि॒म्पश्च॑ विक्षि॒पः । \textbf{ 65} \newline
                  \newline
                                                        (णॊ कॊर्वै fऒर् थिस् अनुवक्कम् ) \textbf{24} \newline \newline
                                \textbf{ T.A.7.25.1} \newline
                  उ॒ग्रश्च॒ धुनि॑श्च ध्वा॒न्तश्च॑ ध्व॒नश्च॑ ध्व॒नयꣳ॑श्च । स॒ह॒स॒ह्वाꣳश्च॒ सह॑मानश्च॒ सह॑स्वाꣳश्च॒ सही॑याꣳश्च ।  एत्य॒ प्रेत्य॑ विक्षि॒पः । \textbf{ 66} \newline
                  \newline
                                                        (णॊ कॊर्वै fऒर् थिस् अनुवक्कम् ) \textbf{25} \newline \newline
                                \textbf{ T.A.7.26.1} \newline
                  अ॒हो॒रा॒त्रे त्वोदी॑रयताम् । अ॒र्द्ध॒मा॒सास्त्वोदीं᳚ जयन्तु । मासा᳚स्त्वा श्रपयन्तु । ऋ॒तव॑स्त्वा पचन्तु ।  स॒म्ॅव॒थ्स॒रस्त्वा॑ हन्त्वसौ । \textbf{ 67} \newline
                  \newline
                                                        (णॊ कॊर्वै fऒर् थिस् अनुवक्कम् ) \textbf{26} \newline \newline
                                \textbf{ T.A.7.27.1} \newline
                  खट् फड् ज॒हि । छि॒न्धी भि॒न्धी ह॒न्धी कट् ।  इति॒ वाचः॑ क्रूरा॒णि । \textbf{ 68} \newline
                  \newline
                                                        (णॊ कॊर्वै fऒर् थिस् अनुवक्कम्) \textbf{27} \newline \newline
                                \textbf{ T.A.7.28.1} \newline
                  विगा इ॑न्द्र वि॒चर᳚न्थ् स्पाशयस्व । स्व॒पन्त॑मिन्द्र पशु॒मन्त॑मिच्छ । वज्रे॑णा॒मुम् बो॑धय दुर्वि॒दत्र᳚म् । स्व॒प॒तो᳚ऽस्य॒ प्रह॑र॒ भोज॑नेभ्यः ॥  अग्ने॑ अ॒ग्निना॒ सम्ॅव॑दस्व । मृत्यो॑ मृ॒त्युना॒ सम्ॅव॑दस्व । नम॑स्ते अस्तु भगवः ॥ स॒कृत्ते॑ अग्ने॒ नमः॑ । द्विस्ते॒ नमः॑ । त्रिस्ते॒ नमः॑ ( ) । च॒तुस्ते॒ नमः॑ । प॒ञ्च॒कृत्व॑स्ते॒ नमः॑ ।  द॒श॒कृत्व॑स्ते॒ नमः॑ । श॒त॒कृत्व॑स्ते॒ नमः॑ । आ॒स॒ह॒स्र॒कृत्व॑स्ते॒ नमः॑ । अ॒प॒रि॒मि॒त॒कृत्व॑स्ते॒ नमः॑ । नम॑स्ते अस्तु॒ मा मा॑ हिꣳसीः । \textbf{ 69} \newline
                  \newline
                                                        (त्रिस्ते॒ नमः॑ स॒प्त च॑) \textbf{28} \newline \newline
                                \textbf{ } \newline
                   \textbf{ 0} \newline
                  \newline
                                                        (4) (णॊ कॊर्वै fऒर् थिस् अनुवक्कम् ) \textbf{29} \newline \newline
                                \textbf{ T.A.7.30.1} \newline
                  यदे॒तद्-वृ॑क॒सो भू॒त्वा । वाग् दे᳚व्यभि॒राय॑सि । द्वि॒षन्त॑म् मे॒ऽभिरा॑य । तम् मृ॑त्यो मृ॒त्यवे॑ नय । स आर्त्याऽऽर्ति॒मार्च्छ॑तु । \textbf{ 71} \newline
                  \newline
                                                        (णॊ कॊर्वै fऒर् थिस् अनुवक्कम् ) \textbf{30} \newline \newline
                                \textbf{ T.A.7.31.1} \newline
                  यदी॑षि॒तो यदि॑ वा स्वका॒मी । भ॒येड॑को॒ वद॑ति॒ वाच॑मे॒ताम् । तामि॑न्द्रा॒ग्नी ब्रह्म॑णा सम्ॅविदा॒नौ ।  शि॒वाम॒स्मभ्यं॑ कृणुतङ्गृ॒हेषु॑ । \textbf{ 72} \newline
                  \newline
                                                        (णॊ कॊर्वै fऒर् थिस् अनुवक्कम् ) \textbf{31} \newline \newline
                                \textbf{ T.A.7.32.1} \newline
                  दीर्घ॑मुखि॒ दुर्.ह॑णु । मा स्म॑ दक्षिण॒तो व॑दः ।  यदि॑ दक्षिण॒तो वदा᳚द् द्वि॒षन्त॒म् मेऽव॑ बाधासै । \textbf{ 73} \newline
                  \newline
                                                        (णॊ कॊर्वै fऒर् थिस् अनुवक्कम् ) \textbf{32} \newline \newline
                                \textbf{ T.A.7.33.1} \newline
                  इ॒त्थादुलू॑क॒ आप॑प्तत् । हि॒र॒ण्या॒क्षो अयो॑मुखः ।  रक्ष॑सान्दू॒त आग॑तः । तमि॒तो ना॑शयाग्ने । \textbf{ 74} \newline
                  \newline
                                                        (णॊ कॊर्वै fऒर् थिस् अनुवक्कम् ) \textbf{33} \newline \newline
                                \textbf{ T.A.7.34.1} \newline
                  यदे॒तद् भू॒तान्य॑न्वा॒विश्य॑ । दैवीं॒ ॅवाचं॑ ॅव॒दसि॑ । द्वि॒षतो॑ नः॒ परा॑वद । तान् मृ॑त्यो मृ॒त्यवे॑ नय । त आर्त्याऽऽर्ति॒मार्च्छ॑न्तु ।  अ॒ग्निना॒ऽग्निः सम्ॅव॑दताम् । \textbf{ 75} \newline
                  \newline
                                                        (णॊ कॊर्वै fऒर् थिस् अनुवक्कम् ) \textbf{34} \newline \newline
                                \textbf{ T.A.7.35.1} \newline
                  प्र॒सार्य॑ स॒क्थ्यौ॑ पत॑सि । स॒व्यमक्षि॑ नि॒पेपि॑ च ।  मेह क॑स्यच॒नाम॑मत् । \textbf{ 76} \newline
                  \newline
                                                        (णॊ कॊर्वै fऒर् थिस् अनुवक्कम् ) \textbf{35} \newline \newline
                                \textbf{ T.A.7.36.1} \newline
                  अत्रि॑णा त्वा क्रिमे हन्मि । कण्वे॑न ज॒मद॑ग्निना ।  वि॒श्वाव॑सो॒र् ब्रह्म॑णा ह॒तः । क्रिमी॑णाꣳ॒॒ राजा᳚ ।  अप्ये॑षाꣳ स्थ॒पति॑र्ह॒तः । अथो॑ मा॒ताऽथो॑ पि॒ता ।  अथो᳚ स्थू॒रा अथो᳚ क्षु॒द्राः । अथो॑ कृ॒ष्णा अथो᳚ श्वे॒ताः । अथो॑ आ॒शाति॑का ह॒ताः ।  श्वे॒ताभिः॑ स॒ह सर्वे॑ ह॒ताः । \textbf{ 77} \newline
                  \newline
                                                        (णॊ कॊर्वै fऒर् थिस् अनुवक्कम् ) \textbf{36} \newline \newline
                                \textbf{ T.A.7.37.1} \newline
                  आह॒राव॑द्य । शृ॒तस्य॑ ह॒विषो॒ यथा᳚ । तथ् स॒त्यम् ।  यद॒मुं ॅय॒मस्य॒ जम्भ॑योः । आद॑धामि॒ तथा॒ हि तत् । खण्फण्म्रसि॑ । \textbf{ 78} \newline
                  \newline
                                                        (णॊ कॊर्वै fऒर् थिस् अनुवक्कम् ) \textbf{37} \newline \newline
                                \textbf{ T.A.7.38.1} \newline
                  ब्रह्म॑णा त्वा शपामि । ब्रह्म॑णस्त्वा श॒पथे॑न शपामि । घो॒रेण॑ त्वा॒ भृगू॑णा॒ञ्चक्षु॑षा॒ प्रेक्षे᳚ । रौ॒द्रेण॒ त्वाङ्गि॑रसा॒म् मन॑सा ध्यायामि ।  अ॒घस्य॑ त्वा॒ धार॑या विद्ध्यामि ।  अध॑रो॒ मत् प॑द्यस्वासौ । \textbf{ 79} \newline
                  \newline
                                                        (णॊ कॊर्वै fऒर् थिस् अनुवक्कम् ) \textbf{38} \newline \newline
                                \textbf{ T.A.7.39.1} \newline
                  उत्तु॑द शिमिजावरि । तल्पे॑जे॒ तल्प॒ उत्तु॑द । गि॒रीꣳ रनु॒ प्रवे॑शय । मरी॑ची॒रुप॒ सन्नु॑द । याव॑दि॒तः पु॒रस्ता॑दु॒दया॑ति॒ सूर्यः॑ ।  ताव॑दि॒तो॑ऽमुन्ना॑शय । यो᳚ऽस्मान् द्वेष्टि॑ ।  यञ्च॑ व॒यम् द्वि॒ष्मः । \textbf{ 80} \newline
                  \newline
                                                        (णॊ कॊर्वै fऒर् थिस् अनुवक्कम् ) \textbf{39} \newline \newline
                                \textbf{ T.A.7.40.1} \newline
                  भूर्भुव॒स्सुवो॒ भूर्भुव॒स्सुवो॒ भूर्भुव॒स्सुवः॑ ।  भुवो᳚ऽद्धायि॒ भुवो᳚ऽद्धायि॒ भुवो᳚ऽद्धायि ।  नृ॒म्णायि नृ॒म्णम् नृ॒म्णायि नृ॒म्णम् नृ॒म्णायि नृ॒म्णम् । नि॒धा य्यो॑ऽवायि नि॒धा य्यो॑ऽवायि नि॒धा य्यो॑ऽवायि । ए अ॒स्मे अ॒स्मे । सुव॒र्न ज्योतीः᳚ । \textbf{ 81} \newline
                  \newline
                                                        (णॊ कॊर्वै fऒर् थिस् अनुवक्कम् ) \textbf{40} \newline \newline
                                \textbf{ T.A.7.41.1} \newline
                  पृ॒थि॒वी स॒मित् । ताम॒ग्निः समि॑न्धे । साऽग्निꣳ समि॑न्धे । ताम॒हꣳ समि॑न्धे । सा मा॒ समि॑द्धा । आयु॑षा॒ तेज॑सा । वर्च॑सा श्रि॒या । यश॑सा ब्रह्मवर्च॒सेन॑ । अ॒न्नाद्ये॑न॒ समि॑न्ताꣳ॒॒ स्वाहा᳚ ॥  अ॒न्तरि॑क्षꣳ स॒मित् \textbf{ 82} \newline
                  \newline
                                                                  \textbf{ T.A.7.41.2} \newline
                  तां ॅवा॒युः समि॑न्धे । सा वा॒युꣳ समि॑न्धे । ताम॒हꣳ समि॑न्धे । सा मा॒ समि॑द्धा । आयु॑षा॒ तेज॑सा । वर्च॑सा श्रि॒या । यश॑सा ब्रह्मवर्च॒सेन॑ । अ॒न्नाद्ये॑न॒ समि॑न्ताꣳ॒॒ स्वाहा᳚ ।  द्यौः स॒मित् । तामा॑दि॒त्यः समि॑न्धे \textbf{ 83} \newline
                  \newline
                                                                  \textbf{ T.A.7.41.3} \newline
                  साऽऽदि॒त्यꣳ समि॑न्धे । ताम॒हꣳ समि॑न्धे । सा मा॒ समि॑द्धा । आयु॑षा॒ तेज॑सा । वर्च॑सा श्रि॒या । यश॑सा ब्रह्मवर्च॒सेन॑ । अ॒न्नाद्ये॑न॒ समि॑न्ताꣳ॒॒ स्वाहा᳚ ॥ प्रा॒जा॒प॒त्या मे॑ स॒मिद॑सि सपत्न॒क्षय॑णी ।  भ्रा॒तृ॒व्य॒हा मे॑ऽसि॒ स्वाहा᳚ ॥ अग्ने᳚ व्रतपते व्र॒तञ्च॑रिष्यामि \textbf{ 84} \newline
                  \newline
                                                                  \textbf{ T.A.7.41.4} \newline
                  तच्छ॑केयं॒ तन्मे॑ राद्ध्यताम् । वायो᳚ व्रतपत॒ आदि॑त्य व्रतपते । व्र॒तानां᳚ ॅव्रतपते व्र॒तञ्च॑रिष्यामि । तच्छ॑केयं॒ तन्मे॑ राद्ध्यताम् ॥ द्यौः स॒मित् । तामा॑दि॒त्यः समि॑न्धे । साऽऽदि॒त्यꣳ समि॑न्धे ।  ताम॒हꣳ समि॑न्धे । सा मा॒ समि॑द्धा । आयु॑षा॒ तेज॑सा \textbf{ 85} \newline
                  \newline
                                                                  \textbf{ T.A.7.41.5} \newline
                  वर्च॑सा श्रि॒या । यश॑सा ब्रह्मवर्च॒सेन॑ । अ॒न्नाद्ये॑न॒ समि॑न्ताꣳ॒॒ स्वाहा᳚ ।  अ॒न्तरि॑क्षꣳ स॒मित् । तां ॅवा॒युः समि॑न्धे । सा वा॒युꣳ समि॑न्धे । ताम॒हꣳ समि॑न्धे । सा मा॒ समि॑द्धा । आयु॑षा॒ तेज॑सा । वर्च॑सा श्रि॒या \textbf{ 86} \newline
                  \newline
                                                                  \textbf{ T.A.7.41.6} \newline
                  यश॑सा ब्रह्मवर्च॒सेन॑ । अ॒न्नाद्ये॑न॒ समि॑न्ताꣳ॒॒ स्वाहा᳚ । पृ॒थि॒वी स॒मित् । ताम॒ग्निः समि॑न्धे । साऽग्निꣳ समि॑न्धे । ताम॒हꣳ समि॑न्धे ।  सा मा॒ समि॑द्धा । आयु॑षा॒ तेज॑सा । वर्च॑सा श्रि॒या ।  यश॑सा ब्रह्मवर्च॒सेन॑ ( ) \textbf{ 87} \newline
                  \newline
                                                                  \textbf{ T.A.7.41.7} \newline
                  अ॒न्नाद्ये॑न॒ समि॑न्ताꣳ॒॒ स्वाहा᳚ । प्रा॒जा॒प॒त्या मे॑ स॒मिद॑सि सपत्न॒क्षय॑णी । भ्रा॒तृ॒व्य॒हा मे॑ऽसि॒ स्वाहा᳚ । आदि॑त्य व्रतपते व्र॒तम॑चारिषम् ।  तद॑शकं॒ तन्मे॑ऽराधि । वायो᳚ व्रतप॒तेऽग्ने᳚ व्रतपते । व्र॒तानां᳚ ॅव्रतपते व्र॒तम॑चारिषम् । तद॑शकं॒ तन्मे॑ऽराधि । \textbf{ 88} \newline
                  \newline
                                                        (स॒मिथ् - समि॑न्धे - व्र॒तञ्च॑रिष्या॒ - म्यायु॑षा॒ तेज॑सा॒ - वर्च॑सा श्रि॒या - यश॑सा ब्रह्मवर्च॒सेना॒ - +ष्टौ च॑) \textbf{41} \newline \newline
                                \textbf{ T.A.7.42.1} \newline
                  शन्नो॒ वातः॑ पवताम् मात॒रिश्वा॒ शन्न॑स्तपतु॒ सूर्यः॑ । अहा॑नि॒ शम् भ॑वन्तु नः॒ शꣳ रात्रिः॒ प्रति॑धीयताम् ॥ शमु॒षा नो॒ व्यु॑च्छतु॒ शमा॑दि॒त्य उदे॑तु नः । शि॒वा नः॒ शन्त॑मा भव सुमृडी॒का सर॑स्वति । मा ते॒ व्यो॑म स॒न्दृशि॑ ॥  इडा॑यै॒ वास्त्व॑सि वास्तु॒मद् वा᳚स्तु॒मन्तो॑ भूयास्म॒ मा वास्तो᳚-श्छिथ्स्मह्य वा॒स्तुः स भू॑या॒द्-यो᳚ऽस्मान् द्वेष्टि॒ यञ्च॑ व॒यं द्वि॒ष्मः ॥  प्र॒ति॒ष्ठाऽसि॑ प्रति॒ष्ठाव॑न्तो भूयास्म॒ मा प्र॑ति॒ष्ठाया᳚- श्छिथ्स्मह्य प्रति॒ष्ठः  स भू॑या॒द् यो᳚ऽस्मान् द्वेष्टि॒ यञ्च॑ व॒यं द्वि॒ष्मः ॥ आ वा॑त वाहि भेष॒जम् ॅवि वा॑त वाहि॒ यद्रपः॑ ।  त्वꣳ हि वि॒श्वभे॑षजो दे॒वानां᳚ दू॒त ईय॑से ॥  द्वावि॒मौ वातौ॑ वात॒ आ सिन्धा॒रा प॑रा॒वतः॑ \textbf{ 89} \newline
                  \newline
                                                                  \textbf{ T.A.7.42.2} \newline
                  दक्ष॑म् मे अ॒न्य आ॒वातु॒ परा॒ऽन्यो वा॑तु॒ यद्-रपः॑ ॥  यद॒दो वा॑त ते गृ॒हे॑ऽमृत॑स्य नि॒धिर्.हि॒तः ।  ततो॑ नो देहि जी॒वसे॒ ततो॑ नो धेहि भेष॒जम् ॥  ततो॑ नो॒ मह॒ आव॑ह॒ वात॒ आवा॑तु भेष॒जम् ।  श॒म्भूर् म॑यो॒भूर्नो॑ हृ॒दे प्र ण॒ आयूꣳ॑षि तारिषत् ॥ इन्द्र॑स्य गृ॒हो॑ऽसि॒ तम् त्वा॒ प्रप॑द्ये॒ सगु॒स्साश्वः॑ । स॒ह यन्मे॒ अस्ति॒ तेन॑ ॥  भूः प्रप॑द्ये॒ भुवः॒ प्रप॑द्ये॒ सुवः॒ प्रप॑द्ये॒ भूर्भुव॒स्सुवः॒ प्रप॑द्ये वा॒युम्  प्रप॒द्येऽना᳚र्ताम् दे॒वता॒म् प्रप॒द्ये ऽश्मा॑नमाख॒णम् प्रप॑द्ये प्र॒जाप॑तेर्  ब्रह्मको॒शम् ब्रह्म॒ प्रप॑द्य॒ ओम् प्रप॑द्ये ॥  अ॒न्तरि॑क्षम् म उ॒र्व॑न्तर॑म् बृ॒हद॒ग्नयः॒ पर्व॑ताश्च॒ यया॒ वातः॑ स्व॒स्त्या स्व॑स्ति॒मान्तया᳚ स्व॒स्त्या स्व॑स्ति॒ मान॑सानि ॥  प्राणा॑पानौ मृ॒त्योर्मा॑ पात॒म् प्राणा॑पानौ॒ मा मा॑ हासिष्ट॒म् मयि॑ मे॒धाम् मयि॑ प्र॒जाम् मय्य॒ग्नि स्तेजो॑ दधातु॒ मयि॑ मे॒धाम् मयि॑ प्र॒जाम् मयीन्द्र॑ इन्द्रि॒यम् द॑धातु॒ मयि॑ मे॒धाम् मयि॑ प्र॒जाम् मयि॒ सूर्यो॒  भ्राजो॑ दधातु । \textbf{ 90} \newline
                  \newline
                                                                  \textbf{ T.A.7.42.3} \newline
                  द्यु॒भिर॒क्तुभिः॒ परि॑पात-म॒स्मानरि॑ष्टेभि-रश्विना॒ सौभ॑गेभिः । तन्नो॑ मि॒त्रो वरु॑णो मामहन्ता॒ मदि॑तिः॒ सिन्धुः॑ पृथि॒वी उ॒त द्यौः ॥  कया॑ नश्चि॒त्र आभु॑वदू॒ती स॒दावृ॑धः॒ सखा᳚ । कया॒ शचि॑ष्ठया वृ॒ता ॥ कस्त्वा॑ स॒त्यो मदा॑ना॒म् मꣳहि॑ष्ठो मथ्स॒दन्ध॑सः ।  दृ॒ढा चि॑दा॒रुजे॒ वसु॑ ॥ अ॒भी षु णः॒ सखी॑नामवि॒ता ज॑रितॄ॒णाम् । श॒तम् भ॑वास्यू॒तिभिः॑ ॥  वयः॑ सुप॒र्णा उप॑सेदु॒रिन्द्र॑म् प्रि॒यमे॑धा॒ ऋष॑यो॒ नाध॑मानाः ।  अप॑ ध्वा॒न्तमू᳚र्णु॒हि पू॒र्द्धि चक्षु॑र् मुमु॒ग्ध्य॑स्मा-न्नि॒धये॑व ब॒द्धान् । \textbf{ 91} \newline
                  \newline
                                                                  \textbf{ T.A.7.42.4} \newline
                  शन्नो॑ दे॒वीर॒भिष्ट॑य॒ आपो॑ भवन्तु पी॒तये᳚ । शं ॅयोर॒भिस्र॑वन्तु नः ॥ ईशा॑ना॒ वार्या॑णा॒म् क्षय॑न्तीश्चर्.षणी॒नाम् । अ॒पो या॑चामि भेष॒जम् ॥ सु॒मि॒त्रा न॒ आप॒ ओष॑धयः सन्तु दुर्मि॒त्रास्तस्मै॑ भूयासु॒र्-यो᳚ऽस्मान्-द्वेष्टि॒ यञ्च॑ व॒यं द्वि॒ष्मः ॥  आपो॒ हि ष्ठा म॑यो॒भुव॒स्ता न॑ ऊ॒र्जे द॑धातन । म॒हे रणा॑य॒ चक्ष॑से ॥ यो वः॑ शि॒वत॑मो॒ रस॒स्तस्य॑ भाजयते॒ह नः॑ । उ॒श॒तीरि॑व मा॒तरः॑ ॥  तस्मा॒ अर॑म् गमाम वो॒ यस्य॒ क्षया॑य॒ जिन्व॑थ \textbf{ 92} \newline
                  \newline
                                                                  \textbf{ T.A.7.42.5} \newline
                  आपो॑ ज॒नय॑था च नः ॥  पृ॒थि॒वी शा॒न्ता साऽग्निना॑ शा॒न्ता सा मे॑ शा॒न्ता शुचꣳ॑ शमयतु ॥ अ॒न्तरि॑क्षꣳ शा॒न्तं तद् वा॒युना॑ शा॒न्तं तन्मे॑ शा॒न्तꣳ शुचꣳ॑ शमयतु ।  द्यौः शा॒न्ता साऽऽदि॒त्येन॑ शा॒न्ता सा मे॑ शा॒न्ता शुचꣳ॑ शमयतु ॥ पृ॒थि॒वी शान्ति॑र॒न्तरि॑क्षꣳ॒॒ शान्ति॒र् द्यौः शान्ति॒र् दिशः॒ शान्ति॑ रवान्तरदि॒शाः शान्ति॑ र॒ग्निः शान्ति॑र् वा॒युः शान्ति॑ रादि॒त्यः शान्ति॑  श्च॒न्द्रमाः॒ शान्ति॒र् नक्ष॑त्राणि॒ शान्ति॒ रापः॒ शान्ति॒ रोष॑धयः॒ शान्ति॒र् वन॒स्पत॑यः॒ शान्ति॒र् गौः शान्ति॑ र॒जा शान्ति॒ रश्वः॒ शान्तिः॒ पुरु॑षः॒ शान्ति॒र् ब्रह्म॒ शान्ति॑र् ब्राह्म॒णः शान्तिः॒ शान्ति॑रे॒व शान्तिः॒ शान्ति॑र्मे अस्तु॒ शान्तिः॑ ॥  तया॒ऽहꣳ शा॒न्त्या स॑र्वशा॒न्त्या मह्यं॑ द्वि॒पदे॒ चतु॑ष्पदे च॒ शान्ति॑म् करोमि॒ शान्ति॑र्मे अस्तु॒ शान्तिः॑ ॥  एह॒ श्रीश्च॒ ह्रीश्च॒ धृति॑श्च॒ तपो॑ मे॒धा प्र॑ति॒ष्ठा श्र॒द्धा स॒त्यं धर्म॑श्चै॒तानि॒ मोत्ति॑ष्ठन्त॒-मनूत्ति॑ष्ठन्तु॒ मा माꣳ॒॒ श्रीश्च॒ ह्रीश्च॒ धृति॑श्च॒ तपो॑ मे॒धा प्र॑ति॒ष्ठा श्र॒द्धा स॒त्यं धर्म॑श्चै॒तानि॑ मा॒ मा हा॑सिषुः ॥  उदायु॑षा स्वा॒युषोदोष॑धीनाꣳ॒॒ रसे॒नोत् प॒र्जन्य॑स्य॒ शुष्मे॒णोद॑स्था-म॒मृताꣳ॒॒ अनु॑ ॥ तच्चक्षु॑र् दे॒वहि॑तम् पु॒रस्ता᳚ च्छु॒क्रमु॒च्चर॑त् ।  पश्ये॑म श॒रदः॑ श॒तम् जीवे॑म श॒रदः॑ श॒तम् नन्दा॑म श॒रदः॑ श॒तम् मोदा॑म श॒रदः॑ श॒तम् भवा॑म श॒रदः॑ श॒तꣳ शृ॒णवा॑म श॒रदः॑ श॒तम् प्रब्र॑वाम श॒रदः॑ श॒त मजी॑ताः स्याम श॒रदः॑ श॒तम् ज्योक्च॒ सूर्य॑म् दृ॒शे ( ) । \textbf{ 93} \newline
                  \newline
                                                                  \textbf{ T.A.7.42.6} \newline
                  य उद॑गान् मह॒तोऽर्णवा᳚द्-वि॒भ्राज॑मानः सरि॒रस्य॒ मद्ध्या॒थ्स मा॑ वृष॒भो लो॑हिता॒क्षः सूर्यो॑ विप॒श्चिन् मन॑सा पुनातु ॥ ब्रह्म॑णः॒ श्चोत॑न्यसि॒ ब्रह्म॑ण आ॒णी स्थो॒ ब्रह्म॑ण आ॒वप॑नमसि धारि॒तेयम् पृ॑थि॒वी ब्रह्म॑णा म॒ही धा॑रि॒तमे॑नेन म॒हद॒न्तरि॑क्ष॒म् दिव॑म् दाधार पृथि॒वीꣳ सदे॑वां॒ ॅयद॒हं ॅवेद॒ तद॒हं धा॑रयाणि॒ मा मद्वेदोऽधि॒ विस्र॑सत् ॥ मे॒धा॒म॒नी॒षे माऽऽवि॑शताꣳ स॒मीची॑ भू॒तस्य॒ भव्य॒स्याव॑रुद्ध्यै॒ सर्व॒मायु॑रयाणि॒  सर्व॒मायु॑रयाणि ॥ आ॒भिर् गी॒र्भिर् यदतो॑ न ऊ॒नमाप्या॑यय हरिवो॒ वर्द्ध॑मानः ।  य॒दा स्तो॒तृभ्यो॒ महि॑ गो॒त्रा रु॒जासि॑ भूयिष्ठ॒भाजो॒ अध॑ ते स्याम ॥ ब्रह्म॒ प्रावा॑दिष्म॒ तन्नो॒ मा हा॑सीत् ॥ ओम् शान्तिः॒ शान्तिः॒ शान्तिः॑ । \textbf{ 94} \newline
                  \newline
                                                        (प॒रा॒वतो॑-दधातु-ब॒द्धान्-जिन्व॑थ-दृ॒शे-+स॒प्त च॑) \textbf{42} \newline \newline
\textbf{Prapaataka Korvai with starting Padams of 1 to 42 Anuvaakams :-} \newline
(नमो॑ - यु॒ञ्जते॒ - वृष्णो॒ अश्व॑स्य॒ - ब्रह्म॑न् प्रव॒र्ग्ये॑ण॒ - ब्रह्म॒न् प्रच॑रिष्यामो॒ - दश॒ प्राची॒ - रप॑श्यङ्गो॒पाम् - दे॒वस्य॑ - समु॒द्रा - ये॒षे पी॑पिहि॒ - घर्म॒ या ते॑ - मही॒नाम् च॒त्वा - र्यस्का॒न्॒. - या पु॒रस्ता᳚थ् स॒प्त स॑प्त - प्रा॒णाय॒ त्रीणि॑ - पू॒ष्णे च॒त्वार् यु - द॒स्यैका॑दश॒ - यास्ते॑ स॒प्ता - ग्निर् द्ध्रु॒वसीः॒ दैका॒न्न विꣳ॑शति॒र् - भूरू॒र्द्ध्व स्त्रिꣳ॒॒शतद् - भूर् मयि॒ षोड॑श॒ - यास्ते॑ घो॒रा नव॒ - स्निक्चा॒ष्टौ - धुनि॑श्च॒ द्वे - उ॒ग्रश्च॒ त्रीण्य॑ - होरा॒त्रे पञ्च॒ - खट् त्रीणि॒ - विगाः स॒प्तद॒शा - सृ॑न् मुखः च॒त्वारि॒ - यदे॒तद् वृ॑क॒सः पञ्च॒ - यदी॑षि॒त श्च॒त्वारि॒ - दीर्घ॑मुखि॒ त्रीणी॒ - त्था च्च॒त्वारि॒ - यदे॒तद् भू॒तानि॒ षट् - प्र॒सार्य॒ त्रीण्य - त्रि॑णा॒ दशा- ह॒राव॑द्य॒ - ब्रह्म॑णा॒ षट्थ् षड् - उत्तु॑दा॒ष्टौ - भूः षट् - पृ॑थि॒व्य॑ष्ट ष॑ष्टिः॒ - शन्नः॑ स॒प्तप॑ञ्चा॒शद् द्विच॑त्वारिꣳशत् ) \newline

\textbf{korvai with starting padams of1, 11, 21 Series of Dasinis :-} \newline
(नमो॑ वा॒चे - ब्रह्म॑न् प्रव॒र्ग्ये॑ण॒ - मयि॒ रु - ग॒न्तरि॑क्षेण - पि॒ता नो॑ऽसि - मही॒ना - मुप॑ नो॒ - यदे॒तद् - भूर् - द्यू॒भि श्चतु॑र् नवतिः) \newline

\textbf{first and last padam in TA, 6th Prapaatakam :-} \newline
एन्द् ऒf fइर्स्त्ळस्त् फदम् \newline 


॥ कृष्ण यजुर्वेदीय तैत्तिरीय ब्राह्मणे आरण्यके सप्तमः प्रपाठकः समाप्तः ॥



Appendix (of Expansions)
ट्.आ.7.11.2 - अनु॑नो॒ऽद्यानु॑मतिः{1} , अन्विद॑नुमते॒ त्वम् {2} अनु॑ नो॒ऽद्याऽनु॑मतिर्य॒ज्ञ्ं दे॒वेषु॑ मन्यतां । 
अ॒ग्निश्च॑ हव्य॒वाह॑नो॒ भव॑तां दा॒शुषे॒ मयः॑ ॥ {1}

अन्विद॑नुमते॒ त्वं मन्या॑सै॒ शञ्च॑नः कृधि । 
क्रत्वे॒ दक्षा॑य नो हिनु॒ प्रण॒ आयूꣳ॑षि तारिषः ॥ {2} 
(Both {1} and {2} appearing in TS 3.3.11.3 and 3.3.11.4)

ट्.आ.7.11.9 - उद्व॒यन्तम॑स॒स्परि॑ {3} , उदु॒त्यं चि॒त्रम् {4} 
उद्व॒यं तम॑स॒स्परि॒ पश्य॑न्तो॒ ज्योति॒रुत्त॑रं । 
दे॒वं दे॑व॒त्रा सूर्य॒मग॑न्म॒ ज्योति॑रुत्त॒मं ॥ {3}
({3} appearing in T.S.4.1.7.4)
उदु॒ त्यं जा॒तवे॑दसं दे॒वं ॅव॑हन्ति के॒तवः॑ । दृ॒शे विश्वा॑य॒ सूर्यं᳚ । 
चि॒त्रं दे॒वाना॒-मुद॑गा॒दनी॑कं॒ चक्षु॑ र्मि॒त्रस्य॒ वरु॑णस्या॒ऽग्नेः ॥ {4}
({4} appearing in T.S.1.4.43.1)

ट्.आ.7.20.2 - पुन॑रू॒र्जा{5}, स॒ह र॒य्या {6} 
पुन॑रू॒र्जा नि व॑र्तस्व॒ पुन॑रग्न इ॒षाऽऽयु॑षा । पुन॑र्नः पाहि वि॒श्वतः॑ ॥ {5}

स॒ह र॒य्या नि व॑र्त॒स्वाग्ने॒ पिन्व॑स्व॒ धार॑या । 
वि॒श्वफ्स्नि॑या वि॒श्वत॒स्परि॑ ॥ {6}
(Both {5} and {6] appearing in T.S.1.5.3.3)

ट्.आ.7.20.3 - इ॒मम् मे॑ वरुण॒{7}, तत्त्वा॑ यामि{8} 
इ॒मं मे॑ वरुण श्रुधी॒ हव॑म॒द्या च॑ मृडय । 
त्वाम॑व॒स्युरा च॑के ॥ {7}

तत्त्वा॑ यामि॒ ब्रह्म॑णा॒ वन्द॑मान॒-स्तदा शा᳚स्ते॒ यज॑मानो ह॒विर्भिः॑ । 
अहे॑डमानो वरुणे॒ह बो॒द्ध्युरु॑शꣳस॒ मा न॒ आयुः॒ प्रमो॑षीः ॥ {8} 
(Both {7} and {8] appearing in T.S.2.1.11.6)

ट्.आ.7.20.3 - त्वन्नो॑ अग्ने॒{9}, स त्वन्नो॑ अग्ने{10} 
त्वं नो॑ अग्ने॒ वरु॑णस्य वि॒द्वान् दे॒वस्य॒ हेडोऽव॑ यासि सीष्ठाः । यजि॑ष्ठो॒ वह्नि॑ तमः॒ शोशु॑चानो॒ विश्वा॒ द्वेषाꣳ॑सि॒ प्रमु॑मुग्ध्य॒स्मत् ॥ {9}

स त्वंनो॑ अन्गेऽव॒मो भ॑वो॒ती नेदि॑ष्ठो अ॒स्या उ॒षसो॒ व्यु॑ष्टौ । 
अव॑ यक्ष्व नो॒ वरु॑णꣳ॒॒ ररा॑णो वी॒हि मृ॑डी॒कꣳ सु॒हवो॑ न एधि ॥ {10}
(Both {9} and {10} appearing in T.S.2.5.12.3)

ट्.आ.7.20.3 - त्वम॑ग्ने अ॒यासि॑{11) 
त्वम॑ग्ने अ॒याऽसि॑ । अ॒या सन्मन॑सा हि॒तः । अ॒या सन्.ह॒व्यमू॑हिषे । अ॒या नो॑ धेहि भेष॒जम् । इ॒ष्टो अ॒ग्निराहु॑तः । स्वाहा॑कृतः पिपर्तु नः । स्व॒गा दे॒वेभ्य॑ इ॒दं नमः॑ ॥ {11}
({11} appearing in T.B.2.4.1.9 )

ट्.आ.7.20.3 - उद्व॒यन्तम॑स॒स्परि॑ {12} , उदु॒त्यं{13}, चि॒त्रम् {14} 
{12}, {13} अन्द् {14} इस् समॆ अस् {3} अन्द् {4} अबॊवॆ

ट्.आ.7.20.3 - वयः॑ सुपर्णाः {15} 
वयः॑ सुप॒र्णा उप॑सेदु॒रिन्द्र᳚म् । प्रि॒यमे॑धा॒ ऋष॑यो॒ नाध॑मानाः । 
अप॑ ध्वा॒न्तमू᳚र्णु॒हि पू॒र्धि चक्षुः॑ । मु॒मु॒ग्ध्य॑स्मान् नि॒धये॑व ब॒द्धान् ॥ {15} 
({15} appearing in T.B.2.5.8.3) \newline
\pagebreak
\pagebreak
        


\end{document}
