\documentclass[17pt]{extarticle}
\usepackage{babel}
\usepackage{fontspec}
\usepackage{polyglossia}
\usepackage{extsizes}

\usepackage{color}   %May be necessary if you want to color links
\usepackage{hyperref}
\hypersetup{
    colorlinks=true, %set true if you want colored links
    linktoc=all,     %set to all if you want both sections and subsections linked
    linkcolor=black,  %choose some color if you want links to stand out
}

\setmainlanguage{sanskrit}
\setotherlanguages{english} %% or other languages
\setlength{\parindent}{0pt}
\pagestyle{myheadings}
\newfontfamily\devanagarifont[Script=Devanagari]{AdishilaVedic}
\renewcommand{\theHsection}{\thepart.section.\thesection}

\newcommand{\VAR}[1]{}
\newcommand{\BLOCK}[1]{}




\begin{document}
\begin{titlepage}
    \begin{center}
 
\begin{sanskrit}
    { \Large
    कृष्ण यजुर्वेदीय तैत्तिरीय संहिता,पद,जटा,घन पाठः 
    }
    \\
    \vspace{2.5cm}
    \mbox{ \Large
    5.6      पञ्चमकाण्डे षष्ठः प्रश्नः - उपानुवाक्याभिधानं   }
\end{sanskrit}
\end{center}

\end{titlepage}
\tableofcontents
\phantomsection
\pagebreak

\markright{ TS 5.6.1.1  \hfill https://www.vedavms.in \hfill}

\section{ TS 5.6.1.1 }

\textbf{TS 5.6.1.1 } \newline
\textbf{Samhita Paata} \newline

हिर॑ण्यवर्णाः॒ शुच॑यः पाव॒का यासु॑ जा॒तः क॒श्यपो॒ यास्विन्द्रः॑ । अ॒ग्निं ॅया गर्भं॑ दधि॒रे विरू॑पा॒स्ता न॒ आपः॒ शꣳ स्यो॒ना भ॑वन्तु ॥ यासाꣳ॒॒ राजा॒ वरु॑णो॒ याति॒ मद्ध्ये॑ सत्यानृ॒ते अ॑व॒पश्य॒न् जना॑नां । म॒धु॒श्चुतः॒ शुच॑यो॒ याः पा॑व॒कास्ता न॒ आपः॒ शꣳ स्यो॒ना भ॑वन्तु ॥ यासां᳚ दे॒वा दि॒वि कृ॒ण्वन्ति॑ भ॒क्षं ॅया अ॒न्तरि॑क्षे बहु॒धा भव॑न्ति । याः पृ॑थि॒वीं पय॑सो॒न्दन्ति॑ - [  ] \newline

\textbf{Pada Paata} \newline

हिर॑ण्यवर्णा॒ इति॒ हिर॑ण्य - व॒र्णाः॒ । शुच॑यः । पा॒व॒काः । यासु॑ । जा॒तः । क॒श्यपः॑ । यासु॑ । इन्द्रः॑ ॥ अ॒ग्निम् । याः । गर्भ᳚म् । द॒धि॒रे । विरू॑पा॒ इति॒ वि - रू॒पाः॒ । ताः । नः॒ । आपः॑ । शम् । स्यो॒नाः । भ॒व॒न्तु॒ ॥ यासा᳚म् । राजा᳚ । वरु॑णः । याति॑ । मद्ध्ये᳚ । स॒त्या॒नृ॒ते इति॑ सत्य - अ॒नृ॒ते । अ॒व॒पश्य॒न्नित्य॑व - पश्यन्न्॑ । जना॑नाम् ॥ म॒धु॒श्चुत॒ इति॑ मधु - श्चुतः॑ । शुच॑यः । याः । पा॒व॒काः । ताः । नः॒ । आपः॑ । शम् । स्यो॒नाः । भ॒व॒न्तु॒ ॥ यासा᳚म् । दे॒वाः । दि॒वि । कृ॒ण्वन्ति॑ । भ॒क्षम् । याः । अ॒न्तरि॑क्षे । ब॒हु॒धेति॑ बहु - धा । भव॑न्ति ॥ याः । पृ॒थि॒वीम् । पय॑सा । उ॒न्दन्ति॑ ।  \newline


\textbf{Krama Paata} \newline

हिर॑ण्यवर्णाः॒ शुच॑यः । हिर॑ण्यवर्णा॒ इति॒ हिर॑ण्य - व॒र्णाः॒ । शुच॑यः पाव॒काः । पा॒व॒का यासु॑ । यासु॑ जा॒तः । जा॒तः क॒श्यपः॑ । क॒श्यपो॒ यासु॑ । यास्विन्द्रः॑ । इन्द्र॒ इतीन्द्रः॑ ॥ अ॒ग्निम् ॅयाः । या गर्भ᳚म् । गर्भ॑म् दधि॒रे । द॒धि॒रे विरू॑पाः । विरू॑पा॒स्ताः । विरू॑पा॒ इति॒ वि - रू॒पाः॒ । ता नः॑ । न॒ आपः॑ । आपः॒ शम् । शꣳ स्यो॒नाः । स्यो॒ना भ॑वन्तु । भ॒व॒न्त्विति॑ भवन्तु ॥ यासाꣳ॒॒ राजा᳚ । राजा॒ वरु॑णः । वरु॑णो॒ याति॑ । याति॒ मद्ध्ये᳚ । मद्ध्ये॑ सत्यानृ॒ते । स॒त्या॒नृ॒ते अ॑व॒पश्यन्न्॑ । स॒त्या॒नृ॒ते इति॑ सत्य - अ॒नृ॒ते । अ॒व॒पश्य॒न् जना॑नाम् । अ॒व॒पश्य॒न्नित्य॑व - पश्यन्न्॑ । जना॑ना॒मिति॒ जना॑नाम् ॥ म॒धु॒श्चुतः॒ शुच॑यः । म॒धु॒श्चुत॒ इति॑ मधु - श्चुतः॑ । शुच॑यो॒ याः । याः पा॑व॒काः । पा॒व॒कास्ताः । ता नः॑ । न॒ आपः॑ । आपः॒ शम् । शꣳ स्यो॒नाः । स्यो॒ना भ॑वन्तु । भ॒व॒न्त्विति॑ भवन्तु ॥ यासा᳚म् दे॒वाः । दे॒वा दि॒वि । दि॒वि कृ॒ण्वन्ति॑ । कृ॒ण्वन्ति॑ भ॒क्षम् । भ॒क्षम् ॅयाः । या अ॒न्तरि॑क्षे । अ॒न्तरि॑क्षे बहु॒धा । ब॒हु॒धा भव॑न्ति । ब॒हु॒धेति॑ बहु - धा । भव॒न्तीति॒ भव॑न्ति ॥ याः पृ॑थि॒वीम् । पृ॒थि॒वीम् पय॑सा । पय॑सो॒न्दन्ति॑ । उ॒न्दन्ति॑ शु॒क्राः \newline

\textbf{Jatai Paata} \newline

1. हिर॑ण्यवर्णाः॒ शुच॑यः॒ शुच॑यो॒ हिर॑ण्यवर्णा॒ हिर॑ण्यवर्णाः॒ शुच॑यः । \newline
2. हिर॑ण्यवर्णा॒ इति॒ हिर॑ण्य - व॒र्णाः॒ । \newline
3. शुच॑यः पाव॒काः पा॑व॒काः शुच॑यः॒ शुच॑यः पाव॒काः । \newline
4. पा॒व॒का यासु॒ यासु॑ पाव॒काः पा॑व॒का यासु॑ । \newline
5. यासु॑ जा॒तो जा॒तो यासु॒ यासु॑ जा॒तः । \newline
6. जा॒तः क॒श्यपः॑ क॒श्यपो॑ जा॒तो जा॒तः क॒श्यपः॑ । \newline
7. क॒श्यपो॒ यासु॒ यासु॑ क॒श्यपः॑ क॒श्यपो॒ यासु॑ । \newline
8. यास्विन्द्र॒ इन्द्रो॒ यासु॒ यास्विन्द्रः॑ । \newline
9. इन्द्र॒ इतीन्द्रः॑ । \newline
10. अ॒ग्निं ॅया या अ॒ग्नि म॒ग्निं ॅयाः । \newline
11. या गर्भ॒म् गर्भं॒ ॅया या गर्भ᳚म् । \newline
12. गर्भ॑म् दधि॒रे द॑धि॒रे गर्भ॒म् गर्भ॑म् दधि॒रे । \newline
13. द॒धि॒रे विरू॑पा॒ विरू॑पा दधि॒रे द॑धि॒रे विरू॑पाः । \newline
14. विरू॑पा॒ स्ता स्ता विरू॑पा॒ विरू॑पा॒ स्ताः । \newline
15. विरू॑पा॒ इति॒ वि - रू॒पाः॒ । \newline
16. ता नो॑ न॒ स्ता स्ता नः॑ । \newline
17. न॒ आप॒ आपो॑ नो न॒ आपः॑ । \newline
18. आपः॒ शꣳ श माप॒ आपः॒ शम् । \newline
19. शꣳ स्यो॒नाः स्यो॒नाः शꣳ शꣳ स्यो॒नाः । \newline
20. स्यो॒ना भ॑वन्तु भवन्तु स्यो॒नाः स्यो॒ना भ॑वन्तु । \newline
21. भ॒व॒न्त्विति॑ भवन्तु । \newline
22. यासाꣳ॒॒ राजा॒ राजा॒ यासां॒ ॅयासाꣳ॒॒ राजा᳚ । \newline
23. राजा॒ वरु॑णो॒ वरु॑णो॒ राजा॒ राजा॒ वरु॑णः । \newline
24. वरु॑णो॒ याति॒ याति॒ वरु॑णो॒ वरु॑णो॒ याति॑ । \newline
25. याति॒ मद्ध्ये॒ मद्ध्ये॒ याति॒ याति॒ मद्ध्ये᳚ । \newline
26. मद्ध्ये॑ सत्यानृ॒ते स॑त्यानृ॒ते मद्ध्ये॒ मद्ध्ये॑ सत्यानृ॒ते । \newline
27. स॒त्या॒नृ॒ते अ॑व॒पश्य॑न् नव॒पश्यन्᳚ थ्सत्यानृ॒ते स॑त्यानृ॒ते अ॑व॒पश्यन्न्॑ । \newline
28. स॒त्या॒नृ॒ते इति॑ सत्य - अ॒नृ॒ते । \newline
29. अ॒व॒पश्य॒न् जना॑ना॒म् जना॑ना मव॒पश्य॑न् नव॒पश्य॒न् जना॑नाम् । \newline
30. अ॒व॒पश्य॒न्नित्य॑व - पश्यन्न्॑ । \newline
31. जना॑ना॒मिति॒ जना॑नाम् । \newline
32. म॒धु॒श्चुतः॒ शुच॑यः॒ शुच॑यो मधु॒श्चुतो॑ मधु॒श्चुतः॒ शुच॑यः । \newline
33. म॒धु॒श्चुत॒ इति॑ मधु - श्चुतः॑ । \newline
34. शुच॑यो॒ या याः शुच॑यः॒ शुच॑यो॒ याः । \newline
35. याः पा॑व॒काः पा॑व॒का या याः पा॑व॒काः । \newline
36. पा॒व॒का स्ता स्ताः पा॑व॒काः पा॑व॒का स्ताः । \newline
37. ता नो॑ न॒ स्ता स्ता नः॑ । \newline
38. न॒ आप॒ आपो॑ नो न॒ आपः॑ । \newline
39. आपः॒ शꣳ श माप॒ आपः॒ शम् । \newline
40. शꣳ स्यो॒नाः स्यो॒नाः शꣳ शꣳ स्यो॒नाः । \newline
41. स्यो॒ना भ॑वन्तु भवन्तु स्यो॒नाः स्यो॒ना भ॑वन्तु । \newline
42. भ॒व॒न्त्विति॑ भवन्तु । \newline
43. यासा᳚म् दे॒वा दे॒वा यासां॒ ॅयासा᳚म् दे॒वाः । \newline
44. दे॒वा दि॒वि दि॒वि दे॒वा दे॒वा दि॒वि । \newline
45. दि॒वि कृ॒ण्वन्ति॑ कृ॒ण्वन्ति॑ दि॒वि दि॒वि कृ॒ण्वन्ति॑ । \newline
46. कृ॒ण्वन्ति॑ भ॒क्षम् भ॒क्षम् कृ॒ण्वन्ति॑ कृ॒ण्वन्ति॑ भ॒क्षम् । \newline
47. भ॒क्षं ॅया या भ॒क्षम् भ॒क्षं ॅयाः । \newline
48. या अ॒न्तरि॑क्षे अ॒न्तरि॑क्षे॒ या या अ॒न्तरि॑क्षे । \newline
49. अ॒न्तरि॑क्षे बहु॒धा ब॑हु॒धा ऽन्तरि॑क्षे अ॒न्तरि॑क्षे बहु॒धा । \newline
50. ब॒हु॒धा भव॑न्ति॒ भव॑न्ति बहु॒धा ब॑हु॒धा भव॑न्ति । \newline
51. ब॒हु॒धेति॑ बहु - धा । \newline
52. भव॒न्त्विति॒ भव॑न्ति । \newline
53. याः पृ॑थि॒वीम् पृ॑थि॒वीं ॅया याः पृ॑थि॒वीम् । \newline
54. पृ॒थि॒वीम् पय॑सा॒ पय॑सा पृथि॒वीम् पृ॑थि॒वीम् पय॑सा । \newline
55. पय॑सो॒न्दन् त्यु॒न्दन्ति॒ पय॑सा॒ पय॑सो॒न्दन्ति॑ । \newline
56. उ॒न्दन्ति॑ शु॒क्राः शु॒क्रा उ॒न्दन् त्यु॒न्दन्ति॑ शु॒क्राः । \newline

\textbf{Ghana Paata } \newline

1. हिर॑ण्यवर्णाः॒ शुच॑यः॒ शुच॑यो॒ हिर॑ण्यवर्णा॒ हिर॑ण्यवर्णाः॒ शुच॑यः पाव॒काः पा॑व॒काः शुच॑यो॒ हिर॑ण्यवर्णा॒ हिर॑ण्यवर्णाः॒ शुच॑यः पाव॒काः । \newline
2. हिर॑ण्यवर्णा॒ इति॒ हिर॑ण्य - व॒र्णाः॒ । \newline
3. शुच॑यः पाव॒काः पा॑व॒काः शुच॑यः॒ शुच॑यः पाव॒का यासु॒ यासु॑ पाव॒काः शुच॑यः॒ शुच॑यः पाव॒का यासु॑ । \newline
4. पा॒व॒का यासु॒ यासु॑ पाव॒काः पा॑व॒का यासु॑ जा॒तो जा॒तो यासु॑ पाव॒काः पा॑व॒का यासु॑ जा॒तः । \newline
5. यासु॑ जा॒तो जा॒तो यासु॒ यासु॑ जा॒तः क॒श्यपः॑ क॒श्यपो॑ जा॒तो यासु॒ यासु॑ जा॒तः क॒श्यपः॑ । \newline
6. जा॒तः क॒श्यपः॑ क॒श्यपो॑ जा॒तो जा॒तः क॒श्यपो॒ यासु॒ यासु॑ क॒श्यपो॑ जा॒तो जा॒तः क॒श्यपो॒ यासु॑ । \newline
7. क॒श्यपो॒ यासु॒ यासु॑ क॒श्यपः॑ क॒श्यपो॒ यास्विन्द्र॒ इन्द्रो॒ यासु॑ क॒श्यपः॑ क॒श्यपो॒ यास्विन्द्रः॑ । \newline
8. यास्विन्द्र॒ इन्द्रो॒ यासु॒ यास्विन्द्रः॑ । \newline
9. इन्द्र॒ इतीन्द्रः॑ । \newline
10. अ॒ग्निं ॅया या अ॒ग्नि म॒ग्निं ॅया गर्भ॒म् गर्भं॒ ॅया अ॒ग्नि म॒ग्निं ॅया गर्भ᳚म् । \newline
11. या गर्भ॒म् गर्भं॒ ॅया या गर्भ॑म् दधि॒रे द॑धि॒रे गर्भं॒ ॅया या गर्भ॑म् दधि॒रे । \newline
12. गर्भ॑म् दधि॒रे द॑धि॒रे गर्भ॒म् गर्भ॑म् दधि॒रे विरू॑पा॒ विरू॑पा दधि॒रे गर्भ॒म् गर्भ॑म् दधि॒रे विरू॑पाः । \newline
13. द॒धि॒रे विरू॑पा॒ विरू॑पा दधि॒रे द॑धि॒रे विरू॑पा॒ स्ता स्ता विरू॑पा दधि॒रे द॑धि॒रे विरू॑पा॒ स्ताः । \newline
14. विरू॑पा॒ स्ता स्ता विरू॑पा॒ विरू॑पा॒ स्ता नो॑ न॒ स्ता विरू॑पा॒ विरू॑पा॒ स्ता नः॑ । \newline
15. विरू॑पा॒ इति॒ वि - रू॒पाः॒ । \newline
16. ता नो॑ न॒ स्ता स्ता न॒ आप॒ आपो॑ न॒ स्ता स्ता न॒ आपः॑ । \newline
17. न॒ आप॒ आपो॑ नो न॒ आपः॒ शꣳ श मापो॑ नो न॒ आपः॒ शम् । \newline
18. आपः॒ शꣳ श माप॒ आपः॒ शꣳ स्यो॒नाः स्यो॒नाः श माप॒ आपः॒ शꣳ स्यो॒नाः । \newline
19. शꣳ स्यो॒नाः स्यो॒नाः शꣳ शꣳ स्यो॒ना भ॑वन्तु भवन्तु स्यो॒नाः शꣳ शꣳ स्यो॒ना भ॑वन्तु । \newline
20. स्यो॒ना भ॑वन्तु भवन्तु स्यो॒नाः स्यो॒ना भ॑वन्तु । \newline
21. भ॒व॒न्त्विति॑ भवन्तु । \newline
22. यासाꣳ॒॒ राजा॒ राजा॒ यासां॒ ॅयासाꣳ॒॒ राजा॒ वरु॑णो॒ वरु॑णो॒ राजा॒ यासां॒ ॅयासाꣳ॒॒ राजा॒ वरु॑णः । \newline
23. राजा॒ वरु॑णो॒ वरु॑णो॒ राजा॒ राजा॒ वरु॑णो॒ याति॒ याति॒ वरु॑णो॒ राजा॒ राजा॒ वरु॑णो॒ याति॑ । \newline
24. वरु॑णो॒ याति॒ याति॒ वरु॑णो॒ वरु॑णो॒ याति॒ मद्ध्ये॒ मद्ध्ये॒ याति॒ वरु॑णो॒ वरु॑णो॒ याति॒ मद्ध्ये᳚ । \newline
25. याति॒ मद्ध्ये॒ मद्ध्ये॒ याति॒ याति॒ मद्ध्ये॑ सत्यानृ॒ते स॑त्यानृ॒ते मद्ध्ये॒ याति॒ याति॒ मद्ध्ये॑ सत्यानृ॒ते । \newline
26. मद्ध्ये॑ सत्यानृ॒ते स॑त्यानृ॒ते मद्ध्ये॒ मद्ध्ये॑ सत्यानृ॒ते अ॑व॒पश्य॑न् नव॒पश्यन्᳚ थ्सत्यानृ॒ते मद्ध्ये॒ मद्ध्ये॑ सत्यानृ॒ते अ॑व॒पश्यन्न्॑ । \newline
27. स॒त्या॒नृ॒ते अ॑व॒पश्य॑न् नव॒पश्यन्᳚ थ्सत्यानृ॒ते स॑त्यानृ॒ते अ॑व॒पश्य॒न् जना॑ना॒म् जना॑ना मव॒पश्यन्᳚ थ्सत्यानृ॒ते स॑त्यानृ॒ते अ॑व॒पश्य॒न् जना॑नाम् । \newline
28. स॒त्या॒नृ॒ते इति॑ सत्य - अ॒नृ॒ते । \newline
29. अ॒व॒पश्य॒न् जना॑ना॒म् जना॑ना मव॒पश्य॑न् नव॒पश्य॒न् जना॑नाम् । \newline
30. अ॒व॒पश्य॒न्नित्य॑व - पश्यन्न्॑ । \newline
31. जना॑ना॒मिति॒ जना॑नाम् । \newline
32. म॒धु॒श्चुतः॒ शुच॑यः॒ शुच॑यो मधु॒श्चुतो॑ मधु॒श्चुतः॒ शुच॑यो॒ या याः शुच॑यो मधु॒श्चुतो॑ मधु॒श्चुतः॒ शुच॑यो॒ याः । \newline
33. म॒धु॒श्चुत॒ इति॑ मधु - श्चुतः॑ । \newline
34. शुच॑यो॒ या याः शुच॑यः॒ शुच॑यो॒ याः पा॑व॒काः पा॑व॒का याः शुच॑यः॒ शुच॑यो॒ याः पा॑व॒काः । \newline
35. याः पा॑व॒काः पा॑व॒का या याः पा॑व॒का स्ता स्ताः पा॑व॒का या याः पा॑व॒का स्ताः । \newline
36. पा॒व॒का स्ता स्ताः पा॑व॒काः पा॑व॒का स्ता नो॑ न॒ स्ताः पा॑व॒काः पा॑व॒का स्ता नः॑ । \newline
37. ता नो॑ न॒ स्ता स्ता न॒ आप॒ आपो॑ न॒ स्ता स्ता न॒ आपः॑ । \newline
38. न॒ आप॒ आपो॑ नो न॒ आपः॒ शꣳ श मापो॑ नो न॒ आपः॒ शम् । \newline
39. आपः॒ शꣳ श माप॒ आपः॒ शꣳ स्यो॒नाः स्यो॒नाः श माप॒ आपः॒ शꣳ स्यो॒नाः । \newline
40. शꣳ स्यो॒नाः स्यो॒नाः शꣳ शꣳ स्यो॒ना भ॑वन्तु भवन्तु स्यो॒नाः शꣳ शꣳ स्यो॒ना भ॑वन्तु । \newline
41. स्यो॒ना भ॑वन्तु भवन्तु स्यो॒नाः स्यो॒ना भ॑वन्तु । \newline
42. भ॒व॒न्त्विति॑ भवन्तु । \newline
43. यासा᳚म् दे॒वा दे॒वा यासां॒ ॅयासा᳚म् दे॒वा दि॒वि दि॒वि दे॒वा यासां॒ ॅयासा᳚म् दे॒वा दि॒वि । \newline
44. दे॒वा दि॒वि दि॒वि दे॒वा दे॒वा दि॒वि कृ॒ण्वन्ति॑ कृ॒ण्वन्ति॑ दि॒वि दे॒वा दे॒वा दि॒वि कृ॒ण्वन्ति॑ । \newline
45. दि॒वि कृ॒ण्वन्ति॑ कृ॒ण्वन्ति॑ दि॒वि दि॒वि कृ॒ण्वन्ति॑ भ॒क्षम् भ॒क्षम् कृ॒ण्वन्ति॑ दि॒वि दि॒वि कृ॒ण्वन्ति॑ भ॒क्षम् । \newline
46. कृ॒ण्वन्ति॑ भ॒क्षम् भ॒क्षम् कृ॒ण्वन्ति॑ कृ॒ण्वन्ति॑ भ॒क्षं ॅया या भ॒क्षम् कृ॒ण्वन्ति॑ कृ॒ण्वन्ति॑ भ॒क्षं ॅयाः । \newline
47. भ॒क्षं ॅया या भ॒क्षम् भ॒क्षं ॅया अ॒न्तरि॑क्षे अ॒न्तरि॑क्षे॒ या भ॒क्षम् भ॒क्षं ॅया अ॒न्तरि॑क्षे । \newline
48. या अ॒न्तरि॑क्षे अ॒न्तरि॑क्षे॒ या या अ॒न्तरि॑क्षे बहु॒धा ब॑हु॒धा ऽन्तरि॑क्षे॒ या या अ॒न्तरि॑क्षे बहु॒धा । \newline
49. अ॒न्तरि॑क्षे बहु॒धा ब॑हु॒धा ऽन्तरि॑क्षे अ॒न्तरि॑क्षे बहु॒धा भव॑न्ति॒ भव॑न्ति बहु॒धा ऽन्तरि॑क्षे 
अ॒न्तरि॑क्षे बहु॒धा भव॑न्ति । \newline
50. ब॒हु॒धा भव॑न्ति॒ भव॑न्ति बहु॒धा ब॑हु॒धा भव॑न्ति । \newline
51. ब॒हु॒धेति॑ बहु - धा । \newline
52. भव॒न्त्विति॒ भव॑न्ति । \newline
53. याः पृ॑थि॒वीम् पृ॑थि॒वीं ॅया याः पृ॑थि॒वीम् पय॑सा॒ पय॑सा पृथि॒वीं ॅया याः पृ॑थि॒वीम् पय॑सा । \newline
54. पृ॒थि॒वीम् पय॑सा॒ पय॑सा पृथि॒वीम् पृ॑थि॒वीम् पय॑सो॒न्दन् त्यु॒न्दन्ति॒ पय॑सा पृथि॒वीम् पृ॑थि॒वीम् पय॑सो॒न्दन्ति॑ । \newline
55. पय॑सो॒न्दन् त्यु॒न्दन्ति॒ पय॑सा॒ पय॑सो॒न्दन्ति॑ शु॒क्राः शु॒क्रा उ॒न्दन्ति॒ पय॑सा॒ पय॑सो॒न्दन्ति॑ शु॒क्राः । \newline
56. उ॒न्दन्ति॑ शु॒क्राः शु॒क्रा उ॒न्दन् त्यु॒न्दन्ति॑ शु॒क्रा स्ता स्ताः शु॒क्रा उ॒न्दन् त्यु॒न्दन्ति॑ शु॒क्रा स्ताः । \newline
\pagebreak
\markright{ TS 5.6.1.2  \hfill https://www.vedavms.in \hfill}

\section{ TS 5.6.1.2 }

\textbf{TS 5.6.1.2 } \newline
\textbf{Samhita Paata} \newline

शु॒क्रास्ता न॒ आपः॒ शꣳ स्यो॒ना भ॑वन्तु ॥ शि॒वेन॑ मा॒ चक्षु॑षा पश्यताऽऽ*पः शि॒वया॑ त॒नुवोप॑ स्पृशत॒ त्वचं॑ मे ।सर्वाꣳ॑ अ॒ग्नीꣳ र॑फ्सु॒षदो॑ हुवे वो॒ मयि॒ वर्चो॒ बल॒मोजो॒ नि ध॑त्त ॥यद॒दः सं॑ प्रय॒॒तीरहा॒॒ वन॑दताह॒ते । तस्मा॒दा न॒द्यो॑ नाम॑ स्थ॒ ता वो॒ नामा॑नि सिन्धवः ॥ यत् प्रेषि॑ता॒ वरु॑णेन॒ ताः शीभꣳ॑ स॒मव॑ल्गत । \newline

\textbf{Pada Paata} \newline

शु॒क्राः । ताः । नः॒ । आपः॑ । शम् । स्यो॒नाः । भ॒व॒न्तु॒ ॥ शि॒वेन॑ । मा॒ । चक्षु॑षा । प॒श्य॒त॒ । आ॒पः॒ । शि॒वया᳚ । त॒नुवा᳚ । उपेति॑ । स्पृ॒श॒त॒ । त्वच᳚म् । मे॒ ॥ सर्वान्॑ । अ॒ग्नीन् । अ॒फ्सु॒षद॒ इत्य॑फ्सु - सदः॑ । हु॒वे॒ । वः॒ । मयि॑ । वर्चः॑ । बल᳚म् । ओजः॑ । नीति॑ । ध॒त्त॒ ॥ यत् । अ॒दः । स॒प्रं॒य॒तीरिति॑ सं - प्र॒य॒तीः । अहौ᳚ । अन॑दत । ह॒ते ॥ तस्मा᳚त् । एति॑ । न॒द्यः॑ । नाम॑ । स्थ॒ । ता । वः॒ । नामा॑नि । सि॒न्ध॒वः॒ ॥ यत् । प्रेषि॑ता॒ इति॒ प्र - इ॒षि॒ताः॒ । वरु॑णेन । ताः । शीभ᳚म् । स॒मव॑ल्ग॒तेति॑ सं - अव॑ल्गत ॥  \newline


\textbf{Krama Paata} \newline

शु॒क्रास्ताः । ता नः॑ । न॒ आपः॑ । आपः॒ शम् । शꣳ स्यो॒नाः । स्यो॒ना भ॑वन्तु । भ॒व॒न्त्विति॑ भवन्तु ॥ शि॒वेन॑ मा । मा॒ चक्षु॑षा । चक्षु॑षा पश्यत । प॒श्य॒ता॒पः॒ । आ॒पः॒ शि॒वया᳚ । शि॒वया॑ त॒नुवा᳚ । त॒नुवोप॑ । उप॑ स्पृशत । स्पृ॒श॒त॒ त्वच᳚म् । त्वच॑म् मे । म॒ इति॑ मे ॥ सर्वाꣳ॑ अ॒ग्नीन् । अ॒ग्नीꣳर॑फ्सु॒षदः॑ । अ॒फ्सु॒षदो॑ हुवे । अ॒फ्सु॒षद॒ इत॑फ्सु - सदः॑ । हु॒वे॒ वः॒ । वो॒ मयि॑ । मयि॒ वर्चः॑ । वर्चो॒ बल᳚म् । बल॒मोजः॑ । ओजो॒ नि । नि ध॑त्त । ध॒त्तेति॑ धत्त ॥ यद॒दः । अ॒दः स॑म्प्रय॒तीः । स॒म्प्र॒य॒तीरहौ᳚ । स॒म्प्र॒य॒तीरिति॑ सम् - प्र॒य॒तीः । अहा॒वन॑दत । अन॑दता ह॒ते । ह॒त इति॑ ह॒ते ॥ तस्मा॒दा । आ न॒द्यः॑ । न॒द्यो॑ नाम॑ । नाम॑ स्थ । स्थ॒ ता । ता वः॑ । वो॒ नामा॑नि । नामा॑नि सिन्धवः । सि॒न्ध॒व॒ इति॑ सिन्धवः ॥ यत् प्रेषि॑ताः । प्रेषि॑ता॒ वरु॑णेन । प्रेषि॑ता॒ इति॒ प्र - इ॒षि॒ताः॒ । वरु॑णेन॒ ताः । ताः शीभ᳚म् । शीभꣳ॑ स॒मव॑ल्गत । स॒मव॑ल्गत॒ तत् । स॒मव॑ल्ग॒तेति॑ सम् - अव॑ल्गत । \newline

\textbf{Jatai Paata} \newline

1. शु॒क्रा स्ता स्ताः शु॒क्राः शु॒क्रा स्ताः । \newline
2. ता नो॑ न॒ स्ता स्ता नः॑ । \newline
3. न॒ आप॒ आपो॑ नो न॒ आपः॑ । \newline
4. आपः॒ शꣳ श माप॒ आपः॒ शम् । \newline
5. शꣳ स्यो॒नाः स्यो॒नाः शꣳ शꣳ स्यो॒नाः । \newline
6. स्यो॒ना भ॑वन्तु भवन्तु स्यो॒नाः स्यो॒ना भ॑वन्तु । \newline
7. भ॒व॒न्त्विति॑ भवन्तु । \newline
8. शि॒वेन॑ मा मा शि॒वेन॑ शि॒वेन॑ मा । \newline
9. मा॒ चक्षु॑षा॒ चक्षु॑षा मा मा॒ चक्षु॑षा । \newline
10. चक्षु॑षा पश्यत पश्यत॒ चक्षु॑षा॒ चक्षु॑षा पश्यत । \newline
11. प॒श्य॒ ता॒प॒ आ॒पः॒ प॒श्य॒त॒ प॒श्य॒ ता॒पः॒ । \newline
12. आ॒पः॒ शि॒वया॑ शि॒वया॑ ऽऽप आपः शि॒वया᳚ । \newline
13. शि॒वया॑ त॒नुवा॑ त॒नुवा॑ शि॒वया॑ शि॒वया॑ त॒नुवा᳚ । \newline
14. त॒नुवोपोप॑ त॒नुवा॑ त॒नुवोप॑ । \newline
15. उप॑ स्पृशत स्पृश॒तो पोप॑ स्पृशत । \newline
16. स्पृ॒श॒त॒ त्वच॒म् त्वचꣳ॑ स्पृशत स्पृशत॒ त्वच᳚म् । \newline
17. त्वच॑म् मे मे॒ त्वच॒म् त्वच॑म् मे । \newline
18. म॒ इति॑ मे । \newline
19. सर्वाꣳ॑ अ॒ग्नीꣳ र॒ग्नीन् थ्सर्वा॒न् थ्सर्वाꣳ॑ अ॒ग्नीन् । \newline
20. अ॒ग्नीꣳ र॑फ्सु॒षदो॑ अफ्सु॒षदो॑ अ॒ग्नीꣳ र॒ग्नीꣳ र॑फ्सु॒षदः॑ । \newline
21. अ॒फ्सु॒षदो॑ हुवे हुवे अफ्सु॒षदो॑ अफ्सु॒षदो॑ हुवे । \newline
22. अ॒फ्सु॒षद॒ इत्य॑फ्सु - सदः॑ । \newline
23. हु॒वे॒ वो॒ वो॒ हु॒वे॒ हु॒वे॒ वः॒ । \newline
24. वो॒ मयि॒ मयि॑ वो वो॒ मयि॑ । \newline
25. मयि॒ वर्चो॒ वर्चो॒ मयि॒ मयि॒ वर्चः॑ । \newline
26. वर्चो॒ बल॒म् बलं॒ ॅवर्चो॒ वर्चो॒ बल᳚म् । \newline
27. बल॒ मोज॒ ओजो॒ बल॒म् बल॒ मोजः॑ । \newline
28. ओजो॒ नि न्योज॒ ओजो॒ नि । \newline
29. नि ध॑त्त धत्त॒ नि नि ध॑त्त । \newline
30. ध॒त्तेति॑ धत्त । \newline
31. यद॒दो॑ ऽदो यद् यद॒दः । \newline
32. अ॒दः सं॑प्रय॒तीः सं॑प्रय॒ती र॒दो॑ ऽदः सं॑प्रय॒तीः । \newline
33. सं॒प्र॒य॒ती रहा॒ वहौ॑ संप्रय॒तीः सं॑प्रय॒ती रहौ᳚ । \newline
34. सं॒प्र॒य॒तीरिति॑ सं - प्र॒य॒तीः । \newline
35. अहा॒ वन॑द॒ता न॑द॒ताहा॒ वहा॒ वन॑दत । \newline
36. अन॑दता ह॒ते ह॒ते ऽन॑द॒ता न॑दता ह॒ते । \newline
37. ह॒त इति॑ ह॒ते । \newline
38. तस्मा॒दा तस्मा॒त् तस्मा॒दा । \newline
39. आ न॒द्यो॑ न॒द्य॑ आ न॒द्यः॑ । \newline
40. न॒द्यो॑ नाम॒ नाम॑ न॒द्यो॑ न॒द्यो॑ नाम॑ । \newline
41. नाम॑ स्थ स्थ॒ नाम॒ नाम॑ स्थ । \newline
42. स्थ॒ ता ता स्थ॑ स्थ॒ ता । \newline
43. ता वो॑ व॒ स्ता ता वः॑ । \newline
44. वो॒ नामा॑नि॒ नामा॑नि वो वो॒ नामा॑नि । \newline
45. नामा॑नि सिन्धवः सिन्धवो॒ नामा॑नि॒ नामा॑नि सिन्धवः । \newline
46. सि॒न्ध॒व॒ इति॑ सिन्धवः । \newline
47. यत् प्रेषि॑ताः॒ प्रेषि॑ता॒ यद् यत् प्रेषि॑ताः । \newline
48. प्रेषि॑ता॒ वरु॑णेन॒ वरु॑णेन॒ प्रेषि॑ताः॒ प्रेषि॑ता॒ वरु॑णेन । \newline
49. प्रेषि॑ता॒ इति॒ प्र - इ॒षि॒ताः॒ । \newline
50. वरु॑णेन॒ ता स्ता वरु॑णेन॒ वरु॑णेन॒ ताः । \newline
51. ताः शीभꣳ॒॒ शीभ॒म् ता स्ताः शीभ᳚म् । \newline
52. शीभꣳ॑ स॒मव॑ल्गत स॒मव॑ल्गत॒ शीभꣳ॒॒ शीभꣳ॑ स॒मव॑ल्गत । \newline
53. स॒मव॑ल्ग॒तेति॑ सं - अव॑ल्गत । \newline

\textbf{Ghana Paata } \newline

1. शु॒क्रा स्ता स्ताः शु॒क्राः शु॒क्रा स्ता नो॑ न॒ स्ताः शु॒क्राः शु॒क्रा स्ता नः॑ । \newline
2. ता नो॑ न॒ स्ता स्ता न॒ आप॒ आपो॑ न॒ स्ता स्ता न॒ आपः॑ । \newline
3. न॒ आप॒ आपो॑ नो न॒ आपः॒ शꣳ श मापो॑ नो न॒ आपः॒ शम् । \newline
4. आपः॒ शꣳ श माप॒ आपः॒ शꣳ स्यो॒नाः स्यो॒नाः श माप॒ आपः॒ शꣳ स्यो॒नाः । \newline
5. शꣳ स्यो॒नाः स्यो॒नाः शꣳ शꣳ स्यो॒ना भ॑वन्तु भवन्तु स्यो॒नाः शꣳ शꣳ स्यो॒ना भ॑वन्तु । \newline
6. स्यो॒ना भ॑वन्तु भवन्तु स्यो॒नाः स्यो॒ना भ॑वन्तु । \newline
7. भ॒व॒न्त्विति॑ भवन्तु । \newline
8. शि॒वेन॑ मा मा शि॒वेन॑ शि॒वेन॑ मा॒ चक्षु॑षा॒ चक्षु॑षा मा शि॒वेन॑ शि॒वेन॑ मा॒ चक्षु॑षा । \newline
9. मा॒ चक्षु॑षा॒ चक्षु॑षा मा मा॒ चक्षु॑षा पश्यत पश्यत॒ चक्षु॑षा मा मा॒ चक्षु॑षा पश्यत । \newline
10. चक्षु॑षा पश्यत पश्यत॒ चक्षु॑षा॒ चक्षु॑षा पश्यताप आपः पश्यत॒ चक्षु॑षा॒ चक्षु॑षा पश्यतापः । \newline
11. प॒श्य॒ता॒प॒ आ॒पः॒ प॒श्य॒त॒ प॒श्य॒ता॒पः॒ शि॒वया॑ शि॒वया॑ ऽऽपः पश्यत पश्यतापः शि॒वया᳚ । \newline
12. आ॒पः॒ शि॒वया॑ शि॒वया॑ ऽऽप आपः शि॒वया॑ त॒नुवा॑ त॒नुवा॑ शि॒वया॑ ऽऽप आपः शि॒वया॑ त॒नुवा᳚ । \newline
13. शि॒वया॑ त॒नुवा॑ त॒नुवा॑ शि॒वया॑ शि॒वया॑ त॒नुवो पोप॑ त॒नुवा॑ शि॒वया॑ शि॒वया॑ त॒नुवोप॑ । \newline
14. त॒नुवो पोप॑ त॒नुवा॑ त॒नुवोप॑ स्पृशत स्पृश॒तोप॑ त॒नुवा॑ त॒नुवोप॑ स्पृशत । \newline
15. उप॑ स्पृशत स्पृश॒तो पोप॑ स्पृशत॒ त्वच॒म् त्वचꣳ॑ स्पृश॒तो पोप॑ स्पृशत॒ त्वच᳚म् । \newline
16. स्पृ॒श॒त॒ त्वच॒म् त्वचꣳ॑ स्पृशत स्पृशत॒ त्वच॑म् मे मे॒ त्वचꣳ॑ स्पृशत स्पृशत॒ त्वच॑म् मे । \newline
17. त्वच॑म् मे मे॒ त्वच॒म् त्वच॑म् मे । \newline
18. म॒ इति॑ मे । \newline
19. सर्वाꣳ॑ अ॒ग्नीꣳ र॒ग्नीन् थ्सर्वा॒न् थ्सर्वाꣳ॑ अ॒ग्नीꣳ र॑फ्सु॒षदो॑ अफ्सु॒षदो॑ अ॒ग्नीन् थ्सर्वा॒न् थ्सर्वाꣳ॑ अ॒ग्नीꣳ र॑फ्सु॒षदः॑ । \newline
20. अ॒ग्नीꣳ र॑फ्सु॒षदो॑ अफ्सु॒षदो॑ अ॒ग्नीꣳ र॒ग्नीꣳ र॑फ्सु॒षदो॑ हुवे हुवे अफ्सु॒षदो॑ अ॒ग्नीꣳ र॒ग्नीꣳ र॑फ्सु॒षदो॑ हुवे । \newline
21. अ॒फ्सु॒षदो॑ हुवे हुवे अफ्सु॒षदो॑ अफ्सु॒षदो॑ हुवे वो वो हुवे अफ्सु॒षदो॑ अफ्सु॒षदो॑ हुवे वः । \newline
22. अ॒फ्सु॒षद॒ इत्य॑फ्सु - सदः॑ । \newline
23. हु॒वे॒ वो॒ वो॒ हु॒वे॒ हु॒वे॒ वो॒ मयि॒ मयि॑ वो हुवे हुवे वो॒ मयि॑ । \newline
24. वो॒ मयि॒ मयि॑ वो वो॒ मयि॒ वर्चो॒ वर्चो॒ मयि॑ वो वो॒ मयि॒ वर्चः॑ । \newline
25. मयि॒ वर्चो॒ वर्चो॒ मयि॒ मयि॒ वर्चो॒ बल॒म् बलं॒ ॅवर्चो॒ मयि॒ मयि॒ वर्चो॒ बल᳚म् । \newline
26. वर्चो॒ बल॒म् बलं॒ ॅवर्चो॒ वर्चो॒ बल॒ मोज॒ ओजो॒ बलं॒ ॅवर्चो॒ वर्चो॒ बल॒ मोजः॑ । \newline
27. बल॒ मोज॒ ओजो॒ बल॒म् बल॒ मोजो॒ नि न्योजो॒ बल॒म् बल॒ मोजो॒ नि । \newline
28. ओजो॒ नि न्योज॒ ओजो॒ नि ध॑त्त धत्त॒ न्योज॒ ओजो॒ नि ध॑त्त । \newline
29. नि ध॑त्त धत्त॒ नि नि ध॑त्त । \newline
30. ध॒त्तेति॑ धत्त । \newline
31. यद॒दो॑ ऽदो यद् यद॒दः सं॑प्रय॒तीः सं॑प्रय॒ती र॒दो यद् यद॒दः सं॑प्रय॒तीः । \newline
32. अ॒दः सं॑प्रय॒तीः सं॑प्रय॒ती र॒दो॑ ऽदः सं॑प्रय॒ती रहा॒ वहौ॑ संप्रय॒ती र॒दो॑ ऽदः सं॑प्रय॒ती रहौ᳚ । \newline
33. सं॒प्र॒य॒ती रहा॒ वहौ॑ संप्रय॒तीः सं॑प्रय॒ती रहा॒ वन॑द॒ता न॑द॒ताहौ॑ संप्रय॒तीः सं॑प्रय॒ती रहा॒ वन॑दत । \newline
34. सं॒प्र॒य॒तीरिति॑ सं - प्र॒य॒तीः । \newline
35. अहा॒ वन॑द॒ता न॑द॒ताहा॒ वहा॒ वन॑दता ह॒ते ह॒ते ऽन॑द॒ताहा॒ वहा॒ वन॑दता ह॒ते । \newline
36. अन॑दता ह॒ते ह॒ते ऽन॑द॒ता न॑दता ह॒ते । \newline
37. ह॒त इति॑ ह॒ते । \newline
38. तस्मा॒दा तस्मा॒त् तस्मा॒दा न॒द्यो॑ न॒द्य॑ आ तस्मा॒त् तस्मा॒दा न॒द्यः॑ । \newline
39. आ न॒द्यो॑ न॒द्य॑ आ न॒द्यो॑ नाम॒ नाम॑ न॒द्य॑ आ न॒द्यो॑ नाम॑ । \newline
40. न॒द्यो॑ नाम॒ नाम॑ न॒द्यो॑ न॒द्यो॑ नाम॑ स्थ स्थ॒ नाम॑ न॒द्यो॑ न॒द्यो॑ नाम॑ स्थ । \newline
41. नाम॑ स्थ स्थ॒ नाम॒ नाम॑ स्थ॒ ता ता स्थ॒ नाम॒ नाम॑ स्थ॒ ता । \newline
42. स्थ॒ ता ता स्थ॑ स्थ॒ ता वो॑ व॒ स्ता स्थ॑ स्थ॒ ता वः॑ । \newline
43. ता वो॑ व॒ स्ता ता वो॒ नामा॑नि॒ नामा॑नि व॒ स्ता ता वो॒ नामा॑नि । \newline
44. वो॒ नामा॑नि॒ नामा॑नि वो वो॒ नामा॑नि सिन्धवः सिन्धवो॒ नामा॑नि वो वो॒ नामा॑नि सिन्धवः । \newline
45. नामा॑नि सिन्धवः सिन्धवो॒ नामा॑नि॒ नामा॑नि सिन्धवः । \newline
46. सि॒न्ध॒व॒ इति॑ सिन्धवः । \newline
47. यत् प्रेषि॑ताः॒ प्रेषि॑ता॒ यद् यत् प्रेषि॑ता॒ वरु॑णेन॒ वरु॑णेन॒ प्रेषि॑ता॒ यद् यत् प्रेषि॑ता॒ वरु॑णेन । \newline
48. प्रेषि॑ता॒ वरु॑णेन॒ वरु॑णेन॒ प्रेषि॑ताः॒ प्रेषि॑ता॒ वरु॑णेन॒ ता स्ता वरु॑णेन॒ प्रेषि॑ताः॒ प्रेषि॑ता॒ वरु॑णेन॒ ताः । \newline
49. प्रेषि॑ता॒ इति॒ प्र - इ॒षि॒ताः॒ । \newline
50. वरु॑णेन॒ ता स्ता वरु॑णेन॒ वरु॑णेन॒ ताः शीभꣳ॒॒ शीभ॒म् ता वरु॑णेन॒ वरु॑णेन॒ ताः शीभ᳚म् । \newline
51. ताः शीभꣳ॒॒ शीभ॒म् ता स्ताः शीभꣳ॑ स॒मव॑ल्गत स॒मव॑ल्गत॒ शीभ॒म् ता स्ताः शीभꣳ॑ स॒मव॑ल्गत । \newline
52. शीभꣳ॑ स॒मव॑ल्गत स॒मव॑ल्गत॒ शीभꣳ॒॒ शीभꣳ॑ स॒मव॑ल्गत । \newline
53. स॒मव॑ल्ग॒तेति॑ सं - अव॑ल्गत । \newline
\pagebreak
\markright{ TS 5.6.1.3  \hfill https://www.vedavms.in \hfill}

\section{ TS 5.6.1.3 }

\textbf{TS 5.6.1.3 } \newline
\textbf{Samhita Paata} \newline

तदा᳚प्नो॒-दिन्द्रो॑ वो य॒ती-स्तस्मा॒-दापो॒ अनु॑ स्थन ॥ अ॒प॒का॒मꣳ स्यन्द॑माना॒ अवी॑वरत वो॒ हिकं᳚ ।इन्द्रो॑ वः॒ शक्ति॑भि र्देवी॒-स्तस्मा॒द्-वार्णाम॑ वो हि॒तं ॥एको॑ दे॒वो अप्य॑तिष्ठ॒थ् स्यन्द॑माना यथा व॒शं । उदा॑निषु-र्म॒हीरिति॒ तस्मा॑-दुद॒क-मु॑च्यते ॥ आपो॑ भ॒द्रा घृ॒तमिदाप॑ आसुर॒ग्नी-षोमौ॑ बिभ्र॒त्याप॒ इत् ताः । ती॒व्रो रसो॑ मधु॒पृचा॑ - [  ] \newline

\textbf{Pada Paata} \newline

तत् । आ॒प्नो॒त् । इन्द्रः॑ । वः॒ । य॒तीः । तस्मा᳚त् । आपः॑ । अन्विति॑ । स्थ॒न॒ ॥ अ॒प॒का॒ममित्य॑प - का॒मम् । स्यन्द॑मानाः । अवी॑वरत । वः॒ । हिक᳚म् ॥ इन्द्रः॑ । वः॒ । शक्ति॑भि॒रिति॒ शक्ति॑-भिः॒ । दे॒वीः॒ । तस्मा᳚त् । वाः । नाम॑ । वः॒ । हि॒तम् ॥ एकः॑ । दे॒वः । अपीति॑ । अ॒ति॒ष्ठ॒त् । स्यन्द॑मानाः । य॒था॒व॒शमिति॑ यथा - व॒शम् ॥ उदिति॑ । आ॒नि॒षुः॒ । म॒हीः । इति॑ । तस्मा᳚त् । उ॒द॒कम् । उ॒च्य॒ते॒ ॥ आपः॑ । भ॒द्राः । घृ॒तम् । इत् । आपः॑ । आ॒सुः॒ । अ॒ग्नीषोमा॒वित्य॒ग्नी - सोमौ᳚ । बि॒भ्र॒ति॒ । आपः॑ । इत् । ताः ॥ ती॒व्रः । रसः॑ । म॒धु॒पृचा॒मिति॑ मधु-पृचा᳚म् ।  \newline


\textbf{Krama Paata} \newline

तदा᳚प्नोत् । आ॒प्नो॒दिन्द्रः॑ । इन्द्रो॑ वः । वो॒ य॒तीः । य॒तीस्तस्मा᳚त् । तस्मा॒दापः॑ । आपो॒ अनु॑ । अनु॑ स्थन । स्थ॒नेति॑ स्थन ॥ अ॒प॒का॒मꣳ स्यन्द॑मानाः । अ॒प॒का॒ममित्य॑प - का॒मम् । स्यन्द॑माना॒ अवी॑वरत । अवी॑वरत वः । वो॒ हिक᳚म् । हिक॒मिति॒ हिक᳚म् ॥ इन्द्रो॑ वः । वः॒ शक्ति॑भिः । शक्ति॑भिर् देवीः । शक्ति॑भि॒रिति॒ शक्ति॑ - भिः॒ । दे॒वी॒स्तस्मा᳚त् । तस्मा॒द् वाः । वार्णाम॑ । नाम॑ वः । वो॒ हि॒तम् । हि॒तमिति॑ हि॒तम् ॥ एको॑ दे॒वः । दे॒वो अपि॑ । अप्य॑तिष्ठत् । अ॒ति॒ष्ठ॒थ् स्यन्द॑मानाः । स्यन्द॑माना यथाव॒शम् । य॒था॒व॒शमिति॑ यथा - व॒शम् ॥ उदा॑निषुः । आ॒नि॒षु॒र् म॒हीः । म॒हीरिति॑ । इति॒ तस्मा᳚त् । तस्मा॑दुद॒कम् । उ॒द॒कमु॑च्यते । उ॒च्य॒त॒ इत्यु॑च्यते ॥ आपो॑ भ॒द्राः । भ॒द्रा घृ॒तम् । घृ॒तमित् । इदापः॑ । आप॑ आसुः । आ॒सु॒र॒ग्नीषोमौ᳚ । अ॒ग्नीषोमौ॑ बिभ्रति । अ॒ग्नीषोमा॒वित्य॒ग्नी - सोमौ᳚ । बि॒भ्र॒त्यापः॑ । आप॒ इत् । इत् ताः । ता इति॒ ताः ॥ ती॒व्रो रसः॑ । रसो॑ मधु॒पृचा᳚म् । म॒धु॒पृचा॑मरङ्ग॒मः । म॒धु॒पृचा॒मिति॑ मधु - पृचा᳚म् \newline

\textbf{Jatai Paata} \newline

1. तदा᳚प्नो दाप्नो॒त् तत् तदा᳚प्नोत् । \newline
2. आ॒प्नो॒ दिन्द्र॒ इन्द्र॑ आप्नो दाप्नो॒ दिन्द्रः॑ । \newline
3. इन्द्रो॑ वो व॒ इन्द्र॒ इन्द्रो॑ वः । \newline
4. वो॒ य॒तीर् य॒तीर् वो॑ वो य॒तीः । \newline
5. य॒ती स्तस्मा॒त् तस्मा᳚द् य॒तीर् य॒ती स्तस्मा᳚त् । \newline
6. तस्मा॒ दाप॒ आप॒ स्तस्मा॒त् तस्मा॒ दापः॑ । \newline
7. आपो॒ अन्वन् वाप॒ आपो॒ अनु॑ । \newline
8. अनु॑ स्थन स्थ॒नान् वनु॑ स्थन । \newline
9. स्थ॒नेति॑ स्थन । \newline
10. अ॒प॒का॒मꣳ स्यन्द॑मानाः॒ स्यन्द॑माना अपका॒म म॑पका॒मꣳ स्यन्द॑मानाः । \newline
11. अ॒प॒का॒ममित्य॑प - का॒मम् । \newline
12. स्यन्द॑माना॒ अवी॑वर॒ता वी॑वरत॒ स्यन्द॑मानाः॒ स्यन्द॑माना॒ अवी॑वरत । \newline
13. अवी॑वरत वो॒ वो ऽवी॑वर॒ता वी॑वरत वः । \newline
14. वो॒ हिकꣳ॒॒ हिकं॑ ॅवो वो॒ हिक᳚म् । \newline
15. हिक॒मिति॒ हिक᳚म् । \newline
16. इन्द्रो॑ वो व॒ इन्द्र॒ इन्द्रो॑ वः । \newline
17. वः॒ शक्ति॑भिः॒ शक्ति॑भिर् वो वः॒ शक्ति॑भिः । \newline
18. शक्ति॑भिर् देवीर् देवीः॒ शक्ति॑भिः॒ शक्ति॑भिर् देवीः । \newline
19. शक्ति॑भि॒रिति॒ शक्ति॑ - भिः॒ । \newline
20. दे॒वी॒ स्तस्मा॒त् तस्मा᳚द् देवीर् देवी॒ स्तस्मा᳚त् । \newline
21. तस्मा॒द् वार् वा स्तस्मा॒त् तस्मा॒द् वाः । \newline
22. वार् नाम॒ नाम॒ वार् वार् नाम॑ । \newline
23. नाम॑ वो वो॒ नाम॒ नाम॑ वः । \newline
24. वो॒ हि॒तꣳ हि॒तं ॅवो॑ वो हि॒तम् । \newline
25. हि॒तमिति॑ हि॒तम् । \newline
26. एको॑ दे॒वो दे॒व एक॒ एको॑ दे॒वः । \newline
27. दे॒वो अप्यपि॑ दे॒वो दे॒वो अपि॑ । \newline
28. अप्य॑ तिष्ठ दतिष्ठ॒ दप्य प्य॑तिष्ठत् । \newline
29. अ॒ति॒ष्ठ॒थ् स्यन्द॑मानाः॒ स्यन्द॑माना अतिष्ठ दतिष्ठ॒थ् स्यन्द॑मानाः । \newline
30. स्यन्द॑माना यथाव॒शं ॅय॑थाव॒शꣳ स्यन्द॑मानाः॒ स्यन्द॑माना यथाव॒शम् । \newline
31. य॒था॒व॒शमिति॑ यथा - व॒शम् । \newline
32. उदा॑निषु रानिषु॒ रुदु दा॑निषुः । \newline
33. आ॒नि॒षु॒र् म॒हीर् म॒ही रा॑निषु रानिषुर् म॒हीः । \newline
34. म॒ही रितीति॑ म॒हीर् म॒ही रिति॑ । \newline
35. इति॒ तस्मा॒त् तस्मा॒दि तीति॒ तस्मा᳚त् । \newline
36. तस्मा॑ दुद॒क मु॑द॒कम् तस्मा॒त् तस्मा॑ दुद॒कम् । \newline
37. उ॒द॒क मु॑च्यत उच्यत उद॒क मु॑द॒क मु॑च्यते । \newline
38. उ॒च्य॒त॒ इत्यु॑च्यते । \newline
39. आपो॑ भ॒द्रा भ॒द्रा आप॒ आपो॑ भ॒द्राः । \newline
40. भ॒द्रा घृ॒तम् घृ॒तम् भ॒द्रा भ॒द्रा घृ॒तम् । \newline
41. घृ॒त मिदिद् घृ॒तम् घृ॒त मित् । \newline
42. इदाप॒ आप॒ इदि दापः॑ । \newline
43. आप॑ आसु रासु॒ राप॒ आप॑ आसुः । \newline
44. आ॒सु॒ र॒ग्नीषोमा॑ व॒ग्नीषोमा॑ वासु रासुर॒ ग्नीषोमौ᳚ । \newline
45. अ॒ग्नीषोमौ॑ बिभ्रति बिभ्र त्य॒ग्नीषोमा॑ व॒ग्नीषोमौ॑ बिभ्रति । \newline
46. अ॒ग्नीषोमा॒वित्य॒ग्नी - सोमौ᳚ । \newline
47. बि॒भ्र॒ त्याप॒ आपो॑ बिभ्रति बिभ्र॒ त्यापः॑ । \newline
48. आप॒ इदिदाप॒ आप॒ इत् । \newline
49. इत् ता स्ता इदित् ताः । \newline
50. ता इति॒ ताः । \newline
51. ती॒व्रो रसो॒ रस॑ स्ती॒व्र स्ती॒व्रो रसः॑ । \newline
52. रसो॑ मधु॒पृचा᳚म् मधु॒पृचाꣳ॒॒ रसो॒ रसो॑ मधु॒पृचा᳚म् । \newline
53. म॒धु॒पृचा॑ मरङ्ग॒मो अ॑रङ्ग॒मो म॑धु॒पृचा᳚म् मधु॒पृचा॑ मरङ्ग॒मः । \newline
54. म॒धु॒पृचा॒मिति॑ मधु - पृचा᳚म् । \newline

\textbf{Ghana Paata } \newline

1. तदा᳚प्नो दाप्नो॒त् तत् तदा᳚प्नो॒ दिन्द्र॒ इन्द्र॑ आप्नो॒त् तत् तदा᳚प्नो॒ दिन्द्रः॑ । \newline
2. आ॒प्नो॒ दिन्द्र॒ इन्द्र॑ आप्नो दाप्नो॒ दिन्द्रो॑ वो व॒ इन्द्र॑ आप्नो दाप्नो॒ दिन्द्रो॑ वः । \newline
3. इन्द्रो॑ वो व॒ इन्द्र॒ इन्द्रो॑ वो य॒तीर् य॒तीर् व॒ इन्द्र॒ इन्द्रो॑ वो य॒तीः । \newline
4. वो॒ य॒तीर् य॒तीर् वो॑ वो य॒ती स्तस्मा॒त् तस्मा᳚द् य॒तीर् वो॑ वो य॒ती स्तस्मा᳚त् । \newline
5. य॒ती स्तस्मा॒त् तस्मा᳚द् य॒तीर् य॒ती स्तस्मा॒ दाप॒ आप॒ स्तस्मा᳚द् य॒तीर् य॒ती स्तस्मा॒ दापः॑ । \newline
6. तस्मा॒ दाप॒ आप॒ स्तस्मा॒त् तस्मा॒ दापो॒ अन्वन् वाप॒ स्तस्मा॒त् तस्मा॒ दापो॒ अनु॑ । \newline
7. आपो॒ अन्वन् वाप॒ आपो॒ अनु॑ स्थन स्थ॒नान् वाप॒ आपो॒ अनु॑ स्थन । \newline
8. अनु॑ स्थन स्थ॒नान् वनु॑ स्थन । \newline
9. स्थ॒नेति॑ स्थन । \newline
10. अ॒प॒का॒मꣳ स्यन्द॑मानाः॒ स्यन्द॑माना अपका॒म म॑पका॒मꣳ स्यन्द॑माना॒ अवी॑वर॒ता वी॑वरत॒ स्यन्द॑माना अपका॒म म॑पका॒मꣳ स्यन्द॑माना॒ अवी॑वरत । \newline
11. अ॒प॒का॒ममित्य॑प - का॒मम् । \newline
12. स्यन्द॑माना॒ अवी॑वर॒ता वी॑वरत॒ स्यन्द॑मानाः॒ स्यन्द॑माना॒ अवी॑वरत वो॒ वो ऽवी॑वरत॒ स्यन्द॑मानाः॒ स्यन्द॑माना॒ अवी॑वरत वः । \newline
13. अवी॑वरत वो॒ वो ऽवी॑वर॒ता वी॑वरत वो॒ हिकꣳ॒॒ हिकं॒ ॅवो ऽवी॑वर॒ता वी॑वरत वो॒ हिक᳚म् । \newline
14. वो॒ हिकꣳ॒॒ हिकं॑ ॅवो वो॒ हिक᳚म् । \newline
15. हिक॒मिति॒ हिक᳚म् । \newline
16. इन्द्रो॑ वो व॒ इन्द्र॒ इन्द्रो॑ वः॒ शक्ति॑भिः॒ शक्ति॑भिर् व॒ इन्द्र॒ इन्द्रो॑ वः॒ शक्ति॑भिः । \newline
17. वः॒ शक्ति॑भिः॒ शक्ति॑भिर् वो वः॒ शक्ति॑भिर् देवीर् देवीः॒ शक्ति॑भिर् वो वः॒ शक्ति॑भिर् देवीः । \newline
18. शक्ति॑भिर् देवीर् देवीः॒ शक्ति॑भिः॒ शक्ति॑भिर् देवी॒ स्तस्मा॒त् तस्मा᳚द् देवीः॒ शक्ति॑भिः॒ शक्ति॑भिर् देवी॒ स्तस्मा᳚त् । \newline
19. शक्ति॑भि॒रिति॒ शक्ति॑ - भिः॒ । \newline
20. दे॒वी॒ स्तस्मा॒त् तस्मा᳚द् देवीर् देवी॒ स्तस्मा॒द् वार् वा स्तस्मा᳚द् देवीर् देवी॒ स्तस्मा॒द् वाः । \newline
21. तस्मा॒द् वार् वा स्तस्मा॒त् तस्मा॒द् वार् नाम॒ नाम॒ वा स्तस्मा॒त् तस्मा॒द् वार् नाम॑ । \newline
22. वार् नाम॒ नाम॒ वार् वार् नाम॑ वो वो॒ नाम॒ वार् वार् नाम॑ वः । \newline
23. नाम॑ वो वो॒ नाम॒ नाम॑ वो हि॒तꣳ हि॒तं ॅवो॒ नाम॒ नाम॑ वो हि॒तम् । \newline
24. वो॒ हि॒तꣳ हि॒तं ॅवो॑ वो हि॒तम् । \newline
25. हि॒तमिति॑ हि॒तम् । \newline
26. एको॑ दे॒वो दे॒व एक॒ एको॑ दे॒वो अप्यपि॑ दे॒व एक॒ एको॑ दे॒वो अपि॑ । \newline
27. दे॒वो अप्यपि॑ दे॒वो दे॒वो अप्य॑तिष्ठ दतिष्ठ॒ दपि॑ दे॒वो दे॒वो अप्य॑तिष्ठत् । \newline
28. अप्य॑तिष्ठ दतिष्ठ॒ दप्यप्य॑ तिष्ठ॒थ् स्यन्द॑मानाः॒ स्यन्द॑माना अतिष्ठ॒ दप्यप्य॑ तिष्ठ॒थ् स्यन्द॑मानाः । \newline
29. अ॒ति॒ष्ठ॒थ् स्यन्द॑मानाः॒ स्यन्द॑माना अतिष्ठ दतिष्ठ॒थ् स्यन्द॑माना यथाव॒शं ॅय॑थाव॒शꣳ स्यन्द॑माना अतिष्ठ दतिष्ठ॒थ् स्यन्द॑माना यथाव॒शम् । \newline
30. स्यन्द॑माना यथाव॒शं ॅय॑थाव॒शꣳ स्यन्द॑मानाः॒ स्यन्द॑माना यथाव॒शम् । \newline
31. य॒था॒व॒शमिति॑ यथा - व॒शम् । \newline
32. उदा॑निषु रानिषु॒ रुदु दा॑निषुर् म॒हीर् म॒ही रा॑निषु॒ रुदु दा॑निषुर् म॒हीः । \newline
33. आ॒नि॒षु॒र् म॒हीर् म॒हीरा॑ निषुरा निषुर् म॒ही रितीति॑ म॒ही रा॑निषु रानिषुर् म॒ही रिति॑ । \newline
34. म॒ही रितीति॑ म॒हीर् म॒हीरिति॒ तस्मा॒त् तस्मा॒ दिति॑ म॒हीर् म॒हीरिति॒ तस्मा᳚त् । \newline
35. इति॒ तस्मा॒त् तस्मा॒ दितीति॒ तस्मा॑ दुद॒क मु॑द॒कम् तस्मा॒ दितीति॒ तस्मा॑ दुद॒कम् । \newline
36. तस्मा॑ दुद॒क मु॑द॒कम् तस्मा॒त् तस्मा॑ दुद॒क मु॑च्यत उच्यत उद॒कम् तस्मा॒त् तस्मा॑ दुद॒क मु॑च्यते । \newline
37. उ॒द॒क मु॑च्यत उच्यत उद॒क मु॑द॒क मु॑च्यते । \newline
38. उ॒च्य॒त॒ इत्यु॑च्यते । \newline
39. आपो॑ भ॒द्रा भ॒द्रा आप॒ आपो॑ भ॒द्रा घृ॒तम् घृ॒तम् भ॒द्रा आप॒ आपो॑ भ॒द्रा घृ॒तम् । \newline
40. भ॒द्रा घृ॒तम् घृ॒तम् भ॒द्रा भ॒द्रा घृ॒त मिदिद् घृ॒तम् भ॒द्रा भ॒द्रा घृ॒त मित् । \newline
41. घृ॒त मिदिद् घृ॒तम् घृ॒त मिदाप॒ आप॒ इद् घृ॒तम् घृ॒त मिदापः॑ । \newline
42. इदाप॒ आप॒ इदिदाप॑ आसु रासु॒ राप॒ इदिदाप॑ आसुः । \newline
43. आप॑ आसु रासु॒ राप॒ आप॑ आसु र॒ग्नीषोमा॑ व॒ग्नीषोमा॑ वासु॒राप॒ आप॑ आसु र॒ग्नीषोमौ᳚ । \newline
44. आ॒सु॒ र॒ग्नीषोमा॑ व॒ग्नीषोमा॑ वासु रासु र॒ग्नीषोमौ॑ बिभ्रति बिभ्र त्य॒ग्नीषोमा॑ वासु रासु र॒ग्नीषोमौ॑ बिभ्रति । \newline
45. अ॒ग्नीषोमौ॑ बिभ्रति बिभ्र त्य॒ग्नीषोमा॑ व॒ग्नीषोमौ॑ बिभ्र॒ त्याप॒ आपो॑ बिभ्र त्य॒ग्नीषोमा॑ व॒ग्नीषोमौ॑ बिभ्र॒ त्यापः॑ । \newline
46. अ॒ग्नीषोमा॒वित्य॒ग्नी - सोमौ᳚ । \newline
47. बि॒भ्र॒ त्याप॒ आपो॑ बिभ्रति बिभ्र॒ त्याप॒ इदिदापो॑ बिभ्रति बिभ्र॒ त्याप॒ इत् । \newline
48. आप॒ इदिदाप॒ आप॒ इत् ता स्ता इदाप॒ आप॒ इत् ताः । \newline
49. इत् ता स्ता इदित् ताः । \newline
50. ता इति॒ ताः । \newline
51. ती॒व्रो रसो॒ रस॑ स्ती॒व्र स्ती॒व्रो रसो॑ मधु॒पृचा᳚म् मधु॒पृचाꣳ॒॒ रस॑ स्ती॒व्र स्ती॒व्रो रसो॑ मधु॒पृचा᳚म् । \newline
52. रसो॑ मधु॒पृचा᳚म् मधु॒पृचाꣳ॒॒ रसो॒ रसो॑ मधु॒पृचा॑ मरङ्ग॒मो अ॑रङ्ग॒मो म॑धु॒पृचाꣳ॒॒ रसो॒ रसो॑ मधु॒पृचा॑ मरङ्ग॒मः । \newline
53. म॒धु॒पृचा॑ मरङ्ग॒मो अ॑रङ्ग॒मो म॑धु॒पृचा᳚म् मधु॒पृचा॑ मरङ्ग॒म आ ऽर॑ङ्ग॒मो म॑धु॒पृचा᳚म् मधु॒पृचा॑ मरङ्ग॒म आ । \newline
54. म॒धु॒पृचा॒मिति॑ मधु - पृचा᳚म् । \newline
\pagebreak
\markright{ TS 5.6.1.4  \hfill https://www.vedavms.in \hfill}

\section{ TS 5.6.1.4 }

\textbf{TS 5.6.1.4 } \newline
\textbf{Samhita Paata} \newline

-मरंग॒म आ मा᳚ प्रा॒णेन॑ स॒ह वर्च॑सा गन्न् ॥ आदित् प॑श्याम्यु॒त वा॑ शृणो॒म्या मा॒ घोषो॑ गच्छति॒ वाङ्न॑ आसां । मन्ये॑ भेजा॒नो अ॒मृत॑स्य॒ तर्.हि॒ हिर॑ण्यवर्णा॒ अतृ॑पं ॅय॒दा वः॑ ॥ आपो॒ हि ष्ठा म॑यो॒ भुव॒स्ता न॑ ऊ॒र्जे द॑धातन । म॒हे रणा॑य॒ चक्ष॑से ॥ यो वः॑ शि॒वत॑मो॒ रस॒स्तस्य॑ भाजयते॒ह नः॑ । उ॒श॒तीरि॑व मा॒तरः॑ ( ) ॥ तस्मा॒ अरं ग॑माम वो॒ यस्य॒ क्षया॑य॒ जिन्व॑थ । आपो॑ ज॒नय॑था च नः ॥ \newline

\textbf{Pada Paata} \newline

अ॒र॒ङ्ग॒म इत्य॑रं - ग॒मः । एति॑ । मा॒ । प्रा॒णेनेति॑ प्र - अ॒नेन॑ । स॒ह । वर्च॑सा । ग॒न्न् ॥ आत् । इत् । प॒श्या॒मि॒ । उ॒त । वा॒ । शृ॒णो॒मि॒ । एति॑ । मा॒ । घोषः॑ । ग॒च्छ॒ति॒ । वाक् । नः॒ । आ॒सा॒म् ॥ मन्ये᳚ । भे॒जा॒नः । अ॒मृत॑स्य । तर्.हि॑ । हिर॑ण्यवर्णा॒ इति॒ हिर॑ण्य - व॒र्णाः॒ । अतृ॑पम् । य॒दा । वः॒ ॥ आपः॑ । हि । स्थ । म॒यो॒भुव॒ इति॑ मयः - भुवः॑ । ताः । नः॒ । ऊ॒र्जे । द॒धा॒त॒न॒ ॥ म॒हे । रणा॑य । चक्ष॑से ॥ यः । वः॒ । शि॒वत॑म॒ इति॑ शि॒व - त॒मः॒ । रसः॑ । तस्य॑ । भा॒ज॒य॒त॒ । इ॒ह । नः॒ ॥ उ॒श॒तीः । इ॒व॒ । मा॒तरः॑ ( ) ॥ तस्मै᳚ । अर᳚म् । ग॒मा॒म॒ । वः॒ । यस्य॑ । क्षया॑य । जिन्व॑थ ॥ आपः॑ । ज॒नय॑थ । च॒ । नः॒ ॥ दि॒वि । श्र॒य॒स्व॒ । अ॒न्तरि॑क्षे । य॒त॒स्व॒ । पृ॒थि॒व्या । समिति॑ । भ॒व॒ । ब्र॒ह्म॒व॒र्च॒समिति॑ ब्रह्म - व॒र्च॒सम् । अ॒सि॒ । ब्र॒ह्म॒व॒र्च॒सायेति॑ ब्रह्म - व॒र्च॒साय॑ । त्वा॒ ॥  \newline


\textbf{Krama Paata} \newline

अ॒र॒ङ्ग॒म आ । अ॒र॒ङ्ग॒म इत्य॑रम् - ग॒मः । आ मा᳚ । मा॒ प्रा॒णेन॑ । प्रा॒णेन॑ स॒ह । प्रा॒णेनेति॑ प्र - अ॒नेन॑ । स॒ह वर्च॑सा । वर्च॑सा गन्न् । ग॒न्निति॑ गन्न् ॥ आदित् । इत् प॑श्यामि । प॒श्या॒म्यु॒त । उ॒त वा᳚ । वा॒ शृ॒णो॒मि॒ । शृ॒णो॒म्या । आ मा᳚ । मा॒ घोषः॑ । घोषो॑ गच्छति । ग॒च्छ॒ति॒ वाक् । वाङ् नः॑ । न॒ आ॒सा॒म् । आ॒सा॒मित्या॑साम् ॥ मन्ये॑ भेजा॒नः । भे॒जा॒नो अ॒मृत॑स्य । अ॒मृत॑स्य॒ तर्.हि॑ । तर्.हि॒ हिर॑ण्यवर्णाः । हिर॑ण्यवर्णा॒ अतृ॑पम् । हिर॑ण्यवर्णा॒ इति॒ हिर॑ण्य - व॒र्णाः॒ । अतृ॑पम् ॅय॒दा । य॒दा वः॑ । व॒ इति॑ वः ॥ आपो॒ हि । हि ष्ठ । स्था म॑यो॒भुवः॑ । म॒यो॒भुव॒स्ताः । म॒यो॒भुव॒ इति॑ मयः - भुवः॑ । ता नः॑ । न॒ ऊ॒र्जे । ऊ॒र्जे द॑धातन । द॒धा॒त॒नेति॑ दधातन ॥ म॒हे रणा॑य । रणा॑य॒ चक्ष॑से । चक्ष॑स॒ इति॒ चक्ष॑से ॥ यो वः॑ । वः॒ शि॒वत॑मः । शि॒वत॑मो॒ रसः॑ । शि॒वत॑म॒ इति॑ शि॒व - त॒मः॒ । रस॒ स्तस्य॑ । तस्य॑ भाजयत । भा॒ज॒य॒ते॒ह । इ॒ह नः॑ । न॒ इति नः॑ ॥ उ॒श॒तीरि॑व । इ॒व॒ मा॒तरः॑ ( ) । मा॒तर॒ इति॑ मा॒तरः॑ ॥ तस्मा॒ अर᳚म् । अर॑म् गमाम । ग॒मा॒म॒ वः॒ । वो॒ यस्य॑ । यस्य॒ क्षया॑य । क्षया॑य॒ जिन्व॑थ । जिन्व॒थेति॒ जिन्व॑थ ॥ आपो॑ ज॒नय॑थ । ज॒नय॑था च । च॒ नः॒ । न॒ इति॑ नः ॥ दि॒वि श्र॑यस्व । श्र॒य॒स्वा॒न्तरि॑क्षे । अ॒न्तरि॑क्षे यतस्व । य॒त॒स्व॒ पृ॒थि॒व्या । पृ॒थि॒व्या सम् । सम् भ॑व । भ॒व॒ ब्र॒ह्म॒व॒र्च॒सम् । ब्र॒ह्म॒व॒र्च॒सम॑सि । ब्र॒ह्म॒व॒र्च॒समिति॑ ब्रह्म - व॒र्च॒सम् । अ॒सि॒ ब्र॒ह्म॒व॒र्च॒साय॑ । ब्र॒ह्म॒व॒र्च॒साय॑ त्वा । ब्र॒ह्म॒व॒र्च॒सायेति॑ ब्रह्म - व॒र्च॒साय॑ । त्वेति॑ त्वा । \newline

\textbf{Jatai Paata} \newline

1. अ॒र॒ङ्ग॒म आ ऽर॑ङ्ग॒मो अ॑रङ्ग॒म आ । \newline
2. अ॒र॒ङ्ग॒म इत्य॑रं - ग॒मः । \newline
3. आ मा॒ मा ऽऽमा᳚ । \newline
4. मा॒ प्रा॒णेन॑ प्रा॒णेन॑ मा मा प्रा॒णेन॑ । \newline
5. प्रा॒णेन॑ स॒ह स॒ह प्रा॒णेन॑ प्रा॒णेन॑ स॒ह । \newline
6. प्रा॒णेनेति॑ प्र - अ॒नेन॑ । \newline
7. स॒ह वर्च॑सा॒ वर्च॑सा स॒ह स॒ह वर्च॑सा । \newline
8. वर्च॑सा गन् ग॒न्॒. वर्च॑सा॒ वर्च॑सा गन्न् । \newline
9. ग॒न्निति॑ गन्न् । \newline
10. आदि दिदा दादित् । \newline
11. इत् प॑श्यामि पश्या॒मीदित् प॑श्यामि । \newline
12. प॒श्या॒ म्यु॒तोत प॑श्यामि पश्या म्यु॒त । \newline
13. उ॒त वा॑ वो॒तोत वा᳚ । \newline
14. वा॒ शृ॒णो॒मि॒ शृ॒णो॒मि॒ वा॒ वा॒ शृ॒णो॒मि॒ । \newline
15. शृ॒णो॒म्या शृ॑णोमि शृणो॒म्या । \newline
16. आ मा॒ मा ऽऽमा᳚ । \newline
17. मा॒ घोषो॒ घोषो॑ मा मा॒ घोषः॑ । \newline
18. घोषो॑ गच्छति गच्छति॒ घोषो॒ घोषो॑ गच्छति । \newline
19. ग॒च्छ॒ति॒ वाग् वाग् ग॑च्छति गच्छति॒ वाक् । \newline
20. वाङ् नो॑ नो॒ वाग् वाङ् नः॑ । \newline
21. न॒ आ॒सा॒ मा॒सा॒म् नो॒ न॒ आ॒सा॒म् । \newline
22. आ॒सा॒मित्या॑साम् । \newline
23. मन्ये॑ भेजा॒नो भे॑जा॒नो मन्ये॒ मन्ये॑ भेजा॒नः । \newline
24. भे॒जा॒नो अ॒मृत॑स्या॒ मृत॑स्य भेजा॒नो भे॑जा॒नो अ॒मृत॑स्य । \newline
25. अ॒मृत॑स्य॒ तर्.हि॒ तर्.ह्य॒मृत॑स्या॒ मृत॑स्य॒ तर्.हि॑ । \newline
26. तर्.हि॒ हिर॑ण्यवर्णा॒ हिर॑ण्यवर्णा॒ स्तर्.हि॒ तर्.हि॒ हिर॑ण्यवर्णाः । \newline
27. हिर॑ण्यवर्णा॒ अतृ॑प॒ मतृ॑पꣳ॒॒ हिर॑ण्यवर्णा॒ हिर॑ण्यवर्णा॒ अतृ॑पम् । \newline
28. हिर॑ण्यवर्णा॒ इति॒ हिर॑ण्य - व॒र्णाः॒ । \newline
29. अतृ॑पं ॅय॒दा य॒दा ऽतृ॑प॒ मतृ॑पं ॅय॒दा । \newline
30. य॒दा वो॑ वो य॒दा य॒दा वः॑ । \newline
31. व॒ इति॑ वः । \newline
32. आपो॒ हि ह्याप॒ आपो॒ हि । \newline
33. हि ष्ठ स्थ हि हि ष्ठ । \newline
34. स्था म॑यो॒भुवो॑ मयो॒भुवः॒ स्थ स्था म॑यो॒भुवः॑ । \newline
35. म॒यो॒भुव॒ स्ता स्ता म॑यो॒भुवो॑ मयो॒भुव॒ स्ताः । \newline
36. म॒यो॒भुव॒ इति॑ मयः - भुवः॑ । \newline
37. ता नो॑ न॒ स्ता स्ता नः॑ । \newline
38. न॒ ऊ॒र्ज ऊ॒र्जे नो॑ न ऊ॒र्जे । \newline
39. ऊ॒र्जे द॑धातन दधातनो॒र्ज ऊ॒र्जे द॑धातन । \newline
40. द॒धा॒त॒नेति॑ दधातन । \newline
41. म॒हे रणा॑य॒ रणा॑य म॒हे म॒हे रणा॑य । \newline
42. रणा॑य॒ चक्ष॑से॒ चक्ष॑से॒ रणा॑य॒ रणा॑य॒ चक्ष॑से । \newline
43. चक्ष॑स॒ इति॒ चक्ष॑से । \newline
44. यो वो॑ वो॒ यो यो वः॑ । \newline
45. वः॒ शि॒वत॑मः शि॒वत॑मो वो वः शि॒वत॑मः । \newline
46. शि॒वत॑मो॒ रसो॒ रसः॑ शि॒वत॑मः शि॒वत॑मो॒ रसः॑ । \newline
47. शि॒वत॑म॒ इति॑ शि॒व - त॒मः॒ । \newline
48. रस॒ स्तस्य॒ तस्य॒ रसो॒ रस॒ स्तस्य॑ । \newline
49. तस्य॑ भाजयत भाजयत॒ तस्य॒ तस्य॑ भाजयत । \newline
50. भा॒ज॒य॒ ते॒हेह भा॑जयत भाजयते॒ह । \newline
51. इ॒ह नो॑ न इ॒हे ह नः॑ । \newline
52. न॒ इति नः॑ । \newline
53. उ॒श॒ती रि॑वे वोश॒ती रु॑श॒ती रि॑व । \newline
54. इ॒व॒ मा॒तरो॑ मा॒तर॑ इवेव मा॒तरः॑ । \newline
55. मा॒तर॒ इति॑ मा॒तरः॑ । \newline
56. तस्मा॒ अर॒ मर॒म् तस्मै॒ तस्मा॒ अर᳚म् । \newline
57. अर॑म् गमाम गमा॒मा र॒ मर॑म् गमाम । \newline
58. ग॒मा॒म॒ वो॒ वो॒ ग॒मा॒म॒ ग॒मा॒म॒ वः॒ । \newline
59. वो॒ यस्य॒ यस्य॑ वो वो॒ यस्य॑ । \newline
60. यस्य॒ क्षया॑य॒ क्षया॑य॒ यस्य॒ यस्य॒ क्षया॑य । \newline
61. क्षया॑य॒ जिन्व॑थ॒ जिन्व॑थ॒ क्षया॑य॒ क्षया॑य॒ जिन्व॑थ । \newline
62. जिन्व॒थेति॒ जिन्व॑थ । \newline
63. आपो॑ ज॒नय॑थ ज॒नय॒थाप॒ आपो॑ ज॒नय॑थ । \newline
64. ज॒नय॑था च च ज॒नय॑थ ज॒नय॑था च । \newline
65. च॒ नो॒ न॒श्च॒ च॒ नः॒ । \newline
66. न॒ इति नः॑ । \newline
67. दि॒वि श्र॑यस्व श्रयस्व दि॒वि दि॒वि श्र॑यस्व । \newline
68. श्र॒य॒स्वा॒ न्तरि॑क्षे अ॒न्तरि॑क्षे श्रयस्व श्रयस्वा॒ न्तरि॑क्षे । \newline
69. अ॒न्तरि॑क्षे यतस्व यतस्वा॒ न्तरि॑क्षे अ॒न्तरि॑क्षे यतस्व । \newline
70. य॒त॒स्व॒ पृ॒थि॒व्या पृ॑थि॒व्या य॑तस्व यतस्व पृथि॒व्या । \newline
71. पृ॒थि॒व्या सꣳ सम् पृ॑थि॒व्या पृ॑थि॒व्या सम् । \newline
72. सम् भ॑व भव॒ सꣳ सम् भ॑व । \newline
73. भ॒व॒ ब्र॒ह्म॒व॒र्च॒सम् ब्र॑ह्मवर्च॒सम् भ॑व भव ब्रह्मवर्च॒सम् । \newline
74. ब्र॒ह्म॒व॒र्च॒स म॑स्यसि ब्रह्मवर्च॒सम् ब्र॑ह्मवर्च॒स म॑सि । \newline
75. ब्र॒ह्म॒व॒र्च॒समिति॑ ब्रह्म - व॒र्च॒सम् । \newline
76. अ॒सि॒ ब्र॒ह्म॒व॒र्च॒साय॑ ब्रह्मवर्च॒साया᳚स्यसि ब्रह्मवर्च॒साय॑ । \newline
77. ब्र॒ह्म॒व॒र्च॒साय॑ त्वा त्वा ब्रह्मवर्च॒साय॑ ब्रह्मवर्च॒साय॑ त्वा । \newline
78. ब्र॒ह्म॒व॒र्च॒सायेति॑ ब्रह्म - व॒र्च॒साय॑ । \newline
79. त्वेति॑ त्वा । \newline

\textbf{Ghana Paata } \newline

1. अ॒र॒ङ्ग॒म आ ऽर॑ङ्ग॒मो अ॑रङ्ग॒म आ मा॒ मा ऽर॑ङ्ग॒मो अ॑रङ्ग॒म आ मा᳚ । \newline
2. अ॒र॒ङ्ग॒म इत्य॑रं - ग॒मः । \newline
3. आ मा॒ मा ऽऽमा᳚ प्रा॒णेन॑ प्रा॒णेन॒ मा ऽऽमा᳚ प्रा॒णेन॑ । \newline
4. मा॒ प्रा॒णेन॑ प्रा॒णेन॑ मा मा प्रा॒णेन॑ स॒ह स॒ह प्रा॒णेन॑ मा मा प्रा॒णेन॑ स॒ह । \newline
5. प्रा॒णेन॑ स॒ह स॒ह प्रा॒णेन॑ प्रा॒णेन॑ स॒ह वर्च॑सा॒ वर्च॑सा स॒ह प्रा॒णेन॑ प्रा॒णेन॑ स॒ह वर्च॑सा । \newline
6. प्रा॒णेनेति॑ प्र - अ॒नेन॑ । \newline
7. स॒ह वर्च॑सा॒ वर्च॑सा स॒ह स॒ह वर्च॑सा गन् ग॒न्॒. वर्च॑सा स॒ह स॒ह वर्च॑सा गन्न् । \newline
8. वर्च॑सा गन् ग॒न्॒. वर्च॑सा॒ वर्च॑सा गन्न् । \newline
9. ग॒न्निति॑ गन्न् । \newline
10. आदिदि दादादित् प॑श्यामि पश्या॒मी दादा दित् प॑श्यामि । \newline
11. इत् प॑श्यामि पश्या॒मी दित् प॑श्या म्यु॒तोत प॑श्या॒ मीदित् प॑श्या म्यु॒त । \newline
12. प॒श्या॒ म्यु॒तोत प॑श्यामि पश्या म्यु॒त वा॑ वो॒त प॑श्यामि पश्या म्यु॒त वा᳚ । \newline
13. उ॒त वा॑ वो॒तोत वा॑ शृणोमि शृणोमि वो॒तोत वा॑ शृणोमि । \newline
14. वा॒ शृ॒णो॒मि॒ शृ॒णो॒मि॒ वा॒ वा॒ शृ॒णो॒म्या शृ॑णोमि वा वा शृणो॒म्या । \newline
15. शृ॒णो॒म्या शृ॑णोमि शृणो॒म्या मा॒ मा ऽऽशृ॑णोमि शृणो॒म्या मा᳚ । \newline
16. आ मा॒ मा ऽऽमा॒ घोषो॒ घोषो॒ मा ऽऽमा॒ घोषः॑ । \newline
17. मा॒ घोषो॒ घोषो॑ मा मा॒ घोषो॑ गच्छति गच्छति॒ घोषो॑ मा मा॒ घोषो॑ गच्छति । \newline
18. घोषो॑ गच्छति गच्छति॒ घोषो॒ घोषो॑ गच्छति॒ वाग् वाग् ग॑च्छति॒ घोषो॒ घोषो॑ गच्छति॒ वाक् । \newline
19. ग॒च्छ॒ति॒ वाग् वाग् ग॑च्छति गच्छति॒ वाङ् नो॑ नो॒ वाग् ग॑च्छति गच्छति॒ वाङ् नः॑ । \newline
20. वाङ् नो॑ नो॒ वाग् वाङ् न॑ आसा मासाम् नो॒ वाग् वाङ् न॑ आसाम् । \newline
21. न॒ आ॒सा॒ मा॒सा॒म् नो॒ न॒ आ॒सा॒म् । \newline
22. आ॒सा॒मित्या॑साम् । \newline
23. मन्ये॑ भेजा॒नो भे॑जा॒नो मन्ये॒ मन्ये॑ भेजा॒नो अ॒मृत॑स्या॒ मृत॑स्य भेजा॒नो मन्ये॒ मन्ये॑ भेजा॒नो अ॒मृत॑स्य । \newline
24. भे॒जा॒नो अ॒मृत॑स्या॒ मृत॑स्य भेजा॒नो भे॑जा॒नो अ॒मृत॑स्य॒ तर्.हि॒ तर्.ह्य॒मृत॑स्य भेजा॒नो भे॑जा॒नो अ॒मृत॑स्य॒ तर्.हि॑ । \newline
25. अ॒मृत॑स्य॒ तर्.हि॒ तर्.ह्य॒मृत॑स्या॒ मृत॑स्य॒ तर्.हि॒ हिर॑ण्यवर्णा॒ हिर॑ण्यवर्णा॒ स्तर्.ह्य॒मृत॑स्या॒ मृत॑स्य॒ तर्.हि॒ हिर॑ण्यवर्णाः । \newline
26. तर्.हि॒ हिर॑ण्यवर्णा॒ हिर॑ण्यवर्णा॒ स्तर्.हि॒ तर्.हि॒ हिर॑ण्यवर्णा॒ अतृ॑प॒ मतृ॑पꣳ॒॒ हिर॑ण्यवर्णा॒ स्तर्.हि॒ तर्.हि॒ हिर॑ण्यवर्णा॒ अतृ॑पम् । \newline
27. हिर॑ण्यवर्णा॒ अतृ॑प॒ मतृ॑पꣳ॒॒ हिर॑ण्यवर्णा॒ हिर॑ण्यवर्णा॒ अतृ॑पं ॅय॒दा य॒दा ऽतृ॑पꣳ॒॒ हिर॑ण्यवर्णा॒ हिर॑ण्यवर्णा॒ अतृ॑पं ॅय॒दा । \newline
28. हिर॑ण्यवर्णा॒ इति॒ हिर॑ण्य - व॒र्णाः॒ । \newline
29. अतृ॑पं ॅय॒दा य॒दा ऽतृ॑प॒ मतृ॑पं ॅय॒दा वो॑ वो य॒दा ऽतृ॑प॒ मतृ॑पं ॅय॒दा वः॑ । \newline
30. य॒दा वो॑ वो य॒दा य॒दा वः॑ । \newline
31. व॒ इति॑ वः । \newline
32. आपो॒ हि ह्याप॒ आपो॒ हि ष्ठ स्थ ह्याप॒ आपो॒ हि ष्ठ । \newline
33. हि ष्ठ स्थ हि हि ष्ठा म॑यो॒भुवो॑ मयो॒भुवः॒ स्थ हि हि ष्ठा म॑यो॒भुवः॑ । \newline
34. स्था म॑यो॒भुवो॑ मयो॒भुवः॒ स्थ स्था म॑यो॒भुव॒ स्ता स्ता म॑यो॒भुवः॒ स्थ स्था म॑यो॒भुव॒ स्ताः । \newline
35. म॒यो॒भुव॒ स्ता स्ता म॑यो॒भुवो॑ मयो॒भुव॒ स्ता नो॑ न॒ स्ता म॑यो॒भुवो॑ मयो॒भुव॒ स्ता नः॑ । \newline
36. म॒यो॒भुव॒ इति॑ मयः - भुवः॑ । \newline
37. ता नो॑ न॒ स्ता स्ता न॑ ऊ॒र्ज ऊ॒र्जे न॒ स्ता स्ता न॑ ऊ॒र्जे । \newline
38. न॒ ऊ॒र्ज ऊ॒र्जे नो॑ न ऊ॒र्जे द॑धातन दधात नो॒र्जे नो॑ न ऊ॒र्जे द॑धातन । \newline
39. ऊ॒र्जे द॑धातन दधात नो॒र्ज ऊ॒र्जे द॑धातन । \newline
40. द॒धा॒त॒नेति॑ दधातन । \newline
41. म॒हे रणा॑य॒ रणा॑य म॒हे म॒हे रणा॑य॒ चक्ष॑से॒ चक्ष॑से॒ रणा॑य म॒हे म॒हे रणा॑य॒ चक्ष॑से । \newline
42. रणा॑य॒ चक्ष॑से॒ चक्ष॑से॒ रणा॑य॒ रणा॑य॒ चक्ष॑से । \newline
43. चक्ष॑स॒ इति॒ चक्ष॑से । \newline
44. यो वो॑ वो॒ यो यो वः॑ शि॒वत॑मः शि॒वत॑मो वो॒ यो यो वः॑ शि॒वत॑मः । \newline
45. वः॒ शि॒वत॑मः शि॒वत॑मो वो वः शि॒वत॑मो॒ रसो॒ रसः॑ शि॒वत॑मो वो वः शि॒वत॑मो॒ रसः॑ । \newline
46. शि॒वत॑मो॒ रसो॒ रसः॑ शि॒वत॑मः शि॒वत॑मो॒ रस॒ स्तस्य॒ तस्य॒ रसः॑ शि॒वत॑मः शि॒वत॑मो॒ रस॒ स्तस्य॑ । \newline
47. शि॒वत॑म॒ इति॑ शि॒व - त॒मः॒ । \newline
48. रस॒ स्तस्य॒ तस्य॒ रसो॒ रस॒ स्तस्य॑ भाजयत भाजयत॒ तस्य॒ रसो॒ रस॒ स्तस्य॑ भाजयत । \newline
49. तस्य॑ भाजयत भाजयत॒ तस्य॒ तस्य॑ भाजय ते॒हेह भा॑जयत॒ तस्य॒ तस्य॑ भाजयते॒ह । \newline
50. भा॒ज॒य॒ ते॒हेह भा॑जयत भाजयते॒ह नो॑ न इ॒ह भा॑जयत भाजयते॒ह नः॑ । \newline
51. इ॒ह नो॑ न इ॒हेह नः॑ । \newline
52. न॒ इति नः॑ । \newline
53. उ॒श॒ती रि॑वे वोश॒ती रु॑श॒ती रि॑व मा॒तरो॑ मा॒तर॑ इवोश॒ती रु॑श॒ती रि॑व मा॒तरः॑ । \newline
54. इ॒व॒ मा॒तरो॑ मा॒तर॑ इवेव मा॒तरः॑ । \newline
55. मा॒तर॒ इति॑ मा॒तरः॑ । \newline
56. तस्मा॒ अर॒ मर॒म् तस्मै॒ तस्मा॒ अर॑म् गमाम गमा॒मा र॒म् तस्मै॒ तस्मा॒ अर॑म् गमाम । \newline
57. अर॑म् गमाम गमा॒मा र॒ मर॑म् गमाम वो वो गमा॒मा र॒ मर॑म् गमाम वः । \newline
58. ग॒मा॒म॒ वो॒ वो॒ ग॒मा॒म॒ ग॒मा॒म॒ वो॒ यस्य॒ यस्य॑ वो गमाम गमाम वो॒ यस्य॑ । \newline
59. वो॒ यस्य॒ यस्य॑ वो वो॒ यस्य॒ क्षया॑य॒ क्षया॑य॒ यस्य॑ वो वो॒ यस्य॒ क्षया॑य । \newline
60. यस्य॒ क्षया॑य॒ क्षया॑य॒ यस्य॒ यस्य॒ क्षया॑य॒ जिन्व॑थ॒ जिन्व॑थ॒ क्षया॑य॒ यस्य॒ यस्य॒ क्षया॑य॒ जिन्व॑थ । \newline
61. क्षया॑य॒ जिन्व॑थ॒ जिन्व॑थ॒ क्षया॑य॒ क्षया॑य॒ जिन्व॑थ । \newline
62. जिन्व॒थेति॒ जिन्व॑थ । \newline
63. आपो॑ ज॒नय॑थ ज॒नय॒ थाप॒ आपो॑ ज॒नय॑था च च ज॒नय॒ थाप॒ आपो॑ ज॒नय॑था च । \newline
64. ज॒नय॑था च च ज॒नय॑थ ज॒नय॑था च नो नश्च ज॒नय॑थ ज॒नय॑था च नः । \newline
65. च॒ नो॒ न॒श्च॒ च॒ नः॒ । \newline
66. न॒ इति नः॑ । \newline
67. दि॒वि श्र॑यस्व श्रयस्व दि॒वि दि॒वि श्र॑यस्वा॒ न्तरि॑क्षे अ॒न्तरि॑क्षे श्रयस्व दि॒वि दि॒वि श्र॑यस्वा॒ न्तरि॑क्षे । \newline
68. श्र॒य॒स्वा॒ न्तरि॑क्षे अ॒न्तरि॑क्षे श्रयस्व श्रयस्वा॒ न्तरि॑क्षे यतस्व यतस्वा॒ न्तरि॑क्षे श्रयस्व श्रयस्वा॒ न्तरि॑क्षे यतस्व । \newline
69. अ॒न्तरि॑क्षे यतस्व यतस्वा॒ न्तरि॑क्षे अ॒न्तरि॑क्षे यतस्व पृथि॒व्या पृ॑थि॒व्या य॑तस्वा॒ न्तरि॑क्षे 
अ॒न्तरि॑क्षे यतस्व पृथि॒व्या । \newline
70. य॒त॒स्व॒ पृ॒थि॒व्या पृ॑थि॒व्या य॑तस्व यतस्व पृथि॒व्या सꣳ सम् पृ॑थि॒व्या य॑तस्व यतस्व पृथि॒व्या सम् । \newline
71. पृ॒थि॒व्या सꣳ सम् पृ॑थि॒व्या पृ॑थि॒व्या सम् भ॑व भव॒ सम् पृ॑थि॒व्या पृ॑थि॒व्या सम् भ॑व । \newline
72. सम् भ॑व भव॒ सꣳ सम् भ॑व ब्रह्मवर्च॒सम् ब्र॑ह्मवर्च॒सम् भ॑व॒ सꣳ सम् भ॑व ब्रह्मवर्च॒सम् । \newline
73. भ॒व॒ ब्र॒ह्म॒व॒र्च॒सम् ब्र॑ह्मवर्च॒सम् भ॑व भव ब्रह्मवर्च॒स म॑स्यसि ब्रह्मवर्च॒सम् भ॑व भव ब्रह्मवर्च॒स म॑सि । \newline
74. ब्र॒ह्म॒व॒र्च॒स म॑स्यसि ब्रह्मवर्च॒सम् ब्र॑ह्मवर्च॒स म॑सि ब्रह्मवर्च॒साय॑ ब्रह्मवर्च॒साया॑सि ब्रह्मवर्च॒सम् ब्र॑ह्मवर्च॒स म॑सि ब्रह्मवर्च॒साय॑ । \newline
75. ब्र॒ह्म॒व॒र्च॒समिति॑ ब्रह्म - व॒र्च॒सम् । \newline
76. अ॒सि॒ ब्र॒ह्म॒व॒र्च॒साय॑ ब्रह्मवर्च॒साया᳚स्यसि ब्रह्मवर्च॒साय॑ त्वा त्वा ब्रह्मवर्च॒साया᳚स्यसि ब्रह्मवर्च॒साय॑ त्वा । \newline
77. ब्र॒ह्म॒व॒र्च॒साय॑ त्वा त्वा ब्रह्मवर्च॒साय॑ ब्रह्मवर्च॒साय॑ त्वा । \newline
78. ब्र॒ह्म॒व॒र्च॒सायेति॑ ब्रह्म - व॒र्च॒साय॑ । \newline
79. त्वेति॑ त्वा । \newline
\pagebreak
\markright{ TS 5.6.2.1  \hfill https://www.vedavms.in \hfill}

\section{ TS 5.6.2.1 }

\textbf{TS 5.6.2.1 } \newline
\textbf{Samhita Paata} \newline

अ॒पां ग्रहा᳚न् गृह्णात्ये॒तद्वाव रा॑ज॒सूयं॒ ॅयदे॒ते ग्रहाः᳚ स॒वो᳚ ऽग्निर्व॑रुणस॒वो रा॑ज॒सूय॑-मग्निस॒व-श्चित्य॒स्ताभ्या॑-मे॒व सू॑य॒तेऽथो॑ उ॒भावे॒व लो॒काव॒भि ज॑यति॒ यश्च॑ राज॒सूये॑नेजा॒नस्य॒ यश्चा᳚ऽग्नि॒चित॒ आपो॑ भव॒न्त्यापो॒ वा अ॒ग्नेर्भ्रातृ॑व्या॒ यद॒पो᳚ ऽग्नेर॒धस्ता॑दुप॒ दधा॑ति॒ भ्रातृ॑व्याभिभूत्यै॒ भव॑त्या॒त्मना॒ परा᳚ऽस्य॒ भ्रातृ॑व्यो भवत्य॒मृतं॒ - [  ] \newline

\textbf{Pada Paata} \newline

अ॒पाम् । ग्रहान्॑ । गृ॒ह्णा॒ति॒ । ए॒तत् । वाव । रा॒ज॒सूय॒मिति॑ राज - सूय᳚म् । यत् । ए॒ते । ग्रहाः᳚ । स॒वः । अ॒ग्निः । व॒रु॒ण॒स॒व इति॑ वरुण - स॒वः । रा॒ज॒सूय॒मिति॑ राज - सूय᳚म् । अ॒ग्नि॒स॒व इत्य॑ग्नि - स॒वः । चित्यः॑ । ताभ्या᳚म् । ए॒व । सू॒य॒ते॒ । अथो॒ इति॑ । उ॒भौ । ए॒व । लो॒कौ । अ॒भीति॑ । ज॒य॒ति॒ । यः । च॒ । रा॒ज॒सूये॒नेति॑ राज - सूये॑न । ई॒जा॒नस्य॑ । यः । च॒ । अ॒ग्नि॒चित॒ इत्य॑ग्नि-चितः॑ । आपः॑ । भ॒व॒न्ति॒ । आपः॑ । वै । अ॒ग्नेः । भ्रातृ॑व्याः । यत् । अ॒पः । अ॒ग्नेः । अ॒धस्ता᳚त् । उ॒प॒दधा॒तीत्यु॑प - दधा॑ति । भ्रातृ॑व्याभिभूत्या॒ इति॒ भ्रातृ॑व्य - अ॒भि॒भू॒त्यै॒ । भव॑ति । आ॒त्मना᳚ । परेति॑ । अ॒स्य॒ । भ्रातृ॑व्यः । भ॒व॒ति॒ । अ॒मृत᳚म् ।  \newline


\textbf{Krama Paata} \newline

अ॒पाम् ग्रहान्॑ । ग्रहा᳚न् गृह्णाति । गृ॒ह्णा॒त्ये॒तत् । ए॒तद् वाव । वाव रा॑ज॒सूय᳚म् । रा॒ज॒सूय॒म् ॅयत् । रा॒ज॒सूय॒मिति॑ राज - सूय᳚म् । यदे॒ते । ए॒ते ग्रहाः᳚ । ग्रहाः᳚ स॒वः । स॒वो᳚ऽग्निः । अ॒ग्निर् व॑रुणस॒वः । व॒रु॒ण॒स॒वो रा॑ज॒सूय᳚म् । व॒रु॒ण॒स॒व इति॑ वरुण - स॒वः । रा॒ज॒सूय॑मग्निस॒वः । रा॒ज॒सूय॒मिति॑ राज - सूय᳚म् । अ॒ग्नि॒स॒वश्चित्यः॑ । अ॒ग्नि॒स॒व इत्य॑ग्नि - स॒वः । चित्य॒स्ताभ्या᳚म् । ताभ्या॑मे॒व । ए॒व सू॑यते । सू॒य॒तेऽथो᳚ । अथो॑ उ॒भौ । अथो॒ इत्यथो᳚ । उ॒भावे॒व । ए॒व लो॒कौ । लो॒काव॒भि । अ॒भि ज॑यति । ज॒य॒ति॒ यः । यश्च॑ । च॒ रा॒ज॒सूये॑न । रा॒ज॒सूये॑नेजा॒नस्य॑ । रा॒ज॒सूये॒नेति॑ राज - सूये॑न । ई॒जा॒नस्य॒ यः । यश्च॑ । चा॒ग्नि॒चितः॑ । अ॒ग्नि॒चित॒ आपः॑ । अ॒ग्नि॒चित॒ इत्य॑ग्नि - चितः॑ । आपो॑ भवन्ति । भ॒व॒न्त्यापः॑ । आपो॒ वै । वा अ॒ग्नेः । अ॒ग्नेर् भ्रातृ॑व्याः । भ्रातृ॑व्या॒ यत् । यद॒पः । अ॒पो᳚ऽग्नेः । अ॒ग्नेर॒धस्ता᳚त् । अ॒धस्ता॑दुप॒दधा॑ति । उ॒प॒दधा॑ति॒ भ्रातृ॑व्याभिभूत्यै । उ॒प॒दधा॒तीत्यु॑प - दधा॑ति । भ्रातृ॑व्याभिभूत्यै॒ भव॑ति । भ्रातृ॑व्याभिभूत्या॒ इति॒ भ्रातृ॑व्य - अ॒भि॒भू॒त्यै॒ । भव॑त्या॒त्मना᳚ । आ॒त्मना॒ परा᳚ । परा᳚ऽस्य । अ॒स्य॒ भ्रातृ॑व्यः । भ्रातृ॑व्यो भवति । भ॒व॒त्य॒मृत᳚म् । अ॒मृत॒म् ॅवै \newline

\textbf{Jatai Paata} \newline

1. अ॒पाम् ग्रहा॒न् ग्रहा॑ न॒पा म॒पाम् ग्रहान्॑ । \newline
2. ग्रहा᳚न् गृह्णाति गृह्णाति॒ ग्रहा॒न् ग्रहा᳚न् गृह्णाति । \newline
3. गृ॒ह्णा॒ त्ये॒त दे॒तद् गृ॑ह्णाति गृह्णा त्ये॒तत् । \newline
4. ए॒तद् वाव वावैत दे॒तद् वाव । \newline
5. वाव रा॑ज॒सूयꣳ॑ राज॒सूयं॒ ॅवाव वाव रा॑ज॒सूय᳚म् । \newline
6. रा॒ज॒सूयं॒ ॅयद् यद् रा॑ज॒सूयꣳ॑ राज॒सूयं॒ ॅयत् । \newline
7. रा॒ज॒सूय॒मिति॑ राज - सूय᳚म् । \newline
8. यदे॒त ए॒ते यद् यदे॒ते । \newline
9. ए॒ते ग्रहा॒ ग्रहा॑ ए॒त ए॒ते ग्रहाः᳚ । \newline
10. ग्रहाः᳚ स॒वः स॒वो ग्रहा॒ ग्रहाः᳚ स॒वः । \newline
11. स॒वो᳚ ऽग्नि र॒ग्निः स॒वः स॒वो᳚ ऽग्निः । \newline
12. अ॒ग्निर् व॑रुणस॒वो व॑रुणस॒वो᳚ ऽग्नि र॒ग्निर् व॑रुणस॒वः । \newline
13. व॒रु॒ण॒स॒वो रा॑ज॒सूयꣳ॑ राज॒सूयं॑ ॅवरुणस॒वो व॑रुणस॒वो रा॑ज॒सूय᳚म् । \newline
14. व॒रु॒ण॒स॒व इति॑ वरुण - स॒वः । \newline
15. रा॒ज॒सूय॑ मग्निस॒वो᳚ ऽग्निस॒वो रा॑ज॒सूयꣳ॑ राज॒सूय॑ मग्निस॒वः । \newline
16. रा॒ज॒सूय॒मिति॑ राज - सूय᳚म् । \newline
17. अ॒ग्नि॒स॒व श्चित्य॒ श्चित्यो᳚ ऽग्निस॒वो᳚ ऽग्निस॒व श्चित्यः॑ । \newline
18. अ॒ग्नि॒स॒व इत्य॑ग्नि - स॒वः । \newline
19. चित्य॒ स्ताभ्या॒म् ताभ्या॒म् चित्य॒ श्चित्य॒ स्ताभ्या᳚म् । \newline
20. ताभ्या॑ मे॒वैव ताभ्या॒म् ताभ्या॑ मे॒व । \newline
21. ए॒व सू॑यते सूयत ए॒वैव सू॑यते । \newline
22. सू॒य॒ते ऽथो॒ अथो॑ सूयते सूय॒ते ऽथो᳚ । \newline
23. अथो॑ उ॒भा वु॒भा वथो॒ अथो॑ उ॒भौ । \newline
24. अथो॒ इत्यथो᳚ । \newline
25. उ॒भा वे॒वै वोभा वु॒भा वे॒व । \newline
26. ए॒व लो॒कौ लो॒का वे॒वैव लो॒कौ । \newline
27. लो॒का व॒भ्य॑भि लो॒कौ लो॒का व॒भि । \newline
28. अ॒भि ज॑यति जय त्य॒भ्य॑भि ज॑यति । \newline
29. ज॒य॒ति॒ यो यो ज॑यति जयति॒ यः । \newline
30. यश्च॑ च॒ यो यश्च॑ । \newline
31. च॒ रा॒ज॒सूये॑न राज॒सूये॑न च च राज॒सूये॑न । \newline
32. रा॒ज॒सूये॑ नेजा॒न स्ये॑जा॒नस्य॑ राज॒सूये॑न राज॒सूये॑ नेजा॒नस्य॑ । \newline
33. रा॒ज॒सूये॒नेति॑ राज - सूये॑न । \newline
34. ई॒जा॒नस्य॒ यो य ई॑जा॒नस्ये॑ जा॒नस्य॒ यः । \newline
35. यश्च॑ च॒ यो यश्च॑ । \newline
36. चा॒ग्नि॒चितो᳚ ऽग्नि॒चित॑श्च चाग्नि॒चितः॑ । \newline
37. अ॒ग्नि॒चित॒ आप॒ आपो᳚ ऽग्नि॒चितो᳚ ऽग्नि॒चित॒ आपः॑ । \newline
38. अ॒ग्नि॒चित॒ इत्य॑ग्नि - चितः॑ । \newline
39. आपो॑ भवन्ति भव॒न् त्याप॒ आपो॑ भवन्ति । \newline
40. भ॒व॒न् त्याप॒ आपो॑ भवन्ति भव॒न् त्यापः॑ । \newline
41. आपो॒ वै वा आप॒ आपो॒ वै । \newline
42. वा अ॒ग्ने र॒ग्नेर् वै वा अ॒ग्नेः । \newline
43. अ॒ग्नेर् भ्रातृ॑व्या॒ भ्रातृ॑व्या अ॒ग्ने र॒ग्नेर् भ्रातृ॑व्याः । \newline
44. भ्रातृ॑व्या॒ यद् यद् भ्रातृ॑व्या॒ भ्रातृ॑व्या॒ यत् । \newline
45. यद॒पो॑ ऽपो यद् यद॒पः । \newline
46. अ॒पो᳚ ऽग्ने र॒ग्ने र॒पो᳚(1॒) ऽपो᳚ ऽग्नेः । \newline
47. अ॒ग्ने र॒धस्ता॑ द॒धस्ता॑ द॒ग्ने र॒ग्ने र॒धस्ता᳚त् । \newline
48. अ॒धस्ता॑ दुप॒दधा᳚ त्युप॒दधा᳚ त्य॒धस्ता॑ द॒धस्ता॑ दुप॒दधा॑ति । \newline
49. उ॒प॒दधा॑ति॒ भ्रातृ॑व्याभिभूत्यै॒ भ्रातृ॑व्याभिभूत्या उप॒दधा᳚ त्युप॒दधा॑ति॒ भ्रातृ॑व्याभिभूत्यै । \newline
50. उ॒प॒दधा॒तीत्यु॑प - दधा॑ति । \newline
51. भ्रातृ॑व्याभिभूत्यै॒ भव॑ति॒ भव॑ति॒ भ्रातृ॑व्याभिभूत्यै॒ भ्रातृ॑व्याभिभूत्यै॒ भव॑ति । \newline
52. भ्रातृ॑व्याभिभूत्या॒ इति॒ भ्रातृ॑व्य - अ॒भि॒भू॒त्यै॒ । \newline
53. भव॑ त्या॒त्मना॒ ऽऽत्मना॒ भव॑ति॒ भव॑ त्या॒त्मना᳚ । \newline
54. आ॒त्मना॒ परा॒ परा॒ ऽऽत्मना॒ ऽऽत्मना॒ परा᳚ । \newline
55. परा᳚ ऽस्यास्य॒ परा॒ परा᳚ ऽस्य । \newline
56. अ॒स्य॒ भ्रातृ॑व्यो॒ भ्रातृ॑व्यो ऽस्यास्य॒ भ्रातृ॑व्यः । \newline
57. भ्रातृ॑व्यो भवति भवति॒ भ्रातृ॑व्यो॒ भ्रातृ॑व्यो भवति । \newline
58. भ॒व॒ त्य॒मृत॑ म॒मृत॑म् भवति भव त्य॒मृत᳚म् । \newline
59. अ॒मृतं॒ ॅवै वा अ॒मृत॑ म॒मृतं॒ ॅवै । \newline

\textbf{Ghana Paata } \newline

1. अ॒पाम् ग्रहा॒न् ग्रहा॑ न॒पा म॒पाम् ग्रहा᳚न् गृह्णाति गृह्णाति॒ ग्रहा॑ न॒पा म॒पाम् ग्रहा᳚न् गृह्णाति । \newline
2. ग्रहा᳚न् गृह्णाति गृह्णाति॒ ग्रहा॒न् ग्रहा᳚न् गृह्णा त्ये॒त दे॒तद् गृ॑ह्णाति॒ ग्रहा॒न् ग्रहा᳚न् गृह्णा त्ये॒तत् । \newline
3. गृ॒ह्णा॒ त्ये॒त दे॒तद् गृ॑ह्णाति गृह्णा त्ये॒तद् वाव वावैतद् गृ॑ह्णाति गृह्णा त्ये॒तद् वाव । \newline
4. ए॒तद् वाव वावैत दे॒तद् वाव रा॑ज॒सूयꣳ॑ राज॒सूयं॒ ॅवावैत दे॒तद् वाव रा॑ज॒सूय᳚म् । \newline
5. वाव रा॑ज॒सूयꣳ॑ राज॒सूयं॒ ॅवाव वाव रा॑ज॒सूयं॒ ॅयद् यद् रा॑ज॒सूयं॒ ॅवाव वाव रा॑ज॒सूयं॒ ॅयत् । \newline
6. रा॒ज॒सूयं॒ ॅयद् यद् रा॑ज॒सूयꣳ॑ राज॒सूयं॒ ॅयदे॒त ए॒ते यद् रा॑ज॒सूयꣳ॑ राज॒सूयं॒ ॅयदे॒ते । \newline
7. रा॒ज॒सूय॒मिति॑ राज - सूय᳚म् । \newline
8. यदे॒त ए॒ते यद् यदे॒ते ग्रहा॒ ग्रहा॑ ए॒ते यद् यदे॒ते ग्रहाः᳚ । \newline
9. ए॒ते ग्रहा॒ ग्रहा॑ ए॒त ए॒ते ग्रहाः᳚ स॒वः स॒वो ग्रहा॑ ए॒त ए॒ते ग्रहाः᳚ स॒वः । \newline
10. ग्रहाः᳚ स॒वः स॒वो ग्रहा॒ ग्रहाः᳚ स॒वो᳚ ऽग्नि र॒ग्निः स॒वो ग्रहा॒ ग्रहाः᳚ स॒वो᳚ ऽग्निः । \newline
11. स॒वो᳚ ऽग्नि र॒ग्निः स॒वः स॒वो᳚ ऽग्निर् व॑रुणस॒वो व॑रुणस॒वो᳚ ऽग्निः स॒वः स॒वो᳚ ऽग्निर् व॑रुणस॒वः । \newline
12. अ॒ग्निर् व॑रुणस॒वो व॑रुणस॒वो᳚ ऽग्नि र॒ग्निर् व॑रुणस॒वो रा॑ज॒सूयꣳ॑ राज॒सूयं॑ ॅवरुणस॒वो᳚ ऽग्नि र॒ग्निर् व॑रुणस॒वो रा॑ज॒सूय᳚म् । \newline
13. व॒रु॒ण॒स॒वो रा॑ज॒सूयꣳ॑ राज॒सूयं॑ ॅवरुणस॒वो व॑रुणस॒वो रा॑ज॒सूय॑ मग्निस॒वो᳚ ऽग्निस॒वो रा॑ज॒सूयं॑ ॅवरुणस॒वो व॑रुणस॒वो रा॑ज॒सूय॑ मग्निस॒वः । \newline
14. व॒रु॒ण॒स॒व इति॑ वरुण - स॒वः । \newline
15. रा॒ज॒सूय॑ मग्निस॒वो᳚ ऽग्निस॒वो रा॑ज॒सूयꣳ॑ राज॒सूय॑ मग्निस॒व श्चित्य॒ श्चित्यो᳚ ऽग्निस॒वो रा॑ज॒सूयꣳ॑ राज॒सूय॑ मग्निस॒व श्चित्यः॑ । \newline
16. रा॒ज॒सूय॒मिति॑ राज - सूय᳚म् । \newline
17. अ॒ग्नि॒स॒व श्चित्य॒ श्चित्यो᳚ ऽग्निस॒वो᳚ ऽग्निस॒व श्चित्य॒ स्ताभ्या॒म् ताभ्या॒म् चित्यो᳚ ऽग्निस॒वो᳚ ऽग्निस॒व श्चित्य॒ स्ताभ्या᳚म् । \newline
18. अ॒ग्नि॒स॒व इत्य॑ग्नि - स॒वः । \newline
19. चित्य॒ स्ताभ्या॒म् ताभ्या॒म् चित्य॒ श्चित्य॒ स्ताभ्या॑ मे॒वैव ताभ्या॒म् चित्य॒ श्चित्य॒ स्ताभ्या॑ मे॒व । \newline
20. ताभ्या॑ मे॒वैव ताभ्या॒म् ताभ्या॑ मे॒व सू॑यते सूयत ए॒व ताभ्या॒म् ताभ्या॑ मे॒व सू॑यते । \newline
21. ए॒व सू॑यते सूयत ए॒वैव सू॑य॒ते ऽथो॒ अथो॑ सूयत ए॒वैव सू॑य॒ते ऽथो᳚ । \newline
22. सू॒य॒ते ऽथो॒ अथो॑ सूयते सूय॒ते ऽथो॑ उ॒भा वु॒भा वथो॑ सूयते सूय॒ते ऽथो॑ उ॒भौ । \newline
23. अथो॑ उ॒भा वु॒भा वथो॒ अथो॑ उ॒भा वे॒वै वोभा वथो॒ अथो॑ उ॒भा वे॒व । \newline
24. अथो॒ इत्यथो᳚ । \newline
25. उ॒भा वे॒वै वोभा वु॒भा वे॒व लो॒कौ लो॒का वे॒वोभा वु॒भा वे॒व लो॒कौ । \newline
26. ए॒व लो॒कौ लो॒का वे॒वैव लो॒का व॒भ्य॑भि लो॒का वे॒वैव लो॒का व॒भि । \newline
27. लो॒का व॒भ्य॑भि लो॒कौ लो॒का व॒भि ज॑यति जय त्य॒भि लो॒कौ लो॒का व॒भि ज॑यति । \newline
28. अ॒भि ज॑यति जय त्य॒भ्य॑भि ज॑यति॒ यो यो ज॑य त्य॒भ्य॑भि ज॑यति॒ यः । \newline
29. ज॒य॒ति॒ यो यो ज॑यति जयति॒ यश्च॑ च॒ यो ज॑यति जयति॒ यश्च॑ । \newline
30. यश्च॑ च॒ यो यश्च॑ राज॒सूये॑न राज॒सूये॑न च॒ यो यश्च॑ राज॒सूये॑न । \newline
31. च॒ रा॒ज॒सूये॑न राज॒सूये॑न च च राज॒सूये॑ नेजा॒न स्ये॑जा॒नस्य॑ राज॒सूये॑न च च राज॒सूये॑ नेजा॒नस्य॑ । \newline
32. रा॒ज॒सूये॑ नेजा॒न स्ये॑जा॒नस्य॑ राज॒सूये॑न राज॒सूये॑ नेजा॒नस्य॒ यो य ई॑जा॒नस्य॑ राज॒सूये॑न राज॒सूये॑
नेजा॒नस्य॒ यः । \newline
33. रा॒ज॒सूये॒नेति॑ राज - सूये॑न । \newline
34. ई॒जा॒नस्य॒ यो य ई॑जा॒न स्ये॑जा॒नस्य॒ यश्च॑ च॒ य ई॑जा॒न स्ये॑जा॒नस्य॒ यश्च॑ । \newline
35. यश्च॑ च॒ यो यश्चा᳚ ग्नि॒चितो᳚ ऽग्नि॒चित॑श्च॒ यो यश्चा᳚ ग्नि॒चितः॑ । \newline
36. चा॒ग्नि॒चितो᳚ ऽग्नि॒चित॑श्च चाग्नि॒चित॒ आप॒ आपो᳚ ऽग्नि॒चित॑श्च चाग्नि॒चित॒ आपः॑ । \newline
37. अ॒ग्नि॒चित॒ आप॒ आपो᳚ ऽग्नि॒चितो᳚ ऽग्नि॒चित॒ आपो॑ भवन्ति भव॒न् त्यापो᳚ ऽग्नि॒चितो᳚ ऽग्नि॒चित॒ आपो॑ भवन्ति । \newline
38. अ॒ग्नि॒चित॒ इत्य॑ग्नि - चितः॑ । \newline
39. आपो॑ भवन्ति भव॒न् त्याप॒ आपो॑ भव॒न्त्याप॒ आपो॑ भव॒न् त्याप॒ आपो॑ भव॒न् त्यापः॑ । \newline
40. भ॒व॒न् त्याप॒ आपो॑ भवन्ति भव॒न् त्यापो॒ वै वा आपो॑ भवन्ति भव॒न् त्यापो॒ वै । \newline
41. आपो॒ वै वा आप॒ आपो॒ वा अ॒ग्ने र॒ग्नेर् वा आप॒ आपो॒ वा अ॒ग्नेः । \newline
42. वा अ॒ग्ने र॒ग्नेर् वै वा अ॒ग्नेर् भ्रातृ॑व्या॒ भ्रातृ॑व्या अ॒ग्नेर् वै वा अ॒ग्नेर् भ्रातृ॑व्याः । \newline
43. अ॒ग्नेर् भ्रातृ॑व्या॒ भ्रातृ॑व्या अ॒ग्ने र॒ग्नेर् भ्रातृ॑व्या॒ यद् यद् भ्रातृ॑व्या अ॒ग्ने र॒ग्नेर् भ्रातृ॑व्या॒ यत् । \newline
44. भ्रातृ॑व्या॒ यद् यद् भ्रातृ॑व्या॒ भ्रातृ॑व्या॒ यद॒पो॑ ऽपो यद् भ्रातृ॑व्या॒ भ्रातृ॑व्या॒ यद॒पः । \newline
45. यद॒पो॑ ऽपो यद् यद॒पो᳚ ऽग्ने र॒ग्ने र॒पो यद् यद॒पो᳚ ऽग्नेः । \newline
46. अ॒पो᳚ ऽग्ने र॒ग्ने र॒पो᳚(1॒) ऽपो᳚ ऽग्ने र॒धस्ता॑ द॒धस्ता॑ द॒ग्ने र॒पो᳚(1॒) ऽपो᳚ ऽग्ने र॒धस्ता᳚त् । \newline
47. आ॒ग्ने र॒धस्ता॑ द॒धस्ता॑ द॒ग्ने र॒ग्ने र॒धस्ता॑ दुप॒दधा᳚ त्युप॒दधा᳚ त्य॒धस्ता॑ द॒ग्ने र॒ग्ने र॒धस्ता॑ दुप॒दधा॑ति । \newline
48. अ॒धस्ता॑ दुप॒दधा᳚ त्युप॒दधा᳚ त्य॒धस्ता॑ द॒धस्ता॑ दुप॒दधा॑ति॒ भ्रातृ॑व्याभिभूत्यै॒ भ्रातृ॑व्याभिभूत्या उप॒दधा᳚ त्य॒धस्ता॑ द॒धस्ता॑ दुप॒दधा॑ति॒ भ्रातृ॑व्याभिभूत्यै । \newline
49. उ॒प॒दधा॑ति॒ भ्रातृ॑व्याभिभूत्यै॒ भ्रातृ॑व्याभिभूत्या उप॒दधा᳚ त्युप॒दधा॑ति॒ भ्रातृ॑व्याभिभूत्यै॒ भव॑ति॒ भव॑ति॒ भ्रातृ॑व्याभिभूत्या उप॒दधा᳚ त्युप॒दधा॑ति॒ भ्रातृ॑व्याभिभूत्यै॒ भव॑ति । \newline
50. उ॒प॒दधा॒तीत्यु॑प - दधा॑ति । \newline
51. भ्रातृ॑व्याभिभूत्यै॒ भव॑ति॒ भव॑ति॒ भ्रातृ॑व्याभिभूत्यै॒ भ्रातृ॑व्याभिभूत्यै॒ भव॑त्या॒त्मना॒ ऽऽत्मना॒ भव॑ति॒ भ्रातृ॑व्याभिभूत्यै॒ भ्रातृ॑व्याभिभूत्यै॒ भव॑त्या॒त्मना᳚ । \newline
52. भ्रातृ॑व्याभिभूत्या॒ इति॒ भ्रातृ॑व्य - अ॒भि॒भू॒त्यै॒ । \newline
53. भव॑त्या॒त्मना॒ ऽऽत्मना॒ भव॑ति॒ भव॑त्या॒त्मना॒ परा॒ परा॒ ऽऽत्मना॒ भव॑ति॒ भव॑त्या॒त्मना॒ परा᳚ । \newline
54. आ॒त्मना॒ परा॒ परा॒ ऽऽत्मना॒ ऽऽत्मना॒ परा᳚ ऽस्यास्य॒ परा॒ ऽऽत्मना॒ ऽऽत्मना॒ परा᳚ ऽस्य । \newline
55. परा᳚ ऽस्यास्य॒ परा॒ परा᳚ ऽस्य॒ भ्रातृ॑व्यो॒ भ्रातृ॑व्यो ऽस्य॒ परा॒ परा᳚ ऽस्य॒ भ्रातृ॑व्यः । \newline
56. अ॒स्य॒ भ्रातृ॑व्यो॒ भ्रातृ॑व्यो ऽस्यास्य॒ भ्रातृ॑व्यो भवति भवति॒ भ्रातृ॑व्यो ऽस्यास्य॒ भ्रातृ॑व्यो भवति । \newline
57. भ्रातृ॑व्यो भवति भवति॒ भ्रातृ॑व्यो॒ भ्रातृ॑व्यो भव त्य॒मृत॑ म॒मृत॑म् भवति॒ भ्रातृ॑व्यो॒ भ्रातृ॑व्यो भव त्य॒मृत᳚म् । \newline
58. भ॒व॒त्य॒मृत॑ म॒मृत॑म् भवति भव त्य॒मृतं॒ ॅवै वा अ॒मृत॑म् भवति भव त्य॒मृतं॒ ॅवै । \newline
59. अ॒मृतं॒ ॅवै वा अ॒मृत॑ म॒मृतं॒ ॅवा आप॒ आपो॒ वा अ॒मृत॑ म॒मृतं॒ ॅवा आपः॑ । \newline
\pagebreak
\markright{ TS 5.6.2.2  \hfill https://www.vedavms.in \hfill}

\section{ TS 5.6.2.2 }

\textbf{TS 5.6.2.2 } \newline
\textbf{Samhita Paata} \newline

ॅवा आप॒स्तस्मा॑-द॒द्भिरव॑तान्त-म॒भि षि॑ञ्चन्ति॒ नाऽऽ*र्ति॒मार्च्छ॑ति॒ सर्व॒मायु॑रेति॒ यस्यै॒ता उ॑पधी॒यन्ते॒ य उ॑ चैना ए॒वं ॅवेदान्नं॒ ॅवा आपः॑ प॒शव॒ आपोऽन्नं॑ प॒शवो᳚ऽन्ना॒दः प॑शु॒मान् भ॑वति॒ यस्यै॒ता उ॑पधी॒यन्ते॒ य उ॑ चैना ए॒वं ॅवेद॒ द्वाद॑श भवन्ति॒ द्वाद॑श॒ मासाः᳚ संॅवथ्स॒रः सं॑ॅवथ्स॒रेणै॒वास्मा॒ - [  ] \newline

\textbf{Pada Paata} \newline

वै । आपः॑ । तस्मा᳚त् । अ॒द्भिरित्य॑त्-भिः । अव॑तान्त॒मित्यव॑-ता॒न्त॒म् । अ॒भीति॑ । सि॒ञ्च॒न्ति॒ । न । आर्ति᳚म् । एति॑ । ऋ॒च्छ॒ति॒ । सर्व᳚म् । आयुः॑ । ए॒ति॒ । यस्य॑ । ए॒ताः । उ॒प॒धी॒यन्त॒ इत्यु॑प - धी॒यन्ते᳚ । यः । उ॒ । च॒ । ए॒नाः॒ । ए॒वम् । वेद॑ । अन्न᳚म् । वै । आपः॑ । प॒शवः॑ । आपः॑ । अन्न᳚म् । प॒शवः॑ । अ॒न्ना॒द इत्य॑न्न - अ॒दः । प॒शु॒मानिति॑ पशु - मान् । भ॒व॒ति॒ । यस्य॑ । ए॒ताः । उ॒प॒धी॒यन्त॒ इत्यु॑प-धी॒यन्ते᳚ । यः । उ॒ । च॒ । ए॒नाः॒ । ए॒वम् । वेद॑ । द्वाद॑श । भ॒व॒न्ति॒ । द्वाद॑श । मासाः᳚ । सं॒ॅव॒थ्स॒र इति॑ सं - व॒थ्स॒रः । सं॒ॅव॒थ्स॒रेणेति॑ सं - व॒थ्स॒रेण॑ । ए॒व । अ॒स्मै॒ ।  \newline


\textbf{Krama Paata} \newline

वा आपः॑ । आप॒स्तस्मा᳚त् । तस्मा॑द॒द्भिः । अ॒द्भिरव॑तान्तम् । अ॒द्भिरित्य॑त् - भिः । अव॑तान्तम॒भि । अव॑तान्त॒मित्यव॑ - ता॒न्त॒म् । अ॒भि षि॑ञ्चन्ति । सि॒ञ्च॒न्ति॒ न । नार्ति᳚म् । आर्ति॒मा । आर्च्छ॑ति । ऋ॒च्छ॒ति॒ सर्व᳚म् । सर्व॒मायुः॑ । आयु॑रेति । ए॒ति॒ यस्य॑ । यस्यै॒ताः । ए॒ता उ॑पधी॒यन्ते᳚ । उ॒प॒धी॒यन्ते॒ यः । उ॒प॒धी॒यन्त॒ इत्यु॑प - धी॒यन्ते᳚ । य उ॑ । उ॒ च॒ । चै॒नाः॒ । ए॒ना॒ ए॒वम् । ए॒वम् ॅवेद॑ । वेदान्न᳚म् । अन्न॒म् ॅवै । वा आपः॑ । आपः॑ प॒शवः॑ । प॒शव॒ आपः॑ । आपोऽन्न᳚म् । अन्न॑म् प॒शवः॑ । प॒शवो᳚ऽन्ना॒दः । अ॒न्ना॒दः प॑शु॒मान् । अ॒न्ना॒द इत्य॑न्न - अ॒दः । प॒शु॒मान् भ॑वति । प॒शु॒मानिति॑ पशु - मान् । भ॒व॒ति॒ यस्य॑ । यस्यै॒ताः । ए॒ता उ॑पधी॒यन्ते᳚ । उ॒प॒धी॒यन्ते॒ यः । उ॒प॒धी॒यन्त॒ इत्यु॑प - धी॒यन्ते᳚ । य उ॑ । उ॒ च॒ । चै॒नाः॒ । ए॒ना॒ ए॒वम् । ए॒वम् ॅवेद॑ । वेद॒ द्वाद॑श । द्वाद॑श भवन्ति । भ॒व॒न्ति॒ द्वाद॑श । द्वाद॑श॒ मासाः᳚ । मासाः᳚ सम्ॅवथ्स॒रः । स॒म्ॅव॒थ्स॒रः स॑म्ॅवथ्स॒रेण॑ । स॒म्ॅव॒थ्स॒र इति॑ सम् - व॒थ्स॒रः । स॒म्ॅव॒थ्स॒रेणै॒व । स॒म्ॅव॒थ्स॒रेणेति॑ सम् - व॒थ्स॒रेण॑ । ए॒वास्मै᳚ । अ॒स्मा॒ अन्न᳚म् \newline

\textbf{Jatai Paata} \newline

1. वा आप॒ आपो॒ वै वा आपः॑ । \newline
2. आप॒ स्तस्मा॒त् तस्मा॒ दाप॒ आप॒ स्तस्मा᳚त् । \newline
3. तस्मा॑ द॒द्भि र॒द्भि स्तस्मा॒त् तस्मा॑ द॒द्भिः । \newline
4. अ॒द्भि रव॑तान्त॒ मव॑तान्त म॒द्भि र॒द्भि रव॑तान्तम् । \newline
5. अ॒द्भिरित्य॑त् - भिः । \newline
6. अव॑तान्त म॒भ्य॑भ्य व॑तान्त॒ मव॑तान्त म॒भि । \newline
7. अव॑तान्त॒मित्यव॑ - ता॒न्त॒म् । \newline
8. अ॒भि षि॑ञ्चन्ति सिञ्चन् त्य॒भ्य॑भि षि॑ञ्चन्ति । \newline
9. सि॒ञ्च॒न्ति॒ न न सि॑ञ्चन्ति सिञ्चन्ति॒ न । \newline
10. नार्ति॒ मार्ति॒म् न नार्ति᳚म् । \newline
11. आर्ति॒ मा ऽऽर्ति॒ मार्ति॒ मा । \newline
12. आर्च्छ॑ त्यृच्छ॒ त्यार्च्छ॑ति । \newline
13. ऋ॒च्छ॒ति॒ सर्वꣳ॒॒ सर्व॑ मृच्छ त्यृच्छति॒ सर्व᳚म् । \newline
14. सर्व॒ मायु॒ रायुः॒ सर्वꣳ॒॒ सर्व॒ मायुः॑ । \newline
15. आयु॑ रेत्ये॒ त्यायु॒ रायु॑ रेति । \newline
16. ए॒ति॒ यस्य॒ यस्यै᳚ त्येति॒ यस्य॑ । \newline
17. यस्यै॒ता ए॒ता यस्य॒ यस्यै॒ताः । \newline
18. ए॒ता उ॑पधी॒यन्त॑ उपधी॒यन्त॑ ए॒ता ए॒ता उ॑पधी॒यन्ते᳚ । \newline
19. उ॒प॒धी॒यन्ते॒ यो य उ॑पधी॒यन्त॑ उपधी॒यन्ते॒ यः । \newline
20. उ॒प॒धी॒यन्त॒ इत्यु॑प - धी॒यन्ते᳚ । \newline
21. य उ॑ वु॒ यो य उ॑ । \newline
22. उ॒ च॒ च॒ वु॒ च॒ । \newline
23. चै॒ना॒ ए॒ना॒श्च॒ चै॒नाः॒ । \newline
24. ए॒ना॒ ए॒व मे॒व मे॑ना एना ए॒वम् । \newline
25. ए॒वं ॅवेद॒ वेदै॒व मे॒वं ॅवेद॑ । \newline
26. वेदान्न॒ मन्नं॒ ॅवेद॒ वेदान्न᳚म् । \newline
27. अन्नं॒ ॅवै वा अन्न॒ मन्नं॒ ॅवै । \newline
28. वा आप॒ आपो॒ वै वा आपः॑ । \newline
29. आपः॑ प॒शवः॑ प॒शव॒ आप॒ आपः॑ प॒शवः॑ । \newline
30. प॒शव॒ आप॒ आपः॑ प॒शवः॑ प॒शव॒ आपः॑ । \newline
31. आपो ऽन्न॒ मन्न॒ माप॒ आपो ऽन्न᳚म् । \newline
32. अन्न॑म् प॒शवः॑ प॒शवो ऽन्न॒ मन्न॑म् प॒शवः॑ । \newline
33. प॒शवो᳚ ऽन्ना॒दो᳚ ऽन्ना॒दः प॒शवः॑ प॒शवो᳚ ऽन्ना॒दः । \newline
34. अ॒न्ना॒दः प॑शु॒मान् प॑शु॒मा न॑न्ना॒दो᳚ ऽन्ना॒दः प॑शु॒मान् । \newline
35. अ॒न्ना॒द इत्य॑न्न - अ॒दः । \newline
36. प॒शु॒मान् भ॑वति भवति पशु॒मान् प॑शु॒मान् भ॑वति । \newline
37. प॒शु॒मानिति॑ पशु - मान् । \newline
38. भ॒व॒ति॒ यस्य॒ यस्य॑ भवति भवति॒ यस्य॑ । \newline
39. यस्यै॒ता ए॒ता यस्य॒ यस्यै॒ताः । \newline
40. ए॒ता उ॑पधी॒यन्त॑ उपधी॒यन्त॑ ए॒ता ए॒ता उ॑पधी॒यन्ते᳚ । \newline
41. उ॒प॒धी॒यन्ते॒ यो य उ॑पधी॒यन्त॑ उपधी॒यन्ते॒ यः । \newline
42. उ॒प॒धी॒यन्त॒ इत्यु॑प - धी॒यन्ते᳚ । \newline
43. य उ॑ वु॒ यो य उ॑ । \newline
44. उ॒ च॒ च॒ वु॒ च॒ । \newline
45. चै॒ना॒ ए॒ना॒श्च॒ चै॒नाः॒ । \newline
46. ए॒ना॒ ए॒व मे॒व मे॑ना एना ए॒वम् । \newline
47. ए॒वं ॅवेद॒ वेदै॒व मे॒वं ॅवेद॑ । \newline
48. वेद॒ द्वाद॑श॒ द्वाद॑श॒ वेद॒ वेद॒ द्वाद॑श । \newline
49. द्वाद॑श भवन्ति भवन्ति॒ द्वाद॑श॒ द्वाद॑श भवन्ति । \newline
50. भ॒व॒न्ति॒ द्वाद॑श॒ द्वाद॑श भवन्ति भवन्ति॒ द्वाद॑श । \newline
51. द्वाद॑श॒ मासा॒ मासा॒ द्वाद॑श॒ द्वाद॑श॒ मासाः᳚ । \newline
52. मासाः᳚ संॅवथ्स॒रः सं॑ॅवथ्स॒रो मासा॒ मासाः᳚ संॅवथ्स॒रः । \newline
53. सं॒ॅव॒थ्स॒रः सं॑ॅवथ्स॒रेण॑ संॅवथ्स॒रेण॑ संॅवथ्स॒रः सं॑ॅवथ्स॒रः सं॑ॅवथ्स॒रेण॑ । \newline
54. सं॒ॅव॒थ्स॒र इति॑ सं - व॒थ्स॒रः । \newline
55. सं॒ॅव॒थ्स॒रे णै॒वैव सं॑ॅवथ्स॒रेण॑ संॅवथ्स॒रेणै॒व । \newline
56. सं॒ॅव॒थ्स॒रेणेति॑ सं - व॒थ्स॒रेण॑ । \newline
57. ए॒वास्मा॑ अस्मा ए॒वै वास्मै᳚ । \newline
58. अ॒स्मा॒ अन्न॒ मन्न॑ मस्मा अस्मा॒ अन्न᳚म् । \newline

\textbf{Ghana Paata } \newline

1. वा आप॒ आपो॒ वै वा आप॒ स्तस्मा॒त् तस्मा॒ दापो॒ वै वा आप॒ स्तस्मा᳚त् । \newline
2. आप॒ स्तस्मा॒त् तस्मा॒ दाप॒ आप॒ स्तस्मा॑ द॒द्भि र॒द्भि स्तस्मा॒ दाप॒ आप॒ स्तस्मा॑ द॒द्भिः । \newline
3. तस्मा॑ द॒द्भि र॒द्भि स्तस्मा॒त् तस्मा॑ द॒द्भि रव॑तान्त॒ मव॑तान्त म॒द्भि स्तस्मा॒त् तस्मा॑ द॒द्भि रव॑तान्तम् । \newline
4. अ॒द्भि रव॑तान्त॒ मव॑तान्त म॒द्भि र॒द्भि रव॑तान्त म॒भ्य॑भ्य व॑तान्त म॒द्भि र॒द्भि रव॑तान्त म॒भि । \newline
5. अ॒द्भिरित्य॑त् - भिः । \newline
6. अव॑तान्त म॒भ्य॑भ्य व॑तान्त॒ मव॑तान्त म॒भि षि॑ञ्चन्ति सिञ्चन् त्य॒भ्यव॑तान्त॒ मव॑तान्त म॒भि षि॑ञ्चन्ति । \newline
7. अव॑तान्त॒मित्यव॑ - ता॒न्त॒म् । \newline
8. अ॒भि षि॑ञ्चन्ति सिञ्चन् त्य॒भ्य॑भि षि॑ञ्चन्ति॒ न न सि॑ञ्चन् त्य॒भ्य॑भि षि॑ञ्चन्ति॒ न । \newline
9. सि॒ञ्च॒न्ति॒ न न सि॑ञ्चन्ति सिञ्चन्ति॒ नार्ति॒ मार्ति॒म् न सि॑ञ्चन्ति सिञ्चन्ति॒ नार्ति᳚म् । \newline
10. नार्ति॒ मार्ति॒म् न नार्ति॒ मा ऽऽर्ति॒म् न नार्ति॒ मा । \newline
11. आर्ति॒ मा ऽऽर्ति॒ मार्ति॒ मार्च्छ॑ त्यृच्छ॒ त्याऽऽर्ति॒ मार्ति॒ मार्च्छ॑ति । \newline
12. आर्च्छ॑ त्यृच्छ॒ त्यार्च्छ॑ति॒ सर्वꣳ॒॒ सर्व॑ मृच्छ॒ त्यार्च्छ॑ति॒ सर्व᳚म् । \newline
13. ऋ॒च्छ॒ति॒ सर्वꣳ॒॒ सर्व॑ मृच्छ त्यृच्छति॒ सर्व॒ मायु॒ रायुः॒ सर्व॑ मृच्छ त्यृच्छति॒ सर्व॒ मायुः॑ । \newline
14. सर्व॒ मायु॒ रायुः॒ सर्वꣳ॒॒ सर्व॒ मायु॑ रेत्ये॒ त्यायुः॒ सर्वꣳ॒॒ सर्व॒ मायु॑रेति । \newline
15. आयु॑ रेत्ये॒ त्यायु॒ रायु॑ रेति॒ यस्य॒ यस्यै॒ त्यायु॒ रायु॑रेति॒ यस्य॑ । \newline
16. ए॒ति॒ यस्य॒ यस्यै᳚त्येति॒ यस्यै॒ता ए॒ता यस्यै᳚त्येति॒ यस्यै॒ताः । \newline
17. यस्यै॒ता ए॒ता यस्य॒ यस्यै॒ता उ॑पधी॒यन्त॑ उपधी॒यन्त॑ ए॒ता यस्य॒ यस्यै॒ता उ॑पधी॒यन्ते᳚ । \newline
18. ए॒ता उ॑पधी॒यन्त॑ उपधी॒यन्त॑ ए॒ता ए॒ता उ॑पधी॒यन्ते॒ यो य उ॑पधी॒यन्त॑ ए॒ता ए॒ता उ॑पधी॒यन्ते॒ यः । \newline
19. उ॒प॒धी॒यन्ते॒ यो य उ॑पधी॒यन्त॑ उपधी॒यन्ते॒ य उ॑ वु॒ य उ॑पधी॒यन्त॑ उपधी॒यन्ते॒ य उ॑ । \newline
20. उ॒प॒धी॒यन्त॒ इत्यु॑प - धी॒यन्ते᳚ । \newline
21. य उ॑ वु॒ यो य उ॑ च चो॒ यो य उ॑ च । \newline
22. उ॒ च॒ च॒ वु॒ चै॒ना॒ ए॒ना॒श्च॒ वु॒ चै॒नाः॒ । \newline
23. चै॒ना॒ ए॒ना॒श्च॒ चै॒ना॒ ए॒व मे॒व मे॑नाश्च चैना ए॒वम् । \newline
24. ए॒ना॒ ए॒व मे॒व मे॑ना एना ए॒वं ॅवेद॒ वेदै॒व मे॑ना एना ए॒वं ॅवेद॑ । \newline
25. ए॒वं ॅवेद॒ वेदै॒व मे॒वं ॅवेदान्न॒ मन्नं॒ ॅवेदै॒व मे॒वं ॅवेदान्न᳚म् । \newline
26. वेदान्न॒ मन्नं॒ ॅवेद॒ वेदान्नं॒ ॅवै वा अन्नं॒ ॅवेद॒ वेदान्नं॒ ॅवै । \newline
27. अन्नं॒ ॅवै वा अन्न॒ मन्नं॒ ॅवा आप॒ आपो॒ वा अन्न॒ मन्नं॒ ॅवा आपः॑ । \newline
28. वा आप॒ आपो॒ वै वा आपः॑ प॒शवः॑ प॒शव॒ आपो॒ वै वा आपः॑ प॒शवः॑ । \newline
29. आपः॑ प॒शवः॑ प॒शव॒ आप॒ आपः॑ प॒शव॒ आप॒ आपः॑ प॒शव॒ आप॒ आपः॑ प॒शव॒ आपः॑ । \newline
30. प॒शव॒ आप॒ आपः॑ प॒शवः॑ प॒शव॒ आपो ऽन्न॒ मन्न॒ मापः॑ प॒शवः॑ प॒शव॒ आपो ऽन्न᳚म् । \newline
31. आपो ऽन्न॒ मन्न॒ माप॒ आपो ऽन्न॑म् प॒शवः॑ प॒शवो ऽन्न॒ माप॒ आपो ऽन्न॑म् प॒शवः॑ । \newline
32. अन्न॑म् प॒शवः॑ प॒शवो ऽन्न॒ मन्न॑म् प॒शवो᳚ ऽन्ना॒दो᳚ ऽन्ना॒दः प॒शवो ऽन्न॒ मन्न॑म् प॒शवो᳚ ऽन्ना॒दः । \newline
33. प॒शवो᳚ ऽन्ना॒दो᳚ ऽन्ना॒दः प॒शवः॑ प॒शवो᳚ ऽन्ना॒दः प॑शु॒मान् प॑शु॒मा न॑न्ना॒दः प॒शवः॑ प॒शवो᳚ ऽन्ना॒दः प॑शु॒मान् । \newline
34. अ॒न्ना॒दः प॑शु॒मान् प॑शु॒मा न॑न्ना॒दो᳚ ऽन्ना॒दः प॑शु॒मान् भ॑वति भवति पशु॒मा न॑न्ना॒दो᳚ ऽन्ना॒दः प॑शु॒मान् भ॑वति । \newline
35. अ॒न्ना॒द इत्य॑न्न - अ॒दः । \newline
36. प॒शु॒मान् भ॑वति भवति पशु॒मान् प॑शु॒मान् भ॑वति॒ यस्य॒ यस्य॑ भवति पशु॒मान् प॑शु॒मान् भ॑वति॒ यस्य॑ । \newline
37. प॒शु॒मानिति॑ पशु - मान् । \newline
38. भ॒व॒ति॒ यस्य॒ यस्य॑ भवति भवति॒ यस्यै॒ता ए॒ता यस्य॑ भवति भवति॒ यस्यै॒ताः । \newline
39. यस्यै॒ता ए॒ता यस्य॒ यस्यै॒ता उ॑पधी॒यन्त॑ उपधी॒यन्त॑ ए॒ता यस्य॒ यस्यै॒ता उ॑पधी॒यन्ते᳚ । \newline
40. ए॒ता उ॑पधी॒यन्त॑ उपधी॒यन्त॑ ए॒ता ए॒ता उ॑पधी॒यन्ते॒ यो य उ॑पधी॒यन्त॑ ए॒ता ए॒ता उ॑पधी॒यन्ते॒ यः । \newline
41. उ॒प॒धी॒यन्ते॒ यो य उ॑पधी॒यन्त॑ उपधी॒यन्ते॒ य उ॑ वु॒ य उ॑पधी॒यन्त॑ उपधी॒यन्ते॒ य उ॑ । \newline
42. उ॒प॒धी॒यन्त॒ इत्यु॑प - धी॒यन्ते᳚ । \newline
43. य उ॑ वु॒ यो य उ॑ च चो॒ यो य उ॑ च । \newline
44. उ॒ च॒ च॒ वु॒ चै॒ना॒ ए॒ना॒श्च॒ वु॒ चै॒नाः॒ । \newline
45. चै॒ना॒ ए॒ना॒श्च॒ चै॒ना॒ ए॒व मे॒व मे॑नाश्च चैना ए॒वम् । \newline
46. ए॒ना॒ ए॒व मे॒व मे॑ना एना ए॒वं ॅवेद॒ वेदै॒व मे॑ना एना ए॒वं ॅवेद॑ । \newline
47. ए॒वं ॅवेद॒ वेदै॒व मे॒वं ॅवेद॒ द्वाद॑श॒ द्वाद॑श॒ वेदै॒व मे॒वं ॅवेद॒ द्वाद॑श । \newline
48. वेद॒ द्वाद॑श॒ द्वाद॑श॒ वेद॒ वेद॒ द्वाद॑श भवन्ति भवन्ति॒ द्वाद॑श॒ वेद॒ वेद॒ द्वाद॑श भवन्ति । \newline
49. द्वाद॑श भवन्ति भवन्ति॒ द्वाद॑श॒ द्वाद॑श भवन्ति॒ द्वाद॑श॒ द्वाद॑श भवन्ति॒ द्वाद॑श॒ द्वाद॑श भवन्ति॒ द्वाद॑श । \newline
50. भ॒व॒न्ति॒ द्वाद॑श॒ द्वाद॑श भवन्ति भवन्ति॒ द्वाद॑श॒ मासा॒ मासा॒ द्वाद॑श भवन्ति भवन्ति॒ द्वाद॑श॒ मासाः᳚ । \newline
51. द्वाद॑श॒ मासा॒ मासा॒ द्वाद॑श॒ द्वाद॑श॒ मासाः᳚ संॅवथ्स॒रः सं॑ॅवथ्स॒रो मासा॒ द्वाद॑श॒ द्वाद॑श॒ मासाः᳚ संॅवथ्स॒रः । \newline
52. मासाः᳚ संॅवथ्स॒रः सं॑ॅवथ्स॒रो मासा॒ मासाः᳚ संॅवथ्स॒रः सं॑ॅवथ्स॒रेण॑ संॅवथ्स॒रेण॑ संॅवथ्स॒रो मासा॒ मासाः᳚ संॅवथ्स॒रः सं॑ॅवथ्स॒रेण॑ । \newline
53. सं॒ॅव॒थ्स॒रः सं॑ॅवथ्स॒रेण॑ संॅवथ्स॒रेण॑ संॅवथ्स॒रः सं॑ॅवथ्स॒रः सं॑ॅवथ्स॒रे णै॒वैव सं॑ॅवथ्स॒रेण॑ संॅवथ्स॒रः सं॑ॅवथ्स॒रः सं॑ॅवथ्स॒रेणै॒व । \newline
54. सं॒ॅव॒थ्स॒र इति॑ सं - व॒थ्स॒रः । \newline
55. सं॒ॅव॒थ्स॒रे णै॒वैव सं॑ॅवथ्स॒रेण॑ संॅवथ्स॒रे णै॒वास्मा॑ अस्मा ए॒व सं॑ॅवथ्स॒रेण॑ संॅवथ्स॒रेणै॒ वास्मै᳚ । \newline
56. सं॒ॅव॒थ्स॒रेणेति॑ सं - व॒थ्स॒रेण॑ । \newline
57. ए॒वास्मा॑ अस्मा ए॒वै वास्मा॒ अन्न॒ मन्न॑ मस्मा ए॒वै वास्मा॒ अन्न᳚म् । \newline
58. अ॒स्मा॒ अन्न॒ मन्न॑ मस्मा अस्मा॒ अन्न॒ मवा वान्न॑ मस्मा अस्मा॒ अन्न॒ मव॑ । \newline
\pagebreak
\markright{ TS 5.6.2.3  \hfill https://www.vedavms.in \hfill}

\section{ TS 5.6.2.3 }

\textbf{TS 5.6.2.3 } \newline
\textbf{Samhita Paata} \newline

अन्न॒मव॑ रुन्धे॒ पात्रा॑णि भवन्ति॒ पात्रे॒ वा अन्न॑मद्यते॒ सयो᳚न्ये॒वान्न॒मव॑ रुन्ध॒ आ द्वा॑द॒शात् पुरु॑षा॒दन्न॑-म॒त्त्यथो॒ पात्रा॒न्न छि॑द्यते॒ यस्यै॒ता उ॑पधी॒यन्ते य उ॑ चैना ए॒वं ॅवेद॑ कुं॒भाश्च॑ कुं॒भीश्च॑ मिथु॒नानि॑ भवन्ति मिथु॒नस्य॒ प्रजा᳚त्यै॒ प्र प्र॒जया॑ प॒शुभि॑-र्मिथु॒नै-र्जा॑यते॒ यस्यै॒ता उ॑पधी॒यन्ते॒ य उ॑ - [  ] \newline

\textbf{Pada Paata} \newline

अन्न᳚म् । अवेति॑ । रु॒न्धे॒ । पात्रा॑णि । भ॒व॒न्ति॒ । पात्रे᳚ । वै । अन्न᳚म् । अ॒द्य॒ते॒ । सयो॒नीति॒ स - यो॒नि॒ । ए॒व । अन्न᳚म् । अवेति॑ । रु॒न्धे॒ । एति॑ । द्वा॒द॒शात् । पुरु॑षात् । अन्न᳚म् । अ॒त्ति॒ । अथो॒ इति॑ । पात्रा᳚त् । न । छि॒द्य॒ते॒ । यस्य॑ । ए॒ताः । उ॒प॒धी॒यन्त॒ इत्यु॑प-धी॒यन्ते᳚ । यः । उ॒ । च॒ । ए॒नाः॒ । ए॒वम् । वेद॑ । कु॒भांः । च॒ । कु॒भींः । च॒ । मि॒थु॒नानि॑ । भ॒व॒न्ति॒ । मि॒थु॒नस्य॑ । प्रजा᳚त्या॒ इति॒ प्र - जा॒त्यै॒ । प्रेति॑ । प्र॒जयेति॑ प्र - जया᳚ । प॒शुभि॒रिति॑ प॒शु - भिः॒ । मि॒थु॒नैः । जा॒य॒ते॒ । यस्य॑ । ए॒ताः । उ॒प॒धी॒यन्त॒ इत्यु॑प - धी॒यन्ते᳚ । यः । उ॒ ।  \newline


\textbf{Krama Paata} \newline

अन्न॒मव॑ । अव॑ रुन्धे । रु॒न्धे॒ पात्रा॑णि । पात्रा॑णि भवन्ति । भ॒व॒न्ति॒ पात्रे᳚ । पात्रे॒ वै । वा अन्न᳚म् । अन्न॑मद्यते । अ॒द्य॒ते॒ सयो॑नि । सयो᳚न्ये॒व । सयो॒नीति॒ स - यो॒नि॒ । ए॒वान्न᳚म् । अन्न॒मव॑ । 
अव॑ रुन्धे । रु॒न्ध॒ आ । आ द्वा॑द॒शात् । द्वा॒द॒शात् पुरु॑षात् । पुरु॑षा॒दन्न᳚म् । अन्न॑मत्ति । अ॒त्त्यथो᳚ । अथो॒ पात्रा᳚त् । अथो॒ इत्यथो᳚ । पात्रा॒न् न । न छि॑द्यते । छि॒द्य॒ते॒ यस्य॑ । यस्यै॒ताः । ए॒ता उ॑पधी॒यन्ते᳚ । उ॒प॒धी॒यन्ते॒ यः । उ॒प॒धी॒यन्त॒ इत्यु॑प - धी॒यन्ते᳚ । य उ॑ । उ॒ च॒ । चै॒नाः॒ । ए॒ना॒ ए॒वम् । ए॒वम् ॅवेद॑ । वेद॑ कु॒म्भाः । कु॒म्भाश्च॑ । च॒ कु॒म्भीः । कु॒म्भीश्च॑ । च॒ मि॒थु॒नानि॑ । मि॒थु॒नानि॑ भवन्ति । भ॒व॒न्ति॒ मि॒थु॒नस्य॑ । मि॒थु॒नस्य॒ प्रजा᳚त्यै । प्रजा᳚त्यै॒ प्र । प्रजा᳚त्या॒ इति॒ प्र - जा॒त्यै॒ । प्र प्र॒जया᳚ । प्र॒जया॑ प॒शुभिः॑ । प्र॒जयेति॑ प्र - जया᳚ । प॒शुभि॑र् मिथु॒नैः । प॒शुभि॒रिति॑ प॒शु - भिः॒ । मि॒थु॒नैर् जा॑यते । जा॒य॒ते॒ यस्य॑ । यस्यै॒ताः । ए॒ता उ॑पधी॒यन्ते᳚ । उ॒प॒धी॒यन्ते॒ यः । उ॒प॒धी॒यन्त॒ इत्यु॑प - धी॒यन्ते᳚ । य उ॑ । उ॒ च॒ \newline

\textbf{Jatai Paata} \newline

1. अन्न॒ मवा वान्न॒ मन्न॒ मव॑ । \newline
2. अव॑ रुन्धे रु॒न्धे ऽवाव॑ रुन्धे । \newline
3. रु॒न्धे॒ पात्रा॑णि॒ पात्रा॑णि रुन्धे रुन्धे॒ पात्रा॑णि । \newline
4. पात्रा॑णि भवन्ति भवन्ति॒ पात्रा॑णि॒ पात्रा॑णि भवन्ति । \newline
5. भ॒व॒न्ति॒ पात्रे॒ पात्रे॑ भवन्ति भवन्ति॒ पात्रे᳚ । \newline
6. पात्रे॒ वै वै पात्रे॒ पात्रे॒ वै । \newline
7. वा अन्न॒ मन्नं॒ ॅवै वा अन्न᳚म् । \newline
8. अन्न॑ मद्यते ऽद्य॒ते ऽन्न॒ मन्न॑ मद्यते । \newline
9. अ॒द्य॒ते॒ सयो॑नि॒ सयो᳚ न्यद्यते ऽद्यते॒ सयो॑नि । \newline
10. सयो᳚ न्ये॒वैव सयो॑नि॒ सयो᳚ न्ये॒व । \newline
11. सयो॒नीति॒ स - यो॒नि॒ । \newline
12. ए॒वान्न॒ मन्न॑ मे॒वै वान्न᳚म् । \newline
13. अन्न॒ मवा वान्न॒ मन्न॒ मव॑ । \newline
14. अव॑ रुन्धे रु॒न्धे ऽवाव॑ रुन्धे । \newline
15. रु॒न्ध॒ आ रु॑न्धे रुन्ध॒ आ । \newline
16. आ द्वा॑द॒शाद् द्वा॑द॒शादा द्वा॑द॒शात् । \newline
17. द्वा॒द॒शात् पुरु॑षा॒त् पुरु॑षाद् द्वाद॒शाद् द्वा॑द॒शात् पुरु॑षात् । \newline
18. पुरु॑षा॒ दन्न॒ मन्न॒म् पुरु॑षा॒त् पुरु॑षा॒ दन्न᳚म् । \newline
19. अन्न॑ मत्त्य॒ त्त्यन्न॒ मन्न॑ मत्ति । \newline
20. अ॒त्त्यथो॒ अथो॑ अत्त्य॒ त्त्यथो᳚ । \newline
21. अथो॒ पात्रा॒त् पात्रा॒ दथो॒ अथो॒ पात्रा᳚त् । \newline
22. अथो॒ इत्यथो᳚ । \newline
23. पात्रा॒न् न न पात्रा॒त् पात्रा॒न् न । \newline
24. न छि॑द्यते छिद्यते॒ न न छि॑द्यते । \newline
25. छि॒द्य॒ते॒ यस्य॒ यस्य॑ छिद्यते छिद्यते॒ यस्य॑ । \newline
26. यस्यै॒ता ए॒ता यस्य॒ यस्यै॒ताः । \newline
27. ए॒ता उ॑पधी॒यन्त॑ उपधी॒यन्त॑ ए॒ता ए॒ता उ॑पधी॒यन्ते᳚ । \newline
28. उ॒प॒धी॒यन्ते॒ यो य उ॑पधी॒यन्त॑ उपधी॒यन्ते॒ यः । \newline
29. उ॒प॒धी॒यन्त॒ इत्यु॑प - धी॒यन्ते᳚ । \newline
30. य उ॑ वु॒ यो य उ॑ । \newline
31. उ॒ च॒ च॒ वु॒ च॒ । \newline
32. चै॒ना॒ ए॒ना॒श्च॒ चै॒नाः॒ । \newline
33. ए॒ना॒ ए॒व मे॒व मे॑ना एना ए॒वम् । \newline
34. ए॒वं ॅवेद॒ वेदै॒व मे॒वं ॅवेद॑ । \newline
35. वेद॑ कुं॒भाः कुं॒भा वेद॒ वेद॑ कुं॒भाः । \newline
36. कुं॒भाश्च॑ च कुं॒भाः कुं॒भाश्च॑ । \newline
37. च॒ कुं॒भीः कुं॒भीश्च॑ च कुं॒भीः । \newline
38. कुं॒भीश्च॑ च कुं॒भीः कुं॒भीश्च॑ । \newline
39. च॒ मि॒थु॒नानि॑ मिथु॒नानि॑ च च मिथु॒नानि॑ । \newline
40. मि॒थु॒नानि॑ भवन्ति भवन्ति मिथु॒नानि॑ मिथु॒नानि॑ भवन्ति । \newline
41. भ॒व॒न्ति॒ मि॒थु॒नस्य॑ मिथु॒नस्य॑ भवन्ति भवन्ति मिथु॒नस्य॑ । \newline
42. मि॒थु॒नस्य॒ प्रजा᳚त्यै॒ प्रजा᳚त्यै मिथु॒नस्य॑ मिथु॒नस्य॒ प्रजा᳚त्यै । \newline
43. प्रजा᳚त्यै॒ प्र प्र प्रजा᳚त्यै॒ प्रजा᳚त्यै॒ प्र । \newline
44. प्रजा᳚त्या॒ इति॒ प्र - जा॒त्यै॒ । \newline
45. प्र प्र॒जया᳚ प्र॒जया॒ प्र प्र प्र॒जया᳚ । \newline
46. प्र॒जया॑ प॒शुभिः॑ प॒शुभिः॑ प्र॒जया᳚ प्र॒जया॑ प॒शुभिः॑ । \newline
47. प्र॒जयेति॑ प्र - जया᳚ । \newline
48. प॒शुभि॑र् मिथु॒नैर् मि॑थु॒नैः प॒शुभिः॑ प॒शुभि॑र् मिथु॒नैः । \newline
49. प॒शुभि॒रिति॑ प॒शु - भिः॒ । \newline
50. मि॒थु॒नैर् जा॑यते जायते मिथु॒नैर् मि॑थु॒नैर् जा॑यते । \newline
51. जा॒य॒ते॒ यस्य॒ यस्य॑ जायते जायते॒ यस्य॑ । \newline
52. यस्यै॒ता ए॒ता यस्य॒ यस्यै॒ताः । \newline
53. ए॒ता उ॑पधी॒यन्त॑ उपधी॒यन्त॑ ए॒ता ए॒ता उ॑पधी॒यन्ते᳚ । \newline
54. उ॒प॒धी॒यन्ते॒ यो य उ॑पधी॒यन्त॑ उपधी॒यन्ते॒ यः । \newline
55. उ॒प॒धी॒यन्त॒ इत्यु॑प - धी॒यन्ते᳚ । \newline
56. य उ॑ वु॒ यो य उ॑ । \newline
57. उ॒ च॒ च॒ वु॒ च॒ । \newline

\textbf{Ghana Paata } \newline

1. अन्न॒ मवावान्न॒ मन्न॒ मव॑ रुन्धे रु॒न्धे ऽवान्न॒ मन्न॒ मव॑ रुन्धे । \newline
2. अव॑ रुन्धे रु॒न्धे ऽवाव॑ रुन्धे॒ पात्रा॑णि॒ पात्रा॑णि रु॒न्धे ऽवाव॑ रुन्धे॒ पात्रा॑णि । \newline
3. रु॒न्धे॒ पात्रा॑णि॒ पात्रा॑णि रुन्धे रुन्धे॒ पात्रा॑णि भवन्ति भवन्ति॒ पात्रा॑णि रुन्धे रुन्धे॒ पात्रा॑णि भवन्ति । \newline
4. पात्रा॑णि भवन्ति भवन्ति॒ पात्रा॑णि॒ पात्रा॑णि भवन्ति॒ पात्रे॒ पात्रे॑ भवन्ति॒ पात्रा॑णि॒ पात्रा॑णि भवन्ति॒ पात्रे᳚ । \newline
5. भ॒व॒न्ति॒ पात्रे॒ पात्रे॑ भवन्ति भवन्ति॒ पात्रे॒ वै वै पात्रे॑ भवन्ति भवन्ति॒ पात्रे॒ वै । \newline
6. पात्रे॒ वै वै पात्रे॒ पात्रे॒ वा अन्न॒ मन्नं॒ ॅवै पात्रे॒ पात्रे॒ वा अन्न᳚म् । \newline
7. वा अन्न॒ मन्नं॒ ॅवै वा अन्न॑ मद्यते ऽद्य॒ते ऽन्नं॒ ॅवै वा अन्न॑ मद्यते । \newline
8. अन्न॑ मद्यते ऽद्य॒ते ऽन्न॒ मन्न॑ मद्यते॒ सयो॑नि॒ सयो᳚ न्यद्य॒ते ऽन्न॒ मन्न॑ मद्यते॒ सयो॑नि । \newline
9. अ॒द्य॒ते॒ सयो॑नि॒ सयो᳚ न्यद्यते ऽद्यते॒ सयो᳚ न्ये॒वैव सयो᳚ न्यद्यते ऽद्यते॒ सयो᳚न्ये॒व । \newline
10. सयो᳚ न्ये॒वैव सयो॑नि॒ सयो᳚ न्ये॒वान्न॒ मन्न॑ मे॒व सयो॑नि॒ सयो᳚ न्ये॒वान्न᳚म् । \newline
11. सयो॒नीति॒ स - यो॒नि॒ । \newline
12. ए॒वान्न॒ मन्न॑ मे॒वै वान्न॒ मवा वान्न॑ मे॒वै वान्न॒ मव॑ । \newline
13. अन्न॒ मवा वान्न॒ मन्न॒ मव॑ रुन्धे रु॒न्धे ऽवान्न॒ मन्न॒ मव॑ रुन्धे । \newline
14. अव॑ रुन्धे रु॒न्धे ऽवाव॑ रुन्ध॒ आ रु॒न्धे ऽवाव॑ रुन्ध॒ आ । \newline
15. रु॒न्ध॒ आ रु॑न्धे रुन्ध॒ आ द्वा॑द॒शाद् द्वा॑द॒शादा रु॑न्धे रुन्ध॒ आ द्वा॑द॒शात् । \newline
16. आ द्वा॑द॒शाद् द्वा॑द॒शादा द्वा॑द॒शात् पुरु॑षा॒त् पुरु॑षाद् द्वाद॒शादा द्वा॑द॒शात् पुरु॑षात् । \newline
17. द्वा॒द॒शात् पुरु॑षा॒त् पुरु॑षाद् द्वाद॒शाद् द्वा॑द॒शात् पुरु॑षा॒ दन्न॒ मन्न॒म् पुरु॑षाद् द्वाद॒शाद् द्वा॑द॒शात् पुरु॑षा॒ दन्न᳚म् । \newline
18. पुरु॑षा॒ दन्न॒ मन्न॒म् पुरु॑षा॒त् पुरु॑षा॒ दन्न॑ मत्त्य॒ त्त्यन्न॒म् पुरु॑षा॒त् पुरु॑षा॒ दन्न॑ मत्ति । \newline
19. अन्न॑ मत्त्य॒ त्त्यन्न॒ मन्न॑ म॒त्त्यथो॒ अथो॑ अ॒त्त्यन्न॒ मन्न॑ म॒त्त्यथो᳚ । \newline
20. अ॒त्त्यथो॒ अथो॑ अत्त्य॒ त्त्यथो॒ पात्रा॒त् पात्रा॒ दथो॑ अत्त्य॒ त्त्यथो॒ पात्रा᳚त् । \newline
21. अथो॒ पात्रा॒त् पात्रा॒ दथो॒ अथो॒ पात्रा॒न् न न पात्रा॒ दथो॒ अथो॒ पात्रा॒न् न । \newline
22. अथो॒ इत्यथो᳚ । \newline
23. पात्रा॒न् न न पात्रा॒त् पात्रा॒न् न छि॑द्यते छिद्यते॒ न पात्रा॒त् पात्रा॒न् न छि॑द्यते । \newline
24. न छि॑द्यते छिद्यते॒ न न छि॑द्यते॒ यस्य॒ यस्य॑ छिद्यते॒ न न छि॑द्यते॒ यस्य॑ । \newline
25. छि॒द्य॒ते॒ यस्य॒ यस्य॑ छिद्यते छिद्यते॒ यस्यै॒ता ए॒ता यस्य॑ छिद्यते छिद्यते॒ यस्यै॒ताः । \newline
26. यस्यै॒ता ए॒ता यस्य॒ यस्यै॒ता उ॑पधी॒यन्त॑ उपधी॒यन्त॑ ए॒ता यस्य॒ यस्यै॒ता उ॑पधी॒यन्ते᳚ । \newline
27. ए॒ता उ॑पधी॒यन्त॑ उपधी॒यन्त॑ ए॒ता ए॒ता उ॑पधी॒यन्ते॒ यो य उ॑पधी॒यन्त॑ ए॒ता ए॒ता उ॑पधी॒यन्ते॒ यः । \newline
28. उ॒प॒धी॒यन्ते॒ यो य उ॑पधी॒यन्त॑ उपधी॒यन्ते॒ य उ॑ वु॒ य उ॑पधी॒यन्त॑ उपधी॒यन्ते॒ य उ॑ । \newline
29. उ॒प॒धी॒यन्त॒ इत्यु॑प - धी॒यन्ते᳚ । \newline
30. य उ॑ वु॒ यो य उ॑ च चो॒ यो य उ॑ च । \newline
31. उ॒ च॒ च॒ वु॒ चै॒ना॒ ए॒ना॒श्च॒ वु॒ चै॒नाः॒ । \newline
32. चै॒ना॒ ए॒ना॒श्च॒ चै॒ना॒ ए॒व मे॒व मे॑नाश्च चैना ए॒वम् । \newline
33. ए॒ना॒ ए॒व मे॒व मे॑ना एना ए॒वं ॅवेद॒ वेदै॒व मे॑ना एना ए॒वं ॅवेद॑ । \newline
34. ए॒वं ॅवेद॒ वेदै॒व मे॒वं ॅवेद॑ कुं॒भाः कुं॒भा वेदै॒व मे॒वं ॅवेद॑ कुं॒भाः । \newline
35. वेद॑ कुं॒भाः कुं॒भा वेद॒ वेद॑ कुं॒भाश्च॑ च कुं॒भा वेद॒ वेद॑ कुं॒भाश्च॑ । \newline
36. कुं॒भाश्च॑ च कुं॒भाः कुं॒भाश्च॑ कुं॒भीः कुं॒भीश्च॑ कुं॒भाः कुं॒भाश्च॑ कुं॒भीः । \newline
37. च॒ कुं॒भीः कुं॒भीश्च॑ च कुं॒भीश्च॑ च कुं॒भीश्च॑ च कुं॒भीश्च॑ । \newline
38. कुं॒भीश्च॑ च कुं॒भीः कुं॒भीश्च॑ मिथु॒नानि॑ मिथु॒नानि॑ च कुं॒भीः कुं॒भीश्च॑ मिथु॒नानि॑ । \newline
39. च॒ मि॒थु॒नानि॑ मिथु॒नानि॑ च च मिथु॒नानि॑ भवन्ति भवन्ति मिथु॒नानि॑ च च मिथु॒नानि॑ भवन्ति । \newline
40. मि॒थु॒नानि॑ भवन्ति भवन्ति मिथु॒नानि॑ मिथु॒नानि॑ भवन्ति मिथु॒नस्य॑ मिथु॒नस्य॑ भवन्ति मिथु॒नानि॑ मिथु॒नानि॑ भवन्ति मिथु॒नस्य॑ । \newline
41. भ॒व॒न्ति॒ मि॒थु॒नस्य॑ मिथु॒नस्य॑ भवन्ति भवन्ति मिथु॒नस्य॒ प्रजा᳚त्यै॒ प्रजा᳚त्यै मिथु॒नस्य॑ भवन्ति भवन्ति मिथु॒नस्य॒ प्रजा᳚त्यै । \newline
42. मि॒थु॒नस्य॒ प्रजा᳚त्यै॒ प्रजा᳚त्यै मिथु॒नस्य॑ मिथु॒नस्य॒ प्रजा᳚त्यै॒ प्र प्र प्रजा᳚त्यै मिथु॒नस्य॑ मिथु॒नस्य॒ प्रजा᳚त्यै॒ प्र । \newline
43. प्रजा᳚त्यै॒ प्र प्र प्रजा᳚त्यै॒ प्रजा᳚त्यै॒ प्र प्र॒जया᳚ प्र॒जया॒ प्र प्रजा᳚त्यै॒ प्रजा᳚त्यै॒ प्र प्र॒जया᳚ । \newline
44. प्रजा᳚त्या॒ इति॒ प्र - जा॒त्यै॒ । \newline
45. प्र प्र॒जया᳚ प्र॒जया॒ प्र प्र प्र॒जया॑ प॒शुभिः॑ प॒शुभिः॑ प्र॒जया॒ प्र प्र प्र॒जया॑ प॒शुभिः॑ । \newline
46. प्र॒जया॑ प॒शुभिः॑ प॒शुभिः॑ प्र॒जया᳚ प्र॒जया॑ प॒शुभि॑र् मिथु॒नैर् मि॑थु॒नैः प॒शुभिः॑ प्र॒जया᳚ प्र॒जया॑ प॒शुभि॑र् मिथु॒नैः । \newline
47. प्र॒जयेति॑ प्र - जया᳚ । \newline
48. प॒शुभि॑र् मिथु॒नैर् मि॑थु॒नैः प॒शुभिः॑ प॒शुभि॑र् मिथु॒नैर् जा॑यते जायते मिथु॒नैः प॒शुभिः॑ प॒शुभि॑र् मिथु॒नैर् जा॑यते । \newline
49. प॒शुभि॒रिति॑ प॒शु - भिः॒ । \newline
50. मि॒थु॒नैर् जा॑यते जायते मिथु॒नैर् मि॑थु॒नैर् जा॑यते॒ यस्य॒ यस्य॑ जायते मिथु॒नैर् मि॑थु॒नैर् जा॑यते॒ यस्य॑ । \newline
51. जा॒य॒ते॒ यस्य॒ यस्य॑ जायते जायते॒ यस्यै॒ता ए॒ता यस्य॑ जायते जायते॒ यस्यै॒ताः । \newline
52. यस्यै॒ता ए॒ता यस्य॒ यस्यै॒ता उ॑पधी॒यन्त॑ उपधी॒यन्त॑ ए॒ता यस्य॒ यस्यै॒ता उ॑पधी॒यन्ते᳚ । \newline
53. ए॒ता उ॑पधी॒यन्त॑ उपधी॒यन्त॑ ए॒ता ए॒ता उ॑पधी॒यन्ते॒ यो य उ॑पधी॒यन्त॑ ए॒ता ए॒ता उ॑पधी॒यन्ते॒ यः । \newline
54. उ॒प॒धी॒यन्ते॒ यो य उ॑पधी॒यन्त॑ उपधी॒यन्ते॒ य उ॑ वु॒ य उ॑पधी॒यन्त॑ उपधी॒यन्ते॒ य उ॑ । \newline
55. उ॒प॒धी॒यन्त॒ इत्यु॑प - धी॒यन्ते᳚ । \newline
56. य उ॑ वु॒ यो य उ॑ च चो॒ यो य उ॑ च । \newline
57. उ॒ च॒ च॒ वु॒ चै॒ना॒ ए॒ना॒श्च॒ वु॒ चै॒नाः॒ । \newline
\pagebreak
\markright{ TS 5.6.2.4  \hfill https://www.vedavms.in \hfill}

\section{ TS 5.6.2.4 }

\textbf{TS 5.6.2.4 } \newline
\textbf{Samhita Paata} \newline

चैना ए॒वं ॅवेद॒ शुग्वा अ॒ग्निः सो᳚ऽद्ध्व॒र्युं ॅयज॑मानं प्र॒जाः शु॒चाऽर्प॑यति॒ यद॒प उ॑प॒दधा॑ति॒ शुच॑मे॒वास्य॑ शमयति॒ नाऽऽ*र्ति॒मार्च्छ॑त्यद्ध्व॒र्युर्न यज॑मानः॒ शाम्य॑न्ति प्र॒जा यत्रै॒ता उ॑पधी॒यन्ते॒ ऽपां ॅवा ए॒तानि॒ हृद॑यानि॒ यदे॒ता आपो॒ यदे॒ता अ॒प उ॑प॒दधा॑ति दि॒व्याभि॑रे॒वैनाः॒ सꣳ सृ॑जति॒ वर्.षु॑कः प॒र्जन्यो॑ - [  ] \newline

\textbf{Pada Paata} \newline

च॒ । ए॒नाः॒ । ए॒वम् । वेद॑ । शुक् । वै । अ॒ग्निः । सः । अ॒द्ध्व॒र्युम् । यज॑मानम् । प्र॒जा इति॑ प्र - जाः । शु॒चा । अ॒र्प॒य॒ति॒ । यत् । अ॒पः । उ॒प॒दधा॒तीत्यु॑प - दधा॑ति । शुच᳚म् । ए॒व । अ॒स्य॒ । श॒म॒य॒ति॒ । न । आर्ति᳚म् । एति॑ । ऋ॒च्छ॒ति॒ । अ॒द्ध्व॒र्युः । न । यज॑मानः । शाम्य॑न्ति । प्र॒जा इति॑ प्र - जाः । यत्र॑ । ए॒ताः । उ॒प॒धी॒यन्त॒ इत्यु॑प - धी॒यन्ते᳚ । अ॒पाम् । वै । ए॒तानि॑ । हृद॑यानि । यत् । ए॒ताः । आपः॑ । यत् । ए॒ताः । अ॒पः । उ॒प॒दधा॒तीत्यु॑प - दधा॑ति । दि॒व्याभिः॑ । ए॒व । ए॒नाः॒ । समिति॑ । सृ॒ज॒ति॒ । वर्.षु॑कः । प॒र्जन्यः॑ ।  \newline


\textbf{Krama Paata} \newline

चै॒नाः॒ । ए॒ना॒ ए॒वम् । ए॒वम् ॅवेद॑ । वेद॒ शुक् । शुग् वै । वा अ॒ग्निः । अ॒ग्निः सः । सो᳚ऽद्ध्व॒र्युम् । अ॒द्ध्व॒र्युम् ॅयज॑मानम् । यज॑मानम् प्र॒जाः । प्र॒जाः शु॒चा । प्र॒जा इति॑ प्र - जाः । शु॒चाऽर्प॑यति । अ॒र्प॒य॒ति॒ यत् । यद॒पः । अ॒प उ॑प॒दधा॑ति । उ॒प॒दधा॑ति॒ शुच᳚म् । उ॒प॒दधा॒तीत्यु॑प - दधा॑ति । शुच॑मे॒व । ए॒वास्य॑ । अ॒स्य॒ श॒म॒य॒ति॒ । श॒म॒य॒ति॒ न । नार्ति᳚म् । आर्ति॒मा । आर्च्छ॑ति । ऋ॒च्छ॒त्य॒द्ध्व॒र्युः । अ॒द्ध्व॒र्युर् न । न यज॑मानः । यज॑मानः॒ शाम्य॑न्ति । शाम्य॑न्ति प्र॒जाः । प्र॒जा यत्र॑ । प्र॒जा इति॑ प्र - जाः । यत्रै॒ताः । ए॒ता उ॑पधी॒यन्ते᳚ । उ॒प॒धी॒यन्ते॒ऽपाम् । उ॒प॒धी॒यन्त॒ इत्यु॑प - धी॒यन्ते᳚ । अ॒पाम् ॅवै । वा ए॒तानि॑ । ए॒तानि॒ हृद॑यानि । हृद॑यानि॒ यत् । यदे॒ताः । ए॒ता आपः॑ । आपो॒ यत् । यदे॒ताः । ए॒ता अ॒पः । अ॒प उ॑प॒दधा॑ति । उ॒प॒दधा॑ति दि॒व्याभिः॑ । उ॒प॒दधा॒तीत्यु॑प - दधा॑ति । दि॒व्याभि॑रे॒व । ए॒वैनाः᳚ । ए॒नाः॒ सम् । सꣳ सृ॑जति । सृ॒ज॒ति॒ वर्.षु॑कः । वर्.षु॑कः प॒र्जन्यः॑ । प॒र्जन्यो॑ भवति \newline

\textbf{Jatai Paata} \newline

1. चै॒ना॒ ए॒ना॒श्च॒ चै॒नाः॒ । \newline
2. ए॒ना॒ ए॒व मे॒व मे॑ना एना ए॒वम् । \newline
3. ए॒वं ॅवेद॒ वेदै॒व मे॒वं ॅवेद॑ । \newline
4. वेद॒ शुक् छुग् वेद॒ वेद॒ शुक् । \newline
5. शुग् वै वै शुक् छुग् वै । \newline
6. वा अ॒ग्नि र॒ग्निर् वै वा अ॒ग्निः । \newline
7. अ॒ग्निः स सो᳚ ऽग्नि र॒ग्निः सः । \newline
8. सो᳚ ऽद्ध्व॒र्यु म॑द्ध्व॒र्युꣳ स सो᳚ ऽद्ध्व॒र्युम् । \newline
9. अ॒द्ध्व॒र्युं ॅयज॑मानं॒ ॅयज॑मान मद्ध्व॒र्यु म॑द्ध्व॒र्युं ॅयज॑मानम् । \newline
10. यज॑मानम् प्र॒जाः प्र॒जा यज॑मानं॒ ॅयज॑मानम् प्र॒जाः । \newline
11. प्र॒जाः शु॒चा शु॒चा प्र॒जाः प्र॒जाः शु॒चा । \newline
12. प्र॒जा इति॑ प्र - जाः । \newline
13. शु॒चा ऽर्प॑य त्यर्पयति शु॒चा शु॒चा ऽर्प॑यति । \newline
14. अ॒र्प॒य॒ति॒ यद् यद॑र्पय त्यर्पयति॒ यत् । \newline
15. यद॒पो॑ ऽपो यद् यद॒पः । \newline
16. अ॒प उ॑प॒दधा᳚ त्युप॒दधा᳚ त्य॒पो॑ ऽप उ॑प॒दधा॑ति । \newline
17. उ॒प॒दधा॑ति॒ शुचꣳ॒॒ शुच॑ मुप॒दधा᳚ त्युप॒दधा॑ति॒ शुच᳚म् । \newline
18. उ॒प॒दधा॒तीत्यु॑प - दधा॑ति । \newline
19. शुच॑ मे॒वैव शुचꣳ॒॒ शुच॑ मे॒व । \newline
20. ए॒वास्या᳚ स्यै॒वै वास्य॑ । \newline
21. अ॒स्य॒ श॒म॒य॒ति॒ श॒म॒य॒त्य॒ स्या॒स्य॒ श॒म॒य॒ति॒ । \newline
22. श॒म॒य॒ति॒ न न श॑मयति शमयति॒ न । \newline
23. नार्ति॒ मार्ति॒म् न नार्ति᳚म् । \newline
24. आर्ति॒ मा ऽऽर्ति॒ मार्ति॒ मा । \newline
25. आर्च्छ॑ त्यृच्छ॒ त्यार्च्छ॑ति । \newline
26. ऋ॒च्छ॒ त्य॒द्ध्व॒र्यु र॑द्ध्व॒र्युर्. ऋ॑च्छ त्यृच्छ त्यद्ध्व॒र्युः । \newline
27. अ॒द्ध्व॒र्युर् न नाद्ध्व॒र्यु र॑द्ध्व॒र्युर् न । \newline
28. न यज॑मानो॒ यज॑मानो॒ न न यज॑मानः । \newline
29. यज॑मानः॒ शाम्य॑न्ति॒ शाम्य॑न्ति॒ यज॑मानो॒ यज॑मानः॒ शाम्य॑न्ति । \newline
30. शाम्य॑न्ति प्र॒जाः प्र॒जाः शाम्य॑न्ति॒ शाम्य॑न्ति प्र॒जाः । \newline
31. प्र॒जा यत्र॒ यत्र॑ प्र॒जाः प्र॒जा यत्र॑ । \newline
32. प्र॒जा इति॑ प्र - जाः । \newline
33. यत्रै॒ता ए॒ता यत्र॒ यत्रै॒ताः । \newline
34. ए॒ता उ॑पधी॒यन्त॑ उपधी॒यन्त॑ ए॒ता ए॒ता उ॑पधी॒यन्ते᳚ । \newline
35. उ॒प॒धी॒यन्ते॒ ऽपा म॒पा मु॑पधी॒यन्त॑ उपधी॒यन्ते॒ ऽपाम् । \newline
36. उ॒प॒धी॒यन्त॒ इत्यु॑प - धी॒यन्ते᳚ । \newline
37. अ॒पां ॅवै वा अ॒पा म॒पां ॅवै । \newline
38. वा ए॒ता न्ये॒तानि॒ वै वा ए॒तानि॑ । \newline
39. ए॒तानि॒ हृद॑यानि॒ हृद॑या न्ये॒ता न्ये॒तानि॒ हृद॑यानि । \newline
40. हृद॑यानि॒ यद् यद्धृद॑यानि॒ हृद॑यानि॒ यत् । \newline
41. यदे॒ता ए॒ता यद् यदे॒ताः । \newline
42. ए॒ता आप॒ आप॑ ए॒ता ए॒ता आपः॑ । \newline
43. आपो॒ यद् यदाप॒ आपो॒ यत् । \newline
44. यदे॒ता ए॒ता यद् यदे॒ताः । \newline
45. ए॒ता अ॒पो॑ ऽप ए॒ता ए॒ता अ॒पः । \newline
46. अ॒प उ॑प॒दधा᳚ त्युप॒दधा᳚ त्य॒पो॑ ऽप उ॑प॒दधा॑ति । \newline
47. उ॒प॒दधा॑ति दि॒व्याभि॑र् दि॒व्याभि॑ रुप॒दधा᳚ त्युप॒दधा॑ति दि॒व्याभिः॑ । \newline
48. उ॒प॒दधा॒तीत्यु॑प - दधा॑ति । \newline
49. दि॒व्याभि॑ रे॒वैव दि॒व्याभि॑र् दि॒व्याभि॑ रे॒व । \newline
50. ए॒वैना॑ एना ए॒वै वैनाः᳚ । \newline
51. ए॒नाः॒ सꣳ स मे॑ना एनाः॒ सम् । \newline
52. सꣳ सृ॑जति सृजति॒ सꣳ सꣳ सृ॑जति । \newline
53. सृ॒ज॒ति॒ वर्.षु॑को॒ वर्.षु॑कः सृजति सृजति॒ वर्.षु॑कः । \newline
54. वर्.षु॑कः प॒र्जन्यः॑ प॒र्जन्यो॒ वर्.षु॑को॒ वर्.षु॑कः प॒र्जन्यः॑ । \newline
55. प॒र्जन्यो॑ भवति भवति प॒र्जन्यः॑ प॒र्जन्यो॑ भवति । \newline

\textbf{Ghana Paata } \newline

1. चै॒ना॒ ए॒ना॒श्च॒ चै॒ना॒ ए॒व मे॒व मे॑नाश्च चैना ए॒वम् । \newline
2. ए॒ना॒ ए॒व मे॒व मे॑ना एना ए॒वं ॅवेद॒ वेदै॒व मे॑ना एना ए॒वं ॅवेद॑ । \newline
3. ए॒वं ॅवेद॒ वेदै॒व मे॒वं ॅवेद॒ शुक् छुग् वेदै॒व मे॒वं ॅवेद॒ शुक् । \newline
4. वेद॒ शुक् छुग् वेद॒ वेद॒ शुग् वै वै शुग् वेद॒ वेद॒ शुग् वै । \newline
5. शुग् वै वै शुक् छुग् वा अ॒ग्नि र॒ग्निर् वै शुक् छुग् वा अ॒ग्निः । \newline
6. वा अ॒ग्नि र॒ग्निर् वै वा अ॒ग्निः स सो᳚ ऽग्निर् वै वा अ॒ग्निः सः । \newline
7. अ॒ग्निः स सो᳚ ऽग्नि र॒ग्निः सो᳚ ऽद्ध्व॒र्यु म॑द्ध्व॒र्युꣳ सो᳚ ऽग्नि र॒ग्निः सो᳚ ऽद्ध्व॒र्युम् । \newline
8. सो᳚ ऽद्ध्व॒र्यु म॑द्ध्व॒र्युꣳ स सो᳚ ऽद्ध्व॒र्युं ॅयज॑मानं॒ ॅयज॑मान मद्ध्व॒र्युꣳ स सो᳚ ऽद्ध्व॒र्युं ॅयज॑मानम् । \newline
9. अ॒द्ध्व॒र्युं ॅयज॑मानं॒ ॅयज॑मान मद्ध्व॒र्यु म॑द्ध्व॒र्युं ॅयज॑मानम् प्र॒जाः प्र॒जा यज॑मान मद्ध्व॒र्यु म॑द्ध्व॒र्युं ॅयज॑मानम् प्र॒जाः । \newline
10. यज॑मानम् प्र॒जाः प्र॒जा यज॑मानं॒ ॅयज॑मानम् प्र॒जाः शु॒चा शु॒चा प्र॒जा यज॑मानं॒ ॅयज॑मानम् प्र॒जाः शु॒चा । \newline
11. प्र॒जाः शु॒चा शु॒चा प्र॒जाः प्र॒जाः शु॒चा ऽर्प॑य त्यर्पयति शु॒चा प्र॒जाः प्र॒जाः शु॒चा ऽर्प॑यति । \newline
12. प्र॒जा इति॑ प्र - जाः । \newline
13. शु॒चा ऽर्प॑य त्यर्पयति शु॒चा शु॒चा ऽर्प॑यति॒ यद् यद॑र्पयति शु॒चा शु॒चा ऽर्प॑यति॒ यत् । \newline
14. अ॒र्प॒य॒ति॒ यद् यद॑र्पय त्यर्पयति॒ यद॒पो॑ ऽपो यद॑र्पय त्यर्पयति॒ यद॒पः । \newline
15. यद॒पो॑ ऽपो यद् यद॒प उ॑प॒दधा᳚ त्युप॒दधा᳚ त्य॒पो यद् यद॒प उ॑प॒दधा॑ति । \newline
16. अ॒प उ॑प॒दधा᳚ त्युप॒दधा᳚ त्य॒पो॑ ऽप उ॑प॒दधा॑ति॒ शुचꣳ॒॒ शुच॑ मुप॒दधा᳚ त्य॒पो॑ ऽप उ॑प॒दधा॑ति॒ शुच᳚म् । \newline
17. उ॒प॒दधा॑ति॒ शुचꣳ॒॒ शुच॑ मुप॒दधा᳚ त्युप॒दधा॑ति॒ शुच॑ मे॒वैव शुच॑ मुप॒दधा᳚ त्युप॒दधा॑ति॒ शुच॑ मे॒व । \newline
18. उ॒प॒दधा॒तीत्यु॑प - दधा॑ति । \newline
19. शुच॑ मे॒वैव शुचꣳ॒॒ शुच॑ मे॒वास्या᳚ स्यै॒व शुचꣳ॒॒ शुच॑ मे॒वास्य॑ । \newline
20. ए॒वास्या᳚ स्यै॒वै वास्य॑ शमयति शमय त्यस्यै॒वै वास्य॑ शमयति । \newline
21. अ॒स्य॒ श॒म॒य॒ति॒ श॒म॒य॒ त्य॒स्या॒स्य॒ श॒म॒य॒ति॒ न न श॑मय त्यस्यास्य शमयति॒ न । \newline
22. श॒म॒य॒ति॒ न न श॑मयति शमयति॒ नार्ति॒ मार्ति॒म् न श॑मयति शमयति॒ नार्ति᳚म् । \newline
23. नार्ति॒ मार्ति॒म् न नार्ति॒ मा ऽऽर्ति॒म् न नार्ति॒ मा । \newline
24. आर्ति॒ मा ऽऽर्ति॒ मार्ति॒ मार्च्छ॑ त्यृच्छ॒ त्याऽऽर्ति॒ मार्ति॒ मार्च्छ॑ति । \newline
25. आर्च्छ॑ त्यृच्छ॒ त्यार्च्छ॑ त्यद्ध्व॒र्यु र॑द्ध्व॒र्युर्. ऋ॑च्छ॒ त्यार्च्छ॑ त्यद्ध्व॒र्युः । \newline
26. ऋ॒च्छ॒ त्य॒द्ध्व॒र्यु र॑द्ध्व॒र्युर्. ऋ॑च्छ त्यृच्छ त्यद्ध्व॒र्युर् न नाद्ध्व॒र्युर्. ऋ॑च्छ त्यृच्छ त्यद्ध्व॒र्युर् न । \newline
27. अ॒द्ध्व॒र्युर् न नाद्ध्व॒र्यु र॑द्ध्व॒र्युर् न यज॑मानो॒ यज॑मानो॒ नाद्ध्व॒र्यु र॑द्ध्व॒र्युर् न यज॑मानः । \newline
28. न यज॑मानो॒ यज॑मानो॒ न न यज॑मानः॒ शाम्य॑न्ति॒ शाम्य॑न्ति॒ यज॑मानो॒ न न यज॑मानः॒ शाम्य॑न्ति । \newline
29. यज॑मानः॒ शाम्य॑न्ति॒ शाम्य॑न्ति॒ यज॑मानो॒ यज॑मानः॒ शाम्य॑न्ति प्र॒जाः प्र॒जाः शाम्य॑न्ति॒ यज॑मानो॒ यज॑मानः॒ शाम्य॑न्ति प्र॒जाः । \newline
30. शाम्य॑न्ति प्र॒जाः प्र॒जाः शाम्य॑न्ति॒ शाम्य॑न्ति प्र॒जा यत्र॒ यत्र॑ प्र॒जाः शाम्य॑न्ति॒ शाम्य॑न्ति प्र॒जा यत्र॑ । \newline
31. प्र॒जा यत्र॒ यत्र॑ प्र॒जाः प्र॒जा यत्रै॒ता ए॒ता यत्र॑ प्र॒जाः प्र॒जा यत्रै॒ताः । \newline
32. प्र॒जा इति॑ प्र - जाः । \newline
33. यत्रै॒ता ए॒ता यत्र॒ यत्रै॒ता उ॑पधी॒यन्त॑ उपधी॒यन्त॑ ए॒ता यत्र॒ यत्रै॒ता उ॑पधी॒यन्ते᳚ । \newline
34. ए॒ता उ॑पधी॒यन्त॑ उपधी॒यन्त॑ ए॒ता ए॒ता उ॑पधी॒यन्ते॒ ऽपा म॒पा मु॑पधी॒यन्त॑ ए॒ता ए॒ता उ॑पधी॒यन्ते॒ ऽपाम् । \newline
35. उ॒प॒धी॒यन्ते॒ ऽपा म॒पा मु॑पधी॒यन्त॑ उपधी॒यन्ते॒ ऽपां ॅवै वा अ॒पा मु॑पधी॒यन्त॑ उपधी॒यन्ते॒ ऽपां ॅवै । \newline
36. उ॒प॒धी॒यन्त॒ इत्यु॑प - धी॒यन्ते᳚ । \newline
37. अ॒पां ॅवै वा अ॒पा म॒पां ॅवा ए॒ता न्ये॒तानि॒ वा अ॒पा म॒पां ॅवा ए॒तानि॑ । \newline
38. वा ए॒ता न्ये॒तानि॒ वै वा ए॒तानि॒ हृद॑यानि॒ हृद॑या न्ये॒तानि॒ वै वा ए॒तानि॒ हृद॑यानि । \newline
39. ए॒तानि॒ हृद॑यानि॒ हृद॑या न्ये॒ता न्ये॒तानि॒ हृद॑यानि॒ यद् यद्धृद॑या न्ये॒ता न्ये॒तानि॒ हृद॑यानि॒ यत् । \newline
40. हृद॑यानि॒ यद् यद्धृद॑यानि॒ हृद॑यानि॒ यदे॒ता ए॒ता यद्धृद॑यानि॒ हृद॑यानि॒ यदे॒ताः । \newline
41. यदे॒ता ए॒ता यद् यदे॒ता आप॒ आप॑ ए॒ता यद् यदे॒ता आपः॑ । \newline
42. ए॒ता आप॒ आप॑ ए॒ता ए॒ता आपो॒ यद् यदाप॑ ए॒ता ए॒ता आपो॒ यत् । \newline
43. आपो॒ यद् यदाप॒ आपो॒ यदे॒ता ए॒ता यदाप॒ आपो॒ यदे॒ताः । \newline
44. यदे॒ता ए॒ता यद् यदे॒ता अ॒पो॑ ऽप ए॒ता यद् यदे॒ता अ॒पः । \newline
45. ए॒ता अ॒पो॑ ऽप ए॒ता ए॒ता अ॒प उ॑प॒दधा᳚ त्युप॒दधा᳚ त्य॒प ए॒ता ए॒ता अ॒प उ॑प॒दधा॑ति । \newline
46. अ॒प उ॑प॒दधा᳚ त्युप॒दधा᳚ त्य॒पो॑ ऽप उ॑प॒दधा॑ति दि॒व्याभि॑र् दि॒व्याभि॑ रुप॒दधा᳚ त्य॒पो॑ ऽप उ॑प॒दधा॑ति दि॒व्याभिः॑ । \newline
47. उ॒प॒दधा॑ति दि॒व्याभि॑र् दि॒व्याभि॑ रुप॒दधा᳚ त्युप॒दधा॑ति दि॒व्याभि॑ रे॒वैव दि॒व्याभि॑ रुप॒दधा᳚ त्युप॒दधा॑ति दि॒व्याभि॑ रे॒व । \newline
48. उ॒प॒दधा॒तीत्यु॑प - दधा॑ति । \newline
49. दि॒व्याभि॑ रे॒वैव दि॒व्याभि॑र् दि॒व्याभि॑ रे॒वैना॑ एना ए॒व दि॒व्याभि॑र् दि॒व्याभि॑ रे॒वैनाः᳚ । \newline
50. ए॒वैना॑ एना ए॒वै वैनाः॒ सꣳ स मे॑ना ए॒वै वैनाः॒ सम् । \newline
51. ए॒नाः॒ सꣳ स मे॑ना एनाः॒ सꣳ सृ॑जति सृजति॒ स मे॑ना एनाः॒ सꣳ सृ॑जति । \newline
52. सꣳ सृ॑जति सृजति॒ सꣳ सꣳ सृ॑जति॒ वर्.षु॑को॒ वर्.षु॑कः सृजति॒ सꣳ सꣳ सृ॑जति॒ वर्.षु॑कः । \newline
53. सृ॒ज॒ति॒ वर्.षु॑को॒ वर्.षु॑कः सृजति सृजति॒ वर्.षु॑कः प॒र्जन्यः॑ प॒र्जन्यो॒ वर्.षु॑कः सृजति सृजति॒ वर्.षु॑कः प॒र्जन्यः॑ । \newline
54. वर्.षु॑कः प॒र्जन्यः॑ प॒र्जन्यो॒ वर्.षु॑को॒ वर्.षु॑कः प॒र्जन्यो॑ भवति भवति प॒र्जन्यो॒ वर्.षु॑को॒ वर्.षु॑कः प॒र्जन्यो॑ भवति । \newline
55. प॒र्जन्यो॑ भवति भवति प॒र्जन्यः॑ प॒र्जन्यो॑ भवति॒ यो यो भ॑वति प॒र्जन्यः॑ प॒र्जन्यो॑ भवति॒ यः । \newline
\pagebreak
\markright{ TS 5.6.2.5  \hfill https://www.vedavms.in \hfill}

\section{ TS 5.6.2.5 }

\textbf{TS 5.6.2.5 } \newline
\textbf{Samhita Paata} \newline

भवति॒ यो वा ए॒तासा॑मा॒यत॑नं॒ क्लृप्तिं॒ ॅवेदा॒ऽऽ*यत॑नवान् भवति॒ कल्प॑ते ऽस्मा अनुसी॒तमुप॑ दधात्ये॒तद्वा आ॑सामा॒यत॑नमे॒षा क्लृप्ति॒र्य ए॒वं ॅवेदा॒ऽऽ*यत॑नवान् भवति॒ कल्प॑तेऽस्मै द्व॒द्वंम॒न्या उप॑ दधाति॒ चत॑स्रो॒ मद्ध्ये॒ धृत्या॒ अन्नं॒ ॅवा इष्ट॑का ए॒तत् खलु॒ वै सा॒क्षादन्नं॒ ॅयदे॒ष च॒रुर्यदे॒तं च॒रुमु॑प॒ दधा॑ति सा॒क्षा - [  ] \newline

\textbf{Pada Paata} \newline

भ॒व॒ति॒ । यः । वै । ए॒तासा᳚म् । आ॒यत॑न॒मित्या᳚ - यत॑नम् । क्लृप्ति᳚म् । वेद॑ । आ॒यत॑नवा॒नित्या॒यत॑न - वा॒न् । भ॒व॒ति॒ । कल्प॑ते । अ॒स्मै॒ । अ॒नु॒सी॒तमित्य॑नु - सी॒तम् । उपेति॑ । द॒धा॒ति॒ । ए॒तत् । वै । आ॒सा॒म् । आ॒यत॑न॒मित्या᳚ - यत॑नम् । ए॒षा । क्लृप्तिः॑ । यः । ए॒वम् । वेद॑ । आ॒यत॑नवा॒नित्या॒यत॑न - वा॒न् । भ॒व॒ति॒ । कल्प॑ते । अ॒स्मै॒ । द्व॒द्वंमिति॑ द्वं - द्वम् । अ॒न्याः । उपेति॑ । द॒धा॒ति॒ । चत॑स्रः । मद्ध्ये᳚ । धृत्यै᳚ । अन्न᳚म् । वै । इष्ट॑काः । ए॒तत् । खलु॑ । वै । सा॒क्षादिति॑ स-अ॒क्षात् । अन्न᳚म् । यत् । ए॒षः । च॒रुः । यत् । ए॒तम् । च॒रुम् । उ॒प॒दधा॒तीत्यु॑प - दधा॑ति । सा॒क्षादिति॑ स - अ॒क्षात् ।  \newline


\textbf{Krama Paata} \newline

भ॒व॒ति॒ यः । यो वै । वा ए॒तासा᳚म् । ए॒तासा॑मा॒यत॑नम् । आ॒यत॑न॒म् क्लृप्ति᳚म् । आ॒यत॑न॒मित्या᳚ - यत॑नम् । क्लृप्ति॒म् ॅवेद॑ । वेदा॒यत॑नवान् । आ॒यत॑नवान् भवति । आ॒यत॑नवा॒नित्या॒यत॑न - वा॒न्॒ । भ॒व॒ति॒ कल्प॑ते । कल्प॑तेऽस्मै । अ॒स्मा॒ अ॒नु॒सी॒तम् । अ॒नु॒सी॒तमुप॑ । अ॒नु॒सी॒तमित्य॑नु - सी॒तम् । उप॑ दधाति । द॒धा॒त्ये॒तत् । ए॒तद् वै । वा आ॑साम् । आ॒सा॒मा॒यत॑नम् । आ॒यत॑नमे॒षा । आ॒यत॑न॒मित्या᳚ - यत॑नम् । ए॒षा क्लृप्तिः॑ । क्लृप्ति॒र् यः । य ए॒वम् । ए॒वम् ॅवेद॑ । वेदा॒यत॑नवान् । आ॒यत॑नवान् भवति । आ॒यत॑नवा॒नित्या॒यत॑न - वा॒न्॒ । भ॒व॒ति॒ कल्प॑ते । कल्प॑तेऽस्मै । अ॒स्मै॒ द्व॒न्द्वम् । द्व॒न्द्वम॒न्याः । द्व॒न्द्वमिति॑ द्वम् - द्वम् । अ॒न्या उप॑ । उप॑ दधाति । द॒धा॒ति॒ चत॑स्रः । चत॑स्रो॒ मद्ध्ये᳚ । मद्ध्ये॒ धृत्यै᳚ । धृत्या॒ अन्न᳚म् । अन्न॒म्ॅ वै । वा इष्ट॑काः । इष्ट॑का ए॒तत् । ए॒तत् खलु॑ । खलु॒ वै । वै सा॒क्षात् । सा॒क्षादन्न᳚म् । सा॒क्षादिति॑ स - अ॒क्षात् । अन्न॒म् ॅयत् । यदे॒षः । ए॒ष च॒रुः । च॒रुर् यत् । यदे॒तम् । ए॒तम् च॒रुम् । च॒रुमु॑प॒दधा॑ति । उ॒प॒दधा॑ति सा॒क्षात् ( ) । उ॒प॒दधा॒तीत्यु॑प - दधा॑ति । सा॒क्षादे॒व । सा॒क्षादिति॑ स - अ॒क्षात् \newline

\textbf{Jatai Paata} \newline

1. भ॒व॒ति॒ यो यो भ॑वति भवति॒ यः । \newline
2. यो वै वै यो यो वै । \newline
3. वा ए॒तासा॑ मे॒तासां॒ ॅवै वा ए॒तासा᳚म् । \newline
4. ए॒तासा॑ मा॒यत॑न मा॒यत॑न मे॒तासा॑ मे॒तासा॑ मा॒यत॑नम् । \newline
5. आ॒यत॑न॒म् क्लृप्ति॒म् क्लृप्ति॑ मा॒यत॑न मा॒यत॑न॒म् क्लृप्ति᳚म् । \newline
6. आ॒यत॑न॒मित्या᳚ - यत॑नम् । \newline
7. क्लृप्तिं॒ ॅवेद॒ वेद॒ क्लृप्ति॒म् क्लृप्तिं॒ ॅवेद॑ । \newline
8. वेदा॒ यत॑नवा ना॒यत॑नवा॒न्॒. वेद॒ वेदा॒ यत॑नवान् । \newline
9. आ॒यत॑नवान् भवति भव त्या॒यत॑नवा ना॒यत॑नवान् भवति । \newline
10. आ॒यत॑नवा॒नित्या॒यत॑न - वा॒न् । \newline
11. भ॒व॒ति॒ कल्प॑ते॒ कल्प॑ते भवति भवति॒ कल्प॑ते । \newline
12. कल्प॑ते ऽस्मा अस्मै॒ कल्प॑ते॒ कल्प॑ते ऽस्मै । \newline
13. अ॒स्मा॒ अ॒नु॒सी॒त म॑नुसी॒त म॑स्मा अस्मा अनुसी॒तम् । \newline
14. अ॒नु॒सी॒त मुपोपा॑ नुसी॒त म॑नुसी॒त मुप॑ । \newline
15. अ॒नु॒सी॒तमित्य॑नु - सी॒तम् । \newline
16. उप॑ दधाति दधा॒ त्युपोप॑ दधाति । \newline
17. द॒धा॒ त्ये॒त दे॒तद् द॑धाति दधा त्ये॒तत् । \newline
18. ए॒तद् वै वा ए॒त दे॒तद् वै । \newline
19. वा आ॑सा मासां॒ ॅवै वा आ॑साम् । \newline
20. आ॒सा॒ मा॒यत॑न मा॒यत॑न मासा मासा मा॒यत॑नम् । \newline
21. आ॒यत॑न मे॒षैषा ऽऽयत॑न मा॒यत॑न मे॒षा । \newline
22. आ॒यत॑न॒मित्या᳚ - यत॑नम् । \newline
23. ए॒षा क्लृप्तिः॒ क्लृप्ति॑ रे॒षैषा क्लृप्तिः॑ । \newline
24. क्लृप्ति॒र् यो यः क्लृप्तिः॒ क्लृप्ति॒र् यः । \newline
25. य ए॒व मे॒वं ॅयो य ए॒वम् । \newline
26. ए॒वं ॅवेद॒ वेदै॒व मे॒वं ॅवेद॑ । \newline
27. वेदा॒यत॑नवा ना॒यत॑नवा॒न्॒. वेद॒ वेदा॒यत॑नवान् । \newline
28. आ॒यत॑नवान् भवति भव त्या॒यत॑नवा ना॒यत॑नवान् भवति । \newline
29. आ॒यत॑नवा॒नित्या॒यत॑न - वा॒न् । \newline
30. भ॒व॒ति॒ कल्प॑ते॒ कल्प॑ते भवति भवति॒ कल्प॑ते । \newline
31. कल्प॑ते ऽस्मा अस्मै॒ कल्प॑ते॒ कल्प॑ते ऽस्मै । \newline
32. अ॒स्मै॒ द्व॒न्द्वम् द्व॒न्द्व म॑स्मा अस्मै द्व॒न्द्वम् । \newline
33. द्व॒न्द्व म॒न्या अ॒न्या द्व॒न्द्वम् द्व॒न्द्व म॒न्याः । \newline
34. द्व॒न्द्वमिति॑ द्वं - द्वम् । \newline
35. अ॒न्या उपोपा॒ न्या अ॒न्या उप॑ । \newline
36. उप॑ दधाति दधा॒ त्युपोप॑ दधाति । \newline
37. द॒धा॒ति॒ चत॑स्र॒ श्चत॑स्रो दधाति दधाति॒ चत॑स्रः । \newline
38. चत॑स्रो॒ मद्ध्ये॒ मद्ध्ये॒ चत॑स्र॒ श्चत॑स्रो॒ मद्ध्ये᳚ । \newline
39. मद्ध्ये॒ धृत्यै॒ धृत्यै॒ मद्ध्ये॒ मद्ध्ये॒ धृत्यै᳚ । \newline
40. धृत्या॒ अन्न॒ मन्न॒म् धृत्यै॒ धृत्या॒ अन्न᳚म् । \newline
41. अन्नं॒ ॅवै वा अन्न॒ मन्नं॒ ॅवै । \newline
42. वा इष्ट॑का॒ इष्ट॑का॒ वै वा इष्ट॑काः । \newline
43. इष्ट॑का ए॒तदे॒ तदिष्ट॑का॒ इष्ट॑का ए॒तत् । \newline
44. ए॒तत् खलु॒ खल्वे॒त दे॒तत् खलु॑ । \newline
45. खलु॒ वै वै खलु॒ खलु॒ वै । \newline
46. वै सा॒क्षाथ् सा॒क्षाद् वै वै सा॒क्षात् । \newline
47. सा॒क्षा दन्न॒ मन्नꣳ॑ सा॒क्षाथ् सा॒क्षा दन्न᳚म् । \newline
48. सा॒क्षादिति॑ स - अ॒क्षात् । \newline
49. अन्नं॒ ॅयद् यदन्न॒ मन्नं॒ ॅयत् । \newline
50. यदे॒ष ए॒ष यद् यदे॒षः । \newline
51. ए॒ष च॒रु श्च॒रु रे॒ष ए॒ष च॒रुः । \newline
52. च॒रुर् यद् यच् च॒रु श्च॒रुर् यत् । \newline
53. यदे॒त मे॒तं ॅयद् यदे॒तम् । \newline
54. ए॒तम् च॒रुम् च॒रु मे॒त मे॒तम् च॒रुम् । \newline
55. च॒रु मु॑प॒दधा᳚ त्युप॒दधा॑ति च॒रुम् च॒रु मु॑प॒दधा॑ति । \newline
56. उ॒प॒दधा॑ति सा॒क्षाथ् सा॒क्षा दु॑प॒दधा᳚ त्युप॒दधा॑ति सा॒क्षात् । \newline
57. उ॒प॒दधा॒तीत्यु॑प - दधा॑ति । \newline
58. सा॒क्षा दे॒वैव सा॒क्षाथ् सा॒क्षा दे॒व । \newline
59. सा॒क्षादिति॑ स - अ॒क्षात् । \newline

\textbf{Ghana Paata } \newline

1. भ॒व॒ति॒ यो यो भ॑वति भवति॒ यो वै वै यो भ॑वति भवति॒ यो वै । \newline
2. यो वै वै यो यो वा ए॒तासा॑ मे॒तासां॒ ॅवै यो यो वा ए॒तासा᳚म् । \newline
3. वा ए॒तासा॑ मे॒तासां॒ ॅवै वा ए॒तासा॑ मा॒यत॑न मा॒यत॑न मे॒तासां॒ ॅवै वा ए॒तासा॑ मा॒यत॑नम् । \newline
4. ए॒तासा॑ मा॒यत॑न मा॒यत॑न मे॒तासा॑ मे॒तासा॑ मा॒यत॑न॒म् क्लृप्ति॒म् क्लृप्ति॑ मा॒यत॑न मे॒तासा॑ मे॒तासा॑ मा॒यत॑न॒म् क्लृप्ति᳚म् । \newline
5. आ॒यत॑न॒म् क्लृप्ति॒म् क्लृप्ति॑ मा॒यत॑न मा॒यत॑न॒म् क्लृप्तिं॒ ॅवेद॒ वेद॒ क्लृप्ति॑ मा॒यत॑न मा॒यत॑न॒म् क्लृप्तिं॒ ॅवेद॑ । \newline
6. आ॒यत॑न॒मित्या᳚ - यत॑नम् । \newline
7. क्लृप्तिं॒ ॅवेद॒ वेद॒ क्लृप्ति॒म् क्लृप्तिं॒ ॅवेदा॒यत॑नवा ना॒यत॑नवा॒न्॒. वेद॒ क्लृप्ति॒म् क्लृप्तिं॒ ॅवेदा॒ यत॑नवान् । \newline
8. वेदा॒यत॑नवा ना॒यत॑नवा॒न्॒. वेद॒ वेदा॒यत॑नवान् भवति भव त्या॒यत॑नवा॒न्॒. वेद॒ वेदा॒यत॑नवान् भवति । \newline
9. आ॒यत॑नवान् भवति भव त्या॒यत॑नवा ना॒यत॑नवान् भवति॒ कल्प॑ते॒ कल्प॑ते भव त्या॒यत॑नवा ना॒यत॑नवान् भवति॒ कल्प॑ते । \newline
10. आ॒यत॑नवा॒नित्या॒यत॑न - वा॒न् । \newline
11. भ॒व॒ति॒ कल्प॑ते॒ कल्प॑ते भवति भवति॒ कल्प॑ते ऽस्मा अस्मै॒ कल्प॑ते भवति भवति॒ कल्प॑ते ऽस्मै । \newline
12. कल्प॑ते ऽस्मा अस्मै॒ कल्प॑ते॒ कल्प॑ते ऽस्मा अनुसी॒त म॑नुसी॒त म॑स्मै॒ कल्प॑ते॒ कल्प॑ते ऽस्मा अनुसी॒तम् । \newline
13. अ॒स्मा॒ अ॒नु॒सी॒त म॑नुसी॒त म॑स्मा अस्मा अनुसी॒त मुपोपा॑नुसी॒त म॑स्मा अस्मा अनुसी॒त मुप॑ । \newline
14. अ॒नु॒सी॒त मुपोपा॑ नुसी॒त म॑नुसी॒त मुप॑ दधाति दधा॒ त्युपा॑नुसी॒त म॑नुसी॒त मुप॑ दधाति । \newline
15. अ॒नु॒सी॒तमित्य॑नु - सी॒तम् । \newline
16. उप॑ दधाति दधा॒ त्युपोप॑ दधा त्ये॒त दे॒तद् द॑धा॒ त्युपोप॑ दधा त्ये॒तत् । \newline
17. द॒धा॒ त्ये॒त दे॒तद् द॑धाति दधा त्ये॒तद् वै वा ए॒तद् द॑धाति दधा त्ये॒तद् वै । \newline
18. ए॒तद् वै वा ए॒त दे॒तद् वा आ॑सा मासां॒ ॅवा ए॒त दे॒तद् वा आ॑साम् । \newline
19. वा आ॑सा मासां॒ ॅवै वा आ॑सा मा॒यत॑न मा॒यत॑न मासां॒ ॅवै वा आ॑सा मा॒यत॑नम् । \newline
20. आ॒सा॒ मा॒यत॑न मा॒यत॑न मासा मासा मा॒यत॑न मे॒षैषा ऽऽयत॑न मासा मासा मा॒यत॑न मे॒षा । \newline
21. आ॒यत॑न मे॒षैषा ऽऽयत॑न मा॒यत॑न मे॒षा क्लृप्तिः॒ क्लृप्ति॑ रे॒षा ऽऽयत॑न मा॒यत॑न मे॒षा क्लृप्तिः॑ । \newline
22. आ॒यत॑न॒मित्या᳚ - यत॑नम् । \newline
23. ए॒षा क्लृप्तिः॒ क्लृप्ति॑ रे॒षैषा क्लृप्ति॒र् यो यः क्लृप्ति॑ रे॒षैषा क्लृप्ति॒र् यः । \newline
24. क्लृप्ति॒र् यो यः क्लृप्तिः॒ क्लृप्ति॒र् य ए॒व मे॒वं ॅयः क्लृप्तिः॒ क्लृप्ति॒र् य ए॒वम् । \newline
25. य ए॒व मे॒वं ॅयो य ए॒वं ॅवेद॒ वेदै॒वं ॅयो य ए॒वं ॅवेद॑ । \newline
26. ए॒वं ॅवेद॒ वेदै॒व मे॒वं ॅवेदा॒यत॑नवा ना॒यत॑नवा॒न्॒. वेदै॒व मे॒वं ॅवेदा॒यत॑नवान् । \newline
27. वेदा॒यत॑नवा ना॒यत॑नवा॒न्॒. वेद॒ वेदा॒यत॑नवान् भवति भव त्या॒यत॑नवा॒न्॒. वेद॒ वेदा॒यत॑नवान् भवति । \newline
28. आ॒यत॑नवान् भवति भव त्या॒यत॑नवा ना॒यत॑नवान् भवति॒ कल्प॑ते॒ कल्प॑ते भव त्या॒यत॑नवा ना॒यत॑नवान् भवति॒ कल्प॑ते । \newline
29. आ॒यत॑नवा॒नित्या॒यत॑न - वा॒न् । \newline
30. भ॒व॒ति॒ कल्प॑ते॒ कल्प॑ते भवति भवति॒ कल्प॑ते ऽस्मा अस्मै॒ कल्प॑ते भवति भवति॒ कल्प॑ते ऽस्मै । \newline
31. कल्प॑ते ऽस्मा अस्मै॒ कल्प॑ते॒ कल्प॑ते ऽस्मै द्व॒न्द्वम् द्व॒न्द्व म॑स्मै॒ कल्प॑ते॒ कल्प॑ते ऽस्मै द्व॒न्द्वम् । \newline
32. अ॒स्मै॒ द्व॒न्द्वम् द्व॒न्द्व म॑स्मा अस्मै द्व॒न्द्व म॒न्या अ॒न्या द्व॒न्द्व म॑स्मा अस्मै द्व॒न्द्व म॒न्याः । \newline
33. द्व॒न्द्व म॒न्या अ॒न्या द्व॒न्द्वम् द्व॒न्द्व म॒न्या उपोपा॒न्या द्व॒न्द्वम् द्व॒न्द्व म॒न्या उप॑ । \newline
34. द्व॒न्द्वमिति॑ द्वं - द्वम् । \newline
35. अ॒न्या उपोपा॒न्या अ॒न्या उप॑ दधाति दधा॒ त्युपा॒न्या अ॒न्या उप॑ दधाति । \newline
36. उप॑ दधाति दधा॒ त्युपोप॑ दधाति॒ चत॑स्र॒ श्चत॑स्रो दधा॒ त्युपोप॑ दधाति॒ चत॑स्रः । \newline
37. द॒धा॒ति॒ चत॑स्र॒ श्चत॑स्रो दधाति दधाति॒ चत॑स्रो॒ मद्ध्ये॒ मद्ध्ये॒ चत॑स्रो दधाति दधाति॒ चत॑स्रो॒ मद्ध्ये᳚ । \newline
38. चत॑स्रो॒ मद्ध्ये॒ मद्ध्ये॒ चत॑स्र॒ श्चत॑स्रो॒ मद्ध्ये॒ धृत्यै॒ धृत्यै॒ मद्ध्ये॒ चत॑स्र॒ श्चत॑स्रो॒ मद्ध्ये॒ धृत्यै᳚ । \newline
39. मद्ध्ये॒ धृत्यै॒ धृत्यै॒ मद्ध्ये॒ मद्ध्ये॒ धृत्या॒ अन्न॒ मन्न॒म् धृत्यै॒ मद्ध्ये॒ मद्ध्ये॒ धृत्या॒ अन्न᳚म् । \newline
40. धृत्या॒ अन्न॒ मन्न॒म् धृत्यै॒ धृत्या॒ अन्नं॒ ॅवै वा अन्न॒म् धृत्यै॒ धृत्या॒ अन्नं॒ ॅवै । \newline
41. अन्नं॒ ॅवै वा अन्न॒ मन्नं॒ ॅवा इष्ट॑का॒ इष्ट॑का॒ वा अन्न॒ मन्नं॒ ॅवा इष्ट॑काः । \newline
42. वा इष्ट॑का॒ इष्ट॑का॒ वै वा इष्ट॑का ए॒त दे॒त दिष्ट॑का॒ वै वा इष्ट॑का ए॒तत् । \newline
43. इष्ट॑का ए॒त दे॒त दिष्ट॑का॒ इष्ट॑का ए॒तत् खलु॒ खल्वे॒त दिष्ट॑का॒ इष्ट॑का ए॒तत् खलु॑ । \newline
44. ए॒तत् खलु॒ खल्वे॒त दे॒तत् खलु॒ वै वै खल्वे॒त दे॒तत् खलु॒ वै । \newline
45. खलु॒ वै वै खलु॒ खलु॒ वै सा॒क्षाथ् सा॒क्षाद् वै खलु॒ खलु॒ वै सा॒क्षात् । \newline
46. वै सा॒क्षाथ् सा॒क्षाद् वै वै सा॒क्षा दन्न॒ मन्नꣳ॑ सा॒क्षाद् वै वै सा॒क्षा दन्न᳚म् । \newline
47. सा॒क्षा दन्न॒ मन्नꣳ॑ सा॒क्षाथ् सा॒क्षा दन्नं॒ ॅयद् यदन्नꣳ॑ सा॒क्षाथ् सा॒क्षा दन्नं॒ ॅयत् । \newline
48. सा॒क्षादिति॑ स - अ॒क्षात् । \newline
49. अन्नं॒ ॅयद् यदन्न॒ मन्नं॒ ॅयदे॒ष ए॒ष यदन्न॒ मन्नं॒ ॅयदे॒षः । \newline
50. यदे॒ष ए॒ष यद् यदे॒ष च॒रु श्च॒रु रे॒ष यद् यदे॒ष च॒रुः । \newline
51. ए॒ष च॒रु श्च॒रु रे॒ष ए॒ष च॒रुर् यद् यच् च॒रु रे॒ष ए॒ष च॒रुर् यत् । \newline
52. च॒रुर् यद् यच् च॒रु श्च॒रुर् यदे॒त मे॒तं ॅयच् च॒रु श्च॒रुर् यदे॒तम् । \newline
53. यदे॒त मे॒तं ॅयद् यदे॒तम् च॒रुम् च॒रु मे॒तं ॅयद् यदे॒तम् च॒रुम् । \newline
54. ए॒तम् च॒रुम् च॒रु मे॒त मे॒तम् च॒रु मु॑प॒दधा᳚ त्युप॒दधा॑ति च॒रु मे॒त मे॒तम् च॒रु मु॑प॒दधा॑ति । \newline
55. च॒रु मु॑प॒दधा᳚ त्युप॒दधा॑ति च॒रुम् च॒रु मु॑प॒दधा॑ति सा॒क्षाथ् सा॒क्षा दु॑प॒दधा॑ति च॒रुम् च॒रु मु॑प॒दधा॑ति सा॒क्षात् । \newline
56. उ॒प॒दधा॑ति सा॒क्षाथ् सा॒क्षा दु॑प॒दधा᳚ त्युप॒दधा॑ति सा॒क्षा दे॒वैव सा॒क्षा दु॑प॒दधा᳚ त्युप॒दधा॑ति सा॒क्षा दे॒व । \newline
57. उ॒प॒दधा॒तीत्यु॑प - दधा॑ति । \newline
58. सा॒क्षा दे॒वैव सा॒क्षाथ् सा॒क्षा दे॒वास्मा॑ अस्मा ए॒व सा॒क्षाथ् सा॒क्षा दे॒वास्मै᳚ । \newline
59. सा॒क्षादिति॑ स - अ॒क्षात् । \newline
\pagebreak
\markright{ TS 5.6.2.6  \hfill https://www.vedavms.in \hfill}

\section{ TS 5.6.2.6 }

\textbf{TS 5.6.2.6 } \newline
\textbf{Samhita Paata} \newline

दे॒वास्मा॒ अन्न॒मव॑ रुन्धे मद्ध्य॒त उप॑ दधाति मद्ध्य॒त ए॒वास्मा॒ अन्नं॑ दधाति॒ तस्मा᳚न् मद्ध्य॒तोऽन्न॑मद्यते बार्.हस्प॒त्यो भ॑वति॒ ब्रह्म॒ वै दे॒वानां॒ बृह॒स्पति॒-र्ब्रह्म॑णै॒वास्मा॒ अन्न॒मव॑ रुन्धे ब्रह्मवर्च॒सम॑सि ब्रह्मवर्च॒साय॒ त्वेत्या॑ह तेज॒स्वी ब्र॑ह्मवर्च॒सी भ॑वति॒ यस्यै॒ष उ॑पधी॒यते॒ य उ॑ चैनमे॒वं ॅवेद॑ ॥ \newline

\textbf{Pada Paata} \newline

ए॒व । अ॒स्मै॒ । अन्न᳚म् । अवेति॑ । रु॒न्धे॒ । म॒द्ध्य॒तः । उपेति॑ । द॒धा॒ति॒ । म॒द्ध्य॒तः । ए॒व । अ॒स्मै॒ । अन्न᳚म् । द॒धा॒ति॒ । तस्मा᳚त् । म॒द्ध्य॒तः । अन्न᳚म् । अ॒द्य॒ते॒ । बा॒र्.॒ह॒स्प॒त्यः । भ॒व॒ति॒ । ब्रह्म॑ । वै । दे॒वाना᳚म् । बृह॒स्पतिः॑ । ब्रह्म॑णा । ए॒व । अ॒स्मै॒ । अन्न᳚म् । अवेति॑ ।       रु॒न्धे॒ । ब्र॒ह्म॒व॒र्च॒समिति॑ ब्रह्म - व॒र्च॒सम् । अ॒सि॒ । ब्र॒ह्म॒व॒र्च॒सायेति॑ ब्रह्म - व॒र्च॒साय॑ । त्वा॒ । इति॑ । आ॒ह॒ । ते॒ज॒स्वी । ब्र॒ह्म॒व॒र्च॒सीति॑ ब्रह्म - व॒र्च॒सी । भ॒व॒ति॒ । यस्य॑ । ए॒षः । उ॒प॒धी॒यत॒ इत्यु॑प - धी॒यते᳚ । यः । उ॒ । च॒ । ए॒न॒म् । ए॒वम् । वेद॑ ॥  \newline


\textbf{Krama Paata} \newline

ए॒वास्मै᳚ । अ॒स्मा॒ अन्न᳚म् । अन्न॒मव॑ । अव॑ रुन्धे । रु॒न्धे॒ म॒द्ध्य॒तः । म॒द्ध्य॒त उप॑ । उप॑ दधाति । द॒धा॒ति॒ म॒द्ध्य॒तः । म॒द्ध्य॒त ए॒व । ए॒वास्मै᳚ । अ॒स्मा॒ अन्न᳚म् । अन्न॑म् दधाति । द॒धा॒ति॒ तस्मा᳚त् । तस्मा᳚न् मद्ध्य॒तः । म॒द्ध्य॒तोऽन्न᳚म् । अन्न॑मद्यते । अ॒द्य॒ते॒ बा॒र्॒.ह॒स्प॒त्यः । बा॒र्॒.ह॒स्प॒त्यो भ॑वति । भ॒व॒ति॒ ब्रह्म॑ । ब्रह्म॒ वै । वै दे॒वाना᳚म् । दे॒वाना॒म् बृह॒स्पतिः॑ । बृह॒स्पति॒र् ब्रह्म॑णा । ब्रह्म॑णै॒व । ए॒वास्मै᳚ । अ॒स्मा॒ अन्न᳚म् । अन्न॒मव॑ । अव॑ रुन्धे । रु॒न्धे॒ ब्र॒ह्म॒व॒र्च॒सम् । ब्र॒ह्म॒व॒र्च॒सम॑सि । ब्र॒ह्म॒व॒र्च॒समिति॑ ब्रह्म - व॒र्च॒सम् । अ॒सि॒ ब्र॒ह्म॒व॒र्च॒साय॑ । ब्र॒ह्म॒व॒र्च॒साय॑ त्वा । ब्र॒ह्म॒व॒र्च॒सायेति॑ ब्रह्म - व॒र्च॒साय॑ । त्वेति॑ । इत्या॑ह । आ॒ह॒ ते॒ज॒स्वी । ते॒ज॒स्वी ब्र॑ह्मवर्च॒सी । ब्र॒ह्म॒व॒र्च॒सी भ॑वति । ब्र॒ह्म॒व॒र्च॒सीति॑ ब्रह्म - व॒र्च॒सी । भ॒व॒ति॒ यस्य॑ । यस्यै॒षः । ए॒ष उ॑पधी॒यते᳚ । उ॒प॒धी॒यते॒ यः । उ॒प॒धी॒यत॒ इत्यु॑प - धी॒यते᳚ । य उ॑ । उ॒ च॒ । चै॒न॒म् । ए॒न॒मे॒वम् । ए॒वम् ॅवेद॑ । वेदेति॒ वेद॑ । \newline

\textbf{Jatai Paata} \newline

1. ए॒वास्मा॑ अस्मा ए॒वै वास्मै᳚ । \newline
2. अ॒स्मा॒ अन्न॒ मन्न॑ मस्मा अस्मा॒ अन्न᳚म् । \newline
3. अन्न॒ मवा वान्न॒ मन्न॒ मव॑ । \newline
4. अव॑ रुन्धे रु॒न्धे ऽवाव॑ रुन्धे । \newline
5. रु॒न्धे॒ म॒द्ध्य॒तो म॑द्ध्य॒तो रु॑न्धे रुन्धे मद्ध्य॒तः । \newline
6. म॒द्ध्य॒त उपोप॑ मद्ध्य॒तो म॑द्ध्य॒त उप॑ । \newline
7. उप॑ दधाति दधा॒ त्युपोप॑ दधाति । \newline
8. द॒धा॒ति॒ म॒द्ध्य॒तो म॑द्ध्य॒तो द॑धाति दधाति मद्ध्य॒तः । \newline
9. म॒द्ध्य॒त ए॒वैव म॑द्ध्य॒तो म॑द्ध्य॒त ए॒व । \newline
10. ए॒वास्मा॑ अस्मा ए॒वै वास्मै᳚ । \newline
11. अ॒स्मा॒ अन्न॒ मन्न॑ मस्मा अस्मा॒ अन्न᳚म् । \newline
12. अन्न॑म् दधाति दधा॒ त्यन्न॒ मन्न॑म् दधाति । \newline
13. द॒धा॒ति॒ तस्मा॒त् तस्मा᳚द् दधाति दधाति॒ तस्मा᳚त् । \newline
14. तस्मा᳚न् मद्ध्य॒तो म॑द्ध्य॒ तस्तस्मा॒त् तस्मा᳚न् मद्ध्य॒तः । \newline
15. म॒द्ध्य॒तो ऽन्न॒ मन्न॑म् मद्ध्य॒तो म॑द्ध्य॒तो ऽन्न᳚म् । \newline
16. अन्न॑ मद्यते ऽद्य॒ते ऽन्न॒ मन्न॑ मद्यते । \newline
17. अ॒द्य॒ते॒ बा॒र्॒.ह॒स्प॒त्यो बा॑र्.हस्प॒त्यो᳚ ऽद्यते ऽद्यते बार्.हस्प॒त्यः । \newline
18. बा॒र्॒.ह॒स्प॒त्यो भ॑वति भवति बार्.हस्प॒त्यो बा॑र्.हस्प॒त्यो भ॑वति । \newline
19. भ॒व॒ति॒ ब्रह्म॒ ब्रह्म॑ भवति भवति॒ ब्रह्म॑ । \newline
20. ब्रह्म॒ वै वै ब्रह्म॒ ब्रह्म॒ वै । \newline
21. वै दे॒वाना᳚म् दे॒वानां॒ ॅवै वै दे॒वाना᳚म् । \newline
22. दे॒वाना॒म् बृह॒स्पति॒र् बृह॒स्पति॑र् दे॒वाना᳚म् दे॒वाना॒म् बृह॒स्पतिः॑ । \newline
23. बृह॒स्पति॒र् ब्रह्म॑णा॒ ब्रह्म॑णा॒ बृह॒स्पति॒र् बृह॒स्पति॒र् ब्रह्म॑णा । \newline
24. ब्रह्म॑णै॒ वैव ब्रह्म॑णा॒ ब्रह्म॑णै॒व । \newline
25. ए॒वास्मा॑ अस्मा ए॒वै वास्मै᳚ । \newline
26. अ॒स्मा॒ अन्न॒ मन्न॑ मस्मा अस्मा॒ अन्न᳚म् । \newline
27. अन्न॒ मवा वान्न॒ मन्न॒ मव॑ । \newline
28. अव॑ रुन्धे रु॒न्धे ऽवाव॑ रुन्धे । \newline
29. रु॒न्धे॒ ब्र॒ह्म॒व॒र्च॒सम् ब्र॑ह्मवर्च॒सꣳ रु॑न्धे रुन्धे ब्रह्मवर्च॒सम् । \newline
30. ब्र॒ह्म॒व॒र्च॒स म॑स्यसि ब्रह्मवर्च॒सम् ब्र॑ह्मवर्च॒स म॑सि । \newline
31. ब्र॒ह्म॒व॒र्च॒समिति॑ ब्रह्म - व॒र्च॒सम् । \newline
32. अ॒सि॒ ब्र॒ह्म॒व॒र्च॒साय॑ ब्रह्मवर्च॒साया᳚स्यसि ब्रह्मवर्च॒साय॑ । \newline
33. ब्र॒ह्म॒व॒र्च॒साय॑ त्वा त्वा ब्रह्मवर्च॒साय॑ ब्रह्मवर्च॒साय॑ त्वा । \newline
34. ब्र॒ह्म॒व॒र्च॒सायेति॑ ब्रह्म - व॒र्च॒साय॑ । \newline
35. त्वेतीति॑ त्वा॒ त्वेति॑ । \newline
36. इत्या॑हा॒हे तीत्या॑ह । \newline
37. आ॒ह॒ ते॒ज॒स्वी ते॑ज॒ स्व्या॑हाह तेज॒स्वी । \newline
38. ते॒ज॒स्वी ब्र॑ह्मवर्च॒सी ब्र॑ह्मवर्च॒सी ते॑ज॒स्वी ते॑ज॒स्वी ब्र॑ह्मवर्च॒सी । \newline
39. ब्र॒ह्म॒व॒र्च॒सी भ॑वति भवति ब्रह्मवर्च॒सी ब्र॑ह्मवर्च॒सी भ॑वति । \newline
40. ब्र॒ह्म॒व॒र्च॒सीति॑ ब्रह्म - व॒र्च॒सी । \newline
41. भ॒व॒ति॒ यस्य॒ यस्य॑ भवति भवति॒ यस्य॑ । \newline
42. यस्यै॒ष ए॒ष यस्य॒ यस्यै॒षः । \newline
43. ए॒ष उ॑पधी॒यत॑ उपधी॒यत॑ ए॒ष ए॒ष उ॑पधी॒यते᳚ । \newline
44. उ॒प॒धी॒यते॒ यो य उ॑पधी॒यत॑ उपधी॒यते॒ यः । \newline
45. उ॒प॒धी॒यत॒ इत्यु॑प - धी॒यते᳚ । \newline
46. य उ॑ वु॒ यो य उ॑ । \newline
47. उ॒ च॒ च॒ वु॒ च॒ । \newline
48. चै॒न॒ मे॒न॒म् च॒ चै॒न॒म् । \newline
49. ए॒न॒ मे॒व मे॒व मे॑न मेन मे॒वम् । \newline
50. ए॒वं ॅवेद॒ वेदै॒व मे॒वं ॅवेद॑ । \newline
51. वेदेति॒ वेद॑ । \newline

\textbf{Ghana Paata } \newline

1. ए॒वास्मा॑ अस्मा ए॒वै वास्मा॒ अन्न॒ मन्न॑ मस्मा ए॒वै वास्मा॒ अन्न᳚म् । \newline
2. अ॒स्मा॒ अन्न॒ मन्न॑ मस्मा अस्मा॒ अन्न॒ मवावान्न॑ मस्मा अस्मा॒ अन्न॒ मव॑ । \newline
3. अन्न॒ मवावान्न॒ मन्न॒ मव॑ रुन्धे रु॒न्धे ऽवान्न॒ मन्न॒ मव॑ रुन्धे । \newline
4. अव॑ रुन्धे रु॒न्धे ऽवाव॑ रुन्धे मद्ध्य॒तो म॑द्ध्य॒तो रु॒न्धे ऽवाव॑ रुन्धे मद्ध्य॒तः । \newline
5. रु॒न्धे॒ म॒द्ध्य॒तो म॑द्ध्य॒तो रु॑न्धे रुन्धे मद्ध्य॒त उपोप॑ मद्ध्य॒तो रु॑न्धे रुन्धे मद्ध्य॒त उप॑ । \newline
6. म॒द्ध्य॒त उपोप॑ मद्ध्य॒तो म॑द्ध्य॒त उप॑ दधाति दधा॒ त्युप॑ मद्ध्य॒तो म॑द्ध्य॒त उप॑ दधाति । \newline
7. उप॑ दधाति दधा॒ त्युपोप॑ दधाति मद्ध्य॒तो म॑द्ध्य॒तो द॑धा॒ त्युपोप॑ दधाति मद्ध्य॒तः । \newline
8. द॒धा॒ति॒ म॒द्ध्य॒तो म॑द्ध्य॒तो द॑धाति दधाति मद्ध्य॒त ए॒वैव म॑द्ध्य॒तो द॑धाति दधाति मद्ध्य॒त ए॒व । \newline
9. म॒द्ध्य॒त ए॒वैव म॑द्ध्य॒तो म॑द्ध्य॒त ए॒वास्मा॑ अस्मा ए॒व म॑द्ध्य॒तो म॑द्ध्य॒त ए॒वास्मै᳚ । \newline
10. ए॒वास्मा॑ अस्मा ए॒वै वास्मा॒ अन्न॒ मन्न॑ मस्मा ए॒वै वास्मा॒ अन्न᳚म् । \newline
11. अ॒स्मा॒ अन्न॒ मन्न॑ मस्मा अस्मा॒ अन्न॑म् दधाति दधा॒ त्यन्न॑ मस्मा अस्मा॒ अन्न॑म् दधाति । \newline
12. अन्न॑म् दधाति दधा॒ त्यन्न॒ मन्न॑म् दधाति॒ तस्मा॒त् तस्मा᳚द् दधा॒ त्यन्न॒ मन्न॑म् दधाति॒ तस्मा᳚त् । \newline
13. द॒धा॒ति॒ तस्मा॒त् तस्मा᳚द् दधाति दधाति॒ तस्मा᳚न् मद्ध्य॒तो म॑द्ध्य॒त स्तस्मा᳚द् दधाति दधाति॒ तस्मा᳚न् मद्ध्य॒तः । \newline
14. तस्मा᳚न् मद्ध्य॒तो म॑द्ध्य॒त स्तस्मा॒त् तस्मा᳚न् मद्ध्य॒तो ऽन्न॒ मन्न॑म् मद्ध्य॒त स्तस्मा॒त् तस्मा᳚न् मद्ध्य॒तो ऽन्न᳚म् । \newline
15. म॒द्ध्य॒तो ऽन्न॒ मन्न॑म् मद्ध्य॒तो म॑द्ध्य॒तो ऽन्न॑ मद्यते ऽद्य॒ते ऽन्न॑म् मद्ध्य॒तो म॑द्ध्य॒तो ऽन्न॑ मद्यते । \newline
16. अन्न॑ मद्यते ऽद्य॒ते ऽन्न॒ मन्न॑ मद्यते बार्.हस्प॒त्यो बा॑र्.हस्प॒त्यो᳚ ऽद्य॒ते ऽन्न॒ मन्न॑ मद्यते बार्.हस्प॒त्यः । \newline
17. अ॒द्य॒ते॒ बा॒र्॒.ह॒स्प॒त्यो बा॑र्.हस्प॒त्यो᳚ ऽद्यते ऽद्यते बार्.हस्प॒त्यो भ॑वति भवति बार्.हस्प॒त्यो᳚ ऽद्यते ऽद्यते बार्.हस्प॒त्यो भ॑वति । \newline
18. बा॒र्॒.ह॒स्प॒त्यो भ॑वति भवति बार्.हस्प॒त्यो बा॑र्.हस्प॒त्यो भ॑वति॒ ब्रह्म॒ ब्रह्म॑ भवति बार्.हस्प॒त्यो बा॑र्.हस्प॒त्यो भ॑वति॒ ब्रह्म॑ । \newline
19. भ॒व॒ति॒ ब्रह्म॒ ब्रह्म॑ भवति भवति॒ ब्रह्म॒ वै वै ब्रह्म॑ भवति भवति॒ ब्रह्म॒ वै । \newline
20. ब्रह्म॒ वै वै ब्रह्म॒ ब्रह्म॒ वै दे॒वाना᳚म् दे॒वानां॒ ॅवै ब्रह्म॒ ब्रह्म॒ वै दे॒वाना᳚म् । \newline
21. वै दे॒वाना᳚म् दे॒वानां॒ ॅवै वै दे॒वाना॒म् बृह॒स्पति॒र् बृह॒स्पति॑र् दे॒वानां॒ ॅवै वै दे॒वाना॒म् बृह॒स्पतिः॑ । \newline
22. दे॒वाना॒म् बृह॒स्पति॒र् बृह॒स्पति॑र् दे॒वाना᳚म् दे॒वाना॒म् बृह॒स्पति॒र् ब्रह्म॑णा॒ ब्रह्म॑णा॒ बृह॒स्पति॑र् दे॒वाना᳚म् दे॒वाना॒म् बृह॒स्पति॒र् ब्रह्म॑णा । \newline
23. बृह॒स्पति॒र् ब्रह्म॑णा॒ ब्रह्म॑णा॒ बृह॒स्पति॒र् बृह॒स्पति॒र् ब्रह्म॑णै॒वैव ब्रह्म॑णा॒ बृह॒स्पति॒र् बृह॒स्पति॒र् ब्रह्म॑णै॒व । \newline
24. ब्रह्म॑णै॒ वैव ब्रह्म॑णा॒ ब्रह्म॑णै॒ वास्मा॑ अस्मा ए॒व ब्रह्म॑णा॒ ब्रह्म॑णै॒ वास्मै᳚ । \newline
25. ए॒वास्मा॑ अस्मा ए॒वै वास्मा॒ अन्न॒ मन्न॑ मस्मा ए॒वै वास्मा॒ अन्न᳚म् । \newline
26. अ॒स्मा॒ अन्न॒ मन्न॑ मस्मा अस्मा॒ अन्न॒ मवावान्न॑ मस्मा अस्मा॒ अन्न॒ मव॑ । \newline
27. अन्न॒ मवावान्न॒ मन्न॒ मव॑ रुन्धे रु॒न्धे ऽवान्न॒ मन्न॒ मव॑ रुन्धे । \newline
28. अव॑ रुन्धे रु॒न्धे ऽवाव॑ रुन्धे ब्रह्मवर्च॒सम् ब्र॑ह्मवर्च॒सꣳ रु॒न्धे ऽवाव॑ रुन्धे ब्रह्मवर्च॒सम् । \newline
29. रु॒न्धे॒ ब्र॒ह्म॒व॒र्च॒सम् ब्र॑ह्मवर्च॒सꣳ रु॑न्धे रुन्धे ब्रह्मवर्च॒स म॑स्यसि ब्रह्मवर्च॒सꣳ रु॑न्धे रुन्धे ब्रह्मवर्च॒स म॑सि । \newline
30. ब्र॒ह्म॒व॒र्च॒स म॑स्यसि ब्रह्मवर्च॒सम् ब्र॑ह्मवर्च॒स म॑सि ब्रह्मवर्च॒साय॑ ब्रह्मवर्च॒सा या॑सि ब्रह्मवर्च॒सम् ब्र॑ह्मवर्च॒स म॑सि ब्रह्मवर्च॒साय॑ । \newline
31. ब्र॒ह्म॒व॒र्च॒समिति॑ ब्रह्म - व॒र्च॒सम् । \newline
32. अ॒सि॒ ब्र॒ह्म॒व॒र्च॒साय॑ ब्रह्मवर्च॒साया᳚स्यसि ब्रह्मवर्च॒साय॑ त्वा त्वा ब्रह्मवर्च॒साया᳚स्यसि ब्रह्मवर्च॒साय॑ त्वा । \newline
33. ब्र॒ह्म॒व॒र्च॒साय॑ त्वा त्वा ब्रह्मवर्च॒साय॑ ब्रह्मवर्च॒साय॒ त्वेतीति॑ त्वा ब्रह्मवर्च॒साय॑ ब्रह्मवर्च॒साय॒ त्वेति॑ । \newline
34. ब्र॒ह्म॒व॒र्च॒सायेति॑ ब्रह्म - व॒र्च॒साय॑ । \newline
35. त्वेतीति॑ त्वा॒ त्वेत्या॑ हा॒हेति॑ त्वा॒ त्वेत्या॑ह । \newline
36. इत्या॑हा॒हे तीत्या॑ह तेज॒स्वी ते॑ज॒ स्व्या॑हे तीत्या॑ह तेज॒स्वी । \newline
37. आ॒ह॒ ते॒ज॒स्वी ते॑ज॒ स्व्या॑हाह तेज॒स्वी ब्र॑ह्मवर्च॒सी ब्र॑ह्मवर्च॒सी ते॑ज॒ स्व्या॑हाह तेज॒स्वी ब्र॑ह्मवर्च॒सी । \newline
38. ते॒ज॒स्वी ब्र॑ह्मवर्च॒सी ब्र॑ह्मवर्च॒सी ते॑ज॒स्वी ते॑ज॒स्वी ब्र॑ह्मवर्च॒सी भ॑वति भवति ब्रह्मवर्च॒सी ते॑ज॒स्वी ते॑ज॒स्वी ब्र॑ह्मवर्च॒सी भ॑वति । \newline
39. ब्र॒ह्म॒व॒र्च॒सी भ॑वति भवति ब्रह्मवर्च॒सी ब्र॑ह्मवर्च॒सी भ॑वति॒ यस्य॒ यस्य॑ भवति ब्रह्मवर्च॒सी ब्र॑ह्मवर्च॒सी भ॑वति॒ यस्य॑ । \newline
40. ब्र॒ह्म॒व॒र्च॒सीति॑ ब्रह्म - व॒र्च॒सी । \newline
41. भ॒व॒ति॒ यस्य॒ यस्य॑ भवति भवति॒ यस्यै॒ष ए॒ष यस्य॑ भवति भवति॒ यस्यै॒षः । \newline
42. यस्यै॒ष ए॒ष यस्य॒ यस्यै॒ष उ॑पधी॒यत॑ उपधी॒यत॑ ए॒ष यस्य॒ यस्यै॒ष उ॑पधी॒यते᳚ । \newline
43. ए॒ष उ॑पधी॒यत॑ उपधी॒यत॑ ए॒ष ए॒ष उ॑पधी॒यते॒ यो य उ॑पधी॒यत॑ ए॒ष ए॒ष उ॑पधी॒यते॒ यः । \newline
44. उ॒प॒धी॒यते॒ यो य उ॑पधी॒यत॑ उपधी॒यते॒ य उ॑ वु॒ य उ॑पधी॒यत॑ उपधी॒यते॒ य उ॑ । \newline
45. उ॒प॒धी॒यत॒ इत्यु॑प - धी॒यते᳚ । \newline
46. य उ॑ वु॒ यो य उ॑ च चो॒ यो य उ॑ च । \newline
47. उ॒ च॒ च॒ वु॒ चै॒न॒ मे॒न॒म् च॒ वु॒ चै॒न॒म् । \newline
48. चै॒न॒ मे॒न॒म् च॒ चै॒न॒ मे॒व मे॒व मे॑नम् च चैन मे॒वम् । \newline
49. ए॒न॒ मे॒व मे॒व मे॑न मेन मे॒वं ॅवेद॒ वेदै॒व मे॑न मेन मे॒वं ॅवेद॑ । \newline
50. ए॒वं ॅवेद॒ वेदै॒व मे॒वं ॅवेद॑ । \newline
51. वेदेति॒ वेद॑ । \newline
\pagebreak
\markright{ TS 5.6.3.1  \hfill https://www.vedavms.in \hfill}

\section{ TS 5.6.3.1 }

\textbf{TS 5.6.3.1 } \newline
\textbf{Samhita Paata} \newline

भू॒ते॒ष्ट॒का उप॑ दधा॒त्यत्रा᳚त्र॒ वै मृ॒त्युर्जा॑यते॒ यत्र॑यत्रै॒व मृ॒त्युर्जाय॑ते॒ तत॑ ए॒वैन॒मव॑ यजते॒ तस्मा॑दग्नि॒चिथ् सर्व॒मायु॑रेति॒ सर्वे॒ ह्य॑स्य मृ॒त्यवो ऽवे᳚ष्टा॒स्तस्मा॑-दग्नि॒चिन्ना-भिच॑रित॒वै प्र॒त्यगे॑न-मभिचा॒रः स्तृ॑णुते सू॒यते॒ वा ए॒ष यो᳚ऽग्निं चि॑नु॒ते दे॑वसु॒वामे॒तानि॑ ह॒वीꣳषि॑ भवन्त्ये॒ताव॑न्तो॒ वै दे॒वानाꣳ॑ स॒वास्त ए॒वा - [  ] \newline

\textbf{Pada Paata} \newline

भू॒ते॒ष्ट॒का इति॑ भूत - इ॒ष्ट॒काः । उपेति॑ । द॒धा॒ति॒ । अत्रा॒त्रेत्यत्र॑-अ॒त्र॒ । वै । मृ॒त्युः । जा॒य॒ते॒ । यत्र॑य॒त्रेति॒ यत्र॑-य॒त्र॒ । ए॒व । मृ॒त्युः । जाय॑ते । ततः॑ । ए॒व । ए॒न॒म् । अवेति॑ । य॒ज॒ते॒ । तस्मा᳚त् । अ॒ग्नि॒चिदित्य॑ग्नि - चित् । सर्व᳚म् । आयुः॑ । ए॒ति॒ । सर्वे᳚ । हि । अ॒स्य॒ । मृ॒त्यवः॑ । अवे᳚ष्टा॒ इत्यव॑ - इ॒ष्टाः॒ । तस्मा᳚त् । अ॒ग्नि॒चिदित्य॑ग्नि - चित् । न । अ॒भिच॑रित॒वा इत्य॒भि-च॒रि॒त॒वै । प्र॒त्यक् । ए॒न॒म् । अ॒भि॒चा॒र इत्य॑भि - चा॒रः । स्तृ॒णु॒ते॒ । सू॒यते᳚ । वै । ए॒षः । यः । अ॒ग्निम् । चि॒नु॒ते । दे॒व॒सु॒वामिति॑ देव - सु॒वाम् । ए॒तानि॑ । ह॒वीꣳषि॑ । भ॒व॒न्ति॒ । ए॒ताव॑न्तः । वै । दे॒वाना᳚म् । स॒वाः । ते । ए॒व ।  \newline


\textbf{Krama Paata} \newline

भू॒ते॒ष्ट॒का उप॑ । भू॒ते॒ष्ट॒का इति॑ भूत - इ॒ष्ट॒काः । उप॑ दधाति । द॒धा॒त्यत्रा᳚त्र । अत्रा᳚त्र॒ वै । अत्रा॒त्रेत्यत्र॑ - अ॒त्र॒ । वै मृ॒त्युः । मृ॒त्युर् जा॑यते । जा॒य॒ते॒ यत्र॑यत्र । यत्र॑यत्रै॒व । यत्र॑य॒त्रेति॒ यत्र॑ - य॒त्र॒ । ए॒व मृ॒त्युः । मृ॒त्युर् जाय॑ते । जाय॑ते॒ ततः॑ । तत॑ ए॒व । ए॒वैन᳚म् । ए॒न॒मव॑ । अव॑ यजते । य॒ज॒ते॒ तस्मा᳚त् । तस्मा॑दग्नि॒चित् । अ॒ग्नि॒चिथ् सर्व᳚म् । अ॒ग्नि॒चिदित्य॑ग्नि - चित् । सर्व॒मायुः॑ । आयु॑रेति । ए॒ति॒ सर्वे᳚ । सर्वे॒ हि । ह्य॑स्य । अ॒स्य॒ मृ॒त्यवः॑ । मृ॒त्यवोऽवे᳚ष्टाः । अवे᳚ष्टा॒स्तस्मा᳚त् । अवे᳚ष्टा॒ इत्यव॑ - इ॒ष्टाः॒ । तस्मा॑दग्नि॒चित् । अ॒ग्नि॒चिन् न । अ॒ग्नि॒चिदित्य॑ग्नि - चित् । नाभिच॑रित॒वै । अ॒भिच॑रित॒वै प्र॒त्यक् । अ॒भिच॑रित॒वा इत्य॒भि - च॒रि॒त॒वै । प्र॒त्यगे॑नम् । ए॒न॒म॒भि॒चा॒रः । अ॒भि॒चा॒र स्तृ॑णुते । अ॒भि॒चा॒र इत्य॑भि - चा॒रः । स्तृ॒णु॒ते॒ सू॒यते᳚ । सू॒यते॒ वै । वा ए॒षः । 
ए॒ष यः । यो᳚ऽग्निम् । अ॒ग्निम् चि॑नु॒ते । चि॒नु॒ते दे॑वसु॒वाम् । दे॒व॒सु॒वामे॒तानि॑ । दे॒व॒सु॒वामिति॑ देव - सु॒वाम् । ए॒तानि॑ ह॒वीꣳषि॑ । ह॒वीꣳषि॑ भवन्ति । भ॒व॒न्त्ये॒ताव॑न्तः । 
ए॒ताव॑न्तो॒ वै । वै दे॒वाना᳚म् । दे॒वानाꣳ॑ स॒वाः । स॒वास्ते । 
त ए॒व । ए॒वास्मै᳚ \newline

\textbf{Jatai Paata} \newline

1. भू॒ते॒ष्ट॒का उपोप॑ भूतेष्ट॒का भू॑तेष्ट॒का उप॑ । \newline
2. भू॒ते॒ष्ट॒का इति॑ भूत - इ॒ष्ट॒काः । \newline
3. उप॑ दधाति दधा॒ त्युपोप॑ दधाति । \newline
4. द॒धा॒ त्यत्रा॒त्रा त्रा᳚त्र दधाति दधा॒ त्यत्रा᳚त्र । \newline
5. अत्रा᳚त्र॒ वै वा अत्रा॒त्रा त्रा᳚त्र॒ वै । \newline
6. अत्रा॒त्रेत्यत्र॑ - अ॒त्र॒ । \newline
7. वै मृ॒त्युर् मृ॒त्युर् वै वै मृ॒त्युः । \newline
8. मृ॒त्युर् जा॑यते जायते मृ॒त्युर् मृ॒त्युर् जा॑यते । \newline
9. जा॒य॒ते॒ यत्र॑यत्र॒ यत्र॑यत्र जायते जायते॒ यत्र॑यत्र । \newline
10. यत्र॑यत्रै॒ वैव यत्र॑यत्र॒ यत्र॑यत्रै॒व । \newline
11. यत्र॑य॒त्रेति॒ यत्र॑ - य॒त्र॒ । \newline
12. ए॒व मृ॒त्युर् मृ॒त्यु रे॒वैव मृ॒त्युः । \newline
13. मृ॒त्युर् जाय॑ते॒ जाय॑ते मृ॒त्युर् मृ॒त्युर् जाय॑ते । \newline
14. जाय॑ते॒ तत॒ स्ततो॒ जाय॑ते॒ जाय॑ते॒ ततः॑ । \newline
15. तत॑ ए॒वैव तत॒ स्तत॑ ए॒व । \newline
16. ए॒वैन॑ मेन मे॒वै वैन᳚म् । \newline
17. ए॒न॒ मवा वै॑न मेन॒ मव॑ । \newline
18. अव॑ यजते यज॒ते ऽवाव॑ यजते । \newline
19. य॒ज॒ते॒ तस्मा॒त् तस्मा᳚द् यजते यजते॒ तस्मा᳚त् । \newline
20. तस्मा॑ दग्नि॒चि द॑ग्नि॒चित् तस्मा॒त् तस्मा॑ दग्नि॒चित् । \newline
21. अ॒ग्नि॒चिथ् सर्वꣳ॒॒ सर्व॑ मग्नि॒चि द॑ग्नि॒चिथ् सर्व᳚म् । \newline
22. अ॒ग्नि॒चिदित्य॑ग्नि - चित् । \newline
23. सर्व॒ मायु॒ रायुः॒ सर्वꣳ॒॒ सर्व॒ मायुः॑ । \newline
24. आयु॑ रेत्ये॒ त्यायु॒ रायु॑ रेति । \newline
25. ए॒ति॒ सर्वे॒ सर्व॑ एत्येति॒ सर्वे᳚ । \newline
26. सर्वे॒ हि हि सर्वे॒ सर्वे॒ हि । \newline
27. ह्य॑स्यास्य॒ हि ह्य॑स्य । \newline
28. अ॒स्य॒ मृ॒त्यवो॑ मृ॒त्यवो᳚ ऽस्यास्य मृ॒त्यवः॑ । \newline
29. मृ॒त्यवो ऽवे᳚ष्टा॒ अवे᳚ष्टा मृ॒त्यवो॑ मृ॒त्यवो ऽवे᳚ष्टाः । \newline
30. अवे᳚ष्टा॒ स्तस्मा॒त् तस्मा॒ दवे᳚ष्टा॒ अवे᳚ष्टा॒ स्तस्मा᳚त् । \newline
31. अवे᳚ष्टा॒ इत्यव॑ - इ॒ष्टाः॒ । \newline
32. तस्मा॑ दग्नि॒चि द॑ग्नि॒चित् तस्मा॒त् तस्मा॑ दग्नि॒चित् । \newline
33. अ॒ग्नि॒चिन् न नाग्नि॒चि द॑ग्नि॒चिन् न । \newline
34. अ॒ग्नि॒चिदित्य॑ग्नि - चित् । \newline
35. नाभिच॑रित॒वा अ॒भिच॑रित॒वै न नाभिच॑रित॒वै । \newline
36. अ॒भिच॑रित॒वै प्र॒त्यक् प्र॒त्यग॒ भिच॑रित॒वा अ॒भिच॑रित॒वै प्र॒त्यक् । \newline
37. अ॒भिच॑रित॒वा इत्य॒भि - च॒रि॒त॒वै । \newline
38. प्र॒त्य गे॑न मेनम् प्र॒त्यक् प्र॒त्य गे॑नम् । \newline
39. ए॒न॒ म॒भि॒चा॒रो॑ ऽभिचा॒र ए॑न मेन मभिचा॒रः । \newline
40. अ॒भि॒चा॒रः स्तृ॑णुते स्तृणुते ऽभिचा॒रो॑ ऽभिचा॒रः स्तृ॑णुते । \newline
41. अ॒भि॒चा॒र इत्य॑भि - चा॒रः । \newline
42. स्तृ॒णु॒ते॒ सू॒यते॑ सू॒यते᳚ स्तृणुते स्तृणुते सू॒यते᳚ । \newline
43. सू॒यते॒ वै वै सू॒यते॑ सू॒यते॒ वै । \newline
44. वा ए॒ष ए॒ष वै वा ए॒षः । \newline
45. ए॒ष यो य ए॒ष ए॒ष यः । \newline
46. यो᳚ ऽग्नि म॒ग्निं ॅयो यो᳚ ऽग्निम् । \newline
47. अ॒ग्निम् चि॑नु॒ते चि॑नु॒ते᳚ ऽग्नि म॒ग्निम् चि॑नु॒ते । \newline
48. चि॒नु॒ते दे॑वसु॒वाम् दे॑वसु॒वाम् चि॑नु॒ते चि॑नु॒ते दे॑वसु॒वाम् । \newline
49. दे॒व॒सु॒वा मे॒ता न्ये॒तानि॑ देवसु॒वाम् दे॑वसु॒वा मे॒तानि॑ । \newline
50. दे॒व॒सु॒वामिति॑ देव - सु॒वाम् । \newline
51. ए॒तानि॑ ह॒वीꣳषि॑ ह॒वी ꣳष्ये॒ता न्ये॒तानि॑ ह॒वीꣳषि॑ । \newline
52. ह॒वीꣳषि॑ भवन्ति भवन्ति ह॒वीꣳषि॑ ह॒वीꣳषि॑ भवन्ति । \newline
53. भ॒व॒न् त्ये॒ताव॑न्त ए॒ताव॑न्तो भवन्ति भवन् त्ये॒ताव॑न्तः । \newline
54. ए॒ताव॑न्तो॒ वै वा ए॒ताव॑न्त ए॒ताव॑न्तो॒ वै । \newline
55. वै दे॒वाना᳚म् दे॒वानां॒ ॅवै वै दे॒वाना᳚म् । \newline
56. दे॒वानाꣳ॑ स॒वाः स॒वा दे॒वाना᳚म् दे॒वानाꣳ॑ स॒वाः । \newline
57. स॒वा स्ते ते स॒वाः स॒वा स्ते । \newline
58. त ए॒वैव ते त ए॒व । \newline
59. ए॒वास्मा॑ अस्मा ए॒वै वास्मै᳚ । \newline

\textbf{Ghana Paata } \newline

1. भू॒ते॒ष्ट॒का उपोप॑ भूतेष्ट॒का भू॑तेष्ट॒का उप॑ दधाति दधा॒ त्युप॑ भूतेष्ट॒का भू॑तेष्ट॒का उप॑ दधाति । \newline
2. भू॒ते॒ष्ट॒का इति॑ भूत - इ॒ष्ट॒काः । \newline
3. उप॑ दधाति दधा॒ त्युपोप॑ दधा॒ त्यत्रा॒ त्रात्रा᳚त्र दधा॒ त्युपोप॑ दधा॒ त्यत्रा᳚त्र । \newline
4. द॒धा॒ त्यत्रा॒ त्रात्रा᳚त्र दधाति दधा॒ त्यत्रा᳚त्र॒ वै वा अत्रा᳚त्र दधाति दधा॒ त्यत्रा᳚त्र॒ वै । \newline
5. अत्रा᳚त्र॒ वै वा अत्रा॒त्रा त्रा᳚त्र॒ वै मृ॒त्युर् मृ॒त्युर् वा अत्रा॒त्रा त्रा᳚त्र॒ वै मृ॒त्युः । \newline
6. अत्रा॒त्रेत्यत्र॑ - अ॒त्र॒ । \newline
7. वै मृ॒त्युर् मृ॒त्युर् वै वै मृ॒त्युर् जा॑यते जायते मृ॒त्युर् वै वै मृ॒त्युर् जा॑यते । \newline
8. मृ॒त्युर् जा॑यते जायते मृ॒त्युर् मृ॒त्युर् जा॑यते॒ यत्र॑यत्र॒ यत्र॑यत्र जायते मृ॒त्युर् मृ॒त्युर् जा॑यते॒ यत्र॑यत्र । \newline
9. जा॒य॒ते॒ यत्र॑यत्र॒ यत्र॑यत्र जायते जायते॒ यत्र॑यत्रै॒वैव यत्र॑यत्र जायते जायते॒ यत्र॑यत्रै॒व । \newline
10. यत्र॑य त्रै॒वैव यत्र॑यत्र॒ यत्र॑यत्रै॒व मृ॒त्युर् मृ॒त्यु रे॒व यत्र॑यत्र॒ यत्र॑यत्रै॒व मृ॒त्युः । \newline
11. यत्र॑य॒त्रेति॒ यत्र॑ - य॒त्र॒ । \newline
12. ए॒व मृ॒त्युर् मृ॒त्यु रे॒वैव मृ॒त्युर् जाय॑ते॒ जाय॑ते मृ॒त्यु रे॒वैव मृ॒त्युर् जाय॑ते । \newline
13. मृ॒त्युर् जाय॑ते॒ जाय॑ते मृ॒त्युर् मृ॒त्युर् जाय॑ते॒ तत॒ स्ततो॒ जाय॑ते मृ॒त्युर् मृ॒त्युर् जाय॑ते॒ ततः॑ । \newline
14. जाय॑ते॒ तत॒ स्ततो॒ जाय॑ते॒ जाय॑ते॒ तत॑ ए॒वैव ततो॒ जाय॑ते॒ जाय॑ते॒ तत॑ ए॒व । \newline
15. तत॑ ए॒वैव तत॒ स्तत॑ ए॒वैन॑ मेन मे॒व तत॒ स्तत॑ ए॒वैन᳚म् । \newline
16. ए॒वैन॑ मेन मे॒वै वैन॒ मवावै॑न मे॒वै वैन॒ मव॑ । \newline
17. ए॒न॒ मवा वै॑न मेन॒ मव॑ यजते यज॒ते ऽवै॑न मेन॒ मव॑ यजते । \newline
18. अव॑ यजते यज॒ते ऽवाव॑ यजते॒ तस्मा॒त् तस्मा᳚द् यज॒ते ऽवाव॑ यजते॒ तस्मा᳚त् । \newline
19. य॒ज॒ते॒ तस्मा॒त् तस्मा᳚द् यजते यजते॒ तस्मा॑ दग्नि॒चि द॑ग्नि॒चित् तस्मा᳚द् यजते यजते॒ तस्मा॑ दग्नि॒चित् । \newline
20. तस्मा॑ दग्नि॒चि द॑ग्नि॒चित् तस्मा॒त् तस्मा॑ दग्नि॒चिथ् सर्वꣳ॒॒ सर्व॑ मग्नि॒चित् तस्मा॒त् तस्मा॑ दग्नि॒चिथ् सर्व᳚म् । \newline
21. अ॒ग्नि॒चिथ् सर्वꣳ॒॒ सर्व॑ मग्नि॒चि द॑ग्नि॒चिथ् सर्व॒ मायु॒ रायुः॒ सर्व॑ मग्नि॒चि द॑ग्नि॒चिथ् सर्व॒ मायुः॑ । \newline
22. अ॒ग्नि॒चिदित्य॑ग्नि - चित् । \newline
23. सर्व॒ मायु॒ रायुः॒ सर्वꣳ॒॒ सर्व॒ मायु॑ रेत्ये॒ त्यायुः॒ सर्वꣳ॒॒ सर्व॒ मायु॑रेति । \newline
24. आयु॑ रेत्ये॒ त्यायु॒ रायु॑ रेति॒ सर्वे॒ सर्व॑ ए॒त्यायु॒ रायु॑ रेति॒ सर्वे᳚ । \newline
25. ए॒ति॒ सर्वे॒ सर्व॑ एत्येति॒ सर्वे॒ हि हि सर्व॑ एत्येति॒ सर्वे॒ हि । \newline
26. सर्वे॒ हि हि सर्वे॒ सर्वे॒ ह्य॑ स्यास्य॒ हि सर्वे॒ सर्वे॒ ह्य॑स्य । \newline
27. ह्य॑ स्यास्य॒ हि ह्य॑स्य मृ॒त्यवो॑ मृ॒त्यवो᳚ ऽस्य॒ हि ह्य॑स्य मृ॒त्यवः॑ । \newline
28. अ॒स्य॒ मृ॒त्यवो॑ मृ॒त्यवो᳚ ऽस्यास्य मृ॒त्यवो ऽवे᳚ष्टा॒ अवे᳚ष्टा मृ॒त्यवो᳚ ऽस्यास्य मृ॒त्यवो ऽवे᳚ष्टाः । \newline
29. मृ॒त्यवो ऽवे᳚ष्टा॒ अवे᳚ष्टा मृ॒त्यवो॑ मृ॒त्यवो ऽवे᳚ष्टा॒ स्तस्मा॒त् तस्मा॒ दवे᳚ष्टा मृ॒त्यवो॑ मृ॒त्यवो ऽवे᳚ष्टा॒ स्तस्मा᳚त् । \newline
30. अवे᳚ष्टा॒ स्तस्मा॒त् तस्मा॒ दवे᳚ष्टा॒ अवे᳚ष्टा॒ स्तस्मा॑ दग्नि॒चि द॑ग्नि॒चित् तस्मा॒ दवे᳚ष्टा॒ अवे᳚ष्टा॒ स्तस्मा॑ दग्नि॒चित् । \newline
31. अवे᳚ष्टा॒ इत्यव॑ - इ॒ष्टाः॒ । \newline
32. तस्मा॑ दग्नि॒चि द॑ग्नि॒चित् तस्मा॒त् तस्मा॑ दग्नि॒चिन् न नाग्नि॒चित् तस्मा॒त् तस्मा॑ दग्नि॒चिन् न । \newline
33. अ॒ग्नि॒चिन् न नाग्नि॒चि द॑ग्नि॒चिन् नाभिच॑रित॒वा अ॒भिच॑रित॒वै नाग्नि॒चि द॑ग्नि॒चिन् नाभिच॑रित॒वै । \newline
34. अ॒ग्नि॒चिदित्य॑ग्नि - चित् । \newline
35. नाभिच॑रित॒वा अ॒भिच॑रित॒वै न नाभिच॑रित॒वै प्र॒त्यक् प्र॒त्य ग॒भिच॑रित॒वै न नाभिच॑रित॒वै प्र॒त्यक् । \newline
36. अ॒भिच॑रित॒वै प्र॒त्यक् प्र॒त्य ग॒भिच॑रित॒वा अ॒भिच॑रित॒वै प्र॒त्यगे॑न मेनम् प्र॒त्य ग॒भिच॑रित॒वा अ॒भिच॑रित॒वै प्र॒त्यगे॑नम् । \newline
37. अ॒भिच॑रित॒वा इत्य॒भि - च॒रि॒त॒वै । \newline
38. प्र॒त्यगे॑न मेनम् प्र॒त्यक् प्र॒त्यगे॑न मभिचा॒रो॑ ऽभिचा॒र ए॑नम् प्र॒त्यक् प्र॒त्यगे॑न मभिचा॒रः । \newline
39. ए॒न॒ म॒भि॒चा॒रो॑ ऽभिचा॒र ए॑न मेन मभिचा॒रः स्तृ॑णुते स्तृणुते ऽभिचा॒र ए॑न मेन मभिचा॒रः स्तृ॑णुते । \newline
40. अ॒भि॒चा॒रः स्तृ॑णुते स्तृणुते ऽभिचा॒रो॑ ऽभिचा॒रः स्तृ॑णुते सू॒यते॑ सू॒यते᳚ स्तृणुते ऽभिचा॒रो॑ ऽभिचा॒रः स्तृ॑णुते सू॒यते᳚ । \newline
41. अ॒भि॒चा॒र इत्य॑भि - चा॒रः । \newline
42. स्तृ॒णु॒ते॒ सू॒यते॑ सू॒यते᳚ स्तृणुते स्तृणुते सू॒यते॒ वै वै सू॒यते᳚ स्तृणुते स्तृणुते सू॒यते॒ वै । \newline
43. सू॒यते॒ वै वै सू॒यते॑ सू॒यते॒ वा ए॒ष ए॒ष वै सू॒यते॑ सू॒यते॒ वा ए॒षः । \newline
44. वा ए॒ष ए॒ष वै वा ए॒ष यो य ए॒ष वै वा ए॒ष यः । \newline
45. ए॒ष यो य ए॒ष ए॒ष यो᳚ ऽग्नि म॒ग्निं ॅय ए॒ष ए॒ष यो᳚ ऽग्निम् । \newline
46. यो᳚ ऽग्नि म॒ग्निं ॅयो यो᳚ ऽग्निम् चि॑नु॒ते चि॑नु॒ते᳚ ऽग्निं ॅयो यो᳚ ऽग्निम् चि॑नु॒ते । \newline
47. अ॒ग्निम् चि॑नु॒ते चि॑नु॒ते᳚ ऽग्नि म॒ग्निम् चि॑नु॒ते दे॑वसु॒वाम् दे॑वसु॒वाम् चि॑नु॒ते᳚ ऽग्नि म॒ग्निम् चि॑नु॒ते दे॑वसु॒वाम् । \newline
48. चि॒नु॒ते दे॑वसु॒वाम् दे॑वसु॒वाम् चि॑नु॒ते चि॑नु॒ते दे॑वसु॒वा मे॒ता न्ये॒तानि॑ देवसु॒वाम् चि॑नु॒ते चि॑नु॒ते दे॑वसु॒वा मे॒तानि॑ । \newline
49. दे॒व॒सु॒वा मे॒ता न्ये॒तानि॑ देवसु॒वाम् दे॑वसु॒वा मे॒तानि॑ ह॒वीꣳषि॑ ह॒वीꣳ ष्ये॒तानि॑ देवसु॒वाम् दे॑वसु॒वा मे॒तानि॑ ह॒वीꣳषि॑ । \newline
50. दे॒व॒सु॒वामिति॑ देव - सु॒वाम् । \newline
51. ए॒तानि॑ ह॒वीꣳषि॑ ह॒वीꣳ ष्ये॒ता न्ये॒तानि॑ ह॒वीꣳषि॑ भवन्ति भवन्ति ह॒वीꣳ ष्ये॒ता न्ये॒तानि॑ ह॒वीꣳषि॑ भवन्ति । \newline
52. ह॒वीꣳषि॑ भवन्ति भवन्ति ह॒वीꣳषि॑ ह॒वीꣳषि॑ भवन् त्ये॒ताव॑न्त ए॒ताव॑न्तो भवन्ति ह॒वीꣳषि॑ ह॒वीꣳषि॑ भवन् त्ये॒ताव॑न्तः । \newline
53. भ॒व॒न् त्ये॒ताव॑न्त ए॒ताव॑न्तो भवन्ति भवन् त्ये॒ताव॑न्तो॒ वै वा ए॒ताव॑न्तो भवन्ति भवन् त्ये॒ताव॑न्तो॒ वै । \newline
54. ए॒ताव॑न्तो॒ वै वा ए॒ताव॑न्त ए॒ताव॑न्तो॒ वै दे॒वाना᳚म् दे॒वानां॒ ॅवा ए॒ताव॑न्त ए॒ताव॑न्तो॒ वै दे॒वाना᳚म् । \newline
55. वै दे॒वाना᳚म् दे॒वानां॒ ॅवै वै दे॒वानाꣳ॑ स॒वाः स॒वा दे॒वानां॒ ॅवै वै दे॒वानाꣳ॑ स॒वाः । \newline
56. दे॒वानाꣳ॑ स॒वाः स॒वा दे॒वाना᳚म् दे॒वानाꣳ॑ स॒वा स्ते ते स॒वा दे॒वाना᳚म् दे॒वानाꣳ॑ स॒वा स्ते । \newline
57. स॒वा स्ते ते स॒वाः स॒वा स्त ए॒वैव ते स॒वाः स॒वा स्त ए॒व । \newline
58. त ए॒वैव ते त ए॒वास्मा॑ अस्मा ए॒व ते त ए॒वास्मै᳚ । \newline
59. ए॒वास्मा॑ अस्मा ए॒वै वास्मै॑ स॒वान् थ्स॒वा न॑स्मा ए॒वै वास्मै॑ स॒वान् । \newline
\pagebreak
\markright{ TS 5.6.3.2  \hfill https://www.vedavms.in \hfill}

\section{ TS 5.6.3.2 }

\textbf{TS 5.6.3.2 } \newline
\textbf{Samhita Paata} \newline

-स्मै॑ स॒वान् प्र य॑च्छन्ति॒ त ए॑नꣳ सुवन्ते स॒वो᳚ऽग्निर्व॑रुणस॒वो रा॑ज॒सूयं॑ ब्रह्मस॒वश्चित्यो॑ दे॒वस्य॑ त्वा सवि॒तुः प्र॑स॒व इत्या॑ह सवि॒तृप्र॑सूत ए॒वैनं॒ ब्रह्म॑णा दे॒वता॑भिर॒भि षि॑ञ्च॒त्यन्न॑-स्यान्नस्या॒भि षि॑ञ्च॒त्यन्न॑-स्यान्न॒स्या-व॑रुद्ध्यै पु॒रस्ता᳚त् प्र॒त्यञ्च॑म॒भि षि॑ञ्चति पु॒रस्ता॒द्धि प्र॑ती॒ची-न॒मन्न॑म॒द्यते॑ शीर्.ष॒तो॑ऽभि षि॑ञ्चति शीर्.ष॒तो ह्यन्न॑म॒द्यत॒ आ मुखा॑द॒न्वव॑स्रावयति - [  ] \newline

\textbf{Pada Paata} \newline

अ॒स्मै॒ । स॒वान् । प्रेति॑ । य॒च्छ॒न्ति॒ । ते । ए॒न॒म् । सु॒व॒न्ते॒ । स॒वः । अ॒ग्निः । व॒रु॒ण॒स॒व इति॑ वरुण - स॒वः । रा॒ज॒सूय॒मिति॑ राज - सूय᳚म् । ब्र॒ह्म॒स॒व इति॑ ब्रह्म-स॒वः । चित्यः॑ । दे॒वस्य॑ । त्वा॒ । स॒वि॒तुः । प्र॒स॒व इति॑ प्र - स॒वे । इति॑ । आ॒ह॒ । स॒वि॒तृप्र॑सूत॒ इति॑ सवि॒तृ-प्र॒सू॒तः॒ । ए॒व । ए॒न॒म् । ब्रह्म॑णा । दे॒वता॑भिः । अ॒भीति॑ । सि॒ञ्च॒ति॒ । अन्न॑स्यान्न॒स्येत्यन्न॑स्य - अ॒न्न॒स्य॒ । अ॒भीति॑ । सि॒ञ्च॒ति॒ । अन्न॑स्यान्न॒स्येत्यन्न॑स्य - अ॒न्न॒स्य॒ । अव॑रुद्ध्या॒ इत्यव॑ - रु॒द्ध्यै॒ । पु॒रस्ता᳚त् । प्र॒त्यञ्च᳚म् । अ॒भीति॑ । सि॒ञ्च॒ति॒ । पु॒रस्ता᳚त् । हि । प्र॒ती॒चीन᳚म् । अन्न᳚म् । अ॒द्यते᳚ । शी॒र्.॒ष॒तः । अ॒भीति॑ । सि॒ञ्च॒ति॒ । शी॒र्.॒ष॒तः । हि । अन्न᳚म् । अ॒द्यते᳚ । एति॑ । मुखा᳚त् । अ॒न्वव॑स्रावय॒तीत्य॑नु - अव॑स्रावयति ।  \newline


\textbf{Krama Paata} \newline

अ॒स्मै॒ स॒वान् । स॒वान् प्र । प्र य॑च्छन्ति । य॒च्छ॒न्ति॒ ते । त ए॑नम् । ए॒नꣳ॒॒ सु॒व॒न्ते॒ । सु॒व॒न्ते॒ स॒वः । स॒वो᳚ऽग्निः । अ॒ग्निर् व॑रुणस॒वः । व॒रु॒ण॒स॒वो रा॑ज॒सूय᳚म् । व॒रु॒ण॒स॒व इति॑ वरुण - स॒वः । रा॒ज॒सूय॑म् ब्रह्मस॒वः । रा॒ज॒सूय॒मिति॑ राज - सूय᳚म् । ब्र॒ह्म॒स॒वश्चित्यः॑ । ब्र॒ह्म॒स॒व इति॑ ब्रह्म - स॒वः । चित्यो॑ दे॒वस्य॑ । दे॒वस्य॑ त्वा । त्वा॒ स॒वि॒तुः । स॒वि॒तुः प्र॑स॒वे । प्र॒स॒व इति॑ । प्र॒स॒व इति॑ प्र - स॒वे । इत्या॑ह । आ॒ह॒ स॒वि॒तृप्र॑सूतः । स॒वि॒तृप्र॑सूत ए॒व । स॒वि॒तृप्र॑सूत॒ इति॑ सवि॒तृ - प्र॒सू॒तः॒ । ए॒वैन᳚म् । ए॒न॒म् ब्रह्म॑णा । ब्रह्म॑णा दे॒वता॑भिः । दे॒वता॑भिर॒भि । अ॒भि षि॑ञ्चति । सि॒ञ्च॒त्यन्न॑स्यान्नस्य । अन्न॑स्यान्नस्या॒भि । अन्य॑स्यान्न॒स्येत्यन्न॑स्य - अ॒न्न॒स्य॒ । 
अ॒भि षि॑ञ्चति । सि॒ञ्च॒त्यन्न॑स्यान्नस्य । अन्न॑स्यान्न॒स्याव॑रुद्ध्यै । अन्न॑स्यान्न॒स्येत्यन्न॑स्य - अ॒न्न॒स्य॒ । अव॑रुद्ध्यै पु॒रस्ता᳚त् । अव॑रुद्ध्या॒ इत्यव॑ - रु॒द्ध्यै॒ । पु॒रस्ता᳚त् प्र॒त्यञ्च᳚म् । प्र॒त्यञ्च॑म॒भि । अ॒भि षि॑ञ्चति । सि॒ञ्च॒ति॒ पु॒रस्ता᳚त् । पु॒रस्ता॒द्धि । हि प्र॑ती॒चीन᳚म् । प्र॒ती॒चीन॒मन्न᳚म् । अन्न॑म॒द्यते᳚ । अ॒द्यते॑ शीर्.ष॒तः । शी॒र्॒.ष॒तो॑ऽभि । अ॒भि षि॑ञ्चति । सि॒ञ्च॒ति॒ शी॒र्॒.ष॒तः । शी॒र्॒.ष॒तो हि । ह्यन्न᳚म् । अन्न॑म॒द्यते᳚ । अ॒द्यत॒ आ । आ मुखा᳚त् । मुखा॑द॒न्वव॑स्रावयति । अ॒न्वव॑स्रावयति मुख॒तः । अ॒न्वव॑स्रावय॒तीत्य॑नु - अव॑स्रावयति \newline

\textbf{Jatai Paata} \newline

1. अ॒स्मै॒ स॒वान् थ्स॒वा न॑स्मा अस्मै स॒वान् । \newline
2. स॒वान् प्र प्र स॒वान् थ्स॒वान् प्र । \newline
3. प्र य॑च्छन्ति यच्छन्ति॒ प्र प्र य॑च्छन्ति । \newline
4. य॒च्छ॒न्ति॒ ते ते य॑च्छन्ति यच्छन्ति॒ ते । \newline
5. त ए॑न मेन॒म् ते त ए॑नम् । \newline
6. ए॒नꣳ॒॒ सु॒व॒न्ते॒ सु॒व॒न्त॒ ए॒न॒ मे॒नꣳ॒॒ सु॒व॒न्ते॒ । \newline
7. सु॒व॒न्ते॒ स॒वः स॒वः सु॑वन्ते सुवन्ते स॒वः । \newline
8. स॒वो᳚ ऽग्नि र॒ग्निः स॒वः स॒वो᳚ ऽग्निः । \newline
9. अ॒ग्निर् व॑रुणस॒वो व॑रुणस॒वो᳚ ऽग्नि र॒ग्निर् व॑रुणस॒वः । \newline
10. व॒रु॒ण॒स॒वो रा॑ज॒सूयꣳ॑ राज॒सूयं॑ ॅवरुणस॒वो व॑रुणस॒वो रा॑ज॒सूय᳚म् । \newline
11. व॒रु॒ण॒स॒व इति॑ वरुण - स॒वः । \newline
12. रा॒ज॒सूय॑म् ब्रह्मस॒वो ब्र॑ह्मस॒वो रा॑ज॒सूयꣳ॑ राज॒सूय॑म् ब्रह्मस॒वः । \newline
13. रा॒ज॒सूय॒मिति॑ राज - सूय᳚म् । \newline
14. ब्र॒ह्म॒स॒व श्चित्य॒ श्चित्यो᳚ ब्रह्मस॒वो ब्र॑ह्मस॒व श्चित्यः॑ । \newline
15. ब्र॒ह्म॒स॒व इति॑ ब्रह्म - स॒वः । \newline
16. चित्यो॑ दे॒वस्य॑ दे॒वस्य॒ चित्य॒ श्चित्यो॑ दे॒वस्य॑ । \newline
17. दे॒वस्य॑ त्वा त्वा दे॒वस्य॑ दे॒वस्य॑ त्वा । \newline
18. त्वा॒ स॒वि॒तुः स॑वि॒तु स्त्वा᳚ त्वा सवि॒तुः । \newline
19. स॒वि॒तुः प्र॑स॒वे प्र॑स॒वे स॑वि॒तुः स॑वि॒तुः प्र॑स॒वे । \newline
20. प्र॒स॒व इतीति॑ प्रस॒वे प्र॑स॒व इति॑ । \newline
21. प्र॒स॒व इति॑ प्र - स॒वे । \newline
22. इत्या॑हा॒हे तीत्या॑ह । \newline
23. आ॒ह॒ स॒वि॒तृप्र॑सूतः सवि॒तृप्र॑सूत आहाह सवि॒तृप्र॑सूतः । \newline
24. स॒वि॒तृप्र॑सूत ए॒वैव स॑वि॒तृप्र॑सूतः सवि॒तृप्र॑सूत ए॒व । \newline
25. स॒वि॒तृप्र॑सूत॒ इति॑ सवि॒तृ - प्र॒सू॒तः॒ । \newline
26. ए॒वैन॑ मेन मे॒वै वैन᳚म् । \newline
27. ए॒न॒म् ब्रह्म॑णा॒ ब्रह्म॑णैन मेन॒म् ब्रह्म॑णा । \newline
28. ब्रह्म॑णा दे॒वता॑भिर् दे॒वता॑भि॒र् ब्रह्म॑णा॒ ब्रह्म॑णा दे॒वता॑भिः । \newline
29. दे॒वता॑भि र॒भ्य॑भि दे॒वता॑भिर् दे॒वता॑भि र॒भि । \newline
30. अ॒भि षि॑ञ्चति सिञ्च त्य॒भ्य॑भि षि॑ञ्चति । \newline
31. सि॒ञ्च॒ त्यन्न॑स्यान्न॒स्या न्न॑स्यान्नस्य सिञ्चति सिञ्च॒ त्यन्न॑स्यान्नस्य । \newline
32. अन्न॑स्यान्नस्या॒ भ्य॑भ्यन्न॑स्यान्न॒स्या न्न॑स्यान्नस्या॒भि । \newline
33. अन्न॑स्यान्न॒स्येत्यन्न॑स्य - अ॒न्न॒स्य॒ । \newline
34. अ॒भि षि॑ञ्चति सिञ्च त्य॒भ्य॑भि षि॑ञ्चति । \newline
35. सि॒ञ्च॒ त्यन्न॑स्यान्न॒स्या न्न॑स्यान्नस्य सिञ्चति सिञ्च॒ त्यन्न॑स्यान्नस्य । \newline
36. अन्न॑स्यान्न॒स्या व॑रुद्ध्या॒ अव॑रुद्ध्या॒ अन्न॑स्यान्न॒स्या न्न॑स्यान्न॒स्या व॑रुद्ध्यै । \newline
37. अन्न॑स्यान्न॒स्येत्यन्न॑स्य - अ॒न्न॒स्य॒ । \newline
38. अव॑रुद्ध्यै पु॒रस्ता᳚त् पु॒रस्ता॒ दव॑रुद्ध्या॒ अव॑रुद्ध्यै पु॒रस्ता᳚त् । \newline
39. अव॑रुद्ध्या॒ इत्यव॑ - रु॒द्ध्यै॒ । \newline
40. पु॒रस्ता᳚त् प्र॒त्यञ्च॑म् प्र॒त्यञ्च॑म् पु॒रस्ता᳚त् पु॒रस्ता᳚त् प्र॒त्यञ्च᳚म् । \newline
41. प्र॒त्यञ्च॑ म॒भ्य॑भि प्र॒त्यञ्च॑म् प्र॒त्यञ्च॑ म॒भि । \newline
42. अ॒भि षि॑ञ्चति सिञ्च त्य॒भ्य॑भि षि॑ञ्चति । \newline
43. सि॒ञ्च॒ति॒ पु॒रस्ता᳚त् पु॒रस्ता᳚थ् सिञ्चति सिञ्चति पु॒रस्ता᳚त् । \newline
44. पु॒रस्ता॒ द्धि हि पु॒रस्ता᳚त् पु॒रस्ता॒ द्धि । \newline
45. हि प्र॑ती॒चीन॑म् प्रती॒चीनꣳ॒॒ हि हि प्र॑ती॒चीन᳚म् । \newline
46. प्र॒ती॒चीन॒ मन्न॒ मन्न॑म् प्रती॒चीन॑म् प्रती॒चीन॒ मन्न᳚म् । \newline
47. अन्न॑ म॒द्यते॒ ऽद्यते ऽन्न॒ मन्न॑ म॒द्यते᳚ । \newline
48. अ॒द्यते॑ शीर्.ष॒तः शी॑र्.ष॒तो᳚ ऽद्यते॒ ऽद्यते॑ शीर्.ष॒तः । \newline
49. शी॒र्॒.ष॒तो᳚(1॒) ऽभ्य॑भि शी॑र्.ष॒तः शी॑र्.ष॒तो॑ ऽभि । \newline
50. अ॒भि षि॑ञ्चति सिञ्च त्य॒भ्य॑भि षि॑ञ्चति । \newline
51. सि॒ञ्च॒ति॒ शी॒र्॒.ष॒तः शी॑र्.ष॒तः सि॑ञ्चति सिञ्चति शीर्.ष॒तः । \newline
52. शी॒र्॒.ष॒तो हि हि शी॑र्.ष॒तः शी॑र्.ष॒तो हि । \newline
53. ह्यन्न॒ मन्नꣳ॒॒ हि ह्यन्न᳚म् । \newline
54. अन्न॑ म॒द्यते॒ ऽद्यते ऽन्न॒ मन्न॑ म॒द्यते᳚ । \newline
55. अ॒द्यत॒ आ ऽद्यते॒ ऽद्यत॒ आ । \newline
56. आ मुखा॒न् मुखा॒दा मुखा᳚त् । \newline
57. मुखा॑ द॒न्वव॑स्रावय त्य॒न्वव॑स्रावयति॒ मुखा॒न् मुखा॑ द॒न्वव॑स्रावयति । \newline
58. अ॒न्वव॑स्रावयति मुख॒तो मु॑ख॒तो᳚ ऽन्वव॑स्रावय त्य॒न्वव॑स्रावयति मुख॒तः । \newline
59. अ॒न्वव॑स्रावय॒तीत्य॑नु - अव॑स्रावयति । \newline

\textbf{Ghana Paata } \newline

1. अ॒स्मै॒ स॒वान् थ्स॒वा न॑स्मा अस्मै स॒वान् प्र प्र स॒वा न॑स्मा अस्मै स॒वान् प्र । \newline
2. स॒वान् प्र प्र स॒वान् थ्स॒वान् प्र य॑च्छन्ति यच्छन्ति॒ प्र स॒वान् थ्स॒वान् प्र य॑च्छन्ति । \newline
3. प्र य॑च्छन्ति यच्छन्ति॒ प्र प्र य॑च्छन्ति॒ ते ते य॑च्छन्ति॒ प्र प्र य॑च्छन्ति॒ ते । \newline
4. य॒च्छ॒न्ति॒ ते ते य॑च्छन्ति यच्छन्ति॒ त ए॑न मेन॒म् ते य॑च्छन्ति यच्छन्ति॒ त ए॑नम् । \newline
5. त ए॑न मेन॒म् ते त ए॑नꣳ सुवन्ते सुवन्त एन॒म् ते त ए॑नꣳ सुवन्ते । \newline
6. ए॒नꣳ॒॒ सु॒व॒न्ते॒ सु॒व॒न्त॒ ए॒न॒ मे॒नꣳ॒॒ सु॒व॒न्ते॒ स॒वः स॒वः सु॑वन्त एन मेनꣳ सुवन्ते स॒वः । \newline
7. सु॒व॒न्ते॒ स॒वः स॒वः सु॑वन्ते सुवन्ते स॒वो᳚ ऽग्नि र॒ग्निः स॒वः सु॑वन्ते सुवन्ते स॒वो᳚ ऽग्निः । \newline
8. स॒वो᳚ ऽग्नि र॒ग्निः स॒वः स॒वो᳚ ऽग्निर् व॑रुणस॒वो व॑रुणस॒वो᳚ ऽग्निः स॒वः स॒वो᳚ ऽग्निर् व॑रुणस॒वः । \newline
9. अ॒ग्निर् व॑रुणस॒वो व॑रुणस॒वो᳚ ऽग्नि र॒ग्निर् व॑रुणस॒वो रा॑ज॒सूयꣳ॑ राज॒सूयं॑ ॅवरुणस॒वो᳚ ऽग्नि र॒ग्निर् व॑रुणस॒वो रा॑ज॒सूय᳚म् । \newline
10. व॒रु॒ण॒स॒वो रा॑ज॒सूयꣳ॑ राज॒सूयं॑ ॅवरुणस॒वो व॑रुणस॒वो रा॑ज॒सूय॑म् ब्रह्मस॒वो ब्र॑ह्मस॒वो रा॑ज॒सूयं॑ ॅवरुणस॒वो व॑रुणस॒वो रा॑ज॒सूय॑म् ब्रह्मस॒वः । \newline
11. व॒रु॒ण॒स॒व इति॑ वरुण - स॒वः । \newline
12. रा॒ज॒सूय॑म् ब्रह्मस॒वो ब्र॑ह्मस॒वो रा॑ज॒सूयꣳ॑ राज॒सूय॑म् ब्रह्मस॒व श्चित्य॒ श्चित्यो᳚ ब्रह्मस॒वो रा॑ज॒सूयꣳ॑ राज॒सूय॑म् ब्रह्मस॒व श्चित्यः॑ । \newline
13. रा॒ज॒सूय॒मिति॑ राज - सूय᳚म् । \newline
14. ब्र॒ह्म॒स॒व श्चित्य॒ श्चित्यो᳚ ब्रह्मस॒वो ब्र॑ह्मस॒व श्चित्यो॑ दे॒वस्य॑ दे॒वस्य॒ चित्यो᳚ ब्रह्मस॒वो ब्र॑ह्मस॒व श्चित्यो॑ दे॒वस्य॑ । \newline
15. ब्र॒ह्म॒स॒व इति॑ ब्रह्म - स॒वः । \newline
16. चित्यो॑ दे॒वस्य॑ दे॒वस्य॒ चित्य॒ श्चित्यो॑ दे॒वस्य॑ त्वा त्वा दे॒वस्य॒ चित्य॒ श्चित्यो॑ दे॒वस्य॑ त्वा । \newline
17. दे॒वस्य॑ त्वा त्वा दे॒वस्य॑ दे॒वस्य॑ त्वा सवि॒तुः स॑वि॒तु स्त्वा॑ दे॒वस्य॑ दे॒वस्य॑ त्वा सवि॒तुः । \newline
18. त्वा॒ स॒वि॒तुः स॑वि॒तु स्त्वा᳚ त्वा सवि॒तुः प्र॑स॒वे प्र॑स॒वे स॑वि॒तु स्त्वा᳚ त्वा सवि॒तुः प्र॑स॒वे । \newline
19. स॒वि॒तुः प्र॑स॒वे प्र॑स॒वे स॑वि॒तुः स॑वि॒तुः प्र॑स॒व इतीति॑ प्रस॒वे स॑वि॒तुः स॑वि॒तुः प्र॑स॒व इति॑ । \newline
20. प्र॒स॒व इतीति॑ प्रस॒वे प्र॑स॒व इत्या॑हा॒हेति॑ प्रस॒वे प्र॑स॒व इत्या॑ह । \newline
21. प्र॒स॒व इति॑ प्र - स॒वे । \newline
22. इत्या॑हा॒हे तीत्या॑ह सवि॒तृप्र॑सूतः सवि॒तृप्र॑सूत आ॒हे तीत्या॑ह सवि॒तृप्र॑सूतः । \newline
23. आ॒ह॒ स॒वि॒तृप्र॑सूतः सवि॒तृप्र॑सूत आहाह सवि॒तृप्र॑सूत ए॒वैव स॑वि॒तृप्र॑सूत आहाह सवि॒तृप्र॑सूत ए॒व । \newline
24. स॒वि॒तृप्र॑सूत ए॒वैव स॑वि॒तृप्र॑सूतः सवि॒तृप्र॑सूत ए॒वैन॑ मेन मे॒व स॑वि॒तृप्र॑सूतः सवि॒तृप्र॑सूत ए॒वैन᳚म् । \newline
25. स॒वि॒तृप्र॑सूत॒ इति॑ सवि॒तृ - प्र॒सू॒तः॒ । \newline
26. ए॒वैन॑ मेन मे॒वै वैन॒म् ब्रह्म॑णा॒ ब्रह्म॑णैन मे॒वै वैन॒म् ब्रह्म॑णा । \newline
27. ए॒न॒म् ब्रह्म॑णा॒ ब्रह्म॑णैन मेन॒म् ब्रह्म॑णा दे॒वता॑भिर् दे॒वता॑भि॒र् ब्रह्म॑णैन मेन॒म् ब्रह्म॑णा दे॒वता॑भिः । \newline
28. ब्रह्म॑णा दे॒वता॑भिर् दे॒वता॑भि॒र् ब्रह्म॑णा॒ ब्रह्म॑णा दे॒वता॑भि र॒भ्य॑भि दे॒वता॑भि॒र् ब्रह्म॑णा॒ ब्रह्म॑णा दे॒वता॑भि र॒भि । \newline
29. दे॒वता॑भि र॒भ्य॑भि दे॒वता॑भिर् दे॒वता॑भि र॒भि षि॑ञ्चति सिञ्च त्य॒भि दे॒वता॑भिर् दे॒वता॑भि र॒भि षि॑ञ्चति । \newline
30. अ॒भि षि॑ञ्चति सिञ्च त्य॒भ्य॑भि षि॑ञ्च॒ त्यन्न॑स्यान्न॒स्या न्न॑स्यान्नस्य सिञ्च त्य॒भ्य॑भि षि॑ञ्च॒
त्यन्न॑स्यान्नस्य । \newline
31. सि॒ञ्च॒ त्यन्न॑स्यान्न॒स्या न्न॑स्यान्नस्य सिञ्चति सिञ्च॒ त्यन्न॑स्यान्नस्या॒ भ्य॑भ्यन्न॑स्यान्नस्य सिञ्चति सिञ्च॒ त्यन्न॑स्यान्नस्या॒भि । \newline
32. अन्न॑स्यान्नस्या॒ भ्य॑भ्यन्न॑स्या न्न॒स्यान्न॑स्या न्नस्या॒भि षि॑ञ्चति सिञ्च त्य॒भ्यन्न॑स्यान्न॒स्या न्न॑स्यान्नस्या॒भि षि॑ञ्चति । \newline
33. अन्न॑स्यान्न॒स्येत्यन्न॑स्य - अ॒न्न॒स्य॒ । \newline
34. अ॒भि षि॑ञ्चति सिञ्चत्य॒ भ्य॑भि षि॑ञ्च॒ त्यन्न॑स्यान्न॒स्या न्न॑स्यान्नस्य सिञ्च त्य॒भ्य॑भि षि॑ञ्च॒
त्यन्न॑स्यान्नस्य । \newline
35. सि॒ञ्च॒ त्यन्न॑स्यान्न॒स्या न्न॑स्यान्नस्य सिञ्चति सिञ्च॒ त्यन्न॑स्यान्न॒स्या व॑रुद्ध्या॒ अव॑रुद्ध्या॒ अन्न॑स्यान्नस्य सिञ्चति सिञ्च॒ त्यन्न॑स्यान्न॒स्या व॑रुद्ध्यै । \newline
36. अन्न॑स्यान्न॒स्या व॑रुद्ध्या॒ अव॑रुद्ध्या॒ अन्न॑स्यान्न॒स्या न्न॑स्यान्न॒स्या व॑रुद्ध्यै पु॒रस्ता᳚त् पु॒रस्ता॒ दव॑रुद्ध्या॒ अन्न॑स्यान्न॒स्या न्न॑स्यान्न॒स्या व॑रुद्ध्यै पु॒रस्ता᳚त् । \newline
37. अन्न॑स्यान्न॒स्येत्यन्न॑स्य - अ॒न्न॒स्य॒ । \newline
38. अव॑रुद्ध्यै पु॒रस्ता᳚त् पु॒रस्ता॒ दव॑रुद्ध्या॒ अव॑रुद्ध्यै पु॒रस्ता᳚त् प्र॒त्यञ्च॑म् प्र॒त्यञ्च॑म् पु॒रस्ता॒ दव॑रुद्ध्या॒ अव॑रुद्ध्यै पु॒रस्ता᳚त् प्र॒त्यञ्च᳚म् । \newline
39. अव॑रुद्ध्या॒ इत्यव॑ - रु॒द्ध्यै॒ । \newline
40. पु॒रस्ता᳚त् प्र॒त्यञ्च॑म् प्र॒त्यञ्च॑म् पु॒रस्ता᳚त् पु॒रस्ता᳚त् प्र॒त्यञ्च॑ म॒भ्य॑भि प्र॒त्यञ्च॑म् पु॒रस्ता᳚त् पु॒रस्ता᳚त् प्र॒त्यञ्च॑ म॒भि । \newline
41. प्र॒त्यञ्च॑ म॒भ्य॑भि प्र॒त्यञ्च॑म् प्र॒त्यञ्च॑ म॒भि षि॑ञ्चति सिञ्च त्य॒भि प्र॒त्यञ्च॑म् प्र॒त्यञ्च॑ म॒भि षि॑ञ्चति । \newline
42. अ॒भि षि॑ञ्चति सिञ्च त्य॒भ्य॑भि षि॑ञ्चति पु॒रस्ता᳚त् पु॒रस्ता᳚थ् सिञ्च त्य॒भ्य॑भि षि॑ञ्चति पु॒रस्ता᳚त् । \newline
43. सि॒ञ्च॒ति॒ पु॒रस्ता᳚त् पु॒रस्ता᳚थ् सिञ्चति सिञ्चति पु॒रस्ता॒द्धि हि पु॒रस्ता᳚थ् सिञ्चति सिञ्चति पु॒रस्ता॒द्धि । \newline
44. पु॒रस्ता॒द्धि हि पु॒रस्ता᳚त् पु॒रस्ता॒द्धि प्र॑ती॒चीन॑म् प्रती॒चीनꣳ॒॒ हि पु॒रस्ता᳚त् पु॒रस्ता॒द्धि प्र॑ती॒चीन᳚म् । \newline
45. हि प्र॑ती॒चीन॑म् प्रती॒चीनꣳ॒॒ हि हि प्र॑ती॒चीन॒ मन्न॒ मन्न॑म् प्रती॒चीनꣳ॒॒ हि हि प्र॑ती॒चीन॒ मन्न᳚म् । \newline
46. प्र॒ती॒चीन॒ मन्न॒ मन्न॑म् प्रती॒चीन॑म् प्रती॒चीन॒ मन्न॑ म॒द्यते॒ ऽद्यते ऽन्न॑म् प्रती॒चीन॑म् प्रती॒चीन॒ मन्न॑ म॒द्यते᳚ । \newline
47. अन्न॑ म॒द्यते॒ ऽद्यते ऽन्न॒ मन्न॑ म॒द्यते॑ शीर्.ष॒तः शी॑र्.ष॒तो᳚ ऽद्यते ऽन्न॒ मन्न॑ म॒द्यते॑ शीर्.ष॒तः । \newline
48. अ॒द्यते॑ शीर्.ष॒तः शी॑र्.ष॒तो᳚ ऽद्यते॒ ऽद्यते॑ शीर्.ष॒तो᳚(1॒) ऽभ्य॑भि शी॑र्.ष॒तो᳚ ऽद्यते॒ ऽद्यते॑ शीर्.ष॒तो॑ ऽभि । \newline
49. शी॒र्॒.ष॒तो᳚(1॒) ऽभ्य॑भि शी॑र्.ष॒तः शी॑र्.ष॒तो॑ ऽभि षि॑ञ्चति सिञ्च त्य॒भि शी॑र्.ष॒तः शी॑र्.ष॒तो॑ ऽभि षि॑ञ्चति । \newline
50. अ॒भि षि॑ञ्चति सिञ्चत्य॒ भ्य॑भि षि॑ञ्चति शीर्.ष॒तः शी॑र्.ष॒तः सि॑ञ्चत्य॒ भ्य॑भि षि॑ञ्चति शीर्.ष॒तः । \newline
51. सि॒ञ्च॒ति॒ शी॒र्॒.ष॒तः शी॑र्.ष॒तः सि॑ञ्चति सिञ्चति शीर्.ष॒तो हि हि शी॑र्.ष॒तः सि॑ञ्चति सिञ्चति शीर्.ष॒तो हि । \newline
52. शी॒र्॒.ष॒तो हि हि शी॑र्.ष॒तः शी॑र्.ष॒तो ह्यन्न॒ मन्नꣳ॒॒ हि शी॑र्.ष॒तः शी॑र्.ष॒तो ह्यन्न᳚म् । \newline
53. ह्यन्न॒ मन्नꣳ॒॒ हि ह्यन्न॑ म॒द्यते॒ ऽद्यते ऽन्नꣳ॒॒ हि ह्यन्न॑ म॒द्यते᳚ । \newline
54. अन्न॑ म॒द्यते॒ ऽद्यते ऽन्न॒ मन्न॑ म॒द्यत॒ आ ऽद्यते ऽन्न॒ मन्न॑ म॒द्यत॒ आ । \newline
55. अ॒द्यत॒ आ ऽद्यते॒ ऽद्यत॒ आ मुखा॒न् मुखा॒दा ऽद्यते॒ ऽद्यत॒ आ मुखा᳚त् । \newline
56. आ मुखा॒न् मुखा॒दा मुखा॑ द॒न्वव॑स्रावय त्य॒न्वव॑स्रावयति॒ मुखा॒दा मुखा॑ द॒न्वव॑स्रावयति । \newline
57. मुखा॑ द॒न्वव॑स्रावय त्य॒न्वव॑स्रावयति॒ मुखा॒न् मुखा॑ द॒न्वव॑स्रावयति मुख॒तो मु॑ख॒तो᳚ ऽन्वव॑स्रावयति॒ मुखा॒न् मुखा॑ द॒न्वव॑स्रावयति मुख॒तः । \newline
58. अ॒न्वव॑स्रावयति मुख॒तो मु॑ख॒तो᳚ ऽन्वव॑स्रावय त्य॒न्वव॑स्रावयति मुख॒त ए॒वैव मु॑ख॒तो᳚ ऽन्वव॑स्रावय त्य॒न्वव॑स्रावयति मुख॒त ए॒व । \newline
59. अ॒न्वव॑स्रावय॒तीत्य॑नु - अव॑स्रावयति । \newline
\pagebreak
\markright{ TS 5.6.3.3  \hfill https://www.vedavms.in \hfill}

\section{ TS 5.6.3.3 }

\textbf{TS 5.6.3.3 } \newline
\textbf{Samhita Paata} \newline

मुख॒त ए॒वास्मा॑ अ॒न्नाद्यं॑ दधात्य॒ग्नेस्त्वा॒ साम्रा᳚ज्येना॒भि षि॑ञ्चा॒मीत्या॑है॒ष वा अ॒ग्नेः स॒वस्तेनै॒वैन॑म॒भि षि॑ञ्चति॒ बृह॒स्पते᳚स्त्वा॒ साम्रा᳚ज्येना॒भिषि॑ञ्चा॒मीत्या॑ह॒ ब्रह्म॒ वै दे॒वानां॒ बृह॒स्पति॒र्ब्रह्म॑णै॒वैन॑म॒भि षि॑ञ्च॒तीन्द्र॑स्य त्वा॒ साम्रा᳚ज्येना॒भि षि॑ञ्चा॒-मीत्या॑हेन्द्रि॒यमे॒वास्मि॑-न्नु॒परि॑ष्टाद् दधात्ये॒ त - [  ] \newline

\textbf{Pada Paata} \newline

मु॒ख॒तः । ए॒व । अ॒स्मै॒ । अ॒न्नाद्य॒मित्य॑न्न -अद्य᳚म् । द॒धा॒ति॒ । अ॒ग्नेः । त्वा॒ । साम्रा᳚ज्ये॒नेति॒ सां - रा॒ज्ये॒न॒ । अ॒भीति॑ । सि॒ञ्चा॒मि॒ । इति॑ । आ॒ह॒ । ए॒षः । वै । अ॒ग्नेः । स॒वः । तेन॑ । ए॒व । ए॒न॒म् । अ॒भीति॑ । सि॒ञ्च॒ति॒ । बृह॒स्पतेः᳚ । त्वा॒ । साम्रा᳚ज्ये॒नेति॒ सां-रा॒ज्ये॒न॒ । अ॒भीति॑ । सि॒ञ्चा॒मि॒ । इति॑ । आ॒ह॒ । ब्रह्म॑ । वै । दे॒वाना᳚म् । बृह॒स्पतिः॑ । ब्रह्म॑णा । ए॒व । ए॒न॒म् । अ॒भीति॑ । सि॒ञ्च॒ति॒ । इन्द्र॑स्य । त्वा॒ । साम्रा᳚ज्ये॒नेति॒ सां - रा॒ज्ये॒न॒ । अ॒भीति॑ । सि॒ञ्चा॒मि॒ । इति॑ । आ॒ह॒ । इ॒न्द्रि॒यम् । ए॒व । अ॒स्मि॒न्न् । उ॒परि॑ष्टात् । द॒धा॒ति॒ । ए॒तत् ।  \newline


\textbf{Krama Paata} \newline

मु॒ख॒त ए॒व । ए॒वास्मै᳚ । अ॒स्मा॒ अ॒न्नाद्य᳚म् । अ॒न्नाद्य॑म् दधाति । अ॒न्नाद्य॒मित्य॑न्न - अद्य᳚म् । द॒धा॒त्य॒ग्नेः । अ॒ग्नेस्त्वा᳚ । त्वा॒ साम्रा᳚ज्येन । साम्रा᳚ज्येना॒भि । साम्रा᳚ज्ये॒नेति॒ साम् - रा॒ज्ये॒न॒ । अ॒भि षि॑ञ्चामि । सि॒ञ्चा॒मीति॑ । इत्या॑ह । आ॒है॒षः । ए॒ष वै । वा अ॒ग्नेः । अ॒ग्नेः स॒वः । स॒वस्तेन॑ । तेनै॒व । ए॒वैन᳚म् । ए॒न॒म॒भि । अ॒भि षि॑ञ्चति । सि॒ञ्च॒ति॒ बृह॒स्पतेः᳚ । बृह॒स्पते᳚स्त्वा । त्वा॒ साम्रा᳚ज्येन । साम्रा᳚ज्येना॒भि । साम्रा᳚ज्ये॒नेति॒ साम् - रा॒ज्ये॒न॒ । अ॒भि षि॑ञ्चामि । सि॒ञ्चा॒मीति॑ । इत्या॑ह । आ॒ह॒ ब्रह्म॑ । ब्रह्म॒ वै । वै दे॒वाना᳚म् । दे॒वाना॒म् बृह॒स्पतिः॑ । बृह॒स्पति॒र् ब्रह्म॑णा । ब्रह्म॑णै॒व । ए॒वैन᳚म् । ए॒न॒म॒भि । अ॒भि षि॑ञ्चति । सि॒ञ्च॒तीन्द्र॑स्य । इन्द्र॑स्य त्वा । त्वा॒ साम्रा᳚ज्येन । साम्रा᳚ज्येना॒भि । साम्रा᳚ज्ये॒नेति॒ साम् - रा॒ज्ये॒न॒ । अ॒भि षि॑ञ्चामि । सि॒ञ्चा॒मीति॑ । इत्या॑ह । आ॒हे॒न्द्रि॒यम् । इ॒न्द्रि॒यमे॒व । ए॒वास्मिन्न्॑ । अ॒स्मि॒न्नु॒परि॑ष्टात् । उ॒परि॑ष्टाद् दधाति । द॒धा॒त्ये॒तत् ( ) । ए॒तद् वै \newline

\textbf{Jatai Paata} \newline

1. मु॒ख॒त ए॒वैव मु॑ख॒तो मु॑ख॒त ए॒व । \newline
2. ए॒वास्मा॑ अस्मा ए॒वैवा स्मै᳚ । \newline
3. अ॒स्मा॒ अ॒न्नाद्य॑ म॒न्नाद्य॑ मस्मा अस्मा अ॒न्नाद्य᳚म् । \newline
4. अ॒न्नाद्य॑म् दधाति दधा त्य॒न्नाद्य॑ म॒न्नाद्य॑म् दधाति । \newline
5. अ॒न्नाद्य॒मित्य॑न्न - अद्य᳚म् । \newline
6. द॒धा॒ त्य॒ग्ने र॒ग्नेर् द॑धाति दधा त्य॒ग्नेः । \newline
7. अ॒ग्ने स्त्वा᳚ त्वा॒ ऽग्ने र॒ग्ने स्त्वा᳚ । \newline
8. त्वा॒ साम्रा᳚ज्येन॒ साम्रा᳚ज्येन त्वा त्वा॒ साम्रा᳚ज्येन । \newline
9. साम्रा᳚ज्येना॒ भ्य॑भि साम्रा᳚ज्येन॒ साम्रा᳚ज्येना॒भि । \newline
10. साम्रा᳚ज्ये॒नेति॒ सां - रा॒ज्ये॒न॒ । \newline
11. अ॒भि षि॑ञ्चामि सिञ्चा म्य॒भ्य॑भि षि॑ञ्चामि । \newline
12. सि॒ञ्चा॒मीतीति॑ सिञ्चामि सिञ्चा॒मीति॑ । \newline
13. इत्या॑हा॒हे तीत्या॑ह । \newline
14. आ॒है॒ष ए॒ष आ॑हा है॒षः । \newline
15. ए॒ष वै वा ए॒ष ए॒ष वै । \newline
16. वा अ॒ग्ने र॒ग्नेर् वै वा अ॒ग्नेः । \newline
17. अ॒ग्नेः स॒वः स॒वो᳚ ऽग्ने र॒ग्नेः स॒वः । \newline
18. स॒व स्तेन॒ तेन॑ स॒वः स॒व स्तेन॑ । \newline
19. तेनै॒वैव तेन॒ तेनै॒व । \newline
20. ए॒वैन॑ मेन मे॒वै वैन᳚म् । \newline
21. ए॒न॒ म॒भ्या᳚(1॒)भ्ये॑न मेन म॒भि । \newline
22. अ॒भि षि॑ञ्चति सिञ्च त्य॒भ्य॑भि षि॑ञ्चति । \newline
23. सि॒ञ्च॒ति॒ बृह॒स्पते॒र् बृह॒स्पतेः᳚ सिञ्चति सिञ्चति॒ बृह॒स्पतेः᳚ । \newline
24. बृह॒स्पते᳚ स्त्वा त्वा॒ बृह॒स्पते॒र् बृह॒स्पते᳚ स्त्वा । \newline
25. त्वा॒ साम्रा᳚ज्येन॒ साम्रा᳚ज्येन त्वा त्वा॒ साम्रा᳚ज्येन । \newline
26. साम्रा᳚ज्येना॒भ्य॑भि साम्रा᳚ज्येन॒ साम्रा᳚ज्येना॒भि । \newline
27. साम्रा᳚ज्ये॒नेति॒ सां - रा॒ज्ये॒न॒ । \newline
28. अ॒भि षि॑ञ्चामि सिञ्चा म्य॒भ्य॑भि षि॑ञ्चामि । \newline
29. सि॒ञ्चा॒मीतीति॑ सिञ्चामि सिञ्चा॒मीति॑ । \newline
30. इत्या॑हा॒हे तीत्या॑ह । \newline
31. आ॒ह॒ ब्रह्म॒ ब्रह्मा॑ हाह॒ ब्रह्म॑ । \newline
32. ब्रह्म॒ वै वै ब्रह्म॒ ब्रह्म॒ वै । \newline
33. वै दे॒वाना᳚म् दे॒वानां॒ ॅवै वै दे॒वाना᳚म् । \newline
34. दे॒वाना॒म् बृह॒स्पति॒र् बृह॒स्पति॑र् दे॒वाना᳚म् दे॒वाना॒म् बृह॒स्पतिः॑ । \newline
35. बृह॒स्पति॒र् ब्रह्म॑णा॒ ब्रह्म॑णा॒ बृह॒स्पति॒र् बृह॒स्पति॒र् ब्रह्म॑णा । \newline
36. ब्रह्म॑ णै॒वैव ब्रह्म॑णा॒ ब्रह्म॑णै॒व । \newline
37. ए॒वैन॑ मेन मे॒वै वैन᳚म् । \newline
38. ए॒न॒ म॒भ्या᳚(1॒)भ्ये॑न मेन म॒भि । \newline
39. अ॒भि षि॑ञ्चति सिञ्च त्य॒भ्य॑भि षि॑ञ्चति । \newline
40. सि॒ञ्च॒तीन्द्र॒ स्येन्द्र॑स्य सिञ्चति सिञ्च॒तीन्द्र॑स्य । \newline
41. इन्द्र॑स्य त्वा॒ त्वेन्द्र॒ स्येन्द्र॑स्य त्वा । \newline
42. त्वा॒ साम्रा᳚ज्येन॒ साम्रा᳚ज्येन त्वा त्वा॒ साम्रा᳚ज्येन । \newline
43. साम्रा᳚ज्येना॒ भ्य॑भि साम्रा᳚ज्येन॒ साम्रा᳚ज्येना॒भि । \newline
44. साम्रा᳚ज्ये॒नेति॒ सां - रा॒ज्ये॒न॒ । \newline
45. अ॒भि षि॑ञ्चामि सिञ्चा म्य॒भ्य॑भि षि॑ञ्चामि । \newline
46. सि॒ञ्चा॒मीतीति॑ सिञ्चामि सिञ्चा॒मीति॑ । \newline
47. इत्या॑हा॒हे तीत्या॑ह । \newline
48. आ॒हे॒न्द्रि॒य मि॑न्द्रि॒य मा॑हा हेन्द्रि॒यम् । \newline
49. इ॒न्द्रि॒य मे॒वैवेन्द्रि॒य मि॑न्द्रि॒य मे॒व । \newline
50. ए॒वास्मि॑न् नस्मिन् ने॒वै वास्मिन्न्॑ । \newline
51. अ॒स्मि॒न् नु॒परि॑ष्टा दु॒परि॑ष्टा दस्मिन् नस्मिन् नु॒परि॑ष्टात् । \newline
52. उ॒परि॑ष्टाद् दधाति दधा त्यु॒परि॑ष्टा दु॒परि॑ष्टाद् दधाति । \newline
53. द॒धा॒ त्ये॒त दे॒तद् द॑धाति दधा त्ये॒तत् । \newline
54. ए॒तद् वै वा ए॒त दे॒तद् वै । \newline

\textbf{Ghana Paata } \newline

1. मु॒ख॒त ए॒वैव मु॑ख॒तो मु॑ख॒त ए॒वास्मा॑ अस्मा ए॒व मु॑ख॒तो मु॑ख॒त ए॒वास्मै᳚ । \newline
2. ए॒वास्मा॑ अस्मा ए॒वैवास्मा॑ अ॒न्नाद्य॑ म॒न्नाद्य॑ मस्मा ए॒वैवास्मा॑ अ॒न्नाद्य᳚म् । \newline
3. अ॒स्मा॒ अ॒न्नाद्य॑ म॒न्नाद्य॑ मस्मा अस्मा अ॒न्नाद्य॑म् दधाति दधा त्य॒न्नाद्य॑ मस्मा अस्मा अ॒न्नाद्य॑म् दधाति । \newline
4. अ॒न्नाद्य॑म् दधाति दधा त्य॒न्नाद्य॑ म॒न्नाद्य॑म् दधा त्य॒ग्ने र॒ग्नेर् द॑धा त्य॒न्नाद्य॑ म॒न्नाद्य॑म् दधा त्य॒ग्नेः । \newline
5. अ॒न्नाद्य॒मित्य॑न्न - अद्य᳚म् । \newline
6. द॒धा॒ त्य॒ग्ने र॒ग्नेर् द॑धाति दधा त्य॒ग्ने स्त्वा᳚ त्वा॒ ऽग्नेर् द॑धाति दधा त्य॒ग्ने स्त्वा᳚ । \newline
7. अ॒ग्ने स्त्वा᳚ त्वा॒ ऽग्ने र॒ग्ने स्त्वा॒ साम्रा᳚ज्येन॒ साम्रा᳚ज्येन त्वा॒ ऽग्ने र॒ग्ने स्त्वा॒ साम्रा᳚ज्येन । \newline
8. त्वा॒ साम्रा᳚ज्येन॒ साम्रा᳚ज्येन त्वा त्वा॒ साम्रा᳚ज्येना॒ भ्य॑भि साम्रा᳚ज्येन त्वा त्वा॒ साम्रा᳚ज्येना॒भि । \newline
9. साम्रा᳚ज्येना॒ भ्य॑भि साम्रा᳚ज्येन॒ साम्रा᳚ज्येना॒भि षि॑ञ्चामि सिञ्चाम्य॒भि साम्रा᳚ज्येन॒ साम्रा᳚ज्येना॒भि षि॑ञ्चामि । \newline
10. साम्रा᳚ज्ये॒नेति॒ सां - रा॒ज्ये॒न॒ । \newline
11. अ॒भि षि॑ञ्चामि सिञ्चा म्य॒भ्य॑भि षि॑ञ्चा॒मी तीति॑ सिञ्चा म्य॒भ्य॑भि षि॑ञ्चा॒मीति॑ । \newline
12. सि॒ञ्चा॒मी तीति॑ सिञ्चामि सिञ्चा॒मीत्या॑ हा॒हेति॑ सिञ्चामि सिञ्चा॒मी त्या॑ह । \newline
13. इत्या॑हा॒हे तीत्या॑ है॒ष ए॒ष आ॒हे तीत्या॑ है॒षः । \newline
14. आ॒है॒ष ए॒ष आ॑हा है॒ष वै वा ए॒ष आ॑हा है॒ष वै । \newline
15. ए॒ष वै वा ए॒ष ए॒ष वा अ॒ग्ने र॒ग्नेर् वा ए॒ष ए॒ष वा अ॒ग्नेः । \newline
16. वा अ॒ग्ने र॒ग्नेर् वै वा अ॒ग्नेः स॒वः स॒वो᳚ ऽग्नेर् वै वा अ॒ग्नेः स॒वः । \newline
17. अ॒ग्नेः स॒वः स॒वो᳚ ऽग्ने र॒ग्नेः स॒व स्तेन॒ तेन॑ स॒वो᳚ ऽग्ने र॒ग्नेः स॒व स्तेन॑ । \newline
18. स॒व स्तेन॒ तेन॑ स॒वः स॒व स्तेनै॒ वैव तेन॑ स॒वः स॒व स्तेनै॒व । \newline
19. टेनै॒ वैव तेन॒ तेनै॒ वैन॑ मेन मे॒व तेन॒ तेनै॒ वैन᳚म् । \newline
20. ए॒वैन॑ मेन मे॒वै वैन॑ म॒भ्या᳚(1॒)भ्ये॑न मे॒वै वैन॑ म॒भि । \newline
21. ए॒न॒ म॒भ्या᳚(1॒)भ्ये॑न मेन म॒भि षि॑ञ्चति सिञ्च त्य॒भ्ये॑न मेन म॒भि षि॑ञ्चति । \newline
22. अ॒भि षि॑ञ्चति सिञ्च त्य॒भ्य॑भि षि॑ञ्चति॒ बृह॒स्पते॒र् बृह॒स्पतेः᳚ सिञ्च त्य॒भ्य॑भि षि॑ञ्चति॒ बृह॒स्पतेः᳚ । \newline
23. सि॒ञ्च॒ति॒ बृह॒स्पते॒र् बृह॒स्पतेः᳚ सिञ्चति सिञ्चति॒ बृह॒स्पते᳚ स्त्वा त्वा॒ बृह॒स्पतेः᳚ सिञ्चति सिञ्चति॒ बृह॒स्पते᳚ स्त्वा । \newline
24. बृह॒स्पते᳚ स्त्वा त्वा॒ बृह॒स्पते॒र् बृह॒स्पते᳚ स्त्वा॒ साम्रा᳚ज्येन॒ साम्रा᳚ज्येन त्वा॒ बृह॒स्पते॒र् बृह॒स्पते᳚ स्त्वा॒ साम्रा᳚ज्येन । \newline
25. त्वा॒ साम्रा᳚ज्येन॒ साम्रा᳚ज्येन त्वा त्वा॒ साम्रा᳚ज्येना॒ भ्य॑भि साम्रा᳚ज्येन त्वा त्वा॒ साम्रा᳚ज्येना॒भि । \newline
26. साम्रा᳚ज्येना॒ भ्य॑भि साम्रा᳚ज्येन॒ साम्रा᳚ज्येना॒भि षि॑ञ्चामि सिञ्चाम्य॒भि साम्रा᳚ज्येन॒ साम्रा᳚ज्येना॒भि षि॑ञ्चामि । \newline
27. साम्रा᳚ज्ये॒नेति॒ सां - रा॒ज्ये॒न॒ । \newline
28. अ॒भि षि॑ञ्चामि सिञ्चाम्य॒ भ्य॑भि षि॑ञ्चा॒मी तीति॑ सिञ्चाम्य॒ भ्य॑भि षि॑ञ्चा॒मीति॑ । \newline
29. सि॒ञ्चा॒मीतीति॑ सिञ्चामि सिञ्चा॒मीत्या॑ हा॒हेति॑ सिञ्चामि सिञ्चा॒मीत्या॑ह । \newline
30. इत्या॑हा॒हे तीत्या॑ह॒ ब्रह्म॒ ब्रह्मा॒हे तीत्या॑ह॒ ब्रह्म॑ । \newline
31. आ॒ह॒ ब्रह्म॒ ब्रह्मा॑हाह॒ ब्रह्म॒ वै वै ब्रह्मा॑हाह॒ ब्रह्म॒ वै । \newline
32. ब्रह्म॒ वै वै ब्रह्म॒ ब्रह्म॒ वै दे॒वाना᳚म् दे॒वानां॒ ॅवै ब्रह्म॒ ब्रह्म॒ वै दे॒वाना᳚म् । \newline
33. वै दे॒वाना᳚म् दे॒वानां॒ ॅवै वै दे॒वाना॒म् बृह॒स्पति॒र् बृह॒स्पति॑र् दे॒वानां॒ ॅवै वै दे॒वाना॒म् बृह॒स्पतिः॑ । \newline
34. दे॒वाना॒म् बृह॒स्पति॒र् बृह॒स्पति॑र् दे॒वाना᳚म् दे॒वाना॒म् बृह॒स्पति॒र् ब्रह्म॑णा॒ ब्रह्म॑णा॒ बृह॒स्पति॑र् दे॒वाना᳚म् दे॒वाना॒म् बृह॒स्पति॒र् ब्रह्म॑णा । \newline
35. बृह॒स्पति॒र् ब्रह्म॑णा॒ ब्रह्म॑णा॒ बृह॒स्पति॒र् बृह॒स्पति॒र् ब्रह्म॑ णै॒वैव ब्रह्म॑णा॒ बृह॒स्पति॒र् बृह॒स्पति॒र् ब्रह्म॑णै॒व । \newline
36. ब्रह्म॑ णै॒वैव ब्रह्म॑णा॒ ब्रह्म॑ णै॒वैन॑ मेन मे॒व ब्रह्म॑णा॒ ब्रह्म॑ णै॒वैन᳚म् । \newline
37. ए॒वैन॑ मेन मे॒वै वैन॑ म॒भ्या᳚(1॒)भ्ये॑न मे॒वै वैन॑ म॒भि । \newline
38. ए॒न॒ म॒भ्या᳚(1॒)भ्ये॑न मेन म॒भि षि॑ञ्चति सिञ्च त्य॒भ्ये॑न मेन म॒भि षि॑ञ्चति । \newline
39. अ॒भि षि॑ञ्चति सिञ्च त्य॒भ्य॑भि षि॑ञ्च॒तीन्द्र॒ स्येन्द्र॑स्य सिञ्च त्य॒भ्य॑भि षि॑ञ्च॒तीन्द्र॑स्य । \newline
40. सि॒ञ्च॒तीन्द्र॒ स्येन्द्र॑स्य सिञ्चति सिञ्च॒तीन्द्र॑स्य त्वा॒ त्वेन्द्र॑स्य सिञ्चति सिञ्च॒तीन्द्र॑स्य त्वा । \newline
41. इन्द्र॑स्य त्वा॒ त्वेन्द्र॒ स्येन्द्र॑स्य त्वा॒ साम्रा᳚ज्येन॒ साम्रा᳚ज्येन॒ त्वेन्द्र॒ स्येन्द्र॑स्य त्वा॒ साम्रा᳚ज्येन । \newline
42. त्वा॒ साम्रा᳚ज्येन॒ साम्रा᳚ज्येन त्वा त्वा॒ साम्रा᳚ज्येना॒ भ्य॑भि साम्रा᳚ज्येन॒ त्वा त्वा॒ साम्रा᳚ज्येना॒भि । \newline
43. साम्रा᳚ज्येना॒ भ्य॑भि साम्रा᳚ज्येन॒ साम्रा᳚ज्येना॒भि षि॑ञ्चामि सिञ्चाम्य॒भि साम्रा᳚ज्येन॒ साम्रा᳚ज्येना॒भि षि॑ञ्चामि । \newline
44. साम्रा᳚ज्ये॒नेति॒ सां - रा॒ज्ये॒न॒ । \newline
45. अ॒भि षि॑ञ्चामि सिञ्चाम्य॒भ्य॑भि षि॑ञ्चा॒मी तीति॑ सिञ्चाम्य॒ भ्य॑भि षि॑ञ्चा॒मीति॑ । \newline
46. सि॒ञ्चा॒मी तीति॑ सिञ्चामि सिञ्चा॒मीत्या॑ हा॒हेति॑ सिञ्चामि सिञ्चा॒मीत्या॑ह । \newline
47. इत्या॑हा॒हे तीत्या॑ हेन्द्रि॒य मि॑न्द्रि॒य मा॒हे तीत्या॑ हेन्द्रि॒यम् । \newline
48. आ॒हे॒न्द्रि॒य मि॑न्द्रि॒य मा॑हाहेन्द्रि॒य मे॒वै वेन्द्रि॒य मा॑हा हेन्द्रि॒य मे॒व । \newline
49. इ॒न्द्रि॒य मे॒वैवेन्द्रि॒य मि॑न्द्रि॒य मे॒वास्मि॑न् नस्मिन् ने॒वेन्द्रि॒य मि॑न्द्रि॒य मे॒वास्मिन्न्॑ । \newline
50. ए॒वास्मि॑न् नस्मिन् ने॒वै वास्मि॑न् नु॒परि॑ष्टा दु॒परि॑ष्टा दस्मिन् ने॒वै वास्मि॑न् नु॒परि॑ष्टात् । \newline
51. अ॒स्मि॒न् नु॒परि॑ष्टा दु॒परि॑ष्टा दस्मिन् नस्मिन् नु॒परि॑ष्टाद् दधाति दधा त्यु॒परि॑ष्टा दस्मिन् नस्मिन् नु॒परि॑ष्टाद् दधाति । \newline
52. उ॒परि॑ष्टाद् दधाति दधा त्यु॒परि॑ष्टा दु॒परि॑ष्टाद् दधा त्ये॒त दे॒तद् द॑धा त्यु॒परि॑ष्टा दु॒परि॑ष्टाद् दधा त्ये॒तत् । \newline
53. द॒धा॒ त्ये॒त दे॒तद् द॑धाति दधा त्ये॒तद् वै वा ए॒तद् द॑धाति दधा त्ये॒तद् वै । \newline
54. ए॒तद् वै वा ए॒त दे॒तद् वै रा॑ज॒सूय॑स्य राज॒सूय॑स्य॒ वा ए॒त दे॒तद् वै रा॑ज॒सूय॑स्य । \newline
\pagebreak
\markright{ TS 5.6.3.4  \hfill https://www.vedavms.in \hfill}

\section{ TS 5.6.3.4 }

\textbf{TS 5.6.3.4 } \newline
\textbf{Samhita Paata} \newline

-द्वै रा॑ज॒सूय॑स्य रू॒पं ॅय ए॒वं ॅवि॒द्वान॒ग्निं चि॑नु॒त उ॒भावे॒व लो॒काव॒भि ज॑यति॒ यश्च॑ राज॒सूये॑नेजा॒नस्य॒ यश्चा᳚ग्नि॒चित॒ इन्द्र॑स्य सुषुवा॒णस्य॑ दश॒धेन्द्रि॒यं ॅवी॒र्यं॑ परा॑ऽपत॒त् तद्दे॒वाः सौ᳚त्राम॒ण्या सम॑भरन्थ् सू॒यते॒ वा ए॒ष यो᳚ऽग्निं चि॑नु॒ते᳚ऽग्निं चि॒त्वा सौ᳚त्राम॒ण्या य॑जेतेन्द्रि॒यमे॒व वी॒र्यꣳ॑ स॒भृंत्या॒ऽऽ*त्मन् ध॑त्ते ॥ \newline

\textbf{Pada Paata} \newline

वै । रा॒ज॒सूय॒स्येति॑ राज - सूय॑स्य । रू॒पम् । यः । ए॒वम् । वि॒द्वान् । अ॒ग्निम् । चि॒नु॒ते । उ॒भौ । ए॒व ।   लो॒कौ । अ॒भीति॑ । ज॒य॒ति॒ । यः । च॒ । रा॒ज॒सूये॒नेति॑ राज - सूये॑न । ई॒जा॒नस्य॑ । यः । च॒ । अ॒ग्नि॒चित॒ इत्य॑ग्नि - चितः॑ । इन्द्र॑स्य । सु॒षु॒वा॒णस्य॑ । द॒श॒धेति॑ दश - धा । इ॒न्द्रि॒यम् । वी॒र्य᳚म् । परेति॑ । अ॒प॒त॒त् । तत् । दे॒वाः । सौ॒त्रा॒म॒ण्या । समिति॑ । अ॒भ॒र॒न्न् । सू॒यते᳚ । वै । ए॒षः । यः । अ॒ग्निम् । चि॒नु॒ते । अ॒ग्निम् । चि॒त्वा । सौ॒त्रा॒म॒ण्या । य॒जे॒त॒ । इ॒न्द्रि॒यम् । ए॒व । वी॒र्य᳚म् । स॒भृंत्येति॑ सं - भृत्य॑ । आ॒त्मन्न् । ध॒त्ते॒ ॥  \newline


\textbf{Krama Paata} \newline

वै रा॑ज॒सूय॑स्य । रा॒ज॒सूय॑स्य रू॒पम् । रा॒ज॒सूय॒स्येति॑ राज - सूय॑स्य । रू॒पम् ॅयः । य ए॒वम् । ए॒वम् ॅवि॒द्वान् । वि॒द्वान॒ग्निम् । अ॒ग्निम् चि॑नु॒ते । चि॒नु॒त उ॒भौ । उ॒भावे॒व । ए॒व लो॒कौ । लो॒काव॒भि । अ॒भि ज॑यति । ज॒य॒ति॒ यः । यश्च॑ । च॒ रा॒ज॒सूये॑न । रा॒ज॒सूये॑नेजा॒नस्य॑ । रा॒ज॒सूये॒नेति॑ राज - सूये॑न । ई॒जा॒नस्य॒ यः । यश्च॑ । चा॒ग्नि॒चितः॑ । अ॒ग्नि॒चित॒ इन्द्र॑स्य । अ॒ग्नि॒चित॒ इत्य॑ग्नि - चितः॑ । इन्द्र॑स्य सुषुवा॒णस्य॑ । सु॒षु॒वा॒णस्य॑ दश॒धा । द॒श॒धेन्द्रि॒यम् । द॒श॒धेति॑ दश - धा । इ॒न्द्रि॒यम् ॅवी॒र्य᳚म् । वी॒र्य॑म् परा᳚ । परा॑ऽपतत् । अ॒प॒त॒त् तत् । तद् दे॒वाः । दे॒वाः सौ᳚त्राम॒ण्या । सौ॒त्रा॒म॒ण्या सम् । सम॑भरन्न् । अ॒भ॒र॒न्थ् सू॒यते᳚ । सू॒यते॒ वै । वा ए॒षः । ए॒ष यः । यो᳚ऽग्निम् । अ॒ग्निम् चि॑नु॒ते । चि॒नु॒ते᳚ऽग्निम् । अ॒ग्निम् चि॒त्वा । चि॒त्वा सौ᳚त्राम॒ण्या । सौ॒त्रा॒म॒ण्या य॑जेत । य॒जे॒ते॒न्द्रि॒यम् । इ॒न्द्रि॒यमे॒व । ए॒व वी॒र्य᳚म् । वी॒र्यꣳ॑ स॒म्भृत्य॑ । स॒म्भृत्या॒त्मन्न् । स॒म्भृत्येति॑ सम् - भृत्य॑ । आ॒त्मन् ध॑त्ते । ध॒त्त॒ इति॑ धत्ते । \newline

\textbf{Jatai Paata} \newline

1. वै रा॑ज॒सूय॑स्य राज॒सूय॑स्य॒ वै वै रा॑ज॒सूय॑स्य । \newline
2. रा॒ज॒सूय॑स्य रू॒पꣳ रू॒पꣳ रा॑ज॒सूय॑स्य राज॒सूय॑स्य रू॒पम् । \newline
3. रा॒ज॒सूय॒स्येति॑ राज - सूय॑स्य । \newline
4. रू॒पं ॅयो यो रू॒पꣳ रू॒पं ॅयः । \newline
5. य ए॒व मे॒वं ॅयो य ए॒वम् । \newline
6. ए॒वं ॅवि॒द्वान्. वि॒द्वा ने॒व मे॒वं ॅवि॒द्वान् । \newline
7. वि॒द्वा न॒ग्नि म॒ग्निं ॅवि॒द्वान्. वि॒द्वा न॒ग्निम् । \newline
8. अ॒ग्निम् चि॑नु॒ते चि॑नु॒ते᳚ ऽग्नि म॒ग्निम् चि॑नु॒ते । \newline
9. चि॒नु॒त उ॒भा वु॒भौ चि॑नु॒ते चि॑नु॒त उ॒भौ । \newline
10. उ॒भा वे॒वैवोभा वु॒भा वे॒व । \newline
11. ए॒व लो॒कौ लो॒का वे॒वैव लो॒कौ । \newline
12. लो॒का व॒भ्य॑भि लो॒कौ लो॒का व॒भि । \newline
13. अ॒भि ज॑यति जय त्य॒भ्य॑भि ज॑यति । \newline
14. ज॒य॒ति॒ यो यो ज॑यति जयति॒ यः । \newline
15. यश्च॑ च॒ यो यश्च॑ । \newline
16. च॒ रा॒ज॒सूये॑न राज॒सूये॑न च च राज॒सूये॑न । \newline
17. रा॒ज॒सूये॑ नेजा॒न स्ये॑जा॒नस्य॑ राज॒सूये॑न राज॒सूये॑ नेजा॒नस्य॑ । \newline
18. रा॒ज॒सूये॒नेति॑ राज - सूये॑न । \newline
19. ई॒जा॒नस्य॒ यो य ई॑जा॒न स्ये॑जा॒नस्य॒ यः । \newline
20. यश्च॑ च॒ यो यश्च॑ । \newline
21. चा॒ग्नि॒चितो᳚ ऽग्नि॒चित॑श्च चाग्नि॒चितः॑ । \newline
22. अ॒ग्नि॒चित॒ इन्द्र॒ स्येन्द्र॑स्या ग्नि॒चितो᳚ ऽग्नि॒चित॒ इन्द्र॑स्य । \newline
23. अ॒ग्नि॒चित॒ इत्य॑ग्नि - चितः॑ । \newline
24. इन्द्र॑स्य सुषुवा॒णस्य॑ सुषुवा॒ण स्येन्द्र॒ स्येन्द्र॑स्य सुषुवा॒णस्य॑ । \newline
25. सु॒षु॒वा॒णस्य॑ दश॒धा द॑श॒धा सु॑षुवा॒णस्य॑ सुषुवा॒णस्य॑ दश॒धा । \newline
26. द॒श॒धेन्द्रि॒य मि॑न्द्रि॒यम् द॑श॒धा द॑श॒धेन्द्रि॒यम् । \newline
27. द॒श॒धेति॑ दश - धा । \newline
28. इ॒न्द्रि॒यं ॅवी॒र्यं॑ ॅवी॒र्य॑ मिन्द्रि॒य मि॑न्द्रि॒यं ॅवी॒र्य᳚म् । \newline
29. वी॒र्य॑म् परा॒ परा॑ वी॒र्यं॑ ॅवी॒र्य॑म् परा᳚ । \newline
30. परा॑ ऽपत दपत॒त् परा॒ परा॑ ऽपतत् । \newline
31. अ॒प॒त॒त् तत् तद॑पत दपत॒त् तत् । \newline
32. तद् दे॒वा दे॒वा स्तत् तद् दे॒वाः । \newline
33. दे॒वाः सौ᳚त्राम॒ण्या सौ᳚त्राम॒ण्या दे॒वा दे॒वाः सौ᳚त्राम॒ण्या । \newline
34. सौ॒त्रा॒म॒ण्या सꣳ सꣳ सौ᳚त्राम॒ण्या सौ᳚त्राम॒ण्या सम् । \newline
35. स म॑भरन् नभर॒न् थ्सꣳ स म॑भरन्न् । \newline
36. अ॒भ॒र॒न् थ्सू॒यते॑ सू॒यते॑ ऽभरन् नभरन् थ्सू॒यते᳚ । \newline
37. सू॒यते॒ वै वै सू॒यते॑ सू॒यते॒ वै । \newline
38. वा ए॒ष ए॒ष वै वा ए॒षः । \newline
39. ए॒ष यो य ए॒ष ए॒ष यः । \newline
40. यो᳚ ऽग्नि म॒ग्निं ॅयो यो᳚ ऽग्निम् । \newline
41. अ॒ग्निम् चि॑नु॒ते चि॑नु॒ते᳚ ऽग्नि म॒ग्निम् चि॑नु॒ते । \newline
42. चि॒नु॒ते᳚ ऽग्नि म॒ग्निम् चि॑नु॒ते चि॑नु॒ते᳚ ऽग्निम् । \newline
43. अ॒ग्निम् चि॒त्वा चि॒त्वा ऽग्नि म॒ग्निम् चि॒त्वा । \newline
44. चि॒त्वा सौ᳚त्राम॒ण्या सौ᳚त्राम॒ण्या चि॒त्वा चि॒त्वा सौ᳚त्राम॒ण्या । \newline
45. सौ॒त्रा॒म॒ण्या य॑जेत यजेत सौत्राम॒ण्या सौ᳚त्राम॒ण्या य॑जेत । \newline
46. य॒जे॒ते॒न्द्रि॒य मि॑न्द्रि॒यं ॅय॑जेत यजेतेन्द्रि॒यम् । \newline
47. इ॒न्द्रि॒य मे॒वैवेन्द्रि॒य मि॑न्द्रि॒य मे॒व । \newline
48. ए॒व वी॒र्यं॑ ॅवी॒र्य॑ मे॒वैव वी॒र्य᳚म् । \newline
49. वी॒र्यꣳ॑ सं॒भृत्य॑ सं॒भृत्य॑ वी॒र्यं॑ ॅवी॒र्यꣳ॑ सं॒भृत्य॑ । \newline
50. सं॒भृ त्या॒त्मन् ना॒त्मन् थ्सं॒भृत्य॑ सं॒भृ त्या॒त्मन्न् । \newline
51. सं॒भृत्येति॑ सं - भृत्य॑ । \newline
52. आ॒त्मन् ध॑त्ते धत्त आ॒त्मन् ना॒त्मन् ध॑त्ते । \newline
53. ध॒त्त॒ इति॑ धत्ते । \newline

\textbf{Ghana Paata } \newline

1. वै रा॑ज॒सूय॑स्य राज॒सूय॑स्य॒ वै वै रा॑ज॒सूय॑स्य रू॒पꣳ रू॒पꣳ रा॑ज॒सूय॑स्य॒ वै वै रा॑ज॒सूय॑स्य रू॒पम् । \newline
2. रा॒ज॒सूय॑स्य रू॒पꣳ रू॒पꣳ रा॑ज॒सूय॑स्य राज॒सूय॑स्य रू॒पं ॅयो यो रू॒पꣳ रा॑ज॒सूय॑स्य राज॒सूय॑स्य रू॒पं ॅयः । \newline
3. रा॒ज॒सूय॒स्येति॑ राज - सूय॑स्य । \newline
4. रू॒पं ॅयो यो रू॒पꣳ रू॒पं ॅय ए॒व मे॒वं ॅयो रू॒पꣳ रू॒पं ॅय ए॒वम् । \newline
5. य ए॒व मे॒वं ॅयो य ए॒वं ॅवि॒द्वान्. वि॒द्वा ने॒वं ॅयो य ए॒वं ॅवि॒द्वान् । \newline
6. ए॒वं ॅवि॒द्वान्. वि॒द्वा ने॒व मे॒वं ॅवि॒द्वा न॒ग्नि म॒ग्निं ॅवि॒द्वा ने॒व मे॒वं ॅवि॒द्वा न॒ग्निम् । \newline
7. वि॒द्वा न॒ग्नि म॒ग्निं ॅवि॒द्वान्. वि॒द्वा न॒ग्निम् चि॑नु॒ते चि॑नु॒ते᳚ ऽग्निं ॅवि॒द्वान्. वि॒द्वा न॒ग्निम् चि॑नु॒ते । \newline
8. अ॒ग्निम् चि॑नु॒ते चि॑नु॒ते᳚ ऽग्नि म॒ग्निम् चि॑नु॒त उ॒भा वु॒भौ चि॑नु॒ते᳚ ऽग्नि म॒ग्निम् चि॑नु॒त उ॒भौ । \newline
9. चि॒नु॒त उ॒भा वु॒भौ चि॑नु॒ते चि॑नु॒त उ॒भा वे॒वै वोभौ चि॑नु॒ते चि॑नु॒त उ॒भा वे॒व । \newline
10. उ॒भा वे॒वै वोभा वु॒भा वे॒व लो॒कौ लो॒का वे॒वोभा वु॒भा वे॒व लो॒कौ । \newline
11. ए॒व लो॒कौ लो॒का वे॒वैव लो॒का व॒भ्य॑भि लो॒का वे॒वैव लो॒का व॒भि । \newline
12. लो॒का व॒भ्य॑भि लो॒कौ लो॒का व॒भि ज॑यति जय त्य॒भि लो॒कौ लो॒का व॒भि ज॑यति । \newline
13. अ॒भि ज॑यति जय त्य॒भ्य॑भि ज॑यति॒ यो यो ज॑य त्य॒भ्य॑भि ज॑यति॒ यः । \newline
14. ज॒य॒ति॒ यो यो ज॑यति जयति॒ यश्च॑ च॒ यो ज॑यति जयति॒ यश्च॑ । \newline
15. यश्च॑ च॒ यो यश्च॑ राज॒सूये॑न राज॒सूये॑न च॒ यो यश्च॑ राज॒सूये॑न । \newline
16. च॒ रा॒ज॒सूये॑न राज॒सूये॑न च च राज॒सूये॑ नेजा॒न स्ये॑जा॒नस्य॑ राज॒सूये॑न च च राज॒सूये॑ नेजा॒नस्य॑ । \newline
17. रा॒ज॒सूये॑ नेजा॒न स्ये॑जा॒नस्य॑ राज॒सूये॑न राज॒सूये॑ नेजा॒नस्य॒ यो य ई॑जा॒नस्य॑ राज॒सूये॑न राज॒सूये॑
नेजा॒नस्य॒ यः । \newline
18. रा॒ज॒सूये॒नेति॑ राज - सूये॑न । \newline
19. ई॒जा॒नस्य॒ यो य ई॑जा॒न स्ये॑जा॒नस्य॒ यश्च॑ च॒ य ई॑जा॒न स्ये॑जा॒नस्य॒ यश्च॑ । \newline
20. यश्च॑ च॒ यो यश्चा᳚ग्नि॒चितो᳚ ऽग्नि॒चित॑श्च॒ यो यश्चा᳚ग्नि॒चितः॑ । \newline
21. चा॒ग्नि॒चितो᳚ ऽग्नि॒चित॑श्च चाग्नि॒चित॒ इन्द्र॒ स्येन्द्र॑स्या ग्नि॒चित॑श्च चाग्नि॒चित॒ इन्द्र॑स्य । \newline
22. अ॒ग्नि॒चित॒ इन्द्र॒ स्येन्द्र॑स्या ग्नि॒चितो᳚ ऽग्नि॒चित॒ इन्द्र॑स्य सुषुवा॒णस्य॑ सुषुवा॒ण स्येन्द्र॑स्या ग्नि॒चितो᳚ ऽग्नि॒चित॒ इन्द्र॑स्य सुषुवा॒णस्य॑ । \newline
23. अ॒ग्नि॒चित॒ इत्य॑ग्नि - चितः॑ । \newline
24. इन्द्र॑स्य सुषुवा॒णस्य॑ सुषुवा॒ण स्येन्द्र॒ स्येन्द्र॑स्य सुषुवा॒णस्य॑ दश॒धा द॑श॒धा सु॑षुवा॒ण
स्येन्द्र॒ स्येन्द्र॑स्य सुषुवा॒णस्य॑ दश॒धा । \newline
25. सु॒षु॒वा॒णस्य॑ दश॒धा द॑श॒धा सु॑षुवा॒णस्य॑ सुषुवा॒णस्य॑ दश॒धेन्द्रि॒य मि॑न्द्रि॒यम् द॑श॒धा सु॑षुवा॒णस्य॑ सुषुवा॒णस्य॑ दश॒धेन्द्रि॒यम् । \newline
26. द॒श॒धेन्द्रि॒य मि॑न्द्रि॒यम् द॑श॒धा द॑श॒धेन्द्रि॒यं ॅवी॒र्यं॑ ॅवी॒र्य॑ मिन्द्रि॒यम् द॑श॒धा द॑श॒धेन्द्रि॒यं ॅवी॒र्य᳚म् । \newline
27. द॒श॒धेति॑ दश - धा । \newline
28. इ॒न्द्रि॒यं ॅवी॒र्यं॑ ॅवी॒र्य॑ मिन्द्रि॒य मि॑न्द्रि॒यं ॅवी॒र्य॑म् परा॒ परा॑ वी॒र्य॑ मिन्द्रि॒य मि॑न्द्रि॒यं ॅवी॒र्य॑म् परा᳚ । \newline
29. वी॒र्य॑म् परा॒ परा॑ वी॒र्यं॑ ॅवी॒र्य॑म् परा॑ ऽपत दपत॒त् परा॑ वी॒र्यं॑ ॅवी॒र्य॑म् परा॑ ऽपतत् । \newline
30. परा॑ ऽपत दपत॒त् परा॒ परा॑ ऽपत॒त् तत् तद॑पत॒त् परा॒ परा॑ ऽपत॒त् तत् । \newline
31. अ॒प॒त॒त् तत् तद॑पत दपत॒त् तद् दे॒वा दे॒वा स्तद॑पत दपत॒त् तद् दे॒वाः । \newline
32. तद् दे॒वा दे॒वा स्तत् तद् दे॒वाः सौ᳚त्राम॒ण्या सौ᳚त्राम॒ण्या दे॒वा स्तत् तद् दे॒वाः सौ᳚त्राम॒ण्या । \newline
33. दे॒वाः सौ᳚त्राम॒ण्या सौ᳚त्राम॒ण्या दे॒वा दे॒वाः सौ᳚त्राम॒ण्या सꣳ सꣳ सौ᳚त्राम॒ण्या दे॒वा दे॒वाः सौ᳚त्राम॒ण्या सम् । \newline
34. सौ॒त्रा॒म॒ण्या सꣳ सꣳ सौ᳚त्राम॒ण्या सौ᳚त्राम॒ण्या स म॑भरन् नभर॒न् थ्सꣳ सौ᳚त्राम॒ण्या सौ᳚त्राम॒ण्या स म॑भरन्न् । \newline
35. स म॑भरन् नभर॒न् थ्सꣳ स म॑भरन् थ्सू॒यते॑ सू॒यते॑ ऽभर॒न् थ्सꣳ स म॑भरन् थ्सू॒यते᳚ । \newline
36. अ॒भ॒र॒न् थ्सू॒यते॑ सू॒यते॑ ऽभरन् नभरन् थ्सू॒यते॒ वै वै सू॒यते॑ ऽभरन् नभरन् थ्सू॒यते॒ वै । \newline
37. सू॒यते॒ वै वै सू॒यते॑ सू॒यते॒ वा ए॒ष ए॒ष वै सू॒यते॑ सू॒यते॒ वा ए॒षः । \newline
38. वा ए॒ष ए॒ष वै वा ए॒ष यो य ए॒ष वै वा ए॒ष यः । \newline
39. ए॒ष यो य ए॒ष ए॒ष यो᳚ ऽग्नि म॒ग्निं ॅय ए॒ष ए॒ष यो᳚ ऽग्निम् । \newline
40. यो᳚ ऽग्नि म॒ग्निं ॅयो यो᳚ ऽग्निम् चि॑नु॒ते चि॑नु॒ते᳚ ऽग्निं ॅयो यो᳚ ऽग्निम् चि॑नु॒ते । \newline
41. अ॒ग्निम् चि॑नु॒ते चि॑नु॒ते᳚ ऽग्नि म॒ग्निम् चि॑नु॒ते᳚ ऽग्नि म॒ग्निम् चि॑नु॒ते᳚ ऽग्नि म॒ग्निम् चि॑नु॒ते᳚ ऽग्निम् । \newline
42. चि॒नु॒ते᳚ ऽग्नि म॒ग्निम् चि॑नु॒ते चि॑नु॒ते᳚ ऽग्निम् चि॒त्वा चि॒त्वा ऽग्निम् चि॑नु॒ते चि॑नु॒ते᳚ ऽग्निम् चि॒त्वा । \newline
43. अ॒ग्निम् चि॒त्वा चि॒त्वा ऽग्नि म॒ग्निम् चि॒त्वा सौ᳚त्राम॒ण्या सौ᳚त्राम॒ण्या चि॒त्वा ऽग्नि म॒ग्निम् चि॒त्वा सौ᳚त्राम॒ण्या । \newline
44. चि॒त्वा सौ᳚त्राम॒ण्या सौ᳚त्राम॒ण्या चि॒त्वा चि॒त्वा सौ᳚त्राम॒ण्या य॑जेत यजेत सौत्राम॒ण्या चि॒त्वा चि॒त्वा सौ᳚त्राम॒ण्या य॑जेत । \newline
45. सौ॒त्रा॒म॒ण्या य॑जेत यजेत सौत्राम॒ण्या सौ᳚त्राम॒ण्या य॑जेतेन्द्रि॒य मि॑न्द्रि॒यं ॅय॑जेत सौत्राम॒ण्या सौ᳚त्राम॒ण्या य॑जेतेन्द्रि॒यम् । \newline
46. य॒जे॒ते॒न्द्रि॒य मि॑न्द्रि॒यं ॅय॑जेत यजेतेन्द्रि॒य मे॒वैवेन्द्रि॒यं ॅय॑जेत यजेतेन्द्रि॒य मे॒व । \newline
47. इ॒न्द्रि॒य मे॒वैवेन्द्रि॒य मि॑न्द्रि॒य मे॒व वी॒र्यं॑ ॅवी॒र्य॑ मे॒वेन्द्रि॒य मि॑न्द्रि॒य मे॒व वी॒र्य᳚म् । \newline
48. ए॒व वी॒र्यं॑ ॅवी॒र्य॑ मे॒वैव वी॒र्यꣳ॑ सं॒भृत्य॑ सं॒भृत्य॑ वी॒र्य॑ मे॒वैव वी॒र्यꣳ॑ सं॒भृत्य॑ । \newline
49. वी॒र्यꣳ॑ सं॒भृत्य॑ सं॒भृत्य॑ वी॒र्यं॑ ॅवी॒र्यꣳ॑ सं॒भृत्या॒त्मन् ना॒त्मन् थ्सं॒भृत्य॑ वी॒र्यं॑ ॅवी॒र्यꣳ॑ सं॒भृत्या॒त्मन्न् । \newline
50. सं॒भृत्या॒त्मन् ना॒त्मन् थ्सं॒भृत्य॑ सं॒भृत्या॒त्मन् ध॑त्ते धत्त आ॒त्मन् थ्सं॒भृत्य॑ सं॒भृत्या॒त्मन् ध॑त्ते । \newline
51. सं॒भृत्येति॑ सं - भृत्य॑ । \newline
52. आ॒त्मन् ध॑त्ते धत्त आ॒त्मन् ना॒त्मन् ध॑त्ते । \newline
53. ध॒त्त॒ इति॑ धत्ते । \newline
\pagebreak
\markright{ TS 5.6.4.1  \hfill https://www.vedavms.in \hfill}

\section{ TS 5.6.4.1 }

\textbf{TS 5.6.4.1 } \newline
\textbf{Samhita Paata} \newline

स॒जूरब्दोऽया॑वभिः स॒जूरु॒षा अरु॑णीभिः स॒जूः सूर्य॒ एत॑शेन स॒जोषा॑व॒श्विना॒ दꣳसो॑भिः स॒जूर॒ग्निर्वै᳚श्वान॒र इडा॑भिर्घृ॒तेन॒ स्वाहा॑ संॅवथ्स॒रो वा अब्दो॒ मासा॒ अया॑वा उ॒षा अरु॑णी॒ सूर्य॒ एत॑श इ॒मे अ॒श्विना॑ संॅवथ्स॒रो᳚ऽग्निर्वै᳚श्वान॒रः प॒शव॒ इडा॑ प॒शवो॑ घृ॒तꣳ सं॑ॅवथ्स॒रं प॒शवोऽनु॒ प्र जा॑यन्ते संॅवथ्स॒रेणै॒वास्मै॑ प॒शून् प्रज॑नयति दर्भस्त॒म्बे जु॑होति॒ य - [  ] \newline

\textbf{Pada Paata} \newline

स॒जूरिति॑ स - जूः । अब्दः॑ । अया॑वभि॒रित्यया॑व - भिः॒ । स॒जूरिति॑ स - जूः । उ॒षाः । अरु॑णीभिः । स॒जूरिति॑ स - जूः । सूर्यः॑ । एत॑शेन । स॒जोषा॒विति॑ स - जोषौ᳚ । अ॒श्विना᳚ । दꣳसो॑भि॒रिति॒ दꣳसः॑ - भिः॒ । स॒जूरिति॑ स-जूः । अ॒ग्निः । वै॒श्वा॒न॒रः । इडा॑भिः । घृ॒तेन॑ । स्वाहा᳚ । सं॒ॅव॒थ्स॒र इति॑ सं-व॒थ्स॒रः । वै । अब्दः॑ । मासाः᳚ । अया॑वाः । उ॒षाः । अरु॑णी । सूर्यः॑ । एत॑शः । इ॒मे इति॑ । अ॒श्विना᳚ । सं॒ॅव॒थ्स॒र इति॑ सं - व॒थ्स॒रः । अ॒ग्निः । वै॒श्वा॒न॒रः । प॒शवः॑ । इडा᳚ । प॒शवः॑ । घृ॒तम् । सं॒ॅव॒थ्स॒रमिति॑ सं - व॒थ्स॒रम् । प॒शवः॑ । अनु॑ । प्रेति॑ । जा॒य॒न्ते॒ । सं॒ॅव॒थ्स॒रेणेति॑ सं - व॒थ्स॒रेण॑ । ए॒व । अ॒स्मै॒ । प॒शून् । प्रेति॑ । ज॒न॒य॒ति॒ । द॒र्भ॒स्त॒बं इति॑ दर्भ - स्त॒बें । जु॒हो॒ति॒ । यत् ।  \newline


\textbf{Krama Paata} \newline

स॒जूरब्दः॑ । स॒जूरिति॑ स - जूः । अब्दोऽया॑वभिः । अया॑वभिः स॒जूः । अया॑वभि॒रित्यया॑व - भिः॒ । स॒जूरु॒षाः । स॒जूरिति॑ स - जूः । उ॒षा अरु॑णीभिः । अरु॑णीभिः स॒जूः । स॒जूः सूर्यः॑ । स॒जूरिति॑ स - जूः । सूर्य॒ एत॑शेन । एत॑शेन स॒जोषौ᳚ । स॒जोषा॑व॒श्विना᳚ । स॒जोषा॒विति॑ स - जोषौ᳚ । अ॒श्विना॒ दꣳसो॑भिः । दꣳसो॑भिः स॒जूः । दꣳसो॑भि॒रिति॒दꣳसः॑ - भिः॒ । स॒जूर॒ग्निः । स॒जूरिति॑ स - जूः । अ॒ग्निर् वै᳚श्वान॒रः । वै॒श्वा॒न॒र इडा॑भिः । इडा॑भिर् घृ॒तेन॑ । घृ॒तेन॒ स्वाहा᳚ । स्वाहा॑ सम्ॅवथ्स॒रः । स॒म्ॅव॒थ्स॒रो वै । स॒म्ॅव॒थ्स॒र इति॑ सम् - व॒थ्स॒रः । वा अब्दः॑ । अब्दो॒ मासाः᳚ । मासा॒ अया॑वाः । अया॑वा उ॒षाः । उ॒षा अरु॑णी । अरु॑णी॒ सूर्यः॑ । सूर्य॒ एत॑शः । एत॑श इ॒मे । इ॒मे अ॒श्विना᳚ । इमे॒ इती॒मे । अ॒श्विना॑ सम्ॅवथ्स॒रः । स॒म्ॅव॒थ्स॒रो᳚ऽग्निः । स॒म्ॅव॒थ्स॒र इति॑ सम् - व॒थ्स॒रः । अ॒ग्निर् वै᳚श्वान॒रः । वै॒श्वा॒न॒रः प॒शवः॑ । प॒शव॒ इडा᳚ । इडा॑ प॒शवः॑ । प॒शवो॑ घृ॒तम् । घृ॒तꣳ स॑म्ॅवथ्स॒रम् । स॒म्ॅव॒थ्स॒रम् प॒शवः॑ । स॒म्ॅव॒थ्स॒रमिति॑ सम् - व॒थ्स॒रम् । प॒शवोऽनु॑ । अनु॒ प्र । प्र जा॑यन्ते । जा॒य॒न्ते॒ स॒म्ॅव॒थ्स॒रेण॑ । स॒म्ॅव॒थ्स॒रेणै॒व । स॒म्ॅव॒थ्स॒रेणेति॑ सम् - व॒थ्स॒रेण॑ । ए॒वास्मै᳚ । अ॒स्मै॒ प॒शून् । प॒शून् प्र । प्र ज॑नयति । ज॒न॒य॒ति॒ द॒र्भ॒स्त॒म्बे । द॒र्भ॒स्त॒म्बे जु॑होति । द॒र्भ॒स्त॒म्ब इति॑ दर्भ - स्त॒म्बे । जु॒हो॒ति॒ यत् । यद् वै \newline

\textbf{Jatai Paata} \newline

1. स॒जू रब्दो ऽब्दः॑ स॒जूः स॒जू रब्दः॑ । \newline
2. स॒जूरिति॑ स - जूः । \newline
3. अब्दो ऽया॑वभि॒ रया॑वभि॒ रब्दो ऽब्दो ऽया॑वभिः । \newline
4. अया॑वभिः स॒जूः स॒जू रया॑वभि॒ रया॑वभिः स॒जूः । \newline
5. अया॑वभि॒रित्यया॑व - भिः॒ । \newline
6. स॒जू रु॒षा उ॒षाः स॒जूः स॒जू रु॒षाः । \newline
7. स॒जूरिति॑ स - जूः । \newline
8. उ॒षा अरु॑णीभि॒ ररु॑णीभि रु॒षा उ॒षा अरु॑णीभिः । \newline
9. अरु॑णीभिः स॒जूः स॒जू ररु॑णीभि॒ ररु॑णीभिः स॒जूः । \newline
10. स॒जूः सूर्यः॒ सूर्यः॑ स॒जूः स॒जूः सूर्यः॑ । \newline
11. स॒जूरिति॑ स - जूः । \newline
12. सूर्य॒ एत॑शे॒ नैत॑शेन॒ सूर्यः॒ सूर्य॒ एत॑शेन । \newline
13. एत॑शेन स॒जोषौ॑ स॒जोषा॒ वेत॑शे॒ नैत॑शेन स॒जोषौ᳚ । \newline
14. स॒जोषा॑ व॒श्विना॒ ऽश्विना॑ स॒जोषौ॑ स॒जोषा॑ व॒श्विना᳚ । \newline
15. स॒जोषा॒विति॑ स - जोषौ᳚ । \newline
16. अ॒श्विना॒ दꣳसो॑भि॒र् दꣳसो॑भि र॒श्विना॒ ऽश्विना॒ दꣳसो॑भिः । \newline
17. दꣳसो॑भिः स॒जूः स॒जूर् दꣳसो॑भि॒र् दꣳसो॑भिः स॒जूः । \newline
18. दꣳसो॑भि॒रिति॒ दꣳसः॑ - भिः॒ । \newline
19. स॒जू र॒ग्नि र॒ग्निः स॒जूः स॒जू र॒ग्निः । \newline
20. स॒जूरिति॑ स - जूः । \newline
21. अ॒ग्निर् वै᳚श्वान॒रो वै᳚श्वान॒रो᳚ ऽग्नि र॒ग्निर् वै᳚श्वान॒रः । \newline
22. वै॒श्वा॒न॒र इडा॑भि॒ रिडा॑भिर् वैश्वान॒रो वै᳚श्वान॒र इडा॑भिः । \newline
23. इडा॑भिर् घृ॒तेन॑ घृ॒तेने डा॑भि॒ रिडा॑भिर् घृ॒तेन॑ । \newline
24. घृ॒तेन॒ स्वाहा॒ स्वाहा॑ घृ॒तेन॑ घृ॒तेन॒ स्वाहा᳚ । \newline
25. स्वाहा॑ संॅवथ्स॒रः सं॑ॅवथ्स॒रः स्वाहा॒ स्वाहा॑ संॅवथ्स॒रः । \newline
26. सं॒ॅव॒थ्स॒रो वै वै सं॑ॅवथ्स॒रः सं॑ॅवथ्स॒रो वै । \newline
27. सं॒ॅव॒थ्स॒र इति॑ सं - व॒थ्स॒रः । \newline
28. वा अब्दो ऽब्दो॒ वै वा अब्दः॑ । \newline
29. अब्दो॒ मासा॒ मासा॒ अब्दो ऽब्दो॒ मासाः᳚ । \newline
30. मासा॒ अया॑वा॒ अया॑वा॒ मासा॒ मासा॒ अया॑वाः । \newline
31. अया॑वा उ॒षा उ॒षा अया॑वा॒ अया॑वा उ॒षाः । \newline
32. उ॒षा अरु॒ ण्यरु॑ ण्यु॒षा उ॒षा अरु॑णी । \newline
33. अरु॑णी॒ सूर्यः॒ सूर्यो ऽरु॒ ण्यरु॑णी॒ सूर्यः॑ । \newline
34. सूर्य॒ एत॑श॒ एत॑शः॒ सूर्यः॒ सूर्य॒ एत॑शः । \newline
35. एत॑श इ॒मे इ॒मे एत॑श॒ एत॑श इ॒मे । \newline
36. इ॒मे अ॒श्विना॒ ऽश्विने॒मे इ॒मे अ॒श्विना᳚ । \newline
37. इ॒मे इती॒मे । \newline
38. अ॒श्विना॑ संॅवथ्स॒रः सं॑ॅवथ्स॒रो᳚ ऽश्विना॒ ऽश्विना॑ संॅवथ्स॒रः । \newline
39. सं॒ॅव॒थ्स॒रो᳚ ऽग्नि र॒ग्निः सं॑ॅवथ्स॒रः सं॑ॅवथ्स॒रो᳚ ऽग्निः । \newline
40. सं॒ॅव॒थ्स॒र इति॑ सं - व॒थ्स॒रः । \newline
41. अ॒ग्निर् वै᳚श्वान॒रो वै᳚श्वान॒रो᳚ ऽग्नि र॒ग्निर् वै᳚श्वान॒रः । \newline
42. वै॒श्वा॒न॒रः प॒शवः॑ प॒शवो॑ वैश्वान॒रो वै᳚श्वान॒रः प॒शवः॑ । \newline
43. प॒शव॒ इडेडा॑ प॒शवः॑ प॒शव॒ इडा᳚ । \newline
44. इडा॑ प॒शवः॑ प॒शव॒ इडेडा॑ प॒शवः॑ । \newline
45. प॒शवो॑ घृ॒तम् घृ॒तम् प॒शवः॑ प॒शवो॑ घृ॒तम् । \newline
46. घृ॒तꣳ सं॑ॅवथ्स॒रꣳ सं॑ॅवथ्स॒रम् घृ॒तम् घृ॒तꣳ सं॑ॅवथ्स॒रम् । \newline
47. सं॒ॅव॒थ्स॒रम् प॒शवः॑ प॒शवः॑ संॅवथ्स॒रꣳ सं॑ॅवथ्स॒रम् प॒शवः॑ । \newline
48. सं॒ॅव॒थ्स॒रमिति॑ सं - व॒थ्स॒रम् । \newline
49. प॒शवो ऽन्वनु॑ प॒शवः॑ प॒शवो ऽनु॑ । \newline
50. अनु॒ प्र प्राण्वनु॒ प्र । \newline
51. प्र जा॑यन्ते जायन्ते॒ प्र प्र जा॑यन्ते । \newline
52. जा॒य॒न्ते॒ सं॒ॅव॒थ्स॒रेण॑ संॅवथ्स॒रेण॑ जायन्ते जायन्ते संॅवथ्स॒रेण॑ । \newline
53. सं॒ॅव॒थ्स॒रे णै॒वैव सं॑ॅवथ्स॒रेण॑ संॅवथ्स॒रेणै॒व । \newline
54. सं॒ॅव॒थ्स॒रेणेति॑ सं - व॒थ्स॒रेण॑ । \newline
55. ए॒वास्मा॑ अस्मा ए॒वै वास्मै᳚ । \newline
56. अ॒स्मै॒ प॒शून् प॒शू न॑स्मा अस्मै प॒शून् । \newline
57. प॒शून् प्र प्र प॒शून् प॒शून् प्र । \newline
58. प्र ज॑नयति जनयति॒ प्र प्र ज॑नयति । \newline
59. ज॒न॒य॒ति॒ द॒र्भ॒स्तं॒बे द॑र्भस्तं॒बे ज॑नयति जनयति दर्भस्तं॒बे । \newline
60. द॒र्भ॒स्तं॒बे जु॑होति जुहोति दर्भस्तं॒बे द॑र्भस्तं॒बे जु॑होति । \newline
61. द॒र्भ॒स्तं॒ब इति॑ दर्भ - स्तं॒बे । \newline
62. जु॒हो॒ति॒ यद् यज् जु॑होति जुहोति॒ यत् । \newline
63. यद् वै वै यद् यद् वै । \newline

\textbf{Ghana Paata } \newline

1. स॒जू रब्दो ऽब्दः॑ स॒जूः स॒जू रब्दो ऽया॑वभि॒ रया॑वभि॒ रब्दः॑ स॒जूः स॒जू रब्दो ऽया॑वभिः । \newline
2. स॒जूरिति॑ स - जूः । \newline
3. अब्दो ऽया॑वभि॒ रया॑वभि॒ रब्दो ऽब्दो ऽया॑वभिः स॒जूः स॒जू रया॑वभि॒ रब्दो ऽब्दो ऽया॑वभिः स॒जूः । \newline
4. अया॑वभिः स॒जूः स॒जू रया॑वभि॒ रया॑वभिः स॒जू रु॒षा उ॒षाः स॒जू रया॑वभि॒ रया॑वभिः स॒जू रु॒षाः । \newline
5. अया॑वभि॒रित्यया॑व - भिः॒ । \newline
6. स॒जू रु॒षा उ॒षाः स॒जूः स॒जू रु॒षा अरु॑णीभि॒ ररु॑णीभि रु॒षाः स॒जूः स॒जू रु॒षा अरु॑णीभिः । \newline
7. स॒जूरिति॑ स - जूः । \newline
8. उ॒षा अरु॑णीभि॒ ररु॑णीभि रु॒षा उ॒षा अरु॑णीभिः स॒जूः स॒जू ररु॑णीभि रु॒षा उ॒षा अरु॑णीभिः स॒जूः । \newline
9. अरु॑णीभिः स॒जूः स॒जू ररु॑णीभि॒ ररु॑णीभिः स॒जूः सूर्यः॒ सूर्यः॑ स॒जू ररु॑णीभि॒ ररु॑णीभिः स॒जूः सूर्यः॑ । \newline
10. स॒जूः सूर्यः॒ सूर्यः॑ स॒जूः स॒जूः सूर्य॒ एत॑शे॒ नैत॑शेन॒ सूर्यः॑ स॒जूः स॒जूः सूर्य॒ एत॑शेन । \newline
11. स॒जूरिति॑ स - जूः । \newline
12. सूर्य॒ एत॑शे॒ नैत॑शेन॒ सूर्यः॒ सूर्य॒ एत॑शेन स॒जोषौ॑ स॒जोषा॒ वेत॑शेन॒ सूर्यः॒ सूर्य॒ एत॑शेन स॒जोषौ᳚ । \newline
13. एत॑शेन स॒जोषौ॑ स॒जोषा॒ वेत॑शे॒ नैत॑शेन स॒जोषा॑ व॒श्विना॒ ऽश्विना॑ स॒जोषा॒ वेत॑शे॒ नैत॑शेन स॒जोषा॑ व॒श्विना᳚ । \newline
14. स॒जोषा॑ व॒श्विना॒ ऽश्विना॑ स॒जोषौ॑ स॒जोषा॑ व॒श्विना॒ दꣳसो॑भि॒र् दꣳसो॑भि र॒श्विना॑ स॒जोषौ॑ स॒जोषा॑ व॒श्विना॒ दꣳसो॑भिः । \newline
15. स॒जोषा॒विति॑ स - जोषौ᳚ । \newline
16. अ॒श्विना॒ दꣳसो॑भि॒र् दꣳसो॑भि र॒श्विना॒ ऽश्विना॒ दꣳसो॑भिः स॒जूः स॒जूर् दꣳसो॑भि र॒श्विना॒ ऽश्विना॒ दꣳसो॑भिः स॒जूः । \newline
17. दꣳसो॑भिः स॒जूः स॒जूर् दꣳसो॑भि॒र् दꣳसो॑भिः स॒जू र॒ग्नि र॒ग्निः स॒जूर् दꣳसो॑भि॒र् दꣳसो॑भिः स॒जू र॒ग्निः । \newline
18. दꣳसो॑भि॒रिति॒ दꣳसः॑ - भिः॒ । \newline
19. स॒जू र॒ग्नि र॒ग्निः स॒जूः स॒जू र॒ग्निर् वै᳚श्वान॒रो वै᳚श्वान॒रो᳚ ऽग्निः स॒जूः स॒जू र॒ग्निर् वै᳚श्वान॒रः । \newline
20. स॒जूरिति॑ स - जूः । \newline
21. अ॒ग्निर् वै᳚श्वान॒रो वै᳚श्वान॒रो᳚ ऽग्नि र॒ग्निर् वै᳚श्वान॒र इडा॑भि॒ रिडा॑भिर् वैश्वान॒रो᳚ ऽग्नि र॒ग्निर् वै᳚श्वान॒र इडा॑भिः । \newline
22. वै॒श्वा॒न॒र इडा॑भि॒ रिडा॑भिर् वैश्वान॒रो वै᳚श्वान॒र इडा॑भिर् घृ॒तेन॑ घृ॒ते नेडा॑भिर् वैश्वान॒रो वै᳚श्वान॒र इडा॑भिर् घृ॒तेन॑ । \newline
23. इडा॑भिर् घृ॒तेन॑ घृ॒ते नेडा॑भि॒ रिडा॑भिर् घृ॒तेन॒ स्वाहा॒ स्वाहा॑ घृ॒ते नेडा॑भि॒ रिडा॑भिर् घृ॒तेन॒ स्वाहा᳚ । \newline
24. घृ॒तेन॒ स्वाहा॒ स्वाहा॑ घृ॒तेन॑ घृ॒तेन॒ स्वाहा॑ संॅवथ्स॒रः सं॑ॅवथ्स॒रः स्वाहा॑ घृ॒तेन॑ घृ॒तेन॒ स्वाहा॑ संॅवथ्स॒रः । \newline
25. स्वाहा॑ संॅवथ्स॒रः सं॑ॅवथ्स॒रः स्वाहा॒ स्वाहा॑ संॅवथ्स॒रो वै वै सं॑ॅवथ्स॒रः स्वाहा॒ स्वाहा॑ संॅवथ्स॒रो वै । \newline
26. सं॒ॅव॒थ्स॒रो वै वै सं॑ॅवथ्स॒रः सं॑ॅवथ्स॒रो वा अब्दो ऽब्दो॒ वै सं॑ॅवथ्स॒रः सं॑ॅवथ्स॒रो वा अब्दः॑ । \newline
27. सं॒ॅव॒थ्स॒र इति॑ सं - व॒थ्स॒रः । \newline
28. वा अब्दो ऽब्दो॒ वै वा अब्दो॒ मासा॒ मासा॒ अब्दो॒ वै वा अब्दो॒ मासाः᳚ । \newline
29. अब्दो॒ मासा॒ मासा॒ अब्दो ऽब्दो॒ मासा॒ अया॑वा॒ अया॑वा॒ मासा॒ अब्दो ऽब्दो॒ मासा॒ अया॑वाः । \newline
30. मासा॒ अया॑वा॒ अया॑वा॒ मासा॒ मासा॒ अया॑वा उ॒षा उ॒षा अया॑वा॒ मासा॒ मासा॒ अया॑वा उ॒षाः । \newline
31. अया॑वा उ॒षा उ॒षा अया॑वा॒ अया॑वा उ॒षा अरु॒ ण्यरु॑ ण्यु॒षा अया॑वा॒ अया॑वा उ॒षा अरु॑णी । \newline
32. उ॒षा अरु॒ ण्यरु॑ ण्यु॒षा उ॒षा अरु॑णी॒ सूर्यः॒ सूर्यो ऽरु॑ ण्यु॒षा उ॒षा अरु॑णी॒ सूर्यः॑ । \newline
33. अरु॑णी॒ सूर्यः॒ सूर्यो ऽरु॒ ण्यरु॑णी॒ सूर्य॒ एत॑श॒ एत॑शः॒ सूर्यो ऽरु॒ ण्यरु॑णी॒ सूर्य॒ एत॑शः । \newline
34. सूर्य॒ एत॑श॒ एत॑शः॒ सूर्यः॒ सूर्य॒ एत॑श इ॒मे इ॒मे एत॑शः॒ सूर्यः॒ सूर्य॒ एत॑श इ॒मे । \newline
35. एत॑श इ॒मे इ॒मे एत॑श॒ एत॑श इ॒मे अ॒श्विना॒ ऽश्विने॒मे एत॑श॒ एत॑श इ॒मे अ॒श्विना᳚ । \newline
36. इ॒मे अ॒श्विना॒ ऽश्विने॒मे इ॒मे अ॒श्विना॑ संॅवथ्स॒रः सं॑ॅवथ्स॒रो᳚ ऽश्विने॒मे इ॒मे अ॒श्विना॑ संॅवथ्स॒रः । \newline
37. इ॒मे इती॒मे । \newline
38. अ॒श्विना॑ संॅवथ्स॒रः सं॑ॅवथ्स॒रो᳚ ऽश्विना॒ ऽश्विना॑ संॅवथ्स॒रो᳚ ऽग्नि र॒ग्निः सं॑ॅवथ्स॒रो᳚ ऽश्विना॒ ऽश्विना॑ संॅवथ्स॒रो᳚ ऽग्निः । \newline
39. सं॒ॅव॒थ्स॒रो᳚ ऽग्नि र॒ग्निः सं॑ॅवथ्स॒रः सं॑ॅवथ्स॒रो᳚ ऽग्निर् वै᳚श्वान॒रो वै᳚श्वान॒रो᳚ ऽग्निः सं॑ॅवथ्स॒रः सं॑ॅवथ्स॒रो᳚ ऽग्निर् वै᳚श्वान॒रः । \newline
40. सं॒ॅव॒थ्स॒र इति॑ सं - व॒थ्स॒रः । \newline
41. अ॒ग्निर् वै᳚श्वान॒रो वै᳚श्वान॒रो᳚ ऽग्नि र॒ग्निर् वै᳚श्वान॒रः प॒शवः॑ प॒शवो॑ वैश्वान॒रो᳚ ऽग्नि र॒ग्निर् वै᳚श्वान॒रः प॒शवः॑ । \newline
42. वै॒श्वा॒न॒रः प॒शवः॑ प॒शवो॑ वैश्वान॒रो वै᳚श्वान॒रः प॒शव॒ इडेडा॑ प॒शवो॑ वैश्वान॒रो वै᳚श्वान॒रः प॒शव॒ इडा᳚ । \newline
43. प॒शव॒ इडेडा॑ प॒शवः॑ प॒शव॒ इडा॑ प॒शवः॑ प॒शव॒ इडा॑ प॒शवः॑ प॒शव॒ इडा॑ प॒शवः॑ । \newline
44. इडा॑ प॒शवः॑ प॒शव॒ इडेडा॑ प॒शवो॑ घृ॒तम् घृ॒तम् प॒शव॒ इडेडा॑ प॒शवो॑ घृ॒तम् । \newline
45. प॒शवो॑ घृ॒तम् घृ॒तम् प॒शवः॑ प॒शवो॑ घृ॒तꣳ सं॑ॅवथ्स॒रꣳ सं॑ॅवथ्स॒रम् घृ॒तम् प॒शवः॑ प॒शवो॑ घृ॒तꣳ सं॑ॅवथ्स॒रम् । \newline
46. घृ॒तꣳ सं॑ॅवथ्स॒रꣳ सं॑ॅवथ्स॒रम् घृ॒तम् घृ॒तꣳ सं॑ॅवथ्स॒रम् प॒शवः॑ प॒शवः॑ संॅवथ्स॒रम् घृ॒तम् घृ॒तꣳ सं॑ॅवथ्स॒रम् प॒शवः॑ । \newline
47. सं॒ॅव॒थ्स॒रम् प॒शवः॑ प॒शवः॑ संॅवथ्स॒रꣳ सं॑ॅवथ्स॒रम् प॒शवो ऽन्वनु॑ प॒शवः॑ संॅवथ्स॒रꣳ सं॑ॅवथ्स॒रम् प॒शवो ऽनु॑ । \newline
48. सं॒ॅव॒थ्स॒रमिति॑ सं - व॒थ्स॒रम् । \newline
49. प॒शवो ऽन्वनु॑ प॒शवः॑ प॒शवो ऽनु॒ प्र प्राणु॑ प॒शवः॑ प॒शवो ऽनु॒ प्र । \newline
50. अनु॒ प्र प्राण्वनु॒ प्र जा॑यन्ते जायन्ते॒ प्राण्वनु॒ प्र जा॑यन्ते । \newline
51. प्र जा॑यन्ते जायन्ते॒ प्र प्र जा॑यन्ते संॅवथ्स॒रेण॑ संॅवथ्स॒रेण॑ जायन्ते॒ प्र प्र जा॑यन्ते संॅवथ्स॒रेण॑ । \newline
52. जा॒य॒न्ते॒ सं॒ॅव॒थ्स॒रेण॑ संॅवथ्स॒रेण॑ जायन्ते जायन्ते संॅवथ्स॒रे णै॒वैव सं॑ॅवथ्स॒रेण॑ जायन्ते जायन्ते संॅवथ्स॒रेणै॒व । \newline
53. सं॒ॅव॒थ्स॒रे णै॒वैव सं॑ॅवथ्स॒रेण॑ संॅवथ्स॒रे णै॒वास्मा॑ अस्मा ए॒व सं॑ॅवथ्स॒रेण॑ संॅवथ्स॒रे णै॒वास्मै᳚ । \newline
54. सं॒ॅव॒थ्स॒रेणेति॑ सं - व॒थ्स॒रेण॑ । \newline
55. ए॒वास्मा॑ अस्मा ए॒वै वास्मै॑ प॒शून् प॒शू न॑स्मा ए॒वै वास्मै॑ प॒शून् । \newline
56. अ॒स्मै॒ प॒शून् प॒शू न॑स्मा अस्मै प॒शून् प्र प्र प॒शू न॑स्मा अस्मै प॒शून् प्र । \newline
57. प॒शून् प्र प्र प॒शून् प॒शून् प्र ज॑नयति जनयति॒ प्र प॒शून् प॒शून् प्र ज॑नयति । \newline
58. प्र ज॑नयति जनयति॒ प्र प्र ज॑नयति दर्भस्तं॒बे द॑र्भस्तं॒बे ज॑नयति॒ प्र प्र ज॑नयति दर्भस्तं॒बे । \newline
59. ज॒न॒य॒ति॒ द॒र्भ॒स्तं॒बे द॑र्भस्तं॒बे ज॑नयति जनयति दर्भस्तं॒बे जु॑होति जुहोति दर्भस्तं॒बे ज॑नयति जनयति दर्भस्तं॒बे जु॑होति । \newline
60. द॒र्भ॒स्तं॒बे जु॑होति जुहोति दर्भस्तं॒बे द॑र्भस्तं॒बे जु॑होति॒ यद् यज् जु॑होति दर्भस्तं॒बे द॑र्भस्तं॒बे जु॑होति॒ यत् । \newline
61. द॒र्भ॒स्तं॒ब इति॑ दर्भ - स्तं॒बे । \newline
62. जु॒हो॒ति॒ यद् यज् जु॑होति जुहोति॒ यद् वै वै यज् जु॑होति जुहोति॒ यद् वै । \newline
63. यद् वै वै यद् यद् वा अ॒स्या अ॒स्या वै यद् यद् वा अ॒स्याः । \newline
\pagebreak
\markright{ TS 5.6.4.2  \hfill https://www.vedavms.in \hfill}

\section{ TS 5.6.4.2 }

\textbf{TS 5.6.4.2 } \newline
\textbf{Samhita Paata} \newline

-द्वा अ॒स्या अ॒मृतं॒ ॅयद्-वी॒र्यं॑ तद्-द॒र्भास्तस्मि॑न् जुहोति॒ प्रैव जा॑यते ऽन्ना॒दो भ॑वति॒ यस्यै॒वं जुह्व॑त्ये॒ता वै दे॒वता॑ अ॒ग्नेः पु॒रस्ता᳚द्भागा॒स्ता ए॒व प्री॑णा॒त्यथो॒ चक्षु॑रे॒वाग्नेः पु॒रस्ता॒त् प्रति॑ दधा॒त्यन॑न्धो भवति॒ य ए॒वं ॅवेदाऽऽ*पो॒ वा इ॒दमग्रे॑ सलि॒लमा॑सी॒थ् स प्र॒जाप॑तिः पुष्करप॒र्णे वातो॑ भू॒तो॑ऽलेलाय॒थ् सः - [  ] \newline

\textbf{Pada Paata} \newline

वै । अ॒स्याः । अ॒मृत᳚म् । यत् । वी॒र्य᳚म् । तत् । द॒र्भाः । तस्मिन्न्॑ । जु॒हो॒ति॒ । प्रेति॑ । ए॒व । जा॒य॒ते॒ । अ॒न्ना॒द इत्य॑न्न - अ॒दः । भ॒व॒ति॒ । यस्य॑ । ए॒वम् । जुह्व॑ति । ए॒ताः । वै । दे॒वताः᳚ । अ॒ग्नेः । पु॒रस्ता᳚द्भागा॒ इति॑ पु॒रस्ता᳚त् - भा॒गाः॒ । ताः । ए॒व । प्री॒णा॒ति॒ । अथो॒ इति॑ । चक्षुः॑ । ए॒व । अ॒ग्नेः । पु॒रस्ता᳚त् । प्रतीति॑ । द॒धा॒ति॒ । अन॑न्धः । भ॒व॒ति॒ । यः । ए॒वम् । वेद॑ । आपः॑ । वै । इ॒दम् । अग्रे᳚ । स॒लि॒लम् । आ॒सी॒त् । सः । प्र॒जाप॑ति॒रिति॑ प्र॒जा - प॒तिः॒ । पु॒ष्क॒र॒प॒र्ण इति॑ पुष्कर - प॒र्णे । वातः॑ । भू॒तः । अ॒ले॒ला॒य॒त् । सः ।  \newline


\textbf{Krama Paata} \newline

वा अ॒स्याः । अ॒स्या अ॒मृत᳚म् । अ॒मृत॒म् ॅयत् । यद् वी॒र्य᳚म् । वी॒र्य॑म् तत् । तद् द॒र्भाः । द॒र्भास्तस्मिन्न्॑ । तस्मि॑न् जुहोति । जु॒हो॒ति॒ प्र । प्रैव । ए॒व जा॑यते । जा॒य॒ते॒ऽन्ना॒दः । अ॒न्ना॒दो भ॑वति । अ॒न्ना॒द इत्य॑न्न - अ॒दः । भ॒व॒ति॒ यस्य॑ । यस्यै॒वम् । ए॒वम् जुह्व॑ति । जुह्व॑त्ये॒ताः । ए॒ता वै । वै दे॒वताः᳚ । दे॒वता॑ अ॒ग्नेः । अ॒ग्नेः पु॒रस्ता᳚द्भागाः । पु॒रस्ता᳚द्भागा॒स्ताः । पु॒रस्ता᳚द्भागा॒ इति॑ पु॒रस्ता᳚त् - भा॒गाः॒ । ता ए॒व । ए॒व प्री॑णाति । प्री॒णा॒त्यथो᳚ । अथो॒ चक्षुः॑ । अथो॒ इत्यथो᳚ । चक्षु॑रे॒व । ए॒वाग्नेः । अ॒ग्नेः पु॒रस्ता᳚त् । पु॒रस्ता॒त् प्रति॑ । प्रति॑ दधाति । द॒धा॒त्यन॑न्धः । अन॑न्धो भवति । भ॒व॒ति॒ यः । य ए॒वम् । ए॒वम् ॅवेद॑ । वेदापः॑ । आपो॒ वै । वा इ॒दम् । इ॒दमग्रे᳚ । अग्रे॑ सलि॒लम् । स॒लि॒लमा॑सीत् । आ॒सी॒थ् सः । स प्र॒जाप॑तिः । प्र॒जाप॑तिः पुष्करप॒र्णे । प्र॒जाप॑ति॒रिति॑ प्र॒जा - प॒तिः॒ । पु॒ष्क॒र॒प॒र्णे वातः॑ । पु॒ष्क॒र॒प॒र्ण इति॑ पुष्कर - प॒र्णे । वातो॑ भू॒तः । भू॒तो॑ऽलेलायत् । अ॒ले॒ला॒य॒थ् सः । स प्र॑ति॒ष्ठाम् \newline

\textbf{Jatai Paata} \newline

1. वा अ॒स्या अ॒स्या वै वा अ॒स्याः । \newline
2. अ॒स्या अ॒मृत॑ म॒मृत॑ म॒स्या अ॒स्या अ॒मृत᳚म् । \newline
3. अ॒मृतं॒ ॅयद् यद॒मृत॑ म॒मृतं॒ ॅयत् । \newline
4. यद् वी॒र्यं॑ ॅवी॒र्यं॑ ॅयद् यद् वी॒र्य᳚म् । \newline
5. वी॒र्य॑म् तत् तद् वी॒र्यं॑ ॅवी॒र्य॑म् तत् । \newline
6. तद् द॒र्भा द॒र्भा स्तत् तद् द॒र्भाः । \newline
7. द॒र्भा स्तस्मिꣳ॒॒ स्तस्मि॑न् द॒र्भा द॒र्भा स्तस्मिन्न्॑ । \newline
8. तस्मि॑न् जुहोति जुहोति॒ तस्मिꣳ॒॒ स्तस्मि॑न् जुहोति । \newline
9. जु॒हो॒ति॒ प्र प्र जु॑होति जुहोति॒ प्र । \newline
10. प्रैवैव प्र प्रैव । \newline
11. ए॒व जा॑यते जायत ए॒वैव जा॑यते । \newline
12. जा॒य॒ते॒ ऽन्ना॒दो᳚ ऽन्ना॒दो जा॑यते जायते ऽन्ना॒दः । \newline
13. अ॒न्ना॒दो भ॑वति भव त्यन्ना॒दो᳚ ऽन्ना॒दो भ॑वति । \newline
14. अ॒न्ना॒द इत्य॑न्न - अ॒दः । \newline
15. भ॒व॒ति॒ यस्य॒ यस्य॑ भवति भवति॒ यस्य॑ । \newline
16. यस्यै॒व मे॒वं ॅयस्य॒ यस्यै॒वम् । \newline
17. ए॒वम् जुह्व॑ति॒ जुह्व॑ त्ये॒व मे॒वम् जुह्व॑ति । \newline
18. जुह्व॑ त्ये॒ता ए॒ता जुह्व॑ति॒ जुह्व॑ त्ये॒ताः । \newline
19. ए॒ता वै वा ए॒ता ए॒ता वै । \newline
20. वै दे॒वता॑ दे॒वता॒ वै वै दे॒वताः᳚ । \newline
21. दे॒वता॑ अ॒ग्ने र॒ग्नेर् दे॒वता॑ दे॒वता॑ अ॒ग्नेः । \newline
22. अ॒ग्नेः पु॒रस्ता᳚द्भागाः पु॒रस्ता᳚द्भागा अ॒ग्ने र॒ग्नेः पु॒रस्ता᳚द्भागाः । \newline
23. पु॒रस्ता᳚द्भागा॒ स्ता स्ताः पु॒रस्ता᳚द्भागाः पु॒रस्ता᳚द्भागा॒ स्ताः । \newline
24. पु॒रस्ता᳚द्भागा॒ इति॑ पु॒रस्ता᳚त् - भा॒गाः॒ । \newline
25. ता ए॒वैव ता स्ता ए॒व । \newline
26. ए॒व प्री॑णाति प्रीणा त्ये॒वैव प्री॑णाति । \newline
27. प्री॒णा॒ त्यथो॒ अथो᳚ प्रीणाति प्रीणा॒ त्यथो᳚ । \newline
28. अथो॒ चक्षु॒ श्चक्षु॒ रथो॒ अथो॒ चक्षुः॑ । \newline
29. अथो॒ इत्यथो᳚ । \newline
30. चक्षु॑ रे॒वैव चक्षु॒ श्चक्षु॑ रे॒व । \newline
31. ए॒वाग्ने र॒ग्ने रे॒वै वाग्नेः । \newline
32. अ॒ग्नेः पु॒रस्ता᳚त् पु॒रस्ता॑ द॒ग्ने र॒ग्नेः पु॒रस्ता᳚त् । \newline
33. पु॒रस्ता॒त् प्रति॒ प्रति॑ पु॒रस्ता᳚त् पु॒रस्ता॒त् प्रति॑ । \newline
34. प्रति॑ दधाति दधाति॒ प्रति॒ प्रति॑ दधाति । \newline
35. द॒धा॒ त्यन॒न्धो ऽन॑न्धो दधाति दधा॒ त्यन॑न्धः । \newline
36. अन॑न्धो भवति भव॒ त्यन॒न्धो ऽन॑न्धो भवति । \newline
37. भ॒व॒ति॒ यो यो भ॑वति भवति॒ यः । \newline
38. य ए॒व मे॒वं ॅयो य ए॒वम् । \newline
39. ए॒वं ॅवेद॒ वेदै॒व मे॒वं ॅवेद॑ । \newline
40. वेदाप॒ आपो॒ वेद॒ वेदापः॑ । \newline
41. आपो॒ वै वा आप॒ आपो॒ वै । \newline
42. वा इ॒द मि॒दं ॅवै वा इ॒दम् । \newline
43. इ॒द मग्रे ऽग्र॑ इ॒द मि॒द मग्रे᳚ । \newline
44. अग्रे॑ सलि॒लꣳ स॑लि॒ल मग्रे ऽग्रे॑ सलि॒लम् । \newline
45. स॒लि॒ल मा॑सी दासीथ् सलि॒लꣳ स॑लि॒ल मा॑सीत् । \newline
46. आ॒सी॒थ् स स आ॑सी दासी॒थ् सः । \newline
47. स प्र॒जाप॑तिः प्र॒जाप॑तिः॒ स स प्र॒जाप॑तिः । \newline
48. प्र॒जाप॑तिः पुष्करप॒र्णे पु॑ष्करप॒र्णे प्र॒जाप॑तिः प्र॒जाप॑तिः पुष्करप॒र्णे । \newline
49. प्र॒जाप॑ति॒रिति॑ प्र॒जा - प॒तिः॒ । \newline
50. पु॒ष्क॒र॒प॒र्णे वातो॒ वातः॑ पुष्करप॒र्णे पु॑ष्करप॒र्णे वातः॑ । \newline
51. पु॒ष्क॒र॒प॒र्ण इति॑ पुष्कर - प॒र्णे । \newline
52. वातो॑ भू॒तो भू॒तो वातो॒ वातो॑ भू॒तः । \newline
53. भू॒तो॑ ऽलेलाय दलेलायद् भू॒तो भू॒तो॑ ऽलेलायत् । \newline
54. अ॒ले॒ला॒य॒थ् स सो॑ ऽलेलाय दलेलाय॒थ् सः । \newline
55. स प्र॑ति॒ष्ठाम् प्र॑ति॒ष्ठाꣳ स स प्र॑ति॒ष्ठाम् । \newline

\textbf{Ghana Paata } \newline

1. वा अ॒स्या अ॒स्या वै वा अ॒स्या अ॒मृत॑ म॒मृत॑ म॒स्या वै वा अ॒स्या अ॒मृत᳚म् । \newline
2. अ॒स्या अ॒मृत॑ म॒मृत॑ म॒स्या अ॒स्या अ॒मृतं॒ ॅयद् यद॒मृत॑ म॒स्या अ॒स्या अ॒मृतं॒ ॅयत् । \newline
3. अ॒मृतं॒ ॅयद् यद॒मृत॑ म॒मृतं॒ ॅयद् वी॒र्यं॑ ॅवी॒र्यं॑ ॅयद॒मृत॑ म॒मृतं॒ ॅयद् वी॒र्य᳚म् । \newline
4. यद् वी॒र्यं॑ ॅवी॒र्यं॑ ॅयद् यद् वी॒र्य॑म् तत् तद् वी॒र्यं॑ ॅयद् यद् वी॒र्य॑म् तत् । \newline
5. वी॒र्य॑म् तत् तद् वी॒र्यं॑ ॅवी॒र्य॑म् तद् द॒र्भा द॒र्भा स्तद् वी॒र्यं॑ ॅवी॒र्य॑म् तद् द॒र्भाः । \newline
6. तद् द॒र्भा द॒र्भा स्तत् तद् द॒र्भा स्तस्मिꣳ॒॒ स्तस्मि॑न् द॒र्भा स्तत् तद् द॒र्भा स्तस्मिन्न्॑ । \newline
7. द॒र्भा स्तस्मिꣳ॒॒ स्तस्मि॑न् द॒र्भा द॒र्भा स्तस्मि॑न् जुहोति जुहोति॒ तस्मि॑न् द॒र्भा द॒र्भा स्तस्मि॑न् जुहोति । \newline
8. तस्मि॑न् जुहोति जुहोति॒ तस्मिꣳ॒॒ स्तस्मि॑न् जुहोति॒ प्र प्र जु॑होति॒ तस्मिꣳ॒॒ स्तस्मि॑न् जुहोति॒ प्र । \newline
9. जु॒हो॒ति॒ प्र प्र जु॑होति जुहोति॒ प्रैवैव प्र जु॑होति जुहोति॒ प्रैव । \newline
10. प्रैवैव प्र प्रैव जा॑यते जायत ए॒व प्र प्रैव जा॑यते । \newline
11. ए॒व जा॑यते जायत ए॒वैव जा॑यते ऽन्ना॒दो᳚ ऽन्ना॒दो जा॑यत ए॒वैव जा॑यते ऽन्ना॒दः । \newline
12. जा॒य॒ते॒ ऽन्ना॒दो᳚ ऽन्ना॒दो जा॑यते जायते ऽन्ना॒दो भ॑वति भव त्यन्ना॒दो जा॑यते जायते ऽन्ना॒दो भ॑वति । \newline
13. अ॒न्ना॒दो भ॑वति भव त्यन्ना॒दो᳚ ऽन्ना॒दो भ॑वति॒ यस्य॒ यस्य॑ भव त्यन्ना॒दो᳚ ऽन्ना॒दो भ॑वति॒ यस्य॑ । \newline
14. अ॒न्ना॒द इत्य॑न्न - अ॒दः । \newline
15. भ॒व॒ति॒ यस्य॒ यस्य॑ भवति भवति॒ यस्यै॒व मे॒वं ॅयस्य॑ भवति भवति॒ यस्यै॒वम् । \newline
16. यस्यै॒व मे॒वं ॅयस्य॒ यस्यै॒वम् जुह्व॑ति॒ जुह्व॑ त्ये॒वं ॅयस्य॒ यस्यै॒वम् जुह्व॑ति । \newline
17. ए॒वम् जुह्व॑ति॒ जुह्व॑ त्ये॒व मे॒वम् जुह्व॑ त्ये॒ता ए॒ता जुह्व॑ त्ये॒व मे॒वम् जुह्व॑ त्ये॒ताः । \newline
18. जुह्व॑ त्ये॒ता ए॒ता जुह्व॑ति॒ जुह्व॑ त्ये॒ता वै वा ए॒ता जुह्व॑ति॒ जुह्व॑ त्ये॒ता वै । \newline
19. ए॒ता वै वा ए॒ता ए॒ता वै दे॒वता॑ दे॒वता॒ वा ए॒ता ए॒ता वै दे॒वताः᳚ । \newline
20. वै दे॒वता॑ दे॒वता॒ वै वै दे॒वता॑ अ॒ग्ने र॒ग्नेर् दे॒वता॒ वै वै दे॒वता॑ अ॒ग्नेः । \newline
21. दे॒वता॑ अ॒ग्ने र॒ग्नेर् दे॒वता॑ दे॒वता॑ अ॒ग्नेः पु॒रस्ता᳚द्भागाः पु॒रस्ता᳚द्भागा अ॒ग्नेर् दे॒वता॑ दे॒वता॑ अ॒ग्नेः पु॒रस्ता᳚द्भागाः । \newline
22. अ॒ग्नेः पु॒रस्ता᳚द्भागाः पु॒रस्ता᳚द्भागा अ॒ग्ने र॒ग्नेः पु॒रस्ता᳚द्भागा॒ स्ता स्ताः पु॒रस्ता᳚द्भागा अ॒ग्ने र॒ग्नेः पु॒रस्ता᳚द्भागा॒स्ताः । \newline
23. पु॒रस्ता᳚द्भागा॒ स्ता स्ताः पु॒रस्ता᳚द्भागाः पु॒रस्ता᳚द्भागा॒ स्ता ए॒वैव ताः पु॒रस्ता᳚द्भागाः पु॒रस्ता᳚द्भागा॒ स्ता ए॒व । \newline
24. पु॒रस्ता᳚द्भागा॒ इति॑ पु॒रस्ता᳚त् - भा॒गाः॒ । \newline
25. ता ए॒वैव तास्ता ए॒व प्री॑णाति प्रीणा त्ये॒व ता स्ता ए॒व प्री॑णाति । \newline
26. ए॒व प्री॑णाति प्रीणा त्ये॒वैव प्री॑ण॒ त्यथो॒ अथो᳚ प्रीणा त्ये॒वैव प्री॑ण॒ त्यथो᳚ । \newline
27. प्री॒णा॒ त्यथो॒ अथो᳚ प्रीणाति प्रीणा॒ त्यथो॒ चक्षु॒ श्चक्षु॒ रथो᳚ प्रीणाति प्रीणा॒ त्यथो॒ चक्षुः॑ । \newline
28. अथो॒ चक्षु॒ श्चक्षु॒ रथो॒ अथो॒ चक्षु॑ रे॒वैव चक्षु॒ रथो॒ अथो॒ चक्षु॑ रे॒व । \newline
29. अथो॒ इत्यथो᳚ । \newline
30. चक्षु॑ रे॒वैव चक्षु॒ श्चक्षु॑ रे॒वाग्ने र॒ग्ने रे॒व चक्षु॒ श्चक्षु॑ रे॒वाग्नेः । \newline
31. ए॒वाग्ने र॒ग्ने रे॒वै वाग्नेः पु॒रस्ता᳚त् पु॒रस्ता॑ द॒ग्ने रे॒वै वाग्नेः पु॒रस्ता᳚त् । \newline
32. अ॒ग्नेः पु॒रस्ता᳚त् पु॒रस्ता॑ द॒ग्ने र॒ग्नेः पु॒रस्ता॒त् प्रति॒ प्रति॑ पु॒रस्ता॑ द॒ग्ने र॒ग्नेः पु॒रस्ता॒त् प्रति॑ । \newline
33. पु॒रस्ता॒त् प्रति॒ प्रति॑ पु॒रस्ता᳚त् पु॒रस्ता॒त् प्रति॑ दधाति दधाति॒ प्रति॑ पु॒रस्ता᳚त् पु॒रस्ता॒त् प्रति॑ दधाति । \newline
34. प्रति॑ दधाति दधाति॒ प्रति॒ प्रति॑ दधा॒ त्यन॒न्धो ऽन॑न्धो दधाति॒ प्रति॒ प्रति॑ दधा॒ त्यन॑न्धः । \newline
35. द॒धा॒ त्यन॒न्धो ऽन॑न्धो दधाति दधा॒ त्यन॑न्धो भवति भव॒ त्यन॑न्धो दधाति दधा॒ त्यन॑न्धो भवति । \newline
36. अन॑न्धो भवति भव॒ त्यन॒न्धो ऽन॑न्धो भवति॒ यो यो भ॑व॒ त्यन॒न्धो ऽन॑न्धो भवति॒ यः । \newline
37. भ॒व॒ति॒ यो यो भ॑वति भवति॒ य ए॒व मे॒वं ॅयो भ॑वति भवति॒ य ए॒वम् । \newline
38. य ए॒व मे॒वं ॅयो य ए॒वं ॅवेद॒ वेदै॒वं ॅयो य ए॒वं ॅवेद॑ । \newline
39. ए॒वं ॅवेद॒ वेदै॒व मे॒वं ॅवेदाप॒ आपो॒ वेदै॒व मे॒वं ॅवेदापः॑ । \newline
40. वेदाप॒ आपो॒ वेद॒ वेदापो॒ वै वा आपो॒ वेद॒ वेदापो॒ वै । \newline
41. आपो॒ वै वा आप॒ आपो॒ वा इ॒द मि॒दं ॅवा आप॒ आपो॒ वा इ॒दम् । \newline
42. वा इ॒द मि॒दं ॅवै वा इ॒द मग्रे ऽग्र॑ इ॒दं ॅवै वा इ॒द मग्रे᳚ । \newline
43. इ॒द मग्रे ऽग्र॑ इ॒द मि॒द मग्रे॑ सलि॒लꣳ स॑लि॒ल मग्र॑ इ॒द मि॒द मग्रे॑ सलि॒लम् । \newline
44. अग्रे॑ सलि॒लꣳ स॑लि॒ल मग्रे ऽग्रे॑ सलि॒ल मा॑सी दासीथ् सलि॒ल मग्रे ऽग्रे॑ सलि॒ल मा॑सीत् । \newline
45. स॒लि॒ल मा॑सी दासीथ् सलि॒लꣳ स॑लि॒ल मा॑सी॒थ् स स आ॑सीथ् सलि॒लꣳ स॑लि॒ल मा॑सी॒थ् सः । \newline
46. आ॒सी॒थ् स स आ॑सी दासी॒थ् स प्र॒जाप॑तिः प्र॒जाप॑तिः॒ स आ॑सी दासी॒थ् स प्र॒जाप॑तिः । \newline
47. स प्र॒जाप॑तिः प्र॒जाप॑तिः॒ स स प्र॒जाप॑तिः पुष्करप॒र्णे पु॑ष्करप॒र्णे प्र॒जाप॑तिः॒ स स प्र॒जाप॑तिः पुष्करप॒र्णे । \newline
48. प्र॒जाप॑तिः पुष्करप॒र्णे पु॑ष्करप॒र्णे प्र॒जाप॑तिः प्र॒जाप॑तिः पुष्करप॒र्णे वातो॒ वातः॑ पुष्करप॒र्णे प्र॒जाप॑तिः प्र॒जाप॑तिः पुष्करप॒र्णे वातः॑ । \newline
49. प्र॒जाप॑ति॒रिति॑ प्र॒जा - प॒तिः॒ । \newline
50. पु॒ष्क॒र॒प॒र्णे वातो॒ वातः॑ पुष्करप॒र्णे पु॑ष्करप॒र्णे वातो॑ भू॒तो भू॒तो वातः॑ पुष्करप॒र्णे पु॑ष्करप॒र्णे वातो॑ भू॒तः । \newline
51. पु॒ष्क॒र॒प॒र्ण इति॑ पुष्कर - प॒र्णे । \newline
52. वातो॑ भू॒तो भू॒तो वातो॒ वातो॑ भू॒तो॑ ऽलेलाय दलेलायद् भू॒तो वातो॒ वातो॑ भू॒तो॑ ऽलेलायत् । \newline
53. भू॒तो॑ ऽलेलाय दलेलायद् भू॒तो भू॒तो॑ ऽलेलाय॒थ् स सो॑ ऽलेलायद् भू॒तो भू॒तो॑ ऽलेलाय॒थ् सः । \newline
54. अ॒ले॒ला॒य॒थ् स सो॑ ऽलेलाय दलेलाय॒थ् स प्र॑ति॒ष्ठाम् प्र॑ति॒ष्ठाꣳ सो॑ ऽलेलाय दलेलाय॒थ् स प्र॑ति॒ष्ठाम् । \newline
55. स प्र॑ति॒ष्ठाम् प्र॑ति॒ष्ठाꣳ स स प्र॑ति॒ष्ठाम् न न प्र॑ति॒ष्ठाꣳ स स प्र॑ति॒ष्ठाम् न । \newline
\pagebreak
\markright{ TS 5.6.4.3  \hfill https://www.vedavms.in \hfill}

\section{ TS 5.6.4.3 }

\textbf{TS 5.6.4.3 } \newline
\textbf{Samhita Paata} \newline

प्र॑ति॒ष्ठां नाऽवि॑न्दत॒ स ए॒तद॒पां कु॒लाय॑मपश्य॒त् तस्मि॑न्न॒ग्निम॑चिनुत॒ तदि॒यम॑भव॒त् ततो॒ वै स प्रत्य॑तिष्ठ॒द्यां पु॒रस्ता॑दु॒पा-द॑धा॒त् तच्छिरो॑ ऽभव॒थ् सा प्राची॒ दिग्यां द॑क्षिण॒त उ॒पाद॑धा॒थ् स दक्षि॑णः प॒क्षो॑ऽभव॒थ् सा द॑क्षि॒णा दिग्यां प॒श्चा-दु॒पाद॑धा॒त् तत् पुच्छ॑मभव॒थ् सा प्र॒तीची॒ दिग्यामु॑त्तर॒त उ॒पाद॑धा॒थ् - [  ] \newline

\textbf{Pada Paata} \newline

प्र॒ति॒ष्ठामिति॑ प्रति - स्थाम् । न । अ॒वि॒न्द॒त॒ । सः । ए॒तत् । अ॒पाम् । कु॒लाय᳚म् । अ॒प॒श्य॒त् । तस्मिन्न्॑ । अ॒ग्निम् । अ॒चि॒नु॒त॒ । तत् । इ॒यम् । अ॒भ॒व॒त् । ततः॑ । वै । सः । प्रतीति॑ । अ॒ति॒ष्ठ॒त् । याम् । पु॒रस्ता᳚त् । उ॒पाद॑धा॒दित्यु॑प - अद॑धात् । तत् । शिरः॑ । अ॒भ॒व॒त् । सा । प्राची᳚ । दिक् । याम् । द॒क्षि॒ण॒तः । उ॒पाद॑धा॒दित्यु॑प - अद॑धात् । सः । दक्षि॑णः । प॒क्षः । अ॒भ॒व॒त् । सा । द॒क्षि॒णा । दिक् । याम् । प॒श्चात् । उ॒पाद॑धा॒दित्यु॑प-अद॑धात् । तत् । पुच्छ᳚म् । अ॒भ॒व॒त् । सा । प्र॒तीची᳚ । दिक् । याम् । उ॒त्त॒र॒त इत्यु॑त् - त॒र॒तः । उ॒पाद॑धा॒दित्यु॑प - अद॑धात् ।  \newline


\textbf{Krama Paata} \newline

प्र॒ति॒ष्ठाम् न । प्र॒ति॒ष्ठामिति॑ प्रति - स्थाम् । नावि॑न्दत । अ॒वि॒न्द॒त॒ सः । स ए॒तत् । ए॒तद॒पाम् । अ॒पाम् कु॒लाय᳚म् । कु॒लाय॑मपश्यत् । अ॒प॒श्य॒त् तस्मिन्न्॑ । तस्मि॑न्न॒ग्निम् । अ॒ग्निम॑चिनुत । अ॒चि॒नु॒त॒ तत् । तदि॒यम् । इ॒यम॑भवत् । अ॒भ॒व॒त् ततः॑ । ततो॒ वै । वै सः । स प्रति॑ । प्रत्य॑तिष्ठत् । अ॒ति॒ष्ठ॒द् याम् । याम् पु॒रस्ता᳚त् । पु॒रस्ता॑दु॒पाद॑धात् । उ॒पाद॑धा॒त् तत् । उ॒पाद॑धा॒दित्यु॑प - अद॑धात् । तच्छिरः॑ । शिरो॑ऽभवत् । अ॒भ॒व॒थ् सा । सा प्राची᳚ । प्राची॒ दिक् । दिग् याम् । याम् द॑क्षिण॒तः । द॒क्षि॒ण॒त उ॒पाद॑धात् । उ॒पाद॑धा॒थ् सः । उ॒पाद॑धा॒दित्यु॑प - अद॑धात् । स दक्षि॑णः । दक्षि॑णः प॒क्षः । प॒क्षो॑ऽभवत् । अ॒भ॒व॒थ् सा । सा द॑क्षि॒णा । द॒क्षि॒णा दिक् । दिग् याम् । याम् प॒श्चात् । प॒श्चादु॒पाद॑धात् । उ॒पाद॑धा॒त् तत् । उ॒पाद॑धा॒दित्यु॑प - अद॑धात् । तत् पुच्छ᳚म् । पुच्छ॑मभवत् । अ॒भ॒व॒थ् सा । सा प्र॒तीची᳚ । प्र॒तिची॒ दिक् । दिग् याम् । यामु॑त्तर॒तः । उ॒त्त॒र॒त उ॒पाद॑धात् । उ॒त्त॒र॒त इत्यु॑त् - त॒र॒तः । उ॒पाद॑धा॒थ् सः । उ॒पाद॑धा॒दित्यु॑प - अद॑धात् \newline

\textbf{Jatai Paata} \newline

1. प्र॒ति॒ष्ठाम् न न प्र॑ति॒ष्ठाम् प्र॑ति॒ष्ठाम् न । \newline
2. प्र॒ति॒ष्ठामिति॑ प्रति - स्थाम् । \newline
3. नावि॑न्दता विन्दत॒ न नावि॑न्दत । \newline
4. अ॒वि॒न्द॒त॒ स सो॑ ऽविन्दता विन्दत॒ सः । \newline
5. स ए॒त दे॒तथ् स स ए॒तत् । \newline
6. ए॒त द॒पा म॒पा मे॒त दे॒त द॒पाम् । \newline
7. अ॒पाम् कु॒लाय॑म् कु॒लाय॑ म॒पा म॒पाम् कु॒लाय᳚म् । \newline
8. कु॒लाय॑ मपश्य दपश्यत् कु॒लाय॑म् कु॒लाय॑ मपश्यत् । \newline
9. अ॒प॒श्य॒त् तस्मिꣳ॒॒ स्तस्मि॑न् नपश्य दपश्य॒त् तस्मिन्न्॑ । \newline
10. तस्मि॑न् न॒ग्नि म॒ग्निम् तस्मिꣳ॒॒ स्तस्मि॑न् न॒ग्निम् । \newline
11. अ॒ग्नि म॑चिनुता चिनुता॒ग्नि म॒ग्नि म॑चिनुत । \newline
12. अ॒चि॒नु॒त॒ तत् तद॑चिनु ताचिनुत॒ तत् । \newline
13. तदि॒य मि॒यम् तत् तदि॒यम् । \newline
14. इ॒य म॑भव दभव दि॒य मि॒य म॑भवत् । \newline
15. अ॒भ॒व॒त् तत॒ स्ततो॑ ऽभव दभव॒त् ततः॑ । \newline
16. ततो॒ वै वै तत॒ स्ततो॒ वै । \newline
17. वै स स वै वै सः । \newline
18. स प्रति॒ प्रति॒ स स प्रति॑ । \newline
19. प्रत्य॑तिष्ठ दतिष्ठ॒त् प्रति॒ प्रत्य॑तिष्ठत् । \newline
20. अ॒ति॒ष्ठ॒द् यां ॅया म॑तिष्ठ दतिष्ठ॒द् याम् । \newline
21. याम् पु॒रस्ता᳚त् पु॒रस्ता॒द् यां ॅयाम् पु॒रस्ता᳚त् । \newline
22. पु॒रस्ता॑ दु॒पाद॑धा दु॒पाद॑धात् पु॒रस्ता᳚त् पु॒रस्ता॑ दु॒पाद॑धात् । \newline
23. उ॒पाद॑धा॒त् तत् तदु॒पाद॑धा दु॒पाद॑धा॒त् तत् । \newline
24. उ॒पाद॑धा॒दित्यु॑प - अद॑धात् । \newline
25. तच्छिरः॒ शिर॒ स्तत् तच्छिरः॑ । \newline
26. शिरो॑ ऽभव दभव॒ च्छिरः॒ शिरो॑ ऽभवत् । \newline
27. अ॒भ॒व॒थ् सा सा ऽभ॑व दभव॒थ् सा । \newline
28. सा प्राची॒ प्राची॒ सा सा प्राची᳚ । \newline
29. प्राची॒ दिग् दिक् प्राची॒ प्राची॒ दिक् । \newline
30. दिग् यां ॅयाम् दिग् दिग् याम् । \newline
31. याम् द॑क्षिण॒तो द॑क्षिण॒तो यां ॅयाम् द॑क्षिण॒तः । \newline
32. द॒क्षि॒ण॒त उ॒पाद॑धा दु॒पाद॑धाद् दक्षिण॒तो द॑क्षिण॒त उ॒पाद॑धात् । \newline
33. उ॒पाद॑धा॒थ् स स उ॒पाद॑धा दु॒पाद॑धा॒थ् सः । \newline
34. उ॒पाद॑धा॒दित्यु॑प - अद॑धात् । \newline
35. स दक्षि॑णो॒ दक्षि॑णः॒ स स दक्षि॑णः । \newline
36. दक्षि॑णः प॒क्षः प॒क्षो दक्षि॑णो॒ दक्षि॑णः प॒क्षः । \newline
37. प॒क्षो॑ ऽभव दभवत् प॒क्षः प॒क्षो॑ ऽभवत् । \newline
38. अ॒भ॒व॒थ् सा सा ऽभ॑व दभव॒थ् सा । \newline
39. सा द॑क्षि॒णा द॑क्षि॒णा सा सा द॑क्षि॒णा । \newline
40. द॒क्षि॒णा दिग् दिग् द॑क्षि॒णा द॑क्षि॒णा दिक् । \newline
41. दिग् यां ॅयाम् दिग् दिग् याम् । \newline
42. याम् प॒श्चात् प॒श्चाद् यां ॅयाम् प॒श्चात् । \newline
43. प॒श्चा दु॒पाद॑धा दु॒पाद॑धात् प॒श्चात् प॒श्चा दु॒पाद॑धात् । \newline
44. उ॒पाद॑धा॒त् तत् तदु॒पाद॑धा दु॒पाद॑धा॒त् तत् । \newline
45. उ॒पाद॑धा॒दित्यु॑प - अद॑धात् । \newline
46. तत् पुच्छ॒म् पुच्छ॒म् तत् तत् पुच्छ᳚म् । \newline
47. पुच्छ॑ मभव दभव॒त् पुच्छ॒म् पुच्छ॑ मभवत् । \newline
48. अ॒भ॒व॒थ् सा सा ऽभ॑व दभव॒थ् सा । \newline
49. सा प्र॒तीची᳚ प्र॒तीची॒ सा सा प्र॒तीची᳚ । \newline
50. प्र॒तीची॒ दिग् दिक् प्र॒तीची᳚ प्र॒तीची॒ दिक् । \newline
51. दिग् यां ॅयाम् दिग् दिग् याम् । \newline
52. या मु॑त्तर॒त उ॑त्तर॒तो यां ॅया मु॑त्तर॒तः । \newline
53. उ॒त्त॒र॒त उ॒पाद॑धा दु॒पाद॑धा दुत्तर॒त उ॑त्तर॒त उ॒पाद॑धात् । \newline
54. उ॒त्त॒र॒त इत्यु॑त् - त॒र॒तः । \newline
55. उ॒पाद॑धा॒थ् स स उ॒पाद॑धा दु॒पाद॑धा॒थ् सः । \newline
56. उ॒पाद॑धा॒दित्यु॑प - अद॑धात् । \newline

\textbf{Ghana Paata } \newline

1. प्र॒ति॒ष्ठाम् न न प्र॑ति॒ष्ठाम् प्र॑ति॒ष्ठाम् नावि॑न्दता विन्दत॒ न प्र॑ति॒ष्ठाम् प्र॑ति॒ष्ठाम् नावि॑न्दत । \newline
2. प्र॒ति॒ष्ठामिति॑ प्रति - स्थाम् । \newline
3. नावि॑न्दता विन्दत॒ न नावि॑न्दत॒ स सो॑ ऽविन्दत॒ न नावि॑न्दत॒ सः । \newline
4. अ॒वि॒न्द॒त॒ स सो॑ ऽविन्दता विन्दत॒ स ए॒त दे॒तथ् सो॑ ऽविन्दता विन्दत॒ स ए॒तत् । \newline
5. स ए॒त दे॒तथ् स स ए॒त द॒पा म॒पा मे॒तथ् स स ए॒त द॒पाम् । \newline
6. ए॒त द॒पा म॒पा मे॒त दे॒त द॒पाम् कु॒लाय॑म् कु॒लाय॑ म॒पा मे॒त दे॒त द॒पाम् कु॒लाय᳚म् । \newline
7. अ॒पाम् कु॒लाय॑म् कु॒लाय॑ म॒पा म॒पाम् कु॒लाय॑ मपश्य दपश्यत् कु॒लाय॑ म॒पा म॒पाम् कु॒लाय॑ मपश्यत् । \newline
8. कु॒लाय॑ मपश्य दपश्यत् कु॒लाय॑म् कु॒लाय॑ मपश्य॒त् तस्मिꣳ॒॒ स्तस्मि॑न् नपश्यत् कु॒लाय॑म् कु॒लाय॑ मपश्य॒त् तस्मिन्न्॑ । \newline
9. अ॒प॒श्य॒त् तस्मिꣳ॒॒ स्तस्मि॑न् नपश्य दपश्य॒त् तस्मि॑न् न॒ग्नि म॒ग्निम् तस्मि॑न् नपश्य दपश्य॒त् तस्मि॑न् न॒ग्निम् । \newline
10. तस्मि॑न् न॒ग्नि म॒ग्निम् तस्मिꣳ॒॒ स्तस्मि॑न् न॒ग्नि म॑चिनुता चिनुता॒ग्निम् तस्मिꣳ॒॒ स्तस्मि॑न् न॒ग्नि म॑चिनुत । \newline
11. अ॒ग्नि म॑चिनुता चिनुता॒ग्नि म॒ग्नि म॑चिनुत॒ तत् तद॑चिनुता॒ग्नि म॒ग्नि म॑चिनुत॒ तत् । \newline
12. अ॒चि॒नु॒त॒ तत् तद॑चिनुता चिनुत॒ तदि॒य मि॒यम् तद॑चिनुता चिनुत॒ तदि॒यम् । \newline
13. तदि॒य मि॒यम् तत् तदि॒य म॑भव दभव दि॒यम् तत् तदि॒य म॑भवत् । \newline
14. इ॒य म॑भव दभव दि॒य मि॒य म॑भव॒त् तत॒ स्ततो॑ ऽभव दि॒य मि॒य म॑भव॒त् ततः॑ । \newline
15. अ॒भ॒व॒त् तत॒ स्ततो॑ ऽभव दभव॒त् ततो॒ वै वै ततो॑ ऽभव दभव॒त् ततो॒ वै । \newline
16. ततो॒ वै वै तत॒ स्ततो॒ वै स स वै तत॒ स्ततो॒ वै सः । \newline
17. वै स स वै वै स प्रति॒ प्रति॒ स वै वै स प्रति॑ । \newline
18. स प्रति॒ प्रति॒ स स प्रत्य॑तिष्ठ दतिष्ठ॒त् प्रति॒ स स प्रत्य॑तिष्ठत् । \newline
19. प्रत्य॑तिष्ठ दतिष्ठ॒त् प्रति॒ प्रत्य॑तिष्ठ॒द् यां ॅया म॑तिष्ठ॒त् प्रति॒ प्रत्य॑तिष्ठ॒द् याम् । \newline
20. अ॒ति॒ष्ठ॒द् यां ॅया म॑तिष्ठ दतिष्ठ॒द् याम् पु॒रस्ता᳚त् पु॒रस्ता॒द् या म॑तिष्ठ दतिष्ठ॒द् याम् पु॒रस्ता᳚त् । \newline
21. याम् पु॒रस्ता᳚त् पु॒रस्ता॒द् यां ॅयाम् पु॒रस्ता॑ दु॒पाद॑धा दु॒पाद॑धात् पु॒रस्ता॒द् यां ॅयाम् पु॒रस्ता॑ दु॒पाद॑धात् । \newline
22. पु॒रस्ता॑ दु॒पाद॑धा दु॒पाद॑धात् पु॒रस्ता᳚त् पु॒रस्ता॑ दु॒पाद॑धा॒त् तत् तदु॒पाद॑धात् पु॒रस्ता᳚त् पु॒रस्ता॑
दु॒पाद॑धा॒त् तत् । \newline
23. उ॒पाद॑धा॒त् तत् तदु॒पाद॑धा दु॒पाद॑धा॒त् तच्छिरः॒ शिर॒ स्तदु॒पाद॑धा दु॒पाद॑धा॒त् तच्छिरः॑ । \newline
24. उ॒पाद॑धा॒दित्यु॑प - अद॑धात् । \newline
25. तच्छिरः॒ शिर॒ स्तत् तच्छिरो॑ ऽभव दभव॒ च्छिर॒ स्तत् तच्छिरो॑ ऽभवत् । \newline
26. शिरो॑ ऽभव दभव॒ च्छिरः॒ शिरो॑ ऽभव॒थ् सा सा ऽभ॑व॒
च्छिरः॒ शिरो॑ ऽभव॒थ् सा । \newline
27. अ॒भ॒व॒थ् सा सा ऽभ॑व दभव॒थ् सा प्राची॒ प्राची॒ सा ऽभ॑व दभव॒थ् सा प्राची᳚ । \newline
28. सा प्राची॒ प्राची॒ सा सा प्राची॒ दिग् दिक् प्राची॒ सा सा प्राची॒ दिक् । \newline
29. प्राची॒ दिग् दिक् प्राची॒ प्राची॒ दिग् यां ॅयाम् दिक् प्राची॒ प्राची॒ दिग् याम् । \newline
30. दिग् यां ॅयाम् दिग् दिग् याम् द॑क्षिण॒तो द॑क्षिण॒तो याम् दिग् दिग् याम् द॑क्षिण॒तः । \newline
31. याम् द॑क्षिण॒तो द॑क्षिण॒तो यां ॅयाम् द॑क्षिण॒त उ॒पाद॑धा दु॒पाद॑धाद् दक्षिण॒तो यां ॅयाम् द॑क्षिण॒त उ॒पाद॑धात् । \newline
32. द॒क्षि॒ण॒त उ॒पाद॑धा दु॒पाद॑धाद् दक्षिण॒तो द॑क्षिण॒त उ॒पाद॑धा॒थ् स स उ॒पाद॑धाद् दक्षिण॒तो द॑क्षिण॒त उ॒पाद॑धा॒थ् सः । \newline
33. उ॒पाद॑धा॒थ् स स उ॒पाद॑धा दु॒पाद॑धा॒थ् स दक्षि॑णो॒ दक्षि॑णः॒ स उ॒पाद॑धा दु॒पाद॑धा॒थ् स दक्षि॑णः । \newline
34. उ॒पाद॑धा॒दित्यु॑प - अद॑धात् । \newline
35. स दक्षि॑णो॒ दक्षि॑णः॒ स स दक्षि॑णः प॒क्षः प॒क्षो दक्षि॑णः॒ स स दक्षि॑णः प॒क्षः । \newline
36. दक्षि॑णः प॒क्षः प॒क्षो दक्षि॑णो॒ दक्षि॑णः प॒क्षो॑ ऽभव दभवत् प॒क्षो दक्षि॑णो॒ दक्षि॑णः प॒क्षो॑ ऽभवत् । \newline
37. प॒क्षो॑ ऽभव दभवत् प॒क्षः प॒क्षो॑ ऽभव॒थ् सा सा ऽभ॑वत् प॒क्षः प॒क्षो॑ ऽभव॒थ् सा । \newline
38. अ॒भ॒व॒थ् सा सा ऽभ॑व दभव॒थ् सा द॑क्षि॒णा द॑क्षि॒णा सा ऽभ॑व दभव॒थ् सा द॑क्षि॒णा । \newline
39. सा द॑क्षि॒णा द॑क्षि॒णा सा सा द॑क्षि॒णा दिग् दिग् द॑क्षि॒णा सा सा द॑क्षि॒णा दिक् । \newline
40. द॒क्षि॒णा दिग् दिग् द॑क्षि॒णा द॑क्षि॒णा दिग् यां ॅयाम् दिग् द॑क्षि॒णा द॑क्षि॒णा दिग् याम् । \newline
41. दिग् यां ॅयाम् दिग् दिग् याम् प॒श्चात् प॒श्चाद् याम् दिग् दिग् याम् प॒श्चात् । \newline
42. याम् प॒श्चात् प॒श्चाद् यां ॅयाम् प॒श्चा दु॒पाद॑धा दु॒पाद॑धात् प॒श्चाद् यां ॅयाम् प॒श्चा दु॒पाद॑धात् । \newline
43. प॒श्चा दु॒पाद॑धा दु॒पाद॑धात् प॒श्चात् प॒श्चा दु॒पाद॑धा॒त् तत् तदु॒पाद॑धात् प॒श्चात् प॒श्चा दु॒पाद॑धा॒त् तत् । \newline
44. उ॒पाद॑धा॒त् तत् तदु॒पाद॑धा दु॒पाद॑धा॒त् तत् पुच्छ॒म् पुच्छ॒म् तदु॒पाद॑धा दु॒पाद॑धा॒त् तत् पुच्छ᳚म् । \newline
45. उ॒पाद॑धा॒दित्यु॑प - अद॑धात् । \newline
46. तत् पुच्छ॒म् पुच्छ॒म् तत् तत् पुच्छ॑ मभव दभव॒त् पुच्छ॒म् तत् तत् पुच्छ॑ मभवत् । \newline
47. पुच्छ॑ मभव दभव॒त् पुच्छ॒म् पुच्छ॑ मभव॒थ् सा सा ऽभ॑व॒त् पुच्छ॒म् पुच्छ॑ मभव॒थ् सा । \newline
48. अ॒भ॒व॒थ् सा सा ऽभ॑व दभव॒थ् सा प्र॒तीची᳚ प्र॒तीची॒ सा ऽभ॑व दभव॒थ् सा प्र॒तीची᳚ । \newline
49. सा प्र॒तीची᳚ प्र॒तीची॒ सा सा प्र॒तीची॒ दिग् दिक् प्र॒तीची॒ सा सा प्र॒तीची॒ दिक् । \newline
50. प्र॒तीची॒ दिग् दिक् प्र॒तीची᳚ प्र॒तीची॒ दिग् यां ॅयाम् दिक् प्र॒तीची᳚ प्र॒तीची॒ दिग् याम् । \newline
51. दिग् यां ॅयाम् दिग् दिग् या मु॑त्तर॒त उ॑त्तर॒तो याम् दिग् दिग् या मु॑त्तर॒तः । \newline
52. या मु॑त्तर॒त उ॑त्तर॒तो यां ॅया मु॑त्तर॒त उ॒पाद॑धा दु॒पाद॑धा दुत्तर॒तो यां ॅया मु॑त्तर॒त उ॒पाद॑धात् । \newline
53. उ॒त्त॒र॒त उ॒पाद॑धा दु॒पाद॑धा दुत्तर॒त उ॑त्तर॒त उ॒पाद॑धा॒थ् स स उ॒पाद॑धा दुत्तर॒त उ॑त्तर॒त उ॒पाद॑धा॒थ् सः । \newline
54. उ॒त्त॒र॒त इत्यु॑त् - त॒र॒तः । \newline
55. उ॒पाद॑धा॒थ् स स उ॒पाद॑धा दु॒पाद॑धा॒थ् स उत्त॑र॒ उत्त॑रः॒ स उ॒पाद॑धा दु॒पाद॑धा॒थ् स उत्त॑रः । \newline
56. उ॒पाद॑धा॒दित्यु॑प - अद॑धात् । \newline
\pagebreak
\markright{ TS 5.6.4.4  \hfill https://www.vedavms.in \hfill}

\section{ TS 5.6.4.4 }

\textbf{TS 5.6.4.4 } \newline
\textbf{Samhita Paata} \newline

स उत्त॑रः प॒क्षो॑ऽभव॒थ् सोदी॑ची॒ दिग्यामु॒परि॑ष्टा-दु॒पाद॑धा॒त् तत् पृ॒ष्ठम॑भव॒थ् सोर्द्ध्वा दिगि॒यं ॅवा अ॒ग्निः पञ्चे᳚ष्टक॒-स्तस्मा॒द्-यद॒स्यां खन॑न्त्य॒भीष्ट॑कां तृ॒न्दन्त्य॒भि शर्क॑राꣳ॒॒ सर्वा॒ वा इ॒यं ॅवयो᳚भ्यो॒ नक्तं॑ दृ॒शे दी᳚प्यते॒ तस्मा॑दि॒मां ॅवयाꣳ॑सि॒ नक्तं॒ नाद्ध्या॑सते॒ य ए॒वं ॅवि॒द्वान॒ग्निं चि॑नु॒ते प्रत्ये॒व - [  ] \newline

\textbf{Pada Paata} \newline

सः । उत्त॑र॒ इत्युत् - त॒रः॒ । प॒क्षः । अ॒भ॒व॒त् । सा । उदी॑ची । दिक् । याम् । उ॒परि॑ष्टात् । उ॒पाद॑धा॒दित्यु॑प - अद॑धात् । तत् । पृ॒ष्ठम् । अ॒भ॒व॒त् । सा । ऊ॒द्‌र्ध्वा । दिक् । इ॒यम् । वै । अ॒ग्निः । पञ्चे᳚ष्टक॒ इति॒ पञ्च॑-इ॒ष्ट॒कः॒ । तस्मा᳚त् । यत् । अ॒स्याम् । खन॑न्ति । अ॒भीति॑ । इष्ट॑काम् । तृ॒न्दन्ति॑ । अ॒भीति॑ । शर्क॑राम् । सर्वा᳚ । वै । इ॒यम् । वयो᳚भ्य॒ इति॒ वयः॑ - भ्यः॒ । नक्त᳚म् । दृ॒शे । दी॒प्य॒ते॒ । तस्मा᳚त् । इ॒माम् । वयाꣳ॑सि । नक्त᳚म् । न । अधीति॑ । आ॒स॒ते॒ । यः । ए॒वम् । वि॒द्वान् । अ॒ग्निम् । चि॒नु॒ते । प्रतीति॑ । ए॒व ।  \newline


\textbf{Krama Paata} \newline

स उत्त॑रः । उत्त॑रः प॒क्षः । उत्त॑र॒ इत्युत् - त॒रः॒ । प॒क्षो॑ऽभवत् । अ॒भ॒व॒थ् सा । सोदी॑ची । उदी॑ची॒ दिक् । दिग् याम् । यामु॒परि॑ष्टात् । उ॒परि॑ष्टादु॒पाद॑धात् । उ॒पाद॑धा॒त् तत् । उ॒पाद॑धा॒दित्यु॑प - अद॑धात् । तत् पृ॒ष्ठम् । पृ॒ष्ठम॑भवत् । अ॒भ॒व॒थ् सा । सोर्द्ध्वा । ऊ॒र्द्ध्वा दिक् । दिगि॒यम् । इ॒यम्ॅवै । वा अ॒ग्निः । अ॒ग्निः पञ्चे᳚ष्टकः । पञ्चे᳚ष्टक॒स्तस्मा᳚त् । पञ्चे᳚ष्टक॒ इति॒ पञ्च॑ - इ॒ष्ट॒कः॒ । तस्मा॒द् यत् । यद॒स्याम् । अ॒स्याम् खन॑न्ति । खन॑न्त्य॒भि । अ॒भीष्ट॑काम् । इष्ट॑काम् तृ॒न्दन्ति॑ । तृ॒न्दन्त्य॒भि । अ॒भि शर्क॑राम् । शर्क॑राꣳ॒॒ सर्वा᳚ । सर्वा॒ वै । वा इ॒यम् । इ॒यम् ॅवयो᳚भ्यः । वयो᳚भ्यो॒ नक्त᳚म् । वयो᳚भ्य॒ इति॒ वयः॑ - भ्यः॒ । नक्त॑म् दृ॒शे । दृ॒शे दी᳚प्यते । दी॒प्य॒ते॒ तस्मा᳚त् । तस्मा॑दि॒माम् । इ॒माम् ॅवयाꣳ॑सि । वयाꣳ॑सि॒ नक्त᳚म् । नक्त॒म् न । नाधि॑ । अद्ध्या॑सते । आ॒स॒ते॒ यः । य ए॒वम् । ए॒वम् ॅवि॒द्वान् । वि॒द्वान॒ग्निम् । अ॒ग्निम् चि॑नु॒ते । चि॒नु॒ते प्रति॑ । प्रत्ये॒व ( ) । ए॒व ति॑ष्ठति \newline

\textbf{Jatai Paata} \newline

1. स उत्त॑र॒ उत्त॑रः॒ स स उत्त॑रः । \newline
2. उत्त॑रः प॒क्षः प॒क्ष उत्त॑र॒ उत्त॑रः प॒क्षः । \newline
3. उत्त॑र॒ इत्युत् - त॒रः॒ । \newline
4. प॒क्षो॑ ऽभव दभवत् प॒क्षः प॒क्षो॑ ऽभवत् । \newline
5. अ॒भ॒व॒थ् सा सा ऽभ॑व दभव॒थ् सा । \newline
6. सोदी॒ च्युदी॑ची॒ सा सोदी॑ची । \newline
7. उदी॑ची॒ दिग् दिगुदी॒ च्युदी॑ची॒ दिक् । \newline
8. दिग् यां ॅयाम् दिग् दिग् याम् । \newline
9. या मु॒परि॑ष्टा दु॒परि॑ष्टा॒द् यां ॅया मु॒परि॑ष्टात् । \newline
10. उ॒परि॑ष्टा दु॒पाद॑धा दु॒पाद॑धा दु॒परि॑ष्टा दु॒परि॑ष्टा दु॒पाद॑धात् । \newline
11. उ॒पाद॑धा॒त् तत् तदु॒पाद॑धा दु॒पाद॑धा॒त् तत् । \newline
12. उ॒पाद॑धा॒दित्यु॑प - अद॑धात् । \newline
13. तत् पृ॒ष्ठम् पृ॒ष्ठम् तत् तत् पृ॒ष्ठम् । \newline
14. पृ॒ष्ठ म॑भव दभवत् पृ॒ष्ठम् पृ॒ष्ठ म॑भवत् । \newline
15. अ॒भ॒व॒थ् सा सा ऽभ॑व दभव॒थ् सा । \newline
16. सोर्द्ध्वो र्द्ध्वा सा सोर्द्ध्वा । \newline
17. ऊ॒र्द्ध्वा दिग् दिगू॒र्द्ध्वो र्द्ध्वा दिक् । \newline
18. दिगि॒य मि॒यम् दिग् दिगि॒यम् । \newline
19. इ॒यं ॅवै वा इ॒य मि॒यं ॅवै । \newline
20. वा अ॒ग्नि र॒ग्निर् वै वा अ॒ग्निः । \newline
21. अ॒ग्निः पञ्चे᳚ष्टकः॒ पञ्चे᳚ष्टको॒ ऽग्नि र॒ग्निः पञ्चे᳚ष्टकः । \newline
22. पञ्चे᳚ष्टक॒ स्तस्मा॒त् तस्मा॒त् पञ्चे᳚ष्टकः॒ पञ्चे᳚ष्टक॒ स्तस्मा᳚त् । \newline
23. पञ्चे᳚ष्टक॒ इति॒ पञ्च॑ - इ॒ष्ट॒कः॒ । \newline
24. तस्मा॒द् यद् यत् तस्मा॒त् तस्मा॒द् यत् । \newline
25. यद॒स्या म॒स्यां ॅयद् यद॒स्याम् । \newline
26. अ॒स्याम् खन॑न्ति॒ खन॑न् त्य॒स्या म॒स्याम् खन॑न्ति । \newline
27. खन॑न् त्य॒भ्य॑भि खन॑न्ति॒ खन॑न् त्य॒भि । \newline
28. अ॒भीष्ट॑का॒ मिष्ट॑का म॒भ्य॑भी ष्ट॑काम् । \newline
29. इष्ट॑काम् तृ॒न्दन्ति॑ तृ॒न्दन्ती ष्ट॑का॒ मिष्ट॑काम् तृ॒न्दन्ति॑ । \newline
30. तृ॒न्दन् त्य॒भ्य॑भि तृ॒न्दन्ति॑ तृ॒न्दन् त्य॒भि । \newline
31. अ॒भि शर्क॑राꣳ॒॒ शर्क॑रा म॒भ्य॑भि शर्क॑राम् । \newline
32. शर्क॑राꣳ॒॒ सर्वा॒ सर्वा॒ शर्क॑राꣳ॒॒ शर्क॑राꣳ॒॒ सर्वा᳚ । \newline
33. सर्वा॒ वै वै सर्वा॒ सर्वा॒ वै । \newline
34. वा इ॒य मि॒यं ॅवै वा इ॒यम् । \newline
35. इ॒यं ॅवयो᳚भ्यो॒ वयो᳚भ्य इ॒य मि॒यं ॅवयो᳚भ्यः । \newline
36. वयो᳚भ्यो॒ नक्त॒म् नक्तं॒ ॅवयो᳚भ्यो॒ वयो᳚भ्यो॒ नक्त᳚म् । \newline
37. वयो᳚भ्य॒ इति॒ वयः॑ - भ्यः॒ । \newline
38. नक्त॑म् दृ॒शे दृ॒शे नक्त॒म् नक्त॑म् दृ॒शे । \newline
39. दृ॒शे दी᳚प्यते दीप्यते दृ॒शे दृ॒शे दी᳚प्यते । \newline
40. दी॒प्य॒ते॒ तस्मा॒त् तस्मा᳚द् दीप्यते दीप्यते॒ तस्मा᳚त् । \newline
41. तस्मा॑ दि॒मा मि॒माम् तस्मा॒त् तस्मा॑ दि॒माम् । \newline
42. इ॒मां ॅवयाꣳ॑सि॒ वयाꣳ॑सी॒मा मि॒मां ॅवयाꣳ॑सि । \newline
43. वयाꣳ॑सि॒ नक्त॒म् नक्तं॒ ॅवयाꣳ॑सि॒ वयाꣳ॑सि॒ नक्त᳚म् । \newline
44. नक्त॒म् न न नक्त॒म् नक्त॒म् न । \newline
45. नाध्यधि॒ न नाधि॑ । \newline
46. अध्या॑सत आस॒ते ऽध्य ध्या॑सते । \newline
47. आ॒स॒ते॒ यो य आ॑सत आसते॒ यः । \newline
48. य ए॒व मे॒वं ॅयो य ए॒वम् । \newline
49. ए॒वं ॅवि॒द्वान्. वि॒द्वा ने॒व मे॒वं ॅवि॒द्वान् । \newline
50. वि॒द्वा न॒ग्नि म॒ग्निं ॅवि॒द्वान्. वि॒द्वा न॒ग्निम् । \newline
51. अ॒ग्निम् चि॑नु॒ते चि॑नु॒ते᳚ ऽग्नि म॒ग्निम् चि॑नु॒ते । \newline
52. चि॒नु॒ते प्रति॒ प्रति॑ चिनु॒ते चि॑नु॒ते प्रति॑ । \newline
53. प्रत्ये॒वैव प्रति॒ प्रत्ये॒व । \newline
54. ए॒व ति॑ष्ठति तिष्ठ त्ये॒वैव ति॑ष्ठति । \newline

\textbf{Ghana Paata } \newline

1. स उत्त॑र॒ उत्त॑रः॒ स स उत्त॑रः प॒क्षः प॒क्ष उत्त॑रः॒ स स उत्त॑रः प॒क्षः । \newline
2. उत्त॑रः प॒क्षः प॒क्ष उत्त॑र॒ उत्त॑रः प॒क्षो॑ ऽभव दभवत् प॒क्ष उत्त॑र॒ उत्त॑रः प॒क्षो॑ ऽभवत् । \newline
3. उत्त॑र॒ इत्युत् - त॒रः॒ । \newline
4. प॒क्षो॑ ऽभव दभवत् प॒क्षः प॒क्षो॑ ऽभव॒थ् सा सा ऽभ॑वत् प॒क्षः प॒क्षो॑ ऽभव॒थ् सा । \newline
5. अ॒भ॒व॒थ् सा सा ऽभ॑व दभव॒थ् सोदी॒ च्युदी॑ची॒ सा ऽभ॑व दभव॒थ् सोदी॑ची । \newline
6. सोदी॒ च्युदी॑ची॒ सा सोदी॑ची॒ दिग् दिगुदी॑ची॒ सा सोदी॑ची॒ दिक् । \newline
7. उदी॑ची॒ दिग् दिगुदी॒ च्युदी॑ची॒ दिग् यां ॅयाम् दिगुदी॒ च्युदी॑ची॒ दिग् याम् । \newline
8. दिग् यां ॅयाम् दिग् दिग् या मु॒परि॑ष्टा दु॒परि॑ष्टा॒द् याम् दिग् दिग् या मु॒परि॑ष्टात् । \newline
9. या मु॒परि॑ष्टा दु॒परि॑ष्टा॒द् यां ॅया मु॒परि॑ष्टा दु॒पाद॑धा दु॒पाद॑धा दु॒परि॑ष्टा॒द् यां ॅया मु॒परि॑ष्टा दु॒पाद॑धात् । \newline
10. उ॒परि॑ष्टा दु॒पाद॑धा दु॒पाद॑धा दु॒परि॑ष्टा दु॒परि॑ष्टा दु॒पाद॑धा॒त् तत् तदु॒पाद॑धा दु॒परि॑ष्टा दु॒परि॑ष्टा दु॒पाद॑धा॒त् तत् । \newline
11. उ॒पाद॑धा॒त् तत् तदु॒पाद॑धा दु॒पाद॑धा॒त् तत् पृ॒ष्ठम् पृ॒ष्ठम् तदु॒पाद॑धा दु॒पाद॑धा॒त् तत् पृ॒ष्ठम् । \newline
12. उ॒पाद॑धा॒दित्यु॑प - अद॑धात् । \newline
13. तत् पृ॒ष्ठम् पृ॒ष्ठम् तत् तत् पृ॒ष्ठ म॑भव दभवत् पृ॒ष्ठम् तत् तत् पृ॒ष्ठ म॑भवत् । \newline
14. पृ॒ष्ठ म॑भव दभवत् पृ॒ष्ठम् पृ॒ष्ठ म॑भव॒थ् सा सा ऽभ॑वत् पृ॒ष्ठम् पृ॒ष्ठ म॑भव॒थ् सा । \newline
15. अ॒भ॒व॒थ् सा सा ऽभ॑व दभव॒थ् सोर्द्ध्वोर्द्ध्वा सा ऽभ॑व दभव॒थ् सोर्द्ध्वा । \newline
16. सोर्द्ध्वोर्द्ध्वा सा सोर्द्ध्वा दिग् दिगू॒र्द्ध्वा सा सोर्द्ध्वा दिक् । \newline
17. ऊ॒र्द्ध्वा दिग् दिगू॒र्द्ध्वो र्द्ध्वा दिगि॒य मि॒यम् दिगू॒र्द्ध्वो र्द्ध्वा दिगि॒यम् । \newline
18. दिगि॒य मि॒यम् दिग् दिगि॒यं ॅवै वा इ॒यम् दिग् दिगि॒यं ॅवै । \newline
19. इ॒यं ॅवै वा इ॒य मि॒यं ॅवा अ॒ग्नि र॒ग्निर् वा इ॒य मि॒यं ॅवा अ॒ग्निः । \newline
20. वा अ॒ग्नि र॒ग्निर् वै वा अ॒ग्निः पञ्चे᳚ष्टकः॒ पञ्चे᳚ष्टको॒ ऽग्निर् वै वा अ॒ग्निः पञ्चे᳚ष्टकः । \newline
21. अ॒ग्निः पञ्चे᳚ष्टकः॒ पञ्चे᳚ष्टको॒ ऽग्नि र॒ग्निः पञ्चे᳚ष्टक॒ स्तस्मा॒त् तस्मा॒त् पञ्चे᳚ष्टको॒ ऽग्नि र॒ग्निः पञ्चे᳚ष्टक॒ स्तस्मा᳚त् । \newline
22. पञ्चे᳚ष्टक॒ स्तस्मा॒त् तस्मा॒त् पञ्चे᳚ष्टकः॒ पञ्चे᳚ष्टक॒ स्तस्मा॒द् यद् यत् तस्मा॒त् पञ्चे᳚ष्टकः॒ पञ्चे᳚ष्टक॒ स्तस्मा॒द् यत् । \newline
23. पञ्चे᳚ष्टक॒ इति॒ पञ्च॑ - इ॒ष्ट॒कः॒ । \newline
24. तस्मा॒द् यद् यत् तस्मा॒त् तस्मा॒द् यद॒स्या म॒स्यां ॅयत् तस्मा॒त् तस्मा॒द् यद॒स्याम् । \newline
25. यद॒स्या म॒स्यां ॅयद् यद॒स्याम् खन॑न्ति॒ खन॑न् त्य॒स्यां ॅयद् यद॒स्याम् खन॑न्ति । \newline
26. अ॒स्याम् खन॑न्ति॒ खन॑न् त्य॒स्या म॒स्याम् खन॑न् त्य॒ भ्य॑भि खन॑न् त्य॒स्या म॒स्याम् खन॑न् त्य॒भि । \newline
27. खन॑न् त्य॒भ्य॑भि खन॑न्ति॒ खन॑न् त्य॒भीष्ट॑का॒ मिष्ट॑का म॒भि खन॑न्ति॒ खन॑न् त्य॒भीष्ट॑काम् । \newline
28. अ॒भीष्ट॑का॒ मिष्ट॑का म॒भ्य॑ भीष्ट॑काम् तृ॒न्दन्ति॑ तृ॒न्दन्तीष्ट॑का म॒भ्य॑ भीष्ट॑काम् तृ॒न्दन्ति॑ । \newline
29. इष्ट॑काम् तृ॒न्दन्ति॑ तृ॒न्दन् तीष्ट॑का॒ मिष्ट॑काम् तृ॒न्दन् त्य॒भ्य॑भि तृ॒न्दन् तीष्ट॑का॒ मिष्ट॑काम् तृ॒न्दन् त्य॒भि । \newline
30. तृ॒न्दन् त्य॒भ्य॑भि तृ॒न्दन्ति॑ तृ॒न्दन् त्य॒भि शर्क॑राꣳ॒॒ शर्क॑रा म॒भि तृ॒न्दन्ति॑ तृ॒न्दन् त्य॒भि शर्क॑राम् । \newline
31. अ॒भि शर्क॑राꣳ॒॒ शर्क॑रा म॒भ्य॑भि शर्क॑राꣳ॒॒ सर्वा॒ सर्वा॒ शर्क॑रा म॒भ्य॑भि शर्क॑राꣳ॒॒ सर्वा᳚ । \newline
32. शर्क॑राꣳ॒॒ सर्वा॒ सर्वा॒ शर्क॑राꣳ॒॒ शर्क॑राꣳ॒॒ सर्वा॒ वै वै सर्वा॒ शर्क॑राꣳ॒॒ शर्क॑राꣳ॒॒ सर्वा॒ वै । \newline
33. सर्वा॒ वै वै सर्वा॒ सर्वा॒ वा इ॒य मि॒यं ॅवै सर्वा॒ सर्वा॒ वा इ॒यम् । \newline
34. वा इ॒य मि॒यं ॅवै वा इ॒यं ॅवयो᳚भ्यो॒ वयो᳚भ्य इ॒यं ॅवै वा इ॒यं ॅवयो᳚भ्यः । \newline
35. इ॒यं ॅवयो᳚भ्यो॒ वयो᳚भ्य इ॒य मि॒यं ॅवयो᳚भ्यो॒ नक्त॒म् नक्तं॒ ॅवयो᳚भ्य इ॒य मि॒यं ॅवयो᳚भ्यो॒ नक्त᳚म् । \newline
36. वयो᳚भ्यो॒ नक्त॒म् नक्तं॒ ॅवयो᳚भ्यो॒ वयो᳚भ्यो॒ नक्त॑म् दृ॒शे दृ॒शे नक्तं॒ ॅवयो᳚भ्यो॒ वयो᳚भ्यो॒ नक्त॑म् दृ॒शे । \newline
37. वयो᳚भ्य॒ इति॒ वयः॑ - भ्यः॒ । \newline
38. नक्त॑म् दृ॒शे दृ॒शे नक्त॒म् नक्त॑म् दृ॒शे दी᳚प्यते दीप्यते दृ॒शे नक्त॒म् नक्त॑म् दृ॒शे दी᳚प्यते । \newline
39. दृ॒शे दी᳚प्यते दीप्यते दृ॒शे दृ॒शे दी᳚प्यते॒ तस्मा॒त् तस्मा᳚द् दीप्यते दृ॒शे दृ॒शे दी᳚प्यते॒ तस्मा᳚त् । \newline
40. दी॒प्य॒ते॒ तस्मा॒त् तस्मा᳚द् दीप्यते दीप्यते॒ तस्मा॑ दि॒मा मि॒माम् तस्मा᳚द् दीप्यते दीप्यते॒ तस्मा॑ दि॒माम् । \newline
41. तस्मा॑ दि॒मा मि॒माम् तस्मा॒त् तस्मा॑ दि॒मां ॅवयाꣳ॑सि॒ वयाꣳ॑सी॒माम् तस्मा॒त् तस्मा॑ दि॒मां ॅवयाꣳ॑सि । \newline
42. इ॒मां ॅवयाꣳ॑सि॒ वयाꣳ॑सी॒मा मि॒मां ॅवयाꣳ॑सि॒ नक्त॒म् नक्तं॒ ॅवयाꣳ॑सी॒मा मि॒मां ॅवयाꣳ॑सि॒ नक्त᳚म् । \newline
43. वयाꣳ॑सि॒ नक्त॒म् नक्तं॒ ॅवयाꣳ॑सि॒ वयाꣳ॑सि॒ नक्त॒म् न न नक्तं॒ ॅवयाꣳ॑सि॒ वयाꣳ॑सि॒ नक्त॒म् न । \newline
44. नक्त॒म् न न नक्त॒म् नक्त॒म् नाध्यधि॒ न नक्त॒म् नक्त॒म् नाधि॑ । \newline
45. नाध्यधि॒ न नाध्या॑सत आस॒ते ऽधि॒ न नाध्या॑सते । \newline
46. अध्या॑सत आस॒ते ऽध्यध्या॑ सते॒ यो य आ॑स॒ते ऽध्यध्या॑ सते॒ यः । \newline
47. आ॒स॒ते॒ यो य आ॑सत आसते॒ य ए॒व मे॒वं ॅय आ॑सत आसते॒ य ए॒वम् । \newline
48. य ए॒व मे॒वं ॅयो य ए॒वं ॅवि॒द्वान्. वि॒द्वा ने॒वं ॅयो य ए॒वं ॅवि॒द्वान् । \newline
49. ए॒वं ॅवि॒द्वान्. वि॒द्वा ने॒व मे॒वं ॅवि॒द्वा न॒ग्नि म॒ग्निं ॅवि॒द्वा ने॒व मे॒वं ॅवि॒द्वा न॒ग्निम् । \newline
50. वि॒द्वा न॒ग्नि म॒ग्निं ॅवि॒द्वान्. वि॒द्वा न॒ग्निम् चि॑नु॒ते चि॑नु॒ते᳚ ऽग्निं ॅवि॒द्वान्. वि॒द्वा न॒ग्निम् चि॑नु॒ते । \newline
51. अ॒ग्निम् चि॑नु॒ते चि॑नु॒ते᳚ ऽग्नि म॒ग्निम् चि॑नु॒ते प्रति॒ प्रति॑ चिनु॒ते᳚ ऽग्नि म॒ग्निम् चि॑नु॒ते प्रति॑ । \newline
52. चि॒नु॒ते प्रति॒ प्रति॑ चिनु॒ते चि॑नु॒ते प्रत्ये॒ वैव प्रति॑ चिनु॒ते चि॑नु॒ते प्रत्ये॒व । \newline
53. प्रत्ये॒ वैव प्रति॒ प्रत्ये॒व ति॑ष्ठति तिष्ठ त्ये॒व प्रति॒ प्रत्ये॒व ति॑ष्ठति । \newline
54. ए॒व ति॑ष्ठति तिष्ठ त्ये॒वैव ति॑ष्ठ त्य॒भ्य॑भि ति॑ष्ठ त्ये॒वैव ति॑ष्ठत्य॒भि । \newline
\pagebreak
\markright{ TS 5.6.4.5  \hfill https://www.vedavms.in \hfill}

\section{ TS 5.6.4.5 }

\textbf{TS 5.6.4.5 } \newline
\textbf{Samhita Paata} \newline

ति॑ष्ठत्य॒भि दिशो॑ जयत्याग्ने॒यो वै ब्रा᳚ह्म॒णस्तस्मा᳚द्-ब्राह्म॒णाय॒ सर्वा॑सु दि॒क्ष्वर्द्धु॑कꣳ॒॒ स्वामे॒व तद्-दिश॒मन्वे᳚त्य॒पां ॅवा अ॒ग्निः कु॒लायं॒ तस्मा॒दापो॒ऽग्निꣳ हारु॑काः॒ स्वामे॒व तद्-योनिं॒ प्रवि॑शन्ति ॥ \newline

\textbf{Pada Paata} \newline

ति॒ष्ठ॒ति॒ । अ॒भीति॑ । दिशः॑ । ज॒य॒ति॒ । आ॒ग्ने॒यः । वै । ब्रा॒ह्म॒णः । तस्मा᳚त् । ब्रा॒ह्म॒णाय॑ । सर्वा॑सु । दि॒क्षु । अद्‌र्धु॑कम् । स्वाम् । ए॒व । तत् । दिश᳚म् । अन्विति॑ । ए॒ति॒ । अ॒पाम् । वै । अ॒ग्निः । कु॒लाय᳚म् । तस्मा᳚त् । आपः॑ । अ॒ग्निम् । हारु॑काः । स्वाम् । ए॒व । तत् । योनि᳚म् । प्रेति॑ । वि॒श॒न्ति॒ ॥  \newline


\textbf{Krama Paata} \newline

ति॒ष्ठ॒त्य॒भि । अ॒भि दिशः॑ । दिशो॑ जयति । ज॒य॒त्या॒ग्ने॒यः । आ॒ग्ने॒यो वै । वै ब्रा᳚ह्म॒णः । ब्रा॒ह्म॒णस्तस्मा᳚त् । तस्मा᳚द् ब्राह्म॒णाय॑ । ब्रा॒ह्म॒णाय॒ सर्वा॑सु । सर्वा॑सु दि॒क्षु । दि॒क्ष्वर्द्धु॑कम् । अर्द्धु॑कꣳ॒॒ स्वाम् । स्वामे॒व । ए॒व तत् । तद् दिश᳚म् । दिश॒मनु॑ । अन्वे॑ति । ए॒त्य॒पाम् । अ॒पाम् ॅवै । वा अ॒ग्निः । अ॒ग्निः कु॒लाय᳚म् । कु॒लाय॒म् तस्मा᳚त् । तस्मा॒दापः॑ । आपो॒ऽग्निम् । अ॒ग्निꣳ हारु॑काः । हारु॑काः॒ स्वाम् । स्वामे॒व । ए॒व तत् । तद् योनि᳚म् । योनि॒म् प्र । प्र वि॑शन्ति । वि॒श॒न्तीति॑ विशन्ति । \newline

\textbf{Jatai Paata} \newline

1. ति॒ष्ठ॒ त्य॒भ्य॑भि ति॑ष्ठति तिष्ठ त्य॒भि । \newline
2. अ॒भि दिशो॒ दिशो॒ ऽभ्य॑भि दिशः॑ । \newline
3. दिशो॑ जयति जयति॒ दिशो॒ दिशो॑ जयति । \newline
4. ज॒य॒ त्या॒ग्ने॒य आ᳚ग्ने॒यो ज॑यति जय त्याग्ने॒यः । \newline
5. आ॒ग्ने॒यो वै वा आ᳚ग्ने॒य आ᳚ग्ने॒यो वै । \newline
6. वै ब्रा᳚ह्म॒णो ब्रा᳚ह्म॒णो वै वै ब्रा᳚ह्म॒णः । \newline
7. ब्रा॒ह्म॒ण स्तस्मा॒त् तस्मा᳚द् ब्राह्म॒णो ब्रा᳚ह्म॒ण स्तस्मा᳚त् । \newline
8. तस्मा᳚द् ब्राह्म॒णाय॑ ब्राह्म॒णाय॒ तस्मा॒त् तस्मा᳚द् ब्राह्म॒णाय॑ । \newline
9. ब्रा॒ह्म॒णाय॒ सर्वा॑सु॒ सर्वा॑सु ब्राह्म॒णाय॑ ब्राह्म॒णाय॒ सर्वा॑सु । \newline
10. सर्वा॑सु दि॒क्षु दि॒क्षु सर्वा॑सु॒ सर्वा॑सु दि॒क्षु । \newline
11. दि॒क्ष्व र्द्धु॑क॒ मर्द्धु॑कम् दि॒क्षु दि॒क्ष्व र्द्धु॑कम् । \newline
12. अर्द्धु॑कꣳ॒॒ स्वाꣳ स्वा मर्द्धु॑क॒ मर्द्धु॑कꣳ॒॒ स्वाम् । \newline
13. स्वा मे॒वैव स्वाꣳ स्वा मे॒व । \newline
14. ए॒व तत् तदे॒ वैव तत् । \newline
15. तद् दिश॒म् दिश॒म् तत् तद् दिश᳚म् । \newline
16. दिश॒ मन्वनु॒ दिश॒म् दिश॒ मनु॑ । \newline
17. अन्वे᳚ त्ये॒त्यन् वन् वे॑ति । \newline
18. ए॒त्य॒पा म॒पा मे᳚त्ये त्य॒पाम् । \newline
19. अ॒पां ॅवै वा अ॒पा म॒पां ॅवै । \newline
20. वा अ॒ग्नि र॒ग्निर् वै वा अ॒ग्निः । \newline
21. अ॒ग्निः कु॒लाय॑म् कु॒लाय॑ म॒ग्नि र॒ग्निः कु॒लाय᳚म् । \newline
22. कु॒लाय॒म् तस्मा॒त् तस्मा᳚त् कु॒लाय॑म् कु॒लाय॒म् तस्मा᳚त् । \newline
23. तस्मा॒ दाप॒ आप॒ स्तस्मा॒त् तस्मा॒ दापः॑ । \newline
24. आपो॒ ऽग्नि म॒ग्नि माप॒ आपो॒ ऽग्निम् । \newline
25. अ॒ग्निꣳ हारु॑का॒ हारु॑का अ॒ग्नि म॒ग्निꣳ हारु॑काः । \newline
26. हारु॑काः॒ स्वाꣳ स्वाꣳ हारु॑का॒ हारु॑काः॒ स्वाम् । \newline
27. स्वा मे॒वैव स्वाꣳ स्वा मे॒व । \newline
28. ए॒व तत् तदे॒ वैव तत् । \newline
29. तद् योनिं॒ ॅयोनि॒म् तत् तद् योनि᳚म् । \newline
30. योनि॒म् प्र प्र योनिं॒ ॅयोनि॒म् प्र । \newline
31. प्र वि॑शन्ति विशन्ति॒ प्र प्र वि॑शन्ति । \newline
32. वि॒श॒न्तीति॑ विशन्ति । \newline

\textbf{Ghana Paata } \newline

1. ति॒ष्ठ॒ त्य॒भ्य॑भि ति॑ष्ठति तिष्ठत्य॒भि दिशो॒ दिशो॒ ऽभि ति॑ष्ठति तिष्ठत्य॒भि दिशः॑ । \newline
2. अ॒भि दिशो॒ दिशो॒ ऽभ्य॑भि दिशो॑ जयति जयति॒ दिशो॒ ऽभ्य॑भि दिशो॑ जयति । \newline
3. दिशो॑ जयति जयति॒ दिशो॒ दिशो॑ जय त्याग्ने॒य आ᳚ग्ने॒यो ज॑यति॒ दिशो॒ दिशो॑ जय त्याग्ने॒यः । \newline
4. ज॒य॒ त्या॒ग्ने॒य आ᳚ग्ने॒यो ज॑यति जय त्याग्ने॒यो वै वा आ᳚ग्ने॒यो ज॑यति जय त्याग्ने॒यो वै । \newline
5. आ॒ग्ने॒यो वै वा आ᳚ग्ने॒य आ᳚ग्ने॒यो वै ब्रा᳚ह्म॒णो ब्रा᳚ह्म॒णो वा आ᳚ग्ने॒य आ᳚ग्ने॒यो वै ब्रा᳚ह्म॒णः । \newline
6. वै ब्रा᳚ह्म॒णो ब्रा᳚ह्म॒णो वै वै ब्रा᳚ह्म॒ण स्तस्मा॒त् तस्मा᳚द् ब्राह्म॒णो वै वै ब्रा᳚ह्म॒ण स्तस्मा᳚त् । \newline
7. ब्रा॒ह्म॒ण स्तस्मा॒त् तस्मा᳚द् ब्राह्म॒णो ब्रा᳚ह्म॒ण स्तस्मा᳚द् ब्राह्म॒णाय॑ ब्राह्म॒णाय॒ तस्मा᳚द् ब्राह्म॒णो ब्रा᳚ह्म॒ण स्तस्मा᳚द् ब्राह्म॒णाय॑ । \newline
8. तस्मा᳚द् ब्राह्म॒णाय॑ ब्राह्म॒णाय॒ तस्मा॒त् तस्मा᳚द् ब्राह्म॒णाय॒ सर्वा॑सु॒ सर्वा॑सु ब्राह्म॒णाय॒ तस्मा॒त् तस्मा᳚द् ब्राह्म॒णाय॒ सर्वा॑सु । \newline
9. ब्रा॒ह्म॒णाय॒ सर्वा॑सु॒ सर्वा॑सु ब्राह्म॒णाय॑ ब्राह्म॒णाय॒ सर्वा॑सु दि॒क्षु दि॒क्षु सर्वा॑सु ब्राह्म॒णाय॑ ब्राह्म॒णाय॒ सर्वा॑सु दि॒क्षु । \newline
10. सर्वा॑सु दि॒क्षु दि॒क्षु सर्वा॑सु॒ सर्वा॑सु दि॒क्ष्व र्द्धु॑क॒ मर्द्धु॑कम् दि॒क्षु सर्वा॑सु॒ सर्वा॑सु दि॒क्ष्व र्द्धु॑कम् । \newline
11. दि॒क्ष्व र्द्धु॑क॒ मर्द्धु॑कम् दि॒क्षु दि॒क्ष्व र्द्धु॑कꣳ॒॒ स्वाꣳ स्वा मर्द्धु॑कम् दि॒क्षु दि॒क्ष्व र्द्धु॑कꣳ॒॒ स्वाम् । \newline
12. अर्द्धु॑कꣳ॒॒ स्वाꣳ स्वा मर्द्धु॑क॒ मर्द्धु॑कꣳ॒॒ स्वा मे॒वैव स्वा मर्द्धु॑क॒ मर्द्धु॑कꣳ॒॒ स्वा मे॒व । \newline
13. स्वा मे॒वैव स्वाꣳ स्वा मे॒व तत् तदे॒व स्वाꣳ स्वा मे॒व तत् । \newline
14. ए॒व तत् तदे॒वैव तद् दिश॒म् दिश॒म् तदे॒वैव तद् दिश᳚म् । \newline
15. तद् दिश॒म् दिश॒म् तत् तद् दिश॒ मन्वनु॒ दिश॒म् तत् तद् दिश॒ मनु॑ । \newline
16. दिश॒ मन्वनु॒ दिश॒म् दिश॒ मन्वे᳚ त्ये॒त्यनु॒ दिश॒म् दिश॒ मन्वे॑ति । \newline
17. अन्वे᳚ त्ये॒त्यन् वन्वे᳚ त्य॒पा म॒पा मे॒त्यन् वन्वे᳚ त्य॒पाम् । \newline
18. ए॒त्य॒पा म॒पा मे᳚त्ये त्य॒पां ॅवै वा अ॒पा मे᳚त्ये त्य॒पां ॅवै । \newline
19. अ॒पां ॅवै वा अ॒पा म॒पां ॅवा अ॒ग्नि र॒ग्निर् वा अ॒पा म॒पां ॅवा अ॒ग्निः । \newline
20. वा अ॒ग्नि र॒ग्निर् वै वा अ॒ग्निः कु॒लाय॑म् कु॒लाय॑ म॒ग्निर् वै वा अ॒ग्निः कु॒लाय᳚म् । \newline
21. अ॒ग्निः कु॒लाय॑म् कु॒लाय॑ म॒ग्नि र॒ग्निः कु॒लाय॒म् तस्मा॒त् तस्मा᳚त् कु॒लाय॑ म॒ग्नि र॒ग्निः कु॒लाय॒म् तस्मा᳚त् । \newline
22. कु॒लाय॒म् तस्मा॒त् तस्मा᳚त् कु॒लाय॑म् कु॒लाय॒म् तस्मा॒ दाप॒ आप॒ स्तस्मा᳚त् कु॒लाय॑म् कु॒लाय॒म् तस्मा॒ दापः॑ । \newline
23. तस्मा॒ दाप॒ आप॒ स्तस्मा॒त् तस्मा॒ दापो॒ ऽग्नि म॒ग्नि माप॒ स्तस्मा॒त् तस्मा॒ दापो॒ ऽग्निम् । \newline
24. आपो॒ ऽग्नि म॒ग्नि माप॒ आपो॒ ऽग्निꣳ हारु॑का॒ हारु॑का अ॒ग्नि माप॒ आपो॒ ऽग्निꣳ हारु॑काः । \newline
25. अ॒ग्निꣳ हारु॑का॒ हारु॑का अ॒ग्नि म॒ग्निꣳ हारु॑काः॒ स्वाꣳ स्वाꣳ हारु॑का अ॒ग्नि म॒ग्निꣳ हारु॑काः॒ स्वाम् । \newline
26. हारु॑काः॒ स्वाꣳ स्वाꣳ हारु॑का॒ हारु॑काः॒ स्वा मे॒वैव स्वाꣳ हारु॑का॒ हारु॑काः॒ स्वा मे॒व । \newline
27. स्वा मे॒वैव स्वाꣳ स्वा मे॒व तत् तदे॒व स्वाꣳ स्वा मे॒व तत् । \newline
28. ए॒व तत् तदे॒ वैव तद् योनिं॒ ॅयोनि॒म् तदे॒ वैव तद् योनि᳚म् । \newline
29. तद् योनिं॒ ॅयोनि॒म् तत् तद् योनि॒म् प्र प्र योनि॒म् तत् तद् योनि॒म् प्र । \newline
30. योनि॒म् प्र प्र योनिं॒ ॅयोनि॒म् प्र वि॑शन्ति विशन्ति॒ प्र योनिं॒ ॅयोनि॒म् प्र वि॑शन्ति । \newline
31. प्र वि॑शन्ति विशन्ति॒ प्र प्र वि॑शन्ति । \newline
32. वि॒श॒न्तीति॑ विशन्ति । \newline
\pagebreak
\markright{ TS 5.6.5.1  \hfill https://www.vedavms.in \hfill}

\section{ TS 5.6.5.1 }

\textbf{TS 5.6.5.1 } \newline
\textbf{Samhita Paata} \newline

सं॒ॅव॒थ्स॒रमुख्यं॑ भृ॒त्वा द्वि॒तीये॑ संॅवथ्स॒र आ᳚ग्ने॒यम॒ष्टाक॑पालं॒ निर्व॑पेदै॒न्द्र-मेका॑दशकपालं ॅवैश्वदे॒वं द्वाद॑शकपालं बार्.हस्प॒त्यं च॒रुं ॅवै᳚ष्ण॒वं त्रि॑कपा॒लं तृ॒तीये॑ संॅवथ्स॒रे॑ऽभि॒जिता॑ यजेत॒ यद॒ष्टाक॑पालो॒ भव॑त्य॒ष्टाक्ष॑रा गाय॒त्र्या᳚ग्ने॒यं गा॑य॒त्रं प्रा॑तस्सव॒नं प्रा॑तस्सव॒नमे॒व तेन॑ दाधार गाय॒त्रीं छन्दो॒ यदेका॑दशकपालो॒ भव॒त्येका॑दशाक्षरा त्रि॒ष्टुगै॒न्द्रं त्रैष्टु॑भं॒ माद्ध्य॑न्दिनꣳ॒॒ सव॑नं॒ माद्ध्य॑न्दिनमे॒व सव॑नं॒ तेन॑ दाधार त्रि॒ष्टुभं॒ - [  ] \newline

\textbf{Pada Paata} \newline

सं॒ॅव॒थ्स॒रमिति॑ सं - व॒थ्स॒रम् । उख्य᳚म् । भृ॒त्वा । द्वि॒तीये᳚ । सं॒ॅव॒थ्स॒र इति॑ सं - व॒थ्स॒रे । आ॒ग्ने॒यम् । अ॒ष्टाक॑पाल॒मित्य॒ष्टा - क॒पा॒ल॒म् । निरिति॑ । व॒पे॒त् । ऐ॒न्द्रम् । एका॑दशकपाल॒मित्येका॑दश - क॒पा॒ल॒म् । वै॒श्व॒दे॒वमिति॑ वैश्व - दे॒वम् । द्वाद॑शकपाल॒मिति॒ द्वाद॑श - क॒पा॒ल॒म् । बा॒र्.॒ह॒स्प॒त्यम् । च॒रुम् । वै॒ष्ण॒वम् । त्रि॒क॒पा॒लमिति॑ त्रि - क॒पा॒लम् । तृ॒तीये᳚ । सं॒ॅव॒थ्स॒र इति॑ सं - व॒थ्स॒रे । अ॒भि॒जितेत्य॑भि - जिता᳚ । य॒जे॒त॒ । यत् । अ॒ष्टाक॑पाल॒ इत्य॒ष्टा - क॒पा॒लः॒ । भव॑ति । अ॒ष्टाक्ष॒रेत्य॒ष्टा - अ॒क्ष॒रा॒ । गा॒य॒त्री । आ॒ग्ने॒यम् । गा॒य॒त्रम् । प्रा॒त॒स्स॒व॒नमिति॑ प्रातः - स॒व॒नम् । प्रा॒त॒स्स॒व॒नमिति॑ प्रातः-स॒व॒नम् । ए॒व । तेन॑ । दा॒धा॒र॒ । गा॒य॒त्रीम् । छन्दः॑ । यत् । एका॑दशकपाल॒ इत्येका॑दश - क॒पा॒लः॒ । भव॑ति । एका॑दशाक्ष॒रेत्येका॑दश - अ॒क्ष॒रा॒ । त्रि॒ष्टुक् । ऐ॒न्द्रम् । त्रैष्टु॑भम् । माद्ध्य॑न्दिनम् । सव॑नम् । माद्ध्य॑न्दिनम् । ए॒व । सव॑नम् । तेन॑ । दा॒धा॒र॒ । त्रि॒ष्टुभ᳚म् ।  \newline


\textbf{Krama Paata} \newline

स॒म्ॅव॒थ्स॒रमुख्य᳚म् । स॒म्ॅव॒थ्स॒रमिति॑ सम् - व॒थ्स॒रम् । उख्य॑म् भृ॒त्वा । भृ॒त्वा द्वि॒तीये᳚ । द्वि॒तीये॑ सम्ॅवथ्स॒रे । स॒म्ॅव॒थ्स॒र आ᳚ग्ने॒यम् । स॒म्ॅव॒थ्स॒र इति॑ सम् - व॒थ्स॒रे । आ॒ग्ने॒यम॒ष्टाक॑पालम् । अ॒ष्टाक॑पाल॒म् निः । अ॒ष्टाक॑पाल॒मित्य॒ष्टा - क॒पा॒ल॒म् । निर् व॑पेत् । व॒पे॒दै॒न्द्रम् । ऐ॒न्द्रमेका॑दशकपालम् । एका॑दशकपालम् ॅवैश्वदे॒वम् । एका॑दशकपाल॒मित्येका॑दश - क॒पा॒ल॒म् । वै॒श्व॒दे॒वम् द्वाद॑शकपालम् । वै॒श्व॒दे॒वमिति॑ वैश्व - दे॒वम् । द्वाद॑शकपालम् बार्.हस्प॒त्यम् । द्वाद॑शकपाल॒मिति॒ द्वाद॑श - क॒पा॒ल॒म् । बा॒र्॒.ह॒स्प॒त्यम् च॒रुम् । च॒रुम् वै᳚ष्ण॒वम् । वै॒ष्ण॒वम् त्रि॑कपा॒लम् । त्रि॒क॒पा॒लम् तृ॒तीये᳚ । त्रि॒क॒पा॒लमिति॑ त्रि - क॒पा॒लम् । तृ॒तीये॑ सम्ॅवथ्स॒रे । स॒म्ॅव॒थ्स॒रे॑ऽभि॒जिता᳚ । स॒म्ॅव॒थ्स॒र इति॑ सम् - व॒थ्स॒रे । अ॒भि॒जिता॑ यजेत । अ॒भि॒जितेत्य॑भि - जिता᳚ । य॒जे॒त॒ यत् । यद॒ष्टाक॑पालः । अ॒ष्टाक॑पालो॒ भव॑ति । अ॒ष्टाक॑पाल॒ इत्य॒ष्टा - क॒पा॒लः॒ । भव॑त्य॒ष्टाक्ष॑रा । अ॒ष्टाक्ष॑रा गाय॒त्री । अ॒ष्टाक्ष॒रेत्य॒ष्टा - अ॒क्ष॒रा॒ । गा॒य॒त्र्या᳚ग्ने॒यम् । आ॒ग्ने॒यम् गा॑य॒त्रम् । गा॒य॒त्रम् प्रा॑तस्सव॒नम् । प्रा॒त॒स्स॒व॒नम् प्रा॑तस्सव॒नम् । प्रा॒त॒स्स॒व॒नमिति॑ प्रातः - स॒व॒नम् । प्रा॒त॒स्स॒व॒नमे॒व । प्रा॒त॒स्स॒व॒नमिति॑ प्रातः - स॒व॒नम् । ए॒व तेन॑ । तेन॑ दाधार । दा॒धा॒र॒ गा॒य॒त्रीम् । गा॒य॒त्रीम् छन्दः॑ । छन्दो॒ यत् । यदेका॑दशकपालः । एका॑दशकपालो॒ भव॑ति । एका॑दशकपाल॒ इत्येका॑दश - क॒पा॒लः॒ । भव॒त्येका॑दशाक्षरा । एका॑दशाक्षरा त्रि॒ष्टुक् । एका॑दशाक्ष॒रेत्येका॑दश - अ॒क्ष॒रा॒ । त्रि॒ष्टुगै॒न्द्रम् । ऐ॒न्दम् त्रैष्टु॑भम् । त्रैष्टु॑भ॒म् माद्ध्य॑न्दिनम् । माद्ध्य॑न्दिनꣳ॒॒ सव॑नम् । सव॑न॒म् माद्ध्य॑न्दिनम् । माद्ध्य॑न्दिनमे॒व । ए॒व सव॑नम् । सव॑न॒म् तेन॑ । तेन॑ दाधार । दा॒धा॒र॒ त्रि॒ष्टुभ᳚म् । त्रि॒ष्टुभ॒म् छन्दः॑ \newline

\textbf{Jatai Paata} \newline

1. सं॒ॅव॒थ्स॒र मुख्य॒ मुख्यꣳ॑ संॅवथ्स॒रꣳ सं॑ॅवथ्स॒र मुख्य᳚म् । \newline
2. सं॒ॅव॒थ्स॒रमिति॑ सं - व॒थ्स॒रम् । \newline
3. उख्य॑म् भृ॒त्वा भृ॒त्वोख्य॒ मुख्य॑म् भृ॒त्वा । \newline
4. भृ॒त्वा द्वि॒तीये᳚ द्वि॒तीये॑ भृ॒त्वा भृ॒त्वा द्वि॒तीये᳚ । \newline
5. द्वि॒तीये॑ संॅवथ्स॒रे सं॑ॅवथ्स॒रे द्वि॒तीये᳚ द्वि॒तीये॑ संॅवथ्स॒रे । \newline
6. सं॒ॅव॒थ्स॒र आ᳚ग्ने॒य मा᳚ग्ने॒यꣳ सं॑ॅवथ्स॒रे सं॑ॅवथ्स॒र आ᳚ग्ने॒यम् । \newline
7. सं॒ॅव॒थ्स॒र इति॑ सं - व॒थ्स॒रे । \newline
8. आ॒ग्ने॒य म॒ष्टाक॑पाल म॒ष्टाक॑पाल माग्ने॒य मा᳚ग्ने॒य म॒ष्टाक॑पालम् । \newline
9. अ॒ष्टाक॑पाल॒म् निर् णिर॒ष्टाक॑पाल म॒ष्टाक॑पाल॒म् निः । \newline
10. अ॒ष्टाक॑पाल॒मित्य॒ष्टा - क॒पा॒ल॒म् । \newline
11. निर् व॑पेद् वपे॒न् निर् णिर् व॑पेत् । \newline
12. व॒पे॒ दै॒न्द्र मै॒न्द्रं ॅव॑पेद् वपे दै॒न्द्रम् । \newline
13. ऐ॒न्द्र मेका॑दशकपाल॒ मेका॑दशकपाल मै॒न्द्र मै॒न्द्र मेका॑दशकपालम् । \newline
14. एका॑दशकपालं ॅवैश्वदे॒वं ॅवै᳚श्वदे॒व मेका॑दशकपाल॒ मेका॑दशकपालं ॅवैश्वदे॒वम् । \newline
15. एका॑दशकपाल॒मित्येका॑दश - क॒पा॒ल॒म् । \newline
16. वै॒श्व॒दे॒वम् द्वाद॑शकपाल॒म् द्वाद॑शकपालं ॅवैश्वदे॒वं ॅवै᳚श्वदे॒वम् द्वाद॑शकपालम् । \newline
17. वै॒श्व॒दे॒वमिति॑ वैश्व - दे॒वम् । \newline
18. द्वाद॑शकपालम् बार्.हस्प॒त्यम् बा॑र्.हस्प॒त्यम् द्वाद॑शकपाल॒म् द्वाद॑शकपालम् बार्.हस्प॒त्यम् । \newline
19. द्वाद॑शकपाल॒मिति॒ द्वाद॑श - क॒पा॒ल॒म् । \newline
20. बा॒र्॒.ह॒स्प॒त्यम् च॒रुम् च॒रुम् बा॑र्.हस्प॒त्यम् बा॑र्.हस्प॒त्यम् च॒रुम् । \newline
21. च॒रुं ॅवै᳚ष्ण॒वं ॅवै᳚ष्ण॒वम् च॒रुम् च॒रुं ॅवै᳚ष्ण॒वम् । \newline
22. वै॒ष्ण॒वम् त्रि॑कपा॒लम् त्रि॑कपा॒लं ॅवै᳚ष्ण॒वं ॅवै᳚ष्ण॒वम् त्रि॑कपा॒लम् । \newline
23. त्रि॒क॒पा॒लम् तृ॒तीये॑ तृ॒तीये᳚ त्रिकपा॒लम् त्रि॑कपा॒लम् तृ॒तीये᳚ । \newline
24. त्रि॒क॒पा॒लमिति॑ त्रि - क॒पा॒लम् । \newline
25. तृ॒तीये॑ संॅवथ्स॒रे सं॑ॅवथ्स॒रे तृ॒तीये॑ तृ॒तीये॑ संॅवथ्स॒रे । \newline
26. सं॒ॅव॒थ्स॒रे॑ ऽभि॒जिता॑ ऽभि॒जिता॑ संॅवथ्स॒रे सं॑ॅवथ्स॒रे॑ ऽभि॒जिता᳚ । \newline
27. सं॒ॅव॒थ्स॒र इति॑ सं - व॒थ्स॒रे । \newline
28. अ॒भि॒जिता॑ यजेत यजेता भि॒जिता॑ ऽभि॒जिता॑ यजेत । \newline
29. अ॒भि॒जितेत्य॑भि - जिता᳚ । \newline
30. य॒जे॒त॒ यद् यद् य॑जेत यजेत॒ यत् । \newline
31. यद॒ष्टाक॑पालो॒ ऽष्टाक॑पालो॒ यद् यद॒ष्टाक॑पालः । \newline
32. अ॒ष्टाक॑पालो॒ भव॑ति॒ भव॑ त्य॒ष्टाक॑पालो॒ ऽष्टाक॑पालो॒ भव॑ति । \newline
33. अ॒ष्टाक॑पाल॒ इत्य॒ष्टा - क॒पा॒लः॒ । \newline
34. भव॑ त्य॒ष्टाक्ष॑रा॒ ऽष्टाक्ष॑रा॒ भव॑ति॒ भव॑ त्य॒ष्टाक्ष॑रा । \newline
35. अ॒ष्टाक्ष॑रा गाय॒त्री गा॑य॒ त्र्य॑ष्टाक्ष॑रा॒ ऽष्टाक्ष॑रा गाय॒त्री । \newline
36. अ॒ष्टाक्ष॒रेत्य॒ष्टा - अ॒क्ष॒रा॒ । \newline
37. गा॒य॒ त्र्या᳚ग्ने॒य मा᳚ग्ने॒यम् गा॑य॒त्री गा॑य॒ त्र्या᳚ग्ने॒यम् । \newline
38. आ॒ग्ने॒यम् गा॑य॒त्रम् गा॑य॒त्र मा᳚ग्ने॒य मा᳚ग्ने॒यम् गा॑य॒त्रम् । \newline
39. गा॒य॒त्रम् प्रा॑तस्सव॒नम् प्रा॑तस्सव॒नम् गा॑य॒त्रम् गा॑य॒त्रम् प्रा॑तस्सव॒नम् । \newline
40. प्रा॒त॒स्स॒व॒नम् प्रा॑तस्सव॒नम् । \newline
41. प्रा॒त॒स्स॒व॒नमिति॑ प्रातः - स॒व॒नम् । \newline
42. प्रा॒त॒स्स॒व॒न मे॒वैव प्रा॑तस्सव॒नम् प्रा॑तस्सव॒न मे॒व । \newline
43. प्रा॒त॒स्स॒व॒नमिति॑ प्रातः - स॒व॒नम् । \newline
44. ए॒व तेन॒ तेनै॒ वैव तेन॑ । \newline
45. तेन॑ दाधार दाधार॒ तेन॒ तेन॑ दाधार । \newline
46. दा॒धा॒र॒ गा॒य॒त्रीम् गा॑य॒त्रीम् दा॑धार दाधार गाय॒त्रीम् । \newline
47. गा॒य॒त्रीम् छन्द॒ श्छन्दो॑ गाय॒त्रीम् गा॑य॒त्रीम् छन्दः॑ । \newline
48. छन्दो॒ यद् यच् छन्द॒ श्छन्दो॒ यत् । \newline
49. यदेका॑दशकपाल॒ एका॑दशकपालो॒ यद् यदेका॑दशकपालः । \newline
50. एका॑दशकपालो॒ भव॑ति॒ भव॒ त्येका॑दशकपाल॒ एका॑दशकपालो॒ भव॑ति । \newline
51. एका॑दशकपाल॒ इत्येका॑दश - क॒पा॒लः॒ । \newline
52. भव॒ त्येका॑दशाक्ष॒ रैका॑दशाक्षरा॒ भव॑ति॒ भव॒ त्येका॑दशाक्षरा । \newline
53. एका॑दशाक्षरा त्रि॒ष्टुक् त्रि॒ष्टु गेका॑दशाक्ष॒ रैका॑दशाक्षरा त्रि॒ष्टुक् । \newline
54. एका॑दशाक्ष॒रेत्येका॑दश - अ॒क्ष॒रा॒ । \newline
55. त्रि॒ष्टु गै॒न्द्र मै॒न्द्रम् त्रि॒ष्टुक् त्रि॒ष्टु गै॒न्द्रम् । \newline
56. ऐ॒न्द्रम् त्रैष्टु॑भ॒म् त्रैष्टु॑भ मै॒न्द्र मै॒न्द्रम् त्रैष्टु॑भम् । \newline
57. त्रैष्टु॑भ॒म् माद्ध्य॑न्दिन॒म् माद्ध्य॑न्दिन॒म् त्रैष्टु॑भ॒म् त्रैष्टु॑भ॒म् माद्ध्य॑न्दिनम् । \newline
58. माद्ध्य॑न्दिनꣳ॒॒ सव॑नꣳ॒॒ सव॑न॒म् माद्ध्य॑न्दिन॒म् माद्ध्य॑न्दिनꣳ॒॒ सव॑नम् । \newline
59. सव॑न॒म् माद्ध्य॑न्दिन॒म् माद्ध्य॑न्दिनꣳ॒॒ सव॑नꣳ॒॒ सव॑न॒म् माद्ध्य॑न्दिनम् । \newline
60. माद्ध्य॑न्दिन मे॒वैव माद्ध्य॑न्दिन॒म् माद्ध्य॑न्दिन मे॒व । \newline
61. ए॒व सव॑नꣳ॒॒ सव॑न मे॒वैव सव॑नम् । \newline
62. सव॑न॒म् तेन॒ तेन॒ सव॑नꣳ॒॒ सव॑न॒म् तेन॑ । \newline
63. तेन॑ दाधार दाधार॒ तेन॒ तेन॑ दाधार । \newline
64. दा॒धा॒र॒ त्रि॒ष्टुभ॑म् त्रि॒ष्टुभ॑म् दाधार दाधार त्रि॒ष्टुभ᳚म् । \newline
65. त्रि॒ष्टुभ॒म् छन्द॒ श्छन्द॑ स्त्रि॒ष्टुभ॑म् त्रि॒ष्टुभ॒म् छन्दः॑ । \newline

\textbf{Ghana Paata } \newline

1. सं॒ॅव॒थ्स॒र मुख्य॒ मुख्यꣳ॑ संॅवथ्स॒रꣳ सं॑ॅवथ्स॒र मुख्य॑म् भृ॒त्वा भृ॒त्वोख्यꣳ॑ संॅवथ्स॒रꣳ सं॑ॅवथ्स॒र मुख्य॑म् भृ॒त्वा । \newline
2. सं॒ॅव॒थ्स॒रमिति॑ सं - व॒थ्स॒रम् । \newline
3. उख्य॑म् भृ॒त्वा भृ॒त्वोख्य॒ मुख्य॑म् भृ॒त्वा द्वि॒तीये᳚ द्वि॒तीये॑ भृ॒त्वोख्य॒ मुख्य॑म् भृ॒त्वा द्वि॒तीये᳚ । \newline
4. भृ॒त्वा द्वि॒तीये᳚ द्वि॒तीये॑ भृ॒त्वा भृ॒त्वा द्वि॒तीये॑ संॅवथ्स॒रे सं॑ॅवथ्स॒रे द्वि॒तीये॑ भृ॒त्वा भृ॒त्वा द्वि॒तीये॑ संॅवथ्स॒रे । \newline
5. द्वि॒तीये॑ संॅवथ्स॒रे सं॑ॅवथ्स॒रे द्वि॒तीये᳚ द्वि॒तीये॑ संॅवथ्स॒र आ᳚ग्ने॒य मा᳚ग्ने॒यꣳ सं॑ॅवथ्स॒रे द्वि॒तीये᳚ द्वि॒तीये॑ संॅवथ्स॒र आ᳚ग्ने॒यम् । \newline
6. सं॒ॅव॒थ्स॒र आ᳚ग्ने॒य मा᳚ग्ने॒यꣳ सं॑ॅवथ्स॒रे सं॑ॅवथ्स॒र आ᳚ग्ने॒य म॒ष्टाक॑पाल म॒ष्टाक॑पाल माग्ने॒यꣳ सं॑ॅवथ्स॒रे सं॑ॅवथ्स॒र आ᳚ग्ने॒य म॒ष्टाक॑पालम् । \newline
7. सं॒ॅव॒थ्स॒र इति॑ सं - व॒थ्स॒रे । \newline
8. आ॒ग्ने॒य म॒ष्टाक॑पाल म॒ष्टाक॑पाल माग्ने॒य मा᳚ग्ने॒य म॒ष्टाक॑पाल॒म् निर् णिर॒ष्टाक॑पाल माग्ने॒य मा᳚ग्ने॒य म॒ष्टाक॑पाल॒म् निः । \newline
9. अ॒ष्टाक॑पाल॒म् निर् णिर॒ष्टाक॑पाल म॒ष्टाक॑पाल॒म् निर् व॑पेद् वपे॒न् निर॒ष्टाक॑पाल म॒ष्टाक॑पाल॒म् निर् व॑पेत् । \newline
10. अ॒ष्टाक॑पाल॒मित्य॒ष्टा - क॒पा॒ल॒म् । \newline
11. निर् व॑पेद् वपे॒न् निर् णिर् व॑पेदै॒न्द्र मै॒न्द्रं ॅव॑पे॒न् निर् णिर् व॑पेदै॒न्द्रम् । \newline
12. व॒पे॒दै॒न्द्र मै॒न्द्रं ॅव॑पेद् वपेदै॒न्द्र मेका॑दशकपाल॒ मेका॑दशकपाल मै॒न्द्रं ॅव॑पेद् 
वपेदै॒न्द्र मेका॑दशकपालम् । \newline
13. ऐ॒न्द्र मेका॑दशकपाल॒ मेका॑दशकपाल मै॒न्द्र मै॒न्द्र मेका॑दशकपालं ॅवैश्वदे॒वं ॅवै᳚श्वदे॒व मेका॑दशकपाल मै॒न्द्र मै॒न्द्र मेका॑दशकपालं ॅवैश्वदे॒वम् । \newline
14. एका॑दशकपालं ॅवैश्वदे॒वं ॅवै᳚श्वदे॒व मेका॑दशकपाल॒ मेका॑दशकपालं ॅवैश्वदे॒वम् द्वाद॑शकपाल॒म् द्वाद॑शकपालं ॅवैश्वदे॒व मेका॑दशकपाल॒ मेका॑दशकपालं ॅवैश्वदे॒वम् द्वाद॑शकपालम् । \newline
15. एका॑दशकपाल॒मित्येका॑दश - क॒पा॒ल॒म् । \newline
16. वै॒श्व॒दे॒वम् द्वाद॑शकपाल॒म् द्वाद॑शकपालं ॅवैश्वदे॒वं ॅवै᳚श्वदे॒वम् द्वाद॑शकपालम् बार्.हस्प॒त्यम् बा॑र्.हस्प॒त्यम् द्वाद॑शकपालं ॅवैश्वदे॒वं ॅवै᳚श्वदे॒वम् द्वाद॑शकपालम् बार्.हस्प॒त्यम् । \newline
17. वै॒श्व॒दे॒वमिति॑ वैश्व - दे॒वम् । \newline
18. द्वाद॑शकपालम् बार्.हस्प॒त्यम् बा॑र्.हस्प॒त्यम् द्वाद॑शकपाल॒म् द्वाद॑शकपालम् बार्.हस्प॒त्यम् च॒रुम् च॒रुम् बा॑र्.हस्प॒त्यम् द्वाद॑शकपाल॒म् द्वाद॑शकपालम् बार्.हस्प॒त्यम् च॒रुम् । \newline
19. द्वाद॑शकपाल॒मिति॒ द्वाद॑श - क॒पा॒ल॒म् । \newline
20. बा॒र्॒.ह॒स्प॒त्यम् च॒रुम् च॒रुम् बा॑र्.हस्प॒त्यम् बा॑र्.हस्प॒त्यम् च॒रुं ॅवै᳚ष्ण॒वं ॅवै᳚ष्ण॒वम् च॒रुम् बा॑र्.हस्प॒त्यम् बा॑र्.हस्प॒त्यम् च॒रुं ॅवै᳚ष्ण॒वम् । \newline
21. च॒रुं ॅवै᳚ष्ण॒वं ॅवै᳚ष्ण॒वम् च॒रुम् च॒रुं ॅवै᳚ष्ण॒वम् त्रि॑कपा॒लम् त्रि॑कपा॒लं ॅवै᳚ष्ण॒वम् च॒रुम् च॒रुं ॅवै᳚ष्ण॒वम् त्रि॑कपा॒लम् । \newline
22. वै॒ष्ण॒वम् त्रि॑कपा॒लम् त्रि॑कपा॒लं ॅवै᳚ष्ण॒वं ॅवै᳚ष्ण॒वम् त्रि॑कपा॒लम् तृ॒तीये॑ तृ॒तीये᳚ त्रिकपा॒लं ॅवै᳚ष्ण॒वं ॅवै᳚ष्ण॒वम् त्रि॑कपा॒लम् तृ॒तीये᳚ । \newline
23. त्रि॒क॒पा॒लम् तृ॒तीये॑ तृ॒तीये᳚ त्रिकपा॒लम् त्रि॑कपा॒लम् तृ॒तीये॑ संॅवथ्स॒रे सं॑ॅवथ्स॒रे तृ॒तीये᳚ त्रिकपा॒लम् त्रि॑कपा॒लम् तृ॒तीये॑ संॅवथ्स॒रे । \newline
24. त्रि॒क॒पा॒लमिति॑ त्रि - क॒पा॒लम् । \newline
25. तृ॒तीये॑ संॅवथ्स॒रे सं॑ॅवथ्स॒रे तृ॒तीये॑ तृ॒तीये॑ संॅवथ्स॒रे॑ ऽभि॒जिता॑ ऽभि॒जिता॑ संॅवथ्स॒रे तृ॒तीये॑ तृ॒तीये॑ संॅवथ्स॒रे॑ ऽभि॒जिता᳚ । \newline
26. सं॒ॅव॒थ्स॒रे॑ ऽभि॒जिता॑ ऽभि॒जिता॑ संॅवथ्स॒रे सं॑ॅवथ्स॒रे॑ ऽभि॒जिता॑ यजेत यजेता भि॒जिता॑ संॅवथ्स॒रे सं॑ॅवथ्स॒रे॑ ऽभि॒जिता॑ यजेत । \newline
27. सं॒ॅव॒थ्स॒र इति॑ सं - व॒थ्स॒रे । \newline
28. अ॒भि॒जिता॑ यजेत यजेता भि॒जिता॑ ऽभि॒जिता॑ यजेत॒ यद् यद् य॑जेता भि॒जिता॑ ऽभि॒जिता॑ यजेत॒ यत् । \newline
29. अ॒भि॒जितेत्य॑भि - जिता᳚ । \newline
30. य॒जे॒त॒ यद् यद् य॑जेत यजेत॒ यद॒ष्टाक॑पालो॒ ऽष्टाक॑पालो॒ यद् य॑जेत यजेत॒ यद॒ष्टाक॑पालः । \newline
31. यद॒ष्टाक॑पालो॒ ऽष्टाक॑पालो॒ यद् यद॒ष्टाक॑पालो॒ भव॑ति॒ भव॑ त्य॒ष्टाक॑पालो॒ यद् यद॒ष्टाक॑पालो॒ भव॑ति । \newline
32. अ॒ष्टाक॑पालो॒ भव॑ति॒ भव॑ त्य॒ष्टाक॑पालो॒ ऽष्टाक॑पालो॒ भव॑ त्य॒ष्टाक्ष॑रा॒ ऽष्टाक्ष॑रा॒ भव॑ त्य॒ष्टाक॑पालो॒ ऽष्टाक॑पालो॒ भव॑ त्य॒ष्टाक्ष॑रा । \newline
33. अ॒ष्टाक॑पाल॒ इत्य॒ष्टा - क॒पा॒लः॒ । \newline
34. भव॑ त्य॒ष्टाक्ष॑रा॒ ऽष्टाक्ष॑रा॒ भव॑ति॒ भव॑ त्य॒ष्टाक्ष॑रा गाय॒त्री गा॑य॒ त्र्य॑ष्टाक्ष॑रा॒ भव॑ति॒ भव॑ त्य॒ष्टाक्ष॑रा गाय॒त्री । \newline
35. अ॒ष्टाक्ष॑रा गाय॒त्री गा॑य॒ त्र्य॑ष्टाक्ष॑रा॒ ऽष्टाक्ष॑रा गाय॒ त्र्या᳚ग्ने॒य मा᳚ग्ने॒यम् गा॑य॒
त्र्य॑ष्टाक्ष॑रा॒ ऽष्टाक्ष॑रा गाय॒ त्र्या᳚ग्ने॒यम् । \newline
36. अ॒ष्टाक्ष॒रेत्य॒ष्टा - अ॒क्ष॒रा॒ । \newline
37. गा॒य॒ त्र्या᳚ग्ने॒य मा᳚ग्ने॒यम् गा॑य॒त्री गा॑य॒ त्र्या᳚ग्ने॒यम् गा॑य॒त्रम् गा॑य॒त्र मा᳚ग्ने॒यम् गा॑य॒त्री गा॑य॒ त्र्या᳚ग्ने॒यम् गा॑य॒त्रम् । \newline
38. आ॒ग्ने॒यम् गा॑य॒त्रम् गा॑य॒त्र मा᳚ग्ने॒य मा᳚ग्ने॒यम् गा॑य॒त्रम् प्रा॑तस्सव॒नम् प्रा॑तस्सव॒नम् गा॑य॒त्र मा᳚ग्ने॒य मा᳚ग्ने॒यम् गा॑य॒त्रम् प्रा॑तस्सव॒नम् । \newline
39. गा॒य॒त्रम् प्रा॑तस्सव॒नम् प्रा॑तस्सव॒नम् गा॑य॒त्रम् गा॑य॒त्रम् प्रा॑तस्सव॒नम् । \newline
40. प्रा॒त॒स्स॒व॒नम् प्रा॑तस्सव॒नम् । \newline
41. प्रा॒त॒स्स॒व॒नमिति॑ प्रातः - स॒व॒नम् । \newline
42. प्रा॒त॒स्स॒व॒न मे॒वैव प्रा॑तस्सव॒नम् प्रा॑तस्सव॒न मे॒व तेन॒ तेनै॒व प्रा॑तस्सव॒नम् प्रा॑तस्सव॒न मे॒व तेन॑ । \newline
43. प्रा॒त॒स्स॒व॒नमिति॑ प्रातः - स॒व॒नम् । \newline
44. ए॒व तेन॒ तेनै॒ वैव तेन॑ दाधार दाधार॒ तेनै॒ वैव तेन॑ दाधार । \newline
45. तेन॑ दाधार दाधार॒ तेन॒ तेन॑ दाधार गाय॒त्रीम् गा॑य॒त्रीम् दा॑धार॒ तेन॒ तेन॑ दाधार गाय॒त्रीम् । \newline
46. दा॒धा॒र॒ गा॒य॒त्रीम् गा॑य॒त्रीम् दा॑धार दाधार गाय॒त्रीम् छन्द॒ श्छन्दो॑ गाय॒त्रीम् दा॑धार दाधार गाय॒त्रीम् छन्दः॑ । \newline
47. गा॒य॒त्रीम् छन्द॒ श्छन्दो॑ गाय॒त्रीम् गा॑य॒त्रीम् छन्दो॒ यद् यच् छन्दो॑ गाय॒त्रीम् गा॑य॒त्रीम् छन्दो॒ यत् । \newline
48. छन्दो॒ यद् यच् छन्द॒ श्छन्दो॒ यदेका॑दशकपाल॒ एका॑दशकपालो॒ यच् छन्द॒ श्छन्दो॒ यदेका॑दशकपालः । \newline
49. यदेका॑दशकपाल॒ एका॑दशकपालो॒ यद् यदेका॑दशकपालो॒ भव॑ति॒ भव॒ त्येका॑दशकपालो॒ यद् यदेका॑दशकपालो॒ भव॑ति । \newline
50. एका॑दशकपालो॒ भव॑ति॒ भव॒ त्येका॑दशकपाल॒ एका॑दशकपालो॒ भव॒ त्येका॑दशाक्ष॒ रैका॑दशाक्षरा॒ भव॒ त्येका॑दशकपाल॒ एका॑दशकपालो॒ भव॒ त्येका॑दशाक्षरा । \newline
51. एका॑दशकपाल॒ इत्येका॑दश - क॒पा॒लः॒ । \newline
52. भव॒ त्येका॑दशाक्ष॒ रैका॑दशाक्षरा॒ भव॑ति॒ भव॒ त्येका॑दशाक्षरा त्रि॒ष्टुक् त्रि॒ष्टु गेका॑दशाक्षरा॒ भव॑ति॒ भव॒ त्येका॑दशाक्षरा त्रि॒ष्टुक् । \newline
53. एका॑दशाक्षरा त्रि॒ष्टुक् त्रि॒ष्टु गेका॑दशाक्ष॒ रैका॑दशाक्षरा त्रि॒ष्टु गै॒न्द्र मै॒न्द्रम् त्रि॒ष्टु गेका॑दशाक्ष॒ रैका॑दशाक्षरा त्रि॒ष्टु गै॒न्द्रम् । \newline
54. एका॑दशाक्ष॒रेत्येका॑दश - अ॒क्ष॒रा॒ । \newline
55. त्रि॒ष्टु गै॒न्द्र मै॒न्द्रम् त्रि॒ष्टुक् त्रि॒ष्टुग् ऐ॒न्द्रम् त्रैष्टु॑भ॒म् त्रैष्टु॑भ मै॒न्द्रम् त्रि॒ष्टुक् त्रि॒ष्टु गै॒न्द्रम् त्रैष्टु॑भम् । \newline
56. ऐ॒न्द्रम् त्रैष्टु॑भ॒म् त्रैष्टु॑भ मै॒न्द्र मै॒न्द्रम् त्रैष्टु॑भ॒म् माद्ध्य॑न्दिन॒म् माद्ध्य॑न्दिन॒म् त्रैष्टु॑भ मै॒न्द्र मै॒न्द्रम् त्रैष्टु॑भ॒म् माद्ध्य॑न्दिनम् । \newline
57. त्रैष्टु॑भ॒म् माद्ध्य॑न्दिन॒म् माद्ध्य॑न्दिन॒म् त्रैष्टु॑भ॒म् त्रैष्टु॑भ॒म् माद्ध्य॑न्दिनꣳ॒॒ सव॑नꣳ॒॒ सव॑न॒म् माद्ध्य॑न्दिन॒म् त्रैष्टु॑भ॒म् त्रैष्टु॑भ॒म् माद्ध्य॑न्दिनꣳ॒॒ सव॑नम् । \newline
58. माद्ध्य॑न्दिनꣳ॒॒ सव॑नꣳ॒॒ सव॑न॒म् माद्ध्य॑न्दिन॒म् माद्ध्य॑न्दिनꣳ॒॒ सव॑न॒म् माद्ध्य॑न्दिन॒म् माद्ध्य॑न्दिनꣳ॒॒ सव॑न॒म् माद्ध्य॑न्दिन॒म् माद्ध्य॑न्दिनꣳ॒॒ सव॑न॒म् माद्ध्य॑न्दिनम् । \newline
59. सव॑न॒म् माद्ध्य॑न्दिन॒म् माद्ध्य॑न्दिनꣳ॒॒ सव॑नꣳ॒॒ सव॑न॒म् माद्ध्य॑न्दिन मे॒वैव माद्ध्य॑न्दिनꣳ॒॒ सव॑नꣳ॒॒ सव॑न॒म् माद्ध्य॑न्दिन मे॒व । \newline
60. माद्ध्य॑न्दिन मे॒वैव माद्ध्य॑न्दिन॒म् माद्ध्य॑न्दिन मे॒व सव॑नꣳ॒॒ सव॑न मे॒व माद्ध्य॑न्दिन॒म् माद्ध्य॑न्दिन मे॒व सव॑नम् । \newline
61. ए॒व सव॑नꣳ॒॒ सव॑न मे॒वैव सव॑न॒म् तेन॒ तेन॒ सव॑न मे॒वैव सव॑न॒म् तेन॑ । \newline
62. सव॑न॒म् तेन॒ तेन॒ सव॑नꣳ॒॒ सव॑न॒म् तेन॑ दाधार दाधार॒ तेन॒ सव॑नꣳ॒॒ सव॑न॒म् तेन॑ दाधार । \newline
63. तेन॑ दाधार दाधार॒ तेन॒ तेन॑ दाधार त्रि॒ष्टुभ॑म् त्रि॒ष्टुभ॑म् दाधार॒ तेन॒ तेन॑ दाधार त्रि॒ष्टुभ᳚म् । \newline
64. दा॒धा॒र॒ त्रि॒ष्टुभ॑म् त्रि॒ष्टुभ॑म् दाधार दाधार त्रि॒ष्टुभ॒म् छन्द॒ श्छन्द॑ स्त्रि॒ष्टुभ॑म् दाधार दाधार त्रि॒ष्टुभ॒म् छन्दः॑ । \newline
65. त्रि॒ष्टुभ॒म् छन्द॒ श्छन्द॑ स्त्रि॒ष्टुभ॑म् त्रि॒ष्टुभ॒म् छन्दो॒ यद् यच् छन्द॑ स्त्रि॒ष्टुभ॑म् त्रि॒ष्टुभ॒म् छन्दो॒ यत् । \newline
\pagebreak
\markright{ TS 5.6.5.2  \hfill https://www.vedavms.in \hfill}

\section{ TS 5.6.5.2 }

\textbf{TS 5.6.5.2 } \newline
\textbf{Samhita Paata} \newline

छन्दो॒ यद्-द्वाद॑शकपालो॒ भव॑ति॒ द्वाद॑शाक्षरा॒ जग॑ती वैश्वदे॒वं जाग॑तं तृतीयसव॒नं तृ॑तीयसव॒नमे॒व तेन॑ दाधार॒ जग॑तीं॒ छन्दो॒ यद्-बा॑र्.हस्प॒त्यश्च॒रुर्भव॑ति॒ ब्रह्म॒ वै दे॒वानां॒ बृह॒स्पति॒र्ब्रह्मै॒व तेन॑ दाधार॒ यद्-वै᳚ष्ण॒वस्त्रि॑कपा॒लो भव॑ति य॒ज्ञो वै विष्णु॑र्य॒ज्ञ्मे॒व तेन॑ दाधार॒ यत् तृ॒तीये॑ संॅवथ्स॒रे॑ऽभि॒जिता॒ यज॑ते॒ऽभिजि॑त्यै॒ यथ् सं॑ॅवथ्स॒रमुख्यं॑ बि॒भर्ती॒ममे॒व - [  ] \newline

\textbf{Pada Paata} \newline

छन्दः॑ । यत् । द्वाद॑शकपाल॒ इति॒ द्वाद॑श - क॒पा॒लः॒ । भव॑ति । द्वाद॑शाक्ष॒रेति॒ द्वाद॑शा-अ॒क्ष॒रा॒ । जग॑ती । वै॒श्व॒दे॒वमिति॑ वैश्व - दे॒वम् । जाग॑तम् । तृ॒ती॒य॒स॒व॒नमिति॑ तृतीय - स॒व॒नम् । तृ॒ती॒य॒स॒व॒नमिति॑ तृतीय - स॒व॒नम् । ए॒व । तेन॑ । दा॒धा॒र॒ । जग॑तीम् । छन्दः॑ । यत् । बा॒र्.॒ह॒स्प॒त्यः । च॒रुः । भव॑ति । ब्रह्म॑ । वै । दे॒वाना᳚म् । बृह॒स्पतिः॑ । ब्रह्म॑ । ए॒व । तेन॑ । दा॒धा॒र॒ । यत् । वै॒ष्ण॒वः । त्रि॒क॒पा॒ल इति॑ त्रि - क॒पा॒लः । भव॑ति । य॒ज्ञ्ः । वै । विष्णुः॑ । य॒ज्ञ्म् । ए॒व । तेन॑ । दा॒धा॒र॒ । यत् । तृ॒तीये᳚ । सं॒ॅव॒थ्स॒र इति॑ सं - व॒थ्स॒रे । अ॒भि॒जितेत्य॑भि - जिता᳚ । यज॑ते । अ॒भिजि॑त्या॒ इत्य॒भि - जि॒त्यै॒ । यत् । सं॒ॅव॒थ्स॒रमिति॑ सं - व॒थ्स॒रम् । उख्य᳚म् । बि॒भर्ति॑ । इ॒मम् । ए॒व ।  \newline


\textbf{Krama Paata} \newline

छन्दो॒ यत् । यद् द्वाद॑शकपालः । द्वाद॑शकपालो॒ भव॑ति । द्वाद॑शकपाल॒ इति॒ द्वाद॑श - क॒पा॒लः॒ । भव॑ति॒ द्वाद॑शाक्षरा । द्वाद॑शाक्षरा॒ जग॑ती । द्वाद॑शाक्ष॒रेति॒ द्वाद॑श - अ॒क्ष॒रा॒ । जग॑ती वैश्वदे॒वम् । वै॒श्व॒दे॒वम् जाग॑तम् । वै॒श्व॒दे॒वमिति॑ वैश्व - दे॒वम् । जाग॑तम् तृतीयसव॒नम् । तृ॒ती॒य॒स॒व॒नम् तृ॑तीयसव॒नम् । तृ॒ती॒य॒स॒व॒नमिति॑ तृतीय - स॒व॒नम् । तृ॒ती॒य॒स॒व॒नमे॒व । तृ॒ती॒य॒स॒व॒नमिति॑ तृतीय - स॒व॒नम् । ए॒व तेन॑ । तेन॑ दाधार । दा॒धा॒र॒ जग॑तीम् । जग॑ती॒म् छन्दः॑ । छन्दो॒ यत् । यद् बा॑र्.हस्प॒त्यः । बा॒र्॒.ह॒स्प॒त्यश्च॒रुः । च॒रुर् भव॑ति । भव॑ति॒ ब्रह्म॑ । ब्रह्म॒ वै । वै दे॒वाना᳚म् । दे॒वाना॒म् बृह॒स्पतिः॑ । बृह॒स्पति॒र् ब्रह्म॑ । ब्रह्मै॒व । ए॒व तेन॑ । तेन॑ दाधार । दा॒धा॒र॒ यत् । यद् वै᳚ष्ण॒वः । वै॒ष्ण॒वस्त्रि॑कपा॒लः । त्रि॒क॒पा॒लो भव॑ति । त्रि॒क॒पा॒ल इति॑ त्रि - क॒पा॒लः । भव॑ति य॒ज्ञ्ः । य॒ज्ञो वै । वै विष्णुः॑ । विष्णु॑र् य॒ज्ञ्म् । य॒ज्ञ्मे॒व । ए॒व तेन॑ । तेन॑ दाधार । दा॒धा॒र॒ यत् । यत् तृ॒तीये᳚ । तृ॒तीये॑ सम्ॅवथ्स॒रे । स॒म्ॅव॒थ्स॒रे॑ऽभि॒जिता᳚ । स॒म्ॅव॒थ्स॒र इति॑ सम् - व॒थ्स॒रे । अ॒भि॒जिता॒ यज॑ते । अ॒भि॒जितेत्य॑भि - जिता᳚ । यज॑ते॒ऽभिजि॑त्यै । अ॒भिजि॑त्यै॒ यत् । अ॒भिजि॑त्या॒ इत्य॒भि - जि॒त्यै॒ । यथ् स॑म्ॅवथ्स॒रम् । स॒म्ॅव॒थ्स॒रमुख्य᳚म् । स॒म्ॅव॒थ्स॒रमिति॑ सम् - व॒थ्स॒रम् । उख्य॑म् बि॒भर्ति॑ । बि॒भर्ती॒मम् । इ॒ममे॒व । ए॒व तेन॑ \newline

\textbf{Jatai Paata} \newline

1. छन्दो॒ यद् यच् छन्द॒ श्छन्दो॒ यत् । \newline
2. यद् द्वाद॑शकपालो॒ द्वाद॑शकपालो॒ यद् यद् द्वाद॑शकपालः । \newline
3. द्वाद॑शकपालो॒ भव॑ति॒ भव॑ति॒ द्वाद॑शकपालो॒ द्वाद॑शकपालो॒ भव॑ति । \newline
4. द्वाद॑शकपाल॒ इति॒ द्वाद॑श - क॒पा॒लः॒ । \newline
5. भव॑ति॒ द्वाद॑शाक्षरा॒ द्वाद॑शाक्षरा॒ भव॑ति॒ भव॑ति॒ द्वाद॑शाक्षरा । \newline
6. द्वाद॑शाक्षरा॒ जग॑ती॒ जग॑ती॒ द्वाद॑शाक्षरा॒ द्वाद॑शाक्षरा॒ जग॑ती । \newline
7. द्वाद॑शाक्ष॒रेति॒ द्वाद॑शा - अ॒क्ष॒रा॒ । \newline
8. जग॑ती वैश्वदे॒वं ॅवै᳚श्वदे॒वम् जग॑ती॒ जग॑ती वैश्वदे॒वम् । \newline
9. वै॒श्व॒दे॒वम् जाग॑त॒म् जाग॑तं ॅवैश्वदे॒वं ॅवै᳚श्वदे॒वम् जाग॑तम् । \newline
10. वै॒श्व॒दे॒वमिति॑ वैश्व - दे॒वम् । \newline
11. जाग॑तम् तृतीयसव॒नम् तृ॑तीयसव॒नम् जाग॑त॒म् जाग॑तम् तृतीयसव॒नम् । \newline
12. तृ॒ती॒य॒स॒व॒नम् तृ॑तीयसव॒नम् । \newline
13. तृ॒ती॒य॒स॒व॒नमिति॑ तृतीय - स॒व॒नम् । \newline
14. तृ॒ती॒य॒स॒व॒न मे॒वैव तृ॑तीयसव॒नम् तृ॑तीयसव॒न मे॒व । \newline
15. तृ॒ती॒य॒स॒व॒नमिति॑ तृतीय - स॒व॒नम् । \newline
16. ए॒व तेन॒ तेनै॒ वैव तेन॑ । \newline
17. तेन॑ दाधार दाधार॒ तेन॒ तेन॑ दाधार । \newline
18. दा॒धा॒र॒ जग॑ती॒म् जग॑तीम् दाधार दाधार॒ जग॑तीम् । \newline
19. जग॑ती॒म् छन्द॒ श्छन्दो॒ जग॑ती॒म् जग॑ती॒म् छन्दः॑ । \newline
20. छन्दो॒ यद् यच् छन्द॒ श्छन्दो॒ यत् । \newline
21. यद् बा॑र्.हस्प॒त्यो बा॑र्.हस्प॒त्यो यद् यद् बा॑र्.हस्प॒त्यः । \newline
22. बा॒र्॒.ह॒स्प॒त्य श्च॒रु श्च॒रुर् बा॑र्.हस्प॒त्यो बा॑र्.हस्प॒त्य श्च॒रुः । \newline
23. च॒रुर् भव॑ति॒ भव॑ति च॒रु श्च॒रुर् भव॑ति । \newline
24. भव॑ति॒ ब्रह्म॒ ब्रह्म॒ भव॑ति॒ भव॑ति॒ ब्रह्म॑ । \newline
25. ब्रह्म॒ वै वै ब्रह्म॒ ब्रह्म॒ वै । \newline
26. वै दे॒वाना᳚म् दे॒वानां॒ ॅवै वै दे॒वाना᳚म् । \newline
27. दे॒वाना॒म् बृह॒स्पति॒र् बृह॒स्पति॑र् दे॒वाना᳚म् दे॒वाना॒म् बृह॒स्पतिः॑ । \newline
28. बृह॒स्पति॒र् ब्रह्म॒ ब्रह्म॒ बृह॒स्पति॒र् बृह॒स्पति॒र् ब्रह्म॑ । \newline
29. ब्रह्मै॒ वैव ब्रह्म॒ ब्रह्मै॒व । \newline
30. ए॒व तेन॒ तेनै॒ वैव तेन॑ । \newline
31. तेन॑ दाधार दाधार॒ तेन॒ तेन॑ दाधार । \newline
32. दा॒धा॒र॒ यद् यद् दा॑धार दाधार॒ यत् । \newline
33. यद् वै᳚ष्ण॒वो वै᳚ष्ण॒वो यद् यद् वै᳚ष्ण॒वः । \newline
34. वै॒ष्ण॒व स्त्रि॑कपा॒ल स्त्रि॑कपा॒लो वै᳚ष्ण॒वो वै᳚ष्ण॒व स्त्रि॑कपा॒लः । \newline
35. त्रि॒क॒पा॒लो भव॑ति॒ भव॑ति त्रिकपा॒ल स्त्रि॑कपा॒लो भव॑ति । \newline
36. त्रि॒क॒पा॒ल इति॑ त्रि - क॒पा॒लः । \newline
37. भव॑ति य॒ज्ञो य॒ज्ञो भव॑ति॒ भव॑ति य॒ज्ञ्ः । \newline
38. य॒ज्ञो वै वै य॒ज्ञो य॒ज्ञो वै । \newline
39. वै विष्णु॒र् विष्णु॒र् वै वै विष्णुः॑ । \newline
40. विष्णु॑र् य॒ज्ञ्ं ॅय॒ज्ञ्ं ॅविष्णु॒र् विष्णु॑र् य॒ज्ञ्म् । \newline
41. य॒ज्ञ् मे॒ वैव य॒ज्ञ्ं ॅय॒ज्ञ् मे॒व । \newline
42. ए॒व तेन॒ तेनै॒ वैव तेन॑ । \newline
43. तेन॑ दाधार दाधार॒ तेन॒ तेन॑ दाधार । \newline
44. दा॒धा॒र॒ यद् यद् दा॑धार दाधार॒ यत् । \newline
45. यत् तृ॒तीये॑ तृ॒तीये॒ यद् यत् तृ॒तीये᳚ । \newline
46. तृ॒तीये॑ संॅवथ्स॒रे सं॑ॅवथ्स॒रे तृ॒तीये॑ तृ॒तीये॑ संॅवथ्स॒रे । \newline
47. सं॒ॅव॒थ्स॒रे॑ ऽभि॒जिता॑ ऽभि॒जिता॑ संॅवथ्स॒रे सं॑ॅवथ्स॒रे॑ ऽभि॒जिता᳚ । \newline
48. सं॒ॅव॒थ्स॒र इति॑ सं - व॒थ्स॒रे । \newline
49. अ॒भि॒जिता॒ यज॑ते॒ यज॑ते ऽभि॒जिता॑ ऽभि॒जिता॒ यज॑ते । \newline
50. अ॒भि॒जितेत्य॑भि - जिता᳚ । \newline
51. यज॑ते॒ ऽभिजि॑त्या अ॒भिजि॑त्यै॒ यज॑ते॒ यज॑ते॒ ऽभिजि॑त्यै । \newline
52. अ॒भिजि॑त्यै॒ यद् यद॒भिजि॑त्या अ॒भिजि॑त्यै॒ यत् । \newline
53. अ॒भिजि॑त्या॒ इत्य॒भि - जि॒त्यै॒ । \newline
54. यथ् सं॑ॅवथ्स॒रꣳ सं॑ॅवथ्स॒रं ॅयद् यथ् सं॑ॅवथ्स॒रम् । \newline
55. सं॒ॅव॒थ्स॒र मुख्य॒ मुख्यꣳ॑ संॅवथ्स॒रꣳ सं॑ॅवथ्स॒र मुख्य᳚म् । \newline
56. सं॒ॅव॒थ्स॒रमिति॑ सं - व॒थ्स॒रम् । \newline
57. उख्य॑म् बि॒भर्ति॑ बि॒भर्त्युख्य॒ मुख्य॑म् बि॒भर्ति॑ । \newline
58. बि॒भर्ती॒म मि॒मम् बि॒भर्ति॑ बि॒भर्ती॒मम् । \newline
59. इ॒म मे॒वैवे म मि॒म मे॒व । \newline
60. ए॒व तेन॒ तेनै॒ वैव तेन॑ । \newline

\textbf{Ghana Paata } \newline

1. छन्दो॒ यद् यच् छन्द॒ श्छन्दो॒ यद् द्वाद॑शकपालो॒ द्वाद॑शकपालो॒ यच् छन्द॒ श्छन्दो॒ यद् द्वाद॑शकपालः । \newline
2. यद् द्वाद॑शकपालो॒ द्वाद॑शकपालो॒ यद् यद् द्वाद॑शकपालो॒ भव॑ति॒ भव॑ति॒ द्वाद॑शकपालो॒ यद् यद् द्वाद॑शकपालो॒ भव॑ति । \newline
3. द्वाद॑शकपालो॒ भव॑ति॒ भव॑ति॒ द्वाद॑शकपालो॒ द्वाद॑शकपालो॒ भव॑ति॒ द्वाद॑शाक्षरा॒ द्वाद॑शाक्षरा॒ भव॑ति॒ द्वाद॑शकपालो॒ द्वाद॑शकपालो॒ भव॑ति॒ द्वाद॑शाक्षरा । \newline
4. द्वाद॑शकपाल॒ इति॒ द्वाद॑श - क॒पा॒लः॒ । \newline
5. भव॑ति॒ द्वाद॑शाक्षरा॒ द्वाद॑शाक्षरा॒ भव॑ति॒ भव॑ति॒ द्वाद॑शाक्षरा॒ जग॑ती॒ जग॑ती॒ द्वाद॑शाक्षरा॒ भव॑ति॒ भव॑ति॒ द्वाद॑शाक्षरा॒ जग॑ती । \newline
6. द्वाद॑शाक्षरा॒ जग॑ती॒ जग॑ती॒ द्वाद॑शाक्षरा॒ द्वाद॑शाक्षरा॒ जग॑ती वैश्वदे॒वं ॅवै᳚श्वदे॒वम् जग॑ती॒ द्वाद॑शाक्षरा॒ द्वाद॑शाक्षरा॒ जग॑ती वैश्वदे॒वम् । \newline
7. द्वाद॑शाक्ष॒रेति॒ द्वाद॑शा - अ॒क्ष॒रा॒ । \newline
8. जग॑ती वैश्वदे॒वं ॅवै᳚श्वदे॒वम् जग॑ती॒ जग॑ती वैश्वदे॒वम् जाग॑त॒म् जाग॑तं ॅवैश्वदे॒वम् जग॑ती॒ जग॑ती वैश्वदे॒वम् जाग॑तम् । \newline
9. वै॒श्व॒दे॒वम् जाग॑त॒म् जाग॑तं ॅवैश्वदे॒वं ॅवै᳚श्वदे॒वम् जाग॑तम् तृतीयसव॒नम् तृ॑तीयसव॒नम् जाग॑तं ॅवैश्वदे॒वं ॅवै᳚श्वदे॒वम् जाग॑तम् तृतीयसव॒नम् । \newline
10. वै॒श्व॒दे॒वमिति॑ वैश्व - दे॒वम् । \newline
11. जाग॑तम् तृतीयसव॒नम् तृ॑तीयसव॒नम् जाग॑त॒म् जाग॑तम् तृतीयसव॒नम् । \newline
12. तृ॒ती॒य॒स॒व॒नम् तृ॑तीयसव॒नम् । \newline
13. तृ॒ती॒य॒स॒व॒नमिति॑ तृतीय - स॒व॒नम् । \newline
14. तृ॒ती॒य॒स॒व॒न मे॒वैव तृ॑तीयसव॒नम् तृ॑तीयसव॒न मे॒व तेन॒ तेनै॒व तृ॑तीयसव॒नम् तृ॑तीयसव॒न मे॒व तेन॑ । \newline
15. तृ॒ती॒य॒स॒व॒नमिति॑ तृतीय - स॒व॒नम् । \newline
16. ए॒व तेन॒ तेनै॒ वैव तेन॑ दाधार दाधार॒ तेनै॒ वैव तेन॑ दाधार । \newline
17. तेन॑ दाधार दाधार॒ तेन॒ तेन॑ दाधार॒ जग॑ती॒म् जग॑तीम् दाधार॒ तेन॒ तेन॑ दाधार॒ जग॑तीम् । \newline
18. दा॒धा॒र॒ जग॑ती॒म् जग॑तीम् दाधार दाधार॒ जग॑ती॒म् छन्द॒ श्छन्दो॒ जग॑तीम् दाधार दाधार॒ जग॑ती॒म् छन्दः॑ । \newline
19. जग॑ती॒म् छन्द॒ श्छन्दो॒ जग॑ती॒म् जग॑ती॒म् छन्दो॒ यद् यच् छन्दो॒ जग॑ती॒म् जग॑ती॒म् छन्दो॒ यत् । \newline
20. छन्दो॒ यद् यच् छन्द॒ श्छन्दो॒ यद् बा॑र्.हस्प॒त्यो बा॑र्.हस्प॒त्यो यच् छन्द॒ श्छन्दो॒ यद् बा॑र्.हस्प॒त्यः । \newline
21. यद् बा॑र्.हस्प॒त्यो बा॑र्.हस्प॒त्यो यद् यद् बा॑र्.हस्प॒त्य श्च॒रु श्च॒रुर् बा॑र्.हस्प॒त्यो यद् यद् बा॑र्.हस्प॒त्य श्च॒रुः । \newline
22. बा॒र्॒.ह॒स्प॒त्य श्च॒रु श्च॒रुर् बा॑र्.हस्प॒त्यो बा॑र्.हस्प॒त्य श्च॒रुर् भव॑ति॒ भव॑ति च॒रुर् बा॑र्.हस्प॒त्यो बा॑र्.हस्प॒त्य श्च॒रुर् भव॑ति । \newline
23. च॒रुर् भव॑ति॒ भव॑ति च॒रु श्च॒रुर् भव॑ति॒ ब्रह्म॒ ब्रह्म॒ भव॑ति च॒रु श्च॒रुर् भव॑ति॒ ब्रह्म॑ । \newline
24. भव॑ति॒ ब्रह्म॒ ब्रह्म॒ भव॑ति॒ भव॑ति॒ ब्रह्म॒ वै वै ब्रह्म॒ भव॑ति॒ भव॑ति॒ ब्रह्म॒ वै । \newline
25. ब्रह्म॒ वै वै ब्रह्म॒ ब्रह्म॒ वै दे॒वाना᳚म् दे॒वानां॒ ॅवै ब्रह्म॒ ब्रह्म॒ वै दे॒वाना᳚म् । \newline
26. वै दे॒वाना᳚म् दे॒वानां॒ ॅवै वै दे॒वाना॒म् बृह॒स्पति॒र् बृह॒स्पति॑र् दे॒वानां॒ ॅवै वै दे॒वाना॒म् बृह॒स्पतिः॑ । \newline
27. दे॒वाना॒म् बृह॒स्पति॒र् बृह॒स्पति॑र् दे॒वाना᳚म् दे॒वाना॒म् बृह॒स्पति॒र् ब्रह्म॒ ब्रह्म॒ बृह॒स्पति॑र् दे॒वाना᳚म् दे॒वाना॒म् बृह॒स्पति॒र् ब्रह्म॑ । \newline
28. बृह॒स्पति॒र् ब्रह्म॒ ब्रह्म॒ बृह॒स्पति॒र् बृह॒स्पति॒र् ब्रह्मै॒वैव ब्रह्म॒ बृह॒स्पति॒र् बृह॒स्पति॒र् ब्रह्मै॒व । \newline
29. ब्रह्मै॒ वैव ब्रह्म॒ ब्रह्मै॒व तेन॒ तेनै॒व ब्रह्म॒ ब्रह्मै॒व तेन॑ । \newline
30. ए॒व तेन॒ तेनै॒वैव तेन॑ दाधार दाधार॒ तेनै॒वैव तेन॑ दाधार । \newline
31. तेन॑ दाधार दाधार॒ तेन॒ तेन॑ दाधार॒ यद् यद् दा॑धार॒ तेन॒ तेन॑ दाधार॒ यत् । \newline
32. दा॒धा॒र॒ यद् यद् दा॑धार दाधार॒ यद् वै᳚ष्ण॒वो वै᳚ष्ण॒वो यद् दा॑धार दाधार॒ यद् वै᳚ष्ण॒वः । \newline
33. यद् वै᳚ष्ण॒वो वै᳚ष्ण॒वो यद् यद् वै᳚ष्ण॒व स्त्रि॑कपा॒ल स्त्रि॑कपा॒लो वै᳚ष्ण॒वो यद् यद् वै᳚ष्ण॒व स्त्रि॑कपा॒लः । \newline
34. वै॒ष्ण॒व स्त्रि॑कपा॒ल स्त्रि॑कपा॒लो वै᳚ष्ण॒वो वै᳚ष्ण॒व स्त्रि॑कपा॒लो भव॑ति॒ भव॑ति त्रिकपा॒लो वै᳚ष्ण॒वो वै᳚ष्ण॒व स्त्रि॑कपा॒लो भव॑ति । \newline
35. त्रि॒क॒पा॒लो भव॑ति॒ भव॑ति त्रिकपा॒ल स्त्रि॑कपा॒लो भव॑ति य॒ज्ञो य॒ज्ञो भव॑ति त्रिकपा॒ल स्त्रि॑कपा॒लो भव॑ति य॒ज्ञ्ः । \newline
36. त्रि॒क॒पा॒ल इति॑ त्रि - क॒पा॒लः । \newline
37. भव॑ति य॒ज्ञो य॒ज्ञो भव॑ति॒ भव॑ति य॒ज्ञो वै वै य॒ज्ञो भव॑ति॒ भव॑ति य॒ज्ञो वै । \newline
38. य॒ज्ञो वै वै य॒ज्ञो य॒ज्ञो वै विष्णु॒र् विष्णु॒र् वै य॒ज्ञो य॒ज्ञो वै विष्णुः॑ । \newline
39. वै विष्णु॒र् विष्णु॒र् वै वै विष्णु॑र् य॒ज्ञ्ं ॅय॒ज्ञ्ं ॅविष्णु॒र् वै वै विष्णु॑र् य॒ज्ञ्म् । \newline
40. विष्णु॑र् य॒ज्ञ्ं ॅय॒ज्ञ्ं ॅविष्णु॒र् विष्णु॑र् य॒ज्ञ् मे॒वैव य॒ज्ञ्ं ॅविष्णु॒र् विष्णु॑र् य॒ज्ञ् मे॒व । \newline
41. य॒ज्ञ् मे॒वैव य॒ज्ञ्ं ॅय॒ज्ञ् मे॒व तेन॒ तेनै॒व य॒ज्ञ्ं ॅय॒ज्ञ् मे॒व तेन॑ । \newline
42. ए॒व तेन॒ तेनै॒ वैव तेन॑ दाधार दाधार॒ तेनै॒ वैव तेन॑ दाधार । \newline
43. तेन॑ दाधार दाधार॒ तेन॒ तेन॑ दाधार॒ यद् यद् दा॑धार॒ तेन॒ तेन॑ दाधार॒ यत् । \newline
44. दा॒धा॒र॒ यद् यद् दा॑धार दाधार॒ यत् तृ॒तीये॑ तृ॒तीये॒ यद् दा॑धार दाधार॒ यत् तृ॒तीये᳚ । \newline
45. यत् तृ॒तीये॑ तृ॒तीये॒ यद् यत् तृ॒तीये॑ संॅवथ्स॒रे सं॑ॅवथ्स॒रे तृ॒तीये॒ यद् यत् तृ॒तीये॑ संॅवथ्स॒रे । \newline
46. तृ॒तीये॑ संॅवथ्स॒रे सं॑ॅवथ्स॒रे तृ॒तीये॑ तृ॒तीये॑ संॅवथ्स॒रे॑ ऽभि॒जिता॑ ऽभि॒जिता॑ संॅवथ्स॒रे तृ॒तीये॑ तृ॒तीये॑ संॅवथ्स॒रे॑ ऽभि॒जिता᳚ । \newline
47. सं॒ॅव॒थ्स॒रे॑ ऽभि॒जिता॑ ऽभि॒जिता॑ संॅवथ्स॒रे सं॑ॅवथ्स॒रे॑ ऽभि॒जिता॒ यज॑ते॒ यज॑ते ऽभि॒जिता॑ संॅवथ्स॒रे सं॑ॅवथ्स॒रे॑ ऽभि॒जिता॒ यज॑ते । \newline
48. सं॒ॅव॒थ्स॒र इति॑ सं - व॒थ्स॒रे । \newline
49. अ॒भि॒जिता॒ यज॑ते॒ यज॑ते ऽभि॒जिता॑ ऽभि॒जिता॒ यज॑ते॒ ऽभिजि॑त्या अ॒भिजि॑त्यै॒ यज॑ते ऽभि॒जिता॑ ऽभि॒जिता॒ यज॑ते॒ ऽभिजि॑त्यै । \newline
50. अ॒भि॒जितेत्य॑भि - जिता᳚ । \newline
51. यज॑ते॒ ऽभिजि॑त्या अ॒भिजि॑त्यै॒ यज॑ते॒ यज॑ते॒ ऽभिजि॑त्यै॒ यद् यद॒भिजि॑त्यै॒ यज॑ते॒ यज॑ते॒ ऽभिजि॑त्यै॒ यत् । \newline
52. अ॒भिजि॑त्यै॒ यद् यद॒भिजि॑त्या अ॒भिजि॑त्यै॒ यथ् सं॑ॅवथ्स॒रꣳ सं॑ॅवथ्स॒रं ॅयद॒भिजि॑त्या अ॒भिजि॑त्यै॒ यथ् सं॑ॅवथ्स॒रम् । \newline
53. अ॒भिजि॑त्या॒ इत्य॒भि - जि॒त्यै॒ । \newline
54. यथ् सं॑ॅवथ्स॒रꣳ सं॑ॅवथ्स॒रं ॅयद् यथ् सं॑ॅवथ्स॒र मुख्य॒ मुख्यꣳ॑ संॅवथ्स॒रं ॅयद् यथ् सं॑ॅवथ्स॒र मुख्य᳚म् । \newline
55. सं॒ॅव॒थ्स॒र मुख्य॒ मुख्यꣳ॑ संॅवथ्स॒रꣳ सं॑ॅवथ्स॒र मुख्य॑म् बि॒भर्ति॑ बि॒भर् त्युख्यꣳ॑ संॅवथ्स॒रꣳ सं॑ॅवथ्स॒र मुख्य॑म् बि॒भर्ति॑ । \newline
56. सं॒ॅव॒थ्स॒रमिति॑ सं - व॒थ्स॒रम् । \newline
57. उख्य॑म् बि॒भर्ति॑ बि॒भर् त्युख्य॒ मुख्य॑म् बि॒भर्ती॒म मि॒मम् बि॒भर् त्युख्य॒ मुख्य॑म् बि॒भर्ती॒मम् । \newline
58. बि॒भर्ती॒म मि॒मम् बि॒भर्ति॑ बि॒भर्ती॒म मे॒वैवेमम् बि॒भर्ति॑ बि॒भर्ती॒म मे॒व । \newline
59. इ॒म मे॒वैवेम मि॒म मे॒व तेन॒ तेनै॒वेम मि॒म मे॒व तेन॑ । \newline
60. ए॒व तेन॒ तेनै॒ वैव तेन॑ लो॒कम् ॅलो॒कम् तेनै॒ वैव तेन॑ लो॒कम् । \newline
\pagebreak
\markright{ TS 5.6.5.3  \hfill https://www.vedavms.in \hfill}

\section{ TS 5.6.5.3 }

\textbf{TS 5.6.5.3 } \newline
\textbf{Samhita Paata} \newline

तेन॑ लो॒कꣳ स्पृ॑णोति॒ यद् द्वि॒तीये॑ संॅवथ्स॒रे᳚ऽग्निं चि॑नु॒ते᳚ ऽन्तरि॑क्षमे॒व तेन॑ स्पृणोति॒ यत् तृ॒तीये॑ संॅवथ्स॒रे यज॑ते॒ऽमुमे॒व तेन॑ लो॒कꣳ स्पृ॑णोत्ये॒तं ॅवै पर॑ आट्णा॒रः क॒क्षीवाꣳ॑ औशि॒जो वी॒तह॑व्यः श्राय॒सस्त्र॒सद॑स्युः पौरुकु॒थ्स्यः प्र॒जाका॑मा अचिन्वत॒ ततो॒ वै ते स॒हस्रꣳ॑ सहस्रं पु॒त्रान॑विन्दन्त॒ प्रथ॑ते प्र॒जया॑ प॒शुभि॒स्तां मात्रा॑माप्नोति॒ यां तेऽग॑च्छ॒न्॒ य ए॒वं ( ) ॅवि॒द्वाने॒तम॒ग्निं चि॑नु॒ते ॥ \newline

\textbf{Pada Paata} \newline

तेन॑ । लो॒कम् । स्पृ॒णो॒ति॒ । यत् । द्वि॒तीये᳚ । सं॒ॅव॒थ्स॒र इति॑ सं - व॒थ्स॒रे । अ॒ग्निम् । चि॒नु॒ते । अ॒न्तरि॑क्षम् । ए॒व । तेन॑ । स्पृ॒णो॒ति॒ । यत् । तृ॒तीये᳚ । सं॒ॅव॒थ्स॒र इति॑ सं - व॒थ्स॒रे । यज॑ते । अ॒मुम् । ए॒व । तेन॑ । लो॒कम् । स्पृ॒णो॒ति॒ । ए॒तम् । वै । परः॑ । आ॒ट्णा॒रः । क॒क्षीवा॒निति॑ क॒क्षी - वा॒न् । औ॒शि॒जः । वी॒तह॑व्य॒ इति॑ वी॒त - ह॒व्यः॒ । श्रा॒य॒सः । त्र॒सद॑स्युः । पौ॒रु॒कु॒थ्स्य इति॑ पौरु - कु॒थ्स्यः । प्र॒जाका॑मा॒ इति॑ प्र॒जा-का॒माः॒ । अ॒चि॒न्व॒त॒ । ततः॑ । वै । ते । स॒हस्रꣳ॑सहस्र॒मिति॑ स॒हस्रं᳚ - स॒ह॒स्र॒म् । पु॒त्रान् । अ॒वि॒न्द॒न्त॒ । प्रथ॑ते । प्र॒जयेति॑ प्र - जया᳚ । प॒शुभि॒रिति॑ प॒शु-भिः॒ । ताम् । मात्रा᳚म् । आ॒प्नो॒ति॒ । याम् । ते॒ । अग॑च्छन्न् । यः । ए॒वम् ( ) । वि॒द्वान् । ए॒तम् । अ॒ग्निम् । चि॒नु॒ते ॥  \newline


\textbf{Krama Paata} \newline

तेन॑ लो॒कम् । लो॒कꣳ स्पृ॑णोति । स्पृ॒णो॒ति॒ यत् । यद् द्वि॒तीये᳚ । द्वि॒तीये॑ सम्ॅवथ्स॒रे । स॒म्ॅव॒थ्स॒रे᳚ऽग्निम् । स॒म्ॅव॒थ्स॒र इति॑ सम् - व॒थ्स॒रे । अ॒ग्निम् चि॑नु॒ते । चि॒नु॒ते᳚ऽन्तरि॑क्षम् । अ॒न्तरि॑क्षमे॒व । ए॒व तेन॑ । तेन॑ स्पृणोति । स्पृ॒णो॒ति॒ यत् । यत् तृ॒तीये᳚ । तृ॒तीये॑ सम्ॅवथ्स॒रे । स॒म्ॅव॒थ्स॒रे यज॑ते । स॒म्ॅव॒थ्स॒र इति॑ सम् - व॒थ्स॒रे । यज॑ते॒ऽमुम् । अ॒मुमे॒व । 
ए॒व तेन॑ । तेन॑ लो॒कम् । लो॒कꣳ स्पृ॑णोति । स्पृ॒णो॒त्ये॒तम् । 
ए॒तम् ॅवै । वै परः॑ । पर॑ आट्णा॒रः । आ॒ट्णा॒रः क॒क्षीवान्॑ । क॒क्षीवाꣳ॑ औशि॒जः । क॒क्षीवा॒निति॑ क॒क्षी - वा॒न्॒ । औ॒शि॒जो वी॒तह॑व्यः । वी॒तह॑व्यः श्राय॒सः । वी॒तह॑व्य॒ इति॑ वी॒त - ह॒व्यः॒ । श्रा॒य॒स स्त्र॒सद॑स्युः । त्र॒सद॑स्युः पौरुकु॒थ्स्यः । पौ॒रु॒कु॒थ्स्यः प्र॒जाका॑माः । पौ॒रु॒कु॒थ्स्य इति॑ पौरु - कु॒थ्स्यः । प्र॒जाका॑मा अचिन्वत । प्र॒जाका॑मा॒ इति॑ प्र॒जा - का॒माः॒ । अ॒चि॒न्व॒त॒ ततः॑ । ततो॒ वै । वै ते । ते स॒हस्रꣳ॑सहस्रम् । स॒हस्रꣳ॑सहस्रम् पु॒त्रान् । स॒हस्रꣳ॑सहस्र॒मिति॑ स॒हस्र᳚म् - स॒ह॒स्र॒म् । पु॒त्रान॑विन्दन्त । अ॒वि॒न्द॒न्त॒ प्रथ॑ते । प्रथ॑ते प्र॒जया᳚ । प्र॒जया॑ प॒शुभिः॑ । प्र॒जयेति॑ प्र - जया᳚ । प॒शुभि॒स्ताम् । प॒शुभि॒रिति॑ प॒शु - भिः॒ । ताम् मात्रा᳚म् । मात्रा॑माप्नोति । आ॒प्नो॒ति॒ याम् । याम् ते । तेऽग॑च्छन्न् । अग॑च्छ॒न्.॒ यः । य ए॒वम् ( ) । ए॒वम् ॅवि॒द्वान् । वि॒द्वाने॒तम् । ए॒तम॒ग्निम् । अ॒ग्निम् चि॑नु॒ते । चि॒नु॒त इति॑ चिनु॒ते । \newline

\textbf{Jatai Paata} \newline

1. तेन॑ लो॒कम् ॅलो॒कम् तेन॒ तेन॑ लो॒कम् । \newline
2. लो॒कꣳ स्पृ॑णोति स्पृणोति लो॒कम् ॅलो॒कꣳ स्पृ॑णोति । \newline
3. स्पृ॒णो॒ति॒ यद् यथ् स्पृ॑णोति स्पृणोति॒ यत् । \newline
4. यद् द्वि॒तीये᳚ द्वि॒तीये॒ यद् यद् द्वि॒तीये᳚ । \newline
5. द्वि॒तीये॑ संॅवथ्स॒रे सं॑ॅवथ्स॒रे द्वि॒तीये᳚ द्वि॒तीये॑ संॅवथ्स॒रे । \newline
6. सं॒ॅव॒थ्स॒रे᳚ ऽग्नि म॒ग्निꣳ सं॑ॅवथ्स॒रे सं॑ॅवथ्स॒रे᳚ ऽग्निम् । \newline
7. सं॒ॅव॒थ्स॒र इति॑ सं - व॒थ्स॒रे । \newline
8. अ॒ग्निम् चि॑नु॒ते चि॑नु॒ते᳚ ऽग्नि म॒ग्निम् चि॑नु॒ते । \newline
9. चि॒नु॒ते᳚ ऽन्तरि॑क्ष म॒न्तरि॑क्षम् चिनु॒ते चि॑नु॒ते᳚ ऽन्तरि॑क्षम् । \newline
10. अ॒न्तरि॑क्ष मे॒वैवा न्तरि॑क्ष म॒न्तरि॑क्ष मे॒व । \newline
11. ए॒व तेन॒ तेनै॒ वैव तेन॑ । \newline
12. तेन॑ स्पृणोति स्पृणोति॒ तेन॒ तेन॑ स्पृणोति । \newline
13. स्पृ॒णो॒ति॒ यद् यथ् स्पृ॑णोति स्पृणोति॒ यत् । \newline
14. यत् तृ॒तीये॑ तृ॒तीये॒ यद् यत् तृ॒तीये᳚ । \newline
15. तृ॒तीये॑ संॅवथ्स॒रे सं॑ॅवथ्स॒रे तृ॒तीये॑ तृ॒तीये॑ संॅवथ्स॒रे । \newline
16. सं॒ॅव॒थ्स॒रे यज॑ते॒ यज॑ते संॅवथ्स॒रे सं॑ॅवथ्स॒रे यज॑ते । \newline
17. सं॒ॅव॒थ्स॒र इति॑ सं - व॒थ्स॒रे । \newline
18. यज॑ते॒ ऽमु म॒मुं ॅयज॑ते॒ यज॑ते॒ ऽमुम् । \newline
19. अ॒मु मे॒वै वामु म॒मु मे॒व । \newline
20. ए॒व तेन॒ तेनै॒ वैव तेन॑ । \newline
21. तेन॑ लो॒कम् ॅलो॒कम् तेन॒ तेन॑ लो॒कम् । \newline
22. लो॒कꣳ स्पृ॑णोति स्पृणोति लो॒कम् ॅलो॒कꣳ स्पृ॑णोति । \newline
23. स्पृ॒णो॒ त्ये॒त मे॒तꣳ स्पृ॑णोति स्पृणो त्ये॒तम् । \newline
24. ए॒तं ॅवै वा ए॒त मे॒तं ॅवै । \newline
25. वै परः॒ परो॒ वै वै परः॑ । \newline
26. पर॑ आट्णा॒र आ᳚ट्णा॒रः परः॒ पर॑ आट्णा॒रः । \newline
27. आ॒ट्णा॒रः क॒क्षीवा᳚न् क॒क्षीवाꣳ॑ आट्णा॒र आ᳚ट्णा॒रः क॒क्षीवान्॑ । \newline
28. क॒क्षीवाꣳ॑ औशि॒ज औ॑शि॒जः क॒क्षीवा᳚न् क॒क्षीवाꣳ॑ औशि॒जः । \newline
29. क॒क्षीवा॒निति॑ क॒क्षी - वा॒न् । \newline
30. औ॒शि॒जो वी॒तह॑व्यो वी॒तह॑व्य औशि॒ज औ॑शि॒जो वी॒तह॑व्यः । \newline
31. वी॒तह॑व्यः श्राय॒सः श्रा॑य॒सो वी॒तह॑व्यो वी॒तह॑व्यः श्राय॒सः । \newline
32. वी॒तह॑व्य॒ इति॑ वी॒त - ह॒व्यः॒ । \newline
33. श्रा॒य॒स स्त्र॒सद॑स्यु स्त्र॒सद॑स्युः श्राय॒सः श्रा॑य॒स स्त्र॒सद॑स्युः । \newline
34. त्र॒सद॑स्युः पौरुकु॒थ्स्यः पौ॑रुकु॒थ्स्य स्त्र॒सद॑स्यु स्त्र॒सद॑स्युः पौरुकु॒थ्स्यः । \newline
35. पौ॒रु॒कु॒थ्स्यः प्र॒जाका॑माः प्र॒जाका॑माः पौरुकु॒थ्स्यः पौ॑रुकु॒थ्स्यः प्र॒जाका॑माः । \newline
36. पौ॒रु॒कु॒थ्स्य इति॑ पौरु - कु॒थ्स्यः । \newline
37. प्र॒जाका॑मा अचिन्वता चिन्वत प्र॒जाका॑माः प्र॒जाका॑मा अचिन्वत । \newline
38. प्र॒जाका॑मा॒ इति॑ प्र॒जा - का॒माः॒ । \newline
39. अ॒चि॒न्व॒त॒ तत॒ स्ततो॑ ऽचिन्वता चिन्वत॒ ततः॑ । \newline
40. ततो॒ वै वै तत॒ स्ततो॒ वै । \newline
41. वै ते ते वै वै ते । \newline
42. ते स॒हस्रꣳ॑सहस्रꣳ स॒हस्रꣳ॑सहस्र॒म् ते ते स॒हस्रꣳ॑सहस्रम् । \newline
43. स॒हस्रꣳ॑सहस्रम् पु॒त्रान् पु॒त्रान् थ्स॒हस्रꣳ॑सहस्रꣳ स॒हस्रꣳ॑सहस्रम् पु॒त्रान् । \newline
44. स॒हस्रꣳ॑सहस्र॒मिति॑ स॒हस्रं᳚ - स॒ह॒स्र॒म् । \newline
45. पु॒त्रा न॑विन्दन्ता विन्दन्त पु॒त्रान् पु॒त्रा न॑विन्दन्त । \newline
46. अ॒वि॒न्द॒न्त॒ प्रथ॑ते॒ प्रथ॑ते ऽविन्दन्ता विन्दन्त॒ प्रथ॑ते । \newline
47. प्रथ॑ते प्र॒जया᳚ प्र॒जया॒ प्रथ॑ते॒ प्रथ॑ते प्र॒जया᳚ । \newline
48. प्र॒जया॑ प॒शुभिः॑ प॒शुभिः॑ प्र॒जया᳚ प्र॒जया॑ प॒शुभिः॑ । \newline
49. प्र॒जयेति॑ प्र - जया᳚ । \newline
50. प॒शुभि॒ स्ताम् ताम् प॒शुभिः॑ प॒शुभि॒ स्ताम् । \newline
51. प॒शुभि॒रिति॑ प॒शु - भिः॒ । \newline
52. ताम् मात्रा॒म् मात्रा॒म् ताम् ताम् मात्रा᳚म् । \newline
53. मात्रा॑ माप्नो त्याप्नोति॒ मात्रा॒म् मात्रा॑ माप्नोति । \newline
54. आ॒प्नो॒ति॒ यां ॅया मा᳚प्नो त्याप्नोति॒ याम् । \newline
55. याम् ते ते यां ॅयाम् ते । \newline
56. ते ऽग॑च्छ॒न् नग॑च्छ॒न् ते ते ऽग॑च्छन्न् । \newline
57. अग॑च्छ॒न्॒. यो यो ऽग॑च्छ॒न् नग॑च्छ॒न्॒. यः । \newline
58. य ए॒व मे॒वं ॅयो य ए॒वम् । \newline
59. ए॒वं ॅवि॒द्वान्. वि॒द्वा ने॒व मे॒वं ॅवि॒द्वान् । \newline
60. वि॒द्वा ने॒त मे॒तं ॅवि॒द्वान्. वि॒द्वा ने॒तम् । \newline
61. ए॒त म॒ग्नि म॒ग्नि मे॒त मे॒त म॒ग्निम् । \newline
62. अ॒ग्निम् चि॑नु॒ते चि॑नु॒ते᳚ ऽग्नि म॒ग्निम् चि॑नु॒ते । \newline
63. चि॒नु॒त इति॑ चिनु॒ते । \newline

\textbf{Ghana Paata } \newline

1. तेन॑ लो॒कम् ॅलो॒कम् तेन॒ तेन॑ लो॒कꣳ स्पृ॑णोति स्पृणोति लो॒कम् तेन॒ तेन॑ लो॒कꣳ स्पृ॑णोति । \newline
2. लो॒कꣳ स्पृ॑णोति स्पृणोति लो॒कम् ॅलो॒कꣳ स्पृ॑णोति॒ यद् यथ् स्पृ॑णोति लो॒कम् ॅलो॒कꣳ स्पृ॑णोति॒ यत् । \newline
3. स्पृ॒णो॒ति॒ यद् यथ् स्पृ॑णोति स्पृणोति॒ यद् द्वि॒तीये᳚ द्वि॒तीये॒ यथ् स्पृ॑णोति स्पृणोति॒ यद् द्वि॒तीये᳚ । \newline
4. यद् द्वि॒तीये᳚ द्वि॒तीये॒ यद् यद् द्वि॒तीये॑ संॅवथ्स॒रे सं॑ॅवथ्स॒रे द्वि॒तीये॒ यद् यद् द्वि॒तीये॑ संॅवथ्स॒रे । \newline
5. द्वि॒तीये॑ संॅवथ्स॒रे सं॑ॅवथ्स॒रे द्वि॒तीये᳚ द्वि॒तीये॑ संॅवथ्स॒रे᳚ ऽग्नि म॒ग्निꣳ सं॑ॅवथ्स॒रे द्वि॒तीये᳚ द्वि॒तीये॑ संॅवथ्स॒रे᳚ ऽग्निम् । \newline
6. सं॒ॅव॒थ्स॒रे᳚ ऽग्नि म॒ग्निꣳ सं॑ॅवथ्स॒रे सं॑ॅवथ्स॒रे᳚ ऽग्निम् चि॑नु॒ते चि॑नु॒ते᳚ ऽग्निꣳ सं॑ॅवथ्स॒रे सं॑ॅवथ्स॒रे᳚ ऽग्निम् चि॑नु॒ते । \newline
7. सं॒ॅव॒थ्स॒र इति॑ सं - व॒थ्स॒रे । \newline
8. अ॒ग्निम् चि॑नु॒ते चि॑नु॒ते᳚ ऽग्नि म॒ग्निम् चि॑नु॒ते᳚ ऽन्तरि॑क्ष म॒न्तरि॑क्षम् चिनु॒ते᳚ ऽग्नि म॒ग्निम् चि॑नु॒ते᳚ ऽन्तरि॑क्षम् । \newline
9. चि॒नु॒ते᳚ ऽन्तरि॑क्ष म॒न्तरि॑क्षम् चिनु॒ते चि॑नु॒ते᳚ ऽन्तरि॑क्ष मे॒वै वान्तरि॑क्षम् चिनु॒ते चि॑नु॒ते᳚ ऽन्तरि॑क्ष मे॒व । \newline
10. अ॒न्तरि॑क्ष मे॒वै वान्तरि॑क्ष म॒न्तरि॑क्ष मे॒व तेन॒ तेनै॒ वान्तरि॑क्ष म॒न्तरि॑क्ष मे॒व तेन॑ । \newline
11. ए॒व तेन॒ तेनै॒ वैव तेन॑ स्पृणोति स्पृणोति॒ तेनै॒ वैव तेन॑ स्पृणोति । \newline
12. तेन॑ स्पृणोति स्पृणोति॒ तेन॒ तेन॑ स्पृणोति॒ यद् यथ् स्पृ॑णोति॒ तेन॒ तेन॑ स्पृणोति॒ यत् । \newline
13. स्पृ॒णो॒ति॒ यद् यथ् स्पृ॑णोति स्पृणोति॒ यत् तृ॒तीये॑ तृ॒तीये॒ यथ् स्पृ॑णोति स्पृणोति॒ यत् तृ॒तीये᳚ । \newline
14. यत् तृ॒तीये॑ तृ॒तीये॒ यद् यत् तृ॒तीये॑ संॅवथ्स॒रे सं॑ॅवथ्स॒रे तृ॒तीये॒ यद् यत् तृ॒तीये॑ संॅवथ्स॒रे । \newline
15. तृ॒तीये॑ संॅवथ्स॒रे सं॑ॅवथ्स॒रे तृ॒तीये॑ तृ॒तीये॑ संॅवथ्स॒रे यज॑ते॒ यज॑ते संॅवथ्स॒रे तृ॒तीये॑ तृ॒तीये॑ संॅवथ्स॒रे यज॑ते । \newline
16. सं॒ॅव॒थ्स॒रे यज॑ते॒ यज॑ते संॅवथ्स॒रे सं॑ॅवथ्स॒रे यज॑ते॒ ऽमु म॒मुं ॅयज॑ते संॅवथ्स॒रे सं॑ॅवथ्स॒रे यज॑ते॒ ऽमुम् । \newline
17. सं॒ॅव॒थ्स॒र इति॑ सं - व॒थ्स॒रे । \newline
18. यज॑ते॒ ऽमु म॒मुं ॅयज॑ते॒ यज॑ते॒ ऽमु मे॒वै वामुं ॅयज॑ते॒ यज॑ते॒ ऽमु मे॒व । \newline
19. अ॒मु मे॒वै वामु म॒मु मे॒व तेन॒ तेनै॒ वामु म॒मु मे॒व तेन॑ । \newline
20. ए॒व तेन॒ तेनै॒ वैव तेन॑ लो॒कम् ॅलो॒कम् तेनै॒ वैव तेन॑ लो॒कम् । \newline
21. तेन॑ लो॒कम् ॅलो॒कम् तेन॒ तेन॑ लो॒कꣳ स्पृ॑णोति स्पृणोति लो॒कम् तेन॒ तेन॑ लो॒कꣳ स्पृ॑णोति । \newline
22. लो॒कꣳ स्पृ॑णोति स्पृणोति लो॒कम् ॅलो॒कꣳ स्पृ॑णो त्ये॒त मे॒तꣳ स्पृ॑णोति लो॒कम् ॅलो॒कꣳ स्पृ॑णो त्ये॒तम् । \newline
23. स्पृ॒णो॒ त्ये॒त मे॒तꣳ स्पृ॑णोति स्पृणो त्ये॒तं ॅवै वा ए॒तꣳ स्पृ॑णोति स्पृणो त्ये॒तं ॅवै । \newline
24. ए॒तं ॅवै वा ए॒त मे॒तं ॅवै परः॒ परो॒ वा ए॒त मे॒तं ॅवै परः॑ । \newline
25. वै परः॒ परो॒ वै वै पर॑ आट्णा॒र आ᳚ट्णा॒रः परो॒ वै वै पर॑ आट्णा॒रः । \newline
26. पर॑ आट्णा॒र आ᳚ट्णा॒रः परः॒ पर॑ आट्णा॒रः क॒क्षीवा᳚न् क॒क्षीवाꣳ॑ आट्णा॒रः परः॒ पर॑ आट्णा॒रः क॒क्षीवान्॑ । \newline
27. आ॒ट्णा॒रः क॒क्षीवा᳚न् क॒क्षीवाꣳ॑ आट्णा॒र आ᳚ट्णा॒रः क॒क्षीवाꣳ॑ औशि॒ज औ॑शि॒जः क॒क्षीवाꣳ॑ आट्णा॒र आ᳚ट्णा॒रः क॒क्षीवाꣳ॑ औशि॒जः । \newline
28. क॒क्षीवाꣳ॑ औशि॒ज औ॑शि॒जः क॒क्षीवा᳚न् क॒क्षीवाꣳ॑ औशि॒जो वी॒तह॑व्यो वी॒तह॑व्य औशि॒जः क॒क्षीवा᳚न् क॒क्षीवाꣳ॑ औशि॒जो वी॒तह॑व्यः । \newline
29. क॒क्षीवा॒निति॑ क॒क्षी - वा॒न् । \newline
30. औ॒शि॒जो वी॒तह॑व्यो वी॒तह॑व्य औशि॒ज औ॑शि॒जो वी॒तह॑व्यः श्राय॒सः श्रा॑य॒सो वी॒तह॑व्य औशि॒ज औ॑शि॒जो वी॒तह॑व्यः श्राय॒सः । \newline
31. वी॒तह॑व्यः श्राय॒सः श्रा॑य॒सो वी॒तह॑व्यो वी॒तह॑व्यः श्राय॒स स्त्र॒सद॑स्यु स्त्र॒सद॑स्युः श्राय॒सो वी॒तह॑व्यो वी॒तह॑व्यः श्राय॒स स्त्र॒सद॑स्युः । \newline
32. वी॒तह॑व्य॒ इति॑ वी॒त - ह॒व्यः॒ । \newline
33. श्रा॒य॒स स्त्र॒सद॑स्यु स्त्र॒सद॑स्युः श्राय॒सः श्रा॑य॒स स्त्र॒सद॑स्युः पौरुकु॒थ्स्यः पौ॑रुकु॒थ्स्य स्त्र॒सद॑स्युः श्राय॒सः श्रा॑य॒स स्त्र॒सद॑स्युः पौरुकु॒थ्स्यः । \newline
34. त्र॒सद॑स्युः पौरुकु॒थ्स्यः पौ॑रुकु॒थ्स्य स्त्र॒सद॑स्यु स्त्र॒सद॑स्युः पौरुकु॒थ्स्यः प्र॒जाका॑माः प्र॒जाका॑माः पौरुकु॒थ्स्य स्त्र॒सद॑स्यु स्त्र॒सद॑स्युः पौरुकु॒थ्स्यः प्र॒जाका॑माः । \newline
35. पौ॒रु॒कु॒थ्स्यः प्र॒जाका॑माः प्र॒जाका॑माः पौरुकु॒थ्स्यः पौ॑रुकु॒थ्स्यः प्र॒जाका॑मा अचिन्वता चिन्वत प्र॒जाका॑माः पौरुकु॒थ्स्यः पौ॑रुकु॒थ्स्यः प्र॒जाका॑मा अचिन्वत । \newline
36. पौ॒रु॒कु॒थ्स्य इति॑ पौरु - कु॒थ्स्यः । \newline
37. प्र॒जाका॑मा अचिन्वता चिन्वत प्र॒जाका॑माः प्र॒जाका॑मा अचिन्वत॒ तत॒ स्ततो॑ ऽचिन्वत प्र॒जाका॑माः प्र॒जाका॑मा अचिन्वत॒ ततः॑ । \newline
38. प्र॒जाका॑मा॒ इति॑ प्र॒जा - का॒माः॒ । \newline
39. अ॒चि॒न्व॒त॒ तत॒ स्ततो॑ ऽचिन्वता चिन्वत॒ ततो॒ वै वै ततो॑ ऽचिन्वता चिन्वत॒ ततो॒ वै । \newline
40. ततो॒ वै वै तत॒ स्ततो॒ वै ते ते वै तत॒ स्ततो॒ वै ते । \newline
41. वै ते ते वै वै ते स॒हस्रꣳ॑सहस्रꣳ स॒हस्रꣳ॑सहस्र॒म् ते वै वै ते स॒हस्रꣳ॑सहस्रम् । \newline
42. ते स॒हस्रꣳ॑सहस्रꣳ स॒हस्रꣳ॑सहस्र॒म् ते ते स॒हस्रꣳ॑सहस्रम् पु॒त्रान् पु॒त्रान् थ्स॒हस्रꣳ॑सहस्र॒म् ते ते स॒हस्रꣳ॑सहस्रम् पु॒त्रान् । \newline
43. स॒हस्रꣳ॑सहस्रम् पु॒त्रान् पु॒त्रान् थ्स॒हस्रꣳ॑सहस्रꣳ स॒हस्रꣳ॑सहस्रम् पु॒त्रा न॑विन्दन्ता विन्दन्त पु॒त्रान् थ्स॒हस्रꣳ॑सहस्रꣳ स॒हस्रꣳ॑सहस्रम् पु॒त्रा न॑विन्दन्त । \newline
44. स॒हस्रꣳ॑सहस्र॒मिति॑ स॒हस्रं᳚ - स॒ह॒स्र॒म् । \newline
45. पु॒त्रा न॑विन्दन्ता विन्दन्त पु॒त्रान् पु॒त्रा न॑विन्दन्त॒ प्रथ॑ते॒ प्रथ॑ते ऽविन्दन्त पु॒त्रान् पु॒त्रा न॑विन्दन्त॒ प्रथ॑ते । \newline
46. अ॒वि॒न्द॒न्त॒ प्रथ॑ते॒ प्रथ॑ते ऽविन्दन्ता विन्दन्त॒ प्रथ॑ते प्र॒जया᳚ प्र॒जया॒ प्रथ॑ते ऽविन्दन्ता विन्दन्त॒ प्रथ॑ते प्र॒जया᳚ । \newline
47. प्रथ॑ते प्र॒जया᳚ प्र॒जया॒ प्रथ॑ते॒ प्रथ॑ते प्र॒जया॑ प॒शुभिः॑ प॒शुभिः॑ प्र॒जया॒ प्रथ॑ते॒ प्रथ॑ते प्र॒जया॑ प॒शुभिः॑ । \newline
48. प्र॒जया॑ प॒शुभिः॑ प॒शुभिः॑ प्र॒जया᳚ प्र॒जया॑ प॒शुभि॒ स्ताम् ताम् प॒शुभिः॑ प्र॒जया᳚ प्र॒जया॑ प॒शुभि॒ स्ताम् । \newline
49. प्र॒जयेति॑ प्र - जया᳚ । \newline
50. प॒शुभि॒ स्ताम् ताम् प॒शुभिः॑ प॒शुभि॒ स्ताम् मात्रा॒म् मात्रा॒म् ताम् प॒शुभिः॑ प॒शुभि॒ स्ताम् मात्रा᳚म् । \newline
51. प॒शुभि॒रिति॑ प॒शु - भिः॒ । \newline
52. ताम् मात्रा॒म् मात्रा॒म् ताम् ताम् मात्रा॑ माप्नो त्याप्नोति॒ मात्रा॒म् ताम् ताम् मात्रा॑ माप्नोति । \newline
53. मात्रा॑ माप्नो त्याप्नोति॒ मात्रा॒म् मात्रा॑ माप्नोति॒ यां ॅया मा᳚प्नोति॒ मात्रा॒म् मात्रा॑ माप्नोति॒ याम् । \newline
54. आ॒प्नो॒ति॒ यां ॅया मा᳚प्नो त्याप्नोति॒ याम् ते ते या मा᳚प्नो त्याप्नोति॒ याम् ते । \newline
55. याम् ते ते यां ॅयाम् ते ऽग॑च्छ॒न् नग॑च्छ॒न् ते यां ॅयाम् ते ऽग॑च्छन्न् । \newline
56. ते ऽग॑च्छ॒न् नग॑च्छ॒न् ते ते ऽग॑च्छ॒न्॒. यो यो ऽग॑च्छ॒न् ते ते ऽग॑च्छ॒न्॒. यः । \newline
57. अग॑च्छ॒न्॒. यो यो ऽग॑च्छ॒न् नग॑च्छ॒न्॒. य ए॒व मे॒वं ॅयो ऽग॑च्छ॒न् नग॑च्छ॒न्॒. य ए॒वम् । \newline
58. य ए॒व मे॒वं ॅयो य ए॒वं ॅवि॒द्वान्. वि॒द्वा ने॒वं ॅयो य ए॒वं ॅवि॒द्वान् । \newline
59. ए॒वं ॅवि॒द्वान्. वि॒द्वा ने॒व मे॒वं ॅवि॒द्वा ने॒त मे॒तं ॅवि॒द्वा ने॒व मे॒वं ॅवि॒द्वा ने॒तम् । \newline
60. वि॒द्वा ने॒त मे॒तं ॅवि॒द्वान्. वि॒द्वा ने॒त म॒ग्नि म॒ग्नि मे॒तं ॅवि॒द्वान्. वि॒द्वा ने॒त म॒ग्निम् । \newline
61. ए॒त म॒ग्नि म॒ग्नि मे॒त मे॒त म॒ग्निम् चि॑नु॒ते चि॑नु॒ते᳚ ऽग्नि मे॒त मे॒त म॒ग्निम् चि॑नु॒ते । \newline
62. अ॒ग्निम् चि॑नु॒ते चि॑नु॒ते᳚ ऽग्नि म॒ग्निम् चि॑नु॒ते । \newline
63. चि॒नु॒त इति॑ चिनु॒ते । \newline
\pagebreak
\markright{ TS 5.6.6.1  \hfill https://www.vedavms.in \hfill}

\section{ TS 5.6.6.1 }

\textbf{TS 5.6.6.1 } \newline
\textbf{Samhita Paata} \newline

प्र॒जाप॑तिर॒ग्निम॑चिनुत॒ स क्षु॒रप॑विर्भू॒त्वाऽति॑ष्ठ॒त् तं दे॒वा बिभ्य॑तो॒ नोपा॑ऽऽ*य॒न् ते छन्दो॑भिरा॒त्मानं॑ छादयि॒त्वोपा॑ऽऽ*य॒न् तच्छन्द॑सां छन्द॒स्त्वं ब्रह्म॒ वै छन्दाꣳ॑सि॒ ब्रह्म॑ण ए॒तद्-रू॒पं ॅयत् कृ॑ष्णाजि॒नं कार्ष्णी॑ उपा॒नहा॒वुप॑ मुञ्चते॒ छन्दो॑भिरे॒वाऽऽ*त्मानं॑ छादयि॒त्वाऽग्निमुप॑ चरत्या॒त्मनोऽहिꣳ॑सायै देवनि॒धिर्वा ए॒ष नि धी॑यते॒ यद॒ग्नि - [  ] \newline

\textbf{Pada Paata} \newline

प्र॒जाप॑ति॒रिति॑ प्र॒जा - प॒तिः॒ । अ॒ग्निम् । अ॒चि॒नु॒त॒ । सः । क्षु॒रप॑वि॒रिति॑ क्षु॒र - प॒विः॒ । भू॒त्वा । अ॒ति॒ष्ठ॒त् । तम् । दे॒वाः । बिभ्य॑तः । न । उपेति॑ । आ॒य॒न्न् । ते । छन्दो॑भि॒रिति॒ छन्दः॑ - भिः॒ । आ॒त्मान᳚म् । छा॒द॒यि॒त्वा । उपेति॑ । आ॒य॒न्न् । तत् । छन्द॑साम् । छ॒न्द॒स्त्वमिति॑ छन्दः - त्वम् । ब्रह्म॑ । वै । छन्दाꣳ॑सि । ब्रह्म॑णः । ए॒तत् । रू॒पम् । यत् । कृ॒ष्णा॒जि॒नमिति॑ कृष्ण - अ॒जि॒नम् । कार्ष्णी॒ इति॑ । उ॒पा॒नहौ᳚ । उपेति॑ । मु॒ञ्च॒ते॒ । छन्दो॑भि॒रिति॒ छन्दः॑ - भिः॒ । ए॒व । आ॒त्मान᳚म् । छा॒द॒यि॒त्वा । अ॒ग्निम् । उपेति॑ । च॒र॒ति॒ । आ॒त्मनः॑ । अहिꣳ॑सायै । दे॒व॒नि॒धिरिति॑ देव - नि॒धिः । वै । ए॒षः । नीति॑ । धी॒य॒ते॒ । यत् । अ॒ग्निः ।  \newline


\textbf{Krama Paata} \newline

प्र॒जाप॑तिर॒ग्निम् । प्र॒जाप॑ति॒रिति॑ प्र॒जा - प॒तिः॒ । अ॒ग्निम॑चिनुत । अ॒चि॒नु॒त॒ सः । स क्षु॒रप॑विः । क्षु॒रप॑विर् भू॒त्वा । क्षु॒रप॑वि॒रिति॑ क्षु॒र - प॒विः॒ । भू॒त्वाऽति॑ष्ठत् । अ॒ति॒ष्ठ॒त् तम् । तम् दे॒वाः । दे॒वा बिभ्य॑तः । बिभ्य॑तो॒ न । नोप॑ । उपा॑यन्न् । आ॒य॒न् ते । ते छन्दो॑भिः । छनो॑भिरा॒त्मान᳚म् । छन्दो॑भि॒रिति॒ छन्दः॑ - भिः॒ । आ॒त्मान॑म् छादयि॒त्वा । छा॒द॒यि॒त्वोप॑ । उपा॑यन्न् । आ॒य॒न् तत् । तच् छन्द॑साम् । छन्द॑साम् छन्द॒स्त्वम् । छ॒न्द॒स्त्वम् ब्रह्म॑ । छ॒न्द॒स्त्वमिति॑ छन्दः - त्वम् । ब्रह्म॒ वै । वै छन्दाꣳ॑सि । छन्दाꣳ॑सि॒ ब्रह्म॑णः । ब्रह्म॑ण ए॒तत् । ए॒तद् रू॒पम् । रू॒पम् ॅयत् । यत् कृ॑ष्णाजि॒नम् । कृ॒ष्णा॒जि॒नम् कार्ष्णी᳚ । कृ॒ष्णा॒जि॒नमिति॑ कृष्ण - अ॒जि॒नम् । कार्ष्णी॑ उपा॒नहौ᳚ । कार्ष्णी॒ इति॒ कार्ष्णी᳚ । उ॒पा॒नहा॒वुप॑ । उप॑ मुञ्चते । मु॒ञ्च॒ते॒ छन्दो॑भिः । छन्दो॑भिरे॒व । छन्दो॑भि॒रिति॒ छन्दः॑ - भिः॒ । ए॒वात्मान᳚म् । आ॒त्मान॑म् छादयि॒त्वा । छा॒द॒यि॒त्वाऽग्निम् । अ॒ग्निमुप॑ । उप॑ चरति । च॒र॒त्या॒त्मनः॑ । आ॒त्मनोऽहिꣳ॑सायै । अहिꣳ॑सायै देवनि॒धिः । दे॒व॒नि॒धिर् वै । दे॒व॒नि॒धिरिति॑ देव - नि॒धिः । वा ए॒षः । ए॒ष नि । नि धी॑यते । धी॒य॒ते॒ यत् । यद॒ग्निः । अ॒ग्निर॒न्ये \newline

\textbf{Jatai Paata} \newline

1. प्र॒जाप॑ति र॒ग्नि म॒ग्निम् प्र॒जाप॑तिः प्र॒जाप॑ति र॒ग्निम् । \newline
2. प्र॒जाप॑ति॒रिति॑ प्र॒जा - प॒तिः॒ । \newline
3. अ॒ग्नि म॑चिनुता चिनुता॒ग्नि म॒ग्नि म॑चिनुत । \newline
4. अ॒चि॒नु॒त॒ स सो॑ ऽचिनुता चिनुत॒ सः । \newline
5. स क्षु॒रप॑विः क्षु॒रप॑विः॒ स स क्षु॒रप॑विः । \newline
6. क्षु॒रप॑विर् भू॒त्वा भू॒त्वा क्षु॒रप॑विः क्षु॒रप॑विर् भू॒त्वा । \newline
7. क्षु॒रप॑वि॒रिति॑ क्षु॒र - प॒विः॒ । \newline
8. भू॒त्वा ऽति॑ष्ठ दतिष्ठद् भू॒त्वा भू॒त्वा ऽति॑ष्ठत् । \newline
9. अ॒ति॒ष्ठ॒त् तम् त म॑तिष्ठ दतिष्ठ॒त् तम् । \newline
10. तम् दे॒वा दे॒वा स्तम् तम् दे॒वाः । \newline
11. दे॒वा बिभ्य॑तो॒ बिभ्य॑तो दे॒वा दे॒वा बिभ्य॑तः । \newline
12. बिभ्य॑तो॒ न न बिभ्य॑तो॒ बिभ्य॑तो॒ न । \newline
13. नोपोप॒ न नोप॑ । \newline
14. उपा॑यन् नाय॒न् नुपोपा॑यन्न् । \newline
15. आ॒य॒न् ते त आ॑यन् नाय॒न् ते । \newline
16. ते छन्दो॑भि॒ श्छन्दो॑भि॒ स्ते ते छन्दो॑भिः । \newline
17. छन्दो॑भि रा॒त्मान॑ मा॒त्मान॒म् छन्दो॑भि॒ श्छन्दो॑भि रा॒त्मान᳚म् । \newline
18. छन्दो॑भि॒रिति॒ छन्दः॑ - भिः॒ । \newline
19. आ॒त्मान॑म् छादयि॒त्वा छा॑दयि॒त्वा ऽऽत्मान॑ मा॒त्मान॑म् छादयि॒त्वा । \newline
20. छा॒द॒यि॒त्वोपोप॑ च्छादयि॒त्वा छा॑दयि॒त्वोप॑ । \newline
21. उपा॑यन् नाय॒न् नुपोपा॑यन्न् । \newline
22. आ॒य॒न् तत् तदा॑यन् नाय॒न् तत् । \newline
23. तच् छन्द॑सा॒म् छन्द॑सा॒म् तत् तच् छन्द॑साम् । \newline
24. छन्द॑साम् छन्द॒स्त्वम् छ॑न्द॒स्त्वम् छन्द॑सा॒म् छन्द॑साम् छन्द॒स्त्वम् । \newline
25. छ॒न्द॒स्त्वम् ब्रह्म॒ ब्रह्म॑ छन्द॒स्त्वम् छ॑न्द॒स्त्वम् ब्रह्म॑ । \newline
26. छ॒न्द॒स्त्वमिति॑ छन्दः - त्वम् । \newline
27. ब्रह्म॒ वै वै ब्रह्म॒ ब्रह्म॒ वै । \newline
28. वै छन्दाꣳ॑सि॒ छन्दाꣳ॑सि॒ वै वै छन्दाꣳ॑सि । \newline
29. छन्दाꣳ॑सि॒ ब्रह्म॑णो॒ ब्रह्म॑ण॒ श्छन्दाꣳ॑सि॒ छन्दाꣳ॑सि॒ ब्रह्म॑णः । \newline
30. ब्रह्म॑ण ए॒त दे॒तद् ब्रह्म॑णो॒ ब्रह्म॑ण ए॒तत् । \newline
31. ए॒तद् रू॒पꣳ रू॒प मे॒त दे॒तद् रू॒पम् । \newline
32. रू॒पं ॅयद् यद् रू॒पꣳ रू॒पं ॅयत् । \newline
33. यत् कृ॑ष्णाजि॒नम् कृ॑ष्णाजि॒नं ॅयद् यत् कृ॑ष्णाजि॒नम् । \newline
34. कृ॒ष्णा॒जि॒नम् कार्ष्णी॒ कार्ष्णी॑ कृष्णाजि॒नम् कृ॑ष्णाजि॒नम् कार्ष्णी᳚ । \newline
35. कृ॒ष्णा॒जि॒नमिति॑ कृष्ण - अ॒जि॒नम् । \newline
36. कार्ष्णी॑ उपा॒नहा॑ वुपा॒नहौ॒ कार्ष्णी॒ कार्ष्णी॑ उपा॒नहौ᳚ । \newline
37. कार्ष्णी॒ इति॒ कार्ष्णी᳚ । \newline
38. उ॒पा॒नहा॒ वुपोपो॑ पा॒नहा॑ वुपा॒नहा॒ वुप॑ । \newline
39. उप॑ मुञ्चते मुञ्चत॒ उपोप॑ मुञ्चते । \newline
40. मु॒ञ्च॒ते॒ छन्दो॑भि॒ श्छन्दो॑भिर् मुञ्चते मुञ्चते॒ छन्दो॑भिः । \newline
41. छन्दो॑भि रे॒वैव छन्दो॑भि॒ श्छन्दो॑भि रे॒व । \newline
42. छन्दो॑भि॒रिति॒ छन्दः॑ - भिः॒ । \newline
43. ए॒वात्मान॑ मा॒त्मान॑ मे॒वै वात्मान᳚म् । \newline
44. आ॒त्मान॑म् छादयि॒त्वा छा॑दयि॒त्वा ऽऽत्मान॑ मा॒त्मान॑म् छादयि॒त्वा । \newline
45. छा॒द॒यि॒त्वा ऽग्नि म॒ग्निम् छा॑दयि॒त्वा छा॑दयि॒त्वा ऽग्निम् । \newline
46. अ॒ग्नि मुपोपा॒ग्नि म॒ग्नि मुप॑ । \newline
47. उप॑ चरति चर॒ त्युपोप॑ चरति । \newline
48. च॒र॒ त्या॒त्मन॑ आ॒त्मन॑ श्चरति चर त्या॒त्मनः॑ । \newline
49. आ॒त्मनो ऽहिꣳ॑साया॒ अहिꣳ॑साया आ॒त्मन॑ आ॒त्मनो ऽहिꣳ॑सायै । \newline
50. अहिꣳ॑सायै देवनि॒धिर् दे॑वनि॒धि रहिꣳ॑साया॒ अहिꣳ॑सायै देवनि॒धिः । \newline
51. दे॒व॒नि॒धिर् वै वै दे॑वनि॒धिर् दे॑वनि॒धिर् वै । \newline
52. दे॒व॒नि॒धिरिति॑ देव - नि॒धिः । \newline
53. वा ए॒ष ए॒ष वै वा ए॒षः । \newline
54. ए॒ष नि न्ये॑ष ए॒ष नि । \newline
55. नि धी॑यते धीयते॒ नि नि धी॑यते । \newline
56. धी॒य॒ते॒ यद् यद् धी॑यते धीयते॒ यत् । \newline
57. यद॒ग्नि र॒ग्निर् यद् यद॒ग्निः । \newline
58. अ॒ग्नि र॒न्ये᳚(1॒) ऽन्ये᳚ ऽग्नि र॒ग्नि र॒न्ये । \newline

\textbf{Ghana Paata } \newline

1. प्र॒जाप॑ति र॒ग्नि म॒ग्निम् प्र॒जाप॑तिः प्र॒जाप॑ति र॒ग्नि म॑चिनुता चिनुता॒ग्निम् प्र॒जाप॑तिः प्र॒जाप॑ति र॒ग्नि म॑चिनुत । \newline
2. प्र॒जाप॑ति॒रिति॑ प्र॒जा - प॒तिः॒ । \newline
3. अ॒ग्नि म॑चिनुता चिनुता॒ग्नि म॒ग्नि म॑चिनुत॒ स सो॑ ऽचिनुता॒ग्नि म॒ग्नि म॑चिनुत॒ सः । \newline
4. अ॒चि॒नु॒त॒ स सो॑ ऽचिनुता चिनुत॒ स क्षु॒रप॑विः क्षु॒रप॑विः॒ सो॑ ऽचिनुता चिनुत॒ स क्षु॒रप॑विः । \newline
5. स क्षु॒रप॑विः क्षु॒रप॑विः॒ स स क्षु॒रप॑विर् भू॒त्वा भू॒त्वा क्षु॒रप॑विः॒ स स क्षु॒रप॑विर् भू॒त्वा । \newline
6. क्षु॒रप॑विर् भू॒त्वा भू॒त्वा क्षु॒रप॑विः क्षु॒रप॑विर् भू॒त्वा ऽति॑ष्ठ दतिष्ठद् भू॒त्वा क्षु॒रप॑विः क्षु॒रप॑विर् भू॒त्वा ऽति॑ष्ठत् । \newline
7. क्षु॒रप॑वि॒रिति॑ क्षु॒र - प॒विः॒ । \newline
8. भू॒त्वा ऽति॑ष्ठ दतिष्ठद् भू॒त्वा भू॒त्वा ऽति॑ष्ठ॒त् तम् त म॑तिष्ठद् भू॒त्वा भू॒त्वा ऽति॑ष्ठ॒त् तम् । \newline
9. अ॒ति॒ष्ठ॒त् तम् त म॑तिष्ठ दतिष्ठ॒त् तम् दे॒वा दे॒वा स्त म॑तिष्ठ दतिष्ठ॒त् तम् दे॒वाः । \newline
10. तम् दे॒वा दे॒वा स्तम् तम् दे॒वा बिभ्य॑तो॒ बिभ्य॑तो दे॒वा स्तम् तम् दे॒वा बिभ्य॑तः । \newline
11. दे॒वा बिभ्य॑तो॒ बिभ्य॑तो दे॒वा दे॒वा बिभ्य॑तो॒ न न बिभ्य॑तो दे॒वा दे॒वा बिभ्य॑तो॒ न । \newline
12. बिभ्य॑तो॒ न न बिभ्य॑तो॒ बिभ्य॑तो॒ नोपोप॒ न बिभ्य॑तो॒ बिभ्य॑तो॒ नोप॑ । \newline
13. नोपोप॒ न नोपा॑यन् नाय॒न् नुप॒ न नोपा॑यन्न् । \newline
14. उपा॑यन् नाय॒न् नुपोपा॑य॒न् ते त आ॑य॒न् नुपोपा॑य॒न् ते । \newline
15. आ॒य॒न् ते त आ॑यन् नाय॒न् ते छन्दो॑भि॒ श्छन्दो॑भि॒ स्त आ॑यन् नाय॒न् ते छन्दो॑भिः । \newline
16. ते छन्दो॑भि॒ श्छन्दो॑भि॒ स्ते ते छन्दो॑भि रा॒त्मान॑ मा॒त्मान॒म् छन्दो॑भि॒ स्ते ते छन्दो॑भि रा॒त्मान᳚म् । \newline
17. छन्दो॑भि रा॒त्मान॑ मा॒त्मान॒म् छन्दो॑भि॒ श्छन्दो॑भि रा॒त्मान॑म् छादयि॒त्वा छा॑दयि॒त्वा ऽऽत्मान॒म् छन्दो॑भि॒ श्छन्दो॑भि रा॒त्मान॑म् छादयि॒त्वा । \newline
18. छन्दो॑भि॒रिति॒ छन्दः॑ - भिः॒ । \newline
19. आ॒त्मान॑म् छादयि॒त्वा छा॑दयि॒त्वा ऽऽत्मान॑ मा॒त्मान॑म् छादयि॒त्वोपोप॑ च्छादयि॒त्वा ऽऽत्मान॑ मा॒त्मान॑म् छादयि॒त्वोप॑ । \newline
20. छा॒द॒यि॒ त्वोपोप॑ च्छादयि॒त्वा छा॑दयि॒त्वोपा॑यन् नाय॒न् नुप॑ च्छादयि॒त्वा छा॑दयि॒ त्वोपा॑यन्न् । \newline
21. उपा॑यन् नाय॒न् नुपोपा॑य॒न् तत् तदा॑य॒न् नुपोपा॑य॒न् तत् । \newline
22. आ॒य॒न् तत् तदा॑यन् नाय॒न् तच् छन्द॑सा॒म् छन्द॑सा॒म् तदा॑यन् नाय॒न् तच् छन्द॑साम् । \newline
23. तच् छन्द॑सा॒म् छन्द॑सा॒म् तत् तच् छन्द॑साम् छन्द॒स्त्वम् छ॑न्द॒स्त्वम् छन्द॑सा॒म् तत् तच् छन्द॑साम् छन्द॒स्त्वम् । \newline
24. छन्द॑साम् छन्द॒स्त्वम् छ॑न्द॒स्त्वम् छन्द॑सा॒म् छन्द॑साम् छन्द॒स्त्वम् ब्रह्म॒ ब्रह्म॑ छन्द॒स्त्वम् छन्द॑सा॒म् छन्द॑साम् छन्द॒स्त्वम् ब्रह्म॑ । \newline
25. छ॒न्द॒स्त्वम् ब्रह्म॒ ब्रह्म॑ छन्द॒स्त्वम् छ॑न्द॒स्त्वम् ब्रह्म॒ वै वै ब्रह्म॑ छन्द॒स्त्वम् छ॑न्द॒स्त्वम् ब्रह्म॒ वै । \newline
26. छ॒न्द॒स्त्वमिति॑ छन्दः - त्वम् । \newline
27. ब्रह्म॒ वै वै ब्रह्म॒ ब्रह्म॒ वै छन्दाꣳ॑सि॒ छन्दाꣳ॑सि॒ वै ब्रह्म॒ ब्रह्म॒ वै छन्दाꣳ॑सि । \newline
28. वै छन्दाꣳ॑सि॒ छन्दाꣳ॑सि॒ वै वै छन्दाꣳ॑सि॒ ब्रह्म॑णो॒ ब्रह्म॑ण॒ श्छन्दाꣳ॑सि॒ वै वै छन्दाꣳ॑सि॒ ब्रह्म॑णः । \newline
29. छन्दाꣳ॑सि॒ ब्रह्म॑णो॒ ब्रह्म॑ण॒ श्छन्दाꣳ॑सि॒ छन्दाꣳ॑सि॒ ब्रह्म॑ण ए॒त दे॒तद् ब्रह्म॑ण॒ श्छन्दाꣳ॑सि॒ छन्दाꣳ॑सि॒ ब्रह्म॑ण ए॒तत् । \newline
30. ब्रह्म॑ण ए॒त दे॒तद् ब्रह्म॑णो॒ ब्रह्म॑ण ए॒तद् रू॒पꣳ रू॒प मे॒तद् ब्रह्म॑णो॒ ब्रह्म॑ण ए॒तद् रू॒पम् । \newline
31. ए॒तद् रू॒पꣳ रू॒प मे॒त दे॒तद् रू॒पं ॅयद् यद् रू॒प मे॒त दे॒तद् रू॒पं ॅयत् । \newline
32. रू॒पं ॅयद् यद् रू॒पꣳ रू॒पं ॅयत् कृ॑ष्णाजि॒नम् कृ॑ष्णाजि॒नं ॅयद् रू॒पꣳ रू॒पं ॅयत् कृ॑ष्णाजि॒नम् । \newline
33. यत् कृ॑ष्णाजि॒नम् कृ॑ष्णाजि॒नं ॅयद् यत् कृ॑ष्णाजि॒नम् कार्ष्णी॒ कार्ष्णी॑ कृष्णाजि॒नं ॅयद् यत् कृ॑ष्णाजि॒नम् कार्ष्णी᳚ । \newline
34. कृ॒ष्णा॒जि॒नम् कार्ष्णी॒ कार्ष्णी॑ कृष्णाजि॒नम् कृ॑ष्णाजि॒नम् कार्ष्णी॑ उपा॒नहा॑ वुपा॒नहौ॒ कार्ष्णी॑ कृष्णाजि॒नम् कृ॑ष्णाजि॒नम् कार्ष्णी॑ उपा॒नहौ᳚ । \newline
35. कृ॒ष्णा॒जि॒नमिति॑ कृष्ण - अ॒जि॒नम् । \newline
36. कार्ष्णी॑ उपा॒नहा॑ वुपा॒नहौ॒ कार्ष्णी॒ कार्ष्णी॑ उपा॒नहा॒ वुपो पो॑पा॒नहौ॒ कार्ष्णी॒ कार्ष्णी॑ उपा॒नहा॒ वुप॑ । \newline
37. कार्ष्णी॒ इति॒ कार्ष्णी᳚ । \newline
38. उ॒पा॒नहा॒ वुपो पो॑पा॒नहा॑ वुपा॒नहा॒ वुप॑ मुञ्चते मुञ्चत॒ उपो॑पा॒नहा॑ वुपा॒नहा॒ वुप॑ मुञ्चते । \newline
39. उप॑ मुञ्चते मुञ्चत॒ उपोप॑ मुञ्चते॒ छन्दो॑भि॒ श्छन्दो॑भिर् मुञ्चत॒ उपोप॑ मुञ्चते॒ छन्दो॑भिः । \newline
40. मु॒ञ्च॒ते॒ छन्दो॑भि॒ श्छन्दो॑भिर् मुञ्चते मुञ्चते॒ छन्दो॑भि रे॒वैव छन्दो॑भिर् मुञ्चते मुञ्चते॒ छन्दो॑भि रे॒व । \newline
41. छन्दो॑भि रे॒वैव छन्दो॑भि॒ श्छन्दो॑भि रे॒वात्मान॑ मा॒त्मान॑ मे॒व छन्दो॑भि॒ श्छन्दो॑भि रे॒वात्मान᳚म् । \newline
42. छन्दो॑भि॒रिति॒ छन्दः॑ - भिः॒ । \newline
43. ए॒वात्मान॑ मा॒त्मान॑ मे॒वै वात्मान॑म् छादयि॒त्वा छा॑दयि॒त्वा ऽऽत्मान॑ मे॒वै वात्मान॑म् छादयि॒त्वा । \newline
44. आ॒त्मान॑म् छादयि॒त्वा छा॑दयि॒त्वा ऽऽत्मान॑ मा॒त्मान॑म् छादयि॒त्वा ऽग्नि म॒ग्निम् छा॑दयि॒त्वा ऽऽत्मान॑ मा॒त्मान॑म् छादयि॒त्वा ऽग्निम् । \newline
45. छा॒द॒यि॒त्वा ऽग्नि म॒ग्निम् छा॑दयि॒त्वा छा॑दयि॒त्वा ऽग्नि मुपोपा॒ग्निम् छा॑दयि॒त्वा छा॑दयि॒त्वा ऽग्नि मुप॑ । \newline
46. अ॒ग्नि मुपोपा॒ग्नि म॒ग्नि मुप॑ चरति चर॒ त्युपा॒ग्नि म॒ग्नि मुप॑ चरति । \newline
47. उप॑ चरति चर॒ त्युपोप॑ चर त्या॒त्मन॑ आ॒त्मन॑ श्चर॒ त्युपोप॑ चर त्या॒त्मनः॑ । \newline
48. च॒र॒ त्या॒त्मन॑ आ॒त्मन॑ श्चरति चर त्या॒त्मनो ऽहिꣳ॑साया॒ अहिꣳ॑साया आ॒त्मन॑ श्चरति चर त्या॒त्मनो ऽहिꣳ॑सायै । \newline
49. आ॒त्मनो ऽहिꣳ॑साया॒ अहिꣳ॑साया आ॒त्मन॑ आ॒त्मनो ऽहिꣳ॑सायै देवनि॒धिर् दे॑वनि॒धि रहिꣳ॑साया आ॒त्मन॑ आ॒त्मनो ऽहिꣳ॑सायै देवनि॒धिः । \newline
50. अहिꣳ॑सायै देवनि॒धिर् दे॑वनि॒धि रहिꣳ॑साया॒ अहिꣳ॑सायै देवनि॒धिर् वै वै दे॑वनि॒धि रहिꣳ॑साया॒ अहिꣳ॑सायै देवनि॒धिर् वै । \newline
51. दे॒व॒नि॒धिर् वै वै दे॑वनि॒धिर् दे॑वनि॒धिर् वा ए॒ष ए॒ष वै दे॑वनि॒धिर् दे॑वनि॒धिर् वा ए॒षः । \newline
52. दे॒व॒नि॒धिरिति॑ देव - नि॒धिः । \newline
53. वा ए॒ष ए॒ष वै वा ए॒ष नि न्ये॑ष वै वा ए॒ष नि । \newline
54. ए॒ष नि न्ये॑ष ए॒ष नि धी॑यते धीयते॒ न्ये॑ष ए॒ष नि धी॑यते । \newline
55. नि धी॑यते धीयते॒ नि नि धी॑यते॒ यद् यद् धी॑यते॒ नि नि धी॑यते॒ यत् । \newline
56. धी॒य॒ते॒ यद् यद् धी॑यते धीयते॒ यद॒ग्नि र॒ग्निर् यद् धी॑यते धीयते॒ यद॒ग्निः । \newline
57. यद॒ग्नि र॒ग्निर् यद् यद॒ग्नि र॒न्ये᳚(1॒) ऽन्ये᳚ ऽग्निर् यद् यद॒ग्नि र॒न्ये । \newline
58. अ॒ग्नि र॒न्ये᳚(1॒) ऽन्ये᳚ ऽग्नि र॒ग्नि र॒न्ये वा॑ वा॒ ऽन्ये᳚ ऽग्नि र॒ग्नि र॒न्ये वा᳚ । \newline
\pagebreak
\markright{ TS 5.6.6.2  \hfill https://www.vedavms.in \hfill}

\section{ TS 5.6.6.2 }

\textbf{TS 5.6.6.2 } \newline
\textbf{Samhita Paata} \newline

-र॒न्ये वा॒ वै नि॒धिमगु॑प्तं ॅवि॒न्दन्ति॒ न वा॒ प्रति॒ प्र जा॑नात्यु॒खामा क्रा॑मत्या॒त्मान॑मे॒वाधि॒पां कु॑रुते॒ गुप्त्या॒ अथो॒ खल्वा॑हु॒र्नाऽऽक्रम्येति॑ नैर्.ऋ॒त्यु॑खा यदा॒क्रामे॒न्निर्.ऋ॑त्या आ॒त्मान॒मपि॑ दद्ध्या॒त् तस्मा॒न्नाऽऽक्रम्या॑ पुरुषशी॒र्॒.षमुप॑ दधाति॒ गुप्त्या॒ अथो॒ यथा᳚ ब्रू॒यादे॒तन्मे॑ गोपा॒येति॑ ता॒दृगे॒व तत् - [  ] \newline

\textbf{Pada Paata} \newline

अ॒न्ये । वा॒ । वै । नि॒धिमिति॑ नि - धिम् । अगु॑प्तम् । वि॒न्दन्ति॑ । न । वा॒ । प्रति॑ । प्रेति॑ । जा॒ना॒ति॒ । उ॒खाम् । एति॑ । क्रा॒म॒ति॒ । आ॒त्मान᳚म् । ए॒व । अ॒धि॒पामित्य॑धि - पाम् । कु॒रु॒ते॒ । गुप्त्यै᳚ । अथो॒ इति॑ । खलु॑ । आ॒हुः॒ । न । आ॒क्रम्येत्या᳚ - क्रम्या᳚ । इति॑ । नै॒र्.॒ऋ॒तीति॑ नैः-ऋ॒ती । उ॒खा । यत् । आ॒क्रामे॒दित्या᳚ - क्रामे᳚त् । निर्.ऋ॑त्या॒ इति॒ निः-ऋ॒त्यै॒ । आ॒त्मान᳚म् । अपीति॑ । द॒द्ध्या॒त् । तस्मा᳚त् । न । आ॒क्रम्येत्या᳚-क्रम्या᳚ । पु॒रु॒ष॒शी॒र्॒.षमिति॑ पुरुष - शी॒र्॒.षम् । उपेति॑ । द॒धा॒ति॒ । गुप्त्यै᳚ । अथो॒ इति॑ । यथा᳚ । ब्रू॒यात् । ए॒तत् । मे॒ । गो॒पा॒य॒ । इति॑ । ता॒दृक् । ए॒व । तत् ।  \newline


\textbf{Krama Paata} \newline

अ॒न्ये वा᳚ । वा॒ वै । वै नि॒धिम् । नि॒धिमगु॑प्तम् । नि॒धिमिति॑ नि - धिम् । अगु॑प्तम् ॅवि॒न्दन्ति॑ । वि॒न्दन्ति॒ न । न वा᳚ । वा॒ प्रति॑ । प्रति॒ प्र । प्र जा॑नाति । जा॒ना॒त्यु॒खाम् । उ॒खामा । आ क्रा॑मति । क्रा॒म॒त्या॒त्मान᳚म् । आ॒त्मान॑मे॒व । ए॒वाधि॒पाम् । अ॒धि॒पाम् कु॑रुते । अ॒धि॒पामित्य॑धि - पाम् । कु॒रु॒ते॒ गुप्त्यै᳚ । गुप्त्या॒ अथो᳚ । अथो॒ खलु॑ । अथो॒ इत्यथो᳚ । खल्वा॑हुः । आ॒हु॒र् न । नाक्रम्या᳚ । आ॒क्रम्येति॑ । आ॒क्रम्येत्या᳚ - क्रम्या᳚ । इति॑ नैर्.ऋ॒ती । नै॒र्॒.ऋ॒त्यु॑खा । नै॒र्॒.ऋ॒तीति॑ नैः - ऋ॒ती । उ॒खा यत् । यदा॒क्रामे᳚त् । आ॒क्रामे॒न् निर्.ऋ॑त्यै । आ॒क्रामे॒दित्या᳚ - क्रामे᳚त् । निर्.ऋ॑त्या आ॒त्मान᳚म् । निर्.ऋ॑त्या॒ इति॒ निः - ऋ॒त्यै॒ । आ॒त्मान॒मपि॑ । अपि॑ दद्ध्यात् । द॒द्ध्या॒त् तस्मा᳚त् । तस्मा॒न् न । नाक्रम्या᳚ । आ॒क्रम्या॑ पुरुषशी॒र्॒.षम् । आ॒क्रम्येत्या᳚ - क्रम्या᳚ । पु॒रु॒ष॒शी॒र्॒.षमुप॑ । पु॒रु॒ष॒शी॒र्॒.षमिति॑ पुरुष - शी॒र्॒.षम् । उप॑ दधाति । द॒धा॒ति॒ गुप्त्यै᳚ । गुप्त्या॒ अथो᳚ । अथो॒ यथा᳚ । अथो॒ इत्यथो᳚ । यथा᳚ ब्रू॒यात् । ब्रू॒यादे॒तत् । ए॒तन् मे᳚ । मे॒ गो॒पा॒य॒ । गो॒पा॒येति॑ । इति॑ ता॒दृक् । ता॒दृगे॒व । ए॒व तत् । तत् प्र॒जाप॑तिः \newline

\textbf{Jatai Paata} \newline

1. अ॒न्ये वा॑ वा॒ ऽन्ये᳚ ऽन्ये वा᳚ । \newline
2. वा॒ वै वै वा॑ वा॒ वै । \newline
3. वै नि॒धिन्नि॒धिं ॅवै वै नि॒धिम् । \newline
4. नि॒धि मगु॑प्त॒ मगु॑प्तन् नि॒धिन् नि॒धि मगु॑प्तम् । \newline
5. नि॒धिमिति॑ नि - धिम् । \newline
6. अगु॑प्तं ॅवि॒न्दन्ति॑ वि॒न्दन् त्यगु॑प्त॒ मगु॑प्तं ॅवि॒न्दन्ति॑ । \newline
7. वि॒न्दन्ति॒ न न वि॒न्दन्ति॑ वि॒न्दन्ति॒ न । \newline
8. न वा॑ वा॒ न न वा᳚ । \newline
9. वा॒ प्रति॒ प्रति॑ वा वा॒ प्रति॑ । \newline
10. प्रति॒ प्र प्र प्रति॒ प्रति॒ प्र । \newline
11. प्र जा॑नाति जानाति॒ प्र प्र जा॑नाति । \newline
12. जा॒ना॒ त्यु॒खा मु॒खाम् जा॑नाति जाना त्यु॒खाम् । \newline
13. उ॒खा मोखा मु॒खा मा । \newline
14. आ क्रा॑मति क्राम॒त्या क्रा॑मति । \newline
15. क्रा॒म॒ त्या॒त्मान॑ मा॒त्मान॑म् क्रामति क्राम त्या॒त्मान᳚म् । \newline
16. आ॒त्मान॑ मे॒वै वात्मान॑ मा॒त्मान॑ मे॒व । \newline
17. ए॒वाधि॒पा म॑धि॒पा मे॒वै वाधि॒पाम् । \newline
18. अ॒धि॒पाम् कु॑रुते कुरुते ऽधि॒पा म॑धि॒पाम् कु॑रुते । \newline
19. अ॒धि॒पामित्य॑धि - पाम् । \newline
20. कु॒रु॒ते॒ गुप्त्यै॒ गुप्त्यै॑ कुरुते कुरुते॒ गुप्त्यै᳚ । \newline
21. गुप्त्या॒ अथो॒ अथो॒ गुप्त्यै॒ गुप्त्या॒ अथो᳚ । \newline
22. अथो॒ खलु॒ खल्वथो॒ अथो॒ खलु॑ । \newline
23. अथो॒ इत्यथो᳚ । \newline
24. खल्वा॑हु राहुः॒ खलु॒ खल्वा॑हुः । \newline
25. आ॒हु॒र् न नाहु॑ राहु॒र् न । \newline
26. नाक्रम्या॒ ऽऽक्रम्या॒ न नाक्रम्या᳚ । \newline
27. आ॒क्रम्येती त्या॒क्रम्या॒ ऽऽक्रम्येति॑ । \newline
28. आ॒क्रम्येत्या᳚ - क्रम्या᳚ । \newline
29. इति॑ नैर्.ऋ॒ती नैर्॑.ऋ॒ती तीति॑ नैर्.ऋ॒ती । \newline
30. नै॒र्॒.ऋ॒ त्यु॑खोखा नैर्॑.ऋ॒ती नैर्॑.ऋ॒ त्यु॑खा । \newline
31. नै॒र्.॒ऋ॒तीति॑ नैः - ऋ॒ती । \newline
32. उ॒खा यद् यदु॒खोखा यत् । \newline
33. यदा॒क्रामे॑ दा॒क्रामे॒द् यद् यदा॒क्रामे᳚त् । \newline
34. आ॒क्रामे॒न् निर्.ऋ॑त्यै॒ निर्.ऋ॑त्या आ॒क्रामे॑ दा॒क्रामे॒न् निर्.ऋ॑त्यै । \newline
35. आ॒क्रामे॒दित्या᳚ - क्रामे᳚त् । \newline
36. निर्.ऋ॑त्या आ॒त्मान॑ मा॒त्मान॒न् निर्.ऋ॑त्यै॒ निर्.ऋ॑त्या आ॒त्मान᳚म् । \newline
37. निर्.ऋ॑त्या॒ इति॒ निः - ऋ॒त्यै॒ । \newline
38. आ॒त्मान॒ मप्य प्या॒त्मान॑ मा॒त्मान॒ मपि॑ । \newline
39. अपि॑ दद्ध्याद् दद्ध्या॒ दप्यपि॑ दद्ध्यात् । \newline
40. द॒द्ध्या॒त् तस्मा॒त् तस्मा᳚द् दद्ध्याद् दद्ध्या॒त् तस्मा᳚त् । \newline
41. तस्मा॒न् न न तस्मा॒त् तस्मा॒न् न । \newline
42. नाक्रम्या॒ ऽऽक्रम्या॒ न नाक्रम्या᳚ । \newline
43. आ॒क्रम्या॑ पुरुषशी॒र्॒.षम् पु॑रुषशी॒र्॒.ष मा॒क्रम्या॒ ऽऽक्रम्या॑ पुरुषशी॒र्॒.षम् । \newline
44. आ॒क्रम्येत्या᳚ - क्रम्या᳚ । \newline
45. पु॒रु॒ष॒शी॒र्॒.ष मुपोप॑ पुरुषशी॒र्॒.षम् पु॑रुषशी॒र्॒.ष मुप॑ । \newline
46. पु॒रु॒ष॒शी॒र्॒.षमिति॑ पुरुष - शी॒र्॒.षम् । \newline
47. उप॑ दधाति दधा॒ त्युपोप॑ दधाति । \newline
48. द॒धा॒ति॒ गुप्त्यै॒ गुप्त्यै॑ दधाति दधाति॒ गुप्त्यै᳚ । \newline
49. गुप्त्या॒ अथो॒ अथो॒ गुप्त्यै॒ गुप्त्या॒ अथो᳚ । \newline
50. अथो॒ यथा॒ यथा ऽथो॒ अथो॒ यथा᳚ । \newline
51. अथो॒ इत्यथो᳚ । \newline
52. यथा᳚ ब्रू॒याद् ब्रू॒याद् यथा॒ यथा᳚ ब्रू॒यात् । \newline
53. ब्रू॒या दे॒त दे॒तद् ब्रू॒याद् ब्रू॒या दे॒तत् । \newline
54. ए॒तन् मे॑ म ए॒त दे॒तन् मे᳚ । \newline
55. मे॒ गो॒पा॒य॒ गो॒पा॒य॒ मे॒ मे॒ गो॒पा॒य॒ । \newline
56. गो॒पा॒ये तीति॑ गोपाय गोपा॒येति॑ । \newline
57. इति॑ ता॒दृक् ता॒दृ गितीति॑ ता॒दृक् । \newline
58. ता॒दृ गे॒वैव ता॒दृक् ता॒दृ गे॒व । \newline
59. ए॒व तत् तदे॒ वैव तत् । \newline
60. तत् प्र॒जाप॑तिः प्र॒जाप॑ति॒ स्तत् तत् प्र॒जाप॑तिः । \newline

\textbf{Ghana Paata } \newline

1. अ॒न्ये वा॑ वा॒ ऽन्ये᳚ ऽन्ये वा॒ वै वै वा॒ ऽन्ये᳚ ऽन्ये वा॒ वै । \newline
2. वा॒ वै वै वा॑ वा॒ वै नि॒धिम् नि॒धिं ॅवै वा॑ वा॒ वै नि॒धिम् । \newline
3. वै नि॒धिम् नि॒धिं ॅवै वै नि॒धि मगु॑प्त॒ मगु॑प्तम् नि॒धिं ॅवै वै नि॒धि मगु॑प्तम् । \newline
4. नि॒धि मगु॑प्त॒ मगु॑प्तम् नि॒धिम् नि॒धि मगु॑प्तं ॅवि॒न्दन्ति॑ वि॒न्दन् त्यगु॑प्तम् नि॒धिम् नि॒धि मगु॑प्तं ॅवि॒न्दन्ति॑ । \newline
5. नि॒धिमिति॑ नि - धिम् । \newline
6. अगु॑प्तं ॅवि॒न्दन्ति॑ वि॒न्दन् त्यगु॑प्त॒ मगु॑प्तं ॅवि॒न्दन्ति॒ न न वि॒न्दन् त्यगु॑प्त॒ मगु॑प्तं ॅवि॒न्दन्ति॒ न । \newline
7. वि॒न्दन्ति॒ न न वि॒न्दन्ति॑ वि॒न्दन्ति॒ न वा॑ वा॒ न वि॒न्दन्ति॑ वि॒न्दन्ति॒ न वा᳚ । \newline
8. न वा॑ वा॒ न न वा॒ प्रति॒ प्रति॑ वा॒ न न वा॒ प्रति॑ । \newline
9. वा॒ प्रति॒ प्रति॑ वा वा॒ प्रति॒ प्र प्र प्रति॑ वा वा॒ प्रति॒ प्र । \newline
10. प्रति॒ प्र प्र प्रति॒ प्रति॒ प्र जा॑नाति जानाति॒ प्र प्रति॒ प्रति॒ प्र जा॑नाति । \newline
11. प्र जा॑नाति जानाति॒ प्र प्र जा॑ना त्यु॒खा मु॒खाम् जा॑नाति॒ प्र प्र जा॑ना त्यु॒खाम् । \newline
12. जा॒ना॒ त्यु॒खा मु॒खाम् जा॑नाति जाना त्यु॒खा मोखाम् जा॑नाति जाना त्यु॒खा मा । \newline
13. उ॒खा मोखा मु॒खा मा क्रा॑मति क्राम॒ त्योखा मु॒खा मा क्रा॑मति । \newline
14. आ क्रा॑मति क्राम॒ त्याक्रा॑म त्या॒त्मान॑ मा॒त्मान॑म् क्राम॒ त्याक्रा॑म त्या॒त्मान᳚म् । \newline
15. क्रा॒म॒ त्या॒त्मान॑ मा॒त्मान॑म् क्रामति क्राम त्या॒त्मान॑ मे॒वै वात्मान॑म् क्रामति क्राम त्या॒त्मान॑ मे॒व । \newline
16. आ॒त्मान॑ मे॒वै वात्मान॑ मा॒त्मान॑ मे॒वाधि॒पा म॑धि॒पा मे॒वात्मान॑ मा॒त्मान॑ मे॒वाधि॒पाम् । \newline
17. ए॒वाधि॒पा म॑धि॒पा मे॒वै वाधि॒पाम् कु॑रुते कुरुते ऽधि॒पा मे॒वै वाधि॒पाम् कु॑रुते । \newline
18. अ॒धि॒पाम् कु॑रुते कुरुते ऽधि॒पा म॑धि॒पाम् कु॑रुते॒ गुप्त्यै॒ गुप्त्यै॑ कुरुते ऽधि॒पा म॑धि॒पाम् कु॑रुते॒ गुप्त्यै᳚ । \newline
19. अ॒धि॒पामित्य॑धि - पाम् । \newline
20. कु॒रु॒ते॒ गुप्त्यै॒ गुप्त्यै॑ कुरुते कुरुते॒ गुप्त्या॒ अथो॒ अथो॒ गुप्त्यै॑ कुरुते कुरुते॒ गुप्त्या॒ अथो᳚ । \newline
21. गुप्त्या॒ अथो॒ अथो॒ गुप्त्यै॒ गुप्त्या॒ अथो॒ खलु॒ खल्वथो॒ गुप्त्यै॒ गुप्त्या॒ अथो॒ खलु॑ । \newline
22. अथो॒ खलु॒ खल्वथो॒ अथो॒ खल्वा॑हु राहुः॒ खल्वथो॒ अथो॒ खल्वा॑हुः । \newline
23. अथो॒ इत्यथो᳚ । \newline
24. खल्वा॑हु राहुः॒ खलु॒ खल्वा॑हु॒र् न नाहुः॒ खलु॒ खल्वा॑हु॒र् न । \newline
25. आ॒हु॒र् न नाहु॑ राहु॒र् नाक्रम्या॒ ऽऽक्रम्या॒ नाहु॑ राहु॒र् नाक्रम्या᳚ । \newline
26. नाक्रम्या॒ ऽऽक्रम्या॒ न नाक्र म्येती त्या॒क्रम्या॒ न नाक्रम्येति॑ । \newline
27. आ॒क्रम् येती त्या॒क्रम्या॒ ऽऽक्रम्येति॑ नैर्.ऋ॒ती नैर्॑.ऋ॒ती त्या॒क्रम्या॒ ऽऽक्रम्येति॑ नैर्.ऋ॒ती । \newline
28. आ॒क्रम्येत्या᳚ - क्रम्या᳚ । \newline
29. इति॑ नैर्.ऋ॒ती नैर्॑.ऋ॒ती तीति॑ नैर्.ऋ॒ त्यु॑खोखा नैर्॑.ऋ॒ती तीति॑ नैर्.ऋ॒ त्यु॑खा । \newline
30. नै॒र्॒.ऋ॒ त्यु॑खोखा नैर्॑.ऋ॒ती नैर्॑.ऋ॒ त्यु॑खा यद् यदु॒खा नैर्॑.ऋ॒ती नैर्॑.ऋ॒ त्यु॑खा यत् । \newline
31. नै॒र्.॒ऋ॒तीति॑ नैः - ऋ॒ती । \newline
32. उ॒खा यद् यदु॒खोखा यदा॒क्रामे॑ दा॒क्रामे॒द् यदु॒खोखा यदा॒क्रामे᳚त् । \newline
33. यदा॒क्रामे॑ दा॒क्रामे॒द् यद् यदा॒क्रामे॒न् निर्.ऋ॑त्यै॒ निर्.ऋ॑त्या आ॒क्रामे॒द् यद् यदा॒क्रामे॒न् निर्.ऋ॑त्यै । \newline
34. आ॒क्रामे॒न् निर्.ऋ॑त्यै॒ निर्.ऋ॑त्या आ॒क्रामे॑ दा॒क्रामे॒न् निर्.ऋ॑त्या आ॒त्मान॑ मा॒त्मान॒म् निर्.ऋ॑त्या आ॒क्रामे॑ दा॒क्रामे॒न् निर्.ऋ॑त्या आ॒त्मान᳚म् । \newline
35. आ॒क्रामे॒दित्या᳚ - क्रामे᳚त् । \newline
36. निर्.ऋ॑त्या आ॒त्मान॑ मा॒त्मान॒म् निर्.ऋ॑त्यै॒ निर्.ऋ॑त्या आ॒त्मान॒ मप्य प्या॒त्मान॒म् निर्.ऋ॑त्यै॒ निर्.ऋ॑त्या आ॒त्मान॒ मपि॑ । \newline
37. निर्.ऋ॑त्या॒ इति॒ निः - ऋ॒त्यै॒ । \newline
38. आ॒त्मान॒ मप्य प्या॒त्मान॑ मा॒त्मान॒ मपि॑ दद्ध्याद् दद्ध्या॒द प्या॒त्मान॑ मा॒त्मान॒ मपि॑ दद्ध्यात् । \newline
39. अपि॑ दद्ध्याद् दद्ध्या॒ दप्यपि॑ दद्ध्या॒त् तस्मा॒त् तस्मा᳚द् दद्ध्या॒ दप्यपि॑ दद्ध्या॒त् तस्मा᳚त् । \newline
40. द॒द्ध्या॒त् तस्मा॒त् तस्मा᳚द् दद्ध्याद् दद्ध्या॒त् तस्मा॒न् न न तस्मा᳚द् दद्ध्याद् दद्ध्या॒त् तस्मा॒न् न । \newline
41. तस्मा॒न् न न तस्मा॒त् तस्मा॒न् नाक्रम्या॒ ऽऽक्रम्या॒ न तस्मा॒त् तस्मा॒न् नाक्रम्या᳚ । \newline
42. नाक्रम्या॒ ऽऽक्रम्या॒ न नाक्रम्या॑ पुरुषशी॒र्॒.षम् पु॑रुषशी॒र्॒.ष मा॒क्रम्या॒ न नाक्रम्या॑ पुरुषशी॒र्॒.षम् । \newline
43. आ॒क्रम्या॑ पुरुषशी॒र्॒.षम् पु॑रुषशी॒र्॒.ष मा॒क्रम्या॒ ऽऽक्रम्या॑ पुरुषशी॒र्॒.ष मुपोप॑ पुरुषशी॒र्॒.ष मा॒क्रम्या॒ ऽऽक्रम्या॑ पुरुषशी॒र्॒.ष मुप॑ । \newline
44. आ॒क्रम्येत्या᳚ - क्रम्या᳚ । \newline
45. पु॒रु॒ष॒शी॒र्॒.ष मुपोप॑ पुरुषशी॒र्॒.षम् पु॑रुषशी॒र्॒.ष मुप॑ दधाति दधा॒ त्युप॑ पुरुषशी॒र्॒.षम् पु॑रुषशी॒र्॒.ष मुप॑ दधाति । \newline
46. पु॒रु॒ष॒शी॒र्॒.षमिति॑ पुरुष - शी॒र्॒.षम् । \newline
47. उप॑ दधाति दधा॒ त्युपोप॑ दधाति॒ गुप्त्यै॒ गुप्त्यै॑ दधा॒ त्युपोप॑ दधाति॒ गुप्त्यै᳚ । \newline
48. द॒धा॒ति॒ गुप्त्यै॒ गुप्त्यै॑ दधाति दधाति॒ गुप्त्या॒ अथो॒ अथो॒ गुप्त्यै॑ दधाति दधाति॒ गुप्त्या॒ अथो᳚ । \newline
49. गुप्त्या॒ अथो॒ अथो॒ गुप्त्यै॒ गुप्त्या॒ अथो॒ यथा॒ यथा ऽथो॒ गुप्त्यै॒ गुप्त्या॒ अथो॒ यथा᳚ । \newline
50. अथो॒ यथा॒ यथा ऽथो॒ अथो॒ यथा᳚ ब्रू॒याद् ब्रू॒याद् यथा ऽथो॒ अथो॒ यथा᳚ ब्रू॒यात् । \newline
51. अथो॒ इत्यथो᳚ । \newline
52. यथा᳚ ब्रू॒याद् ब्रू॒याद् यथा॒ यथा᳚ ब्रू॒या दे॒त दे॒तद् ब्रू॒याद् यथा॒ यथा᳚ ब्रू॒या दे॒तत् । \newline
53. ब्रू॒या दे॒त दे॒तद् ब्रू॒याद् ब्रू॒या दे॒तन् मे॑ म ए॒तद् ब्रू॒याद् ब्रू॒या दे॒तन् मे᳚ । \newline
54. ए॒तन् मे॑ म ए॒त दे॒तन् मे॑ गोपाय गोपाय म ए॒त दे॒तन् मे॑ गोपाय । \newline
55. मे॒ गो॒पा॒य॒ गो॒पा॒य॒ मे॒ मे॒ गो॒पा॒ये तीति॑ गोपाय मे मे गोपा॒येति॑ । \newline
56. गो॒पा॒ये तीति॑ गोपाय गोपा॒येति॑ ता॒दृक् ता॒दृगिति॑ गोपाय गोपा॒येति॑ ता॒दृक् । \newline
57. इति॑ ता॒दृक् ता॒दृ गितीति॑ ता॒दृ गे॒वैव ता॒दृ गितीति॑ ता॒दृ गे॒व । \newline
58. ता॒दृ गे॒वैव ता॒दृक् ता॒दृ गे॒व तत् तदे॒व ता॒दृक् ता॒दृ गे॒व तत् । \newline
59. ए॒व तत् तदे॒ वैव तत् प्र॒जाप॑तिः प्र॒जाप॑ति॒ स्तदे॒वैव तत् प्र॒जाप॑तिः । \newline
60. तत् प्र॒जाप॑तिः प्र॒जाप॑ति॒ स्तत् तत् प्र॒जाप॑ति॒र् वै वै प्र॒जाप॑ति॒ स्तत् तत् प्र॒जाप॑ति॒र् वै । \newline
\pagebreak
\markright{ TS 5.6.6.3  \hfill https://www.vedavms.in \hfill}

\section{ TS 5.6.6.3 }

\textbf{TS 5.6.6.3 } \newline
\textbf{Samhita Paata} \newline

प्र॒जाप॑ति॒र्वा अथ॑र्वा॒ ऽग्निरे॒व द॒द्ध्यङ्ङा॑थर्व॒णस्तस्येष्ट॑का अ॒स्थान्ये॒तꣳ ह॒ वाव तद्-ऋषि॑र॒भ्यनू॑वा॒चेन्द्रो॑ दधी॒चो अ॒स्थभि॒रिति॒ यदिष्ट॑काभिर॒ग्निं चि॒नोति॒ सात्मा॑नमे॒वाग्निं चि॑नुते॒ सात्मा॒मुष्मि॑न् ॅलो॒के भ॑वति॒ य ए॒वं ॅवेद॒ शरी॑रं॒ ॅवा ए॒तद॒ग्नेर्यच्चित्य॑ आ॒त्मा वै᳚श्वान॒रो यच्चि॒ते वै᳚श्वान॒रं जु॒होति॒ शरी॑रमे॒व सꣳ॒॒स्कृत्या॒ - [  ] \newline

\textbf{Pada Paata} \newline

प्र॒जाप॑ति॒रिति॑ प्र॒जा - प॒तिः॒ । वै । अथ॑र्वा । अ॒ग्निः । ए॒व । द॒द्ध्यङ् । आ॒थ॒र्व॒णः । तस्य॑ । इष्ट॑काः । अ॒स्थानि॑ । ए॒तम् । ह॒ । वाव । तत् । ऋषिः॑ । अ॒भ्यनू॑वा॒चेत्य॑भि - अनू॑वाच । इन्द्रः॑ । द॒धी॒चः । अ॒स्थभि॒रित्य॒स्थ - भिः॒ । इति॑ । यत् । इष्ट॑काभिः । अ॒ग्निम् । चि॒नोति॑ । सात्मा॑न॒मिति॒ स - आ॒त्मा॒न॒म् । ए॒व । अ॒ग्निम् । चि॒नु॒ते॒ । सात्मेति॒ स - आ॒त्मा॒ । अ॒मुष्मिन्न्॑ । लो॒के । भ॒व॒ति॒ । यः । ए॒वम् । वेद॑ । शरी॑रम् । वै । ए॒तत् । अ॒ग्नेः । यत् । चित्यः॑ । आ॒त्मा । वै॒श्वा॒न॒रः । यत् । चि॒ते । वै॒श्वा॒न॒रम् । जु॒होति॑ । शरी॑रम् । ए॒व । सꣳ॒॒स्कृत्य॑ ।  \newline


\textbf{Krama Paata} \newline

प्र॒जाप॑ति॒र् वै । प्र॒जाप॑ति॒रिति॑ प्र॒जा - प॒तिः॒ । वा अथ॑र्वा । अथ॑र्वा॒ऽग्निः । अ॒ग्निरे॒व । ए॒व द॒द्ध्यङ् । द॒द्ध्यङ्ङा॑थर्व॒णः । आ॒थ॒र्व॒णस्तस्य॑ । तस्येष्ट॑काः । इष्ट॑का अ॒स्थानि॑ । अ॒स्थान्ये॒तम् । ए॒तꣳ ह॑ । ह॒ वाव । वाव तत् । तदृषिः॑ । ऋषि॑र॒भ्यनू॑वाच । अ॒भ्यनू॑वा॒चेन्द्रः॑ । अ॒भ्यनू॑वा॒चेत्य॑भि - अनू॑वाच । इन्द्रो॑ दधी॒चः । द॒धी॒चो अ॒स्थभिः॑ । अ॒स्थभि॒रिति॑ । अ॒स्थभि॒रित्य॒स्थ - भिः॒ । इति॒ यत् । यदिष्ट॑काभिः । इष्ट॑काभिर॒ग्निम् । अ॒ग्निम् चि॒नोति॑ । चि॒नोति॒ सात्मा॑नम् । सात्मा॑नमे॒व । सात्मा॑न॒मिति॒ स - आ॒त्मा॒न॒म् । ए॒वाग्निम् । अ॒ग्निम् चि॑नुते । चि॒नु॒ते॒ सात्मा᳚ । सात्मा॒ऽमुष्मिन्न्॑ । सात्मेति॒ स - आ॒त्मा॒ । अ॒मुष्मि॑न् ॅलो॒के । लो॒के भ॑वति । भ॒व॒ति॒ यः । य ए॒वम् । ए॒वम् ॅवेद॑ । वेद॒ शरी॑रम् । शरी॑र॒म् ॅवै । वा ए॒तत् । ए॒तद॒ग्नेः । अ॒ग्नेर् यत् । यच् चित्यः॑ । चित्य॑ आ॒त्मा । आ॒त्मा वै᳚श्वान॒रः । वै॒श्वा॒न॒रो यत् । यच् चि॒ते । चि॒ते वै᳚श्वान॒रम् । वै॒श्वा॒न॒रम् जु॒होति॑ । जु॒होति॒ शरी॑रम् । शरी॑रमे॒व । ए॒व सꣳ॒॒स्कृत्य॑ । सꣳ॒॒स्कृत्या॒भ्यारो॑हति \newline

\textbf{Jatai Paata} \newline

1. प्र॒जाप॑ति॒र् वै वै प्र॒जाप॑तिः प्र॒जाप॑ति॒र् वै । \newline
2. प्र॒जाप॑ति॒रिति॑ प्र॒जा - प॒तिः॒ । \newline
3. वा अथ॒र्वा ऽथ॑र्वा॒ वै वा अथ॑र्वा । \newline
4. अथ॑र्वा॒ ऽग्नि र॒ग्नि रथ॒र्वा ऽथ॑र्वा॒ ऽग्निः । \newline
5. अ॒ग्नि रे॒वै वाग्नि र॒ग्नि रे॒व । \newline
6. ए॒व द॒द्ध्यङ् द॒द्ध्य ङे॒वैव द॒द्ध्यङ् । \newline
7. द॒द्ध्य ङा॑थर्व॒ण आ॑थर्व॒णो द॒द्ध्यङ् द॒द्ध्य ङा॑थर्व॒णः । \newline
8. आ॒थ॒र्व॒ण स्तस्य॒ तस्या॑थर्व॒ण आ॑थर्व॒ण स्तस्य॑ । \newline
9. तस्येष्ट॑का॒ इष्ट॑का॒ स्तस्य॒ तस्येष्ट॑काः । \newline
10. इष्ट॑का अ॒स्था न्य॒स्थानीष्ट॑का॒ इष्ट॑का अ॒स्थानि॑ । \newline
11. अ॒स्था न्ये॒त मे॒त म॒स्था न्य॒स्था न्ये॒तम् । \newline
12. ए॒तꣳ ह॑ है॒त मे॒तꣳ ह॑ । \newline
13. ह॒ वाव वाव ह॑ ह॒ वाव । \newline
14. वाव तत् तद् वाव वाव तत् । \newline
15. तदृषि॒र्॒. ऋषि॒ स्तत् तदृषिः॑ । \newline
16. ऋषि॑ र॒भ्यनू॑वाचा॒ भ्यनू॑वा॒च र्.षि॒र्॒. ऋषि॑ र॒भ्यनू॑वाच । \newline
17. अ॒भ्यनू॑वा॒चेन्द्र॒ इन्द्रो॒ ऽभ्यनू॑वाचा॒ भ्यनू॑वा॒चेन्द्रः॑ । \newline
18. अ॒भ्यनू॑वा॒चेत्य॑भि - अनू॑वाच । \newline
19. इन्द्रो॑ दधी॒चो द॑धी॒च इन्द्र॒ इन्द्रो॑ दधी॒चः । \newline
20. द॒धी॒चो अ॒स्थभि॑ र॒स्थभि॑र् दधी॒चो द॑धी॒चो अ॒स्थभिः॑ । \newline
21. अ॒स्थभि॒ रितीत्य॒ स्थभि॑ र॒स्थभि॒ रिति॑ । \newline
22. अ॒स्थभि॒रित्य॒स्थ - भिः॒ । \newline
23. इति॒ यद् यदितीति॒ यत् । \newline
24. यदिष्ट॑काभि॒ रिष्ट॑काभि॒र् यद् यदिष्ट॑काभिः । \newline
25. इष्ट॑काभि र॒ग्नि म॒ग्नि मिष्ट॑काभि॒ रिष्ट॑काभि र॒ग्निम् । \newline
26. अ॒ग्निम् चि॒नोति॑ चि॒नो त्य॒ग्नि म॒ग्निम् चि॒नोति॑ । \newline
27. चि॒नोति॒ सात्मा॑नꣳ॒॒ सात्मा॑नम् चि॒नोति॑ चि॒नोति॒ सात्मा॑नम् । \newline
28. सात्मा॑न मे॒वैव सात्मा॑नꣳ॒॒ सात्मा॑न मे॒व । \newline
29. सात्मा॑न॒मिति॒ स - आ॒त्मा॒न॒म् । \newline
30. ए॒वाग्नि म॒ग्नि मे॒वै वाग्निम् । \newline
31. अ॒ग्निम् चि॑नुते चिनुते॒ ऽग्नि म॒ग्निम् चि॑नुते । \newline
32. चि॒नु॒ते॒ सात्मा॒ सात्मा॑ चिनुते चिनुते॒ सात्मा᳚ । \newline
33. सात्मा॒ ऽमुष्मि॑न् न॒मुष्मि॒न् थ्सात्मा॒ सात्मा॒ ऽमुष्मिन्न्॑ । \newline
34. सात्मेति॒ स - आ॒त्मा॒ । \newline
35. अ॒मुष्मि॑न् ॅलो॒के लो॒के॑ ऽमुष्मि॑न् न॒मुष्मि॑न् ॅलो॒के । \newline
36. लो॒के भ॑वति भवति लो॒के लो॒के भ॑वति । \newline
37. भ॒व॒ति॒ यो यो भ॑वति भवति॒ यः । \newline
38. य ए॒व मे॒वं ॅयो य ए॒वम् । \newline
39. ए॒वं ॅवेद॒ वेदै॒व मे॒वं ॅवेद॑ । \newline
40. वेद॒ शरी॑रꣳ॒॒ शरी॑रं॒ ॅवेद॒ वेद॒ शरी॑रम् । \newline
41. शरी॑रं॒ ॅवै वै शरी॑रꣳ॒॒ शरी॑रं॒ ॅवै । \newline
42. वा ए॒त दे॒तद् वै वा ए॒तत् । \newline
43. ए॒त द॒ग्ने र॒ग्ने रे॒त दे॒त द॒ग्नेः । \newline
44. अ॒ग्नेर् यद् यद॒ग्ने र॒ग्नेर् यत् । \newline
45. यच् चित्य॒ श्चित्यो॒ यद् यच् चित्यः॑ । \newline
46. चित्य॑ आ॒त्मा ऽऽत्मा चित्य॒ श्चित्य॑ आ॒त्मा । \newline
47. आ॒त्मा वै᳚श्वान॒रो वै᳚श्वान॒र आ॒त्मा ऽऽत्मा वै᳚श्वान॒रः । \newline
48. वै॒श्वा॒न॒रो यद् यद् वै᳚श्वान॒रो वै᳚श्वान॒रो यत् । \newline
49. यच् चि॒ते चि॒ते यद् यच् चि॒ते । \newline
50. चि॒ते वै᳚श्वान॒रं ॅवै᳚श्वान॒रम् चि॒ते चि॒ते वै᳚श्वान॒रम् । \newline
51. वै॒श्वा॒न॒रम् जु॒होति॑ जु॒होति॑ वैश्वान॒रं ॅवै᳚श्वान॒रम् जु॒होति॑ । \newline
52. जु॒होति॒ शरी॑रꣳ॒॒ शरी॑रम् जु॒होति॑ जु॒होति॒ शरी॑रम् । \newline
53. शरी॑र मे॒वैव शरी॑रꣳ॒॒ शरी॑र मे॒व । \newline
54. ए॒व सꣳ॒॒स्कृत्य॑ सꣳ॒॒स्कृ त्यै॒वैव सꣳ॒॒स्कृत्य॑ । \newline
55. सꣳ॒॒स्कृत्या॒ भ्यारो॑ह त्य॒भ्यारो॑हति सꣳ॒॒स्कृत्य॑ सꣳ॒॒स्कृत्या॒ भ्यारो॑हति । \newline

\textbf{Ghana Paata } \newline

1. प्र॒जाप॑ति॒र् वै वै प्र॒जाप॑तिः प्र॒जाप॑ति॒र् वा अथ॒र्वा ऽथ॑र्वा॒ वै प्र॒जाप॑तिः प्र॒जाप॑ति॒र् वा अथ॑र्वा । \newline
2. प्र॒जाप॑ति॒रिति॑ प्र॒जा - प॒तिः॒ । \newline
3. वा अथ॒र्वा ऽथ॑र्वा॒ वै वा अथ॑र्वा॒ ऽग्नि र॒ग्नि रथ॑र्वा॒ वै वा अथ॑र्वा॒ ऽग्निः । \newline
4. अथ॑र्वा॒ ऽग्नि र॒ग्नि रथ॒र्वा ऽथ॑र्वा॒ ऽग्नि रे॒वै वाग्नि रथ॒र्वा ऽथ॑र्वा॒ ऽग्नि रे॒व । \newline
5. अ॒ग्नि रे॒वै वाग्नि र॒ग्नि रे॒व द॒द्ध्यङ् द॒द्ध्य ङे॒वाग्नि र॒ग्नि रे॒व द॒द्ध्यङ् । \newline
6. ए॒व द॒द्ध्यङ् द॒द्ध्य ङे॒वैव द॒द्ध्यङ् आ॑थर्व॒ण आ॑थर्व॒णो द॒द्ध्य ङे॒वैव द॒द्ध्यङ् आ॑थर्व॒णः । \newline
7. द॒द्ध्य ङा॑थर्व॒ण आ॑थर्व॒णो द॒द्ध्यङ् द॒द्ध्य ङा॑थर्व॒ण स्तस्य॒ तस्या॑ थर्व॒णो 
द॒द्ध्यङ् द॒द्ध्य ङा॑थर्व॒ण स्तस्य॑ । \newline
8. आ॒थ॒र्व॒ण स्तस्य॒ तस्या॑ थर्व॒ण आ॑थर्व॒ण स्तस्येष्ट॑का॒ इष्ट॑का॒ स्तस्या॑ थर्व॒ण आ॑थर्व॒ण 
स्तस्येष्ट॑काः । \newline
9. तस्येष्ट॑का॒ इष्ट॑का॒ स्तस्य॒ तस्येष्ट॑का अ॒स्था न्य॒स्थानीष्ट॑का॒ स्तस्य॒ तस्येष्ट॑का अ॒स्थानि॑ । \newline
10. इष्ट॑का अ॒स्था न्य॒स्थानीष्ट॑का॒ इष्ट॑का अ॒स्था न्ये॒त मे॒त म॒स्थानीष्ट॑का॒ इष्ट॑का अ॒स्था न्ये॒तम् । \newline
11. अ॒स्था न्ये॒त मे॒त म॒स्था न्य॒स्था न्ये॒तꣳ ह॑ है॒त म॒स्था न्य॒स्था न्ये॒तꣳ ह॑ । \newline
12. ए॒तꣳ ह॑ है॒त मे॒तꣳ ह॒ वाव वाव है॒त मे॒तꣳ ह॒ वाव । \newline
13. ह॒ वाव वाव ह॑ ह॒ वाव तत् तद् वाव ह॑ ह॒ वाव तत् । \newline
14. वाव तत् तद् वाव वाव तदृषि॒र्॒. ऋषि॒ स्तद् वाव वाव तदृषिः॑ । \newline
15. तदृषि॒र्॒. ऋषि॒ स्तत् तदृषि॑ र॒भ्यनू॑वा चा॒भ्यनू॑वा॒च र्.षि॒ स्तत् तदृषि॑ र॒भ्यनू॑वाच । \newline
16. ऋषि॑ र॒भ्यनू॑वाचा॒ भ्यनू॑वा॒च र्.षि॒र्॒. ऋषि॑ र॒भ्यनू॑वा॒चेन्द्र॒ इन्द्रो॒ ऽभ्यनू॑वा॒च र्.षि॒र्॒. ऋषि॑ र॒भ्यनू॑वा॒चेन्द्रः॑ । \newline
17. अ॒भ्यनू॑वा॒चेन्द्र॒ इन्द्रो॒ ऽभ्यनू॑वाचा॒ भ्यनू॑वा॒चेन्द्रो॑ दधी॒चो द॑धी॒च इन्द्रो॒ ऽभ्यनू॑वाचा॒ 
भ्यनू॑वा॒चेन्द्रो॑ दधी॒चः । \newline
18. अ॒भ्यनू॑वा॒चेत्य॑भि - अनू॑वाच । \newline
19. इन्द्रो॑ दधी॒चो द॑धी॒च इन्द्र॒ इन्द्रो॑ दधी॒चो अ॒स्थभि॑ र॒स्थभि॑र् दधी॒च इन्द्र॒ इन्द्रो॑ दधी॒चो अ॒स्थभिः॑ । \newline
20. द॒धी॒चो अ॒स्थभि॑ र॒स्थभि॑र् दधी॒चो द॑धी॒चो अ॒स्थभि॒ रितीत्य॒ स्थभि॑र् दधी॒चो द॑धी॒चो अ॒स्थभि॒ रिति॑ । \newline
21. अ॒स्थभि॒ रिती त्य॒स्थभि॑ र॒स्थभि॒ रिति॒ यद् यदित्य॒ स्थभि॑ र॒स्थभि॒ रिति॒ यत् । \newline
22. अ॒स्थभि॒रित्य॒स्थ - भिः॒ । \newline
23. इति॒ यद् यदितीति॒ यदिष्ट॑काभि॒ रिष्ट॑काभि॒र् यदितीति॒ यदिष्ट॑काभिः । \newline
24. यदिष्ट॑काभि॒ रिष्ट॑काभि॒र् यद् यदिष्ट॑काभि र॒ग्नि म॒ग्नि मिष्ट॑काभि॒र् यद् यदिष्ट॑काभि र॒ग्निम् । \newline
25. इष्ट॑काभि र॒ग्नि म॒ग्नि मिष्ट॑काभि॒ रिष्ट॑काभि र॒ग्निम् चि॒नोति॑ चि॒नो त्य॒ग्नि मिष्ट॑काभि॒ रिष्ट॑काभि र॒ग्निम् चि॒नोति॑ । \newline
26. अ॒ग्निम् चि॒नोति॑ चि॒नो त्य॒ग्नि म॒ग्निम् चि॒नोति॒ सात्मा॑नꣳ॒॒ सात्मा॑नम् चि॒नो त्य॒ग्नि म॒ग्निम् चि॒नोति॒ सात्मा॑नम् । \newline
27. चि॒नोति॒ सात्मा॑नꣳ॒॒ सात्मा॑नम् चि॒नोति॑ चि॒नोति॒ सात्मा॑न मे॒वैव सात्मा॑नम् चि॒नोति॑ चि॒नोति॒ सात्मा॑न मे॒व । \newline
28. सात्मा॑न मे॒वैव सात्मा॑नꣳ॒॒ सात्मा॑न मे॒वाग्नि म॒ग्नि मे॒व सात्मा॑नꣳ॒॒ सात्मा॑न मे॒वाग्निम् । \newline
29. सात्मा॑न॒मिति॒ स - आ॒त्मा॒न॒म् । \newline
30. ए॒वाग्नि म॒ग्नि मे॒वै वाग्निम् चि॑नुते चिनुते॒ ऽग्नि मे॒वै वाग्निम् चि॑नुते । \newline
31. अ॒ग्निम् चि॑नुते चिनुते॒ ऽग्नि म॒ग्निम् चि॑नुते॒ सात्मा॒ सात्मा॑ चिनुते॒ ऽग्नि म॒ग्निम् चि॑नुते॒ सात्मा᳚ । \newline
32. चि॒नु॒ते॒ सात्मा॒ सात्मा॑ चिनुते चिनुते॒ सात्मा॒ ऽमुष्मि॑न् न॒मुष्मि॒न् थ्सात्मा॑ चिनुते चिनुते॒ सात्मा॒ ऽमुष्मिन्न्॑ । \newline
33. सात्मा॒ ऽमुष्मि॑न् न॒मुष्मि॒न् थ्सात्मा॒ सात्मा॒ ऽमुष्मि॑न् ॅलो॒के लो॒के॑ ऽमुष्मि॒न् थ्सात्मा॒ सात्मा॒ ऽमुष्मि॑न् ॅलो॒के । \newline
34. सात्मेति॒ स - आ॒त्मा॒ । \newline
35. अ॒मुष्मि॑न् ॅलो॒के लो॒के॑ ऽमुष्मि॑न् न॒मुष्मि॑न् ॅलो॒के भ॑वति भवति लो॒के॑ ऽमुष्मि॑न् न॒मुष्मि॑न् ॅलो॒के भ॑वति । \newline
36. लो॒के भ॑वति भवति लो॒के लो॒के भ॑वति॒ यो यो भ॑वति लो॒के लो॒के भ॑वति॒ यः । \newline
37. भ॒व॒ति॒ यो यो भ॑वति भवति॒ य ए॒व मे॒वं ॅयो भ॑वति भवति॒ य ए॒वम् । \newline
38. य ए॒व मे॒वं ॅयो य ए॒वं ॅवेद॒ वेदै॒वं ॅयो य ए॒वं ॅवेद॑ । \newline
39. ए॒वं ॅवेद॒ वेदै॒व मे॒वं ॅवेद॒ शरी॑रꣳ॒॒ शरी॑रं॒ ॅवेदै॒व मे॒वं ॅवेद॒ शरी॑रम् । \newline
40. वेद॒ शरी॑रꣳ॒॒ शरी॑रं॒ ॅवेद॒ वेद॒ शरी॑रं॒ ॅवै वै शरी॑रं॒ ॅवेद॒ वेद॒ शरी॑रं॒ ॅवै । \newline
41. शरी॑रं॒ ॅवै वै शरी॑रꣳ॒॒ शरी॑रं॒ ॅवा ए॒त दे॒तद् वै शरी॑रꣳ॒॒ शरी॑रं॒ ॅवा ए॒तत् । \newline
42. वा ए॒त दे॒तद् वै वा ए॒त द॒ग्ने र॒ग्ने रे॒तद् वै वा ए॒त द॒ग्नेः । \newline
43. ए॒त द॒ग्ने र॒ग्ने रे॒त दे॒त द॒ग्नेर् यद् यद॒ग्ने रे॒त दे॒त द॒ग्नेर् यत् । \newline
44. अ॒ग्नेर् यद् यद॒ग्ने र॒ग्नेर् यच् चित्य॒ श्चित्यो॒ यद॒ग्ने र॒ग्नेर् यच् चित्यः॑ । \newline
45. यच् चित्य॒ श्चित्यो॒ यद् यच् चित्य॑ आ॒त्मा ऽऽत्मा चित्यो॒ यद् यच् चित्य॑ आ॒त्मा । \newline
46. चित्य॑ आ॒त्मा ऽऽत्मा चित्य॒ श्चित्य॑ आ॒त्मा वै᳚श्वान॒रो वै᳚श्वान॒र आ॒त्मा चित्य॒ श्चित्य॑ आ॒त्मा वै᳚श्वान॒रः । \newline
47. आ॒त्मा वै᳚श्वान॒रो वै᳚श्वान॒र आ॒त्मा ऽऽत्मा वै᳚श्वान॒रो यद् यद् वै᳚श्वान॒र आ॒त्मा ऽऽत्मा वै᳚श्वान॒रो यत् । \newline
48. वै॒श्वा॒न॒रो यद् यद् वै᳚श्वान॒रो वै᳚श्वान॒रो यच् चि॒ते चि॒ते यद् वै᳚श्वान॒रो वै᳚श्वान॒रो यच् चि॒ते । \newline
49. यच् चि॒ते चि॒ते यद् यच् चि॒ते वै᳚श्वान॒रं ॅवै᳚श्वान॒रम् चि॒ते यद् यच् चि॒ते वै᳚श्वान॒रम् । \newline
50. चि॒ते वै᳚श्वान॒रं ॅवै᳚श्वान॒रम् चि॒ते चि॒ते वै᳚श्वान॒रम् जु॒होति॑ जु॒होति॑ वैश्वान॒रम् चि॒ते चि॒ते वै᳚श्वान॒रम् जु॒होति॑ । \newline
51. वै॒श्वा॒न॒रम् जु॒होति॑ जु॒होति॑ वैश्वान॒रं ॅवै᳚श्वान॒रम् जु॒होति॒ शरी॑रꣳ॒॒ शरी॑रम् जु॒होति॑ वैश्वान॒रं ॅवै᳚श्वान॒रम् जु॒होति॒ शरी॑रम् । \newline
52. जु॒होति॒ शरी॑रꣳ॒॒ शरी॑रम् जु॒होति॑ जु॒होति॒ शरी॑र मे॒वैव शरी॑रम् जु॒होति॑ जु॒होति॒ शरी॑र मे॒व । \newline
53. शरी॑र मे॒वैव शरी॑रꣳ॒॒ शरी॑र मे॒व सꣳ॒॒स्कृत्य॑ सꣳ॒॒स्कृ त्यै॒व शरी॑रꣳ॒॒ शरी॑र मे॒व सꣳ॒॒स्कृत्य॑ । \newline
54. ए॒व सꣳ॒॒स्कृत्य॑ सꣳ॒॒स्कृ त्यै॒वैव सꣳ॒॒स्कृ त्या॒भ्यारो॑ह त्य॒भ्यारो॑हति सꣳ॒॒स्कृ त्यै॒वैव सꣳ॒॒स्कृ त्या॒भ्यारो॑हति । \newline
55. सꣳ॒॒स्कृ त्या॒भ्यारो॑ह त्य॒भ्यारो॑हति सꣳ॒॒स्कृत्य॑ सꣳ॒॒स्कृ त्या॒भ्यारो॑हति॒ शरी॑रꣳ॒॒ शरी॑र म॒भ्यारो॑हति सꣳ॒॒स्कृत्य॑ सꣳ॒॒स्कृ त्या॒भ्यारो॑हति॒ शरी॑रम् । \newline
\pagebreak
\markright{ TS 5.6.6.4  \hfill https://www.vedavms.in \hfill}

\section{ TS 5.6.6.4 }

\textbf{TS 5.6.6.4 } \newline
\textbf{Samhita Paata} \newline

-भ्यारो॑हति॒ शरी॑रं॒ ॅवा ए॒तद्-यज॑मानः॒ सꣳ स्कु॑रुते॒ यद॒ग्निं चि॑नु॒ते यच्चि॒ते वै᳚श्वान॒रं जु॒होति॒ शरी॑रमे॒व सꣳ॒॒स्कृत्या॒ ऽऽ*त्मना॒ऽभ्यारो॑हति॒ तस्मा॒त् तस्य॒ नाव॑ द्यन्ति॒ जीव॑न्ने॒व दे॒वानप्ये॑ति वैश्वान॒र्यर्चा पुरी॑ष॒मुप॑ दधाती॒यं ॅवा अ॒ग्निर्वै᳚श्वान॒रस्तस्यै॒षा चिति॒र्यत् पुरी॑षम॒ग्निमे॒व वै᳚श्वान॒रं चि॑नुत ए॒षा वा अ॒ग्नेः ( ) प्रि॒या त॒नूर्यद्-वै᳚श्वान॒रः प्रि॒यामे॒वास्य॑ त॒नुव॒मव॑ रुन्धे ॥ \newline

\textbf{Pada Paata} \newline

अ॒भ्यारो॑ह॒तीत्य॑भि - आरो॑हति । शरी॑रम् । वै । ए॒तत् । यज॑मानः । समिति॑ । कु॒रु॒ते॒ । यत् । अ॒ग्निम् । चि॒नु॒ते । यत् । चि॒ते । वै॒श्वा॒न॒रम् । जु॒होति॑ । शरी॑रम् । ए॒व । सꣳ॒॒स्कृत्य॑ । आ॒त्मना᳚ । अ॒भ्यारो॑ह॒तीत्य॑भि -आरो॑हति । तस्मा᳚त् । तस्य॑ । न । अवेति॑ । द्य॒न्ति॒ । जीवन्न्॑ । ए॒व । दे॒वान् । अपीति॑ । ए॒ति॒ । वै॒श्वा॒न॒र्या । ऋ॒चा । पुरी॑षम् । उपेति॑ । द॒धा॒ति॒ । इ॒यम् । वै । अ॒ग्निः । वै॒श्वा॒न॒रः । तस्य॑ । ए॒षा । चितिः॑ । यत् । पुरी॑षम् । अ॒ग्निम् । ए॒व । वै॒श्वा॒न॒रम् । चि॒नु॒ते॒ । ए॒षा । वै । अ॒ग्नेः ( ) । प्रि॒या । त॒नूः । यत् । वै॒श्वा॒न॒रः । प्रि॒याम् । ए॒व । अ॒स्य॒ । त॒नुव᳚म् । अवेति॑ । रु॒न्धे॒ ॥  \newline


\textbf{Krama Paata} \newline

अ॒भ्यारो॑हति॒ शरी॑रम् । अ॒भ्यारो॑ह॒तीत्य॑भि - आरो॑हति । शरी॑र॒म् ॅवै । वा ए॒तत् । ए॒तद् यज॑मानः । यज॑मानः॒ सम् । सꣳ स्कु॑रुते । कु॒रु॒ते॒ यत् । यद॒ग्निम् । अ॒ग्निम् चि॑नु॒ते । चि॒नु॒ते यत् । यच् चि॒ते । चि॒ते वै᳚श्वान॒रम् । वै॒श्वा॒न॒रम् जु॒होति॑ । जु॒होति॒ शरी॑रम् । शरी॑रमे॒व । ए॒व सꣳ॒॒स्कृत्य॑ । सꣳ॒॒स्कृत्या॒त्मना᳚ । आ॒त्मना॒ऽभ्यारो॑हति । अ॒भ्यारो॑हति॒ तस्मा᳚त् । अ॒भ्यारो॑ह॒तीत्य॑भि - आरो॑हति । तस्मा॒त् तस्य॑ । तस्य॒ न । नाव॑ । अव॑ द्यन्ति । द्य॒न्ति॒ जीवन्न्॑ । जीव॑न्ने॒व । ए॒व दे॒वान् । दे॒वानपि॑ । अप्ये॑ति । ए॒ति॒ वै॒श्वा॒न॒र्या । वै॒श्वा॒न॒र्यर्.चा । ऋ॒चा पुरी॑षम् । पुरी॑ष॒मुप॑ । उप॑ दधाति । द॒धा॒ती॒यम् । इ॒यम् ॅवै । वा अ॒ग्निः । अ॒ग्निर् वै᳚श्वान॒रः । वै॒श्वा॒न॒रस्तस्य॑ । तस्यै॒षा । ए॒षा चितिः॑ । चिति॒र् यत् । यत् पुरी॑षम् । पुरी॑षम॒ग्निम् । अ॒ग्निमे॒व । ए॒व वै᳚श्वान॒रम् । वै॒श्वा॒न॒रम् चि॑नुते । चि॒नु॒त॒ ए॒षा । ए॒षा वै । वा अ॒ग्नेः ( ) । अ॒ग्नेः प्रि॒या । प्रि॒या त॒नूः । त॒नूर् यत् । यद् वै᳚श्वान॒रः । वै॒श्वा॒न॒रः प्रि॒याम् । प्रि॒यामे॒व । ए॒वास्य॑ । अ॒स्य॒ त॒नुव᳚म् । त॒नुव॒मव॑ । अव॑ रुन्धे । रु॒न्ध॒ इति॑ रुन्धे । \newline

\textbf{Jatai Paata} \newline

1. अ॒भ्यारो॑हति॒ शरी॑रꣳ॒॒ शरी॑र म॒भ्यारो॑ह त्य॒भ्यारो॑हति॒ शरी॑रम् । \newline
2. अ॒भ्यारो॑ह॒तीत्य॑भि - आरो॑हति । \newline
3. शरी॑रं॒ ॅवै वै शरी॑रꣳ॒॒ शरी॑रं॒ ॅवै । \newline
4. वा ए॒त दे॒तद् वै वा ए॒तत् । \newline
5. ए॒तद् यज॑मानो॒ यज॑मान ए॒त दे॒तद् यज॑मानः । \newline
6. यज॑मानः॒ सꣳ सं ॅयज॑मानो॒ यज॑मानः॒ सम् । \newline
7. सꣳ स्कु॑रुते कुरुते॒ सꣳ सꣳ स्कु॑रुते । \newline
8. कु॒रु॒ते॒ यद् यत् कु॑रुते कुरुते॒ यत् । \newline
9. यद॒ग्नि म॒ग्निं ॅयद् यद॒ग्निम् । \newline
10. अ॒ग्निम् चि॑नु॒ते चि॑नु॒ते᳚ ऽग्नि म॒ग्निम् चि॑नु॒ते । \newline
11. चि॒नु॒ते यद् यच् चि॑नु॒ते चि॑नु॒ते यत् । \newline
12. यच् चि॒ते चि॒ते यद् यच् चि॒ते । \newline
13. चि॒ते वै᳚श्वान॒रं ॅवै᳚श्वान॒रम् चि॒ते चि॒ते वै᳚श्वान॒रम् । \newline
14. वै॒श्वा॒न॒रम् जु॒होति॑ जु॒होति॑ वैश्वान॒रं ॅवै᳚श्वान॒रम् जु॒होति॑ । \newline
15. जु॒होति॒ शरी॑रꣳ॒॒ शरी॑रम् जु॒होति॑ जु॒होति॒ शरी॑रम् । \newline
16. शरी॑र मे॒वैव शरी॑रꣳ॒॒ शरी॑र मे॒व । \newline
17. ए॒व सꣳ॒॒स्कृत्य॑ सꣳ॒॒स्कृ त्यै॒वैव सꣳ॒॒स्कृत्य॑ । \newline
18. सꣳ॒॒स्कृ त्या॒त्मना॒ ऽऽत्मना॑ सꣳ॒॒स्कृत्य॑ सꣳ॒॒स्कृ त्या॒त्मना᳚ । \newline
19. आ॒त्मना॒ ऽभ्यारो॑ह त्य॒भ्यारो॑ह त्या॒त्मना॒ ऽऽत्मना॒ ऽभ्यारो॑हति । \newline
20. अ॒भ्यारो॑हति॒ तस्मा॒त् तस्मा॑ द॒भ्यारो॑ह त्य॒भ्यारो॑हति॒ तस्मा᳚त् । \newline
21. अ॒भ्यारो॑ह॒तीत्य॑भि - आरो॑हति । \newline
22. तस्मा॒त् तस्य॒ तस्य॒ तस्मा॒त् तस्मा॒त् तस्य॑ । \newline
23. तस्य॒ न न तस्य॒ तस्य॒ न । \newline
24. नावाव॒ न नाव॑ । \newline
25. अव॑ द्यन्ति द्य॒न् त्यवाव॑ द्यन्ति । \newline
26. द्य॒न्ति॒ जीव॒न् जीव॑न् द्यन्ति द्यन्ति॒ जीवन्न्॑ । \newline
27. जीव॑न् ने॒वैव जीव॒न् जीव॑न् ने॒व । \newline
28. ए॒व दे॒वान् दे॒वा ने॒वैव दे॒वान् । \newline
29. दे॒वा नप्यपि॑ दे॒वान् दे॒वा नपि॑ । \newline
30. अप्ये᳚ त्ये॒ त्यप्य प्ये॑ति । \newline
31. ए॒ति॒ वै॒श्वा॒न॒र्या वै᳚श्वान॒र् यैत्ये॑ति वैश्वान॒र्या । \newline
32. वै॒श्वा॒न॒र्य र्‌च र्‌चा वै᳚श्वान॒र्या वै᳚श्वान॒र्य र्‌चा । \newline
33. ऋ॒चा पुरी॑ष॒म् पुरी॑ष मृ॒च र्‌चा पुरी॑षम् । \newline
34. पुरी॑ष॒ मुपोप॒ पुरी॑ष॒म् पुरी॑ष॒ मुप॑ । \newline
35. उप॑ दधाति दधा॒ त्युपोप॑ दधाति । \newline
36. द॒धा॒ ती॒य मि॒यम् द॑धाति दधा ती॒यम् । \newline
37. इ॒यं ॅवै वा इ॒य मि॒यं ॅवै । \newline
38. वा अ॒ग्नि र॒ग्निर् वै वा अ॒ग्निः । \newline
39. अ॒ग्निर् वै᳚श्वान॒रो वै᳚श्वान॒रो᳚ ऽग्नि र॒ग्निर् वै᳚श्वान॒रः । \newline
40. वै॒श्वा॒न॒र स्तस्य॒ तस्य॑ वैश्वान॒रो वै᳚श्वान॒र स्तस्य॑ । \newline
41. तस्यै॒ षैषा तस्य॒ तस्यै॒षा । \newline
42. ए॒षा चिति॒ श्चिति॑ रे॒षैषा चितिः॑ । \newline
43. चिति॒र् यद् यच् चिति॒ श्चिति॒र् यत् । \newline
44. यत् पुरी॑ष॒म् पुरी॑षं॒ ॅयद् यत् पुरी॑षम् । \newline
45. पुरी॑ष म॒ग्नि म॒ग्निम् पुरी॑ष॒म् पुरी॑ष म॒ग्निम् । \newline
46. अ॒ग्नि मे॒वै वाग्नि म॒ग्नि मे॒व । \newline
47. ए॒व वै᳚श्वान॒रं ॅवै᳚श्वान॒र मे॒वैव वै᳚श्वान॒रम् । \newline
48. वै॒श्वा॒न॒रम् चि॑नुते चिनुते वैश्वान॒रं ॅवै᳚श्वान॒रम् चि॑नुते । \newline
49. चि॒नु॒त॒ ए॒षैषा चि॑नुते चिनुत ए॒षा । \newline
50. ए॒षा वै वा ए॒षैषा वै । \newline
51. वा अ॒ग्ने र॒ग्नेर् वै वा अ॒ग्नेः । \newline
52. अ॒ग्नेः प्रि॒या प्रि॒या ऽग्ने र॒ग्नेः प्रि॒या । \newline
53. प्रि॒या त॒नू स्त॒नूः प्रि॒या प्रि॒या त॒नूः । \newline
54. त॒नूर् यद् यत् त॒नू स्त॒नूर् यत् । \newline
55. यद् वै᳚श्वान॒रो वै᳚श्वान॒रो यद् यद् वै᳚श्वान॒रः । \newline
56. वै॒श्वा॒न॒रः प्रि॒याम् प्रि॒यां ॅवै᳚श्वान॒रो वै᳚श्वान॒रः प्रि॒याम् । \newline
57. प्रि॒या मे॒वैव प्रि॒याम् प्रि॒या मे॒व । \newline
58. ए॒वास्या᳚ स्यै॒वै वास्य॑ । \newline
59. अ॒स्य॒ त॒नुव॑म् त॒नुव॑ मस्यास्य त॒नुव᳚म् । \newline
60. त॒नुव॒ मवाव॑ त॒नुव॑म् त॒नुव॒ मव॑ । \newline
61. अव॑ रुन्धे रु॒न्धे ऽवाव॑ रुन्धे । \newline
62. रु॒न्ध॒ इति॑ रुन्धे । \newline

\textbf{Ghana Paata } \newline

1. अ॒भ्यारो॑हति॒ शरी॑रꣳ॒॒ शरी॑र म॒भ्यारो॑ह त्य॒भ्यारो॑हति॒ शरी॑रं॒ ॅवै वै शरी॑र म॒भ्यारो॑ह
त्य॒भ्यारो॑हति॒ शरी॑रं॒ ॅवै । \newline
2. अ॒भ्यारो॑ह॒तीत्य॑भि - आरो॑हति । \newline
3. शरी॑रं॒ ॅवै वै शरी॑रꣳ॒॒ शरी॑रं॒ ॅवा ए॒त दे॒तद् वै शरी॑रꣳ॒॒ शरी॑रं॒ ॅवा ए॒तत् । \newline
4. वा ए॒त दे॒तद् वै वा ए॒तद् यज॑मानो॒ यज॑मान ए॒तद् वै वा ए॒तद् यज॑मानः । \newline
5. ए॒तद् यज॑मानो॒ यज॑मान ए॒त दे॒तद् यज॑मानः॒ सꣳ सं ॅयज॑मान ए॒त दे॒तद् यज॑मानः॒ सम् । \newline
6. यज॑मानः॒ सꣳ सं ॅयज॑मानो॒ यज॑मानः॒ सꣳस्कु॑रुते कुरुते॒ सं ॅयज॑मानो॒ यज॑मानः॒ सम् कु॑रुते । \newline
7. सम् कु॑रुते कुरुते॒ सꣳ सꣳस्कु॑रुते॒ यद् यत् कु॑रुते॒ सꣳ सꣳस्कु॑रुते॒ यत् । \newline
8. कु॒रु॒ते॒ यद् यत् कु॑रुते कुरुते॒ यद॒ग्नि म॒ग्निं ॅयत् कु॑रुते कुरुते॒ यद॒ग्निम् । \newline
9. यद॒ग्नि म॒ग्निं ॅयद् यद॒ग्निम् चि॑नु॒ते चि॑नु॒ते᳚ ऽग्निं ॅयद् यद॒ग्निम् चि॑नु॒ते । \newline
10. अ॒ग्निम् चि॑नु॒ते चि॑नु॒ते᳚ ऽग्नि म॒ग्निम् चि॑नु॒ते यद् यच् चि॑नु॒ते᳚ ऽग्नि म॒ग्निम् चि॑नु॒ते यत् । \newline
11. चि॒नु॒ते यद् यच् चि॑नु॒ते चि॑नु॒ते यच् चि॒ते चि॒ते यच् चि॑नु॒ते चि॑नु॒ते यच् चि॒ते । \newline
12. यच् चि॒ते चि॒ते यद् यच् चि॒ते वै᳚श्वान॒रं ॅवै᳚श्वान॒रम् चि॒ते यद् यच् चि॒ते वै᳚श्वान॒रम् । \newline
13. चि॒ते वै᳚श्वान॒रं ॅवै᳚श्वान॒रम् चि॒ते चि॒ते वै᳚श्वान॒रम् जु॒होति॑ जु॒होति॑ वैश्वान॒रम् चि॒ते चि॒ते वै᳚श्वान॒रम् जु॒होति॑ । \newline
14. वै॒श्वा॒न॒रम् जु॒होति॑ जु॒होति॑ वैश्वान॒रं ॅवै᳚श्वान॒रम् जु॒होति॒ शरी॑रꣳ॒॒ शरी॑रम् जु॒होति॑ वैश्वान॒रं ॅवै᳚श्वान॒रम् जु॒होति॒ शरी॑रम् । \newline
15. जु॒होति॒ शरी॑रꣳ॒॒ शरी॑रम् जु॒होति॑ जु॒होति॒ शरी॑र मे॒वैव शरी॑रम् जु॒होति॑ जु॒होति॒ शरी॑र मे॒व । \newline
16. शरी॑र मे॒वैव शरी॑रꣳ॒॒ शरी॑र मे॒व सꣳ॒॒स्कृत्य॑ सꣳ॒॒स्कृ त्यै॒व शरी॑रꣳ॒॒ शरी॑र मे॒व सꣳ॒॒स्कृत्य॑ । \newline
17. ए॒व सꣳ॒॒स्कृत्य॑ सꣳ॒॒स्कृ त्यै॒वैव सꣳ॒॒स्कृ त्या॒त्मना॒ ऽऽत्मना॑ सꣳ॒॒स्कृ त्यै॒वैव 
सꣳ॒॒स्कृ त्या॒त्मना᳚ । \newline
18. सꣳ॒॒स्कृ त्या॒त्मना॒ ऽऽत्मना॑ सꣳ॒॒स्कृत्य॑ सꣳ॒॒स्कृ त्या॒त्मना॒ ऽभ्यारो॑ह त्य॒भ्यारो॑ह त्या॒त्मना॑ सꣳ॒॒स्कृ त्य॑सꣳ॒॒स्कृ त्या॒त्मना॒ ऽभ्यारो॑हति । \newline
19. आ॒त्मना॒ ऽभ्यारो॑ह त्य॒भ्यारो॑ह त्या॒त्मना॒ ऽऽत्मना॒ ऽभ्यारो॑हति॒ तस्मा॒त् तस्मा॑ द॒भ्यारो॑ह त्या॒त्मना॒ ऽऽत्मना॒ ऽभ्यारो॑हति॒ तस्मा᳚त् । \newline
20. अ॒भ्यारो॑हति॒ तस्मा॒त् तस्मा॑ द॒भ्यारो॑ह त्य॒भ्यारो॑हति॒ तस्मा॒त् तस्य॒ तस्य॒ तस्मा॑ द॒भ्यारो॑ह त्य॒भ्यारो॑हति॒ तस्मा॒त् तस्य॑ । \newline
21. अ॒भ्यारो॑ह॒तीत्य॑भि - आरो॑हति । \newline
22. तस्मा॒त् तस्य॒ तस्य॒ तस्मा॒त् तस्मा॒त् तस्य॒ न न तस्य॒ तस्मा॒त् तस्मा॒त् तस्य॒ न । \newline
23. तस्य॒ न न तस्य॒ तस्य॒ नावाव॒ न तस्य॒ तस्य॒ नाव॑ । \newline
24. नावाव॒ न नाव॑ द्यन्ति द्य॒न् त्यव॒ न नाव॑ द्यन्ति । \newline
25. अव॑ द्यन्ति द्य॒न् त्यवाव॑ द्यन्ति॒ जीव॒न् जीव॑न् द्य॒न् त्यवाव॑ द्यन्ति॒ जीवन्न्॑ । \newline
26. द्य॒न्ति॒ जीव॒न् जीव॑न् द्यन्ति द्यन्ति॒ जीव॑न् ने॒वैव जीव॑न् द्यन्ति द्यन्ति॒ जीव॑न् ने॒व । \newline
27. जीव॑न् ने॒वैव जीव॒न् जीव॑न् ने॒व दे॒वान् दे॒वा ने॒व जीव॒न् जीव॑न् ने॒व दे॒वान् । \newline
28. ए॒व दे॒वान् दे॒वा ने॒वैव दे॒वा नप्यपि॑ दे॒वा ने॒वैव दे॒वा नपि॑ । \newline
29. दे॒वा नप्यपि॑ दे॒वान् दे॒वा नप्ये᳚ त्ये॒त्यपि॑ दे॒वान् दे॒वा नप्ये॑ति । \newline
30. अप्ये᳚ त्ये॒त्यप्य प्ये॑ति वैश्वान॒र्या वै᳚श्वान॒र् यैत्यप्य प्ये॑ति वैश्वान॒र्या । \newline
31. ए॒ति॒ वै॒श्वा॒न॒र्या वै᳚श्वान॒र् यैत्ये॑ति वैश्वान॒ र्‌यर्च र्‌चा वै᳚श्वान॒र्यै त्ये॑ति वैश्वान॒ र्‌यर्चा । \newline
32. वै॒श्वा॒न॒र्य र्‌च र्‌चा वै᳚श्वान॒र्या वै᳚श्वान॒र्य र्‌चा पुरी॑ष॒म् पुरी॑ष मृ॒चा वै᳚श्वान॒र्या वै᳚श्वान॒र्य र्‌चा पुरी॑षम् । \newline
33. ऋ॒चा पुरी॑ष॒म् पुरी॑ष मृ॒च र्‌चा पुरी॑ष॒ मुपोप॒ पुरी॑ष मृ॒च र्‌चा पुरी॑ष॒ मुप॑ । \newline
34. पुरी॑ष॒ मुपोप॒ पुरी॑ष॒म् पुरी॑ष॒ मुप॑ दधाति दधा॒ त्युप॒ पुरी॑ष॒म् पुरी॑ष॒ मुप॑ दधाति । \newline
35. उप॑ दधाति दधा॒ त्युपोप॑ दधाती॒य मि॒यम् द॑धा॒ त्युपोप॑ दधाती॒यम् । \newline
36. द॒धा॒ती॒य मि॒यम् द॑धाति दधाती॒यं ॅवै वा इ॒यम् द॑धाति दधाती॒यं ॅवै । \newline
37. इ॒यं ॅवै वा इ॒य मि॒यं ॅवा अ॒ग्नि र॒ग्निर् वा इ॒य मि॒यं ॅवा अ॒ग्निः । \newline
38. वा अ॒ग्नि र॒ग्निर् वै वा अ॒ग्निर् वै᳚श्वान॒रो वै᳚श्वान॒रो᳚ ऽग्निर् वै वा अ॒ग्निर् वै᳚श्वान॒रः । \newline
39. अ॒ग्निर् वै᳚श्वान॒रो वै᳚श्वान॒रो᳚ ऽग्नि र॒ग्निर् वै᳚श्वान॒र स्तस्य॒ तस्य॑ वैश्वान॒रो᳚ ऽग्नि र॒ग्निर् वै᳚श्वान॒र स्तस्य॑ । \newline
40. वै॒श्वा॒न॒र स्तस्य॒ तस्य॑ वैश्वान॒रो वै᳚श्वान॒र स्तस्यै॒षैषा तस्य॑ वैश्वान॒रो वै᳚श्वान॒र स्तस्यै॒षा । \newline
41. तस्यै॒षैषा तस्य॒ तस्यै॒षा चिति॒ श्चिति॑ रे॒षा तस्य॒ तस्यै॒षा चितिः॑ । \newline
42. ए॒षा चिति॒ श्चिति॑ रे॒षैषा चिति॒र् यद् यच् चिति॑ रे॒षैषा चिति॒र् यत् । \newline
43. चिति॒र् यद् यच् चिति॒ श्चिति॒र् यत् पुरी॑ष॒म् पुरी॑षं॒ ॅयच् चिति॒ श्चिति॒र् यत् पुरी॑षम् । \newline
44. यत् पुरी॑ष॒म् पुरी॑षं॒ ॅयद् यत् पुरी॑ष म॒ग्नि म॒ग्निम् पुरी॑षं॒ ॅयद् यत् पुरी॑ष म॒ग्निम् । \newline
45. पुरी॑ष म॒ग्नि म॒ग्निम् पुरी॑ष॒म् पुरी॑ष म॒ग्नि मे॒वै वाग्निम् पुरी॑ष॒म् पुरी॑ष म॒ग्नि मे॒व । \newline
46. अ॒ग्नि मे॒वै वाग्नि म॒ग्नि मे॒व वै᳚श्वान॒रं ॅवै᳚श्वान॒र मे॒वाग्नि म॒ग्नि मे॒व वै᳚श्वान॒रम् । \newline
47. ए॒व वै᳚श्वान॒रं ॅवै᳚श्वान॒र मे॒वैव वै᳚श्वान॒रम् चि॑नुते चिनुते वैश्वान॒र मे॒वैव वै᳚श्वान॒रम् चि॑नुते । \newline
48. वै॒श्वा॒न॒रम् चि॑नुते चिनुते वैश्वान॒रं ॅवै᳚श्वान॒रम् चि॑नुत ए॒षैषा चि॑नुते वैश्वान॒रं ॅवै᳚श्वान॒रम् चि॑नुत ए॒षा । \newline
49. चि॒नु॒त॒ ए॒षैषा चि॑नुते चिनुत ए॒षा वै वा ए॒षा चि॑नुते चिनुत ए॒षा वै । \newline
50. ए॒षा वै वा ए॒षैषा वा अ॒ग्ने र॒ग्नेर् वा ए॒षैषा वा अ॒ग्नेः । \newline
51. वा अ॒ग्ने र॒ग्नेर् वै वा अ॒ग्नेः प्रि॒या प्रि॒या ऽग्नेर् वै वा अ॒ग्नेः प्रि॒या । \newline
52. अ॒ग्नेः प्रि॒या प्रि॒या ऽग्ने र॒ग्नेः प्रि॒या त॒नू स्त॒नूः प्रि॒या ऽग्ने र॒ग्नेः प्रि॒या त॒नूः । \newline
53. प्रि॒या त॒नू स्त॒नूः प्रि॒या प्रि॒या त॒नूर् यद् यत् त॒नूः प्रि॒या प्रि॒या त॒नूर् यत् । \newline
54. त॒नूर् यद् यत् त॒नू स्त॒नूर् यद् वै᳚श्वान॒रो वै᳚श्वान॒रो यत् त॒नू स्त॒नूर् यद् वै᳚श्वान॒रः । \newline
55. यद् वै᳚श्वान॒रो वै᳚श्वान॒रो यद् यद् वै᳚श्वान॒रः प्रि॒याम् प्रि॒यां ॅवै᳚श्वान॒रो यद् यद् वै᳚श्वान॒रः प्रि॒याम् । \newline
56. वै॒श्वा॒न॒रः प्रि॒याम् प्रि॒यां ॅवै᳚श्वान॒रो वै᳚श्वान॒रः प्रि॒या मे॒वैव प्रि॒यां ॅवै᳚श्वान॒रो वै᳚श्वान॒रः प्रि॒या मे॒व । \newline
57. प्रि॒या मे॒वैव प्रि॒याम् प्रि॒या मे॒वास्या᳚ स्यै॒व प्रि॒याम् प्रि॒या मे॒वास्य॑ । \newline
58. ए॒वास्या᳚ स्यै॒वै वास्य॑ त॒नुव॑म् त॒नुव॑ मस्यै॒ वैवास्य॑ त॒नुव᳚म् । \newline
59. अ॒स्य॒ त॒नुव॑म् त॒नुव॑ मस्यास्य त॒नुव॒ मवाव॑ त॒नुव॑ मस्यास्य त॒नुव॒ मव॑ । \newline
60. त॒नुव॒ मवाव॑ त॒नुव॑म् त॒नुव॒ मव॑ रुन्धे रु॒न्धे ऽव॑ त॒नुव॑म् त॒नुव॒ मव॑ रुन्धे । \newline
61. अव॑ रुन्धे रु॒न्धे ऽवाव॑ रुन्धे । \newline
62. रु॒न्ध॒ इति॑ रुन्धे । \newline
\pagebreak
\markright{ TS 5.6.7.1  \hfill https://www.vedavms.in \hfill}

\section{ TS 5.6.7.1 }

\textbf{TS 5.6.7.1 } \newline
\textbf{Samhita Paata} \newline

अ॒ग्नेर्वै दी॒क्षया॑ दे॒वा वि॒राज॑माप्नुवन् ति॒स्रो रात्री᳚र्दीक्षि॒तः स्या᳚त् त्रि॒पदा॑ वि॒राड् वि॒राज॑माप्नोति॒ षड्-रात्री᳚र्दीक्षि॒तः स्या॒थ् षड् वा ऋ॒तवः॑ संॅवथ्स॒रः सं॑ॅवथ्स॒रो वि॒राड् वि॒राज॑माप्नोति॒ दश॒ रात्री᳚र्दीक्षि॒तः स्या॒द्-दशा᳚क्षरा वि॒राड् वि॒राज॑माप्नोति॒ द्वाद॑श॒ रात्री᳚र्दीक्षि॒तः स्या॒द् द्वाद॑श॒ मासाः᳚ संॅवथ्स॒रः सं॑ॅवथ्स॒रो वि॒राड् वि॒राज॑माप्नोति॒ त्रयो॑दश॒ रात्री᳚र्दीक्षि॒तः स्या॒त् त्रयो॑दश॒ - [  ] \newline

\textbf{Pada Paata} \newline

अ॒ग्नेः । वै । दी॒क्षया᳚ । दे॒वाः । वि॒राज॒मिति॑ वि-राज᳚म् । आ॒प्नु॒व॒न्न् । ति॒स्रः । रात्रीः᳚ । दी॒क्षि॒तः । स्या॒त् । त्रि॒पदेति॑ त्रि - पदा᳚ । वि॒राडिति॑ वि - राट् । वि॒राज॒मिति॑ वि - राज᳚म् । आ॒प्नो॒ति॒ । षट् । रात्रीः᳚ । दी॒क्षि॒तः । स्या॒त् । षट् । वै । ऋ॒तवः॑ । सं॒ॅव॒थ्स॒र इति॑ सं - व॒थ्स॒रः । सं॒ॅव॒थ्स॒र इति॑ सं - व॒थ्स॒रः । वि॒राडिति॑ वि - राट् । वि॒राज॒मिति॑ वि - राज᳚म् । आ॒प्नो॒ति॒ । दश॑ । रात्रीः᳚ । दी॒क्षि॒तः । स्या॒त् । दशा᳚क्ष॒रेति॒ दश॑ - अ॒क्ष॒रा॒ । वि॒राडिति॑ वि - राट् । वि॒राज॒मिति॑ वि - राज᳚म् । आ॒प्नो॒ति॒ । द्वाद॑श । रात्रीः᳚ । दी॒क्षि॒तः । स्या॒त् । द्वाद॑श । मासाः᳚ । सं॒ॅव॒थ्स॒र इति॑ सं - व॒थ्स॒रः । सं॒ॅव॒थ्स॒र इति॑ सं - व॒थ्स॒रः । वि॒राडिति॑ वि - राट् । वि॒राज॒मिति॑ वि - राज᳚म् । आ॒प्नो॒ति॒ । त्रयो॑द॒शेति॒ त्रयः॑ - द॒श॒ । रात्रीः᳚ । दी॒क्षि॒तः । स्या॒त् । त्रयो॑द॒शेति॒ त्रयः॑ - द॒श॒ ।  \newline


\textbf{Krama Paata} \newline

अ॒ग्नेर् वै । वै दी॒क्षया᳚ । दी॒क्षया॑ दे॒वाः । दे॒वा वि॒राज᳚म् । वि॒राज॑माप्नुवन्न् । वि॒राज॒मिति॑ वि - राज᳚म् । आ॒प्नु॒व॒न् ति॒स्रः । ति॒स्रो रात्रीः᳚ । रात्री᳚र् दीक्षि॒तः । दी॒क्षि॒तः स्या᳚त् । स्या॒त् त्रि॒पदा᳚ । त्रि॒पदा॑ वि॒राट् । त्रि॒पदेति॑ त्रि - पदा᳚ । वि॒राड् वि॒राज᳚म् । वि॒राडिति॑ वि - राट् । वि॒राज॑माप्नोति । वि॒राज॒मिति॑ वि - राज᳚म् । आ॒प्नो॒ति॒ षट् । षड् रात्रीः᳚ । रात्री᳚र् दीक्षि॒तः । दी॒क्षि॒तः स्या᳚त् । स्या॒त् षट् । षड् वै । वा ऋ॒तवः॑ । ऋ॒तवः॑ सम्ॅवथ्स॒रः । स॒म्ॅव॒थ्स॒रः स॑म्ॅवथ्स॒रः । स॒म्ॅव॒थ्स॒र इति॑ सम् - व॒थ्स॒रः । स॒म्ॅव॒थ्स॒रो वि॒राट् । स॒म्ॅव॒थ्स॒र इति॑ सम् - व॒थ्स॒रः । वि॒राड् वि॒राज᳚म् । वि॒राडिति॑ वि - राट् । वि॒राज॑माप्नोति । वि॒राज॒मिति॑ वि - राज᳚म् । आ॒प्नो॒ति॒ दश॑ । दश॒ रात्रीः᳚ । रात्री᳚र् दीक्षि॒तः । दी॒क्षि॒तः स्या᳚त् । स्या॒द् दशा᳚क्षरा । दशा᳚क्षरा वि॒राट् । दशा᳚क्ष॒रेति॒ दश॑ - अ॒क्ष॒रा॒ । वि॒राड् वि॒राज᳚म् । वि॒राडिति॑ वि - राट् । वि॒राज॑माप्नोति । वि॒राज॒मिति॑ वि - राज᳚म् । आ॒प्नो॒ति॒ द्वाद॑श । द्वाद॑श॒ रात्रीः᳚ । रात्री᳚र् दीक्षि॒तः । दी॒क्षि॒तः स्या᳚त् । स्या॒द् द्वाद॑श । द्वाद॑श॒ मासाः᳚ । मासाः᳚ सम्ॅवथ्स॒रः । स॒म्ॅव॒थ्स॒रः स॑म्ॅवथ्स॒रः । स॒म्ॅव॒थ्स॒र इति॑ सम् - व॒थ्स॒रः । स॒म्ॅव॒थ्स॒रो वि॒राट् । स॒म्ॅव॒थ्स॒र इति॑ सम् - व॒थ्स॒रः । वि॒राड् वि॒राज᳚म् । वि॒राडिति॑ वि - राट् । वि॒राज॑माप्नोति । वि॒राज॒मिति॑ वि - राज᳚म् । आ॒प्नो॒ति॒ त्रयो॑दश । त्रयो॑दश॒ रात्रीः᳚ । त्रयो॑द॒शेति॒ त्रयः॑ - द॒श॒ । रात्री᳚र् दीक्षि॒तः । दी॒क्षि॒तः स्या᳚त् । स्या॒त् त्रयो॑दश । त्रयो॑दश॒ मासाः᳚ । त्रयो॑द॒शेति॒ त्रयः॑ - द॒श॒ \newline

\textbf{Jatai Paata} \newline

1. अ॒ग्नेर् वै वा अ॒ग्ने र॒ग्नेर् वै । \newline
2. वै दी॒क्षया॑ दी॒क्षया॒ वै वै दी॒क्षया᳚ । \newline
3. दी॒क्षया॑ दे॒वा दे॒वा दी॒क्षया॑ दी॒क्षया॑ दे॒वाः । \newline
4. दे॒वा वि॒राजं॑ ॅवि॒राज॑म् दे॒वा दे॒वा वि॒राज᳚म् । \newline
5. वि॒राज॑ माप्नुवन् नाप्नुवन् वि॒राजं॑ ॅवि॒राज॑ माप्नुवन्न् । \newline
6. वि॒राज॒मिति॑ वि - राज᳚म् । \newline
7. आ॒प्नु॒व॒न् ति॒स्र स्ति॒स्र आ᳚प्नुवन् नाप्नुवन् ति॒स्रः । \newline
8. ति॒स्रो रात्री॒ रात्री᳚ स्ति॒स्र स्ति॒स्रो रात्रीः᳚ । \newline
9. रात्री᳚र् दीक्षि॒तो दी᳚क्षि॒तो रात्री॒ रात्री᳚र् दीक्षि॒तः । \newline
10. दी॒क्षि॒तः स्या᳚थ् स्याद् दीक्षि॒तो दी᳚क्षि॒तः स्या᳚त् । \newline
11. स्या॒त् त्रि॒पदा᳚ त्रि॒पदा᳚ स्याथ् स्यात् त्रि॒पदा᳚ । \newline
12. त्रि॒पदा॑ वि॒राड् वि॒राट् त्रि॒पदा᳚ त्रि॒पदा॑ वि॒राट् । \newline
13. त्रि॒पदेति॑ त्रि - पदा᳚ । \newline
14. वि॒राड् वि॒राजं॑ ॅवि॒राजं॑ ॅवि॒राड् वि॒राड् वि॒राज᳚म् । \newline
15. वि॒राडिति॑ वि - राट् । \newline
16. वि॒राज॑ माप्नो त्याप्नोति वि॒राजं॑ ॅवि॒राज॑ माप्नोति । \newline
17. वि॒राज॒मिति॑ वि - राज᳚म् । \newline
18. आ॒प्नो॒ति॒ षट् थ्षडा᳚प्नो त्याप्नोति॒ षट् । \newline
19. षड् रात्री॒ रात्री॒ ष्षट् थ्षड् रात्रीः᳚ । \newline
20. रात्री᳚र् दीक्षि॒तो दी᳚क्षि॒तो रात्री॒ रात्री᳚र् दीक्षि॒तः । \newline
21. दी॒क्षि॒तः स्या᳚थ् स्याद् दीक्षि॒तो दी᳚क्षि॒तः स्या᳚त् । \newline
22. स्या॒थ् षट् थ्षट् थ्स्या᳚थ् स्या॒थ् षट् । \newline
23. षड् वै वै षट् थ्षड् वै । \newline
24. वा ऋ॒तव॑ ऋ॒तवो॒ वै वा ऋ॒तवः॑ । \newline
25. ऋ॒तवः॑ संॅवथ्स॒रः सं॑ॅवथ्स॒र ऋ॒तव॑ ऋ॒तवः॑ संॅवथ्स॒रः । \newline
26. सं॒ॅव॒थ्स॒रः सं॑ॅवथ्स॒रः । \newline
27. सं॒ॅव॒थ्स॒र इति॑ सं - व॒थ्स॒रः । \newline
28. सं॒ॅव॒थ्स॒रो वि॒राड् वि॒राट् थ्सं॑ॅवथ्स॒रः सं॑ॅवथ्स॒रो वि॒राट् । \newline
29. सं॒ॅव॒थ्स॒र इति॑ सं - व॒थ्स॒रः । \newline
30. वि॒राड् वि॒राजं॑ ॅवि॒राजं॑ ॅवि॒राड् वि॒राड् वि॒राज᳚म् । \newline
31. वि॒राडिति॑ वि - राट् । \newline
32. वि॒राज॑ माप्नो त्याप्नोति वि॒राजं॑ ॅवि॒राज॑ माप्नोति । \newline
33. वि॒राज॒मिति॑ वि - राज᳚म् । \newline
34. आ॒प्नो॒ति॒ दश॒ दशा᳚प्नो त्याप्नोति॒ दश॑ । \newline
35. दश॒ रात्री॒ रात्री॒र् दश॒ दश॒ रात्रीः᳚ । \newline
36. रात्री᳚र् दीक्षि॒तो दी᳚क्षि॒तो रात्री॒ रात्री᳚र् दीक्षि॒तः । \newline
37. दी॒क्षि॒तः स्या᳚थ् स्याद् दीक्षि॒तो दी᳚क्षि॒तः स्या᳚त् । \newline
38. स्या॒द् दशा᳚क्षरा॒ दशा᳚क्षरा स्याथ् स्या॒द् दशा᳚क्षरा । \newline
39. दशा᳚क्षरा वि॒राड् वि॒राड् दशा᳚क्षरा॒ दशा᳚क्षरा वि॒राट् । \newline
40. दशा᳚क्ष॒रेति॒ दश॑ - अ॒क्ष॒रा॒ । \newline
41. वि॒राड् वि॒राजं॑ ॅवि॒राजं॑ ॅवि॒राड् वि॒राड् वि॒राज᳚म् । \newline
42. वि॒राडिति॑ वि - राट् । \newline
43. वि॒राज॑ माप्नो त्याप्नोति वि॒राजं॑ ॅवि॒राज॑ माप्नोति । \newline
44. वि॒राज॒मिति॑ वि - राज᳚म् । \newline
45. आ॒प्नो॒ति॒ द्वाद॑श॒ द्वाद॑शाप्नो त्याप्नोति॒ द्वाद॑श । \newline
46. द्वाद॑श॒ रात्री॒ रात्री॒र् द्वाद॑श॒ द्वाद॑श॒ रात्रीः᳚ । \newline
47. रात्री᳚र् दीक्षि॒तो दी᳚क्षि॒तो रात्री॒ रात्री᳚र् दीक्षि॒तः । \newline
48. दी॒क्षि॒तः स्या᳚थ् स्याद् दीक्षि॒तो दी᳚क्षि॒तः स्या᳚त् । \newline
49. स्या॒द् द्वाद॑श॒ द्वाद॑श स्याथ् स्या॒द् द्वाद॑श । \newline
50. द्वाद॑श॒ मासा॒ मासा॒ द्वाद॑श॒ द्वाद॑श॒ मासाः᳚ । \newline
51. मासाः᳚ संॅवथ्स॒रः सं॑ॅवथ्स॒रो मासा॒ मासाः᳚ संॅवथ्स॒रः । \newline
52. सं॒ॅव॒थ्स॒रः सं॑ॅवथ्स॒रः । \newline
53. सं॒ॅव॒थ्स॒र इति॑ सं - व॒थ्स॒रः । \newline
54. सं॒ॅव॒थ्स॒रो वि॒राड् वि॒राट् थ्सं॑ॅवथ्स॒रः सं॑ॅवथ्स॒रो वि॒राट् । \newline
55. सं॒ॅव॒थ्स॒र इति॑ सं - व॒थ्स॒रः । \newline
56. वि॒राड् वि॒राजं॑ ॅवि॒राजं॑ ॅवि॒राड् वि॒राड् वि॒राज᳚म् । \newline
57. वि॒राडिति॑ वि - राट् । \newline
58. वि॒राज॑ माप्नो त्याप्नोति वि॒राजं॑ ॅवि॒राज॑ माप्नोति । \newline
59. वि॒राज॒मिति॑ वि - राज᳚म् । \newline
60. आ॒प्नो॒ति॒ त्रयो॑दश॒ त्रयो॑दशाप्नो त्याप्नोति॒ त्रयो॑दश । \newline
61. त्रयो॑दश॒ रात्री॒ रात्री॒ स्त्रयो॑दश॒ त्रयो॑दश॒ रात्रीः᳚ । \newline
62. त्रयो॑द॒शेति॒ त्रयः॑ - द॒श॒ । \newline
63. रात्री᳚र् दीक्षि॒तो दी᳚क्षि॒तो रात्री॒ रात्री᳚र् दीक्षि॒तः । \newline
64. दी॒क्षि॒तः स्या᳚थ् स्याद् दीक्षि॒तो दी᳚क्षि॒तः स्या᳚त् । \newline
65. स्या॒त् त्रयो॑दश॒ त्रयो॑दश स्याथ् स्या॒त् त्रयो॑दश । \newline
66. त्रयो॑दश॒ मासा॒ मासा॒ स्त्रयो॑दश॒ त्रयो॑दश॒ मासाः᳚ । \newline
67. त्रयो॑द॒शेति॒ त्रयः॑ - द॒श॒ । \newline

\textbf{Ghana Paata } \newline

1. अ॒ग्नेर् वै वा अ॒ग्ने र॒ग्नेर् वै दी॒क्षया॑ दी॒क्षया॒ वा अ॒ग्ने र॒ग्नेर् वै दी॒क्षया᳚ । \newline
2. वै दी॒क्षया॑ दी॒क्षया॒ वै वै दी॒क्षया॑ दे॒वा दे॒वा दी॒क्षया॒ वै वै दी॒क्षया॑ दे॒वाः । \newline
3. दी॒क्षया॑ दे॒वा दे॒वा दी॒क्षया॑ दी॒क्षया॑ दे॒वा वि॒राजं॑ ॅवि॒राज॑म् दे॒वा दी॒क्षया॑ दी॒क्षया॑ दे॒वा वि॒राज᳚म् । \newline
4. दे॒वा वि॒राजं॑ ॅवि॒राज॑म् दे॒वा दे॒वा वि॒राज॑ माप्नुवन् नाप्नुवन् वि॒राज॑म् दे॒वा दे॒वा वि॒राज॑ माप्नुवन्न् । \newline
5. वि॒राज॑ माप्नुवन् नाप्नुवन् वि॒राजं॑ ॅवि॒राज॑ माप्नुवन् ति॒स्र स्ति॒स्र आ᳚प्नुवन् वि॒राजं॑ ॅवि॒राज॑ माप्नुवन् ति॒स्रः । \newline
6. वि॒राज॒मिति॑ वि - राज᳚म् । \newline
7. आ॒प्नु॒व॒न् ति॒स्र स्ति॒स्र आ᳚प्नुवन् नाप्नुवन् ति॒स्रो रात्री॒ रात्री᳚ स्ति॒स्र आ᳚प्नुवन् नाप्नुवन् ति॒स्रो रात्रीः᳚ । \newline
8. ति॒स्रो रात्री॒ रात्री᳚ स्ति॒स्र स्ति॒स्रो रात्री᳚र् दीक्षि॒तो दी᳚क्षि॒तो रात्री᳚ स्ति॒स्र स्ति॒स्रो रात्री᳚र् दीक्षि॒तः । \newline
9. रात्री᳚र् दीक्षि॒तो दी᳚क्षि॒तो रात्री॒ रात्री᳚र् दीक्षि॒तः स्या᳚थ् स्याद् दीक्षि॒तो रात्री॒ रात्री᳚र् दीक्षि॒तः स्या᳚त् । \newline
10. दी॒क्षि॒तः स्या᳚थ् स्याद् दीक्षि॒तो दी᳚क्षि॒तः स्या᳚त् त्रि॒पदा᳚ त्रि॒पदा᳚ स्याद् दीक्षि॒तो दी᳚क्षि॒तः स्या᳚त् त्रि॒पदा᳚ । \newline
11. स्या॒त् त्रि॒पदा᳚ त्रि॒पदा᳚ स्याथ् स्यात् त्रि॒पदा॑ वि॒राड् वि॒राट् त्रि॒पदा᳚ स्याथ् स्यात् त्रि॒पदा॑ वि॒राट् । \newline
12. त्रि॒पदा॑ वि॒राड् वि॒राट् त्रि॒पदा᳚ त्रि॒पदा॑ वि॒राड् वि॒राजं॑ ॅवि॒राजं॑ ॅवि॒राट् त्रि॒पदा᳚ त्रि॒पदा॑ वि॒राड् वि॒राज᳚म् । \newline
13. त्रि॒पदेति॑ त्रि - पदा᳚ । \newline
14. वि॒राड् वि॒राजं॑ ॅवि॒राजं॑ ॅवि॒राड् वि॒राड् वि॒राज॑ माप्नो त्याप्नोति वि॒राजं॑ ॅवि॒राड् वि॒राड् वि॒राज॑ माप्नोति । \newline
15. वि॒राडिति॑ वि - राट् । \newline
16. वि॒राज॑ माप्नो त्याप्नोति वि॒राजं॑ ॅवि॒राज॑ माप्नोति॒ षट् थ्षडा᳚प्नोति वि॒राजं॑ ॅवि॒राज॑ माप्नोति॒ षट् । \newline
17. वि॒राज॒मिति॑ वि - राज᳚म् । \newline
18. आ॒प्नो॒ति॒ षट् थ्षडा᳚प्नो त्याप्नोति॒ षड् रात्री॒ रात्री॒ ष्षडा᳚प्नो त्याप्नोति॒ षड् रात्रीः᳚ । \newline
19. षड् रात्री॒ रात्री॒ ष्षट् थ्षड् रात्री᳚र् दीक्षि॒तो दी᳚क्षि॒तो रात्री॒ ष्षट् थ्षड् रात्री᳚र् दीक्षि॒तः । \newline
20. रात्री᳚र् दीक्षि॒तो दी᳚क्षि॒तो रात्री॒ रात्री᳚र् दीक्षि॒तः स्या᳚थ् स्याद् दीक्षि॒तो रात्री॒ रात्री᳚र् दीक्षि॒तः स्या᳚त् । \newline
21. दी॒क्षि॒तः स्या᳚थ् स्याद् दीक्षि॒तो दी᳚क्षि॒तः स्या॒थ् षट् थ्षट् थ्स्या᳚द् दीक्षि॒तो दी᳚क्षि॒तः स्या॒थ् षट् । \newline
22. स्या॒थ् षट् थ्षट् थ्स्या᳚थ् स्या॒थ् षड् वै वै षट् थ्स्या᳚थ् स्या॒थ् षड् वै । \newline
23. षड् वै वै षट् थ्षड् वा ऋ॒तव॑ ऋ॒तवो॒ वै षट् थ्षड् वा ऋ॒तवः॑ । \newline
24. वा ऋ॒तव॑ ऋ॒तवो॒ वै वा ऋ॒तवः॑ संॅवथ्स॒रः सं॑ॅवथ्स॒र ऋ॒तवो॒ वै वा ऋ॒तवः॑ संॅवथ्स॒रः । \newline
25. ऋ॒तवः॑ संॅवथ्स॒रः सं॑ॅवथ्स॒र ऋ॒तव॑ ऋ॒तवः॑ संॅवथ्स॒रः । \newline
26. सं॒ॅव॒थ्स॒रः सं॑ॅवथ्स॒रः । \newline
27. सं॒ॅव॒थ्स॒र इति॑ सं - व॒थ्स॒रः । \newline
28. सं॒ॅव॒थ्स॒रो वि॒राड् वि॒राट् थ्सं॑ॅवथ्स॒रः सं॑ॅवथ्स॒रो वि॒राड् वि॒राजं॑ ॅवि॒राजं॑ ॅवि॒राट् 
थ्सं॑ॅवथ्स॒रः सं॑ॅवथ्स॒रो वि॒राड् वि॒राज᳚म् । \newline
29. सं॒ॅव॒थ्स॒र इति॑ सं - व॒थ्स॒रः । \newline
30. वि॒राड् वि॒राजं॑ ॅवि॒राजं॑ ॅवि॒राड् वि॒राड् वि॒राज॑ माप्नो त्याप्नोति वि॒राजं॑ ॅवि॒राड् वि॒राड् वि॒राज॑ माप्नोति । \newline
31. वि॒राडिति॑ वि - राट् । \newline
32. वि॒राज॑ माप्नो त्याप्नोति वि॒राजं॑ ॅवि॒राज॑ माप्नोति॒ दश॒ दशा᳚प्नोति वि॒राजं॑ ॅवि॒राज॑ माप्नोति॒ दश॑ । \newline
33. वि॒राज॒मिति॑ वि - राज᳚म् । \newline
34. आ॒प्नो॒ति॒ दश॒ दशा᳚प्नो त्याप्नोति॒ दश॒ रात्री॒ रात्री॒र् दशा᳚प्नो त्याप्नोति॒ दश॒ रात्रीः᳚ । \newline
35. दश॒ रात्री॒ रात्री॒र् दश॒ दश॒ रात्री᳚र् दीक्षि॒तो दी᳚क्षि॒तो रात्री॒र् दश॒ दश॒ रात्री᳚र् दीक्षि॒तः । \newline
36. रात्री᳚र् दीक्षि॒तो दी᳚क्षि॒तो रात्री॒ रात्री᳚र् दीक्षि॒तः स्या᳚थ् स्याद् दीक्षि॒तो रात्री॒ रात्री᳚र् दीक्षि॒तः स्या᳚त् । \newline
37. दी॒क्षि॒तः स्या᳚थ् स्याद् दीक्षि॒तो दी᳚क्षि॒तः स्या॒द् दशा᳚क्षरा॒ दशा᳚क्षरा स्याद् दीक्षि॒तो दी᳚क्षि॒तः स्या॒द् दशा᳚क्षरा । \newline
38. स्या॒द् दशा᳚क्षरा॒ दशा᳚क्षरा स्याथ् स्या॒द् दशा᳚क्षरा वि॒राड् वि॒राड् दशा᳚क्षरा स्याथ् स्या॒द् दशा᳚क्षरा वि॒राट् । \newline
39. दशा᳚क्षरा वि॒राड् वि॒राड् दशा᳚क्षरा॒ दशा᳚क्षरा वि॒राड् वि॒राजं॑ ॅवि॒राजं॑ ॅवि॒राड् दशा᳚क्षरा॒ दशा᳚क्षरा वि॒राड् वि॒राज᳚म् । \newline
40. दशा᳚क्ष॒रेति॒ दश॑ - अ॒क्ष॒रा॒ । \newline
41. वि॒राड् वि॒राजं॑ ॅवि॒राजं॑ ॅवि॒राड् वि॒राड् वि॒राज॑ माप्नो त्याप्नोति वि॒राजं॑ ॅवि॒राड् वि॒राड् वि॒राज॑ माप्नोति । \newline
42. वि॒राडिति॑ वि - राट् । \newline
43. वि॒राज॑ माप्नो त्याप्नोति वि॒राजं॑ ॅवि॒राज॑ माप्नोति॒ द्वाद॑श॒ द्वाद॑शाप्नोति वि॒राजं॑ ॅवि॒राज॑ माप्नोति॒ द्वाद॑श । \newline
44. वि॒राज॒मिति॑ वि - राज᳚म् । \newline
45. आ॒प्नो॒ति॒ द्वाद॑श॒ द्वाद॑शप्नो त्याप्नोति॒ द्वाद॑श॒ रात्री॒ रात्री॒र् द्वाद॑शप्नो त्याप्नोति॒ द्वाद॑श॒ रात्रीः᳚ । \newline
46. द्वाद॑श॒ रात्री॒ रात्री॒र् द्वाद॑श॒ द्वाद॑श॒ रात्री᳚र् दीक्षि॒तो दी᳚क्षि॒तो रात्री॒र् द्वाद॑श॒ द्वाद॑श॒ रात्री᳚र् दीक्षि॒तः । \newline
47. रात्री᳚र् दीक्षि॒तो दी᳚क्षि॒तो रात्री॒ रात्री᳚र् दीक्षि॒तः स्या᳚थ् स्याद् दीक्षि॒तो रात्री॒ रात्री᳚र् दीक्षि॒तः स्या᳚त् । \newline
48. दी॒क्षि॒तः स्या᳚थ् स्याद् दीक्षि॒तो दी᳚क्षि॒तः स्या॒द् द्वाद॑श॒ द्वाद॑श स्याद् दीक्षि॒तो दी᳚क्षि॒तः स्या॒द् द्वाद॑श । \newline
49. स्या॒द् द्वाद॑श॒ द्वाद॑श स्याथ् स्या॒द् द्वाद॑श॒ मासा॒ मासा॒ द्वाद॑श स्याथ् स्या॒द् द्वाद॑श॒ मासाः᳚ । \newline
50. द्वाद॑श॒ मासा॒ मासा॒ द्वाद॑श॒ द्वाद॑श॒ मासाः᳚ संॅवथ्स॒रः सं॑ॅवथ्स॒रो मासा॒ द्वाद॑श॒ द्वाद॑श॒ मासाः᳚ संॅवथ्स॒रः । \newline
51. मासाः᳚ संॅवथ्स॒रः सं॑ॅवथ्स॒रो मासा॒ मासाः᳚ संॅवथ्स॒रः । \newline
52. सं॒ॅव॒थ्स॒रः सं॑ॅवथ्स॒रः । \newline
53. सं॒ॅव॒थ्स॒र इति॑ सं - व॒थ्स॒रः । \newline
54. सं॒ॅव॒थ्स॒रो वि॒राड् वि॒राट् थ्सं॑ॅवथ्स॒रः सं॑ॅवथ्स॒रो वि॒राड् वि॒राजं॑ ॅवि॒राजं॑ ॅवि॒राट् 
थ्सं॑ॅवथ्स॒रः सं॑ॅवथ्स॒रो वि॒राड् वि॒राज᳚म् । \newline
55. सं॒ॅव॒थ्स॒र इति॑ सं - व॒थ्स॒रः । \newline
56. वि॒राड् वि॒राजं॑ ॅवि॒राजं॑ ॅवि॒राड् वि॒राड् वि॒राज॑ माप्नो त्याप्नोति वि॒राजं॑ ॅवि॒राड् वि॒राड् वि॒राज॑ माप्नोति । \newline
57. वि॒राडिति॑ वि - राट् । \newline
58. वि॒राज॑ माप्नो त्याप्नोति वि॒राजं॑ ॅवि॒राज॑ माप्नोति॒ त्रयो॑दश॒ त्रयो॑द शाप्नोति वि॒राजं॑ ॅवि॒राज॑ माप्नोति॒ त्रयो॑दश । \newline
59. वि॒राज॒मिति॑ वि - राज᳚म् । \newline
60. आ॒प्नो॒ति॒ त्रयो॑दश॒ त्रयो॑दशाप्नो त्याप्नोति॒ त्रयो॑दश॒ रात्री॒ रात्री॒ स्त्रयो॑दशाप्नो त्याप्नोति॒ त्रयो॑दश॒ रात्रीः᳚ । \newline
61. त्रयो॑दश॒ रात्री॒ रात्री॒ स्त्रयो॑दश॒ त्रयो॑दश॒ रात्री᳚र् दीक्षि॒तो दी᳚क्षि॒तो रात्री॒ स्त्रयो॑दश॒ त्रयो॑दश॒ रात्री᳚र् दीक्षि॒तः । \newline
62. त्रयो॑द॒शेति॒ त्रयः॑ - द॒श॒ । \newline
63. रात्री᳚र् दीक्षि॒तो दी᳚क्षि॒तो रात्री॒ रात्री᳚र् दीक्षि॒तः स्या᳚थ् स्याद् दीक्षि॒तो रात्री॒ रात्री᳚र् दीक्षि॒तः स्या᳚त् । \newline
64. दी॒क्षि॒तः स्या᳚थ् स्याद् दीक्षि॒तो दी᳚क्षि॒तः स्या॒त् त्रयो॑दश॒ त्रयो॑दश स्याद् दीक्षि॒तो दी᳚क्षि॒तः स्या॒त् त्रयो॑दश । \newline
65. स्या॒त् त्रयो॑दश॒ त्रयो॑दश स्याथ् स्या॒त् त्रयो॑दश॒ मासा॒ मासा॒ स्त्रयो॑दश स्याथ् स्या॒त् त्रयो॑दश॒ मासाः᳚ । \newline
66. त्रयो॑दश॒ मासा॒ मासा॒ स्त्रयो॑दश॒ त्रयो॑दश॒ मासाः᳚ संॅवथ्स॒रः सं॑ॅवथ्स॒रो मासा॒ स्त्रयो॑दश॒ त्रयो॑दश॒ मासाः᳚ संॅवथ्स॒रः । \newline
67. त्रयो॑द॒शेति॒ त्रयः॑ - द॒श॒ । \newline
\pagebreak
\markright{ TS 5.6.7.2  \hfill https://www.vedavms.in \hfill}

\section{ TS 5.6.7.2 }

\textbf{TS 5.6.7.2 } \newline
\textbf{Samhita Paata} \newline

मासाः᳚ संॅवथ्स॒रः सं॑ॅवथ्स॒रो वि॒राड् वि॒राज॑माप्नोति॒ पञ्च॑दश॒ रात्री᳚र्दीक्षि॒तः स्या॒त् पञ्च॑दश॒ वा अ॑र्द्धमा॒सस्य॒ रात्र॑योऽर्द्धमास॒शः सं॑ॅवथ्स॒र आ᳚प्यते संॅवथ्स॒रो वि॒राड् वि॒राज॑माप्नोति स॒प्तद॑श॒ रात्री᳚र्दीक्षि॒तः स्या॒द् द्वाद॑श॒ मासाः॒ पञ्च॒र्तवः॒ स सं॑ॅवथ्स॒रः सं॑ॅवथ्स॒रो वि॒राड् वि॒राज॑माप्नोति॒ चतु॑र्विꣳशतिꣳ॒॒ रात्री᳚र्दीक्षि॒तः स्या॒-च्चतु॑र्विꣳशतिरर्द्धमा॒साः सं॑ॅवथ्स॒रः सं॑ॅवथ्स॒रो वि॒राड् वि॒राज॑माप्नोति त्रिꣳ॒॒शतꣳ॒॒ रात्री᳚र्दीक्षि॒तः स्या᳚त् - [  ] \newline

\textbf{Pada Paata} \newline

मासाः᳚ । सं॒ॅव॒थ्स॒र इति॑ सं-व॒थ्स॒रः । सं॒ॅव॒थ्स॒र इति॑ सं-व॒थ्स॒रः । वि॒राडिति॑ वि -   राट् । वि॒राज॒मिति॑ वि - राज᳚म् । आ॒प्नो॒ति॒ । पञ्च॑द॒शेति॒ पञ्च॑ - द॒श॒ । रात्रीः᳚ । दी॒क्षि॒तः । स्या॒त् । पञ्च॑द॒शेति॒ पञ्च॑ - द॒श॒ । वै । अ॒द्‌र्ध॒मा॒सस्येत्य॒॑द्‌र्ध - मा॒सस्य॑ ।   रात्र॑यः । अ॒द्‌र्ध॒मा॒स॒श इत्य॒॑द्‌र्धमास -   शः । सं॒ॅव॒थ्स॒र इति॑ सं - व॒थ्स॒रः । आ॒प्य॒ते॒ । सं॒ॅव॒थ्स॒र इति॑ सं - व॒थ्स॒रः । वि॒राडिति॑ वि - राट् । वि॒राज॒मिति॑ वि - राज᳚म् । आ॒प्नो॒ति॒ । स॒प्तद॒शेति॑ स॒प्त - द॒श॒ । रात्रीः᳚ । दी॒क्षि॒तः । स्या॒त् । द्वाद॑श । मासाः᳚ । पञ्च॑ । ऋ॒तवः॑ । सः । सं॒ॅव॒थ्स॒र इति॑ सं - व॒थ्स॒रः । सं॒ॅव॒थ्स॒र इति॑ सं - व॒थ्स॒रः । वि॒राडिति॑ वि - राट् । वि॒राज॒मिति॑ वि - राज᳚म् । आ॒प्नो॒ति॒ । चतु॑र्विꣳशति॒मिति॒ चतुः॑ - विꣳ॒॒श॒ति॒म् । रात्रीः᳚ । दी॒क्षि॒तः । स्या॒त् । चतु॑र्विꣳशति॒रिति॒ चतुः॑ - विꣳ॒॒श॒तिः॒ । अ॒द्‌र्ध॒मा॒सा इत्य॑द्‌र्ध-मा॒साः । सं॒ॅव॒थ्स॒र इति॑ सं - व॒थ्स॒रः । सं॒ॅव॒थ्स॒र इति॑ सं - व॒थ्स॒रः । वि॒राडिति॑ वि - राट् । वि॒राज॒मिति॑ वि - राज᳚म् । आ॒प्नो॒ति॒ । त्रिꣳ॒॒शत᳚म् । रात्रीः᳚ । दी॒क्षि॒तः । स्या॒त् ।  \newline


\textbf{Krama Paata} \newline

मासाः᳚ सम्ॅवथ्स॒रः । स॒म्ॅव॒थ्स॒रः स॑म्ॅवथ्स॒रः । स॒म्ॅव॒थ्स॒र इति॑ सम् - व॒थ्स॒रः । स॒म्ॅव॒थ्स॒रो वि॒राट् । स॒म्ॅव॒थ्स॒र इति॑ सम् - व॒थ्स॒रः । वि॒राड् वि॒राज᳚म् । वि॒राडिति॑ वि - राट् । वि॒राज॑माप्नोति । वि॒राज॒मिति॑ वि - राज᳚म् । आ॒प्नो॒ति॒ पञ्च॑दश । पञ्च॑दश॒ रात्रीः᳚ । पञ्च॑द॒शेति॒ पञ्च॑ - द॒श॒ । रात्री᳚र् दीक्षि॒तः । दी॒क्षि॒तः स्या᳚त् । स्या॒त् पञ्च॑दश । पञ्च॑दश॒ वै । पञ्च॑द॒शेति॒ पञ्च॑ - द॒श॒ । वा अ॑र्द्धमा॒सस्य॑ । अ॒र्द्ध॒मा॒सस्य॒ रात्र॑यः । अ॒र्द्ध॒मा॒सस्येत्य॑र्द्ध - मा॒सस्य॑ । रात्र॑योऽर्द्धमास॒शः । अ॒र्द्ध॒मा॒स॒शः स॑म्ॅवथ्स॒रः । अ॒र्द्ध॒मा॒स॒श इत्य॑र्द्धमास - शः । स॒म्ॅव॒थ्स॒र आ᳚प्यते । स॒म्ॅव॒थ्स॒र इति॑ सम् - व॒थ्स॒रः । आ॒प्य॒ते॒ स॒म्ॅव॒थ्स॒रः । स॒म्ॅव॒थ्स॒रो वि॒राट् । स॒म्ॅव॒थ्स॒र इति॑ सम् - व॒थ्स॒रः । वि॒राड् वि॒राज᳚म् । वि॒राडिति॑ वि - राट् । वि॒राज॑माप्नोति । वि॒राज॒मिति॑ वि - राज᳚म् । आ॒प्नो॒ति॒ स॒प्तद॑श । स॒प्तद॑श॒ रात्रीः᳚ । स॒प्तद॒शेति॑ स॒प्त - द॒श॒ । रात्री᳚र् दीक्षि॒तः । दी॒क्षि॒तः स्या᳚त् । स्या॒द् द्वाद॑श । द्वाद॑श॒ मासाः᳚ । मासाः॒ पञ्च॑ । पञ्च॒र्तवः॑ । ऋ॒तवः॒ सः । स स॑म्ॅवथ्स॒रः । स॒म्ॅव॒थ्स॒रः स॑म्ॅवथ्स॒रः । स॒म्ॅव॒थ्स॒र इति॑ सम् - व॒थ्स॒रः । स॒म्ॅव॒थ्स॒रो वि॒राट् । स॒म्ॅव॒थ्स॒र इति॑ सम् - व॒थ्स॒रः । वि॒राड् वि॒राज᳚म् । वि॒राडिति॑ वि - राट् । वि॒राज॑माप्नोति । वि॒राज॒मिति॑ वि - राज᳚म् । आ॒प्नो॒ति॒ चतु॑र्विꣳशतिम् । चतु॑र्विꣳशतिꣳ॒॒ रात्रीः᳚ । चतु॑र्विꣳशति॒मिति॒ चतुः॑ - विꣳ॒॒श॒ति॒म् । रात्री᳚र् दीक्षि॒तः । दी॒क्षि॒तः स्या᳚त् । स्या॒च्चतु॑र्विꣳशतिः । चतु॑र्विꣳशतिरर्द्धमा॒साः । चतु॑र्विꣳशति॒रिति॒ चतुः॑ - विꣳ॒॒श॒तिः॒ । अ॒र्द्ध॒मा॒साः स॑म्ॅवथ्स॒रः । अ॒र्द्ध॒मा॒सा इत्य॑र्द्ध - मा॒साः । स॒म्ॅव॒थ्स॒रः स॑म्ॅवथ्स॒रः । स॒म्ॅव॒थ्स॒र इति॑ सम् - व॒थ्स॒रः । स॒म्ॅव॒थ्स॒रो वि॒राट् । स॒म्ॅव॒थ्स॒र इति॑ सम् - व॒थ्स॒रः । वि॒राड् वि॒राज᳚म् । वि॒राडिति॑ वि - राट् । वि॒राज॑माप्नोति । वि॒राज॒मिति॑ वि - राज᳚म् । आ॒प्नो॒ति॒ त्रिꣳ॒॒शत᳚म् । त्रिꣳ॒॒शतꣳ॒॒ रात्रीः᳚ । रात्री᳚र् दीक्षि॒तः । दी॒क्षि॒तः स्या᳚त् । स्या॒त् त्रिꣳ॒॒शद॑क्षरा \newline

\textbf{Jatai Paata} \newline

1. मासाः᳚ संॅवथ्स॒रः सं॑ॅवथ्स॒रो मासा॒ मासाः᳚ संॅवथ्स॒रः । \newline
2. सं॒ॅव॒थ्स॒रः सं॑ॅवथ्स॒रः । \newline
3. सं॒ॅव॒थ्स॒र इति॑ सं - व॒थ्स॒रः । \newline
4. सं॒ॅव॒थ्स॒रो वि॒राड् वि॒राट् थ्सं॑ॅवथ्स॒रः सं॑ॅवथ्स॒रो वि॒राट् । \newline
5. सं॒ॅव॒थ्स॒र इति॑ सं - व॒थ्स॒रः । \newline
6. वि॒राड् वि॒राजं॑ ॅवि॒राजं॑ ॅवि॒राड् वि॒राड् वि॒राज᳚म् । \newline
7. वि॒राडिति॑ वि - राट् । \newline
8. वि॒राज॑ माप्नो त्याप्नोति वि॒राजं॑ ॅवि॒राज॑ माप्नोति । \newline
9. वि॒राज॒मिति॑ वि - राज᳚म् । \newline
10. आ॒प्नो॒ति॒ पञ्च॑दश॒ पञ्च॑दशाप्नो त्याप्नोति॒ पञ्च॑दश । \newline
11. पञ्च॑दश॒ रात्री॒ रात्रीः॒ पञ्च॑दश॒ पञ्च॑दश॒ रात्रीः᳚ । \newline
12. पञ्च॑द॒शेति॒ पञ्च॑ - द॒श॒ । \newline
13. रात्री᳚र् दीक्षि॒तो दी᳚क्षि॒तो रात्री॒ रात्री᳚र् दीक्षि॒तः । \newline
14. दी॒क्षि॒तः स्या᳚थ् स्याद् दीक्षि॒तो दी᳚क्षि॒तः स्या᳚त् । \newline
15. स्या॒त् पञ्च॑दश॒ पञ्च॑दश स्याथ् स्या॒त् पञ्च॑दश । \newline
16. पञ्च॑दश॒ वै वै पञ्च॑दश॒ पञ्च॑दश॒ वै । \newline
17. पञ्च॑द॒शेति॒ पञ्च॑ - द॒श॒ । \newline
18. वा अ॑र्द्धमा॒सस्या᳚ र्द्धमा॒सस्य॒ वै वा अ॑र्द्धमा॒सस्य॑ । \newline
19. अ॒र्द्ध॒मा॒सस्य॒ रात्र॑यो॒ रात्र॑यो ऽर्द्धमा॒सस्या᳚ र्द्धमा॒सस्य॒ रात्र॑यः । \newline
20. अ॒र्द्ध॒मा॒सस्येत्य॑र्द्ध - मा॒सस्य॑ । \newline
21. रात्र॑यो ऽर्द्धमास॒शो᳚ ऽर्द्धमास॒शो रात्र॑यो॒ रात्र॑यो ऽर्द्धमास॒शः । \newline
22. अ॒र्द्ध॒मा॒स॒शः सं॑ॅवथ्स॒रः सं॑ॅवथ्स॒रो᳚ ऽर्द्धमास॒शो᳚ ऽर्द्धमास॒शः सं॑ॅवथ्स॒रः । \newline
23. अ॒र्द्ध॒मा॒स॒श इत्य॑र्द्धमास - शः । \newline
24. सं॒ॅव॒थ्स॒र आ᳚प्यत आप्यते संॅवथ्स॒रः सं॑ॅवथ्स॒र आ᳚प्यते । \newline
25. सं॒ॅव॒थ्स॒र इति॑ सं - व॒थ्स॒रः । \newline
26. आ॒प्य॒ते॒ सं॒ॅव॒थ्स॒रः सं॑ॅवथ्स॒र आ᳚प्यत आप्यते संॅवथ्स॒रः । \newline
27. सं॒ॅव॒थ्स॒रो वि॒राड् वि॒राट् थ् सं॑ॅवथ्स॒रः सं॑ॅवथ्स॒रो वि॒राट् । \newline
28. सं॒ॅव॒थ्स॒र इति॑ सं - व॒थ्स॒रः । \newline
29. वि॒राड् वि॒राजं॑ ॅवि॒राजं॑ ॅवि॒राड् वि॒राड् वि॒राज᳚म् । \newline
30. वि॒राडिति॑ वि - राट् । \newline
31. वि॒राज॑ माप्नो त्याप्नोति वि॒राजं॑ ॅवि॒राज॑ माप्नोति । \newline
32. वि॒राज॒मिति॑ वि - राज᳚म् । \newline
33. आ॒प्नो॒ति॒ स॒प्तद॑श स॒प्तद॑शाप्नो त्याप्नोति स॒प्तद॑श । \newline
34. स॒प्तद॑श॒ रात्री॒ रात्रीः᳚ स॒प्तद॑श स॒प्तद॑श॒ रात्रीः᳚ । \newline
35. स॒प्तद॒शेति॑ स॒प्त - द॒श॒ । \newline
36. रात्री᳚र् दीक्षि॒तो दी᳚क्षि॒तो रात्री॒ रात्री᳚र् दीक्षि॒तः । \newline
37. दी॒क्षि॒तः स्या᳚थ् स्याद् दीक्षि॒तो दी᳚क्षि॒तः स्या᳚त् । \newline
38. स्या॒द् द्वाद॑श॒ द्वाद॑श स्याथ् स्या॒द् द्वाद॑श । \newline
39. द्वाद॑श॒ मासा॒ मासा॒ द्वाद॑श॒ द्वाद॑श॒ मासाः᳚ । \newline
40. मासाः॒ पञ्च॒ पञ्च॒ मासा॒ मासाः॒ पञ्च॑ । \newline
41. पञ्च॒ र्‌तव॑ ऋ॒तवः॒ पञ्च॒ पञ्च॒ र्‌तवः॑ । \newline
42. ऋ॒तवः॒ स स ऋ॒तव॑ ऋ॒तवः॒ सः । \newline
43. स सं॑ॅवथ्स॒रः सं॑ॅवथ्स॒रः स स सं॑ॅवथ्स॒रः । \newline
44. सं॒ॅव॒थ्स॒रः सं॑ॅवथ्स॒रः । \newline
45. सं॒ॅव॒थ्स॒र इति॑ सं - व॒थ्स॒रः । \newline
46. सं॒ॅव॒थ्स॒रो वि॒राड् वि॒राट् थ्सं॑ॅवथ्स॒रः सं॑ॅवथ्स॒रो वि॒राट् । \newline
47. सं॒ॅव॒थ्स॒र इति॑ सं - व॒थ्स॒रः । \newline
48. वि॒राड् वि॒राजं॑ ॅवि॒राजं॑ ॅवि॒राड् वि॒राड् वि॒राज᳚म् । \newline
49. वि॒राडिति॑ वि - राट् । \newline
50. वि॒राज॑ माप्नो त्याप्नोति वि॒राजं॑ ॅवि॒राज॑ माप्नोति । \newline
51. वि॒राज॒मिति॑ वि - राज᳚म् । \newline
52. आ॒प्नो॒ति॒ चतु॑र्विꣳशति॒म् चतु॑र्विꣳशति माप्नो त्याप्नोति॒ चतु॑र्विꣳशतिम् । \newline
53. चतु॑र्विꣳशतिꣳ॒॒ रात्री॒ रात्री॒ श्चतु॑र्विꣳशति॒म् चतु॑र्विꣳशतिꣳ॒॒ रात्रीः᳚ । \newline
54. चतु॑र्विꣳशति॒मिति॒ चतुः॑ - विꣳ॒॒श॒ति॒म् । \newline
55. रात्री᳚र् दीक्षि॒तो दी᳚क्षि॒तो रात्री॒ रात्री᳚र् दीक्षि॒तः । \newline
56. दी॒क्षि॒तः स्या᳚थ् स्याद् दीक्षि॒तो दी᳚क्षि॒तः स्या᳚त् । \newline
57. स्या॒च् चतु॑र्विꣳशति॒ श्चतु॑र्विꣳशति स्याथ् स्या॒च् चतु॑र्विꣳशतिः । \newline
58. चतु॑र्विꣳशति रर्द्धमा॒सा अ॑र्द्धमा॒सा श्चतु॑र्विꣳशति॒ श्चतु॑र्विꣳशति रर्द्धमा॒साः । \newline
59. चतु॑र्विꣳशति॒रिति॒ चतुः॑ - विꣳ॒॒श॒तिः॒ । \newline
60. अ॒र्द्ध॒मा॒साः सं॑ॅवथ्स॒रः सं॑ॅवथ्स॒रो᳚ ऽर्द्धमा॒सा अ॑र्द्धमा॒साः सं॑ॅवथ्स॒रः । \newline
61. अ॒र्द्ध॒मा॒सा इत्य॑र्द्ध - मा॒साः । \newline
62. सं॒ॅव॒थ्स॒रः सं॑ॅवथ्स॒रः । \newline
63. सं॒ॅव॒थ्स॒र इति॑ सं - व॒थ्स॒रः । \newline
64. सं॒ॅव॒थ्स॒रो वि॒राड् वि॒राट् थ्सं॑ॅवथ्स॒रः सं॑ॅवथ्स॒रो वि॒राट् । \newline
65. सं॒ॅव॒थ्स॒र इति॑ सं - व॒थ्स॒रः । \newline
66. वि॒राड् वि॒राजं॑ ॅवि॒राजं॑ ॅवि॒राड् वि॒राड् वि॒राज᳚म् । \newline
67. वि॒राडिति॑ वि - राट् । \newline
68. वि॒राज॑ माप्नो त्याप्नोति वि॒राजं॑ ॅवि॒राज॑ माप्नोति । \newline
69. वि॒राज॒मिति॑ वि - राज᳚म् । \newline
70. आ॒प्नो॒ति॒ त्रिꣳ॒॒शत॑म् त्रिꣳ॒॒शत॑ माप्नो त्याप्नोति त्रिꣳ॒॒शत᳚म् । \newline
71. त्रिꣳ॒॒शतꣳ॒॒ रात्री॒ रात्री᳚ स्त्रिꣳ॒॒शत॑म् त्रिꣳ॒॒शतꣳ॒॒ रात्रीः᳚ । \newline
72. रात्री᳚र् दीक्षि॒तो दी᳚क्षि॒तो रात्री॒ रात्री᳚र् दीक्षि॒तः । \newline
73. दी॒क्षि॒तः स्या᳚थ् स्याद् दीक्षि॒तो दी᳚क्षि॒तः स्या᳚त् । \newline
74. स्या॒त् त्रिꣳ॒॒शद॑क्षरा त्रिꣳ॒॒शद॑क्षरा स्याथ् स्यात् त्रिꣳ॒॒शद॑क्षरा । \newline

\textbf{Ghana Paata } \newline

1. मासाः᳚ संॅवथ्स॒रः सं॑ॅवथ्स॒रो मासा॒ मासाः᳚ संॅवथ्स॒रः । \newline
2. सं॒ॅव॒थ्स॒रः सं॑ॅवथ्स॒रः । \newline
3. सं॒ॅव॒थ्स॒र इति॑ सं - व॒थ्स॒रः । \newline
4. सं॒ॅव॒थ्स॒रो वि॒राड् वि॒राट् थ्सं॑ॅवथ्स॒रः सं॑ॅवथ्स॒रो वि॒राड् वि॒राजं॑ ॅवि॒राजं॑ ॅवि॒राट् 
थ्सं॑ॅवथ्स॒रः सं॑ॅवथ्स॒रो वि॒राड् वि॒राज᳚म् । \newline
5. सं॒ॅव॒थ्स॒र इति॑ सं - व॒थ्स॒रः । \newline
6. वि॒राड् वि॒राजं॑ ॅवि॒राजं॑ ॅवि॒राड् वि॒राड् वि॒राज॑ माप्नो त्याप्नोति वि॒राजं॑ ॅवि॒राड् वि॒राड् वि॒राज॑ माप्नोति । \newline
7. वि॒राडिति॑ वि - राट् । \newline
8. वि॒राज॑ माप्नो त्याप्नोति वि॒राजं॑ ॅवि॒राज॑ माप्नोति॒ पञ्च॑दश॒ पञ्च॑दशाप्नोति वि॒राजं॑ ॅवि॒राज॑ माप्नोति॒ पञ्च॑दश । \newline
9. वि॒राज॒मिति॑ वि - राज᳚म् । \newline
10. आ॒प्नो॒ति॒ पञ्च॑दश॒ पञ्च॑दशाप्नो त्याप्नोति॒ पञ्च॑दश॒ रात्री॒ रात्रीः॒ पञ्च॑दशाप्नो त्याप्नोति॒ पञ्च॑दश॒ रात्रीः᳚ । \newline
11. पञ्च॑दश॒ रात्री॒ रात्रीः॒ पञ्च॑दश॒ पञ्च॑दश॒ रात्री᳚र् दीक्षि॒तो दी᳚क्षि॒तो रात्रीः॒ पञ्च॑दश॒ पञ्च॑दश॒ रात्री᳚र् दीक्षि॒तः । \newline
12. पञ्च॑द॒शेति॒ पञ्च॑ - द॒श॒ । \newline
13. रात्री᳚र् दीक्षि॒तो दी᳚क्षि॒तो रात्री॒ रात्री᳚र् दीक्षि॒तः स्या᳚थ् स्याद् दीक्षि॒तो रात्री॒ रात्री᳚र् दीक्षि॒तः स्या᳚त् । \newline
14. दी॒क्षि॒तः स्या᳚थ् स्याद् दीक्षि॒तो दी᳚क्षि॒तः स्या॒त् पञ्च॑दश॒ पञ्च॑दश स्याद् दीक्षि॒तो दी᳚क्षि॒तः स्या॒त् पञ्च॑दश । \newline
15. स्या॒त् पञ्च॑दश॒ पञ्च॑दश स्याथ् स्या॒त् पञ्च॑दश॒ वै वै पञ्च॑दश स्याथ् स्या॒त् पञ्च॑दश॒ वै । \newline
16. पञ्च॑दश॒ वै वै पञ्च॑दश॒ पञ्च॑दश॒ वा अ॑र्द्धमा॒सस्या᳚ र्द्धमा॒सस्य॒ वै पञ्च॑दश॒ पञ्च॑दश॒ वा अ॑र्द्धमा॒सस्य॑ । \newline
17. पञ्च॑द॒शेति॒ पञ्च॑ - द॒श॒ । \newline
18. वा अ॑र्द्धमा॒सस्या᳚ र्द्धमा॒सस्य॒ वै वा अ॑र्द्धमा॒सस्य॒ रात्र॑यो॒ रात्र॑यो ऽर्द्धमा॒सस्य॒ वै वा अ॑र्द्धमा॒सस्य॒ रात्र॑यः । \newline
19. अ॒र्द्ध॒मा॒सस्य॒ रात्र॑यो॒ रात्र॑यो ऽर्द्धमा॒सस्या᳚ र्द्धमा॒सस्य॒ रात्र॑यो ऽर्द्धमास॒शो᳚ ऽर्द्धमास॒शो रात्र॑यो ऽर्द्धमा॒सस्या᳚ र्द्धमा॒सस्य॒ रात्र॑यो ऽर्द्धमास॒शः । \newline
20. अ॒र्द्ध॒मा॒सस्येत्य॑र्द्ध - मा॒सस्य॑ । \newline
21. रात्र॑यो ऽर्द्धमास॒शो᳚ ऽर्द्धमास॒शो रात्र॑यो॒ रात्र॑यो ऽर्द्धमास॒शः सं॑ॅवथ्स॒रः सं॑ॅवथ्स॒रो᳚ ऽर्द्धमास॒शो रात्र॑यो॒ रात्र॑यो ऽर्द्धमास॒शः सं॑ॅवथ्स॒रः । \newline
22. अ॒र्द्ध॒मा॒स॒शः सं॑ॅवथ्स॒रः सं॑ॅवथ्स॒रो᳚ ऽर्द्धमास॒शो᳚ ऽर्द्धमास॒शः सं॑ॅवथ्स॒र आ᳚प्यत आप्यते संॅवथ्स॒रो᳚ ऽर्द्धमास॒शो᳚ ऽर्द्धमास॒शः सं॑ॅवथ्स॒र आ᳚प्यते । \newline
23. अ॒र्द्ध॒मा॒स॒श इत्य॑र्द्धमास - शः । \newline
24. सं॒ॅव॒थ्स॒र आ᳚प्यत आप्यते संॅवथ्स॒रः सं॑ॅवथ्स॒र आ᳚प्यते संॅवथ्स॒रः सं॑ॅवथ्स॒र आ᳚प्यते संॅवथ्स॒रः सं॑ॅवथ्स॒र आ᳚प्यते संॅवथ्स॒रः । \newline
25. सं॒ॅव॒थ्स॒र इति॑ सं - व॒थ्स॒रः । \newline
26. आ॒प्य॒ते॒ सं॒ॅव॒थ्स॒रः सं॑ॅवथ्स॒र आ᳚प्यत आप्यते संॅवथ्स॒रो वि॒राड् वि॒राट् थ्सं॑ॅवथ्स॒र आ᳚प्यत आप्यते संॅवथ्स॒रो वि॒राट् । \newline
27. सं॒ॅव॒थ्स॒रो वि॒राड् वि॒राट् थ्सं॑ॅवथ्स॒रः सं॑ॅवथ्स॒रो वि॒राड् वि॒राजं॑ ॅवि॒राजं॑ ॅवि॒राट् 
थ्सं॑ॅवथ्स॒रः सं॑ॅवथ्स॒रो वि॒राड् वि॒राज᳚म् । \newline
28. सं॒ॅव॒थ्स॒र इति॑ सं - व॒थ्स॒रः । \newline
29. वि॒राड् वि॒राजं॑ ॅवि॒राजं॑ ॅवि॒राड् वि॒राड् वि॒राज॑ माप्नो त्याप्नोति वि॒राजं॑ ॅवि॒राड् वि॒राड् वि॒राज॑ माप्नोति । \newline
30. वि॒राडिति॑ वि - राट् । \newline
31. वि॒राज॑ माप्नो त्याप्नोति वि॒राजं॑ ॅवि॒राज॑ माप्नोति स॒प्तद॑श स॒प्तद॑शाप्नोति वि॒राजं॑ ॅवि॒राज॑ माप्नोति स॒प्तद॑श । \newline
32. वि॒राज॒मिति॑ वि - राज᳚म् । \newline
33. आ॒प्नो॒ति॒ स॒प्तद॑श स॒प्तद॑शप्नो त्याप्नोति स॒प्तद॑श॒ रात्री॒ रात्रीः᳚ स॒प्तद॑शप्नो त्याप्नोति स॒प्तद॑श॒ रात्रीः᳚ । \newline
34. स॒प्तद॑श॒ रात्री॒ रात्रीः᳚ स॒प्तद॑श स॒प्तद॑श॒ रात्री᳚र् दीक्षि॒तो दी᳚क्षि॒तो रात्रीः᳚ स॒प्तद॑श स॒प्तद॑श॒ रात्री᳚र् दीक्षि॒तः । \newline
35. स॒प्तद॒शेति॑ स॒प्त - द॒श॒ । \newline
36. रात्री᳚र् दीक्षि॒तो दी᳚क्षि॒तो रात्री॒ रात्री᳚र् दीक्षि॒तः स्या᳚थ् स्याद् दीक्षि॒तो रात्री॒ रात्री᳚र् दीक्षि॒तः स्या᳚त् । \newline
37. दी॒क्षि॒तः स्या᳚थ् स्याद् दीक्षि॒तो दी᳚क्षि॒तः स्या॒द् द्वाद॑श॒ द्वाद॑श स्याद् दीक्षि॒तो दी᳚क्षि॒तः स्या॒द् द्वाद॑श । \newline
38. स्या॒द् द्वाद॑श॒ द्वाद॑श स्याथ् स्या॒द् द्वाद॑श॒ मासा॒ मासा॒ द्वाद॑श स्याथ् स्या॒द् द्वाद॑श॒ मासाः᳚ । \newline
39. द्वाद॑श॒ मासा॒ मासा॒ द्वाद॑श॒ द्वाद॑श॒ मासाः॒ पञ्च॒ पञ्च॒ मासा॒ द्वाद॑श॒ द्वाद॑श॒ मासाः॒ पञ्च॑ । \newline
40. मासाः॒ पञ्च॒ पञ्च॒ मासा॒ मासाः॒ पञ्च॒ र्‌तव॑ ऋ॒तवः॒ पञ्च॒ मासा॒ मासाः॒ पञ्च॒ र्‌तवः॑ । \newline
41. पञ्च॒ र्‌तव॑ ऋ॒तवः॒ पञ्च॒ पञ्च॒ र्‌तवः॒ स स ऋ॒तवः॒ पञ्च॒ पञ्च॒ र्‌तवः॒ सः । \newline
42. ऋ॒तवः॒ स स ऋ॒तव॑ ऋ॒तवः॒ स सं॑ॅवथ्स॒रः सं॑ॅवथ्स॒रः स ऋ॒तव॑ ऋ॒तवः॒ स सं॑ॅवथ्स॒रः । \newline
43. स सं॑ॅवथ्स॒रः सं॑ॅवथ्स॒रः स स सं॑ॅवथ्स॒रः । \newline
44. सं॒ॅव॒थ्स॒रः सं॑ॅवथ्स॒रः । \newline
45. सं॒ॅव॒थ्स॒र इति॑ सं - व॒थ्स॒रः । \newline
46. सं॒ॅव॒थ्स॒रो वि॒राड् वि॒राट् थ्सं॑ॅवथ्स॒रः सं॑ॅवथ्स॒रो वि॒राड् वि॒राजं॑ ॅवि॒राजं॑ ॅवि॒राट् 
थ्सं॑ॅवथ्स॒रः सं॑ॅवथ्स॒रो वि॒राड् वि॒राज᳚म् । \newline
47. सं॒ॅव॒थ्स॒र इति॑ सं - व॒थ्स॒रः । \newline
48. वि॒राड् वि॒राजं॑ ॅवि॒राजं॑ ॅवि॒राड् वि॒राड् वि॒राज॑ माप्नो त्याप्नोति वि॒राजं॑ ॅवि॒राड् वि॒राड् वि॒राज॑ माप्नोति । \newline
49. वि॒राडिति॑ वि - राट् । \newline
50. वि॒राज॑ माप्नो त्याप्नोति वि॒राजं॑ ॅवि॒राज॑ माप्नोति॒ चतु॑र्विꣳशति॒म् चतु॑र्विꣳशति माप्नोति वि॒राजं॑ ॅवि॒राज॑ माप्नोति॒ चतु॑र्विꣳशतिम् । \newline
51. वि॒राज॒मिति॑ वि - राज᳚म् । \newline
52. आ॒प्नो॒ति॒ चतु॑र्विꣳशति॒म् चतु॑र्विꣳशति माप्नो त्याप्नोति॒ चतु॑र्विꣳशतिꣳ॒॒ रात्री॒ रात्री॒ श्चतु॑र्विꣳशति माप्नो त्याप्नोति॒ चतु॑र्विꣳशतिꣳ॒॒ रात्रीः᳚ । \newline
53. चतु॑र्विꣳशतिꣳ॒॒ रात्री॒ रात्री॒ श्चतु॑र्विꣳशति॒म् चतु॑र्विꣳशतिꣳ॒॒ रात्री᳚र् दीक्षि॒तो दी᳚क्षि॒तो रात्री॒ श्चतु॑र्विꣳशति॒म् चतु॑र्विꣳशतिꣳ॒॒ रात्री᳚र् दीक्षि॒तः । \newline
54. चतु॑र्विꣳशति॒मिति॒ चतुः॑ - विꣳ॒॒श॒ति॒म् । \newline
55. रात्री᳚र् दीक्षि॒तो दी᳚क्षि॒तो रात्री॒ रात्री᳚र् दीक्षि॒तः स्या᳚थ् स्याद् दीक्षि॒तो रात्री॒ रात्री᳚र् दीक्षि॒तः स्या᳚त् । \newline
56. दी॒क्षि॒तः स्या᳚थ् स्याद् दीक्षि॒तो दी᳚क्षि॒तः स्या॒च् चतु॑र्विꣳशति॒ श्चतु॑र्विꣳशतिः स्या॒द् दीक्षि॒तो दी᳚क्षि॒तः स्या॒च् चतु॑र्विꣳशतिः । \newline
57. स्या॒च् चतु॑र्विꣳशति॒ श्चतु॑र्विꣳशतिः स्याथ् स्या॒च् चतु॑र्विꣳशति रर्द्धमा॒सा अ॑र्द्धमा॒सा श्चतु॑र्विꣳशतिः स्याथ् स्या॒च् चतु॑र्विꣳशति रर्द्धमा॒साः । \newline
58. चतु॑र्विꣳशति रर्द्धमा॒सा अ॑र्द्धमा॒सा श्चतु॑र्विꣳशति॒ श्चतु॑र्विꣳशति रर्द्धमा॒साः सं॑ॅवथ्स॒रः सं॑ॅवथ्स॒रो᳚ ऽर्द्धमा॒सा श्चतु॑र्विꣳशति॒ श्चतु॑र्विꣳशति रर्द्धमा॒साः सं॑ॅवथ्स॒रः । \newline
59. चतु॑र्विꣳशति॒रिति॒ चतुः॑ - विꣳ॒॒श॒तिः॒ । \newline
60. अ॒र्द्ध॒मा॒साः सं॑ॅवथ्स॒रः सं॑ॅवथ्स॒रो᳚ ऽर्द्धमा॒सा अ॑र्द्धमा॒साः सं॑ॅवथ्स॒रः । \newline
61. अ॒र्द्ध॒मा॒सा इत्य॑र्द्ध - मा॒साः । \newline
62. सं॒ॅव॒थ्स॒रः सं॑ॅवथ्स॒रः । \newline
63. सं॒ॅव॒थ्स॒र इति॑ सं - व॒थ्स॒रः । \newline
64. सं॒ॅव॒थ्स॒रो वि॒राड् वि॒राट् थ्सं॑ॅवथ्स॒रः सं॑ॅवथ्स॒रो वि॒राड् वि॒राजं॑ ॅवि॒राजं॑ ॅवि॒राट् 
थ्सं॑ॅवथ्स॒रः सं॑ॅवथ्स॒रो वि॒राड् वि॒राज᳚म् । \newline
65. सं॒ॅव॒थ्स॒र इति॑ सं - व॒थ्स॒रः । \newline
66. वि॒राड् वि॒राजं॑ ॅवि॒राजं॑ ॅवि॒राड् वि॒राड् वि॒राज॑ माप्नो त्याप्नोति वि॒राजं॑ ॅवि॒राड् वि॒राड् वि॒राज॑ माप्नोति । \newline
67. वि॒राडिति॑ वि - राट् । \newline
68. वि॒राज॑ माप्नो त्याप्नोति वि॒राजं॑ ॅवि॒राज॑ माप्नोति त्रिꣳ॒॒शत॑म् त्रिꣳ॒॒शत॑ माप्नोति वि॒राजं॑ ॅवि॒राज॑ माप्नोति त्रिꣳ॒॒शत᳚म् । \newline
69. वि॒राज॒मिति॑ वि - राज᳚म् । \newline
70. आ॒प्नो॒ति॒ त्रिꣳ॒॒शत॑म् त्रिꣳ॒॒शत॑ माप्नो त्याप्नोति त्रिꣳ॒॒शतꣳ॒॒ रात्री॒ रात्री᳚ स्त्रिꣳ॒॒शत॑ माप्नो त्याप्नोति त्रिꣳ॒॒शतꣳ॒॒ रात्रीः᳚ । \newline
71. त्रिꣳ॒॒शतꣳ॒॒ रात्री॒ रात्री᳚ स्त्रिꣳ॒॒शत॑म् त्रिꣳ॒॒शतꣳ॒॒ रात्री᳚र् दीक्षि॒तो दी᳚क्षि॒तो रात्री᳚ स्त्रिꣳ॒॒शत॑म् त्रिꣳ॒॒शतꣳ॒॒ रात्री᳚र् दीक्षि॒तः । \newline
72. रात्री᳚र् दीक्षि॒तो दी᳚क्षि॒तो रात्री॒ रात्री᳚र् दीक्षि॒तः स्या᳚थ् स्याद् दीक्षि॒तो रात्री॒ रात्री᳚र् दीक्षि॒तः स्या᳚त् । \newline
73. दी॒क्षि॒तः स्या᳚थ् स्याद् दीक्षि॒तो दी᳚क्षि॒तः स्या᳚त् त्रिꣳ॒॒शद॑क्षरा त्रिꣳ॒॒शद॑क्षरा स्याद् दीक्षि॒तो दी᳚क्षि॒तः स्या᳚त् त्रिꣳ॒॒शद॑क्षरा । \newline
74. स्या॒त् त्रिꣳ॒॒शद॑क्षरा त्रिꣳ॒॒शद॑क्षरा स्याथ् स्यात् त्रिꣳ॒॒शद॑क्षरा वि॒राड् वि॒राट् त्रिꣳ॒॒शद॑क्षरा स्याथ् स्यात् त्रिꣳ॒॒शद॑क्षरा वि॒राट् । \newline
\pagebreak
\markright{ TS 5.6.7.3  \hfill https://www.vedavms.in \hfill}

\section{ TS 5.6.7.3 }

\textbf{TS 5.6.7.3 } \newline
\textbf{Samhita Paata} \newline

त्रिꣳ॒॒शद॑क्षरा वि॒राड् वि॒राज॑माप्नोति॒ मासं॑ दीक्षि॒तः स्या॒द्-यो मासः॒ स सं॑ॅवथ्स॒रः सं॑ॅवथ्स॒रो वि॒राड् वि॒राज॑माप्नोति च॒तुरो॑ मा॒सो दी᳚क्षि॒तः स्या᳚च्च॒तुरो॒ वा ए॒तं मा॒सो वस॑वोऽबिभरु॒स्ते पृ॑थि॒वीमाऽज॑यन् गाय॒त्रीं छन्दो॒ऽष्टौ रु॒द्रास्ते᳚-ऽन्तरि॑क्ष॒माऽज॑यन् त्रि॒ष्टुभं॒ छन्दो॒ द्वाद॑शा-ऽऽदि॒त्यास्ते दिव॒माऽज॑य॒न् जग॑तीं॒ छन्द॒स्ततो॒ वै ते ( ) व्या॒वृत॑-मगच्छ॒ञ्छ्रैष्ठ्यं॑ दे॒वानां॒ तस्मा॒द् द्वाद॑श मा॒सो भृ॒त्वाऽग्निं चि॑न्वीत॒ द्वाद॑श॒ मासाः᳚ संॅवथ्स॒रः सं॑ॅवथ्स॒रो᳚ -ऽग्निश्चित्य॒स्तस्या॑-होरा॒त्राणीष्ट॑का आ॒प्तेष्ट॑कमेनं चिनु॒तेऽथो᳚ व्या॒वृत॑मे॒व ग॑च्छति॒ श्रैष्ठ्यꣳ॑ समा॒नानां᳚ ॥ \newline

\textbf{Pada Paata} \newline

त्रिꣳ॒॒शद॑क्ष॒रेति॑ त्रिꣳ॒॒शत् - अ॒क्ष॒रा॒ । वि॒राडिति॑ वि - राट् । वि॒राज॒मिति॑ वि - राज᳚म् । आ॒प्नो॒ति॒ । मास᳚म् । दी॒क्षि॒तः । स्या॒त् । यः । मासः॑ । सः । सं॒ॅव॒थ्स॒र इति॑ सं - व॒थ्स॒रः । सं॒ॅव॒थ्स॒र इति॑ सं - व॒थ्स॒रः । वि॒राडिति॑ वि - राट् । वि॒राज॒मिति॑ वि - राज᳚म् । आ॒प्नो॒ति॒ । च॒तुरः॑ । मा॒सः । दी॒क्षि॒तः । स्या॒त् । च॒तुरः॑ । वै । ए॒तम् । मा॒सः । वस॑वः । अ॒बि॒भ॒रुः॒ । ते । पृ॒थि॒वीम् । एति॑ । अ॒ज॒य॒न्न् । गा॒य॒त्रीम् । छन्दः॑ । अ॒ष्टौ । रु॒द्राः । ते । अ॒न्तरि॑क्षम् । एति॑ । अ॒ज॒य॒न्न् । त्रि॒ष्टुभ᳚म् । छन्दः॑ । द्वाद॑श । आ॒दि॒त्याः । ते । दिव᳚म् । एति॑ । अ॒ज॒य॒न्न् । जग॑तीम् । छन्दः॑ । ततः॑ । वै । ते ( ) । व्या॒वृत॒मिति॑ वि - आ॒वृत᳚म् । अ॒ग॒च्छ॒न्न् । श्रैष्ठ्य᳚म् । दे॒वाना᳚म् । तस्मा᳚त् । द्वाद॑श । मा॒सः । भृ॒त्वा । अ॒ग्निम् । चि॒न्वी॒त॒ । द्वाद॑श । मासाः᳚ । सं॒ॅव॒थ्स॒र इति॑ सं - व॒थ्स॒रः । सं॒ॅव॒थ्स॒र इति॑ सं - व॒थ्स॒रः । अ॒ग्निः । चित्यः॑ । तस्य॑ । अ॒हो॒रा॒त्राणीत्य॑हः - रा॒त्राणि॑ । इष्ट॑काः । आ॒प्तेष्ट॑क॒मित्या॒प्त-इ॒ष्ट॒क॒म् । ए॒न॒म् । चि॒नु॒ते॒ । अथो॒ इति॑ । व्या॒वृत॒मिति॑ वि - आ॒वृत᳚म् । ए॒व । ग॒च्छ॒ति॒ । श्रैष्ठ्य᳚म् । स॒मा॒नाना᳚म् ॥  \newline


\textbf{Krama Paata} \newline

त्रिꣳ॒॒शद॑क्षरा वि॒राट् । त्रिꣳ॒॒शद॑क्ष॒रेति॑ त्रिꣳ॒॒शत् - अ॒क्ष॒रा॒ । वि॒राड् वि॒राज᳚म् । वि॒राडिति॑ वि - राट् । वि॒राज॑माप्नोति । वि॒राज॒मिति॑ वि - राज᳚म् । आ॒प्नो॒ति॒ मास᳚म् । मास॑म् दीक्षि॒तः । दी॒क्षि॒तः स्या᳚त् । स्या॒द् यः । यो मासः॑ । मासः॒ सः । स स॑म्ॅवथ्स॒रः । स॒म्ॅव॒थ्स॒रः स॑म्ॅवथ्स॒रः । स॒म्ॅव॒थ्स॒र इति॑ सम् - व॒थ्स॒रः । स॒म्ॅव॒थ्स॒रो वि॒राट् । स॒म्ॅव॒थ्स॒र इति॑ सम् - व॒थ्स॒रः । वि॒राड् वि॒राज᳚म् । वि॒राडिति॑ वि - राट् । वि॒राज॑माप्नोति । वि॒राज॒मिति॑ वि - राज᳚म् । आ॒प्नो॒ति॒ च॒तुरः॑ । च॒तुरो॑ मा॒सः । मा॒सो दी᳚क्षि॒तः । दी॒क्षि॒तः स्या᳚त् । स्या॒च्च॒तुरः॑ । च॒तुरो॒ वै । वा ए॒तम् । ए॒तम् मा॒सः । मा॒सो वस॑वः । वस॑वोऽबिभरुः । अ॒बि॒भ॒रु॒स्ते । ते पृ॑थि॒वीम् । पृ॒थि॒वीमा । आऽज॑यन्न् । अ॒ज॒य॒न् गा॒य॒त्रीम् । गा॒य॒त्रीम् छन्दः॑ । छन्दो॒ऽष्टौ । अ॒ष्टौ रु॒द्राः । रु॒द्रास्ते । ते᳚ऽन्तरि॑क्षम् । अ॒न्तरि॑क्ष॒मा । आऽज॑यन्न् । अ॒ज॒य॒न् त्रि॒ष्टुभ᳚म् । त्रि॒ष्टुभ॒म् छन्दः॑ । छन्दो॒ द्वाद॑श । द्वाद॑शादि॒त्याः । आ॒दि॒त्यास्ते । ते दिव᳚म् । दिव॒मा । आऽज॑यन्न् । अ॒ज॒य॒न् जग॑तीम् । जग॑ती॒म् छन्दः॑ । छन्द॒स्ततः॑ । ततो॒ वै । वै ते ( ) । ते व्या॒वृत᳚म् । व्या॒वृत॑मगच्छन्न् । व्या॒वृत॒मिति॑ वि - आ॒वृत᳚म् । अ॒ग॒च्छ॒न् श्रैष्ठ्य᳚म् । श्रैष्ठ्य॑म् दे॒वाना᳚म् । दे॒वाना॒म् तस्मा᳚त् । तस्मा॒द् द्वाद॑श । द्वाद॑श मा॒सः । मा॒सो भृ॒त्वा । भृ॒त्वाऽग्निम् । अ॒ग्निम् चि॑न्वीत । चि॒न्वी॒त॒ द्वाद॑श । द्वाद॑श॒ मासाः᳚ । मासाः᳚ सम्ॅवथ्स॒रः । स॒म्ॅव॒थ्स॒रः स॑म्ॅवथ्स॒रः । स॒म्ॅव॒थ्स॒र इति॑ सम् - व॒थ्स॒रः । स॒म्ॅव॒थ्स॒रो᳚ऽग्निः । स॒म्ॅव॒थ्स॒र इति॑ सम् - व॒थ्स॒रः । अ॒ग्निश्चित्यः॑ । चित्य॒स्तस्य॑ । तस्या॑होरा॒त्राणि॑ । अ॒हो॒रा॒त्राणीष्ट॑काः । अ॒हो॒रा॒त्राणीत्य॑हः - रा॒त्राणि॑ । इष्ट॑का आ॒प्तेष्ट॑कम् । आ॒प्तेष्ट॑कमेनम् । आ॒प्तेष्ट॑क॒मित्या॒प्त - इ॒ष्ट॒क॒म् । ए॒न॒म् चि॒नु॒ते॒ । चि॒नु॒तेऽथो᳚ । अथो᳚ व्या॒वृत᳚म् । अथो॒ इत्यथो᳚ । व्या॒वृत॑मे॒व । व्या॒वृत॒मिति॑ वि - आ॒वृत᳚म् । ए॒व ग॑च्छति । ग॒च्छ॒ति॒ श्रैष्ठ्य᳚म् । श्रैष्ठ्यꣳ॑ समा॒नाना᳚म् । स॒मा॒नाना॒मिति॑ समा॒नाना᳚म् । \newline

\textbf{Jatai Paata} \newline

1. त्रिꣳ॒॒शद॑क्षरा वि॒राड् वि॒राट् त्रिꣳ॒॒शद॑क्षरा त्रिꣳ॒॒शद॑क्षरा वि॒राट् । \newline
2. त्रिꣳ॒॒शद॑क्ष॒रेति॑ त्रिꣳ॒॒शत् - अ॒क्ष॒रा॒ । \newline
3. वि॒राड् वि॒राजं॑ ॅवि॒राजं॑ ॅवि॒राड् वि॒राड् वि॒राज᳚म् । \newline
4. वि॒राडिति॑ वि - राट् । \newline
5. वि॒राज॑ माप्नो त्याप्नोति वि॒राजं॑ ॅवि॒राज॑ माप्नोति । \newline
6. वि॒राज॒मिति॑ वि - राज᳚म् । \newline
7. आ॒प्नो॒ति॒ मास॒म् मास॑ माप्नो त्याप्नोति॒ मास᳚म् । \newline
8. मास॑म् दीक्षि॒तो दी᳚क्षि॒तो मास॒म् मास॑म् दीक्षि॒तः । \newline
9. दी॒क्षि॒तः स्या᳚थ् स्याद् दीक्षि॒तो दी᳚क्षि॒तः स्या᳚त् । \newline
10. स्या॒द् यो यः स्या᳚थ् स्या॒द् यः । \newline
11. यो मासो॒ मासो॒ यो यो मासः॑ । \newline
12. मासः॒ स स मासो॒ मासः॒ सः । \newline
13. स सं॑ॅवथ्स॒रः सं॑ॅवथ्स॒रः स स सं॑ॅवथ्स॒रः । \newline
14. सं॒ॅव॒थ्स॒रः सं॑ॅवथ्स॒रः । \newline
15. सं॒ॅव॒थ्स॒र इति॑ सं - व॒थ्स॒रः । \newline
16. सं॒ॅव॒थ्स॒रो वि॒राड् वि॒राट् थ्सं॑ॅवथ्स॒रः सं॑ॅवथ्स॒रो वि॒राट् । \newline
17. सं॒ॅव॒थ्स॒र इति॑ सं - व॒थ्स॒रः । \newline
18. वि॒राड् वि॒राजं॑ ॅवि॒राजं॑ ॅवि॒राड् वि॒राड् वि॒राज᳚म् । \newline
19. वि॒राडिति॑ वि - राट् । \newline
20. वि॒राज॑ माप्नो त्याप्नोति वि॒राजं॑ ॅवि॒राज॑ माप्नोति । \newline
21. वि॒राज॒मिति॑ वि - राज᳚म् । \newline
22. आ॒प्नो॒ति॒ च॒तुर॑ श्च॒तुर॑ आप्नो त्याप्नोति च॒तुरः॑ । \newline
23. च॒तुरो॑ मा॒सो मा॒स श्च॒तुर॑ श्च॒तुरो॑ मा॒सः । \newline
24. मा॒सो दी᳚क्षि॒तो दी᳚क्षि॒तो मा॒सो मा॒सो दी᳚क्षि॒तः । \newline
25. दी॒क्षि॒तः स्या᳚थ् स्याद् दीक्षि॒तो दी᳚क्षि॒तः स्या᳚त् । \newline
26. स्या॒च् च॒तुर॑ श्च॒तुरः॑ स्याथ् स्याच् च॒तुरः॑ । \newline
27. च॒तुरो॒ वै वै च॒तुर॑ श्च॒तुरो॒ वै । \newline
28. वा ए॒त मे॒तं ॅवै वा ए॒तम् । \newline
29. ए॒तम् मा॒सो मा॒स ए॒त मे॒तम् मा॒सः । \newline
30. मा॒सो वस॑वो॒ वस॑वो मा॒सो मा॒सो वस॑वः । \newline
31. वस॑वो ऽबिभरु रबिभरु॒र् वस॑वो॒ वस॑वो ऽबिभरुः । \newline
32. अ॒बि॒भ॒रु॒ स्ते ते॑ ऽबिभरु रबिभरु॒ स्ते । \newline
33. ते पृ॑थि॒वीम् पृ॑थि॒वीम् ते ते पृ॑थि॒वीम् । \newline
34. पृ॒थि॒वी मा पृ॑थि॒वीम् पृ॑थि॒वी मा । \newline
35. आ ऽज॑यन् नजय॒न् ना ऽज॑यन्न् । \newline
36. अ॒ज॒य॒न् गा॒य॒त्रीम् गा॑य॒त्री म॑जयन् नजयन् गाय॒त्रीम् । \newline
37. गा॒य॒त्रीम् छन्द॒ श्छन्दो॑ गाय॒त्रीम् गा॑य॒त्रीम् छन्दः॑ । \newline
38. छन्दो॒ ऽष्टा व॒ष्टौ छन्द॒ श्छन्दो॒ ऽष्टौ । \newline
39. अ॒ष्टौ रु॒द्रा रु॒द्रा अ॒ष्टा व॒ष्टौ रु॒द्राः । \newline
40. रु॒द्रा स्ते ते रु॒द्रा रु॒द्रा स्ते । \newline
41. ते᳚ ऽन्तरि॑क्ष म॒न्तरि॑क्ष॒म् ते ते᳚ ऽन्तरि॑क्षम् । \newline
42. अ॒न्तरि॑क्ष॒ मा ऽन्तरि॑क्ष म॒न्तरि॑क्ष॒ मा । \newline
43. आ ऽज॑यन् नजय॒न् ना ऽज॑यन्न् । \newline
44. अ॒ज॒य॒न् त्रि॒ष्टुभ॑म् त्रि॒ष्टुभ॑ मजयन् नजयन् त्रि॒ष्टुभ᳚म् । \newline
45. त्रि॒ष्टुभ॒म् छन्द॒ श्छन्द॑ स्त्रि॒ष्टुभ॑म् त्रि॒ष्टुभ॒म् छन्दः॑ । \newline
46. छन्दो॒ द्वाद॑श॒ द्वाद॑श॒ छन्द॒ श्छन्दो॒ द्वाद॑श । \newline
47. द्वाद॑शादि॒त्या आ॑दि॒त्या द्वाद॑श॒ द्वाद॑शादि॒त्याः । \newline
48. आ॒दि॒त्या स्ते त आ॑दि॒त्या आ॑दि॒त्या स्ते । \newline
49. ते दिव॒म् दिव॒म् ते ते दिव᳚म् । \newline
50. दिव॒ मा दिव॒म् दिव॒ मा । \newline
51. आ ऽज॑यन् नजय॒न् ना ऽज॑यन्न् । \newline
52. अ॒ज॒य॒न् जग॑ती॒म् जग॑ती मजयन् नजय॒न् जग॑तीम् । \newline
53. जग॑ती॒म् छन्द॒ श्छन्दो॒ जग॑ती॒म् जग॑ती॒म् छन्दः॑ । \newline
54. छन्द॒ स्तत॒ स्तत॒ श्छन्द॒ श्छन्द॒ स्ततः॑ । \newline
55. ततो॒ वै वै तत॒ स्ततो॒ वै । \newline
56. वै ते ते वै वै ते । \newline
57. ते व्या॒वृतं॑ ॅव्या॒वृत॒म् ते ते व्या॒वृत᳚म् । \newline
58. व्या॒वृत॑ मगच्छन् नगच्छन् व्या॒वृतं॑ ॅव्या॒वृत॑ मगच्छन्न् । \newline
59. व्या॒वृत॒मिति॑ वि - आ॒वृत᳚म् । \newline
60. अ॒ग॒च्छ॒ञ् छ्रैष्ठ्यꣳ॒॒ श्रैष्ठ्य॑ मगच्छन् नगच्छ॒ञ् छ्रैष्ठ्य᳚म् । \newline
61. श्रैष्ठ्य॑म् दे॒वाना᳚म् दे॒वानाꣳ॒॒ श्रैष्ठ्यꣳ॒॒ श्रैष्ठ्य॑म् दे॒वाना᳚म् । \newline
62. दे॒वाना॒म् तस्मा॒त् तस्मा᳚द् दे॒वाना᳚म् दे॒वाना॒म् तस्मा᳚त् । \newline
63. तस्मा॒द् द्वाद॑श॒ द्वाद॑श॒ तस्मा॒त् तस्मा॒द् द्वाद॑श । \newline
64. द्वाद॑श मा॒सो मा॒सो द्वाद॑श॒ द्वाद॑श मा॒सः । \newline
65. मा॒सो भृ॒त्वा भृ॒त्वा मा॒सो मा॒सो भृ॒त्वा । \newline
66. भृ॒त्वा ऽग्नि म॒ग्निम् भृ॒त्वा भृ॒त्वा ऽग्निम् । \newline
67. अ॒ग्निम् चि॑न्वीत चिन्वीता॒ग्नि म॒ग्निम् चि॑न्वीत । \newline
68. चि॒न्वी॒त॒ द्वाद॑श॒ द्वाद॑श चिन्वीत चिन्वीत॒ द्वाद॑श । \newline
69. द्वाद॑श॒ मासा॒ मासा॒ द्वाद॑श॒ द्वाद॑श॒ मासाः᳚ । \newline
70. मासाः᳚ संॅवथ्स॒रः सं॑ॅवथ्स॒रो मासा॒ मासाः᳚ संॅवथ्स॒रः । \newline
71. सं॒ॅव॒थ्स॒रः सं॑ॅवथ्स॒रः । \newline
72. सं॒ॅव॒थ्स॒र इति॑ सं - व॒थ्स॒रः । \newline
73. सं॒ॅव॒थ्स॒रो᳚ ऽग्नि र॒ग्निः सं॑ॅवथ्स॒रः सं॑ॅवथ्स॒रो᳚ ऽग्निः । \newline
74. सं॒ॅव॒थ्स॒र इति॑ सं - व॒थ्स॒रः । \newline
75. अ॒ग्नि श्चित्य॒ श्चित्यो॒ ऽग्नि र॒ग्नि श्चित्यः॑ । \newline
76. चित्य॒ स्तस्य॒ तस्य॒ चित्य॒ श्चित्य॒ स्तस्य॑ । \newline
77. तस्या॑ होरा॒त्रा ण्य॑होरा॒त्राणि॒ तस्य॒ तस्या॑ होरा॒त्राणि॑ । \newline
78. अ॒हो॒रा॒त्राणी ष्ट॑का॒ इष्ट॑का अहोरा॒त्रा ण्य॑होरा॒त्राणी ष्ट॑काः । \newline
79. अ॒हो॒रा॒त्राणीत्य॑हः - रा॒त्राणि॑ । \newline
80. इष्ट॑का आ॒प्तेष्ट॑क मा॒प्तेष्ट॑क॒ मिष्ट॑का॒ इष्ट॑का आ॒प्तेष्ट॑कम् । \newline
81. आ॒प्तेष्ट॑क मेन मेन मा॒प्तेष्ट॑क मा॒प्तेष्ट॑क मेनम् । \newline
82. आ॒प्तेष्ट॑क॒मित्या॒प्त - इ॒ष्ट॒क॒म् । \newline
83. ए॒न॒म् चि॒नु॒ते॒ चि॒नु॒त॒ ए॒न॒ मे॒न॒म् चि॒नु॒ते॒ । \newline
84. चि॒नु॒ते ऽथो॒ अथो॑ चिनुते चिनु॒ते ऽथो᳚ । \newline
85. अथो᳚ व्या॒वृतं॑ ॅव्या॒वृत॒ मथो॒ अथो᳚ व्या॒वृत᳚म् । \newline
86. अथो॒ इत्यथो᳚ । \newline
87. व्या॒वृत॑ मे॒वैव व्या॒वृतं॑ ॅव्या॒वृत॑ मे॒व । \newline
88. व्या॒वृत॒मिति॑ वि - आ॒वृत᳚म् । \newline
89. ए॒व ग॑च्छति गच्छ त्ये॒वैव ग॑च्छति । \newline
90. ग॒च्छ॒ति॒ श्रैष्ठ्यꣳ॒॒ श्रैष्ठ्य॑म् गच्छति गच्छति॒ श्रैष्ठ्य᳚म् । \newline
91. श्रैष्ठ्यꣳ॑ समा॒नानाꣳ॑ समा॒नानाꣳ॒॒ श्रैष्ठ्यꣳ॒॒ श्रैष्ठ्यꣳ॑ समा॒नाना᳚म् । \newline
92. स॒मा॒नाना॒मिति॑ समा॒नाना᳚म् । \newline

\textbf{Ghana Paata } \newline

1. त्रिꣳ॒॒शद॑क्षरा वि॒राड् वि॒राट् त्रिꣳ॒॒शद॑क्षरा त्रिꣳ॒॒शद॑क्षरा वि॒राड् वि॒राजं॑ ॅवि॒राजं॑ ॅवि॒राट् त्रिꣳ॒॒शद॑क्षरा त्रिꣳ॒॒शद॑क्षरा वि॒राड् वि॒राज᳚म् । \newline
2. त्रिꣳ॒॒शद॑क्ष॒रेति॑ त्रिꣳ॒॒शत् - अ॒क्ष॒रा॒ । \newline
3. वि॒राड् वि॒राजं॑ ॅवि॒राजं॑ ॅवि॒राड् वि॒राड् वि॒राज॑ माप्नो त्याप्नोति वि॒राजं॑ ॅवि॒राड् वि॒राड् वि॒राज॑ माप्नोति । \newline
4. वि॒राडिति॑ वि - राट् । \newline
5. वि॒राज॑ माप्नो त्याप्नोति वि॒राजं॑ ॅवि॒राज॑ माप्नोति॒ मास॒म् मास॑ माप्नोति वि॒राजं॑ ॅवि॒राज॑ माप्नोति॒ मास᳚म् । \newline
6. वि॒राज॒मिति॑ वि - राज᳚म् । \newline
7. आ॒प्नो॒ति॒ मास॒म् मास॑ माप्नो त्याप्नोति॒ मास॑म् दीक्षि॒तो दी᳚क्षि॒तो मास॑ माप्नो त्याप्नोति॒ मास॑म् दीक्षि॒तः । \newline
8. मास॑म् दीक्षि॒तो दी᳚क्षि॒तो मास॒म् मास॑म् दीक्षि॒तः स्या᳚थ् स्याद् दीक्षि॒तो मास॒म् मास॑म् दीक्षि॒तः स्या᳚त् । \newline
9. दी॒क्षि॒तः स्या᳚थ् स्याद् दीक्षि॒तो दी᳚क्षि॒तः स्या॒द् यो यः स्या᳚द् दीक्षि॒तो दी᳚क्षि॒तः स्या॒द् यः । \newline
10. स्या॒द् यो यः स्या᳚थ् स्या॒द् यो मासो॒ मासो॒ यः स्या᳚थ् स्या॒द् यो मासः॑ । \newline
11. यो मासो॒ मासो॒ यो यो मासः॒ स स मासो॒ यो यो मासः॒ सः । \newline
12. मासः॒ स स मासो॒ मासः॒ स सं॑ॅवथ्स॒रः सं॑ॅवथ्स॒रः स मासो॒ मासः॒ स सं॑ॅवथ्स॒रः । \newline
13. स सं॑ॅवथ्स॒रः सं॑ॅवथ्स॒रः स स सं॑ॅवथ्स॒रः । \newline
14. सं॒ॅव॒थ्स॒रः सं॑ॅवथ्स॒रः । \newline
15. सं॒ॅव॒थ्स॒र इति॑ सं - व॒थ्स॒रः । \newline
16. सं॒ॅव॒थ्स॒रो वि॒राड् वि॒राट् थ्सं॑ॅवथ्स॒रः सं॑ॅवथ्स॒रो वि॒राड् वि॒राजं॑ ॅवि॒राजं॑ ॅवि॒राट् 
थ्सं॑ॅवथ्स॒रः सं॑ॅवथ्स॒रो वि॒राड् वि॒राज᳚म् । \newline
17. सं॒ॅव॒थ्स॒र इति॑ सं - व॒थ्स॒रः । \newline
18. वि॒राड् वि॒राजं॑ ॅवि॒राजं॑ ॅवि॒राड् वि॒राड् वि॒राज॑ माप्नो त्याप्नोति वि॒राजं॑ ॅवि॒राड् वि॒राड् वि॒राज॑ माप्नोति । \newline
19. वि॒राडिति॑ वि - राट् । \newline
20. वि॒राज॑ माप्नो त्याप्नोति वि॒राजं॑ ॅवि॒राज॑ माप्नोति च॒तुर॑ श्च॒तुर॑ आप्नोति वि॒राजं॑ ॅवि॒राज॑ माप्नोति च॒तुरः॑ । \newline
21. वि॒राज॒मिति॑ वि - राज᳚म् । \newline
22. आ॒प्नो॒ति॒ च॒तुर॑ श्च॒तुर॑ आप्नो त्याप्नोति च॒तुरो॑ मा॒सो मा॒स श्च॒तुर॑ आप्नो त्याप्नोति च॒तुरो॑ मा॒सः । \newline
23. च॒तुरो॑ मा॒सो मा॒स श्च॒तुर॑ श्च॒तुरो॑ मा॒सो दी᳚क्षि॒तो दी᳚क्षि॒तो मा॒स श्च॒तुर॑ श्च॒तुरो॑ मा॒सो दी᳚क्षि॒तः । \newline
24. मा॒सो दी᳚क्षि॒तो दी᳚क्षि॒तो मा॒सो मा॒सो दी᳚क्षि॒तः स्या᳚थ् स्याद् दीक्षि॒तो मा॒सो मा॒सो दी᳚क्षि॒तः स्या᳚त् । \newline
25. दी॒क्षि॒तः स्या᳚थ् स्याद् दीक्षि॒तो दी᳚क्षि॒तः स्या᳚च् च॒तुर॑ श्च॒तुरः॑ स्याद् दीक्षि॒तो दी᳚क्षि॒तः स्या᳚च् च॒तुरः॑ । \newline
26. स्या॒च् च॒तुर॑ श्च॒तुरः॑ स्याथ् स्याच् च॒तुरो॒ वै वै च॒तुरः॑ स्याथ् स्याच् च॒तुरो॒ वै । \newline
27. च॒तुरो॒ वै वै च॒तुर॑ श्च॒तुरो॒ वा ए॒त मे॒तं ॅवै च॒तुर॑ श्च॒तुरो॒ वा ए॒तम् । \newline
28. वा ए॒त मे॒तं ॅवै वा ए॒तम् मा॒सो मा॒स ए॒तं ॅवै वा ए॒तम् मा॒सः । \newline
29. ए॒तम् मा॒सो मा॒स ए॒त मे॒तम् मा॒सो वस॑वो॒ वस॑वो मा॒स ए॒त मे॒तम् मा॒सो वस॑वः । \newline
30. मा॒सो वस॑वो॒ वस॑वो मा॒सो मा॒सो वस॑वो ऽबिभरु रबिभरु॒र् वस॑वो मा॒सो मा॒सो वस॑वो ऽबिभरुः । \newline
31. वस॑वो ऽबिभरु रबिभरु॒र् वस॑वो॒ वस॑वो ऽबिभरु॒ स्ते ते॑ ऽबिभरु॒र् वस॑वो॒ वस॑वो ऽबिभरु॒ स्ते । \newline
32. अ॒बि॒भ॒रु॒ स्ते ते॑ ऽबिभरु रबिभरु॒ स्ते पृ॑थि॒वीम् पृ॑थि॒वीम् ते॑ ऽबिभरु रबिभरु॒ स्ते पृ॑थि॒वीम् । \newline
33. ते पृ॑थि॒वीम् पृ॑थि॒वीम् ते ते पृ॑थि॒वी मा पृ॑थि॒वीम् ते ते पृ॑थि॒वी मा । \newline
34. पृ॒थि॒वी मा पृ॑थि॒वीम् पृ॑थि॒वी मा ऽज॑यन् नजय॒न् ना पृ॑थि॒वीम् पृ॑थि॒वी मा ऽज॑यन्न् । \newline
35. आ ऽज॑यन् नजय॒न्ना ऽज॑यन् गाय॒त्रीम् गा॑य॒त्री म॑जय॒न्ना ऽज॑यन् गाय॒त्रीम् । \newline
36. अ॒ज॒य॒न् गा॒य॒त्रीम् गा॑य॒त्री म॑जयन् नजयन् गाय॒त्रीम् छन्द॒ श्छन्दो॑ गाय॒त्री म॑जयन् नजयन् गाय॒त्रीम् छन्दः॑ । \newline
37. गा॒य॒त्रीम् छन्द॒ श्छन्दो॑ गाय॒त्रीम् गा॑य॒त्रीम् छन्दो॒ ऽष्टा व॒ष्टौ छन्दो॑ गाय॒त्रीम् गा॑य॒त्रीम् छन्दो॒ ऽष्टौ । \newline
38. छन्दो॒ ऽष्टा व॒ष्टौ छन्द॒ श्छन्दो॒ ऽष्टौ रु॒द्रा रु॒द्रा अ॒ष्टौ छन्द॒ श्छन्दो॒ ऽष्टौ रु॒द्राः । \newline
39. अ॒ष्टौ रु॒द्रा रु॒द्रा अ॒ष्टा व॒ष्टौ रु॒द्रा स्ते ते रु॒द्रा अ॒ष्टा व॒ष्टौ रु॒द्रा स्ते । \newline
40. रु॒द्रा स्ते ते रु॒द्रा रु॒द्रा स्ते᳚ ऽन्तरि॑क्ष म॒न्तरि॑क्ष॒म् ते रु॒द्रा रु॒द्रा स्ते᳚ ऽन्तरि॑क्षम् । \newline
41. ते᳚ ऽन्तरि॑क्ष म॒न्तरि॑क्ष॒म् ते ते᳚ ऽन्तरि॑क्ष॒ मा ऽन्तरि॑क्ष॒म् ते ते᳚ ऽन्तरि॑क्ष॒ मा । \newline
42. अ॒न्तरि॑क्ष॒ मा ऽन्तरि॑क्ष म॒न्तरि॑क्ष॒ मा ऽज॑यन् नजय॒न्ना ऽन्तरि॑क्ष म॒न्तरि॑क्ष॒ मा ऽज॑यन्न् । \newline
43. आ ऽज॑यन् नजय॒न्ना ऽज॑यन् त्रि॒ष्टुभ॑म् त्रि॒ष्टुभ॑ मजय॒न्ना ऽज॑यन् त्रि॒ष्टुभ᳚म् । \newline
44. अ॒ज॒य॒न् त्रि॒ष्टुभ॑म् त्रि॒ष्टुभ॑ मजयन् नजयन् त्रि॒ष्टुभ॒म् छन्द॒ श्छन्द॑ स्त्रि॒ष्टुभ॑ मजयन् नजयन् त्रि॒ष्टुभ॒म् छन्दः॑ । \newline
45. त्रि॒ष्टुभ॒म् छन्द॒ श्छन्द॑ स्त्रि॒ष्टुभ॑म् त्रि॒ष्टुभ॒म् छन्दो॒ द्वाद॑श॒ द्वाद॑श॒ छन्द॑ स्त्रि॒ष्टुभ॑म् त्रि॒ष्टुभ॒म् छन्दो॒ द्वाद॑श । \newline
46. छन्दो॒ द्वाद॑श॒ द्वाद॑श॒ छन्द॒ श्छन्दो॒ द्वाद॑शादि॒त्या आ॑दि॒त्या द्वाद॑श॒ छन्द॒ श्छन्दो॒ द्वाद॑शादि॒त्याः । \newline
47. द्वाद॑शादि॒त्या आ॑दि॒त्या द्वाद॑श॒ द्वाद॑शादि॒त्या स्ते त आ॑दि॒त्या द्वाद॑श॒ द्वाद॑शादि॒त्या स्ते । \newline
48. आ॒दि॒त्या स्ते त आ॑दि॒त्या आ॑दि॒त्या स्ते दिव॒म् दिव॒म् त आ॑दि॒त्या आ॑दि॒त्या स्ते दिव᳚म् । \newline
49. ते दिव॒म् दिव॒म् ते ते दिव॒ मा दिव॒म् ते ते दिव॒ मा । \newline
50. दिव॒ मा दिव॒म् दिव॒ मा ऽज॑यन् नजय॒न्ना दिव॒म् दिव॒ मा ऽज॑यन्न् । \newline
51. आ ऽज॑यन् नजय॒न्ना ऽज॑य॒न् जग॑ती॒म् जग॑ती मजय॒न् ना ऽज॑य॒न् जग॑तीम् । \newline
52. अ॒ज॒य॒न् जग॑ती॒म् जग॑ती मजयन् नजय॒न् जग॑ती॒म् छन्द॒ श्छन्दो॒ जग॑ती मजयन् नजय॒न् जग॑ती॒म् छन्दः॑ । \newline
53. जग॑ती॒म् छन्द॒ श्छन्दो॒ जग॑ती॒म् जग॑ती॒म् छन्द॒ स्तत॒ स्तत॒ श्छन्दो॒ जग॑ती॒म् जग॑ती॒म् छन्द॒ स्ततः॑ । \newline
54. छन्द॒ स्तत॒ स्तत॒ श्छन्द॒ श्छन्द॒ स्ततो॒ वै वै तत॒ श्छन्द॒ श्छन्द॒ स्ततो॒ वै । \newline
55. ततो॒ वै वै तत॒ स्ततो॒ वै ते ते वै तत॒ स्ततो॒ वै ते । \newline
56. वै ते ते वै वै ते व्या॒वृतं॑ ॅव्या॒वृत॒म् ते वै वै ते व्या॒वृत᳚म् । \newline
57. ते व्या॒वृतं॑ ॅव्या॒वृत॒म् ते ते व्या॒वृत॑ मगच्छन् नगच्छन् व्या॒वृत॒म् ते ते व्या॒वृत॑ मगच्छन्न् । \newline
58. व्या॒वृत॑ मगच्छन् नगच्छन् व्या॒वृतं॑ ॅव्या॒वृत॑ मगच्छ॒ञ् छ्रैष्ठ्यꣳ॒॒ श्रैष्ठ्य॑ मगच्छन् व्या॒वृतं॑ ॅव्या॒वृत॑ मगच्छ॒ञ् छ्रैष्ठ्य᳚म् । \newline
59. व्या॒वृत॒मिति॑ वि - आ॒वृत᳚म् । \newline
60. अ॒ग॒च्छ॒ञ् छ्रैष्ठ्यꣳ॒॒ श्रैष्ठ्य॑ मगच्छन् नगच्छ॒ञ् छ्रैष्ठ्य॑म् दे॒वाना᳚म् दे॒वानाꣳ॒॒ श्रैष्ठ्य॑ मगच्छन् नगच्छ॒ञ् छ्रैष्ठ्य॑म् दे॒वाना᳚म् । \newline
61. श्रैष्ठ्य॑म् दे॒वाना᳚म् दे॒वानाꣳ॒॒ श्रैष्ठ्यꣳ॒॒ श्रैष्ठ्य॑म् दे॒वाना॒म् तस्मा॒त् तस्मा᳚द् दे॒वानाꣳ॒॒ श्रैष्ठ्यꣳ॒॒ श्रैष्ठ्य॑म् दे॒वाना॒म् तस्मा᳚त् । \newline
62. दे॒वाना॒म् तस्मा॒त् तस्मा᳚द् दे॒वाना᳚म् दे॒वाना॒म् तस्मा॒द् द्वाद॑श॒ द्वाद॑श॒ तस्मा᳚द् दे॒वाना᳚म् दे॒वाना॒म् तस्मा॒द् द्वाद॑श । \newline
63. तस्मा॒द् द्वाद॑श॒ द्वाद॑श॒ तस्मा॒त् तस्मा॒द् द्वाद॑श मा॒सो मा॒सो द्वाद॑श॒ तस्मा॒त् तस्मा॒द् द्वाद॑श मा॒सः । \newline
64. द्वाद॑श मा॒सो मा॒सो द्वाद॑श॒ द्वाद॑श मा॒सो भृ॒त्वा भृ॒त्वा मा॒सो द्वाद॑श॒ द्वाद॑श मा॒सो भृ॒त्वा । \newline
65. मा॒सो भृ॒त्वा भृ॒त्वा मा॒सो मा॒सो भृ॒त्वा ऽग्नि म॒ग्निम् भृ॒त्वा मा॒सो मा॒सो भृ॒त्वा ऽग्निम् । \newline
66. भृ॒त्वा ऽग्नि म॒ग्निम् भृ॒त्वा भृ॒त्वा ऽग्निम् चि॑न्वीत चिन्वीता॒ग्निम् भृ॒त्वा भृ॒त्वा ऽग्निम् चि॑न्वीत । \newline
67. अ॒ग्निम् चि॑न्वीत चिन्वीता॒ग्नि म॒ग्निम् चि॑न्वीत॒ द्वाद॑श॒ द्वाद॑श चिन्वीता॒ग्नि म॒ग्निम् चि॑न्वीत॒ द्वाद॑श । \newline
68. चि॒न्वी॒त॒ द्वाद॑श॒ द्वाद॑श चिन्वीत चिन्वीत॒ द्वाद॑श॒ मासा॒ मासा॒ द्वाद॑श चिन्वीत चिन्वीत॒ द्वाद॑श॒ मासाः᳚ । \newline
69. द्वाद॑श॒ मासा॒ मासा॒ द्वाद॑श॒ द्वाद॑श॒ मासाः᳚ संॅवथ्स॒रः सं॑ॅवथ्स॒रो मासा॒ द्वाद॑श॒ द्वाद॑श॒ मासाः᳚ संॅवथ्स॒रः । \newline
70. मासाः᳚ संॅवथ्स॒रः सं॑ॅवथ्स॒रो मासा॒ मासाः᳚ संॅवथ्स॒रः । \newline
71. सं॒ॅव॒थ्स॒रः सं॑ॅवथ्स॒रः । \newline
72. सं॒ॅव॒थ्स॒र इति॑ सं - व॒थ्स॒रः । \newline
73. सं॒ॅव॒थ्स॒रो᳚ ऽग्नि र॒ग्निः सं॑ॅवथ्स॒रः सं॑ॅवथ्स॒रो᳚ ऽग्नि श्चित्य॒ श्चित्यो॒ ऽग्निः सं॑ॅवथ्स॒रः सं॑ॅवथ्स॒रो᳚ ऽग्नि श्चित्यः॑ । \newline
74. सं॒ॅव॒थ्स॒र इति॑ सं - व॒थ्स॒रः । \newline
75. अ॒ग्नि श्चित्य॒ श्चित्यो॒ ऽग्नि र॒ग्नि श्चित्य॒ स्तस्य॒ तस्य॒ चित्यो॒ ऽग्नि र॒ग्नि श्चित्य॒ स्तस्य॑ । \newline
76. चित्य॒ स्तस्य॒ तस्य॒ चित्य॒ श्चित्य॒ स्तस्या॑ होरा॒त्राण्य॑ होरा॒त्राणि॒ तस्य॒ चित्य॒ श्चित्य॒ स्तस्या॑ होरा॒त्राणि॑ । \newline
77. तस्या॑ होरा॒त्रा ण्य॑होरा॒त्राणि॒ तस्य॒ तस्या॑ होरा॒त्राणी ष्ट॑का॒ इष्ट॑का अहोरा॒त्राणि॒ तस्य॒ तस्या॑ होरा॒त्राणी ष्ट॑काः । \newline
78. अ॒हो॒रा॒त्राणीष्ट॑का॒ इष्ट॑का अहोरा॒त्रा ण्य॑होरा॒त्रा णीष्ट॑का आ॒प्तेष्ट॑क मा॒प्तेष्ट॑क॒ मिष्ट॑का अहोरा॒त्रा ण्य॑होरा॒त्रा णीष्ट॑का आ॒प्तेष्ट॑कम् । \newline
79. अ॒हो॒रा॒त्राणीत्य॑हः - रा॒त्राणि॑ । \newline
80. इष्ट॑का आ॒प्तेष्ट॑क मा॒प्तेष्ट॑क॒ मिष्ट॑का॒ इष्ट॑का आ॒प्तेष्ट॑क मेन मेन मा॒प्तेष्ट॑क॒ मिष्ट॑का॒ इष्ट॑का आ॒प्तेष्ट॑क मेनम् । \newline
81. आ॒प्तेष्ट॑क मेन मेन मा॒प्तेष्ट॑क मा॒प्तेष्ट॑क मेनम् चिनुते चिनुत एन मा॒प्तेष्ट॑क मा॒प्तेष्ट॑क मेनम् चिनुते । \newline
82. आ॒प्तेष्ट॑क॒मित्या॒प्त - इ॒ष्ट॒क॒म् । \newline
83. ए॒न॒म् चि॒नु॒ते॒ चि॒नु॒त॒ ए॒न॒ मे॒न॒म् चि॒नु॒ते ऽथो॒ अथो॑ चिनुत एन मेनम् चिनु॒ते ऽथो᳚ । \newline
84. चि॒नु॒ते ऽथो॒ अथो॑ चिनुते चिनु॒ते ऽथो᳚ व्या॒वृतं॑ ॅव्या॒वृत॒ मथो॑ चिनुते चिनु॒ते ऽथो᳚ व्या॒वृत᳚म् । \newline
85. अथो᳚ व्या॒वृतं॑ ॅव्या॒वृत॒ मथो॒ अथो᳚ व्या॒वृत॑ मे॒वैव व्या॒वृत॒ मथो॒ अथो᳚ व्या॒वृत॑ मे॒व । \newline
86. अथो॒ इत्यथो᳚ । \newline
87. व्या॒वृत॑ मे॒वैव व्या॒वृतं॑ ॅव्या॒वृत॑ मे॒व ग॑च्छति गच्छ त्ये॒व व्या॒वृतं॑ ॅव्या॒वृत॑ मे॒व ग॑च्छति । \newline
88. व्या॒वृत॒मिति॑ वि - आ॒वृत᳚म् । \newline
89. ए॒व ग॑च्छति गच्छ त्ये॒वैव ग॑च्छति॒ श्रैष्ठ्यꣳ॒॒ श्रैष्ठ्य॑म् गच्छ त्ये॒वैव ग॑च्छति॒ श्रैष्ठ्य᳚म् । \newline
90. ग॒च्छ॒ति॒ श्रैष्ठ्यꣳ॒॒ श्रैष्ठ्य॑म् गच्छति गच्छति॒ श्रैष्ठ्यꣳ॑ समा॒नानाꣳ॑ समा॒नानाꣳ॒॒ श्रैष्ठ्य॑म् गच्छति गच्छति॒ श्रैष्ठ्यꣳ॑ समा॒नाना᳚म् । \newline
91. श्रैष्ठ्यꣳ॑ समा॒नानाꣳ॑ समा॒नानाꣳ॒॒ श्रैष्ठ्यꣳ॒॒ श्रैष्ठ्यꣳ॑ समा॒नाना᳚म् । \newline
92. स॒मा॒नाना॒मिति॑ समा॒नाना᳚म् । \newline
\pagebreak
\markright{ TS 5.6.8.1  \hfill https://www.vedavms.in \hfill}

\section{ TS 5.6.8.1 }

\textbf{TS 5.6.8.1 } \newline
\textbf{Samhita Paata} \newline

सु॒व॒र्गाय॒ वा ए॒ष लो॒काय॑ चीयते॒ यद॒ग्निस्तं ॅयन्नान्वा॒रोहे᳚थ् सुव॒र्गाल्लो॒काद्-यज॑मानो हीयेत पृथि॒वीमाऽक्र॑मिषं प्रा॒णो मा॒ मा हा॑सीद॒न्तरि॑क्ष॒माऽक्र॑मिषं प्र॒जा मा॒ मा हा॑सी॒द्-दिव॒माऽक्र॑मिषꣳ॒॒ सुव॑रग॒न्मेत्या॑है॒ष वा अ॒ग्नेर॑न्वारो॒हस्तेनै॒वैन॑-म॒न्वारो॑हति सुव॒र्गस्य॑ लो॒कस्य॒ सम॑ष्ट्यै॒ यत् प॒क्षस॑म्मितां मिनु॒यात् - [  ] \newline

\textbf{Pada Paata} \newline

सु॒व॒र्गायेति॑ सुवः - गाय॑ । वै । ए॒षः । लो॒काय॑ । ची॒य॒ते॒ । यत् । अ॒ग्निः । तम् । यत् । न । अ॒न्वा॒रोहे॒दित्य॑नु - आ॒रोहे᳚त् । सु॒व॒र्गादिति॑ सुवः-गात् । लो॒कात् । यज॑मानः । ही॒ये॒त॒ । पृ॒थि॒वीम् । एति॑ । अ॒क्र॒मि॒ष॒म् । प्रा॒ण इति॑ प्र - अ॒नः । मा॒ । मा । हा॒सी॒त् । अ॒न्तरि॑क्षम् । एति॑ । अ॒क्र॒मि॒ष॒म् । प्र॒जेति॑ प्र - जा । मा॒ । मा । हा॒सी॒त् । दिव᳚म् । एति॑ । अ॒क्र॒मि॒ष॒म् । सुवः॑ । अ॒ग॒न्म॒ । इति॑ । आ॒ह॒ । ए॒षः । वै । अ॒ग्नेः । अ॒न्वा॒रो॒ह इत्य॑नु-आ॒रो॒हः । तेन॑ । ए॒व । ए॒न॒म् । अ॒न्वारो॑ह॒तीत्य॑नु - आरो॑हति । सु॒व॒र्गस्येति॑ सुवः - गस्य॑ । लो॒कस्य॑ । सम॑ष्ट्या॒ इति॒ सं - अ॒ष्ट्यै॒ । यत् । प॒क्षस॑म्मिता॒मिति॑ प॒क्ष - स॒म्मि॒ता॒म् । मि॒नु॒यात् ।  \newline


\textbf{Krama Paata} \newline

सु॒व॒र्गाय॒ वै । सु॒व॒र्गायेति॑ सुवः - गाय॑ । वा ए॒षः । ए॒ष लो॒काय॑ । लो॒काय॑ चीयते । ची॒य॒ते॒ यत् । यद॒ग्निः । अ॒ग्निस्तम् । तम् ॅयत् । यन् न । नान्वा॒रोहे᳚त् । अ॒न्वा॒रोहे᳚थ् सुव॒र्गात् । अ॒न्वा॒रोहे॒दित्य॑नु - आ॒रोहे᳚त् । सु॒व॒र्गाल्लो॒कात् । सु॒व॒र्गादिति॑ सुवः - गात् । लो॒काद् यज॑मानः । यज॑मानो हीयेत । ही॒ये॒त॒ पृ॒थि॒वीम् । पृ॒थि॒वीमा । आऽक्र॑मिषम् । अ॒क्र॒मि॒ष॒म् प्रा॒णः । प्रा॒णो मा᳚ । प्रा॒ण इति॑ प्र - अ॒नः । मा॒ मा । मा हा॑सीत् । हा॒सी॒द॒न्तरि॑क्षम् । अ॒न्तरि॑क्ष॒मा । आऽक्र॑मिषम् । अ॒क्र॒मि॒ष॒म् प्र॒जा । प्र॒जा मा᳚ । प्र॒जेति॑ प्र - जा । मा॒ मा । मा हा॑सीत् । हा॒सी॒द् दिव᳚म् । दिव॒मा । आऽक्र॑मिषम् । अ॒क्र॒मि॒षꣳ॒॒ सुवः॑ । सुव॑रगन्म । अ॒ग॒न्मेति॑ । इत्या॑ह । आ॒है॒षः । ए॒ष वै । वा अ॒ग्नेः । अ॒ग्नेर॑न्वारो॒हः । अ॒न्वा॒रो॒हस्तेन॑ । अ॒न्वा॒रो॒ह इत्य॑नु - आ॒रो॒हः । तेनै॒व । ए॒वैन᳚म् । ए॒न॒म॒न्वारो॑हति । अ॒न्वारो॑हति सुव॒र्गस्य॑ । अ॒न्वारो॑ह॒तीत्य॑नु - आरो॑हति । सु॒व॒र्गस्य॑ लो॒कस्य॑ । सु॒व॒र्गस्येति॑ सुवः - गस्य॑ । लो॒कस्य॒ सम॑ष्ट्यै । सम॑ष्ट्यै॒ यत् । सम॑ष्ट्या॒ इति॒ सम् - अ॒ष्ट्यै॒ । यत् प॒क्षस॑म्मिताम् । प॒क्षस॑म्मिताम् मिनु॒यात् । प॒क्षस॑म्मिता॒मिति॑ प॒क्ष - स॒म्मि॒ता॒म् । मि॒नु॒यात् कनी॑याꣳसम् \newline

\textbf{Jatai Paata} \newline

1. सु॒व॒र्गाय॒ वै वै सु॑व॒र्गाय॑ सुव॒र्गाय॒ वै । \newline
2. सु॒व॒र्गायेति॑ सुवः - गाय॑ । \newline
3. वा ए॒ष ए॒ष वै वा ए॒षः । \newline
4. ए॒ष लो॒काय॑ लो॒का यै॒ष ए॒ष लो॒काय॑ । \newline
5. लो॒काय॑ चीयते चीयते लो॒काय॑ लो॒काय॑ चीयते । \newline
6. ची॒य॒ते॒ यद् यच् ची॑यते चीयते॒ यत् । \newline
7. यद॒ग्नि र॒ग्निर् यद् यद॒ग्निः । \newline
8. अ॒ग्नि स्तम् त म॒ग्नि र॒ग्नि स्तम् । \newline
9. तं ॅयद् यत् तम् तं ॅयत् । \newline
10. यन् न न यद् यन् न । \newline
11. नान्वा॒रोहे॑ दन्वा॒रोहे॒न् न नान्वा॒रोहे᳚त् । \newline
12. अ॒न्वा॒रोहे᳚थ् सुव॒र्गाथ् सु॑व॒र्गा द॑न्वा॒रोहे॑ दन्वा॒रोहे᳚थ् सुव॒र्गात् । \newline
13. अ॒न्वा॒रोहे॒दित्य॑नु - आ॒रोहे᳚त् । \newline
14. सु॒व॒र्गाल् लो॒काल् लो॒काथ् सु॑व॒र्गाथ् सु॑व॒र्गाल् लो॒कात् । \newline
15. सु॒व॒र्गादिति॑ सुवः - गात् । \newline
16. लो॒काद् यज॑मानो॒ यज॑मानो लो॒काल् लो॒काद् यज॑मानः । \newline
17. यज॑मानो हीयेत हीयेत॒ यज॑मानो॒ यज॑मानो हीयेत । \newline
18. ही॒ये॒त॒ पृ॒थि॒वीम् पृ॑थि॒वीꣳ ही॑येत हीयेत पृथि॒वीम् । \newline
19. पृ॒थि॒वी मा पृ॑थि॒वीम् पृ॑थि॒वी मा । \newline
20. आ ऽक्र॑मिष मक्रमिष॒ मा ऽक्र॑मिषम् । \newline
21. अ॒क्र॒मि॒ष॒म् प्रा॒णः प्रा॒णो᳚ ऽक्रमिष मक्रमिषम् प्रा॒णः । \newline
22. प्रा॒णो मा॑ मा प्रा॒णः प्रा॒णो मा᳚ । \newline
23. प्रा॒ण इति॑ प्र - अ॒नः । \newline
24. मा॒ मा मा मा॑ मा॒ मा । \newline
25. मा हा॑सी द्धासी॒न् मा मा हा॑सीत् । \newline
26. हा॒सी॒ द॒न्तरि॑क्ष म॒न्तरि॑क्षꣳ हासी द्धासी द॒न्तरि॑क्षम् । \newline
27. अ॒न्तरि॑क्ष॒ मा ऽन्तरि॑क्ष म॒न्तरि॑क्ष॒ मा । \newline
28. आ ऽक्र॑मिष मक्रमिष॒ मा ऽक्र॑मिषम् । \newline
29. अ॒क्र॒मि॒ष॒म् प्र॒जा प्र॒जा ऽक्र॑मिष मक्रमिषम् प्र॒जा । \newline
30. प्र॒जा मा॑ मा प्र॒जा प्र॒जा मा᳚ । \newline
31. प्र॒जेति॑ प्र - जा । \newline
32. मा॒ मा मा मा॑ मा॒ मा । \newline
33. मा हा॑सी द्धासी॒न् मा मा हा॑सीत् । \newline
34. हा॒सी॒द् दिव॒म् दिवꣳ॑ हासी द्धासी॒द् दिव᳚म् । \newline
35. दिव॒ मा दिव॒म् दिव॒ मा । \newline
36. आ ऽक्र॑मिष मक्रमिष॒ मा ऽक्र॑मिषम् । \newline
37. अ॒क्र॒मि॒षꣳ॒॒ सुवः॒ सुव॑ रक्रमिष मक्रमिषꣳ॒॒ सुवः॑ । \newline
38. सुव॑ रगन्मा गन्म॒ सुवः॒ सुव॑ रगन्म । \newline
39. अ॒ग॒न्मे तीत्य॑ गन्मा ग॒न्मेति॑ । \newline
40. इत्या॑हा॒हे तीत्या॑ह । \newline
41. आ॒है॒ष ए॒ष आ॑हा है॒षः । \newline
42. ए॒ष वै वा ए॒ष ए॒ष वै । \newline
43. वा अ॒ग्ने र॒ग्नेर् वै वा अ॒ग्नेः । \newline
44. अ॒ग्ने र॑न्वारो॒हो᳚ ऽन्वारो॒हो᳚ ऽग्ने र॒ग्ने र॑न्वारो॒हः । \newline
45. अ॒न्वा॒रो॒ह स्तेन॒ तेना᳚ न्वारो॒हो᳚ ऽन्वारो॒ह स्तेन॑ । \newline
46. अ॒न्वा॒रो॒ह इत्य॑नु - आ॒रो॒हः । \newline
47. तेनै॒ वैव तेन॒ तेनै॒व । \newline
48. ए॒वैन॑ मेन मे॒वै वैन᳚म् । \newline
49. ए॒न॒ म॒न्वारो॑ह त्य॒न्वारो॑ह त्येन मेन म॒न्वारो॑हति । \newline
50. अ॒न्वारो॑हति सुव॒र्गस्य॑ सुव॒र्गस्या॒ न्वारो॑ह त्य॒न्वारो॑हति सुव॒र्गस्य॑ । \newline
51. अ॒न्वारो॑ह॒तीत्य॑नु - आरो॑हति । \newline
52. सु॒व॒र्गस्य॑ लो॒कस्य॑ लो॒कस्य॑ सुव॒र्गस्य॑ सुव॒र्गस्य॑ लो॒कस्य॑ । \newline
53. सु॒व॒र्गस्येति॑ सुवः - गस्य॑ । \newline
54. लो॒कस्य॒ सम॑ष्ट्यै॒ सम॑ष्ट्यै लो॒कस्य॑ लो॒कस्य॒ सम॑ष्ट्यै । \newline
55. सम॑ष्ट्यै॒ यद् यथ् सम॑ष्ट्यै॒ सम॑ष्ट्यै॒ यत् । \newline
56. सम॑ष्ट्या॒ इति॒ सं - अ॒ष्ट्यै॒ । \newline
57. यत् प॒क्षस॑म्मिताम् प॒क्षस॑म्मितां॒ ॅयद् यत् प॒क्षस॑म्मिताम् । \newline
58. प॒क्षस॑म्मिताम् मिनु॒यान् मि॑नु॒यात् प॒क्षस॑म्मिताम् प॒क्षस॑म्मिताम् मिनु॒यात् । \newline
59. प॒क्षस॑म्मिता॒मिति॑ प॒क्ष - स॒म्मि॒ता॒म् । \newline
60. मि॒नु॒यात् कनी॑याꣳस॒म् कनी॑याꣳसम् मिनु॒यान् मि॑नु॒यात् कनी॑याꣳसम् । \newline

\textbf{Ghana Paata } \newline

1. सु॒व॒र्गाय॒ वै वै सु॑व॒र्गाय॑ सुव॒र्गाय॒ वा ए॒ष ए॒ष वै सु॑व॒र्गाय॑ सुव॒र्गाय॒ वा ए॒षः । \newline
2. सु॒व॒र्गायेति॑ सुवः - गाय॑ । \newline
3. वा ए॒ष ए॒ष वै वा ए॒ष लो॒काय॑ लो॒कायै॒ष वै वा ए॒ष लो॒काय॑ । \newline
4. ए॒ष लो॒काय॑ लो॒कायै॒ष ए॒ष लो॒काय॑ चीयते चीयते लो॒कायै॒ष ए॒ष लो॒काय॑ चीयते । \newline
5. लो॒काय॑ चीयते चीयते लो॒काय॑ लो॒काय॑ चीयते॒ यद् यच् ची॑यते लो॒काय॑ लो॒काय॑ चीयते॒ यत् । \newline
6. ची॒य॒ते॒ यद् यच् ची॑यते चीयते॒ यद॒ग्नि र॒ग्निर् यच् ची॑यते चीयते॒ यद॒ग्निः । \newline
7. यद॒ग्नि र॒ग्निर् यद् यद॒ग्नि स्तम् त म॒ग्निर् यद् यद॒ग्नि स्तम् । \newline
8. अ॒ग्नि स्तम् त म॒ग्नि र॒ग्नि स्तं ॅयद् यत् त म॒ग्नि र॒ग्नि स्तं ॅयत् । \newline
9. तं ॅयद् यत् तम् तं ॅयन् न न यत् तम् तं ॅयन् न । \newline
10. यन् न न यद् यन् नान्वा॒रोहे॑ दन्वा॒रोहे॒न् न यद् यन् नान्वा॒रोहे᳚त् । \newline
11. नान्वा॒रोहे॑ दन्वा॒रोहे॒न् न नान्वा॒रोहे᳚थ् सुव॒र्गाथ् सु॑व॒र्गा द॑न्वा॒रोहे॒न् न नान्वा॒रोहे᳚थ् सुव॒र्गात् । \newline
12. अ॒न्वा॒रोहे᳚थ् सुव॒र्गाथ् सु॑व॒र्गा द॑न्वा॒रोहे॑ दन्वा॒रोहे᳚थ् सुव॒र्गाल् लो॒काल् लो॒काथ् सु॑व॒र्गा द॑न्वा॒रोहे॑ दन्वा॒रोहे᳚थ् सुव॒र्गाल् लो॒कात् । \newline
13. अ॒न्वा॒रोहे॒दित्य॑नु - आ॒रोहे᳚त् । \newline
14. सु॒व॒र्गाल् लो॒काल् लो॒काथ् सु॑व॒र्गाथ् सु॑व॒र्गाल् लो॒काद् यज॑मानो॒ यज॑मानो लो॒काथ् सु॑व॒र्गाथ् सु॑व॒र्गाल् लो॒काद् यज॑मानः । \newline
15. सु॒व॒र्गादिति॑ सुवः - गात् । \newline
16. लो॒काद् यज॑मानो॒ यज॑मानो लो॒काल् लो॒काद् यज॑मानो हीयेत हीयेत॒ यज॑मानो लो॒काल् लो॒काद् यज॑मानो हीयेत । \newline
17. यज॑मानो हीयेत हीयेत॒ यज॑मानो॒ यज॑मानो हीयेत पृथि॒वीम् पृ॑थि॒वीꣳ ही॑येत॒ यज॑मानो॒ यज॑मानो हीयेत पृथि॒वीम् । \newline
18. ही॒ये॒त॒ पृ॒थि॒वीम् पृ॑थि॒वीꣳ ही॑येत हीयेत पृथि॒वी मा पृ॑थि॒वीꣳ ही॑येत हीयेत पृथि॒वी मा । \newline
19. पृ॒थि॒वी मा पृ॑थि॒वीम् पृ॑थि॒वी मा ऽक्र॑मिष मक्रमिष॒ मा पृ॑थि॒वीम् पृ॑थि॒वी मा ऽक्र॑मिषम् । \newline
20. आ ऽक्र॑मिष मक्रमिष॒ मा ऽक्र॑मिषम् प्रा॒णः प्रा॒णो᳚ ऽक्रमिष॒ मा ऽक्र॑मिषम् प्रा॒णः । \newline
21. अ॒क्र॒मि॒ष॒म् प्रा॒णः प्रा॒णो᳚ ऽक्रमिष मक्रमिषम् प्रा॒णो मा॑ मा प्रा॒णो᳚ ऽक्रमिष मक्रमिषम् प्रा॒णो मा᳚ । \newline
22. प्रा॒णो मा॑ मा प्रा॒णः प्रा॒णो मा॒ मा मा मा᳚ प्रा॒णः प्रा॒णो मा॒ मा । \newline
23. प्रा॒ण इति॑ प्र - अ॒नः । \newline
24. मा॒ मा मा मा॑ मा॒ मा हा॑सी द्धासी॒न् मा मा॑ मा॒ मा हा॑सीत् । \newline
25. मा हा॑सी द्धासी॒न् मा मा हा॑सी द॒न्तरि॑क्ष म॒न्तरि॑क्षꣳ हासी॒न् मा मा हा॑सी द॒न्तरि॑क्षम् । \newline
26. हा॒सी॒ द॒न्तरि॑क्ष म॒न्तरि॑क्षꣳ हासी द्धासी द॒न्तरि॑क्ष॒ मा ऽन्तरि॑क्षꣳ हासी द्धासी द॒न्तरि॑क्ष॒ मा । \newline
27. अ॒न्तरि॑क्ष॒ मा ऽन्तरि॑क्ष म॒न्तरि॑क्ष॒ मा ऽक्र॑मिष मक्रमिष॒ मा ऽन्तरि॑क्ष म॒न्तरि॑क्ष॒ मा ऽक्र॑मिषम् । \newline
28. आ ऽक्र॑मिष मक्रमिष॒ मा ऽक्र॑मिषम् प्र॒जा प्र॒जा ऽक्र॑मिष॒ मा ऽक्र॑मिषम् प्र॒जा । \newline
29. अ॒क्र॒मि॒ष॒म् प्र॒जा प्र॒जा ऽक्र॑मिष मक्रमिषम् प्र॒जा मा॑ मा प्र॒जा ऽक्र॑मिष मक्रमिषम् प्र॒जा मा᳚ । \newline
30. प्र॒जा मा॑ मा प्र॒जा प्र॒जा मा॒ मा मा मा᳚ प्र॒जा प्र॒जा मा॒ मा । \newline
31. प्र॒जेति॑ प्र - जा । \newline
32. मा॒ मा मा मा॑ मा॒ मा हा॑सी द्धासी॒न् मा मा॑ मा॒ मा हा॑सीत् । \newline
33. मा हा॑सी द्धासी॒न् मा मा हा॑सी॒द् दिव॒म् दिवꣳ॑ हासी॒न् मा मा हा॑सी॒द् दिव᳚म् । \newline
34. हा॒सी॒द् दिव॒म् दिवꣳ॑ हासी द्धासी॒द् दिव॒ मा दिवꣳ॑ हासी द्धासी॒द् दिव॒ मा । \newline
35. दिव॒ मा दिव॒म् दिव॒ मा ऽक्र॑मिष मक्रमिष॒ मा दिव॒म् दिव॒ मा ऽक्र॑मिषम् । \newline
36. आ ऽक्र॑मिष मक्रमिष॒ मा ऽक्र॑मिषꣳ॒॒ सुवः॒ सुव॑ रक्रमिष॒ मा ऽक्र॑मिषꣳ॒॒ सुवः॑ । \newline
37. अ॒क्र॒मि॒षꣳ॒॒ सुवः॒ सुव॑ रक्रमिष मक्रमिषꣳ॒॒ सुव॑ रगन्मा गन्म॒ सुव॑ रक्रमिष मक्रमिषꣳ॒॒ सुव॑ रगन्म । \newline
38. सुव॑ रगन्मा गन्म॒ सुवः॒ सुव॑ रग॒न्मे तीत्य॑गन्म॒ सुवः॒ सुव॑ रग॒न्मेति॑ । \newline
39. अ॒ग॒न्मे तीत्य॑गन्मा ग॒न्मे त्या॑हा॒हे त्य॑गन्मा ग॒न्मे त्या॑ह । \newline
40. इत्या॑हा॒हे तीत्या॑ है॒ष ए॒ष आ॒हे तीत्या॑ है॒षः । \newline
41. आ॒है॒ष ए॒ष आ॑हाहै॒ष वै वा ए॒ष आ॑हाहै॒ष वै । \newline
42. ए॒ष वै वा ए॒ष ए॒ष वा अ॒ग्ने र॒ग्नेर् वा ए॒ष ए॒ष वा अ॒ग्नेः । \newline
43. वा अ॒ग्ने र॒ग्नेर् वै वा अ॒ग्ने र॑न्वारो॒हो᳚ ऽन्वारो॒हो᳚ ऽग्नेर् वै वा अ॒ग्ने र॑न्वारो॒हः । \newline
44. आ॒ग्ने र॑न्वारो॒हो᳚ ऽन्वारो॒हो᳚ ऽग्ने र॒ग्ने र॑न्वारो॒ह स्तेन॒ तेना᳚ न्वारो॒हो᳚ ऽग्ने र॒ग्ने र॑न्वारो॒ह स्तेन॑ । \newline
45. अ॒न्वा॒रो॒ह स्तेन॒ तेना᳚ न्वारो॒हो᳚ ऽन्वारो॒ह स्तेनै॒वैव तेना᳚ न्वारो॒हो᳚ ऽन्वारो॒ह स्तेनै॒व । \newline
46. अ॒न्वा॒रो॒ह इत्य॑नु - आ॒रो॒हः । \newline
47. तेनै॒ वैव तेन॒ तेनै॒ वैन॑ मेन मे॒व तेन॒ तेनै॒ वैन᳚म् । \newline
48. ए॒वैन॑ मेन मे॒वै वैन॑ म॒न्वारो॑ह त्य॒न्वारो॑ह त्येन मे॒वै वैन॑ म॒न्वारो॑हति । \newline
49. ए॒न॒ म॒न्वारो॑ह त्य॒न्वारो॑ह त्येन मेन म॒न्वारो॑हति सुव॒र्गस्य॑ सुव॒र्गस्या॒ न्वारो॑ह त्येन मेन म॒न्वारो॑हति सुव॒र्गस्य॑ । \newline
50. अ॒न्वारो॑हति सुव॒र्गस्य॑ सुव॒र्गस्या॒ न्वारो॑ह त्य॒न्वारो॑हति सुव॒र्गस्य॑ लो॒कस्य॑ लो॒कस्य॑ सुव॒र्गस्या॒ न्वारो॑ह त्य॒न्वारो॑हति सुव॒र्गस्य॑ लो॒कस्य॑ । \newline
51. अ॒न्वारो॑ह॒तीत्य॑नु - आरो॑हति । \newline
52. सु॒व॒र्गस्य॑ लो॒कस्य॑ लो॒कस्य॑ सुव॒र्गस्य॑ सुव॒र्गस्य॑ लो॒कस्य॒ सम॑ष्ट्यै॒ सम॑ष्ट्यै लो॒कस्य॑ सुव॒र्गस्य॑ सुव॒र्गस्य॑ लो॒कस्य॒ सम॑ष्ट्यै । \newline
53. सु॒व॒र्गस्येति॑ सुवः - गस्य॑ । \newline
54. लो॒कस्य॒ सम॑ष्ट्यै॒ सम॑ष्ट्यै लो॒कस्य॑ लो॒कस्य॒ सम॑ष्ट्यै॒ यद् यथ् सम॑ष्ट्यै लो॒कस्य॑ लो॒कस्य॒ सम॑ष्ट्यै॒ यत् । \newline
55. सम॑ष्ट्यै॒ यद् यथ् सम॑ष्ट्यै॒ सम॑ष्ट्यै॒ यत् प॒क्षस॑म्मिताम् प॒क्षस॑म्मितां॒ ॅयथ् सम॑ष्ट्यै॒ सम॑ष्ट्यै॒ यत् प॒क्षस॑म्मिताम् । \newline
56. सम॑ष्ट्या॒ इति॒ सं - अ॒ष्ट्यै॒ । \newline
57. यत् प॒क्षस॑म्मिताम् प॒क्षस॑म्मितां॒ ॅयद् यत् प॒क्षस॑म्मिताम् मिनु॒यान् मि॑नु॒यात् प॒क्षस॑म्मितां॒ ॅयद् यत् प॒क्षस॑म्मिताम् मिनु॒यात् । \newline
58. प॒क्षस॑म्मिताम् मिनु॒यान् मि॑नु॒यात् प॒क्षस॑म्मिताम् प॒क्षस॑म्मिताम् मिनु॒यात् कनी॑याꣳस॒म् कनी॑याꣳसम् मिनु॒यात् प॒क्षस॑म्मिताम् प॒क्षस॑म्मिताम् मिनु॒यात् कनी॑याꣳसम् । \newline
59. प॒क्षस॑म्मिता॒मिति॑ प॒क्ष - स॒म्मि॒ता॒म् । \newline
60. मि॒नु॒यात् कनी॑याꣳस॒म् कनी॑याꣳसम् मिनु॒यान् मि॑नु॒यात् कनी॑याꣳसं ॅयज्ञ्क्र॒तुं ॅय॑ज्ञ्क्र॒तुम् कनी॑याꣳसम् मिनु॒यान् मि॑नु॒यात् कनी॑याꣳसं ॅयज्ञ्क्र॒तुम् । \newline
\pagebreak
\markright{ TS 5.6.8.2  \hfill https://www.vedavms.in \hfill}

\section{ TS 5.6.8.2 }

\textbf{TS 5.6.8.2 } \newline
\textbf{Samhita Paata} \newline

कनी॑याꣳसं ॅयज्ञ्क्र॒तुमुपे॑या॒त् पापी॑यस्यस्या॒ ऽऽत्मनः॑ प्र॒जा स्या॒द्-वेदि॑सम्मितां मिनोति॒ ज्यायाꣳ॑समे॒व य॑ज्ञ्क्र॒तुमुपै॑ति॒ नास्या॒ऽऽ*त्मनः॒ पापी॑यसी प्र॒जा भ॑वति साह॒स्रं चि॑न्वीत प्रथ॒मं चि॑न्वा॒नः स॒हस्र॑सम्मितो॒ वा अ॒यं ॅलो॒क इ॒ममे॒व लो॒कम॒भि ज॑यति॒ द्विषा॑हस्रं चिन्वीत द्वि॒तीयं॑ चिन्वा॒नो द्विषा॑हस्रं॒ ॅवा अ॒न्तरि॑क्ष-म॒न्तरि॑क्षमे॒वाभि ज॑यति॒ त्रिषा॑हस्रं चिन्वीत तृ॒तीयं॑ चिन्वा॒न - [  ] \newline

\textbf{Pada Paata} \newline

कनी॑याꣳसम् । य॒ज्ञ्॒क्र॒तुमिति॑ यज्ञ् - क्र॒तुम् । उपेति॑ । इ॒या॒त् । पापी॑यसी । अ॒स्य॒ । आ॒त्मनः॑ । प्र॒जेति॑ प्र - जा । स्या॒त् । वेदि॑सम्मिता॒मिति॒ वेदि॑ - स॒म्मि॒ता॒म् । मि॒नो॒ति॒ । ज्यायाꣳ॑सम् । ए॒व । य॒ज्ञ्॒क्र॒तुमिति॑ यज्ञ् - क्र॒तुम् । उपेति॑ । ए॒ति॒ । न । अ॒स्य॒ । आ॒त्मनः॑ । पापी॑यसी । प्र॒जेति॑ प्र - जा । भ॒व॒ति॒ । सा॒ह॒स्रम् । चि॒न्वी॒त॒ । प्र॒थ॒मम् । चि॒न्वा॒नः । स॒हस्र॑सम्मित॒ इति॑ स॒हस्र॑ - स॒म्मि॒तः॒ । वै । अ॒यम् । लो॒कः ।   इ॒मम् । ए॒व । लो॒कम् । अ॒भीति॑ । ज॒य॒ति॒ । द्विषा॑हस्र॒मिति॒ द्वि - सा॒ह॒स्र॒म् । चि॒न्वी॒त॒ । द्वि॒तीय᳚म् । चि॒न्वा॒नः । द्विषा॑हस्र॒मिति॒ द्वि - सा॒ह॒स्र॒म् । वै । अ॒न्तरि॑क्षम् । अ॒न्तरि॑क्षम् । ए॒व । अ॒भीति॑ । ज॒य॒ति॒ । त्रिषा॑हस्र॒मिति॒ त्रि - सा॒ह॒स्र॒म् । चि॒न्वी॒त॒ । तृ॒तीय᳚म् । चि॒न्वा॒नः ।  \newline


\textbf{Krama Paata} \newline

कनी॑याꣳसम् ॅयज्ञ्क्र॒तुम् । य॒ज्ञ्॒क्र॒तुमुप॑ । य॒ज्ञ्॒क्र॒तुमिति॑ यज्ञ् - क्र॒तुम् । उपे॑यात् । इ॒या॒त् पापी॑यसी । पापी॑यस्यस्य । अ॒स्या॒त्मनः॑ । आ॒त्मनः॑ प्र॒जा । प्र॒जा स्या᳚त् । प्र॒जेति॑ प्र - जा । स्या॒द् वेदि॑सम्मिताम् । वेदि॑सम्मिताम् मिनोति । वेदि॑सम्मिता॒मिति॒ वेदि॑ - स॒म्मि॒ता॒म् । मि॒नो॒ति॒ ज्यायाꣳ॑सम् । ज्यायाꣳ॑समे॒व । ए॒व य॑ज्ञ्क्र॒तुम् । य॒ज्ञ्॒क्र॒तुमुप॑ । य॒ज्ञ्॒क्र॒तुमिति॑ यज्ञ् - क्र॒तुम् । उपै॑ति । ए॒ति॒ न । नास्य॑ । अ॒स्या॒त्मनः॑ । आ॒त्मनः॒ पापी॑यसी । पापी॑यसी प्र॒जा । प्र॒जा भ॑वति । प्र॒जेति॑ प्र - जा । भ॒व॒ति॒ सा॒ह॒स्रम् । सा॒ह॒स्रम् चि॑न्वीत । चि॒न्वी॒त॒ प्र॒थ॒मम् । प्र॒थ॒मम् चि॑न्वा॒नः । चि॒न्वा॒नः स॒हस्र॑सम्मितः । स॒हस्र॑सम्मितो॒ वै । स॒हस्र॑सम्मित॒ इति॑ स॒हस्र॑ - स॒म्मि॒तः॒ । वा अ॒यम् । अ॒यम् ॅलो॒कः । लो॒क इ॒मम् । इ॒ममे॒व । ए॒व लो॒कम् । लो॒कम॒भि । अ॒भि ज॑यति । ज॒य॒ति॒ द्विषा॑हस्रम् । द्विषा॑हस्रम् चिन्वीत । द्विषा॑हस्र॒मिति॒ द्वि - सा॒ह॒स्र॒म् । चि॒न्वी॒त॒ द्वि॒तीय᳚म् । द्वि॒तीय॑म् चिन्वा॒नः । चि॒न्वा॒नो द्विषा॑हस्रम् । द्विषा॑हस्र॒म् ॅवै । द्विषा॑हस्र॒मिति॒ द्वि - सा॒ह॒स्र॒म् । वा अ॒न्तरि॑क्षम् । अ॒न्तरि॑क्षम॒न्तरि॑क्षम् । अ॒न्तरि॑क्षमे॒व । ए॒वाभि । अ॒भि ज॑यति । ज॒य॒ति॒ त्रिषा॑हस्रम् । त्रिषा॑हस्रम् चिन्वीत । त्रिषा॑हस्र॒मिति॒ त्रि - सा॒ह॒स्र॒म् । चि॒न्वी॒त॒ तृ॒तीय᳚म् । तृ॒तीय॑म् चिन्वा॒नः । चि॒न्वा॒नस्त्रिषा॑हस्रः \newline

\textbf{Jatai Paata} \newline

1. कनी॑याꣳसं ॅयज्ञ्क्र॒तुं ॅय॑ज्ञ्क्र॒तुम् कनी॑याꣳस॒म् कनी॑याꣳसं ॅयज्ञ्क्र॒तुम् । \newline
2. य॒ज्ञ्॒क्र॒तु मुपोप॑ यज्ञ्क्र॒तुं ॅय॑ज्ञ्क्र॒तु मुप॑ । \newline
3. य॒ज्ञ्॒क्र॒तुमिति॑ यज्ञ् - क्र॒तुम् । \newline
4. उपे॑ या दिया॒ दुपोपे॑ यात् । \newline
5. इ॒या॒त् पापी॑यसी॒ पापी॑यसीया दिया॒त् पापी॑यसी । \newline
6. पापी॑यस्य स्यास्य॒ पापी॑यसी॒ पापी॑यस्यस्य । \newline
7. अ॒स्या॒ त्मन॑ आ॒त्मनो᳚ ऽस्या स्या॒त्मनः॑ । \newline
8. आ॒त्मनः॑ प्र॒जा प्र॒जा ऽऽत्मन॑ आ॒त्मनः॑ प्र॒जा । \newline
9. प्र॒जा स्या᳚थ् स्यात् प्र॒जा प्र॒जा स्या᳚त् । \newline
10. प्र॒जेति॑ प्र - जा । \newline
11. स्या॒द् वेदि॑सम्मितां॒ ॅवेदि॑सम्मिताꣳ स्याथ् स्या॒द् वेदि॑सम्मिताम् । \newline
12. वेदि॑सम्मिताम् मिनोति मिनोति॒ वेदि॑सम्मितां॒ ॅवेदि॑सम्मिताम् मिनोति । \newline
13. वेदि॑सम्मिता॒मिति॒ वेदि॑ - स॒म्मि॒ता॒म् । \newline
14. मि॒नो॒ति॒ ज्यायाꣳ॑स॒म् ज्यायाꣳ॑सम् मिनोति मिनोति॒ ज्यायाꣳ॑सम् । \newline
15. ज्यायाꣳ॑स मे॒वैव ज्यायाꣳ॑स॒म् ज्यायाꣳ॑स मे॒व । \newline
16. ए॒व य॑ज्ञ्क्र॒तुं ॅय॑ज्ञ्क्र॒तु मे॒वैव य॑ज्ञ्क्र॒तुम् । \newline
17. य॒ज्ञ्॒क्र॒तु मुपोप॑ यज्ञ्क्र॒तुं ॅय॑ज्ञ्क्र॒तु मुप॑ । \newline
18. य॒ज्ञ्॒क्र॒तुमिति॑ यज्ञ् - क्र॒तुम् । \newline
19. उपै᳚त्ये॒ त्युपोपै॑ति । \newline
20. ए॒ति॒ न नैत्ये॑ति॒ न । \newline
21. नास्या᳚स्य॒ न नास्य॑ । \newline
22. अ॒स्या॒त्मन॑ आ॒त्मनो᳚ ऽस्या स्या॒त्मनः॑ । \newline
23. आ॒त्मनः॒ पापी॑यसी॒ पापी॑य स्या॒त्मन॑ आ॒त्मनः॒ पापी॑यसी । \newline
24. पापी॑यसी प्र॒जा प्र॒जा पापी॑यसी॒ पापी॑यसी प्र॒जा । \newline
25. प्र॒जा भ॑वति भवति प्र॒जा प्र॒जा भ॑वति । \newline
26. प्र॒जेति॑ प्र - जा । \newline
27. भ॒व॒ति॒ सा॒ह॒स्रꣳ सा॑ह॒स्रम् भ॑वति भवति साह॒स्रम् । \newline
28. सा॒ह॒स्रम् चि॑न्वीत चिन्वीत साह॒स्रꣳ सा॑ह॒स्रम् चि॑न्वीत । \newline
29. चि॒न्वी॒त॒ प्र॒थ॒मम् प्र॑थ॒मम् चि॑न्वीत चिन्वीत प्रथ॒मम् । \newline
30. प्र॒थ॒मम् चि॑न्वा॒न श्चि॑न्वा॒नः प्र॑थ॒मम् प्र॑थ॒मम् चि॑न्वा॒नः । \newline
31. चि॒न्वा॒नः स॒हस्र॑सम्मितः स॒हस्र॑सम्मित श्चिन्वा॒न श्चि॑न्वा॒नः स॒हस्र॑सम्मितः । \newline
32. स॒हस्र॑सम्मितो॒ वै वै स॒हस्र॑सम्मितः स॒हस्र॑सम्मितो॒ वै । \newline
33. स॒हस्र॑सम्मित॒ इति॑ स॒हस्र॑ - स॒म्मि॒तः॒ । \newline
34. वा अ॒य म॒यं ॅवै वा अ॒यम् । \newline
35. अ॒यम् ॅलो॒को लो॒को॑ ऽय म॒यम् ॅलो॒कः । \newline
36. लो॒क इ॒म मि॒मम् ॅलो॒को लो॒क इ॒मम् । \newline
37. इ॒म मे॒वै वेम मि॒म मे॒व । \newline
38. ए॒व लो॒कम् ॅलो॒क मे॒वैव लो॒कम् । \newline
39. लो॒क म॒भ्य॑भि लो॒कम् ॅलो॒क म॒भि । \newline
40. अ॒भि ज॑यति जय त्य॒भ्य॑भि ज॑यति । \newline
41. ज॒य॒ति॒ द्विषा॑हस्र॒म् द्विषा॑हस्रम् जयति जयति॒ द्विषा॑हस्रम् । \newline
42. द्विषा॑हस्रम् चिन्वीत चिन्वीत॒ द्विषा॑हस्र॒म् द्विषा॑हस्रम् चिन्वीत । \newline
43. द्विषा॑हस्र॒मिति॒ द्वि - सा॒ह॒स्र॒म् । \newline
44. चि॒न्वी॒त॒ द्वि॒तीय॑म् द्वि॒तीय॑म् चिन्वीत चिन्वीत द्वि॒तीय᳚म् । \newline
45. द्वि॒तीय॑म् चिन्वा॒न श्चि॑न्वा॒नो द्वि॒तीय॑म् द्वि॒तीय॑म् चिन्वा॒नः । \newline
46. चि॒न्वा॒नो द्विषा॑हस्र॒म् द्विषा॑हस्रम् चिन्वा॒न श्चि॑न्वा॒नो द्विषा॑हस्रम् । \newline
47. द्विषा॑हस्रं॒ ॅवै वै द्विषा॑हस्र॒म् द्विषा॑हस्रं॒ ॅवै । \newline
48. द्विषा॑हस्र॒मिति॒ द्वि - सा॒ह॒स्र॒म् । \newline
49. वा अ॒न्तरि॑क्ष म॒न्तरि॑क्षं॒ ॅवै वा अ॒न्तरि॑क्षम् । \newline
50. अ॒न्तरि॑क्ष म॒न्तरि॑क्षम् । \newline
51. अ॒न्तरि॑क्ष मे॒वै वान्तरि॑क्ष म॒न्तरि॑क्ष मे॒व । \newline
52. ए॒वाभ्या᳚(1॒) भ्ये॑वै वाभि । \newline
53. अ॒भि ज॑यति जय त्य॒भ्य॑भि ज॑यति । \newline
54. ज॒य॒ति॒ त्रिषा॑हस्र॒म् त्रिषा॑हस्रम् जयति जयति॒ त्रिषा॑हस्रम् । \newline
55. त्रिषा॑हस्रम् चिन्वीत चिन्वीत॒ त्रिषा॑हस्र॒म् त्रिषा॑हस्रम् चिन्वीत । \newline
56. त्रिषा॑हस्र॒मिति॒ त्रि - सा॒ह॒स्र॒म् । \newline
57. चि॒न्वी॒त॒ तृ॒तीय॑म् तृ॒तीय॑म् चिन्वीत चिन्वीत तृ॒तीय᳚म् । \newline
58. तृ॒तीय॑म् चिन्वा॒न श्चि॑न्वा॒न स्तृ॒तीय॑म् तृ॒तीय॑म् चिन्वा॒नः । \newline
59. चि॒न्वा॒न स्त्रिषा॑हस्र॒ स्त्रिषा॑हस्र श्चिन्वा॒न श्चि॑न्वा॒न स्त्रिषा॑हस्रः । \newline

\textbf{Ghana Paata } \newline

1. कनी॑याꣳसं ॅयज्ञ्क्र॒तुं ॅय॑ज्ञ्क्र॒तुम् कनी॑याꣳस॒म् कनी॑याꣳसं ॅयज्ञ्क्र॒तु मुपोप॑ यज्ञ्क्र॒तुम् कनी॑याꣳस॒म् कनी॑याꣳसं ॅयज्ञ्क्र॒तु मुप॑ । \newline
2. य॒ज्ञ्॒क्र॒तु मुपोप॑ यज्ञ्क्र॒तुं ॅय॑ज्ञ्क्र॒तु मुपे॑ यादिया॒दुप॑ यज्ञ्क्र॒तुं ॅय॑ज्ञ्क्र॒तु मुपे॑ यात् । \newline
3. य॒ज्ञ्॒क्र॒तुमिति॑ यज्ञ् - क्र॒तुम् । \newline
4. उपे॑ यादिया॒ दुपोपे॑ या॒त् पापी॑यसी॒ पापी॑यसीया॒ दुपोपे॑ या॒त् पापी॑यसी । \newline
5. इ॒या॒त् पापी॑यसी॒ पापी॑यसीया दिया॒त् पापी॑यस्य स्यास्य॒ पापी॑यसीया दिया॒त् पापी॑यस्यस्य । \newline
6. पापी॑यस्य स्यास्य॒ पापी॑यसी॒ पापी॑यस्य स्या॒त्मन॑ आ॒त्मनो᳚ ऽस्य॒ पापी॑यसी॒ पापी॑यस्य स्या॒त्मनः॑ । \newline
7. अ॒स्या॒त्मन॑ आ॒त्मनो᳚ ऽस्यास्या॒त्मनः॑ प्र॒जा प्र॒जा ऽऽत्मनो᳚ ऽस्या स्या॒त्मनः॑ प्र॒जा । \newline
8. आ॒त्मनः॑ प्र॒जा प्र॒जा ऽऽत्मन॑ आ॒त्मनः॑ प्र॒जा स्या᳚थ् स्यात् प्र॒जा ऽऽत्मन॑ आ॒त्मनः॑ प्र॒जा स्या᳚त् । \newline
9. प्र॒जा स्या᳚थ् स्यात् प्र॒जा प्र॒जा स्या॒द् वेदि॑सम्मितां॒ ॅवेदि॑सम्मिताꣳ स्यात् प्र॒जा प्र॒जा स्या॒द् वेदि॑सम्मिताम् । \newline
10. प्र॒जेति॑ प्र - जा । \newline
11. स्या॒द् वेदि॑सम्मितां॒ ॅवेदि॑सम्मिताꣳ स्याथ् स्या॒द् वेदि॑सम्मिताम् मिनोति मिनोति॒ वेदि॑सम्मिताꣳ स्याथ् स्या॒द् वेदि॑सम्मिताम् मिनोति । \newline
12. वेदि॑सम्मिताम् मिनोति मिनोति॒ वेदि॑सम्मितां॒ ॅवेदि॑सम्मिताम् मिनोति॒ ज्यायाꣳ॑स॒म् ज्यायाꣳ॑सम् मिनोति॒ वेदि॑सम्मितां॒ ॅवेदि॑सम्मिताम् मिनोति॒ ज्यायाꣳ॑सम् । \newline
13. वेदि॑सम्मिता॒मिति॒ वेदि॑ - स॒म्मि॒ता॒म् । \newline
14. मि॒नो॒ति॒ ज्यायाꣳ॑स॒म् ज्यायाꣳ॑सम् मिनोति मिनोति॒ ज्यायाꣳ॑स मे॒वैव ज्यायाꣳ॑सम् मिनोति मिनोति॒ ज्यायाꣳ॑स मे॒व । \newline
15. ज्यायाꣳ॑स मे॒वैव ज्यायाꣳ॑स॒म् ज्यायाꣳ॑स मे॒व य॑ज्ञ्क्र॒तुं ॅय॑ज्ञ्क्र॒तु मे॒व ज्यायाꣳ॑स॒म् ज्यायाꣳ॑स मे॒व य॑ज्ञ्क्र॒तुम् । \newline
16. ए॒व य॑ज्ञ्क्र॒तुं ॅय॑ज्ञ्क्र॒तु मे॒वैव य॑ज्ञ्क्र॒तु मुपोप॑ यज्ञ्क्र॒तु मे॒वैव य॑ज्ञ्क्र॒तु मुप॑ । \newline
17. य॒ज्ञ्॒क्र॒तु मुपोप॑ यज्ञ्क्र॒तुं ॅय॑ज्ञ्क्र॒तु मुपै᳚ त्ये॒त्युप॑ यज्ञ्क्र॒तुं ॅय॑ज्ञ्क्र॒तु मुपै॑ति । \newline
18. य॒ज्ञ्॒क्र॒तुमिति॑ यज्ञ् - क्र॒तुम् । \newline
19. उपै᳚त्ये॒ त्युपो पै॑ति॒ न नैत्युपो पै॑ति॒ न । \newline
20. ए॒ति॒ न नैत्ये॑ति॒ नास्या᳚स्य॒ नैत्ये॑ति॒ नास्य॑ । \newline
21. नास्या᳚स्य॒ न नास्या॒त्मन॑ आ॒त्मनो᳚ ऽस्य॒ न नास्या॒त्मनः॑ । \newline
22. अ॒स्या॒त्मन॑ आ॒त्मनो᳚ ऽस्या स्या॒त्मनः॒ पापी॑यसी॒ पापी॑य स्या॒त्मनो᳚ ऽस्या स्या॒त्मनः॒ पापी॑यसी । \newline
23. आ॒त्मनः॒ पापी॑यसी॒ पापी॑य स्या॒त्मन॑ आ॒त्मनः॒ पापी॑यसी प्र॒जा प्र॒जा पापी॑य स्या॒त्मन॑ आ॒त्मनः॒ पापी॑यसी प्र॒जा । \newline
24. पापी॑यसी प्र॒जा प्र॒जा पापी॑यसी॒ पापी॑यसी प्र॒जा भ॑वति भवति प्र॒जा पापी॑यसी॒ पापी॑यसी प्र॒जा भ॑वति । \newline
25. प्र॒जा भ॑वति भवति प्र॒जा प्र॒जा भ॑वति साह॒स्रꣳ सा॑ह॒स्रम् भ॑वति प्र॒जा प्र॒जा भ॑वति साह॒स्रम् । \newline
26. प्र॒जेति॑ प्र - जा । \newline
27. भ॒व॒ति॒ सा॒ह॒स्रꣳ सा॑ह॒स्रम् भ॑वति भवति साह॒स्रम् चि॑न्वीत चिन्वीत साह॒स्रम् भ॑वति भवति साह॒स्रम् चि॑न्वीत । \newline
28. सा॒ह॒स्रम् चि॑न्वीत चिन्वीत साह॒स्रꣳ सा॑ह॒स्रम् चि॑न्वीत प्रथ॒मम् प्र॑थ॒मम् चि॑न्वीत साह॒स्रꣳ सा॑ह॒स्रम् चि॑न्वीत प्रथ॒मम् । \newline
29. चि॒न्वी॒त॒ प्र॒थ॒मम् प्र॑थ॒मम् चि॑न्वीत चिन्वीत प्रथ॒मम् चि॑न्वा॒न श्चि॑न्वा॒नः प्र॑थ॒मम् चि॑न्वीत चिन्वीत प्रथ॒मम् चि॑न्वा॒नः । \newline
30. प्र॒थ॒मम् चि॑न्वा॒न श्चि॑न्वा॒नः प्र॑थ॒मम् प्र॑थ॒मम् चि॑न्वा॒नः स॒हस्र॑सम्मितः स॒हस्र॑सम्मित श्चिन्वा॒नः प्र॑थ॒मम् प्र॑थ॒मम् चि॑न्वा॒नः स॒हस्र॑सम्मितः । \newline
31. चि॒न्वा॒नः स॒हस्र॑सम्मितः स॒हस्र॑सम्मित श्चिन्वा॒न श्चि॑न्वा॒नः स॒हस्र॑सम्मितो॒ वै वै स॒हस्र॑सम्मित श्चिन्वा॒न श्चि॑न्वा॒नः स॒हस्र॑सम्मितो॒ वै । \newline
32. स॒हस्र॑सम्मितो॒ वै वै स॒हस्र॑सम्मितः स॒हस्र॑सम्मितो॒ वा अ॒य म॒यं ॅवै स॒हस्र॑सम्मितः स॒हस्र॑सम्मितो॒ वा अ॒यम् । \newline
33. स॒हस्र॑सम्मित॒ इति॑ स॒हस्र॑ - स॒म्मि॒तः॒ । \newline
34. वा अ॒य म॒यं ॅवै वा अ॒यम् ॅलो॒को लो॒को॑ ऽयं ॅवै वा अ॒यम् ॅलो॒कः । \newline
35. अ॒यम् ॅलो॒को लो॒को॑ ऽय म॒यम् ॅलो॒क इ॒म मि॒मम् ॅलो॒को॑ ऽय म॒यम् ॅलो॒क इ॒मम् । \newline
36. लो॒क इ॒म मि॒मम् ॅलो॒को लो॒क इ॒म मे॒वैवेमम् ॅलो॒को लो॒क इ॒म मे॒व । \newline
37. इ॒म मे॒वैवेम मि॒म मे॒व लो॒कम् ॅलो॒क मे॒वेम मि॒म मे॒व लो॒कम् । \newline
38. ए॒व लो॒कम् ॅलो॒क मे॒वैव लो॒क म॒भ्य॑भि लो॒क मे॒वैव लो॒क म॒भि । \newline
39. लो॒क म॒भ्य॑भि लो॒कम् ॅलो॒क म॒भि ज॑यति जय त्य॒भि लो॒कम् ॅलो॒क म॒भि ज॑यति । \newline
40. अ॒भि ज॑यति जय त्य॒भ्य॑भि ज॑यति॒ द्विषा॑हस्र॒म् द्विषा॑हस्रम् जय त्य॒भ्य॑भि ज॑यति॒ द्विषा॑हस्रम् । \newline
41. ज॒य॒ति॒ द्विषा॑हस्र॒म् द्विषा॑हस्रम् जयति जयति॒ द्विषा॑हस्रम् चिन्वीत चिन्वीत॒ द्विषा॑हस्रम् जयति जयति॒ द्विषा॑हस्रम् चिन्वीत । \newline
42. द्विषा॑हस्रम् चिन्वीत चिन्वीत॒ द्विषा॑हस्र॒म् द्विषा॑हस्रम् चिन्वीत द्वि॒तीय॑म् द्वि॒तीय॑म् चिन्वीत॒ द्विषा॑हस्र॒म् द्विषा॑हस्रम् चिन्वीत द्वि॒तीय᳚म् । \newline
43. द्विषा॑हस्र॒मिति॒ द्वि - सा॒ह॒स्र॒म् । \newline
44. चि॒न्वी॒त॒ द्वि॒तीय॑म् द्वि॒तीय॑म् चिन्वीत चिन्वीत द्वि॒तीय॑म् चिन्वा॒न श्चि॑न्वा॒नो द्वि॒तीय॑म् चिन्वीत चिन्वीत द्वि॒तीय॑म् चिन्वा॒नः । \newline
45. द्वि॒तीय॑म् चिन्वा॒न श्चि॑न्वा॒नो द्वि॒तीय॑म् द्वि॒तीय॑म् चिन्वा॒नो द्विषा॑हस्र॒म् द्विषा॑हस्रम् चिन्वा॒नो द्वि॒तीय॑म् द्वि॒तीय॑म् चिन्वा॒नो द्विषा॑हस्रम् । \newline
46. चि॒न्वा॒नो द्विषा॑हस्र॒म् द्विषा॑हस्रम् चिन्वा॒न श्चि॑न्वा॒नो द्विषा॑हस्रं॒ ॅवै वै द्विषा॑हस्रम् चिन्वा॒न श्चि॑न्वा॒नो द्विषा॑हस्रं॒ ॅवै । \newline
47. द्विषा॑हस्रं॒ ॅवै वै द्विषा॑हस्र॒म् द्विषा॑हस्रं॒ ॅवा अ॒न्तरि॑क्ष म॒न्तरि॑क्षं॒ ॅवै द्विषा॑हस्र॒म् द्विषा॑हस्रं॒ ॅवा अ॒न्तरि॑क्षम् । \newline
48. द्विषा॑हस्र॒मिति॒ द्वि - सा॒ह॒स्र॒म् । \newline
49. वा अ॒न्तरि॑क्ष म॒न्तरि॑क्षं॒ ॅवै वा अ॒न्तरि॑क्षम् । \newline
50. अ॒न्तरि॑क्ष म॒न्तरि॑क्षम् । \newline
51. अ॒न्तरि॑क्ष मे॒वै वान्तरि॑क्ष म॒न्तरि॑क्ष मे॒वाभ्या᳚(1॒)भ्ये॑ वान्तरि॑क्ष म॒न्तरि॑क्ष मे॒वाभि । \newline
52. ए॒वाभ्या᳚(1॒)भ्ये॑ वैवाभि ज॑यति जय त्य॒भ्ये॑ वैवाभि ज॑यति । \newline
53. अ॒भि ज॑यति जय त्य॒भ्य॑भि ज॑यति॒ त्रिषा॑हस्र॒म् त्रिषा॑हस्रम् जय त्य॒भ्य॑भि ज॑यति॒ त्रिषा॑हस्रम् । \newline
54. ज॒य॒ति॒ त्रिषा॑हस्र॒म् त्रिषा॑हस्रम् जयति जयति॒ त्रिषा॑हस्रम् चिन्वीत चिन्वीत॒ त्रिषा॑हस्रम् जयति जयति॒ त्रिषा॑हस्रम् चिन्वीत । \newline
55. त्रिषा॑हस्रम् चिन्वीत चिन्वीत॒ त्रिषा॑हस्र॒म् त्रिषा॑हस्रम् चिन्वीत तृ॒तीय॑म् तृ॒तीय॑म् चिन्वीत॒ त्रिषा॑हस्र॒म् त्रिषा॑हस्रम् चिन्वीत तृ॒तीय᳚म् । \newline
56. त्रिषा॑हस्र॒मिति॒ त्रि - सा॒ह॒स्र॒म् । \newline
57. चि॒न्वी॒त॒ तृ॒तीय॑म् तृ॒तीय॑म् चिन्वीत चिन्वीत तृ॒तीय॑म् चिन्वा॒न श्चि॑न्वा॒न स्तृ॒तीय॑म् चिन्वीत चिन्वीत तृ॒तीय॑म् चिन्वा॒नः । \newline
58. तृ॒तीय॑म् चिन्वा॒न श्चि॑न्वा॒न स्तृ॒तीय॑म् तृ॒तीय॑म् चिन्वा॒न स्त्रिषा॑हस्र॒ स्त्रिषा॑हस्र श्चिन्वा॒न स्तृ॒तीय॑म् तृ॒तीय॑म् चिन्वा॒न स्त्रिषा॑हस्रः । \newline
59. चि॒न्वा॒न स्त्रिषा॑हस्र॒ स्त्रिषा॑हस्र श्चिन्वा॒न श्चि॑न्वा॒न स्त्रिषा॑हस्रो॒ वै वै त्रिषा॑हस्र श्चिन्वा॒न श्चि॑न्वा॒न स्त्रिषा॑हस्रो॒ वै । \newline
\pagebreak
\markright{ TS 5.6.8.3  \hfill https://www.vedavms.in \hfill}

\section{ TS 5.6.8.3 }

\textbf{TS 5.6.8.3 } \newline
\textbf{Samhita Paata} \newline

-स्त्रिषा॑हस्रो॒ वा अ॒सौ लो॒को॑ ऽमुमे॒व लो॒कम॒भि ज॑यति जानुद॒घ्नं चि॑न्वीत प्रथ॒मं चि॑न्वा॒नो गा॑यत्रि॒यैवेमं ॅलो॒कम॒भ्यारो॑हति नाभिद॒घ्नं चि॑न्वीत द्वि॒तीयं॑ चिन्वा॒नस्त्रि॒ष्टुभै॒वा-न्तरि॑क्ष-म॒भ्यारो॑हति ग्रीवद॒घ्नं चि॑न्वीत तृ॒तीयं॑ चिन्वा॒नो जग॑त्यै॒वामुन् ॅलो॒कम॒भ्यारो॑हति॒ नाग्निं चि॒त्वा रा॒मामुपे॑यादयो॒नौ रेतो॑ धास्या॒मीति॒ न द्वि॒तीयं॑ चि॒त्वाऽन्यस्य॒ स्त्रिय॒ - [  ] \newline

\textbf{Pada Paata} \newline

त्रिषा॑हस्र॒ इति॒ त्रि - सा॒ह॒स्रः॒ । वै । अ॒सौ । लो॒कः । अ॒मुम् । ए॒व । लो॒कम् । अ॒भीति॑ । ज॒य॒ति॒ । जा॒नु॒द॒घ्नमिति॑ जानु - द॒घ्नम् । चि॒न्वी॒त॒ । प्र॒थ॒मम् । चि॒न्वा॒नः । गा॒य॒त्रि॒या । ए॒व । इ॒मम् । लो॒कम् । अ॒भ्यारो॑ह॒तीत्य॑भि - आरो॑हति । ना॒भि॒द॒घ्नमिति॑ नाभि - द॒घ्नम् । चि॒न्वी॒त॒ ।  द्वि॒तीय᳚म् । चि॒न्वा॒नः । त्रि॒ष्टुभा᳚ । ए॒व । अ॒न्तरि॑क्षम् । अ॒भ्यारो॑ह॒तीत्य॑भि - आरो॑हति । ग्री॒व॒द॒घ्नमिति॑ ग्रीव - द॒घ्नम् । चि॒न्वी॒त॒ । तृ॒तीय᳚म् । चि॒न्वा॒नः । जग॑त्या । ए॒व । अ॒मुम् ।  लो॒कम् । अ॒भ्यारो॑ह॒तीत्य॑भि - आरो॑हति । न । अ॒ग्निम् । चि॒त्वा । रा॒माम् । उपेति॑ । इ॒या॒त् । अ॒यो॒नौ । रेतः॑ । धा॒स्या॒मि॒ । इति॑ । न । द्वि॒तीय᳚म् । चि॒त्वा । अ॒न्यस्य॑ । स्त्रिय᳚म् ।  \newline


\textbf{Krama Paata} \newline

त्रिषा॑हस्रो॒ वै । त्रिषा॑हस्र॒ इति॒ त्रि - सा॒ह॒स्रः॒ । वा अ॒सौ । अ॒सौ लो॒कः । लो॒को॑ऽमुम् । अ॒मुमे॒व । ए॒व लो॒कम् । लो॒कम॒भि । अ॒भि ज॑यति । ज॒य॒ति॒ जा॒नु॒द॒घ्नम् । जा॒नु॒द॒घ्नम् चि॑न्वीत । जा॒नु॒द॒घ्नमिति॑ जानु - द॒घ्नम् । चि॒न्वी॒त॒ प्र॒थ॒मम् । प्र॒थ॒मम् चि॑न्वा॒नः । चि॒न्वा॒नो गा॑यत्रि॒या । गा॒य॒त्रि॒यैव । ए॒वेमम् । इ॒मम् ॅलो॒कम् । लो॒कम॒भ्यारो॑हति । अ॒भ्यारो॑हति नाभिद॒घ्नम् । अ॒भ्यारो॑ह॒तीत्य॑भि - आरो॑हति । ना॒भि॒द॒घ्नम् चि॑न्वीत । ना॒भि॒द॒घ्नमिति॑ नाभि - द॒घ्नम् । चि॒न्वी॒त॒ द्वि॒तीय᳚म् । द्वि॒तीय॑म् चिन्वा॒नः । चि॒न्वा॒नस्त्रि॒ष्टुभा᳚ । त्रि॒ष्टुभै॒व । ए॒वान्तरि॑क्षम् । अ॒न्तरि॑क्षम॒भ्यारो॑हति । अ॒भ्यारो॑हति ग्रीवद॒घ्नम् । अ॒भ्यारो॑ह॒तीत्य॑भि - आरो॑हति । ग्री॒व॒द॒घ्नम् चि॑न्वीत । ग्री॒व॒द॒घ्नमिति॑ ग्रीव - द॒घ्नम् । चि॒न्वी॒त॒ तृ॒तीय᳚म् । तृ॒तीय॑म् चिन्वा॒नः । चि॒न्वा॒नो जग॑त्या । जग॑त्यै॒व । ए॒वामुम् । अ॒मुम् ॅलो॒कम् । लो॒कम॒भ्यारो॑हति । अ॒भ्यारो॑हति॒ न । अ॒भ्यारो॑ह॒तीत्य॑भि - आरो॑हति । नाग्निम् । अ॒ग्निम् चि॒त्वा । चि॒त्वा रा॒माम् । रा॒मामुप॑ । उपे॑यात् । इ॒या॒द॒यो॒नौ । अ॒यो॒नौ रेतः॑ । रेतो॑ धास्यामि । धा॒स्या॒मीति॑ । इति॒ न । न द्वि॒तीय᳚म् । द्वि॒तीय॑म् चि॒त्वा । चि॒त्वाऽन्यस्य॑ । अ॒न्यस्य॒ स्त्रिय᳚म् । स्त्रिय॒मुप॑ \newline

\textbf{Jatai Paata} \newline

1. त्रिषा॑हस्रो॒ वै वै त्रिषा॑हस्र॒ स्त्रिषा॑हस्रो॒ वै । \newline
2. त्रिषा॑हस्र॒ इति॒ त्रि - सा॒ह॒स्रः॒ । \newline
3. वा अ॒सा व॒सौ वै वा अ॒सौ । \newline
4. अ॒सौ लो॒को लो॒को॑ ऽसा व॒सौ लो॒कः । \newline
5. लो॒को॑ ऽमु म॒मुम् ॅलो॒को लो॒को॑ ऽमुम् । \newline
6. अ॒मु मे॒वै वामु म॒मु मे॒व । \newline
7. ए॒व लो॒कम् ॅलो॒क मे॒वैव लो॒कम् । \newline
8. लो॒क म॒भ्य॑भि लो॒कम् ॅलो॒क म॒भि । \newline
9. अ॒भि ज॑यति जय त्य॒भ्य॑भि ज॑यति । \newline
10. ज॒य॒ति॒ जा॒नु॒द॒घ्नम् जा॑नुद॒घ्नम् ज॑यति जयति जानुद॒घ्नम् । \newline
11. जा॒नु॒द॒घ्नम् चि॑न्वीत चिन्वीत जानुद॒घ्नम् जा॑नुद॒घ्नम् चि॑न्वीत । \newline
12. जा॒नु॒द॒घ्नमिति॑ जानु - द॒घ्नम् । \newline
13. चि॒न्वी॒त॒ प्र॒थ॒मम् प्र॑थ॒मम् चि॑न्वीत चिन्वीत प्रथ॒मम् । \newline
14. प्र॒थ॒मम् चि॑न्वा॒न श्चि॑न्वा॒नः प्र॑थ॒मम् प्र॑थ॒मम् चि॑न्वा॒नः । \newline
15. चि॒न्वा॒नो गा॑यत्रि॒या गा॑यत्रि॒या चि॑न्वा॒न श्चि॑न्वा॒नो गा॑यत्रि॒या । \newline
16. गा॒य॒त्रि॒ यैवैव गा॑यत्रि॒या गा॑यत्रि॒यैव । \newline
17. ए॒वेम मि॒म मे॒वै वेमम् । \newline
18. इ॒मम् ॅलो॒कम् ॅलो॒क मि॒म मि॒मम् ॅलो॒कम् । \newline
19. लो॒क म॒भ्यारो॑ह त्य॒भ्यारो॑हति लो॒कम् ॅलो॒क म॒भ्यारो॑हति । \newline
20. अ॒भ्यारो॑हति नाभिद॒घ्नम् ना॑भिद॒घ्न म॒भ्यारो॑ह त्य॒भ्यारो॑हति नाभिद॒घ्नम् । \newline
21. अ॒भ्यारो॑ह॒तीत्य॑भि - आरो॑हति । \newline
22. ना॒भि॒द॒घ्नम् चि॑न्वीत चिन्वीत नाभिद॒घ्नम् ना॑भिद॒घ्नम् चि॑न्वीत । \newline
23. ना॒भि॒द॒घ्नमिति॑ नाभि - द॒घ्नम् । \newline
24. चि॒न्वी॒त॒ द्वि॒तीय॑म् द्वि॒तीय॑म् चिन्वीत चिन्वीत द्वि॒तीय᳚म् । \newline
25. द्वि॒तीय॑म् चिन्वा॒न श्चि॑न्वा॒नो द्वि॒तीय॑म् द्वि॒तीय॑म् चिन्वा॒नः । \newline
26. चि॒न्वा॒न स्त्रि॒ष्टुभा᳚ त्रि॒ष्टुभा॑ चिन्वा॒न श्चि॑न्वा॒न स्त्रि॒ष्टुभा᳚ । \newline
27. त्रि॒ष्टुभै॒वैव त्रि॒ष्टुभा᳚ त्रि॒ष्टुभै॒व । \newline
28. ए॒वान्तरि॑क्ष म॒न्तरि॑क्ष मे॒वै वान्तरि॑क्षम् । \newline
29. अ॒न्तरि॑क्ष म॒भ्यारो॑ह त्य॒भ्यारो॑ह त्य॒न्तरि॑क्ष म॒न्तरि॑क्ष म॒भ्यारो॑हति । \newline
30. अ॒भ्यारो॑हति ग्रीवद॒घ्नम् ग्री॑वद॒घ्न म॒भ्यारो॑ह त्य॒भ्यारो॑हति ग्रीवद॒घ्नम् । \newline
31. अ॒भ्यारो॑ह॒तीत्य॑भि - आरो॑हति । \newline
32. ग्री॒व॒द॒घ्नम् चि॑न्वीत चिन्वीत ग्रीवद॒घ्नम् ग्री॑वद॒घ्नम् चि॑न्वीत । \newline
33. ग्री॒व॒द॒घ्नमिति॑ ग्रीव - द॒घ्नम् । \newline
34. चि॒न्वी॒त॒ तृ॒तीय॑म् तृ॒तीय॑म् चिन्वीत चिन्वीत तृ॒तीय᳚म् । \newline
35. तृ॒तीय॑म् चिन्वा॒न श्चि॑न्वा॒न स्तृ॒तीय॑म् तृ॒तीय॑म् चिन्वा॒नः । \newline
36. चि॒न्वा॒नो जग॑त्या॒ जग॑त्या चिन्वा॒न श्चि॑न्वा॒नो जग॑त्या । \newline
37. जग॑ त्यै॒वैव जग॑त्या॒ जग॑ त्यै॒व । \newline
38. ए॒वामु म॒मु मे॒वै वामुम् । \newline
39. अ॒मुम् ॅलो॒कम् ॅलो॒क म॒मु म॒मुम् ॅलो॒कम् । \newline
40. लो॒क म॒भ्यारो॑ह त्य॒भ्यारो॑हति लो॒कम् ॅलो॒क म॒भ्यारो॑हति । \newline
41. अ॒भ्यारो॑हति॒ न नाभ्यारो॑ह त्य॒भ्यारो॑हति॒ न । \newline
42. अ॒भ्यारो॑ह॒तीत्य॑भि - आरो॑हति । \newline
43. नाग्नि म॒ग्निम् न नाग्निम् । \newline
44. अ॒ग्निम् चि॒त्वा चि॒त्वा ऽग्नि म॒ग्निम् चि॒त्वा । \newline
45. चि॒त्वा रा॒माꣳ रा॒माम् चि॒त्वा चि॒त्वा रा॒माम् । \newline
46. रा॒मा मुपोप॑ रा॒माꣳ रा॒मा मुप॑ । \newline
47. उपे॑ या दिया॒ दुपोपे॑ यात् । \newline
48. इ॒या॒ द॒यो॒ना व॑यो॒ना वि॑या दिया दयो॒नौ । \newline
49. अ॒यो॒नौ रेतो॒ रेतो॑ ऽयो॒ना व॑यो॒नौ रेतः॑ । \newline
50. रेतो॑ धास्यामि धास्यामि॒ रेतो॒ रेतो॑ धास्यामि । \newline
51. धा॒स्या॒मी तीति॑ धास्यामि धास्या॒मीति॑ । \newline
52. इति॒ न नेतीति॒ न । \newline
53. न द्वि॒तीय॑म् द्वि॒तीय॒म् न न द्वि॒तीय᳚म् । \newline
54. द्वि॒तीय॑म् चि॒त्वा चि॒त्वा द्वि॒तीय॑म् द्वि॒तीय॑म् चि॒त्वा । \newline
55. चि॒त्वा ऽन्यस्या॒ न्यस्य॑ चि॒त्वा चि॒त्वा ऽन्यस्य॑ । \newline
56. अ॒न्यस्य॒ स्त्रियꣳ॒॒ स्त्रिय॑ म॒न्यस्या॒ न्यस्य॒ स्त्रिय᳚म् । \newline
57. स्त्रिय॒ मुपोप॒ स्त्रियꣳ॒॒ स्त्रिय॒ मुप॑ । \newline

\textbf{Ghana Paata } \newline

1. त्रिषा॑हस्रो॒ वै वै त्रिषा॑हस्र॒ स्त्रिषा॑हस्रो॒ वा अ॒सा व॒सौ वै त्रिषा॑हस्र॒ स्त्रिषा॑हस्रो॒ वा अ॒सौ । \newline
2. त्रिषा॑हस्र॒ इति॒ त्रि - सा॒ह॒स्रः॒ । \newline
3. वा अ॒सा व॒सौ वै वा अ॒सौ लो॒को लो॒को॑ ऽसौ वै वा अ॒सौ लो॒कः । \newline
4. अ॒सौ लो॒को लो॒को॑ ऽसा व॒सौ लो॒को॑ ऽमु म॒मुम् ॅलो॒को॑ ऽसा व॒सौ लो॒को॑ ऽमुम् । \newline
5. लो॒को॑ ऽमु म॒मुम् ॅलो॒को लो॒को॑ ऽमु मे॒वैवामुम् ॅलो॒को लो॒को॑ ऽमु मे॒व । \newline
6. अ॒मु मे॒वैवामु म॒मु मे॒व लो॒कम् ॅलो॒क मे॒वामु म॒मु मे॒व लो॒कम् । \newline
7. ए॒व लो॒कम् ॅलो॒क मे॒वैव लो॒क म॒भ्य॑भि लो॒क मे॒वैव लो॒क म॒भि । \newline
8. लो॒क म॒भ्य॑भि लो॒कम् ॅलो॒क म॒भि ज॑यति जय त्य॒भि लो॒कम् ॅलो॒क म॒भि ज॑यति । \newline
9. अ॒भि ज॑यति जय त्य॒भ्य॑भि ज॑यति जानुद॒घ्नम् जा॑नुद॒घ्नम् ज॑य त्य॒भ्य॑भि ज॑यति जानुद॒घ्नम् । \newline
10. ज॒य॒ति॒ जा॒नु॒द॒घ्नम् जा॑नुद॒घ्नम् ज॑यति जयति जानुद॒घ्नम् चि॑न्वीत चिन्वीत जानुद॒घ्नम् ज॑यति जयति जानुद॒घ्नम् चि॑न्वीत । \newline
11. जा॒नु॒द॒घ्नम् चि॑न्वीत चिन्वीत जानुद॒घ्नम् जा॑नुद॒घ्नम् चि॑न्वीत प्रथ॒मम् प्र॑थ॒मम् चि॑न्वीत जानुद॒घ्नम् जा॑नुद॒घ्नम् चि॑न्वीत प्रथ॒मम् । \newline
12. जा॒नु॒द॒घ्नमिति॑ जानु - द॒घ्नम् । \newline
13. चि॒न्वी॒त॒ प्र॒थ॒मम् प्र॑थ॒मम् चि॑न्वीत चिन्वीत प्रथ॒मम् चि॑न्वा॒न श्चि॑न्वा॒नः प्र॑थ॒मम् चि॑न्वीत चिन्वीत प्रथ॒मम् चि॑न्वा॒नः । \newline
14. प्र॒थ॒मम् चि॑न्वा॒न श्चि॑न्वा॒नः प्र॑थ॒मम् प्र॑थ॒मम् चि॑न्वा॒नो गा॑यत्रि॒या गा॑यत्रि॒या चि॑न्वा॒नः प्र॑थ॒मम् प्र॑थ॒मम् चि॑न्वा॒नो गा॑यत्रि॒या । \newline
15. चि॒न्वा॒नो गा॑यत्रि॒या गा॑यत्रि॒या चि॑न्वा॒न श्चि॑न्वा॒नो गा॑यत्रि॒यैवैव गा॑यत्रि॒या चि॑न्वा॒न श्चि॑न्वा॒नो गा॑यत्रि॒यैव । \newline
16. गा॒य॒त्रि॒यै वैव गा॑यत्रि॒या गा॑यत्रि॒ यैवेम मि॒म मे॒व गा॑यत्रि॒या गा॑यत्रि॒यैवेमम् । \newline
17. ए॒वेम मि॒म मे॒वैवेमम् ॅलो॒कम् ॅलो॒क मि॒म मे॒वैवेमम् ॅलो॒कम् । \newline
18. इ॒मम् ॅलो॒कम् ॅलो॒क मि॒म मि॒मम् ॅलो॒क म॒भ्यारो॑ह त्य॒भ्यारो॑हति लो॒क मि॒म मि॒मम् ॅलो॒क म॒भ्यारो॑हति । \newline
19. लो॒क म॒भ्यारो॑ह त्य॒भ्यारो॑हति लो॒कम् ॅलो॒क म॒भ्यारो॑हति नाभिद॒घ्नम् ना॑भिद॒घ्न म॒भ्यारो॑हति लो॒कम् ॅलो॒क म॒भ्यारो॑हति नाभिद॒घ्नम् । \newline
20. अ॒भ्यारो॑हति नाभिद॒घ्नम् ना॑भिद॒घ्न म॒भ्यारो॑ह त्य॒भ्यारो॑हति नाभिद॒घ्नम् चि॑न्वीत चिन्वीत नाभिद॒घ्न म॒भ्यारो॑ह त्य॒भ्यारो॑हति नाभिद॒घ्नम् चि॑न्वीत । \newline
21. अ॒भ्यारो॑ह॒तीत्य॑भि - आरो॑हति । \newline
22. ना॒भि॒द॒घ्नम् चि॑न्वीत चिन्वीत नाभिद॒घ्नम् ना॑भिद॒घ्नम् चि॑न्वीत द्वि॒तीय॑म् द्वि॒तीय॑म् चिन्वीत नाभिद॒घ्नम् ना॑भिद॒घ्नम् चि॑न्वीत द्वि॒तीय᳚म् । \newline
23. ना॒भि॒द॒घ्नमिति॑ नाभि - द॒घ्नम् । \newline
24. चि॒न्वी॒त॒ द्वि॒तीय॑म् द्वि॒तीय॑म् चिन्वीत चिन्वीत द्वि॒तीय॑म् चिन्वा॒न श्चि॑न्वा॒नो द्वि॒तीय॑म् चिन्वीत चिन्वीत द्वि॒तीय॑म् चिन्वा॒नः । \newline
25. द्वि॒तीय॑म् चिन्वा॒न श्चि॑न्वा॒नो द्वि॒तीय॑म् द्वि॒तीय॑म् चिन्वा॒न स्त्रि॒ष्टुभा᳚ त्रि॒ष्टुभा॑ चिन्वा॒नो द्वि॒तीय॑म् द्वि॒तीय॑म् चिन्वा॒न स्त्रि॒ष्टुभा᳚ । \newline
26. चि॒न्वा॒न स्त्रि॒ष्टुभा᳚ त्रि॒ष्टुभा॑ चिन्वा॒न श्चि॑न्वा॒न स्त्रि॒ष्टुभै॒ वैव त्रि॒ष्टुभा॑ चिन्वा॒न श्चि॑न्वा॒न स्त्रि॒ष्टुभै॒व । \newline
27. त्रि॒ष्टुभै॒ वैव त्रि॒ष्टुभा᳚ त्रि॒ष्टुभै॒ वान्तरि॑क्ष म॒न्तरि॑क्ष मे॒व त्रि॒ष्टुभा᳚ त्रि॒ष्टुभै॒ वान्तरि॑क्षम् । \newline
28. ए॒वान्तरि॑क्ष म॒न्तरि॑क्ष मे॒वै वान्तरि॑क्ष म॒भ्यारो॑ह त्य॒भ्यारो॑ह त्य॒न्तरि॑क्ष मे॒वै वान्तरि॑क्ष म॒भ्यारो॑हति । \newline
29. अ॒न्तरि॑क्ष म॒भ्यारो॑ह त्य॒भ्यारो॑ह त्य॒न्तरि॑क्ष म॒न्तरि॑क्ष म॒भ्यारो॑हति ग्रीवद॒घ्नम् ग्री॑वद॒घ्न म॒भ्यारो॑ह त्य॒न्तरि॑क्ष म॒न्तरि॑क्ष म॒भ्यारो॑हति ग्रीवद॒घ्नम् । \newline
30. अ॒भ्यारो॑हति ग्रीवद॒घ्नम् ग्री॑वद॒घ्न म॒भ्यारो॑ह त्य॒भ्यारो॑हति ग्रीवद॒घ्नम् चि॑न्वीत चिन्वीत ग्रीवद॒घ्न म॒भ्यारो॑ह त्य॒भ्यारो॑हति ग्रीवद॒घ्नम् चि॑न्वीत । \newline
31. अ॒भ्यारो॑ह॒तीत्य॑भि - आरो॑हति । \newline
32. ग्री॒व॒द॒घ्नम् चि॑न्वीत चिन्वीत ग्रीवद॒घ्नम् ग्री॑वद॒घ्नम् चि॑न्वीत तृ॒तीय॑म् तृ॒तीय॑म् चिन्वीत ग्रीवद॒घ्नम् ग्री॑वद॒घ्नम् चि॑न्वीत तृ॒तीय᳚म् । \newline
33. ग्री॒व॒द॒घ्नमिति॑ ग्रीव - द॒घ्नम् । \newline
34. चि॒न्वी॒त॒ तृ॒तीय॑म् तृ॒तीय॑म् चिन्वीत चिन्वीत तृ॒तीय॑म् चिन्वा॒न श्चि॑न्वा॒न स्तृ॒तीय॑म् चिन्वीत चिन्वीत तृ॒तीय॑म् चिन्वा॒नः । \newline
35. तृ॒तीय॑म् चिन्वा॒न श्चि॑न्वा॒न स्तृ॒तीय॑म् तृ॒तीय॑म् चिन्वा॒नो जग॑त्या॒ जग॑त्या चिन्वा॒न स्तृ॒तीय॑म् तृ॒तीय॑म् चिन्वा॒नो जग॑त्या । \newline
36. चि॒न्वा॒नो जग॑त्या॒ जग॑त्या चिन्वा॒न श्चि॑न्वा॒नो जग॑ त्यै॒वैव जग॑त्या चिन्वा॒न श्चि॑न्वा॒नो जग॑त्यै॒व । \newline
37. जग॑ त्यै॒वैव जग॑त्या॒ जग॑त्यै॒ वामु म॒मु मे॒व जग॑त्या॒ जग॑त्यै॒ वामुम् । \newline
38. ए॒वामु म॒मु मे॒वै वामुम् ॅलो॒कम् ॅलो॒क म॒मु मे॒वै वामुम् ॅलो॒कम् । \newline
39. अ॒मुम् ॅलो॒कम् ॅलो॒क म॒मु म॒मुम् ॅलो॒क म॒भ्यारो॑ह त्य॒भ्यारो॑हति लो॒क म॒मु म॒मुम् ॅलो॒क म॒भ्यारो॑हति । \newline
40. लो॒क म॒भ्यारो॑ह त्य॒भ्यारो॑हति लो॒कम् ॅलो॒क म॒भ्यारो॑हति॒ न नाभ्यारो॑हति लो॒कम् ॅलो॒क म॒भ्यारो॑हति॒ न । \newline
41. अ॒भ्यारो॑हति॒ न नाभ्यारो॑ह त्य॒भ्यारो॑हति॒ नाग्नि म॒ग्निम् नाभ्यारो॑ह त्य॒भ्यारो॑हति॒ नाग्निम् । \newline
42. अ॒भ्यारो॑ह॒तीत्य॑भि - आरो॑हति । \newline
43. नाग्नि म॒ग्निम् न नाग्निम् चि॒त्वा चि॒त्वा ऽग्निम् न नाग्निम् चि॒त्वा । \newline
44. अ॒ग्निम् चि॒त्वा चि॒त्वा ऽग्नि म॒ग्निम् चि॒त्वा रा॒माꣳ रा॒माम् चि॒त्वा ऽग्नि म॒ग्निम् चि॒त्वा रा॒माम् । \newline
45. चि॒त्वा रा॒माꣳ रा॒माम् चि॒त्वा चि॒त्वा रा॒मा मुपोप॑ रा॒माम् चि॒त्वा चि॒त्वा रा॒मा मुप॑ । \newline
46. रा॒मा मुपोप॑ रा॒माꣳ रा॒मा मुपे॑या दिया॒ दुप॑ रा॒माꣳ रा॒मा मुपे॑यात् । \newline
47. उपे॑या दिया॒ दुपोपे॑या दयो॒ना व॑यो॒ना वि॑या॒ दुपोपे॑या दयो॒नौ । \newline
48. इ॒या॒ द॒यो॒ना व॑यो॒ना वि॑या दिया दयो॒नौ रेतो॒ रेतो॑ ऽयो॒ना वि॑या दिया दयो॒नौ रेतः॑ । \newline
49. अ॒यो॒नौ रेतो॒ रेतो॑ ऽयो॒ना व॑यो॒नौ रेतो॑ धास्यामि धास्यामि॒ रेतो॑ ऽयो॒ना व॑यो॒नौ रेतो॑ धास्यामि । \newline
50. रेतो॑ धास्यामि धास्यामि॒ रेतो॒ रेतो॑ धास्या॒मीतीति॑ धास्यामि॒ रेतो॒ रेतो॑ धास्या॒मीति॑ । \newline
51. धा॒स्या॒मीतीति॑ धास्यामि धास्या॒मीति॒ न नेति॑ धास्यामि धास्या॒मीति॒ न । \newline
52. इति॒ न नेतीति॒ न द्वि॒तीय॑म् द्वि॒तीय॒म् नेतीति॒ न द्वि॒तीय᳚म् । \newline
53. न द्वि॒तीय॑म् द्वि॒तीय॒म् न न द्वि॒तीय॑म् चि॒त्वा चि॒त्वा द्वि॒तीय॒म् न न द्वि॒तीय॑म् चि॒त्वा । \newline
54. द्वि॒तीय॑म् चि॒त्वा चि॒त्वा द्वि॒तीय॑म् द्वि॒तीय॑म् चि॒त्वा ऽन्यस्या॒ न्यस्य॑ चि॒त्वा द्वि॒तीय॑म् द्वि॒तीय॑म् चि॒त्वा ऽन्यस्य॑ । \newline
55. चि॒त्वा ऽन्यस्या॒ न्यस्य॑ चि॒त्वा चि॒त्वा ऽन्यस्य॒ स्त्रियꣳ॒॒ स्त्रिय॑ म॒न्यस्य॑ चि॒त्वा चि॒त्वा ऽन्यस्य॒ स्त्रिय᳚म् । \newline
56. अ॒न्यस्य॒ स्त्रियꣳ॒॒ स्त्रिय॑ म॒न्यस्या॒ न्यस्य॒ स्त्रिय॒ मुपोप॒ स्त्रिय॑ म॒न्यस्या॒ न्यस्य॒ स्त्रिय॒ मुप॑ । \newline
57. स्त्रिय॒ मुपोप॒ स्त्रियꣳ॒॒ स्त्रिय॒ मुपे॑ यादिया॒ दुप॒ स्त्रियꣳ॒॒ स्त्रिय॒ मुपे॑ यात् । \newline
\pagebreak
\markright{ TS 5.6.8.4  \hfill https://www.vedavms.in \hfill}

\section{ TS 5.6.8.4 }

\textbf{TS 5.6.8.4 } \newline
\textbf{Samhita Paata} \newline

-मुपे॑या॒न्न तृ॒तीयं॑ चि॒त्वा कां च॒नोपे॑या॒द्-रेतो॒ वा ए॒तन्नि ध॑त्ते॒ यद॒ग्निं चि॑नु॒ते यदु॑पे॒याद्-रेत॑सा॒ व्यृ॑द्ध्ये॒ताऽथो॒ खल्वा॑हुर प्रज॒स्यं तद्-यन्नोपे॒यादिति॒ यद्-रे॑त॒स्सिचा॑वुप॒दधा॑ति॒ ते ए॒व यज॑मानस्य॒ रेतो॑ बिभृत॒स्तस्मा॒-दुपे॑या॒द्-रेत॒सो-ऽस्क॑न्दाय॒ त्रीणि॒ वाव रेताꣳ॑सि पि॒ता पु॒त्रः पौत्रो॒ - [  ] \newline

\textbf{Pada Paata} \newline

उपेति॑ । इ॒या॒त् । न । तृ॒तीय᳚म् । चि॒त्वा । काम् । च॒न । उपेति॑ । इ॒या॒त् । रेतः॑ । वै । ए॒तत् । नीति॑ । ध॒त्ते॒ । यत् । अ॒ग्निम् । चि॒नु॒ते । यत् । उ॒पे॒यादित्यु॑प - इ॒यात् । रेत॑सा । वीति॑ । ऋ॒द्ध्ये॒त॒ । अथो॒ इति॑ । खलु॑ । आ॒हुः॒ । अ॒प्र॒ज॒स्यमित्य॑प्र-ज॒स्यम् । तत् । यत् । न । उ॒पे॒यादित्यु॑प - इ॒यात् । इति॑ । यत् । रे॒त॒स्सिचा॒विति॑ रेतः - सिचौ᳚ । उ॒प॒दधा॒तीत्यु॑प - दधा॑ति । ते इति॑ । ए॒व । यज॑मानस्य । रेतः॑ । बि॒भृ॒तः॒ । तस्मा᳚त् । उपेति॑ । इ॒या॒त् । रेत॑सः । अस्क॑न्दाय । त्रीणि॑ । वाव । रेताꣳ॑सि । पि॒ता । पु॒त्रः । पौत्रः॑ ।  \newline


\textbf{Krama Paata} \newline

उपे॑यात् । इ॒या॒न् न । न तृ॒तीय᳚म् । तृ॒तीय॑म् चि॒त्वा । चि॒त्वा काम् । काम् च॒न । च॒नोप॑ । उपे॑यात् । इ॒या॒द् रेतः॑ । रेतो॒ वै । वा ए॒तत् । ए॒तन् नि । नि ध॑त्ते । ध॒त्ते॒ यत् । यद॒ग्निम् । अ॒ग्निम् चि॑नु॒ते । चि॒नु॒ते यत् । यदु॑पे॒यात् । उ॒पे॒याद् रेत॑सा । उ॒पे॒यादित्यु॑प - इ॒यात् । रेत॑सा॒ वि । व्यृ॑द्ध्येत । ऋ॒द्ध्ये॒ताथो᳚ । अथो॒ खलु॑ । अथो॒ इत्यथो᳚ । खल्वा॑हुः । आ॒हु॒र॒प्र॒ज॒स्यम् । अ॒प्र॒ज॒स्यम् तत् । अ॒प्र॒ज॒स्यमित्य॑प्र - ज॒स्यम् । तद् यत् । यन् न । नोपे॒यात् । उ॒पे॒यादिति॑ । उ॒पे॒यादित्यु॑प - इ॒यात् । इति॒ यत् । यद् रे॑त॒स्सिचौ᳚ । रे॒त॒स्सिचा॑वुप॒दधा॑ति । रे॒त॒स्सिचा॒विति॑ रेतः - सिचौ᳚ । उ॒प॒दधा॑ति॒ ते । उ॒प॒दधा॒तीत्यु॑प - दधा॑ति । ते ए॒व । ते इति॒ ते । ए॒व यज॑मानस्य । यज॑मानस्य॒ रेतः॑ । रेतो॑ बिभृतः । बि॒भृ॒त॒स्तस्मा᳚त् । तस्मा॒दुप॑ । उपे॑यात् । इ॒या॒द् रेत॑सः । रेत॒सोऽस्क॑न्दाय । अस्क॑न्दाय॒ त्रीणि॑ । त्रीणि॒ वाव । वाव रेताꣳ॑सि । रेताꣳ॑सि पि॒ता । पि॒ता पु॒त्रः । पु॒त्रः पौत्रः॑ । पौत्रो॒ यत् \newline

\textbf{Jatai Paata} \newline

1. उपे॑ या दिया॒ दुपोपे॑ यात् । \newline
2. इ॒या॒न् न नेया॑ दिया॒न् न । \newline
3. न तृ॒तीय॑म् तृ॒तीय॒म् न न तृ॒तीय᳚म् । \newline
4. तृ॒तीय॑म् चि॒त्वा चि॒त्वा तृ॒तीय॑म् तृ॒तीय॑म् चि॒त्वा । \newline
5. चि॒त्वा काम् काम् चि॒त्वा चि॒त्वा काम् । \newline
6. काम् च॒न च॒न काम् काम् च॒न । \newline
7. च॒नोपोप॑ च॒न च॒नोप॑ । \newline
8. उपे॑या दिया॒ दुपोपे॑ यात् । \newline
9. इ॒या॒द् रेतो॒ रेत॑ इया दिया॒द् रेतः॑ । \newline
10. रेतो॒ वै वै रेतो॒ रेतो॒ वै । \newline
11. वा ए॒त दे॒तद् वै वा ए॒तत् । \newline
12. ए॒तन् नि न्ये॑त दे॒तन् नि । \newline
13. नि ध॑त्ते धत्ते॒ नि नि ध॑त्ते । \newline
14. ध॒त्ते॒ यद् यद् ध॑त्ते धत्ते॒ यत् । \newline
15. यद॒ग्नि म॒ग्निं ॅयद् यद॒ग्निम् । \newline
16. अ॒ग्निम् चि॑नु॒ते चि॑नु॒ते᳚ ऽग्नि म॒ग्निम् चि॑नु॒ते । \newline
17. चि॒नु॒ते यद् यच् चि॑नु॒ते चि॑नु॒ते यत् । \newline
18. यदु॑पे॒या दु॑पे॒याद् यद् यदु॑पे॒यात् । \newline
19. उ॒पे॒याद् रेत॑सा॒ रेत॑सो पे॒या दु॑पे॒याद् रेत॑सा । \newline
20. उ॒पे॒यादित्यु॑प - इ॒यात् । \newline
21. रेत॑सा॒ वि वि रेत॑सा॒ रेत॑सा॒ वि । \newline
22. व्यृ॑द्ध्येत र्‌द्ध्येत॒ वि व्यृ॑द्ध्येत । \newline
23. ऋ॒द्ध्ये॒ ताथो॒ अथो॑ ऋद्ध्येत र्‌द्ध्ये॒ ताथो᳚ । \newline
24. अथो॒ खलु॒ खल्वथो॒ अथो॒ खलु॑ । \newline
25. अथो॒ इत्यथो᳚ । \newline
26. खल्वा॑हु राहुः॒ खलु॒ खल्वा॑हुः । \newline
27. आ॒हु॒ र॒प्र॒ज॒स्य म॑प्रज॒स्य मा॑हु राहु रप्रज॒स्यम् । \newline
28. अ॒प्र॒ज॒स्यम् तत् तद॑प्रज॒स्य म॑प्रज॒स्यम् तत् । \newline
29. अ॒प्र॒ज॒स्यमित्य॑प्र - ज॒स्यम् । \newline
30. तद् यद् यत् तत् तद् यत् । \newline
31. यन् न न यद् यन् न । \newline
32. नोपे॒या दु॑पे॒यान् न नोपे॒यात् । \newline
33. उ॒पे॒या दिती त्यु॑पे॒या दु॑पे॒या दिति॑ । \newline
34. उ॒पे॒यादित्यु॑प - इ॒यात् । \newline
35. इति॒ यद् यदितीति॒ यत् । \newline
36. यद् रे॑त॒स्सिचौ॑ रेत॒स्सिचौ॒ यद् यद् रे॑त॒स्सिचौ᳚ । \newline
37. रे॒त॒स्सिचा॑ वुप॒दधा᳚ त्युप॒दधा॑ति रेत॒स्सिचौ॑ रेत॒स्सिचा॑ वुप॒दधा॑ति । \newline
38. रे॒त॒स्सिचा॒विति॑ रेतः - सिचौ᳚ । \newline
39. उ॒प॒दधा॑ति॒ ते ते उ॑प॒दधा᳚ त्युप॒दधा॑ति॒ ते । \newline
40. उ॒प॒दधा॒तीत्यु॑प - दधा॑ति । \newline
41. ते ए॒वैव ते ते ए॒व । \newline
42. ते इति॒ ते । \newline
43. ए॒व यज॑मानस्य॒ यज॑मान स्यै॒वैव यज॑मानस्य । \newline
44. यज॑मानस्य॒ रेतो॒ रेतो॒ यज॑मानस्य॒ यज॑मानस्य॒ रेतः॑ । \newline
45. रेतो॑ बिभृतो बिभृतो॒ रेतो॒ रेतो॑ बिभृतः । \newline
46. बि॒भृ॒त॒ स्तस्मा॒त् तस्मा᳚द् बिभृतो बिभृत॒ स्तस्मा᳚त् । \newline
47. तस्मा॒ दुपोप॒ तस्मा॒त् तस्मा॒ दुप॑ । \newline
48. उपे॑या दिया॒ दुपोपे॑ यात् । \newline
49. इ॒या॒द् रेत॑सो॒ रेत॑स इया दिया॒द् रेत॑सः । \newline
50. रेत॒सो ऽस्क॑न्दा॒या स्क॑न्दाय॒ रेत॑सो॒ रेत॒सो ऽस्क॑न्दाय । \newline
51. अस्क॑न्दाय॒ त्रीणि॒ त्रीण्यस्क॑न्दा॒या स्क॑न्दाय॒ त्रीणि॑ । \newline
52. त्रीणि॒ वाव वाव त्रीणि॒ त्रीणि॒ वाव । \newline
53. वाव रेताꣳ॑सि॒ रेताꣳ॑सि॒ वाव वाव रेताꣳ॑सि । \newline
54. रेताꣳ॑सि पि॒ता पि॒ता रेताꣳ॑सि॒ रेताꣳ॑सि पि॒ता । \newline
55. पि॒ता पु॒त्रः पु॒त्रः पि॒ता पि॒ता पु॒त्रः । \newline
56. पु॒त्रः पौत्रः॒ पौत्रः॑ पु॒त्रः पु॒त्रः पौत्रः॑ । \newline
57. पौत्रो॒ यद् यत् पौत्रः॒ पौत्रो॒ यत् । \newline

\textbf{Ghana Paata } \newline

1. उपे॑ यादिया॒ दुपोपे॑ या॒न् न नेया॒दु पोपे॑ या॒न् न । \newline
2. इ॒या॒न् न ने या॑दिया॒न् न तृ॒तीय॑म् तृ॒तीय॒न् नेया॑ दिया॒न् न तृ॒तीय᳚म् । \newline
3. न तृ॒तीय॑म् तृ॒तीय॒म् न न तृ॒तीय॑म् चि॒त्वा चि॒त्वा तृ॒तीय॒म् न न तृ॒तीय॑म् चि॒त्वा । \newline
4. तृ॒तीय॑म् चि॒त्वा चि॒त्वा तृ॒तीय॑म् तृ॒तीय॑म् चि॒त्वा काम् काम् चि॒त्वा तृ॒तीय॑म् तृ॒तीय॑म् चि॒त्वा काम् । \newline
5. चि॒त्वा काम् काम् चि॒त्वा चि॒त्वा काम् च॒न च॒न काम् चि॒त्वा चि॒त्वा काम् च॒न । \newline
6. काम् च॒न च॒न काम् काम् च॒नो पोप॑ च॒न काम् काम् च॒नोप॑ । \newline
7. च॒नो पोप॑ च॒न च॒नोपे॑या दिया॒ दुप॑ च॒न च॒नोपे॑ यात् । \newline
8. उपे॑ यादिया॒ दुपोपे॑या॒द् रेतो॒ रेत॑ इया॒दुपो पे॑या॒द् रेतः॑ । \newline
9. इ॒या॒द् रेतो॒ रेत॑ इया दिया॒द् रेतो॒ वै वै रेत॑ इया दिया॒द् रेतो॒ वै । \newline
10. रेतो॒ वै वै रेतो॒ रेतो॒ वा ए॒त दे॒तद् वै रेतो॒ रेतो॒ वा ए॒तत् । \newline
11. वा ए॒त दे॒तद् वै वा ए॒तन् नि न्ये॑तद् वै वा ए॒तन् नि । \newline
12. ए॒तन् नि न्ये॑त दे॒तन् नि ध॑त्ते धत्ते॒ न्ये॑त दे॒तन् नि ध॑त्ते । \newline
13. नि ध॑त्ते धत्ते॒ नि नि ध॑त्ते॒ यद् यद् ध॑त्ते॒ नि नि ध॑त्ते॒ यत् । \newline
14. ध॒त्ते॒ यद् यद् ध॑त्ते धत्ते॒ यद॒ग्नि म॒ग्निं ॅयद् ध॑त्ते धत्ते॒ यद॒ग्निम् । \newline
15. यद॒ग्नि म॒ग्निं ॅयद् यद॒ग्निम् चि॑नु॒ते चि॑नु॒ते᳚ ऽग्निं ॅयद् यद॒ग्निम् चि॑नु॒ते । \newline
16. अ॒ग्निम् चि॑नु॒ते चि॑नु॒ते᳚ ऽग्नि म॒ग्निम् चि॑नु॒ते यद् यच् चि॑नु॒ते᳚ ऽग्नि म॒ग्निम् चि॑नु॒ते यत् । \newline
17. चि॒नु॒ते यद् यच् चि॑नु॒ते चि॑नु॒ते यदु॑पे॒या दु॑पे॒याद् यच् चि॑नु॒ते चि॑नु॒ते यदु॑पे॒यात् । \newline
18. यदु॑पे॒या दु॑पे॒याद् यद् यदु॑पे॒याद् रेत॑सा॒ रेत॑सोपे॒याद् यद् यदु॑पे॒याद् रेत॑सा । \newline
19. उ॒पे॒याद् रेत॑सा॒ रेत॑सोपे॒या दु॑पे॒याद् रेत॑सा॒ वि वि रेत॑सोपे॒या दु॑पे॒याद् रेत॑सा॒ वि । \newline
20. उ॒पे॒यादित्यु॑प - इ॒यात् । \newline
21. रेत॑सा॒ वि वि रेत॑सा॒ रेत॑सा॒ व्यृ॑द्ध्येत र्‌द्ध्येत॒ वि रेत॑सा॒ रेत॑सा॒ व्यृ॑द्ध्येत । \newline
22. व्यृ॑द्ध्येत र्‌द्ध्येत॒ वि व्यृ॑द्ध्ये॒ताथो॒ अथो॑ ऋद्ध्येत॒ वि व्यृ॑द्ध्ये॒ताथो᳚ । \newline
23. ऋ॒द्ध्ये॒ ताथो॒ अथो॑ ऋद्ध्येत र्‌द्ध्ये॒ ताथो॒ खलु॒ खल्वथो॑ ऋद्ध्येत र्‌द्ध्ये॒ ताथो॒ खलु॑ । \newline
24. अथो॒ खलु॒ खल्वथो॒ अथो॒ खल्वा॑हु राहुः॒ खल्वथो॒ अथो॒ खल्वा॑हुः । \newline
25. अथो॒ इत्यथो᳚ । \newline
26. खल्वा॑हु राहुः॒ खलु॒ खल्वा॑हु रप्रज॒स्य म॑प्रज॒स्य मा॑हुः॒ खलु॒ खल्वा॑हु रप्रज॒स्यम् । \newline
27. आ॒हु॒ र॒प्र॒ज॒स्य म॑प्रज॒स्य मा॑हु राहु रप्रज॒स्यम् तत् तद॑प्रज॒स्य मा॑हुराहु रप्रज॒स्यम् तत् । \newline
28. अ॒प्र॒ज॒स्यम् तत् तद॑प्रज॒स्य म॑प्रज॒स्यम् तद् यद् यत् तद॑प्रज॒स्य म॑प्रज॒स्यम् तद् यत् । \newline
29. अ॒प्र॒ज॒स्यमित्य॑प्र - ज॒स्यम् । \newline
30. तद् यद् यत् तत् तद् यन् न न यत् तत् तद् यन् न । \newline
31. यन् न न यद् यन् नोपे॒या दु॑पे॒यान् न यद् यन् नोपे॒यात् । \newline
32. नोपे॒या दु॑पे॒यान् न नोपे॒यादिती त्यु॑पे॒यान् न नोपे॒या दिति॑ । \newline
33. उ॒पे॒या दिती त्यु॑पे॒या दु॑पे॒या दिति॒ यद् यदि त्यु॑पे॒या दु॑पे॒या दिति॒ यत् । \newline
34. उ॒पे॒यादित्यु॑प - इ॒यात् । \newline
35. इति॒ यद् यदितीति॒ यद् रे॑त॒स्सिचौ॑ रेत॒स्सिचौ॒ यदितीति॒ यद् रे॑त॒स्सिचौ᳚ । \newline
36. यद् रे॑त॒स्सिचौ॑ रेत॒स्सिचौ॒ यद् यद् रे॑त॒स्सिचा॑ वुप॒दधा᳚ त्युप॒दधा॑ति रेत॒स्सिचौ॒ यद् यद् रे॑त॒स्सिचा॑ वुप॒दधा॑ति । \newline
37. रे॒त॒स्सिचा॑ वुप॒दधा᳚ त्युप॒दधा॑ति रेत॒स्सिचौ॑ रेत॒स्सिचा॑ वुप॒दधा॑ति॒ ते ते उ॑प॒दधा॑ति रेत॒स्सिचौ॑ रेत॒स्सिचा॑ वुप॒दधा॑ति॒ ते । \newline
38. रे॒त॒स्सिचा॒विति॑ रेतः - सिचौ᳚ । \newline
39. उ॒प॒दधा॑ति॒ ते ते उ॑प॒दधा᳚ त्युप॒दधा॑ति॒ ते ए॒वैव ते उ॑प॒दधा᳚ त्युप॒दधा॑ति॒ ते ए॒व । \newline
40. उ॒प॒दधा॒तीत्यु॑प - दधा॑ति । \newline
41. ते ए॒वैव ते ते ए॒व यज॑मानस्य॒ यज॑मानस्यै॒व ते ते ए॒व यज॑मानस्य । \newline
42. ते इति॒ ते । \newline
43. ए॒व यज॑मानस्य॒ यज॑मान स्यै॒वैव यज॑मानस्य॒ रेतो॒ रेतो॒ यज॑मान स्यै॒वैव यज॑मानस्य॒ रेतः॑ । \newline
44. यज॑मानस्य॒ रेतो॒ रेतो॒ यज॑मानस्य॒ यज॑मानस्य॒ रेतो॑ बिभृतो बिभृतो॒ रेतो॒ यज॑मानस्य॒ यज॑मानस्य॒ रेतो॑ बिभृतः । \newline
45. रेतो॑ बिभृतो बिभृतो॒ रेतो॒ रेतो॑ बिभृत॒ स्तस्मा॒त् तस्मा᳚द् बिभृतो॒ रेतो॒ रेतो॑ बिभृत॒ स्तस्मा᳚त् । \newline
46. बि॒भृ॒त॒ स्तस्मा॒त् तस्मा᳚द् बिभृतो बिभृत॒ स्तस्मा॒ दुपोप॒ तस्मा᳚द् बिभृतो बिभृत॒ स्तस्मा॒ दुप॑ । \newline
47. तस्मा॒ दुपोप॒ तस्मा॒त् तस्मा॒ दुपे॑या दिया॒ दुप॒ तस्मा॒त् तस्मा॒ दुपे॑ यात् । \newline
48. उपे॑या दिया॒ दुपोपे॑या॒द् रेत॑सो॒ रेत॑स इया॒ दुपोपे॑या॒द् रेत॑सः । \newline
49. इ॒या॒द् रेत॑सो॒ रेत॑स इया दिया॒द् रेत॒सो ऽस्क॑न्दा॒या स्क॑न्दाय॒ रेत॑स इया दिया॒द् रेत॒सो ऽस्क॑न्दाय । \newline
50. रेत॒सो ऽस्क॑न्दा॒या स्क॑न्दाय॒ रेत॑सो॒ रेत॒सो ऽस्क॑न्दाय॒ त्रीणि॒ त्रीण्य स्क॑न्दाय॒ रेत॑सो॒ रेत॒सो ऽस्क॑न्दाय॒ त्रीणि॑ । \newline
51. अस्क॑न्दाय॒ त्रीणि॒ त्रीण्य स्क॑न्दा॒या स्क॑न्दाय॒ त्रीणि॒ वाव वाव त्रीण्य स्क॑न्दा॒या स्क॑न्दाय॒ त्रीणि॒ वाव । \newline
52. त्रीणि॒ वाव वाव त्रीणि॒ त्रीणि॒ वाव रेताꣳ॑सि॒ रेताꣳ॑सि॒ वाव त्रीणि॒ त्रीणि॒ वाव रेताꣳ॑सि । \newline
53. वाव रेताꣳ॑सि॒ रेताꣳ॑सि॒ वाव वाव रेताꣳ॑सि पि॒ता पि॒ता रेताꣳ॑सि॒ वाव वाव रेताꣳ॑सि पि॒ता । \newline
54. रेताꣳ॑सि पि॒ता पि॒ता रेताꣳ॑सि॒ रेताꣳ॑सि पि॒ता पु॒त्रः पु॒त्रः पि॒ता रेताꣳ॑सि॒ रेताꣳ॑सि पि॒ता पु॒त्रः । \newline
55. पि॒ता पु॒त्रः पु॒त्रः पि॒ता पि॒ता पु॒त्रः पौत्रः॒ पौत्रः॑ पु॒त्रः पि॒ता पि॒ता पु॒त्रः पौत्रः॑ । \newline
56. पु॒त्रः पौत्रः॒ पौत्रः॑ पु॒त्रः पु॒त्रः पौत्रो॒ यद् यत् पौत्रः॑ पु॒त्रः पु॒त्रः पौत्रो॒ यत् । \newline
57. पौत्रो॒ यद् यत् पौत्रः॒ पौत्रो॒ यद् द्वे द्वे यत् पौत्रः॒ पौत्रो॒ यद् द्वे । \newline
\pagebreak
\markright{ TS 5.6.8.5  \hfill https://www.vedavms.in \hfill}

\section{ TS 5.6.8.5 }

\textbf{TS 5.6.8.5 } \newline
\textbf{Samhita Paata} \newline

यद् द्वे रे॑त॒स्सिचा॑वुपद॒द्ध्याद्-रेतो᳚ऽस्य॒ विच्छि॑न्द्यात् ति॒स्र उप॑ दधाति॒ रेत॑सः॒ संत॑त्या इ॒यं ॅवाव प्र॑थ॒मा रे॑त॒स्सिग् वाग्वा इ॒यं तस्मा॒त् पश्य॑न्ती॒मां पश्य॑न्ति॒ वाचं॒ ॅवद॑न्तीम॒न्तरि॑क्षं द्वि॒तीया᳚ प्रा॒णो वा अ॒न्तरि॑क्षं॒ तस्मा॒न्नाऽन्तरि॑क्षं॒ पश्य॑न्ति॒ न प्रा॒णम॒सौ तृ॒तीया॒ चक्षु॒र्वा अ॒सौ तस्मा॒त् पश्य॑न्त्य॒मूं पश्य॑न्ति॒ चक्षु॒र्यजु॑षे॒मां चा॒ - [  ] \newline

\textbf{Pada Paata} \newline

यत् । द्वे इति॑ । रे॒त॒स्सिचा॒विति॑ रेतः-सिचौ᳚ । उ॒प॒द॒द्ध्यादित्यु॑प-द॒ध्यात् । रेतः॑ । अ॒स्य॒ । वीति॑ । छि॒न्द्या॒त् । ति॒स्रः । उपेति॑ । द॒धा॒ति॒ । रेत॑सः । संत॑त्या॒ इति॒ सं - त॒त्यै॒ । इ॒यम् । वाव । प्र॒थ॒मा । रे॒त॒स्सिगिति॑ रेतः - सिक् । वाक् । वै । इ॒यम् । तस्मा᳚त् । पश्य॑न्ति । इ॒माम् । पश्य॑न्ति । वाच᳚म् । वद॑न्तीम् । अ॒न्तरि॑क्षम् । द्वि॒तीया᳚ । प्रा॒ण इति॑ प्र - अ॒नः । वै । अ॒न्तरि॑क्षम् । तस्मा᳚त् । न । अ॒न्तरि॑क्षम् । पश्य॑न्ति । न । प्रा॒णमिति॑ प्र - अ॒नम् । अ॒सौ । तृ॒तीया᳚ ।   चक्षुः॑ । वै । अ॒सौ । तस्मा᳚त् । पश्य॑न्ति । अ॒मूम् । पश्य॑न्ति । चक्षुः॑ । यजु॑षा । इ॒माम् । च॒ ।  \newline


\textbf{Krama Paata} \newline

यद् द्वे । द्वे रे॑त॒स्सिचौ᳚ । द्वे इति॒ द्वे । रे॒त॒स्सिचा॑,वुपद॒द्ध्यात् । रे॒त॒स्सिचा॒विति॑ रेतः - सिचौ᳚ । उ॒प॒द॒द्ध्याद् रेतः॑ । उ॒प॒द॒द्ध्यादित्यु॑प - द॒द्ध्यात् । रेतो᳚ऽस्य । अ॒स्य॒ वि । विच्छि॑न्द्यात् । छि॒न्द्या॒त् ति॒स्रः । ति॒स्र उप॑ । उप॑ दधाति । द॒धा॒ति॒ रेत॑सः । रेत॑सः॒ सन्त॑त्यै । सन्त॑त्या इ॒यम् । सन्त॑त्या॒ इति॒ सम् - त॒त्यै॒ । इ॒यम् ॅवाव । वाव प्र॑थ॒मा । प्र॒थ॒मा रे॑त॒स्सिक् । रे॒त॒स्सिग् वाक् । रे॒त॒स्सिगिति॑ रेतः - सिक् । वाग् वै । वा इ॒यम् । इ॒यम् तस्मा᳚त् । तस्मा॒त् पश्य॑न्ति । पश्य॑न्ती॒माम् । इ॒माम् पश्य॑न्ति । पश्य॑न्ति॒ वाच᳚म् । वाच॒म् ॅवद॑न्तीम् । वद॑न्तीम॒न्तरि॑क्षम् । अ॒न्तरि॑क्षम् द्वि॒तीया᳚ । द्वि॒तीया᳚ प्रा॒णः । प्रा॒णो वै । प्रा॒ण इति॑ प्र - अ॒नः । वा अ॒न्तरि॑क्षम् । अ॒न्तरि॑क्ष॒म् तस्मा᳚त् । तस्मा॒न् न । नान्तरि॑क्षम् । अ॒न्तरि॑क्ष॒म् पश्य॑न्ति । पश्य॑न्ति॒ न । न प्रा॒णम् । प्रा॒णम॒सौ । प्रा॒णमिति॑ प्र - अ॒नम् । अ॒सौ तृ॒तीया᳚ । तृ॒तीया॒ चक्षुः॑ । चक्षु॒र् वै । वा अ॒सौ । अ॒सौ तस्मा᳚त् । तस्मा॒त् पश्य॑न्ति । पश्य॑न्त्य॒मूम् । अ॒मूम् पश्य॑न्ति । पश्य॑न्ति॒ चक्षुः॑ । चक्षु॒र् यजु॑षा । यजु॑षे॒माम् । इ॒माम् च॑ । चा॒मूम् \newline

\textbf{Jatai Paata} \newline

1. यद् द्वे द्वे यद् यद् द्वे । \newline
2. द्वे रे॑त॒स्सिचौ॑ रेत॒स्सिचौ॒ द्वे द्वे रे॑त॒स्सिचौ᳚ । \newline
3. द्वे इति॒ द्वे । \newline
4. रे॒त॒स्सिचा॑ वुपद॒द्ध्या दु॑पद॒द्ध्याद् रे॑त॒स्सिचौ॑ रेत॒स्सिचा॑ वुपद॒द्ध्यात् । \newline
5. रे॒त॒स्सिचा॒विति॑ रेतः - सिचौ᳚ । \newline
6. उ॒प॒द॒द्ध्याद् रेतो॒ रेत॑ उपद॒द्ध्या दु॑पद॒द्ध्याद् रेतः॑ । \newline
7. उ॒प॒द॒द्ध्यादित्यु॑प - द॒ध्यात् । \newline
8. रेतो᳚ ऽस्यास्य॒ रेतो॒ रेतो᳚ ऽस्य । \newline
9. अ॒स्य॒ वि व्य॑स्यास्य॒ वि । \newline
10. वि च्छि॑न्द्याच् छिन्द्या॒द् वि वि च्छि॑न्द्यात् । \newline
11. छि॒न्द्या॒त् ति॒स्र स्ति॒स्र श्छि॑न्द्याच् छिन्द्यात् ति॒स्रः । \newline
12. ति॒स्र उपोप॑ ति॒स्र स्ति॒स्र उप॑ । \newline
13. उप॑ दधाति दधा॒ त्युपोप॑ दधाति । \newline
14. द॒धा॒ति॒ रेत॑सो॒ रेत॑सो दधाति दधाति॒ रेत॑सः । \newline
15. रेत॑सः॒ सन्त॑त्यै॒ सन्त॑त्यै॒ रेत॑सो॒ रेत॑सः॒ सन्त॑त्यै । \newline
16. सन्त॑त्या इ॒य मि॒यꣳ सन्त॑त्यै॒ सन्त॑त्या इ॒यम् । \newline
17. सन्त॑त्या॒ इति॒ सं - त॒त्यै॒ । \newline
18. इ॒यं ॅवाव वावेय मि॒यं ॅवाव । \newline
19. वाव प्र॑थ॒मा प्र॑थ॒मा वाव वाव प्र॑थ॒मा । \newline
20. प्र॒थ॒मा रे॑त॒स्सिग् रे॑त॒स्सिक् प्र॑थ॒मा प्र॑थ॒मा रे॑त॒स्सिक् । \newline
21. रे॒त॒स्सिग् वाग् वाग् रे॑त॒स्सिग् रे॑त॒स्सिग् वाक् । \newline
22. रे॒त॒स्सिगिति॑ रेतः - सिक् । \newline
23. वाग् वै वै वाग् वाग् वै । \newline
24. वा इ॒य मि॒यं ॅवै वा इ॒यम् । \newline
25. इ॒यम् तस्मा॒त् तस्मा॑ दि॒य मि॒यम् तस्मा᳚त् । \newline
26. तस्मा॒त् पश्य॑न्ति॒ पश्य॑न्ति॒ तस्मा॒त् तस्मा॒त् पश्य॑न्ति । \newline
27. पश्य॑न्ती॒मा मि॒माम् पश्य॑न्ति॒ पश्य॑न्ती॒माम् । \newline
28. इ॒माम् पश्य॑न्ति॒ पश्य॑न्ती॒मा मि॒माम् पश्य॑न्ति । \newline
29. पश्य॑न्ति॒ वाचं॒ ॅवाच॒म् पश्य॑न्ति॒ पश्य॑न्ति॒ वाच᳚म् । \newline
30. वाचं॒ ॅवद॑न्तीं॒ ॅवद॑न्तीं॒ ॅवाचं॒ ॅवाचं॒ ॅवद॑न्तीम् । \newline
31. वद॑न्ती म॒न्तरि॑क्ष म॒न्तरि॑क्षं॒ ॅवद॑न्तीं॒ ॅवद॑न्ती म॒न्तरि॑क्षम् । \newline
32. अ॒न्तरि॑क्षम् द्वि॒तीया᳚ द्वि॒तीया॒ ऽन्तरि॑क्ष म॒न्तरि॑क्षम् द्वि॒तीया᳚ । \newline
33. द्वि॒तीया᳚ प्रा॒णः प्रा॒णो द्वि॒तीया᳚ द्वि॒तीया᳚ प्रा॒णः । \newline
34. प्रा॒णो वै वै प्रा॒णः प्रा॒णो वै । \newline
35. प्रा॒ण इति॑ प्र - अ॒नः । \newline
36. वा अ॒न्तरि॑क्ष म॒न्तरि॑क्षं॒ ॅवै वा अ॒न्तरि॑क्षम् । \newline
37. अ॒न्तरि॑क्ष॒म् तस्मा॒त् तस्मा॑ द॒न्तरि॑क्ष म॒न्तरि॑क्ष॒म् तस्मा᳚त् । \newline
38. तस्मा॒न् न न तस्मा॒त् तस्मा॒न् न । \newline
39. नान्तरि॑क्ष म॒न्तरि॑क्ष॒न्न नान्तरि॑क्षम् । \newline
40. अ॒न्तरि॑क्ष॒म् पश्य॑न्ति॒ पश्य॑न् त्य॒न्तरि॑क्ष म॒न्तरि॑क्ष॒म् पश्य॑न्ति । \newline
41. पश्य॑न्ति॒ न न पश्य॑न्ति॒ पश्य॑न्ति॒ न । \newline
42. न प्रा॒णम् प्रा॒णम् न न प्रा॒णम् । \newline
43. प्रा॒ण म॒सा व॒सौ प्रा॒णम् प्रा॒ण म॒सौ । \newline
44. प्रा॒णमिति॑ प्र - अ॒नम् । \newline
45. अ॒सौ तृ॒तीया॑ तृ॒तीया॒ ऽसा व॒सौ तृ॒तीया᳚ । \newline
46. तृ॒तीया॒ चक्षु॒ श्चक्षु॑ स्तृ॒तीया॑ तृ॒तीया॒ चक्षुः॑ । \newline
47. चक्षु॒र् वै वै चक्षु॒ श्चक्षु॒र् वै । \newline
48. वा अ॒सा व॒सौ वै वा अ॒सौ । \newline
49. अ॒सौ तस्मा॒त् तस्मा॑ द॒सा व॒सौ तस्मा᳚त् । \newline
50. तस्मा॒त् पश्य॑न्ति॒ पश्य॑न्ति॒ तस्मा॒त् तस्मा॒त् पश्य॑न्ति । \newline
51. पश्य॑न् त्य॒मू म॒मूम् पश्य॑न्ति॒ पश्य॑न् त्य॒मूम् । \newline
52. अ॒मूम् पश्य॑न्ति॒ पश्य॑न् त्य॒मू म॒मूम् पश्य॑न्ति । \newline
53. पश्य॑न्ति॒ चक्षु॒ श्चक्षुः॒ पश्य॑न्ति॒ पश्य॑न्ति॒ चक्षुः॑ । \newline
54. चक्षु॒र् यजु॑षा॒ यजु॑षा॒ चक्षु॒ श्चक्षु॒र् यजु॑षा । \newline
55. यजु॑षे॒मा मि॒मां ॅयजु॑षा॒ यजु॑षे॒माम् । \newline
56. इ॒माम् च॑ चे॒मा मि॒माम् च॑ । \newline
57. चा॒मू म॒मूम् च॑ चा॒मूम् । \newline

\textbf{Ghana Paata } \newline

1. यद् द्वे द्वे यद् यद् द्वे रे॑त॒स्सिचौ॑ रेत॒स्सिचौ॒ द्वे यद् यद् द्वे रे॑त॒स्सिचौ᳚ । \newline
2. द्वे रे॑त॒स्सिचौ॑ रेत॒स्सिचौ॒ द्वे द्वे रे॑त॒स्सिचा॑ वुपद॒द्ध्या दु॑पद॒द्ध्याद् रे॑त॒स्सिचौ॒ द्वे द्वे रे॑त॒स्सिचा॑ वुपद॒द्ध्यात् । \newline
3. द्वे इति॒ द्वे । \newline
4. रे॒त॒स्सिचा॑ वुपद॒द्ध्या दु॑पद॒द्ध्याद् रे॑त॒स्सिचौ॑ रेत॒स्सिचा॑ वुपद॒द्ध्याद् रेतो॒ रेत॑ उपद॒द्ध्याद् रे॑त॒स्सिचौ॑ रेत॒स्सिचा॑ वुपद॒द्ध्याद् रेतः॑ । \newline
5. रे॒त॒स्सिचा॒विति॑ रेतः - सिचौ᳚ । \newline
6. उ॒प॒द॒द्ध्याद् रेतो॒ रेत॑ उपद॒द्ध्या दु॑पद॒द्ध्याद् रेतो᳚ ऽस्यास्य॒ रेत॑ उपद॒द्ध्या दु॑पद॒द्ध्याद् रेतो᳚ ऽस्य । \newline
7. उ॒प॒द॒द्ध्यादित्यु॑प - द॒ध्यात् । \newline
8. रेतो᳚ ऽस्यास्य॒ रेतो॒ रेतो᳚ ऽस्य॒ वि व्य॑स्य॒ रेतो॒ रेतो᳚ ऽस्य॒ वि । \newline
9. अ॒स्य॒ वि व्य॑स्यास्य॒ वि च्छि॑न्द्याच् छिन्द्या॒द् व्य॑स्यास्य॒ वि च्छि॑न्द्यात् । \newline
10. वि च्छि॑न्द्याच् छिन्द्या॒द् वि वि च्छि॑न्द्यात् ति॒स्र स्ति॒स्र श्छि॑न्द्या॒द् वि वि च्छि॑न्द्यात् ति॒स्रः । \newline
11. छि॒न्द्या॒त् ति॒स्र स्ति॒स्र श्छि॑न्द्याच् छिन्द्यात् ति॒स्र उपोप॑ ति॒स्र श्छि॑न्द्या च्छिन्द्यात् ति॒स्र उप॑ । \newline
12. ति॒स्र उपोप॑ ति॒स्र स्ति॒स्र उप॑ दधाति दधा॒ त्युप॑ ति॒स्र स्ति॒स्र उप॑ दधाति । \newline
13. उप॑ दधाति दधा॒ त्युपोप॑ दधाति॒ रेत॑सो॒ रेत॑सो दधा॒ त्युपोप॑ दधाति॒ रेत॑सः । \newline
14. द॒धा॒ति॒ रेत॑सो॒ रेत॑सो दधाति दधाति॒ रेत॑सः॒ सन्त॑त्यै॒ सन्त॑त्यै॒ रेत॑सो दधाति दधाति॒ रेत॑सः॒ सन्त॑त्यै । \newline
15. रेत॑सः॒ सन्त॑त्यै॒ सन्त॑त्यै॒ रेत॑सो॒ रेत॑सः॒ सन्त॑त्या इ॒य मि॒यꣳ सन्त॑त्यै॒ रेत॑सो॒ रेत॑सः॒ सन्त॑त्या इ॒यम् । \newline
16. सन्त॑त्या इ॒य मि॒यꣳ सन्त॑त्यै॒ सन्त॑त्या इ॒यं ॅवाव वावेयꣳ सन्त॑त्यै॒ सन्त॑त्या इ॒यं ॅवाव । \newline
17. सन्त॑त्या॒ इति॒ सं - त॒त्यै॒ । \newline
18. इ॒यं ॅवाव वावेय मि॒यं ॅवाव प्र॑थ॒मा प्र॑थ॒मा वावेय मि॒यं ॅवाव प्र॑थ॒मा । \newline
19. वाव प्र॑थ॒मा प्र॑थ॒मा वाव वाव प्र॑थ॒मा रे॑त॒स्सिग् रे॑त॒स्सिक् प्र॑थ॒मा वाव वाव प्र॑थ॒मा रे॑त॒स्सिक् । \newline
20. प्र॒थ॒मा रे॑त॒स्सिग् रे॑त॒स्सिक् प्र॑थ॒मा प्र॑थ॒मा रे॑त॒स्सिग् वाग् वाग् रे॑त॒स्सिक् प्र॑थ॒मा प्र॑थ॒मा रे॑त॒स्सिग् वाक् । \newline
21. रे॒त॒स्सिग् वाग् वाग् रे॑त॒स्सिग् रे॑त॒स्सिग् वाग् वै वै वाग् रे॑त॒स्सिग् रे॑त॒स्सिग् वाग् वै । \newline
22. रे॒त॒स्सिगिति॑ रेतः - सिक् । \newline
23. वाग् वै वै वाग् वाग् वा इ॒य मि॒यं ॅवै वाग् वाग् वा इ॒यम् । \newline
24. वा इ॒य मि॒यं ॅवै वा इ॒यम् तस्मा॒त् तस्मा॑ दि॒यं ॅवै वा इ॒यम् तस्मा᳚त् । \newline
25. इ॒यम् तस्मा॒त् तस्मा॑ दि॒य मि॒यम् तस्मा॒त् पश्य॑न्ति॒ पश्य॑न्ति॒ तस्मा॑ दि॒य मि॒यम् तस्मा॒त् पश्य॑न्ति । \newline
26. तस्मा॒त् पश्य॑न्ति॒ पश्य॑न्ति॒ तस्मा॒त् तस्मा॒त् पश्य॑न्ती॒मा मि॒माम् पश्य॑न्ति॒ तस्मा॒त् तस्मा॒त् पश्य॑न्ती॒माम् । \newline
27. पश्य॑न्ती॒मा मि॒माम् पश्य॑न्ति॒ पश्य॑न्ती॒माम् पश्य॑न्ति॒ पश्य॑न्ती॒माम् पश्य॑न्ति॒ पश्य॑न्ती॒माम् पश्य॑न्ति । \newline
28. इ॒माम् पश्य॑न्ति॒ पश्य॑न्ती॒मा मि॒माम् पश्य॑न्ति॒ वाचं॒ ॅवाच॒म् पश्य॑न्ती॒मा मि॒माम् पश्य॑न्ति॒ वाच᳚म् । \newline
29. पश्य॑न्ति॒ वाचं॒ ॅवाच॒म् पश्य॑न्ति॒ पश्य॑न्ति॒ वाचं॒ ॅवद॑न्तीं॒ ॅवद॑न्तीं॒ ॅवाच॒म् पश्य॑न्ति॒ पश्य॑न्ति॒ वाचं॒ ॅवद॑न्तीम् । \newline
30. वाचं॒ ॅवद॑न्तीं॒ ॅवद॑न्तीं॒ ॅवाचं॒ ॅवाचं॒ ॅवद॑न्ती म॒न्तरि॑क्ष म॒न्तरि॑क्षं॒ ॅवद॑न्तीं॒ ॅवाचं॒ ॅवाचं॒ ॅवद॑न्ती म॒न्तरि॑क्षम् । \newline
31. वद॑न्ती म॒न्तरि॑क्ष म॒न्तरि॑क्षं॒ ॅवद॑न्तीं॒ ॅवद॑न्ती म॒न्तरि॑क्षम् द्वि॒तीया᳚ द्वि॒तीया॒ ऽन्तरि॑क्षं॒ ॅवद॑न्तीं॒ ॅवद॑न्ती म॒न्तरि॑क्षम् द्वि॒तीया᳚ । \newline
32. अ॒न्तरि॑क्षम् द्वि॒तीया᳚ द्वि॒तीया॒ ऽन्तरि॑क्ष म॒न्तरि॑क्षम् द्वि॒तीया᳚ प्रा॒णः प्रा॒णो द्वि॒तीया॒ ऽन्तरि॑क्ष म॒न्तरि॑क्षम् द्वि॒तीया᳚ प्रा॒णः । \newline
33. द्वि॒तीया᳚ प्रा॒णः प्रा॒णो द्वि॒तीया᳚ द्वि॒तीया᳚ प्रा॒णो वै वै प्रा॒णो द्वि॒तीया᳚ द्वि॒तीया᳚ प्रा॒णो वै । \newline
34. प्रा॒णो वै वै प्रा॒णः प्रा॒णो वा अ॒न्तरि॑क्ष म॒न्तरि॑क्षं॒ ॅवै प्रा॒णः प्रा॒णो वा अ॒न्तरि॑क्षम् । \newline
35. प्रा॒ण इति॑ प्र - अ॒नः । \newline
36. वा अ॒न्तरि॑क्ष म॒न्तरि॑क्षं॒ ॅवै वा अ॒न्तरि॑क्ष॒म् तस्मा॒त् तस्मा॑ द॒न्तरि॑क्षं॒ ॅवै वा अ॒न्तरि॑क्ष॒म् तस्मा᳚त् । \newline
37. अ॒न्तरि॑क्ष॒म् तस्मा॒त् तस्मा॑ द॒न्तरि॑क्ष म॒न्तरि॑क्ष॒म् तस्मा॒न् न न तस्मा॑ द॒न्तरि॑क्ष म॒न्तरि॑क्ष॒म् तस्मा॒न् न । \newline
38. तस्मा॒न् न न तस्मा॒त् तस्मा॒न् नान्तरि॑क्ष म॒न्तरि॑क्ष॒म् न तस्मा॒त् तस्मा॒न् नान्तरि॑क्षम् । \newline
39. नान्तरि॑क्ष म॒न्तरि॑क्ष॒म् न नान्तरि॑क्ष॒म् पश्य॑न्ति॒ पश्य॑ न्त्य॒न्तरि॑क्ष॒म् न नान्तरि॑क्ष॒म् पश्य॑न्ति । \newline
40. अ॒न्तरि॑क्ष॒म् पश्य॑न्ति॒ पश्य॑ न्त्य॒न्तरि॑क्ष म॒न्तरि॑क्ष॒म् पश्य॑न्ति॒ न न पश्य॑ न्त्य॒न्तरि॑क्ष म॒न्तरि॑क्ष॒म् पश्य॑न्ति॒ न । \newline
41. पश्य॑न्ति॒ न न पश्य॑न्ति॒ पश्य॑न्ति॒ न प्रा॒णम् प्रा॒णम् न पश्य॑न्ति॒ पश्य॑न्ति॒ न प्रा॒णम् । \newline
42. न प्रा॒णम् प्रा॒णम् न न प्रा॒ण म॒सा व॒सौ प्रा॒णम् न न प्रा॒ण म॒सौ । \newline
43. प्रा॒ण म॒सा व॒सौ प्रा॒णम् प्रा॒ण म॒सौ तृ॒तीया॑ तृ॒तीया॒ ऽसौ प्रा॒णम् प्रा॒ण म॒सौ तृ॒तीया᳚ । \newline
44. प्रा॒णमिति॑ प्र - अ॒नम् । \newline
45. अ॒सौ तृ॒तीया॑ तृ॒तीया॒ ऽसा व॒सौ तृ॒तीया॒ चक्षु॒ श्चक्षु॑ स्तृ॒तीया॒ ऽसा व॒सौ तृ॒तीया॒ चक्षुः॑ । \newline
46. तृ॒तीया॒ चक्षु॒ श्चक्षु॑ स्तृ॒तीया॑ तृ॒तीया॒ चक्षु॒र् वै वै चक्षु॑ स्तृ॒तीया॑ तृ॒तीया॒ चक्षु॒र् वै । \newline
47. चक्षु॒र् वै वै चक्षु॒ श्चक्षु॒र् वा अ॒सा व॒सौ वै चक्षु॒ श्चक्षु॒र् वा अ॒सौ । \newline
48. वा अ॒सा व॒सौ वै वा अ॒सौ तस्मा॒त् तस्मा॑ द॒सौ वै वा अ॒सौ तस्मा᳚त् । \newline
49. अ॒सौ तस्मा॒त् तस्मा॑ द॒सा व॒सौ तस्मा॒त् पश्य॑न्ति॒ पश्य॑न्ति॒ तस्मा॑ द॒सा व॒सौ तस्मा॒त् पश्य॑न्ति । \newline
50. तस्मा॒त् पश्य॑न्ति॒ पश्य॑न्ति॒ तस्मा॒त् तस्मा॒त् पश्य॑न्त्य॒मू म॒मूम् पश्य॑न्ति॒ तस्मा॒त् तस्मा॒त् पश्य॑न्त्य॒मूम् । \newline
51. पश्य॑न्त्य॒मू म॒मूम् पश्य॑न्ति॒ पश्य॑न्त्य॒मूम् पश्य॑न्ति॒ पश्य॑न् त्य॒मूम् पश्य॑न्ति॒ पश्य॑न्त्य॒मूम् पश्य॑न्ति । \newline
52. अ॒मूम् पश्य॑न्ति॒ पश्य॑न्त्य॒मू म॒मूम् पश्य॑न्ति॒ चक्षु॒ श्चक्षुः॒ पश्य॑न्त्य॒मू म॒मूम् पश्य॑न्ति॒ चक्षुः॑ । \newline
53. पश्य॑न्ति॒ चक्षु॒ श्चक्षुः॒ पश्य॑न्ति॒ पश्य॑न्ति॒ चक्षु॒र् यजु॑षा॒ यजु॑षा॒ चक्षुः॒ पश्य॑न्ति॒ पश्य॑न्ति॒ चक्षु॒र् यजु॑षा । \newline
54. चक्षु॒र् यजु॑षा॒ यजु॑षा॒ चक्षु॒ श्चक्षु॒र् यजु॑षे॒मा मि॒मां ॅयजु॑षा॒ चक्षु॒ श्चक्षु॒र् यजु॑षे॒माम् । \newline
55. यजु॑षे॒मा मि॒मां ॅयजु॑षा॒ यजु॑षे॒माम् च॑ चे॒मां ॅयजु॑षा॒ यजु॑षे॒माम् च॑ । \newline
56. इ॒माम् च॑ चे॒मा मि॒माम् चा॒मू म॒मूम् चे॒मा मि॒माम् चा॒मूम् । \newline
57. चा॒मू म॒मूम् च॑ चा॒मूम् च॑ चा॒मूम् च॑ चा॒मूम् च॑ । \newline
\pagebreak
\markright{ TS 5.6.8.6  \hfill https://www.vedavms.in \hfill}

\section{ TS 5.6.8.6 }

\textbf{TS 5.6.8.6 } \newline
\textbf{Samhita Paata} \newline

-मूं चोप॑ दधाति॒ मन॑सा मद्ध्य॒मामे॒षां ॅलो॒कानां॒ क्लृप्त्या॒ अथो᳚ प्रा॒णाना॑मि॒ष्टो य॒ज्ञो भृगु॑भिराशी॒र्दा वसु॑भि॒स्तस्य॑ त इ॒ष्टस्य॑ वी॒तस्य॒ द्रवि॑णे॒ह भ॑क्षी॒येत्या॑ह स्तुतश॒स्त्रे ए॒वैतेन॑ दुहे पि॒ता मा॑त॒रिश्वाऽच्छि॑द्रा प॒दा धा॒ अच्छि॑द्रा उ॒शिजः॑ प॒दाऽनु॑ तक्षुः॒ सोमो॑ विश्व॒विन्ने॒ता ने॑ष॒द्-बृह॒स्पति॑रुक्थाम॒दानि॑ शꣳसिष॒दित्या॑है॒तद्वा ( ) अ॒ग्नेरु॒क्थं तेनै॒वैन॒मनु॑ शꣳसति ॥ \newline

\textbf{Pada Paata} \newline

अ॒मूम् । च॒ । उपेति॑ । द॒धा॒ति॒ । मन॑सा । म॒द्ध्य॒माम् । ए॒षाम् । लो॒काना᳚म् । क्लृप्त्यै᳚ । अथो॒ इति॑ । प्रा॒णाना॒मिति॑ प्र - अ॒नाना᳚म् । इ॒ष्टः । य॒ज्ञ्ः । भृगु॑भि॒रिति॒ भृगु॑ - भिः॒ । आ॒शी॒र्दा इत्या॑शीः - दाः । वसु॑भि॒रिति॒ वसु॑ - भिः॒ । तस्य॑ । ते॒ । इ॒ष्टस्य॑ । वी॒तस्य॑ । द्रवि॑णा । इ॒ह । भ॒क्षी॒य॒ । इति॑ । आ॒ह॒ । स्तु॒त॒श॒स्त्रे इति॑ स्तुत - श॒स्त्रे । ए॒व । ए॒तेन॑ । दु॒हे॒ । पि॒ता । मा॒त॒रिश्वा᳚ । अच्छि॑द्रा । प॒दा । धाः॒ । अच्छि॑द्राः । उ॒शिजः॑ । प॒दा । अन्विति॑ । त॒क्षुः॒ । सोमः॑ । वि॒श्व॒विदिति॑ विश्व - वित् । ने॒ता । ने॒ष॒त् । बृह॒स्पतिः॑ । उ॒क्था॒म॒दानीत्यु॑क्थ - म॒दानि॑ । शꣳ॒॒सि॒ष॒त् । इति॑ । आ॒ह॒ । ए॒तत् । वै ( ) । अ॒ग्नेः । उ॒क्थम् । तेन॑ । ए॒व । ए॒न॒म् । अन्विति॑ । शꣳ॒॒स॒ति॒ ॥  \newline


\textbf{Krama Paata} \newline

अ॒मूम् च॑ । चोप॑ । उप॑ दधाति । द॒धा॒ति॒ मन॑सा । मन॑सा मद्ध्य॒माम् । म॒द्ध्य॒मामे॒षाम् । ए॒षाम् ॅलो॒काना᳚म् । लो॒काना॒म् क्लृप्त्यै᳚ । क्लृप्त्या॒ अथो᳚ । अथो᳚ प्रा॒णाना᳚म् । अथो॒ इत्यथो᳚ । प्रा॒णाना॑मि॒ष्टः । प्रा॒णाना॒मिति॑ प्र - अ॒नाना᳚म् । इ॒ष्टो य॒ज्ञ्ः । य॒ज्ञो भृगु॑भिः । भृगु॑भिराशी॒र्दाः । भृगु॑भि॒रिति॒ भृगु॑ - भिः॒ । आ॒शी॒र्दा वसु॑भिः । आ॒शी॒र्दा इत्या॑शीः - दाः । वसु॑भि॒स्तस्य॑ । वसु॑भि॒रिति॒ वसु॑ - भिः॒ । तस्य॑ ते । त॒ इ॒ष्टस्य॑ । इ॒ष्टस्य॑ वी॒तस्य॑ । वी॒तस्य॒ द्रवि॑णा । द्रवि॑णे॒ह । इ॒ह भ॑क्षीय । भ॒क्षी॒येति॑ । इत्या॑ह । आ॒ह॒ स्तु॒त॒श॒स्त्रे । स्तु॒त॒श॒स्त्रे ए॒व । स्तु॒त॒श॒स्त्रे इति॑ स्तुत - श॒स्त्रे । ए॒वैतेन॑ । ए॒तेन॑ दुहे । दु॒हे॒ पि॒ता । पि॒ता मा॑त॒रिश्वा᳚ । मा॒त॒रिश्वाऽच्छि॑द्रा । अच्छि॑द्रा प॒दा । प॒दा धाः᳚ । धा॒ अच्छि॑द्राः । अच्छि॑द्रा उ॒शिजः॑ । उ॒शिजः॑ प॒दा । प॒दाऽनु॑ । अनु॑ तक्षुः । त॒क्षुः॒ सोमः॑ । सोमो॑ विश्व॒वित् । वि॒श्व॒विन् ने॒ता । वि॒श्व॒विदिति॑ विश्व - वित् । ने॒ता ने॑षत् । ने॒ष॒द् बृह॒स्पतिः॑ । बृह॒स्पति॑रुक्थाम॒दानि॑ । उ॒क्था॒म॒दानि॑ शꣳसिषत् । उ॒क्था॒म॒दानीत्यु॑क्थ - म॒दानि॑ । शꣳ॒॒सि॒ष॒दिति॑ । इत्या॑ह । आ॒है॒तत् । ए॒तद् वै ( ) । वा अ॒ग्नेः । अ॒ग्नेरु॒क्थम् । उ॒क्थम् तेन॑ । तेनै॒व । ए॒वैन᳚म् । ए॒न॒मनु॑ । अनु॑ शꣳसति । शꣳ॒॒स॒तीति॑ सꣳसति । \newline

\textbf{Jatai Paata} \newline

1. अ॒मूम् च॑ चा॒मू म॒मूम् च॑ । \newline
2. चोपोप॑ च॒ चोप॑ । \newline
3. उप॑ दधाति दधा॒ त्युपोप॑ दधाति । \newline
4. द॒धा॒ति॒ मन॑सा॒ मन॑सा दधाति दधाति॒ मन॑सा । \newline
5. मन॑सा मद्ध्य॒माम् म॑द्ध्य॒माम् मन॑सा॒ मन॑सा मद्ध्य॒माम् । \newline
6. म॒द्ध्य॒मा मे॒षा मे॒षाम् म॑द्ध्य॒माम् म॑द्ध्य॒मा मे॒षाम् । \newline
7. ए॒षाम् ॅलो॒काना᳚म् ॅलो॒काना॑ मे॒षा मे॒षाम् ॅलो॒काना᳚म् । \newline
8. लो॒काना॒म् क्लृप्त्यै॒ क्लृप्त्यै॑ लो॒काना᳚म् ॅलो॒काना॒म् क्लृप्त्यै᳚ । \newline
9. क्लृप्त्या॒ अथो॒ अथो॒ क्लृप्त्यै॒ क्लृप्त्या॒ अथो᳚ । \newline
10. अथो᳚ प्रा॒णाना᳚म् प्रा॒णाना॒ मथो॒ अथो᳚ प्रा॒णाना᳚म् । \newline
11. अथो॒ इत्यथो᳚ । \newline
12. प्रा॒णाना॑ मि॒ष्ट इ॒ष्टः प्रा॒णाना᳚म् प्रा॒णाना॑ मि॒ष्टः । \newline
13. प्रा॒णाना॒मिति॑ प्र - अ॒नाना᳚म् । \newline
14. इ॒ष्टो य॒ज्ञो य॒ज्ञ् इ॒ष्ट इ॒ष्टो य॒ज्ञ्ः । \newline
15. य॒ज्ञो भृगु॑भि॒र् भृगु॑भिर् य॒ज्ञो य॒ज्ञो भृगु॑भिः । \newline
16. भृगु॑भि राशी॒र्दा आ॑शी॒र्दा भृगु॑भि॒र् भृगु॑भि राशी॒र्दाः । \newline
17. भृगु॑भि॒रिति॒ भृगु॑ - भिः॒ । \newline
18. आ॒शी॒र्दा वसु॑भि॒र् वसु॑भि राशी॒र्दा आ॑शी॒र्दा वसु॑भिः । \newline
19. आ॒शी॒र्दा इत्या॑शीः - दाः । \newline
20. वसु॑भि॒ स्तस्य॒ तस्य॒ वसु॑भि॒र् वसु॑भि॒ स्तस्य॑ । \newline
21. वसु॑भि॒रिति॒ वसु॑ - भिः॒ । \newline
22. तस्य॑ ते ते॒ तस्य॒ तस्य॑ ते । \newline
23. त॒ इ॒ष्ट स्ये॒ष्टस्य॑ ते त इ॒ष्टस्य॑ । \newline
24. इ॒ष्टस्य॑ वी॒तस्य॑ वी॒त स्ये॒ष्ट स्ये॒ष्टस्य॑ वी॒तस्य॑ । \newline
25. वी॒तस्य॒ द्रवि॑णा॒ द्रवि॑णा वी॒तस्य॑ वी॒तस्य॒ द्रवि॑णा । \newline
26. द्रवि॑णे॒हेह द्रवि॑णा॒ द्रवि॑णे॒ह । \newline
27. इ॒ह भ॑क्षीय भक्षीये॒हेह भ॑क्षीय । \newline
28. भ॒क्षी॒ये तीति॑ भक्षीय भक्षी॒येति॑ । \newline
29. इत्या॑हा॒हे तीत्या॑ह । \newline
30. आ॒ह॒ स्तु॒त॒श॒स्त्रे स्तु॑तश॒स्त्रे आ॑हाह स्तुतश॒स्त्रे । \newline
31. स्तु॒त॒श॒स्त्रे ए॒वैव स्तु॑तश॒स्त्रे स्तु॑तश॒स्त्रे ए॒व । \newline
32. स्तु॒त॒श॒स्त्रे इति॑ स्तुत - श॒स्त्रे । \newline
33. ए॒वैते नै॒ते नै॒वै वैतेन॑ । \newline
34. ए॒तेन॑ दुहे दुह ए॒ते नै॒तेन॑ दुहे । \newline
35. दु॒हे॒ पि॒ता पि॒ता दु॑हे दुहे पि॒ता । \newline
36. पि॒ता मा॑त॒रिश्वा॑ मात॒रिश्वा॑ पि॒ता पि॒ता मा॑त॒रिश्वा᳚ । \newline
37. मा॒त॒रिश्वा ऽच्छि॒द्रा ऽच्छि॑द्रा मात॒रिश्वा॑ मात॒रिश्वा ऽच्छि॑द्रा । \newline
38. अच्छि॑द्रा प॒दा प॒दा ऽच्छि॒द्रा ऽच्छि॑द्रा प॒दा । \newline
39. प॒दा धा॑ धाः प॒दा प॒दा धाः᳚ । \newline
40. धा॒ अच्छि॑द्रा॒ अच्छि॑द्रा धा धा॒ अच्छि॑द्राः । \newline
41. अच्छि॑द्रा उ॒शिज॑ उ॒शिजो ऽच्छि॑द्रा॒ अच्छि॑द्रा उ॒शिजः॑ । \newline
42. उ॒शिजः॑ प॒दा प॒दोशिज॑ उ॒शिजः॑ प॒दा । \newline
43. प॒दा ऽन्वनु॑ प॒दा प॒दा ऽनु॑ । \newline
44. अनु॑ तक्षु स्तक्षु॒ रन्वनु॑ तक्षुः । \newline
45. त॒क्षुः॒ सोमः॒ सोम॑ स्तक्षु स्तक्षुः॒ सोमः॑ । \newline
46. सोमो॑ विश्व॒विद् वि॑श्व॒विथ् सोमः॒ सोमो॑ विश्व॒वित् । \newline
47. वि॒श्व॒विन् ने॒ता ने॒ता वि॑श्व॒विद् वि॑श्व॒विन् ने॒ता । \newline
48. वि॒श्व॒विदिति॑ विश्व - वित् । \newline
49. ने॒ता ने॑षन् नेषन् ने॒ता ने॒ता ने॑षत् । \newline
50. ने॒ष॒द् बृह॒स्पति॒र् बृह॒स्पति॑र् नेषन् नेष॒द् बृह॒स्पतिः॑ । \newline
51. बृह॒स्पति॑ रुक्थाम॒दा न्यु॑क्थाम॒दानि॒ बृह॒स्पति॒र् बृह॒स्पति॑ रुक्थाम॒दानि॑ । \newline
52. उ॒क्था॒म॒दानि॑ शꣳसिष च्छꣳसिष दुक्थाम॒दा न्यु॑क्थाम॒दानि॑ शꣳसिषत् । \newline
53. उ॒क्था॒म॒दानीत्यु॑क्थ - म॒दानि॑ । \newline
54. शꣳ॒॒सि॒ष॒दितीति॑ शꣳसिष च्छꣳसिष॒दिति॑ । \newline
55. इत्या॑हा॒हे तीत्या॑ह । \newline
56. आ॒है॒ तदे॒त दा॑हा है॒तत् । \newline
57. ए॒तद् वै वा ए॒त दे॒तद् वै । \newline
58. वा अ॒ग्ने र॒ग्नेर् वै वा अ॒ग्नेः । \newline
59. अ॒ग्ने रु॒क्थ मु॒क्थ म॒ग्ने र॒ग्ने रु॒क्थम् । \newline
60. उ॒क्थम् तेन॒ तेनो॒क्थ मु॒क्थम् तेन॑ । \newline
61. तेनै॒ वैव तेन॒ तेनै॒व । \newline
62. ए॒वैन॑ मेन मे॒वै वैन᳚म् । \newline
63. ए॒न॒ मन् वन् वे॑न मेन॒ मनु॑ । \newline
64. अनु॑ शꣳसति शꣳस॒ त्यन् वनु॑ शꣳसति । \newline
65. शꣳ॒॒स॒तीति॑ सꣳसति । \newline

\textbf{Ghana Paata } \newline

1. अ॒मूम् च॑ चा॒मू म॒मूम् चोपोप॑ चा॒मू म॒मूम् चोप॑ । \newline
2. चोपोप॑ च॒ चोप॑ दधाति दधा॒ त्युप॑ च॒ चोप॑ दधाति । \newline
3. उप॑ दधाति दधा॒ त्युपोप॑ दधाति॒ मन॑सा॒ मन॑सा दधा॒ त्युपोप॑ दधाति॒ मन॑सा । \newline
4. द॒धा॒ति॒ मन॑सा॒ मन॑सा दधाति दधाति॒ मन॑सा मद्ध्य॒माम् म॑द्ध्य॒माम् मन॑सा दधाति दधाति॒ मन॑सा मद्ध्य॒माम् । \newline
5. मन॑सा मद्ध्य॒माम् म॑द्ध्य॒माम् मन॑सा॒ मन॑सा मद्ध्य॒मा मे॒षा मे॒षाम् म॑द्ध्य॒माम् मन॑सा॒ मन॑सा मद्ध्य॒मा मे॒षाम् । \newline
6. म॒द्ध्य॒मा मे॒षा मे॒षाम् म॑द्ध्य॒माम् म॑द्ध्य॒मा मे॒षाम् ॅलो॒काना᳚म् ॅलो॒काना॑ मे॒षाम् म॑द्ध्य॒माम् म॑द्ध्य॒मा मे॒षाम् ॅलो॒काना᳚म् । \newline
7. ए॒षाम् ॅलो॒काना᳚म् ॅलो॒काना॑ मे॒षा मे॒षाम् ॅलो॒काना॒म् क्लृप्त्यै॒ क्लृप्त्यै॑ लो॒काना॑ मे॒षा मे॒षाम् ॅलो॒काना॒म् क्लृप्त्यै᳚ । \newline
8. लो॒काना॒म् क्लृप्त्यै॒ क्लृप्त्यै॑ लो॒काना᳚म् ॅलो॒काना॒म् क्लृप्त्या॒ अथो॒ अथो॒ क्लृप्त्यै॑ लो॒काना᳚म् ॅलो॒काना॒म् क्लृप्त्या॒ अथो᳚ । \newline
9. क्लृप्त्या॒ अथो॒ अथो॒ क्लृप्त्यै॒ क्लृप्त्या॒ अथो᳚ प्रा॒णाना᳚म् प्रा॒णाना॒ मथो॒ क्लृप्त्यै॒ क्लृप्त्या॒ अथो᳚ प्रा॒णाना᳚म् । \newline
10. अथो᳚ प्रा॒णाना᳚म् प्रा॒णाना॒ मथो॒ अथो᳚ प्रा॒णाना॑ मि॒ष्ट इ॒ष्टः प्रा॒णाना॒ मथो॒ अथो᳚ प्रा॒णाना॑ मि॒ष्टः । \newline
11. अथो॒ इत्यथो᳚ । \newline
12. प्रा॒णाना॑ मि॒ष्ट इ॒ष्टः प्रा॒णाना᳚म् प्रा॒णाना॑ मि॒ष्टो य॒ज्ञो य॒ज्ञ् इ॒ष्टः प्रा॒णाना᳚म् प्रा॒णाना॑ मि॒ष्टो य॒ज्ञ्ः । \newline
13. प्रा॒णाना॒मिति॑ प्र - अ॒नाना᳚म् । \newline
14. इ॒ष्टो य॒ज्ञो य॒ज्ञ् इ॒ष्ट इ॒ष्टो य॒ज्ञो भृगु॑भि॒र् भृगु॑भिर् य॒ज्ञ् इ॒ष्ट इ॒ष्टो य॒ज्ञो भृगु॑भिः । \newline
15. य॒ज्ञो भृगु॑भि॒र् भृगु॑भिर् य॒ज्ञो य॒ज्ञो भृगु॑भि राशी॒र्दा आ॑शी॒र्दा भृगु॑भिर् य॒ज्ञो य॒ज्ञो भृगु॑भि राशी॒र्दाः । \newline
16. भृगु॑भि राशी॒र्दा आ॑शी॒र्दा भृगु॑भि॒र् भृगु॑भि राशी॒र्दा वसु॑भि॒र् वसु॑भि राशी॒र्दा भृगु॑भि॒र् भृगु॑भि राशी॒र्दा वसु॑भिः । \newline
17. भृगु॑भि॒रिति॒ भृगु॑ - भिः॒ । \newline
18. आ॒शी॒र्दा वसु॑भि॒र् वसु॑भि राशी॒र्दा आ॑शी॒र्दा वसु॑भि॒ स्तस्य॒ तस्य॒ वसु॑भि राशी॒र्दा आ॑शी॒र्दा वसु॑भि॒ स्तस्य॑ । \newline
19. आ॒शी॒र्दा इत्या॑शीः - दाः । \newline
20. वसु॑भि॒ स्तस्य॒ तस्य॒ वसु॑भि॒र् वसु॑भि॒ स्तस्य॑ ते ते॒ तस्य॒ वसु॑भि॒र् वसु॑भि॒ स्तस्य॑ ते । \newline
21. वसु॑भि॒रिति॒ वसु॑ - भिः॒ । \newline
22. तस्य॑ ते ते॒ तस्य॒ तस्य॑ त इ॒ष्ट स्ये॒ष्टस्य॑ ते॒ तस्य॒ तस्य॑ त इ॒ष्टस्य॑ । \newline
23. त॒ इ॒ष्ट स्ये॒ष्टस्य॑ ते त इ॒ष्टस्य॑ वी॒तस्य॑ वी॒त स्ये॒ष्टस्य॑ ते त इ॒ष्टस्य॑ वी॒तस्य॑ । \newline
24. इ॒ष्टस्य॑ वी॒तस्य॑ वी॒त स्ये॒ष्ट स्ये॒ष्टस्य॑ वी॒तस्य॒ द्रवि॑णा॒ द्रवि॑णा वी॒त स्ये॒ष्ट स्ये॒ष्टस्य॑ वी॒तस्य॒ द्रवि॑णा । \newline
25. वी॒तस्य॒ द्रवि॑णा॒ द्रवि॑णा वी॒तस्य॑ वी॒तस्य॒ द्रवि॑णे॒हेह द्रवि॑णा वी॒तस्य॑ वी॒तस्य॒ द्रवि॑णे॒ह । \newline
26. द्रवि॑णे॒हेह द्रवि॑णा॒ द्रवि॑णे॒ह भ॑क्षीय भक्षीये॒ह द्रवि॑णा॒ द्रवि॑णे॒ह भ॑क्षीय । \newline
27. इ॒ह भ॑क्षीय भक्षी ये॒हेह भ॑क्षी॒ये तीति॑ भक्षी ये॒हेह भ॑क्षी॒येति॑ । \newline
28. भ॒क्षी॒ये तीति॑ भक्षीय भक्षी॒ये त्या॑हा॒ हेति॑ भक्षीय भक्षी॒ये त्या॑ह । \newline
29. इत्या॑हा॒हे तीत्या॑ह स्तुतश॒स्त्रे स्तु॑तश॒स्त्रे आ॒हे तीत्या॑ह स्तुतश॒स्त्रे । \newline
30. आ॒ह॒ स्तु॒त॒श॒स्त्रे स्तु॑तश॒स्त्रे आ॑हाह स्तुतश॒स्त्रे ए॒वैव स्तु॑तश॒स्त्रे आ॑हाह स्तुतश॒स्त्रे ए॒व । \newline
31. स्तु॒त॒श॒स्त्रे ए॒वैव स्तु॑तश॒स्त्रे स्तु॑तश॒स्त्रे ए॒वैते नै॒ते नै॒व स्तु॑तश॒स्त्रे स्तु॑तश॒स्त्रे ए॒वैतेन॑ । \newline
32. स्तु॒त॒श॒स्त्रे इति॑ स्तुत - श॒स्त्रे । \newline
33. ए॒वैते नै॒ते नै॒वैवै तेन॑ दुहे दुह ए॒ते नै॒वै वैतेन॑ दुहे । \newline
34. ए॒तेन॑ दुहे दुह ए॒तेनै॒तेन॑ दुहे पि॒ता पि॒ता दु॑ह ए॒तेनै॒तेन॑ दुहे पि॒ता । \newline
35. दु॒हे॒ पि॒ता पि॒ता दु॑हे दुहे पि॒ता मा॑त॒रिश्वा॑ मात॒रिश्वा॑ पि॒ता दु॑हे दुहे पि॒ता मा॑त॒रिश्वा᳚ । \newline
36. पि॒ता मा॑त॒रिश्वा॑ मात॒रिश्वा॑ पि॒ता पि॒ता मा॑त॒रिश्वा ऽच्छि॒द्रा ऽच्छि॑द्रा मात॒रिश्वा॑ पि॒ता पि॒ता मा॑त॒रिश्वा ऽच्छि॑द्रा । \newline
37. मा॒त॒रिश्वा ऽच्छि॒द्रा ऽच्छि॑द्रा मात॒रिश्वा॑ मात॒रिश्वा ऽच्छि॑द्रा प॒दा प॒दा ऽच्छि॑द्रा मात॒रिश्वा॑ मात॒रिश्वा ऽच्छि॑द्रा प॒दा । \newline
38. अच्छि॑द्रा प॒दा प॒दा ऽच्छि॒द्रा ऽच्छि॑द्रा प॒दा धा॑ धाः प॒दा ऽच्छि॒द्रा ऽच्छि॑द्रा प॒दा धाः᳚ । \newline
39. प॒दा धा॑ धाः प॒दा प॒दा धा॒ अच्छि॑द्रा॒ अच्छि॑द्रा धाः प॒दा प॒दा धा॒ अच्छि॑द्राः । \newline
40. धा॒ अच्छि॑द्रा॒ अच्छि॑द्रा धा धा॒ अच्छि॑द्रा उ॒शिज॑ उ॒शिजो ऽच्छि॑द्रा धा धा॒ अच्छि॑द्रा उ॒शिजः॑ । \newline
41. अच्छि॑द्रा उ॒शिज॑ उ॒शिजो ऽच्छि॑द्रा॒ अच्छि॑द्रा उ॒शिजः॑ प॒दा प॒दोशिजो ऽच्छि॑द्रा॒ अच्छि॑द्रा उ॒शिजः॑ प॒दा । \newline
42. उ॒शिजः॑ प॒दा प॒दोशिज॑ उ॒शिजः॑ प॒दा ऽन्वनु॑ प॒दोशिज॑ उ॒शिजः॑ प॒दा ऽनु॑ । \newline
43. प॒दा ऽन्वनु॑ प॒दा प॒दा ऽनु॑ तक्षु स्तक्षु॒ रनु॑ प॒दा प॒दा ऽनु॑ तक्षुः । \newline
44. अनु॑ तक्षु स्तक्षु॒ रन्वनु॑ तक्षुः॒ सोमः॒ सोम॑ स्तक्षु॒ रन्वनु॑ तक्षुः॒ सोमः॑ । \newline
45. त॒क्षुः॒ सोमः॒ सोम॑ स्तक्षु स्तक्षुः॒ सोमो॑ विश्व॒विद् वि॑श्व॒विथ् सोम॑ स्तक्षु स्तक्षुः॒ सोमो॑ विश्व॒वित् । \newline
46. सोमो॑ विश्व॒विद् वि॑श्व॒विथ् सोमः॒ सोमो॑ विश्व॒विन् ने॒ता ने॒ता वि॑श्व॒विथ् सोमः॒ सोमो॑ विश्व॒विन् ने॒ता । \newline
47. वि॒श्व॒विन् ने॒ता ने॒ता वि॑श्व॒विद् वि॑श्व॒विन् ने॒ता ने॑षन् नेषन् ने॒ता वि॑श्व॒विद् वि॑श्व॒विन् ने॒ता ने॑षत् । \newline
48. वि॒श्व॒विदिति॑ विश्व - वित् । \newline
49. ने॒ता ने॑षन् नेषन् ने॒ता ने॒ता ने॑ष॒द् बृह॒स्पति॒र् बृह॒स्पति॑र् नेषन् ने॒ता ने॒ता ने॑ष॒द् बृह॒स्पतिः॑ । \newline
50. ने॒ष॒द् बृह॒स्पति॒र् बृह॒स्पति॑र् नेषन् नेष॒द् बृह॒स्पति॑ रुक्थाम॒दा न्यु॑क्थाम॒दानि॒ बृह॒स्पति॑र् नेषन् नेष॒द् बृह॒स्पति॑ रुक्थाम॒दानि॑ । \newline
51. बृह॒स्पति॑ रुक्थाम॒दा न्यु॑क्थाम॒दानि॒ बृह॒स्पति॒र् बृह॒स्पति॑ रुक्थाम॒दानि॑ शꣳसिष च्छꣳसिष दुक्थाम॒दानि॒ बृह॒स्पति॒र् बृह॒स्पति॑ रुक्थाम॒दानि॑ शꣳसिषत् । \newline
52. उ॒क्था॒म॒दानि॑ शꣳसिष च्छꣳसिष दुक्थाम॒दा न्यु॑क्थाम॒दानि॑ शꣳसिष॒दितीति॑ शꣳसिष दुक्थाम॒दा न्यु॑क्थाम॒दानि॑ शꣳसिष॒दिति॑ । \newline
53. उ॒क्था॒म॒दानीत्यु॑क्थ - म॒दानि॑ । \newline
54. शꣳ॒॒सि॒ष॒ दितीति॑ शꣳसिष च्छꣳसिष॒ दित्या॑ हा॒हेति॑ शꣳसिष च्छꣳसिष॒ दित्या॑ह । \newline
55. इत्या॑हा॒हे तीत्या॑है॒त दे॒त दा॒हे तीत्या॑ है॒तत् । \newline
56. आ॒है॒त दे॒तदा॑हा है॒तद् वै वा ए॒तदा॑हा है॒तद् वै । \newline
57. ए॒तद् वै वा ए॒त दे॒तद् वा अ॒ग्ने र॒ग्नेर् वा ए॒त दे॒तद् वा अ॒ग्नेः । \newline
58. वा अ॒ग्ने र॒ग्नेर् वै वा अ॒ग्ने रु॒क्थ मु॒क्थ म॒ग्नेर् वै वा अ॒ग्ने रु॒क्थम् । \newline
59. अ॒ग्ने रु॒क्थ मु॒क्थ म॒ग्ने र॒ग्ने रु॒क्थम् तेन॒ तेनो॒क्थ म॒ग्ने र॒ग्ने रु॒क्थम् तेन॑ । \newline
60. उ॒क्थम् तेन॒ तेनो॒क्थ मु॒क्थम् तेनै॒ वैव तेनो॒क्थ मु॒क्थम् तेनै॒व । \newline
61. तेनै॒ वैव तेन॒ तेनै॒ वैन॑ मेन मे॒व तेन॒ तेनै॒ वैन᳚म् । \newline
62. ए॒वैन॑ मेन मे॒वै वैन॒ मन् वन् वे॑न मे॒वै वैन॒ मनु॑ । \newline
63. ए॒न॒ मन् वन् वे॑न मेन॒ मनु॑ शꣳसति शꣳस॒ त्यन् वे॑न मेन॒ मनु॑ शꣳसति । \newline
64. अनु॑ शꣳसति शꣳस॒ त्यन् वनु॑ शꣳसति । \newline
65. शꣳ॒॒स॒तीति॑ सꣳसति । \newline
\pagebreak
\markright{ TS 5.6.9.1  \hfill https://www.vedavms.in \hfill}

\section{ TS 5.6.9.1 }

\textbf{TS 5.6.9.1 } \newline
\textbf{Samhita Paata} \newline

सू॒यते॒ वा ए॒षो᳚ऽग्नी॒नां ॅय उ॒खायां᳚ भ्रि॒यते॒ यद॒धः सा॒दये॒द्-गर्भाः᳚ प्र॒पादु॑काः स्यु॒रथो॒ यथा॑ स॒वात् प्र॑त्यव॒रोह॑ति ता॒दृगे॒व तदा॑स॒न्दी सा॑दयति॒ गर्भा॑णां॒ धृत्या॒ अप्र॑पादा॒याथो॑ स॒वमे॒वैनं॑ करोति॒ गर्भो॒ वा ए॒ष यदुख्यो॒ योनिः॑ शि॒क्यं॑ ॅयच्छि॒क्या॑दु॒खां नि॒रूहे॒द्-योने॒र्गर्भं॒ निर्.ह॑ण्या॒थ् षडु॑द्यामꣳ शि॒क्यं॑ भवति षोढा विहि॒तो वै - [  ] \newline

\textbf{Pada Paata} \newline

सू॒यते᳚ । वै । ए॒षः । अ॒ग्नी॒नाम् । यः । उ॒खाया᳚म् । भ्रि॒यते᳚ । यत् । अ॒धः । सा॒दये᳚त् । गर्भाः᳚ । प्र॒पादु॑का॒ इति॑ प्र - पादु॑काः । स्युः॒ । अथो॒ इति॑ । यथा᳚ । स॒वात् । प्र॒त्य॒व॒रोह॒तीति॑ प्रति - अ॒व॒रोह॑ति । ता॒दृक् । ए॒व । तत् । आ॒स॒न्दी । सा॒द॒य॒ति॒ । गर्भा॑णाम् । धृत्यै᳚ । अप्र॑पादा॒येत्यप्र॑ - पा॒दा॒य॒ । अथो॒ इति॑ । स॒वम् । ए॒व । ए॒न॒म् । क॒रो॒ति॒ । गर्भः॑ । वै । ए॒षः । यत् । उख्यः॑ । योनिः॑ । शि॒क्य᳚म् । यत् । शि॒क्या᳚त् । उ॒खाम् । नि॒रूहे॒दिति॑ निः - ऊहे᳚त् । योनेः᳚ । गर्भ᳚म् । निरिति॑ । ह॒न्या॒त् । षडु॑द्याम॒मिति॒ षट्-उ॒द्या॒म॒म् । शि॒क्य᳚म् । भ॒व॒ति॒ । षो॒ढा॒वि॒हि॒त इति॑ षोढा - वि॒हि॒तः । वै ।  \newline


\textbf{Krama Paata} \newline

सू॒यते॒ वै । वा ए॒षः । ए॒षो᳚ऽग्नी॒नाम् । अ॒ग्नी॒नाम् ॅयः । य उ॒खाया᳚म् । उ॒खाया᳚म् भ्रि॒यते᳚ । भ्रि॒यते॒ यत् । यद॒धः । अ॒धः सा॒दये᳚त् । सा॒दये॒द् गर्भाः᳚ । गर्भाः᳚ प्र॒पादु॑काः । प्र॒पादु॑काः स्युः । प्र॒पादु॑का॒ इति॑ प्र - पादु॑काः । स्यु॒रथो᳚ । अथो॒ यथा᳚ । अथो॒ इत्यथो᳚ । यथा॑ स॒वात् । स॒वात् प्र॑त्यव॒रोह॑ति । प्र॒त्य॒व॒रोह॑ति ता॒दृक् । प्र॒त्य॒व॒रोह॒तीति॑ प्रति - अ॒व॒रोह॑ति । ता॒दृगे॒व । ए॒व तत् । तदा॑स॒न्दी । आ॒स॒न्दी सा॑दयति । सा॒द॒य॒ति॒ गर्भा॑णाम् । गर्भा॑णा॒म् धृत्यै᳚ । धृत्या॒ अप्र॑पादाय । अप्र॑पादा॒याथो᳚ । अप्र॑पादा॒येत्यप्र॑ - पा॒दा॒य॒ । अथो॑ स॒वम् । अथो॒ इत्यथो᳚ । स॒वमे॒व । ए॒वैन᳚म् । ए॒न॒म् क॒रो॒ति॒ । क॒रो॒ति॒ गर्भः॑ । गर्भो॒ वै । वा ए॒षः । ए॒ष यत् । यदुख्यः॑ । उख्यो॒ योनिः॑ । योनिः॑ शि॒क्य᳚म् । शि॒क्य॑म् ॅयत् । यच्छि॒क्या᳚त् । शि॒क्या॑दु॒खाम् । उ॒खाम् नि॒रूहे᳚त् । नि॒रूहे॒द् योनेः᳚ । नि॒रूहे॒दिति॑ निः - ऊहे᳚त् । योने॒र् गर्भ᳚म् । गर्भ॒म् निः । निर्. ह॑ण्यात् । ह॒ण्या॒थ् षडु॑द्यामम् । षडु॑द्यामꣳ शि॒क्य᳚म् । षडु॑द्याम॒मिति॒ षट् - उ॒द्या॒म॒म् । शि॒क्य॑म् भवति । भ॒व॒ति॒ षो॒ढा॒वि॒हि॒तः । षो॒ढा॒वि॒हि॒तो वै । षो॒ढा॒वि॒हि॒त इति॑ षोढा - वि॒हि॒तः । वै पुरु॑षः \newline

\textbf{Jatai Paata} \newline

1. सू॒यते॒ वै वै सू॒यते॑ सू॒यते॒ वै । \newline
2. वा ए॒ष ए॒ष वै वा ए॒षः । \newline
3. ए॒षो᳚ ऽग्नी॒ना म॑ग्नी॒ना मे॒ष ए॒षो᳚ ऽग्नी॒नाम् । \newline
4. अ॒ग्नी॒नां ॅयो यो᳚ ऽग्नी॒ना म॑ग्नी॒नां ॅयः । \newline
5. य उ॒खाया॑ मु॒खायां॒ ॅयो य उ॒खाया᳚म् । \newline
6. उ॒खाया᳚म् भ्रि॒यते᳚ भ्रि॒यत॑ उ॒खाया॑ मु॒खाया᳚म् भ्रि॒यते᳚ । \newline
7. भ्रि॒यते॒ यद् यद् भ्रि॒यते᳚ भ्रि॒यते॒ यत् । \newline
8. यद॒धो॑ ऽधो यद् यद॒धः । \newline
9. अ॒धः सा॒दये᳚थ् सा॒दये॑ द॒धो॑ ऽधः सा॒दये᳚त् । \newline
10. सा॒दये॒द् गर्भा॒ गर्भाः᳚ सा॒दये᳚थ् सा॒दये॒द् गर्भाः᳚ । \newline
11. गर्भाः᳚ प्र॒पादु॑काः प्र॒पादु॑का॒ गर्भा॒ गर्भाः᳚ प्र॒पादु॑काः । \newline
12. प्र॒पादु॑काः स्युः स्युः प्र॒पादु॑काः प्र॒पादु॑काः स्युः । \newline
13. प्र॒पादु॑का॒ इति॑ प्र - पादु॑काः । \newline
14. स्यु॒ रथो॒ अथो᳚ स्युः स्यु॒ रथो᳚ । \newline
15. अथो॒ यथा॒ यथा ऽथो॒ अथो॒ यथा᳚ । \newline
16. अथो॒ इत्यथो᳚ । \newline
17. यथा॑ स॒वाथ् स॒वाद् यथा॒ यथा॑ स॒वात् । \newline
18. स॒वात् प्र॑त्यव॒रोह॑ति प्रत्यव॒रोह॑ति स॒वाथ् स॒वात् प्र॑त्यव॒रोह॑ति । \newline
19. प्र॒त्य॒व॒रोह॑ति ता॒दृक् ता॒दृक् प्र॑त्यव॒रोह॑ति प्रत्यव॒रोह॑ति ता॒दृक् । \newline
20. प्र॒त्य॒व॒रोह॒तीति॑ प्रति - अ॒व॒रोह॑ति । \newline
21. ता॒दृ गे॒वैव ता॒दृक् ता॒दृ गे॒व । \newline
22. ए॒व तत् तदे॒ वैव तत् । \newline
23. तदा॑स॒न् द्या॑स॒न्दी तत् तदा॑स॒न्दी । \newline
24. आ॒स॒न्दी सा॑दयति सादय त्यास॒न् द्या॑स॒न्दी सा॑दयति । \newline
25. सा॒द॒य॒ति॒ गर्भा॑णा॒म् गर्भा॑णाꣳ सादयति सादयति॒ गर्भा॑णाम् । \newline
26. गर्भा॑णा॒म् धृत्यै॒ धृत्यै॒ गर्भा॑णा॒म् गर्भा॑णा॒म् धृत्यै᳚ । \newline
27. धृत्या॒ अप्र॑पादा॒या प्र॑पादाय॒ धृत्यै॒ धृत्या॒ अप्र॑पादाय । \newline
28. अप्र॑पादा॒ याथो॒ अथो॒ अप्र॑पादा॒या प्र॑पादा॒ याथो᳚ । \newline
29. अप्र॑पादा॒येत्यप्र॑ - पा॒दा॒य॒ । \newline
30. अथो॑ स॒वꣳ स॒व मथो॒ अथो॑ स॒वम् । \newline
31. अथो॒ इत्यथो᳚ । \newline
32. स॒व मे॒वैव स॒वꣳ स॒व मे॒व । \newline
33. ए॒वैन॑ मेन मे॒वै वैन᳚म् । \newline
34. ए॒न॒म् क॒रो॒ति॒ क॒रो॒ त्ये॒न॒ मे॒न॒म् क॒रो॒ति॒ । \newline
35. क॒रो॒ति॒ गर्भो॒ गर्भः॑ करोति करोति॒ गर्भः॑ । \newline
36. गर्भो॒ वै वै गर्भो॒ गर्भो॒ वै । \newline
37. वा ए॒ष ए॒ष वै वा ए॒षः । \newline
38. ए॒ष यद् यदे॒ष ए॒ष यत् । \newline
39. यदुख्य॒ उख्यो॒ यद् यदुख्यः॑ । \newline
40. उख्यो॒ योनि॒र् योनि॒ रुख्य॒ उख्यो॒ योनिः॑ । \newline
41. योनिः॑ शि॒क्यꣳ॑ शि॒क्यं॑ ॅयोनि॒र् योनिः॑ शि॒क्य᳚म् । \newline
42. शि॒क्यं॑ ॅयद् य च्छि॒क्यꣳ॑ शि॒क्यं॑ ॅयत् । \newline
43. यच्छि॒क्या᳚ च्छि॒क्या᳚द् यद् यच्छि॒क्या᳚त् । \newline
44. शि॒क्या॑दु॒खा मु॒खाꣳ शि॒क्या᳚ च्छि॒क्या॑ दु॒खाम् । \newline
45. उ॒खान् नि॒रूहे᳚न् नि॒रूहे॑ दु॒खा मु॒खान् नि॒रूहे᳚त् । \newline
46. नि॒रूहे॒द् योने॒र् योने᳚र् नि॒रूहे᳚न् नि॒रूहे॒द् योनेः᳚ । \newline
47. नि॒रूहे॒दिति॑ निः - ऊहे᳚त् । \newline
48. योने॒र् गर्भ॒म् गर्भं॒ ॅयोने॒र् योने॒र् गर्भ᳚म् । \newline
49. गर्भ॒म् निर् णिर् गर्भ॒म् गर्भ॒म् निः । \newline
50. निर्. ह॑ण्या द्धन्या॒न् निर् णिर्. ह॑ण्यात् । \newline
51. ह॒न्या॒ थ्षडु॑द्यामꣳ॒॒ षडु॑द्यामꣳ हन्या द्धन्या॒ थ्षडु॑द्यामम् । \newline
52. षडु॑द्यामꣳ शि॒क्यꣳ॑ शि॒क्यꣳ॑ षडु॑द्यामꣳ॒॒ षडु॑द्यामꣳ शि॒क्य᳚म् । \newline
53. षडु॑द्याम॒मिति॒ षट् - उ॒द्या॒म॒म् । \newline
54. शि॒क्य॑म् भवति भवति शि॒क्यꣳ॑ शि॒क्य॑म् भवति । \newline
55. भ॒व॒ति॒ षो॒ढा॒वि॒हि॒त ष्षो॑ढाविहि॒तो भ॑वति भवति षोढाविहि॒तः । \newline
56. षो॒ढा॒वि॒हि॒तो वै वै षो॑ढाविहि॒त ष्षो॑ढाविहि॒तो वै । \newline
57. षो॒ढा॒वि॒हि॒त इति॑ षोढा - वि॒हि॒तः । \newline
58. वै पुरु॑षः॒ पुरु॑षो॒ वै वै पुरु॑षः । \newline

\textbf{Ghana Paata } \newline

1. सू॒यते॒ वै वै सू॒यते॑ सू॒यते॒ वा ए॒ष ए॒ष वै सू॒यते॑ सू॒यते॒ वा ए॒षः । \newline
2. वा ए॒ष ए॒ष वै वा ए॒षो᳚ ऽग्नी॒ना म॑ग्नी॒ना मे॒ष वै वा ए॒षो᳚ ऽग्नी॒नाम् । \newline
3. ए॒षो᳚ ऽग्नी॒ना म॑ग्नी॒ना मे॒ष ए॒षो᳚ ऽग्नी॒नां ॅयो यो᳚ ऽग्नी॒ना मे॒ष ए॒षो᳚ ऽग्नी॒नां ॅयः । \newline
4. अ॒ग्नी॒नां ॅयो यो᳚ ऽग्नी॒ना म॑ग्नी॒नां ॅय उ॒खाया॑ मु॒खायां॒ ॅयो᳚ ऽग्नी॒ना म॑ग्नी॒नां ॅय उ॒खाया᳚म् । \newline
5. य उ॒खाया॑ मु॒खायां॒ ॅयो य उ॒खाया᳚म् भ्रि॒यते᳚ भ्रि॒यत॑ उ॒खायां॒ ॅयो य उ॒खाया᳚म् भ्रि॒यते᳚ । \newline
6. उ॒खाया᳚म् भ्रि॒यते᳚ भ्रि॒यत॑ उ॒खाया॑ मु॒खाया᳚म् भ्रि॒यते॒ यद् यद् भ्रि॒यत॑ उ॒खाया॑ मु॒खाया᳚म् भ्रि॒यते॒ यत् । \newline
7. भ्रि॒यते॒ यद् यद् भ्रि॒यते᳚ भ्रि॒यते॒ यद॒धो॑ ऽधो यद् भ्रि॒यते᳚ भ्रि॒यते॒ यद॒धः । \newline
8. यद॒धो॑ ऽधो यद् यद॒धः सा॒दये᳚थ् सा॒दये॑ द॒धो यद् यद॒धः सा॒दये᳚त् । \newline
9. अ॒धः सा॒दये᳚थ् सा॒दये॑ द॒धो॑ ऽधः सा॒दये॒द् गर्भा॒ गर्भाः᳚ सा॒दये॑ द॒धो॑ ऽधः सा॒दये॒द् गर्भाः᳚ । \newline
10. सा॒दये॒द् गर्भा॒ गर्भाः᳚ सा॒दये᳚थ् सा॒दये॒द् गर्भाः᳚ प्र॒पादु॑काः प्र॒पादु॑का॒ गर्भाः᳚ सा॒दये᳚थ् सा॒दये॒द् गर्भाः᳚ प्र॒पादु॑काः । \newline
11. गर्भाः᳚ प्र॒पादु॑काः प्र॒पादु॑का॒ गर्भा॒ गर्भाः᳚ प्र॒पादु॑काः स्युः स्युः प्र॒पादु॑का॒ गर्भा॒ गर्भाः᳚ प्र॒पादु॑काः स्युः । \newline
12. प्र॒पादु॑काः स्युः स्युः प्र॒पादु॑काः प्र॒पादु॑काः स्यु॒ रथो॒ अथो᳚ स्युः प्र॒पादु॑काः प्र॒पादु॑काः स्यु॒ रथो᳚ । \newline
13. प्र॒पादु॑का॒ इति॑ प्र - पादु॑काः । \newline
14. स्यु॒ रथो॒ अथो᳚ स्युः स्यु॒ रथो॒ यथा॒ यथा ऽथो᳚ स्युः स्यु॒ रथो॒ यथा᳚ । \newline
15. अथो॒ यथा॒ यथा ऽथो॒ अथो॒ यथा॑ स॒वाथ् स॒वाद् यथा ऽथो॒ अथो॒ यथा॑ स॒वात् । \newline
16. अथो॒ इत्यथो᳚ । \newline
17. यथा॑ स॒वाथ् स॒वाद् यथा॒ यथा॑ स॒वात् प्र॑त्यव॒रोह॑ति प्रत्यव॒रोह॑ति स॒वाद् यथा॒ यथा॑ स॒वात् प्र॑त्यव॒रोह॑ति । \newline
18. स॒वात् प्र॑त्यव॒रोह॑ति प्रत्यव॒रोह॑ति स॒वाथ् स॒वात् प्र॑त्यव॒रोह॑ति ता॒दृक् ता॒दृक् प्र॑त्यव॒रोह॑ति स॒वाथ् स॒वात् प्र॑त्यव॒रोह॑ति ता॒दृक् । \newline
19. प्र॒त्य॒व॒रोह॑ति ता॒दृक् ता॒दृक् प्र॑त्यव॒रोह॑ति प्रत्यव॒रोह॑ति ता॒दृ गे॒वैव ता॒दृक् प्र॑त्यव॒रोह॑ति प्रत्यव॒रोह॑ति ता॒दृ गे॒व । \newline
20. प्र॒त्य॒व॒रोह॒तीति॑ प्रति - अ॒व॒रोह॑ति । \newline
21. ता॒दृ गे॒वैव ता॒दृक् ता॒दृ गे॒व तत् तदे॒व ता॒दृक् ता॒दृ गे॒व तत् । \newline
22. ए॒व तत् तदे॒ वैव तदा॑स॒न्द्या॑ स॒न्दी तदे॒ वैव तदा॑स॒न्दी । \newline
23. तदा॑स॒न्द्या॑ स॒न्दी तत् तदा॑स॒न्दी सा॑दयति सादय त्यास॒न्दी तत् तदा॑स॒न्दी सा॑दयति । \newline
24. आ॒स॒न्दी सा॑दयति सादय त्यास॒न्द्या॑ स॒न्दी सा॑दयति॒ गर्भा॑णा॒म् गर्भा॑णाꣳ सादय त्यास॒न्द्या॑ स॒न्दी सा॑दयति॒ गर्भा॑णाम् । \newline
25. सा॒द॒य॒ति॒ गर्भा॑णा॒म् गर्भा॑णाꣳ सादयति सादयति॒ गर्भा॑णा॒म् धृत्यै॒ धृत्यै॒ गर्भा॑णाꣳ सादयति सादयति॒ गर्भा॑णा॒म् धृत्यै᳚ । \newline
26. गर्भा॑णा॒म् धृत्यै॒ धृत्यै॒ गर्भा॑णा॒म् गर्भा॑णा॒म् धृत्या॒ अप्र॑पादा॒या प्र॑पादाय॒ धृत्यै॒ गर्भा॑णा॒म् गर्भा॑णा॒म् धृत्या॒ अप्र॑पादाय । \newline
27. धृत्या॒ अप्र॑पादा॒या प्र॑पादाय॒ धृत्यै॒ धृत्या॒ अप्र॑पादा॒ याथो॒ अथो॒ अप्र॑पादाय॒ धृत्यै॒ धृत्या॒ अप्र॑पादा॒ याथो᳚ । \newline
28. अप्र॑पादा॒ याथो॒ अथो॒ अप्र॑पादा॒या प्र॑पादा॒ याथो॑ स॒वꣳ स॒व मथो॒ अप्र॑पादा॒या प्र॑पादा॒याथो॑ स॒वम् । \newline
29. अप्र॑पादा॒येत्यप्र॑ - पा॒दा॒य॒ । \newline
30. अथो॑ स॒वꣳ स॒व मथो॒ अथो॑ स॒व मे॒वैव स॒व मथो॒ अथो॑ स॒व मे॒व । \newline
31. अथो॒ इत्यथो᳚ । \newline
32. स॒व मे॒वैव स॒वꣳ स॒व मे॒वैन॑ मेन मे॒व स॒वꣳ स॒व मे॒वैन᳚म् । \newline
33. ए॒वैन॑ मेन मे॒वै वैन॑म् करोति करो त्येन मे॒वै वैन॑म् करोति । \newline
34. ए॒न॒म् क॒रो॒ति॒ क॒रो॒ त्ये॒न॒ मे॒न॒म् क॒रो॒ति॒ गर्भो॒ गर्भः॑ करो त्येन मेनम् करोति॒ गर्भः॑ । \newline
35. क॒रो॒ति॒ गर्भो॒ गर्भः॑ करोति करोति॒ गर्भो॒ वै वै गर्भः॑ करोति करोति॒ गर्भो॒ वै । \newline
36. गर्भो॒ वै वै गर्भो॒ गर्भो॒ वा ए॒ष ए॒ष वै गर्भो॒ गर्भो॒ वा ए॒षः । \newline
37. वा ए॒ष ए॒ष वै वा ए॒ष यद् यदे॒ष वै वा ए॒ष यत् । \newline
38. ए॒ष यद् यदे॒ष ए॒ष यदुख्य॒ उख्यो॒ यदे॒ष ए॒ष यदुख्यः॑ । \newline
39. यदुख्य॒ उख्यो॒ यद् यदुख्यो॒ योनि॒र् योनि॒ रुख्यो॒ यद् यदुख्यो॒ योनिः॑ । \newline
40. उख्यो॒ योनि॒र् योनि॒ रुख्य॒ उख्यो॒ योनिः॑ शि॒क्यꣳ॑ शि॒क्यं॑ ॅयोनि॒ रुख्य॒ उख्यो॒ योनिः॑ शि॒क्य᳚म् । \newline
41. योनिः॑ शि॒क्यꣳ॑ शि॒क्यं॑ ॅयोनि॒र् योनिः॑ शि॒क्यं॑ ॅयद् यच्छि॒क्यं॑ ॅयोनि॒र् योनिः॑ शि॒क्यं॑ ॅयत् । \newline
42. शि॒क्यं॑ ॅयद् यच्छि॒क्यꣳ॑ शि॒क्यं॑ ॅयच्छि॒क्या᳚ च्छि॒क्या᳚द् यच्छि॒क्यꣳ॑ शि॒क्यं॑ ॅयच्छि॒क्या᳚त् । \newline
43. यच्छि॒क्या᳚ च्छि॒क्या᳚द् यद् यच्छि॒क्या॑ दु॒खा मु॒खाꣳ शि॒क्या᳚द् यद् यच्छि॒क्या॑ दु॒खाम् । \newline
44. शि॒क्या॑ दु॒खा मु॒खाꣳ शि॒क्या᳚ च्छि॒क्या॑ दु॒खान् नि॒रूहे᳚न् नि॒रूहे॑ दु॒खाꣳ शि॒क्या᳚ च्छि॒क्या॑ दु॒खाम् नि॒रूहे᳚त् । \newline
45. उ॒खान् नि॒रूहे᳚न् नि॒रूहे॑ दु॒खा मु॒खान् नि॒रूहे॒द् योने॒र् योने᳚र् नि॒रूहे॑ दु॒खा मु॒खान् नि॒रूहे॒द् योनेः᳚ । \newline
46. नि॒रूहे॒द् योने॒र् योने᳚र् नि॒रूहे᳚न् नि॒रूहे॒द् योने॒र् गर्भ॒म् गर्भं॒ ॅयोने᳚र् नि॒रूहे᳚न् नि॒रूहे॒द् योने॒र् गर्भ᳚म् । \newline
47. नि॒रूहे॒दिति॑ निः - ऊहे᳚त् । \newline
48. योने॒र् गर्भ॒म् गर्भं॒ ॅयोने॒र् योने॒र् गर्भ॒न् निर् णिर् गर्भं॒ ॅयोने॒र् योने॒र् गर्भ॒न् निः । \newline
49. गर्भ॒न् निर् णिर् गर्भ॒म् गर्भ॒न् निर्. ह॑ण्या द्धन्या॒न् निर् गर्भ॒म् गर्भ॒न् निर्. ह॑ण्यात् । \newline
50. निर्. ह॑ण्या द्धन्या॒न् निर् णिर्. ह॑ण्या॒ थ्षडु॑द्यामꣳ॒॒ षडु॑द्यामꣳ हन्या॒न् निर् णिर्. ह॑ण्या॒ थ्षडु॑द्यामम् । \newline
51. ह॒न्या॒ थ्षडु॑द्यामꣳ॒॒ षडु॑द्यामꣳ हन्या द्धन्या॒ थ्षडु॑द्यामꣳ शि॒क्यꣳ॑ शि॒क्यꣳ॑ षडु॑द्यामꣳ हन्या द्धन्या॒ थ्षडु॑द्यामꣳ शि॒क्य᳚म् । \newline
52. षडु॑द्यामꣳ शि॒क्यꣳ॑ शि॒क्यꣳ॑ षडु॑द्यामꣳ॒॒ षडु॑द्यामꣳ शि॒क्य॑म् भवति भवति शि॒क्यꣳ॑ षडु॑द्यामꣳ॒॒ षडु॑द्यामꣳ शि॒क्य॑म् भवति । \newline
53. षडु॑द्याम॒मिति॒ षट् - उ॒द्या॒म॒म् । \newline
54. शि॒क्य॑म् भवति भवति शि॒क्यꣳ॑ शि॒क्य॑म् भवति षोढाविहि॒त ष्षो॑ढाविहि॒तो भ॑वति शि॒क्यꣳ॑ शि॒क्य॑म् भवति षोढाविहि॒तः । \newline
55. भ॒व॒ति॒ षो॒ढा॒वि॒हि॒त ष्षो॑ढाविहि॒तो भ॑वति भवति षोढाविहि॒तो वै वै षो॑ढाविहि॒तो भ॑वति भवति षोढाविहि॒तो वै । \newline
56. षो॒ढा॒वि॒हि॒तो वै वै षो॑ढाविहि॒त ष्षो॑ढाविहि॒तो वै पुरु॑षः॒ पुरु॑षो॒ वै षो॑ढाविहि॒त ष्षो॑ढाविहि॒तो वै पुरु॑षः । \newline
57. षो॒ढा॒वि॒हि॒त इति॑ षोढा - वि॒हि॒तः । \newline
58. वै पुरु॑षः॒ पुरु॑षो॒ वै वै पुरु॑ष आ॒त्मा ऽऽत्मा पुरु॑षो॒ वै वै पुरु॑ष आ॒त्मा । \newline
\pagebreak
\markright{ TS 5.6.9.2  \hfill https://www.vedavms.in \hfill}

\section{ TS 5.6.9.2 }

\textbf{TS 5.6.9.2 } \newline
\textbf{Samhita Paata} \newline

पुरु॑ष आ॒त्मा च॒ शिर॑श्च च॒त्वार्यङ्गा᳚न्या॒त्मन्ने॒वैनं॑ बिभर्ति प्र॒जाप॑ति॒र्वा ए॒ष यद॒ग्निस्तस्यो॒खा चो॒लूख॑लं च॒ स्तनौ॒ ताव॑स्य प्र॒जा उप॑ जीवन्ति॒ यदु॒खां चो॒लूख॑लं चोप॒दधा॑ति॒ ताभ्या॑मे॒व यज॑मानो॒ऽमुष्मि॑न् ॅलो॒के᳚ऽग्निं दु॑हे संॅवथ्स॒रो वा ए॒ष यद॒ग्निस्तस्य॑ त्रेधाविहि॒ता इष्ट॑काः प्राजाप॒त्या वै᳚ष्ण॒वी - [  ] \newline

\textbf{Pada Paata} \newline

पुरु॑षः । आ॒त्मा । च॒ । शिरः॑ । च॒ । च॒त्वारि॑ । अङ्गा॑नि । आ॒त्मन्न् । ए॒व । ए॒न॒म् । बि॒भ॒र्ति॒ । प्र॒जाप॑ति॒रिति॑ प्र॒जा - प॒तिः॒ । वै । ए॒षः । यत् । अ॒ग्निः । तस्य॑ । उ॒खा । च॒ । उ॒लूख॑लम् । च॒ । स्तनौ᳚ । तौ । अ॒स्य॒ । प्र॒जा इति॑ प्र - जाः । उपेति॑ । जी॒व॒न्ति॒ । यत् । उ॒खाम् । च॒ । उ॒लूख॑लम् । च॒ । उ॒प॒दधा॒तीत्यु॑प - दधा॑ति । ताभ्या᳚म् । ए॒व । यज॑मानः । अ॒मुष्मिन्न्॑ । लो॒के । अ॒ग्निम् । दु॒हे॒ । सं॒ॅव॒थ्स॒र इति॑ सं - व॒थ्स॒रः । वै । ए॒षः । यत् । अ॒ग्निः । तस्य॑ । त्रे॒धा॒वि॒हि॒ता इति॑ त्रेधा - वि॒हि॒ताः । इष्ट॑काः । प्रा॒जा॒प॒त्या इति॑ प्राजा - प॒त्याः॒ । वै॒ष्ण॒वीः ।  \newline


\textbf{Krama Paata} \newline

पुरु॑ष आ॒त्मा । आ॒त्मा च॑ । च॒ शिरः॑ । शिर॑श्च । च॒ च॒त्वारि॑ । च॒त्वार्यङ्गा॑नि । अङ्गा᳚न्या॒त्मन्न् । आ॒त्मन्ने॒व । ए॒वैन᳚म् । ए॒न॒म् बि॒भ॒र्ति॒ । बि॒भ॒र्ति॒ प्र॒जाप॑तिः । प्र॒जाप॑ति॒र् वै । प्र॒जाप॑ति॒रिति॑ प्र॒जा - प॒तिः॒ । वा ए॒षः । ए॒ष यत् । यद॒ग्निः । अ॒ग्निस्तस्य॑ । तस्यो॒खा । उ॒खा च॑ । चो॒लूख॑लम् । उ॒लूख॑लम् च । च॒ स्तनौ᳚ । स्तनौ॒ तौ । ताव॑स्य । अ॒स्य॒ प्र॒जाः । प्र॒जा उप॑ । प्र॒जा इति॑ प्र - जाः । उप॑ जीवन्ति । जी॒व॒न्ति॒ यत् । यदु॒खाम् । उ॒खाम् च॑ । चो॒लूख॑लम् । उ॒लूख॑लम् च । चो॒प॒दधा॑ति । उ॒प॒दधा॑ति॒ ताभ्या᳚म् । उ॒प॒दधा॒तीत्यु॑प - दधा॑ति । ताभ्या॑मे॒व । ए॒व यज॑मानः । यज॑मानो॒ऽमुष्मिन्न्॑ । अ॒मुष्मि॑न् ॅलो॒के । लो॒के᳚ऽग्निम् । अ॒ग्निम् दु॑हे । दु॒हे॒ स॒म्ॅव॒थ्स॒रः । स॒म्ॅव॒थ्स॒रो वै । स॒म्ॅव॒थ्स॒र इति॑ सम् - व॒थ्स॒रः । वा ए॒षः । ए॒ष यत् । यद॒ग्निः । अ॒ग्निस्तस्य॑ । तस्य॑ त्रेधाविहि॒ताः । त्रे॒धा॒वि॒हि॒ता इष्ट॑काः । त्रे॒धा॒वि॒हि॒ता इति॑ त्रेधा - वि॒हि॒ताः । इष्ट॑काः प्राजाप॒त्याः । प्रा॒जा॒प॒त्या वै᳚ष्ण॒वीः । प्रा॒जा॒प॒त्या इति॑ प्राजा - प॒त्याः । वै॒ष्ण॒वीर् वै᳚श्वकर्म॒णीः \newline

\textbf{Jatai Paata} \newline

1. पुरु॑ष आ॒त्मा ऽऽत्मा पुरु॑षः॒ पुरु॑ष आ॒त्मा । \newline
2. आ॒त्मा च॑ चा॒त्मा ऽऽत्मा च॑ । \newline
3. च॒ शिरः॒ शिर॑श्च च॒ शिरः॑ । \newline
4. शिर॑श्च च॒ शिरः॒ शिर॑श्च । \newline
5. च॒ च॒त्वारि॑ च॒त्वारि॑ च च च॒त्वारि॑ । \newline
6. च॒त्वार्यङ्गा॒ न्यङ्गा॑नि च॒त्वारि॑ च॒त्वार्यङ्गा॑नि । \newline
7. अङ्गा᳚ न्या॒त्मन् ना॒त्मन् नङ्गा॒ न्यङ्गा᳚ न्या॒त्मन्न् । \newline
8. आ॒त्मन् ने॒वैवात्मन् ना॒त्मन् ने॒व । \newline
9. ए॒वैन॑ मेन मे॒वै वैन᳚म् । \newline
10. ए॒न॒म् बि॒भ॒र्ति॒ बि॒भ॒र् त्ये॒न॒ मे॒न॒म् बि॒भ॒र्ति॒ । \newline
11. बि॒भ॒र्ति॒ प्र॒जाप॑तिः प्र॒जाप॑तिर् बिभर्ति बिभर्ति प्र॒जाप॑तिः । \newline
12. प्र॒जाप॑ति॒र् वै वै प्र॒जाप॑तिः प्र॒जाप॑ति॒र् वै । \newline
13. प्र॒जाप॑ति॒रिति॑ प्र॒जा - प॒तिः॒ । \newline
14. वा ए॒ष ए॒ष वै वा ए॒षः । \newline
15. ए॒ष यद् यदे॒ष ए॒ष यत् । \newline
16. यद॒ग्नि र॒ग्निर् यद् यद॒ग्निः । \newline
17. अ॒ग्नि स्तस्य॒ तस्या॒ग्नि र॒ग्नि स्तस्य॑ । \newline
18. तस्यो॒खोखा तस्य॒ तस्यो॒खा । \newline
19. उ॒खा च॑ चो॒खोखा च॑ । \newline
20. चो॒लूख॑ल मु॒लूख॑लम् च चो॒लूख॑लम् । \newline
21. उ॒लूख॑लम् च चो॒लूख॑ल मु॒लूख॑लम् च । \newline
22. च॒ स्तनौ॒ स्तनौ॑ च च॒ स्तनौ᳚ । \newline
23. स्तनौ॒ तौ तौ स्तनौ॒ स्तनौ॒ तौ । \newline
24. ता व॑स्यास्य॒ तौ ता व॑स्य । \newline
25. अ॒स्य॒ प्र॒जाः प्र॒जा अ॑स्यास्य प्र॒जाः । \newline
26. प्र॒जा उपोप॑ प्र॒जाः प्र॒जा उप॑ । \newline
27. प्र॒जा इति॑ प्र - जाः । \newline
28. उप॑ जीवन्ति जीव॒न् त्युपोप॑ जीवन्ति । \newline
29. जी॒व॒न्ति॒ यद् यज् जी॑वन्ति जीवन्ति॒ यत् । \newline
30. यदु॒खा मु॒खां ॅयद् यदु॒खाम् । \newline
31. उ॒खाम् च॑ चो॒खा मु॒खाम् च॑ । \newline
32. चो॒लूख॑ल मु॒लूख॑लम् च चो॒लूख॑लम् । \newline
33. उ॒लूख॑लम् च चो॒लूख॑ल मु॒लूख॑लम् च । \newline
34. चो॒प॒दधा᳚ त्युप॒दधा॑ति च चोप॒दधा॑ति । \newline
35. उ॒प॒दधा॑ति॒ ताभ्या॒म् ताभ्या॑ मुप॒दधा᳚ त्युप॒दधा॑ति॒ ताभ्या᳚म् । \newline
36. उ॒प॒दधा॒तीत्यु॑प - दधा॑ति । \newline
37. ताभ्या॑ मे॒वैव ताभ्या॒म् ताभ्या॑ मे॒व । \newline
38. ए॒व यज॑मानो॒ यज॑मान ए॒वैव यज॑मानः । \newline
39. यज॑मानो॒ ऽमुष्मि॑न् न॒मुष्मि॒न्॒. यज॑मानो॒ यज॑मानो॒ ऽमुष्मिन्न्॑ । \newline
40. अ॒मुष्मि॑न् ॅलो॒के लो॒के॑ ऽमुष्मि॑न् न॒मुष्मि॑न् ॅलो॒के । \newline
41. लो॒के᳚ ऽग्नि म॒ग्निम् ॅलो॒के लो॒के᳚ ऽग्निम् । \newline
42. अ॒ग्निम् दु॑हे दुहे॒ ऽग्नि म॒ग्निम् दु॑हे । \newline
43. दु॒हे॒ सं॒ॅव॒थ्स॒रः सं॑ॅवथ्स॒रो दु॑हे दुहे संॅवथ्स॒रः । \newline
44. सं॒ॅव॒थ्स॒रो वै वै सं॑ॅवथ्स॒रः सं॑ॅवथ्स॒रो वै । \newline
45. सं॒ॅव॒थ्स॒र इति॑ सं - व॒थ्स॒रः । \newline
46. वा ए॒ष ए॒ष वै वा ए॒षः । \newline
47. ए॒ष यद् यदे॒ष ए॒ष यत् । \newline
48. यद॒ग्नि र॒ग्निर् यद् यद॒ग्निः । \newline
49. अ॒ग्नि स्तस्य॒ तस्या॒ग्नि र॒ग्नि स्तस्य॑ । \newline
50. तस्य॑ त्रेधाविहि॒ता स्त्रे॑धाविहि॒ता स्तस्य॒ तस्य॑ त्रेधाविहि॒ताः । \newline
51. त्रे॒धा॒वि॒हि॒ता इष्ट॑का॒ इष्ट॑का स्त्रेधाविहि॒ता स्त्रे॑धाविहि॒ता इष्ट॑काः । \newline
52. त्रे॒धा॒वि॒हि॒ता इति॑ त्रेधा - वि॒हि॒ताः । \newline
53. इष्ट॑काः प्राजाप॒त्याः प्रा॑जाप॒त्या इष्ट॑का॒ इष्ट॑काः प्राजाप॒त्याः । \newline
54. प्रा॒जा॒प॒त्या वै᳚ष्ण॒वीर् वै᳚ष्ण॒वीः प्रा॑जाप॒त्याः प्रा॑जाप॒त्या वै᳚ष्ण॒वीः । \newline
55. प्रा॒जा॒प॒त्या इति॑ प्राजा - प॒त्याः॒ । \newline
56. वै॒ष्ण॒वीर् वै᳚श्वकर्म॒णीर् वै᳚श्वकर्म॒णीर् वै᳚ष्ण॒वीर् वै᳚ष्ण॒वीर् वै᳚श्वकर्म॒णीः । \newline

\textbf{Ghana Paata } \newline

1. पुरु॑ष आ॒त्मा ऽऽत्मा पुरु॑षः॒ पुरु॑ष आ॒त्मा च॑ चा॒त्मा पुरु॑षः॒ पुरु॑ष आ॒त्मा च॑ । \newline
2. आ॒त्मा च॑ चा॒त्मा ऽऽत्मा च॒ शिरः॒ शिर॑ श्चा॒त्मा ऽऽत्मा च॒ शिरः॑ । \newline
3. च॒ शिरः॒ शिर॑श्च च॒ शिर॑श्च च॒ शिर॑श्च च॒ शिर॑श्च । \newline
4. शिर॑श्च च॒ शिरः॒ शिर॑श्च च॒त्वारि॑ च॒त्वारि॑ च॒ शिरः॒ शिर॑श्च च॒त्वारि॑ । \newline
5. च॒ च॒त्वारि॑ च॒त्वारि॑ च च च॒त्वार्यङ्गा॒ न्यङ्गा॑नि च॒त्वारि॑ च च च॒त्वार्यङ्गा॑नि । \newline
6. च॒त्वा र्यङ्गा॒ न्यङ्गा॑नि च॒त्वारि॑ च॒त्वार्यङ्गा᳚ न्या॒त्मन् ना॒त्मन् नङ्गा॑नि च॒त्वारि॑ च॒त्वार्यङ्गा᳚ न्या॒त्मन्न् । \newline
7. अङ्गा᳚ न्या॒त्मन् ना॒त्मन् नङ्गा॒ न्यङ्गा᳚ न्या॒त्मन् ने॒वै वात्मन् नङ्गा॒ न्यङ्गा᳚ न्या॒त्मन् ने॒व । \newline
8. आ॒त्मन् ने॒वै वात्मन् ना॒त्मन् ने॒वैन॑ मेन मे॒वात्मन् ना॒त्मन् ने॒वैन᳚म् । \newline
9. ए॒वैन॑ मेन मे॒वै वैन॑म् बिभर्ति बिभर्त्येन मे॒वै वैन॑म् बिभर्ति । \newline
10. ए॒न॒म् बि॒भ॒र्ति॒ बि॒भ॒र्त्ये॒न॒ मे॒न॒म् बि॒भ॒र्ति॒ प्र॒जाप॑तिः प्र॒जाप॑तिर् बिभर्त्येन मेनम् बिभर्ति प्र॒जाप॑तिः । \newline
11. बि॒भ॒र्ति॒ प्र॒जाप॑तिः प्र॒जाप॑तिर् बिभर्ति बिभर्ति प्र॒जाप॑ति॒र् वै वै प्र॒जाप॑तिर् बिभर्ति बिभर्ति प्र॒जाप॑ति॒र् वै । \newline
12. प्र॒जाप॑ति॒र् वै वै प्र॒जाप॑तिः प्र॒जाप॑ति॒र् वा ए॒ष ए॒ष वै प्र॒जाप॑तिः प्र॒जाप॑ति॒र् वा ए॒षः । \newline
13. प्र॒जाप॑ति॒रिति॑ प्र॒जा - प॒तिः॒ । \newline
14. वा ए॒ष ए॒ष वै वा ए॒ष यद् यदे॒ष वै वा ए॒ष यत् । \newline
15. ए॒ष यद् यदे॒ष ए॒ष यद॒ग्नि र॒ग्निर् यदे॒ष ए॒ष यद॒ग्निः । \newline
16. यद॒ग्नि र॒ग्निर् यद् यद॒ग्नि स्तस्य॒ तस्या॒ग्निर् यद् यद॒ग्नि स्तस्य॑ । \newline
17. अ॒ग्नि स्तस्य॒ तस्या॒ग्नि र॒ग्नि स्तस्यो॒ खोखा तस्या॒ग्नि र॒ग्नि स्तस्यो॒खा । \newline
18. तस्यो॒ खोखा तस्य॒ तस्यो॒खा च॑ चो॒खा तस्य॒ तस्यो॒खा च॑ । \newline
19. उ॒खा च॑ चो॒खोखा चो॒लूख॑ल मु॒लूख॑लम् चो॒खोखा चो॒लूख॑लम् । \newline
20. चो॒लूख॑ल मु॒लूख॑लम् च चो॒लूख॑लम् च चो॒लूख॑लम् च चो॒लूख॑लम् च । \newline
21. उ॒लूख॑लम् च चो॒लूख॑ल मु॒लूख॑लम् च॒ स्तनौ॒ स्तनौ॑ चो॒लूख॑ल मु॒लूख॑लम् च॒ स्तनौ᳚ । \newline
22. च॒ स्तनौ॒ स्तनौ॑ च च॒ स्तनौ॒ तौ तौ स्तनौ॑ च च॒ स्तनौ॒ तौ । \newline
23. स्तनौ॒ तौ तौ स्तनौ॒ स्तनौ॒ ता व॑स्यास्य॒ तौ स्तनौ॒ स्तनौ॒ ता व॑स्य । \newline
24. ता व॑स्यास्य॒ तौ ता व॑स्य प्र॒जाः प्र॒जा अ॑स्य॒ तौ ता व॑स्य प्र॒जाः । \newline
25. अ॒स्य॒ प्र॒जाः प्र॒जा अ॑स्यास्य प्र॒जा उपोप॑ प्र॒जा अ॑स्यास्य प्र॒जा उप॑ । \newline
26. प्र॒जा उपोप॑ प्र॒जाः प्र॒जा उप॑ जीवन्ति जीव॒न्त्युप॑ प्र॒जाः प्र॒जा उप॑ जीवन्ति । \newline
27. प्र॒जा इति॑ प्र - जाः । \newline
28. उप॑ जीवन्ति जीव॒न् त्युपोप॑ जीवन्ति॒ यद् यज् जी॑व॒न् त्युपोप॑ जीवन्ति॒ यत् । \newline
29. जी॒व॒न्ति॒ यद् यज् जी॑वन्ति जीवन्ति॒ यदु॒खा मु॒खां ॅयज् जी॑वन्ति जीवन्ति॒ यदु॒खाम् । \newline
30. यदु॒खा मु॒खां ॅयद् यदु॒खाम् च॑ चो॒खां ॅयद् यदु॒खाम् च॑ । \newline
31. उ॒खाम् च॑ चो॒खा मु॒खाम् चो॒लूख॑ल मु॒लूख॑लम् चो॒खा मु॒खाम् चो॒लूख॑लम् । \newline
32. चो॒लूख॑ल मु॒लूख॑लम् च चो॒लूख॑लम् च चो॒लूख॑लम् च चो॒लूख॑लम् च । \newline
33. उ॒लूख॑लम् च चो॒लूख॑ल मु॒लूख॑लम् चोप॒दधा᳚ त्युप॒दधा॑ति चो॒लूख॑ल मु॒लूख॑लम् चोप॒दधा॑ति । \newline
34. चो॒प॒दधा᳚ त्युप॒दधा॑ति च चोप॒दधा॑ति॒ ताभ्या॒म् ताभ्या॑ मुप॒दधा॑ति च चोप॒दधा॑ति॒ ताभ्या᳚म् । \newline
35. उ॒प॒दधा॑ति॒ ताभ्या॒म् ताभ्या॑ मुप॒दधा᳚ त्युप॒दधा॑ति॒ ताभ्या॑ मे॒वैव ताभ्या॑ मुप॒दधा᳚ त्युप॒दधा॑ति॒ ताभ्या॑ मे॒व । \newline
36. उ॒प॒दधा॒तीत्यु॑प - दधा॑ति । \newline
37. ताभ्या॑ मे॒वैव ताभ्या॒म् ताभ्या॑ मे॒व यज॑मानो॒ यज॑मान ए॒व ताभ्या॒म् ताभ्या॑ मे॒व यज॑मानः । \newline
38. ए॒व यज॑मानो॒ यज॑मान ए॒वैव यज॑मानो॒ ऽमुष्मि॑न् न॒मुष्मि॒न्॒. यज॑मान ए॒वैव यज॑मानो॒ ऽमुष्मिन्न्॑ । \newline
39. यज॑मानो॒ ऽमुष्मि॑न् न॒मुष्मि॒न्॒. यज॑मानो॒ यज॑मानो॒ ऽमुष्मि॑न् ॅलो॒के लो॒के॑ ऽमुष्मि॒न्॒. यज॑मानो॒ यज॑मानो॒ ऽमुष्मि॑न् ॅलो॒के । \newline
40. अ॒मुष्मि॑न् ॅलो॒के लो॒के॑ ऽमुष्मि॑न् न॒मुष्मि॑न् ॅलो॒के᳚ ऽग्नि म॒ग्निम् ॅलो॒के॑ ऽमुष्मि॑न् न॒मुष्मि॑न् ॅलो॒के᳚ ऽग्निम् । \newline
41. लो॒के᳚ ऽग्नि म॒ग्निम् ॅलो॒के लो॒के᳚ ऽग्निम् दु॑हे दुहे॒ ऽग्निम् ॅलो॒के लो॒के᳚ ऽग्निम् दु॑हे । \newline
42. अ॒ग्निम् दु॑हे दुहे॒ ऽग्नि म॒ग्निम् दु॑हे संॅवथ्स॒रः सं॑ॅवथ्स॒रो दु॑हे॒ ऽग्नि म॒ग्निम् दु॑हे संॅवथ्स॒रः । \newline
43. दु॒हे॒ सं॒ॅव॒थ्स॒रः सं॑ॅवथ्स॒रो दु॑हे दुहे संॅवथ्स॒रो वै वै सं॑ॅवथ्स॒रो दु॑हे दुहे संॅवथ्स॒रो वै । \newline
44. सं॒ॅव॒थ्स॒रो वै वै सं॑ॅवथ्स॒रः सं॑ॅवथ्स॒रो वा ए॒ष ए॒ष वै सं॑ॅवथ्स॒रः सं॑ॅवथ्स॒रो वा ए॒षः । \newline
45. सं॒ॅव॒थ्स॒र इति॑ सं - व॒थ्स॒रः । \newline
46. वा ए॒ष ए॒ष वै वा ए॒ष यद् यदे॒ष वै वा ए॒ष यत् । \newline
47. ए॒ष यद् यदे॒ष ए॒ष यद॒ग्नि र॒ग्निर् यदे॒ष ए॒ष यद॒ग्निः । \newline
48. यद॒ग्नि र॒ग्निर् यद् यद॒ग्नि स्तस्य॒ तस्या॒ग्निर् यद् यद॒ग्नि स्तस्य॑ । \newline
49. अ॒ग्नि स्तस्य॒ तस्या॒ग्नि र॒ग्नि स्तस्य॑ त्रेधाविहि॒ता स्त्रे॑धाविहि॒ता स्तस्या॒ग्नि र॒ग्नि स्तस्य॑ त्रेधाविहि॒ताः । \newline
50. तस्य॑ त्रेधाविहि॒ता स्त्रे॑धाविहि॒ता स्तस्य॒ तस्य॑ त्रेधाविहि॒ता इष्ट॑का॒ इष्ट॑का स्त्रेधाविहि॒ता स्तस्य॒ तस्य॑ त्रेधाविहि॒ता इष्ट॑काः । \newline
51. त्रे॒धा॒वि॒हि॒ता इष्ट॑का॒ इष्ट॑का स्त्रेधाविहि॒ता स्त्रे॑धाविहि॒ता इष्ट॑काः प्राजाप॒त्याः प्रा॑जाप॒त्या इष्ट॑का स्त्रेधाविहि॒ता स्त्रे॑धाविहि॒ता इष्ट॑काः प्राजाप॒त्याः । \newline
52. त्रे॒धा॒वि॒हि॒ता इति॑ त्रेधा - वि॒हि॒ताः । \newline
53. इष्ट॑काः प्राजाप॒त्याः प्रा॑जाप॒त्या इष्ट॑का॒ इष्ट॑काः प्राजाप॒त्या वै᳚ष्ण॒वीर् वै᳚ष्ण॒वीः प्रा॑जाप॒त्या इष्ट॑का॒ इष्ट॑काः प्राजाप॒त्या वै᳚ष्ण॒वीः । \newline
54. प्रा॒जा॒प॒त्या वै᳚ष्ण॒वीर् वै᳚ष्ण॒वीः प्रा॑जाप॒त्याः प्रा॑जाप॒त्या वै᳚ष्ण॒वीर् वै᳚श्वकर्म॒णीर् वै᳚श्वकर्म॒णीर् वै᳚ष्ण॒वीः प्रा॑जाप॒त्याः प्रा॑जाप॒त्या वै᳚ष्ण॒वीर् वै᳚श्वकर्म॒णीः । \newline
55. प्रा॒जा॒प॒त्या इति॑ प्राजा - प॒त्याः॒ । \newline
56. वै॒ष्ण॒वीर् वै᳚श्वकर्म॒णीर् वै᳚श्वकर्म॒णीर् वै᳚ष्ण॒वीर् वै᳚ष्ण॒वीर् वै᳚श्वकर्म॒णी र॑होरा॒त्रा
ण्य॑होरा॒त्राणि॑ वैश्वकर्म॒णीर् वै᳚ष्ण॒वीर् वै᳚ष्ण॒वीर् वै᳚श्वकर्म॒णी र॑होरा॒त्राणि॑ । \newline
\pagebreak
\markright{ TS 5.6.9.3  \hfill https://www.vedavms.in \hfill}

\section{ TS 5.6.9.3 }

\textbf{TS 5.6.9.3 } \newline
\textbf{Samhita Paata} \newline

-र्वै᳚श्वकर्म॒णी-र॑होरा॒त्राण्ये॒वास्य॑ प्राजाप॒त्या यदुख्यं॑ बि॒भर्ति॑ प्राजाप॒त्या ए॒व तदुप॑ धत्ते॒ यथ् स॒मिध॑ आ॒दधा॑ति वैष्ण॒वा वै वन॒स्पत॑यो वैष्ण॒वीरे॒व तदुप॑ धत्ते॒ यदिष्ट॑काभिर॒ग्निं चि॒नोती॒यं ॅवै वि॒श्वक॑र्मा वैश्वकर्म॒णीरे॒व तदुप॑ धत्ते॒ तस्मा॑-दाहु-स्त्रि॒वृद॒ग्निरिति॒ तं ॅवा ए॒तं ॅयज॑मान ए॒व चि॑न्वीत॒ यद॑स्या॒न्य ( ) श्चि॑नु॒याद्यत् तं दक्षि॑णाभि॒र्न रा॒धये॑द॒ग्निम॑स्य वृञ्जीत॒ यो᳚ऽस्या॒ऽग्निं चि॑नु॒यात् तं दक्षि॑णाभी राधयेद॒ग्निमे॒व तथ् स्पृ॑णोति ॥ \newline

\textbf{Pada Paata} \newline

वै॒श्व॒क॒र्म॒णीरिति॑ वैश्व - क॒र्म॒णीः । अ॒हो॒रा॒त्राणीत्य॑हः - रा॒त्राणि॑ । ए॒व । अ॒स्य॒ । प्रा॒जा॒प॒त्या इति॑ प्राजा - प॒त्याः । यत् । उख्य᳚म् । बि॒भर्ति॑ । प्रा॒जा॒प॒त्या इति॑ प्राजा-प॒त्याः । ए॒व । तत् । उपेति॑ । ध॒त्ते॒ । यत् । स॒मिध॒ इति॑ सं - इधः॑ । आ॒दधा॒तीत्या᳚ - दधा॑ति । वै॒ष्ण॒वाः । वै । वन॒स्पत॑यः । वै॒ष्ण॒वीः । ए॒व । तत् । उपेति॑ । ध॒त्ते॒ । यत् । इष्ट॑काभिः । अ॒ग्निम् । चि॒नोति॑ । इ॒यम् । वै । वि॒श्वक॒र्मेति॑ वि॒श्व - क॒र्मा॒ । वै॒श्व॒क॒र्म॒णीरिति॑ वैश्व - क॒र्म॒णीः । ए॒व । तत् । उपेति॑ । ध॒त्ते॒ । तस्मा᳚त् । आ॒हुः॒ । त्रि॒वृदिति॑ त्रि - वृत् । अ॒ग्निः । इति॑ । तम् । वै । ए॒तम् । यज॑मानः । ए॒व । चि॒न्वी॒त॒ । यत् । अ॒स्य॒ । अ॒न्यः ( ) । चि॒नु॒यात् । यत् । तम् । दक्षि॑णाभिः । न । रा॒धये᳚त् । अ॒ग्निम् । अ॒स्य॒ । वृ॒ञ्जी॒त॒ । यः । अ॒स्य॒ । अ॒ग्निम् । चि॒नु॒यात् । तम् । दक्षि॑णाभिः । रा॒ध॒ये॒त् । अ॒ग्निम् । ए॒व । तत् । स्पृ॒णो॒ति॒ ॥  \newline


\textbf{Krama Paata} \newline

वै॒श्व॒क॒र्म॒णीर॑होरा॒त्राणि॑ । वै॒श्व॒क॒र्म॒णीरिति॑ वैश्व - क॒र्म॒णीः । अ॒हो॒रा॒त्राण्ये॒व । अ॒हो॒रा॒त्राणीत्य॑हः - रा॒त्राणि॑ । ए॒वास्य॑ । अ॒स्य॒ प्रा॒जा॒प॒त्याः । प्रा॒जा॒प॒त्या यत् । प्रा॒जा॒प॒त्या इति॑ प्राजा - प॒त्याः । यदुख्य᳚म् । उख्य॑म् बि॒भर्ति॑ । बि॒भर्ति॑ प्राजाप॒त्याः । प्रा॒ज॒प॒त्या ए॒व । प्रा॒जा॒प॒त्या इति॑ प्राजा - प॒त्याः । ए॒व तत् । तदुप॑ । उप॑ धत्ते । ध॒त्ते॒ यत् । यथ् स॒मिधः॑ । स॒मिध॑ आ॒दधा॑ति । स॒मिध॒ इति॑ सम् - इधः॑ । आ॒दधा॑ति वैष्ण॒वाः । आ॒दधा॒तीत्या᳚ - दधा॑ति । वै॒ष्ण॒वा वै । वै वन॒स्पत॑यः । वन॒स्पत॑यो वैष्ण॒वीः । वै॒ष्ण॒वीरे॒व । ए॒व तत् । तदुप॑ । उप॑ धत्ते । ध॒त्ते॒ यत् । यदिष्ट॑काभिः । इष्ट॑काभिर॒ग्निम् । अ॒ग्निम् चि॒नोति॑ । चि॒नोती॒यम् । इ॒यम् ॅवै । वै वि॒श्वक॑र्मा । वि॒श्वक॑र्मा वैश्वकर्म॒णीः । वि॒श्वक॒र्मेति॑ वि॒श्व - क॒र्मा॒ । वै॒श्व॒क॒र्म॒णीरे॒व । वै॒श्व॒क॒र्म॒णीरिति॑ वैश्व - क॒र्म॒णीः । ए॒व तत् । तदुप॑ । उप॑ धत्ते । ध॒त्ते॒ तस्मा᳚त् । तस्मा॑दाहुः । आ॒हु॒स्त्रि॒वृत् । त्रि॒वृद॒ग्निः । त्रि॒वृदिति॑ त्रि - वृत् । अ॒ग्निरिति॑ । इति॒ तम् । तम् ॅवै । वा ए॒तम् । ए॒तम् ॅयज॑मानः । यज॑मान ए॒व । ए॒व चि॑न्वीत । चि॒न्वी॒त॒ यत् । यद॑स्य ( ) । अ॒स्या॒न्यः । अ॒न्यश्चि॑नु॒यात् । चि॒नु॒याद् यत् । यत् तम् । तम् दक्षि॑णाभिः । दक्षि॑णाभि॒र् न । न रा॒धये᳚त् । रा॒धये॑द॒ग्निम् । अ॒ग्निम॑स्य । अ॒स्य॒ वृ॒ञ्जी॒त॒ । वृ॒ञ्जी॒त॒ यः । यो᳚ऽस्य । अ॒स्या॒ग्निम् । अ॒ग्निम् चि॑नु॒यात् । चि॒नु॒यात् तम् । तम् दक्षि॑णाभिः । 
दक्षि॑णाभी राधयेत् । रा॒ध॒ये॒द॒ग्निम् । अ॒ग्निमे॒व । ए॒व तत् । तथ् स्पृ॑णोति । स्पृ॒णो॒तीति॑ स्पृणोति । \newline

\textbf{Jatai Paata} \newline

1. वै॒श्व॒क॒र्म॒णी र॑होरा॒त्राण्य॑ होरा॒त्राणि॑ वैश्वकर्म॒णीर् वै᳚श्वकर्म॒णी र॑होरा॒त्राणि॑ । \newline
2. वै॒श्व॒क॒र्म॒णीरिति॑ वैश्व - क॒र्म॒णीः । \newline
3. अ॒हो॒रा॒त्रा ण्ये॒वैवा हो॑रा॒त्रा ण्य॑होरा॒त्राण्ये॒व । \newline
4. अ॒हो॒रा॒त्राणीत्य॑हः - रा॒त्राणि॑ । \newline
5. ए॒वास्या᳚ स्यै॒वै वास्य॑ । \newline
6. अ॒स्य॒ प्रा॒जा॒प॒त्याः प्रा॑जाप॒त्या अ॑स्यास्य प्राजाप॒त्याः । \newline
7. प्रा॒जा॒प॒त्या यद् यत् प्रा॑जाप॒त्याः प्रा॑जाप॒त्या यत् । \newline
8. प्रा॒जा॒प॒त्या इति॑ प्राजा - प॒त्याः । \newline
9. यदुख्य॒ मुख्यं॒ ॅयद् यदुख्य᳚म् । \newline
10. उख्य॑म् बि॒भर्ति॑ बि॒भर् त्युख्य॒ मुख्य॑म् बि॒भर्ति॑ । \newline
11. बि॒भर्ति॑ प्राजाप॒त्याः प्रा॑जाप॒त्या बि॒भर्ति॑ बि॒भर्ति॑ प्राजाप॒त्याः । \newline
12. प्रा॒जा॒प॒त्या ए॒वैव प्रा॑जाप॒त्याः प्रा॑जाप॒त्या ए॒व । \newline
13. प्रा॒जा॒प॒त्या इति॑ प्राजा - प॒त्याः । \newline
14. ए॒व तत् तदे॒ वैव तत् । \newline
15. तदु पोप॒ तत् तदुप॑ । \newline
16. उप॑ धत्ते धत्त॒ उपोप॑ धत्ते । \newline
17. ध॒त्ते॒ यद् यद् ध॑त्ते धत्ते॒ यत् । \newline
18. यथ् स॒मिधः॑ स॒मिधो॒ यद् यथ् स॒मिधः॑ । \newline
19. स॒मिध॑ आ॒दधा᳚ त्या॒दधा॑ति स॒मिधः॑ स॒मिध॑ आ॒दधा॑ति । \newline
20. स॒मिध॒ इति॑ सं - इधः॑ । \newline
21. आ॒दधा॑ति वैष्ण॒वा वै᳚ष्ण॒वा आ॒दधा᳚ त्या॒दधा॑ति वैष्ण॒वाः । \newline
22. आ॒दधा॒तीत्या᳚ - दधा॑ति । \newline
23. वै॒ष्ण॒वा वै वै वै᳚ष्ण॒वा वै᳚ष्ण॒वा वै । \newline
24. वै वन॒स्पत॑यो॒ वन॒स्पत॑यो॒ वै वै वन॒स्पत॑यः । \newline
25. वन॒स्पत॑यो वैष्ण॒वीर् वै᳚ष्ण॒वीर् वन॒स्पत॑यो॒ वन॒स्पत॑यो वैष्ण॒वीः । \newline
26. वै॒ष्ण॒वी रे॒वैव वै᳚ष्ण॒वीर् वै᳚ष्ण॒वी रे॒व । \newline
27. ए॒व तत् तदे॒ वैव तत् । \newline
28. तदुपोप॒ तत् तदुप॑ । \newline
29. उप॑ धत्ते धत्त॒ उपोप॑ धत्ते । \newline
30. ध॒त्ते॒ यद् यद् ध॑त्ते धत्ते॒ यत् । \newline
31. यदिष्ट॑काभि॒ रिष्ट॑काभि॒र् यद् यदिष्ट॑काभिः । \newline
32. इष्ट॑काभि र॒ग्नि म॒ग्नि मिष्ट॑काभि॒ रिष्ट॑काभि र॒ग्निम् । \newline
33. अ॒ग्निम् चि॒नोति॑ चि॒नो त्य॒ग्नि म॒ग्निम् चि॒नोति॑ । \newline
34. चि॒नोती॒य मि॒यम् चि॒नोति॑ चि॒नोती॒यम् । \newline
35. इ॒यं ॅवै वा इ॒य मि॒यं ॅवै । \newline
36. वै वि॒श्वक॑र्मा वि॒श्वक॑र्मा॒ वै वै वि॒श्वक॑र्मा । \newline
37. वि॒श्वक॑र्मा वैश्वकर्म॒णीर् वै᳚श्वकर्म॒णीर् वि॒श्वक॑र्मा वि॒श्वक॑र्मा वैश्वकर्म॒णीः । \newline
38. वि॒श्वक॒र्मेति॑ वि॒श्व - क॒र्मा॒ । \newline
39. वै॒श्व॒क॒र्म॒णी रे॒वैव वै᳚श्वकर्म॒णीर् वै᳚श्वकर्म॒णी रे॒व । \newline
40. वै॒श्व॒क॒र्म॒णीरिति॑ वैश्व - क॒र्म॒णीः । \newline
41. ए॒व तत् तदे॒ वैव तत् । \newline
42. तदुपोप॒ तत् तदुप॑ । \newline
43. उप॑ धत्ते धत्त॒ उपोप॑ धत्ते । \newline
44. ध॒त्ते॒ तस्मा॒त् तस्मा᳚द् धत्ते धत्ते॒ तस्मा᳚त् । \newline
45. तस्मा॑ दाहु राहु॒ स्तस्मा॒त् तस्मा॑ दाहुः । \newline
46. आ॒हु॒ स्त्रि॒वृत् त्रि॒वृ दा॑हु राहु स्त्रि॒वृत् । \newline
47. त्रि॒वृ द॒ग्नि र॒ग्नि स्त्रि॒वृत् त्रि॒वृ द॒ग्निः । \newline
48. त्रि॒वृदिति॑ त्रि - वृत् । \newline
49. अ॒ग्नि रिती त्य॒ग्नि र॒ग्नि रिति॑ । \newline
50. इति॒ तम् त मितीति॒ तम् । \newline
51. तं ॅवै वै तम् तं ॅवै । \newline
52. वा ए॒त मे॒तं ॅवै वा ए॒तम् । \newline
53. ए॒तं ॅयज॑मानो॒ यज॑मान ए॒त मे॒तं ॅयज॑मानः । \newline
54. यज॑मान ए॒वैव यज॑मानो॒ यज॑मान ए॒व । \newline
55. ए॒व चि॑न्वीत चिन्वीतै॒ वैव चि॑न्वीत । \newline
56. चि॒न्वी॒त॒ यद् यच् चि॑न्वीत चिन्वीत॒ यत् । \newline
57. यद॑स्यास्य॒ यद् यद॑स्य । \newline
58. अ॒स्या॒न्यो᳚(1॒) ऽन्यो᳚ ऽस्यास्या॒न्यः । \newline
59. अ॒न्य श्चि॑नु॒याच् चि॑नु॒या द॒न्यो᳚ ऽन्य श्चि॑नु॒यात् । \newline
60. चि॒नु॒याद् यद् यच् चि॑नु॒याच् चि॑नु॒याद् यत् । \newline
61. यत् तम् तं ॅयद् यत् तम् । \newline
62. तम् दक्षि॑णाभि॒र् दक्षि॑णाभि॒ स्तम् तम् दक्षि॑णाभिः । \newline
63. दक्षि॑णाभि॒र् न न दक्षि॑णाभि॒र् दक्षि॑णाभि॒र् न । \newline
64. न रा॒धये᳚द् रा॒धये॒न् न न रा॒धये᳚त् । \newline
65. रा॒धये॑ द॒ग्नि म॒ग्निꣳ रा॒धये᳚द् रा॒धये॑ द॒ग्निम् । \newline
66. अ॒ग्नि म॑स्या स्या॒ग्नि म॒ग्नि म॑स्य । \newline
67. अ॒स्य॒ वृ॒ञ्जी॒त॒ वृ॒ञ्जी॒ता॒ स्या॒स्य॒ वृ॒ञ्जी॒त॒ । \newline
68. वृ॒ञ्जी॒त॒ यो यो वृ॑ञ्जीत वृञ्जीत॒ यः । \newline
69. यो᳚ ऽस्यास्य॒ यो यो᳚ ऽस्य । \newline
70. अ॒स्या॒ग्नि म॒ग्नि म॑स्या स्या॒ग्निम् । \newline
71. अ॒ग्निम् चि॑नु॒याच् चि॑नु॒या द॒ग्नि म॒ग्निम् चि॑नु॒यात् । \newline
72. चि॒नु॒यात् तम् तम् चि॑नु॒याच् चि॑नु॒यात् तम् । \newline
73. तम् दक्षि॑णाभि॒र् दक्षि॑णाभि॒ स्तम् तम् दक्षि॑णाभिः । \newline
74. दक्षि॑णाभी राधयेद् राधये॒द् दक्षि॑णाभि॒र् दक्षि॑णाभी राधयेत् । \newline
75. रा॒ध॒ये॒ द॒ग्नि म॒ग्निꣳ रा॑धयेद् राधये द॒ग्निम् । \newline
76. अ॒ग्नि मे॒वै वाग्नि म॒ग्नि मे॒व । \newline
77. ए॒व तत् तदे॒ वैव तत् । \newline
78. तथ् स्पृ॑णोति स्पृणोति॒ तत् तथ् स्पृ॑णोति । \newline
79. स्पृ॒णो॒तीति॑ स्पृणोति । \newline

\textbf{Ghana Paata } \newline

1. वै॒श्व॒क॒र्म॒णी र॑होरा॒त्रा ण्य॑होरा॒त्राणि॑ वैश्वकर्म॒णीर् वै᳚श्वकर्म॒णी र॑होरा॒त्राण्ये॒वै वाहो॑रा॒त्राणि॑ वैश्वकर्म॒णीर् वै᳚श्वकर्म॒णी र॑होरा॒त्राण्ये॒व । \newline
2. वै॒श्व॒क॒र्म॒णीरिति॑ वैश्व - क॒र्म॒णीः । \newline
3. अ॒हो॒रा॒त्राण्ये॒वै वाहो॑रा॒त्रा ण्य॑होरा॒त्रा ण्ये॒वास्या᳚ स्यै॒वाहो॑रा॒त्रा ण्य॑होरा॒त्रा ण्ये॒वास्य॑ । \newline
4. अ॒हो॒रा॒त्राणीत्य॑हः - रा॒त्राणि॑ । \newline
5. ए॒वास्या᳚ स्यै॒वैवास्य॑ प्राजाप॒त्याः प्रा॑जाप॒त्या अ॑स्यै॒वैवास्य॑ प्राजाप॒त्याः । \newline
6. अ॒स्य॒ प्रा॒जा॒प॒त्याः प्रा॑जाप॒त्या अ॑स्यास्य प्राजाप॒त्या यद् यत् प्रा॑जाप॒त्या अ॑स्यास्य प्राजाप॒त्या यत् । \newline
7. प्रा॒जा॒प॒त्या यद् यत् प्रा॑जाप॒त्याः प्रा॑जाप॒त्या यदुख्य॒ मुख्यं॒ ॅयत् प्रा॑जाप॒त्याः प्रा॑जाप॒त्या यदुख्य᳚म् । \newline
8. प्रा॒जा॒प॒त्या इति॑ प्राजा - प॒त्याः । \newline
9. यदुख्य॒ मुख्यं॒ ॅयद् यदुख्य॑म् बि॒भर्ति॑ बि॒भर्त्युख्यं॒ ॅयद् यदुख्य॑म् बि॒भर्ति॑ । \newline
10. उख्य॑म् बि॒भर्ति॑ बि॒भर्त्युख्य॒ मुख्य॑म् बि॒भर्ति॑ प्राजाप॒त्याः प्रा॑जाप॒त्या बि॒भर्त्युख्य॒ मुख्य॑म् बि॒भर्ति॑ प्राजाप॒त्याः । \newline
11. बि॒भर्ति॑ प्राजाप॒त्याः प्रा॑जाप॒त्या बि॒भर्ति॑ बि॒भर्ति॑ प्राजाप॒त्या ए॒वैव प्रा॑जाप॒त्या बि॒भर्ति॑ बि॒भर्ति॑ प्राजाप॒त्या ए॒व । \newline
12. प्रा॒जा॒प॒त्या ए॒वैव प्रा॑जाप॒त्याः प्रा॑जाप॒त्या ए॒व तत् तदे॒व प्रा॑जाप॒त्याः प्रा॑जाप॒त्या ए॒व तत् । \newline
13. प्रा॒जा॒प॒त्या इति॑ प्राजा - प॒त्याः । \newline
14. ए॒व तत् तदे॒ वैव तदुपोप॒ तदे॒ वैव तदुप॑ । \newline
15. तदुपोप॒ तत् तदुप॑ धत्ते धत्त॒ उप॒ तत् तदुप॑ धत्ते । \newline
16. उप॑ धत्ते धत्त॒ उपोप॑ धत्ते॒ यद् यद् ध॑त्त॒ उपोप॑ धत्ते॒ यत् । \newline
17. ध॒त्ते॒ यद् यद् ध॑त्ते धत्ते॒ यथ् स॒मिधः॑ स॒मिधो॒ यद् ध॑त्ते धत्ते॒ यथ् स॒मिधः॑ । \newline
18. यथ् स॒मिधः॑ स॒मिधो॒ यद् यथ् स॒मिध॑ आ॒दधा᳚ त्या॒दधा॑ति स॒मिधो॒ यद् यथ् स॒मिध॑ आ॒दधा॑ति । \newline
19. स॒मिध॑ आ॒दधा᳚ त्या॒दधा॑ति स॒मिधः॑ स॒मिध॑ आ॒दधा॑ति वैष्ण॒वा वै᳚ष्ण॒वा आ॒दधा॑ति स॒मिधः॑ स॒मिध॑ आ॒दधा॑ति वैष्ण॒वाः । \newline
20. स॒मिध॒ इति॑ सं - इधः॑ । \newline
21. आ॒दधा॑ति वैष्ण॒वा वै᳚ष्ण॒वा आ॒दधा᳚ त्या॒दधा॑ति वैष्ण॒वा वै वै वै᳚ष्ण॒वा आ॒दधा᳚ त्या॒दधा॑ति वैष्ण॒वा वै । \newline
22. आ॒दधा॒तीत्या᳚ - दधा॑ति । \newline
23. वै॒ष्ण॒वा वै वै वै᳚ष्ण॒वा वै᳚ष्ण॒वा वै वन॒स्पत॑यो॒ वन॒स्पत॑यो॒ वै वै᳚ष्ण॒वा वै᳚ष्ण॒वा वै वन॒स्पत॑यः । \newline
24. वै वन॒स्पत॑यो॒ वन॒स्पत॑यो॒ वै वै वन॒स्पत॑यो वैष्ण॒वीर् वै᳚ष्ण॒वीर् वन॒स्पत॑यो॒ वै वै वन॒स्पत॑यो वैष्ण॒वीः । \newline
25. वन॒स्पत॑यो वैष्ण॒वीर् वै᳚ष्ण॒वीर् वन॒स्पत॑यो॒ वन॒स्पत॑यो वैष्ण॒वी रे॒वैव वै᳚ष्ण॒वीर् वन॒स्पत॑यो॒ वन॒स्पत॑यो वैष्ण॒वी रे॒व । \newline
26. वै॒ष्ण॒वी रे॒वैव वै᳚ष्ण॒वीर् वै᳚ष्ण॒वी रे॒व तत् तदे॒व वै᳚ष्ण॒वीर् वै᳚ष्ण॒वी रे॒व तत् । \newline
27. ए॒व तत् तदे॒ वैव तदुपोप॒ तदे॒ वैव तदुप॑ । \newline
28. तदुपोप॒ तत् तदुप॑ धत्ते धत्त॒ उप॒ तत् तदुप॑ धत्ते । \newline
29. उप॑ धत्ते धत्त॒ उपोप॑ धत्ते॒ यद् यद् ध॑त्त॒ उपोप॑ धत्ते॒ यत् । \newline
30. ध॒त्ते॒ यद् यद् ध॑त्ते धत्ते॒ यदिष्ट॑काभि॒ रिष्ट॑काभि॒र् यद् ध॑त्ते धत्ते॒ यदिष्ट॑काभिः । \newline
31. यदिष्ट॑काभि॒ रिष्ट॑काभि॒र् यद् यदिष्ट॑काभि र॒ग्नि म॒ग्नि मिष्ट॑काभि॒र् यद् यदिष्ट॑काभि र॒ग्निम् । \newline
32. इष्ट॑काभि र॒ग्नि म॒ग्नि मिष्ट॑काभि॒ रिष्ट॑काभि र॒ग्निम् चि॒नोति॑ चि॒नो त्य॒ग्नि मिष्ट॑काभि॒ रिष्ट॑काभि र॒ग्निम् चि॒नोति॑ । \newline
33. अ॒ग्निम् चि॒नोति॑ चि॒नो त्य॒ग्नि म॒ग्निम् चि॒नोती॒य मि॒यम् चि॒नो त्य॒ग्नि म॒ग्निम् चि॒नोती॒यम् । \newline
34. चि॒नोती॒य मि॒यम् चि॒नोति॑ चि॒नोती॒यं ॅवै वा इ॒यम् चि॒नोति॑ चि॒नोती॒यं ॅवै । \newline
35. इ॒यं ॅवै वा इ॒य मि॒यं ॅवै वि॒श्वक॑र्मा वि॒श्वक॑र्मा॒ वा इ॒य मि॒यं ॅवै वि॒श्वक॑र्मा । \newline
36. वै वि॒श्वक॑र्मा वि॒श्वक॑र्मा॒ वै वै वि॒श्वक॑र्मा वैश्वकर्म॒णीर् वै᳚श्वकर्म॒णीर् वि॒श्वक॑र्मा॒ वै वै वि॒श्वक॑र्मा वैश्वकर्म॒णीः । \newline
37. वि॒श्वक॑र्मा वैश्वकर्म॒णीर् वै᳚श्वकर्म॒णीर् वि॒श्वक॑र्मा वि॒श्वक॑र्मा वैश्वकर्म॒णी रे॒वैव वै᳚श्वकर्म॒णीर् वि॒श्वक॑र्मा वि॒श्वक॑र्मा वैश्वकर्म॒णी रे॒व । \newline
38. वि॒श्वक॒र्मेति॑ वि॒श्व - क॒र्मा॒ । \newline
39. वै॒श्व॒क॒र्म॒णी रे॒वैव वै᳚श्वकर्म॒णीर् वै᳚श्वकर्म॒णी रे॒व तत् तदे॒व वै᳚श्वकर्म॒णीर् वै᳚श्वकर्म॒णीरे॒व तत् । \newline
40. वै॒श्व॒क॒र्म॒णीरिति॑ वैश्व - क॒र्म॒णीः । \newline
41. ए॒व तत् तदे॒ वैव तदुपोप॒ तदे॒ वैव तदुप॑ । \newline
42. तदुपोप॒ तत् तदुप॑ धत्ते धत्त॒ उप॒ तत् तदुप॑ धत्ते । \newline
43. उप॑ धत्ते धत्त॒ उपोप॑ धत्ते॒ तस्मा॒त् तस्मा᳚द् धत्त॒ उपोप॑ धत्ते॒ तस्मा᳚त् । \newline
44. ध॒त्ते॒ तस्मा॒त् तस्मा᳚द् धत्ते धत्ते॒ तस्मा॑ दाहु राहु॒ स्तस्मा᳚द् धत्ते धत्ते॒ तस्मा॑ दाहुः । \newline
45. तस्मा॑ दाहु राहु॒ स्तस्मा॒त् तस्मा॑ दाहु स्त्रि॒वृत् त्रि॒वृ दा॑हु॒ स्तस्मा॒त् तस्मा॑ दाहु स्त्रि॒वृत् । \newline
46. आ॒हु॒ स्त्रि॒वृत् त्रि॒वृ दा॑हु राहु स्त्रि॒वृ द॒ग्नि र॒ग्नि स्त्रि॒वृ दा॑हु राहु स्त्रि॒वृ द॒ग्निः । \newline
47. त्रि॒वृ द॒ग्नि र॒ग्नि स्त्रि॒वृत् त्रि॒वृ द॒ग्निरिती त्य॒ग्नि स्त्रि॒वृत् त्रि॒वृ द॒ग्निरिति॑ । \newline
48. त्रि॒वृदिति॑ त्रि - वृत् । \newline
49. अ॒ग्निरिती त्य॒ग्नि र॒ग्निरिति॒ तम् त मित्य॒ग्नि र॒ग्निरिति॒ तम् । \newline
50. इति॒ तम् त मितीति॒ तं ॅवै वै त मितीति॒ तं ॅवै । \newline
51. तं ॅवै वै तम् तं ॅवा ए॒त मे॒तं ॅवै तम् तं ॅवा ए॒तम् । \newline
52. वा ए॒त मे॒तं ॅवै वा ए॒तं ॅयज॑मानो॒ यज॑मान ए॒तं ॅवै वा ए॒तं ॅयज॑मानः । \newline
53. ए॒तं ॅयज॑मानो॒ यज॑मान ए॒त मे॒तं ॅयज॑मान ए॒वैव यज॑मान ए॒त मे॒तं ॅयज॑मान ए॒व । \newline
54. यज॑मान ए॒वैव यज॑मानो॒ यज॑मान ए॒व चि॑न्वीत चिन्वीतै॒व यज॑मानो॒ यज॑मान ए॒व चि॑न्वीत । \newline
55. ए॒व चि॑न्वीत चिन्वी तै॒वैव चि॑न्वीत॒ यद् यच् चि॑न्वी तै॒वैव चि॑न्वीत॒ यत् । \newline
56. चि॒न्वी॒त॒ यद् यच् चि॑न्वीत चिन्वीत॒ यद॑स्यास्य॒ यच् चि॑न्वीत चिन्वीत॒ यद॑स्य । \newline
57. यद॑स्यास्य॒ यद् यद॑स्या॒न्यो᳚(1॒) ऽन्यो᳚ ऽस्य॒ यद् यद॑स्या॒न्यः । \newline
58. अ॒स्या॒न्यो᳚(1॒) ऽन्यो᳚ ऽस्यास्या॒न्य श्चि॑नु॒याच् चि॑नु॒या द॒न्यो᳚ ऽस्यास्या॒न्य श्चि॑नु॒यात् । \newline
59. आ॒न्य श्चि॑नु॒या च्चि॑नु॒या द॒न्यो᳚ ऽन्य श्चि॑नु॒याद् यद् य च्चि॑नु॒या द॒न्यो᳚ ऽन्य श्चि॑नु॒याद् यत् । \newline
60. चि॒नु॒याद् यद् यच् चि॑नु॒याच् चि॑नु॒याद् यत् तम् तं ॅयच् चि॑नु॒याच् चि॑नु॒याद् यत् तम् । \newline
61. यत् तम् तं ॅयद् यत् तम् दक्षि॑णाभि॒र् दक्षि॑णाभि॒ स्तं ॅयद् यत् तम् दक्षि॑णाभिः । \newline
62. तम् दक्षि॑णाभि॒र् दक्षि॑णाभि॒ स्तम् तम् दक्षि॑णाभि॒र् न न दक्षि॑णाभि॒ स्तम् तम् दक्षि॑णाभि॒र् न । \newline
63. दक्षि॑णाभि॒र् न न दक्षि॑णाभि॒र् दक्षि॑णाभि॒र् न रा॒धये᳚द् रा॒धये॒न् न दक्षि॑णाभि॒र् दक्षि॑णाभि॒र् न रा॒धये᳚त् । \newline
64. न रा॒धये᳚द् रा॒धये॒न् न न रा॒धये॑ द॒ग्नि म॒ग्निꣳ रा॒धये॒न् न न रा॒धये॑ द॒ग्निम् । \newline
65. रा॒धये॑ द॒ग्नि म॒ग्निꣳ रा॒धये᳚द् रा॒धये॑ द॒ग्नि म॑स्या स्या॒ग्निꣳ रा॒धये᳚द् रा॒धये॑ द॒ग्नि म॑स्य । \newline
66. अ॒ग्नि म॑स्या स्या॒ग्नि म॒ग्नि म॑स्य वृञ्जीत वृञ्जीता स्या॒ग्नि म॒ग्नि म॑स्य वृञ्जीत । \newline
67. अ॒स्य॒ वृ॒ञ्जी॒त॒ वृ॒ञ्जी॒ता॒ स्या॒स्य॒ वृ॒ञ्जी॒त॒ यो यो वृ॑ञ्जीता स्यास्य वृञ्जीत॒ यः । \newline
68. वृ॒ञ्जी॒त॒ यो यो वृ॑ञ्जीत वृञ्जीत॒ यो᳚ ऽस्यास्य॒ यो वृ॑ञ्जीत वृञ्जीत॒ यो᳚ ऽस्य । \newline
69. यो᳚ ऽस्यास्य॒ यो यो᳚ ऽस्या॒ग्नि म॒ग्नि म॑स्य॒ यो यो᳚ ऽस्या॒ग्निम् । \newline
70. अ॒स्या॒ग्नि म॒ग्नि म॑स्या स्या॒ग्निम् चि॑नु॒याच् चि॑नु॒या द॒ग्नि म॑स्या स्या॒ग्निम् चि॑नु॒यात् । \newline
71. अ॒ग्निम् चि॑नु॒याच् चि॑नु॒या द॒ग्नि म॒ग्निम् चि॑नु॒यात् तम् तम् चि॑नु॒या द॒ग्नि म॒ग्निम् चि॑नु॒यात् तम् । \newline
72. चि॒नु॒यात् तम् तम् चि॑नु॒या च्चि॑नु॒यात् तम् दक्षि॑णाभि॒र् दक्षि॑णाभि॒ स्तम् चि॑नु॒याच् चि॑नु॒यात् तम् दक्षि॑णाभिः । \newline
73. तम् दक्षि॑णाभि॒र् दक्षि॑णाभि॒ स्तम् तम् दक्षि॑णाभी राधयेद् राधये॒द् दक्षि॑णाभि॒ स्तम् तम् दक्षि॑णाभी राधयेत् । \newline
74. दक्षि॑णाभी राधयेद् राधये॒द् दक्षि॑णाभि॒र् दक्षि॑णाभी राधये द॒ग्नि म॒ग्निꣳ रा॑धये॒द् दक्षि॑णाभि॒र् दक्षि॑णाभी राधये द॒ग्निम् । \newline
75. रा॒ध॒ये॒ द॒ग्नि म॒ग्निꣳ रा॑धयेद् राधये द॒ग्नि मे॒वै वाग्निꣳ रा॑धयेद् राधये द॒ग्नि मे॒व । \newline
76. अ॒ग्नि मे॒वै वाग्नि म॒ग्नि मे॒व तत् तदे॒ वाग्नि म॒ग्नि मे॒व तत् । \newline
77. ए॒व तत् तदे॒ वैव तथ् स्पृ॑णोति स्पृणोति॒ तदे॒ वैव तथ् स्पृ॑णोति । \newline
78. तथ् स्पृ॑णोति स्पृणोति॒ तत् तथ् स्पृ॑णोति । \newline
79. स्पृ॒णो॒तीति॑ स्पृणोति । \newline
\pagebreak
\markright{ TS 5.6.10.1  \hfill https://www.vedavms.in \hfill}

\section{ TS 5.6.10.1 }

\textbf{TS 5.6.10.1 } \newline
\textbf{Samhita Paata} \newline

प्र॒जाप॑ति-र॒ग्नि-म॑चिनुत॒र्तुभिः॑ संॅवथ्स॒रं ॅव॑स॒न्तेनै॒वास्य॑ पूर्वा॒र्द्धम॑चिनुत ग्री॒ष्मेण॒ दक्षि॑णं प॒क्षं ॅव॒र्॒.षाभिः॒ पुच्छꣳ॑ श॒रदोत्त॑रं प॒क्षꣳ हे॑म॒न्तेन॒ मद्ध्यं॒ ब्रह्म॑णा॒ वा अ॑स्य॒ तत् पू᳚र्वा॒र्द्धम॑चिनुत क्ष॒त्रेण॒ दक्षि॑णं प॒क्षं प॒शुभिः॒ पुच्छं॑ ॅवि॒शोत्त॑रं प॒क्षमा॒शया॒ मद्ध्यं॒ ॅय ए॒वं ॅवि॒द्वान॒ग्निं चि॑नु॒त ऋ॒तुभि॑रे॒वैनं॑ चिनु॒तेऽथो॑ ए॒तदे॒व सर्व॒मव॑ - [  ] \newline

\textbf{Pada Paata} \newline

प्र॒जाप॑ति॒रिति॑ प्र॒जा - प॒तिः॒ । अ॒ग्निम् । अ॒चि॒नु॒त॒ । ऋ॒तुभि॒रित्यृ॒तु - भिः॒ । सं॒ॅव॒थ्स॒रमिति॑ सं - व॒थ्स॒रम् । व॒स॒न्तेन॑ । ए॒व । अ॒स्य॒ । पू॒र्वा॒द्‌र्धमिति॑ पूर्व - अ॒द्‌र्धम् । अ॒चि॒नु॒त॒ । ग्री॒ष्मेण॑ । दक्षि॑णम् । प॒क्षम् । व॒र्॒.षाभिः॑ । पुच्छ᳚म् । श॒रदा᳚ । उत्त॑र॒मित्युत् - त॒र॒म् । प॒क्षम् । हे॒म॒न्तेन॑ । मद्ध्य᳚म् । ब्रह्म॑णा । वै । अ॒स्य॒ । तत् । पू॒र्वा॒द्‌र्धमिति॑ पूर्व - अ॒द्‌र्धम् । अ॒चि॒नु॒त॒ । क्ष॒त्रेण॑ । दक्षि॑णम् । प॒क्षम् । प॒शुभि॒रिति॑ प॒शु - भिः॒ । पुच्छ᳚म् । वि॒शा । उत्त॑र॒मित्युत् - त॒र॒म् । प॒क्षम् । आ॒शया᳚ । मद्ध्य᳚म् । यः । ए॒वम् । वि॒द्वान् । अ॒ग्निम् । चि॒नु॒ते । ऋ॒तुभि॒रित्यृ॒तु - भिः॒ । ए॒व । ए॒न॒म् । चि॒नु॒ते॒ । अथो॒ इति॑ । ए॒तत् । ए॒व । सर्व᳚म् । अवेति॑ ।  \newline


\textbf{Krama Paata} \newline

प्र॒जाप॑तिर॒ग्निम् । प्र॒जाप॑ति॒रिति॑ प्र॒जा - प॒तिः॒ । अ॒ग्निम॑चिनुत । अ॒चि॒नु॒त॒र्तुभिः॑ । ऋ॒तुभिः॑ सम्ॅवथ्स॒रम् । ऋ॒तुभि॒रित्यृ॒तु - भिः॒ । स॒म्ॅव॒थ्स॒रम् ॅव॑स॒न्तेन॑ । स॒म्ॅव॒थ्स॒रमिति॑ सम् - व॒थ्स॒रम् । व॒स॒न्तेनै॒व । ए॒वास्य॑ । अ॒स्य॒ पू॒र्वा॒र्द्धम् । पू॒र्वा॒र्द्धम॑चिनुत । पू॒र्वा॒र्द्धमिति॑ पूर्व - अ॒र्द्धम् । अ॒चि॒नु॒त॒ ग्री॒ष्मेण॑ । ग्री॒ष्मेण॒ दक्षि॑णम् । दक्षि॑णम् प॒क्षम् । प॒क्षम् ॅव॒र्॒.षाभिः॑ । व॒र्॒.षाभिः॒ पुच्छ᳚म् । पुच्छꣳ॑ श॒रदा᳚ । श॒रदोत्त॑रम् । उत्त॑रम् प॒क्षम् । उत्त॑र॒मित्युत् - त॒र॒म् । प॒क्षꣳ हे॑म॒न्तेन॑ । हे॒म॒न्तेन॒ मद्ध्य᳚म् । मद्ध्य॒म् ब्रह्म॑णा । ब्रह्म॑णा॒ वै । वा अ॑स्य । अ॒स्य॒ तत् । तत् पू᳚र्वा॒र्द्धम् । पू॒र्वा॒र्द्धम॑चिनुत । पू॒र्वा॒र्द्धमिति॑ पूर्व - अ॒र्द्धम् । अ॒चि॒नु॒त॒ क्ष॒त्रेण॑ । क्ष॒त्रेण॒ दक्षि॑णम् । दक्षि॑णम् प॒क्षम् । प॒क्षम् प॒शुभिः॑ । प॒शुभिः॒ पुच्छ᳚म् । प॒शुभि॒रिति॑ प॒शु - भिः॒ । पुच्छ॑म् ॅवि॒शा । वि॒शोत्त॑रम् । उत्त॑रम् प॒क्षम् । उत्त॑र॒मित्युत् - त॒र॒म् । प॒क्षमा॒शया᳚ । आ॒शया॒ मद्ध्य᳚म् । मद्ध्य॒म् ॅयः । य ए॒वम् । ए॒वम् ॅवि॒द्वान् । वि॒द्वान॒ग्निम् । अ॒ग्निम् चि॑नु॒ते । चि॒नु॒त ऋ॒तुभिः॑ । ऋ॒तुभि॑रे॒व । ऋ॒तुभि॒रित्यृ॒तु - भिः॒ । ए॒वैन᳚म् । ए॒न॒म् चि॒नु॒ते॒ । चि॒नु॒तेऽथो᳚ । अथो॑ ए॒तत् । अथो॒ इत्यथो᳚ । ए॒तदे॒व । ए॒व सर्व᳚म् । सर्व॒मव॑ । अव॑ रुन्धे \newline

\textbf{Jatai Paata} \newline

1. प्र॒जाप॑ति र॒ग्नि म॒ग्निम् प्र॒जाप॑तिः प्र॒जाप॑ति र॒ग्निम् । \newline
2. प्र॒जाप॑ति॒रिति॑ प्र॒जा - प॒तिः॒ । \newline
3. अ॒ग्नि म॑चिनुता चिनुता॒ग्नि म॒ग्नि म॑चिनुत । \newline
4. अ॒चि॒नु॒त॒ र्‌तुभिर्॑. ऋ॒तुभि॑ रचिनुता चिनुत॒ र्‌तुभिः॑ । \newline
5. ऋ॒तुभिः॑ संॅवथ्स॒रꣳ सं॑ॅवथ्स॒र मृ॒तुभिर्॑. ऋ॒तुभिः॑ संॅवथ्स॒रम् । \newline
6. ऋ॒तुभि॒रित्यृ॒तु - भिः॒ । \newline
7. सं॒ॅव॒थ्स॒रं ॅव॑स॒न्तेन॑ वस॒न्तेन॑ संॅवथ्स॒रꣳ सं॑ॅवथ्स॒रं ॅव॑स॒न्तेन॑ । \newline
8. सं॒ॅव॒थ्स॒रमिति॑ सं - व॒थ्स॒रम् । \newline
9. व॒स॒न्ते नै॒वैव व॑स॒न्तेन॑ वस॒न्तेनै॒व । \newline
10. ए॒वास्या᳚ स्यै॒वै वास्य॑ । \newline
11. अ॒स्य॒ पू॒र्वा॒र्द्धम् पू᳚र्वा॒र्द्ध म॑स्यास्य पूर्वा॒र्द्धम् । \newline
12. पू॒र्वा॒र्द्ध म॑चिनुता चिनुत पूर्वा॒र्द्धम् पू᳚र्वा॒र्द्ध म॑चिनुत । \newline
13. पू॒र्वा॒र्द्धमिति॑ पूर्व - अ॒र्द्धम् । \newline
14. अ॒चि॒नु॒त॒ ग्री॒ष्मेण॑ ग्री॒ष्मेणा॑ चिनुता चिनुत ग्री॒ष्मेण॑ । \newline
15. ग्री॒ष्मेण॒ दक्षि॑ण॒म् दक्षि॑णम् ग्री॒ष्मेण॑ ग्री॒ष्मेण॒ दक्षि॑णम् । \newline
16. दक्षि॑णम् प॒क्षम् प॒क्षम् दक्षि॑ण॒म् दक्षि॑णम् प॒क्षम् । \newline
17. प॒क्षं ॅव॒र्॒.षाभि॑र् व॒र्॒.षाभिः॑ प॒क्षम् प॒क्षं ॅव॒र्॒.षाभिः॑ । \newline
18. व॒र्॒.षाभिः॒ पुच्छ॒म् पुच्छं॑ ॅव॒र्॒.षाभि॑र् व॒र्॒.षाभिः॒ पुच्छ᳚म् । \newline
19. पुच्छꣳ॑ श॒रदा॑ श॒रदा॒ पुच्छ॒म् पुच्छꣳ॑ श॒रदा᳚ । \newline
20. श॒रदोत्त॑र॒ मुत्त॑रꣳ श॒रदा॑ श॒रदोत्त॑रम् । \newline
21. उत्त॑रम् प॒क्षम् प॒क्ष मुत्त॑र॒ मुत्त॑रम् प॒क्षम् । \newline
22. उत्त॑र॒मित्युत् - त॒र॒म् । \newline
23. प॒क्षꣳ हे॑म॒न्तेन॑ हेम॒न्तेन॑ प॒क्षम् प॒क्षꣳ हे॑म॒न्तेन॑ । \newline
24. हे॒म॒न्तेन॒ मद्ध्य॒म् मद्ध्यꣳ॑ हेम॒न्तेन॑ हेम॒न्तेन॒ मद्ध्य᳚म् । \newline
25. मद्ध्य॒म् ब्रह्म॑णा॒ ब्रह्म॑णा॒ मद्ध्य॒म् मद्ध्य॒म् ब्रह्म॑णा । \newline
26. ब्रह्म॑णा॒ वै वै ब्रह्म॑णा॒ ब्रह्म॑णा॒ वै । \newline
27. वा अ॑स्यास्य॒ वै वा अ॑स्य । \newline
28. अ॒स्य॒ तत् तद॑स्यास्य॒ तत् । \newline
29. तत् पू᳚र्वा॒र्द्धम् पू᳚र्वा॒र्द्धम् तत् तत् पू᳚र्वा॒र्द्धम् । \newline
30. पू॒र्वा॒र्द्ध म॑चिनुता चिनुत पूर्वा॒र्द्धम् पू᳚र्वा॒र्द्ध म॑चिनुत । \newline
31. पू॒र्वा॒र्द्धमिति॑ पूर्व - अ॒र्द्धम् । \newline
32. अ॒चि॒नु॒त॒ क्ष॒त्रेण॑ क्ष॒त्रेणा॑ चिनुता चिनुत क्ष॒त्रेण॑ । \newline
33. क्ष॒त्रेण॒ दक्षि॑ण॒म् दक्षि॑णम् क्ष॒त्रेण॑ क्ष॒त्रेण॒ दक्षि॑णम् । \newline
34. दक्षि॑णम् प॒क्षम् प॒क्षम् दक्षि॑ण॒म् दक्षि॑णम् प॒क्षम् । \newline
35. प॒क्षम् प॒शुभिः॑ प॒शुभिः॑ प॒क्षम् प॒क्षम् प॒शुभिः॑ । \newline
36. प॒शुभिः॒ पुच्छ॒म् पुच्छ॑म् प॒शुभिः॑ प॒शुभिः॒ पुच्छ᳚म् । \newline
37. प॒शुभि॒रिति॑ प॒शु - भिः॒ । \newline
38. पुच्छं॑ ॅवि॒शा वि॒शा पुच्छ॒म् पुच्छं॑ ॅवि॒शा । \newline
39. वि॒शोत्त॑र॒ मुत्त॑रं ॅवि॒शा वि॒शोत्त॑रम् । \newline
40. उत्त॑रम् प॒क्षम् प॒क्ष मुत्त॑र॒ मुत्त॑रम् प॒क्षम् । \newline
41. उत्त॑र॒मित्युत् - त॒र॒म् । \newline
42. प॒क्ष मा॒शया॒ ऽऽशया॑ प॒क्षम् प॒क्ष मा॒शया᳚ । \newline
43. आ॒शया॒ मद्ध्य॒म् मद्ध्य॑ मा॒शया॒ ऽऽशया॒ मद्ध्य᳚म् । \newline
44. मद्ध्यं॒ ॅयो यो मद्ध्य॒म् मद्ध्यं॒ ॅयः । \newline
45. य ए॒व मे॒वं ॅयो य ए॒वम् । \newline
46. ए॒वं ॅवि॒द्वान्. वि॒द्वा ने॒व मे॒वं ॅवि॒द्वान् । \newline
47. वि॒द्वा न॒ग्नि म॒ग्निं ॅवि॒द्वान्. वि॒द्वा न॒ग्निम् । \newline
48. अ॒ग्निम् चि॑नु॒ते चि॑नु॒ते᳚ ऽग्नि म॒ग्निम् चि॑नु॒ते । \newline
49. चि॒नु॒त ऋ॒तुभिर्॑. ऋ॒तुभि॑ श्चिनु॒ते चि॑नु॒त ऋ॒तुभिः॑ । \newline
50. ऋ॒तुभि॑ रे॒वैव र्‌तुभिर्॑. ऋ॒तुभि॑ रे॒व । \newline
51. ऋ॒तुभि॒रित्यृ॒तु - भिः॒ । \newline
52. ए॒वैन॑ मेन मे॒वै वैन᳚म् । \newline
53. ए॒न॒म् चि॒नु॒ते॒ चि॒नु॒त॒ ए॒न॒ मे॒न॒म् चि॒नु॒ते॒ । \newline
54. चि॒नु॒ते ऽथो॒ अथो॑ चिनुते चिनु॒ते ऽथो᳚ । \newline
55. अथो॑ ए॒त दे॒त दथो॒ अथो॑ ए॒तत् । \newline
56. अथो॒ इत्यथो᳚ । \newline
57. ए॒तदे॒ वैवै तदे॒ तदे॒व । \newline
58. ए॒व सर्वꣳ॒॒ सर्व॑ मे॒वैव सर्व᳚म् । \newline
59. सर्व॒ मवाव॒ सर्वꣳ॒॒ सर्व॒ मव॑ । \newline
60. अव॑ रुन्धे रु॒न्धे ऽवाव॑ रुन्धे । \newline

\textbf{Ghana Paata } \newline

1. प्र॒जाप॑ति र॒ग्नि म॒ग्निम् प्र॒जाप॑तिः प्र॒जाप॑ति र॒ग्नि म॑चिनुता चिनुता॒ग्निम् प्र॒जाप॑तिः प्र॒जाप॑ति र॒ग्नि म॑चिनुत । \newline
2. प्र॒जाप॑ति॒रिति॑ प्र॒जा - प॒तिः॒ । \newline
3. अ॒ग्नि म॑चिनुता चिनुता॒ग्नि म॒ग्नि म॑चिनुत॒ र्‌तुभिर्॑. ऋ॒तुभि॑ रचिनुता॒ग्नि म॒ग्नि म॑चिनुत॒ र्‌तुभिः॑ । \newline
4. अ॒चि॒नु॒त॒ र्‌तुभिर्॑. ऋ॒तुभि॑ रचिनुता चिनुत॒ र्‌तुभिः॑ संॅवथ्स॒रꣳ सं॑ॅवथ्स॒र मृ॒तुभि॑ रचिनुता चिनुत॒ र्‌तुभिः॑ संॅवथ्स॒रम् । \newline
5. ऋ॒तुभिः॑ संॅवथ्स॒रꣳ सं॑ॅवथ्स॒र मृ॒तुभिर्॑. ऋ॒तुभिः॑ संॅवथ्स॒रं ॅव॑स॒न्तेन॑ वस॒न्तेन॑ संॅवथ्स॒र मृ॒तुभिर्॑. ऋ॒तुभिः॑ संॅवथ्स॒रं ॅव॑स॒न्तेन॑ । \newline
6. ऋ॒तुभि॒रित्यृ॒तु - भिः॒ । \newline
7. सं॒ॅव॒थ्स॒रं ॅव॑स॒न्तेन॑ वस॒न्तेन॑ संॅवथ्स॒रꣳ सं॑ॅवथ्स॒रं ॅव॑स॒न्ते नै॒वैव व॑स॒न्तेन॑ संॅवथ्स॒रꣳ सं॑ॅवथ्स॒रं ॅव॑स॒न्तेनै॒व । \newline
8. सं॒ॅव॒थ्स॒रमिति॑ सं - व॒थ्स॒रम् । \newline
9. व॒स॒न्ते नै॒वैव व॑स॒न्तेन॑ वस॒न्ते नै॒वास्या᳚ स्यै॒व व॑स॒न्तेन॑ वस॒न्ते नै॒वास्य॑ । \newline
10. ए॒वास्या᳚ स्यै॒वैवास्य॑ पूर्वा॒र्द्धम् पू᳚र्वा॒र्द्ध म॑स्यै॒ वैवास्य॑ पूर्वा॒र्द्धम् । \newline
11. अ॒स्य॒ पू॒र्वा॒र्द्धम् पू᳚र्वा॒र्द्ध म॑स्यास्य पूर्वा॒र्द्ध म॑चिनुता चिनुत पूर्वा॒र्द्ध म॑स्यास्य पूर्वा॒र्द्ध म॑चिनुत । \newline
12. पू॒र्वा॒र्द्ध म॑चिनुता चिनुत पूर्वा॒र्द्धम् पू᳚र्वा॒र्द्ध म॑चिनुत ग्री॒ष्मेण॑ ग्री॒ष्मेणा॑ चिनुत पूर्वा॒र्द्धम् पू᳚र्वा॒र्द्ध म॑चिनुत ग्री॒ष्मेण॑ । \newline
13. पू॒र्वा॒र्द्धमिति॑ पूर्व - अ॒र्द्धम् । \newline
14. अ॒चि॒नु॒त॒ ग्री॒ष्मेण॑ ग्री॒ष्मेणा॑ चिनुता चिनुत ग्री॒ष्मेण॒ दक्षि॑ण॒म् दक्षि॑णम् ग्री॒ष्मेणा॑ चिनुता चिनुत ग्री॒ष्मेण॒ दक्षि॑णम् । \newline
15. ग्री॒ष्मेण॒ दक्षि॑ण॒म् दक्षि॑णम् ग्री॒ष्मेण॑ ग्री॒ष्मेण॒ दक्षि॑णम् प॒क्षम् प॒क्षम् दक्षि॑णम् ग्री॒ष्मेण॑ ग्री॒ष्मेण॒ दक्षि॑णम् प॒क्षम् । \newline
16. दक्षि॑णम् प॒क्षम् प॒क्षम् दक्षि॑ण॒म् दक्षि॑णम् प॒क्षं ॅव॒र्॒.षाभि॑र् व॒र्॒.षाभिः॑ प॒क्षम् दक्षि॑ण॒म् दक्षि॑णम् प॒क्षं ॅव॒र्॒.षाभिः॑ । \newline
17. प॒क्षं ॅव॒र्॒.षाभि॑र् व॒र्॒.षाभिः॑ प॒क्षम् प॒क्षं ॅव॒र्॒.षाभिः॒ पुच्छ॒म् पुच्छं॑ ॅव॒र्॒.षाभिः॑ प॒क्षम् प॒क्षं ॅव॒र्॒.षाभिः॒ पुच्छ᳚म् । \newline
18. व॒र्॒.षाभिः॒ पुच्छ॒म् पुच्छं॑ ॅव॒र्॒.षाभि॑र् व॒र्॒.षाभिः॒ पुच्छꣳ॑ श॒रदा॑ श॒रदा॒ पुच्छं॑ ॅव॒र्॒.षाभि॑र् व॒र्॒.षाभिः॒ पुच्छꣳ॑ श॒रदा᳚ । \newline
19. पुच्छꣳ॑ श॒रदा॑ श॒रदा॒ पुच्छ॒म् पुच्छꣳ॑ श॒रदोत्त॑र॒ मुत्त॑रꣳ श॒रदा॒ पुच्छ॒म् पुच्छꣳ॑ श॒रदोत्त॑रम् । \newline
20. श॒रदोत्त॑र॒ मुत्त॑रꣳ श॒रदा॑ श॒रदोत्त॑रम् प॒क्षम् प॒क्ष मुत्त॑रꣳ श॒रदा॑ श॒रदोत्त॑रम् प॒क्षम् । \newline
21. उत्त॑रम् प॒क्षम् प॒क्ष मुत्त॑र॒ मुत्त॑रम् प॒क्षꣳ हे॑म॒न्तेन॑ हेम॒न्तेन॑ प॒क्ष मुत्त॑र॒ मुत्त॑रम् प॒क्षꣳ हे॑म॒न्तेन॑ । \newline
22. उत्त॑र॒मित्युत् - त॒र॒म् । \newline
23. प॒क्षꣳ हे॑म॒न्तेन॑ हेम॒न्तेन॑ प॒क्षम् प॒क्षꣳ हे॑म॒न्तेन॒ मद्ध्य॒म् मद्ध्यꣳ॑ हेम॒न्तेन॑ प॒क्षम् प॒क्षꣳ हे॑म॒न्तेन॒ मद्ध्य᳚म् । \newline
24. हे॒म॒न्तेन॒ मद्ध्य॒म् मद्ध्यꣳ॑ हेम॒न्तेन॑ हेम॒न्तेन॒ मद्ध्य॒म् ब्रह्म॑णा॒ ब्रह्म॑णा॒ मद्ध्यꣳ॑ हेम॒न्तेन॑ हेम॒न्तेन॒ मद्ध्य॒म् ब्रह्म॑णा । \newline
25. मद्ध्य॒म् ब्रह्म॑णा॒ ब्रह्म॑णा॒ मद्ध्य॒म् मद्ध्य॒म् ब्रह्म॑णा॒ वै वै ब्रह्म॑णा॒ मद्ध्य॒म् मद्ध्य॒म् ब्रह्म॑णा॒ वै । \newline
26. ब्रह्म॑णा॒ वै वै ब्रह्म॑णा॒ ब्रह्म॑णा॒ वा अ॑स्यास्य॒ वै ब्रह्म॑णा॒ ब्रह्म॑णा॒ वा अ॑स्य । \newline
27. वा अ॑स्यास्य॒ वै वा अ॑स्य॒ तत् तद॑स्य॒ वै वा अ॑स्य॒ तत् । \newline
28. अ॒स्य॒ तत् तद॑ स्यास्य॒ तत् पू᳚र्वा॒र्द्धम् पू᳚र्वा॒र्द्धम् तद॑ स्यास्य॒ तत् पू᳚र्वा॒र्द्धम् । \newline
29. तत् पू᳚र्वा॒र्द्धम् पू᳚र्वा॒र्द्धम् तत् तत् पू᳚र्वा॒र्द्ध म॑चिनुता चिनुत पूर्वा॒र्द्धम् तत् तत् पू᳚र्वा॒र्द्ध म॑चिनुत । \newline
30. पू॒र्वा॒र्द्ध म॑चिनुता चिनुत पूर्वा॒र्द्धम् पू᳚र्वा॒र्द्ध म॑चिनुत क्ष॒त्रेण॑ क्ष॒त्रेणा॑ चिनुत पूर्वा॒र्द्धम् पू᳚र्वा॒र्द्ध म॑चिनुत क्ष॒त्रेण॑ । \newline
31. पू॒र्वा॒र्द्धमिति॑ पूर्व - अ॒र्द्धम् । \newline
32. अ॒चि॒नु॒त॒ क्ष॒त्रेण॑ क्ष॒त्रेणा॑ चिनुता चिनुत क्ष॒त्रेण॒ दक्षि॑ण॒म् दक्षि॑णम् क्ष॒त्रेणा॑ चिनुता चिनुत क्ष॒त्रेण॒ दक्षि॑णम् । \newline
33. क्ष॒त्रेण॒ दक्षि॑ण॒म् दक्षि॑णम् क्ष॒त्रेण॑ क्ष॒त्रेण॒ दक्षि॑णम् प॒क्षम् प॒क्षम् दक्षि॑णम् क्ष॒त्रेण॑ क्ष॒त्रेण॒ दक्षि॑णम् प॒क्षम् । \newline
34. दक्षि॑णम् प॒क्षम् प॒क्षम् दक्षि॑ण॒म् दक्षि॑णम् प॒क्षम् प॒शुभिः॑ प॒शुभिः॑ प॒क्षम् दक्षि॑ण॒म् दक्षि॑णम् प॒क्षम् प॒शुभिः॑ । \newline
35. प॒क्षम् प॒शुभिः॑ प॒शुभिः॑ प॒क्षम् प॒क्षम् प॒शुभिः॒ पुच्छ॒म् पुच्छ॑म् प॒शुभिः॑ प॒क्षम् प॒क्षम् प॒शुभिः॒ पुच्छ᳚म् । \newline
36. प॒शुभिः॒ पुच्छ॒म् पुच्छ॑म् प॒शुभिः॑ प॒शुभिः॒ पुच्छं॑ ॅवि॒शा वि॒शा पुच्छ॑म् प॒शुभिः॑ प॒शुभिः॒ पुच्छं॑ ॅवि॒शा । \newline
37. प॒शुभि॒रिति॑ प॒शु - भिः॒ । \newline
38. पुच्छं॑ ॅवि॒शा वि॒शा पुच्छ॒म् पुच्छं॑ ॅवि॒शोत्त॑र॒ मुत्त॑रं ॅवि॒शा पुच्छ॒म् पुच्छं॑ ॅवि॒शोत्त॑रम् । \newline
39. वि॒शोत्त॑र॒ मुत्त॑रं ॅवि॒शा वि॒शोत्त॑रम् प॒क्षम् प॒क्ष मुत्त॑रं ॅवि॒शा वि॒शोत्त॑रम् प॒क्षम् । \newline
40. उत्त॑रम् प॒क्षम् प॒क्ष मुत्त॑र॒ मुत्त॑रम् प॒क्ष मा॒शया॒ ऽऽशया॑ प॒क्ष मुत्त॑र॒ मुत्त॑रम् प॒क्ष मा॒शया᳚ । \newline
41. उत्त॑र॒मित्युत् - त॒र॒म् । \newline
42. प॒क्ष मा॒शया॒ ऽऽशया॑ प॒क्षम् प॒क्ष मा॒शया॒ मद्ध्य॒म् मद्ध्य॑ मा॒शया॑ प॒क्षम् प॒क्ष मा॒शया॒ मद्ध्य᳚म् । \newline
43. आ॒शया॒ मद्ध्य॒म् मद्ध्य॑ मा॒शया॒ ऽऽशया॒ मद्ध्यं॒ ॅयो यो मद्ध्य॑ मा॒शया॒ ऽऽशया॒ मद्ध्यं॒ ॅयः । \newline
44. मद्ध्यं॒ ॅयो यो मद्ध्य॒म् मद्ध्यं॒ ॅय ए॒व मे॒वं ॅयो मद्ध्य॒म् मद्ध्यं॒ ॅय ए॒वम् । \newline
45. य ए॒व मे॒वं ॅयो य ए॒वं ॅवि॒द्वान्. वि॒द्वा ने॒वं ॅयो य ए॒वं ॅवि॒द्वान् । \newline
46. ए॒वं ॅवि॒द्वान्. वि॒द्वा ने॒व मे॒वं ॅवि॒द्वा न॒ग्नि म॒ग्निं ॅवि॒द्वा ने॒व मे॒वं ॅवि॒द्वा न॒ग्निम् । \newline
47. वि॒द्वा न॒ग्नि म॒ग्निं ॅवि॒द्वान्. वि॒द्वा न॒ग्निम् चि॑नु॒ते चि॑नु॒ते᳚ ऽग्निं ॅवि॒द्वान्. वि॒द्वा न॒ग्निम् चि॑नु॒ते । \newline
48. अ॒ग्निम् चि॑नु॒ते चि॑नु॒ते᳚ ऽग्नि म॒ग्निम् चि॑नु॒त ऋ॒तुभिर्॑. ऋ॒तुभि॑ श्चिनु॒ते᳚ ऽग्नि म॒ग्निम् चि॑नु॒त ऋ॒तुभिः॑ । \newline
49. चि॒नु॒त ऋ॒तुभिर्॑. ऋ॒तुभि॑ श्चिनु॒ते चि॑नु॒त ऋ॒तुभि॑ रे॒वैव र्‌तुभि॑ श्चिनु॒ते चि॑नु॒त ऋ॒तुभि॑ रे॒व । \newline
50. ऋ॒तुभि॑ रे॒वैव र्‌तुभिर्॑. ऋ॒तुभि॑ रे॒वैन॑ मेन मे॒व र्‌तुभिर्॑. ऋ॒तुभि॑ रे॒वैन᳚म् । \newline
51. ऋ॒तुभि॒रित्यृ॒तु - भिः॒ । \newline
52. ए॒वैन॑ मेन मे॒वै वैन॑म् चिनुते चिनुत एन मे॒वै वैन॑म् चिनुते । \newline
53. ए॒न॒म् चि॒नु॒ते॒ चि॒नु॒त॒ ए॒न॒ मे॒न॒म् चि॒नु॒ते ऽथो॒ अथो॑ चिनुत एन मेनम् चिनु॒ते ऽथो᳚ । \newline
54. चि॒नु॒ते ऽथो॒ अथो॑ चिनुते चिनु॒ते ऽथो॑ ए॒त दे॒त दथो॑ चिनुते चिनु॒ते ऽथो॑ ए॒तत् । \newline
55. अथो॑ ए॒त दे॒त दथो॒ अथो॑ ए॒त दे॒वैवै तदथो॒ अथो॑ ए॒त दे॒व । \newline
56. अथो॒ इत्यथो᳚ । \newline
57. ए॒तदे॒वै वैत दे॒त दे॒व सर्वꣳ॒॒ सर्व॑ मे॒वैत दे॒त दे॒व सर्व᳚म् । \newline
58. ए॒व सर्वꣳ॒॒ सर्व॑ मे॒वैव सर्व॒ मवाव॒ सर्व॑ मे॒वैव सर्व॒ मव॑ । \newline
59. सर्व॒ मवाव॒ सर्वꣳ॒॒ सर्व॒ मव॑ रुन्धे रु॒न्धे ऽव॒ सर्वꣳ॒॒ सर्व॒ मव॑ रुन्धे । \newline
60. अव॑ रुन्धे रु॒न्धे ऽवाव॑ रुन्धे शृ॒ण्वन्ति॑ शृ॒ण्वन्ति॑ रु॒न्धे ऽवाव॑ रुन्धे शृ॒ण्वन्ति॑ । \newline
\pagebreak
\markright{ TS 5.6.10.2  \hfill https://www.vedavms.in \hfill}

\section{ TS 5.6.10.2 }

\textbf{TS 5.6.10.2 } \newline
\textbf{Samhita Paata} \newline

रुन्धे शृ॒ण्वन्त्ये॑नम॒ग्निं चि॑क्या॒नमत्त्यन्नꣳ॒॒ रोच॑त इ॒यं ॅवाव प्र॑थ॒मा चिति॒रोष॑धयो॒ वन॒स्पत॑यः॒ पुरी॑षम॒न्तरि॑क्षं द्वि॒तीया॒ वयाꣳ॑सि॒ पुरी॑षम॒सौ तृ॒तीया॒ नक्ष॑त्राणि॒ पुरी॑षं ॅय॒ज्ञ्श्च॑तु॒र्थी दक्षि॑णा॒ पुरी॑षं॒ ॅयज॑मानः पञ्च॒मी प्र॒जा पुरी॑षं॒ ॅयत् त्रिचि॑तीकं चिन्वी॒त य॒ज्ञ्ं दक्षि॑णामा॒त्मानं॑ प्र॒जाम॒न्तरि॑या॒त् तस्मा॒त् पञ्च॑चितीकश्चेत॒व्य॑ ए॒तदे॒व सर्वꣳ॑ स्पृणोति॒ यत् ति॒स्रश्चित॑य - [  ] \newline

\textbf{Pada Paata} \newline

रु॒न्धे॒ । शृ॒ण्वन्ति॑ । ए॒न॒म् । अ॒ग्निम् । चि॒क्या॒नम् । अत्ति॑ । अन्न᳚म् । रोच॑ते । इ॒यम् । वाव । प्र॒थ॒मा । चितिः॑ । ओष॑धयः । वन॒स्पत॑यः । पुरी॑षम् । अ॒न्तरि॑क्षम् । द्वि॒तीया᳚ । वयाꣳ॑सि । पुरी॑षम् । अ॒सौ । तृ॒तीया᳚ । नक्ष॑त्राणि । पुरी॑षम् । य॒ज्ञ्ः । च॒तु॒र्थी । दक्षि॑णा । पुरी॑षम् । यज॑मानः । प॒ञ्च॒मी । प्र॒जेति॑ प्र - जा । पुरी॑षम् । यत् । त्रिचि॑तीक॒मिति॒ त्रि - चि॒ती॒क॒म् । चि॒न्वी॒त । य॒ज्ञ्म् । दक्षि॑णाम् । आ॒त्मान᳚म् । प्र॒जामिति॑ प्र - जाम् । अ॒न्तः । इ॒या॒त् । तस्मा᳚त् । पञ्च॑चितीक॒ इति॒ पञ्च॑ - चि॒ती॒कः॒ । चे॒त॒व्यः॑ । ए॒तत् । ए॒व । सर्व᳚म् । स्पृ॒णो॒ति॒ । यत् । ति॒स्रः । चित॑यः ।  \newline


\textbf{Krama Paata} \newline

रु॒न्धे॒ शृ॒ण्वन्ति॑ । शृ॒ण्वन्त्ये॑नम् । ए॒न॒म॒ग्निम् । अ॒ग्निम् चि॑क्या॒नम् । चि॒क्या॒नमत्ति॑ । अत्त्यन्न᳚म् । अन्नꣳ॒॒ रोच॑ते । रोच॑त इ॒यम् । इ॒यम् ॅवाव । वाव प्र॑थ॒मा । प्र॒थ॒मा चितिः॑ । चिति॒रोष॑धयः । ओष॑धयो॒ वन॒स्पत॑यः । वन॒स्पत॑यः॒ पुरी॑षम् । पुरी॑षम॒न्तरि॑क्षम् । अ॒न्तरि॑क्षम् द्वि॒तीया᳚ । द्वि॒तीया॒ वयाꣳ॑सि । वयाꣳ॑सि॒ पुरी॑षम् । पुरी॑षम॒सौ । अ॒सौ तृ॒तीया᳚ । तृ॒तीया॒ नक्ष॑त्राणि । नक्ष॑त्राणि॒ पुरी॑षम् । पुरी॑षम् ॅय॒ज्ञ्ः । य॒ज्ञ्श्च॑तु॒र्त्थी । च॒तु॒र्त्थी दक्षि॑णा । दक्षि॑णा॒ पुरी॑षम् । पुरी॑ष॒म् ॅयज॑मानः । यज॑मानः पञ्च॒मी । प॒ञ्च॒मी प्र॒जा । प्र॒जा पुरी॑षम् । प्र॒जेति॑ प्र - जा । पुरी॑ष॒म् ॅयत् । यत् त्रिचि॑तीकम् । त्रिचि॑तीकम् चिन्वी॒त । त्रिचि॑तीक॒मिति॒ त्रि - चि॒ती॒क॒म् । चि॒न्वी॒त य॒ज्ञ्म् । य॒ज्ञ्म् दक्षि॑णाम् । दक्षि॑णामा॒त्मान᳚म् । आ॒त्मान॑म् प्र॒जाम् । प्र॒जाम॒न्तः । प्र॒जामिति॑ प्र - जाम् । अ॒न्तरि॑यात् । इ॒या॒त् तस्मा᳚त् । तस्मा॒त् पञ्च॑चितीकः । पञ्च॑चितीकश्चेत॒व्यः॑ । पञ्च॑चितीक॒ इति॒ पञ्च॑ - चि॒ती॒कः॒ । चे॒त॒व्य॑ ए॒तत् । ए॒तदे॒व । ए॒व सर्व᳚म् । सर्वꣳ॑ स्पृणोति । स्पृ॒णो॒ति॒ यत् । यत् ति॒स्रः । ति॒स्रश्चित॑यः । चित॑यस्त्रि॒वृत् \newline

\textbf{Jatai Paata} \newline

1. रु॒न्धे॒ शृ॒ण्वन्ति॑ शृ॒ण्वन्ति॑ रुन्धे रुन्धे शृ॒ण्वन्ति॑ । \newline
2. शृ॒ण्वन् त्ये॑न मेनꣳ शृ॒ण्वन्ति॑ शृ॒ण्वन् त्ये॑नम् । \newline
3. ए॒न॒ म॒ग्नि म॒ग्नि मे॑न मेन म॒ग्निम् । \newline
4. अ॒ग्निम् चि॑क्या॒नम् चि॑क्या॒न म॒ग्नि म॒ग्निम् चि॑क्या॒नम् । \newline
5. चि॒क्या॒न मत्त्यत्ति॑ चिक्या॒नम् चि॑क्या॒न मत्ति॑ । \newline
6. अत्त्यन्न॒ मन्न॒ मत्त्य त्त्यन्न᳚म् । \newline
7. अन्नꣳ॒॒ रोच॑ते॒ रोच॒ते ऽन्न॒ मन्नꣳ॒॒ रोच॑ते । \newline
8. रोच॑त इ॒य मि॒यꣳ रोच॑ते॒ रोच॑त इ॒यम् । \newline
9. इ॒यं ॅवाव वावेय मि॒यं ॅवाव । \newline
10. वाव प्र॑थ॒मा प्र॑थ॒मा वाव वाव प्र॑थ॒मा । \newline
11. प्र॒थ॒मा चिति॒ श्चितिः॑ प्रथ॒मा प्र॑थ॒मा चितिः॑ । \newline
12. चिति॒ रोष॑धय॒ ओष॑धय॒ श्चिति॒ श्चिति॒ रोष॑धयः । \newline
13. ओष॑धयो॒ वन॒स्पत॑यो॒ वन॒स्पत॑य॒ ओष॑धय॒ ओष॑धयो॒ वन॒स्पत॑यः । \newline
14. वन॒स्पत॑यः॒ पुरी॑ष॒म् पुरी॑षं॒ ॅवन॒स्पत॑यो॒ वन॒स्पत॑यः॒ पुरी॑षम् । \newline
15. पुरी॑ष म॒न्तरि॑क्ष म॒न्तरि॑क्ष॒म् पुरी॑ष॒म् पुरी॑ष म॒न्तरि॑क्षम् । \newline
16. अ॒न्तरि॑क्षम् द्वि॒तीया᳚ द्वि॒तीया॒ ऽन्तरि॑क्ष म॒न्तरि॑क्षम् द्वि॒तीया᳚ । \newline
17. द्वि॒तीया॒ वयाꣳ॑सि॒ वयाꣳ॑सि द्वि॒तीया᳚ द्वि॒तीया॒ वयाꣳ॑सि । \newline
18. वयाꣳ॑सि॒ पुरी॑ष॒म् पुरी॑षं॒ ॅवयाꣳ॑सि॒ वयाꣳ॑सि॒ पुरी॑षम् । \newline
19. पुरी॑ष म॒सा व॒सौ पुरी॑ष॒म् पुरी॑ष म॒सौ । \newline
20. अ॒सौ तृ॒तीया॑ तृ॒तीया॒ ऽसा व॒सौ तृ॒तीया᳚ । \newline
21. तृ॒तीया॒ नक्ष॑त्राणि॒ नक्ष॑त्राणि तृ॒तीया॑ तृ॒तीया॒ नक्ष॑त्राणि । \newline
22. नक्ष॑त्राणि॒ पुरी॑ष॒म् पुरी॑ष॒न् नक्ष॑त्राणि॒ नक्ष॑त्राणि॒ पुरी॑षम् । \newline
23. पुरी॑षं ॅय॒ज्ञो य॒ज्ञ्ः पुरी॑ष॒म् पुरी॑षं ॅय॒ज्ञ्ः । \newline
24. य॒ज्ञ् श्च॑तु॒र्थी च॑तु॒र्थी य॒ज्ञो य॒ज्ञ् श्च॑तु॒र्थी । \newline
25. च॒तु॒र्थी दक्षि॑णा॒ दक्षि॑णा चतु॒र्थी च॑तु॒र्थी दक्षि॑णा । \newline
26. दक्षि॑णा॒ पुरी॑ष॒म् पुरी॑ष॒म् दक्षि॑णा॒ दक्षि॑णा॒ पुरी॑षम् । \newline
27. पुरी॑षं॒ ॅयज॑मानो॒ यज॑मानः॒ पुरी॑ष॒म् पुरी॑षं॒ ॅयज॑मानः । \newline
28. यज॑मानः पञ्च॒मी प॑ञ्च॒मी यज॑मानो॒ यज॑मानः पञ्च॒मी । \newline
29. प॒ञ्च॒मी प्र॒जा प्र॒जा प॑ञ्च॒मी प॑ञ्च॒मी प्र॒जा । \newline
30. प्र॒जा पुरी॑ष॒म् पुरी॑षम् प्र॒जा प्र॒जा पुरी॑षम् । \newline
31. प्र॒जेति॑ प्र - जा । \newline
32. पुरी॑षं॒ ॅयद् यत् पुरी॑ष॒म् पुरी॑षं॒ ॅयत् । \newline
33. यत् त्रिचि॑तीक॒म् त्रिचि॑तीकं॒ ॅयद् यत् त्रिचि॑तीकम् । \newline
34. त्रिचि॑तीकम् चिन्वी॒त चि॑न्वी॒त त्रिचि॑तीक॒म् त्रिचि॑तीक॒म् चिन्वी॒त । \newline
35. त्रिचि॑तीक॒मिति॒ त्रि - चि॒ती॒क॒म् । \newline
36. चि॒न्वी॒त य॒ज्ञ्ं ॅय॒ज्ञ्म् चि॑न्वी॒त चि॑न्वी॒त य॒ज्ञ्म् । \newline
37. य॒ज्ञ्म् दक्षि॑णा॒म् दक्षि॑णां ॅय॒ज्ञ्ं ॅय॒ज्ञ्म् दक्षि॑णाम् । \newline
38. दक्षि॑णा मा॒त्मान॑ मा॒त्मान॒म् दक्षि॑णा॒म् दक्षि॑णा मा॒त्मान᳚म् । \newline
39. आ॒त्मान॑म् प्र॒जाम् प्र॒जा मा॒त्मान॑ मा॒त्मान॑म् प्र॒जाम् । \newline
40. प्र॒जा म॒न्त र॒न्तः प्र॒जाम् प्र॒जा म॒न्तः । \newline
41. प्र॒जामिति॑ प्र - जाम् । \newline
42. अ॒न्त रि॑या दिया द॒न्त र॒न्त रि॑यात् । \newline
43. इ॒या॒त् तस्मा॒त् तस्मा॑ दिया दिया॒त् तस्मा᳚त् । \newline
44. तस्मा॒त् पञ्च॑चितीकः॒ पञ्च॑चितीक॒ स्तस्मा॒त् तस्मा॒त् पञ्च॑चितीकः । \newline
45. पञ्च॑चितीक श्चेत॒व्य॑ श्चेत॒व्यः॑ पञ्च॑चितीकः॒ पञ्च॑चितीक श्चेत॒व्यः॑ । \newline
46. पञ्च॑चितीक॒ इति॒ पञ्च॑ - चि॒ती॒कः॒ । \newline
47. चे॒त॒व्य॑ ए॒त दे॒तच् चे॑त॒व्य॑ श्चेत॒व्य॑ ए॒तत् । \newline
48. ए॒त दे॒वै वैत दे॒त दे॒व । \newline
49. ए॒व सर्वꣳ॒॒ सर्व॑ मे॒वैव सर्व᳚म् । \newline
50. सर्वꣳ॑ स्पृणोति स्पृणोति॒ सर्वꣳ॒॒ सर्वꣳ॑ स्पृणोति । \newline
51. स्पृ॒णो॒ति॒ यद् यथ् स्पृ॑णोति स्पृणोति॒ यत् । \newline
52. यत् ति॒स्र स्ति॒स्रो यद् यत् ति॒स्रः । \newline
53. ति॒स्र श्चित॑य॒ श्चित॑य स्ति॒स्र स्ति॒स्र श्चित॑यः । \newline
54. चित॑य स्त्रि॒वृत् त्रि॒वृच् चित॑य॒ श्चित॑य स्त्रि॒वृत् । \newline

\textbf{Ghana Paata } \newline

1. रु॒न्धे॒ शृ॒ण्वन्ति॑ शृ॒ण्वन्ति॑ रुन्धे रुन्धे शृ॒ण्वन् त्ये॑न मेनꣳ शृ॒ण्वन्ति॑ रुन्धे रुन्धे शृ॒ण्वन् त्ये॑नम् । \newline
2. शृ॒ण्वन् त्ये॑न मेनꣳ शृ॒ण्वन्ति॑ शृ॒ण्वन् त्ये॑न म॒ग्नि म॒ग्नि मे॑नꣳ शृ॒ण्वन्ति॑ शृ॒ण्वन् त्ये॑न म॒ग्निम् । \newline
3. ए॒न॒ म॒ग्नि म॒ग्नि मे॑न मेन म॒ग्निम् चि॑क्या॒नम् चि॑क्या॒न म॒ग्नि मे॑न मेन म॒ग्निम् चि॑क्या॒नम् । \newline
4. अ॒ग्निम् चि॑क्या॒नम् चि॑क्या॒न म॒ग्नि म॒ग्निम् चि॑क्या॒न मत्त्यत्ति॑ चिक्या॒न म॒ग्नि म॒ग्निम् चि॑क्या॒न मत्ति॑ । \newline
5. चि॒क्या॒न मत्त्यत्ति॑ चिक्या॒नम् चि॑क्या॒न मत्त्यन्न॒ मन्न॒ मत्ति॑ चिक्या॒नम् चि॑क्या॒न मत्त्यन्न᳚म् । \newline
6. अत्त्यन्न॒ मन्न॒ मत्त्यत् त्यन्नꣳ॒॒ रोच॑ते॒ रोच॒ते ऽन्न॒ मत्त्यत् त्यन्नꣳ॒॒ रोच॑ते । \newline
7. अन्नꣳ॒॒ रोच॑ते॒ रोच॒ते ऽन्न॒ मन्नꣳ॒॒ रोच॑त इ॒य मि॒यꣳ रोच॒ते ऽन्न॒ मन्नꣳ॒॒ रोच॑त इ॒यम् । \newline
8. रोच॑त इ॒य मि॒यꣳ रोच॑ते॒ रोच॑त इ॒यं ॅवाव वावेयꣳ रोच॑ते॒ रोच॑त इ॒यं ॅवाव । \newline
9. इ॒यं ॅवाव वावेय मि॒यं ॅवाव प्र॑थ॒मा प्र॑थ॒मा वावेय मि॒यं ॅवाव प्र॑थ॒मा । \newline
10. वाव प्र॑थ॒मा प्र॑थ॒मा वाव वाव प्र॑थ॒मा चिति॒ श्चितिः॑ प्रथ॒मा वाव वाव प्र॑थ॒मा चितिः॑ । \newline
11. प्र॒थ॒मा चिति॒ श्चितिः॑ प्रथ॒मा प्र॑थ॒मा चिति॒ रोष॑धय॒ ओष॑धय॒ श्चितिः॑ प्रथ॒मा प्र॑थ॒मा चिति॒ रोष॑धयः । \newline
12. चिति॒ रोष॑धय॒ ओष॑धय॒ श्चिति॒ श्चिति॒ रोष॑धयो॒ वन॒स्पत॑यो॒ वन॒स्पत॑य॒ ओष॑धय॒ श्चिति॒ श्चिति॒ रोष॑धयो॒ वन॒स्पत॑यः । \newline
13. ओष॑धयो॒ वन॒स्पत॑यो॒ वन॒स्पत॑य॒ ओष॑धय॒ ओष॑धयो॒ वन॒स्पत॑यः॒ पुरी॑ष॒म् पुरी॑षं॒ ॅवन॒स्पत॑य॒ ओष॑धय॒ ओष॑धयो॒ वन॒स्पत॑यः॒ पुरी॑षम् । \newline
14. वन॒स्पत॑यः॒ पुरी॑ष॒म् पुरी॑षं॒ ॅवन॒स्पत॑यो॒ वन॒स्पत॑यः॒ पुरी॑ष म॒न्तरि॑क्ष म॒न्तरि॑क्ष॒म् पुरी॑षं॒ ॅवन॒स्पत॑यो॒ वन॒स्पत॑यः॒ पुरी॑ष म॒न्तरि॑क्षम् । \newline
15. पुरी॑ष म॒न्तरि॑क्ष म॒न्तरि॑क्ष॒म् पुरी॑ष॒म् पुरी॑ष म॒न्तरि॑क्षम् द्वि॒तीया᳚ द्वि॒तीया॒ ऽन्तरि॑क्ष॒म् पुरी॑ष॒म् पुरी॑ष म॒न्तरि॑क्षम् द्वि॒तीया᳚ । \newline
16. अ॒न्तरि॑क्षम् द्वि॒तीया᳚ द्वि॒तीया॒ ऽन्तरि॑क्ष म॒न्तरि॑क्षम् द्वि॒तीया॒ वयाꣳ॑सि॒ वयाꣳ॑सि द्वि॒तीया॒ ऽन्तरि॑क्ष म॒न्तरि॑क्षम् द्वि॒तीया॒ वयाꣳ॑सि । \newline
17. द्वि॒तीया॒ वयाꣳ॑सि॒ वयाꣳ॑सि द्वि॒तीया᳚ द्वि॒तीया॒ वयाꣳ॑सि॒ पुरी॑ष॒म् पुरी॑षं॒ ॅवयाꣳ॑सि द्वि॒तीया᳚ द्वि॒तीया॒ वयाꣳ॑सि॒ पुरी॑षम् । \newline
18. वयाꣳ॑सि॒ पुरी॑ष॒म् पुरी॑षं॒ ॅवयाꣳ॑सि॒ वयाꣳ॑सि॒ पुरी॑ष म॒सा व॒सौ पुरी॑षं॒ ॅवयाꣳ॑सि॒ वयाꣳ॑सि॒ पुरी॑ष म॒सौ । \newline
19. पुरी॑ष म॒सा व॒सौ पुरी॑ष॒म् पुरी॑ष म॒सौ तृ॒तीया॑ तृ॒तीया॒ ऽसौ पुरी॑ष॒म् पुरी॑ष म॒सौ तृ॒तीया᳚ । \newline
20. अ॒सौ तृ॒तीया॑ तृ॒तीया॒ ऽसा व॒सौ तृ॒तीया॒ नक्ष॑त्राणि॒ नक्ष॑त्राणि तृ॒तीया॒ ऽसा व॒सौ तृ॒तीया॒ नक्ष॑त्राणि । \newline
21. तृ॒तीया॒ नक्ष॑त्राणि॒ नक्ष॑त्राणि तृ॒तीया॑ तृ॒तीया॒ नक्ष॑त्राणि॒ पुरी॑ष॒म् पुरी॑ष॒म् नक्ष॑त्राणि तृ॒तीया॑ तृ॒तीया॒ नक्ष॑त्राणि॒ पुरी॑षम् । \newline
22. नक्ष॑त्राणि॒ पुरी॑ष॒म् पुरी॑ष॒म् नक्ष॑त्राणि॒ नक्ष॑त्राणि॒ पुरी॑षं ॅय॒ज्ञो य॒ज्ञ्ः पुरी॑ष॒म् नक्ष॑त्राणि॒ नक्ष॑त्राणि॒ पुरी॑षं ॅय॒ज्ञ्ः । \newline
23. पुरी॑षं ॅय॒ज्ञो य॒ज्ञ्ः पुरी॑ष॒म् पुरी॑षं ॅय॒ज्ञ् श्च॑तु॒र्थी च॑तु॒र्थी य॒ज्ञ्ः पुरी॑ष॒म् पुरी॑षं ॅय॒ज्ञ् श्च॑तु॒र्थी । \newline
24. य॒ज्ञ् श्च॑तु॒र्थी च॑तु॒र्थी य॒ज्ञो य॒ज्ञ् श्च॑तु॒र्थी दक्षि॑णा॒ दक्षि॑णा चतु॒र्थी य॒ज्ञो य॒ज्ञ् श्च॑तु॒र्थी दक्षि॑णा । \newline
25. च॒तु॒र्थी दक्षि॑णा॒ दक्षि॑णा चतु॒र्थी च॑तु॒र्थी दक्षि॑णा॒ पुरी॑ष॒म् पुरी॑ष॒म् दक्षि॑णा चतु॒र्थी च॑तु॒र्थी दक्षि॑णा॒ पुरी॑षम् । \newline
26. दक्षि॑णा॒ पुरी॑ष॒म् पुरी॑ष॒म् दक्षि॑णा॒ दक्षि॑णा॒ पुरी॑षं॒ ॅयज॑मानो॒ यज॑मानः॒ पुरी॑ष॒म् दक्षि॑णा॒ दक्षि॑णा॒ पुरी॑षं॒ ॅयज॑मानः । \newline
27. पुरी॑षं॒ ॅयज॑मानो॒ यज॑मानः॒ पुरी॑ष॒म् पुरी॑षं॒ ॅयज॑मानः पञ्च॒मी प॑ञ्च॒मी यज॑मानः॒ पुरी॑ष॒म् पुरी॑षं॒ ॅयज॑मानः पञ्च॒मी । \newline
28. यज॑मानः पञ्च॒मी प॑ञ्च॒मी यज॑मानो॒ यज॑मानः पञ्च॒मी प्र॒जा प्र॒जा प॑ञ्च॒मी यज॑मानो॒ यज॑मानः पञ्च॒मी प्र॒जा । \newline
29. प॒ञ्च॒मी प्र॒जा प्र॒जा प॑ञ्च॒मी प॑ञ्च॒मी प्र॒जा पुरी॑ष॒म् पुरी॑षम् प्र॒जा प॑ञ्च॒मी प॑ञ्च॒मी प्र॒जा पुरी॑षम् । \newline
30. प्र॒जा पुरी॑ष॒म् पुरी॑षम् प्र॒जा प्र॒जा पुरी॑षं॒ ॅयद् यत् पुरी॑षम् प्र॒जा प्र॒जा पुरी॑षं॒ ॅयत् । \newline
31. प्र॒जेति॑ प्र - जा । \newline
32. पुरी॑षं॒ ॅयद् यत् पुरी॑ष॒म् पुरी॑षं॒ ॅयत् त्रिचि॑तीक॒म् त्रिचि॑तीकं॒ ॅयत् पुरी॑ष॒म् पुरी॑षं॒ ॅयत् त्रिचि॑तीकम् । \newline
33. यत् त्रिचि॑तीक॒म् त्रिचि॑तीकं॒ ॅयद् यत् त्रिचि॑तीकम् चिन्वी॒त चि॑न्वी॒त त्रिचि॑तीकं॒ ॅयद् यत् त्रिचि॑तीकम् चिन्वी॒त । \newline
34. त्रिचि॑तीकम् चिन्वी॒त चि॑न्वी॒त त्रिचि॑तीक॒म् त्रिचि॑तीकम् चिन्वी॒त य॒ज्ञ्ं ॅय॒ज्ञ्म् चि॑न्वी॒त त्रिचि॑तीक॒म् त्रिचि॑तीकम् चिन्वी॒त य॒ज्ञ्म् । \newline
35. त्रिचि॑तीक॒मिति॒ त्रि - चि॒ती॒क॒म् । \newline
36. चि॒न्वी॒त य॒ज्ञ्ं ॅय॒ज्ञ्म् चि॑न्वी॒त चि॑न्वी॒त य॒ज्ञ्म् दक्षि॑णा॒म् दक्षि॑णां ॅय॒ज्ञ्म् चि॑न्वी॒त चि॑न्वी॒त य॒ज्ञ्म् दक्षि॑णाम् । \newline
37. य॒ज्ञ्म् दक्षि॑णा॒म् दक्षि॑णां ॅय॒ज्ञ्ं ॅय॒ज्ञ्म् दक्षि॑णा मा॒त्मान॑ मा॒त्मान॒म् दक्षि॑णां ॅय॒ज्ञ्ं ॅय॒ज्ञ्म् दक्षि॑णा मा॒त्मान᳚म् । \newline
38. दक्षि॑णा मा॒त्मान॑ मा॒त्मान॒म् दक्षि॑णा॒म् दक्षि॑णा मा॒त्मान॑म् प्र॒जाम् प्र॒जा मा॒त्मान॒म् दक्षि॑णा॒म् दक्षि॑णा मा॒त्मान॑म् प्र॒जाम् । \newline
39. आ॒त्मान॑म् प्र॒जाम् प्र॒जा मा॒त्मान॑ मा॒त्मान॑म् प्र॒जा म॒न्त र॒न्तः प्र॒जा मा॒त्मान॑ मा॒त्मान॑म् प्र॒जा म॒न्तः । \newline
40. प्र॒जा म॒न्त र॒न्तः प्र॒जाम् प्र॒जा म॒न्त रि॑या दिया द॒न्तः प्र॒जाम् प्र॒जा म॒न्त रि॑यात् । \newline
41. प्र॒जामिति॑ प्र - जाम् । \newline
42. अ॒न्त रि॑या दिया द॒न्त र॒न्त रि॑या॒त् तस्मा॒त् तस्मा॑ दिया द॒न्त र॒न्त रि॑या॒त् तस्मा᳚त् । \newline
43. इ॒या॒त् तस्मा॒त् तस्मा॑ दिया दिया॒त् तस्मा॒त् पञ्च॑चितीकः॒ पञ्च॑चितीक॒ स्तस्मा॑ दिया दिया॒त् तस्मा॒त् पञ्च॑चितीकः । \newline
44. तस्मा॒त् पञ्च॑चितीकः॒ पञ्च॑चितीक॒ स्तस्मा॒त् तस्मा॒त् पञ्च॑चितीक श्चेत॒व्य॑ श्चेत॒व्यः॑ पञ्च॑चितीक॒ स्तस्मा॒त् तस्मा॒त् पञ्च॑चितीक श्चेत॒व्यः॑ । \newline
45. पञ्च॑चितीक श्चेत॒व्य॑ श्चेत॒व्यः॑ पञ्च॑चितीकः॒ पञ्च॑चितीक श्चेत॒व्य॑ ए॒त दे॒तच् चे॑त॒व्यः॑ पञ्च॑चितीकः॒ पञ्च॑चितीक श्चेत॒व्य॑ ए॒तत् । \newline
46. पञ्च॑चितीक॒ इति॒ पञ्च॑ - चि॒ती॒कः॒ । \newline
47. चे॒त॒व्य॑ ए॒त दे॒तच् चे॑त॒व्य॑ श्चेत॒व्य॑ ए॒त दे॒वैवैतच् चे॑त॒व्य॑ श्चेत॒व्य॑ ए॒त दे॒व । \newline
48. ए॒तदे॒वैवै तदे॒ तदे॒व सर्वꣳ॒॒ सर्व॑ मे॒वै तदे॒ तदे॒व सर्व᳚म् । \newline
49. ए॒व सर्वꣳ॒॒ सर्व॑ मे॒वैव सर्वꣳ॑ स्पृणोति स्पृणोति॒ सर्व॑ मे॒वैव सर्वꣳ॑ स्पृणोति । \newline
50. सर्वꣳ॑ स्पृणोति स्पृणोति॒ सर्वꣳ॒॒ सर्वꣳ॑ स्पृणोति॒ यद् यथ् स्पृ॑णोति॒ सर्वꣳ॒॒ सर्वꣳ॑ स्पृणोति॒ यत् । \newline
51. स्पृ॒णो॒ति॒ यद् यथ् स्पृ॑णोति स्पृणोति॒ यत् ति॒स्र स्ति॒स्रो यथ् स्पृ॑णोति स्पृणोति॒ यत् ति॒स्रः । \newline
52. यत् ति॒स्र स्ति॒स्रो यद् यत् ति॒स्र श्चित॑य॒ श्चित॑य स्ति॒स्रो यद् यत् ति॒स्र श्चित॑यः । \newline
53. ति॒स्र श्चित॑य॒ श्चित॑य स्ति॒स्र स्ति॒स्र श्चित॑य स्त्रि॒वृत् त्रि॒वृच् चित॑य स्ति॒स्र स्ति॒स्र श्चित॑य स्त्रि॒वृत् । \newline
54. चित॑य स्त्रि॒वृत् त्रि॒वृच् चित॑य॒ श्चित॑य स्त्रि॒वृद्धि हि त्रि॒वृच् चित॑य॒ श्चित॑य स्त्रि॒वृद्धि । \newline
\pagebreak
\markright{ TS 5.6.10.3  \hfill https://www.vedavms.in \hfill}

\section{ TS 5.6.10.3 }

\textbf{TS 5.6.10.3 } \newline
\textbf{Samhita Paata} \newline

-स्त्रि॒वृद्ध्य॑ग्निर्यद् द्वे द्वि॒पाद्-यज॑मानः॒ प्रति॑ष्ठित्यै॒ पञ्च॒ चित॑यो भवन्ति॒ पाङ्क्तः॒ पुरु॑ष आ॒त्मान॑मे॒व स्पृ॑णोति॒ पञ्च॒ चित॑यो भवन्ति प॒ञ्चभिः॒ पुरी॑षैर॒भ्यू॑हति॒ दश॒ सं प॑द्यन्ते॒ दशा᳚क्षरो॒ वै पुरु॑षो॒ यावा॑ने॒व पुरु॑ष॒स्तꣳ स्पृ॑णो॒त्यथो॒ दशा᳚क्षरा वि॒राडन्नं॑ ॅवि॒राड् वि॒राज्ये॒वान्नाद्ये॒ प्रति॑ तिष्ठति संॅवथ्स॒रो वै ष॒ष्ठी चिति॑र्.ऋ॒तवः॒ पुरी॑षꣳ॒॒ ( ) षट् चित॑यो भवन्ति॒ षट् पुरी॑षाणि॒ द्वाद॑श॒ सं प॑द्यन्ते॒ द्वाद॑श॒ मासाः᳚ संॅवथ्स॒रः सं॑ॅवथ्स॒र ए॒व प्रति॑ तिष्ठति ॥ \newline

\textbf{Pada Paata} \newline

त्रि॒वृदिति॑ त्रि - वृत् । हि । अ॒ग्निः । यत् । द्वे इति॑ । द्वि॒पादिति॑ द्वि - पात् । यज॑मानः । प्रति॑ष्ठित्या॒ इति॒ प्रति॑ - स्थि॒त्यै॒ । पञ्च॑ । चित॑यः । भ॒व॒न्ति॒ । पाङ्क्तः॑ । पुरु॑षः । आ॒त्मान᳚म् । ए॒व । स्पृ॒णो॒ति॒ । पञ्च॑ । चित॑यः । भ॒व॒न्ति॒ । प॒ञ्चभि॒रिति॑ प॒ञ्च - भिः॒ । पुरी॑षैः । अ॒भीति॑ । ऊ॒ह॒ति॒ । दश॑ । समिति॑ । प॒द्य॒न्ते॒ । दशा᳚क्षर॒ इति॒ दश॑ - अ॒क्ष॒रः॒ । वै । पुरु॑षः । यावान्॑ । ए॒व । पुरु॑षः । तम् । स्पृ॒णो॒ति॒ । अथो॒ इति॑ । दशा᳚क्ष॒रेति॒ दश॑ - अ॒क्ष॒रा॒ । वि॒राडिति॑ वि - राट् । अन्न᳚म् । वि॒राडिति॑ वि - राट् । वि॒राजीति॑ वि - राजि॑ । ए॒व । अ॒न्नाद्य॒ इत्य॑न्न - अद्ये᳚ । प्रतीति॑ । ति॒ष्ठ॒ति॒ । सं॒ॅव॒थ्स॒र इति॑ सं - व॒थ्स॒रः । वै । ष॒ष्ठी । चितिः॑ । ऋ॒तवः॑ । पुरी॑षम् ( ) । षट् । चित॑यः । भ॒व॒न्ति॒ । षट् । पुरी॑षाणि । द्वाद॑श । समिति॑ । प॒द्य॒न्ते॒ । द्वाद॑श । मासाः᳚ । सं॒ॅव॒थ्स॒र इति॑ सं - व॒थ्स॒रः । सं॒ॅव॒थ्स॒र इति॑ सं - व॒थ्स॒रे । ए॒व । प्रतीति॑ । ति॒ष्ठ॒ति॒ ॥  \newline


\textbf{Krama Paata} \newline

त्रि॒वृद्धि । त्रि॒वृदिति॑ त्रि - वृत् । ह्य॑ग्निः । अ॒ग्निर् यत् । यद् द्वे । द्वे द्वि॒पात् । द्वे इति॒ द्वे । द्वि॒पाद् यज॑मानः । द्वि॒पादिति॑ द्वि - पात् । यज॑मानः॒ प्रति॑ष्ठित्यै । प्रति॑ष्ठित्यै॒ पञ्च॑ । प्रति॑ष्ठित्या॒ इति॒ प्रति॑ - स्थि॒त्यै॒ । पञ्च॒ चित॑यः । चित॑यो भवन्ति । भ॒व॒न्ति॒ पाङ्क्तः॑ । पाङ्क्तः॒ पुरु॑षः । पुरु॑ष आ॒त्मान᳚म् । आ॒त्मान॑मे॒व । ए॒व स्पृ॑णोति । स्पृ॒णो॒ति॒ पञ्च॑ । पञ्च॒ चित॑यः । चित॑यो भवन्ति । भ॒व॒न्ति॒ प॒ञ्चभिः॑ । प॒ञ्चभिः॒ पुरी॑षैः । प॒ञ्चभि॒रिति॑ प॒ञ्च - भिः॒ । पुरी॑षैर॒भि । अ॒भ्यू॑हति । ऊ॒ह॒ति॒ दश॑ । दश॒ सम् । सम् प॑द्यन्ते । प॒द्य॒न्ते॒ दशा᳚क्षरः । दशा᳚क्षरो॒ वै । दशा᳚क्षर॒ इति॒ दश॑ - अ॒क्ष॒रः॒ । वै पुरु॑षः । पुरु॑षो॒ यावान्॑ । यावा॑ने॒व । ए॒व पुरु॑षः । पुरु॑ष॒स्तम् । तꣳ स्पृ॑णोति । स्पृ॒णो॒त्यथो᳚ । अथो॒ दशा᳚क्षरा । अथो॒ इत्यथो᳚ । दशा᳚क्षरा वि॒राट् । दशा᳚क्ष॒रेति॒ दश॑ - अ॒क्ष॒रा॒ । वि॒राडन्न᳚म् । वि॒राडिति॑ वि - राट् । अन्न॑म् ॅवि॒राट् । वि॒राड् वि॒राजि॑ । वि॒राडिति॑ वि - राट् । वि॒राज्ये॒व । वि॒राजीति॑ वि - राजि॑ । ए॒वान्नाद्ये᳚ । अ॒न्नाद्ये॒ प्रति॑ । अ॒न्नाद्य॒ इत्य॑न्न - अद्ये᳚ । प्रति॑ तिष्ठति । ति॒ष्ठ॒ति॒ स॒म्ॅव॒थ्स॒रः । स॒म्ॅव॒थ्स॒रो वै । स॒म्ॅव॒थ्स॒र इति॑ सम् - व॒थ्स॒रः । वै ष॒ष्ठी । ष॒ष्ठी चितिः॑ । चिति॑र्. ऋ॒तवः॑ । ऋ॒तवः॒ पुरी॑षम् ( ) । पुरी॑षꣳ॒॒ षट् । षट् चित॑यः । चित॑यो भवन्ति । भ॒व॒न्ति॒ षट् । षट् पुरी॑षाणि । पुरी॑षाणि॒ द्वाद॑श । द्वाद॑श॒ सम् । सम् प॑द्यन्ते । प॒द्य॒न्ते॒ द्वाद॑श । द्वाद॑श॒ मासाः᳚ । मासाः᳚ सम्ॅवथ्स॒रः । स॒म्ॅव॒थ्स॒रः स॑म्ॅवथ्स॒रे । स॒म्ॅव॒थ्स॒र इति॑ सम् - व॒थ्स॒रः । स॒म्ॅव॒थ्स॒र ए॒व । स॒म्ॅव॒थ्स॒र इति॑ सम् - व॒थ्स॒रे । ए॒व प्रति॑ । प्रति॑ तिष्ठति । ति॒ष्ठ॒तीति॑ तिष्ठति । \newline

\textbf{Jatai Paata} \newline

1. त्रि॒वृद्धि हि त्रि॒वृत् त्रि॒वृद्धि । \newline
2. त्रि॒वृदिति॑ त्रि - वृत् । \newline
3. ह्य॑ग्नि र॒ग्निर्. हि ह्य॑ग्निः । \newline
4. अ॒ग्निर् यद् यद॒ग्नि र॒ग्निर् यत् । \newline
5. यद् द्वे द्वे यद् यद् द्वे । \newline
6. द्वे द्वि॒पाद् द्वि॒पाद् द्वे द्वे द्वि॒पात् । \newline
7. द्वे इति॒ द्वे । \newline
8. द्वि॒पाद् यज॑मानो॒ यज॑मानो द्वि॒पाद् द्वि॒पाद् यज॑मानः । \newline
9. द्वि॒पादिति॑ द्वि - पात् । \newline
10. यज॑मानः॒ प्रति॑ष्ठित्यै॒ प्रति॑ष्ठित्यै॒ यज॑मानो॒ यज॑मानः॒ प्रति॑ष्ठित्यै । \newline
11. प्रति॑ष्ठित्यै॒ पञ्च॒ पञ्च॒ प्रति॑ष्ठित्यै॒ प्रति॑ष्ठित्यै॒ पञ्च॑ । \newline
12. प्रति॑ष्ठित्या॒ इति॒ प्रति॑ - स्थि॒त्यै॒ । \newline
13. पञ्च॒ चित॑य॒ श्चित॑यः॒ पञ्च॒ पञ्च॒ चित॑यः । \newline
14. चित॑यो भवन्ति भवन्ति॒ चित॑य॒ श्चित॑यो भवन्ति । \newline
15. भ॒व॒न्ति॒ पाङ्क्तः॒ पाङ्क्तो॑ भवन्ति भवन्ति॒ पाङ्क्तः॑ । \newline
16. पाङ्क्तः॒ पुरु॑षः॒ पुरु॑षः॒ पाङ्क्तः॒ पाङ्क्तः॒ पुरु॑षः । \newline
17. पुरु॑ष आ॒त्मान॑ मा॒त्मान॒म् पुरु॑षः॒ पुरु॑ष आ॒त्मान᳚म् । \newline
18. आ॒त्मान॑ मे॒वै वात्मान॑ मा॒त्मान॑ मे॒व । \newline
19. ए॒व स्पृ॑णोति स्पृणो त्ये॒वैव स्पृ॑णोति । \newline
20. स्पृ॒णो॒ति॒ पञ्च॒ पञ्च॑ स्पृणोति स्पृणोति॒ पञ्च॑ । \newline
21. पञ्च॒ चित॑य॒ श्चित॑यः॒ पञ्च॒ पञ्च॒ चित॑यः । \newline
22. चित॑यो भवन्ति भवन्ति॒ चित॑य॒ श्चित॑यो भवन्ति । \newline
23. भ॒व॒न्ति॒ प॒ञ्चभिः॑ प॒ञ्चभि॑र् भवन्ति भवन्ति प॒ञ्चभिः॑ । \newline
24. प॒ञ्चभिः॒ पुरी॑षैः॒ पुरी॑षैः प॒ञ्चभिः॑ प॒ञ्चभिः॒ पुरी॑षैः । \newline
25. प॒ञ्चभि॒रिति॑ प॒ञ्च - भिः॒ । \newline
26. पुरी॑षै र॒भ्य॑भि पुरी॑षैः॒ पुरी॑षै र॒भि । \newline
27. अ॒भ्यू॑ह त्यूह त्य॒भ्या᳚(1॒) भ्यू॑हति । \newline
28. ऊ॒ह॒ति॒ दश॒ दशो॑ह त्यूहति॒ दश॑ । \newline
29. दश॒ सꣳ सम् दश॒ दश॒ सम् । \newline
30. सम् प॑द्यन्ते पद्यन्ते॒ सꣳ सम् प॑द्यन्ते । \newline
31. प॒द्य॒न्ते॒ दशा᳚क्षरो॒ दशा᳚क्षरः पद्यन्ते पद्यन्ते॒ दशा᳚क्षरः । \newline
32. दशा᳚क्षरो॒ वै वै दशा᳚क्षरो॒ दशा᳚क्षरो॒ वै । \newline
33. दशा᳚क्षर॒ इति॒ दश॑ - अ॒क्ष॒रः॒ । \newline
34. वै पुरु॑षः॒ पुरु॑षो॒ वै वै पुरु॑षः । \newline
35. पुरु॑षो॒ यावा॒न्॒. यावा॒न् पुरु॑षः॒ पुरु॑षो॒ यावान्॑ । \newline
36. यावा॑ ने॒वैव यावा॒न्॒. यावा॑ ने॒व । \newline
37. ए॒व पुरु॑षः॒ पुरु॑ष ए॒वैव पुरु॑षः । \newline
38. पुरु॑ष॒ स्तम् तम् पुरु॑षः॒ पुरु॑ष॒ स्तम् । \newline
39. तꣳ स्पृ॑णोति स्पृणोति॒ तम् तꣳ स्पृ॑णोति । \newline
40. स्पृ॒णो॒ त्यथो॒ अथो᳚ स्पृणोति स्पृणो॒ त्यथो᳚ । \newline
41. अथो॒ दशा᳚क्षरा॒ दशा᳚क्ष॒रा ऽथो॒ अथो॒ दशा᳚क्षरा । \newline
42. अथो॒ इत्यथो᳚ । \newline
43. दशा᳚क्षरा वि॒राड् वि॒राड् दशा᳚क्षरा॒ दशा᳚क्षरा वि॒राट् । \newline
44. दशा᳚क्ष॒रेति॒ दश॑ - अ॒क्ष॒रा॒ । \newline
45. वि॒राडन्न॒ मन्नं॑ ॅवि॒राड् वि॒राडन्न᳚म् । \newline
46. वि॒राडिति॑ वि - राट् । \newline
47. अन्नं॑ ॅवि॒राड् वि॒राडन्न॒ मन्नं॑ ॅवि॒राट् । \newline
48. वि॒राड् वि॒राजि॑ वि॒राजि॑ वि॒राड् वि॒राड् वि॒राजि॑ । \newline
49. वि॒राडिति॑ वि - राट् । \newline
50. वि॒रा ज्ये॒वैव वि॒राजि॑ वि॒रा ज्ये॒व । \newline
51. वि॒राजीति॑ वि - राजि॑ । \newline
52. ए॒वान्नाद्ये॒ ऽन्नाद्य॑ ए॒वै वान्नाद्ये᳚ । \newline
53. अ॒न्नाद्ये॒ प्रति॒ प्रत्य॒न्नाद्ये॒ ऽन्नाद्ये॒ प्रति॑ । \newline
54. अ॒न्नाद्य॒ इत्य॑न्न - अद्ये᳚ । \newline
55. प्रति॑ तिष्ठति तिष्ठति॒ प्रति॒ प्रति॑ तिष्ठति । \newline
56. ति॒ष्ठ॒ति॒ सं॒ॅव॒थ्स॒रः सं॑ॅवथ्स॒र स्ति॑ष्ठति तिष्ठति संॅवथ्स॒रः । \newline
57. सं॒ॅव॒थ्स॒रो वै वै सं॑ॅवथ्स॒रः सं॑ॅवथ्स॒रो वै । \newline
58. सं॒ॅव॒थ्स॒र इति॑ सं - व॒थ्स॒रः । \newline
59. वै ष॒ष्ठी ष॒ष्ठी वै वै ष॒ष्ठी । \newline
60. ष॒ष्ठी चिति॒ श्चिति॑ ष्ष॒ष्ठी ष॒ष्ठी चितिः॑ । \newline
61. चितिर्॑. ऋ॒तव॑ ऋ॒तव॒ श्चिति॒ श्चितिर्॑. ऋ॒तवः॑ । \newline
62. ऋ॒तवः॒ पुरी॑ष॒म् पुरी॑ष मृ॒तव॑ ऋ॒तवः॒ पुरी॑षम् । \newline
63. पुरी॑षꣳ॒॒ षट् थ्षट् पुरी॑ष॒म् पुरी॑षꣳ॒॒ षट् । \newline
64. षट् चित॑य॒ श्चित॑य॒ ष्षट् थ्षट् चित॑यः । \newline
65. चित॑यो भवन्ति भवन्ति॒ चित॑य॒ श्चित॑यो भवन्ति । \newline
66. भ॒व॒न्ति॒ षट् थ्षड् भ॑वन्ति भवन्ति॒ षट् । \newline
67. षट् पुरी॑षाणि॒ पुरी॑षाणि॒ षट् थ्षट् पुरी॑षाणि । \newline
68. पुरी॑षाणि॒ द्वाद॑श॒ द्वाद॑श॒ पुरी॑षाणि॒ पुरी॑षाणि॒ द्वाद॑श । \newline
69. द्वाद॑श॒ सꣳ सम् द्वाद॑श॒ द्वाद॑श॒ सम् । \newline
70. सम् प॑द्यन्ते पद्यन्ते॒ सꣳ सम् प॑द्यन्ते । \newline
71. प॒द्य॒न्ते॒ द्वाद॑श॒ द्वाद॑श पद्यन्ते पद्यन्ते॒ द्वाद॑श । \newline
72. द्वाद॑श॒ मासा॒ मासा॒ द्वाद॑श॒ द्वाद॑श॒ मासाः᳚ । \newline
73. मासाः᳚ संॅवथ्स॒रः सं॑ॅवथ्स॒रो मासा॒ मासाः᳚ संॅवथ्स॒रः । \newline
74. सं॒ॅव॒थ्स॒रः सं॑ॅवथ्स॒रे सं॑ॅवथ्स॒रे सं॑ॅवथ्स॒रः सं॑ॅवथ्स॒रः सं॑ॅवथ्स॒रे । \newline
75. सं॒ॅव॒थ्स॒र इति॑ सं - व॒थ्स॒रः । \newline
76. सं॒ॅव॒थ्स॒र ए॒वैव सं॑ॅवथ्स॒रे सं॑ॅवथ्स॒र ए॒व । \newline
77. सं॒ॅव॒थ्स॒र इति॑ सं - व॒थ्स॒रे । \newline
78. ए॒व प्रति॒ प्रत्ये॒ वैव प्रति॑ । \newline
79. प्रति॑ तिष्ठति तिष्ठति॒ प्रति॒ प्रति॑ तिष्ठति । \newline
80. ति॒ष्ठ॒तीति॑ तिष्ठति । \newline

\textbf{Ghana Paata } \newline

1. त्रि॒वृद्धि हि त्रि॒वृत् त्रि॒वृ द्ध्य॑ग्नि र॒ग्निर्. हि त्रि॒वृत् त्रि॒वृ द्ध्य॑ग्निः । \newline
2. त्रि॒वृदिति॑ त्रि - वृत् । \newline
3. ह्य॑ग्नि र॒ग्निर्. हि ह्य॑ग्निर् यद् यद॒ग्निर्. हि ह्य॑ग्निर् यत् । \newline
4. अ॒ग्निर् यद् यद॒ग्नि र॒ग्निर् यद् द्वे द्वे यद॒ग्नि र॒ग्निर् यद् द्वे । \newline
5. यद् द्वे द्वे यद् यद् द्वे द्वि॒पाद् द्वि॒पाद् द्वे यद् यद् द्वे द्वि॒पात् । \newline
6. द्वे द्वि॒पाद् द्वि॒पाद् द्वे द्वे द्वि॒पाद् यज॑मानो॒ यज॑मानो द्वि॒पाद् द्वे द्वे द्वि॒पाद् यज॑मानः । \newline
7. द्वे इति॒ द्वे । \newline
8. द्वि॒पाद् यज॑मानो॒ यज॑मानो द्वि॒पाद् द्वि॒पाद् यज॑मानः॒ प्रति॑ष्ठित्यै॒ प्रति॑ष्ठित्यै॒ यज॑मानो द्वि॒पाद् द्वि॒पाद् यज॑मानः॒ प्रति॑ष्ठित्यै । \newline
9. द्वि॒पादिति॑ द्वि - पात् । \newline
10. यज॑मानः॒ प्रति॑ष्ठित्यै॒ प्रति॑ष्ठित्यै॒ यज॑मानो॒ यज॑मानः॒ प्रति॑ष्ठित्यै॒ पञ्च॒ पञ्च॒ प्रति॑ष्ठित्यै॒ यज॑मानो॒ यज॑मानः॒ प्रति॑ष्ठित्यै॒ पञ्च॑ । \newline
11. प्रति॑ष्ठित्यै॒ पञ्च॒ पञ्च॒ प्रति॑ष्ठित्यै॒ प्रति॑ष्ठित्यै॒ पञ्च॒ चित॑य॒ श्चित॑यः॒ पञ्च॒ प्रति॑ष्ठित्यै॒ प्रति॑ष्ठित्यै॒ पञ्च॒ चित॑यः । \newline
12. प्रति॑ष्ठित्या॒ इति॒ प्रति॑ - स्थि॒त्यै॒ । \newline
13. पञ्च॒ चित॑य॒ श्चित॑यः॒ पञ्च॒ पञ्च॒ चित॑यो भवन्ति भवन्ति॒ चित॑यः॒ पञ्च॒ पञ्च॒ चित॑यो भवन्ति । \newline
14. चित॑यो भवन्ति भवन्ति॒ चित॑य॒ श्चित॑यो भवन्ति॒ पाङ्क्तः॒ पाङ्क्तो॑ भवन्ति॒ चित॑य॒ श्चित॑यो भवन्ति॒ पाङ्क्तः॑ । \newline
15. भ॒व॒न्ति॒ पाङ्क्तः॒ पाङ्क्तो॑ भवन्ति भवन्ति॒ पाङ्क्तः॒ पुरु॑षः॒ पुरु॑षः॒ पाङ्क्तो॑ भवन्ति भवन्ति॒ पाङ्क्तः॒ पुरु॑षः । \newline
16. पाङ्क्तः॒ पुरु॑षः॒ पुरु॑षः॒ पाङ्क्तः॒ पाङ्क्तः॒ पुरु॑ष आ॒त्मान॑ मा॒त्मान॒म् पुरु॑षः॒ पाङ्क्तः॒ पाङ्क्तः॒ पुरु॑ष आ॒त्मान᳚म् । \newline
17. पुरु॑ष आ॒त्मान॑ मा॒त्मान॒म् पुरु॑षः॒ पुरु॑ष आ॒त्मान॑ मे॒वै वात्मान॒म् पुरु॑षः॒ पुरु॑ष आ॒त्मान॑ मे॒व । \newline
18. आ॒त्मान॑ मे॒वै वात्मान॑ मा॒त्मान॑ मे॒व स्पृ॑णोति स्पृणो त्ये॒वात्मान॑ मा॒त्मान॑ मे॒व स्पृ॑णोति । \newline
19. ए॒व स्पृ॑णोति स्पृणो त्ये॒वैव स्पृ॑णोति॒ पञ्च॒ पञ्च॑ स्पृणो त्ये॒वैव स्पृ॑णोति॒ पञ्च॑ । \newline
20. स्पृ॒णो॒ति॒ पञ्च॒ पञ्च॑ स्पृणोति स्पृणोति॒ पञ्च॒ चित॑य॒ श्चित॑यः॒ पञ्च॑ स्पृणोति स्पृणोति॒ पञ्च॒ चित॑यः । \newline
21. पञ्च॒ चित॑य॒ श्चित॑यः॒ पञ्च॒ पञ्च॒ चित॑यो भवन्ति भवन्ति॒ चित॑यः॒ पञ्च॒ पञ्च॒ चित॑यो भवन्ति । \newline
22. चित॑यो भवन्ति भवन्ति॒ चित॑य॒ श्चित॑यो भवन्ति प॒ञ्चभिः॑ प॒ञ्चभि॑र् भवन्ति॒ चित॑य॒ श्चित॑यो भवन्ति प॒ञ्चभिः॑ । \newline
23. भ॒व॒न्ति॒ प॒ञ्चभिः॑ प॒ञ्चभि॑र् भवन्ति भवन्ति प॒ञ्चभिः॒ पुरी॑षैः॒ पुरी॑षैः प॒ञ्चभि॑र् भवन्ति भवन्ति प॒ञ्चभिः॒ पुरी॑षैः । \newline
24. प॒ञ्चभिः॒ पुरी॑षैः॒ पुरी॑षैः प॒ञ्चभिः॑ प॒ञ्चभिः॒ पुरी॑षै र॒भ्य॑भि पुरी॑षैः प॒ञ्चभिः॑ प॒ञ्चभिः॒ पुरी॑षै र॒भि । \newline
25. प॒ञ्चभि॒रिति॑ प॒ञ्च - भिः॒ । \newline
26. पुरी॑षै र॒भ्य॑भि पुरी॑षैः॒ पुरी॑षै र॒भ्यू॑ह त्यूह त्य॒भि पुरी॑षैः॒ पुरी॑षै र॒भ्यू॑हति । \newline
27. अ॒भ्यू॑ह त्यूह त्य॒भ्या᳚(1॒)भ्यू॑हति॒ दश॒ दशो॑ह त्य॒भ्या᳚(1॒)भ्यू॑हति॒ दश॑ । \newline
28. ऊ॒ह॒ति॒ दश॒ दशो॑ह त्यूहति॒ दश॒ सꣳ सम् दशो॑ह त्यूहति॒ दश॒ सम् । \newline
29. दश॒ सꣳ सम् दश॒ दश॒ सम् प॑द्यन्ते पद्यन्ते॒ सम् दश॒ दश॒ सम् प॑द्यन्ते । \newline
30. सम् प॑द्यन्ते पद्यन्ते॒ सꣳ सम् प॑द्यन्ते॒ दशा᳚क्षरो॒ दशा᳚क्षरः पद्यन्ते॒ सꣳ सम् प॑द्यन्ते॒ दशा᳚क्षरः । \newline
31. प॒द्य॒न्ते॒ दशा᳚क्षरो॒ दशा᳚क्षरः पद्यन्ते पद्यन्ते॒ दशा᳚क्षरो॒ वै वै दशा᳚क्षरः पद्यन्ते पद्यन्ते॒ दशा᳚क्षरो॒ वै । \newline
32. दशा᳚क्षरो॒ वै वै दशा᳚क्षरो॒ दशा᳚क्षरो॒ वै पुरु॑षः॒ पुरु॑षो॒ वै दशा᳚क्षरो॒ दशा᳚क्षरो॒ वै पुरु॑षः । \newline
33. दशा᳚क्षर॒ इति॒ दश॑ - अ॒क्ष॒रः॒ । \newline
34. वै पुरु॑षः॒ पुरु॑षो॒ वै वै पुरु॑षो॒ यावा॒न्॒. यावा॒न् पुरु॑षो॒ वै वै पुरु॑षो॒ यावान्॑ । \newline
35. पुरु॑षो॒ यावा॒न्॒. यावा॒न् पुरु॑षः॒ पुरु॑षो॒ यावा॑ ने॒वैव यावा॒न् पुरु॑षः॒ पुरु॑षो॒ यावा॑ ने॒व । \newline
36. यावा॑ ने॒वैव यावा॒न्॒. यावा॑ ने॒व पुरु॑षः॒ पुरु॑ष ए॒व यावा॒न्॒. यावा॑ ने॒व पुरु॑षः । \newline
37. ए॒व पुरु॑षः॒ पुरु॑ष ए॒वैव पुरु॑ष॒ स्तम् तम् पुरु॑ष ए॒वैव पुरु॑ष॒ स्तम् । \newline
38. पुरु॑ष॒ स्तम् तम् पुरु॑षः॒ पुरु॑ष॒ स्तꣳ स्पृ॑णोति स्पृणोति॒ तम् पुरु॑षः॒ पुरु॑ष॒ स्तꣳ स्पृ॑णोति । \newline
39. तꣳ स्पृ॑णोति स्पृणोति॒ तम् तꣳ स्पृ॑णो॒ त्यथो॒ अथो᳚ स्पृणोति॒ तम् तꣳ स्पृ॑णो॒ त्यथो᳚ । \newline
40. स्पृ॒णो॒ त्यथो॒ अथो᳚ स्पृणोति स्पृणो॒ त्यथो॒ दशा᳚क्षरा॒ दशा᳚क्ष॒रा ऽथो᳚ स्पृणोति स्पृणो॒ त्यथो॒ दशा᳚क्षरा । \newline
41. अथो॒ दशा᳚क्षरा॒ दशा᳚क्ष॒रा ऽथो॒ अथो॒ दशा᳚क्षरा वि॒राड् वि॒राड् दशा᳚क्ष॒रा ऽथो॒ अथो॒ दशा᳚क्षरा वि॒राट् । \newline
42. अथो॒ इत्यथो᳚ । \newline
43. दशा᳚क्षरा वि॒राड् वि॒राड् दशा᳚क्षरा॒ दशा᳚क्षरा वि॒राडन्न॒ मन्नं॑ ॅवि॒राड् दशा᳚क्षरा॒ दशा᳚क्षरा वि॒राडन्न᳚म् । \newline
44. दशा᳚क्ष॒रेति॒ दश॑ - अ॒क्ष॒रा॒ । \newline
45. वि॒राडन्न॒ मन्नं॑ ॅवि॒राड् वि॒राडन्नं॑ ॅवि॒राड् वि॒राडन्नं॑ ॅवि॒राड् वि॒राडन्नं॑ ॅवि॒राट् । \newline
46. वि॒राडिति॑ वि - राट् । \newline
47. अन्नं॑ ॅवि॒राड् वि॒राडन्न॒ मन्नं॑ ॅवि॒राड् वि॒राजि॑ वि॒राजि॑ वि॒राडन्न॒ मन्नं॑ ॅवि॒राड् वि॒राजि॑ । \newline
48. वि॒राड् वि॒राजि॑ वि॒राजि॑ वि॒राड् वि॒राड् वि॒राज्ये॒वैव वि॒राजि॑ वि॒राड् वि॒राड् वि॒राज्ये॒व । \newline
49. वि॒राडिति॑ वि - राट् । \newline
50. वि॒राज्ये॒वैव वि॒राजि॑ वि॒राज्ये॒ वान्नाद्ये॒ ऽन्नाद्य॑ ए॒व वि॒राजि॑ वि॒राज्ये॒ वान्नाद्ये᳚ । \newline
51. वि॒राजीति॑ वि - राजि॑ । \newline
52. ए॒वान्नाद्ये॒ ऽन्नाद्य॑ ए॒वै वान्नाद्ये॒ प्रति॒ प्रत्य॒न्नाद्य॑ ए॒वै वान्नाद्ये॒ प्रति॑ । \newline
53. अ॒न्नाद्ये॒ प्रति॒ प्रत्य॒न्नाद्ये॒ ऽन्नाद्ये॒ प्रति॑ तिष्ठति तिष्ठति॒ प्रत्य॒न्नाद्ये॒ ऽन्नाद्ये॒ प्रति॑ तिष्ठति । \newline
54. अ॒न्नाद्य॒ इत्य॑न्न - अद्ये᳚ । \newline
55. प्रति॑ तिष्ठति तिष्ठति॒ प्रति॒ प्रति॑ तिष्ठति संॅवथ्स॒रः सं॑ॅवथ्स॒र स्ति॑ष्ठति॒ प्रति॒ प्रति॑ तिष्ठति संॅवथ्स॒रः । \newline
56. ति॒ष्ठ॒ति॒ सं॒ॅव॒थ्स॒रः सं॑ॅवथ्स॒र स्ति॑ष्ठति तिष्ठति संॅवथ्स॒रो वै वै सं॑ॅवथ्स॒र स्ति॑ष्ठति तिष्ठति संॅवथ्स॒रो वै । \newline
57. सं॒ॅव॒थ्स॒रो वै वै सं॑ॅवथ्स॒रः सं॑ॅवथ्स॒रो वै ष॒ष्ठी ष॒ष्ठी वै सं॑ॅवथ्स॒रः सं॑ॅवथ्स॒रो वै ष॒ष्ठी । \newline
58. सं॒ॅव॒थ्स॒र इति॑ सं - व॒थ्स॒रः । \newline
59. वै ष॒ष्ठी ष॒ष्ठी वै वै ष॒ष्ठी चिति॒ श्चिति॑ ष्ष॒ष्ठी वै वै ष॒ष्ठी चितिः॑ । \newline
60. ष॒ष्ठी चिति॒ श्चिति॑ ष्ष॒ष्ठी ष॒ष्ठी चितिर्॑. ऋ॒तव॑ ऋ॒तव॒ श्चिति॑ ष्ष॒ष्ठी ष॒ष्ठी चितिर्॑. ऋ॒तवः॑ । \newline
61. चितिर्॑. ऋ॒तव॑ ऋ॒तव॒ श्चिति॒ श्चितिर्॑. ऋ॒तवः॒ पुरी॑ष॒म् पुरी॑ष मृ॒तव॒ श्चिति॒ श्चितिर्॑. ऋ॒तवः॒ पुरी॑षम् । \newline
62. ऋ॒तवः॒ पुरी॑ष॒म् पुरी॑ष मृ॒तव॑ ऋ॒तवः॒ पुरी॑षꣳ॒॒ षट् थ्षट् पुरी॑ष मृ॒तव॑ ऋ॒तवः॒ पुरी॑षꣳ॒॒ षट् । \newline
63. पुरी॑षꣳ॒॒ षट् थ्षट् पुरी॑ष॒म् पुरी॑षꣳ॒॒ षट् चित॑य॒ श्चित॑य॒ ष्षट् पुरी॑ष॒म् पुरी॑षꣳ॒॒ षट् चित॑यः । \newline
64. षट् चित॑य॒ श्चित॑य॒ ष्षट् थ्षट् चित॑यो भवन्ति भवन्ति॒ चित॑य॒ ष्षट् थ्षट् चित॑यो भवन्ति । \newline
65. चित॑यो भवन्ति भवन्ति॒ चित॑य॒ श्चित॑यो भवन्ति॒ षट् थ्षड् भ॑वन्ति॒ चित॑य॒ श्चित॑यो भवन्ति॒ षट् । \newline
66. भ॒व॒न्ति॒ षट् थ्षड् भ॑वन्ति भवन्ति॒ षट् पुरी॑षाणि॒ पुरी॑षाणि॒ षड् भ॑वन्ति भवन्ति॒ षट् पुरी॑षाणि । \newline
67. षट् पुरी॑षाणि॒ पुरी॑षाणि॒ षट् थ्षट् पुरी॑षाणि॒ द्वाद॑श॒ द्वाद॑श॒ पुरी॑षाणि॒ षट् थ्षट् पुरी॑षाणि॒ द्वाद॑श । \newline
68. पुरी॑षाणि॒ द्वाद॑श॒ द्वाद॑श॒ पुरी॑षाणि॒ पुरी॑षाणि॒ द्वाद॑श॒ सꣳ सम् द्वाद॑श॒ पुरी॑षाणि॒ पुरी॑षाणि॒ द्वाद॑श॒ सम् । \newline
69. द्वाद॑श॒ सꣳ सम् द्वाद॑श॒ द्वाद॑श॒ सम् प॑द्यन्ते पद्यन्ते॒ सम् द्वाद॑श॒ द्वाद॑श॒ सम् प॑द्यन्ते । \newline
70. सम् प॑द्यन्ते पद्यन्ते॒ सꣳ सम् प॑द्यन्ते॒ द्वाद॑श॒ द्वाद॑श पद्यन्ते॒ सꣳ सम् प॑द्यन्ते॒ द्वाद॑श । \newline
71. प॒द्य॒न्ते॒ द्वाद॑श॒ द्वाद॑श पद्यन्ते पद्यन्ते॒ द्वाद॑श॒ मासा॒ मासा॒ द्वाद॑श पद्यन्ते पद्यन्ते॒ द्वाद॑श॒ मासाः᳚ । \newline
72. द्वाद॑श॒ मासा॒ मासा॒ द्वाद॑श॒ द्वाद॑श॒ मासाः᳚ संॅवथ्स॒रः सं॑ॅवथ्स॒रो मासा॒ द्वाद॑श॒ द्वाद॑श॒ मासाः᳚ संॅवथ्स॒रः । \newline
73. मासाः᳚ संॅवथ्स॒रः सं॑ॅवथ्स॒रो मासा॒ मासाः᳚ संॅवथ्स॒रः सं॑ॅवथ्स॒रे सं॑ॅवथ्स॒रे सं॑ॅवथ्स॒रो मासा॒ मासाः᳚ संॅवथ्स॒रः सं॑ॅवथ्स॒रे । \newline
74. सं॒ॅव॒थ्स॒रः सं॑ॅवथ्स॒रे सं॑ॅवथ्स॒रे सं॑ॅवथ्स॒रः सं॑ॅवथ्स॒रः सं॑ॅवथ्स॒र ए॒वैव सं॑ॅवथ्स॒रे सं॑ॅवथ्स॒रः सं॑ॅवथ्स॒रः सं॑ॅवथ्स॒र ए॒व । \newline
75. सं॒ॅव॒थ्स॒र इति॑ सं - व॒थ्स॒रः । \newline
76. सं॒ॅव॒थ्स॒र ए॒वैव सं॑ॅवथ्स॒रे सं॑ॅवथ्स॒र ए॒व प्रति॒ प्रत्ये॒व सं॑ॅवथ्स॒रे सं॑ॅवथ्स॒र ए॒व प्रति॑ । \newline
77. सं॒ॅव॒थ्स॒र इति॑ सं - व॒थ्स॒रे । \newline
78. ए॒व प्रति॒ प्रत्ये॒ वैव प्रति॑ तिष्ठति तिष्ठति॒ प्रत्ये॒ वैव प्रति॑ तिष्ठति । \newline
79. प्रति॑ तिष्ठति तिष्ठति॒ प्रति॒ प्रति॑ तिष्ठति । \newline
80. ति॒ष्ठ॒तीति॑ तिष्ठति । \newline
\pagebreak
\markright{ TS 5.6.11.1  \hfill https://www.vedavms.in \hfill}

\section{ TS 5.6.11.1 }

\textbf{TS 5.6.11.1 } \newline
\textbf{Samhita Paata} \newline

रोहि॑तो धू॒म्ररो॑हितः क॒र्कन्धु॑रोहित॒स्ते प्रा॑जाप॒त्या ब॒भ्रुर॑रु॒णब॑भ्रुः॒ शुक॑बभ्रु॒स्ते रौ॒द्राः श्येतः॑ श्येता॒क्षः श्येत॑ग्रीव॒स्ते पि॑तृदेव॒त्या᳚स्ति॒स्रः कृ॒ष्णा व॒शा वा॑रु॒ण्य॑स्ति॒स्रः श्वे॒ता व॒शाः सौ॒र्यो॑ मैत्राबार्.हस्प॒त्या धू॒म्रल॑लामास्तूप॒राः ॥ \newline

\textbf{Pada Paata} \newline

रोहि॑तः । धू॒म्ररो॑हित॒ इति॑ धू॒म्र - रो॒हि॒तः॒ । क॒र्कन्धु॑रोहित॒ इति॑ क॒र्कन्धु॑ - रो॒हि॒तः॒ । ते । प्रा॒जा॒प॒त्या इति॑ प्राजा - प॒त्याः । ब॒भ्रुः । अ॒रु॒णब॑भ्रु॒रित्य॑रु॒ण - ब॒भ्रुः॒ । शुक॑बभ्रु॒रिति॒ शुक॑ - ब॒भ्रुः॒ । ते । रौ॒द्राः । श्येतः॑ । श्ये॒ता॒क्ष इति॑ श्येत - अ॒क्षः । श्येत॑ग्रीव॒ इति॒ श्येत॑ - ग्री॒वः॒ । ते । पि॒तृ॒दे॒व॒त्या॑ इति॑ पितृ - दे॒व॒त्याः᳚ । ति॒स्रः । कृ॒ष्णाः । व॒शाः । वा॒रु॒ण्यः॑ । ति॒स्रः । श्वे॒ताः । व॒शाः । सौ॒र्यः॑ । मै॒त्रा॒बा॒र्.॒ह॒स्प॒त्या इति॑ मैत्रा - बा॒र्.॒ह॒स्प॒त्याः । धू॒म्रल॑लामा॒ इति॑ धू॒म्र - ल॒ला॒माः॒ । तू॒प॒राः ॥  \newline


\textbf{Krama Paata} \newline

रोहि॑तो धू॒म्ररो॑हितः । धू॒म्ररो॑हितः क॒र्कन्धु॑रोहितः । धू॒म्ररो॑हित॒ इति॑ धू॒म्र - रो॒हि॒तः॒ । क॒र्कन्धु॑रोहित॒स्ते । क॒र्कन्धु॑रोहित॒ इति॑ क॒र्कन्धु॑ - रो॒हि॒तः॒ । ते प्रा॑जाप॒त्याः । प्रा॒जा॒प॒त्या ब॒भ्रुः । प्रा॒जा॒प॒त्या इति॑ प्राजा - प॒त्याः । ब॒भ्रुर॑रु॒णब॑भ्रुः । अ॒रु॒णब॑भ्रुः॒ शुक॑बभ्रुः । अ॒रु॒णब॑भ्रु॒रित्य॑रु॒ण - ब॒भ्रुः॒ । शुक॑बभ्रु॒स्ते । शुक॑बभ्रु॒रिति॒ शुक॑ - ब॒भ्रुः॒ । ते रौ॒द्राः । रौ॒द्राः श्येतः॑ । श्येतः॑ श्येता॒क्षः । श्ये॒ता॒क्षः श्येत॑ग्रीवः । श्ये॒ता॒क्ष इति॑ श्येत - अ॒क्षः । श्येत॑ग्रीव॒स्ते । श्येत॑ग्रीव॒ इति॒ श्येत॑ - ग्री॒वः॒ । ते पि॑तृदेव॒त्याः᳚ । पि॒तृ॒दे॒व॒त्या᳚स्ति॒स्रः । पि॒तृ॒दे॒व॒त्या॑ इति॑ पितृ - दे॒व॒त्याः᳚ । ति॒स्रः कृ॒ष्णाः । कृ॒ष्णा व॒शाः । व॒शा वा॑रु॒ण्यः॑ । वा॒रु॒ण्य॑स्ति॒स्रः । ति॒स्रः श्वे॒ताः । श्वे॒ता व॒शाः । व॒शाः सौ॒र्यः॑ । सौ॒र्यो॑ मैत्राबार्.हस्प॒त्याः । मै॒त्रा॒बा॒र्॒.ह॒स्प॒त्या धू॒म्रल॑लामाः । मै॒त्रा॒बा॒र्॒.ह॒स्प॒त्या इति॑ मैत्रा - बा॒र्॒.ह॒स्प॒त्याः । धू॒म्रल॑लामास्तूप॒राः । धू॒म्रल॑लामा॒ इति॑ धू॒म्र - ल॒ला॒माः॒ । तू॒प॒रा इति॑ तूप॒राः । \newline

\textbf{Jatai Paata} \newline

1. रोहि॑तो धू॒म्ररो॑हितो धू॒म्ररो॑हितो॒ रोहि॑तो॒ रोहि॑तो धू॒म्ररो॑हितः । \newline
2. धू॒म्ररो॑हितः क॒र्कन्धु॑रोहितः क॒र्कन्धु॑रोहितो धू॒म्ररो॑हितो धू॒म्ररो॑हितः क॒र्कन्धु॑रोहितः । \newline
3. धू॒म्ररो॑हित॒ इति॑ धू॒म्र - रो॒हि॒तः॒ । \newline
4. क॒र्कन्धु॑रोहित॒ स्ते ते क॒र्कन्धु॑रोहितः क॒र्कन्धु॑रोहित॒ स्ते । \newline
5. क॒र्कन्धु॑रोहित॒ इति॑ क॒र्कन्धु॑ - रो॒हि॒तः॒ । \newline
6. ते प्रा॑जाप॒त्याः प्रा॑जाप॒त्या स्ते ते प्रा॑जाप॒त्याः । \newline
7. प्रा॒जा॒प॒त्या ब॒भ्रुर् ब॒भ्रुः प्रा॑जाप॒त्याः प्रा॑जाप॒त्या ब॒भ्रुः । \newline
8. प्रा॒जा॒प॒त्या इति॑ प्राजा - प॒त्याः । \newline
9. ब॒भ्रु र॑रु॒णब॑भ्रु ररु॒णब॑भ्रुर् ब॒भ्रुर् ब॒भ्रु र॑रु॒णब॑भ्रुः । \newline
10. अ॒रु॒णब॑भ्रुः॒ शुक॑बभ्रुः॒ शुक॑बभ्रु ररु॒णब॑भ्रु ररु॒णब॑भ्रुः॒ शुक॑बभ्रुः । \newline
11. अ॒रु॒णब॑भ्रु॒रित्य॑रु॒ण - ब॒भ्रुः॒ । \newline
12. शुक॑बभ्रु॒ स्ते ते शुक॑बभ्रुः॒ शुक॑बभ्रु॒ स्ते । \newline
13. शुक॑बभ्रु॒रिति॒ शुक॑ - ब॒भ्रुः॒ । \newline
14. ते रौ॒द्रा रौ॒द्रा स्ते ते रौ॒द्राः । \newline
15. रौ॒द्राः श्येतः॒ श्येतो॑ रौ॒द्रा रौ॒द्राः श्येतः॑ । \newline
16. श्येतः॑ श्येता॒क्षः श्ये॑ता॒क्षः श्येतः॒ श्येतः॑ श्येता॒क्षः । \newline
17. श्ये॒ता॒क्षः श्येत॑ग्रीवः॒ श्येत॑ग्रीवः श्येता॒क्षः श्ये॑ता॒क्षः श्येत॑ग्रीवः । \newline
18. श्ये॒ता॒क्ष इति॑ श्येत - अ॒क्षः । \newline
19. श्येत॑ग्रीव॒ स्ते ते श्येत॑ग्रीवः॒ श्येत॑ग्रीव॒ स्ते । \newline
20. श्येत॑ग्रीव॒ इति॒ श्येत॑ - ग्री॒वः॒ । \newline
21. ते पि॑तृदेव॒त्याः᳚ पितृदेव॒त्या᳚ स्ते ते पि॑तृदेव॒त्याः᳚ । \newline
22. पि॒तृ॒दे॒व॒त्या᳚ स्ति॒स्र स्ति॒स्रः पि॑तृदेव॒त्याः᳚ पितृदेव॒त्या᳚ स्ति॒स्रः । \newline
23. पि॒तृ॒दे॒व॒त्या॑ इति॑ पितृ - दे॒व॒त्याः᳚ । \newline
24. ति॒स्रः कृ॒ष्णाः कृ॒ष्णा स्ति॒स्र स्ति॒स्रः कृ॒ष्णाः । \newline
25. कृ॒ष्णा व॒शा व॒शाः कृ॒ष्णाः कृ॒ष्णा व॒शाः । \newline
26. व॒शा वा॑रु॒ण्यो॑ वारु॒ण्यो॑ व॒शा व॒शा वा॑रु॒ण्यः॑ । \newline
27. वा॒रु॒ण्य॑ स्ति॒स्र स्ति॒स्रो वा॑रु॒ण्यो॑ वारु॒ण्य॑ स्ति॒स्रः । \newline
28. ति॒स्रः श्वे॒ताः श्वे॒ता स्ति॒स्र स्ति॒स्रः श्वे॒ताः । \newline
29. श्वे॒ता व॒शा व॒शाः श्वे॒ताः श्वे॒ता व॒शाः । \newline
30. व॒शाः सौ॒र्यः॑ सौ॒र्यो॑ व॒शा व॒शाः सौ॒र्यः॑ । \newline
31. सौ॒र्यो॑ मैत्राबार्.हस्प॒त्या मै᳚त्राबार्.हस्प॒त्याः सौ॒र्यः॑ सौ॒र्यो॑ मैत्राबार्.हस्प॒त्याः । \newline
32. मै॒त्रा॒बा॒र्॒.ह॒स्प॒त्या धू॒म्रल॑लामा धू॒म्रल॑लामा मैत्राबार्.हस्प॒त्या मै᳚त्राबार्.हस्प॒त्या धू॒म्रल॑लामाः । \newline
33. मै॒त्रा॒बा॒र्.॒ह॒स्प॒त्या इति॑ मैत्रा - बा॒र्.॒ह॒स्प॒त्याः । \newline
34. धू॒म्रल॑लामा स्तूप॒रा स्तू॑प॒रा धू॒म्रल॑लामा धू॒म्रल॑लामा स्तूप॒राः । \newline
35. धू॒म्रल॑लामा॒ इति॑ धू॒म्र - ल॒ला॒माः॒ । \newline
36. तू॒प॒रा इति॑ तूप॒राः । \newline

\textbf{Ghana Paata } \newline

1. रोहि॑तो धू॒म्ररो॑हितो धू॒म्ररो॑हितो॒ रोहि॑तो॒ रोहि॑तो धू॒म्ररो॑हितः क॒र्कन्धु॑रोहितः क॒र्कन्धु॑रोहितो धू॒म्ररो॑हितो॒ रोहि॑तो॒ रोहि॑तो धू॒म्ररो॑हितः क॒र्कन्धु॑रोहितः । \newline
2. धू॒म्ररो॑हितः क॒र्कन्धु॑रोहितः क॒र्कन्धु॑रोहितो धू॒म्ररो॑हितो धू॒म्ररो॑हितः क॒र्कन्धु॑रोहित॒ स्ते ते क॒र्कन्धु॑रोहितो धू॒म्ररो॑हितो धू॒म्ररो॑हितः क॒र्कन्धु॑रोहित॒ स्ते । \newline
3. धू॒म्ररो॑हित॒ इति॑ धू॒म्र - रो॒हि॒तः॒ । \newline
4. क॒र्कन्धु॑रोहित॒ स्ते ते क॒र्कन्धु॑रोहितः क॒र्कन्धु॑रोहित॒ स्ते प्रा॑जाप॒त्याः प्रा॑जाप॒त्या स्ते क॒र्कन्धु॑रोहितः क॒र्कन्धु॑रोहित॒ स्ते प्रा॑जाप॒त्याः । \newline
5. क॒र्कन्धु॑रोहित॒ इति॑ क॒र्कन्धु॑ - रो॒हि॒तः॒ । \newline
6. ते प्रा॑जाप॒त्याः प्रा॑जाप॒त्या स्ते ते प्रा॑जाप॒त्या ब॒भ्रुर् ब॒भ्रुः प्रा॑जाप॒त्या स्ते ते प्रा॑जाप॒त्या ब॒भ्रुः । \newline
7. प्रा॒जा॒प॒त्या ब॒भ्रुर् ब॒भ्रुः प्रा॑जाप॒त्याः प्रा॑जाप॒त्या ब॒भ्रु र॑रु॒ण ब॑भ्रु ररु॒णब॑भ्रुर् ब॒भ्रुः प्रा॑जाप॒त्याः प्रा॑जाप॒त्या ब॒भ्रु र॑रु॒णब॑भ्रुः । \newline
8. प्रा॒जा॒प॒त्या इति॑ प्राजा - प॒त्याः । \newline
9. ब॒भ्रु र॑रु॒णब॑भ्रु ररु॒णब॑भ्रुर् ब॒भ्रुर् ब॒भ्रु र॑रु॒णब॑भ्रुः॒ शुक॑बभ्रुः॒ शुक॑बभ्रु ररु॒णब॑भ्रुर् ब॒भ्रुर् ब॒भ्रु र॑रु॒णब॑भ्रुः॒ शुक॑बभ्रुः । \newline
10. अ॒रु॒णब॑भ्रुः॒ शुक॑बभ्रुः॒ शुक॑बभ्रु ररु॒णब॑भ्रु ररु॒णब॑भ्रुः॒ शुक॑बभ्रु॒ स्ते ते शुक॑बभ्रु ररु॒णब॑भ्रु ररु॒णब॑भ्रुः॒ शुक॑बभ्रु॒ स्ते । \newline
11. अ॒रु॒णब॑भ्रु॒रित्य॑रु॒ण - ब॒भ्रुः॒ । \newline
12. शुक॑बभ्रु॒ स्ते ते शुक॑बभ्रुः॒ शुक॑बभ्रु॒ स्ते रौ॒द्रा रौ॒द्रा स्ते शुक॑बभ्रुः॒ शुक॑बभ्रु॒ स्ते रौ॒द्राः । \newline
13. शुक॑बभ्रु॒रिति॒ शुक॑ - ब॒भ्रुः॒ । \newline
14. ते रौ॒द्रा रौ॒द्रा स्ते ते रौ॒द्राः श्येतः॒ श्येतो॑ रौ॒द्रा स्ते ते रौ॒द्राः श्येतः॑ । \newline
15. रौ॒द्राः श्येतः॒ श्येतो॑ रौ॒द्रा रौ॒द्राः श्येतः॑ श्येता॒क्षः श्ये॑ता॒क्षः श्येतो॑ रौ॒द्रा रौ॒द्राः श्येतः॑ श्येता॒क्षः । \newline
16. श्येतः॑ श्येता॒क्षः श्ये॑ता॒क्षः श्येतः॒ श्येतः॑ श्येता॒क्षः श्येत॑ग्रीवः॒ श्येत॑ग्रीवः श्येता॒क्षः श्येतः॒ श्येतः॑ श्येता॒क्षः श्येत॑ग्रीवः । \newline
17. श्ये॒ता॒क्षः श्येत॑ग्रीवः॒ श्येत॑ग्रीवः श्येता॒क्षः श्ये॑ता॒क्षः श्येत॑ग्रीव॒ स्ते ते श्येत॑ग्रीवः श्येता॒क्षः श्ये॑ता॒क्षः श्येत॑ग्रीव॒ स्ते । \newline
18. श्ये॒ता॒क्ष इति॑ श्येत - अ॒क्षः । \newline
19. श्येत॑ग्रीव॒ स्ते ते श्येत॑ग्रीवः॒ श्येत॑ग्रीव॒ स्ते पि॑तृदेव॒त्याः᳚ पितृदेव॒त्या᳚ स्ते श्येत॑ग्रीवः॒ श्येत॑ग्रीव॒ स्ते पि॑तृदेव॒त्याः᳚ । \newline
20. श्येत॑ग्रीव॒ इति॒ श्येत॑ - ग्री॒वः॒ । \newline
21. ते पि॑तृदेव॒त्याः᳚ पितृदेव॒त्या᳚ स्ते ते पि॑तृदेव॒त्या᳚ स्ति॒स्र स्ति॒स्रः पि॑तृदेव॒त्या᳚ स्ते ते पि॑तृदेव॒त्या᳚ स्ति॒स्रः । \newline
22. पि॒तृ॒दे॒व॒त्या᳚ स्ति॒स्र स्ति॒स्रः पि॑तृदेव॒त्याः᳚ पितृदेव॒त्या᳚ स्ति॒स्रः कृ॒ष्णाः कृ॒ष्णा स्ति॒स्रः पि॑तृदेव॒त्याः᳚ पितृदेव॒त्या᳚ स्ति॒स्रः कृ॒ष्णाः । \newline
23. पि॒तृ॒दे॒व॒त्या॑ इति॑ पितृ - दे॒व॒त्याः᳚ । \newline
24. ति॒स्रः कृ॒ष्णाः कृ॒ष्णा स्ति॒स्र स्ति॒स्रः कृ॒ष्णा व॒शा व॒शाः कृ॒ष्णा स्ति॒स्र स्ति॒स्रः कृ॒ष्णा व॒शाः । \newline
25. कृ॒ष्णा व॒शा व॒शाः कृ॒ष्णाः कृ॒ष्णा व॒शा वा॑रु॒ण्यो॑ वारु॒ण्यो॑ व॒शाः कृ॒ष्णाः कृ॒ष्णा व॒शा वा॑रु॒ण्यः॑ । \newline
26. व॒शा वा॑रु॒ण्यो॑ वारु॒ण्यो॑ व॒शा व॒शा वा॑रु॒ण्य॑ स्ति॒स्र स्ति॒स्रो वा॑रु॒ण्यो॑ व॒शा व॒शा वा॑रु॒ण्य॑ स्ति॒स्रः । \newline
27. वा॒रु॒ण्य॑ स्ति॒स्र स्ति॒स्रो वा॑रु॒ण्यो॑ वारु॒ण्य॑ स्ति॒स्रः श्वे॒ताः श्वे॒ता स्ति॒स्रो वा॑रु॒ण्यो॑ वारु॒ण्य॑ स्ति॒स्रः श्वे॒ताः । \newline
28. ति॒स्रः श्वे॒ताः श्वे॒ता स्ति॒स्र स्ति॒स्रः श्वे॒ता व॒शा व॒शाः श्वे॒ता स्ति॒स्र स्ति॒स्रः श्वे॒ता व॒शाः । \newline
29. श्वे॒ता व॒शा व॒शाः श्वे॒ताः श्वे॒ता व॒शाः सौ॒र्यः॑ सौ॒र्यो॑ व॒शाः श्वे॒ताः श्वे॒ता व॒शाः सौ॒र्यः॑ । \newline
30. व॒शाः सौ॒र्यः॑ सौ॒र्यो॑ व॒शा व॒शाः सौ॒र्यो॑ मैत्राबार्.हस्प॒त्या मै᳚त्राबार्.हस्प॒त्याः सौ॒र्यो॑ व॒शा व॒शाः सौ॒र्यो॑ मैत्राबार्.हस्प॒त्याः । \newline
31. सौ॒र्यो॑ मैत्राबार्.हस्प॒त्या मै᳚त्राबार्.हस्प॒त्याः सौ॒र्यः॑ सौ॒र्यो॑ मैत्राबार्.हस्प॒त्या धू॒म्रल॑लामा धू॒म्रल॑लामा मैत्राबार्.हस्प॒त्याः सौ॒र्यः॑ सौ॒र्यो॑ मैत्राबार्.हस्प॒त्या धू॒म्रल॑लामाः । \newline
32. मै॒त्रा॒बा॒र्॒.ह॒स्प॒त्या धू॒म्रल॑लामा धू॒म्रल॑लामा मैत्राबार्.हस्प॒त्या मै᳚त्राबार्.हस्प॒त्या धू॒म्रल॑लामास्तूप॒रा स्तू॑प॒रा धू॒म्रल॑लामा मैत्राबार्.हस्प॒त्या मै᳚त्राबार्.हस्प॒त्या धू॒म्रल॑लामा स्तूप॒राः । \newline
33. मै॒त्रा॒बा॒र्.॒ह॒स्प॒त्या इति॑ मैत्रा - बा॒र्.॒ह॒स्प॒त्याः । \newline
34. धू॒म्रल॑लामा स्तूप॒रा स्तू॑प॒रा धू॒म्रल॑लामा धू॒म्रल॑लामा स्तूप॒राः । \newline
35. धू॒म्रल॑लामा॒ इति॑ धू॒म्र - ल॒ला॒माः॒ । \newline
36. तू॒प॒रा इति॑ तूप॒राः । \newline
\pagebreak
\markright{ TS 5.6.12.1  \hfill https://www.vedavms.in \hfill}

\section{ TS 5.6.12.1 }

\textbf{TS 5.6.12.1 } \newline
\textbf{Samhita Paata} \newline

पृश्नि॑-स्तिर॒श्चीन॑-पृश्निरू॒र्द्ध्व-पृ॑श्नि॒स्ते मा॑रु॒ताः फ॒ल्गूर्लो॑हितो॒र्णी ब॑ल॒क्षी ताः सा॑रस्व॒त्यः॑ पृष॑ती स्थू॒लपृ॑षती क्षु॒द्रपृ॑षती॒ ता वै᳚श्वदे॒व्य॑स्ति॒स्रः श्या॒मा व॒शाः पौ॒ष्णिय॑स्ति॒स्रो रोहि॑णीर्व॒शा मै॒त्रिय॑ ऐन्द्राबार्.हस्प॒त्या अ॑रु॒णल॑लामास्तूप॒राः ॥ \newline

\textbf{Pada Paata} \newline

पृश्निः॑ । ति॒र॒श्चीन॑पृश्नि॒रिति॑ तिर॒श्चीन॑ - पृ॒श्निः॒ । ऊ॒द्‌र्ध्वपृ॑श्नि॒रित्यू॒द्‌र्ध्व - पृ॒श्निः॒ । ते । मा॒रु॒ताः । फ॒ल्गूः । लो॒हि॒तो॒र्णीति॑ लोहित - ऊ॒र्णीः । ब॒ल॒क्षी । ताः । सा॒र॒स्व॒त्यः॑ । पृष॑ती । स्थू॒लपृ॑ष॒तीति॑ स्थू॒ल - पृ॒ष॒ती॒ । क्षु॒द्रपृ॑ष॒तीति॑ क्षु॒द्र-पृ॒ष॒ती॒ । ताः । वै॒श्व॒दे॒व्य॑ इति॑ वैश्व - दे॒व्यः॑ । ति॒स्रः । श्या॒माः । व॒शाः । पौ॒ष्णियः॑ । ति॒स्रः । रोहि॑णीः । व॒शाः । मै॒त्रियः॑ । ऐ॒न्द्रा॒बा॒र्.॒ह॒स्प॒त्या इत्यै᳚न्द्रा - बा॒र्.॒ह॒स्प॒त्याः । अ॒रु॒णल॑लामा॒ इत्य॑रु॒ण - ल॒ला॒माः॒ । तू॒प॒राः ॥  \newline


\textbf{Krama Paata} \newline

पृश्ञि॑स्तिर॒श्चीन॑पृश्ञिः । ति॒र॒श्चीन॑पृश्ञि,रू॒र्द्ध्वपृ॑श्ञिः । ति॒र॒श्चीन॑पृश्ञि॒रिति॑ तिर॒श्चीन॑ - पृ॒श्ञिः॒ । ऊ॒र्द्ध्वपृ॑श्ञि॒स्ते । ऊ॒र्द्ध्वपृ॑श्ञि॒रित्यू॒र्द्ध्व - पृ॒श्ञिः॒ । ते मा॑रु॒ताः । मा॒रु॒ताः फ॒ल्गूः । फ॒ल्गूर् लो॑हितो॒र्णी । लो॒हि॒तो॒र्णी ब॑ल॒क्षी । लो॒हि॒तो॒र्णीति॑ लोहित - ऊ॒र्णी । ब॒ल॒क्षी ताः । ताः सा॑रस्व॒त्यः॑ । सा॒र॒स्व॒त्यः॑ पृष॑ती । पृष॑ती स्थू॒लपृ॑षती । स्थू॒लपृ॑षती क्षु॒द्रपृ॑षती । स्थू॒लपृ॑ष॒तीति॑ स्थू॒ल - पृ॒ष॒ती॒ । क्षु॒द्रपृ॑षती॒ ताः । क्षु॒द्रपृ॑ष॒तीति॑ क्षु॒द्र - पृ॒ष॒ती॒ । ता वै᳚श्वदे॒व्यः॑ । वै॒श्व॒दे॒व्य॑स्ति॒स्रः । वै॒श्व॒दे॒व्य॑ इति॑ वैश्व - दे॒व्यः॑ । ति॒स्रः श्या॒माः । श्या॒मा व॒शाः । व॒शाः पौ॒ष्णियः॑ । पौ॒ष्णिय॑ स्ति॒स्रः । ति॒स्रो रोहि॑णीः । रोहि॑णीर् व॒शाः । व॒शा मै॒त्रियः॑ । मै॒त्रिय॑ ऐन्द्राबार्.हस्प॒त्याः । ऐ॒न्द्रा॒बा॒र्॒.ह॒स्प॒त्या अ॑रु॒णल॑लामाः । ऐ॒न्द्रा॒बा॒र्॒.ह॒स्प॒त्या इत्यै᳚न्द्रा - बा॒र्॒.ह॒स्प॒त्याः । अ॒रु॒णल॑लामास्तूप॒राः । अ॒रु॒णल॑लामा॒ इत्य॑रु॒ण - ल॒ला॒माः॒ । तू॒प॒रा इति॑ तूप॒राः । \newline

\textbf{Jatai Paata} \newline

1. पृश्ञि॑ स्तिर॒श्चीन॑पृश्ञि स्तिर॒श्चीन॑पृश्ञिः॒ पृश्ञिः॒ पृश्ञि॑ स्तिर॒श्चीन॑पृश्ञिः । \newline
2. ति॒र॒श्चीन॑पृश्ञि रू॒र्द्ध्वपृ॑श्ञि रू॒र्द्ध्वपृ॑श्ञि स्तिर॒श्चीन॑पृश्ञि स्तिर॒श्चीन॑पृश्ञि रू॒र्द्ध्वपृ॑श्ञिः । \newline
3. ति॒र॒श्चीन॑पृश्ञि॒रिति॑ तिर॒श्चीन॑ - पृ॒श्ञिः॒ । \newline
4. ऊ॒र्द्ध्वपृ॑श्ञि॒ स्ते त ऊ॒र्द्ध्वपृ॑श्ञि रू॒र्द्ध्वपृ॑श्ञि॒ स्ते । \newline
5. ऊ॒र्द्ध्वपृ॑श्ञि॒रित्यू॒र्द्ध्व - पृ॒श्ञिः॒ । \newline
6. ते मा॑रु॒ता मा॑रु॒ता स्ते ते मा॑रु॒ताः । \newline
7. मा॒रु॒ताः फ॒ल्गूः फ॒ल्गूर् मा॑रु॒ता मा॑रु॒ताः फ॒ल्गूः । \newline
8. फ॒ल्गूर् लो॑हितो॒र्णी लो॑हितो॒र्णी फ॒ल्गूः फ॒ल्गूर् लो॑हितो॒र्णी । \newline
9. लो॒हि॒तो॒र्णी ब॑ल॒क्षी ब॑ल॒क्षी लो॑हितो॒र्णी लो॑हितो॒र्णी ब॑ल॒क्षी । \newline
10. लो॒हि॒तो॒र्णीति॑ लोहित - ऊ॒र्णी । \newline
11. ब॒ल॒क्षी ता स्ता ब॑ल॒क्षी ब॑ल॒क्षी ताः । \newline
12. ताः सा॑रस्व॒त्यः॑ सारस्व॒त्य॑ स्ता स्ताः सा॑रस्व॒त्यः॑ । \newline
13. सा॒र॒स्व॒त्यः॑ पृष॑ती॒ पृष॑ती सारस्व॒त्यः॑ सारस्व॒त्यः॑ पृष॑ती । \newline
14. पृष॑ती स्थू॒लपृ॑षती स्थू॒लपृ॑षती॒ पृष॑ती॒ पृष॑ती स्थू॒लपृ॑षती । \newline
15. स्थू॒लपृ॑षती क्षु॒द्रपृ॑षती क्षु॒द्रपृ॑षती स्थू॒लपृ॑षती स्थू॒लपृ॑षती क्षु॒द्रपृ॑षती । \newline
16. स्थू॒लपृ॑ष॒तीति॑ स्थू॒ल - पृ॒ष॒ती॒ । \newline
17. क्षु॒द्रपृ॑षती॒ ता स्ताः क्षु॒द्रपृ॑षती क्षु॒द्रपृ॑षती॒ ताः । \newline
18. क्षु॒द्रपृ॑ष॒तीति॑ क्षु॒द्र - पृ॒ष॒ती॒ । \newline
19. ता वै᳚श्वदे॒व्यो॑ वैश्वदे॒व्य॑ स्ता स्ता वै᳚श्वदे॒व्यः॑ । \newline
20. वै॒श्व॒दे॒व्य॑ स्ति॒स्र स्ति॒स्रो वै᳚श्वदे॒व्यो॑ वैश्वदे॒व्य॑ स्ति॒स्रः । \newline
21. वै॒श्व॒दे॒व्य॑ इति॑ वैश्व - दे॒व्यः॑ । \newline
22. ति॒स्रः श्या॒माः श्या॒मा स्ति॒स्र स्ति॒स्रः श्या॒माः । \newline
23. श्या॒मा व॒शा व॒शाः श्या॒माः श्या॒मा व॒शाः । \newline
24. व॒शाः पौ॒ष्णियः॑ पौ॒ष्णियो॑ व॒शा व॒शाः पौ॒ष्णियः॑ । \newline
25. पौ॒ष्णिय॑ स्ति॒स्र स्ति॒स्रः पौ॒ष्णियः॑ पौ॒ष्णिय॑ स्ति॒स्रः । \newline
26. ति॒स्रो रोहि॑णी॒ रोहि॑णी स्ति॒स्र स्ति॒स्रो रोहि॑णीः । \newline
27. रोहि॑णीर् व॒शा व॒शा रोहि॑णी॒ रोहि॑णीर् व॒शाः । \newline
28. व॒शा मै॒त्रियो॑ मै॒त्रियो॑ व॒शा व॒शा मै॒त्रियः॑ । \newline
29. मै॒त्रिय॑ ऐन्द्राबार्.हस्प॒त्या ऐ᳚न्द्राबार्.हस्प॒त्या मै॒त्रियो॑ मै॒त्रिय॑ ऐन्द्राबार्.हस्प॒त्याः । \newline
30. ऐ॒न्द्रा॒बा॒र्॒.ह॒स्प॒त्या अ॑रु॒णल॑लामा अरु॒णल॑लामा ऐन्द्राबार्.हस्प॒त्या ऐ᳚न्द्राबार्.हस्प॒त्या अ॑रु॒णल॑लामाः । \newline
31. ऐ॒न्द्रा॒बा॒र्.॒ह॒स्प॒त्या इत्यै᳚न्द्रा - बा॒र्.॒ह॒स्प॒त्याः । \newline
32. अ॒रु॒णल॑लामा स्तूप॒रा स्तू॑प॒रा अ॑रु॒णल॑लामा अरु॒णल॑लामा स्तूप॒राः । \newline
33. अ॒रु॒णल॑लामा॒ इत्य॑रु॒ण - ल॒ला॒माः॒ । \newline
34. तू॒प॒रा इति॑ तूप॒राः । \newline

\textbf{Ghana Paata } \newline

1. पृश्ञि॑ स्तिर॒श्चीन॑पृश्ञि स्तिर॒श्चीन॑पृश्ञिः॒ पृश्ञिः॒ पृश्ञि॑ स्तिर॒श्चीन॑पृश्ञि रू॒र्द्ध्वपृ॑श्ञि रू॒र्द्ध्वपृ॑श्ञि स्तिर॒श्चीन॑पृश्ञिः॒ पृश्ञिः॒ पृश्ञि॑ स्तिर॒श्चीन॑पृश्ञि रू॒र्द्ध्वपृ॑श्ञिः । \newline
2. ति॒र॒श्चीन॑पृश्ञि रू॒र्द्ध्वपृ॑श्ञि रू॒र्द्ध्वपृ॑श्ञि स्तिर॒श्चीन॑पृश्ञि स्तिर॒श्चीन॑पृश्ञि रू॒र्द्ध्वपृ॑श्ञि॒ स्ते त ऊ॒र्द्ध्वपृ॑श्ञि स्तिर॒श्चीन॑पृश्ञिस्ति र॒श्चीन॑पृश्ञि रू॒र्द्ध्वपृ॑श्ञि॒ स्ते । \newline
3. ति॒र॒श्चीन॑पृश्ञि॒रिति॑ तिर॒श्चीन॑ - पृ॒श्ञिः॒ । \newline
4. ऊ॒र्द्ध्वपृ॑श्ञि॒ स्ते त ऊ॒र्द्ध्वपृ॑श्ञि रू॒र्द्ध्वपृ॑श्ञि॒ स्ते मा॑रु॒ता मा॑रु॒ता स्त ऊ॒र्द्ध्वपृ॑श्ञि रू॒र्द्ध्वपृ॑श्ञि॒ स्ते मा॑रु॒ताः । \newline
5. ऊ॒र्द्ध्वपृ॑श्ञि॒रित्यू॒र्द्ध्व - पृ॒श्ञिः॒ । \newline
6. ते मा॑रु॒ता मा॑रु॒ता स्ते ते मा॑रु॒ताः फ॒ल्गूः फ॒ल्गूर् मा॑रु॒ता स्ते ते मा॑रु॒ताः फ॒ल्गूः । \newline
7. मा॒रु॒ताः फ॒ल्गूः फ॒ल्गूर् मा॑रु॒ता मा॑रु॒ताः फ॒ल्गूर् लो॑हितो॒र्णी लो॑हितो॒र्णी फ॒ल्गूर् मा॑रु॒ता मा॑रु॒ताः फ॒ल्गूर् लो॑हितो॒र्णी । \newline
8. फ॒ल्गूर् लो॑हितो॒र्णी लो॑हितो॒र्णी फ॒ल्गूः फ॒ल्गूर् लो॑हितो॒र्णी ब॑ल॒क्षी ब॑ल॒क्षी लो॑हितो॒र्णी फ॒ल्गूः फ॒ल्गूर् लो॑हितो॒र्णी ब॑ल॒क्षी । \newline
9. लो॒हि॒तो॒र्णी ब॑ल॒क्षी ब॑ल॒क्षी लो॑हितो॒र्णी लो॑हितो॒र्णी ब॑ल॒क्षी ता स्ता ब॑ल॒क्षी लो॑हितो॒र्णी लो॑हितो॒र्णी ब॑ल॒क्षी ताः । \newline
10. लो॒हि॒तो॒र्णीति॑ लोहित - ऊ॒र्णी । \newline
11. ब॒ल॒क्षी ता स्ता ब॑ल॒क्षी ब॑ल॒क्षी ताः सा॑रस्व॒त्यः॑ सारस्व॒त्य॑ स्ता ब॑ल॒क्षी ब॑ल॒क्षी ताः सा॑रस्व॒त्यः॑ । \newline
12. ताः सा॑रस्व॒त्यः॑ सारस्व॒त्य॑ स्ता स्ताः सा॑रस्व॒त्यः॑ पृष॑ती॒ पृष॑ती सारस्व॒त्य॑ स्ता स्ताः सा॑रस्व॒त्यः॑ पृष॑ती । \newline
13. सा॒र॒स्व॒त्यः॑ पृष॑ती॒ पृष॑ती सारस्व॒त्यः॑ सारस्व॒त्यः॑ पृष॑ती स्थू॒लपृ॑षती स्थू॒लपृ॑षती॒ पृष॑ती सारस्व॒त्यः॑ सारस्व॒त्यः॑ पृष॑ती स्थू॒लपृ॑षती । \newline
14. पृष॑ती स्थू॒लपृ॑षती स्थू॒लपृ॑षती॒ पृष॑ती॒ पृष॑ती स्थू॒लपृ॑षती क्षु॒द्रपृ॑षती क्षु॒द्रपृ॑षती स्थू॒लपृ॑षती॒ पृष॑ती॒ पृष॑ती स्थू॒लपृ॑षती क्षु॒द्रपृ॑षती । \newline
15. स्थू॒लपृ॑षती क्षु॒द्रपृ॑षती क्षु॒द्रपृ॑षती स्थू॒लपृ॑षती स्थू॒लपृ॑षती क्षु॒द्रपृ॑षती॒ ता स्ताः क्षु॒द्रपृ॑षती॒ स्थू॒लपृ॑षती स्थू॒लपृ॑षती क्षु॒द्रपृ॑षती॒ ताः । \newline
16. स्थू॒लपृ॑ष॒तीति॑ स्थू॒ल - पृ॒ष॒ती॒ । \newline
17. क्षु॒द्रपृ॑षती॒ ता स्ताः क्षु॒द्रपृ॑षती क्षु॒द्रपृ॑षती॒ ता वै᳚श्वदे॒व्यो॑ वैश्वदे॒व्य॑ स्ताः क्षु॒द्रपृ॑षती क्षु॒द्रपृ॑षती॒ ता वै᳚श्वदे॒व्यः॑ । \newline
18. क्षु॒द्रपृ॑ष॒तीति॑ क्षु॒द्र - पृ॒ष॒ती॒ । \newline
19. ता वै᳚श्वदे॒व्यो॑ वैश्वदे॒व्य॑ स्ता स्ता वै᳚श्वदे॒व्य॑ स्ति॒स्र स्ति॒स्रो वै᳚श्वदे॒व्य॑ स्ता स्ता वै᳚श्वदे॒व्य॑ स्ति॒स्रः । \newline
20. वै॒श्व॒दे॒व्य॑ स्ति॒स्र स्ति॒स्रो वै᳚श्वदे॒व्यो॑ वैश्वदे॒व्य॑ स्ति॒स्रः श्या॒माः श्या॒मा स्ति॒स्रो वै᳚श्वदे॒व्यो॑ वैश्वदे॒व्य॑ स्ति॒स्रः श्या॒माः । \newline
21. वै॒श्व॒दे॒व्य॑ इति॑ वैश्व - दे॒व्यः॑ । \newline
22. ति॒स्रः श्या॒माः श्या॒मा स्ति॒स्र स्ति॒स्रः श्या॒मा व॒शा व॒शाः श्या॒मा स्ति॒स्र स्ति॒स्रः श्या॒मा व॒शाः । \newline
23. श्या॒मा व॒शा व॒शाः श्या॒माः श्या॒मा व॒शाः पौ॒ष्णियः॑ पौ॒ष्णियो॑ व॒शाः श्या॒माः श्या॒मा व॒शाः पौ॒ष्णियः॑ । \newline
24. व॒शाः पौ॒ष्णियः॑ पौ॒ष्णियो॑ व॒शा व॒शाः पौ॒ष्णिय॑ स्ति॒स्र स्ति॒स्रः पौ॒ष्णियो॑ व॒शा व॒शाः पौ॒ष्णिय॑ स्ति॒स्रः । \newline
25. पौ॒ष्णिय॑ स्ति॒स्र स्ति॒स्रः पौ॒ष्णियः॑ पौ॒ष्णिय॑ स्ति॒स्रो रोहि॑णी॒ रोहि॑णी स्ति॒स्रः पौ॒ष्णियः॑ पौ॒ष्णिय॑ स्ति॒स्रो रोहि॑णीः । \newline
26. ति॒स्रो रोहि॑णी॒ रोहि॑णी स्ति॒स्र स्ति॒स्रो रोहि॑णीर् व॒शा व॒शा रोहि॑णी स्ति॒स्र 
स्ति॒स्रो रोहि॑णीर् व॒शाः । \newline
27. रोहि॑णीर् व॒शा व॒शा रोहि॑णी॒ रोहि॑णीर् व॒शा मै॒त्रियो॑ मै॒त्रियो॑ व॒शा रोहि॑णी॒ रोहि॑णीर् व॒शा मै॒त्रियः॑ । \newline
28. व॒शा मै॒त्रियो॑ मै॒त्रियो॑ व॒शा व॒शा मै॒त्रिय॑ ऐन्द्राबार्.हस्प॒त्या ऐ᳚न्द्राबार्.हस्प॒त्या मै॒त्रियो॑ व॒शा व॒शा मै॒त्रिय॑ ऐन्द्राबार्.हस्प॒त्याः । \newline
29. मै॒त्रिय॑ ऐन्द्राबार्.हस्प॒त्या ऐ᳚न्द्राबार्.हस्प॒त्या मै॒त्रियो॑ मै॒त्रिय॑ ऐन्द्राबार्.हस्प॒त्या अ॑रु॒णल॑लामा अरु॒णल॑लामा ऐन्द्राबार्.हस्प॒त्या मै॒त्रियो॑ मै॒त्रिय॑ ऐन्द्राबार्.हस्प॒त्या अ॑रु॒णल॑लामाः । \newline
30. ऐ॒न्द्रा॒बा॒र्॒.ह॒स्प॒त्या अ॑रु॒णल॑लामा अरु॒णल॑लामा ऐन्द्राबार्.हस्प॒त्या ऐ᳚न्द्राबार्.हस्प॒त्या अ॑रु॒णल॑लामा स्तूप॒रा स्तू॑प॒रा अ॑रु॒णल॑लामा ऐन्द्राबार्.हस्प॒त्या ऐ᳚न्द्राबार्.हस्प॒त्या अ॑रु॒णल॑लामा स्तूप॒राः । \newline
31. ऐ॒न्द्रा॒बा॒र्.॒ह॒स्प॒त्या इत्यै᳚न्द्रा - बा॒र्.॒ह॒स्प॒त्याः । \newline
32. अ॒रु॒णल॑लामा स्तूप॒रा स्तू॑प॒रा अ॑रु॒णल॑लामा अरु॒णल॑लामा स्तूप॒राः । \newline
33. अ॒रु॒णल॑लामा॒ इत्य॑रु॒ण - ल॒ला॒माः॒ । \newline
34. तू॒प॒रा इति॑ तूप॒राः । \newline
\pagebreak
\markright{ TS 5.6.13.1  \hfill https://www.vedavms.in \hfill}

\section{ TS 5.6.13.1 }

\textbf{TS 5.6.13.1 } \newline
\textbf{Samhita Paata} \newline

शि॒ति॒बा॒हु-र॒न्यत॑श्शितिबाहुः सम॒न्त शि॑तिबाहु॒स्त ऐ᳚न्द्रवाय॒वाः शि॑ति॒रन्ध्रो॒ ऽन्यत॑श्शितिरन्ध्रः सम॒न्तशि॑तिरन्ध्र॒स्ते मै᳚त्रावरु॒णाः शु॒द्धवा॑लः स॒र्वशु॑द्धवालो म॒णिवा॑ल॒स्त आ᳚श्वि॒नास्ति॒स्रः शि॒ल्पा व॒शा वै᳚श्वदे॒व्य॑स्ति॒स्रः श्येनीः᳚ परमे॒ष्ठिने॑ सोमापौ॒ष्णाः श्या॒मल॑लामास्तूप॒राः ॥ \newline

\textbf{Pada Paata} \newline

शि॒ति॒बा॒हुरिति॑ शिति - बा॒हुः । अ॒न्यत॑श्शितिबाहु॒रित्य॒न्यतः॑ -शि॒ति॒बा॒हुः॒ । स॒म॒न्तशि॑तिबाहु॒रिति॑ सम॒न्त - शि॒ति॒बा॒हुः॒ । ते । ऐ॒न्द्र॒वा॒य॒वा इत्यै᳚न्द्र - वा॒य॒वाः । शि॒ति॒रन्ध्र॒ इति॑ शिति - रन्ध्रः॑ । अ॒न्यत॑श्शितिरन्ध्र॒ इत्य॒न्यतः॑ - शि॒ति॒र॒न्ध्रः॒ । स॒म॒न्तशि॑तिरन्ध्र॒ इति॑ सम॒न्त - शि॒ति॒र॒न्ध्रः॒ । ते । मै॒त्रा॒व॒रु॒णा इति॑ मैत्रा - व॒रु॒णाः । शु॒द्धवा॑ल॒ इति॑ शु॒द्ध-वा॒लः॒ । स॒र्वशु॑द्धवाल॒ इति॑ स॒र्व-शु॒द्ध॒वा॒लः॒ । म॒णिवा॑ल॒ इति॑ म॒णि - वा॒लः॒ । ते । आ॒श्वि॒नाः । ति॒स्रः । शि॒ल्पाः । व॒शाः । वै॒श्व॒दे॒व्य॑ इति॑ वैश्व - दे॒व्यः॑ । ति॒स्रः । श्येनीः᳚ । प॒र॒मे॒ष्ठिने᳚ । सो॒मा॒पौ॒ष्णा इति॑ सोमा - पौ॒ष्णाः । श्या॒मल॑लामा॒ इति॑ श्या॒म - ल॒ला॒माः॒ । तू॒प॒राः ॥  \newline


\textbf{Krama Paata} \newline

शि॒ति॒बा॒हुर॒न्यत॑श्शितिबाहुः । शि॒ति॒बा॒हुरिति॑ शिति - बा॒हुः । अ॒न्यत॑श्शितिबाहुः सम॒न्तशि॑तिबाहुः । अ॒न्यत॑श्शितिबाहु॒रित्य॒न्यतः॑ - शि॒ति॒बा॒हुः॒ । स॒म॒न्तशि॑तिबाहु॒स्ते । स॒म॒न्तशि॑तिबाहु॒रिति॑ सम॒न्त - शि॒ति॒बा॒हुः॒ । त ऐ᳚न्द्रवाय॒वाः । ऐ॒न्द्रा॒वा॒य॒वाः शि॑ति॒रन्ध्रः॑ । ऐ॒न्द्र॒वा॒य॒वा इत्यै᳚न्द्र - वा॒य॒वाः । शि॒ति॒रन्ध्रो॒ऽन्यत॑श्शितिरन्ध्रः । शि॒ति॒रन्ध्र॒ इति॑ शिति - रन्ध्रः॑ । अ॒न्यत॑श्शितिरन्ध्रः सम॒न्तशि॑तिरन्ध्रः । अ॒न्यत॑श्शितिरन्ध्र॒ इत्य॒न्यतः॑ - शि॒ति॒र॒न्ध्रः॒ । स॒म॒न्तशि॑तिरन्ध्र॒स्ते । स॒म॒न्तशि॑तिरन्ध्र॒ इति॑ सम॒न्त - शि॒ति॒र॒न्ध्रः॒ । ते मै᳚त्रावरु॒णाः । मै॒त्रा॒व॒रु॒णाः शु॒द्धवा॑लः । मै॒त्रा॒व॒रु॒णा इति॑ मैत्रा - व॒रु॒णाः । शु॒द्धवा॑लः स॒र्वशु॑द्धवालः । शु॒द्धवा॑ल॒ इति॑ शु॒द्ध - वा॒लः॒ । स॒र्वशु॑द्धवालो म॒णिवा॑लः । स॒र्वशु॑द्धवाल॒ इति॑ स॒र्व - शु॒द्ध॒वा॒लः॒ । म॒णिवा॑ल॒स्ते । म॒णिवा॑ल॒ इति॑ म॒णि - वा॒लः॒ । त आ᳚श्वि॒नाः । आ॒श्वि॒नास्ति॒स्रः । ति॒स्रः शि॒ल्पाः । शि॒ल्पा व॒शाः । व॒शा वै᳚श्वदे॒व्यः॑ । वै॒श्व॒दे॒व्य॑स्ति॒स्रः । वै॒श्व॒दे॒व्य॑ इति॑ वैश्व - दे॒व्यः॑ । ति॒स्रः श्येनीः᳚ । श्येनीः᳚ परमे॒ष्ठिने᳚ । प॒र॒मे॒ष्ठिने॑ सोमापौ॒ष्णाः । सो॒मा॒पौ॒ष्णाः श्या॒मल॑लामाः । सो॒मा॒पौ॒ष्णा इति॑ सोमा - पौ॒ष्णाः । श्या॒मल॑लामास्तूप॒राः । श्या॒मल॑लामा॒ इति॑ श्या॒म - ल॒ला॒माः॒ । तू॒प॒रा इति॑ तूप॒राः । \newline

\textbf{Jatai Paata} \newline

1. शि॒ति॒बा॒हु र॒न्यत॑श्शितिबाहु र॒न्यत॑श्शितिबाहुः शितिबा॒हुः शि॑तिबा॒हु र॒न्यत॑श्शितिबाहुः । \newline
2. शि॒ति॒बा॒हुरिति॑ शिति - बा॒हुः । \newline
3. अ॒न्यत॑श्शितिबाहुः सम॒न्तशि॑तिबाहुः सम॒न्तशि॑तिबाहु र॒न्यत॑श्शितिबाहु र॒न्यत॑श्शितिबाहुः सम॒न्तशि॑तिबाहुः । \newline
4. अ॒न्यत॑श्शितिबाहु॒रित्य॒न्यतः॑ - शि॒ति॒बा॒हुः॒ । \newline
5. स॒म॒न्तशि॑तिबाहु॒ स्ते ते स॑म॒न्तशि॑तिबाहुः सम॒न्तशि॑तिबाहु॒ स्ते । \newline
6. स॒म॒न्तशि॑तिबाहु॒रिति॑ सम॒न्त - शि॒ति॒बा॒हुः॒ । \newline
7. त ऐ᳚न्द्रवाय॒वा ऐ᳚न्द्रवाय॒वा स्ते त ऐ᳚न्द्रवाय॒वाः । \newline
8. ऐ॒न्द्र॒वा॒य॒वाः शि॑ति॒रन्ध्रः॑ शिति॒रन्ध्र॑ ऐन्द्रवाय॒वा ऐ᳚न्द्रवाय॒वाः शि॑ति॒रन्ध्रः॑ । \newline
9. ऐ॒न्द्र॒वा॒य॒वा इत्यै᳚न्द्र - वा॒य॒वाः । \newline
10. शि॒ति॒रन्ध्रो॒ ऽन्यत॑श्शितिरन्ध्रो॒ ऽन्यत॑श्शितिरन्ध्रः शिति॒रन्ध्रः॑ शिति॒रन्ध्रो॒ ऽन्यत॑श्शितिरन्ध्रः । \newline
11. शि॒ति॒रन्ध्र॒ इति॑ शिति - रन्ध्रः॑ । \newline
12. अ॒न्यत॑श्शितिरन्ध्रः सम॒न्तशि॑तिरन्ध्रः सम॒न्तशि॑तिरन्ध्रो॒ ऽन्यत॑श्शितिरन्ध्रो॒ ऽन्यत॑श्शितिरन्ध्रः सम॒न्तशि॑तिरन्ध्रः । \newline
13. अ॒न्यत॑श्शितिरन्ध्र॒ इत्य॒न्यतः॑ - शि॒ति॒र॒न्ध्रः॒ । \newline
14. स॒म॒न्तशि॑तिरन्ध्र॒ स्ते ते स॑म॒न्तशि॑तिरन्ध्रः सम॒न्तशि॑तिरन्ध्र॒ स्ते । \newline
15. स॒म॒न्तशि॑तिरन्ध्र॒ इति॑ सम॒न्त - शि॒ति॒र॒न्ध्रः॒ । \newline
16. ते मै᳚त्रावरु॒णा मै᳚त्रावरु॒णा स्ते ते मै᳚त्रावरु॒णाः । \newline
17. मै॒त्रा॒व॒रु॒णाः शु॒द्धवा॑लः शु॒द्धवा॑लो मैत्रावरु॒णा मै᳚त्रावरु॒णाः शु॒द्धवा॑लः । \newline
18. मै॒त्रा॒व॒रु॒णा इति॑ मैत्रा - व॒रु॒णाः । \newline
19. शु॒द्धवा॑लः स॒र्वशु॑द्धवालः स॒र्वशु॑द्धवालः शु॒द्धवा॑लः शु॒द्धवा॑लः स॒र्वशु॑द्धवालः । \newline
20. शु॒द्धवा॑ल॒ इति॑ शु॒द्ध - वा॒लः॒ । \newline
21. स॒र्वशु॑द्धवालो म॒णिवा॑लो म॒णिवा॑लः स॒र्वशु॑द्धवालः स॒र्वशु॑द्धवालो म॒णिवा॑लः । \newline
22. स॒र्वशु॑द्धवाल॒ इति॑ स॒र्व - शु॒द्ध॒वा॒लः॒ । \newline
23. म॒णिवा॑ल॒ स्ते ते म॒णिवा॑लो म॒णिवा॑ल॒ स्ते । \newline
24. म॒णिवा॑ल॒ इति॑ म॒णि - वा॒लः॒ । \newline
25. त आ᳚श्वि॒ना आ᳚श्वि॒ना स्ते त आ᳚श्वि॒नाः । \newline
26. आ॒श्वि॒ना स्ति॒स्र स्ति॒स्र आ᳚श्वि॒ना आ᳚श्वि॒ना स्ति॒स्रः । \newline
27. ति॒स्रः शि॒ल्पाः शि॒ल्पा स्ति॒स्र स्ति॒स्रः शि॒ल्पाः । \newline
28. शि॒ल्पा व॒शा व॒शाः शि॒ल्पाः शि॒ल्पा व॒शाः । \newline
29. व॒शा वै᳚श्वदे॒व्यो॑ वैश्वदे॒व्यो॑ व॒शा व॒शा वै᳚श्वदे॒व्यः॑ । \newline
30. वै॒श्व॒दे॒व्य॑ स्ति॒स्र स्ति॒स्रो वै᳚श्वदे॒व्यो॑ वैश्वदे॒व्य॑ स्ति॒स्रः । \newline
31. वै॒श्व॒दे॒व्य॑ इति॑ वैश्व - दे॒व्यः॑ । \newline
32. ति॒स्रः श्येनीः॒ श्येनी᳚ स्ति॒स्र स्ति॒स्रः श्येनीः᳚ । \newline
33. श्येनीः᳚ परमे॒ष्ठिने॑ परमे॒ष्ठिने॒ श्येनीः॒ श्येनीः᳚ परमे॒ष्ठिने᳚ । \newline
34. प॒र॒मे॒ष्ठिने॑ सोमापौ॒ष्णाः सो॑मापौ॒ष्णाः प॑रमे॒ष्ठिने॑ परमे॒ष्ठिने॑ सोमापौ॒ष्णाः । \newline
35. सो॒मा॒पौ॒ष्णाः श्या॒मल॑लामाः श्या॒मल॑लामाः सोमापौ॒ष्णाः सो॑मापौ॒ष्णाः श्या॒मल॑लामाः । \newline
36. सो॒मा॒पौ॒ष्णा इति॑ सोमा - पौ॒ष्णाः । \newline
37. श्या॒मल॑लामा स्तूप॒रा स्तू॑प॒राः श्या॒मल॑लामाः श्या॒मल॑लामा स्तूप॒राः । \newline
38. श्या॒मल॑लामा॒ इति॑ श्या॒म - ल॒ला॒माः॒ । \newline
39. तू॒प॒रा इति॑ तूप॒राः । \newline

\textbf{Ghana Paata } \newline

1. शि॒ति॒बा॒हु र॒न्यत॑श्शितिबाहु र॒न्यत॑श्शितिबाहुः शितिबा॒हुः शि॑तिबा॒हु र॒न्यत॑श्शितिबाहुः सम॒न्तशि॑तिबाहुः सम॒न्तशि॑तिबाहु र॒न्यत॑श्शितिबाहुः शितिबा॒हुः शि॑तिबा॒हु र॒न्यत॑श्शितिबाहुः सम॒न्तशि॑तिबाहुः । \newline
2. शि॒ति॒बा॒हुरिति॑ शिति - बा॒हुः । \newline
3. अ॒न्यत॑श्शितिबाहुः सम॒न्तशि॑तिबाहुः सम॒न्तशि॑तिबाहु र॒न्यत॑श्शितिबाहु र॒न्यत॑श्शितिबाहुः सम॒न्तशि॑तिबाहु॒ स्ते ते स॑म॒न्तशि॑तिबाहु र॒न्यत॑श्शितिबाहु र॒न्यत॑श्शितिबाहुः सम॒न्तशि॑तिबाहु॒ स्ते । \newline
4. अ॒न्यत॑श्शितिबाहु॒रित्य॒न्यतः॑ - शि॒ति॒बा॒हुः॒ । \newline
5. स॒म॒न्तशि॑तिबाहु॒ स्ते ते स॑म॒न्तशि॑तिबाहुः सम॒न्तशि॑तिबाहु॒ स्त ऐ᳚न्द्रवाय॒वा ऐ᳚न्द्रवाय॒वा स्ते स॑म॒न्तशि॑तिबाहुः सम॒न्तशि॑तिबाहु॒ स्त ऐ᳚न्द्रवाय॒वाः । \newline
6. स॒म॒न्तशि॑तिबाहु॒रिति॑ सम॒न्त - शि॒ति॒बा॒हुः॒ । \newline
7. त ऐ᳚न्द्रवाय॒वा ऐ᳚न्द्रवाय॒वा स्ते त ऐ᳚न्द्रवाय॒वाः शि॑ति॒रन्ध्रः॑ शिति॒रन्ध्र॑ ऐन्द्रवाय॒वा स्ते त ऐ᳚न्द्रवाय॒वाः शि॑ति॒रन्ध्रः॑ । \newline
8. ऐ॒न्द्र॒वा॒य॒वाः शि॑ति॒रन्ध्रः॑ शिति॒रन्ध्र॑ ऐन्द्रवाय॒वा ऐ᳚न्द्रवाय॒वाः शि॑ति॒रन्ध्रो॒ ऽन्यत॑श्शितिरन्ध्रो॒ ऽन्यत॑श्शितिरन्ध्रः शिति॒रन्ध्र॑ ऐन्द्रवाय॒वा ऐ᳚न्द्रवाय॒वाः शि॑ति॒रन्ध्रो॒ ऽन्यत॑श्शितिरन्ध्रः । \newline
9. ऐ॒न्द्र॒वा॒य॒वा इत्यै᳚न्द्र - वा॒य॒वाः । \newline
10. शि॒ति॒रन्ध्रो॒ ऽन्यत॑श्शितिरन्ध्रो॒ ऽन्यत॑श्शितिरन्ध्रः शिति॒रन्ध्रः॑ शिति॒रन्ध्रो॒ ऽन्यत॑श्शितिरन्ध्रः सम॒न्तशि॑तिरन्ध्रः सम॒न्तशि॑तिरन्ध्रो॒ ऽन्यत॑श्शितिरन्ध्रः शिति॒रन्ध्रः॑ शिति॒रन्ध्रो॒ ऽन्यत॑श्शितिरन्ध्रः सम॒न्तशि॑तिरन्ध्रः । \newline
11. शि॒ति॒रन्ध्र॒ इति॑ शिति - रन्ध्रः॑ । \newline
12. अ॒न्यत॑श्शितिरन्ध्रः सम॒न्तशि॑तिरन्ध्रः सम॒न्तशि॑तिरन्ध्रो॒ ऽन्यत॑श्शितिरन्ध्रो॒ ऽन्यत॑श्शितिरन्ध्रः सम॒न्तशि॑तिरन्ध्र॒ स्ते ते स॑म॒न्तशि॑तिरन्ध्रो॒ ऽन्यत॑श्शितिरन्ध्रो॒ ऽन्यत॑श्शितिरन्ध्रः सम॒न्तशि॑तिरन्ध्र॒ स्ते । \newline
13. अ॒न्यत॑श्शितिरन्ध्र॒ इत्य॒न्यतः॑ - शि॒ति॒र॒न्ध्रः॒ । \newline
14. स॒म॒न्तशि॑तिरन्ध्र॒ स्ते ते स॑म॒न्तशि॑तिरन्ध्रः सम॒न्तशि॑तिरन्ध्र॒ स्ते मै᳚त्रावरु॒णा मै᳚त्रावरु॒णा स्ते स॑म॒न्तशि॑तिरन्ध्रः सम॒न्तशि॑तिरन्ध्र॒ स्ते मै᳚त्रावरु॒णाः । \newline
15. स॒म॒न्तशि॑तिरन्ध्र॒ इति॑ सम॒न्त - शि॒ति॒र॒न्ध्रः॒ । \newline
16. ते मै᳚त्रावरु॒णा मै᳚त्रावरु॒णा स्ते ते मै᳚त्रावरु॒णाः शु॒द्धवा॑लः शु॒द्धवा॑लो मैत्रावरु॒णा स्ते ते मै᳚त्रावरु॒णाः शु॒द्धवा॑लः । \newline
17. मै॒त्रा॒व॒रु॒णाः शु॒द्धवा॑लः शु॒द्धवा॑लो मैत्रावरु॒णा मै᳚त्रावरु॒णाः शु॒द्धवा॑लः स॒र्वशु॑द्धवालः स॒र्वशु॑द्धवालः शु॒द्धवा॑लो मैत्रावरु॒णा मै᳚त्रावरु॒णाः शु॒द्धवा॑लः स॒र्वशु॑द्धवालः । \newline
18. मै॒त्रा॒व॒रु॒णा इति॑ मैत्रा - व॒रु॒णाः । \newline
19. शु॒द्धवा॑लः स॒र्वशु॑द्धवालः स॒र्वशु॑द्धवालः शु॒द्धवा॑लः शु॒द्धवा॑लः स॒र्वशु॑द्धवालो म॒णिवा॑लो म॒णिवा॑लः स॒र्वशु॑द्धवालः शु॒द्धवा॑लः शु॒द्धवा॑लः स॒र्वशु॑द्धवालो म॒णिवा॑लः । \newline
20. शु॒द्धवा॑ल॒ इति॑ शु॒द्ध - वा॒लः॒ । \newline
21. स॒र्वशु॑द्धवालो म॒णिवा॑लो म॒णिवा॑लः स॒र्वशु॑द्धवालः स॒र्वशु॑द्धवालो म॒णिवा॑ल॒ स्ते ते म॒णिवा॑लः स॒र्वशु॑द्धवालः स॒र्वशु॑द्धवालो म॒णिवा॑ल॒ स्ते । \newline
22. स॒र्वशु॑द्धवाल॒ इति॑ स॒र्व - शु॒द्ध॒वा॒लः॒ । \newline
23. म॒णिवा॑ल॒ स्ते ते म॒णिवा॑लो म॒णिवा॑ल॒ स्त आ᳚श्वि॒ना आ᳚श्वि॒ना स्ते म॒णिवा॑लो म॒णिवा॑ल॒ स्त आ᳚श्वि॒नाः । \newline
24. म॒णिवा॑ल॒ इति॑ म॒णि - वा॒लः॒ । \newline
25. त आ᳚श्वि॒ना आ᳚श्वि॒ना स्ते त आ᳚श्वि॒ना स्ति॒स्र स्ति॒स्र आ᳚श्वि॒ना स्ते त आ᳚श्वि॒ना स्ति॒स्रः । \newline
26. आ॒श्वि॒ना स्ति॒स्र स्ति॒स्र आ᳚श्वि॒ना आ᳚श्वि॒ना स्ति॒स्रः शि॒ल्पाः शि॒ल्पा स्ति॒स्र आ᳚श्वि॒ना आ᳚श्वि॒ना स्ति॒स्रः शि॒ल्पाः । \newline
27. ति॒स्रः शि॒ल्पाः शि॒ल्पा स्ति॒स्र स्ति॒स्रः शि॒ल्पा व॒शा व॒शाः शि॒ल्पा स्ति॒स्र स्ति॒स्रः शि॒ल्पा व॒शाः । \newline
28. शि॒ल्पा व॒शा व॒शाः शि॒ल्पाः शि॒ल्पा व॒शा वै᳚श्वदे॒व्यो॑ वैश्वदे॒व्यो॑ व॒शाः शि॒ल्पाः शि॒ल्पा व॒शा वै᳚श्वदे॒व्यः॑ । \newline
29. व॒शा वै᳚श्वदे॒व्यो॑ वैश्वदे॒व्यो॑ व॒शा व॒शा वै᳚श्वदे॒व्य॑ स्ति॒स्र स्ति॒स्रो वै᳚श्वदे॒व्यो॑ व॒शा व॒शा वै᳚श्वदे॒व्य॑ स्ति॒स्रः । \newline
30. वै॒श्व॒दे॒व्य॑ स्ति॒स्र स्ति॒स्रो वै᳚श्वदे॒व्यो॑ वैश्वदे॒व्य॑ स्ति॒स्रः श्येनीः॒ श्येनी᳚ स्ति॒स्रो वै᳚श्वदे॒व्यो॑ वैश्वदे॒व्य॑ स्ति॒स्रः श्येनीः᳚ । \newline
31. वै॒श्व॒दे॒व्य॑ इति॑ वैश्व - दे॒व्यः॑ । \newline
32. ति॒स्रः श्येनीः॒ श्येनी᳚ स्ति॒स्र स्ति॒स्रः श्येनीः᳚ परमे॒ष्ठिने॑ परमे॒ष्ठिने॒ श्येनी᳚ स्ति॒स्र स्ति॒स्रः श्येनीः᳚ परमे॒ष्ठिने᳚ । \newline
33. श्येनीः᳚ परमे॒ष्ठिने॑ परमे॒ष्ठिने॒ श्येनीः॒ श्येनीः᳚ परमे॒ष्ठिने॑ सोमापौ॒ष्णाः सो॑मापौ॒ष्णाः प॑रमे॒ष्ठिने॒ श्येनीः॒ श्येनीः᳚ परमे॒ष्ठिने॑ सोमापौ॒ष्णाः । \newline
34. प॒र॒मे॒ष्ठिने॑ सोमापौ॒ष्णाः सो॑मापौ॒ष्णाः प॑रमे॒ष्ठिने॑ परमे॒ष्ठिने॑ सोमापौ॒ष्णाः श्या॒मल॑लामाः श्या॒मल॑लामाः सोमापौ॒ष्णाः प॑रमे॒ष्ठिने॑ परमे॒ष्ठिने॑ सोमापौ॒ष्णाः श्या॒मल॑लामाः । \newline
35. सो॒मा॒पौ॒ष्णाः श्या॒मल॑लामाः श्या॒मल॑लामाः सोमापौ॒ष्णाः सो॑मापौ॒ष्णाः श्या॒मल॑लामा स्तूप॒रा स्तू॑प॒राः श्या॒मल॑लामाः सोमापौ॒ष्णाः सो॑मापौ॒ष्णाः श्या॒मल॑लामा स्तूप॒राः । \newline
36. सो॒मा॒पौ॒ष्णा इति॑ सोमा - पौ॒ष्णाः । \newline
37. श्या॒मल॑लामा स्तूप॒रा स्तू॑प॒राः श्या॒मल॑लामाः श्या॒मल॑लामा स्तूप॒राः । \newline
38. श्या॒मल॑लामा॒ इति॑ श्या॒म - ल॒ला॒माः॒ । \newline
39. तू॒प॒रा इति॑ तूप॒राः । \newline
\pagebreak
\markright{ TS 5.6.14.1  \hfill https://www.vedavms.in \hfill}

\section{ TS 5.6.14.1 }

\textbf{TS 5.6.14.1 } \newline
\textbf{Samhita Paata} \newline

उ॒न्न॒त ऋ॑ष॒भो वा॑म॒नस्त ऐ᳚न्द्रावरु॒णाः शिति॑ककुच्छितिपृ॒ष्ठः शिति॑भस॒त् त ऐ᳚न्द्राबार्.हस्प॒त्याः शि॑ति॒पाच्छि॒त्योष्ठः॑ शिति॒भ्रुस्त ऐ᳚न्द्रावैष्ण॒वास्ति॒स्रः सि॒द्ध्मा व॒शा वै᳚श्वकर्म॒ण्य॑स्ति॒स्रो धा॒त्रे पृ॑षोद॒रा ऐ᳚न्द्रापौ॒ष्णाः श्येत॑ललामास्तूप॒राः ॥ \newline

\textbf{Pada Paata} \newline

उ॒न्न॒त इत्यु॑त् - न॒तः । ऋ॒ष॒भः । वा॒म॒नः । ते । ऐ॒न्द्रा॒व॒रु॒णा इत्यै᳚न्द्रा - व॒रु॒णाः । शिति॑ककु॒दिति॒ शिति॑ - क॒कु॒त् । शि॒ति॒पृ॒ष्ठ इति॑ शिति - पृ॒ष्ठः । शिति॑भस॒दिति॒ शिति॑ - भ॒स॒त् । ते । ऐ॒न्द्रा॒बा॒र्.॒ह॒स्प॒त्या इत्यै᳚न्द्रा - बा॒र्.॒ह॒स्प॒त्याः । शि॒ति॒पादिति॑ शिति - पात् । शि॒त्योष्ठ॒ इति॑ शिति - ओष्ठः॑ । शि॒ति॒भ्रुरिति॑ शिति - भ्रुः । ते । ऐ॒न्द्रा॒वै॒ष्ण॒वा इत्यै᳚न्द्रा - वै॒ष्ण॒वाः । ति॒स्रः । सि॒द्ध्माः । व॒शाः । वै॒श्व॒क॒र्म॒ण्य॑ इति॑ वैश्व-क॒र्म॒ण्यः॑ । ति॒स्रः । धा॒त्रे । पृ॒षो॒द॒रा इति॑ पृष - उ॒द॒राः । ऐ॒न्द्रा॒पौ॒ष्णा इत्यै᳚न्द्रा - पौ॒ष्णाः । श्येत॑ललामा॒ इति॒ श्येत॑ - ल॒ला॒माः॒ । तू॒प॒राः ॥  \newline


\textbf{Krama Paata} \newline

उ॒न्न॒त ऋ॑ष॒भः । उ॒न्न॒त इत्यु॑त् - न॒तः । ऋ॒ष॒भो वा॑म॒नः । वा॒म॒नस्ते । त ऐ᳚न्द्रावरु॒णाः । ऐ॒न्द्रा॒व॒रु॒णाः शिति॑ककुत् । ऐ॒न्द्रा॒व॒रु॒णा इत्यै᳚न्द्रा - व॒रु॒णाः । शिति॑ककुच्छितिपृ॒ष्ठः । शिति॑ककु॒दिति॒ शिति॑ - क॒कु॒त्॒ । शि॒ति॒पृ॒ष्ठः शिति॑भसत् । शि॒ति॒पृ॒ष्ठ इति॑ शिति - पृ॒ष्ठः । शिति॑भस॒त् ते । शिति॑भस॒दिति॒ शिति॑ - भ॒स॒त्॒ । त ऐ᳚न्द्राबार्.हस्प॒त्याः । ऐ॒न्द्रा॒,बा॒र्॒.ह॒स्प॒त्याः शि॑ति॒पात् । ऐ॒न्द्रा॒बा॒र्॒.ह॒स्प॒त्या इत्यै᳚न्द्रा - बा॒र्॒.ह॒स्प॒त्याः । शि॒ति॒पाच्छि॒त्योष्ठः॑ । शि॒ति॒पादिति॑ शिति - पात् । शि॒त्योष्ठः॑ शिति॒भ्रुः । शि॒त्योष्ठ॒ इति॑ शिति - ओष्ठः॑ । शि॒ति॒भ्रुस्ते । शि॒ति॒भ्रुरिति॑ शिति - भ्रुः । त ऐ᳚न्द्रावैष्ण॒वाः । ऐ॒न्द्रा॒वै॒ष्ण॒वास्ति॒स्रः । ऐ॒न्द्रा॒वै॒ष्ण॒वा इत्यै᳚न्द्रा - वै॒ष्ण॒वाः । ति॒स्रः सि॒द्ध्माः । सि॒द्ध्मा व॒शाः । व॒शा वै᳚श्वकर्म॒ण्यः॑ । वै॒श्व॒क॒र्म॒ण्य॑स्ति॒स्रः । वै॒श्व॒क॒र्म॒ण्य॑ इति॑ वैश्व - क॒र्म॒ण्यः॑ । ति॒स्रो धा॒त्रे । धा॒त्रे पृ॑षोद॒राः । पृ॒षो॒द॒रा ऐ᳚न्द्रापौ॒ष्णाः । पृ॒षो॒द॒रा इति॑ पृष - उ॒द॒राः । ऐ॒न्द्रा॒पौ॒ष्णाः श्येत॑ललामाः । ऐ॒न्द्रा॒पौ॒ष्णा इत्यै᳚न्द्रा - पौ॒ष्णाः । श्येत॑ललामा,स्तूप॒राः । श्येत॑ललामा॒ इति॒ शेत॑ - ल॒ला॒माः॒ । तू॒प॒रा इति॑ तूप॒राः । \newline

\textbf{Jatai Paata} \newline

1. उ॒न्न॒त ऋ॑ष॒भ ऋ॑ष॒भ उ॑न्न॒त उ॑न्न॒त ऋ॑ष॒भः । \newline
2. उ॒न्न॒त इत्यु॑त् - न॒तः । \newline
3. ऋ॒ष॒भो वा॑म॒नो वा॑म॒न ऋ॑ष॒भ ऋ॑ष॒भो वा॑म॒नः । \newline
4. वा॒म॒न स्ते ते वा॑म॒नो वा॑म॒न स्ते । \newline
5. त ऐ᳚न्द्रावरु॒णा ऐ᳚न्द्रावरु॒णा स्ते त ऐ᳚न्द्रावरु॒णाः । \newline
6. ऐ॒न्द्रा॒व॒रु॒णाः शिति॑ककु॒ च्छिति॑ककु दैन्द्रावरु॒णा ऐ᳚न्द्रावरु॒णाः शिति॑ककुत् । \newline
7. ऐ॒न्द्रा॒व॒रु॒णा इत्यै᳚न्द्रा - व॒रु॒णाः । \newline
8. शिति॑ककु च्छितिपृ॒ष्ठः शि॑तिपृ॒ष्ठः शिति॑ककु॒ च्छिति॑ककु च्छितिपृ॒ष्ठः । \newline
9. शिति॑ककु॒दिति॒ शिति॑ - क॒कु॒त् । \newline
10. शि॒ति॒पृ॒ष्ठः शिति॑भस॒ च्छिति॑भस च्छितिपृ॒ष्ठः शि॑तिपृ॒ष्ठः शिति॑भसत् । \newline
11. शि॒ति॒पृ॒ष्ठ इति॑ शिति - पृ॒ष्ठः । \newline
12. शिति॑भस॒त् ते ते शिति॑भस॒ च्छिति॑भस॒त् ते । \newline
13. शिति॑भस॒दिति॒ शिति॑ - भ॒स॒त् । \newline
14. त ऐ᳚न्द्राबार्.हस्प॒त्या ऐ᳚न्द्राबार्.हस्प॒त्या स्ते त ऐ᳚न्द्राबार्.हस्प॒त्याः । \newline
15. ऐ॒न्द्रा॒बा॒र्॒.ह॒स्प॒त्याः शि॑ति॒पा च्छि॑ति॒पा दै᳚न्द्राबार्.हस्प॒त्या ऐ᳚न्द्राबार्.हस्प॒त्याः शि॑ति॒पात् । \newline
16. ऐ॒न्द्रा॒बा॒र्.॒ह॒स्प॒त्या इत्यै᳚न्द्रा - बा॒र्.॒ह॒स्प॒त्याः । \newline
17. शि॒ति॒पा च्छि॒त्योष्ठः॑ शि॒त्योष्ठः॑ शिति॒पा च्छि॑ति॒पा च्छि॒त्योष्ठः॑ । \newline
18. शि॒ति॒पादिति॑ शिति - पात् । \newline
19. शि॒त्योष्ठः॑ शिति॒भ्रुः शि॑ति॒भ्रुः शि॒त्योष्ठः॑ शि॒त्योष्ठः॑ शिति॒भ्रुः । \newline
20. शि॒त्योष्ठ॒ इति॑ शिति - ओष्ठः॑ । \newline
21. शि॒ति॒भ्रु स्ते ते शि॑ति॒भ्रुः शि॑ति॒भ्रु स्ते । \newline
22. शि॒ति॒भ्रुरिति॑ शिति - भ्रुः । \newline
23. त ऐ᳚न्द्रावैष्ण॒वा ऐ᳚न्द्रावैष्ण॒वा स्ते त ऐ᳚न्द्रावैष्ण॒वाः । \newline
24. ऐ॒न्द्रा॒वै॒ष्ण॒वा स्ति॒स्र स्ति॒स्र ऐ᳚न्द्रावैष्ण॒वा ऐ᳚न्द्रावैष्ण॒वा स्ति॒स्रः । \newline
25. ऐ॒न्द्रा॒वै॒ष्ण॒वा इत्यै᳚न्द्रा - वै॒ष्ण॒वाः । \newline
26. ति॒स्रः सि॒द्ध्माः सि॒द्ध्मा स्ति॒स्र स्ति॒स्रः सि॒द्ध्माः । \newline
27. सि॒द्ध्मा व॒शा व॒शाः सि॒द्ध्माः सि॒द्ध्मा व॒शाः । \newline
28. व॒शा वै᳚श्वकर्म॒ण्यो॑ वैश्वकर्म॒ण्यो॑ व॒शा व॒शा वै᳚श्वकर्म॒ण्यः॑ । \newline
29. वै॒श्व॒क॒र्म॒ण्य॑ स्ति॒स्र स्ति॒स्रो वै᳚श्वकर्म॒ण्यो॑ वैश्वकर्म॒ण्य॑ स्ति॒स्रः । \newline
30. वै॒श्व॒क॒र्म॒ण्य॑ इति॑ वैश्व - क॒र्म॒ण्यः॑ । \newline
31. ति॒स्रो धा॒त्रे धा॒त्रे ति॒स्र स्ति॒स्रो धा॒त्रे । \newline
32. धा॒त्रे पृ॑षोद॒राः पृ॑षोद॒रा धा॒त्रे धा॒त्रे पृ॑षोद॒राः । \newline
33. पृ॒षो॒द॒रा ऐ᳚न्द्रापौ॒ष्णा ऐ᳚न्द्रापौ॒ष्णाः पृ॑षोद॒राः पृ॑षोद॒रा ऐ᳚न्द्रापौ॒ष्णाः । \newline
34. पृ॒षो॒द॒रा इति॑ पृष - उ॒द॒राः । \newline
35. ऐ॒न्द्रा॒पौ॒ष्णाः श्येत॑ललामाः॒ श्येत॑ललामा ऐन्द्रापौ॒ष्णा ऐ᳚न्द्रापौ॒ष्णाः श्येत॑ललामाः । \newline
36. ऐ॒न्द्रा॒पौ॒ष्णा इत्यै᳚न्द्रा - पौ॒ष्णाः । \newline
37. श्येत॑ललामा स्तूप॒रा स्तू॑प॒राः श्येत॑ललामाः॒ श्येत॑ललामा स्तूप॒राः । \newline
38. श्येत॑ललामा॒ इति॒ श्येत॑ - ल॒ला॒माः॒ । \newline
39. तू॒प॒रा इति॑ तूप॒राः । \newline

\textbf{Ghana Paata } \newline

1. उ॒न्न॒त ऋ॑ष॒भ ऋ॑ष॒भ उ॑न्न॒त उ॑न्न॒त ऋ॑ष॒भो वा॑म॒नो वा॑म॒न ऋ॑ष॒भ उ॑न्न॒त उ॑न्न॒त ऋ॑ष॒भो वा॑म॒नः । \newline
2. उ॒न्न॒त इत्यु॑त् - न॒तः । \newline
3. ऋ॒ष॒भो वा॑म॒नो वा॑म॒न ऋ॑ष॒भ ऋ॑ष॒भो वा॑म॒न स्ते ते वा॑म॒न ऋ॑ष॒भ ऋ॑ष॒भो वा॑म॒न स्ते । \newline
4. वा॒म॒न स्ते ते वा॑म॒नो वा॑म॒न स्त ऐ᳚न्द्रावरु॒णा ऐ᳚न्द्रावरु॒णा स्ते वा॑म॒नो वा॑म॒न स्त ऐ᳚न्द्रावरु॒णाः । \newline
5. त ऐ᳚न्द्रावरु॒णा ऐ᳚न्द्रावरु॒णा स्ते त ऐ᳚न्द्रावरु॒णाः शिति॑ककु॒ च्छिति॑ककु दैन्द्रावरु॒णा स्ते त ऐ᳚न्द्रावरु॒णाः शिति॑ककुत् । \newline
6. ऐ॒न्द्रा॒व॒रु॒णाः शिति॑ककु॒ च्छिति॑ककु दैन्द्रावरु॒णा ऐ᳚न्द्रावरु॒णाः शिति॑ककु च्छितिपृ॒ष्ठः शि॑तिपृ॒ष्ठः शिति॑ककु दैन्द्रावरु॒णा ऐ᳚न्द्रावरु॒णाः शिति॑ककु च्छितिपृ॒ष्ठः । \newline
7. ऐ॒न्द्रा॒व॒रु॒णा इत्यै᳚न्द्रा - व॒रु॒णाः । \newline
8. शिति॑ककु च्छितिपृ॒ष्ठः शि॑तिपृ॒ष्ठः शिति॑ककु॒ च्छिति॑ककु च्छितिपृ॒ष्ठः शिति॑भस॒ च्छिति॑भस
च्छितिपृ॒ष्ठः शिति॑ककु॒ च्छिति॑ककु च्छितिपृ॒ष्ठः शिति॑भसत् । \newline
9. शिति॑ककु॒दिति॒ शिति॑ - क॒कु॒त् । \newline
10. शि॒ति॒पृ॒ष्ठः शिति॑भस॒ च्छिति॑भस च्छितिपृ॒ष्ठः शि॑तिपृ॒ष्ठः शिति॑भस॒त् ते ते शिति॑भस च्छितिपृ॒ष्ठः शि॑तिपृ॒ष्ठः शिति॑भस॒त् ते । \newline
11. शि॒ति॒पृ॒ष्ठ इति॑ शिति - पृ॒ष्ठः । \newline
12. शिति॑भस॒त् ते ते शिति॑भस॒ च्छिति॑भस॒त् त ऐ᳚न्द्राबार्.हस्प॒त्या ऐ᳚न्द्राबार्.हस्प॒त्या स्ते शिति॑भस॒ च्छिति॑भस॒त् त ऐ᳚न्द्राबार्.हस्प॒त्याः । \newline
13. शिति॑भस॒दिति॒ शिति॑ - भ॒स॒त् । \newline
14. त ऐ᳚न्द्राबार्.हस्प॒त्या ऐ᳚न्द्राबार्.हस्प॒त्या स्ते त ऐ᳚न्द्राबार्.हस्प॒त्याः शि॑ति॒पा च्छि॑ति॒पा दै᳚न्द्राबार्.हस्प॒त्या स्ते त ऐ᳚न्द्राबार्.हस्प॒त्याः शि॑ति॒पात् । \newline
15. ऐ॒न्द्रा॒बा॒र्॒.ह॒स्प॒त्याः शि॑ति॒पा च्छि॑ति॒पा दै᳚न्द्राबार्.हस्प॒त्या ऐ᳚न्द्राबार्.हस्प॒त्याः शि॑ति॒पा च्छि॒त्योष्ठः॑ शि॒त्योष्ठः॑ शिति॒पा दै᳚न्द्राबार्.हस्प॒त्या ऐ᳚न्द्राबार्.हस्प॒त्याः शि॑ति॒पा च्छि॒त्योष्ठः॑ । \newline
16. ऐ॒न्द्रा॒बा॒र्.॒ह॒स्प॒त्या इत्यै᳚न्द्रा - बा॒र्.॒ह॒स्प॒त्याः । \newline
17. शि॒ति॒पा च्छि॒त्योष्ठः॑ शि॒त्योष्ठः॑ शिति॒पा च्छि॑ति॒पा च्छि॒त्योष्ठः॑ शिति॒भ्रुः शि॑ति॒भ्रुः शि॒त्योष्ठः॑ शिति॒पा
च्छि॑ति॒पा च्छि॒त्योष्ठः॑ शिति॒भ्रुः । \newline
18. शि॒ति॒पादिति॑ शिति - पात् । \newline
19. शि॒त्योष्ठः॑ शिति॒भ्रुः शि॑ति॒भ्रुः शि॒त्योष्ठः॑ शि॒त्योष्ठः॑ शिति॒भ्रु स्ते ते शि॑ति॒भ्रुः शि॒त्योष्ठः॑ शि॒त्योष्ठः॑ शिति॒भ्रु स्ते । \newline
20. शि॒त्योष्ठ॒ इति॑ शिति - ओष्ठः॑ । \newline
21. शि॒ति॒भ्रु स्ते ते शि॑ति॒भ्रुः शि॑ति॒भ्रु स्त ऐ᳚न्द्रावैष्ण॒वा ऐ᳚न्द्रावैष्ण॒वा स्ते शि॑ति॒भ्रुः शि॑ति॒भ्रु स्त ऐ᳚न्द्रावैष्ण॒वाः । \newline
22. शि॒ति॒भ्रुरिति॑ शिति - भ्रुः । \newline
23. त ऐ᳚न्द्रावैष्ण॒वा ऐ᳚न्द्रावैष्ण॒वा स्ते त ऐ᳚न्द्रावैष्ण॒वा स्ति॒स्र स्ति॒स्र ऐ᳚न्द्रावैष्ण॒वा स्ते त ऐ᳚न्द्रावैष्ण॒वा स्ति॒स्रः । \newline
24. ऐ॒न्द्रा॒वै॒ष्ण॒वा स्ति॒स्र स्ति॒स्र ऐ᳚न्द्रावैष्ण॒वा ऐ᳚न्द्रावैष्ण॒वा स्ति॒स्रः सि॒द्ध्माः सि॒द्ध्मा स्ति॒स्र ऐ᳚न्द्रावैष्ण॒वा ऐ᳚न्द्रावैष्ण॒वा स्ति॒स्रः सि॒द्ध्माः । \newline
25. ऐ॒न्द्रा॒वै॒ष्ण॒वा इत्यै᳚न्द्रा - वै॒ष्ण॒वाः । \newline
26. ति॒स्रः सि॒द्ध्माः सि॒द्ध्मा स्ति॒स्र स्ति॒स्रः सि॒द्ध्मा व॒शा व॒शाः सि॒द्ध्मा स्ति॒स्र स्ति॒स्रः सि॒द्ध्मा व॒शाः । \newline
27. सि॒द्ध्मा व॒शा व॒शाः सि॒द्ध्माः सि॒द्ध्मा व॒शा वै᳚श्वकर्म॒ण्यो॑ वैश्वकर्म॒ण्यो॑ व॒शाः सि॒द्ध्माः सि॒द्ध्मा व॒शा वै᳚श्वकर्म॒ण्यः॑ । \newline
28. व॒शा वै᳚श्वकर्म॒ण्यो॑ वैश्वकर्म॒ण्यो॑ व॒शा व॒शा वै᳚श्वकर्म॒ण्य॑ स्ति॒स्र स्ति॒स्रो वै᳚श्वकर्म॒ण्यो॑ व॒शा व॒शा वै᳚श्वकर्म॒ण्य॑ स्ति॒स्रः । \newline
29. वै॒श्व॒क॒र्म॒ण्य॑ स्ति॒स्र स्ति॒स्रो वै᳚श्वकर्म॒ण्यो॑ वैश्वकर्म॒ण्य॑ स्ति॒स्रो धा॒त्रे धा॒त्रे ति॒स्रो वै᳚श्वकर्म॒ण्यो॑ वैश्वकर्म॒ण्य॑ स्ति॒स्रो धा॒त्रे । \newline
30. वै॒श्व॒क॒र्म॒ण्य॑ इति॑ वैश्व - क॒र्म॒ण्यः॑ । \newline
31. ति॒स्रो धा॒त्रे धा॒त्रे ति॒स्र स्ति॒स्रो धा॒त्रे पृ॑षोद॒राः पृ॑षोद॒रा धा॒त्रे ति॒स्र स्ति॒स्रो धा॒त्रे पृ॑षोद॒राः । \newline
32. धा॒त्रे पृ॑षोद॒राः पृ॑षोद॒रा धा॒त्रे धा॒त्रे पृ॑षोद॒रा ऐ᳚न्द्रापौ॒ष्णा ऐ᳚न्द्रापौ॒ष्णाः पृ॑षोद॒रा धा॒त्रे धा॒त्रे पृ॑षोद॒रा ऐ᳚न्द्रापौ॒ष्णाः । \newline
33. पृ॒षो॒द॒रा ऐ᳚न्द्रापौ॒ष्णा ऐ᳚न्द्रापौ॒ष्णाः पृ॑षोद॒राः पृ॑षोद॒रा ऐ᳚न्द्रापौ॒ष्णाः श्येत॑ललामाः॒ श्येत॑ललामा ऐन्द्रापौ॒ष्णाः पृ॑षोद॒राः पृ॑षोद॒रा ऐ᳚न्द्रापौ॒ष्णाः श्येत॑ललामाः । \newline
34. पृ॒षो॒द॒रा इति॑ पृष - उ॒द॒राः । \newline
35. ऐ॒न्द्रा॒पौ॒ष्णाः श्येत॑ललामाः॒ श्येत॑ललामा ऐन्द्रापौ॒ष्णा ऐ᳚न्द्रापौ॒ष्णाः श्येत॑ललामा स्तूप॒रा स्तू॑प॒राः श्येत॑ललामा ऐन्द्रापौ॒ष्णा ऐ᳚न्द्रापौ॒ष्णाः श्येत॑ललामा स्तूप॒राः । \newline
36. ऐ॒न्द्रा॒पौ॒ष्णा इत्यै᳚न्द्रा - पौ॒ष्णाः । \newline
37. श्येत॑ललामा स्तूप॒रा स्तू॑प॒राः श्येत॑ललामाः॒ श्येत॑ललामा स्तूप॒राः । \newline
38. श्येत॑ललामा॒ इति॒ श्येत॑ - ल॒ला॒माः॒ । \newline
39. तू॒प॒रा इति॑ तूप॒राः । \newline
\pagebreak
\markright{ TS 5.6.15.1  \hfill https://www.vedavms.in \hfill}

\section{ TS 5.6.15.1 }

\textbf{TS 5.6.15.1 } \newline
\textbf{Samhita Paata} \newline

क॒र्णास्त्रयो॑ या॒माः सौ॒म्यास्त्रयः॑ श्विति॒ङ्गा अ॒ग्नये॒ यवि॑ष्ठाय॒ त्रयो॑ नकु॒लास्ति॒स्रो रोहि॑णी॒स्त्र्यव्य॒स्ता वसू॑नां ति॒स्रो॑ऽरु॒णा दि॑त्यौ॒ह्य॑स्ता रु॒द्राणाꣳ॑ सोमै॒न्द्रा ब॒भ्रुल॑लामास्तूप॒राः ॥ \newline

\textbf{Pada Paata} \newline

क॒र्णाः । त्रयः॑ । या॒माः । सौ॒म्याः । त्रयः॑ । श्वि॒ति॒ङ्गाः । अ॒ग्नये᳚ । यवि॑ष्ठाय । त्रयः॑ । न॒कु॒लाः । ति॒स्रः । रोहि॑णीः । त्र्यव्य॒ इति॑ त्रि - अव्यः॑ । ताः । वसू॑नाम् । ति॒स्रः । अ॒रु॒णाः । दि॒त्यौ॒ह्यः॑ । ताः । रु॒द्राणा᳚म् । सो॒मै॒न्द्रा इति॑ सोम - ऐ॒न्द्राः । ब॒भ्रुल॑लामा॒ इति॑ ब॒भ्रु - ल॒ला॒माः॒ । तू॒प॒राः ॥  \newline


\textbf{Krama Paata} \newline

क॒र्णास्त्रयः॑ । त्रयो॑ या॒माः । या॒माः सौ॒म्याः । सौ॒म्यास्त्रयः॑ । त्रयः॑ श्विति॒ङ्गाः । श्वि॒ति॒ङ्गा अ॒ग्नये᳚ । अ॒ग्नये॒ यवि॑ष्ठाय । यवि॑ष्ठाय॒ त्रयः॑ । त्रयो॑ नकु॒लाः । न॒कु॒लास्ति॒स्रः । ति॒स्रो रोहि॑णीः । रोहि॑णी॒स्त्र्यव्यः॑ । त्र्यव्य॒स्ताः । त्र्यव्य॒ इति॑ त्रि - अव्यः॑ । ता वसू॑नाम् । वसू॑नाम् ति॒स्रः । ति॒स्रो॑ऽरु॒णाः । अ॒रु॒णा दि॑त्यौ॒ह्यः॑ । दि॒त्यौ॒ह्य॑स्ताः । ता रु॒द्राणा᳚म् । रु॒द्राणाꣳ॑ सोमै॒न्द्राः । सो॒मै॒न्द्रा ब॒भ्रुल॑लामाः । सो॒मै॒न्द्रा इति॑ सोम - ऐ॒न्द्राः । ब॒भ्रुल॑लामास्तूप॒राः । ब॒भ्रुल॑लामा॒ इति॑ ब॒भ्रु - ल॒ला॒माः॒ । तू॒प॒रा इति॑ तूप॒राः । \newline

\textbf{Jatai Paata} \newline

1. क॒र्णा स्त्रय॒ स्त्रयः॑ क॒र्णाः क॒र्णा स्त्रयः॑ । \newline
2. त्रयो॑ या॒मा या॒मा स्त्रय॒ स्त्रयो॑ या॒माः । \newline
3. या॒माः सौ॒म्याः सौ॒म्या या॒मा या॒माः सौ॒म्याः । \newline
4. सौ॒म्या स्त्रय॒ स्त्रयः॑ सौ॒म्याः सौ॒म्या स्त्रयः॑ । \newline
5. त्रयः॑ श्विति॒ङ्गाः श्वि॑ति॒ङ्गा स्त्रय॒ स्त्रयः॑ श्विति॒ङ्गाः । \newline
6. श्वि॒ति॒ङ्गा अ॒ग्नये॒ ऽग्नये᳚ श्विति॒ङ्गाः श्वि॑ति॒ङ्गा अ॒ग्नये᳚ । \newline
7. अ॒ग्नये॒ यवि॑ष्ठाय॒ यवि॑ष्ठा या॒ग्नये॒ ऽग्नये॒ यवि॑ष्ठाय । \newline
8. यवि॑ष्ठाय॒ त्रय॒ स्त्रयो॒ यवि॑ष्ठाय॒ यवि॑ष्ठाय॒ त्रयः॑ । \newline
9. त्रयो॑ नकु॒ला न॑कु॒ला स्त्रय॒ स्त्रयो॑ नकु॒लाः । \newline
10. न॒कु॒ला स्ति॒स्र स्ति॒स्रो न॑कु॒ला न॑कु॒ला स्ति॒स्रः । \newline
11. ति॒स्रो रोहि॑णी॒ रोहि॑णी स्ति॒स्र स्ति॒स्रो रोहि॑णीः । \newline
12. रोहि॑णि॒ स्त्र्यव्य॒ स्त्र्यव्यो॒ रोहि॑णी॒ रोहि॑णि॒ स्त्र्यव्यः॑ । \newline
13. त्र्यव्य॒ स्ता स्ता स्त्र्यव्य॒ स्त्र्यव्य॒ स्ताः । \newline
14. त्र्यव्य॒ इति॑ त्रि - अव्यः॑ । \newline
15. ता वसू॑नां॒ ॅवसू॑ना॒म् ता स्ता वसू॑नाम् । \newline
16. वसू॑नाम् ति॒स्र स्ति॒स्रो वसू॑नां॒ ॅवसू॑नाम् ति॒स्रः । \newline
17. ति॒स्रो॑ ऽरु॒णा अ॑रु॒णा स्ति॒स्र स्ति॒स्रो॑ ऽरु॒णाः । \newline
18. अ॒रु॒णा दि॑त्यौ॒ह्यो॑ दित्यौ॒ह्यो॑ ऽरु॒णा अ॑रु॒णा दि॑त्यौ॒ह्यः॑ । \newline
19. दि॒त्यौ॒ह्य॑ स्ता स्ता दि॑त्यौ॒ह्यो॑ दित्यौ॒ह्य॑ स्ताः । \newline
20. ता रु॒द्राणाꣳ॑ रु॒द्राणा॒म् ता स्ता रु॒द्राणा᳚म् । \newline
21. रु॒द्राणाꣳ॑ सोमै॒न्द्राः सो॑मै॒न्द्रा रु॒द्राणाꣳ॑ रु॒द्राणाꣳ॑ सोमै॒न्द्राः । \newline
22. सो॒मै॒न्द्रा ब॒भ्रुल॑लामा ब॒भ्रुल॑लामाः सोमै॒न्द्राः सो॑मै॒न्द्रा ब॒भ्रुल॑लामाः । \newline
23. सो॒मै॒न्द्रा इति॑ सोम - ऐ॒न्द्राः । \newline
24. ब॒भ्रुल॑लामा स्तूप॒रा स्तू॑प॒रा ब॒भ्रुल॑लामा ब॒भ्रुल॑लामा स्तूप॒राः । \newline
25. ब॒भ्रुल॑लामा॒ इति॑ ब॒भ्रु - ल॒ला॒माः॒ । \newline
26. तू॒प॒रा इति॑ तूप॒राः । \newline

\textbf{Ghana Paata } \newline

1. क॒र्णा स्त्रय॒ स्त्रयः॑ क॒र्णाः क॒र्णा स्त्रयो॑ या॒मा या॒मा स्त्रयः॑ क॒र्णाः क॒र्णा स्त्रयो॑ या॒माः । \newline
2. त्रयो॑ या॒मा या॒मा स्त्रय॒ स्त्रयो॑ या॒माः सौ॒म्याः सौ॒म्या या॒मा स्त्रय॒ स्त्रयो॑ या॒माः सौ॒म्याः । \newline
3. या॒माः सौ॒म्याः सौ॒म्या या॒मा या॒माः सौ॒म्या स्त्रय॒ स्त्रयः॑ सौ॒म्या या॒मा या॒माः सौ॒म्या स्त्रयः॑ । \newline
4. सौ॒म्या स्त्रय॒ स्त्रयः॑ सौ॒म्याः सौ॒म्या स्त्रयः॑ श्विति॒ङ्गाः श्वि॑ति॒ङ्गा स्त्रयः॑ सौ॒म्याः सौ॒म्या स्त्रयः॑ श्विति॒ङ्गाः । \newline
5. त्रयः॑ श्विति॒ङ्गाः श्वि॑ति॒ङ्गा स्त्रय॒ स्त्रयः॑ श्विति॒ङ्गा अ॒ग्नये॒ ऽग्नये᳚ श्विति॒ङ्गा स्त्रय॒ स्त्रयः॑ श्विति॒ङ्गा अ॒ग्नये᳚ । \newline
6. श्वि॒ति॒ङ्गा अ॒ग्नये॒ ऽग्नये᳚ श्विति॒ङ्गाः श्वि॑ति॒ङ्गा अ॒ग्नये॒ यवि॑ष्ठाय॒ यवि॑ष्ठा या॒ग्नये᳚ श्विति॒ङ्गाः श्वि॑ति॒ङ्गा अ॒ग्नये॒ यवि॑ष्ठाय । \newline
7. अ॒ग्नये॒ यवि॑ष्ठाय॒ यवि॑ष्ठा या॒ग्नये॒ ऽग्नये॒ यवि॑ष्ठाय॒ त्रय॒ स्त्रयो॒ यवि॑ष्ठा या॒ग्नये॒ ऽग्नये॒ यवि॑ष्ठाय॒ त्रयः॑ । \newline
8. यवि॑ष्ठाय॒ त्रय॒ स्त्रयो॒ यवि॑ष्ठाय॒ यवि॑ष्ठाय॒ त्रयो॑ नकु॒ला न॑कु॒ला स्त्रयो॒ यवि॑ष्ठाय॒ यवि॑ष्ठाय॒ त्रयो॑ नकु॒लाः । \newline
9. त्रयो॑ नकु॒ला न॑कु॒ला स्त्रय॒ स्त्रयो॑ नकु॒ला स्ति॒स्र स्ति॒स्रो न॑कु॒ला स्त्रय॒ स्त्रयो॑ नकु॒ला स्ति॒स्रः । \newline
10. न॒कु॒ला स्ति॒स्र स्ति॒स्रो न॑कु॒ला न॑कु॒ला स्ति॒स्रो रोहि॑णी॒ रोहि॑णी स्ति॒स्रो न॑कु॒ला न॑कु॒ला स्ति॒स्रो रोहि॑णीः । \newline
11. ति॒स्रो रोहि॑णी॒ रोहि॑णी स्ति॒स्र स्ति॒स्रो रोहि॑णि॒ स्त्र्यव्य॒ स्त्र्यव्यो॒ रोहि॑णी स्ति॒स्र स्ति॒स्रो रोहि॑णि॒ स्त्र्यव्यः॑ । \newline
12. रोहि॑णि॒ स्त्र्यव्य॒ स्त्र्यव्यो॒ रोहि॑णी॒ रोहि॑णि॒ स्त्र्यव्य॒ स्ता स्ता स्त्र्यव्यो॒ रोहि॑णी॒ रोहि॑णि॒ स्त्र्यव्य॒ स्ताः । \newline
13. त्र्यव्य॒ स्ता स्ता स्त्र्यव्य॒ स्त्र्यव्य॒ स्ता वसू॑नां॒ ॅवसू॑ना॒म् ता स्त्र्यव्य॒ स्त्र्यव्य॒ स्ता वसू॑नाम् । \newline
14. त्र्यव्य॒ इति॑ त्रि - अव्यः॑ । \newline
15. ता वसू॑नां॒ ॅवसू॑ना॒म् ता स्ता वसू॑नाम् ति॒स्र स्ति॒स्रो वसू॑ना॒म् ता स्ता वसू॑नाम् ति॒स्रः । \newline
16. वसू॑नाम् ति॒स्र स्ति॒स्रो वसू॑नां॒ ॅवसू॑नाम् ति॒स्रो॑ ऽरु॒णा अ॑रु॒णा स्ति॒स्रो वसू॑नां॒ ॅवसू॑नाम् ति॒स्रो॑ ऽरु॒णाः । \newline
17. ति॒स्रो॑ ऽरु॒णा अ॑रु॒णा स्ति॒स्र स्ति॒स्रो॑ ऽरु॒णा दि॑त्यौ॒ह्यो॑ दित्यौ॒ह्यो॑ ऽरु॒णा स्ति॒स्र स्ति॒स्रो॑ ऽरु॒णा दि॑त्यौ॒ह्यः॑ । \newline
18. अ॒रु॒णा दि॑त्यौ॒ह्यो॑ दित्यौ॒ह्यो॑ ऽरु॒णा अ॑रु॒णा दि॑त्यौ॒ह्य॑ स्ता स्ता दि॑त्यौ॒ह्यो॑ ऽरु॒णा अ॑रु॒णा दि॑त्यौ॒ह्य॑ स्ताः । \newline
19. दि॒त्यौ॒ह्य॑ स्ता स्ता दि॑त्यौ॒ह्यो॑ दित्यौ॒ह्य॑ स्ता रु॒द्राणाꣳ॑ रु॒द्राणा॒म् ता दि॑त्यौ॒ह्यो॑ दित्यौ॒ह्य॑ स्ता रु॒द्राणा᳚म् । \newline
20. ता रु॒द्राणाꣳ॑ रु॒द्राणा॒म् ता स्ता रु॒द्राणाꣳ॑ सोमै॒न्द्राः सो॑मै॒न्द्रा रु॒द्राणा॒म् ता स्ता रु॒द्राणाꣳ॑ सोमै॒न्द्राः । \newline
21. रु॒द्राणाꣳ॑ सोमै॒न्द्राः सो॑मै॒न्द्रा रु॒द्राणाꣳ॑ रु॒द्राणाꣳ॑ सोमै॒न्द्रा ब॒भ्रुल॑लामा ब॒भ्रुल॑लामाः सोमै॒न्द्रा रु॒द्राणाꣳ॑ रु॒द्राणाꣳ॑ सोमै॒न्द्रा ब॒भ्रुल॑लामाः । \newline
22. सो॒मै॒न्द्रा ब॒भ्रुल॑लामा ब॒भ्रुल॑लामाः सोमै॒न्द्राः सो॑मै॒न्द्रा ब॒भ्रुल॑लामा स्तूप॒रा स्तू॑प॒रा ब॒भ्रुल॑लामाः सोमै॒न्द्राः सो॑मै॒न्द्रा ब॒भ्रुल॑लामा स्तूप॒राः । \newline
23. सो॒मै॒न्द्रा इति॑ सोम - ऐ॒न्द्राः । \newline
24. ब॒भ्रुल॑लामा स्तूप॒रा स्तू॑प॒रा ब॒भ्रुल॑लामा ब॒भ्रुल॑लामा स्तूप॒राः । \newline
25. ब॒भ्रुल॑लामा॒ इति॑ ब॒भ्रु - ल॒ला॒माः॒ । \newline
26. तू॒प॒रा इति॑ तूप॒राः । \newline
\pagebreak
\markright{ TS 5.6.16.1  \hfill https://www.vedavms.in \hfill}

\section{ TS 5.6.16.1 }

\textbf{TS 5.6.16.1 } \newline
\textbf{Samhita Paata} \newline

शु॒ण्ठास्त्रयो॑ वैष्ण॒वा अ॑धीलोध॒कर्णा॒स्त्रयो॒ विष्ण॑व उरुक्र॒माय॑ लफ्सु॒दिन॒स्त्रयो॒ विष्ण॑व उरुगा॒याय॒ पञ्चा॑वीस्ति॒स्र आ॑दि॒त्यानां᳚ त्रिव॒थ्सा-स्ति॒स्रो-ऽङ्गि॑रसामैन्द्रावैष्ण॒वा गौ॒रल॑लामास्तूप॒राः ॥ \newline

\textbf{Pada Paata} \newline

शु॒ण्ठाः । त्रयः॑ । वै॒ष्ण॒वाः । अ॒धी॒लो॒ध॒कर्णा॒ इत्य॑धीलोध - कर्णाः᳚ । त्रयः॑ । विष्ण॑वे । उ॒रु॒क्र॒मायेत्यु॑रु - क्र॒माय॑ । ल॒फ्सु॒दिनः॑ । त्रयः॑ । विष्ण॑वे । उ॒रु॒गा॒यायेत्यु॑रु - गा॒याय॑ । पञ्चा॑वी॒रिति॒ पञ्च॑ - अ॒वीः॒ । ति॒स्रः । आ॒दि॒त्याना᳚म् । त्रि॒व॒थ्सा इति॑ त्रि - व॒थ्साः । ति॒स्रः । अङ्गि॑रसाम् । ऐ॒न्द्रा॒वै॒ष्ण॒वा इत्यै᳚न्द्रा - वै॒ष्ण॒वाः । गौ॒रल॑लामा॒ इति॑ गौ॒र - ल॒ला॒माः॒ । तू॒प॒राः ॥  \newline


\textbf{Krama Paata} \newline

शु॒ण्ठास्त्रयः॑ । त्रयो॑ वैष्ण॒वाः । वै॒ष्ण॒वा अ॑धीलोध॒कर्णाः᳚ । अ॒धी॒लो॒ध॒कर्णा॒,स्त्रयः॑ । अ॒धी॒लो॒ध॒कर्णा॒ इत्य॑धीलोध - कर्णाः᳚ । त्रयो॒ विष्ण॑वे । विष्ण॑व उरुक्र॒माय॑ । उ॒रु॒क्र॒माय॑ लफ्सु॒दिनः॑ । उ॒रु॒क्र॒मायेत्यु॑रु - क्र॒माय॑ । ल॒फ्सु॒दिन॒ स्त्रयः॑ । त्रयो॒ विष्ण॑वे । विष्ण॑व उरुगा॒याय॑ । उ॒रु॒गा॒याय॒ पञ्चा॑वीः । उ॒रु॒गा॒यायेत्यु॑रु - गा॒याय॑ । पञ्चा॑वीस्ति॒स्रः । पञ्चा॑वी॒रिति॒ पञ्च॑ - अ॒वीः॒ । ति॒स्र आ॑दि॒त्याना᳚म् । आ॒दि॒त्याना᳚म् त्रिव॒थ्साः । त्रि॒व॒थ्सा,स्ति॒स्रः । त्रि॒व॒थ्सा इति॑ त्रि - व॒थ्साः । ति॒स्रोऽङ्गि॑रसाम् । अङ्गि॑रसामैन्द्रावैष्ण॒वाः । ऐ॒न्द्रा॒वै॒ष्ण॒वा गौ॒रल॑लामाः । ऐ॒न्द्रा॒वै॒ष्ण॒वा इत्यै᳚न्द्रा - वै॒ष्ण॒वाः । गौ॒रल॑लामास्तूप॒राः । गौ॒रल॑लामा॒ इति॑ गौ॒र - ल॒ला॒माः॒ । तू॒प॒रा इति॑ तूप॒राः । \newline

\textbf{Jatai Paata} \newline

1. शु॒ण्ठा स्त्रय॒ स्त्रयः॑ शु॒ण्ठाः शु॒ण्ठा स्त्रयः॑ । \newline
2. त्रयो॑ वैष्ण॒वा वै᳚ष्ण॒वा स्त्रय॒ स्त्रयो॑ वैष्ण॒वाः । \newline
3. वै॒ष्ण॒वा अ॑धीलोध॒कर्णा॑ अधीलोध॒कर्णा॑ वैष्ण॒वा वै᳚ष्ण॒वा अ॑धीलोध॒कर्णाः᳚ । \newline
4. अ॒धी॒लो॒ध॒कर्णा॒ स्त्रय॒ स्त्रयो॑ ऽधीलोध॒कर्णा॑ अधीलोध॒कर्णा॒ स्त्रयः॑ । \newline
5. अ॒धी॒लो॒ध॒कर्णा॒ इत्य॑धीलोध - कर्णाः᳚ । \newline
6. त्रयो॒ विष्ण॑वे॒ विष्ण॑वे॒ त्रय॒ स्त्रयो॒ विष्ण॑वे । \newline
7. विष्ण॑व उरुक्र॒मायो॑ रुक्र॒माय॒ विष्ण॑वे॒ विष्ण॑व उरुक्र॒माय॑ । \newline
8. उ॒रु॒क्र॒माय॑ लफ्सु॒दिनो॑ लफ्सु॒दिन॑ उरुक्र॒मायो॑ रुक्र॒माय॑ लफ्सु॒दिनः॑ । \newline
9. उ॒रु॒क्र॒मायेत्यु॑रु - क्र॒माय॑ । \newline
10. ल॒फ्सु॒दिन॒ स्त्रय॒ स्त्रयो॑ लफ्सु॒दिनो॑ लफ्सु॒दिन॒ स्त्रयः॑ । \newline
11. त्रयो॒ विष्ण॑वे॒ विष्ण॑वे॒ त्रय॒ स्त्रयो॒ विष्ण॑वे । \newline
12. विष्ण॑व उरुगा॒यायो॑ रुगा॒याय॒ विष्ण॑वे॒ विष्ण॑व उरुगा॒याय॑ । \newline
13. उ॒रु॒गा॒याय॒ पञ्चा॑वीः॒ पञ्चा॑वी रुरुगा॒यायो॑ रुगा॒याय॒ पञ्चा॑वीः । \newline
14. उ॒रु॒गा॒यायेत्यु॑रु - गा॒याय॑ । \newline
15. पञ्चा॑वी स्ति॒स्र स्ति॒स्रः पञ्चा॑वीः॒ पञ्चा॑वी स्ति॒स्रः । \newline
16. पञ्चा॑वी॒रिति॒ पञ्च॑ - अ॒वीः॒ । \newline
17. ति॒स्र आ॑दि॒त्याना॑ मादि॒त्याना᳚म् ति॒स्र स्ति॒स्र आ॑दि॒त्याना᳚म् । \newline
18. आ॒दि॒त्याना᳚म् त्रिव॒थ्सा स्त्रि॑व॒थ्सा आ॑दि॒त्याना॑ मादि॒त्याना᳚म् त्रिव॒थ्साः । \newline
19. त्रि॒व॒थ्सा स्ति॒स्र स्ति॒स्र स्त्रि॑व॒थ्सा स्त्रि॑व॒थ्सा स्ति॒स्रः । \newline
20. त्रि॒व॒थ्सा इति॑ त्रि - व॒थ्साः । \newline
21. ति॒स्रो ऽङ्गि॑रसा॒ मङ्गि॑रसाम् ति॒स्र स्ति॒स्रो ऽङ्गि॑रसाम् । \newline
22. अङ्गि॑रसा मैन्द्रावैष्ण॒वा ऐ᳚न्द्रावैष्ण॒वा अङ्गि॑रसा॒ मङ्गि॑रसा मैन्द्रावैष्ण॒वाः । \newline
23. ऐ॒न्द्रा॒वै॒ष्ण॒वा गौ॒रल॑लामा गौ॒रल॑लामा ऐन्द्रावैष्ण॒वा ऐ᳚न्द्रावैष्ण॒वा गौ॒रल॑लामाः । \newline
24. ऐ॒न्द्रा॒वै॒ष्ण॒वा इत्यै᳚न्द्रा - वै॒ष्ण॒वाः । \newline
25. गौ॒रल॑लामा स्तूप॒रा स्तू॑प॒रा गौ॒रल॑लामा गौ॒रल॑लामा स्तूप॒राः । \newline
26. गौ॒रल॑लामा॒ इति॑ गौ॒र - ल॒ला॒माः॒ । \newline
27. तू॒प॒रा इति॑ तूप॒राः । \newline

\textbf{Ghana Paata } \newline

1. शु॒ण्ठा स्त्रय॒ स्त्रयः॑ शु॒ण्ठाः शु॒ण्ठा स्त्रयो॑ वैष्ण॒वा वै᳚ष्ण॒वा स्त्रयः॑ शु॒ण्ठाः शु॒ण्ठा स्त्रयो॑ वैष्ण॒वाः । \newline
2. त्रयो॑ वैष्ण॒वा वै᳚ष्ण॒वा स्त्रय॒ स्त्रयो॑ वैष्ण॒वा अ॑धीलोध॒कर्णा॑ अधीलोध॒कर्णा॑ वैष्ण॒वा स्त्रय॒ स्त्रयो॑ वैष्ण॒वा अ॑धीलोध॒कर्णाः᳚ । \newline
3. वै॒ष्ण॒वा अ॑धीलोध॒कर्णा॑ अधीलोध॒कर्णा॑ वैष्ण॒वा वै᳚ष्ण॒वा अ॑धीलोध॒कर्णा॒ स्त्रय॒ स्त्रयो॑ ऽधीलोध॒कर्णा॑ वैष्ण॒वा वै᳚ष्ण॒वा अ॑धीलोध॒कर्णा॒ स्त्रयः॑ । \newline
4. अ॒धी॒लो॒ध॒कर्णा॒ स्त्रय॒ स्त्रयो॑ ऽधीलोध॒कर्णा॑ अधीलोध॒कर्णा॒ स्त्रयो॒ विष्ण॑वे॒ विष्ण॑वे॒ त्रयो॑ ऽधीलोध॒कर्णा॑ अधीलोध॒कर्णा॒ स्त्रयो॒ विष्ण॑वे । \newline
5. अ॒धी॒लो॒ध॒कर्णा॒ इत्य॑धीलोध - कर्णाः᳚ । \newline
6. त्रयो॒ विष्ण॑वे॒ विष्ण॑वे॒ त्रय॒ स्त्रयो॒ विष्ण॑व उरुक्र॒मायो॑ रुक्र॒माय॒ विष्ण॑वे॒ त्रय॒ स्त्रयो॒ विष्ण॑व उरुक्र॒माय॑ । \newline
7. विष्ण॑व उरुक्र॒मायो॑ रुक्र॒माय॒ विष्ण॑वे॒ विष्ण॑व उरुक्र॒माय॑ लफ्सु॒दिनो॑ लफ्सु॒दिन॑ उरुक्र॒माय॒ विष्ण॑वे॒ विष्ण॑व उरुक्र॒माय॑ लफ्सु॒दिनः॑ । \newline
8. उ॒रु॒क्र॒माय॑ लफ्सु॒दिनो॑ लफ्सु॒दिन॑ उरुक्र॒मायो॑ रुक्र॒माय॑ लफ्सु॒दिन॒ स्त्रय॒ स्त्रयो॑ लफ्सु॒दिन॑ उरुक्र॒मायो॑ रुक्र॒माय॑ लफ्सु॒दिन॒ स्त्रयः॑ । \newline
9. उ॒रु॒क्र॒मायेत्यु॑रु - क्र॒माय॑ । \newline
10. ल॒फ्सु॒दिन॒ स्त्रय॒ स्त्रयो॑ लफ्सु॒दिनो॑ लफ्सु॒दिन॒ स्त्रयो॒ विष्ण॑वे॒ विष्ण॑वे॒ त्रयो॑ लफ्सु॒दिनो॑ लफ्सु॒दिन॒ स्त्रयो॒ विष्ण॑वे । \newline
11. त्रयो॒ विष्ण॑वे॒ विष्ण॑वे॒ त्रय॒ स्त्रयो॒ विष्ण॑व उरुगा॒यायो॑ रुगा॒याय॒ विष्ण॑वे॒ त्रय॒ स्त्रयो॒ विष्ण॑व उरुगा॒याय॑ । \newline
12. विष्ण॑व उरुगा॒यायो॑ रुगा॒याय॒ विष्ण॑वे॒ विष्ण॑व उरुगा॒याय॒ पञ्चा॑वीः॒ पञ्चा॑वी रुरुगा॒याय॒ विष्ण॑वे॒ विष्ण॑व उरुगा॒याय॒ पञ्चा॑वीः । \newline
13. उ॒रु॒गा॒याय॒ पञ्चा॑वीः॒ पञ्चा॑वी रुरुगा॒यायो॑ रुगा॒याय॒ पञ्चा॑वी स्ति॒स्र स्ति॒स्रः पञ्चा॑वी रुरुगा॒यायो॑ रुगा॒याय॒ पञ्चा॑वी स्ति॒स्रः । \newline
14. उ॒रु॒गा॒यायेत्यु॑रु - गा॒याय॑ । \newline
15. पञ्चा॑वी स्ति॒स्र स्ति॒स्रः पञ्चा॑वीः॒ पञ्चा॑वी स्ति॒स्र आ॑दि॒त्याना॑ मादि॒त्याना᳚म् ति॒स्रः पञ्चा॑वीः॒ पञ्चा॑वी स्ति॒स्र आ॑दि॒त्याना᳚म् । \newline
16. पञ्चा॑वी॒रिति॒ पञ्च॑ - अ॒वीः॒ । \newline
17. ति॒स्र आ॑दि॒त्याना॑ मादि॒त्याना᳚म् ति॒स्र स्ति॒स्र आ॑दि॒त्याना᳚म् त्रिव॒थ्सा स्त्रि॑व॒थ्सा आ॑दि॒त्याना᳚म् ति॒स्र स्ति॒स्र आ॑दि॒त्याना᳚म् त्रिव॒थ्साः । \newline
18. आ॒दि॒त्याना᳚म् त्रिव॒थ्सा स्त्रि॑व॒थ्सा आ॑दि॒त्याना॑ मादि॒त्याना᳚म् त्रिव॒थ्सा स्ति॒स्र स्ति॒स्र स्त्रि॑व॒थ्सा आ॑दि॒त्याना॑ मादि॒त्याना᳚म् त्रिव॒थ्सा स्ति॒स्रः । \newline
19. त्रि॒व॒थ्सा स्ति॒स्र स्ति॒स्र स्त्रि॑व॒थ्सा स्त्रि॑व॒थ्सा स्ति॒स्रो ऽङ्गि॑रसा॒ मङ्गि॑रसाम् ति॒स्र स्त्रि॑व॒थ्सा स्त्रि॑व॒थ्सा स्ति॒स्रो ऽङ्गि॑रसाम् । \newline
20. त्रि॒व॒थ्सा इति॑ त्रि - व॒थ्साः । \newline
21. ति॒स्रो ऽङ्गि॑रसा॒ मङ्गि॑रसाम् ति॒स्र स्ति॒स्रो ऽङ्गि॑रसा मैन्द्रावैष्ण॒वा ऐ᳚न्द्रावैष्ण॒वा अङ्गि॑रसाम् ति॒स्र स्ति॒स्रो ऽङ्गि॑रसा मैन्द्रावैष्ण॒वाः । \newline
22. अङ्गि॑रसा मैन्द्रावैष्ण॒वा ऐ᳚न्द्रावैष्ण॒वा अङ्गि॑रसा॒ मङ्गि॑रसा मैन्द्रावैष्ण॒वा गौ॒रल॑लामा गौ॒रल॑लामा ऐन्द्रावैष्ण॒वा अङ्गि॑रसा॒ मङ्गि॑रसा मैन्द्रावैष्ण॒वा गौ॒रल॑लामाः । \newline
23. ऐ॒न्द्रा॒वै॒ष्ण॒वा गौ॒रल॑लामा गौ॒रल॑लामा ऐन्द्रावैष्ण॒वा ऐ᳚न्द्रावैष्ण॒वा गौ॒रल॑लामा स्तूप॒रा स्तू॑प॒रा गौ॒रल॑लामा ऐन्द्रावैष्ण॒वा ऐ᳚न्द्रावैष्ण॒वा गौ॒रल॑लामा स्तूप॒राः । \newline
24. ऐ॒न्द्रा॒वै॒ष्ण॒वा इत्यै᳚न्द्रा - वै॒ष्ण॒वाः । \newline
25. गौ॒रल॑लामा स्तूप॒रा स्तू॑प॒रा गौ॒रल॑लामा गौ॒रल॑लामा स्तूप॒राः । \newline
26. गौ॒रल॑लामा॒ इति॑ गौ॒र - ल॒ला॒माः॒ । \newline
27. तू॒प॒रा इति॑ तूप॒राः । \newline
\pagebreak
\markright{ TS 5.6.17.1  \hfill https://www.vedavms.in \hfill}

\section{ TS 5.6.17.1 }

\textbf{TS 5.6.17.1 } \newline
\textbf{Samhita Paata} \newline

इन्द्रा॑य॒ राज्ञे॒ त्रयः॑ शितिपृ॒ष्ठा इन्द्रा॑या-धिरा॒जाय॒ त्रयः॒ शिति॑ककुद॒ इन्द्रा॑य स्व॒राज्ञे॒ त्रयः॒ शिति॑भस-दस्ति॒स्रस्तु॑र्यौ॒ह्यः॑ सा॒द्ध्यानां᳚ ति॒स्रः प॑ष्ठौ॒ह्यो॑ विश्वे॑षां दे॒वाना॑माग्ने॒न्द्राः कृ॒ष्णल॑लामास्तूप॒राः ॥ \newline

\textbf{Pada Paata} \newline

इन्द्रा॑य । राज्ञे᳚ । त्रयः॑ । शि॒ति॒पृ॒ष्ठा इति॑ शिति - पृ॒ष्ठाः । इन्द्रा॑य । अ॒धि॒रा॒जायेत्य॑धि - रा॒जाय॑ । त्रयः॑ । शिति॑ककुद॒ इति॒ शिति॑ - क॒कु॒दः॒ । इन्द्रा॑य । स्व॒राज्ञ्॒ इति॑ स्व - राज्ञे᳚ । त्रयः॑ । शिति॑भसद॒ इति॒ शिति॑ - भ॒स॒दः॒ । ति॒स्रः । तु॒र्यौ॒ह्यः॑ । सा॒द्ध्याना᳚म् । ति॒स्रः । प॒ष्ठौ॒ह्यः॑ । विश्वे॑षाम् । दे॒वाना᳚म् । आ॒ग्ने॒न्द्राः । कृ॒ष्णल॑लामा॒ इति॑ कृ॒ष्ण - ल॒ला॒माः । तू॒प॒राः ॥  \newline


\textbf{Krama Paata} \newline

इन्द्रा॑य॒ राज्ञे᳚ । राज्ञे॒ त्रयः॑ । त्रयः॑ शितिपृ॒ष्ठाः । शि॒ति॒पृ॒ष्ठा इन्द्रा॑य । शि॒ति॒पृ॒ष्ठा इति॑ शिति - पृ॒ष्ठाः । इन्द्रा॑याधिरा॒जाय॑ । अ॒धि॒रा॒जाय॒ त्रयः॑ । अ॒धि॒रा॒जायेत्य॑धि - रा॒जाय॑ । त्रयः॒ शिति॑ककुदः । शिति॑ककुद॒ इन्द्रा॑य । शिति॑ककुद॒ इति॒ शिति॑ - क॒कु॒दः॒ । इन्द्रा॑य स्व॒राज्ञे᳚ । स्व॒राज्ञे॒ त्रयः॑ । स्व॒राज्ञ्॒ इति॑ स्व - राज्ञे᳚ । त्रयः॒ शिति॑भसदः । शिति॑भसदस्ति॒स्रः । शिति॑भसद॒ इति॒ शिति॑ - भ॒स॒दः॒ । ति॒स्रस्तु॑र्यौ॒ह्यः॑ । तु॒र्यौ॒ह्यः॑ सा॒द्ध्याना᳚म् । सा॒द्ध्याना᳚म् ति॒स्रः । ति॒स्रः प॑ष्ठौ॒ह्यः॑ । प॒ष्ठौ॒ह्यो॑ विश्वे॑षाम् । विश्वे॑षाम् दे॒वाना᳚म् । दे॒वाना॑माग्ने॒न्द्राः । आ॒ग्ने॒न्द्राः कृ॒ष्णल॑लामाः । कृ॒ष्णल॑लामास्तूप॒राः । कृ॒ष्णल॑लामा॒ इति॑ कृ॒ष्ण - ल॒ला॒माः॒ । तू॒प॒रा इति॑ तूप॒राः । \newline

\textbf{Jatai Paata} \newline

1. इन्द्रा॑य॒ राज्ञे॒ राज्ञ्॒ इन्द्रा॒ येन्द्रा॑य॒ राज्ञे᳚ । \newline
2. राज्ञे॒ त्रय॒ स्त्रयो॒ राज्ञे॒ राज्ञे॒ त्रयः॑ । \newline
3. त्रयः॑ शितिपृ॒ष्ठाः शि॑तिपृ॒ष्ठा स्त्रय॒ स्त्रयः॑ शितिपृ॒ष्ठाः । \newline
4. शि॒ति॒पृ॒ष्ठा इन्द्रा॒ येन्द्रा॑य शितिपृ॒ष्ठाः शि॑तिपृ॒ष्ठा इन्द्रा॑य । \newline
5. शि॒ति॒पृ॒ष्ठा इति॑ शिति - पृ॒ष्ठाः । \newline
6. इन्द्रा॑या धिरा॒जाया॑ धिरा॒जा येन्द्रा॒ येन्द्रा॑या धिरा॒जाय॑ । \newline
7. अ॒धि॒रा॒जाय॒ त्रय॒ स्त्रयो॑ ऽधिरा॒जाया॑ धिरा॒जाय॒ त्रयः॑ । \newline
8. अ॒धि॒रा॒जायेत्य॑धि - रा॒जाय॑ । \newline
9. त्रयः॒ शिति॑ककुदः॒ शिति॑ककुद॒ स्त्रय॒ स्त्रयः॒ शिति॑ककुदः । \newline
10. शिति॑ककुद॒ इन्द्रा॒ येन्द्रा॑य॒ शिति॑ककुदः॒ शिति॑ककुद॒ इन्द्रा॑य । \newline
11. शिति॑ककुद॒ इति॒ शिति॑ - क॒कु॒दः॒ । \newline
12. इन्द्रा॑य स्व॒राज्ञे᳚ स्व॒राज्ञ्॒ इन्द्रा॒ येन्द्रा॑य स्व॒राज्ञे᳚ । \newline
13. स्व॒राज्ञे॒ त्रय॒ स्त्रयः॑ स्व॒राज्ञे᳚ स्व॒राज्ञे॒ त्रयः॑ । \newline
14. स्व॒राज्ञ्॒ इति॑ स्व - राज्ञे᳚ । \newline
15. त्रयः॒ शिति॑भसदः॒ शिति॑भसद॒ स्त्रय॒ स्त्रयः॒ शिति॑भसदः । \newline
16. शिति॑भसद स्ति॒स्र स्ति॒स्रः शिति॑भसदः॒ शिति॑भसद स्ति॒स्रः । \newline
17. शिति॑भसद॒ इति॒ शिति॑ - भ॒स॒दः॒ । \newline
18. ति॒स्र स्तु॑र्यौ॒ह्य॑ स्तुर्यौ॒ह्य॑ स्ति॒स्र स्ति॒स्र स्तु॑र्यौ॒ह्यः॑ । \newline
19. तु॒र्यौ॒ह्यः॑ सा॒द्ध्यानाꣳ॑ सा॒द्ध्याना᳚म् तुर्यौ॒ह्य॑ स्तुर्यौ॒ह्यः॑ सा॒द्ध्याना᳚म् । \newline
20. सा॒द्ध्याना᳚म् ति॒स्र स्ति॒स्रः सा॒द्ध्यानाꣳ॑ सा॒द्ध्याना᳚म् ति॒स्रः । \newline
21. ति॒स्रः प॑ष्ठौ॒ह्यः॑ पष्ठौ॒ह्य॑ स्ति॒स्र स्ति॒स्रः प॑ष्ठौ॒ह्यः॑ । \newline
22. प॒ष्ठौ॒ह्यो॑ विश्वे॑षां॒ ॅविश्वे॑षाम् पष्ठौ॒ह्यः॑ पष्ठौ॒ह्यो॑ विश्वे॑षाम् । \newline
23. विश्वे॑षाम् दे॒वाना᳚म् दे॒वानां॒ ॅविश्वे॑षां॒ ॅविश्वे॑षाम् दे॒वाना᳚म् । \newline
24. दे॒वाना॑ माग्ने॒न्द्रा आ᳚ग्ने॒न्द्रा दे॒वाना᳚म् दे॒वाना॑ माग्ने॒न्द्राः । \newline
25. आ॒ग्ने॒न्द्राः कृ॒ष्णल॑लामाः कृ॒ष्णल॑लामा आग्ने॒न्द्रा आ᳚ग्ने॒न्द्राः कृ॒ष्णल॑लामाः । \newline
26. कृ॒ष्णल॑लामा स्तूप॒रा स्तू॑प॒राः कृ॒ष्णल॑लामाः कृ॒ष्णल॑लामा स्तूप॒राः । \newline
27. कृ॒ष्णल॑लामा॒ इति॑ कृ॒ष्ण - ल॒ला॒माः॒ । \newline
28. तू॒प॒रा इति॑ तूप॒राः । \newline

\textbf{Ghana Paata } \newline

1. इन्द्रा॑य॒ राज्ञे॒ राज्ञ्॒ इन्द्रा॒ येन्द्रा॑य॒ राज्ञे॒ त्रय॒ स्त्रयो॒ राज्ञ्॒ इन्द्रा॒ येन्द्रा॑य॒ राज्ञे॒ त्रयः॑ । \newline
2. राज्ञे॒ त्रय॒ स्त्रयो॒ राज्ञे॒ राज्ञे॒ त्रयः॑ शितिपृ॒ष्ठाः शि॑तिपृ॒ष्ठा स्त्रयो॒ राज्ञे॒ राज्ञे॒ त्रयः॑ शितिपृ॒ष्ठाः । \newline
3. त्रयः॑ शितिपृ॒ष्ठाः शि॑तिपृ॒ष्ठा स्त्रय॒ स्त्रयः॑ शितिपृ॒ष्ठा इन्द्रा॒ येन्द्रा॑य शितिपृ॒ष्ठा स्त्रय॒ स्त्रयः॑ शितिपृ॒ष्ठा इन्द्रा॑य । \newline
4. शि॒ति॒पृ॒ष्ठा इन्द्रा॒ येन्द्रा॑य शितिपृ॒ष्ठाः शि॑तिपृ॒ष्ठा इन्द्रा॑या धिरा॒जाया॑ धिरा॒जा येन्द्रा॑य शितिपृ॒ष्ठाः शि॑तिपृ॒ष्ठा इन्द्रा॑या धिरा॒जाय॑ । \newline
5. शि॒ति॒पृ॒ष्ठा इति॑ शिति - पृ॒ष्ठाः । \newline
6. इन्द्रा॑या धिरा॒जा या॑धिरा॒जा येन्द्रा॒ येन्द्रा॑या धिरा॒जाय॒ त्रय॒ स्त्रयो॑ ऽधिरा॒जा येन्द्रा॒ येन्द्रा॑या धिरा॒जाय॒ त्रयः॑ । \newline
7. अ॒धि॒रा॒जाय॒ त्रय॒ स्त्रयो॑ ऽधिरा॒जाया॑ धिरा॒जाय॒ त्रयः॒ शिति॑ककुदः॒ शिति॑ककुद॒ स्त्रयो॑ ऽधिरा॒जाया॑ धिरा॒जाय॒ त्रयः॒ शिति॑ककुदः । \newline
8. अ॒धि॒रा॒जायेत्य॑धि - रा॒जाय॑ । \newline
9. त्रयः॒ शिति॑ककुदः॒ शिति॑ककुद॒ स्त्रय॒ स्त्रयः॒ शिति॑ककुद॒ इन्द्रा॒ येन्द्रा॑य॒ शिति॑ककुद॒ स्त्रय॒ स्त्रयः॒ शिति॑ककुद॒ इन्द्रा॑य । \newline
10. शिति॑ककुद॒ इन्द्रा॒ येन्द्रा॑य॒ शिति॑ककुदः॒ शिति॑ककुद॒ इन्द्रा॑य स्व॒राज्ञे᳚ स्व॒राज्ञ्॒ इन्द्रा॑य॒ शिति॑ककुदः॒ शिति॑ककुद॒ इन्द्रा॑य स्व॒राज्ञे᳚ । \newline
11. शिति॑ककुद॒ इति॒ शिति॑ - क॒कु॒दः॒ । \newline
12. इन्द्रा॑य स्व॒राज्ञे᳚ स्व॒राज्ञ्॒ इन्द्रा॒ येन्द्रा॑य स्व॒राज्ञे॒ त्रय॒ स्त्रयः॑ स्व॒राज्ञ्॒ इन्द्रा॒ येन्द्रा॑य स्व॒राज्ञे॒ त्रयः॑ । \newline
13. स्व॒राज्ञे॒ त्रय॒ स्त्रयः॑ स्व॒राज्ञे᳚ स्व॒राज्ञे॒ त्रयः॒ शिति॑भसदः॒ शिति॑भसद॒ स्त्रयः॑ स्व॒राज्ञे᳚ स्व॒राज्ञे॒ त्रयः॒ शिति॑भसदः । \newline
14. स्व॒राज्ञ्॒ इति॑ स्व - राज्ञे᳚ । \newline
15. त्रयः॒ शिति॑भसदः॒ शिति॑भसद॒ स्त्रय॒ स्त्रयः॒ शिति॑भसद स्ति॒स्र स्ति॒स्रः शिति॑भसद॒ स्त्रय॒ स्त्रयः॒ शिति॑भसद स्ति॒स्रः । \newline
16. शिति॑भसद स्ति॒स्र स्ति॒स्रः शिति॑भसदः॒ शिति॑भसद स्ति॒स्र स्तु॑र्यौ॒ह्य॑ स्तुर्यौ॒ह्य॑ स्ति॒स्रः शिति॑भसदः॒ शिति॑भसद स्ति॒स्र स्तु॑र्यौ॒ह्यः॑ । \newline
17. शिति॑भसद॒ इति॒ शिति॑ - भ॒स॒दः॒ । \newline
18. ति॒स्र स्तु॑र्यौ॒ह्य॑ स्तुर्यौ॒ह्य॑ स्ति॒स्र स्ति॒स्र स्तु॑र्यौ॒ह्यः॑ सा॒द्ध्यानाꣳ॑ सा॒द्ध्याना᳚म् तुर्यौ॒ह्य॑ स्ति॒स्र स्ति॒स्र स्तु॑र्यौ॒ह्यः॑ सा॒द्ध्याना᳚म् । \newline
19. तु॒र्यौ॒ह्यः॑ सा॒द्ध्यानाꣳ॑ सा॒द्ध्याना᳚म् तुर्यौ॒ह्य॑ स्तुर्यौ॒ह्यः॑ सा॒द्ध्याना᳚म् ति॒स्र स्ति॒स्रः सा॒द्ध्याना᳚म् तुर्यौ॒ह्य॑ स्तुर्यौ॒ह्यः॑ सा॒द्ध्याना᳚म् ति॒स्रः । \newline
20. सा॒द्ध्याना᳚म् ति॒स्र स्ति॒स्रः सा॒द्ध्यानाꣳ॑ सा॒द्ध्याना᳚म् ति॒स्रः प॑ष्ठौ॒ह्यः॑ पष्ठौ॒ह्य॑ स्ति॒स्रः सा॒द्ध्यानाꣳ॑ सा॒द्ध्याना᳚म् ति॒स्रः प॑ष्ठौ॒ह्यः॑ । \newline
21. ति॒स्रः प॑ष्ठौ॒ह्यः॑ पष्ठौ॒ह्य॑ स्ति॒स्र स्ति॒स्रः प॑ष्ठौ॒ह्यो॑ विश्वे॑षां॒ ॅविश्वे॑षाम् पष्ठौ॒ह्य॑ स्ति॒स्र स्ति॒स्रः प॑ष्ठौ॒ह्यो॑ विश्वे॑षाम् । \newline
22. प॒ष्ठौ॒ह्यो॑ विश्वे॑षां॒ ॅविश्वे॑षाम् पष्ठौ॒ह्यः॑ पष्ठौ॒ह्यो॑ विश्वे॑षाम् दे॒वाना᳚म् दे॒वानां॒ ॅविश्वे॑षाम् पष्ठौ॒ह्यः॑ पष्ठौ॒ह्यो॑ विश्वे॑षाम् दे॒वाना᳚म् । \newline
23. विश्वे॑षाम् दे॒वाना᳚म् दे॒वानां॒ ॅविश्वे॑षां॒ ॅविश्वे॑षाम् दे॒वाना॑ माग्ने॒न्द्रा आ᳚ग्ने॒न्द्रा दे॒वानां॒ ॅविश्वे॑षां॒ ॅविश्वे॑षाम् दे॒वाना॑ माग्ने॒न्द्राः । \newline
24. दे॒वाना॑ माग्ने॒न्द्रा आ᳚ग्ने॒न्द्रा दे॒वाना᳚म् दे॒वाना॑ माग्ने॒न्द्राः कृ॒ष्णल॑लामाः कृ॒ष्णल॑लामा आग्ने॒न्द्रा दे॒वाना᳚म् दे॒वाना॑ माग्ने॒न्द्राः कृ॒ष्णल॑लामाः । \newline
25. आ॒ग्ने॒न्द्राः कृ॒ष्णल॑लामाः कृ॒ष्णल॑लामा आग्ने॒न्द्रा आ᳚ग्ने॒न्द्राः कृ॒ष्णल॑लामा स्तूप॒रा स्तू॑प॒राः कृ॒ष्णल॑लामा आग्ने॒न्द्रा आ᳚ग्ने॒न्द्राः कृ॒ष्णल॑लामा स्तूप॒राः । \newline
26. कृ॒ष्णल॑लामा स्तूप॒रा स्तू॑प॒राः कृ॒ष्णल॑लामाः कृ॒ष्णल॑लामा स्तूप॒राः । \newline
27. कृ॒ष्णल॑लामा॒ इति॑ कृ॒ष्ण - ल॒ला॒माः॒ । \newline
28. तू॒प॒रा इति॑ तूप॒राः । \newline
\pagebreak
\markright{ TS 5.6.18.1  \hfill https://www.vedavms.in \hfill}

\section{ TS 5.6.18.1 }

\textbf{TS 5.6.18.1 } \newline
\textbf{Samhita Paata} \newline

अदि॑त्यै॒ त्रयो॑ रोहितै॒ता इ॑न्द्रा॒ण्यै त्रयः॑ कृष्णै॒ताः कु॒ह्वै᳚ त्रयो॑ऽरुणै॒तास्ति॒स्रो धे॒नवो॑ रा॒कायै॒ त्रयो॑ऽन॒ड्वाहः॑ सिनीवा॒ल्या आ᳚ग्नावैष्ण॒वा रोहि॑तललामास्तूप॒राः ॥ \newline

\textbf{Pada Paata} \newline

अदि॑त्यै । त्रयः॑ । रो॒हि॒तै॒ता इति॑ रोहित - ए॒ताः । इ॒न्द्रा॒ण्यै । त्रयः॑ । कृ॒ष्णै॒ता इति॑ कृष्ण - ए॒ताः । कु॒ह्वै᳚ । त्रयः॑ । अ॒रु॒णै॒ता इत्य॑रुण - ए॒ताः । ति॒स्रः । धे॒नवः॑ । रा॒कायै᳚ । त्रयः॑ । अ॒न॒ड्वाहः॑ । सि॒नी॒वा॒ल्यै । आ॒ग्ना॒वै॒ष्ण॒वा इत्या᳚ग्ना - वै॒ष्ण॒वाः । रोहि॑तललामा॒ इति॒ रोहि॑त - ल॒ला॒माः॒ । तू॒प॒राः ॥  \newline


\textbf{Krama Paata} \newline

अदि॑त्यै॒ त्रयः॑ । त्रयो॑ रोहितै॒ताः । रो॒हि॒तै॒ता इ॑न्द्रा॒ण्यै । रो॒हि॒तै॒ता इति॑ रोहित - ए॒ताः । इ॒न्द्रा॒ण्यै त्रयः॑ । त्रयः॑ कृष्णै॒ताः । कृ॒ष्णै॒ताः कु॒ह्वै᳚ । कृ॒ष्णै॒ता इति॑ कृष्ण - ए॒ताः । कु॒ह्वै᳚ त्रयः॑ । त्रयो॑ऽरुणै॒ताः । अ॒रु॒णै॒तास्ति॒स्रः । अ॒रु॒णै॒ता इत्य॑रुण - ए॒ताः । ति॒स्रो धे॒नवः॑ । धे॒नवो॑ रा॒कायै᳚ । रा॒कायै॒ त्रयः॑ । त्रयो॑ऽन॒ड्वाहः॑ । अ॒न॒ड्वाहः॑ सिनीवा॒ल्यै । सि॒नी॒वा॒ल्या आ᳚ग्नावैष्ण॒वाः । आ॒ग्ना॒वै॒ष्ण॒वा रोहि॑तललामाः । आ॒ग्ना॒वै॒ष्ण॒वा इत्या᳚ग्ना - वै॒ष्ण॒वाः । रोहि॑तललामा,स्तूप॒राः । रोहि॑तललामा॒ इति॒ रोहि॑त - ल॒ला॒माः॒ । तू॒प॒रा इति॑ तूप॒राः । \newline

\textbf{Jatai Paata} \newline

1. अदि॑त्यै॒ त्रय॒ स्त्रयो ऽदि॑त्या॒ अदि॑त्यै॒ त्रयः॑ । \newline
2. त्रयो॑ रोहितै॒ता रो॑हितै॒ता स्त्रय॒ स्त्रयो॑ रोहितै॒ताः । \newline
3. रो॒हि॒तै॒ता इ॑न्द्रा॒ण्या इ॑न्द्रा॒ण्यै रो॑हितै॒ता रो॑हितै॒ता इ॑न्द्रा॒ण्यै । \newline
4. रो॒हि॒तै॒ता इति॑ रोहित - ए॒ताः । \newline
5. इ॒न्द्रा॒ण्यै त्रय॒ स्त्रय॑ इन्द्रा॒ण्या इ॑न्द्रा॒ण्यै त्रयः॑ । \newline
6. त्रयः॑ कृष्णै॒ताः कृ॑ष्णै॒ता स्त्रय॒ स्त्रयः॑ कृष्णै॒ताः । \newline
7. कृ॒ष्णै॒ताः कु॒ह्वै॑ कु॒ह्वै॑ कृष्णै॒ताः कृ॑ष्णै॒ताः कु॒ह्वै᳚ । \newline
8. कृ॒ष्णै॒ता इति॑ कृष्ण - ए॒ताः । \newline
9. कु॒ह्वै᳚ त्रय॒ स्त्रयः॑ कु॒ह्वै॑ कु॒ह्वै᳚ त्रयः॑ । \newline
10. त्रयो॑ ऽरुणै॒ता अ॑रुणै॒ता स्त्रय॒ स्त्रयो॑ ऽरुणै॒ताः । \newline
11. अ॒रु॒णै॒ता स्ति॒स्र स्ति॒स्रो॑ ऽरुणै॒ता अ॑रुणै॒ता स्ति॒स्रः । \newline
12. अ॒रु॒णै॒ता इत्य॑रुण - ए॒ताः । \newline
13. ति॒स्रो धे॒नवो॑ धे॒नव॑ स्ति॒स्र स्ति॒स्रो धे॒नवः॑ । \newline
14. धे॒नवो॑ रा॒कायै॑ रा॒कायै॑ धे॒नवो॑ धे॒नवो॑ रा॒कायै᳚ । \newline
15. रा॒कायै॒ त्रय॒ स्त्रयो॑ रा॒कायै॑ रा॒कायै॒ त्रयः॑ । \newline
16. त्रयो॑ ऽन॒ड्वाहो॑ ऽन॒ड्वाह॒ स्त्रय॒ स्त्रयो॑ ऽन॒ड्वाहः॑ । \newline
17. अ॒न॒ड्वाहः॑ सिनीवा॒ल्यै सि॑नीवा॒ल्या अ॑न॒ड्वाहो॑ ऽन॒ड्वाहः॑ सिनीवा॒ल्यै । \newline
18. सि॒नी॒वा॒ल्या आ᳚ग्नावैष्ण॒वा आ᳚ग्नावैष्ण॒वाः सि॑नीवा॒ल्यै सि॑नीवा॒ल्या आ᳚ग्नावैष्ण॒वाः । \newline
19. आ॒ग्ना॒वै॒ष्ण॒वा रोहि॑तललामा॒ रोहि॑तललामा आग्नावैष्ण॒वा आ᳚ग्नावैष्ण॒वा रोहि॑तललामाः । \newline
20. आ॒ग्ना॒वै॒ष्ण॒वा इत्या᳚ग्ना - वै॒ष्ण॒वाः । \newline
21. रोहि॑तललामा स्तूप॒रा स्तू॑प॒रा रोहि॑तललामा॒ रोहि॑तललामा स्तूप॒राः । \newline
22. रोहि॑तललामा॒ इति॒ रोहि॑त - ल॒ला॒माः॒ । \newline
23. तू॒प॒रा इति॑ तूप॒राः । \newline

\textbf{Ghana Paata } \newline

1. अदि॑त्यै॒ त्रय॒ स्त्रयो ऽदि॑त्या॒ अदि॑त्यै॒ त्रयो॑ रोहितै॒ता रो॑हितै॒ता स्त्रयो ऽदि॑त्या॒ अदि॑त्यै॒ त्रयो॑ रोहितै॒ताः । \newline
2. त्रयो॑ रोहितै॒ता रो॑हितै॒ता स्त्रय॒ स्त्रयो॑ रोहितै॒ता इ॑न्द्रा॒ण्या इ॑न्द्रा॒ण्यै रो॑हितै॒ता स्त्रय॒ स्त्रयो॑ रोहितै॒ता इ॑न्द्रा॒ण्यै । \newline
3. रो॒हि॒तै॒ता इ॑न्द्रा॒ण्या इ॑न्द्रा॒ण्यै रो॑हितै॒ता रो॑हितै॒ता इ॑न्द्रा॒ण्यै त्रय॒ स्त्रय॑ इन्द्रा॒ण्यै रो॑हितै॒ता रो॑हितै॒ता इ॑न्द्रा॒ण्यै त्रयः॑ । \newline
4. रो॒हि॒तै॒ता इति॑ रोहित - ए॒ताः । \newline
5. इ॒न्द्रा॒ण्यै त्रय॒ स्त्रय॑ इन्द्रा॒ण्या इ॑न्द्रा॒ण्यै त्रयः॑ कृष्णै॒ताः कृ॑ष्णै॒ता स्त्रय॑ इन्द्रा॒ण्या इ॑न्द्रा॒ण्यै त्रयः॑ कृष्णै॒ताः । \newline
6. त्रयः॑ कृष्णै॒ताः कृ॑ष्णै॒ता स्त्रय॒ स्त्रयः॑ कृष्णै॒ताः कु॒ह्वै॑ कु॒ह्वै॑ कृष्णै॒ता स्त्रय॒ स्त्रयः॑ कृष्णै॒ताः कु॒ह्वै᳚ । \newline
7. कृ॒ष्णै॒ताः कु॒ह्वै॑ कु॒ह्वै॑ कृष्णै॒ताः कृ॑ष्णै॒ताः कु॒ह्वै᳚ त्रय॒ स्त्रयः॑ कु॒ह्वै॑ कृष्णै॒ताः कृ॑ष्णै॒ताः कु॒ह्वै᳚ त्रयः॑ । \newline
8. कृ॒ष्णै॒ता इति॑ कृष्ण - ए॒ताः । \newline
9. कु॒ह्वै᳚ त्रय॒ स्त्रयः॑ कु॒ह्वै॑ कु॒ह्वै᳚ त्रयो॑ ऽरुणै॒ता अ॑रुणै॒ता स्त्रयः॑ कु॒ह्वै॑ कु॒ह्वै᳚ त्रयो॑ ऽरुणै॒ताः । \newline
10. त्रयो॑ ऽरुणै॒ता अ॑रुणै॒ता स्त्रय॒ स्त्रयो॑ ऽरुणै॒ता स्ति॒स्र स्ति॒स्रो॑ ऽरुणै॒ता स्त्रय॒ स्त्रयो॑ ऽरुणै॒ता स्ति॒स्रः । \newline
11. अ॒रु॒णै॒ता स्ति॒स्र स्ति॒स्रो॑ ऽरुणै॒ता अ॑रुणै॒ता स्ति॒स्रो धे॒नवो॑ धे॒नव॑ स्ति॒स्रो॑ ऽरुणै॒ता अ॑रुणै॒ता स्ति॒स्रो धे॒नवः॑ । \newline
12. अ॒रु॒णै॒ता इत्य॑रुण - ए॒ताः । \newline
13. ति॒स्रो धे॒नवो॑ धे॒नव॑ स्ति॒स्र स्ति॒स्रो धे॒नवो॑ रा॒कायै॑ रा॒कायै॑ धे॒नव॑ स्ति॒स्र स्ति॒स्रो धे॒नवो॑ रा॒कायै᳚ । \newline
14. धे॒नवो॑ रा॒कायै॑ रा॒कायै॑ धे॒नवो॑ धे॒नवो॑ रा॒कायै॒ त्रय॒ स्त्रयो॑ रा॒कायै॑ धे॒नवो॑ धे॒नवो॑ रा॒कायै॒ त्रयः॑ । \newline
15. रा॒कायै॒ त्रय॒ स्त्रयो॑ रा॒कायै॑ रा॒कायै॒ त्रयो॑ ऽन॒ड्वाहो॑ ऽन॒ड्वाह॒ स्त्रयो॑ रा॒कायै॑ रा॒कायै॒ त्रयो॑ ऽन॒ड्वाहः॑ । \newline
16. त्रयो॑ ऽन॒ड्वाहो॑ ऽन॒ड्वाह॒ स्त्रय॒ स्त्रयो॑ ऽन॒ड्वाहः॑ सिनीवा॒ल्यै सि॑नीवा॒ल्या अ॑न॒ड्वाह॒ स्त्रय॒ स्त्रयो॑ ऽन॒ड्वाहः॑ सिनीवा॒ल्यै । \newline
17. अ॒न॒ड्वाहः॑ सिनीवा॒ल्यै सि॑नीवा॒ल्या अ॑न॒ड्वाहो॑ ऽन॒ड्वाहः॑ सिनीवा॒ल्या आ᳚ग्नावैष्ण॒वा आ᳚ग्नावैष्ण॒वाः सि॑नीवा॒ल्या अ॑न॒ड्वाहो॑ ऽन॒ड्वाहः॑ सिनीवा॒ल्या आ᳚ग्नावैष्ण॒वाः । \newline
18. सि॒नी॒वा॒ल्या आ᳚ग्नावैष्ण॒वा आ᳚ग्नावैष्ण॒वाः सि॑नीवा॒ल्यै सि॑नीवा॒ल्या आ᳚ग्नावैष्ण॒वा रोहि॑तललामा॒ रोहि॑तललामा आग्नावैष्ण॒वाः सि॑नीवा॒ल्यै सि॑नीवा॒ल्या आ᳚ग्नावैष्ण॒वा रोहि॑तललामाः । \newline
19. आ॒ग्ना॒वै॒ष्ण॒वा रोहि॑तललामा॒ रोहि॑तललामा आग्नावैष्ण॒वा आ᳚ग्नावैष्ण॒वा रोहि॑तललामा स्तूप॒रा स्तू॑प॒रा रोहि॑तललामा आग्नावैष्ण॒वा आ᳚ग्नावैष्ण॒वा रोहि॑तललामा स्तूप॒राः । \newline
20. आ॒ग्ना॒वै॒ष्ण॒वा इत्या᳚ग्ना - वै॒ष्ण॒वाः । \newline
21. रोहि॑तललामा स्तूप॒रा स्तू॑प॒रा रोहि॑तललामा॒ रोहि॑तललामा स्तूप॒राः । \newline
22. रोहि॑तललामा॒ इति॒ रोहि॑त - ल॒ला॒माः॒ । \newline
23. तू॒प॒रा इति॑ तूप॒राः । \newline
\pagebreak
\markright{ TS 5.6.19.1  \hfill https://www.vedavms.in \hfill}

\section{ TS 5.6.19.1 }

\textbf{TS 5.6.19.1 } \newline
\textbf{Samhita Paata} \newline

सौ॒म्यास्त्रयः॑ पि॒शंगाः॒ सोमा॑य॒ राज्ञे॒ त्रयः॑ सा॒रङ्गाः᳚ पार्ज॒न्या नभो॑रूपास्ति॒स्रो॑ऽजा म॒ल॒.हा इ॑न्द्रा॒ण्यै ति॒स्रो मे॒ष्य॑ आदि॒त्या द्या॑वापृथि॒व्या॑ मा॒लंगा᳚स्तूप॒राः ॥ \newline

\textbf{Pada Paata} \newline

सौ॒म्याः । त्रयः॑ । पि॒शङ्गाः᳚ । सोमा॑य । राज्ञे᳚ । त्रयः॑ । सा॒रङ्गाः᳚ । पा॒र्ज॒न्याः । नभो॑रूपा॒ इति॒ नभः॑-रू॒पाः॒ । ति॒स्रः । अ॒जाः । म॒ल॒.हाः । इ॒न्द्रा॒ण्यै । ति॒स्रः । मे॒ष्यः॑ । आ॒दि॒त्याः । द्या॒वा॒पृ॒थि॒व्या॑ इति॑ द्यावा - पृ॒थि॒व्याः᳚ । मा॒लङ्गाः᳚ । तू॒प॒राः ॥  \newline


\textbf{Krama Paata} \newline

सौ॒म्यास्त्रयः॑ । त्रयः॑ पि॒शङ्गाः᳚ । पि॒शङ्गाः॒ सोमा॑य । सोमा॑य॒ राज्ञे᳚ । राज्ञे॒ त्रयः॑ । त्रयः॑ सा॒रङ्गाः᳚ । सा॒रङ्गाः᳚ पार्ज॒न्याः । पा॒र्ज॒न्या नभो॑रूपाः । नभो॑रूपास्ति॒स्रः । नभो॑रूपा॒ इति॒ नभः॑ - रू॒पाः॒ । ति॒स्रो॑ऽजाः । अ॒जा म॒ल्॒.हाः । म॒ल्॒.हा इ॑न्द्रा॒ण्यै । इ॒न्द्रा॒ण्यै ति॒स्रः । ति॒स्रो मे॒ष्यः॑ । मे॒ष्य॑ आदि॒त्याः । आ॒दि॒त्या द्या॑वापृथि॒व्याः᳚ । द्या॒वा॒पृ॒थि॒व्या॑ मा॒लङ्गाः᳚ । द्या॒वा॒पृ॒थि॒व्या॑ इति॑ द्यावा - पृ॒थि॒व्याः᳚ । मा॒लङ्गा᳚स्तूप॒राः । तू॒प॒रा इति॑ तूप॒राः । \newline

\textbf{Jatai Paata} \newline

1. सौ॒म्या स्त्रय॒ स्त्रयः॑ सौ॒म्याः सौ॒म्या स्त्रयः॑ । \newline
2. त्रयः॑ पि॒शङ्गाः᳚ पि॒शङ्गा॒ स्त्रय॒ स्त्रयः॑ पि॒शङ्गाः᳚ । \newline
3. पि॒शङ्गाः॒ सोमा॑य॒ सोमा॑य पि॒शङ्गाः᳚ पि॒शङ्गाः॒ सोमा॑य । \newline
4. सोमा॑य॒ राज्ञे॒ राज्ञे॒ सोमा॑य॒ सोमा॑य॒ राज्ञे᳚ । \newline
5. राज्ञे॒ त्रय॒ स्त्रयो॒ राज्ञे॒ राज्ञे॒ त्रयः॑ । \newline
6. त्रयः॑ सा॒रङ्गाः᳚ सा॒रङ्गा॒ स्त्रय॒ स्त्रयः॑ सा॒रङ्गाः᳚ । \newline
7. सा॒रङ्गाः᳚ पार्ज॒न्याः पा᳚र्ज॒न्याः सा॒रङ्गाः᳚ सा॒रङ्गाः᳚ पार्ज॒न्याः । \newline
8. पा॒र्ज॒न्या नभो॑रूपा॒ नभो॑रूपाः पार्ज॒न्याः पा᳚र्ज॒न्या नभो॑रूपाः । \newline
9. नभो॑रूपा स्ति॒स्र स्ति॒स्रो नभो॑रूपा॒ नभो॑रूपा स्ति॒स्रः । \newline
10. नभो॑रूपा॒ इति॒ नभः॑ - रू॒पाः॒ । \newline
11. ति॒स्रो॑ ऽजा अ॒जा स्ति॒स्र स्ति॒स्रो॑ ऽजाः । \newline
12. अ॒जा म॒ल्॒.हा म॒ल्॒.हा अ॒जा अ॒जा म॒ल्॒.हाः । \newline
13. म॒ल॒.हा इ॑न्द्रा॒ण्या इ॑न्द्रा॒ण्यै म॒ल्॒.हा म॒ल्॒.हा इ॑न्द्रा॒ण्यै । \newline
14. इ॒न्द्रा॒ण्यै ति॒स्र स्ति॒स्र इ॑न्द्रा॒ण्या इ॑न्द्रा॒ण्यै ति॒स्रः । \newline
15. ति॒स्रो मे॒ष्यो॑ मे॒ष्य॑ स्ति॒स्र स्ति॒स्रो मे॒ष्यः॑ । \newline
16. मे॒ष्य॑ आदि॒त्या आ॑दि॒त्या मे॒ष्यो॑ मे॒ष्य॑ आदि॒त्याः । \newline
17. आ॒दि॒त्या द्या॑वापृथि॒व्या᳚ द्यावापृथि॒व्या॑ आदि॒त्या आ॑दि॒त्या द्या॑वापृथि॒व्याः᳚ । \newline
18. द्या॒वा॒पृ॒थि॒व्या॑ मा॒लङ्गा॑ मा॒लङ्गा᳚ द्यावापृथि॒व्या᳚ द्यावापृथि॒व्या॑ मा॒लङ्गाः᳚ । \newline
19. द्या॒वा॒पृ॒थि॒व्या॑ इति॑ द्यावा - पृ॒थि॒व्याः᳚ । \newline
20. मा॒लङ्गा᳚ स्तूप॒रा स्तू॑प॒रा मा॒लङ्गा॑ मा॒लङ्गा᳚ स्तूप॒राः । \newline
21. तू॒प॒रा इति॑ तूप॒राः । \newline

\textbf{Ghana Paata } \newline

1. सौ॒म्या स्त्रय॒ स्त्रयः॑ सौ॒म्याः सौ॒म्या स्त्रयः॑ पि॒शङ्गाः᳚ पि॒शङ्गा॒ स्त्रयः॑ सौ॒म्याः सौ॒म्या स्त्रयः॑ पि॒शङ्गाः᳚ । \newline
2. त्रयः॑ पि॒शङ्गाः᳚ पि॒शङ्गा॒ स्त्रय॒ स्त्रयः॑ पि॒शङ्गाः॒ सोमा॑य॒ सोमा॑य पि॒शङ्गा॒ स्त्रय॒ स्त्रयः॑ पि॒शङ्गाः॒ सोमा॑य । \newline
3. पि॒शङ्गाः॒ सोमा॑य॒ सोमा॑य पि॒शङ्गाः᳚ पि॒शङ्गाः॒ सोमा॑य॒ राज्ञे॒ राज्ञे॒ सोमा॑य पि॒शङ्गाः᳚ पि॒शङ्गाः॒ सोमा॑य॒ राज्ञे᳚ । \newline
4. सोमा॑य॒ राज्ञे॒ राज्ञे॒ सोमा॑य॒ सोमा॑य॒ राज्ञे॒ त्रय॒ स्त्रयो॒ राज्ञे॒ सोमा॑य॒ सोमा॑य॒ राज्ञे॒ त्रयः॑ । \newline
5. राज्ञे॒ त्रय॒ स्त्रयो॒ राज्ञे॒ राज्ञे॒ त्रयः॑ सा॒रङ्गाः᳚ सा॒रङ्गा॒ स्त्रयो॒ राज्ञे॒ राज्ञे॒ त्रयः॑ सा॒रङ्गाः᳚ । \newline
6. त्रयः॑ सा॒रङ्गाः᳚ सा॒रङ्गा॒ स्त्रय॒ स्त्रयः॑ सा॒रङ्गाः᳚ पार्ज॒न्याः पा᳚र्ज॒न्याः सा॒रङ्गा॒ स्त्रय॒ स्त्रयः॑ सा॒रङ्गाः᳚ पार्ज॒न्याः । \newline
7. सा॒रङ्गाः᳚ पार्ज॒न्याः पा᳚र्ज॒न्याः सा॒रङ्गाः᳚ सा॒रङ्गाः᳚ पार्ज॒न्या नभो॑रूपा॒ नभो॑रूपाः पार्ज॒न्याः सा॒रङ्गाः᳚ सा॒रङ्गाः᳚ पार्ज॒न्या नभो॑रूपाः । \newline
8. पा॒र्ज॒न्या नभो॑रूपा॒ नभो॑रूपाः पार्ज॒न्याः पा᳚र्ज॒न्या नभो॑रूपा स्ति॒स्र स्ति॒स्रो नभो॑रूपाः पार्ज॒न्याः पा᳚र्ज॒न्या नभो॑रूपा स्ति॒स्रः । \newline
9. नभो॑रूपा स्ति॒स्र स्ति॒स्रो नभो॑रूपा॒ नभो॑रूपा स्ति॒स्रो॑ ऽजा अ॒जा स्ति॒स्रो नभो॑रूपा॒ नभो॑रूपा स्ति॒स्रो॑ ऽजाः । \newline
10. नभो॑रूपा॒ इति॒ नभः॑ - रू॒पाः॒ । \newline
11. ति॒स्रो॑ ऽजा अ॒जा स्ति॒स्र स्ति॒स्रो॑ ऽजा म॒ल्॒.हा म॒ल्॒.हा अ॒जा स्ति॒स्र स्ति॒स्रो॑ ऽजा म॒ल्॒.हाः । \newline
12. अ॒जा म॒ल्॒.हा म॒ल्॒.हा अ॒जा अ॒जा म॒ल्॒.हा इ॑न्द्रा॒ण्या इ॑न्द्रा॒ण्यै म॒ल्॒.हा अ॒जा अ॒जा म॒ल्॒.हा इ॑न्द्रा॒ण्यै । \newline
13. म॒ल्॒.हा इ॑न्द्रा॒ण्या इ॑न्द्रा॒ण्यै म॒ल्॒.हा म॒ल्॒.हा इ॑न्द्रा॒ण्यै ति॒स्र स्ति॒स्र इ॑न्द्रा॒ण्यै म॒ल्॒.हा म॒ल्॒.हा इ॑न्द्रा॒ण्यै ति॒स्रः । \newline
14. इ॒न्द्रा॒ण्यै ति॒स्र स्ति॒स्र इ॑न्द्रा॒ण्या इ॑न्द्रा॒ण्यै ति॒स्रो मे॒ष्यो॑ मे॒ष्य॑ स्ति॒स्र इ॑न्द्रा॒ण्या इ॑न्द्रा॒ण्यै ति॒स्रो मे॒ष्यः॑ । \newline
15. ति॒स्रो मे॒ष्यो॑ मे॒ष्य॑ स्ति॒स्र स्ति॒स्रो मे॒ष्य॑ आदि॒त्या आ॑दि॒त्या मे॒ष्य॑ स्ति॒स्र स्ति॒स्रो मे॒ष्य॑ आदि॒त्याः । \newline
16. मे॒ष्य॑ आदि॒त्या आ॑दि॒त्या मे॒ष्यो॑ मे॒ष्य॑ आदि॒त्या द्या॑वापृथि॒व्या᳚ द्यावापृथि॒व्या॑ आदि॒त्या मे॒ष्यो॑ मे॒ष्य॑ आदि॒त्या द्या॑वापृथि॒व्याः᳚ । \newline
17. आ॒दि॒त्या द्या॑वापृथि॒व्या᳚ द्यावापृथि॒व्या॑ आदि॒त्या आ॑दि॒त्या द्या॑वापृथि॒व्या॑ मा॒लङ्गा॑ मा॒लङ्गा᳚ द्यावापृथि॒व्या॑ आदि॒त्या आ॑दि॒त्या द्या॑वापृथि॒व्या॑ मा॒लङ्गाः᳚ । \newline
18. द्या॒वा॒पृ॒थि॒व्या॑ मा॒लङ्गा॑ मा॒लङ्गा᳚ द्यावापृथि॒व्या᳚ द्यावापृथि॒व्या॑ मा॒लङ्गा᳚ स्तूप॒रा स्तू॑प॒रा मा॒लङ्गा᳚ द्यावापृथि॒व्या᳚ द्यावापृथि॒व्या॑ मा॒लङ्गा᳚ स्तूप॒राः । \newline
19. द्या॒वा॒पृ॒थि॒व्या॑ इति॑ द्यावा - पृ॒थि॒व्याः᳚ । \newline
20. मा॒लङ्गा᳚ स्तूप॒रा स्तू॑प॒रा मा॒लङ्गा॑ मा॒लङ्गा᳚ स्तूप॒राः । \newline
21. तू॒प॒रा इति॑ तूप॒राः । \newline
\pagebreak
\markright{ TS 5.6.20.1  \hfill https://www.vedavms.in \hfill}

\section{ TS 5.6.20.1 }

\textbf{TS 5.6.20.1 } \newline
\textbf{Samhita Paata} \newline

वा॒रु॒णास्त्रयः॑ कृ॒ष्णल॑लामा॒ वरु॑णाय॒ राज्ञे॒ त्रयो॒ रोहि॑तललामा॒ वरु॑णाय रि॒शाद॑से॒ त्रयो॑ऽरु॒णल॑लामाः शि॒ल्पास्त्रयो॑ वैश्वदे॒वास्त्रयः॒ पृश्न॑यः सर्वदेव॒त्या॑ ऐन्द्रासू॒राः श्येत॑ललामास्तूप॒राः ॥ \newline

\textbf{Pada Paata} \newline

वा॒रु॒णाः । त्रयः॑ । कृ॒ष्णल॑लामा॒ इति॑ कृ॒ष्ण - ल॒ला॒माः॒ । वरु॑णाय । राज्ञे᳚ । त्रयः॑ । रोहि॑तललामा॒ इति॒ रोहि॑त - ल॒ला॒माः॒ । वरु॑णाय । रि॒शाद॑स॒ इति॑ रिश - अद॑से । त्रयः॑ । अ॒रु॒णल॑लामा॒ इत्य॑रु॒ण - ल॒ला॒माः॒ । शि॒ल्पाः । त्रयः॑ । वै॒श्व॒दे॒वा इति॑ वैश्व - दे॒वाः । त्रयः॑ । पृश्न॑यः । स॒र्व॒दे॒व॒त्या॑ इति॑ सर्व - दे॒व॒त्याः᳚ । ऐ॒न्द्रा॒सू॒रा इत्यै᳚न्द्रा - सू॒राः । श्येत॑ललामा॒ इति॒ श्येत॑ - ल॒ला॒माः॒ । तू॒प॒राः ॥  \newline


\textbf{Krama Paata} \newline

वा॒रु॒णास्त्रयः॑ । त्रयः॑ कृ॒ष्णल॑लामाः । कृ॒ष्णल॑लामा॒ वरु॑णाय । कृ॒ष्णल॑लामा॒ इति॑ कृ॒ष्ण - ल॒ला॒माः॒ । वरु॑णाय॒ राज्ञे᳚ । राज्ञे॒ त्रयः॑ । त्रयो॒ रोहि॑तललामाः । रोहि॑तललामा॒ वरु॑णाय । रोहि॑तललामा॒ इति॒ रोहि॑त - ल॒ला॒माः॒ । वरु॑णाय रि॒शाद॑से । रि॒शाद॑से॒ त्रयः॑ । रि॒शाद॑स॒ इति॑ रिश - अद॑से । त्रयो॑ऽरु॒णल॑लामाः । अ॒रु॒णल॑लामाः शि॒ल्पाः । अ॒रु॒णल॑लामा॒ इत्य॑रु॒ण - ल॒ला॒माः॒ । शि॒ल्पास्त्रयः॑ । त्रयो॑ वैश्वदे॒वाः । वै॒श्व॒दे॒वास्त्रयः॑ । वै॒श्व॒दे॒वा इति॑ वैश्व - दे॒वाः । त्रयः॒ पृश्ञ॑यः । पृश्ञ॑यः सर्वदेव॒त्याः᳚ । स॒र्व॒दे॒व॒त्या॑ ऐन्द्रासू॒राः । स॒र्व॒दे॒व॒त्या॑ इति॑ सर्व - दे॒व॒त्याः᳚ । ऐ॒न्द्रा॒सू॒राः श्येत॑ललामाः । ऐ॒न्द्रा॒सू॒रा इत्यै᳚न्द्रा - सू॒राः । श्येत॑ललामास्तूप॒राः । श्येत॑ललामा॒ इति॒ श्येत॑ - ल॒ला॒माः॒ । तू॒प॒रा इति॑ तूप॒राः । \newline

\textbf{Jatai Paata} \newline

1. वा॒रु॒णा स्त्रय॒ स्त्रयो॑ वारु॒णा वा॑रु॒णा स्त्रयः॑ । \newline
2. त्रयः॑ कृ॒ष्णल॑लामाः कृ॒ष्णल॑लामा॒ स्त्रय॒ स्त्रयः॑ कृ॒ष्णल॑लामाः । \newline
3. कृ॒ष्णल॑लामा॒ वरु॑णाय॒ वरु॑णाय कृ॒ष्णल॑लामाः कृ॒ष्णल॑लामा॒ वरु॑णाय । \newline
4. कृ॒ष्णल॑लामा॒ इति॑ कृ॒ष्ण - ल॒ला॒माः॒ । \newline
5. वरु॑णाय॒ राज्ञे॒ राज्ञे॒ वरु॑णाय॒ वरु॑णाय॒ राज्ञे᳚ । \newline
6. राज्ञे॒ त्रय॒ स्त्रयो॒ राज्ञे॒ राज्ञे॒ त्रयः॑ । \newline
7. त्रयो॒ रोहि॑तललामा॒ रोहि॑तललामा॒ स्त्रय॒ स्त्रयो॒ रोहि॑तललामाः । \newline
8. रोहि॑तललामा॒ वरु॑णाय॒ वरु॑णाय॒ रोहि॑तललामा॒ रोहि॑तललामा॒ वरु॑णाय । \newline
9. रोहि॑तललामा॒ इति॒ रोहि॑त - ल॒ला॒माः॒ । \newline
10. वरु॑णाय रि॒शाद॑से रि॒शाद॑से॒ वरु॑णाय॒ वरु॑णाय रि॒शाद॑से । \newline
11. रि॒शाद॑से॒ त्रय॒ स्त्रयो॑ रि॒शाद॑से रि॒शाद॑से॒ त्रयः॑ । \newline
12. रि॒शाद॑स॒ इति॑ रिश - अद॑से । \newline
13. त्रयो॑ ऽरु॒णल॑लामा अरु॒णल॑लामा॒ स्त्रय॒ स्त्रयो॑ ऽरु॒णल॑लामाः । \newline
14. अ॒रु॒णल॑लामाः शि॒ल्पाः शि॒ल्पा अ॑रु॒णल॑लामा अरु॒णल॑लामाः शि॒ल्पाः । \newline
15. अ॒रु॒णल॑लामा॒ इत्य॑रु॒ण - ल॒ला॒माः॒ । \newline
16. शि॒ल्पा स्त्रय॒ स्त्रयः॑ शि॒ल्पाः शि॒ल्पा स्त्रयः॑ । \newline
17. त्रयो॑ वैश्वदे॒वा वै᳚श्वदे॒वा स्त्रय॒ स्त्रयो॑ वैश्वदे॒वाः । \newline
18. वै॒श्व॒दे॒वा स्त्रय॒ स्त्रयो॑ वैश्वदे॒वा वै᳚श्वदे॒वा स्त्रयः॑ । \newline
19. वै॒श्व॒दे॒वा इति॑ वैश्व - दे॒वाः । \newline
20. त्रयः॒ पृश्ञ॑यः॒ पृश्ञ॑य॒ स्त्रय॒ स्त्रयः॒ पृश्ञ॑यः । \newline
21. पृश्ञ॑यः सर्वदेव॒त्याः᳚ सर्वदेव॒त्याः᳚ पृश्ञ॑यः॒ पृश्ञ॑यः सर्वदेव॒त्याः᳚ । \newline
22. स॒र्व॒दे॒व॒त्या॑ ऐन्द्रासू॒रा ऐ᳚न्द्रासू॒राः स॑र्वदेव॒त्याः᳚ सर्वदेव॒त्या॑ ऐन्द्रासू॒राः । \newline
23. स॒र्व॒दे॒व॒त्या॑ इति॑ सर्व - दे॒व॒त्याः᳚ । \newline
24. ऐ॒न्द्रा॒सू॒राः श्येत॑ललामाः॒ श्येत॑ललामा ऐन्द्रासू॒रा ऐ᳚न्द्रासू॒राः श्येत॑ललामाः । \newline
25. ऐ॒न्द्रा॒सू॒रा इत्यै᳚न्द्रा - सू॒राः । \newline
26. श्येत॑ललामा स्तूप॒रा स्तू॑प॒राः श्येत॑ललामाः॒ श्येत॑ललामा स्तूप॒राः । \newline
27. श्येत॑ललामा॒ इति॒ श्येत॑ - ल॒ला॒माः॒ । \newline
28. तू॒प॒रा इति॑ तूप॒राः । \newline

\textbf{Ghana Paata } \newline

1. वा॒रु॒णा स्त्रय॒ स्त्रयो॑ वारु॒णा वा॑रु॒णा स्त्रयः॑ कृ॒ष्णल॑लामाः कृ॒ष्णल॑लामा॒ स्त्रयो॑ वारु॒णा वा॑रु॒णा स्त्रयः॑ कृ॒ष्णल॑लामाः । \newline
2. त्रयः॑ कृ॒ष्णल॑लामाः कृ॒ष्णल॑लामा॒ स्त्रय॒ स्त्रयः॑ कृ॒ष्णल॑लामा॒ वरु॑णाय॒ वरु॑णाय कृ॒ष्णल॑लामा॒ स्त्रय॒ स्त्रयः॑ कृ॒ष्णल॑लामा॒ वरु॑णाय । \newline
3. कृ॒ष्णल॑लामा॒ वरु॑णाय॒ वरु॑णाय कृ॒ष्णल॑लामाः कृ॒ष्णल॑लामा॒ वरु॑णाय॒ राज्ञे॒ राज्ञे॒ वरु॑णाय कृ॒ष्णल॑लामाः कृ॒ष्णल॑लामा॒ वरु॑णाय॒ राज्ञे᳚ । \newline
4. कृ॒ष्णल॑लामा॒ इति॑ कृ॒ष्ण - ल॒ला॒माः॒ । \newline
5. वरु॑णाय॒ राज्ञे॒ राज्ञे॒ वरु॑णाय॒ वरु॑णाय॒ राज्ञे॒ त्रय॒ स्त्रयो॒ राज्ञे॒ वरु॑णाय॒ वरु॑णाय॒ राज्ञे॒ त्रयः॑ । \newline
6. राज्ञे॒ त्रय॒ स्त्रयो॒ राज्ञे॒ राज्ञे॒ त्रयो॒ रोहि॑तललामा॒ रोहि॑तललामा॒ स्त्रयो॒ राज्ञे॒ राज्ञे॒ त्रयो॒ रोहि॑तललामाः । \newline
7. त्रयो॒ रोहि॑तललामा॒ रोहि॑तललामा॒ स्त्रय॒ स्त्रयो॒ रोहि॑तललामा॒ वरु॑णाय॒ वरु॑णाय॒ रोहि॑तललामा॒ स्त्रय॒ स्त्रयो॒ रोहि॑तललामा॒ वरु॑णाय । \newline
8. रोहि॑तललामा॒ वरु॑णाय॒ वरु॑णाय॒ रोहि॑तललामा॒ रोहि॑तललामा॒ वरु॑णाय रि॒शाद॑से रि॒शाद॑से॒ वरु॑णाय॒ रोहि॑तललामा॒ रोहि॑तललामा॒ वरु॑णाय रि॒शाद॑से । \newline
9. रोहि॑तललामा॒ इति॒ रोहि॑त - ल॒ला॒माः॒ । \newline
10. वरु॑णाय रि॒शाद॑से रि॒शाद॑से॒ वरु॑णाय॒ वरु॑णाय रि॒शाद॑से॒ त्रय॒ स्त्रयो॑ रि॒शाद॑से॒ वरु॑णाय॒ वरु॑णाय रि॒शाद॑से॒ त्रयः॑ । \newline
11. रि॒शाद॑से॒ त्रय॒ स्त्रयो॑ रि॒शाद॑से रि॒शाद॑से॒ त्रयो॑ ऽरु॒णल॑लामा अरु॒णल॑लामा॒ स्त्रयो॑ रि॒शाद॑से रि॒शाद॑से॒ त्रयो॑ ऽरु॒णल॑लामाः । \newline
12. रि॒शाद॑स॒ इति॑ रिश - अद॑से । \newline
13. त्रयो॑ ऽरु॒णल॑लामा अरु॒णल॑लामा॒ स्त्रय॒ स्त्रयो॑ ऽरु॒णल॑लामाः शि॒ल्पाः शि॒ल्पा अ॑रु॒णल॑लामा॒ स्त्रय॒ स्त्रयो॑ ऽरु॒णल॑लामाः शि॒ल्पाः । \newline
14. अ॒रु॒णल॑लामाः शि॒ल्पाः शि॒ल्पा अ॑रु॒णल॑लामा अरु॒णल॑लामाः शि॒ल्पा स्त्रय॒ स्त्रयः॑ शि॒ल्पा अ॑रु॒णल॑लामा अरु॒णल॑लामाः शि॒ल्पा स्त्रयः॑ । \newline
15. अ॒रु॒णल॑लामा॒ इत्य॑रु॒ण - ल॒ला॒माः॒ । \newline
16. शि॒ल्पा स्त्रय॒ स्त्रयः॑ शि॒ल्पाः शि॒ल्पा स्त्रयो॑ वैश्वदे॒वा वै᳚श्वदे॒वा स्त्रयः॑ शि॒ल्पाः शि॒ल्पा स्त्रयो॑ वैश्वदे॒वाः । \newline
17. त्रयो॑ वैश्वदे॒वा वै᳚श्वदे॒वा स्त्रय॒ स्त्रयो॑ वैश्वदे॒वा स्त्रय॒ स्त्रयो॑ वैश्वदे॒वा स्त्रय॒ स्त्रयो॑ वैश्वदे॒वा स्त्रयः॑ । \newline
18. वै॒श्व॒दे॒वा स्त्रय॒ स्त्रयो॑ वैश्वदे॒वा वै᳚श्वदे॒वा स्त्रयः॒ पृश्ञ॑यः॒ पृश्ञ॑य॒ स्त्रयो॑ वैश्वदे॒वा वै᳚श्वदे॒वा स्त्रयः॒ पृश्ञ॑यः । \newline
19. वै॒श्व॒दे॒वा इति॑ वैश्व - दे॒वाः । \newline
20. त्रयः॒ पृश्ञ॑यः॒ पृश्ञ॑य॒ स्त्रय॒ स्त्रयः॒ पृश्ञ॑यः सर्वदेव॒त्याः᳚ सर्वदेव॒त्याः᳚ पृश्ञ॑य॒ स्त्रय॒ स्त्रयः॒ पृश्ञ॑यः सर्वदेव॒त्याः᳚ । \newline
21. पृश्ञ॑यः सर्वदेव॒त्याः᳚ सर्वदेव॒त्याः᳚ पृश्ञ॑यः॒ पृश्ञ॑यः सर्वदेव॒त्या॑ ऐन्द्रासू॒रा ऐ᳚न्द्रासू॒राः स॑र्वदेव॒त्याः᳚ पृश्ञ॑यः॒ पृश्ञ॑यः सर्वदेव॒त्या॑ ऐन्द्रासू॒राः । \newline
22. स॒र्व॒दे॒व॒त्या॑ ऐन्द्रासू॒रा ऐ᳚न्द्रासू॒राः स॑र्वदेव॒त्याः᳚ सर्वदेव॒त्या॑ ऐन्द्रासू॒राः श्येत॑ललामाः॒ श्येत॑ललामा ऐन्द्रासू॒राः स॑र्वदेव॒त्याः᳚ सर्वदेव॒त्या॑ ऐन्द्रासू॒राः श्येत॑ललामाः । \newline
23. स॒र्व॒दे॒व॒त्या॑ इति॑ सर्व - दे॒व॒त्याः᳚ । \newline
24. ऐ॒न्द्रा॒सू॒राः श्येत॑ललामाः॒ श्येत॑ललामा ऐन्द्रासू॒रा ऐ᳚न्द्रासू॒राः श्येत॑ललामा स्तूप॒रा स्तू॑प॒राः श्येत॑ललामा ऐन्द्रासू॒रा ऐ᳚न्द्रासू॒राः श्येत॑ललामा स्तूप॒राः । \newline
25. ऐ॒न्द्रा॒सू॒रा इत्यै᳚न्द्रा - सू॒राः । \newline
26. श्येत॑ललामा स्तूप॒रा स्तू॑प॒राः श्येत॑ललामाः॒ श्येत॑ललामा स्तूप॒राः । \newline
27. श्येत॑ललामा॒ इति॒ श्येत॑ - ल॒ला॒माः॒ । \newline
28. तू॒प॒रा इति॑ तूप॒राः । \newline
\pagebreak
\markright{ TS 5.6.21.1  \hfill https://www.vedavms.in \hfill}

\section{ TS 5.6.21.1 }

\textbf{TS 5.6.21.1 } \newline
\textbf{Samhita Paata} \newline

सोमा॑य स्व॒राज्ञे॑-ऽनोवा॒हाव॑न॒ड्वाहा॑-विन्द्रा॒ग्निभ्या॑-मोजो॒दाभ्या॒मुष्टा॑रा-विन्द्रा॒ग्निभ्यां᳚ बल॒दाभ्याꣳ॑ सीरवा॒हाववी॒ द्वे धे॒नू भौ॒मी दि॒ग्भ्यो वड॑बे॒ द्वे धे॒नू भौ॒मी वै॑रा॒जी पु॑रु॒षी द्वे धे॒नू भौ॒मी व॒यव॑ आरोहणवा॒हाव॑न॒ड्वाहौ॑ वारु॒णी कृ॒ष्णे व॒शे अ॑रा॒ड्यौ॑ दि॒व्यावृ॑ष॒भौ प॑रिम॒रौ ॥ \newline

\textbf{Pada Paata} \newline

सोमा॑य । स्व॒राज्ञ्॒ इति॑ स्व - राज्ञे᳚ । अ॒नो॒वा॒हावित्य॑नः - वा॒हौ । अ॒न॒ड्वाहौ᳚ । इ॒न्द्रा॒ग्निभ्या॒मिती᳚न्द्रा॒ग्नि - भ्या॒म् । ओ॒जो॒दाभ्या॒मित्यो॑जः-दाभ्या᳚म् । उष्टा॑रौ । इ॒न्द्रा॒ग्निभ्या॒मिती᳚न्द्रा॒ग्नि-भ्या॒म् । ब॒ल॒दाभ्या॒मिति॑ बल - दाभ्या᳚म् । सी॒र॒वा॒हाविति॑ सीर - वा॒हौ । अवी॒ इति॑ । द्वे इति॑ । धे॒नू इति॑ । भौ॒मी इति॑ । दि॒ग्भ्य इति॑ दिक् - भ्यः । वड॑बे॒ इति॑ । द्वे इति॑ । धे॒नू इति॑ । भौ॒मी इति॑ । वै॒रा॒जी इति॑ । पु॒रु॒षी इति॑ । द्वे इति॑ । धे॒नू इति॑ । भौ॒मी इति॑ । वा॒यवे᳚ । आ॒रो॒ह॒ण॒वा॒हावित्या॑रोहण - वा॒हौ । अ॒न॒ड्वाहौ᳚ । वा॒रु॒णी इति॑ । कृ॒ष्णे इति॑ । व॒शे इति॑ । अ॒रा॒ड्यौ᳚ । दि॒व्यौ । ऋ॒ष॒भौ । प॒रि॒म॒राविति॑ परि - म॒रौ ॥  \newline


\textbf{Krama Paata} \newline

सोमा॑य स्व॒राज्ञे᳚ । स्व॒राज्ञे॑ऽनोवा॒हौ । स्व॒राज्ञ्॒ इति॑ स्व - राज्ञे᳚ । अ॒नो॒वा॒हाव॑न॒ड्वाहौ᳚ । अ॒नो॒वा॒हावित्य॑नः - वा॒हौ । अ॒न॒ड्वाहा॑विन्द्रा॒ग्निभ्या᳚म् । इ॒न्द्रा॒ग्निभ्या॑मोजो॒दाभ्या᳚म् । इ॒न्द्रा॒ग्निभ्या॒मिती᳚न्द्रा॒ग्नि - भ्या॒म् । ओ॒जो॒दाभ्या॒मुष्टा॑रौ । ओ॒जो॒दाभ्या॒मित्यो॑जः - दाभ्य᳚म् । उष्टा॑राविन्द्रा॒ग्निभ्या᳚म् । इ॒न्द्रा॒ग्निभ्या᳚म् बल॒दाभ्या᳚म् । इ॒न्द्रा॒ग्निभ्या॒मिती᳚न्द्रा॒ग्नि - भ्या॒म् । ब॒ल॒दाभ्याꣳ ॑सीरवा॒हौ । ब॒ल॒दाभ्या॒मिति॑ बल - दाभ्या᳚म् । सी॒र॒वा॒हाववी᳚ । सी॒र॒वा॒हाविति॑ सीर - वा॒हौ । अवी॒ द्वे । अवी॒ इत्यवी᳚ । द्वे धे॒नू । द्वे इति॒ द्वे । धे॒नू भौ॒मी । धे॒नू इति॑ धे॒नू । भौ॒मी दि॒ग्भ्यः । भौ॒मी इति॑ भौ॒मी । दि॒ग्भ्यो वड॑बे । दि॒ग्भ्य इति॑ दिक् - भ्यः । वड॑बे॒ द्वे । वड॑बे॒ इति॒ वड॑बे । द्वे धे॒नू । द्वे इति॒ द्वे । धे॒नू भौ॒मी । धे॒नू इति॑ धे॒नू । भौ॒मी वै॑रा॒जी । भौ॒मी इति॑ भौ॒मी । वै॒रा॒जी पु॑रु॒षी । वै॒रा॒जी इति॑ वैरा॒जी । पु॒रु॒षी द्वे । पु॒रु॒षी इति॑ पुरु॒षी । द्वे धे॒नू । द्वे इति॒ द्वे । धे॒नू भौ॒मी । धे॒नू इति॑ धे॒नू । भौ॒मी वा॒यवे᳚ । भौ॒मी इति॑ भौ॒मी । वा॒यव॑ आरोहणवा॒हौ । आ॒रो॒ह॒ण॒वा॒हाव॑न॒ड्वाहौ᳚ । आ॒रो॒ह॒ण॒वा॒हावित्या॑रोहण - वा॒हौ । अ॒न॒ड्॒वाहौ॑ वारु॒णी । वा॒रु॒णी कृ॒ष्णे । वा॒रु॒णी इति॑ वारु॒णी । कृ॒ष्णे व॒शे । कृ॒ष्णे इति॑ कृ॒ष्णे । व॒शे अ॑रा॒ड्यौ᳚ । व॒शे इति॑ व॒शे । अ॒रा॒ड्यौ॑ दि॒व्यौ । दि॒व्यावृ॑ष॒भौ । ऋ॒ष॒भौ प॑रिम॒रौ । प॒रि॒म॒राविति॑ परि - म॒रौ । \newline

\textbf{Jatai Paata} \newline

1. सोमा॑य स्व॒राज्ञे᳚ स्व॒राज्ञे॒ सोमा॑य॒ सोमा॑य स्व॒राज्ञे᳚ । \newline
2. स्व॒राज्ञे॑ ऽनोवा॒हा व॑नोवा॒हौ स्व॒राज्ञे᳚ स्व॒राज्ञे॑ ऽनोवा॒हौ । \newline
3. स्व॒राज्ञ्॒ इति॑ स्व - राज्ञे᳚ । \newline
4. अ॒नो॒वा॒हा व॑न॒ड्वाहा॑ वन॒ड्वाहा॑ वनोवा॒हा व॑नोवा॒हा व॑न॒ड्वाहौ᳚ । \newline
5. अ॒नो॒वा॒हावित्य॑नः - वा॒हौ । \newline
6. अ॒न॒ड्वाहा॑ विन्द्रा॒ग्निभ्या॑ मिन्द्रा॒ग्निभ्या॑ मन॒ड्वाहा॑ वन॒ड्वाहा॑ विन्द्रा॒ग्निभ्या᳚म् । \newline
7. इ॒न्द्रा॒ग्निभ्या॑ मोजो॒दाभ्या॑ मोजो॒दाभ्या॑ मिन्द्रा॒ग्निभ्या॑ मिन्द्रा॒ग्निभ्या॑ मोजो॒दाभ्या᳚म् । \newline
8. इ॒न्द्रा॒ग्निभ्या॒मिती᳚न्द्रा॒ग्नि - भ्या॒म् । \newline
9. ओ॒जो॒दाभ्या॒ मुष्टा॑रा॒ वुष्टा॑रा वोजो॒दाभ्या॑ मोजो॒दाभ्या॒ मुष्टा॑रौ । \newline
10. ओ॒जो॒दाभ्या॒मित्यो॑जः - दाभ्या᳚म् । \newline
11. उष्टा॑रा विन्द्रा॒ग्निभ्या॑ मिन्द्रा॒ग्निभ्या॒ मुष्टा॑रा॒ वुष्टा॑रा विन्द्रा॒ग्निभ्या᳚म् । \newline
12. इ॒न्द्रा॒ग्निभ्या᳚म् बल॒दाभ्या᳚म् बल॒दाभ्या॑ मिन्द्रा॒ग्निभ्या॑ मिन्द्रा॒ग्निभ्या᳚म् बल॒दाभ्या᳚म् । \newline
13. इ॒न्द्रा॒ग्निभ्या॒मिती᳚न्द्रा॒ग्नि - भ्या॒म् । \newline
14. ब॒ल॒दाभ्याꣳ॑ सीरवा॒हौ सी॑रवा॒हौ ब॑ल॒दाभ्या᳚म् बल॒दाभ्याꣳ॑ सीरवा॒हौ । \newline
15. ब॒ल॒दाभ्या॒मिति॑ बल - दाभ्या᳚म् । \newline
16. सी॒र॒वा॒हा ववी॒ अवी॑ सीरवा॒हौ सी॑रवा॒हा ववी᳚ । \newline
17. सी॒र॒वा॒हाविति॑ सीर - वा॒हौ । \newline
18. अवी॒ द्वे द्वे अवी॒ अवी॒ द्वे । \newline
19. अवी॒ इत्यवी᳚ । \newline
20. द्वे धे॒नू धे॒नू द्वे द्वे धे॒नू । \newline
21. द्वे इति॒ द्वे । \newline
22. धे॒नू भौ॒मी भौ॒मी धे॒नू धे॒नू भौ॒मी । \newline
23. धे॒नू इति॑ धे॒नू । \newline
24. भौ॒मी दि॒ग्भ्यो दि॒ग्भ्यो भौ॒मी भौ॒मी दि॒ग्भ्यः । \newline
25. भौ॒मी इति॑ भौ॒मी । \newline
26. दि॒ग्भ्यो वड॑बे॒ वड॑बे दि॒ग्भ्यो दि॒ग्भ्यो वड॑बे । \newline
27. दि॒ग्भ्य इति॑ दिक् - भ्यः । \newline
28. वड॑बे॒ द्वे द्वे वड॑बे॒ वड॑बे॒ द्वे । \newline
29. वड॑बे॒ इति॒ वड॑बे । \newline
30. द्वे धे॒नू धे॒नू द्वे द्वे धे॒नू । \newline
31. द्वे इति॒ द्वे । \newline
32. धे॒नू भौ॒मी भौ॒मी धे॒नू धे॒नू भौ॒मी । \newline
33. धे॒नू इति॑ धे॒नू । \newline
34. भौ॒मी वै॑रा॒जी वै॑रा॒जी भौ॒मी भौ॒मी वै॑रा॒जी । \newline
35. भौ॒मी इति॑ भौ॒मी । \newline
36. वै॒रा॒जी पु॑रु॒षी पु॑रु॒षी वै॑रा॒जी वै॑रा॒जी पु॑रु॒षी । \newline
37. वै॒रा॒जी इति॑ वैरा॒जी । \newline
38. पु॒रु॒षी द्वे द्वे पु॑रु॒षी पु॑रु॒षी द्वे । \newline
39. पु॒रु॒षी इति॑ पुरु॒षी । \newline
40. द्वे धे॒नू धे॒नू द्वे द्वे धे॒नू । \newline
41. द्वे इति॒ द्वे । \newline
42. धे॒नू भौ॒मी भौ॒मी धे॒नू धे॒नू भौ॒मी । \newline
43. धे॒नू इति॑ धे॒नू । \newline
44. भौ॒मी वा॒यवे॑ वा॒यवे॑ भौ॒मी भौ॒मी वा॒यवे᳚ । \newline
45. भौ॒मी इति॑ भौ॒मी । \newline
46. वा॒यव॑ आरोहणवा॒हा वा॑रोहणवा॒हौ वा॒यवे॑ वा॒यव॑ आरोहणवा॒हौ । \newline
47. आ॒रो॒ह॒ण॒वा॒हा व॑न॒ड्वाहा॑ वन॒ड्वाहा॑ वारोहणवा॒हा वा॑रोहणवा॒हा व॑न॒ड्वाहौ᳚ । \newline
48. आ॒रो॒ह॒ण॒वा॒हावित्या॑रोहण - वा॒हौ । \newline
49. अ॒न॒ड्वाहौ॑ वारु॒णी वा॑रु॒णी अ॑न॒ड्वाहा॑ वन॒ड्वाहौ॑ वारु॒णी । \newline
50. वा॒रु॒णी कृ॒ष्णे कृ॒ष्णे वा॑रु॒णी वा॑रु॒णी कृ॒ष्णे । \newline
51. वा॒रु॒णी इति॑ वारु॒णी । \newline
52. कृ॒ष्णे व॒शे व॒शे कृ॒ष्णे कृ॒ष्णे व॒शे । \newline
53. कृ॒ष्णे इति॑ कृ॒ष्णे । \newline
54. व॒शे अ॑रा॒ड्या॑ वरा॒ड्यौ॑ व॒शे व॒शे अ॑रा॒ड्यौ᳚ । \newline
55. व॒शे इति॑ व॒शे । \newline
56. अ॒रा॒ड्यौ॑ दि॒व्यौ दि॒व्या व॑रा॒ड्या॑ वरा॒ड्यौ॑ दि॒व्यौ । \newline
57. दि॒व्या वृ॑ष॒भा वृ॑ष॒भौ दि॒व्यौ दि॒व्या वृ॑ष॒भौ । \newline
58. ऋ॒ष॒भौ प॑रिम॒रौ प॑रिम॒रा वृ॑ष॒भा वृ॑ष॒भौ प॑रिम॒रौ । \newline
59. प॒रि॒म॒राविति॑ परि - म॒रौ । \newline

\textbf{Ghana Paata } \newline

1. सोमा॑य स्व॒राज्ञे᳚ स्व॒राज्ञे॒ सोमा॑य॒ सोमा॑य स्व॒राज्ञे॑ ऽनोवा॒हा व॑नोवा॒हौ स्व॒राज्ञे॒ सोमा॑य॒ सोमा॑य स्व॒राज्ञे॑ ऽनोवा॒हौ । \newline
2. स्व॒राज्ञे॑ ऽनोवा॒हा व॑नोवा॒हौ स्व॒राज्ञे᳚ स्व॒राज्ञे॑ ऽनोवा॒हा व॑न॒ड्वाहा॑ वन॒ड्वाहा॑ वनोवा॒हौ स्व॒राज्ञे᳚ स्व॒राज्ञे॑ ऽनोवा॒हा व॑न॒ड्वाहौ᳚ । \newline
3. स्व॒राज्ञ्॒ इति॑ स्व - राज्ञे᳚ । \newline
4. अ॒नो॒वा॒हा व॑न॒ड्वाहा॑ वन॒ड्वाहा॑ वनोवा॒हा व॑नोवा॒हा व॑न॒ड्वाहा॑ विन्द्रा॒ग्निभ्या॑ मिन्द्रा॒ग्निभ्या॑ मन॒ड्वाहा॑ वनोवा॒हा व॑नोवा॒हा व॑न॒ड्वाहा॑ विन्द्रा॒ग्निभ्या᳚म् । \newline
5. अ॒नो॒वा॒हावित्य॑नः - वा॒हौ । \newline
6. अ॒न॒ड्वाहा॑ विन्द्रा॒ग्निभ्या॑ मिन्द्रा॒ग्निभ्या॑ मन॒ड्वाहा॑ वन॒ड्वाहा॑ विन्द्रा॒ग्निभ्या॑ मोजो॒दाभ्या॑ मोजो॒दाभ्या॑ मिन्द्रा॒ग्निभ्या॑ मन॒ड्वाहा॑ वन॒ड्वाहा॑ विन्द्रा॒ग्निभ्या॑ मोजो॒दाभ्या᳚म् । \newline
7. इ॒न्द्रा॒ग्निभ्या॑ मोजो॒दाभ्या॑ मोजो॒दाभ्या॑ मिन्द्रा॒ग्निभ्या॑ मिन्द्रा॒ग्निभ्या॑ मोजो॒दाभ्या॒ मुष्टा॑रा॒ वुष्टा॑रा वोजो॒दाभ्या॑ मिन्द्रा॒ग्निभ्या॑ मिन्द्रा॒ग्निभ्या॑ मोजो॒दाभ्या॒ मुष्टा॑रौ । \newline
8. इ॒न्द्रा॒ग्निभ्या॒मिती᳚न्द्रा॒ग्नि - भ्या॒म् । \newline
9. ओ॒जो॒दाभ्या॒ मुष्टा॑रा॒ वुष्टा॑रा वोजो॒दाभ्या॑ मोजो॒दाभ्या॒ मुष्टा॑रा विन्द्रा॒ग्निभ्या॑ मिन्द्रा॒ग्निभ्या॒ मुष्टा॑रा वोजो॒दाभ्या॑ मोजो॒दाभ्या॒ मुष्टा॑रा विन्द्रा॒ग्निभ्या᳚म् । \newline
10. ओ॒जो॒दाभ्या॒मित्यो॑जः - दाभ्या᳚म् । \newline
11. उष्टा॑रा विन्द्रा॒ग्निभ्या॑ मिन्द्रा॒ग्निभ्या॒ मुष्टा॑रा॒ वुष्टा॑रा विन्द्रा॒ग्निभ्या᳚म् बल॒दाभ्या᳚म् बल॒दाभ्या॑ मिन्द्रा॒ग्निभ्या॒ मुष्टा॑रा॒ वुष्टा॑रा विन्द्रा॒ग्निभ्या᳚म् बल॒दाभ्या᳚म् । \newline
12. इ॒न्द्रा॒ग्निभ्या᳚म् बल॒दाभ्या᳚म् बल॒दाभ्या॑ मिन्द्रा॒ग्निभ्या॑ मिन्द्रा॒ग्निभ्या᳚म् बल॒दाभ्याꣳ॑ सीरवा॒हौ सी॑रवा॒हौ ब॑ल॒दाभ्या॑ मिन्द्रा॒ग्निभ्या॑ मिन्द्रा॒ग्निभ्या᳚म् बल॒दाभ्याꣳ॑ सीरवा॒हौ । \newline
13. इ॒न्द्रा॒ग्निभ्या॒मिती᳚न्द्रा॒ग्नि - भ्या॒म् । \newline
14. ब॒ल॒दाभ्याꣳ॑ सीरवा॒हौ सी॑रवा॒हौ ब॑ल॒दाभ्या᳚म् बल॒दाभ्याꣳ॑ सीरवा॒हा ववी॒ अवी॑ सीरवा॒हौ ब॑ल॒दाभ्या᳚म् बल॒दाभ्याꣳ॑ सीरवा॒हा ववी᳚ । \newline
15. ब॒ल॒दाभ्या॒मिति॑ बल - दाभ्या᳚म् । \newline
16. सी॒र॒वा॒हा ववी॒ अवी॑ सीरवा॒हौ सी॑रवा॒हा ववी॒ द्वे द्वे अवी॑ सीरवा॒हौ सी॑रवा॒हा ववी॒ द्वे । \newline
17. सी॒र॒वा॒हाविति॑ सीर - वा॒हौ । \newline
18. अवी॒ द्वे द्वे अवी॒ अवी॒ द्वे धे॒नू धे॒नू द्वे अवी॒ अवी॒ द्वे धे॒नू । \newline
19. अवी॒ इत्यवी᳚ । \newline
20. द्वे धे॒नू धे॒नू द्वे द्वे धे॒नू भौ॒मी भौ॒मी धे॒नू द्वे द्वे धे॒नू भौ॒मी । \newline
21. द्वे इति॒ द्वे । \newline
22. धे॒नू भौ॒मी भौ॒मी धे॒नू धे॒नू भौ॒मी दि॒ग्भ्यो दि॒ग्भ्यो भौ॒मी धे॒नू धे॒नू भौ॒मी दि॒ग्भ्यः । \newline
23. धे॒नू इति॑ धे॒नू । \newline
24. भौ॒मी दि॒ग्भ्यो दि॒ग्भ्यो भौ॒मी भौ॒मी दि॒ग्भ्यो वड॑बे॒ वड॑बे दि॒ग्भ्यो भौ॒मी भौ॒मी दि॒ग्भ्यो वड॑बे । \newline
25. भौ॒मी इति॑ भौ॒मी । \newline
26. दि॒ग्भ्यो वड॑बे॒ वड॑बे दि॒ग्भ्यो दि॒ग्भ्यो वड॑बे॒ द्वे द्वे वड॑बे दि॒ग्भ्यो दि॒ग्भ्यो वड॑बे॒ द्वे । \newline
27. दि॒ग्भ्य इति॑ दिक् - भ्यः । \newline
28. वड॑बे॒ द्वे द्वे वड॑बे॒ वड॑बे॒ द्वे धे॒नू धे॒नू द्वे वड॑बे॒ वड॑बे॒ द्वे धे॒नू । \newline
29. वड॑बे॒ इति॒ वड॑बे । \newline
30. द्वे धे॒नू धे॒नू द्वे द्वे धे॒नू भौ॒मी भौ॒मी धे॒नू द्वे द्वे धे॒नू भौ॒मी । \newline
31. द्वे इति॒ द्वे । \newline
32. धे॒नू भौ॒मी भौ॒मी धे॒नू धे॒नू भौ॒मी वै॑रा॒जी वै॑रा॒जी भौ॒मी धे॒नू धे॒नू भौ॒मी वै॑रा॒जी । \newline
33. धे॒नू इति॑ धे॒नू । \newline
34. भौ॒मी वै॑रा॒जी वै॑रा॒जी भौ॒मी भौ॒मी वै॑रा॒जी पु॑रु॒षी पु॑रु॒षी वै॑रा॒जी भौ॒मी भौ॒मी वै॑रा॒जी पु॑रु॒षी । \newline
35. भौ॒मी इति॑ भौ॒मी । \newline
36. वै॒रा॒जी पु॑रु॒षी पु॑रु॒षी वै॑रा॒जी वै॑रा॒जी पु॑रु॒षी द्वे द्वे पु॑रु॒षी वै॑रा॒जी वै॑रा॒जी पु॑रु॒षी द्वे । \newline
37. वै॒रा॒जी इति॑ वैरा॒जी । \newline
38. पु॒रु॒षी द्वे द्वे पु॑रु॒षी पु॑रु॒षी द्वे धे॒नू धे॒नू द्वे पु॑रु॒षी पु॑रु॒षी द्वे धे॒नू । \newline
39. पु॒रु॒षी इति॑ पुरु॒षी । \newline
40. द्वे धे॒नू धे॒नू द्वे द्वे धे॒नू भौ॒मी भौ॒मी धे॒नू द्वे द्वे धे॒नू भौ॒मी । \newline
41. द्वे इति॒ द्वे । \newline
42. धे॒नू भौ॒मी भौ॒मी धे॒नू धे॒नू भौ॒मी वा॒यवे॑ वा॒यवे॑ भौ॒मी धे॒नू धे॒नू भौ॒मी वा॒यवे᳚ । \newline
43. धे॒नू इति॑ धे॒नू । \newline
44. भौ॒मी वा॒यवे॑ वा॒यवे॑ भौ॒मी भौ॒मी वा॒यव॑ आरोहणवा॒हा वा॑रोहणवा॒हौ वा॒यवे॑ भौ॒मी भौ॒मी वा॒यव॑ आरोहणवा॒हौ । \newline
45. भौ॒मी इति॑ भौ॒मी । \newline
46. वा॒यव॑ आरोहणवा॒हा वा॑रोहणवा॒हौ वा॒यवे॑ वा॒यव॑ आरोहणवा॒हा व॑न॒ड्वाहा॑ वन॒ड्वाहा॑ वारोहणवा॒हौ वा॒यवे॑ वा॒यव॑ आरोहणवा॒हा व॑न॒ड्वाहौ᳚ । \newline
47. आ॒रो॒ह॒ण॒वा॒हा व॑न॒ड्वाहा॑ वन॒ड्वाहा॑ वारोहणवा॒हा वा॑रोहणवा॒हा व॑न॒ड्वाहौ॑ वारु॒णी वा॑रु॒णी अ॑न॒ड्वाहा॑ वारोहणवा॒हा वा॑रोहणवा॒हा व॑न॒ड्वाहौ॑ वारु॒णी । \newline
48. आ॒रो॒ह॒ण॒वा॒हावित्या॑रोहण - वा॒हौ । \newline
49. अ॒न॒ड्वाहौ॑ वारु॒णी वा॑रु॒णी अ॑न॒ड्वाहा॑ वन॒ड्वाहौ॑ वारु॒णी कृ॒ष्णे कृ॒ष्णे वा॑रु॒णी अ॑न॒ड्वाहा॑ वन॒ड्वाहौ॑ वारु॒णी कृ॒ष्णे । \newline
50. वा॒रु॒णी कृ॒ष्णे कृ॒ष्णे वा॑रु॒णी वा॑रु॒णी कृ॒ष्णे व॒शे व॒शे कृ॒ष्णे वा॑रु॒णी वा॑रु॒णी कृ॒ष्णे व॒शे । \newline
51. वा॒रु॒णी इति॑ वारु॒णी । \newline
52. कृ॒ष्णे व॒शे व॒शे कृ॒ष्णे कृ॒ष्णे व॒शे अ॑रा॒ड्या॑ वरा॒ड्यौ॑ व॒शे कृ॒ष्णे कृ॒ष्णे व॒शे अ॑रा॒ड्यौ᳚ । \newline
53. कृ॒ष्णे इति॑ कृ॒ष्णे । \newline
54. व॒शे अ॑रा॒ड्या॑ वरा॒ड्यौ॑ व॒शे व॒शे अ॑रा॒ड्यौ॑ दि॒व्यौ दि॒व्या व॑रा॒ड्यौ॑ व॒शे व॒शे अ॑रा॒ड्यौ॑ दि॒व्यौ । \newline
55. व॒शे इति॑ व॒शे । \newline
56. अ॒रा॒ड्यौ॑ दि॒व्यौ दि॒व्या व॑रा॒ड्या॑ वरा॒ड्यौ॑ दि॒व्या वृ॑ष॒भा वृ॑ष॒भौ दि॒व्या व॑रा॒ड्या॑ वरा॒ड्यौ॑ दि॒व्या वृ॑ष॒भौ । \newline
57. दि॒व्या वृ॑ष॒भा वृ॑ष॒भौ दि॒व्यौ दि॒व्या वृ॑ष॒भौ प॑रिम॒रौ प॑रिम॒रा वृ॑ष॒भौ दि॒व्यौ दि॒व्या वृ॑ष॒भौ प॑रिम॒रौ । \newline
58. ऋ॒ष॒भौ प॑रिम॒रौ प॑रिम॒रा वृ॑ष॒भा वृ॑ष॒भौ प॑रिम॒रौ । \newline
59. प॒रि॒म॒राविति॑ परि - म॒रौ । \newline
\pagebreak
\markright{ TS 5.6.22.1  \hfill https://www.vedavms.in \hfill}

\section{ TS 5.6.22.1 }

\textbf{TS 5.6.22.1 } \newline
\textbf{Samhita Paata} \newline

एका॑दश प्रा॒तर्ग॒व्याः प॒शव॒ आ ल॑भ्यन्ते छग॒लः क॒ल्माषः॑ किकिदी॒विर्वि॑दी॒गय॒स्ते त्वा॒ष्ट्राः सौ॒रीर्नव॑ श्वे॒ता व॒शा अ॑नूब॒न्ध्या॑ भवन्त्याग्ने॒य ऐ᳚न्द्रा॒ग्न आ᳚श्वि॒नस्ते वि॑शालयू॒प आ ल॑भ्यन्ते ॥ \newline

\textbf{Pada Paata} \newline

एका॑दश । प्रा॒तः । ग॒व्याः । प॒शवः॑ । एति॑ । ल॒भ्य॒न्ते॒ । छ॒ग॒लः । क॒ल्माषः॑ । कि॒कि॒दी॒विः । वि॒दी॒गयः॑ । ते । त्वा॒ष्ट्राः । सौ॒रीः । नव॑ । श्वे॒ताः । व॒शाः । अ॒नू॒ब॒न्ध्या॑ इत्य॑नु - ब॒न्ध्याः᳚ । भ॒व॒न्ति॒ । आ॒ग्ने॒यः । ऐ॒न्द्रा॒ग्न इत्यै᳚न्द्र - अ॒ग्नः । आ॒श्वि॒नः । ते । वि॒शा॒ल॒यू॒प इति॑ विशाल - यू॒पे । एति॑ । ल॒भ्य॒न्ते॒ ॥  \newline


\textbf{Krama Paata} \newline

एका॑दश प्रा॒तः । प्रा॒तर् ग॒व्याः । ग॒व्याः प॒शवः॑ । प॒शव॒ आ । आ ल॑भ्यन्ते । ल॒भ्य॒न्ते॒ छ॒ग॒लः । छ॒ग॒लः क॒ल्माषः॑ । क॒ल्माषः॑ किकिदी॒विः । कि॒कि॒दी॒विर् वि॑दी॒गयः॑ । वि॒दी॒गय॒स्ते । ते त्वा॒ष्ट्राः । त्वा॒ष्ट्राः सौ॒रीः । सौ॒रीर् नव॑ । नव॑ श्वे॒ताः । श्वे॒ता व॒शाः । व॒शा अ॑नूब॒न्द्ध्याः᳚ । अ॒नू॒ब॒न्द्ध्या॑ भवन्ति । अ॒नू॒ब॒न्द्ध्या॑ इत्य॑नु - ब॒न्द्ध्याः᳚ । भ॒व॒न्त्या॒ग्ने॒यः । आ॒ग्ने॒य ऐ᳚न्द्रा॒ग्नः । ऐ॒न्द्रा॒ग्न आ᳚श्वि॒नः । ऐ॒न्द्रा॒ग्न इत्यै᳚न्द्र - अ॒ग्नः । आ॒श्वि॒नस्ते । ते वि॑शालयू॒पे । वि॒शा॒ल॒यू॒प आ । वि॒शा॒ल॒यू॒प इति॑ विशाल - यू॒पे । आ ल॑भ्यन्ते । ल॒भ्य॒न्त॒ इति॑ लभ्यन्ते । \newline

\textbf{Jatai Paata} \newline

1. एका॑दश प्रा॒तः प्रा॒त रेका॑द॒शै का॑दश प्रा॒तः । \newline
2. प्रा॒तर् ग॒व्या ग॒व्याः प्रा॒तः प्रा॒तर् ग॒व्याः । \newline
3. ग॒व्याः प॒शवः॑ प॒शवो॑ ग॒व्या ग॒व्याः प॒शवः॑ । \newline
4. प॒शव॒ आ प॒शवः॑ प॒शव॒ आ । \newline
5. आ ल॑भ्यन्ते लभ्यन्त॒ आ ल॑भ्यन्ते । \newline
6. ल॒भ्य॒न्ते॒ छ॒ग॒ल श्छ॑ग॒लो ल॑भ्यन्ते लभ्यन्ते छग॒लः । \newline
7. छ॒ग॒लः क॒ल्माषः॑ क॒ल्माष॑ श्छग॒ल श्छ॑ग॒लः क॒ल्माषः॑ । \newline
8. क॒ल्माषः॑ किकिदी॒विः कि॑किदी॒विः क॒ल्माषः॑ क॒ल्माषः॑ किकिदी॒विः । \newline
9. कि॒कि॒दी॒विर् वि॑दी॒गयो॑ विदी॒गयः॑ किकिदी॒विः कि॑किदी॒विर् वि॑दी॒गयः॑ । \newline
10. वि॒दी॒गय॒ स्ते ते वि॑दी॒गयो॑ विदी॒गय॒ स्ते । \newline
11. ते त्वा॒ष्ट्रा स्त्वा॒ष्ट्रा स्ते ते त्वा॒ष्ट्राः । \newline
12. त्वा॒ष्ट्राः सौ॒रीः सौ॒री स्त्वा॒ष्ट्रा स्त्वा॒ष्ट्राः सौ॒रीः । \newline
13. सौ॒रीर् नव॒ नव॑ सौ॒रीः सौ॒रीर् नव॑ । \newline
14. नव॑ श्वे॒ताः श्वे॒ता नव॒ नव॑ श्वे॒ताः । \newline
15. श्वे॒ता व॒शा व॒शाः श्वे॒ताः श्वे॒ता व॒शाः । \newline
16. व॒शा अ॑नूब॒न्ध्या॑ अनूब॒न्ध्या॑ व॒शा व॒शा अ॑नूब॒न्ध्याः᳚ । \newline
17. अ॒नू॒ब॒न्ध्या॑ भवन्ति भवन् त्यनूब॒न्ध्या॑ अनूब॒न्ध्या॑ भवन्ति । \newline
18. अ॒नू॒ब॒न्ध्या॑ इत्य॑नु - ब॒न्ध्याः᳚ । \newline
19. भ॒व॒न् त्या॒ग्ने॒य आ᳚ग्ने॒यो भ॑वन्ति भवन् त्याग्ने॒यः । \newline
20. आ॒ग्ने॒य ऐ᳚न्द्रा॒ग्न ऐ᳚न्द्रा॒ग्न आ᳚ग्ने॒य आ᳚ग्ने॒य ऐ᳚न्द्रा॒ग्नः । \newline
21. ऐ॒न्द्रा॒ग्न आ᳚श्वि॒न आ᳚श्वि॒न ऐ᳚न्द्रा॒ग्न ऐ᳚न्द्रा॒ग्न आ᳚श्वि॒नः । \newline
22. ऐ॒न्द्रा॒ग्न इत्यै᳚न्द्र - अ॒ग्नः । \newline
23. आ॒श्वि॒न स्ते त आ᳚श्वि॒न आ᳚श्वि॒न स्ते । \newline
24. ते वि॑शालयू॒पे वि॑शालयू॒पे ते ते वि॑शालयू॒पे । \newline
25. वि॒शा॒ल॒यू॒प आ वि॑शालयू॒पे वि॑शालयू॒प आ । \newline
26. वि॒शा॒ल॒यू॒प इति॑ विशाल - यू॒पे । \newline
27. आ ल॑भ्यन्ते लभ्यन्त॒ आ ल॑भ्यन्ते । \newline
28. ल॒भ्य॒न्त॒ इति॑ लभ्यन्ते । \newline

\textbf{Ghana Paata } \newline

1. एका॑दश प्रा॒तः प्रा॒त रेका॑द॒ शैका॑दश प्रा॒तर् ग॒व्या ग॒व्याः प्रा॒त रेका॑द॒ शैका॑दश प्रा॒तर् ग॒व्याः । \newline
2. प्रा॒तर् ग॒व्या ग॒व्याः प्रा॒तः प्रा॒तर् ग॒व्याः प॒शवः॑ प॒शवो॑ ग॒व्याः प्रा॒तः प्रा॒तर् ग॒व्याः प॒शवः॑ । \newline
3. ग॒व्याः प॒शवः॑ प॒शवो॑ ग॒व्या ग॒व्याः प॒शव॒ आ प॒शवो॑ ग॒व्या ग॒व्याः प॒शव॒ आ । \newline
4. प॒शव॒ आ प॒शवः॑ प॒शव॒ आ ल॑भ्यन्ते लभ्यन्त॒ आ प॒शवः॑ प॒शव॒ आ ल॑भ्यन्ते । \newline
5. आ ल॑भ्यन्ते लभ्यन्त॒ आ ल॑भ्यन्ते छग॒ल श्छ॑ग॒लो ल॑भ्यन्त॒ आ ल॑भ्यन्ते छग॒लः । \newline
6. ल॒भ्य॒न्ते॒ छ॒ग॒ल श्छ॑ग॒लो ल॑भ्यन्ते लभ्यन्ते छग॒लः क॒ल्माषः॑ क॒ल्माष॑ श्छग॒लो ल॑भ्यन्ते लभ्यन्ते छग॒लः क॒ल्माषः॑ । \newline
7. छ॒ग॒लः क॒ल्माषः॑ क॒ल्माष॑ श्छग॒ल श्छ॑ग॒लः क॒ल्माषः॑ किकिदी॒विः कि॑किदी॒विः क॒ल्माष॑ श्छग॒ल श्छ॑ग॒लः क॒ल्माषः॑ किकिदी॒विः । \newline
8. क॒ल्माषः॑ किकिदी॒विः कि॑किदी॒विः क॒ल्माषः॑ क॒ल्माषः॑ किकिदी॒विर् वि॑दी॒गयो॑ विदी॒गयः॑ किकिदी॒विः क॒ल्माषः॑ क॒ल्माषः॑ किकिदी॒विर् वि॑दी॒गयः॑ । \newline
9. कि॒कि॒दी॒विर् वि॑दी॒गयो॑ विदी॒गयः॑ किकिदी॒विः कि॑किदी॒विर् वि॑दी॒गय॒ स्ते ते वि॑दी॒गयः॑ किकिदी॒विः कि॑किदी॒विर् वि॑दी॒गय॒ स्ते । \newline
10. वि॒दी॒गय॒ स्ते ते वि॑दी॒गयो॑ विदी॒गय॒ स्ते त्वा॒ष्ट्रा स्त्वा॒ष्ट्रा स्ते वि॑दी॒गयो॑ विदी॒गय॒ स्ते त्वा॒ष्ट्राः । \newline
11. ते त्वा॒ष्ट्रा स्त्वा॒ष्ट्रा स्ते ते त्वा॒ष्ट्राः सौ॒रीः सौ॒री स्त्वा॒ष्ट्रा स्ते ते त्वा॒ष्ट्राः सौ॒रीः । \newline
12. त्वा॒ष्ट्राः सौ॒रीः सौ॒री स्त्वा॒ष्ट्रा स्त्वा॒ष्ट्राः सौ॒रीर् नव॒ नव॑ सौ॒री स्त्वा॒ष्ट्रा स्त्वा॒ष्ट्राः सौ॒रीर् नव॑ । \newline
13. सौ॒रीर् नव॒ नव॑ सौ॒रीः सौ॒रीर् नव॑ श्वे॒ताः श्वे॒ता नव॑ सौ॒रीः सौ॒रीर् नव॑ श्वे॒ताः । \newline
14. नव॑ श्वे॒ताः श्वे॒ता नव॒ नव॑ श्वे॒ता व॒शा व॒शाः श्वे॒ता नव॒ नव॑ श्वे॒ता व॒शाः । \newline
15. श्वे॒ता व॒शा व॒शाः श्वे॒ताः श्वे॒ता व॒शा अ॑नूब॒न्ध्या॑ अनूब॒न्ध्या॑ व॒शाः श्वे॒ताः श्वे॒ता व॒शा अ॑नूब॒न्ध्याः᳚ । \newline
16. व॒शा अ॑नूब॒न्ध्या॑ अनूब॒न्ध्या॑ व॒शा व॒शा अ॑नूब॒न्ध्या॑ भवन्ति भवन् त्यनूब॒न्ध्या॑ व॒शा व॒शा अ॑नूब॒न्ध्या॑ भवन्ति । \newline
17. अ॒नू॒ब॒न्ध्या॑ भवन्ति भवन् त्यनूब॒न्ध्या॑ अनूब॒न्ध्या॑ भवन् त्याग्ने॒य आ᳚ग्ने॒यो भ॑वन्त्य नूब॒न्ध्या॑ अनूब॒न्ध्या॑ भवन् त्याग्ने॒यः । \newline
18. अ॒नू॒ब॒न्ध्या॑ इत्य॑नु - ब॒न्ध्याः᳚ । \newline
19. भ॒व॒न् त्या॒ग्ने॒य आ᳚ग्ने॒यो भ॑वन्ति भवन् त्याग्ने॒य ऐ᳚न्द्रा॒ग्न ऐ᳚न्द्रा॒ग्न आ᳚ग्ने॒यो भ॑वन्ति भवन् त्याग्ने॒य ऐ᳚न्द्रा॒ग्नः । \newline
20. आ॒ग्ने॒य ऐ᳚न्द्रा॒ग्न ऐ᳚न्द्रा॒ग्न आ᳚ग्ने॒य आ᳚ग्ने॒य ऐ᳚न्द्रा॒ग्न आ᳚श्वि॒न आ᳚श्वि॒न ऐ᳚न्द्रा॒ग्न आ᳚ग्ने॒य आ᳚ग्ने॒य ऐ᳚न्द्रा॒ग्न आ᳚श्वि॒नः । \newline
21. ऐ॒न्द्रा॒ग्न आ᳚श्वि॒न आ᳚श्वि॒न ऐ᳚न्द्रा॒ग्न ऐ᳚न्द्रा॒ग्न आ᳚श्वि॒न स्ते त आ᳚श्वि॒न ऐ᳚न्द्रा॒ग्न ऐ᳚न्द्रा॒ग्न आ᳚श्वि॒न स्ते । \newline
22. ऐ॒न्द्रा॒ग्न इत्यै᳚न्द्र - अ॒ग्नः । \newline
23. आ॒श्वि॒न स्ते त आ᳚श्वि॒न आ᳚श्वि॒न स्ते वि॑शालयू॒पे वि॑शालयू॒पे त आ᳚श्वि॒न आ᳚श्वि॒न स्ते वि॑शालयू॒पे । \newline
24. ते वि॑शालयू॒पे वि॑शालयू॒पे ते ते वि॑शालयू॒प आ वि॑शालयू॒पे ते ते वि॑शालयू॒प आ । \newline
25. वि॒शा॒ल॒यू॒प आ वि॑शालयू॒पे वि॑शालयू॒प आ ल॑भ्यन्ते लभ्यन्त॒ आ वि॑शालयू॒पे वि॑शालयू॒प आ ल॑भ्यन्ते । \newline
26. वि॒शा॒ल॒यू॒प इति॑ विशाल - यू॒पे । \newline
27. आ ल॑भ्यन्ते लभ्यन्त॒ आ ल॑भ्यन्ते । \newline
28. ल॒भ्य॒न्त॒ इति॑ लभ्यन्ते । \newline
\pagebreak
\markright{ TS 5.6.23.1  \hfill https://www.vedavms.in \hfill}

\section{ TS 5.6.23.1 }

\textbf{TS 5.6.23.1 } \newline
\textbf{Samhita Paata} \newline

पि॒शङ्गा॒स्त्रयो॑ वास॒न्ताः सा॒रङ्गा॒स्त्रयो॒ ग्रैष्माः॒ पृष॑न्त॒स्त्रयो॒ वार्.षि॑काः॒ पृश्न॑य॒स्त्रयः शार॒दाः पृ॑श्निस॒क्थास्त्रयो॒ हैम॑न्तिका अवलि॒प्तास्त्रयः॑ शैशि॒राः सं॑ॅवथ्स॒राय॒ निव॑क्षसः ॥स्पॆचिअल् कोर्वै fऒर् अनुवाकम् 12 तॊरोहि॑तः कृ॒ष्णा धू॒म्रल॑लामाः॒ - पृश्निः॑ श्या॒मा अ॑रु॒णल॑लामाः -शितिबा॒हुः शि॒ल्पाः श्येनीः᳚ श्या॒मल॑लामा - उन्न॒तः सि॒द्ध्मा धा॒त्रे पौ॒ष्णाः श्येत॑ललामाः - क॒र्णा ब॒भ्रुल॑लामाः - शु॒ण्ठा गौ॒रल॑लामा॒ - इन्द्रा॑य कृ॒ष्णाल॑लामा॒ - अदि॑त्यै॒ रोहि॑त ललामः -सौ॒म्या मा॒लङ्गा॑ - वारु॒णाः सू॒राः श्येत॑ललामा॒ - दश॑ । \newline

\textbf{Pada Paata} \newline

पि॒शङ्गाः᳚ । त्रयः॑ । वा॒स॒न्ताः । सा॒रङ्गाः᳚ । त्रयः॑ । ग्रैष्माः᳚ । पृष॑न्तः । त्रयः॑ । वार्.षि॑काः । पृश्न॑यः । त्रयः॑ । शा॒र॒दाः । पृ॒श्नि॒स॒क्था इति॑ पृश्नि - स॒क्थाः । त्रयः॑ । हैम॑न्तिकाः । अ॒व॒लि॒प्ता इत्य॑व-लि॒प्ताः । त्रयः॑ । शै॒शि॒राः । सं॒ॅव॒थ्स॒रायेति॑ सं - व॒थ्स॒राय॑ । निव॑क्षस॒ इति॒ नि - व॒क्ष॒सः॒ ॥रोहि॑तः कृ॒ष्णा धू॒म्रल॑लामाः॒ - पृश्निः॑ श्या॒मा अ॑रु॒णल॑लामाः -शितिबा॒हुः शि॒ल्पाः श्येनीः᳚ श्या॒मल॑लामा - उन्न॒तः सि॒द्ध्मा धा॒त्रे पौ॒ष्णाः श्येत॑ललामाः - क॒र्णा ब॒भ्रुल॑लामाः - शु॒ण्ठा गौ॒रल॑लामा॒ - इन्द्रा॑य कृ॒ष्णाल॑लामा॒ - अदि॑त्यै॒ रोहि॑त ललामः -सौ॒म्या मा॒लङ्गा॑ - वारु॒णाः सू॒राः श्येत॑ललामा॒ - दश॑ ।  \newline


\textbf{Krama Paata} \newline

पि॒शङ्गा॒स्त्रयः॑ । त्रयो॑ वास॒न्ताः । वा॒स॒न्ताः सा॒रङ्गाः᳚ । सा॒रङ्गा॒स्त्रयः॑ । त्रयो॒ ग्रैष्माः᳚ । ग्रैष्माः॒ पृष॑न्तः । पृष॑न्त॒स्त्रयः॑ । त्रयो॒ वार्.षि॑काः । वार्.षि॑काः॒ पृश्ञ॑यः । पृश्ञ॑य॒स्त्रयः॑ । त्रयः॑ शार॒दाः । शा॒र॒दाः पृ॑श्ञिस॒क्थाः । पृ॒श्ञि॒स॒क्थास्त्रयः॑ । पृ॒श्ञि॒स॒क्था इति॑ पृश्ञि - स॒क्थाः । त्रयो॒ हैम॑न्तिकाः । हैम॑न्तिका अवलि॒प्ताः । अ॒व॒लि॒प्ता स्त्रयः॑ । अ॒व॒लि॒प्ता इत्य॑व - लि॒प्ताः । त्रयः॑ शैशि॒राः । शै॒शि॒राः स॑म्ॅवथ्स॒राय॑ । स॒म्ॅव॒थ्स॒राय॒ निव॑क्षसः । स॒म्ॅव॒थ्स॒रायेति॑ सम् - व॒थ्स॒राय॑ । निव॑क्षस॒ इति॒ नि - व॒क्ष॒सः॒ । \newline

\textbf{Jatai Paata} \newline

1. पि॒शङ्गा॒ स्त्रय॒ स्त्रयः॑ पि॒शङ्गाः᳚ पि॒शङ्गा॒ स्त्रयः॑ । \newline
2. त्रयो॑ वास॒न्ता वा॑स॒न्ता स्त्रय॒ स्त्रयो॑ वास॒न्ताः । \newline
3. वा॒स॒न्ताः सा॒रङ्गाः᳚ सा॒रङ्गा॑ वास॒न्ता वा॑स॒न्ताः सा॒रङ्गाः᳚ । \newline
4. सा॒रङ्गा॒ स्त्रय॒ स्त्रयः॑ सा॒रङ्गाः᳚ सा॒रङ्गा॒ स्त्रयः॑ । \newline
5. त्रयो॒ ग्रैष्मा॒ ग्रैष्मा॒ स्त्रय॒ स्त्रयो॒ ग्रैष्माः᳚ । \newline
6. ग्रैष्माः॒ पृष॑न्तः॒ पृष॑न्तो॒ ग्रैष्मा॒ ग्रैष्माः॒ पृष॑न्तः । \newline
7. पृष॑न्त॒ स्त्रय॒ स्त्रयः॒ पृष॑न्तः॒ पृष॑न्त॒ स्त्रयः॑ । \newline
8. त्रयो॒ वार्.षि॑का॒ वार्.षि॑का॒ स्त्रय॒ स्त्रयो॒ वार्.षि॑काः । \newline
9. वार्.षि॑काः॒ पृश्ञ॑यः॒ पृश्ञ॑यो॒ वार्.षि॑का॒ वार्.षि॑काः॒ पृश्ञ॑यः । \newline
10. पृश्ञ॑य॒ स्त्रय॒ स्त्रयः॒ पृश्ञ॑यः॒ पृश्ञ॑य॒ स्त्रयः॑ । \newline
11. त्रयः॑ शार॒दाः शा॑र॒दा स्त्रय॒ स्त्रयः॑ शार॒दाः । \newline
12. शा॒र॒दाः पृ॑श्ञिस॒क्थाः पृ॑श्ञिस॒क्थाः शा॑र॒दाः शा॑र॒दाः पृ॑श्ञिस॒क्थाः । \newline
13. पृ॒श्ञि॒स॒क्था स्त्रय॒ स्त्रयः॑ पृश्ञिस॒क्थाः पृ॑श्ञिस॒क्था स्त्रयः॑ । \newline
14. पृ॒श्ञि॒स॒क्था इति॑ पृश्ञि - स॒क्थाः । \newline
15. त्रयो॒ हैम॑न्तिका॒ हैम॑न्तिका॒ स्त्रय॒ स्त्रयो॒ हैम॑न्तिकाः । \newline
16. हैम॑न्तिका अवलि॒प्ता अ॑वलि॒प्ता हैम॑न्तिका॒ हैम॑न्तिका अवलि॒प्ताः । \newline
17. अ॒व॒लि॒प्ता स्त्रय॒ स्त्रयो॑ ऽवलि॒प्ता अ॑वलि॒प्ता स्त्रयः॑ । \newline
18. अ॒व॒लि॒प्ता इत्य॑व - लि॒प्ताः । \newline
19. त्रयः॑ शैशि॒राः शै॑शि॒रा स्त्रय॒ स्त्रयः॑ शैशि॒राः । \newline
20. शै॒शि॒राः सं॑ॅवथ्स॒राय॑ संॅवथ्स॒राय॑ शैशि॒राः शै॑शि॒राः सं॑ॅवथ्स॒राय॑ । \newline
21. सं॒ॅव॒थ्स॒राय॒ निव॑क्षसो॒ निव॑क्षसः संॅवथ्स॒राय॑ संॅवथ्स॒राय॒ निव॑क्षसः । \newline
22. सं॒ॅव॒थ्स॒रायेति॑ सं - व॒थ्स॒राय॑ । \newline
23. निव॑क्षस॒ इति॒ नि - व॒क्ष॒सः॒ । \newline

\textbf{Ghana Paata } \newline

1. पि॒शङ्गा॒ स्त्रय॒ स्त्रयः॑ पि॒शङ्गाः᳚ पि॒शङ्गा॒ स्त्रयो॑ वास॒न्ता वा॑स॒न्ता स्त्रयः॑ पि॒शङ्गाः᳚ पि॒शङ्गा॒ स्त्रयो॑ वास॒न्ताः । \newline
2. त्रयो॑ वास॒न्ता वा॑स॒न्ता स्त्रय॒ स्त्रयो॑ वास॒न्ताः सा॒रङ्गाः᳚ सा॒रङ्गा॑ वास॒न्ता स्त्रय॒ स्त्रयो॑ वास॒न्ताः सा॒रङ्गाः᳚ । \newline
3. वा॒स॒न्ताः सा॒रङ्गाः᳚ सा॒रङ्गा॑ वास॒न्ता वा॑स॒न्ताः सा॒रङ्गा॒ स्त्रय॒ स्त्रयः॑ सा॒रङ्गा॑ वास॒न्ता वा॑स॒न्ताः सा॒रङ्गा॒ स्त्रयः॑ । \newline
4. सा॒रङ्गा॒ स्त्रय॒ स्त्रयः॑ सा॒रङ्गाः᳚ सा॒रङ्गा॒ स्त्रयो॒ ग्रैष्मा॒ ग्रैष्मा॒ स्त्रयः॑ सा॒रङ्गाः᳚ सा॒रङ्गा॒ स्त्रयो॒ ग्रैष्माः᳚ । \newline
5. त्रयो॒ ग्रैष्मा॒ ग्रैष्मा॒ स्त्रय॒ स्त्रयो॒ ग्रैष्माः॒ पृष॑न्तः॒ पृष॑न्तो॒ ग्रैष्मा॒ स्त्रय॒ स्त्रयो॒ ग्रैष्माः॒ पृष॑न्तः । \newline
6. ग्रैष्माः॒ पृष॑न्तः॒ पृष॑न्तो॒ ग्रैष्मा॒ ग्रैष्माः॒ पृष॑न्त॒ स्त्रय॒ स्त्रयः॒ पृष॑न्तो॒ ग्रैष्मा॒ ग्रैष्माः॒ पृष॑न्त॒ स्त्रयः॑ । \newline
7. पृष॑न्त॒ स्त्रय॒ स्त्रयः॒ पृष॑न्तः॒ पृष॑न्त॒ स्त्रयो॒ वार्.षि॑का॒ वार्.षि॑का॒ स्त्रयः॒ पृष॑न्तः॒ पृष॑न्त॒ स्त्रयो॒ वार्.षि॑काः । \newline
8. त्रयो॒ वार्.षि॑का॒ वार्.षि॑का॒ स्त्रय॒ स्त्रयो॒ वार्.षि॑काः॒ पृश्ञ॑यः॒ पृश्ञ॑यो॒ वार्.षि॑का॒ स्त्रय॒ स्त्रयो॒ वार्.षि॑काः॒ पृश्ञ॑यः । \newline
9. वार्.षि॑काः॒ पृश्ञ॑यः॒ पृश्ञ॑यो॒ वार्.षि॑का॒ वार्.षि॑काः॒ पृश्ञ॑य॒ स्त्रय॒ स्त्रयः॒ पृश्ञ॑यो॒ वार्.षि॑का॒ वार्.षि॑काः॒ पृश्ञ॑य॒ स्त्रयः॑ । \newline
10. पृश्ञ॑य॒ स्त्रय॒ स्त्रयः॒ पृश्ञ॑यः॒ पृश्ञ॑य॒ स्त्रयः॑ शार॒दाः शा॑र॒दा स्त्रयः॒ पृश्ञ॑यः॒ पृश्ञ॑य॒ स्त्रयः॑ शार॒दाः । \newline
11. त्रयः॑ शार॒दाः शा॑र॒दा स्त्रय॒ स्त्रयः॑ शार॒दाः पृ॑श्ञिस॒क्थाः पृ॑श्ञिस॒क्थाः शा॑र॒दा स्त्रय॒ स्त्रयः॑ शार॒दाः पृ॑श्ञिस॒क्थाः । \newline
12. शा॒र॒दाः पृ॑श्ञिस॒क्थाः पृ॑श्ञिस॒क्थाः शा॑र॒दाः शा॑र॒दाः पृ॑श्ञिस॒क्था स्त्रय॒ स्त्रयः॑ पृश्ञिस॒क्थाः शा॑र॒दाः शा॑र॒दाः पृ॑श्ञिस॒क्था स्त्रयः॑ । \newline
13. पृ॒श्ञि॒स॒क्था स्त्रय॒ स्त्रयः॑ पृश्ञिस॒क्थाः पृ॑श्ञिस॒क्था स्त्रयो॒ हैम॑न्तिका॒ हैम॑न्तिका॒ स्त्रयः॑ पृश्ञिस॒क्थाः पृ॑श्ञिस॒क्था स्त्रयो॒ हैम॑न्तिकाः । \newline
14. पृ॒श्ञि॒स॒क्था इति॑ पृश्ञि - स॒क्थाः । \newline
15. त्रयो॒ हैम॑न्तिका॒ हैम॑न्तिका॒ स्त्रय॒ स्त्रयो॒ हैम॑न्तिका अवलि॒प्ता अ॑वलि॒प्ता हैम॑न्तिका॒ स्त्रय॒ स्त्रयो॒ हैम॑न्तिका अवलि॒प्ताः । \newline
16. हैम॑न्तिका अवलि॒प्ता अ॑वलि॒प्ता हैम॑न्तिका॒ हैम॑न्तिका अवलि॒प्ता स्त्रय॒ स्त्रयो॑ ऽवलि॒प्ता हैम॑न्तिका॒ हैम॑न्तिका अवलि॒प्ता स्त्रयः॑ । \newline
17. अ॒व॒लि॒प्ता स्त्रय॒ स्त्रयो॑ ऽवलि॒प्ता अ॑वलि॒प्ता स्त्रयः॑ शैशि॒राः शै॑शि॒रा स्त्रयो॑ ऽवलि॒प्ता अ॑वलि॒प्ता स्त्रयः॑ शैशि॒राः । \newline
18. अ॒व॒लि॒प्ता इत्य॑व - लि॒प्ताः । \newline
19. त्रयः॑ शैशि॒राः शै॑शि॒रा स्त्रय॒ स्त्रयः॑ शैशि॒राः सं॑ॅवथ्स॒राय॑ संॅवथ्स॒राय॑ शैशि॒रा स्त्रय॒ स्त्रयः॑ शैशि॒राः सं॑ॅवथ्स॒राय॑ । \newline
20. शै॒शि॒राः सं॑ॅवथ्स॒राय॑ संॅवथ्स॒राय॑ शैशि॒राः शै॑शि॒राः सं॑ॅवथ्स॒राय॒ निव॑क्षसो॒ निव॑क्षसः संॅवथ्स॒राय॑ शैशि॒राः शै॑शि॒राः सं॑ॅवथ्स॒राय॒ निव॑क्षसः । \newline
21. सं॒ॅव॒थ्स॒राय॒ निव॑क्षसो॒ निव॑क्षसः संॅवथ्स॒राय॑ संॅवथ्स॒राय॒ निव॑क्षसः । \newline
22. सं॒ॅव॒थ्स॒रायेति॑ सं - व॒थ्स॒राय॑ । \newline
23. निव॑क्षस॒ इति॒ नि - व॒क्ष॒सः॒ । \newline
\pagebreak


\end{document}