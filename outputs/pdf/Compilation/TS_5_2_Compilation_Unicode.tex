\documentclass[17pt]{extarticle}
\usepackage{babel}
\usepackage{fontspec}
\usepackage{polyglossia}
\usepackage{extsizes}

\usepackage{color}   %May be necessary if you want to color links
\usepackage{hyperref}
\hypersetup{
    colorlinks=true, %set true if you want colored links
    linktoc=all,     %set to all if you want both sections and subsections linked
    linkcolor=black,  %choose some color if you want links to stand out
}

\setmainlanguage{sanskrit}
\setotherlanguages{english} %% or other languages
\setlength{\parindent}{0pt}
\pagestyle{myheadings}
\newfontfamily\devanagarifont[Script=Devanagari]{AdishilaVedic}
\renewcommand{\theHsection}{\thepart.section.\thesection}

\newcommand{\VAR}[1]{}
\newcommand{\BLOCK}[1]{}




\begin{document}
\begin{titlepage}
    \begin{center}
 
\begin{sanskrit}
    { \Large
    कृष्ण यजुर्वेदीय तैत्तिरीय संहिता,पद,जटा,घन पाठः 
    }
    \\
    \vspace{2.5cm}
    \mbox{ \Large
    5.2      पञ्चमकाण्डे द्वितीयः प्रश्नः - चित्युपक्रमाभिधानं   }
\end{sanskrit}
\end{center}

\end{titlepage}
\tableofcontents
\phantomsection
\pagebreak

\markright{ TS 5.2.1.1  \hfill https://www.vedavms.in \hfill}

\section{ TS 5.2.1.1 }

\textbf{TS 5.2.1.1 } \newline
\textbf{Samhita Paata} \newline

विष्णु॑मुखा॒ वै दे॒वा श्छन्दो॑भिरि॒मान् ॅलो॒कान॑नपज॒य्य म॒भ्य॑जय॒न्॒.यद्-वि॑ष्णुक्र॒मान् क्रम॑ते॒ विष्णु॑रे॒व भू॒त्वा यज॑मान॒श्छन्दो॑भिरि॒मान् ॅलो॒कान॑नपज॒य्यम॒भि ज॑यति॒ विष्णोः॒ क्रमो᳚ऽस्य-भिमाति॒हेत्या॑ह गाय॒त्री वै पृ॑थि॒वी त्रैष्ठु॑भम॒न्तरि॑क्षं॒ जाग॑ती॒ द्यौरानु॑ष्टुभी॒र्दिश॒ श्छन्दो॑भिरे॒वेमान् ॅलो॒कान्. य॑था पू॒र्वम॒भि ज॑यति प्र॒जाप॑तिर॒ग्निम॑सृजत॒ सो᳚ऽस्माथ् सृ॒ष्टः - [  ] \newline

\textbf{Pada Paata} \newline

विष्णु॑मुखा॒ इति॒ विष्णु॑ - मु॒खाः॒ । वै । दे॒वाः । छन्दो॑भि॒रिति॒ छन्दः॑ - भिः॒ । इ॒मान् । लो॒कान् । अ॒न॒प॒ज॒य्यमित्य॑नप - ज॒य्यम् । अ॒भीति॑ । अ॒ज॒य॒न्न् । यत् । वि॒ष्णु॒क्र॒मानिति॑ विष्णु-क्र॒मान् । क्रम॑ते । विष्णुः॑ । ए॒व । भू॒त्वा । यज॑मानः । छन्दो॑भि॒रिति॒ छन्दः॑ - भिः॒ । इ॒मान् । लो॒कान् । अ॒न॒प॒ज॒य्यमित्य॑नप-ज॒य्यम् । अ॒भीति॑ । ज॒य॒ति॒ । विष्णोः᳚ । क्रमः॑ । अ॒सि॒ । अ॒भि॒मा॒ति॒हेत्य॑भिमाति-हा । इति॑ । आ॒ह॒ । गा॒य॒त्री । वै । पृ॒थि॒वी । त्रैष्टु॑भम् । अ॒न्तरि॑क्षम् । जाग॑ती । द्यौः । आनु॑ष्टुभी॒रित्यानु॑ - स्तु॒भीः॒ । दिशः॑ । छन्दो॑भि॒रिति॒ छन्दः॑ - भिः॒ । ए॒व । इ॒मान् । लो॒कान् । य॒था॒पू॒र्वमिति॑ यथा - पू॒र्वम् । अ॒भीति॑ । ज॒य॒ति॒ । प्र॒जाप॑ति॒रिति॑ प्र॒जा - प॒तिः॒ । अ॒ग्निम् । अ॒सृ॒ज॒त॒ । सः॒ । अ॒स्मा॒त् । सृ॒ष्टः ।  \newline




\markright{ TS 5.2.1.2  \hfill https://www.vedavms.in \hfill}

\section{ TS 5.2.1.2 }

\textbf{TS 5.2.1.2 } \newline
\textbf{Samhita Paata} \newline

परा॑ङै॒त् तमे॒तया ऽन्वै॒दक्र॑न्द॒दिति॒ तया॒ वै सो᳚ऽग्नेः प्रि॒यं धामाऽवा॑रुन्ध॒ यदे॒ताम॒न्वाहा॒-ग्नेरे॒वैतया᳚ प्रि॒यं धामाऽव॑ रुन्ध ईश्व॒रो वा ए॒ष परा᳚ङ् प्र॒दघो॒ यो वि॑ष्णुक्र॒मान् क्रम॑ते चत॒सृभि॒रा व॑र्तते च॒त्वारि॒ छन्दाꣳ॑सि॒ छन्दाꣳ॑सि॒ खलु॒ वा अ॒ग्नेः प्रि॒या त॒नूः प्रि॒यामे॒वास्य॑ त॒नुव॑म॒भि - [  ] \newline

\textbf{Pada Paata} \newline

पराङ्॑ । ऐ॒त् । तम् । ए॒तया᳚ । अन्विति॑ । ऐ॒त् । अक्र॑न्दत् । इति॑ । तया᳚ । वै । सः । अ॒ग्नेः । प्रि॒यम् । धाम॑ । अवेति॑ । अ॒रु॒न्ध॒ । यत् । ए॒ताम् । अ॒न्वाहेत्य॑नु-आह॑ । अ॒ग्नेः । ए॒व । ए॒तया᳚ । प्रि॒यम् । धाम॑ । अवेति॑ । रु॒न्धे॒ । ई॒श्व॒रः । वै । ए॒षः । पराङ्॑ । प्र॒दघ॒ इति॑ प्र -दघः॑ । यः । वि॒ष्णु॒क्र॒मानिति॑ विष्णु - क्र॒मान् । क्रम॑ते । च॒त॒सृभि॒रिति॑ चत॒सृ - भिः॒ । एति॑ । व॒र्त॒ते॒ । च॒त्वारि॑ । छन्दाꣳ॑सि । छन्दाꣳ॑सि । खलु॑ । वै । अ॒ग्नेः । प्रि॒या । त॒नूः । प्रि॒याम् । ए॒व । अ॒स्य॒ । त॒नुव᳚म् । अ॒भीति॑ ।  \newline




\markright{ TS 5.2.1.3  \hfill https://www.vedavms.in \hfill}

\section{ TS 5.2.1.3 }

\textbf{TS 5.2.1.3 } \newline
\textbf{Samhita Paata} \newline

प॒र्याव॑र्तते दक्षि॒णा प॒र्याव॑र्तते॒ स्वमे॒व वी॒र्य॑मनु॑ प॒र्याव॑र्तते॒ तस्मा॒द्-दक्षि॒णोऽर्द्ध॑ आ॒त्मनो॑ वी॒र्या॑वत्त॒रोऽथो॑ आदि॒त्यस्यै॒वाऽऽ*वृत॒मनु॑ प॒र्याव॑र्तते॒ शुन॒श्शेप॒माजी॑गर्तिं॒ ॅवरु॑णोऽगृह्णा॒थ् स ए॒तां ॅवा॑रु॒णीम॑पश्य॒त् तया॒ वै स आ॒त्मानं॑ ॅवरुणपा॒शाद॑मुञ्च॒द्-वरु॑णो॒ वा ए॒तं गृ॑ह्णाति॒ य उ॒खां प्र॑तिमु॒ञ्चत॒ उदु॑त्त॒मं ॅव॑रुण॒पाश॑-म॒स्मदित्या॑हा॒-ऽऽ*त्मान॑मे॒वैतया॑ - [  ] \newline

\textbf{Pada Paata} \newline

प॒र्याव॑र्तत॒ इति॑ परि - आव॑र्तते । द॒क्षि॒णा । प॒र्याव॑र्तत॒ इति॑ परि - आव॑र्तते । स्वम् । ए॒व । वी॒र्य᳚म् । अन्विति॑ । प॒र्याव॑र्तत॒ इति॑ परि - आव॑र्तते । तस्मा᳚त् । दक्षि॑णः । अद्‌र्धः॑ । आ॒त्मनः॑ । वी॒र्या॑वत्तर॒ इति॑ वी॒र्या॑वत् - त॒रः॒ । अथो॒ इति॑ । आ॒दि॒त्यस्य॑ । ए॒व । आ॒वृत॒मित्या᳚ - वृत᳚म् । अन्विति॑ । प॒र्याव॑र्तत॒ इति॑ परि - आव॑र्तते । शुन॒श्शेप᳚म् । आजी॑गर्तिम् । वरु॑णः । अ॒गृ॒ह्णा॒त् । सः । ए॒ताम् । वा॒रु॒णीम् । अ॒प॒श्य॒त् । तया᳚ । वै । सः । आ॒त्मान᳚म् । व॒रु॒ण॒पा॒शादिति॑ वरुण - पा॒शात् । अ॒मु॒ञ्च॒त् । वरु॑णः । वै । ए॒तम् । गृ॒ह्णा॒ति॒ । यः । उ॒खाम् । प्र॒ति॒मु॒ञ्चत॒ इति॑ प्रति - मु॒ञ्चते᳚ । उदिति॑ । उ॒त्त॒ममित्यु॑त् - त॒मम् । व॒रु॒ण॒ । पाश᳚म् । अ॒स्मत् । इति॑ । आ॒ह॒ । आ॒त्मान᳚म् । ए॒व । ए॒तया᳚ ।  \newline




\markright{ TS 5.2.1.4  \hfill https://www.vedavms.in \hfill}

\section{ TS 5.2.1.4 }

\textbf{TS 5.2.1.4 } \newline
\textbf{Samhita Paata} \newline

वरुणपा॒शान् मु॑ञ्च॒त्या त्वा॑ऽ*हार्.ष॒मित्या॒हा ऽऽ*ह्य॑नꣳ॒॒ हर॑ति ध्रु॒वस्ति॒ष्ठा ऽवि॑चाचलि॒रित्या॑ह॒ प्रति॑ष्ठित्यै॒ विश॑स्त्वा॒ सर्वा॑ वाञ्छ॒न्त्वित्या॑ह वि॒शैवैनꣳ॒॒ सम॑र्द्धयत्य॒स्मिन् रा॒ष्ट्रमधि॑ श्र॒येत्या॑ह रा॒ष्ट्रमे॒वास्मि॑न् ध्रु॒वम॑क॒र्यं का॒मये॑त रा॒ष्ट्रꣳ स्या॒दिति॒ तं मन॑सा ध्यायेद्-रा॒ष्ट्रमे॒व भ॑व॒त्य - [  ] \newline

\textbf{Pada Paata} \newline

व॒रु॒ण॒पा॒शादिति॑ वरुण-पा॒शात् । मु॒ञ्च॒ति॒ । एति॑ । त्वा॒ । अ॒हा॒र्॒.ष॒म् । इति॑ । आ॒ह॒ । एति॑ । हि । ए॒न॒म् । हर॑ति । ध्रु॒वः । ति॒ष्ठ॒ । अवि॑चाचलि॒रित्यवि॑ - चा॒च॒लिः॒ । इति॑ । आ॒ह॒ । प्रति॑ष्ठित्या॒ इति॒ प्रति॑ - स्थि॒त्यै॒ । विशः॑ । त्वा॒ । सर्वाः᳚ । वा॒ञ्छ॒न्तु॒ । इति॑ । आ॒ह॒ । वि॒शा । ए॒व । ए॒न॒म् । समिति॑ । अ॒द्‌र्ध॒य॒ति॒ । अ॒स्मिन्न् । रा॒ष्ट्रम् । अधीति॑ । श्र॒य॒ । इति॑ । आ॒ह॒ । रा॒ष्ट्रम् । ए॒व । अ॒स्मि॒न्न् । ध्रु॒वम् । अ॒कः॒ । यम् । का॒मये॑त । रा॒ष्ट्रम् । स्या॒त् । इति॑ । तम् । मन॑सा । ध्या॒ये॒त् । रा॒ष्ट्रम् । ए॒व । भ॒व॒ति॒ ।  \newline




\markright{ TS 5.2.1.5  \hfill https://www.vedavms.in \hfill}

\section{ TS 5.2.1.5 }

\textbf{TS 5.2.1.5 } \newline
\textbf{Samhita Paata} \newline

-ग्रे॑ बृ॒हन्नु॒षसा॑मू॒र्द्ध्वो अ॑स्था॒दित्या॒हाऽग्र॑मे॒वैनꣳ॑ समा॒नानां᳚ करोति निर्जग्मि॒वान् तम॑स॒ इत्या॑ह॒ तम॑ ए॒वास्मा॒दप॑ हन्ति॒ ज्योति॒षा- ऽऽगा॒दित्या॑ह॒ ज्योति॑रे॒वा-स्मि॑न् दधाति चत॒सृभिः॑ सादयति च॒त्वारि॒ छन्दाꣳ॑सि॒ छन्दो॑भिरे॒वाऽ-ति॑च्छन्दसोत्त॒मया॒ वर्ष्म॒ वा ए॒षा छन्द॑सां॒ ॅयदति॑च्छन्दा॒ वर्ष्मै॒वैनꣳ॑ समा॒नानां᳚ करोति॒ सद्व॑ती - [  ] \newline

\textbf{Pada Paata} \newline

अग्रे᳚ । बृ॒हन्न् । उ॒षसा᳚म् । ऊ॒द्‌र्ध्वः । अ॒स्था॒त् । इति॑ । आ॒ह॒ । अग्र᳚म् । ए॒व । ए॒न॒म् । स॒मा॒नाना᳚म् । क॒रो॒ति॒ । नि॒र्ज॒ग्मि॒वानिति॑ निः-ज॒ग्मि॒वान् । तम॑सः । इति॑ । आ॒ह॒ । तमः॑ । ए॒व । अ॒स्मा॒त् । अपेति॑ । ह॒न्ति॒ । ज्योति॑षा । एति॑ । अ॒गा॒त् । इति॑ । आ॒ह॒ । ज्योतिः॑ । ए॒व । अ॒स्मि॒न्न् । द॒धा॒ति॒ । च॒त॒सृभि॒रिति॑ चत॒सृ - भिः॒ । सा॒द॒य॒ति॒ । च॒त्वारि॑ । छन्दाꣳ॑सि । छन्दो॑भि॒रिति॒ छन्दः॑ - भिः॒ । ए॒व । अति॑च्छन्द॒सेत्यति॑ - छ॒न्द॒सा॒ । उ॒त्त॒मयेत्यु॑त् - त॒मया᳚ । वर्ष्म॑ । वै । ए॒षा । छन्द॑साम् । यत् । अति॑च्छन्दा॒ इत्यति॑ - छ॒न्दाः॒ । वर्ष्म॑ । ए॒व । ए॒न॒म् । स॒मा॒नाना᳚म् । क॒रो॒ति॒ । सद्व॒तीति॒ सत् - व॒ती॒ ।  \newline




\markright{ TS 5.2.1.6  \hfill https://www.vedavms.in \hfill}

\section{ TS 5.2.1.6 }

\textbf{TS 5.2.1.6 } \newline
\textbf{Samhita Paata} \newline

भवति स॒त्त्वमे॒वैनं॑ गमयति वाथ्स॒प्रेणोप॑ तिष्ठत ए॒तेन॒ वै व॑थ्स॒प्रीर्भा॑लन्द॒नो᳚ऽग्नेः प्रि॒यं धामाऽवा॑ऽरुन्धा॒ऽग्नेरे॒वैतेन॑ प्रि॒यं धामाऽव॑ रुन्ध एकाद॒शं भ॑वत्येक॒धैव यज॑माने वी॒र्यं॑ दधाति॒ स्तोमे॑न॒ वै दे॒वा अ॒स्मिन् ॅलो॒क आ᳚र्द्ध्नुव॒न् छन्दो॑भिर॒मुष्मि॒न्थ् स्तोम॑स्येव॒ खलु॒ वा ए॒तद्-रू॒पं ॅयद्-वा᳚थ्स॒प्रं ॅयद्-वा᳚थ्स॒प्रेणो॑प॒तिष्ठ॑त - [  ] \newline

\textbf{Pada Paata} \newline

भ॒व॒ति॒ । स॒त्त्वमिति॑ सत् - त्वम् । ए॒व । ए॒न॒म् । ग॒म॒य॒ति॒ । वा॒थ्स॒प्रेणेति॑ वाथ्स - प्रेण॑ । उपेति॑ । ति॒ष्ठ॒ते॒ । ए॒तेन॑ । वै । व॒थ्स॒प्रीरिति॑ वथ्स - प्रीः । भा॒ल॒न्द॒नः । अ॒ग्नेः । प्रि॒यम् । धाम॑ । अवेति॑ । अ॒रु॒न्ध॒ । अ॒ग्नेः । ए॒व । ए॒तेन॑ । प्रि॒यम् । धाम॑ । अवेति॑ । रु॒न्धे॒ । ए॒का॒द॒शम् । भ॒व॒ति॒ । ए॒क॒धेत्ये॑क - धा । ए॒व । यज॑माने । वी॒र्य᳚म् । द॒धा॒ति॒ । स्तोमे॑न । वै । दे॒वाः । अ॒स्मिन्न् । लो॒के । आ॒द्‌र्ध्नु॒व॒न्न् । छन्दो॑भि॒रिति॒ छन्दः॑ - भिः॒ । अ॒मुष्मिन्न्॑ । स्तोम॑स्य । इ॒व॒ । खलु॑ । वै । ए॒तत् । रू॒पम् । यत् । वा॒थ्स॒प्रमिति॑ वाथ्स-प्रम् । यत् । वा॒थ्स॒प्रेणेति॑ वाथ्स-प्रेण॑ । उ॒प॒तिष्ठ॑त॒ इत्यु॑प-तिष्ठ॑ते ।  \newline




\markright{ TS 5.2.1.7  \hfill https://www.vedavms.in \hfill}

\section{ TS 5.2.1.7 }

\textbf{TS 5.2.1.7 } \newline
\textbf{Samhita Paata} \newline

इ॒ममे॒व तेन॑ लो॒कम॒भि ज॑यति॒ यद्-वि॑ष्णुक्र॒मान् क्रम॑ते॒ऽमुमे॒व तैर्लो॒कम॒भि ज॑यति पूर्वे॒द्युः प्रक्रा॑मत्युत्तरे॒द्युरुप॑ तिष्ठते॒ तस्मा॒द्-योगे॒ऽन्यासां᳚ प्र॒जानां॒ मनः॒ क्षेमे॒ऽन्यासां॒ तस्मा᳚द्-यायाव॒रः क्षे॒म्यस्ये॑शे॒ तस्मा᳚द्-यायाव॒रः क्षे॒म्यम॒द्ध्यव॑स्यति मु॒ष्टी क॑रोति॒ वाचं॑ ॅयच्छति य॒ज्ञ्स्य॒ धृत्यै᳚ ॥ \newline

\textbf{Pada Paata} \newline

इ॒मम् । ए॒व । तेन॑ । लो॒कम् । अ॒भीति॑ । ज॒य॒ति॒ । यत् । वि॒ष्णु॒क्र॒मानिति॑ विष्णु - क्र॒मान् । क्रम॑ते । अ॒मुम् । ए॒व । तैः । लो॒कम् । अ॒भीति॑ । ज॒य॒ति॒ । पू॒र्वे॒द्युः । प्रेति॑ । क्रा॒म॒ति॒ । उ॒त्त॒रे॒द्युः । उपेति॑ । ति॒ष्ठ॒ते॒ । तस्मा᳚त् । योगे᳚ । अ॒न्यासा᳚म् । प्र॒जाना॒मिति॑ प्र - जाना᳚म् । मनः॑ । क्षेमे᳚ । अ॒न्यासा᳚म् । तस्मा᳚त् । या॒या॒व॒रः । क्षे॒म्यस्य॑ । ई॒शे॒ । तस्मा᳚त् । या॒या॒व॒रः । क्षे॒म्यम् । अ॒द्ध्यव॑स्य॒तीत्य॑धि - अव॑स्यति । मु॒ष्टी इति॑ । क॒रो॒ति॒ । वाच᳚म् । य॒च्छ॒ति॒ । य॒ज्ञ्स्य॑ । धृत्यै᳚ ॥  \newline




\markright{ TS 5.2.2.1  \hfill https://www.vedavms.in \hfill}

\section{ TS 5.2.2.1 }

\textbf{TS 5.2.2.1 } \newline
\textbf{Samhita Paata} \newline

अन्न॑प॒तेऽन्न॑स्य नो दे॒हीत्या॑हा॒-ग्निर्वा अन्न॑पतिः॒ स ए॒वास्मा॒ अन्नं॒ प्रय॑च्छत्यनमी॒वस्य॑ शु॒ष्मिण॒ इत्या॑हा-य॒क्ष्मस्येति॒ वावैतदा॑ह॒ प्र प्र॑दा॒तारं॑ तारिष॒ ऊर्जं॑ नो धेहि द्वि॒पदे॒ चतु॑ष्पद॒ इत्या॑हा॒ऽऽ*शिष॑मे॒वैतामा शा᳚स्त॒ उदु॑ त्वा॒ विश्वे॑ दे॒वा इत्या॑ह प्रा॒णा वै विश्वे॑ दे॒वाः - [  ] \newline

\textbf{Pada Paata} \newline

अन्न॑पत॒ इत्यन्न॑ - प॒ते॒ । अन्न॑स्य । नः॒ । दे॒हि॒ । इति॑ । आ॒ह॒ । अ॒ग्निः । वै । अन्न॑पति॒रित्यन्न॑ - प॒तिः॒ । सः । ए॒व । अ॒स्मै॒ । अन्न᳚म् । प्रेति॑ । य॒च्छ॒ति॒ । अ॒न॒मी॒वस्य॑ । शु॒ष्मिणः॑ । इति॑ । आ॒ह॒ । अ॒य॒क्ष्मस्य॑ । इति॑ । वाव । ए॒तत् । आ॒ह॒ । प्रेति॑ । प्र॒दा॒तार॒मिति॑ प्र - दा॒तार᳚म् । ता॒रि॒षः॒ । ऊर्ज᳚म् । नः॒ । धे॒हि॒ । द्वि॒पद॒ इति॑ द्वि - पदे᳚ । चतु॑ष्पद॒ इति॒ चतुः॑ - प॒दे॒ । इति॑ । आ॒ह॒ । आ॒शिष॒मित्या᳚ - शिष᳚म् । ए॒व । ए॒ताम् । एति॑ । शा॒स्ते॒ । उदिति॑ । उ॒ । त्वा॒ । विश्वे᳚ । दे॒वाः । इति॑ । आ॒ह॒ । प्रा॒णा इति॑ प्र-अ॒नाः । वै । विश्वे᳚ । दे॒वाः ।  \newline




\markright{ TS 5.2.2.2  \hfill https://www.vedavms.in \hfill}

\section{ TS 5.2.2.2 }

\textbf{TS 5.2.2.2 } \newline
\textbf{Samhita Paata} \newline

प्रा॒णैरे॒वैन॒मुद्य॑च्छ॒ते ऽग्ने॒ भर॑न्तु॒ चित्ति॑भि॒रित्या॑ह॒ यस्मा॑ ए॒वैनं॑ चि॒त्तायो॒द्यच्छ॑ते॒ तेनै॒वैनꣳ॒॒ सम॑र्द्धयति चत॒सृभि॒रा सा॑दयति च॒त्वारि॒ छन्दाꣳ॑सि॒ छन्दो॑भिरे॒वा-ति॑च्छन्दसोत्त॒मया॒ वर्ष्म॒ वा ए॒षा छन्द॑सां॒ ॅयदति॑च्छन्दा॒ वर्ष्मै॒वैनꣳ॑ समा॒नानां᳚ करोति॒ सद्व॑ती भवति स॒त्त्वमे॒वैनं॑ गमयति॒ प्रेद॑ग्ने॒ ज्योति॑ष्मान् - [  ] \newline

\textbf{Pada Paata} \newline

प्रा॒णैरिति॑ प्र - अ॒नैः । ए॒व । ए॒न॒म् । उदिति॑ । य॒च्छ॒ते॒ । अग्ने᳚ । भर॑न्तु । चित्ति॑भि॒रिति॒ चित्ति॑ - भिः॒ । इति॑ । आ॒ह॒ । यस्मै᳚ । ए॒व । ए॒न॒म् । चि॒त्ताय॑ । उ॒द्यच्छ॑त॒ इत्यु॑त् - यच्छ॑ते । तेन॑ । ए॒व । ए॒न॒म् । समिति॑ । अ॒द्‌र्ध॒य॒ति॒ । च॒त॒सृभि॒रिति॑ चत॒सृ - भिः॒ । एति॑ । सा॒द॒य॒ति॒ । च॒त्वारि॑ । छन्दाꣳ॑सि । छन्दो॑भि॒रिति॒ छन्दः॑ - भिः॒ । ए॒व । अति॑च्छन्द॒सेत्यति॑ - छ॒न्द॒सा॒ । उ॒त्त॒मयेत्यु॑त् - त॒मया᳚ । वर्ष्म॑ । वै । ए॒षा । छन्द॑साम् । यत् । अति॑च्छन्दा॒ इत्यति॑ - छ॒न्दाः॒ । वर्ष्म॑ । ए॒व । ए॒न॒म् । स॒मा॒नाना᳚म् । क॒रो॒ति॒ । सद्व॒तीति॒ सत् - व॒ती॒ । भ॒व॒ति॒ । स॒त्त्वमिति॑ सत्-त्वम् । ए॒व । ए॒न॒म् । ग॒म॒य॒ति॒ । प्रेति॑ । इत् । अ॒ग्ने॒ । ज्योति॑ष्मान् ।  \newline




\markright{ TS 5.2.2.3  \hfill https://www.vedavms.in \hfill}

\section{ TS 5.2.2.3 }

\textbf{TS 5.2.2.3 } \newline
\textbf{Samhita Paata} \newline

या॒हीत्या॑ह॒ ज्योति॑रे॒वास्मि॑न् दधाति त॒नुवा॒ वा ए॒ष हि॑नस्ति॒ यꣳ हि॒नस्ति॒ मा हिꣳ॑सीस्त॒नुवा᳚ प्र॒जा इत्या॑ह प्र॒जाभ्य॑ ए॒वैनꣳ॑ शमयति॒ रक्षाꣳ॑सि॒ वा ए॒तद्-य॒ज्ञ्ꣳ स॑चन्ते॒ यदन॑ उ॒थ्सर्ज॒-त्यक्र॑न्द॒दित्यन्वा॑ह॒ रक्ष॑सा॒मप॑हत्या॒ अन॑सा वह॒न्-त्यप॑चिति-मे॒वास्मि॑न् दधाति॒ तस्मा॑दन॒स्वी च॑ र॒थी चाति॑थीना॒-मप॑चिततमा॒ - [  ] \newline

\textbf{Pada Paata} \newline

या॒हि॒ । इति॑ । आ॒ह॒ । ज्योतिः॑ । ए॒व । अ॒स्मि॒न्न् । द॒धा॒ति॒ । त॒नुवा᳚ । वै । ए॒षः । हि॒न॒स्ति॒ । यम् । हि॒नस्ति॑ । मा । हिꣳ॒॒सीः॒ । त॒नुवा᳚ । प्र॒जा इति॑ प्र - जाः । इति॑ । आ॒ह॒ । प्र॒जाभ्य॒ इति॑ प्र - जाभ्यः॑ । ए॒व । ए॒न॒म् । श॒म॒य॒ति॒ । रक्षाꣳ॑सि । वै । ए॒तत् । य॒ज्ञ्म् । स॒च॒न्ते॒ । यत् । अनः॑ । उ॒थ्सर्ज॒तीत्यु॑त्-सर्ज॑ति । अक्र॑न्दत् । इति॑ । अन्विति॑ । आ॒ह॒ । रक्ष॑साम् । अप॑हत्या॒ इत्यप॑ - ह॒त्यै॒ । अन॑सा । व॒ह॒न्ति॒ । अप॑चिति॒मित्यप॑ - चि॒ति॒म् । ए॒व । अ॒स्मि॒न्न् । द॒धा॒ति॒ । तस्मा᳚त् । अ॒न॒स्वी । च॒ । र॒थी । च॒ । अति॑थीनाम् । अप॑चिततमा॒वित्यप॑चित - त॒मौ॒ ।  \newline




\markright{ TS 5.2.2.4  \hfill https://www.vedavms.in \hfill}

\section{ TS 5.2.2.4 }

\textbf{TS 5.2.2.4 } \newline
\textbf{Samhita Paata} \newline

वप॑चितिमान् भवति॒ य ए॒वं ॅवेद॑ स॒मिधा॒ऽग्निं दु॑वस्य॒तेति॑ घृतानुषि॒क्तामव॑सिते स॒मिध॒मा द॑धाति॒ यथाऽति॑थय॒ आग॑ताय स॒र्पिष्व॑दाति॒थ्यं क्रि॒यते॑ ता॒दृगे॒व तद्-गा॑यत्रि॒या ब्रा᳚ह्म॒णस्य॑ गाय॒त्रो हि ब्रा᳚ह्म॒णस्त्रि॒ष्टुभा॑ राज॒न्य॑स्य॒ त्रैष्टु॑भो॒ हि रा॑ज॒न्यो᳚ ऽफ्सु भस्म॒ प्र वे॑शयत्य॒फ्सुयो॑नि॒र्वा अ॒ग्निः स्वामे॒वैनं॒ ॅयोनिं॑ गमयति ति॒सृभिः॒ प्रवे॑शयति त्रि॒वृद्वा - [  ] \newline

\textbf{Pada Paata} \newline

अप॑चितिमा॒नित्यप॑चिति - मा॒न् । भ॒व॒ति॒ । यः । ए॒वम् । वेद॑ । स॒मिधेति॑ सं - इधा᳚ । अ॒ग्निम् । दु॒व॒स्य॒त॒ । इति॑ । घृ॒ता॒नु॒षि॒क्तामिति॑ घृत - अ॒नु॒षि॒क्ताम् । अव॑सित॒ इत्यव॑ - सि॒ते॒ । स॒मिध॒मिति॑ सं - इध᳚म् । एति॑ । द॒धा॒ति॒ । यथा᳚ । अति॑थये । आग॑ता॒येत्या - ग॒ता॒य॒ । स॒र्पिष्व॑त् । आ॒ति॒थ्यम् । क्रि॒यते᳚ । ता॒दृक् । ए॒व । तत् । गा॒य॒त्रि॒या । ब्रा॒ह्म॒णस्य॑ । गा॒य॒त्रः । हि । ब्रा॒ह्म॒णः । त्रि॒ष्टुभा᳚ । रा॒ज॒न्य॑स्य । त्रैष्टु॑भः । हि । रा॒ज॒न्यः॑ । अ॒फ्स्वित्य॑प्-सु । भस्म॑ । प्रेति॑ । वे॒श॒य॒ति॒ । अ॒फ्सुयो॑नि॒रितिय॒फ्सु - यो॒निः॒ । वै । अ॒ग्निः । स्वाम् । ए॒व । ए॒न॒म् । योनि᳚म् । ग॒म॒य॒ति॒ । ति॒सृभि॒रिति॑ ति॒सृ-भिः॒ । प्रेति॑ । वे॒श॒य॒ति॒ । त्रि॒वृदिति॑ त्रि - वृत् । वै ।  \newline




\markright{ TS 5.2.2.5  \hfill https://www.vedavms.in \hfill}

\section{ TS 5.2.2.5 }

\textbf{TS 5.2.2.5 } \newline
\textbf{Samhita Paata} \newline

अ॒ग्नि-र्यावा॑ने॒वाऽग्निस्तं प्र॑ति॒ष्ठां ग॑मयति॒ परा॒ वा ए॒षो᳚ऽग्निं ॅव॑पति॒ यो᳚ऽफ्सु भस्म॑ प्रवे॒शय॑ति॒ ज्योति॑ष्मतीभ्या॒-मव॑ दधाति॒ ज्योति॑रे॒वाऽस्मि॑न् दधाति॒ द्वाभ्यां॒ प्रति॑ष्ठित्यै॒ परा॒ वा ए॒ष प्र॒जां प॒शून् व॑पति॒ यो᳚ऽफ्सु भस्म॑ प्रवे॒शय॑ति॒ पुन॑रू॒र्जा स॒ह र॒य्येति॒ पुन॑रु॒दैति॑ प्र॒जामे॒व प॒शूना॒त्मन् ध॑त्ते॒ पुन॑स्त्वाऽऽदि॒त्या - [  ] \newline

\textbf{Pada Paata} \newline

अ॒ग्निः । यावान्॑ । ए॒व । अ॒ग्निः । तम् । प्र॒ति॒ष्ठामिति॑ प्रति - स्थाम् । ग॒म॒य॒ति॒ । परेति॑ । वै । ए॒षः । अ॒ग्निम् । व॒प॒ति॒ । यः । अ॒फ्स्वित्य॑प् - सु । भस्म॑ । प्र॒वे॒शय॒तीति॑ प्र - वे॒शय॑ति । ज्योति॑ष्मतीभ्याम् । अवेति॑ । द॒धा॒ति॒ । ज्योतिः॑ । ए॒व । अ॒स्मि॒न्न् । द॒धा॒ति॒ । द्वाभ्या᳚म् । प्रति॑ष्ठित्या॒ इति॒ प्रति॑ - स्थि॒त्यै॒ । परेति॑ । वै । ए॒षः । प्र॒जामिति॑ प्र - जाम् । प॒शून् । व॒प॒ति॒ । यः । अ॒फ्स्वित्य॑प् - सु । भस्म॑ । प्र॒वे॒शय॒तीति॑ प्र - वे॒शय॑ति । पुनः॑ । ऊ॒र्जा । स॒ह । र॒य्या । इति॑ । पुनः॑ । उ॒दैतीत्यु॑त् - ऐति॑ । प्र॒जामिति॑ प्र - जाम् । ए॒व । प॒शून् । आ॒त्मन्न् । ध॒त्ते॒ । पुनः॑ । त्वा॒ । आ॒दि॒त्याः ।  \newline




\markright{ TS 5.2.2.6  \hfill https://www.vedavms.in \hfill}

\section{ TS 5.2.2.6 }

\textbf{TS 5.2.2.6 } \newline
\textbf{Samhita Paata} \newline

रु॒द्रा वस॑वः॒ समि॑न्धता॒-मित्या॑है॒ता वा ए॒तं दे॒वता॒ अग्रे॒ समै᳚न्धत॒ ताभि॑रे॒वैनꣳ॒॒ समि॑न्धे॒ बोधा॒ स बो॒धीत्युप॑ तिष्ठते बो॒धय॑त्ये॒वैनं॒ तस्मा᳚थ् सु॒प्त्वा प्र॒जाः प्रबु॑द्ध्यन्ते यथास्था॒नमुप॑ तिष्ठते॒ तस्मा᳚द्-यथास्था॒नं प॒शवः॒ पुन॒रेत्योप॑ तिष्ठन्ते ॥ \newline

\textbf{Pada Paata} \newline

रु॒द्राः । वस॑वः । समिति॑ । इ॒न्ध॒ता॒म् । इति॑ । आ॒ह॒ । ए॒ताः । वै । ए॒तम् । दे॒वताः᳚ । अग्रे᳚ । समिति॑ । ऐ॒न्ध॒त॒ । ताभिः॑ । ए॒व । ए॒न॒म् । समिति॑ । इ॒न्धे॒ । बोध॑ । सः । बो॒धि॒ । इति॑ । उपेति॑ । ति॒ष्ठ॒ते॒ । बो॒धय॑ति । ए॒व । ए॒न॒म् । तस्मा᳚त् । सु॒प्त्वा । प्र॒जा इति॑ प्र - जाः । प्रेति॑ । बु॒द्ध्य॒न्ते॒ । य॒था॒स्था॒नमिति॑ यथा - स्था॒नम् । उपेति॑ । ति॒ष्ठ॒ते॒ । तस्मा᳚त् । य॒था॒स्था॒नमिति॑ यथा - स्था॒नम् । प॒शवः॑ । पुनः॑ । एत्येत्या᳚ - इत्य॑ । उपेति॑ । ति॒ष्ठ॒न्ते॒ ॥  \newline




\markright{ TS 5.2.3.1  \hfill https://www.vedavms.in \hfill}

\section{ TS 5.2.3.1 }

\textbf{TS 5.2.3.1 } \newline
\textbf{Samhita Paata} \newline

याव॑ती॒ वै पृ॑थि॒वी तस्यै॑ य॒म आधि॑पत्यं॒ परी॑याय॒ यो वै य॒मं दे॑व॒यज॑नम॒स्या अनि॑र्याच्या॒ऽग्निं चि॑नु॒ते य॒मायै॑नꣳ॒॒ स चि॑नु॒तेऽपे॒ते-त्य॒द्ध्यव॑साययति य॒ममे॒व दे॑व॒यज॑नम॒स्यै नि॒र्याच्या॒- ऽऽत्मने॒ऽग्निं चि॑नुत इष्व॒ग्रेण॒ वा अ॒स्या अना॑मृत-मि॒च्छन्तो॒ नावि॑न्द॒न् ते दे॒वा ए॒तद्-यजु॑रपश्य॒न्नपे॒तेति॒ यदे॒तेना᳚-ध्यवसा॒यय॒त्य - [  ] \newline

\textbf{Pada Paata} \newline

याव॑ती । वै । पृ॒थि॒वी । तस्यै᳚ । य॒मः । आधि॑पत्य॒मित्याधि॑ - प॒त्य॒म् । परीति॑ । इ॒या॒य॒ । यः । वै । य॒मम् । दे॒व॒यज॑न॒मिति॑ देव - यज॑नम् । अ॒स्याः । अनि॑र्या॒च्येत्यनिः॑ - या॒च्य॒ । अ॒ग्निम् । चि॒नु॒ते । य॒माय॑ । ए॒न॒म् । सः । चि॒नु॒ते॒ । अपेति॑ । इ॒त॒ । इति॑ । अ॒द्ध्यव॑सायय॒तीत्य॑धि - अव॑साययति । य॒मम् । ए॒व । दे॒व॒यज॑न॒मिति॑ देव-यज॑नम् । अ॒स्यै । नि॒र्याच्येति॑ निः - याच्य॑ । आ॒त्मने᳚ । अ॒ग्निम् । चि॒नु॒ते॒ । इ॒ष्व॒ग्रेणेती॑षु - अ॒ग्रेण॑ । वै । अ॒स्याः । अना॑मृत॒मित्यना᳚ - मृ॒त॒म् । इ॒च्छन्तः॑ । न । अ॒वि॒न्द॒न्न् । ते । दे॒वाः । ए॒तत् । यजुः॑ । अ॒प॒श्य॒न्न् । अपेति॑ । इ॒त॒ । इति॑ । यत् । ए॒तेन॑ । अ॒द्ध्य॒व॒सा॒यय॒तीत्य॑धि - अ॒व॒सा॒यय॑ति ।  \newline




\markright{ TS 5.2.3.2  \hfill https://www.vedavms.in \hfill}

\section{ TS 5.2.3.2 }

\textbf{TS 5.2.3.2 } \newline
\textbf{Samhita Paata} \newline

-ना॑मृत ए॒वाग्निं चि॑नुत॒ उद्ध॑न्ति॒ यदे॒वास्या॑ अमे॒द्ध्यं तदप॑ हन्त्य॒पोऽवो᳚क्षति॒ शान्त्यै॒ सिक॑ता॒ नि व॑पत्ये॒तद्वा अ॒ग्नेर्वै᳚श्वान॒रस्य॑ रू॒पꣳ रू॒पेणै॒व वै᳚श्वान॒रमव॑ रुन्ध॒ ऊषा॒न् निव॑पति॒ पुष्टि॒र्वा ए॒षा प्र॒जन॑नं॒ ॅयदूषाः॒ पुष्ट्या॑मे॒व प्र॒जन॑ने॒ऽग्निं चि॑नु॒तेऽथो॑ स॒ज्ञांन॑ ए॒व स॒ज्ञांनꣳ॒॒ ह्ये॑तत् - [  ] \newline

\textbf{Pada Paata} \newline

अना॑मृत॒ इत्यना᳚ - मृ॒ते॒ । ए॒व । अ॒ग्निम् । चि॒नु॒ते॒ । उदिति॑ । ह॒न्ति॒ । यत् । ए॒व । अ॒स्याः॒ । अ॒मे॒द्ध्यम् । तत् । अपेति॑ । ह॒न्ति॒ । अ॒पः । अवेति॑ । उ॒क्ष॒ति॒ । शान्त्यै᳚ । सिक॑ताः । नीति॑ । व॒प॒ति॒ । ए॒तत् । वै । अ॒ग्नेः । वै॒श्वा॒न॒रस्य॑ । रू॒पम् । रू॒पेण॑ । ए॒व । वै॒श्वा॒न॒रम् । अवेति॑ । रु॒न्धे॒ । ऊषान्॑ । नीति॑ । व॒प॒ति॒ । पुष्टिः॑ । वै । ए॒षा । प्र॒जन॑न॒मिति॑ प्र - जन॑नम् । यत् । ऊषाः᳚ । पुष्ट्या᳚म् । ए॒व । प्र॒जन॑न॒ इति॑ प्र - जन॑ने । अ॒ग्निम् । चि॒नु॒ते॒ । अथो॒ इति॑ । स॒ज्ञांन॒ इति॑ सं - ज्ञाने᳚ । ए॒व । स॒ज्ञांन॒मिति॑ सं-ज्ञान᳚म् । हि । ए॒तत् ।  \newline




\markright{ TS 5.2.3.3  \hfill https://www.vedavms.in \hfill}

\section{ TS 5.2.3.3 }

\textbf{TS 5.2.3.3 } \newline
\textbf{Samhita Paata} \newline

प॑शू॒नां ॅयदूषा॒ द्यावा॑पृथि॒वी स॒हाऽऽस्तां॒ ते वि॑य॒ती अ॑ब्रूता॒मस्त्वे॒व नौ॑ स॒ह य॒ज्ञिय॒मिति॒ यद॒मुष्या॑ य॒ज्ञिय॒मासी॒त् तद॒स्याम॑दधा॒त् त ऊषा॑ अभव॒न्॒ यद॒स्या य॒ज्ञिय॒मासी॒त् तद॒मुष्या॑मदधा॒त् तद॒दश्च॒न्द्रम॑सि कृ॒ष्णमूषा᳚न् नि॒वप॑न्न॒दो ध्या॑ये॒द् द्यावा॑पृथि॒व्योरे॒व य॒ज्ञिये॒ऽग्निं चि॑नुते॒ ऽयꣳ सो अ॒ग्निरिति॑ वि॒श्वामि॑त्रस्य - [  ] \newline

\textbf{Pada Paata} \newline

प॒शू॒नाम् । यत् । ऊषाः᳚ । द्यावा॑पृथि॒वी इति॒ द्यावा᳚ - पृ॒थि॒वी । स॒ह । आ॒स्ता॒म् । ते इति॑ । वि॒य॒ती इति॑ वि - य॒ती । अ॒ब्रू॒ता॒म् । अस्तु॑ । ए॒व । नौ॒ । स॒ह । य॒ज्ञिय᳚म् । इति॑ । यत् । अ॒मुष्याः᳚ । य॒ज्ञिय᳚म् । आसी᳚त् । तत् । अ॒स्याम् । अ॒द॒धा॒त् । ते । ऊषाः᳚ । अ॒भ॒व॒न्न् । यत् । अ॒स्याः । य॒ज्ञिय᳚म् । आसी᳚त् । तत् । अ॒मुष्या᳚म् । अ॒द॒धा॒त् । तत् । अ॒दः । च॒न्द्रम॑सि । कृ॒ष्णम् । ऊषान्॑ । नि॒वप॒न्निति॑ नि-वपन्न्॑ । अ॒दः । ध्या॒ये॒त् । द्यावा॑पृथि॒व्योरिति॒ द्यावा᳚ - पृ॒थि॒व्योः । ए॒व । य॒ज्ञिये᳚ । अ॒ग्निम् । चि॒नु॒ते॒ । अ॒यम् । सः । अ॒ग्निः । इति॑ । वि॒श्वामि॑त्र॒स्येति॑ वि॒श्व - मि॒त्र॒स्य॒ ।  \newline




\markright{ TS 5.2.3.4  \hfill https://www.vedavms.in \hfill}

\section{ TS 5.2.3.4 }

\textbf{TS 5.2.3.4 } \newline
\textbf{Samhita Paata} \newline

सू॒क्तं भ॑वत्ये॒तेन॒ वै वि॒श्वामि॑त्रो॒ऽग्नेः प्रि॒यं धामा ऽवा॑रुन्धा॒ग्नेरे॒वैतेन॑ प्रि॒यं धामाव॑ रुन्धे॒ छन्दो॑भि॒र्वै दे॒वाः सु॑व॒र्गं ॅलो॒कमा॑य॒न् चत॑स्रः॒ प्राची॒रुप॑ दधाति च॒त्वारि॒ छन्दाꣳ॑सि॒ छन्दो॑भिरे॒व तद्-यज॑मानः सुव॒र्गं ॅलो॒कमे॑ति॒ तेषाꣳ॑ सुव॒र्गं ॅलो॒कं ॅय॒तां दिशः॒ सम॑व्लीयन्त॒ ते द्वे पु॒रस्ता᳚थ् स॒मीची॒ उपा॑दधत॒ द्वे - [  ] \newline

\textbf{Pada Paata} \newline

सू॒क्तमिति॑ सु - उ॒क्तम् । भ॒व॒ति॒ । ए॒तेन॑ । वै । वि॒श्वामि॑त्र॒ इति॑ वि॒श्व - मि॒त्रः॒ । अ॒ग्नेः । प्रि॒यम् । धाम॑ । अवेति॑ । अ॒रु॒न्ध॒ । अ॒ग्नेः । ए॒व । ए॒तेन॑ । प्रि॒यम् । धाम॑ । अवेति॑ । रु॒न्धे॒ । छन्दो॑भि॒रिति॒ छन्दः॑ - भिः॒ । वै । दे॒वाः । सु॒व॒र्गमिति॑ सुवः - गम् । लो॒कम् । आ॒य॒न्न् । चत॑स्रः । प्राचीः᳚ । उपेति॑ । द॒धा॒ति॒ । च॒त्वारि॑ । छन्दाꣳ॑सि । छन्दो॑भि॒रिति॒ छन्दः॑ - भिः॒ । ए॒व । तत् । यज॑मानः । सु॒व॒र्गमिति॑ सुवः - गम् । लो॒कम् । ए॒ति॒ । तेषा᳚म् । सु॒व॒र्गमिति॑ सुवः - गम् । लो॒कम् । य॒ताम् । दिशः॑ । समिति॑ । अ॒व्ली॒य॒न्त॒ । ते । द्वे इति॑ । पु॒रस्ता᳚त् । स॒मीची॒ इति॑ । उपेति॑ । अ॒द॒ध॒त॒ । द्वे इति॑ ।  \newline




\markright{ TS 5.2.3.5  \hfill https://www.vedavms.in \hfill}

\section{ TS 5.2.3.5 }

\textbf{TS 5.2.3.5 } \newline
\textbf{Samhita Paata} \newline

प॒श्चाथ् स॒मीची॒ ताभि॒र्वै तेदिशो॑ऽदृꣳह॒न्॒ यद्द्वे पु॒रस्ता᳚थ् स॒मीची॑ उप॒दधा॑ति॒ द्वे प॒॒श्चाथ् स॒मीची॑ दि॒शां ॅविधृ॑त्या॒ अथो॑ प॒शवो॒ वै छन्दाꣳ॑सि प॒शूने॒वास्मै॑ स॒मीचो॑ दधात्य॒ष्टावुप॑ दधात्य॒ष्टाक्ष॑रा गाय॒त्री गा॑य॒त्रो᳚ऽग्नि-र्यावा॑ने॒वाग्निस्तं चि॑नुते॒ऽष्टावुप॑ दधात्य॒ष्टाक्ष॑रा गाय॒त्री गा॑य॒त्री सु॑व॒र्गं ॅलो॒कमञ्ज॑सा वेद सुव॒र्गस्य॑ लो॒कस्य॒ - [  ] \newline

\textbf{Pada Paata} \newline

प॒श्चात् । स॒मीची॒ इति॑ । ताभिः॑ । वै । ते । दिशः॑ । अ॒दृꣳ॒॒ह॒न्न् । यत् । द्वे इति॑ । पु॒रस्ता᳚त् । स॒मीची॒ इति॑ । उ॒प॒दधा॒तीत्यु॑प - दधा॑ति । द्वे इति॑ । प॒श्चात् । स॒मीची॒ इति॑ । दि॒शाम् । विधृ॑त्या॒ इति॒ वि-धृ॒त्यै॒ । अथो॒ इति॑ । प॒शवः॑ । वै । छन्दाꣳ॑सि । प॒शून् । ए॒व । अ॒स्मै॒ । स॒मीचः॑ । द॒धा॒ति॒ । अ॒ष्टौ । उपेति॑ । द॒धा॒ति॒ । अ॒ष्टाक्ष॒रेत्य॒ष्टा-अ॒क्ष॒रा॒ । गा॒य॒त्री । गा॒य॒त्रः । अ॒ग्निः । यावान्॑ । ए॒व । अ॒ग्निः । तम् । चि॒नु॒ते॒ । अ॒ष्टौ । उपेति॑ । द॒धा॒ति॒ । अ॒ष्टाक्ष॒रेत्य॒ष्टा-अ॒क्ष॒रा॒ । गा॒य॒त्री । गा॒य॒त्री । सु॒व॒र्गमिति॑ सुवः - गम् । लो॒कम् । अञ्ज॑सा । वे॒द॒ । सु॒व॒र्गस्येति॑ सुवः - गस्य॑ । लो॒कस्य॑ ।  \newline




\markright{ TS 5.2.3.6  \hfill https://www.vedavms.in \hfill}

\section{ TS 5.2.3.6 }

\textbf{TS 5.2.3.6 } \newline
\textbf{Samhita Paata} \newline

प्रज्ञा᳚त्यै॒ त्रयो॑दश लोकंपृ॒णा उप॑ दधा॒त्येक॑विꣳशतिः॒ संप॑द्यन्ते प्रति॒ष्ठा वा ए॑कविꣳ॒॒शः प्र॑ति॒ष्ठा गार्.ह॑पत्य एकविꣳ॒॒शस्यै॒व प्र॑ति॒ष्ठां गार्.ह॑पत्य॒मनु॒ प्रति॑ तिष्ठति॒ प्रत्य॒ग्निं चि॑क्या॒नस्ति॑ष्ठति॒ य ए॒वं ॅवेद॒ पञ्च॑चितीकं चिन्वीत प्रथ॒मं चि॑न्वा॒नः पाङ्क्तो॑ य॒ज्ञ्ः पाङ्क्ताः᳚ प॒शवो॑ य॒ज्ञ्मे॒व प॒शूनव॑ रुन्धे॒ त्रिचि॑तीकं चिन्वीत द्वि॒तीयं॑ चिन्वा॒नस्त्रय॑ इ॒मे लो॒का ए॒ष्वे॑व लो॒केषु॒ - [  ] \newline

\textbf{Pada Paata} \newline

प्रज्ञा᳚त्या॒ इति॒ प्र - ज्ञा॒त्यै॒ । त्रयो॑द॒शेति॒ त्रयः॑ - द॒श॒ । लो॒क॒पृं॒णा इति॑ लोकं - पृ॒णाः । उपेति॑ । द॒धा॒ति॒ । एक॑विꣳशति॒रित्येक॑ - विꣳ॒॒श॒तिः॒ । समिति॑ । प॒द्य॒न्ते॒ । प्र॒ति॒ष्ठेति॑ प्रति - स्था । वै । ए॒क॒विꣳ॒॒श इत्ये॑क - विꣳ॒॒शः । प्र॒ति॒ष्ठेति॑ प्रति - स्था । गार्.ह॑पत्य॒ इति॒ गार्.ह॑ - प॒त्यः॒ । ए॒क॒विꣳ॒॒शस्येत्ये॑क - विꣳ॒॒शस्य॑ । ए॒व । प्र॒ति॒ष्ठामिति॑ प्रति-स्थाम् । गार्.ह॑पत्य॒मिति॒ गार्.ह॑ - प॒त्य॒म् । अनु॑ । प्रतीति॑ । ति॒ष्ठ॒ति॒ । प्रतीति॑ । अ॒ग्निम् । चि॒क्या॒नः । ति॒ष्ठ॒ति॒ । यः । ए॒वम् । वेद॑ । पञ्च॑चितीक॒मिति॒ पञ्च॑-चि॒ती॒क॒म् । चि॒न्वी॒त॒ । प्र॒थ॒मम् । चि॒न्वा॒नः । पाङ्क्तः॑ । य॒ज्ञ्ः । पाङ्क्ताः᳚ । प॒शवः॑ । य॒ज्ञ्म् । ए॒व । प॒शून् । अवेति॑ । रु॒न्धे॒ । त्रिचि॑तीक॒मिति॒ त्रि - चि॒ती॒क॒म् । चि॒न्वी॒त॒ । द्वि॒तीय᳚म् । चि॒न्वा॒नः । त्रयः॑ । इ॒मे । लो॒काः । ए॒षु । ए॒व । लो॒केषु॑ ।  \newline




\markright{ TS 5.2.3.7  \hfill https://www.vedavms.in \hfill}

\section{ TS 5.2.3.7 }

\textbf{TS 5.2.3.7 } \newline
\textbf{Samhita Paata} \newline

प्रति॑तिष्ठ॒-त्येक॑चितीकं चिन्वीत तृ॒तीयं॑ चिन्वा॒न ए॑क॒धा वै सु॑व॒र्गो लो॒क ए॑क॒वृतै॒व सु॑वर्गं ॅलो॒कमे॑ति॒ पुरी॑षेणा॒भ्यू॑हति॒ तस्मा᳚न्माꣳ॒॒ सेनास्थि॑ छ॒न्नं न दु॒श्चर्मा॑ भवति॒ य ए॒वं ॅवेद॒ पञ्च॒ चित॑यो भवन्ति प॒ञ्चभिः॒ पुरी॑षैर॒भ्यू॑हति॒ दश॒ संप॑द्यन्ते॒ दशा᳚क्षरा वि॒राडन्नं॑ ॅवि॒राड् वि॒राज्ये॒वाऽन्नाद्ये॒ प्रति॑तिष्ठति ॥ \newline

\textbf{Pada Paata} \newline

प्रतीति॑ । ति॒ष्ठ॒ति॒ । एक॑चितीक॒मित्येक॑ - चि॒ती॒क॒म् । चि॒न्वी॒त॒ । तृ॒तीय᳚म् । चि॒न्वा॒नः । ए॒क॒धेत्ये॑क - धा । वै । सु॒व॒र्ग इति॑ सुवः - गः । लो॒कः । ए॒क॒वृतेत्ये॑क - वृता᳚ । ए॒व । सु॒व॒र्गमिति॑ सुवः - गम् । लो॒कम् । ए॒ति॒ । पुरी॑षेण । अ॒भीति॑ । ऊ॒ह॒ति॒ । तस्मा᳚त् । माꣳ॒॒सेन॑ । अस्थि॑ । छ॒न्नम् । न । दु॒श्चर्मेति॑ दुः - चर्मा᳚ । भ॒व॒ति॒ । यः । ए॒वम् । वेद॑ । पञ्च॑ । चित॑यः । भ॒व॒न्ति॒ । प॒ञ्चभि॒रिति॑ प॒ञ्च - भिः॒ । पुरी॑षैः । अ॒भीति॑ । ऊ॒ह॒ति॒ । दश॑ । समिति॑ । प॒द्य॒न्ते॒ । दशा᳚क्ष॒रेति॒ दश॑ - अ॒क्ष॒रा॒ । वि॒राडिति॑ वि-राट् । अन्न᳚म् । वि॒राडिति॑ वि - राट् । वि॒राजीति॑ वि - राजि॑ । ए॒व । अ॒न्नाद्य॒ इत्य॑न्न - अद्ये᳚ । प्रतीति॑ । ति॒ष्ठ॒ति॒ ॥  \newline




\markright{ TS 5.2.4.1  \hfill https://www.vedavms.in \hfill}

\section{ TS 5.2.4.1 }

\textbf{TS 5.2.4.1 } \newline
\textbf{Samhita Paata} \newline

वि वा ए॒तौ द्वि॑षाते॒ यश्च॑ पु॒राऽग्निर्यश्चो॒खायाꣳ॒॒ समि॑त॒मिति॑ चत॒सृभिः॒ सं निव॑पति च॒त्वारि॒ छन्दाꣳ॑सि॒ छन्दाꣳ॑सि॒ खलु॒ वा अ॒ग्नेः प्रि॒या त॒नूः प्रि॒ययै॒वैनौ॑ त॒नुवा॒ सꣳ शा᳚स्ति॒ समि॑त॒मित्या॑ह॒ तस्मा॒द्ब्रह्म॑णा क्ष॒त्रꣳ समे॑ति॒ यथ्सं॒ न्युप्य॑ वि॒हर॑ति॒ तस्मा॒द् ब्रह्म॑णा क्ष॒त्रं ॅव्ये᳚त्यृ॒तुभि॒ - [  ] \newline

\textbf{Pada Paata} \newline

वीति॑ । वै । ए॒तौ । द्वि॒षा॒ते॒ इति॑ । यः । च॒ । पु॒रा । अ॒ग्निः । यः । च॒ । उ॒खाया᳚म् । समिति॑ । इ॒त॒म् । इति॑ । च॒त॒सृभि॒रिति॑ चत॒सृ-भिः॒ । सम् । नीति॑ । व॒प॒ति॒ । च॒त्वारि॑ । छन्दाꣳ॑सि । छन्दाꣳ॑सि । खलु॑ । वै । अ॒ग्नेः । प्रि॒या । त॒नूः । प्रि॒यया᳚ । ए॒व । ए॒नौ॒ । त॒नुवा᳚ । समिति॑ । शा॒स्ति॒ । समिति॑ । इ॒त॒म् । इति॑ । आ॒ह॒ । तस्मा᳚त् । ब्रह्म॑णा । क्ष॒त्रम् । समिति॑ । ए॒ति॒ । यत् । स॒न्युंप्येति॑ सं - न्युप्य॑ । वि॒हर॒तीति॑ वि - हर॑ति । तस्मा᳚त् । ब्रह्म॑णा । क्ष॒त्रम् । वीति॑ । ए॒ति॒ । ऋ॒तुभि॒रित्यृ॒तु - भिः॒ ।  \newline




\markright{ TS 5.2.4.2  \hfill https://www.vedavms.in \hfill}

\section{ TS 5.2.4.2 }

\textbf{TS 5.2.4.2 } \newline
\textbf{Samhita Paata} \newline

-र्वा ए॒तं दी᳚क्षयन्ति॒ स ऋ॒तुभि॑रे॒व वि॒मुच्यो॑ मा॒तेव॑ पु॒त्रं पृ॑थि॒वी पु॑री॒ष्य॑मित्या॑ह॒-र्तुभि॑रे॒वैनं॑ दीक्षयि॒त्वर्तुभि॒र्वि मु॑ञ्चति वैश्वान॒र्या शि॒क्य॑मा द॑त्ते स्व॒दय॑त्ये॒वैन॑-न्नैर्.ऋ॒तीः कृ॒ष्णा-स्ति॒स्र-स्तुष॑पक्वा भवन्ति॒ निर्.ऋ॑त्यै॒ वा ए॒तद्-भा॑ग॒धेयं॒ ॅयत् तुषा॒ निर्.ऋ॑त्यै रू॒पं कृ॒ष्णꣳ रू॒पेणै॒व निर्.ऋ॑तिं नि॒रव॑दयत इ॒मां दिशं॑ ॅयन्त्ये॒षा - [  ] \newline

\textbf{Pada Paata} \newline

वै । ए॒तम् । दी॒क्ष॒य॒न्ति॒ । सः । ऋ॒तुभि॒रित्यृ॒तु - भिः॒ । ए॒व । वि॒मुच्य॒ इति॑ वि - मुच्यः॑ । मा॒ता । इ॒व॒ । पु॒त्रम् । पृ॒थि॒वी । पु॒री॒ष्य᳚म् । इति॑ । आ॒ह॒ । ऋ॒तुभि॒रित्यृ॒तु - भिः॒ । ए॒व । ए॒न॒म् । दी॒क्ष॒यि॒त्वा । ऋ॒तुभि॒रित्यृ॒तु-भिः॒ । वीति॑ । मु॒ञ्च॒ति॒ । वै॒श्वा॒न॒र्या । शि॒क्य᳚म् । एति॑ । द॒त्ते॒ । स्व॒दय॑ति । ए॒व । ए॒न॒त्॒ । नै॒र्॒.ऋ॒तीरिति॑ नैः-ऋ॒तीः । कृ॒ष्णाः । ति॒स्रः । तुष॑पक्वा॒ इति॒ तुष॑ - प॒क्वाः॒ । भ॒व॒न्ति॒ । निर्.ऋ॑त्या॒ इति॒ निः - ऋ॒त्यै॒ । वै । ए॒तत् । भा॒ग॒धेय॒मिति॑ भाग - धेय᳚म् । यत् । तुषाः᳚ । निर्.ऋ॑त्या॒ इति॒ निः-ऋ॒त्यै॒ । रू॒पम् । कृ॒ष्णम् । रू॒पेण॑ । ए॒व । निर्.ऋ॑ति॒मिति॒ निः - ऋ॒ति॒म् । नि॒रव॑दयत॒ इति॑ निः - अव॑दयते । इ॒माम् । दिश᳚म् । य॒न्ति॒ । ए॒षा ।  \newline




\markright{ TS 5.2.4.3  \hfill https://www.vedavms.in \hfill}

\section{ TS 5.2.4.3 }

\textbf{TS 5.2.4.3 } \newline
\textbf{Samhita Paata} \newline

वै निर्.ऋ॑त्यै॒ दिक् स्वाया॑मे॒व दि॒शि निर्.ऋ॑तिं नि॒रव॑दयते॒ स्वकृ॑त॒ इरि॑ण॒ उप॑ दधाति प्रद॒रे वै॒तद्वै निर्.ऋ॑त्या आ॒यत॑नꣳ॒॒ स्व ए॒वाऽऽ*यत॑ने॒ निर्.ऋ॑तिं नि॒रव॑दयते शि॒क्य॑म॒भ्युप॑ दधाति नैर्.ऋ॒तो वै पाशः॑ सा॒क्षादे॒वैनं॑ निर्.ऋतिपा॒शान्-मु॑ञ्चति ति॒स्र उप॑ दधाति त्रेधाविहि॒तो वै पुरु॑षो॒ यावा॑ने॒व पुरु॑ष॒स्तस्मा॒न्-निर्.ऋ॑ति॒मव॑ यजते॒ परा॑ची॒रुप॑ - [  ] \newline

\textbf{Pada Paata} \newline

वै । निर्.ऋ॑त्या॒ इति॒ निः - ऋ॒त्यै॒ । दिक् । स्वाया᳚म् । ए॒व । दि॒शि । निर्.ऋ॑ति॒मिति॒ निः - ऋ॒ति॒म् । नि॒रव॑दयत॒ इति॑ निः - अव॑दयते । स्वकृ॑त॒ इति॒ स्व - कृ॒ते॒ । इरि॑णे । उपेति॑ । द॒धा॒ति॒ । प्र॒द॒र इति॑ प्र - द॒रे । वा॒ । ए॒तत् । वै । निर्.ऋ॑त्या॒ इति॒ निः - ऋ॒त्याः॒ । आ॒यत॑न॒मित्या᳚ - यत॑नम् । स्वे । ए॒व । आ॒यत॑न॒ इत्या᳚ - यत॑ने । निर्.ऋ॑ति॒मिति॒ निः - ऋ॒ति॒म् । नि॒रव॑दयत॒ इति॑ निः - अव॑दयते । शि॒क्य᳚म् । अ॒भि । उपेति॑ । द॒धा॒ति॒ । नै॒र्॒.ऋ॒त इति॑ नैः - ऋ॒तः । वै । पाशः॑ । सा॒क्षादिति॑ स - अ॒क्षात् । ए॒व । ए॒न॒म् । नि॒र्॒.ऋ॒ति॒पा॒शादिति॑ निर्.ऋति - पा॒शात् । मु॒ञ्च॒ति॒ । ति॒स्रः । उपेति॑ । द॒धा॒ति॒ । त्रे॒धा॒वि॒हि॒त इति॑ त्रेधा - वि॒हि॒तः । वै । पुरु॑षः । यावान्॑ । ए॒व । पुरु॑षः । तस्मा᳚त् । निर्.ऋ॑ति॒मिति॒ निः - ऋ॒ति॒म् । अवेति॑ । य॒ज॒ते॒ । परा॑चीः । उपेति॑ ।  \newline




\markright{ TS 5.2.4.4  \hfill https://www.vedavms.in \hfill}

\section{ TS 5.2.4.4 }

\textbf{TS 5.2.4.4 } \newline
\textbf{Samhita Paata} \newline

दधाति॒ परा॑चीमे॒वास्मा॒न्-निर्.ऋ॑तिं॒ प्रणु॑द॒ते ऽप्र॑तीक्ष॒मा य॑न्ति॒ निर्.ऋ॑त्या अ॒न्तर्.हि॑त्यै मार्जयि॒त्वोप॑ तिष्ठन्ते मेद्ध्य॒त्वाय॒ गार्.ह॑पत्य॒मुप॑ तिष्ठन्ते निर्.ऋति लो॒क ए॒व च॑रि॒त्वा पू॒ता दे॑वलो॒कमु॒पाव॑र्तन्त॒ एक॒योप॑ तिष्ठन्त एक॒धैव यज॑माने वी॒र्यं॑ दधति नि॒वेश॑नः स॒ङ्गम॑नो॒ वसू॑ना॒मित्या॑ह प्र॒जा वै प॒शवो॒ वसु॑ प्र॒जयै॒वैनं॑ प॒शुभिः॒ सम॑र्द्धयन्ति ॥ \newline

\textbf{Pada Paata} \newline

द॒धा॒ति॒ । परा॑चीम् । ए॒व । अ॒स्मा॒त् । निर्.ऋ॑ति॒मिति॒ निः - ऋ॒ति॒म् । प्रेति॑ । नु॒द॒ते॒ । अप्र॑तीक्ष॒मित्यप्र॑ति - ई॒क्ष॒म् । एति॑ । य॒न्ति॒ । निर्.ऋ॑त्या॒ इति॒ निः - ऋ॒त्याः॒ । अ॒न्तर्.हि॑त्या॒ इत्य॒न्तः - हि॒त्यै॒ । मा॒र्ज॒यि॒त्वा । उपेति॑ । ति॒ष्ठ॒न्ते॒ । मे॒द्ध्य॒त्वायेति॑ मेद्ध्य - त्वाय॑ । गार्.ह॑पत्य॒मिति॒ गार्.ह॑ - प॒त्य॒म् । उपेति॑ । ति॒ष्ठ॒न्ते॒ । नि॒र्॒.ऋ॒ति॒लो॒क इति॑ निर्.ऋति - लो॒के । ए॒व । च॒रि॒त्वा । पू॒ताः । दे॒व॒लो॒कमिति॑ देव -  लो॒कम् । उ॒पाव॑र्तन्त॒ इत्यु॑प - आव॑र्तन्ते । एक॑या । उपेति॑ । ति॒ष्ठ॒न्ते॒ । ए॒क॒धेत्ये॑क - धा । ए॒व । यज॑माने । वी॒र्य᳚म् । द॒ध॒ति॒ । नि॒वेश॑न॒ इति॑ नि-वेश॑नः । स॒ङ्गम॑न॒ इति॑ सं- गम॑नः । वसू॑नाम् । इति॑ । आ॒ह॒ । प्र॒जेति॑ प्र - जा । वै ।   प॒शवः॑ । वसु॑ । प्र॒जयेति॑ प्र - जया᳚ । ए॒व । ए॒न॒म् । प॒शुभि॒रिति॑ प॒शु - भिः॒ । समिति॑ । अ॒द्‌र्ध॒य॒न्ति॒ ॥  \newline




\markright{ TS 5.2.5.1  \hfill https://www.vedavms.in \hfill}

\section{ TS 5.2.5.1 }

\textbf{TS 5.2.5.1 } \newline
\textbf{Samhita Paata} \newline

पु॒रु॒ष॒मा॒त्रेण॒ वि मि॑मीते य॒ज्ञेन॒ वै पुरु॑षः॒ संमि॑तो यज्ञ्प॒रुषै॒वैनं॒ ॅविमि॑मीते॒ यावा॒न् पुरु॑ष ऊ॒र्द्ध्वबा॑हु॒स्तावा᳚न् भवत्ये॒ताव॒द्वै पुरु॑षे वी॒र्यं॑ ॅवी॒र्ये॑णै॒वैनं॒ ॅवि मि॑मीते प॒क्षी भ॑वति॒ न ह्य॑प॒क्षः पति॑तु॒-मर्.ह॑त्यर॒त्निना॑ प॒क्षौ द्राघी॑याꣳसौ भवत॒स्तस्मा᳚त् प॒क्षप्र॑वयाꣳसि॒ वयाꣳ॑सि व्याममा॒त्रौ प॒क्षौ च॒ पुच्छं॑ च भवत्ये॒ताव॒द्वै पुरु॑षे वी॒र्यं॑ - [  ] \newline

\textbf{Pada Paata} \newline

पु॒रु॒ष॒मा॒त्रेणेति॑ पुरुष - मा॒त्रेण॑ । वीति॑ । मि॒मी॒ते॒ । य॒ज्ञेन॑ । वै । पुरु॑षः । सम्मि॑त॒ इति॒ सं - मि॒तः॒ । य॒ज्ञ्॒प॒रुषेति॑ यज्ञ्-प॒रुषा᳚ । ए॒व । ए॒न॒म् । वीति॑ । मि॒मी॒ते॒ । यावान्॑ । पुरु॑षः । ऊ॒द्‌र्ध्वबा॑हु॒रित्यु॒द्‌र्ध्व-बा॒हुः॒ । तावान्॑ । भ॒व॒ति॒ । ए॒ताव॑त् । वै । पुरु॑षे । वी॒र्य᳚म् । वी॒र्ये॑ण । ए॒व । ए॒न॒म् । वीति॑ । मि॒मी॒ते॒ । प॒क्षी । भ॒व॒ति॒ । न । हि । अ॒प॒क्षः । पति॑तुम् । अर्.ह॑ति । अ॒र॒त्निना᳚ । प॒क्षौ । द्राघी॑याꣳसौ । भ॒व॒तः॒ । तस्मा᳚त् । प॒क्षप्र॑वयाꣳ॒॒सीति॑ प॒क्ष - प्र॒व॒याꣳ॒॒सि॒ । वयाꣳ॑सि । व्या॒म॒मा॒त्राविति॑ व्याम - मा॒त्रौ । प॒क्षौ । च॒ । पुच्छ᳚म् । च॒ । भ॒व॒ति॒ । ए॒ताव॑त् । वै । पुरु॑षे । वी॒र्य᳚म् ।  \newline




\markright{ TS 5.2.5.2  \hfill https://www.vedavms.in \hfill}

\section{ TS 5.2.5.2 }

\textbf{TS 5.2.5.2 } \newline
\textbf{Samhita Paata} \newline

ॅवी॒र्य॑संमितो॒ वेणु॑ना॒ वि मि॑मीत आग्ने॒यो वै वेणुः॑ सयोनि॒त्वाय॒ यजु॑षा युनक्ति॒ यजु॑षा कृषति॒ व्यावृ॑त्त्यै षड्ग॒वेन॑ कृषति॒ षड् वा ऋ॒तव॑ ऋ॒तुभि॑रे॒वैनं॑ कृषति॒ यद् द्वा॑दशग॒वेन॑ संॅवथ्स॒रेणै॒वे यं ॅवा अ॒ग्ने-र॑तिदा॒हाद॑बिभे॒थ् सैतद् द्वि॑गु॒णम॑पश्यत् कृ॒ष्टं चाकृ॑ष्टं च॒ ततो॒ वा इ॒मां नाऽत्य॑दह॒द्यत् कृ॒ष्टं चाकृ॑ष्टं च॒ - [  ] \newline

\textbf{Pada Paata} \newline

वी॒र्य॑सम्मित॒ इति॑ वी॒र्य॑ - स॒म्मि॒तः॒ । वेणु॑ना । वीति॑ । मि॒मी॒ते॒ । आ॒ग्ने॒यः । वै । वेणुः॑ । स॒यो॒नि॒त्वायेति॑ सयोनि - त्वाय॑ । यजु॑षा । यु॒न॒क्ति॒ । यजु॑षा । कृ॒ष॒ति॒ । व्यावृ॑त्त्या॒ इति॑ वि - आवृ॑त्त्यै । ष॒ड्ग॒वेनेति॑ षट् - ग॒वेन॑ । कृ॒ष॒ति॒ । षट् । वै । ऋ॒तवः॑ । ऋ॒तुभि॒रित्यृ॒तु - भिः॒ । ए॒व । ए॒न॒म् । कृ॒ष॒ति॒ । यत् । द्वा॒द॒श॒ग॒वेनेति॑ द्वादश - ग॒वेन॑ । सं॒ॅव॒थ्स॒रेणेति॑ सं - व॒थ्स॒रेण॑ । ए॒व । इ॒यम् । वै । अ॒ग्नेः । अ॒ति॒दा॒हादित्य॑ति - दा॒हात् । अ॒बि॒भे॒त् । सा । ए॒तत् । द्वि॒गु॒णमिति॑ द्वि - गु॒णम् । अ॒प॒श्य॒त् । कृ॒ष्टम् । च॒ । अकृ॑ष्टम् । च॒ । ततः॑ । वै । इ॒माम् । न । अतीति॑ । अ॒द॒ह॒त् । यत् । कृ॒ष्टम् । च॒ । अकृ॑ष्टम् । च॒ ।  \newline




\markright{ TS 5.2.5.3  \hfill https://www.vedavms.in \hfill}

\section{ TS 5.2.5.3 }

\textbf{TS 5.2.5.3 } \newline
\textbf{Samhita Paata} \newline

भव॑त्य॒स्या अन॑तिदाहाय द्विगु॒णं त्वा अ॒ग्नि-मुद्य॑न्तु-मर्.ह॒तीत्या॑हु॒र्यत् कृ॒ष्टं चाकृ॑ष्टं च॒ भव॑त्य॒ग्नेरुद्य॑त्या ए॒ताव॑न्तो॒ वै प॒शवो᳚ द्वि॒पाद॑श्च॒ चतु॑ष्पादश्च॒ तान्. यत् प्राच॑ उथ्सृ॒जेद्-रु॒द्रायापि॑ दद्ध्या॒द्-यद्-द॑क्षि॒णा पि॒तृभ्यो॒ निधु॑वे॒द्यत् प्र॒तीचो॒ रक्षाꣳ॑सि हन्यु॒रुदी॑च॒ उथ्सृ॑जत्ये॒षा वै दे॑वमनु॒ष्याणाꣳ॑ शा॒न्ता दिक् - [  ] \newline

\textbf{Pada Paata} \newline

भव॑ति । अ॒स्याः । अन॑तिदाहा॒येत्यन॑ति - दा॒हा॒य॒ । द्वि॒गु॒णमिति॑ द्वि - गु॒णम् । तु । वै । अ॒ग्निम् । उद्य॑न्तु॒मित्युत्-य॒न्तु॒म् । अ॒र्.॒ह॒ति॒ । इति॑ । आ॒हुः॒ । यत् । कृ॒ष्टम् । च॒ । अकृ॑ष्टम् । च॒ । भव॑ति । अ॒ग्नेः । उद्य॑त्या॒ इत्युत् - य॒त्यै॒ । ए॒ताव॑न्तः । वै । प॒शवः॑ । द्वि॒पाद॒ इति॑ द्वि - पादः॑ । च॒ । चतु॑ष्पाद॒ इति॒ चतुः॑ - पा॒दः॒ । च॒ । तान् । यत् । प्राचः॑ । उ॒थ्सृ॒जेदित्यु॑त् - सृ॒जेत् । रु॒द्राय॑ । अपीति॑ । द॒द्ध्या॒त् । यत् । द॒क्षि॒णा । पि॒तृभ्य॒ इति॑ पि॒तृ - भ्यः॒ । नीति॑ । धु॒वे॒त् । यत् । प्र॒तीचः॑ । रक्षाꣳ॑सि । ह॒न्युः॒ । उदी॑चः । उदिति॑ । सृ॒ज॒ति॒ । ए॒षा । वै । दे॒व॒म॒नु॒ष्याणा॒मिति॑ देव - म॒नु॒ष्याणा᳚म् । शा॒न्ता । दिक् ।  \newline




\markright{ TS 5.2.5.4  \hfill https://www.vedavms.in \hfill}

\section{ TS 5.2.5.4 }

\textbf{TS 5.2.5.4 } \newline
\textbf{Samhita Paata} \newline

तामे॒वैना॒ननूथ् सृ॑ज॒त्यथो॒ खल्वि॒मां दिश॒मुथ् सृ॑जत्य॒सौ वा आ॑दि॒त्यः प्रा॒णः प्रा॒णमे॒वैना॒-ननूथ्सृ॑जति दक्षि॒णा प॒र्याव॑र्तन्ते॒ स्वमे॒व वी॒र्य॑मनु॑ प॒र्याव॑र्तन्ते॒ तस्मा॒द्-दक्षि॒णोऽर्द्ध॑ आ॒त्मनो॑ वी॒र्या॑वत्त॒रोऽथो॑ आदि॒त्यस्यै॒वाऽऽ*वृत॒मनु॑ प॒र्याव॑र्तन्ते॒ तस्मा॒त् परा᳚ञ्चः प॒शवो॒ वि ति॑ष्ठन्ते प्र॒त्यञ्च॒ आ व॑र्तन्ते ति॒स्रस्ति॑स्रः॒ सीताः᳚ - [  ] \newline

\textbf{Pada Paata} \newline

ताम् । ए॒व । ए॒ना॒न् । अनु॑ । उदिति॑ । सृ॒ज॒ति॒ । अथो॒ इति॑ । खलु॑ । इ॒माम् । दिश᳚म् । उदिति॑ । सृ॒ज॒ति॒ । अ॒सौ । वै । आ॒दि॒त्यः । प्रा॒ण इति॑ प्र - अ॒नः । प्रा॒णमिति॑ प्र - अ॒नम् । ए॒व । ए॒ना॒न् । अनु॑ । उदिति॑ । सृ॒ज॒ति॒ । द॒क्षि॒णा । प॒र्याव॑र्तन्त॒ इति॑ परि - आव॑र्तन्ते । स्वम् । ए॒व । वी॒र्य᳚म् । अन्विति॑ । प॒र्याव॑र्तन्त॒ इति॑ परि - आव॑र्तन्ते । तस्मा᳚त् । दक्षि॑णः । अद्‌र्धः॑ । आ॒त्मनः॑ । वी॒र्या॑वत्तर॒ इति॑ वी॒र्या॑वत् - त॒रः॒ । अथो॒ इति॑ । आ॒दि॒त्यस्य॑ । ए॒व । आ॒वृत॒मित्या᳚ - वृत᳚म् । अन्विति॑ । प॒र्याव॑र्तन्त॒ इति॑ परि-आव॑र्तन्ते । तस्मा᳚त् । परा᳚ञ्चः । प॒शवः॑ । वीति॑ । ति॒ष्ठ॒न्ते॒ । प्र॒त्यञ्चः॑ । एति॑ । व॒र्त॒न्ते॒ । ति॒स्रस्ति॑स्र॒ इति॑ ति॒स्रः - ति॒स्रः॒ । सीताः᳚ ।  \newline




\markright{ TS 5.2.5.5  \hfill https://www.vedavms.in \hfill}

\section{ TS 5.2.5.5 }

\textbf{TS 5.2.5.5 } \newline
\textbf{Samhita Paata} \newline

कृषति त्रि॒वृत॑मे॒व य॑ज्ञ्मु॒खे वि या॑तय॒त्योष॑धीर्वपति॒ ब्रह्म॒णाऽन्न॒मव॑ रुन्धे॒ ऽर्के᳚ऽर्कश्ची॑यते चतुर्द॒शभि॑र्वपति स॒प्त ग्रा॒म्या ओष॑धयः स॒प्ताऽऽर॒ण्या उ॒भयी॑षा॒मव॑रुद्ध्या॒ अन्न॑स्यान्नस्य वप॒त्यन्न॑स्या-न्न॒स्याव॑रुद्ध्यै कृ॒ष्टे व॑पति कृ॒ष्टे ह्योष॑धयः प्रति॒तिष्ठ॑न्त्यनुसी॒तं ॅव॑पति॒ प्रजा᳚त्यै द्वाद॒शसु॒ सीता॑सु वपति॒ द्वाद॑श॒ मासाः᳚ संॅवथ्स॒रः सं॑ॅवथ्स॒रेणै॒वास्मा॒ अन्नं॑ पचति॒ यद॑ग्नि॒चि - [  ] \newline

\textbf{Pada Paata} \newline

कृ॒ष॒ति॒ । त्रि॒वृत॒मिति॑ त्रि - वृत᳚म् । ए॒व । य॒ज्ञ्॒मु॒ख इति॑ यज्ञ्-मु॒खे । वीति॑ । या॒त॒य॒ति॒ । ओष॑धीः । व॒प॒ति॒ । ब्रह्म॑णा । अन्न᳚म् । अवेति॑ । रु॒न्धे॒ । अ॒र्के । अ॒र्कः । ची॒य॒ते॒ । च॒तु॒र्द॒शभि॒रिति॑ चतुर्द॒श - भिः॒ । व॒प॒ति॒ । स॒प्त । ग्रा॒म्याः । ओष॑धयः । स॒प्त । आ॒र॒ण्याः । उ॒भयी॑षाम् । अव॑रुद्ध्या॒ इत्यव॑ - रु॒द्ध्यै॒ । अन्न॑स्यान्न॒स्येत्यन्न॑स्य - अ॒न्न॒स्य॒ । व॒प॒ति॒ । अन्न॑स्यान्न॒स्येत्यन्न॑स्य-अ॒न्न॒स्य॒ । अव॑रुद्ध्या॒ इत्यव॑-रु॒द्ध्यै॒ । कृ॒ष्टे । व॒प॒ति॒ । कृ॒ष्टे । हि । ओष॑धयः । प्र॒ति॒तिष्ठ॒न्तीति॑ प्रति-तिष्ठ॑न्ति । अ॒नु॒सी॒तमित्य॑नु - सी॒तम् । व॒प॒ति॒ । प्रजा᳚त्या॒ इति॒ प्र - जा॒त्यै॒ । द्वा॒द॒शस्विति॑ द्वाद॒श - सु॒ । सीता॑सु । व॒प॒ति॒ । द्वाद॑श । मासाः᳚ । सं॒ॅव॒थ्स॒र इति॑ सं - व॒थ्स॒रः । सं॒ॅव॒थ्स॒रेणेति॑ सं - व॒थ्स॒रेण॑ । ए॒व । अ॒स्मै॒ । अन्न᳚म् । प॒च॒ति॒ । यत् । अ॒ग्नि॒चिदित्य॑ग्नि - चित् ।  \newline




\markright{ TS 5.2.5.6  \hfill https://www.vedavms.in \hfill}

\section{ TS 5.2.5.6 }

\textbf{TS 5.2.5.6 } \newline
\textbf{Samhita Paata} \newline

-दन॑वरुद्धस्या-श्नी॒यादव॑-रुद्धेन॒ व्यृ॑द्ध्येत॒ ये वन॒स्पती॑नां फल॒ग्रह॑य॒-स्तानि॒द्ध्मेऽपि॒ प्रोक्षे॒-दन॑वरुद्ध॒स्या-व॑रुद्ध्यै दि॒ग्भ्यो लो॒ष्टान्थ् सम॑स्यति दि॒शामे॒व वी॒र्य॑मव॒रुद्ध्य॑ दि॒शां ॅवी॒र्ये᳚ऽग्निं चि॑नुते॒ यं द्वि॒ष्याद्-यत्र॒ स स्यात् तस्यै॑ दि॒शो लो॒ष्टमा ह॑रे॒दिष॒-मूर्ज॑म॒हमि॒त आ द॑द॒ इतीष॑मे॒वोर्जं॒ तस्यै॑ दि॒शोऽव॑ ( ) रुन्धे॒ क्षोधु॑को भवति॒ यस्तस्यां᳚ दि॒शि भव॑त्युत्तरवे॒दिमुप॑ वपत्युत्तरवे॒द्याꣳ ह्य॑ग्निश्ची॒यते ऽथो॑ प॒शवो॒ वा उ॑त्तरवे॒दिः प॒शूने॒वाव॑ रु॒न्धेऽथो॑ यज्ञ्प॒रुषोऽन॑न्तरित्यै ॥ \newline

\textbf{Pada Paata} \newline

अन॑वरुद्ध॒स्येत्यन॑व-रु॒द्ध॒स्य॒ । अ॒श्नी॒यात् । अव॑रुद्धे॒नेत्यव॑ - रु॒द्धे॒न॒ । वीति॑ । ऋ॒द्ध्ये॒त॒ । ये । वन॒स्पती॑नाम् । फ॒ल॒ग्रह॑य॒ इति॑ फल-ग्रह॑यः । तान् । इ॒द्ध्मे । अपि॑ । प्रेति॑ । उ॒क्षे॒त् । अन॑वरुद्ध॒स्येत्यन॑व-रु॒द्ध॒स्य॒ । अव॑रुद्ध्या॒ इत्यव॑ - रु॒द्ध्यै॒ । दि॒ग्भ्य इति॑ दिक् - भ्यः । लो॒ष्टान् । समिति॑ । अ॒स्य॒ति॒ । दि॒शाम् । ए॒व । वी॒र्य᳚म् । अ॒व॒रुद्ध्येत्य॑व-रुद्ध्य॑ । दि॒शाम् । वी॒र्ये᳚ । अ॒ग्निम् । चि॒नु॒ते॒ । यम् । द्वि॒ष्यात् । यत्र॑ । सः । स्यात् । तस्यै᳚ । दि॒शः । लो॒ष्टम् । एति॑ । ह॒रे॒त् । इष᳚म् । ऊर्ज᳚म् । अ॒हम् । इ॒तः । एति॑ । द॒दे॒ । इति॑ । इष᳚म् । ए॒व । ऊर्ज᳚म् । तस्यै᳚ । दि॒शः । अवेति॑ ( ) । रु॒न्धे॒ । क्षोधु॑कः । भ॒व॒ति॒ । यः । तस्या᳚म् । दि॒शि । भव॑ति । उ॒त्त॒र॒वे॒दिमित्यु॑त्तर - वे॒दिम् । उपेति॑ । व॒प॒ति॒ । उ॒त्त॒र॒वे॒द्यामित्यु॑त्तर - वे॒द्याम् । हि । अ॒ग्निः । ची॒यते᳚ । अथो॒ इति॑ । प॒शवः॑ । वै । उ॒त्त॒र॒वे॒दिरित्यु॑त्तर - वे॒दिः । प॒शून् । ए॒व । अवेति॑ । रु॒न्धे॒ । अथो॒ इति॑ । य॒ज्ञ्॒प॒रुष॒ इति॑ यज्ञ् - प॒रुषः॑ । अन॑न्तरित्या॒ इत्यन॑न्तः - इ॒त्यै॒ ॥  \newline




\markright{ TS 5.2.6.1  \hfill https://www.vedavms.in \hfill}

\section{ TS 5.2.6.1 }

\textbf{TS 5.2.6.1 } \newline
\textbf{Samhita Paata} \newline

अग्ने॒ तव॒ श्रवो॒ वय॒ इति॒ सिक॑ता॒ नि व॑पत्ये॒तद्वा अ॒ग्नेर्वै᳚श्वान॒रस्य॑ सू॒क्तꣳ सू॒क्तेनै॒व वै᳚श्वान॒रमव॑ रुन्धे ष॒ड्भिर्नि व॑पति॒ षड्वा ऋ॒तवः॑ संॅवथ्स॒रः सं॑ॅवथ्स॒रो᳚ऽग्निर्वै᳚श्वान॒रः सा॒क्षादे॒व वै᳚श्वान॒रमव॑ रुन्धे समु॒द्रं ॅवै नामै॒तच्छन्दः॑ समु॒द्रमनु॑ प्र॒जाः प्रजा॑यन्ते॒ यदे॒तेन॒ सिक॑ता नि॒ वप॑ति प्र॒जानां᳚ प्र॒जन॑ना॒येन्द्रो॑ - [  ] \newline

\textbf{Pada Paata} \newline

अग्ने᳚ । तव॑ । श्रवः॑ । वयः॑ । इति॑ । सिक॑ताः । नीति॑ । व॒प॒ति॒ । ए॒तत् । वै । अ॒ग्नेः । वै॒श्वा॒न॒रस्य॑ । सू॒क्तमिति॑ सु - उ॒क्तम् । सू॒क्तेनेति॑ सु - उ॒क्तेन॑ । ए॒व । वै॒श्वा॒न॒रम् । अवेति॑ । रु॒न्धे॒ । ष॒ड्भिरिति॑ षट् - भिः । नीति॑ । व॒प॒ति॒ । षट् । वै । ऋ॒तवः॑ । सं॒ॅव॒थ्स॒र इति॑ सं - व॒थ्स॒रः । सं॒ॅव॒थ्स॒र इति॑ सं - व॒थ्स॒रः । अ॒ग्निः । वै॒श्वा॒न॒रः । सा॒क्षादिति॑ स - अ॒क्षात् । ए॒व । वै॒श्वा॒न॒रम् । अवेति॑ । रु॒न्धे॒ । स॒मु॒द्रम् । वै । नाम॑ । ए॒तत् । छन्दः॑ । स॒मु॒द्रम् । अन्विति॑ । प्र॒जा इति॑ प्र - जाः । प्रेति॑ । जा॒य॒न्ते॒ । यत् । ए॒तेन॑ । सिक॑ताः । नि॒वप॒तीति॑ नि - वप॑ति । प्र॒जाना॒मिति॑ प्र - जाना᳚म् । प्र॒जन॑ना॒येति॑ प्र - जन॑नाय । इन्द्रः॑ ।  \newline




\markright{ TS 5.2.6.2  \hfill https://www.vedavms.in \hfill}

\section{ TS 5.2.6.2 }

\textbf{TS 5.2.6.2 } \newline
\textbf{Samhita Paata} \newline

वृ॒त्राय॒ वज्रं॒ प्राह॑र॒थ् स त्रे॒धा व्य॑भव॒थ् स्फ्यस्तृती॑यꣳ॒॒ रथ॒स्तृती॑यं॒ ॅयूप॒स्तृती॑यं॒ ॅये᳚ऽन्तश्श॒रा अशी᳚र्यन्त॒ ताः शर्क॑रा अभव॒न् तच्छर्क॑राणाꣳ शर्कर॒त्वं ॅवज्रो॒ वै शर्क॑राः प॒शुर॒ग्नि-र्यच्छर्क॑राभिर॒ग्निं प॑रिमि॒नोति॒ वज्रे॑णै॒वास्मै॑ प॒शून् परि॑ गृह्णाति॒ तस्मा॒द्-वज्रे॑ण प॒शवः॒ परि॑गृहीता॒स्तस्मा॒थ् स्थेया॒नस्थे॑यसो॒ नोप॑ हरते त्रिस॒प्ताभिः॑ प॒शुका॑मस्य॒ - [  ] \newline

\textbf{Pada Paata} \newline

वृ॒त्राय॑ । वज्र᳚म् । प्रेति॑ । अ॒ह॒र॒त् । सः । त्रे॒धा । वीति॑ । अ॒भ॒व॒त् । स्फ्यः । तृती॑यम् । रथः॑ । तृती॑यम् । यूपः॑ । तृती॑यम् । ये । अ॒न्त॒श्श॒रा इत्य॑न्तः-श॒राः । अशी᳚र्यन्त । ताः । शर्क॑राः । अ॒भ॒व॒न्न् । तत् । शर्क॑राणाम् । श॒र्क॒र॒त्वमिति॑ शर्कर - त्वम् । वज्रः॑ । वै । शर्क॑राः । प॒शुः । अ॒ग्निः । यत् । शर्क॑राभिः । अ॒ग्निम् । प॒रि॒मि॒नोतीति॑ परि - मि॒नोति॑ । वज्रे॑ण । ए॒व । अ॒स्मै॒ । प॒शून् । परीति॑ । गृ॒ह्णा॒ति॒ । तस्मा᳚त् । वज्रे॑ण । प॒शवः॑ । परि॑गृहीता॒ इति॒ परि॑ - गृ॒ही॒ताः॒ । तस्मा᳚त् । स्थेयान्॑ । अस्थे॑यसः । न । उपेति॑ । ह॒र॒ते॒ । त्रि॒स॒प्ताभि॒रिति॑ त्रि - स॒प्ताभिः॑ । प॒शुका॑म॒स्येति॑ प॒शु - का॒म॒स्य॒ ।  \newline




\markright{ TS 5.2.6.3  \hfill https://www.vedavms.in \hfill}

\section{ TS 5.2.6.3 }

\textbf{TS 5.2.6.3 } \newline
\textbf{Samhita Paata} \newline

परि॑ मिनुयाथ् स॒प्त वै शी॑र्.ष॒ण्याः᳚ प्रा॒णाः प्रा॒णाः प॒शवः॑ प्रा॒णैरे॒वास्मै॑ प॒शूनव॑ रुन्धे त्रिण॒वाभि॒-र्भ्रातृ॑व्यवत-स्त्रि॒वृत॑मे॒व वज्रꣳ॑ स॒भृंत्य॒ भ्रातृ॑व्याय॒ प्रह॑रति॒ स्तृत्या॒ अप॑रिमिताभिः॒ परि॑ मिनुया॒-दप॑रिमित॒स्या-व॑रुद्ध्यै॒ यं का॒मये॑ताप॒शुः स्या॒दित्यप॑रिमित्य॒ तस्य॒ शर्क॑राः॒ सिक॑ता॒ व्यू॑हे॒दप॑रिगृहीत ए॒वास्य॑ विषू॒चीनꣳ॒॒ रेतः॒ परा॑ सिञ्चत्यप॒शुरे॒व भ॑वति॒ - [  ] \newline

\textbf{Pada Paata} \newline

परीति॑ । मि॒नु॒या॒त् । स॒प्त । वै । शी॒र्॒.ष॒ण्याः᳚ । प्रा॒णा इति॑ प्र-अ॒नाः । प्रा॒णा इति॑ प्र - अ॒नाः । प॒शवः॑ । प्रा॒णैरिति॑ प्र - अ॒नैः । ए॒व । अ॒स्मै॒ । प॒शून् । अवेति॑ । रु॒न्धे॒ । त्रि॒ण॒वाभि॒रिति॑ त्रि - न॒वाभिः॑ । भ्रातृ॑व्यवत॒ इति॒ भ्रातृ॑व्य - व॒तः॒ । त्रि॒वृत॒मिति॑ त्रि - वृत᳚म् । ए॒व । वज्र᳚म् । स॒भृंत्येति॑ सं - भृत्य॑ । भ्रातृ॑व्याय । प्रेति॑ । ह॒र॒ति॒ । स्तृत्यै᳚ । अप॑रिमिताभि॒रित्यप॑रि - मि॒ता॒भिः॒ । परीति॑ । मि॒नु॒या॒त् । अप॑रिमित॒स्येत्यप॑रि - मि॒त॒स्य॒ । अव॑रुद्ध्या॒ इत्यव॑-रु॒द्ध्यै॒ । यम् । का॒मये॑त । अ॒प॒शुः । स्या॒त् । इति॑ । अप॑रिमि॒त्येत्यप॑रि - मि॒त्य॒ । तस्य॑ । शर्क॑राः । सिक॑ताः । वीति॑ । ऊ॒हे॒त् । अप॑रिगृहीत॒ इत्यप॑रि - गृ॒ही॒ते॒ । ए॒व । अ॒स्य॒ । वि॒षू॒चीन᳚म् । रेतः॑ । परेति॑ । सि॒ञ्च॒ति॒ । अ॒प॒शुः । ए॒व । भ॒व॒ति॒ ।  \newline




\markright{ TS 5.2.6.4  \hfill https://www.vedavms.in \hfill}

\section{ TS 5.2.6.4 }

\textbf{TS 5.2.6.4 } \newline
\textbf{Samhita Paata} \newline

यं का॒मये॑त पशु॒मान्थ् स्या॒दिति॑ परि॒मित्य॒ तस्य॒ शर्क॑राः॒ सिक॑ता॒ व्यू॑हे॒त् परि॑गृहीत ए॒वास्मै॑ समी॒चीनꣳ॒॒ रेतः॑ सिञ्चति पशु॒माने॒व भ॑वति सौ॒म्या व्यू॑हति॒ सोमो॒ वै रे॑तो॒धा रेत॑ ए॒व तद्-द॑धाति गायत्रि॒या ब्रा᳚ह्म॒णस्य॑ गाय॒त्रो हि ब्रा᳚ह्म॒ण-स्त्रि॒ष्टुभा॑ राज॒न्य॑स्य॒ त्रैष्टु॑भो॒ हि रा॑ज॒न्यः॑ श॒म्युं बा॑र्.हस्प॒त्यं मेधो॒ नोपा॑नम॒थ् सो᳚ऽग्निं प्राऽवि॑श॒थ् - [  ] \newline

\textbf{Pada Paata} \newline

यम् । का॒मये॑त । प॒शु॒मानिति॑ पशु - मान् । स्या॒त् । इति॑ । प॒रि॒मित्येति॑ परि-मित्य॑ । तस्य॑ । शर्क॑राः । सिक॑ताः । वीति॑ । ऊ॒हे॒त् । परि॑गृहीत॒ इति॒ परि॑ - गृ॒ही॒ते॒ । ए॒व । अ॒स्मै॒ । स॒मी॒चीन᳚म् । रेतः॑ । सि॒ञ्च॒ति॒ । प॒शु॒मानिति॑ पशु-मान् । ए॒व । भ॒व॒ति॒ । सौ॒म्या । वीति॑ । ऊ॒ह॒ति॒ । सोमः॑ । वै । रे॒तो॒धा इति॑ रेतः - धाः । रेतः॑ । ए॒व । तत् । द॒धा॒ति॒ । गा॒य॒त्रि॒या । ब्रा॒ह्म॒णस्य॑ । गा॒य॒त्रः । हि । ब्रा॒ह्म॒णः । त्रि॒ष्टुभा᳚ । रा॒ज॒न्य॑स्य । त्रैष्टु॑भः । हि । रा॒ज॒न्यः॑ । शं॒ॅयुमिति॑ शं - युम् । बा॒र्॒.ह॒स्प॒त्यम् । मेधः॑ । न । उपेति॑ । अ॒न॒म॒त् । सः । अ॒ग्निम् । प्रेति॑ । अ॒वि॒श॒त् ।  \newline




\markright{ TS 5.2.6.5  \hfill https://www.vedavms.in \hfill}

\section{ TS 5.2.6.5 }

\textbf{TS 5.2.6.5 } \newline
\textbf{Samhita Paata} \newline

सो᳚ऽग्नेः कृष्णो॑ रू॒पं कृ॒त्वोदा॑यत॒ सोऽश्वं॒ प्राऽवि॑श॒थ् सोऽश्व॑स्या-वान्तरश॒फो॑-भव॒द्-यदश्व॑माक्र॒मय॑ति॒ य ए॒व मेधोऽश्वं॒ प्राऽवि॑श॒त् तमे॒वाव॑ रुन्धे प्र॒जाप॑तिना॒ऽग्निश्चे॑त॒व्य॑ इत्या॑हुः प्राजाप॒त्योऽश्वो॒ यदश्व॑माक्र॒मय॑ति प्र॒जाप॑तिनै॒वाऽग्निं चि॑नुते पुष्करप॒र्णमुप॑ दधाति॒ योनि॒र्वा अ॒ग्नेः पु॑ष्करप॒र्णꣳ सयो॑नि- ( ) -मे॒वाग्निं चि॑नुते॒ ऽपां पृ॒ष्ठम॒सीत्युप॑ दधात्य॒पां ॅवा ए॒तत् पृ॒ष्ठं ॅयत् पु॑ष्करप॒र्णꣳ रू॒पेणै॒वैन॒दुप॑ दधाति ॥ \newline

\textbf{Pada Paata} \newline

सः । अ॒ग्नेः । कृष्णः॑ । रू॒पम् । कृ॒त्वा । उदिति॑ । आ॒य॒त॒ । सः । अश्व᳚म् । प्रेति॑ । अ॒वि॒श॒त् । सः । अश्व॑स्य । अ॒वा॒न्त॒र॒श॒फ इत्य॑वान्तर - श॒फः । अ॒भ॒व॒त् । यत् । अश्व᳚म् । आ॒क्र॒मय॒तीत्या᳚ - क्र॒मय॑ति । यः । ए॒व । मेधः॑ । अश्व᳚म् । प्रेति॑ । अवि॑शत् । तम् । ए॒व । अवेति॑ । रु॒न्धे॒ । प्र॒जाप॑ति॒नेति॑ प्र॒जा - प॒ति॒ना॒ । अ॒ग्निः । चे॒त॒व्यः॑ । इति॑ । आ॒हुः॒ । प्रा॒जा॒प॒त्य इति॑ प्राजा - प॒त्यः । अश्वः॑ । यत् । अश्व᳚म् । आ॒क्र॒मय॒तीत्या᳚-क्र॒मय॑ति । प्र॒जाप॑ति॒नेति॑ प्र॒जा - प॒ति॒ना॒ । ए॒व । अ॒ग्निम् । चि॒नु॒ते॒ । पु॒ष्क॒र॒प॒र्णमिति॑ पुष्कर - प॒र्णम् । उपेति॑ । द॒धा॒ति॒ । योनिः॑ । वै । अ॒ग्नेः । पु॒ष्क॒र॒प॒र्णमिति॑ पुष्कर - प॒र्णम् । सयो॑नि॒मिति॒ स-यो॒नि॒म् ( ) । ए॒व । अ॒ग्निम् । चि॒नु॒ते॒ । अ॒पाम् । पृ॒ष्ठम् । अ॒सि॒ । इति॑ । उपेति॑ । द॒धा॒ति॒ । अ॒पाम् । वै । ए॒तत् । पृ॒ष्ठम् । यत् । पु॒ष्क॒र॒प॒र्णमिति॑ पुष्कर - प॒र्णम् । रू॒पेण॑ । ए॒व । एन॒त्॒ । उपेति॑ । द॒धा॒ति॒ ॥  \newline




\markright{ TS 5.2.7.1  \hfill https://www.vedavms.in \hfill}

\section{ TS 5.2.7.1 }

\textbf{TS 5.2.7.1 } \newline
\textbf{Samhita Paata} \newline

ब्रह्म॑ जज्ञा॒नमिति॑ रु॒क्ममुप॑ दधाति॒ ब्रह्म॑मुखा॒ वै प्र॒जाप॑तिः प्र॒जा अ॑सृजत॒ ब्रह्म॑मुखा ए॒व तत् प्र॒जा यज॑मानः सृजते॒ ब्रह्म॑ जज्ञा॒नमित्या॑ह॒ तस्मा᳚द्ब्राह्म॒णो मुख्यो॒ मुख्यो॑ भवति॒ य ए॒वं ॅवेद॑ ब्रह्मवा॒दिनो॑ वदन्ति॒ न पृ॑थि॒व्यां नान्तरि॑क्षे॒ न दि॒व्य॑ग्निश्चे॑त॒व्य॑ इति॒ यत् पृ॑थि॒व्यां चि॑न्वी॒त पृ॑थि॒वीꣳ शु॒चाऽर्प॑ये॒न्नौष॑धयो॒ न वन॒स्पत॑यः॒ - [  ] \newline

\textbf{Pada Paata} \newline

ब्रह्म॑ । ज॒ज्ञा॒नम् । इति॑ । रु॒क्मम् । उपेति॑ । द॒धा॒ति॒ । ब्रह्म॑मुखा॒ इति॒ ब्रह्म॑ - मु॒खाः॒ । वै । प्र॒जाप॑ति॒रिति॑ प्र॒जा - प॒तिः॒ । प्र॒जा इति॑ प्र - जाः । अ॒सृ॒ज॒त॒ । ब्रह्म॑मुखा॒ इति॒ ब्रह्म॑ - मु॒खाः॒ । ए॒व । तत् । प्र॒जा इति॑ प्र - जाः । यज॑मानः । सृ॒ज॒ते॒ । ब्रह्म॑ । ज॒ज्ञा॒नम् । इति॑ । आ॒ह॒ । तस्मा᳚त् । ब्रा॒ह्म॒णः । मुख्यः॑ । मुख्यः॑ । भ॒व॒ति॒ । यः । ए॒वम् । वेद॑ । ब्र॒ह्म॒वा॒दिन॒ इति॑ ब्रह्म - वा॒दिनः॑ । व॒द॒न्ति॒ । न । पृ॒थि॒व्याम् । न । अ॒न्तरि॑क्षे । न । दि॒वि । अ॒ग्निः । चे॒त॒व्यः॑ । इति॑ । यत् । पृ॒थि॒व्याम् । चि॒न्वी॒त । पृ॒थि॒वीम् । शु॒चा । अ॒र्प॒ये॒त् । न । ओष॑धयः । न । वन॒स्पत॑यः ।  \newline




\markright{ TS 5.2.7.2  \hfill https://www.vedavms.in \hfill}

\section{ TS 5.2.7.2 }

\textbf{TS 5.2.7.2 } \newline
\textbf{Samhita Paata} \newline

प्र जा॑येर॒न॒. यद॒न्तरि॑क्षे चिन्वी॒तान्तरि॑क्षꣳ शु॒चाऽर्प॑ये॒न्न वयाꣳ॑सि॒ प्र जा॑येर॒न्॒. यद्-दि॒वि चि॑न्वी॒त दिवꣳ॑ शु॒चाऽर्प॑ये॒न्न प॒र्जन्यो॑ वर्.षेद्रु॒क्ममुप॑ दधात्य॒मृतं॒ ॅवै हिर॑ण्यम॒मृत॑ ए॒वाग्निं चि॑नुते॒ प्रजा᳚त्यै हिर॒ण्मयं॒ पुरु॑ष॒मुप॑ दधाति यजमानलो॒कस्य॒ विधृ॑त्यै॒ यदिष्ट॑काया॒ आतृ॑ण्णमनूपद॒द्ध्यात् प॑शू॒नां च॒ यज॑मानस्य च प्रा॒णमपि॑ दद्ध्याद् दक्षिण॒तः - [  ] \newline

\textbf{Pada Paata} \newline

प्रेति॑ । जा॒ये॒र॒न्न् । यत् । अ॒न्तरि॑क्षे । चि॒न्वी॒त । अ॒न्तरि॑क्षम् । शु॒चा । अ॒र्प॒ये॒त् । न । वयाꣳ॑सि । प्रेति॑ । जा॒ये॒र॒न्न् । यत् । दि॒वि । चि॒न्वी॒त । दिव᳚म् । शु॒चा । अ॒र्प॒ये॒त् । न । प॒र्जन्यः॑ । व॒र्॒.षे॒त् । रु॒क्मम् । उपेति॑ । द॒धा॒ति॒ । अ॒मृत᳚म् । वै । हिर॑ण्यम् । अ॒मृते᳚ । ए॒व । अ॒ग्निम् । चि॒नु॒ते॒ । प्रजा᳚त्या॒ इति॒ प्र - जा॒त्यै॒ । हि॒र॒ण्मय᳚म् । पुरु॑षम् । उपेति॑ । द॒धा॒ति॒ । य॒ज॒मा॒न॒लो॒कस्येति॑ यजमान - लो॒कस्य॑ । विधृ॑त्या॒ इति॒ वि-धृ॒त्यै॒ । यत् । इष्ट॑कायाः । आतृ॑ण्ण॒मित्या-तृ॒ण्ण॒म् । अ॒नू॒प॒द॒द्ध्यादित्य॑नु - उ॒प॒द॒द्ध्यात् । प॒शू॒नाम् । च॒ । यज॑मानस्य । च॒ । प्रा॒णमिति॑ प्र - अ॒नम् । अपीति॑ । द॒द्ध्या॒त् । द॒क्षि॒ण॒तः ।  \newline




\markright{ TS 5.2.7.3  \hfill https://www.vedavms.in \hfill}

\section{ TS 5.2.7.3 }

\textbf{TS 5.2.7.3 } \newline
\textbf{Samhita Paata} \newline

प्राञ्च॒मुप॑ दधाति दा॒धार॑ यजमानलो॒कं न प॑शू॒नां च॒ यज॑मानस्य च प्रा॒णमपि॑ दधा॒त्यथो॒ खल्विष्ट॑काया॒ आतृ॑ण्ण॒मनूप॑ दधाति प्रा॒णाना॒मुथ्सृ॑ष्ट्यै द्र॒फ्सश्च॑स्क॒न्देत्य॒भि मृ॑शति॒ होत्रा᳚स्वे॒वैनं॒ प्रति॑ष्ठापयति॒ स्रुचा॒वुप॑ दधा॒त्याज्य॑स्य पू॒र्णां का᳚र्ष्मर्य॒मयीं᳚ द॒द्ध्नः पू॒र्णा-मौदु॑बंरीमि॒यं ॅवै का᳚र्ष्मर्य॒मय्य॒सावौ-दु॑बंरी॒मे ए॒वोप॑ धत्ते - [  ] \newline

\textbf{Pada Paata} \newline

प्राञ्च᳚म् । उपेति॑ । द॒धा॒ति॒ । दा॒धार॑ । य॒ज॒मा॒न॒लो॒कमिति॑ यजमान - लो॒कम् । न । प॒शू॒नाम् । च॒ । यज॑मानस्य । च॒ । प्रा॒णमिति॑ प्र - अ॒नम् । अपीति॑ । द॒धा॒ति॒ । अथो॒ इति॑ । खलु॑ । इष्ट॑कायाः । आतृ॑ण्ण॒मित्या - तृ॒ण्ण॒म् । अनु॑ । उपेति॑ । द॒धा॒ति॒ । प्रा॒णाना॒मिति॑ प्र - अ॒नाना᳚म् । उथ्सृ॑ष्ट्या॒ इत्युत् - सृ॒ष्ट्यै॒ । द्र॒फ्सः । च॒स्क॒न्द॒ । इति॑ । अ॒भीति॑ । मृ॒श॒ति॒ । होत्रा॑सु । ए॒व । ए॒न॒म् । प्रतीति॑ । स्था॒प॒य॒ति॒ । स्रुचौ᳚ । उपेति॑ । द॒धा॒ति॒ । आज्य॑स्य । पू॒र्णाम् । का॒र्ष्म॒र्य॒मयी॒मिति॑ कार्ष्मर्य - मयी᳚म् । द॒द्ध्नः । पू॒र्णाम् । औदु॑बंरीम् । इ॒यम् । वै । का॒र्ष्म॒र्य॒मयीति॑ कार्ष्मर्य - मयी᳚ । अ॒सौ । औदु॑बंरी । इ॒मे इति॑ । ए॒व । उपेति॑ । ध॒त्ते॒ ।  \newline




\markright{ TS 5.2.7.4  \hfill https://www.vedavms.in \hfill}

\section{ TS 5.2.7.4 }

\textbf{TS 5.2.7.4 } \newline
\textbf{Samhita Paata} \newline

तू॒ष्णीमुप॑ दधाति॒ न हीमे यजु॒षाऽऽप्तु॒मर्.ह॑ति॒ दक्षि॑णां कार्ष्मर्य॒मयी॒-मुत्त॑रा॒मौ-दु॑म्बरीं॒ तस्मा॑द॒स्या अ॒सावुत्त॒रा ऽऽज्य॑स्य पू॒र्णां का᳚र्ष्मर्य॒मयीं॒ ॅवज्रो॒ वा आज्यं॒ ॅवज्रः॑ कार्ष्म॒र्यो॑ वज्रे॑णै॒व य॒ज्ञ्स्य॑ दक्षिण॒तो रक्षाꣳ॒॒स्यप॑ हन्ति द॒द्ध्नः पू॒र्णामौदु॑म्बरीं प॒शवो॒ वै दद्ध्यूर्गु॑दु॒म्बरः॑ प॒शुष्वे॒वोर्जं॑ दधाति पू॒र्णे उप॑ दधाति पू॒र्णे ए॒वैन॑ - [  ] \newline

\textbf{Pada Paata} \newline

तू॒ष्णीम् । उपेति॑ । द॒धा॒ति॒ । न । हि । इ॒मे इति॑ । यजु॑षा । आप्तु᳚म् । अर्.ह॑ति । दक्षि॑णाम् । का॒र्ष्म॒र्य॒मयी॒मिति॑ कार्ष्मर्य - मयी᳚म् । उत्त॑रा॒मित्युत् - त॒रा॒म् । औदु॑बंरीम् । तस्मा᳚त् । अ॒स्याः । अ॒सौ । उत्त॒रेत्युत् - त॒रा॒ । आज्य॑स्य । पू॒र्णाम् । का॒र्ष्म॒र्य॒मयी॒मिति॑ कार्ष्मर्य - मयी᳚म् । वज्रः॑ । वै । आज्य᳚म् । वज्रः॑ । का॒र्ष्म॒र्यः॑ । वज्रे॑ण । ए॒व । य॒ज्ञ्स्य॑ । द॒क्षि॒ण॒तः । रक्षाꣳ॑सि । अपेति॑ । ह॒न्ति॒ । द॒द्ध्नः । पू॒र्णाम् । औदु॑बंरीम् । प॒शवः॑ । वै । दधि॑ । ऊर्क् । उ॒दु॒बंरः॑ । प॒शुषु॑ । ए॒व । ऊर्ज᳚म् । द॒धा॒ति॒ । पू॒र्णे इति॑ । उपेति॑ । द॒धा॒ति॒ । पू॒र्णे इति॑ । ए॒व । ए॒न॒म् ।  \newline




\markright{ TS 5.2.7.5  \hfill https://www.vedavms.in \hfill}

\section{ TS 5.2.7.5 }

\textbf{TS 5.2.7.5 } \newline
\textbf{Samhita Paata} \newline

म॒मुष्मि॑न् ॅलो॒क उप॑तिष्ठेते वि॒राज्य॒ग्निश्चे॑त॒व्य॑ इत्या॑हुः॒ स्रुग्वै वि॒राड्यथ् स्रुचा॑वुप॒दधा॑ति वि॒राज्ये॒वाग्निं चि॑नुते यज्ञ्मु॒खेय॑ज्ञ्मुखे॒ वै क्रि॒यमा॑णे य॒ज्ञ्ꣳ रक्षाꣳ॑सि जिघाꣳसन्ति यज्ञ्मु॒खꣳ रु॒क्मो यद्-रु॒क्मं ॅव्या॑घा॒रय॑ति यज्ञ्मु॒खादे॒व रक्षाꣳ॒॒स्यप॑ हन्ति प॒ञ्चभि॒व्या घा॑रयति॒ पाङ्क्तो॑ य॒ज्ञो यावा॑ने॒व य॒ज्ञ्स्तस्मा॒द्-रक्षाꣳ॒॒स्यप॑ हन्त्यक्ष्ण॒याव्या घा॑रयति॒ तस्मा॑दक्ष्ण॒या ( ) प॒शवोऽङ्गा॑नि॒ प्र ह॑रन्ति॒ प्रति॑ष्ठित्यै ॥ \newline

\textbf{Pada Paata} \newline

अ॒मुष्मिन्न्॑ । लो॒के । उपेति॑ । ति॒ष्ठे॒ते॒ इति॑ । वि॒राजीति॑ वि - राजि॑ । अ॒ग्निः । चे॒त॒व्यः॑ । इति॑ । आ॒हुः॒ । स्रुक् । वै । वि॒राडिति॑ वि-राट् । यत् । स्रुचौ᳚ । उ॒प॒दधा॒तीत्यु॑प - दधा॑ति । वि॒राजीति॑ वि - राजि॑ । ए॒व । अ॒ग्निम् । चि॒नु॒ते॒ । य॒ज्ञ्॒मु॒खे य॑ज्ञ्मुख॒ इति॑ यज्ञ्मु॒खे-य॒ज्ञ्॒मु॒खे॒ । वै । क्रि॒यमा॑णे । य॒ज्ञ्म् । रक्षाꣳ॑सि । जि॒घाꣳ॒॒स॒न्ति॒ । य॒ज्ञ्॒मु॒खमिति॑ यज्ञ् - मु॒खम् । रु॒क्मः । यत् । रु॒क्मम् । व्या॒घा॒रय॒तीति॑ वि - आ॒घा॒रय॑ति । य॒ज्ञ्॒मु॒खादिति॑ यज्ञ् - मु॒खात् । ए॒व । रक्षाꣳ॑सि । अपेति॑ । ह॒न्ति॒ । प॒ञ्चभि॒रिति॑ प॒ञ्च - भिः॒ । व्याघा॑रय॒तीति॑ वि - आघा॑रयति । पाङ्क्तः॑ । य॒ज्ञ्ः । यावान्॑ । ए॒व । य॒ज्ञ्ः । तस्मा᳚त् । रक्षाꣳ॑सि । अपेति॑ । ह॒न्ति॒ । अ॒क्ष्ण॒या । व्याघा॑रय॒तीति॑ वि - आघा॑रयति । तस्मा᳚त् । अ॒क्ष्ण॒या ( ) । प॒शवः॑ । अङ्गा॑नि । प्रेति॑ । ह॒र॒न्ति॒ । प्रति॑ष्ठित्या॒ इति॒ प्रति॑ - स्थि॒त्यै॒ ॥  \newline




\markright{ TS 5.2.8.1  \hfill https://www.vedavms.in \hfill}

\section{ TS 5.2.8.1 }

\textbf{TS 5.2.8.1 } \newline
\textbf{Samhita Paata} \newline

स्व॒य॒मा॒तृ॒ण्णामुप॑ दधाती॒यं ॅवै स्व॑यमातृ॒ण्णेमामे॒वोप॑ ध॒त्ते ऽश्व॒मुप॑ घ्रापयति प्रा॒णमे॒वास्यां᳚ दधा॒त्यथो᳚ प्राजाप॒त्यो वा अश्वः॑ प्र॒जाप॑तिनै॒वाऽग्निं चि॑नुते प्रथ॒मेष्ट॑कोपधी॒यमा॑ना पशू॒नां च॒ यज॑मानस्य च प्रा॒णमपि॑ दधाति स्वयमातृ॒ण्णा भ॑वति प्रा॒णाना॒मुथ्सृ॑ष्ट्या॒ अथो॑ सुव॒र्गस्य॑ लो॒कस्यानु॑ख्यात्या अ॒ग्नाव॒ग्निश्चे॑त॒व्य॑ इत्या॑हुरे॒ष वा - [  ] \newline

\textbf{Pada Paata} \newline

स्व॒य॒मा॒तृ॒ण्णामिति॑ स्वयं - आ॒तृ॒ण्णाम् । उपेति॑ । द॒धा॒ति॒ । इ॒यम् । वै । स्व॒य॒मा॒तृ॒ण्णेति॑ स्वयं-आ॒तृ॒ण्णा । इ॒माम् । ए॒व । उपेति॑ । ध॒त्ते॒ । अश्व᳚म् । उपेति॑ । घ्रा॒प॒य॒ति॒ । प्रा॒णमिति॑ प्र-अ॒नम् । ए॒व । अ॒स्या॒म् । द॒धा॒ति॒ । अथो॒ इति॑ । प्रा॒जा॒प॒त्य इति॑ प्रजा - प॒त्यः । वै । अश्वः॑ । प्र॒जाप॑ति॒नेति॑ प्र॒जा - प॒ति॒ना॒ । ए॒व । अ॒ग्निम् । चि॒नु॒ते॒ । प्र॒थ॒मा । इष्ट॑का । उ॒प॒धी॒यमा॒नेत्यु॑प-धी॒यमा॑ना । प॒शू॒नाम् । च॒ । यज॑मानस्य । च॒ । प्रा॒णमिति॑ प्र - अ॒नम् । अपीति॑ । द॒धा॒ति॒ । स्व॒य॒मा॒तृ॒ण्णेति॑ स्वयं - आ॒तृ॒ण्णा । भ॒व॒ति॒ । प्रा॒णाना॒मिति॑ प्र - अ॒नाना᳚म् । उथ्सृ॑ष्ट्या॒ इत्युत् - सृ॒ष्ट्यै॒ । अथो॒ इति॑ । सु॒व॒र्गस्येति॑ सुवः-गस्य॑ । लो॒कस्य॑ । अनु॑ख्यात्या॒ इत्यनु॑ - ख्या॒त्यै॒ । अ॒ग्नौ । अ॒ग्निः । चे॒त॒व्यः॑ । इति॑ । आ॒हुः॒ । ए॒षः । वै ।  \newline




\markright{ TS 5.2.8.2  \hfill https://www.vedavms.in \hfill}

\section{ TS 5.2.8.2 }

\textbf{TS 5.2.8.2 } \newline
\textbf{Samhita Paata} \newline

अ॒ग्निर्वै᳚श्वान॒रो यद्ब्रा᳚ह्म॒णस्तस्मै᳚ प्रथ॒मामिष्ट॑कां॒ ॅयजु॑ष्कृतां॒ प्रय॑च्छे॒त् तां ब्रा᳚ह्म॒णश्चोप॑ दद्ध्याता-म॒ग्नावे॒व तद॒ग्निं चि॑नुत ईश्व॒रो वा ए॒ष आर्ति॒मार्तो॒र्यो-ऽवि॑द्वा॒निष्ट॑का-मुप॒दधा॑ति॒ त्रीन्. वरा᳚न् दद्या॒त् त्रयो॒ वै प्रा॒णाः प्रा॒णानाꣳ॒॒ स्पृत्यै॒ द्वावे॒व देयौ॒ द्वौ हि प्रा॒णावेक॑ ए॒व देय॒ एको॒ हि प्रा॒णः प॒शु - [  ] \newline

\textbf{Pada Paata} \newline

अ॒ग्निः । वै॒श्वा॒न॒रः । यत् । ब्रा॒ह्म॒णः । तस्मै᳚ । प्र॒थ॒माम् । इष्ट॑काम् । यजु॑ष्कृता॒मिति॒ यजुः॑ - कृ॒ता॒म् । प्रेति॑ । य॒च्छे॒त् । ताम् । ब्रा॒ह्म॒णः । च॒ । उपेति॑ । द॒द्ध्या॒ता॒म् । अ॒ग्नौ । ए॒व । तत् । अ॒ग्निम् । चि॒नु॒ते॒ । ई॒श्व॒रः । वै । ए॒षः । आर्ति᳚म् । आर्तो॒रित्या-अ॒र्तोः॒ । यः । अवि॑द्वान् । इष्ट॑काम् । उ॒प॒द॒धा॒तीत्यु॑प - दधा॑ति । त्रीन् । वरान्॑ । द॒द्या॒त् । त्रयः॑ । वै । प्रा॒णा इति॑ प्र - अ॒नाः । प्रा॒णाना॒मिति॑ प्र - अ॒नाना᳚म् । स्पृत्यै᳚ । द्वौ । ए॒व । देयौ᳚ । द्वौ । हि । प्रा॒णाविति॑ प्र - अ॒नौ । एकः॑ । ए॒व । देयः॑ । एकः॑ । हि । प्रा॒ण इति॑ प्र - अ॒नः । प॒शुः ।  \newline




\markright{ TS 5.2.8.3  \hfill https://www.vedavms.in \hfill}

\section{ TS 5.2.8.3 }

\textbf{TS 5.2.8.3 } \newline
\textbf{Samhita Paata} \newline

-र्वा ए॒ष यद॒ग्निर्न खलु॒ वै प॒शव॒ आय॑वसे रमन्ते दूर्वेष्ट॒कामुप॑ दधाति पशू॒नां धृत्यै॒ द्वाभ्यां॒ प्रति॑ष्ठित्यै॒ काण्डा᳚त् काण्डात् प्र॒रोह॒न्तीत्या॑ह॒ काण्डे॑नकाण्डेन॒ ह्ये॑षा प्र॑ति॒तिष्ठ॑त्ये॒वा नो॑ दूर्वे॒ प्रत॑नु स॒हस्रे॑ण श॒तेन॒ चेत्या॑ह साह॒स्रः प्र॒जाप॑तिः प्र॒जाप॑ते॒राप्त्यै॑ देवल॒क्ष्मं ॅवै त्र्या॑लिखि॒ता तामुत्त॑रलक्ष्माणं दे॒वा उपा॑दध॒ता-ध॑रलक्ष्माण॒-मसु॑रा॒ यं - [  ] \newline

\textbf{Pada Paata} \newline

वै । ए॒षः । यत् । अ॒ग्निः । न । खलु॑ । वै । प॒शवः॑ । आय॑वसे । र॒म॒न्ते॒ । दू॒र्वे॒ष्ट॒कामिति॑ दूर्वा - इ॒ष्ट॒काम् । उपेति॑ । द॒धा॒ति॒ । प॒शू॒नाम् । धृत्यै᳚ । द्वाभ्या᳚म् । प्रति॑ष्ठित्या॒ इति॒ प्रति॑ - स्थि॒त्यै॒ । काण्डा᳚त्काण्डा॒दिति॒ काण्डा᳚त् - का॒ण्डा॒त् । प्र॒रोह॒न्तीति॑ प्र-रोह॑न्ती । इति॑ । आ॒ह॒ । काण्डे॑नकाण्डे॒नेति॒ काण्डे॑न - का॒ण्डे॒न॒ । हि । ए॒षा । प्र॒ति॒तिष्ठ॒तीति॑ प्रति - तिष्ठ॑ति । ए॒वा । नः॒ । दू॒र्वे॒ । प्रेति॑ । त॒नु॒ । स॒हस्रे॑ण । श॒तेन॑ । च॒ । इति॑ । आ॒ह॒ । सा॒ह॒स्रः । प्र॒जाप॑ति॒रिति॑ प्र॒जा - प॒तिः॒ । प्र॒जाप॑ते॒रिति॑ प्र॒जा - प॒तेः॒ । आप्त्यै᳚ । दे॒व॒ल॒क्ष्ममिति॑ देव-ल॒क्ष्मम् । वै । त्र्या॒लि॒खि॒तेति॑ त्रि - आ॒लि॒खि॒ता । ताम् । उत्त॑रलक्ष्माण॒मित्युत्त॑र - ल॒क्ष्मा॒ण॒म् । दे॒वाः । उपेति॑ । अ॒द॒ध॒त॒ । अध॑रलक्ष्माण॒मित्यध॑र - ल॒क्ष्मा॒ण॒म् । असु॑राः । यम् ।  \newline




\markright{ TS 5.2.8.4  \hfill https://www.vedavms.in \hfill}

\section{ TS 5.2.8.4 }

\textbf{TS 5.2.8.4 } \newline
\textbf{Samhita Paata} \newline

का॒मये॑त॒ वसी॑यान्थ् स्या॒दित्युत्त॑रलक्ष्माणं॒ तस्योप॑ दद्ध्या॒द्-वसी॑याने॒व भ॑वति॒ यं का॒मये॑त॒ पापी॑यान्थ् स्या॒दित्यध॑रलक्ष्माणं॒ तस्योप॑ दद्ध्यादसुरयो॒नि-मे॒वैन॒मनु॒ परा॑ भावयति॒ पापी॑यान् भवति त्र्यालिखि॒ता भ॑वती॒मे वै लो॒कास्त्र्या॑लिखि॒तैभ्य ए॒व लो॒केभ्यो॒ भ्रातृ॑व्यम॒न्तरे॒त्यङ्गि॑रसः सुव॒र्गं ॅलो॒कं ॅय॒तः पु॑रो॒डाशः॑ कू॒र्मो भू॒त्वाऽनु॒ प्रास॑र्प॒ - [  ] \newline

\textbf{Pada Paata} \newline

का॒मये॑त । वसी॑यान् । स्या॒त् । इति॑ । उत्त॑रलक्ष्माण॒मित्युत्त॑र-ल॒क्ष्मा॒ण॒म् । तस्य॑ । उपेति॑ । द॒द्ध्या॒त् । वसी॑यान् । ए॒व । भ॒व॒ति॒ । यम् । का॒मये॑त । पापी॑यान् । स्या॒त् । इति॑ । अध॑रलक्ष्माण॒मित्यध॑र- ल॒क्ष्मा॒ण॒म् । तस्य॑ । उपेति॑ । द॒द्ध्या॒त् । अ॒सु॒र॒यो॒निमित्य॑सुर - यो॒निम् । ए॒व । ए॒न॒म् । अनु॑ । परेति॑ । भा॒व॒य॒ति॒ । पापी॑यान् । भ॒व॒ति॒ । त्र्या॒लि॒खि॒तेति॑ त्रि - आ॒लि॒खि॒ता । भ॒व॒ति॒ । इ॒मे । वै । लो॒काः । त्र्या॒लि॒खि॒तेति॑ त्रि - आ॒लि॒खि॒ता । ए॒भ्यः । ए॒व । लो॒केभ्यः॑ । भ्रातृ॑व्यम् । अ॒न्तः । ए॒ति॒ । अङ्गि॑रसः । सु॒व॒र्गमिति॑ सुवः - गम् । लो॒कम् । य॒तः । पु॒रो॒डाशः॑ । कू॒र्मः । भू॒त्वा । अनु॑ । प्रेति॑ । अ॒स॒र्प॒त् ।  \newline




\markright{ TS 5.2.8.5  \hfill https://www.vedavms.in \hfill}

\section{ TS 5.2.8.5 }

\textbf{TS 5.2.8.5 } \newline
\textbf{Samhita Paata} \newline

-द्यत् कू॒र्ममु॑प॒दधा॑ति॒ यथा᳚ क्षेत्र॒विदञ्ज॑सा॒ नय॑त्ये॒वमे॒वैनं॑ कू॒र्मः सु॑व॒र्गं ॅलो॒कमञ्ज॑सा नयति॒ मेधो॒ वा ए॒ष प॑शू॒नां ॅयत् कू॒र्मो यत् कू॒र्ममु॑प॒ दधा॑ति॒ स्वमे॒व मेधं॒ पश्य॑न्तः प॒शव॒ उप॑ तिष्ठन्ते श्मशा॒नं ॅवा ए॒तत् क्रि॑यते॒ यन्मृ॒तानां᳚ पशू॒नाꣳ शी॒र्॒.षाण्यु॑पधी॒यन्ते॒ यज्जीव॑न्तं कू॒र्ममु॑प॒ दधा॑ति॒ तेनाश्म॑शानचिद्वास्त॒व्यो॑ वा ए॒ष यत् - [  ] \newline

\textbf{Pada Paata} \newline

यत् । कू॒र्मम् । उ॒प॒दधा॒तीत्यु॑प - दधा॑ति । यथा᳚ । क्षे॒त्र॒विदिति॑ क्षेत्र - वित् । अञ्ज॑सा । नय॑ति । ए॒वम् । ए॒व । ए॒न॒म् । कू॒र्मः । सु॒व॒र्गमिति॑ सुवः - गम् । लो॒कम् । अञ्ज॑सा । न॒य॒ति॒ । मेधः॑ । वै । ए॒षः । प॒शू॒नाम् । यत् । कू॒र्मः । यत् । कू॒र्मम् । उ॒प॒दधा॒तीत्यु॑प - दधा॑ति । स्वम् । ए॒व । मेध᳚म् । पश्य॑न्तः । प॒शवः॑ । उपेति॑ । ति॒ष्ठ॒न्ते॒ । श्म॒शा॒नम् । वै । ए॒तत् । क्रि॒य॒ते॒ । यत् । मृ॒ताना᳚म् । प॒शू॒नाम् । शी॒र्॒.षाणि॑ । उ॒प॒धी॒यन्त॒ इत्यु॑प - धी॒यन्ते᳚ । यत् । जीव॑न्तम् । कू॒र्मम् । उ॒प॒दधा॒तीत्यु॑प - दधा॑ति । तेन॑ । अश्म॑शानचि॒दित्यश्म॑शान - चि॒त् । वा॒स्त॒व्यः॑ । वै । ए॒षः । यत् ।  \newline




\markright{ TS 5.2.8.6  \hfill https://www.vedavms.in \hfill}

\section{ TS 5.2.8.6 }

\textbf{TS 5.2.8.6 } \newline
\textbf{Samhita Paata} \newline

कू॒र्मो मधु॒ वाता॑ ऋताय॒त इति॑ द॒द्ध्ना म॑धुमि॒श्रेणा॒भ्य॑नक्ति स्व॒दय॑त्ये॒वैनं॑ ग्रा॒म्यं ॅवा ए॒तदन्नं॒ ॅयद्-दद्ध्या॑र॒ण्यं मधु॒ यद्द॒द्ध्ना म॑धुमि॒श्रेणा᳚ भ्य॒नक्त्यु॒भय॒स्या ऽव॑रुद्ध्यै म॒ही द्यौः पृ॑थि॒वी च॑ न॒ इत्या॑हा॒ ऽऽभ्यामे॒वैन॑मुभ॒यतः॒ परि॑गृह्णाति॒ प्राञ्च॒मुप॑ दधाति सुव॒र्गस्य॑ लो॒कस्य॒ सम॑ष्ट्यै पु॒रस्ता᳚त् प्र॒त्यञ्च॒मुप॑ दधाति॒ तस्मा᳚त् - [  ] \newline

\textbf{Pada Paata} \newline

कू॒र्मः । मधु॑ । वाताः᳚ । ऋ॒ता॒य॒त इत्यृ॑त - य॒ते । इति॑ । द॒द्ध्ना । म॒धु॒मि॒श्रेणेति॑ मधु - मि॒श्रेण॑ । अ॒भीति॑ । अ॒न॒क्ति॒ । स्व॒दय॑ति । ए॒व । ए॒न॒म् । ग्रा॒म्यम् । वै । ए॒तत् । अन्न᳚म् । यत् । दधि॑ । आ॒र॒ण्यम् । मधु॑ । यत् । द॒द्ध्ना । म॒धु॒मि॒श्रेणेति॑ मधु - मि॒श्रेण॑ । अ॒भ्य॒नक्तीत्य॑भि - अ॒नक्ति॑ । उ॒भय॑स्य । अव॑रुद्ध्या॒ इत्यव॑-रु॒द्ध्यै॒ । म॒ही । द्यौः । पृ॒थि॒वी । च॒ । नः॒ । इति॑ । आ॒ह॒ । आ॒भ्याम् । ए॒व । ए॒न॒म् । उ॒भ॒यतः॑ । परीति॑ । गृ॒ह्णा॒ति॒ । प्राञ्च᳚म् । उपेति॑ । द॒धा॒ति॒ । सु॒व॒र्गस्येति॑ सुवः - गस्य॑ । लो॒कस्य॑ । सम॑ष्ट्या॒ इति॒ सं-अ॒ष्ट्यै॒ । पु॒रस्ता᳚त् । प्र॒त्यञ्च᳚म् । उपेति॑ । द॒धा॒ति॒ । तस्मा᳚त् ।  \newline




\markright{ TS 5.2.8.7  \hfill https://www.vedavms.in \hfill}

\section{ TS 5.2.8.7 }

\textbf{TS 5.2.8.7 } \newline
\textbf{Samhita Paata} \newline

पु॒रस्ता᳚त् प्र॒त्यञ्चः॑ प॒शवो॒ मेध॒मुप॑ तिष्ठन्ते॒ यो वा अप॑नाभिम॒ग्निं चि॑नु॒ते यज॑मानस्य॒ नाभि॒मनु॒ प्रवि॑शति॒ स ए॑नमीश्व॒रो हिꣳसि॑तोरु॒लूख॑ल॒मुप॑ दधात्ये॒षा वा अ॒ग्नेर्नाभिः॒ सना॑भिमे॒वाऽग्निं चि॑नु॒ते हिꣳ॑साया॒ औदु॑म्बरं भव॒त्यूर्ग्वा उ॑दु॒म्बर॒ ऊर्ज॑मे॒वाव॑ रुन्धे मद्ध्य॒त उप॑ दधाति मद्ध्य॒त ए॒वास्मा॒ ऊर्जं॑ दधाति॒ तस्मा᳚न्-( )-मद्ध्य॒त ऊ॒र्जा भु॑ञ्जत॒ इय॑द्-भवति प्र॒जाप॑तिना यज्ञ्मु॒खेन॒ संमि॑त॒मव॑ ह॒न्त्यन्न॑मे॒वाक॑-र्वैष्ण॒व्यर्चोप॑ दधाति॒ विष्णु॒र्वै य॒ज्ञो वै᳚ष्ण॒वा वन॒स्पत॑यो य॒ज्ञ् ए॒व य॒ज्ञ्ं प्रति॑ष्ठापयति ॥ \newline

\textbf{Pada Paata} \newline

पु॒रस्ता᳚त् । प्र॒त्यञ्चः॑ । प॒शवः॑ । मेध᳚म् । उपेति॑ । ति॒ष्ठ॒न्ते॒ । यः । वै । अप॑नाभि॒मित्यप॑ - ना॒भि॒म् । अ॒ग्निम् । चि॒नु॒ते । यज॑मानस्य । नाभि᳚म् । अनु॑ । प्रेति॑ । वि॒श॒ति॒ । सः । ए॒न॒म् । ई॒श्व॒रः । हिꣳसि॑तोः । उ॒लूख॑लम् । उपेति॑ । द॒धा॒ति॒ । ए॒षा । वै । अ॒ग्नेः । नाभिः॑ । सना॑भि॒मिति॒ स - ना॒भि॒म् । ए॒व । अ॒ग्निम् । चि॒नु॒त॒ । अहिꣳ॑सायै । औदु॑बंरम् । भ॒व॒ति॒ । ऊर्क् । वै । उ॒दु॒बंरः॑ । ऊर्ज᳚म् । ए॒व । अवेति॑ । रु॒न्धे॒ । म॒द्ध्य॒तः । उपेति॑ । द॒धा॒ति॒ । म॒द्ध्य॒तः । ए॒व । अ॒स्मै॒ । ऊर्ज᳚म् । द॒धा॒ति॒ । तस्मा᳚त् ( ) । म॒द्ध्य॒तः । ऊ॒र्जा । भु॒ञ्ज॒ते॒ । इय॑त् । भ॒व॒ति॒ । प्र॒जाप॑ति॒नेति॑ प्र॒जा - प॒ति॒ना॒ । य॒ज्ञ्॒मु॒खेनेति॑ यज्ञ् - मु॒खेन॑ । सम्मि॑त॒मिति॒ सं - मि॒त॒म् । अवेति॑ । ह॒न्ति॒ । अन्न᳚म् । ए॒व । अ॒कः॒ । वै॒ष्ण॒व्या । ऋ॒चा । उपेति॑ । द॒धा॒ति॒ । विष्णुः॑ । वै । य॒ज्ञ्ः । वै॒ष्ण॒वाः । वन॒स्पत॑यः । य॒ज्ञे । ए॒व । य॒ज्ञ्म् । प्रतीति॑ । स्था॒प॒य॒ति॒ ॥  \newline




\markright{ TS 5.2.9.1  \hfill https://www.vedavms.in \hfill}

\section{ TS 5.2.9.1 }

\textbf{TS 5.2.9.1 } \newline
\textbf{Samhita Paata} \newline

ए॒षां ॅवा ए॒तल्लो॒कानां॒ ज्योतिः॒ सम्भृ॑तं॒ ॅयदु॒खा यदु॒खा-मु॑प॒दधा᳚त्ये॒भ्य ए॒व लो॒केभ्यो॒ ज्योति॒रव॑ रुन्धे मद्ध्य॒त उप॑ दधाति मद्ध्य॒त ए॒वास्मै॒ ज्योति॑र्दधाति॒ तस्मा᳚न्मद्ध्य॒तो ज्योति॒रुपा᳚ऽऽस्महे॒ सिक॑ताभिः पूरयत्ये॒तद्वा अ॒ग्नेर्वै᳚श्वान॒रस्य॑ रू॒पꣳ रू॒पेणै॒व वै᳚श्वान॒रमव॑ रुन्धे॒ यं का॒मये॑त॒ क्षोधु॑कः स्या॒दित्यू॒नां तस्योप॑ - [  ] \newline

\textbf{Pada Paata} \newline

ए॒षाम् । वै । ए॒तत् । लो॒काना᳚म् । ज्योतिः॑ । संभृ॑त॒मिति॒ सं-भृ॒त॒म् । यत् । उ॒खा । यत् । उ॒खाम् । उ॒प॒दधा॒तीत्यु॑प-दधा॑ति । ए॒भ्यः । ए॒व । लो॒केभ्यः॑ । ज्योतिः॑ । अवेति॑ । रु॒न्धे॒ । म॒द्ध्य॒तः । उपेति॑ । द॒धा॒ति॒ । म॒द्ध्य॒तः । ए॒व । अ॒स्मै॒ । ज्योतिः॑ । द॒धा॒ति॒ । तस्मा᳚त् । म॒द्ध्य॒तः । ज्योतिः॑ । उपेति॑ । आ॒स्म॒हे॒ । सिक॑ताभिः । पू॒र॒य॒ति॒ । ए॒तत् । वै । अ॒ग्नेः । वै॒श्वा॒न॒रस्य॑ । रू॒पम् । रू॒पेण॑ । ए॒व । वै॒श्वा॒न॒रम् । अवेति॑ । रु॒न्धे॒ । यम् । का॒मये॑त । क्षोधु॑कः । स्या॒त् । इति॑ । ऊ॒नाम् । तस्य॑ । उपेति॑ ।  \newline




\markright{ TS 5.2.9.2  \hfill https://www.vedavms.in \hfill}

\section{ TS 5.2.9.2 }

\textbf{TS 5.2.9.2 } \newline
\textbf{Samhita Paata} \newline

दद्ध्या॒त् क्षोधु॑क ए॒व भ॑वति॒ यं का॒मये॒ता-नु॑पदस्य॒-दन्न॑मद्या॒दिति॑ पू॒र्णां तस्योप॑ दद्ध्या॒दनु॑पदस्य-दे॒वान्न॑मत्ति स॒हस्रं॒ ॅवै प्रति॒ पुरु॑षः पशू॒नां ॅय॑च्छति स॒हस्र॑म॒न्ये प॒शवो॒ मद्ध्ये॑ पुरुषशी॒र्॒.षमुप॑ दधाति सवीर्य॒त्वायो॒-खाया॒मपि॑ दधाति प्रति॒ष्ठामे॒वैन॑द्-गमयति॒ व्यृ॑द्धं॒ ॅवा ए॒तत् प्रा॒णैर॑मे॒द्ध्यं ॅयत् पु॑रुषशी॒र्.षम॒मृतं॒ खलु॒ वै प्रा॒णा - [  ] \newline

\textbf{Pada Paata} \newline

द॒द्ध्या॒त् । क्षोधु॑कः । ए॒व । भ॒व॒ति॒ । यम् । का॒मये॑त । अनु॑पदस्य॒दित्यनु॑प - द॒स्य॒त् । अन्न᳚म् । अ॒द्या॒त् । इति॑ । पू॒र्णाम् । तस्य॑ । उपेति॑ । द॒द्ध्या॒त् । अनु॑पदस्य॒दित्यनु॑प - द॒स्य॒त् । ए॒व । अन्न᳚म् । अ॒त्ति॒ । स॒हस्र᳚म् । वै । प्रतीति॑ । पुरु॑षः । प॒शू॒नाम् । य॒च्छ॒ति॒ । स॒हस्र᳚म् । अ॒न्ये । प॒शवः॑ । मद्ध्ये᳚ । पु॒रु॒ष॒शी॒र्॒.षमिति॑ पुरुष - शी॒र्॒.षम् । उपेति॑ । द॒धा॒ति॒ । स॒वी॒र्य॒त्वायेति॑ सवीर्य-त्वाय॑ । उ॒खाया᳚म् । अपीति॑ । द॒धा॒ति॒ । प्र॒ति॒ष्ठामिति॑ प्रति - स्थाम् । ए॒व । ए॒न॒त् । ग॒म॒य॒ति॒ । व्यृ॑द्ध॒मिति॒ वि - ऋ॒द्ध॒म् । वै । ए॒तत् । प्रा॒णैरिति॑ प्र - अ॒नैः । अ॒मे॒द्ध्यम् । यत् । पु॒रु॒ष॒शी॒र्॒.षमिति॑ पुरुष - शी॒र्॒.षम् । अ॒मृत᳚म् । खलु॑ । वै । प्रा॒णा इति॑ प्र - अ॒नाः ।  \newline




\markright{ TS 5.2.9.3  \hfill https://www.vedavms.in \hfill}

\section{ TS 5.2.9.3 }

\textbf{TS 5.2.9.3 } \newline
\textbf{Samhita Paata} \newline

अ॒मृतꣳ॒॒ हिर॑ण्यं प्रा॒णेषु॑ हिरण्यश॒ल्कान् प्रत्य॑स्यति प्रति॒ष्ठामे॒वैन॑द्-गमयि॒त्वा प्रा॒णैः सम॑र्द्धयति द॒द्ध्ना म॑धुमि॒श्रेण॑ पूरयति मध॒व्यो॑ऽसा॒नीति॑ शृतात॒ङ्क्ये॑न मेद्ध्य॒त्वाय॑ ग्रा॒म्यं ॅवा ए॒तदन्नं॒ ॅयद्-दद्ध्या॑र॒ण्यं मधु॒ यद्द॒द्ध्ना म॑धुमि॒श्रेण॑ पू॒रय॑त्यु॒भय॒स्या-व॑रुद्ध्यै पशुशी॒र्॒.षाण्युप॑ दधाति प॒शवो॒ वै प॑शुशी॒र्॒.षाणि॑ प॒शूने॒वाव॑ रुन्धे॒ यं का॒मये॑ताप॒शुः स्या॒दिति॑ - [  ] \newline

\textbf{Pada Paata} \newline

अ॒मृत᳚म् । हिर॑ण्यम् । प्रा॒णेष्विति॑ प्र - अ॒नेषु॑ । हि॒र॒ण्य॒श॒ल्कानिति॑ हिरण्य - श॒ल्कान् । प्रतीति॑ । अ॒स्य॒ति॒ । प्र॒ति॒ष्ठामिति॑ प्रति-स्थाम् । ए॒व । ए॒न॒त् । ग॒म॒यि॒त्वा । प्रा॒णैरिति॑ प्र-अ॒नैः । समिति॑ । अ॒द्‌र्ध॒य॒ति॒ । द॒द्ध्ना । म॒धु॒मि॒श्रेणेति॑ मधु-मि॒श्रेण॑ । पू॒र॒य॒ति॒ । म॒ध॒व्यः॑ । अ॒सा॒नि॒ । इति॑ । शृ॒ता॒त॒ङ्क्ये॑नेति॑ शृत - आ॒त॒ङ्क्ये॑न । मे॒द्ध्य॒त्वायेति॑ मेद्ध्य - त्वाय॑ । ग्रा॒म्यम् । वै । ए॒तत् । अन्न᳚म् । यत् । दधि॑ । आ॒र॒ण्यम् । मधु॑ । यत् । द॒द्ध्ना । म॒धु॒मि॒श्रेणेति॑ मधु - मि॒श्रेण॑ । पू॒रय॑ति । उ॒भय॑स्य । अव॑रुद्ध्या॒ इत्यव॑ - रु॒द्ध्यै॒ । प॒शु॒शी॒र्॒.षाणीति॑ पशु - शी॒र्॒.षाणि॑ । उपेति॑ । द॒धा॒ति॒ । प॒शवः॑ । वै । प॒शु॒शी॒र्॒.षाणीति॑ पशु - शी॒र्॒.षाणि॑ । प॒शून् । ए॒व । अवेति॑ । रु॒न्धे॒ । यम् । का॒मये॑त । अ॒प॒शुः । स्या॒त् । इति॑ ।  \newline




\markright{ TS 5.2.9.4  \hfill https://www.vedavms.in \hfill}

\section{ TS 5.2.9.4 }

\textbf{TS 5.2.9.4 } \newline
\textbf{Samhita Paata} \newline

विषू॒चीना॑नि॒ तस्योप॑ दद्ध्या॒द्-विषू॑च ए॒वास्मा᳚त् प॒शून् द॑धात्यप॒शुरे॒व भ॑वति॒ यं का॒मये॑त पशु॒मान्थ्-स्या॒दिति॑ समी॒चीना॑नि॒ तस्योप॑ दद्ध्याथ् स॒मीच॑ ए॒वास्मै॑ प॒शून् द॑धाति पशु॒माने॒व भ॑वति पु॒रस्ता᳚त् प्रती॒चीन॒मश्व॒स्योप॑ दधाति प॒श्चात् प्रा॒चीन॑मृष॒भस्या-प॑शवो॒ वा अ॒न्ये गो॑ अ॒श्वेभ्यः॑ प॒शवो॑ गो अ॒श्वाने॒वास्मै॑ स॒मीचो॑ दधात्ये॒-ताव॑न्तो॒ वै प॒शवो᳚ - [  ] \newline

\textbf{Pada Paata} \newline

वि॒षू॒चीना॑नि । तस्य॑ । उपेति॑ । द॒द्ध्या॒त् । विषू॑चः । ए॒व । अ॒स्मा॒त् । प॒शून् । द॒धा॒ति॒ । अ॒प॒शुः । ए॒व । भ॒व॒ति॒ । यम् । का॒मये॑त । प॒शु॒मानिति॑ पशु - मान् । स्या॒त् । इति॑ । स॒मी॒चीना॑नि । तस्य॑ । उपेति॑ । द॒द्ध्या॒त् । स॒मीचः॑ । ए॒व । अ॒स्मै॒ । प॒शून् । द॒धा॒ति॒ । प॒शु॒मानिति॑ पशु - मान् । ए॒व । भ॒व॒ति॒ । पु॒रस्ता᳚त् । प्र॒ती॒चीन᳚म् । अश्व॑स्य । उपेति॑ । द॒धा॒ति॒ । प॒श्चात् । प्रा॒चीन᳚म् । ऋ॒ष॒भस्य॑ । अप॑शवः । वै । अ॒न्ये । गो॒ अ॒श्वेभ्य॒ इति॑ गो -अ॒श्वेभ्यः॑ । प॒शवः॑ । गो॒ अ॒श्वानिति॑ गो - अ॒श्वान् । ए॒व । अ॒स्मै॒ । स॒मीचः॑ । द॒धा॒ति॒ । ए॒ताव॑न्तः । वै । प॒शवः॑ ।  \newline




\markright{ TS 5.2.9.5  \hfill https://www.vedavms.in \hfill}

\section{ TS 5.2.9.5 }

\textbf{TS 5.2.9.5 } \newline
\textbf{Samhita Paata} \newline

द्वि॒पाद॑श्च॒ चतु॑ष्पादश्च॒ तान्. वा ए॒तद॒ग्नौ प्रद॑धाति॒ यत् प॑शुशी॒र्॒.षाण्यु॑प॒-दधा᳚त्य॒-मुमा॑र॒ण्यमनु॑ ते दिशा॒मीत्या॑ह ग्रा॒म्येभ्य॑ ए॒व प॒शुभ्य॑ आर॒ण्यान् प॒शूञ्छुच॒मनूथ्सृ॑जति॒ तस्मा᳚थ् स॒माव॑त् पशू॒नां प्र॒जाय॑मानाना-मार॒ण्याः प॒शवः॒ कनी॑याꣳसः शु॒चा ह्यृ॑ताः स॑र्पशी॒र्॒.षमुप॑ दधाति॒ यैव स॒र्पे त्विषि॒स्तामे॒वाव॑ रुन्धे॒ - [  ] \newline

\textbf{Pada Paata} \newline

द्वि॒पाद॒ इति॑ द्वि - पादः॑ । च॒ । चतु॑ष्पाद॒ इति॒ चतुः॑ - पा॒दः॒ । च॒ । तान् । वै । ए॒तत् । अ॒ग्नौ । प्रेति॑ । द॒धा॒ति॒ । यत् । प॒शु॒शी॒र्॒.षाणीति॑ पशु - शी॒र्॒.षाणि॑ । उ॒प॒दधा॒तीत्यु॑प - दधा॑ति । अ॒मुम् । आ॒र॒ण्यम् । अन्विति॑ । ते॒ । दि॒शा॒मि॒ । इति॑ । आ॒ह॒ । ग्रा॒म्येभ्यः॑ । ए॒व । प॒शुभ्य॒ इति॑ प॒शु - भ्यः॒ । आ॒र॒ण्यान् । प॒शून् । शुच᳚म् । अनू॑ । उदिति॑ । सृ॒ज॒ति॒ । तस्मा᳚त् । स॒माव॑त् । प॒शू॒नाम् । प्र॒जाय॑मानाना॒मिति॑ प्र - जाय॑मानानाम् । आ॒र॒ण्याः । प॒शवः॑ । कनी॑याꣳसः । शु॒चा । हि । ऋ॒ताः । स॒र्प॒शी॒र्॒.षमिति॑ सर्प - शी॒र्॒.षम् । उपेति॑ । द॒धा॒ति॒ । या । ए॒व । स॒र्पे । त्विषिः॑ । ताम् । ए॒व । अवेति॑ । रु॒न्धे॒ ।  \newline




\markright{ TS 5.2.9.6  \hfill https://www.vedavms.in \hfill}

\section{ TS 5.2.9.6 }

\textbf{TS 5.2.9.6 } \newline
\textbf{Samhita Paata} \newline

यथ् स॑मी॒चीनं॑ पशुशी॒र्॒.षैरु॑प द॒द्ध्याद् ग्रा॒म्यान् प॒शून् दꣳशु॑काः स्यु॒र्यद्-वि॑षू॒चीन॑-मार॒ण्यान्. यजु॑रे॒व व॑दे॒दव॒ तां त्विषिꣳ॑ रुन्धे॒ या स॒र्पे न ग्रा॒म्यान् प॒शून्. हि॒नस्ति॒ नाऽऽ*र॒ण्यानथो॒ खलू॑प॒धेय॑मे॒व यदु॑प॒दधा॑ति॒ तेन॒ तां त्विषि॒मव॑ रुन्धे॒ या स॒र्पे यद्-यजु॒र्वद॑ति॒ तेन॑ शा॒न्तं ॥ \newline

\textbf{Pada Paata} \newline

यत् । स॒मी॒चीन᳚म् । प॒शु॒शी॒र्॒.षैरिति॑ पशु - शी॒र्॒.षैः । उ॒प॒द॒द्ध्यादित्यु॑प - द॒द्ध्यात् । ग्रा॒म्यान् । प॒शून् । दꣳशु॑काः । स्युः॒ । यत् । वि॒षू॒चीन᳚म् । आ॒र॒ण्यान् । यजुः॑ । ए॒व । व॒दे॒त् । अवेति॑ । ताम् । त्विषि᳚म् । रु॒न्धे॒ । या । स॒र्पे । न । ग्रा॒म्यान् । प॒शून् । हि॒नस्ति॑ । न । आ॒र॒ण्यान् । अथो॒ इति॑ । खलु॑ । उ॒प॒धेय॒मित्यु॑प - धेय᳚म् । ए॒व । यत् । उ॒प॒दधा॒तीत्यु॑प - दधा॑ति । तेन॑ । ताम् । त्विषि᳚म् । अवेति॑ । रु॒न्धे॒ । या । स॒र्पे । यत् । यजुः॑ । वद॑ति । तेन॑ । शा॒न्तम् ॥  \newline




\markright{ TS 5.2.10.1  \hfill https://www.vedavms.in \hfill}

\section{ TS 5.2.10.1 }

\textbf{TS 5.2.10.1 } \newline
\textbf{Samhita Paata} \newline

प॒शुर्वा ए॒ष यद॒ग्निर्योनिः॒ खलु॒ वा ए॒षा प॒शोर्वि क्रि॑यते॒ यत् प्रा॒चीन॑मैष्ट॒काद्-यजुः॑ क्रि॒यते॒ रेतो॑ऽप॒स्या॑ अप॒स्या॑ उप॑ दधाति॒ योना॑वे॒व रेतो॑ दधाति॒ पञ्चोप॑ दधाति॒ पाङ्क्ताः᳚ प॒शवः॑ प॒शूने॒वास्मै॒ प्रज॑नयति॒ पञ्च॑ दक्षिण॒तो वज्रो॒ वा अ॑प॒स्या॑ वज्रे॑णै॒व य॒ज्ञ्स्य॑ दक्षिण॒तो रक्षाꣳ॒॒स्यप॑ हन्ति॒ पञ्च॑ प॒श्चात् - [  ] \newline

\textbf{Pada Paata} \newline

प॒शुः । वै । ए॒षः । यत् । अ॒ग्निः । योनिः॑ । खलु॑ । वै । ए॒षा । प॒शोः । वीति॑ । क्रि॒य॒ते॒ । यत् । प्रा॒चीन᳚म् । ऐ॒ष्ट॒कात् । यजुः॑ । क्रि॒यते᳚ । रेतः॑ । अ॒प॒स्याः᳚ । अ॒प॒स्याः᳚ । उपेति॑ । द॒धा॒ति॒ । योनौ᳚ । ए॒व । रेतः॑ । द॒धा॒ति॒ । पञ्च॑ । उपेति॑ । द॒धा॒ति॒ । पाङ्क्ताः᳚ । प॒शवः॑ । प॒शून् । ए॒व । अ॒स्मै॒ । प्रेति॑ । ज॒न॒य॒ति॒ । पञ्च॑ । द॒क्षि॒ण॒तः । वज्रः॑ । वै । अ॒प॒स्याः᳚ । वज्रे॑ण । ए॒व । य॒ज्ञ्स्य॑ । द॒क्षि॒ण॒तः । रक्षाꣳ॑सि । अपेति॑ । ह॒न्ति॒ । पञ्च॑ । प॒श्चात् ।  \newline




\markright{ TS 5.2.10.2  \hfill https://www.vedavms.in \hfill}

\section{ TS 5.2.10.2 }

\textbf{TS 5.2.10.2 } \newline
\textbf{Samhita Paata} \newline

प्राची॒रुप॑ दधाति प॒श्चाद्वै प्रा॒चीनꣳ॒॒ रेतो॑ धीयते प॒श्चादे॒वास्मै᳚ प्रा॒चीनꣳ॒॒ रेतो॑ दधाति॒ पञ्च॑ पु॒रस्ता᳚त् प्र॒तीची॒रुप॑ दधाति॒ पञ्च॑ प॒श्चात् प्राची॒स्तस्मा᳚त् प्रा॒चीनꣳ॒॒ रेतो॑ धीयते प्र॒तीचीः᳚ प्र॒जा जा॑यन्ते॒ पञ्चो᳚त्तर॒त श्छ॑न्द॒स्याः᳚ प॒शवो॒ वै छ॑न्द॒स्याः᳚ प॒शूने॒व प्रजा॑ता॒न्थ् स्वमा॒यत॑नम॒भि पर्यू॑हत इ॒यं ॅवा अ॒ग्ने-र॑तिदा॒हाद॑बिभे॒थ् सैता - [  ] \newline

\textbf{Pada Paata} \newline

प्राचीः᳚ । उपेति॑ । द॒धा॒ति॒ । प॒श्चात् । वै । प्रा॒चीन᳚म् । रेतः॑ । धी॒य॒ते॒ । प॒श्चात् । ए॒व । अ॒स्मै॒ । प्रा॒चीन᳚म् । रेतः॑ । द॒धा॒ति॒ । पञ्च॑ । पु॒रस्ता᳚त् । प्र॒तीचीः᳚ । उपेति॑ । द॒धा॒ति॒ । पञ्च॑ । प॒श्चात् । प्राचीः᳚ । तस्मा᳚त् । प्रा॒चीन᳚म् । रेतः॑ । धी॒य॒ते॒ । प्र॒तीचीः᳚ । प्र॒जा इति॑ प्र-जाः । जा॒य॒न्ते॒ । पञ्च॑ । उ॒त्त॒र॒त इत्यु॑त् - त॒र॒तः । छ॒न्द॒स्याः᳚ । प॒शवः॑ । वै । छ॒न्द॒स्याः᳚ । प॒शून् । ए॒व । प्रजा॑ता॒निति॒ प्र - जा॒ता॒न् । स्वम् । आ॒यत॑न॒मित्या᳚ - यत॑नम् । अ॒भि । परीति॑ । ऊ॒ह॒ते॒ । इ॒यम् । वै । अ॒ग्नेः । अ॒ति॒दा॒हादित्य॑ति - दा॒हात् । अ॒बि॒भे॒त् । सा । ए॒ताः ।  \newline




\markright{ TS 5.2.10.3  \hfill https://www.vedavms.in \hfill}

\section{ TS 5.2.10.3 }

\textbf{TS 5.2.10.3 } \newline
\textbf{Samhita Paata} \newline

अ॑प॒स्या॑ अपश्य॒त् ता उपा॑धत्त॒ ततो॒ वा इ॒मां नात्य॑दह॒द्-यद॑प॒स्या॑ उप॒दधा᳚त्य॒स्या अन॑तिदाहायो॒वाच॑ हे॒यमद॒दिथ् स ब्रह्म॒णाऽन्नं॒ ॅयस्यै॒ता उ॑पधी॒यान्तै॒ य उ॑ चैना ए॒वंॅवे द॒दिति॑ प्राण॒भृत॒ उप॑ दधाति॒ रेत॑स्ये॒व प्रा॒णान् द॑धाति॒ तस्मा॒द्-वद॑न् प्रा॒णन् पश्य॑ञ्छृ॒ण्वन् प॒शुर्जा॑यते॒ ऽयं पु॒रो - [  ] \newline

\textbf{Pada Paata} \newline

अ॒प॒स्याः᳚ । अ॒प॒श्य॒त् । ताः । उपेति॑ । अ॒ध॒त्त॒ । ततः॑ । वै । इ॒माम् । न । अतीति॑ । अ॒द॒ह॒त् । यत् । अ॒प॒स्याः᳚ । उ॒प॒दधा॒तीत्यु॑प - दधा॑ति । अ॒स्याः । अन॑तिदाहा॒येत्यन॑ति - दा॒हा॒य॒ । उ॒वाच॑ । ह॒ । इ॒यम् । अद॑त् । इत् । सः । ब्रह्म॑णा । अन्न᳚म् । यस्य॑ । ए॒ताः । उ॒प॒धी॒यान्ता॒ इत्यु॑प - धी॒यान्तै᳚ । यः । उ॒ । च॒ । ए॒नाः॒ । ए॒वम् । वेद॑त् । इति॑ । प्रा॒ण॒भृत॒ इति॑ प्राण-भृतः॑ । उपेति॑ । द॒धा॒ति॒ । रेत॑सि । ए॒व । प्रा॒णानिति॑ प्र - अ॒नान् । द॒धा॒ति॒ । तस्मा᳚त् । वदन्न्॑ । प्रा॒णन्निति॑ प्र - अ॒नन्न् । पश्यन्न्॑ । शृ॒ण्वन्न् । प॒शुः । जा॒य॒ते॒ । अ॒यम् । पु॒रः ।  \newline




\markright{ TS 5.2.10.4  \hfill https://www.vedavms.in \hfill}

\section{ TS 5.2.10.4 }

\textbf{TS 5.2.10.4 } \newline
\textbf{Samhita Paata} \newline

भुव॒ इति॑ पु॒रस्ता॒दुप॑ दधाति प्रा॒णमे॒वैताभि॑-र्दाधारा॒ऽयं द॑क्षि॒णा वि॒श्वक॒र्मेति॑ दक्षिण॒तो मन॑ ए॒वैताभि॑र्दाधारा॒यं प॒श्चाद्-वि॒श्वव्य॑चा॒ इति॑ प॒श्चा-च्चक्षु॑रे॒वैताभि॑-र्दाधारे॒द-मु॑त्त॒राथ् सुव॒रित्यु॑त्तर॒तः श्रोत्र॑मे॒वैताभि॑-र्दाधारे॒यमु॒परि॑ म॒तिरित्यु॒परि॑ष्टा॒द्-वाच॑मे॒वैताभि॑-र्दाधार॒ दश॑द॒शोप॑ दधाति सवीर्य॒त्वाया᳚क्ष्ण॒यो - [  ] \newline

\textbf{Pada Paata} \newline

भुवः॑ । इति॑ । पु॒रस्ता᳚त् । उपेति॑ । द॒धा॒ति॒ । प्रा॒णमिति॑ प्र - अ॒नम् । ए॒व । ए॒ताभिः॑ । दा॒धा॒र॒ । अ॒यम् । द॒क्षि॒णा । वि॒श्वक॒र्मेति॑ वि॒श्व-क॒र्मा॒ । इति॑ । द॒क्षि॒ण॒तः । मनः॑ । ए॒व । ए॒ताभिः॑ । दा॒धा॒र॒ । अ॒यम् । प॒श्चात् । वि॒श्वव्य॑चा॒ इति॑ वि॒श्व - व्य॒चाः॒ । इति॑ । प॒श्चात् । चक्षुः॑ । ए॒व । ए॒ताभिः॑ । दा॒धा॒र॒ । इ॒दम् । उ॒त्त॒रादित्यु॑त् - त॒रात् । सुवः॑ । इति॑ । उ॒त्त॒र॒त इत्यु॑त् - त॒र॒तः । श्रोत्र᳚म् । ए॒व । ए॒ताभिः॑ । दा॒धा॒र॒ । इ॒यम् । उ॒परि॑ । म॒तिः । इति॑ । उ॒परि॑ष्टात् । वाच᳚म् । ए॒व । ए॒ताभिः॑ । दा॒धा॒र॒ । दश॑द॒शेति॒ दश॑-द॒श॒ । उपेति॑ । द॒धा॒ति॒ । स॒वी॒र्य॒त्वायेति॑ सवीर्य-त्वाय॑ । अ॒क्ष्ण॒या ।  \newline




\markright{ TS 5.2.10.5  \hfill https://www.vedavms.in \hfill}

\section{ TS 5.2.10.5 }

\textbf{TS 5.2.10.5 } \newline
\textbf{Samhita Paata} \newline

-प॑ दधाति॒ तस्मा॑दक्ष्ण॒या प॒शवोऽङ्गा॑नि॒ प्रह॑रन्ति॒ प्रति॑ष्ठित्यै॒ याः प्राची॒स्ताभि॒-र्वसि॑ष्ठ आर्द्ध्नो॒द्या द॑क्षि॒णा ताभि॑र्भ॒रद्वा॑जो॒ याः प्र॒तीची॒स्ताभि॑ र्वि॒श्वामि॑त्रो॒ या उदी॑ची॒स्ताभि॑-र्ज॒मद॑ग्नि॒र्या ऊ॒र्द्ध्वास्ताभि॑-र्वि॒श्वक॑र्मा॒ य ए॒वमे॒तासा॒मृद्धिं॒ ॅवेद॒र्द्ध्नोत्ये॒व य आ॑सामे॒वं ब॒न्धुतां॒ ॅवेद॒ बन्धु॑मान् भवति॒ य आ॑सामे॒वं क्लृप्तिं॒ ॅवेद॒ कल्प॑ते - [  ] \newline

\textbf{Pada Paata} \newline

उपेति॑ । द॒धा॒ति॒ । तस्मा᳚त् । अ॒क्ष्ण॒या । प॒शवः॑ । अङ्गा॑नि । प्रेति॑ । ह॒र॒न्ति॒ । प्रति॑ष्ठित्या॒ इति॒ प्रति॑ - स्थि॒त्यै॒ । याः । प्राचीः᳚ । ताभिः॑ । वसि॑ष्ठः । आ॒द्‌र्ध्नो॒त् । याः । द॒क्षि॒णा । ताभिः॑ । भ॒रद्वा॑जः । याः । प्र॒तीचीः᳚ । ताभिः॑ । वि॒श्वामि॑त्र॒ इति॑ वि॒श्व - मि॒त्रः॒ । याः । उदी॑चीः । ताभिः॑ । ज॒मद॑ग्निः । याः । ऊ॒द्‌र्ध्वाः । ताभिः॑ । वि॒श्वक॒र्मेति॑ वि॒श्व - क॒र्मा॒ । यः । ए॒वम् । ए॒तासा᳚म् । ऋद्धि᳚म् । वेद॑ । ऋ॒द्ध्नोति॑ । ए॒व । यः । आ॒सा॒म् । ए॒वम् । ब॒न्धुता᳚म् । वेद॑ । बन्धु॑मा॒निति॒ बन्धु॑ - मा॒न् । भ॒व॒ति॒ । यः । आ॒सा॒म् । ए॒वम् । क्लृप्ति᳚म् । वेद॑ । कल्प॑ते ।  \newline




\markright{ TS 5.2.10.6  \hfill https://www.vedavms.in \hfill}

\section{ TS 5.2.10.6 }

\textbf{TS 5.2.10.6 } \newline
\textbf{Samhita Paata} \newline

ऽस्मै॒ य आ॑सामे॒वमा॒यत॑नं॒ ॅवेदा॒ऽऽ*यत॑नवान् भवति॒ य आ॑सामे॒वं प्र॑ति॒ष्ठां ॅवेद॒ प्रत्ये॒व ति॑ष्ठति प्राण॒भृत॑ उप॒धाय॑ सं॒ॅयत॒ उप॑ दधाति प्रा॒णाने॒वा ऽस्मि॑न् धि॒त्वा सं॒ॅयद्भिः॒ संॅय॑च्छति॒ तथ् सं॒ॅयताꣳ॑ संॅय॒त्त्वमथो᳚ प्रा॒ण ए॒वापा॒नं द॑धाति॒ तस्मा᳚त् प्राणापा॒नौ सं च॑रतो॒ विषू॑ची॒रुप॑ दधाति॒ तस्मा॒द्-विष्व॑ञ्चौ प्राणापा॒नौ यद्वा अ॒ग्नेरसं॑ॅयत॒ - [  ] \newline

\textbf{Pada Paata} \newline

अ॒स्मै॒ । यः । आ॒सा॒म् । ए॒वम् । आ॒यत॑न॒मित्या᳚ - यत॑नम् । वेद॑ । आ॒यत॑नवा॒नित्या॒यत॑न - वा॒न् । भ॒व॒ति॒ । यः । आ॒सा॒म् । ए॒वम् । प्र॒ति॒ष्ठामिति॑ प्रति - स्थाम् । वेद॑ । प्रतीति॑ । ए॒व । ति॒ष्ठ॒ति॒ । प्रा॒ण॒भृत॒ इति॑ प्राण - भृतः॑ । उ॒प॒धायेत्यु॑प - धाय॑ । सं॒ॅयत॒ इति॑ सं - यतः॑ । उपेति॑ । द॒धा॒ति॒ । प्रा॒णा॒निति॑ प्र - अ॒नान् । ए॒व । अ॒स्मि॒न्न् । धि॒त्वा । सं॒ॅयद्भि॒रिति॑ सं॒ॅयत् - भिः॒ । समिति॑ । य॒च्छ॒ति॒ । तत् । सं॒ॅयता॒मिति॑ सं - यता᳚म् । सं॒ॅय॒त्त्वमिति॑ संॅयत् - त्वम् । अथो॒ इति॑ । प्रा॒ण इति॑ प्र - अ॒ने । ए॒व । अ॒पा॒नमित्य॑प - अ॒नम् । द॒धा॒ति॒ । तस्मा᳚त् । प्रा॒णा॒पा॒नाविति॑ प्राण - अ॒पा॒नौ । समिति॑ । च॒र॒तः॒ । विषू॑चीः । उपेति॑ । द॒धा॒ति॒ । तस्मा᳚त् । विष्व॑ञ्चौ । प्रा॒णा॒पा॒नाविति॑ प्राण-अ॒पा॒नौ । यत् । वै । अ॒ग्नेः । असं॑ॅयत॒मित्यसं᳚ - य॒त॒म् ।  \newline




\markright{ TS 5.2.10.7  \hfill https://www.vedavms.in \hfill}

\section{ TS 5.2.10.7 }

\textbf{TS 5.2.10.7 } \newline
\textbf{Samhita Paata} \newline

-मसु॑वर्ग्यमस्य॒ तथ् सु॑व॒र्ग्यो᳚ऽग्निर्यथ् सं॒ॅयत॑ उप॒ दधा॑ति॒ समे॒वैनं॑ ॅयच्छति उव॒र्ग्य॑मे॒वाक॒ -स्त्र्यवि॒र्वयः॑ कृ॒तमया॑ना॒मित्या॑ह॒ वयो॑भिरे॒वाया॒नव॑ रु॒न्धे ऽयै॒र्वयाꣳ॑सि स॒र्वतो॑ वायु॒मती᳚र्भवन्ति॒ तस्मा॑द॒यꣳ स॒र्वतः॑ पवते ॥ \newline

\textbf{Pada Paata} \newline

असु॑वर्ग्य॒मित्यसु॑वः - ग्य॒म् । अ॒स्य॒ । तत् । सु॒व॒र्ग्य॑ इति॑ सुवः - ग्यः॑ । अ॒ग्निः । यत् । सं॒ॅयत॒ इति॑ सं - यतः॑ । उ॒प॒दधा॒तीत्यु॑प - दधा॑ति । समिति॑ । ए॒व । ए॒न॒म् । य॒च्छ॒ति॒ । सु॒व॒र्ग्य॑मिति॑ सुवः - ग्य᳚म् । ए॒व । अ॒कः॒ । त्र्यवि॒रिति॑ त्रि - अविः॑ । वयः॑ । कृ॒तम् । अया॑नाम् । इति॑ । आ॒ह॒ । वयो॑भि॒रिति॒ वयः॑ - भिः॒ । ए॒व । अयान्॑ । अवेति॑ । रु॒न्धे॒ । अयैः᳚ । वयाꣳ॑सि । स॒र्वतः॑ । वा॒यु॒मती॒रिति॑ वायु - मतीः᳚ । भ॒व॒न्ति॒ । तस्मा᳚त् । अ॒यम् । स॒र्वतः॑ । प॒व॒ते॒ ॥  \newline




\markright{ TS 5.2.11.1  \hfill https://www.vedavms.in \hfill}

\section{ TS 5.2.11.1 }

\textbf{TS 5.2.11.1 } \newline
\textbf{Samhita Paata} \newline

गा॒य॒त्री त्रि॒ष्टुब् जग॑त्यनु॒ष्टुक् प॒ङ्क्त्या॑ स॒ह । बृ॒ह॒त्यु॑ष्णिहा॑ क॒कुथ् सू॒चीभिः॑ शिम्यन्तु त्वा ॥द्वि॒पदा॒ या चतु॑ष्पदा त्रि॒पदा॒ याच॒ षट्प॑दा । सछ॑न्दा॒ या च॒ विच्छ॑न्दाः सू॒चीभिः॑ शिम्यन्तु त्वा ॥म॒हाना᳚म्नी रे॒वत॑यो॒ विश्वा॒ आशाः᳚ प्र॒सूव॑रीः । मेघ्या॑ वि॒द्युतो॒ वाचः॑ सू॒चीभिः॑ शिम्यन्तु त्वा ॥र॒ज॒ता हरि॑णीः॒ सीसा॒ युजो॑ युज्यन्ते॒ कर्म॑भिः । अश्व॑स्य वा॒जिन॑स्त्व॒चि सू॒चीभिः॑ शिम्यन्तु त्वा ॥ नारी᳚ - [  ] \newline

\textbf{Pada Paata} \newline

गा॒य॒त्री । त्रि॒ष्टुप् । जग॑ती । अ॒नु॒ष्टुगित्य॑नु-स्तुक् । प॒ङ्क्त्या᳚ । स॒ह ॥ बृ॒ह॒ती । उ॒ष्णिहा᳚ । क॒कुत् । सू॒चीभिः॑ । शि॒म्य॒न्तु॒ । त्वा॒ ॥ द्वि॒पदेति॑ द्वि - पदा᳚ । या । चतु॑ष्प॒देति॒ चतुः॑ - प॒दा॒ । त्रि॒पदेति॑ त्रि - पदा᳚ । या । च॒ । षट्प॒देति॒ षट् - प॒दा॒ ॥ सछ॑न्दा॒ इति॒ स - छ॒न्दाः॒ । या । च॒ । विच्छ॑न्दा॒ इति॒ वि - छ॒न्दाः॒ । सू॒चीभिः॑ । शि॒म्य॒न्तु॒ । त्वा॒ ॥ म॒हाना᳚म्नी॒रिति॑ म॒हा - ना॒म्नीः॒ । रे॒वत॑यः । विश्वाः᳚ । आशाः᳚ । प्र॒सूव॑री॒रिति॑ प्र - सूव॑रीः ॥ मेघ्‌याः᳚ । वि॒द्युत॒ इति॑ वि - द्युतः॑ । वाचः॑ । सू॒चीभिः॑ । शि॒म्य॒न्तु॒ । त्वा॒ ॥ र॒ज॒ताः । हरि॑णीः । सीसाः᳚ । युजः॑ । यु॒ज्य॒न्ते॒ । कर्म॑भि॒रिति॒ कर्म॑ -  भिः॒ ॥ अश्व॑स्य । वा॒जिनः॑ । त्व॒चि । सू॒चीभिः॑ । शि॒म्य॒न्तु॒ । त्वा॒ ॥ नारीः᳚ ।  \newline




\markright{ TS 5.2.11.2  \hfill https://www.vedavms.in \hfill}

\section{ TS 5.2.11.2 }

\textbf{TS 5.2.11.2 } \newline
\textbf{Samhita Paata} \newline

-स्ते॒ पत्न॑यो॒ लोम॒ विचि॑न्वन्तु मनी॒षया᳚ । दे॒वानां॒ पत्नी॒र्दिशः॑ सू॒चीभिः॑ शिम्यन्तु त्वा ॥कु॒विद॒ङ्ग यव॑मन्तो॒ यवं॑ चि॒द्यथा॒ दान्त्य॑नुपू॒र्वं ॅवि॒यूय॑ ।इ॒हेहै॑षां कृणुत॒ भोज॑नानि॒ ये ब॒र्॒.हिषो॒ नमो॑वृक्तिं॒ नज॒ग्मुः ॥ \newline

\textbf{Pada Paata} \newline

ते । पत्न॑यः । लोम॑ । वीति॑ । चि॒न्व॒न्तु॒ । म॒नी॒षया᳚ ॥ दे॒वाना᳚म् । पत्नीः᳚ । दिशः॑ । सू॒चीभिः॑ । शि॒म्य॒न्तु॒ । त्वा॒ ॥ कु॒वित् । अ॒ङ्ग । यव॑मन्त॒ इति॒ यव॑ - म॒न्तः॒ । यव᳚म् । चि॒त् । यथा᳚ । दान्ति॑ । अ॒नु॒पू॒र्वमित्य॑नु - पू॒र्वम् । वि॒यूयेति॑ वि - यूय॑ ॥ इ॒हेहेती॒ह - इ॒ह॒ । ए॒षा॒म् । कृ॒णु॒त॒ । भोज॑नानि । ये । ब॒र्॒.हिषः॑ । नमो॑वृक्ति॒मिति॒ नमः॑ - वृ॒क्ति॒म् । न । ज॒ग्मुः ॥  \newline




\markright{ TS 5.2.12.1  \hfill https://www.vedavms.in \hfill}

\section{ TS 5.2.12.1 }

\textbf{TS 5.2.12.1 } \newline
\textbf{Samhita Paata} \newline

कस्त्वा᳚ च्छ्यति॒ कस्त्वा॒ वि शा᳚स्ति॒ कस्ते॒ गात्रा॑णि शिम्यति । क उ॑ ते शमि॒ता क॒विः ॥ ऋ॒तव॑स्त ऋतु॒धा परुः॑ शमि॒तारो॒ विशा॑सतु । सं॒ॅव॒थ्स॒रस्य॒ धाय॑सा॒ शिमी॑भिः शिम्यन्तु त्वा ॥दैव्या॑ अद्ध्व॒र्यव॑स्त्वा॒ च्छ्यन्तु॒ वि च॑ शासतु । गात्रा॑णि पर्व॒शस्ते॒ शिमाः᳚ कृण्वन्तु॒ शिम्य॑न्तः ॥अ॒र्द्ध॒मा॒साः परूꣳ॑षि ते॒ मासाः᳚ छ्यन्तु॒ शिम्य॑न्तः । अ॒हो॒रा॒त्राणि॑ म॒रुतो॒ विलि॑ष्टꣳ - [  ] \newline

\textbf{Pada Paata} \newline

कः । त्वा॒ । छ्य॒ति॒ । कः । त्वा॒ । वीति॑ । शा॒स्ति॒ । कः । ते॒ । गात्रा॑णि । शि॒म्य॒ति॒ ॥ कः । उ॒ । ते॒ । श॒मि॒ता । क॒विः ॥ ऋ॒तवः॑ । ते॒ । ऋ॒तु॒धेत्यृ॑तु - धा । परुः॑ । श॒मि॒तारः॑ । वीति॑ । शा॒स॒तु॒ ॥ सं॒ॅव॒थ्स॒रस्येति॑ सं - व॒थ्स॒रस्य॑ । धाय॑सा । शिमी॑भिः । शि॒म्य॒न्तु॒ । त्वा॒ ॥ दैव्याः᳚ । अ॒द्ध्व॒र्यवः॑ । त्वा॒ । छ्यन्तु॑ । वीति॑ । च॒ । शा॒स॒तु॒ ॥ गात्रा॑णि । प॒र्व॒श इति॑ पर्व - शः । ते॒ । शिमाः᳚ । कृ॒ण्व॒न्तु॒ । शिम्य॑न्तः ॥ अ॒द्‌र्ध॒मा॒सा इत्य॑द्‌र्ध - मा॒साः । परूꣳ॑षि । ते॒ । मासाः᳚ । छ्य॒न्तु॒ । शिम्य॑न्तः ॥ अ॒हो॒रा॒त्राणीत्य॑हः - रा॒त्राणि॑ । म॒रुतः॑ । विलि॑ष्ट॒मिति॒ वि - लि॒ष्ट॒म् ।  \newline




\markright{ TS 5.2.12.2  \hfill https://www.vedavms.in \hfill}

\section{ TS 5.2.12.2 }

\textbf{TS 5.2.12.2 } \newline
\textbf{Samhita Paata} \newline

सूदयन्तु ते ॥ पृ॒थि॒वी ते॒ ऽन्तरि॑क्षेण वा॒युश्छि॒द्रं भि॑षज्यतु । द्यौस्ते॒ नक्ष॑त्रैः स॒ह रू॒पं कृ॑णोतु साधु॒या ॥ शं ते॒ परे᳚भ्यो॒ गात्रे᳚भ्यः॒ शम॒स्त्वव॑रेभ्यः । शम॒स्थभ्यो॑ म॒ज्जभ्यः॒ शमु॑ ते त॒नुवे॑ भुवत् ॥ \newline

\textbf{Pada Paata} \newline

सू॒द॒य॒न्तु॒ । ते॒ ॥ पृ॒थि॒वी । ते॒ । अ॒न्तरि॑क्षेण । वा॒युः । छि॒द्रम् । भि॒ष॒ज्य॒तु॒ ॥ द्यौः । ते॒ । नक्ष॑त्रैः । स॒ह । रू॒पम् । कृ॒णो॒तु॒ । सा॒धु॒या ॥ शम् । ते॒ । परे᳚भ्यः । गात्रे᳚भ्यः । शम् । अ॒स्तु॒ । अव॑रेभ्यः ॥ शम् । अ॒स्थभ्य॒ इत्य॒स्थ - भ्यः॒ । म॒ज्जभ्य॒ इति॑ म॒ज्ज - भ्यः॒ । शम् । उ॒ । ते॒ । त॒नुवे᳚ । भु॒व॒त् ॥  \newline






\end{document}