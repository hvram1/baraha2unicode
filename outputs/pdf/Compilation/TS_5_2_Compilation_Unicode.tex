\documentclass[17pt]{extarticle}
\usepackage{babel}
\usepackage{fontspec}
\usepackage{polyglossia}
\usepackage{extsizes}

\usepackage{color}   %May be necessary if you want to color links
\usepackage{hyperref}
\hypersetup{
    colorlinks=true, %set true if you want colored links
    linktoc=all,     %set to all if you want both sections and subsections linked
    linkcolor=black,  %choose some color if you want links to stand out
}

\setmainlanguage{sanskrit}
\setotherlanguages{english} %% or other languages
\setlength{\parindent}{0pt}
\pagestyle{myheadings}
\newfontfamily\devanagarifont[Script=Devanagari]{AdishilaVedic}
\renewcommand{\theHsection}{\thepart.section.\thesection}

\newcommand{\VAR}[1]{}
\newcommand{\BLOCK}[1]{}




\begin{document}
\begin{titlepage}
    \begin{center}
 
\begin{sanskrit}
    { \Large
    कृष्ण यजुर्वेदीय तैत्तिरीय संहिता,पद,जटा,घन पाठः 
    }
    \\
    \vspace{2.5cm}
    \mbox{ \Large
    5.2      पञ्चमकाण्डे द्वितीयः प्रश्नः - चित्युपक्रमाभिधानं   }
\end{sanskrit}
\end{center}

\end{titlepage}
\tableofcontents
\phantomsection
\pagebreak

\markright{ TS 5.2.1.1  \hfill https://www.vedavms.in \hfill}

\section{ TS 5.2.1.1 }

\textbf{TS 5.2.1.1 } \newline
\textbf{Samhita Paata} \newline

विष्णु॑मुखा॒ वै दे॒वा श्छन्दो॑भिरि॒मान् ॅलो॒कान॑नपज॒य्य म॒भ्य॑जय॒न्॒.यद्-वि॑ष्णुक्र॒मान् क्रम॑ते॒ विष्णु॑रे॒व भू॒त्वा यज॑मान॒श्छन्दो॑भिरि॒मान् ॅलो॒कान॑नपज॒य्यम॒भि ज॑यति॒ विष्णोः॒ क्रमो᳚ऽस्य-भिमाति॒हेत्या॑ह गाय॒त्री वै पृ॑थि॒वी त्रैष्ठु॑भम॒न्तरि॑क्षं॒ जाग॑ती॒ द्यौरानु॑ष्टुभी॒र्दिश॒ श्छन्दो॑भिरे॒वेमान् ॅलो॒कान्. य॑था पू॒र्वम॒भि ज॑यति प्र॒जाप॑तिर॒ग्निम॑सृजत॒ सो᳚ऽस्माथ् सृ॒ष्टः - [  ] \newline

\textbf{Pada Paata} \newline

विष्णु॑मुखा॒ इति॒ विष्णु॑ - मु॒खाः॒ । वै । दे॒वाः । छन्दो॑भि॒रिति॒ छन्दः॑ - भिः॒ । इ॒मान् । लो॒कान् । अ॒न॒प॒ज॒य्यमित्य॑नप - ज॒य्यम् । अ॒भीति॑ । अ॒ज॒य॒न्न् । यत् । वि॒ष्णु॒क्र॒मानिति॑ विष्णु-क्र॒मान् । क्रम॑ते । विष्णुः॑ । ए॒व । भू॒त्वा । यज॑मानः । छन्दो॑भि॒रिति॒ छन्दः॑ - भिः॒ । इ॒मान् । लो॒कान् । अ॒न॒प॒ज॒य्यमित्य॑नप-ज॒य्यम् । अ॒भीति॑ । ज॒य॒ति॒ । विष्णोः᳚ । क्रमः॑ । अ॒सि॒ । अ॒भि॒मा॒ति॒हेत्य॑भिमाति-हा । इति॑ । आ॒ह॒ । गा॒य॒त्री । वै । पृ॒थि॒वी । त्रैष्टु॑भम् । अ॒न्तरि॑क्षम् । जाग॑ती । द्यौः । आनु॑ष्टुभी॒रित्यानु॑ - स्तु॒भीः॒ । दिशः॑ । छन्दो॑भि॒रिति॒ छन्दः॑ - भिः॒ । ए॒व । इ॒मान् । लो॒कान् । य॒था॒पू॒र्वमिति॑ यथा - पू॒र्वम् । अ॒भीति॑ । ज॒य॒ति॒ । प्र॒जाप॑ति॒रिति॑ प्र॒जा - प॒तिः॒ । अ॒ग्निम् । अ॒सृ॒ज॒त॒ । सः॒ । अ॒स्मा॒त् । सृ॒ष्टः ।  \newline


\textbf{Krama Paata} \newline

विष्णु॑मुखा॒ वै । विष्णु॑मुखा॒ इति॒ विष्णु॑ - मु॒खाः॒ । वै दे॒वाः । दे॒वाश्छन्दो॑भिः । छन्दो॑भिरि॒मान् । छन्दो॑भि॒रिति॒ छन्दः॑ - भिः॒ । इ॒मान् ॅलो॒कान् । लो॒कान॑नपज॒य्यम् । अ॒न॒प॒ज॒य्यम॒भि । अ॒न॒प॒ज॒य्यमित्य॑नप - ज॒य्यम् । अ॒भ्य॑जयन्न् । अ॒ज॒य॒न्॒. यत् । यद् वि॑ष्णुक्र॒मान् । वि॒ष्णु॒क्र॒मान् क्रम॑ते । वि॒ष्णु॒क्र॒मानिति॑ विष्णु - क्र॒मान् । क्रम॑ते॒ विष्णुः॑ । विष्णु॑रे॒व । ए॒व भू॒त्वा । भू॒त्वा यज॑मानः । यज॑मान॒श्छन्दो॑भिः । छन्दो॑भिरि॒मान् । छन्दो॑भि॒रिति॒ छन्दः॑ - भिः॒ । इ॒मान् ॅलो॒कान् । लो॒कान॑नपज॒य्यम् । अ॒न॒प॒ज॒य्यम॒भि । अ॒न॒प॒ज॒य्यमित्य॑नप - ज॒य्यम् । अ॒भि ज॑यति । ज॒य॒ति॒ विष्णोः᳚ । विष्णोः॒ क्रमः॑ । क्रमो॑ऽसि । अ॒स्य॒भि॒मा॒ति॒हा । अ॒भि॒मा॒ति॒हेति॑ । अ॒भि॒मा॒ति॒हेत्य॑भिमाति - हा । इत्या॑ह । आ॒ह॒ गा॒य॒त्री । गा॒य॒त्री वै । वै पृ॑थि॒वी । पृ॒थि॒वी त्रैष्टु॑भम् । त्रैष्टु॑भम॒न्तरि॑क्षम् । अ॒न्तरि॑क्ष॒म् जाग॑ती । जाग॑ती॒ द्यौः । द्यौरानु॑ष्टुभीः । आनु॑ष्टुभी॒र् दिशः॑ । आनु॑ष्टुभी॒रित्यानु॑ - स्तु॒भीः॒ । दिश॒श्छन्दो॑भिः । छन्दो॑भिरे॒व । छन्दो॑भि॒रिति॒ छन्दः॑ - भिः॒ । ए॒वेमान् । इ॒मान् ॅलो॒कान् । लो॒कान्. य॑थापू॒र्वम् । य॒था॒पू॒र्वम॒भि । य॒था॒पू॒र्वमिति॑ यथा - पू॒र्वम् । अ॒भि ज॑यति । ज॒य॒ति॒ प्र॒जाप॑तिः । प्र॒जाप॑तिर॒ग्निम् । प्र॒जाप॑ति॒रिति॑ प्र॒जा - प॒तिः॒ । अ॒ग्निम॑सृजत । अ॒सृ॒ज॒त॒ सः । सो᳚ऽस्मात् । अ॒स्मा॒थ् सृ॒ष्टः । सृ॒ष्टः पराङ्॑ \newline

\textbf{Jatai Paata} \newline

1. विष्णु॑मुखा॒ वै वै विष्णु॑मुखा॒ विष्णु॑मुखा॒ वै । \newline
2. विष्णु॑मुखा॒ इति॒ विष्णु॑ - मु॒खाः॒ । \newline
3. वै दे॒वा दे॒वा वै वै दे॒वाः । \newline
4. दे॒वा श्छन्दो॑भि॒ श्छन्दो॑भिर् दे॒वा दे॒वा श्छन्दो॑भिः । \newline
5. छन्दो॑भि रि॒मा नि॒मान् छन्दो॑भि॒ श्छन्दो॑भि रि॒मान् । \newline
6. छन्दो॑भि॒रिति॒ छन्दः॑ - भिः॒ । \newline
7. इ॒मान् ॅलो॒कान् ॅलो॒का नि॒मा नि॒मान् ॅलो॒कान् । \newline
8. लो॒का न॑नपज॒य्य म॑नपज॒य्यम् ॅलो॒कान् ॅलो॒का न॑नपज॒य्यम् । \newline
9. अ॒न॒प॒ज॒य्य म॒भ्या᳚(1॒)भ्य॑ नपज॒य्य म॑नपज॒य्य म॒भि । \newline
10. अ॒न॒प॒ज॒य्यमित्य॑नप - ज॒य्यम् । \newline
11. अ॒भ्य॑जयन् नजयन् न॒भ्या᳚(1॒) भ्य॑जयन्न् । \newline
12. अ॒ज॒य॒न्॒. यद् य द॑जयन् नजय॒न्॒. यत् । \newline
13. यद् वि॑ष्णुक्र॒मान्. वि॑ष्णुक्र॒मान्. यद् यद् वि॑ष्णुक्र॒मान् । \newline
14. वि॒ष्णु॒क्र॒मान् क्रम॑ते॒ क्रम॑ते विष्णुक्र॒मान्. वि॑ष्णुक्र॒मान् क्रम॑ते । \newline
15. वि॒ष्णु॒क्र॒मानिति॑ विष्णु - क्र॒मान् । \newline
16. क्रम॑ते॒ विष्णु॒र् विष्णुः॒ क्रम॑ते॒ क्रम॑ते॒ विष्णुः॑ । \newline
17. विष्णु॑ रे॒वैव विष्णु॒र् विष्णु॑ रे॒व । \newline
18. ए॒व भू॒त्वा भू॒त्वै वैव भू॒त्वा । \newline
19. भू॒त्वा यज॑मानो॒ यज॑मानो भू॒त्वा भू॒त्वा यज॑मानः । \newline
20. यज॑मान॒ श्छन्दो॑भि॒ श्छन्दो॑भि॒र् यज॑मानो॒ यज॑मान॒ श्छन्दो॑भिः । \newline
21. छन्दो॑भिरि॒मा नि॒मान् छन्दो॑भि॒ श्छन्दो॑भि रि॒मान् । \newline
22. छन्दो॑भि॒रिति॒ छन्दः॑ - भिः॒ । \newline
23. इ॒मान् ॅलो॒कान् ॅलो॒का नि॒मा नि॒मान् ॅलो॒कान् । \newline
24. लो॒का न॑नपज॒य्य म॑नपज॒य्यम् ॅलो॒कान् ॅलो॒का न॑नपज॒य्यम् । \newline
25. अ॒न॒प॒ज॒य्य म॒भ्या᳚(1॒) भ्य॑नपज॒य्य म॑नपज॒य्य म॒भि । \newline
26. अ॒न॒प॒ज॒य्यमित्य॑नप - ज॒य्यम् । \newline
27. अ॒भि ज॑यति जय त्य॒भ्य॑भि ज॑यति । \newline
28. ज॒य॒ति॒ विष्णो॒र् विष्णो᳚र् जयति जयति॒ विष्णोः᳚ । \newline
29. विष्णोः॒ क्रमः॒ क्रमो॒ विष्णो॒र् विष्णोः॒ क्रमः॑ । \newline
30. क्रमो᳚ ऽस्यसि॒ क्रमः॒ क्रमो॑ ऽसि । \newline
31. अ॒स्य॒भि॒मा॒ति॒हा ऽभि॑माति॒हा ऽस्य॑स्य भिमाति॒हा । \newline
32. अ॒भि॒मा॒ति॒ हेतीत्य॑ भिमाति॒हा ऽभि॑माति॒हेति॑ । \newline
33. अ॒भि॒मा॒ति॒हेत्य॑भिमाति - हा । \newline
34. इत्या॑ हा॒हे तीत्या॑ह । \newline
35. आ॒ह॒ गा॒य॒त्री गा॑य॒ त्र्या॑हाह गाय॒त्री । \newline
36. गा॒य॒त्री वै वै गा॑य॒त्री गा॑य॒त्री वै । \newline
37. वै पृ॑थि॒वी पृ॑थि॒वी वै वै पृ॑थि॒वी । \newline
38. पृ॒थि॒वी त्रैष्टु॑भ॒म् त्रैष्टु॑भम् पृथि॒वी पृ॑थि॒वी त्रैष्टु॑भम् । \newline
39. त्रैष्टु॑भ म॒न्तरि॑क्ष म॒न्तरि॑क्ष॒म् त्रैष्टु॑भ॒म् त्रैष्टु॑भ म॒न्तरि॑क्षम् । \newline
40. अ॒न्तरि॑क्ष॒म् जाग॑ती॒ जाग॑त्य॒ न्तरि॑क्ष म॒न्तरि॑क्ष॒म् जाग॑ती । \newline
41. जाग॑ती॒ द्यौर् द्यौर् जाग॑ती॒ जाग॑ती॒ द्यौः । \newline
42. द्यौ रानु॑ष्टुभी॒ रानु॑ष्टुभी॒र् द्यौर् द्यौ रानु॑ष्टुभीः । \newline
43. आनु॑ष्टुभी॒र् दिशो॒ दिश॒ आनु॑ष्टुभी॒ रानु॑ष्टुभी॒र् दिशः॑ । \newline
44. आनु॑ष्टुभी॒रित्यानु॑ - स्तु॒भीः॒ । \newline
45. दिश॒ श्छन्दो॑भि॒ श्छन्दो॑भि॒र् दिशो॒ दिश॒ श्छन्दो॑भिः । \newline
46. छन्दो॑भि रे॒वैव छन्दो॑भि॒ श्छन्दो॑भि रे॒व । \newline
47. छन्दो॑भि॒रिति॒ छन्दः॑ - भिः॒ । \newline
48. ए॒वे मा नि॒मा ने॒वैवे मान् । \newline
49. इ॒मान् ॅलो॒कान् ॅलो॒का नि॒मा नि॒मान् ॅलो॒कान् । \newline
50. लो॒कान्. य॑थापू॒र्वं ॅय॑थापू॒र्वम् ॅलो॒कान् ॅलो॒कान्. य॑थापू॒र्वम् । \newline
51. य॒था॒पू॒र्व म॒भ्य॑भि य॑थापू॒र्वं ॅय॑थापू॒र्व म॒भि । \newline
52. य॒था॒पू॒र्वमिति॑ यथा - पू॒र्वम् । \newline
53. अ॒भि ज॑यति जय त्य॒भ्य॑भि ज॑यति । \newline
54. ज॒य॒ति॒ प्र॒जाप॑तिः प्र॒जाप॑तिर् जयति जयति प्र॒जाप॑तिः । \newline
55. प्र॒जाप॑ति र॒ग्नि म॒ग्निम् प्र॒जाप॑तिः प्र॒जाप॑ति र॒ग्निम् । \newline
56. प्र॒जाप॑ति॒रिति॑ प्र॒जा - प॒तिः॒ । \newline
57. अ॒ग्नि म॑सृजता सृजता॒ग्नि म॒ग्नि म॑सृजत । \newline
58. अ॒सृ॒ज॒त॒ स सो॑ ऽसृजता सृजत॒ सः । \newline
59. सो᳚ ऽस्मा दस्मा॒थ् स सो᳚ ऽस्मात् । \newline
60. अ॒स्मा॒थ् सृ॒ष्टः सृ॒ष्टो᳚ ऽस्मा दस्माथ् सृ॒ष्टः । \newline
61. सृ॒ष्टः परा॒ङ् परा᳚ङ् ख्सृ॒ष्टः सृ॒ष्टः पराङ्॑ । \newline

\textbf{Ghana Paata } \newline

1. विष्णु॑मुखा॒ वै वै विष्णु॑मुखा॒ विष्णु॑मुखा॒ वै दे॒वा दे॒वा वै विष्णु॑मुखा॒ विष्णु॑मुखा॒ वै दे॒वाः । \newline
2. विष्णु॑मुखा॒ इति॒ विष्णु॑ - मु॒खाः॒ । \newline
3. वै दे॒वा दे॒वा वै वै दे॒वा श्छन्दो॑भि॒ श्छन्दो॑भिर् दे॒वा वै वै दे॒वा श्छन्दो॑भिः । \newline
4. दे॒वा श्छन्दो॑भि॒ श्छन्दो॑भिर् दे॒वा दे॒वा श्छन्दो॑भि रि॒मा नि॒मान् छन्दो॑भिर् दे॒वा दे॒वा श्छन्दो॑भि रि॒मान् । \newline
5. छन्दो॑भि रि॒मा नि॒मान् छन्दो॑भि॒ श्छन्दो॑भि रि॒मान् ॅलो॒कान् ॅलो॒का नि॒मान् छन्दो॑भि॒ श्छन्दो॑भि रि॒मान् ॅलो॒कान् । \newline
6. छन्दो॑भि॒रिति॒ छन्दः॑ - भिः॒ । \newline
7. इ॒मान् ॅलो॒कान् ॅलो॒का नि॒मा नि॒मान् ॅलो॒का न॑नपज॒य्य म॑नपज॒य्यम् ॅलो॒का नि॒मा नि॒मान् ॅलो॒का न॑नपज॒य्यम् । \newline
8. लो॒का न॑नपज॒य्य म॑नपज॒य्यम् ॅलो॒कान् ॅलो॒का न॑नपज॒य्य म॒भ्या᳚(1॒)भ्य॑नपज॒य्यम् ॅलो॒कान् ॅलो॒का न॑नपज॒य्य म॒भि । \newline
9. अ॒न॒प॒ज॒य्य म॒भ्या᳚(1॒)भ्य॑नपज॒य्य म॑नपज॒य्य म॒भ्य॑जयन् नजयन् न॒भ्य॑नपज॒य्य म॑नपज॒य्य म॒भ्य॑जयन्न् । \newline
10. अ॒न॒प॒ज॒य्यमित्य॑नप - ज॒य्यम् । \newline
11. अ॒भ्य॑जयन् नजयन् न॒भ्या᳚(1॒)भ्य॑जय॒न्॒. यद् यद॑जयन् न॒भ्या᳚(1॒)भ्य॑जय॒न्॒. यत् । \newline
12. अ॒ज॒य॒न्॒. यद् यद॑जयन् नजय॒न्॒. यद् वि॑ष्णुक्र॒मान्. वि॑ष्णुक्र॒मान्. यद॑जयन् नजय॒न्॒. यद् वि॑ष्णुक्र॒मान् । \newline
13. यद् वि॑ष्णुक्र॒मान्. वि॑ष्णुक्र॒मान्. यद् यद् वि॑ष्णुक्र॒मान् क्रम॑ते॒ क्रम॑ते विष्णुक्र॒मान्. यद् यद् वि॑ष्णुक्र॒मान् क्रम॑ते । \newline
14. वि॒ष्णु॒क्र॒मान् क्रम॑ते॒ क्रम॑ते विष्णुक्र॒मान्. वि॑ष्णुक्र॒मान् क्रम॑ते॒ विष्णु॒र् विष्णुः॒ क्रम॑ते विष्णुक्र॒मान्. वि॑ष्णुक्र॒मान् क्रम॑ते॒ विष्णुः॑ । \newline
15. वि॒ष्णु॒क्र॒मानिति॑ विष्णु - क्र॒मान् । \newline
16. क्रम॑ते॒ विष्णु॒र् विष्णुः॒ क्रम॑ते॒ क्रम॑ते॒ विष्णु॑ रे॒वैव विष्णुः॒ क्रम॑ते॒ क्रम॑ते॒ विष्णु॑ रे॒व । \newline
17. विष्णु॑ रे॒वैव विष्णु॒र् विष्णु॑रे॒व भू॒त्वा भू॒त्वैव विष्णु॒र् विष्णु॑रे॒व भू॒त्वा । \newline
18. ए॒व भू॒त्वा भू॒त्वैवैव भू॒त्वा यज॑मानो॒ यज॑मानो भू॒त्वैवैव भू॒त्वा यज॑मानः । \newline
19. भू॒त्वा यज॑मानो॒ यज॑मानो भू॒त्वा भू॒त्वा यज॑मान॒ श्छन्दो॑भि॒ श्छन्दो॑भि॒र् यज॑मानो भू॒त्वा भू॒त्वा यज॑मान॒ श्छन्दो॑भिः । \newline
20. यज॑मान॒ श्छन्दो॑भि॒ श्छन्दो॑भि॒र् यज॑मानो॒ यज॑मान॒ श्छन्दो॑भि रि॒मा नि॒मान् छन्दो॑भि॒र् यज॑मानो॒ यज॑मान॒ श्छन्दो॑भि रि॒मान् । \newline
21. छन्दो॑भि रि॒मा नि॒मान् छन्दो॑भि॒ श्छन्दो॑भि रि॒मान् ॅलो॒कान् ॅलो॒का नि॒मान् छन्दो॑भि॒ श्छन्दो॑भि रि॒मान् ॅलो॒कान् । \newline
22. छन्दो॑भि॒रिति॒ छन्दः॑ - भिः॒ । \newline
23. इ॒मान् ॅलो॒कान् ॅलो॒का नि॒मा नि॒मान् ॅलो॒का न॑नपज॒य्य म॑नपज॒य्यम् ॅलो॒का नि॒मा नि॒मान् ॅलो॒का न॑नपज॒य्यम् । \newline
24. लो॒का न॑नपज॒य्य म॑नपज॒य्यम् ॅलो॒कान् ॅलो॒का न॑नपज॒य्य म॒भ्या᳚(1॒)भ्य॑नपज॒य्यम् ॅलो॒कान् ॅलो॒का न॑नपज॒य्य म॒भि । \newline
25. अ॒न॒प॒ज॒य्य म॒भ्या᳚(1॒)भ्य॑नपज॒य्य म॑नपज॒य्य म॒भि ज॑यति जय त्य॒भ्य॑नपज॒य्य म॑नपज॒य्य म॒भि ज॑यति । \newline
26. अ॒न॒प॒ज॒य्यमित्य॑नप - ज॒य्यम् । \newline
27. अ॒भि ज॑यति जय त्य॒भ्य॑भि ज॑यति॒ विष्णो॒र् विष्णो᳚र् जय त्य॒भ्य॑भि ज॑यति॒ विष्णोः᳚ । \newline
28. ज॒य॒ति॒ विष्णो॒र् विष्णो᳚र् जयति जयति॒ विष्णोः॒ क्रमः॒ क्रमो॒ विष्णो᳚र् जयति जयति॒ विष्णोः॒ क्रमः॑ । \newline
29. विष्णोः॒ क्रमः॒ क्रमो॒ विष्णो॒र् विष्णोः॒ क्रमो᳚ ऽस्यसि॒ क्रमो॒ विष्णो॒र् विष्णोः॒ क्रमो॑ ऽसि । \newline
30. क्रमो᳚ ऽस्यसि॒ क्रमः॒ क्रमो᳚ ऽस्यभिमाति॒हा ऽभि॑माति॒हा ऽसि॒ क्रमः॒ क्रमो᳚ ऽस्यभिमाति॒हा । \newline
31. अ॒स्य॒भि॒मा॒ति॒हा ऽभि॑माति॒हा ऽस्य॑ स्यभिमाति॒हेती त्य॑भिमाति॒हा ऽस्य॑ स्यभिमाति॒हेति॑ । \newline
32. अ॒भि॒मा॒ति॒हेती त्य॑भिमाति॒हा ऽभि॑माति॒हे त्या॑हा॒हे त्य॑भिमाति॒हा ऽभि॑माति॒हेत्या॑ह । \newline
33. अ॒भि॒मा॒ति॒हेत्य॑भिमाति - हा । \newline
34. इत्या॑हा॒हे तीत्या॑ह गाय॒त्री गा॑य॒त्र्या॑हे तीत्या॑ह गाय॒त्री । \newline
35. आ॒ह॒ गा॒य॒त्री गा॑य॒ त्र्या॑हाह गाय॒त्री वै वै गा॑य॒ त्र्या॑हाह गाय॒त्री वै । \newline
36. गा॒य॒त्री वै वै गा॑य॒त्री गा॑य॒त्री वै पृ॑थि॒वी पृ॑थि॒वी वै गा॑य॒त्री गा॑य॒त्री वै पृ॑थि॒वी । \newline
37. वै पृ॑थि॒वी पृ॑थि॒वी वै वै पृ॑थि॒वी त्रैष्टु॑भ॒म् त्रैष्टु॑भम् पृथि॒वी वै वै पृ॑थि॒वी त्रैष्टु॑भम् । \newline
38. पृ॒थि॒वी त्रैष्टु॑भ॒म् त्रैष्टु॑भम् पृथि॒वी पृ॑थि॒वी त्रैष्टु॑भ म॒न्तरि॑क्ष म॒न्तरि॑क्ष॒म् त्रैष्टु॑भम् पृथि॒वी पृ॑थि॒वी त्रैष्टु॑भ म॒न्तरि॑क्षम् । \newline
39. त्रैष्टु॑भ म॒न्तरि॑क्ष म॒न्तरि॑क्ष॒म् त्रैष्टु॑भ॒म् त्रैष्टु॑भ म॒न्तरि॑क्ष॒म् जाग॑ती॒ जाग॑ त्य॒न्तरि॑क्ष॒म् त्रैष्टु॑भ॒म् त्रैष्टु॑भ म॒न्तरि॑क्ष॒म् जाग॑ती । \newline
40. अ॒न्तरि॑क्ष॒म् जाग॑ती॒ जाग॑ त्य॒न्तरि॑क्ष म॒न्तरि॑क्ष॒म् जाग॑ती॒ द्यौर् द्यौर् जाग॑ त्य॒न्तरि॑क्ष म॒न्तरि॑क्ष॒म् जाग॑ती॒ द्यौः । \newline
41. जाग॑ती॒ द्यौर् द्यौर् जाग॑ती॒ जाग॑ती॒ द्यौरानु॑ष्टुभी॒ रानु॑ष्टुभी॒र् द्यौर् जाग॑ती॒ जाग॑ती॒ द्यौरानु॑ष्टुभीः । \newline
42. द्यौरानु॑ष्टुभी॒ रानु॑ष्टुभी॒र् द्यौर् द्यौरानु॑ष्टुभी॒र् दिशो॒ दिश॒ आनु॑ष्टुभी॒र् द्यौर् द्यौरानु॑ष्टुभी॒र् दिशः॑ । \newline
43. आनु॑ष्टुभी॒र् दिशो॒ दिश॒ आनु॑ष्टुभी॒ रानु॑ष्टुभी॒र् दिश॒ श्छन्दो॑भि॒ श्छन्दो॑भि॒र् दिश॒ आनु॑ष्टुभी॒ रानु॑ष्टुभी॒र् दिश॒ श्छन्दो॑भिः । \newline
44. आनु॑ष्टुभी॒रित्यानु॑ - स्तु॒भीः॒ । \newline
45. दिश॒ श्छन्दो॑भि॒ श्छन्दो॑भि॒र् दिशो॒ दिश॒ श्छन्दो॑भि रे॒वैव छन्दो॑भि॒र् दिशो॒ दिश॒ श्छन्दो॑भि रे॒व । \newline
46. छन्दो॑भि रे॒वैव छन्दो॑भि॒ श्छन्दो॑भि रे॒वेमा नि॒मा ने॒व छन्दो॑भि॒ श्छन्दो॑भि रे॒वेमान् । \newline
47. छन्दो॑भि॒रिति॒ छन्दः॑ - भिः॒ । \newline
48. ए॒वेमा नि॒मा ने॒वैवेमान् ॅलो॒कान् ॅलो॒का नि॒मा ने॒वैवेमान् ॅलो॒कान् । \newline
49. इ॒मान् ॅलो॒कान् ॅलो॒का नि॒मा नि॒मान् ॅलो॒कान्. य॑थापू॒र्वं ॅय॑थापू॒र्वम् ॅलो॒का नि॒मा नि॒मान् ॅलो॒कान्. य॑थापू॒र्वम् । \newline
50. लो॒कान्. य॑थापू॒र्वं ॅय॑थापू॒र्वम् ॅलो॒कान् ॅलो॒कान्. य॑थापू॒र्व म॒भ्य॑भि य॑थापू॒र्वम् ॅलो॒कान् ॅलो॒कान्. य॑थापू॒र्व म॒भि । \newline
51. य॒था॒पू॒र्व म॒भ्य॑भि य॑थापू॒र्वं ॅय॑थापू॒र्व म॒भि ज॑यति जय त्य॒भि य॑थापू॒र्वं ॅय॑थापू॒र्व म॒भि ज॑यति । \newline
52. य॒था॒पू॒र्वमिति॑ यथा - पू॒र्वम् । \newline
53. अ॒भि ज॑यति जय त्य॒भ्य॑भि ज॑यति प्र॒जाप॑तिः प्र॒जाप॑तिर् जय त्य॒भ्य॑भि ज॑यति प्र॒जाप॑तिः । \newline
54. ज॒य॒ति॒ प्र॒जाप॑तिः प्र॒जाप॑तिर् जयति जयति प्र॒जाप॑ति र॒ग्नि म॒ग्निम् प्र॒जाप॑तिर् जयति जयति प्र॒जाप॑ति र॒ग्निम् । \newline
55. प्र॒जाप॑ति र॒ग्नि म॒ग्निम् प्र॒जाप॑तिः प्र॒जाप॑ति र॒ग्नि म॑सृजता सृजता॒ग्निम् प्र॒जाप॑तिः प्र॒जाप॑ति र॒ग्नि म॑सृजत । \newline
56. प्र॒जाप॑ति॒रिति॑ प्र॒जा - प॒तिः॒ । \newline
57. अ॒ग्नि म॑सृजता सृजता॒ग्नि म॒ग्नि म॑सृजत॒ स सो॑ ऽसृजता॒ग्नि म॒ग्नि म॑सृजत॒ सः । \newline
58. अ॒सृ॒ज॒त॒ स सो॑ ऽसृजता सृजत॒ सो᳚ ऽस्मा दस्मा॒थ् सो॑ ऽसृजता सृजत॒ सो᳚ ऽस्मात् । \newline
59. सो᳚ ऽस्मा दस्मा॒थ् स सो᳚ ऽस्माथ् सृ॒ष्टः सृ॒ष्टो᳚ ऽस्मा॒थ् स सो᳚ ऽस्माथ् सृ॒ष्टः । \newline
60. अ॒स्मा॒थ् सृ॒ष्टः सृ॒ष्टो᳚ ऽस्मा दस्माथ् सृ॒ष्टः परा॒ङ् परा᳚ङ् ख्सृ॒ष्टो᳚ ऽस्मा दस्माथ् सृ॒ष्टः पराङ्॑ । \newline
61. सृ॒ष्टः परा॒ङ् परा᳚ङ् ख्सृ॒ष्टः सृ॒ष्टः परा॑ ङैदै॒त् परा᳚ङ् ख्सृ॒ष्टः सृ॒ष्टः परा॑ ङैत् । \newline
\pagebreak
\markright{ TS 5.2.1.2  \hfill https://www.vedavms.in \hfill}

\section{ TS 5.2.1.2 }

\textbf{TS 5.2.1.2 } \newline
\textbf{Samhita Paata} \newline

परा॑ङै॒त् तमे॒तया ऽन्वै॒दक्र॑न्द॒दिति॒ तया॒ वै सो᳚ऽग्नेः प्रि॒यं धामाऽवा॑रुन्ध॒ यदे॒ताम॒न्वाहा॒-ग्नेरे॒वैतया᳚ प्रि॒यं धामाऽव॑ रुन्ध ईश्व॒रो वा ए॒ष परा᳚ङ् प्र॒दघो॒ यो वि॑ष्णुक्र॒मान् क्रम॑ते चत॒सृभि॒रा व॑र्तते च॒त्वारि॒ छन्दाꣳ॑सि॒ छन्दाꣳ॑सि॒ खलु॒ वा अ॒ग्नेः प्रि॒या त॒नूः प्रि॒यामे॒वास्य॑ त॒नुव॑म॒भि - [  ] \newline

\textbf{Pada Paata} \newline

पराङ्॑ । ऐ॒त् । तम् । ए॒तया᳚ । अन्विति॑ । ऐ॒त् । अक्र॑न्दत् । इति॑ । तया᳚ । वै । सः । अ॒ग्नेः । प्रि॒यम् । धाम॑ । अवेति॑ । अ॒रु॒न्ध॒ । यत् । ए॒ताम् । अ॒न्वाहेत्य॑नु-आह॑ । अ॒ग्नेः । ए॒व । ए॒तया᳚ । प्रि॒यम् । धाम॑ । अवेति॑ । रु॒न्धे॒ । ई॒श्व॒रः । वै । ए॒षः । पराङ्॑ । प्र॒दघ॒ इति॑ प्र -दघः॑ । यः । वि॒ष्णु॒क्र॒मानिति॑ विष्णु - क्र॒मान् । क्रम॑ते । च॒त॒सृभि॒रिति॑ चत॒सृ - भिः॒ । एति॑ । व॒र्त॒ते॒ । च॒त्वारि॑ । छन्दाꣳ॑सि । छन्दाꣳ॑सि । खलु॑ । वै । अ॒ग्नेः । प्रि॒या । त॒नूः । प्रि॒याम् । ए॒व । अ॒स्य॒ । त॒नुव᳚म् । अ॒भीति॑ ।  \newline


\textbf{Krama Paata} \newline

परा॑ङैत् । ऐ॒त् तम् । तमे॒तया᳚ । ए॒तयाऽनु॑ । अन्वै᳚त् । ऐ॒दक्र॑न्दत् । अक्र॑न्द॒दिति॑ । इति॒ तया᳚ । तया॒ वै । वै सः । 
सो᳚ऽग्नेः । अ॒ग्नेः प्रि॒यम् । प्रि॒यम् धाम॑ । धामाव॑ । अवा॑रुन्ध । अ॒रु॒न्ध॒ यत् । यदे॒ताम् । ए॒ताम॒न्वाह॑ । अ॒न्वाहा॒ऽग्नेः । अ॒न्वाहेत्य॑नु - आह॑ । अ॒ग्नेरे॒व । ए॒वैतया᳚ । ए॒तया᳚ प्रि॒यम् । प्रि॒यम् धाम॑ । धामाव॑ । 
अव॑ रुन्धे । रु॒न्ध॒ ई॒श्व॒रः । ई॒श्व॒रो वै । वा ए॒षः । ए॒ष पराङ्॑ । परा᳚ङ् प्र॒दघः॑ । प्र॒दघो॒ यः । प्र॒दघ॒ इति॑ प्र - दघः॑ । यो वि॑ष्णुक्र॒मान् । वि॒ष्णु॒क्र॒मान् क्रम॑ते । वि॒ष्णु॒क्र॒मानिति॑ विष्णु - क्र॒मान् । क्रम॑ते चत॒सृभिः॑ । च॒त॒सृभि॒रा । च॒त॒सृभि॒रिति॑ चत॒सृ - भिः॒ । आ व॑र्तते । व॒र्त॒ते॒ च॒त्वारि॑ । च॒त्वारि॒ छन्दाꣳ॑सि । छन्दाꣳ॑सि॒ छन्दाꣳ॑सि । छन्दाꣳ॑सि॒ खलु॑ । खलु॒ वै । वा अ॒ग्नेः । अ॒ग्नेः प्रि॒या । प्रि॒या त॒नूः । त॒नूः प्रि॒याम् । प्रि॒यामे॒व । ए॒वास्य॑ । अ॒स्य॒ त॒नुव᳚म् । त॒नुव॑म॒भि । अ॒भि प॒र्याव॑र्तते \newline

\textbf{Jatai Paata} \newline

1. परा॑ ङैदै॒त् परा॒ङ् परा॑ ङैत् । \newline
2. ऐ॒त् तम् त मै॑दै॒त् तम् । \newline
3. त मे॒तयै॒ तया॒ तम् त मे॒तया᳚ । \newline
4. ए॒तया ऽन्वन् वे॒त यै॒तया ऽनु॑ । \newline
5. अन्वै॑ दै॒दन् वन् वै᳚त् । \newline
6. ऐ॒ दक्र॑न् द॒दक्र॑न् ददै दै॒ दक्र॑न्दत् । \newline
7. अक्र॑न् द॒दिती त्यक्र॑न्द॒ दक्र॑न्द॒ दिति॑ । \newline
8. इति॒ तया॒ तयेतीति॒ तया᳚ । \newline
9. तया॒ वै वै तया॒ तया॒ वै । \newline
10. वै स स वै वै सः । \newline
11. सो᳚ ऽग्ने र॒ग्नेः स सो᳚ ऽग्नेः । \newline
12. अ॒ग्नेः प्रि॒यम् प्रि॒य म॒ग्ने र॒ग्नेः प्रि॒यम् । \newline
13. प्रि॒यम् धाम॒ धाम॑ प्रि॒यम् प्रि॒यम् धाम॑ । \newline
14. धामा वाव॒ धाम॒ धामाव॑ । \newline
15. अवा॑ रुन्धा रु॒न्धा वावा॑ रुन्ध । \newline
16. अ॒रु॒न्ध॒ यद् यद॑रुन्धा रुन्ध॒ यत् । \newline
17. यदे॒ता मे॒तां ॅयद् यदे॒ताम् । \newline
18. ए॒ता म॒न्वा हा॒न्वा है॒ता मे॒ता म॒न्वाह॑ । \newline
19. अ॒न्वा हा॒ग्ने र॒ग्ने र॒न्वाहा॒ न्वाहा॒ग्नेः । \newline
20. अ॒न्वाहेत्य॑नु - आह॑ । \newline
21. अ॒ग्ने रे॒वै वाग्ने र॒ग्ने रे॒व । \newline
22. ए॒वैत यै॒त यै॒वै वैतया᳚ । \newline
23. ए॒तया᳚ प्रि॒यम् प्रि॒य मे॒त यै॒तया᳚ प्रि॒यम् । \newline
24. प्रि॒यम् धाम॒ धाम॑ प्रि॒यम् प्रि॒यम् धाम॑ । \newline
25. धामा वाव॒ धाम॒ धामाव॑ । \newline
26. अव॑ रुन्धे रु॒न्धे ऽवाव॑ रुन्धे । \newline
27. रु॒न्ध॒ ई॒श्व॒र ई᳚श्व॒रो रु॑न्धे रुन्ध ईश्व॒रः । \newline
28. ई॒श्व॒रो वै वा ई᳚श्व॒र ई᳚श्व॒रो वै । \newline
29. वा ए॒ष ए॒ष वै वा ए॒षः । \newline
30. ए॒ष परा॒ङ् परा॑ ङे॒ष ए॒ष पराङ्॑ । \newline
31. परा᳚ङ् प्र॒दघः॑ प्र॒दघः॒ परा॒ङ् परा᳚ङ् प्र॒दघः॑ । \newline
32. प्र॒दघो॒ यो यः प्र॒दघः॑ प्र॒दघो॒ यः । \newline
33. प्र॒दघ॒ इति॑ प्र - दघः॑ । \newline
34. यो वि॑ष्णुक्र॒मान्. वि॑ष्णुक्र॒मान्. यो यो वि॑ष्णुक्र॒मान् । \newline
35. वि॒ष्णु॒क्र॒मान् क्रम॑ते॒ क्रम॑ते विष्णुक्र॒मान्. वि॑ष्णुक्र॒मान् क्रम॑ते । \newline
36. वि॒ष्णु॒क्र॒मानिति॑ विष्णु - क्र॒मान् । \newline
37. क्रम॑ते चत॒सृभि॑ श्चत॒सृभिः॒ क्रम॑ते॒ क्रम॑ते चत॒सृभिः॑ । \newline
38. च॒त॒सृभि॒रा च॑त॒सृभि॑ श्चत॒सृभि॒रा । \newline
39. च॒त॒सृभि॒रिति॑ चत॒सृ - भिः॒ । \newline
40. आ व॑र्तते वर्तत॒ आ व॑र्तते । \newline
41. व॒र्त॒ते॒ च॒त्वारि॑ च॒त्वारि॑ वर्तते वर्तते च॒त्वारि॑ । \newline
42. च॒त्वारि॒ छन्दाꣳ॑सि॒ छन्दाꣳ॑सि च॒त्वारि॑ च॒त्वारि॒ छन्दाꣳ॑सि । \newline
43. छन्दाꣳ॑सि॒ छन्दाꣳ॑सि । \newline
44. छन्दाꣳ॑सि॒ खलु॒ खलु॒ छन्दाꣳ॑सि॒ छन्दाꣳ॑सि॒ खलु॑ । \newline
45. खलु॒ वै वै खलु॒ खलु॒ वै । \newline
46. वा अ॒ग्ने र॒ग्नेर् वै वा अ॒ग्नेः । \newline
47. अ॒ग्नेः प्रि॒या प्रि॒या ऽग्ने र॒ग्नेः प्रि॒या । \newline
48. प्रि॒या त॒नू स्त॒नूः प्रि॒या प्रि॒या त॒नूः । \newline
49. त॒नूः प्रि॒याम् प्रि॒याम् त॒नू स्त॒नूः प्रि॒याम् । \newline
50. प्रि॒या मे॒वैव प्रि॒याम् प्रि॒या मे॒व । \newline
51. ए॒वा स्या᳚ स्यै॒वैवा स्य॑ । \newline
52. अ॒स्य॒ त॒नुव॑म् त॒नुव॑ मस्यास्य त॒नुव᳚म् । \newline
53. त॒नुव॑ म॒भ्य॑भि त॒नुव॑म् त॒नुव॑ म॒भि । \newline
54. अ॒भि प॒र्याव॑र्तते प॒र्याव॑र्तते॒ ऽभ्य॑भि प॒र्याव॑र्तते । \newline

\textbf{Ghana Paata } \newline

1. परा॑ ङैदै॒त् परा॒ङ् परा॑ ङै॒त् तम् त मै॒त् परा॒ङ् परा॑ ङै॒त् तम् । \newline
2. ऐ॒त् तम् त मै॑दै॒त् त मे॒त यै॒तया॒ त मै॑दै॒त् त मे॒तया᳚ । \newline
3. त मे॒त यै॒तया॒ तम् त मे॒तया ऽन्वन् वे॒तया॒ तम् त मे॒तया ऽनु॑ । \newline
4. ए॒तया ऽन्वन् वे॒त यै॒तया ऽन्वै॑दै॒ दन्वे॒त यै॒तया ऽन्वै᳚त् । \newline
5. अन्वै॑दै॒ दन् वन् वै॒दक्र॑न्द॒ दक्र॑न्द दै॒दन् वन्वै॒ दक्र॑न्दत् । \newline
6. ऐ॒दक्र॑न्द॒ दक्र॑न्द दैदै॒ दक्र॑न्द॒दिती त्यक्र॑न्द दैदै॒ दक्र॑न्द॒ दिति॑ । \newline
7. अक्र॑न्द॒दिती त्यक्र॑न्द॒ दक्र॑न्द॒ दिति॒ तया॒ तये त्यक्र॑न्द॒ दक्र॑न्द॒ दिति॒ तया᳚ । \newline
8. इति॒ तया॒ तयेतीति॒ तया॒ वै वै तयेतीति॒ तया॒ वै । \newline
9. तया॒ वै वै तया॒ तया॒ वै स स वै तया॒ तया॒ वै सः । \newline
10. वै स स वै वै सो᳚ ऽग्ने र॒ग्नेः स वै वै सो᳚ ऽग्नेः । \newline
11. सो᳚ ऽग्ने र॒ग्नेः स सो᳚ ऽग्नेः प्रि॒यम् प्रि॒य म॒ग्नेः स सो᳚ ऽग्नेः प्रि॒यम् । \newline
12. अ॒ग्नेः प्रि॒यम् प्रि॒य म॒ग्ने र॒ग्नेः प्रि॒यम् धाम॒ धाम॑ प्रि॒य म॒ग्ने र॒ग्नेः प्रि॒यम् धाम॑ । \newline
13. प्रि॒यम् धाम॒ धाम॑ प्रि॒यम् प्रि॒यम् धामा वाव॒ धाम॑ प्रि॒यम् प्रि॒यम् धामाव॑ । \newline
14. धामावाव॒ धाम॒ धामा वा॑रुन्धा रु॒न्धाव॒ धाम॒ धामा वा॑रुन्ध । \newline
15. अवा॑रुन्धा रु॒न्धा वावा॑ रुन्ध॒ यद् यद॑रु॒न्धा वावा॑ रुन्ध॒ यत् । \newline
16. अ॒रु॒न्ध॒ यद् यद॑रुन्धा रुन्ध॒ यदे॒ता मे॒तां ॅयद॑रुन्धा रुन्ध॒ यदे॒ताम् । \newline
17. यदे॒ता मे॒तां ॅयद् यदे॒ता म॒न्वाहा॒ न्वाहै॒तां ॅयद् यदे॒ता म॒न्वाह॑ । \newline
18. ए॒ता म॒न्वाहा॒ न्वाहै॒ता मे॒ता म॒न्वाहा॒ग्ने र॒ग्ने र॒न्वाहै॒ता मे॒ता म॒न्वाहा॒ग्नेः । \newline
19. अ॒न्वाहा॒ग्ने र॒ग्ने र॒न्वाहा॒ न्वाहा॒ग्ने रे॒वैवाग्ने र॒न्वाहा॒ न्वाहा॒ग्ने रे॒व । \newline
20. अ॒न्वाहेत्य॑नु - आह॑ । \newline
21. अ॒ग्ने रे॒वैवाग्ने र॒ग्ने रे॒वैत यै॒त यै॒वाग्ने र॒ग्ने रे॒वैतया᳚ । \newline
22. ए॒वैत यै॒त यै॒वै वैतया᳚ प्रि॒यम् प्रि॒य मे॒त यै॒वै वैतया᳚ प्रि॒यम् । \newline
23. ए॒तया᳚ प्रि॒यम् प्रि॒य मे॒त यै॒तया᳚ प्रि॒यम् धाम॒ धाम॑ प्रि॒य मे॒त यै॒तया᳚ प्रि॒यम् धाम॑ । \newline
24. प्रि॒यम् धाम॒ धाम॑ प्रि॒यम् प्रि॒यम् धामावाव॒ धाम॑ प्रि॒यम् प्रि॒यम् धामाव॑ । \newline
25. धामावाव॒ धाम॒ धामाव॑ रुन्धे रु॒न्धे ऽव॒ धाम॒ धामाव॑ रुन्धे । \newline
26. अव॑ रुन्धे रु॒न्धे ऽवाव॑ रुन्ध ईश्व॒र ई᳚श्व॒रो रु॒न्धे ऽवाव॑ रुन्ध ईश्व॒रः । \newline
27. रु॒न्ध॒ ई॒श्व॒र ई᳚श्व॒रो रु॑न्धे रुन्ध ईश्व॒रो वै वा ई᳚श्व॒रो रु॑न्धे रुन्ध ईश्व॒रो वै । \newline
28. ई॒श्व॒रो वै वा ई᳚श्व॒र ई᳚श्व॒रो वा ए॒ष ए॒ष वा ई᳚श्व॒र ई᳚श्व॒रो वा ए॒षः । \newline
29. वा ए॒ष ए॒ष वै वा ए॒ष परा॒ङ् परा॑ ङे॒ष वै वा ए॒ष पराङ्॑ । \newline
30. ए॒ष परा॒ङ् परा॑ ङे॒ष ए॒ष परा᳚ङ् प्र॒दघः॑ प्र॒दघः॒ परा॑ ङे॒ष ए॒ष परा᳚ङ् प्र॒दघः॑ । \newline
31. परा᳚ङ् प्र॒दघः॑ प्र॒दघः॒ परा॒ङ् परा᳚ङ् प्र॒दघो॒ यो यः प्र॒दघः॒ परा॒ङ् परा᳚ङ् प्र॒दघो॒ यः । \newline
32. प्र॒दघो॒ यो यः प्र॒दघः॑ प्र॒दघो॒ यो वि॑ष्णुक्र॒मान्. वि॑ष्णुक्र॒मान्. यः प्र॒दघः॑ प्र॒दघो॒ यो वि॑ष्णुक्र॒मान् । \newline
33. प्र॒दघ॒ इति॑ प्र - दघः॑ । \newline
34. यो वि॑ष्णुक्र॒मान्. वि॑ष्णुक्र॒मान्. यो यो वि॑ष्णुक्र॒मान् क्रम॑ते॒ क्रम॑ते विष्णुक्र॒मान्. यो यो वि॑ष्णुक्र॒मान् क्रम॑ते । \newline
35. वि॒ष्णु॒क्र॒मान् क्रम॑ते॒ क्रम॑ते विष्णुक्र॒मान्. वि॑ष्णुक्र॒मान् क्रम॑ते चत॒सृभि॑ श्चत॒सृभिः॒ क्रम॑ते विष्णुक्र॒मान्. वि॑ष्णुक्र॒मान् क्रम॑ते चत॒सृभिः॑ । \newline
36. वि॒ष्णु॒क्र॒मानिति॑ विष्णु - क्र॒मान् । \newline
37. क्रम॑ते चत॒सृभि॑ श्चत॒सृभिः॒ क्रम॑ते॒ क्रम॑ते चत॒सृभि॒रा च॑त॒सृभिः॒ क्रम॑ते॒ क्रम॑ते चत॒सृभि॒रा । \newline
38. च॒त॒सृभि॒रा च॑त॒सृभि॑ श्चत॒सृभि॒रा व॑र्तते वर्तत॒ आ च॑त॒सृभि॑ श्चत॒सृभि॒रा व॑र्तते । \newline
39. च॒त॒सृभि॒रिति॑ चत॒सृ - भिः॒ । \newline
40. आ व॑र्तते वर्तत॒ आ व॑र्तते च॒त्वारि॑ च॒त्वारि॑ वर्तत॒ आ व॑र्तते च॒त्वारि॑ । \newline
41. व॒र्त॒ते॒ च॒त्वारि॑ च॒त्वारि॑ वर्तते वर्तते च॒त्वारि॒ छन्दाꣳ॑सि॒ छन्दाꣳ॑सि च॒त्वारि॑ वर्तते वर्तते च॒त्वारि॒ छन्दाꣳ॑सि । \newline
42. च॒त्वारि॒ छन्दाꣳ॑सि॒ छन्दाꣳ॑सि च॒त्वारि॑ च॒त्वारि॒ छन्दाꣳ॑सि । \newline
43. छन्दाꣳ॑सि॒ छन्दाꣳ॑सि । \newline
44. छन्दाꣳ॑सि॒ खलु॒ खलु॒ छन्दाꣳ॑सि॒ छन्दाꣳ॑सि॒ खलु॒ वै वै खलु॒ छन्दाꣳ॑सि॒ छन्दाꣳ॑सि॒ खलु॒ वै । \newline
45. खलु॒ वै वै खलु॒ खलु॒ वा अ॒ग्ने र॒ग्नेर् वै खलु॒ खलु॒ वा अ॒ग्नेः । \newline
46. वा अ॒ग्ने र॒ग्नेर् वै वा अ॒ग्नेः प्रि॒या प्रि॒या ऽग्नेर् वै वा अ॒ग्नेः प्रि॒या । \newline
47. अ॒ग्नेः प्रि॒या प्रि॒या ऽग्ने र॒ग्नेः प्रि॒या त॒नू स्त॒नूः प्रि॒या ऽग्ने र॒ग्नेः प्रि॒या त॒नूः । \newline
48. प्रि॒या त॒नू स्त॒नूः प्रि॒या प्रि॒या त॒नूः प्रि॒याम् प्रि॒याम् त॒नूः प्रि॒या प्रि॒या त॒नूः प्रि॒याम् । \newline
49. त॒नूः प्रि॒याम् प्रि॒याम् त॒नू स्त॒नूः प्रि॒या मे॒वैव प्रि॒याम् त॒नू स्त॒नूः प्रि॒या मे॒व । \newline
50. प्रि॒या मे॒वैव प्रि॒याम् प्रि॒या मे॒वास्या᳚ स्यै॒व प्रि॒याम् प्रि॒या मे॒वास्य॑ । \newline
51. ए॒वास्या᳚ स्यै॒वैवास्य॑ त॒नुव॑म् त॒नुव॑ मस्यै॒वैवास्य॑ त॒नुव᳚म् । \newline
52. अ॒स्य॒ त॒नुव॑म् त॒नुव॑ मस्यास्य त॒नुव॑ म॒भ्य॑भि त॒नुव॑ मस्यास्य त॒नुव॑ म॒भि । \newline
53. त॒नुव॑ म॒भ्य॑भि त॒नुव॑म् त॒नुव॑ म॒भि प॒र्याव॑र्तते प॒र्याव॑र्तते॒ ऽभि त॒नुव॑म् त॒नुव॑ म॒भि प॒र्याव॑र्तते । \newline
54. अ॒भि प॒र्याव॑र्तते प॒र्याव॑र्तते॒ ऽभ्य॑भि प॒र्याव॑र्तते दक्षि॒णा द॑क्षि॒णा प॒र्याव॑र्तते॒ ऽभ्य॑भि प॒र्याव॑र्तते दक्षि॒णा । \newline
\pagebreak
\markright{ TS 5.2.1.3  \hfill https://www.vedavms.in \hfill}

\section{ TS 5.2.1.3 }

\textbf{TS 5.2.1.3 } \newline
\textbf{Samhita Paata} \newline

प॒र्याव॑र्तते दक्षि॒णा प॒र्याव॑र्तते॒ स्वमे॒व वी॒र्य॑मनु॑ प॒र्याव॑र्तते॒ तस्मा॒द्-दक्षि॒णोऽर्द्ध॑ आ॒त्मनो॑ वी॒र्या॑वत्त॒रोऽथो॑ आदि॒त्यस्यै॒वाऽऽ*वृत॒मनु॑ प॒र्याव॑र्तते॒ शुन॒श्शेप॒माजी॑गर्तिं॒ ॅवरु॑णोऽगृह्णा॒थ् स ए॒तां ॅवा॑रु॒णीम॑पश्य॒त् तया॒ वै स आ॒त्मानं॑ ॅवरुणपा॒शाद॑मुञ्च॒द्-वरु॑णो॒ वा ए॒तं गृ॑ह्णाति॒ य उ॒खां प्र॑तिमु॒ञ्चत॒ उदु॑त्त॒मं ॅव॑रुण॒पाश॑-म॒स्मदित्या॑हा॒-ऽऽ*त्मान॑मे॒वैतया॑ - [  ] \newline

\textbf{Pada Paata} \newline

प॒र्याव॑र्तत॒ इति॑ परि - आव॑र्तते । द॒क्षि॒णा । प॒र्याव॑र्तत॒ इति॑ परि - आव॑र्तते । स्वम् । ए॒व । वी॒र्य᳚म् । अन्विति॑ । प॒र्याव॑र्तत॒ इति॑ परि - आव॑र्तते । तस्मा᳚त् । दक्षि॑णः । अद्‌र्धः॑ । आ॒त्मनः॑ । वी॒र्या॑वत्तर॒ इति॑ वी॒र्या॑वत् - त॒रः॒ । अथो॒ इति॑ । आ॒दि॒त्यस्य॑ । ए॒व । आ॒वृत॒मित्या᳚ - वृत᳚म् । अन्विति॑ । प॒र्याव॑र्तत॒ इति॑ परि - आव॑र्तते । शुन॒श्शेप᳚म् । आजी॑गर्तिम् । वरु॑णः । अ॒गृ॒ह्णा॒त् । सः । ए॒ताम् । वा॒रु॒णीम् । अ॒प॒श्य॒त् । तया᳚ । वै । सः । आ॒त्मान᳚म् । व॒रु॒ण॒पा॒शादिति॑ वरुण - पा॒शात् । अ॒मु॒ञ्च॒त् । वरु॑णः । वै । ए॒तम् । गृ॒ह्णा॒ति॒ । यः । उ॒खाम् । प्र॒ति॒मु॒ञ्चत॒ इति॑ प्रति - मु॒ञ्चते᳚ । उदिति॑ । उ॒त्त॒ममित्यु॑त् - त॒मम् । व॒रु॒ण॒ । पाश᳚म् । अ॒स्मत् । इति॑ । आ॒ह॒ । आ॒त्मान᳚म् । ए॒व । ए॒तया᳚ ।  \newline


\textbf{Krama Paata} \newline

प॒र्याव॑र्तते दक्षि॒णा । प॒र्याव॑र्तत॒ इति॑ परि - आव॑र्तते । द॒क्षि॒णा प॒र्याव॑र्तते । प॒र्याव॑र्तते॒ स्वम् । प॒र्याव॑र्तत॒ इति॑ परि - आव॑र्तते । स्वमे॒व । ए॒व वी॒र्य᳚म् । वी॒र्य॑मनु॑ । अनु॑ प॒र्याव॑र्तते । प॒र्याव॑र्तते॒ तस्मा᳚त् । प॒र्याव॑र्तत॒ इति॑ परि - आव॑र्तते । तस्मा॒द् दक्षि॑णः । दक्षि॒णोऽर्द्धः॑ । अर्द्ध॑ आ॒त्मनः॑ । आ॒त्मनो॑ वी॒र्या॑वत्तरः । वी॒र्या॑वत्त॒रोऽथो᳚ । वी॒र्या॑वत्तर॒ इति॑ वी॒र्या॑वत् - त॒रः॒ । अथो॑ आदि॒त्यस्य॑ । अथो॒ इत्यथो᳚ । आ॒दि॒त्यस्यै॒व । ए॒वावृत᳚म् । आ॒वृत॒मनु॑ । आ॒वृत॒मित्या᳚ - वृत᳚म् । अनु॑ प॒र्याव॑र्तते । प॒र्याव॑र्तते॒ शुन॒श्शेप᳚म् । प॒र्याव॑र्तत॒ इति॑ परि - आव॑र्तते । शुन॒श्शेप॒माजी॑गर्तिम् । आजी॑गर्ति॒म् वरु॑णः । वरु॑णोऽगृह्णात् । अ॒गृ॒ह्णा॒थ् सः । स ए॒ताम् । ए॒ताम् ॅवा॑रु॒णीम् । वा॒रु॒णीम॑पश्यत् । अ॒प॒श्य॒त् तया᳚ । तया॒ वै । वै सः । स आ॒त्मान᳚म् । आ॒त्मान॑म् ॅवरुणपा॒शात् । व॒रु॒ण॒पा॒शाद॑मुञ्चत् । व॒रु॒ण॒पा॒शादिति॑ वरुण - पा॒शात् । अ॒मु॒ञ्च॒द् वरु॑णः । वरु॑णो॒ वै । वा ए॒तम् । ए॒तम् गृ॑ह्णाति । गृ॒ह्णा॒ति॒ यः । य उ॒खाम् । उ॒खाम् प्र॑तिमु॒ञ्चते᳚ । प्र॒ति॒मु॒ञ्चत॒ उत् । प्र॒ति॒मु॒ञ्चत॒ इति॑ प्रति - मु॒ञ्चते᳚ । उदु॑त्त॒मम् । उ॒त्त॒मम् ॅव॑रुण । उ॒त्त॒ममित्यु॑त् - त॒मम् । व॒रु॒ण॒ पाश᳚म् । पाश॑म॒स्मत् । अ॒स्मदिति॑ । इत्या॑ह । आ॒हा॒त्मान᳚म् । आ॒त्मान॑मे॒व । ए॒वैतया᳚ । ए॒तया॑ वरुणपा॒शात् \newline

\textbf{Jatai Paata} \newline

1. प॒र्याव॑र्तते दक्षि॒णा द॑क्षि॒णा प॒र्याव॑र्तते प॒र्याव॑र्तते दक्षि॒णा । \newline
2. प॒र्याव॑र्तत॒ इति॑ परि - आव॑र्तते । \newline
3. द॒क्षि॒णा प॒र्याव॑र्तते प॒र्याव॑र्तते दक्षि॒णा द॑क्षि॒णा प॒र्याव॑र्तते । \newline
4. प॒र्याव॑र्तते॒ स्वꣳ स्वम् प॒र्याव॑र्तते प॒र्याव॑र्तते॒ स्वम् । \newline
5. प॒र्याव॑र्तत॒ इति॑ परि - आव॑र्तते । \newline
6. स्व मे॒वैव स्वꣳ स्व मे॒व । \newline
7. ए॒व वी॒र्यं॑ ॅवी॒र्य॑ मे॒वैव वी॒र्य᳚म् । \newline
8. वी॒र्य॑ मन्वनु॑ वी॒र्यं॑ ॅवी॒र्य॑ मनु॑ । \newline
9. अनु॑ प॒र्याव॑र्तते प॒र्याव॑र्त॒ते ऽन्वनु॑ प॒र्याव॑र्तते । \newline
10. प॒र्याव॑र्तते॒ तस्मा॒त् तस्मा᳚त् प॒र्याव॑र्तते प॒र्याव॑र्तते॒ तस्मा᳚त् । \newline
11. प॒र्याव॑र्तत॒ इति॑ परि - आव॑र्तते । \newline
12. तस्मा॒द् दक्षि॑णो॒ दक्षि॑ण॒ स्तस्मा॒त् तस्मा॒द् दक्षि॑णः । \newline
13. दक्षि॒णो ऽर्द्धो ऽर्द्धो॒ दक्षि॑णो॒ दक्षि॒णो ऽर्द्धः॑ । \newline
14. अर्द्ध॑ आ॒त्मन॑ आ॒त्मनो ऽर्द्धो ऽर्द्ध॑ आ॒त्मनः॑ । \newline
15. आ॒त्मनो॑ वी॒र्या॑वत्तरो वी॒र्या॑वत्तर आ॒त्मन॑ आ॒त्मनो॑ वी॒र्या॑वत्तरः । \newline
16. वी॒र्या॑वत्त॒रो ऽथो॒ अथो॑ वी॒र्या॑वत्तरो वी॒र्या॑वत्त॒रो ऽथो᳚ । \newline
17. वी॒र्या॑वत्तर॒ इति॑ वी॒र्या॑वत् - त॒रः॒ । \newline
18. अथो॑ आदि॒त्यस्या॑ दि॒त्यस्या थो॒ अथो॑ आदि॒त्यस्य॑ । \newline
19. अथो॒ इत्यथो᳚ । \newline
20. आ॒दि॒त्य स्यै॒वै वादि॒त्यस्या॑ दि॒त्य स्यै॒व । \newline
21. ए॒वावृत॑ मा॒वृत॑ मे॒वैवा वृत᳚म् । \newline
22. आ॒वृत॒ मन् वन् वा॒वृत॑ मा॒वृत॒ मनु॑ । \newline
23. आ॒वृत॒मित्या᳚ - वृत᳚म् । \newline
24. अनु॑ प॒र्याव॑र्तते प॒र्याव॑र्त॒ते ऽन्वनु॑ प॒र्याव॑र्तते । \newline
25. प॒र्याव॑र्तते॒ शुन॒श्शेपꣳ॒॒ शुन॒श्शेप॑म् प॒र्याव॑र्तते प॒र्याव॑र्तते॒ शुन॒श्शेप᳚म् । \newline
26. प॒र्याव॑र्तत॒ इति॑ परि - आव॑र्तते । \newline
27. शुन॒श्शेप॒ माजी॑गर्ति॒ माजी॑गर्तिꣳ॒॒ शुन॒श्शेपꣳ॒॒ शुन॒श्शेप॒ माजी॑गर्तिम् । \newline
28. आजी॑गर्तिं॒ ॅवरु॑णो॒ वरु॑ण॒ आजी॑गर्ति॒ माजी॑गर्तिं॒ ॅवरु॑णः । \newline
29. वरु॑णो ऽगृह्णा दगृह्णा॒द् वरु॑णो॒ वरु॑णो ऽगृह्णात् । \newline
30. अ॒गृ॒ह्णा॒थ् स सो॑ ऽगृह्णा दगृह्णा॒थ् सः । \newline
31. स ए॒ता मे॒ताꣳ स स ए॒ताम् । \newline
32. ए॒तां ॅवा॑रु॒णीं ॅवा॑रु॒णी मे॒ता मे॒तां ॅवा॑रु॒णीम् । \newline
33. वा॒रु॒णी म॑पश्य दपश्यद् वारु॒णीं ॅवा॑रु॒णी म॑पश्यत् । \newline
34. अ॒प॒श्य॒त् तया॒ तया॑ ऽपश्य दपश्य॒त् तया᳚ । \newline
35. तया॒ वै वै तया॒ तया॒ वै । \newline
36. वै स स वै वै सः । \newline
37. स आ॒त्मान॑ मा॒त्मानꣳ॒॒ स स आ॒त्मान᳚म् । \newline
38. आ॒त्मानं॑ ॅवरुणपा॒शाद् व॑रुणपा॒शा दा॒त्मान॑ मा॒त्मानं॑ ॅवरुणपा॒शात् । \newline
39. व॒रु॒ण॒पा॒शा द॑मुञ्च दमुञ्चद् वरुणपा॒शाद् व॑रुणपा॒शा द॑मुञ्चत् । \newline
40. व॒रु॒ण॒पा॒शादिति॑ वरुण - पा॒शात् । \newline
41. अ॒मु॒ञ्च॒द् वरु॑णो॒ वरु॑णो ऽमुञ्च दमुञ्च॒द् वरु॑णः । \newline
42. वरु॑णो॒ वै वै वरु॑णो॒ वरु॑णो॒ वै । \newline
43. वा ए॒त मे॒तं ॅवै वा ए॒तम् । \newline
44. ए॒तम् गृ॑ह्णाति गृह्णा त्ये॒त मे॒तम् गृ॑ह्णाति । \newline
45. गृ॒ह्णा॒ति॒ यो यो गृ॑ह्णाति गृह्णाति॒ यः । \newline
46. य उ॒खा मु॒खां ॅयो य उ॒खाम् । \newline
47. उ॒खाम् प्र॑तिमु॒ञ्चते᳚ प्रतिमु॒ञ्चत॑ उ॒खा मु॒खाम् प्र॑तिमु॒ञ्चते᳚ । \newline
48. प्र॒ति॒मु॒ञ्चत॒ उदुत् प्र॑तिमु॒ञ्चते᳚ प्रतिमु॒ञ्चत॒ उत् । \newline
49. प्र॒ति॒मु॒ञ्चत॒ इति॑ प्रति - मु॒ञ्चते᳚ । \newline
50. उदु॑त्त॒म मु॑त्त॒म मुदु दु॑त्त॒मम् । \newline
51. उ॒त्त॒मं ॅव॑रुण वरुणोत्त॒म मु॑त्त॒मं ॅव॑रुण । \newline
52. उ॒त्त॒ममित्यु॑त् - त॒मम् । \newline
53. व॒रु॒ण॒ पाश॒म् पाशं॑ ॅवरुण वरुण॒ पाश᳚म् । \newline
54. पाश॑ म॒स्म द॒स्मत् पाश॒म् पाश॑ म॒स्मत् । \newline
55. अ॒स्म दिती त्य॒स्म द॒स्म दिति॑ । \newline
56. इत्या॑हा॒हे तीत्या॑ह । \newline
57. आ॒हा॒ त्मान॑ मा॒त्मान॑ माहा हा॒त्मान᳚म् । \newline
58. आ॒त्मान॑ मे॒वै वात्मान॑ मा॒त्मान॑ मे॒व । \newline
59. ए॒वैत यै॒त यै॒वै वैतया᳚ । \newline
60. ए॒तया॑ वरुणपा॒शाद् व॑रुणपा॒शा दे॒तयै॒ तया॑ वरुणपा॒शात् । \newline

\textbf{Ghana Paata } \newline

1. प॒र्याव॑र्तते दक्षि॒णा द॑क्षि॒णा प॒र्याव॑र्तते प॒र्याव॑र्तते दक्षि॒णा प॒र्याव॑र्तते प॒र्याव॑र्तते दक्षि॒णा प॒र्याव॑र्तते प॒र्याव॑र्तते दक्षि॒णा प॒र्याव॑र्तते । \newline
2. प॒र्याव॑र्तत॒ इति॑ परि - आव॑र्तते । \newline
3. द॒क्षि॒णा प॒र्याव॑र्तते प॒र्याव॑र्तते दक्षि॒णा द॑क्षि॒णा प॒र्याव॑र्तते॒ स्वꣳ स्वम् प॒र्याव॑र्तते दक्षि॒णा द॑क्षि॒णा प॒र्याव॑र्तते॒ स्वम् । \newline
4. प॒र्याव॑र्तते॒ स्वꣳ स्वम् प॒र्याव॑र्तते प॒र्याव॑र्तते॒ स्व मे॒वैव स्वम् प॒र्याव॑र्तते प॒र्याव॑र्तते॒ स्व मे॒व । \newline
5. प॒र्याव॑र्तत॒ इति॑ परि - आव॑र्तते । \newline
6. स्व मे॒वैव स्वꣳ स्व मे॒व वी॒र्यं॑ ॅवी॒र्य॑ मे॒व स्वꣳ स्व मे॒व वी॒र्य᳚म् । \newline
7. ए॒व वी॒र्यं॑ ॅवी॒र्य॑ मे॒वैव वी॒र्य॑ मन्वनु॑ वी॒र्य॑ मे॒वैव वी॒र्य॑ मनु॑ । \newline
8. वी॒र्य॑ मन्वनु॑ वी॒र्यं॑ ॅवी॒र्य॑ मनु॑ प॒र्याव॑र्तते प॒र्याव॑र्त॒ते ऽनु॑ वी॒र्यं॑ ॅवी॒र्य॑ मनु॑ प॒र्याव॑र्तते । \newline
9. अनु॑ प॒र्याव॑र्तते प॒र्याव॑र्त॒ते ऽन्वनु॑ प॒र्याव॑र्तते॒ तस्मा॒त् तस्मा᳚त् प॒र्याव॑र्त॒ते ऽन्वनु॑ प॒र्याव॑र्तते॒ तस्मा᳚त् । \newline
10. प॒र्याव॑र्तते॒ तस्मा॒त् तस्मा᳚त् प॒र्याव॑र्तते प॒र्याव॑र्तते॒ तस्मा॒द् दक्षि॑णो॒ दक्षि॑ण॒ स्तस्मा᳚त् प॒र्याव॑र्तते प॒र्याव॑र्तते॒ तस्मा॒द् दक्षि॑णः । \newline
11. प॒र्याव॑र्तत॒ इति॑ परि - आव॑र्तते । \newline
12. तस्मा॒द् दक्षि॑णो॒ दक्षि॑ण॒ स्तस्मा॒त् तस्मा॒द् दक्षि॒णो ऽर्द्धो ऽर्द्धो॒ दक्षि॑ण॒ स्तस्मा॒त् तस्मा॒द् दक्षि॒णो ऽर्द्धः॑ । \newline
13. दक्षि॒णो ऽर्द्धो ऽर्द्धो॒ दक्षि॑णो॒ दक्षि॒णो ऽर्द्ध॑ आ॒त्मन॑ आ॒त्मनो ऽर्द्धो॒ दक्षि॑णो॒ दक्षि॒णो ऽर्द्ध॑ आ॒त्मनः॑ । \newline
14. अर्द्ध॑ आ॒त्मन॑ आ॒त्मनो ऽर्द्धो ऽर्द्ध॑ आ॒त्मनो॑ वी॒र्या॑वत्तरो वी॒र्या॑वत्तर आ॒त्मनो ऽर्द्धो ऽर्द्ध॑ आ॒त्मनो॑ वी॒र्या॑वत्तरः । \newline
15. आ॒त्मनो॑ वी॒र्या॑वत्तरो वी॒र्या॑वत्तर आ॒त्मन॑ आ॒त्मनो॑ वी॒र्या॑वत्त॒रो ऽथो॒ अथो॑ वी॒र्या॑वत्तर आ॒त्मन॑ आ॒त्मनो॑ वी॒र्या॑वत्त॒रो ऽथो᳚ । \newline
16. वी॒र्या॑वत्त॒रो ऽथो॒ अथो॑ वी॒र्या॑वत्तरो वी॒र्या॑वत्त॒रो ऽथो॑ आदि॒त्यस्या॑ दि॒त्यस्याथो॑ वी॒र्या॑वत्तरो वी॒र्या॑वत्त॒रो ऽथो॑ आदि॒त्यस्य॑ । \newline
17. वी॒र्या॑वत्तर॒ इति॑ वी॒र्या॑वत् - त॒रः॒ । \newline
18. अथो॑ आदि॒त्यस्या॑ दि॒त्यस्याथो॒ अथो॑ आदि॒त्य स्यै॒वै वादि॒त्यस्या थो॒ अथो॑ आदि॒त्यस्यै॒व । \newline
19. अथो॒ इत्यथो᳚ । \newline
20. आ॒दि॒त्य स्यै॒वै वादि॒त्यस्या॑ दि॒त्यस्यै॒ वावृत॑ मा॒वृत॑ मे॒वा दि॒त्य स्या॑दि॒त्य स्यै॒वावृत᳚म् । \newline
21. ए॒वावृत॑ मा॒वृत॑ मे॒वैवावृत॒ मन्वन् वा॒वृत॑ मे॒वैवावृत॒ मनु॑ । \newline
22. आ॒वृत॒ मन्वन् वा॒वृत॑ मा॒वृत॒ मनु॑ प॒र्याव॑र्तते प॒र्याव॑र्त॒ते ऽन्वा॒वृत॑ मा॒वृत॒ मनु॑ प॒र्याव॑र्तते । \newline
23. आ॒वृत॒मित्या᳚ - वृत᳚म् । \newline
24. अनु॑ प॒र्याव॑र्तते प॒र्याव॑र्त॒ते ऽन्वनु॑ प॒र्याव॑र्तते॒ शुन॒श्शेपꣳ॒॒ शुन॒श्शेप॑म् प॒र्याव॑र्त॒ते ऽन्वनु॑ प॒र्याव॑र्तते॒ शुन॒श्शेप᳚म् । \newline
25. प॒र्याव॑र्तते॒ शुन॒श्शेपꣳ॒॒ शुन॒श्शेप॑म् प॒र्याव॑र्तते प॒र्याव॑र्तते॒ शुन॒श्शेप॒ माजी॑गर्ति॒ माजी॑गर्तिꣳ॒॒ शुन॒श्शेप॑म् प॒र्याव॑र्तते प॒र्याव॑र्तते॒ शुन॒श्शेप॒ माजी॑गर्तिम् । \newline
26. प॒र्याव॑र्तत॒ इति॑ परि - आव॑र्तते । \newline
27. शुन॒श्शेप॒ माजी॑गर्ति॒ माजी॑गर्तिꣳ॒॒ शुन॒श्शेपꣳ॒॒ शुन॒श्शेप॒ माजी॑गर्तिं॒ ॅवरु॑णो॒ वरु॑ण॒ आजी॑गर्तिꣳ॒॒ शुन॒श्शेपꣳ॒॒ शुन॒श्शेप॒ माजी॑गर्तिं॒ ॅवरु॑णः । \newline
28. आजी॑गर्तिं॒ ॅवरु॑णो॒ वरु॑ण॒ आजी॑गर्ति॒ माजी॑गर्तिं॒ ॅवरु॑णो ऽगृह्णा दगृह्णा॒द् वरु॑ण॒ आजी॑गर्ति॒ माजी॑गर्तिं॒ ॅवरु॑णो ऽगृह्णात् । \newline
29. वरु॑णो ऽगृह्णा दगृह्णा॒द् वरु॑णो॒ वरु॑णो ऽगृह्णा॒थ् स सो॑ ऽगृह्णा॒द् वरु॑णो॒ वरु॑णो ऽगृह्णा॒थ् सः । \newline
30. अ॒गृ॒ह्णा॒थ् स सो॑ ऽगृह्णा दगृह्णा॒थ् स ए॒ता मे॒ताꣳ सो॑ ऽगृह्णा दगृह्णा॒थ् स ए॒ताम् । \newline
31. स ए॒ता मे॒ताꣳ स स ए॒तां ॅवा॑रु॒णीं ॅवा॑रु॒णी मे॒ताꣳ स स ए॒तां ॅवा॑रु॒णीम् । \newline
32. ए॒तां ॅवा॑रु॒णीं ॅवा॑रु॒णी मे॒ता मे॒तां ॅवा॑रु॒णी म॑पश्य दपश्यद् वारु॒णी मे॒ता मे॒तां ॅवा॑रु॒णी म॑पश्यत् । \newline
33. वा॒रु॒णी म॑पश्य दपश्यद् वारु॒णीं ॅवा॑रु॒णी म॑पश्य॒त् तया॒ तया॑ ऽपश्यद् वारु॒णीं ॅवा॑रु॒णी म॑पश्य॒त् तया᳚ । \newline
34. अ॒प॒श्य॒त् तया॒ तया॑ ऽपश्य दपश्य॒त् तया॒ वै वै तया॑ ऽपश्य दपश्य॒त् तया॒ वै । \newline
35. तया॒ वै वै तया॒ तया॒ वै स स वै तया॒ तया॒ वै सः । \newline
36. वै स स वै वै स आ॒त्मान॑ मा॒त्मानꣳ॒॒ स वै वै स आ॒त्मान᳚म् । \newline
37. स आ॒त्मान॑ मा॒त्मानꣳ॒॒ स स आ॒त्मानं॑ ॅवरुणपा॒शाद् व॑रुणपा॒शा दा॒त्मानꣳ॒॒ स स आ॒त्मानं॑ ॅवरुणपा॒शात् । \newline
38. आ॒त्मानं॑ ॅवरुणपा॒शाद् व॑रुणपा॒शा दा॒त्मान॑ मा॒त्मानं॑ ॅवरुणपा॒शा द॑मुञ्च दमुञ्चद् वरुणपा॒शा दा॒त्मान॑ मा॒त्मानं॑ ॅवरुणपा॒शा द॑मुञ्चत् । \newline
39. व॒रु॒ण॒पा॒शा द॑मुञ्च दमुञ्चद् वरुणपा॒शाद् व॑रुणपा॒शा द॑मुञ्च॒द् वरु॑णो॒ वरु॑णो ऽमुञ्चद् वरुणपा॒शाद् व॑रुणपा॒शा द॑मुञ्च॒द् वरु॑णः । \newline
40. व॒रु॒ण॒पा॒शादिति॑ वरुण - पा॒शात् । \newline
41. अ॒मु॒ञ्च॒द् वरु॑णो॒ वरु॑णो ऽमुञ्च दमुञ्च॒द् वरु॑णो॒ वै वै वरु॑णो ऽमुञ्च दमुञ्च॒द् वरु॑णो॒ वै । \newline
42. वरु॑णो॒ वै वै वरु॑णो॒ वरु॑णो॒ वा ए॒त मे॒तं ॅवै वरु॑णो॒ वरु॑णो॒ वा ए॒तम् । \newline
43. वा ए॒त मे॒तं ॅवै वा ए॒तम् गृ॑ह्णाति गृह्णा त्ये॒तं ॅवै वा ए॒तम् गृ॑ह्णाति । \newline
44. ए॒तम् गृ॑ह्णाति गृह्णा त्ये॒त मे॒तम् गृ॑ह्णाति॒ यो यो गृ॑ह्णा त्ये॒त मे॒तम् गृ॑ह्णाति॒ यः । \newline
45. गृ॒ह्णा॒ति॒ यो यो गृ॑ह्णाति गृह्णाति॒ य उ॒खा मु॒खां ॅयो गृ॑ह्णाति गृह्णाति॒ य उ॒खाम् । \newline
46. य उ॒खा मु॒खां ॅयो य उ॒खाम् प्र॑तिमु॒ञ्चते᳚ प्रतिमु॒ञ्चत॑ उ॒खां ॅयो य उ॒खाम् प्र॑तिमु॒ञ्चते᳚ । \newline
47. उ॒खाम् प्र॑तिमु॒ञ्चते᳚ प्रतिमु॒ञ्चत॑ उ॒खा मु॒खाम् प्र॑तिमु॒ञ्चत॒ उदुत् प्र॑तिमु॒ञ्चत॑ उ॒खा मु॒खाम् प्र॑तिमु॒ञ्चत॒ उत् । \newline
48. प्र॒ति॒मु॒ञ्चत॒ उदुत् प्र॑तिमु॒ञ्चते᳚ प्रतिमु॒ञ्चत॒ उदु॑त्त॒म मु॑त्त॒म मुत् प्र॑तिमु॒ञ्चते᳚ प्रतिमु॒ञ्चत॒ उदु॑त्त॒मम् । \newline
49. प्र॒ति॒मु॒ञ्चत॒ इति॑ प्रति - मु॒ञ्चते᳚ । \newline
50. उदु॑त्त॒म मु॑त्त॒म मुद् उदु॑त्त॒मं ॅव॑रुण वरुणोत्त॒म मुदु दु॑त्त॒मं ॅव॑रुण । \newline
51. उ॒त्त॒मं ॅव॑रुण वरुणोत्त॒म मु॑त्त॒मं ॅव॑रुण॒ पाश॒म् पाशं॑ ॅवरुणोत्त॒म मु॑त्त॒मं ॅव॑रुण॒ पाश᳚म् । \newline
52. उ॒त्त॒ममित्यु॑त् - त॒मम् । \newline
53. व॒रु॒ण॒ पाश॒म् पाशं॑ ॅवरुण वरुण॒ पाश॑ म॒स्म द॒स्मत् पाशं॑ ॅवरुण वरुण॒ पाश॑ म॒स्मत् । \newline
54. पाश॑ म॒स्म द॒स्मत् पाश॒म् पाश॑ म॒स्म दितीत्य॒स्मत् पाश॒म् पाश॑ म॒स्मदिति॑ । \newline
55. अ॒स्मदिती त्य॒स्म द॒स्म दित्या॑हा॒हे त्य॒स्म द॒स्म दित्या॑ह । \newline
56. इत्या॑हा॒हेती त्या॑हा॒त्मान॑ मा॒त्मान॑ मा॒हेती त्या॑हा॒त्मान᳚म् । \newline
57. आ॒हा॒त्मान॑ मा॒त्मान॑ माहा हा॒त्मान॑ मे॒वैवात्मान॑ माहा हा॒त्मान॑ मे॒व । \newline
58. आ॒त्मान॑ मे॒वैवात्मान॑ मा॒त्मान॑ मे॒वैत यै॒त यै॒वा त्मान॑ मा॒त्मान॑ मे॒वैतया᳚ । \newline
59. ए॒वैत यै॒त यै॒वै वैतया॑ वरुणपा॒शाद् व॑रुणपा॒शा दे॒त यै॒वै वैतया॑ वरुणपा॒शात् । \newline
60. ए॒तया॑ वरुणपा॒शाद् व॑रुणपा॒शा दे॒त यै॒तया॑ वरुणपा॒शान् मु॑ञ्चति मुञ्चति वरुणपा॒शा दे॒त यै॒तया॑ वरुणपा॒शान् मु॑ञ्चति । \newline
\pagebreak
\markright{ TS 5.2.1.4  \hfill https://www.vedavms.in \hfill}

\section{ TS 5.2.1.4 }

\textbf{TS 5.2.1.4 } \newline
\textbf{Samhita Paata} \newline

वरुणपा॒शान् मु॑ञ्च॒त्या त्वा॑ऽ*हार्.ष॒मित्या॒हा ऽऽ*ह्य॑नꣳ॒॒ हर॑ति ध्रु॒वस्ति॒ष्ठा ऽवि॑चाचलि॒रित्या॑ह॒ प्रति॑ष्ठित्यै॒ विश॑स्त्वा॒ सर्वा॑ वाञ्छ॒न्त्वित्या॑ह वि॒शैवैनꣳ॒॒ सम॑र्द्धयत्य॒स्मिन् रा॒ष्ट्रमधि॑ श्र॒येत्या॑ह रा॒ष्ट्रमे॒वास्मि॑न् ध्रु॒वम॑क॒र्यं का॒मये॑त रा॒ष्ट्रꣳ स्या॒दिति॒ तं मन॑सा ध्यायेद्-रा॒ष्ट्रमे॒व भ॑व॒त्य - [  ] \newline

\textbf{Pada Paata} \newline

व॒रु॒ण॒पा॒शादिति॑ वरुण-पा॒शात् । मु॒ञ्च॒ति॒ । एति॑ । त्वा॒ । अ॒हा॒र्॒.ष॒म् । इति॑ । आ॒ह॒ । एति॑ । हि । ए॒न॒म् । हर॑ति । ध्रु॒वः । ति॒ष्ठ॒ । अवि॑चाचलि॒रित्यवि॑ - चा॒च॒लिः॒ । इति॑ । आ॒ह॒ । प्रति॑ष्ठित्या॒ इति॒ प्रति॑ - स्थि॒त्यै॒ । विशः॑ । त्वा॒ । सर्वाः᳚ । वा॒ञ्छ॒न्तु॒ । इति॑ । आ॒ह॒ । वि॒शा । ए॒व । ए॒न॒म् । समिति॑ । अ॒द्‌र्ध॒य॒ति॒ । अ॒स्मिन्न् । रा॒ष्ट्रम् । अधीति॑ । श्र॒य॒ । इति॑ । आ॒ह॒ । रा॒ष्ट्रम् । ए॒व । अ॒स्मि॒न्न् । ध्रु॒वम् । अ॒कः॒ । यम् । का॒मये॑त । रा॒ष्ट्रम् । स्या॒त् । इति॑ । तम् । मन॑सा । ध्या॒ये॒त् । रा॒ष्ट्रम् । ए॒व । भ॒व॒ति॒ ।  \newline


\textbf{Krama Paata} \newline

व॒रु॒ण॒पा॒शान् मु॑ञ्चति । व॒रु॒ण॒पा॒शादिति॑ वरुण - पा॒शात् । मु॒ञ्च॒त्या । आ त्वा᳚ । त्वा॒ऽहा॒र्॒.ष॒म् । अ॒हा॒र्॒.ष॒मिति॑ । इत्या॑ह । आ॒हा । आ हि । ह्ये॑नम् । ए॒नꣳ॒॒ हर॑ति । हर॑ति ध्रु॒वः । ध्रु॒वस्ति॑ष्ठ । ति॒ष्ठावि॑चाचलिः । अवि॑चाचलि॒रिति॑ । अवि॑चाचलि॒रित्यवि॑ - चा॒च॒लिः॒ । इत्या॑ह । आ॒ह॒ प्रति॑ष्ठित्यै । प्रति॑ष्ठित्यै॒ विशः॑ । प्रति॑ष्ठित्या॒ इति॒ प्रति॑ - स्थि॒त्यै॒ । विश॑स्त्वा । त्वा॒ सर्वाः᳚ । सर्वा॑ वाञ्छन्तु । वा॒ञ्छ॒न्त्विति॑ । इत्या॑ह । आ॒ह॒ वि॒शा । वि॒शैव । ए॒वैन᳚म् । ए॒नꣳ॒॒ सम् । सम॑र्द्धयति । अ॒र्द्ध॒य॒त्य॒स्मिन्न् । अ॒स्मिन् रा॒ष्ट्रम् । रा॒ष्ट्रमधि॑ । अधि॑ श्रय । श्र॒येति॑ । इत्या॑ह । आ॒ह॒ रा॒ष्ट्रम् । रा॒ष्ट्रमे॒व । ए॒वास्मिन्न्॑ । अ॒स्मि॒न् ध्रु॒वम् । ध्रु॒वम॑कः । अ॒क॒र् यम् । यम् का॒मये॑त । का॒मये॑त रा॒ष्ट्रम् । रा॒ष्ट्रꣳ स्या᳚त् । स्या॒दिति॑ । इति॒ तम् । तम् मन॑सा । मन॑सा ध्यायेत् । ध्या॒ये॒द् रा॒ष्टम् । रा॒ष्ट्रमे॒व । ए॒व भ॑वति । भ॒व॒त्यग्रे᳚ \newline

\textbf{Jatai Paata} \newline

1. व॒रु॒ण॒पा॒शान् मु॑ञ्चति मुञ्चति वरुणपा॒शाद् व॑रुणपा॒शान् मु॑ञ्चति । \newline
2. व॒रु॒ण॒पा॒शादिति॑ वरुण - पा॒शात् । \newline
3. मु॒ञ्च॒त्या मु॑ञ्चति मुञ्च॒त्या । \newline
4. आ त्वा॒ त्वा ऽऽत्वा᳚ । \newline
5. त्वा॒ ऽहा॒र्॒.ष॒ म॒हा॒र्॒.ष॒म् त्वा॒ त्वा॒ ऽहा॒र्॒.ष॒म् । \newline
6. अ॒हा॒र्॒.ष॒ मिती त्य॑हार्.ष महार्.ष॒ मिति॑ । \newline
7. इत्या॑हा॒हे तीत्या॑ह । \newline
8. आ॒हा ऽऽहा॒हा । \newline
9. आ हि ह्या हि । \newline
10. ह्ये॑न मेनꣳ॒॒ हि ह्ये॑नम् । \newline
11. ए॒नꣳ॒॒ हर॑ति॒ हर॑त्येन मेनꣳ॒॒ हर॑ति । \newline
12. हर॑ति ध्रु॒वो ध्रु॒वो हर॑ति॒ हर॑ति ध्रु॒वः । \newline
13. ध्रु॒व स्ति॑ष्ठ तिष्ठ ध्रु॒वो ध्रु॒व स्ति॑ष्ठ । \newline
14. ति॒ष्ठा वि॑चाचलि॒ रवि॑चाचलि स्तिष्ठ ति॒ष्ठा वि॑चाचलिः । \newline
15. अवि॑चाचलि॒ रितीत्य वि॑चाचलि॒ रवि॑चाचलि॒ रिति॑ । \newline
16. अवि॑चाचलि॒रित्यवि॑ - चा॒च॒लिः॒ । \newline
17. इत्या॑हा॒हे तीत्या॑ह । \newline
18. आ॒ह॒ प्रति॑ष्ठित्यै॒ प्रति॑ष्ठित्या आहाह॒ प्रति॑ष्ठित्यै । \newline
19. प्रति॑ष्ठित्यै॒ विशो॒ विशः॒ प्रति॑ष्ठित्यै॒ प्रति॑ष्ठित्यै॒ विशः॑ । \newline
20. प्रति॑ष्ठित्या॒ इति॒ प्रति॑ - स्थि॒त्यै॒ । \newline
21. विश॑ स्त्वा त्वा॒ विशो॒ विश॑ स्त्वा । \newline
22. त्वा॒ सर्वाः॒ सर्वा᳚ स्त्वा त्वा॒ सर्वाः᳚ । \newline
23. सर्वा॑ वाञ्छन्तु वाञ्छन्तु॒ सर्वाः॒ सर्वा॑ वाञ्छन्तु । \newline
24. वा॒ञ्छ॒न् त्वितीति॑ वाञ्छन्तु वाञ्छ॒न् त्विति॑ । \newline
25. इत्या॑हा॒हे तीत्या॑ह । \newline
26. आ॒ह॒ वि॒शा वि॒शा ऽऽहा॑ह वि॒शा । \newline
27. वि॒शै वैव वि॒शा वि॒शैव । \newline
28. ए॒वैन॑ मेन मे॒वै वैन᳚म् । \newline
29. ए॒नꣳ॒॒ सꣳ स मे॑न मेनꣳ॒॒ सम् । \newline
30. स म॑र्द्धय त्यर्द्धयति॒ सꣳ स म॑र्द्धयति । \newline
31. अ॒र्द्ध॒य॒ त्य॒स्मिन् न॒स्मिन् न॑र्द्धय त्यर्द्धय त्य॒स्मिन्न् । \newline
32. अ॒स्मिन् रा॒ष्ट्रꣳ रा॒ष्ट्र म॒स्मिन् न॒स्मिन् रा॒ष्ट्रम् । \newline
33. रा॒ष्ट्र मध्यधि॑ रा॒ष्ट्रꣳ रा॒ष्ट्र मधि॑ । \newline
34. अधि॑ श्रय श्र॒या ध्यधि॑ श्रय । \newline
35. श्र॒ये तीति॑ श्रय श्र॒येति॑ । \newline
36. इत्या॑हा॒हे तीत्या॑ह । \newline
37. आ॒ह॒ रा॒ष्ट्रꣳ रा॒ष्ट्र मा॑हाह रा॒ष्ट्रम् । \newline
38. रा॒ष्ट्र मे॒वैव रा॒ष्ट्रꣳ रा॒ष्ट्र मे॒व । \newline
39. ए॒वास्मि॑न् नस्मिन् ने॒वै वास्मिन्न्॑ । \newline
40. अ॒स्मि॒न् ध्रु॒वम् ध्रु॒व म॑स्मिन् नस्मिन् ध्रु॒वम् । \newline
41. ध्रु॒व म॑क रकर् ध्रु॒वम् ध्रु॒व म॑कः । \newline
42. अ॒क॒र् यं ॅय म॑क रक॒र् यम् । \newline
43. यम् का॒मये॑त का॒मये॑त॒ यं ॅयम् का॒मये॑त । \newline
44. का॒मये॑त रा॒ष्ट्रꣳ रा॒ष्ट्रम् का॒मये॑त का॒मये॑त रा॒ष्ट्रम् । \newline
45. रा॒ष्ट्रꣳ स्या᳚थ् स्याद् रा॒ष्ट्रꣳ रा॒ष्ट्रꣳ स्या᳚त् । \newline
46. स्या॒ दितीति॑ स्याथ् स्या॒ दिति॑ । \newline
47. इति॒ तम् त मितीति॒ तम् । \newline
48. तम् मन॑सा॒ मन॑सा॒ तम् तम् मन॑सा । \newline
49. मन॑सा ध्यायेद् ध्याये॒न् मन॑सा॒ मन॑सा ध्यायेत् । \newline
50. ध्या॒ये॒द् रा॒ष्ट्रꣳ रा॒ष्ट्रम् ध्या॑येद् ध्यायेद् रा॒ष्ट्रम् । \newline
51. रा॒ष्ट्र मे॒वैव रा॒ष्ट्रꣳ रा॒ष्ट्र मे॒व । \newline
52. ए॒व भ॑वति भव त्ये॒वैव भ॑वति । \newline
53. भ॒व॒ त्यग्रे ऽग्रे॑ भवति भव॒ त्यग्रे᳚ । \newline

\textbf{Ghana Paata } \newline

1. व॒रु॒ण॒पा॒शान् मु॑ञ्चति मुञ्चति वरुणपा॒शाद् व॑रुणपा॒शान् मु॑ञ्च॒त्या मु॑ञ्चति वरुणपा॒शाद् व॑रुणपा॒शान् मु॑ञ्च॒त्या । \newline
2. व॒रु॒ण॒पा॒शादिति॑ वरुण - पा॒शात् । \newline
3. मु॒ञ्च॒त्या मु॑ञ्चति मुञ्च॒त्या त्वा॒ त्वा ऽऽमु॑ञ्चति मुञ्च॒त्या त्वा᳚ । \newline
4. आ त्वा॒ त्वा ऽऽत्वा॑ ऽहार्.ष महार्.ष॒म् त्वा ऽऽत्वा॑ ऽहार्.षम् । \newline
5. त्वा॒ ऽहा॒र्॒.ष॒ म॒हा॒र्॒.ष॒म् त्वा॒ त्वा॒ ऽहा॒र्॒.ष॒ मिती त्य॑हार्.षम् त्वा त्वा ऽहार्.ष॒ मिति॑ । \newline
6. अ॒हा॒र्॒.ष॒ मिती त्य॑हार्.ष महार्.ष॒ मित्या॑हा॒हे त्य॑हार्.ष महार्.ष॒ मित्या॑ह । \newline
7. इत्या॑हा॒हे तीत्या॒हा ऽऽहे तीत्या॒हा । \newline
8. आ॒हा ऽऽहा॒हा हि ह्या ऽऽहा॒हा हि । \newline
9. आ हि ह्या ह्ये॑न मेनꣳ॒॒ ह्या ह्ये॑नम् । \newline
10. ह्ये॑न मेनꣳ॒॒ हि ह्ये॑नꣳ॒॒ हर॑ति॒ हर॑ त्येनꣳ॒॒ हि ह्ये॑नꣳ॒॒ हर॑ति । \newline
11. ए॒नꣳ॒॒ हर॑ति॒ हर॑ त्येन मेनꣳ॒॒ हर॑ति ध्रु॒वो ध्रु॒वो हर॑ त्येन मेनꣳ॒॒ हर॑ति ध्रु॒वः । \newline
12. हर॑ति ध्रु॒वो ध्रु॒वो हर॑ति॒ हर॑ति ध्रु॒व स्ति॑ष्ठ तिष्ठ ध्रु॒वो हर॑ति॒ हर॑ति ध्रु॒व स्ति॑ष्ठ । \newline
13. ध्रु॒व स्ति॑ष्ठ तिष्ठ ध्रु॒वो ध्रु॒व स्ति॒ष्ठा वि॑चाचलि॒ रवि॑चाचलि स्तिष्ठ ध्रु॒वो ध्रु॒व स्ति॒ष्ठा वि॑चाचलिः । \newline
14. ति॒ष्ठा वि॑चाचलि॒ रवि॑चाचलि स्तिष्ठ ति॒ष्ठा वि॑चाचलि॒ रिती त्यवि॑चाचलि स्तिष्ठ ति॒ष्ठा वि॑चाचलि॒रिति॑ । \newline
15. अवि॑चाचलि॒ रिती त्यवि॑चाचलि॒ रवि॑चाचलि॒ रित्या॑हा॒हे त्यवि॑चाचलि॒ रवि॑चाचलि॒ रित्या॑ह । \newline
16. अवि॑चाचलि॒रित्यवि॑ - चा॒च॒लिः॒ । \newline
17. इत्या॑हा॒हे तीत्या॑ह॒ प्रति॑ष्ठित्यै॒ प्रति॑ष्ठित्या आ॒हे तीत्या॑ह॒ प्रति॑ष्ठित्यै । \newline
18. आ॒ह॒ प्रति॑ष्ठित्यै॒ प्रति॑ष्ठित्या आहाह॒ प्रति॑ष्ठित्यै॒ विशो॒ विशः॒ प्रति॑ष्ठित्या आहाह॒ प्रति॑ष्ठित्यै॒ विशः॑ । \newline
19. प्रति॑ष्ठित्यै॒ विशो॒ विशः॒ प्रति॑ष्ठित्यै॒ प्रति॑ष्ठित्यै॒ विश॑ स्त्वा त्वा॒ विशः॒ प्रति॑ष्ठित्यै॒ प्रति॑ष्ठित्यै॒ विश॑ स्त्वा । \newline
20. प्रति॑ष्ठित्या॒ इति॒ प्रति॑ - स्थि॒त्यै॒ । \newline
21. विश॑ स्त्वा त्वा॒ विशो॒ विश॑ स्त्वा॒ सर्वाः॒ सर्वा᳚ स्त्वा॒ विशो॒ विश॑ स्त्वा॒ सर्वाः᳚ । \newline
22. त्वा॒ सर्वाः॒ सर्वा᳚ स्त्वा त्वा॒ सर्वा॑ वाञ्छन्तु वाञ्छन्तु॒ सर्वा᳚ स्त्वा त्वा॒ सर्वा॑ वाञ्छन्तु । \newline
23. सर्वा॑ वाञ्छन्तु वाञ्छन्तु॒ सर्वाः॒ सर्वा॑ वाञ्छ॒न्त्वितीति॑ वाञ्छन्तु॒ सर्वाः॒ सर्वा॑ वाञ्छ॒न्त्विति॑ । \newline
24. वा॒ञ्छ॒न्त्वितीति॑ वाञ्छन्तु वाञ्छ॒न्त्वि त्या॑हा॒हेति॑ वाञ्छन्तु वाञ्छ॒न्त्वित्या॑ह । \newline
25. इत्या॑हा॒हे तीत्या॑ह वि॒शा वि॒शा ऽऽहेतीत्या॑ह वि॒शा । \newline
26. आ॒ह॒ वि॒शा वि॒शा ऽऽहा॑ह वि॒शैवैव वि॒शा ऽऽहा॑ह वि॒शैव । \newline
27. वि॒शैवैव वि॒शा वि॒शैवैन॑ मेन मे॒व वि॒शा वि॒शैवैन᳚म् । \newline
28. ए॒वैन॑ मेन मे॒वैवैनꣳ॒॒ सꣳ स मे॑न मे॒वैवैनꣳ॒॒ सम् । \newline
29. ए॒नꣳ॒॒ सꣳ स मे॑न मेनꣳ॒॒ स म॑र्द्धय त्यर्द्धयति॒ स मे॑न मेनꣳ॒॒ स म॑र्द्धयति । \newline
30. स म॑र्द्धय त्यर्द्धयति॒ सꣳ स म॑र्द्धय त्य॒स्मिन् न॒स्मिन् न॑र्द्धयति॒ सꣳ स म॑र्द्धय त्य॒स्मिन्न् । \newline
31. अ॒र्द्ध॒य॒ त्य॒स्मिन् न॒स्मिन् न॑र्द्धय त्यर्द्धय त्य॒स्मिन् रा॒ष्ट्रꣳ रा॒ष्ट्र म॒स्मिन् न॑र्द्धय त्यर्द्धय त्य॒स्मिन् रा॒ष्ट्रम् । \newline
32. अ॒स्मिन् रा॒ष्ट्रꣳ रा॒ष्ट्र म॒स्मिन् न॒स्मिन् रा॒ष्ट्र मध्यधि॑ रा॒ष्ट्र म॒स्मिन् न॒स्मिन् रा॒ष्ट्र मधि॑ । \newline
33. रा॒ष्ट्र मध्यधि॑ रा॒ष्ट्रꣳ रा॒ष्ट्र मधि॑ श्रय श्र॒याधि॑ रा॒ष्ट्रꣳ रा॒ष्ट्र मधि॑ श्रय । \newline
34. अधि॑ श्रय श्र॒याध्यधि॑ श्र॒ये तीति॑ श्र॒याध्यधि॑ श्र॒येति॑ । \newline
35. श्र॒ये तीति॑ श्रय श्र॒ये त्या॑हा॒हेति॑ श्रय श्र॒ये त्या॑ह । \newline
36. इत्या॑हा॒हे तीत्या॑ह रा॒ष्ट्रꣳ रा॒ष्ट्र मा॒हे तीत्या॑ह रा॒ष्ट्रम् । \newline
37. आ॒ह॒ रा॒ष्ट्रꣳ रा॒ष्ट्र मा॑हाह रा॒ष्ट्र मे॒वैव रा॒ष्ट्र मा॑हाह रा॒ष्ट्र मे॒व । \newline
38. रा॒ष्ट्र मे॒वैव रा॒ष्ट्रꣳ रा॒ष्ट्र मे॒वास्मि॑न् नस्मिन् ने॒व रा॒ष्ट्रꣳ रा॒ष्ट्र मे॒वास्मिन्न्॑ । \newline
39. ए॒वास्मि॑न् नस्मिन् ने॒वैवास्मि॑न् ध्रु॒वम् ध्रु॒व म॑स्मिन् ने॒वैवास्मि॑न् ध्रु॒वम् । \newline
40. अ॒स्मि॒न् ध्रु॒वम् ध्रु॒व म॑स्मिन् नस्मिन् ध्रु॒व म॑क रकर् ध्रु॒व म॑स्मिन् नस्मिन् ध्रु॒व म॑कः । \newline
41. ध्रु॒व म॑क रकर् ध्रु॒वम् ध्रु॒व म॑क॒र् यं ॅय म॑कर् ध्रु॒वम् ध्रु॒व म॑क॒र् यम् । \newline
42. अ॒क॒र् यं ॅय म॑क रक॒र् यम् का॒मये॑त का॒मये॑त॒ य म॑क रक॒र् यम् का॒मये॑त । \newline
43. यम् का॒मये॑त का॒मये॑त॒ यं ॅयम् का॒मये॑त रा॒ष्ट्रꣳ रा॒ष्ट्रम् का॒मये॑त॒ यं ॅयम् का॒मये॑त रा॒ष्ट्रम् । \newline
44. का॒मये॑त रा॒ष्ट्रꣳ रा॒ष्ट्रम् का॒मये॑त का॒मये॑त रा॒ष्ट्रꣳ स्या᳚थ् स्याद् रा॒ष्ट्रम् का॒मये॑त का॒मये॑त रा॒ष्ट्रꣳ स्या᳚त् । \newline
45. रा॒ष्ट्रꣳ स्या᳚थ् स्याद् रा॒ष्ट्रꣳ रा॒ष्ट्रꣳ स्या॒दितीति॑ स्याद् रा॒ष्ट्रꣳ रा॒ष्ट्रꣳ स्या॒दिति॑ । \newline
46. स्या॒दितीति॑ स्याथ् स्या॒दिति॒ तम् तमिति॑ स्याथ् स्या॒दिति॒ तम् । \newline
47. इति॒ तम् तमितीति॒ तम् मन॑सा॒ मन॑सा॒ तमितीति॒ तम् मन॑सा । \newline
48. तम् मन॑सा॒ मन॑सा॒ तम् तम् मन॑सा ध्यायेद् ध्याये॒न् मन॑सा॒ तम् तम् मन॑सा ध्यायेत् । \newline
49. मन॑सा ध्यायेद् ध्याये॒न् मन॑सा॒ मन॑सा ध्यायेद् रा॒ष्ट्रꣳ रा॒ष्ट्रम् ध्या॑ये॒न् मन॑सा॒ मन॑सा ध्यायेद् रा॒ष्ट्रम् । \newline
50. ध्या॒ये॒द् रा॒ष्ट्रꣳ रा॒ष्ट्रम् ध्या॑येद् ध्यायेद् रा॒ष्ट्र मे॒वैव रा॒ष्ट्रम् ध्या॑येद् ध्यायेद् रा॒ष्ट्र मे॒व । \newline
51. रा॒ष्ट्र मे॒वैव रा॒ष्ट्रꣳ रा॒ष्ट्र मे॒व भ॑वति भवत्ये॒व रा॒ष्ट्रꣳ रा॒ष्ट्र मे॒व भ॑वति । \newline
52. ए॒व भ॑वति भव त्ये॒वैव भ॑व॒त्यग्रे ऽग्रे॑ भव त्ये॒वैव भ॑व॒त्यग्रे᳚ । \newline
53. भ॒व॒त्यग्रे ऽग्रे॑ भवति भव॒त्यग्रे॑ बृ॒हन् बृ॒हन् नग्रे॑ भवति भव॒त्यग्रे॑ बृ॒हन्न् । \newline
\pagebreak
\markright{ TS 5.2.1.5  \hfill https://www.vedavms.in \hfill}

\section{ TS 5.2.1.5 }

\textbf{TS 5.2.1.5 } \newline
\textbf{Samhita Paata} \newline

-ग्रे॑ बृ॒हन्नु॒षसा॑मू॒र्द्ध्वो अ॑स्था॒दित्या॒हाऽग्र॑मे॒वैनꣳ॑ समा॒नानां᳚ करोति निर्जग्मि॒वान् तम॑स॒ इत्या॑ह॒ तम॑ ए॒वास्मा॒दप॑ हन्ति॒ ज्योति॒षा- ऽऽगा॒दित्या॑ह॒ ज्योति॑रे॒वा-स्मि॑न् दधाति चत॒सृभिः॑ सादयति च॒त्वारि॒ छन्दाꣳ॑सि॒ छन्दो॑भिरे॒वाऽ-ति॑च्छन्दसोत्त॒मया॒ वर्ष्म॒ वा ए॒षा छन्द॑सां॒ ॅयदति॑च्छन्दा॒ वर्ष्मै॒वैनꣳ॑ समा॒नानां᳚ करोति॒ सद्व॑ती - [  ] \newline

\textbf{Pada Paata} \newline

अग्रे᳚ । बृ॒हन्न् । उ॒षसा᳚म् । ऊ॒द्‌र्ध्वः । अ॒स्था॒त् । इति॑ । आ॒ह॒ । अग्र᳚म् । ए॒व । ए॒न॒म् । स॒मा॒नाना᳚म् । क॒रो॒ति॒ । नि॒र्ज॒ग्मि॒वानिति॑ निः-ज॒ग्मि॒वान् । तम॑सः । इति॑ । आ॒ह॒ । तमः॑ । ए॒व । अ॒स्मा॒त् । अपेति॑ । ह॒न्ति॒ । ज्योति॑षा । एति॑ । अ॒गा॒त् । इति॑ । आ॒ह॒ । ज्योतिः॑ । ए॒व । अ॒स्मि॒न्न् । द॒धा॒ति॒ । च॒त॒सृभि॒रिति॑ चत॒सृ - भिः॒ । सा॒द॒य॒ति॒ । च॒त्वारि॑ । छन्दाꣳ॑सि । छन्दो॑भि॒रिति॒ छन्दः॑ - भिः॒ । ए॒व । अति॑च्छन्द॒सेत्यति॑ - छ॒न्द॒सा॒ । उ॒त्त॒मयेत्यु॑त् - त॒मया᳚ । वर्ष्म॑ । वै । ए॒षा । छन्द॑साम् । यत् । अति॑च्छन्दा॒ इत्यति॑ - छ॒न्दाः॒ । वर्ष्म॑ । ए॒व । ए॒न॒म् । स॒मा॒नाना᳚म् । क॒रो॒ति॒ । सद्व॒तीति॒ सत् - व॒ती॒ ।  \newline


\textbf{Krama Paata} \newline

अग्रे॑ बृ॒हन्न् । बृ॒हन्नु॒षसा᳚म् । उ॒षसा॑मू॒र्द्ध्वः । ऊ॒र्द्ध्वो अ॑स्थात् । अ॒स्था॒दिति॑ । इत्या॑ह । आ॒हाग्र᳚म् । अग्र॑मे॒व । ए॒वैन᳚म् । 
ए॒नꣳ॒॒ स॒मा॒नाना᳚म् । स॒मा॒नाना᳚म् करोति । क॒रो॒ति॒ नि॒र्ज॒ग्मि॒वान् । नि॒र्ज॒ग्मि॒वान् तम॑सः । नि॒र्ज॒ग्मि॒वानिति॑ निः - ज॒ग्मि॒वान् । तम॑स॒ इति॑ । इत्या॑ह । आ॒ह॒ तमः॑ । तम॑ ए॒व । ए॒वास्मा᳚त् । अ॒स्मा॒दप॑ । अप॑ हन्ति । ह॒न्ति॒ ज्योति॑षा । ज्योति॒षा । आऽगा᳚त् । अ॒गा॒दिति॑ । इत्या॑ह । आ॒ह॒ ज्योतिः॑ । ज्योति॑रे॒व । ए॒वास्मिन्न्॑ । अ॒स्मि॒न् द॒धा॒ति॒ । द॒धा॒ति॒ च॒त॒सृभिः॑ । च॒त॒सृभिः॑ सादयति । च॒त॒सृभि॒रिति॑ चत॒सृ - भिः॒ । सा॒द॒य॒ति॒ च॒त्वारि॑ । च॒त्वारि॒ छन्दाꣳ॑सि । छन्दाꣳ॑सि॒ छन्दो॑भिः । छन्दो॑भिरे॒व । छन्दो॑भि॒रिति॒ छन्दः॑ - भिः॒ । ए॒वाति॑च्छन्दसा । अति॑च्छन्दसोत्त॒मया᳚ । अति॑च्छन्द॒सेत्यति॑ - छ॒न्द॒सा॒ । उ॒त्त॒मया॒ वर्ष्म॑ । उ॒त्त॒मयेत्यु॑त् - त॒मया᳚ । वर्ष्म॒ वै । वा ए॒षा । ए॒षा छन्द॑साम् । छन्द॑सा॒म् ॅयत् । यदति॑च्छन्दाः । अति॑च्छन्दा॒ वर्ष्म॑ । अति॑च्छन्दा॒ इत्यति॑ - छ॒न्दाः॒ । वर्ष्मै॒व । ए॒वैन᳚म् । ए॒नꣳ॒॒ स॒मा॒नाना᳚म् । स॒मा॒नाना᳚म् करोति । क॒रो॒ति॒ सद्व॑ती । सद्व॑ती भवति । सद्व॒तीति॒ सत् - व॒ती॒ \newline

\textbf{Jatai Paata} \newline

1. अग्रे॑ बृ॒हन् बृ॒हन् नग्रे॒ अग्रे॑ बृ॒हन्न् । \newline
2. बृ॒हन् नु॒षसा॑ मु॒षसा᳚म् बृ॒हन् बृ॒हन् नु॒षसा᳚म् । \newline
3. उ॒षसा॑ मू॒र्द्ध्व ऊ॒र्द्ध्व उ॒षसा॑ मु॒षसा॑ मू॒र्द्ध्वः । \newline
4. ऊ॒र्द्ध्वो अ॑स्था दस्था दू॒र्द्ध्व ऊ॒र्द्ध्वो अ॑स्थात् । \newline
5. अ॒स्था॒ दितीत्य॑ स्था दस्था॒ दिति॑ । \newline
6. इत्या॑हा॒हे तीत्या॑ह । \newline
7. आ॒हाग्र॒ मग्र॑ माहा॒हाग्र᳚म् । \newline
8. अग्र॑ मे॒वै वाग्र॒ मग्र॑ मे॒व । \newline
9. ए॒वैन॑ मेन मे॒वै वैन᳚म् । \newline
10. ए॒नꣳ॒॒ स॒मा॒नानाꣳ॑ समा॒नाना॑ मेन मेनꣳ समा॒नाना᳚म् । \newline
11. स॒मा॒नाना᳚म् करोति करोति समा॒नानाꣳ॑ समा॒नाना᳚म् करोति । \newline
12. क॒रो॒ति॒ नि॒र्ज॒ग्मि॒वान् नि॑र्जग्मि॒वान् क॑रोति करोति निर्जग्मि॒वान् । \newline
13. नि॒र्ज॒ग्मि॒वान् तम॑स॒ स्तम॑सो निर्जग्मि॒वान् नि॑र्जग्मि॒वान् तम॑सः । \newline
14. नि॒र्ज॒ग्मि॒वानिति॑ निः - ज॒ग्मि॒वान् । \newline
15. तम॑स॒ इतीति॒ तम॑स॒ स्तम॑स॒ इति॑ । \newline
16. इत्या॑हा॒हे तीत्या॑ह । \newline
17. आ॒ह॒ तम॒ स्तम॑ आहाह॒ तमः॑ । \newline
18. तम॑ ए॒वैव तम॒ स्तम॑ ए॒व । \newline
19. ए॒वास्मा॑ दस्मा दे॒वै वास्मा᳚त् । \newline
20. अ॒स्मा॒ दपापा᳚ स्मा दस्मा॒ दप॑ । \newline
21. अप॑ हन्ति ह॒न्त्य पाप॑ हन्ति । \newline
22. ह॒न्ति॒ ज्योति॑षा॒ ज्योति॑षा हन्ति हन्ति॒ ज्योति॑षा । \newline
23. ज्योति॒षा ऽऽज्योति॑षा॒ ज्योति॒षा । \newline
24. आ ऽगा॑ दगा॒ दा ऽगा᳚त् । \newline
25. अ॒गा॒ दिती त्य॑गा दगा॒ दिति॑ । \newline
26. इत्या॑हा॒हे तीत्या॑ह । \newline
27. आ॒ह॒ ज्योति॒र् ज्योति॑ राहाह॒ ज्योतिः॑ । \newline
28. ज्योति॑ रे॒वैव ज्योति॒र् ज्योति॑ रे॒व । \newline
29. ए॒वास्मि॑न् नस्मिन् ने॒वै वास्मिन्न्॑ । \newline
30. अ॒स्मि॒न् द॒धा॒ति॒ द॒धा॒ त्य॒स्मि॒न् न॒स्मि॒न् द॒धा॒ति॒ । \newline
31. द॒धा॒ति॒ च॒त॒सृभि॑ श्चत॒सृभि॑र् दधाति दधाति चत॒सृभिः॑ । \newline
32. च॒त॒सृभिः॑ सादयति सादयति चत॒सृभि॑ श्चत॒सृभिः॑ सादयति । \newline
33. च॒त॒सृभि॒रिति॑ चत॒सृ - भिः॒ । \newline
34. सा॒द॒य॒ति॒ च॒त्वारि॑ च॒त्वारि॑ सादयति सादयति च॒त्वारि॑ । \newline
35. च॒त्वारि॒ छन्दाꣳ॑सि॒ छन्दाꣳ॑सि च॒त्वारि॑ च॒त्वारि॒ छन्दाꣳ॑सि । \newline
36. छन्दाꣳ॑सि॒ छन्दो॑भि॒ श्छन्दो॑भि॒ श्छन्दाꣳ॑सि॒ छन्दाꣳ॑सि॒ छन्दो॑भिः । \newline
37. छन्दो॑भि रे॒वैव छन्दो॑भि॒ श्छन्दो॑भि रे॒व । \newline
38. छन्दो॑भि॒रिति॒ छन्दः॑ - भिः॒ । \newline
39. ए॒वा ति॑च्छन्द॒सा ऽति॑च्छन्द सै॒वैवा ति॑च्छन्दसा । \newline
40. अति॑च्छन्दसो त्त॒मयो᳚ त्त॒मया ऽति॑च्छन्द॒सा ऽति॑च्छन्दसो त्त॒मया᳚ । \newline
41. अति॑च्छन्द॒सेत्यति॑ - छ॒न्द॒सा॒ । \newline
42. उ॒त्त॒मया॒ वर्ष्म॒ वर्ष्मो᳚ त्त॒मयो᳚ त्त॒मया॒ वर्ष्म॑ । \newline
43. उ॒त्त॒मयेत्यु॑त् - त॒मया᳚ । \newline
44. वर्ष्म॒ वै वै वर्ष्म॒ वर्ष्म॒ वै । \newline
45. वा ए॒षैषा वै वा ए॒षा । \newline
46. ए॒षा छन्द॑सा॒म् छन्द॑सा मे॒षैषा छन्द॑साम् । \newline
47. छन्द॑सां॒ ॅयद् यच् छन्द॑सा॒म् छन्द॑सां॒ ॅयत् । \newline
48. य दति॑च्छन्दा॒ अति॑च्छन्दा॒ यद् य दति॑च्छन्दाः । \newline
49. अति॑च्छन्दा॒ वर्ष्म॒ वर्ष्माति॑च्छन्दा॒ अति॑च्छन्दा॒ वर्ष्म॑ । \newline
50. अति॑च्छन्दा॒ इत्यति॑ - छ॒न्दाः॒ । \newline
51. वर्ष्मै॒ वैव वर्ष्म॒ वर्ष्मै॒व । \newline
52. ए॒वैन॑ मेन मे॒वै वैन᳚म् । \newline
53. ए॒नꣳ॒॒ स॒मा॒नानाꣳ॑ समा॒नाना॑ मेन मेनꣳ समा॒नाना᳚म् । \newline
54. स॒मा॒नाना᳚म् करोति करोति समा॒नानाꣳ॑ समा॒नाना᳚म् करोति । \newline
55. क॒रो॒ति॒ सद्व॑ती॒ सद्व॑ती करोति करोति॒ सद्व॑ती । \newline
56. सद्व॑ती भवति भवति॒ सद्व॑ती॒ सद्व॑ती भवति । \newline
57. सद्व॒तीति॒ सत् - व॒ती॒ । \newline

\textbf{Ghana Paata } \newline

1. अग्रे॑ बृ॒हन् बृ॒हन् नग्रे॒ अग्रे॑ बृ॒हन् नु॒षसा॑ मु॒षसा᳚म् बृ॒हन् नग्रे॒ अग्रे॑ बृ॒हन् नु॒षसा᳚म् । \newline
2. बृ॒हन् नु॒षसा॑ मु॒षसा᳚म् बृ॒हन् बृ॒हन् नु॒षसा॑ मू॒र्द्ध्व ऊ॒र्द्ध्व उ॒षसा᳚म् बृ॒हन् बृ॒हन् नु॒षसा॑ मू॒र्द्ध्वः । \newline
3. उ॒षसा॑ मू॒र्द्ध्व ऊ॒र्द्ध्व उ॒षसा॑ मु॒षसा॑ मू॒र्द्ध्वो अ॑स्था दस्था दू॒र्द्ध्व उ॒षसा॑ मु॒षसा॑ मू॒र्द्ध्वो अ॑स्थात् । \newline
4. ऊ॒र्द्ध्वो अ॑स्था दस्था दू॒र्द्ध्व ऊ॒र्द्ध्वो अ॑स्था॒ दिती त्य॑स्था दू॒र्द्ध्व ऊ॒र्द्ध्वो अ॑स्था॒दिति॑ । \newline
5. अ॒स्था॒ दितीत्य॑स्था दस्था॒ दित्या॑हा॒हे त्य॑स्था दस्था॒ दित्या॑ह । \newline
6. इत्या॑हा॒हे तीत्या॒हाग्र॒ मग्र॑ मा॒हे तीत्या॒हाग्र᳚म् । \newline
7. आ॒हाग्र॒ मग्र॑ माहा॒हाग्र॑ मे॒वैवाग्र॑ माहा॒हाग्र॑ मे॒व । \newline
8. अग्र॑ मे॒वैवाग्र॒ मग्र॑ मे॒वैन॑ मेन मे॒वाग्र॒ मग्र॑ मे॒वैन᳚म् । \newline
9. ए॒वैन॑ मेन मे॒वैवैनꣳ॑ समा॒नानाꣳ॑ समा॒नाना॑ मेन मे॒वैवैनꣳ॑ समा॒नाना᳚म् । \newline
10. ए॒नꣳ॒॒ स॒मा॒नानाꣳ॑ समा॒नाना॑ मेन मेनꣳ समा॒नाना᳚म् करोति करोति समा॒नाना॑ मेन मेनꣳ समा॒नाना᳚म् करोति । \newline
11. स॒मा॒नाना᳚म् करोति करोति समा॒नानाꣳ॑ समा॒नाना᳚म् करोति निर्जग्मि॒वान् नि॑र्जग्मि॒वान् क॑रोति समा॒नानाꣳ॑ समा॒नाना᳚म् करोति निर्जग्मि॒वान् । \newline
12. क॒रो॒ति॒ नि॒र्ज॒ग्मि॒वान् नि॑र्जग्मि॒वान् क॑रोति करोति निर्जग्मि॒वान् तम॑स॒ स्तम॑सो निर्जग्मि॒वान् क॑रोति करोति निर्जग्मि॒वान् तम॑सः । \newline
13. नि॒र्ज॒ग्मि॒वान् तम॑स॒ स्तम॑सो निर्जग्मि॒वान् नि॑र्जग्मि॒वान् तम॑स॒ इतीति॒ तम॑सो निर्जग्मि॒वान् नि॑र्जग्मि॒वान् तम॑स॒ इति॑ । \newline
14. नि॒र्ज॒ग्मि॒वानिति॑ निः - ज॒ग्मि॒वान् । \newline
15. तम॑स॒ इतीति॒ तम॑स॒ स्तम॑स॒ इत्या॑हा॒हेति॒ तम॑स॒ स्तम॑स॒ इत्या॑ह । \newline
16. इत्या॑हा॒हे तीत्या॑ह॒ तम॒ स्तम॑ आ॒हे तीत्या॑ह॒ तमः॑ । \newline
17. आ॒ह॒ तम॒ स्तम॑ आहाह॒ तम॑ ए॒वैव तम॑ आहाह॒ तम॑ ए॒व । \newline
18. तम॑ ए॒वैव तम॒ स्तम॑ ए॒वास्मा॑ दस्मादे॒व तम॒ स्तम॑ ए॒वास्मा᳚त् । \newline
19. ए॒वास्मा॑ दस्मा दे॒वैवास्मा॒ दपापा᳚स्मा दे॒वैवास्मा॒ दप॑ । \newline
20. अ॒स्मा॒ दपापा᳚स्मा दस्मा॒दप॑ हन्ति ह॒न्त्यपा᳚स्मा दस्मा॒दप॑ हन्ति । \newline
21. अप॑ हन्ति ह॒न्त्यपाप॑ हन्ति॒ ज्योति॑षा॒ ज्योति॑षा ह॒न्त्यपाप॑ हन्ति॒ ज्योति॑षा । \newline
22. ह॒न्ति॒ ज्योति॑षा॒ ज्योति॑षा हन्ति हन्ति॒ ज्योति॒षा ऽऽज्योति॑षा हन्ति हन्ति॒ ज्योति॒षा । \newline
23. ज्योति॒षा ऽऽज्योति॑षा॒ ज्योति॒षा ऽगा॑ दगा॒दा ज्योति॑षा॒ ज्योति॒षा ऽगा᳚त् । \newline
24. आ ऽगा॑ दगा॒दा ऽगा॒दिती त्य॑गा॒दा ऽगा॒दिति॑ । \newline
25. अ॒गा॒दिती त्य॑गा दगा॒ दित्या॑हा॒हे त्य॑गा दगा॒ दित्या॑ह । \newline
26. इत्या॑हा॒हे तीत्या॑ह॒ ज्योति॒र् ज्योति॑रा॒हे तीत्या॑ह॒ ज्योतिः॑ । \newline
27. आ॒ह॒ ज्योति॒र् ज्योति॑ राहाह॒ ज्योति॑ रे॒वैव ज्योति॑ राहाह॒ ज्योति॑ रे॒व । \newline
28. ज्योति॑ रे॒वैव ज्योति॒र् ज्योति॑ रे॒वास्मि॑न् नस्मिन् ने॒व ज्योति॒र् ज्योति॑ रे॒वास्मिन्न्॑ । \newline
29. ए॒वास्मि॑न् नस्मिन् ने॒वैवास्मि॑न् दधाति दधा त्यस्मिन् ने॒वैवास्मि॑न् दधाति । \newline
30. अ॒स्मि॒न् द॒धा॒ति॒ द॒धा॒ त्य॒स्मि॒न् न॒स्मि॒न् द॒धा॒ति॒ च॒त॒सृभि॑ श्चत॒सृभि॑र् दधा त्यस्मिन् नस्मिन् दधाति चत॒सृभिः॑ । \newline
31. द॒धा॒ति॒ च॒त॒सृभि॑ श्चत॒सृभि॑र् दधाति दधाति चत॒सृभिः॑ सादयति सादयति चत॒सृभि॑र् दधाति दधाति चत॒सृभिः॑ सादयति । \newline
32. च॒त॒सृभिः॑ सादयति सादयति चत॒सृभि॑ श्चत॒सृभिः॑ सादयति च॒त्वारि॑ च॒त्वारि॑ सादयति चत॒सृभि॑ श्चत॒सृभिः॑ सादयति च॒त्वारि॑ । \newline
33. च॒त॒सृभि॒रिति॑ चत॒सृ - भिः॒ । \newline
34. सा॒द॒य॒ति॒ च॒त्वारि॑ च॒त्वारि॑ सादयति सादयति च॒त्वारि॒ छन्दाꣳ॑सि॒ छन्दाꣳ॑सि च॒त्वारि॑ सादयति सादयति च॒त्वारि॒ छन्दाꣳ॑सि । \newline
35. च॒त्वारि॒ छन्दाꣳ॑सि॒ छन्दाꣳ॑सि च॒त्वारि॑ च॒त्वारि॒ छन्दाꣳ॑सि॒ छन्दो॑भि॒ श्छन्दो॑भि॒ श्छन्दाꣳ॑सि च॒त्वारि॑ च॒त्वारि॒ छन्दाꣳ॑सि॒ छन्दो॑भिः । \newline
36. छन्दाꣳ॑सि॒ छन्दो॑भि॒ श्छन्दो॑भि॒ श्छन्दाꣳ॑सि॒ छन्दाꣳ॑सि॒ छन्दो॑भि रे॒वैव छन्दो॑भि॒ श्छन्दाꣳ॑सि॒ छन्दाꣳ॑सि॒ छन्दो॑भि रे॒व । \newline
37. छन्दो॑भि रे॒वैव छन्दो॑भि॒ श्छन्दो॑भि रे॒वाति॑ च्छन्द॒सा ऽति॑च्छन्दसै॒व छन्दो॑भि॒ श्छन्दो॑भि रे॒वाति॑च्छन्दसा । \newline
38. छन्दो॑भि॒रिति॒ छन्दः॑ - भिः॒ । \newline
39. ए॒वाति॑च्छन्द॒सा ऽति॑च्छन्द सै॒वैवाति॑च्छन्द सोत्त॒म यो᳚त्त॒मया ऽति॑च्छन्द सै॒वैवाति॑च्छन्द सोत्त॒मया᳚ । \newline
40. अति॑च्छन्द सोत्त॒म यो᳚त्त॒मया ऽति॑च्छन्द॒सा ऽति॑च्छन्द सोत्त॒मया॒ वर्ष्म॒ वर्ष्मो᳚त्त॒मया ऽति॑च्छन्द॒सा ऽति॑च्छन्द सोत्त॒मया॒ वर्ष्म॑ । \newline
41. अति॑च्छन्द॒सेत्यति॑ - छ॒न्द॒सा॒ । \newline
42. उ॒त्त॒मया॒ वर्ष्म॒ वर्ष्मो᳚त्त॒म यो᳚त्त॒मया॒ वर्ष्म॒ वै वै वर्ष्मो᳚त्त॒म यो᳚त्त॒मया॒ वर्ष्म॒ वै । \newline
43. उ॒त्त॒मयेत्यु॑त् - त॒मया᳚ । \newline
44. वर्ष्म॒ वै वै वर्ष्म॒ वर्ष्म॒ वा ए॒षैषा वै वर्ष्म॒ वर्ष्म॒ वा ए॒षा । \newline
45. वा ए॒षैषा वै वा ए॒षा छन्द॑सा॒म् छन्द॑सा मे॒षा वै वा ए॒षा छन्द॑साम् । \newline
46. ए॒षा छन्द॑सा॒म् छन्द॑सा मे॒षैषा छन्द॑सां॒ ॅयद् यच् छन्द॑सा मे॒षैषा छन्द॑सां॒ ॅयत् । \newline
47. छन्द॑सां॒ ॅयद् यच् छन्द॑सा॒म् छन्द॑सां॒ ॅयदति॑च्छन्दा॒ अति॑च्छन्दा॒ यच् छन्द॑सा॒म् छन्द॑सां॒ ॅयदति॑च्छन्दाः । \newline
48. यदति॑च्छन्दा॒ अति॑च्छन्दा॒ यद् यदति॑च्छन्दा॒ वर्ष्म॒ वर्ष्माति॑च्छन्दा॒ यद् यदति॑च्छन्दा॒ वर्ष्म॑ । \newline
49. अति॑च्छन्दा॒ वर्ष्म॒ वर्ष्माति॑च्छन्दा॒ अति॑च्छन्दा॒ वर्ष्मै॒वैव वर्ष्माति॑च्छन्दा॒ अति॑च्छन्दा॒ वर्ष्मै॒व । \newline
50. अति॑च्छन्दा॒ इत्यति॑ - छ॒न्दाः॒ । \newline
51. वर्ष्मै॒वैव वर्ष्म॒ वर्ष्मै॒वैन॑ मेन मे॒व वर्ष्म॒ वर्ष्मै॒वैन᳚म् । \newline
52. ए॒वैन॑ मेन मे॒वैवैनꣳ॑ समा॒नानाꣳ॑ समा॒नाना॑ मेन मे॒वैवैनꣳ॑ समा॒नाना᳚म् । \newline
53. ए॒नꣳ॒॒ स॒मा॒नानाꣳ॑ समा॒नाना॑ मेन मेनꣳ समा॒नाना᳚म् करोति करोति समा॒नाना॑ मेन मेनꣳ समा॒नाना᳚म् करोति । \newline
54. स॒मा॒नाना᳚म् करोति करोति समा॒नानाꣳ॑ समा॒नाना᳚म् करोति॒ सद्व॑ती॒ सद्व॑ती करोति समा॒नानाꣳ॑ समा॒नाना᳚म् करोति॒ सद्व॑ती । \newline
55. क॒रो॒ति॒ सद्व॑ती॒ सद्व॑ती करोति करोति॒ सद्व॑ती भवति भवति॒ सद्व॑ती करोति करोति॒ सद्व॑ती भवति । \newline
56. सद्व॑ती भवति भवति॒ सद्व॑ती॒ सद्व॑ती भवति स॒त्त्वꣳ स॒त्त्वम् भ॑वति॒ सद्व॑ती॒ सद्व॑ती भवति स॒त्त्वम् । \newline
57. सद्व॒तीति॒ सत् - व॒ती॒ । \newline
\pagebreak
\markright{ TS 5.2.1.6  \hfill https://www.vedavms.in \hfill}

\section{ TS 5.2.1.6 }

\textbf{TS 5.2.1.6 } \newline
\textbf{Samhita Paata} \newline

भवति स॒त्त्वमे॒वैनं॑ गमयति वाथ्स॒प्रेणोप॑ तिष्ठत ए॒तेन॒ वै व॑थ्स॒प्रीर्भा॑लन्द॒नो᳚ऽग्नेः प्रि॒यं धामाऽवा॑ऽरुन्धा॒ऽग्नेरे॒वैतेन॑ प्रि॒यं धामाऽव॑ रुन्ध एकाद॒शं भ॑वत्येक॒धैव यज॑माने वी॒र्यं॑ दधाति॒ स्तोमे॑न॒ वै दे॒वा अ॒स्मिन् ॅलो॒क आ᳚र्द्ध्नुव॒न् छन्दो॑भिर॒मुष्मि॒न्थ् स्तोम॑स्येव॒ खलु॒ वा ए॒तद्-रू॒पं ॅयद्-वा᳚थ्स॒प्रं ॅयद्-वा᳚थ्स॒प्रेणो॑प॒तिष्ठ॑त - [  ] \newline

\textbf{Pada Paata} \newline

भ॒व॒ति॒ । स॒त्त्वमिति॑ सत् - त्वम् । ए॒व । ए॒न॒म् । ग॒म॒य॒ति॒ । वा॒थ्स॒प्रेणेति॑ वाथ्स - प्रेण॑ । उपेति॑ । ति॒ष्ठ॒ते॒ । ए॒तेन॑ । वै । व॒थ्स॒प्रीरिति॑ वथ्स - प्रीः । भा॒ल॒न्द॒नः । अ॒ग्नेः । प्रि॒यम् । धाम॑ । अवेति॑ । अ॒रु॒न्ध॒ । अ॒ग्नेः । ए॒व । ए॒तेन॑ । प्रि॒यम् । धाम॑ । अवेति॑ । रु॒न्धे॒ । ए॒का॒द॒शम् । भ॒व॒ति॒ । ए॒क॒धेत्ये॑क - धा । ए॒व । यज॑माने । वी॒र्य᳚म् । द॒धा॒ति॒ । स्तोमे॑न । वै । दे॒वाः । अ॒स्मिन्न् । लो॒के । आ॒द्‌र्ध्नु॒व॒न्न् । छन्दो॑भि॒रिति॒ छन्दः॑ - भिः॒ । अ॒मुष्मिन्न्॑ । स्तोम॑स्य । इ॒व॒ । खलु॑ । वै । ए॒तत् । रू॒पम् । यत् । वा॒थ्स॒प्रमिति॑ वाथ्स-प्रम् । यत् । वा॒थ्स॒प्रेणेति॑ वाथ्स-प्रेण॑ । उ॒प॒तिष्ठ॑त॒ इत्यु॑प-तिष्ठ॑ते ।  \newline


\textbf{Krama Paata} \newline

भ॒व॒ति॒ स॒त्वम् । स॒त्वमे॒व । स॒त्वमिति॑ सत् - त्वम् । ए॒वैन᳚म् । ए॒न॒म् ग॒म॒य॒ति॒ । ग॒म॒य॒ति॒ वा॒थ्स॒प्रेण॑ । वा॒थ्स॒प्रेणोऽप॑ । वा॒थ्स॒प्रेणेति॑ वाथ्स - प्रेण॑ । उप॑ तिष्ठते । ति॒ष्ठ॒त॒ ए॒तेन॑ । ए॒तेन॒ वै । वै व॑थ्स॒प्रीः । व॒थ्स॒प्रीर् भा॑लन्द॒नः । व॒थ्स॒प्रीरिति॑ वथ्स - प्रीः । भा॒ल॒न्द॒नो᳚ऽग्नेः । अ॒ग्नेः प्रि॒यम् । प्रि॒यम् धाम॑ । धामाव॑ । अवा॑रुन्ध । अ॒रु॒न्धा॒ग्नेः । अ॒ग्नेरे॒व । ए॒वैतेन॑ । ए॒तेन॑ प्रि॒यम् । प्रि॒यम् धाम॑ । धामाव॑ । अव॑ रुन्धे । रु॒न्ध॒ ए॒का॒द॒शम् । ए॒का॒द॒शम् भ॑वति । भ॒व॒त्ये॒क॒धा । ए॒क॒धैव । ए॒क॒धेत्ये॑क - धा । ए॒व यज॑माने । यज॑माने वी॒र्य᳚म् । वी॒र्य॑म् दधाति । द॒धा॒ति॒ स्तोमे॑न । स्तोमे॑न॒ वै । वै दे॒वाः । दे॒वा अ॒स्मिन्न् । अ॒स्मिन् ॅलो॒के । लो॒क आ᳚र्द्ध्नुवन्न् । आ॒र्द्ध्नु॒वञ्छन्दो॑भिः । छन्दो॑भिर॒मुष्मिन्न्॑ । छन्दो॑भि॒रिति॒ छन्दः॑ - भिः॒ । अ॒मुष्मि॒न्थ् स्तोम॑स्य । स्तोम॑स्ये॒व । इ॒व॒ खलु॑ । खलु॒ वै । वा ए॒तत् । ए॒तद् रू॒पम् । रू॒पम् ॅयत् । यद् वा᳚थ्स॒प्रम् । वा॒थ्स॒प्रम् ॅयत् । वा॒थ्स॒प्रमिति॑ वाथ्स - प्रम् । यद् वा᳚थ्स॒प्रेण॑ । वा॒थ्स॒प्रेणो॑प॒तिष्ठ॑ते ( ) । वा॒थ्स॒प्रेणेति॑ वाथ्स - प्रेण॑ । उ॒प॒तिष्ठ॑त इ॒मम् । उ॒प॒तिष्ठ॑त॒ इत्यु॑प - तिष्ठ॑ते \newline

\textbf{Jatai Paata} \newline

1. भ॒व॒ति॒ स॒त्त्वꣳ स॒त्त्वम् भ॑वति भवति स॒त्त्वम् । \newline
2. स॒त्त्व मे॒वैव स॒त्त्वꣳ स॒त्त्व मे॒व । \newline
3. स॒त्त्वमिति॑ सत् - त्वम् । \newline
4. ए॒वैन॑ मेन मे॒वै वैन᳚म् । \newline
5. ए॒न॒म् ग॒म॒य॒ति॒ ग॒म॒य॒ त्ये॒न॒ मे॒न॒म् ग॒म॒य॒ति॒ । \newline
6. ग॒म॒य॒ति॒ वा॒थ्स॒प्रेण॑ वाथ्स॒प्रेण॑ गमयति गमयति वाथ्स॒प्रेण॑ । \newline
7. वा॒थ्स॒प्रेणो पोप॑ वाथ्स॒प्रेण॑ वाथ्स॒प्रे णोप॑ । \newline
8. वा॒थ्स॒प्रेणेति॑ वाथ्स - प्रेण॑ । \newline
9. उप॑ तिष्ठते तिष्ठत॒ उपोप॑ तिष्ठते । \newline
10. ति॒ष्ठ॒त॒ ए॒ते नै॒तेन॑ तिष्ठते तिष्ठत ए॒तेन॑ । \newline
11. ए॒तेन॒ वै वा ए॒ते नै॒तेन॒ वै । \newline
12. वै व॑थ्स॒प्रीर् व॑थ्स॒प्रीर् वै वै व॑थ्स॒प्रीः । \newline
13. व॒थ्स॒प्रीर् भा॑लन्द॒नो भा॑लन्द॒नो व॑थ्स॒प्रीर् व॑थ्स॒प्रीर् भा॑लन्द॒नः । \newline
14. व॒थ्स॒प्रीरिति॑ वथ्स - प्रीः । \newline
15. भा॒ल॒न्द॒नो᳚ ऽग्ने र॒ग्नेर् भा॑लन्द॒नो भा॑लन्द॒नो᳚ ऽग्नेः । \newline
16. अ॒ग्नेः प्रि॒यम् प्रि॒य म॒ग्ने र॒ग्नेः प्रि॒यम् । \newline
17. प्रि॒यम् धाम॒ धाम॑ प्रि॒यम् प्रि॒यम् धाम॑ । \newline
18. धामा वाव॒ धाम॒ धामाव॑ । \newline
19. अवा॑ रुन्धा रु॒न्धा वावा॑ रुन्ध । \newline
20. अ॒रु॒न् धा॒ग्ने र॒ग्ने र॑रुन्धा रुन्धा॒ग्नेः । \newline
21. अ॒ग्ने रे॒वैवाग्ने र॒ग्ने रे॒व । \newline
22. ए॒वैते नै॒ते नै॒वै वैतेन॑ । \newline
23. ए॒तेन॑ प्रि॒यम् प्रि॒य मे॒ते नै॒तेन॑ प्रि॒यम् । \newline
24. प्रि॒यम् धाम॒ धाम॑ प्रि॒यम् प्रि॒यम् धाम॑ । \newline
25. धामा वाव॒ धाम॒ धामाव॑ । \newline
26. अव॑ रुन्धे रु॒न्धे ऽवाव॑ रुन्धे । \newline
27. रु॒न्ध॒ ए॒का॒द॒श मे॑काद॒शꣳ रु॑न्धे रुन्ध एकाद॒शम् । \newline
28. ए॒का॒द॒शम् भ॑वति भव त्येकाद॒श मे॑काद॒शम् भ॑वति । \newline
29. भ॒व॒ त्ये॒क॒ धैक॒धा भ॑वति भव त्येक॒धा । \newline
30. ए॒क॒ धैवै वैक॒ धैक॒ धैव । \newline
31. ए॒क॒धेत्ये॑क - धा । \newline
32. ए॒व यज॑माने॒ यज॑मान ए॒वैव यज॑माने । \newline
33. यज॑माने वी॒र्यं॑ ॅवी॒र्यं॑ ॅयज॑माने॒ यज॑माने वी॒र्य᳚म् । \newline
34. वी॒र्य॑म् दधाति दधाति वी॒र्यं॑ ॅवी॒र्य॑म् दधाति । \newline
35. द॒धा॒ति॒ स्तोमे॑न॒ स्तोमे॑न दधाति दधाति॒ स्तोमे॑न । \newline
36. स्तोमे॑न॒ वै वै स्तोमे॑न॒ स्तोमे॑न॒ वै । \newline
37. वै दे॒वा दे॒वा वै वै दे॒वाः । \newline
38. दे॒वा अ॒स्मिन् न॒स्मिन् दे॒वा दे॒वा अ॒स्मिन्न् । \newline
39. अ॒स्मिन् ॅलो॒के लो॒के᳚ ऽस्मिन् न॒स्मिन् ॅलो॒के । \newline
40. लो॒क आ᳚र्द्ध्नुवन् नार्द्ध्नुवन् ॅलो॒के लो॒क आ᳚र्द्ध्नुवन्न् । \newline
41. आ॒र्द्ध्नु॒व॒न् छन्दो॑भि॒ श्छन्दो॑भि रार्द्ध्नुवन् नार्द्ध्नुव॒न् छन्दो॑भिः । \newline
42. छन्दो॑भि र॒मुष्मि॑न् न॒मुष्मि॒न् छन्दो॑भि॒ श्छन्दो॑भि र॒मुष्मिन्न्॑ । \newline
43. छन्दो॑भि॒रिति॒ छन्दः॑ - भिः॒ । \newline
44. अ॒मुष्मि॒न् थ्स्तोम॑स्य॒ स्तोम॑स्या॒ मुष्मि॑न् न॒मुष्मि॒न् थ्स्तोम॑स्य । \newline
45. स्तोम॑ स्येवेव॒ स्तोम॑स्य॒ स्तोम॑स्येव । \newline
46. इ॒व॒ खलु॒ खल्वि॑वेव॒ खलु॑ । \newline
47. खलु॒ वै वै खलु॒ खलु॒ वै । \newline
48. वा ए॒त दे॒तद् वै वा ए॒तत् । \newline
49. ए॒तद् रू॒पꣳ रू॒प मे॒त दे॒तद् रू॒पम् । \newline
50. रू॒पं ॅयद् यद् रू॒पꣳ रू॒पं ॅयत् । \newline
51. यद् वा᳚थ्स॒प्रं ॅवा᳚थ्स॒प्रं ॅयद् यद् वा᳚थ्स॒प्रम् । \newline
52. वा॒थ्स॒प्रं ॅयद् यद् वा᳚थ्स॒प्रं ॅवा᳚थ्स॒प्रं ॅयत् । \newline
53. वा॒थ्स॒प्रमिति॑ वाथ्स - प्रम् । \newline
54. यद् वा᳚थ्स॒प्रेण॑ वाथ्स॒प्रेण॒ यद् यद् वा᳚थ्स॒प्रेण॑ । \newline
55. वा॒थ्स॒प्रेणो॑ प॒तिष्ठ॑त उप॒तिष्ठ॑ते वाथ्स॒प्रेण॑ वाथ्स॒प्रेणो॑ प॒तिष्ठ॑ते । \newline
56. वा॒थ्स॒प्रेणेति॑ वाथ्स - प्रेण॑ । \newline
57. उ॒प॒तिष्ठ॑त इ॒म मि॒म मु॑प॒तिष्ठ॑त उप॒तिष्ठ॑त इ॒मम् । \newline
58. उ॒प॒तिष्ठ॑त॒ इत्यु॑प - तिष्ठ॑ते । \newline

\textbf{Ghana Paata } \newline

1. भ॒व॒ति॒ स॒त्त्वꣳ स॒त्त्वम् भ॑वति भवति स॒त्त्व मे॒वैव स॒त्त्वम् भ॑वति भवति स॒त्त्व मे॒व । \newline
2. स॒त्त्व मे॒वैव स॒त्त्वꣳ स॒त्त्व मे॒वैन॑ मेन मे॒व स॒त्त्वꣳ स॒त्त्व मे॒वैन᳚म् । \newline
3. स॒त्त्वमिति॑ सत् - त्वम् । \newline
4. ए॒वैन॑ मेन मे॒वैवैन॑म् गमयति गमय त्येन मे॒वैवैन॑म् गमयति । \newline
5. ए॒न॒म् ग॒म॒य॒ति॒ ग॒म॒य॒ त्ये॒न॒ मे॒न॒म् ग॒म॒य॒ति॒ वा॒थ्स॒प्रेण॑ वाथ्स॒प्रेण॑ गमय त्येन मेनम् गमयति वाथ्स॒प्रेण॑ । \newline
6. ग॒म॒य॒ति॒ वा॒थ्स॒प्रेण॑ वाथ्स॒प्रेण॑ गमयति गमयति वाथ्स॒प्रेणोपोप॑ वाथ्स॒प्रेण॑ गमयति गमयति वाथ्स॒प्रेणोप॑ । \newline
7. वा॒थ्स॒प्रेणोपोप॑ वाथ्स॒प्रेण॑ वाथ्स॒प्रेणोप॑ तिष्ठते तिष्ठत॒ उप॑ वाथ्स॒प्रेण॑ वाथ्स॒प्रेणोप॑ तिष्ठते । \newline
8. वा॒थ्स॒प्रेणेति॑ वाथ्स - प्रेण॑ । \newline
9. उप॑ तिष्ठते तिष्ठत॒ उपोप॑ तिष्ठत ए॒तेनै॒तेन॑ तिष्ठत॒ उपोप॑ तिष्ठत ए॒तेन॑ । \newline
10. ति॒ष्ठ॒त॒ ए॒तेनै॒तेन॑ तिष्ठते तिष्ठत ए॒तेन॒ वै वा ए॒तेन॑ तिष्ठते तिष्ठत ए॒तेन॒ वै । \newline
11. ए॒तेन॒ वै वा ए॒ते नै॒तेन॒ वै व॑थ्स॒प्रीर् व॑थ्स॒प्रीर् वा ए॒ते नै॒तेन॒ वै व॑थ्स॒प्रीः । \newline
12. वै व॑थ्स॒प्रीर् व॑थ्स॒प्रीर् वै वै व॑थ्स॒प्रीर् भा॑लन्द॒नो भा॑लन्द॒नो व॑थ्स॒प्रीर् वै वै व॑थ्स॒प्रीर् भा॑लन्द॒नः । \newline
13. व॒थ्स॒प्रीर् भा॑लन्द॒नो भा॑लन्द॒नो व॑थ्स॒प्रीर् व॑थ्स॒प्रीर् भा॑लन्द॒नो᳚ ऽग्ने र॒ग्नेर् भा॑लन्द॒नो व॑थ्स॒प्रीर् व॑थ्स॒प्रीर् भा॑लन्द॒नो᳚ ऽग्नेः । \newline
14. व॒थ्स॒प्रीरिति॑ वथ्स - प्रीः । \newline
15. भा॒ल॒न्द॒नो᳚ ऽग्ने र॒ग्नेर् भा॑लन्द॒नो भा॑लन्द॒नो᳚ ऽग्नेः प्रि॒यम् प्रि॒य म॒ग्नेर् भा॑लन्द॒नो भा॑लन्द॒नो᳚ ऽग्नेः प्रि॒यम् । \newline
16. अ॒ग्नेः प्रि॒यम् प्रि॒य म॒ग्ने र॒ग्नेः प्रि॒यम् धाम॒ धाम॑ प्रि॒य म॒ग्ने र॒ग्नेः प्रि॒यम् धाम॑ । \newline
17. प्रि॒यम् धाम॒ धाम॑ प्रि॒यम् प्रि॒यम् धामावाव॒ धाम॑ प्रि॒यम् प्रि॒यम् धामाव॑ । \newline
18. धामा वाव॒ धाम॒ धामा वा॑रुन्धा रु॒न्धाव॒ धाम॒ धामा वा॑रुन्ध । \newline
19. अवा॑रुन्धा रु॒न्धा वावा॑ रुन्धा॒ग्ने र॒ग्ने र॑रु॒न्धा वावा॑ रुन्धा॒ग्नेः । \newline
20. अ॒रु॒न्धा॒ग्ने र॒ग्ने र॑रुन्धा रुन्धा॒ग्ने रे॒वैवाग्ने र॑रुन्धा रुन्धा॒ग्ने रे॒व । \newline
21. अ॒ग्ने रे॒वैवाग्ने र॒ग्ने रे॒वैते नै॒ते नै॒वाग्ने र॒ग्ने रे॒वैतेन॑ । \newline
22. ए॒वैते नै॒ते नै॒वैवैतेन॑ प्रि॒यम् प्रि॒य मे॒ते नै॒वैवैतेन॑ प्रि॒यम् । \newline
23. ए॒तेन॑ प्रि॒यम् प्रि॒य मे॒तेनै॒तेन॑ प्रि॒यम् धाम॒ धाम॑ प्रि॒य मे॒तेनै॒तेन॑ प्रि॒यम् धाम॑ । \newline
24. प्रि॒यम् धाम॒ धाम॑ प्रि॒यम् प्रि॒यम् धामा वाव॒ धाम॑ प्रि॒यम् प्रि॒यम् धामाव॑ । \newline
25. धामा वाव॒ धाम॒ धामाव॑ रुन्धे रु॒न्धे ऽव॒ धाम॒ धामाव॑ रुन्धे । \newline
26. अव॑ रुन्धे रु॒न्धे ऽवाव॑ रुन्ध एकाद॒श मे॑काद॒शꣳ रु॒न्धे ऽवाव॑ रुन्ध एकाद॒शम् । \newline
27. रु॒न्ध॒ ए॒का॒द॒श मे॑काद॒शꣳ रु॑न्धे रुन्ध एकाद॒शम् भ॑वति भव त्येकाद॒शꣳ रु॑न्धे रुन्ध एकाद॒शम् भ॑वति । \newline
28. ए॒का॒द॒शम् भ॑वति भव त्येकाद॒श मे॑काद॒शम् भ॑व त्येक॒ धैक॒धा भ॑व त्येकाद॒श मे॑काद॒शम् भ॑व त्येक॒धा । \newline
29. भ॒व॒ त्ये॒क॒ धैक॒धा भ॑वति भव त्येक॒ धैवैवैक॒धा भ॑वति भव त्येक॒धैव । \newline
30. ए॒क॒ धैवैवैक॒ धैक॒ धैव यज॑माने॒ यज॑मान ए॒वैक॒ धैक॒ धैव यज॑माने । \newline
31. ए॒क॒धेत्ये॑क - धा । \newline
32. ए॒व यज॑माने॒ यज॑मान ए॒वैव यज॑माने वी॒र्यं॑ ॅवी॒र्यं॑ ॅयज॑मान ए॒वैव यज॑माने वी॒र्य᳚म् । \newline
33. यज॑माने वी॒र्यं॑ ॅवी॒र्यं॑ ॅयज॑माने॒ यज॑माने वी॒र्य॑म् दधाति दधाति वी॒र्यं॑ ॅयज॑माने॒ यज॑माने वी॒र्य॑म् दधाति । \newline
34. वी॒र्य॑म् दधाति दधाति वी॒र्यं॑ ॅवी॒र्य॑म् दधाति॒ स्तोमे॑न॒ स्तोमे॑न दधाति वी॒र्यं॑ ॅवी॒र्य॑म् दधाति॒ स्तोमे॑न । \newline
35. द॒धा॒ति॒ स्तोमे॑न॒ स्तोमे॑न दधाति दधाति॒ स्तोमे॑न॒ वै वै स्तोमे॑न दधाति दधाति॒ स्तोमे॑न॒ वै । \newline
36. स्तोमे॑न॒ वै वै स्तोमे॑न॒ स्तोमे॑न॒ वै दे॒वा दे॒वा वै स्तोमे॑न॒ स्तोमे॑न॒ वै दे॒वाः । \newline
37. वै दे॒वा दे॒वा वै वै दे॒वा अ॒स्मिन् न॒स्मिन् दे॒वा वै वै दे॒वा अ॒स्मिन्न् । \newline
38. दे॒वा अ॒स्मिन् न॒स्मिन् दे॒वा दे॒वा अ॒स्मिन् ॅलो॒के लो॒के᳚ ऽस्मिन् दे॒वा दे॒वा अ॒स्मिन् ॅलो॒के । \newline
39. अ॒स्मिन् ॅलो॒के लो॒के᳚ ऽस्मिन् न॒स्मिन् ॅलो॒क आ᳚र्द्ध्नुवन् नार्द्ध्नुवन् ॅलो॒के᳚ ऽस्मिन् न॒स्मिन् ॅलो॒क आ᳚र्द्ध्नुवन्न् । \newline
40. लो॒क आ᳚र्द्ध्नुवन् नार्द्ध्नुवन् ॅलो॒के लो॒क आ᳚र्द्ध्नुव॒न् छन्दो॑भि॒ श्छन्दो॑भि रार्द्ध्नुवन् ॅलो॒के लो॒क आ᳚र्द्ध्नुव॒न् छन्दो॑भिः । \newline
41. आ॒र्द्ध्नु॒व॒न् छन्दो॑भि॒ श्छन्दो॑भि रार्द्ध्नुवन् नार्द्ध्नुव॒न् छन्दो॑भि र॒मुष्मि॑न् न॒मुष्मि॒न् छन्दो॑भि रार्द्ध्नुवन् नार्द्ध्नुव॒न् छन्दो॑भि र॒मुष्मिन्न्॑ । \newline
42. छन्दो॑भि र॒मुष्मि॑न् न॒मुष्मि॒न् छन्दो॑भि॒ श्छन्दो॑भि र॒मुष्मि॒न् थ्स्तोम॑स्य॒ स्तोम॑स्या॒ मुष्मि॒न् छन्दो॑भि॒ श्छन्दो॑भि र॒मुष्मि॒न् थ्स्तोम॑स्य । \newline
43. छन्दो॑भि॒रिति॒ छन्दः॑ - भिः॒ । \newline
44. अ॒मुष्मि॒न् थ्स्तोम॑स्य॒ स्तोम॑स्या॒ मुष्मि॑न् न॒मुष्मि॒न् थ्स्तोम॑स्ये वेव॒ स्तोम॑स्या॒ मुष्मि॑न् न॒मुष्मि॒न् थ्स्तोम॑स्येव । \newline
45. स्तोम॑स्ये वे व॒ स्तोम॑स्य॒ स्तोम॑स्येव॒ खलु॒ खल्वि॑व॒ स्तोम॑स्य॒ स्तोम॑स्येव॒ खलु॑ । \newline
46. इ॒व॒ खलु॒ खल्वि॑वेव॒ खलु॒ वै वै खल्वि॑वेव॒ खलु॒ वै । \newline
47. खलु॒ वै वै खलु॒ खलु॒ वा ए॒त दे॒तद् वै खलु॒ खलु॒ वा ए॒तत् । \newline
48. वा ए॒त दे॒तद् वै वा ए॒तद् रू॒पꣳ रू॒प मे॒तद् वै वा ए॒तद् रू॒पम् । \newline
49. ए॒तद् रू॒पꣳ रू॒प मे॒त दे॒तद् रू॒पं ॅयद् यद् रू॒प मे॒त दे॒तद् रू॒पं ॅयत् । \newline
50. रू॒पं ॅयद् यद् रू॒पꣳ रू॒पं ॅयद् वा᳚थ्स॒प्रं ॅवा᳚थ्स॒प्रं ॅयद् रू॒पꣳ रू॒पं ॅयद् वा᳚थ्स॒प्रम् । \newline
51. यद् वा᳚थ्स॒प्रं ॅवा᳚थ्स॒प्रं ॅयद् यद् वा᳚थ्स॒प्रं ॅयद् यद् वा᳚थ्स॒प्रं ॅयद् यद् वा᳚थ्स॒प्रं ॅयत् । \newline
52. वा॒थ्स॒प्रं ॅयद् यद् वा᳚थ्स॒प्रं ॅवा᳚थ्स॒प्रं ॅयद् वा᳚थ्स॒प्रेण॑ वाथ्स॒प्रेण॒ यद् वा᳚थ्स॒प्रं ॅवा᳚थ्स॒प्रं ॅयद् वा᳚थ्स॒प्रेण॑ । \newline
53. वा॒थ्स॒प्रमिति॑ वाथ्स - प्रम् । \newline
54. यद् वा᳚थ्स॒प्रेण॑ वाथ्स॒प्रेण॒ यद् यद् वा᳚थ्स॒प्रे णो॑प॒तिष्ठ॑त उप॒तिष्ठ॑ते वाथ्स॒प्रेण॒ यद् यद् वा᳚थ्स॒प्रे णो॑प॒तिष्ठ॑ते । \newline
55. वा॒थ्स॒प्रे णो॑प॒तिष्ठ॑त उप॒तिष्ठ॑ते वाथ्स॒प्रेण॑ वाथ्स॒प्रे णो॑प॒तिष्ठ॑त इ॒म मि॒म मु॑प॒तिष्ठ॑ते वाथ्स॒प्रेण॑ वाथ्स॒प्रे णो॑प॒तिष्ठ॑त इ॒मम् । \newline
56. वा॒थ्स॒प्रेणेति॑ वाथ्स - प्रेण॑ । \newline
57. उ॒प॒तिष्ठ॑त इ॒म मि॒म मु॑प॒तिष्ठ॑त उप॒तिष्ठ॑त इ॒म मे॒वैवेम मु॑प॒तिष्ठ॑त उप॒तिष्ठ॑त इ॒म मे॒व । \newline
58. उ॒प॒तिष्ठ॑त॒ इत्यु॑प - तिष्ठ॑ते । \newline
\pagebreak
\markright{ TS 5.2.1.7  \hfill https://www.vedavms.in \hfill}

\section{ TS 5.2.1.7 }

\textbf{TS 5.2.1.7 } \newline
\textbf{Samhita Paata} \newline

इ॒ममे॒व तेन॑ लो॒कम॒भि ज॑यति॒ यद्-वि॑ष्णुक्र॒मान् क्रम॑ते॒ऽमुमे॒व तैर्लो॒कम॒भि ज॑यति पूर्वे॒द्युः प्रक्रा॑मत्युत्तरे॒द्युरुप॑ तिष्ठते॒ तस्मा॒द्-योगे॒ऽन्यासां᳚ प्र॒जानां॒ मनः॒ क्षेमे॒ऽन्यासां॒ तस्मा᳚द्-यायाव॒रः क्षे॒म्यस्ये॑शे॒ तस्मा᳚द्-यायाव॒रः क्षे॒म्यम॒द्ध्यव॑स्यति मु॒ष्टी क॑रोति॒ वाचं॑ ॅयच्छति य॒ज्ञ्स्य॒ धृत्यै᳚ ॥ \newline

\textbf{Pada Paata} \newline

इ॒मम् । ए॒व । तेन॑ । लो॒कम् । अ॒भीति॑ । ज॒य॒ति॒ । यत् । वि॒ष्णु॒क्र॒मानिति॑ विष्णु - क्र॒मान् । क्रम॑ते । अ॒मुम् । ए॒व । तैः । लो॒कम् । अ॒भीति॑ । ज॒य॒ति॒ । पू॒र्वे॒द्युः । प्रेति॑ । क्रा॒म॒ति॒ । उ॒त्त॒रे॒द्युः । उपेति॑ । ति॒ष्ठ॒ते॒ । तस्मा᳚त् । योगे᳚ । अ॒न्यासा᳚म् । प्र॒जाना॒मिति॑ प्र - जाना᳚म् । मनः॑ । क्षेमे᳚ । अ॒न्यासा᳚म् । तस्मा᳚त् । या॒या॒व॒रः । क्षे॒म्यस्य॑ । ई॒शे॒ । तस्मा᳚त् । या॒या॒व॒रः । क्षे॒म्यम् । अ॒द्ध्यव॑स्य॒तीत्य॑धि - अव॑स्यति । मु॒ष्टी इति॑ । क॒रो॒ति॒ । वाच᳚म् । य॒च्छ॒ति॒ । य॒ज्ञ्स्य॑ । धृत्यै᳚ ॥  \newline


\textbf{Krama Paata} \newline

इ॒ममे॒व । ए॒व तेन॑ । तेन॑ लो॒कम् । लो॒कम॒भि । अ॒भि ज॑यति । ज॒य॒ति॒ यत् । यद् वि॑ष्णुक्र॒मान् । वि॒ष्णु॒क्र॒मान् क्रम॑ते । वि॒ष्णु॒क्र॒मानिति॑ विष्णु - क्र॒मान् । क्रम॑ते॒ऽमुम् । अ॒मुमे॒व । ए॒व तैः । तैर् लो॒कम् । लो॒कम॒भि । अ॒भि ज॑यति । ज॒य॒ति॒ पू॒र्वे॒द्युः । पू॒र्वे॒द्युः प्र । प्र क्रा॑मति । क्रा॒म॒त्यु॒त्त॒रे॒द्युः । उ॒त्त॒रे॒द्युरुप॑ । उप॑ तिष्ठते । ति॒ष्ठ॒ते॒ तस्मा᳚त् । तस्मा॒द् योगे᳚ । योगे॒ऽन्यासा᳚म् । अ॒न्यासा᳚म् प्र॒जाना᳚म् । प्र॒जाना॒म् मनः॑ । प्र॒जाना॒मिति॑ प्र - जाना᳚म् । मनः॒ क्षेमे᳚ । क्षेमे॒ऽन्यासा᳚म् । अ॒न्यासा॒म् तस्मा᳚त् । तस्मा᳚द् यायाव॒रः । या॒या॒व॒रः क्षे॒म्यस्य॑ । क्षे॒म्यस्ये॑शे । ई॒शे॒ तस्मा᳚त् । तस्मा᳚द् यायाव॒रः । या॒या॒व॒रः क्षे॒म्यम् । क्षे॒म्यम॒द्ध्यव॑स्यति । अ॒द्ध्यव॑स्यति मु॒ष्टी । अ॒द्ध्यव॑स्य॒तीत्य॑धि - अव॑स्यति । मु॒ष्टी क॑रोति । मु॒ष्टी इति॑ मु॒ष्टी । क॒रो॒ति॒ वाच᳚म् । वाच॑म् ॅयच्छति । य॒च्छ॒ति॒ य॒ज्ञ्स्य॑ । य॒ज्ञ्स्य॒ धृत्यै᳚ । धृत्या॒ इति॒ धृत्यै᳚ । \newline

\textbf{Jatai Paata} \newline

1. इ॒म मे॒वैवेम मि॒म मे॒व । \newline
2. ए॒व तेन॒ तेनै॒ वैव तेन॑ । \newline
3. तेन॑ लो॒कम् ॅलो॒कम् तेन॒ तेन॑ लो॒कम् । \newline
4. लो॒क म॒भ्य॑भि लो॒कम् ॅलो॒क म॒भि । \newline
5. अ॒भि ज॑यति जय त्य॒भ्य॑भि ज॑यति । \newline
6. ज॒य॒ति॒ यद् यज् ज॑यति जयति॒ यत् । \newline
7. यद् वि॑ष्णुक्र॒मान्. वि॑ष्णुक्र॒मान्. यद् यद् वि॑ष्णुक्र॒मान् । \newline
8. वि॒ष्णु॒क्र॒मान् क्रम॑ते॒ क्रम॑ते विष्णुक्र॒मान्. वि॑ष्णुक्र॒मान् क्रम॑ते । \newline
9. वि॒ष्णु॒क्र॒मानिति॑ विष्णु - क्र॒मान् । \newline
10. क्रम॑ते॒ ऽमु म॒मुम् क्रम॑ते॒ क्रम॑ते॒ ऽमुम् । \newline
11. अ॒मु मे॒वैवामु म॒मु मे॒व । \newline
12. ए॒व तै स्तै रे॒वैव तैः । \newline
13. तैर् लो॒कम् ॅलो॒कम् तै स्तैर् लो॒कम् । \newline
14. लो॒क म॒भ्य॑भि लो॒कम् ॅलो॒क म॒भि । \newline
15. अ॒भि ज॑यति जय त्य॒भ्य॑भि ज॑यति । \newline
16. ज॒य॒ति॒ पू॒र्वे॒द्युः पू᳚र्वे॒द्युर् ज॑यति जयति पूर्वे॒द्युः । \newline
17. पू॒र्वे॒द्युः प्र प्र पू᳚र्वे॒द्युः पू᳚र्वे॒द्युः प्र । \newline
18. प्र क्रा॑मति क्रामति॒ प्र प्र क्रा॑मति । \newline
19. क्रा॒म॒ त्यु॒त्त॒रे॒द्यु रु॑त्तरे॒द्युः क्रा॑मति क्राम त्युत्तरे॒द्युः । \newline
20. उ॒त्त॒रे॒द्यु रुपोपो᳚ त्तरे॒द्यु रु॑त्तरे॒द्यु रुप॑ । \newline
21. उप॑ तिष्ठते तिष्ठत॒ उपोप॑ तिष्ठते । \newline
22. ति॒ष्ठ॒ते॒ तस्मा॒त् तस्मा᳚त् तिष्ठते तिष्ठते॒ तस्मा᳚त् । \newline
23. तस्मा॒द् योगे॒ योगे॒ तस्मा॒त् तस्मा॒द् योगे᳚ । \newline
24. योगे॒ ऽन्यासा॑ म॒न्यासां॒ ॅयोगे॒ योगे॒ ऽन्यासा᳚म् । \newline
25. अ॒न्यासा᳚म् प्र॒जाना᳚म् प्र॒जाना॑ म॒न्यासा॑ म॒न्यासा᳚म् प्र॒जाना᳚म् । \newline
26. प्र॒जाना॒म् मनो॒ मनः॑ प्र॒जाना᳚म् प्र॒जाना॒म् मनः॑ । \newline
27. प्र॒जाना॒मिति॑ प्र - जाना᳚म् । \newline
28. मनः॒ क्षेमे॒ क्षेमे॒ मनो॒ मनः॒ क्षेमे᳚ । \newline
29. क्षेमे॒ ऽन्यासा॑ म॒न्यासा॒म् क्षेमे॒ क्षेमे॒ ऽन्यासा᳚म् । \newline
30. अ॒न्यासा॒म् तस्मा॒त् तस्मा॑ द॒न्यासा॑ म॒न्यासा॒म् तस्मा᳚त् । \newline
31. तस्मा᳚द् यायाव॒रो या॑याव॒र स्तस्मा॒त् तस्मा᳚द् यायाव॒रः । \newline
32. या॒या॒व॒रः क्षे॒म्यस्य॑ क्षे॒म्यस्य॑ यायाव॒रो या॑याव॒रः क्षे॒म्यस्य॑ । \newline
33. क्षे॒म्य स्ये॑श ईशे क्षे॒म्यस्य॑ क्षे॒म्य स्ये॑शे । \newline
34. ई॒शे॒ तस्मा॒त् तस्मा॑ दीश ईशे॒ तस्मा᳚त् । \newline
35. तस्मा᳚द् यायाव॒रो या॑याव॒र स्तस्मा॒त् तस्मा᳚द् यायाव॒रः । \newline
36. या॒या॒व॒रः क्षे॒म्यम् क्षे॒म्यं ॅया॑याव॒रो या॑याव॒रः क्षे॒म्यम् । \newline
37. क्षे॒म्य म॒द्ध्यव॑स्य त्य॒द्ध्यव॑स्यति क्षे॒म्यम् क्षे॒म्य म॒द्ध्यव॑स्यति । \newline
38. अ॒द्ध्यव॑स्यति मु॒ष्टी मु॒ष्टी अ॒द्ध्यव॑स्य त्य॒द्ध्यव॑स्यति मु॒ष्टी । \newline
39. अ॒द्ध्यव॑स्य॒तीत्य॑धि - अव॑स्यति । \newline
40. मु॒ष्टी क॑रोति करोति मु॒ष्टी मु॒ष्टी क॑रोति । \newline
41. मु॒ष्टी इति॑ मु॒ष्टी । \newline
42. क॒रो॒ति॒ वाचं॒ ॅवाच॑म् करोति करोति॒ वाच᳚म् । \newline
43. वाचं॑ ॅयच्छति यच्छति॒ वाचं॒ ॅवाचं॑ ॅयच्छति । \newline
44. य॒च्छ॒ति॒ य॒ज्ञ्स्य॑ य॒ज्ञ्स्य॑ यच्छति यच्छति य॒ज्ञ्स्य॑ । \newline
45. य॒ज्ञ्स्य॒ धृत्यै॒ धृत्यै॑ य॒ज्ञ्स्य॑ य॒ज्ञ्स्य॒ धृत्यै᳚ । \newline
46. धृत्या॒ इति॒ धृत्यै᳚ । \newline

\textbf{Ghana Paata } \newline

1. इ॒म मे॒वैवेम मि॒म मे॒व तेन॒ तेनै॒वेम मि॒म मे॒व तेन॑ । \newline
2. ए॒व तेन॒ तेनै॒वैव तेन॑ लो॒कम् ॅलो॒कम् तेनै॒वैव तेन॑ लो॒कम् । \newline
3. तेन॑ लो॒कम् ॅलो॒कम् तेन॒ तेन॑ लो॒क म॒भ्य॑भि लो॒कम् तेन॒ तेन॑ लो॒क म॒भि । \newline
4. लो॒क म॒भ्य॑भि लो॒कम् ॅलो॒क म॒भि ज॑यति जयत्य॒भि लो॒कम् ॅलो॒क म॒भि ज॑यति । \newline
5. अ॒भि ज॑यति जय त्य॒भ्य॑भि ज॑यति॒ यद् यज् ज॑य त्य॒भ्य॑भि ज॑यति॒ यत् । \newline
6. ज॒य॒ति॒ यद् यज् ज॑यति जयति॒ यद् वि॑ष्णुक्र॒मान्. वि॑ष्णुक्र॒मान्. यज् ज॑यति जयति॒ यद् वि॑ष्णुक्र॒मान् । \newline
7. यद् वि॑ष्णुक्र॒मान्. वि॑ष्णुक्र॒मान्. यद् यद् वि॑ष्णुक्र॒मान् क्रम॑ते॒ क्रम॑ते विष्णुक्र॒मान्. यद् यद् वि॑ष्णुक्र॒मान् क्रम॑ते । \newline
8. वि॒ष्णु॒क्र॒मान् क्रम॑ते॒ क्रम॑ते विष्णुक्र॒मान्. वि॑ष्णुक्र॒मान् क्रम॑ते॒ ऽमु म॒मुम् क्रम॑ते विष्णुक्र॒मान्. वि॑ष्णुक्र॒मान् क्रम॑ते॒ ऽमुम् । \newline
9. वि॒ष्णु॒क्र॒मानिति॑ विष्णु - क्र॒मान् । \newline
10. क्रम॑ते॒ ऽमु म॒मुम् क्रम॑ते॒ क्रम॑ते॒ ऽमु मे॒वैवामुम् क्रम॑ते॒ क्रम॑ते॒ ऽमु मे॒व । \newline
11. अ॒मु मे॒वैवामु म॒मु मे॒व तै स्तै रे॒वामु म॒मु मे॒व तैः । \newline
12. ए॒व तै स्तै रे॒वैव तैर् लो॒कम् ॅलो॒कम् तै रे॒वैव तैर् लो॒कम् । \newline
13. तैर् लो॒कम् ॅलो॒कम् तै स्तैर् लो॒क म॒भ्य॑भि लो॒कम् तै स्तैर् लो॒क म॒भि । \newline
14. लो॒क म॒भ्य॑भि लो॒कम् ॅलो॒क म॒भि ज॑यति जयत्य॒भि लो॒कम् ॅलो॒क म॒भि ज॑यति । \newline
15. अ॒भि ज॑यति जय त्य॒भ्य॑भि ज॑यति पूर्वे॒द्युः पू᳚र्वे॒द्युर् ज॑य त्य॒भ्य॑भि ज॑यति पूर्वे॒द्युः । \newline
16. ज॒य॒ति॒ पू॒र्वे॒द्युः पू᳚र्वे॒द्युर् ज॑यति जयति पूर्वे॒द्युः प्र प्र पू᳚र्वे॒द्युर् ज॑यति जयति पूर्वे॒द्युः प्र । \newline
17. पू॒र्वे॒द्युः प्र प्र पू᳚र्वे॒द्युः पू᳚र्वे॒द्युः प्र क्रा॑मति क्रामति॒ प्र पू᳚र्वे॒द्युः पू᳚र्वे॒द्युः प्र क्रा॑मति । \newline
18. प्र क्रा॑मति क्रामति॒ प्र प्र क्रा॑म त्युत्तरे॒द्यु रु॑त्तरे॒द्युः क्रा॑मति॒ प्र प्र क्रा॑म त्युत्तरे॒द्युः । \newline
19. क्रा॒म॒ त्यु॒त्त॒रे॒द्यु रु॑त्तरे॒द्युः क्रा॑मति क्राम त्युत्तरे॒द्यु रुपोपो᳚त्तरे॒द्युः क्रा॑मति क्राम त्युत्त रे॒द्युरुप॑ । \newline
20. उ॒त्त॒रे॒द्यु रुपोपो᳚त्तरे॒द्यु रु॑त्तरे॒द्युरुप॑ तिष्ठते तिष्ठत॒ उपो᳚त्तरे॒द्यु रु॑त्तरे॒द्युरुप॑ तिष्ठते । \newline
21. उप॑ तिष्ठते तिष्ठत॒ उपोप॑ तिष्ठते॒ तस्मा॒त् तस्मा᳚त् तिष्ठत॒ उपोप॑ तिष्ठते॒ तस्मा᳚त् । \newline
22. ति॒ष्ठ॒ते॒ तस्मा॒त् तस्मा᳚त् तिष्ठते तिष्ठते॒ तस्मा॒द् योगे॒ योगे॒ तस्मा᳚त् तिष्ठते तिष्ठते॒ तस्मा॒द् योगे᳚ । \newline
23. तस्मा॒द् योगे॒ योगे॒ तस्मा॒त् तस्मा॒द् योगे॒ ऽन्यासा॑ म॒न्यासां॒ ॅयोगे॒ तस्मा॒त् तस्मा॒द् योगे॒ ऽन्यासा᳚म् । \newline
24. योगे॒ ऽन्यासा॑ म॒न्यासां॒ ॅयोगे॒ योगे॒ ऽन्यासा᳚म् प्र॒जाना᳚म् प्र॒जाना॑ म॒न्यासां॒ ॅयोगे॒ योगे॒ ऽन्यासा᳚म् प्र॒जाना᳚म् । \newline
25. अ॒न्यासा᳚म् प्र॒जाना᳚म् प्र॒जाना॑ म॒न्यासा॑ म॒न्यासा᳚म् प्र॒जाना॒म् मनो॒ मनः॑ प्र॒जाना॑ म॒न्यासा॑ म॒न्यासा᳚म् प्र॒जाना॒म् मनः॑ । \newline
26. प्र॒जाना॒म् मनो॒ मनः॑ प्र॒जाना᳚म् प्र॒जाना॒म् मनः॒ क्षेमे॒ क्षेमे॒ मनः॑ प्र॒जाना᳚म् प्र॒जाना॒म् मनः॒ क्षेमे᳚ । \newline
27. प्र॒जाना॒मिति॑ प्र - जाना᳚म् । \newline
28. मनः॒ क्षेमे॒ क्षेमे॒ मनो॒ मनः॒ क्षेमे॒ ऽन्यासा॑ म॒न्यासा॒म् क्षेमे॒ मनो॒ मनः॒ क्षेमे॒ ऽन्यासा᳚म् । \newline
29. क्षेमे॒ ऽन्यासा॑ म॒न्यासा॒म् क्षेमे॒ क्षेमे॒ ऽन्यासा॒म् तस्मा॒त् तस्मा॑ द॒न्यासा॒म् क्षेमे॒ क्षेमे॒ ऽन्यासा॒म् तस्मा᳚त् । \newline
30. अ॒न्यासा॒म् तस्मा॒त् तस्मा॑ द॒न्यासा॑ म॒न्यासा॒म् तस्मा᳚द् यायाव॒रो या॑याव॒र स्तस्मा॑ द॒न्यासा॑ म॒न्यासा॒म् तस्मा᳚द् यायाव॒रः । \newline
31. तस्मा᳚द् यायाव॒रो या॑याव॒र स्तस्मा॒त् तस्मा᳚द् यायाव॒रः क्षे॒म्यस्य॑ क्षे॒म्यस्य॑ यायाव॒र स्तस्मा॒त् तस्मा᳚द् यायाव॒रः क्षे॒म्यस्य॑ । \newline
32. या॒या॒व॒रः क्षे॒म्यस्य॑ क्षे॒म्यस्य॑ यायाव॒रो या॑याव॒रः क्षे॒म्यस्ये॑श ईशे क्षे॒म्यस्य॑ यायाव॒रो या॑याव॒रः क्षे॒म्यस्ये॑शे । \newline
33. क्षे॒म्यस्ये॑श ईशे क्षे॒म्यस्य॑ क्षे॒म्यस्ये॑शे॒ तस्मा॒त् तस्मा॑दीशे क्षे॒म्यस्य॑ क्षे॒म्यस्ये॑शे॒ तस्मा᳚त् । \newline
34. ई॒शे॒ तस्मा॒त् तस्मा॑दीश ईशे॒ तस्मा᳚द् यायाव॒रो या॑याव॒र स्तस्मा॑दीश ईशे॒ तस्मा᳚द् यायाव॒रः । \newline
35. तस्मा᳚द् यायाव॒रो या॑याव॒र स्तस्मा॒त् तस्मा᳚द् यायाव॒रः क्षे॒म्यम् क्षे॒म्यं ॅया॑याव॒र स्तस्मा॒त् तस्मा᳚द् यायाव॒रः क्षे॒म्यम् । \newline
36. या॒या॒व॒रः क्षे॒म्यम् क्षे॒म्यं ॅया॑याव॒रो या॑याव॒रः क्षे॒म्य म॒द्ध्यव॑स्य त्य॒द्ध्यव॑स्यति क्षे॒म्यं ॅया॑याव॒रो या॑याव॒रः क्षे॒म्य म॒द्ध्यव॑स्यति । \newline
37. क्षे॒म्य म॒द्ध्यव॑स्य त्य॒द्ध्यव॑स्यति क्षे॒म्यम् क्षे॒म्य म॒द्ध्यव॑स्यति मु॒ष्टी मु॒ष्टी अ॒द्ध्यव॑स्यति क्षे॒म्यम् क्षे॒म्य म॒द्ध्यव॑स्यति मु॒ष्टी । \newline
38. अ॒द्ध्यव॑स्यति मु॒ष्टी मु॒ष्टी अ॒द्ध्यव॑स्य त्य॒द्ध्यव॑स्यति मु॒ष्टी क॑रोति करोति मु॒ष्टी अ॒द्ध्यव॑स्य त्य॒द्ध्यव॑स्यति मु॒ष्टी क॑रोति । \newline
39. अ॒द्ध्यव॑स्य॒तीत्य॑धि - अव॑स्यति । \newline
40. मु॒ष्टी क॑रोति करोति मु॒ष्टी मु॒ष्टी क॑रोति॒ वाचं॒ ॅवाच॑म् करोति मु॒ष्टी मु॒ष्टी क॑रोति॒ वाच᳚म् । \newline
41. मु॒ष्टी इति॑ मु॒ष्टी । \newline
42. क॒रो॒ति॒ वाचं॒ ॅवाच॑म् करोति करोति॒ वाचं॑ ॅयच्छति यच्छति॒ वाच॑म् करोति करोति॒ वाचं॑ ॅयच्छति । \newline
43. वाचं॑ ॅयच्छति यच्छति॒ वाचं॒ ॅवाचं॑ ॅयच्छति य॒ज्ञ्स्य॑ य॒ज्ञ्स्य॑ यच्छति॒ वाचं॒ ॅवाचं॑ ॅयच्छति य॒ज्ञ्स्य॑ । \newline
44. य॒च्छ॒ति॒ य॒ज्ञ्स्य॑ य॒ज्ञ्स्य॑ यच्छति यच्छति य॒ज्ञ्स्य॒ धृत्यै॒ धृत्यै॑ य॒ज्ञ्स्य॑ यच्छति यच्छति य॒ज्ञ्स्य॒ धृत्यै᳚ । \newline
45. य॒ज्ञ्स्य॒ धृत्यै॒ धृत्यै॑ य॒ज्ञ्स्य॑ य॒ज्ञ्स्य॒ धृत्यै᳚ । \newline
46. धृत्या॒ इति॒ धृत्यै᳚ । \newline
\pagebreak
\markright{ TS 5.2.2.1  \hfill https://www.vedavms.in \hfill}

\section{ TS 5.2.2.1 }

\textbf{TS 5.2.2.1 } \newline
\textbf{Samhita Paata} \newline

अन्न॑प॒तेऽन्न॑स्य नो दे॒हीत्या॑हा॒-ग्निर्वा अन्न॑पतिः॒ स ए॒वास्मा॒ अन्नं॒ प्रय॑च्छत्यनमी॒वस्य॑ शु॒ष्मिण॒ इत्या॑हा-य॒क्ष्मस्येति॒ वावैतदा॑ह॒ प्र प्र॑दा॒तारं॑ तारिष॒ ऊर्जं॑ नो धेहि द्वि॒पदे॒ चतु॑ष्पद॒ इत्या॑हा॒ऽऽ*शिष॑मे॒वैतामा शा᳚स्त॒ उदु॑ त्वा॒ विश्वे॑ दे॒वा इत्या॑ह प्रा॒णा वै विश्वे॑ दे॒वाः - [  ] \newline

\textbf{Pada Paata} \newline

अन्न॑पत॒ इत्यन्न॑ - प॒ते॒ । अन्न॑स्य । नः॒ । दे॒हि॒ । इति॑ । आ॒ह॒ । अ॒ग्निः । वै । अन्न॑पति॒रित्यन्न॑ - प॒तिः॒ । सः । ए॒व । अ॒स्मै॒ । अन्न᳚म् । प्रेति॑ । य॒च्छ॒ति॒ । अ॒न॒मी॒वस्य॑ । शु॒ष्मिणः॑ । इति॑ । आ॒ह॒ । अ॒य॒क्ष्मस्य॑ । इति॑ । वाव । ए॒तत् । आ॒ह॒ । प्रेति॑ । प्र॒दा॒तार॒मिति॑ प्र - दा॒तार᳚म् । ता॒रि॒षः॒ । ऊर्ज᳚म् । नः॒ । धे॒हि॒ । द्वि॒पद॒ इति॑ द्वि - पदे᳚ । चतु॑ष्पद॒ इति॒ चतुः॑ - प॒दे॒ । इति॑ । आ॒ह॒ । आ॒शिष॒मित्या᳚ - शिष᳚म् । ए॒व । ए॒ताम् । एति॑ । शा॒स्ते॒ । उदिति॑ । उ॒ । त्वा॒ । विश्वे᳚ । दे॒वाः । इति॑ । आ॒ह॒ । प्रा॒णा इति॑ प्र-अ॒नाः । वै । विश्वे᳚ । दे॒वाः ।  \newline


\textbf{Krama Paata} \newline

अन्न॑प॒तेऽन्न॑स्य । अन्न॑पत॒ इत्यन्न॑ - प॒ते॒ । अन्न॑स्य नः । नो॒ दे॒हि॒ । दे॒हीति॑ । इत्या॑ह । आ॒हा॒ग्निः । अ॒ग्निर् वै । वा अन्न॑पतिः । अन्न॑पतिः॒ सः । अन्न॑पति॒रित्यन्न॑ - प॒तिः॒ । स ए॒व । ए॒वास्मै᳚ । अ॒स्मा॒ अन्न᳚म् । अन्न॒म् प्र । 
प्र य॑च्छति । य॒च्छ॒त्य॒न॒मी॒वस्य॑ । अ॒न॒मी॒वस्य॑ शु॒ष्मिणः॑ । शु॒ष्मिण॒ इति॑ । इत्या॑ह । आ॒हा॒य॒क्ष्मस्य॑ । अ॒य॒क्ष्मस्येति॑ । इति॒ वाव । वावैतत् । ए॒तदा॑ह । आ॒ह॒ प्र । प्र प्र॑दा॒तार᳚म् । प्र॒दा॒तार॑म् तारिषः । प्र॒दा॒तार॒मिति॑ प्र - दा॒तार᳚म् । ता॒रि॒ष॒ ऊर्ज᳚म् । ऊर्ज॑म् नः । नो॒ धे॒हि॒ । धे॒हि॒ द्वि॒पदे᳚ । द्वि॒पदे॒ चतु॑ष्पदे । द्वि॒पद॒ इति॑ द्वि - पदे᳚ । चतु॑ष्पद॒ इति॑ । चतु॑ष्पद॒ इति॒ चतुः॑ - प॒दे॒ । इत्या॑ह । आ॒हा॒शिष᳚म् । आ॒शिष॑मे॒व । आ॒शिष॒मित्या᳚ - शिष᳚म् । ए॒वैताम् । ए॒तामा । आ शा᳚स्ते । शा॒स्त॒ उत् । उदु॑ । उ॒ त्वा॒ । त्वा॒ विश्वे᳚ । विश्वे॑ दे॒वाः । दे॒वा इति॑ । इत्या॑ह । आ॒ह॒ प्रा॒णाः । प्रा॒णा वै । प्रा॒णा इति॑ प्र - अ॒नाः । वै विश्वे᳚ । विश्वे॑ दे॒वाः । दे॒वाः प्रा॒णैः \newline

\textbf{Jatai Paata} \newline

1. अन्न॑प॒ते ऽन्न॒स्या न्न॒स्या न्न॑प॒ते ऽन्न॑प॒ते ऽन्न॑स्य । \newline
2. अन्न॑पत॒ इत्यन्न॑ - प॒ते॒ । \newline
3. अन्न॑स्य नो नो॒ अन्न॒स्या न्न॑स्य नः । \newline
4. नो॒ दे॒हि॒ दे॒हि॒ नो॒ नो॒ दे॒हि॒ । \newline
5. दे॒हीतीति॑ देहि दे॒हीति॑ । \newline
6. इत्या॑हा॒हे तीत्या॑ह । \newline
7. आ॒हा॒ग्नि र॒ग्नि रा॑हा हा॒ग्निः । \newline
8. अ॒ग्निर् वै वा अ॒ग्नि र॒ग्निर् वै । \newline
9. वा अन्न॑पति॒ रन्न॑पति॒र् वै वा अन्न॑पतिः । \newline
10. अन्न॑पतिः॒ स सो ऽन्न॑पति॒ रन्न॑पतिः॒ सः । \newline
11. अन्न॑पति॒रित्यन्न॑ - प॒तिः॒ । \newline
12. स ए॒वैव स स ए॒व । \newline
13. ए॒वास्मा॑ अस्मा ए॒वै वास्मै᳚ । \newline
14. अ॒स्मा॒ अन्न॒ मन्न॑ मस्मा अस्मा॒ अन्न᳚म् । \newline
15. अन्न॒म् प्र प्रान्न॒ मन्न॒म् प्र । \newline
16. प्र य॑च्छति यच्छति॒ प्र प्र य॑च्छति । \newline
17. य॒च्छ॒ त्य॒न॒मी॒वस्या॑ नमी॒वस्य॑ यच्छति यच्छ त्यनमी॒वस्य॑ । \newline
18. अ॒न॒मी॒वस्य॑ शु॒ष्मिणः॑ शु॒ष्मिणो॑ ऽनमी॒व स्या॑नमी॒वस्य॑ शु॒ष्मिणः॑ । \newline
19. शु॒ष्मिण॒ इतीति॑ शु॒ष्मिणः॑ शु॒ष्मिण॒ इति॑ । \newline
20. इत्या॑हा॒हे तीत्या॑ह । \newline
21. आ॒हा॒ य॒क्ष्मस्या॑ य॒क्ष्मस्या॑ हाहा य॒क्ष्मस्य॑ । \newline
22. अ॒य॒क्ष्म स्येती त्य॑य॒क्ष्मस्या॑ य॒क्ष्मस्येति॑ । \newline
23. इति॒ वाव वावे तीति॒ वाव । \newline
24. वावैत दे॒तद् वाव वावैतत् । \newline
25. ए॒त दा॑हा है॒त दे॒त दा॑ह । \newline
26. आ॒ह॒ प्र प्राहा॑ह॒ प्र । \newline
27. प्र प्र॑दा॒तार॑म् प्रदा॒तार॒म् प्र प्र प्र॑दा॒तार᳚म् । \newline
28. प्र॒दा॒तार॑म् तारिष स्तारिषः प्रदा॒तार॑म् प्रदा॒तार॑म् तारिषः । \newline
29. प्र॒दा॒तार॒मिति॑ प्र - दा॒तार᳚म् । \newline
30. ता॒रि॒ष॒ ऊर्ज॒ मूर्ज॑म् तारिष स्तारिष॒ ऊर्ज᳚म् । \newline
31. ऊर्ज॑न् नो न॒ ऊर्ज॒ मूर्ज॑न्नः । \newline
32. नो॒ धे॒हि॒ धे॒हि॒ नो॒ नो॒ धे॒हि॒ । \newline
33. धे॒हि॒ द्वि॒पदे᳚ द्वि॒पदे॑ धेहि धेहि द्वि॒पदे᳚ । \newline
34. द्वि॒पदे॒ चतु॑ष्पदे॒ चतु॑ष्पदे द्वि॒पदे᳚ द्वि॒पदे॒ चतु॑ष्पदे । \newline
35. द्वि॒पद॒ इति॑ द्वि - पदे᳚ । \newline
36. चतु॑ष्पद॒ इतीति॒ चतु॑ष्पदे॒ चतु॑ष्पद॒ इति॑ । \newline
37. चतु॑ष्पद॒ इति॒ चतुः॑ - प॒दे॒ । \newline
38. इत्या॑हा॒हे तीत्या॑ह । \newline
39. आ॒हा॒शिष॑ मा॒शिष॑ माहा हा॒शिष᳚म् । \newline
40. आ॒शिष॑ मे॒वै वाशिष॑ मा॒शिष॑ मे॒व । \newline
41. आ॒शिष॒मित्या᳚ - शिष᳚म् । \newline
42. ए॒वैता मे॒ता मे॒वै वैताम् । \newline
43. ए॒ता मैता मे॒ता मा । \newline
44. आ शा᳚स्ते शास्त॒ आ शा᳚स्ते । \newline
45. शा॒स्त॒ उदु च्छा᳚स्ते शास्त॒ उत् । \newline
46. उदु॑ वु॒ वु दुदु॑ । \newline
47. उ॒ त्वा॒ त्व॒ वु॒ त्वा॒ । \newline
48. त्वा॒ विश्वे॒ विश्वे᳚ त्वा त्वा॒ विश्वे᳚ । \newline
49. विश्वे॑ दे॒वा दे॒वा विश्वे॒ विश्वे॑ दे॒वाः । \newline
50. दे॒वा इतीति॑ दे॒वा दे॒वा इति॑ । \newline
51. इत्या॑हा॒हे तीत्या॑ह । \newline
52. आ॒ह॒ प्रा॒णाः प्रा॒णा आ॑हाह प्रा॒णाः । \newline
53. प्रा॒णा वै वै प्रा॒णाः प्रा॒णा वै । \newline
54. प्रा॒णा इति॑ प्र - अ॒नाः । \newline
55. वै विश्वे॒ विश्वे॒ वै वै विश्वे᳚ । \newline
56. विश्वे॑ दे॒वा दे॒वा विश्वे॒ विश्वे॑ दे॒वाः । \newline
57. दे॒वाः प्रा॒णैः प्रा॒णैर् दे॒वा दे॒वाः प्रा॒णैः । \newline

\textbf{Ghana Paata } \newline

1. अन्न॑प॒ते ऽन्न॒स्या न्न॒स्या न्न॑प॒ते ऽन्न॑प॒ते ऽन्न॑स्य नो नो॒ अन्न॒स्या न्न॑प॒ते ऽन्न॑प॒ते ऽन्न॑स्य नः । \newline
2. अन्न॑पत॒ इत्यन्न॑ - प॒ते॒ । \newline
3. अन्न॑स्य नो नो॒ अन्न॒स्या न्न॑स्य नो देहि देहि नो॒ अन्न॒स्या न्न॑स्य नो देहि । \newline
4. नो॒ दे॒हि॒ दे॒हि॒ नो॒ नो॒ दे॒हीतीति॑ देहि नो नो दे॒हीति॑ । \newline
5. दे॒हीतीति॑ देहि दे॒ही त्या॑हा॒हेति॑ देहि दे॒हीत्या॑ह । \newline
6. इत्या॑हा॒हेती त्या॑हा॒ग्नि र॒ग्नि रा॒हेती त्या॑हा॒ग्निः । \newline
7. आ॒हा॒ग्नि र॒ग्नि रा॑हाहा॒ग्निर् वै वा अ॒ग्नि रा॑हाहा॒ग्निर् वै । \newline
8. अ॒ग्निर् वै वा अ॒ग्नि र॒ग्निर् वा अन्न॑पति॒ रन्न॑पति॒र् वा अ॒ग्नि र॒ग्निर् वा अन्न॑पतिः । \newline
9. वा अन्न॑पति॒ रन्न॑पति॒र् वै वा अन्न॑पतिः॒ स सो ऽन्न॑पति॒र् वै वा अन्न॑पतिः॒ सः । \newline
10. अन्न॑पतिः॒ स सो ऽन्न॑पति॒ रन्न॑पतिः॒ स ए॒वैव सो ऽन्न॑पति॒ रन्न॑पतिः॒ स ए॒व । \newline
11. अन्न॑पति॒रित्यन्न॑ - प॒तिः॒ । \newline
12. स ए॒वैव स स ए॒वास्मा॑ अस्मा ए॒व स स ए॒वास्मै᳚ । \newline
13. ए॒वास्मा॑ अस्मा ए॒वैवास्मा॒ अन्न॒ मन्न॑ मस्मा ए॒वैवास्मा॒ अन्न᳚म् । \newline
14. अ॒स्मा॒ अन्न॒ मन्न॑ मस्मा अस्मा॒ अन्न॒म् प्र प्रान्न॑ मस्मा अस्मा॒ अन्न॒म् प्र । \newline
15. अन्न॒म् प्र प्रान्न॒ मन्न॒म् प्र य॑च्छति यच्छति॒ प्रान्न॒ मन्न॒म् प्र य॑च्छति । \newline
16. प्र य॑च्छति यच्छति॒ प्र प्र य॑च्छ त्यनमी॒वस्या॑ नमी॒वस्य॑ यच्छति॒ प्र प्र य॑च्छ त्यनमी॒वस्य॑ । \newline
17. य॒च्छ॒ त्य॒न॒मी॒वस्या॑ नमी॒वस्य॑ यच्छति यच्छ त्यनमी॒वस्य॑ शु॒ष्मिणः॑ शु॒ष्मिणो॑ ऽनमी॒वस्य॑ यच्छति यच्छ त्यनमी॒वस्य॑ शु॒ष्मिणः॑ । \newline
18. अ॒न॒मी॒वस्य॑ शु॒ष्मिणः॑ शु॒ष्मिणो॑ ऽनमी॒वस्या॑ नमी॒वस्य॑ शु॒ष्मिण॒ इतीति॑ शु॒ष्मिणो॑ ऽनमी॒वस्या॑ नमी॒वस्य॑ शु॒ष्मिण॒ इति॑ । \newline
19. शु॒ष्मिण॒ इतीति॑ शु॒ष्मिणः॑ शु॒ष्मिण॒ इत्या॑हा॒हेति॑ शु॒ष्मिणः॑ शु॒ष्मिण॒ इत्या॑ह । \newline
20. इत्या॑हा॒हेती त्या॑हा य॒क्ष्मस्या॑ य॒क्ष्मस्या॒हेती त्या॑हाय॒क्ष्मस्य॑ । \newline
21. आ॒हा॒य॒क्ष्मस्या॑ य॒क्ष्मस्या॑ हाहा य॒क्ष्मस्ये तीत्य॑य॒क्ष्मस्या॑ हाहा य॒क्ष्मस्येति॑ । \newline
22. अ॒य॒क्ष्मस्येती त्य॑य॒क्ष्मस्या॑ य॒क्ष्मस्येति॒ वाव वावे त्य॑य॒क्ष्मस्या॑ य॒क्ष्मस्येति॒ वाव । \newline
23. इति॒ वाव वावे तीति॒ वावैत दे॒तद् वावे तीति॒ वावैतत् । \newline
24. वावैत दे॒तद् वाव वावैत दा॑हा है॒तद् वाव वावैतदा॑ह । \newline
25. ए॒तदा॑हा है॒त दे॒तदा॑ह॒ प्र प्राहै॒त दे॒तदा॑ह॒ प्र । \newline
26. आ॒ह॒ प्र प्राहा॑ह॒ प्र प्र॑दा॒तार॑म् प्रदा॒तार॒म् प्राहा॑ह॒ प्र प्र॑दा॒तार᳚म् । \newline
27. प्र प्र॑दा॒तार॑म् प्रदा॒तार॒म् प्र प्र प्र॑दा॒तार॑म् तारिष स्तारिषः प्रदा॒तार॒म् प्र प्र प्र॑दा॒तार॑म् तारिषः । \newline
28. प्र॒दा॒तार॑म् तारिष स्तारिषः प्रदा॒तार॑म् प्रदा॒तार॑म् तारिष॒ ऊर्ज॒ मूर्ज॑म् तारिषः प्रदा॒तार॑म् प्रदा॒तार॑म् तारिष॒ ऊर्ज᳚म् । \newline
29. प्र॒दा॒तार॒मिति॑ प्र - दा॒तार᳚म् । \newline
30. ता॒रि॒ष॒ ऊर्ज॒ मूर्ज॑म् तारिष स्तारिष॒ ऊर्ज॑न्नो न॒ ऊर्ज॑म् तारिष स्तारिष॒ ऊर्ज॑न्नः । \newline
31. ऊर्ज॑न्नो न॒ ऊर्ज॒ मूर्ज॑न्नो धेहि धेहि न॒ ऊर्ज॒ मूर्ज॑न्नो धेहि । \newline
32. नो॒ धे॒हि॒ धे॒हि॒ नो॒ नो॒ धे॒हि॒ द्वि॒पदे᳚ द्वि॒पदे॑ धेहि नो नो धेहि द्वि॒पदे᳚ । \newline
33. धे॒हि॒ द्वि॒पदे᳚ द्वि॒पदे॑ धेहि धेहि द्वि॒पदे॒ चतु॑ष्पदे॒ चतु॑ष्पदे द्वि॒पदे॑ धेहि धेहि द्वि॒पदे॒ चतु॑ष्पदे । \newline
34. द्वि॒पदे॒ चतु॑ष्पदे॒ चतु॑ष्पदे द्वि॒पदे᳚ द्वि॒पदे॒ चतु॑ष्पद॒ इतीति॒ चतु॑ष्पदे द्वि॒पदे᳚ द्वि॒पदे॒ चतु॑ष्पद॒ इति॑ । \newline
35. द्वि॒पद॒ इति॑ द्वि - पदे᳚ । \newline
36. चतु॑ष्पद॒ इतीति॒ चतु॑ष्पदे॒ चतु॑ष्पद॒ इत्या॑हा॒हेति॒ चतु॑ष्पदे॒ चतु॑ष्पद॒ इत्या॑ह । \newline
37. चतु॑ष्पद॒ इति॒ चतुः॑ - प॒दे॒ । \newline
38. इत्या॑हा॒हेती त्या॑हा॒शिष॑ मा॒शिष॑ मा॒हेती त्या॑हा॒शिष᳚म् । \newline
39. आ॒हा॒शिष॑ मा॒शिष॑ माहाहा॒शिष॑ मे॒वैवाशिष॑ माहाहा॒शिष॑ मे॒व । \newline
40. आ॒शिष॑ मे॒वैवाशिष॑ मा॒शिष॑ मे॒वैता मे॒ता मे॒वाशिष॑ मा॒शिष॑ मे॒वैताम् । \newline
41. आ॒शिष॒मित्या᳚ - शिष᳚म् । \newline
42. ए॒वैता मे॒ता मे॒वैवैता मैता मे॒वैवैता मा । \newline
43. ए॒ता मैता मे॒ता मा शा᳚स्ते शास्त॒ ऐता मे॒ता मा शा᳚स्ते । \newline
44. आ शा᳚स्ते शास्त॒ आ शा᳚स्त॒ उदु च्छा᳚स्त॒ आ शा᳚स्त॒ उत् । \newline
45. शा॒स्त॒ उदु च्छा᳚स्ते शास्त॒ उदु॑ वु॒ वु च्छा᳚स्ते शास्त॒ उदु॑ । \newline
46. उदु॑ वु॒ वु दुदु॑ त्वा त्व॒ वु दुदु॑ त्वा । \newline
47. उ॒ त्वा॒ त्व॒ वु॒ त्वा॒ विश्वे॒ विश्वे᳚ त्व वु त्वा॒ विश्वे᳚ । \newline
48. त्वा॒ विश्वे॒ विश्वे᳚ त्वा त्वा॒ विश्वे॑ दे॒वा दे॒वा विश्वे᳚ त्वा त्वा॒ विश्वे॑ दे॒वाः । \newline
49. विश्वे॑ दे॒वा दे॒वा विश्वे॒ विश्वे॑ दे॒वा इतीति॑ दे॒वा विश्वे॒ विश्वे॑ दे॒वा इति॑ । \newline
50. दे॒वा इतीति॑ दे॒वा दे॒वा इत्या॑ हा॒हेति॑ दे॒वा दे॒वा इत्या॑ह । \newline
51. इत्या॑हा॒हे तीत्या॑ह प्रा॒णाः प्रा॒णा आ॒हे तीत्या॑ह प्रा॒णाः । \newline
52. आ॒ह॒ प्रा॒णाः प्रा॒णा आ॑हाह प्रा॒णा वै वै प्रा॒णा आ॑हाह प्रा॒णा वै । \newline
53. प्रा॒णा वै वै प्रा॒णाः प्रा॒णा वै विश्वे॒ विश्वे॒ वै प्रा॒णाः प्रा॒णा वै विश्वे᳚ । \newline
54. प्रा॒णा इति॑ प्र - अ॒नाः । \newline
55. वै विश्वे॒ विश्वे॒ वै वै विश्वे॑ दे॒वा दे॒वा विश्वे॒ वै वै विश्वे॑ दे॒वाः । \newline
56. विश्वे॑ दे॒वा दे॒वा विश्वे॒ विश्वे॑ दे॒वाः प्रा॒णैः प्रा॒णैर् दे॒वा विश्वे॒ विश्वे॑ दे॒वाः प्रा॒णैः । \newline
57. दे॒वाः प्रा॒णैः प्रा॒णैर् दे॒वा दे॒वाः प्रा॒णै रे॒वैव प्रा॒णैर् दे॒वा दे॒वाः प्रा॒णैरे॒व । \newline
\pagebreak
\markright{ TS 5.2.2.2  \hfill https://www.vedavms.in \hfill}

\section{ TS 5.2.2.2 }

\textbf{TS 5.2.2.2 } \newline
\textbf{Samhita Paata} \newline

प्रा॒णैरे॒वैन॒मुद्य॑च्छ॒ते ऽग्ने॒ भर॑न्तु॒ चित्ति॑भि॒रित्या॑ह॒ यस्मा॑ ए॒वैनं॑ चि॒त्तायो॒द्यच्छ॑ते॒ तेनै॒वैनꣳ॒॒ सम॑र्द्धयति चत॒सृभि॒रा सा॑दयति च॒त्वारि॒ छन्दाꣳ॑सि॒ छन्दो॑भिरे॒वा-ति॑च्छन्दसोत्त॒मया॒ वर्ष्म॒ वा ए॒षा छन्द॑सां॒ ॅयदति॑च्छन्दा॒ वर्ष्मै॒वैनꣳ॑ समा॒नानां᳚ करोति॒ सद्व॑ती भवति स॒त्त्वमे॒वैनं॑ गमयति॒ प्रेद॑ग्ने॒ ज्योति॑ष्मान् - [  ] \newline

\textbf{Pada Paata} \newline

प्रा॒णैरिति॑ प्र - अ॒नैः । ए॒व । ए॒न॒म् । उदिति॑ । य॒च्छ॒ते॒ । अग्ने᳚ । भर॑न्तु । चित्ति॑भि॒रिति॒ चित्ति॑ - भिः॒ । इति॑ । आ॒ह॒ । यस्मै᳚ । ए॒व । ए॒न॒म् । चि॒त्ताय॑ । उ॒द्यच्छ॑त॒ इत्यु॑त् - यच्छ॑ते । तेन॑ । ए॒व । ए॒न॒म् । समिति॑ । अ॒द्‌र्ध॒य॒ति॒ । च॒त॒सृभि॒रिति॑ चत॒सृ - भिः॒ । एति॑ । सा॒द॒य॒ति॒ । च॒त्वारि॑ । छन्दाꣳ॑सि । छन्दो॑भि॒रिति॒ छन्दः॑ - भिः॒ । ए॒व । अति॑च्छन्द॒सेत्यति॑ - छ॒न्द॒सा॒ । उ॒त्त॒मयेत्यु॑त् - त॒मया᳚ । वर्ष्म॑ । वै । ए॒षा । छन्द॑साम् । यत् । अति॑च्छन्दा॒ इत्यति॑ - छ॒न्दाः॒ । वर्ष्म॑ । ए॒व । ए॒न॒म् । स॒मा॒नाना᳚म् । क॒रो॒ति॒ । सद्व॒तीति॒ सत् - व॒ती॒ । भ॒व॒ति॒ । स॒त्त्वमिति॑ सत्-त्वम् । ए॒व । ए॒न॒म् । ग॒म॒य॒ति॒ । प्रेति॑ । इत् । अ॒ग्ने॒ । ज्योति॑ष्मान् ।  \newline


\textbf{Krama Paata} \newline

प्रा॒णैरे॒व । प्रा॒णैरिति॑ प्र - अ॒नैः । ए॒वैन᳚म् । ए॒न॒मुत् । उद् य॑च्छते । य॒च्छ॒तेऽग्ने᳚ । अग्ने॒ भर॑न्तु । भर॑न्तु॒ चित्ति॑भिः । चित्ति॑भि॒रिति॑ । चित्ति॑भि॒रिति॒ चित्ति॑ - भिः॒ । इत्या॑ह । आ॒ह॒ यस्मै᳚ । यस्मा॑ ए॒व । ए॒वैन᳚म् । ए॒न॒म् चि॒त्ताय॑ । चि॒त्तायो॒द्यच्छ॑ते । उ॒द्यच्छ॑ते॒ तेन॑ । उ॒द्यच्छ॑त॒ इत्यु॑त् - यच्छ॑ते । तेनै॒व । ए॒वैन᳚म् । ए॒नꣳ॒॒ सम् । सम॑र्द्धयति । अ॒र्द्ध॒य॒ति॒ च॒त॒सृभिः॑ । च॒त॒सृभि॒रा । च॒त॒सृभि॒रिति॑ चत॒सृ - भिः॒ । आ सा॑दयति । सा॒द॒य॒ति॒ च॒त्वारि॑ । च॒त्वारि॒ छन्दाꣳ॑सि । छन्दाꣳ॑सि॒ छन्दो॑भिः । छन्दो॑भिरे॒व । छन्दो॑भि॒रिति॒ छन्दः॑ - भिः॒ । ए॒वाति॑च्छन्दसा । अति॑च्छन्दसोत्त॒मया᳚ । अति॑च्छन्द॒सेत्यति॑ - छ॒न्द॒सा॒ । उ॒त्त॒मया॒ वर्ष्म॑ । उ॒त्त॒मयेत्यु॑त् - त॒मया᳚ । वर्ष्म॒ वै । वा ए॒षा । ए॒षा छन्द॑साम् । छन्द॑सा॒म् ॅयत् । यदति॑च्छन्दाः । अति॑च्छन्दा॒ वर्ष्म॑ । अति॑च्छन्दा॒ इत्यति॑ - छ॒न्दाः॒ । वर्ष्मै॒व । ए॒वैन᳚म् । ए॒नꣳ॒॒ स॒मा॒नाना᳚म् । स॒मा॒नाना᳚म् करोति । क॒रो॒ति॒ सद्व॑ती । सद्व॑ती भवति । सद्व॒तीति॒ सत् - व॒ती॒ । भ॒व॒ति॒ स॒त्वम् । स॒त्वमे॒व । स॒त्वमिति॑ सत् - त्वम् । ए॒वैन᳚म् । ए॒न॒म् ग॒म॒य॒ति॒ । ग॒म॒य॒ति॒ प्र । प्रेत् । इद॑ग्ने । अ॒ग्ने॒ ज्योति॑ष्मान् । ज्योति॑ष्मान्. याहि \newline

\textbf{Jatai Paata} \newline

1. प्रा॒णै रे॒वैव प्रा॒णैः प्रा॒णै रे॒व । \newline
2. प्रा॒णैरिति॑ प्र - अ॒नैः । \newline
3. ए॒वैन॑ मेन मे॒वै वैन᳚म् । \newline
4. ए॒न॒ मुदु दे॑न मेन॒ मुत् । \newline
5. उद् य॑च्छते यच्छत॒ उदुद् य॑च्छते । \newline
6. य॒च्छ॒ते ऽग्ने ऽग्ने॑ यच्छते यच्छ॒ते ऽग्ने᳚ । \newline
7. अग्ने॒ भर॑न्तु॒ भर॒न् त्वग्ने ऽग्ने॒ भर॑न्तु । \newline
8. भर॑न्तु॒ चित्ति॑भि॒ श्चित्ति॑भि॒र् भर॑न्तु॒ भर॑न्तु॒ चित्ति॑भिः । \newline
9. चित्ति॑भि॒ रितीति॒ चित्ति॑भि॒ श्चित्ति॑भि॒ रिति॑ । \newline
10. चित्ति॑भि॒रिति॒ चित्ति॑ - भिः॒ । \newline
11. इत्या॑हा॒हे तीत्या॑ह । \newline
12. आ॒ह॒ यस्मै॒ यस्मा॑ आहाह॒ यस्मै᳚ । \newline
13. यस्मा॑ ए॒वैव यस्मै॒ यस्मा॑ ए॒व । \newline
14. ए॒वैन॑ मेन मे॒वै वैन᳚म् । \newline
15. ए॒न॒म् चि॒त्ताय॑ चि॒त्तायै॑न मेनम् चि॒त्ताय॑ । \newline
16. चि॒त्तायो॒ द्यच्छ॑त उ॒द्यच्छ॑ते चि॒त्ताय॑ चि॒त्तायो॒ द्यच्छ॑ते । \newline
17. उ॒द्यच्छ॑ते॒ तेन॒ तेनो॒ द्यच्छ॑त उ॒द्यच्छ॑ते॒ तेन॑ । \newline
18. उ॒द्यच्छ॑त॒ इत्यु॑त् - यच्छ॑ते । \newline
19. तेनै॒ वैव तेन॒ तेनै॒व । \newline
20. ए॒वैन॑ मेन मे॒वै वैन᳚म् । \newline
21. ए॒नꣳ॒॒ सꣳ स मे॑न मेनꣳ॒॒ सम् । \newline
22. स म॑र्द्धय त्यर्द्धयति॒ सꣳ स म॑र्द्धयति । \newline
23. अ॒र्द्ध॒य॒ति॒ च॒त॒सृभि॑ श्चत॒सृभि॑ रर्द्धय त्यर्द्धयति चत॒सृभिः॑ । \newline
24. च॒त॒सृभि॒रा च॑त॒सृभि॑ श्चत॒सृभि॒रा । \newline
25. च॒त॒सृभि॒रिति॑ चत॒सृ - भिः॒ । \newline
26. आ सा॑दयति सादय॒त्या सा॑दयति । \newline
27. सा॒द॒य॒ति॒ च॒त्वारि॑ च॒त्वारि॑ सादयति सादयति च॒त्वारि॑ । \newline
28. च॒त्वारि॒ छन्दाꣳ॑सि॒ छन्दाꣳ॑सि च॒त्वारि॑ च॒त्वारि॒ छन्दाꣳ॑सि । \newline
29. छन्दाꣳ॑सि॒ छन्दो॑भि॒ श्छन्दो॑भि॒ श्छन्दाꣳ॑सि॒ छन्दाꣳ॑सि॒ छन्दो॑भिः । \newline
30. छन्दो॑भि रे॒वैव छन्दो॑भि॒ श्छन्दो॑भि रे॒व । \newline
31. छन्दो॑भि॒रिति॒ छन्दः॑ - भिः॒ । \newline
32. ए॒वा ति॑च्छन्द॒सा ऽति॑च्छन्दसै॒ वैवा ति॑च्छन्दसा । \newline
33. अति॑च्छन्द सोत्त॒म यो᳚त्त॒मया ऽति॑च्छन्द॒सा ऽति॑च्छन्द सोत्त॒मया᳚ । \newline
34. अति॑च्छन्द॒सेत्यति॑ - छ॒न्द॒सा॒ । \newline
35. उ॒त्त॒मया॒ वर्ष्म॒ वर्ष्मो᳚ त्त॒मयो᳚ त्त॒मया॒ वर्ष्म॑ । \newline
36. उ॒त्त॒मयेत्यु॑त् - त॒मया᳚ । \newline
37. वर्ष्म॒ वै वै वर्ष्म॒ वर्ष्म॒ वै । \newline
38. वा ए॒षैषा वै वा ए॒षा । \newline
39. ए॒षा छन्द॑सा॒म् छन्द॑सा मे॒षैषा छन्द॑साम् । \newline
40. छन्द॑सां॒ ॅयद् यच् छन्द॑सा॒म् छन्द॑सां॒ ॅयत् । \newline
41. यदति॑च्छन्दा॒ अति॑च्छन्दा॒ यद् यदति॑च्छन्दाः । \newline
42. अति॑च्छन्दा॒ वर्ष्म॒ वर्ष्मा ति॑च्छन्दा॒ अति॑च्छन्दा॒ वर्ष्म॑ । \newline
43. अति॑च्छन्दा॒ इत्यति॑ - छ॒न्दाः॒ । \newline
44. वर्ष्मै॒ वैव वर्ष्म॒ वर्ष्मै॒व । \newline
45. ए॒वैन॑ मेन मे॒वै वैन᳚म् । \newline
46. ए॒नꣳ॒॒ स॒मा॒नानाꣳ॑ समा॒नाना॑ मेन मेनꣳ समा॒नाना᳚म् । \newline
47. स॒मा॒नाना᳚म् करोति करोति समा॒नानाꣳ॑ समा॒नाना᳚म् करोति । \newline
48. क॒रो॒ति॒ सद्व॑ती॒ सद्व॑ती करोति करोति॒ सद्व॑ती । \newline
49. सद्व॑ती भवति भवति॒ सद्व॑ती॒ सद्व॑ती भवति । \newline
50. सद्व॒तीति॒ सत् - व॒ती॒ । \newline
51. भ॒व॒ति॒ स॒त्त्वꣳ स॒त्त्वम् भ॑वति भवति स॒त्त्वम् । \newline
52. स॒त्त्व मे॒वैव स॒त्त्वꣳ स॒त्त्व मे॒व । \newline
53. स॒त्त्वमिति॑ सत् - त्वम् । \newline
54. ए॒वैन॑ मेन मे॒वै वैन᳚म् । \newline
55. ए॒न॒म् ग॒म॒य॒ति॒ ग॒म॒य॒ त्ये॒न॒ मे॒न॒म् ग॒म॒य॒ति॒ । \newline
56. ग॒म॒य॒ति॒ प्र प्र ग॑मयति गमयति॒ प्र । \newline
57. प्रेदित् प्र प्रेत् । \newline
58. इद॑ग्ने अग्न॒ इदि द॑ग्ने । \newline
59. अ॒ग्ने॒ ज्योति॑ष्मा॒न् ज्योति॑ष्मा नग्ने अग्ने॒ ज्योति॑ष्मान् । \newline
60. ज्योति॑ष्मान्. याहि याहि॒ ज्योति॑ष्मा॒न् ज्योति॑ष्मान्. याहि । \newline

\textbf{Ghana Paata } \newline

1. प्रा॒णै रे॒वैव प्रा॒णैः प्रा॒णै रे॒वैन॑ मेन मे॒व प्रा॒णैः प्रा॒णै रे॒वैन᳚म् । \newline
2. प्रा॒णैरिति॑ प्र - अ॒नैः । \newline
3. ए॒वैन॑ मेन मे॒वैवैन॒ मुदुदे॑न मे॒वैवैन॒ मुत् । \newline
4. ए॒न॒ मुदुदे॑न मेन॒ मुद् य॑च्छते यच्छत॒ उदे॑न मेन॒ मुद् य॑च्छते । \newline
5. उद् य॑च्छते यच्छत॒ उदुद् य॑च्छ॒ते ऽग्ने ऽग्ने॑ यच्छत॒ उदुद् य॑च्छ॒ते ऽग्ने᳚ । \newline
6. य॒च्छ॒ते ऽग्ने ऽग्ने॑ यच्छते यच्छ॒ते ऽग्ने॒ भर॑न्तु॒ भर॒न्त्वग्ने॑ यच्छते यच्छ॒ते ऽग्ने॒ भर॑न्तु । \newline
7. अग्ने॒ भर॑न्तु॒ भर॒न्त्वग्ने ऽग्ने॒ भर॑न्तु॒ चित्ति॑भि॒ श्चित्ति॑भि॒र् भर॒न्त्वग्ने ऽग्ने॒ भर॑न्तु॒ चित्ति॑भिः । \newline
8. भर॑न्तु॒ चित्ति॑भि॒ श्चित्ति॑भि॒र् भर॑न्तु॒ भर॑न्तु॒ चित्ति॑भि॒ रितीति॒ चित्ति॑भि॒र् भर॑न्तु॒ भर॑न्तु॒ चित्ति॑भि॒ रिति॑ । \newline
9. चित्ति॑भि॒ रितीति॒ चित्ति॑भि॒ श्चित्ति॑भि॒ रित्या॑हा॒हेति॒ चित्ति॑भि॒ श्चित्ति॑भि॒ रित्या॑ह । \newline
10. चित्ति॑भि॒रिति॒ चित्ति॑ - भिः॒ । \newline
11. इत्या॑हा॒हे तीत्या॑ह॒ यस्मै॒ यस्मा॑ आ॒हे तीत्या॑ह॒ यस्मै᳚ । \newline
12. आ॒ह॒ यस्मै॒ यस्मा॑ आहाह॒ यस्मा॑ ए॒वैव यस्मा॑ आहाह॒ यस्मा॑ ए॒व । \newline
13. यस्मा॑ ए॒वैव यस्मै॒ यस्मा॑ ए॒वैन॑ मेन मे॒व यस्मै॒ यस्मा॑ ए॒वैन᳚म् । \newline
14. ए॒वैन॑ मेन मे॒वैवैन॑म् चि॒त्ताय॑ चि॒त्तायै॑न मे॒वैवैन॑म् चि॒त्ताय॑ । \newline
15. ए॒न॒म् चि॒त्ताय॑ चि॒त्तायै॑न मेनम् चि॒त्ता यो॒द्यच्छ॑त उ॒द्यच्छ॑ते चि॒त्तायै॑न मेनम् चि॒त्ता यो॒द्यच्छ॑ते । \newline
16. चि॒त्ता यो॒द्यच्छ॑त उ॒द्यच्छ॑ते चि॒त्ताय॑ चि॒त्ता यो॒द्यच्छ॑ते॒ तेन॒ तेनो॒द्यच्छ॑ते चि॒त्ताय॑ चि॒त्ता यो॒द्यच्छ॑ते॒ तेन॑ । \newline
17. उ॒द्यच्छ॑ते॒ तेन॒ तेनो॒द्यच्छ॑त उ॒द्यच्छ॑ते॒ तेनै॒वैव तेनो॒द्यच्छ॑त उ॒द्यच्छ॑ते॒ तेनै॒व । \newline
18. उ॒द्यच्छ॑त॒ इत्यु॑त् - यच्छ॑ते । \newline
19. तेनै॒वैव तेन॒ तेनै॒वैन॑ मेन मे॒व तेन॒ तेनै॒वैन᳚म् । \newline
20. ए॒वैन॑ मेन मे॒वैवैनꣳ॒॒ सꣳ स मे॑न मे॒वैवैनꣳ॒॒ सम् । \newline
21. ए॒नꣳ॒॒ सꣳ स मे॑न मेनꣳ॒॒ स म॑र्द्धय त्यर्द्धयति॒ स मे॑न मेनꣳ॒॒ स म॑र्द्धयति । \newline
22. स म॑र्द्धय त्यर्द्धयति॒ सꣳ स म॑र्द्धयति चत॒सृभि॑ श्चत॒सृभि॑ रर्द्धयति॒ सꣳ स म॑र्द्धयति चत॒सृभिः॑ । \newline
23. अ॒र्द्ध॒य॒ति॒ च॒त॒सृभि॑ श्चत॒सृभि॑ रर्द्धय त्यर्द्धयति चत॒सृभि॒रा च॑त॒सृभि॑ रर्द्धय त्यर्द्धयति चत॒सृभि॒रा । \newline
24. च॒त॒सृभि॒रा च॑त॒सृभि॑ श्चत॒सृभि॒रा सा॑दयति सादय॒त्या च॑त॒सृभि॑ श्चत॒सृभि॒रा सा॑दयति । \newline
25. च॒त॒सृभि॒रिति॑ चत॒सृ - भिः॒ । \newline
26. आ सा॑दयति सादय॒त्या सा॑दयति च॒त्वारि॑ च॒त्वारि॑ सादय॒त्या सा॑दयति च॒त्वारि॑ । \newline
27. सा॒द॒य॒ति॒ च॒त्वारि॑ च॒त्वारि॑ सादयति सादयति च॒त्वारि॒ छन्दाꣳ॑सि॒ छन्दाꣳ॑सि च॒त्वारि॑ सादयति सादयति च॒त्वारि॒ छन्दाꣳ॑सि । \newline
28. च॒त्वारि॒ छन्दाꣳ॑सि॒ छन्दाꣳ॑सि च॒त्वारि॑ च॒त्वारि॒ छन्दाꣳ॑सि॒ छन्दो॑भि॒ श्छन्दो॑भि॒ श्छन्दाꣳ॑सि च॒त्वारि॑ च॒त्वारि॒ छन्दाꣳ॑सि॒ छन्दो॑भिः । \newline
29. छन्दाꣳ॑सि॒ छन्दो॑भि॒ श्छन्दो॑भि॒ श्छन्दाꣳ॑सि॒ छन्दाꣳ॑सि॒ छन्दो॑भि रे॒वैव छन्दो॑भि॒ श्छन्दाꣳ॑सि॒ छन्दाꣳ॑सि॒ छन्दो॑भि रे॒व । \newline
30. छन्दो॑भि रे॒वैव छन्दो॑भि॒ श्छन्दो॑भि रे॒वाति॑च्छन्द॒सा ऽति॑च्छन्दसै॒व छन्दो॑भि॒ श्छन्दो॑भि रे॒वाति॑च्छन्दसा । \newline
31. छन्दो॑भि॒रिति॒ छन्दः॑ - भिः॒ । \newline
32. ए॒वाति॑च्छन्द॒सा ऽति॑च्छन्द सै॒वैवाति॑च्छन्द सोत्त॒म यो᳚त्त॒मया ऽति॑च्छन्द सै॒वैवाति॑च्छन्द सोत्त॒मया᳚ । \newline
33. अति॑च्छन्द सोत्त॒म यो᳚त्त॒मया ऽति॑च्छन्द॒सा ऽति॑च्छन्द सोत्त॒मया॒ वर्ष्म॒ वर्ष्मो᳚त्त॒मया ऽति॑च्छन्द॒सा ऽति॑च्छन्द सोत्त॒मया॒ वर्ष्म॑ । \newline
34. अति॑च्छन्द॒सेत्यति॑ - छ॒न्द॒सा॒ । \newline
35. उ॒त्त॒मया॒ वर्ष्म॒ वर्ष्मो᳚त्त॒म यो᳚त्त॒मया॒ वर्ष्म॒ वै वै वर्ष्मो᳚त्त॒म यो᳚त्त॒मया॒ वर्ष्म॒ वै । \newline
36. उ॒त्त॒मयेत्यु॑त् - त॒मया᳚ । \newline
37. वर्ष्म॒ वै वै वर्ष्म॒ वर्ष्म॒ वा ए॒षैषा वै वर्ष्म॒ वर्ष्म॒ वा ए॒षा । \newline
38. वा ए॒षैषा वै वा ए॒षा छन्द॑सा॒म् छन्द॑सा मे॒षा वै वा ए॒षा छन्द॑साम् । \newline
39. ए॒षा छन्द॑सा॒म् छन्द॑सा मे॒षैषा छन्द॑सां॒ ॅयद् यच् छन्द॑सा मे॒षैषा छन्द॑सां॒ ॅयत् । \newline
40. छन्द॑सां॒ ॅयद् यच् छन्द॑सा॒म् छन्द॑सां॒ ॅयदति॑च्छन्दा॒ अति॑च्छन्दा॒ यच् छन्द॑सा॒म् छन्द॑सां॒ ॅयदति॑च्छन्दाः । \newline
41. यदति॑च्छन्दा॒ अति॑च्छन्दा॒ यद् यदति॑च्छन्दा॒ वर्ष्म॒ वर्ष्माति॑च्छन्दा॒ यद् यदति॑च्छन्दा॒ वर्ष्म॑ । \newline
42. अति॑च्छन्दा॒ वर्ष्म॒ वर्ष्माति॑च्छन्दा॒ अति॑च्छन्दा॒ वर्ष्मै॒वैव वर्ष्माति॑च्छन्दा॒ अति॑च्छन्दा॒ वर्ष्मै॒व । \newline
43. अति॑च्छन्दा॒ इत्यति॑ - छ॒न्दाः॒ । \newline
44. वर्ष्मै॒वैव वर्ष्म॒ वर्ष्मै॒वैन॑ मेन मे॒व वर्ष्म॒ वर्ष्मै॒वैन᳚म् । \newline
45. ए॒वैन॑ मेन मे॒वैवैनꣳ॑ समा॒नानाꣳ॑ समा॒नाना॑ मेन मे॒वैवैनꣳ॑ समा॒नाना᳚म् । \newline
46. ए॒नꣳ॒॒ स॒मा॒नानाꣳ॑ समा॒नाना॑ मेन मेनꣳ समा॒नाना᳚म् करोति करोति समा॒नाना॑ मेन मेनꣳ समा॒नाना᳚म् करोति । \newline
47. स॒मा॒नाना᳚म् करोति करोति समा॒नानाꣳ॑ समा॒नाना᳚म् करोति॒ सद्व॑ती॒ सद्व॑ती करोति समा॒नानाꣳ॑ समा॒नाना᳚म् करोति॒ सद्व॑ती । \newline
48. क॒रो॒ति॒ सद्व॑ती॒ सद्व॑ती करोति करोति॒ सद्व॑ती भवति भवति॒ सद्व॑ती करोति करोति॒ सद्व॑ती भवति । \newline
49. सद्व॑ती भवति भवति॒ सद्व॑ती॒ सद्व॑ती भवति स॒त्त्वꣳ स॒त्त्वम् भ॑वति॒ सद्व॑ती॒ सद्व॑ती भवति स॒त्त्वम् । \newline
50. सद्व॒तीति॒ सत् - व॒ती॒ । \newline
51. भ॒व॒ति॒ स॒त्त्वꣳ स॒त्त्वम् भ॑वति भवति स॒त्त्व मे॒वैव स॒त्त्वम् भ॑वति भवति स॒त्त्व मे॒व । \newline
52. स॒त्त्व मे॒वैव स॒त्त्वꣳ स॒त्त्व मे॒वैन॑ मेन मे॒व स॒त्त्वꣳ स॒त्त्व मे॒वैन᳚म् । \newline
53. स॒त्त्वमिति॑ सत् - त्वम् । \newline
54. ए॒वैन॑ मेन मे॒वैवैन॑म् गमयति गमय त्येन मे॒वैवैन॑म् गमयति । \newline
55. ए॒न॒म् ग॒म॒य॒ति॒ ग॒म॒य॒ त्ये॒न॒ मे॒न॒म् ग॒म॒य॒ति॒ प्र प्र ग॑मय त्येन मेनम् गमयति॒ प्र । \newline
56. ग॒म॒य॒ति॒ प्र प्र ग॑मयति गमयति॒ प्रेदित् प्र ग॑मयति गमयति॒ प्रेत् । \newline
57. प्रे दित् प्र प्रे द॑ग्ने अग्न॒ इत् प्र प्रे द॑ग्ने । \newline
58. इद॑ग्ने अग्न॒ इदिद॑ग्ने॒ ज्योति॑ष्मा॒न् ज्योति॑ष्मा नग्न॒ इदिद॑ग्ने॒ ज्योति॑ष्मान् । \newline
59. अ॒ग्ने॒ ज्योति॑ष्मा॒न् ज्योति॑ष्मा नग्ने अग्ने॒ ज्योति॑ष्मान्. याहि याहि॒ ज्योति॑ष्मा नग्ने अग्ने॒ ज्योति॑ष्मान्. याहि । \newline
60. ज्योति॑ष्मान्. याहि याहि॒ ज्योति॑ष्मा॒न् ज्योति॑ष्मान्. या॒हीतीति॑ याहि॒ ज्योति॑ष्मा॒न् ज्योति॑ष्मान्. या॒हीति॑ । \newline
\pagebreak
\markright{ TS 5.2.2.3  \hfill https://www.vedavms.in \hfill}

\section{ TS 5.2.2.3 }

\textbf{TS 5.2.2.3 } \newline
\textbf{Samhita Paata} \newline

या॒हीत्या॑ह॒ ज्योति॑रे॒वास्मि॑न् दधाति त॒नुवा॒ वा ए॒ष हि॑नस्ति॒ यꣳ हि॒नस्ति॒ मा हिꣳ॑सीस्त॒नुवा᳚ प्र॒जा इत्या॑ह प्र॒जाभ्य॑ ए॒वैनꣳ॑ शमयति॒ रक्षाꣳ॑सि॒ वा ए॒तद्-य॒ज्ञ्ꣳ स॑चन्ते॒ यदन॑ उ॒थ्सर्ज॒-त्यक्र॑न्द॒दित्यन्वा॑ह॒ रक्ष॑सा॒मप॑हत्या॒ अन॑सा वह॒न्-त्यप॑चिति-मे॒वास्मि॑न् दधाति॒ तस्मा॑दन॒स्वी च॑ र॒थी चाति॑थीना॒-मप॑चिततमा॒ - [  ] \newline

\textbf{Pada Paata} \newline

या॒हि॒ । इति॑ । आ॒ह॒ । ज्योतिः॑ । ए॒व । अ॒स्मि॒न्न् । द॒धा॒ति॒ । त॒नुवा᳚ । वै । ए॒षः । हि॒न॒स्ति॒ । यम् । हि॒नस्ति॑ । मा । हिꣳ॒॒सीः॒ । त॒नुवा᳚ । प्र॒जा इति॑ प्र - जाः । इति॑ । आ॒ह॒ । प्र॒जाभ्य॒ इति॑ प्र - जाभ्यः॑ । ए॒व । ए॒न॒म् । श॒म॒य॒ति॒ । रक्षाꣳ॑सि । वै । ए॒तत् । य॒ज्ञ्म् । स॒च॒न्ते॒ । यत् । अनः॑ । उ॒थ्सर्ज॒तीत्यु॑त्-सर्ज॑ति । अक्र॑न्दत् । इति॑ । अन्विति॑ । आ॒ह॒ । रक्ष॑साम् । अप॑हत्या॒ इत्यप॑ - ह॒त्यै॒ । अन॑सा । व॒ह॒न्ति॒ । अप॑चिति॒मित्यप॑ - चि॒ति॒म् । ए॒व । अ॒स्मि॒न्न् । द॒धा॒ति॒ । तस्मा᳚त् । अ॒न॒स्वी । च॒ । र॒थी । च॒ । अति॑थीनाम् । अप॑चिततमा॒वित्यप॑चित - त॒मौ॒ ।  \newline


\textbf{Krama Paata} \newline

या॒हीति॑ । इत्या॑ह । आ॒ह॒ ज्योतिः॑ । ज्योति॑रे॒व । ए॒वास्मिन्न्॑ । अ॒स्मि॒न् द॒धा॒ति॒ । द॒धा॒ति॒ त॒नुवा᳚ । त॒नुवा॒ वै । वा ए॒षः । ए॒ष हि॑नस्ति । हि॒न॒स्ति॒ यम् । यꣳ हि॒नस्ति॑ । हि॒नस्ति॒ मा । मा हिꣳ॑सीः । हिꣳ॒॒सी॒स्त॒नुवा᳚ । त॒नुवा᳚ प्र॒जाः । प्र॒जा इति॑ । प्र॒जा इति॑ प्र - जाः । इत्या॑ह । आ॒ह॒ प्र॒जाभ्यः॑ । प्र॒जाभ्य॑ ए॒व । प्र॒जाभ्य॒ इति॑ प्र - जाभ्यः॑ । ए॒वैन᳚म् । ए॒नꣳ॒॒ श॒म॒य॒ति॒ । श॒म॒य॒ति॒ रक्षाꣳ॑सि । रक्षाꣳ॑सि॒ वै । वा ए॒तत् । ए॒तद् य॒ज्ञ्म् । य॒ज्ञ्ꣳ स॑चन्ते । स॒च॒न्ते॒ यत् । यदनः॑ । अन॑ उ॒थ्सर्ज॑ति । उ॒थ्सर्ज॒त्यक्र॑न्दत् । उ॒थ्सर्ज॒तीत्यु॑त् - सर्ज॑ति । अक्र॑न्द॒दिति॑ । इत्यनु॑ । अन्वा॑ह । आ॒ह॒ रक्ष॑साम् । रक्ष॑सा॒मप॑हत्यै । अप॑हत्या॒ अन॑सा । अप॑हत्या॒ इत्यप॑ - ह॒त्यै॒ । अन॑सा वहन्ति । व॒ह॒न्त्यप॑चितिम् । अप॑चितिमे॒व । अप॑चिति॒मित्यप॑ - चि॒ति॒म् । ए॒वास्मिन्न्॑ । अ॒स्मि॒न् द॒धा॒ति॒ । द॒धा॒ति॒ तस्मा᳚त् । तस्मा॑दन॒स्वी । अ॒न॒स्वी च॑ । च॒ र॒थी । र॒थी च॑ । चाति॑थीनाम् । अति॑थीना॒मप॑चिततमौ । अप॑चिततमा॒वप॑चितिमान् । अप॑चिततमा॒वित्यप॑चित - त॒मौ॒ \newline

\textbf{Jatai Paata} \newline

1. या॒हीतीति॑ याहि या॒हीति॑ । \newline
2. इत्या॑हा॒हे तीत्या॑ह । \newline
3. आ॒ह॒ ज्योति॒र् ज्योति॑ राहाह॒ ज्योतिः॑ । \newline
4. ज्योति॑ रे॒वैव ज्योति॒र् ज्योति॑ रे॒व । \newline
5. ए॒वास्मि॑न् नस्मिन् ने॒वै वास्मिन्न्॑ । \newline
6. अ॒स्मि॒न् द॒धा॒ति॒ द॒धा॒ त्य॒स्मि॒न् न॒स्मि॒न् द॒धा॒ति॒ । \newline
7. द॒धा॒ति॒ त॒नुवा॑ त॒नुवा॑ दधाति दधाति त॒नुवा᳚ । \newline
8. त॒नुवा॒ वै वै त॒नुवा॑ त॒नुवा॒ वै । \newline
9. वा ए॒ष ए॒ष वै वा ए॒षः । \newline
10. ए॒ष हि॑नस्ति हिन स्त्ये॒ष ए॒ष हि॑नस्ति । \newline
11. हि॒न॒स्ति॒ यं ॅयꣳ हि॑नस्ति हिनस्ति॒ यम् । \newline
12. यꣳ हि॒नस्ति॑ हि॒नस्ति॒ यं ॅयꣳ हि॒नस्ति॑ । \newline
13. हि॒नस्ति॒ मा मा हि॒नस्ति॑ हि॒नस्ति॒ मा । \newline
14. मा हिꣳ॑सीर्. हिꣳसी॒र् मा मा हिꣳ॑सीः । \newline
15. हिꣳ॒॒सी॒ स्त॒नुवा॑ त॒नुवा॑ हिꣳसीर्. हिꣳसी स्त॒नुवा᳚ । \newline
16. त॒नुवा᳚ प्र॒जाः प्र॒जा स्त॒नुवा॑ त॒नुवा᳚ प्र॒जाः । \newline
17. प्र॒जा इतीति॑ प्र॒जाः प्र॒जा इति॑ । \newline
18. प्र॒जा इति॑ प्र - जाः । \newline
19. इत्या॑हा॒हे तीत्या॑ह । \newline
20. आ॒ह॒ प्र॒जाभ्यः॑ प्र॒जाभ्य॑ आहाह प्र॒जाभ्यः॑ । \newline
21. प्र॒जाभ्य॑ ए॒वैव प्र॒जाभ्यः॑ प्र॒जाभ्य॑ ए॒व । \newline
22. प्र॒जाभ्य॒ इति॑ प्र - जाभ्यः॑ । \newline
23. ए॒वैन॑ मेन मे॒वै वैन᳚म् । \newline
24. ए॒नꣳ॒॒ श॒म॒य॒ति॒ श॒म॒य॒ त्ये॒न॒ मे॒नꣳ॒॒ श॒म॒य॒ति॒ । \newline
25. श॒म॒य॒ति॒ रक्षाꣳ॑सि॒ रक्षाꣳ॑सि शमयति शमयति॒ रक्षाꣳ॑सि । \newline
26. रक्षाꣳ॑सि॒ वै वै रक्षाꣳ॑सि॒ रक्षाꣳ॑सि॒ वै । \newline
27. वा ए॒त दे॒तद् वै वा ए॒तत् । \newline
28. ए॒तद् य॒ज्ञ्ं ॅय॒ज्ञ् मे॒त दे॒तद् य॒ज्ञ्म् । \newline
29. य॒ज्ञ्ꣳ स॑चन्ते सचन्ते य॒ज्ञ्ं ॅय॒ज्ञ्ꣳ स॑चन्ते । \newline
30. स॒च॒न्ते॒ यद् यथ् स॑चन्ते सचन्ते॒ यत् । \newline
31. य दनो ऽनो॒ यद् य दनः॑ । \newline
32. अन॑ उ॒थ्सर्ज॑ त्यु॒थ्सर्ज॒ त्यनो ऽन॑ उ॒थ्सर्ज॑ति । \newline
33. उ॒थ्सर्ज॒ त्यक्र॑न्द॒ दक्र॑न्द दु॒थ्सर्ज॑ त्यु॒थ्सर्ज॒ त्यक्र॑न्दत् । \newline
34. उ॒थ्सर्ज॒तीत्यु॑त् - सर्ज॑ति । \newline
35. अक्र॑न्द॒ दिती त्यक्र॑न्द॒ दक्र॑न्द॒ दिति॑ । \newline
36. इत्यन् वन् विती त्यनु॑ । \newline
37. अन्वा॑हा॒ हान् वन् वा॑ह । \newline
38. आ॒ह॒ रक्ष॑साꣳ॒॒ रक्ष॑सा माहाह॒ रक्ष॑साम् । \newline
39. रक्ष॑सा॒ मप॑हत्या॒ अप॑हत्यै॒ रक्ष॑साꣳ॒॒ रक्ष॑सा॒ मप॑हत्यै । \newline
40. अप॑हत्या॒ अन॒सा ऽन॒सा ऽप॑हत्या॒ अप॑हत्या॒ अन॑सा । \newline
41. अप॑हत्या॒ इत्यप॑ - ह॒त्यै॒ । \newline
42. अन॑सा वहन्ति वह॒न् त्यन॒सा ऽन॑सा वहन्ति । \newline
43. व॒ह॒न् त्यप॑चिति॒ मप॑चितिं ॅवहन्ति वह॒न् त्यप॑चितिम् । \newline
44. अप॑चिति मे॒वैवा प॑चिति॒ मप॑चिति मे॒व । \newline
45. अप॑चिति॒मित्यप॑ - चि॒ति॒म् । \newline
46. ए॒वास्मि॑न् नस्मिन् ने॒वै वास्मिन्न्॑ । \newline
47. अ॒स्मि॒न् द॒धा॒ति॒ द॒धा॒ त्य॒स्मि॒न् न॒स्मि॒न् द॒धा॒ति॒ । \newline
48. द॒धा॒ति॒ तस्मा॒त् तस्मा᳚द् दधाति दधाति॒ तस्मा᳚त् । \newline
49. तस्मा॑ दन॒ स्व्य॑न॒स्वी तस्मा॒त् तस्मा॑ दन॒स्वी । \newline
50. अ॒न॒स्वी च॑ चान॒ स्व्य॑न॒स्वी च॑ । \newline
51. च॒ र॒थी र॒थी च॑ च र॒थी । \newline
52. र॒थी च॑ च र॒थी र॒थी च॑ । \newline
53. चाति॑थीना॒ मति॑थीनाम् च॒ चाति॑थीनाम् । \newline
54. अति॑थीना॒ मप॑चिततमा॒ वप॑चिततमा॒ वति॑थीना॒ मति॑थीना॒ मप॑चिततमौ । \newline
55. अप॑चिततमा॒ वप॑चितिमा॒ नप॑चितिमा॒ नप॑चिततमा॒ वप॑चिततमा॒ वप॑चितिमान् । \newline
56. अप॑चिततमा॒वित्यप॑चित - त॒मौ॒ । \newline

\textbf{Ghana Paata } \newline

1. या॒हीतीति॑ याहि या॒ही त्या॑हा॒हेति॑ याहि या॒हीत्या॑ह । \newline
2. इत्या॑हा॒हे तीत्या॑ह॒ ज्योति॒र् ज्योति॑ रा॒हेतीत्या॑ह॒ ज्योतिः॑ । \newline
3. आ॒ह॒ ज्योति॒र् ज्योति॑ राहाह॒ ज्योति॑ रे॒वैव ज्योति॑ राहाह॒ ज्योति॑ रे॒व । \newline
4. ज्योति॑ रे॒वैव ज्योति॒र् ज्योति॑ रे॒वास्मि॑न् नस्मिन् ने॒व ज्योति॒र् ज्योति॑ रे॒वास्मिन्न्॑ । \newline
5. ए॒वास्मि॑न् नस्मिन् ने॒वैवास्मि॑न् दधाति दधा त्यस्मिन् ने॒वैवास्मि॑न् दधाति । \newline
6. अ॒स्मि॒न् द॒धा॒ति॒ द॒धा॒ त्य॒स्मि॒न् न॒स्मि॒न् द॒धा॒ति॒ त॒नुवा॑ त॒नुवा॑ दधा त्यस्मिन् नस्मिन् दधाति त॒नुवा᳚ । \newline
7. द॒धा॒ति॒ त॒नुवा॑ त॒नुवा॑ दधाति दधाति त॒नुवा॒ वै वै त॒नुवा॑ दधाति दधाति त॒नुवा॒ वै । \newline
8. त॒नुवा॒ वै वै त॒नुवा॑ त॒नुवा॒ वा ए॒ष ए॒ष वै त॒नुवा॑ त॒नुवा॒ वा ए॒षः । \newline
9. वा ए॒ष ए॒ष वै वा ए॒ष हि॑नस्ति हिनस्त्ये॒ष वै वा ए॒ष हि॑नस्ति । \newline
10. ए॒ष हि॑नस्ति हिनस्त्ये॒ष ए॒ष हि॑नस्ति॒ यं ॅयꣳ हि॑नस्त्ये॒ष ए॒ष हि॑नस्ति॒ यम् । \newline
11. हि॒न॒स्ति॒ यं ॅयꣳ हि॑नस्ति हिनस्ति॒ यꣳ हि॒नस्ति॑ हि॒नस्ति॒ यꣳ हि॑नस्ति हिनस्ति॒ यꣳ हि॒नस्ति॑ । \newline
12. यꣳ हि॒नस्ति॑ हि॒नस्ति॒ यं ॅयꣳ हि॒नस्ति॒ मा मा हि॒नस्ति॒ यं ॅयꣳ हि॒नस्ति॒ मा । \newline
13. हि॒नस्ति॒ मा मा हि॒नस्ति॑ हि॒नस्ति॒ मा हिꣳ॑सीर्. हिꣳसी॒र् मा हि॒नस्ति॑ हि॒नस्ति॒ मा हिꣳ॑सीः । \newline
14. मा हिꣳ॑सीर्. हिꣳसी॒र् मा मा हिꣳ॑सी स्त॒नुवा॑ त॒नुवा॑ हिꣳसी॒र् मा मा हिꣳ॑सी स्त॒नुवा᳚ । \newline
15. हिꣳ॒॒सी॒ स्त॒नुवा॑ त॒नुवा॑ हिꣳसीर्. हिꣳसी स्त॒नुवा᳚ प्र॒जाः प्र॒जा स्त॒नुवा॑ हिꣳसीर्. हिꣳसी स्त॒नुवा᳚ प्र॒जाः । \newline
16. त॒नुवा᳚ प्र॒जाः प्र॒जा स्त॒नुवा॑ त॒नुवा᳚ प्र॒जा इतीति॑ प्र॒जा स्त॒नुवा॑ त॒नुवा᳚ प्र॒जा इति॑ । \newline
17. प्र॒जा इतीति॑ प्र॒जाः प्र॒जा इत्या॑हा॒हेति॑ प्र॒जाः प्र॒जा इत्या॑ह । \newline
18. प्र॒जा इति॑ प्र - जाः । \newline
19. इत्या॑हा॒हे तीत्या॑ह प्र॒जाभ्यः॑ प्र॒जाभ्य॑ आ॒हे तीत्या॑ह प्र॒जाभ्यः॑ । \newline
20. आ॒ह॒ प्र॒जाभ्यः॑ प्र॒जाभ्य॑ आहाह प्र॒जाभ्य॑ ए॒वैव प्र॒जाभ्य॑ आहाह प्र॒जाभ्य॑ ए॒व । \newline
21. प्र॒जाभ्य॑ ए॒वैव प्र॒जाभ्यः॑ प्र॒जाभ्य॑ ए॒वैन॑ मेन मे॒व प्र॒जाभ्यः॑ प्र॒जाभ्य॑ ए॒वैन᳚म् । \newline
22. प्र॒जाभ्य॒ इति॑ प्र - जाभ्यः॑ । \newline
23. ए॒वैन॑ मेन मे॒वैवैनꣳ॑ शमयति शमय त्येन मे॒वैवैनꣳ॑ शमयति । \newline
24. ए॒नꣳ॒॒ श॒म॒य॒ति॒ श॒म॒य॒त्ये॒न॒ मे॒नꣳ॒॒ श॒म॒य॒ति॒ रक्षाꣳ॑सि॒ रक्षाꣳ॑सि शमयत्येन मेनꣳ शमयति॒ रक्षाꣳ॑सि । \newline
25. श॒म॒य॒ति॒ रक्षाꣳ॑सि॒ रक्षाꣳ॑सि शमयति शमयति॒ रक्षाꣳ॑सि॒ वै वै रक्षाꣳ॑सि शमयति शमयति॒ रक्षाꣳ॑सि॒ वै । \newline
26. रक्षाꣳ॑सि॒ वै वै रक्षाꣳ॑सि॒ रक्षाꣳ॑सि॒ वा ए॒त दे॒तद् वै रक्षाꣳ॑सि॒ रक्षाꣳ॑सि॒ वा ए॒तत् । \newline
27. वा ए॒त दे॒तद् वै वा ए॒तद् य॒ज्ञ्ं ॅय॒ज्ञ् मे॒तद् वै वा ए॒तद् य॒ज्ञ्म् । \newline
28. ए॒तद् य॒ज्ञ्ं ॅय॒ज्ञ् मे॒त दे॒तद् य॒ज्ञ्ꣳ स॑चन्ते सचन्ते य॒ज्ञ् मे॒त दे॒तद् य॒ज्ञ्ꣳ स॑चन्ते । \newline
29. य॒ज्ञ्ꣳ स॑चन्ते सचन्ते य॒ज्ञ्ं ॅय॒ज्ञ्ꣳ स॑चन्ते॒ यद् यथ् स॑चन्ते य॒ज्ञ्ं ॅय॒ज्ञ्ꣳ स॑चन्ते॒ यत् । \newline
30. स॒च॒न्ते॒ यद् यथ् स॑चन्ते सचन्ते॒ यदनो ऽनो॒ यथ् स॑चन्ते सचन्ते॒ यदनः॑ । \newline
31. यदनो ऽनो॒ यद् यदन॑ उ॒थ्सर्ज॑ त्यु॒थ्सर्ज॒ त्यनो॒ यद् यदन॑ उ॒थ्सर्ज॑ति । \newline
32. अन॑ उ॒थ्सर्ज॑ त्यु॒थ्सर्ज॒ त्यनो ऽन॑ उ॒थ्सर्ज॒ त्यक्र॑न्द॒ दक्र॑न्द दु॒थ्सर्ज॒ त्यनो ऽन॑ उ॒थ्सर्ज॒ त्यक्र॑न्दत् । \newline
33. उ॒थ्सर्ज॒ त्यक्र॑न्द॒ दक्र॑न्द दु॒थ्सर्ज॑ त्यु॒थ्सर्ज॒ त्यक्र॑न्द॒ दितीत्यक्र॑न्द दु॒थ्सर्ज॑ त्यु॒थ्सर्ज॒ त्यक्र॑न्द॒ दिति॑ । \newline
34. उ॒थ्सर्ज॒तीत्यु॑त् - सर्ज॑ति । \newline
35. अक्र॑न्द॒ दितीत्यक्र॑न्द॒ दक्र॑न्द॒दि त्यन् वन् वित्यक्र॑न्द॒ दक्र॑न्द॒ दित्यनु॑ । \newline
36. इत्यन्वन् विती त्यन्वा॑हा॒हान् वितीत्यन्वा॑ह । \newline
37. अन्वा॑हा॒हान् वन्वा॑ह॒ रक्ष॑साꣳ॒॒ रक्ष॑सा मा॒हान् वन्वा॑ह॒ रक्ष॑साम् । \newline
38. आ॒ह॒ रक्ष॑साꣳ॒॒ रक्ष॑सा माहाह॒ रक्ष॑सा॒ मप॑हत्या॒ अप॑हत्यै॒ रक्ष॑सा माहाह॒ रक्ष॑सा॒ मप॑हत्यै । \newline
39. रक्ष॑सा॒ मप॑हत्या॒ अप॑हत्यै॒ रक्ष॑साꣳ॒॒ रक्ष॑सा॒ मप॑हत्या॒ अन॒सा ऽन॒सा ऽप॑हत्यै॒ रक्ष॑साꣳ॒॒ रक्ष॑सा॒ मप॑हत्या॒ अन॑सा । \newline
40. अप॑हत्या॒ अन॒सा ऽन॒सा ऽप॑हत्या॒ अप॑हत्या॒ अन॑सा वहन्ति वह॒न् त्यन॒सा ऽप॑हत्या॒ अप॑हत्या॒ अन॑सा वहन्ति । \newline
41. अप॑हत्या॒ इत्यप॑ - ह॒त्यै॒ । \newline
42. अन॑सा वहन्ति वह॒न् त्यन॒सा ऽन॑सा वह॒न् त्यप॑चिति॒ मप॑चितिं ॅवह॒न् त्यन॒सा ऽन॑सा वह॒न् त्यप॑चितिम् । \newline
43. व॒ह॒न् त्यप॑चिति॒ मप॑चितिं ॅवहन्ति वह॒न् त्यप॑चिति मे॒वैवा प॑चितिं ॅवहन्ति वह॒न् त्यप॑चिति मे॒व । \newline
44. अप॑चिति मे॒वैवा प॑चिति॒ मप॑चिति मे॒वास्मि॑न् नस्मिन् ने॒वाप॑चिति॒ मप॑चिति मे॒वास्मिन्न्॑ । \newline
45. अप॑चिति॒मित्यप॑ - चि॒ति॒म् । \newline
46. ए॒वास्मि॑न् नस्मिन् ने॒वैवास्मि॑न् दधाति दधा त्यस्मिन् ने॒वैवास्मि॑न् दधाति । \newline
47. अ॒स्मि॒न् द॒धा॒ति॒ द॒धा॒ त्य॒स्मि॒न् न॒स्मि॒न् द॒धा॒ति॒ तस्मा॒त् तस्मा᳚द् दधा त्यस्मिन् नस्मिन् दधाति॒ तस्मा᳚त् । \newline
48. द॒धा॒ति॒ तस्मा॒त् तस्मा᳚द् दधाति दधाति॒ तस्मा॑ दन॒ स्व्य॑न॒स्वी तस्मा᳚द् दधाति दधाति॒ तस्मा॑ दन॒स्वी । \newline
49. तस्मा॑ दन॒ स्व्य॑न॒स्वी तस्मा॒त् तस्मा॑ दन॒स्वी च॑ चान॒स्वी तस्मा॒त् तस्मा॑ दन॒स्वी च॑ । \newline
50. अ॒न॒स्वी च॑ चान॒ स्व्य॑न॒स्वी च॑ र॒थी र॒थी चा॑न॒ स्व्य॑न॒स्वी च॑ र॒थी । \newline
51. च॒ र॒थी र॒थी च॑ च र॒थी च॑ च र॒थी च॑ च र॒थी च॑ । \newline
52. र॒थी च॑ च र॒थी र॒थी चाति॑थीना॒ मति॑थीनाम् च र॒थी र॒थी चाति॑थीनाम् । \newline
53. चाति॑थीना॒ मति॑थीनाम् च॒ चाति॑थीना॒ मप॑चिततमा॒ वप॑चिततमा॒ वति॑थीनाम् च॒ चाति॑थीना॒ मप॑चिततमौ । \newline
54. अति॑थीना॒ मप॑चिततमा॒ वप॑चिततमा॒ वति॑थीना॒ मति॑थीना॒ मप॑चिततमा॒ वप॑चितिमा॒ नप॑चितिमा॒ नप॑चिततमा॒ वति॑थीना॒ मति॑थीना॒ मप॑चिततमा॒ वप॑चितिमान् । \newline
55. अप॑चिततमा॒ वप॑चितिमा॒ नप॑चितिमा॒ नप॑चिततमा॒ वप॑चिततमा॒ वप॑चितिमान् भवति भव॒ त्यप॑चितिमा॒ नप॑चिततमा॒ वप॑चिततमा॒ वप॑चितिमान् भवति । \newline
56. अप॑चिततमा॒वित्यप॑चित - त॒मौ॒ । \newline
\pagebreak
\markright{ TS 5.2.2.4  \hfill https://www.vedavms.in \hfill}

\section{ TS 5.2.2.4 }

\textbf{TS 5.2.2.4 } \newline
\textbf{Samhita Paata} \newline

वप॑चितिमान् भवति॒ य ए॒वं ॅवेद॑ स॒मिधा॒ऽग्निं दु॑वस्य॒तेति॑ घृतानुषि॒क्तामव॑सिते स॒मिध॒मा द॑धाति॒ यथाऽति॑थय॒ आग॑ताय स॒र्पिष्व॑दाति॒थ्यं क्रि॒यते॑ ता॒दृगे॒व तद्-गा॑यत्रि॒या ब्रा᳚ह्म॒णस्य॑ गाय॒त्रो हि ब्रा᳚ह्म॒णस्त्रि॒ष्टुभा॑ राज॒न्य॑स्य॒ त्रैष्टु॑भो॒ हि रा॑ज॒न्यो᳚ ऽफ्सु भस्म॒ प्र वे॑शयत्य॒फ्सुयो॑नि॒र्वा अ॒ग्निः स्वामे॒वैनं॒ ॅयोनिं॑ गमयति ति॒सृभिः॒ प्रवे॑शयति त्रि॒वृद्वा - [  ] \newline

\textbf{Pada Paata} \newline

अप॑चितिमा॒नित्यप॑चिति - मा॒न् । भ॒व॒ति॒ । यः । ए॒वम् । वेद॑ । स॒मिधेति॑ सं - इधा᳚ । अ॒ग्निम् । दु॒व॒स्य॒त॒ । इति॑ । घृ॒ता॒नु॒षि॒क्तामिति॑ घृत - अ॒नु॒षि॒क्ताम् । अव॑सित॒ इत्यव॑ - सि॒ते॒ । स॒मिध॒मिति॑ सं - इध᳚म् । एति॑ । द॒धा॒ति॒ । यथा᳚ । अति॑थये । आग॑ता॒येत्या - ग॒ता॒य॒ । स॒र्पिष्व॑त् । आ॒ति॒थ्यम् । क्रि॒यते᳚ । ता॒दृक् । ए॒व । तत् । गा॒य॒त्रि॒या । ब्रा॒ह्म॒णस्य॑ । गा॒य॒त्रः । हि । ब्रा॒ह्म॒णः । त्रि॒ष्टुभा᳚ । रा॒ज॒न्य॑स्य । त्रैष्टु॑भः । हि । रा॒ज॒न्यः॑ । अ॒फ्स्वित्य॑प्-सु । भस्म॑ । प्रेति॑ । वे॒श॒य॒ति॒ । अ॒फ्सुयो॑नि॒रितिय॒फ्सु - यो॒निः॒ । वै । अ॒ग्निः । स्वाम् । ए॒व । ए॒न॒म् । योनि᳚म् । ग॒म॒य॒ति॒ । ति॒सृभि॒रिति॑ ति॒सृ-भिः॒ । प्रेति॑ । वे॒श॒य॒ति॒ । त्रि॒वृदिति॑ त्रि - वृत् । वै ।  \newline


\textbf{Krama Paata} \newline

अप॑चितिमान् भवति । अप॑चितिमा॒नित्यप॑चिति - मा॒न्॒ । भ॒व॒ति॒ यः । य ए॒वम् । ए॒वम् ॅवेद॑ । वेद॑ स॒मिधा᳚ । स॒मिधा॒ऽग्निम् । स॒मिधेति॑ सम् - इधा᳚ । अ॒ग्निम् दु॑वस्यत । दु॒व॒स्य॒तेति॑ । इति॑ घृतानुषि॒क्ताम् । घृ॒ता॒नु॒षि॒क्तामव॑सिते । घृ॒ता॒नु॒षि॒क्तामिति॑ घृत - अ॒नु॒षि॒क्ताम् । अव॑सिते स॒मिध᳚म् । अव॑सित॒ इत्यव॑ - सि॒ते॒ । स॒मिध॒मा । स॒मिध॒मिति॑ सम् - इध᳚म् । आ द॑धाति । द॒धा॒ति॒ यथा᳚ । यथाऽति॑थये । अति॑थय॒ आग॑ताय । आग॑ताय स॒र्पिष्व॑त् । आग॑ता॒येत्या - ग॒ता॒य॒ । स॒र्पिष्व॑दाति॒थ्यम् । आ॒ति॒थ्यम् क्रि॒यते᳚ । क्रि॒यते॑ ता॒दृक् । ता॒दृगे॒व । ए॒व तत् । तद् गा॑यत्रि॒या । गा॒य॒त्रि॒या ब्रा᳚ह्म॒णस्य॑ । ब्रा॒ह्म॒णस्य॑ गाय॒त्रः । गा॒य॒त्रो हि । हि ब्रा᳚ह्म॒णः । ब्रा॒ह्म॒णस्त्रि॒ष्टुभा᳚ । त्रि॒ष्टुभा॑ राज॒न्य॑स्य । रा॒ज॒न्य॑स्य॒ त्रैष्टु॑भः । त्रैष्टु॑भो॒ हि । हि रा॑ज॒न्यः॑ । रा॒ज॒न्यो᳚ऽफ्सु । अ॒फ्सु भस्म॑ । अ॒फ्स्वित्य॑प् - सु । भस्म॒ प्र । प्र वे॑शयति । वे॒श॒य॒त्य॒फ्सुयो॑निः । अ॒फ्सुयो॑नि॒र् वै । अ॒फ्सुयो॑नि॒रित्य॒फ्सु - यो॒निः॒ । वा अ॒ग्निः । अ॒ग्निः स्वाम् । स्वामे॒व । ए॒वैन᳚म् । ए॒न॒म् ॅयोनि᳚म् । योनि॑म् गमयति । ग॒म॒य॒ति॒ ति॒सृभिः॑ । ति॒सृभिः॒ प्र । ति॒सृभि॒रिति॑ ति॒सृ - भिः॒ । प्र वे॑शयति । वे॒श॒य॒ति॒ त्रि॒वृत् । त्रि॒वृद् वै । त्रि॒वृदिति॑ त्रि - वृत् । वा अ॒ग्निः \newline

\textbf{Jatai Paata} \newline

1. अप॑चितिमान् भवति भव॒ त्यप॑चितिमा॒ नप॑चितिमान् भवति । \newline
2. अप॑चितिमा॒नित्यप॑चिति - मा॒न् । \newline
3. भ॒व॒ति॒ यो यो भ॑वति भवति॒ यः । \newline
4. य ए॒व मे॒वं ॅयो य ए॒वम् । \newline
5. ए॒वं ॅवेद॒ वेदै॒व मे॒वं ॅवेद॑ । \newline
6. वेद॑ स॒मिधा॑ स॒मिधा॒ वेद॒ वेद॑ स॒मिधा᳚ । \newline
7. स॒मिधा॒ ऽग्नि म॒ग्निꣳ स॒मिधा॑ स॒मिधा॒ ऽग्निम् । \newline
8. स॒मिधेति॑ सं - इधा᳚ । \newline
9. अ॒ग्निम् दु॑वस्यत दुवस्य ता॒ग्नि म॒ग्निम् दु॑वस्यत । \newline
10. दु॒व॒स्य॒ते तीति॑ दुवस्यत दुवस्य॒तेति॑ । \newline
11. इति॑ घृतानुषि॒क्ताम् घृ॑तानुषि॒क्ता मितीति॑ घृतानुषि॒क्ताम् । \newline
12. घृ॒ता॒नु॒षि॒क्ता मव॑सि॒ते ऽव॑सिते घृतानुषि॒क्ताम् घृ॑तानुषि॒क्ता मव॑सिते । \newline
13. घृ॒ता॒नु॒षि॒क्तामिति॑ घृत - अ॒नु॒षि॒क्ताम् । \newline
14. अव॑सिते स॒मिधꣳ॑ स॒मिध॒ मव॑सि॒ते ऽव॑सिते स॒मिध᳚म् । \newline
15. अव॑सित॒ इत्यव॑ - सि॒ते॒ । \newline
16. स॒मिध॒ मा स॒मिधꣳ॑ स॒मिध॒ मा । \newline
17. स॒मिध॒मिति॑ सं - इध᳚म् । \newline
18. आ द॑धाति दधा॒त्या द॑धाति । \newline
19. द॒धा॒ति॒ यथा॒ यथा॑ दधाति दधाति॒ यथा᳚ । \newline
20. यथा ऽति॑थ॒ये ऽति॑थये॒ यथा॒ यथा ऽति॑थये । \newline
21. अति॑थय॒ आग॑ता॒या ग॑ता॒या ति॑थ॒ये ऽति॑थय॒ आग॑ताय । \newline
22. आग॑ताय स॒र्पिष्व॑थ् स॒र्पिष्व॒दा ग॑ता॒या ग॑ताय स॒र्पिष्व॑त् । \newline
23. आग॑ता॒येत्या - ग॒ता॒य॒ । \newline
24. स॒र्पिष्व॑ दाति॒थ्य मा॑ति॒थ्यꣳ स॒र्पिष्व॑थ् स॒र्पिष्व॑ दाति॒थ्यम् । \newline
25. आ॒ति॒थ्यम् क्रि॒यते᳚ क्रि॒यत॑ आति॒थ्य मा॑ति॒थ्यम् क्रि॒यते᳚ । \newline
26. क्रि॒यते॑ ता॒दृक् ता॒दृक् क्रि॒यते᳚ क्रि॒यते॑ ता॒दृक् । \newline
27. ता॒दृ गे॒वैव ता॒दृक् ता॒दृग् ए॒व । \newline
28. ए॒व तत् तदे॒ वैव तत् । \newline
29. तद् गा॑यत्रि॒या गा॑यत्रि॒या तत् तद् गा॑यत्रि॒या । \newline
30. गा॒य॒त्रि॒या ब्रा᳚ह्म॒णस्य॑ ब्राह्म॒णस्य॑ गायत्रि॒या गा॑यत्रि॒या ब्रा᳚ह्म॒णस्य॑ । \newline
31. ब्रा॒ह्म॒णस्य॑ गाय॒त्रो गा॑य॒त्रो ब्रा᳚ह्म॒णस्य॑ ब्राह्म॒णस्य॑ गाय॒त्रः । \newline
32. गा॒य॒त्रो हि हि गा॑य॒त्रो गा॑य॒त्रो हि । \newline
33. हि ब्रा᳚ह्म॒णो ब्रा᳚ह्म॒णो हि हि ब्रा᳚ह्म॒णः । \newline
34. ब्रा॒ह्म॒ण स्त्रि॒ष्टुभा᳚ त्रि॒ष्टुभा᳚ ब्राह्म॒णो ब्रा᳚ह्म॒ण स्त्रि॒ष्टुभा᳚ । \newline
35. त्रि॒ष्टुभा॑ राज॒न्य॑स्य राज॒न्य॑स्य त्रि॒ष्टुभा᳚ त्रि॒ष्टुभा॑ राज॒न्य॑स्य । \newline
36. रा॒ज॒न्य॑स्य॒ त्रैष्टु॑भ॒ स्त्रैष्टु॑भो राज॒न्य॑स्य राज॒न्य॑स्य॒ त्रैष्टु॑भः । \newline
37. त्रैष्टु॑भो॒ हि हि त्रैष्टु॑भ॒ स्त्रैष्टु॑भो॒ हि । \newline
38. हि रा॑ज॒न्यो॑ राज॒न्यो॑ हि हि रा॑ज॒न्यः॑ । \newline
39. रा॒ज॒न्यो᳚(1॒) ऽफ्स्व॑फ्सु रा॑ज॒न्यो॑ राज॒न्यो᳚ ऽफ्सु । \newline
40. अ॒फ्सु भस्म॒ भस्मा॒ फ्स्व॑फ्सु भस्म॑ । \newline
41. अ॒फ्स्वित्य॑प् - सु । \newline
42. भस्म॒ प्र प्र भस्म॒ भस्म॒ प्र । \newline
43. प्र वे॑शयति वेशयति॒ प्र प्र वे॑शयति । \newline
44. वे॒श॒य॒ त्य॒फ्सुयो॑नि र॒फ्सुयो॑निर् वेशयति वेशय त्य॒फ्सुयो॑निः । \newline
45. अ॒फ्सुयो॑नि॒र् वै वा अ॒फ्सुयो॑नि र॒फ्सुयो॑नि॒र् वै । \newline
46. अ॒फ्सुयो॑नि॒रित्य॒फ्सु - यो॒निः॒ । \newline
47. वा अ॒ग्नि र॒ग्निर् वै वा अ॒ग्निः । \newline
48. अ॒ग्निः स्वाꣳ स्वा म॒ग्नि र॒ग्निः स्वाम् । \newline
49. स्वा मे॒वैव स्वाꣳ स्वा मे॒व । \newline
50. ए॒वैन॑ मेन मे॒वै वैन᳚म् । \newline
51. ए॒नं॒ ॅयोनिं॒ ॅयोनि॑ मेन मेनं॒ ॅयोनि᳚म् । \newline
52. योनि॑म् गमयति गमयति॒ योनिं॒ ॅयोनि॑म् गमयति । \newline
53. ग॒म॒य॒ति॒ ति॒सृभि॑ स्ति॒सृभि॑र् गमयति गमयति ति॒सृभिः॑ । \newline
54. ति॒सृभिः॒ प्र प्र ति॒सृभि॑ स्ति॒सृभिः॒ प्र । \newline
55. ति॒सृभि॒रिति॑ ति॒सृ - भिः॒ । \newline
56. प्र वे॑शयति वेशयति॒ प्र प्र वे॑शयति । \newline
57. वे॒श॒य॒ति॒ त्रि॒वृत् त्रि॒वृद् वे॑शयति वेशयति त्रि॒वृत् । \newline
58. त्रि॒वृद् वै वै त्रि॒वृत् त्रि॒वृद् वै । \newline
59. त्रि॒वृदिति॑ त्रि - वृत् । \newline
60. वा अ॒ग्नि र॒ग्निर् वै वा अ॒ग्निः । \newline

\textbf{Ghana Paata } \newline

1. अप॑चितिमान् भवति भव॒ त्यप॑चितिमा॒ नप॑चितिमान् भवति॒ यो यो भ॑व॒ त्यप॑चितिमा॒ नप॑चितिमान् भवति॒ यः । \newline
2. अप॑चितिमा॒नित्यप॑चिति - मा॒न् । \newline
3. भ॒व॒ति॒ यो यो भ॑वति भवति॒ य ए॒व मे॒वं ॅयो भ॑वति भवति॒ य ए॒वम् । \newline
4. य ए॒व मे॒वं ॅयो य ए॒वं ॅवेद॒ वेदै॒वं ॅयो य ए॒वं ॅवेद॑ । \newline
5. ए॒वं ॅवेद॒ वेदै॒व मे॒वं ॅवेद॑ स॒मिधा॑ स॒मिधा॒ वेदै॒व मे॒वं ॅवेद॑ स॒मिधा᳚ । \newline
6. वेद॑ स॒मिधा॑ स॒मिधा॒ वेद॒ वेद॑ स॒मिधा॒ ऽग्नि म॒ग्निꣳ स॒मिधा॒ वेद॒ वेद॑ स॒मिधा॒ ऽग्निम् । \newline
7. स॒मिधा॒ ऽग्नि म॒ग्निꣳ स॒मिधा॑ स॒मिधा॒ ऽग्निम् दु॑वस्यत दुवस्यता॒ग्निꣳ स॒मिधा॑ स॒मिधा॒ ऽग्निम् दु॑वस्यत । \newline
8. स॒मिधेति॑ सं - इधा᳚ । \newline
9. अ॒ग्निम् दु॑वस्यत दुवस्यता॒ग्नि म॒ग्निम् दु॑वस्य॒ते तीति॑ दुवस्यता॒ग्नि म॒ग्निम् दु॑वस्य॒तेति॑ । \newline
10. दु॒व॒स्य॒ते तीति॑ दुवस्यत दुवस्य॒तेति॑ घृतानुषि॒क्ताम् घृ॑तानुषि॒क्ता मिति॑ दुवस्यत दुवस्य॒तेति॑ घृतानुषि॒क्ताम् । \newline
11. इति॑ घृतानुषि॒क्ताम् घृ॑तानुषि॒क्ता मितीति॑ घृतानुषि॒क्ता मव॑सि॒ते ऽव॑सिते घृतानुषि॒क्ता मितीति॑ घृतानुषि॒क्ता मव॑सिते । \newline
12. घृ॒ता॒नु॒षि॒क्ता मव॑सि॒ते ऽव॑सिते घृतानुषि॒क्ताम् घृ॑तानुषि॒क्ता मव॑सिते स॒मिधꣳ॑ स॒मिध॒ मव॑सिते घृतानुषि॒क्ताम् घृ॑तानुषि॒क्ता मव॑सिते स॒मिध᳚म् । \newline
13. घृ॒ता॒नु॒षि॒क्तामिति॑ घृत - अ॒नु॒षि॒क्ताम् । \newline
14. अव॑सिते स॒मिधꣳ॑ स॒मिध॒ मव॑सि॒ते ऽव॑सिते स॒मिध॒ मा स॒मिध॒ मव॑सि॒ते ऽव॑सिते स॒मिध॒ मा । \newline
15. अव॑सित॒ इत्यव॑ - सि॒ते॒ । \newline
16. स॒मिध॒ मा स॒मिधꣳ॑ स॒मिध॒ मा द॑धाति दधा॒त्या स॒मिधꣳ॑ स॒मिध॒ मा द॑धाति । \newline
17. स॒मिध॒मिति॑ सं - इध᳚म् । \newline
18. आ द॑धाति दधा॒त्या द॑धाति॒ यथा॒ यथा॑ दधा॒त्या द॑धाति॒ यथा᳚ । \newline
19. द॒धा॒ति॒ यथा॒ यथा॑ दधाति दधाति॒ यथा ऽति॑थ॒ये ऽति॑थये॒ यथा॑ दधाति दधाति॒ यथा ऽति॑थये । \newline
20. यथा ऽति॑थ॒ये ऽति॑थये॒ यथा॒ यथा ऽति॑थय॒ आग॑ता॒या ग॑ता॒या ति॑थये॒ यथा॒ यथा ऽति॑थय॒ आग॑ताय । \newline
21. अति॑थय॒ आग॑ता॒या ग॑ता॒या ति॑थ॒ये ऽति॑थय॒ आग॑ताय स॒र्पिष्व॑थ् स॒र्पिष्व॒ दाग॑ता॒या ति॑थ॒ये ऽति॑थय॒ आग॑ताय स॒र्पिष्व॑त् । \newline
22. आग॑ताय स॒र्पिष्व॑थ् स॒र्पिष्व॒ दाग॑ता॒या ग॑ताय स॒र्पिष्व॑ दाति॒थ्य मा॑ति॒थ्यꣳ स॒र्पिष्व॒ दाग॑ता॒या ग॑ताय स॒र्पिष्व॑ दाति॒थ्यम् । \newline
23. आग॑ता॒येत्या - ग॒ता॒य॒ । \newline
24. स॒र्पिष्व॑ दाति॒थ्य मा॑ति॒थ्यꣳ स॒र्पिष्व॑थ् स॒र्पिष्व॑ दाति॒थ्यम् क्रि॒यते᳚ क्रि॒यत॑ आति॒थ्यꣳ स॒र्पिष्व॑थ् स॒र्पिष्व॑ दाति॒थ्यम् क्रि॒यते᳚ । \newline
25. आ॒ति॒थ्यम् क्रि॒यते᳚ क्रि॒यत॑ आति॒थ्य मा॑ति॒थ्यम् क्रि॒यते॑ ता॒दृक् ता॒दृक् क्रि॒यत॑ आति॒थ्य मा॑ति॒थ्यम् क्रि॒यते॑ ता॒दृक् । \newline
26. क्रि॒यते॑ ता॒दृक् ता॒दृक् क्रि॒यते᳚ क्रि॒यते॑ ता॒दृ गे॒वैव ता॒दृक् क्रि॒यते᳚ क्रि॒यते॑ ता॒दृगे॒व । \newline
27. ता॒दृ गे॒वैव ता॒दृक् ता॒दृगे॒व तत् तदे॒व ता॒दृक् ता॒दृगे॒व तत् । \newline
28. ए॒व तत् तदे॒वैव तद् गा॑यत्रि॒या गा॑यत्रि॒या तदे॒वैव तद् गा॑यत्रि॒या । \newline
29. तद् गा॑यत्रि॒या गा॑यत्रि॒या तत् तद् गा॑यत्रि॒या ब्रा᳚ह्म॒णस्य॑ ब्राह्म॒णस्य॑ गायत्रि॒या तत् तद् गा॑यत्रि॒या ब्रा᳚ह्म॒णस्य॑ । \newline
30. गा॒य॒त्रि॒या ब्रा᳚ह्म॒णस्य॑ ब्राह्म॒णस्य॑ गायत्रि॒या गा॑यत्रि॒या ब्रा᳚ह्म॒णस्य॑ गाय॒त्रो गा॑य॒त्रो ब्रा᳚ह्म॒णस्य॑ गायत्रि॒या गा॑यत्रि॒या ब्रा᳚ह्म॒णस्य॑ गाय॒त्रः । \newline
31. ब्रा॒ह्म॒णस्य॑ गाय॒त्रो गा॑य॒त्रो ब्रा᳚ह्म॒णस्य॑ ब्राह्म॒णस्य॑ गाय॒त्रो हि हि गा॑य॒त्रो ब्रा᳚ह्म॒णस्य॑ ब्राह्म॒णस्य॑ गाय॒त्रो हि । \newline
32. गा॒य॒त्रो हि हि गा॑य॒त्रो गा॑य॒त्रो हि ब्रा᳚ह्म॒णो ब्रा᳚ह्म॒णो हि गा॑य॒त्रो गा॑य॒त्रो हि ब्रा᳚ह्म॒णः । \newline
33. हि ब्रा᳚ह्म॒णो ब्रा᳚ह्म॒णो हि हि ब्रा᳚ह्म॒ण स्त्रि॒ष्टुभा᳚ त्रि॒ष्टुभा᳚ ब्राह्म॒णो हि हि ब्रा᳚ह्म॒ण स्त्रि॒ष्टुभा᳚ । \newline
34. ब्रा॒ह्म॒ण स्त्रि॒ष्टुभा᳚ त्रि॒ष्टुभा᳚ ब्राह्म॒णो ब्रा᳚ह्म॒ण स्त्रि॒ष्टुभा॑ राज॒न्य॑स्य राज॒न्य॑स्य त्रि॒ष्टुभा᳚ ब्राह्म॒णो ब्रा᳚ह्म॒ण स्त्रि॒ष्टुभा॑ राज॒न्य॑स्य । \newline
35. त्रि॒ष्टुभा॑ राज॒न्य॑स्य राज॒न्य॑स्य त्रि॒ष्टुभा᳚ त्रि॒ष्टुभा॑ राज॒न्य॑स्य॒ त्रैष्टु॑भ॒ स्त्रैष्टु॑भो राज॒न्य॑स्य त्रि॒ष्टुभा᳚ त्रि॒ष्टुभा॑ राज॒न्य॑स्य॒ त्रैष्टु॑भः । \newline
36. रा॒ज॒न्य॑स्य॒ त्रैष्टु॑भ॒ स्त्रैष्टु॑भो राज॒न्य॑स्य राज॒न्य॑स्य॒ त्रैष्टु॑भो॒ हि हि त्रैष्टु॑भो राज॒न्य॑स्य राज॒न्य॑स्य॒ त्रैष्टु॑भो॒ हि । \newline
37. त्रैष्टु॑भो॒ हि हि त्रैष्टु॑भ॒ स्त्रैष्टु॑भो॒ हि रा॑ज॒न्यो॑ राज॒न्यो॑ हि त्रैष्टु॑भ॒ स्त्रैष्टु॑भो॒ हि रा॑ज॒न्यः॑ । \newline
38. हि रा॑ज॒न्यो॑ राज॒न्यो॑ हि हि रा॑ज॒न्यो᳚(1॒) ऽफ्स्व॑फ्सु रा॑ज॒न्यो॑ हि हि रा॑ज॒न्यो᳚ ऽफ्सु । \newline
39. रा॒ज॒न्यो᳚(1॒) ऽफ्स्व॑फ्सु रा॑ज॒न्यो॑ राज॒न्यो᳚ ऽफ्सु भस्म॒ भस्मा॒फ्सु रा॑ज॒न्यो॑ राज॒न्यो᳚ ऽफ्सु भस्म॑ । \newline
40. अ॒फ्सु भस्म॒ भस्मा॒ फ्स्व॑फ्सु भस्म॒ प्र प्र भस्मा॒ फ्स्व॑फ्सु भस्म॒ प्र । \newline
41. अ॒फ्स्वित्य॑प् - सु । \newline
42. भस्म॒ प्र प्र भस्म॒ भस्म॒ प्र वे॑शयति वेशयति॒ प्र भस्म॒ भस्म॒ प्र वे॑शयति । \newline
43. प्र वे॑शयति वेशयति॒ प्र प्र वे॑शय त्य॒फ्सुयो॑नि र॒फ्सुयो॑निर् वेशयति॒ प्र प्र वे॑शय त्य॒फ्सुयो॑निः । \newline
44. वे॒श॒य॒ त्य॒फ्सुयो॑नि र॒फ्सुयो॑निर् वेशयति वेशय त्य॒फ्सुयो॑नि॒र् वै वा अ॒फ्सुयो॑निर् वेशयति वेशय त्य॒फ्सुयो॑नि॒र् वै । \newline
45. अ॒फ्सुयो॑नि॒र् वै वा अ॒फ्सुयो॑नि र॒फ्सुयो॑नि॒र् वा अ॒ग्नि र॒ग्निर् वा अ॒फ्सुयो॑नि र॒फ्सुयो॑नि॒र् वा अ॒ग्निः । \newline
46. अ॒फ्सुयो॑नि॒रित्य॒फ्सु - यो॒निः॒ । \newline
47. वा अ॒ग्नि र॒ग्निर् वै वा अ॒ग्निः स्वाꣳ स्वा म॒ग्निर् वै वा अ॒ग्निः स्वाम् । \newline
48. अ॒ग्निः स्वाꣳ स्वा म॒ग्नि र॒ग्निः स्वा मे॒वैव स्वा म॒ग्नि र॒ग्निः स्वा मे॒व । \newline
49. स्वा मे॒वैव स्वाꣳ स्वा मे॒वैन॑ मेन मे॒व स्वाꣳ स्वा मे॒वैन᳚म् । \newline
50. ए॒वैन॑ मेन मे॒वैवैनं॒ ॅयोनिं॒ ॅयोनि॑ मेन मे॒वैवैनं॒ ॅयोनि᳚म् । \newline
51. ए॒नं॒ ॅयोनिं॒ ॅयोनि॑ मेन मेनं॒ ॅयोनि॑म् गमयति गमयति॒ योनि॑ मेन मेनं॒ ॅयोनि॑म् गमयति । \newline
52. योनि॑म् गमयति गमयति॒ योनिं॒ ॅयोनि॑म् गमयति ति॒सृभि॑ स्ति॒सृभि॑र् गमयति॒ योनिं॒ ॅयोनि॑म् गमयति ति॒सृभिः॑ । \newline
53. ग॒म॒य॒ति॒ ति॒सृभि॑ स्ति॒सृभि॑र् गमयति गमयति ति॒सृभिः॒ प्र प्र ति॒सृभि॑र् गमयति गमयति ति॒सृभिः॒ प्र । \newline
54. ति॒सृभिः॒ प्र प्र ति॒सृभि॑ स्ति॒सृभिः॒ प्र वे॑शयति वेशयति॒ प्र ति॒सृभि॑ स्ति॒सृभिः॒ प्र वे॑शयति । \newline
55. ति॒सृभि॒रिति॑ ति॒सृ - भिः॒ । \newline
56. प्र वे॑शयति वेशयति॒ प्र प्र वे॑शयति त्रि॒वृत् त्रि॒वृद् वे॑शयति॒ प्र प्र वे॑शयति त्रि॒वृत् । \newline
57. वे॒श॒य॒ति॒ त्रि॒वृत् त्रि॒वृद् वे॑शयति वेशयति त्रि॒वृद् वै वै त्रि॒वृद् वे॑शयति वेशयति त्रि॒वृद् वै । \newline
58. त्रि॒वृद् वै वै त्रि॒वृत् त्रि॒वृद् वा अ॒ग्नि र॒ग्निर् वै त्रि॒वृत् त्रि॒वृद् वा अ॒ग्निः । \newline
59. त्रि॒वृदिति॑ त्रि - वृत् । \newline
60. वा अ॒ग्नि र॒ग्निर् वै वा अ॒ग्निर् यावा॒न्॒. यावा॑ न॒ग्निर् वै वा अ॒ग्निर् यावान्॑ । \newline
\pagebreak
\markright{ TS 5.2.2.5  \hfill https://www.vedavms.in \hfill}

\section{ TS 5.2.2.5 }

\textbf{TS 5.2.2.5 } \newline
\textbf{Samhita Paata} \newline

अ॒ग्नि-र्यावा॑ने॒वाऽग्निस्तं प्र॑ति॒ष्ठां ग॑मयति॒ परा॒ वा ए॒षो᳚ऽग्निं ॅव॑पति॒ यो᳚ऽफ्सु भस्म॑ प्रवे॒शय॑ति॒ ज्योति॑ष्मतीभ्या॒-मव॑ दधाति॒ ज्योति॑रे॒वाऽस्मि॑न् दधाति॒ द्वाभ्यां॒ प्रति॑ष्ठित्यै॒ परा॒ वा ए॒ष प्र॒जां प॒शून् व॑पति॒ यो᳚ऽफ्सु भस्म॑ प्रवे॒शय॑ति॒ पुन॑रू॒र्जा स॒ह र॒य्येति॒ पुन॑रु॒दैति॑ प्र॒जामे॒व प॒शूना॒त्मन् ध॑त्ते॒ पुन॑स्त्वाऽऽदि॒त्या - [  ] \newline

\textbf{Pada Paata} \newline

अ॒ग्निः । यावान्॑ । ए॒व । अ॒ग्निः । तम् । प्र॒ति॒ष्ठामिति॑ प्रति - स्थाम् । ग॒म॒य॒ति॒ । परेति॑ । वै । ए॒षः । अ॒ग्निम् । व॒प॒ति॒ । यः । अ॒फ्स्वित्य॑प् - सु । भस्म॑ । प्र॒वे॒शय॒तीति॑ प्र - वे॒शय॑ति । ज्योति॑ष्मतीभ्याम् । अवेति॑ । द॒धा॒ति॒ । ज्योतिः॑ । ए॒व । अ॒स्मि॒न्न् । द॒धा॒ति॒ । द्वाभ्या᳚म् । प्रति॑ष्ठित्या॒ इति॒ प्रति॑ - स्थि॒त्यै॒ । परेति॑ । वै । ए॒षः । प्र॒जामिति॑ प्र - जाम् । प॒शून् । व॒प॒ति॒ । यः । अ॒फ्स्वित्य॑प् - सु । भस्म॑ । प्र॒वे॒शय॒तीति॑ प्र - वे॒शय॑ति । पुनः॑ । ऊ॒र्जा । स॒ह । र॒य्या । इति॑ । पुनः॑ । उ॒दैतीत्यु॑त् - ऐति॑ । प्र॒जामिति॑ प्र - जाम् । ए॒व । प॒शून् । आ॒त्मन्न् । ध॒त्ते॒ । पुनः॑ । त्वा॒ । आ॒दि॒त्याः ।  \newline


\textbf{Krama Paata} \newline

अ॒ग्निर् यावान्॑ । यावा॑ने॒व । ए॒वाग्निः । अ॒ग्निस्तम् । तम् प्र॑ति॒ष्ठाम् । प्र॒ति॒ष्ठाम् ग॑मयति । प्र॒ति॒ष्ठामिति॑ प्रति - स्थाम् । ग॒म॒य॒ति॒ परा᳚ । परा॒ वै । वा ए॒षः । ए॒षो᳚ऽग्निम् । अ॒ग्निम् ॅव॑पति । व॒प॒ति॒ यः । यो᳚ऽफ्सु । अ॒फ्सु भस्म॑ । अ॒फ्स्वित्य॑प् - सु । भस्म॑ प्रवे॒शय॑ति । प्र॒वे॒शय॑ति॒ ज्योति॑ष्मतीभ्याम् । प्र॒वे॒शय॒तीति॑ प्र - वे॒शय॑ति । ज्योति॑ष्मतीभ्या॒मव॑ । अव॑ दधाति । द॒धा॒ति॒ ज्योतिः॑ । ज्योति॑रे॒व । ए॒वास्मिन्न्॑ । अ॒स्मि॒न् द॒धा॒ति॒ । द॒धा॒ति॒ द्वाभ्या᳚म् । द्वाभ्या॒म् प्रति॑ष्ठित्यै । प्रति॑ष्ठित्यै॒ परा᳚ । प्रति॑ष्ठित्या॒ इति॒ प्रति॑ - स्थि॒त्यै॒ । परा॒ वै । वा ए॒षः । ए॒ष प्र॒जाम् । प्र॒जाम् प॒शून् । प्र॒जामिति॑ प्र - जाम् । प॒शून्. व॑पति । व॒प॒ति॒ यः । यो᳚ऽफ्सु । अ॒फ्सु भस्म॑ । अ॒फ्स्वित्य॑प् - सु । भस्म॑ प्रवे॒शय॑ति । प्र॒वे॒शय॑ति॒ पुनः॑ । प्र॒वे॒शय॒तीति॑ प्र - वे॒शय॑ति । पुन॑रू॒र्जा । ऊ॒र्जा स॒ह । स॒ह र॒य्या । र॒य्येति॑ । इति॒ पुनः॑ । पुन॑रु॒दैति॑ । उ॒दैति॑ प्र॒जाम् । उ॒दैतीत्यु॑त् - ऐति॑ । प्र॒जामे॒व । प्र॒जामिति॑ प्र - जाम् । ए॒व प॒शून् । प॒शूना॒त्मन्न् । आ॒त्मन् ध॑त्ते । ध॒त्ते॒ पुनः॑ । पुन॑स्त्वा । त्वा॒ऽऽदि॒त्याः ( ) । आ॒दि॒त्या रु॒द्राः \newline

\textbf{Jatai Paata} \newline

1. अ॒ग्निर् यावा॒न्॒. यावा॑ न॒ग्नि र॒ग्निर् यावान्॑ । \newline
2. यावा॑ ने॒वैव यावा॒न्॒. यावा॑ ने॒व । \newline
3. ए॒वाग्नि र॒ग्नि रे॒वै वाग्निः । \newline
4. अ॒ग्नि स्तम् त म॒ग्नि र॒ग्नि स्तम् । \newline
5. तम् प्र॑ति॒ष्ठाम् प्र॑ति॒ष्ठाम् तम् तम् प्र॑ति॒ष्ठाम् । \newline
6. प्र॒ति॒ष्ठाम् ग॑मयति गमयति प्रति॒ष्ठाम् प्र॑ति॒ष्ठाम् ग॑मयति । \newline
7. प्र॒ति॒ष्ठामिति॑ प्रति - स्थाम् । \newline
8. ग॒म॒य॒ति॒ परा॒ परा॑ गमयति गमयति॒ परा᳚ । \newline
9. परा॒ वै वै परा॒ परा॒ वै । \newline
10. वा ए॒ष ए॒ष वै वा ए॒षः । \newline
11. ए॒षो᳚ ऽग्नि म॒ग्नि मे॒ष ए॒षो᳚ ऽग्निम् । \newline
12. अ॒ग्निं ॅव॑पति वप त्य॒ग्नि म॒ग्निं ॅव॑पति । \newline
13. व॒प॒ति॒ यो यो व॑पति वपति॒ यः । \newline
14. यो᳚(1॒) ऽफ्स्व॑फ्सु यो यो᳚ ऽफ्सु । \newline
15. अ॒फ्सु भस्म॒ भस्मा॒ फ्स्व॑फ्सु भस्म॑ । \newline
16. अ॒फ्स्वित्य॑प् - सु । \newline
17. भस्म॑ प्रवे॒शय॑ति प्रवे॒शय॑ति॒ भस्म॒ भस्म॑ प्रवे॒शय॑ति । \newline
18. प्र॒वे॒शय॑ति॒ ज्योति॑ष्मतीभ्या॒म् ज्योति॑ष्मतीभ्याम् प्रवे॒शय॑ति प्रवे॒शय॑ति॒ ज्योति॑ष्मतीभ्याम् । \newline
19. प्र॒वे॒शय॒तीति॑ प्र - वे॒शय॑ति । \newline
20. ज्योति॑ष्मतीभ्या॒ मवाव॒ ज्योति॑ष्मतीभ्या॒म् ज्योति॑ष्मतीभ्या॒ मव॑ । \newline
21. अव॑ दधाति दधा॒ त्यवाव॑ दधाति । \newline
22. द॒धा॒ति॒ ज्योति॒र् ज्योति॑र् दधाति दधाति॒ ज्योतिः॑ । \newline
23. ज्योति॑ रे॒वैव ज्योति॒र् ज्योति॑ रे॒व । \newline
24. ए॒वास्मि॑न् नस्मिन् ने॒वै वास्मिन्न्॑ । \newline
25. अ॒स्मि॒न् द॒धा॒ति॒ द॒धा॒ त्य॒स्मि॒न् न॒स्मि॒न् द॒धा॒ति॒ । \newline
26. द॒धा॒ति॒ द्वाभ्या॒म् द्वाभ्या᳚म् दधाति दधाति॒ द्वाभ्या᳚म् । \newline
27. द्वाभ्या॒म् प्रति॑ष्ठित्यै॒ प्रति॑ष्ठित्यै॒ द्वाभ्या॒म् द्वाभ्या॒म् प्रति॑ष्ठित्यै । \newline
28. प्रति॑ष्ठित्यै॒ परा॒ परा॒ प्रति॑ष्ठित्यै॒ प्रति॑ष्ठित्यै॒ परा᳚ । \newline
29. प्रति॑ष्ठित्या॒ इति॒ प्रति॑ - स्थि॒त्यै॒ । \newline
30. परा॒ वै वै परा॒ परा॒ वै । \newline
31. वा ए॒ष ए॒ष वै वा ए॒षः । \newline
32. ए॒ष प्र॒जाम् प्र॒जा मे॒ष ए॒ष प्र॒जाम् । \newline
33. प्र॒जाम् प॒शून् प॒शून् प्र॒जाम् प्र॒जाम् प॒शून् । \newline
34. प्र॒जामिति॑ प्र - जाम् । \newline
35. प॒शून्. व॑पति वपति प॒शून् प॒शून्. व॑पति । \newline
36. व॒प॒ति॒ यो यो व॑पति वपति॒ यः । \newline
37. यो᳚(1॒) ऽफ्स्व॑फ्सु यो यो᳚ ऽफ्सु । \newline
38. अ॒फ्सु भस्म॒ भस्मा॒ फ्स्व॑फ्सु भस्म॑ । \newline
39. अ॒फ्स्वित्य॑प् - सु । \newline
40. भस्म॑ प्रवे॒शय॑ति प्रवे॒शय॑ति॒ भस्म॒ भस्म॑ प्रवे॒शय॑ति । \newline
41. प्र॒वे॒शय॑ति॒ पुनः॒ पुनः॑ प्रवे॒शय॑ति प्रवे॒शय॑ति॒ पुनः॑ । \newline
42. प्र॒वे॒शय॒तीति॑ प्र - वे॒शय॑ति । \newline
43. पुन॑ रू॒र्जोर्जा पुनः॒ पुन॑ रू॒र्जा । \newline
44. ऊ॒र्जा स॒ह स॒होर् जोर्जा स॒ह । \newline
45. स॒ह र॒य्या र॒य्या स॒ह स॒ह र॒य्या । \newline
46. र॒य्येतीति॑ र॒य्या र॒य्येति॑ । \newline
47. इति॒ पुनः॒ पुन॒ रितीति॒ पुनः॑ । \newline
48. पुन॑ रु॒दै त्यु॒दैति॒ पुनः॒ पुन॑ रु॒दैति॑ । \newline
49. उ॒दैति॑ प्र॒जाम् प्र॒जा मु॒दै त्यु॒दैति॑ प्र॒जाम् । \newline
50. उ॒दैतीत्यु॑त् - ऐति॑ । \newline
51. प्र॒जा मे॒वैव प्र॒जाम् प्र॒जा मे॒व । \newline
52. प्र॒जामिति॑ प्र - जाम् । \newline
53. ए॒व प॒शून् प॒शू ने॒वैव प॒शून् । \newline
54. प॒शू ना॒त्मन् ना॒त्मन् प॒शून् प॒शू ना॒त्मन्न् । \newline
55. आ॒त्मन् ध॑त्ते धत्त आ॒त्मन् ना॒त्मन् ध॑त्ते । \newline
56. ध॒त्ते॒ पुनः॒ पुन॑र् धत्ते धत्ते॒ पुनः॑ । \newline
57. पुन॑ स्त्वा त्वा॒ पुनः॒ पुन॑ स्त्वा । \newline
58. त्वा॒ ऽऽदि॒त्या आ॑दि॒त्या स्त्वा᳚ त्वा ऽऽदि॒त्याः । \newline
59. आ॒दि॒त्या रु॒द्रा रु॒द्रा आ॑दि॒त्या आ॑दि॒त्या रु॒द्राः । \newline

\textbf{Ghana Paata } \newline

1. अ॒ग्निर् यावा॒न्॒. यावा॑ न॒ग्नि र॒ग्निर् यावा॑ ने॒वैव यावा॑ न॒ग्नि र॒ग्निर् यावा॑ ने॒व । \newline
2. यावा॑ ने॒वैव यावा॒न्॒. यावा॑ ने॒वाग्नि र॒ग्नि रे॒व यावा॒न्॒. यावा॑ ने॒वाग्निः । \newline
3. ए॒वाग्नि र॒ग्नि रे॒वैवाग्नि स्तम् त म॒ग्नि रे॒वैवाग्नि स्तम् । \newline
4. अ॒ग्नि स्तम् त म॒ग्नि र॒ग्नि स्तम् प्र॑ति॒ष्ठाम् प्र॑ति॒ष्ठाम् त म॒ग्नि र॒ग्नि स्तम् प्र॑ति॒ष्ठाम् । \newline
5. तम् प्र॑ति॒ष्ठाम् प्र॑ति॒ष्ठाम् तम् तम् प्र॑ति॒ष्ठाम् ग॑मयति गमयति प्रति॒ष्ठाम् तम् तम् प्र॑ति॒ष्ठाम् ग॑मयति । \newline
6. प्र॒ति॒ष्ठाम् ग॑मयति गमयति प्रति॒ष्ठाम् प्र॑ति॒ष्ठाम् ग॑मयति॒ परा॒ परा॑ गमयति प्रति॒ष्ठाम् प्र॑ति॒ष्ठाम् ग॑मयति॒ परा᳚ । \newline
7. प्र॒ति॒ष्ठामिति॑ प्रति - स्थाम् । \newline
8. ग॒म॒य॒ति॒ परा॒ परा॑ गमयति गमयति॒ परा॒ वै वै परा॑ गमयति गमयति॒ परा॒ वै । \newline
9. परा॒ वै वै परा॒ परा॒ वा ए॒ष ए॒ष वै परा॒ परा॒ वा ए॒षः । \newline
10. वा ए॒ष ए॒ष वै वा ए॒षो᳚ ऽग्नि म॒ग्नि मे॒ष वै वा ए॒षो᳚ ऽग्निम् । \newline
11. ए॒षो᳚ ऽग्नि म॒ग्नि मे॒ष ए॒षो᳚ ऽग्निं ॅव॑पति वप त्य॒ग्नि मे॒ष ए॒षो᳚ ऽग्निं ॅव॑पति । \newline
12. अ॒ग्निं ॅव॑पति वप त्य॒ग्नि म॒ग्निं ॅव॑पति॒ यो यो व॑प त्य॒ग्नि म॒ग्निं ॅव॑पति॒ यः । \newline
13. व॒प॒ति॒ यो यो व॑पति वपति॒ यो᳚(1॒) ऽफ्स्व॑फ्सु यो व॑पति वपति॒ यो᳚ ऽफ्सु । \newline
14. यो᳚(1॒) ऽफ्स्व॑फ्सु यो यो᳚ ऽफ्सु भस्म॒ भस्मा॒फ्सु यो यो᳚ ऽफ्सु भस्म॑ । \newline
15. अ॒फ्सु भस्म॒ भस्मा॒ फ्स्व॑फ्सु भस्म॑ प्रवे॒शय॑ति प्रवे॒शय॑ति॒ भस्मा॒ फ्स्व॑फ्सु भस्म॑ प्रवे॒शय॑ति । \newline
16. अ॒फ्स्वित्य॑प् - सु । \newline
17. भस्म॑ प्रवे॒शय॑ति प्रवे॒शय॑ति॒ भस्म॒ भस्म॑ प्रवे॒शय॑ति॒ ज्योति॑ष्मतीभ्या॒म् ज्योति॑ष्मतीभ्याम् प्रवे॒शय॑ति॒ भस्म॒ भस्म॑ प्रवे॒शय॑ति॒ ज्योति॑ष्मतीभ्याम् । \newline
18. प्र॒वे॒शय॑ति॒ ज्योति॑ष्मतीभ्या॒म् ज्योति॑ष्मतीभ्याम् प्रवे॒शय॑ति प्रवे॒शय॑ति॒ ज्योति॑ष्मतीभ्या॒ मवाव॒ ज्योति॑ष्मतीभ्याम् प्रवे॒शय॑ति प्रवे॒शय॑ति॒ ज्योति॑ष्मतीभ्या॒ मव॑ । \newline
19. प्र॒वे॒शय॒तीति॑ प्र - वे॒शय॑ति । \newline
20. ज्योति॑ष्मतीभ्या॒ मवाव॒ ज्योति॑ष्मतीभ्या॒म् ज्योति॑ष्मतीभ्या॒ मव॑ दधाति दधा॒त्यव॒ ज्योति॑ष्मतीभ्या॒म् ज्योति॑ष्मतीभ्या॒ मव॑ दधाति । \newline
21. अव॑ दधाति दधा॒ त्यवाव॑ दधाति॒ ज्योति॒र् ज्योति॑र् दधा॒ त्यवाव॑ दधाति॒ ज्योतिः॑ । \newline
22. द॒धा॒ति॒ ज्योति॒र् ज्योति॑र् दधाति दधाति॒ ज्योति॑ रे॒वैव ज्योति॑र् दधाति दधाति॒ ज्योति॑ रे॒व । \newline
23. ज्योति॑ रे॒वैव ज्योति॒र् ज्योति॑ रे॒वास्मि॑न् नस्मिन् ने॒व ज्योति॒र् ज्योति॑ रे॒वास्मिन्न्॑ । \newline
24. ए॒वास्मि॑न् नस्मिन् ने॒वैवास्मि॑न् दधाति दधा त्यस्मिन् ने॒वैवास्मि॑न् दधाति । \newline
25. अ॒स्मि॒न् द॒धा॒ति॒ द॒धा॒ त्य॒स्मि॒न् न॒स्मि॒न् द॒धा॒ति॒ द्वाभ्या॒म् द्वाभ्या᳚म् दधा त्यस्मिन् नस्मिन् दधाति॒ द्वाभ्या᳚म् । \newline
26. द॒धा॒ति॒ द्वाभ्या॒म् द्वाभ्या᳚म् दधाति दधाति॒ द्वाभ्या॒म् प्रति॑ष्ठित्यै॒ प्रति॑ष्ठित्यै॒ द्वाभ्या᳚म् दधाति दधाति॒ द्वाभ्या॒म् प्रति॑ष्ठित्यै । \newline
27. द्वाभ्या॒म् प्रति॑ष्ठित्यै॒ प्रति॑ष्ठित्यै॒ द्वाभ्या॒म् द्वाभ्या॒म् प्रति॑ष्ठित्यै॒ परा॒ परा॒ प्रति॑ष्ठित्यै॒ द्वाभ्या॒म् द्वाभ्या॒म् प्रति॑ष्ठित्यै॒ परा᳚ । \newline
28. प्रति॑ष्ठित्यै॒ परा॒ परा॒ प्रति॑ष्ठित्यै॒ प्रति॑ष्ठित्यै॒ परा॒ वै वै परा॒ प्रति॑ष्ठित्यै॒ प्रति॑ष्ठित्यै॒ परा॒ वै । \newline
29. प्रति॑ष्ठित्या॒ इति॒ प्रति॑ - स्थि॒त्यै॒ । \newline
30. परा॒ वै वै परा॒ परा॒ वा ए॒ष ए॒ष वै परा॒ परा॒ वा ए॒षः । \newline
31. वा ए॒ष ए॒ष वै वा ए॒ष प्र॒जाम् प्र॒जा मे॒ष वै वा ए॒ष प्र॒जाम् । \newline
32. ए॒ष प्र॒जाम् प्र॒जा मे॒ष ए॒ष प्र॒जाम् प॒शून् प॒शून् प्र॒जा मे॒ष ए॒ष प्र॒जाम् प॒शून् । \newline
33. प्र॒जाम् प॒शून् प॒शून् प्र॒जाम् प्र॒जाम् प॒शून्. व॑पति वपति प॒शून् प्र॒जाम् प्र॒जाम् प॒शून्. व॑पति । \newline
34. प्र॒जामिति॑ प्र - जाम् । \newline
35. प॒शून्. व॑पति वपति प॒शून् प॒शून्. व॑पति॒ यो यो व॑पति प॒शून् प॒शून्. व॑पति॒ यः । \newline
36. व॒प॒ति॒ यो यो व॑पति वपति॒ यो᳚(1॒) ऽफ्स्व॑फ्सु यो व॑पति वपति॒ यो᳚ ऽफ्सु । \newline
37. यो᳚(1॒) ऽफ्स्व॑फ्सु यो यो᳚ ऽफ्सु भस्म॒ भस्मा॒फ्सु यो यो᳚ ऽफ्सु भस्म॑ । \newline
38. अ॒फ्सु भस्म॒ भस्मा॒ फ्स्व॑फ्सु भस्म॑ प्रवे॒शय॑ति प्रवे॒शय॑ति॒ भस्मा॒ फ्स्व॑फ्सु भस्म॑ प्रवे॒शय॑ति । \newline
39. अ॒फ्स्वित्य॑प् - सु । \newline
40. भस्म॑ प्रवे॒शय॑ति प्रवे॒शय॑ति॒ भस्म॒ भस्म॑ प्रवे॒शय॑ति॒ पुनः॒ पुनः॑ प्रवे॒शय॑ति॒ भस्म॒ भस्म॑ प्रवे॒शय॑ति॒ पुनः॑ । \newline
41. प्र॒वे॒शय॑ति॒ पुनः॒ पुनः॑ प्रवे॒शय॑ति प्रवे॒शय॑ति॒ पुन॑ रू॒र्जोर्जा पुनः॑ प्रवे॒शय॑ति प्रवे॒शय॑ति॒ पुन॑ रू॒र्जा । \newline
42. प्र॒वे॒शय॒तीति॑ प्र - वे॒शय॑ति । \newline
43. पुन॑ रू॒र्जोर्जा पुनः॒ पुन॑ रू॒र्जा स॒ह स॒होर्जा पुनः॒ पुन॑ रू॒र्जा स॒ह । \newline
44. ऊ॒र्जा स॒ह स॒होर्जोर्जा स॒ह र॒य्या र॒य्या स॒होर्जोर्जा स॒ह र॒य्या । \newline
45. स॒ह र॒य्या र॒य्या स॒ह स॒ह र॒य्येतीति॑ र॒य्या स॒ह स॒ह र॒य्येति॑ । \newline
46. र॒य्येतीति॑ र॒य्या र॒य्येति॒ पुनः॒ पुन॒ रिति॑ र॒य्या र॒य्येति॒ पुनः॑ । \newline
47. इति॒ पुनः॒ पुन॒ रितीति॒ पुन॑ रु॒दै त्यु॒दैति॒ पुन॒ रितीति॒ पुन॑ रु॒दैति॑ । \newline
48. पुन॑ रु॒दै त्यु॒दैति॒ पुनः॒ पुन॑ रु॒दैति॑ प्र॒जाम् प्र॒जा मु॒दैति॒ पुनः॒ पुन॑ रु॒दैति॑ प्र॒जाम् । \newline
49. उ॒दैति॑ प्र॒जाम् प्र॒जा मु॒दै त्यु॒दैति॑ प्र॒जा मे॒वैव प्र॒जा मु॒दै त्यु॒दैति॑ प्र॒जा मे॒व । \newline
50. उ॒दैतीत्यु॑त् - ऐति॑ । \newline
51. प्र॒जा मे॒वैव प्र॒जाम् प्र॒जा मे॒व प॒शून् प॒शू ने॒व प्र॒जाम् प्र॒जा मे॒व प॒शून् । \newline
52. प्र॒जामिति॑ प्र - जाम् । \newline
53. ए॒व प॒शून् प॒शू ने॒वैव प॒शू ना॒त्मन् ना॒त्मन् प॒शू ने॒वैव प॒शू ना॒त्मन्न् । \newline
54. प॒शू ना॒त्मन् ना॒त्मन् प॒शून् प॒शू ना॒त्मन् ध॑त्ते धत्त आ॒त्मन् प॒शून् प॒शू ना॒त्मन् ध॑त्ते । \newline
55. आ॒त्मन् ध॑त्ते धत्त आ॒त्मन् ना॒त्मन् ध॑त्ते॒ पुनः॒ पुन॑र् धत्त आ॒त्मन् ना॒त्मन् ध॑त्ते॒ पुनः॑ । \newline
56. ध॒त्ते॒ पुनः॒ पुन॑र् धत्ते धत्ते॒ पुन॑ स्त्वा त्वा॒ पुन॑र् धत्ते धत्ते॒ पुन॑ स्त्वा । \newline
57. पुन॑ स्त्वा त्वा॒ पुनः॒ पुन॑ स्त्वा ऽऽदि॒त्या आ॑दि॒त्या स्त्वा॒ पुनः॒ पुन॑ स्त्वा ऽऽदि॒त्याः । \newline
58. त्वा॒ ऽऽदि॒त्या आ॑दि॒त्या स्त्वा᳚ त्वा ऽऽदि॒त्या रु॒द्रा रु॒द्रा आ॑दि॒त्या स्त्वा᳚ त्वा ऽऽदि॒त्या रु॒द्राः । \newline
59. आ॒दि॒त्या रु॒द्रा रु॒द्रा आ॑दि॒त्या आ॑दि॒त्या रु॒द्रा वस॑वो॒ वस॑वो रु॒द्रा आ॑दि॒त्या आ॑दि॒त्या रु॒द्रा वस॑वः । \newline
\pagebreak
\markright{ TS 5.2.2.6  \hfill https://www.vedavms.in \hfill}

\section{ TS 5.2.2.6 }

\textbf{TS 5.2.2.6 } \newline
\textbf{Samhita Paata} \newline

रु॒द्रा वस॑वः॒ समि॑न्धता॒-मित्या॑है॒ता वा ए॒तं दे॒वता॒ अग्रे॒ समै᳚न्धत॒ ताभि॑रे॒वैनꣳ॒॒ समि॑न्धे॒ बोधा॒ स बो॒धीत्युप॑ तिष्ठते बो॒धय॑त्ये॒वैनं॒ तस्मा᳚थ् सु॒प्त्वा प्र॒जाः प्रबु॑द्ध्यन्ते यथास्था॒नमुप॑ तिष्ठते॒ तस्मा᳚द्-यथास्था॒नं प॒शवः॒ पुन॒रेत्योप॑ तिष्ठन्ते ॥ \newline

\textbf{Pada Paata} \newline

रु॒द्राः । वस॑वः । समिति॑ । इ॒न्ध॒ता॒म् । इति॑ । आ॒ह॒ । ए॒ताः । वै । ए॒तम् । दे॒वताः᳚ । अग्रे᳚ । समिति॑ । ऐ॒न्ध॒त॒ । ताभिः॑ । ए॒व । ए॒न॒म् । समिति॑ । इ॒न्धे॒ । बोध॑ । सः । बो॒धि॒ । इति॑ । उपेति॑ । ति॒ष्ठ॒ते॒ । बो॒धय॑ति । ए॒व । ए॒न॒म् । तस्मा᳚त् । सु॒प्त्वा । प्र॒जा इति॑ प्र - जाः । प्रेति॑ । बु॒द्ध्य॒न्ते॒ । य॒था॒स्था॒नमिति॑ यथा - स्था॒नम् । उपेति॑ । ति॒ष्ठ॒ते॒ । तस्मा᳚त् । य॒था॒स्था॒नमिति॑ यथा - स्था॒नम् । प॒शवः॑ । पुनः॑ । एत्येत्या᳚ - इत्य॑ । उपेति॑ । ति॒ष्ठ॒न्ते॒ ॥  \newline


\textbf{Krama Paata} \newline

रु॒द्रा वस॑वः । वस॑वः॒ सम् । समि॑न्धताम् । इ॒न्ध॒ता॒मिति॑ । इत्या॑ह । आ॒है॒ताः । ए॒ता वै । वा ए॒तम् । ए॒तम् दे॒वताः᳚ । दे॒वता॒ अग्रे᳚ । अग्रे॒ सम् । समै᳚न्धत । ऐ॒न्ध॒त॒ ताभिः॑ । ताभि॑रे॒व । ए॒वैन᳚म् । ए॒नꣳ॒॒ सम् । समि॑न्धे । इ॒न्धे॒ बोध॑ । बोधा॒ सः । स बो॑धि । बो॒धीति॑ । इत्युप॑ । उप॑ तिष्ठते । ति॒ष्ठ॒ते॒ बो॒धय॑ति । बो॒धय॑त्ये॒व । ए॒वैन᳚म् । ए॒न॒म् तस्मा᳚त् । तस्मा᳚थ् सु॒प्त्वा । सु॒प्त्वा प्र॒जाः । प्र॒जाः प्र । प्र॒जा इति॑ प्र - जाः । प्र बु॑द्ध्यन्ते । बु॒द्ध्य॒न्ते॒ य॒था॒स्था॒नम् । य॒था॒स्था॒नमुप॑ । य॒था॒स्था॒नमिति॑ यथा - स्था॒नम् । उप॑ तिष्ठते । ति॒ष्ठ॒ते॒ तस्मा᳚त् । तस्मा᳚द् यथास्था॒नम् । य॒था॒स्था॒नम् प॒शवः॑ । य॒था॒स्था॒नमिति॑ यथा - स्था॒नम् । प॒शवः॒ पुनः॑ । पुन॒रेत्य॑ । एत्योप॑ । एत्येत्या᳚ - इत्य॑ । उप॑ तिष्ठन्ते । ति॒ष्ठ॒न्त॒ इति॑ तिष्ठन्ते । \newline

\textbf{Jatai Paata} \newline

1. रु॒द्रा वस॑वो॒ वस॑वो रु॒द्रा रु॒द्रा वस॑वः । \newline
2. वस॑वः॒ सꣳ सं ॅवस॑वो॒ वस॑वः॒ सम् । \newline
3. स मि॑न्धता मिन्धताꣳ॒॒ सꣳ स मि॑न्धताम् । \newline
4. इ॒न्ध॒ता॒ मितीती᳚ न्धता मिन्धता॒ मिति॑ । \newline
5. इत्या॑हा॒हे तीत्या॑ह । \newline
6. आ॒है॒ता ए॒ता आ॑हा है॒ताः । \newline
7. ए॒ता वै वा ए॒ता ए॒ता वै । \newline
8. वा ए॒त मे॒तं ॅवै वा ए॒तम् । \newline
9. ए॒तम् दे॒वता॑ दे॒वता॑ ए॒त मे॒तम् दे॒वताः᳚ । \newline
10. दे॒वता॒ अग्रे ऽग्रे॑ दे॒वता॑ दे॒वता॒ अग्रे᳚ । \newline
11. अग्रे॒ सꣳ स मग्रे ऽग्रे॒ सम् । \newline
12. स मै᳚न्ध तैन्धत॒ सꣳ स मै᳚न्धत । \newline
13. ऐ॒न्ध॒त॒ ताभि॒ स्ताभि॑ रैन्ध तैन्धत॒ ताभिः॑ । \newline
14. ताभि॑ रे॒वैव ताभि॒ स्ताभि॑ रे॒व । \newline
15. ए॒वैन॑ मेन मे॒वै वैन᳚म् । \newline
16. ए॒नꣳ॒॒ सꣳ स मे॑न मेनꣳ॒॒ सम् । \newline
17. स मि॑न्ध इन्धे॒ सꣳ स मि॑न्धे । \newline
18. इ॒न्धे॒ बोध॒ बोधे᳚ न्ध इन्धे॒ बोध॑ । \newline
19. बोधा॒ स स बोध॒ बोधा॒ सः । \newline
20. स बो॑धि बोधि॒ स स बो॑धि । \newline
21. बो॒धीतीति॑ बोधि बो॒धीति॑ । \newline
22. इत्युपोपे तीत्युप॑ । \newline
23. उप॑ तिष्ठते तिष्ठत॒ उपोप॑ तिष्ठते । \newline
24. ति॒ष्ठ॒ते॒ बो॒धय॑ति बो॒धय॑ति तिष्ठते तिष्ठते बो॒धय॑ति । \newline
25. बो॒धय॑ त्ये॒वैव बो॒धय॑ति बो॒धय॑ त्ये॒व । \newline
26. ए॒वैन॑ मेन मे॒वै वैन᳚म् । \newline
27. ए॒न॒म् तस्मा॒त् तस्मा॑ देन मेन॒म् तस्मा᳚त् । \newline
28. तस्मा᳚थ् सु॒प्त्वा सु॒प्त्वा तस्मा॒त् तस्मा᳚थ् सु॒प्त्वा । \newline
29. सु॒प्त्वा प्र॒जाः प्र॒जाः सु॒प्त्वा सु॒प्त्वा प्र॒जाः । \newline
30. प्र॒जाः प्र प्र प्र॒जाः प्र॒जाः प्र । \newline
31. प्र॒जा इति॑ प्र - जाः । \newline
32. प्र बु॑द्ध्यन्ते बुद्ध्यन्ते॒ प्र प्र बु॑द्ध्यन्ते । \newline
33. बु॒द्ध्य॒न्ते॒ य॒था॒स्था॒नं ॅय॑थास्था॒नम् बु॑द्ध्यन्ते बुद्ध्यन्ते यथास्था॒नम् । \newline
34. य॒था॒स्था॒न मुपोप॑ यथास्था॒नं ॅय॑थास्था॒न मुप॑ । \newline
35. य॒था॒स्था॒नमिति॑ यथा - स्था॒नम् । \newline
36. उप॑ तिष्ठते तिष्ठत॒ उपोप॑ तिष्ठते । \newline
37. ति॒ष्ठ॒ते॒ तस्मा॒त् तस्मा᳚त् तिष्ठते तिष्ठते॒ तस्मा᳚त् । \newline
38. तस्मा᳚द् यथास्था॒नं ॅय॑थास्था॒नम् तस्मा॒त् तस्मा᳚द् यथास्था॒नम् । \newline
39. य॒था॒स्था॒नम् प॒शवः॑ प॒शवो॑ यथास्था॒नं ॅय॑थास्था॒नम् प॒शवः॑ । \newline
40. य॒था॒स्था॒नमिति॑ यथा - स्था॒नम् । \newline
41. प॒शवः॒ पुनः॒ पुनः॑ प॒शवः॑ प॒शवः॒ पुनः॑ । \newline
42. पुन॒ रेत्येत्य॒ पुनः॒ पुन॒ रेत्य॑ । \newline
43. एत्योपो पेत्ये त्योप॑ । \newline
44. एत्येत्या᳚ - इत्य॑ । \newline
45. उप॑ तिष्ठन्ते तिष्ठन्त॒ उपोप॑ तिष्ठन्ते । \newline
46. ति॒ष्ठ॒न्त॒ इति॑ तिष्ठन्ते । \newline

\textbf{Ghana Paata } \newline

1. रु॒द्रा वस॑वो॒ वस॑वो रु॒द्रा रु॒द्रा वस॑वः॒ सꣳ सं ॅवस॑वो रु॒द्रा रु॒द्रा वस॑वः॒ सम् । \newline
2. वस॑वः॒ सꣳ सं ॅवस॑वो॒ वस॑वः॒ स मि॑न्धता मिन्धताꣳ॒॒ सं ॅवस॑वो॒ वस॑वः॒ स मि॑न्धताम् । \newline
3. स मि॑न्धता मिन्धताꣳ॒॒ सꣳ स मि॑न्धता॒ मितीती᳚न्धताꣳ॒॒ सꣳ स मि॑न्धता॒ मिति॑ । \newline
4. इ॒न्ध॒ता॒ मितीती᳚न्धता मिन्धता॒ मित्या॑हा॒हे ती᳚न्धता मिन्धता॒ मित्या॑ह । \newline
5. इत्या॑हा॒हे तीत्या॑है॒ता ए॒ता आ॒हे तीत्या॑है॒ताः । \newline
6. आ॒है॒ता ए॒ता आ॑हाहै॒ता वै वा ए॒ता आ॑हाहै॒ता वै । \newline
7. ए॒ता वै वा ए॒ता ए॒ता वा ए॒त मे॒तं ॅवा ए॒ता ए॒ता वा ए॒तम् । \newline
8. वा ए॒त मे॒तं ॅवै वा ए॒तम् दे॒वता॑ दे॒वता॑ ए॒तं ॅवै वा ए॒तम् दे॒वताः᳚ । \newline
9. ए॒तम् दे॒वता॑ दे॒वता॑ ए॒त मे॒तम् दे॒वता॒ अग्रे ऽग्रे॑ दे॒वता॑ ए॒त मे॒तम् दे॒वता॒ अग्रे᳚ । \newline
10. दे॒वता॒ अग्रे ऽग्रे॑ दे॒वता॑ दे॒वता॒ अग्रे॒ सꣳ स मग्रे॑ दे॒वता॑ दे॒वता॒ अग्रे॒ सम् । \newline
11. अग्रे॒ सꣳ स मग्रे ऽग्रे॒ स मै᳚न्ध तैन्धत॒ स मग्रे ऽग्रे॒ स मै᳚न्धत । \newline
12. स मै᳚न्ध तैन्धत॒ सꣳ स मै᳚न्धत॒ ताभि॒ स्ताभि॑ रैन्धत॒ सꣳ स मै᳚न्धत॒ ताभिः॑ । \newline
13. ऐ॒न्ध॒त॒ ताभि॒ स्ताभि॑ रैन्ध तैन्धत॒ ताभि॑ रे॒वैव ताभि॑ रैन्ध तैन्धत॒ ताभि॑ रे॒व । \newline
14. ताभि॑ रे॒वैव ताभि॒ स्ताभि॑ रे॒वैन॑ मेन मे॒व ताभि॒ स्ताभि॑ रे॒वैन᳚म् । \newline
15. ए॒वैन॑ मेन मे॒वैवैनꣳ॒॒ सꣳ स मे॑न मे॒वैवैनꣳ॒॒ सम् । \newline
16. ए॒नꣳ॒॒ सꣳ स मे॑न मेनꣳ॒॒ स मि॑न्ध इन्धे॒ स मे॑न मेनꣳ॒॒ स मि॑न्धे । \newline
17. स मि॑न्ध इन्धे॒ सꣳ स मि॑न्धे॒ बोध॒ बोधे᳚न्धे॒ सꣳ स मि॑न्धे॒ बोध॑ । \newline
18. इ॒न्धे॒ बोध॒ बोधे᳚न्ध इन्धे॒ बोधा॒ स स बोधे᳚न्ध इन्धे॒ बोधा॒ सः । \newline
19. बोधा॒ स स बोध॒ बोधा॒ स बो॑धि बोधि॒ स बोध॒ बोधा॒ स बो॑धि । \newline
20. स बो॑धि बोधि॒ स स बो॒धीतीति॑ बोधि॒ स स बो॒धीति॑ । \newline
21. बो॒धीतीति॑ बोधि बो॒धी त्युपोपेति॑ बोधि बो॒धी त्युप॑ । \newline
22. इत्युपोपे तीत्युप॑ तिष्ठते तिष्ठत॒ उपे तीत्युप॑ तिष्ठते । \newline
23. उप॑ तिष्ठते तिष्ठत॒ उपोप॑ तिष्ठते बो॒धय॑ति बो॒धय॑ति तिष्ठत॒ उपोप॑ तिष्ठते बो॒धय॑ति । \newline
24. ति॒ष्ठ॒ते॒ बो॒धय॑ति बो॒धय॑ति तिष्ठते तिष्ठते बो॒धय॑ त्ये॒वैव बो॒धय॑ति तिष्ठते तिष्ठते बो॒धय॑ त्ये॒व । \newline
25. बो॒धय॑ त्ये॒वैव बो॒धय॑ति बो॒धय॑ त्ये॒वैन॑ मेन मे॒व बो॒धय॑ति बो॒धय॑ त्ये॒वैन᳚म् । \newline
26. ए॒वैन॑ मेन मे॒वैवैन॒म् तस्मा॒त् तस्मा॑ देन मे॒वैवैन॒म् तस्मा᳚त् । \newline
27. ए॒न॒म् तस्मा॒त् तस्मा॑ देन मेन॒म् तस्मा᳚थ् सु॒प्त्वा सु॒प्त्वा तस्मा॑ देन मेन॒म् तस्मा᳚थ् सु॒प्त्वा । \newline
28. तस्मा᳚थ् सु॒प्त्वा सु॒प्त्वा तस्मा॒त् तस्मा᳚थ् सु॒प्त्वा प्र॒जाः प्र॒जाः सु॒प्त्वा तस्मा॒त् तस्मा᳚थ् सु॒प्त्वा प्र॒जाः । \newline
29. सु॒प्त्वा प्र॒जाः प्र॒जाः सु॒प्त्वा सु॒प्त्वा प्र॒जाः प्र प्र प्र॒जाः सु॒प्त्वा सु॒प्त्वा प्र॒जाः प्र । \newline
30. प्र॒जाः प्र प्र प्र॒जाः प्र॒जाः प्र बु॑द्ध्यन्ते बुद्ध्यन्ते॒ प्र प्र॒जाः प्र॒जाः प्र बु॑द्ध्यन्ते । \newline
31. प्र॒जा इति॑ प्र - जाः । \newline
32. प्र बु॑द्ध्यन्ते बुद्ध्यन्ते॒ प्र प्र बु॑द्ध्यन्ते यथास्था॒नं ॅय॑थास्था॒नम् बु॑द्ध्यन्ते॒ प्र प्र बु॑द्ध्यन्ते यथास्था॒नम् । \newline
33. बु॒द्ध्य॒न्ते॒ य॒था॒स्था॒नं ॅय॑थास्था॒नम् बु॑द्ध्यन्ते बुद्ध्यन्ते यथास्था॒न मुपोप॑ यथास्था॒नम् बु॑द्ध्यन्ते बुद्ध्यन्ते यथास्था॒न मुप॑ । \newline
34. य॒था॒स्था॒न मुपोप॑ यथास्था॒नं ॅय॑थास्था॒न मुप॑ तिष्ठते तिष्ठत॒ उप॑ यथास्था॒नं ॅय॑थास्था॒न मुप॑ तिष्ठते । \newline
35. य॒था॒स्था॒नमिति॑ यथा - स्था॒नम् । \newline
36. उप॑ तिष्ठते तिष्ठत॒ उपोप॑ तिष्ठते॒ तस्मा॒त् तस्मा᳚त् तिष्ठत॒ उपोप॑ तिष्ठते॒ तस्मा᳚त् । \newline
37. ति॒ष्ठ॒ते॒ तस्मा॒त् तस्मा᳚त् तिष्ठते तिष्ठते॒ तस्मा᳚द् यथास्था॒नं ॅय॑थास्था॒नम् तस्मा᳚त् तिष्ठते तिष्ठते॒ तस्मा᳚द् यथास्था॒नम् । \newline
38. तस्मा᳚द् यथास्था॒नं ॅय॑थास्था॒नम् तस्मा॒त् तस्मा᳚द् यथास्था॒नम् प॒शवः॑ प॒शवो॑ यथास्था॒नम् तस्मा॒त् तस्मा᳚द् यथास्था॒नम् प॒शवः॑ । \newline
39. य॒था॒स्था॒नम् प॒शवः॑ प॒शवो॑ यथास्था॒नं ॅय॑थास्था॒नम् प॒शवः॒ पुनः॒ पुनः॑ प॒शवो॑ यथास्था॒नं ॅय॑थास्था॒नम् प॒शवः॒ पुनः॑ । \newline
40. य॒था॒स्था॒नमिति॑ यथा - स्था॒नम् । \newline
41. प॒शवः॒ पुनः॒ पुनः॑ प॒शवः॑ प॒शवः॒ पुन॒ रेत्येत्य॒ पुनः॑ प॒शवः॑ प॒शवः॒ पुन॒ रेत्य॑ । \newline
42. पुन॒ रेत्येत्य॒ पुनः॒ पुन॒ रेत्यो पोपेत्य॒ पुनः॒ पुन॒ रेत्योप॑ । \newline
43. एत्यो पोपे त्येत्योप॑ तिष्ठन्ते तिष्ठन्त॒ उपे त्येत्योप॑ तिष्ठन्ते । \newline
44. एत्येत्या᳚ - इत्य॑ । \newline
45. उप॑ तिष्ठन्ते तिष्ठन्त॒ उपोप॑ तिष्ठन्ते । \newline
46. ति॒ष्ठ॒न्त॒ इति॑ तिष्ठन्ते । \newline
\pagebreak
\markright{ TS 5.2.3.1  \hfill https://www.vedavms.in \hfill}

\section{ TS 5.2.3.1 }

\textbf{TS 5.2.3.1 } \newline
\textbf{Samhita Paata} \newline

याव॑ती॒ वै पृ॑थि॒वी तस्यै॑ य॒म आधि॑पत्यं॒ परी॑याय॒ यो वै य॒मं दे॑व॒यज॑नम॒स्या अनि॑र्याच्या॒ऽग्निं चि॑नु॒ते य॒मायै॑नꣳ॒॒ स चि॑नु॒तेऽपे॒ते-त्य॒द्ध्यव॑साययति य॒ममे॒व दे॑व॒यज॑नम॒स्यै नि॒र्याच्या॒- ऽऽत्मने॒ऽग्निं चि॑नुत इष्व॒ग्रेण॒ वा अ॒स्या अना॑मृत-मि॒च्छन्तो॒ नावि॑न्द॒न् ते दे॒वा ए॒तद्-यजु॑रपश्य॒न्नपे॒तेति॒ यदे॒तेना᳚-ध्यवसा॒यय॒त्य - [  ] \newline

\textbf{Pada Paata} \newline

याव॑ती । वै । पृ॒थि॒वी । तस्यै᳚ । य॒मः । आधि॑पत्य॒मित्याधि॑ - प॒त्य॒म् । परीति॑ । इ॒या॒य॒ । यः । वै । य॒मम् । दे॒व॒यज॑न॒मिति॑ देव - यज॑नम् । अ॒स्याः । अनि॑र्या॒च्येत्यनिः॑ - या॒च्य॒ । अ॒ग्निम् । चि॒नु॒ते । य॒माय॑ । ए॒न॒म् । सः । चि॒नु॒ते॒ । अपेति॑ । इ॒त॒ । इति॑ । अ॒द्ध्यव॑सायय॒तीत्य॑धि - अव॑साययति । य॒मम् । ए॒व । दे॒व॒यज॑न॒मिति॑ देव-यज॑नम् । अ॒स्यै । नि॒र्याच्येति॑ निः - याच्य॑ । आ॒त्मने᳚ । अ॒ग्निम् । चि॒नु॒ते॒ । इ॒ष्व॒ग्रेणेती॑षु - अ॒ग्रेण॑ । वै । अ॒स्याः । अना॑मृत॒मित्यना᳚ - मृ॒त॒म् । इ॒च्छन्तः॑ । न । अ॒वि॒न्द॒न्न् । ते । दे॒वाः । ए॒तत् । यजुः॑ । अ॒प॒श्य॒न्न् । अपेति॑ । इ॒त॒ । इति॑ । यत् । ए॒तेन॑ । अ॒द्ध्य॒व॒सा॒यय॒तीत्य॑धि - अ॒व॒सा॒यय॑ति ।  \newline


\textbf{Krama Paata} \newline

याव॑ती॒ वै । वै पृ॑थि॒वी । पृ॒थि॒वी तस्यै᳚ । तस्यै॑ य॒मः । य॒म आधि॑पत्यम् । आधि॑पत्य॒म् परि॑ । आधि॑पत्य॒मित्याधि॑ - प॒त्य॒म् । परी॑याय । इ॒या॒य॒ यः । यो वै । वै य॒मम् । य॒मम् दे॑व॒यज॑नम् । दे॒व॒यज॑नम॒स्याः । दे॒व॒यज॑न॒मिति॑ देव - यज॑नम् । अ॒स्या अनि॑र्याच्य । अनि॑र्याच्या॒ग्निम् । अनि॑र्या॒च्येत्यनिः॑ - या॒च्य॒ । अ॒ग्निम् चि॑नु॒ते । चि॒नु॒ते य॒माय॑ । य॒मायै॑नम् । ए॒नꣳ॒॒ सः । स चि॑नुते । चि॒नु॒तेऽप॑ । अपेत॑ । इ॒तेति॑ । इत्य॒द्ध्यव॑साययति । अ॒द्ध्यव॑साययति य॒मम् । अ॒द्ध्यव॑सायय॒तीत्य॑धि - अव॑साययति । य॒ममे॒व । ए॒व दे॑व॒यज॑नम् । दे॒व॒यज॑नम॒स्यै । दे॒व॒यज॑न॒मिति॑ देव - यज॑नम् । अ॒स्यै नि॒र्याच्य॑ । नि॒र्याच्या॒त्मने᳚ । नि॒र्याच्येति॑ निः - याच्य॑ । आ॒त्मने॒ऽग्निम् । अ॒ग्निम् चि॑नुते । चि॒नु॒त॒ इ॒ष्व॒ग्रेण॑ । इ॒ष्व॒ग्रेण॒ वै । इ॒ष्व॒ग्रेणेती॑षु - अ॒ग्रेण॑ । वा अ॒स्याः । अ॒स्या अना॑मृतम् । अना॑मृतमि॒च्छन्तः॑ । अना॑मृत॒मित्यना᳚ - मृ॒त॒म् । इ॒च्छन्तो॒ न । नावि॑न्दन्न् । अ॒वि॒न्द॒न् ते । ते दे॒वाः । दे॒वा ए॒तत् । ए॒तद् यजुः॑ । यजु॑रपश्यन्न् । अ॒प॒श्य॒न्नप॑ । अपे॑त । इ॒तेति॑ । इति॒ यत् । यदे॒तेन॑ । ए॒तेना᳚द्ध्यवसा॒यय॑ति । अ॒द्ध्य॒व॒सा॒यय॒त्यना॑मृते । अ॒द्ध्य॒व॒सा॒यय॒तीत्य॑धि - अ॒व॒सा॒यय॑ति \newline

\textbf{Jatai Paata} \newline

1. याव॑ती॒ वै वै याव॑ती॒ याव॑ती॒ वै । \newline
2. वै पृ॑थि॒वी पृ॑थि॒वी वै वै पृ॑थि॒वी । \newline
3. पृ॒थि॒वी तस्यै॒ तस्यै॑ पृथि॒वी पृ॑थि॒वी तस्यै᳚ । \newline
4. तस्यै॑ य॒मो य॒म स्तस्यै॒ तस्यै॑ य॒मः । \newline
5. य॒म आधि॑पत्य॒ माधि॑पत्यं ॅय॒मो य॒म आधि॑पत्यम् । \newline
6. आधि॑पत्य॒म् परि॒ पर्याधि॑पत्य॒ माधि॑पत्य॒म् परि॑ । \newline
7. आधि॑पत्य॒मित्याधि॑ - प॒त्य॒म् । \newline
8. परी॑या येयाय॒ परि॒ परी॑याय । \newline
9. इ॒या॒य॒ यो य इ॑या येयाय॒ यः । \newline
10. यो वै वै यो यो वै । \newline
11. वै य॒मं ॅय॒मं ॅवै वै य॒मम् । \newline
12. य॒मम् दे॑व॒यज॑नम् देव॒यज॑नं ॅय॒मं ॅय॒मम् दे॑व॒यज॑नम् । \newline
13. दे॒व॒यज॑न म॒स्या अ॒स्या दे॑व॒यज॑नम् देव॒यज॑न म॒स्याः । \newline
14. दे॒व॒यज॑न॒मिति॑ देव - यज॑नम् । \newline
15. अ॒स्या अनि॑र्या॒च्या नि॑र्याच्या॒स्या अ॒स्या अनि॑र्याच्य । \newline
16. अनि॑र्याच्या॒ग्नि म॒ग्नि मनि॑र्या॒च्या नि॑र्या च्या॒ग्निम् । \newline
17. अनि॑र्या॒च्येत्यनिः॑ - या॒च्य॒ । \newline
18. अ॒ग्निम् चि॑नु॒ते चि॑नु॒ते᳚ ऽग्नि म॒ग्निम् चि॑नु॒ते । \newline
19. चि॒नु॒ते य॒माय॑ य॒माय॑ चिनु॒ते चि॑नु॒ते य॒माय॑ । \newline
20. य॒मायै॑न मेनं ॅय॒माय॑ य॒मायै॑नम् । \newline
21. ए॒नꣳ॒॒ स स ए॑न मेनꣳ॒॒ सः । \newline
22. स चि॑नुते चिनुते॒ स स चि॑नुते । \newline
23. चि॒नु॒ते ऽपाप॑ चिनुते चिनु॒ते ऽप॑ । \newline
24. अपे॑ते॒ तापापे॑त । \newline
25. इ॒ते तीती॑ते॒ तेति॑ । \newline
26. इत्य॒द्ध्यव॑सायय त्य॒द्ध्यव॑सायय॒ती तीत्य॒द्ध्यव॑साययति । \newline
27. अ॒द्ध्यव॑साययति य॒मं ॅय॒म म॒द्ध्यव॑सायय त्य॒द्ध्यव॑साययति य॒मम् । \newline
28. अ॒द्ध्यव॑सायय॒तीत्य॑धि - अव॑साययति । \newline
29. य॒म मे॒वैव य॒मं ॅय॒म मे॒व । \newline
30. ए॒व दे॑व॒यज॑नम् देव॒यज॑न मे॒वैव दे॑व॒यज॑नम् । \newline
31. दे॒व॒यज॑न म॒स्या अ॒स्यै दे॑व॒यज॑नम् देव॒यज॑न म॒स्यै । \newline
32. दे॒व॒यज॑न॒मिति॑ देव - यज॑नम् । \newline
33. अ॒स्यै नि॒र्याच्य॑ नि॒र्याच्या॒स्या अ॒स्यै नि॒र्याच्य॑ । \newline
34. नि॒र्या च्या॒त्मन॑ आ॒त्मने॑ नि॒र्याच्य॑ नि॒र्या च्या॒त्मने᳚ । \newline
35. नि॒र्याच्येति॑ निः - याच्य॑ । \newline
36. आ॒त्मने॒ ऽग्नि म॒ग्नि मा॒त्मन॑ आ॒त्मने॒ ऽग्निम् । \newline
37. अ॒ग्निम् चि॑नुते चिनुते॒ ऽग्नि म॒ग्निम् चि॑नुते । \newline
38. चि॒नु॒त॒ इ॒ष्व॒ग्रेणे᳚ ष्व॒ग्रेण॑ चिनुते चिनुत इष्व॒ग्रेण॑ । \newline
39. इ॒ष्व॒ग्रेण॒ वै वा इ॑ष्व॒ग्रेणे᳚ ष्व॒ग्रेण॒ वै । \newline
40. इ॒ष्व॒ग्रेणेती॑षु - अ॒ग्रेण॑ । \newline
41. वा अ॒स्या अ॒स्या वै वा अ॒स्याः । \newline
42. अ॒स्या अना॑मृत॒ मना॑मृत म॒स्या अ॒स्या अना॑मृतम् । \newline
43. अना॑मृत मि॒च्छन्त॑ इ॒च्छन्तो ऽना॑मृत॒ मना॑मृत मि॒च्छन्तः॑ । \newline
44. अना॑मृत॒मित्यना᳚ - मृ॒त॒म् । \newline
45. इ॒च्छन्तो॒ न ने च्छन्त॑ इ॒च्छन्तो॒ न । \newline
46. नावि॑न्दन् नविन्द॒न् न नावि॑न्दन्न् । \newline
47. अ॒वि॒न्द॒न् ते ते॑ ऽविन्दन् नविन्द॒न् ते । \newline
48. ते दे॒वा दे॒वा स्ते ते दे॒वाः । \newline
49. दे॒वा ए॒त दे॒तद् दे॒वा दे॒वा ए॒तत् । \newline
50. ए॒तद् यजु॒र् यजु॑ रे॒त दे॒तद् यजुः॑ । \newline
51. यजु॑ रपश्यन् नपश्य॒न्॒. यजु॒र् यजु॑ रपश्यन्न् । \newline
52. अ॒प॒श्य॒न् नपापा॑ पश्यन् नपश्य॒न् नप॑ । \newline
53. अपे॑ते॒ तापापे॑त । \newline
54. इ॒ते तीती॑ते॒ तेति॑ । \newline
55. इति॒ यद् यदितीति॒ यत् । \newline
56. यदे॒ते नै॒तेन॒ यद् यदे॒तेन॑ । \newline
57. ए॒ते ना᳚द्ध्यवसा॒यय॑ त्यद्ध्यवसा॒यय॑ त्ये॒तेनै॒ते ना᳚द्ध्यवसा॒यय॑ति । \newline
58. अ॒द्ध्य॒व॒सा॒यय॒ त्यना॑मृ॒ते ऽना॑मृते ऽद्ध्यवसा॒यय॑ त्यद्ध्यवसा॒यय॒ त्यना॑मृते । \newline
59. अ॒द्ध्य॒व॒सा॒यय॒तीत्य॑धि - अ॒व॒सा॒यय॑ति । \newline

\textbf{Ghana Paata } \newline

1. याव॑ती॒ वै वै याव॑ती॒ याव॑ती॒ वै पृ॑थि॒वी पृ॑थि॒वी वै याव॑ती॒ याव॑ती॒ वै पृ॑थि॒वी । \newline
2. वै पृ॑थि॒वी पृ॑थि॒वी वै वै पृ॑थि॒वी तस्यै॒ तस्यै॑ पृथि॒वी वै वै पृ॑थि॒वी तस्यै᳚ । \newline
3. पृ॒थि॒वी तस्यै॒ तस्यै॑ पृथि॒वी पृ॑थि॒वी तस्यै॑ य॒मो य॒म स्तस्यै॑ पृथि॒वी पृ॑थि॒वी तस्यै॑ य॒मः । \newline
4. तस्यै॑ य॒मो य॒म स्तस्यै॒ तस्यै॑ य॒म आधि॑पत्य॒ माधि॑पत्यं ॅय॒म स्तस्यै॒ तस्यै॑ य॒म आधि॑पत्यम् । \newline
5. य॒म आधि॑पत्य॒ माधि॑पत्यं ॅय॒मो य॒म आधि॑पत्य॒म् परि॒ पर्याधि॑पत्यं ॅय॒मो य॒म आधि॑पत्य॒म् परि॑ । \newline
6. आधि॑पत्य॒म् परि॒ पर्याधि॑पत्य॒ माधि॑पत्य॒म् परी॑या येयाय॒ पर्याधि॑पत्य॒ माधि॑पत्य॒म् परी॑याय । \newline
7. आधि॑पत्य॒मित्याधि॑ - प॒त्य॒म् । \newline
8. परी॑या येयाय॒ परि॒ परी॑याय॒ यो य इ॑याय॒ परि॒ परी॑याय॒ यः । \newline
9. इ॒या॒य॒ यो य इ॑याये याय॒ यो वै वै य इ॑याये याय॒ यो वै । \newline
10. यो वै वै यो यो वै य॒मं ॅय॒मं ॅवै यो यो वै य॒मम् । \newline
11. वै य॒मं ॅय॒मं ॅवै वै य॒मम् दे॑व॒यज॑नम् देव॒यज॑नं ॅय॒मं ॅवै वै य॒मम् दे॑व॒यज॑नम् । \newline
12. य॒मम् दे॑व॒यज॑नम् देव॒यज॑नं ॅय॒मं ॅय॒मम् दे॑व॒यज॑न म॒स्या अ॒स्या दे॑व॒यज॑नं ॅय॒मं ॅय॒मम् दे॑व॒यज॑न म॒स्याः । \newline
13. दे॒व॒यज॑न म॒स्या अ॒स्या दे॑व॒यज॑नम् देव॒यज॑न म॒स्या अनि॑र्या॒च्या नि॑र्याच्या॒स्या दे॑व॒यज॑नम् देव॒यज॑न म॒स्या अनि॑र्याच्य । \newline
14. दे॒व॒यज॑न॒मिति॑ देव - यज॑नम् । \newline
15. अ॒स्या अनि॑र्या॒च्या नि॑र्याच्या॒ स्या अ॒स्या अनि॑र्याच्या॒ग्नि म॒ग्नि मनि॑र्याच्या॒स्या अ॒स्या अनि॑र्याच्या॒ग्निम् । \newline
16. अनि॑र्याच्या॒ग्नि म॒ग्नि मनि॑र्या॒च्या नि॑र्याच्या॒ग्निम् चि॑नु॒ते चि॑नु॒ते᳚ ऽग्नि मनि॑र्या॒च्या नि॑र्याच्या॒ग्निम् चि॑नु॒ते । \newline
17. अनि॑र्या॒च्येत्यनिः॑ - या॒च्य॒ । \newline
18. अ॒ग्निम् चि॑नु॒ते चि॑नु॒ते᳚ ऽग्नि म॒ग्निम् चि॑नु॒ते य॒माय॑ य॒माय॑ चिनु॒ते᳚ ऽग्नि म॒ग्निम् चि॑नु॒ते य॒माय॑ । \newline
19. चि॒नु॒ते य॒माय॑ य॒माय॑ चिनु॒ते चि॑नु॒ते य॒मायै॑न मेनं ॅय॒माय॑ चिनु॒ते चि॑नु॒ते य॒मायै॑नम् । \newline
20. य॒मायै॑न मेनं ॅय॒माय॑ य॒मायै॑नꣳ॒॒ स स ए॑नं ॅय॒माय॑ य॒मायै॑नꣳ॒॒ सः । \newline
21. ए॒नꣳ॒॒ स स ए॑न मेनꣳ॒॒ स चि॑नुते चिनुते॒ स ए॑न मेनꣳ॒॒ स चि॑नुते । \newline
22. स चि॑नुते चिनुते॒ स स चि॑नु॒ते ऽपाप॑ चिनुते॒ स स चि॑नु॒ते ऽप॑ । \newline
23. चि॒नु॒ते ऽपाप॑ चिनुते चिनु॒ते ऽपे॑ते॒ ताप॑ चिनुते चिनु॒ते ऽपे॑त । \newline
24. अपे॑ते॒ तापा पे॒ते तीती॒ता पापे॒ तेति॑ । \newline
25. इ॒ते तीती॑ते॒ते त्य॒द्ध्यव॑सायय त्य॒द्ध्यव॑सायय॒तीती॑ ते॒ते त्य॒द्ध्यव॑साययति । \newline
26. इत्य॒द्ध्यव॑सायय त्य॒द्ध्यव॑सायय॒तीती त्य॒द्ध्यव॑साययति य॒मं ॅय॒म म॒द्ध्यव॑सायय॒तीती त्य॒द्ध्यव॑साययति य॒मम् । \newline
27. अ॒द्ध्यव॑साययति य॒मं ॅय॒म म॒द्ध्यव॑सायय त्य॒द्ध्यव॑साययति य॒म मे॒वैव य॒म म॒द्ध्यव॑सायय त्य॒द्ध्यव॑साययति य॒म मे॒व । \newline
28. अ॒द्ध्यव॑सायय॒तीत्य॑धि - अव॑साययति । \newline
29. य॒म मे॒वैव य॒मं ॅय॒म मे॒व दे॑व॒यज॑नम् देव॒यज॑न मे॒व य॒मं ॅय॒म मे॒व दे॑व॒यज॑नम् । \newline
30. ए॒व दे॑व॒यज॑नम् देव॒यज॑न मे॒वैव दे॑व॒यज॑न म॒स्या अ॒स्यै दे॑व॒यज॑न मे॒वैव दे॑व॒यज॑न म॒स्यै । \newline
31. दे॒व॒यज॑न म॒स्या अ॒स्यै दे॑व॒यज॑नम् देव॒यज॑न म॒स्यै नि॒र्याच्य॑ नि॒र्याच्या॒स्यै दे॑व॒यज॑नम् देव॒यज॑न म॒स्यै नि॒र्याच्य॑ । \newline
32. दे॒व॒यज॑न॒मिति॑ देव - यज॑नम् । \newline
33. अ॒स्यै नि॒र्याच्य॑ नि॒र्याच्या॒स्या अ॒स्यै नि॒र्याच्या॒त्मन॑ आ॒त्मने॑ नि॒र्याच्या॒स्या अ॒स्यै नि॒र्याच्या॒त्मने᳚ । \newline
34. नि॒र्याच्या॒त्मन॑ आ॒त्मने॑ नि॒र्याच्य॑ नि॒र्याच्या॒त्मने॒ ऽग्नि म॒ग्नि मा॒त्मने॑ नि॒र्याच्य॑ नि॒र्याच्या॒त्मने॒ ऽग्निम् । \newline
35. नि॒र्याच्येति॑ निः - याच्य॑ । \newline
36. आ॒त्मने॒ ऽग्नि म॒ग्नि मा॒त्मन॑ आ॒त्मने॒ ऽग्निम् चि॑नुते चिनुते॒ ऽग्नि मा॒त्मन॑ आ॒त्मने॒ ऽग्निम् चि॑नुते । \newline
37. अ॒ग्निम् चि॑नुते चिनुते॒ ऽग्नि म॒ग्निम् चि॑नुत इष्व॒ग्रेणे᳚ ष्व॒ग्रेण॑ चिनुते॒ ऽग्नि म॒ग्निम् चि॑नुत इष्व॒ग्रेण॑ । \newline
38. चि॒नु॒त॒ इ॒ष्व॒ग्रेणे᳚ ष्व॒ग्रेण॑ चिनुते चिनुत इष्व॒ग्रेण॒ वै वा इ॑ष्व॒ग्रेण॑ चिनुते चिनुत इष्व॒ग्रेण॒ वै । \newline
39. इ॒ष्व॒ग्रेण॒ वै वा इ॑ष्व॒ग्रेणे᳚ ष्व॒ग्रेण॒ वा अ॒स्या अ॒स्या वा इ॑ष्व॒ग्रेणे᳚ ष्व॒ग्रेण॒ वा अ॒स्याः । \newline
40. इ॒ष्व॒ग्रेणेती॑षु - अ॒ग्रेण॑ । \newline
41. वा अ॒स्या अ॒स्या वै वा अ॒स्या अना॑मृत॒ मना॑मृत म॒स्या वै वा अ॒स्या अना॑मृतम् । \newline
42. अ॒स्या अना॑मृत॒ मना॑मृत म॒स्या अ॒स्या अना॑मृत मि॒च्छन्त॑ इ॒च्छन्तो ऽना॑मृत म॒स्या अ॒स्या अना॑मृत मि॒च्छन्तः॑ । \newline
43. अना॑मृत मि॒च्छन्त॑ इ॒च्छन्तो ऽना॑मृत॒ मना॑मृत मि॒च्छन्तो॒ न नेच्छन्तो ऽना॑मृत॒ मना॑मृत मि॒च्छन्तो॒ न । \newline
44. अना॑मृत॒मित्यना᳚ - मृ॒त॒म् । \newline
45. इ॒च्छन्तो॒ न नेच्छन्त॑ इ॒च्छन्तो॒ नावि॑न्दन् नविन्द॒न् नेच्छन्त॑ इ॒च्छन्तो॒ नावि॑न्दन्न् । \newline
46. नावि॑न्दन् नविन्द॒न् न नावि॑न्द॒न् ते ते॑ ऽविन्द॒न् न नावि॑न्द॒न् ते । \newline
47. अ॒वि॒न्द॒न् ते ते॑ ऽविन्दन् नविन्द॒न् ते दे॒वा दे॒वा स्ते॑ ऽविन्दन् नविन्द॒न् ते दे॒वाः । \newline
48. ते दे॒वा दे॒वा स्ते ते दे॒वा ए॒त दे॒तद् दे॒वा स्ते ते दे॒वा ए॒तत् । \newline
49. दे॒वा ए॒त दे॒तद् दे॒वा दे॒वा ए॒तद् यजु॒र् यजु॑ रे॒तद् दे॒वा दे॒वा ए॒तद् यजुः॑ । \newline
50. ए॒तद् यजु॒र् यजु॑ रे॒त दे॒तद् यजु॑ रपश्यन् नपश्य॒न्॒. यजु॑ रे॒त दे॒तद् यजु॑ रपश्यन्न् । \newline
51. यजु॑ रपश्यन् नपश्य॒न्॒. यजु॒र् यजु॑ रपश्य॒न् नपापा॑ पश्य॒न्॒. यजु॒र् यजु॑ रपश्य॒न् नप॑ । \newline
52. अ॒प॒श्य॒न् नपापा॑ पश्यन् नपश्य॒न् नपे॑ते॒ तापा॑ पश्यन् नपश्य॒न् नपे॑त । \newline
53. अपे॑ते॒ तापापे॒ते तीती॒ता पापे॒ तेति॑ । \newline
54. इ॒ते तीती॑ते॒ तेति॒ यद् यदिती॑ते॒ तेति॒ यत् । \newline
55. इति॒ यद् यदितीति॒ यदे॒ते नै॒तेन॒ यदितीति॒ यदे॒तेन॑ । \newline
56. यदे॒ते नै॒तेन॒ यद् यदे॒तेना᳚द्ध्यवसा॒यय॑ त्यद्ध्यवसा॒यय॑ त्ये॒तेन॒ यद् यदे॒तेना᳚द्ध्यवसा॒यय॑ति । \newline
57. ए॒तेना᳚ द्ध्यवसा॒यय॑ त्यद्ध्यवसा॒यय॑ त्ये॒तेनै॒तेना᳚ द्ध्यवसा॒यय॒ त्यना॑मृ॒ते ऽना॑मृते ऽद्ध्यवसा॒यय॑ त्ये॒तेनै॒तेना᳚ द्ध्यवसा॒यय॒ त्यना॑मृते । \newline
58. अ॒द्ध्य॒व॒सा॒यय॒ त्यना॑मृ॒ते ऽना॑मृते ऽद्ध्यवसा॒यय॑ त्यद्ध्यवसा॒यय॒ त्यना॑मृत ए॒वैवाना॑मृते ऽद्ध्यवसा॒यय॑ त्यद्ध्यवसा॒यय॒ त्यना॑मृत ए॒व । \newline
59. अ॒द्ध्य॒व॒सा॒यय॒तीत्य॑धि - अ॒व॒सा॒यय॑ति । \newline
\pagebreak
\markright{ TS 5.2.3.2  \hfill https://www.vedavms.in \hfill}

\section{ TS 5.2.3.2 }

\textbf{TS 5.2.3.2 } \newline
\textbf{Samhita Paata} \newline

-ना॑मृत ए॒वाग्निं चि॑नुत॒ उद्ध॑न्ति॒ यदे॒वास्या॑ अमे॒द्ध्यं तदप॑ हन्त्य॒पोऽवो᳚क्षति॒ शान्त्यै॒ सिक॑ता॒ नि व॑पत्ये॒तद्वा अ॒ग्नेर्वै᳚श्वान॒रस्य॑ रू॒पꣳ रू॒पेणै॒व वै᳚श्वान॒रमव॑ रुन्ध॒ ऊषा॒न् निव॑पति॒ पुष्टि॒र्वा ए॒षा प्र॒जन॑नं॒ ॅयदूषाः॒ पुष्ट्या॑मे॒व प्र॒जन॑ने॒ऽग्निं चि॑नु॒तेऽथो॑ स॒ज्ञांन॑ ए॒व स॒ज्ञांनꣳ॒॒ ह्ये॑तत् - [  ] \newline

\textbf{Pada Paata} \newline

अना॑मृत॒ इत्यना᳚ - मृ॒ते॒ । ए॒व । अ॒ग्निम् । चि॒नु॒ते॒ । उदिति॑ । ह॒न्ति॒ । यत् । ए॒व । अ॒स्याः॒ । अ॒मे॒द्ध्यम् । तत् । अपेति॑ । ह॒न्ति॒ । अ॒पः । अवेति॑ । उ॒क्ष॒ति॒ । शान्त्यै᳚ । सिक॑ताः । नीति॑ । व॒प॒ति॒ । ए॒तत् । वै । अ॒ग्नेः । वै॒श्वा॒न॒रस्य॑ । रू॒पम् । रू॒पेण॑ । ए॒व । वै॒श्वा॒न॒रम् । अवेति॑ । रु॒न्धे॒ । ऊषान्॑ । नीति॑ । व॒प॒ति॒ । पुष्टिः॑ । वै । ए॒षा । प्र॒जन॑न॒मिति॑ प्र - जन॑नम् । यत् । ऊषाः᳚ । पुष्ट्या᳚म् । ए॒व । प्र॒जन॑न॒ इति॑ प्र - जन॑ने । अ॒ग्निम् । चि॒नु॒ते॒ । अथो॒ इति॑ । स॒ज्ञांन॒ इति॑ सं - ज्ञाने᳚ । ए॒व । स॒ज्ञांन॒मिति॑ सं-ज्ञान᳚म् । हि । ए॒तत् ।  \newline


\textbf{Krama Paata} \newline

अना॑मृत ए॒व । अना॑मृत॒ इत्यना᳚ - मृ॒ते॒ । ए॒वाग्निम् । अ॒ग्निम् चि॑नुते । चि॒नु॒त॒ उत् । उद्ध॑न्ति । ह॒न्ति॒ यत् । 
यदे॒व । ए॒वास्याः᳚ । अ॒स्या॒ अ॒मे॒द्ध्यम् । अ॒मे॒द्ध्यम् तत् । तदप॑ । अप॑ हन्ति । ह॒न्त्य॒पः । अ॒पोऽव॑ । 
अवो᳚क्षति । उ॒क्ष॒ति॒ शान्त्यै᳚ । शान्त्यै॒ सिक॑ताः । सिक॑ता॒ नि । नि व॑पति । व॒प॒त्ये॒तत् । ए॒तद् वै । वा अ॒ग्नेः । अ॒ग्नेर् वै᳚श्वान॒रस्य॑ । वै॒श्वा॒न॒रस्य॑ रू॒पम् । रू॒पꣳ रू॒पेण॑ । रू॒पेणै॒व । ए॒व वै᳚श्वान॒रम् । वै॒श्वा॒न॒रमव॑ । अव॑ रुन्धे । रु॒न्ध॒ ऊषान्॑ । ऊषा॒न् नि । नि व॑पति । व॒प॒ति॒ पुष्टिः॑ । पुष्टि॒र् वै । वा ए॒षा । ए॒षा प्र॒जन॑नम् । प्र॒जन॑न॒म् ॅयत् । प्र॒जन॑न॒मिति॑ प्र - जन॑नम् । यदूषाः᳚ । ऊषाः॒ पुष्ट्या᳚म् । पुष्ट्या॑मे॒व । ए॒व प्र॒जन॑ने । प्र॒जन॑ने॒ऽग्निम् । प्र॒जन॑न॒ इति॑ प्र - जन॑ने । अ॒ग्निम् चि॑नुते । चि॒नु॒तेऽथो᳚ । अथो॑ स॒म्(2)ज्ञाने᳚ । अथो॒ इत्यथो᳚ । स॒म्(2)ज्ञान॑ ए॒व । स॒म्(2)ज्ञान॒ इति॑ सम्. - ज्ञाने᳚ । ए॒व स॒म्(2)ज्ञान᳚म् । स॒म्(2)ज्ञानꣳ॒॒ हि । स॒म्(2)ज्ञान॒मिति॑ सम्. - ज्ञान᳚म् । ह्ये॑तत् । ए॒तत् प॑शू॒नाम् \newline

\textbf{Jatai Paata} \newline

1. अना॑मृत ए॒वैवा ना॑मृ॒ते ऽना॑मृत ए॒व । \newline
2. अना॑मृत॒ इत्यना᳚ - मृ॒ते॒ । \newline
3. ए॒वाग्नि म॒ग्नि मे॒वै वाग्निम् । \newline
4. अ॒ग्निम् चि॑नुते चिनुते॒ ऽग्नि म॒ग्निम् चि॑नुते । \newline
5. चि॒नु॒त॒ उदुच् चि॑नुते चिनुत॒ उत् । \newline
6. उद्ध॑न्ति ह॒न्त्यु दुद्ध॑न्ति । \newline
7. ह॒न्ति॒ यद् यद्ध॑न्ति हन्ति॒ यत् । \newline
8. यदे॒ वैव यद् यदे॒व । \newline
9. ए॒वास्या॑ अस्या ए॒वै वास्याः᳚ । \newline
10. अ॒स्या॒ अ॒मे॒द्ध्य म॑मे॒द्ध्य म॑स्या अस्या अमे॒द्ध्यम् । \newline
11. अ॒मे॒द्ध्यम् तत् तद॑मे॒द्ध्य म॑मे॒द्ध्यम् तत् । \newline
12. तद पाप॒ तत् तदप॑ । \newline
13. अप॑ हन्ति ह॒न्त्यपाप॑ हन्ति । \newline
14. ह॒न्त्य॒पो॑ ऽपो ह॑न्ति हन्त्य॒पः । \newline
15. अ॒पो ऽवावा॒पो॑ ऽपो ऽव॑ । \newline
16. अवो᳚क्ष त्युक्ष॒ त्यवा वो᳚क्षति । \newline
17. उ॒क्ष॒ति॒ शान्त्यै॒ शान्त्या॑ उक्ष त्युक्षति॒ शान्त्यै᳚ । \newline
18. शान्त्यै॒ सिक॑ताः॒ सिक॑ताः॒ शान्त्यै॒ शान्त्यै॒ सिक॑ताः । \newline
19. सिक॑ता॒ नि नि सिक॑ताः॒ सिक॑ता॒ नि । \newline
20. नि व॑पति वपति॒ नि नि व॑पति । \newline
21. व॒प॒ त्ये॒त दे॒तद् व॑पति वप त्ये॒तत् । \newline
22. ए॒तद् वै वा ए॒त दे॒तद् वै । \newline
23. वा अ॒ग्ने र॒ग्नेर् वै वा अ॒ग्नेः । \newline
24. अ॒ग्नेर् वै᳚श्वान॒रस्य॑ वैश्वान॒र स्या॒ग्ने र॒ग्नेर् वै᳚श्वान॒रस्य॑ । \newline
25. वै॒श्वा॒न॒रस्य॑ रू॒पꣳ रू॒पं ॅवै᳚श्वान॒रस्य॑ वैश्वान॒रस्य॑ रू॒पम् । \newline
26. रू॒पꣳ रू॒पेण॑ रू॒पेण॑ रू॒पꣳ रू॒पꣳ रू॒पेण॑ । \newline
27. रू॒पे णै॒वैव रू॒पेण॑ रू॒पेणै॒व । \newline
28. ए॒व वै᳚श्वान॒रं ॅवै᳚श्वान॒र मे॒वैव वै᳚श्वान॒रम् । \newline
29. वै॒श्वा॒न॒र मवाव॑ वैश्वान॒रं ॅवै᳚श्वान॒र मव॑ । \newline
30. अव॑ रुन्धे रु॒न्धे ऽवाव॑ रुन्धे । \newline
31. रु॒न्ध॒ ऊषा॒ नूषा᳚न् रुन्धे रुन्ध॒ ऊषान्॑ । \newline
32. ऊषा॒न् नि न्यूषा॒ नूषा॒न् नि । \newline
33. नि व॑पति वपति॒ नि नि व॑पति । \newline
34. व॒प॒ति॒ पुष्टिः॒ पुष्टि॑र् वपति वपति॒ पुष्टिः॑ । \newline
35. पुष्टि॒र् वै वै पुष्टिः॒ पुष्टि॒र् वै । \newline
36. वा ए॒षैषा वै वा ए॒षा । \newline
37. ए॒षा प्र॒जन॑नम् प्र॒जन॑न मे॒षैषा प्र॒जन॑नम् । \newline
38. प्र॒जन॑नं॒ ॅयद् यत् प्र॒जन॑नम् प्र॒जन॑नं॒ ॅयत् । \newline
39. प्र॒जन॑न॒मिति॑ प्र - जन॑नम् । \newline
40. यदूषा॒ ऊषा॒ यद् यदूषाः᳚ । \newline
41. ऊषाः॒ पुष्ट्या॒म् पुष्ट्या॒ मूषा॒ ऊषाः॒ पुष्ट्या᳚म् । \newline
42. पुष्ट्या॑ मे॒वैव पुष्ट्या॒म् पुष्ट्या॑ मे॒व । \newline
43. ए॒व प्र॒जन॑ने प्र॒जन॑न ए॒वैव प्र॒जन॑ने । \newline
44. प्र॒जन॑ने॒ ऽग्नि म॒ग्निम् प्र॒जन॑ने प्र॒जन॑ने॒ ऽग्निम् । \newline
45. प्र॒जन॑न॒ इति॑ प्र - जन॑ने । \newline
46. अ॒ग्निम् चि॑नुते चिनुते॒ ऽग्नि म॒ग्निम् चि॑नुते । \newline
47. चि॒नु॒ते ऽथो॒ अथो॑ चिनुते चिनु॒ते ऽथो᳚ । \newline
48. अथो॑ सं॒.ज्ञाने॑ सं॒.ज्ञाने ऽथो॒ अथो॑ सं॒.ज्ञाने᳚ । \newline
49. अथो॒ इत्यथो᳚ । \newline
50. सं॒.ज्ञान॑ ए॒वैव सं॒.ज्ञाने॑ सं॒.ज्ञान॑ ए॒व । \newline
51. सं॒.ज्ञान॒ इति॑ सं - ज्ञाने᳚ । \newline
52. ए॒व सं॒.ज्ञानꣳ॑ सं॒.ज्ञान॑ मे॒वैव सं॒.ज्ञान᳚म् । \newline
53. सं॒.ज्ञानꣳ॒॒ हि हि सं॒.ज्ञानꣳ॑ सं॒.ज्ञानꣳ॒॒ हि । \newline
54. सं॒.ज्ञान॒मिति॑ सं - ज्ञान᳚म् । \newline
55. ह्ये॑त दे॒तद्धि ह्ये॑तत् । \newline
56. ए॒तत् प॑शू॒नाम् प॑शू॒ना मे॒त दे॒तत् प॑शू॒नाम् । \newline

\textbf{Ghana Paata } \newline

1. अना॑मृत ए॒वैवाना॑मृ॒ते ऽना॑मृत ए॒वाग्नि म॒ग्नि मे॒वाना॑मृ॒ते ऽना॑मृत ए॒वाग्निम् । \newline
2. अना॑मृत॒ इत्यना᳚ - मृ॒ते॒ । \newline
3. ए॒वाग्नि म॒ग्नि मे॒वैवाग्निम् चि॑नुते चिनुते॒ ऽग्नि मे॒वैवाग्निम् चि॑नुते । \newline
4. अ॒ग्निम् चि॑नुते चिनुते॒ ऽग्नि म॒ग्निम् चि॑नुत॒ उदुच् चि॑नुते॒ ऽग्नि म॒ग्निम् चि॑नुत॒ उत् । \newline
5. चि॒नु॒त॒ उदुच् चि॑नुते चिनुत॒ उद्ध॑न्ति ह॒न्त्युच् चि॑नुते चिनुत॒ उद्ध॑न्ति । \newline
6. उद्ध॑न्ति ह॒न्त्यु दु द्ध॑न्ति॒ यद् यद्ध॒न्त्यु दु द्ध॑न्ति॒ यत् । \newline
7. ह॒न्ति॒ यद् यद्ध॑न्ति हन्ति॒ यदे॒वैव यद्ध॑न्ति हन्ति॒ यदे॒व । \newline
8. यदे॒वैव यद् यदे॒वास्या॑ अस्या ए॒व यद् यदे॒वास्याः᳚ । \newline
9. ए॒वास्या॑ अस्या ए॒वैवास्या॑ अमे॒द्ध्य म॑मे॒द्ध्य म॑स्या ए॒वैवास्या॑ अमे॒द्ध्यम् । \newline
10. अ॒स्या॒ अ॒मे॒द्ध्य म॑मे॒द्ध्य म॑स्या अस्या अमे॒द्ध्यम् तत् तद॑मे॒द्ध्य म॑स्या अस्या अमे॒द्ध्यम् तत् । \newline
11. अ॒मे॒द्ध्यम् तत् तद॑मे॒द्ध्य म॑मे॒द्ध्यम् तदपाप॒ तद॑मे॒द्ध्य म॑मे॒द्ध्यम् तदप॑ । \newline
12. तदपाप॒ तत् तदप॑ हन्ति ह॒न्त्यप॒ तत् तदप॑ हन्ति । \newline
13. अप॑ हन्ति ह॒न्त्यपाप॑ हन्त्य॒पो॑ ऽपो ह॒न्त्यपाप॑ हन्त्य॒पः । \newline
14. ह॒न्त्य॒पो॑ ऽपो ह॑न्ति हन्त्य॒पो ऽवावा॒पो ह॑न्ति हन्त्य॒पो ऽव॑ । \newline
15. अ॒पो ऽवावा॒पो॑ ऽपो ऽवो᳚क्ष त्युक्ष॒ त्यवा॒पो॑ ऽपो ऽवो᳚क्षति । \newline
16. अवो᳚क्ष त्युक्ष॒ त्यवावो᳚क्षति॒ शान्त्यै॒ शान्त्या॑ उक्ष॒ त्यवावो᳚क्षति॒ शान्त्यै᳚ । \newline
17. उ॒क्ष॒ति॒ शान्त्यै॒ शान्त्या॑ उक्ष त्युक्षति॒ शान्त्यै॒ सिक॑ताः॒ सिक॑ताः॒ शान्त्या॑ उक्ष त्युक्षति॒ शान्त्यै॒ सिक॑ताः । \newline
18. शान्त्यै॒ सिक॑ताः॒ सिक॑ताः॒ शान्त्यै॒ शान्त्यै॒ सिक॑ता॒ नि नि सिक॑ताः॒ शान्त्यै॒ शान्त्यै॒ सिक॑ता॒ नि । \newline
19. सिक॑ता॒ नि नि सिक॑ताः॒ सिक॑ता॒ नि व॑पति वपति॒ नि सिक॑ताः॒ सिक॑ता॒ नि व॑पति । \newline
20. नि व॑पति वपति॒ नि नि व॑प त्ये॒त दे॒तद् व॑पति॒ नि नि व॑प त्ये॒तत् । \newline
21. व॒प॒ त्ये॒त दे॒तद् व॑पति वप त्ये॒तद् वै वा ए॒तद् व॑पति वप त्ये॒तद् वै । \newline
22. ए॒तद् वै वा ए॒त दे॒तद् वा अ॒ग्ने र॒ग्नेर् वा ए॒त दे॒तद् वा अ॒ग्नेः । \newline
23. वा अ॒ग्ने र॒ग्नेर् वै वा अ॒ग्नेर् वै᳚श्वान॒रस्य॑ वैश्वान॒रस्या॒ग्नेर् वै वा अ॒ग्नेर् वै᳚श्वान॒रस्य॑ । \newline
24. अ॒ग्नेर् वै᳚श्वान॒रस्य॑ वैश्वान॒रस्या॒ ग्ने र॒ग्नेर् वै᳚श्वान॒रस्य॑ रू॒पꣳ रू॒पं ॅवै᳚श्वान॒रस्या॒ ग्ने र॒ग्नेर् वै᳚श्वान॒रस्य॑ रू॒पम् । \newline
25. वै॒श्वा॒न॒रस्य॑ रू॒पꣳ रू॒पं ॅवै᳚श्वान॒रस्य॑ वैश्वान॒रस्य॑ रू॒पꣳ रू॒पेण॑ रू॒पेण॑ रू॒पं ॅवै᳚श्वान॒रस्य॑ वैश्वान॒रस्य॑ रू॒पꣳ रू॒पेण॑ । \newline
26. रू॒पꣳ रू॒पेण॑ रू॒पेण॑ रू॒पꣳ रू॒पꣳ रू॒पेणै॒वैव रू॒पेण॑ रू॒पꣳ रू॒पꣳ रू॒पेणै॒व । \newline
27. रू॒पेणै॒वैव रू॒पेण॑ रू॒पेणै॒व वै᳚श्वान॒रं ॅवै᳚श्वान॒र मे॒व रू॒पेण॑ रू॒पेणै॒व वै᳚श्वान॒रम् । \newline
28. ए॒व वै᳚श्वान॒रं ॅवै᳚श्वान॒र मे॒वैव वै᳚श्वान॒र मवाव॑ वैश्वान॒र मे॒वैव वै᳚श्वान॒र मव॑ । \newline
29. वै॒श्वा॒न॒र मवाव॑ वैश्वान॒रं ॅवै᳚श्वान॒र मव॑ रुन्धे रु॒न्धे ऽव॑ वैश्वान॒रं ॅवै᳚श्वान॒र मव॑ रुन्धे । \newline
30. अव॑ रुन्धे रु॒न्धे ऽवाव॑ रुन्ध॒ ऊषा॒ नूषा᳚न् रु॒न्धे ऽवाव॑ रुन्ध॒ ऊषान्॑ । \newline
31. रु॒न्ध॒ ऊषा॒ नूषा᳚न् रुन्धे रुन्ध॒ ऊषा॒न् नि न्यूषा᳚न् रुन्धे रुन्ध॒ ऊषा॒न् नि । \newline
32. ऊषा॒न् नि न्यूषा॒ नूषा॒न् नि व॑पति वपति॒ न्यूषा॒ नूषा॒न् नि व॑पति । \newline
33. नि व॑पति वपति॒ नि नि व॑पति॒ पुष्टिः॒ पुष्टि॑र् वपति॒ नि नि व॑पति॒ पुष्टिः॑ । \newline
34. व॒प॒ति॒ पुष्टिः॒ पुष्टि॑र् वपति वपति॒ पुष्टि॒र् वै वै पुष्टि॑र् वपति वपति॒ पुष्टि॒र् वै । \newline
35. पुष्टि॒र् वै वै पुष्टिः॒ पुष्टि॒र् वा ए॒षैषा वै पुष्टिः॒ पुष्टि॒र् वा ए॒षा । \newline
36. वा ए॒षैषा वै वा ए॒षा प्र॒जन॑नम् प्र॒जन॑न मे॒षा वै वा ए॒षा प्र॒जन॑नम् । \newline
37. ए॒षा प्र॒जन॑नम् प्र॒जन॑न मे॒षैषा प्र॒जन॑नं॒ ॅयद् यत् प्र॒जन॑न मे॒षैषा प्र॒जन॑नं॒ ॅयत् । \newline
38. प्र॒जन॑नं॒ ॅयद् यत् प्र॒जन॑नम् प्र॒जन॑नं॒ ॅयदूषा॒ ऊषा॒ यत् प्र॒जन॑नम् प्र॒जन॑नं॒ ॅयदूषाः᳚ । \newline
39. प्र॒जन॑न॒मिति॑ प्र - जन॑नम् । \newline
40. यदूषा॒ ऊषा॒ यद् यदूषाः॒ पुष्ट्या॒म् पुष्ट्या॒ मूषा॒ यद् यदूषाः॒ पुष्ट्या᳚म् । \newline
41. ऊषाः॒ पुष्ट्या॒म् पुष्ट्या॒ मूषा॒ ऊषाः॒ पुष्ट्या॑ मे॒वैव पुष्ट्या॒ मूषा॒ ऊषाः॒ पुष्ट्या॑ मे॒व । \newline
42. पुष्ट्या॑ मे॒वैव पुष्ट्या॒म् पुष्ट्या॑ मे॒व प्र॒जन॑ने प्र॒जन॑न ए॒व पुष्ट्या॒म् पुष्ट्या॑ मे॒व प्र॒जन॑ने । \newline
43. ए॒व प्र॒जन॑ने प्र॒जन॑न ए॒वैव प्र॒जन॑ने॒ ऽग्नि म॒ग्निम् प्र॒जन॑न ए॒वैव प्र॒जन॑ने॒ ऽग्निम् । \newline
44. प्र॒जन॑ने॒ ऽग्नि म॒ग्निम् प्र॒जन॑ने प्र॒जन॑ने॒ ऽग्निम् चि॑नुते चिनुते॒ ऽग्निम् प्र॒जन॑ने प्र॒जन॑ने॒ ऽग्निम् चि॑नुते । \newline
45. प्र॒जन॑न॒ इति॑ प्र - जन॑ने । \newline
46. अ॒ग्निम् चि॑नुते चिनुते॒ ऽग्नि म॒ग्निम् चि॑नु॒ते ऽथो॒ अथो॑ चिनुते॒ ऽग्नि म॒ग्निम् चि॑नु॒ते ऽथो᳚ । \newline
47. चि॒नु॒ते ऽथो॒ अथो॑ चिनुते चिनु॒ते ऽथो॑ सं॒.ज्ञाने॑ सं॒.ज्ञाने ऽथो॑ चिनुते चिनु॒ते ऽथो॑ सं॒.ज्ञाने᳚ । \newline
48. अथो॑ सं॒.ज्ञाने॑ सं॒.ज्ञाने ऽथो॒ अथो॑ सं॒.ज्ञान॑ ए॒वैव सं॒.ज्ञाने ऽथो॒ अथो॑ सं॒.ज्ञान॑ ए॒व । \newline
49. अथो॒ इत्यथो᳚ । \newline
50. सं॒.ज्ञान॑ ए॒वैव सं॒.ज्ञाने॑ सं॒.ज्ञान॑ ए॒व सं॒.ज्ञानꣳ॑ सं॒.ज्ञान॑ मे॒व सं॒.ज्ञाने॑ सं॒.ज्ञान॑ ए॒व सं॒.ज्ञान᳚म् । \newline
51. सं॒.ज्ञान॒ इति॑ सं - ज्ञाने᳚ । \newline
52. ए॒व सं॒.ज्ञानꣳ॑ सं॒.ज्ञान॑ मे॒वैव सं॒.ज्ञानꣳ॒॒ हि हि सं॒.ज्ञान॑ मे॒वैव सं॒.ज्ञानꣳ॒॒ हि । \newline
53. सं॒.ज्ञानꣳ॒॒ हि हि सं॒.ज्ञानꣳ॑ सं॒.ज्ञानꣳ॒॒ ह्ये॑त दे॒तद्धि सं॒.ज्ञानꣳ॑ सं॒.ज्ञानꣳ॒॒ ह्ये॑तत् । \newline
54. सं॒.ज्ञान॒मिति॑ सं - ज्ञान᳚म् । \newline
55. ह्ये॑त दे॒तद्धि ह्ये॑तत् प॑शू॒नाम् प॑शू॒ना मे॒तद्धि ह्ये॑तत् प॑शू॒नाम् । \newline
56. ए॒तत् प॑शू॒नाम् प॑शू॒ना मे॒त दे॒तत् प॑शू॒नां ॅयद् यत् प॑शू॒ना मे॒त दे॒तत् प॑शू॒नां ॅयत् । \newline
\pagebreak
\markright{ TS 5.2.3.3  \hfill https://www.vedavms.in \hfill}

\section{ TS 5.2.3.3 }

\textbf{TS 5.2.3.3 } \newline
\textbf{Samhita Paata} \newline

प॑शू॒नां ॅयदूषा॒ द्यावा॑पृथि॒वी स॒हाऽऽस्तां॒ ते वि॑य॒ती अ॑ब्रूता॒मस्त्वे॒व नौ॑ स॒ह य॒ज्ञिय॒मिति॒ यद॒मुष्या॑ य॒ज्ञिय॒मासी॒त् तद॒स्याम॑दधा॒त् त ऊषा॑ अभव॒न्॒ यद॒स्या य॒ज्ञिय॒मासी॒त् तद॒मुष्या॑मदधा॒त् तद॒दश्च॒न्द्रम॑सि कृ॒ष्णमूषा᳚न् नि॒वप॑न्न॒दो ध्या॑ये॒द् द्यावा॑पृथि॒व्योरे॒व य॒ज्ञिये॒ऽग्निं चि॑नुते॒ ऽयꣳ सो अ॒ग्निरिति॑ वि॒श्वामि॑त्रस्य - [  ] \newline

\textbf{Pada Paata} \newline

प॒शू॒नाम् । यत् । ऊषाः᳚ । द्यावा॑पृथि॒वी इति॒ द्यावा᳚ - पृ॒थि॒वी । स॒ह । आ॒स्ता॒म् । ते इति॑ । वि॒य॒ती इति॑ वि - य॒ती । अ॒ब्रू॒ता॒म् । अस्तु॑ । ए॒व । नौ॒ । स॒ह । य॒ज्ञिय᳚म् । इति॑ । यत् । अ॒मुष्याः᳚ । य॒ज्ञिय᳚म् । आसी᳚त् । तत् । अ॒स्याम् । अ॒द॒धा॒त् । ते । ऊषाः᳚ । अ॒भ॒व॒न्न् । यत् । अ॒स्याः । य॒ज्ञिय᳚म् । आसी᳚त् । तत् । अ॒मुष्या᳚म् । अ॒द॒धा॒त् । तत् । अ॒दः । च॒न्द्रम॑सि । कृ॒ष्णम् । ऊषान्॑ । नि॒वप॒न्निति॑ नि-वपन्न्॑ । अ॒दः । ध्या॒ये॒त् । द्यावा॑पृथि॒व्योरिति॒ द्यावा᳚ - पृ॒थि॒व्योः । ए॒व । य॒ज्ञिये᳚ । अ॒ग्निम् । चि॒नु॒ते॒ । अ॒यम् । सः । अ॒ग्निः । इति॑ । वि॒श्वामि॑त्र॒स्येति॑ वि॒श्व - मि॒त्र॒स्य॒ ।  \newline


\textbf{Krama Paata} \newline

प॒शू॒नाम् ॅयत् । यदूषाः᳚ । ऊषा॒ द्यावा॑पृथि॒वी । द्यावा॑पृथि॒वी स॒ह । द्यावा॑पृथि॒वी इति॒ द्यावा᳚ - पृ॒थि॒वी । स॒हास्ता᳚म् । आ॒स्ता॒म् ते । ते वि॑य॒ती । ते इति॒ ते । वि॒य॒ती अ॑ब्रूताम् । वि॒य॒ती इति॑ वि - य॒ती । अ॒ब्रू॒ता॒मस्तु॑ । अस्त्वे॒व । ए॒व नौ᳚ । नौ॒ स॒ह । स॒ह य॒ज्ञिय᳚म् । य॒ज्ञिय॒मिति॑ । इति॒ यत् । यद॒मुष्याः᳚ । अ॒मुष्या॑ य॒ज्ञिय᳚म् । य॒ज्ञिय॒मासी᳚त् । आसी॒त् तत् । तद॒स्याम् । अ॒स्याम॑दधात् । अ॒द॒धा॒त् ते । त ऊषाः᳚ । ऊषा॑ अभवन्न् । अ॒भ॒व॒न्॒. यत् । यद॒स्याः । अ॒स्या य॒ज्ञिय᳚म् । य॒ज्ञिय॒मासी᳚त् । आसी॒त् तत् । तद॒मुष्या᳚म् । अ॒मुष्या॑मदधात् । अ॒द॒धा॒त् तत् । तद॒दः । अ॒दश्च॒न्द्रम॑सि । च॒न्द्रम॑सि कृ॒ष्णम् । कृ॒ष्णमूषान्॑ । ऊषा᳚न् नि॒वपन्न्॑ । नि॒वप॑न्न॒दः । नि॒वप॒न्निति॑ नि - वपन्न्॑ । अ॒दो ध्या॑येत् । ध्या॒ये॒द् द्यावा॑पृथि॒व्योः । द्यावा॑पृथि॒व्योरे॒व । द्यावा॑पृथि॒व्योरिति॒ द्यावा᳚ - पृ॒थि॒व्योः । ए॒व य॒ज्ञिये᳚ । य॒ज्ञिये॒ऽग्निम् । अ॒ग्निम् चि॑नुते । चि॒नु॒ते॒ऽयम् । अ॒यꣳ सः । सो अ॒ग्निः । अ॒ग्निरिति॑ । इति॑ वि॒श्वामि॑त्रस्य । वि॒श्वा॑मित्रस्य सू॒क्तम् । वि॒श्वामि॑त्र॒स्येति॑ वि॒श्व - मि॒त्र॒स्य॒ \newline

\textbf{Jatai Paata} \newline

1. प॒शू॒नां ॅयद् यत् प॑शू॒नाम् प॑शू॒नां ॅयत् । \newline
2. यदूषा॒ ऊषा॒ यद् यदूषाः᳚ । \newline
3. ऊषा॒ द्यावा॑पृथि॒वी द्यावा॑पृथि॒वी ऊषा॒ ऊषा॒ द्यावा॑पृथि॒वी । \newline
4. द्यावा॑पृथि॒वी स॒ह स॒ह द्यावा॑पृथि॒वी द्यावा॑पृथि॒वी स॒ह । \newline
5. द्यावा॑पृथि॒वी इति॒ द्यावा᳚ - पृ॒थि॒वी । \newline
6. स॒हास्ता॑ मास्ताꣳ स॒ह स॒हास्ता᳚म् । \newline
7. आ॒स्ता॒म् ते ते आ᳚स्ता मास्ता॒म् ते । \newline
8. ते वि॑य॒ती वि॑य॒ती ते ते वि॑य॒ती । \newline
9. ते इति॒ ते । \newline
10. वि॒य॒ती अ॑ब्रूता मब्रूतां ॅविय॒ती वि॑य॒ती अ॑ब्रूताम् । \newline
11. वि॒य॒ती इति॑ वि - य॒ती । \newline
12. अ॒ब्रू॒ता॒ मस्त्व स्त्व॑ब्रूता मब्रूता॒ मस्तु॑ । \newline
13. अस्त्वे॒ वैवा स्त्व स्त्वे॒व । \newline
14. ए॒व नौ॑ ना वे॒वैव नौ᳚ । \newline
15. नौ॒ स॒ह स॒ह नौ॑ नौ स॒ह । \newline
16. स॒ह य॒ज्ञियं॑ ॅय॒ज्ञियꣳ॑ स॒ह स॒ह य॒ज्ञिय᳚म् । \newline
17. य॒ज्ञिय॒ मितीति॑ य॒ज्ञियं॑ ॅय॒ज्ञिय॒ मिति॑ । \newline
18. इति॒ यद् यदितीति॒ यत् । \newline
19. यद॒मुष्या॑ अ॒मुष्या॒ यद् यद॒मुष्याः᳚ । \newline
20. अ॒मुष्या॑ य॒ज्ञियं॑ ॅय॒ज्ञिय॑ म॒मुष्या॑ अ॒मुष्या॑ य॒ज्ञिय᳚म् । \newline
21. य॒ज्ञिय॒ मासी॒ दासी᳚द् य॒ज्ञियं॑ ॅय॒ज्ञिय॒ मासी᳚त् । \newline
22. आसी॒त् तत् तदासी॒ दासी॒त् तत् । \newline
23. तद॒स्या म॒स्याम् तत् तद॒स्याम् । \newline
24. अ॒स्या म॑दधा ददधा द॒स्या म॒स्या म॑दधात् । \newline
25. अ॒द॒धा॒त् ते ते॑ ऽदधाद दधा॒त् ते । \newline
26. त ऊषा॒ ऊषा॒ स्ते त ऊषाः᳚ । \newline
27. ऊषा॑ अभवन् नभव॒न् नूषा॒ ऊषा॑ अभवन्न् । \newline
28. अ॒भ॒व॒न्॒. यद् यद॑भवन् नभव॒न्॒. यत् । \newline
29. यद॒स्या अ॒स्या यद् यद॒स्याः । \newline
30. अ॒स्या य॒ज्ञियं॑ ॅय॒ज्ञिय॑ म॒स्या अ॒स्या य॒ज्ञिय᳚म् । \newline
31. य॒ज्ञिय॒ मासी॒ दासी᳚द् य॒ज्ञियं॑ ॅय॒ज्ञिय॒ मासी᳚त् । \newline
32. आसी॒त् तत् तदासी॒ दासी॒त् तत् । \newline
33. तद॒मुष्या॑ म॒मुष्या॒म् तत् तद॒मुष्या᳚म् । \newline
34. अ॒मुष्या॑ मदधा ददधा द॒मुष्या॑ म॒मुष्या॑ मदधात् । \newline
35. अ॒द॒धा॒त् तत् तद॑दधा ददधा॒त् तत् । \newline
36. तद॒दो॑ ऽद स्तत् तद॒दः । \newline
37. अ॒द श्च॒न्द्रम॑सि च॒न्द्रम॑ स्य॒दो॑ ऽद श्च॒न्द्रम॑सि । \newline
38. च॒न्द्रम॑सि कृ॒ष्णम् कृ॒ष्णम् च॒न्द्रम॑सि च॒न्द्रम॑सि कृ॒ष्णम् । \newline
39. कृ॒ष्ण मूषा॒ नूषा᳚न् कृ॒ष्णम् कृ॒ष्ण मूषान्॑ । \newline
40. ऊषा᳚न् नि॒वप॑न् नि॒वप॒न् नूषा॒ नूषा᳚न् नि॒वपन्न्॑ । \newline
41. नि॒वप॑न् न॒दो॑ ऽदो नि॒वप॑न् नि॒वप॑न् न॒दः । \newline
42. नि॒वप॒न्निति॑ नि - वपन्न्॑ । \newline
43. अ॒दो ध्या॑येद् ध्याये द॒दो॑ ऽदो ध्या॑येत् । \newline
44. ध्या॒ये॒द् द्यावा॑पृथि॒व्योर् द्यावा॑पृथि॒व्योर् ध्या॑येद् ध्याये॒द् द्यावा॑पृथि॒व्योः । \newline
45. द्यावा॑पृथि॒व्यो रे॒वैव द्यावा॑पृथि॒व्योर् द्यावा॑पृथि॒व्यो रे॒व । \newline
46. द्यावा॑पृथि॒व्योरिति॒ द्यावा᳚ - पृ॒थि॒व्योः । \newline
47. ए॒व य॒ज्ञिये॑ य॒ज्ञिय॑ ए॒वैव य॒ज्ञिये᳚ । \newline
48. य॒ज्ञिये॒ ऽग्नि म॒ग्निं ॅय॒ज्ञिये॑ य॒ज्ञिये॒ ऽग्निम् । \newline
49. अ॒ग्निम् चि॑नुते चिनुते॒ ऽग्नि म॒ग्निम् चि॑नुते । \newline
50. चि॒नु॒ते॒ ऽय म॒यम् चि॑नुते चिनुते॒ ऽयम् । \newline
51. अ॒यꣳ स सो॑ ऽय म॒यꣳ सः । \newline
52. सो अ॒ग्नि र॒ग्निः स सो अ॒ग्निः । \newline
53. अ॒ग्नि रिती त्य॒ग्नि र॒ग्नि रिति॑ । \newline
54. इति॑ वि॒श्वामि॑त्रस्य वि॒श्वामि॑त्र॒ स्येतीति॑ वि॒श्वामि॑त्रस्य । \newline
55. वि॒श्वामि॑त्रस्य सू॒क्तꣳ सू॒क्तं ॅवि॒श्वामि॑त्रस्य वि॒श्वामि॑त्रस्य सू॒क्तम् । \newline
56. वि॒श्वामि॑त्र॒स्येति॑ वि॒श्व - मि॒त्र॒स्य॒ । \newline

\textbf{Ghana Paata } \newline

1. प॒शू॒नां ॅयद् यत् प॑शू॒नाम् प॑शू॒नां ॅयदूषा॒ ऊषा॒ यत् प॑शू॒नाम् प॑शू॒नां ॅयदूषाः᳚ । \newline
2. यदूषा॒ ऊषा॒ यद् यदूषा॒ द्यावा॑पृथि॒वी द्यावा॑पृथि॒वी ऊषा॒ यद् यदूषा॒ द्यावा॑पृथि॒वी । \newline
3. ऊषा॒ द्यावा॑पृथि॒वी द्यावा॑पृथि॒वी ऊषा॒ ऊषा॒ द्यावा॑पृथि॒वी स॒ह स॒ह द्यावा॑पृथि॒वी ऊषा॒ ऊषा॒ द्यावा॑पृथि॒वी स॒ह । \newline
4. द्यावा॑पृथि॒वी स॒ह स॒ह द्यावा॑पृथि॒वी द्यावा॑पृथि॒वी स॒हास्ता॑ मास्ताꣳ स॒ह द्यावा॑पृथि॒वी द्यावा॑पृथि॒वी स॒हास्ता᳚म् । \newline
5. द्यावा॑पृथि॒वी इति॒ द्यावा᳚ - पृ॒थि॒वी । \newline
6. स॒हास्ता॑ मास्ताꣳ स॒ह स॒हास्ता॒म् ते ते आ᳚स्ताꣳ स॒ह स॒हास्ता॒म् ते । \newline
7. आ॒स्ता॒म् ते ते आ᳚स्ता मास्ता॒म् ते वि॑य॒ती वि॑य॒ती ते आ᳚स्ता मास्ता॒म् ते वि॑य॒ती । \newline
8. ते वि॑य॒ती वि॑य॒ती ते ते वि॑य॒ती अ॑ब्रूता मब्रूतां ॅविय॒ती ते ते वि॑य॒ती अ॑ब्रूताम् । \newline
9. ते इति॒ ते । \newline
10. वि॒य॒ती अ॑ब्रूता मब्रूतां ॅविय॒ती वि॑य॒ती अ॑ब्रूता॒ मस्त्व स्त्व॑ब्रूतां ॅविय॒ती वि॑य॒ती अ॑ब्रूता॒ मस्तु॑ । \newline
11. वि॒य॒ती इति॑ वि - य॒ती । \newline
12. अ॒ब्रू॒ता॒ मस्त्व स्त्व॑ब्रूता मब्रूता॒ मस्त्वे॒ वैवा स्त्व॑ब्रूता मब्रूता॒ मस्त्वे॒व । \newline
13. अस्त्वे॒ वैवा स्त्व स्त्वे॒व नौ॑ ना वे॒वा स्त्व स्त्वे॒व नौ᳚ । \newline
14. ए॒व नौ॑ ना वे॒वैव नौ॑ स॒ह स॒ह ना॑ वे॒वैव नौ॑ स॒ह । \newline
15. नौ॒ स॒ह स॒ह नौ॑ नौ स॒ह य॒ज्ञियं॑ ॅय॒ज्ञियꣳ॑ स॒ह नौ॑ नौ स॒ह य॒ज्ञिय᳚म् । \newline
16. स॒ह य॒ज्ञियं॑ ॅय॒ज्ञियꣳ॑ स॒ह स॒ह य॒ज्ञिय॒ मितीति॑ य॒ज्ञियꣳ॑ स॒ह स॒ह य॒ज्ञिय॒ मिति॑ । \newline
17. य॒ज्ञिय॒ मितीति॑ य॒ज्ञियं॑ ॅय॒ज्ञिय॒ मिति॒ यद् यदिति॑ य॒ज्ञियं॑ ॅय॒ज्ञिय॒ मिति॒ यत् । \newline
18. इति॒ यद् यदितीति॒ यद॒मुष्या॑ अ॒मुष्या॒ यदितीति॒ यद॒मुष्याः᳚ । \newline
19. यद॒मुष्या॑ अ॒मुष्या॒ यद् यद॒मुष्या॑ य॒ज्ञियं॑ ॅय॒ज्ञिय॑ म॒मुष्या॒ यद् यद॒मुष्या॑ य॒ज्ञिय᳚म् । \newline
20. अ॒मुष्या॑ य॒ज्ञियं॑ ॅय॒ज्ञिय॑ म॒मुष्या॑ अ॒मुष्या॑ य॒ज्ञिय॒ मासी॒ दासी᳚द् य॒ज्ञिय॑ म॒मुष्या॑ अ॒मुष्या॑ य॒ज्ञिय॒ मासी᳚त् । \newline
21. य॒ज्ञिय॒ मासी॒ दासी᳚द् य॒ज्ञियं॑ ॅय॒ज्ञिय॒ मासी॒त् तत् तदासी᳚द् य॒ज्ञियं॑ ॅय॒ज्ञिय॒ मासी॒त् तत् । \newline
22. आसी॒त् तत् तदासी॒ दासी॒त् तद॒स्या म॒स्याम् तदासी॒ दासी॒त् तद॒स्याम् । \newline
23. तद॒स्या म॒स्याम् तत् तद॒स्या म॑दधा ददधा द॒स्याम् तत् तद॒स्या म॑दधात् । \newline
24. अ॒स्या म॑दधा ददधा द॒स्या म॒स्या म॑दधा॒त् ते ते॑ ऽदधा द॒स्या म॒स्या म॑दधा॒त् ते । \newline
25. अ॒द॒धा॒त् ते ते॑ ऽदधा ददधा॒त् त ऊषा॒ ऊषा॒ स्ते॑ ऽदधाद दधा॒त् त ऊषाः᳚ । \newline
26. त ऊषा॒ ऊषा॒ स्ते त ऊषा॑ अभवन् नभव॒न् नूषा॒ स्ते त ऊषा॑ अभवन्न् । \newline
27. ऊषा॑ अभवन् नभव॒न् नूषा॒ ऊषा॑ अभव॒न्॒. यद् यद॑भव॒न् नूषा॒ ऊषा॑ अभव॒न्॒. यत् । \newline
28. अ॒भ॒व॒न्॒. यद् यद॑भवन् नभव॒न्॒. यद॒स्या अ॒स्या यद॑भवन् नभव॒न्॒. यद॒स्याः । \newline
29. यद॒स्या अ॒स्या यद् यद॒स्या य॒ज्ञियं॑ ॅय॒ज्ञिय॑ म॒स्या यद् यद॒स्या य॒ज्ञिय᳚म् । \newline
30. अ॒स्या य॒ज्ञियं॑ ॅय॒ज्ञिय॑ म॒स्या अ॒स्या य॒ज्ञिय॒ मासी॒ दासी᳚द् य॒ज्ञिय॑ म॒स्या अ॒स्या य॒ज्ञिय॒ मासी᳚त् । \newline
31. य॒ज्ञिय॒ मासी॒ दासी᳚द् य॒ज्ञियं॑ ॅय॒ज्ञिय॒ मासी॒त् तत् तदासी᳚द् य॒ज्ञियं॑ ॅय॒ज्ञिय॒ मासी॒त् तत् । \newline
32. आसी॒त् तत् तदासी॒ दासी॒त् तद॒मुष्या॑ म॒मुष्या॒म् तदासी॒ दासी॒त् तद॒मुष्या᳚म् । \newline
33. तद॒मुष्या॑ म॒मुष्या॒म् तत् तद॒मुष्या॑ मदधा ददधा द॒मुष्या॒म् तत् तद॒मुष्या॑ मदधात् । \newline
34. अ॒मुष्या॑ मदधा ददधा द॒मुष्या॑ म॒मुष्या॑ मदधा॒त् तत् तद॑दधा द॒मुष्या॑ म॒मुष्या॑ मदधा॒त् तत् । \newline
35. अ॒द॒धा॒त् तत् तद॑दधा ददधा॒त् तद॒दो॑ ऽदस्त द॑दधा ददधा॒त् तद॒दः । \newline
36. तद॒दो॑ ऽदस्तत् तद॒द श्च॒न्द्रम॑सि च॒न्द्रम॑ स्य॒द स्तत् तद॒द श्च॒न्द्रम॑सि । \newline
37. अ॒द श्च॒न्द्रम॑सि च॒न्द्रम॑ स्य॒दो॑ ऽद श्च॒न्द्रम॑सि कृ॒ष्णम् कृ॒ष्णम् च॒न्द्रम॑ स्य॒दो॑ ऽद श्च॒न्द्रम॑सि कृ॒ष्णम् । \newline
38. च॒न्द्रम॑सि कृ॒ष्णम् कृ॒ष्णम् च॒न्द्रम॑सि च॒न्द्रम॑सि कृ॒ष्ण मूषा॒ नूषा᳚न् कृ॒ष्णम् च॒न्द्रम॑सि च॒न्द्रम॑सि कृ॒ष्ण मूषान्॑ । \newline
39. कृ॒ष्ण मूषा॒ नूषा᳚न् कृ॒ष्णम् कृ॒ष्ण मूषा᳚न् नि॒वप॑न् नि॒वप॒न् नूषा᳚न् कृ॒ष्णम् कृ॒ष्ण मूषा᳚न् नि॒वपन्न्॑ । \newline
40. ऊषा᳚न् नि॒वप॑न् नि॒वप॒न् नूषा॒ नूषा᳚न् नि॒वप॑न् न॒दो॑ ऽदो नि॒वप॒न् नूषा॒ नूषा᳚न् नि॒वप॑न् न॒दः । \newline
41. नि॒वप॑न् न॒दो॑ ऽदो नि॒वप॑न् नि॒वप॑न् न॒दो ध्या॑येद् ध्यायेद॒दो नि॒वप॑न् नि॒वप॑न् न॒दो ध्या॑येत् । \newline
42. नि॒वप॒न्निति॑ नि - वपन्न्॑ । \newline
43. अ॒दो ध्या॑येद् ध्याये द॒दो॑ ऽदो ध्या॑ये॒द् द्यावा॑पृथि॒व्योर् द्यावा॑पृथि॒व्योर् ध्या॑ये द॒दो॑ ऽदो ध्या॑ये॒द् द्यावा॑पृथि॒व्योः । \newline
44. ध्या॒ये॒द् द्यावा॑पृथि॒व्योर् द्यावा॑पृथि॒व्योर् ध्या॑येद् ध्याये॒द् द्यावा॑पृथि॒व्यो रे॒वैव द्यावा॑पृथि॒व्योर् ध्या॑येद् ध्याये॒द् द्यावा॑पृथि॒व्यो रे॒व । \newline
45. द्यावा॑पृथि॒व्यो रे॒वैव द्यावा॑पृथि॒व्योर् द्यावा॑पृथि॒व्यो रे॒व य॒ज्ञिये॑ य॒ज्ञिय॑ ए॒व द्यावा॑पृथि॒व्योर् द्यावा॑पृथि॒व्यो रे॒व य॒ज्ञिये᳚ । \newline
46. द्यावा॑पृथि॒व्योरिति॒ द्यावा᳚ - पृ॒थि॒व्योः । \newline
47. ए॒व य॒ज्ञिये॑ य॒ज्ञिय॑ ए॒वैव य॒ज्ञिये॒ ऽग्नि म॒ग्निं ॅय॒ज्ञिय॑ ए॒वैव य॒ज्ञिये॒ ऽग्निम् । \newline
48. य॒ज्ञिये॒ ऽग्नि म॒ग्निं ॅय॒ज्ञिये॑ य॒ज्ञिये॒ ऽग्निम् चि॑नुते चिनुते॒ ऽग्निं ॅय॒ज्ञिये॑ य॒ज्ञिये॒ ऽग्निम् चि॑नुते । \newline
49. अ॒ग्निम् चि॑नुते चिनुते॒ ऽग्नि म॒ग्निम् चि॑नुते॒ ऽय म॒यम् चि॑नुते॒ ऽग्नि म॒ग्निम् चि॑नुते॒ ऽयम् । \newline
50. चि॒नु॒ते॒ ऽय म॒यम् चि॑नुते चिनुते॒ ऽयꣳ स सो॑ ऽयम् चि॑नुते चिनुते॒ ऽयꣳ सः । \newline
51. अ॒यꣳ स सो॑ ऽय म॒यꣳ सो अ॒ग्नि र॒ग्निः सो॑ ऽय म॒यꣳ सो अ॒ग्निः । \newline
52. सो अ॒ग्नि र॒ग्निः स सो अ॒ग्निरिती त्य॒ग्निः स सो अ॒ग्निरिति॑ । \newline
53. अ॒ग्निरिती त्य॒ग्नि र॒ग्निरिति॑ वि॒श्वामि॑त्रस्य वि॒श्वामि॑त्र॒स्ये त्य॒ग्नि र॒ग्निरिति॑ वि॒श्वामि॑त्रस्य । \newline
54. इति॑ वि॒श्वामि॑त्रस्य वि॒श्वामि॑त्र॒स्ये तीति॑ वि॒श्वामि॑त्रस्य सू॒क्तꣳ सू॒क्तं ॅवि॒श्वामि॑त्र॒स्ये तीति॑ वि॒श्वामि॑त्रस्य सू॒क्तम् । \newline
55. वि॒श्वामि॑त्रस्य सू॒क्तꣳ सू॒क्तं ॅवि॒श्वामि॑त्रस्य वि॒श्वामि॑त्रस्य सू॒क्तम् भ॑वति भवति सू॒क्तं ॅवि॒श्वामि॑त्रस्य वि॒श्वामि॑त्रस्य सू॒क्तम् भ॑वति । \newline
56. वि॒श्वामि॑त्र॒स्येति॑ वि॒श्व - मि॒त्र॒स्य॒ । \newline
\pagebreak
\markright{ TS 5.2.3.4  \hfill https://www.vedavms.in \hfill}

\section{ TS 5.2.3.4 }

\textbf{TS 5.2.3.4 } \newline
\textbf{Samhita Paata} \newline

सू॒क्तं भ॑वत्ये॒तेन॒ वै वि॒श्वामि॑त्रो॒ऽग्नेः प्रि॒यं धामा ऽवा॑रुन्धा॒ग्नेरे॒वैतेन॑ प्रि॒यं धामाव॑ रुन्धे॒ छन्दो॑भि॒र्वै दे॒वाः सु॑व॒र्गं ॅलो॒कमा॑य॒न् चत॑स्रः॒ प्राची॒रुप॑ दधाति च॒त्वारि॒ छन्दाꣳ॑सि॒ छन्दो॑भिरे॒व तद्-यज॑मानः सुव॒र्गं ॅलो॒कमे॑ति॒ तेषाꣳ॑ सुव॒र्गं ॅलो॒कं ॅय॒तां दिशः॒ सम॑व्लीयन्त॒ ते द्वे पु॒रस्ता᳚थ् स॒मीची॒ उपा॑दधत॒ द्वे - [  ] \newline

\textbf{Pada Paata} \newline

सू॒क्तमिति॑ सु - उ॒क्तम् । भ॒व॒ति॒ । ए॒तेन॑ । वै । वि॒श्वामि॑त्र॒ इति॑ वि॒श्व - मि॒त्रः॒ । अ॒ग्नेः । प्रि॒यम् । धाम॑ । अवेति॑ । अ॒रु॒न्ध॒ । अ॒ग्नेः । ए॒व । ए॒तेन॑ । प्रि॒यम् । धाम॑ । अवेति॑ । रु॒न्धे॒ । छन्दो॑भि॒रिति॒ छन्दः॑ - भिः॒ । वै । दे॒वाः । सु॒व॒र्गमिति॑ सुवः - गम् । लो॒कम् । आ॒य॒न्न् । चत॑स्रः । प्राचीः᳚ । उपेति॑ । द॒धा॒ति॒ । च॒त्वारि॑ । छन्दाꣳ॑सि । छन्दो॑भि॒रिति॒ छन्दः॑ - भिः॒ । ए॒व । तत् । यज॑मानः । सु॒व॒र्गमिति॑ सुवः - गम् । लो॒कम् । ए॒ति॒ । तेषा᳚म् । सु॒व॒र्गमिति॑ सुवः - गम् । लो॒कम् । य॒ताम् । दिशः॑ । समिति॑ । अ॒व्ली॒य॒न्त॒ । ते । द्वे इति॑ । पु॒रस्ता᳚त् । स॒मीची॒ इति॑ । उपेति॑ । अ॒द॒ध॒त॒ । द्वे इति॑ ।  \newline


\textbf{Krama Paata} \newline

सू॒क्तम् भ॑वति । सू॒क्तमिति॑ सु - उ॒क्तम् । भ॒व॒त्ये॒तेन॑ । ए॒तेन॒ वै । वै वि॒श्वामि॑त्रः । वि॒श्वामि॑त्रो॒ऽग्नेः । वि॒श्वामि॑त्र॒ इति॑ वि॒श्व - मि॒त्रः॒ । अ॒ग्नेः प्रि॒यम् । प्रि॒यम् धाम॑ । धामाव॑ । अवा॑रुन्ध । अ॒रु॒न्धा॒ग्नेः । अ॒ग्नेरे॒व । ए॒वैतेन॑ । ए॒तेन॑ प्रि॒यम् । प्रि॒यम् धाम॑ । धामाव॑ । अव॑ रुन्धे । रु॒न्धे॒ छन्दो॑भिः । छन्दो॑भि॒र् वै । छन्दो॑भि॒रिति॒ छन्दः॑ - भिः॒ । वै दे॒वाः । दे॒वाः सु॑व॒र्गम् । सु॒व॒र्गम् ॅलो॒कम् । सु॒व॒र्गमिति॑ सुवः - गम् । लो॒कमा॑यन्न् । आ॒य॒न् चत॑स्रः । चत॑स्रः॒ प्राचीः᳚ । प्राची॒रुप॑ । उप॑ दधाति । द॒धा॒ति॒ च॒त्वारि॑ । च॒त्वारि॒ छन्दाꣳ॑सि । छन्दाꣳ॑सि॒ छन्दो॑भिः । छन्दो॑भिरे॒व । छन्दो॑भि॒रिति॒ छन्दः॑ - भिः॒ । ए॒व तत् । तद् यज॑मानः । यज॑मानः सुव॒र्गम् । सु॒व॒र्गम् ॅलो॒कम् । सु॒व॒र्गमिति॑ सुवः - गम् । लो॒कमे॑ति । ए॒ति॒ तेषा᳚म् । तेषाꣳ॑ सुव॒र्गम् । सु॒व॒र्गम् ॅलो॒कम् । सु॒व॒र्गमिति॑ सुवः - गम् । लो॒कम् ॅय॒ताम् । य॒ताम् दिशः॑ । दिशः॒ सम् । सम॑व्लीयन्त । अ॒व्ली॒य॒न्त॒ ते । ते द्वे । द्वे पु॒रस्ता᳚त् । द्वे इति॒ द्वे । पु॒रस्ता᳚थ् स॒मीची᳚ । स॒मीची॒ उप॑ । स॒मीची॒ इति॑ स॒मीची᳚ । उपा॑दधत । अ॒द॒ध॒त॒ द्वे । द्वे प॒श्चात् । द्वे इति॒ द्वे \newline

\textbf{Jatai Paata} \newline

1. सू॒क्तम् भ॑वति भवति सू॒क्तꣳ सू॒क्तम् भ॑वति । \newline
2. सू॒क्तमिति॑ सु - उ॒क्तम् । \newline
3. भ॒व॒ त्ये॒ते नै॒तेन॑ भवति भव त्ये॒तेन॑ । \newline
4. ए॒तेन॒ वै वा ए॒ते नै॒तेन॒ वै । \newline
5. वै वि॒श्वामि॑त्रो वि॒श्वामि॑त्रो॒ वै वै वि॒श्वामि॑त्रः । \newline
6. वि॒श्वामि॑त्रो॒ ऽग्ने र॒ग्नेर् वि॒श्वामि॑त्रो वि॒श्वामि॑त्रो॒ ऽग्नेः । \newline
7. वि॒श्वामि॑त्र॒ इति॑ वि॒श्व - मि॒त्रः॒ । \newline
8. अ॒ग्नेः प्रि॒यम् प्रि॒य म॒ग्ने र॒ग्नेः प्रि॒यम् । \newline
9. प्रि॒यम् धाम॒ धाम॑ प्रि॒यम् प्रि॒यम् धाम॑ । \newline
10. धामा वाव॒ धाम॒ धामाव॑ । \newline
11. अवा॑ रुन्धा रु॒न्धा वावा॑ रुन्ध । \newline
12. अ॒रु॒न्धा॒ग्ने र॒ग्ने र॑रुन्धा रुन्धा॒ग्नेः । \newline
13. अ॒ग्ने रे॒वै वाग्ने र॒ग्ने रे॒व । \newline
14. ए॒वै तेनै॒ते नै॒वै वैतेन॑ । \newline
15. ए॒तेन॑ प्रि॒यम् प्रि॒य मे॒ते नै॒तेन॑ प्रि॒यम् । \newline
16. प्रि॒यम् धाम॒ धाम॑ प्रि॒यम् प्रि॒यम् धाम॑ । \newline
17. धामा वाव॒ धाम॒ धामाव॑ । \newline
18. अव॑ रुन्धे रु॒न्धे ऽवाव॑ रुन्धे । \newline
19. रु॒न्धे॒ छन्दो॑भि॒ श्छन्दो॑भी रुन्धे रुन्धे॒ छन्दो॑भिः । \newline
20. छन्दो॑भि॒र् वै वै छन्दो॑भि॒ श्छन्दो॑भि॒र् वै । \newline
21. छन्दो॑भि॒रिति॒ छन्दः॑ - भिः॒ । \newline
22. वै दे॒वा दे॒वा वै वै दे॒वाः । \newline
23. दे॒वाः सु॑व॒र्गꣳ सु॑व॒र्गम् दे॒वा दे॒वाः सु॑व॒र्गम् । \newline
24. सु॒व॒र्गम् ॅलो॒कम् ॅलो॒कꣳ सु॑व॒र्गꣳ सु॑व॒र्गम् ॅलो॒कम् । \newline
25. सु॒व॒र्गमिति॑ सुवः - गम् । \newline
26. लो॒क मा॑यन् नायन् ॅलो॒कम् ॅलो॒क मा॑यन्न् । \newline
27. आ॒य॒न् चत॑स्र॒ श्चत॑स्र आयन् नाय॒न् चत॑स्रः । \newline
28. चत॑स्रः॒ प्राचीः॒ प्राची॒ श्चत॑स्र॒ श्चत॑स्रः॒ प्राचीः᳚ । \newline
29. प्राची॒ रुपोप॒ प्राचीः॒ प्राची॒ रुप॑ । \newline
30. उप॑ दधाति दधा॒ त्युपोप॑ दधाति । \newline
31. द॒धा॒ति॒ च॒त्वारि॑ च॒त्वारि॑ दधाति दधाति च॒त्वारि॑ । \newline
32. च॒त्वारि॒ छन्दाꣳ॑सि॒ छन्दाꣳ॑सि च॒त्वारि॑ च॒त्वारि॒ छन्दाꣳ॑सि । \newline
33. छन्दाꣳ॑सि॒ छन्दो॑भि॒ श्छन्दो॑भि॒ श्छन्दाꣳ॑सि॒ छन्दाꣳ॑सि॒ छन्दो॑भिः । \newline
34. छन्दो॑भि रे॒वैव छन्दो॑भि॒ श्छन्दो॑भि रे॒व । \newline
35. छन्दो॑भि॒रिति॒ छन्दः॑ - भिः॒ । \newline
36. ए॒व तत् तदे॒वैव तत् । \newline
37. तद् यज॑मानो॒ यज॑मान॒ स्तत् तद् यज॑मानः । \newline
38. यज॑मानः सुव॒र्गꣳ सु॑व॒र्गं ॅयज॑मानो॒ यज॑मानः सुव॒र्गम् । \newline
39. सु॒व॒र्गम् ॅलो॒कम् ॅलो॒कꣳ सु॑व॒र्गꣳ सु॑व॒र्गम् ॅलो॒कम् । \newline
40. सु॒व॒र्गमिति॑ सुवः - गम् । \newline
41. लो॒क मे᳚त्येति लो॒कम् ॅलो॒क मे॑ति । \newline
42. ए॒ति॒ तेषा॒म् तेषा॑ मेत्येति॒ तेषा᳚म् । \newline
43. तेषाꣳ॑ सुव॒र्गꣳ सु॑व॒र्गम् तेषा॒म् तेषाꣳ॑ सुव॒र्गम् । \newline
44. सु॒व॒र्गम् ॅलो॒कम् ॅलो॒कꣳ सु॑व॒र्गꣳ सु॑व॒र्गम् ॅलो॒कम् । \newline
45. सु॒व॒र्गमिति॑ सुवः - गम् । \newline
46. लो॒कं ॅय॒तां ॅय॒ताम् ॅलो॒कम् ॅलो॒कं ॅय॒ताम् । \newline
47. य॒ताम् दिशो॒ दिशो॑ य॒तां ॅय॒ताम् दिशः॑ । \newline
48. दिशः॒ सꣳ सम् दिशो॒ दिशः॒ सम् । \newline
49. स म॑व्लीयन्ता व्लीयन्त॒ सꣳ स म॑व्लीयन्त । \newline
50. अ॒व्ली॒य॒न्त॒ ते ते᳚ ऽव्लीयन्ता व्लीयन्त॒ ते । \newline
51. ते द्वे द्वे ते ते द्वे । \newline
52. द्वे पु॒रस्ता᳚त् पु॒रस्ता॒द् द्वे द्वे पु॒रस्ता᳚त् । \newline
53. द्वे इति॒ द्वे । \newline
54. पु॒रस्ता᳚थ् स॒मीची॑ स॒मीची॑ पु॒रस्ता᳚त् पु॒रस्ता᳚थ् स॒मीची᳚ । \newline
55. स॒मीची॒ उपोप॑ स॒मीची॑ स॒मीची॒ उप॑ । \newline
56. स॒मीची॒ इति॑ स॒मीची᳚ । \newline
57. उपा॑ दधता दध॒ तोपोपा॑ दधत । \newline
58. अ॒द॒ध॒त॒ द्वे द्वे अ॑दधता दधत॒ द्वे । \newline
59. द्वे प॒श्चात् प॒श्चाद् द्वे द्वे प॒श्चात् । \newline
60. द्वे इति॒ द्वे । \newline

\textbf{Ghana Paata } \newline

1. सू॒क्तम् भ॑वति भवति सू॒क्तꣳ सू॒क्तम् भ॑व त्ये॒ते नै॒तेन॑ भवति सू॒क्तꣳ सू॒क्तम् भ॑व त्ये॒तेन॑ । \newline
2. सू॒क्तमिति॑ सु - उ॒क्तम् । \newline
3. भ॒व॒ त्ये॒ते नै॒तेन॑ भवति भव त्ये॒तेन॒ वै वा ए॒तेन॑ भवति भव त्ये॒तेन॒ वै । \newline
4. ए॒तेन॒ वै वा ए॒ते नै॒तेन॒ वै वि॒श्वामि॑त्रो वि॒श्वामि॑त्रो॒ वा ए॒ते नै॒तेन॒ वै वि॒श्वामि॑त्रः । \newline
5. वै वि॒श्वामि॑त्रो वि॒श्वामि॑त्रो॒ वै वै वि॒श्वामि॑त्रो॒ ऽग्ने र॒ग्नेर् वि॒श्वामि॑त्रो॒ वै वै वि॒श्वामि॑त्रो॒ ऽग्नेः । \newline
6. वि॒श्वामि॑त्रो॒ ऽग्ने र॒ग्नेर् वि॒श्वामि॑त्रो वि॒श्वामि॑त्रो॒ ऽग्नेः प्रि॒यम् प्रि॒य म॒ग्नेर् वि॒श्वामि॑त्रो वि॒श्वामि॑त्रो॒ ऽग्नेः प्रि॒यम् । \newline
7. वि॒श्वामि॑त्र॒ इति॑ वि॒श्व - मि॒त्रः॒ । \newline
8. अ॒ग्नेः प्रि॒यम् प्रि॒य म॒ग्ने र॒ग्नेः प्रि॒यम् धाम॒ धाम॑ प्रि॒य म॒ग्ने र॒ग्नेः प्रि॒यम् धाम॑ । \newline
9. प्रि॒यम् धाम॒ धाम॑ प्रि॒यम् प्रि॒यम् धामावाव॒ धाम॑ प्रि॒यम् प्रि॒यम् धामाव॑ । \newline
10. धामावाव॒ धाम॒ धामावा॑ रुन्धा रु॒न्धाव॒ धाम॒ धामावा॑ रुन्ध । \newline
11. अवा॑ रुन्धा रु॒न्धा वावा॑ रुन्धा॒ग्ने र॒ग्ने र॑रु॒न्धा वावा॑ रुन्धा॒ग्नेः । \newline
12. अ॒रु॒न्धा॒ग्ने र॒ग्ने र॑रुन्धा रुन्धा॒ ग्ने रे॒वैवाग्ने र॑रुन्धा रुन्धा॒ग्ने रे॒व । \newline
13. अ॒ग्ने रे॒वैवाग्ने र॒ग्ने रे॒वैते नै॒ते नै॒वाग्ने र॒ग्ने रे॒वैतेन॑ । \newline
14. ए॒वैते नै॒ते नै॒वैवैतेन॑ प्रि॒यम् प्रि॒य मे॒ते नै॒वैवैतेन॑ प्रि॒यम् । \newline
15. ए॒तेन॑ प्रि॒यम् प्रि॒य मे॒ते नै॒तेन॑ प्रि॒यम् धाम॒ धाम॑ प्रि॒य मे॒ते नै॒तेन॑ प्रि॒यम् धाम॑ । \newline
16. प्रि॒यम् धाम॒ धाम॑ प्रि॒यम् प्रि॒यम् धामावाव॒ धाम॑ प्रि॒यम् प्रि॒यम् धामाव॑ । \newline
17. धामावाव॒ धाम॒ धामाव॑ रुन्धे रु॒न्धे ऽव॒ धाम॒ धामाव॑ रुन्धे । \newline
18. अव॑ रुन्धे रु॒न्धे ऽवाव॑ रुन्धे॒ छन्दो॑भि॒ श्छन्दो॑भी रु॒न्धे ऽवाव॑ रुन्धे॒ छन्दो॑भिः । \newline
19. रु॒न्धे॒ छन्दो॑भि॒ श्छन्दो॑भी रुन्धे रुन्धे॒ छन्दो॑भि॒र् वै वै छन्दो॑भी रुन्धे रुन्धे॒ छन्दो॑भि॒र् वै । \newline
20. छन्दो॑भि॒र् वै वै छन्दो॑भि॒ श्छन्दो॑भि॒र् वै दे॒वा दे॒वा वै छन्दो॑भि॒ श्छन्दो॑भि॒र् वै दे॒वाः । \newline
21. छन्दो॑भि॒रिति॒ छन्दः॑ - भिः॒ । \newline
22. वै दे॒वा दे॒वा वै वै दे॒वाः सु॑व॒र्गꣳ सु॑व॒र्गम् दे॒वा वै वै दे॒वाः सु॑व॒र्गम् । \newline
23. दे॒वाः सु॑व॒र्गꣳ सु॑व॒र्गम् दे॒वा दे॒वाः सु॑व॒र्गम् ॅलो॒कम् ॅलो॒कꣳ सु॑व॒र्गम् दे॒वा दे॒वाः सु॑व॒र्गम् ॅलो॒कम् । \newline
24. सु॒व॒र्गम् ॅलो॒कम् ॅलो॒कꣳ सु॑व॒र्गꣳ सु॑व॒र्गम् ॅलो॒क मा॑यन् नायन् ॅलो॒कꣳ सु॑व॒र्गꣳ सु॑व॒र्गम् ॅलो॒क मा॑यन्न् । \newline
25. सु॒व॒र्गमिति॑ सुवः - गम् । \newline
26. लो॒क मा॑यन् नायन् ॅलो॒कम् ॅलो॒क मा॑य॒न् चत॑स्र॒ श्चत॑स्र आयन् ॅलो॒कम् ॅलो॒क मा॑य॒न् चत॑स्रः । \newline
27. आ॒य॒न् चत॑स्र॒ श्चत॑स्र आयन् नाय॒न् चत॑स्रः॒ प्राचीः॒ प्राची॒ श्चत॑स्र आयन् नाय॒न् चत॑स्रः॒ प्राचीः᳚ । \newline
28. चत॑स्रः॒ प्राचीः॒ प्राची॒ श्चत॑स्र॒ श्चत॑स्रः॒ प्राची॒ रुपोप॒ प्राची॒ श्चत॑स्र॒ श्चत॑स्रः॒ प्राची॒ रुप॑ । \newline
29. प्राची॒ रुपोप॒ प्राचीः॒ प्राची॒ रुप॑ दधाति दधा॒ त्युप॒ प्राचीः॒ प्राची॒ रुप॑ दधाति । \newline
30. उप॑ दधाति दधा॒ त्युपोप॑ दधाति च॒त्वारि॑ च॒त्वारि॑ दधा॒ त्युपोप॑ दधाति च॒त्वारि॑ । \newline
31. द॒धा॒ति॒ च॒त्वारि॑ च॒त्वारि॑ दधाति दधाति च॒त्वारि॒ छन्दाꣳ॑सि॒ छन्दाꣳ॑सि च॒त्वारि॑ दधाति दधाति च॒त्वारि॒ छन्दाꣳ॑सि । \newline
32. च॒त्वारि॒ छन्दाꣳ॑सि॒ छन्दाꣳ॑सि च॒त्वारि॑ च॒त्वारि॒ छन्दाꣳ॑सि॒ छन्दो॑भि॒ श्छन्दो॑भि॒ श्छन्दाꣳ॑सि च॒त्वारि॑ च॒त्वारि॒ छन्दाꣳ॑सि॒ छन्दो॑भिः । \newline
33. छन्दाꣳ॑सि॒ छन्दो॑भि॒ श्छन्दो॑भि॒ श्छन्दाꣳ॑सि॒ छन्दाꣳ॑सि॒ छन्दो॑भि रे॒वैव छन्दो॑भि॒ श्छन्दाꣳ॑सि॒ छन्दाꣳ॑सि॒ छन्दो॑भि रे॒व । \newline
34. छन्दो॑भि रे॒वैव छन्दो॑भि॒ श्छन्दो॑भि रे॒व तत् तदे॒व छन्दो॑भि॒ श्छन्दो॑भि रे॒व तत् । \newline
35. छन्दो॑भि॒रिति॒ छन्दः॑ - भिः॒ । \newline
36. ए॒व तत् तदे॒वैव तद् यज॑मानो॒ यज॑मान॒ स्तदे॒वैव तद् यज॑मानः । \newline
37. तद् यज॑मानो॒ यज॑मान॒ स्तत् तद् यज॑मानः सुव॒र्गꣳ सु॑व॒र्गं ॅयज॑मान॒ स्तत् तद् यज॑मानः सुव॒र्गम् । \newline
38. यज॑मानः सुव॒र्गꣳ सु॑व॒र्गं ॅयज॑मानो॒ यज॑मानः सुव॒र्गम् ॅलो॒कम् ॅलो॒कꣳ सु॑व॒र्गं ॅयज॑मानो॒ यज॑मानः सुव॒र्गम् ॅलो॒कम् । \newline
39. सु॒व॒र्गम् ॅलो॒कम् ॅलो॒कꣳ सु॑व॒र्गꣳ सु॑व॒र्गम् ॅलो॒क मे᳚त्येति लो॒कꣳ सु॑व॒र्गꣳ सु॑व॒र्गम् ॅलो॒क मे॑ति । \newline
40. सु॒व॒र्गमिति॑ सुवः - गम् । \newline
41. लो॒क मे᳚त्येति लो॒कम् ॅलो॒क मे॑ति॒ तेषा॒म् तेषा॑ मेति लो॒कम् ॅलो॒क मे॑ति॒ तेषा᳚म् । \newline
42. ए॒ति॒ तेषा॒म् तेषा॑ मेत्येति॒ तेषाꣳ॑ सुव॒र्गꣳ सु॑व॒र्गम् तेषा॑ मेत्येति॒ तेषाꣳ॑ सुव॒र्गम् । \newline
43. तेषाꣳ॑ सुव॒र्गꣳ सु॑व॒र्गम् तेषा॒म् तेषाꣳ॑ सुव॒र्गम् ॅलो॒कम् ॅलो॒कꣳ सु॑व॒र्गम् तेषा॒म् तेषाꣳ॑ सुव॒र्गम् ॅलो॒कम् । \newline
44. सु॒व॒र्गम् ॅलो॒कम् ॅलो॒कꣳ सु॑व॒र्गꣳ सु॑व॒र्गम् ॅलो॒कं ॅय॒तां ॅय॒ताम् ॅलो॒कꣳ सु॑व॒र्गꣳ सु॑व॒र्गम् ॅलो॒कं ॅय॒ताम् । \newline
45. सु॒व॒र्गमिति॑ सुवः - गम् । \newline
46. लो॒कं ॅय॒तां ॅय॒ताम् ॅलो॒कम् ॅलो॒कं ॅय॒ताम् दिशो॒ दिशो॑ य॒ताम् ॅलो॒कम् ॅलो॒कं ॅय॒ताम् दिशः॑ । \newline
47. य॒ताम् दिशो॒ दिशो॑ य॒तां ॅय॒ताम् दिशः॒ सꣳ सम् दिशो॑ य॒तां ॅय॒ताम् दिशः॒ सम् । \newline
48. दिशः॒ सꣳ सम् दिशो॒ दिशः॒ स म॑व्लीयन्ता व्लीयन्त॒ सम् दिशो॒ दिशः॒ स म॑व्लीयन्त । \newline
49. स म॑व्लीयन्ता व्लीयन्त॒ सꣳ स म॑व्लीयन्त॒ ते ते᳚ ऽव्लीयन्त॒ सꣳ स म॑व्लीयन्त॒ ते । \newline
50. अ॒व्ली॒य॒न्त॒ ते ते᳚ ऽव्लीयन्ता व्लीयन्त॒ ते द्वे द्वे ते᳚ ऽव्लीयन्ता व्लीयन्त॒ ते द्वे । \newline
51. ते द्वे द्वे ते ते द्वे पु॒रस्ता᳚त् पु॒रस्ता॒द् द्वे ते ते द्वे पु॒रस्ता᳚त् । \newline
52. द्वे पु॒रस्ता᳚त् पु॒रस्ता॒द् द्वे द्वे पु॒रस्ता᳚थ् स॒मीची॑ स॒मीची॑ पु॒रस्ता॒द् द्वे द्वे पु॒रस्ता᳚थ् स॒मीची᳚ । \newline
53. द्वे इति॒ द्वे । \newline
54. पु॒रस्ता᳚थ् स॒मीची॑ स॒मीची॑ पु॒रस्ता᳚त् पु॒रस्ता᳚थ् स॒मीची॒ उपोप॑ स॒मीची॑ पु॒रस्ता᳚त् पु॒रस्ता᳚थ् स॒मीची॒ उप॑ । \newline
55. स॒मीची॒ उपोप॑ स॒मीची॑ स॒मीची॒ उपा॑दधता दध॒तोप॑ स॒मीची॑ स॒मीची॒ उपा॑दधत । \newline
56. स॒मीची॒ इति॑ स॒मीची᳚ । \newline
57. उपा॑दधता दध॒ तोपोपा॑द धत॒ द्वे द्वे अ॑दध॒ तोपोपा॑ दधत॒ द्वे । \newline
58. अ॒द॒ध॒त॒ द्वे द्वे अ॑दधता दधत॒ द्वे प॒श्चात् प॒श्चाद् द्वे अ॑दधता दधत॒ द्वे प॒श्चात् । \newline
59. द्वे प॒श्चात् प॒श्चाद् द्वे द्वे प॒श्चाथ् स॒मीची॑ स॒मीची॑ प॒श्चाद् द्वे द्वे प॒श्चाथ् स॒मीची᳚ । \newline
60. द्वे इति॒ द्वे । \newline
\pagebreak
\markright{ TS 5.2.3.5  \hfill https://www.vedavms.in \hfill}

\section{ TS 5.2.3.5 }

\textbf{TS 5.2.3.5 } \newline
\textbf{Samhita Paata} \newline

प॒श्चाथ् स॒मीची॒ ताभि॒र्वै तेदिशो॑ऽदृꣳह॒न्॒ यद्द्वे पु॒रस्ता᳚थ् स॒मीची॑ उप॒दधा॑ति॒ द्वे प॒॒श्चाथ् स॒मीची॑ दि॒शां ॅविधृ॑त्या॒ अथो॑ प॒शवो॒ वै छन्दाꣳ॑सि प॒शूने॒वास्मै॑ स॒मीचो॑ दधात्य॒ष्टावुप॑ दधात्य॒ष्टाक्ष॑रा गाय॒त्री गा॑य॒त्रो᳚ऽग्नि-र्यावा॑ने॒वाग्निस्तं चि॑नुते॒ऽष्टावुप॑ दधात्य॒ष्टाक्ष॑रा गाय॒त्री गा॑य॒त्री सु॑व॒र्गं ॅलो॒कमञ्ज॑सा वेद सुव॒र्गस्य॑ लो॒कस्य॒ - [  ] \newline

\textbf{Pada Paata} \newline

प॒श्चात् । स॒मीची॒ इति॑ । ताभिः॑ । वै । ते । दिशः॑ । अ॒दृꣳ॒॒ह॒न्न् । यत् । द्वे इति॑ । पु॒रस्ता᳚त् । स॒मीची॒ इति॑ । उ॒प॒दधा॒तीत्यु॑प - दधा॑ति । द्वे इति॑ । प॒श्चात् । स॒मीची॒ इति॑ । दि॒शाम् । विधृ॑त्या॒ इति॒ वि-धृ॒त्यै॒ । अथो॒ इति॑ । प॒शवः॑ । वै । छन्दाꣳ॑सि । प॒शून् । ए॒व । अ॒स्मै॒ । स॒मीचः॑ । द॒धा॒ति॒ । अ॒ष्टौ । उपेति॑ । द॒धा॒ति॒ । अ॒ष्टाक्ष॒रेत्य॒ष्टा-अ॒क्ष॒रा॒ । गा॒य॒त्री । गा॒य॒त्रः । अ॒ग्निः । यावान्॑ । ए॒व । अ॒ग्निः । तम् । चि॒नु॒ते॒ । अ॒ष्टौ । उपेति॑ । द॒धा॒ति॒ । अ॒ष्टाक्ष॒रेत्य॒ष्टा-अ॒क्ष॒रा॒ । गा॒य॒त्री । गा॒य॒त्री । सु॒व॒र्गमिति॑ सुवः - गम् । लो॒कम् । अञ्ज॑सा । वे॒द॒ । सु॒व॒र्गस्येति॑ सुवः - गस्य॑ । लो॒कस्य॑ ।  \newline


\textbf{Krama Paata} \newline

प॒श्चाथ् स॒मीची᳚ । स॒मीची॒ ताभिः॑ । स॒मीची॒ इति॑ स॒मीची᳚ । ताभि॒र् वै । वै ते । ते दिशः॑ । दिशो॑ऽदृꣳहन्न् । अ॒दृꣳ॒॒ह॒न्॒. यत् । यद् द्वे । द्वे पु॒रस्ता᳚त् । द्वे इति॒ द्वे । पु॒रस्ता᳚थ् स॒मीची᳚ । स॒मीची॑ उप॒दधा॑ति । स॒मीची॒ इति॑ स॒मीची᳚ । उ॒प॒दधा॑ति॒ द्वे । उ॒प॒दधा॒तीत्यु॑प - दधा॑ति । द्वे प॒श्चात् । द्वे इति॒ द्वे । प॒श्चाथ् स॒मीची᳚ । स॒मीची॑ दि॒शाम् । स॒मीची॒ इति॑ स॒मीची᳚ । दि॒शाम् ॅविधृ॑त्यै । विधृ॑त्या॒ अथो᳚ । विधृ॑त्या॒ इति॒ वि - धृ॒त्यै॒ । अथो॑ प॒शवः॑ । अथो॒ इत्यथो᳚ । प॒शवो॒ वै । वै छन्दाꣳ॑सि । छन्दाꣳ॑सि प॒शून् । प॒शूने॒व । ए॒वास्मै᳚ । अ॒स्मै॒ स॒मीचः॑ । स॒मीचो॑ दधाति । द॒धा॒त्य॒ष्टौ । अ॒ष्टावुप॑ । उप॑ दधाति । द॒धा॒त्य॒ष्टाक्ष॑रा । अ॒ष्टाक्ष॑रा गाय॒त्री । अ॒ष्टाक्ष॒रेत्य॒ष्टा - अ॒क्ष॒रा॒ । गा॒य॒त्री गा॑य॒त्रः । गा॒य॒त्रो᳚ऽग्निः । अ॒ग्निर् यावान्॑ । यावा॑ने॒व । ए॒वाग्निः । अ॒ग्निस्तम् । तम् चि॑नुते । चि॒नु॒ते॒ऽष्टौ । अ॒ष्टावुप॑ । उप॑ दधाति । द॒धा॒त्य॒ष्टाक्ष॑रा । अ॒ष्टाक्ष॑रा गाय॒त्री । अ॒ष्टाक्ष॒रेत्य॒ष्टा - अ॒क्ष॒रा॒ । गा॒य॒त्री गा॑य॒त्री । गा॒य॒त्री सु॑व॒र्गम् । सु॒व॒र्गम् ॅलो॒कम् । सु॒व॒र्गमिति॑ सुवः - गम् । लो॒कमञ्ज॑सा । अञ्ज॑सा वेद । वे॒द॒ सु॒व॒र्गस्य॑ । सु॒व॒र्गस्य॑ लो॒कस्य॑ । सु॒व॒र्गस्येति॑ सुवः - गस्य॑ । लो॒कस्य॒ प्रज्ञा᳚त्यै \newline

\textbf{Jatai Paata} \newline

1. प॒श्चाथ् स॒मीची॑ स॒मीची॑ प॒श्चात् प॒श्चाथ् स॒मीची᳚ । \newline
2. स॒मीची॒ ताभि॒ स्ताभिः॑ स॒मीची॑ स॒मीची॒ ताभिः॑ । \newline
3. स॒मीची॒ इति॑ स॒मीची᳚ । \newline
4. ताभि॒र् वै वै ताभि॒ स्ताभि॒र् वै । \newline
5. वै ते ते वै वै ते । \newline
6. ते दिशो॒ दिश॒ स्ते ते दिशः॑ । \newline
7. दिशो॑ ऽदृꣳहन् नदृꣳह॒न् दिशो॒ दिशो॑ ऽदृꣳहन्न् । \newline
8. अ॒दृꣳ॒॒ह॒न्॒. यद् यद॑दृꣳहन् नदृꣳह॒न्॒. यत् । \newline
9. यद् द्वे द्वे यद् यद् द्वे । \newline
10. द्वे पु॒रस्ता᳚त् पु॒रस्ता॒द् द्वे द्वे पु॒रस्ता᳚त् । \newline
11. द्वे इति॒ द्वे । \newline
12. पु॒रस्ता᳚थ् स॒मीची॑ स॒मीची॑ पु॒रस्ता᳚त् पु॒रस्ता᳚थ् स॒मीची᳚ । \newline
13. स॒मीची॑ उप॒दधा᳚ त्युप॒दधा॑ति स॒मीची॑ स॒मीची॑ उप॒दधा॑ति । \newline
14. स॒मीची॒ इति॑ स॒मीची᳚ । \newline
15. उ॒प॒दधा॑ति॒ द्वे द्वे उ॑प॒दधा᳚ त्युप॒दधा॑ति॒ द्वे । \newline
16. उ॒प॒दधा॒तीत्यु॑प - दधा॑ति । \newline
17. द्वे प॒श्चात् प॒श्चाद् द्वे द्वे प॒श्चात् । \newline
18. द्वे इति॒ द्वे । \newline
19. प॒श्चाथ् स॒मीची॑ स॒मीची॑ प॒श्चात् प॒श्चाथ् स॒मीची᳚ । \newline
20. स॒मीची॑ दि॒शाम् दि॒शाꣳ स॒मीची॑ स॒मीची॑ दि॒शाम् । \newline
21. स॒मीची॒ इति॑ स॒मीची᳚ । \newline
22. दि॒शां ॅविधृ॑त्यै॒ विधृ॑त्यै दि॒शाम् दि॒शां ॅविधृ॑त्यै । \newline
23. विधृ॑त्या॒ अथो॒ अथो॒ विधृ॑त्यै॒ विधृ॑त्या॒ अथो᳚ । \newline
24. विधृ॑त्या॒ इति॒ वि - धृ॒त्यै॒ । \newline
25. अथो॑ प॒शवः॑ प॒शवो ऽथो॒ अथो॑ प॒शवः॑ । \newline
26. अथो॒ इत्यथो᳚ । \newline
27. प॒शवो॒ वै वै प॒शवः॑ प॒शवो॒ वै । \newline
28. वै छन्दाꣳ॑सि॒ छन्दाꣳ॑सि॒ वै वै छन्दाꣳ॑सि । \newline
29. छन्दाꣳ॑सि प॒शून् प॒शून् छन्दाꣳ॑सि॒ छन्दाꣳ॑सि प॒शून् । \newline
30. प॒शू ने॒वैव प॒शून् प॒शू ने॒व । \newline
31. ए॒वास्मा॑ अस्मा ए॒वै वास्मै᳚ । \newline
32. अ॒स्मै॒ स॒मीचः॑ स॒मीचो᳚ ऽस्मा अस्मै स॒मीचः॑ । \newline
33. स॒मीचो॑ दधाति दधाति स॒मीचः॑ स॒मीचो॑ दधाति । \newline
34. द॒धा॒ त्य॒ष्टा व॒ष्टौ द॑धाति दधा त्य॒ष्टौ । \newline
35. अ॒ष्टा वुपोपा॒ष्टा व॒ष्टा वुप॑ । \newline
36. उप॑ दधाति दधा॒ त्युपोप॑ दधाति । \newline
37. द॒धा॒ त्य॒ष्टाक्ष॑रा॒ ऽष्टाक्ष॑रा दधाति दधा त्य॒ष्टाक्ष॑रा । \newline
38. अ॒ष्टाक्ष॑रा गाय॒त्री गा॑य॒ त्र्य॑ष्टाक्ष॑रा॒ ऽष्टाक्ष॑रा गाय॒त्री । \newline
39. अ॒ष्टाक्ष॒रेत्य॒ष्टा - अ॒क्ष॒रा॒ । \newline
40. गा॒य॒त्री गा॑य॒त्रो गा॑य॒त्रो गा॑य॒त्री गा॑य॒त्री गा॑य॒त्रः । \newline
41. गा॒य॒त्रो᳚ ऽग्नि र॒ग्निर् गा॑य॒त्रो गा॑य॒त्रो᳚ ऽग्निः । \newline
42. अ॒ग्निर् यावा॒न्॒. यावा॑ न॒ग्नि र॒ग्निर् यावान्॑ । \newline
43. यावा॑ ने॒वैव यावा॒न्॒. यावा॑ ने॒व । \newline
44. ए॒वाग्नि र॒ग्नि रे॒वै वाग्निः । \newline
45. अ॒ग्नि स्तम् त म॒ग्नि र॒ग्नि स्तम् । \newline
46. तम् चि॑नुते चिनुते॒ तम् तम् चि॑नुते । \newline
47. चि॒नु॒ते॒ ऽष्टा व॒ष्टौ चि॑नुते चिनुते॒ ऽष्टौ । \newline
48. अ॒ष्टा वुपोपा॒ष्टा व॒ष्टा वुप॑ । \newline
49. उप॑ दधाति दधा॒ त्युपोप॑ दधाति । \newline
50. द॒धा॒ त्य॒ष्टाक्ष॑रा॒ ऽष्टाक्ष॑रा दधाति दधा त्य॒ष्टाक्ष॑रा । \newline
51. अ॒ष्टाक्ष॑रा गाय॒त्री गा॑य॒ त्र्य॑ष्टाक्ष॑रा॒ ऽष्टाक्ष॑रा गाय॒त्री । \newline
52. अ॒ष्टाक्ष॒रेत्य॒ष्टा - अ॒क्ष॒रा॒ । \newline
53. गा॒य॒त्री गा॑य॒त्री । \newline
54. गा॒य॒त्री सु॑व॒र्गꣳ सु॑व॒र्गम् गा॑य॒त्री गा॑य॒त्री सु॑व॒र्गम् । \newline
55. सु॒व॒र्गम् ॅलो॒कम् ॅलो॒कꣳ सु॑व॒र्गꣳ सु॑व॒र्गम् ॅलो॒कम् । \newline
56. सु॒व॒र्गमिति॑ सुवः - गम् । \newline
57. लो॒क मञ्ज॒सा ऽञ्ज॑सा लो॒कम् ॅलो॒क मञ्ज॑सा । \newline
58. अञ्ज॑सा वेद वे॒दाञ्ज॒सा ऽञ्ज॑सा वेद । \newline
59. वे॒द॒ सु॒व॒र्गस्य॑ सुव॒र्गस्य॑ वेद वेद सुव॒र्गस्य॑ । \newline
60. सु॒व॒र्गस्य॑ लो॒कस्य॑ लो॒कस्य॑ सुव॒र्गस्य॑ सुव॒र्गस्य॑ लो॒कस्य॑ । \newline
61. सु॒व॒र्गस्येति॑ सुवः - गस्य॑ । \newline
62. लो॒कस्य॒ प्रज्ञा᳚त्यै॒ प्रज्ञा᳚त्यै लो॒कस्य॑ लो॒कस्य॒ प्रज्ञा᳚त्यै । \newline

\textbf{Ghana Paata } \newline

1. प॒श्चाथ् स॒मीची॑ स॒मीची॑ प॒श्चात् प॒श्चाथ् स॒मीची॒ ताभि॒ स्ताभिः॑ स॒मीची॑ प॒श्चात् प॒श्चाथ् स॒मीची॒ ताभिः॑ । \newline
2. स॒मीची॒ ताभि॒ स्ताभिः॑ स॒मीची॑ स॒मीची॒ ताभि॒र् वै वै ताभिः॑ स॒मीची॑ स॒मीची॒ ताभि॒र् वै । \newline
3. स॒मीची॒ इति॑ स॒मीची᳚ । \newline
4. ताभि॒र् वै वै ताभि॒ स्ताभि॒र् वै ते ते वै ताभि॒ स्ताभि॒र् वै ते । \newline
5. वै ते ते वै वै ते दिशो॒ दिश॒ स्ते वै वै ते दिशः॑ । \newline
6. ते दिशो॒ दिश॒ स्ते ते दिशो॑ ऽदृꣳहन् नदृꣳह॒न् दिश॒ स्ते ते दिशो॑ ऽदृꣳहन्न् । \newline
7. दिशो॑ ऽदृꣳहन् नदृꣳह॒न् दिशो॒ दिशो॑ ऽदृꣳह॒न्॒. यद् यद॑दृꣳह॒न् दिशो॒ दिशो॑ ऽदृꣳह॒न्॒. यत् । \newline
8. अ॒दृꣳ॒॒ह॒न्॒. यद् यद॑दृꣳहन् नदृꣳह॒न्॒. यद् द्वे द्वे यद॑दृꣳहन् नदृꣳह॒न्॒. यद् द्वे । \newline
9. यद् द्वे द्वे यद् यद् द्वे पु॒रस्ता᳚त् पु॒रस्ता॒द् द्वे यद् यद् द्वे पु॒रस्ता᳚त् । \newline
10. द्वे पु॒रस्ता᳚त् पु॒रस्ता॒द् द्वे द्वे पु॒रस्ता᳚थ् स॒मीची॑ स॒मीची॑ पु॒रस्ता॒द् द्वे द्वे पु॒रस्ता᳚थ् स॒मीची᳚ । \newline
11. द्वे इति॒ द्वे । \newline
12. पु॒रस्ता᳚थ् स॒मीची॑ स॒मीची॑ पु॒रस्ता᳚त् पु॒रस्ता᳚थ् स॒मीची॑ उप॒दधा᳚ त्युप॒दधा॑ति स॒मीची॑ पु॒रस्ता᳚त् पु॒रस्ता᳚थ् स॒मीची॑ उप॒दधा॑ति । \newline
13. स॒मीची॑ उप॒दधा᳚ त्युप॒दधा॑ति स॒मीची॑ स॒मीची॑ उप॒दधा॑ति॒ द्वे द्वे उ॑प॒दधा॑ति स॒मीची॑ स॒मीची॑ उप॒दधा॑ति॒ द्वे । \newline
14. स॒मीची॒ इति॑ स॒मीची᳚ । \newline
15. उ॒प॒दधा॑ति॒ द्वे द्वे उ॑प॒दधा᳚ त्युप॒दधा॑ति॒ द्वे प॒श्चात् प॒श्चाद् द्वे उ॑प॒दधा᳚ त्युप॒दधा॑ति॒ द्वे प॒श्चात् । \newline
16. उ॒प॒दधा॒तीत्यु॑प - दधा॑ति । \newline
17. द्वे प॒श्चात् प॒श्चाद् द्वे द्वे प॒श्चाथ् स॒मीची॑ स॒मीची॑ प॒श्चाद् द्वे द्वे प॒श्चाथ् स॒मीची᳚ । \newline
18. द्वे इति॒ द्वे । \newline
19. प॒श्चाथ् स॒मीची॑ स॒मीची॑ प॒श्चात् प॒श्चाथ् स॒मीची॑ दि॒शाम् दि॒शाꣳ स॒मीची॑ प॒श्चात् प॒श्चाथ् स॒मीची॑ दि॒शाम् । \newline
20. स॒मीची॑ दि॒शाम् दि॒शाꣳ स॒मीची॑ स॒मीची॑ दि॒शां ॅविधृ॑त्यै॒ विधृ॑त्यै दि॒शाꣳ स॒मीची॑ स॒मीची॑ दि॒शां ॅविधृ॑त्यै । \newline
21. स॒मीची॒ इति॑ स॒मीची᳚ । \newline
22. दि॒शां ॅविधृ॑त्यै॒ विधृ॑त्यै दि॒शाम् दि॒शां ॅविधृ॑त्या॒ अथो॒ अथो॒ विधृ॑त्यै दि॒शाम् दि॒शां ॅविधृ॑त्या॒ अथो᳚ । \newline
23. विधृ॑त्या॒ अथो॒ अथो॒ विधृ॑त्यै॒ विधृ॑त्या॒ अथो॑ प॒शवः॑ प॒शवो ऽथो॒ विधृ॑त्यै॒ विधृ॑त्या॒ अथो॑ प॒शवः॑ । \newline
24. विधृ॑त्या॒ इति॒ वि - धृ॒त्यै॒ । \newline
25. अथो॑ प॒शवः॑ प॒शवो ऽथो॒ अथो॑ प॒शवो॒ वै वै प॒शवो ऽथो॒ अथो॑ प॒शवो॒ वै । \newline
26. अथो॒ इत्यथो᳚ । \newline
27. प॒शवो॒ वै वै प॒शवः॑ प॒शवो॒ वै छन्दाꣳ॑सि॒ छन्दाꣳ॑सि॒ वै प॒शवः॑ प॒शवो॒ वै छन्दाꣳ॑सि । \newline
28. वै छन्दाꣳ॑सि॒ छन्दाꣳ॑सि॒ वै वै छन्दाꣳ॑सि प॒शून् प॒शून् छन्दाꣳ॑सि॒ वै वै छन्दाꣳ॑सि प॒शून् । \newline
29. छन्दाꣳ॑सि प॒शून् प॒शून् छन्दाꣳ॑सि॒ छन्दाꣳ॑सि प॒शू ने॒वैव प॒शून् छन्दाꣳ॑सि॒ छन्दाꣳ॑सि प॒शू ने॒व । \newline
30. प॒शू ने॒वैव प॒शून् प॒शू ने॒वास्मा॑ अस्मा ए॒व प॒शून् प॒शू ने॒वास्मै᳚ । \newline
31. ए॒वास्मा॑ अस्मा ए॒वैवास्मै॑ स॒मीचः॑ स॒मीचो᳚ ऽस्मा ए॒वैवास्मै॑ स॒मीचः॑ । \newline
32. अ॒स्मै॒ स॒मीचः॑ स॒मीचो᳚ ऽस्मा अस्मै स॒मीचो॑ दधाति दधाति स॒मीचो᳚ ऽस्मा अस्मै स॒मीचो॑ दधाति । \newline
33. स॒मीचो॑ दधाति दधाति स॒मीचः॑ स॒मीचो॑ दधा त्य॒ष्टा व॒ष्टौ द॑धाति स॒मीचः॑ स॒मीचो॑ दधा त्य॒ष्टौ । \newline
34. द॒धा॒ त्य॒ष्टा व॒ष्टौ द॑धाति दधा त्य॒ष्टा वुपोपा॒ष्टौ द॑धाति दधा त्य॒ष्टा वुप॑ । \newline
35. अ॒ष्टा वुपोपा॒ष्टा व॒ष्टा वुप॑ दधाति दधा॒ त्युपा॒ष्टा व॒ष्टा वुप॑ दधाति । \newline
36. उप॑ दधाति दधा॒ त्युपोप॑ दधा त्य॒ष्टाक्ष॑रा॒ ऽष्टाक्ष॑रा दधा॒ त्युपोप॑ दधा त्य॒ष्टाक्ष॑रा । \newline
37. द॒धा॒ त्य॒ष्टाक्ष॑रा॒ ऽष्टाक्ष॑रा दधाति दधा त्य॒ष्टाक्ष॑रा गाय॒त्री गा॑य॒ त्र्य॑ष्टाक्ष॑रा दधाति दधा त्य॒ष्टाक्ष॑रा गाय॒त्री । \newline
38. अ॒ष्टाक्ष॑रा गाय॒त्री गा॑य॒ त्र्य॑ष्टाक्ष॑रा॒ ऽष्टाक्ष॑रा गाय॒त्री गा॑य॒त्रो गा॑य॒त्रो गा॑य॒ त्र्य॑ष्टाक्ष॑रा॒ ऽष्टाक्ष॑रा गाय॒त्री गा॑य॒त्रः । \newline
39. अ॒ष्टाक्ष॒रेत्य॒ष्टा - अ॒क्ष॒रा॒ । \newline
40. गा॒य॒त्री गा॑य॒त्रो गा॑य॒त्रो गा॑य॒त्री गा॑य॒त्री गा॑य॒त्रो᳚ ऽग्नि र॒ग्निर् गा॑य॒त्रो गा॑य॒त्री गा॑य॒त्री गा॑य॒त्रो᳚ ऽग्निः । \newline
41. गा॒य॒त्रो᳚ ऽग्नि र॒ग्निर् गा॑य॒त्रो गा॑य॒त्रो᳚ ऽग्निर् यावा॒न्॒. यावा॑ न॒ग्निर् गा॑य॒त्रो गा॑य॒त्रो᳚ ऽग्निर् यावान्॑ । \newline
42. अ॒ग्निर् यावा॒न्॒. यावा॑ न॒ग्नि र॒ग्निर् यावा॑ ने॒वैव यावा॑ न॒ग्नि र॒ग्निर् यावा॑ ने॒व । \newline
43. यावा॑ ने॒वैव यावा॒न्॒. यावा॑ ने॒वाग्नि र॒ग्नि रे॒व यावा॒न्॒. यावा॑ ने॒वाग्निः । \newline
44. ए॒वाग्नि र॒ग्नि रे॒वैवाग्नि स्तम् तम॒ग्नि रे॒वैवाग्नि स्तम् । \newline
45. अ॒ग्नि स्तम् त म॒ग्नि र॒ग्नि स्तम् चि॑नुते चिनुते॒ त म॒ग्नि र॒ग्नि स्तम् चि॑नुते । \newline
46. तम् चि॑नुते चिनुते॒ तम् तम् चि॑नुते॒ ऽष्टा व॒ष्टौ चि॑नुते॒ तम् तम् चि॑नुते॒ ऽष्टौ । \newline
47. चि॒नु॒ते॒ ऽष्टा व॒ष्टौ चि॑नुते चिनुते॒ ऽष्टा वुपोपा॒ष्टौ चि॑नुते चिनुते॒ ऽष्टा वुप॑ । \newline
48. अ॒ष्टा वुपोपा॒ष्टा व॒ष्टा वुप॑ दधाति दधा॒ त्युपा॒ष्टा व॒ष्टा वुप॑ दधाति । \newline
49. उप॑ दधाति दधा॒ त्युपोप॑ दधा त्य॒ष्टाक्ष॑रा॒ ऽष्टाक्ष॑रा दधा॒ त्युपोप॑ दधा त्य॒ष्टाक्ष॑रा । \newline
50. द॒धा॒ त्य॒ष्टाक्ष॑रा॒ ऽष्टाक्ष॑रा दधाति दधा त्य॒ष्टाक्ष॑रा गाय॒त्री गा॑य॒ त्र्य॑ष्टाक्ष॑रा दधाति दधा त्य॒ष्टाक्ष॑रा गाय॒त्री । \newline
51. अ॒ष्टाक्ष॑रा गाय॒त्री गा॑य॒ त्र्य॑ष्टाक्ष॑रा॒ ऽष्टाक्ष॑रा गाय॒त्री । \newline
52. अ॒ष्टाक्ष॒रेत्य॒ष्टा - अ॒क्ष॒रा॒ । \newline
53. गा॒य॒त्री गा॑य॒त्री । \newline
54. गा॒य॒त्री सु॑व॒र्गꣳ सु॑व॒र्गम् गा॑य॒त्री गा॑य॒त्री सु॑व॒र्गम् ॅलो॒कम् ॅलो॒कꣳ सु॑व॒र्गम् गा॑य॒त्री गा॑य॒त्री सु॑व॒र्गम् ॅलो॒कम् । \newline
55. सु॒व॒र्गम् ॅलो॒कम् ॅलो॒कꣳ सु॑व॒र्गꣳ सु॑व॒र्गम् ॅलो॒क मञ्ज॒सा ऽञ्ज॑सा लो॒कꣳ सु॑व॒र्गꣳ सु॑व॒र्गम् ॅलो॒क मञ्ज॑सा । \newline
56. सु॒व॒र्गमिति॑ सुवः - गम् । \newline
57. लो॒क मञ्ज॒सा ऽञ्ज॑सा लो॒कम् ॅलो॒क मञ्ज॑सा वेद वे॒दाञ्ज॑सा लो॒कम् ॅलो॒क मञ्ज॑सा वेद । \newline
58. अञ्ज॑सा वेद वे॒दाञ्ज॒सा ऽञ्ज॑सा वेद सुव॒र्गस्य॑ सुव॒र्गस्य॑ वे॒दाञ्ज॒सा ऽञ्ज॑सा वेद सुव॒र्गस्य॑ । \newline
59. वे॒द॒ सु॒व॒र्गस्य॑ सुव॒र्गस्य॑ वेद वेद सुव॒र्गस्य॑ लो॒कस्य॑ लो॒कस्य॑ सुव॒र्गस्य॑ वेद वेद सुव॒र्गस्य॑ लो॒कस्य॑ । \newline
60. सु॒व॒र्गस्य॑ लो॒कस्य॑ लो॒कस्य॑ सुव॒र्गस्य॑ सुव॒र्गस्य॑ लो॒कस्य॒ प्रज्ञा᳚त्यै॒ प्रज्ञा᳚त्यै लो॒कस्य॑ सुव॒र्गस्य॑ सुव॒र्गस्य॑ लो॒कस्य॒ प्रज्ञा᳚त्यै । \newline
61. सु॒व॒र्गस्येति॑ सुवः - गस्य॑ । \newline
62. लो॒कस्य॒ प्रज्ञा᳚त्यै॒ प्रज्ञा᳚त्यै लो॒कस्य॑ लो॒कस्य॒ प्रज्ञा᳚त्यै॒ त्रयो॑दश॒ त्रयो॑दश॒ प्रज्ञा᳚त्यै लो॒कस्य॑ लो॒कस्य॒ प्रज्ञा᳚त्यै॒ त्रयो॑दश । \newline
\pagebreak
\markright{ TS 5.2.3.6  \hfill https://www.vedavms.in \hfill}

\section{ TS 5.2.3.6 }

\textbf{TS 5.2.3.6 } \newline
\textbf{Samhita Paata} \newline

प्रज्ञा᳚त्यै॒ त्रयो॑दश लोकंपृ॒णा उप॑ दधा॒त्येक॑विꣳशतिः॒ संप॑द्यन्ते प्रति॒ष्ठा वा ए॑कविꣳ॒॒शः प्र॑ति॒ष्ठा गार्.ह॑पत्य एकविꣳ॒॒शस्यै॒व प्र॑ति॒ष्ठां गार्.ह॑पत्य॒मनु॒ प्रति॑ तिष्ठति॒ प्रत्य॒ग्निं चि॑क्या॒नस्ति॑ष्ठति॒ य ए॒वं ॅवेद॒ पञ्च॑चितीकं चिन्वीत प्रथ॒मं चि॑न्वा॒नः पाङ्क्तो॑ य॒ज्ञ्ः पाङ्क्ताः᳚ प॒शवो॑ य॒ज्ञ्मे॒व प॒शूनव॑ रुन्धे॒ त्रिचि॑तीकं चिन्वीत द्वि॒तीयं॑ चिन्वा॒नस्त्रय॑ इ॒मे लो॒का ए॒ष्वे॑व लो॒केषु॒ - [  ] \newline

\textbf{Pada Paata} \newline

प्रज्ञा᳚त्या॒ इति॒ प्र - ज्ञा॒त्यै॒ । त्रयो॑द॒शेति॒ त्रयः॑ - द॒श॒ । लो॒क॒पृं॒णा इति॑ लोकं - पृ॒णाः । उपेति॑ । द॒धा॒ति॒ । एक॑विꣳशति॒रित्येक॑ - विꣳ॒॒श॒तिः॒ । समिति॑ । प॒द्य॒न्ते॒ । प्र॒ति॒ष्ठेति॑ प्रति - स्था । वै । ए॒क॒विꣳ॒॒श इत्ये॑क - विꣳ॒॒शः । प्र॒ति॒ष्ठेति॑ प्रति - स्था । गार्.ह॑पत्य॒ इति॒ गार्.ह॑ - प॒त्यः॒ । ए॒क॒विꣳ॒॒शस्येत्ये॑क - विꣳ॒॒शस्य॑ । ए॒व । प्र॒ति॒ष्ठामिति॑ प्रति-स्थाम् । गार्.ह॑पत्य॒मिति॒ गार्.ह॑ - प॒त्य॒म् । अनु॑ । प्रतीति॑ । ति॒ष्ठ॒ति॒ । प्रतीति॑ । अ॒ग्निम् । चि॒क्या॒नः । ति॒ष्ठ॒ति॒ । यः । ए॒वम् । वेद॑ । पञ्च॑चितीक॒मिति॒ पञ्च॑-चि॒ती॒क॒म् । चि॒न्वी॒त॒ । प्र॒थ॒मम् । चि॒न्वा॒नः । पाङ्क्तः॑ । य॒ज्ञ्ः । पाङ्क्ताः᳚ । प॒शवः॑ । य॒ज्ञ्म् । ए॒व । प॒शून् । अवेति॑ । रु॒न्धे॒ । त्रिचि॑तीक॒मिति॒ त्रि - चि॒ती॒क॒म् । चि॒न्वी॒त॒ । द्वि॒तीय᳚म् । चि॒न्वा॒नः । त्रयः॑ । इ॒मे । लो॒काः । ए॒षु । ए॒व । लो॒केषु॑ ।  \newline


\textbf{Krama Paata} \newline

प्रज्ञा᳚त्यै॒ त्रयो॑दश । प्रज्ञा᳚त्या॒ इति॒ प्र - ज्ञा॒त्यै॒ । त्रयो॑दश लोकम्पृ॒णाः । त्रयो॑द॒शेति॒ त्रयः॑ - द॒श॒ । लो॒क॒म्पृ॒णा उप॑ । लो॒क॒म्पृ॒णा इति॑ लोकम् - पृ॒णाः । उप॑ दधाति । द॒धा॒त्येक॑विꣳशतिः । एक॑विꣳशतिः॒ सम् । एक॑विꣳशति॒रित्येक॑ - विꣳ॒॒श॒तिः॒ । सम् प॑द्यन्ते । प॒द्य॒न्ते॒ प्र॒ति॒ष्ठा । प्र॒ति॒ष्ठा वै । प्र॒ति॒ष्ठेति॑ प्रति - स्था । वा ए॑कविꣳ॒॒शः । ए॒क॒विꣳ॒॒शः प्र॑ति॒ष्ठा । ए॒क॒विꣳ॒॒श इत्ये॑क - विꣳ॒॒शः । प्र॒ति॒ष्ठा गार्.ह॑पत्यः । प्र॒ति॒ष्ठेति॑ प्रति - स्था । गार्.ह॑पत्य एकविꣳ॒॒शस्य॑ । गार्.ह॑पत्य॒ इति॒ गार्.ह॑ - प॒त्यः॒ । ए॒क॒विꣳ॒॒शस्यै॒व । ए॒क॒विꣳ॒॒शस्येत्ये॑क - विꣳ॒॒शस्य॑ । ए॒व प्र॑ति॒ष्ठाम् । प्र॒ति॒ष्ठाम् गार्.ह॑पत्यम् । प्र॒ति॒ष्ठामिति॑ प्रति - स्थाम् । गार्.ह॑पत्य॒मनु॑ । गार्.ह॑पत्य॒मिति॒ गार्.ह॑ - प॒त्य॒म् । अनु॒ प्रति॑ । प्रति॑ तिष्ठति । ति॒ष्ठ॒ति॒ प्रति॑ । प्रत्य॒ग्निम् । अ॒ग्निम् चि॑क्या॒नः । चि॒क्या॒नस्ति॑ष्ठति । ति॒ष्ठ॒ति॒ यः । य ए॒वम् । ए॒वम् ॅवेद॑ । वेद॒ पञ्च॑चितीकम् । पञ्च॑चितीकम् चिन्वीत । पञ्च॑चितीक॒मिति॒ पञ्च॑ - चि॒ती॒क॒म् । चि॒न्वी॒त॒ प्र॒थ॒मम् । प्र॒थ॒मम् चि॑न्वा॒नः । चि॒न्वा॒नः पाङ्क्तः॑ । पाङ्क्तो॑ य॒ज्ञ्ः । य॒ज्ञ्ः पाङ्क्ताः᳚ । पाङ्क्ताः᳚ प॒शवः॑ । प॒शवो॑ य॒ज्ञ्म् । य॒ज्ञ्मे॒व । ए॒व प॒शून् । प॒शूनव॑ । अव॑ रुन्धे । रु॒न्धे॒ त्रिचि॑तीकम् । त्रिचि॑तीकम् चिन्वीत । त्रिचि॑तीक॒मिति॒ त्रि - चि॒ती॒क॒म् । चि॒न्वी॒त॒ द्वि॒तीय᳚म् । द्वि॒तीय॑म् चिन्वा॒नः । चि॒न्वा॒नस्त्रयः॑ । त्रय॑ इ॒मे । इ॒मे लो॒काः । लो॒का ए॒षु । ए॒ष्वे॑व । ए॒व लो॒केषु॑ ( ) । लो॒केषु॒ प्रति॑ \newline

\textbf{Jatai Paata} \newline

1. प्रज्ञा᳚त्यै॒ त्रयो॑दश॒ त्रयो॑दश॒ प्रज्ञा᳚त्यै॒ प्रज्ञा᳚त्यै॒ त्रयो॑दश । \newline
2. प्रज्ञा᳚त्या॒ इति॒ प्र - ज्ञा॒त्यै॒ । \newline
3. त्रयो॑दश लोकंपृ॒णा लो॑कंपृ॒णा स्त्रयो॑दश॒ त्रयो॑दश लोकंपृ॒णाः । \newline
4. त्रयो॑द॒शेति॒ त्रयः॑ - द॒श॒ । \newline
5. लो॒कं॒पृ॒णा उपोप॑ लोकंपृ॒णा लो॑कंपृ॒णा उप॑ । \newline
6. लो॒कं॒पृ॒णा इति॑ लोकं - पृ॒णाः । \newline
7. उप॑ दधाति दधा॒ त्युपोप॑ दधाति । \newline
8. द॒धा॒ त्येक॑विꣳशति॒ रेक॑विꣳशतिर् दधाति दधा॒ त्येक॑विꣳशतिः । \newline
9. एक॑विꣳशतिः॒ सꣳ स मेक॑विꣳशति॒ रेक॑विꣳशतिः॒ सम् । \newline
10. एक॑विꣳशति॒रित्येक॑ - विꣳ॒॒श॒तिः॒ । \newline
11. सम् प॑द्यन्ते पद्यन्ते॒ सꣳ सम् प॑द्यन्ते । \newline
12. प॒द्य॒न्ते॒ प्र॒ति॒ष्ठा प्र॑ति॒ष्ठा प॑द्यन्ते पद्यन्ते प्रति॒ष्ठा । \newline
13. प्र॒ति॒ष्ठा वै वै प्र॑ति॒ष्ठा प्र॑ति॒ष्ठा वै । \newline
14. प्र॒ति॒ष्ठेति॑ प्रति - स्था । \newline
15. वा ए॑कविꣳ॒॒श ए॑कविꣳ॒॒शो वै वा ए॑कविꣳ॒॒शः । \newline
16. ए॒क॒विꣳ॒॒शः प्र॑ति॒ष्ठा प्र॑ति॒ष्ठै क॑विꣳ॒॒श ए॑कविꣳ॒॒शः प्र॑ति॒ष्ठा । \newline
17. ए॒क॒विꣳ॒॒श इत्ये॑क - विꣳ॒॒शः । \newline
18. प्र॒ति॒ष्ठा गार्.ह॑पत्यो॒ गार्.ह॑पत्यः प्रति॒ष्ठा प्र॑ति॒ष्ठा गार्.ह॑पत्यः । \newline
19. प्र॒ति॒ष्ठेति॑ प्रति - स्था । \newline
20. गार्.ह॑पत्य एकविꣳ॒॒शस्यै॑ कविꣳ॒॒शस्य॒ गार्.ह॑पत्यो॒ गार्.ह॑पत्य एकविꣳ॒॒शस्य॑ । \newline
21. गार्.ह॑पत्य॒ इति॒ गार्.ह॑ - प॒त्यः॒ । \newline
22. ए॒क॒विꣳ॒॒श स्यै॒वैवै क॑विꣳ॒॒श स्यै॑कविꣳ॒॒शस्यै॒व । \newline
23. ए॒क॒विꣳ॒॒शस्येत्ये॑क - विꣳ॒॒शस्य॑ । \newline
24. ए॒व प्र॑ति॒ष्ठाम् प्र॑ति॒ष्ठा मे॒वैव प्र॑ति॒ष्ठाम् । \newline
25. प्र॒ति॒ष्ठाम् गार्.ह॑पत्य॒म् गार्.ह॑पत्यम् प्रति॒ष्ठाम् प्र॑ति॒ष्ठाम् गार्.ह॑पत्यम् । \newline
26. प्र॒ति॒ष्ठामिति॑ प्रति - स्थाम् । \newline
27. गार्.ह॑पत्य॒ मन्वनु॒ गार्.ह॑पत्य॒म् गार्.ह॑पत्य॒ मनु॑ । \newline
28. गार्.ह॑पत्य॒मिति॒ गार्.ह॑ - प॒त्य॒म् । \newline
29. अनु॒ प्रति॒ प्र त्यन्वनु॒ प्रति॑ । \newline
30. प्रति॑ तिष्ठति तिष्ठति॒ प्रति॒ प्रति॑ तिष्ठति । \newline
31. ति॒ष्ठ॒ति॒ प्रति॒ प्रति॑ तिष्ठति तिष्ठति॒ प्रति॑ । \newline
32. प्रत्य॒ग्नि म॒ग्निम् प्रति॒ प्रत्य॒ग्निम् । \newline
33. अ॒ग्निम् चि॑क्या॒न श्चि॑क्या॒नो᳚ ऽग्नि म॒ग्निम् चि॑क्या॒नः । \newline
34. चि॒क्या॒न स्ति॑ष्ठति तिष्ठति चिक्या॒न श्चि॑क्या॒न स्ति॑ष्ठति । \newline
35. ति॒ष्ठ॒ति॒ यो य स्ति॑ष्ठति तिष्ठति॒ यः । \newline
36. य ए॒व मे॒वं ॅयो य ए॒वम् । \newline
37. ए॒वं ॅवेद॒ वेदै॒व मे॒वं ॅवेद॑ । \newline
38. वेद॒ पञ्च॑चितीक॒म् पञ्च॑चितीकं॒ ॅवेद॒ वेद॒ पञ्च॑चितीकम् । \newline
39. पञ्च॑चितीकम् चिन्वीत चिन्वीत॒ पञ्च॑चितीक॒म् पञ्च॑चितीकम् चिन्वीत । \newline
40. पञ्च॑चितीक॒मिति॒ पञ्च॑ - चि॒ती॒क॒म् । \newline
41. चि॒न्वी॒त॒ प्र॒थ॒मम् प्र॑थ॒मम् चि॑न्वीत चिन्वीत प्रथ॒मम् । \newline
42. प्र॒थ॒मम् चि॑न्वा॒न श्चि॑न्वा॒नः प्र॑थ॒मम् प्र॑थ॒मम् चि॑न्वा॒नः । \newline
43. चि॒न्वा॒नः पाङ्क्तः॒ पाङ्क्त॑ श्चिन्वा॒न श्चि॑न्वा॒नः पाङ्क्तः॑ । \newline
44. पाङ्क्तो॑ य॒ज्ञो य॒ज्ञ्ः पाङ्क्तः॒ पाङ्क्तो॑ य॒ज्ञ्ः । \newline
45. य॒ज्ञ्ः पाङ्क्ताः॒ पाङ्क्ता॑ य॒ज्ञो य॒ज्ञ्ः पाङ्क्ताः᳚ । \newline
46. पाङ्क्ताः᳚ प॒शवः॑ प॒शवः॒ पाङ्क्ताः॒ पाङ्क्ताः᳚ प॒शवः॑ । \newline
47. प॒शवो॑ य॒ज्ञ्ं ॅय॒ज्ञ्म् प॒शवः॑ प॒शवो॑ य॒ज्ञ्म् । \newline
48. य॒ज्ञ् मे॒वैव य॒ज्ञ्ं ॅय॒ज्ञ् मे॒व । \newline
49. ए॒व प॒शून् प॒शू ने॒वैव प॒शून् । \newline
50. प॒शू नवाव॑ प॒शून् प॒शू नव॑ । \newline
51. अव॑ रुन्धे रु॒न्धे ऽवाव॑ रुन्धे । \newline
52. रु॒न्धे॒ त्रिचि॑तीक॒म् त्रिचि॑तीकꣳ रुन्धे रुन्धे॒ त्रिचि॑तीकम् । \newline
53. त्रिचि॑तीकम् चिन्वीत चिन्वीत॒ त्रिचि॑तीक॒म् त्रिचि॑तीकम् चिन्वीत । \newline
54. त्रिचि॑तीक॒मिति॒ त्रि - चि॒ती॒क॒म् । \newline
55. चि॒न्वी॒त॒ द्वि॒तीय॑म् द्वि॒तीय॑म् चिन्वीत चिन्वीत द्वि॒तीय᳚म् । \newline
56. द्वि॒तीय॑म् चिन्वा॒न श्चि॑न्वा॒नो द्वि॒तीय॑म् द्वि॒तीय॑म् चिन्वा॒नः । \newline
57. चि॒न्वा॒न स्त्रय॒ स्त्रय॑ श्चिन्वा॒न श्चि॑न्वा॒न स्त्रयः॑ । \newline
58. त्रय॑ इ॒म इ॒मे त्रय॒ स्त्रय॑ इ॒मे । \newline
59. इ॒मे लो॒का लो॒का इ॒म इ॒मे लो॒काः । \newline
60. लो॒का ए॒ष्वे॑षु लो॒का लो॒का ए॒षु । \newline
61. ए॒ष्वे॑वै वैष्वे᳚(1॒)ष्वे॑व । \newline
62. ए॒व लो॒केषु॑ लो॒केष्वे॒ वैव लो॒केषु॑ । \newline
63. लो॒केषु॒ प्रति॒ प्रति॑ लो॒केषु॑ लो॒केषु॒ प्रति॑ । \newline

\textbf{Ghana Paata } \newline

1. प्रज्ञा᳚त्यै॒ त्रयो॑दश॒ त्रयो॑दश॒ प्रज्ञा᳚त्यै॒ प्रज्ञा᳚त्यै॒ त्रयो॑दश लोकंपृ॒णा लो॑कंपृ॒णा स्त्रयो॑दश॒ प्रज्ञा᳚त्यै॒ प्रज्ञा᳚त्यै॒ त्रयो॑दश लोकंपृ॒णाः । \newline
2. प्रज्ञा᳚त्या॒ इति॒ प्र - ज्ञा॒त्यै॒ । \newline
3. त्रयो॑दश लोकंपृ॒णा लो॑कंपृ॒णा स्त्रयो॑दश॒ त्रयो॑दश लोकंपृ॒णा उपोप॑ लोकंपृ॒णा स्त्रयो॑दश॒ त्रयो॑दश लोकंपृ॒णा उप॑ । \newline
4. त्रयो॑द॒शेति॒ त्रयः॑ - द॒श॒ । \newline
5. लो॒कं॒पृ॒णा उपोप॑ लोकंपृ॒णा लो॑कंपृ॒णा उप॑ दधाति दधा॒त्युप॑ लोकंपृ॒णा लो॑कंपृ॒णा उप॑ दधाति । \newline
6. लो॒कं॒पृ॒णा इति॑ लोकं - पृ॒णाः । \newline
7. उप॑ दधाति दधा॒ त्युपोप॑ दधा॒ त्येक॑विꣳशति॒ रेक॑विꣳशतिर् दधा॒ त्युपोप॑ दधा॒ त्येक॑विꣳशतिः । \newline
8. द॒धा॒ त्येक॑विꣳशति॒ रेक॑विꣳशतिर् दधाति दधा॒ त्येक॑विꣳशतिः॒ सꣳ स मेक॑विꣳशतिर् दधाति दधा॒ त्येक॑विꣳशतिः॒ सम् । \newline
9. एक॑विꣳशतिः॒ सꣳ स मेक॑विꣳशति॒ रेक॑विꣳशतिः॒ सम् प॑द्यन्ते पद्यन्ते॒ स मेक॑विꣳशति॒ रेक॑विꣳशतिः॒ सम् प॑द्यन्ते । \newline
10. एक॑विꣳशति॒रित्येक॑ - विꣳ॒॒श॒तिः॒ । \newline
11. सम् प॑द्यन्ते पद्यन्ते॒ सꣳ सम् प॑द्यन्ते प्रति॒ष्ठा प्र॑ति॒ष्ठा प॑द्यन्ते॒ सꣳ सम् प॑द्यन्ते प्रति॒ष्ठा । \newline
12. प॒द्य॒न्ते॒ प्र॒ति॒ष्ठा प्र॑ति॒ष्ठा प॑द्यन्ते पद्यन्ते प्रति॒ष्ठा वै वै प्र॑ति॒ष्ठा प॑द्यन्ते पद्यन्ते प्रति॒ष्ठा वै । \newline
13. प्र॒ति॒ष्ठा वै वै प्र॑ति॒ष्ठा प्र॑ति॒ष्ठा वा ए॑कविꣳ॒॒श ए॑कविꣳ॒॒शो वै प्र॑ति॒ष्ठा प्र॑ति॒ष्ठा वा ए॑कविꣳ॒॒शः । \newline
14. प्र॒ति॒ष्ठेति॑ प्रति - स्था । \newline
15. वा ए॑कविꣳ॒॒श ए॑कविꣳ॒॒शो वै वा ए॑कविꣳ॒॒शः प्र॑ति॒ष्ठा प्र॑ति॒ष्ठैक॑विꣳ॒॒शो वै वा ए॑कविꣳ॒॒शः प्र॑ति॒ष्ठा । \newline
16. ए॒क॒विꣳ॒॒शः प्र॑ति॒ष्ठा प्र॑ति॒ष्ठैक॑विꣳ॒॒श ए॑कविꣳ॒॒शः प्र॑ति॒ष्ठा गार्.ह॑पत्यो॒ गार्.ह॑पत्यः प्रति॒ष्ठैक॑विꣳ॒॒श ए॑कविꣳ॒॒शः प्र॑ति॒ष्ठा गार्.ह॑पत्यः । \newline
17. ए॒क॒विꣳ॒॒श इत्ये॑क - विꣳ॒॒शः । \newline
18. प्र॒ति॒ष्ठा गार्.ह॑पत्यो॒ गार्.ह॑पत्यः प्रति॒ष्ठा प्र॑ति॒ष्ठा गार्.ह॑पत्य एकविꣳ॒॒श स्यै॑कविꣳ॒॒शस्य॒ गार्.ह॑पत्यः प्रति॒ष्ठा प्र॑ति॒ष्ठा गार्.ह॑पत्य एकविꣳ॒॒शस्य॑ । \newline
19. प्र॒ति॒ष्ठेति॑ प्रति - स्था । \newline
20. गार्.ह॑पत्य एकविꣳ॒॒श स्यै॑कविꣳ॒॒शस्य॒ गार्.ह॑पत्यो॒ गार्.ह॑पत्य एकविꣳ॒॒श स्यै॒वैवैक॑विꣳ॒॒शस्य॒ गार्.ह॑पत्यो॒ गार्.ह॑पत्य एकविꣳ॒॒शस्यै॒व । \newline
21. गार्.ह॑पत्य॒ इति॒ गार्.ह॑ - प॒त्यः॒ । \newline
22. ए॒क॒विꣳ॒॒श स्यै॒वैवैक॑विꣳ॒॒श स्यै॑कविꣳ॒॒शस्यै॒व प्र॑ति॒ष्ठाम् प्र॑ति॒ष्ठा मे॒वैक॑विꣳ॒॒श स्यै॑कविꣳ॒॒शस्यै॒व प्र॑ति॒ष्ठाम् । \newline
23. ए॒क॒विꣳ॒॒शस्येत्ये॑क - विꣳ॒॒शस्य॑ । \newline
24. ए॒व प्र॑ति॒ष्ठाम् प्र॑ति॒ष्ठा मे॒वैव प्र॑ति॒ष्ठाम् गार्.ह॑पत्य॒म् गार्.ह॑पत्यम् प्रति॒ष्ठा मे॒वैव प्र॑ति॒ष्ठाम् गार्.ह॑पत्यम् । \newline
25. प्र॒ति॒ष्ठाम् गार्.ह॑पत्य॒म् गार्.ह॑पत्यम् प्रति॒ष्ठाम् प्र॑ति॒ष्ठाम् गार्.ह॑पत्य॒ मन्वनु॒ गार्.ह॑पत्यम् प्रति॒ष्ठाम् प्र॑ति॒ष्ठाम् गार्.ह॑पत्य॒ मनु॑ । \newline
26. प्र॒ति॒ष्ठामिति॑ प्रति - स्थाम् । \newline
27. गार्.ह॑पत्य॒ मन्वनु॒ गार्.ह॑पत्य॒म् गार्.ह॑पत्य॒ मनु॒ प्रति॒ प्रत्यनु॒ गार्.ह॑पत्य॒म् गार्.ह॑पत्य॒ मनु॒ प्रति॑ । \newline
28. गार्.ह॑पत्य॒मिति॒ गार्.ह॑ - प॒त्य॒म् । \newline
29. अनु॒ प्रति॒ प्रत्यन्वनु॒ प्रति॑ तिष्ठति तिष्ठति॒ प्रत्यन्वनु॒ प्रति॑ तिष्ठति । \newline
30. प्रति॑ तिष्ठति तिष्ठति॒ प्रति॒ प्रति॑ तिष्ठति॒ प्रति॒ प्रति॑ तिष्ठति॒ प्रति॒ प्रति॑ तिष्ठति॒ प्रति॑ । \newline
31. ति॒ष्ठ॒ति॒ प्रति॒ प्रति॑ तिष्ठति तिष्ठति॒ प्रत्य॒ग्नि म॒ग्निम् प्रति॑ तिष्ठति तिष्ठति॒ प्रत्य॒ग्निम् । \newline
32. प्रत्य॒ग्नि म॒ग्निम् प्रति॒ प्रत्य॒ग्निम् चि॑क्या॒न श्चि॑क्या॒नो᳚ ऽग्निम् प्रति॒ प्रत्य॒ग्निम् चि॑क्या॒नः । \newline
33. अ॒ग्निम् चि॑क्या॒न श्चि॑क्या॒नो᳚ ऽग्नि म॒ग्निम् चि॑क्या॒न स्ति॑ष्ठति तिष्ठति चिक्या॒नो᳚ ऽग्नि म॒ग्निम् चि॑क्या॒न स्ति॑ष्ठति । \newline
34. चि॒क्या॒न स्ति॑ष्ठति तिष्ठति चिक्या॒न श्चि॑क्या॒न स्ति॑ष्ठति॒ यो य स्ति॑ष्ठति चिक्या॒न श्चि॑क्या॒न स्ति॑ष्ठति॒ यः । \newline
35. ति॒ष्ठ॒ति॒ यो य स्ति॑ष्ठति तिष्ठति॒ य ए॒व मे॒वं ॅय स्ति॑ष्ठति तिष्ठति॒ य ए॒वम् । \newline
36. य ए॒व मे॒वं ॅयो य ए॒वं ॅवेद॒ वेदै॒वं ॅयो य ए॒वं ॅवेद॑ । \newline
37. ए॒वं ॅवेद॒ वेदै॒व मे॒वं ॅवेद॒ पञ्च॑चितीक॒म् पञ्च॑चितीकं॒ ॅवेदै॒व मे॒वं ॅवेद॒ पञ्च॑चितीकम् । \newline
38. वेद॒ पञ्च॑चितीक॒म् पञ्च॑चितीकं॒ ॅवेद॒ वेद॒ पञ्च॑चितीकम् चिन्वीत चिन्वीत॒ पञ्च॑चितीकं॒ ॅवेद॒ वेद॒ पञ्च॑चितीकम् चिन्वीत । \newline
39. पञ्च॑चितीकम् चिन्वीत चिन्वीत॒ पञ्च॑चितीक॒म् पञ्च॑चितीकम् चिन्वीत प्रथ॒मम् प्र॑थ॒मम् चि॑न्वीत॒ पञ्च॑चितीक॒म् पञ्च॑चितीकम् चिन्वीत प्रथ॒मम् । \newline
40. पञ्च॑चितीक॒मिति॒ पञ्च॑ - चि॒ती॒क॒म् । \newline
41. चि॒न्वी॒त॒ प्र॒थ॒मम् प्र॑थ॒मम् चि॑न्वीत चिन्वीत प्रथ॒मम् चि॑न्वा॒न श्चि॑न्वा॒नः प्र॑थ॒मम् चि॑न्वीत चिन्वीत प्रथ॒मम् चि॑न्वा॒नः । \newline
42. प्र॒थ॒मम् चि॑न्वा॒न श्चि॑न्वा॒नः प्र॑थ॒मम् प्र॑थ॒मम् चि॑न्वा॒नः पाङ्क्तः॒ पाङ्क्त॑ श्चिन्वा॒नः प्र॑थ॒मम् प्र॑थ॒मम् चि॑न्वा॒नः पाङ्क्तः॑ । \newline
43. चि॒न्वा॒नः पाङ्क्तः॒ पाङ्क्त॑ श्चिन्वा॒न श्चि॑न्वा॒नः पाङ्क्तो॑ य॒ज्ञो य॒ज्ञ्ः पाङ्क्त॑ श्चिन्वा॒न श्चि॑न्वा॒नः पाङ्क्तो॑ य॒ज्ञ्ः । \newline
44. पाङ्क्तो॑ य॒ज्ञो य॒ज्ञ्ः पाङ्क्तः॒ पाङ्क्तो॑ य॒ज्ञ्ः पाङ्क्ताः॒ पाङ्क्ता॑ य॒ज्ञ्ः पाङ्क्तः॒ पाङ्क्तो॑ य॒ज्ञ्ः पाङ्क्ताः᳚ । \newline
45. य॒ज्ञ्ः पाङ्क्ताः॒ पाङ्क्ता॑ य॒ज्ञो य॒ज्ञ्ः पाङ्क्ताः᳚ प॒शवः॑ प॒शवः॒ पाङ्क्ता॑ य॒ज्ञो य॒ज्ञ्ः पाङ्क्ताः᳚ प॒शवः॑ । \newline
46. पाङ्क्ताः᳚ प॒शवः॑ प॒शवः॒ पाङ्क्ताः॒ पाङ्क्ताः᳚ प॒शवो॑ य॒ज्ञ्ं ॅय॒ज्ञ्म् प॒शवः॒ पाङ्क्ताः॒ पाङ्क्ताः᳚ प॒शवो॑ य॒ज्ञ्म् । \newline
47. प॒शवो॑ य॒ज्ञ्ं ॅय॒ज्ञ्म् प॒शवः॑ प॒शवो॑ य॒ज्ञ् मे॒वैव य॒ज्ञ्म् प॒शवः॑ प॒शवो॑ य॒ज्ञ् मे॒व । \newline
48. य॒ज्ञ् मे॒वैव य॒ज्ञ्ं ॅय॒ज्ञ् मे॒व प॒शून् प॒शू ने॒व य॒ज्ञ्ं ॅय॒ज्ञ् मे॒व प॒शून् । \newline
49. ए॒व प॒शून् प॒शू ने॒वैव प॒शू नवाव॑ प॒शू ने॒वैव प॒शू नव॑ । \newline
50. प॒शू नवाव॑ प॒शून् प॒शू नव॑ रुन्धे रु॒न्धे ऽव॑ प॒शून् प॒शू नव॑ रुन्धे । \newline
51. अव॑ रुन्धे रु॒न्धे ऽवाव॑ रुन्धे॒ त्रिचि॑तीक॒म् त्रिचि॑तीकꣳ रु॒न्धे ऽवाव॑ रुन्धे॒ त्रिचि॑तीकम् । \newline
52. रु॒न्धे॒ त्रिचि॑तीक॒म् त्रिचि॑तीकꣳ रुन्धे रुन्धे॒ त्रिचि॑तीकम् चिन्वीत चिन्वीत॒ त्रिचि॑तीकꣳ रुन्धे रुन्धे॒ त्रिचि॑तीकम् चिन्वीत । \newline
53. त्रिचि॑तीकम् चिन्वीत चिन्वीत॒ त्रिचि॑तीक॒म् त्रिचि॑तीकम् चिन्वीत द्वि॒तीय॑म् द्वि॒तीय॑म् चिन्वीत॒ त्रिचि॑तीक॒म् त्रिचि॑तीकम् चिन्वीत द्वि॒तीय᳚म् । \newline
54. त्रिचि॑तीक॒मिति॒ त्रि - चि॒ती॒क॒म् । \newline
55. चि॒न्वी॒त॒ द्वि॒तीय॑म् द्वि॒तीय॑म् चिन्वीत चिन्वीत द्वि॒तीय॑म् चिन्वा॒न श्चि॑न्वा॒नो द्वि॒तीय॑म् चिन्वीत चिन्वीत द्वि॒तीय॑म् चिन्वा॒नः । \newline
56. द्वि॒तीय॑म् चिन्वा॒न श्चि॑न्वा॒नो द्वि॒तीय॑म् द्वि॒तीय॑म् चिन्वा॒न स्त्रय॒ स्त्रय॑ श्चिन्वा॒नो द्वि॒तीय॑म् द्वि॒तीय॑म् चिन्वा॒न स्त्रयः॑ । \newline
57. चि॒न्वा॒न स्त्रय॒ स्त्रय॑ श्चिन्वा॒न श्चि॑न्वा॒न स्त्रय॑ इ॒म इ॒मे त्रय॑ श्चिन्वा॒न श्चि॑न्वा॒न स्त्रय॑ इ॒मे । \newline
58. त्रय॑ इ॒म इ॒मे त्रय॒ स्त्रय॑ इ॒मे लो॒का लो॒का इ॒मे त्रय॒ स्त्रय॑ इ॒मे लो॒काः । \newline
59. इ॒मे लो॒का लो॒का इ॒म इ॒मे लो॒का ए॒ष्वे॑षु लो॒का इ॒म इ॒मे लो॒का ए॒षु । \newline
60. लो॒का ए॒ष्वे॑षु लो॒का लो॒का ए॒ष्वे॑ वैवैषु लो॒का लो॒का ए॒ष्वे॑व । \newline
61. ए॒ष्वे॑ वैवैष्वे᳚(1॒)ष्वे॑व लो॒केषु॑ लो॒के ष्वे॒वैष्वे᳚(1॒)ष्वे॑व लो॒केषु॑ । \newline
62. ए॒व लो॒केषु॑ लो॒के ष्वे॒वैव लो॒केषु॒ प्रति॒ प्रति॑ लो॒के ष्वे॒वैव लो॒केषु॒ प्रति॑ । \newline
63. लो॒केषु॒ प्रति॒ प्रति॑ लो॒केषु॑ लो॒केषु॒ प्रति॑ तिष्ठति तिष्ठति॒ प्रति॑ लो॒केषु॑ लो॒केषु॒ प्रति॑ तिष्ठति । \newline
\pagebreak
\markright{ TS 5.2.3.7  \hfill https://www.vedavms.in \hfill}

\section{ TS 5.2.3.7 }

\textbf{TS 5.2.3.7 } \newline
\textbf{Samhita Paata} \newline

प्रति॑तिष्ठ॒-त्येक॑चितीकं चिन्वीत तृ॒तीयं॑ चिन्वा॒न ए॑क॒धा वै सु॑व॒र्गो लो॒क ए॑क॒वृतै॒व सु॑वर्गं ॅलो॒कमे॑ति॒ पुरी॑षेणा॒भ्यू॑हति॒ तस्मा᳚न्माꣳ॒॒ सेनास्थि॑ छ॒न्नं न दु॒श्चर्मा॑ भवति॒ य ए॒वं ॅवेद॒ पञ्च॒ चित॑यो भवन्ति प॒ञ्चभिः॒ पुरी॑षैर॒भ्यू॑हति॒ दश॒ संप॑द्यन्ते॒ दशा᳚क्षरा वि॒राडन्नं॑ ॅवि॒राड् वि॒राज्ये॒वाऽन्नाद्ये॒ प्रति॑तिष्ठति ॥ \newline

\textbf{Pada Paata} \newline

प्रतीति॑ । ति॒ष्ठ॒ति॒ । एक॑चितीक॒मित्येक॑ - चि॒ती॒क॒म् । चि॒न्वी॒त॒ । तृ॒तीय᳚म् । चि॒न्वा॒नः । ए॒क॒धेत्ये॑क - धा । वै । सु॒व॒र्ग इति॑ सुवः - गः । लो॒कः । ए॒क॒वृतेत्ये॑क - वृता᳚ । ए॒व । सु॒व॒र्गमिति॑ सुवः - गम् । लो॒कम् । ए॒ति॒ । पुरी॑षेण । अ॒भीति॑ । ऊ॒ह॒ति॒ । तस्मा᳚त् । माꣳ॒॒सेन॑ । अस्थि॑ । छ॒न्नम् । न । दु॒श्चर्मेति॑ दुः - चर्मा᳚ । भ॒व॒ति॒ । यः । ए॒वम् । वेद॑ । पञ्च॑ । चित॑यः । भ॒व॒न्ति॒ । प॒ञ्चभि॒रिति॑ प॒ञ्च - भिः॒ । पुरी॑षैः । अ॒भीति॑ । ऊ॒ह॒ति॒ । दश॑ । समिति॑ । प॒द्य॒न्ते॒ । दशा᳚क्ष॒रेति॒ दश॑ - अ॒क्ष॒रा॒ । वि॒राडिति॑ वि-राट् । अन्न᳚म् । वि॒राडिति॑ वि - राट् । वि॒राजीति॑ वि - राजि॑ । ए॒व । अ॒न्नाद्य॒ इत्य॑न्न - अद्ये᳚ । प्रतीति॑ । ति॒ष्ठ॒ति॒ ॥  \newline


\textbf{Krama Paata} \newline

प्रति॑ तिष्ठति । ति॒ष्ठ॒त्येक॑चितीकम् । एक॑चितीकम् चिन्वीत । एक॑चितीक॒मित्येक॑ - चि॒ती॒क॒म् । चि॒न्वी॒त॒ तृ॒तीय᳚म् । तृ॒तीय॑म् चिन्वा॒नः । चि॒न्वा॒न ए॑क॒धा । ए॒क॒धा वै । ए॒क॒धेत्ये॑क - धा । वै सु॑व॒र्गः । सु॒व॒र्गो लो॒कः । सु॒व॒र्ग इति॑ सुवः - गः । लो॒क ए॑क॒वृता᳚ । ए॒क॒वृतै॒व । ए॒क॒वृतेत्ये॑क - वृता᳚ । ए॒व सु॑व॒र्गम् । सु॒व॒र्गम् ॅलो॒कम् । सु॒व॒र्गमिति॑ सुवः - गम् । लो॒कमे॑ति । ए॒ति॒ पुरी॑षेण । पुरी॑षेणा॒भि । अ॒भ्यू॑हति । ऊ॒ह॒ति॒ तस्मा᳚त् । तस्मा᳚न् माꣳ॒॒सेन॑ । माꣳ॒॒सेनास्थि॑ । अस्थि॑ छ॒न्नम् । छ॒न्नम् न । न दु॒श्चर्मा᳚ । दु॒श्चर्मा॑ भवति । दु॒श्चर्मेति॑ दुः - चर्मा᳚ । भ॒व॒ति॒ यः । य ए॒वम् । ए॒वम् ॅवेद॑ । वेद॒ पञ्च॑ । पञ्च॒ चित॑यः । चित॑यो भवन्ति । भ॒व॒न्ति॒ प॒ञ्चभिः॑ । प॒ञ्चभिः॒ पुरी॑षैः । प॒ञ्चभि॒रिति॑ प॒ञ्च - भिः॒ । पुरी॑षैर॒भि । अ॒भ्यू॑हति । ऊ॒ह॒ति॒ दश॑ । दश॒ सम् । सम् प॑द्यन्ते । प॒द्य॒न्ते॒ दशा᳚क्षरा । दशा᳚क्षरा वि॒राट् । दशा᳚क्ष॒रेति॒ दश॑ - अ॒क्ष॒रा॒ । वि॒राडन्न᳚म् । वि॒राडिति॑ वि - राट् । अन्न॑म् ॅवि॒राट् । वि॒राड् वि॒राजि॑ । वि॒राडिति॑ वि - राट् । वि॒राज्ये॒व । वि॒राजीति॑ वि - राजि॑ । ए॒वान्नाद्ये᳚ । अ॒न्नाद्ये॒ प्रति॑ । अ॒न्नाद्य॒ इत्य॑न्न - अद्ये᳚ । प्रति॑ तिष्ठति । ति॒ष्ठ॒तीति॑ तिष्ठति । \newline

\textbf{Jatai Paata} \newline

1. प्रति॑ तिष्ठति तिष्ठति॒ प्रति॒ प्रति॑ तिष्ठति । \newline
2. ति॒ष्ठ॒ त्येक॑चितीक॒ मेक॑चितीकम् तिष्ठति तिष्ठ॒ त्येक॑चितीकम् । \newline
3. एक॑चितीकम् चिन्वीत चिन्वी॒तै क॑चितीक॒ मेक॑चितीकम् चिन्वीत । \newline
4. एक॑चितीक॒मित्येक॑ - चि॒ती॒क॒म् । \newline
5. चि॒न्वी॒त॒ तृ॒तीय॑म् तृ॒तीय॑म् चिन्वीत चिन्वीत तृ॒तीय᳚म् । \newline
6. तृ॒तीय॑म् चिन्वा॒न श्चि॑न्वा॒न स्तृ॒तीय॑म् तृ॒तीय॑म् चिन्वा॒नः । \newline
7. चि॒न्वा॒न ए॑क॒ धैक॒धा चि॑न्वा॒न श्चि॑न्वा॒न ए॑क॒धा । \newline
8. ए॒क॒धा वै वा ए॑क॒ धैक॒धा वै । \newline
9. ए॒क॒धेत्ये॑क - धा । \newline
10. वै सु॑व॒र्गः सु॑व॒र्गो वै वै सु॑व॒र्गः । \newline
11. सु॒व॒र्गो लो॒को लो॒कः सु॑व॒र्गः सु॑व॒र्गो लो॒कः । \newline
12. सु॒व॒र्ग इति॑ सुवः - गः । \newline
13. लो॒क ए॑क॒वृतै॑ क॒वृता॑ लो॒को लो॒क ए॑क॒वृता᳚ । \newline
14. ए॒क॒वृ तै॒वै वैक॒वृ तै॑क॒वृतै॒व । \newline
15. ए॒क॒वृतेत्ये॑क - वृता᳚ । \newline
16. ए॒व सु॑व॒र्गꣳ सु॑व॒र्ग मे॒वैव सु॑व॒र्गम् । \newline
17. सु॒व॒र्गम् ॅलो॒कम् ॅलो॒कꣳ सु॑व॒र्गꣳ सु॑व॒र्गम् ॅलो॒कम् । \newline
18. सु॒व॒र्गमिति॑ सुवः - गम् । \newline
19. लो॒क मे᳚त्येति लो॒कम् ॅलो॒क मे॑ति । \newline
20. ए॒ति॒ पुरी॑षेण॒ पुरी॑षेणै त्येति॒ पुरी॑षेण । \newline
21. पुरी॑षेणा॒ भ्य॑भि पुरी॑षेण॒ पुरी॑षेणा॒भि । \newline
22. अ॒भ्यू॑ह त्यूह त्य॒भ्या᳚(1॒) भ्यू॑हति । \newline
23. ऊ॒ह॒ति॒ तस्मा॒त् तस्मा॑ दूह त्यूहति॒ तस्मा᳚त् । \newline
24. तस्मा᳚न् माꣳ॒॒सेन॑ माꣳ॒॒सेन॒ तस्मा॒त् तस्मा᳚न् माꣳ॒॒सेन॑ । \newline
25. माꣳ॒॒सेना स्थ्यस्थि॑ माꣳ॒॒सेन॑ माꣳ॒॒सेनास्थि॑ । \newline
26. अस्थि॑ छ॒न्नम् छ॒न्न मस्थ्यस्थि॑ छ॒न्नम् । \newline
27. छ॒न्नम् न न छ॒न्नम् छ॒न्नम् न । \newline
28. न दु॒श्चर्मा॑ दु॒श्चर्मा॒ न न दु॒श्चर्मा᳚ । \newline
29. दु॒श्चर्मा॑ भवति भवति दु॒श्चर्मा॑ दु॒श्चर्मा॑ भवति । \newline
30. दु॒श्चर्मेति॑ दुः - चर्मा᳚ । \newline
31. भ॒व॒ति॒ यो यो भ॑वति भवति॒ यः । \newline
32. य ए॒व मे॒वं ॅयो य ए॒वम् । \newline
33. ए॒वं ॅवेद॒ वेदै॒व मे॒वं ॅवेद॑ । \newline
34. वेद॒ पञ्च॒ पञ्च॒ वेद॒ वेद॒ पञ्च॑ । \newline
35. पञ्च॒ चित॑य॒ श्चित॑यः॒ पञ्च॒ पञ्च॒ चित॑यः । \newline
36. चित॑यो भवन्ति भवन्ति॒ चित॑य॒ श्चित॑यो भवन्ति । \newline
37. भ॒व॒न्ति॒ प॒ञ्चभिः॑ प॒ञ्चभि॑र् भवन्ति भवन्ति प॒ञ्चभिः॑ । \newline
38. प॒ञ्चभिः॒ पुरी॑षैः॒ पुरी॑षैः प॒ञ्चभिः॑ प॒ञ्चभिः॒ पुरी॑षैः । \newline
39. प॒ञ्चभि॒रिति॑ प॒ञ्च - भिः॒ । \newline
40. पुरी॑षै र॒भ्य॑भि पुरी॑षैः॒ पुरी॑षै र॒भि । \newline
41. अ॒भ्यू॑ह त्यूह त्य॒भ्या᳚(1॒) भ्यू॑हति । \newline
42. ऊ॒ह॒ति॒ दश॒ दशो॑ हत्यूहति॒ दश॑ । \newline
43. दश॒ सꣳ सम् दश॒ दश॒ सम् । \newline
44. सम् प॑द्यन्ते पद्यन्ते॒ सꣳ सम् प॑द्यन्ते । \newline
45. प॒द्य॒न्ते॒ दशा᳚क्षरा॒ दशा᳚क्षरा पद्यन्ते पद्यन्ते॒ दशा᳚क्षरा । \newline
46. दशा᳚क्षरा वि॒राड् वि॒राड् दशा᳚क्षरा॒ दशा᳚क्षरा वि॒राट् । \newline
47. दशा᳚क्ष॒रेति॒ दश॑ - अ॒क्ष॒रा॒ । \newline
48. वि॒रा डन्न॒ मन्नं॑ ॅवि॒राड् वि॒रा डन्न᳚म् । \newline
49. वि॒राडिति॑ वि - राट् । \newline
50. अन्नं॑ ॅवि॒राड् वि॒रा डन्न॒ मन्नं॑ ॅवि॒राट् । \newline
51. वि॒राड् वि॒राजि॑ वि॒राजि॑ वि॒राड् वि॒राड् वि॒राजि॑ । \newline
52. वि॒राडिति॑ वि - राट् । \newline
53. वि॒रा ज्ये॒वैव वि॒राजि॑ वि॒रा ज्ये॒व । \newline
54. वि॒राजीति॑ वि - राजि॑ । \newline
55. ए॒वान्नाद्ये॒ ऽन्नाद्य॑ ए॒वै वान्नाद्ये᳚ । \newline
56. अ॒न्नाद्ये॒ प्रति॒ प्रत्य॒न्नाद्ये॒ ऽन्नाद्ये॒ प्रति॑ । \newline
57. अ॒न्नाद्य॒ इत्य॑न्न - अद्ये᳚ । \newline
58. प्रति॑ तिष्ठति तिष्ठति॒ प्रति॒ प्रति॑ तिष्ठति । \newline
59. ति॒ष्ठ॒तीति॑ तिष्ठति । \newline

\textbf{Ghana Paata } \newline

1. प्रति॑ तिष्ठति तिष्ठति॒ प्रति॒ प्रति॑ तिष्ठ॒ त्येक॑चितीक॒ मेक॑चितीकम् तिष्ठति॒ प्रति॒ प्रति॑ तिष्ठ॒ त्येक॑चितीकम् । \newline
2. ति॒ष्ठ॒ त्येक॑चितीक॒ मेक॑चितीकम् तिष्ठति तिष्ठ॒ त्येक॑चितीकम् चिन्वीत चिन्वी॒ तैक॑चितीकम् तिष्ठति तिष्ठ॒ त्येक॑चितीकम् चिन्वीत । \newline
3. एक॑चितीकम् चिन्वीत चिन्वी॒ तैक॑चितीक॒ मेक॑चितीकम् चिन्वीत तृ॒तीय॑म् तृ॒तीय॑म् चिन्वी॒ तैक॑चितीक॒ मेक॑चितीकम् चिन्वीत तृ॒तीय᳚म् । \newline
4. एक॑चितीक॒मित्येक॑ - चि॒ती॒क॒म् । \newline
5. चि॒न्वी॒त॒ तृ॒तीय॑म् तृ॒तीय॑म् चिन्वीत चिन्वीत तृ॒तीय॑म् चिन्वा॒न श्चि॑न्वा॒न स्तृ॒तीय॑म् चिन्वीत चिन्वीत तृ॒तीय॑म् चिन्वा॒नः । \newline
6. तृ॒तीय॑म् चिन्वा॒न श्चि॑न्वा॒न स्तृ॒तीय॑म् तृ॒तीय॑म् चिन्वा॒न ए॑क॒ धैक॒धा चि॑न्वा॒न स्तृ॒तीय॑म् तृ॒तीय॑म् चिन्वा॒न ए॑क॒धा । \newline
7. चि॒न्वा॒न ए॑क॒ धैक॒धा चि॑न्वा॒न श्चि॑न्वा॒न ए॑क॒धा वै वा ए॑क॒धा चि॑न्वा॒न श्चि॑न्वा॒न ए॑क॒धा वै । \newline
8. ए॒क॒धा वै वा ए॑क॒ धैक॒धा वै सु॑व॒र्गः सु॑व॒र्गो वा ए॑क॒ धैक॒धा वै सु॑व॒र्गः । \newline
9. ए॒क॒धेत्ये॑क - धा । \newline
10. वै सु॑व॒र्गः सु॑व॒र्गो वै वै सु॑व॒र्गो लो॒को लो॒कः सु॑व॒र्गो वै वै सु॑व॒र्गो लो॒कः । \newline
11. सु॒व॒र्गो लो॒को लो॒कः सु॑व॒र्गः सु॑व॒र्गो लो॒क ए॑क॒वृ तै॑क॒वृता॑ लो॒कः सु॑व॒र्गः सु॑व॒र्गो लो॒क ए॑क॒वृता᳚ । \newline
12. सु॒व॒र्ग इति॑ सुवः - गः । \newline
13. लो॒क ए॑क॒वृ तै॑क॒वृता॑ लो॒को लो॒क ए॑क॒वृ तै॒वै वैक॒वृता॑ लो॒को लो॒क ए॑क॒वृतै॒व । \newline
14. ए॒क॒वृतै॒ वैवैक॒वृ तै॑क॒वृतै॒व सु॑व॒र्गꣳ सु॑व॒र्ग मे॒वैक॒वृ तै॑क॒वृतै॒व सु॑व॒र्गम् । \newline
15. ए॒क॒वृतेत्ये॑क - वृता᳚ । \newline
16. ए॒व सु॑व॒र्गꣳ सु॑व॒र्ग मे॒वैव सु॑व॒र्गम् ॅलो॒कम् ॅलो॒कꣳ सु॑व॒र्ग मे॒वैव सु॑व॒र्गम् ॅलो॒कम् । \newline
17. सु॒व॒र्गम् ॅलो॒कम् ॅलो॒कꣳ सु॑व॒र्गꣳ सु॑व॒र्गम् ॅलो॒क मे᳚त्येति लो॒कꣳ सु॑व॒र्गꣳ सु॑व॒र्गम् ॅलो॒क मे॑ति । \newline
18. सु॒व॒र्गमिति॑ सुवः - गम् । \newline
19. लो॒क मे᳚त्येति लो॒कम् ॅलो॒क मे॑ति॒ पुरी॑षेण॒ पुरी॑षे णैति लो॒कम् ॅलो॒क मे॑ति॒ पुरी॑षेण । \newline
20. ए॒ति॒ पुरी॑षेण॒ पुरी॑षे णैत्येति॒ पुरी॑षेणा॒भ्य॑भि पुरी॑षे णैत्येति॒ पुरी॑षेणा॒भि । \newline
21. पुरी॑षेणा॒भ्य॑भि पुरी॑षेण॒ पुरी॑षेणा॒ भ्यू॑ह त्यूह त्य॒भि पुरी॑षेण॒ पुरी॑षेणा॒ भ्यू॑हति । \newline
22. अ॒भ्यू॑ह त्यूह त्य॒भ्या᳚(1॒)भ्यू॑हति॒ तस्मा॒त् तस्मा॑ दूह त्य॒भ्या᳚(1॒)भ्यू॑हति॒ तस्मा᳚त् । \newline
23. ऊ॒ह॒ति॒ तस्मा॒त् तस्मा॑ दूह त्यूहति॒ तस्मा᳚न् माꣳ॒॒सेन॑ माꣳ॒॒सेन॒ तस्मा॑ दूह त्यूहति॒ तस्मा᳚न् माꣳ॒॒सेन॑ । \newline
24. तस्मा᳚न् माꣳ॒॒सेन॑ माꣳ॒॒सेन॒ तस्मा॒त् तस्मा᳚न् माꣳ॒॒सेना स्थ्यस्थि॑ माꣳ॒॒सेन॒ तस्मा॒त् तस्मा᳚न् माꣳ॒॒सेनास्थि॑ । \newline
25. माꣳ॒॒सेना स्थ्यस्थि॑ माꣳ॒॒सेन॑ माꣳ॒॒सेनास्थि॑ छ॒न्नम् छ॒न्न मस्थि॑ माꣳ॒॒सेन॑ माꣳ॒॒सेनास्थि॑ छ॒न्नम् । \newline
26. अस्थि॑ छ॒न्नम् छ॒म् न मस्थ्यस्थि॑ छ॒न्नन्न न छ॒न्न मस्थ्यस्थि॑ छ॒न्नम् न । \newline
27. छ॒न्नम् न न छ॒न्नम् छ॒न्नम् न दु॒श्चर्मा॑ दु॒श्चर्मा॒ न छ॒न्नम् छ॒न्नम् न दु॒श्चर्मा᳚ । \newline
28. न दु॒श्चर्मा॑ दु॒श्चर्मा॒ न न दु॒श्चर्मा॑ भवति भवति दु॒श्चर्मा॒ न न दु॒श्चर्मा॑ भवति । \newline
29. दु॒श्चर्मा॑ भवति भवति दु॒श्चर्मा॑ दु॒श्चर्मा॑ भवति॒ यो यो भ॑वति दु॒श्चर्मा॑ दु॒श्चर्मा॑ भवति॒ यः । \newline
30. दु॒श्चर्मेति॑ दुः - चर्मा᳚ । \newline
31. भ॒व॒ति॒ यो यो भ॑वति भवति॒ य ए॒व मे॒वं ॅयो भ॑वति भवति॒ य ए॒वम् । \newline
32. य ए॒व मे॒वं ॅयो य ए॒वं ॅवेद॒ वेदै॒वं ॅयो य ए॒वं ॅवेद॑ । \newline
33. ए॒वं ॅवेद॒ वेदै॒व मे॒वं ॅवेद॒ पञ्च॒ पञ्च॒ वेदै॒व मे॒वं ॅवेद॒ पञ्च॑ । \newline
34. वेद॒ पञ्च॒ पञ्च॒ वेद॒ वेद॒ पञ्च॒ चित॑य॒ श्चित॑यः॒ पञ्च॒ वेद॒ वेद॒ पञ्च॒ चित॑यः । \newline
35. पञ्च॒ चित॑य॒ श्चित॑यः॒ पञ्च॒ पञ्च॒ चित॑यो भवन्ति भवन्ति॒ चित॑यः॒ पञ्च॒ पञ्च॒ चित॑यो भवन्ति । \newline
36. चित॑यो भवन्ति भवन्ति॒ चित॑य॒ श्चित॑यो भवन्ति प॒ञ्चभिः॑ प॒ञ्चभि॑र् भवन्ति॒ चित॑य॒ श्चित॑यो भवन्ति प॒ञ्चभिः॑ । \newline
37. भ॒व॒न्ति॒ प॒ञ्चभिः॑ प॒ञ्चभि॑र् भवन्ति भवन्ति प॒ञ्चभिः॒ पुरी॑षैः॒ पुरी॑षैः प॒ञ्चभि॑र् भवन्ति भवन्ति प॒ञ्चभिः॒ पुरी॑षैः । \newline
38. प॒ञ्चभिः॒ पुरी॑षैः॒ पुरी॑षैः प॒ञ्चभिः॑ प॒ञ्चभिः॒ पुरी॑षै र॒भ्य॑भि पुरी॑षैः प॒ञ्चभिः॑ प॒ञ्चभिः॒ पुरी॑षै र॒भि । \newline
39. प॒ञ्चभि॒रिति॑ प॒ञ्च - भिः॒ । \newline
40. पुरी॑षै र॒भ्य॑भि पुरी॑षैः॒ पुरी॑षै र॒भ्यू॑ह त्यूह त्य॒भि पुरी॑षैः॒ पुरी॑षै र॒भ्यू॑हति । \newline
41. अ॒भ्यू॑ह त्यूह त्य॒भ्या᳚(1॒)भ्यू॑हति॒ दश॒ दशो॑ह त्य॒भ्या᳚(1॒)भ्यू॑हति॒ दश॑ । \newline
42. ऊ॒ह॒ति॒ दश॒ दशो॑ह त्यूहति॒ दश॒ सꣳ सम् दशो॑ह त्यूहति॒ दश॒ सम् । \newline
43. दश॒ सꣳ सम् दश॒ दश॒ सम् प॑द्यन्ते पद्यन्ते॒ सम् दश॒ दश॒ सम् प॑द्यन्ते । \newline
44. सम् प॑द्यन्ते पद्यन्ते॒ सꣳ सम् प॑द्यन्ते॒ दशा᳚क्षरा॒ दशा᳚क्षरा पद्यन्ते॒ सꣳ सम् प॑द्यन्ते॒ दशा᳚क्षरा । \newline
45. प॒द्य॒न्ते॒ दशा᳚क्षरा॒ दशा᳚क्षरा पद्यन्ते पद्यन्ते॒ दशा᳚क्षरा वि॒राड् वि॒राड् दशा᳚क्षरा पद्यन्ते पद्यन्ते॒ दशा᳚क्षरा वि॒राट् । \newline
46. दशा᳚क्षरा वि॒राड् वि॒राड् दशा᳚क्षरा॒ दशा᳚क्षरा वि॒राडन्न॒ मन्नं॑ ॅवि॒राड् दशा᳚क्षरा॒ दशा᳚क्षरा वि॒राडन्न᳚म् । \newline
47. दशा᳚क्ष॒रेति॒ दश॑ - अ॒क्ष॒रा॒ । \newline
48. वि॒राडन्न॒ मन्नं॑ ॅवि॒राड् वि॒राडन्नं॑ ॅवि॒राड् वि॒राडन्नं॑ ॅवि॒राड् वि॒राडन्नं॑ ॅवि॒राट् । \newline
49. वि॒राडिति॑ वि - राट् । \newline
50. अन्नं॑ ॅवि॒राड् वि॒राडन्न॒ मन्नं॑ ॅवि॒राड् वि॒राजि॑ वि॒राजि॑ वि॒राडन्न॒ मन्नं॑ ॅवि॒राड् वि॒राजि॑ । \newline
51. वि॒राड् वि॒राजि॑ वि॒राजि॑ वि॒राड् वि॒राड् वि॒रा ज्ये॒वैव वि॒राजि॑ वि॒राड् वि॒राड् वि॒राज्ये॒व । \newline
52. वि॒राडिति॑ वि - राट् । \newline
53. वि॒रा ज्ये॒वैव वि॒राजि॑ वि॒रा ज्ये॒वान्नाद्ये॒ ऽन्नाद्य॑ ए॒व वि॒राजि॑ वि॒रा ज्ये॒वान्नाद्ये᳚ । \newline
54. वि॒राजीति॑ वि - राजि॑ । \newline
55. ए॒वान्नाद्ये॒ ऽन्नाद्य॑ ए॒वैवान्नाद्ये॒ प्रति॒ प्रत्य॒न्नाद्य॑ ए॒वैवान्नाद्ये॒ प्रति॑ । \newline
56. अ॒न्नाद्ये॒ प्रति॒ प्रत्य॒न्नाद्ये॒ ऽन्नाद्ये॒ प्रति॑ तिष्ठति तिष्ठति॒ प्रत्य॒न्नाद्ये॒ ऽन्नाद्ये॒ प्रति॑ तिष्ठति । \newline
57. अ॒न्नाद्य॒ इत्य॑न्न - अद्ये᳚ । \newline
58. प्रति॑ तिष्ठति तिष्ठति॒ प्रति॒ प्रति॑ तिष्ठति । \newline
59. ति॒ष्ठ॒तीति॑ तिष्ठति । \newline
\pagebreak
\markright{ TS 5.2.4.1  \hfill https://www.vedavms.in \hfill}

\section{ TS 5.2.4.1 }

\textbf{TS 5.2.4.1 } \newline
\textbf{Samhita Paata} \newline

वि वा ए॒तौ द्वि॑षाते॒ यश्च॑ पु॒राऽग्निर्यश्चो॒खायाꣳ॒॒ समि॑त॒मिति॑ चत॒सृभिः॒ सं निव॑पति च॒त्वारि॒ छन्दाꣳ॑सि॒ छन्दाꣳ॑सि॒ खलु॒ वा अ॒ग्नेः प्रि॒या त॒नूः प्रि॒ययै॒वैनौ॑ त॒नुवा॒ सꣳ शा᳚स्ति॒ समि॑त॒मित्या॑ह॒ तस्मा॒द्ब्रह्म॑णा क्ष॒त्रꣳ समे॑ति॒ यथ्सं॒ न्युप्य॑ वि॒हर॑ति॒ तस्मा॒द् ब्रह्म॑णा क्ष॒त्रं ॅव्ये᳚त्यृ॒तुभि॒ - [  ] \newline

\textbf{Pada Paata} \newline

वीति॑ । वै । ए॒तौ । द्वि॒षा॒ते॒ इति॑ । यः । च॒ । पु॒रा । अ॒ग्निः । यः । च॒ । उ॒खाया᳚म् । समिति॑ । इ॒त॒म् । इति॑ । च॒त॒सृभि॒रिति॑ चत॒सृ-भिः॒ । सम् । नीति॑ । व॒प॒ति॒ । च॒त्वारि॑ । छन्दाꣳ॑सि । छन्दाꣳ॑सि । खलु॑ । वै । अ॒ग्नेः । प्रि॒या । त॒नूः । प्रि॒यया᳚ । ए॒व । ए॒नौ॒ । त॒नुवा᳚ । समिति॑ । शा॒स्ति॒ । समिति॑ । इ॒त॒म् । इति॑ । आ॒ह॒ । तस्मा᳚त् । ब्रह्म॑णा । क्ष॒त्रम् । समिति॑ । ए॒ति॒ । यत् । स॒न्युंप्येति॑ सं - न्युप्य॑ । वि॒हर॒तीति॑ वि - हर॑ति । तस्मा᳚त् । ब्रह्म॑णा । क्ष॒त्रम् । वीति॑ । ए॒ति॒ । ऋ॒तुभि॒रित्यृ॒तु - भिः॒ ।  \newline


\textbf{Krama Paata} \newline

वि वै । वा ए॒तौ । ए॒तौ द्वि॑षाते । द्वि॒षा॒ते॒ यः । द्वि॒षा॒ते॒ इति॑ द्विषाते । यश्च॑ । च॒ पु॒रा । पु॒राऽग्निः । अ॒ग्निर् यः । यश्च॑ । चो॒खाया᳚म् । उ॒खायाꣳ॒॒ सम् । समि॑तम् । इ॒त॒मिति॑ । इति॑ चत॒सृभिः॑ । च॒त॒सृभिः॒ सम् । च॒त॒सृभि॒रिति॑ चत॒सृ - भिः॒ । सम् नि । नि व॑पति । व॒प॒ति॒ च॒त्वारि॑ । च॒त्वारि॒ छन्दाꣳ॑सि । छन्दाꣳ॑सि॒ छन्दाꣳ॑सि । छन्दाꣳ॑सि॒ खलु॑ । खलु॒ वै । वा अ॒ग्नेः । अ॒ग्नेः प्रि॒या । प्रि॒या त॒नूः । त॒नूः प्रि॒यया᳚ । प्रि॒ययै॒व । ए॒वैनौ᳚ । ए॒नौ॒ त॒नुवा᳚ । त॒नुवा॒ सम् । सꣳ शा᳚स्ति । शा॒स्ति॒ सम् । समि॑तम् । इ॒त॒मिति॑ । इत्या॑ह । आ॒ह॒ तस्मा᳚त् । तस्मा॒द् ब्रह्म॑णा । ब्रह्म॑णा क्ष॒त्रम् । क्ष॒त्रꣳ सम् । समे॑ति । ए॒ति॒ यत् । यथ् स॒न्न्युप्य॑ । स॒न्न्युप्य॑ वि॒हर॑ति । स॒न्न्युप्येति॑ सम् - न्युप्य॑ । वि॒हर॑ति॒ तस्मा᳚त् । वि॒हर॒तीति॑ वि - हर॑ति । तस्मा॒द् ब्रह्म॑णा । ब्रह्म॑णा क्ष॒त्रम् । क्ष॒त्रम् ॅवि । व्ये॑ति । ए॒त्यृ॒तुभिः॑ । ऋ॒तुभि॒र् वै । ऋ॒तुभि॒रित्यृ॒तु - भिः॒ \newline

\textbf{Jatai Paata} \newline

1. वि वै वै वि वि वै । \newline
2. वा ए॒ता वे॒तौ वै वा ए॒तौ । \newline
3. ए॒तौ द्वि॑षाते द्विषाते ए॒ता वे॒तौ द्वि॑षाते । \newline
4. द्वि॒षा॒ते॒ यो यो द्वि॑षाते द्विषाते॒ यः । \newline
5. द्वि॒षा॒ते॒ इति॑ द्विषाते । \newline
6. यश्च॑ च॒ यो यश्च॑ । \newline
7. च॒ पु॒रा पु॒रा च॑ च पु॒रा । \newline
8. पु॒रा ऽग्नि र॒ग्निः पु॒रा पु॒रा ऽग्निः । \newline
9. अ॒ग्निर् यो यो᳚ ऽग्नि र॒ग्निर् यः । \newline
10. यश्च॑ च॒ यो यश्च॑ । \newline
11. चो॒खाया॑ मु॒खाया᳚म् च चो॒खाया᳚म् । \newline
12. उ॒खायाꣳ॒॒ सꣳ स मु॒खाया॑ मु॒खायाꣳ॒॒ सम् । \newline
13. स मि॑त मितꣳ॒॒ सꣳ स मि॑तम् । \newline
14. इ॒त॒ मितीती॑त मित॒ मिति॑ । \newline
15. इति॑ चत॒सृभि॑ श्चत॒सृभि॒ रितीति॑ चत॒सृभिः॑ । \newline
16. च॒त॒सृभिः॒ सꣳ सम् च॑त॒सृभि॑ श्चत॒सृभिः॒ सम् । \newline
17. च॒त॒सृभि॒रिति॑ चत॒सृ - भिः॒ । \newline
18. सम् नि नि सꣳ सम् नि । \newline
19. नि व॑पति वपति॒ नि नि व॑पति । \newline
20. व॒प॒ति॒ च॒त्वारि॑ च॒त्वारि॑ वपति वपति च॒त्वारि॑ । \newline
21. च॒त्वारि॒ छन्दाꣳ॑सि॒ छन्दाꣳ॑सि च॒त्वारि॑ च॒त्वारि॒ छन्दाꣳ॑सि । \newline
22. छन्दाꣳ॑सि॒ छन्दाꣳ॑सि । \newline
23. छन्दाꣳ॑सि॒ खलु॒ खलु॒ छन्दाꣳ॑सि॒ छन्दाꣳ॑सि॒ खलु॑ । \newline
24. खलु॒ वै वै खलु॒ खलु॒ वै । \newline
25. वा अ॒ग्ने र॒ग्नेर् वै वा अ॒ग्नेः । \newline
26. अ॒ग्नेः प्रि॒या प्रि॒या ऽग्ने र॒ग्नेः प्रि॒या । \newline
27. प्रि॒या त॒नू स्त॒नूः प्रि॒या प्रि॒या त॒नूः । \newline
28. त॒नूः प्रि॒यया᳚ प्रि॒यया॑ त॒नू स्त॒नूः प्रि॒यया᳚ । \newline
29. प्रि॒य यै॒वैव प्रि॒यया᳚ प्रि॒ययै॒व । \newline
30. ए॒वैना॑ वेना वे॒वै वैनौ᳚ । \newline
31. ए॒नौ॒ त॒नुवा॑ त॒नुवै॑ना वेनौ त॒नुवा᳚ । \newline
32. त॒नुवा॒ सꣳ सम् त॒नुवा॑ त॒नुवा॒ सम् । \newline
33. सꣳ शा᳚स्ति शास्ति॒ सꣳ सꣳ शा᳚स्ति । \newline
34. शा॒स्ति॒ सꣳ सꣳ शा᳚स्ति शास्ति॒ सम् । \newline
35. स मि॑त मितꣳ॒॒ सꣳ स मि॑तम् । \newline
36. इ॒त॒ मितीती॑त मित॒ मिति॑ । \newline
37. इत्या॑हा॒हे तीत्या॑ह । \newline
38. आ॒ह॒ तस्मा॒त् तस्मा॑ दाहाह॒ तस्मा᳚त् । \newline
39. तस्मा॒द् ब्रह्म॑णा॒ ब्रह्म॑णा॒ तस्मा॒त् तस्मा॒द् ब्रह्म॑णा । \newline
40. ब्रह्म॑णा क्ष॒त्रम् क्ष॒त्रम् ब्रह्म॑णा॒ ब्रह्म॑णा क्ष॒त्रम् । \newline
41. क्ष॒त्रꣳ सꣳ सम् क्ष॒त्रम् क्ष॒त्रꣳ सम् । \newline
42. स मे᳚त्येति॒ सꣳ स मे॑ति । \newline
43. ए॒ति॒ यद् यदे᳚ त्येति॒ यत् । \newline
44. यथ् स॒न्न्युप्य॑ स॒न्न्युप्य॒ यद् यथ् स॒न्न्युप्य॑ । \newline
45. स॒न्न्युप्य॑ वि॒हर॑ति वि॒हर॑ति स॒न्न्युप्य॑ स॒न्न्युप्य॑ वि॒हर॑ति । \newline
46. स॒न्न्युप्येति॑ सं - न्युप्य॑ । \newline
47. वि॒हर॑ति॒ तस्मा॒त् तस्मा᳚द् वि॒हर॑ति वि॒हर॑ति॒ तस्मा᳚त् । \newline
48. वि॒हर॒तीति॑ वि - हर॑ति । \newline
49. तस्मा॒द् ब्रह्म॑णा॒ ब्रह्म॑णा॒ तस्मा॒त् तस्मा॒द् ब्रह्म॑णा । \newline
50. ब्रह्म॑णा क्ष॒त्रम् क्ष॒त्रम् ब्रह्म॑णा॒ ब्रह्म॑णा क्ष॒त्रम् । \newline
51. क्ष॒त्रं ॅवि वि क्ष॒त्रम् क्ष॒त्रं ॅवि । \newline
52. व्ये᳚त्येति॒ वि व्ये॑ति । \newline
53. ए॒त्यृ॒तुभिर्॑. ऋ॒तुभि॑ रेत्ये त्यृ॒तुभिः॑ । \newline
54. ऋ॒तुभि॒र् वै वा ऋ॒तुभिर्॑. ऋ॒तुभि॒र् वै । \newline
55. ऋ॒तुभि॒रित्यृ॒तु - भिः॒ । \newline

\textbf{Ghana Paata } \newline

1. वि वै वै वि वि वा ए॒ता वे॒तौ वै वि वि वा ए॒तौ । \newline
2. वा ए॒ता वे॒तौ वै वा ए॒तौ द्वि॑षाते द्विषाते ए॒तौ वै वा ए॒तौ द्वि॑षाते । \newline
3. ए॒तौ द्वि॑षाते द्विषाते ए॒ता वे॒तौ द्वि॑षाते॒ यो यो द्वि॑षाते ए॒ता वे॒तौ द्वि॑षाते॒ यः । \newline
4. द्वि॒षा॒ते॒ यो यो द्वि॑षाते द्विषाते॒ यश्च॑ च॒ यो द्वि॑षाते द्विषाते॒ यश्च॑ । \newline
5. द्वि॒षा॒ते॒ इति॑ द्विषाते । \newline
6. यश्च॑ च॒ यो यश्च॑ पु॒रा पु॒रा च॒ यो यश्च॑ पु॒रा । \newline
7. च॒ पु॒रा पु॒रा च॑ च पु॒रा ऽग्नि र॒ग्निः पु॒रा च॑ च पु॒रा ऽग्निः । \newline
8. पु॒रा ऽग्नि र॒ग्निः पु॒रा पु॒रा ऽग्निर् यो यो᳚ ऽग्निः पु॒रा पु॒रा ऽग्निर् यः । \newline
9. अ॒ग्निर् यो यो᳚ ऽग्नि र॒ग्निर् यश्च॑ च॒ यो᳚ ऽग्नि र॒ग्निर् यश्च॑ । \newline
10. यश्च॑ च॒ यो यश्चो॒खाया॑ मु॒खाया᳚म् च॒ यो यश्चो॒खाया᳚म् । \newline
11. चो॒खाया॑ मु॒खाया᳚म् च चो॒खायाꣳ॒॒ सꣳ स मु॒खाया᳚म् च चो॒खायाꣳ॒॒ सम् । \newline
12. उ॒खायाꣳ॒॒ सꣳ स मु॒खाया॑ मु॒खायाꣳ॒॒ स मि॑त मितꣳ॒॒ स मु॒खाया॑ मु॒खायाꣳ॒॒ स मि॑तम् । \newline
13. स मि॑त मितꣳ॒॒ सꣳ स मि॑त॒ मितीती॑तꣳ॒॒ सꣳ स मि॑त॒ मिति॑ । \newline
14. इ॒त॒ मितीती॑त मित॒ मिति॑ चत॒सृभि॑ श्चत॒सृभि॒ रिती॑त मित॒ मिति॑ चत॒सृभिः॑ । \newline
15. इति॑ चत॒सृभि॑ श्चत॒सृभि॒ रितीति॑ चत॒सृभिः॒ सꣳ सम् च॑त॒सृभि॒ रितीति॑ चत॒सृभिः॒ सम् । \newline
16. च॒त॒सृभिः॒ सꣳ सम् च॑त॒सृभि॑ श्चत॒सृभिः॒ सम् नि नि सम् च॑त॒सृभि॑ श्चत॒सृभिः॒ सम् नि । \newline
17. च॒त॒सृभि॒रिति॑ चत॒सृ - भिः॒ । \newline
18. सम् नि नि सꣳ सम् नि व॑पति वपति॒ नि सꣳ सम् नि व॑पति । \newline
19. नि व॑पति वपति॒ नि नि व॑पति च॒त्वारि॑ च॒त्वारि॑ वपति॒ नि नि व॑पति च॒त्वारि॑ । \newline
20. व॒प॒ति॒ च॒त्वारि॑ च॒त्वारि॑ वपति वपति च॒त्वारि॒ छन्दाꣳ॑सि॒ छन्दाꣳ॑सि च॒त्वारि॑ वपति वपति च॒त्वारि॒ छन्दाꣳ॑सि । \newline
21. च॒त्वारि॒ छन्दाꣳ॑सि॒ छन्दाꣳ॑सि च॒त्वारि॑ च॒त्वारि॒ छन्दाꣳ॑सि । \newline
22. छन्दाꣳ॑सि॒ छन्दाꣳ॑सि । \newline
23. छन्दाꣳ॑सि॒ खलु॒ खलु॒ छन्दाꣳ॑सि॒ छन्दाꣳ॑सि॒ खलु॒ वै वै खलु॒ छन्दाꣳ॑सि॒ छन्दाꣳ॑सि॒ खलु॒ वै । \newline
24. खलु॒ वै वै खलु॒ खलु॒ वा अ॒ग्ने र॒ग्नेर् वै खलु॒ खलु॒ वा अ॒ग्नेः । \newline
25. वा अ॒ग्ने र॒ग्नेर् वै वा अ॒ग्नेः प्रि॒या प्रि॒या ऽग्नेर् वै वा अ॒ग्नेः प्रि॒या । \newline
26. अ॒ग्नेः प्रि॒या प्रि॒या ऽग्ने र॒ग्नेः प्रि॒या त॒नू स्त॒नूः प्रि॒या ऽग्ने र॒ग्नेः प्रि॒या त॒नूः । \newline
27. प्रि॒या त॒नू स्त॒नूः प्रि॒या प्रि॒या त॒नूः प्रि॒यया᳚ प्रि॒यया॑ त॒नूः प्रि॒या प्रि॒या त॒नूः प्रि॒यया᳚ । \newline
28. त॒नूः प्रि॒यया᳚ प्रि॒यया॑ त॒नू स्त॒नूः प्रि॒ययै॒वैव प्रि॒यया॑ त॒नू स्त॒नूः प्रि॒ययै॒व । \newline
29. प्रि॒ययै॒वैव प्रि॒यया᳚ प्रि॒ययै॒वैना॑ वेना वे॒व प्रि॒यया᳚ प्रि॒ययै॒वैनौ᳚ । \newline
30. ए॒वैना॑ वेना वे॒वैवैनौ॑ त॒नुवा॑ त॒नुवै॑ना वे॒वैवैनौ॑ त॒नुवा᳚ । \newline
31. ए॒नौ॒ त॒नुवा॑ त॒नुवै॑ना वेनौ त॒नुवा॒ सꣳ सम् त॒नुवै॑ना वेनौ त॒नुवा॒ सम् । \newline
32. त॒नुवा॒ सꣳ सम् त॒नुवा॑ त॒नुवा॒ सꣳ शा᳚स्ति शास्ति॒ सम् त॒नुवा॑ त॒नुवा॒ सꣳ शा᳚स्ति । \newline
33. सꣳ शा᳚स्ति शास्ति॒ सꣳ सꣳ शा᳚स्ति॒ सꣳ सꣳ शा᳚स्ति॒ सꣳ सꣳ शा᳚स्ति॒ सम् । \newline
34. शा॒स्ति॒ सꣳ सꣳ शा᳚स्ति शास्ति॒ स मि॑त मितꣳ॒॒ सꣳ शा᳚स्ति शास्ति॒ स मि॑तम् । \newline
35. स मि॑त मितꣳ॒॒ सꣳ स मि॑त॒ मितीती॑तꣳ॒॒ सꣳ स मि॑त॒ मिति॑ । \newline
36. इ॒त॒ मितीती॑त मित॒ मित्या॑हा॒हे ती॑त मित॒ मित्या॑ह । \newline
37. इत्या॑हा॒हे तीत्या॑ह॒ तस्मा॒त् तस्मा॑दा॒हे तीत्या॑ह॒ तस्मा᳚त् । \newline
38. आ॒ह॒ तस्मा॒त् तस्मा॑ दाहाह॒ तस्मा॒द् ब्रह्म॑णा॒ ब्रह्म॑णा॒ तस्मा॑ दाहाह॒ तस्मा॒द् ब्रह्म॑णा । \newline
39. तस्मा॒द् ब्रह्म॑णा॒ ब्रह्म॑णा॒ तस्मा॒त् तस्मा॒द् ब्रह्म॑णा क्ष॒त्रम् क्ष॒त्रम् ब्रह्म॑णा॒ तस्मा॒त् तस्मा॒द् ब्रह्म॑णा क्ष॒त्रम् । \newline
40. ब्रह्म॑णा क्ष॒त्रम् क्ष॒त्रम् ब्रह्म॑णा॒ ब्रह्म॑णा क्ष॒त्रꣳ सꣳ सम् क्ष॒त्रम् ब्रह्म॑णा॒ ब्रह्म॑णा क्ष॒त्रꣳ सम् । \newline
41. क्ष॒त्रꣳ सꣳ सम् क्ष॒त्रम् क्ष॒त्रꣳ स मे᳚त्येति॒ सम् क्ष॒त्रम् क्ष॒त्रꣳ स मे॑ति । \newline
42. स मे᳚त्येति॒ सꣳ स मे॑ति॒ यद् यदे॑ति॒ सꣳ स मे॑ति॒ यत् । \newline
43. ए॒ति॒ यद् यदे᳚त्येति॒ यथ् स॒न्न्युप्य॑ स॒न्न्युप्य॒ यदे᳚त्येति॒ यथ् स॒न्न्युप्य॑ । \newline
44. यथ् स॒न्न्युप्य॑ स॒न्न्युप्य॒ यद् यथ् स॒न्न्युप्य॑ वि॒हर॑ति वि॒हर॑ति स॒न्न्युप्य॒ यद् यथ् स॒न्न्युप्य॑ वि॒हर॑ति । \newline
45. स॒न्न्युप्य॑ वि॒हर॑ति वि॒हर॑ति स॒न्न्युप्य॑ स॒न्न्युप्य॑ वि॒हर॑ति॒ तस्मा॒त् तस्मा᳚द् वि॒हर॑ति स॒न्न्युप्य॑ स॒न्न्युप्य॑ वि॒हर॑ति॒ तस्मा᳚त् । \newline
46. स॒न्न्युप्येति॑ सं - न्युप्य॑ । \newline
47. वि॒हर॑ति॒ तस्मा॒त् तस्मा᳚द् वि॒हर॑ति वि॒हर॑ति॒ तस्मा॒द् ब्रह्म॑णा॒ ब्रह्म॑णा॒ तस्मा᳚द् वि॒हर॑ति वि॒हर॑ति॒ तस्मा॒द् ब्रह्म॑णा । \newline
48. वि॒हर॒तीति॑ वि - हर॑ति । \newline
49. तस्मा॒द् ब्रह्म॑णा॒ ब्रह्म॑णा॒ तस्मा॒त् तस्मा॒द् ब्रह्म॑णा क्ष॒त्रम् क्ष॒त्रम् ब्रह्म॑णा॒ तस्मा॒त् तस्मा॒द् ब्रह्म॑णा क्ष॒त्रम् । \newline
50. ब्रह्म॑णा क्ष॒त्रम् क्ष॒त्रम् ब्रह्म॑णा॒ ब्रह्म॑णा क्ष॒त्रं ॅवि वि क्ष॒त्रम् ब्रह्म॑णा॒ ब्रह्म॑णा क्ष॒त्रं ॅवि । \newline
51. क्ष॒त्रं ॅवि वि क्ष॒त्रम् क्ष॒त्रं ॅव्ये᳚त्येति॒ वि क्ष॒त्रम् क्ष॒त्रं ॅव्ये॑ति । \newline
52. व्ये᳚त्येति॒ वि व्ये᳚त्यृ॒तुभिर्॑. ऋ॒तुभि॑रेति॒ वि व्ये᳚त्यृ॒तुभिः॑ । \newline
53. ए॒त्यृ॒तुभिर्॑. ऋ॒तुभि॑ रेत्ये त्यृ॒तुभि॒र् वै वा ऋ॒तुभि॑ रेत्ये त्यृ॒तुभि॒र् वै । \newline
54. ऋ॒तुभि॒र् वै वा ऋ॒तुभिर्॑. ऋ॒तुभि॒र् वा ए॒त मे॒तं ॅवा ऋ॒तुभिर्॑. ऋ॒तुभि॒र् वा ए॒तम् । \newline
55. ऋ॒तुभि॒रित्यृ॒तु - भिः॒ । \newline
\pagebreak
\markright{ TS 5.2.4.2  \hfill https://www.vedavms.in \hfill}

\section{ TS 5.2.4.2 }

\textbf{TS 5.2.4.2 } \newline
\textbf{Samhita Paata} \newline

-र्वा ए॒तं दी᳚क्षयन्ति॒ स ऋ॒तुभि॑रे॒व वि॒मुच्यो॑ मा॒तेव॑ पु॒त्रं पृ॑थि॒वी पु॑री॒ष्य॑मित्या॑ह॒-र्तुभि॑रे॒वैनं॑ दीक्षयि॒त्वर्तुभि॒र्वि मु॑ञ्चति वैश्वान॒र्या शि॒क्य॑मा द॑त्ते स्व॒दय॑त्ये॒वैन॑-न्नैर्.ऋ॒तीः कृ॒ष्णा-स्ति॒स्र-स्तुष॑पक्वा भवन्ति॒ निर्.ऋ॑त्यै॒ वा ए॒तद्-भा॑ग॒धेयं॒ ॅयत् तुषा॒ निर्.ऋ॑त्यै रू॒पं कृ॒ष्णꣳ रू॒पेणै॒व निर्.ऋ॑तिं नि॒रव॑दयत इ॒मां दिशं॑ ॅयन्त्ये॒षा - [  ] \newline

\textbf{Pada Paata} \newline

वै । ए॒तम् । दी॒क्ष॒य॒न्ति॒ । सः । ऋ॒तुभि॒रित्यृ॒तु - भिः॒ । ए॒व । वि॒मुच्य॒ इति॑ वि - मुच्यः॑ । मा॒ता । इ॒व॒ । पु॒त्रम् । पृ॒थि॒वी । पु॒री॒ष्य᳚म् । इति॑ । आ॒ह॒ । ऋ॒तुभि॒रित्यृ॒तु - भिः॒ । ए॒व । ए॒न॒म् । दी॒क्ष॒यि॒त्वा । ऋ॒तुभि॒रित्यृ॒तु-भिः॒ । वीति॑ । मु॒ञ्च॒ति॒ । वै॒श्वा॒न॒र्या । शि॒क्य᳚म् । एति॑ । द॒त्ते॒ । स्व॒दय॑ति । ए॒व । ए॒न॒त्॒ । नै॒र्॒.ऋ॒तीरिति॑ नैः-ऋ॒तीः । कृ॒ष्णाः । ति॒स्रः । तुष॑पक्वा॒ इति॒ तुष॑ - प॒क्वाः॒ । भ॒व॒न्ति॒ । निर्.ऋ॑त्या॒ इति॒ निः - ऋ॒त्यै॒ । वै । ए॒तत् । भा॒ग॒धेय॒मिति॑ भाग - धेय᳚म् । यत् । तुषाः᳚ । निर्.ऋ॑त्या॒ इति॒ निः-ऋ॒त्यै॒ । रू॒पम् । कृ॒ष्णम् । रू॒पेण॑ । ए॒व । निर्.ऋ॑ति॒मिति॒ निः - ऋ॒ति॒म् । नि॒रव॑दयत॒ इति॑ निः - अव॑दयते । इ॒माम् । दिश᳚म् । य॒न्ति॒ । ए॒षा ।  \newline


\textbf{Krama Paata} \newline

वा ए॒तम् । ए॒तम् दी᳚क्षयन्ति । दी॒क्ष॒य॒न्ति॒ सः । स ऋ॒तुभिः॑ । ऋ॒तुभि॑रे॒व । ऋ॒तुभि॒रित्यृ॒तु - भिः॒ । ए॒व वि॒मुच्यः॑ । वि॒मुच्यो॑ मा॒ता । वि॒मुच्य॒ इति॑ वि - मुच्यः॑ । मा॒तेव॑ । इ॒व॒ पु॒त्रम् । पु॒त्रम् पृ॑थि॒वी । पृ॒थि॒वी पु॑री॒ष्य᳚म् । पु॒री॒ष्य॑मिति॑ । इत्या॑ह । आ॒ह॒र्तुभिः॑ । ऋ॒तुभि॑रे॒व । ऋ॒तुभि॒रित्यृ॒तु - भिः॒ । ए॒वैन᳚म् । ए॒न॒म् दी॒क्ष॒यि॒त्वा । दी॒क्ष॒यि॒त्वर्तुभिः॑ । ऋ॒तुभि॒र् वि । ऋ॒तुभि॒रित्यृ॒तु - भिः॒ । वि मु॑ञ्चति । मु॒ञ्च॒ति॒ वै॒श्वा॒न॒र्या । वै॒श्वा॒न॒र्या शि॒क्य᳚म् । शि॒क्य॑मा । आ द॑त्ते । द॒त्ते॒ स्व॒दय॑ति । स्व॒दय॑त्ये॒व । ए॒वैन॑त् । ए॒न॒न् नै॒र्॒.ऋ॒तीः । नै॒र्॒.ऋ॒तीः कृ॒ष्णाः । नै॒र्॒.ऋ॒तीरिति॑ नैः - ऋ॒तीः । कृ॒ष्णास्ति॒स्रः । त्रि॒स्रस्तुष॑पक्वाः । तुष॑पक्वा भवन्ति । तुष॑पक्वा॒ इति॒ तुष॑ - प॒क्वाः॒ । भ॒व॒न्ति॒ निर्.ऋ॑त्यै । निर्.ऋ॑त्यै॒ वै । निर्.ऋ॑त्या॒ इति॒ निः - ऋ॒त्यै॒ । वा ए॒तत् । ए॒तद् भा॑ग॒धेय᳚म् । भा॒ग॒धेय॒म् ॅयत् । भा॒ग॒धेय॒मिति॑ भाग - धेय᳚म् । यत् तुषाः᳚ । तुषा॒ निर्.ऋ॑त्यै । निर्.ऋ॑त्यै रू॒पम् । निर्.ऋ॑त्या॒ इति॒ निः - ऋ॒त्यै॒ । रू॒पम् कृ॒ष्णम् । कृ॒ष्णꣳ रू॒पेण॑ । रू॒पेणै॒व । ए॒व निर्.ऋ॑तिम् । निर्.ऋ॑तिम् नि॒रव॑दयते । निर्.ऋ॑ति॒मिति॒ निः - ऋ॒ति॒म् । नि॒रव॑दयत इ॒माम् । नि॒रव॑दयत॒ इति॑ निः - अव॑दयते । इ॒माम् दिश᳚म् । दिश॑म् ॅयन्ति । य॒न्त्ये॒षा । ए॒षा वै \newline

\textbf{Jatai Paata} \newline

1. वा ए॒त मे॒तं ॅवै वा ए॒तम् । \newline
2. ए॒तम् दी᳚क्षयन्ति दीक्षयन् त्ये॒त मे॒तम् दी᳚क्षयन्ति । \newline
3. दी॒क्ष॒य॒न्ति॒ स स दी᳚क्षयन्ति दीक्षयन्ति॒ सः । \newline
4. स ऋ॒तुभिर्॑. ऋ॒तुभिः॒ स स ऋ॒तुभिः॑ । \newline
5. ऋ॒तुभि॑ रे॒वैव र्‌तुभिर्॑. ऋ॒तुभि॑ रे॒व । \newline
6. ऋ॒तुभि॒रित्यृ॒तु - भिः॒ । \newline
7. ए॒व वि॒मुच्यो॑ वि॒मुच्य॑ ए॒वैव वि॒मुच्यः॑ । \newline
8. वि॒मुच्यो॑ मा॒ता मा॒ता वि॒मुच्यो॑ वि॒मुच्यो॑ मा॒ता । \newline
9. वि॒मुच्य॒ इति॑ वि - मुच्यः॑ । \newline
10. मा॒तेवे॑व मा॒ता मा॒तेव॑ । \newline
11. इ॒व॒ पु॒त्रम् पु॒त्र मि॑वेव पु॒त्रम् । \newline
12. पु॒त्रम् पृ॑थि॒वी पृ॑थि॒वी पु॒त्रम् पु॒त्रम् पृ॑थि॒वी । \newline
13. पृ॒थि॒वी पु॑री॒ष्य॑म् पुरी॒ष्य॑म् पृथि॒वी पृ॑थि॒वी पु॑री॒ष्य᳚म् । \newline
14. पु॒री॒ष्य॑ मितीति॑ पुरी॒ष्य॑म् पुरी॒ष्य॑ मिति॑ । \newline
15. इत्या॑हा॒हे तीत्या॑ह । \newline
16. आ॒ह॒ र्‌तुभिर्॑. ऋ॒तुभि॑ राहाह॒ र्‌तुभिः॑ । \newline
17. ऋ॒तुभि॑ रे॒वैव र्‌तुभिर्॑. ऋ॒तुभि॑ रे॒व । \newline
18. ऋ॒तुभि॒रित्यृ॒तु - भिः॒ । \newline
19. ए॒वैन॑ मेन मे॒वैवैन᳚म् । \newline
20. ए॒न॒म् दी॒क्ष॒यि॒त्वा दी᳚क्षयि॒त्वैन॑ मेनम् दीक्षयि॒त्वा । \newline
21. दी॒क्ष॒यि॒ त्वर्तुभिर्॑. ऋ॒तुभि॑र् दीक्षयि॒त्वा दी᳚क्षयि॒त्व र्‌तुभिः॑ । \newline
22. ऋ॒तुभि॒र् वि व्यृ॑तुभिर्॑. ऋ॒तुभि॒र् वि । \newline
23. ऋ॒तुभि॒रित्यृ॒तु - भिः॒ । \newline
24. वि मु॑ञ्चति मुञ्चति॒ वि वि मु॑ञ्चति । \newline
25. मु॒ञ्च॒ति॒ वै॒श्वा॒न॒र्या वै᳚श्वान॒र्या मु॑ञ्चति मुञ्चति वैश्वान॒र्या । \newline
26. वै॒श्वा॒न॒र्या शि॒क्यꣳ॑ शि॒क्यं॑ ॅवैश्वान॒र्या वै᳚श्वान॒र्या शि॒क्य᳚म् । \newline
27. शि॒क्य॑ मा शि॒क्यꣳ॑ शि॒क्य॑ मा । \newline
28. आ द॑त्ते दत्त॒ आ द॑त्ते । \newline
29. द॒त्ते॒ स्व॒दय॑ति स्व॒दय॑ति दत्ते दत्ते स्व॒दय॑ति । \newline
30. स्व॒दय॑ त्ये॒वैव स्व॒दय॑ति स्व॒दय॑ त्ये॒व । \newline
31. ए॒वैन॑ देन दे॒वै वैन॑त् । \newline
32. ए॒न॒न् नै॒र्॒.ऋ॒तीर् नैर्॑.ऋ॒ती रे॑न देनन् नैर्.ऋ॒तीः । \newline
33. नै॒र्॒.ऋ॒तीः कृ॒ष्णाः कृ॒ष्णा नैर्॑.ऋ॒तीर् नैर्॑.ऋ॒तीः कृ॒ष्णाः । \newline
34. नै॒र्॒.ऋ॒तीरिति॑ नैः - ऋ॒तीः । \newline
35. कृ॒ष्णा स्ति॒स्र स्ति॒स्रः कृ॒ष्णाः कृ॒ष्णा स्ति॒स्रः । \newline
36. ति॒स्र स्तुष॑पक्वा॒ स्तुष॑पक्वा स्ति॒स्र स्ति॒स्र स्तुष॑पक्वाः । \newline
37. तुष॑पक्वा भवन्ति भवन्ति॒ तुष॑पक्वा॒ स्तुष॑पक्वा भवन्ति । \newline
38. तुष॑पक्वा॒ इति॒ तुष॑ - प॒क्वाः॒ । \newline
39. भ॒व॒न्ति॒ निर्.ऋ॑त्यै॒ निर्.ऋ॑त्यै भवन्ति भवन्ति॒ निर्.ऋ॑त्यै । \newline
40. निर्.ऋ॑त्यै॒ वै वै निर्.ऋ॑त्यै॒ निर्.ऋ॑त्यै॒ वै । \newline
41. निर्.ऋ॑त्या॒ इति॒ निः - ऋ॒त्यै॒ । \newline
42. वा ए॒त दे॒तद् वै वा ए॒तत् । \newline
43. ए॒तद् भा॑ग॒धेय॑म् भाग॒धेय॑ मे॒त दे॒तद् भा॑ग॒धेय᳚म् । \newline
44. भा॒ग॒धेयं॒ ॅयद् यद् भा॑ग॒धेय॑म् भाग॒धेयं॒ ॅयत् । \newline
45. भा॒ग॒धेय॒मिति॑ भाग - धेय᳚म् । \newline
46. यत् तुषा॒ स्तुषा॒ यद् यत् तुषाः᳚ । \newline
47. तुषा॒ निर्.ऋ॑त्यै॒ निर्.ऋ॑त्यै॒ तुषा॒ स्तुषा॒ निर्.ऋ॑त्यै । \newline
48. निर्.ऋ॑त्यै रू॒पꣳ रू॒पम् निर्.ऋ॑त्यै॒ निर्.ऋ॑त्यै रू॒पम् । \newline
49. निर्.ऋ॑त्या॒ इति॒ निः - ऋ॒त्यै॒ । \newline
50. रू॒पम् कृ॒ष्णम् कृ॒ष्णꣳ रू॒पꣳ रू॒पम् कृ॒ष्णम् । \newline
51. कृ॒ष्णꣳ रू॒पेण॑ रू॒पेण॑ कृ॒ष्णम् कृ॒ष्णꣳ रू॒पेण॑ । \newline
52. रू॒पे णै॒वैव रू॒पेण॑ रू॒पेणै॒व । \newline
53. ए॒व निर्.ऋ॑ति॒म् निर्.ऋ॑ति मे॒वैव निर्.ऋ॑तिम् । \newline
54. निर्.ऋ॑तिम् नि॒रव॑दयते नि॒रव॑दयते॒ निर्.ऋ॑ति॒म् निर्.ऋ॑तिम् नि॒रव॑दयते । \newline
55. निर्.ऋ॑ति॒मिति॒ निः - ऋ॒ति॒म् । \newline
56. नि॒रव॑दयत इ॒मा मि॒माम् नि॒रव॑दयते नि॒रव॑दयत इ॒माम् । \newline
57. नि॒रव॑दयत॒ इति॑ निः - अव॑दयते । \newline
58. इ॒माम् दिश॒म् दिश॑ मि॒मा मि॒माम् दिश᳚म् । \newline
59. दिशं॑ ॅयन्ति यन्ति॒ दिश॒म् दिशं॑ ॅयन्ति । \newline
60. य॒न्त्ये॒षैषा य॑न्ति यन्त्ये॒षा । \newline
61. ए॒षा वै वा ए॒षैषा वै । \newline

\textbf{Ghana Paata } \newline

1. वा ए॒त मे॒तं ॅवै वा ए॒तम् दी᳚क्षयन्ति दीक्षयन् त्ये॒तं ॅवै वा ए॒तम् दी᳚क्षयन्ति । \newline
2. ए॒तम् दी᳚क्षयन्ति दीक्षयन् त्ये॒त मे॒तम् दी᳚क्षयन्ति॒ स स दी᳚क्षयन् त्ये॒त मे॒तम् दी᳚क्षयन्ति॒ सः । \newline
3. दी॒क्ष॒य॒न्ति॒ स स दी᳚क्षयन्ति दीक्षयन्ति॒ स ऋ॒तुभिर्॑. ऋ॒तुभिः॒ स दी᳚क्षयन्ति दीक्षयन्ति॒ स ऋ॒तुभिः॑ । \newline
4. स ऋ॒तुभिर्॑. ऋ॒तुभिः॒ स स ऋ॒तुभि॑ रे॒वैव र्तुभिः॒ स स ऋ॒तुभि॑ रे॒व । \newline
5. ऋ॒तुभि॑ रे॒वैव र्‌तुभिर्॑. ऋ॒तुभि॑ रे॒व वि॒मुच्यो॑ वि॒मुच्य॑ ए॒व र्‌तुभिर्॑. ऋ॒तुभि॑ रे॒व वि॒मुच्यः॑ । \newline
6. ऋ॒तुभि॒रित्यृ॒तु - भिः॒ । \newline
7. ए॒व वि॒मुच्यो॑ वि॒मुच्य॑ ए॒वैव वि॒मुच्यो॑ मा॒ता मा॒ता वि॒मुच्य॑ ए॒वैव वि॒मुच्यो॑ मा॒ता । \newline
8. वि॒मुच्यो॑ मा॒ता मा॒ता वि॒मुच्यो॑ वि॒मुच्यो॑ मा॒तेवे॑व मा॒ता वि॒मुच्यो॑ वि॒मुच्यो॑ मा॒तेव॑ । \newline
9. वि॒मुच्य॒ इति॑ वि - मुच्यः॑ । \newline
10. मा॒तेवे॑व मा॒ता मा॒तेव॑ पु॒त्रम् पु॒त्र मि॑व मा॒ता मा॒तेव॑ पु॒त्रम् । \newline
11. इ॒व॒ पु॒त्रम् पु॒त्र मि॑वेव पु॒त्रम् पृ॑थि॒वी पृ॑थि॒वी पु॒त्र मि॑वेव पु॒त्रम् पृ॑थि॒वी । \newline
12. पु॒त्रम् पृ॑थि॒वी पृ॑थि॒वी पु॒त्रम् पु॒त्रम् पृ॑थि॒वी पु॑री॒ष्य॑म् पुरी॒ष्य॑म् पृथि॒वी पु॒त्रम् पु॒त्रम् पृ॑थि॒वी पु॑री॒ष्य᳚म् । \newline
13. पृ॒थि॒वी पु॑री॒ष्य॑म् पुरी॒ष्य॑म् पृथि॒वी पृ॑थि॒वी पु॑री॒ष्य॑ मितीति॑ पुरी॒ष्य॑म् पृथि॒वी पृ॑थि॒वी पु॑री॒ष्य॑ मिति॑ । \newline
14. पु॒री॒ष्य॑ मितीति॑ पुरी॒ष्य॑म् पुरी॒ष्य॑ मित्या॑हा॒हेति॑ पुरी॒ष्य॑म् पुरी॒ष्य॑ मित्या॑ह । \newline
15. इत्या॑हा॒हे तीत्या॑ह॒ र्‌तुभिर्॑. ऋ॒तुभि॑ रा॒हे तीत्या॑ह॒ र्‌तुभिः॑ । \newline
16. आ॒ह॒ र्‌तुभिर्॑. ऋ॒तुभि॑ राहाह॒ र्‌तुभि॑ रे॒वैव र्‌तुभि॑ राहाह॒ र्‌तुभि॑ रे॒व । \newline
17. ऋ॒तुभि॑ रे॒वैव र्‌तुभिर्॑. ऋ॒तुभि॑ रे॒वैन॑ मेन मे॒व र्‌तुभिर्॑. ऋ॒तुभि॑ रे॒वैन᳚म् । \newline
18. ऋ॒तुभि॒रित्यृ॒तु - भिः॒ । \newline
19. ए॒वैन॑ मेन मे॒वैवैन॑म् दीक्षयि॒त्वा दी᳚क्षयि॒त्वैन॑ मे॒वैवैन॑म् दीक्षयि॒त्वा । \newline
20. ए॒न॒म् दी॒क्ष॒यि॒त्वा दी᳚क्षयि॒त्वैन॑ मेनम् दीक्षयि॒त्व र्‌तुभिर्॑. ऋ॒तुभि॑र् दीक्षयि॒त्वैन॑ मेनम् दीक्षयि॒त्व र्‌तुभिः॑ । \newline
21. दी॒क्ष॒यि॒त्व र्‌तुभिर्॑. ऋ॒तुभि॑र् दीक्षयि॒त्वा दी᳚क्षयि॒त्व र्‌तुभि॒र् वि व्यृ॑तुभि॑र् दीक्षयि॒त्वा दी᳚क्षयि॒त्व र्‌तुभि॒र् वि । \newline
22. ऋ॒तुभि॒र् वि व्यृ॑तुभिर्॑. ऋ॒तुभि॒र् वि मु॑ञ्चति मुञ्चति॒ व्यृ॑तुभिर्॑. ऋ॒तुभि॒र् वि मु॑ञ्चति । \newline
23. ऋ॒तुभि॒रित्यृ॒तु - भिः॒ । \newline
24. वि मु॑ञ्चति मुञ्चति॒ वि वि मु॑ञ्चति वैश्वान॒र्या वै᳚श्वान॒र्या मु॑ञ्चति॒ वि वि मु॑ञ्चति वैश्वान॒र्या । \newline
25. मु॒ञ्च॒ति॒ वै॒श्वा॒न॒र्या वै᳚श्वान॒र्या मु॑ञ्चति मुञ्चति वैश्वान॒र्या शि॒क्यꣳ॑ शि॒क्यं॑ ॅवैश्वान॒र्या मु॑ञ्चति मुञ्चति वैश्वान॒र्या शि॒क्य᳚म् । \newline
26. वै॒श्वा॒न॒र्या शि॒क्यꣳ॑ शि॒क्यं॑ ॅवैश्वान॒र्या वै᳚श्वान॒र्या शि॒क्य॑ मा शि॒क्यं॑ ॅवैश्वान॒र्या वै᳚श्वान॒र्या शि॒क्य॑ मा । \newline
27. शि॒क्य॑ मा शि॒क्यꣳ॑ शि॒क्य॑ मा द॑त्ते दत्त॒ आ शि॒क्यꣳ॑ शि॒क्य॑ मा द॑त्ते । \newline
28. आ द॑त्ते दत्त॒ आ द॑त्ते स्व॒दय॑ति स्व॒दय॑ति दत्त॒ आ द॑त्ते स्व॒दय॑ति । \newline
29. द॒त्ते॒ स्व॒दय॑ति स्व॒दय॑ति दत्ते दत्ते स्व॒दय॑ त्ये॒वैव स्व॒दय॑ति दत्ते दत्ते स्व॒दय॑ त्ये॒व । \newline
30. स्व॒दय॑ त्ये॒वैव स्व॒दय॑ति स्व॒दय॑ त्ये॒वैन॑ देन दे॒व स्व॒दय॑ति स्व॒दय॑ त्ये॒वैन॑त् । \newline
31. ए॒वैन॑ देन दे॒वैवैन॑न् नैर्.ऋ॒तीर् नैर्॑.ऋ॒ती रे॑न दे॒वैवैन॑न् नैर्.ऋ॒तीः । \newline
32. ए॒न॒न् नै॒र्॒.ऋ॒तीर् नैर्॑.ऋ॒ती रे॑न देनन् नैर्.ऋ॒तीः कृ॒ष्णाः कृ॒ष्णा नैर्॑.ऋ॒ती रे॑न देनन् नैर्.ऋ॒तीः कृ॒ष्णाः । \newline
33. नै॒र्॒.ऋ॒तीः कृ॒ष्णाः कृ॒ष्णा नैर्॑.ऋ॒तीर् नैर्॑.ऋ॒तीः कृ॒ष्णा स्ति॒स्र स्ति॒स्रः कृ॒ष्णा नैर्॑.ऋ॒तीर् नैर्॑.ऋ॒तीः कृ॒ष्णा स्ति॒स्रः । \newline
34. नै॒र्॒.ऋ॒तीरिति॑ नैः - ऋ॒तीः । \newline
35. कृ॒ष्णा स्ति॒स्र स्ति॒स्रः कृ॒ष्णाः कृ॒ष्णा स्ति॒स्र स्तुष॑पक्वा॒ स्तुष॑पक्वा स्ति॒स्रः कृ॒ष्णाः कृ॒ष्णा स्ति॒स्र स्तुष॑पक्वाः । \newline
36. ति॒स्र स्तुष॑पक्वा॒ स्तुष॑पक्वा स्ति॒स्र स्ति॒स्र स्तुष॑पक्वा भवन्ति भवन्ति॒ तुष॑पक्वा स्ति॒स्र स्ति॒स्र स्तुष॑पक्वा भवन्ति । \newline
37. तुष॑पक्वा भवन्ति भवन्ति॒ तुष॑पक्वा॒ स्तुष॑पक्वा भवन्ति॒ निर्.ऋ॑त्यै॒ निर्.ऋ॑त्यै भवन्ति॒ तुष॑पक्वा॒ स्तुष॑पक्वा भवन्ति॒ निर्.ऋ॑त्यै । \newline
38. तुष॑पक्वा॒ इति॒ तुष॑ - प॒क्वाः॒ । \newline
39. भ॒व॒न्ति॒ निर्.ऋ॑त्यै॒ निर्.ऋ॑त्यै भवन्ति भवन्ति॒ निर्.ऋ॑त्यै॒ वै वै निर्.ऋ॑त्यै भवन्ति भवन्ति॒ निर्.ऋ॑त्यै॒ वै । \newline
40. निर्.ऋ॑त्यै॒ वै वै निर्.ऋ॑त्यै॒ निर्.ऋ॑त्यै॒ वा ए॒त दे॒तद् वै निर्.ऋ॑त्यै॒ निर्.ऋ॑त्यै॒ वा ए॒तत् । \newline
41. निर्.ऋ॑त्या॒ इति॒ निः - ऋ॒त्यै॒ । \newline
42. वा ए॒त दे॒तद् वै वा ए॒तद् भा॑ग॒धेय॑म् भाग॒धेय॑ मे॒तद् वै वा ए॒तद् भा॑ग॒धेय᳚म् । \newline
43. ए॒तद् भा॑ग॒धेय॑म् भाग॒धेय॑ मे॒त दे॒तद् भा॑ग॒धेयं॒ ॅयद् यद् भा॑ग॒धेय॑ मे॒त दे॒तद् भा॑ग॒धेयं॒ ॅयत् । \newline
44. भा॒ग॒धेयं॒ ॅयद् यद् भा॑ग॒धेय॑म् भाग॒धेयं॒ ॅयत् तुषा॒ स्तुषा॒ यद् भा॑ग॒धेय॑म् भाग॒धेयं॒ ॅयत् तुषाः᳚ । \newline
45. भा॒ग॒धेय॒मिति॑ भाग - धेय᳚म् । \newline
46. यत् तुषा॒ स्तुषा॒ यद् यत् तुषा॒ निर्.ऋ॑त्यै॒ निर्.ऋ॑त्यै॒ तुषा॒ यद् यत् तुषा॒ निर्.ऋ॑त्यै । \newline
47. तुषा॒ निर्.ऋ॑त्यै॒ निर्.ऋ॑त्यै॒ तुषा॒ स्तुषा॒ निर्.ऋ॑त्यै रू॒पꣳ रू॒पम् निर्.ऋ॑त्यै॒ तुषा॒ स्तुषा॒ निर्.ऋ॑त्यै रू॒पम् । \newline
48. निर्.ऋ॑त्यै रू॒पꣳ रू॒पम् निर्.ऋ॑त्यै॒ निर्.ऋ॑त्यै रू॒पम् कृ॒ष्णम् कृ॒ष्णꣳ रू॒पम् निर्.ऋ॑त्यै॒ निर्.ऋ॑त्यै रू॒पम् कृ॒ष्णम् । \newline
49. निर्.ऋ॑त्या॒ इति॒ निः - ऋ॒त्यै॒ । \newline
50. रू॒पम् कृ॒ष्णम् कृ॒ष्णꣳ रू॒पꣳ रू॒पम् कृ॒ष्णꣳ रू॒पेण॑ रू॒पेण॑ कृ॒ष्णꣳ रू॒पꣳ रू॒पम् कृ॒ष्णꣳ रू॒पेण॑ । \newline
51. कृ॒ष्णꣳ रू॒पेण॑ रू॒पेण॑ कृ॒ष्णम् कृ॒ष्णꣳ रू॒पेणै॒वैव रू॒पेण॑ कृ॒ष्णम् कृ॒ष्णꣳ रू॒पेणै॒व । \newline
52. रू॒पेणै॒वैव रू॒पेण॑ रू॒पेणै॒व निर्.ऋ॑ति॒म् निर्.ऋ॑ति मे॒व रू॒पेण॑ रू॒पेणै॒व निर्.ऋ॑तिम् । \newline
53. ए॒व निर्.ऋ॑ति॒म् निर्.ऋ॑ति मे॒वैव निर्.ऋ॑तिम् नि॒रव॑दयते नि॒रव॑दयते॒ निर्.ऋ॑ति मे॒वैव निर्.ऋ॑तिम् नि॒रव॑दयते । \newline
54. निर्.ऋ॑तिम् नि॒रव॑दयते नि॒रव॑दयते॒ निर्.ऋ॑ति॒म् निर्.ऋ॑तिम् नि॒रव॑दयत इ॒मा मि॒माम् नि॒रव॑दयते॒ निर्.ऋ॑ति॒म् निर्.ऋ॑तिम् नि॒रव॑दयत इ॒माम् । \newline
55. निर्.ऋ॑ति॒मिति॒ निः - ऋ॒ति॒म् । \newline
56. नि॒रव॑दयत इ॒मा मि॒माम् नि॒रव॑दयते नि॒रव॑दयत इ॒माम् दिश॒म् दिश॑ मि॒माम् नि॒रव॑दयते नि॒रव॑दयत इ॒माम् दिश᳚म् । \newline
57. नि॒रव॑दयत॒ इति॑ निः - अव॑दयते । \newline
58. इ॒माम् दिश॒म् दिश॑ मि॒मा मि॒माम् दिशं॑ ॅयन्ति यन्ति॒ दिश॑ मि॒मा मि॒माम् दिशं॑ ॅयन्ति । \newline
59. दिशं॑ ॅयन्ति यन्ति॒ दिश॒म् दिशं॑ ॅयन्त्ये॒षैषा य॑न्ति॒ दिश॒म् दिशं॑ ॅयन्त्ये॒षा । \newline
60. य॒न्त्ये॒षैषा य॑न्ति यन्त्ये॒षा वै वा ए॒षा य॑न्ति यन्त्ये॒षा वै । \newline
61. ए॒षा वै वा ए॒षैषा वै निर्.ऋ॑त्यै॒ निर्.ऋ॑त्यै॒ वा ए॒षैषा वै निर्.ऋ॑त्यै । \newline
\pagebreak
\markright{ TS 5.2.4.3  \hfill https://www.vedavms.in \hfill}

\section{ TS 5.2.4.3 }

\textbf{TS 5.2.4.3 } \newline
\textbf{Samhita Paata} \newline

वै निर्.ऋ॑त्यै॒ दिक् स्वाया॑मे॒व दि॒शि निर्.ऋ॑तिं नि॒रव॑दयते॒ स्वकृ॑त॒ इरि॑ण॒ उप॑ दधाति प्रद॒रे वै॒तद्वै निर्.ऋ॑त्या आ॒यत॑नꣳ॒॒ स्व ए॒वाऽऽ*यत॑ने॒ निर्.ऋ॑तिं नि॒रव॑दयते शि॒क्य॑म॒भ्युप॑ दधाति नैर्.ऋ॒तो वै पाशः॑ सा॒क्षादे॒वैनं॑ निर्.ऋतिपा॒शान्-मु॑ञ्चति ति॒स्र उप॑ दधाति त्रेधाविहि॒तो वै पुरु॑षो॒ यावा॑ने॒व पुरु॑ष॒स्तस्मा॒न्-निर्.ऋ॑ति॒मव॑ यजते॒ परा॑ची॒रुप॑ - [  ] \newline

\textbf{Pada Paata} \newline

वै । निर्.ऋ॑त्या॒ इति॒ निः - ऋ॒त्यै॒ । दिक् । स्वाया᳚म् । ए॒व । दि॒शि । निर्.ऋ॑ति॒मिति॒ निः - ऋ॒ति॒म् । नि॒रव॑दयत॒ इति॑ निः - अव॑दयते । स्वकृ॑त॒ इति॒ स्व - कृ॒ते॒ । इरि॑णे । उपेति॑ । द॒धा॒ति॒ । प्र॒द॒र इति॑ प्र - द॒रे । वा॒ । ए॒तत् । वै । निर्.ऋ॑त्या॒ इति॒ निः - ऋ॒त्याः॒ । आ॒यत॑न॒मित्या᳚ - यत॑नम् । स्वे । ए॒व । आ॒यत॑न॒ इत्या᳚ - यत॑ने । निर्.ऋ॑ति॒मिति॒ निः - ऋ॒ति॒म् । नि॒रव॑दयत॒ इति॑ निः - अव॑दयते । शि॒क्य᳚म् । अ॒भि । उपेति॑ । द॒धा॒ति॒ । नै॒र्॒.ऋ॒त इति॑ नैः - ऋ॒तः । वै । पाशः॑ । सा॒क्षादिति॑ स - अ॒क्षात् । ए॒व । ए॒न॒म् । नि॒र्॒.ऋ॒ति॒पा॒शादिति॑ निर्.ऋति - पा॒शात् । मु॒ञ्च॒ति॒ । ति॒स्रः । उपेति॑ । द॒धा॒ति॒ । त्रे॒धा॒वि॒हि॒त इति॑ त्रेधा - वि॒हि॒तः । वै । पुरु॑षः । यावान्॑ । ए॒व । पुरु॑षः । तस्मा᳚त् । निर्.ऋ॑ति॒मिति॒ निः - ऋ॒ति॒म् । अवेति॑ । य॒ज॒ते॒ । परा॑चीः । उपेति॑ ।  \newline


\textbf{Krama Paata} \newline

वै निर्.ऋ॑त्यै । निर्.ऋ॑त्यै॒ दिक् । निर्.ऋ॑त्या॒ इति॒ निः - ऋ॒त्यै॒ । दिख् स्वाया᳚म् । स्वाया॑मे॒व । ए॒व दि॒शि । दि॒शि निर्.ऋ॑तिम् । निर्.ऋ॑तिम् नि॒रव॑दयते । निर्.ऋ॑ति॒मिति॒ निः - ऋ॒ति॒म् । नि॒रव॑दयते॒ स्वकृ॑ते । नि॒रव॑दयत॒ इति॑ निः - अव॑दयते । स्वकृ॑त॒ इरि॑णे । स्वकृ॑त॒ इति॒ स्व - कृ॒ते॒ । इरि॑ण॒ उप॑ । उप॑ दधाति । द॒धा॒ति॒ प्र॒द॒रे । प्र॒द॒रे वा᳚ । प्र॒द॒र इति॑ प्र - द॒रे । वै॒तत् । ए॒तद् वै । वै निर्.ऋ॑त्याः । निर्.ऋ॑त्या आ॒यत॑नम् । निर्.ऋ॑त्या॒ इति॒ निः - ऋ॒त्याः॒ । आ॒यत॑नꣳ॒॒ स्वे । आ॒यत॑न॒मित्या᳚ - यत॑नम् । स्व ए॒व । ए॒वायत॑ने । आ॒यत॑ने॒ निर्.ऋ॑तिम् । आ॒यत॑न॒ इत्या᳚ - यत॑ने । निर्.ऋ॑तिम् नि॒रव॑दयते । निर्.ऋ॑ति॒मिति॒ निः - ऋ॒ति॒म् । नि॒रव॑दयते शि॒क्य᳚म् । नि॒रव॑दयत॒ इति॑ निः - अव॑दयते । शि॒क्य॑म॒भि । अ॒भ्युप॑ । उप॑ दधाति । द॒धा॒ति॒ नै॒र्॒.ऋ॒तः । नै॒र्॒.ऋ॒तो वै । नै॒र्॒.ऋ॒त इति॑ नैः - ऋ॒तः । वै पाशः॑ । पाशः॑ सा॒क्षात् । सा॒क्षादे॒व । सा॒क्षादिति॑ स - अ॒क्षात् । ए॒वैन᳚म् । ए॒न॒म् नि॒र्॒.ऋ॒ति॒पा॒शात् । नि॒र्॒.ऋ॒ति॒पा॒शान् मु॑ञ्चति । नि॒र्॒.ऋ॒ति॒पा॒शादिति॑ निर्.ऋति - पा॒शात् । मु॒ञ्च॒ति॒ ति॒स्रः । ति॒स्र उप॑ । उप॑ दधाति । द॒धा॒ति॒ त्रे॒धा॒वि॒हि॒तः । त्रे॒धा॒वि॒हि॒तो वै । त्रे॒धा॒वि॒हि॒त इति॑ त्रेधा - वि॒हि॒तः । वै पुरु॑षः । पुरु॑षो॒ यावान्॑ । यावा॑ने॒व । ए॒व पुरु॑षः । पुरु॑ष॒स्तस्मा᳚त् । तस्मा॒न् निर्.ऋ॑तिम् । निर्.ऋ॑ति॒मव॑ । निर्.ऋ॑ति॒मिति॒ निः - ऋ॒ति॒म् । अव॑ यजते । य॒ज॒ते॒ परा॑चीः । परा॑ची॒रुप॑ ( ) । उप॑ दधाति \newline

\textbf{Jatai Paata} \newline

1. वै निर्.ऋ॑त्यै॒ निर्.ऋ॑त्यै॒ वै वै निर्.ऋ॑त्यै । \newline
2. निर्.ऋ॑त्यै॒ दिग् दिङ् निर्.ऋ॑त्यै॒ निर्.ऋ॑त्यै॒ दिक् । \newline
3. निर्.ऋ॑त्या॒ इति॒ निः - ऋ॒त्यै॒ । \newline
4. दिख् स्वायाꣳ॒॒ स्वाया॒म् दिग् दिख् स्वाया᳚म् । \newline
5. स्वाया॑ मे॒वैव स्वायाꣳ॒॒ स्वाया॑ मे॒व । \newline
6. ए॒व दि॒शि दि॒श्ये॑वैव दि॒शि । \newline
7. दि॒शि निर्.ऋ॑ति॒न्निर्.ऋ॑तिम् दि॒शि दि॒शि निर्.ऋ॑तिम् । \newline
8. निर्.ऋ॑तिम् नि॒रव॑दयते नि॒रव॑दयते॒ निर्.ऋ॑ति॒म् निर्.ऋ॑तिम् नि॒रव॑दयते । \newline
9. निर्.ऋ॑ति॒मिति॒ निः - ऋ॒ति॒म् । \newline
10. नि॒रव॑दयते॒ स्वकृ॑ते॒ स्वकृ॑ते नि॒रव॑दयते नि॒रव॑दयते॒ स्वकृ॑ते । \newline
11. नि॒रव॑दयत॒ इति॑ निः - अव॑दयते । \newline
12. स्वकृ॑त॒ इरि॑ण॒ इरि॑णे॒ स्वकृ॑ते॒ स्वकृ॑त॒ इरि॑णे । \newline
13. स्वकृ॑त॒ इति॒ स्व - कृ॒ते॒ । \newline
14. इरि॑ण॒ उपोपे रि॑ण॒ इरि॑ण॒ उप॑ । \newline
15. उप॑ दधाति दधा॒ त्युपोप॑ दधाति । \newline
16. द॒धा॒ति॒ प्र॒द॒रे प्र॑द॒रे द॑धाति दधाति प्रद॒रे । \newline
17. प्र॒द॒रे वा॑ वा प्रद॒रे प्र॑द॒रे वा᳚ । \newline
18. प्र॒द॒र इति॑ प्र - द॒रे । \newline
19. वै॒त दे॒तद् वा॑ वै॒तत् । \newline
20. ए॒तद् वै वा ए॒त दे॒तद् वै । \newline
21. वै निर्.ऋ॑त्या॒ निर्.ऋ॑त्या॒ वै वै निर्.ऋ॑त्याः । \newline
22. निर्.ऋ॑त्या आ॒यत॑न मा॒यत॑न॒म् निर्.ऋ॑त्या॒ निर्.ऋ॑त्या आ॒यत॑नम् । \newline
23. निर्.ऋ॑त्या॒ इति॒ निः - ऋ॒त्याः॒ । \newline
24. आ॒यत॑नꣳ॒॒ स्वे स्व आ॒यत॑न मा॒यत॑नꣳ॒॒ स्वे । \newline
25. आ॒यत॑न॒मित्या᳚ - यत॑नम् । \newline
26. स्व ए॒वैव स्वे स्व ए॒व । \newline
27. ए॒वा यत॑न आ॒यत॑न ए॒वै वायत॑ने । \newline
28. आ॒यत॑ने॒ निर्.ऋ॑ति॒म् निर्.ऋ॑ति मा॒यत॑न आ॒यत॑ने॒ निर्.ऋ॑तिम् । \newline
29. आ॒यत॑न॒ इत्या᳚ - यत॑ने । \newline
30. निर्.ऋ॑तिम् नि॒रव॑दयते नि॒रव॑दयते॒ निर्.ऋ॑ति॒म् निर्.ऋ॑तिम् नि॒रव॑दयते । \newline
31. निर्.ऋ॑ति॒मिति॒ निः - ऋ॒ति॒म् । \newline
32. नि॒रव॑दयते शि॒क्यꣳ॑ शि॒क्य॑म् नि॒रव॑दयते नि॒रव॑दयते शि॒क्य᳚म् । \newline
33. नि॒रव॑दयत॒ इति॑ निः - अव॑दयते । \newline
34. शि॒क्य॑ म॒भ्य॑भि शि॒क्यꣳ॑ शि॒क्य॑ म॒भि । \newline
35. अ॒भ्यु पोपा॒ भ्य॑ भ्युप॑ । \newline
36. उप॑ दधाति दधा॒ त्युपोप॑ दधाति । \newline
37. द॒धा॒ति॒ नै॒र्॒.ऋ॒तो नैर्॑.ऋ॒तो द॑धाति दधाति नैर्.ऋ॒तः । \newline
38. नै॒र्॒.ऋ॒तो वै वै नैर्॑.ऋ॒तो नैर्॑.ऋ॒तो वै । \newline
39. नै॒र्॒.ऋ॒त इति॑ नैः - ऋ॒तः । \newline
40. वै पाशः॒ पाशो॒ वै वै पाशः॑ । \newline
41. पाशः॑ सा॒क्षाथ् सा॒क्षात् पाशः॒ पाशः॑ सा॒क्षात् । \newline
42. सा॒क्षा दे॒वैव सा॒क्षाथ् सा॒क्षा दे॒व । \newline
43. सा॒क्षादिति॑ स - अ॒क्षात् । \newline
44. ए॒वैन॑ मेन मे॒वै वैन᳚म् । \newline
45. ए॒न॒म् नि॒र्॒.ऋ॒ति॒पा॒शान् निर्॑.ऋतिपा॒शादे॑न मेनम् निर्.ऋतिपा॒शात् । \newline
46. नि॒र्॒.ऋ॒ति॒पा॒शान् मु॑ञ्चति मुञ्चति निर्.ऋतिपा॒शान् निर्॑.ऋतिपा॒शान् मु॑ञ्चति । \newline
47. नि॒र्॒.ऋ॒ति॒पा॒शादिति॑ निर्.ऋति - पा॒शात् । \newline
48. मु॒ञ्च॒ति॒ ति॒स्र स्ति॒स्रो मु॑ञ्चति मुञ्चति ति॒स्रः । \newline
49. ति॒स्र उपोप॑ ति॒स्र स्ति॒स्र उप॑ । \newline
50. उप॑ दधाति दधा॒ त्युपोप॑ दधाति । \newline
51. द॒धा॒ति॒ त्रे॒धा॒वि॒हि॒त स्त्रे॑धाविहि॒तो द॑धाति दधाति त्रेधाविहि॒तः । \newline
52. त्रे॒धा॒वि॒हि॒तो वै वै त्रे॑धाविहि॒त स्त्रे॑धाविहि॒तो वै । \newline
53. त्रे॒धा॒वि॒हि॒त इति॑ त्रेधा - वि॒हि॒तः । \newline
54. वै पुरु॑षः॒ पुरु॑षो॒ वै वै पुरु॑षः । \newline
55. पुरु॑षो॒ यावा॒न्॒. यावा॒न् पुरु॑षः॒ पुरु॑षो॒ यावान्॑ । \newline
56. यावा॑ ने॒वैव यावा॒न्॒. यावा॑ ने॒व । \newline
57. ए॒व पुरु॑षः॒ पुरु॑ष ए॒वैव पुरु॑षः । \newline
58. पुरु॑ष॒ स्तस्मा॒त् तस्मा॒त् पुरु॑षः॒ पुरु॑ष॒ स्तस्मा᳚त् । \newline
59. तस्मा॒न् निर्.ऋ॑ति॒म् निर्.ऋ॑ति॒म् तस्मा॒त् तस्मा॒न् निर्.ऋ॑तिम् । \newline
60. निर्.ऋ॑ति॒ मवाव॒ निर्.ऋ॑ति॒म् निर्.ऋ॑ति॒ मव॑ । \newline
61. निर्.ऋ॑ति॒मिति॒ निः - ऋ॒ति॒म् । \newline
62. अव॑ यजते यज॒ते ऽवाव॑ यजते । \newline
63. य॒ज॒ते॒ परा॑चीः॒ परा॑चीर् यजते यजते॒ परा॑चीः । \newline
64. परा॑ची॒ रुपोप॒ परा॑चीः॒ परा॑ची॒ रुप॑ । \newline
65. उप॑ दधाति दधा॒ त्युपोप॑ दधाति । \newline

\textbf{Ghana Paata } \newline

1. वै निर्.ऋ॑त्यै॒ निर्.ऋ॑त्यै॒ वै वै निर्.ऋ॑त्यै॒ दिग् दिङ् निर्.ऋ॑त्यै॒ वै वै निर्.ऋ॑त्यै॒ दिक् । \newline
2. निर्.ऋ॑त्यै॒ दिग् दिङ् निर्.ऋ॑त्यै॒ निर्.ऋ॑त्यै॒ दिख् स्वायाꣳ॒॒ स्वाया॒म् दिङ् निर्.ऋ॑त्यै॒ निर्.ऋ॑त्यै॒ दिख् स्वाया᳚म् । \newline
3. निर्.ऋ॑त्या॒ इति॒ निः - ऋ॒त्यै॒ । \newline
4. दिख् स्वायाꣳ॒॒ स्वाया॒म् दिग् दिख् स्वाया॑ मे॒वैव स्वाया॒म् दिग् दिख् स्वाया॑ मे॒व । \newline
5. स्वाया॑ मे॒वैव स्वायाꣳ॒॒ स्वाया॑ मे॒व दि॒शि दि॒श्ये॑व स्वायाꣳ॒॒ स्वाया॑ मे॒व दि॒शि । \newline
6. ए॒व दि॒शि दि॒श्ये॑वैव दि॒शि निर्.ऋ॑ति॒म् निर्.ऋ॑तिम् दि॒श्ये॑वैव दि॒शि निर्.ऋ॑तिम् । \newline
7. दि॒शि निर्.ऋ॑ति॒म् निर्.ऋ॑तिम् दि॒शि दि॒शि निर्.ऋ॑तिम् नि॒रव॑दयते नि॒रव॑दयते॒ निर्.ऋ॑तिम् दि॒शि दि॒शि निर्.ऋ॑तिम् नि॒रव॑दयते । \newline
8. निर्.ऋ॑तिम् नि॒रव॑दयते नि॒रव॑दयते॒ निर्.ऋ॑ति॒म् निर्.ऋ॑तिम् नि॒रव॑दयते॒ स्वकृ॑ते॒ स्वकृ॑ते नि॒रव॑दयते॒ निर्.ऋ॑ति॒म् निर्.ऋ॑तिम् नि॒रव॑दयते॒ स्वकृ॑ते । \newline
9. निर्.ऋ॑ति॒मिति॒ निः - ऋ॒ति॒म् । \newline
10. नि॒रव॑दयते॒ स्वकृ॑ते॒ स्वकृ॑ते नि॒रव॑दयते नि॒रव॑दयते॒ स्वकृ॑त॒ इरि॑ण॒ इरि॑णे॒ स्वकृ॑ते नि॒रव॑दयते नि॒रव॑दयते॒ स्वकृ॑त॒ इरि॑णे । \newline
11. नि॒रव॑दयत॒ इति॑ निः - अव॑दयते । \newline
12. स्वकृ॑त॒ इरि॑ण॒ इरि॑णे॒ स्वकृ॑ते॒ स्वकृ॑त॒ इरि॑ण॒ उपोपे रि॑णे॒ स्वकृ॑ते॒ स्वकृ॑त॒ इरि॑ण॒ उप॑ । \newline
13. स्वकृ॑त॒ इति॒ स्व - कृ॒ते॒ । \newline
14. इरि॑ण॒ उपोपे रि॑ण॒ इरि॑ण॒ उप॑ दधाति दधा॒त्युपे रि॑ण॒ इरि॑ण॒ उप॑ दधाति । \newline
15. उप॑ दधाति दधा॒ त्युपोप॑ दधाति प्रद॒रे प्र॑द॒रे द॑धा॒ त्युपोप॑ दधाति प्रद॒रे । \newline
16. द॒धा॒ति॒ प्र॒द॒रे प्र॑द॒रे द॑धाति दधाति प्रद॒रे वा॑ वा प्रद॒रे द॑धाति दधाति प्रद॒रे वा᳚ । \newline
17. प्र॒द॒रे वा॑ वा प्रद॒रे प्र॑द॒रे वै॒त दे॒तद् वा᳚ प्रद॒रे प्र॑द॒रे वै॒तत् । \newline
18. प्र॒द॒र इति॑ प्र - द॒रे । \newline
19. वै॒त दे॒तद् वा॑ वै॒तद् वै वा ए॒तद् वा॑ वै॒तद् वै । \newline
20. ए॒तद् वै वा ए॒त दे॒तद् वै निर्.ऋ॑त्या॒ निर्.ऋ॑त्या॒ वा ए॒त दे॒तद् वै निर्.ऋ॑त्याः । \newline
21. वै निर्.ऋ॑त्या॒ निर्.ऋ॑त्या॒ वै वै निर्.ऋ॑त्या आ॒यत॑न मा॒यत॑न॒म् निर्.ऋ॑त्या॒ वै वै निर्.ऋ॑त्या आ॒यत॑नम् । \newline
22. निर्.ऋ॑त्या आ॒यत॑न मा॒यत॑न॒म् निर्.ऋ॑त्या॒ निर्.ऋ॑त्या आ॒यत॑नꣳ॒॒ स्वे स्व आ॒यत॑न॒म् निर्.ऋ॑त्या॒ निर्.ऋ॑त्या आ॒यत॑नꣳ॒॒ स्वे । \newline
23. निर्.ऋ॑त्या॒ इति॒ निः - ऋ॒त्याः॒ । \newline
24. आ॒यत॑नꣳ॒॒ स्वे स्व आ॒यत॑न मा॒यत॑नꣳ॒॒ स्व ए॒वैव स्व आ॒यत॑न मा॒यत॑नꣳ॒॒ स्व ए॒व । \newline
25. आ॒यत॑न॒मित्या᳚ - यत॑नम् । \newline
26. स्व ए॒वैव स्वे स्व ए॒वायत॑न आ॒यत॑न ए॒व स्वे स्व ए॒वायत॑ने । \newline
27. ए॒वायत॑न आ॒यत॑न ए॒वैवायत॑ने॒ निर्.ऋ॑ति॒म् निर्.ऋ॑ति मा॒यत॑न ए॒वैवायत॑ने॒ निर्.ऋ॑तिम् । \newline
28. आ॒यत॑ने॒ निर्.ऋ॑ति॒म् निर्.ऋ॑ति मा॒यत॑न आ॒यत॑ने॒ निर्.ऋ॑तिम् नि॒रव॑दयते नि॒रव॑दयते॒ निर्.ऋ॑ति मा॒यत॑न आ॒यत॑ने॒ निर्.ऋ॑तिम् नि॒रव॑दयते । \newline
29. आ॒यत॑न॒ इत्या᳚ - यत॑ने । \newline
30. निर्.ऋ॑तिम् नि॒रव॑दयते नि॒रव॑दयते॒ निर्.ऋ॑ति॒म् निर्.ऋ॑तिम् नि॒रव॑दयते शि॒क्यꣳ॑ शि॒क्य॑म् नि॒रव॑दयते॒ निर्.ऋ॑ति॒म् निर्.ऋ॑तिम् नि॒रव॑दयते शि॒क्य᳚म् । \newline
31. निर्.ऋ॑ति॒मिति॒ निः - ऋ॒ति॒म् । \newline
32. नि॒रव॑दयते शि॒क्यꣳ॑ शि॒क्य॑म् नि॒रव॑दयते नि॒रव॑दयते शि॒क्य॑ म॒भ्य॑भि शि॒क्य॑म् नि॒रव॑दयते नि॒रव॑दयते शि॒क्य॑ म॒भि । \newline
33. नि॒रव॑दयत॒ इति॑ निः - अव॑दयते । \newline
34. शि॒क्य॑ म॒भ्य॑भि शि॒क्यꣳ॑ शि॒क्य॑ म॒भ्युपोपा॒भि शि॒क्यꣳ॑ शि॒क्य॑ म॒भ्युप॑ । \newline
35. अ॒भ्युपोपा॒ भ्य॑ भ्युप॑ दधाति दधा॒ त्युपा॒ भ्य॑ भ्युप॑ दधाति । \newline
36. उप॑ दधाति दधा॒ त्युपोप॑ दधाति नैर्.ऋ॒तो नैर्॑.ऋ॒तो द॑धा॒ त्युपोप॑ दधाति नैर्.ऋ॒तः । \newline
37. द॒धा॒ति॒ नै॒र्॒.ऋ॒तो नैर्॑.ऋ॒तो द॑धाति दधाति नैर्.ऋ॒तो वै वै नैर्॑.ऋ॒तो द॑धाति दधाति नैर्.ऋ॒तो वै । \newline
38. नै॒र्॒.ऋ॒तो वै वै नैर्॑.ऋ॒तो नैर्॑.ऋ॒तो वै पाशः॒ पाशो॒ वै नैर्॑.ऋ॒तो नैर्॑.ऋ॒तो वै पाशः॑ । \newline
39. नै॒र्॒.ऋ॒त इति॑ नैः - ऋ॒तः । \newline
40. वै पाशः॒ पाशो॒ वै वै पाशः॑ सा॒क्षाथ् सा॒क्षात् पाशो॒ वै वै पाशः॑ सा॒क्षात् । \newline
41. पाशः॑ सा॒क्षाथ् सा॒क्षात् पाशः॒ पाशः॑ सा॒क्षा दे॒वैव सा॒क्षात् पाशः॒ पाशः॑ सा॒क्षा दे॒व । \newline
42. सा॒क्षा दे॒वैव सा॒क्षाथ् सा॒क्षा दे॒वैन॑ मेन मे॒व सा॒क्षाथ् सा॒क्षा दे॒वैन᳚म् । \newline
43. सा॒क्षादिति॑ स - अ॒क्षात् । \newline
44. ए॒वैन॑ मेन मे॒वैवैन॑म् निर्.ऋतिपा॒शान् निर्॑.ऋतिपा॒शा दे॑न मे॒वैवैन॑म् निर्.ऋतिपा॒शात् । \newline
45. ए॒न॒म् नि॒र्॒.ऋ॒ति॒पा॒शान् निर्॑.ऋतिपा॒शा दे॑न मेनम् निर्.ऋतिपा॒शान् मु॑ञ्चति मुञ्चति निर्.ऋतिपा॒शा दे॑न मेनम् निर्.ऋतिपा॒शान् मु॑ञ्चति । \newline
46. नि॒र्॒.ऋ॒ति॒पा॒शान् मु॑ञ्चति मुञ्चति निर्.ऋतिपा॒शान् निर्॑.ऋतिपा॒शान् मु॑ञ्चति ति॒स्र स्ति॒स्रो मु॑ञ्चति निर्.ऋतिपा॒शान् निर्॑.ऋतिपा॒शान् मु॑ञ्चति ति॒स्रः । \newline
47. नि॒र्॒.ऋ॒ति॒पा॒शादिति॑ निर्.ऋति - पा॒शात् । \newline
48. मु॒ञ्च॒ति॒ ति॒स्र स्ति॒स्रो मु॑ञ्चति मुञ्चति ति॒स्र उपोप॑ ति॒स्रो मु॑ञ्चति मुञ्चति ति॒स्र उप॑ । \newline
49. ति॒स्र उपोप॑ ति॒स्र स्ति॒स्र उप॑ दधाति दधा॒ त्युप॑ ति॒स्र स्ति॒स्र उप॑ दधाति । \newline
50. उप॑ दधाति दधा॒ त्युपोप॑ दधाति त्रेधाविहि॒त स्त्रे॑धाविहि॒तो द॑धा॒ त्युपोप॑ दधाति त्रेधाविहि॒तः । \newline
51. द॒धा॒ति॒ त्रे॒धा॒वि॒हि॒त स्त्रे॑धाविहि॒तो द॑धाति दधाति त्रेधाविहि॒तो वै वै त्रे॑धाविहि॒तो द॑धाति दधाति त्रेधाविहि॒तो वै । \newline
52. त्रे॒धा॒वि॒हि॒तो वै वै त्रे॑धाविहि॒त स्त्रे॑धाविहि॒तो वै पुरु॑षः॒ पुरु॑षो॒ वै त्रे॑धाविहि॒त स्त्रे॑धाविहि॒तो वै पुरु॑षः । \newline
53. त्रे॒धा॒वि॒हि॒त इति॑ त्रेधा - वि॒हि॒तः । \newline
54. वै पुरु॑षः॒ पुरु॑षो॒ वै वै पुरु॑षो॒ यावा॒न्॒. यावा॒न् पुरु॑षो॒ वै वै पुरु॑षो॒ यावान्॑ । \newline
55. पुरु॑षो॒ यावा॒न्॒. यावा॒न् पुरु॑षः॒ पुरु॑षो॒ यावा॑ ने॒वैव यावा॒न् पुरु॑षः॒ पुरु॑षो॒ यावा॑ ने॒व । \newline
56. यावा॑ ने॒वैव यावा॒न्॒. यावा॑ ने॒व पुरु॑षः॒ पुरु॑ष ए॒व यावा॒न्॒. यावा॑ ने॒व पुरु॑षः । \newline
57. ए॒व पुरु॑षः॒ पुरु॑ष ए॒वैव पुरु॑ष॒ स्तस्मा॒त् तस्मा॒त् पुरु॑ष ए॒वैव पुरु॑ष॒ स्तस्मा᳚त् । \newline
58. पुरु॑ष॒ स्तस्मा॒त् तस्मा॒त् पुरु॑षः॒ पुरु॑ष॒ स्तस्मा॒न् निर्.ऋ॑ति॒म् निर्.ऋ॑ति॒म् तस्मा॒त् पुरु॑षः॒ पुरु॑ष॒ स्तस्मा॒न् निर्.ऋ॑तिम् । \newline
59. तस्मा॒न् निर्.ऋ॑ति॒म् निर्.ऋ॑ति॒म् तस्मा॒त् तस्मा॒न् निर्.ऋ॑ति॒ मवाव॒ निर्.ऋ॑ति॒म् तस्मा॒त् तस्मा॒न् निर्.ऋ॑ति॒ मव॑ । \newline
60. निर्.ऋ॑ति॒ मवाव॒ निर्.ऋ॑ति॒म् निर्.ऋ॑ति॒ मव॑ यजते यज॒ते ऽव॒ निर्.ऋ॑ति॒म् निर्.ऋ॑ति॒ मव॑ यजते । \newline
61. निर्.ऋ॑ति॒मिति॒ निः - ऋ॒ति॒म् । \newline
62. अव॑ यजते यज॒ते ऽवाव॑ यजते॒ परा॑चीः॒ परा॑चीर् यज॒ते ऽवाव॑ यजते॒ परा॑चीः । \newline
63. य॒ज॒ते॒ परा॑चीः॒ परा॑चीर् यजते यजते॒ परा॑ची॒ रुपोप॒ परा॑चीर् यजते यजते॒ परा॑ची॒ रुप॑ । \newline
64. परा॑ची॒ रुपोप॒ परा॑चीः॒ परा॑ची॒ रुप॑ दधाति दधा॒ त्युप॒ परा॑चीः॒ परा॑ची॒ रुप॑ दधाति । \newline
65. उप॑ दधाति दधा॒ त्युपोप॑ दधाति॒ परा॑ची॒म् परा॑चीम् दधा॒ त्युपोप॑ दधाति॒ परा॑चीम् । \newline
\pagebreak
\markright{ TS 5.2.4.4  \hfill https://www.vedavms.in \hfill}

\section{ TS 5.2.4.4 }

\textbf{TS 5.2.4.4 } \newline
\textbf{Samhita Paata} \newline

दधाति॒ परा॑चीमे॒वास्मा॒न्-निर्.ऋ॑तिं॒ प्रणु॑द॒ते ऽप्र॑तीक्ष॒मा य॑न्ति॒ निर्.ऋ॑त्या अ॒न्तर्.हि॑त्यै मार्जयि॒त्वोप॑ तिष्ठन्ते मेद्ध्य॒त्वाय॒ गार्.ह॑पत्य॒मुप॑ तिष्ठन्ते निर्.ऋति लो॒क ए॒व च॑रि॒त्वा पू॒ता दे॑वलो॒कमु॒पाव॑र्तन्त॒ एक॒योप॑ तिष्ठन्त एक॒धैव यज॑माने वी॒र्यं॑ दधति नि॒वेश॑नः स॒ङ्गम॑नो॒ वसू॑ना॒मित्या॑ह प्र॒जा वै प॒शवो॒ वसु॑ प्र॒जयै॒वैनं॑ प॒शुभिः॒ सम॑र्द्धयन्ति ॥ \newline

\textbf{Pada Paata} \newline

द॒धा॒ति॒ । परा॑चीम् । ए॒व । अ॒स्मा॒त् । निर्.ऋ॑ति॒मिति॒ निः - ऋ॒ति॒म् । प्रेति॑ । नु॒द॒ते॒ । अप्र॑तीक्ष॒मित्यप्र॑ति - ई॒क्ष॒म् । एति॑ । य॒न्ति॒ । निर्.ऋ॑त्या॒ इति॒ निः - ऋ॒त्याः॒ । अ॒न्तर्.हि॑त्या॒ इत्य॒न्तः - हि॒त्यै॒ । मा॒र्ज॒यि॒त्वा । उपेति॑ । ति॒ष्ठ॒न्ते॒ । मे॒द्ध्य॒त्वायेति॑ मेद्ध्य - त्वाय॑ । गार्.ह॑पत्य॒मिति॒ गार्.ह॑ - प॒त्य॒म् । उपेति॑ । ति॒ष्ठ॒न्ते॒ । नि॒र्॒.ऋ॒ति॒लो॒क इति॑ निर्.ऋति - लो॒के । ए॒व । च॒रि॒त्वा । पू॒ताः । दे॒व॒लो॒कमिति॑ देव -  लो॒कम् । उ॒पाव॑र्तन्त॒ इत्यु॑प - आव॑र्तन्ते । एक॑या । उपेति॑ । ति॒ष्ठ॒न्ते॒ । ए॒क॒धेत्ये॑क - धा । ए॒व । यज॑माने । वी॒र्य᳚म् । द॒ध॒ति॒ । नि॒वेश॑न॒ इति॑ नि-वेश॑नः । स॒ङ्गम॑न॒ इति॑ सं- गम॑नः । वसू॑नाम् । इति॑ । आ॒ह॒ । प्र॒जेति॑ प्र - जा । वै ।   प॒शवः॑ । वसु॑ । प्र॒जयेति॑ प्र - जया᳚ । ए॒व । ए॒न॒म् । प॒शुभि॒रिति॑ प॒शु - भिः॒ । समिति॑ । अ॒द्‌र्ध॒य॒न्ति॒ ॥  \newline


\textbf{Krama Paata} \newline

द॒धा॒ति॒ परा॑चीम् । परा॑चीमे॒व । ए॒वास्मा᳚त् । अ॒स्मा॒न् निर्.ऋ॑तिम् । निर्.ऋ॑ति॒म् प्र । निर्.ऋ॑ति॒मिति॒ निः - ऋ॒ति॒म् । प्र णु॑दते । नु॒द॒तेऽप्र॑तीक्षम् । अप्र॑तीक्ष॒मा । अप्र॑तीक्ष॒मित्यप्र॑ति - 
ई॒क्ष॒म् । आ य॑न्ति । य॒न्ति॒ निर्.ऋ॑त्याः । निर्.ऋ॑त्या अ॒न्तर्.हि॑त्यै । निर्.ऋ॑त्या॒ इति॒ निः - ऋ॒त्याः॒ । अ॒न्तर्.हि॑त्यै मार्जयि॒त्वा । अ॒न्तर्.हि॑त्या॒ इत्य॒न्तः - हि॒त्यै॒ । मा॒र्ज॒यि॒त्वोप॑ । उप॑ तिष्ठन्ते । ति॒ष्ठ॒न्ते॒ मे॒द्ध्य॒त्वाय॑ । मे॒द्ध्य॒त्वाय॒ गार्.ह॑पत्यम् । मे॒द्ध्य॒त्वायेति॑ मेद्ध्य - त्वाय॑ । गार्.ह॑पत्य॒मुप॑ । गार्.ह॑पत्य॒मिति॒ गार्.ह॑ - प॒त्य॒म् । उप॑ तिष्ठन्ते । ति॒ष॒न्ते॒ नि॒र्॒.ऋ॒ति॒लो॒के । नि॒र्॒.ऋ॒ति॒लो॒क ए॒व । नि॒र्॒.ऋ॒ति॒लो॒क इति॑ निर्.ऋति - लो॒के । ए॒व च॑रि॒त्वा । च॒रि॒त्वा पू॒ताः । पू॒ता दे॑वलो॒कम् । दे॒व॒लो॒कमु॒पाव॑र्तन्ते । दे॒व॒लो॒कमिति॑ देव - लो॒कम् । उ॒पाव॑र्तन्त॒ एक॑या । उ॒पाव॑र्तन्त॒ इत्यु॑प - आव॑र्तन्ते । एक॒योप॑ । उप॑ तिष्ठन्ते । ति॒ष्ठ॒न्त॒ ए॒क॒धा । ए॒क॒धैव । ए॒क॒धेत्ये॑क - धा । ए॒व यज॑माने । यज॑माने वी॒र्य᳚म् । वी॒र्य॑म् दधति । द॒ध॒ति॒ नि॒वेश॑नः । नि॒वेश॑नः स॒ङ्गम॑नः । नि॒वेश॑न॒ इति॑ नि - वेश॑नः । स॒ङ्गम॑नो॒ वसू॑नाम् । स॒ङ्गम॑न॒ इति॑ सम् - गम॑नः । वसू॑ना॒मिति॑ । इत्या॑ह । आ॒ह॒ प्र॒जा । प्र॒जा वै । प्र॒जेति॑ प्र - जा । वै प॒शवः॑ । प॒शवो॒ वसु॑ । वसु॑ प्र॒जया᳚ । प्र॒जयै॒व । प्र॒जयेति॑ प्र - जया᳚ । ए॒वैन᳚म् । ए॒न॒म् प॒शुभिः॑ । प॒शुभिः॒ सम् । प॒शुभि॒रिति॑ प॒शु - भिः॒ । सम॑र्द्धयन्ति । अ॒र्द्ध॒य॒न्तीत्य॑र्द्धयन्ति । \newline

\textbf{Jatai Paata} \newline

1. द॒धा॒ति॒ परा॑ची॒म् परा॑चीम् दधाति दधाति॒ परा॑चीम् । \newline
2. परा॑ची मे॒वैव परा॑ची॒म् परा॑ची मे॒व । \newline
3. ए॒वास्मा॑ दस्मा दे॒वैवा स्मा᳚त् । \newline
4. अ॒स्मा॒न् निर्.ऋ॑ति॒म् निर्.ऋ॑ति मस्मा दस्मा॒न् निर्.ऋ॑तिम् । \newline
5. निर्.ऋ॑ति॒म् प्र प्र णिर्.ऋ॑ति॒म् निर्.ऋ॑ति॒म् प्र । \newline
6. निर्.ऋ॑ति॒मिति॒ निः - ऋ॒ति॒म् । \newline
7. प्र णु॑दते नुदते॒ प्र प्र णु॑दते । \newline
8. नु॒द॒ते ऽप्र॑तीक्ष॒ मप्र॑तीक्षम् नुदते नुद॒ते ऽप्र॑तीक्षम् । \newline
9. अप्र॑तीक्ष॒ मा ऽप्र॑तीक्ष॒ मप्र॑तीक्ष॒ मा । \newline
10. अप्र॑तीक्ष॒मित्यप्र॑ति - ई॒क्ष॒म् । \newline
11. आ य॑न्ति य॒न्त्या य॑न्ति । \newline
12. य॒न्ति॒ निर्.ऋ॑त्या॒ निर्.ऋ॑त्या यन्ति यन्ति॒ निर्.ऋ॑त्याः । \newline
13. निर्.ऋ॑त्या अ॒न्तर्.हि॑त्या अ॒न्तर्.हि॑त्यै॒ निर्.ऋ॑त्या॒ निर्.ऋ॑त्या अ॒न्तर्.हि॑त्यै । \newline
14. निर्.ऋ॑त्या॒ इति॒ निः - ऋ॒त्याः॒ । \newline
15. अ॒न्तर्.हि॑त्यै मार्जयि॒त्वा मा᳚र्जयि॒त्वा ऽन्तर्.हि॑त्या अ॒न्तर्.हि॑त्यै मार्जयि॒त्वा । \newline
16. अ॒न्तर्.हि॑त्या॒ इत्य॒न्तः - हि॒त्यै॒ । \newline
17. मा॒र्ज॒यि॒ त्वोपोप॑ मार्जयि॒त्वा मा᳚र्जयि॒ त्वोप॑ । \newline
18. उप॑ तिष्ठन्ते तिष्ठन्त॒ उपोप॑ तिष्ठन्ते । \newline
19. ति॒ष्ठ॒न्ते॒ मे॒द्ध्य॒त्वाय॑ मेद्ध्य॒त्वाय॑ तिष्ठन्ते तिष्ठन्ते मेद्ध्य॒त्वाय॑ । \newline
20. मे॒द्ध्य॒त्वाय॒ गार्.ह॑पत्य॒म् गार्.ह॑पत्यम् मेद्ध्य॒त्वाय॑ मेद्ध्य॒त्वाय॒ गार्.ह॑पत्यम् । \newline
21. मे॒द्ध्य॒त्वायेति॑ मेद्ध्य - त्वाय॑ । \newline
22. गार्.ह॑पत्य॒ मुपोप॒ गार्.ह॑पत्य॒म् गार्.ह॑पत्य॒ मुप॑ । \newline
23. गार्.ह॑पत्य॒मिति॒ गार्.ह॑ - प॒त्य॒म् । \newline
24. उप॑ तिष्ठन्ते तिष्ठन्त॒ उपोप॑ तिष्ठन्ते । \newline
25. ति॒ष्ठ॒न्ते॒ नि॒र्॒.ऋ॒ति॒लो॒के निर्॑.ऋतिलो॒के ति॑ष्ठन्ते तिष्ठन्ते निर्.ऋतिलो॒के । \newline
26. नि॒र्॒.ऋ॒ति॒लो॒क ए॒वैव निर्॑.ऋतिलो॒के निर्॑.ऋतिलो॒क ए॒व । \newline
27. नि॒र्॒.ऋ॒ति॒लो॒क इति॑ निर्.ऋति - लो॒के । \newline
28. ए॒व च॑रि॒त्वा च॑रि॒ त्वैवैव च॑रि॒त्वा । \newline
29. च॒रि॒त्वा पू॒ताः पू॒ता श्च॑रि॒त्वा च॑रि॒त्वा पू॒ताः । \newline
30. पू॒ता दे॑वलो॒कम् दे॑वलो॒कम् पू॒ताः पू॒ता दे॑वलो॒कम् । \newline
31. दे॒व॒लो॒क मु॒पाव॑र्तन्त उ॒पाव॑र्तन्ते देवलो॒कम् दे॑वलो॒क मु॒पाव॑र्तन्ते । \newline
32. दे॒व॒लो॒कमिति॑ देव - लो॒कम् । \newline
33. उ॒पाव॑र्तन्त॒ एक॒ यैक॑ यो॒पाव॑र्तन्त उ॒पाव॑र्तन्त॒ एक॑या । \newline
34. उ॒पाव॑र्तन्त॒ इत्यु॑प - आव॑र्तन्ते । \newline
35. एक॒ योपो पैक॒ यैक॒ योप॑ । \newline
36. उप॑ तिष्ठन्ते तिष्ठन्त॒ उपोप॑ तिष्ठन्ते । \newline
37. ति॒ष्ठ॒न्त॒ ए॒क॒ धैक॒धा ति॑ष्ठन्ते तिष्ठन्त एक॒धा । \newline
38. ए॒क॒ धैवै वैक॒ धैक॒ धैव । \newline
39. ए॒क॒धेत्ये॑क - धा । \newline
40. ए॒व यज॑माने॒ यज॑मान ए॒वैव यज॑माने । \newline
41. यज॑माने वी॒र्यं॑ ॅवी॒र्यं॑ ॅयज॑माने॒ यज॑माने वी॒र्य᳚म् । \newline
42. वी॒र्य॑म् दधति दधति वी॒र्यं॑ ॅवी॒र्य॑म् दधति । \newline
43. द॒ध॒ति॒ नि॒वेश॑नो नि॒वेश॑नो दधति दधति नि॒वेश॑नः । \newline
44. नि॒वेश॑नः स॒ङ्गम॑नः स॒ङ्गम॑नो नि॒वेश॑नो नि॒वेश॑नः स॒ङ्गम॑नः । \newline
45. नि॒वेश॑न॒ इति॑ नि - वेश॑नः । \newline
46. स॒ङ्गम॑नो॒ वसू॑नां॒ ॅवसू॑नाꣳ स॒ङ्गम॑नः स॒ङ्गम॑नो॒ वसू॑नाम् । \newline
47. स॒ङ्गम॑न॒ इति॑ सं - गम॑नः । \newline
48. वसू॑ना॒ मितीति॒ वसू॑नां॒ ॅवसू॑ना॒ मिति॑ । \newline
49. इत्या॑हा॒हे तीत्या॑ह । \newline
50. आ॒ह॒ प्र॒जा प्र॒जा ऽऽहा॑ह प्र॒जा । \newline
51. प्र॒जा वै वै प्र॒जा प्र॒जा वै । \newline
52. प्र॒जेति॑ प्र - जा । \newline
53. वै प॒शवः॑ प॒शवो॒ वै वै प॒शवः॑ । \newline
54. प॒शवो॒ वसु॒ वसु॑ प॒शवः॑ प॒शवो॒ वसु॑ । \newline
55. वसु॑ प्र॒जया᳚ प्र॒जया॒ वसु॒ वसु॑ प्र॒जया᳚ । \newline
56. प्र॒ज यै॒वैव प्र॒जया᳚ प्र॒जयै॒व । \newline
57. प्र॒जयेति॑ प्र - जया᳚ । \newline
58. ए॒वैन॑ मेन मे॒वैवैन᳚म् । \newline
59. ए॒न॒म् प॒शुभिः॑ प॒शुभि॑ रेन मेनम् प॒शुभिः॑ । \newline
60. प॒शुभिः॒ सꣳ सम् प॒शुभिः॑ प॒शुभिः॒ सम् । \newline
61. प॒शुभि॒रिति॑ प॒शु - भिः॒ । \newline
62. स म॑र्द्धयन् त्यर्द्धयन्ति॒ सꣳ स म॑र्द्धयन्ति । \newline
63. अ॒र्द्ध॒य॒न्तीत्य॑र्द्धयन्ति । \newline

\textbf{Ghana Paata } \newline

1. द॒धा॒ति॒ परा॑ची॒म् परा॑चीम् दधाति दधाति॒ परा॑ची मे॒वैव परा॑चीम् दधाति दधाति॒ परा॑ची मे॒व । \newline
2. परा॑ची मे॒वैव परा॑ची॒म् परा॑ची मे॒वास्मा॑ दस्मा दे॒व परा॑ची॒म् परा॑ची मे॒वास्मा᳚त् । \newline
3. ए॒वास्मा॑ दस्मा दे॒वैवास्मा॒न् निर्.ऋ॑ति॒म् निर्.ऋ॑ति मस्मा दे॒वैवास्मा॒न् निर्.ऋ॑तिम् । \newline
4. अ॒स्मा॒न् निर्.ऋ॑ति॒म् निर्.ऋ॑ति मस्मा दस्मा॒न् निर्.ऋ॑ति॒म् प्र प्र णिर्.ऋ॑ति मस्मा दस्मा॒न् निर्.ऋ॑ति॒म् प्र । \newline
5. निर्.ऋ॑ति॒म् प्र प्र णिर्.ऋ॑ति॒म् निर्.ऋ॑ति॒म् प्र णु॑दते नुदते॒ प्र णिर्.ऋ॑ति॒म् निर्.ऋ॑ति॒म् प्र णु॑दते । \newline
6. निर्.ऋ॑ति॒मिति॒ निः - ऋ॒ति॒म् । \newline
7. प्र णु॑दते नुदते॒ प्र प्र णु॑द॒ते ऽप्र॑तीक्ष॒ मप्र॑तीक्षम् नुदते॒ प्र प्र णु॑द॒ते ऽप्र॑तीक्षम् । \newline
8. नु॒द॒ते ऽप्र॑तीक्ष॒ मप्र॑तीक्षम् नुदते नुद॒ते ऽप्र॑तीक्ष॒ मा ऽप्र॑तीक्षम् नुदते नुद॒ते ऽप्र॑तीक्ष॒ मा । \newline
9. अप्र॑तीक्ष॒ मा ऽप्र॑तीक्ष॒ मप्र॑तीक्ष॒ मा य॑न्ति य॒न्त्या ऽप्र॑तीक्ष॒ मप्र॑तीक्ष॒ मा य॑न्ति । \newline
10. अप्र॑तीक्ष॒मित्यप्र॑ति - ई॒क्ष॒म् । \newline
11. आ य॑न्ति य॒न्त्या य॑न्ति॒ निर्.ऋ॑त्या॒ निर्.ऋ॑त्या य॒न्त्या य॑न्ति॒ निर्.ऋ॑त्याः । \newline
12. य॒न्ति॒ निर्.ऋ॑त्या॒ निर्.ऋ॑त्या यन्ति यन्ति॒ निर्.ऋ॑त्या अ॒न्तर्.हि॑त्या अ॒न्तर्.हि॑त्यै॒ निर्.ऋ॑त्या यन्ति यन्ति॒ निर्.ऋ॑त्या अ॒न्तर्.हि॑त्यै । \newline
13. निर्.ऋ॑त्या अ॒न्तर्.हि॑त्या अ॒न्तर्.हि॑त्यै॒ निर्.ऋ॑त्या॒ निर्.ऋ॑त्या अ॒न्तर्.हि॑त्यै मार्जयि॒त्वा मा᳚र्जयि॒त्वा ऽन्तर्.हि॑त्यै॒ निर्.ऋ॑त्या॒ निर्.ऋ॑त्या अ॒न्तर्.हि॑त्यै मार्जयि॒त्वा । \newline
14. निर्.ऋ॑त्या॒ इति॒ निः - ऋ॒त्याः॒ । \newline
15. अ॒न्तर्.हि॑त्यै मार्जयि॒त्वा मा᳚र्जयि॒त्वा ऽन्तर्.हि॑त्या अ॒न्तर्.हि॑त्यै मार्जयि॒ त्वोपोप॑ मार्जयि॒त्वा ऽन्तर्.हि॑त्या अ॒न्तर्.हि॑त्यै मार्जयि॒त्वोप॑ । \newline
16. अ॒न्तर्.हि॑त्या॒ इत्य॒न्तः - हि॒त्यै॒ । \newline
17. मा॒र्ज॒यि॒ त्वोपोप॑ मार्जयि॒त्वा मा᳚र्जयि॒त्वोप॑ तिष्ठन्ते तिष्ठन्त॒ उप॑ मार्जयि॒त्वा मा᳚र्जयि॒त्वोप॑ तिष्ठन्ते । \newline
18. उप॑ तिष्ठन्ते तिष्ठन्त॒ उपोप॑ तिष्ठन्ते मेद्ध्य॒त्वाय॑ मेद्ध्य॒त्वाय॑ तिष्ठन्त॒ उपोप॑ तिष्ठन्ते मेद्ध्य॒त्वाय॑ । \newline
19. ति॒ष्ठ॒न्ते॒ मे॒द्ध्य॒त्वाय॑ मेद्ध्य॒त्वाय॑ तिष्ठन्ते तिष्ठन्ते मेद्ध्य॒त्वाय॒ गार्.ह॑पत्य॒म् गार्.ह॑पत्यम् मेद्ध्य॒त्वाय॑ तिष्ठन्ते तिष्ठन्ते मेद्ध्य॒त्वाय॒ गार्.ह॑पत्यम् । \newline
20. मे॒द्ध्य॒त्वाय॒ गार्.ह॑पत्य॒म् गार्.ह॑पत्यम् मेद्ध्य॒त्वाय॑ मेद्ध्य॒त्वाय॒ गार्.ह॑पत्य॒ मुपोप॒ गार्.ह॑पत्यम् मेद्ध्य॒त्वाय॑ मेद्ध्य॒त्वाय॒ गार्.ह॑पत्य॒ मुप॑ । \newline
21. मे॒द्ध्य॒त्वायेति॑ मेद्ध्य - त्वाय॑ । \newline
22. गार्.ह॑पत्य॒ मुपोप॒ गार्.ह॑पत्य॒म् गार्.ह॑पत्य॒ मुप॑ तिष्ठन्ते तिष्ठन्त॒ उप॒ गार्.ह॑पत्य॒म् गार्.ह॑पत्य॒ मुप॑ तिष्ठन्ते । \newline
23. गार्.ह॑पत्य॒मिति॒ गार्.ह॑ - प॒त्य॒म् । \newline
24. उप॑ तिष्ठन्ते तिष्ठन्त॒ उपोप॑ तिष्ठन्ते निर्.ऋतिलो॒के निर्॑.ऋतिलो॒के ति॑ष्ठन्त॒ उपोप॑ तिष्ठन्ते निर्.ऋतिलो॒के । \newline
25. ति॒ष्ठ॒न्ते॒ नि॒र्॒.ऋ॒ति॒लो॒के निर्॑.ऋतिलो॒के ति॑ष्ठन्ते तिष्ठन्ते निर्.ऋतिलो॒क ए॒वैव निर्॑.ऋतिलो॒के ति॑ष्ठन्ते तिष्ठन्ते निर्.ऋतिलो॒क ए॒व । \newline
26. नि॒र्॒.ऋ॒ति॒लो॒क ए॒वैव निर्॑.ऋतिलो॒के निर्॑.ऋतिलो॒क ए॒व च॑रि॒त्वा च॑रि॒त्वैव निर्॑.ऋतिलो॒के निर्॑.ऋतिलो॒क ए॒व च॑रि॒त्वा । \newline
27. नि॒र्॒.ऋ॒ति॒लो॒क इति॑ निर्.ऋति - लो॒के । \newline
28. ए॒व च॑रि॒त्वा च॑रि॒ त्वैवैव च॑रि॒त्वा पू॒ताः पू॒ता श्च॑रि॒त्वैवैव च॑रि॒त्वा पू॒ताः । \newline
29. च॒रि॒त्वा पू॒ताः पू॒ता श्च॑रि॒त्वा च॑रि॒त्वा पू॒ता दे॑वलो॒कम् दे॑वलो॒कम् पू॒ता श्च॑रि॒त्वा च॑रि॒त्वा पू॒ता दे॑वलो॒कम् । \newline
30. पू॒ता दे॑वलो॒कम् दे॑वलो॒कम् पू॒ताः पू॒ता दे॑वलो॒क मु॒पाव॑र्तन्त उ॒पाव॑र्तन्ते देवलो॒कम् पू॒ताः पू॒ता दे॑वलो॒क मु॒पाव॑र्तन्ते । \newline
31. दे॒व॒लो॒क मु॒पाव॑र्तन्त उ॒पाव॑र्तन्ते देवलो॒कम् दे॑वलो॒क मु॒पाव॑र्तन्त॒ एक॒ यैक॑यो॒पाव॑र्तन्ते देवलो॒कम् दे॑वलो॒क मु॒पाव॑र्तन्त॒ एक॑या । \newline
32. दे॒व॒लो॒कमिति॑ देव - लो॒कम् । \newline
33. उ॒पाव॑र्तन्त॒ एक॒ यैक॑यो॒पाव॑र्तन्त उ॒पाव॑र्तन्त॒ एक॒ योपो पैक॑ यो॒पाव॑र्तन्त उ॒पाव॑र्तन्त॒ एक॒योप॑ । \newline
34. उ॒पाव॑र्तन्त॒ इत्यु॑प - आव॑र्तन्ते । \newline
35. एक॒ योपो पैक॒ यैक॒योप॑ तिष्ठन्ते तिष्ठन्त॒ उपैक॒ यैक॒योप॑ तिष्ठन्ते । \newline
36. उप॑ तिष्ठन्ते तिष्ठन्त॒ उपोप॑ तिष्ठन्त एक॒ धैक॒धा ति॑ष्ठन्त॒ उपोप॑ तिष्ठन्त एक॒धा । \newline
37. ति॒ष्ठ॒न्त॒ ए॒क॒ धैक॒धा ति॑ष्ठन्ते तिष्ठन्त एक॒ धैवै वैक॒धा ति॑ष्ठन्ते तिष्ठन्त एक॒धैव । \newline
38. ए॒क॒ धैवै वैक॒ धैक॒ धैव यज॑माने॒ यज॑मान ए॒वैक॒ धैक॒ धैव यज॑माने । \newline
39. ए॒क॒धेत्ये॑क - धा । \newline
40. ए॒व यज॑माने॒ यज॑मान ए॒वैव यज॑माने वी॒र्यं॑ ॅवी॒र्यं॑ ॅयज॑मान ए॒वैव यज॑माने वी॒र्य᳚म् । \newline
41. यज॑माने वी॒र्यं॑ ॅवी॒र्यं॑ ॅयज॑माने॒ यज॑माने वी॒र्य॑म् दधति दधति वी॒र्यं॑ ॅयज॑माने॒ यज॑माने वी॒र्य॑म् दधति । \newline
42. वी॒र्य॑म् दधति दधति वी॒र्यं॑ ॅवी॒र्य॑म् दधति नि॒वेश॑नो नि॒वेश॑नो दधति वी॒र्यं॑ ॅवी॒र्य॑म् दधति नि॒वेश॑नः । \newline
43. द॒ध॒ति॒ नि॒वेश॑नो नि॒वेश॑नो दधति दधति नि॒वेश॑नः स॒ङ्गम॑नः स॒ङ्गम॑नो नि॒वेश॑नो दधति दधति नि॒वेश॑नः स॒ङ्गम॑नः । \newline
44. नि॒वेश॑नः स॒ङ्गम॑नः स॒ङ्गम॑नो नि॒वेश॑नो नि॒वेश॑नः स॒ङ्गम॑नो॒ वसू॑नां॒ ॅवसू॑नाꣳ स॒ङ्गम॑नो नि॒वेश॑नो नि॒वेश॑नः स॒ङ्गम॑नो॒ वसू॑नाम् । \newline
45. नि॒वेश॑न॒ इति॑ नि - वेश॑नः । \newline
46. स॒ङ्गम॑नो॒ वसू॑नां॒ ॅवसू॑नाꣳ स॒ङ्गम॑नः स॒ङ्गम॑नो॒ वसू॑ना॒ मितीति॒ वसू॑नाꣳ स॒ङ्गम॑नः स॒ङ्गम॑नो॒ वसू॑ना॒ मिति॑ । \newline
47. स॒ङ्गम॑न॒ इति॑ सं - गम॑नः । \newline
48. वसू॑ना॒ मितीति॒ वसू॑नां॒ ॅवसू॑ना॒ मित्या॑हा॒हेति॒ वसू॑नां॒ ॅवसू॑ना॒ मित्या॑ह । \newline
49. इत्या॑हा॒हे तीत्या॑ह प्र॒जा प्र॒जा ऽऽहे तीत्या॑ह प्र॒जा । \newline
50. आ॒ह॒ प्र॒जा प्र॒जा ऽऽहा॑ह प्र॒जा वै वै प्र॒जा ऽऽहा॑ह प्र॒जा वै । \newline
51. प्र॒जा वै वै प्र॒जा प्र॒जा वै प॒शवः॑ प॒शवो॒ वै प्र॒जा प्र॒जा वै प॒शवः॑ । \newline
52. प्र॒जेति॑ प्र - जा । \newline
53. वै प॒शवः॑ प॒शवो॒ वै वै प॒शवो॒ वसु॒ वसु॑ प॒शवो॒ वै वै प॒शवो॒ वसु॑ । \newline
54. प॒शवो॒ वसु॒ वसु॑ प॒शवः॑ प॒शवो॒ वसु॑ प्र॒जया᳚ प्र॒जया॒ वसु॑ प॒शवः॑ प॒शवो॒ वसु॑ प्र॒जया᳚ । \newline
55. वसु॑ प्र॒जया᳚ प्र॒जया॒ वसु॒ वसु॑ प्र॒ज यै॒वैव प्र॒जया॒ वसु॒ वसु॑ प्र॒जयै॒व । \newline
56. प्र॒जयै॒वैव प्र॒जया᳚ प्र॒ज यै॒वैन॑ मेन मे॒व प्र॒जया᳚ प्र॒ज यै॒वैन᳚म् । \newline
57. प्र॒जयेति॑ प्र - जया᳚ । \newline
58. ए॒वैन॑ मेन मे॒वैवैन॑म् प॒शुभिः॑ प॒शुभि॑ रेन मे॒वैवैन॑म् प॒शुभिः॑ । \newline
59. ए॒न॒म् प॒शुभिः॑ प॒शुभि॑ रेन मेनम् प॒शुभिः॒ सꣳ सम् प॒शुभि॑ रेन मेनम् प॒शुभिः॒ सम् । \newline
60. प॒शुभिः॒ सꣳ सम् प॒शुभिः॑ प॒शुभिः॒ स म॑र्द्धयन् त्यर्द्धयन्ति॒ सम् प॒शुभिः॑ प॒शुभिः॒ स म॑र्द्धयन्ति । \newline
61. प॒शुभि॒रिति॑ प॒शु - भिः॒ । \newline
62. स म॑र्द्धयन् त्यर्द्धयन्ति॒ सꣳ स म॑र्द्धयन्ति । \newline
63. अ॒र्द्ध॒य॒न्तीत्य॑र्द्धयन्ति । \newline
\pagebreak
\markright{ TS 5.2.5.1  \hfill https://www.vedavms.in \hfill}

\section{ TS 5.2.5.1 }

\textbf{TS 5.2.5.1 } \newline
\textbf{Samhita Paata} \newline

पु॒रु॒ष॒मा॒त्रेण॒ वि मि॑मीते य॒ज्ञेन॒ वै पुरु॑षः॒ संमि॑तो यज्ञ्प॒रुषै॒वैनं॒ ॅविमि॑मीते॒ यावा॒न् पुरु॑ष ऊ॒र्द्ध्वबा॑हु॒स्तावा᳚न् भवत्ये॒ताव॒द्वै पुरु॑षे वी॒र्यं॑ ॅवी॒र्ये॑णै॒वैनं॒ ॅवि मि॑मीते प॒क्षी भ॑वति॒ न ह्य॑प॒क्षः पति॑तु॒-मर्.ह॑त्यर॒त्निना॑ प॒क्षौ द्राघी॑याꣳसौ भवत॒स्तस्मा᳚त् प॒क्षप्र॑वयाꣳसि॒ वयाꣳ॑सि व्याममा॒त्रौ प॒क्षौ च॒ पुच्छं॑ च भवत्ये॒ताव॒द्वै पुरु॑षे वी॒र्यं॑ - [  ] \newline

\textbf{Pada Paata} \newline

पु॒रु॒ष॒मा॒त्रेणेति॑ पुरुष - मा॒त्रेण॑ । वीति॑ । मि॒मी॒ते॒ । य॒ज्ञेन॑ । वै । पुरु॑षः । सम्मि॑त॒ इति॒ सं - मि॒तः॒ । य॒ज्ञ्॒प॒रुषेति॑ यज्ञ्-प॒रुषा᳚ । ए॒व । ए॒न॒म् । वीति॑ । मि॒मी॒ते॒ । यावान्॑ । पुरु॑षः । ऊ॒द्‌र्ध्वबा॑हु॒रित्यु॒द्‌र्ध्व-बा॒हुः॒ । तावान्॑ । भ॒व॒ति॒ । ए॒ताव॑त् । वै । पुरु॑षे । वी॒र्य᳚म् । वी॒र्ये॑ण । ए॒व । ए॒न॒म् । वीति॑ । मि॒मी॒ते॒ । प॒क्षी । भ॒व॒ति॒ । न । हि । अ॒प॒क्षः । पति॑तुम् । अर्.ह॑ति । अ॒र॒त्निना᳚ । प॒क्षौ । द्राघी॑याꣳसौ । भ॒व॒तः॒ । तस्मा᳚त् । प॒क्षप्र॑वयाꣳ॒॒सीति॑ प॒क्ष - प्र॒व॒याꣳ॒॒सि॒ । वयाꣳ॑सि । व्या॒म॒मा॒त्राविति॑ व्याम - मा॒त्रौ । प॒क्षौ । च॒ । पुच्छ᳚म् । च॒ । भ॒व॒ति॒ । ए॒ताव॑त् । वै । पुरु॑षे । वी॒र्य᳚म् ।  \newline


\textbf{Krama Paata} \newline

पु॒रु॒ष॒मा॒त्रेण॒ वि । पु॒रु॒ष॒मा॒त्रेणेति॑ पुरुष - मा॒त्रेण॑ । वि मि॑मीते । मि॒मी॒ते॒ य॒ज्ञेन॑ । य॒ज्ञेन॒ वै । वै पुरु॑षः । पुरु॑षः॒ सम्मि॑तः । सम्मि॑तो यज्ञ्प॒रुषा᳚ । सम्मि॑त॒ इति॒ सम् - मि॒तः॒ । य॒ज्ञ्॒प॒रुषै॒व । य॒ज्ञ्॒प॒रुषेति॑ यज्ञ् - प॒रुषा᳚ । ए॒वैन᳚म् । ए॒न॒म् ॅवि । वि मि॑मीते । मि॒मी॒ते॒ यावान्॑ । यावा॒न् पुरु॑षः । पुरु॑ष ऊ॒र्द्ध्वबा॑हुः । ऊ॒र्द्ध्वबा॑हु॒स्तावान्॑ । ऊ॒र्द्ध्वबा॑हु॒रित्यू॒र्द्ध्व - बा॒हुः॒ । तावा᳚न् भवति । भ॒व॒त्ये॒ताव॑त् । ए॒ताव॒द् वै । वै पुरु॑षे । पुरु॑षे वी॒र्य᳚म् । वी॒र्य॑म् ॅवी॒र्ये॑ण । वी॒र्ये॑णै॒व । ए॒वैन᳚म् । ए॒न॒म् ॅवि । वि मि॑मीते । मि॒मी॒ते॒ प॒क्षी । प॒क्षी भ॑वति । भ॒व॒ति॒ न । न हि । ह्य॑प॒क्षः । अ॒प॒क्षः पति॑तुम् । पति॑तु॒मर्.ह॑ति । अर्.ह॑त्यर॒त्निना᳚ । अ॒र॒त्निना॑ प॒क्षौ । प॒क्षौ द्राघी॑याꣳसौ । द्राघी॑याꣳसौ भवतः । भ॒व॒त॒स्तस्मा᳚त् । तस्मा᳚त् प॒क्षप्र॑वयाꣳसि । प॒क्षप्र॑वयाꣳसि॒ वयाꣳ॑सि । प॒क्षप्र॑वयाꣳ॒॒सीति॑ प॒क्ष - प्र॒व॒याꣳ॒॒सि॒ । वयाꣳ॑सि व्याममा॒त्रौ । व्या॒म॒मा॒त्रौ प॒क्षौ । व्या॒म॒मा॒त्राविति॑ व्याम - मा॒त्रौ । प॒क्षौ च॑ । च॒ पुच्छ᳚म् । पुच्छ॑म् च । च॒ भ॒व॒ति॒ । भ॒व॒त्ये॒ताव॑त् । ए॒ताव॒द् वै । वै पुरु॑षे । पुरु॑षे वी॒र्य᳚म् । वी॒र्य॑म् ॅवी॒र्य॑सम्मितः \newline

\textbf{Jatai Paata} \newline

1. पु॒रु॒ष॒मा॒त्रेण॒ वि वि पु॑रुषमा॒त्रेण॑ पुरुषमा॒त्रेण॒ वि । \newline
2. पु॒रु॒ष॒मा॒त्रेणेति॑ पुरुष - मा॒त्रेण॑ । \newline
3. वि मि॑मीते मिमीते॒ वि वि मि॑मीते । \newline
4. मि॒मी॒ते॒ य॒ज्ञेन॑ य॒ज्ञेन॑ मिमीते मिमीते य॒ज्ञेन॑ । \newline
5. य॒ज्ञेन॒ वै वै य॒ज्ञेन॑ य॒ज्ञेन॒ वै । \newline
6. वै पुरु॑षः॒ पुरु॑षो॒ वै वै पुरु॑षः । \newline
7. पुरु॑षः॒ सम्मि॑तः॒ सम्मि॑तः॒ पुरु॑षः॒ पुरु॑षः॒ सम्मि॑तः । \newline
8. सम्मि॑तो यज्ञ्प॒रुषा॑ यज्ञ्प॒रुषा॒ सम्मि॑तः॒ सम्मि॑तो यज्ञ्प॒रुषा᳚ । \newline
9. सम्मि॑त॒ इति॒ सं - मि॒तः॒ । \newline
10. य॒ज्ञ्॒प॒रु षै॒वैव य॑ज्ञ्प॒रुषा॑ यज्ञ्प॒रुषै॒व । \newline
11. य॒ज्ञ्॒प॒रुषेति॑ यज्ञ् - प॒रुषा᳚ । \newline
12. ए॒वैन॑ मेन मे॒वैवैन᳚म् । \newline
13. ए॒नं॒ ॅवि व्ये॑न मेनं॒ ॅवि । \newline
14. वि मि॑मीते मिमीते॒ वि वि मि॑मीते । \newline
15. मि॒मी॒ते॒ यावा॒न्॒. यावा᳚न् मिमीते मिमीते॒ यावान्॑ । \newline
16. यावा॒न् पुरु॑षः॒ पुरु॑षो॒ यावा॒न्॒. यावा॒न् पुरु॑षः । \newline
17. पुरु॑ष ऊ॒र्द्ध्वबा॑हु रू॒र्द्ध्वबा॑हुः॒ पुरु॑षः॒ पुरु॑ष ऊ॒र्द्ध्वबा॑हुः । \newline
18. ऊ॒र्द्ध्वबा॑हु॒ स्तावा॒न् तावा॑ नू॒र्द्ध्वबा॑हु रू॒र्द्ध्वबा॑हु॒ स्तावान्॑ । \newline
19. ऊ॒र्द्ध्वबा॑हु॒रित्यु॒र्द्ध्व - बा॒हुः॒ । \newline
20. तावा᳚न् भवति भवति॒ तावा॒न् तावा᳚न् भवति । \newline
21. भ॒व॒ त्ये॒ताव॑ दे॒ताव॑द् भवति भव त्ये॒ताव॑त् । \newline
22. ए॒ताव॒द् वै वा ए॒ताव॑ दे॒ताव॒द् वै । \newline
23. वै पुरु॑षे॒ पुरु॑षे॒ वै वै पुरु॑षे । \newline
24. पुरु॑षे वी॒र्यं॑ ॅवी॒र्य॑म् पुरु॑षे॒ पुरु॑षे वी॒र्य᳚म् । \newline
25. वी॒र्यं॑ ॅवी॒र्ये॑ण वी॒र्ये॑ण वी॒र्यं॑ ॅवी॒र्यं॑ ॅवी॒र्ये॑ण । \newline
26. वी॒र्ये॑ णै॒वैव वी॒र्ये॑ण वी॒र्ये॑णै॒व । \newline
27. ए॒वैन॑ मेन मे॒वैवैन᳚म् । \newline
28. ए॒नं॒ ॅवि व्ये॑न मेनं॒ ॅवि । \newline
29. वि मि॑मीते मिमीते॒ वि वि मि॑मीते । \newline
30. मि॒मी॒ते॒ प॒क्षी प॒क्षी मि॑मीते मिमीते प॒क्षी । \newline
31. प॒क्षी भ॑वति भवति प॒क्षी प॒क्षी भ॑वति । \newline
32. भ॒व॒ति॒ न न भ॑वति भवति॒ न । \newline
33. न हि हि न न हि । \newline
34. ह्य॑प॒क्षो॑ ऽप॒क्षो हि ह्य॑प॒क्षः । \newline
35. अ॒प॒क्षः पति॑तु॒म् पति॑तु मप॒क्षो॑ ऽप॒क्षः पति॑तुम् । \newline
36. पति॑तु॒ मर्.ह॒ त्यर्.ह॑ति॒ पति॑तु॒म् पति॑तु॒ मर्.ह॑ति । \newline
37. अर्.ह॑ त्यर॒त्निना॑ ऽर॒त्निना ऽर्.ह॒ त्यर्.ह॑ त्यर॒त्निना᳚ । \newline
38. अ॒र॒त्निना॑ प॒क्षौ प॒क्षा व॑र॒त्निना॑ ऽर॒त्निना॑ प॒क्षौ । \newline
39. प॒क्षौ द्राघी॑याꣳसौ॒ द्राघी॑याꣳसौ प॒क्षौ प॒क्षौ द्राघी॑याꣳसौ । \newline
40. द्राघी॑याꣳसौ भवतो भवतो॒ द्राघी॑याꣳसौ॒ द्राघी॑याꣳसौ भवतः । \newline
41. भ॒व॒त॒ स्तस्मा॒त् तस्मा᳚द् भवतो भवत॒ स्तस्मा᳚त् । \newline
42. तस्मा᳚त् प॒क्षप्र॑वयाꣳसि प॒क्षप्र॑वयाꣳसि॒ तस्मा॒त् तस्मा᳚त् प॒क्षप्र॑वयाꣳसि । \newline
43. प॒क्षप्र॑वयाꣳसि॒ वयाꣳ॑सि॒ वयाꣳ॑सि प॒क्षप्र॑वयाꣳसि प॒क्षप्र॑वयाꣳसि॒ वयाꣳ॑सि । \newline
44. प॒क्षप्र॑वयाꣳ॒॒सीति॑ प॒क्ष - प्र॒व॒याꣳ॒॒सि॒ । \newline
45. वयाꣳ॑सि व्याममा॒त्रौ व्या॑ममा॒त्रौ वयाꣳ॑सि॒ वयाꣳ॑सि व्याममा॒त्रौ । \newline
46. व्या॒म॒मा॒त्रौ प॒क्षौ प॒क्षौ व्या॑ममा॒त्रौ व्या॑ममा॒त्रौ प॒क्षौ । \newline
47. व्या॒म॒मा॒त्राविति॑ व्याम - मा॒त्रौ । \newline
48. प॒क्षौ च॑ च प॒क्षौ प॒क्षौ च॑ । \newline
49. च॒ पुच्छ॒म् पुच्छ॑म् च च॒ पुच्छ᳚म् । \newline
50. पुच्छ॑म् च च॒ पुच्छ॒म् पुच्छ॑म् च । \newline
51. च॒ भ॒व॒ति॒ भ॒व॒ति॒ च॒ च॒ भ॒व॒ति॒ । \newline
52. भ॒व॒ त्ये॒ताव॑ दे॒ताव॑द् भवति भव त्ये॒ताव॑त् । \newline
53. ए॒ताव॒द् वै वा ए॒ताव॑ दे॒ताव॒द् वै । \newline
54. वै पुरु॑षे॒ पुरु॑षे॒ वै वै पुरु॑षे । \newline
55. पुरु॑षे वी॒र्यं॑ ॅवी॒र्य॑म् पुरु॑षे॒ पुरु॑षे वी॒र्य᳚म् । \newline
56. वी॒र्यं॑ ॅवी॒र्य॑सम्मितो वी॒र्य॑सम्मितो वी॒र्यं॑ ॅवी॒र्यं॑ ॅवी॒र्य॑सम्मितः । \newline

\textbf{Ghana Paata } \newline

1. पु॒रु॒ष॒मा॒त्रेण॒ वि वि पु॑रुषमा॒त्रेण॑ पुरुषमा॒त्रेण॒ वि मि॑मीते मिमीते॒ वि पु॑रुषमा॒त्रेण॑ पुरुषमा॒त्रेण॒ वि मि॑मीते । \newline
2. पु॒रु॒ष॒मा॒त्रेणेति॑ पुरुष - मा॒त्रेण॑ । \newline
3. वि मि॑मीते मिमीते॒ वि वि मि॑मीते य॒ज्ञेन॑ य॒ज्ञेन॑ मिमीते॒ वि वि मि॑मीते य॒ज्ञेन॑ । \newline
4. मि॒मी॒ते॒ य॒ज्ञेन॑ य॒ज्ञेन॑ मिमीते मिमीते य॒ज्ञेन॒ वै वै य॒ज्ञेन॑ मिमीते मिमीते य॒ज्ञेन॒ वै । \newline
5. य॒ज्ञेन॒ वै वै य॒ज्ञेन॑ य॒ज्ञेन॒ वै पुरु॑षः॒ पुरु॑षो॒ वै य॒ज्ञेन॑ य॒ज्ञेन॒ वै पुरु॑षः । \newline
6. वै पुरु॑षः॒ पुरु॑षो॒ वै वै पुरु॑षः॒ सम्मि॑तः॒ सम्मि॑तः॒ पुरु॑षो॒ वै वै पुरु॑षः॒ सम्मि॑तः । \newline
7. पुरु॑षः॒ सम्मि॑तः॒ सम्मि॑तः॒ पुरु॑षः॒ पुरु॑षः॒ सम्मि॑तो यज्ञ्प॒रुषा॑ यज्ञ्प॒रुषा॒ सम्मि॑तः॒ पुरु॑षः॒ पुरु॑षः॒ सम्मि॑तो यज्ञ्प॒रुषा᳚ । \newline
8. सम्मि॑तो यज्ञ्प॒रुषा॑ यज्ञ्प॒रुषा॒ सम्मि॑तः॒ सम्मि॑तो यज्ञ्प॒रु षै॒वैव य॑ज्ञ्प॒रुषा॒ सम्मि॑तः॒ सम्मि॑तो यज्ञ्प॒रुषै॒व । \newline
9. सम्मि॑त॒ इति॒ सं - मि॒तः॒ । \newline
10. य॒ज्ञ्॒प॒रुषै॒वैव य॑ज्ञ्प॒रुषा॑ यज्ञ्प॒रुषै॒वैन॑ मेन मे॒व य॑ज्ञ्प॒रुषा॑ यज्ञ्प॒रु षै॒वैन᳚म् । \newline
11. य॒ज्ञ्॒प॒रुषेति॑ यज्ञ् - प॒रुषा᳚ । \newline
12. ए॒वैन॑ मेन मे॒वैवैनं॒ ॅवि व्ये॑न मे॒वैवैनं॒ ॅवि । \newline
13. ए॒नं॒ ॅवि व्ये॑न मेनं॒ ॅवि मि॑मीते मिमीते॒ व्ये॑न मेनं॒ ॅवि मि॑मीते । \newline
14. वि मि॑मीते मिमीते॒ वि वि मि॑मीते॒ यावा॒न्॒. यावा᳚न् मिमीते॒ वि वि मि॑मीते॒ यावान्॑ । \newline
15. मि॒मी॒ते॒ यावा॒न्॒. यावा᳚न् मिमीते मिमीते॒ यावा॒न् पुरु॑षः॒ पुरु॑षो॒ यावा᳚न् मिमीते मिमीते॒ यावा॒न् पुरु॑षः । \newline
16. यावा॒न् पुरु॑षः॒ पुरु॑षो॒ यावा॒न्॒. यावा॒न् पुरु॑ष ऊ॒र्द्ध्वबा॑हु रू॒र्द्ध्वबा॑हुः॒ पुरु॑षो॒ यावा॒न्॒. यावा॒न् पुरु॑ष ऊ॒र्द्ध्वबा॑हुः । \newline
17. पुरु॑ष ऊ॒र्द्ध्वबा॑हु रू॒र्द्ध्वबा॑हुः॒ पुरु॑षः॒ पुरु॑ष ऊ॒र्द्ध्वबा॑हु॒ स्तावा॒न् तावा॑ नू॒र्द्ध्वबा॑हुः॒ पुरु॑षः॒ पुरु॑ष ऊ॒र्द्ध्वबा॑हु॒ स्तावान्॑ । \newline
18. ऊ॒र्द्ध्वबा॑हु॒ स्तावा॒न् तावा॑ नू॒र्द्ध्वबा॑हु रू॒र्द्ध्वबा॑हु॒ स्तावा᳚न् भवति भवति॒ तावा॑ नू॒र्द्ध्वबा॑हु रू॒र्द्ध्वबा॑हु॒ स्तावा᳚न् भवति । \newline
19. ऊ॒र्द्ध्वबा॑हु॒रित्यु॒र्द्ध्व - बा॒हुः॒ । \newline
20. तावा᳚न् भवति भवति॒ तावा॒न् तावा᳚न् भव त्ये॒ताव॑ दे॒ताव॑द् भवति॒ तावा॒न् तावा᳚न् भव त्ये॒ताव॑त् । \newline
21. भ॒व॒ त्ये॒ताव॑ दे॒ताव॑द् भवति भव त्ये॒ताव॒द् वै वा ए॒ताव॑द् भवति भव त्ये॒ताव॒द् वै । \newline
22. ए॒ताव॒द् वै वा ए॒ताव॑ दे॒ताव॒द् वै पुरु॑षे॒ पुरु॑षे॒ वा ए॒ताव॑ दे॒ताव॒द् वै पुरु॑षे । \newline
23. वै पुरु॑षे॒ पुरु॑षे॒ वै वै पुरु॑षे वी॒र्यं॑ ॅवी॒र्य॑म् पुरु॑षे॒ वै वै पुरु॑षे वी॒र्य᳚म् । \newline
24. पुरु॑षे वी॒र्यं॑ ॅवी॒र्य॑म् पुरु॑षे॒ पुरु॑षे वी॒र्यं॑ ॅवी॒र्ये॑ण वी॒र्ये॑ण वी॒र्य॑म् पुरु॑षे॒ पुरु॑षे वी॒र्यं॑ ॅवी॒र्ये॑ण । \newline
25. वी॒र्यं॑ ॅवी॒र्ये॑ण वी॒र्ये॑ण वी॒र्यं॑ ॅवी॒र्यं॑ ॅवी॒र्ये॑ णै॒वैव वी॒र्ये॑ण वी॒र्यं॑ ॅवी॒र्यं॑ ॅवी॒र्ये॑णै॒व । \newline
26. वी॒र्ये॑ णै॒वैव वी॒र्ये॑ण वी॒र्ये॑ णै॒वैन॑ मेन मे॒व वी॒र्ये॑ण वी॒र्ये॑ णै॒वैन᳚म् । \newline
27. ए॒वैन॑ मेन मे॒वैवैनं॒ ॅवि व्ये॑न मे॒वैवैनं॒ ॅवि । \newline
28. ए॒नं॒ ॅवि व्ये॑न मेनं॒ ॅवि मि॑मीते मिमीते॒ व्ये॑न मेनं॒ ॅवि मि॑मीते । \newline
29. वि मि॑मीते मिमीते॒ वि वि मि॑मीते प॒क्षी प॒क्षी मि॑मीते॒ वि वि मि॑मीते प॒क्षी । \newline
30. मि॒मी॒ते॒ प॒क्षी प॒क्षी मि॑मीते मिमीते प॒क्षी भ॑वति भवति प॒क्षी मि॑मीते मिमीते प॒क्षी भ॑वति । \newline
31. प॒क्षी भ॑वति भवति प॒क्षी प॒क्षी भ॑वति॒ न न भ॑वति प॒क्षी प॒क्षी भ॑वति॒ न । \newline
32. भ॒व॒ति॒ न न भ॑वति भवति॒ न हि हि न भ॑वति भवति॒ न हि । \newline
33. न हि हि न न ह्य॑प॒क्षो॑ ऽप॒क्षो हि न न ह्य॑प॒क्षः । \newline
34. ह्य॑प॒क्षो॑ ऽप॒क्षो हि ह्य॑प॒क्षः पति॑तु॒म् पति॑तु मप॒क्षो हि ह्य॑प॒क्षः पति॑तुम् । \newline
35. अ॒प॒क्षः पति॑तु॒म् पति॑तु मप॒क्षो॑ ऽप॒क्षः पति॑तु॒ मर्.ह॒ त्यर्.ह॑ति॒ पति॑तु मप॒क्षो॑ ऽप॒क्षः पति॑तु॒ मर्.ह॑ति । \newline
36. पति॑तु॒ मर्.ह॒ त्यर्.ह॑ति॒ पति॑तु॒म् पति॑तु॒ मर्.ह॑ त्यर॒त्निना॑ ऽर॒त्निना ऽर्.ह॑ति॒ पति॑तु॒म् पति॑तु॒ मर्.ह॑ त्यर॒त्निना᳚ । \newline
37. अर्.ह॑ त्यर॒त्निना॑ ऽर॒त्निना ऽर्.ह॒ त्यर्.ह॑ त्यर॒त्निना॑ प॒क्षौ प॒क्षा व॑र॒त्निना ऽर्.ह॒ त्यर्.ह॑ त्यर॒त्निना॑ प॒क्षौ । \newline
38. अ॒र॒त्निना॑ प॒क्षौ प॒क्षा व॑र॒त्निना॑ ऽर॒त्निना॑ प॒क्षौ द्राघी॑याꣳसौ॒ द्राघी॑याꣳसौ प॒क्षा व॑र॒त्निना॑ ऽर॒त्निना॑ प॒क्षौ द्राघी॑याꣳसौ । \newline
39. प॒क्षौ द्राघी॑याꣳसौ॒ द्राघी॑याꣳसौ प॒क्षौ प॒क्षौ द्राघी॑याꣳसौ भवतो भवतो॒ द्राघी॑याꣳसौ प॒क्षौ प॒क्षौ द्राघी॑याꣳसौ भवतः । \newline
40. द्राघी॑याꣳसौ भवतो भवतो॒ द्राघी॑याꣳसौ॒ द्राघी॑याꣳसौ भवत॒ स्तस्मा॒त् तस्मा᳚द् भवतो॒ द्राघी॑याꣳसौ॒ द्राघी॑याꣳसौ भवत॒ स्तस्मा᳚त् । \newline
41. भ॒व॒त॒ स्तस्मा॒त् तस्मा᳚द् भवतो भवत॒ स्तस्मा᳚त् प॒क्षप्र॑वयाꣳसि प॒क्षप्र॑वयाꣳसि॒ तस्मा᳚द् भवतो भवत॒ स्तस्मा᳚त् प॒क्षप्र॑वयाꣳसि । \newline
42. तस्मा᳚त् प॒क्षप्र॑वयाꣳसि प॒क्षप्र॑वयाꣳसि॒ तस्मा॒त् तस्मा᳚त् प॒क्षप्र॑वयाꣳसि॒ वयाꣳ॑सि॒ वयाꣳ॑सि प॒क्षप्र॑वयाꣳसि॒ तस्मा॒त् तस्मा᳚त् प॒क्षप्र॑वयाꣳसि॒ वयाꣳ॑सि । \newline
43. प॒क्षप्र॑वयाꣳसि॒ वयाꣳ॑सि॒ वयाꣳ॑सि प॒क्षप्र॑वयाꣳसि प॒क्षप्र॑वयाꣳसि॒ वयाꣳ॑सि व्याममा॒त्रौ व्या॑ममा॒त्रौ वयाꣳ॑सि प॒क्षप्र॑वयाꣳसि प॒क्षप्र॑वयाꣳसि॒ वयाꣳ॑सि व्याममा॒त्रौ । \newline
44. प॒क्षप्र॑वयाꣳ॒॒सीति॑ प॒क्ष - प्र॒व॒याꣳ॒॒सि॒ । \newline
45. वयाꣳ॑सि व्याममा॒त्रौ व्या॑ममा॒त्रौ वयाꣳ॑सि॒ वयाꣳ॑सि व्याममा॒त्रौ प॒क्षौ प॒क्षौ व्या॑ममा॒त्रौ वयाꣳ॑सि॒ वयाꣳ॑सि व्याममा॒त्रौ प॒क्षौ । \newline
46. व्या॒म॒मा॒त्रौ प॒क्षौ प॒क्षौ व्या॑ममा॒त्रौ व्या॑ममा॒त्रौ प॒क्षौ च॑ च प॒क्षौ व्या॑ममा॒त्रौ व्या॑ममा॒त्रौ प॒क्षौ च॑ । \newline
47. व्या॒म॒मा॒त्राविति॑ व्याम - मा॒त्रौ । \newline
48. प॒क्षौ च॑ च प॒क्षौ प॒क्षौ च॒ पुच्छ॒म् पुच्छ॑म् च प॒क्षौ प॒क्षौ च॒ पुच्छ᳚म् । \newline
49. च॒ पुच्छ॒म् पुच्छ॑म् च च॒ पुच्छ॑म् च च॒ पुच्छ॑म् च च॒ पुच्छ॑म् च । \newline
50. पुच्छ॑म् च च॒ पुच्छ॒म् पुच्छ॑म् च भवति भवति च॒ पुच्छ॒म् पुच्छ॑म् च भवति । \newline
51. च॒ भ॒व॒ति॒ भ॒व॒ति॒ च॒ च॒ भ॒व॒ त्ये॒ताव॑ दे॒ताव॑द् भवति च च भव त्ये॒ताव॑त् । \newline
52. भ॒व॒ त्ये॒ताव॑ दे॒ताव॑द् भवति भव त्ये॒ताव॒द् वै वा ए॒ताव॑द् भवति भव त्ये॒ताव॒द् वै । \newline
53. ए॒ताव॒द् वै वा ए॒ताव॑ दे॒ताव॒द् वै पुरु॑षे॒ पुरु॑षे॒ वा ए॒ताव॑ दे॒ताव॒द् वै पुरु॑षे । \newline
54. वै पुरु॑षे॒ पुरु॑षे॒ वै वै पुरु॑षे वी॒र्यं॑ ॅवी॒र्य॑म् पुरु॑षे॒ वै वै पुरु॑षे वी॒र्य᳚म् । \newline
55. पुरु॑षे वी॒र्यं॑ ॅवी॒र्य॑म् पुरु॑षे॒ पुरु॑षे वी॒र्यं॑ ॅवी॒र्य॑सम्मितो वी॒र्य॑सम्मितो वी॒र्य॑म् पुरु॑षे॒ पुरु॑षे वी॒र्यं॑ ॅवी॒र्य॑सम्मितः । \newline
56. वी॒र्यं॑ ॅवी॒र्य॑सम्मितो वी॒र्य॑सम्मितो वी॒र्यं॑ ॅवी॒र्यं॑ ॅवी॒र्य॑सम्मितो॒ वेणु॑ना॒ वेणु॑ना वी॒र्य॑सम्मितो वी॒र्यं॑ ॅवी॒र्यं॑ ॅवी॒र्य॑सम्मितो॒ वेणु॑ना । \newline
\pagebreak
\markright{ TS 5.2.5.2  \hfill https://www.vedavms.in \hfill}

\section{ TS 5.2.5.2 }

\textbf{TS 5.2.5.2 } \newline
\textbf{Samhita Paata} \newline

ॅवी॒र्य॑संमितो॒ वेणु॑ना॒ वि मि॑मीत आग्ने॒यो वै वेणुः॑ सयोनि॒त्वाय॒ यजु॑षा युनक्ति॒ यजु॑षा कृषति॒ व्यावृ॑त्त्यै षड्ग॒वेन॑ कृषति॒ षड् वा ऋ॒तव॑ ऋ॒तुभि॑रे॒वैनं॑ कृषति॒ यद् द्वा॑दशग॒वेन॑ संॅवथ्स॒रेणै॒वे यं ॅवा अ॒ग्ने-र॑तिदा॒हाद॑बिभे॒थ् सैतद् द्वि॑गु॒णम॑पश्यत् कृ॒ष्टं चाकृ॑ष्टं च॒ ततो॒ वा इ॒मां नाऽत्य॑दह॒द्यत् कृ॒ष्टं चाकृ॑ष्टं च॒ - [  ] \newline

\textbf{Pada Paata} \newline

वी॒र्य॑सम्मित॒ इति॑ वी॒र्य॑ - स॒म्मि॒तः॒ । वेणु॑ना । वीति॑ । मि॒मी॒ते॒ । आ॒ग्ने॒यः । वै । वेणुः॑ । स॒यो॒नि॒त्वायेति॑ सयोनि - त्वाय॑ । यजु॑षा । यु॒न॒क्ति॒ । यजु॑षा । कृ॒ष॒ति॒ । व्यावृ॑त्त्या॒ इति॑ वि - आवृ॑त्त्यै । ष॒ड्ग॒वेनेति॑ षट् - ग॒वेन॑ । कृ॒ष॒ति॒ । षट् । वै । ऋ॒तवः॑ । ऋ॒तुभि॒रित्यृ॒तु - भिः॒ । ए॒व । ए॒न॒म् । कृ॒ष॒ति॒ । यत् । द्वा॒द॒श॒ग॒वेनेति॑ द्वादश - ग॒वेन॑ । सं॒ॅव॒थ्स॒रेणेति॑ सं - व॒थ्स॒रेण॑ । ए॒व । इ॒यम् । वै । अ॒ग्नेः । अ॒ति॒दा॒हादित्य॑ति - दा॒हात् । अ॒बि॒भे॒त् । सा । ए॒तत् । द्वि॒गु॒णमिति॑ द्वि - गु॒णम् । अ॒प॒श्य॒त् । कृ॒ष्टम् । च॒ । अकृ॑ष्टम् । च॒ । ततः॑ । वै । इ॒माम् । न । अतीति॑ । अ॒द॒ह॒त् । यत् । कृ॒ष्टम् । च॒ । अकृ॑ष्टम् । च॒ ।  \newline


\textbf{Krama Paata} \newline

वी॒र्य॑सम्मितो॒ वेणु॑ना । वी॒र्य॑सम्मित॒ इति॑ वी॒र्य॑ - स॒म्मि॒तः॒ । वेणु॑ना॒ वि । वि मि॑मीते । मि॒मी॒त॒ आ॒ग्ने॒यः । आ॒ग्ने॒यो वै । वै वेणुः॑ । वेणुः॑ सयोनि॒त्वाय॑ । स॒यो॒नि॒त्वाय॒ यजु॑षा । स॒यो॒नि॒त्वायेति॑ सयोनि - त्वाय॑ । यजु॑षा युनक्ति । य॒न॒क्ति॒ यजु॑षा । यजु॑षा कृषति । कृ॒ष॒ति॒ व्यावृ॑त्त्यै । व्यावृ॑त्त्यै षड्ग॒वेन॑ । व्यावृ॑त्त्या॒ इति॑ वि - आवृ॑त्त्यै । ष॒ड्ग॒वेन॑ कृषति । ष॒ड्ग॒वेनेति॑ षट् - ग॒वेन॑ । कृ॒ष॒ति॒ षट् । षड् वै । वा ऋ॒तवः॑ । ऋ॒तव॑ ऋ॒तुभिः॑ । ऋ॒तुभि॑रे॒व । ऋ॒तुभि॒रितित्यृ॒तु - भिः॒ । ए॒वैन᳚म् । ए॒न॒म् कृ॒ष॒ति॒ । कृ॒ष॒ति॒ यत् । यद् द्वा॑दशग॒वेन॑ । द्वा॒द॒श॒ग॒वेन॑ सम्ॅवथ्स॒रेण॑ । द्वा॒द॒श॒ग॒वेनेति॑ द्वादश - ग॒वेन॑ । स॒म्ॅव॒थ्स॒रेणै॒व । स॒म्ॅव॒थ्स॒रेणेति॑ सम् - व॒थ्स॒रेण॑ । ए॒वेयम् । इ॒यम् ॅवै । वा अ॒ग्नेः । अ॒ग्नेर॑तिदा॒हात् । अ॒ति॒दा॒हाद॑बिभेत् । अ॒ति॒दा॒हादित्य॑ति - दा॒हात् । अ॒बि॒भे॒थ् सा । सैतत् । ए॒तद् द्वि॑गु॒णम् । द्वि॒गु॒णम॑पश्यत् । द्वि॒गु॒णमिति॑ द्वि - गु॒णम् । अ॒प॒श्य॒त् कृ॒ष्टम् । कृ॒ष्टम् च॑ । चाकृ॑ष्टम् । अकृ॑ष्टम् च । च॒ ततः॑ । ततो॒ वै । वा इ॒माम् । इ॒माम् न । नाति॑ । अत्य॑दहत् । अ॒द॒ह॒द् यत् । यत् कृ॒ष्टम् । कृ॒ष्टम् च॑ । चाकृ॑ष्टम् । अकृ॑ष्टम् च । च॒ भव॑ति \newline

\textbf{Jatai Paata} \newline

1. वी॒र्य॑सम्मितो॒ वेणु॑ना॒ वेणु॑ना वी॒र्य॑सम्मितो वी॒र्य॑सम्मितो॒ वेणु॑ना । \newline
2. वी॒र्य॑सम्मित॒ इति॑ वी॒र्य॑ - स॒म्मि॒तः॒ । \newline
3. वेणु॑ना॒ वि वि वेणु॑ना॒ वेणु॑ना॒ वि । \newline
4. वि मि॑मीते मिमीते॒ वि वि मि॑मीते । \newline
5. मि॒मी॒त॒ आ॒ग्ने॒य आ᳚ग्ने॒यो मि॑मीते मिमीत आग्ने॒यः । \newline
6. आ॒ग्ने॒यो वै वा आ᳚ग्ने॒य आ᳚ग्ने॒यो वै । \newline
7. वै वेणु॒र् वेणु॒र् वै वै वेणुः॑ । \newline
8. वेणुः॑ सयोनि॒त्वाय॑ सयोनि॒त्वाय॒ वेणु॒र् वेणुः॑ सयोनि॒त्वाय॑ । \newline
9. स॒यो॒नि॒त्वाय॒ यजु॑षा॒ यजु॑षा सयोनि॒त्वाय॑ सयोनि॒त्वाय॒ यजु॑षा । \newline
10. स॒यो॒नि॒त्वायेति॑ सयोनि - त्वाय॑ । \newline
11. यजु॑षा युनक्ति युनक्ति॒ यजु॑षा॒ यजु॑षा युनक्ति । \newline
12. यु॒न॒क्ति॒ यजु॑षा॒ यजु॑षा युनक्ति युनक्ति॒ यजु॑षा । \newline
13. यजु॑षा कृषति कृषति॒ यजु॑षा॒ यजु॑षा कृषति । \newline
14. कृ॒ष॒ति॒ व्यावृ॑त्त्यै॒ व्यावृ॑त्त्यै कृषति कृषति॒ व्यावृ॑त्त्यै । \newline
15. व्यावृ॑त्त्यै षड्ग॒वेन॑ षड्ग॒वेन॒ व्यावृ॑त्त्यै॒ व्यावृ॑त्त्यै षड्ग॒वेन॑ । \newline
16. व्यावृ॑त्त्या॒ इति॑ वि - आवृ॑त्त्यै । \newline
17. ष॒ड्ग॒वेन॑ कृषति कृषति षड्ग॒वेन॑ षड्ग॒वेन॑ कृषति । \newline
18. ष॒ड्ग॒वेनेति॑ षट् - ग॒वेन॑ । \newline
19. कृ॒ष॒ति॒ षट् थ्षट् कृ॑षति कृषति॒ षट् । \newline
20. षड् वै वै षट् थ्षड् वै । \newline
21. वा ऋ॒तव॑ ऋ॒तवो॒ वै वा ऋ॒तवः॑ । \newline
22. ऋ॒तव॑ ऋ॒तुभिर्॑. ऋ॒तुभिर्॑. ऋ॒तव॑ ऋ॒तव॑ ऋ॒तुभिः॑ । \newline
23. ऋ॒तुभि॑ रे॒वैव र्‌तुभिर्॑. ऋ॒तुभि॑ रे॒व । \newline
24. ऋ॒तुभि॒रित्यृ॒तु - भिः॒ । \newline
25. ए॒वैन॑ मेन मे॒वैवैन᳚म् । \newline
26. ए॒न॒म् कृ॒ष॒ति॒ कृ॒ष॒ त्ये॒न॒ मे॒न॒म् कृ॒ष॒ति॒ । \newline
27. कृ॒ष॒ति॒ यद् यत् कृ॑षति कृषति॒ यत् । \newline
28. यद् द्वा॑दशग॒वेन॑ द्वादशग॒वेन॒ यद् यद् द्वा॑दशग॒वेन॑ । \newline
29. द्वा॒द॒श॒ग॒वेन॑ संॅवथ्स॒रेण॑ संॅवथ्स॒रेण॑ द्वादशग॒वेन॑ द्वादशग॒वेन॑ संॅवथ्स॒रेण॑ । \newline
30. द्वा॒द॒श॒ग॒वेनेति॑ द्वादश - ग॒वेन॑ । \newline
31. सं॒ॅव॒थ्स॒रे णै॒वैव सं॑ॅवथ्स॒रेण॑ संॅवथ्स॒रे णै॒व । \newline
32. सं॒ॅव॒थ्स॒रेणेति॑ सं - व॒थ्स॒रेण॑ । \newline
33. ए॒वेय मि॒य मे॒वै वेयम् । \newline
34. इ॒यं ॅवै वा इ॒य मि॒यं ॅवै । \newline
35. वा अ॒ग्ने र॒ग्नेर् वै वा अ॒ग्नेः । \newline
36. अ॒ग्ने र॑तिदा॒हा द॑तिदा॒हा द॒ग्ने र॒ग्ने र॑तिदा॒हात् । \newline
37. अ॒ति॒दा॒हा द॑बिभे दबिभे दतिदा॒हा द॑तिदा॒हा द॑बिभेत् । \newline
38. अ॒ति॒दा॒हादित्य॑ति - दा॒हात् । \newline
39. अ॒बि॒भे॒थ् सा सा ऽबि॑भे दबिभे॒थ् सा । \newline
40. सैत दे॒तथ् सा सैतत् । \newline
41. ए॒तद् द्वि॑गु॒णम् द्वि॑गु॒ण मे॒त दे॒तद् द्वि॑गु॒णम् । \newline
42. द्वि॒गु॒ण म॑पश्य दपश्यद् द्विगु॒णम् द्वि॑गु॒ण म॑पश्यत् । \newline
43. द्वि॒गु॒णमिति॑ द्वि - गु॒णम् । \newline
44. अ॒प॒श्य॒त् कृ॒ष्टम् कृ॒ष्ट म॑पश्य दपश्यत् कृ॒ष्टम् । \newline
45. कृ॒ष्टम् च॑ च कृ॒ष्टम् कृ॒ष्टम् च॑ । \newline
46. चाकृ॑ष्ट॒ मकृ॑ष्टम् च॒ चाकृ॑ष्टम् । \newline
47. अकृ॑ष्टम् च॒ चाकृ॑ष्ट॒ मकृ॑ष्टम् च । \newline
48. च॒ तत॒ स्तत॑श्च च॒ ततः॑ । \newline
49. ततो॒ वै वै तत॒ स्ततो॒ वै । \newline
50. वा इ॒मा मि॒मां ॅवै वा इ॒माम् । \newline
51. इ॒माम् न ने मा मि॒माम् न । \newline
52. नात्यति॒ न नाति॑ । \newline
53. अत्य॑दह ददह॒ दत्य त्य॑दहत् । \newline
54. अ॒द॒ह॒द् यद् यद॑दह ददह॒द् यत् । \newline
55. यत् कृ॒ष्टम् कृ॒ष्टं ॅयद् यत् कृ॒ष्टम् । \newline
56. कृ॒ष्टम् च॑ च कृ॒ष्टम् कृ॒ष्टम् च॑ । \newline
57. चाकृ॑ष्ट॒ मकृ॑ष्टम् च॒ चाकृ॑ष्टम् । \newline
58. अकृ॑ष्टम् च॒ चाकृ॑ष्ट॒ मकृ॑ष्टम् च । \newline
59. च॒ भव॑ति॒ भव॑ति च च॒ भव॑ति । \newline

\textbf{Ghana Paata } \newline

1. वी॒र्य॑सम्मितो॒ वेणु॑ना॒ वेणु॑ना वी॒र्य॑सम्मितो वी॒र्य॑सम्मितो॒ वेणु॑ना॒ वि वि वेणु॑ना वी॒र्य॑सम्मितो वी॒र्य॑सम्मितो॒ वेणु॑ना॒ वि । \newline
2. वी॒र्य॑सम्मित॒ इति॑ वी॒र्य॑ - स॒म्मि॒तः॒ । \newline
3. वेणु॑ना॒ वि वि वेणु॑ना॒ वेणु॑ना॒ वि मि॑मीते मिमीते॒ वि वेणु॑ना॒ वेणु॑ना॒ वि मि॑मीते । \newline
4. वि मि॑मीते मिमीते॒ वि वि मि॑मीत आग्ने॒य आ᳚ग्ने॒यो मि॑मीते॒ वि वि मि॑मीत आग्ने॒यः । \newline
5. मि॒मी॒त॒ आ॒ग्ने॒य आ᳚ग्ने॒यो मि॑मीते मिमीत आग्ने॒यो वै वा आ᳚ग्ने॒यो मि॑मीते मिमीत आग्ने॒यो वै । \newline
6. आ॒ग्ने॒यो वै वा आ᳚ग्ने॒य आ᳚ग्ने॒यो वै वेणु॒र् वेणु॒र् वा आ᳚ग्ने॒य आ᳚ग्ने॒यो वै वेणुः॑ । \newline
7. वै वेणु॒र् वेणु॒र् वै वै वेणुः॑ सयोनि॒त्वाय॑ सयोनि॒त्वाय॒ वेणु॒र् वै वै वेणुः॑ सयोनि॒त्वाय॑ । \newline
8. वेणुः॑ सयोनि॒त्वाय॑ सयोनि॒त्वाय॒ वेणु॒र् वेणुः॑ सयोनि॒त्वाय॒ यजु॑षा॒ यजु॑षा सयोनि॒त्वाय॒ वेणु॒र् वेणुः॑ सयोनि॒त्वाय॒ यजु॑षा । \newline
9. स॒यो॒नि॒त्वाय॒ यजु॑षा॒ यजु॑षा सयोनि॒त्वाय॑ सयोनि॒त्वाय॒ यजु॑षा युनक्ति युनक्ति॒ यजु॑षा सयोनि॒त्वाय॑ सयोनि॒त्वाय॒ यजु॑षा युनक्ति । \newline
10. स॒यो॒नि॒त्वायेति॑ सयोनि - त्वाय॑ । \newline
11. यजु॑षा युनक्ति युनक्ति॒ यजु॑षा॒ यजु॑षा युनक्ति॒ यजु॑षा॒ यजु॑षा युनक्ति॒ यजु॑षा॒ यजु॑षा युनक्ति॒ यजु॑षा । \newline
12. यु॒न॒क्ति॒ यजु॑षा॒ यजु॑षा युनक्ति युनक्ति॒ यजु॑षा कृषति कृषति॒ यजु॑षा युनक्ति युनक्ति॒ यजु॑षा कृषति । \newline
13. यजु॑षा कृषति कृषति॒ यजु॑षा॒ यजु॑षा कृषति॒ व्यावृ॑त्त्यै॒ व्यावृ॑त्त्यै कृषति॒ यजु॑षा॒ यजु॑षा कृषति॒ व्यावृ॑त्त्यै । \newline
14. कृ॒ष॒ति॒ व्यावृ॑त्त्यै॒ व्यावृ॑त्त्यै कृषति कृषति॒ व्यावृ॑त्त्यै षड्ग॒वेन॑ षड्ग॒वेन॒ व्यावृ॑त्त्यै कृषति कृषति॒ व्यावृ॑त्त्यै षड्ग॒वेन॑ । \newline
15. व्यावृ॑त्त्यै षड्ग॒वेन॑ षड्ग॒वेन॒ व्यावृ॑त्त्यै॒ व्यावृ॑त्त्यै षड्ग॒वेन॑ कृषति कृषति षड्ग॒वेन॒ व्यावृ॑त्त्यै॒ व्यावृ॑त्त्यै षड्ग॒वेन॑ कृषति । \newline
16. व्यावृ॑त्त्या॒ इति॑ वि - आवृ॑त्त्यै । \newline
17. ष॒ड्ग॒वेन॑ कृषति कृषति षड्ग॒वेन॑ षड्ग॒वेन॑ कृषति॒ षट् थ्षट् कृ॑षति षड्ग॒वेन॑ षड्ग॒वेन॑ कृषति॒ षट् । \newline
18. ष॒ड्ग॒वेनेति॑ षट् - ग॒वेन॑ । \newline
19. कृ॒ष॒ति॒ षट् थ्षट् कृ॑षति कृषति॒ षड् वै वै षट् कृ॑षति कृषति॒ षड् वै । \newline
20. षड् वै वै षट् थ्षड् वा ऋ॒तव॑ ऋ॒तवो॒ वै षट् थ्षड् वा ऋ॒तवः॑ । \newline
21. वा ऋ॒तव॑ ऋ॒तवो॒ वै वा ऋ॒तव॑ ऋ॒तुभिर्॑. ऋ॒तुभिर्॑. ऋ॒तवो॒ वै वा ऋ॒तव॑ ऋ॒तुभिः॑ । \newline
22. ऋ॒तव॑ ऋ॒तुभिर्॑. ऋ॒तुभिर्॑. ऋ॒तव॑ ऋ॒तव॑ ऋ॒तुभि॑ रे॒वैव र्‌तुभिर्॑. ऋ॒तव॑ ऋ॒तव॑ ऋ॒तुभि॑ रे॒व । \newline
23. ऋ॒तुभि॑ रे॒वैव र्‌तुभिर्॑. ऋ॒तुभि॑ रे॒वैन॑ मेन मे॒व र्‌तुभिर्॑. ऋ॒तुभि॑ रे॒वैन᳚म् । \newline
24. ऋ॒तुभि॒रित्यृ॒तु - भिः॒ । \newline
25. ए॒वैन॑ मेन मे॒वैवैन॑म् कृषति कृष त्येन मे॒वैवैन॑म् कृषति । \newline
26. ए॒न॒म् कृ॒ष॒ति॒ कृ॒ष॒ त्ये॒न॒ मे॒न॒म् कृ॒ष॒ति॒ यद् यत् कृ॑ष त्येन मेनम् कृषति॒ यत् । \newline
27. कृ॒ष॒ति॒ यद् यत् कृ॑षति कृषति॒ यद् द्वा॑दशग॒वेन॑ द्वादशग॒वेन॒ यत् कृ॑षति कृषति॒ यद् द्वा॑दशग॒वेन॑ । \newline
28. यद् द्वा॑दशग॒वेन॑ द्वादशग॒वेन॒ यद् यद् द्वा॑दशग॒वेन॑ संॅवथ्स॒रेण॑ संॅवथ्स॒रेण॑ द्वादशग॒वेन॒ यद् यद् द्वा॑दशग॒वेन॑ संॅवथ्स॒रेण॑ । \newline
29. द्वा॒द॒श॒ग॒वेन॑ संॅवथ्स॒रेण॑ संॅवथ्स॒रेण॑ द्वादशग॒वेन॑ द्वादशग॒वेन॑ संॅवथ्स॒रे णै॒वैव सं॑ॅवथ्स॒रेण॑ द्वादशग॒वेन॑ द्वादशग॒वेन॑ संॅवथ्स॒रेणै॒व । \newline
30. द्वा॒द॒श॒ग॒वेनेति॑ द्वादश - ग॒वेन॑ । \newline
31. सं॒ॅव॒थ्स॒रे णै॒वैव सं॑ॅवथ्स॒रेण॑ संॅवथ्स॒रे णै॒वेय मि॒य मे॒व सं॑ॅवथ्स॒रेण॑ संॅवथ्स॒रे णै॒वेयम् । \newline
32. सं॒ॅव॒थ्स॒रेणेति॑ सं - व॒थ्स॒रेण॑ । \newline
33. ए॒वेय मि॒य मे॒वैवेयं ॅवै वा इ॒य मे॒वैवेयं ॅवै । \newline
34. इ॒यं ॅवै वा इ॒य मि॒यं ॅवा अ॒ग्ने र॒ग्नेर् वा इ॒य मि॒यं ॅवा अ॒ग्नेः । \newline
35. वा अ॒ग्ने र॒ग्नेर् वै वा अ॒ग्ने र॑तिदा॒हा द॑तिदा॒हा द॒ग्नेर् वै वा अ॒ग्ने र॑तिदा॒हात् । \newline
36. अ॒ग्ने र॑तिदा॒हा द॑तिदा॒हा द॒ग्ने र॒ग्ने र॑तिदा॒हा द॑बिभे दबिभे दतिदा॒हा द॒ग्ने र॒ग्ने र॑तिदा॒हा द॑बिभेत् । \newline
37. अ॒ति॒दा॒हा द॑बिभे दबिभे दतिदा॒हा द॑तिदा॒हा द॑बिभे॒थ् सा सा ऽबि॑भे दतिदा॒हा द॑तिदा॒हा द॑बिभे॒थ् सा । \newline
38. अ॒ति॒दा॒हादित्य॑ति - दा॒हात् । \newline
39. अ॒बि॒भे॒थ् सा सा ऽबि॑भे दबिभे॒थ् सैत दे॒तथ् सा ऽबि॑भे दबिभे॒थ् सैतत् । \newline
40. सैत दे॒तथ् सा सैतद् द्वि॑गु॒णम् द्वि॑गु॒ण मे॒तथ् सा सैतद् द्वि॑गु॒णम् । \newline
41. ए॒तद् द्वि॑गु॒णम् द्वि॑गु॒ण मे॒त दे॒तद् द्वि॑गु॒ण म॑पश्य दपश्यद् द्विगु॒ण मे॒त दे॒तद् द्वि॑गु॒ण म॑पश्यत् । \newline
42. द्वि॒गु॒ण म॑पश्य दपश्यद् द्विगु॒णम् द्वि॑गु॒ण म॑पश्यत् कृ॒ष्टम् कृ॒ष्ट म॑पश्यद् द्विगु॒णम् द्वि॑गु॒ण म॑पश्यत् कृ॒ष्टम् । \newline
43. द्वि॒गु॒णमिति॑ द्वि - गु॒णम् । \newline
44. अ॒प॒श्य॒त् कृ॒ष्टम् कृ॒ष्ट म॑पश्य दपश्यत् कृ॒ष्टम् च॑ च कृ॒ष्ट म॑पश्य दपश्यत् कृ॒ष्टम् च॑ । \newline
45. कृ॒ष्टम् च॑ च कृ॒ष्टम् कृ॒ष्टम् चाकृ॑ष्ट॒ मकृ॑ष्टम् च कृ॒ष्टम् कृ॒ष्टम् चाकृ॑ष्टम् । \newline
46. चाकृ॑ष्ट॒ मकृ॑ष्टम् च॒ चाकृ॑ष्टम् च॒ चाकृ॑ष्टम् च॒ चाकृ॑ष्टम् च । \newline
47. अकृ॑ष्टम् च॒ चाकृ॑ष्ट॒ मकृ॑ष्टम् च॒ तत॒ स्तत॒ श्चाकृ॑ष्ट॒ मकृ॑ष्टम् च॒ ततः॑ । \newline
48. च॒ तत॒ स्तत॑श्च च॒ ततो॒ वै वै तत॑श्च च॒ ततो॒ वै । \newline
49. ततो॒ वै वै तत॒ स्ततो॒ वा इ॒मा मि॒मां ॅवै तत॒ स्ततो॒ वा इ॒माम् । \newline
50. वा इ॒मा मि॒मां ॅवै वा इ॒माम् न ने मां ॅवै वा इ॒माम् न । \newline
51. इ॒माम् न ने मा मि॒माम् नात्यति॒ ने मा मि॒माम् नाति॑ । \newline
52. नात्यति॒ न नात्य॑दहद दह॒दति॒ न नात्य॑दहत् । \newline
53. अत्य॑दह ददह॒ दत्य त्य॑दह॒द् यद् यद॑दह॒ दत्य त्य॑दह॒द् यत् । \newline
54. अ॒द॒ह॒द् यद् यद॑दह ददह॒द् यत् कृ॒ष्टम् कृ॒ष्टं ॅयद॑दह ददह॒द् यत् कृ॒ष्टम् । \newline
55. यत् कृ॒ष्टम् कृ॒ष्टं ॅयद् यत् कृ॒ष्टम् च॑ च कृ॒ष्टं ॅयद् यत् कृ॒ष्टम् च॑ । \newline
56. कृ॒ष्टम् च॑ च कृ॒ष्टम् कृ॒ष्टम् चाकृ॑ष्ट॒ मकृ॑ष्टम् च कृ॒ष्टम् कृ॒ष्टम् चाकृ॑ष्टम् । \newline
57. चाकृ॑ष्ट॒ मकृ॑ष्टम् च॒ चाकृ॑ष्टम् च॒ चाकृ॑ष्टम् च॒ चाकृ॑ष्टम् च । \newline
58. अकृ॑ष्टम् च॒ चाकृ॑ष्ट॒ मकृ॑ष्टम् च॒ भव॑ति॒ भव॑ति॒ चाकृ॑ष्ट॒ मकृ॑ष्टम् च॒ भव॑ति । \newline
59. च॒ भव॑ति॒ भव॑ति च च॒ भव॑ त्य॒स्या अ॒स्या भव॑ति च च॒ भव॑ त्य॒स्याः । \newline
\pagebreak
\markright{ TS 5.2.5.3  \hfill https://www.vedavms.in \hfill}

\section{ TS 5.2.5.3 }

\textbf{TS 5.2.5.3 } \newline
\textbf{Samhita Paata} \newline

भव॑त्य॒स्या अन॑तिदाहाय द्विगु॒णं त्वा अ॒ग्नि-मुद्य॑न्तु-मर्.ह॒तीत्या॑हु॒र्यत् कृ॒ष्टं चाकृ॑ष्टं च॒ भव॑त्य॒ग्नेरुद्य॑त्या ए॒ताव॑न्तो॒ वै प॒शवो᳚ द्वि॒पाद॑श्च॒ चतु॑ष्पादश्च॒ तान्. यत् प्राच॑ उथ्सृ॒जेद्-रु॒द्रायापि॑ दद्ध्या॒द्-यद्-द॑क्षि॒णा पि॒तृभ्यो॒ निधु॑वे॒द्यत् प्र॒तीचो॒ रक्षाꣳ॑सि हन्यु॒रुदी॑च॒ उथ्सृ॑जत्ये॒षा वै दे॑वमनु॒ष्याणाꣳ॑ शा॒न्ता दिक् - [  ] \newline

\textbf{Pada Paata} \newline

भव॑ति । अ॒स्याः । अन॑तिदाहा॒येत्यन॑ति - दा॒हा॒य॒ । द्वि॒गु॒णमिति॑ द्वि - गु॒णम् । तु । वै । अ॒ग्निम् । उद्य॑न्तु॒मित्युत्-य॒न्तु॒म् । अ॒र्.॒ह॒ति॒ । इति॑ । आ॒हुः॒ । यत् । कृ॒ष्टम् । च॒ । अकृ॑ष्टम् । च॒ । भव॑ति । अ॒ग्नेः । उद्य॑त्या॒ इत्युत् - य॒त्यै॒ । ए॒ताव॑न्तः । वै । प॒शवः॑ । द्वि॒पाद॒ इति॑ द्वि - पादः॑ । च॒ । चतु॑ष्पाद॒ इति॒ चतुः॑ - पा॒दः॒ । च॒ । तान् । यत् । प्राचः॑ । उ॒थ्सृ॒जेदित्यु॑त् - सृ॒जेत् । रु॒द्राय॑ । अपीति॑ । द॒द्ध्या॒त् । यत् । द॒क्षि॒णा । पि॒तृभ्य॒ इति॑ पि॒तृ - भ्यः॒ । नीति॑ । धु॒वे॒त् । यत् । प्र॒तीचः॑ । रक्षाꣳ॑सि । ह॒न्युः॒ । उदी॑चः । उदिति॑ । सृ॒ज॒ति॒ । ए॒षा । वै । दे॒व॒म॒नु॒ष्याणा॒मिति॑ देव - म॒नु॒ष्याणा᳚म् । शा॒न्ता । दिक् ।  \newline


\textbf{Krama Paata} \newline

भव॑त्य॒स्याः । अ॒स्या अन॑तिदाहाय । अन॑तिदाहाय द्विगु॒णम् । अन॑तिदाहा॒येत्यन॑ति - दा॒हा॒य॒ । द्वि॒गु॒णम् तु । द्वि॒गु॒णमिति॑ द्वि - गु॒णम् । त्वै । वा अ॒ग्निम् । अ॒ग्निमुद्य॑न्तुम् । उद्य॑न्तुमर्.हति । उद्य॑न्तु॒मित्युत् - य॒न्तु॒म् । अ॒र्.॒ह॒तीति॑ । इत्या॑हुः । आ॒हु॒र् यत् । यत् कृ॒ष्टम् । कृ॒ष्टम् च॑ । चाकृ॑ष्टम् । अकृ॑ष्टम् च । च॒ भव॑ति । भव॑त्य॒ग्नेः । अ॒ग्नेरुद्य॑त्यै । उद्य॑त्या ए॒ताव॑न्तः । उद्य॑त्या॒ इत्युत् - य॒त्यै॒ । ए॒ताव॑न्तो॒ वै । वै प॒शवः॑ । प॒शवो᳚ द्वि॒पादः॑ । द्वि॒पाद॑श्च । द्वि॒पाद॒ इति॑ द्वि - पादः॑ । च॒ चतु॑ष्पादः । चतु॑ष्पादश्च । चतु॑ष्पाद॒ इति॒ चतुः॑ - पा॒दः॒ । च॒ तान् । तान्. यत् । यत् प्राचः॑ । प्राच॑ उथ्सृ॒जेत् । उ॒थ्सृ॒जेद् रु॒द्राय॑ । उ॒थ्सृ॒जेदित्यु॑त् - सृ॒जेत् । रु॒द्रायापि॑ । अपि॑ दद्ध्यात् । द॒द्ध्या॒द् यत् । यद् द॑क्षि॒णा । द॒क्षि॒णा पि॒तृभ्यः॑ । पि॒तृभ्यो॒ नि । पि॒तृभ्य॒ इति॑ पि॒तृ - भ्यः॒ । नि धु॑वेत् । धु॒वे॒द् यत् । यत् प्र॒तीचः॑ । प्र॒तीचो॒ रक्षाꣳ॑सि । रक्षाꣳ॑सि हन्युः । ह॒न्यु॒रुदी॑चः । उदी॑च॒ उत् । उथ् सृ॑जति । सृ॒ज॒त्ये॒षा । ए॒षा वै । वै दे॑वमनु॒ष्याणा᳚म् । दे॒व॒म॒नु॒ष्याणाꣳ॑ शा॒न्ता । दे॒व॒म॒नु॒ष्याणा॒मिति॑ देव - म॒नु॒ष्याणा᳚म् । शा॒न्ता दिक् । दिक् ताम् \newline

\textbf{Jatai Paata} \newline

1. भव॑ त्य॒स्या अ॒स्या भव॑ति॒ भव॑ त्य॒स्याः । \newline
2. अ॒स्या अन॑तिदाहा॒या न॑तिदाहाया॒स्या अ॒स्या अन॑तिदाहाय । \newline
3. अन॑तिदाहाय द्विगु॒णम् द्वि॑गु॒ण मन॑तिदाहा॒या न॑तिदाहाय द्विगु॒णम् । \newline
4. अन॑तिदाहा॒येत्यन॑ति - दा॒हा॒य॒ । \newline
5. द्वि॒गु॒णम् तु तु द्वि॑गु॒णम् द्वि॑गु॒णम् तु । \newline
6. द्वि॒गु॒णमिति॑ द्वि - गु॒णम् । \newline
7. त्वै वै तु त्वै । \newline
8. वा अ॒ग्नि म॒ग्निं ॅवै वा अ॒ग्निम् । \newline
9. अ॒ग्नि मुद्य॑न्तु॒ मुद्य॑न्तु म॒ग्नि म॒ग्नि मुद्य॑न्तुम् । \newline
10. उद्य॑न्तु मर्.ह त्यर्.ह॒ त्युद्य॑न्तु॒ मुद्य॑न्तु मर्.हति । \newline
11. उद्य॑न्तु॒मित्युत् - य॒न्तु॒म् । \newline
12. अ॒र्॒.ह॒तीती त्य॑र्.ह त्यर्.ह॒तीति॑ । \newline
13. इत्या॑हु राहु॒ रिती त्या॑हुः । \newline
14. आ॒हु॒र् यद् यदा॑हु राहु॒र् यत् । \newline
15. यत् कृ॒ष्टम् कृ॒ष्टं ॅयद् यत् कृ॒ष्टम् । \newline
16. कृ॒ष्टम् च॑ च कृ॒ष्टम् कृ॒ष्टम् च॑ । \newline
17. चाकृ॑ष्ट॒ मकृ॑ष्टम् च॒ चाकृ॑ष्टम् । \newline
18. अकृ॑ष्टम् च॒ चाकृ॑ष्ट॒ मकृ॑ष्टम् च । \newline
19. च॒ भव॑ति॒ भव॑ति च च॒ भव॑ति । \newline
20. भव॑ त्य॒ग्ने र॒ग्नेर् भव॑ति॒ भव॑ त्य॒ग्नेः । \newline
21. अ॒ग्ने रुद्य॑त्या॒ उद्य॑त्या अ॒ग्ने र॒ग्ने रुद्य॑त्यै । \newline
22. उद्य॑त्या ए॒ताव॑न्त ए॒ताव॑न्त॒ उद्य॑त्या॒ उद्य॑त्या ए॒ताव॑न्तः । \newline
23. उद्य॑त्या॒ इत्युत् - य॒त्यै॒ । \newline
24. ए॒ताव॑न्तो॒ वै वा ए॒ताव॑न्त ए॒ताव॑न्तो॒ वै । \newline
25. वै प॒शवः॑ प॒शवो॒ वै वै प॒शवः॑ । \newline
26. प॒शवो᳚ द्वि॒पादो᳚ द्वि॒पादः॑ प॒शवः॑ प॒शवो᳚ द्वि॒पादः॑ । \newline
27. द्वि॒पाद॑श्च च द्वि॒पादो᳚ द्वि॒पाद॑श्च । \newline
28. द्वि॒पाद॒ इति॑ द्वि - पादः॑ । \newline
29. च॒ चतु॑ष्पाद॒ श्चतु॑ष्पादश्च च॒ चतु॑ष्पादः । \newline
30. चतु॑ष्पादश्च च॒ चतु॑ष्पाद॒ श्चतु॑ष्पादश्च । \newline
31. चतु॑ष्पाद॒ इति॒ चतुः॑ - पा॒दः॒ । \newline
32. च॒ ताꣳ स्ताꣳ श्च॑ च॒ तान् । \newline
33. तान्. यद् यत् ताꣳ स्तान्. यत् । \newline
34. यत् प्राचः॒ प्राचो॒ यद् यत् प्राचः॑ । \newline
35. प्राच॑ उथ्सृ॒जे दु॑थ्सृ॒जेत् प्राचः॒ प्राच॑ उथ्सृ॒जेत् । \newline
36. उ॒थ्सृ॒जेद् रु॒द्राय॑ रु॒द्रायो᳚थ्सृ॒जे दु॑थ्सृ॒जेद् रु॒द्राय॑ । \newline
37. उ॒थ्सृ॒जेदित्यु॑त् - सृ॒जेत् । \newline
38. रु॒द्राया प्यपि॑ रु॒द्राय॑ रु॒द्रायापि॑ । \newline
39. अपि॑ दद्ध्याद् दद्ध्या॒ दप्यपि॑ दद्ध्यात् । \newline
40. द॒द्ध्या॒द् यद् यद् द॑द्ध्याद् दद्ध्या॒द् यत् । \newline
41. यद् द॑क्षि॒णा द॑क्षि॒णा यद् यद् द॑क्षि॒णा । \newline
42. द॒क्षि॒णा पि॒तृभ्यः॑ पि॒तृभ्यो॑ दक्षि॒णा द॑क्षि॒णा पि॒तृभ्यः॑ । \newline
43. पि॒तृभ्यो॒ नि नि पि॒तृभ्यः॑ पि॒तृभ्यो॒ नि । \newline
44. पि॒तृभ्य॒ इति॑ पि॒तृ - भ्यः॒ । \newline
45. नि धु॑वेद् धुवे॒न् नि नि धु॑वेत् । \newline
46. धु॒वे॒द् यद् यद् धु॑वेद् धुवे॒द् यत् । \newline
47. यत् प्र॒तीचः॑ प्र॒तीचो॒ यद् यत् प्र॒तीचः॑ । \newline
48. प्र॒तीचो॒ रक्षाꣳ॑सि॒ रक्षाꣳ॑सि प्र॒तीचः॑ प्र॒तीचो॒ रक्षाꣳ॑सि । \newline
49. रक्षाꣳ॑सि हन्युर्. हन्यू॒ रक्षाꣳ॑सि॒ रक्षाꣳ॑सि हन्युः । \newline
50. ह॒न्यु॒ रुदी॑च॒ उदी॑चो हन्युर्. हन्यु॒ रुदी॑चः । \newline
51. उदी॑च॒ उदु दुदी॑च॒ उदी॑च॒ उत् । \newline
52. उथ् सृ॑जति सृज॒ त्युदुथ् सृ॑जति । \newline
53. सृ॒ज॒ त्ये॒षैषा सृ॑जति सृज त्ये॒षा । \newline
54. ए॒षा वै वा ए॒षैषा वै । \newline
55. वै दे॑वमनु॒ष्याणा᳚म् देवमनु॒ष्याणां॒ ॅवै वै दे॑वमनु॒ष्याणा᳚म् । \newline
56. दे॒व॒म॒नु॒ष्याणाꣳ॑ शा॒न्ता शा॒न्ता दे॑वमनु॒ष्याणा᳚म् देवमनु॒ष्याणाꣳ॑ शा॒न्ता । \newline
57. दे॒व॒म॒नु॒ष्याणा॒मिति॑ देव - म॒नु॒ष्याणा᳚म् । \newline
58. शा॒न्ता दिग् दिक् छा॒न्ता शा॒न्ता दिक् । \newline
59. दिक् ताम् ताम् दिग् दिक् ताम् । \newline

\textbf{Ghana Paata } \newline

1. भव॑त्य॒स्या अ॒स्या भव॑ति॒ भव॑ त्य॒स्या अन॑तिदाहा॒या न॑तिदाहाया॒ स्या भव॑ति॒ भव॑ त्य॒स्या अन॑तिदाहाय । \newline
2. अ॒स्या अन॑तिदाहा॒या न॑तिदाहाया॒ स्या अ॒स्या अन॑तिदाहाय द्विगु॒णम् द्वि॑गु॒ण मन॑तिदाहाया॒ स्या अ॒स्या अन॑तिदाहाय द्विगु॒णम् । \newline
3. अन॑तिदाहाय द्विगु॒णम् द्वि॑गु॒ण मन॑तिदाहा॒या न॑तिदाहाय द्विगु॒णम् तु तु द्वि॑गु॒ण मन॑तिदाहा॒या न॑तिदाहाय द्विगु॒णम् तु । \newline
4. अन॑तिदाहा॒येत्यन॑ति - दा॒हा॒य॒ । \newline
5. द्वि॒गु॒णम् तु तु द्वि॑गु॒णम् द्वि॑गु॒णम् त्वै वै तु द्वि॑गु॒णम् द्वि॑गु॒णम् त्वै । \newline
6. द्वि॒गु॒णमिति॑ द्वि - गु॒णम् । \newline
7. त्वै वै तु त्वा अ॒ग्नि म॒ग्निं ॅवै तु त्वा अ॒ग्निम् । \newline
8. वा अ॒ग्नि म॒ग्निं ॅवै वा अ॒ग्नि मुद्य॑न्तु॒ मुद्य॑न्तु म॒ग्निं ॅवै वा अ॒ग्नि मुद्य॑न्तुम् । \newline
9. अ॒ग्नि मुद्य॑न्तु॒ मुद्य॑न्तु म॒ग्नि म॒ग्नि मुद्य॑न्तु मर्.ह त्यर्.ह॒ त्युद्य॑न्तु म॒ग्नि म॒ग्नि मुद्य॑न्तु मर्.हति । \newline
10. उद्य॑न्तु मर्.ह त्यर्.ह॒ त्युद्य॑न्तु॒ मुद्य॑न्तु मर्.ह॒तीती त्य॑र्.ह॒ त्युद्य॑न्तु॒ मुद्य॑न्तु मर्.ह॒तीति॑ । \newline
11. उद्य॑न्तु॒मित्युत् - य॒न्तु॒म् । \newline
12. अ॒र्॒.ह॒तीती त्य॑र्.ह त्यर्.ह॒ती त्या॑हु राहु॒ रित्य॑र्.ह त्यर्.ह॒तीत्या॑हुः । \newline
13. इत्या॑हु राहु॒रिती त्या॑हु॒र् यद् यदा॑हु॒रिती त्या॑हु॒र् यत् । \newline
14. आ॒हु॒र् यद् यदा॑हु राहु॒र् यत् कृ॒ष्टम् कृ॒ष्टं ॅयदा॑हु राहु॒र् यत् कृ॒ष्टम् । \newline
15. यत् कृ॒ष्टम् कृ॒ष्टं ॅयद् यत् कृ॒ष्टम् च॑ च कृ॒ष्टं ॅयद् यत् कृ॒ष्टम् च॑ । \newline
16. कृ॒ष्टम् च॑ च कृ॒ष्टम् कृ॒ष्टम् चाकृ॑ष्ट॒ मकृ॑ष्टम् च कृ॒ष्टम् कृ॒ष्टम् चाकृ॑ष्टम् । \newline
17. चाकृ॑ष्ट॒ मकृ॑ष्टम् च॒ चाकृ॑ष्टम् च॒ चाकृ॑ष्टम् च॒ चाकृ॑ष्टम् च । \newline
18. अकृ॑ष्टम् च॒ चाकृ॑ष्ट॒ मकृ॑ष्टम् च॒ भव॑ति॒ भव॑ति॒ चाकृ॑ष्ट॒ मकृ॑ष्टम् च॒ भव॑ति । \newline
19. च॒ भव॑ति॒ भव॑ति च च॒ भव॑ त्य॒ग्ने र॒ग्नेर् भव॑ति च च॒ भव॑ त्य॒ग्नेः । \newline
20. भव॑ त्य॒ग्ने र॒ग्नेर् भव॑ति॒ भव॑ त्य॒ग्ने रुद्य॑त्या॒ उद्य॑त्या अ॒ग्नेर् भव॑ति॒ भव॑ त्य॒ग्ने रुद्य॑त्यै । \newline
21. अ॒ग्ने रुद्य॑त्या॒ उद्य॑त्या अ॒ग्ने र॒ग्ने रुद्य॑त्या ए॒ताव॑न्त ए॒ताव॑न्त॒ उद्य॑त्या अ॒ग्ने र॒ग्ने रुद्य॑त्या ए॒ताव॑न्तः । \newline
22. उद्य॑त्या ए॒ताव॑न्त ए॒ताव॑न्त॒ उद्य॑त्या॒ उद्य॑त्या ए॒ताव॑न्तो॒ वै वा ए॒ताव॑न्त॒ उद्य॑त्या॒ उद्य॑त्या ए॒ताव॑न्तो॒ वै । \newline
23. उद्य॑त्या॒ इत्युत् - य॒त्यै॒ । \newline
24. ए॒ताव॑न्तो॒ वै वा ए॒ताव॑न्त ए॒ताव॑न्तो॒ वै प॒शवः॑ प॒शवो॒ वा ए॒ताव॑न्त ए॒ताव॑न्तो॒ वै प॒शवः॑ । \newline
25. वै प॒शवः॑ प॒शवो॒ वै वै प॒शवो᳚ द्वि॒पादो᳚ द्वि॒पादः॑ प॒शवो॒ वै वै प॒शवो᳚ द्वि॒पादः॑ । \newline
26. प॒शवो᳚ द्वि॒पादो᳚ द्वि॒पादः॑ प॒शवः॑ प॒शवो᳚ द्वि॒पाद॑श्च च द्वि॒पादः॑ प॒शवः॑ प॒शवो᳚ द्वि॒पाद॑श्च । \newline
27. द्वि॒पाद॑श्च च द्वि॒पादो᳚ द्वि॒पाद॑श्च॒ चतु॑ष्पाद॒ श्चतु॑ष्पादश्च द्वि॒पादो᳚ द्वि॒पाद॑श्च॒ चतु॑ष्पादः । \newline
28. द्वि॒पाद॒ इति॑ द्वि - पादः॑ । \newline
29. च॒ चतु॑ष्पाद॒ श्चतु॑ष्पादश्च च॒ चतु॑ष्पादश्च च॒ चतु॑ष्पादश्च च॒ चतु॑ष्पादश्च । \newline
30. चतु॑ष्पादश्च च॒ चतु॑ष्पाद॒ श्चतु॑ष्पादश्च॒ ताꣳ स्ताꣳ श्च॒ चतु॑ष्पाद॒ श्चतु॑ष्पादश्च॒ तान् । \newline
31. चतु॑ष्पाद॒ इति॒ चतुः॑ - पा॒दः॒ । \newline
32. च॒ ताꣳ स्ताꣳश्च॑ च॒ तान्. यद् यत् ताꣳश्च॑ च॒ तान्. यत् । \newline
33. तान्. यद् यत् ताꣳ स्तान्. यत् प्राचः॒ प्राचो॒ यत् ताꣳ स्तान्. यत् प्राचः॑ । \newline
34. यत् प्राचः॒ प्राचो॒ यद् यत् प्राच॑ उथ्सृ॒जे दु॑थ्सृ॒जेत् प्राचो॒ यद् यत् प्राच॑ उथ्सृ॒जेत् । \newline
35. प्राच॑ उथ्सृ॒जे दु॑थ्सृ॒जेत् प्राचः॒ प्राच॑ उथ्सृ॒जेद् रु॒द्राय॑ रु॒द्रा यो᳚थ्सृ॒जेत् प्राचः॒ प्राच॑ उथ्सृ॒जेद् रु॒द्राय॑ । \newline
36. उ॒थ्सृ॒जेद् रु॒द्राय॑ रु॒द्रा यो᳚थ्सृ॒जे दु॑थ्सृ॒जेद् रु॒द्रायाप्यपि॑ रु॒द्रा यो᳚थ्सृ॒जे दु॑थ्सृ॒जेद् रु॒द्रायापि॑ । \newline
37. उ॒थ्सृ॒जेदित्यु॑त् - सृ॒जेत् । \newline
38. रु॒द्रायाप्यपि॑ रु॒द्राय॑ रु॒द्रायापि॑ दद्ध्याद् दद्ध्या॒दपि॑ रु॒द्राय॑ रु॒द्रायापि॑ दद्ध्यात् । \newline
39. अपि॑ दद्ध्याद् दद्ध्या॒ दप्यपि॑ दद्ध्या॒द् यद् यद् द॑द्ध्या॒ दप्यपि॑ दद्ध्या॒द् यत् । \newline
40. द॒द्ध्या॒द् यद् यद् द॑द्ध्याद् दद्ध्या॒द् यद् द॑क्षि॒णा द॑क्षि॒णा यद् द॑द्ध्याद् दद्ध्या॒द् यद् द॑क्षि॒णा । \newline
41. यद् द॑क्षि॒णा द॑क्षि॒णा यद् यद् द॑क्षि॒णा पि॒तृभ्यः॑ पि॒तृभ्यो॑ दक्षि॒णा यद् यद् द॑क्षि॒णा पि॒तृभ्यः॑ । \newline
42. द॒क्षि॒णा पि॒तृभ्यः॑ पि॒तृभ्यो॑ दक्षि॒णा द॑क्षि॒णा पि॒तृभ्यो॒ नि नि पि॒तृभ्यो॑ दक्षि॒णा द॑क्षि॒णा पि॒तृभ्यो॒ नि । \newline
43. पि॒तृभ्यो॒ नि नि पि॒तृभ्यः॑ पि॒तृभ्यो॒ नि धु॑वेद् धुवे॒न् नि पि॒तृभ्यः॑ पि॒तृभ्यो॒ नि धु॑वेत् । \newline
44. पि॒तृभ्य॒ इति॑ पि॒तृ - भ्यः॒ । \newline
45. नि धु॑वेद् धुवे॒न् नि नि धु॑वे॒द् यद् यद् धु॑वे॒न् नि नि धु॑वे॒द् यत् । \newline
46. धु॒वे॒द् यद् यद् धु॑वेद् धुवे॒द् यत् प्र॒तीचः॑ प्र॒तीचो॒ यद् धु॑वेद् धुवे॒द् यत् प्र॒तीचः॑ । \newline
47. यत् प्र॒तीचः॑ प्र॒तीचो॒ यद् यत् प्र॒तीचो॒ रक्षाꣳ॑सि॒ रक्षाꣳ॑सि प्र॒तीचो॒ यद् यत् प्र॒तीचो॒ रक्षाꣳ॑सि । \newline
48. प्र॒तीचो॒ रक्षाꣳ॑सि॒ रक्षाꣳ॑सि प्र॒तीचः॑ प्र॒तीचो॒ रक्षाꣳ॑सि हन्युर्. हन्यू॒ रक्षाꣳ॑सि प्र॒तीचः॑ प्र॒तीचो॒ रक्षाꣳ॑सि हन्युः । \newline
49. रक्षाꣳ॑सि हन्युर्. हन्यू॒ रक्षाꣳ॑सि॒ रक्षाꣳ॑सि हन्यु॒ रुदी॑च॒ उदी॑चो हन्यू॒ रक्षाꣳ॑सि॒ रक्षाꣳ॑सि हन्यु॒ रुदी॑चः । \newline
50. ह॒न्यु॒रुदी॑च॒ उदी॑चो हन्युर्. हन्यु॒ रुदी॑च॒ उदु दुदी॑चो हन्युर्. हन्यु॒ रुदी॑च॒ उत् । \newline
51. उदी॑च॒ उदु दुदी॑च॒ उदी॑च॒ उथ् सृ॑जति सृज॒ त्युदुदी॑च॒ उदी॑च॒ उथ् सृ॑जति । \newline
52. उथ् सृ॑जति सृज॒ त्युदुथ् सृ॑ज त्ये॒षैषा सृ॑ज॒ त्युदुथ् सृ॑ज त्ये॒षा । \newline
53. सृ॒ज॒ त्ये॒षैषा सृ॑जति सृज त्ये॒षा वै वा ए॒षा सृ॑जति सृज त्ये॒षा वै । \newline
54. ए॒षा वै वा ए॒षैषा वै दे॑वमनु॒ष्याणा᳚म् देवमनु॒ष्याणां॒ ॅवा ए॒षैषा वै दे॑वमनु॒ष्याणा᳚म् । \newline
55. वै दे॑वमनु॒ष्याणा᳚म् देवमनु॒ष्याणां॒ ॅवै वै दे॑वमनु॒ष्याणाꣳ॑ शा॒न्ता शा॒न्ता दे॑वमनु॒ष्याणां॒ ॅवै वै दे॑वमनु॒ष्याणाꣳ॑ शा॒न्ता । \newline
56. दे॒व॒म॒नु॒ष्याणाꣳ॑ शा॒न्ता शा॒न्ता दे॑वमनु॒ष्याणा᳚म् देवमनु॒ष्याणाꣳ॑ शा॒न्ता दिग् दिक् छा॒न्ता दे॑वमनु॒ष्याणा᳚म् देवमनु॒ष्याणाꣳ॑ शा॒न्ता दिक् । \newline
57. दे॒व॒म॒नु॒ष्याणा॒मिति॑ देव - म॒नु॒ष्याणा᳚म् । \newline
58. शा॒न्ता दिग् दिक् छा॒न्ता शा॒न्ता दिक् ताम् ताम् दिक् छा॒न्ता शा॒न्ता दिक् ताम् । \newline
59. दिक् ताम् ताम् दिग् दिक् ता मे॒वैव ताम् दिग् दिक् ता मे॒व । \newline
\pagebreak
\markright{ TS 5.2.5.4  \hfill https://www.vedavms.in \hfill}

\section{ TS 5.2.5.4 }

\textbf{TS 5.2.5.4 } \newline
\textbf{Samhita Paata} \newline

तामे॒वैना॒ननूथ् सृ॑ज॒त्यथो॒ खल्वि॒मां दिश॒मुथ् सृ॑जत्य॒सौ वा आ॑दि॒त्यः प्रा॒णः प्रा॒णमे॒वैना॒-ननूथ्सृ॑जति दक्षि॒णा प॒र्याव॑र्तन्ते॒ स्वमे॒व वी॒र्य॑मनु॑ प॒र्याव॑र्तन्ते॒ तस्मा॒द्-दक्षि॒णोऽर्द्ध॑ आ॒त्मनो॑ वी॒र्या॑वत्त॒रोऽथो॑ आदि॒त्यस्यै॒वाऽऽ*वृत॒मनु॑ प॒र्याव॑र्तन्ते॒ तस्मा॒त् परा᳚ञ्चः प॒शवो॒ वि ति॑ष्ठन्ते प्र॒त्यञ्च॒ आ व॑र्तन्ते ति॒स्रस्ति॑स्रः॒ सीताः᳚ - [  ] \newline

\textbf{Pada Paata} \newline

ताम् । ए॒व । ए॒ना॒न् । अनु॑ । उदिति॑ । सृ॒ज॒ति॒ । अथो॒ इति॑ । खलु॑ । इ॒माम् । दिश᳚म् । उदिति॑ । सृ॒ज॒ति॒ । अ॒सौ । वै । आ॒दि॒त्यः । प्रा॒ण इति॑ प्र - अ॒नः । प्रा॒णमिति॑ प्र - अ॒नम् । ए॒व । ए॒ना॒न् । अनु॑ । उदिति॑ । सृ॒ज॒ति॒ । द॒क्षि॒णा । प॒र्याव॑र्तन्त॒ इति॑ परि - आव॑र्तन्ते । स्वम् । ए॒व । वी॒र्य᳚म् । अन्विति॑ । प॒र्याव॑र्तन्त॒ इति॑ परि - आव॑र्तन्ते । तस्मा᳚त् । दक्षि॑णः । अद्‌र्धः॑ । आ॒त्मनः॑ । वी॒र्या॑वत्तर॒ इति॑ वी॒र्या॑वत् - त॒रः॒ । अथो॒ इति॑ । आ॒दि॒त्यस्य॑ । ए॒व । आ॒वृत॒मित्या᳚ - वृत᳚म् । अन्विति॑ । प॒र्याव॑र्तन्त॒ इति॑ परि-आव॑र्तन्ते । तस्मा᳚त् । परा᳚ञ्चः । प॒शवः॑ । वीति॑ । ति॒ष्ठ॒न्ते॒ । प्र॒त्यञ्चः॑ । एति॑ । व॒र्त॒न्ते॒ । ति॒स्रस्ति॑स्र॒ इति॑ ति॒स्रः - ति॒स्रः॒ । सीताः᳚ ।  \newline


\textbf{Krama Paata} \newline

तामे॒व । ए॒वैनान्॑ । ए॒ना॒ननु॑ । अनूत् । उथ् सृ॑जति । सृ॒ज॒त्यथो᳚ । अथो॒ खलु॑ । अथो॒ इत्यथो᳚ । खल्वि॒माम् । इ॒माम् दिश᳚म् । दिश॒मुत् । उथ् सृ॑जति । सृ॒ज॒त्य॒सौ । अ॒सौ वै । वा आ॑दि॒त्यः । आ॒दि॒त्यः प्रा॒णः । प्रा॒णः प्रा॒णम् । प्रा॒ण इति॑ प्र - अ॒नः । प्रा॒णमे॒व । प्रा॒णमिति॑ प्र - अ॒नम् । ए॒वैनान्॑ । ए॒ना॒ननु॑ । अनूत् । उथ् सृ॑जति । सृ॒ज॒ति॒ द॒क्षि॒णा । द॒क्षि॒णा प॒र्याव॑र्तन्ते । प॒र्याव॑र्तन्ते॒ स्वम् । प॒र्याव॑र्तन्त॒ इति॑ परि - आव॑र्तन्ते । स्वमे॒व । ए॒व वी॒र्य᳚म् । वी॒र्य॑मनु॑ । अनु॑ प॒र्याव॑र्तन्ते । प॒र्याव॑र्तन्ते॒ तस्मा᳚त् । प॒र्याव॑र्तन्त॒ इति॑ परि - आव॑र्तन्ते । तस्मा॒द् दक्षि॑णः । दक्षि॒णोऽर्द्धः॑ । अर्द्ध॑ आ॒त्मनः॑ । आ॒त्मनो॑ वी॒र्या॑वत्तरः । वी॒र्या॑वत्त॒रोऽथो᳚ । वी॒र्या॑वत्तर॒ इति॑ वी॒र्या॑वत् - त॒रः॒ । अथो॑ आदि॒त्यस्य॑ । अथो॒ इत्यथो᳚ । आ॒दि॒त्यस्यै॒व । ए॒वावृत᳚म् । आ॒वृत॒मनु॑ । आ॒वृत॒मित्या᳚ - वृत᳚म् । अनु॑प॒र्याव॑र्तन्ते । प॒र्याव॑र्तन्ते॒ तस्मा᳚त् । प॒र्याव॑र्तन्त॒ इति॑ परि - आव॑र्तन्ते । तस्मा॒त् परा᳚ञ्चः । परा᳚ञ्चः प॒शवः॑ । प॒शवो॒ वि । वि ति॑ष्ठन्ते । ति॒ष्ठ॒न्ते॒ प्र॒त्यञ्चः॑ । प्र॒त्यञ्च॒ आ । आ व॑र्तन्ते । व॒र्त॒न्ते॒ ति॒स्रस्ति॑स्रः । ति॒स्रस्ति॑स्रः॒ सीताः᳚ । ति॒स्रस्ति॑स्र॒ इति॑ ति॒स्रः - ति॒स्रः॒ । सीताः᳚ कृषति \newline

\textbf{Jatai Paata} \newline

1. ता मे॒वैव ताम् ता मे॒व । \newline
2. ए॒वैना॑ नेना ने॒वै वैनान्॑ । \newline
3. ए॒ना॒ नन् वन् वे॑ना नेना॒ ननु॑ । \newline
4. अनू दु दन् वनूत् । \newline
5. उथ् सृ॑जति सृज॒ त्युदुथ् सृ॑जति । \newline
6. सृ॒ज॒ त्यथो॒ अथो॑ सृजति सृज॒ त्यथो᳚ । \newline
7. अथो॒ खलु॒ खल्वथो॒ अथो॒ खलु॑ । \newline
8. अथो॒ इत्यथो᳚ । \newline
9. खल्वि॒मा मि॒माम् खलु॒ खल्वि॒माम् । \newline
10. इ॒माम् दिश॒म् दिश॑ मि॒मा मि॒माम् दिश᳚म् । \newline
11. दिश॒ मुदुद् दिश॒म् दिश॒ मुत् । \newline
12. उथ् सृ॑जति सृज॒ त्युदुथ् सृ॑जति । \newline
13. सृ॒ज॒ त्य॒सा व॒सौ सृ॑जति सृज त्य॒सौ । \newline
14. अ॒सौ वै वा अ॒सा व॒सौ वै । \newline
15. वा आ॑दि॒त्य आ॑दि॒त्यो वै वा आ॑दि॒त्यः । \newline
16. आ॒दि॒त्यः प्रा॒णः प्रा॒ण आ॑दि॒त्य आ॑दि॒त्यः प्रा॒णः । \newline
17. प्रा॒णः प्रा॒णम् प्रा॒णम् प्रा॒णः प्रा॒णः प्रा॒णम् । \newline
18. प्रा॒ण इति॑ प्र - अ॒नः । \newline
19. प्रा॒ण मे॒वैव प्रा॒णम् प्रा॒ण मे॒व । \newline
20. प्रा॒णमिति॑ प्र - अ॒नम् । \newline
21. ए॒वैना॑ नेना ने॒वै वैनान्॑ । \newline
22. ए॒ना॒ नन् वन् वे॑ना नेना॒ ननु॑ । \newline
23. अनू दुदन् वनूत् । \newline
24. उथ् सृ॑जति सृज॒ त्युदुथ् सृ॑जति । \newline
25. सृ॒ज॒ति॒ द॒क्षि॒णा द॑क्षि॒णा सृ॑जति सृजति दक्षि॒णा । \newline
26. द॒क्षि॒णा प॒र्याव॑र्तन्ते प॒र्याव॑र्तन्ते दक्षि॒णा द॑क्षि॒णा प॒र्याव॑र्तन्ते । \newline
27. प॒र्याव॑र्तन्ते॒ स्वꣳ स्वम् प॒र्याव॑र्तन्ते प॒र्याव॑र्तन्ते॒ स्वम् । \newline
28. प॒र्याव॑र्तन्त॒ इति॑ परि - आव॑र्तन्ते । \newline
29. स्व मे॒वैव स्वꣳ स्व मे॒व । \newline
30. ए॒व वी॒र्यं॑ ॅवी॒र्य॑ मे॒वैव वी॒र्य᳚म् । \newline
31. वी॒र्य॑ मन् वनु॑ वी॒र्यं॑ ॅवी॒र्य॑ मनु॑ । \newline
32. अनु॑ प॒र्याव॑र्तन्ते प॒र्याव॑र्त॒न्ते ऽन्वनु॑ प॒र्याव॑र्तन्ते । \newline
33. प॒र्याव॑र्तन्ते॒ तस्मा॒त् तस्मा᳚त् प॒र्याव॑र्तन्ते प॒र्याव॑र्तन्ते॒ तस्मा᳚त् । \newline
34. प॒र्याव॑र्तन्त॒ इति॑ परि - आव॑र्तन्ते । \newline
35. तस्मा॒द् दक्षि॑णो॒ दक्षि॑ण॒ स्तस्मा॒त् तस्मा॒द् दक्षि॑णः । \newline
36. दक्षि॒णो ऽर्द्धो ऽर्द्धो॒ दक्षि॑णो॒ दक्षि॒णो ऽर्द्धः॑ । \newline
37. अर्द्ध॑ आ॒त्मन॑ आ॒त्मनो ऽर्द्धो ऽर्द्ध॑ आ॒त्मनः॑ । \newline
38. आ॒त्मनो॑ वी॒र्या॑वत्तरो वी॒र्या॑वत्तर आ॒त्मन॑ आ॒त्मनो॑ वी॒र्या॑वत्तरः । \newline
39. वी॒र्या॑वत्त॒रो ऽथो॒ अथो॑ वी॒र्या॑वत्तरो वी॒र्या॑वत्त॒रो ऽथो᳚ । \newline
40. वी॒र्या॑वत्तर॒ इति॑ वी॒र्या॑वत् - त॒रः॒ । \newline
41. अथो॑ आदि॒त्यस्या॑ दि॒त्यस्याथो॒ अथो॑ आदि॒त्यस्य॑ । \newline
42. अथो॒ इत्यथो᳚ । \newline
43. आ॒दि॒त्य स्यै॒वैवादि॒त्य स्या॑दि॒त्य स्यै॒व । \newline
44. ए॒वावृत॑ मा॒वृत॑ मे॒वै वावृत᳚म् । \newline
45. आ॒वृत॒ मन् वन् वा॒वृत॑ मा॒वृत॒ मनु॑ । \newline
46. आ॒वृत॒मित्या᳚ - वृत᳚म् । \newline
47. अनु॑ प॒र्याव॑र्तन्ते प॒र्याव॑र्त॒न्ते ऽन्वनु॑ प॒र्याव॑र्तन्ते । \newline
48. प॒र्याव॑र्तन्ते॒ तस्मा॒त् तस्मा᳚त् प॒र्याव॑र्तन्ते प॒र्याव॑र्तन्ते॒ तस्मा᳚त् । \newline
49. प॒र्याव॑र्तन्त॒ इति॑ परि - आव॑र्तन्ते । \newline
50. तस्मा॒त् परा᳚ञ्चः॒ परा᳚ञ्च॒ स्तस्मा॒त् तस्मा॒त् परा᳚ञ्चः । \newline
51. परा᳚ञ्चः प॒शवः॑ प॒शवः॒ परा᳚ञ्चः॒ परा᳚ञ्चः प॒शवः॑ । \newline
52. प॒शवो॒ वि वि प॒शवः॑ प॒शवो॒ वि । \newline
53. वि ति॑ष्ठन्ते तिष्ठन्ते॒ वि वि ति॑ष्ठन्ते । \newline
54. ति॒ष्ठ॒न्ते॒ प्र॒त्यञ्चः॑ प्र॒त्यञ्च॑ स्तिष्ठन्ते तिष्ठन्ते प्र॒त्यञ्चः॑ । \newline
55. प्र॒त्यञ्च॒ आ प्र॒त्यञ्चः॑ प्र॒त्यञ्च॒ आ । \newline
56. आ व॑र्तन्ते वर्तन्त॒ आ व॑र्तन्ते । \newline
57. व॒र्त॒न्ते॒ ति॒स्रस्ति॑स्र स्ति॒स्रस्ति॑स्रो वर्तन्ते वर्तन्ते ति॒स्रस्ति॑स्रः । \newline
58. ति॒स्रस्ति॑स्रः॒ सीताः॒ सीता᳚ स्ति॒स्रस्ति॑स्र स्ति॒स्रस्ति॑स्रः॒ सीताः᳚ । \newline
59. ति॒स्रस्ति॑स्र॒ इति॑ ति॒स्रः - ति॒स्रः॒ । \newline
60. सीताः᳚ कृषति कृषति॒ सीताः॒ सीताः᳚ कृषति । \newline

\textbf{Ghana Paata } \newline

1. ता मे॒वैव ताम् ता मे॒वैना॑ नेना ने॒व ताम् ता मे॒वैनान्॑ । \newline
2. ए॒वैना॑ नेना ने॒वैवैना॒ नन्वन् वे॑ना ने॒वैवैना॒ ननु॑ । \newline
3. ए॒ना॒ नन्वन् वे॑ना नेना॒ ननूदु दन्वे॑ना नेना॒ ननूत् । \newline
4. अनूदु दन्वनूथ् सृ॑जति सृज॒ त्युदन्वनूथ् सृ॑जति । \newline
5. उथ् सृ॑जति सृज॒ त्युदुथ् सृ॑ज॒ त्यथो॒ अथो॑ सृज॒ त्युदुथ् सृ॑ज॒ त्यथो᳚ । \newline
6. सृ॒ज॒ त्यथो॒ अथो॑ सृजति सृज॒ त्यथो॒ खलु॒ खल्वथो॑ सृजति सृज॒ त्यथो॒ खलु॑ । \newline
7. अथो॒ खलु॒ खल्वथो॒ अथो॒ खल्वि॒मा मि॒माम् खल्वथो॒ अथो॒ खल्वि॒माम् । \newline
8. अथो॒ इत्यथो᳚ । \newline
9. खल्वि॒मा मि॒माम् खलु॒ खल्वि॒माम् दिश॒म् दिश॑ मि॒माम् खलु॒ खल्वि॒माम् दिश᳚म् । \newline
10. इ॒माम् दिश॒म् दिश॑ मि॒मा मि॒माम् दिश॒ मुदुद् दिश॑ मि॒मा मि॒माम् दिश॒ मुत् । \newline
11. दिश॒ मुदुद् दिश॒म् दिश॒ मुथ् सृ॑जति सृज॒ त्युद् दिश॒म् दिश॒ मुथ् सृ॑जति । \newline
12. उथ् सृ॑जति सृज॒ त्युदुथ् सृ॑ज त्य॒सा व॒सौ सृ॑ज॒ त्युदुथ् सृ॑ज त्य॒सौ । \newline
13. सृ॒ज॒ त्य॒सा व॒सौ सृ॑जति सृज त्य॒सौ वै वा अ॒सौ सृ॑जति सृज त्य॒सौ वै । \newline
14. अ॒सौ वै वा अ॒सा व॒सौ वा आ॑दि॒त्य आ॑दि॒त्यो वा अ॒सा व॒सौ वा आ॑दि॒त्यः । \newline
15. वा आ॑दि॒त्य आ॑दि॒त्यो वै वा आ॑दि॒त्यः प्रा॒णः प्रा॒ण आ॑दि॒त्यो वै वा आ॑दि॒त्यः प्रा॒णः । \newline
16. आ॒दि॒त्यः प्रा॒णः प्रा॒ण आ॑दि॒त्य आ॑दि॒त्यः प्रा॒णः प्रा॒णम् प्रा॒णम् प्रा॒ण आ॑दि॒त्य आ॑दि॒त्यः प्रा॒णः प्रा॒णम् । \newline
17. प्रा॒णः प्रा॒णम् प्रा॒णम् प्रा॒णः प्रा॒णः प्रा॒ण मे॒वैव प्रा॒णम् प्रा॒णः प्रा॒णः प्रा॒ण मे॒व । \newline
18. प्रा॒ण इति॑ प्र - अ॒नः । \newline
19. प्रा॒ण मे॒वैव प्रा॒णम् प्रा॒ण मे॒वैना॑ नेना ने॒व प्रा॒णम् प्रा॒ण मे॒वैनान्॑ । \newline
20. प्रा॒णमिति॑ प्र - अ॒नम् । \newline
21. ए॒वैना॑ नेना ने॒वैवैना॒ नन्वन् वे॑ना ने॒वैवैना॒ ननु॑ । \newline
22. ए॒ना॒ नन्वन् वे॑ना नेना॒ ननूदु दन्वे॑ना नेना॒ ननूत् । \newline
23. अनूदु दन्वनूथ् सृ॑जति सृज॒ त्युदन्वनूथ् सृ॑जति । \newline
24. उथ् सृ॑जति सृज॒ त्युदुथ् सृ॑जति दक्षि॒णा द॑क्षि॒णा सृ॑ज॒ त्युदुथ् सृ॑जति दक्षि॒णा । \newline
25. सृ॒ज॒ति॒ द॒क्षि॒णा द॑क्षि॒णा सृ॑जति सृजति दक्षि॒णा प॒र्याव॑र्तन्ते प॒र्याव॑र्तन्ते दक्षि॒णा सृ॑जति सृजति दक्षि॒णा प॒र्याव॑र्तन्ते । \newline
26. द॒क्षि॒णा प॒र्याव॑र्तन्ते प॒र्याव॑र्तन्ते दक्षि॒णा द॑क्षि॒णा प॒र्याव॑र्तन्ते॒ स्वꣳ स्वम् प॒र्याव॑र्तन्ते दक्षि॒णा द॑क्षि॒णा प॒र्याव॑र्तन्ते॒ स्वम् । \newline
27. प॒र्याव॑र्तन्ते॒ स्वꣳ स्वम् प॒र्याव॑र्तन्ते प॒र्याव॑र्तन्ते॒ स्व मे॒वैव स्वम् प॒र्याव॑र्तन्ते प॒र्याव॑र्तन्ते॒ स्व मे॒व । \newline
28. प॒र्याव॑र्तन्त॒ इति॑ परि - आव॑र्तन्ते । \newline
29. स्व मे॒वैव स्वꣳ स्व मे॒व वी॒र्यं॑ ॅवी॒र्य॑ मे॒व स्वꣳ स्व मे॒व वी॒र्य᳚म् । \newline
30. ए॒व वी॒र्यं॑ ॅवी॒र्य॑ मे॒वैव वी॒र्य॑ मन्वनु॑ वी॒र्य॑ मे॒वैव वी॒र्य॑ मनु॑ । \newline
31. वी॒र्य॑ मन्वनु॑ वी॒र्यं॑ ॅवी॒र्य॑ मनु॑ प॒र्याव॑र्तन्ते प॒र्याव॑र्त॒न्ते ऽनु॑ वी॒र्यं॑ ॅवी॒र्य॑ मनु॑ प॒र्याव॑र्तन्ते । \newline
32. अनु॑ प॒र्याव॑र्तन्ते प॒र्याव॑र्त॒न्ते ऽन्वनु॑ प॒र्याव॑र्तन्ते॒ तस्मा॒त् तस्मा᳚त् प॒र्याव॑र्त॒न्ते ऽन्वनु॑ प॒र्याव॑र्तन्ते॒ तस्मा᳚त् । \newline
33. प॒र्याव॑र्तन्ते॒ तस्मा॒त् तस्मा᳚त् प॒र्याव॑र्तन्ते प॒र्याव॑र्तन्ते॒ तस्मा॒द् दक्षि॑णो॒ दक्षि॑ण॒ स्तस्मा᳚त् प॒र्याव॑र्तन्ते प॒र्याव॑र्तन्ते॒ तस्मा॒द् दक्षि॑णः । \newline
34. प॒र्याव॑र्तन्त॒ इति॑ परि - आव॑र्तन्ते । \newline
35. तस्मा॒द् दक्षि॑णो॒ दक्षि॑ण॒ स्तस्मा॒त् तस्मा॒द् दक्षि॒णो ऽर्द्धो ऽर्द्धो॒ दक्षि॑ण॒ स्तस्मा॒त् तस्मा॒द् दक्षि॒णो ऽर्द्धः॑ । \newline
36. दक्षि॒णो ऽर्द्धो ऽर्द्धो॒ दक्षि॑णो॒ दक्षि॒णो ऽर्द्ध॑ आ॒त्मन॑ आ॒त्मनो ऽर्द्धो॒ दक्षि॑णो॒ दक्षि॒णो ऽर्द्ध॑ आ॒त्मनः॑ । \newline
37. अर्द्ध॑ आ॒त्मन॑ आ॒त्मनो ऽर्द्धो ऽर्द्ध॑ आ॒त्मनो॑ वी॒र्या॑वत्तरो वी॒र्या॑वत्तर आ॒त्मनो ऽर्द्धो ऽर्द्ध॑ आ॒त्मनो॑ वी॒र्या॑वत्तरः । \newline
38. आ॒त्मनो॑ वी॒र्या॑वत्तरो वी॒र्या॑वत्तर आ॒त्मन॑ आ॒त्मनो॑ वी॒र्या॑वत्त॒रो ऽथो॒ अथो॑ वी॒र्या॑वत्तर आ॒त्मन॑ आ॒त्मनो॑ वी॒र्या॑वत्त॒रो ऽथो᳚ । \newline
39. वी॒र्या॑वत्त॒रो ऽथो॒ अथो॑ वी॒र्या॑वत्तरो वी॒र्या॑वत्त॒रो ऽथो॑ आदि॒त्यस्या॑ दि॒त्यस्याथो॑ वी॒र्या॑वत्तरो वी॒र्या॑वत्त॒रो ऽथो॑ आदि॒त्यस्य॑ । \newline
40. वी॒र्या॑वत्तर॒ इति॑ वी॒र्या॑वत् - त॒रः॒ । \newline
41. अथो॑ आदि॒त्यस्या॑ दि॒त्यस्याथो॒ अथो॑ आदि॒त्य स्यै॒वैवादि॒ त्यस्याथो॒ अथो॑ आदि॒ त्यस्यै॒व । \newline
42. अथो॒ इत्यथो᳚ । \newline
43. आ॒दि॒त्य स्यै॒वैवादि॒त्य स्या॑दि॒त्य स्यै॒वावृत॑ मा॒वृत॑ मे॒वादि॒ त्यस्या॑दि॒त्य स्यै॒वावृत᳚म् । \newline
44. ए॒वावृत॑ मा॒वृत॑ मे॒वैवावृत॒ मन् वन् वा॒वृत॑ मे॒वैवावृत॒ मनु॑ । \newline
45. आ॒वृत॒ मन् वन् वा॒वृत॑ मा॒वृत॒ मनु॑ प॒र्याव॑र्तन्ते प॒र्याव॑र्त॒न्ते ऽन्वा॒वृत॑ मा॒वृत॒ मनु॑ प॒र्याव॑र्तन्ते । \newline
46. आ॒वृत॒मित्या᳚ - वृत᳚म् । \newline
47. अनु॑ प॒र्याव॑र्तन्ते प॒र्याव॑र्त॒न्ते ऽन्वनु॑ प॒र्याव॑र्तन्ते॒ तस्मा॒त् तस्मा᳚त् प॒र्याव॑र्त॒न्ते ऽन्वनु॑ प॒र्याव॑र्तन्ते॒ तस्मा᳚त् । \newline
48. प॒र्याव॑र्तन्ते॒ तस्मा॒त् तस्मा᳚त् प॒र्याव॑र्तन्ते प॒र्याव॑र्तन्ते॒ तस्मा॒त् परा᳚ञ्चः॒ परा᳚ञ्च॒ स्तस्मा᳚त् प॒र्याव॑र्तन्ते प॒र्याव॑र्तन्ते॒ तस्मा॒त् परा᳚ञ्चः । \newline
49. प॒र्याव॑र्तन्त॒ इति॑ परि - आव॑र्तन्ते । \newline
50. तस्मा॒त् परा᳚ञ्चः॒ परा᳚ञ्च॒ स्तस्मा॒त् तस्मा॒त् परा᳚ञ्चः प॒शवः॑ प॒शवः॒ परा᳚ञ्च॒ स्तस्मा॒त् तस्मा॒त् परा᳚ञ्चः प॒शवः॑ । \newline
51. परा᳚ञ्चः प॒शवः॑ प॒शवः॒ परा᳚ञ्चः॒ परा᳚ञ्चः प॒शवो॒ वि वि प॒शवः॒ परा᳚ञ्चः॒ परा᳚ञ्चः प॒शवो॒ वि । \newline
52. प॒शवो॒ वि वि प॒शवः॑ प॒शवो॒ वि ति॑ष्ठन्ते तिष्ठन्ते॒ वि प॒शवः॑ प॒शवो॒ वि ति॑ष्ठन्ते । \newline
53. वि ति॑ष्ठन्ते तिष्ठन्ते॒ वि वि ति॑ष्ठन्ते प्र॒त्यञ्चः॑ प्र॒त्यञ्च॑ स्तिष्ठन्ते॒ वि वि ति॑ष्ठन्ते प्र॒त्यञ्चः॑ । \newline
54. ति॒ष्ठ॒न्ते॒ प्र॒त्यञ्चः॑ प्र॒त्यञ्च॑ स्तिष्ठन्ते तिष्ठन्ते प्र॒त्यञ्च॒ आ प्र॒त्यञ्च॑ स्तिष्ठन्ते तिष्ठन्ते प्र॒त्यञ्च॒ आ । \newline
55. प्र॒त्यञ्च॒ आ प्र॒त्यञ्चः॑ प्र॒त्यञ्च॒ आ व॑र्तन्ते वर्तन्त॒ आ प्र॒त्यञ्चः॑ प्र॒त्यञ्च॒ आ व॑र्तन्ते । \newline
56. आ व॑र्तन्ते वर्तन्त॒ आ व॑र्तन्ते ति॒स्रस्ति॑स्र स्ति॒स्रस्ति॑स्रो वर्तन्त॒ आ व॑र्तन्ते ति॒स्रस्ति॑स्रः । \newline
57. व॒र्त॒न्ते॒ ति॒स्रस्ति॑स्र स्ति॒स्रस्ति॑स्रो वर्तन्ते वर्तन्ते ति॒स्र स्ति॑स्रः॒ सीताः॒ सीता᳚ स्ति॒स्रस्ति॑स्रो वर्तन्ते वर्तन्ते ति॒स्रस्ति॑स्रः॒ सीताः᳚ । \newline
58. ति॒स्रस्ति॑स्रः॒ सीताः॒ सीता᳚ स्ति॒स्रस्ति॑स्र स्ति॒स्रस्ति॑स्रः॒ सीताः᳚ कृषति कृषति॒ सीता᳚ स्ति॒स्रस्ति॑स्र स्ति॒स्रस्ति॑स्रः॒ सीताः᳚ कृषति । \newline
59. ति॒स्रस्ति॑स्र॒ इति॑ ति॒स्रः - ति॒स्रः॒ । \newline
60. सीताः᳚ कृषति कृषति॒ सीताः॒ सीताः᳚ कृषति त्रि॒वृत॑म् त्रि॒वृत॑म् कृषति॒ सीताः॒ सीताः᳚ कृषति त्रि॒वृत᳚म् । \newline
\pagebreak
\markright{ TS 5.2.5.5  \hfill https://www.vedavms.in \hfill}

\section{ TS 5.2.5.5 }

\textbf{TS 5.2.5.5 } \newline
\textbf{Samhita Paata} \newline

कृषति त्रि॒वृत॑मे॒व य॑ज्ञ्मु॒खे वि या॑तय॒त्योष॑धीर्वपति॒ ब्रह्म॒णाऽन्न॒मव॑ रुन्धे॒ ऽर्के᳚ऽर्कश्ची॑यते चतुर्द॒शभि॑र्वपति स॒प्त ग्रा॒म्या ओष॑धयः स॒प्ताऽऽर॒ण्या उ॒भयी॑षा॒मव॑रुद्ध्या॒ अन्न॑स्यान्नस्य वप॒त्यन्न॑स्या-न्न॒स्याव॑रुद्ध्यै कृ॒ष्टे व॑पति कृ॒ष्टे ह्योष॑धयः प्रति॒तिष्ठ॑न्त्यनुसी॒तं ॅव॑पति॒ प्रजा᳚त्यै द्वाद॒शसु॒ सीता॑सु वपति॒ द्वाद॑श॒ मासाः᳚ संॅवथ्स॒रः सं॑ॅवथ्स॒रेणै॒वास्मा॒ अन्नं॑ पचति॒ यद॑ग्नि॒चि - [  ] \newline

\textbf{Pada Paata} \newline

कृ॒ष॒ति॒ । त्रि॒वृत॒मिति॑ त्रि - वृत᳚म् । ए॒व । य॒ज्ञ्॒मु॒ख इति॑ यज्ञ्-मु॒खे । वीति॑ । या॒त॒य॒ति॒ । ओष॑धीः । व॒प॒ति॒ । ब्रह्म॑णा । अन्न᳚म् । अवेति॑ । रु॒न्धे॒ । अ॒र्के । अ॒र्कः । ची॒य॒ते॒ । च॒तु॒र्द॒शभि॒रिति॑ चतुर्द॒श - भिः॒ । व॒प॒ति॒ । स॒प्त । ग्रा॒म्याः । ओष॑धयः । स॒प्त । आ॒र॒ण्याः । उ॒भयी॑षाम् । अव॑रुद्ध्या॒ इत्यव॑ - रु॒द्ध्यै॒ । अन्न॑स्यान्न॒स्येत्यन्न॑स्य - अ॒न्न॒स्य॒ । व॒प॒ति॒ । अन्न॑स्यान्न॒स्येत्यन्न॑स्य-अ॒न्न॒स्य॒ । अव॑रुद्ध्या॒ इत्यव॑-रु॒द्ध्यै॒ । कृ॒ष्टे । व॒प॒ति॒ । कृ॒ष्टे । हि । ओष॑धयः । प्र॒ति॒तिष्ठ॒न्तीति॑ प्रति-तिष्ठ॑न्ति । अ॒नु॒सी॒तमित्य॑नु - सी॒तम् । व॒प॒ति॒ । प्रजा᳚त्या॒ इति॒ प्र - जा॒त्यै॒ । द्वा॒द॒शस्विति॑ द्वाद॒श - सु॒ । सीता॑सु । व॒प॒ति॒ । द्वाद॑श । मासाः᳚ । सं॒ॅव॒थ्स॒र इति॑ सं - व॒थ्स॒रः । सं॒ॅव॒थ्स॒रेणेति॑ सं - व॒थ्स॒रेण॑ । ए॒व । अ॒स्मै॒ । अन्न᳚म् । प॒च॒ति॒ । यत् । अ॒ग्नि॒चिदित्य॑ग्नि - चित् ।  \newline


\textbf{Krama Paata} \newline

कृ॒ष॒ति॒ त्रि॒वृत᳚म् । त्रि॒वृत॑मे॒व । त्रि॒वृत॒मिति॑ त्रि - वृत᳚म् । ए॒व य॑ज्ञ्मु॒खे । य॒ज्ञ्॒मु॒खे वि । य॒ज्ञ्॒मु॒ख इति॑ यज्ञ् - मु॒खे । वि या॑तयति । या॒त॒य॒त्योष॑धीः । ओष॑धीर् वपति । व॒प॒ति॒ ब्रह्म॑णा । ब्रह्म॒णाऽन्न᳚म् । अन्न॒मव॑ । अव॑ रुन्धे । रु॒न्धे॒ऽर्के । अ॒र्के᳚ऽर्कः । अ॒र्कश्ची॑यते । ची॒य॒ते॒ च॒तु॒र्द॒शभिः॑ । च॒तु॒र्द॒शभि॑र् वपति । च॒तु॒र्द॒शभि॒रिति॑ चतुर्द॒श - भिः॒ । व॒प॒ति॒ स॒प्त । स॒प्त ग्रा॒म्याः । ग्रा॒म्या ओष॑धयः । ओष॑धयः स॒प्त । स॒प्तार॒ण्याः । आ॒र॒ण्या उ॒भयी॑षाम् । उ॒भयी॑षा॒मव॑रुद्ध्यै । अव॑रुद्ध्या॒ अन्न॑स्यान्नस्य । अव॑रुद्ध्या॒ इत्यव॑ - रु॒द्ध्यै॒ । अन्न॑स्यान्नस्य वपति । अन्न॑स्यान्न॒स्येत्यन्न॑स्य - अ॒न्न॒स्य॒ । व॒प॒त्यन्न॑स्यान्नस्य । अन्न॑स्यान्न॒स्याव॑रुद्ध्यै । अन्न॑स्यान्न॒स्येत्यन्न॑स्य - अ॒न्न॒स्य॒ । अव॑रुद्ध्यै कृ॒ष्टे । अव॑रुद्ध्या॒ इत्यव॑ - रु॒द्ध्यै॒ । कृ॒ष्टे व॑पति । व॒प॒ति॒ कृ॒ष्टे । कृ॒ष्टे हि । ह्योष॑धयः । ओष॑धयः प्रति॒तिष्ठ॑न्ति । प्र॒ति॒तिष्ठ॑न्त्यनुसी॒तम् । प्र॒ति॒तिष्ठ॒न्तीति॑ प्रति - तिष्ठ॑न्ति । अ॒नु॒सी॒तम् ॅव॑पति । अ॒नु॒सी॒तमित्य॑नु - सी॒तम् । व॒प॒ति॒ प्रजा᳚त्यै । प्रजा᳚त्यै द्वाद॒शसु॑ । प्रजा᳚त्या॒ इति॒ प्र - जा॒त्यै॒ । द्वा॒द॒शसु॒ सीता॑सु । द्वा॒द॒शस्विति॑ द्वाद॒श - सु॒ । सीता॑सु वपति । व॒प॒ति॒ द्वाद॑श । द्वाद॑श॒ मासाः᳚ । मासाः᳚ सम्ॅवथ्स॒रः । स॒म्ॅव॒थ्स॒रः स॑म्ॅवथ्स॒रेण॑ । स॒म्ॅव॒थ्स॒र इति॑ सम् - व॒थ्स॒रः । स॒म्ॅव॒थ्स॒रेणै॒व । स॒म्ॅव॒थ्स॒रेणेति॑ सम् - व॒थ्स॒रेण॑ । ए॒वास्मै᳚ । अ॒स्मा॒ अन्न᳚म् । अन्न॑म् पचति । प॒च॒ति॒ यत् । यद॑ग्नि॒चित् । अ॒ग्नि॒चिदन॑वरुद्धस्य । अ॒ग्नि॒चिदित्य॑ग्नि - चित् \newline

\textbf{Jatai Paata} \newline

1. कृ॒ष॒ति॒ त्रि॒वृत॑म् त्रि॒वृत॑म् कृषति कृषति त्रि॒वृत᳚म् । \newline
2. त्रि॒वृत॑ मे॒वैव त्रि॒वृत॑म् त्रि॒वृत॑ मे॒व । \newline
3. त्रि॒वृत॒मिति॑ त्रि - वृत᳚म् । \newline
4. ए॒व य॑ज्ञ्मु॒खे य॑ज्ञ्मु॒ख ए॒वैव य॑ज्ञ्मु॒खे । \newline
5. य॒ज्ञ्॒मु॒खे वि वि य॑ज्ञ्मु॒खे य॑ज्ञ्मु॒खे वि । \newline
6. य॒ज्ञ्॒मु॒ख इति॑ यज्ञ् - मु॒खे । \newline
7. वि या॑तयति यातयति॒ वि वि या॑तयति । \newline
8. या॒त॒य॒ त्योष॑धी॒ रोष॑धीर् यातयति यातय॒ त्योष॑धीः । \newline
9. ओष॑धीर् वपति वप॒ त्योष॑धी॒ रोष॑धीर् वपति । \newline
10. व॒प॒ति॒ ब्रह्म॑णा॒ ब्रह्म॑णा वपति वपति॒ ब्रह्म॑णा । \newline
11. ब्रह्म॒णा ऽन्न॒ मन्न॒म् ब्रह्म॑णा॒ ब्रह्म॒णा ऽन्न᳚म् । \newline
12. अन्न॒ मवा वान्न॒ मन्न॒ मव॑ । \newline
13. अव॑ रुन्धे रु॒न्धे ऽवाव॑ रुन्धे । \newline
14. रु॒न्धे॒ ऽर्के᳚ ऽर्के रु॑न्धे रुन्धे॒ ऽर्के । \newline
15. अ॒र्के᳚(1॒) ऽर्को᳚(1॒) ऽर्को᳚(1॒) ऽर्के᳚(1॒) ऽर्के᳚ ऽर्कः । \newline
16. अ॒र्क श्ची॑यते चीयते॒ ऽर्को᳚ ऽर्क श्ची॑यते । \newline
17. ची॒य॒ते॒ च॒तु॒र्द॒शभि॑ श्चतुर्द॒शभि॑ श्चीयते चीयते चतुर्द॒शभिः॑ । \newline
18. च॒तु॒र्द॒शभि॑र् वपति वपति चतुर्द॒शभि॑ श्चतुर्द॒शभि॑र् वपति । \newline
19. च॒तु॒र्द॒शभि॒रिति॑ चतुर्द॒श - भिः॒ । \newline
20. व॒प॒ति॒ स॒प्त स॒प्त व॑पति वपति स॒प्त । \newline
21. स॒प्त ग्रा॒म्या ग्रा॒म्याः स॒प्त स॒प्त ग्रा॒म्याः । \newline
22. ग्रा॒म्या ओष॑धय॒ ओष॑धयो ग्रा॒म्या ग्रा॒म्या ओष॑धयः । \newline
23. ओष॑धयः स॒प्त स॒प्तौष॑धय॒ ओष॑धयः स॒प्त । \newline
24. स॒प्तार॒ण्या आ॑र॒ण्याः स॒प्त स॒प्तार॒ण्याः । \newline
25. आ॒र॒ण्या उ॒भयी॑षा मु॒भयी॑षा मार॒ण्या आ॑र॒ण्या उ॒भयी॑षाम् । \newline
26. उ॒भयी॑षा॒ मव॑रुद्ध्या॒ अव॑रुद्ध्या उ॒भयी॑षा मु॒भयी॑षा॒ मव॑रुद्ध्यै । \newline
27. अव॑रुद्ध्या॒ अन्न॑स्यान्न॒स्या न्न॑स्यान्न॒स्या व॑रुद्ध्या॒ अव॑रुद्ध्या॒ अन्न॑स्यान्नस्य । \newline
28. अव॑रुद्ध्या॒ इत्यव॑ - रु॒द्ध्यै॒ । \newline
29. अन्न॑स्यान्नस्य वपति वप॒ त्यन्न॑स्यान्न॒स्या न्न॑स्यान्नस्य वपति । \newline
30. अन्न॑स्यान्न॒स्येत्यन्न॑स्य - अ॒न्न॒स्य॒ । \newline
31. व॒प॒ त्यन्न॑स्यान्न॒स्या न्न॑स्यान्नस्य वपति वप॒ त्यन्न॑स्यान्नस्य । \newline
32. अन्न॑स्यान्न॒स्या व॑रुद्ध्या॒ अव॑रुद्ध्या॒ अन्न॑स्यान्न॒स्या न्न॑स्यान्न॒स्या व॑रुद्ध्यै । \newline
33. अन्न॑स्यान्न॒स्येत्यन्न॑स्य - अ॒न्न॒स्य॒ । \newline
34. अव॑रुद्ध्यै कृ॒ष्टे कृ॒ष्टे ऽव॑रुद्ध्या॒ अव॑रुद्ध्यै कृ॒ष्टे । \newline
35. अव॑रुद्ध्या॒ इत्यव॑ - रु॒द्ध्यै॒ । \newline
36. कृ॒ष्टे व॑पति वपति कृ॒ष्टे कृ॒ष्टे व॑पति । \newline
37. व॒प॒ति॒ कृ॒ष्टे कृ॒ष्टे व॑पति वपति कृ॒ष्टे । \newline
38. कृ॒ष्टे हि हि कृ॒ष्टे कृ॒ष्टे हि । \newline
39. ह्योष॑धय॒ ओष॑धयो॒ हि ह्योष॑धयः । \newline
40. ओष॑धयः प्रति॒तिष्ठ॑न्ति प्रति॒तिष्ठ॒न् त्योष॑धय॒ ओष॑धयः प्रति॒तिष्ठ॑न्ति । \newline
41. प्र॒ति॒तिष्ठ॑न् त्यनुसी॒त म॑नुसी॒तम् प्र॑ति॒तिष्ठ॑न्ति प्रति॒तिष्ठ॑न् त्यनुसी॒तम् । \newline
42. प्र॒ति॒तिष्ठ॒न्तीति॑ प्रति - तिष्ठ॑न्ति । \newline
43. अ॒नु॒सी॒तं ॅव॑पति वप त्यनुसी॒त म॑नुसी॒तं ॅव॑पति । \newline
44. अ॒नु॒सी॒तमित्य॑नु - सी॒तम् । \newline
45. व॒प॒ति॒ प्रजा᳚त्यै॒ प्रजा᳚त्यै वपति वपति॒ प्रजा᳚त्यै । \newline
46. प्रजा᳚त्यै द्वाद॒शसु॑ द्वाद॒शसु॒ प्रजा᳚त्यै॒ प्रजा᳚त्यै द्वाद॒शसु॑ । \newline
47. प्रजा᳚त्या॒ इति॒ प्र - जा॒त्यै॒ । \newline
48. द्वा॒द॒शसु॒ सीता॑सु॒ सीता॑सु द्वाद॒शसु॑ द्वाद॒शसु॒ सीता॑सु । \newline
49. द्वा॒द॒शस्विति॑ द्वाद॒श - सु॒ । \newline
50. सीता॑सु वपति वपति॒ सीता॑सु॒ सीता॑सु वपति । \newline
51. व॒प॒ति॒ द्वाद॑श॒ द्वाद॑श वपति वपति॒ द्वाद॑श । \newline
52. द्वाद॑श॒ मासा॒ मासा॒ द्वाद॑श॒ द्वाद॑श॒ मासाः᳚ । \newline
53. मासाः᳚ संॅवथ्स॒रः सं॑ॅवथ्स॒रो मासा॒ मासाः᳚ संॅवथ्स॒रः । \newline
54. सं॒ॅव॒थ्स॒रः सं॑ॅवथ्स॒रेण॑ संॅवथ्स॒रेण॑ संॅवथ्स॒रः सं॑ॅवथ्स॒रः सं॑ॅवथ्स॒रेण॑ । \newline
55. सं॒ॅव॒थ्स॒र इति॑ सं - व॒थ्स॒रः । \newline
56. सं॒ॅव॒थ्स॒रे णै॒वैव सं॑ॅवथ्स॒रेण॑ संॅवथ्स॒रे णै॒व । \newline
57. सं॒ॅव॒थ्स॒रेणेति॑ सं - व॒थ्स॒रेण॑ । \newline
58. ए॒वास्मा॑ अस्मा ए॒वैवास्मै᳚ । \newline
59. अ॒स्मा॒ अन्न॒ मन्न॑ मस्मा अस्मा॒ अन्न᳚म् । \newline
60. अन्न॑म् पचति पच॒ त्यन्न॒ मन्न॑म् पचति । \newline
61. प॒च॒ति॒ यद् यत् प॑चति पचति॒ यत् । \newline
62. यद॑ग्नि॒चि द॑ग्नि॒चिद् यद् यद॑ग्नि॒चित् । \newline
63. अ॒ग्नि॒चि दन॑वरुद्ध॒स्या न॑वरुद्धस्या ग्नि॒चि द॑ग्नि॒चि दन॑वरुद्धस्य । \newline
64. अ॒ग्नि॒चिदित्य॑ग्नि - चित् । \newline

\textbf{Ghana Paata } \newline

1. कृ॒ष॒ति॒ त्रि॒वृत॑म् त्रि॒वृत॑म् कृषति कृषति त्रि॒वृत॑ मे॒वैव त्रि॒वृत॑म् कृषति कृषति त्रि॒वृत॑ मे॒व । \newline
2. त्रि॒वृत॑ मे॒वैव त्रि॒वृत॑म् त्रि॒वृत॑ मे॒व य॑ज्ञ्मु॒खे य॑ज्ञ्मु॒ख ए॒व त्रि॒वृत॑म् त्रि॒वृत॑ मे॒व य॑ज्ञ्मु॒खे । \newline
3. त्रि॒वृत॒मिति॑ त्रि - वृत᳚म् । \newline
4. ए॒व य॑ज्ञ्मु॒खे य॑ज्ञ्मु॒ख ए॒वैव य॑ज्ञ्मु॒खे वि वि य॑ज्ञ्मु॒ख ए॒वैव य॑ज्ञ्मु॒खे वि । \newline
5. य॒ज्ञ्॒मु॒खे वि वि य॑ज्ञ्मु॒खे य॑ज्ञ्मु॒खे वि या॑तयति यातयति॒ वि य॑ज्ञ्मु॒खे य॑ज्ञ्मु॒खे वि या॑तयति । \newline
6. य॒ज्ञ्॒मु॒ख इति॑ यज्ञ् - मु॒खे । \newline
7. वि या॑तयति यातयति॒ वि वि या॑तय॒ त्योष॑धी॒ रोष॑धीर् यातयति॒ वि वि या॑तय॒ त्योष॑धीः । \newline
8. या॒त॒य॒ त्योष॑धी॒ रोष॑धीर् यातयति यातय॒ त्योष॑धीर् वपति वप॒ त्योष॑धीर् यातयति यातय॒ त्योष॑धीर् वपति । \newline
9. ओष॑धीर् वपति वप॒ त्योष॑धी॒ रोष॑धीर् वपति॒ ब्रह्म॑णा॒ ब्रह्म॑णा वप॒ त्योष॑धी॒ रोष॑धीर् वपति॒ ब्रह्म॑णा । \newline
10. व॒प॒ति॒ ब्रह्म॑णा॒ ब्रह्म॑णा वपति वपति॒ ब्रह्म॒णा ऽन्न॒ मन्न॒म् ब्रह्म॑णा वपति वपति॒ ब्रह्म॒णा ऽन्न᳚म् । \newline
11. ब्रह्म॒णा ऽन्न॒ मन्न॒म् ब्रह्म॑णा॒ ब्रह्म॒णा ऽन्न॒ मवावा न्न॒म् ब्रह्म॑णा॒ ब्रह्म॒णा ऽन्न॒ मव॑ । \newline
12. अन्न॒ मवावा न्न॒ मन्न॒ मव॑ रुन्धे रु॒न्धे ऽवान्न॒ मन्न॒ मव॑ रुन्धे । \newline
13. अव॑ रुन्धे रु॒न्धे ऽवाव॑ रुन्धे॒ ऽर्के᳚ ऽर्के रु॒न्धे ऽवाव॑ रुन्धे॒ ऽर्के । \newline
14. रु॒न्धे॒ ऽर्के᳚ ऽर्के रु॑न्धे रुन्धे॒ ऽर्के᳚(1॒) ऽर्को᳚(1॒) ऽर्को᳚ ऽर्के रु॑न्धे रुन्धे॒ ऽर्के᳚ ऽर्कः । \newline
15. अ॒र्के᳚(1॒) ऽर्को᳚(1॒) ऽर्को᳚(1॒) ऽर्के᳚(1॒) ऽर्के᳚ ऽर्क श्ची॑यते चीयते॒ ऽर्को᳚(1॒) ऽर्के᳚(1॒) ऽर्के᳚ ऽर्क श्ची॑यते । \newline
16. अ॒र्क श्ची॑यते चीयते॒ ऽर्को᳚ ऽर्कश्ची॑यते चतुर्द॒शभि॑ श्चतुर्द॒शभि॑ श्चीयते॒ ऽर्को᳚ ऽर्कश्ची॑यते चतुर्द॒शभिः॑ । \newline
17. ची॒य॒ते॒ च॒तु॒र्द॒शभि॑ श्चतुर्द॒शभि॑ श्चीयते चीयते चतुर्द॒शभि॑र् वपति वपति चतुर्द॒शभि॑श्चीयते चीयते चतुर्द॒शभि॑र् वपति । \newline
18. च॒तु॒र्द॒शभि॑र् वपति वपति चतुर्द॒शभि॑ श्चतुर्द॒शभि॑र् वपति स॒प्त स॒प्त व॑पति चतुर्द॒शभि॑ श्चतुर्द॒शभि॑र् वपति स॒प्त । \newline
19. च॒तु॒र्द॒शभि॒रिति॑ चतुर्द॒श - भिः॒ । \newline
20. व॒प॒ति॒ स॒प्त स॒प्त व॑पति वपति स॒प्त ग्रा॒म्या ग्रा॒म्याः स॒प्त व॑पति वपति स॒प्त ग्रा॒म्याः । \newline
21. स॒प्त ग्रा॒म्या ग्रा॒म्याः स॒प्त स॒प्त ग्रा॒म्या ओष॑धय॒ ओष॑धयो ग्रा॒म्याः स॒प्त स॒प्त ग्रा॒म्या ओष॑धयः । \newline
22. ग्रा॒म्या ओष॑धय॒ ओष॑धयो ग्रा॒म्या ग्रा॒म्या ओष॑धयः स॒प्त स॒प्तौष॑धयो ग्रा॒म्या ग्रा॒म्या ओष॑धयः स॒प्त । \newline
23. ओष॑धयः स॒प्त स॒प्तौष॑धय॒ ओष॑धयः स॒प्तार॒ण्या आ॑र॒ण्याः स॒प्तौष॑धय॒ ओष॑धयः स॒प्तार॒ण्याः । \newline
24. स॒प्तार॒ण्या आ॑र॒ण्याः स॒प्त स॒प्तार॒ण्या उ॒भयी॑षा मु॒भयी॑षा मार॒ण्याः स॒प्त स॒प्तार॒ण्या उ॒भयी॑षाम् । \newline
25. आ॒र॒ण्या उ॒भयी॑षा मु॒भयी॑षा मार॒ण्या आ॑र॒ण्या उ॒भयी॑षा॒ मव॑रुद्ध्या॒ अव॑रुद्ध्या उ॒भयी॑षा मार॒ण्या आ॑र॒ण्या उ॒भयी॑षा॒ मव॑रुद्ध्यै । \newline
26. उ॒भयी॑षा॒ मव॑रुद्ध्या॒ अव॑रुद्ध्या उ॒भयी॑षा मु॒भयी॑षा॒ मव॑रुद्ध्या॒ 
अन्न॑स्यान्न॒स्या न्न॑स्यान्न॒स्या व॑रुद्ध्या उ॒भयी॑षा मु॒भयी॑षा॒ मव॑रुद्ध्या॒ अन्न॑स्यान्नस्य । \newline
27. अव॑रुद्ध्या॒ अन्न॑स्यान्न॒स्या न्न॑स्यान्न॒स्या व॑रुद्ध्या॒ अव॑रुद्ध्या॒ अन्न॑स्यान्नस्य वपति वप॒त्यन्न॑ स्यान्न॒स्या व॑रुद्ध्या॒ अव॑रुद्ध्या॒ अन्न॑स्यान्नस्य वपति । \newline
28. अव॑रुद्ध्या॒ इत्यव॑ - रु॒द्ध्यै॒ । \newline
29. अन्न॑स्यान्नस्य वपति वप॒ त्यन्न॑स्यान्न॒स्या न्न॑स्यान्नस्य वप॒ त्यन्न॑स्यान्न॒स्या न्न॑स्यान्नस्य वप॒ त्यन्न॑स्यान्न॒ स्यान्न॑स्यान्नस्य वप॒ त्यन्न॑स्यान्नस्य । \newline
30. अन्न॑स्यान्न॒स्येत्यन्न॑स्य - अ॒न्न॒स्य॒ । \newline
31. व॒प॒ त्यन्न॑स्या न्न॒स्यान्न॑स्यान्नस्य वपति वप॒ त्यन्न॑स्यान्न॒स्या व॑रुद्ध्या॒ अव॑रुद्ध्या॒ अन्न॑स्यान्नस्य वपति वप॒ त्यन्न॑स्यान्न॒स्या व॑रुद्ध्यै । \newline
32. अन्न॑स्यान्न॒स्या व॑रुद्ध्या॒ अव॑रुद्ध्या॒ अन्न॑स्यान्न॒स्या न्न॑स्यान्न॒स्या व॑रुद्ध्यै कृ॒ष्टे कृ॒ष्टे ऽव॑रुद्ध्या॒ अन्न॑स्यान्न॒स्या न्न॑स्यान्न॒स्या व॑रुद्ध्यै कृ॒ष्टे । \newline
33. अन्न॑स्यान्न॒स्येत्यन्न॑स्य - अ॒न्न॒स्य॒ । \newline
34. अव॑रुद्ध्यै कृ॒ष्टे कृ॒ष्टे ऽव॑रुद्ध्या॒ अव॑रुद्ध्यै कृ॒ष्टे व॑पति वपति कृ॒ष्टे ऽव॑रुद्ध्या॒ अव॑रुद्ध्यै कृ॒ष्टे व॑पति । \newline
35. अव॑रुद्ध्या॒ इत्यव॑ - रु॒द्ध्यै॒ । \newline
36. कृ॒ष्टे व॑पति वपति कृ॒ष्टे कृ॒ष्टे व॑पति कृ॒ष्टे कृ॒ष्टे व॑पति कृ॒ष्टे कृ॒ष्टे व॑पति कृ॒ष्टे । \newline
37. व॒प॒ति॒ कृ॒ष्टे कृ॒ष्टे व॑पति वपति कृ॒ष्टे हि हि कृ॒ष्टे व॑पति वपति कृ॒ष्टे हि । \newline
38. कृ॒ष्टे हि हि कृ॒ष्टे कृ॒ष्टे ह्योष॑धय॒ ओष॑धयो॒ हि कृ॒ष्टे कृ॒ष्टे ह्योष॑धयः । \newline
39. ह्योष॑धय॒ ओष॑धयो॒ हि ह्योष॑धयः प्रति॒तिष्ठ॑न्ति प्रति॒तिष्ठ॒न् त्योष॑धयो॒ हि ह्योष॑धयः प्रति॒तिष्ठ॑न्ति । \newline
40. ओष॑धयः प्रति॒तिष्ठ॑न्ति प्रति॒तिष्ठ॒न् त्योष॑धय॒ ओष॑धयः प्रति॒तिष्ठ॑न् त्यनुसी॒त म॑नुसी॒तम् प्र॑ति॒तिष्ठ॒न् त्योष॑धय॒ ओष॑धयः प्रति॒तिष्ठ॑न् त्यनुसी॒तम् । \newline
41. प्र॒ति॒तिष्ठ॑न् त्यनुसी॒त म॑नुसी॒तम् प्र॑ति॒तिष्ठ॑न्ति प्रति॒तिष्ठ॑न् त्यनुसी॒तं ॅव॑पति वप त्यनुसी॒तम् प्र॑ति॒तिष्ठ॑न्ति प्रति॒तिष्ठ॑न् त्यनुसी॒तं ॅव॑पति । \newline
42. प्र॒ति॒तिष्ठ॒न्तीति॑ प्रति - तिष्ठ॑न्ति । \newline
43. अ॒नु॒सी॒तं ॅव॑पति वप त्यनुसी॒त म॑नुसी॒तं ॅव॑पति॒ प्रजा᳚त्यै॒ प्रजा᳚त्यै वप त्यनुसी॒त म॑नुसी॒तं ॅव॑पति॒ प्रजा᳚त्यै । \newline
44. अ॒नु॒सी॒तमित्य॑नु - सी॒तम् । \newline
45. व॒प॒ति॒ प्रजा᳚त्यै॒ प्रजा᳚त्यै वपति वपति॒ प्रजा᳚त्यै द्वाद॒शसु॑ द्वाद॒शसु॒ प्रजा᳚त्यै वपति वपति॒ प्रजा᳚त्यै द्वाद॒शसु॑ । \newline
46. प्रजा᳚त्यै द्वाद॒शसु॑ द्वाद॒शसु॒ प्रजा᳚त्यै॒ प्रजा᳚त्यै द्वाद॒शसु॒ सीता॑सु॒ सीता॑सु द्वाद॒शसु॒ प्रजा᳚त्यै॒ प्रजा᳚त्यै द्वाद॒शसु॒ सीता॑सु । \newline
47. प्रजा᳚त्या॒ इति॒ प्र - जा॒त्यै॒ । \newline
48. द्वा॒द॒शसु॒ सीता॑सु॒ सीता॑सु द्वाद॒शसु॑ द्वाद॒शसु॒ सीता॑सु वपति वपति॒ सीता॑सु द्वाद॒शसु॑ द्वाद॒शसु॒ सीता॑सु वपति । \newline
49. द्वा॒द॒शस्विति॑ द्वाद॒श - सु॒ । \newline
50. सीता॑सु वपति वपति॒ सीता॑सु॒ सीता॑सु वपति॒ द्वाद॑श॒ द्वाद॑श वपति॒ सीता॑सु॒ सीता॑सु वपति॒ द्वाद॑श । \newline
51. व॒प॒ति॒ द्वाद॑श॒ द्वाद॑श वपति वपति॒ द्वाद॑श॒ मासा॒ मासा॒ द्वाद॑श वपति वपति॒ द्वाद॑श॒ मासाः᳚ । \newline
52. द्वाद॑श॒ मासा॒ मासा॒ द्वाद॑श॒ द्वाद॑श॒ मासाः᳚ संॅवथ्स॒रः सं॑ॅवथ्स॒रो मासा॒ द्वाद॑श॒ द्वाद॑श॒ मासाः᳚ संॅवथ्स॒रः । \newline
53. मासाः᳚ संॅवथ्स॒रः सं॑ॅवथ्स॒रो मासा॒ मासाः᳚ संॅवथ्स॒रः सं॑ॅवथ्स॒रेण॑ संॅवथ्स॒रेण॑ संॅवथ्स॒रो मासा॒ मासाः᳚ संॅवथ्स॒रः सं॑ॅवथ्स॒रेण॑ । \newline
54. सं॒ॅव॒थ्स॒रः सं॑ॅवथ्स॒रेण॑ संॅवथ्स॒रेण॑ संॅवथ्स॒रः सं॑ॅवथ्स॒रः सं॑ॅवथ्स॒रे णै॒वैव सं॑ॅवथ्स॒रेण॑ संॅवथ्स॒रः सं॑ॅवथ्स॒रः सं॑ॅवथ्स॒रेणै॒व । \newline
55. सं॒ॅव॒थ्स॒र इति॑ सं - व॒थ्स॒रः । \newline
56. सं॒ॅव॒थ्स॒रे णै॒वैव सं॑ॅवथ्स॒रेण॑ संॅवथ्स॒रे णै॒वास्मा॑ अस्मा ए॒व सं॑ॅवथ्स॒रेण॑ संॅवथ्स॒रे णै॒वास्मै᳚ । \newline
57. सं॒ॅव॒थ्स॒रेणेति॑ सं - व॒थ्स॒रेण॑ । \newline
58. ए॒वास्मा॑ अस्मा ए॒वैवास्मा॒ अन्न॒ मन्न॑ मस्मा ए॒वैवास्मा॒ अन्न᳚म् । \newline
59. अ॒स्मा॒ अन्न॒ मन्न॑ मस्मा अस्मा॒ अन्न॑म् पचति पच॒त्यन्न॑ मस्मा अस्मा॒ अन्न॑म् पचति । \newline
60. अन्न॑म् पचति पच॒त्यन्न॒ मन्न॑म् पचति॒ यद् यत् प॑च॒त्यन्न॒ मन्न॑म् पचति॒ यत् । \newline
61. प॒च॒ति॒ यद् यत् प॑चति पचति॒ यद॑ग्नि॒चि द॑ग्नि॒चिद् यत् प॑चति पचति॒ यद॑ग्नि॒चित् । \newline
62. यद॑ग्नि॒चि द॑ग्नि॒चिद् यद् यद॑ग्नि॒चि दन॑वरुद्ध॒स्या न॑वरुद्धस्या ग्नि॒चिद् यद् यद॑ग्नि॒चि दन॑वरुद्धस्य । \newline
63. अ॒ग्नि॒चि दन॑वरुद्ध॒स्या न॑वरुद्धस्या ग्नि॒चि द॑ग्नि॒चि दन॑वरुद्धस्या श्ञी॒या द॑श्ञी॒या दन॑वरुद्धस्या ग्नि॒चि द॑ग्नि॒चि दन॑वरुद्धस्या श्ञी॒यात् । \newline
64. अ॒ग्नि॒चिदित्य॑ग्नि - चित् । \newline
\pagebreak
\markright{ TS 5.2.5.6  \hfill https://www.vedavms.in \hfill}

\section{ TS 5.2.5.6 }

\textbf{TS 5.2.5.6 } \newline
\textbf{Samhita Paata} \newline

-दन॑वरुद्धस्या-श्नी॒यादव॑-रुद्धेन॒ व्यृ॑द्ध्येत॒ ये वन॒स्पती॑नां फल॒ग्रह॑य॒-स्तानि॒द्ध्मेऽपि॒ प्रोक्षे॒-दन॑वरुद्ध॒स्या-व॑रुद्ध्यै दि॒ग्भ्यो लो॒ष्टान्थ् सम॑स्यति दि॒शामे॒व वी॒र्य॑मव॒रुद्ध्य॑ दि॒शां ॅवी॒र्ये᳚ऽग्निं चि॑नुते॒ यं द्वि॒ष्याद्-यत्र॒ स स्यात् तस्यै॑ दि॒शो लो॒ष्टमा ह॑रे॒दिष॒-मूर्ज॑म॒हमि॒त आ द॑द॒ इतीष॑मे॒वोर्जं॒ तस्यै॑ दि॒शोऽव॑ ( ) रुन्धे॒ क्षोधु॑को भवति॒ यस्तस्यां᳚ दि॒शि भव॑त्युत्तरवे॒दिमुप॑ वपत्युत्तरवे॒द्याꣳ ह्य॑ग्निश्ची॒यते ऽथो॑ प॒शवो॒ वा उ॑त्तरवे॒दिः प॒शूने॒वाव॑ रु॒न्धेऽथो॑ यज्ञ्प॒रुषोऽन॑न्तरित्यै ॥ \newline

\textbf{Pada Paata} \newline

अन॑वरुद्ध॒स्येत्यन॑व-रु॒द्ध॒स्य॒ । अ॒श्नी॒यात् । अव॑रुद्धे॒नेत्यव॑ - रु॒द्धे॒न॒ । वीति॑ । ऋ॒द्ध्ये॒त॒ । ये । वन॒स्पती॑नाम् । फ॒ल॒ग्रह॑य॒ इति॑ फल-ग्रह॑यः । तान् । इ॒द्ध्मे । अपि॑ । प्रेति॑ । उ॒क्षे॒त् । अन॑वरुद्ध॒स्येत्यन॑व-रु॒द्ध॒स्य॒ । अव॑रुद्ध्या॒ इत्यव॑ - रु॒द्ध्यै॒ । दि॒ग्भ्य इति॑ दिक् - भ्यः । लो॒ष्टान् । समिति॑ । अ॒स्य॒ति॒ । दि॒शाम् । ए॒व । वी॒र्य᳚म् । अ॒व॒रुद्ध्येत्य॑व-रुद्ध्य॑ । दि॒शाम् । वी॒र्ये᳚ । अ॒ग्निम् । चि॒नु॒ते॒ । यम् । द्वि॒ष्यात् । यत्र॑ । सः । स्यात् । तस्यै᳚ । दि॒शः । लो॒ष्टम् । एति॑ । ह॒रे॒त् । इष᳚म् । ऊर्ज᳚म् । अ॒हम् । इ॒तः । एति॑ । द॒दे॒ । इति॑ । इष᳚म् । ए॒व । ऊर्ज᳚म् । तस्यै᳚ । दि॒शः । अवेति॑ ( ) । रु॒न्धे॒ । क्षोधु॑कः । भ॒व॒ति॒ । यः । तस्या᳚म् । दि॒शि । भव॑ति । उ॒त्त॒र॒वे॒दिमित्यु॑त्तर - वे॒दिम् । उपेति॑ । व॒प॒ति॒ । उ॒त्त॒र॒वे॒द्यामित्यु॑त्तर - वे॒द्याम् । हि । अ॒ग्निः । ची॒यते᳚ । अथो॒ इति॑ । प॒शवः॑ । वै । उ॒त्त॒र॒वे॒दिरित्यु॑त्तर - वे॒दिः । प॒शून् । ए॒व । अवेति॑ । रु॒न्धे॒ । अथो॒ इति॑ । य॒ज्ञ्॒प॒रुष॒ इति॑ यज्ञ् - प॒रुषः॑ । अन॑न्तरित्या॒ इत्यन॑न्तः - इ॒त्यै॒ ॥  \newline


\textbf{Krama Paata} \newline

अन॑वरुद्धस्याश्ञी॒यात् । अन॑वरुद्ध॒स्येत्यन॑व - रु॒द्ध॒स्य॒ । अ॒श्ञी॒याद,व॑रुद्धेन । अव॑रुद्धेन॒ वि । अव॑रुद्धे॒नेत्यव॑ - रु॒द्धे॒न॒ । व्यृ॑द्ध्येत । ऋ॒द्ध्ये॒त॒ ये । ये वन॒स्पती॑नाम् । वन॒स्पती॑नाम् फल॒ग्रह॑यः । फ॒ल॒ग्रह॑य॒स्तान् । फ॒ल॒ग्रह॑य॒ इति॑ फल - ग्रह॑यः । तानि॒द्ध्मे । इ॒द्ध्मेऽपि॑ । अपि॒ प्र । प्रोक्षे᳚त् । उ॒क्षे॒दन॑वरुद्धस्य । अन॑वरुद्ध॒स्याव॑रुद्ध्यै । अन॑वरुद्ध॒स्येत्यन॑व - रु॒द्ध॒स्य॒ । अव॑रुद्ध्यै दि॒ग्भ्यः । अव॑रुद्ध्या॒ इत्यव॑ - रु॒द्ध्यै॒ । दि॒ग्भ्यो लो॒ष्टान् । दि॒ग्भ्य इति॑ दिक् - भ्यः॑ । लो॒ष्टान्थ् सम् । सम॑स्यति । अ॒स्य॒ति॒ दि॒शाम् । दि॒शामे॒व । ए॒व वी॒र्य᳚म् । वी॒र्य॑मव॒रुद्ध्य॑ । अ॒व॒रुद्ध्य॑ दि॒शाम् । अ॒व॒रुद्ध्येत्य॑व - रुद्ध्य॑ । दि॒शाम् ॅवी॒र्ये᳚ । वी॒र्ये᳚ऽग्निम् । अ॒ग्निम् चि॑नुते । चि॒नु॒ते॒ यम् । यम् द्वि॒ष्यात् । द्वि॒ष्याद् यत्र॑ । यत्र॒ सः । स स्यात् । स्यात् तस्यै᳚ । तस्यै॑ दि॒शः । दि॒शो लो॒ष्टम् । लो॒ष्टमा । आ ह॑रेत् । ह॒रे॒दिष᳚म् । इष॒मूर्ज᳚म् । ऊर्ज॑म॒हम् । अ॒हमि॒तः । इ॒त आ । आ द॑दे । द॒द॒ इति॑ । इतीष᳚म् । इष॑मे॒व । ए॒वोर्ज᳚म् । ऊर्ज॒म् तस्यै᳚ । तस्यै॑ दि॒शः । दि॒शोऽव॑ ( ) । अव॑ रुन्धे । रु॒न्धे॒ क्षोधु॑कः । क्षोधु॑को भवति । भ॒व॒ति॒ यः । यस्तस्या᳚म् । तस्या᳚म् दि॒शि । दि॒शि भव॑ति । भव॑त्युत्तरवे॒दिम् । उ॒त्त॒र॒वे॒दिमुप॑ । उ॒त्त॒र॒वे॒दिमित्यु॑त्तर - वे॒दिम् । उप॑ वपति । व॒प॒त्यु॒त्त॒र॒वे॒द्याम् । उ॒त्त॒र॒वे॒द्याꣳ हि । उ॒त्त॒र॒वे॒द्यामित्यु॑त्तर - वे॒द्याम् । ह्य॑ग्निः । अ॒ग्निश्ची॒यते᳚ । ची॒यतेऽथो᳚ । अथो॑ प॒शवः॑ । अथो॒ इत्यथो᳚ । प॒शवो॒ वै । वा उ॑त्तरवे॒दिः । उ॒त्त॒र॒वे॒दिः प॒शून् । उ॒त्त॒र॒वे॒दिरित्यु॑त्तर - वे॒दिः । प॒शूने॒व । ए॒वाव॑ । अव॑ रुन्धे । रु॒न्धेऽथो᳚ । अथो॑ यज्ञ्प॒रुषः॑ । अथो॒ इत्यथो᳚ । य॒ज्ञ्॒प॒रुषोऽन॑न्तरित्यै । य॒ज्ञ्॒प॒रुष॒ इति॑ यज्ञ् - प॒रुषः॑ । अन॑न्तरित्या॒ इत्यन॑न्तः - इ॒त्यै॒ । \newline

\textbf{Jatai Paata} \newline

1. अन॑वरुद्धस्या श्ञी॒या द॑श्ञी॒या दन॑वरुद्ध॒स्या न॑वरुद्धस्या श्ञी॒यात् । \newline
2. अन॑वरुद्ध॒स्येत्यन॑व - रु॒द्ध॒स्य॒ । \newline
3. अ॒श्ञी॒या दव॑रुद्धे॒ना व॑रुद्धेना श्ञी॒या द॑श्ञी॒या दव॑रुद्धेन । \newline
4. अव॑रुद्धेन॒ वि व्यव॑रुद्धे॒ना व॑रुद्धेन॒ वि । \newline
5. अव॑रुद्धे॒नेत्यव॑ - रु॒द्धे॒न॒ । \newline
6. व्यृ॑द्ध्येत र्‌द्ध्येत॒ वि व्यृ॑द्ध्येत । \newline
7. ऋ॒द्ध्ये॒त॒ ये य ऋ॑द्ध्येत र्‌द्ध्येत॒ ये । \newline
8. ये वन॒स्पती॑नां॒ ॅवन॒स्पती॑नां॒ ॅये ये वन॒स्पती॑नाम् । \newline
9. वन॒स्पती॑नाम् फल॒ग्रह॑यः फल॒ग्रह॑यो॒ वन॒स्पती॑नां॒ ॅवन॒स्पती॑नाम् फल॒ग्रह॑यः । \newline
10. फ॒ल॒ग्रह॑य॒ स्ताꣳ स्तान् फ॑ल॒ग्रह॑यः फल॒ग्रह॑य॒ स्तान् । \newline
11. फ॒ल॒ग्रह॑य॒ इति॑ फल - ग्रह॑यः । \newline
12. ता नि॒द्ध्म इ॒द्ध्मे ताꣳ स्ता नि॒द्ध्मे । \newline
13. इ॒द्ध्मे ऽप्यपी॒ द्ध्म इ॒द्ध्मे ऽपि॑ । \newline
14. अपि॒ प्र प्राप्यपि॒ प्र । \newline
15. प्रोक्षे॑ दुक्षे॒त् प्र प्रोक्षे᳚त् । \newline
16. उ॒क्षे॒ दन॑वरुद्ध॒स्या न॑वरुद्ध स्योक्षे दुक्षे॒ दन॑वरुद्धस्य । \newline
17. अन॑वरुद्ध॒स्या व॑रुद्ध्या॒ अव॑रुद्ध्या॒ अन॑वरुद्ध॒स्या न॑वरुद्ध॒स्या व॑रुद्ध्यै । \newline
18. अन॑वरुद्ध॒स्येत्यन॑व - रु॒द्ध॒स्य॒ । \newline
19. अव॑रुद्ध्यै दि॒ग्भ्यो दि॒ग्भ्यो ऽव॑रुद्ध्या॒ अव॑रुद्ध्यै दि॒ग्भ्यः । \newline
20. अव॑रुद्ध्या॒ इत्यव॑ - रु॒द्ध्यै॒ । \newline
21. दि॒ग्भ्यो लो॒ष्टान् ॅलो॒ष्टान् दि॒ग्भ्यो दि॒ग्भ्यो लो॒ष्टान् । \newline
22. दि॒ग्भ्य इति॑ दिक् - भ्यः । \newline
23. लो॒ष्टान् थ्सꣳ सम् ॅलो॒ष्टान् ॅलो॒ष्टान् थ्सम् । \newline
24. स म॑स्य त्यस्यति॒ सꣳ स म॑स्यति । \newline
25. अ॒स्य॒ति॒ दि॒शाम् दि॒शा म॑स्य त्यस्यति दि॒शाम् । \newline
26. दि॒शा मे॒वैव दि॒शाम् दि॒शा मे॒व । \newline
27. ए॒व वी॒र्यं॑ ॅवी॒र्य॑ मे॒वैव वी॒र्य᳚म् । \newline
28. वी॒र्य॑ मव॒रुद्ध्या॑ व॒रुद्ध्य॑ वी॒र्यं॑ ॅवी॒र्य॑ मव॒रुद्ध्य॑ । \newline
29. अ॒व॒रुद्ध्य॑ दि॒शाम् दि॒शा म॑व॒रुद्ध्या॑ व॒रुद्ध्य॑ दि॒शाम् । \newline
30. अ॒व॒रुद्ध्येत्य॑व - रुद्ध्य॑ । \newline
31. दि॒शां ॅवी॒र्ये॑ वी॒र्ये॑ दि॒शाम् दि॒शां ॅवी॒र्ये᳚ । \newline
32. वी॒र्ये᳚ ऽग्नि म॒ग्निं ॅवी॒र्ये॑ वी॒र्ये᳚ ऽग्निम् । \newline
33. अ॒ग्निम् चि॑नुते चिनुते॒ ऽग्नि म॒ग्निम् चि॑नुते । \newline
34. चि॒नु॒ते॒ यं ॅयम् चि॑नुते चिनुते॒ यम् । \newline
35. यम् द्वि॒ष्याद् द्वि॒ष्याद् यं ॅयम् द्वि॒ष्यात् । \newline
36. द्वि॒ष्याद् यत्र॒ यत्र॑ द्वि॒ष्याद् द्वि॒ष्याद् यत्र॑ । \newline
37. यत्र॒ स स यत्र॒ यत्र॒ सः । \newline
38. स स्याथ् स्याथ् स स स्यात् । \newline
39. स्यात् तस्यै॒ तस्यै॒ स्याथ् स्यात् तस्यै᳚ । \newline
40. तस्यै॑ दि॒शो दि॒श स्तस्यै॒ तस्यै॑ दि॒शः । \newline
41. दि॒शो लो॒ष्टम् ॅलो॒ष्टम् दि॒शो दि॒शो लो॒ष्टम् । \newline
42. लो॒ष्ट मा लो॒ष्टम् ॅलो॒ष्ट मा । \newline
43. आ ह॑रे द्धरे॒दा ह॑रेत् । \newline
44. ह॒रे॒ दिष॒ मिषꣳ॑ हरे द्धरे॒ दिष᳚म् । \newline
45. इष॒ मूर्ज॒ मूर्ज॒ मिष॒ मिष॒ मूर्ज᳚म् । \newline
46. ऊर्ज॑ म॒ह म॒ह मूर्ज॒ मूर्ज॑ म॒हम् । \newline
47. अ॒ह मि॒त इ॒तो॑ ऽह म॒ह मि॒तः । \newline
48. इ॒त एत इ॒त आ । \newline
49. आ द॑दे दद॒ आ द॑दे । \newline
50. द॒द॒ इतीति॑ ददे दद॒ इति॑ । \newline
51. इतीष॒ मिष॒ मितीतीष᳚म् । \newline
52. इष॑ मे॒वै वेष॒ मिष॑ मे॒व । \newline
53. ए॒वोर्ज॒ मूर्ज॑ मे॒वैवोर्ज᳚म् । \newline
54. ऊर्ज॒म् तस्यै॒ तस्या॒ ऊर्ज॒ मूर्ज॒म् तस्यै᳚ । \newline
55. तस्यै॑ दि॒शो दि॒श स्तस्यै॒ तस्यै॑ दि॒शः । \newline
56. दि॒शो ऽवाव॑ दि॒शो दि॒शो ऽव॑ । \newline
57. अव॑ रुन्धे रु॒न्धे ऽवाव॑ रुन्धे । \newline
58. रु॒न्धे॒ क्षोधु॑कः॒ क्षोधु॑को रुन्धे रुन्धे॒ क्षोधु॑कः । \newline
59. क्षोधु॑को भवति भवति॒ क्षोधु॑कः॒ क्षोधु॑को भवति । \newline
60. भ॒व॒ति॒ यो यो भ॑वति भवति॒ यः । \newline
61. य स्तस्या॒म् तस्यां॒ ॅयो य स्तस्या᳚म् । \newline
62. तस्या᳚म् दि॒शि दि॒शि तस्या॒म् तस्या᳚म् दि॒शि । \newline
63. दि॒शि भव॑ति॒ भव॑ति दि॒शि दि॒शि भव॑ति । \newline
64. भव॑ त्युत्तरवे॒दि मु॑त्तरवे॒दिम् भव॑ति॒ भव॑ त्युत्तरवे॒दिम् । \newline
65. उ॒त्त॒र॒वे॒दि मुपोपो᳚त्तरवे॒दि मु॑त्तरवे॒दि मुप॑ । \newline
66. उ॒त्त॒र॒वे॒दिमित्यु॑त्तर - वे॒दिम् । \newline
67. उप॑ वपति वप॒ त्युपोप॑ वपति । \newline
68. व॒प॒ त्यु॒त्त॒र॒वे॒द्या मु॑त्तरवे॒द्यां ॅव॑पति वप त्युत्तरवे॒द्याम् । \newline
69. उ॒त्त॒र॒वे॒द्याꣳ हि ह्यु॑त्तरवे॒द्या मु॑त्तरवे॒द्याꣳ हि । \newline
70. उ॒त्त॒र॒वे॒द्यामित्यु॑त्तर - वे॒द्याम् । \newline
71. ह्य॑ग्नि र॒ग्निर्. हि ह्य॑ग्निः । \newline
72. अ॒ग्नि श्ची॒यते॑ ची॒यते॒ ऽग्नि र॒ग्नि श्ची॒यते᳚ । \newline
73. ची॒यते ऽथो॒ अथो॑ ची॒यते॑ ची॒यते ऽथो᳚ । \newline
74. अथो॑ प॒शवः॑ प॒शवो ऽथो॒ अथो॑ प॒शवः॑ । \newline
75. अथो॒ इत्यथो᳚ । \newline
76. प॒शवो॒ वै वै प॒शवः॑ प॒शवो॒ वै । \newline
77. वा उ॑त्तरवे॒दि रु॑त्तरवे॒दिर् वै वा उ॑त्तरवे॒दिः । \newline
78. उ॒त्त॒र॒वे॒दिः प॒शून् प॒शू नु॑त्तरवे॒दि रु॑त्तरवे॒दिः प॒शून् । \newline
79. उ॒त्त॒र॒वे॒दिरित्यु॑त्तर - वे॒दिः । \newline
80. प॒शू ने॒वैव प॒शून् प॒शू ने॒व । \newline
81. ए॒वावा वै॒वै वाव॑ । \newline
82. अव॑ रुन्धे रु॒न्धे ऽवाव॑ रुन्धे । \newline
83. रु॒न्धे ऽथो॒ अथो॑ रुन्धे रु॒न्धे ऽथो᳚ । \newline
84. अथो॑ यज्ञ्प॒रुषो॑ यज्ञ्प॒रुषो ऽथो॒ अथो॑ यज्ञ्प॒रुषः॑ । \newline
85. अथो॒ इत्यथो᳚ । \newline
86. य॒ज्ञ्॒प॒रुषो ऽन॑न्तरित्या॒ अन॑न्तरित्यै यज्ञ्प॒रुषो॑ यज्ञ्प॒रुषो ऽन॑न्तरित्यै । \newline
87. य॒ज्ञ्॒प॒रुष॒ इति॑ यज्ञ् - प॒रुषः॑ । \newline
88. अन॑न्तरित्या॒ इत्यन॑न्तः - इ॒त्यै॒ । \newline

\textbf{Ghana Paata } \newline

1. अन॑वरुद्धस्या श्ञी॒या द॑श्ञी॒या दन॑वरुद्ध॒स्या न॑वरुद्धस्या श्ञी॒या दव॑रुद्धे॒ना व॑रुद्धेना श्ञी॒या दन॑वरुद्ध॒स्या न॑वरुद्धस्या श्ञी॒या दव॑रुद्धेन । \newline
2. अन॑वरुद्ध॒स्येत्यन॑व - रु॒द्ध॒स्य॒ । \newline
3. अ॒श्ञी॒या दव॑रुद्धे॒ना व॑रुद्धेना श्ञी॒या द॑श्ञी॒या दव॑रुद्धेन॒ वि व्यव॑रुद्धेना श्ञी॒या द॑श्ञी॒या दव॑रुद्धेन॒ वि । \newline
4. अव॑रुद्धेन॒ वि व्यव॑रुद्धे॒ना व॑रुद्धेन॒ व्यृ॑द्ध्येत र्द्ध्येत॒ व्यव॑रुद्धे॒ना व॑रुद्धेन॒ व्यृ॑द्ध्येत । \newline
5. अव॑रुद्धे॒नेत्यव॑ - रु॒द्धे॒न॒ । \newline
6. व्यृ॑द्ध्येत र्द्ध्येत॒ वि व्यृ॑द्ध्येत॒ ये य ऋ॑द्ध्येत॒ वि व्यृ॑द्ध्येत॒ ये । \newline
7. ऋ॒द्ध्ये॒त॒ ये य ऋ॑द्ध्येत र्द्ध्येत॒ ये वन॒स्पती॑नां॒ ॅवन॒स्पती॑नां॒ ॅय ऋ॑द्ध्येत र्द्ध्येत॒ ये वन॒स्पती॑नाम् । \newline
8. ये वन॒स्पती॑नां॒ ॅवन॒स्पती॑नां॒ ॅये ये वन॒स्पती॑नाम् फल॒ग्रह॑यः फल॒ग्रह॑यो॒ वन॒स्पती॑नां॒ ॅये ये वन॒स्पती॑नाम् फल॒ग्रह॑यः । \newline
9. वन॒स्पती॑नाम् फल॒ग्रह॑यः फल॒ग्रह॑यो॒ वन॒स्पती॑नां॒ ॅवन॒स्पती॑नाम् फल॒ग्रह॑य॒ स्ताꣳ स्तान् फ॑ल॒ग्रह॑यो॒ वन॒स्पती॑नां॒ ॅवन॒स्पती॑नाम् फल॒ग्रह॑य॒ स्तान् । \newline
10. फ॒ल॒ग्रह॑य॒ स्ताꣳ स्तान् फ॑ल॒ग्रह॑यः फल॒ग्रह॑य॒ स्ता नि॒द्ध्म इ॒द्ध्मे तान् फ॑ल॒ग्रह॑यः फल॒ग्रह॑य॒ स्ता नि॒द्ध्मे । \newline
11. फ॒ल॒ग्रह॑य॒ इति॑ फल - ग्रह॑यः । \newline
12. ता नि॒द्ध्म इ॒द्ध्मे ताꣳ स्ता नि॒द्ध्मे ऽप्यपी॒द्ध्मे ताꣳ स्ता नि॒द्ध्मे ऽपि॑ । \newline
13. इ॒द्ध्मे ऽप्यपी॒ द्ध्म इ॒द्ध्मे ऽपि॒ प्र प्रापी॒ द्ध्म इ॒द्ध्मे ऽपि॒ प्र । \newline
14. अपि॒ प्र प्राप्यपि॒ प्रोक्षे॑ दुक्षे॒त् प्राप्यपि॒ प्रोक्षे᳚त् । \newline
15. प्रोक्षे॑ दुक्षे॒त् प्र प्रोक्षे॒ दन॑वरुद्ध॒स्या न॑वरुद्ध स्योक्षे॒त् प्र प्रोक्षे॒ दन॑वरुद्धस्य । \newline
16. उ॒क्षे॒ दन॑वरुद्ध॒स्या न॑वरुद्ध स्योक्षे दुक्षे॒ दन॑वरुद्ध॒स्या व॑रुद्ध्या॒ अव॑रुद्ध्या॒ अन॑वरुद्ध स्योक्षे दुक्षे॒ दन॑वरुद्ध॒स्या व॑रुद्ध्यै । \newline
17. अन॑वरुद्ध॒स्या व॑रुद्ध्या॒ अव॑रुद्ध्या॒ अन॑वरुद्ध॒स्या न॑वरुद्ध॒स्या व॑रुद्ध्यै दि॒ग्भ्यो दि॒ग्भ्यो ऽव॑रुद्ध्या॒ अन॑वरुद्ध॒स्या न॑वरुद्ध॒स्या व॑रुद्ध्यै दि॒ग्भ्यः । \newline
18. अन॑वरुद्ध॒स्येत्यन॑व - रु॒द्ध॒स्य॒ । \newline
19. अव॑रुद्ध्यै दि॒ग्भ्यो दि॒ग्भ्यो ऽव॑रुद्ध्या॒ अव॑रुद्ध्यै दि॒ग्भ्यो लो॒ष्टान् ॅलो॒ष्टान् दि॒ग्भ्यो ऽव॑रुद्ध्या॒ अव॑रुद्ध्यै दि॒ग्भ्यो लो॒ष्टान् । \newline
20. अव॑रुद्ध्या॒ इत्यव॑ - रु॒द्ध्यै॒ । \newline
21. दि॒ग्भ्यो लो॒ष्टान् ॅलो॒ष्टान् दि॒ग्भ्यो दि॒ग्भ्यो लो॒ष्टान् थ्सꣳ सम् ॅलो॒ष्टान् दि॒ग्भ्यो दि॒ग्भ्यो लो॒ष्टान् थ्सम् । \newline
22. दि॒ग्भ्य इति॑ दिक् - भ्यः । \newline
23. लो॒ष्टान् थ्सꣳ सम् ॅलो॒ष्टान् ॅलो॒ष्टान् थ्स म॑स्य त्यस्यति॒ सम् ॅलो॒ष्टान् ॅलो॒ष्टान् थ्स म॑स्यति । \newline
24. स म॑स्य त्यस्यति॒ सꣳ स म॑स्यति दि॒शाम् दि॒शा म॑स्यति॒ सꣳ स म॑स्यति दि॒शाम् । \newline
25. अ॒स्य॒ति॒ दि॒शाम् दि॒शा म॑स्य त्यस्यति दि॒शा मे॒वैव दि॒शा म॑स्य त्यस्यति दि॒शा मे॒व । \newline
26. दि॒शा मे॒वैव दि॒शाम् दि॒शा मे॒व वी॒र्यं॑ ॅवी॒र्य॑ मे॒व दि॒शाम् दि॒शा मे॒व वी॒र्य᳚म् । \newline
27. ए॒व वी॒र्यं॑ ॅवी॒र्य॑ मे॒वैव वी॒र्य॑ मव॒रुद्ध्या॑ व॒रुद्ध्य॑ वी॒र्य॑ मे॒वैव वी॒र्य॑ मव॒रुद्ध्य॑ । \newline
28. वी॒र्य॑ मव॒रुद्ध्या॑ व॒रुद्ध्य॑ वी॒र्यं॑ ॅवी॒र्य॑ मव॒रुद्ध्य॑ दि॒शाम् दि॒शा म॑व॒रुद्ध्य॑ वी॒र्यं॑ ॅवी॒र्य॑ मव॒रुद्ध्य॑ दि॒शाम् । \newline
29. अ॒व॒रुद्ध्य॑ दि॒शाम् दि॒शा म॑व॒रुद्ध्या॑ व॒रुद्ध्य॑ दि॒शां ॅवी॒र्ये॑ वी॒र्ये॑ दि॒शा म॑व॒रुद्ध्या॑ व॒रुद्ध्य॑ दि॒शां ॅवी॒र्ये᳚ । \newline
30. अ॒व॒रुद्ध्येत्य॑व - रुद्ध्य॑ । \newline
31. दि॒शां ॅवी॒र्ये॑ वी॒र्ये॑ दि॒शाम् दि॒शां ॅवी॒र्ये᳚ ऽग्नि म॒ग्निं ॅवी॒र्ये॑ दि॒शाम् दि॒शां ॅवी॒र्ये᳚ ऽग्निम् । \newline
32. वी॒र्ये᳚ ऽग्नि म॒ग्निं ॅवी॒र्ये॑ वी॒र्ये᳚ ऽग्निम् चि॑नुते चिनुते॒ ऽग्निं ॅवी॒र्ये॑ वी॒र्ये᳚ ऽग्निम् चि॑नुते । \newline
33. अ॒ग्निम् चि॑नुते चिनुते॒ ऽग्नि म॒ग्निम् चि॑नुते॒ यं ॅयम् चि॑नुते॒ ऽग्नि म॒ग्निम् चि॑नुते॒ यम् । \newline
34. चि॒नु॒ते॒ यं ॅयम् चि॑नुते चिनुते॒ यम् द्वि॒ष्याद् द्वि॒ष्याद् यम् चि॑नुते चिनुते॒ यम् द्वि॒ष्यात् । \newline
35. यम् द्वि॒ष्याद् द्वि॒ष्याद् यं ॅयम् द्वि॒ष्याद् यत्र॒ यत्र॑ द्वि॒ष्याद् यं ॅयम् द्वि॒ष्याद् यत्र॑ । \newline
36. द्वि॒ष्याद् यत्र॒ यत्र॑ द्वि॒ष्याद् द्वि॒ष्याद् यत्र॒ स स यत्र॑ द्वि॒ष्याद् द्वि॒ष्याद् यत्र॒ सः । \newline
37. यत्र॒ स स यत्र॒ यत्र॒ स स्याथ् स्याथ् स यत्र॒ यत्र॒ स स्यात् । \newline
38. स स्याथ् स्याथ् स स स्यात् तस्यै॒ तस्यै॒ स्याथ् स स स्यात् तस्यै᳚ । \newline
39. स्यात् तस्यै॒ तस्यै॒ स्याथ् स्यात् तस्यै॑ दि॒शो दि॒श स्तस्यै॒ स्याथ् स्यात् तस्यै॑ दि॒शः । \newline
40. तस्यै॑ दि॒शो दि॒श स्तस्यै॒ तस्यै॑ दि॒शो लो॒ष्टम् ॅलो॒ष्टम् दि॒श स्तस्यै॒ तस्यै॑ दि॒शो लो॒ष्टम् । \newline
41. दि॒शो लो॒ष्टम् ॅलो॒ष्टम् दि॒शो दि॒शो लो॒ष्ट मा लो॒ष्टम् दि॒शो दि॒शो लो॒ष्ट मा । \newline
42. लो॒ष्ट मा लो॒ष्टम् ॅलो॒ष्ट मा ह॑रे द्धरे॒दा लो॒ष्टम् ॅलो॒ष्ट मा ह॑रेत् । \newline
43. आ ह॑रे द्धरे॒दा ह॑रे॒ दिष॒ मिषꣳ॑ हरे॒दा ह॑रे॒ दिष᳚म् । \newline
44. ह॒रे॒ दिष॒ मिषꣳ॑ हरे द्धरे॒ दिष॒ मूर्ज॒ मूर्ज॒ मिषꣳ॑ हरे द्धरे॒ दिष॒ मूर्ज᳚म् । \newline
45. इष॒ मूर्ज॒ मूर्ज॒ मिष॒ मिष॒ मूर्ज॑ म॒ह म॒ह मूर्ज॒ मिष॒ मिष॒ मूर्ज॑ म॒हम् । \newline
46. ऊर्ज॑ म॒ह म॒ह मूर्ज॒ मूर्ज॑ म॒ह मि॒त इ॒तो॑ ऽह मूर्ज॒ मूर्ज॑ म॒ह मि॒तः । \newline
47. अ॒ह मि॒त इ॒तो॑ ऽह म॒ह मि॒त एतो॑ ऽह म॒ह मि॒त आ । \newline
48. इ॒त एत इ॒त आ द॑दे दद॒ एत इ॒त आ द॑दे । \newline
49. आ द॑दे दद॒ आ द॑द॒ इतीति॑ दद॒ आ द॑द॒ इति॑ । \newline
50. द॒द॒ इतीति॑ ददे दद॒ इतीष॒ मिष॒ मिति॑ ददे दद॒ इतीष᳚म् । \newline
51. इतीष॒ मिष॒ मितीतीष॑ मे॒वैवेष॒ मितीतीष॑ मे॒व । \newline
52. इष॑ मे॒वैवेष॒ मिष॑ मे॒वोर्ज॒ मूर्ज॑ मे॒वेष॒ मिष॑ मे॒वोर्ज᳚म् । \newline
53. ए॒वोर्ज॒ मूर्ज॑ मे॒वैवोर्ज॒म् तस्यै॒ तस्या॒ ऊर्ज॑ मे॒वैवोर्ज॒म् तस्यै᳚ । \newline
54. ऊर्ज॒म् तस्यै॒ तस्या॒ ऊर्ज॒ मूर्ज॒म् तस्यै॑ दि॒शो दि॒श स्तस्या॒ ऊर्ज॒ मूर्ज॒म् तस्यै॑ दि॒शः । \newline
55. तस्यै॑ दि॒शो दि॒श स्तस्यै॒ तस्यै॑ दि॒शो ऽवाव॑ दि॒श स्तस्यै॒ तस्यै॑ दि॒शो ऽव॑ । \newline
56. दि॒शो ऽवाव॑ दि॒शो दि॒शो ऽव॑ रुन्धे रु॒न्धे ऽव॑ दि॒शो दि॒शो ऽव॑ रुन्धे । \newline
57. अव॑ रुन्धे रु॒न्धे ऽवाव॑ रुन्धे॒ क्षोधु॑कः॒ क्षोधु॑को रु॒न्धे ऽवाव॑ रुन्धे॒ क्षोधु॑कः । \newline
58. रु॒न्धे॒ क्षोधु॑कः॒ क्षोधु॑को रुन्धे रुन्धे॒ क्षोधु॑को भवति भवति॒ क्षोधु॑को रुन्धे रुन्धे॒ क्षोधु॑को भवति । \newline
59. क्षोधु॑को भवति भवति॒ क्षोधु॑कः॒ क्षोधु॑को भवति॒ यो यो भ॑वति॒ क्षोधु॑कः॒ क्षोधु॑को भवति॒ यः । \newline
60. भ॒व॒ति॒ यो यो भ॑वति भवति॒ य स्तस्या॒म् तस्यां॒ ॅयो भ॑वति भवति॒ य स्तस्या᳚म् । \newline
61. य स्तस्या॒म् तस्यां॒ ॅयो य स्तस्या᳚म् दि॒शि दि॒शि तस्यां॒ ॅयो य स्तस्या᳚म् दि॒शि । \newline
62. तस्या᳚म् दि॒शि दि॒शि तस्या॒म् तस्या᳚म् दि॒शि भव॑ति॒ भव॑ति दि॒शि तस्या॒म् तस्या᳚म् दि॒शि भव॑ति । \newline
63. दि॒शि भव॑ति॒ भव॑ति दि॒शि दि॒शि भव॑ त्युत्तरवे॒दि मु॑त्तरवे॒दिम् भव॑ति दि॒शि दि॒शि भव॑ त्युत्तरवे॒दिम् । \newline
64. भव॑ त्युत्तरवे॒दि मु॑त्तरवे॒दिम् भव॑ति॒ भव॑ त्युत्तरवे॒दि मुपोपो᳚त्तरवे॒दिम् भव॑ति॒ भव॑ त्युत्तरवे॒दि मुप॑ । \newline
65. उ॒त्त॒र॒वे॒दि मुपोपो᳚त्तरवे॒दि मु॑त्तरवे॒दि मुप॑ वपति वप॒ त्युपो᳚त्तरवे॒दि मु॑त्तरवे॒दि मुप॑ वपति । \newline
66. उ॒त्त॒र॒वे॒दिमित्यु॑त्तर - वे॒दिम् । \newline
67. उप॑ वपति वप॒ त्युपोप॑ वप त्युत्तरवे॒द्या मु॑त्तरवे॒द्यां ॅव॑प॒ त्युपोप॑ वप त्युत्तरवे॒द्याम् । \newline
68. व॒प॒ त्यु॒त्त॒र॒वे॒द्या मु॑त्तरवे॒द्यां ॅव॑पति वप त्युत्तरवे॒द्याꣳ हि ह्यु॑त्तरवे॒द्यां ॅव॑पति वप त्युत्तरवे॒द्याꣳ हि । \newline
69. उ॒त्त॒र॒वे॒द्याꣳ हि ह्यु॑त्तरवे॒द्या मु॑त्तरवे॒द्याꣳ ह्य॑ग्नि र॒ग्निर् ह्यु॑त्तरवे॒द्या मु॑त्तरवे॒द्याꣳ ह्य॑ग्निः । \newline
70. उ॒त्त॒र॒वे॒द्यामित्यु॑त्तर - वे॒द्याम् । \newline
71. ह्य॑ग्नि र॒ग्निर्. हि ह्य॑ग्नि श्ची॒यते॑ ची॒यते॒ ऽग्निर्. हि ह्य॑ग्नि श्ची॒यते᳚ । \newline
72. अ॒ग्नि श्ची॒यते॑ ची॒यते॒ ऽग्नि र॒ग्नि श्ची॒यते ऽथो॒ अथो॑ ची॒यते॒ ऽग्नि र॒ग्नि श्ची॒यते ऽथो᳚ । \newline
73. ची॒यते ऽथो॒ अथो॑ ची॒यते॑ ची॒यते ऽथो॑ प॒शवः॑ प॒शवो ऽथो॑ ची॒यते॑ ची॒यते ऽथो॑ प॒शवः॑ । \newline
74. अथो॑ प॒शवः॑ प॒शवो ऽथो॒ अथो॑ प॒शवो॒ वै वै प॒शवो ऽथो॒ अथो॑ प॒शवो॒ वै । \newline
75. अथो॒ इत्यथो᳚ । \newline
76. प॒शवो॒ वै वै प॒शवः॑ प॒शवो॒ वा उ॑त्तरवे॒दि रु॑त्तरवे॒दिर् वै प॒शवः॑ प॒शवो॒ वा उ॑त्तरवे॒दिः । \newline
77. वा उ॑त्तरवे॒दि रु॑त्तरवे॒दिर् वै वा उ॑त्तरवे॒दिः प॒शून् प॒शू नु॑त्तरवे॒दिर् वै वा उ॑त्तरवे॒दिः प॒शून् । \newline
78. उ॒त्त॒र॒वे॒दिः प॒शून् प॒शू नु॑त्तरवे॒दि रु॑त्तरवे॒दिः प॒शू ने॒वैव प॒शू नु॑त्तरवे॒दि रु॑त्तरवे॒दिः प॒शू ने॒व । \newline
79. उ॒त्त॒र॒वे॒दिरित्यु॑त्तर - वे॒दिः । \newline
80. प॒शू ने॒वैव प॒शून् प॒शू ने॒वावा वै॒व प॒शून् प॒शू ने॒वाव॑ । \newline
81. ए॒वावा वै॒वैवाव॑ रुन्धे रु॒न्धे ऽवै॒वै वाव॑ रुन्धे । \newline
82. अव॑ रुन्धे रु॒न्धे ऽवाव॑ रु॒न्धे ऽथो॒ अथो॑ रु॒न्धे ऽवाव॑ रु॒न्धे ऽथो᳚ । \newline
83. रु॒न्धे ऽथो॒ अथो॑ रुन्धे रु॒न्धे ऽथो॑ यज्ञ्प॒रुषो॑ यज्ञ्प॒रुषो ऽथो॑ रुन्धे रु॒न्धे ऽथो॑ यज्ञ्प॒रुषः॑ । \newline
84. अथो॑ यज्ञ्प॒रुषो॑ यज्ञ्प॒रुषो ऽथो॒ अथो॑ यज्ञ्प॒रुषो ऽन॑न्तरित्या॒ अन॑न्तरित्यै यज्ञ्प॒रुषो ऽथो॒ अथो॑ यज्ञ्प॒रुषो ऽन॑न्तरित्यै । \newline
85. अथो॒ इत्यथो᳚ । \newline
86. य॒ज्ञ्॒प॒रुषो ऽन॑न्तरित्या॒ अन॑न्तरित्यै यज्ञ्प॒रुषो॑ यज्ञ्प॒रुषो ऽन॑न्तरित्यै । \newline
87. य॒ज्ञ्॒प॒रुष॒ इति॑ यज्ञ् - प॒रुषः॑ । \newline
88. अन॑न्तरित्या॒ इत्यन॑न्तः - इ॒त्यै॒ । \newline
\pagebreak
\markright{ TS 5.2.6.1  \hfill https://www.vedavms.in \hfill}

\section{ TS 5.2.6.1 }

\textbf{TS 5.2.6.1 } \newline
\textbf{Samhita Paata} \newline

अग्ने॒ तव॒ श्रवो॒ वय॒ इति॒ सिक॑ता॒ नि व॑पत्ये॒तद्वा अ॒ग्नेर्वै᳚श्वान॒रस्य॑ सू॒क्तꣳ सू॒क्तेनै॒व वै᳚श्वान॒रमव॑ रुन्धे ष॒ड्भिर्नि व॑पति॒ षड्वा ऋ॒तवः॑ संॅवथ्स॒रः सं॑ॅवथ्स॒रो᳚ऽग्निर्वै᳚श्वान॒रः सा॒क्षादे॒व वै᳚श्वान॒रमव॑ रुन्धे समु॒द्रं ॅवै नामै॒तच्छन्दः॑ समु॒द्रमनु॑ प्र॒जाः प्रजा॑यन्ते॒ यदे॒तेन॒ सिक॑ता नि॒ वप॑ति प्र॒जानां᳚ प्र॒जन॑ना॒येन्द्रो॑ - [  ] \newline

\textbf{Pada Paata} \newline

अग्ने᳚ । तव॑ । श्रवः॑ । वयः॑ । इति॑ । सिक॑ताः । नीति॑ । व॒प॒ति॒ । ए॒तत् । वै । अ॒ग्नेः । वै॒श्वा॒न॒रस्य॑ । सू॒क्तमिति॑ सु - उ॒क्तम् । सू॒क्तेनेति॑ सु - उ॒क्तेन॑ । ए॒व । वै॒श्वा॒न॒रम् । अवेति॑ । रु॒न्धे॒ । ष॒ड्भिरिति॑ षट् - भिः । नीति॑ । व॒प॒ति॒ । षट् । वै । ऋ॒तवः॑ । सं॒ॅव॒थ्स॒र इति॑ सं - व॒थ्स॒रः । सं॒ॅव॒थ्स॒र इति॑ सं - व॒थ्स॒रः । अ॒ग्निः । वै॒श्वा॒न॒रः । सा॒क्षादिति॑ स - अ॒क्षात् । ए॒व । वै॒श्वा॒न॒रम् । अवेति॑ । रु॒न्धे॒ । स॒मु॒द्रम् । वै । नाम॑ । ए॒तत् । छन्दः॑ । स॒मु॒द्रम् । अन्विति॑ । प्र॒जा इति॑ प्र - जाः । प्रेति॑ । जा॒य॒न्ते॒ । यत् । ए॒तेन॑ । सिक॑ताः । नि॒वप॒तीति॑ नि - वप॑ति । प्र॒जाना॒मिति॑ प्र - जाना᳚म् । प्र॒जन॑ना॒येति॑ प्र - जन॑नाय । इन्द्रः॑ ।  \newline


\textbf{Krama Paata} \newline

अग्ने॒ तव॑ । तव॒ श्रवः॑ । श्रवो॒ वयः॑ । वय॒ इति॑ । इति॒ सिक॑ताः । सिक॑ता॒ नि । नि व॑पति । व॒प॒त्ये॒तत् । ए॒तद् वै । वा अ॒ग्नेः । अ॒ग्नेर् वै᳚श्वान॒रस्य॑ । वै॒श्वा॒न॒रस्य॑ सू॒क्तम् । सू॒क्तꣳ सू॒क्तेन॑ । सू॒क्तमिति॑ सु - उ॒क्तम् । सू॒क्तेनै॒व । सू॒क्तेनेति॑ सु - उ॒क्तेन॑ । ए॒व वै᳚श्वान॒रम् । वै॒श्वा॒न॒रमव॑ । अव॑ रुन्धे । रु॒न्धे॒ ष॒ड्भिः । ष॒ड्भिर् नि । ष॒ड्भिरिति॑ षट् - भिः । नि व॑पति । व॒प॒ति॒ षट् । षड् वै । वा ऋ॒तवः॑ । ऋ॒तवः॑ सम्ॅवथ्स॒रः । स॒म्ॅव॒थ्स॒रः स॑म्ॅवथ्स॒रः । स॒म्ॅव॒थ्स॒र इति॑ सम् - व॒थ्स॒रः । स॒म्ॅव॒थ्स॒रो᳚ऽग्निः । स॒म्ॅव॒थ्स॒र इति॑ सम् - व॒थ्स॒रः । अ॒ग्निर् वै᳚श्वान॒रः । वै॒श्वा॒न॒रः सा॒क्षात् । सा॒क्षादे॒व । सा॒क्षादिति॑ स - अ॒क्षात् । ए॒व वै᳚श्वान॒रम् । वै॒श्वा॒न॒रमव॑ । अव॑ रुन्धे । रु॒न्धे॒ स॒मु॒द्रम् । स॒मु॒द्रम् ॅवै । वै नाम॑ । नामै॒तत् । ए॒तच् छन्दः॑ । छन्दः॑ समु॒द्रम् । स॒मु॒द्रमनु॑ । अनु॑ प्र॒जाः । प्र॒जाः 
प्र । प्र॒जा इति॑ प्र - जाः । प्र जा॑यन्ते । जा॒य॒न्ते॒ यत् । यदे॒तेन॑ । ए॒तेन॒ सिक॑ताः । सिक॑ता नि॒वप॑ति । नि॒वप॑ति प्र॒जाना᳚म् । नि॒वप॒तीति॑ नि - वप॑ति । प्र॒जाना᳚म् प्र॒जन॑नाय । प्र॒जाना॒मिति॑ प्र - जाना᳚म् । प्र॒जन॑ना॒येन्द्रः॑ । प्र॒जन॑ना॒येति॑ प्र - जन॑नाय । इन्द्रो॑ वृ॒त्राय॑ \newline

\textbf{Jatai Paata} \newline

1. अग्ने॒ तव॒ तवाग्ने ऽग्ने॒ तव॑ । \newline
2. तव॒ श्रवः॒ श्रव॒ स्तव॒ तव॒ श्रवः॑ । \newline
3. श्रवो॒ वयो॒ वयः॒ श्रवः॒ श्रवो॒ वयः॑ । \newline
4. वय॒ इतीति॒ वयो॒ वय॒ इति॑ । \newline
5. इति॒ सिक॑ताः॒ सिक॑ता॒ इतीति॒ सिक॑ताः । \newline
6. सिक॑ता॒ नि नि सिक॑ताः॒ सिक॑ता॒ नि । \newline
7. नि व॑पति वपति॒ नि नि व॑पति । \newline
8. व॒प॒ त्ये॒त दे॒तद् व॑पति वप त्ये॒तत् । \newline
9. ए॒तद् वै वा ए॒त दे॒तद् वै । \newline
10. वा अ॒ग्ने र॒ग्नेर् वै वा अ॒ग्नेः । \newline
11. अ॒ग्नेर् वै᳚श्वान॒रस्य॑ वैश्वान॒रस्या॒ ग्ने र॒ग्नेर् वै᳚श्वान॒रस्य॑ । \newline
12. वै॒श्वा॒न॒रस्य॑ सू॒क्तꣳ सू॒क्तं ॅवै᳚श्वान॒रस्य॑ वैश्वान॒रस्य॑ सू॒क्तम् । \newline
13. सू॒क्तꣳ सू॒क्तेन॑ सू॒क्तेन॑ सू॒क्तꣳ सू॒क्तꣳ सू॒क्तेन॑ । \newline
14. सू॒क्तमिति॑ सु - उ॒क्तम् । \newline
15. सू॒क्तेनै॒ वैव सू॒क्तेन॑ सू॒क्तेनै॒व । \newline
16. सू॒क्तेनेति॑ सु - उ॒क्तेन॑ । \newline
17. ए॒व वै᳚श्वान॒रं ॅवै᳚श्वान॒र मे॒वैव वै᳚श्वान॒रम् । \newline
18. वै॒श्वा॒न॒र मवाव॑ वैश्वान॒रं ॅवै᳚श्वान॒र मव॑ । \newline
19. अव॑ रुन्धे रु॒न्धे ऽवाव॑ रुन्धे । \newline
20. रु॒न्धे॒ ष॒ड्भि ष्ष॒ड्भी रु॑न्धे रुन्धे ष॒ड्भिः । \newline
21. ष॒ड्भिर् नि नि ष॒ड्भि ष्ष॒ड्भिर् नि । \newline
22. ष॒ड्भिरिति॑ षट् - भिः । \newline
23. नि व॑पति वपति॒ नि नि व॑पति । \newline
24. व॒प॒ति॒ षट् थ्षड् व॑पति वपति॒ षट् । \newline
25. षड् वै वै षट् थ्षड् वै । \newline
26. वा ऋ॒तव॑ ऋ॒तवो॒ वै वा ऋ॒तवः॑ । \newline
27. ऋ॒तवः॑ संॅवथ्स॒रः सं॑ॅवथ्स॒र ऋ॒तव॑ ऋ॒तवः॑ संॅवथ्स॒रः । \newline
28. सं॒ॅव॒थ्स॒रः सं॑ॅवथ्स॒रः । \newline
29. सं॒ॅव॒थ्स॒र इति॑ सं - व॒थ्स॒रः । \newline
30. सं॒ॅव॒थ्स॒रो᳚ ऽग्नि र॒ग्निः सं॑ॅवथ्स॒रः सं॑ॅवथ्स॒रो᳚ ऽग्निः । \newline
31. सं॒ॅव॒थ्स॒र इति॑ सं - व॒थ्स॒रः । \newline
32. अ॒ग्निर् वै᳚श्वान॒रो वै᳚श्वान॒रो᳚ ऽग्नि र॒ग्निर् वै᳚श्वान॒रः । \newline
33. वै॒श्वा॒न॒रः सा॒क्षाथ् सा॒क्षाद् वै᳚श्वान॒रो वै᳚श्वान॒रः सा॒क्षात् । \newline
34. सा॒क्षा दे॒वैव सा॒क्षाथ् सा॒क्षा दे॒व । \newline
35. सा॒क्षादिति॑ स - अ॒क्षात् । \newline
36. ए॒व वै᳚श्वान॒रं ॅवै᳚श्वान॒र मे॒वैव वै᳚श्वान॒रम् । \newline
37. वै॒श्वा॒न॒र मवाव॑ वैश्वान॒रं ॅवै᳚श्वान॒र मव॑ । \newline
38. अव॑ रुन्धे रु॒न्धे ऽवाव॑ रुन्धे । \newline
39. रु॒न्धे॒ स॒मु॒द्रꣳ स॑मु॒द्रꣳ रु॑न्धे रुन्धे समु॒द्रम् । \newline
40. स॒मु॒द्रं ॅवै वै स॑मु॒द्रꣳ स॑मु॒द्रं ॅवै । \newline
41. वै नाम॒ नाम॒ वै वै नाम॑ । \newline
42. नामै॒त दे॒तन् नाम॒ नामै॒तत् । \newline
43. ए॒तच् छन्द॒ श्छन्द॑ ए॒त दे॒तच् छन्दः॑ । \newline
44. छन्दः॑ समु॒द्रꣳ स॑मु॒द्रम् छन्द॒ श्छन्दः॑ समु॒द्रम् । \newline
45. स॒मु॒द्र मन्वनु॑ समु॒द्रꣳ स॑मु॒द्र मनु॑ । \newline
46. अनु॑ प्र॒जाः प्र॒जा अन्वनु॑ प्र॒जाः । \newline
47. प्र॒जाः प्र प्र प्र॒जाः प्र॒जाः प्र । \newline
48. प्र॒जा इति॑ प्र - जाः । \newline
49. प्र जा॑यन्ते जायन्ते॒ प्र प्र जा॑यन्ते । \newline
50. जा॒य॒न्ते॒ यद् यज् जा॑यन्ते जायन्ते॒ यत् । \newline
51. यदे॒ते नै॒तेन॒ यद् यदे॒तेन॑ । \newline
52. ए॒तेन॒ सिक॑ताः॒ सिक॑ता ए॒तेन् ऐ॒तेन॒ सिक॑ताः । \newline
53. सिक॑ता नि॒वप॑ति नि॒वप॑ति॒ सिक॑ताः॒ सिक॑ता नि॒वप॑ति । \newline
54. नि॒वप॑ति प्र॒जाना᳚म् प्र॒जाना᳚न् नि॒वप॑ति नि॒वप॑ति प्र॒जाना᳚म् । \newline
55. नि॒वप॒तीति॑ नि - वप॑ति । \newline
56. प्र॒जाना᳚म् प्र॒जन॑नाय प्र॒जन॑नाय प्र॒जाना᳚म् प्र॒जाना᳚म् प्र॒जन॑नाय । \newline
57. प्र॒जाना॒मिति॑ प्र - जाना᳚म् । \newline
58. प्र॒जन॑ना॒येन्द्र॒ इन्द्रः॑ प्र॒जन॑नाय प्र॒जन॑ना॒येन्द्रः॑ । \newline
59. प्र॒जन॑ना॒येति॑ प्र - जन॑नाय । \newline
60. इन्द्रो॑ वृ॒त्राय॑ वृ॒त्रायेन्द्र॒ इन्द्रो॑ वृ॒त्राय॑ । \newline

\textbf{Ghana Paata } \newline

1. अग्ने॒ तव॒ तवाग्ने ऽग्ने॒ तव॒ श्रवः॒ श्रव॒ स्तवाग्ने ऽग्ने॒ तव॒ श्रवः॑ । \newline
2. तव॒ श्रवः॒ श्रव॒ स्तव॒ तव॒ श्रवो॒ वयो॒ वयः॒ श्रव॒ स्तव॒ तव॒ श्रवो॒ वयः॑ । \newline
3. श्रवो॒ वयो॒ वयः॒ श्रवः॒ श्रवो॒ वय॒ इतीति॒ वयः॒ श्रवः॒ श्रवो॒ वय॒ इति॑ । \newline
4. वय॒ इतीति॒ वयो॒ वय॒ इति॒ सिक॑ताः॒ सिक॑ता॒ इति॒ वयो॒ वय॒ इति॒ सिक॑ताः । \newline
5. इति॒ सिक॑ताः॒ सिक॑ता॒ इतीति॒ सिक॑ता॒ नि नि सिक॑ता॒ इतीति॒ सिक॑ता॒ नि । \newline
6. सिक॑ता॒ नि नि सिक॑ताः॒ सिक॑ता॒ नि व॑पति वपति॒ नि सिक॑ताः॒ सिक॑ता॒ नि व॑पति । \newline
7. नि व॑पति वपति॒ नि नि व॑प त्ये॒त दे॒तद् व॑पति॒ नि नि व॑प त्ये॒तत् । \newline
8. व॒प॒ त्ये॒त दे॒तद् व॑पति वप त्ये॒तद् वै वा ए॒तद् व॑पति वप त्ये॒तद् वै । \newline
9. ए॒तद् वै वा ए॒त दे॒तद् वा अ॒ग्ने र॒ग्नेर् वा ए॒त दे॒तद् वा अ॒ग्नेः । \newline
10. वा अ॒ग्ने र॒ग्नेर् वै वा अ॒ग्नेर् वै᳚श्वान॒रस्य॑ वैश्वान॒रस्या॒ग्नेर् वै वा अ॒ग्नेर् वै᳚श्वान॒रस्य॑ । \newline
11. अ॒ग्नेर् वै᳚श्वान॒रस्य॑ वैश्वान॒रस्या॒ग्ने र॒ग्नेर् वै᳚श्वान॒रस्य॑ सू॒क्तꣳ सू॒क्तं ॅवै᳚श्वान॒रस्या॒ग्ने र॒ग्नेर् वै᳚श्वान॒रस्य॑ सू॒क्तम् । \newline
12. वै॒श्वा॒न॒रस्य॑ सू॒क्तꣳ सू॒क्तं ॅवै᳚श्वान॒रस्य॑ वैश्वान॒रस्य॑ सू॒क्तꣳ सू॒क्तेन॑ सू॒क्तेन॑ सू॒क्तं ॅवै᳚श्वान॒रस्य॑ वैश्वान॒रस्य॑ सू॒क्तꣳ सू॒क्तेन॑ । \newline
13. सू॒क्तꣳ सू॒क्तेन॑ सू॒क्तेन॑ सू॒क्तꣳ सू॒क्तꣳ सू॒क्तेनै॒वैव सू॒क्तेन॑ सू॒क्तꣳ सू॒क्तꣳ सू॒क्तेनै॒व । \newline
14. सू॒क्तमिति॑ सु - उ॒क्तम् । \newline
15. सू॒क्तेनै॒वैव सू॒क्तेन॑ सू॒क्तेनै॒व वै᳚श्वान॒रं ॅवै᳚श्वान॒र मे॒व सू॒क्तेन॑ सू॒क्तेनै॒व वै᳚श्वान॒रम् । \newline
16. सू॒क्तेनेति॑ सु - उ॒क्तेन॑ । \newline
17. ए॒व वै᳚श्वान॒रं ॅवै᳚श्वान॒र मे॒वैव वै᳚श्वान॒र मवाव॑ वैश्वान॒र मे॒वैव वै᳚श्वान॒र मव॑ । \newline
18. वै॒श्वा॒न॒र मवाव॑ वैश्वान॒रं ॅवै᳚श्वान॒र मव॑ रुन्धे रु॒न्धे ऽव॑ वैश्वान॒रं ॅवै᳚श्वान॒र मव॑ रुन्धे । \newline
19. अव॑ रुन्धे रु॒न्धे ऽवाव॑ रुन्धे ष॒ड्भि ष्ष॒ड्भी रु॒न्धे ऽवाव॑ रुन्धे ष॒ड्भिः । \newline
20. रु॒न्धे॒ ष॒ड्भि ष्ष॒ड्भी रु॑न्धे रुन्धे ष॒ड्भिर् नि नि ष॒ड्भी रु॑न्धे रुन्धे ष॒ड्भिर् नि । \newline
21. ष॒ड्भिर् नि नि ष॒ड्भि ष्ष॒ड्भिर् नि व॑पति वपति॒ नि ष॒ड्भि ष्ष॒ड्भिर् नि व॑पति । \newline
22. ष॒ड्भिरिति॑ षट् - भिः । \newline
23. नि व॑पति वपति॒ नि नि व॑पति॒ षट् थ्षड् व॑पति॒ नि नि व॑पति॒ षट् । \newline
24. व॒प॒ति॒ षट् थ्षड् व॑पति वपति॒ षड् वै वै षड् व॑पति वपति॒ षड् वै । \newline
25. षड् वै वै षट् थ्षड् वा ऋ॒तव॑ ऋ॒तवो॒ वै षट् थ्षड् वा ऋ॒तवः॑ । \newline
26. वा ऋ॒तव॑ ऋ॒तवो॒ वै वा ऋ॒तवः॑ संॅवथ्स॒रः सं॑ॅवथ्स॒र ऋ॒तवो॒ वै वा ऋ॒तवः॑ संॅवथ्स॒रः । \newline
27. ऋ॒तवः॑ संॅवथ्स॒रः सं॑ॅवथ्स॒र ऋ॒तव॑ ऋ॒तवः॑ संॅवथ्स॒रः । \newline
28. सं॒ॅव॒थ्स॒रः सं॑ॅवथ्स॒रः । \newline
29. सं॒ॅव॒थ्स॒र इति॑ सं - व॒थ्स॒रः । \newline
30. सं॒ॅव॒थ्स॒रो᳚ ऽग्नि र॒ग्निः सं॑ॅवथ्स॒रः सं॑ॅवथ्स॒रो᳚ ऽग्निर् वै᳚श्वान॒रो वै᳚श्वान॒रो᳚ ऽग्निः सं॑ॅवथ्स॒रः सं॑ॅवथ्स॒रो᳚ ऽग्निर् वै᳚श्वान॒रः । \newline
31. सं॒ॅव॒थ्स॒र इति॑ सं - व॒थ्स॒रः । \newline
32. अ॒ग्निर् वै᳚श्वान॒रो वै᳚श्वान॒रो᳚ ऽग्नि र॒ग्निर् वै᳚श्वान॒रः सा॒क्षाथ् सा॒क्षाद् वै᳚श्वान॒रो᳚ ऽग्नि र॒ग्निर् वै᳚श्वान॒रः सा॒क्षात् । \newline
33. वै॒श्वा॒न॒रः सा॒क्षाथ् सा॒क्षाद् वै᳚श्वान॒रो वै᳚श्वान॒रः सा॒क्षा दे॒वैव सा॒क्षाद् वै᳚श्वान॒रो वै᳚श्वान॒रः सा॒क्षादे॒व । \newline
34. सा॒क्षा दे॒वैव सा॒क्षाथ् सा॒क्षादे॒व वै᳚श्वान॒रं ॅवै᳚श्वान॒र मे॒व सा॒क्षाथ् सा॒क्षा दे॒व वै᳚श्वान॒रम् । \newline
35. सा॒क्षादिति॑ स - अ॒क्षात् । \newline
36. ए॒व वै᳚श्वान॒रं ॅवै᳚श्वान॒र मे॒वैव वै᳚श्वान॒र मवाव॑ वैश्वान॒र मे॒वैव वै᳚श्वान॒र मव॑ । \newline
37. वै॒श्वा॒न॒र मवाव॑ वैश्वान॒रं ॅवै᳚श्वान॒र मव॑ रुन्धे रु॒न्धे ऽव॑ वैश्वान॒रं ॅवै᳚श्वान॒र मव॑ रुन्धे । \newline
38. अव॑ रुन्धे रु॒न्धे ऽवाव॑ रुन्धे समु॒द्रꣳ स॑मु॒द्रꣳ रु॒न्धे ऽवाव॑ रुन्धे समु॒द्रम् । \newline
39. रु॒न्धे॒ स॒मु॒द्रꣳ स॑मु॒द्रꣳ रु॑न्धे रुन्धे समु॒द्रं ॅवै वै स॑मु॒द्रꣳ रु॑न्धे रुन्धे समु॒द्रं ॅवै । \newline
40. स॒मु॒द्रं ॅवै वै स॑मु॒द्रꣳ स॑मु॒द्रं ॅवै नाम॒ नाम॒ वै स॑मु॒द्रꣳ स॑मु॒द्रं ॅवै नाम॑ । \newline
41. वै नाम॒ नाम॒ वै वै नामै॒त दे॒तन् नाम॒ वै वै नामै॒तत् । \newline
42. नामै॒त दे॒तन् नाम॒ नामै॒तच् छन्द॒ श्छन्द॑ ए॒तन् नाम॒ नामै॒तच् छन्दः॑ । \newline
43. ए॒तच् छन्द॒ श्छन्द॑ ए॒त दे॒तच् छन्दः॑ समु॒द्रꣳ स॑मु॒द्रम् छन्द॑ ए॒त दे॒तच् छन्दः॑ समु॒द्रम् । \newline
44. छन्दः॑ समु॒द्रꣳ स॑मु॒द्रम् छन्द॒ श्छन्दः॑ समु॒द्र मन्वनु॑ समु॒द्रम् छन्द॒ श्छन्दः॑ समु॒द्र मनु॑ । \newline
45. स॒मु॒द्र मन्वनु॑ समु॒द्रꣳ स॑मु॒द्र मनु॑ प्र॒जाः प्र॒जा अनु॑ समु॒द्रꣳ स॑मु॒द्र मनु॑ प्र॒जाः । \newline
46. अनु॑ प्र॒जाः प्र॒जा अन्वनु॑ प्र॒जाः प्र प्र प्र॒जा अन्वनु॑ प्र॒जाः प्र । \newline
47. प्र॒जाः प्र प्र प्र॒जाः प्र॒जाः प्र जा॑यन्ते जायन्ते॒ प्र प्र॒जाः प्र॒जाः प्र जा॑यन्ते । \newline
48. प्र॒जा इति॑ प्र - जाः । \newline
49. प्र जा॑यन्ते जायन्ते॒ प्र प्र जा॑यन्ते॒ यद् यज् जा॑यन्ते॒ प्र प्र जा॑यन्ते॒ यत् । \newline
50. जा॒य॒न्ते॒ यद् यज् जा॑यन्ते जायन्ते॒ यदे॒ते नै॒तेन॒ यज् जा॑यन्ते जायन्ते॒ यदे॒तेन॑ । \newline
51. यदे॒ते नै॒तेन॒ यद् यदे॒तेन॒ सिक॑ताः॒ सिक॑ता ए॒तेन॒ यद् यदे॒तेन॒ सिक॑ताः । \newline
52. ए॒तेन॒ सिक॑ताः॒ सिक॑ता ए॒ते नै॒तेन॒ सिक॑ता नि॒वप॑ति नि॒वप॑ति॒ सिक॑ता ए॒ते नै॒तेन॒ सिक॑ता नि॒वप॑ति । \newline
53. सिक॑ता नि॒वप॑ति नि॒वप॑ति॒ सिक॑ताः॒ सिक॑ता नि॒वप॑ति प्र॒जाना᳚म् प्र॒जाना᳚म् नि॒वप॑ति॒ सिक॑ताः॒ सिक॑ता नि॒वप॑ति प्र॒जाना᳚म् । \newline
54. नि॒वप॑ति प्र॒जाना᳚म् प्र॒जाना᳚म् नि॒वप॑ति नि॒वप॑ति प्र॒जाना᳚म् प्र॒जन॑नाय प्र॒जन॑नाय प्र॒जाना᳚म् नि॒वप॑ति नि॒वप॑ति प्र॒जाना᳚म् प्र॒जन॑नाय । \newline
55. नि॒वप॒तीति॑ नि - वप॑ति । \newline
56. प्र॒जाना᳚म् प्र॒जन॑नाय प्र॒जन॑नाय प्र॒जाना᳚म् प्र॒जाना᳚म् प्र॒जन॑ना॒येन्द्र॒ इन्द्रः॑ प्र॒जन॑नाय प्र॒जाना᳚म् प्र॒जाना᳚म् प्र॒जन॑ना॒येन्द्रः॑ । \newline
57. प्र॒जाना॒मिति॑ प्र - जाना᳚म् । \newline
58. प्र॒जन॑ना॒येन्द्र॒ इन्द्रः॑ प्र॒जन॑नाय प्र॒जन॑ना॒येन्द्रो॑ वृ॒त्राय॑ वृ॒त्रायेन्द्रः॑ प्र॒जन॑नाय प्र॒जन॑ना॒येन्द्रो॑ वृ॒त्राय॑ । \newline
59. प्र॒जन॑ना॒येति॑ प्र - जन॑नाय । \newline
60. इन्द्रो॑ वृ॒त्राय॑ वृ॒त्रायेन्द्र॒ इन्द्रो॑ वृ॒त्राय॒ वज्रं॒ ॅवज्रं॑ ॅवृ॒त्रायेन्द्र॒ इन्द्रो॑ वृ॒त्राय॒ वज्र᳚म् । \newline
\pagebreak
\markright{ TS 5.2.6.2  \hfill https://www.vedavms.in \hfill}

\section{ TS 5.2.6.2 }

\textbf{TS 5.2.6.2 } \newline
\textbf{Samhita Paata} \newline

वृ॒त्राय॒ वज्रं॒ प्राह॑र॒थ् स त्रे॒धा व्य॑भव॒थ् स्फ्यस्तृती॑यꣳ॒॒ रथ॒स्तृती॑यं॒ ॅयूप॒स्तृती॑यं॒ ॅये᳚ऽन्तश्श॒रा अशी᳚र्यन्त॒ ताः शर्क॑रा अभव॒न् तच्छर्क॑राणाꣳ शर्कर॒त्वं ॅवज्रो॒ वै शर्क॑राः प॒शुर॒ग्नि-र्यच्छर्क॑राभिर॒ग्निं प॑रिमि॒नोति॒ वज्रे॑णै॒वास्मै॑ प॒शून् परि॑ गृह्णाति॒ तस्मा॒द्-वज्रे॑ण प॒शवः॒ परि॑गृहीता॒स्तस्मा॒थ् स्थेया॒नस्थे॑यसो॒ नोप॑ हरते त्रिस॒प्ताभिः॑ प॒शुका॑मस्य॒ - [  ] \newline

\textbf{Pada Paata} \newline

वृ॒त्राय॑ । वज्र᳚म् । प्रेति॑ । अ॒ह॒र॒त् । सः । त्रे॒धा । वीति॑ । अ॒भ॒व॒त् । स्फ्यः । तृती॑यम् । रथः॑ । तृती॑यम् । यूपः॑ । तृती॑यम् । ये । अ॒न्त॒श्श॒रा इत्य॑न्तः-श॒राः । अशी᳚र्यन्त । ताः । शर्क॑राः । अ॒भ॒व॒न्न् । तत् । शर्क॑राणाम् । श॒र्क॒र॒त्वमिति॑ शर्कर - त्वम् । वज्रः॑ । वै । शर्क॑राः । प॒शुः । अ॒ग्निः । यत् । शर्क॑राभिः । अ॒ग्निम् । प॒रि॒मि॒नोतीति॑ परि - मि॒नोति॑ । वज्रे॑ण । ए॒व । अ॒स्मै॒ । प॒शून् । परीति॑ । गृ॒ह्णा॒ति॒ । तस्मा᳚त् । वज्रे॑ण । प॒शवः॑ । परि॑गृहीता॒ इति॒ परि॑ - गृ॒ही॒ताः॒ । तस्मा᳚त् । स्थेयान्॑ । अस्थे॑यसः । न । उपेति॑ । ह॒र॒ते॒ । त्रि॒स॒प्ताभि॒रिति॑ त्रि - स॒प्ताभिः॑ । प॒शुका॑म॒स्येति॑ प॒शु - का॒म॒स्य॒ ।  \newline


\textbf{Krama Paata} \newline

वृ॒त्राय॒ वज्र᳚म् । वज्र॒म् प्र । प्राह॑रत् । अ॒ह॒र॒थ् सः । स त्रे॒धा । त्रे॒धा वि । व्य॑भवत् । अ॒भ॒व॒थ् स्फ्यः । स्फ्यस्तृती॑यम् । तृती॑यꣳ॒॒ रथः॑ । रथ॒स्तृती॑यम् । तृती॑य॒म् ॅयूपः॑ । यूप॒स्तृती॑यम् । तृती॑य॒म् ॅये । ये᳚ऽन्तश्श॒राः । अ॒न्त॒श्श॒रा अशी᳚र्यन्त । अ॒न्त॒श्श॒रा इत्य॑न्तः - श॒राः । अशी᳚र्यन्त॒ ताः । ताः शर्क॑राः । शर्क॑रा अभवन्न् । अ॒भ॒व॒न् तत् । तच्छर्क॑राणाम् । शर्क॑राणाꣳ शर्कर॒त्वम् । श॒र्क॒र॒त्वम् ॅवज्रः॑ । श॒र्क॒र॒त्वमिति॑ शर्कर - त्वम् । वज्रो॒ वै । वै शर्क॑राः । शर्क॑राः प॒शुः । प॒शुर॒ग्निः । अ॒ग्निर् यत् । यच्छर्क॑राभिः । शर्क॑राभिर॒ग्निम् । अ॒ग्निम् प॑रिमि॒नोति॑ । प॒रि॒मि॒नोति॒ वज्रे॑ण । प॒रि॒मि॒नोतीति॑ परि - मि॒नोति॑ । वज्रे॑णै॒व । ए॒वास्मै᳚ । अ॒स्मै॒ प॒शून् । प॒शून् परि॑ । परि॑ गृह्णाति । गृ॒ह्णा॒ति॒ तस्मा᳚त् । तस्मा॒द् वज्रे॑ण । वज्रे॑ण प॒शवः॑ । प॒शवः॒ परि॑गृहीताः । परि॑गृहीता॒स्तस्मा᳚त् । परि॑गृहीता॒ इति॒ परि॑ - गृ॒ही॒ताः॒ । तस्मा॒थ् स्थेयान्॑ । स्थेया॒नस्थे॑यसः । अस्थे॑यसो॒ न । नोप॑ । उप॑ हरते । ह॒र॒ते॒ त्रि॒स॒प्ताभिः॑ । त्रि॒स॒प्ताभिः॑ प॒शुका॑मस्य । त्रि॒स॒प्ताभि॒रिति॑ त्रि - स॒प्ताभिः॑ । प॒शुका॑मस्य॒ परि॑ । प॒शुका॑म॒स्येति॑ प॒शु - का॒म॒स्य॒ \newline

\textbf{Jatai Paata} \newline

1. वृ॒त्राय॒ वज्रं॒ ॅवज्रं॑ ॅवृ॒त्राय॑ वृ॒त्राय॒ वज्र᳚म् । \newline
2. वज्र॒म् प्र प्र वज्रं॒ ॅवज्र॒म् प्र । \newline
3. प्राह॑र दहर॒त् प्र प्राह॑रत् । \newline
4. अ॒ह॒र॒थ् स सो॑ ऽहर दहर॒थ् सः । \newline
5. स त्रे॒धा त्रे॒धा स स त्रे॒धा । \newline
6. त्रे॒धा वि वि त्रे॒धा त्रे॒धा वि । \newline
7. व्य॑भव दभव॒द् वि व्य॑भवत् । \newline
8. अ॒भ॒व॒थ् स्फ्यः स्फ्यो॑ ऽभव दभव॒थ् स्फ्यः । \newline
9. स्फ्य स्तृती॑य॒म् तृती॑यꣳ॒॒ स्फ्यः स्फ्य स्तृती॑यम् । \newline
10. तृती॑यꣳ॒॒ रथो॒ रथ॒ स्तृती॑य॒म् तृती॑यꣳ॒॒ रथः॑ । \newline
11. रथ॒ स्तृती॑य॒म् तृती॑यꣳ॒॒ रथो॒ रथ॒ स्तृती॑यम् । \newline
12. तृती॑यं॒ ॅयूपो॒ यूप॒ स्तृती॑य॒म् तृती॑यं॒ ॅयूपः॑ । \newline
13. यूप॒ स्तृती॑य॒म् तृती॑यं॒ ॅयूपो॒ यूप॒ स्तृती॑यम् । \newline
14. तृती॑यं॒ ॅये ये तृती॑य॒म् तृती॑यं॒ ॅये । \newline
15. ये᳚ ऽन्तश्श॒रा अ॑न्तश्श॒रा ये ये᳚ ऽन्तश्श॒राः । \newline
16. अ॒न्त॒श्श॒रा अशी᳚र्य॒न् ताशी᳚र्यन् तान्तश्श॒रा अ॑न्तश्श॒रा अशी᳚र्यन्त । \newline
17. अ॒न्त॒श्श॒रा इत्य॑न्तः - श॒राः । \newline
18. अशी᳚र्यन्त॒ ता स्ता अशी᳚र्य॒न् ताशी᳚र्यन्त॒ ताः । \newline
19. ताः शर्क॑राः॒ शर्क॑रा॒ स्ता स्ताः शर्क॑राः । \newline
20. शर्क॑रा अभवन् नभव॒ञ् छर्क॑राः॒ शर्क॑रा अभवन्न् । \newline
21. अ॒भ॒व॒न् तत् तद॑भवन् नभव॒न् तत् । \newline
22. तच्छर्क॑राणाꣳ॒॒ शर्क॑राणा॒म् तत् तच्छर्क॑राणाम् । \newline
23. शर्क॑राणाꣳ शर्कर॒त्वꣳ श॑र्कर॒त्वꣳ शर्क॑राणाꣳ॒॒ शर्क॑राणाꣳ शर्कर॒त्वम् । \newline
24. श॒र्क॒र॒त्वं ॅवज्रो॒ वज्रः॑ शर्कर॒त्वꣳ श॑र्कर॒त्वं ॅवज्रः॑ । \newline
25. श॒र्क॒र॒त्वमिति॑ शर्कर - त्वम् । \newline
26. वज्रो॒ वै वै वज्रो॒ वज्रो॒ वै । \newline
27. वै शर्क॑राः॒ शर्क॑रा॒ वै वै शर्क॑राः । \newline
28. शर्क॑राः प॒शुः प॒शुः शर्क॑राः॒ शर्क॑राः प॒शुः । \newline
29. प॒शु र॒ग्नि र॒ग्निः प॒शुः प॒शु र॒ग्निः । \newline
30. अ॒ग्निर् यद् यद॒ग्नि र॒ग्निर् यत् । \newline
31. यच्छर्क॑राभिः॒ शर्क॑राभि॒र् यद् यच्छर्क॑राभिः । \newline
32. शर्क॑राभि र॒ग्नि म॒ग्निꣳ शर्क॑राभिः॒ शर्क॑राभि र॒ग्निम् । \newline
33. अ॒ग्निम् प॑रिमि॒नोति॑ परिमि॒नो त्य॒ग्नि म॒ग्निम् प॑रिमि॒नोति॑ । \newline
34. प॒रि॒मि॒नोति॒ वज्रे॑ण॒ वज्रे॑ण परिमि॒नोति॑ परिमि॒नोति॒ वज्रे॑ण । \newline
35. प॒रि॒मि॒नोतीति॑ परि - मि॒नोति॑ । \newline
36. वज्रे॑ णै॒वैव वज्रे॑ण॒ वज्रे॑णै॒व । \newline
37. ए॒वास्मा॑ अस्मा ए॒वै वास्मै᳚ । \newline
38. अ॒स्मै॒ प॒शून् प॒शू न॑स्मा अस्मै प॒शून् । \newline
39. प॒शून् परि॒ परि॑ प॒शून् प॒शून् परि॑ । \newline
40. परि॑ गृह्णाति गृह्णाति॒ परि॒ परि॑ गृह्णाति । \newline
41. गृ॒ह्णा॒ति॒ तस्मा॒त् तस्मा᳚द् गृह्णाति गृह्णाति॒ तस्मा᳚त् । \newline
42. तस्मा॒द् वज्रे॑ण॒ वज्रे॑ण॒ तस्मा॒त् तस्मा॒द् वज्रे॑ण । \newline
43. वज्रे॑ण प॒शवः॑ प॒शवो॒ वज्रे॑ण॒ वज्रे॑ण प॒शवः॑ । \newline
44. प॒शवः॒ परि॑गृहीताः॒ परि॑गृहीताः प॒शवः॑ प॒शवः॒ परि॑गृहीताः । \newline
45. परि॑गृहीता॒ स्तस्मा॒त् तस्मा॒त् परि॑गृहीताः॒ परि॑गृहीता॒ स्तस्मा᳚त् । \newline
46. परि॑गृहीता॒ इति॒ परि॑ - गृ॒ही॒ताः॒ । \newline
47. तस्मा॒थ् स्थेया॒न् थ्स्थेया॒न् तस्मा॒त् तस्मा॒थ् स्थेयान्॑ । \newline
48. स्थेया॒ नस्थे॑य॒सो ऽस्थे॑यसः॒ स्थेया॒न् थ्स्थेया॒ नस्थे॑यसः । \newline
49. अस्थे॑यसो॒ न नास्थे॑य॒सो ऽस्थे॑यसो॒ न । \newline
50. नोपोप॒ न नोप॑ । \newline
51. उप॑ हरते हरत॒ उपोप॑ हरते । \newline
52. ह॒र॒ते॒ त्रि॒स॒प्ताभि॑ स्त्रिस॒प्ताभिर्॑. हरते हरते त्रिस॒प्ताभिः॑ । \newline
53. त्रि॒स॒प्ताभिः॑ प॒शुका॑मस्य प॒शुका॑मस्य त्रिस॒प्ताभि॑ स्त्रिस॒प्ताभिः॑ प॒शुका॑मस्य । \newline
54. त्रि॒स॒प्ताभि॒रिति॑ त्रि - स॒प्ताभिः॑ । \newline
55. प॒शुका॑मस्य॒ परि॒ परि॑ प॒शुका॑मस्य प॒शुका॑मस्य॒ परि॑ । \newline
56. प॒शुका॑म॒स्येति॑ प॒शु - का॒म॒स्य॒ । \newline

\textbf{Ghana Paata } \newline

1. वृ॒त्राय॒ वज्रं॒ ॅवज्रं॑ ॅवृ॒त्राय॑ वृ॒त्राय॒ वज्र॒म् प्र प्र वज्रं॑ ॅवृ॒त्राय॑ वृ॒त्राय॒ वज्र॒म् प्र । \newline
2. वज्र॒म् प्र प्र वज्रं॒ ॅवज्र॒म् प्राह॑र दहर॒त् प्र वज्रं॒ ॅवज्र॒म् प्राह॑रत् । \newline
3. प्राह॑र दहर॒त् प्र प्राह॑र॒थ् स सो॑ ऽहर॒त् प्र प्राह॑र॒थ् सः । \newline
4. अ॒ह॒र॒थ् स सो॑ ऽहर दहर॒थ् स त्रे॒धा त्रे॒धा सो॑ ऽहर दहर॒थ् स त्रे॒धा । \newline
5. स त्रे॒धा त्रे॒धा स स त्रे॒धा वि वि त्रे॒धा स स त्रे॒धा वि । \newline
6. त्रे॒धा वि वि त्रे॒धा त्रे॒धा व्य॑भव दभव॒द् वि त्रे॒धा त्रे॒धा व्य॑भवत् । \newline
7. व्य॑भव दभव॒द् वि व्य॑भव॒थ् स्फ्यः स्फ्यो॑ ऽभव॒द् वि व्य॑भव॒थ् स्फ्यः । \newline
8. अ॒भ॒व॒थ् स्फ्यः स्फ्यो॑ ऽभव दभव॒थ् स्फ्य स्तृती॑य॒म् तृती॑यꣳ॒॒ स्फ्यो॑ ऽभव दभव॒थ् स्फ्य स्तृती॑यम् । \newline
9. स्फ्य स्तृती॑य॒म् तृती॑यꣳ॒॒ स्फ्यः स्फ्य स्तृती॑यꣳ॒॒ रथो॒ रथ॒ स्तृती॑यꣳ॒॒ स्फ्यः स्फ्य स्तृती॑यꣳ॒॒ रथः॑ । \newline
10. तृती॑यꣳ॒॒ रथो॒ रथ॒ स्तृती॑य॒म् तृती॑यꣳ॒॒ रथ॒ स्तृती॑य॒म् तृती॑यꣳ॒॒ रथ॒ स्तृती॑य॒म् तृती॑यꣳ॒॒ रथ॒ स्तृती॑यम् । \newline
11. रथ॒ स्तृती॑य॒म् तृती॑यꣳ॒॒ रथो॒ रथ॒ स्तृती॑यं॒ ॅयूपो॒ यूप॒ स्तृती॑यꣳ॒॒ रथो॒ रथ॒ स्तृती॑यं॒ ॅयूपः॑ । \newline
12. तृती॑यं॒ ॅयूपो॒ यूप॒ स्तृती॑य॒म् तृती॑यं॒ ॅयूप॒ स्तृती॑य॒म् तृती॑यं॒ ॅयूप॒ स्तृती॑य॒म् तृती॑यं॒ ॅयूप॒ स्तृती॑यम् । \newline
13. यूप॒ स्तृती॑य॒म् तृती॑यं॒ ॅयूपो॒ यूप॒ स्तृती॑यं॒ ॅये ये तृती॑यं॒ ॅयूपो॒ यूप॒ स्तृती॑यं॒ ॅये । \newline
14. तृती॑यं॒ ॅये ये तृती॑य॒म् तृती॑यं॒ ॅये᳚ ऽन्तश्श॒रा अ॑न्तश्श॒रा ये तृती॑य॒म् तृती॑यं॒ ॅये᳚ ऽन्तश्श॒राः । \newline
15. ये᳚ ऽन्तश्श॒रा अ॑न्तश्श॒रा ये ये᳚ ऽन्तश्श॒रा अशी᳚र्य॒न्ता शी᳚र्यन्ता न्तश्श॒रा ये ये᳚ ऽन्तश्श॒रा अशी᳚र्यन्त । \newline
16. अ॒न्त॒श्श॒रा अशी᳚र्य॒न्ता शी᳚र्यन्ता न्तश्श॒रा अ॑न्तश्श॒रा अशी᳚र्यन्त॒ ता स्ता अशी᳚र्यन्ता न्तश्श॒रा अ॑न्तश्श॒रा अशी᳚र्यन्त॒ ताः । \newline
17. अ॒न्त॒श्श॒रा इत्य॑न्तः - श॒राः । \newline
18. अशी᳚र्यन्त॒ ता स्ता अशी᳚र्य॒न्ता शी᳚र्यन्त॒ ताः शर्क॑राः॒ शर्क॑रा॒ स्ता अशी᳚र्य॒न्ता शी᳚र्यन्त॒ ताः शर्क॑राः । \newline
19. ताः शर्क॑राः॒ शर्क॑रा॒ स्ता स्ताः शर्क॑रा अभवन् नभव॒ञ् छर्क॑रा॒ स्ता स्ताः शर्क॑रा अभवन्न् । \newline
20. शर्क॑रा अभवन् नभव॒ञ् छर्क॑राः॒ शर्क॑रा अभव॒न् तत् तद॑भव॒ञ् छर्क॑राः॒ शर्क॑रा अभव॒न् तत् । \newline
21. अ॒भ॒व॒न् तत् तद॑भवन् नभव॒न् तच्छर्क॑राणाꣳ॒॒ शर्क॑राणा॒म् तद॑भवन् नभव॒न् तच्छर्क॑राणाम् । \newline
22. तच्छर्क॑राणाꣳ॒॒ शर्क॑राणा॒म् तत् तच्छर्क॑राणाꣳ शर्कर॒त्वꣳ श॑र्कर॒त्वꣳ शर्क॑राणा॒म् तत् तच्छर्क॑राणाꣳ शर्कर॒त्वम् । \newline
23. शर्क॑राणाꣳ शर्कर॒त्वꣳ श॑र्कर॒त्वꣳ शर्क॑राणाꣳ॒॒ शर्क॑राणाꣳ शर्कर॒त्वं ॅवज्रो॒ वज्रः॑ शर्कर॒त्वꣳ शर्क॑राणाꣳ॒॒ शर्क॑राणाꣳ शर्कर॒त्वं ॅवज्रः॑ । \newline
24. श॒र्क॒र॒त्वं ॅवज्रो॒ वज्रः॑ शर्कर॒त्वꣳ श॑र्कर॒त्वं ॅवज्रो॒ वै वै वज्रः॑ शर्कर॒त्वꣳ श॑र्कर॒त्वं ॅवज्रो॒ वै । \newline
25. श॒र्क॒र॒त्वमिति॑ शर्कर - त्वम् । \newline
26. वज्रो॒ वै वै वज्रो॒ वज्रो॒ वै शर्क॑राः॒ शर्क॑रा॒ वै वज्रो॒ वज्रो॒ वै शर्क॑राः । \newline
27. वै शर्क॑राः॒ शर्क॑रा॒ वै वै शर्क॑राः प॒शुः प॒शुः शर्क॑रा॒ वै वै शर्क॑राः प॒शुः । \newline
28. शर्क॑राः प॒शुः प॒शुः शर्क॑राः॒ शर्क॑राः प॒शु र॒ग्नि र॒ग्निः प॒शुः शर्क॑राः॒ शर्क॑राः प॒शु र॒ग्निः । \newline
29. प॒शु र॒ग्नि र॒ग्निः प॒शुः प॒शु र॒ग्निर् यद् यद॒ग्निः प॒शुः प॒शु र॒ग्निर् यत् । \newline
30. अ॒ग्निर् यद् यद॒ग्नि र॒ग्निर् यच्छर्क॑राभिः॒ शर्क॑राभि॒र् यद॒ग्नि र॒ग्निर् यच्छर्क॑राभिः । \newline
31. यच्छर्क॑राभिः॒ शर्क॑राभि॒र् यद् यच्छर्क॑राभि र॒ग्नि म॒ग्निꣳ शर्क॑राभि॒र् यद् यच्छर्क॑राभि र॒ग्निम् । \newline
32. शर्क॑राभिर॒ग्नि म॒ग्निꣳ शर्क॑राभिः॒ शर्क॑राभि र॒ग्निम् प॑रिमि॒नोति॑ परिमि॒नो त्य॒ग्निꣳ शर्क॑राभिः॒ शर्क॑राभि र॒ग्निम् प॑रिमि॒नोति॑ । \newline
33. अ॒ग्निम् प॑रिमि॒नोति॑ परिमि॒नो त्य॒ग्नि म॒ग्निम् प॑रिमि॒नोति॒ वज्रे॑ण॒ वज्रे॑ण परिमि॒नो त्य॒ग्नि म॒ग्निम् प॑रिमि॒नोति॒ वज्रे॑ण । \newline
34. प॒रि॒मि॒नोति॒ वज्रे॑ण॒ वज्रे॑ण परिमि॒नोति॑ परिमि॒नोति॒ वज्रे॑णै॒वैव वज्रे॑ण परिमि॒नोति॑ परिमि॒नोति॒ वज्रे॑णै॒व । \newline
35. प॒रि॒मि॒नोतीति॑ परि - मि॒नोति॑ । \newline
36. वज्रे॑णै॒वैव वज्रे॑ण॒ वज्रे॑णै॒वास्मा॑ अस्मा ए॒व वज्रे॑ण॒ वज्रे॑णै॒वास्मै᳚ । \newline
37. ए॒वास्मा॑ अस्मा ए॒वैवास्मै॑ प॒शून् प॒शू न॑स्मा ए॒वैवास्मै॑ प॒शून् । \newline
38. अ॒स्मै॒ प॒शून् प॒शू न॑स्मा अस्मै प॒शून् परि॒ परि॑ प॒शू न॑स्मा अस्मै प॒शून् परि॑ । \newline
39. प॒शून् परि॒ परि॑ प॒शून् प॒शून् परि॑ गृह्णाति गृह्णाति॒ परि॑ प॒शून् प॒शून् परि॑ गृह्णाति । \newline
40. परि॑ गृह्णाति गृह्णाति॒ परि॒ परि॑ गृह्णाति॒ तस्मा॒त् तस्मा᳚द् गृह्णाति॒ परि॒ परि॑ गृह्णाति॒ तस्मा᳚त् । \newline
41. गृ॒ह्णा॒ति॒ तस्मा॒त् तस्मा᳚द् गृह्णाति गृह्णाति॒ तस्मा॒द् वज्रे॑ण॒ वज्रे॑ण॒ तस्मा᳚द् गृह्णाति गृह्णाति॒ तस्मा॒द् वज्रे॑ण । \newline
42. तस्मा॒द् वज्रे॑ण॒ वज्रे॑ण॒ तस्मा॒त् तस्मा॒द् वज्रे॑ण प॒शवः॑ प॒शवो॒ वज्रे॑ण॒ तस्मा॒त् तस्मा॒द् वज्रे॑ण प॒शवः॑ । \newline
43. वज्रे॑ण प॒शवः॑ प॒शवो॒ वज्रे॑ण॒ वज्रे॑ण प॒शवः॒ परि॑गृहीताः॒ परि॑गृहीताः प॒शवो॒ वज्रे॑ण॒ वज्रे॑ण प॒शवः॒ परि॑गृहीताः । \newline
44. प॒शवः॒ परि॑गृहीताः॒ परि॑गृहीताः प॒शवः॑ प॒शवः॒ परि॑गृहीता॒ स्तस्मा॒त् तस्मा॒त् परि॑गृहीताः प॒शवः॑ प॒शवः॒ परि॑गृहीता॒ स्तस्मा᳚त् । \newline
45. परि॑गृहीता॒ स्तस्मा॒त् तस्मा॒त् परि॑गृहीताः॒ परि॑गृहीता॒ स्तस्मा॒थ् स्थेया॒न् थ्स्थेया॒न् तस्मा॒त् परि॑गृहीताः॒ परि॑गृहीता॒ स्तस्मा॒थ् स्थेयान्॑ । \newline
46. परि॑गृहीता॒ इति॒ परि॑ - गृ॒ही॒ताः॒ । \newline
47. तस्मा॒थ् स्थेया॒न् थ्स्थेया॒न् तस्मा॒त् तस्मा॒थ् स्थेया॒ नस्थे॑य॒सो ऽस्थे॑यसः॒ स्थेया॒न् तस्मा॒त् तस्मा॒थ् स्थेया॒ नस्थे॑यसः । \newline
48. स्थेया॒ नस्थे॑य॒सो ऽस्थे॑यसः॒ स्थेया॒न् थ्स्थेया॒ नस्थे॑यसो॒ न नास्थे॑यसः॒ स्थेया॒न् थ्स्थेया॒ नस्थे॑यसो॒ न । \newline
49. अस्थे॑यसो॒ न नास्थे॑य॒सो ऽस्थे॑यसो॒ नोपोप॒ नास्थे॑य॒सो ऽस्थे॑यसो॒ नोप॑ । \newline
50. नोपोप॒ न नोप॑ हरते हरत॒ उप॒ न नोप॑ हरते । \newline
51. उप॑ हरते हरत॒ उपोप॑ हरते त्रिस॒प्ताभि॑ स्त्रिस॒प्ताभिर्॑. हरत॒ उपोप॑ हरते त्रिस॒प्ताभिः॑ । \newline
52. ह॒र॒ते॒ त्रि॒स॒प्ताभि॑ स्त्रिस॒प्ताभिर्॑. हरते हरते त्रिस॒प्ताभिः॑ प॒शुका॑मस्य प॒शुका॑मस्य त्रिस॒प्ताभिर्॑. हरते हरते त्रिस॒प्ताभिः॑ प॒शुका॑मस्य । \newline
53. त्रि॒स॒प्ताभिः॑ प॒शुका॑मस्य प॒शुका॑मस्य त्रिस॒प्ताभि॑ स्त्रिस॒प्ताभिः॑ प॒शुका॑मस्य॒ परि॒ परि॑ प॒शुका॑मस्य त्रिस॒प्ताभि॑ स्त्रिस॒प्ताभिः॑ प॒शुका॑मस्य॒ परि॑ । \newline
54. त्रि॒स॒प्ताभि॒रिति॑ त्रि - स॒प्ताभिः॑ । \newline
55. प॒शुका॑मस्य॒ परि॒ परि॑ प॒शुका॑मस्य प॒शुका॑मस्य॒ परि॑ मिनुयान् मिनुया॒त् परि॑ प॒शुका॑मस्य प॒शुका॑मस्य॒ परि॑ मिनुयात् । \newline
56. प॒शुका॑म॒स्येति॑ प॒शु - का॒म॒स्य॒ । \newline
\pagebreak
\markright{ TS 5.2.6.3  \hfill https://www.vedavms.in \hfill}

\section{ TS 5.2.6.3 }

\textbf{TS 5.2.6.3 } \newline
\textbf{Samhita Paata} \newline

परि॑ मिनुयाथ् स॒प्त वै शी॑र्.ष॒ण्याः᳚ प्रा॒णाः प्रा॒णाः प॒शवः॑ प्रा॒णैरे॒वास्मै॑ प॒शूनव॑ रुन्धे त्रिण॒वाभि॒-र्भ्रातृ॑व्यवत-स्त्रि॒वृत॑मे॒व वज्रꣳ॑ स॒भृंत्य॒ भ्रातृ॑व्याय॒ प्रह॑रति॒ स्तृत्या॒ अप॑रिमिताभिः॒ परि॑ मिनुया॒-दप॑रिमित॒स्या-व॑रुद्ध्यै॒ यं का॒मये॑ताप॒शुः स्या॒दित्यप॑रिमित्य॒ तस्य॒ शर्क॑राः॒ सिक॑ता॒ व्यू॑हे॒दप॑रिगृहीत ए॒वास्य॑ विषू॒चीनꣳ॒॒ रेतः॒ परा॑ सिञ्चत्यप॒शुरे॒व भ॑वति॒ - [  ] \newline

\textbf{Pada Paata} \newline

परीति॑ । मि॒नु॒या॒त् । स॒प्त । वै । शी॒र्॒.ष॒ण्याः᳚ । प्रा॒णा इति॑ प्र-अ॒नाः । प्रा॒णा इति॑ प्र - अ॒नाः । प॒शवः॑ । प्रा॒णैरिति॑ प्र - अ॒नैः । ए॒व । अ॒स्मै॒ । प॒शून् । अवेति॑ । रु॒न्धे॒ । त्रि॒ण॒वाभि॒रिति॑ त्रि - न॒वाभिः॑ । भ्रातृ॑व्यवत॒ इति॒ भ्रातृ॑व्य - व॒तः॒ । त्रि॒वृत॒मिति॑ त्रि - वृत᳚म् । ए॒व । वज्र᳚म् । स॒भृंत्येति॑ सं - भृत्य॑ । भ्रातृ॑व्याय । प्रेति॑ । ह॒र॒ति॒ । स्तृत्यै᳚ । अप॑रिमिताभि॒रित्यप॑रि - मि॒ता॒भिः॒ । परीति॑ । मि॒नु॒या॒त् । अप॑रिमित॒स्येत्यप॑रि - मि॒त॒स्य॒ । अव॑रुद्ध्या॒ इत्यव॑-रु॒द्ध्यै॒ । यम् । का॒मये॑त । अ॒प॒शुः । स्या॒त् । इति॑ । अप॑रिमि॒त्येत्यप॑रि - मि॒त्य॒ । तस्य॑ । शर्क॑राः । सिक॑ताः । वीति॑ । ऊ॒हे॒त् । अप॑रिगृहीत॒ इत्यप॑रि - गृ॒ही॒ते॒ । ए॒व । अ॒स्य॒ । वि॒षू॒चीन᳚म् । रेतः॑ । परेति॑ । सि॒ञ्च॒ति॒ । अ॒प॒शुः । ए॒व । भ॒व॒ति॒ ।  \newline


\textbf{Krama Paata} \newline

परि॑ मिनुयात् । मि॒नु॒या॒थ् स॒प्त । स॒प्त वै । वै शी॑र्.ष॒ण्याः᳚ । शी॒र्॒.ष॒ण्याः᳚ प्रा॒णाः । प्रा॒णाः प्रा॒णाः । प्रा॒णा इति॑ प्र - अ॒नाः । प्रा॒णाः प॒शवः॑ । प्रा॒णा इति॑ प्र - अ॒नाः । प॒शवः॑ प्रा॒णैः । प्रा॒णैरे॒व । प्रा॒णैरिति॑ प्र - अ॒नैः । ए॒वास्मै᳚ । अ॒स्मै॒ प॒शून् । प॒शूनव॑ । अव॑ रुन्धे । रु॒न्धे॒ त्रि॒ण॒वाभिः॑ । त्रि॒ण॒वाभि॒र् भ्रातृ॑व्यवतः । त्रि॒ण॒वाभि॒रिति॑ त्रि - न॒वाभिः॑ । भ्रातृ॑व्यवत स्त्रि॒वृत᳚म् । भ्रातृ॑व्यवत॒ इति॒ भातृ॑व्य - व॒तः॒ । त्रि॒वृत॑मे॒व । त्रि॒वृत॒मिति॑ त्रि - वृत᳚म् । ए॒व वज्र᳚म् । वज्रꣳ॑ स॒म्भृत्य॑ । स॒म्भृत्य॒ भ्रातृ॑व्याय । स॒म्भृत्येति॑ सम् - भृत्य॑ । भ्रातृ॑व्याय॒ प्र । प्र ह॑रति । ह॒र॒ति॒ स्तृत्यै᳚ । स्तृत्या॒ अप॑रिमिताभिः । अप॑रिमिताभिः॒ परि॑ । अप॑रिमिताभि॒रित्यप॑रि - मि॒ता॒भिः॒ । परि॑ मिनुयात् । मि॒नु॒या॒दप॑रिमितस्य । अप॑रिमित॒स्याव॑रुद्ध्यै । अप॑रिमित॒स्येत्यप॑रि - मि॒त॒स्य॒ । अव॑रुद्ध्यै॒ यम् । अव॑रुद्ध्या॒ इत्यव॑ - रु॒द्ध्यै॒ । यम् का॒मये॑त । का॒मये॑ताप॒शुः । अ॒प॒शुः स्या᳚त् । स्या॒दिति॑ । इत्यप॑रिमित्य । अप॑रिमित्य॒ तस्य॑ । अप॑रिमि॒त्येत्यप॑रि - मि॒त्य॒ । तस्य॒ शर्क॑राः । शर्क॑राः॒ सिक॑ताः । सिक॑ता॒ वि । व्यू॑हेत् । ऊ॒हे॒दप॑रिगृहीते । अप॑रिगृहीत ए॒व । अप॑रिगृहीत॒ इत्यप॑रि - गृ॒ही॒ते॒ । ए॒वास्य॑ । अ॒स्य॒ वि॒षू॒चीन᳚म् । वि॒षू॒चीनꣳ॒॒ रेतः॑ । रेतः॒ परा᳚ । परा॑ सिञ्चति । सि॒ञ्च॒त्य॒प॒शुः । अ॒प॒शुरे॒व । ए॒व भ॑वति । भ॒व॒ति॒ यम् \newline

\textbf{Jatai Paata} \newline

1. परि॑ मिनुयान् मिनुया॒त् परि॒ परि॑ मिनुयात् । \newline
2. मि॒नु॒या॒थ् स॒प्त स॒प्त मि॑नुयान् मिनुयाथ् स॒प्त । \newline
3. स॒प्त वै वै स॒प्त स॒प्त वै । \newline
4. वै शी॑र्.ष॒ण्याः᳚ शीर्.ष॒ण्या॑ वै वै शी॑र्.ष॒ण्याः᳚ । \newline
5. शी॒र्॒.ष॒ण्याः᳚ प्रा॒णाः प्रा॒णाः शी॑र्.ष॒ण्याः᳚ शीर्.ष॒ण्याः᳚ प्रा॒णाः । \newline
6. प्रा॒णाः प्रा॒णाः । \newline
7. प्रा॒णा इति॑ प्र - अ॒नाः । \newline
8. प्रा॒णाः प॒शवः॑ प॒शवः॑ प्रा॒णाः प्रा॒णाः प॒शवः॑ । \newline
9. प्रा॒णा इति॑ प्र - अ॒नाः । \newline
10. प॒शवः॑ प्रा॒णैः प्रा॒णैः प॒शवः॑ प॒शवः॑ प्रा॒णैः । \newline
11. प्रा॒णै रे॒वैव प्रा॒णैः प्रा॒णै रे॒व । \newline
12. प्रा॒णैरिति॑ प्र - अ॒नैः । \newline
13. ए॒वास्मा॑ अस्मा ए॒वै वास्मै᳚ । \newline
14. अ॒स्मै॒ प॒शून् प॒शू न॑स्मा अस्मै प॒शून् । \newline
15. प॒शू नवाव॑ प॒शून् प॒शू नव॑ । \newline
16. अव॑ रुन्धे रु॒न्धे ऽवाव॑ रुन्धे । \newline
17. रु॒न्धे॒ त्रि॒ण॒वाभि॑ स्त्रिण॒वाभी॑ रुन्धे रुन्धे त्रिण॒वाभिः॑ । \newline
18. त्रि॒ण॒वाभि॒र् भ्रातृ॑व्यवतो॒ भ्रातृ॑व्यवत स्त्रिण॒वाभि॑ स्त्रिण॒वाभि॒र् भ्रातृ॑व्यवतः । \newline
19. त्रि॒ण॒वाभि॒रिति॑ त्रि - न॒वाभिः॑ । \newline
20. भ्रातृ॑व्यवत स्त्रि॒वृत॑म् त्रि॒वृत॒म् भ्रातृ॑व्यवतो॒ भ्रातृ॑व्यवत स्त्रि॒वृत᳚म् । \newline
21. भ्रातृ॑व्यवत॒ इति॒ भ्रातृ॑व्य - व॒तः॒ । \newline
22. त्रि॒वृत॑ मे॒वैव त्रि॒वृत॑म् त्रि॒वृत॑ मे॒व । \newline
23. त्रि॒वृत॒मिति॑ त्रि - वृत᳚म् । \newline
24. ए॒व वज्रं॒ ॅवज्र॑ मे॒वैव वज्र᳚म् । \newline
25. वज्रꣳ॑ सं॒भृत्य॑ सं॒भृत्य॒ वज्रं॒ ॅवज्रꣳ॑ सं॒भृत्य॑ । \newline
26. सं॒भृत्य॒ भ्रातृ॑व्याय॒ भ्रातृ॑व्याय सं॒भृत्य॑ सं॒भृत्य॒ भ्रातृ॑व्याय । \newline
27. सं॒भृत्येति॑ सं - भृत्य॑ । \newline
28. भ्रातृ॑व्याय॒ प्र प्र भ्रातृ॑व्याय॒ भ्रातृ॑व्याय॒ प्र । \newline
29. प्र ह॑रति हरति॒ प्र प्र ह॑रति । \newline
30. ह॒र॒ति॒ स्तृत्यै॒ स्तृत्यै॑ हरति हरति॒ स्तृत्यै᳚ । \newline
31. स्तृत्या॒ अप॑रिमिताभि॒ रप॑रिमिताभिः॒ स्तृत्यै॒ स्तृत्या॒ अप॑रिमिताभिः । \newline
32. अप॑रिमिताभिः॒ परि॒ पर्यप॑रिमिताभि॒ रप॑रिमिताभिः॒ परि॑ । \newline
33. अप॑रिमिताभि॒रित्यप॑रि - मि॒ता॒भिः॒ । \newline
34. परि॑ मिनुयान् मिनुया॒त् परि॒ परि॑ मिनुयात् । \newline
35. मि॒नु॒या॒ दप॑रिमित॒स्या प॑रिमितस्य मिनुयान् मिनुया॒ दप॑रिमितस्य । \newline
36. अप॑रिमित॒स्या व॑रुद्ध्या॒ अव॑रुद्ध्या॒ अप॑रिमित॒स्या प॑रिमित॒स्या व॑रुद्ध्यै । \newline
37. अप॑रिमित॒स्येत्यप॑रि - मि॒त॒स्य॒ । \newline
38. अव॑रुद्ध्यै॒ यं ॅय मव॑रुद्ध्या॒ अव॑रुद्ध्यै॒ यम् । \newline
39. अव॑रुद्ध्या॒ इत्यव॑ - रु॒द्ध्यै॒ । \newline
40. यम् का॒मये॑त का॒मये॑त॒ यं ॅयम् का॒मये॑त । \newline
41. का॒मये॑ता प॒शु र॑प॒शुः का॒मये॑त का॒मये॑ ताप॒शुः । \newline
42. अ॒प॒शुः स्या᳚थ् स्या दप॒शु र॑प॒शुः स्या᳚त् । \newline
43. स्या॒ दितीति॑ स्याथ् स्या॒दिति॑ । \newline
44. इत्यप॑रिमि॒त्या प॑रिमि॒त्ये तीत्यप॑रिमित्य । \newline
45. अप॑रिमित्य॒ तस्य॒ तस्या प॑रिमि॒त्या प॑रिमित्य॒ तस्य॑ । \newline
46. अप॑रिमि॒त्येत्यप॑रि - मि॒त्य॒ । \newline
47. तस्य॒ शर्क॑राः॒ शर्क॑रा॒ स्तस्य॒ तस्य॒ शर्क॑राः । \newline
48. शर्क॑राः॒ सिक॑ताः॒ सिक॑ताः॒ शर्क॑राः॒ शर्क॑राः॒ सिक॑ताः । \newline
49. सिक॑ता॒ वि वि सिक॑ताः॒ सिक॑ता॒ वि । \newline
50. व्यू॑हे दूहे॒द् वि व्यू॑हेत् । \newline
51. ऊ॒हे॒ दप॑रिगृही॒ते ऽप॑रिगृहीत ऊहे दूहे॒ दप॑रिगृहीते । \newline
52. अप॑रिगृहीत ए॒वैवा प॑रिगृही॒ते ऽप॑रिगृहीत ए॒व । \newline
53. अप॑रिगृहीत॒ इत्यप॑रि - गृ॒ही॒ते॒ । \newline
54. ए॒वास्या᳚ स्यै॒वै वास्य॑ । \newline
55. अ॒स्य॒ वि॒षू॒चीनं॑ ॅविषू॒चीन॑ मस्यास्य विषू॒चीन᳚म् । \newline
56. वि॒षू॒चीनꣳ॒॒ रेतो॒ रेतो॑ विषू॒चीनं॑ ॅविषू॒चीनꣳ॒॒ रेतः॑ । \newline
57. रेतः॒ परा॒ परा॒ रेतो॒ रेतः॒ परा᳚ । \newline
58. परा॑ सिञ्चति सिञ्चति॒ परा॒ परा॑ सिञ्चति । \newline
59. सि॒ञ्च॒ त्य॒प॒शु र॑प॒शुः सि॑ञ्चति सिञ्च त्यप॒शुः । \newline
60. अ॒प॒शु रे॒वैवा प॒शु र॑प॒शु रे॒व । \newline
61. ए॒व भ॑वति भव त्ये॒वैव भ॑वति । \newline
62. भ॒व॒ति॒ यं ॅयम् भ॑वति भवति॒ यम् । \newline

\textbf{Ghana Paata } \newline

1. परि॑ मिनुयान् मिनुया॒त् परि॒ परि॑ मिनुयाथ् स॒प्त स॒प्त मि॑नुया॒त् परि॒ परि॑ मिनुयाथ् स॒प्त । \newline
2. मि॒नु॒या॒थ् स॒प्त स॒प्त मि॑नुयान् मिनुयाथ् स॒प्त वै वै स॒प्त मि॑नुयान् मिनुयाथ् स॒प्त वै । \newline
3. स॒प्त वै वै स॒प्त स॒प्त वै शी॑र्.ष॒ण्याः᳚ शीर्.ष॒ण्या॑ वै स॒प्त स॒प्त वै शी॑र्.ष॒ण्याः᳚ । \newline
4. वै शी॑र्.ष॒ण्याः᳚ शीर्.ष॒ण्या॑ वै वै शी॑र्.ष॒ण्याः᳚ प्रा॒णाः प्रा॒णाः शी॑र्.ष॒ण्या॑ वै वै शी॑र्.ष॒ण्याः᳚ प्रा॒णाः । \newline
5. शी॒र्॒.ष॒ण्याः᳚ प्रा॒णाः प्रा॒णाः शी॑र्.ष॒ण्याः᳚ शीर्.ष॒ण्याः᳚ प्रा॒णाः । \newline
6. प्रा॒णाः प्रा॒णाः । \newline
7. प्रा॒णा इति॑ प्र - अ॒नाः । \newline
8. प्रा॒णाः प॒शवः॑ प॒शवः॑ प्रा॒णाः प्रा॒णाः प॒शवः॑ प्रा॒णैः प्रा॒णैः प॒शवः॑ प्रा॒णाः प्रा॒णाः प॒शवः॑ प्रा॒णैः । \newline
9. प्रा॒णा इति॑ प्र - अ॒नाः । \newline
10. प॒शवः॑ प्रा॒णैः प्रा॒णैः प॒शवः॑ प॒शवः॑ प्रा॒णै रे॒वैव प्रा॒णैः प॒शवः॑ प॒शवः॑ प्रा॒णै रे॒व । \newline
11. प्रा॒णै रे॒वैव प्रा॒णैः प्रा॒णै रे॒वास्मा॑ अस्मा ए॒व प्रा॒णैः प्रा॒णै रे॒वास्मै᳚ । \newline
12. प्रा॒णैरिति॑ प्र - अ॒नैः । \newline
13. ए॒वास्मा॑ अस्मा ए॒वैवास्मै॑ प॒शून् प॒शू न॑स्मा ए॒वैवास्मै॑ प॒शून् । \newline
14. अ॒स्मै॒ प॒शून् प॒शू न॑स्मा अस्मै प॒शू नवाव॑ प॒शू न॑स्मा अस्मै प॒शू नव॑ । \newline
15. प॒शू नवाव॑ प॒शून् प॒शू नव॑ रुन्धे रु॒न्धे ऽव॑ प॒शून् प॒शू नव॑ रुन्धे । \newline
16. अव॑ रुन्धे रु॒न्धे ऽवाव॑ रुन्धे त्रिण॒वाभि॑ स्त्रिण॒वाभी॑ रु॒न्धे ऽवाव॑ रुन्धे त्रिण॒वाभिः॑ । \newline
17. रु॒न्धे॒ त्रि॒ण॒वाभि॑ स्त्रिण॒वाभी॑ रुन्धे रुन्धे त्रिण॒वाभि॒र् भ्रातृ॑व्यवतो॒ भ्रातृ॑व्यवत स्त्रिण॒वाभी॑ रुन्धे रुन्धे त्रिण॒वाभि॒र् भ्रातृ॑व्यवतः । \newline
18. त्रि॒ण॒वाभि॒र् भ्रातृ॑व्यवतो॒ भ्रातृ॑व्यवत स्त्रिण॒वाभि॑ स्त्रिण॒वाभि॒र् भ्रातृ॑व्यवत स्त्रि॒वृत॑म् त्रि॒वृत॒म् भ्रातृ॑व्यवत स्त्रिण॒वाभि॑ स्त्रिण॒वाभि॒र् भ्रातृ॑व्यवत स्त्रि॒वृत᳚म् । \newline
19. त्रि॒ण॒वाभि॒रिति॑ त्रि - न॒वाभिः॑ । \newline
20. भ्रातृ॑व्यवत स्त्रि॒वृत॑म् त्रि॒वृत॒म् भ्रातृ॑व्यवतो॒ भ्रातृ॑व्यवत स्त्रि॒वृत॑ मे॒वैव त्रि॒वृत॒म् भ्रातृ॑व्यवतो॒ भ्रातृ॑व्यवत स्त्रि॒वृत॑ मे॒व । \newline
21. भ्रातृ॑व्यवत॒ इति॒ भ्रातृ॑व्य - व॒तः॒ । \newline
22. त्रि॒वृत॑ मे॒वैव त्रि॒वृत॑म् त्रि॒वृत॑ मे॒व वज्रं॒ ॅवज्र॑ मे॒व त्रि॒वृत॑म् त्रि॒वृत॑ मे॒व वज्र᳚म् । \newline
23. त्रि॒वृत॒मिति॑ त्रि - वृत᳚म् । \newline
24. ए॒व वज्रं॒ ॅवज्र॑ मे॒वैव वज्रꣳ॑ सं॒भृत्य॑ सं॒भृत्य॒ वज्र॑ मे॒वैव वज्रꣳ॑ सं॒भृत्य॑ । \newline
25. वज्रꣳ॑ सं॒भृत्य॑ सं॒भृत्य॒ वज्रं॒ ॅवज्रꣳ॑ सं॒भृत्य॒ भ्रातृ॑व्याय॒ भ्रातृ॑व्याय सं॒भृत्य॒ वज्रं॒ ॅवज्रꣳ॑ सं॒भृत्य॒ भ्रातृ॑व्याय । \newline
26. सं॒भृत्य॒ भ्रातृ॑व्याय॒ भ्रातृ॑व्याय सं॒भृत्य॑ सं॒भृत्य॒ भ्रातृ॑व्याय॒ प्र प्र भ्रातृ॑व्याय सं॒भृत्य॑ सं॒भृत्य॒ भ्रातृ॑व्याय॒ प्र । \newline
27. सं॒भृत्येति॑ सं - भृत्य॑ । \newline
28. भ्रातृ॑व्याय॒ प्र प्र भ्रातृ॑व्याय॒ भ्रातृ॑व्याय॒ प्र ह॑रति हरति॒ प्र भ्रातृ॑व्याय॒ भ्रातृ॑व्याय॒ प्र ह॑रति । \newline
29. प्र ह॑रति हरति॒ प्र प्र ह॑रति॒ स्तृत्यै॒ स्तृत्यै॑ हरति॒ प्र प्र ह॑रति॒ स्तृत्यै᳚ । \newline
30. ह॒र॒ति॒ स्तृत्यै॒ स्तृत्यै॑ हरति हरति॒ स्तृत्या॒ अप॑रिमिताभि॒ रप॑रिमिताभिः॒ स्तृत्यै॑ हरति हरति॒ स्तृत्या॒ अप॑रिमिताभिः । \newline
31. स्तृत्या॒ अप॑रिमिताभि॒ रप॑रिमिताभिः॒ स्तृत्यै॒ स्तृत्या॒ अप॑रिमिताभिः॒ परि॒ पर्यप॑रिमिताभिः॒ स्तृत्यै॒ स्तृत्या॒ अप॑रिमिताभिः॒ परि॑ । \newline
32. अप॑रिमिताभिः॒ परि॒ पर्यप॑रिमिताभि॒ रप॑रिमिताभिः॒ परि॑ मिनुयान् मिनुया॒त् पर्यप॑रिमिताभि॒ रप॑रिमिताभिः॒ परि॑ मिनुयात् । \newline
33. अप॑रिमिताभि॒रित्यप॑रि - मि॒ता॒भिः॒ । \newline
34. परि॑ मिनुयान् मिनुया॒त् परि॒ परि॑ मिनुया॒ दप॑रिमित॒स्या प॑रिमितस्य मिनुया॒त् परि॒ परि॑ मिनुया॒ दप॑रिमितस्य । \newline
35. मि॒नु॒या॒ दप॑रिमित॒स्या प॑रिमितस्य मिनुयान् मिनुया॒ दप॑रिमित॒स्या व॑रुद्ध्या॒ अव॑रुद्ध्या॒ अप॑रिमितस्य मिनुयान् मिनुया॒ दप॑रिमित॒स्या व॑रुद्ध्यै । \newline
36. अप॑रिमित॒स्या व॑रुद्ध्या॒ अव॑रुद्ध्या॒ अप॑रिमित॒स्या प॑रिमित॒स्या व॑रुद्ध्यै॒ यं ॅय मव॑रुद्ध्या॒ अप॑रिमित॒स्या प॑रिमित॒स्या व॑रुद्ध्यै॒ यम् । \newline
37. अप॑रिमित॒स्येत्यप॑रि - मि॒त॒स्य॒ । \newline
38. अव॑रुद्ध्यै॒ यं ॅय मव॑रुद्ध्या॒ अव॑रुद्ध्यै॒ यम् का॒मये॑त का॒मये॑त॒ य मव॑रुद्ध्या॒ अव॑रुद्ध्यै॒ यम् का॒मये॑त । \newline
39. अव॑रुद्ध्या॒ इत्यव॑ - रु॒द्ध्यै॒ । \newline
40. यम् का॒मये॑त का॒मये॑त॒ यं ॅयम् का॒मये॑ता प॒शु र॑प॒शुः का॒मये॑त॒ यं ॅयम् का॒मये॑ता प॒शुः । \newline
41. का॒मये॑ता प॒शु र॑प॒शुः का॒मये॑त का॒मये॑ता प॒शुः स्या᳚थ् स्या दप॒शुः का॒मये॑त का॒मये॑ता प॒शुः स्या᳚त् । \newline
42. अ॒प॒शुः स्या᳚थ् स्या दप॒शु र॑प॒शुः स्या॒दितीति॑ स्या दप॒शु र॑प॒शुः स्या॒दिति॑ । \newline
43. स्या॒ दितीति॑ स्याथ् स्या॒ दित्यप॑रिमि॒त्या प॑रिमि॒त्येति॑ स्याथ् स्या॒ दित्यप॑रिमित्य । \newline
44. इत्यप॑रिमि॒त्या प॑रिमि॒त्ये तीत्यप॑रिमित्य॒ तस्य॒ तस्या प॑रिमि॒त्ये तीत्यप॑रिमित्य॒ तस्य॑ । \newline
45. अप॑रिमित्य॒ तस्य॒ तस्या प॑रिमि॒त्या प॑रिमित्य॒ तस्य॒ शर्क॑राः॒ शर्क॑रा॒ स्तस्या प॑रिमि॒त्या प॑रिमित्य॒ तस्य॒ शर्क॑राः । \newline
46. अप॑रिमि॒त्येत्यप॑रि - मि॒त्य॒ । \newline
47. तस्य॒ शर्क॑राः॒ शर्क॑रा॒ स्तस्य॒ तस्य॒ शर्क॑राः॒ सिक॑ताः॒ सिक॑ताः॒ शर्क॑रा॒ स्तस्य॒ तस्य॒ शर्क॑राः॒ सिक॑ताः । \newline
48. शर्क॑राः॒ सिक॑ताः॒ सिक॑ताः॒ शर्क॑राः॒ शर्क॑राः॒ सिक॑ता॒ वि वि सिक॑ताः॒ शर्क॑राः॒ शर्क॑राः॒ सिक॑ता॒ वि । \newline
49. सिक॑ता॒ वि वि सिक॑ताः॒ सिक॑ता॒ व्यू॑हे दूहे॒द् वि सिक॑ताः॒ सिक॑ता॒ व्यू॑हेत् । \newline
50. व्यू॑हेदूहे॒द् वि व्यू॑हे॒ दप॑रिगृही॒ते ऽप॑रिगृहीत ऊहे॒द् वि व्यू॑हे॒ दप॑रिगृहीते । \newline
51. ऊ॒हे॒ दप॑रिगृही॒ते ऽप॑रिगृहीत ऊहे दूहे॒ दप॑रिगृहीत ए॒वैवा प॑रिगृहीत ऊहे दूहे॒ दप॑रिगृहीत ए॒व । \newline
52. अप॑रिगृहीत ए॒वैवा प॑रिगृही॒ते ऽप॑रिगृहीत ए॒वास्या᳚ स्यै॒वा प॑रिगृही॒ते ऽप॑रिगृहीत ए॒वास्य॑ । \newline
53. अप॑रिगृहीत॒ इत्यप॑रि - गृ॒ही॒ते॒ । \newline
54. ए॒वास्या᳚ स्यै॒वैवास्य॑ विषू॒चीनं॑ ॅविषू॒चीन॑ मस्यै॒वैवास्य॑ विषू॒चीन᳚म् । \newline
55. अ॒स्य॒ वि॒षू॒चीनं॑ ॅविषू॒चीन॑ मस्यास्य विषू॒चीनꣳ॒॒ रेतो॒ रेतो॑ विषू॒चीन॑ मस्यास्य विषू॒चीनꣳ॒॒ रेतः॑ । \newline
56. वि॒षू॒चीनꣳ॒॒ रेतो॒ रेतो॑ विषू॒चीनं॑ ॅविषू॒चीनꣳ॒॒ रेतः॒ परा॒ परा॒ रेतो॑ विषू॒चीनं॑ ॅविषू॒चीनꣳ॒॒ रेतः॒ परा᳚ । \newline
57. रेतः॒ परा॒ परा॒ रेतो॒ रेतः॒ परा॑ सिञ्चति सिञ्चति॒ परा॒ रेतो॒ रेतः॒ परा॑ सिञ्चति । \newline
58. परा॑ सिञ्चति सिञ्चति॒ परा॒ परा॑ सिञ्च त्यप॒शु र॑प॒शुः सि॑ञ्चति॒ परा॒ परा॑ सिञ्च त्यप॒शुः । \newline
59. सि॒ञ्च॒ त्य॒प॒शु र॑प॒शुः सि॑ञ्चति सिञ्च त्यप॒शु रे॒वैवाप॒शुः सि॑ञ्चति सिञ्च त्यप॒शु रे॒व । \newline
60. अ॒प॒शु रे॒वैवा प॒शु र॑प॒शु रे॒व भ॑वति भव त्ये॒वा प॒शु र॑प॒शु रे॒व भ॑वति । \newline
61. ए॒व भ॑वति भव त्ये॒वैव भ॑वति॒ यं ॅयम् भ॑व त्ये॒वैव भ॑वति॒ यम् । \newline
62. भ॒व॒ति॒ यं ॅयम् भ॑वति भवति॒ यम् का॒मये॑त का॒मये॑त॒ यम् भ॑वति भवति॒ यम् का॒मये॑त । \newline
\pagebreak
\markright{ TS 5.2.6.4  \hfill https://www.vedavms.in \hfill}

\section{ TS 5.2.6.4 }

\textbf{TS 5.2.6.4 } \newline
\textbf{Samhita Paata} \newline

यं का॒मये॑त पशु॒मान्थ् स्या॒दिति॑ परि॒मित्य॒ तस्य॒ शर्क॑राः॒ सिक॑ता॒ व्यू॑हे॒त् परि॑गृहीत ए॒वास्मै॑ समी॒चीनꣳ॒॒ रेतः॑ सिञ्चति पशु॒माने॒व भ॑वति सौ॒म्या व्यू॑हति॒ सोमो॒ वै रे॑तो॒धा रेत॑ ए॒व तद्-द॑धाति गायत्रि॒या ब्रा᳚ह्म॒णस्य॑ गाय॒त्रो हि ब्रा᳚ह्म॒ण-स्त्रि॒ष्टुभा॑ राज॒न्य॑स्य॒ त्रैष्टु॑भो॒ हि रा॑ज॒न्यः॑ श॒म्युं बा॑र्.हस्प॒त्यं मेधो॒ नोपा॑नम॒थ् सो᳚ऽग्निं प्राऽवि॑श॒थ् - [  ] \newline

\textbf{Pada Paata} \newline

यम् । का॒मये॑त । प॒शु॒मानिति॑ पशु - मान् । स्या॒त् । इति॑ । प॒रि॒मित्येति॑ परि-मित्य॑ । तस्य॑ । शर्क॑राः । सिक॑ताः । वीति॑ । ऊ॒हे॒त् । परि॑गृहीत॒ इति॒ परि॑ - गृ॒ही॒ते॒ । ए॒व । अ॒स्मै॒ । स॒मी॒चीन᳚म् । रेतः॑ । सि॒ञ्च॒ति॒ । प॒शु॒मानिति॑ पशु-मान् । ए॒व । भ॒व॒ति॒ । सौ॒म्या । वीति॑ । ऊ॒ह॒ति॒ । सोमः॑ । वै । रे॒तो॒धा इति॑ रेतः - धाः । रेतः॑ । ए॒व । तत् । द॒धा॒ति॒ । गा॒य॒त्रि॒या । ब्रा॒ह्म॒णस्य॑ । गा॒य॒त्रः । हि । ब्रा॒ह्म॒णः । त्रि॒ष्टुभा᳚ । रा॒ज॒न्य॑स्य । त्रैष्टु॑भः । हि । रा॒ज॒न्यः॑ । शं॒ॅयुमिति॑ शं - युम् । बा॒र्॒.ह॒स्प॒त्यम् । मेधः॑ । न । उपेति॑ । अ॒न॒म॒त् । सः । अ॒ग्निम् । प्रेति॑ । अ॒वि॒श॒त् ।  \newline


\textbf{Krama Paata} \newline

यम् का॒मये॑त । का॒मये॑त पशु॒मान् । प॒शु॒मान्थ् स्या᳚त् । प॒शु॒मानिति॑ पशु - मान् । स्या॒दिति॑ । इति॑ परि॒मित्य॑ । प॒रि॒मित्य॒ तस्य॑ । प॒रि॒मित्येति॑ परि - मित्य॑ । तस्य॒ शर्क॑राः । शर्क॑राः॒ सिक॑ताः । सिक॑ता॒ वि । व्यू॑हेत् । ऊ॒हे॒त् परि॑गृहीते । परि॑गृहीत ए॒व । परि॑गृहीत॒ इति॒ परि॑ - गृ॒ही॒ते॒ । ए॒वास्मै᳚ । अ॒स्मै॒ स॒मी॒चीन᳚म् । स॒मी॒चीनꣳ॒॒ रेतः॑ । रेतः॑ सिञ्चति । सि॒ञ्च॒ति॒ प॒शु॒मान् । प॒शु॒माने॒व । प॒शु॒मानिति॑ पशु - मान् । ए॒व भ॑वति । भ॒व॒ति॒ सौ॒म्या । सौ॒म्या वि । व्यू॑हति । ऊ॒ह॒ति॒ सोमः॑ । सोमो॒ वै । वै रे॑तो॒धाः । रे॒तो॒धा रेतः॑ । रे॒तो॒धा इति॑ रेतः - धाः । रेत॑ ए॒व । ए॒व तत् । तद् द॑धाति । द॒धा॒ति॒ गा॒य॒त्रि॒या । गा॒य॒त्रि॒या ब्रा᳚ह्म॒णस्य॑ । ब्रा॒ह्म॒णस्य॑ गाय॒त्रः । गा॒य॒त्रो हि । हि ब्रा᳚ह्म॒णः । ब्रा॒ह्म॒णस्त्रि॒ष्टुभा᳚ । त्रि॒ष्टुभा॑ राज॒न्य॑स्य । रा॒ज॒न्य॑स्य॒ त्रैष्टु॑भः । त्रैष्टु॑भो॒ हि । हि रा॑ज॒न्यः॑ । रा॒ज॒न्यः॑ श॒म्ॅयुम् । श॒म्ॅयुम् बा॑र्.हस्प॒त्यम् । श॒म्ॅयुमिति॑ शम् - युम् । बा॒र्॒.ह॒स्प॒त्यम् मेधः॑ । मेधो॒ न । नोप॑ । उपा॑नमत् । अ॒न॒म॒थ् सः । सो᳚ऽग्निम् । अ॒ग्निम् प्र । प्रावि॑शत् । अ॒वि॒श॒थ् सः \newline

\textbf{Jatai Paata} \newline

1. यम् का॒मये॑त का॒मये॑त॒ यं ॅयम् का॒मये॑त । \newline
2. का॒मये॑त पशु॒मान् प॑शु॒मान् का॒मये॑त का॒मये॑त पशु॒मान् । \newline
3. प॒शु॒मान् थ्स्या᳚थ् स्यात् पशु॒मान् प॑शु॒मान् थ्स्या᳚त् । \newline
4. प॒शु॒मानिति॑ पशु - मान् । \newline
5. स्या॒ दितीति॑ स्याथ् स्या॒दिति॑ । \newline
6. इति॑ परि॒मित्य॑ परि॒मित्ये तीति॑ परि॒मित्य॑ । \newline
7. प॒रि॒मित्य॒ तस्य॒ तस्य॑ परि॒मित्य॑ परि॒मित्य॒ तस्य॑ । \newline
8. प॒रि॒मित्येति॑ परि - मित्य॑ । \newline
9. तस्य॒ शर्क॑राः॒ शर्क॑रा॒ स्तस्य॒ तस्य॒ शर्क॑राः । \newline
10. शर्क॑राः॒ सिक॑ताः॒ सिक॑ताः॒ शर्क॑राः॒ शर्क॑राः॒ सिक॑ताः । \newline
11. सिक॑ता॒ वि वि सिक॑ताः॒ सिक॑ता॒ वि । \newline
12. व्यू॑हे दूहे॒द् वि व्यू॑हेत् । \newline
13. ऊ॒हे॒त् परि॑गृहीते॒ परि॑गृहीत ऊहे दूहे॒त् परि॑गृहीते । \newline
14. परि॑गृहीत ए॒वैव परि॑गृहीते॒ परि॑गृहीत ए॒व । \newline
15. परि॑गृहीत॒ इति॒ परि॑ - गृ॒ही॒ते॒ । \newline
16. ए॒वास्मा॑ अस्मा ए॒वै वास्मै᳚ । \newline
17. अ॒स्मै॒ स॒मी॒चीनꣳ॑ समी॒चीन॑ मस्मा अस्मै समी॒चीन᳚म् । \newline
18. स॒मी॒चीनꣳ॒॒ रेतो॒ रेतः॑ समी॒चीनꣳ॑ समी॒चीनꣳ॒॒ रेतः॑ । \newline
19. रेतः॑ सिञ्चति सिञ्चति॒ रेतो॒ रेतः॑ सिञ्चति । \newline
20. सि॒ञ्च॒ति॒ प॒शु॒मान् प॑शु॒मान् थ्सि॑ञ्चति सिञ्चति पशु॒मान् । \newline
21. प॒शु॒मा ने॒वैव प॑शु॒मान् प॑शु॒मा ने॒व । \newline
22. प॒शु॒मानिति॑ पशु - मान् । \newline
23. ए॒व भ॑वति भव त्ये॒वैव भ॑वति । \newline
24. भ॒व॒ति॒ सौ॒म्या सौ॒म्या भ॑वति भवति सौ॒म्या । \newline
25. सौ॒म्या वि वि सौ॒म्या सौ॒म्या वि । \newline
26. व्यू॑ह त्यूहति॒ वि व्यू॑हति । \newline
27. ऊ॒ह॒ति॒ सोमः॒ सोम॑ ऊह त्यूहति॒ सोमः॑ । \newline
28. सोमो॒ वै वै सोमः॒ सोमो॒ वै । \newline
29. वै रे॑तो॒धा रे॑तो॒धा वै वै रे॑तो॒धाः । \newline
30. रे॒तो॒धा रेतो॒ रेतो॑ रेतो॒धा रे॑तो॒धा रेतः॑ । \newline
31. रे॒तो॒धा इति॑ रेतः - धाः । \newline
32. रेत॑ ए॒वैव रेतो॒ रेत॑ ए॒व । \newline
33. ए॒व तत् तदे॒वैव तत् । \newline
34. तद् द॑धाति दधाति॒ तत् तद् द॑धाति । \newline
35. द॒धा॒ति॒ गा॒य॒त्रि॒या गा॑यत्रि॒या द॑धाति दधाति गायत्रि॒या । \newline
36. गा॒य॒त्रि॒या ब्रा᳚ह्म॒णस्य॑ ब्राह्म॒णस्य॑ गायत्रि॒या गा॑यत्रि॒या ब्रा᳚ह्म॒णस्य॑ । \newline
37. ब्रा॒ह्म॒णस्य॑ गाय॒त्रो गा॑य॒त्रो ब्रा᳚ह्म॒णस्य॑ ब्राह्म॒णस्य॑ गाय॒त्रः । \newline
38. गा॒य॒त्रो हि हि गा॑य॒त्रो गा॑य॒त्रो हि । \newline
39. हि ब्रा᳚ह्म॒णो ब्रा᳚ह्म॒णो हि हि ब्रा᳚ह्म॒णः । \newline
40. ब्रा॒ह्म॒ण स्त्रि॒ष्टुभा᳚ त्रि॒ष्टुभा᳚ ब्राह्म॒णो ब्रा᳚ह्म॒ण स्त्रि॒ष्टुभा᳚ । \newline
41. त्रि॒ष्टुभा॑ राज॒न्य॑स्य राज॒न्य॑स्य त्रि॒ष्टुभा᳚ त्रि॒ष्टुभा॑ राज॒न्य॑स्य । \newline
42. रा॒ज॒न्य॑स्य॒ त्रैष्टु॑भ॒स्त्रैष्टु॑भो राज॒न्य॑स्य राज॒न्य॑स्य॒ त्रैष्टु॑भः । \newline
43. त्रैष्टु॑भो॒ हि हि त्रैष्टु॑भ॒ स्त्रैष्टु॑भो॒ हि । \newline
44. हि रा॑ज॒न्यो॑ राज॒न्यो॑ हि हि रा॑ज॒न्यः॑ । \newline
45. रा॒ज॒न्यः॑ शं॒ॅयुꣳ शं॒ॅयुꣳ रा॑ज॒न्यो॑ राज॒न्यः॑ शं॒ॅयुम् । \newline
46. शं॒ॅयुम् बा॑र्.हस्प॒त्यम् बा॑र्.हस्प॒त्यꣳ शं॒ॅयुꣳ शं॒ॅयुम् बा॑र्.हस्प॒त्यम् । \newline
47. शं॒ॅयुमिति॑ शं - युम् । \newline
48. बा॒र्॒.ह॒स्प॒त्यम् मेधो॒ मेधो॑ बार्.हस्प॒त्यम् बा॑र्.हस्प॒त्यम् मेधः॑ । \newline
49. मेधो॒ न न मेधो॒ मेधो॒ न । \newline
50. नोपोप॒ न नोप॑ । \newline
51. उपा॑नम दनम॒ दुपोपा॑ नमत् । \newline
52. अ॒न॒म॒थ् स सो॑ ऽनम दनम॒थ् सः । \newline
53. सो᳚ ऽग्नि म॒ग्निꣳ स सो᳚ ऽग्निम् । \newline
54. अ॒ग्निम् प्र प्राग्नि म॒ग्निम् प्र । \newline
55. प्रावि॑श दविश॒त् प्र प्रावि॑शत् । \newline
56. अ॒वि॒श॒थ् स सो॑ ऽविश दविश॒थ् सः । \newline

\textbf{Ghana Paata } \newline

1. यम् का॒मये॑त का॒मये॑त॒ यं ॅयम् का॒मये॑त पशु॒मान् प॑शु॒मान् का॒मये॑त॒ यं ॅयम् का॒मये॑त पशु॒मान् । \newline
2. का॒मये॑त पशु॒मान् प॑शु॒मान् का॒मये॑त का॒मये॑त पशु॒मान् थ्स्या᳚थ् स्यात् पशु॒मान् का॒मये॑त का॒मये॑त पशु॒मान् थ्स्या᳚त् । \newline
3. प॒शु॒मान् थ्स्या᳚थ् स्यात् पशु॒मान् प॑शु॒मान् थ्स्या॒ दितीति॑ स्यात् पशु॒मान् प॑शु॒मान् थ्स्या॒दिति॑ । \newline
4. प॒शु॒मानिति॑ पशु - मान् । \newline
5. स्या॒ दितीति॑ स्याथ् स्या॒दिति॑ परि॒मित्य॑ परि॒मित्येति॑ स्याथ् स्या॒दिति॑ परि॒मित्य॑ । \newline
6. इति॑ परि॒मित्य॑ परि॒मित्ये तीति॑ परि॒मित्य॒ तस्य॒ तस्य॑ परि॒मित्ये तीति॑ परि॒मित्य॒ तस्य॑ । \newline
7. प॒रि॒मित्य॒ तस्य॒ तस्य॑ परि॒मित्य॑ परि॒मित्य॒ तस्य॒ शर्क॑राः॒ शर्क॑रा॒ स्तस्य॑ परि॒मित्य॑ परि॒मित्य॒ तस्य॒ शर्क॑राः । \newline
8. प॒रि॒मित्येति॑ परि - मित्य॑ । \newline
9. तस्य॒ शर्क॑राः॒ शर्क॑रा॒ स्तस्य॒ तस्य॒ शर्क॑राः॒ सिक॑ताः॒ सिक॑ताः॒ शर्क॑रा॒ स्तस्य॒ तस्य॒ शर्क॑राः॒ सिक॑ताः । \newline
10. शर्क॑राः॒ सिक॑ताः॒ सिक॑ताः॒ शर्क॑राः॒ शर्क॑राः॒ सिक॑ता॒ वि वि सिक॑ताः॒ शर्क॑राः॒ शर्क॑राः॒ सिक॑ता॒ वि । \newline
11. सिक॑ता॒ वि वि सिक॑ताः॒ सिक॑ता॒ व्यू॑हे दूहे॒द् वि सिक॑ताः॒ सिक॑ता॒ व्यू॑हेत् । \newline
12. व्यू॑हे दूहे॒द् वि व्यू॑हे॒त् परि॑गृहीते॒ परि॑गृहीत ऊहे॒द् वि व्यू॑हे॒त् परि॑गृहीते । \newline
13. ऊ॒हे॒त् परि॑गृहीते॒ परि॑गृहीत ऊहे दूहे॒त् परि॑गृहीत ए॒वैव परि॑गृहीत ऊहे दूहे॒त् परि॑गृहीत ए॒व । \newline
14. परि॑गृहीत ए॒वैव परि॑गृहीते॒ परि॑गृहीत ए॒वास्मा॑ अस्मा ए॒व परि॑गृहीते॒ परि॑गृहीत ए॒वास्मै᳚ । \newline
15. परि॑गृहीत॒ इति॒ परि॑ - गृ॒ही॒ते॒ । \newline
16. ए॒वास्मा॑ अस्मा ए॒वैवास्मै॑ समी॒चीनꣳ॑ समी॒चीन॑ मस्मा ए॒वैवास्मै॑ समी॒चीन᳚म् । \newline
17. अ॒स्मै॒ स॒मी॒चीनꣳ॑ समी॒चीन॑ मस्मा अस्मै समी॒चीनꣳ॒॒ रेतो॒ रेतः॑ समी॒चीन॑ मस्मा अस्मै समी॒चीनꣳ॒॒ रेतः॑ । \newline
18. स॒मी॒चीनꣳ॒॒ रेतो॒ रेतः॑ समी॒चीनꣳ॑ समी॒चीनꣳ॒॒ रेतः॑ सिञ्चति सिञ्चति॒ रेतः॑ समी॒चीनꣳ॑ समी॒चीनꣳ॒॒ रेतः॑ सिञ्चति । \newline
19. रेतः॑ सिञ्चति सिञ्चति॒ रेतो॒ रेतः॑ सिञ्चति पशु॒मान् प॑शु॒मान् थ्सि॑ञ्चति॒ रेतो॒ रेतः॑ सिञ्चति पशु॒मान् । \newline
20. सि॒ञ्च॒ति॒ प॒शु॒मान् प॑शु॒मान् थ्सि॑ञ्चति सिञ्चति पशु॒मा ने॒वैव प॑शु॒मान् थ्सि॑ञ्चति सिञ्चति पशु॒मा ने॒व । \newline
21. प॒शु॒मा ने॒वैव प॑शु॒मान् प॑शु॒मा ने॒व भ॑वति भव त्ये॒व प॑शु॒मान् प॑शु॒मा ने॒व भ॑वति । \newline
22. प॒शु॒मानिति॑ पशु - मान् । \newline
23. ए॒व भ॑वति भव त्ये॒वैव भ॑वति सौ॒म्या सौ॒म्या भ॑व त्ये॒वैव भ॑वति सौ॒म्या । \newline
24. भ॒व॒ति॒ सौ॒म्या सौ॒म्या भ॑वति भवति सौ॒म्या वि वि सौ॒म्या भ॑वति भवति सौ॒म्या वि । \newline
25. सौ॒म्या वि वि सौ॒म्या सौ॒म्या व्यू॑ह त्यूहति॒ वि सौ॒म्या सौ॒म्या व्यू॑हति । \newline
26. व्यू॑ह त्यूहति॒ वि व्यू॑हति॒ सोमः॒ सोम॑ ऊहति॒ वि व्यू॑हति॒ सोमः॑ । \newline
27. ऊ॒ह॒ति॒ सोमः॒ सोम॑ ऊह त्यूहति॒ सोमो॒ वै वै सोम॑ ऊह त्यूहति॒ सोमो॒ वै । \newline
28. सोमो॒ वै वै सोमः॒ सोमो॒ वै रे॑तो॒धा रे॑तो॒धा वै सोमः॒ सोमो॒ वै रे॑तो॒धाः । \newline
29. वै रे॑तो॒धा रे॑तो॒धा वै वै रे॑तो॒धा रेतो॒ रेतो॑ रेतो॒धा वै वै रे॑तो॒धा रेतः॑ । \newline
30. रे॒तो॒धा रेतो॒ रेतो॑ रेतो॒धा रे॑तो॒धा रेत॑ ए॒वैव रेतो॑ रेतो॒धा रे॑तो॒धा रेत॑ ए॒व । \newline
31. रे॒तो॒धा इति॑ रेतः - धाः । \newline
32. रेत॑ ए॒वैव रेतो॒ रेत॑ ए॒व तत् तदे॒व रेतो॒ रेत॑ ए॒व तत् । \newline
33. ए॒व तत् तदे॒वैव तद् द॑धाति दधाति॒ तदे॒वैव तद् द॑धाति । \newline
34. तद् द॑धाति दधाति॒ तत् तद् द॑धाति गायत्रि॒या गा॑यत्रि॒या द॑धाति॒ तत् तद् द॑धाति गायत्रि॒या । \newline
35. द॒धा॒ति॒ गा॒य॒त्रि॒या गा॑यत्रि॒या द॑धाति दधाति गायत्रि॒या ब्रा᳚ह्म॒णस्य॑ ब्राह्म॒णस्य॑ गायत्रि॒या द॑धाति दधाति गायत्रि॒या ब्रा᳚ह्म॒णस्य॑ । \newline
36. गा॒य॒त्रि॒या ब्रा᳚ह्म॒णस्य॑ ब्राह्म॒णस्य॑ गायत्रि॒या गा॑यत्रि॒या ब्रा᳚ह्म॒णस्य॑ गाय॒त्रो गा॑य॒त्रो ब्रा᳚ह्म॒णस्य॑ गायत्रि॒या गा॑यत्रि॒या ब्रा᳚ह्म॒णस्य॑ गाय॒त्रः । \newline
37. ब्रा॒ह्म॒णस्य॑ गाय॒त्रो गा॑य॒त्रो ब्रा᳚ह्म॒णस्य॑ ब्राह्म॒णस्य॑ गाय॒त्रो हि हि गा॑य॒त्रो ब्रा᳚ह्म॒णस्य॑ ब्राह्म॒णस्य॑ गाय॒त्रो हि । \newline
38. गा॒य॒त्रो हि हि गा॑य॒त्रो गा॑य॒त्रो हि ब्रा᳚ह्म॒णो ब्रा᳚ह्म॒णो हि गा॑य॒त्रो गा॑य॒त्रो हि ब्रा᳚ह्म॒णः । \newline
39. हि ब्रा᳚ह्म॒णो ब्रा᳚ह्म॒णो हि हि ब्रा᳚ह्म॒ण स्त्रि॒ष्टुभा᳚ त्रि॒ष्टुभा᳚ ब्राह्म॒णो हि हि ब्रा᳚ह्म॒ण स्त्रि॒ष्टुभा᳚ । \newline
40. ब्रा॒ह्म॒ण स्त्रि॒ष्टुभा᳚ त्रि॒ष्टुभा᳚ ब्राह्म॒णो ब्रा᳚ह्म॒ण स्त्रि॒ष्टुभा॑ राज॒न्य॑स्य राज॒न्य॑स्य त्रि॒ष्टुभा᳚ ब्राह्म॒णो ब्रा᳚ह्म॒ण स्त्रि॒ष्टुभा॑ राज॒न्य॑स्य । \newline
41. त्रि॒ष्टुभा॑ राज॒न्य॑स्य राज॒न्य॑स्य त्रि॒ष्टुभा᳚ त्रि॒ष्टुभा॑ राज॒न्य॑स्य॒ त्रैष्टु॑भ॒ स्त्रैष्टु॑भो राज॒न्य॑स्य त्रि॒ष्टुभा᳚ त्रि॒ष्टुभा॑ राज॒न्य॑स्य॒ त्रैष्टु॑भः । \newline
42. रा॒ज॒न्य॑स्य॒ त्रैष्टु॑भ॒ स्त्रैष्टु॑भो राज॒न्य॑स्य राज॒न्य॑स्य॒ त्रैष्टु॑भो॒ हि हि त्रैष्टु॑भो राज॒न्य॑स्य राज॒न्य॑स्य॒ त्रैष्टु॑भो॒ हि । \newline
43. त्रैष्टु॑भो॒ हि हि त्रैष्टु॑भ॒ स्त्रैष्टु॑भो॒ हि रा॑ज॒न्यो॑ राज॒न्यो॑ हि त्रैष्टु॑भ॒ स्त्रैष्टु॑भो॒ हि रा॑ज॒न्यः॑ । \newline
44. हि रा॑ज॒न्यो॑ राज॒न्यो॑ हि हि रा॑ज॒न्यः॑ शं॒ॅयुꣳ शं॒ॅयुꣳ रा॑ज॒न्यो॑ हि हि रा॑ज॒न्यः॑ शं॒ॅयुम् । \newline
45. रा॒ज॒न्यः॑ शं॒ॅयुꣳ शं॒ॅयुꣳ रा॑ज॒न्यो॑ राज॒न्यः॑ शं॒ॅयुम् बा॑र्.हस्प॒त्यम् बा॑र्.हस्प॒त्यꣳ शं॒ॅयुꣳ रा॑ज॒न्यो॑ राज॒न्यः॑ शं॒ॅयुम् बा॑र्.हस्प॒त्यम् । \newline
46. शं॒ॅयुम् बा॑र्.हस्प॒त्यम् बा॑र्.हस्प॒त्यꣳ शं॒ॅयुꣳ शं॒ॅयुम् बा॑र्.हस्प॒त्यम् मेधो॒ मेधो॑ बार्.हस्प॒त्यꣳ शं॒ॅयुꣳ शं॒ॅयुम् बा॑र्.हस्प॒त्यम् मेधः॑ । \newline
47. शं॒ॅयुमिति॑ शं - युम् । \newline
48. बा॒र्॒.ह॒स्प॒त्यम् मेधो॒ मेधो॑ बार्.हस्प॒त्यम् बा॑र्.हस्प॒त्यम् मेधो॒ न न मेधो॑ बार्.हस्प॒त्यम् बा॑र्.हस्प॒त्यम् मेधो॒ न । \newline
49. मेधो॒ न न मेधो॒ मेधो॒ नोपोप॒ न मेधो॒ मेधो॒ नोप॑ । \newline
50. नोपोप॒ न नोपा॑ नम दनम॒ दुप॒ न नोपा॑ नमत् । \newline
51. उपा॑ नम दनम॒ दुपोपा॑ नम॒थ् स सो॑ ऽनम॒ दुपोपा॑ नम॒थ् सः । \newline
52. अ॒न॒म॒थ् स सो॑ ऽनम दनम॒थ् सो᳚ ऽग्नि म॒ग्निꣳ सो॑ ऽनम दनम॒थ् सो᳚ ऽग्निम् । \newline
53. सो᳚ ऽग्नि म॒ग्निꣳ स सो᳚ ऽग्निम् प्र प्राग्निꣳ स सो᳚ ऽग्निम् प्र । \newline
54. अ॒ग्निम् प्र प्राग्नि म॒ग्निम् प्रावि॑श दविश॒त् प्राग्नि म॒ग्निम् प्रावि॑शत् । \newline
55. प्रावि॑श दविश॒त् प्र प्रावि॑श॒थ् स सो॑ ऽविश॒त् प्र प्रावि॑श॒थ् सः । \newline
56. अ॒वि॒श॒थ् स सो॑ ऽविश दविश॒थ् सो᳚ ऽग्ने र॒ग्नेः सो॑ ऽविश दविश॒थ् सो᳚ ऽग्नेः । \newline
\pagebreak
\markright{ TS 5.2.6.5  \hfill https://www.vedavms.in \hfill}

\section{ TS 5.2.6.5 }

\textbf{TS 5.2.6.5 } \newline
\textbf{Samhita Paata} \newline

सो᳚ऽग्नेः कृष्णो॑ रू॒पं कृ॒त्वोदा॑यत॒ सोऽश्वं॒ प्राऽवि॑श॒थ् सोऽश्व॑स्या-वान्तरश॒फो॑-भव॒द्-यदश्व॑माक्र॒मय॑ति॒ य ए॒व मेधोऽश्वं॒ प्राऽवि॑श॒त् तमे॒वाव॑ रुन्धे प्र॒जाप॑तिना॒ऽग्निश्चे॑त॒व्य॑ इत्या॑हुः प्राजाप॒त्योऽश्वो॒ यदश्व॑माक्र॒मय॑ति प्र॒जाप॑तिनै॒वाऽग्निं चि॑नुते पुष्करप॒र्णमुप॑ दधाति॒ योनि॒र्वा अ॒ग्नेः पु॑ष्करप॒र्णꣳ सयो॑नि- ( ) -मे॒वाग्निं चि॑नुते॒ ऽपां पृ॒ष्ठम॒सीत्युप॑ दधात्य॒पां ॅवा ए॒तत् पृ॒ष्ठं ॅयत् पु॑ष्करप॒र्णꣳ रू॒पेणै॒वैन॒दुप॑ दधाति ॥ \newline

\textbf{Pada Paata} \newline

सः । अ॒ग्नेः । कृष्णः॑ । रू॒पम् । कृ॒त्वा । उदिति॑ । आ॒य॒त॒ । सः । अश्व᳚म् । प्रेति॑ । अ॒वि॒श॒त् । सः । अश्व॑स्य । अ॒वा॒न्त॒र॒श॒फ इत्य॑वान्तर - श॒फः । अ॒भ॒व॒त् । यत् । अश्व᳚म् । आ॒क्र॒मय॒तीत्या᳚ - क्र॒मय॑ति । यः । ए॒व । मेधः॑ । अश्व᳚म् । प्रेति॑ । अवि॑शत् । तम् । ए॒व । अवेति॑ । रु॒न्धे॒ । प्र॒जाप॑ति॒नेति॑ प्र॒जा - प॒ति॒ना॒ । अ॒ग्निः । चे॒त॒व्यः॑ । इति॑ । आ॒हुः॒ । प्रा॒जा॒प॒त्य इति॑ प्राजा - प॒त्यः । अश्वः॑ । यत् । अश्व᳚म् । आ॒क्र॒मय॒तीत्या᳚-क्र॒मय॑ति । प्र॒जाप॑ति॒नेति॑ प्र॒जा - प॒ति॒ना॒ । ए॒व । अ॒ग्निम् । चि॒नु॒ते॒ । पु॒ष्क॒र॒प॒र्णमिति॑ पुष्कर - प॒र्णम् । उपेति॑ । द॒धा॒ति॒ । योनिः॑ । वै । अ॒ग्नेः । पु॒ष्क॒र॒प॒र्णमिति॑ पुष्कर - प॒र्णम् । सयो॑नि॒मिति॒ स-यो॒नि॒म् ( ) । ए॒व । अ॒ग्निम् । चि॒नु॒ते॒ । अ॒पाम् । पृ॒ष्ठम् । अ॒सि॒ । इति॑ । उपेति॑ । द॒धा॒ति॒ । अ॒पाम् । वै । ए॒तत् । पृ॒ष्ठम् । यत् । पु॒ष्क॒र॒प॒र्णमिति॑ पुष्कर - प॒र्णम् । रू॒पेण॑ । ए॒व । एन॒त्॒ । उपेति॑ । द॒धा॒ति॒ ॥  \newline


\textbf{Krama Paata} \newline

सो᳚ऽग्नेः । अ॒ग्नेः कृष्णः॑ । कृष्णो॑ रू॒पम् । रू॒पम् कृ॒त्वा । कृ॒त्वोत् । उदा॑यत । आ॒य॒त॒ सः । सोऽश्व᳚म् । अश्व॒म् प्र । प्रावि॑शत् । अ॒वि॒श॒थ् सः । सोऽश्व॑स्य । अश्व॑स्यावान्तरश॒फः । अ॒वा॒न्त॒र॒श॒फो॑ऽभवत् । अ॒वा॒न्त॒र॒श॒फ इत्य॑वान्तर - श॒फः । अ॒भ॒व॒द् यत् । यदश्व᳚म् । अश्व॑माक्र॒मय॑ति । आ॒क्र॒मय॑ति॒ यः । आ॒क्र॒मय॒तीत्या᳚ - क्र॒मय॑ति । य ए॒व । ए॒व मेधः॑ । मेधोऽश्व᳚म् । अश्व॒म् प्र । प्रावि॑शत् । अवि॑श॒त् तम् । तमे॒व । ए॒वाव॑ । अव॑ रुन्धे । रु॒न्धे॒ प्र॒जाप॑तिना । प्र॒जाप॑तिना॒ऽग्निः । प्र॒जाप॑ति॒नेति॑ प्र॒जा - प॒ति॒ना॒ । अ॒ग्निश्चे॑त॒व्यः॑ । चे॒त॒व्य॑ इति॑ । इत्या॑हुः । आ॒हुः॒ प्रा॒जा॒प॒त्यः । प्रा॒जा॒प॒त्योऽश्वः॑ । प्रा॒जा॒प॒त्य इति॑ प्राजा - प॒त्यः । अश्वो॒ यत् । यदश्व᳚म् । अश्व॑माक्र॒मय॑ति । आ॒क्र॒मय॑ति प्र॒जाप॑तिना । आ॒क्र॒मय॒तीत्या᳚ - क्र॒मय॑ति । प्र॒जाप॑तिनै॒व । प्र॒जाप॑ति॒नेति॑ प्र॒जा - प॒ति॒ना॒ । ए॒वाग्निम् । अ॒ग्निम् चि॑नुते । चि॒नु॒ते॒ पु॒ष्क॒र॒प॒र्णम् । पु॒ष्क॒र॒प॒र्णमुप॑ । पु॒ष्क॒र॒प॒र्णमिति॑ पुष्कर - प॒र्णम् । उप॑ दधाति । द॒धा॒ति॒ योनिः॑ । योनि॒र् वै । वा अ॒ग्नेः । अ॒ग्नेः पु॑ष्करप॒र्णम् । पु॒ष्क॒र॒प॒र्णꣳ सयो॑निम् ( ) । पु॒ष्क॒र॒प॒र्णमिति॑ पुष्कर - प॒र्णम् । सयो॑निमे॒व । सयो॑नि॒मिति॒ स - यो॒नि॒म् । ए॒वाग्निम् । अ॒ग्निम् चि॑नुते । चि॒नु॒ते॒ऽपाम् । अ॒पाम् पृ॒ष्ठम् । पृ॒ष्ठम॑सि । अ॒सीति॑ । इत्युप॑ । उप॑ दधाति । द॒धा॒त्य॒पाम् । अ॒पाम् ॅवै । वा ए॒तत् । ए॒तत् पृ॒ष्ठम् । पृ॒ष्ठम् ॅयत् । यत् पु॑ष्करप॒र्णम् । पु॒ष्क॒र॒प॒र्णꣳ रू॒पेण॑ । पु॒ष्क॒र॒प॒र्णमिति॑ पुष्कर - प॒र्णम् । रू॒पेणै॒व । ए॒वैन॑त् । ए॒न॒दुप॑ । उप॑ दधाति । द॒धा॒तीति॑ दधाति । \newline

\textbf{Jatai Paata} \newline

1. सो᳚ ऽग्ने र॒ग्नेः स सो᳚ ऽग्नेः । \newline
2. अ॒ग्नेः कृष्णः॒ कृष्णो॒ ऽग्ने र॒ग्नेः कृष्णः॑ । \newline
3. कृष्णो॑ रू॒पꣳ रू॒पम् कृष्णः॒ कृष्णो॑ रू॒पम् । \newline
4. रू॒पम् कृ॒त्वा कृ॒त्वा रू॒पꣳ रू॒पम् कृ॒त्वा । \newline
5. कृ॒त्वो दुत् कृ॒त्वा कृ॒त्वोत् । \newline
6. उदा॑यता य॒तोदु दा॑यत । \newline
7. आ॒य॒त॒ स स आ॑यता यत॒ सः । \newline
8. सो ऽश्व॒ मश्वꣳ॒॒ स सो ऽश्व᳚म् । \newline
9. अश्व॒म् प्र प्राश्व॒ मश्व॒म् प्र । \newline
10. प्रावि॑श दविश॒त् प्र प्रावि॑शत् । \newline
11. अ॒वि॒श॒थ् स सो॑ ऽविश दविश॒थ् सः । \newline
12. सो ऽश्व॒स्या श्व॑स्य॒ स सो ऽश्व॑स्य । \newline
13. अश्व॑स्यावा न्तरश॒फो॑ ऽवान्तरश॒फो ऽश्व॒स्या श्व॑स्या वान्तरश॒फः । \newline
14. अ॒वा॒न्त॒र॒श॒फो॑ ऽभव दभव दवान्तरश॒फो॑ ऽवान्तरश॒फो॑ ऽभवत् । \newline
15. अ॒वा॒न्त॒र॒श॒फ इत्य॑वान्तर - श॒फः । \newline
16. अ॒भ॒व॒द् यद् यद॑भव दभव॒द् यत् । \newline
17. यदश्व॒ मश्वं॒ ॅयद् यदश्व᳚म् । \newline
18. अश्व॑ माक्र॒मय॑ त्याक्र॒मय॒ त्यश्व॒ मश्व॑ माक्र॒मय॑ति । \newline
19. आ॒क्र॒मय॑ति॒ यो य आ᳚क्र॒मय॑ त्याक्र॒मय॑ति॒ यः । \newline
20. आ॒क्र॒मय॒तीत्या᳚ - क्र॒मय॑ति । \newline
21. य ए॒वैव यो य ए॒व । \newline
22. ए॒व मेधो॒ मेध॑ ए॒वैव मेधः॑ । \newline
23. मेधो ऽश्व॒ मश्व॒म् मेधो॒ मेधो ऽश्व᳚म् । \newline
24. अश्व॒म् प्र प्राश्व॒ मश्व॒म् प्र । \newline
25. प्रावि॑श॒ दवि॑श॒त् प्र प्रावि॑शत् । \newline
26. अवि॑श॒त् तम् त मवि॑श॒ दवि॑श॒त् तम् । \newline
27. त मे॒वैव तम् त मे॒व । \newline
28. ए॒वावा वै॒वै वाव॑ । \newline
29. अव॑ रुन्धे रु॒न्धे ऽवाव॑ रुन्धे । \newline
30. रु॒न्धे॒ प्र॒जाप॑तिना प्र॒जाप॑तिना रुन्धे रुन्धे प्र॒जाप॑तिना । \newline
31. प्र॒जाप॑तिना॒ ऽग्नि र॒ग्निः प्र॒जाप॑तिना प्र॒जाप॑तिना॒ ऽग्निः । \newline
32. प्र॒जाप॑ति॒नेति॑ प्र॒जा - प॒ति॒ना॒ । \newline
33. अ॒ग्नि श्चे॑त॒व्य॑ श्चेत॒व्यो᳚ ऽग्नि र॒ग्नि श्चे॑त॒व्यः॑ । \newline
34. चे॒त॒व्य॑ इतीति॑ चेत॒व्य॑ श्चेत॒व्य॑ इति॑ । \newline
35. इत्या॑हु राहु॒ रिती त्या॑हुः । \newline
36. आ॒हुः॒ प्रा॒जा॒प॒त्यः प्रा॑जाप॒त्य आ॑हु राहुः प्राजाप॒त्यः । \newline
37. प्रा॒जा॒प॒त्यो ऽश्वो ऽश्वः॑ प्राजाप॒त्यः प्रा॑जाप॒त्यो ऽश्वः॑ । \newline
38. प्रा॒जा॒प॒त्य इति॑ प्राजा - प॒त्यः । \newline
39. अश्वो॒ यद् यदश्वो ऽश्वो॒ यत् । \newline
40. यदश्व॒ मश्वं॒ ॅयद् यदश्व᳚म् । \newline
41. अश्व॑ माक्र॒मय॑ त्याक्र॒मय॒ त्यश्व॒ मश्व॑ माक्र॒मय॑ति । \newline
42. आ॒क्र॒मय॑ति प्र॒जाप॑तिना प्र॒जाप॑तिना ऽऽक्र॒मय॑ त्याक्र॒मय॑ति प्र॒जाप॑तिना । \newline
43. आ॒क्र॒मय॒तीत्या᳚ - क्र॒मय॑ति । \newline
44. प्र॒जाप॑ति नै॒वैव प्र॒जाप॑तिना प्र॒जाप॑ति नै॒व । \newline
45. प्र॒जाप॑ति॒नेति॑ प्र॒जा - प॒ति॒ना॒ । \newline
46. ए॒वाग्नि म॒ग्नि मे॒वै वाग्निम् । \newline
47. अ॒ग्निम् चि॑नुते चिनुते॒ ऽग्नि म॒ग्निम् चि॑नुते । \newline
48. चि॒नु॒ते॒ पु॒ष्क॒र॒प॒र्णम् पु॑ष्करप॒र्णम् चि॑नुते चिनुते पुष्करप॒र्णम् । \newline
49. पु॒ष्क॒र॒प॒र्ण मुपोप॑ पुष्करप॒र्णम् पु॑ष्करप॒र्ण मुप॑ । \newline
50. पु॒ष्क॒र॒प॒र्णमिति॑ पुष्कर - प॒र्णम् । \newline
51. उप॑ दधाति दधा॒ त्युपोप॑ दधाति । \newline
52. द॒धा॒ति॒ योनि॒र् योनि॑र् दधाति दधाति॒ योनिः॑ । \newline
53. योनि॒र् वै वै योनि॒र् योनि॒र् वै । \newline
54. वा अ॒ग्ने र॒ग्नेर् वै वा अ॒ग्नेः । \newline
55. अ॒ग्नेः पु॑ष्करप॒र्णम् पु॑ष्करप॒र्ण म॒ग्नेर॒ग्नेः पु॑ष्करप॒र्णम् । \newline
56. पु॒ष्क॒र॒प॒र्णꣳ सयो॑निꣳ॒॒ सयो॑निम् पुष्करप॒र्णम् पु॑ष्करप॒र्णꣳ सयो॑निम् । \newline
57. पु॒ष्क॒र॒प॒र्णमिति॑ पुष्कर - प॒र्णम् । \newline
58. सयो॑नि मे॒वैव सयो॑निꣳ॒॒ सयो॑नि मे॒व । \newline
59. सयो॑नि॒मिति॒ स - यो॒नि॒म् । \newline
60. ए॒वाग्नि म॒ग्नि मे॒वै वाग्निम् । \newline
61. अ॒ग्निम् चि॑नुते चिनुते॒ ऽग्नि म॒ग्निम् चि॑नुते । \newline
62. चि॒नु॒ते॒ ऽपा म॒पाम् चि॑नुते चिनुते॒ ऽपाम् । \newline
63. अ॒पाम् पृ॒ष्ठम् पृ॒ष्ठ म॒पा म॒पाम् पृ॒ष्ठम् । \newline
64. पृ॒ष्ठ म॑स्यसि पृ॒ष्ठम् पृ॒ष्ठ म॑सि । \newline
65. अ॒सी तीत्य॑स्य॒ सीति॑ । \newline
66. इत्यु पोपे तीत्युप॑ । \newline
67. उप॑ दधाति दधा॒ त्युपोप॑ दधाति । \newline
68. द॒धा॒ त्य॒पा म॒पाम् द॑धाति दधा त्य॒पाम् । \newline
69. अ॒पां ॅवै वा अ॒पा म॒पां ॅवै । \newline
70. वा ए॒त दे॒तद् वै वा ए॒तत् । \newline
71. ए॒तत् पृ॒ष्ठम् पृ॒ष्ठ मे॒त दे॒तत् पृ॒ष्ठम् । \newline
72. पृ॒ष्ठं ॅयद् यत् पृ॒ष्ठम् पृ॒ष्ठं ॅयत् । \newline
73. यत् पु॑ष्करप॒र्णम् पु॑ष्करप॒र्णं ॅयद् यत् पु॑ष्करप॒र्णम् । \newline
74. पु॒ष्क॒र॒प॒र्णꣳ रू॒पेण॑ रू॒पेण॑ पुष्करप॒र्णम् पु॑ष्करप॒र्णꣳ रू॒पेण॑ । \newline
75. पु॒ष्क॒र॒प॒र्णमिति॑ पुष्कर - प॒र्णम् । \newline
76. रू॒पे णै॒वैव रू॒पेण॑ रू॒पेणै॒व । \newline
77. ए॒वैन॑ देन दे॒वै वैन॑त् । \newline
78. ए॒न॒ दुपोपै॑न देन॒ दुप॑ । \newline
79. उप॑ दधाति दधा॒ त्युपोप॑ दधाति । \newline
80. द॒धा॒तीति॑ दधाति । \newline

\textbf{Ghana Paata } \newline

1. सो᳚ ऽग्ने र॒ग्नेः स सो᳚ ऽग्नेः कृष्णः॒ कृष्णो॒ ऽग्नेः स सो᳚ ऽग्नेः कृष्णः॑ । \newline
2. अ॒ग्नेः कृष्णः॒ कृष्णो॒ ऽग्ने र॒ग्नेः कृष्णो॑ रू॒पꣳ रू॒पम् कृष्णो॒ ऽग्ने र॒ग्नेः कृष्णो॑ रू॒पम् । \newline
3. कृष्णो॑ रू॒पꣳ रू॒पम् कृष्णः॒ कृष्णो॑ रू॒पम् कृ॒त्वा कृ॒त्वा रू॒पम् कृष्णः॒ कृष्णो॑ रू॒पम् कृ॒त्वा । \newline
4. रू॒पम् कृ॒त्वा कृ॒त्वा रू॒पꣳ रू॒पम् कृ॒त्वो दुत् कृ॒त्वा रू॒पꣳ रू॒पम् कृ॒त्वोत् । \newline
5. कृ॒त्वो दुत् कृ॒त्वा कृ॒त्वो दा॑यता य॒तोत् कृ॒त्वा कृ॒त्वो दा॑यत । \newline
6. उदा॑यता य॒तो दुदा॑यत॒ स स आ॑य॒तो दुदा॑यत॒ सः । \newline
7. आ॒य॒त॒ स स आ॑यता यत॒ सो ऽश्व॒ मश्वꣳ॒॒ स आ॑यता यत॒ सो ऽश्व᳚म् । \newline
8. सो ऽश्व॒ मश्वꣳ॒॒ स सो ऽश्व॒म् प्र प्राश्वꣳ॒॒ स सो ऽश्व॒म् प्र । \newline
9. अश्व॒म् प्र प्राश्व॒ मश्व॒म् प्रावि॑श दविश॒त् प्राश्व॒ मश्व॒म् प्रावि॑शत् । \newline
10. प्रावि॑श दविश॒त् प्र प्रावि॑श॒थ् स सो॑ ऽविश॒त् प्र प्रावि॑श॒थ् सः । \newline
11. अ॒वि॒श॒थ् स सो॑ ऽविश दविश॒थ् सो ऽश्व॒स्या श्व॑स्य॒ सो॑ ऽविश दविश॒थ् सो ऽश्व॑स्य । \newline
12. सो ऽश्व॒स्या श्व॑स्य॒ स सो ऽश्व॑स्या वान्तरश॒फो॑ ऽवान्तरश॒फो ऽश्व॑स्य॒ स सो ऽश्व॑स्या वान्तरश॒फः । \newline
13. अश्व॑स्या वान्तरश॒फो॑ ऽवान्तरश॒फो ऽश्व॒स्या श्व॑स्याव् आन्तरश॒फो॑ ऽभव दभव दवान्तरश॒फो ऽश्व॒स्या श्व॑स्या वान्तरश॒फो॑ ऽभवत् । \newline
14. अ॒वा॒न्त॒र॒श॒फो॑ ऽभव दभव दवान्तरश॒फो॑ ऽवान्तरश॒फो॑ ऽभव॒द् यद् यद॑भव दवान्तरश॒फो॑ ऽवान्तरश॒फो॑ ऽभव॒द् यत् । \newline
15. अ॒वा॒न्त॒र॒श॒फ इत्य॑वान्तर - श॒फः । \newline
16. अ॒भ॒व॒द् यद् यद॑भव दभव॒द् यदश्व॒ मश्वं॒ ॅयद॑भव दभव॒द् यदश्व᳚म् । \newline
17. यदश्व॒ मश्वं॒ ॅयद् यदश्व॑ माक्र॒मय॑ त्याक्र॒मय॒ त्यश्वं॒ ॅयद् यदश्व॑ माक्र॒मय॑ति । \newline
18. अश्व॑ माक्र॒मय॑ त्याक्र॒मय॒ त्यश्व॒ मश्व॑ माक्र॒मय॑ति॒ यो य आ᳚क्र॒मय॒ त्यश्व॒ मश्व॑ माक्र॒मय॑ति॒ यः । \newline
19. आ॒क्र॒मय॑ति॒ यो य आ᳚क्र॒मय॑ त्याक्र॒मय॑ति॒ य ए॒वैव य आ᳚क्र॒मय॑ त्याक्र॒मय॑ति॒ य ए॒व । \newline
20. आ॒क्र॒मय॒तीत्या᳚ - क्र॒मय॑ति । \newline
21. य ए॒वैव यो य ए॒व मेधो॒ मेध॑ ए॒व यो य ए॒व मेधः॑ । \newline
22. ए॒व मेधो॒ मेध॑ ए॒वैव मेधो ऽश्व॒ मश्व॒म् मेध॑ ए॒वैव मेधो ऽश्व᳚म् । \newline
23. मेधो ऽश्व॒ मश्व॒म् मेधो॒ मेधो ऽश्व॒म् प्र प्राश्व॒म् मेधो॒ मेधो ऽश्व॒म् प्र । \newline
24. अश्व॒म् प्र प्राश्व॒ मश्व॒म् प्रावि॑श॒ दवि॑श॒त् प्राश्व॒ मश्व॒म् प्रावि॑शत् । \newline
25. प्रावि॑श॒ दवि॑श॒त् प्र प्रावि॑श॒त् तम् त मवि॑श॒त् प्र प्रावि॑श॒त् तम् । \newline
26. अवि॑श॒त् तम् त मवि॑श॒ दवि॑श॒त् त मे॒वैव त मवि॑श॒ दवि॑श॒त् त मे॒व । \newline
27. त मे॒वैव तम् त मे॒वावा वै॒व तम् त मे॒वाव॑ । \newline
28. ए॒वावा वै॒वै वाव॑ रुन्धे रु॒न्धे ऽवै॒वै वाव॑ रुन्धे । \newline
29. अव॑ रुन्धे रु॒न्धे ऽवाव॑ रुन्धे प्र॒जाप॑तिना प्र॒जाप॑तिना रु॒न्धे ऽवाव॑ रुन्धे प्र॒जाप॑तिना । \newline
30. रु॒न्धे॒ प्र॒जाप॑तिना प्र॒जाप॑तिना रुन्धे रुन्धे प्र॒जाप॑तिना॒ ऽग्नि र॒ग्निः प्र॒जाप॑तिना रुन्धे रुन्धे प्र॒जाप॑तिना॒ ऽग्निः । \newline
31. प्र॒जाप॑तिना॒ ऽग्नि र॒ग्निः प्र॒जाप॑तिना प्र॒जाप॑तिना॒ ऽग्नि श्चे॑त॒व्य॑ श्चेत॒व्यो᳚ ऽग्निः प्र॒जाप॑तिना प्र॒जाप॑तिना॒ ऽग्नि श्चे॑त॒व्यः॑ । \newline
32. प्र॒जाप॑ति॒नेति॑ प्र॒जा - प॒ति॒ना॒ । \newline
33. अ॒ग्नि श्चे॑त॒व्य॑ श्चेत॒व्यो᳚ ऽग्नि र॒ग्नि श्चे॑त॒व्य॑ इतीति॑ चेत॒व्यो᳚ ऽग्नि र॒ग्नि श्चे॑त॒व्य॑ इति॑ । \newline
34. चे॒त॒व्य॑ इतीति॑ चेत॒व्य॑ श्चेत॒व्य॑ इत्या॑हु राहु॒ रिति॑ चेत॒व्य॑ श्चेत॒व्य॑ इत्या॑हुः । \newline
35. इत्या॑हु राहु॒ रितीत्या॑हुः प्राजाप॒त्यः प्रा॑जाप॒त्य आ॑हु॒ रितीत्या॑हुः प्राजाप॒त्यः । \newline
36. आ॒हुः॒ प्रा॒जा॒प॒त्यः प्रा॑जाप॒त्य आ॑हु राहुः प्राजाप॒त्यो ऽश्वो ऽश्वः॑ प्राजाप॒त्य आ॑हु राहुः प्राजाप॒त्यो ऽश्वः॑ । \newline
37. प्रा॒जा॒प॒त्यो ऽश्वो ऽश्वः॑ प्राजाप॒त्यः प्रा॑जाप॒त्यो ऽश्वो॒ यद् यदश्वः॑ प्राजाप॒त्यः प्रा॑जाप॒त्यो ऽश्वो॒ यत् । \newline
38. प्रा॒जा॒प॒त्य इति॑ प्राजा - प॒त्यः । \newline
39. अश्वो॒ यद् यदश्वो ऽश्वो॒ यदश्व॒ मश्वं॒ ॅयदश्वो ऽश्वो॒ यदश्व᳚म् । \newline
40. यदश्व॒ मश्वं॒ ॅयद् यदश्व॑ माक्र॒मय॑ त्याक्र॒मय॒ त्यश्वं॒ ॅयद् यदश्व॑ माक्र॒मय॑ति । \newline
41. अश्व॑ माक्र॒मय॑ त्याक्र॒मय॒ त्यश्व॒ मश्व॑ माक्र॒मय॑ति प्र॒जाप॑तिना प्र॒जाप॑तिना ऽऽक्र॒मय॒ त्यश्व॒ मश्व॑ माक्र॒मय॑ति प्र॒जाप॑तिना । \newline
42. आ॒क्र॒मय॑ति प्र॒जाप॑तिना प्र॒जाप॑तिना ऽऽक्र॒मय॑ त्याक्र॒मय॑ति प्र॒जाप॑ति नै॒वैव प्र॒जाप॑तिना ऽऽक्र॒मय॑ त्याक्र॒मय॑ति प्र॒जाप॑ति नै॒व । \newline
43. आ॒क्र॒मय॒तीत्या᳚ - क्र॒मय॑ति । \newline
44. प्र॒जाप॑ति नै॒वैव प्र॒जाप॑तिना प्र॒जाप॑ति नै॒वाग्नि म॒ग्नि मे॒व प्र॒जाप॑तिना प्र॒जाप॑ति नै॒वाग्निम् । \newline
45. प्र॒जाप॑ति॒नेति॑ प्र॒जा - प॒ति॒ना॒ । \newline
46. ए॒वाग्नि म॒ग्नि मे॒वैवाग्निम् चि॑नुते चिनुते॒ ऽग्नि मे॒वैवाग्निम् चि॑नुते । \newline
47. अ॒ग्निम् चि॑नुते चिनुते॒ ऽग्नि म॒ग्निम् चि॑नुते पुष्करप॒र्णम् पु॑ष्करप॒र्णम् चि॑नुते॒ ऽग्नि म॒ग्निम् चि॑नुते पुष्करप॒र्णम् । \newline
48. चि॒नु॒ते॒ पु॒ष्क॒र॒प॒र्णम् पु॑ष्करप॒र्णम् चि॑नुते चिनुते पुष्करप॒र्ण मुपोप॑ पुष्करप॒र्णम् चि॑नुते चिनुते पुष्करप॒र्ण मुप॑ । \newline
49. पु॒ष्क॒र॒प॒र्ण मुपोप॑ पुष्करप॒र्णम् पु॑ष्करप॒र्ण मुप॑ दधाति दधा॒ त्युप॑ पुष्करप॒र्णम् पु॑ष्करप॒र्ण मुप॑ दधाति । \newline
50. पु॒ष्क॒र॒प॒र्णमिति॑ पुष्कर - प॒र्णम् । \newline
51. उप॑ दधाति दधा॒ त्युपोप॑ दधाति॒ योनि॒र् योनि॑र् दधा॒ त्युपोप॑ दधाति॒ योनिः॑ । \newline
52. द॒धा॒ति॒ योनि॒र् योनि॑र् दधाति दधाति॒ योनि॒र् वै वै योनि॑र् दधाति दधाति॒ योनि॒र् वै । \newline
53. योनि॒र् वै वै योनि॒र् योनि॒र् वा अ॒ग्ने र॒ग्नेर् वै योनि॒र् योनि॒र् वा अ॒ग्नेः । \newline
54. वा अ॒ग्ने र॒ग्नेर् वै वा अ॒ग्नेः पु॑ष्करप॒र्णम् पु॑ष्करप॒र्ण म॒ग्नेर् वै वा अ॒ग्नेः पु॑ष्करप॒र्णम् । \newline
55. अ॒ग्नेः पु॑ष्करप॒र्णम् पु॑ष्करप॒र्ण म॒ग्ने र॒ग्नेः पु॑ष्करप॒र्णꣳ सयो॑निꣳ॒॒ सयो॑निम् पुष्करप॒र्ण म॒ग्ने र॒ग्नेः पु॑ष्करप॒र्णꣳ सयो॑निम् । \newline
56. पु॒ष्क॒र॒प॒र्णꣳ सयो॑निꣳ॒॒ सयो॑निम् पुष्करप॒र्णम् पु॑ष्करप॒र्णꣳ सयो॑नि मे॒वैव सयो॑निम् पुष्करप॒र्णम् पु॑ष्करप॒र्णꣳ सयो॑नि मे॒व । \newline
57. पु॒ष्क॒र॒प॒र्णमिति॑ पुष्कर - प॒र्णम् । \newline
58. सयो॑नि मे॒वैव सयो॑निꣳ॒॒ सयो॑नि मे॒वाग्नि म॒ग्नि मे॒व सयो॑निꣳ॒॒ सयो॑नि मे॒वाग्निम् । \newline
59. सयो॑नि॒मिति॒ स - यो॒नि॒म् । \newline
60. ए॒वाग्नि म॒ग्नि मे॒वैवाग्निम् चि॑नुते चिनुते॒ ऽग्नि मे॒वैवाग्निम् चि॑नुते । \newline
61. अ॒ग्निम् चि॑नुते चिनुते॒ ऽग्नि म॒ग्निम् चि॑नुते॒ ऽपा म॒पाम् चि॑नुते॒ ऽग्नि म॒ग्निम् चि॑नुते॒ ऽपाम् । \newline
62. चि॒नु॒ते॒ ऽपा म॒पाम् चि॑नुते चिनुते॒ ऽपाम् पृ॒ष्ठम् पृ॒ष्ठ म॒पाम् चि॑नुते चिनुते॒ ऽपाम् पृ॒ष्ठम् । \newline
63. अ॒पाम् पृ॒ष्ठम् पृ॒ष्ठ म॒पा म॒पाम् पृ॒ष्ठ म॑स्यसि पृ॒ष्ठ म॒पा म॒पाम् पृ॒ष्ठ म॑सि । \newline
64. पृ॒ष्ठ म॑स्यसि पृ॒ष्ठम् पृ॒ष्ठ म॒सीती त्य॑सि पृ॒ष्ठम् पृ॒ष्ठ म॒सीति॑ । \newline
65. अ॒सीती त्य॑स्य॒सी त्युपोपे त्य॑स्य॒सी त्युप॑ । \newline
66. इत्युपोपेती त्युप॑ दधाति दधा॒ त्युपेती त्युप॑ दधाति । \newline
67. उप॑ दधाति दधा॒ त्युपोप॑ दधा त्य॒पा म॒पाम् द॑धा॒ त्युपोप॑ दधा त्य॒पाम् । \newline
68. द॒धा॒ त्य॒पा म॒पाम् द॑धाति दधा त्य॒पां ॅवै वा अ॒पाम् द॑धाति दधा त्य॒पां ॅवै । \newline
69. अ॒पां ॅवै वा अ॒पा म॒पां ॅवा ए॒त दे॒तद् वा अ॒पा म॒पां ॅवा ए॒तत् । \newline
70. वा ए॒त दे॒तद् वै वा ए॒तत् पृ॒ष्ठम् पृ॒ष्ठ मे॒तद् वै वा ए॒तत् पृ॒ष्ठम् । \newline
71. ए॒तत् पृ॒ष्ठम् पृ॒ष्ठ मे॒त दे॒तत् पृ॒ष्ठं ॅयद् यत् पृ॒ष्ठ मे॒त दे॒तत् पृ॒ष्ठं ॅयत् । \newline
72. पृ॒ष्ठं ॅयद् यत् पृ॒ष्ठम् पृ॒ष्ठं ॅयत् पु॑ष्करप॒र्णम् पु॑ष्करप॒र्णं ॅयत् पृ॒ष्ठम् पृ॒ष्ठं ॅयत् पु॑ष्करप॒र्णम् । \newline
73. यत् पु॑ष्करप॒र्णम् पु॑ष्करप॒र्णं ॅयद् यत् पु॑ष्करप॒र्णꣳ रू॒पेण॑ रू॒पेण॑ पुष्करप॒र्णं ॅयद् यत् पु॑ष्करप॒र्णꣳ रू॒पेण॑ । \newline
74. पु॒ष्क॒र॒प॒र्णꣳ रू॒पेण॑ रू॒पेण॑ पुष्करप॒र्णम् पु॑ष्करप॒र्णꣳ रू॒पे णै॒वैव रू॒पेण॑ पुष्करप॒र्णम् पु॑ष्करप॒र्णꣳ रू॒पेणै॒व । \newline
75. पु॒ष्क॒र॒प॒र्णमिति॑ पुष्कर - प॒र्णम् । \newline
76. रू॒पे णै॒वैव रू॒पेण॑ रू॒पे णै॒वैन॑ देन दे॒व रू॒पेण॑ रू॒पे णै॒वैन॑त् । \newline
77. ए॒वैन॑ देन दे॒वैवैन॒ दुपो पै॑न दे॒वैवैन॒ दुप॑ । \newline
78. ए॒न॒ दुपो पै॑न देन॒ दुप॑ दधाति दधा॒ त्युपै॑न देन॒ दुप॑ दधाति । \newline
79. उप॑ दधाति दधा॒ त्युपोप॑ दधाति । \newline
80. द॒धा॒तीति॑ दधाति । \newline
\pagebreak
\markright{ TS 5.2.7.1  \hfill https://www.vedavms.in \hfill}

\section{ TS 5.2.7.1 }

\textbf{TS 5.2.7.1 } \newline
\textbf{Samhita Paata} \newline

ब्रह्म॑ जज्ञा॒नमिति॑ रु॒क्ममुप॑ दधाति॒ ब्रह्म॑मुखा॒ वै प्र॒जाप॑तिः प्र॒जा अ॑सृजत॒ ब्रह्म॑मुखा ए॒व तत् प्र॒जा यज॑मानः सृजते॒ ब्रह्म॑ जज्ञा॒नमित्या॑ह॒ तस्मा᳚द्ब्राह्म॒णो मुख्यो॒ मुख्यो॑ भवति॒ य ए॒वं ॅवेद॑ ब्रह्मवा॒दिनो॑ वदन्ति॒ न पृ॑थि॒व्यां नान्तरि॑क्षे॒ न दि॒व्य॑ग्निश्चे॑त॒व्य॑ इति॒ यत् पृ॑थि॒व्यां चि॑न्वी॒त पृ॑थि॒वीꣳ शु॒चाऽर्प॑ये॒न्नौष॑धयो॒ न वन॒स्पत॑यः॒ - [  ] \newline

\textbf{Pada Paata} \newline

ब्रह्म॑ । ज॒ज्ञा॒नम् । इति॑ । रु॒क्मम् । उपेति॑ । द॒धा॒ति॒ । ब्रह्म॑मुखा॒ इति॒ ब्रह्म॑ - मु॒खाः॒ । वै । प्र॒जाप॑ति॒रिति॑ प्र॒जा - प॒तिः॒ । प्र॒जा इति॑ प्र - जाः । अ॒सृ॒ज॒त॒ । ब्रह्म॑मुखा॒ इति॒ ब्रह्म॑ - मु॒खाः॒ । ए॒व । तत् । प्र॒जा इति॑ प्र - जाः । यज॑मानः । सृ॒ज॒ते॒ । ब्रह्म॑ । ज॒ज्ञा॒नम् । इति॑ । आ॒ह॒ । तस्मा᳚त् । ब्रा॒ह्म॒णः । मुख्यः॑ । मुख्यः॑ । भ॒व॒ति॒ । यः । ए॒वम् । वेद॑ । ब्र॒ह्म॒वा॒दिन॒ इति॑ ब्रह्म - वा॒दिनः॑ । व॒द॒न्ति॒ । न । पृ॒थि॒व्याम् । न । अ॒न्तरि॑क्षे । न । दि॒वि । अ॒ग्निः । चे॒त॒व्यः॑ । इति॑ । यत् । पृ॒थि॒व्याम् । चि॒न्वी॒त । पृ॒थि॒वीम् । शु॒चा । अ॒र्प॒ये॒त् । न । ओष॑धयः । न । वन॒स्पत॑यः ।  \newline


\textbf{Krama Paata} \newline

ब्रह्म॑ जज्ञा॒नम् । ज॒ज्ञा॒नमिति॑ । इति॑ रु॒क्मम् । रु॒क्ममुप॑ । उप॑ दधाति । द॒धा॒ति॒ ब्रह्म॑मुखाः । ब्रह्म॑मुखा॒ वै । ब्रह्म॑मुखा॒ इति॒ ब्रह्म॑ - मु॒खाः॒ । वै प्र॒जाप॑तिः । प्र॒जाप॑तिः प्र॒जाः । प्र॒जाप॑ति॒रिति॑ प्र॒जा - प॒तिः॒ । प्र॒जा अ॑सृजत । प्र॒जा इति॑ प्र - जाः । अ॒सृ॒ज॒त॒ ब्रह्म॑मुखाः । ब्रह्म॑मुखा ए॒व । ब्रह्म॑मुखा॒ इति॒ ब्रह्म॑ - मु॒खाः॒ । ए॒व तत् । तत् प्र॒जाः । प्र॒जा यज॑मानः । प्र॒जा इति॑ प्र - जाः । यज॑मानः सृजते । सृ॒ज॒ते॒ ब्रह्म॑ । ब्रह्म॑ जज्ञा॒नम् । ज॒ज्ञा॒नमिति॑ । इत्या॑ह । आ॒ह॒ तस्मा᳚त् । तस्मा᳚द् ब्राह्म॒णः । ब्रा॒ह्म॒णो मुख्यः॑ । मुख्यो॒ मुख्यः॑ । मुख्यो॑ भवति । भ॒व॒ति॒ यः । य ए॒वम् । ए॒वम् ॅवेद॑ । वेद॑ ब्रह्मवा॒दिनः॑ । ब्र॒ह्म॒वा॒दिनो॑ वदन्ति । ब्र॒ह्म॒वा॒दिन॒ इति॑ ब्रह्म - वा॒दिनः॑ । व॒द॒न्ति॒ न । न पृ॑थि॒व्याम् । पृ॒थि॒व्याम् न । नान्तरि॑क्षे । अ॒न्तरि॑क्षे॒ न । न दि॒वि । दि॒व्य॑ग्निः । अ॒ग्निश्चे॑त॒व्यः॑ । चे॒त॒व्य॑ इति॑ । इति॒ यत् । यत् पृ॑थि॒व्याम् । पृ॒थि॒व्याम् चि॑न्वी॒त । चि॒न्वी॒त पृ॑थि॒वीम् । पृ॒थि॒वीꣳ शु॒चा । शु॒चाऽर्प॑येत् । अ॒र्प॒ये॒न् न । नौष॑धयः । ओष॑धयो॒ न । न वन॒स्पत॑यः । वन॒स्पत॑यः॒ प्र \newline

\textbf{Jatai Paata} \newline

1. ब्रह्म॑ जज्ञा॒नम् ज॑ज्ञा॒नम् ब्रह्म॒ ब्रह्म॑ जज्ञा॒नम् । \newline
2. ज॒ज्ञा॒न मितीति॑ जज्ञा॒नम् ज॑ज्ञा॒न मिति॑ । \newline
3. इति॑ रु॒क्मꣳ रु॒क्म मितीति॑ रु॒क्मम् । \newline
4. रु॒क्म मुपोप॑ रु॒क्मꣳ रु॒क्म मुप॑ । \newline
5. उप॑ दधाति दधा॒ त्युपोप॑ दधाति । \newline
6. द॒धा॒ति॒ ब्रह्म॑मुखा॒ ब्रह्म॑मुखा दधाति दधाति॒ ब्रह्म॑मुखाः । \newline
7. ब्रह्म॑मुखा॒ वै वै ब्रह्म॑मुखा॒ ब्रह्म॑मुखा॒ वै । \newline
8. ब्रह्म॑मुखा॒ इति॒ ब्रह्म॑ - मु॒खाः॒ । \newline
9. वै प्र॒जाप॑तिः प्र॒जाप॑ति॒र् वै वै प्र॒जाप॑तिः । \newline
10. प्र॒जाप॑तिः प्र॒जाः प्र॒जाः प्र॒जाप॑तिः प्र॒जाप॑तिः प्र॒जाः । \newline
11. प्र॒जाप॑ति॒रिति॑ प्र॒जा - प॒तिः॒ । \newline
12. प्र॒जा अ॑सृजता सृजत प्र॒जाः प्र॒जा अ॑सृजत । \newline
13. प्र॒जा इति॑ प्र - जाः । \newline
14. अ॒सृ॒ज॒त॒ ब्रह्म॑मुखा॒ ब्रह्म॑मुखा असृजता सृजत॒ ब्रह्म॑मुखाः । \newline
15. ब्रह्म॑मुखा ए॒वैव ब्रह्म॑मुखा॒ ब्रह्म॑मुखा ए॒व । \newline
16. ब्रह्म॑मुखा॒ इति॒ ब्रह्म॑ - मु॒खाः॒ । \newline
17. ए॒व तत् तदे॒ वैव तत् । \newline
18. तत् प्र॒जाः प्र॒जा स्तत् तत् प्र॒जाः । \newline
19. प्र॒जा यज॑मानो॒ यज॑मानः प्र॒जाः प्र॒जा यज॑मानः । \newline
20. प्र॒जा इति॑ प्र - जाः । \newline
21. यज॑मानः सृजते सृजते॒ यज॑मानो॒ यज॑मानः सृजते । \newline
22. सृ॒ज॒ते॒ ब्रह्म॒ ब्रह्म॑ सृजते सृजते॒ ब्रह्म॑ । \newline
23. ब्रह्म॑ जज्ञा॒नम् ज॑ज्ञा॒नम् ब्रह्म॒ ब्रह्म॑ जज्ञा॒नम् । \newline
24. ज॒ज्ञा॒न मितीति॑ जज्ञा॒नम् ज॑ज्ञा॒न मिति॑ । \newline
25. इत्या॑ हा॒हेती त्या॑ह । \newline
26. आ॒ह॒ तस्मा॒त् तस्मा॑ दाहाह॒ तस्मा᳚त् । \newline
27. तस्मा᳚द् ब्राह्म॒णो ब्रा᳚ह्म॒ण स्तस्मा॒त् तस्मा᳚द् ब्राह्म॒णः । \newline
28. ब्रा॒ह्म॒णो मुख्यो॒ मुख्यो᳚ ब्राह्म॒णो ब्रा᳚ह्म॒णो मुख्यः॑ । \newline
29. मुख्यो॒ मुख्यः॑ । \newline
30. मुख्यो॑ भवति भवति॒ मुख्यो॒ मुख्यो॑ भवति । \newline
31. भ॒व॒ति॒ यो यो भ॑वति भवति॒ यः । \newline
32. य ए॒व मे॒वं ॅयो य ए॒वम् । \newline
33. ए॒वं ॅवेद॒ वेदै॒व मे॒वं ॅवेद॑ । \newline
34. वेद॑ ब्रह्मवा॒दिनो᳚ ब्रह्मवा॒दिनो॒ वेद॒ वेद॑ ब्रह्मवा॒दिनः॑ । \newline
35. ब्र॒ह्म॒वा॒दिनो॑ वदन्ति वदन्ति ब्रह्मवा॒दिनो᳚ ब्रह्मवा॒दिनो॑ वदन्ति । \newline
36. ब्र॒ह्म॒वा॒दिन॒ इति॑ ब्रह्म - वा॒दिनः॑ । \newline
37. व॒द॒न्ति॒ न न व॑दन्ति वदन्ति॒ न । \newline
38. न पृ॑थि॒व्याम् पृ॑थि॒व्यान् न न पृ॑थि॒व्याम् । \newline
39. पृ॒थि॒व्याम् न न पृ॑थि॒व्याम् पृ॑थि॒व्याम् न । \newline
40. नान्तरि॑क्षे॒ ऽन्तरि॑क्षे॒ न नान्तरि॑क्षे । \newline
41. अ॒न्तरि॑क्षे॒ न नान्तरि॑क्षे॒ ऽन्तरि॑क्षे॒ न । \newline
42. न दि॒वि दि॒वि न न दि॒वि । \newline
43. दि॒व्य॑ग्नि र॒ग्निर् दि॒वि दि॒व्य॑ग्निः । \newline
44. अ॒ग्नि श्चे॑त॒व्य॑ श्चेत॒व्यो᳚ ऽग्नि र॒ग्नि श्चे॑त॒व्यः॑ । \newline
45. चे॒त॒व्य॑ इतीति॑ चेत॒व्य॑ श्चेत॒व्य॑ इति॑ । \newline
46. इति॒ यद् यदितीति॒ यत् । \newline
47. यत् पृ॑थि॒व्याम् पृ॑थि॒व्यां ॅयद् यत् पृ॑थि॒व्याम् । \newline
48. पृ॒थि॒व्याम् चि॑न्वी॒त चि॑न्वी॒त पृ॑थि॒व्याम् पृ॑थि॒व्याम् चि॑न्वी॒त । \newline
49. चि॒न्वी॒त पृ॑थि॒वीम् पृ॑थि॒वीम् चि॑न्वी॒त चि॑न्वी॒त पृ॑थि॒वीम् । \newline
50. पृ॒थि॒वीꣳ शु॒चा शु॒चा पृ॑थि॒वीम् पृ॑थि॒वीꣳ शु॒चा । \newline
51. शु॒चा ऽर्प॑ये दर्पये च्छु॒चा शु॒चा ऽर्प॑येत् । \newline
52. अ॒र्प॒ये॒न् न नार्प॑ये दर्पये॒न् न । \newline
53. नौष॑धय॒ ओष॑धयो॒ न नौष॑धयः । \newline
54. ओष॑धयो॒ न नौष॑धय॒ ओष॑धयो॒ न । \newline
55. न वन॒स्पत॑यो॒ वन॒स्पत॑यो॒ न न वन॒स्पत॑यः । \newline
56. वन॒स्पत॑यः॒ प्र प्र वण॒स्पत॑यो॒ वन॒स्पत॑यः॒ प्र । \newline

\textbf{Ghana Paata } \newline

1. ब्रह्म॑ जज्ञा॒नम् ज॑ज्ञा॒नम् ब्रह्म॒ ब्रह्म॑ जज्ञा॒न मितीति॑ जज्ञा॒नम् ब्रह्म॒ ब्रह्म॑ जज्ञा॒न मिति॑ । \newline
2. ज॒ज्ञा॒न मितीति॑ जज्ञा॒नम् ज॑ज्ञा॒न मिति॑ रु॒क्मꣳ रु॒क्म मिति॑ जज्ञा॒नम् ज॑ज्ञा॒न मिति॑ रु॒क्मम् । \newline
3. इति॑ रु॒क्मꣳ रु॒क्म मितीति॑ रु॒क्म मुपोप॑ रु॒क्म मितीति॑ रु॒क्म मुप॑ । \newline
4. रु॒क्म मुपोप॑ रु॒क्मꣳ रु॒क्म मुप॑ दधाति दधा॒ त्युप॑ रु॒क्मꣳ रु॒क्म मुप॑ दधाति । \newline
5. उप॑ दधाति दधा॒ त्युपोप॑ दधाति॒ ब्रह्म॑मुखा॒ ब्रह्म॑मुखा दधा॒ त्युपोप॑ दधाति॒ ब्रह्म॑मुखाः । \newline
6. द॒धा॒ति॒ ब्रह्म॑मुखा॒ ब्रह्म॑मुखा दधाति दधाति॒ ब्रह्म॑मुखा॒ वै वै ब्रह्म॑मुखा दधाति दधाति॒ ब्रह्म॑मुखा॒ वै । \newline
7. ब्रह्म॑मुखा॒ वै वै ब्रह्म॑मुखा॒ ब्रह्म॑मुखा॒ वै प्र॒जाप॑तिः प्र॒जाप॑ति॒र् वै ब्रह्म॑मुखा॒ ब्रह्म॑मुखा॒ वै प्र॒जाप॑तिः । \newline
8. ब्रह्म॑मुखा॒ इति॒ ब्रह्म॑ - मु॒खाः॒ । \newline
9. वै प्र॒जाप॑तिः प्र॒जाप॑ति॒र् वै वै प्र॒जाप॑तिः प्र॒जाः प्र॒जाः प्र॒जाप॑ति॒र् वै वै प्र॒जाप॑तिः प्र॒जाः । \newline
10. प्र॒जाप॑तिः प्र॒जाः प्र॒जाः प्र॒जाप॑तिः प्र॒जाप॑तिः प्र॒जा अ॑सृजता सृजत प्र॒जाः प्र॒जाप॑तिः प्र॒जाप॑तिः प्र॒जा अ॑सृजत । \newline
11. प्र॒जाप॑ति॒रिति॑ प्र॒जा - प॒तिः॒ । \newline
12. प्र॒जा अ॑सृजता सृजत प्र॒जाः प्र॒जा अ॑सृजत॒ ब्रह्म॑मुखा॒ ब्रह्म॑मुखा असृजत प्र॒जाः प्र॒जा अ॑सृजत॒ ब्रह्म॑मुखाः । \newline
13. प्र॒जा इति॑ प्र - जाः । \newline
14. अ॒सृ॒ज॒त॒ ब्रह्म॑मुखा॒ ब्रह्म॑मुखा असृजता सृजत॒ ब्रह्म॑मुखा ए॒वैव ब्रह्म॑मुखा असृजता सृजत॒ ब्रह्म॑मुखा ए॒व । \newline
15. ब्रह्म॑मुखा ए॒वैव ब्रह्म॑मुखा॒ ब्रह्म॑मुखा ए॒व तत् तदे॒व ब्रह्म॑मुखा॒ ब्रह्म॑मुखा ए॒व तत् । \newline
16. ब्रह्म॑मुखा॒ इति॒ ब्रह्म॑ - मु॒खाः॒ । \newline
17. ए॒व तत् तदे॒वैव तत् प्र॒जाः प्र॒जा स्तदे॒वैव तत् प्र॒जाः । \newline
18. तत् प्र॒जाः प्र॒जा स्तत् तत् प्र॒जा यज॑मानो॒ यज॑मानः प्र॒जा स्तत् तत् प्र॒जा यज॑मानः । \newline
19. प्र॒जा यज॑मानो॒ यज॑मानः प्र॒जाः प्र॒जा यज॑मानः सृजते सृजते॒ यज॑मानः प्र॒जाः प्र॒जा यज॑मानः सृजते । \newline
20. प्र॒जा इति॑ प्र - जाः । \newline
21. यज॑मानः सृजते सृजते॒ यज॑मानो॒ यज॑मानः सृजते॒ ब्रह्म॒ ब्रह्म॑ सृजते॒ यज॑मानो॒ यज॑मानः सृजते॒ ब्रह्म॑ । \newline
22. सृ॒ज॒ते॒ ब्रह्म॒ ब्रह्म॑ सृजते सृजते॒ ब्रह्म॑ जज्ञा॒नम् ज॑ज्ञा॒नम् ब्रह्म॑ सृजते सृजते॒ ब्रह्म॑ जज्ञा॒नम् । \newline
23. ब्रह्म॑ जज्ञा॒नम् ज॑ज्ञा॒नम् ब्रह्म॒ ब्रह्म॑ जज्ञा॒न मितीति॑ जज्ञा॒नम् ब्रह्म॒ ब्रह्म॑ जज्ञा॒न मिति॑ । \newline
24. ज॒ज्ञा॒न मितीति॑ जज्ञा॒नम् ज॑ज्ञा॒न मित्या॑ हा॒हेति॑ जज्ञा॒नम् ज॑ज्ञा॒न मित्या॑ह । \newline
25. इत्या॑हा॒हे तीत्या॑ह॒ तस्मा॒त् तस्मा॑ दा॒हे तीत्या॑ह॒ तस्मा᳚त् । \newline
26. आ॒ह॒ तस्मा॒त् तस्मा॑ दाहाह॒ तस्मा᳚द् ब्राह्म॒णो ब्रा᳚ह्म॒ण स्तस्मा॑ दाहाह॒ तस्मा᳚द् ब्राह्म॒णः । \newline
27. तस्मा᳚द् ब्राह्म॒णो ब्रा᳚ह्म॒ण स्तस्मा॒त् तस्मा᳚द् ब्राह्म॒णो मुख्यो॒ मुख्यो᳚ ब्राह्म॒ण स्तस्मा॒त् तस्मा᳚द् ब्राह्म॒णो मुख्यः॑ । \newline
28. ब्रा॒ह्म॒णो मुख्यो॒ मुख्यो᳚ ब्राह्म॒णो ब्रा᳚ह्म॒णो मुख्यः॑ । \newline
29. मुख्यो॒ मुख्यः॑ । \newline
30. मुख्यो॑ भवति भवति॒ मुख्यो॒ मुख्यो॑ भवति॒ यो यो भ॑वति॒ मुख्यो॒ मुख्यो॑ भवति॒ यः । \newline
31. भ॒व॒ति॒ यो यो भ॑वति भवति॒ य ए॒व मे॒वं ॅयो भ॑वति भवति॒ य ए॒वम् । \newline
32. य ए॒व मे॒वं ॅयो य ए॒वं ॅवेद॒ वेदै॒वं ॅयो य ए॒वं ॅवेद॑ । \newline
33. ए॒वं ॅवेद॒ वेदै॒व मे॒वं ॅवेद॑ ब्रह्मवा॒दिनो᳚ ब्रह्मवा॒दिनो॒ वेदै॒व मे॒वं ॅवेद॑ ब्रह्मवा॒दिनः॑ । \newline
34. वेद॑ ब्रह्मवा॒दिनो᳚ ब्रह्मवा॒दिनो॒ वेद॒ वेद॑ ब्रह्मवा॒दिनो॑ वदन्ति वदन्ति ब्रह्मवा॒दिनो॒ वेद॒ वेद॑ ब्रह्मवा॒दिनो॑ वदन्ति । \newline
35. ब्र॒ह्म॒वा॒दिनो॑ वदन्ति वदन्ति ब्रह्मवा॒दिनो᳚ ब्रह्मवा॒दिनो॑ वदन्ति॒ न न व॑दन्ति ब्रह्मवा॒दिनो᳚ ब्रह्मवा॒दिनो॑ वदन्ति॒ न । \newline
36. ब्र॒ह्म॒वा॒दिन॒ इति॑ ब्रह्म - वा॒दिनः॑ । \newline
37. व॒द॒न्ति॒ न न व॑दन्ति वदन्ति॒ न पृ॑थि॒व्याम् पृ॑थि॒व्याम् न व॑दन्ति वदन्ति॒ न पृ॑थि॒व्याम् । \newline
38. न पृ॑थि॒व्याम् पृ॑थि॒व्याम् न न पृ॑थि॒व्याम् न न पृ॑थि॒व्याम् न न पृ॑थि॒व्याम् न । \newline
39. पृ॒थि॒व्याम् न न पृ॑थि॒व्याम् पृ॑थि॒व्याम् नान्तरि॑क्षे॒ ऽन्तरि॑क्षे॒ न पृ॑थि॒व्याम् पृ॑थि॒व्याम् नान्तरि॑क्षे । \newline
40. नान्तरि॑क्षे॒ ऽन्तरि॑क्षे॒ न नान्तरि॑क्षे॒ न नान्तरि॑क्षे॒ न नान्तरि॑क्षे॒ न । \newline
41. अ॒न्तरि॑क्षे॒ न नान्तरि॑क्षे॒ ऽन्तरि॑क्षे॒ न दि॒वि दि॒वि नान्तरि॑क्षे॒ ऽन्तरि॑क्षे॒ न दि॒वि । \newline
42. न दि॒वि दि॒वि न न दि॒व्य॑ग्नि र॒ग्निर् दि॒वि न न दि॒व्य॑ग्निः । \newline
43. दि॒व्य॑ग्नि र॒ग्निर् दि॒वि दि॒व्य॑ग्नि श्चे॑त॒व्य॑ श्चेत॒व्यो᳚ ऽग्निर् दि॒वि दि॒व्य॑ग्नि श्चे॑त॒व्यः॑ । \newline
44. अ॒ग्नि श्चे॑त॒व्य॑ श्चेत॒व्यो᳚ ऽग्नि र॒ग्नि श्चे॑त॒व्य॑ इतीति॑ चेत॒व्यो᳚ ऽग्नि र॒ग्नि श्चे॑त॒व्य॑ इति॑ । \newline
45. चे॒त॒व्य॑ इतीति॑ चेत॒व्य॑ श्चेत॒व्य॑ इति॒ यद् यदिति॑ चेत॒व्य॑ श्चेत॒व्य॑ इति॒ यत् । \newline
46. इति॒ यद् यदितीति॒ यत् पृ॑थि॒व्याम् पृ॑थि॒व्यां ॅयदितीति॒ यत् पृ॑थि॒व्याम् । \newline
47. यत् पृ॑थि॒व्याम् पृ॑थि॒व्यां ॅयद् यत् पृ॑थि॒व्याम् चि॑न्वी॒त चि॑न्वी॒त पृ॑थि॒व्यां ॅयद् यत् पृ॑थि॒व्याम् चि॑न्वी॒त । \newline
48. पृ॒थि॒व्याम् चि॑न्वी॒त चि॑न्वी॒त पृ॑थि॒व्याम् पृ॑थि॒व्याम् चि॑न्वी॒त पृ॑थि॒वीम् पृ॑थि॒वीम् चि॑न्वी॒त पृ॑थि॒व्याम् पृ॑थि॒व्याम् चि॑न्वी॒त पृ॑थि॒वीम् । \newline
49. चि॒न्वी॒त पृ॑थि॒वीम् पृ॑थि॒वीम् चि॑न्वी॒त चि॑न्वी॒त पृ॑थि॒वीꣳ शु॒चा शु॒चा पृ॑थि॒वीम् चि॑न्वी॒त चि॑न्वी॒त पृ॑थि॒वीꣳ शु॒चा । \newline
50. पृ॒थि॒वीꣳ शु॒चा शु॒चा पृ॑थि॒वीम् पृ॑थि॒वीꣳ शु॒चा ऽर्प॑ये दर्पये च्छु॒चा पृ॑थि॒वीम् पृ॑थि॒वीꣳ शु॒चा ऽर्प॑येत् । \newline
51. शु॒चा ऽर्प॑ये दर्पये च्छु॒चा शु॒चा ऽर्प॑ये॒न् न नार्प॑ये च्छु॒चा शु॒चा ऽर्प॑ये॒न् न । \newline
52. अ॒र्प॒ये॒न् न नार्प॑ये दर्पये॒न् नौष॑धय॒ ओष॑धयो॒ नार्प॑ये दर्पये॒न् नौष॑धयः । \newline
53. नौष॑धय॒ ओष॑धयो॒ न नौष॑धयो॒ न नौष॑धयो॒ न नौष॑धयो॒ न । \newline
54. ओष॑धयो॒ न नौष॑धय॒ ओष॑धयो॒ न वन॒स्पत॑यो॒ वन॒स्पत॑यो॒ नौष॑धय॒ ओष॑धयो॒ न वन॒स्पत॑यः । \newline
55. न वन॒स्पत॑यो॒ वन॒स्पत॑यो॒ न न वन॒स्पत॑यः॒ प्र प्र वण॒स्पत॑यो॒ न न वन॒स्पत॑यः॒ प्र । \newline
56. वन॒स्पत॑यः॒ प्र प्र वण॒स्पत॑यो॒ वन॒स्पत॑यः॒ प्र जा॑येरन् जायेर॒न् प्र वण॒स्पत॑यो॒ वन॒स्पत॑यः॒ प्र जा॑येरन्न् । \newline
\pagebreak
\markright{ TS 5.2.7.2  \hfill https://www.vedavms.in \hfill}

\section{ TS 5.2.7.2 }

\textbf{TS 5.2.7.2 } \newline
\textbf{Samhita Paata} \newline

प्र जा॑येर॒न॒. यद॒न्तरि॑क्षे चिन्वी॒तान्तरि॑क्षꣳ शु॒चाऽर्प॑ये॒न्न वयाꣳ॑सि॒ प्र जा॑येर॒न्॒. यद्-दि॒वि चि॑न्वी॒त दिवꣳ॑ शु॒चाऽर्प॑ये॒न्न प॒र्जन्यो॑ वर्.षेद्रु॒क्ममुप॑ दधात्य॒मृतं॒ ॅवै हिर॑ण्यम॒मृत॑ ए॒वाग्निं चि॑नुते॒ प्रजा᳚त्यै हिर॒ण्मयं॒ पुरु॑ष॒मुप॑ दधाति यजमानलो॒कस्य॒ विधृ॑त्यै॒ यदिष्ट॑काया॒ आतृ॑ण्णमनूपद॒द्ध्यात् प॑शू॒नां च॒ यज॑मानस्य च प्रा॒णमपि॑ दद्ध्याद् दक्षिण॒तः - [  ] \newline

\textbf{Pada Paata} \newline

प्रेति॑ । जा॒ये॒र॒न्न् । यत् । अ॒न्तरि॑क्षे । चि॒न्वी॒त । अ॒न्तरि॑क्षम् । शु॒चा । अ॒र्प॒ये॒त् । न । वयाꣳ॑सि । प्रेति॑ । जा॒ये॒र॒न्न् । यत् । दि॒वि । चि॒न्वी॒त । दिव᳚म् । शु॒चा । अ॒र्प॒ये॒त् । न । प॒र्जन्यः॑ । व॒र्॒.षे॒त् । रु॒क्मम् । उपेति॑ । द॒धा॒ति॒ । अ॒मृत᳚म् । वै । हिर॑ण्यम् । अ॒मृते᳚ । ए॒व । अ॒ग्निम् । चि॒नु॒ते॒ । प्रजा᳚त्या॒ इति॒ प्र - जा॒त्यै॒ । हि॒र॒ण्मय᳚म् । पुरु॑षम् । उपेति॑ । द॒धा॒ति॒ । य॒ज॒मा॒न॒लो॒कस्येति॑ यजमान - लो॒कस्य॑ । विधृ॑त्या॒ इति॒ वि-धृ॒त्यै॒ । यत् । इष्ट॑कायाः । आतृ॑ण्ण॒मित्या-तृ॒ण्ण॒म् । अ॒नू॒प॒द॒द्ध्यादित्य॑नु - उ॒प॒द॒द्ध्यात् । प॒शू॒नाम् । च॒ । यज॑मानस्य । च॒ । प्रा॒णमिति॑ प्र - अ॒नम् । अपीति॑ । द॒द्ध्या॒त् । द॒क्षि॒ण॒तः ।  \newline


\textbf{Krama Paata} \newline

प्र जा॑येरन्न् । जा॒ये॒र॒न्॒. यत् । यद॒न्तरि॑क्षे । अ॒न्तरि॑क्षे चिन्वी॒त । चि॒न्वी॒तान्तरि॑क्षम् । अ॒न्तरि॑क्षꣳ शु॒चा । शु॒चाऽर्प॑येत् । अ॒र्प॒ये॒न् न । न वयाꣳ॑सि । वयाꣳ॑सि॒ प्र । प्र जा॑येरन्न् । जा॒ये॒र॒न्॒. यत् । यद् दि॒वि । दि॒वि चि॑न्वी॒त । चि॒न्वी॒त दिव᳚म् । दिवꣳ॑ शु॒चा । शु॒चाऽर्प॑येत् । अ॒र्प॒ये॒न् न । न प॒र्जन्यः॑ । प॒र्जन्यो॑ वर्.षेत् । व॒र्॒.षे॒द् रु॒क्मम् । रु॒क्ममुप॑ । उप॑ दधाति । द॒धा॒त्य॒मृत᳚म् । अ॒मृत॒म् ॅवै । वै हिर॑ण्यम् । हिर॑ण्यम॒मृते᳚ । अ॒मृत॑ ए॒व । ए॒वाग्निम् । अ॒ग्निम् चि॑नुते । चि॒नु॒ते॒ प्रजा᳚त्यै । प्रजा᳚त्यै हिर॒ण्मय᳚म् । प्रजा᳚त्या॒ इति॒ प्र - जा॒त्यै॒ । हि॒र॒ण्मय॒म् पुरु॑षम् । पुरु॑ष॒मुप॑ । उप॑ दधाति । द॒धा॒ति॒ य॒ज॒मा॒न॒लो॒कस्य॑ । य॒ज॒मा॒न॒लो॒कस्य॒ विधृ॑त्यै । य॒ज॒मा॒न॒लो॒कस्येति॑ यजमान - लो॒कस्य॑ । विधृ॑त्यै॒ यत् । विधृ॑त्या॒ इति॒ वि - धृ॒त्यै॒ । यदिष्ट॑कायाः । इष्ट॑काया॒ आतृ॑ण्णम् । आतृ॑ण्णमनूपद॒द्ध्यात् । आतृ॑ण्ण॒मित्या - तृ॒ण्ण॒म् । अ॒नू॒प॒द॒द्ध्यात् प॑शू॒नाम् । अ॒नू॒प॒द॒द्ध्यादित्य॑नु - उ॒प॒द॒द्ध्यात् । प॒शू॒नाम् च॑ । च॒ यज॑मानस्य । यज॑मानस्य च । च॒ प्रा॒णम् । प्रा॒णमपि॑ । प्रा॒णमिति॑ प्र - अ॒नम् । अपि॑ दद्ध्यात् । द॒द्ध्या॒द् द॒क्षि॒ण॒तः । द॒क्षि॒ण॒तः प्राञ्च᳚म् \newline

\textbf{Jatai Paata} \newline

1. प्र जा॑येरन् जायेर॒न् प्र प्र जा॑येरन्न् । \newline
2. जा॒ये॒र॒न्॒. यद् यज् जा॑येरन् जायेर॒न्॒. यत् । \newline
3. यद॒न्तरि॑क्षे॒ ऽन्तरि॑क्षे॒ यद् यद॒न्तरि॑क्षे । \newline
4. अ॒न्तरि॑क्षे चिन्वी॒त चि॑न्वी॒ता न्तरि॑क्षे॒ ऽन्तरि॑क्षे चिन्वी॒त । \newline
5. चि॒न्वी॒ता न्तरि॑क्ष म॒न्तरि॑क्षम् चिन्वी॒त चि॑न्वी॒ता न्तरि॑क्षम् । \newline
6. अ॒न्तरि॑क्षꣳ शु॒चा शु॒चा ऽन्तरि॑क्ष म॒न्तरि॑क्षꣳ शु॒चा । \newline
7. शु॒चा ऽर्प॑ये दर्पये च्छु॒चा शु॒चा ऽर्प॑येत् । \newline
8. अ॒र्प॒ये॒न् न नार्प॑ये दर्पये॒न् न । \newline
9. न वयाꣳ॑सि॒ वयाꣳ॑सि॒ न न वयाꣳ॑सि । \newline
10. वयाꣳ॑सि॒ प्र प्र वयाꣳ॑सि॒ वयाꣳ॑सि॒ प्र । \newline
11. प्र जा॑येरन् जायेर॒न् प्र प्र जा॑येरन्न् । \newline
12. जा॒ये॒र॒न्॒. यद् यज् जा॑येरन् जायेर॒न्॒. यत् । \newline
13. यद् दि॒वि दि॒वि यद् यद् दि॒वि । \newline
14. दि॒वि चि॑न्वी॒त चि॑न्वी॒त दि॒वि दि॒वि चि॑न्वी॒त । \newline
15. चि॒न्वी॒त दिव॒म् दिव॑म् चिन्वी॒त चि॑न्वी॒त दिव᳚म् । \newline
16. दिवꣳ॑ शु॒चा शु॒चा दिव॒म् दिवꣳ॑ शु॒चा । \newline
17. शु॒चा ऽर्प॑ये दर्पये च्छु॒चा शु॒चा ऽर्प॑येत् । \newline
18. अ॒र्प॒ये॒न् न नार्प॑ये दर्पये॒न् न । \newline
19. न प॒र्जन्यः॑ प॒र्जन्यो॒ न न प॒र्जन्यः॑ । \newline
20. प॒र्जन्यो॑ वर्.षेद् वर्.षेत् प॒र्जन्यः॑ प॒र्जन्यो॑ वर्.षेत् । \newline
21. व॒र्॒.षे॒द् रु॒क्मꣳ रु॒क्मं ॅव॑र्.षेद् वर्.षेद् रु॒क्मम् । \newline
22. रु॒क्म मुपोप॑ रु॒क्मꣳ रु॒क्म मुप॑ । \newline
23. उप॑ दधाति दधा॒ त्युपोप॑ दधाति । \newline
24. द॒धा॒ त्य॒मृत॑ म॒मृत॑म् दधाति दधा त्य॒मृत᳚म् । \newline
25. अ॒मृतं॒ ॅवै वा अ॒मृत॑ म॒मृतं॒ ॅवै । \newline
26. वै हिर॑ण्यꣳ॒॒ हिर॑ण्यं॒ ॅवै वै हिर॑ण्यम् । \newline
27. हिर॑ण्य म॒मृते॒ ऽमृते॒ हिर॑ण्यꣳ॒॒ हिर॑ण्य म॒मृते᳚ । \newline
28. अ॒मृत॑ ए॒वैवा मृते॒ ऽमृत॑ ए॒व । \newline
29. ए॒वाग्नि म॒ग्नि मे॒वै वाग्निम् । \newline
30. अ॒ग्निम् चि॑नुते चिनुते॒ ऽग्नि म॒ग्निम् चि॑नुते । \newline
31. चि॒नु॒ते॒ प्रजा᳚त्यै॒ प्रजा᳚त्यै चिनुते चिनुते॒ प्रजा᳚त्यै । \newline
32. प्रजा᳚त्यै हिर॒ण्मयꣳ॑ हिर॒ण्मय॒म् प्रजा᳚त्यै॒ प्रजा᳚त्यै हिर॒ण्मय᳚म् । \newline
33. प्रजा᳚त्या॒ इति॒ प्र - जा॒त्यै॒ । \newline
34. हि॒र॒ण्मय॒म् पुरु॑ष॒म् पुरु॑षꣳ हिर॒ण्मयꣳ॑ हिर॒ण्मय॒म् पुरु॑षम् । \newline
35. पुरु॑ष॒ मुपोप॒ पुरु॑ष॒म् पुरु॑ष॒ मुप॑ । \newline
36. उप॑ दधाति दधा॒ त्युपोप॑ दधाति । \newline
37. द॒धा॒ति॒ य॒ज॒मा॒न॒लो॒कस्य॑ यजमानलो॒कस्य॑ दधाति दधाति यजमानलो॒कस्य॑ । \newline
38. य॒ज॒मा॒न॒लो॒कस्य॒ विधृ॑त्यै॒ विधृ॑त्यै यजमानलो॒कस्य॑ यजमानलो॒कस्य॒ विधृ॑त्यै । \newline
39. य॒ज॒मा॒न॒लो॒कस्येति॑ यजमान - लो॒कस्य॑ । \newline
40. विधृ॑त्यै॒ यद् यद् विधृ॑त्यै॒ विधृ॑त्यै॒ यत् । \newline
41. विधृ॑त्या॒ इति॒ वि - धृ॒त्यै॒ । \newline
42. यदिष्ट॑काया॒ इष्ट॑काया॒ यद् यदिष्ट॑कायाः । \newline
43. इष्ट॑काया॒ आतृ॑ण्ण॒ मातृ॑ण्ण॒ मिष्ट॑काया॒ इष्ट॑काया॒ आतृ॑ण्णम् । \newline
44. आतृ॑ण्ण मनूपद॒द्ध्या द॑नूपद॒द्ध्या दातृ॑ण्ण॒ मातृ॑ण्ण मनूपद॒द्ध्यात् । \newline
45. आतृ॑ण्ण॒मित्या - तृ॒ण्ण॒म् । \newline
46. अ॒नू॒प॒द॒द्ध्यात् प॑शू॒नाम् प॑शू॒ना म॑नूपद॒द्ध्या द॑नूपद॒द्ध्यात् प॑शू॒नाम् । \newline
47. अ॒नू॒प॒द॒द्ध्यादित्य॑नु - उ॒प॒द॒द्ध्यात् । \newline
48. प॒शू॒नाम् च॑ च पशू॒नाम् प॑शू॒नाम् च॑ । \newline
49. च॒ यज॑मानस्य॒ यज॑मानस्य च च॒ यज॑मानस्य । \newline
50. यज॑मानस्य च च॒ यज॑मानस्य॒ यज॑मानस्य च । \newline
51. च॒ प्रा॒णम् प्रा॒णम् च॑ च प्रा॒णम् । \newline
52. प्रा॒ण मप्यपि॑ प्रा॒णम् प्रा॒ण मपि॑ । \newline
53. प्रा॒णमिति॑ प्र - अ॒नम् । \newline
54. अपि॑ दद्ध्याद् दद्ध्या॒ दप्यपि॑ दद्ध्यात् । \newline
55. द॒द्ध्या॒द् द॒क्षि॒ण॒तो द॑क्षिण॒तो द॑द्ध्याद् दद्ध्याद् दक्षिण॒तः । \newline
56. द॒क्षि॒ण॒तः प्राञ्च॒म् प्राञ्च॑म् दक्षिण॒तो द॑क्षिण॒तः प्राञ्च᳚म् । \newline

\textbf{Ghana Paata } \newline

1. प्र जा॑येरन् जायेर॒न् प्र प्र जा॑येर॒न्॒. यद् यज् जा॑येर॒न् प्र प्र जा॑येर॒न्॒. यत् । \newline
2. जा॒ये॒र॒न्॒. यद् यज् जा॑येरन् जायेर॒न्॒. यद॒न्तरि॑क्षे॒ ऽन्तरि॑क्षे॒ यज् जा॑येरन् जायेर॒न्॒. यद॒न्तरि॑क्षे । \newline
3. यद॒न्तरि॑क्षे॒ ऽन्तरि॑क्षे॒ यद् यद॒न्तरि॑क्षे चिन्वी॒त चि॑न्वी॒ता न्तरि॑क्षे॒ यद् यद॒न्तरि॑क्षे चिन्वी॒त । \newline
4. अ॒न्तरि॑क्षे चिन्वी॒त चि॑न्वी॒ता न्तरि॑क्षे॒ ऽन्तरि॑क्षे चिन्वी॒ता न्तरि॑क्ष म॒न्तरि॑क्षम् चिन्वी॒ता न्तरि॑क्षे॒ ऽन्तरि॑क्षे चिन्वी॒ता न्तरि॑क्षम् । \newline
5. चि॒न्वी॒ता न्तरि॑क्ष म॒न्तरि॑क्षम् चिन्वी॒त चि॑न्वी॒ता न्तरि॑क्षꣳ शु॒चा शु॒चा ऽन्तरि॑क्षम् चिन्वी॒त चि॑न्वी॒ता न्तरि॑क्षꣳ शु॒चा । \newline
6. अ॒न्तरि॑क्षꣳ शु॒चा शु॒चा ऽन्तरि॑क्ष म॒न्तरि॑क्षꣳ शु॒चा ऽर्प॑ये दर्पये च्छु॒चा ऽन्तरि॑क्ष म॒न्तरि॑क्षꣳ शु॒चा ऽर्प॑येत् । \newline
7. शु॒चा ऽर्प॑ये दर्पये च्छु॒चा शु॒चा ऽर्प॑ये॒न् न नार्प॑ये च्छु॒चा शु॒चा ऽर्प॑ये॒न् न । \newline
8. अ॒र्प॒ये॒न् न नार्प॑ये दर्पये॒न् न वयाꣳ॑सि॒ वयाꣳ॑सि॒ नार्प॑ये दर्पये॒न् न वयाꣳ॑सि । \newline
9. न वयाꣳ॑सि॒ वयाꣳ॑सि॒ न न वयाꣳ॑सि॒ प्र प्र वयाꣳ॑सि॒ न न वयाꣳ॑सि॒ प्र । \newline
10. वयाꣳ॑सि॒ प्र प्र वयाꣳ॑सि॒ वयाꣳ॑सि॒ प्र जा॑येरन् जायेर॒न् प्र वयाꣳ॑सि॒ वयाꣳ॑सि॒ प्र जा॑येरन्न् । \newline
11. प्र जा॑येरन् जायेर॒न् प्र प्र जा॑येर॒न्॒. यद् यज् जा॑येर॒न् प्र प्र जा॑येर॒न्॒. यत् । \newline
12. जा॒ये॒र॒न्॒. यद् यज् जा॑येरन् जायेर॒न्॒. यद् दि॒वि दि॒वि यज् जा॑येरन् जायेर॒न्॒. यद् दि॒वि । \newline
13. यद् दि॒वि दि॒वि यद् यद् दि॒वि चि॑न्वी॒त चि॑न्वी॒त दि॒वि यद् यद् दि॒वि चि॑न्वी॒त । \newline
14. दि॒वि चि॑न्वी॒त चि॑न्वी॒त दि॒वि दि॒वि चि॑न्वी॒त दिव॒म् दिव॑म् चिन्वी॒त दि॒वि दि॒वि चि॑न्वी॒त दिव᳚म् । \newline
15. चि॒न्वी॒त दिव॒म् दिव॑म् चिन्वी॒त चि॑न्वी॒त दिवꣳ॑ शु॒चा शु॒चा दिव॑म् चिन्वी॒त चि॑न्वी॒त दिवꣳ॑ शु॒चा । \newline
16. दिवꣳ॑ शु॒चा शु॒चा दिव॒म् दिवꣳ॑ शु॒चा ऽर्प॑ये दर्पये च्छु॒चा दिव॒म् दिवꣳ॑ शु॒चा ऽर्प॑येत् । \newline
17. शु॒चा ऽर्प॑ये दर्पये च्छु॒चा शु॒चा ऽर्प॑ये॒न् न नार्प॑ये च्छु॒चा शु॒चा ऽर्प॑ये॒न् न । \newline
18. अ॒र्प॒ये॒न् न नार्प॑ये दर्पये॒न् न प॒र्जन्यः॑ प॒र्जन्यो॒ नार्प॑ये दर्पये॒न् न प॒र्जन्यः॑ । \newline
19. न प॒र्जन्यः॑ प॒र्जन्यो॒ न न प॒र्जन्यो॑ वर्.षेद् वर्.षेत् प॒र्जन्यो॒ न न प॒र्जन्यो॑ वर्.षेत् । \newline
20. प॒र्जन्यो॑ वर्.षेद् वर्.षेत् प॒र्जन्यः॑ प॒र्जन्यो॑ वर्.षेद् रु॒क्मꣳ रु॒क्मं ॅव॑र्.षेत् प॒र्जन्यः॑ प॒र्जन्यो॑ वर्.षेद् रु॒क्मम् । \newline
21. व॒र्॒.षे॒द् रु॒क्मꣳ रु॒क्मं ॅव॑रे.षेद् वर्.षेद् रु॒क्म मुपोप॑ रु॒क्मं ॅव॑र्.षेद् वर्.षेद् रु॒क्म मुप॑ । \newline
22. रु॒क्म मुपोप॑ रु॒क्मꣳ रु॒क्म मुप॑ दधाति दधा॒ त्युप॑ रु॒क्मꣳ रु॒क्म मुप॑ दधाति । \newline
23. उप॑ दधाति दधा॒ त्युपोप॑ दधा त्य॒मृत॑ म॒मृत॑म् दधा॒ त्युपोप॑ दधा त्य॒मृत᳚म् । \newline
24. द॒धा॒ त्य॒मृत॑ म॒मृत॑म् दधाति दधा त्य॒मृतं॒ ॅवै वा अ॒मृत॑म् दधाति दधा त्य॒मृतं॒ ॅवै । \newline
25. अ॒मृतं॒ ॅवै वा अ॒मृत॑ म॒मृतं॒ ॅवै हिर॑ण्यꣳ॒॒ हिर॑ण्यं॒ ॅवा अ॒मृत॑ म॒मृतं॒ ॅवै हिर॑ण्यम् । \newline
26. वै हिर॑ण्यꣳ॒॒ हिर॑ण्यं॒ ॅवै वै हिर॑ण्य म॒मृते॒ ऽमृते॒ हिर॑ण्यं॒ ॅवै वै हिर॑ण्य म॒मृते᳚ । \newline
27. हिर॑ण्य म॒मृते॒ ऽमृते॒ हिर॑ण्यꣳ॒॒ हिर॑ण्य म॒मृत॑ ए॒वैवामृते॒ हिर॑ण्यꣳ॒॒ हिर॑ण्य म॒मृत॑ ए॒व । \newline
28. अ॒मृत॑ ए॒वैवामृते॒ ऽमृत॑ ए॒वाग्नि म॒ग्नि मे॒वामृते॒ ऽमृत॑ ए॒वाग्निम् । \newline
29. ए॒वाग्नि म॒ग्नि मे॒वैवाग्निम् चि॑नुते चिनुते॒ ऽग्नि मे॒वैवाग्निम् चि॑नुते । \newline
30. अ॒ग्निम् चि॑नुते चिनुते॒ ऽग्नि म॒ग्निम् चि॑नुते॒ प्रजा᳚त्यै॒ प्रजा᳚त्यै चिनुते॒ ऽग्नि म॒ग्निम् चि॑नुते॒ प्रजा᳚त्यै । \newline
31. चि॒नु॒ते॒ प्रजा᳚त्यै॒ प्रजा᳚त्यै चिनुते चिनुते॒ प्रजा᳚त्यै हिर॒ण्मयꣳ॑ हिर॒ण्मय॒म् प्रजा᳚त्यै चिनुते चिनुते॒ प्रजा᳚त्यै हिर॒ण्मय᳚म् । \newline
32. प्रजा᳚त्यै हिर॒ण्मयꣳ॑ हिर॒ण्मय॒म् प्रजा᳚त्यै॒ प्रजा᳚त्यै हिर॒ण्मय॒म् पुरु॑ष॒म् पुरु॑षꣳ हिर॒ण्मय॒म् प्रजा᳚त्यै॒ प्रजा᳚त्यै हिर॒ण्मय॒म् पुरु॑षम् । \newline
33. प्रजा᳚त्या॒ इति॒ प्र - जा॒त्यै॒ । \newline
34. हि॒र॒ण्मय॒म् पुरु॑ष॒म् पुरु॑षꣳ हिर॒ण्मयꣳ॑ हिर॒ण्मय॒म् पुरु॑ष॒ मुपोप॒ पुरु॑षꣳ हिर॒ण्मयꣳ॑ हिर॒ण्मय॒म् पुरु॑ष॒ मुप॑ । \newline
35. पुरु॑ष॒ मुपोप॒ पुरु॑ष॒म् पुरु॑ष॒ मुप॑ दधाति दधा॒ त्युप॒ पुरु॑ष॒म् पुरु॑ष॒ मुप॑ दधाति । \newline
36. उप॑ दधाति दधा॒ त्युपोप॑ दधाति यजमानलो॒कस्य॑ यजमानलो॒कस्य॑ दधा॒ त्युपोप॑ दधाति यजमानलो॒कस्य॑ । \newline
37. द॒धा॒ति॒ य॒ज॒मा॒न॒लो॒कस्य॑ यजमानलो॒कस्य॑ दधाति दधाति यजमानलो॒कस्य॒ विधृ॑त्यै॒ विधृ॑त्यै यजमानलो॒कस्य॑ दधाति दधाति यजमानलो॒कस्य॒ विधृ॑त्यै । \newline
38. य॒ज॒मा॒न॒लो॒कस्य॒ विधृ॑त्यै॒ विधृ॑त्यै यजमानलो॒कस्य॑ यजमानलो॒कस्य॒ विधृ॑त्यै॒ यद् यद् विधृ॑त्यै यजमानलो॒कस्य॑ यजमानलो॒कस्य॒ विधृ॑त्यै॒ यत् । \newline
39. य॒ज॒मा॒न॒लो॒कस्येति॑ यजमान - लो॒कस्य॑ । \newline
40. विधृ॑त्यै॒ यद् यद् विधृ॑त्यै॒ विधृ॑त्यै॒ यदिष्ट॑काया॒ इष्ट॑काया॒ यद् विधृ॑त्यै॒ विधृ॑त्यै॒ यदिष्ट॑कायाः । \newline
41. विधृ॑त्या॒ इति॒ वि - धृ॒त्यै॒ । \newline
42. यदिष्ट॑काया॒ इष्ट॑काया॒ यद् यदिष्ट॑काया॒ आतृ॑ण्ण॒ मातृ॑ण्ण॒ मिष्ट॑काया॒ यद् यदिष्ट॑काया॒ आतृ॑ण्णम् । \newline
43. इष्ट॑काया॒ आतृ॑ण्ण॒ मातृ॑ण्ण॒ मिष्ट॑काया॒ इष्ट॑काया॒ आतृ॑ण्ण मनूपद॒द्ध्या द॑नूपद॒द्ध्या दातृ॑ण्ण॒ मिष्ट॑काया॒ इष्ट॑काया॒ आतृ॑ण्ण मनूपद॒द्ध्यात् । \newline
44. आतृ॑ण्ण मनूपद॒द्ध्या द॑नूपद॒द्ध्या दातृ॑ण्ण॒ मातृ॑ण्ण मनूपद॒द्ध्यात् प॑शू॒नाम् प॑शू॒ना म॑नूपद॒द्ध्या दातृ॑ण्ण॒ मातृ॑ण्ण मनूपद॒द्ध्यात् प॑शू॒नाम् । \newline
45. आतृ॑ण्ण॒मित्या - तृ॒ण्ण॒म् । \newline
46. अ॒नू॒प॒द॒द्ध्यात् प॑शू॒नाम् प॑शू॒ना म॑नूपद॒द्ध्या द॑नूपद॒द्ध्यात् प॑शू॒नाम् च॑ च पशू॒ना म॑नूपद॒द्ध्या द॑नूपद॒द्ध्यात् प॑शू॒नाम् च॑ । \newline
47. अ॒नू॒प॒द॒द्ध्यादित्य॑नु - उ॒प॒द॒द्ध्यात् । \newline
48. प॒शू॒नाम् च॑ च पशू॒नाम् प॑शू॒नाम् च॒ यज॑मानस्य॒ यज॑मानस्य च पशू॒नाम् प॑शू॒नाम् च॒ यज॑मानस्य । \newline
49. च॒ यज॑मानस्य॒ यज॑मानस्य च च॒ यज॑मानस्य च च॒ यज॑मानस्य च च॒ यज॑मानस्य च । \newline
50. यज॑मानस्य च च॒ यज॑मानस्य॒ यज॑मानस्य च प्रा॒णम् प्रा॒णम् च॒ यज॑मानस्य॒ यज॑मानस्य च प्रा॒णम् । \newline
51. च॒ प्रा॒णम् प्रा॒णम् च॑ च प्रा॒ण मप्यपि॑ प्रा॒णम् च॑ च प्रा॒ण मपि॑ । \newline
52. प्रा॒ण मप्यपि॑ प्रा॒णम् प्रा॒ण मपि॑ दद्ध्याद् दद्ध्या॒दपि॑ प्रा॒णम् प्रा॒ण मपि॑ दद्ध्यात् । \newline
53. प्रा॒णमिति॑ प्र - अ॒नम् । \newline
54. अपि॑ दद्ध्याद् दद्ध्या॒ दप्यपि॑ दद्ध्याद् दक्षिण॒तो द॑क्षिण॒तो द॑द्ध्या॒ दप्यपि॑ दद्ध्याद् दक्षिण॒तः । \newline
55. द॒द्ध्या॒द् द॒क्षि॒ण॒तो द॑क्षिण॒तो द॑द्ध्याद् दद्ध्याद् दक्षिण॒तः प्राञ्च॒म् प्राञ्च॑म् दक्षिण॒तो द॑द्ध्याद् दद्ध्याद् दक्षिण॒तः प्राञ्च᳚म् । \newline
56. द॒क्षि॒ण॒तः प्राञ्च॒म् प्राञ्च॑म् दक्षिण॒तो द॑क्षिण॒तः प्राञ्च॒ मुपोप॒ प्राञ्च॑म् दक्षिण॒तो द॑क्षिण॒तः प्राञ्च॒ मुप॑ । \newline
\pagebreak
\markright{ TS 5.2.7.3  \hfill https://www.vedavms.in \hfill}

\section{ TS 5.2.7.3 }

\textbf{TS 5.2.7.3 } \newline
\textbf{Samhita Paata} \newline

प्राञ्च॒मुप॑ दधाति दा॒धार॑ यजमानलो॒कं न प॑शू॒नां च॒ यज॑मानस्य च प्रा॒णमपि॑ दधा॒त्यथो॒ खल्विष्ट॑काया॒ आतृ॑ण्ण॒मनूप॑ दधाति प्रा॒णाना॒मुथ्सृ॑ष्ट्यै द्र॒फ्सश्च॑स्क॒न्देत्य॒भि मृ॑शति॒ होत्रा᳚स्वे॒वैनं॒ प्रति॑ष्ठापयति॒ स्रुचा॒वुप॑ दधा॒त्याज्य॑स्य पू॒र्णां का᳚र्ष्मर्य॒मयीं᳚ द॒द्ध्नः पू॒र्णा-मौदु॑बंरीमि॒यं ॅवै का᳚र्ष्मर्य॒मय्य॒सावौ-दु॑बंरी॒मे ए॒वोप॑ धत्ते - [  ] \newline

\textbf{Pada Paata} \newline

प्राञ्च᳚म् । उपेति॑ । द॒धा॒ति॒ । दा॒धार॑ । य॒ज॒मा॒न॒लो॒कमिति॑ यजमान - लो॒कम् । न । प॒शू॒नाम् । च॒ । यज॑मानस्य । च॒ । प्रा॒णमिति॑ प्र - अ॒नम् । अपीति॑ । द॒धा॒ति॒ । अथो॒ इति॑ । खलु॑ । इष्ट॑कायाः । आतृ॑ण्ण॒मित्या - तृ॒ण्ण॒म् । अनु॑ । उपेति॑ । द॒धा॒ति॒ । प्रा॒णाना॒मिति॑ प्र - अ॒नाना᳚म् । उथ्सृ॑ष्ट्या॒ इत्युत् - सृ॒ष्ट्यै॒ । द्र॒फ्सः । च॒स्क॒न्द॒ । इति॑ । अ॒भीति॑ । मृ॒श॒ति॒ । होत्रा॑सु । ए॒व । ए॒न॒म् । प्रतीति॑ । स्था॒प॒य॒ति॒ । स्रुचौ᳚ । उपेति॑ । द॒धा॒ति॒ । आज्य॑स्य । पू॒र्णाम् । का॒र्ष्म॒र्य॒मयी॒मिति॑ कार्ष्मर्य - मयी᳚म् । द॒द्ध्नः । पू॒र्णाम् । औदु॑बंरीम् । इ॒यम् । वै । का॒र्ष्म॒र्य॒मयीति॑ कार्ष्मर्य - मयी᳚ । अ॒सौ । औदु॑बंरी । इ॒मे इति॑ । ए॒व । उपेति॑ । ध॒त्ते॒ ।  \newline


\textbf{Krama Paata} \newline

प्राञ्च॒मुप॑ । उप॑ दधाति । द॒धा॒ति॒ दा॒धार॑ । दा॒धार॑ यजमानलो॒कम् । य॒ज॒मा॒न॒लो॒कम् न । य॒ज॒मा॒न॒लो॒कमिति॑ यजमान - लो॒कम् । न प॑शू॒नाम् । प॒शू॒नाम् च॑ । च॒ यज॑मानस्य । यज॑मानस्य च । च॒ प्रा॒णम् । प्रा॒णमपि॑ । प्रा॒णमिति॑ प्र - अ॒नम् । अपि॑ दधाति । द॒धा॒त्यथो᳚ । अथो॒ खलु॑ । अथो॒ इत्यथो᳚ । खल्विष्ट॑कायाः । इष्ट॑काया॒ आतृ॑ण्णम् । आतृ॑ण्ण॒मनु॑ । आतृ॑ण्ण॒मित्या - तृ॒ण्ण॒म् । अनूप॑ । उप॑ दधाति । द॒धा॒ति॒ प्रा॒णाना᳚म् । प्रा॒णाना॒मुथ्सृ॑ष्ट्यै । प्रा॒णाना॒मिति॑ प्र - अ॒नाना᳚म् । उथ्सृ॑ष्ट्यै द्र॒फ्सः । उथ्सृ॑ष्ट्या॒ इत्युत् - सृ॒ष्ट्यै॒ । द्र॒फ्सश्च॑स्कन्द । च॒स्क॒न्देति॑ । इत्य॒भि । अ॒भि मृ॑शति । मृ॒श॒ति॒ होत्रा॑सु । होत्रा᳚स्वे॒व । ए॒वैन᳚म् । ए॒न॒म् प्रति॑ । प्रति॑ ष्ठापयति । स्था॒प॒य॒ति॒ सुचौ᳚ । स्रुचा॒वुप॑ । उप॑ दधाति । द॒धा॒त्याज्य॑स्य । आज्य॑स्य पू॒र्णाम् । पू॒र्णाम् का᳚र्ष्मर्य॒मयी᳚म् । का॒र्ष्म॒र्य॒मयी᳚म् द॒द्ध्नः । का॒र्ष्म॒र्य॒मयी॒मिति॑ काष्मर्य - मयी᳚म् । द॒द्ध्नः पू॒र्णाम् । पू॒र्णामौदु॑म्बरीम् । औदु॑म्बरीमि॒यम् । इ॒यम् ॅवै । वै का᳚र्ष्मर्य॒मयी᳚ । का॒र्ष्म॒र्य॒मय्य॒सौ । का॒र्ष्म॒र्य॒मयीति॑ कार्ष्मर्य - मयी᳚ । अ॒सावौदु॑म्बरी । औदु॑म्बरी॒मे । इ॒मे ए॒व । इ॒मे इती॒मे । ए॒वोप॑ । उप॑ धत्ते । ध॒त्ते॒ तू॒ष्णीम् \newline

\textbf{Jatai Paata} \newline

1. प्राञ्च॒ मुपोप॒ प्राञ्च॒म् प्राञ्च॒ मुप॑ । \newline
2. उप॑ दधाति दधा॒ त्युपोप॑ दधाति । \newline
3. द॒धा॒ति॒ दा॒धार॑ दा॒धार॑ दधाति दधाति दा॒धार॑ । \newline
4. दा॒धार॑ यजमानलो॒कं ॅय॑जमानलो॒कम् दा॒धार॑ दा॒धार॑ यजमानलो॒कम् । \newline
5. य॒ज॒मा॒न॒लो॒कम् न न य॑जमानलो॒कं ॅय॑जमानलो॒कम् न । \newline
6. य॒ज॒मा॒न॒लो॒कमिति॑ यजमान - लो॒कम् । \newline
7. न प॑शू॒नाम् प॑शू॒नाम् न न प॑शू॒नाम् । \newline
8. प॒शू॒नाम् च॑ च पशू॒नाम् प॑शू॒नाम् च॑ । \newline
9. च॒ यज॑मानस्य॒ यज॑मानस्य च च॒ यज॑मानस्य । \newline
10. यज॑मानस्य च च॒ यज॑मानस्य॒ यज॑मानस्य च । \newline
11. च॒ प्रा॒णम् प्रा॒णम् च॑ च प्रा॒णम् । \newline
12. प्रा॒ण मप्यपि॑ प्रा॒णम् प्रा॒ण मपि॑ । \newline
13. प्रा॒णमिति॑ प्र - अ॒नम् । \newline
14. अपि॑ दधाति दधा॒ त्यप्यपि॑ दधाति । \newline
15. द॒धा॒ त्यथो॒ अथो॑ दधाति दधा॒ त्यथो᳚ । \newline
16. अथो॒ खलु॒ खल्वथो॒ अथो॒ खलु॑ । \newline
17. अथो॒ इत्यथो᳚ । \newline
18. खल्विष्ट॑काया॒ इष्ट॑कायाः॒ खलु॒ खल्विष्ट॑कायाः । \newline
19. इष्ट॑काया॒ आतृ॑ण्ण॒ मातृ॑ण्ण॒ मिष्ट॑काया॒ इष्ट॑काया॒ आतृ॑ण्णम् । \newline
20. आतृ॑ण्ण॒ मन्वन् वातृ॑ण्ण॒ मातृ॑ण्ण॒ मनु॑ । \newline
21. आतृ॑ण्ण॒मित्या - तृ॒ण्ण॒म् । \newline
22. अनू पोपान् वनूप॑ । \newline
23. उप॑ दधाति दधा॒ त्युपोप॑ दधाति । \newline
24. द॒धा॒ति॒ प्रा॒णाना᳚म् प्रा॒णाना᳚म् दधाति दधाति प्रा॒णाना᳚म् । \newline
25. प्रा॒णाना॒ मुथ्सृ॑ष्ट्या॒ उथ्सृ॑ष्ट्यै प्रा॒णाना᳚म् प्रा॒णाना॒ मुथ्सृ॑ष्ट्यै । \newline
26. प्रा॒णाना॒मिति॑ प्र - अ॒नाना᳚म् । \newline
27. उथ्सृ॑ष्ट्यै द्र॒फ्सो द्र॒फ्स उथ्सृ॑ष्ट्या॒ उथ्सृ॑ष्ट्यै द्र॒फ्सः । \newline
28. उथ्सृ॑ष्ट्या॒ इत्युत् - सृ॒ष्ट्यै॒ । \newline
29. द्र॒फ्स श्च॑स्कन्द चस्कन्द द्र॒फ्सो द्र॒फ्स श्च॑स्कन्द । \newline
30. च॒स्क॒न्देतीति॑ चस्कन्द चस्क॒न्देति॑ । \newline
31. इत्य॒ भ्य॑भीती त्य॒भि । \newline
32. अ॒भि मृ॑शति मृश त्य॒भ्य॑भि मृ॑शति । \newline
33. मृ॒श॒ति॒ होत्रा॑सु॒ होत्रा॑सु मृशति मृशति॒ होत्रा॑सु । \newline
34. होत्रा᳚ स्वे॒वैव होत्रा॑सु॒ होत्रा᳚ स्वे॒व । \newline
35. ए॒वैन॑ मेन मे॒वै वैन᳚म् । \newline
36. ए॒न॒म् प्रति॒ प्रत्ये॑न मेन॒म् प्रति॑ । \newline
37. प्रति॑ ष्ठापयति स्थापयति॒ प्रति॒ प्रति॑ ष्ठापयति । \newline
38. स्था॒प॒य॒ति॒ स्रुचौ॒ स्रुचौ᳚ स्थापयति स्थापयति॒ स्रुचौ᳚ । \newline
39. स्रुचा॒ वुपोप॒ स्रुचौ॒ स्रुचा॒ वुप॑ । \newline
40. उप॑ दधाति दधा॒ त्युपोप॑ दधाति । \newline
41. द॒धा॒ त्याज्य॒स्या ज्य॑स्य दधाति दधा॒ त्याज्य॑स्य । \newline
42. आज्य॑स्य पू॒र्णाम् पू॒र्णा माज्य॒स्या ज्य॑स्य पू॒र्णाम् । \newline
43. पू॒र्णाम् का᳚र्ष्मर्य॒मयी᳚म् कार्ष्मर्य॒मयी᳚म् पू॒र्णाम् पू॒र्णाम् का᳚र्ष्मर्य॒मयी᳚म् । \newline
44. का॒र्ष्म॒र्य॒मयी᳚म् द॒द्ध्नो द॒द्ध्नः का᳚र्ष्मर्य॒मयी᳚म् कार्ष्मर्य॒मयी᳚म् द॒द्ध्नः । \newline
45. का॒र्ष्म॒र्य॒मयी॒मिति॑ कार्ष्मर्य - मयी᳚म् । \newline
46. द॒द्ध्नः पू॒र्णाम् पू॒र्णाम् द॒द्ध्नो द॒द्ध्नः पू॒र्णाम् । \newline
47. पू॒र्णा मौदुं॑बरी॒ मौदुं॑बरीम् पू॒र्णाम् पू॒र्णा मौदुं॑बरीम् । \newline
48. औदुं॑बरी मि॒य मि॒य मौदुं॑बरी॒ मौदुं॑बरी मि॒यम् । \newline
49. इ॒यं ॅवै वा इ॒य मि॒यं ॅवै । \newline
50. वै का᳚र्ष्मर्य॒मयी॑ कार्ष्मर्य॒मयी॒ वै वै का᳚र्ष्मर्य॒मयी᳚ । \newline
51. का॒र्ष्म॒र्य॒मय्य॒सा व॒सौ का᳚र्ष्मर्य॒मयी॑ कार्ष्मर्य॒मय्य॒सौ । \newline
52. का॒र्ष्म॒र्य॒मयीति॑ कार्ष्मर्य - मयी᳚ । \newline
53. अ॒सा वौदुं॑ब॒र्यौ दुं॑बर्य॒सा व॒सा वौदुं॑बरी । \newline
54. औदुं॑बरी॒मे इ॒मे औदुं॑ब॒र्यौ दुं॑बरी॒मे । \newline
55. इ॒मे ए॒वैवेमे इ॒मे ए॒व । \newline
56. इ॒मे इती॒मे । \newline
57. ए॒वोपो पै॒वै वोप॑ । \newline
58. उप॑ धत्ते धत्त॒ उपोप॑ धत्ते । \newline
59. ध॒त्ते॒ तू॒ष्णीम् तू॒ष्णीम् ध॑त्ते धत्ते तू॒ष्णीम् । \newline

\textbf{Ghana Paata } \newline

1. प्राञ्च॒ मुपोप॒ प्राञ्च॒म् प्राञ्च॒ मुप॑ दधाति दधा॒ त्युप॒ प्राञ्च॒म् प्राञ्च॒ मुप॑ दधाति । \newline
2. उप॑ दधाति दधा॒ त्युपोप॑ दधाति दा॒धार॑ दा॒धार॑ दधा॒ त्युपोप॑ दधाति दा॒धार॑ । \newline
3. द॒धा॒ति॒ दा॒धार॑ दा॒धार॑ दधाति दधाति दा॒धार॑ यजमानलो॒कं ॅय॑जमानलो॒कम् दा॒धार॑ दधाति दधाति दा॒धार॑ यजमानलो॒कम् । \newline
4. दा॒धार॑ यजमानलो॒कं ॅय॑जमानलो॒कम् दा॒धार॑ दा॒धार॑ यजमानलो॒कम् न न य॑जमानलो॒कम् दा॒धार॑ दा॒धार॑ यजमानलो॒कम् न । \newline
5. य॒ज॒मा॒न॒लो॒कम् न न य॑जमानलो॒कं ॅय॑जमानलो॒कम् न प॑शू॒नाम् प॑शू॒नाम् न य॑जमानलो॒कं ॅय॑जमानलो॒कम् न प॑शू॒नाम् । \newline
6. य॒ज॒मा॒न॒लो॒कमिति॑ यजमान - लो॒कम् । \newline
7. न प॑शू॒नाम् प॑शू॒नाम् न न प॑शू॒नाम् च॑ च पशू॒नाम् न न प॑शू॒नाम् च॑ । \newline
8. प॒शू॒नाम् च॑ च पशू॒नाम् प॑शू॒नाम् च॒ यज॑मानस्य॒ यज॑मानस्य च पशू॒नाम् प॑शू॒नाम् च॒ यज॑मानस्य । \newline
9. च॒ यज॑मानस्य॒ यज॑मानस्य च च॒ यज॑मानस्य च च॒ यज॑मानस्य च च॒ यज॑मानस्य च । \newline
10. यज॑मानस्य च च॒ यज॑मानस्य॒ यज॑मानस्य च प्रा॒णम् प्रा॒णम् च॒ यज॑मानस्य॒ यज॑मानस्य च प्रा॒णम् । \newline
11. च॒ प्रा॒णम् प्रा॒णम् च॑ च प्रा॒ण मप्यपि॑ प्रा॒णम् च॑ च प्रा॒ण मपि॑ । \newline
12. प्रा॒ण मप्यपि॑ प्रा॒णम् प्रा॒ण मपि॑ दधाति दधा॒ त्यपि॑ प्रा॒णम् प्रा॒ण मपि॑ दधाति । \newline
13. प्रा॒णमिति॑ प्र - अ॒नम् । \newline
14. अपि॑ दधाति दधा॒ त्यप्यपि॑ दधा॒ त्यथो॒ अथो॑ दधा॒ त्यप्यपि॑ दधा॒ त्यथो᳚ । \newline
15. द॒धा॒ त्यथो॒ अथो॑ दधाति दधा॒ त्यथो॒ खलु॒ खल्वथो॑ दधाति दधा॒ त्यथो॒ खलु॑ । \newline
16. अथो॒ खलु॒ खल्वथो॒ अथो॒ खल्विष्ट॑काया॒ इष्ट॑कायाः॒ खल्वथो॒ अथो॒ खल्विष्ट॑कायाः । \newline
17. अथो॒ इत्यथो᳚ । \newline
18. खल्विष्ट॑काया॒ इष्ट॑कायाः॒ खलु॒ खल्विष्ट॑काया॒ आतृ॑ण्ण॒ मातृ॑ण्ण॒ मिष्ट॑कायाः॒ खलु॒ खल्विष्ट॑काया॒ आतृ॑ण्णम् । \newline
19. इष्ट॑काया॒ आतृ॑ण्ण॒ मातृ॑ण्ण॒ मिष्ट॑काया॒ इष्ट॑काया॒ आतृ॑ण्ण॒ मन् वन् वातृ॑ण्ण॒ मिष्ट॑काया॒ इष्ट॑काया॒ आतृ॑ण्ण॒ मनु॑ । \newline
20. आतृ॑ण्ण॒ मन् वन् वातृ॑ण्ण॒ मातृ॑ण्ण॒ मनूपोपान् वातृ॑ण्ण॒ मातृ॑ण्ण॒ मनूप॑ । \newline
21. आतृ॑ण्ण॒मित्या - तृ॒ण्ण॒म् । \newline
22. अनूपोपान् वनूप॑ दधाति दधा॒ त्युपान् वनूप॑ दधाति । \newline
23. उप॑ दधाति दधा॒ त्युपोप॑ दधाति प्रा॒णाना᳚म् प्रा॒णाना᳚म् दधा॒ त्युपोप॑ दधाति प्रा॒णाना᳚म् । \newline
24. द॒धा॒ति॒ प्रा॒णाना᳚म् प्रा॒णाना᳚म् दधाति दधाति प्रा॒णाना॒ मुथ्सृ॑ष्ट्या॒ उथ्सृ॑ष्ट्यै प्रा॒णाना᳚म् दधाति दधाति प्रा॒णाना॒ मुथ्सृ॑ष्ट्यै । \newline
25. प्रा॒णाना॒ मुथ्सृ॑ष्ट्या॒ उथ्सृ॑ष्ट्यै प्रा॒णाना᳚म् प्रा॒णाना॒ मुथ्सृ॑ष्ट्यै द्र॒फ्सो द्र॒फ्स उथ्सृ॑ष्ट्यै प्रा॒णाना᳚म् प्रा॒णाना॒ मुथ्सृ॑ष्ट्यै द्र॒फ्सः । \newline
26. प्रा॒णाना॒मिति॑ प्र - अ॒नाना᳚म् । \newline
27. उथ्सृ॑ष्ट्यै द्र॒फ्सो द्र॒फ्स उथ्सृ॑ष्ट्या॒ उथ्सृ॑ष्ट्यै द्र॒फ्स श्च॑स्कन्द चस्कन्द द्र॒फ्स उथ्सृ॑ष्ट्या॒ उथ्सृ॑ष्ट्यै द्र॒फ्स श्च॑स्कन्द । \newline
28. उथ्सृ॑ष्ट्या॒ इत्युत् - सृ॒ष्ट्यै॒ । \newline
29. द्र॒फ्स श्च॑स्कन्द चस्कन्द द्र॒फ्सो द्र॒फ्स श्च॑स्क॒न्दे तीति॑ चस्कन्द द्र॒फ्सो द्र॒फ्स श्च॑स्क॒न्देति॑ । \newline
30. च॒स्क॒न्दे तीति॑ चस्कन्द चस्क॒न्दे त्य॒भ्य॑भीति॑ चस्कन्द चस्क॒न्दे त्य॒भि । \newline
31. इत्य॒भ्य॑भीती त्य॒भि मृ॑शति मृश त्य॒भीती त्य॒भि मृ॑शति । \newline
32. अ॒भि मृ॑शति मृश त्य॒भ्य॑भि मृ॑शति॒ होत्रा॑सु॒ होत्रा॑सु मृश त्य॒भ्य॑भि मृ॑शति॒ होत्रा॑सु । \newline
33. मृ॒श॒ति॒ होत्रा॑सु॒ होत्रा॑सु मृशति मृशति॒ होत्रा᳚ स्वे॒वैव होत्रा॑सु मृशति मृशति॒ होत्रा᳚स्वे॒व । \newline
34. होत्रा᳚ स्वे॒वैव होत्रा॑सु॒ होत्रा᳚ स्वे॒वैन॑ मेन मे॒व होत्रा॑सु॒ होत्रा᳚ स्वे॒वैन᳚म् । \newline
35. ए॒वैन॑ मेन मे॒वैवैन॒म् प्रति॒ प्रत्ये॑न मे॒वैवैन॒म् प्रति॑ । \newline
36. ए॒न॒म् प्रति॒ प्रत्ये॑न मेन॒म् प्रति॑ ष्ठापयति स्थापयति॒ प्रत्ये॑न मेन॒म् प्रति॑ ष्ठापयति । \newline
37. प्रति॑ ष्ठापयति स्थापयति॒ प्रति॒ प्रति॑ ष्ठापयति॒ स्रुचौ॒ स्रुचौ᳚ स्थापयति॒ प्रति॒ प्रति॑ ष्ठापयति॒ स्रुचौ᳚ । \newline
38. स्था॒प॒य॒ति॒ स्रुचौ॒ स्रुचौ᳚ स्थापयति स्थापयति॒ स्रुचा॒ वुपोप॒ स्रुचौ᳚ स्थापयति स्थापयति॒ स्रुचा॒ वुप॑ । \newline
39. स्रुचा॒ वुपोप॒ स्रुचौ॒ स्रुचा॒ वुप॑ दधाति दधा॒ त्युप॒ स्रुचौ॒ स्रुचा॒ वुप॑ दधाति । \newline
40. उप॑ दधाति दधा॒ त्युपोप॑ दधा॒ त्याज्य॒स्या ज्य॑स्य दधा॒ त्युपोप॑ दधा॒ त्याज्य॑स्य । \newline
41. द॒धा॒ त्याज्य॒स्या ज्य॑स्य दधाति दधा॒ त्याज्य॑स्य पू॒र्णाम् पू॒र्णा माज्य॑स्य दधाति दधा॒ त्याज्य॑स्य पू॒र्णाम् । \newline
42. आज्य॑स्य पू॒र्णाम् पू॒र्णा माज्य॒स्या ज्य॑स्य पू॒र्णाम् का᳚र्ष्मर्य॒मयी᳚म् कार्ष्मर्य॒मयी᳚म् पू॒र्णा माज्य॒ स्याज्य॑स्य पू॒र्णाम् का᳚र्ष्मर्य॒मयी᳚म् । \newline
43. पू॒र्णाम् का᳚र्ष्मर्य॒मयी᳚म् कार्ष्मर्य॒मयी᳚म् पू॒र्णाम् पू॒र्णाम् का᳚र्ष्मर्य॒मयी᳚म् द॒द्ध्नो द॒द्ध्नः का᳚र्ष्मर्य॒मयी᳚म् पू॒र्णाम् पू॒र्णाम् का᳚र्ष्मर्य॒मयी᳚म् द॒द्ध्नः । \newline
44. का॒र्ष्म॒र्य॒मयी᳚म् द॒द्ध्नो द॒द्ध्नः का᳚र्ष्मर्य॒मयी᳚म् कार्ष्मर्य॒मयी᳚म् द॒द्ध्नः पू॒र्णाम् पू॒र्णाम् द॒द्ध्नः का᳚र्ष्मर्य॒मयी᳚म् कार्ष्मर्य॒मयी᳚म् द॒द्ध्नः पू॒र्णाम् । \newline
45. का॒र्ष्म॒र्य॒मयी॒मिति॑ कार्ष्मर्य - मयी᳚म् । \newline
46. द॒द्ध्नः पू॒र्णाम् पू॒र्णाम् द॒द्ध्नो द॒द्ध्नः पू॒र्णा मौदुं॑बरी॒ मौदुं॑बरीम् पू॒र्णाम् द॒द्ध्नो द॒द्ध्नः पू॒र्णा मौदुं॑बरीम् । \newline
47. पू॒र्णा मौदुं॑बरी॒ मौदुं॑बरीम् पू॒र्णाम् पू॒र्णा मौदुं॑बरी मि॒य मि॒य मौदुं॑बरीम् पू॒र्णाम् पू॒र्णा मौदुं॑बरी मि॒यम् । \newline
48. औदुं॑बरी मि॒य मि॒य मौदुं॑बरी॒ मौदुं॑बरी मि॒यं ॅवै वा इ॒य मौदुं॑बरी॒ मौदुं॑बरी मि॒यं ॅवै । \newline
49. इ॒यं ॅवै वा इ॒य मि॒यं ॅवै का᳚र्ष्मर्य॒मयी॑ कार्ष्मर्य॒मयी॒ वा इ॒य मि॒यं ॅवै का᳚र्ष्मर्य॒मयी᳚ । \newline
50. वै का᳚र्ष्मर्य॒मयी॑ कार्ष्मर्य॒मयी॒ वै वै का᳚र्ष्मर्य॒मय्य॒सा व॒सौ का᳚र्ष्मर्य॒मयी॒ वै वै का᳚र्ष्मर्य॒मय्य॒सौ । \newline
51. का॒र्ष्म॒र्य॒मय्य॒सा व॒सौ का᳚र्ष्मर्य॒मयी॑ कार्ष्मर्य॒मय्य॒सा वौदुं॑ब॒र् यौदुं॑बर्य॒सौ का᳚र्ष्मर्य॒मयी॑ कार्ष्मर्य॒मय्य॒सा वौदुं॑बरी । \newline
52. का॒र्ष्म॒र्य॒मयीति॑ कार्ष्मर्य - मयी᳚ । \newline
53. अ॒सा वौदुं॑ब॒र् यौदुं॑बर्य॒सा व॒सा वौदुं॑बरी॒मे इ॒मे औदुं॑बर्य॒सा व॒सा वौदुं॑बरी॒मे । \newline
54. औदुं॑बरी॒मे इ॒मे औदुं॑ब॒र् यौदुं॑बरी॒मे ए॒वैवेमे औदुं॑ब॒र् यौदुं॑बरी॒मे ए॒व । \newline
55. इ॒मे ए॒वैवेमे इ॒मे ए॒वोपो पै॒वेमे इ॒मे ए॒वोप॑ । \newline
56. इ॒मे इती॒मे । \newline
57. ए॒वोपो पै॒वैवोप॑ धत्ते धत्त॒ उपै॒वैवोप॑ धत्ते । \newline
58. उप॑ धत्ते धत्त॒ उपोप॑ धत्ते तू॒ष्णीम् तू॒ष्णीम् ध॑त्त॒ उपोप॑ धत्ते तू॒ष्णीम् । \newline
59. ध॒त्ते॒ तू॒ष्णीम् तू॒ष्णीम् ध॑त्ते धत्ते तू॒ष्णी मुपोप॑ तू॒ष्णीम् ध॑त्ते धत्ते तू॒ष्णी मुप॑ । \newline
\pagebreak
\markright{ TS 5.2.7.4  \hfill https://www.vedavms.in \hfill}

\section{ TS 5.2.7.4 }

\textbf{TS 5.2.7.4 } \newline
\textbf{Samhita Paata} \newline

तू॒ष्णीमुप॑ दधाति॒ न हीमे यजु॒षाऽऽप्तु॒मर्.ह॑ति॒ दक्षि॑णां कार्ष्मर्य॒मयी॒-मुत्त॑रा॒मौ-दु॑म्बरीं॒ तस्मा॑द॒स्या अ॒सावुत्त॒रा ऽऽज्य॑स्य पू॒र्णां का᳚र्ष्मर्य॒मयीं॒ ॅवज्रो॒ वा आज्यं॒ ॅवज्रः॑ कार्ष्म॒र्यो॑ वज्रे॑णै॒व य॒ज्ञ्स्य॑ दक्षिण॒तो रक्षाꣳ॒॒स्यप॑ हन्ति द॒द्ध्नः पू॒र्णामौदु॑म्बरीं प॒शवो॒ वै दद्ध्यूर्गु॑दु॒म्बरः॑ प॒शुष्वे॒वोर्जं॑ दधाति पू॒र्णे उप॑ दधाति पू॒र्णे ए॒वैन॑ - [  ] \newline

\textbf{Pada Paata} \newline

तू॒ष्णीम् । उपेति॑ । द॒धा॒ति॒ । न । हि । इ॒मे इति॑ । यजु॑षा । आप्तु᳚म् । अर्.ह॑ति । दक्षि॑णाम् । का॒र्ष्म॒र्य॒मयी॒मिति॑ कार्ष्मर्य - मयी᳚म् । उत्त॑रा॒मित्युत् - त॒रा॒म् । औदु॑बंरीम् । तस्मा᳚त् । अ॒स्याः । अ॒सौ । उत्त॒रेत्युत् - त॒रा॒ । आज्य॑स्य । पू॒र्णाम् । का॒र्ष्म॒र्य॒मयी॒मिति॑ कार्ष्मर्य - मयी᳚म् । वज्रः॑ । वै । आज्य᳚म् । वज्रः॑ । का॒र्ष्म॒र्यः॑ । वज्रे॑ण । ए॒व । य॒ज्ञ्स्य॑ । द॒क्षि॒ण॒तः । रक्षाꣳ॑सि । अपेति॑ । ह॒न्ति॒ । द॒द्ध्नः । पू॒र्णाम् । औदु॑बंरीम् । प॒शवः॑ । वै । दधि॑ । ऊर्क् । उ॒दु॒बंरः॑ । प॒शुषु॑ । ए॒व । ऊर्ज᳚म् । द॒धा॒ति॒ । पू॒र्णे इति॑ । उपेति॑ । द॒धा॒ति॒ । पू॒र्णे इति॑ । ए॒व । ए॒न॒म् ।  \newline


\textbf{Krama Paata} \newline

तू॒ष्णीमुप॑ । उप॑ दधाति । द॒धा॒ति॒ न । न हि । हीमे । इ॒मे यजु॑षा । इ॒मे इती॒मे । यजु॒षाऽप्तु᳚म् । आप्तु॒मर्.ह॑ति । अर्.ह॑ति॒ दक्षि॑णाम् । दक्षि॑णाम् कार्ष्मर्य॒मयी᳚म् । का॒र्ष्म॒र्य॒मयी॒मुत्त॑राम् । का॒र्ष्म॒र्य॒मयी॒मिति॑ कार्ष्मर्य - मयी᳚म् । उत्त॑रा॒मौदु॑म्बरीम् । उत्त॑रा॒मित्युत् - त॒रा॒म् । औदु॑म्बरी॒म् तस्मा᳚त् । तस्मा॑द॒स्याः । अ॒स्या अ॒सौ । अ॒सावुत्त॑रा । उत्त॒राऽऽज्य॑स्य । उत्त॒रेत्युत् - त॒रा॒ । आज्य॑स्य पू॒र्णाम् । पू॒र्णाम् का᳚र्ष्मर्य॒मयी᳚म् । का॒र्ष्म॒र्य॒मयी॒म् ॅवज्रः॑ । का॒र्ष्म॒र्य॒मयी॒मिति॑ कार्ष्मर्य - मयी᳚म् । वज्रो॒ वै । 
वा आज्य᳚म् । आज्य॒म् ॅवज्रः॑ । वज्रः॑ कार्ष्म॒र्यः॑ । का॒र्ष्म॒र्यो॑ वज्रे॑ण । वज्रे॑णै॒व । ए॒व य॒ज्ञ्स्य॑ । य॒ज्ञ्स्य॑ दक्षिण॒तः । द॒क्षि॒ण॒तो रक्षाꣳ॑सि । रक्षाꣳ॒॒स्यप॑ । अप॑ हन्ति । ह॒न्ति॒ द॒द्ध्नः । द॒द्ध्नः पू॒र्णाम् । पू॒र्णामौदु॑म्बरीम् । औदु॑म्बरीम् प॒शवः॑ । प॒शवो॒ वै । वै दधि॑ । दद्ध्यूर्क् । ऊर्गु॑दु॒म्बरः॑ । उ॒दु॒म्बरः॑ प॒शुषु॑ । प॒शुष्वे॒व । ए॒वोर्ज᳚म् । ऊर्ज॑म् दधाति । द॒धा॒ति॒ पू॒र्णे । पू॒र्णे उप॑ । पू॒र्णे इति॑ पू॒र्णे । उप॑ दधाति । द॒धा॒ति॒ पू॒र्णे । पू॒र्णे ए॒व । पू॒र्णे इति॑ पू॒र्णे । ए॒वैन᳚म् । ए॒न॒म॒मुष्मिन्न्॑ \newline

\textbf{Jatai Paata} \newline

1. तू॒ष्णी मुपोप॑ तू॒ष्णीम् तू॒ष्णी मुप॑ । \newline
2. उप॑ दधाति दधा॒ त्युपोप॑ दधाति । \newline
3. द॒धा॒ति॒ न न द॑धाति दधाति॒ न । \newline
4. न हि हि न न हि । \newline
5. हीमे इ॒मे हि हीमे । \newline
6. इ॒मे यजु॑षा॒ यजु॑षे॒मे इ॒मे यजु॑षा । \newline
7. इ॒मे इती॒मे । \newline
8. यजु॒षा ऽऽप्तु॒ माप्तुं॒ ॅयजु॑षा॒ यजु॒षा ऽऽप्तु᳚म् । \newline
9. आप्तु॒ मर्.ह॒ त्यर्.ह॒ त्याप्तु॒ माप्तु॒ मर्.ह॑ति । \newline
10. अर्.ह॑ति॒ दक्षि॑णा॒म् दक्षि॑णा॒ मर्.ह॒ त्यर्.ह॑ति॒ दक्षि॑णाम् । \newline
11. दक्षि॑णाम् कार्ष्मर्य॒मयी᳚म् कार्ष्मर्य॒मयी॒म् दक्षि॑णा॒म् दक्षि॑णाम् कार्ष्मर्य॒मयी᳚म् । \newline
12. का॒र्ष्म॒र्य॒मयी॒ मुत्त॑रा॒ मुत्त॑राम् कार्ष्मर्य॒मयी᳚म् कार्ष्मर्य॒मयी॒ मुत्त॑राम् । \newline
13. का॒र्ष्म॒र्य॒मयी॒मिति॑ कार्ष्मर्य - मयी᳚म् । \newline
14. उत्त॑रा॒ मौदुं॑बरी॒ मौदुं॑बरी॒ मुत्त॑रा॒ मुत्त॑रा॒ मौदुं॑बरीम् । \newline
15. उत्त॑रा॒मित्युत् - त॒रा॒म् । \newline
16. औदुं॑बरी॒म् तस्मा॒त् तस्मा॒ दौदुं॑बरी॒ मौदुं॑बरी॒म् तस्मा᳚त् । \newline
17. तस्मा॑ द॒स्या अ॒स्या स्तस्मा॒त् तस्मा॑ द॒स्याः । \newline
18. अ॒स्या अ॒सा व॒सा व॒स्या अ॒स्या अ॒सौ । \newline
19. अ॒सा वुत्त॒ रोत्त॑रा॒ ऽसा व॒सा वुत्त॑रा । \newline
20. उत्त॒रा ऽऽज्य॒स्या ज्य॒स्यो त्त॒रोत्त॒रा ऽऽज्य॑स्य । \newline
21. उत्त॒रेत्युत् - त॒रा॒ । \newline
22. आज्य॑स्य पू॒र्णाम् पू॒र्णा माज्य॒स्या ज्य॑स्य पू॒र्णाम् । \newline
23. पू॒र्णाम् का᳚र्ष्मर्य॒मयी᳚म् कार्ष्मर्य॒मयी᳚म् पू॒र्णाम् पू॒र्णाम् का᳚र्ष्मर्य॒मयी᳚म् । \newline
24. का॒र्ष्म॒र्य॒मयीं॒ ॅवज्रो॒ वज्रः॑ कार्ष्मर्य॒मयी᳚म् कार्ष्मर्य॒मयीं॒ ॅवज्रः॑ । \newline
25. का॒र्ष्म॒र्य॒मयी॒मिति॑ कार्ष्मर्य - मयी᳚म् । \newline
26. वज्रो॒ वै वै वज्रो॒ वज्रो॒ वै । \newline
27. वा आज्य॒ माज्यं॒ ॅवै वा आज्य᳚म् । \newline
28. आज्यं॒ ॅवज्रो॒ वज्र॒ आज्य॒ माज्यं॒ ॅवज्रः॑ । \newline
29. वज्रः॑ कार्ष्म॒र्यः॑ कार्ष्म॒र्यो॑ वज्रो॒ वज्रः॑ कार्ष्म॒र्यः॑ । \newline
30. का॒र्ष्म॒र्यो॑ वज्रे॑ण॒ वज्रे॑ण कार्ष्म॒र्यः॑ कार्ष्म॒र्यो॑ वज्रे॑ण । \newline
31. वज्रे॑ णै॒वैव वज्रे॑ण॒ वज्रे॑णै॒व । \newline
32. ए॒व य॒ज्ञ्स्य॑ य॒ज्ञ् स्यै॒वैव य॒ज्ञ्स्य॑ । \newline
33. य॒ज्ञ्स्य॑ दक्षिण॒तो द॑क्षिण॒तो य॒ज्ञ्स्य॑ य॒ज्ञ्स्य॑ दक्षिण॒तः । \newline
34. द॒क्षि॒ण॒तो रक्षाꣳ॑सि॒ रक्षाꣳ॑सि दक्षिण॒तो द॑क्षिण॒तो रक्षाꣳ॑सि । \newline
35. रक्षाꣳ॒॒ स्यपाप॒ रक्षाꣳ॑सि॒ रक्षाꣳ॒॒ स्यप॑ । \newline
36. अप॑ हन्ति ह॒न्त्य पाप॑ हन्ति । \newline
37. ह॒न्ति॒ द॒द्ध्नो द॒द्ध्नो ह॑न्ति हन्ति द॒द्ध्नः । \newline
38. द॒द्ध्नः पू॒र्णाम् पू॒र्णाम् द॒द्ध्नो द॒द्ध्नः पू॒र्णाम् । \newline
39. पू॒र्णा मौदुं॑बरी॒ मौदुं॑बरीम् पू॒र्णाम् पू॒र्णा मौदुं॑बरीम् । \newline
40. औदुं॑बरीम् प॒शवः॑ प॒शव॒ औदुं॑बरी॒ मौदुं॑बरीम् प॒शवः॑ । \newline
41. प॒शवो॒ वै वै प॒शवः॑ प॒शवो॒ वै । \newline
42. वै दधि॒ दधि॒ वै वै दधि॑ । \newline
43. दध्यूर् गूर्ग् दधि॒ दध्यूर्क् । \newline
44. ऊर्गु॑दुं॒बर॑ उदुं॒बर॒ ऊर् गूर् गु॑दुं॒बरः॑ । \newline
45. उ॒दुं॒बरः॑ प॒शुषु॑ प॒शुषू॑ दुं॒बर॑ उदुं॒बरः॑ प॒शुषु॑ । \newline
46. प॒शु ष्वे॒वैव प॒शुषु॑ प॒शु ष्वे॒व । \newline
47. ए॒वोर्ज॒ मूर्ज॑ मे॒वैवोर्ज᳚म् । \newline
48. ऊर्ज॑म् दधाति दधा॒ त्यूर्ज॒ मूर्ज॑म् दधाति । \newline
49. द॒धा॒ति॒ पू॒र्णे पू॒र्णे द॑धाति दधाति पू॒र्णे । \newline
50. पू॒र्णे उपोप॑ पू॒र्णे पू॒र्णे उप॑ । \newline
51. पू॒र्णे इति॑ पू॒र्णे । \newline
52. उप॑ दधाति दधा॒ त्युपोप॑ दधाति । \newline
53. द॒धा॒ति॒ पू॒र्णे पू॒र्णे द॑धाति दधाति पू॒र्णे । \newline
54. पू॒र्णे ए॒वैव पू॒र्णे पू॒र्णे ए॒व । \newline
55. पू॒र्णे इति॑ पू॒र्णे । \newline
56. ए॒वैन॑ मेन मे॒वै वैन᳚म् । \newline
57. ए॒न॒ म॒मुष्मि॑न् न॒मुष्मि॑न् नेन मेन म॒मुष्मिन्न्॑ । \newline

\textbf{Ghana Paata } \newline

1. तू॒ष्णी मुपोप॑ तू॒ष्णीम् तू॒ष्णी मुप॑ दधाति दधा॒ त्युप॑ तू॒ष्णीम् तू॒ष्णी मुप॑ दधाति । \newline
2. उप॑ दधाति दधा॒ त्युपोप॑ दधाति॒ न न द॑धा॒ त्युपोप॑ दधाति॒ न । \newline
3. द॒धा॒ति॒ न न द॑धाति दधाति॒ न हि हि न द॑धाति दधाति॒ न हि । \newline
4. न हि हि न न हीमे इ॒मे हि न न हीमे । \newline
5. हीमे इ॒मे हि हीमे यजु॑षा॒ यजु॑षे॒मे हि हीमे यजु॑षा । \newline
6. इ॒मे यजु॑षा॒ यजु॑षे॒मे इ॒मे यजु॒षा ऽऽप्तु॒ माप्तुं॒ ॅयजु॑षे॒मे इ॒मे यजु॒षा ऽऽप्तु᳚म् । \newline
7. इ॒मे इती॒मे । \newline
8. यजु॒षा ऽऽप्तु॒ माप्तुं॒ ॅयजु॑षा॒ यजु॒षा ऽऽप्तु॒ मर्.ह॒ त्यर्.ह॒ त्याप्तुं॒ ॅयजु॑षा॒ यजु॒षा ऽऽप्तु॒ मर्.ह॑ति । \newline
9. आप्तु॒ मर्.ह॒ त्यर्.ह॒ त्याप्तु॒ माप्तु॒ मर्.ह॑ति॒ दक्षि॑णा॒म् दक्षि॑णा॒ मर्.ह॒ त्याप्तु॒ माप्तु॒ मर्.ह॑ति॒ दक्षि॑णाम् । \newline
10. अर्.ह॑ति॒ दक्षि॑णा॒म् दक्षि॑णा॒ मर्.ह॒ त्यर्.ह॑ति॒ दक्षि॑णाम् कार्ष्मर्य॒मयी᳚म् कार्ष्मर्य॒मयी॒म् दक्षि॑णा॒ मर्.ह॒ त्यर्.ह॑ति॒ दक्षि॑णाम् कार्ष्मर्य॒मयी᳚म् । \newline
11. दक्षि॑णाम् कार्ष्मर्य॒मयी᳚म् कार्ष्मर्य॒मयी॒म् दक्षि॑णा॒म् दक्षि॑णाम् कार्ष्मर्य॒मयी॒ मुत्त॑रा॒ मुत्त॑राम् कार्ष्मर्य॒मयी॒म् दक्षि॑णा॒म् दक्षि॑णाम् कार्ष्मर्य॒मयी॒ मुत्त॑राम् । \newline
12. का॒र्ष्म॒र्य॒मयी॒ मुत्त॑रा॒ मुत्त॑राम् कार्ष्मर्य॒मयी᳚म् कार्ष्मर्य॒मयी॒ मुत्त॑रा॒ मौदुं॑बरी॒ मौदुं॑बरी॒ मुत्त॑राम् कार्ष्मर्य॒मयी᳚म् कार्ष्मर्य॒मयी॒ मुत्त॑रा॒ मौदुं॑बरीम् । \newline
13. का॒र्ष्म॒र्य॒मयी॒मिति॑ कार्ष्मर्य - मयी᳚म् । \newline
14. उत्त॑रा॒ मौदुं॑बरी॒ मौदुं॑बरी॒ मुत्त॑रा॒ मुत्त॑रा॒ मौदुं॑बरी॒म् तस्मा॒त् तस्मा॒ दौदुं॑बरी॒ मुत्त॑रा॒ मुत्त॑रा॒ मौदुं॑बरी॒म् तस्मा᳚त् । \newline
15. उत्त॑रा॒मित्युत् - त॒रा॒म् । \newline
16. औदुं॑बरी॒म् तस्मा॒त् तस्मा॒ दौदुं॑बरी॒ मौदुं॑बरी॒म् तस्मा॑ द॒स्या अ॒स्या स्तस्मा॒ दौदुं॑बरी॒ मौदुं॑बरी॒म् तस्मा॑ द॒स्याः । \newline
17. तस्मा॑ द॒स्या अ॒स्या स्तस्मा॒त् तस्मा॑ द॒स्या अ॒सा व॒सा व॒स्या स्तस्मा॒त् तस्मा॑ द॒स्या अ॒सौ । \newline
18. अ॒स्या अ॒सा व॒सा व॒स्या अ॒स्या अ॒सा वुत्त॒ रोत्त॑रा॒ ऽसा व॒स्या अ॒स्या अ॒सा वुत्त॑रा । \newline
19. अ॒सा वुत्त॒ रोत्त॑रा॒ ऽसा व॒सा वुत्त॒रा ऽऽज्य॒स्या ज्य॒स्योत्त॑रा॒ ऽसा व॒सा वुत्त॒रा ऽऽज्य॑स्य । \newline
20. उत्त॒रा ऽऽज्य॒स्या ज्य॒स्योत्त॒ रोत्त॒रा ऽऽज्य॑स्य पू॒र्णाम् पू॒र्णा माज्य॒ स्योत्त॒ रोत्त॒रा ऽऽज्य॑स्य पू॒र्णाम् । \newline
21. उत्त॒रेत्युत् - त॒रा॒ । \newline
22. आज्य॑स्य पू॒र्णाम् पू॒र्णा माज्य॒स्या ज्य॑स्य पू॒र्णाम् का᳚र्ष्मर्य॒मयी᳚म् कार्ष्मर्य॒मयी᳚म् पू॒र्णा माज्य॒स्या ज्य॑स्य पू॒र्णाम् का᳚र्ष्मर्य॒मयी᳚म् । \newline
23. पू॒र्णाम् का᳚र्ष्मर्य॒मयी᳚म् कार्ष्मर्य॒मयी᳚म् पू॒र्णाम् पू॒र्णाम् का᳚र्ष्मर्य॒मयीं॒ ॅवज्रो॒ वज्रः॑ कार्ष्मर्य॒मयी᳚म् पू॒र्णाम् पू॒र्णाम् का᳚र्ष्मर्य॒मयीं॒ ॅवज्रः॑ । \newline
24. का॒र्ष्म॒र्य॒मयीं॒ ॅवज्रो॒ वज्रः॑ कार्ष्मर्य॒मयी᳚म् कार्ष्मर्य॒मयीं॒ ॅवज्रो॒ वै वै वज्रः॑ कार्ष्मर्य॒मयी᳚म् कार्ष्मर्य॒मयीं॒ ॅवज्रो॒ वै । \newline
25. का॒र्ष्म॒र्य॒मयी॒मिति॑ कार्ष्मर्य - मयी᳚म् । \newline
26. वज्रो॒ वै वै वज्रो॒ वज्रो॒ वा आज्य॒ माज्यं॒ ॅवै वज्रो॒ वज्रो॒ वा आज्य᳚म् । \newline
27. वा आज्य॒ माज्यं॒ ॅवै वा आज्यं॒ ॅवज्रो॒ वज्र॒ आज्यं॒ ॅवै वा आज्यं॒ ॅवज्रः॑ । \newline
28. आज्यं॒ ॅवज्रो॒ वज्र॒ आज्य॒ माज्यं॒ ॅवज्रः॑ कार्ष्म॒र्यः॑ कार्ष्म॒र्यो॑ वज्र॒ आज्य॒ माज्यं॒ ॅवज्रः॑ कार्ष्म॒र्यः॑ । \newline
29. वज्रः॑ कार्ष्म॒र्यः॑ कार्ष्म॒र्यो॑ वज्रो॒ वज्रः॑ कार्ष्म॒र्यो॑ वज्रे॑ण॒ वज्रे॑ण कार्ष्म॒र्यो॑ वज्रो॒ वज्रः॑ कार्ष्म॒र्यो॑ वज्रे॑ण । \newline
30. का॒र्ष्म॒र्यो॑ वज्रे॑ण॒ वज्रे॑ण कार्ष्म॒र्यः॑ कार्ष्म॒र्यो॑ वज्रे॑ णै॒वैव वज्रे॑ण कार्ष्म॒र्यः॑ कार्ष्म॒र्यो॑ वज्रे॑णै॒व । \newline
31. वज्रे॑ णै॒वैव वज्रे॑ण॒ वज्रे॑णै॒व य॒ज्ञ्स्य॑ य॒ज्ञ्स्यै॒व वज्रे॑ण॒ वज्रे॑णै॒व य॒ज्ञ्स्य॑ । \newline
32. ए॒व य॒ज्ञ्स्य॑ य॒ज्ञ् स्यै॒वैव य॒ज्ञ्स्य॑ दक्षिण॒तो द॑क्षिण॒तो य॒ज्ञ् स्यै॒वैव य॒ज्ञ्स्य॑ दक्षिण॒तः । \newline
33. य॒ज्ञ्स्य॑ दक्षिण॒तो द॑क्षिण॒तो य॒ज्ञ्स्य॑ य॒ज्ञ्स्य॑ दक्षिण॒तो रक्षाꣳ॑सि॒ रक्षाꣳ॑सि दक्षिण॒तो य॒ज्ञ्स्य॑ य॒ज्ञ्स्य॑ दक्षिण॒तो रक्षाꣳ॑सि । \newline
34. द॒क्षि॒ण॒तो रक्षाꣳ॑सि॒ रक्षाꣳ॑सि दक्षिण॒तो द॑क्षिण॒तो रक्षाꣳ॒॒ स्यपाप॒ रक्षाꣳ॑सि दक्षिण॒तो द॑क्षिण॒तो रक्षाꣳ॒॒स्यप॑ । \newline
35. रक्षाꣳ॒॒ स्यपाप॒ रक्षाꣳ॑सि॒ रक्षाꣳ॒॒स्यप॑ हन्ति ह॒न्त्यप॒ रक्षाꣳ॑सि॒ रक्षाꣳ॒॒स्यप॑ हन्ति । \newline
36. अप॑ हन्ति ह॒न्त्यपाप॑ हन्ति द॒द्ध्नो द॒द्ध्नो ह॒न्त्यपाप॑ हन्ति द॒द्ध्नः । \newline
37. ह॒न्ति॒ द॒द्ध्नो द॒द्ध्नो ह॑न्ति हन्ति द॒द्ध्नः पू॒र्णाम् पू॒र्णाम् द॒द्ध्नो ह॑न्ति हन्ति द॒द्ध्नः पू॒र्णाम् । \newline
38. द॒द्ध्नः पू॒र्णाम् पू॒र्णाम् द॒द्ध्नो द॒द्ध्नः पू॒र्णा मौदुं॑बरी॒ मौदुं॑बरीम् पू॒र्णाम् द॒द्ध्नो द॒द्ध्नः पू॒र्णा मौदुं॑बरीम् । \newline
39. पू॒र्णा मौदुं॑बरी॒ मौदुं॑बरीम् पू॒र्णाम् पू॒र्णा मौदुं॑बरीम् प॒शवः॑ प॒शव॒ औदुं॑बरीम् पू॒र्णाम् पू॒र्णा मौदुं॑बरीम् प॒शवः॑ । \newline
40. औदुं॑बरीम् प॒शवः॑ प॒शव॒ औदुं॑बरी॒ मौदुं॑बरीम् प॒शवो॒ वै वै प॒शव॒ औदुं॑बरी॒ मौदुं॑बरीम् प॒शवो॒ वै । \newline
41. प॒शवो॒ वै वै प॒शवः॑ प॒शवो॒ वै दधि॒ दधि॒ वै प॒शवः॑ प॒शवो॒ वै दधि॑ । \newline
42. वै दधि॒ दधि॒ वै वै दध्यूर्गूर्ग् दधि॒ वै वै दध्यूर्क् । \newline
43. दध्यूर्गूर्ग् दधि॒ दध्यूर्गु॑दुं॒बर॑ उदुं॒बर॒ ऊर्ग् दधि॒ दध्यूर्गु॑दुं॒बरः॑ । \newline
44. ऊर्गु॑दुं॒बर॑ उदुं॒बर॒ ऊर्गूर्गु॑दुं॒बरः॑ प॒शुषु॑ प॒शुषू॑दुं॒बर॒ ऊर्गूर्गु॑दुं॒बरः॑ प॒शुषु॑ । \newline
45. उ॒दुं॒बरः॑ प॒शुषु॑ प॒शुषू॑दुं॒बर॑ उदुं॒बरः॑ प॒शु ष्वे॒वैव प॒शुषू॑दुं॒बर॑ उदुं॒बरः॑ प॒शुष्वे॒व । \newline
46. प॒शु ष्वे॒वैव प॒शुषु॑ प॒शु ष्वे॒वोर्ज॒ मूर्ज॑ मे॒व प॒शुषु॑ प॒शु ष्वे॒वोर्ज᳚म् । \newline
47. ए॒वोर्ज॒ मूर्ज॑ मे॒वैवोर्ज॑म् दधाति दधा॒ त्यूर्ज॑ मे॒वैवोर्ज॑म् दधाति । \newline
48. ऊर्ज॑म् दधाति दधा॒ त्यूर्ज॒ मूर्ज॑म् दधाति पू॒र्णे पू॒र्णे द॑धा॒ त्यूर्ज॒ मूर्ज॑म् दधाति पू॒र्णे । \newline
49. द॒धा॒ति॒ पू॒र्णे पू॒र्णे द॑धाति दधाति पू॒र्णे उपोप॑ पू॒र्णे द॑धाति दधाति पू॒र्णे उप॑ । \newline
50. पू॒र्णे उपोप॑ पू॒र्णे पू॒र्णे उप॑ दधाति दधा॒ त्युप॑ पू॒र्णे पू॒र्णे उप॑ दधाति । \newline
51. पू॒र्णे इति॑ पू॒र्णे । \newline
52. उप॑ दधाति दधा॒ त्युपोप॑ दधाति पू॒र्णे पू॒र्णे द॑धा॒ त्युपोप॑ दधाति पू॒र्णे । \newline
53. द॒धा॒ति॒ पू॒र्णे पू॒र्णे द॑धाति दधाति पू॒र्णे ए॒वैव पू॒र्णे द॑धाति दधाति पू॒र्णे ए॒व । \newline
54. पू॒र्णे ए॒वैव पू॒र्णे पू॒र्णे ए॒वैन॑ मेन मे॒व पू॒र्णे पू॒र्णे ए॒वैन᳚म् । \newline
55. पू॒र्णे इति॑ पू॒र्णे । \newline
56. ए॒वैन॑ मेन मे॒वैवैन॑ म॒मुष्मि॑न् न॒मुष्मि॑न् नेन मे॒वैवैन॑ म॒मुष्मिन्न्॑ । \newline
57. ए॒न॒ म॒मुष्मि॑न् न॒मुष्मि॑न् नेन मेन म॒मुष्मि॑न् ॅलो॒के लो॒के॑ ऽमुष्मि॑न् नेन मेन म॒मुष्मि॑न् ॅलो॒के । \newline
\pagebreak
\markright{ TS 5.2.7.5  \hfill https://www.vedavms.in \hfill}

\section{ TS 5.2.7.5 }

\textbf{TS 5.2.7.5 } \newline
\textbf{Samhita Paata} \newline

म॒मुष्मि॑न् ॅलो॒क उप॑तिष्ठेते वि॒राज्य॒ग्निश्चे॑त॒व्य॑ इत्या॑हुः॒ स्रुग्वै वि॒राड्यथ् स्रुचा॑वुप॒दधा॑ति वि॒राज्ये॒वाग्निं चि॑नुते यज्ञ्मु॒खेय॑ज्ञ्मुखे॒ वै क्रि॒यमा॑णे य॒ज्ञ्ꣳ रक्षाꣳ॑सि जिघाꣳसन्ति यज्ञ्मु॒खꣳ रु॒क्मो यद्-रु॒क्मं ॅव्या॑घा॒रय॑ति यज्ञ्मु॒खादे॒व रक्षाꣳ॒॒स्यप॑ हन्ति प॒ञ्चभि॒व्या घा॑रयति॒ पाङ्क्तो॑ य॒ज्ञो यावा॑ने॒व य॒ज्ञ्स्तस्मा॒द्-रक्षाꣳ॒॒स्यप॑ हन्त्यक्ष्ण॒याव्या घा॑रयति॒ तस्मा॑दक्ष्ण॒या ( ) प॒शवोऽङ्गा॑नि॒ प्र ह॑रन्ति॒ प्रति॑ष्ठित्यै ॥ \newline

\textbf{Pada Paata} \newline

अ॒मुष्मिन्न्॑ । लो॒के । उपेति॑ । ति॒ष्ठे॒ते॒ इति॑ । वि॒राजीति॑ वि - राजि॑ । अ॒ग्निः । चे॒त॒व्यः॑ । इति॑ । आ॒हुः॒ । स्रुक् । वै । वि॒राडिति॑ वि-राट् । यत् । स्रुचौ᳚ । उ॒प॒दधा॒तीत्यु॑प - दधा॑ति । वि॒राजीति॑ वि - राजि॑ । ए॒व । अ॒ग्निम् । चि॒नु॒ते॒ । य॒ज्ञ्॒मु॒खे य॑ज्ञ्मुख॒ इति॑ यज्ञ्मु॒खे-य॒ज्ञ्॒मु॒खे॒ । वै । क्रि॒यमा॑णे । य॒ज्ञ्म् । रक्षाꣳ॑सि । जि॒घाꣳ॒॒स॒न्ति॒ । य॒ज्ञ्॒मु॒खमिति॑ यज्ञ् - मु॒खम् । रु॒क्मः । यत् । रु॒क्मम् । व्या॒घा॒रय॒तीति॑ वि - आ॒घा॒रय॑ति । य॒ज्ञ्॒मु॒खादिति॑ यज्ञ् - मु॒खात् । ए॒व । रक्षाꣳ॑सि । अपेति॑ । ह॒न्ति॒ । प॒ञ्चभि॒रिति॑ प॒ञ्च - भिः॒ । व्याघा॑रय॒तीति॑ वि - आघा॑रयति । पाङ्क्तः॑ । य॒ज्ञ्ः । यावान्॑ । ए॒व । य॒ज्ञ्ः । तस्मा᳚त् । रक्षाꣳ॑सि । अपेति॑ । ह॒न्ति॒ । अ॒क्ष्ण॒या । व्याघा॑रय॒तीति॑ वि - आघा॑रयति । तस्मा᳚त् । अ॒क्ष्ण॒या ( ) । प॒शवः॑ । अङ्गा॑नि । प्रेति॑ । ह॒र॒न्ति॒ । प्रति॑ष्ठित्या॒ इति॒ प्रति॑ - स्थि॒त्यै॒ ॥  \newline


\textbf{Krama Paata} \newline

अ॒मुष्मि॑न् ॅलो॒के । लो॒क उप॑ । उप॑ तिष्ठेते । ति॒ष्ठे॒ते॒ वि॒राजि॑ । ति॒ष्ठे॒ते॒ इति॑ तिष्ठेते । वि॒राज्य॒ग्निः । वि॒राजीति॑ वि - राजि॑ । अ॒ग्निश्चे॑त॒व्यः॑ । चे॒त॒व्य॑ इति॑ । इत्या॑हुः । आ॒हुः॒ स्रुक् । स्रुग् वै । वै वि॒राट् । वि॒राड् यत् । वि॒राडिति॑ वि - राट् । यथ् स्रुचौ᳚ । स्रुचा॑वुप॒दधा॑ति । उ॒प॒दधा॑ति वि॒राजि॑ । उ॒प॒दधा॒तीत्यु॑प - दधा॑ति । वि॒राज्ये॒व । वि॒राजीति॑ वि - राजि॑ । ए॒वाग्निम् । अ॒ग्निम् चि॑नुते । चि॒नु॒ते॒ य॒ज्ञ्॒मु॒खेय॑ज्ञ्मुखे । य॒ज्ञ्॒मु॒खेय॑ज्ञ्मुखे॒ वै । य॒ज्ञ्॒मु॒खेय॑ज्ञ्मुख॒ इति॑ यज्ञ्मु॒खे - य॒ज्ञ्॒मु॒खे॒ । वै क्रि॒यमा॑णे । क्रि॒यमा॑णे य॒ज्ञ्म् । य॒ज्ञ्ꣳ रक्षाꣳ॑सि । रक्षाꣳ॑सि जिघाꣳसन्ति । जि॒घाꣳ॒॒स॒न्ति॒ य॒ज्ञ्॒मु॒खम् । य॒ज्ञ्॒मु॒खꣳ रु॒क्मः । य॒ज्ञ्॒मु॒खमिति॑ यज्ञ् - मु॒खम् । रु॒क्मो यत् । यद् रु॒क्मम् । रु॒क्मम् ॅव्या॑घा॒रय॑ति । व्या॒घा॒रय॑ति यज्ञ्मु॒खात् । व्या॒घा॒रय॒तीति॑ वि - आ॒घा॒रय॑ति । य॒ज्ञ्॒मु॒खादे॒व । य॒ज्ञ्॒मु॒खादिति॑ यज्ञ् - मु॒खात् । ए॒व रक्षाꣳ॑सि । रक्षाꣳ॒॒स्यप॑ । अप॑ हन्ति । ह॒न्ति॒ प॒ञ्चभिः॑ । प॒ञ्चभि॒र् व्याघा॑रयति । प॒ञ्चभि॒रिति॑ प॒ञ्च - भिः॒ । व्याघा॑रयति॒ पाङ्क्तः॑ । व्याघा॑रय॒तीति॑ वि - आघा॑रयति । पाङ्क्तो॑ य॒ज्ञ्ः । य॒ज्ञो यावान्॑ । यावा॑ने॒व । ए॒व य॒ज्ञ्ः । य॒ज्ञ्स्तस्मा᳚त् । तस्मा॒द् रक्षाꣳ॑सि । रक्षाꣳ॒॒स्यप॑ । अप॑ हन्ति । ह॒न्त्य॒क्ष्ण॒या । अ॒क्ष्ण॒या व्याघा॑रयति । व्याघा॑रयति॒ तस्मा᳚त् । व्याघा॑रय॒तीति॑ वि - आघा॑रयति । तस्मा॑दक्ष्ण॒या ( ) । अ॒क्ष्ण॒या प॒शवः॑ । प॒शवोऽङ्गा॑नि । अङ्गा॑नि॒ प्र । प्र ह॑रन्ति । ह॒र॒न्ति॒ प्रति॑ष्ठित्यै । प्रति॑ष्ठित्या॒ इति॒ प्रति॑ - स्थि॒त्यै॒ । \newline

\textbf{Jatai Paata} \newline

1. अ॒मुष्मि॑न् ॅलो॒के लो॒के॑ ऽमुष्मि॑न् न॒मुष्मि॑न् ॅलो॒के । \newline
2. लो॒क उपोप॑ लो॒के लो॒क उप॑ । \newline
3. उप॑ तिष्ठेते तिष्ठेते॒ उपोप॑ तिष्ठेते । \newline
4. ति॒ष्ठे॒ते॒ वि॒राजि॑ वि॒राजि॑ तिष्ठेते तिष्ठेते वि॒राजि॑ । \newline
5. ति॒ष्ठे॒ते॒ इति॑ तिष्ठेते । \newline
6. वि॒राज्य॒ग्नि र॒ग्निर् वि॒राजि॑ वि॒राज्य॒ग्निः । \newline
7. वि॒राजीति॑ वि - राजि॑ । \newline
8. अ॒ग्नि श्चे॑त॒व्य॑ श्चेत॒व्यो᳚ ऽग्नि र॒ग्नि श्चे॑त॒व्यः॑ । \newline
9. चे॒त॒व्य॑ इतीति॑ चेत॒व्य॑ श्चेत॒व्य॑ इति॑ । \newline
10. इत्या॑हु राहु॒ रिती त्या॑हुः । \newline
11. आ॒हुः॒ स्रुख् स्रुगा॑हु राहुः॒ स्रुक् । \newline
12. स्रुग् वै वै स्रुख् स्रुग् वै । \newline
13. वै वि॒राड् वि॒राड् वै वै वि॒राट् । \newline
14. वि॒राड् यद् यद् वि॒राड् वि॒राड् यत् । \newline
15. वि॒राडिति॑ वि - राट् । \newline
16. यथ् स्रुचौ॒ स्रुचौ॒ यद् यथ् स्रुचौ᳚ । \newline
17. स्रुचा॑ वुप॒दधा᳚ त्युप॒दधा॑ति॒ स्रुचौ॒ स्रुचा॑ वुप॒दधा॑ति । \newline
18. उ॒प॒दधा॑ति वि॒राजि॑ वि॒राज्यु॑प॒दधा᳚ त्युप॒दधा॑ति वि॒राजि॑ । \newline
19. उ॒प॒दधा॒तीत्यु॑प - दधा॑ति । \newline
20. वि॒रा ज्ये॒वैव वि॒राजि॑ वि॒रा ज्ये॒व । \newline
21. वि॒राजीति॑ वि - राजि॑ । \newline
22. ए॒वाग्नि म॒ग्नि मे॒वै वाग्निम् । \newline
23. अ॒ग्निम् चि॑नुते चिनुते॒ ऽग्नि म॒ग्निम् चि॑नुते । \newline
24. चि॒नु॒ते॒ य॒ज्ञ्॒मु॒खेय॑ज्ञ्मुखे यज्ञ्मु॒खेय॑ज्ञ्मुखे चिनुते चिनुते यज्ञ्मु॒खेय॑ज्ञ्मुखे । \newline
25. य॒ज्ञ्॒मु॒खेय॑ज्ञ्मुखे॒ वै वै य॑ज्ञ्मु॒खेय॑ज्ञ्मुखे यज्ञ्मु॒खेय॑ज्ञ्मुखे॒ वै । \newline
26. य॒ज्ञ्॒मु॒खेय॑ज्ञ्मुख॒ इति॑ यज्ञ्मु॒खे - य॒ज्ञ्॒मु॒खे॒ । \newline
27. वै क्रि॒यमा॑णे क्रि॒यमा॑णे॒ वै वै क्रि॒यमा॑णे । \newline
28. क्रि॒यमा॑णे य॒ज्ञ्ं ॅय॒ज्ञ्म् क्रि॒यमा॑णे क्रि॒यमा॑णे य॒ज्ञ्म् । \newline
29. य॒ज्ञ्ꣳ रक्षाꣳ॑सि॒ रक्षाꣳ॑सि य॒ज्ञ्ं ॅय॒ज्ञ्ꣳ रक्षाꣳ॑सि । \newline
30. रक्षाꣳ॑सि जिघाꣳसन्ति जिघाꣳसन्ति॒ रक्षाꣳ॑सि॒ रक्षाꣳ॑सि जिघाꣳसन्ति । \newline
31. जि॒घाꣳ॒॒स॒न्ति॒ य॒ज्ञ्॒मु॒खं ॅय॑ज्ञ्मु॒खम् जि॑घाꣳसन्ति जिघाꣳसन्ति यज्ञ्मु॒खम् । \newline
32. य॒ज्ञ्॒मु॒खꣳ रु॒क्मो रु॒क्मो य॑ज्ञ्मु॒खं ॅय॑ज्ञ्मु॒खꣳ रु॒क्मः । \newline
33. य॒ज्ञ्॒मु॒खमिति॑ यज्ञ् - मु॒खम् । \newline
34. रु॒क्मो यद् यद् रु॒क्मो रु॒क्मो यत् । \newline
35. यद् रु॒क्मꣳ रु॒क्मं ॅयद् यद् रु॒क्मम् । \newline
36. रु॒क्मं ॅव्या॑घा॒रय॑ति व्याघा॒रय॑ति रु॒क्मꣳ रु॒क्मं ॅव्या॑घा॒रय॑ति । \newline
37. व्या॒घा॒रय॑ति यज्ञ्मु॒खाद् य॑ज्ञ्मु॒खाद् व्या॑घा॒रय॑ति व्याघा॒रय॑ति यज्ञ्मु॒खात् । \newline
38. व्या॒घा॒रय॒तीति॑ वि - आ॒घा॒रय॑ति । \newline
39. य॒ज्ञ्॒मु॒खा दे॒वैव य॑ज्ञ्मु॒खाद् य॑ज्ञ्मु॒खा दे॒व । \newline
40. य॒ज्ञ्॒मु॒खादिति॑ यज्ञ् - मु॒खात् । \newline
41. ए॒व रक्षाꣳ॑सि॒ रक्षाꣳ॑ स्ये॒वैव रक्षाꣳ॑सि । \newline
42. रक्षाꣳ॒॒ स्यपाप॒ रक्षाꣳ॑सि॒ रक्षाꣳ॒॒ स्यप॑ । \newline
43. अप॑ हन्ति ह॒न्त्य पाप॑ हन्ति । \newline
44. ह॒न्ति॒ प॒ञ्चभिः॑ प॒ञ्चभिर्॑. हन्ति हन्ति प॒ञ्चभिः॑ । \newline
45. प॒ञ्चभि॒र् व्याघा॑रयति॒ व्याघा॑रयति प॒ञ्चभिः॑ प॒ञ्चभि॒र् व्याघा॑रयति । \newline
46. प॒ञ्चभि॒रिति॑ प॒ञ्च - भिः॒ । \newline
47. व्याघा॑रयति॒ पाङ्क्तः॒ पाङ्क्तो॒ व्याघा॑रयति॒ व्याघा॑रयति॒ पाङ्क्तः॑ । \newline
48. व्याघा॑रय॒तीति॑ वि - आघा॑रयति । \newline
49. पाङ्क्तो॑ य॒ज्ञो य॒ज्ञ्ः पाङ्क्तः॒ पाङ्क्तो॑ य॒ज्ञ्ः । \newline
50. य॒ज्ञो यावा॒न्॒. यावान्॑. य॒ज्ञो य॒ज्ञो यावान्॑ । \newline
51. यावा॑ ने॒वैव यावा॒न्॒. यावा॑ ने॒व । \newline
52. ए॒व य॒ज्ञो य॒ज्ञ् ए॒वैव य॒ज्ञ्ः । \newline
53. य॒ज्ञ् स्तस्मा॒त् तस्मा᳚द् य॒ज्ञो य॒ज्ञ् स्तस्मा᳚त् । \newline
54. तस्मा॒द् रक्षाꣳ॑सि॒ रक्षाꣳ॑सि॒ तस्मा॒त् तस्मा॒द् रक्षाꣳ॑सि । \newline
55. रक्षाꣳ॒॒ स्यपाप॒ रक्षाꣳ॑सि॒ रक्षाꣳ॒॒ स्यप॑ । \newline
56. अप॑ हन्ति ह॒न्त्य पाप॑ हन्ति । \newline
57. ह॒न्त्य॒ क्ष्ण॒या ऽक्ष्ण॒या ह॑न्ति हन्त्य क्ष्ण॒या । \newline
58. अ॒क्ष्ण॒या व्याघा॑रयति॒ व्याघा॑रय त्यक्ष्ण॒या ऽक्ष्ण॒या व्याघा॑रयति । \newline
59. व्याघा॑रयति॒ तस्मा॒त् तस्मा॒द् व्याघा॑रयति॒ व्याघा॑रयति॒ तस्मा᳚त् । \newline
60. व्याघा॑रय॒तीति॑ वि - आघा॑रयति । \newline
61. तस्मा॑ दक्ष्ण॒या ऽक्ष्ण॒या तस्मा॒त् तस्मा॑ दक्ष्ण॒या । \newline
62. अ॒क्ष्ण॒या प॒शवः॑ प॒शवो᳚ ऽक्ष्ण॒या ऽक्ष्ण॒या प॒शवः॑ । \newline
63. प॒शवो ऽङ्गा॒ न्यङ्गा॑नि प॒शवः॑ प॒शवो ऽङ्गा॑नि । \newline
64. अङ्गा॑नि॒ प्र प्राङ्गा॒ न्यङ्गा॑नि॒ प्र । \newline
65. प्र ह॑रन्ति हरन्ति॒ प्र प्र ह॑रन्ति । \newline
66. ह॒र॒न्ति॒ प्रति॑ष्ठित्यै॒ प्रति॑ष्ठित्यै हरन्ति हरन्ति॒ प्रति॑ष्ठित्यै । \newline
67. प्रति॑ष्ठित्या॒ इति॒ प्रति॑ - स्थि॒त्यै॒ । \newline

\textbf{Ghana Paata } \newline

1. अ॒मुष्मि॑न् ॅलो॒के लो॒के॑ ऽमुष्मि॑न् न॒मुष्मि॑न् ॅलो॒क उपोप॑ लो॒के॑ ऽमुष्मि॑न् न॒मुष्मि॑न् ॅलो॒क उप॑ । \newline
2. लो॒क उपोप॑ लो॒के लो॒क उप॑ तिष्ठेते तिष्ठेते॒ उप॑ लो॒के लो॒क उप॑ तिष्ठेते । \newline
3. उप॑ तिष्ठेते तिष्ठेते॒ उपोप॑ तिष्ठेते वि॒राजि॑ वि॒राजि॑ तिष्ठेते॒ उपोप॑ तिष्ठेते वि॒राजि॑ । \newline
4. ति॒ष्ठे॒ते॒ वि॒राजि॑ वि॒राजि॑ तिष्ठेते तिष्ठेते वि॒राज्य॒ग्नि र॒ग्निर् वि॒राजि॑ तिष्ठेते तिष्ठेते वि॒राज्य॒ग्निः । \newline
5. ति॒ष्ठे॒ते॒ इति॑ तिष्ठेते । \newline
6. वि॒राज्य॒ग्नि र॒ग्निर् वि॒राजि॑ वि॒राज्य॒ग्नि श्चे॑त॒व्य॑ श्चेत॒व्यो᳚ ऽग्निर् वि॒राजि॑ वि॒राज्य॒ग्नि श्चे॑त॒व्यः॑ । \newline
7. वि॒राजीति॑ वि - राजि॑ । \newline
8. अ॒ग्नि श्चे॑त॒व्य॑ श्चेत॒व्यो᳚ ऽग्नि र॒ग्नि श्चे॑त॒व्य॑ इतीति॑ चेत॒व्यो᳚ ऽग्नि र॒ग्नि श्चे॑त॒व्य॑ इति॑ । \newline
9. चे॒त॒व्य॑ इतीति॑ चेत॒व्य॑ श्चेत॒व्य॑ इत्या॑हु राहु॒रिति॑ चेत॒व्य॑ श्चेत॒व्य॑ इत्या॑हुः । \newline
10. इत्या॑हु राहु॒रितीत्या॑हुः॒ स्रुख् स्रुगा॑हु॒ रितीत्या॑हुः॒ स्रुक् । \newline
11. आ॒हुः॒ स्रुख् स्रुगा॑हु राहुः॒ स्रुग् वै वै स्रुगा॑हु राहुः॒ स्रुग् वै । \newline
12. स्रुग् वै वै स्रुख् स्रुग् वै वि॒राड् वि॒राड् वै स्रुख् स्रुग् वै वि॒राट् । \newline
13. वै वि॒राड् वि॒राड् वै वै वि॒राड् यद् यद् वि॒राड् वै वै वि॒राड् यत् । \newline
14. वि॒राड् यद् यद् वि॒राड् वि॒राड् यथ् स्रुचौ॒ स्रुचौ॒ यद् वि॒राड् वि॒राड् यथ् स्रुचौ᳚ । \newline
15. वि॒राडिति॑ वि - राट् । \newline
16. यथ् स्रुचौ॒ स्रुचौ॒ यद् यथ् स्रुचा॑ वुप॒दधा᳚ त्युप॒दधा॑ति॒ स्रुचौ॒ यद् यथ् स्रुचा॑ वुप॒दधा॑ति । \newline
17. स्रुचा॑ वुप॒दधा᳚ त्युप॒दधा॑ति॒ स्रुचौ॒ स्रुचा॑ वुप॒दधा॑ति वि॒राजि॑ वि॒रा ज्यु॑प॒दधा॑ति॒ स्रुचौ॒ स्रुचा॑ वुप॒दधा॑ति वि॒राजि॑ । \newline
18. उ॒प॒दधा॑ति वि॒राजि॑ वि॒रा ज्यु॑प॒दधा᳚ त्युप॒दधा॑ति वि॒राज्ये॒वैव वि॒रा ज्यु॑प॒दधा᳚ त्युप॒दधा॑ति वि॒राज्ये॒व । \newline
19. उ॒प॒दधा॒तीत्यु॑प - दधा॑ति । \newline
20. वि॒रा ज्ये॒वैव वि॒राजि॑ वि॒राज्ये॒वाग्नि म॒ग्नि मे॒व वि॒राजि॑ वि॒राज्ये॒वाग्निम् । \newline
21. वि॒राजीति॑ वि - राजि॑ । \newline
22. ए॒वाग्नि म॒ग्नि मे॒वैवाग्निम् चि॑नुते चिनुते॒ ऽग्नि मे॒वैवाग्निम् चि॑नुते । \newline
23. अ॒ग्निम् चि॑नुते चिनुते॒ ऽग्नि म॒ग्निम् चि॑नुते यज्ञ्मु॒खेय॑ज्ञ्मुखे यज्ञ्मु॒खेय॑ज्ञ्मुखे चिनुते॒ ऽग्नि म॒ग्निम् चि॑नुते यज्ञ्मु॒खेय॑ज्ञ्मुखे । \newline
24. चि॒नु॒ते॒ य॒ज्ञ्॒मु॒खेय॑ज्ञ्मुखे यज्ञ्मु॒खेय॑ज्ञ्मुखे चिनुते चिनुते यज्ञ्मु॒खेय॑ज्ञ्मुखे॒ वै वै य॑ज्ञ्मु॒खेय॑ज्ञ्मुखे चिनुते चिनुते यज्ञ्मु॒खेय॑ज्ञ्मुखे॒ वै । \newline
25. य॒ज्ञ्॒मु॒खेय॑ज्ञ्मुखे॒ वै वै य॑ज्ञ्मु॒खेय॑ज्ञ्मुखे यज्ञ्मु॒खेय॑ज्ञ्मुखे॒ वै क्रि॒यमा॑णे क्रि॒यमा॑णे॒ वै य॑ज्ञ्मु॒खेय॑ज्ञ्मुखे यज्ञ्मु॒खेय॑ज्ञ्मुखे॒ वै क्रि॒यमा॑णे । \newline
26. य॒ज्ञ्॒मु॒खेय॑ज्ञ्मुख॒ इति॑ यज्ञ्मु॒खे - य॒ज्ञ्॒मु॒खे॒ । \newline
27. वै क्रि॒यमा॑णे क्रि॒यमा॑णे॒ वै वै क्रि॒यमा॑णे य॒ज्ञ्ं ॅय॒ज्ञ्म् क्रि॒यमा॑णे॒ वै वै क्रि॒यमा॑णे य॒ज्ञ्म् । \newline
28. क्रि॒यमा॑णे य॒ज्ञ्ं ॅय॒ज्ञ्म् क्रि॒यमा॑णे क्रि॒यमा॑णे य॒ज्ञ्ꣳ रक्षाꣳ॑सि॒ रक्षाꣳ॑सि य॒ज्ञ्म् क्रि॒यमा॑णे क्रि॒यमा॑णे य॒ज्ञ्ꣳ रक्षाꣳ॑सि । \newline
29. य॒ज्ञ्ꣳ रक्षाꣳ॑सि॒ रक्षाꣳ॑सि य॒ज्ञ्ं ॅय॒ज्ञ्ꣳ रक्षाꣳ॑सि जिघाꣳसन्ति जिघाꣳसन्ति॒ रक्षाꣳ॑सि य॒ज्ञ्ं ॅय॒ज्ञ्ꣳ रक्षाꣳ॑सि जिघाꣳसन्ति । \newline
30. रक्षाꣳ॑सि जिघाꣳसन्ति जिघाꣳसन्ति॒ रक्षाꣳ॑सि॒ रक्षाꣳ॑सि जिघाꣳसन्ति यज्ञ्मु॒खं ॅय॑ज्ञ्मु॒खम् जि॑घाꣳसन्ति॒ रक्षाꣳ॑सि॒ रक्षाꣳ॑सि जिघाꣳसन्ति यज्ञ्मु॒खम् । \newline
31. जि॒घाꣳ॒॒स॒न्ति॒ य॒ज्ञ्॒मु॒खं ॅय॑ज्ञ्मु॒खम् जि॑घाꣳसन्ति जिघाꣳसन्ति यज्ञ्मु॒खꣳ रु॒क्मो रु॒क्मो य॑ज्ञ्मु॒खम् जि॑घाꣳसन्ति जिघाꣳसन्ति यज्ञ्मु॒खꣳ रु॒क्मः । \newline
32. य॒ज्ञ्॒मु॒खꣳ रु॒क्मो रु॒क्मो य॑ज्ञ्मु॒खं ॅय॑ज्ञ्मु॒खꣳ रु॒क्मो यद् यद् रु॒क्मो य॑ज्ञ्मु॒खं ॅय॑ज्ञ्मु॒खꣳ रु॒क्मो यत् । \newline
33. य॒ज्ञ्॒मु॒खमिति॑ यज्ञ् - मु॒खम् । \newline
34. रु॒क्मो यद् यद् रु॒क्मो रु॒क्मो यद् रु॒क्मꣳ रु॒क्मं ॅयद् रु॒क्मो रु॒क्मो यद् रु॒क्मम् । \newline
35. यद् रु॒क्मꣳ रु॒क्मं ॅयद् यद् रु॒क्मं ॅव्या॑घा॒रय॑ति व्याघा॒रय॑ति रु॒क्मं ॅयद् यद् रु॒क्मं ॅव्या॑घा॒रय॑ति । \newline
36. रु॒क्मं ॅव्या॑घा॒रय॑ति व्याघा॒रय॑ति रु॒क्मꣳ रु॒क्मं ॅव्या॑घा॒रय॑ति यज्ञ्मु॒खाद् य॑ज्ञ्मु॒खाद् व्या॑घा॒रय॑ति रु॒क्मꣳ रु॒क्मं ॅव्या॑घा॒रय॑ति यज्ञ्मु॒खात् । \newline
37. व्या॒घा॒रय॑ति यज्ञ्मु॒खाद् य॑ज्ञ्मु॒खाद् व्या॑घा॒रय॑ति व्याघा॒रय॑ति यज्ञ्मु॒खा दे॒वैव य॑ज्ञ्मु॒खाद् व्या॑घा॒रय॑ति व्याघा॒रय॑ति यज्ञ्मु॒खा दे॒व । \newline
38. व्या॒घा॒रय॒तीति॑ वि - आ॒घा॒रय॑ति । \newline
39. य॒ज्ञ्॒मु॒खा दे॒वैव य॑ज्ञ्मु॒खाद् य॑ज्ञ्मु॒खादे॒व रक्षाꣳ॑सि॒ रक्षाꣳ॑स्ये॒व य॑ज्ञ्मु॒खाद् य॑ज्ञ्मु॒खादे॒व रक्षाꣳ॑सि । \newline
40. य॒ज्ञ्॒मु॒खादिति॑ यज्ञ् - मु॒खात् । \newline
41. ए॒व रक्षाꣳ॑सि॒ रक्षाꣳ॑ स्ये॒वैव रक्षाꣳ॒॒ स्यपाप॒ रक्षाꣳ॑ स्ये॒वैव रक्षाꣳ॒॒स्यप॑ । \newline
42. रक्षाꣳ॒॒ स्यपाप॒ रक्षाꣳ॑सि॒ रक्षाꣳ॒॒स्यप॑ हन्ति ह॒न्त्यप॒ रक्षाꣳ॑सि॒ रक्षाꣳ॒॒स्यप॑ हन्ति । \newline
43. अप॑ हन्ति ह॒न्त्यपाप॑ हन्ति प॒ञ्चभिः॑ प॒ञ्चभिर्॑. ह॒न्त्यपाप॑ हन्ति प॒ञ्चभिः॑ । \newline
44. ह॒न्ति॒ प॒ञ्चभिः॑ प॒ञ्चभिर्॑. हन्ति हन्ति प॒ञ्चभि॒र् व्याघा॑रयति॒ व्याघा॑रयति प॒ञ्चभिर्॑. हन्ति हन्ति प॒ञ्चभि॒र् व्याघा॑रयति । \newline
45. प॒ञ्चभि॒र् व्याघा॑रयति॒ व्याघा॑रयति प॒ञ्चभिः॑ प॒ञ्चभि॒र् व्याघा॑रयति॒ पाङ्क्तः॒ पाङ्क्तो॒ व्याघा॑रयति प॒ञ्चभिः॑ प॒ञ्चभि॒र् व्याघा॑रयति॒ पाङ्क्तः॑ । \newline
46. प॒ञ्चभि॒रिति॑ प॒ञ्च - भिः॒ । \newline
47. व्याघा॑रयति॒ पाङ्क्तः॒ पाङ्क्तो॒ व्याघा॑रयति॒ व्याघा॑रयति॒ पाङ्क्तो॑ य॒ज्ञो य॒ज्ञ्ः पाङ्क्तो॒ व्याघा॑रयति॒ व्याघा॑रयति॒ पाङ्क्तो॑ य॒ज्ञ्ः । \newline
48. व्याघा॑रय॒तीति॑ वि - आघा॑रयति । \newline
49. पाङ्क्तो॑ य॒ज्ञो य॒ज्ञ्ः पाङ्क्तः॒ पाङ्क्तो॑ य॒ज्ञो यावा॒न्॒. यावान्॑. य॒ज्ञ्ः पाङ्क्तः॒ पाङ्क्तो॑ य॒ज्ञो यावान्॑ । \newline
50. य॒ज्ञो यावा॒न्॒. यावान्॑. य॒ज्ञो य॒ज्ञो यावा॑ ने॒वैव यावान्॑. य॒ज्ञो य॒ज्ञो यावा॑ ने॒व । \newline
51. यावा॑ ने॒वैव यावा॒न्॒. यावा॑ ने॒व य॒ज्ञो य॒ज्ञ् ए॒व यावा॒न्॒. यावा॑ ने॒व य॒ज्ञ्ः । \newline
52. ए॒व य॒ज्ञो य॒ज्ञ् ए॒वैव य॒ज्ञ् स्तस्मा॒त् तस्मा᳚द् य॒ज्ञ् ए॒वैव य॒ज्ञ् स्तस्मा᳚त् । \newline
53. य॒ज्ञ् स्तस्मा॒त् तस्मा᳚द् य॒ज्ञो य॒ज्ञ् स्तस्मा॒द् रक्षाꣳ॑सि॒ रक्षाꣳ॑सि॒ तस्मा᳚द् य॒ज्ञो य॒ज्ञ् स्तस्मा॒द् रक्षाꣳ॑सि । \newline
54. तस्मा॒द् रक्षाꣳ॑सि॒ रक्षाꣳ॑सि॒ तस्मा॒त् तस्मा॒द् रक्षाꣳ॒॒ स्यपाप॒ रक्षाꣳ॑सि॒ तस्मा॒त् तस्मा॒द् रक्षाꣳ॒॒स्यप॑ । \newline
55. रक्षाꣳ॒॒ स्यपाप॒ रक्षाꣳ॑सि॒ रक्षाꣳ॒॒स्यप॑ हन्ति ह॒न्त्यप॒ रक्षाꣳ॑सि॒ रक्षाꣳ॒॒स्यप॑ हन्ति । \newline
56. अप॑ हन्ति ह॒न्त्यपाप॑ हन्त्यक्ष्ण॒या ऽक्ष्ण॒या ह॒न्त्यपाप॑ हन्त्यक्ष्ण॒या । \newline
57. ह॒न्त्य॒क्ष्ण॒या ऽक्ष्ण॒या ह॑न्ति हन्त्यक्ष्ण॒या व्याघा॑रयति॒ व्याघा॑रय त्यक्ष्ण॒या ह॑न्ति हन्त्यक्ष्ण॒या व्याघा॑रयति । \newline
58. अ॒क्ष्ण॒या व्याघा॑रयति॒ व्याघा॑रय त्यक्ष्ण॒या ऽक्ष्ण॒या व्याघा॑रयति॒ तस्मा॒त् तस्मा॒द् व्याघा॑रय त्यक्ष्ण॒या ऽक्ष्ण॒या व्याघा॑रयति॒ तस्मा᳚त् । \newline
59. व्याघा॑रयति॒ तस्मा॒त् तस्मा॒द् व्याघा॑रयति॒ व्याघा॑रयति॒ तस्मा॑ दक्ष्ण॒या ऽक्ष्ण॒या तस्मा॒द् व्याघा॑रयति॒ व्याघा॑रयति॒ तस्मा॑ दक्ष्ण॒या । \newline
60. व्याघा॑रय॒तीति॑ वि - आघा॑रयति । \newline
61. तस्मा॑ दक्ष्ण॒या ऽक्ष्ण॒या तस्मा॒त् तस्मा॑ दक्ष्ण॒या प॒शवः॑ प॒शवो᳚ ऽक्ष्ण॒या तस्मा॒त् तस्मा॑ दक्ष्ण॒या प॒शवः॑ । \newline
62. अ॒क्ष्ण॒या प॒शवः॑ प॒शवो᳚ ऽक्ष्ण॒या ऽक्ष्ण॒या प॒शवो ऽङ्गा॒ न्यङ्गा॑नि प॒शवो᳚ ऽक्ष्ण॒या ऽक्ष्ण॒या प॒शवो ऽङ्गा॑नि । \newline
63. प॒शवो ऽङ्गा॒ न्यङ्गा॑नि प॒शवः॑ प॒शवो ऽङ्गा॑नि॒ प्र प्राङ्गा॑नि प॒शवः॑ प॒शवो ऽङ्गा॑नि॒ प्र । \newline
64. अङ्गा॑नि॒ प्र प्राङ्गा॒ न्यङ्गा॑नि॒ प्र ह॑रन्ति हरन्ति॒ प्राङ्गा॒ न्यङ्गा॑नि॒ प्र ह॑रन्ति । \newline
65. प्र ह॑रन्ति हरन्ति॒ प्र प्र ह॑रन्ति॒ प्रति॑ष्ठित्यै॒ प्रति॑ष्ठित्यै हरन्ति॒ प्र प्र ह॑रन्ति॒ प्रति॑ष्ठित्यै । \newline
66. ह॒र॒न्ति॒ प्रति॑ष्ठित्यै॒ प्रति॑ष्ठित्यै हरन्ति हरन्ति॒ प्रति॑ष्ठित्यै । \newline
67. प्रति॑ष्ठित्या॒ इति॒ प्रति॑ - स्थि॒त्यै॒ । \newline
\pagebreak
\markright{ TS 5.2.8.1  \hfill https://www.vedavms.in \hfill}

\section{ TS 5.2.8.1 }

\textbf{TS 5.2.8.1 } \newline
\textbf{Samhita Paata} \newline

स्व॒य॒मा॒तृ॒ण्णामुप॑ दधाती॒यं ॅवै स्व॑यमातृ॒ण्णेमामे॒वोप॑ ध॒त्ते ऽश्व॒मुप॑ घ्रापयति प्रा॒णमे॒वास्यां᳚ दधा॒त्यथो᳚ प्राजाप॒त्यो वा अश्वः॑ प्र॒जाप॑तिनै॒वाऽग्निं चि॑नुते प्रथ॒मेष्ट॑कोपधी॒यमा॑ना पशू॒नां च॒ यज॑मानस्य च प्रा॒णमपि॑ दधाति स्वयमातृ॒ण्णा भ॑वति प्रा॒णाना॒मुथ्सृ॑ष्ट्या॒ अथो॑ सुव॒र्गस्य॑ लो॒कस्यानु॑ख्यात्या अ॒ग्नाव॒ग्निश्चे॑त॒व्य॑ इत्या॑हुरे॒ष वा - [  ] \newline

\textbf{Pada Paata} \newline

स्व॒य॒मा॒तृ॒ण्णामिति॑ स्वयं - आ॒तृ॒ण्णाम् । उपेति॑ । द॒धा॒ति॒ । इ॒यम् । वै । स्व॒य॒मा॒तृ॒ण्णेति॑ स्वयं-आ॒तृ॒ण्णा । इ॒माम् । ए॒व । उपेति॑ । ध॒त्ते॒ । अश्व᳚म् । उपेति॑ । घ्रा॒प॒य॒ति॒ । प्रा॒णमिति॑ प्र-अ॒नम् । ए॒व । अ॒स्या॒म् । द॒धा॒ति॒ । अथो॒ इति॑ । प्रा॒जा॒प॒त्य इति॑ प्रजा - प॒त्यः । वै । अश्वः॑ । प्र॒जाप॑ति॒नेति॑ प्र॒जा - प॒ति॒ना॒ । ए॒व । अ॒ग्निम् । चि॒नु॒ते॒ । प्र॒थ॒मा । इष्ट॑का । उ॒प॒धी॒यमा॒नेत्यु॑प-धी॒यमा॑ना । प॒शू॒नाम् । च॒ । यज॑मानस्य । च॒ । प्रा॒णमिति॑ प्र - अ॒नम् । अपीति॑ । द॒धा॒ति॒ । स्व॒य॒मा॒तृ॒ण्णेति॑ स्वयं - आ॒तृ॒ण्णा । भ॒व॒ति॒ । प्रा॒णाना॒मिति॑ प्र - अ॒नाना᳚म् । उथ्सृ॑ष्ट्या॒ इत्युत् - सृ॒ष्ट्यै॒ । अथो॒ इति॑ । सु॒व॒र्गस्येति॑ सुवः-गस्य॑ । लो॒कस्य॑ । अनु॑ख्यात्या॒ इत्यनु॑ - ख्या॒त्यै॒ । अ॒ग्नौ । अ॒ग्निः । चे॒त॒व्यः॑ । इति॑ । आ॒हुः॒ । ए॒षः । वै ।  \newline


\textbf{Krama Paata} \newline

स्व॒य॒मा॒तृ॒ण्णामुप॑ । स्व॒य॒मा॒तृ॒ण्णामिति॑ स्वयम् - आ॒तृ॒ण्णाम् । उप॑ दधाति । द॒धा॒ती॒यम् । इ॒यम् ॅवै । वै स्व॑यमातृ॒ण्णा । स्व॒य॒मा॒तृ॒ण्णेमाम् । स्व॒य॒मा॒तृ॒ण्णेति॑ स्वयम् - आ॒तृ॒ण्णा । इ॒मामे॒व । ए॒वोप॑ । उप॑ धत्ते । ध॒त्तेऽश्व᳚म् । अश्व॒मुप॑ । उप॑ घ्रापयति । घ्रा॒प॒य॒ति॒ प्रा॒णम् । प्रा॒णमे॒व । प्रा॒णमिति॑ प्र - अ॒नम् । ए॒वास्या᳚म् । अ॒स्या॒म् द॒धा॒ति॒ । द॒धा॒त्यथो᳚ । अथो᳚ प्राजाप॒त्यः । अथो॒ इत्यथो᳚ । प्रा॒जा॒प॒त्यो वै । प्रा॒जा॒प॒त्य इति॑ प्राजा - प॒त्यः । वा अश्वः॑ । अश्वः॑ प्र॒जाप॑तिना । प्र॒जाप॑तिनै॒व । प्र॒जाप॑ति॒नेति॑ प्र॒जा - प॒ति॒ना॒ । ए॒वाग्निम् । अ॒ग्निम् चि॑नुते । चि॒नु॒ते॒ प्र॒थ॒मा । प्र॒थ॒मेष्ट॑का । इष्ट॑कोपधी॒यमा॑ना । उ॒प॒धी॒यमा॑ना पशू॒नाम् । उ॒प॒धी॒यमा॒नेत्यु॑प - धी॒यमा॑ना । प॒शू॒नाम् च॑ । च॒ यज॑मानस्य । यज॑मानस्य च । च॒ प्रा॒णम् । प्रा॒णमपि॑ । प्रा॒णमिति॑ प्र - अ॒नम् । अपि॑ दधाति । द॒धा॒ति॒ स्व॒य॒मा॒तृ॒ण्णा । स्व॒य॒मा॒तृ॒ण्णा भ॑वति । स्व॒य॒मा॒तृ॒ण्णेति॑ स्वयम् - आ॒तृ॒ण्णा । भ॒व॒ति॒ प्रा॒णाना᳚म् । प्रा॒णाना॒मुथ्सृ॑ष्ट्यै । प्रा॒णाना॒मिति॑ प्र - अ॒नाना᳚म् । उथ्सृ॑ष्ट्या॒ अथो᳚ । उथ्सृ॑ष्ट्या॒ इत्युत् - सृ॒ष्ट्यै॒ । अथो॑ सुव॒र्गस्य॑ । अथो॒ इत्यथो᳚ । सु॒व॒र्गस्य॑ लो॒कस्य॑ । सु॒व॒र्गस्येति॑ सुवः - गस्य॑ । लो॒कस्यानु॑ख्यात्यै । अनु॑ख्यात्या अ॒ग्नौ । अनु॑ख्यात्या॒ इत्यनु॑ - ख्या॒त्यै॒ । अ॒ग्नाव॒ग्निः । अ॒ग्निश्चे॑त॒व्यः॑ । चे॒त॒व्य॑ इति॑ । इत्या॑हुः । आ॒हु॒रे॒षः । ए॒ष वै । वा अ॒ग्निः \newline

\textbf{Jatai Paata} \newline

1. स्व॒य॒मा॒तृ॒ण्णा मुपोप॑ स्वयमातृ॒ण्णाꣳ स्व॑यमातृ॒ण्णा मुप॑ । \newline
2. स्व॒य॒मा॒तृ॒ण्णामिति॑ स्वयं - आ॒तृ॒ण्णाम् । \newline
3. उप॑ दधाति दधा॒ त्युपोप॑ दधाति । \newline
4. द॒धा॒ती॒य मि॒यम् द॑धाति दधाती॒यम् । \newline
5. इ॒यं ॅवै वा इ॒य मि॒यं ॅवै । \newline
6. वै स्व॑यमातृ॒ण्णा स्व॑यमातृ॒ण्णा वै वै स्व॑यमातृ॒ण्णा । \newline
7. स्व॒य॒मा॒तृ॒ण्णेमा मि॒माꣳ स्व॑यमातृ॒ण्णा स्व॑यमातृ॒ण्णेमाम् । \newline
8. स्व॒य॒मा॒तृ॒ण्णेति॑ स्वयं - आ॒तृ॒ण्णा । \newline
9. इ॒मा मे॒वैवे मा मि॒मा मे॒व । \newline
10. ए॒वो पोपै॒ वैवोप॑ । \newline
11. उप॑ धत्ते धत्त॒ उपोप॑ धत्ते । \newline
12. ध॒त्ते ऽश्व॒ मश्व॑म् धत्ते ध॒त्ते ऽश्व᳚म् । \newline
13. अश्व॒ मुपोपाश्व॒ मश्व॒ मुप॑ । \newline
14. उप॑ घ्रापयति घ्रापय॒ त्युपोप॑ घ्रापयति । \newline
15. घ्रा॒प॒य॒ति॒ प्रा॒णम् प्रा॒णम् घ्रा॑पयति घ्रापयति प्रा॒णम् । \newline
16. प्रा॒ण मे॒वैव प्रा॒णम् प्रा॒ण मे॒व । \newline
17. प्रा॒णमिति॑ प्र - अ॒नम् । \newline
18. ए॒वास्या॑ मस्या मे॒वै वास्या᳚म् । \newline
19. अ॒स्या॒म् द॒धा॒ति॒ द॒धा॒ त्य॒स्या॒ म॒स्या॒म् द॒धा॒ति॒ । \newline
20. द॒धा॒ त्यथो॒ अथो॑ दधाति दधा॒ त्यथो᳚ । \newline
21. अथो᳚ प्राजाप॒त्यः प्रा॑जाप॒त्यो ऽथो॒ अथो᳚ प्राजाप॒त्यः । \newline
22. अथो॒ इत्यथो᳚ । \newline
23. प्रा॒जा॒प॒त्यो वै वै प्रा॑जाप॒त्यः प्रा॑जाप॒त्यो वै । \newline
24. प्रा॒जा॒प॒त्य इति॑ प्राजा - प॒त्यः । \newline
25. वा अश्वो ऽश्वो॒ वै वा अश्वः॑ । \newline
26. अश्वः॑ प्र॒जाप॑तिना प्र॒जाप॑ति॒ना ऽश्वो ऽश्वः॑ प्र॒जाप॑तिना । \newline
27. प्र॒जाप॑ति नै॒वैव प्र॒जाप॑तिना प्र॒जाप॑तिनै॒व । \newline
28. प्र॒जाप॑ति॒नेति॑ प्र॒जा - प॒ति॒ना॒ । \newline
29. ए॒वाग्नि म॒ग्नि मे॒वै वाग्निम् । \newline
30. अ॒ग्निम् चि॑नुते चिनुते॒ ऽग्नि म॒ग्निम् चि॑नुते । \newline
31. चि॒नु॒ते॒ प्र॒थ॒मा प्र॑थ॒मा चि॑नुते चिनुते प्रथ॒मा । \newline
32. प्र॒थ॒मेष्ट॒ केष्ट॑का प्रथ॒मा प्र॑थ॒मे ष्ट॑का । \newline
33. इष्ट॑ कोपधी॒यमा॑ नोपधी॒यमा॒ नेष्ट॒केष्ट॑ कोपधी॒यमा॑ना । \newline
34. उ॒प॒धी॒यमा॑ना पशू॒नाम् प॑शू॒ना मु॑पधी॒यमा॑ नोपधी॒यमा॑ना पशू॒नाम् । \newline
35. उ॒प॒धी॒यमा॒नेत्यु॑प - धी॒यमा॑ना । \newline
36. प॒शू॒नाम् च॑ च पशू॒नाम् प॑शू॒नाम् च॑ । \newline
37. च॒ यज॑मानस्य॒ यज॑मानस्य च च॒ यज॑मानस्य । \newline
38. यज॑मानस्य च च॒ यज॑मानस्य॒ यज॑मानस्य च । \newline
39. च॒ प्रा॒णम् प्रा॒णम् च॑ च प्रा॒णम् । \newline
40. प्रा॒ण मप्यपि॑ प्रा॒णम् प्रा॒ण मपि॑ । \newline
41. प्रा॒णमिति॑ प्र - अ॒नम् । \newline
42. अपि॑ दधाति दधा॒ त्यप्यपि॑ दधाति । \newline
43. द॒धा॒ति॒ स्व॒य॒मा॒तृ॒ण्णा स्व॑यमातृ॒ण्णा द॑धाति दधाति स्वयमातृ॒ण्णा । \newline
44. स्व॒य॒मा॒तृ॒ण्णा भ॑वति भवति स्वयमातृ॒ण्णा स्व॑यमातृ॒ण्णा भ॑वति । \newline
45. स्व॒य॒मा॒तृ॒ण्णेति॑ स्वयं - आ॒तृ॒ण्णा । \newline
46. भ॒व॒ति॒ प्रा॒णाना᳚म् प्रा॒णाना᳚म् भवति भवति प्रा॒णाना᳚म् । \newline
47. प्रा॒णाना॒ मुथ्सृ॑ष्ट्या॒ उथ्सृ॑ष्ट्यै प्रा॒णाना᳚म् प्रा॒णाना॒ मुथ्सृ॑ष्ट्यै । \newline
48. प्रा॒णाना॒मिति॑ प्र - अ॒नाना᳚म् । \newline
49. उथ्सृ॑ष्ट्या॒ अथो॒ अथो॒ उथ्सृ॑ष्ट्या॒ उथ्सृ॑ष्ट्या॒ अथो᳚ । \newline
50. उथ्सृ॑ष्ट्या॒ इत्युत् - सृ॒ष्ट्यै॒ । \newline
51. अथो॑ सुव॒र्गस्य॑ सुव॒र्गस्याथो॒ अथो॑ सुव॒र्गस्य॑ । \newline
52. अथो॒ इत्यथो᳚ । \newline
53. सु॒व॒र्गस्य॑ लो॒कस्य॑ लो॒कस्य॑ सुव॒र्गस्य॑ सुव॒र्गस्य॑ लो॒कस्य॑ । \newline
54. सु॒व॒र्गस्येति॑ सुवः - गस्य॑ । \newline
55. लो॒कस्या नु॑ख्यात्या॒ अनु॑ख्यात्यै लो॒कस्य॑ लो॒कस्या नु॑ख्यात्यै । \newline
56. अनु॑ख्यात्या अ॒ग्ना व॒ग्ना वनु॑ख्यात्या॒ अनु॑ख्यात्या अ॒ग्नौ । \newline
57. अनु॑ख्यात्या॒ इत्यनु॑ - ख्या॒त्यै॒ । \newline
58. अ॒ग्ना व॒ग्नि र॒ग्नि र॒ग्ना व॒ग्ना व॒ग्निः । \newline
59. अ॒ग्नि श्चे॑त॒व्य॑ श्चेत॒व्यो᳚ ऽग्नि र॒ग्नि श्चे॑त॒व्यः॑ । \newline
60. चे॒त॒व्य॑ इतीति॑ चेत॒व्य॑ श्चेत॒व्य॑ इति॑ । \newline
61. इत्या॑हु राहु॒ रिती त्या॑हुः । \newline
62. आ॒हु॒ रे॒ष ए॒ष आ॑हु राहु रे॒षः । \newline
63. ए॒ष वै वा ए॒ष ए॒ष वै । \newline
64. वा अ॒ग्नि र॒ग्निर् वै वा अ॒ग्निः । \newline

\textbf{Ghana Paata } \newline

1. स्व॒य॒मा॒तृ॒ण्णा मुपोप॑ स्वयमातृ॒ण्णाꣳ स्व॑यमातृ॒ण्णा मुप॑ दधाति दधा॒त्युप॑ स्वयमातृ॒ण्णाꣳ स्व॑यमातृ॒ण्णा मुप॑ दधाति । \newline
2. स्व॒य॒मा॒तृ॒ण्णामिति॑ स्वयं - आ॒तृ॒ण्णाम् । \newline
3. उप॑ दधाति दधा॒ त्युपोप॑ दधाती॒य मि॒यम् द॑धा॒ त्युपोप॑ दधाती॒यम् । \newline
4. द॒धा॒ती॒य मि॒यम् द॑धाति दधाती॒यं ॅवै वा इ॒यम् द॑धाति दधाती॒यं ॅवै । \newline
5. इ॒यं ॅवै वा इ॒य मि॒यं ॅवै स्व॑यमातृ॒ण्णा स्व॑यमातृ॒ण्णा वा इ॒य मि॒यं ॅवै स्व॑यमातृ॒ण्णा । \newline
6. वै स्व॑यमातृ॒ण्णा स्व॑यमातृ॒ण्णा वै वै स्व॑यमातृ॒ण्णेमा मि॒माꣳ स्व॑यमातृ॒ण्णा वै वै स्व॑यमातृ॒ण्णेमाम् । \newline
7. स्व॒य॒मा॒तृ॒ण्णेमा मि॒माꣳ स्व॑यमातृ॒ण्णा स्व॑यमातृ॒ण्णेमा मे॒वैवे माꣳ स्व॑यमातृ॒ण्णा स्व॑यमातृ॒ण्णेमा मे॒व । \newline
8. स्व॒य॒मा॒तृ॒ण्णेति॑ स्वयं - आ॒तृ॒ण्णा । \newline
9. इ॒मा मे॒वैवेमा मि॒मा मे॒वोपो पै॒वेमा मि॒मा मे॒वोप॑ । \newline
10. ए॒वोपो पै॒वैवोप॑ धत्ते धत्त॒ उपै॒वैवोप॑ धत्ते । \newline
11. उप॑ धत्ते धत्त॒ उपोप॑ ध॒त्ते ऽश्व॒ मश्व॑म् धत्त॒ उपोप॑ ध॒त्ते ऽश्व᳚म् । \newline
12. ध॒त्ते ऽश्व॒ मश्व॑म् धत्ते ध॒त्ते ऽश्व॒ मुपोपाश्व॑म् धत्ते ध॒त्ते ऽश्व॒ मुप॑ । \newline
13. अश्व॒ मुपोपाश्व॒ मश्व॒ मुप॑ घ्रापयति घ्रापय॒ त्युपाश्व॒ मश्व॒ मुप॑ घ्रापयति । \newline
14. उप॑ घ्रापयति घ्रापय॒ त्युपोप॑ घ्रापयति प्रा॒णम् प्रा॒णम् घ्रा॑पय॒ त्युपोप॑ घ्रापयति प्रा॒णम् । \newline
15. घ्रा॒प॒य॒ति॒ प्रा॒णम् प्रा॒णम् घ्रा॑पयति घ्रापयति प्रा॒ण मे॒वैव प्रा॒णम् घ्रा॑पयति घ्रापयति प्रा॒ण मे॒व । \newline
16. प्रा॒ण मे॒वैव प्रा॒णम् प्रा॒ण मे॒वास्या॑ मस्या मे॒व प्रा॒णम् प्रा॒ण मे॒वास्या᳚म् । \newline
17. प्रा॒णमिति॑ प्र - अ॒नम् । \newline
18. ए॒वास्या॑ मस्या मे॒वैवास्या᳚म् दधाति दधा त्यस्या मे॒वैवास्या᳚म् दधाति । \newline
19. अ॒स्या॒म् द॒धा॒ति॒ द॒धा॒ त्य॒स्या॒ म॒स्या॒म् द॒धा॒त्यथो॒ अथो॑ दधा त्यस्या मस्याम् दधा॒ त्यथो᳚ । \newline
20. द॒धा॒ त्यथो॒ अथो॑ दधाति दधा॒ त्यथो᳚ प्राजाप॒त्यः प्रा॑जाप॒त्यो ऽथो॑ दधाति दधा॒ त्यथो᳚ प्राजाप॒त्यः । \newline
21. अथो᳚ प्राजाप॒त्यः प्रा॑जाप॒त्यो ऽथो॒ अथो᳚ प्राजाप॒त्यो वै वै प्रा॑जाप॒त्यो ऽथो॒ अथो᳚ प्राजाप॒त्यो वै । \newline
22. अथो॒ इत्यथो᳚ । \newline
23. प्रा॒जा॒प॒त्यो वै वै प्रा॑जाप॒त्यः प्रा॑जाप॒त्यो वा अश्वो ऽश्वो॒ वै प्रा॑जाप॒त्यः प्रा॑जाप॒त्यो वा अश्वः॑ । \newline
24. प्रा॒जा॒प॒त्य इति॑ प्राजा - प॒त्यः । \newline
25. वा अश्वो ऽश्वो॒ वै वा अश्वः॑ प्र॒जाप॑तिना प्र॒जाप॑ति॒ना ऽश्वो॒ वै वा अश्वः॑ प्र॒जाप॑तिना । \newline
26. अश्वः॑ प्र॒जाप॑तिना प्र॒जाप॑ति॒ना ऽश्वो ऽश्वः॑ प्र॒जाप॑ति नै॒वैव प्र॒जाप॑ति॒ना ऽश्वो ऽश्वः॑ प्र॒जाप॑तिनै॒व । \newline
27. प्र॒जाप॑ति नै॒वैव प्र॒जाप॑तिना प्र॒जाप॑ति नै॒वाग्नि म॒ग्नि मे॒व प्र॒जाप॑तिना प्र॒जाप॑ति नै॒वाग्निम् । \newline
28. प्र॒जाप॑ति॒नेति॑ प्र॒जा - प॒ति॒ना॒ । \newline
29. ए॒वाग्नि म॒ग्नि मे॒वैवाग्निम् चि॑नुते चिनुते॒ ऽग्नि मे॒वैवाग्निम् चि॑नुते । \newline
30. अ॒ग्निम् चि॑नुते चिनुते॒ ऽग्नि म॒ग्निम् चि॑नुते प्रथ॒मा प्र॑थ॒मा चि॑नुते॒ ऽग्नि म॒ग्निम् चि॑नुते प्रथ॒मा । \newline
31. चि॒नु॒ते॒ प्र॒थ॒मा प्र॑थ॒मा चि॑नुते चिनुते प्रथ॒ मेष्ट॒केष्ट॑का प्रथ॒मा चि॑नुते चिनुते प्रथ॒मेष्ट॑का । \newline
32. प्र॒थ॒ मेष्ट॒केष्ट॑का प्रथ॒मा प्र॑थ॒मेष्ट॑ कोपधी॒यमा॑ नोपधी॒यमा॒ नेष्ट॑का प्रथ॒मा प्र॑थ॒मेष्ट॑ कोपधी॒यमा॑ना । \newline
33. इष्ट॑ कोपधी॒यमा॑ नोपधी॒यमा॒ नेष्ट॒केष्ट॑ कोपधी॒यमा॑ना पशू॒नाम् प॑शू॒ना मु॑पधी॒यमा॒ नेष्ट॒केष्ट॑ कोपधी॒यमा॑ना पशू॒नाम् । \newline
34. उ॒प॒धी॒यमा॑ना पशू॒नाम् प॑शू॒ना मु॑पधी॒यमा॑ नोपधी॒यमा॑ना पशू॒नाम् च॑ च पशू॒ना मु॑पधी॒यमा॑ नोपधी॒यमा॑ना पशू॒नाम् च॑ । \newline
35. उ॒प॒धी॒यमा॒नेत्यु॑प - धी॒यमा॑ना । \newline
36. प॒शू॒नाम् च॑ च पशू॒नाम् प॑शू॒नाम् च॒ यज॑मानस्य॒ यज॑मानस्य च पशू॒नाम् प॑शू॒नाम् च॒ यज॑मानस्य । \newline
37. च॒ यज॑मानस्य॒ यज॑मानस्य च च॒ यज॑मानस्य च च॒ यज॑मानस्य च च॒ यज॑मानस्य च । \newline
38. यज॑मानस्य च च॒ यज॑मानस्य॒ यज॑मानस्य च प्रा॒णम् प्रा॒णम् च॒ यज॑मानस्य॒ यज॑मानस्य च प्रा॒णम् । \newline
39. च॒ प्रा॒णम् प्रा॒णम् च॑ च प्रा॒ण मप्यपि॑ प्रा॒णम् च॑ च प्रा॒ण मपि॑ । \newline
40. प्रा॒ण मप्यपि॑ प्रा॒णम् प्रा॒ण मपि॑ दधाति दधा॒ त्यपि॑ प्रा॒णम् प्रा॒ण मपि॑ दधाति । \newline
41. प्रा॒णमिति॑ प्र - अ॒नम् । \newline
42. अपि॑ दधाति दधा॒ त्यप्यपि॑ दधाति स्वयमातृ॒ण्णा स्व॑यमातृ॒ण्णा द॑धा॒ त्यप्यपि॑ दधाति स्वयमातृ॒ण्णा । \newline
43. द॒धा॒ति॒ स्व॒य॒मा॒तृ॒ण्णा स्व॑यमातृ॒ण्णा द॑धाति दधाति स्वयमातृ॒ण्णा भ॑वति भवति स्वयमातृ॒ण्णा द॑धाति दधाति स्वयमातृ॒ण्णा भ॑वति । \newline
44. स्व॒य॒मा॒तृ॒ण्णा भ॑वति भवति स्वयमातृ॒ण्णा स्व॑यमातृ॒ण्णा भ॑वति प्रा॒णाना᳚म् प्रा॒णाना᳚म् भवति स्वयमातृ॒ण्णा स्व॑यमातृ॒ण्णा भ॑वति प्रा॒णाना᳚म् । \newline
45. स्व॒य॒मा॒तृ॒ण्णेति॑ स्वयं - आ॒तृ॒ण्णा । \newline
46. भ॒व॒ति॒ प्रा॒णाना᳚म् प्रा॒णाना᳚म् भवति भवति प्रा॒णाना॒ मुथ्सृ॑ष्ट्या॒ उथ्सृ॑ष्ट्यै प्रा॒णाना᳚म् भवति भवति प्रा॒णाना॒ मुथ्सृ॑ष्ट्यै । \newline
47. प्रा॒णाना॒ मुथ्सृ॑ष्ट्या॒ उथ्सृ॑ष्ट्यै प्रा॒णाना᳚म् प्रा॒णाना॒ मुथ्सृ॑ष्ट्या॒ अथो॒ अथो॒ उथ्सृ॑ष्ट्यै प्रा॒णाना᳚म् प्रा॒णाना॒ मुथ्सृ॑ष्ट्या॒ अथो᳚ । \newline
48. प्रा॒णाना॒मिति॑ प्र - अ॒नाना᳚म् । \newline
49. उथ्सृ॑ष्ट्या॒ अथो॒ अथो॒ उथ्सृ॑ष्ट्या॒ उथ्सृ॑ष्ट्या॒ अथो॑ सुव॒र्गस्य॑ सुव॒र्गस्याथो॒ उथ्सृ॑ष्ट्या॒ उथ्सृ॑ष्ट्या॒ अथो॑ सुव॒र्गस्य॑ । \newline
50. उथ्सृ॑ष्ट्या॒ इत्युत् - सृ॒ष्ट्यै॒ । \newline
51. अथो॑ सुव॒र्गस्य॑ सुव॒र्गस्याथो॒ अथो॑ सुव॒र्गस्य॑ लो॒कस्य॑ लो॒कस्य॑ सुव॒र्गस्याथो॒ अथो॑ सुव॒र्गस्य॑ लो॒कस्य॑ । \newline
52. अथो॒ इत्यथो᳚ । \newline
53. सु॒व॒र्गस्य॑ लो॒कस्य॑ लो॒कस्य॑ सुव॒र्गस्य॑ सुव॒र्गस्य॑ लो॒कस्या नु॑ख्यात्या॒ अनु॑ख्यात्यै लो॒कस्य॑ सुव॒र्गस्य॑ सुव॒र्गस्य॑ लो॒कस्या नु॑ख्यात्यै । \newline
54. सु॒व॒र्गस्येति॑ सुवः - गस्य॑ । \newline
55. लो॒कस्या नु॑ख्यात्या॒ अनु॑ख्यात्यै लो॒कस्य॑ लो॒कस्या नु॑ख्यात्या अ॒ग्ना व॒ग्ना वनु॑ख्यात्यै लो॒कस्य॑ लो॒कस्या नु॑ख्यात्या अ॒ग्नौ । \newline
56. अनु॑ख्यात्या अ॒ग्ना व॒ग्ना वनु॑ख्यात्या॒ अनु॑ख्यात्या अ॒ग्ना व॒ग्नि र॒ग्नि र॒ग्ना वनु॑ख्यात्या॒ अनु॑ख्यात्या अ॒ग्ना व॒ग्निः । \newline
57. अनु॑ख्यात्या॒ इत्यनु॑ - ख्या॒त्यै॒ । \newline
58. अ॒ग्ना व॒ग्नि र॒ग्नि र॒ग्ना व॒ग्ना व॒ग्नि श्चे॑त॒व्य॑ श्चेत॒व्यो᳚ ऽग्नि र॒ग्ना व॒ग्ना व॒ग्नि श्चे॑त॒व्यः॑ । \newline
59. अ॒ग्नि श्चे॑त॒व्य॑ श्चेत॒व्यो᳚ ऽग्नि र॒ग्नि श्चे॑त॒व्य॑ इतीति॑ चेत॒व्यो᳚ ऽग्नि र॒ग्नि श्चे॑त॒व्य॑ इति॑ । \newline
60. चे॒त॒व्य॑ इतीति॑ चेत॒व्य॑ श्चेत॒व्य॑ इत्या॑हु राहु॒रिति॑ चेत॒व्य॑ श्चेत॒व्य॑ इत्या॑हुः । \newline
61. इत्या॑हु राहु॒रिती त्या॑हु रे॒ष ए॒ष आ॑हु॒रिती त्या॑हु रे॒षः । \newline
62. आ॒हु॒ रे॒ष ए॒ष आ॑हु राहु रे॒ष वै वा ए॒ष आ॑हु राहु रे॒ष वै । \newline
63. ए॒ष वै वा ए॒ष ए॒ष वा अ॒ग्नि र॒ग्निर् वा ए॒ष ए॒ष वा अ॒ग्निः । \newline
64. वा अ॒ग्नि र॒ग्निर् वै वा अ॒ग्निर् वै᳚श्वान॒रो वै᳚श्वान॒रो᳚ ऽग्निर् वै वा अ॒ग्निर् वै᳚श्वान॒रः । \newline
\pagebreak
\markright{ TS 5.2.8.2  \hfill https://www.vedavms.in \hfill}

\section{ TS 5.2.8.2 }

\textbf{TS 5.2.8.2 } \newline
\textbf{Samhita Paata} \newline

अ॒ग्निर्वै᳚श्वान॒रो यद्ब्रा᳚ह्म॒णस्तस्मै᳚ प्रथ॒मामिष्ट॑कां॒ ॅयजु॑ष्कृतां॒ प्रय॑च्छे॒त् तां ब्रा᳚ह्म॒णश्चोप॑ दद्ध्याता-म॒ग्नावे॒व तद॒ग्निं चि॑नुत ईश्व॒रो वा ए॒ष आर्ति॒मार्तो॒र्यो-ऽवि॑द्वा॒निष्ट॑का-मुप॒दधा॑ति॒ त्रीन्. वरा᳚न् दद्या॒त् त्रयो॒ वै प्रा॒णाः प्रा॒णानाꣳ॒॒ स्पृत्यै॒ द्वावे॒व देयौ॒ द्वौ हि प्रा॒णावेक॑ ए॒व देय॒ एको॒ हि प्रा॒णः प॒शु - [  ] \newline

\textbf{Pada Paata} \newline

अ॒ग्निः । वै॒श्वा॒न॒रः । यत् । ब्रा॒ह्म॒णः । तस्मै᳚ । प्र॒थ॒माम् । इष्ट॑काम् । यजु॑ष्कृता॒मिति॒ यजुः॑ - कृ॒ता॒म् । प्रेति॑ । य॒च्छे॒त् । ताम् । ब्रा॒ह्म॒णः । च॒ । उपेति॑ । द॒द्ध्या॒ता॒म् । अ॒ग्नौ । ए॒व । तत् । अ॒ग्निम् । चि॒नु॒ते॒ । ई॒श्व॒रः । वै । ए॒षः । आर्ति᳚म् । आर्तो॒रित्या-अ॒र्तोः॒ । यः । अवि॑द्वान् । इष्ट॑काम् । उ॒प॒द॒धा॒तीत्यु॑प - दधा॑ति । त्रीन् । वरान्॑ । द॒द्या॒त् । त्रयः॑ । वै । प्रा॒णा इति॑ प्र - अ॒नाः । प्रा॒णाना॒मिति॑ प्र - अ॒नाना᳚म् । स्पृत्यै᳚ । द्वौ । ए॒व । देयौ᳚ । द्वौ । हि । प्रा॒णाविति॑ प्र - अ॒नौ । एकः॑ । ए॒व । देयः॑ । एकः॑ । हि । प्रा॒ण इति॑ प्र - अ॒नः । प॒शुः ।  \newline


\textbf{Krama Paata} \newline

अ॒ग्निर् वै᳚श्वान॒रः । वै॒श्वा॒न॒रो यत् । यद् ब्रा᳚ह्म॒णः । ब्रा॒ह्म॒णस्तस्मै᳚ । तस्मै᳚ प्रथ॒माम् । प्र॒थ॒मामिष्ट॑काम् । इष्ट॑का॒म् ॅयजु॑ष्कृताम् । यजु॑ष्कृता॒म् प्र । यजु॑ष्कृता॒मिति॒ यजुः॑ - कृ॒ता॒म् । प्र य॑च्छेत् । य॒च्छे॒त् ताम् । ताम् ब्रा᳚ह्म॒णः । ब्रा॒ह्म॒णश्च॑ । चोप॑ । उप॑ दद्ध्याताम् । द॒द्ध्या॒ता॒म॒ग्नौ । अ॒ग्नावे॒व । ए॒व तत् । तद॒ग्निम् । अ॒ग्निम् चि॑नुते । चि॒नु॒त॒ ई॒श्व॒रः । ई॒श्व॒रो वै । वा ए॒षः । ए॒ष आर्ति᳚म् । आर्ति॒मार्तोः᳚ । आर्तो॒र् यः । आर्तो॒रित्या - अ॒र्तोः॒ । योऽवि॑द्वान् । अवि॑द्वा॒निष्ट॑काम् । इष्ट॑कामुप॒दधा॑ति । उ॒प॒दधा॑ति॒ त्रीन् । उ॒प॒दधा॒तीत्यु॑प - दधा॑ति । त्रीन्. वरान्॑ । वरा᳚न् दद्यात् । द॒द्या॒त् त्रयः॑ । त्रयो॒ वै । वै प्रा॒णाः । प्रा॒णाः प्रा॒णाना᳚म् । प्रा॒णा इति॑ प्र - अ॒नाः । प्रा॒णानाꣳ॒॒ स्पृत्यै᳚ । प्रा॒णाना॒मिति॑ प्र - अ॒नाना᳚म् । स्पृत्यै॒ द्वौ । द्वावे॒व । ए॒व देयौ᳚ । देयौ॒ द्वौ । द्वौ हि । हि प्रा॒णौ । प्रा॒णावेकः॑ । प्रा॒णाविति॑ प्र - अ॒नौ । एक॑ ए॒व । ए॒व देयः॑ । देय॒ एकः॑ । एको॒ हि । हि प्रा॒णः । प्रा॒णः प॒शुः । प्रा॒ण इति॑ प्र - अ॒नः । प॒शुर् वै \newline

\textbf{Jatai Paata} \newline

1. अ॒ग्निर् वै᳚श्वान॒रो वै᳚श्वान॒रो᳚ ऽग्नि र॒ग्निर् वै᳚श्वान॒रः । \newline
2. वै॒श्वा॒न॒रो यद् यद् वै᳚श्वान॒रो वै᳚श्वान॒रो यत् । \newline
3. यद् ब्रा᳚ह्म॒णो ब्रा᳚ह्म॒णो यद् यद् ब्रा᳚ह्म॒णः । \newline
4. ब्रा॒ह्म॒ण स्तस्मै॒ तस्मै᳚ ब्राह्म॒णो ब्रा᳚ह्म॒ण स्तस्मै᳚ । \newline
5. तस्मै᳚ प्रथ॒माम् प्र॑थ॒माम् तस्मै॒ तस्मै᳚ प्रथ॒माम् । \newline
6. प्र॒थ॒मा मिष्ट॑का॒ मिष्ट॑काम् प्रथ॒माम् प्र॑थ॒मा मिष्ट॑काम् । \newline
7. इष्ट॑कां॒ ॅयजु॑ष्कृतां॒ ॅयजु॑ष्कृता॒ मिष्ट॑का॒ मिष्ट॑कां॒ ॅयजु॑ष्कृताम् । \newline
8. यजु॑ष्कृता॒म् प्र प्र यजु॑ष्कृतां॒ ॅयजु॑ष्कृता॒म् प्र । \newline
9. यजु॑ष्कृता॒मिति॒ यजुः॑ - कृ॒ता॒म् । \newline
10. प्र य॑च्छेद् यच्छे॒त् प्र प्र य॑च्छेत् । \newline
11. य॒च्छे॒त् ताम् तां ॅय॑च्छेद् यच्छे॒त् ताम् । \newline
12. ताम् ब्रा᳚ह्म॒णो ब्रा᳚ह्म॒ण स्ताम् ताम् ब्रा᳚ह्म॒णः । \newline
13. ब्रा॒ह्म॒णश्च॑ च ब्राह्म॒णो ब्रा᳚ह्म॒णश्च॑ । \newline
14. चोपोप॑ च॒ चोप॑ । \newline
15. उप॑ दद्ध्याताम् दद्ध्याता॒ मुपोप॑ दद्ध्याताम् । \newline
16. द॒द्ध्या॒ता॒ म॒ग्ना व॒ग्नौ द॑द्ध्याताम् दद्ध्याता म॒ग्नौ । \newline
17. अ॒ग्ना वे॒वै वाग्ना व॒ग्ना वे॒व । \newline
18. ए॒व तत् तदे॒ वैव तत् । \newline
19. तद॒ग्नि म॒ग्निम् तत् तद॒ग्निम् । \newline
20. अ॒ग्निम् चि॑नुते चिनुते॒ ऽग्नि म॒ग्निम् चि॑नुते । \newline
21. चि॒नु॒त॒ ई॒श्व॒र ई᳚श्व॒र श्चि॑नुते चिनुत ईश्व॒रः । \newline
22. ई॒श्व॒रो वै वा ई᳚श्व॒र ई᳚श्व॒रो वै । \newline
23. वा ए॒ष ए॒ष वै वा ए॒षः । \newline
24. ए॒ष आर्ति॒ मार्ति॑ मे॒ष ए॒ष आर्ति᳚म् । \newline
25. आर्ति॒ मार्तो॒ रार्तो॒ रार्ति॒ मार्ति॒ मार्तोः᳚ । \newline
26. आर्तो॒र् यो य आर्तो॒ रार्तो॒र् यः । \newline
27. आर्तो॒रित्या - अ॒र्तोः॒ । \newline
28. यो ऽवि॑द्वा॒ नवि॑द्वा॒न्॒. यो यो ऽवि॑द्वान् । \newline
29. अवि॑द्वा॒ निष्ट॑का॒ मिष्ट॑का॒ मवि॑द्वा॒ नवि॑द्वा॒ निष्ट॑काम् । \newline
30. इष्ट॑का मुप॒दधा᳚ त्युप॒दधा॒ती ष्ट॑का॒ मिष्ट॑का मुप॒दधा॑ति । \newline
31. उ॒प॒दधा॑ति॒ त्रीꣳस् त्री नु॑प॒दधा᳚ त्युप॒दधा॑ति॒ त्रीन् । \newline
32. उ॒प॒द॒धा॒तीत्यु॑प - दधा॑ति । \newline
33. त्रीन्. वरा॒न्॒. वरा॒न् त्रीꣳ स्त्रीन्. वरान्॑ । \newline
34. वरा᳚न् दद्याद् दद्या॒द् वरा॒न्॒. वरा᳚न् दद्यात् । \newline
35. द॒द्या॒त् त्रय॒ स्त्रयो॑ दद्याद् दद्या॒त् त्रयः॑ । \newline
36. त्रयो॒ वै वै त्रय॒ स्त्रयो॒ वै । \newline
37. वै प्रा॒णाः प्रा॒णा वै वै प्रा॒णाः । \newline
38. प्रा॒णाः प्रा॒णाना᳚म् प्रा॒णाना᳚म् प्रा॒णाः प्रा॒णाः प्रा॒णाना᳚म् । \newline
39. प्रा॒णा इति॑ प्र - अ॒नाः । \newline
40. प्रा॒णानाꣳ॒॒ स्पृत्यै॒ स्पृत्यै᳚ प्रा॒णाना᳚म् प्रा॒णानाꣳ॒॒ स्पृत्यै᳚ । \newline
41. प्रा॒णाना॒मिति॑ प्र - अ॒नाना᳚म् । \newline
42. स्पृत्यै॒ द्वौ द्वौ स्पृत्यै॒ स्पृत्यै॒ द्वौ । \newline
43. द्वा वे॒वैव द्वौ द्वा वे॒व । \newline
44. ए॒व देयौ॒ देया॑ वे॒वैव देयौ᳚ । \newline
45. देयौ॒ द्वौ द्वौ देयौ॒ देयौ॒ द्वौ । \newline
46. द्वौ हि हि द्वौ द्वौ हि । \newline
47. हि प्रा॒णौ प्रा॒णौ हि हि प्रा॒णौ । \newline
48. प्रा॒णा वेक॒ एकः॑ प्रा॒णौ प्रा॒णा वेकः॑ । \newline
49. प्रा॒णाविति॑ प्र - अ॒नौ । \newline
50. एक॑ ए॒वै वैक॒ एक॑ ए॒व । \newline
51. ए॒व देयो॒ देय॑ ए॒वैव देयः॑ । \newline
52. देय॒ एक॒ एको॒ देयो॒ देय॒ एकः॑ । \newline
53. एको॒ हि ह्येक॒ एको॒ हि । \newline
54. हि प्रा॒णः प्रा॒णो हि हि प्रा॒णः । \newline
55. प्रा॒णः प॒शुः प॒शुः प्रा॒णः प्रा॒णः प॒शुः । \newline
56. प्रा॒ण इति॑ प्र - अ॒नः । \newline
57. प॒शुर् वै वै प॒शुः प॒शुर् वै । \newline

\textbf{Ghana Paata } \newline

1. अ॒ग्निर् वै᳚श्वान॒रो वै᳚श्वान॒रो᳚ ऽग्नि र॒ग्निर् वै᳚श्वान॒रो यद् यद् वै᳚श्वान॒रो᳚ ऽग्नि र॒ग्निर् वै᳚श्वान॒रो यत् । \newline
2. वै॒श्वा॒न॒रो यद् यद् वै᳚श्वान॒रो वै᳚श्वान॒रो यद् ब्रा᳚ह्म॒णो ब्रा᳚ह्म॒णो यद् वै᳚श्वान॒रो वै᳚श्वान॒रो यद् ब्रा᳚ह्म॒णः । \newline
3. यद् ब्रा᳚ह्म॒णो ब्रा᳚ह्म॒णो यद् यद् ब्रा᳚ह्म॒ण स्तस्मै॒ तस्मै᳚ ब्राह्म॒णो यद् यद् ब्रा᳚ह्म॒ण स्तस्मै᳚ । \newline
4. ब्रा॒ह्म॒ण स्तस्मै॒ तस्मै᳚ ब्राह्म॒णो ब्रा᳚ह्म॒ण स्तस्मै᳚ प्रथ॒माम् प्र॑थ॒माम् तस्मै᳚ ब्राह्म॒णो ब्रा᳚ह्म॒ण स्तस्मै᳚ प्रथ॒माम् । \newline
5. तस्मै᳚ प्रथ॒माम् प्र॑थ॒माम् तस्मै॒ तस्मै᳚ प्रथ॒मा मिष्ट॑का॒ मिष्ट॑काम् प्रथ॒माम् तस्मै॒ तस्मै᳚ प्रथ॒मा मिष्ट॑काम् । \newline
6. प्र॒थ॒मा मिष्ट॑का॒ मिष्ट॑काम् प्रथ॒माम् प्र॑थ॒मा मिष्ट॑कां॒ ॅयजु॑ष्कृतां॒ ॅयजु॑ष्कृता॒ मिष्ट॑काम् प्रथ॒माम् प्र॑थ॒मा मिष्ट॑कां॒ ॅयजु॑ष्कृताम् । \newline
7. इष्ट॑कां॒ ॅयजु॑ष्कृतां॒ ॅयजु॑ष्कृता॒ मिष्ट॑का॒ मिष्ट॑कां॒ ॅयजु॑ष्कृता॒म् प्र प्र यजु॑ष्कृता॒ मिष्ट॑का॒ मिष्ट॑कां॒ ॅयजु॑ष्कृता॒म् प्र । \newline
8. यजु॑ष्कृता॒म् प्र प्र यजु॑ष्कृतां॒ ॅयजु॑ष्कृता॒म् प्र य॑च्छेद् यच्छे॒त् प्र यजु॑ष्कृतां॒ ॅयजु॑ष्कृता॒म् प्र य॑च्छेत् । \newline
9. यजु॑ष्कृता॒मिति॒ यजुः॑ - कृ॒ता॒म् । \newline
10. प्र य॑च्छेद् यच्छे॒त् प्र प्र य॑च्छे॒त् ताम् तां ॅय॑च्छे॒त् प्र प्र य॑च्छे॒त् ताम् । \newline
11. य॒च्छे॒त् ताम् तां ॅय॑च्छेद् यच्छे॒त् ताम् ब्रा᳚ह्म॒णो ब्रा᳚ह्म॒ण स्तां ॅय॑च्छेद् यच्छे॒त् ताम् ब्रा᳚ह्म॒णः । \newline
12. ताम् ब्रा᳚ह्म॒णो ब्रा᳚ह्म॒ण स्ताम् ताम् ब्रा᳚ह्म॒णश्च॑ च ब्राह्म॒ण स्ताम् ताम् ब्रा᳚ह्म॒णश्च॑ । \newline
13. ब्रा॒ह्म॒णश्च॑ च ब्राह्म॒णो ब्रा᳚ह्म॒ण श्चोपोप॑ च ब्राह्म॒णो ब्रा᳚ह्म॒ण श्चोप॑ । \newline
14. चोपोप॑ च॒ चोप॑ दद्ध्याताम् दद्ध्याता॒ मुप॑ च॒ चोप॑ दद्ध्याताम् । \newline
15. उप॑ दद्ध्याताम् दद्ध्याता॒ मुपोप॑ दद्ध्याता म॒ग्ना व॒ग्नौ द॑द्ध्याता॒ मुपोप॑ दद्ध्याता म॒ग्नौ । \newline
16. द॒द्ध्या॒ता॒ म॒ग्ना व॒ग्नौ द॑द्ध्याताम् दद्ध्याता म॒ग्ना वे॒वैवाग्नौ द॑द्ध्याताम् दद्ध्याता म॒ग्ना वे॒व । \newline
17. अ॒ग्ना वे॒वैवाग्ना व॒ग्ना वे॒व तत् तदे॒वाग्ना व॒ग्ना वे॒व तत् । \newline
18. ए॒व तत् तदे॒वैव तद॒ग्नि म॒ग्निम् तदे॒वैव तद॒ग्निम् । \newline
19. तद॒ग्नि म॒ग्निम् तत् तद॒ग्निम् चि॑नुते चिनुते॒ ऽग्निम् तत् तद॒ग्निम् चि॑नुते । \newline
20. अ॒ग्निम् चि॑नुते चिनुते॒ ऽग्नि म॒ग्निम् चि॑नुत ईश्व॒र ई᳚श्व॒र श्चि॑नुते॒ ऽग्नि म॒ग्निम् चि॑नुत ईश्व॒रः । \newline
21. चि॒नु॒त॒ ई॒श्व॒र ई᳚श्व॒र श्चि॑नुते चिनुत ईश्व॒रो वै वा ई᳚श्व॒र श्चि॑नुते चिनुत ईश्व॒रो वै । \newline
22. ई॒श्व॒रो वै वा ई᳚श्व॒र ई᳚श्व॒रो वा ए॒ष ए॒ष वा ई᳚श्व॒र ई᳚श्व॒रो वा ए॒षः । \newline
23. वा ए॒ष ए॒ष वै वा ए॒ष आर्ति॒ मार्ति॑ मे॒ष वै वा ए॒ष आर्ति᳚म् । \newline
24. ए॒ष आर्ति॒ मार्ति॑ मे॒ष ए॒ष आर्ति॒ मार्तो॒ रार्तो॒ रार्ति॑ मे॒ष ए॒ष आर्ति॒ मार्तोः᳚ । \newline
25. आर्ति॒ मार्तो॒ रार्तो॒ रार्ति॒ मार्ति॒ मार्तो॒र् यो य आर्तो॒ रार्ति॒ मार्ति॒ मार्तो॒र् यः । \newline
26. आर्तो॒र् यो य आर्तो॒ रार्तो॒र् यो ऽवि॑द्वा॒ नवि॑द्वा॒न्॒. य आर्तो॒ रार्तो॒र् यो ऽवि॑द्वान् । \newline
27. आर्तो॒रित्या - अ॒र्तोः॒ । \newline
28. यो ऽवि॑द्वा॒ नवि॑द्वा॒न्॒. यो यो ऽवि॑द्वा॒ निष्ट॑का॒ मिष्ट॑का॒ मवि॑द्वा॒न्॒. यो यो ऽवि॑द्वा॒ निष्ट॑काम् । \newline
29. अवि॑द्वा॒ निष्ट॑का॒ मिष्ट॑का॒ मवि॑द्वा॒ नवि॑द्वा॒ निष्ट॑का मुप॒दधा᳚ त्युप॒दधा॒तीष्ट॑का॒ मवि॑द्वा॒ नवि॑द्वा॒ निष्ट॑का मुप॒दधा॑ति । \newline
30. इष्ट॑का मुप॒दधा᳚ त्युप॒दधा॒तीष्ट॑का॒ मिष्ट॑का मुप॒दधा॑ति॒ त्रीꣳ स्त्री नु॑प॒दधा॒तीष्ट॑का॒ मिष्ट॑का मुप॒दधा॑ति॒ त्रीन् । \newline
31. उ॒प॒दधा॑ति॒ त्रीꣳ स्त्री नु॑प॒दधा᳚ त्युप॒दधा॑ति॒ त्रीन्. वरा॒न्॒. वरा॒न् त्री नु॑प॒दधा᳚ त्युप॒दधा॑ति॒ त्रीन्. वरान्॑ । \newline
32. उ॒प॒द॒धा॒तीत्यु॑प - दधा॑ति । \newline
33. त्रीन्. वरा॒न्॒. वरा॒न् त्रीꣳ स्त्रीन्. वरा᳚न् दद्याद् दद्या॒द् वरा॒न् त्रीꣳ स्त्रीन्. वरा᳚न् दद्यात् । \newline
34. वरा᳚न् दद्याद् दद्या॒द् वरा॒न्॒. वरा᳚न् दद्या॒त् त्रय॒ स्त्रयो॑ दद्या॒द् वरा॒न्॒. वरा᳚न् दद्या॒त् त्रयः॑ । \newline
35. द॒द्या॒त् त्रय॒ स्त्रयो॑ दद्याद् दद्या॒त् त्रयो॒ वै वै त्रयो॑ दद्याद् दद्या॒त् त्रयो॒ वै । \newline
36. त्रयो॒ वै वै त्रय॒ स्त्रयो॒ वै प्रा॒णाः प्रा॒णा वै त्रय॒ स्त्रयो॒ वै प्रा॒णाः । \newline
37. वै प्रा॒णाः प्रा॒णा वै वै प्रा॒णाः प्रा॒णाना᳚म् प्रा॒णाना᳚म् प्रा॒णा वै वै प्रा॒णाः प्रा॒णाना᳚म् । \newline
38. प्रा॒णाः प्रा॒णाना᳚म् प्रा॒णाना᳚म् प्रा॒णाः प्रा॒णाः प्रा॒णानाꣳ॒॒ स्पृत्यै॒ स्पृत्यै᳚ प्रा॒णाना᳚म् प्रा॒णाः प्रा॒णाः प्रा॒णानाꣳ॒॒ स्पृत्यै᳚ । \newline
39. प्रा॒णा इति॑ प्र - अ॒नाः । \newline
40. प्रा॒णानाꣳ॒॒ स्पृत्यै॒ स्पृत्यै᳚ प्रा॒णाना᳚म् प्रा॒णानाꣳ॒॒ स्पृत्यै॒ द्वौ द्वौ स्पृत्यै᳚ प्रा॒णाना᳚म् प्रा॒णानाꣳ॒॒ स्पृत्यै॒ द्वौ । \newline
41. प्रा॒णाना॒मिति॑ प्र - अ॒नाना᳚म् । \newline
42. स्पृत्यै॒ द्वौ द्वौ स्पृत्यै॒ स्पृत्यै॒ द्वा वे॒वैव द्वौ स्पृत्यै॒ स्पृत्यै॒ द्वा वे॒व । \newline
43. द्वा वे॒वैव द्वौ द्वा वे॒व देयौ॒ देया॑ वे॒व द्वौ द्वा वे॒व देयौ᳚ । \newline
44. ए॒व देयौ॒ देया॑ वे॒वैव देयौ॒ द्वौ द्वौ देया॑ वे॒वैव देयौ॒ द्वौ । \newline
45. देयौ॒ द्वौ द्वौ देयौ॒ देयौ॒ द्वौ हि हि द्वौ देयौ॒ देयौ॒ द्वौ हि । \newline
46. द्वौ हि हि द्वौ द्वौ हि प्रा॒णौ प्रा॒णौ हि द्वौ द्वौ हि प्रा॒णौ । \newline
47. हि प्रा॒णौ प्रा॒णौ हि हि प्रा॒णा वेक॒ एकः॑ प्रा॒णौ हि हि प्रा॒णा वेकः॑ । \newline
48. प्रा॒णा वेक॒ एकः॑ प्रा॒णौ प्रा॒णा वेक॑ ए॒वैवैकः॑ प्रा॒णौ प्रा॒णा वेक॑ ए॒व । \newline
49. प्रा॒णाविति॑ प्र - अ॒नौ । \newline
50. एक॑ ए॒वैवैक॒ एक॑ ए॒व देयो॒ देय॑ ए॒वैक॒ एक॑ ए॒व देयः॑ । \newline
51. ए॒व देयो॒ देय॑ ए॒वैव देय॒ एक॒ एको॒ देय॑ ए॒वैव देय॒ एकः॑ । \newline
52. देय॒ एक॒ एको॒ देयो॒ देय॒ एको॒ हि ह्येको॒ देयो॒ देय॒ एको॒ हि । \newline
53. एको॒ हि ह्येक॒ एको॒ हि प्रा॒णः प्रा॒णो ह्येक॒ एको॒ हि प्रा॒णः । \newline
54. हि प्रा॒णः प्रा॒णो हि हि प्रा॒णः प॒शुः प॒शुः प्रा॒णो हि हि प्रा॒णः प॒शुः । \newline
55. प्रा॒णः प॒शुः प॒शुः प्रा॒णः प्रा॒णः प॒शुर् वै वै प॒शुः प्रा॒णः प्रा॒णः प॒शुर् वै । \newline
56. प्रा॒ण इति॑ प्र - अ॒नः । \newline
57. प॒शुर् वै वै प॒शुः प॒शुर् वा ए॒ष ए॒ष वै प॒शुः प॒शुर् वा ए॒षः । \newline
\pagebreak
\markright{ TS 5.2.8.3  \hfill https://www.vedavms.in \hfill}

\section{ TS 5.2.8.3 }

\textbf{TS 5.2.8.3 } \newline
\textbf{Samhita Paata} \newline

-र्वा ए॒ष यद॒ग्निर्न खलु॒ वै प॒शव॒ आय॑वसे रमन्ते दूर्वेष्ट॒कामुप॑ दधाति पशू॒नां धृत्यै॒ द्वाभ्यां॒ प्रति॑ष्ठित्यै॒ काण्डा᳚त् काण्डात् प्र॒रोह॒न्तीत्या॑ह॒ काण्डे॑नकाण्डेन॒ ह्ये॑षा प्र॑ति॒तिष्ठ॑त्ये॒वा नो॑ दूर्वे॒ प्रत॑नु स॒हस्रे॑ण श॒तेन॒ चेत्या॑ह साह॒स्रः प्र॒जाप॑तिः प्र॒जाप॑ते॒राप्त्यै॑ देवल॒क्ष्मं ॅवै त्र्या॑लिखि॒ता तामुत्त॑रलक्ष्माणं दे॒वा उपा॑दध॒ता-ध॑रलक्ष्माण॒-मसु॑रा॒ यं - [  ] \newline

\textbf{Pada Paata} \newline

वै । ए॒षः । यत् । अ॒ग्निः । न । खलु॑ । वै । प॒शवः॑ । आय॑वसे । र॒म॒न्ते॒ । दू॒र्वे॒ष्ट॒कामिति॑ दूर्वा - इ॒ष्ट॒काम् । उपेति॑ । द॒धा॒ति॒ । प॒शू॒नाम् । धृत्यै᳚ । द्वाभ्या᳚म् । प्रति॑ष्ठित्या॒ इति॒ प्रति॑ - स्थि॒त्यै॒ । काण्डा᳚त्काण्डा॒दिति॒ काण्डा᳚त् - का॒ण्डा॒त् । प्र॒रोह॒न्तीति॑ प्र-रोह॑न्ती । इति॑ । आ॒ह॒ । काण्डे॑नकाण्डे॒नेति॒ काण्डे॑न - का॒ण्डे॒न॒ । हि । ए॒षा । प्र॒ति॒तिष्ठ॒तीति॑ प्रति - तिष्ठ॑ति । ए॒वा । नः॒ । दू॒र्वे॒ । प्रेति॑ । त॒नु॒ । स॒हस्रे॑ण । श॒तेन॑ । च॒ । इति॑ । आ॒ह॒ । सा॒ह॒स्रः । प्र॒जाप॑ति॒रिति॑ प्र॒जा - प॒तिः॒ । प्र॒जाप॑ते॒रिति॑ प्र॒जा - प॒तेः॒ । आप्त्यै᳚ । दे॒व॒ल॒क्ष्ममिति॑ देव-ल॒क्ष्मम् । वै । त्र्या॒लि॒खि॒तेति॑ त्रि - आ॒लि॒खि॒ता । ताम् । उत्त॑रलक्ष्माण॒मित्युत्त॑र - ल॒क्ष्मा॒ण॒म् । दे॒वाः । उपेति॑ । अ॒द॒ध॒त॒ । अध॑रलक्ष्माण॒मित्यध॑र - ल॒क्ष्मा॒ण॒म् । असु॑राः । यम् ।  \newline


\textbf{Krama Paata} \newline

वा ए॒षः । ए॒ष यत् । यद॒ग्निः । अ॒ग्निर् न । न खलु॑ । खलु॒ वै । वै प॒शवः॑ । प॒शव॒ आय॑वसे । आय॑वसे रमन्ते । र॒म॒न्ते॒ दू॒र्वे॒ष्ट॒काम् । दू॒र्वे॒ष्ट॒कामुप॑ । दू॒र्वे॒ष्ट॒कामिति॑ दूर्वा - इ॒ष्ट॒काम् । उप॑ दधाति । द॒धा॒ति॒ प॒शू॒नाम् । 
प॒शू॒नाम् धृत्यै᳚ । धृत्यै॒ द्वाभ्या᳚म् । द्वाभ्या॒म् प्रति॑ष्ठित्यै । प्रति॑ष्ठित्यै॒ काण्डा᳚त् काण्डात् । प्रति॑ष्ठित्या॒ इति॒ प्रति॑ - स्थि॒त्यै॒ । काण्डा᳚त्,काण्डात् प्र॒रोह॑न्ती । काण्डा᳚त्,काण्डा॒दिति॒ काण्डा᳚त् - का॒ण्डा॒त्॒ । प्र॒रोह॒न्तीति॑ । प्र॒रोह॒न्तीति॑ प्र - रोह॑न्ती । इत्या॑ह । आ॒ह॒ काण्डे॑न,काण्डेन । काण्डे॑न काण्डेन॒ हि । काण्डे॑न काण्डे॒नेति॒ काण्डे॑न - का॒ण्डे॒न॒ । ह्ये॑षा । ए॒षा प्र॑ति॒तिष्ठ॑ति । प्र॒ति॒तिष्ठ॑त्ये॒वा । प्र॒ति॒तिष्ठ॒तीति॑ प्रति - तिष्ठ॑ति । ए॒वा नः॑ । नो॒ दू॒र्वे॒ । दू॒र्वे॒ प्र । प्र त॑नु । त॒नु॒ स॒हस्रे॑ण । स॒हस्रे॑ण श॒तेन॑ । श॒तेन॑ च । चेति॑ । इत्या॑ह । आ॒ह॒ सा॒ह॒स्रः । सा॒ह॒स्रः प्र॒जाप॑तिः । प्र॒जाप॑तिः प्र॒जाप॑तेः । प्र॒जाप॑ति॒रिति॑ प्र॒जा - प॒तिः॒ । प्र॒जाप॑ते॒राप्त्यै᳚ । प्र॒जाप॑ते॒रिति॑ प्र॒जा - प॒तेः॒ । आप्त्यै॑ देवल॒क्ष्मम् । दे॒व॒ल॒क्ष्मम् ॅवै । दे॒व॒ल॒क्ष्ममिति॑ देव - ल॒क्ष्मम् । वै त्र्या॑लिखि॒ता । त्र्या॒लि॒खि॒ता ताम् । त्र्या॒लि॒खि॒तेति॑ त्रि - आ॒लि॒खि॒ता । तामुत्त॑रलक्ष्माणम् । उत्त॑रलक्ष्माणम् दे॒वाः । उत्त॑रलख्माण॒मित्युत्त॑र - ल॒क्ष्मा॒ण॒म् । दे॒वा उप॑ । उपा॑दधत । अ॒द॒ध॒ताध॑रलक्ष्माणम् । अध॑रलक्ष्माण॒मसु॑राः । अध॑रलक्ष्माण॒मित्यध॑र - ल॒क्ष्मा॒ण॒म् । असु॑रा॒ यम् । यम् का॒मये॑त \newline

\textbf{Jatai Paata} \newline

1. वा ए॒ष ए॒ष वै वा ए॒षः । \newline
2. ए॒ष यद् यदे॒ष ए॒ष यत् । \newline
3. यद॒ग्नि र॒ग्निर् यद् यद॒ग्निः । \newline
4. अ॒ग्निर् न नाग्नि र॒ग्निर् न । \newline
5. न खलु॒ खलु॒ न न खलु॑ । \newline
6. खलु॒ वै वै खलु॒ खलु॒ वै । \newline
7. वै प॒शवः॑ प॒शवो॒ वै वै प॒शवः॑ । \newline
8. प॒शव॒ आय॑वस॒ आय॑वसे प॒शवः॑ प॒शव॒ आय॑वसे । \newline
9. आय॑वसे रमन्ते रमन्त॒ आय॑वस॒ आय॑वसे रमन्ते । \newline
10. र॒म॒न्ते॒ दू॒र्वे॒ष्ट॒काम् दू᳚र्वेष्ट॒काꣳ र॑मन्ते रमन्ते दूर्वेष्ट॒काम् । \newline
11. दू॒र्वे॒ष्ट॒का मुपोप॑ दूर्वेष्ट॒काम् दू᳚र्वेष्ट॒का मुप॑ । \newline
12. दू॒र्वे॒ष्ट॒कामिति॑ दूर्वा - इ॒ष्ट॒काम् । \newline
13. उप॑ दधाति दधा॒ त्युपोप॑ दधाति । \newline
14. द॒धा॒ति॒ प॒शू॒नाम् प॑शू॒नाम् द॑धाति दधाति पशू॒नाम् । \newline
15. प॒शू॒नाम् धृत्यै॒ धृत्यै॑ पशू॒नाम् प॑शू॒नाम् धृत्यै᳚ । \newline
16. धृत्यै॒ द्वाभ्या॒म् द्वाभ्या॒म् धृत्यै॒ धृत्यै॒ द्वाभ्या᳚म् । \newline
17. द्वाभ्या॒म् प्रति॑ष्ठित्यै॒ प्रति॑ष्ठित्यै॒ द्वाभ्या॒म् द्वाभ्या॒म् प्रति॑ष्ठित्यै । \newline
18. प्रति॑ष्ठित्यै॒ काण्डा᳚त्काण्डा॒त् काण्डा᳚त्काण्डा॒त् प्रति॑ष्ठित्यै॒ प्रति॑ष्ठित्यै॒ काण्डा᳚त्काण्डात् । \newline
19. प्रति॑ष्ठित्या॒ इति॒ प्रति॑ - स्थि॒त्यै॒ । \newline
20. काण्डा᳚त्काण्डात् प्र॒रोह॑न्ती प्र॒रोह॑न्ती॒ काण्डा᳚त्काण्डा॒त् काण्डा᳚त्काण्डात् प्र॒रोह॑न्ती । \newline
21. काण्डा᳚त्काण्डा॒दिति॒ काण्डा᳚त् - का॒ण्डा॒त् । \newline
22. प्र॒रोह॒न्तीतीति॑ प्र॒रोह॑न्ती प्र॒रोह॒न्तीति॑ । \newline
23. प्र॒रोह॒न्तीति॑ प्र - रोह॑न्ती । \newline
24. इत्या॑हा॒हेती त्या॑ह । \newline
25. आ॒ह॒ काण्डे॑नकाण्डेन॒ काण्डे॑नकाण्डेना हाह॒ काण्डे॑नकाण्डेन । \newline
26. काण्डे॑नकाण्डेन॒ हि हि काण्डे॑नकाण्डेन॒ काण्डे॑नकाण्डेन॒ हि । \newline
27. काण्डे॑नकाण्डे॒नेति॒ काण्डे॑न - का॒ण्डे॒न॒ । \newline
28. ह्ये॑षैषा हि ह्ये॑षा । \newline
29. ए॒षा प्र॑ति॒तिष्ठ॑ति प्रति॒तिष्ठ॑ त्ये॒षैषा प्र॑ति॒तिष्ठ॑ति । \newline
30. प्र॒ति॒तिष्ठ॑ त्ये॒वैवा प्र॑ति॒तिष्ठ॑ति प्रति॒तिष्ठ॑ त्ये॒वा । \newline
31. प्र॒ति॒तिष्ठ॒तीति॑ प्रति - तिष्ठ॑ति । \newline
32. ए॒वा नो॑ न ए॒वैवा नः॑ । \newline
33. नो॒ दू॒र्वे॒ दू॒र्वे॒ नो॒ नो॒ दू॒र्वे॒ । \newline
34. दू॒र्वे॒ प्र प्र दू᳚र्वे दूर्वे॒ प्र । \newline
35. प्र त॑नु तनु॒ प्र प्र त॑नु । \newline
36. त॒नु॒ स॒हस्रे॑ण स॒हस्रे॑ण तनु तनु स॒हस्रे॑ण । \newline
37. स॒हस्रे॑ण श॒तेन॑ श॒तेन॑ स॒हस्रे॑ण स॒हस्रे॑ण श॒तेन॑ । \newline
38. श॒तेन॑ च च श॒तेन॑ श॒तेन॑ च । \newline
39. चेतीति॑ च॒ चेति॑ । \newline
40. इत्या॑ हा॒हेती त्या॑ह । \newline
41. आ॒ह॒ सा॒ह॒स्रः सा॑ह॒स्र आ॑हाह साह॒स्रः । \newline
42. सा॒ह॒स्रः प्र॒जाप॑तिः प्र॒जाप॑तिः साह॒स्रः सा॑ह॒स्रः प्र॒जाप॑तिः । \newline
43. प्र॒जाप॑तिः प्र॒जाप॑तेः प्र॒जाप॑तेः प्र॒जाप॑तिः प्र॒जाप॑तिः प्र॒जाप॑तेः । \newline
44. प्र॒जाप॑ति॒रिति॑ प्र॒जा - प॒तिः॒ । \newline
45. प्र॒जाप॑त्ये॒ राप्त्या॒ आप्त्यै᳚ प्र॒जाप॑तेः प्र॒जाप॑ते॒ राप्त्यै᳚ । \newline
46. प्र॒जाप॑ते॒रिति॑ प्र॒जा - प॒तेः॒ । \newline
47. आप्त्यै॑ देवल॒क्ष्मम् दे॑वल॒क्ष्म माप्त्या॒ आप्त्यै॑ देवल॒क्ष्मम् । \newline
48. दे॒व॒ल॒क्ष्मं ॅवै वै दे॑वल॒क्ष्मम् दे॑वल॒क्ष्मं ॅवै । \newline
49. दे॒व॒ल॒क्ष्ममिति॑ देव - ल॒क्ष्मम् । \newline
50. वै त्र्या॑लिखि॒ता त्र्या॑लिखि॒ता वै वै त्र्या॑लिखि॒ता । \newline
51. त्र्या॒लि॒खि॒ता ताम् ताम् त्र्या॑लिखि॒ता त्र्या॑लिखि॒ता ताम् । \newline
52. त्र्या॒लि॒खि॒तेति॑ त्रि - आ॒लि॒खि॒ता । \newline
53. ता मुत्त॑रलक्ष्माण॒ मुत्त॑रलक्ष्माण॒म् ताम् ता मुत्त॑रलक्ष्माणम् । \newline
54. उत्त॑रलक्ष्माणम् दे॒वा दे॒वा उत्त॑रलक्ष्माण॒ मुत्त॑रलक्ष्माणम् दे॒वाः । \newline
55. उत्त॑रलक्ष्माण॒मित्युत्त॑र - ल॒क्ष्मा॒ण॒म् । \newline
56. दे॒वा उपोप॑ दे॒वा दे॒वा उप॑ । \newline
57. उपा॑ दधता दध॒ तोपोपा॑ दधत । \newline
58. अ॒द॒ध॒ता ध॑रलक्ष्माण॒ मध॑रलक्ष्माण मदधता दध॒ता ध॑रलक्ष्माणम् । \newline
59. अध॑रलक्ष्माण॒ मसु॑रा॒ असु॑रा॒ अध॑रलक्ष्माण॒ मध॑रलक्ष्माण॒ मसु॑राः । \newline
60. अध॑रलक्ष्माण॒मित्यध॑र - ल॒क्ष्मा॒ण॒म् । \newline
61. असु॑रा॒ यं ॅय मसु॑रा॒ असु॑रा॒ यम् । \newline
62. यम् का॒मये॑त का॒मये॑त॒ यं ॅयम् का॒मये॑त । \newline

\textbf{Ghana Paata } \newline

1. वा ए॒ष ए॒ष वै वा ए॒ष यद् यदे॒ष वै वा ए॒ष यत् । \newline
2. ए॒ष यद् यदे॒ष ए॒ष यद॒ग्नि र॒ग्निर् यदे॒ष ए॒ष यद॒ग्निः । \newline
3. यद॒ग्नि र॒ग्निर् यद् यद॒ग्निर् न नाग्निर् यद् यद॒ग्निर् न । \newline
4. अ॒ग्निर् न नाग्नि र॒ग्निर् न खलु॒ खलु॒ नाग्नि र॒ग्निर् न खलु॑ । \newline
5. न खलु॒ खलु॒ न न खलु॒ वै वै खलु॒ न न खलु॒ वै । \newline
6. खलु॒ वै वै खलु॒ खलु॒ वै प॒शवः॑ प॒शवो॒ वै खलु॒ खलु॒ वै प॒शवः॑ । \newline
7. वै प॒शवः॑ प॒शवो॒ वै वै प॒शव॒ आय॑वस॒ आय॑वसे प॒शवो॒ वै वै प॒शव॒ आय॑वसे । \newline
8. प॒शव॒ आय॑वस॒ आय॑वसे प॒शवः॑ प॒शव॒ आय॑वसे रमन्ते रमन्त॒ आय॑वसे प॒शवः॑ प॒शव॒ आय॑वसे रमन्ते । \newline
9. आय॑वसे रमन्ते रमन्त॒ आय॑वस॒ आय॑वसे रमन्ते दूर्वेष्ट॒काम् दू᳚र्वेष्ट॒काꣳ र॑मन्त॒ आय॑वस॒ आय॑वसे रमन्ते दूर्वेष्ट॒काम् । \newline
10. र॒म॒न्ते॒ दू॒र्वे॒ष्ट॒काम् दू᳚र्वेष्ट॒काꣳ र॑मन्ते रमन्ते दूर्वेष्ट॒का मुपोप॑ दूर्वेष्ट॒काꣳ र॑मन्ते रमन्ते दूर्वेष्ट॒का मुप॑ । \newline
11. दू॒र्वे॒ष्ट॒का मुपोप॑ दूर्वेष्ट॒काम् दू᳚र्वेष्ट॒का मुप॑ दधाति दधा॒ त्युप॑ दूर्वेष्ट॒काम् दू᳚र्वेष्ट॒का मुप॑ दधाति । \newline
12. दू॒र्वे॒ष्ट॒कामिति॑ दूर्वा - इ॒ष्ट॒काम् । \newline
13. उप॑ दधाति दधा॒ त्युपोप॑ दधाति पशू॒नाम् प॑शू॒नाम् द॑धा॒ त्युपोप॑ दधाति पशू॒नाम् । \newline
14. द॒धा॒ति॒ प॒शू॒नाम् प॑शू॒नाम् द॑धाति दधाति पशू॒नाम् धृत्यै॒ धृत्यै॑ पशू॒नाम् द॑धाति दधाति पशू॒नाम् धृत्यै᳚ । \newline
15. प॒शू॒नाम् धृत्यै॒ धृत्यै॑ पशू॒नाम् प॑शू॒नाम् धृत्यै॒ द्वाभ्या॒म् द्वाभ्या॒म् धृत्यै॑ पशू॒नाम् प॑शू॒नाम् धृत्यै॒ द्वाभ्या᳚म् । \newline
16. धृत्यै॒ द्वाभ्या॒म् द्वाभ्या॒म् धृत्यै॒ धृत्यै॒ द्वाभ्या॒म् प्रति॑ष्ठित्यै॒ प्रति॑ष्ठित्यै॒ द्वाभ्या॒म् धृत्यै॒ धृत्यै॒ द्वाभ्या॒म् प्रति॑ष्ठित्यै । \newline
17. द्वाभ्या॒म् प्रति॑ष्ठित्यै॒ प्रति॑ष्ठित्यै॒ द्वाभ्या॒म् द्वाभ्या॒म् प्रति॑ष्ठित्यै॒ काण्डा᳚त्काण्डा॒त् काण्डा᳚त्काण्डा॒त् प्रति॑ष्ठित्यै॒ द्वाभ्या॒म् द्वाभ्या॒म् प्रति॑ष्ठित्यै॒ काण्डा᳚त्काण्डात् । \newline
18. प्रति॑ष्ठित्यै॒ काण्डा᳚त्काण्डा॒त् काण्डा᳚त्काण्डा॒त् प्रति॑ष्ठित्यै॒ प्रति॑ष्ठित्यै॒ काण्डा᳚त्काण्डात् प्र॒रोह॑न्ती प्र॒रोह॑न्ती॒ काण्डा᳚त्काण्डा॒त् प्रति॑ष्ठित्यै॒ प्रति॑ष्ठित्यै॒ काण्डा᳚त्काण्डात् प्र॒रोह॑न्ती । \newline
19. प्रति॑ष्ठित्या॒ इति॒ प्रति॑ - स्थि॒त्यै॒ । \newline
20. काण्डा᳚त्काण्डात् प्र॒रोह॑न्ती प्र॒रोह॑न्ती॒ काण्डा᳚त्काण्डा॒त् काण्डा᳚त्काण्डात् प्र॒रोह॒न्तीतीति॑ प्र॒रोह॑न्ती॒ काण्डा᳚त्काण्डा॒त् काण्डा᳚त्काण्डात् प्र॒रोह॒न्तीति॑ । \newline
21. काण्डा᳚त्काण्डा॒दिति॒ काण्डा᳚त् - का॒ण्डा॒त् । \newline
22. प्र॒रोह॒न्तीतीति॑ प्र॒रोह॑न्ती प्र॒रोह॒न्ती त्या॑हा॒हेति॑ प्र॒रोह॑न्ती प्र॒रोह॒न्तीत्या॑ह । \newline
23. प्र॒रोह॒न्तीति॑ प्र - रोह॑न्ती । \newline
24. इत्या॑हा॒हे तीत्या॑ह॒ काण्डे॑नकाण्डेन॒ काण्डे॑नकाण्डेना॒हे तीत्या॑ह॒ काण्डे॑नकाण्डेन । \newline
25. आ॒ह॒ काण्डे॑नकाण्डेन॒ काण्डे॑नकाण्डेना हाह॒ काण्डे॑नकाण्डेन॒ हि हि काण्डे॑नकाण्डेना हाह॒ काण्डे॑नकाण्डेन॒ हि । \newline
26. काण्डे॑नकाण्डेन॒ हि हि काण्डे॑नकाण्डेन॒ काण्डे॑नकाण्डेन॒ ह्ये॑षैषा हि काण्डे॑नकाण्डेन॒ काण्डे॑नकाण्डेन॒ ह्ये॑षा । \newline
27. काण्डे॑नकाण्डे॒नेति॒ काण्डे॑न - का॒ण्डे॒न॒ । \newline
28. ह्ये॑षैषा हि ह्ये॑षा प्र॑ति॒तिष्ठ॑ति प्रति॒तिष्ठ॑ त्ये॒षा हि ह्ये॑षा प्र॑ति॒तिष्ठ॑ति । \newline
29. ए॒षा प्र॑ति॒तिष्ठ॑ति प्रति॒तिष्ठ॑ त्ये॒षैषा प्र॑ति॒तिष्ठ॑ त्ये॒वैवा प्र॑ति॒तिष्ठ॑ त्ये॒षैषा प्र॑ति॒तिष्ठ॑ त्ये॒वा । \newline
30. प्र॒ति॒तिष्ठ॑ त्ये॒वैवा प्र॑ति॒तिष्ठ॑ति प्रति॒तिष्ठ॑ त्ये॒वा नो॑ न ए॒वा प्र॑ति॒तिष्ठ॑ति प्रति॒तिष्ठ॑ त्ये॒वा नः॑ । \newline
31. प्र॒ति॒तिष्ठ॒तीति॑ प्रति - तिष्ठ॑ति । \newline
32. ए॒वा नो॑ न ए॒वैवा नो॑ दूर्वे दूर्वे न ए॒वैवा नो॑ दूर्वे । \newline
33. नो॒ दू॒र्वे॒ दू॒र्वे॒ नो॒ नो॒ दू॒र्वे॒ प्र प्र दू᳚र्वे नो नो दूर्वे॒ प्र । \newline
34. दू॒र्वे॒ प्र प्र दू᳚र्वे दूर्वे॒ प्र त॑नु तनु॒ प्र दू᳚र्वे दूर्वे॒ प्र त॑नु । \newline
35. प्र त॑नु तनु॒ प्र प्र त॑नु स॒हस्रे॑ण स॒हस्रे॑ण तनु॒ प्र प्र त॑नु स॒हस्रे॑ण । \newline
36. त॒नु॒ स॒हस्रे॑ण स॒हस्रे॑ण तनु तनु स॒हस्रे॑ण श॒तेन॑ श॒तेन॑ स॒हस्रे॑ण तनु तनु स॒हस्रे॑ण श॒तेन॑ । \newline
37. स॒हस्रे॑ण श॒तेन॑ श॒तेन॑ स॒हस्रे॑ण स॒हस्रे॑ण श॒तेन॑ च च श॒तेन॑ स॒हस्रे॑ण स॒हस्रे॑ण श॒तेन॑ च । \newline
38. श॒तेन॑ च च श॒तेन॑ श॒तेन॒ चेतीति॑ च श॒तेन॑ श॒तेन॒ चेति॑ । \newline
39. चेतीति॑ च॒ चेत्या॑ हा॒हेति॑ च॒ चेत्या॑ह । \newline
40. इत्या॑हा॒हे तीत्या॑ह साह॒स्रः सा॑ह॒स्र आ॒हे तीत्या॑ह साह॒स्रः । \newline
41. आ॒ह॒ सा॒ह॒स्रः सा॑ह॒स्र आ॑हाह साह॒स्रः प्र॒जाप॑तिः प्र॒जाप॑तिः साह॒स्र आ॑हाह साह॒स्रः प्र॒जाप॑तिः । \newline
42. सा॒ह॒स्रः प्र॒जाप॑तिः प्र॒जाप॑तिः साह॒स्रः सा॑ह॒स्रः प्र॒जाप॑तिः प्र॒जाप॑तेः प्र॒जाप॑तेः प्र॒जाप॑तिः साह॒स्रः सा॑ह॒स्रः प्र॒जाप॑तिः प्र॒जाप॑तेः । \newline
43. प्र॒जाप॑तिः प्र॒जाप॑तेः प्र॒जाप॑तेः प्र॒जाप॑तिः प्र॒जाप॑तिः प्र॒जाप॑ते॒ राप्त्या॒ आप्त्यै᳚ प्र॒जाप॑तेः प्र॒जाप॑तिः प्र॒जाप॑तिः प्र॒जाप॑ते॒ राप्त्यै᳚ । \newline
44. प्र॒जाप॑ति॒रिति॑ प्र॒जा - प॒तिः॒ । \newline
45. प्र॒जाप॑त्ये॒ राप्त्या॒ आप्त्यै᳚ प्र॒जाप॑तेः प्र॒जाप॑ते॒ राप्त्यै॑ देवल॒क्ष्मम् दे॑वल॒क्ष्म माप्त्यै᳚ प्र॒जाप॑तेः प्र॒जाप॑ते॒ राप्त्यै॑ देवल॒क्ष्मम् । \newline
46. प्र॒जाप॑ते॒रिति॑ प्र॒जा - प॒तेः॒ । \newline
47. आप्त्यै॑ देवल॒क्ष्मम् दे॑वल॒क्ष्म माप्त्या॒ आप्त्यै॑ देवल॒क्ष्मं ॅवै वै दे॑वल॒क्ष्म माप्त्या॒ आप्त्यै॑ देवल॒क्ष्मं ॅवै । \newline
48. दे॒व॒ल॒क्ष्मं ॅवै वै दे॑वल॒क्ष्मम् दे॑वल॒क्ष्मं ॅवै त्र्या॑लिखि॒ता त्र्या॑लिखि॒ता वै दे॑वल॒क्ष्मम् दे॑वल॒क्ष्मं ॅवै त्र्या॑लिखि॒ता । \newline
49. दे॒व॒ल॒क्ष्ममिति॑ देव - ल॒क्ष्मम् । \newline
50. वै त्र्या॑लिखि॒ता त्र्या॑लिखि॒ता वै वै त्र्या॑लिखि॒ता ताम् ताम् त्र्या॑लिखि॒ता वै वै त्र्या॑लिखि॒ता ताम् । \newline
51. त्र्या॒लि॒खि॒ता ताम् ताम् त्र्या॑लिखि॒ता त्र्या॑लिखि॒ता ता मुत्त॑रलक्ष्माण॒ मुत्त॑रलक्ष्माण॒म् ताम् त्र्या॑लिखि॒ता त्र्या॑लिखि॒ता ता मुत्त॑रलक्ष्माणम् । \newline
52. त्र्या॒लि॒खि॒तेति॑ त्रि - आ॒लि॒खि॒ता । \newline
53. ता मुत्त॑रलक्ष्माण॒ मुत्त॑रलक्ष्माण॒म् ताम् ता मुत्त॑रलक्ष्माणम् दे॒वा दे॒वा उत्त॑रलक्ष्माण॒म् ताम् ता मुत्त॑रलक्ष्माणम् दे॒वाः । \newline
54. उत्त॑रलक्ष्माणम् दे॒वा दे॒वा उत्त॑रलक्ष्माण॒ मुत्त॑रलक्ष्माणम् दे॒वा उपोप॑ दे॒वा उत्त॑रलक्ष्माण॒ मुत्त॑रलक्ष्माणम् दे॒वा उप॑ । \newline
55. उत्त॑रलक्ष्माण॒मित्युत्त॑र - ल॒क्ष्मा॒ण॒म् । \newline
56. दे॒वा उपोप॑ दे॒वा दे॒वा उपा॑दधता दध॒तोप॑ दे॒वा दे॒वा उपा॑दधत । \newline
57. उपा॑दधता दध॒तोपोपा॑ दध॒ता ध॑रलक्ष्माण॒ मध॑रलक्ष्माण मदध॒तोपोपा॑ दध॒ता ध॑रलक्ष्माणम् । \newline
58. अ॒द॒ध॒ता ध॑रलक्ष्माण॒ मध॑रलक्ष्माण मदधता दध॒ता ध॑रलक्ष्माण॒ मसु॑रा॒ असु॑रा॒ अध॑रलक्ष्माण मदधता दध॒ता ध॑रलक्ष्माण॒ मसु॑राः । \newline
59. अध॑रलक्ष्माण॒ मसु॑रा॒ असु॑रा॒ अध॑रलक्ष्माण॒ मध॑रलक्ष्माण॒ मसु॑रा॒ यं ॅय मसु॑रा॒ अध॑रलक्ष्माण॒ मध॑रलक्ष्माण॒ मसु॑रा॒ यम् । \newline
60. अध॑रलक्ष्माण॒मित्यध॑र - ल॒क्ष्मा॒ण॒म् । \newline
61. असु॑रा॒ यं ॅय मसु॑रा॒ असु॑रा॒ यम् का॒मये॑त का॒मये॑त॒ य मसु॑रा॒ असु॑रा॒ यम् का॒मये॑त । \newline
62. यम् का॒मये॑त का॒मये॑त॒ यं ॅयम् का॒मये॑त॒ वसी॑या॒न्॒. वसी॑यान् का॒मये॑त॒ यं ॅयम् का॒मये॑त॒ वसी॑यान् । \newline
\pagebreak
\markright{ TS 5.2.8.4  \hfill https://www.vedavms.in \hfill}

\section{ TS 5.2.8.4 }

\textbf{TS 5.2.8.4 } \newline
\textbf{Samhita Paata} \newline

का॒मये॑त॒ वसी॑यान्थ् स्या॒दित्युत्त॑रलक्ष्माणं॒ तस्योप॑ दद्ध्या॒द्-वसी॑याने॒व भ॑वति॒ यं का॒मये॑त॒ पापी॑यान्थ् स्या॒दित्यध॑रलक्ष्माणं॒ तस्योप॑ दद्ध्यादसुरयो॒नि-मे॒वैन॒मनु॒ परा॑ भावयति॒ पापी॑यान् भवति त्र्यालिखि॒ता भ॑वती॒मे वै लो॒कास्त्र्या॑लिखि॒तैभ्य ए॒व लो॒केभ्यो॒ भ्रातृ॑व्यम॒न्तरे॒त्यङ्गि॑रसः सुव॒र्गं ॅलो॒कं ॅय॒तः पु॑रो॒डाशः॑ कू॒र्मो भू॒त्वाऽनु॒ प्रास॑र्प॒ - [  ] \newline

\textbf{Pada Paata} \newline

का॒मये॑त । वसी॑यान् । स्या॒त् । इति॑ । उत्त॑रलक्ष्माण॒मित्युत्त॑र-ल॒क्ष्मा॒ण॒म् । तस्य॑ । उपेति॑ । द॒द्ध्या॒त् । वसी॑यान् । ए॒व । भ॒व॒ति॒ । यम् । का॒मये॑त । पापी॑यान् । स्या॒त् । इति॑ । अध॑रलक्ष्माण॒मित्यध॑र- ल॒क्ष्मा॒ण॒म् । तस्य॑ । उपेति॑ । द॒द्ध्या॒त् । अ॒सु॒र॒यो॒निमित्य॑सुर - यो॒निम् । ए॒व । ए॒न॒म् । अनु॑ । परेति॑ । भा॒व॒य॒ति॒ । पापी॑यान् । भ॒व॒ति॒ । त्र्या॒लि॒खि॒तेति॑ त्रि - आ॒लि॒खि॒ता । भ॒व॒ति॒ । इ॒मे । वै । लो॒काः । त्र्या॒लि॒खि॒तेति॑ त्रि - आ॒लि॒खि॒ता । ए॒भ्यः । ए॒व । लो॒केभ्यः॑ । भ्रातृ॑व्यम् । अ॒न्तः । ए॒ति॒ । अङ्गि॑रसः । सु॒व॒र्गमिति॑ सुवः - गम् । लो॒कम् । य॒तः । पु॒रो॒डाशः॑ । कू॒र्मः । भू॒त्वा । अनु॑ । प्रेति॑ । अ॒स॒र्प॒त् ।  \newline


\textbf{Krama Paata} \newline

का॒मये॑त॒ वसी॑यान् । वसी॑यान्थ् स्यात् । स्या॒दिति॑ । इत्युत्त॑रलक्ष्माणम् । उत्त॑रलक्ष्माण॒म् तस्य॑ । उत्त॑रलक्ष्माण॒मित्युत्त॑र - ल॒क्ष्मा॒ण॒म् । तस्योप॑ । उप॑ दद्ध्यात् । द॒द्ध्या॒द् वसी॑यान् । वसी॑याने॒व । ए॒व भ॑वति । भ॒व॒ति॒ यम् । यम् का॒मये॑त । का॒मये॑त॒ पापी॑यान् । पापी॑यान्थ् स्यात् । स्या॒दिति॑ । इत्यध॑रलक्ष्माणम् । अध॑रलक्ष्माण॒म् तस्य॑ । अध॑रलक्ष्माण॒मित्यध॑र - ल॒क्ष्मा॒ण॒म् । तस्योप॑ । उप॑ दद्ध्यात् । द॒द्ध्या॒द॒सु॒र॒यो॒निम् । अ॒सु॒र॒यो॒निमे॒व । अ॒सु॒र॒यो॒निमित्य॑सुर - यो॒निम् । ए॒वैन᳚म् । ए॒न॒मनु॑ । अनु॒ परा᳚ । परा॑ भावयति । भा॒व॒य॒ति॒ पापी॑यान् । पापी॑यान् भवति । भ॒व॒ति॒ त्र्या॒लि॒खि॒ता । त्र्या॒लि॒खि॒ता भ॑वति । त्र्या॒लि॒खि॒तेति॑ त्रि - आ॒लि॒खि॒ता । भ॒व॒ती॒मे । इ॒मे वै । वै लो॒काः । लो॒का स्त्र्या॑लिखि॒ता । त्र्या॒लि॒खि॒तैभ्यः । त्र्या॒लि॒खि॒तेति॑ त्रि - आ॒लि॒खि॒ता । ए॒भ्य ए॒व । ए॒व लो॒केभ्यः॑ । लो॒केभ्यो॒ भ्रातृ॑व्यम् । भ्रातृ॑व्यम॒न्तः । अ॒न्तरे॑ति । ए॒त्यङ्गि॑रसः । अङ्गि॑रसः सुव॒र्गम् । सु॒व॒र्गम् ॅलो॒कम् । सु॒व॒र्गमिति॑ सुवः - गम् । लो॒कम् ॅय॒तः । य॒तः पु॑रो॒डाशः॑ । पु॒रो॒डाशः॑ कू॒र्मः । कू॒र्मो भू॒त्वा । भू॒त्वाऽनु॑ । अनु॒ प्र । प्रास॑र्पत् । अ॒स॒र्प॒द् य॒त् \newline

\textbf{Jatai Paata} \newline

1. का॒मये॑त॒ वसी॑या॒न्॒. वसी॑यान् का॒मये॑त का॒मये॑त॒ वसी॑यान् । \newline
2. वसी॑यान् थ्स्याथ् स्या॒द् वसी॑या॒न्॒. वसी॑यान् थ्स्यात् । \newline
3. स्या॒ दितीति॑ स्याथ् स्या॒दिति॑ । \newline
4. इत्युत्त॑रलक्ष्माण॒ मुत्त॑रलक्ष्माण॒ मिती त्युत्त॑रलक्ष्माणम् । \newline
5. उत्त॑रलक्ष्माण॒म् तस्य॒ तस्यो त्त॑रलक्ष्माण॒ मुत्त॑रलक्ष्माण॒म् तस्य॑ । \newline
6. उत्त॑रलक्ष्माण॒मित्युत्त॑र - ल॒क्ष्मा॒ण॒म् । \newline
7. तस्योपोप॒ तस्य॒ तस्योप॑ । \newline
8. उप॑ दद्ध्याद् दद्ध्या॒ दुपोप॑ दद्ध्यात् । \newline
9. द॒द्ध्या॒द् वसी॑या॒न्॒. वसी॑यान् दद्ध्याद् दद्ध्या॒द् वसी॑यान् । \newline
10. वसी॑या ने॒वैव वसी॑या॒न्॒. वसी॑या ने॒व । \newline
11. ए॒व भ॑वति भव त्ये॒वैव भ॑वति । \newline
12. भ॒व॒ति॒ यं ॅयम् भ॑वति भवति॒ यम् । \newline
13. यम् का॒मये॑त का॒मये॑त॒ यं ॅयम् का॒मये॑त । \newline
14. का॒मये॑त॒ पापी॑या॒न् पापी॑यान् का॒मये॑त का॒मये॑त॒ पापी॑यान् । \newline
15. पापी॑यान् थ्स्याथ् स्या॒त् पापी॑या॒न् पापी॑यान् थ्स्यात् । \newline
16. स्या॒ दितीति॑ स्याथ् स्या॒दिति॑ । \newline
17. इत्यध॑रलक्ष्माण॒ मध॑रलक्ष्माण॒ मिती त्यध॑रलक्ष्माणम् । \newline
18. अध॑रलक्ष्माण॒म् तस्य॒ तस्या ध॑रलक्ष्माण॒ मध॑रलक्ष्माण॒म् तस्य॑ । \newline
19. अध॑रलक्ष्माण॒मित्यध॑र - ल॒क्ष्मा॒ण॒म् । \newline
20. तस्योपोप॒ तस्य॒ तस्योप॑ । \newline
21. उप॑ दद्ध्याद् दद्ध्या॒ दुपोप॑ दद्ध्यात् । \newline
22. द॒द्ध्या॒ द॒सु॒र॒यो॒नि म॑सुरयो॒निम् द॑द्ध्याद् दद्ध्या दसुरयो॒निम् । \newline
23. अ॒सु॒र॒यो॒नि मे॒वैवा सु॑रयो॒नि म॑सुरयो॒नि मे॒व । \newline
24. अ॒सु॒र॒यो॒निमित्य॑सुर - यो॒निम् । \newline
25. ए॒वैन॑ मेन मे॒वै वैन᳚म् । \newline
26. ए॒न॒ मन् वन् वे॑न मेन॒ मनु॑ । \newline
27. अनु॒ परा॒ परा ऽन्वनु॒ परा᳚ । \newline
28. परा॑ भावयति भावयति॒ परा॒ परा॑ भावयति । \newline
29. भा॒व॒य॒ति॒ पापी॑या॒न् पापी॑यान् भावयति भावयति॒ पापी॑यान् । \newline
30. पापी॑यान् भवति भवति॒ पापी॑या॒न् पापी॑यान् भवति । \newline
31. भ॒व॒ति॒ त्र्या॒लि॒खि॒ता त्र्या॑लिखि॒ता भ॑वति भवति त्र्यालिखि॒ता । \newline
32. त्र्या॒लि॒खि॒ता भ॑वति भवति त्र्यालिखि॒ता त्र्या॑लिखि॒ता भ॑वति । \newline
33. त्र्या॒लि॒खि॒तेति॑ त्रि - आ॒लि॒खि॒ता । \newline
34. भ॒व॒ती॒म इ॒मे भ॑वति भवती॒मे । \newline
35. इ॒मे वै वा इ॒म इ॒मे वै । \newline
36. वै लो॒का लो॒का वै वै लो॒काः । \newline
37. लो॒का स्त्र्या॑लिखि॒ता त्र्या॑लिखि॒ता लो॒का लो॒का स्त्र्या॑लिखि॒ता । \newline
38. त्र्या॒लि॒खि॒ तैभ्य ए॒भ्य स्त्र्या॑लिखि॒ता त्र्या॑लिखि॒ तैभ्यः । \newline
39. त्र्या॒लि॒खि॒तेति॑ त्रि - आ॒लि॒खि॒ता । \newline
40. ए॒भ्य ए॒वै वैभ्य ए॒भ्य ए॒व । \newline
41. ए॒व लो॒केभ्यो॑ लो॒केभ्य॑ ए॒वैव लो॒केभ्यः॑ । \newline
42. लो॒केभ्यो॒ भ्रातृ॑व्य॒म् भ्रातृ॑व्यम् ॅलो॒केभ्यो॑ लो॒केभ्यो॒ भ्रातृ॑व्यम् । \newline
43. भ्रातृ॑व्य म॒न्त र॒न्तर् भ्रातृ॑व्य॒म् भ्रातृ॑व्य म॒न्तः । \newline
44. अ॒न्त रे᳚त्ये त्य॒न्त र॒न्त रे॑ति । \newline
45. ए॒त्यङ्गि॑र॒सो ऽङ्गि॑रस एत्ये॒ त्यङ्गि॑रसः । \newline
46. अङ्गि॑रसः सुव॒र्गꣳ सु॑व॒र्ग मङ्गि॑र॒सो ऽङ्गि॑रसः सुव॒र्गम् । \newline
47. सु॒व॒र्गम् ॅलो॒कम् ॅलो॒कꣳ सु॑व॒र्गꣳ सु॑व॒र्गम् ॅलो॒कम् । \newline
48. सु॒व॒र्गमिति॑ सुवः - गम् । \newline
49. लो॒कं ॅय॒तो य॒तो लो॒कम् ॅलो॒कं ॅय॒तः । \newline
50. य॒तः पु॑रो॒डाशः॑ पुरो॒डाशो॑ य॒तो य॒तः पु॑रो॒डाशः॑ । \newline
51. पु॒रो॒डाशः॑ कू॒र्मः कू॒र्मः पु॑रो॒डाशः॑ पुरो॒डाशः॑ कू॒र्मः । \newline
52. कू॒र्मो भू॒त्वा भू॒त्वा कू॒र्मः कू॒र्मो भू॒त्वा । \newline
53. भू॒त्वा ऽन्वनु॑ भू॒त्वा भू॒त्वा ऽनु॑ । \newline
54. अनु॒ प्र प्राण् वनु॒ प्र । \newline
55. प्रास॑र्प दसर्प॒त् प्र प्रास॑र्पत् । \newline
56. अ॒स॒र्प॒द् यद् यद॑सर्प दसर्प॒द् यत् । \newline

\textbf{Ghana Paata } \newline

1. का॒मये॑त॒ वसी॑या॒न्॒. वसी॑यान् का॒मये॑त का॒मये॑त॒ वसी॑यान् थ्स्याथ् स्या॒द् वसी॑यान् का॒मये॑त का॒मये॑त॒ वसी॑यान् थ्स्यात् । \newline
2. वसी॑यान् थ्स्याथ् स्या॒द् वसी॑या॒न्॒. वसी॑यान् थ्स्या॒दितीति॑ स्या॒द् वसी॑या॒न्॒. वसी॑यान् थ्स्या॒दिति॑ । \newline
3. स्या॒दितीति॑ स्याथ् स्या॒दि त्युत्त॑रलक्ष्माण॒ मुत्त॑रलक्ष्माण॒ मिति॑ स्याथ् स्या॒दि त्युत्त॑रलक्ष्माणम् । \newline
4. इत्युत्त॑रलक्ष्माण॒ मुत्त॑रलक्ष्माण॒ मिती त्युत्त॑रलक्ष्माण॒म् तस्य॒ तस्योत्त॑रलक्ष्माण॒ मिती त्युत्त॑रलक्ष्माण॒म् तस्य॑ । \newline
5. उत्त॑रलक्ष्माण॒म् तस्य॒ तस्योत्त॑रलक्ष्माण॒ मुत्त॑रलक्ष्माण॒म् तस्योपोप॒ तस्योत्त॑रलक्ष्माण॒ मुत्त॑रलक्ष्माण॒म् तस्योप॑ । \newline
6. उत्त॑रलक्ष्माण॒मित्युत्त॑र - ल॒क्ष्मा॒ण॒म् । \newline
7. तस्योपोप॒ तस्य॒ तस्योप॑ दद्ध्याद् दद्ध्या॒ दुप॒ तस्य॒ तस्योप॑ दद्ध्यात् । \newline
8. उप॑ दद्ध्याद् दद्ध्या॒ दुपोप॑ दद्ध्या॒द् वसी॑या॒न्॒. वसी॑यान् दद्ध्या॒ दुपोप॑ दद्ध्या॒द् वसी॑यान् । \newline
9. द॒द्ध्या॒द् वसी॑या॒न्॒. वसी॑यान् दद्ध्याद् दद्ध्या॒द् वसी॑या ने॒वैव वसी॑यान् दद्ध्याद् दद्ध्या॒द् वसी॑या ने॒व । \newline
10. वसी॑या ने॒वैव वसी॑या॒न्॒. वसी॑या ने॒व भ॑वति भव त्ये॒व वसी॑या॒न्॒. वसी॑या ने॒व भ॑वति । \newline
11. ए॒व भ॑वति भव त्ये॒वैव भ॑वति॒ यं ॅयम् भ॑व त्ये॒वैव भ॑वति॒ यम् । \newline
12. भ॒व॒ति॒ यं ॅयम् भ॑वति भवति॒ यम् का॒मये॑त का॒मये॑त॒ यम् भ॑वति भवति॒ यम् का॒मये॑त । \newline
13. यम् का॒मये॑त का॒मये॑त॒ यं ॅयम् का॒मये॑त॒ पापी॑या॒न् पापी॑यान् का॒मये॑त॒ यं ॅयम् का॒मये॑त॒ पापी॑यान् । \newline
14. का॒मये॑त॒ पापी॑या॒न् पापी॑यान् का॒मये॑त का॒मये॑त॒ पापी॑यान् थ्स्याथ् स्या॒त् पापी॑यान् का॒मये॑त का॒मये॑त॒ पापी॑यान् थ्स्यात् । \newline
15. पापी॑यान् थ्स्याथ् स्या॒त् पापी॑या॒न् पापी॑यान् थ्स्या॒दितीति॑ स्या॒त् पापी॑या॒न् पापी॑यान् थ्स्या॒दिति॑ । \newline
16. स्या॒दितीति॑ स्याथ् स्या॒दि त्यध॑रलक्ष्माण॒ मध॑रलक्ष्माण॒ मिति॑ स्याथ् स्या॒दि त्यध॑रलक्ष्माणम् । \newline
17. इत्यध॑रलक्ष्माण॒ मध॑रलक्ष्माण॒ मिती त्यध॑रलक्ष्माण॒म् तस्य॒ तस्या ध॑रलक्ष्माण॒ मिती त्यध॑रलक्ष्माण॒म् तस्य॑ । \newline
18. अध॑रलक्ष्माण॒म् तस्य॒ तस्या ध॑रलक्ष्माण॒ मध॑रलक्ष्माण॒म् तस्योपोप॒ तस्या ध॑रलक्ष्माण॒ मध॑रलक्ष्माण॒म् तस्योप॑ । \newline
19. अध॑रलक्ष्माण॒मित्यध॑र - ल॒क्ष्मा॒ण॒म् । \newline
20. तस्योपोप॒ तस्य॒ तस्योप॑ दद्ध्याद् दद्ध्या॒दुप॒ तस्य॒ तस्योप॑ दद्ध्यात् । \newline
21. उप॑ दद्ध्याद् दद्ध्या॒दुपोप॑ दद्ध्या दसुरयो॒नि म॑सुरयो॒निम् द॑द्ध्या॒ दुपोप॑ दद्ध्या दसुरयो॒निम् । \newline
22. द॒द्ध्या॒ द॒सु॒र॒यो॒नि म॑सुरयो॒निम् द॑द्ध्याद् दद्ध्या दसुरयो॒नि मे॒वैवा सु॑रयो॒निम् द॑द्ध्याद् दद्ध्या दसुरयो॒नि मे॒व । \newline
23. अ॒सु॒र॒यो॒नि मे॒वैवा सु॑रयो॒नि म॑सुरयो॒नि मे॒वैन॑ मेन मे॒वासु॑रयो॒नि म॑सुरयो॒नि मे॒वैन᳚म् । \newline
24. अ॒सु॒र॒यो॒निमित्य॑सुर - यो॒निम् । \newline
25. ए॒वैन॑ मेन मे॒वैवैन॒ मन् वन् वे॑न मे॒वैवैन॒ मनु॑ । \newline
26. ए॒न॒ मन् वन् वे॑न मेन॒ मनु॒ परा॒ परा ऽन्वे॑न मेन॒ मनु॒ परा᳚ । \newline
27. अनु॒ परा॒ परा ऽन्वनु॒ परा॑ भावयति भावयति॒ परा ऽन्वनु॒ परा॑ भावयति । \newline
28. परा॑ भावयति भावयति॒ परा॒ परा॑ भावयति॒ पापी॑या॒न् पापी॑यान् भावयति॒ परा॒ परा॑ भावयति॒ पापी॑यान् । \newline
29. भा॒व॒य॒ति॒ पापी॑या॒न् पापी॑यान् भावयति भावयति॒ पापी॑यान् भवति भवति॒ पापी॑यान् भावयति भावयति॒ पापी॑यान् भवति । \newline
30. पापी॑यान् भवति भवति॒ पापी॑या॒न् पापी॑यान् भवति त्र्यालिखि॒ता त्र्या॑लिखि॒ता भ॑वति॒ पापी॑या॒न् पापी॑यान् भवति त्र्यालिखि॒ता । \newline
31. भ॒व॒ति॒ त्र्या॒लि॒खि॒ता त्र्या॑लिखि॒ता भ॑वति भवति त्र्यालिखि॒ता भ॑वति भवति त्र्यालिखि॒ता भ॑वति भवति त्र्यालिखि॒ता भ॑वति । \newline
32. त्र्या॒लि॒खि॒ता भ॑वति भवति त्र्यालिखि॒ता त्र्या॑लिखि॒ता भ॑वती॒म इ॒मे भ॑वति त्र्यालिखि॒ता त्र्या॑लिखि॒ता भ॑वती॒मे । \newline
33. त्र्या॒लि॒खि॒तेति॑ त्रि - आ॒लि॒खि॒ता । \newline
34. भ॒व॒ती॒म इ॒मे भ॑वति भवती॒मे वै वा इ॒मे भ॑वति भवती॒मे वै । \newline
35. इ॒मे वै वा इ॒म इ॒मे वै लो॒का लो॒का वा इ॒म इ॒मे वै लो॒काः । \newline
36. वै लो॒का लो॒का वै वै लो॒का स्त्र्या॑लिखि॒ता त्र्या॑लिखि॒ता लो॒का वै वै लो॒का स्त्र्या॑लिखि॒ता । \newline
37. लो॒का स्त्र्या॑लिखि॒ता त्र्या॑लिखि॒ता लो॒का लो॒का स्त्र्या॑लिखि॒तैभ्य ए॒भ्य स्त्र्या॑लिखि॒ता लो॒का लो॒का स्त्र्या॑लिखि॒तैभ्यः । \newline
38. त्र्या॒लि॒खि॒तैभ्य ए॒भ्य स्त्र्या॑लिखि॒ता त्र्या॑लिखि॒तैभ्य ए॒वैवैभ्य स्त्र्या॑लिखि॒ता त्र्या॑लिखि॒तैभ्य ए॒व । \newline
39. त्र्या॒लि॒खि॒तेति॑ त्रि - आ॒लि॒खि॒ता । \newline
40. ए॒भ्य ए॒वैवैभ्य ए॒भ्य ए॒व लो॒केभ्यो॑ लो॒केभ्य॑ ए॒वैभ्य ए॒भ्य ए॒व लो॒केभ्यः॑ । \newline
41. ए॒व लो॒केभ्यो॑ लो॒केभ्य॑ ए॒वैव लो॒केभ्यो॒ भ्रातृ॑व्य॒म् भ्रातृ॑व्यम् ॅलो॒केभ्य॑ ए॒वैव लो॒केभ्यो॒ भ्रातृ॑व्यम् । \newline
42. लो॒केभ्यो॒ भ्रातृ॑व्य॒म् भ्रातृ॑व्यम् ॅलो॒केभ्यो॑ लो॒केभ्यो॒ भ्रातृ॑व्य म॒न्त र॒न्तर् भ्रातृ॑व्यम् ॅलो॒केभ्यो॑ लो॒केभ्यो॒ भ्रातृ॑व्य म॒न्तः । \newline
43. भ्रातृ॑व्य म॒न्त र॒न्तर् भ्रातृ॑व्य॒म् भ्रातृ॑व्य म॒न्त रे᳚त्ये त्य॒न्तर् भ्रातृ॑व्य॒म् भ्रातृ॑व्य म॒न्त रे॑ति । \newline
44. अ॒न्त रे᳚त्ये त्य॒न्त र॒न्त रे॒त्यङ्गि॑र॒सो ऽङ्गि॑रस एत्य॒न्त र॒न्त रे॒त्यङ्गि॑रसः । \newline
45. ए॒त्यङ्गि॑र॒सो ऽङ्गि॑रस एत्ये॒ त्यङ्गि॑रसः सुव॒र्गꣳ सु॑व॒र्ग मङ्गि॑रस एत्ये॒ त्यङ्गि॑रसः सुव॒र्गम् । \newline
46. अङ्गि॑रसः सुव॒र्गꣳ सु॑व॒र्ग मङ्गि॑र॒सो ऽङ्गि॑रसः सुव॒र्गम् ॅलो॒कम् ॅलो॒कꣳ सु॑व॒र्ग मङ्गि॑र॒सो ऽङ्गि॑रसः सुव॒र्गम् ॅलो॒कम् । \newline
47. सु॒व॒र्गम् ॅलो॒कम् ॅलो॒कꣳ सु॑व॒र्गꣳ सु॑व॒र्गम् ॅलो॒कं ॅय॒तो य॒तो लो॒कꣳ सु॑व॒र्गꣳ सु॑व॒र्गम् ॅलो॒कं ॅय॒तः । \newline
48. सु॒व॒र्गमिति॑ सुवः - गम् । \newline
49. लो॒कं ॅय॒तो य॒तो लो॒कम् ॅलो॒कं ॅय॒तः पु॑रो॒डाशः॑ पुरो॒डाशो॑ य॒तो लो॒कम् ॅलो॒कं ॅय॒तः पु॑रो॒डाशः॑ । \newline
50. य॒तः पु॑रो॒डाशः॑ पुरो॒डाशो॑ य॒तो य॒तः पु॑रो॒डाशः॑ कू॒र्मः कू॒र्मः पु॑रो॒डाशो॑ य॒तो य॒तः पु॑रो॒डाशः॑ कू॒र्मः । \newline
51. पु॒रो॒डाशः॑ कू॒र्मः कू॒र्मः पु॑रो॒डाशः॑ पुरो॒डाशः॑ कू॒र्मो भू॒त्वा भू॒त्वा कू॒र्मः पु॑रो॒डाशः॑ पुरो॒डाशः॑ कू॒र्मो भू॒त्वा । \newline
52. कू॒र्मो भू॒त्वा भू॒त्वा कू॒र्मः कू॒र्मो भू॒त्वा ऽन्वनु॑ भू॒त्वा कू॒र्मः कू॒र्मो भू॒त्वा ऽनु॑ । \newline
53. भू॒त्वा ऽन्वनु॑ भू॒त्वा भू॒त्वा ऽनु॒ प्र प्राणु॑ भू॒त्वा भू॒त्वा ऽनु॒ प्र । \newline
54. अनु॒ प्र प्राण् वनु॒ प्रास॑र्प दसर्प॒त् प्राण् वनु॒ प्रास॑र्पत् । \newline
55. प्रास॑र्प दसर्प॒त् प्र प्रास॑र्प॒द् यद् यद॑सर्प॒त् प्र प्रास॑र्प॒द् यत् । \newline
56. अ॒स॒र्प॒द् यद् यद॑सर्प दसर्प॒द् यत् कू॒र्मम् कू॒र्मं ॅयद॑सर्प दसर्प॒द् यत् कू॒र्मम् । \newline
\pagebreak
\markright{ TS 5.2.8.5  \hfill https://www.vedavms.in \hfill}

\section{ TS 5.2.8.5 }

\textbf{TS 5.2.8.5 } \newline
\textbf{Samhita Paata} \newline

-द्यत् कू॒र्ममु॑प॒दधा॑ति॒ यथा᳚ क्षेत्र॒विदञ्ज॑सा॒ नय॑त्ये॒वमे॒वैनं॑ कू॒र्मः सु॑व॒र्गं ॅलो॒कमञ्ज॑सा नयति॒ मेधो॒ वा ए॒ष प॑शू॒नां ॅयत् कू॒र्मो यत् कू॒र्ममु॑प॒ दधा॑ति॒ स्वमे॒व मेधं॒ पश्य॑न्तः प॒शव॒ उप॑ तिष्ठन्ते श्मशा॒नं ॅवा ए॒तत् क्रि॑यते॒ यन्मृ॒तानां᳚ पशू॒नाꣳ शी॒र्॒.षाण्यु॑पधी॒यन्ते॒ यज्जीव॑न्तं कू॒र्ममु॑प॒ दधा॑ति॒ तेनाश्म॑शानचिद्वास्त॒व्यो॑ वा ए॒ष यत् - [  ] \newline

\textbf{Pada Paata} \newline

यत् । कू॒र्मम् । उ॒प॒दधा॒तीत्यु॑प - दधा॑ति । यथा᳚ । क्षे॒त्र॒विदिति॑ क्षेत्र - वित् । अञ्ज॑सा । नय॑ति । ए॒वम् । ए॒व । ए॒न॒म् । कू॒र्मः । सु॒व॒र्गमिति॑ सुवः - गम् । लो॒कम् । अञ्ज॑सा । न॒य॒ति॒ । मेधः॑ । वै । ए॒षः । प॒शू॒नाम् । यत् । कू॒र्मः । यत् । कू॒र्मम् । उ॒प॒दधा॒तीत्यु॑प - दधा॑ति । स्वम् । ए॒व । मेध᳚म् । पश्य॑न्तः । प॒शवः॑ । उपेति॑ । ति॒ष्ठ॒न्ते॒ । श्म॒शा॒नम् । वै । ए॒तत् । क्रि॒य॒ते॒ । यत् । मृ॒ताना᳚म् । प॒शू॒नाम् । शी॒र्॒.षाणि॑ । उ॒प॒धी॒यन्त॒ इत्यु॑प - धी॒यन्ते᳚ । यत् । जीव॑न्तम् । कू॒र्मम् । उ॒प॒दधा॒तीत्यु॑प - दधा॑ति । तेन॑ । अश्म॑शानचि॒दित्यश्म॑शान - चि॒त् । वा॒स्त॒व्यः॑ । वै । ए॒षः । यत् ।  \newline


\textbf{Krama Paata} \newline

यत् कू॒र्मम् । कू॒र्ममु॑प॒दधा॑ति । उ॒प॒दधा॑ति॒ यथा᳚ । उ॒प॒दधा॒तीत्यु॑प दधा॑ति । यथा᳚ क्षेत्र॒वित् । क्षे॒त्र॒विदञ्ज॑सा । क्षे॒त्र॒विदिति॑ क्षेत्र - वित् । अञ्ज॑सा॒ नय॑ति । नय॑त्ये॒वम् । ए॒वमे॒व । ए॒वैन᳚म् । ए॒न॒म् कू॒र्मः । कू॒र्मः सु॑व॒र्गम् । सु॒व॒र्गम् ॅलो॒कम् । सु॒व॒र्गमिति॑ सुवः - गम् । लो॒कमञ्ज॑सा । अञ्ज॑सा नयति । न॒य॒ति॒ मेधः॑ । मेधो॒ वै । वा ए॒षः । ए॒ष प॑शू॒नाम् । प॒शू॒नाम् ॅयत् । यत् कू॒र्मः । कू॒र्मो यत् । यत् कू॒र्मम् । कू॒र्ममु॑प॒दधा॑ति । उ॒प॒दधा॑ति॒ स्वम् । उ॒प॒दधा॒तीत्यु॑प - दधा॑ति । स्वमे॒व । ए॒व मेध᳚म् । मेध॒म् पश्य॑न्तः । पश्य॑न्तः प॒शवः॑ । प॒शव॒ उप॑ । उप॑ तिष्ठन्ते । ति॒ष्ठ॒न्ते॒ श्म॒शा॒नम् । श्म॒शा॒नम् ॅवै । वा ए॒तत् । ए॒तत् क्रि॑यते । क्रि॒य॒ते॒ यत् । यन् मृ॒ताना᳚म् । मृ॒ताना᳚म् पशू॒नाम् । प॒शू॒नाꣳ शी॒र्.॒षाणि॑ । शी॒र्.॒षाण्यु॑पधी॒यन्ते᳚ । उ॒प॒धी॒यन्ते॒ यत् । उ॒प॒धी॒यन्त॒ इत्यु॑प - धी॒यन्ते᳚ । यज् जीव॑न्तम् । जीव॑न्तम् कू॒र्मम् । कू॒र्ममु॑प॒दधा॑ति । उ॒प॒दधा॑ति॒ तेन॑ । उ॒प॒दधा॒तीत्यु॑प - दधा॑ति । तेनाश्म॑शानचित् । अश्म॑शानचिद् वास्त॒व्यः॑ । अश्म॑शानचि॒दित्यश्म॑शान - चि॒त्॒ । वा॒स्त॒व्यो॑ वै । वा ए॒षः । ए॒ष यत् । यत् कू॒र्मः \newline

\textbf{Jatai Paata} \newline

1. यत् कू॒र्मम् कू॒र्मं ॅयद् यत् कू॒र्मम् । \newline
2. कू॒र्म मु॑प॒दधा᳚ त्युप॒दधा॑ति कू॒र्मम् कू॒र्म मु॑प॒दधा॑ति । \newline
3. उ॒प॒दधा॑ति॒ यथा॒ यथो॑ प॒दधा᳚ त्युप॒दधा॑ति॒ यथा᳚ । \newline
4. उ॒प॒दधा॒तीत्यु॑प - दधा॑ति । \newline
5. यथा᳚ क्षेत्र॒वित् क्षे᳚त्र॒विद् यथा॒ यथा᳚ क्षेत्र॒वित् । \newline
6. क्षे॒त्र॒वि दञ्ज॒सा ऽञ्ज॑सा क्षेत्र॒वित् क्षे᳚त्र॒वि दञ्ज॑सा । \newline
7. क्षे॒त्र॒विदिति॑ क्षेत्र - वित् । \newline
8. अञ्ज॑सा॒ नय॑ति॒ नय॒ त्यञ्ज॒सा ऽञ्ज॑सा॒ नय॑ति । \newline
9. नय॑ त्ये॒व मे॒वन् नय॑ति॒ नय॑ त्ये॒वम् । \newline
10. ए॒व मे॒वै वैव मे॒व मे॒व । \newline
11. ए॒वैन॑ मेन मे॒वै वैन᳚म् । \newline
12. ए॒न॒म् कू॒र्मः कू॒र्म ए॑न मेनम् कू॒र्मः । \newline
13. कू॒र्मः सु॑व॒र्गꣳ सु॑व॒र्गम् कू॒र्मः कू॒र्मः सु॑व॒र्गम् । \newline
14. सु॒व॒र्गम् ॅलो॒कम् ॅलो॒कꣳ सु॑व॒र्गꣳ सु॑व॒र्गम् ॅलो॒कम् । \newline
15. सु॒व॒र्गमिति॑ सुवः - गम् । \newline
16. लो॒क मञ्ज॒सा ऽञ्ज॑सा लो॒कम् ॅलो॒क मञ्ज॑सा । \newline
17. अञ्ज॑सा नयति नय॒ त्यञ्ज॒सा ऽञ्ज॑सा नयति । \newline
18. न॒य॒ति॒ मेधो॒ मेधो॑ नयति नयति॒ मेधः॑ । \newline
19. मेधो॒ वै वै मेधो॒ मेधो॒ वै । \newline
20. वा ए॒ष ए॒ष वै वा ए॒षः । \newline
21. ए॒ष प॑शू॒नाम् प॑शू॒ना मे॒ष ए॒ष प॑शू॒नाम् । \newline
22. प॒शू॒नां ॅयद् यत् प॑शू॒नाम् प॑शू॒नां ॅयत् । \newline
23. यत् कू॒र्मः कू॒र्मो यद् यत् कू॒र्मः । \newline
24. कू॒र्मो यद् यत् कू॒र्मः कू॒र्मो यत् । \newline
25. यत् कू॒र्मम् कू॒र्मं ॅयद् यत् कू॒र्मम् । \newline
26. कू॒र्म मु॑प॒दधा᳚ त्युप॒दधा॑ति कू॒र्मम् कू॒र्म मु॑प॒दधा॑ति । \newline
27. उ॒प॒दधा॑ति॒ स्वꣳ स्व मु॑प॒दधा᳚ त्युप॒दधा॑ति॒ स्वम् । \newline
28. उ॒प॒दधा॒तीत्यु॑प - दधा॑ति । \newline
29. स्व मे॒वैव स्वꣳ स्व मे॒व । \newline
30. ए॒व मेध॒म् मेध॑ मे॒वैव मेध᳚म् । \newline
31. मेध॒म् पश्य॑न्तः॒ पश्य॑न्तो॒ मेध॒म् मेध॒म् पश्य॑न्तः । \newline
32. पश्य॑न्तः प॒शवः॑ प॒शवः॒ पश्य॑न्तः॒ पश्य॑न्तः प॒शवः॑ । \newline
33. प॒शव॒ उपोप॑ प॒शवः॑ प॒शव॒ उप॑ । \newline
34. उप॑ तिष्ठन्ते तिष्ठन्त॒ उपोप॑ तिष्ठन्ते । \newline
35. ति॒ष्ठ॒न्ते॒ श्म॒शा॒नꣳ श्म॑शा॒नम् ति॑ष्ठन्ते तिष्ठन्ते श्मशा॒नम् । \newline
36. श्म॒शा॒नं ॅवै वै श्म॑शा॒नꣳ श्म॑शा॒नं ॅवै । \newline
37. वा ए॒त दे॒तद् वै वा ए॒तत् । \newline
38. ए॒तत् क्रि॑यते क्रियत ए॒त दे॒तत् क्रि॑यते । \newline
39. क्रि॒य॒ते॒ यद् यत् क्रि॑यते क्रियते॒ यत् । \newline
40. यन् मृ॒ताना᳚म् मृ॒तानां॒ ॅयद् यन् मृ॒ताना᳚म् । \newline
41. मृ॒ताना᳚म् पशू॒नाम् प॑शू॒नाम् मृ॒ताना᳚म् मृ॒ताना᳚म् पशू॒नाम् । \newline
42. प॒शू॒नाꣳ शी॒र्॒.षाणि॑ शी॒र्॒.षाणि॑ पशू॒नाम् प॑शू॒नाꣳ शी॒र्॒.षाणि॑ । \newline
43. शी॒र्॒.षा ण्यु॑पधी॒यन्त॑ उपधी॒यन्ते॑ शी॒र्॒.षाणि॑ शी॒र्॒.षा ण्यु॑पधी॒यन्ते᳚ । \newline
44. उ॒प॒धी॒यन्ते॒ यद् यदु॑पधी॒यन्त॑ उपधी॒यन्ते॒ यत् । \newline
45. उ॒प॒धी॒यन्त॒ इत्यु॑प - धी॒यन्ते᳚ । \newline
46. यज् जीव॑न्त॒म् जीव॑न्तं॒ ॅयद् यज् जीव॑न्तम् । \newline
47. जीव॑न्तम् कू॒र्मम् कू॒र्मम् जीव॑न्त॒म् जीव॑न्तम् कू॒र्मम् । \newline
48. कू॒र्म मु॑प॒दधा᳚ त्युप॒दधा॑ति कू॒र्मम् कू॒र्म मु॑प॒दधा॑ति । \newline
49. उ॒प॒दधा॑ति॒ तेन॒ तेनो॑प॒दधा᳚ त्युप॒दधा॑ति॒ तेन॑ । \newline
50. उ॒प॒दधा॒तीत्यु॑प - दधा॑ति । \newline
51. तेना श्म॑शानचि॒ दश्म॑शानचि॒त् तेन॒ तेना श्म॑शानचित् । \newline
52. अश्म॑शानचिद् वास्त॒व्यो॑ वास्त॒व्यो ऽश्म॑शानचि॒ दश्म॑शानचिद् वास्त॒व्यः॑ । \newline
53. अश्म॑शानचि॒दित्यश्म॑शान - चि॒त् । \newline
54. वा॒स्त॒व्यो॑ वै वै वा᳚स्त॒व्यो॑ वास्त॒व्यो॑ वै । \newline
55. वा ए॒ष ए॒ष वै वा ए॒षः । \newline
56. ए॒ष यद् यदे॒ष ए॒ष यत् । \newline
57. यत् कू॒र्मः कू॒र्मो यद् यत् कू॒र्मः । \newline

\textbf{Ghana Paata } \newline

1. यत् कू॒र्मम् कू॒र्मं ॅयद् यत् कू॒र्म मु॑प॒दधा᳚ त्युप॒दधा॑ति कू॒र्मं ॅयद् यत् कू॒र्म मु॑प॒दधा॑ति । \newline
2. कू॒र्म मु॑प॒दधा᳚ त्युप॒दधा॑ति कू॒र्मम् कू॒र्म मु॑प॒दधा॑ति॒ यथा॒ यथो॑प॒दधा॑ति कू॒र्मम् कू॒र्म मु॑प॒दधा॑ति॒ यथा᳚ । \newline
3. उ॒प॒दधा॑ति॒ यथा॒ यथो॑प॒दधा᳚ त्युप॒दधा॑ति॒ यथा᳚ क्षेत्र॒वित् क्षे᳚त्र॒विद् यथो॑प॒दधा᳚त्युप॒दधा॑ति॒ यथा᳚ क्षेत्र॒वित् । \newline
4. उ॒प॒दधा॒तीत्यु॑प - दधा॑ति । \newline
5. यथा᳚ क्षेत्र॒वित् क्षे᳚त्र॒विद् यथा॒ यथा᳚ क्षेत्र॒वि दञ्ज॒सा ऽञ्ज॑सा क्षेत्र॒विद् यथा॒ यथा᳚ क्षेत्र॒वि दञ्ज॑सा । \newline
6. क्षे॒त्र॒वि दञ्ज॒सा ऽञ्ज॑सा क्षेत्र॒वित् क्षे᳚त्र॒वि दञ्ज॑सा॒ नय॑ति॒ नय॒ त्यञ्ज॑सा क्षेत्र॒वित् क्षे᳚त्र॒वि दञ्ज॑सा॒ नय॑ति । \newline
7. क्षे॒त्र॒विदिति॑ क्षेत्र - वित् । \newline
8. अञ्ज॑सा॒ नय॑ति॒ नय॒ त्यञ्ज॒सा ऽञ्ज॑सा॒ नय॑ त्ये॒व मे॒वम् नय॒ त्यञ्ज॒सा ऽञ्ज॑सा॒ नय॑ त्ये॒वम् । \newline
9. नय॑त्ये॒व मे॒वम् नय॑ति॒ नय॑ त्ये॒व मे॒वै वैवम् नय॑ति॒ नय॑ त्ये॒व मे॒व । \newline
10. ए॒व मे॒वैवैव मे॒व मे॒वैन॑ मेन मे॒वैव मे॒व मे॒वैन᳚म् । \newline
11. ए॒वैन॑ मेन मे॒वैवैन॑म् कू॒र्मः कू॒र्म ए॑न मे॒वैवैन॑म् कू॒र्मः । \newline
12. ए॒न॒म् कू॒र्मः कू॒र्म ए॑न मेनम् कू॒र्मः सु॑व॒र्गꣳ सु॑व॒र्गम् कू॒र्म ए॑न मेनम् कू॒र्मः सु॑व॒र्गम् । \newline
13. कू॒र्मः सु॑व॒र्गꣳ सु॑व॒र्गम् कू॒र्मः कू॒र्मः सु॑व॒र्गम् ॅलो॒कम् ॅलो॒कꣳ सु॑व॒र्गम् कू॒र्मः कू॒र्मः सु॑व॒र्गम् ॅलो॒कम् । \newline
14. सु॒व॒र्गम् ॅलो॒कम् ॅलो॒कꣳ सु॑व॒र्गꣳ सु॑व॒र्गम् ॅलो॒क मञ्ज॒सा ऽञ्ज॑सा लो॒कꣳ सु॑व॒र्गꣳ सु॑व॒र्गम् ॅलो॒क मञ्ज॑सा । \newline
15. सु॒व॒र्गमिति॑ सुवः - गम् । \newline
16. लो॒क मञ्ज॒सा ऽञ्ज॑सा लो॒कम् ॅलो॒क मञ्ज॑सा नयति नय॒ त्यञ्ज॑सा लो॒कम् ॅलो॒क मञ्ज॑सा नयति । \newline
17. अञ्ज॑सा नयति नय॒ त्यञ्ज॒सा ऽञ्ज॑सा नयति॒ मेधो॒ मेधो॑ नय॒ त्यञ्ज॒सा ऽञ्ज॑सा नयति॒ मेधः॑ । \newline
18. न॒य॒ति॒ मेधो॒ मेधो॑ नयति नयति॒ मेधो॒ वै वै मेधो॑ नयति नयति॒ मेधो॒ वै । \newline
19. मेधो॒ वै वै मेधो॒ मेधो॒ वा ए॒ष ए॒ष वै मेधो॒ मेधो॒ वा ए॒षः । \newline
20. वा ए॒ष ए॒ष वै वा ए॒ष प॑शू॒नाम् प॑शू॒ना मे॒ष वै वा ए॒ष प॑शू॒नाम् । \newline
21. ए॒ष प॑शू॒नाम् प॑शू॒ना मे॒ष ए॒ष प॑शू॒नां ॅयद् यत् प॑शू॒ना मे॒ष ए॒ष प॑शू॒नां ॅयत् । \newline
22. प॒शू॒नां ॅयद् यत् प॑शू॒नाम् प॑शू॒नां ॅयत् कू॒र्मः कू॒र्मो यत् प॑शू॒नाम् प॑शू॒नां ॅयत् कू॒र्मः । \newline
23. यत् कू॒र्मः कू॒र्मो यद् यत् कू॒र्मो यद् यत् कू॒र्मो यद् यत् कू॒र्मो यत् । \newline
24. कू॒र्मो यद् यत् कू॒र्मः कू॒र्मो यत् कू॒र्मम् कू॒र्मं ॅयत् कू॒र्मः कू॒र्मो यत् कू॒र्मम् । \newline
25. यत् कू॒र्मम् कू॒र्मं ॅयद् यत् कू॒र्म मु॑प॒दधा᳚ त्युप॒दधा॑ति कू॒र्मं ॅयद् यत् कू॒र्म मु॑प॒दधा॑ति । \newline
26. कू॒र्म मु॑प॒दधा᳚ त्युप॒दधा॑ति कू॒र्मम् कू॒र्म मु॑प॒दधा॑ति॒ स्वꣳ स्व मु॑प॒दधा॑ति कू॒र्मम् कू॒र्म मु॑प॒दधा॑ति॒ स्वम् । \newline
27. उ॒प॒दधा॑ति॒ स्वꣳ स्व मु॑प॒दधा᳚ त्युप॒दधा॑ति॒ स्व मे॒वैव स्व मु॑प॒दधा᳚ त्युप॒दधा॑ति॒ स्व मे॒व । \newline
28. उ॒प॒दधा॒तीत्यु॑प - दधा॑ति । \newline
29. स्व मे॒वैव स्वꣳ स्व मे॒व मेध॒म् मेध॑ मे॒व स्वꣳ स्व मे॒व मेध᳚म् । \newline
30. ए॒व मेध॒म् मेध॑ मे॒वैव मेध॒म् पश्य॑न्तः॒ पश्य॑न्तो॒ मेध॑ मे॒वैव मेध॒म् पश्य॑न्तः । \newline
31. मेध॒म् पश्य॑न्तः॒ पश्य॑न्तो॒ मेध॒म् मेध॒म् पश्य॑न्तः प॒शवः॑ प॒शवः॒ पश्य॑न्तो॒ मेध॒म् मेध॒म् पश्य॑न्तः प॒शवः॑ । \newline
32. पश्य॑न्तः प॒शवः॑ प॒शवः॒ पश्य॑न्तः॒ पश्य॑न्तः प॒शव॒ उपोप॑ प॒शवः॒ पश्य॑न्तः॒ पश्य॑न्तः प॒शव॒ उप॑ । \newline
33. प॒शव॒ उपोप॑ प॒शवः॑ प॒शव॒ उप॑ तिष्ठन्ते तिष्ठन्त॒ उप॑ प॒शवः॑ प॒शव॒ उप॑ तिष्ठन्ते । \newline
34. उप॑ तिष्ठन्ते तिष्ठन्त॒ उपोप॑ तिष्ठन्ते श्मशा॒नꣳ श्म॑शा॒नम् ति॑ष्ठन्त॒ उपोप॑ तिष्ठन्ते श्मशा॒नम् । \newline
35. ति॒ष्ठ॒न्ते॒ श्म॒शा॒नꣳ श्म॑शा॒नम् ति॑ष्ठन्ते तिष्ठन्ते श्मशा॒नं ॅवै वै श्म॑शा॒नम् ति॑ष्ठन्ते तिष्ठन्ते श्मशा॒नं ॅवै । \newline
36. श्म॒शा॒नं ॅवै वै श्म॑शा॒नꣳ श्म॑शा॒नं ॅवा ए॒त दे॒तद् वै श्म॑शा॒नꣳ श्म॑शा॒नं ॅवा ए॒तत् । \newline
37. वा ए॒त दे॒तद् वै वा ए॒तत् क्रि॑यते क्रियत ए॒तद् वै वा ए॒तत् क्रि॑यते । \newline
38. ए॒तत् क्रि॑यते क्रियत ए॒त दे॒तत् क्रि॑यते॒ यद् यत् क्रि॑यत ए॒त दे॒तत् क्रि॑यते॒ यत् । \newline
39. क्रि॒य॒ते॒ यद् यत् क्रि॑यते क्रियते॒ यन् मृ॒ताना᳚म् मृ॒तानां॒ ॅयत् क्रि॑यते क्रियते॒ यन् मृ॒ताना᳚म् । \newline
40. यन् मृ॒ताना᳚म् मृ॒तानां॒ ॅयद् यन् मृ॒ताना᳚म् पशू॒नाम् प॑शू॒नाम् मृ॒तानां॒ ॅयद् यन् मृ॒ताना᳚म् पशू॒नाम् । \newline
41. मृ॒ताना᳚म् पशू॒नाम् प॑शू॒नाम् मृ॒ताना᳚म् मृ॒ताना᳚म् पशू॒नाꣳ शी॒र्॒.षाणि॑ शी॒र्॒.षाणि॑ पशू॒नाम् मृ॒ताना᳚म् मृ॒ताना᳚म् पशू॒नाꣳ शी॒र्॒.षाणि॑ । \newline
42. प॒शू॒नाꣳ शी॒र्॒.षाणि॑ शी॒र्॒.षाणि॑ पशू॒नाम् प॑शू॒नाꣳ शी॒र्॒.षा ण्यु॑पधी॒यन्त॑ उपधी॒यन्ते॑ शी॒र्॒.षाणि॑ पशू॒नाम् प॑शू॒नाꣳ शी॒र्॒.षा ण्यु॑पधी॒यन्ते᳚ । \newline
43. शी॒र्॒.षा ण्यु॑पधी॒यन्त॑ उपधी॒यन्ते॑ शी॒र्॒.षाणि॑ शी॒र्॒.षा ण्यु॑पधी॒यन्ते॒ यद् यदु॑पधी॒यन्ते॑ शी॒र्॒.षाणि॑ शी॒र्॒.षा ण्यु॑पधी॒यन्ते॒ यत् । \newline
44. उ॒प॒धी॒यन्ते॒ यद् यदु॑पधी॒यन्त॑ उपधी॒यन्ते॒ यज् जीव॑न्त॒म् जीव॑न्तं॒ ॅयदु॑पधी॒यन्त॑ उपधी॒यन्ते॒ यज् जीव॑न्तम् । \newline
45. उ॒प॒धी॒यन्त॒ इत्यु॑प - धी॒यन्ते᳚ । \newline
46. यज् जीव॑न्त॒म् जीव॑न्तं॒ ॅयद् यज् जीव॑न्तम् कू॒र्मम् कू॒र्मम् जीव॑न्तं॒ ॅयद् यज् जीव॑न्तम् कू॒र्मम् । \newline
47. जीव॑न्तम् कू॒र्मम् कू॒र्मम् जीव॑न्त॒म् जीव॑न्तम् कू॒र्म मु॑प॒दधा᳚ त्युप॒दधा॑ति कू॒र्मम् जीव॑न्त॒म् जीव॑न्तम् कू॒र्म मु॑प॒दधा॑ति । \newline
48. कू॒र्म मु॑प॒दधा᳚ त्युप॒दधा॑ति कू॒र्मम् कू॒र्म मु॑प॒दधा॑ति॒ तेन॒ तेनो॑प॒दधा॑ति कू॒र्मम् कू॒र्म मु॑प॒दधा॑ति॒ तेन॑ । \newline
49. उ॒प॒दधा॑ति॒ तेन॒ तेनो॑प॒दधा᳚ त्युप॒दधा॑ति॒ तेनाश्म॑शानचि॒ दश्म॑शानचि॒त् तेनो॑प॒दधा᳚त्युप॒दधा॑ति॒ तेनाश्म॑शानचित् । \newline
50. उ॒प॒दधा॒तीत्यु॑प - दधा॑ति । \newline
51. तेनाश्म॑शानचि॒ दश्म॑शानचि॒त् तेन॒ तेनाश्म॑शानचिद् वास्त॒व्यो॑ वास्त॒व्यो ऽश्म॑शानचि॒त् तेन॒ तेनाश्म॑शानचिद् वास्त॒व्यः॑ । \newline
52. अश्म॑शानचिद् वास्त॒व्यो॑ वास्त॒व्यो ऽश्म॑शानचि॒ दश्म॑शानचिद् वास्त॒व्यो॑ वै वै वा᳚स्त॒व्यो ऽश्म॑शानचि॒ दश्म॑शानचिद् वास्त॒व्यो॑ वै । \newline
53. अश्म॑शानचि॒दित्यश्म॑शान - चि॒त् । \newline
54. वा॒स्त॒व्यो॑ वै वै वा᳚स्त॒व्यो॑ वास्त॒व्यो॑ वा ए॒ष ए॒ष वै वा᳚स्त॒व्यो॑ वास्त॒व्यो॑ वा ए॒षः । \newline
55. वा ए॒ष ए॒ष वै वा ए॒ष यद् यदे॒ष वै वा ए॒ष यत् । \newline
56. ए॒ष यद् यदे॒ष ए॒ष यत् कू॒र्मः कू॒र्मो यदे॒ष ए॒ष यत् कू॒र्मः । \newline
57. यत् कू॒र्मः कू॒र्मो यद् यत् कू॒र्मो मधु॒ मधु॑ कू॒र्मो यद् यत् कू॒र्मो मधु॑ । \newline
\pagebreak
\markright{ TS 5.2.8.6  \hfill https://www.vedavms.in \hfill}

\section{ TS 5.2.8.6 }

\textbf{TS 5.2.8.6 } \newline
\textbf{Samhita Paata} \newline

कू॒र्मो मधु॒ वाता॑ ऋताय॒त इति॑ द॒द्ध्ना म॑धुमि॒श्रेणा॒भ्य॑नक्ति स्व॒दय॑त्ये॒वैनं॑ ग्रा॒म्यं ॅवा ए॒तदन्नं॒ ॅयद्-दद्ध्या॑र॒ण्यं मधु॒ यद्द॒द्ध्ना म॑धुमि॒श्रेणा᳚ भ्य॒नक्त्यु॒भय॒स्या ऽव॑रुद्ध्यै म॒ही द्यौः पृ॑थि॒वी च॑ न॒ इत्या॑हा॒ ऽऽभ्यामे॒वैन॑मुभ॒यतः॒ परि॑गृह्णाति॒ प्राञ्च॒मुप॑ दधाति सुव॒र्गस्य॑ लो॒कस्य॒ सम॑ष्ट्यै पु॒रस्ता᳚त् प्र॒त्यञ्च॒मुप॑ दधाति॒ तस्मा᳚त् - [  ] \newline

\textbf{Pada Paata} \newline

कू॒र्मः । मधु॑ । वाताः᳚ । ऋ॒ता॒य॒त इत्यृ॑त - य॒ते । इति॑ । द॒द्ध्ना । म॒धु॒मि॒श्रेणेति॑ मधु - मि॒श्रेण॑ । अ॒भीति॑ । अ॒न॒क्ति॒ । स्व॒दय॑ति । ए॒व । ए॒न॒म् । ग्रा॒म्यम् । वै । ए॒तत् । अन्न᳚म् । यत् । दधि॑ । आ॒र॒ण्यम् । मधु॑ । यत् । द॒द्ध्ना । म॒धु॒मि॒श्रेणेति॑ मधु - मि॒श्रेण॑ । अ॒भ्य॒नक्तीत्य॑भि - अ॒नक्ति॑ । उ॒भय॑स्य । अव॑रुद्ध्या॒ इत्यव॑-रु॒द्ध्यै॒ । म॒ही । द्यौः । पृ॒थि॒वी । च॒ । नः॒ । इति॑ । आ॒ह॒ । आ॒भ्याम् । ए॒व । ए॒न॒म् । उ॒भ॒यतः॑ । परीति॑ । गृ॒ह्णा॒ति॒ । प्राञ्च᳚म् । उपेति॑ । द॒धा॒ति॒ । सु॒व॒र्गस्येति॑ सुवः - गस्य॑ । लो॒कस्य॑ । सम॑ष्ट्या॒ इति॒ सं-अ॒ष्ट्यै॒ । पु॒रस्ता᳚त् । प्र॒त्यञ्च᳚म् । उपेति॑ । द॒धा॒ति॒ । तस्मा᳚त् ।  \newline


\textbf{Krama Paata} \newline

कू॒र्मो मधु॑ । मधु॒ वाताः᳚ । वाता॑ ऋताय॒ते । ऋ॒ता॒य॒त इति॑ । ऋ॒ता॒य॒त इत्यृ॑त - य॒ते । इति॑ द॒द्ध्ना । द॒द्ध्ना म॑धुमि॒श्रेण॑ । म॒धु॒मि॒श्रेणा॒भि । म॒धु॒मि॒श्रेणेति॑ मधु - मि॒श्रेण॑ । अ॒भ्य॑नक्ति । अ॒न॒क्ति॒ स्व॒दय॑ति । स्व॒दय॑त्ये॒व । ए॒वैन᳚म् । ए॒न॒म् ग्रा॒म्यम् । ग्रा॒म्यम् ॅवै । वा ए॒तत् । ए॒तदन्न᳚म् । अन्न॒म् ॅयत् । यद् दद्धि॑ । दद्ध्या॑र॒ण्यम् । आ॒र॒ण्यम् मधु॑ । मधु॒ यत् । यद् द॒द्ध्ना । द॒द्ध्ना म॑धुमि॒श्रेण॑ । म॒धु॒मि॒श्रेणा᳚भ्य॒नक्ति॑ । म॒धु॒मि॒श्रेणेति॑ मधु - मि॒श्रेण॑ । अ॒भ्य॒नक्त्यु॒भय॑स्य । अ॒भ्य॒नक्तीत्य॑भि - अ॒नक्ति॑ । उ॒भय॒स्याव॑रुद्ध्यै । अव॑रुद्ध्यै म॒ही । अव॑रुद्ध्या॒ इत्यव॑ - रु॒द्ध्यै॒ । म॒ही द्यौः । द्यौः पृ॑थि॒वी । पृ॒थि॒वी च॑ । च॒ नः॒ । न॒ इति॑ । इत्या॑ह । आ॒हा॒भ्याम् । आ॒भ्यामे॒व । ए॒वैन᳚म् । ए॒न॒मु॒भ॒यतः॑ । उ॒भ॒यतः॒ परि॑ । परि॑ गृह्णाति । गृ॒ह्णा॒ति॒ प्राञ्च᳚म् । प्राञ्च॒मुप॑ । उप॑ दधाति । द॒धा॒ति॒ सु॒व॒र्गस्य॑ । सु॒व॒र्गस्य॑ लो॒कस्य॑ । सु॒व॒र्गस्येति॑ सुवः - गस्य॑ । लो॒कस्य॒ सम॑ष्ट्यै । सम॑ष्ट्यै पु॒रस्ता᳚त् । सम॑ष्ट्या॒ इति॒ सम् - अ॒ष्ट्यै॒ । पु॒रस्ता᳚त् प्र॒त्यञ्च᳚म् । प्र॒त्यञ्च॒मुप॑ । उप॑ दधाति । द॒धा॒ति॒ तस्मा᳚त् । तस्मा᳚त् पु॒रस्ता᳚त् \newline

\textbf{Jatai Paata} \newline

1. कू॒र्मो मधु॒ मधु॑ कू॒र्मः कू॒र्मो मधु॑ । \newline
2. मधु॒ वाता॒ वाता॒ मधु॒ मधु॒ वाताः᳚ । \newline
3. वाता॑ ऋताय॒त ऋ॑ताय॒ते वाता॒ वाता॑ ऋताय॒ते । \newline
4. ऋ॒ता॒य॒त इती त्यृ॑ताय॒त ऋ॑ताय॒त इति॑ । \newline
5. ऋ॒ता॒य॒त इत्यृ॑त - य॒ते । \newline
6. इति॑ द॒द्ध्ना द॒द्ध्नेतीति॑ द॒द्ध्ना । \newline
7. द॒द्ध्ना म॑धुमि॒श्रेण॑ मधुमि॒श्रेण॑ द॒द्ध्ना द॒द्ध्ना म॑धुमि॒श्रेण॑ । \newline
8. म॒धु॒मि॒श्रेणा॒ भ्य॑भि म॑धुमि॒श्रेण॑ मधुमि॒श्रेणा॒भि । \newline
9. म॒धु॒मि॒श्रेणेति॑ मधु - मि॒श्रेण॑ । \newline
10. अ॒भ्य॑ नक्त्यनक् त्य॒भ्या᳚(1॒)भ्य॑नक्ति । \newline
11. अ॒न॒क्ति॒ स्व॒दय॑ति स्व॒दय॑ त्यनक् त्यनक्ति स्व॒दय॑ति । \newline
12. स्व॒दय॑ त्ये॒वैव स्व॒दय॑ति स्व॒दय॑ त्ये॒व । \newline
13. ए॒वैन॑ मेन मे॒वै वैन᳚म् । \newline
14. ए॒न॒म् ग्रा॒म्यम् ग्रा॒म्य मे॑न मेनम् ग्रा॒म्यम् । \newline
15. ग्रा॒म्यं ॅवै वै ग्रा॒म्यम् ग्रा॒म्यं ॅवै । \newline
16. वा ए॒त दे॒तद् वै वा ए॒तत् । \newline
17. ए॒त दन्न॒ मन्न॑ मे॒त दे॒त दन्न᳚म् । \newline
18. अन्नं॒ ॅयद् यदन्न॒ मन्नं॒ ॅयत् । \newline
19. यद् दधि॒ दधि॒ यद् यद् दधि॑ । \newline
20. दध्या॑ र॒ण्य मा॑र॒ण्यम् दधि॒ दध्या॑ र॒ण्यम् । \newline
21. आ॒र॒ण्यम् मधु॒ मध्वा॑र॒ण्य मा॑र॒ण्यम् मधु॑ । \newline
22. मधु॒ यद् यन् मधु॒ मधु॒ यत् । \newline
23. यद् द॒द्ध्ना द॒द्ध्ना यद् यद् द॒द्ध्ना । \newline
24. द॒द्ध्ना म॑धुमि॒श्रेण॑ मधुमि॒श्रेण॑ द॒द्ध्ना द॒द्ध्ना म॑धुमि॒श्रेण॑ । \newline
25. म॒धु॒मि॒श्रेणा᳚ भ्य॒नक्त्य॑ भ्य॒नक्ति॑ मधुमि॒श्रेण॑ मधुमि॒श्रेणा᳚ भ्य॒नक्ति॑ । \newline
26. म॒धु॒मि॒श्रेणेति॑ मधु - मि॒श्रेण॑ । \newline
27. अ॒भ्य॒नक् त्यु॒भय॑स्यो॒ भय॑स्या भ्य॒नक्त्य॑ भ्य॒नक् त्यु॒भय॑स्य । \newline
28. अ॒भ्य॒नक्तीत्य॑भि - अ॒नक्ति॑ । \newline
29. उ॒भय॒स्या व॑रुद्ध्या॒ अव॑रुद्ध्या उ॒भय॑स्यो॒ भय॒स्या व॑रुद्ध्यै । \newline
30. अव॑रुद्ध्यै म॒ही म॒ह्य व॑रुद्ध्या॒ अव॑रुद्ध्यै म॒ही । \newline
31. अव॑रुद्ध्या॒ इत्यव॑ - रु॒द्ध्यै॒ । \newline
32. म॒ही द्यौर् द्यौर् म॒ही म॒ही द्यौः । \newline
33. द्यौः पृ॑थि॒वी पृ॑थि॒वी द्यौर् द्यौः पृ॑थि॒वी । \newline
34. पृ॒थि॒वी च॑ च पृथि॒वी पृ॑थि॒वी च॑ । \newline
35. च॒ नो॒ न॒श्च॒ च॒ नः॒ । \newline
36. न॒ इतीति॑ नो न॒ इति॑ । \newline
37. इत्या॑ हा॒हेती त्या॑ह । \newline
38. आ॒हा॒भ्या मा॒भ्या मा॑हा हा॒भ्याम् । \newline
39. आ॒भ्या मे॒वै वाभ्या मा॒भ्या मे॒व । \newline
40. ए॒वैन॑ मेन मे॒वै वैन᳚म् । \newline
41. ए॒न॒ मु॒भ॒यत॑ उभ॒यत॑ एन मेन मुभ॒यतः॑ । \newline
42. उ॒भ॒यतः॒ परि॒ पर्यु॑भ॒यत॑ उभ॒यतः॒ परि॑ । \newline
43. परि॑ गृह्णाति गृह्णाति॒ परि॒ परि॑ गृह्णाति । \newline
44. गृ॒ह्णा॒ति॒ प्राञ्च॒म् प्राञ्च॑म् गृह्णाति गृह्णाति॒ प्राञ्च᳚म् । \newline
45. प्राञ्च॒ मुपोप॒ प्राञ्च॒म् प्राञ्च॒ मुप॑ । \newline
46. उप॑ दधाति दधा॒ त्युपोप॑ दधाति । \newline
47. द॒धा॒ति॒ सु॒व॒र्गस्य॑ सुव॒र्गस्य॑ दधाति दधाति सुव॒र्गस्य॑ । \newline
48. सु॒व॒र्गस्य॑ लो॒कस्य॑ लो॒कस्य॑ सुव॒र्गस्य॑ सुव॒र्गस्य॑ लो॒कस्य॑ । \newline
49. सु॒व॒र्गस्येति॑ सुवः - गस्य॑ । \newline
50. लो॒कस्य॒ सम॑ष्ट्यै॒ सम॑ष्ट्यै लो॒कस्य॑ लो॒कस्य॒ सम॑ष्ट्यै । \newline
51. सम॑ष्ट्यै पु॒रस्ता᳚त् पु॒रस्ता॒थ् सम॑ष्ट्यै॒ सम॑ष्ट्यै पु॒रस्ता᳚त् । \newline
52. सम॑ष्ट्या॒ इति॒ सं - अ॒ष्ट्यै॒ । \newline
53. पु॒रस्ता᳚त् प्र॒त्यञ्च॑म् प्र॒त्यञ्च॑म् पु॒रस्ता᳚त् पु॒रस्ता᳚त् प्र॒त्यञ्च᳚म् । \newline
54. प्र॒त्यञ्च॒ मुपोप॑ प्र॒त्यञ्च॑म् प्र॒त्यञ्च॒ मुप॑ । \newline
55. उप॑ दधाति दधा॒ त्युपोप॑ दधाति । \newline
56. द॒धा॒ति॒ तस्मा॒त् तस्मा᳚द् दधाति दधाति॒ तस्मा᳚त् । \newline
57. तस्मा᳚त् पु॒रस्ता᳚त् पु॒रस्ता॒त् तस्मा॒त् तस्मा᳚त् पु॒रस्ता᳚त् । \newline

\textbf{Ghana Paata } \newline

1. कू॒र्मो मधु॒ मधु॑ कू॒र्मः कू॒र्मो मधु॒ वाता॒ वाता॒ मधु॑ कू॒र्मः कू॒र्मो मधु॒ वाताः᳚ । \newline
2. मधु॒ वाता॒ वाता॒ मधु॒ मधु॒ वाता॑ ऋताय॒त ऋ॑ताय॒ते वाता॒ मधु॒ मधु॒ वाता॑ ऋताय॒ते । \newline
3. वाता॑ ऋताय॒त ऋ॑ताय॒ते वाता॒ वाता॑ ऋताय॒त इतीत्यृ॑ताय॒ते वाता॒ वाता॑ ऋताय॒त इति॑ । \newline
4. ऋ॒ता॒य॒त इती त्यृ॑ताय॒त ऋ॑ताय॒त इति॑ द॒द्ध्ना द॒द्ध्ने त्यृ॑ताय॒त ऋ॑ताय॒त इति॑ द॒द्ध्ना । \newline
5. ऋ॒ता॒य॒त इत्यृ॑त - य॒ते । \newline
6. इति॑ द॒द्ध्ना द॒द्ध्नेतीति॑ द॒द्ध्ना म॑धुमि॒श्रेण॑ मधुमि॒श्रेण॑ द॒द्ध्नेतीति॑ द॒द्ध्ना म॑धुमि॒श्रेण॑ । \newline
7. द॒द्ध्ना म॑धुमि॒श्रेण॑ मधुमि॒श्रेण॑ द॒द्ध्ना द॒द्ध्ना म॑धुमि॒श्रेणा॒ भ्य॑भि म॑धुमि॒श्रेण॑ द॒द्ध्ना द॒द्ध्ना म॑धुमि॒श्रेणा॒भि । \newline
8. म॒धु॒मि॒श्रेणा॒ भ्य॑भि म॑धुमि॒श्रेण॑ मधुमि॒श्रेणा॒ भ्य॑नक् त्यनक्त्य॒भि म॑धुमि॒श्रेण॑ मधुमि॒श्रेणा॒ भ्य॑नक्ति । \newline
9. म॒धु॒मि॒श्रेणेति॑ मधु - मि॒श्रेण॑ । \newline
10. अ॒भ्य॑नक् त्यनक्त्य॒भ्या᳚(1॒)भ्य॑नक्ति स्व॒दय॑ति स्व॒दय॑ त्यनक्त्य॒भ्या᳚(1॒)भ्य॑नक्ति स्व॒दय॑ति । \newline
11. अ॒न॒क्ति॒ स्व॒दय॑ति स्व॒दय॑ त्यनक्त्यनक्ति स्व॒दय॑ त्ये॒वैव स्व॒दय॑ त्यनक् त्यनक्ति स्व॒दय॑ त्ये॒व । \newline
12. स्व॒दय॑ त्ये॒वैव स्व॒दय॑ति स्व॒दय॑ त्ये॒वैन॑ मेन मे॒व स्व॒दय॑ति स्व॒दय॑ त्ये॒वैन᳚म् । \newline
13. ए॒वैन॑ मेन मे॒वैवैन॑म् ग्रा॒म्यम् ग्रा॒म्य मे॑न मे॒वैवैन॑म् ग्रा॒म्यम् । \newline
14. ए॒न॒म् ग्रा॒म्यम् ग्रा॒म्य मे॑न मेनम् ग्रा॒म्यं ॅवै वै ग्रा॒म्य मे॑न मेनम् ग्रा॒म्यं ॅवै । \newline
15. ग्रा॒म्यं ॅवै वै ग्रा॒म्यम् ग्रा॒म्यं ॅवा ए॒त दे॒तद् वै ग्रा॒म्यम् ग्रा॒म्यं ॅवा ए॒तत् । \newline
16. वा ए॒त दे॒तद् वै वा ए॒त दन्न॒ मन्न॑ मे॒तद् वै वा ए॒त दन्न᳚म् । \newline
17. ए॒त दन्न॒ मन्न॑ मे॒त दे॒त दन्नं॒ ॅयद् यदन्न॑ मे॒त दे॒त दन्नं॒ ॅयत् । \newline
18. अन्नं॒ ॅयद् यदन्न॒ मन्नं॒ ॅयद् दधि॒ दधि॒ यदन्न॒ मन्नं॒ ॅयद् दधि॑ । \newline
19. यद् दधि॒ दधि॒ यद् यद् दध्या॑र॒ण्य मा॑र॒ण्यम् दधि॒ यद् यद् दध्या॑र॒ण्यम् । \newline
20. दध्या॑र॒ण्य मा॑र॒ण्यम् दधि॒ दध्या॑र॒ण्यम् मधु॒ मध्वा॑र॒ण्यम् दधि॒ दध्या॑र॒ण्यम् मधु॑ । \newline
21. आ॒र॒ण्यम् मधु॒ मध्वा॑र॒ण्य मा॑र॒ण्यम् मधु॒ यद् यन् मध्वा॑र॒ण्य मा॑र॒ण्यम् मधु॒ यत् । \newline
22. मधु॒ यद् यन् मधु॒ मधु॒ यद् द॒द्ध्ना द॒द्ध्ना यन् मधु॒ मधु॒ यद् द॒द्ध्ना । \newline
23. यद् द॒द्ध्ना द॒द्ध्ना यद् यद् द॒द्ध्ना म॑धुमि॒श्रेण॑ मधुमि॒श्रेण॑ द॒द्ध्ना यद् यद् द॒द्ध्ना म॑धुमि॒श्रेण॑ । \newline
24. द॒द्ध्ना म॑धुमि॒श्रेण॑ मधुमि॒श्रेण॑ द॒द्ध्ना द॒द्ध्ना म॑धुमि॒श्रेणा᳚ भ्य॒नक् त्य॑भ्य॒नक्ति॑ मधुमि॒श्रेण॑ द॒द्ध्ना द॒द्ध्ना म॑धुमि॒श्रेणा᳚ भ्य॒नक्ति॑ । \newline
25. म॒धु॒मि॒श्रेणा᳚ भ्य॒नक् त्य॑भ्य॒नक्ति॑ मधुमि॒श्रेण॑ मधुमि॒श्रेणा᳚ भ्य॒नक् त्यु॒भय॑ स्यो॒भय॑स्या भ्य॒नक्ति॑ मधुमि॒श्रेण॑ मधुमि॒श्रेणा᳚ भ्य॒नक् त्यु॒भय॑स्य । \newline
26. म॒धु॒मि॒श्रेणेति॑ मधु - मि॒श्रेण॑ । \newline
27. अ॒भ्य॒नक् त्यु॒भय॑स्यो॒ भय॑स्या भ्य॒नक् त्य॑भ्य॒नक् त्यु॒भय॒स्या व॑रुद्ध्या॒ अव॑रुद्ध्या उ॒भय॑स्या भ्य॒नक् त्य॑भ्य॒नक् त्यु॒भय॒स्या व॑रुद्ध्यै । \newline
28. अ॒भ्य॒नक्तीत्य॑भि - अ॒नक्ति॑ । \newline
29. उ॒भय॒स्या व॑रुद्ध्या॒ अव॑रुद्ध्या उ॒भय॑ स्यो॒भय॒स्या व॑रुद्ध्यै म॒ही म॒ह्यव॑रुद्ध्या उ॒भय॑स्यो॒ भय॒स्या व॑रुद्ध्यै म॒ही । \newline
30. अव॑रुद्ध्यै म॒ही म॒ह्यव॑रुद्ध्या॒ अव॑रुद्ध्यै म॒ही द्यौर् द्यौर् म॒ह्यव॑रुद्ध्या॒ अव॑रुद्ध्यै म॒ही द्यौः । \newline
31. अव॑रुद्ध्या॒ इत्यव॑ - रु॒द्ध्यै॒ । \newline
32. म॒ही द्यौर् द्यौर् म॒ही म॒ही द्यौः पृ॑थि॒वी पृ॑थि॒वी द्यौर् म॒ही म॒ही द्यौः पृ॑थि॒वी । \newline
33. द्यौः पृ॑थि॒वी पृ॑थि॒वी द्यौर् द्यौः पृ॑थि॒वी च॑ च पृथि॒वी द्यौर् द्यौः पृ॑थि॒वी च॑ । \newline
34. पृ॒थि॒वी च॑ च पृथि॒वी पृ॑थि॒वी च॑ नो नश्च पृथि॒वी पृ॑थि॒वी च॑ नः । \newline
35. च॒ नो॒ न॒श्च॒ च॒ न॒ इतीति॑ नश्च च न॒ इति॑ । \newline
36. न॒ इतीति॑ नो न॒ इत्या॑हा॒हेति॑ नो न॒ इत्या॑ह । \newline
37. इत्या॑हा॒हे तीत्या॑हा॒भ्या मा॒भ्या मा॒हे तीत्या॑हा॒भ्याम् । \newline
38. आ॒हा॒भ्या मा॒भ्या मा॑हाहा॒भ्या मे॒वैवाभ्या मा॑हाहा॒भ्या मे॒व । \newline
39. आ॒भ्या मे॒वैवाभ्या मा॒भ्या मे॒वैन॑ मेन मे॒वाभ्या मा॒भ्या मे॒वैन᳚म् । \newline
40. ए॒वैन॑ मेन मे॒वैवैन॑ मुभ॒यत॑ उभ॒यत॑ एन मे॒वैवैन॑ मुभ॒यतः॑ । \newline
41. ए॒न॒ मु॒भ॒यत॑ उभ॒यत॑ एन मेन मुभ॒यतः॒ परि॒ पर्यु॑भ॒यत॑ एन मेन मुभ॒यतः॒ परि॑ । \newline
42. उ॒भ॒यतः॒ परि॒ पर्यु॑भ॒यत॑ उभ॒यतः॒ परि॑ गृह्णाति गृह्णाति॒ पर्यु॑भ॒यत॑ उभ॒यतः॒ परि॑ गृह्णाति । \newline
43. परि॑ गृह्णाति गृह्णाति॒ परि॒ परि॑ गृह्णाति॒ प्राञ्च॒म् प्राञ्च॑म् गृह्णाति॒ परि॒ परि॑ गृह्णाति॒ प्राञ्च᳚म् । \newline
44. गृ॒ह्णा॒ति॒ प्राञ्च॒म् प्राञ्च॑म् गृह्णाति गृह्णाति॒ प्राञ्च॒ मुपोप॒ प्राञ्च॑म् गृह्णाति गृह्णाति॒ प्राञ्च॒ मुप॑ । \newline
45. प्राञ्च॒ मुपोप॒ प्राञ्च॒म् प्राञ्च॒ मुप॑ दधाति दधा॒ त्युप॒ प्राञ्च॒म् प्राञ्च॒ मुप॑ दधाति । \newline
46. उप॑ दधाति दधा॒ त्युपोप॑ दधाति सुव॒र्गस्य॑ सुव॒र्गस्य॑ दधा॒ त्युपोप॑ दधाति सुव॒र्गस्य॑ । \newline
47. द॒धा॒ति॒ सु॒व॒र्गस्य॑ सुव॒र्गस्य॑ दधाति दधाति सुव॒र्गस्य॑ लो॒कस्य॑ लो॒कस्य॑ सुव॒र्गस्य॑ दधाति दधाति सुव॒र्गस्य॑ लो॒कस्य॑ । \newline
48. सु॒व॒र्गस्य॑ लो॒कस्य॑ लो॒कस्य॑ सुव॒र्गस्य॑ सुव॒र्गस्य॑ लो॒कस्य॒ सम॑ष्ट्यै॒ सम॑ष्ट्यै लो॒कस्य॑ सुव॒र्गस्य॑ सुव॒र्गस्य॑ लो॒कस्य॒ सम॑ष्ट्यै । \newline
49. सु॒व॒र्गस्येति॑ सुवः - गस्य॑ । \newline
50. लो॒कस्य॒ सम॑ष्ट्यै॒ सम॑ष्ट्यै लो॒कस्य॑ लो॒कस्य॒ सम॑ष्ट्यै पु॒रस्ता᳚त् पु॒रस्ता॒थ् सम॑ष्ट्यै लो॒कस्य॑ लो॒कस्य॒ सम॑ष्ट्यै पु॒रस्ता᳚त् । \newline
51. सम॑ष्ट्यै पु॒रस्ता᳚त् पु॒रस्ता॒थ् सम॑ष्ट्यै॒ सम॑ष्ट्यै पु॒रस्ता᳚त् प्र॒त्यञ्च॑म् प्र॒त्यञ्च॑म् पु॒रस्ता॒थ् सम॑ष्ट्यै॒ सम॑ष्ट्यै पु॒रस्ता᳚त् प्र॒त्यञ्च᳚म् । \newline
52. सम॑ष्ट्या॒ इति॒ सं - अ॒ष्ट्यै॒ । \newline
53. पु॒रस्ता᳚त् प्र॒त्यञ्च॑म् प्र॒त्यञ्च॑म् पु॒रस्ता᳚त् पु॒रस्ता᳚त् प्र॒त्यञ्च॒ मुपोप॑ प्र॒त्यञ्च॑म् पु॒रस्ता᳚त् पु॒रस्ता᳚त् प्र॒त्यञ्च॒ मुप॑ । \newline
54. प्र॒त्यञ्च॒ मुपोप॑ प्र॒त्यञ्च॑म् प्र॒त्यञ्च॒ मुप॑ दधाति दधा॒ त्युप॑ प्र॒त्यञ्च॑म् प्र॒त्यञ्च॒ मुप॑ दधाति । \newline
55. उप॑ दधाति दधा॒ त्युपोप॑ दधाति॒ तस्मा॒त् तस्मा᳚द् दधा॒ त्युपोप॑ दधाति॒ तस्मा᳚त् । \newline
56. द॒धा॒ति॒ तस्मा॒त् तस्मा᳚द् दधाति दधाति॒ तस्मा᳚त् पु॒रस्ता᳚त् पु॒रस्ता॒त् तस्मा᳚द् दधाति दधाति॒ तस्मा᳚त् पु॒रस्ता᳚त् । \newline
57. तस्मा᳚त् पु॒रस्ता᳚त् पु॒रस्ता॒त् तस्मा॒त् तस्मा᳚त् पु॒रस्ता᳚त् प्र॒त्यञ्चः॑ प्र॒त्यञ्चः॑ पु॒रस्ता॒त् तस्मा॒त् तस्मा᳚त् पु॒रस्ता᳚त् प्र॒त्यञ्चः॑ । \newline
\pagebreak
\markright{ TS 5.2.8.7  \hfill https://www.vedavms.in \hfill}

\section{ TS 5.2.8.7 }

\textbf{TS 5.2.8.7 } \newline
\textbf{Samhita Paata} \newline

पु॒रस्ता᳚त् प्र॒त्यञ्चः॑ प॒शवो॒ मेध॒मुप॑ तिष्ठन्ते॒ यो वा अप॑नाभिम॒ग्निं चि॑नु॒ते यज॑मानस्य॒ नाभि॒मनु॒ प्रवि॑शति॒ स ए॑नमीश्व॒रो हिꣳसि॑तोरु॒लूख॑ल॒मुप॑ दधात्ये॒षा वा अ॒ग्नेर्नाभिः॒ सना॑भिमे॒वाऽग्निं चि॑नु॒ते हिꣳ॑साया॒ औदु॑म्बरं भव॒त्यूर्ग्वा उ॑दु॒म्बर॒ ऊर्ज॑मे॒वाव॑ रुन्धे मद्ध्य॒त उप॑ दधाति मद्ध्य॒त ए॒वास्मा॒ ऊर्जं॑ दधाति॒ तस्मा᳚न्-( )-मद्ध्य॒त ऊ॒र्जा भु॑ञ्जत॒ इय॑द्-भवति प्र॒जाप॑तिना यज्ञ्मु॒खेन॒ संमि॑त॒मव॑ ह॒न्त्यन्न॑मे॒वाक॑-र्वैष्ण॒व्यर्चोप॑ दधाति॒ विष्णु॒र्वै य॒ज्ञो वै᳚ष्ण॒वा वन॒स्पत॑यो य॒ज्ञ् ए॒व य॒ज्ञ्ं प्रति॑ष्ठापयति ॥ \newline

\textbf{Pada Paata} \newline

पु॒रस्ता᳚त् । प्र॒त्यञ्चः॑ । प॒शवः॑ । मेध᳚म् । उपेति॑ । ति॒ष्ठ॒न्ते॒ । यः । वै । अप॑नाभि॒मित्यप॑ - ना॒भि॒म् । अ॒ग्निम् । चि॒नु॒ते । यज॑मानस्य । नाभि᳚म् । अनु॑ । प्रेति॑ । वि॒श॒ति॒ । सः । ए॒न॒म् । ई॒श्व॒रः । हिꣳसि॑तोः । उ॒लूख॑लम् । उपेति॑ । द॒धा॒ति॒ । ए॒षा । वै । अ॒ग्नेः । नाभिः॑ । सना॑भि॒मिति॒ स - ना॒भि॒म् । ए॒व । अ॒ग्निम् । चि॒नु॒त॒ । अहिꣳ॑सायै । औदु॑बंरम् । भ॒व॒ति॒ । ऊर्क् । वै । उ॒दु॒बंरः॑ । ऊर्ज᳚म् । ए॒व । अवेति॑ । रु॒न्धे॒ । म॒द्ध्य॒तः । उपेति॑ । द॒धा॒ति॒ । म॒द्ध्य॒तः । ए॒व । अ॒स्मै॒ । ऊर्ज᳚म् । द॒धा॒ति॒ । तस्मा᳚त् ( ) । म॒द्ध्य॒तः । ऊ॒र्जा । भु॒ञ्ज॒ते॒ । इय॑त् । भ॒व॒ति॒ । प्र॒जाप॑ति॒नेति॑ प्र॒जा - प॒ति॒ना॒ । य॒ज्ञ्॒मु॒खेनेति॑ यज्ञ् - मु॒खेन॑ । सम्मि॑त॒मिति॒ सं - मि॒त॒म् । अवेति॑ । ह॒न्ति॒ । अन्न᳚म् । ए॒व । अ॒कः॒ । वै॒ष्ण॒व्या । ऋ॒चा । उपेति॑ । द॒धा॒ति॒ । विष्णुः॑ । वै । य॒ज्ञ्ः । वै॒ष्ण॒वाः । वन॒स्पत॑यः । य॒ज्ञे । ए॒व । य॒ज्ञ्म् । प्रतीति॑ । स्था॒प॒य॒ति॒ ॥  \newline


\textbf{Krama Paata} \newline

पु॒रस्ता᳚त् प्र॒त्यञ्चः॑ । प्र॒त्यञ्चः॑ प॒शवः॑ । प॒शवो॒ मेध᳚म् । मेध॒मुप॑ । उप॑ तिष्ठन्ते । ति॒ष्ठ॒न्ते॒ यः । यो वै । वा अप॑नाभिम् । अप॑नाभिम॒ग्निम् । अप॑नाभि॒मित्यप॑ - ना॒भि॒म् । अ॒ग्निम् चि॑नु॒ते । चि॒नु॒ते यज॑मानस्य । यज॑मानस्य॒ नाभि᳚म् । नाभि॒मनु॑ । अनु॒ प्र । प्र वि॑शति । वि॒श॒ति॒ सः । स ए॑नम् । ए॒न॒मी॒श्व॒रः । ई॒श्व॒रो हिꣳसि॑तोः । हिꣳसि॑तोरु॒लूख॑लम् । उ॒लूख॑ल॒मुप॑ । उप॑ दधाति । द॒धा॒त्ये॒षा । ए॒षा वै । वा अ॒ग्नेः । अ॒ग्नेर् नाभिः॑ । नाभिः॒ सना॑भिम् । सना॑भिमे॒व । सना॑भि॒मिति॒ स - ना॒भि॒म् । ए॒वाग्निम् । अ॒ग्निम् चि॑नुते । चि॒नु॒तेऽहिꣳ॑सायै । अहिꣳ॑साया॒ औदु॑म्बरम् । औदु॑म्बरम् भवति । भ॒व॒त्यूर्क् । ऊर्ग् वै । वा उ॑दु॒म्बरः॑ । उ॒दु॒म्बर॒ ऊर्ज᳚म् । ऊर्ज॑मे॒व । ए॒वाव॑ । अव॑ रुन्धे । रु॒न्धे॒ म॒द्ध्य॒तः । म॒द्ध्य॒त उप॑ । उप॑ दधाति । द॒धा॒ति॒ म॒द्ध्य॒तः । म॒द्ध्य॒त ए॒व । ए॒वास्मै᳚ । अ॒स्मा॒ ऊर्ज᳚म् । ऊर्ज॑म् दधाति । द॒धा॒ति॒ तस्मा᳚त् ( ) । तस्मा᳚न् मद्ध्य॒तः । म॒द्ध्य॒त ऊ॒र्जा । ऊ॒र्जा भु॑ञ्जते । भु॒ञ्ज॒त॒ इय॑त् । इय॑द् भवति । भ॒व॒ति॒ प्र॒जाप॑तिना । प्र॒जाप॑तिना यज्ञ्मु॒खेन॑ । प्र॒जाप॑ति॒नेति॑ प्र॒जा - प॒ति॒ना॒ । य॒ज्ञ्॒मु॒खेन॒ सम्मि॑तम् । य॒ज्ञ्॒मु॒खेनेति॑ यज्ञ् - मु॒खेन॑ । सम्मि॑त॒मव॑ । सम्मि॑त॒मिति॒ सम् - मि॒त॒म् । अव॑ हन्ति । ह॒न्त्यन्न᳚म् । अन्न॑मे॒व । ए॒वाकः॑ । अ॒क॒र् वै॒ष्ण॒व्या । वै॒ष्ण॒व्यर्चा । ऋ॒चोप॑ । उप॑ दधाति । द॒धा॒ति॒ विष्णुः॑ । विष्णु॒र् वै । वै य॒ज्ञ्ः । य॒ज्ञो वै᳚ष्ण॒वाः । वै॒ष्ण॒वा वन॒स्पत॑यः । वन॒स्पत॑यो य॒ज्ञे । य॒ज्ञ् ए॒व । ए॒व य॒ज्ञ्म् । य॒ज्ञ्म् प्रति॑ । प्रति॑ष्ठापयति । स्था॒प॒य॒तीति॑ स्थापयति । \newline

\textbf{Jatai Paata} \newline

1. पु॒रस्ता᳚त् प्र॒त्यञ्चः॑ प्र॒त्यञ्चः॑ पु॒रस्ता᳚त् पु॒रस्ता᳚त् प्र॒त्यञ्चः॑ । \newline
2. प्र॒त्यञ्चः॑ प॒शवः॑ प॒शवः॑ प्र॒त्यञ्चः॑ प्र॒त्यञ्चः॑ प॒शवः॑ । \newline
3. प॒शवो॒ मेध॒म् मेध॑म् प॒शवः॑ प॒शवो॒ मेध᳚म् । \newline
4. मेध॒ मुपोप॒ मेध॒म् मेध॒ मुप॑ । \newline
5. उप॑ तिष्ठन्ते तिष्ठन्त॒ उपोप॑ तिष्ठन्ते । \newline
6. ति॒ष्ठ॒न्ते॒ यो य स्ति॑ष्ठन्ते तिष्ठन्ते॒ यः । \newline
7. यो वै वै यो यो वै । \newline
8. वा अप॑नाभि॒ मप॑नाभिं॒ ॅवै वा अप॑नाभिम् । \newline
9. अप॑नाभि म॒ग्नि म॒ग्नि मप॑नाभि॒ मप॑नाभि म॒ग्निम् । \newline
10. अप॑नाभि॒मित्यप॑ - ना॒भि॒म् । \newline
11. अ॒ग्निम् चि॑नु॒ते चि॑नु॒ते᳚ ऽग्नि म॒ग्निम् चि॑नु॒ते । \newline
12. चि॒नु॒ते यज॑मानस्य॒ यज॑मानस्य चिनु॒ते चि॑नु॒ते यज॑मानस्य । \newline
13. यज॑मानस्य॒ नाभि॒न् नाभिं॒ ॅयज॑मानस्य॒ यज॑मानस्य॒ नाभि᳚म् । \newline
14. नाभि॒ मन्वनु॒ नाभि॒न् नाभि॒ मनु॑ । \newline
15. अनु॒ प्र प्राण् वनु॒ प्र । \newline
16. प्र वि॑शति विशति॒ प्र प्र वि॑शति । \newline
17. वि॒श॒ति॒ स स वि॑शति विशति॒ सः । \newline
18. स ए॑न मेनꣳ॒॒ स स ए॑नम् । \newline
19. ए॒न॒ मी॒श्व॒र ई᳚श्व॒र ए॑न मेन मीश्व॒रः । \newline
20. ई॒श्व॒रो हिꣳसि॑तो॒र्॒. हिꣳसि॑तो रीश्व॒र ई᳚श्व॒रो हिꣳसि॑तोः । \newline
21. हिꣳसि॑तो रु॒लूख॑ल मु॒लूख॑लꣳ॒॒ हिꣳसि॑तो॒र्॒. हिꣳसि॑तो रु॒लूख॑लम् । \newline
22. उ॒लूख॑ल॒ मुपोपो॒ लूख॑ल मु॒लूख॑ल॒ मुप॑ । \newline
23. उप॑ दधाति दधा॒ त्युपोप॑ दधाति । \newline
24. द॒धा॒ त्ये॒षैषा द॑धाति दधा त्ये॒षा । \newline
25. ए॒षा वै वा ए॒षैषा वै । \newline
26. वा अ॒ग्ने र॒ग्नेर् वै वा अ॒ग्नेः । \newline
27. अ॒ग्नेर् नाभि॒र् नाभि॑ र॒ग्ने र॒ग्नेर् नाभिः॑ । \newline
28. नाभिः॒ सना॑भिꣳ॒॒ सना॑भि॒न् नाभि॒र् नाभिः॒ सना॑भिम् । \newline
29. सना॑भि मे॒वैव सना॑भिꣳ॒॒ सना॑भि मे॒व । \newline
30. सना॑भि॒मिति॒ स - ना॒भि॒म् । \newline
31. ए॒वाग्नि म॒ग्नि मे॒वै वाग्निम् । \newline
32. अ॒ग्निम् चि॑नुते चिनुते॒ ऽग्नि म॒ग्निम् चि॑नुते । \newline
33. चि॒नु॒ते ऽहिꣳ॑साया॒ अहिꣳ॑सायै चिनुते चिनु॒ते ऽहिꣳ॑सायै । \newline
34. अहिꣳ॑साया॒ औदुं॑बर॒ मौदुं॑बर॒ महिꣳ॑साया॒ अहिꣳ॑साया॒ औदुं॑बरम् । \newline
35. औदुं॑बरम् भवति भव॒ त्यौदुं॑बर॒ मौदुं॑बरम् भवति । \newline
36. भ॒व॒ त्यूर्गूर्ग् भ॑वति भव॒ त्यूर्क् । \newline
37. ऊर्ग् वै वा ऊर्गूर्ग् वै । \newline
38. वा उ॑दुं॒बर॑ उदुं॒बरो॒ वै वा उ॑दुं॒बरः॑ । \newline
39. उ॒दुं॒बर॒ ऊर्ज॒ मूर्ज॑ मुदुं॒बर॑ उदुं॒बर॒ ऊर्ज᳚म् । \newline
40. ऊर्ज॑ मे॒वै वोर्ज॒ मूर्ज॑ मे॒व । \newline
41. ए॒वावा वै॒वै वाव॑ । \newline
42. अव॑ रुन्धे रु॒न्धे ऽवाव॑ रुन्धे । \newline
43. रु॒न्धे॒ म॒द्ध्य॒तो म॑द्ध्य॒तो रु॑न्धे रुन्धे मद्ध्य॒तः । \newline
44. म॒द्ध्य॒त उपोप॑ मद्ध्य॒तो म॑द्ध्य॒त उप॑ । \newline
45. उप॑ दधाति दधा॒ त्युपोप॑ दधाति । \newline
46. द॒धा॒ति॒ म॒द्ध्य॒तो म॑द्ध्य॒तो द॑धाति दधाति मद्ध्य॒तः । \newline
47. म॒द्ध्य॒त ए॒वैव म॑द्ध्य॒तो म॑द्ध्य॒त ए॒व । \newline
48. ए॒वास्मा॑ अस्मा ए॒वैवास्मै᳚ । \newline
49. अ॒स्मा॒ ऊर्ज॒ मूर्ज॑ मस्मा अस्मा॒ ऊर्ज᳚म् । \newline
50. ऊर्ज॑म् दधाति दधा॒ त्यूर्ज॒ मूर्ज॑म् दधाति । \newline
51. द॒धा॒ति॒ तस्मा॒त् तस्मा᳚द् दधाति दधाति॒ तस्मा᳚त् । \newline
52. तस्मा᳚न् मद्ध्य॒तो म॑द्ध्य॒त स्तस्मा॒त् तस्मा᳚न् मद्ध्य॒तः । \newline
53. म॒द्ध्य॒त ऊ॒र्जोर्जा म॑द्ध्य॒तो म॑द्ध्य॒त ऊ॒र्जा । \newline
54. ऊ॒र्जा भु॑ञ्जते भुञ्जत ऊ॒र्जोर्जा भु॑ञ्जते । \newline
55. भु॒ञ्ज॒त॒ इय॒ दिय॑द् भुञ्जते भुञ्जत॒ इय॑त् । \newline
56. इय॑द् भवति भव॒तीय॒ दिय॑द् भवति । \newline
57. भ॒व॒ति॒ प्र॒जाप॑तिना प्र॒जाप॑तिना भवति भवति प्र॒जाप॑तिना । \newline
58. प्र॒जाप॑तिना यज्ञ्मु॒खेन॑ यज्ञ्मु॒खेन॑ प्र॒जाप॑तिना प्र॒जाप॑तिना यज्ञ्मु॒खेन॑ । \newline
59. प्र॒जाप॑ति॒नेति॑ प्र॒जा - प॒ति॒ना॒ । \newline
60. य॒ज्ञ्॒मु॒खेन॒ सम्मि॑तꣳ॒॒ सम्मि॑तं ॅयज्ञ्मु॒खेन॑ यज्ञ्मु॒खेन॒ सम्मि॑तम् । \newline
61. य॒ज्ञ्॒मु॒खेनेति॑ यज्ञ् - मु॒खेन॑ । \newline
62. सम्मि॑त॒ मवाव॒ सम्मि॑तꣳ॒॒ सम्मि॑त॒ मव॑ । \newline
63. सम्मि॑त॒मिति॒ सं - मि॒त॒म् । \newline
64. अव॑ हन्ति ह॒न् त्यवाव॑ हन्ति । \newline
65. ह॒न्त्यन्न॒ मन्नꣳ॑ हन्ति ह॒न्त्यन्न᳚म् । \newline
66. अन्न॑ मे॒वै वान्न॒ मन्न॑ मे॒व । \newline
67. ए॒वाक॑ रक रे॒वै वाकः॑ । \newline
68. अ॒क॒र् वै॒ष्ण॒व्या वै᳚ष्ण॒व्या ऽक॑ रकर् वैष्ण॒व्या । \newline
69. वै॒ष्ण॒व्य र्‌चर्चा वै᳚ष्ण॒व्या वै᳚ष्ण॒व्य र्‌चा । \newline
70. ऋ॒चोपोपा॒ र्‌च र्‌चोप॑ । \newline
71. उप॑ दधाति दधा॒ त्युपोप॑ दधाति । \newline
72. द॒धा॒ति॒ विष्णु॒र् विष्णु॑र् दधाति दधाति॒ विष्णुः॑ । \newline
73. विष्णु॒र् वै वै विष्णु॒र् विष्णु॒र् वै । \newline
74. वै य॒ज्ञो य॒ज्ञो वै वै य॒ज्ञ्ः । \newline
75. य॒ज्ञो वै᳚ष्ण॒वा वै᳚ष्ण॒वा य॒ज्ञो य॒ज्ञो वै᳚ष्ण॒वाः । \newline
76. वै॒ष्ण॒वा वन॒स्पत॑यो॒ वन॒स्पत॑यो वैष्ण॒वा वै᳚ष्ण॒वा वन॒स्पत॑यः । \newline
77. वन॒स्पत॑यो य॒ज्ञे य॒ज्ञे वन॒स्पत॑यो॒ वन॒स्पत॑यो य॒ज्ञे । \newline
78. य॒ज्ञ् ए॒वैव य॒ज्ञे य॒ज्ञ् ए॒व । \newline
79. ए॒व य॒ज्ञ्ं ॅय॒ज्ञ् मे॒वैव य॒ज्ञ्म् । \newline
80. य॒ज्ञ्म् प्रति॒ प्रति॑ य॒ज्ञ्ं ॅय॒ज्ञ्म् प्रति॑ । \newline
81. प्रति॑ ष्ठापयति स्थापयति॒ प्रति॒ प्रति॑ ष्ठापयति । \newline
82. स्था॒प॒य॒तीति॑ स्थापयति । \newline

\textbf{Ghana Paata } \newline

1. पु॒रस्ता᳚त् प्र॒त्यञ्चः॑ प्र॒त्यञ्चः॑ पु॒रस्ता᳚त् पु॒रस्ता᳚त् प्र॒त्यञ्चः॑ प॒शवः॑ प॒शवः॑ प्र॒त्यञ्चः॑ पु॒रस्ता᳚त् पु॒रस्ता᳚त् प्र॒त्यञ्चः॑ प॒शवः॑ । \newline
2. प्र॒त्यञ्चः॑ प॒शवः॑ प॒शवः॑ प्र॒त्यञ्चः॑ प्र॒त्यञ्चः॑ प॒शवो॒ मेध॒म् मेध॑म् प॒शवः॑ प्र॒त्यञ्चः॑ प्र॒त्यञ्चः॑ प॒शवो॒ मेध᳚म् । \newline
3. प॒शवो॒ मेध॒म् मेध॑म् प॒शवः॑ प॒शवो॒ मेध॒ मुपोप॒ मेध॑म् प॒शवः॑ प॒शवो॒ मेध॒ मुप॑ । \newline
4. मेध॒ मुपोप॒ मेध॒म् मेध॒ मुप॑ तिष्ठन्ते तिष्ठन्त॒ उप॒ मेध॒म् मेध॒ मुप॑ तिष्ठन्ते । \newline
5. उप॑ तिष्ठन्ते तिष्ठन्त॒ उपोप॑ तिष्ठन्ते॒ यो य स्ति॑ष्ठन्त॒ उपोप॑ तिष्ठन्ते॒ यः । \newline
6. ति॒ष्ठ॒न्ते॒ यो य स्ति॑ष्ठन्ते तिष्ठन्ते॒ यो वै वै य स्ति॑ष्ठन्ते तिष्ठन्ते॒ यो वै । \newline
7. यो वै वै यो यो वा अप॑नाभि॒ मप॑नाभिं॒ ॅवै यो यो वा अप॑नाभिम् । \newline
8. वा अप॑नाभि॒ मप॑नाभिं॒ ॅवै वा अप॑नाभि म॒ग्नि म॒ग्नि मप॑नाभिं॒ ॅवै वा अप॑नाभि म॒ग्निम् । \newline
9. अप॑नाभि म॒ग्नि म॒ग्नि मप॑नाभि॒ मप॑नाभि म॒ग्निम् चि॑नु॒ते चि॑नु॒ते᳚ ऽग्नि मप॑नाभि॒ मप॑नाभि म॒ग्निम् चि॑नु॒ते । \newline
10. अप॑नाभि॒मित्यप॑ - ना॒भि॒म् । \newline
11. अ॒ग्निम् चि॑नु॒ते चि॑नु॒ते᳚ ऽग्नि म॒ग्निम् चि॑नु॒ते यज॑मानस्य॒ यज॑मानस्य चिनु॒ते᳚ ऽग्नि म॒ग्निम् चि॑नु॒ते यज॑मानस्य । \newline
12. चि॒नु॒ते यज॑मानस्य॒ यज॑मानस्य चिनु॒ते चि॑नु॒ते यज॑मानस्य॒ नाभि॒म् नाभिं॒ ॅयज॑मानस्य चिनु॒ते चि॑नु॒ते यज॑मानस्य॒ नाभि᳚म् । \newline
13. यज॑मानस्य॒ नाभि॒म् नाभिं॒ ॅयज॑मानस्य॒ यज॑मानस्य॒ नाभि॒ मन्वनु॒ नाभिं॒ ॅयज॑मानस्य॒ यज॑मानस्य॒ नाभि॒ मनु॑ । \newline
14. नाभि॒ मन्वनु॒ नाभि॒म् नाभि॒ मनु॒ प्र प्राणु॒ नाभि॒म् नाभि॒ मनु॒ प्र । \newline
15. अनु॒ प्र प्राण् वनु॒ प्र वि॑शति विशति॒ प्राण् वनु॒ प्र वि॑शति । \newline
16. प्र वि॑शति विशति॒ प्र प्र वि॑शति॒ स स वि॑शति॒ प्र प्र वि॑शति॒ सः । \newline
17. वि॒श॒ति॒ स स वि॑शति विशति॒ स ए॑न मेनꣳ॒॒ स वि॑शति विशति॒ स ए॑नम् । \newline
18. स ए॑न मेनꣳ॒॒ स स ए॑न मीश्व॒र ई᳚श्व॒र ए॑नꣳ॒॒ स स ए॑न मीश्व॒रः । \newline
19. ए॒न॒ मी॒श्व॒र ई᳚श्व॒र ए॑न मेन मीश्व॒रो हिꣳसि॑तो॒र्॒. हिꣳसि॑तो रीश्व॒र ए॑न मेन मीश्व॒रो हिꣳसि॑तोः । \newline
20. ई॒श्व॒रो हिꣳसि॑तो॒र्॒. हिꣳसि॑तो रीश्व॒र ई᳚श्व॒रो हिꣳसि॑तो रु॒लूख॑ल मु॒लूख॑लꣳ॒॒ हिꣳसि॑तो रीश्व॒र ई᳚श्व॒रो हिꣳसि॑तो रु॒लूख॑लम् । \newline
21. हिꣳसि॑तो रु॒लूख॑ल मु॒लूख॑लꣳ॒॒ हिꣳसि॑तो॒र्॒. हिꣳसि॑तो रु॒लूख॑ल॒ मुपोपो॒ लूख॑लꣳ॒॒ हिꣳसि॑तो॒र्॒. हिꣳसि॑तो रु॒लूख॑ल॒ मुप॑ । \newline
22. उ॒लूख॑ल॒ मुपोपो॒लूख॑ल मु॒लूख॑ल॒ मुप॑ दधाति दधा॒ त्युपो॒लूख॑ल मु॒लूख॑ल॒ मुप॑ दधाति । \newline
23. उप॑ दधाति दधा॒ त्युपोप॑ दधा त्ये॒षैषा द॑धा॒ त्युपोप॑ दधा त्ये॒षा । \newline
24. द॒धा॒ त्ये॒षैषा द॑धाति दधा त्ये॒षा वै वा ए॒षा द॑धाति दधा त्ये॒षा वै । \newline
25. ए॒षा वै वा ए॒षैषा वा अ॒ग्ने र॒ग्नेर् वा ए॒षैषा वा अ॒ग्नेः । \newline
26. वा अ॒ग्ने र॒ग्नेर् वै वा अ॒ग्नेर् नाभि॒र् नाभि॑ र॒ग्नेर् वै वा अ॒ग्नेर् नाभिः॑ । \newline
27. अ॒ग्नेर् नाभि॒र् नाभि॑ र॒ग्ने र॒ग्नेर् नाभिः॒ सना॑भिꣳ॒॒ सना॑भि॒म् नाभि॑ र॒ग्ने र॒ग्नेर् नाभिः॒ सना॑भिम् । \newline
28. नाभिः॒ सना॑भिꣳ॒॒ सना॑भि॒म् नाभि॒र् नाभिः॒ सना॑भि मे॒वैव सना॑भि॒म् नाभि॒र् नाभिः॒ सना॑भि मे॒व । \newline
29. सना॑भि मे॒वैव सना॑भिꣳ॒॒ सना॑भि मे॒वाग्नि म॒ग्नि मे॒व सना॑भिꣳ॒॒ सना॑भि मे॒वाग्निम् । \newline
30. सना॑भि॒मिति॒ स - ना॒भि॒म् । \newline
31. ए॒वाग्नि म॒ग्नि मे॒वैवाग्निम् चि॑नुते चिनुते॒ ऽग्नि मे॒वैवाग्निम् चि॑नुते । \newline
32. अ॒ग्निम् चि॑नुते चिनुते॒ ऽग्नि म॒ग्निम् चि॑नु॒ते ऽहिꣳ॑साया॒ अहिꣳ॑सायै चिनुते॒ ऽग्नि म॒ग्निम् चि॑नु॒ते ऽहिꣳ॑सायै । \newline
33. चि॒नु॒ते ऽहिꣳ॑साया॒ अहिꣳ॑सायै चिनुते चिनु॒ते ऽहिꣳ॑साया॒ औदुं॑बर॒ मौदुं॑बर॒ महिꣳ॑सायै चिनुते चिनु॒ते ऽहिꣳ॑साया॒ औदुं॑बरम् । \newline
34. अहिꣳ॑साया॒ औदुं॑बर॒ मौदुं॑बर॒ महिꣳ॑साया॒ अहिꣳ॑साया॒ औदुं॑बरम् भवति भव॒ त्यौदुं॑बर॒ महिꣳ॑साया॒ अहिꣳ॑साया॒ औदुं॑बरम् भवति । \newline
35. औदुं॑बरम् भवति भव॒ त्यौदुं॑बर॒ मौदुं॑बरम् भव॒ त्यूर्गूर्ग् भ॑व॒ त्यौदुं॑बर॒ मौदुं॑बरम् भव॒त्यूर्क् । \newline
36. भ॒व॒ त्यूर्गूर्ग् भ॑वति भव॒त्यूर्ग् वै वा ऊर्ग् भ॑वति भव॒त्यूर्ग् वै । \newline
37. ऊर्ग् वै वा ऊर्गूर्ग् वा उ॑दुं॒बर॑ उदुं॒बरो॒ वा ऊर्गूर्ग् वा उ॑दुं॒बरः॑ । \newline
38. वा उ॑दुं॒बर॑ उदुं॒बरो॒ वै वा उ॑दुं॒बर॒ ऊर्ज॒ मूर्ज॑ मुदुं॒बरो॒ वै वा उ॑दुं॒बर॒ ऊर्ज᳚म् । \newline
39. उ॒दुं॒बर॒ ऊर्ज॒ मूर्ज॑ मुदुं॒बर॑ उदुं॒बर॒ ऊर्ज॑ मे॒वैवोर्ज॑ मुदुं॒बर॑ उदुं॒बर॒ ऊर्ज॑ मे॒व । \newline
40. ऊर्ज॑ मे॒वैवोर्ज॒ मूर्ज॑ मे॒वावा वै॒वोर्ज॒ मूर्ज॑ मे॒वाव॑ । \newline
41. ए॒वावा वै॒वै वाव॑ रुन्धे रु॒न्धे ऽवै॒वै वाव॑ रुन्धे । \newline
42. अव॑ रुन्धे रु॒न्धे ऽवाव॑ रुन्धे मद्ध्य॒तो म॑द्ध्य॒तो रु॒न्धे ऽवाव॑ रुन्धे मद्ध्य॒तः । \newline
43. रु॒न्धे॒ म॒द्ध्य॒तो म॑द्ध्य॒तो रु॑न्धे रुन्धे मद्ध्य॒त उपोप॑ मद्ध्य॒तो रु॑न्धे रुन्धे मद्ध्य॒त उप॑ । \newline
44. म॒द्ध्य॒त उपोप॑ मद्ध्य॒तो म॑द्ध्य॒त उप॑ दधाति दधा॒ त्युप॑ मद्ध्य॒तो म॑द्ध्य॒त उप॑ दधाति । \newline
45. उप॑ दधाति दधा॒ त्युपोप॑ दधाति मद्ध्य॒तो म॑द्ध्य॒तो द॑धा॒ त्युपोप॑ दधाति मद्ध्य॒तः । \newline
46. द॒धा॒ति॒ म॒द्ध्य॒तो म॑द्ध्य॒तो द॑धाति दधाति मद्ध्य॒त ए॒वैव म॑द्ध्य॒तो द॑धाति दधाति मद्ध्य॒त ए॒व । \newline
47. म॒द्ध्य॒त ए॒वैव म॑द्ध्य॒तो म॑द्ध्य॒त ए॒वास्मा॑ अस्मा ए॒व म॑द्ध्य॒तो म॑द्ध्य॒त ए॒वास्मै᳚ । \newline
48. ए॒वास्मा॑ अस्मा ए॒वैवास्मा॒ ऊर्ज॒ मूर्ज॑ मस्मा ए॒वैवास्मा॒ ऊर्ज᳚म् । \newline
49. अ॒स्मा॒ ऊर्ज॒ मूर्ज॑ मस्मा अस्मा॒ ऊर्ज॑म् दधाति दधा॒ त्यूर्ज॑ मस्मा अस्मा॒ ऊर्ज॑म् दधाति । \newline
50. ऊर्ज॑म् दधाति दधा॒त्यूर्ज॒ मूर्ज॑म् दधाति॒ तस्मा॒त् तस्मा᳚द् दधा॒त्यूर्ज॒ मूर्ज॑म् दधाति॒ तस्मा᳚त् । \newline
51. द॒धा॒ति॒ तस्मा॒त् तस्मा᳚द् दधाति दधाति॒ तस्मा᳚न् मद्ध्य॒तो म॑द्ध्य॒त स्तस्मा᳚द् दधाति दधाति॒ तस्मा᳚न् मद्ध्य॒तः । \newline
52. तस्मा᳚न् मद्ध्य॒तो म॑द्ध्य॒त स्तस्मा॒त् तस्मा᳚न् मद्ध्य॒त ऊ॒र्जोर्जा म॑द्ध्य॒त स्तस्मा॒त् तस्मा᳚न् मद्ध्य॒त ऊ॒र्जा । \newline
53. म॒द्ध्य॒त ऊ॒र्जोर्जा म॑द्ध्य॒तो म॑द्ध्य॒त ऊ॒र्जा भु॑ञ्जते भुञ्जत ऊ॒र्जा म॑द्ध्य॒तो म॑द्ध्य॒त ऊ॒र्जा भु॑ञ्जते । \newline
54. ऊ॒र्जा भु॑ञ्जते भुञ्जत ऊ॒र्जोर्जा भु॑ञ्जत॒ इय॒दिय॑द् भुञ्जत ऊ॒र्जोर्जा भु॑ञ्जत॒ इय॑त् । \newline
55. भु॒ञ्ज॒त॒ इय॒ दिय॑द् भुञ्जते भुञ्जत॒ इय॑द् भवति भव॒तीय॑द् भुञ्जते भुञ्जत॒ इय॑द् भवति । \newline
56. इय॑द् भवति भव॒तीय॒ दिय॑द् भवति प्र॒जाप॑तिना प्र॒जाप॑तिना भव॒तीय॒ दिय॑द् भवति प्र॒जाप॑तिना । \newline
57. भ॒व॒ति॒ प्र॒जाप॑तिना प्र॒जाप॑तिना भवति भवति प्र॒जाप॑तिना यज्ञ्मु॒खेन॑ यज्ञ्मु॒खेन॑ प्र॒जाप॑तिना भवति भवति प्र॒जाप॑तिना यज्ञ्मु॒खेन॑ । \newline
58. प्र॒जाप॑तिना यज्ञ्मु॒खेन॑ यज्ञ्मु॒खेन॑ प्र॒जाप॑तिना प्र॒जाप॑तिना यज्ञ्मु॒खेन॒ सम्मि॑तꣳ॒॒ सम्मि॑तं ॅयज्ञ्मु॒खेन॑ प्र॒जाप॑तिना प्र॒जाप॑तिना यज्ञ्मु॒खेन॒ सम्मि॑तम् । \newline
59. प्र॒जाप॑ति॒नेति॑ प्र॒जा - प॒ति॒ना॒ । \newline
60. य॒ज्ञ्॒मु॒खेन॒ सम्मि॑तꣳ॒॒ सम्मि॑तं ॅयज्ञ्मु॒खेन॑ यज्ञ्मु॒खेन॒ सम्मि॑त॒ मवाव॒ सम्मि॑तं ॅयज्ञ्मु॒खेन॑ यज्ञ्मु॒खेन॒ सम्मि॑त॒ मव॑ । \newline
61. य॒ज्ञ्॒मु॒खेनेति॑ यज्ञ् - मु॒खेन॑ । \newline
62. सम्मि॑त॒ मवाव॒ सम्मि॑तꣳ॒॒ सम्मि॑त॒ मव॑ हन्ति ह॒न्त्यव॒ सम्मि॑तꣳ॒॒ सम्मि॑त॒ मव॑ हन्ति । \newline
63. सम्मि॑त॒मिति॒ सं - मि॒त॒म् । \newline
64. अव॑ हन्ति ह॒न्त्यवाव॑ ह॒न्त्यन्न॒ मन्नꣳ॑ ह॒न्त्यवाव॑ ह॒न्त्यन्न᳚म् । \newline
65. ह॒न्त्यन्न॒ मन्नꣳ॑ हन्ति ह॒न्त्यन्न॑ मे॒वैवान्नꣳ॑ हन्ति ह॒न्त्यन्न॑ मे॒व । \newline
66. अन्न॑ मे॒वैवान्न॒ मन्न॑ मे॒वाक॑ रक रे॒वान्न॒ मन्न॑ मे॒वाकः॑ । \newline
67. ए॒वाक॑ रक रे॒वैवाक॑र् वैष्ण॒व्या वै᳚ष्ण॒व्या ऽक॑ रे॒वैवाक॑र् वैष्ण॒व्या । \newline
68. अ॒क॒र् वै॒ष्ण॒व्या वै᳚ष्ण॒व्या ऽक॑ रकर् वैष्ण॒व्य र्‌च र्‌चा वै᳚ष्ण॒व्या ऽक॑ रकर् वैष्ण॒व्य र्‌चा । \newline
69. वै॒ष्ण॒व्य र्‌च र्‌चा वै᳚ष्ण॒व्या वै᳚ष्ण॒व्य र्‌चोपोपा॒ र्‌चा वै᳚ष्ण॒व्या वै᳚ष्ण॒व्यर्चोप॑ । \newline
70. ऋ॒चोपोपा॒ र्‌चर्चोप॑ दधाति दधा॒ त्युपा॒ र्‌चर्चोप॑ दधाति । \newline
71. उप॑ दधाति दधा॒ त्युपोप॑ दधाति॒ विष्णु॒र् विष्णु॑र् दधा॒ त्युपोप॑ दधाति॒ विष्णुः॑ । \newline
72. द॒धा॒ति॒ विष्णु॒र् विष्णु॑र् दधाति दधाति॒ विष्णु॒र् वै वै विष्णु॑र् दधाति दधाति॒ विष्णु॒र् वै । \newline
73. विष्णु॒र् वै वै विष्णु॒र् विष्णु॒र् वै य॒ज्ञो य॒ज्ञो वै विष्णु॒र् विष्णु॒र् वै य॒ज्ञ्ः । \newline
74. वै य॒ज्ञो य॒ज्ञो वै वै य॒ज्ञो वै᳚ष्ण॒वा वै᳚ष्ण॒वा य॒ज्ञो वै वै य॒ज्ञो वै᳚ष्ण॒वाः । \newline
75. य॒ज्ञो वै᳚ष्ण॒वा वै᳚ष्ण॒वा य॒ज्ञो य॒ज्ञो वै᳚ष्ण॒वा वन॒स्पत॑यो॒ वन॒स्पत॑यो वैष्ण॒वा य॒ज्ञो य॒ज्ञो वै᳚ष्ण॒वा वन॒स्पत॑यः । \newline
76. वै॒ष्ण॒वा वन॒स्पत॑यो॒ वन॒स्पत॑यो वैष्ण॒वा वै᳚ष्ण॒वा वन॒स्पत॑यो य॒ज्ञे य॒ज्ञे वन॒स्पत॑यो वैष्ण॒वा वै᳚ष्ण॒वा वन॒स्पत॑यो य॒ज्ञे । \newline
77. वन॒स्पत॑यो य॒ज्ञे य॒ज्ञे वन॒स्पत॑यो॒ वन॒स्पत॑यो य॒ज्ञ् ए॒वैव य॒ज्ञे वन॒स्पत॑यो॒ वन॒स्पत॑यो य॒ज्ञ् ए॒व । \newline
78. य॒ज्ञ् ए॒वैव य॒ज्ञे य॒ज्ञ् ए॒व य॒ज्ञ्ं ॅय॒ज्ञ् मे॒व य॒ज्ञे य॒ज्ञ् ए॒व य॒ज्ञ्म् । \newline
79. ए॒व य॒ज्ञ्ं ॅय॒ज्ञ् मे॒वैव य॒ज्ञ्म् प्रति॒ प्रति॑ य॒ज्ञ् मे॒वैव य॒ज्ञ्म् प्रति॑ । \newline
80. य॒ज्ञ्म् प्रति॒ प्रति॑ य॒ज्ञ्ं ॅय॒ज्ञ्म् प्रति॑ ष्ठापयति स्थापयति॒ प्रति॑ य॒ज्ञ्ं ॅय॒ज्ञ्म् प्रति॑ ष्ठापयति । \newline
81. प्रति॑ ष्ठापयति स्थापयति॒ प्रति॒ प्रति॑ ष्ठापयति । \newline
82. स्था॒प॒य॒तीति॑ स्थापयति । \newline
\pagebreak
\markright{ TS 5.2.9.1  \hfill https://www.vedavms.in \hfill}

\section{ TS 5.2.9.1 }

\textbf{TS 5.2.9.1 } \newline
\textbf{Samhita Paata} \newline

ए॒षां ॅवा ए॒तल्लो॒कानां॒ ज्योतिः॒ सम्भृ॑तं॒ ॅयदु॒खा यदु॒खा-मु॑प॒दधा᳚त्ये॒भ्य ए॒व लो॒केभ्यो॒ ज्योति॒रव॑ रुन्धे मद्ध्य॒त उप॑ दधाति मद्ध्य॒त ए॒वास्मै॒ ज्योति॑र्दधाति॒ तस्मा᳚न्मद्ध्य॒तो ज्योति॒रुपा᳚ऽऽस्महे॒ सिक॑ताभिः पूरयत्ये॒तद्वा अ॒ग्नेर्वै᳚श्वान॒रस्य॑ रू॒पꣳ रू॒पेणै॒व वै᳚श्वान॒रमव॑ रुन्धे॒ यं का॒मये॑त॒ क्षोधु॑कः स्या॒दित्यू॒नां तस्योप॑ - [  ] \newline

\textbf{Pada Paata} \newline

ए॒षाम् । वै । ए॒तत् । लो॒काना᳚म् । ज्योतिः॑ । संभृ॑त॒मिति॒ सं-भृ॒त॒म् । यत् । उ॒खा । यत् । उ॒खाम् । उ॒प॒दधा॒तीत्यु॑प-दधा॑ति । ए॒भ्यः । ए॒व । लो॒केभ्यः॑ । ज्योतिः॑ । अवेति॑ । रु॒न्धे॒ । म॒द्ध्य॒तः । उपेति॑ । द॒धा॒ति॒ । म॒द्ध्य॒तः । ए॒व । अ॒स्मै॒ । ज्योतिः॑ । द॒धा॒ति॒ । तस्मा᳚त् । म॒द्ध्य॒तः । ज्योतिः॑ । उपेति॑ । आ॒स्म॒हे॒ । सिक॑ताभिः । पू॒र॒य॒ति॒ । ए॒तत् । वै । अ॒ग्नेः । वै॒श्वा॒न॒रस्य॑ । रू॒पम् । रू॒पेण॑ । ए॒व । वै॒श्वा॒न॒रम् । अवेति॑ । रु॒न्धे॒ । यम् । का॒मये॑त । क्षोधु॑कः । स्या॒त् । इति॑ । ऊ॒नाम् । तस्य॑ । उपेति॑ ।  \newline


\textbf{Krama Paata} \newline

ए॒षाम् ॅवै । वा ए॒तत् । ए॒तल्लो॒काना᳚म् । लो॒काना॒म् ज्योतिः॑ । ज्योतिः॒ सम्भृ॑तम् । सम्भृ॑त॒म् ॅयत् । सम्भृ॑त॒मिति॒ सम् - भृ॒त॒म् । यदु॒खा । उ॒खा यत् । यदु॒खाम् । उ॒खामु॑प॒दधा॑ति । उ॒प॒दधा᳚त्ये॒भ्यः । उ॒प॒दधा॒तीत्यु॑प - दधा॑ति । ए॒भ्य ए॒व । ए॒व लो॒केभ्यः॑ । लो॒केभ्यो॒ ज्योतिः॑ । ज्योति॒रव॑ । अव॑ रुन्धे । रु॒न्धे॒ म॒द्ध्य॒तः । म॒द्ध्य॒त उप॑ । उप॑ दधाति । द॒धा॒ति॒ म॒द्ध्य॒तः । म॒द्ध्य॒त ए॒व । ए॒वास्मै᳚ । अ॒स्मै॒ ज्योतिः॑ । ज्योति॑र् दधाति । द॒धा॒ति॒ तस्मा᳚त् । तस्मा᳚न् मद्ध्य॒तः । म॒द्ध्य॒तो ज्योतिः॑ । ज्योति॒रुप॑ । उपा᳚स्महे । आ॒स्म॒हे॒ सिक॑ताभिः । सिक॑ताभिः पूरयति । पू॒र॒य॒त्ये॒तत् । ए॒तद् वै । वा अ॒ग्नेः । अ॒ग्नेर् वै᳚श्वान॒रस्य॑ । वै॒श्वा॒न॒रस्य॑ रू॒पम् । रू॒पꣳ रू॒पेण॑ । रू॒पेणै॒व । ए॒व वै᳚श्वान॒रम् । वै॒श्वा॒न॒रमव॑ । अव॑ रुन्धे । रु॒न्धे॒ यम् । यम् का॒मये॑त । का॒मये॑त॒ क्षोधु॑कः । क्षोधु॑कः स्यात् । स्या॒दिति॑ । इत्यू॒नाम् । ऊ॒नाम् तस्य॑ । तस्योप॑ । उप॑ दद्ध्यात् \newline

\textbf{Jatai Paata} \newline

1. ए॒षां ॅवै वा ए॒षा मे॒षां ॅवै । \newline
2. वा ए॒त दे॒तद् वै वा ए॒तत् । \newline
3. ए॒त ल्लो॒काना᳚म् ॅलो॒काना॑ मे॒त दे॒त ल्लो॒काना᳚म् । \newline
4. लो॒काना॒म् ज्योति॒र् ज्योति॑र् लो॒काना᳚म् ॅलो॒काना॒म् ज्योतिः॑ । \newline
5. ज्योतिः॒ संभृ॑तꣳ॒॒ संभृ॑त॒म् ज्योति॒र् ज्योतिः॒ संभृ॑तम् । \newline
6. संभृ॑तं॒ ॅयद् यथ् संभृ॑तꣳ॒॒ संभृ॑तं॒ ॅयत् । \newline
7. संभृ॑त॒मिति॒ सं - भृ॒त॒म् । \newline
8. यदु॒खोखा यद् यदु॒खा । \newline
9. उ॒खा यद् यदु॒खोखा यत् । \newline
10. यदु॒खा मु॒खां ॅयद् यदु॒खाम् । \newline
11. उ॒खा मु॑प॒दधा᳚ त्युप॒दधा᳚ त्यु॒खा मु॒खा मु॑प॒दधा॑ति । \newline
12. उ॒प॒दधा᳚ त्ये॒भ्य ए॒भ्य उ॑प॒दधा᳚ त्युप॒दधा᳚ त्ये॒भ्यः । \newline
13. उ॒प॒दधा॒तीत्यु॑प - दधा॑ति । \newline
14. ए॒भ्य ए॒वैवैभ्य ए॒भ्य ए॒व । \newline
15. ए॒व लो॒केभ्यो॑ लो॒केभ्य॑ ए॒वैव लो॒केभ्यः॑ । \newline
16. लो॒केभ्यो॒ ज्योति॒र् ज्योति॑र् लो॒केभ्यो॑ लो॒केभ्यो॒ ज्योतिः॑ । \newline
17. ज्योति॒ रवाव॒ ज्योति॒र् ज्योति॒ रव॑ । \newline
18. अव॑ रुन्धे रु॒न्धे ऽवाव॑ रुन्धे । \newline
19. रु॒न्धे॒ म॒द्ध्य॒तो म॑द्ध्य॒तो रु॑न्धे रुन्धे मद्ध्य॒तः । \newline
20. म॒द्ध्य॒त उपोप॑ मद्ध्य॒तो म॑द्ध्य॒त उप॑ । \newline
21. उप॑ दधाति दधा॒ त्युपोप॑ दधाति । \newline
22. द॒धा॒ति॒ म॒द्ध्य॒तो म॑द्ध्य॒तो द॑धाति दधाति मद्ध्य॒तः । \newline
23. म॒द्ध्य॒त ए॒वैव म॑द्ध्य॒तो म॑द्ध्य॒त ए॒व । \newline
24. ए॒वास्मा॑ अस्मा ए॒वै वास्मै᳚ । \newline
25. अ॒स्मै॒ ज्योति॒र् ज्योति॑ रस्मा अस्मै॒ ज्योतिः॑ । \newline
26. ज्योति॑र् दधाति दधाति॒ ज्योति॒र् ज्योति॑र् दधाति । \newline
27. द॒धा॒ति॒ तस्मा॒त् तस्मा᳚द् दधाति दधाति॒ तस्मा᳚त् । \newline
28. तस्मा᳚न् मद्ध्य॒तो म॑द्ध्य॒त स्तस्मा॒त् तस्मा᳚न् मद्ध्य॒तः । \newline
29. म॒द्ध्य॒तो ज्योति॒र् ज्योति॑र् मद्ध्य॒तो म॑द्ध्य॒तो ज्योतिः॑ । \newline
30. ज्योति॒ रुपोप॒ ज्योति॒र् ज्योति॒ रुप॑ । \newline
31. उपा᳚स्मह आस्मह॒ उपोपा᳚ स्महे । \newline
32. आ॒स्म॒हे॒ सिक॑ताभिः॒ सिक॑ताभि रास्मह आस्महे॒ सिक॑ताभिः । \newline
33. सिक॑ताभिः पूरयति पूरयति॒ सिक॑ताभिः॒ सिक॑ताभिः पूरयति । \newline
34. पू॒र॒य॒ त्ये॒त दे॒तत् पू॑रयति पूरय त्ये॒तत् । \newline
35. ए॒तद् वै वा ए॒त दे॒तद् वै । \newline
36. वा अ॒ग्ने र॒ग्नेर् वै वा अ॒ग्नेः । \newline
37. अ॒ग्नेर् वै᳚श्वान॒रस्य॑ वैश्वान॒रस्या॒ ग्ने र॒ग्नेर् वै᳚श्वान॒रस्य॑ । \newline
38. वै॒श्वा॒न॒रस्य॑ रू॒पꣳ रू॒पं ॅवै᳚श्वान॒रस्य॑ वैश्वान॒रस्य॑ रू॒पम् । \newline
39. रू॒पꣳ रू॒पेण॑ रू॒पेण॑ रू॒पꣳ रू॒पꣳ रू॒पेण॑ । \newline
40. रू॒पे णै॒वैव रू॒पेण॑ रू॒पेणै॒व । \newline
41. ए॒व वै᳚श्वान॒रं ॅवै᳚श्वान॒र मे॒वैव वै᳚श्वान॒रम् । \newline
42. वै॒श्वा॒न॒र मवाव॑ वैश्वान॒रं ॅवै᳚श्वान॒र मव॑ । \newline
43. अव॑ रुन्धे रु॒न्धे ऽवाव॑ रुन्धे । \newline
44. रु॒न्धे॒ यं ॅयꣳ रु॑न्धे रुन्धे॒ यम् । \newline
45. यम् का॒मये॑त का॒मये॑त॒ यं ॅयम् का॒मये॑त । \newline
46. का॒मये॑त॒ क्षोधु॑कः॒ क्षोधु॑कः का॒मये॑त का॒मये॑त॒ क्षोधु॑कः । \newline
47. क्षोधु॑कः स्याथ् स्या॒त् क्षोधु॑कः॒ क्षोधु॑कः स्यात् । \newline
48. स्या॒दितीति॑ स्याथ् स्या॒दिति॑ । \newline
49. इत्यू॒ना मू॒ना मिती त्यू॒नाम् । \newline
50. ऊ॒नाम् तस्य॒ तस्यो॒ना मू॒नाम् तस्य॑ । \newline
51. तस्योपोप॒ तस्य॒ तस्योप॑ । \newline
52. उप॑ दद्ध्याद् दद्ध्या॒ दुपोप॑ दद्ध्यात् । \newline

\textbf{Ghana Paata } \newline

1. ए॒षां ॅवै वा ए॒षा मे॒षां ॅवा ए॒त दे॒तद् वा ए॒षा मे॒षां ॅवा ए॒तत् । \newline
2. वा ए॒त दे॒तद् वै वा ए॒त ल्लो॒काना᳚म् ॅलो॒काना॑ मे॒तद् वै वा ए॒त ल्लो॒काना᳚म् । \newline
3. ए॒त ल्लो॒काना᳚म् ॅलो॒काना॑ मे॒त दे॒त ल्लो॒काना॒म् ज्योति॒र् ज्योति॑र् लो॒काना॑ मे॒त दे॒त ल्लो॒काना॒म् ज्योतिः॑ । \newline
4. लो॒काना॒म् ज्योति॒र् ज्योति॑र् लो॒काना᳚म् ॅलो॒काना॒म् ज्योतिः॒ संभृ॑तꣳ॒॒ संभृ॑त॒म् ज्योति॑र् लो॒काना᳚म् ॅलो॒काना॒म् ज्योतिः॒ संभृ॑तम् । \newline
5. ज्योतिः॒ संभृ॑तꣳ॒॒ संभृ॑त॒म् ज्योति॒र् ज्योतिः॒ संभृ॑तं॒ ॅयद् यथ् संभृ॑त॒म् ज्योति॒र् ज्योतिः॒ संभृ॑तं॒ ॅयत् । \newline
6. संभृ॑तं॒ ॅयद् यथ् संभृ॑तꣳ॒॒ संभृ॑तं॒ ॅयदु॒खोखा यथ् संभृ॑तꣳ॒॒ संभृ॑तं॒ ॅयदु॒खा । \newline
7. संभृ॑त॒मिति॒ सं - भृ॒त॒म् । \newline
8. यदु॒खोखा यद् यदु॒खा यद् यदु॒खा यद् यदु॒खा यत् । \newline
9. उ॒खा यद् यदु॒खोखा यदु॒खा मु॒खां ॅयदु॒खोखा यदु॒खाम् । \newline
10. यदु॒खा मु॒खां ॅयद् यदु॒खा मु॑प॒दधा᳚ त्युप॒दधा᳚ त्यु॒खां ॅयद् यदु॒खा मु॑प॒दधा॑ति । \newline
11. उ॒खा मु॑प॒दधा᳚ त्युप॒दधा᳚ त्यु॒खा मु॒खा मु॑प॒दधा᳚ त्ये॒भ्य ए॒भ्य उ॑प॒दधा᳚ त्यु॒खा मु॒खा मु॑प॒दधा᳚ त्ये॒भ्यः । \newline
12. उ॒प॒दधा᳚ त्ये॒भ्य ए॒भ्य उ॑प॒दधा᳚ त्युप॒दधा᳚ त्ये॒भ्य ए॒वैवैभ्य उ॑प॒दधा᳚ त्युप॒दधा᳚ त्ये॒भ्य ए॒व । \newline
13. उ॒प॒दधा॒तीत्यु॑प - दधा॑ति । \newline
14. ए॒भ्य ए॒वैवैभ्य ए॒भ्य ए॒व लो॒केभ्यो॑ लो॒केभ्य॑ ए॒वैभ्य ए॒भ्य ए॒व लो॒केभ्यः॑ । \newline
15. ए॒व लो॒केभ्यो॑ लो॒केभ्य॑ ए॒वैव लो॒केभ्यो॒ ज्योति॒र् ज्योति॑र् लो॒केभ्य॑ ए॒वैव लो॒केभ्यो॒ ज्योतिः॑ । \newline
16. लो॒केभ्यो॒ ज्योति॒र् ज्योति॑र् लो॒केभ्यो॑ लो॒केभ्यो॒ ज्योति॒ रवाव॒ ज्योति॑र् लो॒केभ्यो॑ लो॒केभ्यो॒ ज्योति॒रव॑ । \newline
17. ज्योति॒ रवाव॒ ज्योति॒र् ज्योति॒रव॑ रुन्धे रु॒न्धे ऽव॒ ज्योति॒र् ज्योति॒रव॑ रुन्धे । \newline
18. अव॑ रुन्धे रु॒न्धे ऽवाव॑ रुन्धे मद्ध्य॒तो म॑द्ध्य॒तो रु॒न्धे ऽवाव॑ रुन्धे मद्ध्य॒तः । \newline
19. रु॒न्धे॒ म॒द्ध्य॒तो म॑द्ध्य॒तो रु॑न्धे रुन्धे मद्ध्य॒त उपोप॑ मद्ध्य॒तो रु॑न्धे रुन्धे मद्ध्य॒त उप॑ । \newline
20. म॒द्ध्य॒त उपोप॑ मद्ध्य॒तो म॑द्ध्य॒त उप॑ दधाति दधा॒त्युप॑ मद्ध्य॒तो म॑द्ध्य॒त उप॑ दधाति । \newline
21. उप॑ दधाति दधा॒ त्युपोप॑ दधाति मद्ध्य॒तो म॑द्ध्य॒तो द॑धा॒ त्युपोप॑ दधाति मद्ध्य॒तः । \newline
22. द॒धा॒ति॒ म॒द्ध्य॒तो म॑द्ध्य॒तो द॑धाति दधाति मद्ध्य॒त ए॒वैव म॑द्ध्य॒तो द॑धाति दधाति मद्ध्य॒त ए॒व । \newline
23. म॒द्ध्य॒त ए॒वैव म॑द्ध्य॒तो म॑द्ध्य॒त ए॒वास्मा॑ अस्मा ए॒व म॑द्ध्य॒तो म॑द्ध्य॒त ए॒वास्मै᳚ । \newline
24. ए॒वास्मा॑ अस्मा ए॒वैवास्मै॒ ज्योति॒र् ज्योति॑ रस्मा ए॒वैवास्मै॒ ज्योतिः॑ । \newline
25. अ॒स्मै॒ ज्योति॒र् ज्योति॑ रस्मा अस्मै॒ ज्योति॑र् दधाति दधाति॒ ज्योति॑ रस्मा अस्मै॒ ज्योति॑र् दधाति । \newline
26. ज्योति॑र् दधाति दधाति॒ ज्योति॒र् ज्योति॑र् दधाति॒ तस्मा॒त् तस्मा᳚द् दधाति॒ ज्योति॒र् ज्योति॑र् दधाति॒ तस्मा᳚त् । \newline
27. द॒धा॒ति॒ तस्मा॒त् तस्मा᳚द् दधाति दधाति॒ तस्मा᳚न् मद्ध्य॒तो म॑द्ध्य॒त स्तस्मा᳚द् दधाति दधाति॒ तस्मा᳚न् मद्ध्य॒तः । \newline
28. तस्मा᳚न् मद्ध्य॒तो म॑द्ध्य॒त स्तस्मा॒त् तस्मा᳚न् मद्ध्य॒तो ज्योति॒र् ज्योति॑र् मद्ध्य॒त स्तस्मा॒त् तस्मा᳚न् मद्ध्य॒तो ज्योतिः॑ । \newline
29. म॒द्ध्य॒तो ज्योति॒र् ज्योति॑र् मद्ध्य॒तो म॑द्ध्य॒तो ज्योति॒ रुपोप॒ ज्योति॑र् मद्ध्य॒तो म॑द्ध्य॒तो ज्योति॒रुप॑ । \newline
30. ज्योति॒ रुपोप॒ ज्योति॒र् ज्योति॒ रुपा᳚स्मह आस्मह॒ उप॒ ज्योति॒र् ज्योति॒ रुपा᳚स्महे । \newline
31. उपा᳚स्मह आस्मह॒ उपोपा᳚स्महे॒ सिक॑ताभिः॒ सिक॑ताभि रास्मह॒ उपोपा᳚स्महे॒ सिक॑ताभिः । \newline
32. आ॒स्म॒हे॒ सिक॑ताभिः॒ सिक॑ताभि रास्मह आस्महे॒ सिक॑ताभिः पूरयति पूरयति॒ सिक॑ताभि रास्मह आस्महे॒ सिक॑ताभिः पूरयति । \newline
33. सिक॑ताभिः पूरयति पूरयति॒ सिक॑ताभिः॒ सिक॑ताभिः पूरय त्ये॒त दे॒तत् पू॑रयति॒ सिक॑ताभिः॒ सिक॑ताभिः पूरय त्ये॒तत् । \newline
34. पू॒र॒य॒ त्ये॒त दे॒तत् पू॑रयति पूरय त्ये॒तद् वै वा ए॒तत् पू॑रयति पूरय त्ये॒तद् वै । \newline
35. ए॒तद् वै वा ए॒त दे॒तद् वा अ॒ग्ने र॒ग्नेर् वा ए॒त दे॒तद् वा अ॒ग्नेः । \newline
36. वा अ॒ग्ने र॒ग्नेर् वै वा अ॒ग्नेर् वै᳚श्वान॒रस्य॑ वैश्वान॒रस्या॒ग्नेर् वै वा अ॒ग्नेर् वै᳚श्वान॒रस्य॑ । \newline
37. अ॒ग्नेर् वै᳚श्वान॒रस्य॑ वैश्वान॒रस्या॒ ग्ने र॒ग्नेर् वै᳚श्वान॒रस्य॑ रू॒पꣳ रू॒पं ॅवै᳚श्वान॒रस्या॒ ग्ने र॒ग्नेर् वै᳚श्वान॒रस्य॑ रू॒पम् । \newline
38. वै॒श्वा॒न॒रस्य॑ रू॒पꣳ रू॒पं ॅवै᳚श्वान॒रस्य॑ वैश्वान॒रस्य॑ रू॒पꣳ रू॒पेण॑ रू॒पेण॑ रू॒पं ॅवै᳚श्वान॒रस्य॑ वैश्वान॒रस्य॑ रू॒पꣳ रू॒पेण॑ । \newline
39. रू॒पꣳ रू॒पेण॑ रू॒पेण॑ रू॒पꣳ रू॒पꣳ रू॒पेणै॒वैव रू॒पेण॑ रू॒पꣳ रू॒पꣳ रू॒पेणै॒व । \newline
40. रू॒पेणै॒वैव रू॒पेण॑ रू॒पेणै॒व वै᳚श्वान॒रं ॅवै᳚श्वान॒र मे॒व रू॒पेण॑ रू॒पेणै॒व वै᳚श्वान॒रम् । \newline
41. ए॒व वै᳚श्वान॒रं ॅवै᳚श्वान॒र मे॒वैव वै᳚श्वान॒र मवाव॑ वैश्वान॒र मे॒वैव वै᳚श्वान॒र मव॑ । \newline
42. वै॒श्वा॒न॒र मवाव॑ वैश्वान॒रं ॅवै᳚श्वान॒र मव॑ रुन्धे रु॒न्धे ऽव॑ वैश्वान॒रं ॅवै᳚श्वान॒र मव॑ रुन्धे । \newline
43. अव॑ रुन्धे रु॒न्धे ऽवाव॑ रुन्धे॒ यं ॅयꣳ रु॒न्धे ऽवाव॑ रुन्धे॒ यम् । \newline
44. रु॒न्धे॒ यं ॅयꣳ रु॑न्धे रुन्धे॒ यम् का॒मये॑त का॒मये॑त॒ यꣳ रु॑न्धे रुन्धे॒ यम् का॒मये॑त । \newline
45. यम् का॒मये॑त का॒मये॑त॒ यं ॅयम् का॒मये॑त॒ क्षोधु॑कः॒ क्षोधु॑कः का॒मये॑त॒ यं ॅयम् का॒मये॑त॒ क्षोधु॑कः । \newline
46. का॒मये॑त॒ क्षोधु॑कः॒ क्षोधु॑कः का॒मये॑त का॒मये॑त॒ क्षोधु॑कः स्याथ् स्या॒त् क्षोधु॑कः का॒मये॑त का॒मये॑त॒ क्षोधु॑कः स्यात् । \newline
47. क्षोधु॑कः स्याथ् स्या॒त् क्षोधु॑कः॒ क्षोधु॑कः स्या॒ दितीति॑ स्या॒त् क्षोधु॑कः॒ क्षोधु॑कः स्या॒दिति॑ । \newline
48. स्या॒ दितीति॑ स्याथ् स्या॒दि त्यू॒ना मू॒ना मिति॑ स्याथ् स्या॒ दित्यू॒नाम् । \newline
49. इत्यू॒ना मू॒ना मिती त्यू॒नाम् तस्य॒ तस्यो॒ना मिती त्यू॒नाम् तस्य॑ । \newline
50. ऊ॒नाम् तस्य॒ तस्यो॒ना मू॒नाम् तस्योपोप॒ तस्यो॒ना मू॒नाम् तस्योप॑ । \newline
51. तस्योपोप॒ तस्य॒ तस्योप॑ दद्ध्याद् दद्ध्या॒ दुप॒ तस्य॒ तस्योप॑ दद्ध्यात् । \newline
52. उप॑ दद्ध्याद् दद्ध्या॒ दुपोप॑ दद्ध्या॒त् क्षोधु॑कः॒ क्षोधु॑को दद्ध्या॒ दुपोप॑ दद्ध्या॒त् क्षोधु॑कः । \newline
\pagebreak
\markright{ TS 5.2.9.2  \hfill https://www.vedavms.in \hfill}

\section{ TS 5.2.9.2 }

\textbf{TS 5.2.9.2 } \newline
\textbf{Samhita Paata} \newline

दद्ध्या॒त् क्षोधु॑क ए॒व भ॑वति॒ यं का॒मये॒ता-नु॑पदस्य॒-दन्न॑मद्या॒दिति॑ पू॒र्णां तस्योप॑ दद्ध्या॒दनु॑पदस्य-दे॒वान्न॑मत्ति स॒हस्रं॒ ॅवै प्रति॒ पुरु॑षः पशू॒नां ॅय॑च्छति स॒हस्र॑म॒न्ये प॒शवो॒ मद्ध्ये॑ पुरुषशी॒र्॒.षमुप॑ दधाति सवीर्य॒त्वायो॒-खाया॒मपि॑ दधाति प्रति॒ष्ठामे॒वैन॑द्-गमयति॒ व्यृ॑द्धं॒ ॅवा ए॒तत् प्रा॒णैर॑मे॒द्ध्यं ॅयत् पु॑रुषशी॒र्.षम॒मृतं॒ खलु॒ वै प्रा॒णा - [  ] \newline

\textbf{Pada Paata} \newline

द॒द्ध्या॒त् । क्षोधु॑कः । ए॒व । भ॒व॒ति॒ । यम् । का॒मये॑त । अनु॑पदस्य॒दित्यनु॑प - द॒स्य॒त् । अन्न᳚म् । अ॒द्या॒त् । इति॑ । पू॒र्णाम् । तस्य॑ । उपेति॑ । द॒द्ध्या॒त् । अनु॑पदस्य॒दित्यनु॑प - द॒स्य॒त् । ए॒व । अन्न᳚म् । अ॒त्ति॒ । स॒हस्र᳚म् । वै । प्रतीति॑ । पुरु॑षः । प॒शू॒नाम् । य॒च्छ॒ति॒ । स॒हस्र᳚म् । अ॒न्ये । प॒शवः॑ । मद्ध्ये᳚ । पु॒रु॒ष॒शी॒र्॒.षमिति॑ पुरुष - शी॒र्॒.षम् । उपेति॑ । द॒धा॒ति॒ । स॒वी॒र्य॒त्वायेति॑ सवीर्य-त्वाय॑ । उ॒खाया᳚म् । अपीति॑ । द॒धा॒ति॒ । प्र॒ति॒ष्ठामिति॑ प्रति - स्थाम् । ए॒व । ए॒न॒त् । ग॒म॒य॒ति॒ । व्यृ॑द्ध॒मिति॒ वि - ऋ॒द्ध॒म् । वै । ए॒तत् । प्रा॒णैरिति॑ प्र - अ॒नैः । अ॒मे॒द्ध्यम् । यत् । पु॒रु॒ष॒शी॒र्॒.षमिति॑ पुरुष - शी॒र्॒.षम् । अ॒मृत᳚म् । खलु॑ । वै । प्रा॒णा इति॑ प्र - अ॒नाः ।  \newline


\textbf{Krama Paata} \newline

द॒द्ध्या॒त् क्षोधु॑कः । क्षोधु॑क ए॒व । ए॒व भ॑वति । भ॒व॒ति॒ यम् । यम् का॒मये॑त । का॒मये॒तानु॑पदस्यत् । अनु॑पदस्य॒दन्न᳚म् । अनु॑पदस्य॒दित्यनु॑प - द॒स्य॒त्॒ । अन्न॑मद्यात् । अ॒द्या॒दिति॑ । इति॑ पू॒र्णाम् । पू॒र्णाम् तस्य॑ । तस्योप॑ । उप॑ दद्ध्यात् । द॒द्ध्या॒दनु॑पदस्यत् । अनु॑पदस्यदे॒व । अनु॑पदस्य॒दित्यनु॑प - द॒स्य॒त्॒ । ए॒वान्न᳚म् । अन्न॑मत्ति । अ॒त्ति॒ स॒हस्र᳚म् । स॒हस्र॒म् ॅवै । वै प्रति॑ । प्रति॒ पुरु॑षः । पुरु॑षः पशू॒नाम् । प॒शू॒नाम् ॅय॑च्छति । य॒च्छ॒ति॒ स॒हस्र᳚म् । स॒हस्र॑म॒न्ये । अ॒न्ये प॒शवः॑ । प॒शवो॒ मद्ध्ये᳚ । मद्ध्ये॑ पुरुषशी॒र्.॒षम् । पु॒रु॒ष॒शी॒र्.॒षमुप॑ । पु॒रु॒ष॒शी॒र्.॒षमिति॑ पुरुष - शी॒र्.॒षम् । उप॑ दधाति । द॒धा॒ति॒ स॒वी॒र्य॒त्वाय॑ । स॒वी॒र्य॒त्वायो॒खाया᳚म् । स॒वी॒र्य॒त्वायेति॑ सवीर्य - त्वाय॑ । उ॒खाया॒मपि॑ । अपि॑ दधाति । द॒धा॒ति॒ प्र॒ति॒ष्ठाम् । प्र॒ति॒ष्ठामे॒व । प्र॒ति॒ष्ठामिति॑ प्रति - स्थाम् । ए॒वैन॑त् । ए॒न॒द् ग॒म॒य॒ति॒ । ग॒म॒य॒ति॒ व्यृ॑द्धम् । व्यृ॑द्ध॒म् ॅवै । व्यृ॑द्ध॒मिति॒ वि - ऋ॒द्ध॒म् । वा ए॒तत् । ए॒तत् प्रा॒णैः । प्रा॒णैर॑मे॒द्ध्यम् । प्रा॒णैरिति॑ प्र - अ॒नैः । अ॒मे॒द्ध्यम् ॅयत् । यत् पु॑रुषशी॒र्.॒षम् । पु॒रु॒ष॒शी॒र्.॒षम॒मृत᳚म् । पु॒रु॒ष॒शी॒र्.॒षमिति॑ पुरुष - शी॒र्.॒षम् । अ॒मृत॒म् खलु॑ । खलु॒ वै । वै प्रा॒णाः । प्रा॒णा अ॒मृत᳚म् । प्रा॒णा इति॑ प्र - अ॒नाः \newline

\textbf{Jatai Paata} \newline

1. द॒द्ध्या॒त् क्षोधु॑कः॒ क्षोधु॑को दद्ध्याद् दद्ध्या॒त् क्षोधु॑कः । \newline
2. क्षोधु॑क ए॒वैव क्षोधु॑कः॒ क्षोधु॑क ए॒व । \newline
3. ए॒व भ॑वति भव त्ये॒वैव भ॑वति । \newline
4. भ॒व॒ति॒ यं ॅयम् भ॑वति भवति॒ यम् । \newline
5. यम् का॒मये॑त का॒मये॑त॒ यं ॅयम् का॒मये॑त । \newline
6. का॒मये॒ता नु॑पदस्य॒ दनु॑पदस्यत् का॒मये॑त का॒मये॒ता नु॑पदस्यत् । \newline
7. अनु॑पदस्य॒ दन्न॒ मन्न॒ मनु॑पदस्य॒ दनु॑पदस्य॒ दन्न᳚म् । \newline
8. अनु॑पदस्य॒दित्यनु॑प - द॒स्य॒त् । \newline
9. अन्न॑ मद्या दद्या॒ दन्न॒ मन्न॑ मद्यात् । \newline
10. अ॒द्या॒दिती त्य॑द्या दद्या॒ दिति॑ । \newline
11. इति॑ पू॒र्णाम् पू॒र्णा मितीति॑ पू॒र्णाम् । \newline
12. पू॒र्णाम् तस्य॒ तस्य॑ पू॒र्णाम् पू॒र्णाम् तस्य॑ । \newline
13. तस्योपोप॒ तस्य॒ तस्योप॑ । \newline
14. उप॑ दद्ध्याद् दद्ध्या॒ दुपोप॑ दद्ध्यात् । \newline
15. द॒द्ध्या॒ दनु॑पदस्य॒ दनु॑पदस्यद् दद्ध्याद् दद्ध्या॒ दनु॑पदस्यत् । \newline
16. अनु॑पदस्य दे॒वैवा नु॑पदस्य॒ दनु॑पदस्य दे॒व । \newline
17. अनु॑पदस्य॒दित्यनु॑प - द॒स्य॒त् । \newline
18. ए॒वान्न॒ मन्न॑ मे॒वै वान्न᳚म् । \newline
19. अन्न॑ मत्त्य॒त् त्यन्न॒ मन्न॑ मत्ति । \newline
20. अ॒त्ति॒ स॒हस्रꣳ॑ स॒हस्र॑ मत्त्यत्ति स॒हस्र᳚म् । \newline
21. स॒हस्रं॒ ॅवै वै स॒हस्रꣳ॑ स॒हस्रं॒ ॅवै । \newline
22. वै प्रति॒ प्रति॒ वै वै प्रति॑ । \newline
23. प्रति॒ पुरु॑षः॒ पुरु॑षः॒ प्रति॒ प्रति॒ पुरु॑षः । \newline
24. पुरु॑षः पशू॒नाम् प॑शू॒नाम् पुरु॑षः॒ पुरु॑षः पशू॒नाम् । \newline
25. प॒शू॒नां ॅय॑च्छति यच्छति पशू॒नाम् प॑शू॒नां ॅय॑च्छति । \newline
26. य॒च्छ॒ति॒ स॒हस्रꣳ॑ स॒हस्रं॑ ॅयच्छति यच्छति स॒हस्र᳚म् । \newline
27. स॒हस्र॑ म॒न्ये᳚ ऽन्ये स॒हस्रꣳ॑ स॒हस्र॑ म॒न्ये । \newline
28. अ॒न्ये प॒शवः॑ प॒शवो॒ ऽन्ये᳚ ऽन्ये प॒शवः॑ । \newline
29. प॒शवो॒ मद्ध्ये॒ मद्ध्ये॑ प॒शवः॑ प॒शवो॒ मद्ध्ये᳚ । \newline
30. मद्ध्ये॑ पुरुषशी॒र्॒.षम् पु॑रुषशी॒र्॒.षम् मद्ध्ये॒ मद्ध्ये॑ पुरुषशी॒र्॒.षम् । \newline
31. पु॒रु॒ष॒शी॒र्॒.ष मुपोप॑ पुरुषशी॒र्॒.षम् पु॑रुषशी॒र्॒.ष मुप॑ । \newline
32. पु॒रु॒ष॒शी॒र्॒.षमिति॑ पुरुष - शी॒र्॒.षम् । \newline
33. उप॑ दधाति दधा॒ त्युपोप॑ दधाति । \newline
34. द॒धा॒ति॒ स॒वी॒र्य॒त्वाय॑ सवीर्य॒त्वाय॑ दधाति दधाति सवीर्य॒त्वाय॑ । \newline
35. स॒वी॒र्य॒त्वायो॒ खाया॑ मु॒खायाꣳ॑ सवीर्य॒त्वाय॑ सवीर्य॒त्वायो॒ खाया᳚म् । \newline
36. स॒वी॒र्य॒त्वायेति॑ सवीर्य - त्वाय॑ । \newline
37. उ॒खाया॒ मप्य प्यु॒खाया॑ मु॒खाया॒ मपि॑ । \newline
38. अपि॑ दधाति दधा॒ त्यप्यपि॑ दधाति । \newline
39. द॒धा॒ति॒ प्र॒ति॒ष्ठाम् प्र॑ति॒ष्ठाम् द॑धाति दधाति प्रति॒ष्ठाम् । \newline
40. प्र॒ति॒ष्ठा मे॒वैव प्र॑ति॒ष्ठाम् प्र॑ति॒ष्ठा मे॒व । \newline
41. प्र॒ति॒ष्ठामिति॑ प्रति - स्थाम् । \newline
42. ए॒वैन॑ देन दे॒वै वैन॑त् । \newline
43. ए॒न॒द् ग॒म॒य॒ति॒ ग॒म॒य॒ त्ये॒न॒ दे॒न॒द् ग॒म॒य॒ति॒ । \newline
44. ग॒म॒य॒ति॒ व्यृ॑द्धं॒ ॅव्यृ॑द्धम् गमयति गमयति॒ व्यृ॑द्धम् । \newline
45. व्यृ॑द्धं॒ ॅवै वै व्यृ॑द्धं॒ ॅव्यृ॑द्धं॒ ॅवै । \newline
46. व्यृ॑द्ध॒मिति॒ वि - ऋ॒द्ध॒म् । \newline
47. वा ए॒त दे॒तद् वै वा ए॒तत् । \newline
48. ए॒तत् प्रा॒णैः प्रा॒णै रे॒त दे॒तत् प्रा॒णैः । \newline
49. प्रा॒णै र॑मे॒द्ध्य म॑मे॒द्ध्यम् प्रा॒णैः प्रा॒णै र॑मे॒द्ध्यम् । \newline
50. प्रा॒णैरिति॑ प्र - अ॒नैः । \newline
51. अ॒मे॒द्ध्यं ॅयद् यद॑मे॒द्ध्य म॑मे॒द्ध्यं ॅयत् । \newline
52. यत् पु॑रुषशी॒र्॒.षम् पु॑रुषशी॒र्॒.षं ॅयद् यत् पु॑रुषशी॒र्॒.षम् । \newline
53. पु॒रु॒ष॒शी॒र्॒.ष म॒मृत॑ म॒मृत॑म् पुरुषशी॒र्॒.षम् पु॑रुषशी॒र्॒.ष म॒मृत᳚म् । \newline
54. पु॒रु॒ष॒शी॒र्॒.षमिति॑ पुरुष - शी॒र्॒.षम् । \newline
55. अ॒मृत॒म् खलु॒ खल्व॒मृत॑ म॒मृत॒म् खलु॑ । \newline
56. खलु॒ वै वै खलु॒ खलु॒ वै । \newline
57. वै प्रा॒णाः प्रा॒णा वै वै प्रा॒णाः । \newline
58. प्रा॒णा अ॒मृत॑ म॒मृत॑म् प्रा॒णाः प्रा॒णा अ॒मृत᳚म् । \newline
59. प्रा॒णा इति॑ प्र - अ॒नाः । \newline

\textbf{Ghana Paata } \newline

1. द॒द्ध्या॒त् क्षोधु॑कः॒ क्षोधु॑को दद्ध्याद् दद्ध्या॒त् क्षोधु॑क ए॒वैव क्षोधु॑को दद्ध्याद् दद्ध्या॒त् क्षोधु॑क ए॒व । \newline
2. क्षोधु॑क ए॒वैव क्षोधु॑कः॒ क्षोधु॑क ए॒व भ॑वति भवत्ये॒व क्षोधु॑कः॒ क्षोधु॑क ए॒व भ॑वति । \newline
3. ए॒व भ॑वति भव त्ये॒वैव भ॑वति॒ यं ॅयम् भ॑व त्ये॒वैव भ॑वति॒ यम् । \newline
4. भ॒व॒ति॒ यं ॅयम् भ॑वति भवति॒ यम् का॒मये॑त का॒मये॑त॒ यम् भ॑वति भवति॒ यम् का॒मये॑त । \newline
5. यम् का॒मये॑त का॒मये॑त॒ यं ॅयम् का॒मये॒ता नु॑पदस्य॒ दनु॑पदस्यत् का॒मये॑त॒ यं ॅयम् का॒मये॒ता नु॑पदस्यत् । \newline
6. का॒मये॒ता नु॑पदस्य॒ दनु॑पदस्यत् का॒मये॑त का॒मये॒ता नु॑पदस्य॒ दन्न॒ मन्न॒ मनु॑पदस्यत् का॒मये॑त का॒मये॒ता नु॑पदस्य॒ दन्न᳚म् । \newline
7. अनु॑पदस्य॒ दन्न॒ मन्न॒ मनु॑पदस्य॒ दनु॑पदस्य॒ दन्न॑ मद्यादद्या॒ दन्न॒ मनु॑पदस्य॒ दनु॑पदस्य॒ दन्न॑ मद्यात् । \newline
8. अनु॑पदस्य॒दित्यनु॑प - द॒स्य॒त् । \newline
9. अन्न॑ मद्या दद्या॒ दन्न॒ मन्न॑ मद्या॒दिती त्य॑द्या॒ दन्न॒ मन्न॑ मद्या॒दिति॑ । \newline
10. अ॒द्या॒दिती त्य॑द्या दद्या॒ दिति॑ पू॒र्णाम् पू॒र्णा मित्य॑द्या दद्या॒ दिति॑ पू॒र्णाम् । \newline
11. इति॑ पू॒र्णाम् पू॒र्णा मितीति॑ पू॒र्णाम् तस्य॒ तस्य॑ पू॒र्णा मितीति॑ पू॒र्णाम् तस्य॑ । \newline
12. पू॒र्णाम् तस्य॒ तस्य॑ पू॒र्णाम् पू॒र्णाम् तस्योपोप॒ तस्य॑ पू॒र्णाम् पू॒र्णाम् तस्योप॑ । \newline
13. तस्योपोप॒ तस्य॒ तस्योप॑ दद्ध्याद् दद्ध्या॒ दुप॒ तस्य॒ तस्योप॑ दद्ध्यात् । \newline
14. उप॑ दद्ध्याद् दद्ध्या॒ दुपोप॑ दद्ध्या॒ दनु॑पदस्य॒ दनु॑पदस्यद् दद्ध्या॒ दुपोप॑ दद्ध्या॒ दनु॑पदस्यत् । \newline
15. द॒द्ध्या॒ दनु॑पदस्य॒ दनु॑पदस्यद् दद्ध्याद् दद्ध्या॒ दनु॑पदस्य दे॒वैवा नु॑पदस्यद् दद्ध्याद् दद्ध्या॒ दनु॑पदस्यदे॒व । \newline
16. अनु॑पदस्य दे॒वैवा नु॑पदस्य॒ दनु॑पदस्य दे॒वान्न॒ मन्न॑ मे॒वानु॑पदस्य॒ दनु॑पदस्य दे॒वान्न᳚म् । \newline
17. अनु॑पदस्य॒दित्यनु॑प - द॒स्य॒त् । \newline
18. ए॒वान्न॒ मन्न॑ मे॒वैवान्न॑ मत्त्य॒ त्त्यन्न॑ मे॒वैवान्न॑ मत्ति । \newline
19. अन्न॑ मत्त्य॒ त्त्यन्न॒ मन्न॑ मत्ति स॒हस्रꣳ॑ स॒हस्र॑ म॒त्त्यन्न॒ मन्न॑ मत्ति स॒हस्र᳚म् । \newline
20. अ॒त्ति॒ स॒हस्रꣳ॑ स॒हस्र॑ मत्त्यत्ति स॒हस्रं॒ ॅवै वै स॒हस्र॑ मत्त्यत्ति स॒हस्रं॒ ॅवै । \newline
21. स॒हस्रं॒ ॅवै वै स॒हस्रꣳ॑ स॒हस्रं॒ ॅवै प्रति॒ प्रति॒ वै स॒हस्रꣳ॑ स॒हस्रं॒ ॅवै प्रति॑ । \newline
22. वै प्रति॒ प्रति॒ वै वै प्रति॒ पुरु॑षः॒ पुरु॑षः॒ प्रति॒ वै वै प्रति॒ पुरु॑षः । \newline
23. प्रति॒ पुरु॑षः॒ पुरु॑षः॒ प्रति॒ प्रति॒ पुरु॑षः पशू॒नाम् प॑शू॒नाम् पुरु॑षः॒ प्रति॒ प्रति॒ पुरु॑षः पशू॒नाम् । \newline
24. पुरु॑षः पशू॒नाम् प॑शू॒नाम् पुरु॑षः॒ पुरु॑षः पशू॒नां ॅय॑च्छति यच्छति पशू॒नाम् पुरु॑षः॒ पुरु॑षः पशू॒नां ॅय॑च्छति । \newline
25. प॒शू॒नां ॅय॑च्छति यच्छति पशू॒नाम् प॑शू॒नां ॅय॑च्छति स॒हस्रꣳ॑ स॒हस्रं॑ ॅयच्छति पशू॒नाम् प॑शू॒नां ॅय॑च्छति स॒हस्र᳚म् । \newline
26. य॒च्छ॒ति॒ स॒हस्रꣳ॑ स॒हस्रं॑ ॅयच्छति यच्छति स॒हस्र॑ म॒न्ये᳚ ऽन्ये स॒हस्रं॑ ॅयच्छति यच्छति स॒हस्र॑ म॒न्ये । \newline
27. स॒हस्र॑ म॒न्ये᳚ ऽन्ये स॒हस्रꣳ॑ स॒हस्र॑ म॒न्ये प॒शवः॑ प॒शवो॒ ऽन्ये स॒हस्रꣳ॑ स॒हस्र॑ म॒न्ये प॒शवः॑ । \newline
28. अ॒न्ये प॒शवः॑ प॒शवो॒ ऽन्ये᳚ ऽन्ये प॒शवो॒ मद्ध्ये॒ मद्ध्ये॑ प॒शवो॒ ऽन्ये᳚ ऽन्ये प॒शवो॒ मद्ध्ये᳚ । \newline
29. प॒शवो॒ मद्ध्ये॒ मद्ध्ये॑ प॒शवः॑ प॒शवो॒ मद्ध्ये॑ पुरुषशी॒र्॒.षम् पु॑रुषशी॒र्॒.षम् मद्ध्ये॑ प॒शवः॑ प॒शवो॒ मद्ध्ये॑ पुरुषशी॒र्॒.षम् । \newline
30. मद्ध्ये॑ पुरुषशी॒र्॒.षम् पु॑रुषशी॒र्॒.षम् मद्ध्ये॒ मद्ध्ये॑ पुरुषशी॒र्॒.ष मुपोप॑ पुरुषशी॒र्॒.षम् मद्ध्ये॒ मद्ध्ये॑ पुरुषशी॒र्॒.ष मुप॑ । \newline
31. पु॒रु॒ष॒शी॒र्॒.ष मुपोप॑ पुरुषशी॒र्॒.षम् पु॑रुषशी॒र्॒.ष मुप॑ दधाति दधा॒ त्युप॑ पुरुषशी॒र्॒.षम् पु॑रुषशी॒र्॒.ष मुप॑ दधाति । \newline
32. पु॒रु॒ष॒शी॒र्॒.षमिति॑ पुरुष - शी॒र्॒.षम् । \newline
33. उप॑ दधाति दधा॒ त्युपोप॑ दधाति सवीर्य॒त्वाय॑ सवीर्य॒त्वाय॑ दधा॒ त्युपोप॑ दधाति सवीर्य॒त्वाय॑ । \newline
34. द॒धा॒ति॒ स॒वी॒र्य॒त्वाय॑ सवीर्य॒त्वाय॑ दधाति दधाति सवीर्य॒त्वा यो॒खाया॑ मु॒खायाꣳ॑ सवीर्य॒त्वाय॑ दधाति दधाति सवीर्य॒त्वा यो॒खाया᳚म् । \newline
35. स॒वी॒र्य॒त्वा यो॒खाया॑ मु॒खायाꣳ॑ सवीर्य॒त्वाय॑ सवीर्य॒त्वा यो॒खाया॒ मप्य प्यु॒खायाꣳ॑ सवीर्य॒त्वाय॑ सवीर्य॒त्वा यो॒खाया॒ मपि॑ । \newline
36. स॒वी॒र्य॒त्वायेति॑ सवीर्य - त्वाय॑ । \newline
37. उ॒खाया॒ मप्य प्यु॒खाया॑ मु॒खाया॒ मपि॑ दधाति दधा॒ त्यप्यु॒खाया॑ मु॒खाया॒ मपि॑ दधाति । \newline
38. अपि॑ दधाति दधा॒ त्यप्यपि॑ दधाति प्रति॒ष्ठाम् प्र॑ति॒ष्ठाम् द॑धा॒ त्यप्यपि॑ दधाति प्रति॒ष्ठाम् । \newline
39. द॒धा॒ति॒ प्र॒ति॒ष्ठाम् प्र॑ति॒ष्ठाम् द॑धाति दधाति प्रति॒ष्ठा मे॒वैव प्र॑ति॒ष्ठाम् द॑धाति दधाति प्रति॒ष्ठा मे॒व । \newline
40. प्र॒ति॒ष्ठा मे॒वैव प्र॑ति॒ष्ठाम् प्र॑ति॒ष्ठा मे॒वैन॑ देनदे॒व प्र॑ति॒ष्ठाम् प्र॑ति॒ष्ठा मे॒वैन॑त् । \newline
41. प्र॒ति॒ष्ठामिति॑ प्रति - स्थाम् । \newline
42. ए॒वैन॑ देन दे॒वैवैन॑द् गमयति गमय त्येन दे॒वैवैन॑द् गमयति । \newline
43. ए॒न॒द् ग॒म॒य॒ति॒ ग॒म॒य॒ त्ये॒न॒ दे॒न॒द् ग॒म॒य॒ति॒ व्यृ॑द्धं॒ ॅव्यृ॑द्धम् गमय त्येन देनद् गमयति॒ व्यृ॑द्धम् । \newline
44. ग॒म॒य॒ति॒ व्यृ॑द्धं॒ ॅव्यृ॑द्धम् गमयति गमयति॒ व्यृ॑द्धं॒ ॅवै वै व्यृ॑द्धम् गमयति गमयति॒ व्यृ॑द्धं॒ ॅवै । \newline
45. व्यृ॑द्धं॒ ॅवै वै व्यृ॑द्धं॒ ॅव्यृ॑द्धं॒ ॅवा ए॒त दे॒तद् वै व्यृ॑द्धं॒ ॅव्यृ॑द्धं॒ ॅवा ए॒तत् । \newline
46. व्यृ॑द्ध॒मिति॒ वि - ऋ॒द्ध॒म् । \newline
47. वा ए॒त दे॒तद् वै वा ए॒तत् प्रा॒णैः प्रा॒णै रे॒तद् वै वा ए॒तत् प्रा॒णैः । \newline
48. ए॒तत् प्रा॒णैः प्रा॒णै रे॒त दे॒तत् प्रा॒णै र॑मे॒द्ध्य म॑मे॒द्ध्यम् प्रा॒णै रे॒त दे॒तत् प्रा॒णै र॑मे॒द्ध्यम् । \newline
49. प्रा॒णै र॑मे॒द्ध्य म॑मे॒द्ध्यम् प्रा॒णैः प्रा॒णै र॑मे॒द्ध्यं ॅयद् यद॑मे॒द्ध्यम् प्रा॒णैः प्रा॒णै र॑मे॒द्ध्यं ॅयत् । \newline
50. प्रा॒णैरिति॑ प्र - अ॒नैः । \newline
51. अ॒मे॒द्ध्यं ॅयद् यद॑मे॒द्ध्य म॑मे॒द्ध्यं ॅयत् पु॑रुषशी॒र्॒.षम् पु॑रुषशी॒र्॒.षं ॅयद॑मे॒द्ध्य म॑मे॒द्ध्यं ॅयत् पु॑रुषशी॒र्॒.षम् । \newline
52. यत् पु॑रुषशी॒र्॒.षम् पु॑रुषशी॒र्॒.षं ॅयद् यत् पु॑रुषशी॒र्॒.ष म॒मृत॑ म॒मृत॑म् पुरुषशी॒र्॒.षं ॅयद् यत् पु॑रुषशी॒र्॒.ष म॒मृत᳚म् । \newline
53. पु॒रु॒ष॒शी॒र्॒.ष म॒मृत॑ म॒मृत॑म् पुरुषशी॒र्॒.षम् पु॑रुषशी॒र्॒.ष म॒मृत॒म् खलु॒ खल्व॒मृत॑म् पुरुषशी॒र्॒.षम् पु॑रुषशी॒र्॒.ष म॒मृत॒म् खलु॑ । \newline
54. पु॒रु॒ष॒शी॒र्॒.षमिति॑ पुरुष - शी॒र्॒.षम् । \newline
55. अ॒मृत॒म् खलु॒ खल्व॒मृत॑ म॒मृत॒म् खलु॒ वै वै खल्व॒मृत॑ म॒मृत॒म् खलु॒ वै । \newline
56. खलु॒ वै वै खलु॒ खलु॒ वै प्रा॒णाः प्रा॒णा वै खलु॒ खलु॒ वै प्रा॒णाः । \newline
57. वै प्रा॒णाः प्रा॒णा वै वै प्रा॒णा अ॒मृत॑ म॒मृत॑म् प्रा॒णा वै वै प्रा॒णा अ॒मृत᳚म् । \newline
58. प्रा॒णा अ॒मृत॑ म॒मृत॑म् प्रा॒णाः प्रा॒णा अ॒मृतꣳ॒॒ हिर॑ण्यꣳ॒॒ हिर॑ण्य म॒मृत॑म् प्रा॒णाः प्रा॒णा अ॒मृतꣳ॒॒ हिर॑ण्यम् । \newline
59. प्रा॒णा इति॑ प्र - अ॒नाः । \newline
\pagebreak
\markright{ TS 5.2.9.3  \hfill https://www.vedavms.in \hfill}

\section{ TS 5.2.9.3 }

\textbf{TS 5.2.9.3 } \newline
\textbf{Samhita Paata} \newline

अ॒मृतꣳ॒॒ हिर॑ण्यं प्रा॒णेषु॑ हिरण्यश॒ल्कान् प्रत्य॑स्यति प्रति॒ष्ठामे॒वैन॑द्-गमयि॒त्वा प्रा॒णैः सम॑र्द्धयति द॒द्ध्ना म॑धुमि॒श्रेण॑ पूरयति मध॒व्यो॑ऽसा॒नीति॑ शृतात॒ङ्क्ये॑न मेद्ध्य॒त्वाय॑ ग्रा॒म्यं ॅवा ए॒तदन्नं॒ ॅयद्-दद्ध्या॑र॒ण्यं मधु॒ यद्द॒द्ध्ना म॑धुमि॒श्रेण॑ पू॒रय॑त्यु॒भय॒स्या-व॑रुद्ध्यै पशुशी॒र्॒.षाण्युप॑ दधाति प॒शवो॒ वै प॑शुशी॒र्॒.षाणि॑ प॒शूने॒वाव॑ रुन्धे॒ यं का॒मये॑ताप॒शुः स्या॒दिति॑ - [  ] \newline

\textbf{Pada Paata} \newline

अ॒मृत᳚म् । हिर॑ण्यम् । प्रा॒णेष्विति॑ प्र - अ॒नेषु॑ । हि॒र॒ण्य॒श॒ल्कानिति॑ हिरण्य - श॒ल्कान् । प्रतीति॑ । अ॒स्य॒ति॒ । प्र॒ति॒ष्ठामिति॑ प्रति-स्थाम् । ए॒व । ए॒न॒त् । ग॒म॒यि॒त्वा । प्रा॒णैरिति॑ प्र-अ॒नैः । समिति॑ । अ॒द्‌र्ध॒य॒ति॒ । द॒द्ध्ना । म॒धु॒मि॒श्रेणेति॑ मधु-मि॒श्रेण॑ । पू॒र॒य॒ति॒ । म॒ध॒व्यः॑ । अ॒सा॒नि॒ । इति॑ । शृ॒ता॒त॒ङ्क्ये॑नेति॑ शृत - आ॒त॒ङ्क्ये॑न । मे॒द्ध्य॒त्वायेति॑ मेद्ध्य - त्वाय॑ । ग्रा॒म्यम् । वै । ए॒तत् । अन्न᳚म् । यत् । दधि॑ । आ॒र॒ण्यम् । मधु॑ । यत् । द॒द्ध्ना । म॒धु॒मि॒श्रेणेति॑ मधु - मि॒श्रेण॑ । पू॒रय॑ति । उ॒भय॑स्य । अव॑रुद्ध्या॒ इत्यव॑ - रु॒द्ध्यै॒ । प॒शु॒शी॒र्॒.षाणीति॑ पशु - शी॒र्॒.षाणि॑ । उपेति॑ । द॒धा॒ति॒ । प॒शवः॑ । वै । प॒शु॒शी॒र्॒.षाणीति॑ पशु - शी॒र्॒.षाणि॑ । प॒शून् । ए॒व । अवेति॑ । रु॒न्धे॒ । यम् । का॒मये॑त । अ॒प॒शुः । स्या॒त् । इति॑ ।  \newline


\textbf{Krama Paata} \newline

अ॒मृतꣳ॒॒ हिर॑ण्यम् । हिर॑ण्यम् प्रा॒णेषु॑ । प्रा॒णेषु॑ हिरण्यश॒ल्कान् । प्रा॒णेष्विति॑ प्र - अ॒नेषु॑ । हि॒र॒ण्य॒श॒ल्कान् प्रति॑ । हि॒र॒ण्य॒श॒ल्कानिति॑ हिरण्य - श॒ल्कान् । प्रत्य॑स्यति । अ॒स्य॒ति॒ प्र॒ति॒ष्ठाम् । प्र॒ति॒ष्ठामे॒व । प्र॒ति॒ष्ठामिति॑ प्रति - स्थाम् । ए॒वैन॑त् । ए॒न॒द् ग॒म॒यि॒त्वा । ग॒म॒यि॒त्वा प्रा॒णैः । प्रा॒णैः सम् । प्रा॒णैरिति॑ प्र - अ॒नैः । सम॑र्द्धयति । अ॒र्द्ध॒य॒ति॒ द॒द्ध्ना । द॒द्ध्ना म॑धुमि॒श्रेण॑ । म॒धु॒मि॒श्रेण॑ पूरयति । म॒धु॒मि॒श्रेणेति॑ मधु - मि॒श्रेण॑ । पू॒र॒य॒ति॒ म॒ध॒व्यः॑ । म॒ध॒व्यो॑ऽसानि । अ॒सा॒नीति॑ । इति॑ शृतात॒ङ्क्ये॑न । शृ॒ता॒त॒ङ्क्ये॑न मेद्ध्य॒त्वाय॑ । शृ॒ता॒त॒ङ्क्ये॑नेति॑ शृत - आ॒त॒ङ्क्ये॑न । मे॒द्ध्य॒त्वाय॑ ग्रा॒म्यम् । मे॒द्ध्य॒त्वायेति॑ मेद्ध्य - त्वाय॑ । ग्रा॒म्यम् ॅवै । वा ए॒तत् । ए॒तदन्न᳚म् । अन्न॒म् ॅयत् । यद् दधि॑ । दद्ध्या॑र॒ण्यम् । आ॒र॒ण्यम् मधु॑ । मधु॒ यत् । यद् द॒द्ध्ना । द॒द्ध्ना म॑धुमि॒श्रेण॑ । म॒धु॒मि॒श्रेण॑ पू॒रय॑ति । म॒धु॒मि॒श्रेणेति॑ मधु - मि॒श्रेण॑ । पू॒रय॑त्यु॒भय॑स्य । उ॒भय॒स्याव॑रुद्ध्ये । अव॑रुद्ध्ये पशुशी॒र्.॒षाणि॑ । अव॑रुद्ध्या॒ इत्यव॑ - रु॒द्ध्यै॒ । प॒शु॒शी॒र्.॒षाण्युप॑ । प॒शु॒शी॒र्.॒षाणीति॑ पशु - शी॒र्.॒षाणि॑ । उप॑ दधाति । द॒धा॒ति॒ प॒शवः॑ । प॒शवो॒ वै । वै प॑शुशी॒र्.॒षाणि॑ । प॒शु॒शी॒र्.॒षाणि॑ प॒शून् । प॒शु॒शी॒र्.॒षाणीति॑ पशु - शी॒र्.॒षाणि॑ । प॒शूने॒व । ए॒वाव॑ । अव॑ रुन्धे । रु॒न्धे॒ यम् । यम् का॒मये॑त । का॒मये॑ताप॒शुः । अ॒प॒शुः स्या᳚त् । स्या॒दिति॑ । इति॑ विषू॒चीना॑नि \newline

\textbf{Jatai Paata} \newline

1. अ॒मृतꣳ॒॒ हिर॑ण्यꣳ॒॒ हिर॑ण्य म॒मृत॑ म॒मृतꣳ॒॒ हिर॑ण्यम् । \newline
2. हिर॑ण्यम् प्रा॒णेषु॑ प्रा॒णेषु॒ हिर॑ण्यꣳ॒॒ हिर॑ण्यम् प्रा॒णेषु॑ । \newline
3. प्रा॒णेषु॑ हिरण्यश॒ल्कान्. हि॑रण्यश॒ल्कान् प्रा॒णेषु॑ प्रा॒णेषु॑ हिरण्यश॒ल्कान् । \newline
4. प्रा॒णेष्विति॑ प्र - अ॒नेषु॑ । \newline
5. हि॒र॒ण्य॒श॒ल्कान् प्रति॒ प्रति॑ हिरण्यश॒ल्कान्. हि॑रण्यश॒ल्कान् प्रति॑ । \newline
6. हि॒र॒ण्य॒श॒ल्कानिति॑ हिरण्य - श॒ल्कान् । \newline
7. प्रत्य॑स्य त्यस्यति॒ प्रति॒ प्रत्य॑स्यति । \newline
8. अ॒स्य॒ति॒ प्र॒ति॒ष्ठाम् प्र॑ति॒ष्ठा म॑स्य त्यस्यति प्रति॒ष्ठाम् । \newline
9. प्र॒ति॒ष्ठा मे॒वैव प्र॑ति॒ष्ठाम् प्र॑ति॒ष्ठा मे॒व । \newline
10. प्र॒ति॒ष्ठामिति॑ प्रति - स्थाम् । \newline
11. ए॒वैन॑ देन दे॒वै वैन॑त् । \newline
12. ए॒न॒द् ग॒म॒यि॒त्वा ग॑मयि॒ त्वैन॑ देनद् गमयि॒त्वा । \newline
13. ग॒म॒यि॒त्वा प्रा॒णैः प्रा॒णैर् ग॑मयि॒त्वा ग॑मयि॒त्वा प्रा॒णैः । \newline
14. प्रा॒णैः सꣳ सम् प्रा॒णैः प्रा॒णैः सम् । \newline
15. प्रा॒णैरिति॑ प्र - अ॒नैः । \newline
16. स म॑र्द्धय त्यर्द्धयति॒ सꣳ स म॑र्द्धयति । \newline
17. अ॒र्द्ध॒य॒ति॒ द॒द्ध्ना द॒द्ध्ना ऽर्द्ध॑य त्यर्द्धयति द॒द्ध्ना । \newline
18. द॒द्ध्ना म॑धुमि॒श्रेण॑ मधुमि॒श्रेण॑ द॒द्ध्ना द॒द्ध्ना म॑धुमि॒श्रेण॑ । \newline
19. म॒धु॒मि॒श्रेण॑ पूरयति पूरयति मधुमि॒श्रेण॑ मधुमि॒श्रेण॑ पूरयति । \newline
20. म॒धु॒मि॒श्रेणेति॑ मधु - मि॒श्रेण॑ । \newline
21. पू॒र॒य॒ति॒ म॒ध॒व्यो॑ मध॒व्यः॑ पूरयति पूरयति मध॒व्यः॑ । \newline
22. म॒ध॒व्यो॑ ऽसा न्यसानि मध॒व्यो॑ मध॒व्यो॑ ऽसानि । \newline
23. अ॒सा॒ नीती त्य॑सा न्यसा॒नीति॑ । \newline
24. इति॑ शृतात॒ङ्क्ये॑न शृतात॒ङ्क्ये॑ नेतीति॑ शृतात॒ङ्क्ये॑न । \newline
25. शृ॒ता॒त॒ङ्क्ये॑न मेद्ध्य॒त्वाय॑ मेद्ध्य॒त्वाय॑ शृतात॒ङ्क्ये॑न शृतात॒ङ्क्ये॑न मेद्ध्य॒त्वाय॑ । \newline
26. शृ॒ता॒त॒ङ्क्ये॑नेति॑ शृत - आ॒त॒ङ्क्ये॑न । \newline
27. मे॒द्ध्य॒त्वाय॑ ग्रा॒म्यम् ग्रा॒म्यम् मे᳚द्ध्य॒त्वाय॑ मेद्ध्य॒त्वाय॑ ग्रा॒म्यम् । \newline
28. मे॒द्ध्य॒त्वायेति॑ मेद्ध्य - त्वाय॑ । \newline
29. ग्रा॒म्यं ॅवै वै ग्रा॒म्यम् ग्रा॒म्यं ॅवै । \newline
30. वा ए॒त दे॒तद् वै वा ए॒तत् । \newline
31. ए॒त दन्न॒ मन्न॑ मे॒त दे॒त दन्न᳚म् । \newline
32. अन्नं॒ ॅयद् यदन्न॒ मन्नं॒ ॅयत् । \newline
33. यद् दधि॒ दधि॒ यद् यद् दधि॑ । \newline
34. दध्या॑र॒ण्य मा॑र॒ण्यम् दधि॒ दध्या॑र॒ण्यम् । \newline
35. आ॒र॒ण्यम् मधु॒ मध्वा॑ र॒ण्य मा॑र॒ण्यम् मधु॑ । \newline
36. मधु॒ यद् यन् मधु॒ मधु॒ यत् । \newline
37. यद् द॒द्ध्ना द॒द्ध्ना यद् यद् द॒द्ध्ना । \newline
38. द॒द्ध्ना म॑धुमि॒श्रेण॑ मधुमि॒श्रेण॑ द॒द्ध्ना द॒द्ध्ना म॑धुमि॒श्रेण॑ । \newline
39. म॒धु॒मि॒श्रेण॑ पू॒रय॑ति पू॒रय॑ति मधुमि॒श्रेण॑ मधुमि॒श्रेण॑ पू॒रय॑ति । \newline
40. म॒धु॒मि॒श्रेणेति॑ मधु - मि॒श्रेण॑ । \newline
41. पू॒रय॑ त्यु॒भय॑स्यो॒ भय॑स्य पू॒रय॑ति पू॒रय॑ त्यु॒भय॑स्य । \newline
42. उ॒भय॒स्या व॑रुद्ध्या॒ अव॑रुद्ध्या उ॒भय॑स्यो॒ भय॒स्या व॑रुद्ध्यै । \newline
43. अव॑रुद्ध्यै पशुशी॒र्॒.षाणि॑ पशुशी॒र्॒.षाण्य व॑रुद्ध्या॒ अव॑रुद्ध्यै पशुशी॒र्॒.षाणि॑ । \newline
44. अव॑रुद्ध्या॒ इत्यव॑ - रु॒द्ध्यै॒ । \newline
45. प॒शु॒शी॒र्॒.षा ण्युपोप॑ पशुशी॒र्॒.षाणि॑ पशुशी॒र्॒.षा ण्युप॑ । \newline
46. प॒शु॒शी॒र्॒.षाणीति॑ पशु - शी॒र्॒.षाणि॑ । \newline
47. उप॑ दधाति दधा॒ त्युपोप॑ दधाति । \newline
48. द॒धा॒ति॒ प॒शवः॑ प॒शवो॑ दधाति दधाति प॒शवः॑ । \newline
49. प॒शवो॒ वै वै प॒शवः॑ प॒शवो॒ वै । \newline
50. वै प॑शुशी॒र्॒.षाणि॑ पशुशी॒र्॒.षाणि॒ वै वै प॑शुशी॒र्॒.षाणि॑ । \newline
51. प॒शु॒शी॒र्॒.षाणि॑ प॒शून् प॒शून् प॑शुशी॒र्॒.षाणि॑ पशुशी॒र्॒.षाणि॑ प॒शून् । \newline
52. प॒शु॒शी॒र्॒.षाणीति॑ पशु - शी॒र्॒.षाणि॑ । \newline
53. प॒शू ने॒वैव प॒शून् प॒शू ने॒व । \newline
54. ए॒वावा वै॒वै वाव॑ । \newline
55. अव॑ रुन्धे रु॒न्धे ऽवाव॑ रुन्धे । \newline
56. रु॒न्धे॒ यं ॅयꣳ रु॑न्धे रुन्धे॒ यम् । \newline
57. यम् का॒मये॑त का॒मये॑त॒ यं ॅयम् का॒मये॑त । \newline
58. का॒मये॑ता प॒शु र॑प॒शुः का॒मये॑त का॒मये॑ता प॒शुः । \newline
59. अ॒प॒शुः स्या᳚थ् स्या दप॒शु र॑प॒शुः स्या᳚त् । \newline
60. स्या॒दितीति॑ स्याथ् स्या॒दिति॑ । \newline
61. इति॑ विषू॒चीना॑नि विषू॒चीना॒ नीतीति॑ विषू॒चीना॑नि । \newline

\textbf{Ghana Paata } \newline

1. अ॒मृतꣳ॒॒ हिर॑ण्यꣳ॒॒ हिर॑ण्य म॒मृत॑ म॒मृतꣳ॒॒ हिर॑ण्यम् प्रा॒णेषु॑ प्रा॒णेषु॒ हिर॑ण्य म॒मृत॑ म॒मृतꣳ॒॒ हिर॑ण्यम् प्रा॒णेषु॑ । \newline
2. हिर॑ण्यम् प्रा॒णेषु॑ प्रा॒णेषु॒ हिर॑ण्यꣳ॒॒ हिर॑ण्यम् प्रा॒णेषु॑ हिरण्यश॒ल्कान्. हि॑रण्यश॒ल्कान् प्रा॒णेषु॒ हिर॑ण्यꣳ॒॒ हिर॑ण्यम् प्रा॒णेषु॑ हिरण्यश॒ल्कान् । \newline
3. प्रा॒णेषु॑ हिरण्यश॒ल्कान्. हि॑रण्यश॒ल्कान् प्रा॒णेषु॑ प्रा॒णेषु॑ हिरण्यश॒ल्कान् प्रति॒ प्रति॑ हिरण्यश॒ल्कान् प्रा॒णेषु॑ प्रा॒णेषु॑ हिरण्यश॒ल्कान् प्रति॑ । \newline
4. प्रा॒णेष्विति॑ प्र - अ॒नेषु॑ । \newline
5. हि॒र॒ण्य॒श॒ल्कान् प्रति॒ प्रति॑ हिरण्यश॒ल्कान्. हि॑रण्यश॒ल्कान् प्रत्य॑स्य त्यस्यति॒ प्रति॑ हिरण्यश॒ल्कान्. हि॑रण्यश॒ल्कान् प्रत्य॑स्यति । \newline
6. हि॒र॒ण्य॒श॒ल्कानिति॑ हिरण्य - श॒ल्कान् । \newline
7. प्रत्य॑स्य त्यस्यति॒ प्रति॒ प्रत्य॑स्यति प्रति॒ष्ठाम् प्र॑ति॒ष्ठा म॑स्यति॒ प्रति॒ प्रत्य॑स्यति प्रति॒ष्ठाम् । \newline
8. अ॒स्य॒ति॒ प्र॒ति॒ष्ठाम् प्र॑ति॒ष्ठा म॑स्य त्यस्यति प्रति॒ष्ठा मे॒वैव प्र॑ति॒ष्ठा म॑स्य त्यस्यति प्रति॒ष्ठा मे॒व । \newline
9. प्र॒ति॒ष्ठा मे॒वैव प्र॑ति॒ष्ठाम् प्र॑ति॒ष्ठा मे॒वैन॑ देन दे॒व प्र॑ति॒ष्ठाम् प्र॑ति॒ष्ठा मे॒वैन॑त् । \newline
10. प्र॒ति॒ष्ठामिति॑ प्रति - स्थाम् । \newline
11. ए॒वैन॑ देन दे॒वैवैन॑द् गमयि॒त्वा ग॑मयि॒ त्वैन॑ दे॒वैवैन॑द् गमयि॒त्वा । \newline
12. ए॒न॒द् ग॒म॒यि॒त्वा ग॑मयि॒ त्वैन॑देनद् गमयि॒त्वा प्रा॒णैः प्रा॒णैर् ग॑मयि॒ त्वैन॑ देनद् गमयि॒त्वा प्रा॒णैः । \newline
13. ग॒म॒यि॒त्वा प्रा॒णैः प्रा॒णैर् ग॑मयि॒त्वा ग॑मयि॒त्वा प्रा॒णैः सꣳ सम् प्रा॒णैर् ग॑मयि॒त्वा ग॑मयि॒त्वा प्रा॒णैः सम् । \newline
14. प्रा॒णैः सꣳ सम् प्रा॒णैः प्रा॒णैः स म॑र्द्धय त्यर्द्धयति॒ सम् प्रा॒णैः प्रा॒णैः स म॑र्द्धयति । \newline
15. प्रा॒णैरिति॑ प्र - अ॒नैः । \newline
16. स म॑र्द्धय त्यर्द्धयति॒ सꣳ स म॑र्द्धयति द॒द्ध्ना द॒द्ध्ना ऽर्द्ध॑यति॒ सꣳ स म॑र्द्धयति द॒द्ध्ना । \newline
17. अ॒र्द्ध॒य॒ति॒ द॒द्ध्ना द॒द्ध्ना ऽर्द्ध॑य त्यर्द्धयति द॒द्ध्ना म॑धुमि॒श्रेण॑ मधुमि॒श्रेण॑ द॒द्ध्ना ऽर्द्ध॑य त्यर्द्धयति द॒द्ध्ना म॑धुमि॒श्रेण॑ । \newline
18. द॒द्ध्ना म॑धुमि॒श्रेण॑ मधुमि॒श्रेण॑ द॒द्ध्ना द॒द्ध्ना म॑धुमि॒श्रेण॑ पूरयति पूरयति मधुमि॒श्रेण॑ द॒द्ध्ना द॒द्ध्ना म॑धुमि॒श्रेण॑ पूरयति । \newline
19. म॒धु॒मि॒श्रेण॑ पूरयति पूरयति मधुमि॒श्रेण॑ मधुमि॒श्रेण॑ पूरयति मध॒व्यो॑ मध॒व्यः॑ पूरयति मधुमि॒श्रेण॑ मधुमि॒श्रेण॑ पूरयति मध॒व्यः॑ । \newline
20. म॒धु॒मि॒श्रेणेति॑ मधु - मि॒श्रेण॑ । \newline
21. पू॒र॒य॒ति॒ म॒ध॒व्यो॑ मध॒व्यः॑ पूरयति पूरयति मध॒व्यो॑ ऽसा न्यसानि मध॒व्यः॑ पूरयति पूरयति मध॒व्यो॑ ऽसानि । \newline
22. म॒ध॒व्यो॑ ऽसा न्यसानि मध॒व्यो॑ मध॒व्यो॑ ऽसा॒नीती त्य॑सानि मध॒व्यो॑ मध॒व्यो॑ ऽसा॒नीति॑ । \newline
23. अ॒सा॒नीती त्य॑सा न्यसा॒नीति॑ शृतात॒ङ्क्ये॑न शृतात॒ङ्क्ये॑ने त्य॑सा न्यसा॒नीति॑ शृतात॒ङ्क्ये॑न । \newline
24. इति॑ शृतात॒ङ्क्ये॑न शृतात॒ङ्क्ये॑ नेतीति॑ शृतात॒ङ्क्ये॑न मेद्ध्य॒त्वाय॑ मेद्ध्य॒त्वाय॑ शृतात॒ङ्क्ये॑ने तीति॑ शृतात॒ङ्क्ये॑न मेद्ध्य॒त्वाय॑ । \newline
25. शृ॒ता॒त॒ङ्क्ये॑न मेद्ध्य॒त्वाय॑ मेद्ध्य॒त्वाय॑ शृतात॒ङ्क्ये॑न शृतात॒ङ्क्ये॑न मेद्ध्य॒त्वाय॑ ग्रा॒म्यम् ग्रा॒म्यम् मे᳚द्ध्य॒त्वाय॑ शृतात॒ङ्क्ये॑न शृतात॒ङ्क्ये॑न मेद्ध्य॒त्वाय॑ ग्रा॒म्यम् । \newline
26. शृ॒ता॒त॒ङ्क्ये॑नेति॑ शृत - आ॒त॒ङ्क्ये॑न । \newline
27. मे॒द्ध्य॒त्वाय॑ ग्रा॒म्यम् ग्रा॒म्यम् मे᳚द्ध्य॒त्वाय॑ मेद्ध्य॒त्वाय॑ ग्रा॒म्यं ॅवै वै ग्रा॒म्यम् मे᳚द्ध्य॒त्वाय॑ मेद्ध्य॒त्वाय॑ ग्रा॒म्यं ॅवै । \newline
28. मे॒द्ध्य॒त्वायेति॑ मेद्ध्य - त्वाय॑ । \newline
29. ग्रा॒म्यं ॅवै वै ग्रा॒म्यम् ग्रा॒म्यं ॅवा ए॒त दे॒तद् वै ग्रा॒म्यम् ग्रा॒म्यं ॅवा ए॒तत् । \newline
30. वा ए॒त दे॒तद् वै वा ए॒त दन्न॒ मन्न॑ मे॒तद् वै वा ए॒त दन्न᳚म् । \newline
31. ए॒त दन्न॒ मन्न॑ मे॒त दे॒त दन्नं॒ ॅयद् यदन्न॑ मे॒त दे॒त दन्नं॒ ॅयत् । \newline
32. अन्नं॒ ॅयद् यदन्न॒ मन्नं॒ ॅयद् दधि॒ दधि॒ यदन्न॒ मन्नं॒ ॅयद् दधि॑ । \newline
33. यद् दधि॒ दधि॒ यद् यद् दध्या॑र॒ण्य मा॑र॒ण्यम् दधि॒ यद् यद् दध्या॑र॒ण्यम् । \newline
34. दध्या॑र॒ण्य मा॑र॒ण्यम् दधि॒ दध्या॑र॒ण्यम् मधु॒ मध्वा॑र॒ण्यम् दधि॒ दध्या॑र॒ण्यम् मधु॑ । \newline
35. आ॒र॒ण्यम् मधु॒ मध्वा॑र॒ण्य मा॑र॒ण्यम् मधु॒ यद् यन् मध्वा॑र॒ण्य मा॑र॒ण्यम् मधु॒ यत् । \newline
36. मधु॒ यद् यन् मधु॒ मधु॒ यद् द॒द्ध्ना द॒द्ध्ना यन् मधु॒ मधु॒ यद् द॒द्ध्ना । \newline
37. यद् द॒द्ध्ना द॒द्ध्ना यद् यद् द॒द्ध्ना म॑धुमि॒श्रेण॑ मधुमि॒श्रेण॑ द॒द्ध्ना यद् यद् द॒द्ध्ना म॑धुमि॒श्रेण॑ । \newline
38. द॒द्ध्ना म॑धुमि॒श्रेण॑ मधुमि॒श्रेण॑ द॒द्ध्ना द॒द्ध्ना म॑धुमि॒श्रेण॑ पू॒रय॑ति पू॒रय॑ति मधुमि॒श्रेण॑ द॒द्ध्ना द॒द्ध्ना म॑धुमि॒श्रेण॑ पू॒रय॑ति । \newline
39. म॒धु॒मि॒श्रेण॑ पू॒रय॑ति पू॒रय॑ति मधुमि॒श्रेण॑ मधुमि॒श्रेण॑ पू॒रय॑ त्यु॒भय॑ स्यो॒भय॑स्य पू॒रय॑ति मधुमि॒श्रेण॑ मधुमि॒श्रेण॑ पू॒रय॑ त्यु॒भय॑स्य । \newline
40. म॒धु॒मि॒श्रेणेति॑ मधु - मि॒श्रेण॑ । \newline
41. पू॒रय॑त्यु॒भय॑ स्यो॒भय॑स्य पू॒रय॑ति पू॒रय॑ त्यु॒भय॒स्या व॑रुद्ध्या॒ अव॑रुद्ध्या उ॒भय॑स्य पू॒रय॑ति पू॒रय॑ त्यु॒भय॒स्या व॑रुद्ध्यै । \newline
42. उ॒भय॒स्या व॑रुद्ध्या॒ अव॑रुद्ध्या उ॒भय॑ स्यो॒भय॒स्या व॑रुद्ध्यै पशुशी॒र्॒.षाणि॑ पशुशी॒र्॒.षा ण्यव॑रुद्ध्या उ॒भय॑ स्यो॒भय॒स्या व॑रुद्ध्यै पशुशी॒र्॒.षाणि॑ । \newline
43. अव॑रुद्ध्यै पशुशी॒र्॒.षाणि॑ पशुशी॒र्॒.षा ण्यव॑रुद्ध्या॒ अव॑रुद्ध्यै पशुशी॒र्॒.षा ण्युपोप॑ पशुशी॒र्॒.षा ण्यव॑रुद्ध्या॒ अव॑रुद्ध्यै पशुशी॒र्॒.षाण्युप॑ । \newline
44. अव॑रुद्ध्या॒ इत्यव॑ - रु॒द्ध्यै॒ । \newline
45. प॒शु॒शी॒र्॒.षा ण्युपोप॑ पशुशी॒र्॒.षाणि॑ पशुशी॒र्॒.षा ण्युप॑ दधाति दधा॒ त्युप॑ पशुशी॒र्॒.षाणि॑ पशुशी॒र्॒.षा ण्युप॑ दधाति । \newline
46. प॒शु॒शी॒र्॒.षाणीति॑ पशु - शी॒र्॒.षाणि॑ । \newline
47. उप॑ दधाति दधा॒ त्युपोप॑ दधाति प॒शवः॑ प॒शवो॑ दधा॒ त्युपोप॑ दधाति प॒शवः॑ । \newline
48. द॒धा॒ति॒ प॒शवः॑ प॒शवो॑ दधाति दधाति प॒शवो॒ वै वै प॒शवो॑ दधाति दधाति प॒शवो॒ वै । \newline
49. प॒शवो॒ वै वै प॒शवः॑ प॒शवो॒ वै प॑शुशी॒र्॒.षाणि॑ पशुशी॒र्॒.षाणि॒ वै प॒शवः॑ प॒शवो॒ वै प॑शुशी॒र्॒.षाणि॑ । \newline
50. वै प॑शुशी॒र्॒.षाणि॑ पशुशी॒र्॒.षाणि॒ वै वै प॑शुशी॒र्॒.षाणि॑ प॒शून् प॒शून् प॑शुशी॒र्॒.षाणि॒ वै वै प॑शुशी॒र्॒.षाणि॑ प॒शून् । \newline
51. प॒शु॒शी॒र्॒.षाणि॑ प॒शून् प॒शून् प॑शुशी॒र्॒.षाणि॑ पशुशी॒र्॒.षाणि॑ प॒शू ने॒वैव प॒शून् प॑शुशी॒र्॒.षाणि॑ पशुशी॒र्॒.षाणि॑ प॒शू ने॒व । \newline
52. प॒शु॒शी॒र्॒.षाणीति॑ पशु - शी॒र्॒.षाणि॑ । \newline
53. प॒शू ने॒वैव प॒शून् प॒शू ने॒वावा वै॒व प॒शून् प॒शू ने॒वाव॑ । \newline
54. ए॒वावा वै॒वै वाव॑ रुन्धे रु॒न्धे ऽवै॒वै वाव॑ रुन्धे । \newline
55. अव॑ रुन्धे रु॒न्धे ऽवाव॑ रुन्धे॒ यं ॅयꣳ रु॒न्धे ऽवाव॑ रुन्धे॒ यम् । \newline
56. रु॒न्धे॒ यं ॅयꣳ रु॑न्धे रुन्धे॒ यम् का॒मये॑त का॒मये॑त॒ यꣳ रु॑न्धे रुन्धे॒ यम् का॒मये॑त । \newline
57. यम् का॒मये॑त का॒मये॑त॒ यं ॅयम् का॒मये॑ता प॒शु र॑प॒शुः का॒मये॑त॒ यं ॅयम् का॒मये॑ता प॒शुः । \newline
58. का॒मये॑ता प॒शु र॑प॒शुः का॒मये॑त का॒मये॑ता प॒शुः स्या᳚थ् स्यादप॒शुः का॒मये॑त का॒मये॑ता प॒शुः स्या᳚त् । \newline
59. अ॒प॒शुः स्या᳚थ् स्या दप॒शु र॑प॒शुः स्या॒ दितीति॑ स्या दप॒शु र॑प॒शुः स्या॒दिति॑ । \newline
60. स्या॒ दितीति॑ स्याथ् स्या॒दिति॑ विषू॒चीना॑नि विषू॒चीना॒नीति॑ स्याथ् स्या॒दिति॑ विषू॒चीना॑नि । \newline
61. इति॑ विषू॒चीना॑नि विषू॒चीना॒ नीतीति॑ विषू॒चीना॑नि॒ तस्य॒ तस्य॑ विषू॒चीना॒ नीतीति॑ विषू॒चीना॑नि॒ तस्य॑ । \newline
\pagebreak
\markright{ TS 5.2.9.4  \hfill https://www.vedavms.in \hfill}

\section{ TS 5.2.9.4 }

\textbf{TS 5.2.9.4 } \newline
\textbf{Samhita Paata} \newline

विषू॒चीना॑नि॒ तस्योप॑ दद्ध्या॒द्-विषू॑च ए॒वास्मा᳚त् प॒शून् द॑धात्यप॒शुरे॒व भ॑वति॒ यं का॒मये॑त पशु॒मान्थ्-स्या॒दिति॑ समी॒चीना॑नि॒ तस्योप॑ दद्ध्याथ् स॒मीच॑ ए॒वास्मै॑ प॒शून् द॑धाति पशु॒माने॒व भ॑वति पु॒रस्ता᳚त् प्रती॒चीन॒मश्व॒स्योप॑ दधाति प॒श्चात् प्रा॒चीन॑मृष॒भस्या-प॑शवो॒ वा अ॒न्ये गो॑ अ॒श्वेभ्यः॑ प॒शवो॑ गो अ॒श्वाने॒वास्मै॑ स॒मीचो॑ दधात्ये॒-ताव॑न्तो॒ वै प॒शवो᳚ - [  ] \newline

\textbf{Pada Paata} \newline

वि॒षू॒चीना॑नि । तस्य॑ । उपेति॑ । द॒द्ध्या॒त् । विषू॑चः । ए॒व । अ॒स्मा॒त् । प॒शून् । द॒धा॒ति॒ । अ॒प॒शुः । ए॒व । भ॒व॒ति॒ । यम् । का॒मये॑त । प॒शु॒मानिति॑ पशु - मान् । स्या॒त् । इति॑ । स॒मी॒चीना॑नि । तस्य॑ । उपेति॑ । द॒द्ध्या॒त् । स॒मीचः॑ । ए॒व । अ॒स्मै॒ । प॒शून् । द॒धा॒ति॒ । प॒शु॒मानिति॑ पशु - मान् । ए॒व । भ॒व॒ति॒ । पु॒रस्ता᳚त् । प्र॒ती॒चीन᳚म् । अश्व॑स्य । उपेति॑ । द॒धा॒ति॒ । प॒श्चात् । प्रा॒चीन᳚म् । ऋ॒ष॒भस्य॑ । अप॑शवः । वै । अ॒न्ये । गो॒ अ॒श्वेभ्य॒ इति॑ गो -अ॒श्वेभ्यः॑ । प॒शवः॑ । गो॒ अ॒श्वानिति॑ गो - अ॒श्वान् । ए॒व । अ॒स्मै॒ । स॒मीचः॑ । द॒धा॒ति॒ । ए॒ताव॑न्तः । वै । प॒शवः॑ ।  \newline


\textbf{Krama Paata} \newline

वि॒षू॒चीना॑नि॒ तस्य॑ । तस्योप॑ । उप॑ दद्ध्यात् । द॒द्ध्या॒द् विषू॑चः । विषू॑च ए॒व । ए॒वास्मा᳚त् । अ॒स्मा॒त् प॒शून् । प॒शून् द॑धाति । द॒धा॒त्य॒प॒शुः । अ॒प॒शुरे॒व । ए॒व भ॑वति । भ॒व॒ति॒ यम् । यम् का॒मये॑त । का॒मये॑त पशु॒मान् । प॒शु॒मान्थ् स्या᳚त् । प॒शु॒मानिति॑ पशु - मान् । स्या॒दिति॑ । इति॑ समी॒चीना॑नि । स॒मी॒चीना॑नि॒ तस्य॑ । तस्योप॑ । उप॑ दद्ध्यात् । द॒द्ध्या॒थ् स॒मीचः॑ । स॒मीच॑ ए॒व । ए॒वास्मै᳚ । अ॒स्मै॒ प॒शून् । प॒शून् द॑धाति । द॒धा॒ति॒ प॒शु॒मान् । प॒शु॒माने॒व । प॒शु॒मानिति॑ पशु - मान् । ए॒व भ॑वति । भ॒व॒ति॒ पु॒रस्ता᳚त् । पु॒रस्ता᳚त् प्रती॒चीन᳚म् । प्र॒ती॒चीन॒मश्व॑स्य । अश्व॒स्योप॑ । उप॑ दधाति । द॒धा॒ति॒ प॒श्चात् । प॒श्चात् प्रा॒चीन᳚म् । प्रा॒चीन॑मृष॒भस्य॑ । ऋ॒ष॒भस्याप॑शवः । अप॑शवो॒ वै । वा अ॒न्ये । अ॒न्ये गो॑ अ॒श्वेभ्यः॑ । गो॒ अ॒श्वेभ्यः॑ प॒शवः॑ । गो॒ अ॒श्वेभ्य॒ इति॑ गो - अ॒श्वेभ्यः॑ । प॒शवो॑ गो अ॒श्वान् । गो॒ अ॒श्वाने॒व । गो॒ अ॒श्वानिति॑ गो - अ॒श्वान् । ए॒वास्मै᳚ । अ॒स्मै॒ स॒मीचः॑ । स॒मीचो॑ दधाति । द॒धा॒त्ये॒ताव॑न्तः । ए॒ताव॑न्तो॒ वै । वै प॒शवः॑ । प॒शवो᳚ द्वि॒पादः॑ \newline

\textbf{Jatai Paata} \newline

1. वि॒षू॒चीना॑नि॒ तस्य॒ तस्य॑ विषू॒चीना॑नि विषू॒चीना॑नि॒ तस्य॑ । \newline
2. तस्यो पोप॒ तस्य॒ तस्योप॑ । \newline
3. उप॑ दद्ध्याद् दद्ध्या॒ दुपोप॑ दद्ध्यात् । \newline
4. द॒द्ध्या॒द् विषू॑चो॒ विषू॑चो दद्ध्याद् दद्ध्या॒द् विषू॑चः । \newline
5. विषू॑च ए॒वैव विषू॑चो॒ विषू॑च ए॒व । \newline
6. ए॒वास्मा॑ दस्मा दे॒वै वास्मा᳚त् । \newline
7. अ॒स्मा॒त् प॒शून् प॒शू न॑स्मा दस्मात् प॒शून् । \newline
8. प॒शून् द॑धाति दधाति प॒शून् प॒शून् द॑धाति । \newline
9. द॒धा॒ त्य॒प॒शु र॑प॒शुर् द॑धाति दधा त्यप॒शुः । \newline
10. अ॒प॒शु रे॒वैवा प॒शु र॑प॒शु रे॒व । \newline
11. ए॒व भ॑वति भव त्ये॒वैव भ॑वति । \newline
12. भ॒व॒ति॒ यं ॅयम् भ॑वति भवति॒ यम् । \newline
13. यम् का॒मये॑त का॒मये॑त॒ यं ॅयम् का॒मये॑त । \newline
14. का॒मये॑त पशु॒मान् प॑शु॒मान् का॒मये॑त का॒मये॑त पशु॒मान् । \newline
15. प॒शु॒मान् थ्स्या᳚थ् स्यात् पशु॒मान् प॑शु॒मान् थ्स्या᳚त् । \newline
16. प॒शु॒मानिति॑ पशु - मान् । \newline
17. स्या॒ दितीति॑ स्याथ् स्या॒दिति॑ । \newline
18. इति॑ समी॒चीना॑नि समी॒चीना॒ नीतीति॑ समी॒चीना॑नि । \newline
19. स॒मी॒चीना॑नि॒ तस्य॒ तस्य॑ समी॒चीना॑नि समी॒चीना॑नि॒ तस्य॑ । \newline
20. तस्यो पोप॒ तस्य॒ तस्योप॑ । \newline
21. उप॑ दद्ध्याद् दद्ध्या॒ दुपोप॑ दद्ध्यात् । \newline
22. द॒द्ध्या॒थ् स॒मीचः॑ स॒मीचो॑ दद्ध्याद् दद्ध्याथ् स॒मीचः॑ । \newline
23. स॒मीच॑ ए॒वैव स॒मीचः॑ स॒मीच॑ ए॒व । \newline
24. ए॒वास्मा॑ अस्मा ए॒वै वास्मै᳚ । \newline
25. अ॒स्मै॒ प॒शून् प॒शू न॑स्मा अस्मै प॒शून् । \newline
26. प॒शून् द॑धाति दधाति प॒शून् प॒शून् द॑धाति । \newline
27. द॒धा॒ति॒ प॒शु॒मान् प॑शु॒मान् द॑धाति दधाति पशु॒मान् । \newline
28. प॒शु॒मा ने॒वैव प॑शु॒मान् प॑शु॒मा ने॒व । \newline
29. प॒शु॒मानिति॑ पशु - मान् । \newline
30. ए॒व भ॑वति भव त्ये॒वैव भ॑वति । \newline
31. भ॒व॒ति॒ पु॒रस्ता᳚त् पु॒रस्ता᳚द् भवति भवति पु॒रस्ता᳚त् । \newline
32. पु॒रस्ता᳚त् प्रती॒चीन॑म् प्रती॒चीन॑म् पु॒रस्ता᳚त् पु॒रस्ता᳚त् प्रती॒चीन᳚म् । \newline
33. प्र॒ती॒चीन॒ मश्व॒स्या श्व॑स्य प्रती॒चीन॑म् प्रती॒चीन॒ मश्व॑स्य । \newline
34. अश्व॒स्यो पोपा श्व॒स्या श्व॒स्योप॑ । \newline
35. उप॑ दधाति दधा॒ त्युपोप॑ दधाति । \newline
36. द॒धा॒ति॒ प॒श्चात् प॒श्चाद् द॑धाति दधाति प॒श्चात् । \newline
37. प॒श्चात् प्रा॒चीन॑म् प्रा॒चीन॑म् प॒श्चात् प॒श्चात् प्रा॒चीन᳚म् । \newline
38. प्रा॒चीन॑ मृष॒भस्य॑ र्.ष॒भस्य॑ प्रा॒चीन॑म् प्रा॒चीन॑ मृष॒भस्य॑ । \newline
39. ऋ॒ष॒भस्या प॑श॒वो ऽप॑शव ऋष॒भस्य॑ र्.ष॒भस्या प॑शवः । \newline
40. अप॑शवो॒ वै वा अप॑श॒वो ऽप॑शवो॒ वै । \newline
41. वा अ॒न्ये᳚ ऽन्ये वै वा अ॒न्ये । \newline
42. अ॒न्ये गो॑अ॒श्वेभ्यो॑ गोअ॒श्वेभ्यो॒ ऽन्ये᳚ ऽन्ये गो॑अ॒श्वेभ्यः॑ । \newline
43. गो॒अ॒श्वेभ्यः॑ प॒शवः॑ प॒शवो॑ गोअ॒श्वेभ्यो॑ गोअ॒श्वेभ्यः॑ प॒शवः॑ । \newline
44. गो॒अ॒श्वेभ्य॒ इति॑ गो - अ॒श्वेभ्यः॑ । \newline
45. प॒शवो॑ गोअ॒श्वान् गो॑अ॒श्वान् प॒शवः॑ प॒शवो॑ गोअ॒श्वान् । \newline
46. गो॒अ॒श्वा ने॒वैव गो॑अ॒श्वान् गो॑अ॒श्वा ने॒व । \newline
47. गो॒अ॒श्वानिति॑ गो - अ॒श्वान् । \newline
48. ए॒वास्मा॑ अस्मा ए॒वै वास्मै᳚ । \newline
49. अ॒स्मै॒ स॒मीचः॑ स॒मीचो᳚ ऽस्मा अस्मै स॒मीचः॑ । \newline
50. स॒मीचो॑ दधाति दधाति स॒मीचः॑ स॒मीचो॑ दधाति । \newline
51. द॒धा॒ त्ये॒ताव॑न्त ए॒ताव॑न्तो दधाति दधा त्ये॒ताव॑न्तः । \newline
52. ए॒ताव॑न्तो॒ वै वा ए॒ताव॑न्त ए॒ताव॑न्तो॒ वै । \newline
53. वै प॒शवः॑ प॒शवो॒ वै वै प॒शवः॑ । \newline
54. प॒शवो᳚ द्वि॒पादो᳚ द्वि॒पादः॑ प॒शवः॑ प॒शवो᳚ द्वि॒पादः॑ । \newline

\textbf{Ghana Paata } \newline

1. वि॒षू॒चीना॑नि॒ तस्य॒ तस्य॑ विषू॒चीना॑नि विषू॒चीना॑नि॒ तस्योपोप॒ तस्य॑ विषू॒चीना॑नि विषू॒चीना॑नि॒ तस्योप॑ । \newline
2. तस्योपोप॒ तस्य॒ तस्योप॑ दद्ध्याद् दद्ध्या॒ दुप॒ तस्य॒ तस्योप॑ दद्ध्यात् । \newline
3. उप॑ दद्ध्याद् दद्ध्या॒ दुपोप॑ दद्ध्या॒द् विषू॑चो॒ विषू॑चो दद्ध्या॒ दुपोप॑ दद्ध्या॒द् विषू॑चः । \newline
4. द॒द्ध्या॒द् विषू॑चो॒ विषू॑चो दद्ध्याद् दद्ध्या॒द् विषू॑च ए॒वैव विषू॑चो दद्ध्याद् दद्ध्या॒द् विषू॑च ए॒व । \newline
5. विषू॑च ए॒वैव विषू॑चो॒ विषू॑च ए॒वास्मा॑ दस्मा दे॒व विषू॑चो॒ विषू॑च ए॒वास्मा᳚त् । \newline
6. ए॒वास्मा॑ दस्मा दे॒वैवास्मा᳚त् प॒शून् प॒शू न॑स्मा दे॒वैवास्मा᳚त् प॒शून् । \newline
7. अ॒स्मा॒त् प॒शून् प॒शू न॑स्मा दस्मात् प॒शून् द॑धाति दधाति प॒शू न॑स्मा दस्मात् प॒शून् द॑धाति । \newline
8. प॒शून् द॑धाति दधाति प॒शून् प॒शून् द॑धा त्यप॒शु र॑प॒शुर् द॑धाति प॒शून् प॒शून् द॑धा त्यप॒शुः । \newline
9. द॒धा॒ त्य॒प॒शु र॑प॒शुर् द॑धाति दधा त्यप॒शु रे॒वैवा प॒शुर् द॑धाति दधा त्यप॒शु रे॒व । \newline
10. अ॒प॒शु रे॒वैवा प॒शु र॑प॒शु रे॒व भ॑वति भव त्ये॒वा प॒शु र॑प॒शु रे॒व भ॑वति । \newline
11. ए॒व भ॑वति भव त्ये॒वैव भ॑वति॒ यं ॅयम् भ॑व त्ये॒वैव भ॑वति॒ यम् । \newline
12. भ॒व॒ति॒ यं ॅयम् भ॑वति भवति॒ यम् का॒मये॑त का॒मये॑त॒ यम् भ॑वति भवति॒ यम् का॒मये॑त । \newline
13. यम् का॒मये॑त का॒मये॑त॒ यं ॅयम् का॒मये॑त पशु॒मान् प॑शु॒मान् का॒मये॑त॒ यं ॅयम् का॒मये॑त पशु॒मान् । \newline
14. का॒मये॑त पशु॒मान् प॑शु॒मान् का॒मये॑त का॒मये॑त पशु॒मान् थ्स्या᳚थ् स्यात् पशु॒मान् का॒मये॑त का॒मये॑त पशु॒मान् थ्स्या᳚त् । \newline
15. प॒शु॒मान् थ्स्या᳚थ् स्यात् पशु॒मान् प॑शु॒मान् थ्स्या॒ दितीति॑ स्यात् पशु॒मान् प॑शु॒मान् थ्स्या॒दिति॑ । \newline
16. प॒शु॒मानिति॑ पशु - मान् । \newline
17. स्या॒दितीति॑ स्याथ् स्या॒दिति॑ समी॒चीना॑नि समी॒चीना॒नीति॑ स्याथ् स्या॒दिति॑ समी॒चीना॑नि । \newline
18. इति॑ समी॒चीना॑नि समी॒चीना॒नीतीति॑ समी॒चीना॑नि॒ तस्य॒ तस्य॑ समी॒चीना॒नीतीति॑ समी॒चीना॑नि॒ तस्य॑ । \newline
19. स॒मी॒चीना॑नि॒ तस्य॒ तस्य॑ समी॒चीना॑नि समी॒चीना॑नि॒ तस्योपोप॒ तस्य॑ समी॒चीना॑नि समी॒चीना॑नि॒ तस्योप॑ । \newline
20. तस्योपोप॒ तस्य॒ तस्योप॑ दद्ध्याद् दद्ध्या॒ दुप॒ तस्य॒ तस्योप॑ दद्ध्यात् । \newline
21. उप॑ दद्ध्याद् दद्ध्या॒ दुपोप॑ दद्ध्याथ् स॒मीचः॑ स॒मीचो॑ दद्ध्या॒ दुपोप॑ दद्ध्याथ् स॒मीचः॑ । \newline
22. द॒द्ध्या॒थ् स॒मीचः॑ स॒मीचो॑ दद्ध्याद् दद्ध्याथ् स॒मीच॑ ए॒वैव स॒मीचो॑ दद्ध्याद् दद्ध्याथ् स॒मीच॑ ए॒व । \newline
23. स॒मीच॑ ए॒वैव स॒मीचः॑ स॒मीच॑ ए॒वास्मा॑ अस्मा ए॒व स॒मीचः॑ स॒मीच॑ ए॒वास्मै᳚ । \newline
24. ए॒वास्मा॑ अस्मा ए॒वैवास्मै॑ प॒शून् प॒शू न॑स्मा ए॒वैवास्मै॑ प॒शून् । \newline
25. अ॒स्मै॒ प॒शून् प॒शू न॑स्मा अस्मै प॒शून् द॑धाति दधाति प॒शू न॑स्मा अस्मै प॒शून् द॑धाति । \newline
26. प॒शून् द॑धाति दधाति प॒शून् प॒शून् द॑धाति पशु॒मान् प॑शु॒मान् द॑धाति प॒शून् प॒शून् द॑धाति पशु॒मान् । \newline
27. द॒धा॒ति॒ प॒शु॒मान् प॑शु॒मान् द॑धाति दधाति पशु॒मा ने॒वैव प॑शु॒मान् द॑धाति दधाति पशु॒मा ने॒व । \newline
28. प॒शु॒मा ने॒वैव प॑शु॒मान् प॑शु॒मा ने॒व भ॑वति भव त्ये॒व प॑शु॒मान् प॑शु॒मा ने॒व भ॑वति । \newline
29. प॒शु॒मानिति॑ पशु - मान् । \newline
30. ए॒व भ॑वति भव त्ये॒वैव भ॑वति पु॒रस्ता᳚त् पु॒रस्ता᳚द् भव त्ये॒वैव भ॑वति पु॒रस्ता᳚त् । \newline
31. भ॒व॒ति॒ पु॒रस्ता᳚त् पु॒रस्ता᳚द् भवति भवति पु॒रस्ता᳚त् प्रती॒चीन॑म् प्रती॒चीन॑म् पु॒रस्ता᳚द् भवति भवति पु॒रस्ता᳚त् प्रती॒चीन᳚म् । \newline
32. पु॒रस्ता᳚त् प्रती॒चीन॑म् प्रती॒चीन॑म् पु॒रस्ता᳚त् पु॒रस्ता᳚त् प्रती॒चीन॒ मश्व॒स्या श्व॑स्य प्रती॒चीन॑म् पु॒रस्ता᳚त् पु॒रस्ता᳚त् प्रती॒चीन॒ मश्व॑स्य । \newline
33. प्र॒ती॒चीन॒ मश्व॒स्या श्व॑स्य प्रती॒चीन॑म् प्रती॒चीन॒ मश्व॒ स्योपोपा श्व॑स्य प्रती॒चीन॑म् प्रती॒चीन॒ मश्व॒स्योप॑ । \newline
34. अश्व॒ स्योपोपा श्व॒स्या श्व॒स्योप॑ दधाति दधा॒ त्युपा श्व॒स्या श्व॒स्योप॑ दधाति । \newline
35. उप॑ दधाति दधा॒ त्युपोप॑ दधाति प॒श्चात् प॒श्चाद् द॑धा॒ त्युपोप॑ दधाति प॒श्चात् । \newline
36. द॒धा॒ति॒ प॒श्चात् प॒श्चाद् द॑धाति दधाति प॒श्चात् प्रा॒चीन॑म् प्रा॒चीन॑म् प॒श्चाद् द॑धाति दधाति प॒श्चात् प्रा॒चीन᳚म् । \newline
37. प॒श्चात् प्रा॒चीन॑म् प्रा॒चीन॑म् प॒श्चात् प॒श्चात् प्रा॒चीन॑ मृष॒भस्य॑ र्.ष॒भस्य॑ प्रा॒चीन॑म् प॒श्चात् प॒श्चात् प्रा॒चीन॑ मृष॒भस्य॑ । \newline
38. प्रा॒चीन॑ मृष॒भस्य॑ र्.ष॒भस्य॑ प्रा॒चीन॑म् प्रा॒चीन॑ मृष॒भस्या प॑श॒वो ऽप॑शव ऋष॒भस्य॑ प्रा॒चीन॑म् प्रा॒चीन॑ मृष॒भस्या प॑शवः । \newline
39. ऋ॒ष॒भस्या प॑श॒वो ऽप॑शव ऋष॒भस्य॑ र्.ष॒भस्या प॑शवो॒ वै वा अप॑शव ऋष॒भस्य॑ र्.ष॒भस्या प॑शवो॒ वै । \newline
40. अप॑शवो॒ वै वा अप॑श॒वो ऽप॑शवो॒ वा अ॒न्ये᳚ ऽन्ये वा अप॑श॒वो ऽप॑शवो॒ वा अ॒न्ये । \newline
41. वा अ॒न्ये᳚ ऽन्ये वै वा अ॒न्ये गो॑अ॒श्वेभ्यो॑ गोअ॒श्वेभ्यो॒ ऽन्ये वै वा अ॒न्ये गो॑अ॒श्वेभ्यः॑ । \newline
42. अ॒न्ये गो॑अ॒श्वेभ्यो॑ गोअ॒श्वेभ्यो॒ ऽन्ये᳚ ऽन्ये गो॑अ॒श्वेभ्यः॑ प॒शवः॑ प॒शवो॑ गोअ॒श्वेभ्यो॒ ऽन्ये᳚ ऽन्ये 
गो॑अ॒श्वेभ्यः॑ प॒शवः॑ । \newline
43. गो॒अ॒श्वेभ्यः॑ प॒शवः॑ प॒शवो॑ गोअ॒श्वेभ्यो॑ गोअ॒श्वेभ्यः॑ प॒शवो॑ गोअ॒श्वान् गो॑अ॒श्वान् प॒शवो॑ 
गोअ॒श्वेभ्यो॑ गोअ॒श्वेभ्यः॑ प॒शवो॑ गोअ॒श्वान् । \newline
44. गो॒अ॒श्वेभ्य॒ इति॑ गो - अ॒श्वेभ्यः॑ । \newline
45. प॒शवो॑ गोअ॒श्वान् गो॑अ॒श्वान् प॒शवः॑ प॒शवो॑ गोअ॒श्वा ने॒वैव गो॑अ॒श्वान् प॒शवः॑ प॒शवो॑ गोअ॒श्वा ने॒व । \newline
46. गो॒अ॒श्वा ने॒वैव गो॑अ॒श्वान् गो॑अ॒श्वा ने॒वास्मा॑ अस्मा ए॒व गो॑अ॒श्वान् गो॑अ॒श्वा ने॒वास्मै᳚ । \newline
47. गो॒अ॒श्वानिति॑ गो - अ॒श्वान् । \newline
48. ए॒वास्मा॑ अस्मा ए॒वैवास्मै॑ स॒मीचः॑ स॒मीचो᳚ ऽस्मा ए॒वैवास्मै॑ स॒मीचः॑ । \newline
49. अ॒स्मै॒ स॒मीचः॑ स॒मीचो᳚ ऽस्मा अस्मै स॒मीचो॑ दधाति दधाति स॒मीचो᳚ ऽस्मा अस्मै स॒मीचो॑ दधाति । \newline
50. स॒मीचो॑ दधाति दधाति स॒मीचः॑ स॒मीचो॑ दधा त्ये॒ताव॑न्त ए॒ताव॑न्तो दधाति स॒मीचः॑ स॒मीचो॑ दधा त्ये॒ताव॑न्तः । \newline
51. द॒धा॒ त्ये॒ताव॑न्त ए॒ताव॑न्तो दधाति दधा त्ये॒ताव॑न्तो॒ वै वा ए॒ताव॑न्तो दधाति दधा त्ये॒ताव॑न्तो॒ वै । \newline
52. ए॒ताव॑न्तो॒ वै वा ए॒ताव॑न्त ए॒ताव॑न्तो॒ वै प॒शवः॑ प॒शवो॒ वा ए॒ताव॑न्त ए॒ताव॑न्तो॒ वै प॒शवः॑ । \newline
53. वै प॒शवः॑ प॒शवो॒ वै वै प॒शवो᳚ द्वि॒पादो᳚ द्वि॒पादः॑ प॒शवो॒ वै वै प॒शवो᳚ द्वि॒पादः॑ । \newline
54. प॒शवो᳚ द्वि॒पादो᳚ द्वि॒पादः॑ प॒शवः॑ प॒शवो᳚ द्वि॒पाद॑श्च च द्वि॒पादः॑ प॒शवः॑ प॒शवो᳚ द्वि॒पाद॑श्च । \newline
\pagebreak
\markright{ TS 5.2.9.5  \hfill https://www.vedavms.in \hfill}

\section{ TS 5.2.9.5 }

\textbf{TS 5.2.9.5 } \newline
\textbf{Samhita Paata} \newline

द्वि॒पाद॑श्च॒ चतु॑ष्पादश्च॒ तान्. वा ए॒तद॒ग्नौ प्रद॑धाति॒ यत् प॑शुशी॒र्॒.षाण्यु॑प॒-दधा᳚त्य॒-मुमा॑र॒ण्यमनु॑ ते दिशा॒मीत्या॑ह ग्रा॒म्येभ्य॑ ए॒व प॒शुभ्य॑ आर॒ण्यान् प॒शूञ्छुच॒मनूथ्सृ॑जति॒ तस्मा᳚थ् स॒माव॑त् पशू॒नां प्र॒जाय॑मानाना-मार॒ण्याः प॒शवः॒ कनी॑याꣳसः शु॒चा ह्यृ॑ताः स॑र्पशी॒र्॒.षमुप॑ दधाति॒ यैव स॒र्पे त्विषि॒स्तामे॒वाव॑ रुन्धे॒ - [  ] \newline

\textbf{Pada Paata} \newline

द्वि॒पाद॒ इति॑ द्वि - पादः॑ । च॒ । चतु॑ष्पाद॒ इति॒ चतुः॑ - पा॒दः॒ । च॒ । तान् । वै । ए॒तत् । अ॒ग्नौ । प्रेति॑ । द॒धा॒ति॒ । यत् । प॒शु॒शी॒र्॒.षाणीति॑ पशु - शी॒र्॒.षाणि॑ । उ॒प॒दधा॒तीत्यु॑प - दधा॑ति । अ॒मुम् । आ॒र॒ण्यम् । अन्विति॑ । ते॒ । दि॒शा॒मि॒ । इति॑ । आ॒ह॒ । ग्रा॒म्येभ्यः॑ । ए॒व । प॒शुभ्य॒ इति॑ प॒शु - भ्यः॒ । आ॒र॒ण्यान् । प॒शून् । शुच᳚म् । अनू॑ । उदिति॑ । सृ॒ज॒ति॒ । तस्मा᳚त् । स॒माव॑त् । प॒शू॒नाम् । प्र॒जाय॑मानाना॒मिति॑ प्र - जाय॑मानानाम् । आ॒र॒ण्याः । प॒शवः॑ । कनी॑याꣳसः । शु॒चा । हि । ऋ॒ताः । स॒र्प॒शी॒र्॒.षमिति॑ सर्प - शी॒र्॒.षम् । उपेति॑ । द॒धा॒ति॒ । या । ए॒व । स॒र्पे । त्विषिः॑ । ताम् । ए॒व । अवेति॑ । रु॒न्धे॒ ।  \newline


\textbf{Krama Paata} \newline

द्वि॒पाद॑श्च । द्वि॒पाद॒ इति॑ द्वि - पादः॑ । च॒ चतु॑ष्पादः । चतु॑ष्पादश्च । चतु॑ष्पाद॒ इति॒ चतुः॑ - पा॒दः॒ । च॒ तान् । तान्. वै । वा ए॒तत् । ए॒तद॒ग्नौ । अ॒ग्नौ प्र । प्र द॑धाति । द॒धा॒ति॒ यत् । यत् प॑शुशी॒र्.॒षाणि॑ । प॒शु॒शी॒र्.॒षाण्यु॑प॒दधा॑ति । प॒शु॒शी॒र्॒.षाणीति॑ पशु - शी॒र्॒.षाणि॑ । उ॒प॒दधा᳚त्य॒मुम् । उ॒प॒दधा॒तीत्यु॑प - दधा॑ति । अ॒मुमा॑र॒ण्यम् । आ॒र॒ण्यमनु॑ । अनु॑ ते । ते॒ दि॒शा॒मि॒ । दि॒शा॒मीति॑ । इत्या॑ह । आ॒ह॒ ग्रा॒म्येभ्यः॑ । ग्रा॒म्येभ्य॑ ए॒व । ए॒व प॒शुभ्यः॑ । प॒शुभ्य॑ आर॒ण्यान् । प॒शुभ्य॒ इति॑ प॒शु - भ्यः॒ । आ॒र॒ण्यान् प॒शून् । प॒शूञ्छुच᳚म् । शुच॒मनु॑ । अनूत् । उथ् सृ॑जति । सृ॒ज॒ति॒ तस्मा᳚त् । तस्मा᳚थ् स॒माव॑त् । स॒माव॑त् पशू॒नाम् । प॒शू॒नाम् प्र॒जाय॑मानानाम् । प्र॒जाय॑मानानामार॒ण्याः । प्र॒जाय॑मानाना॒मिति॑ प्र - जाय॑मानानाम् । आ॒र॒ण्याः प॒शवः॑ । प॒शवः॒ कनी॑याꣳसः । कनी॑याꣳसः शु॒चा । शु॒चा हि । ह्यृ॑ताः । ऋ॒ताः स॑र्पशी॒र्॒.षम् । स॒र्प॒शी॒र्॒.षमुप॑ । स॒र्प॒शी॒र्॒.षमिति॑ सर्प - शी॒र्॒.षम् । उप॑ दधाति । द॒धा॒ति॒ या । यैव । ए॒व स॒र्पे । स॒र्पे त्विषिः॑ । त्विषि॒स्ताम् । तामे॒व । ए॒वाव॑ । अव॑ रुन्धे ( ) । रु॒न्धे॒ यत् \newline

\textbf{Jatai Paata} \newline

1. द्वि॒पाद॑श्च च द्वि॒पादो᳚ द्वि॒पाद॑श्च । \newline
2. द्वि॒पाद॒ इति॑ द्वि - पादः॑ । \newline
3. च॒ चतु॑ष्पाद॒ श्चतु॑ष्पादश्च च॒ चतु॑ष्पादः । \newline
4. चतु॑ष्पाद श्च च॒ चतु॑ष्पाद॒ श्चतु॑ष्पाद श्च । \newline
5. चतु॑ष्पाद॒ इति॒ चतुः॑ - पा॒दः॒ । \newline
6. च॒ ताꣳ स्ताꣳ श्च॑ च॒ तान् । \newline
7. तान्. वै वै ताꣳ स्तान्. वै । \newline
8. वा ए॒त दे॒तद् वै वा ए॒तत् । \newline
9. ए॒त द॒ग्ना व॒ग्ना वे॒त दे॒त द॒ग्नौ । \newline
10. अ॒ग्नौ प्र प्राग्ना व॒ग्नौ प्र । \newline
11. प्र द॑धाति दधाति॒ प्र प्र द॑धाति । \newline
12. द॒धा॒ति॒ यद् यद् द॑धाति दधाति॒ यत् । \newline
13. यत् प॑शुशी॒र्॒.षाणि॑ पशुशी॒र्॒.षाणि॒ यद् यत् प॑शुशी॒र्॒.षाणि॑ । \newline
14. प॒शु॒शी॒र्॒.षाण्यु॑ प॒दधा᳚ त्युप॒दधा॑ति पशुशी॒र्॒.षाणि॑ पशुशी॒र्॒.षाण्यु॑ प॒दधा॑ति । \newline
15. प॒शु॒शी॒र्॒.षाणीति॑ पशु - शी॒र्॒.षाणि॑ । \newline
16. उ॒प॒दधा᳚ त्य॒मु म॒मु मु॑प॒दधा᳚ त्युप॒दधा᳚ त्य॒मुम् । \newline
17. उ॒प॒दधा॒तीत्यु॑प - दधा॑ति । \newline
18. अ॒मु मा॑र॒ण्य मा॑र॒ण्य म॒मु म॒मु मा॑र॒ण्यम् । \newline
19. आ॒र॒ण्य मन्वन् वा॑र॒ण्य मा॑र॒ण्य मनु॑ । \newline
20. अनु॑ ते ते॒ अन्वनु॑ ते । \newline
21. ते॒ दि॒शा॒मि॒ दि॒शा॒मि॒ ते॒ ते॒ दि॒शा॒मि॒ । \newline
22. दि॒शा॒ मीतीति॑ दिशामि दिशा॒मीति॑ । \newline
23. इत्या॑ हा॒हेती त्या॑ह । \newline
24. आ॒ह॒ ग्रा॒म्येभ्यो᳚ ग्रा॒म्येभ्य॑ आहाह ग्रा॒म्येभ्यः॑ । \newline
25. ग्रा॒म्येभ्य॑ ए॒वैव ग्रा॒म्येभ्यो᳚ ग्रा॒म्येभ्य॑ ए॒व । \newline
26. ए॒व प॒शुभ्यः॑ प॒शुभ्य॑ ए॒वैव प॒शुभ्यः॑ । \newline
27. प॒शुभ्य॑ आर॒ण्या ना॑र॒ण्यान् प॒शुभ्यः॑ प॒शुभ्य॑ आर॒ण्यान् । \newline
28. प॒शुभ्य॒ इति॑ प॒शु - भ्यः॒ । \newline
29. आ॒र॒ण्यान् प॒शून् प॒शू ना॑र॒ण्या ना॑र॒ण्यान् प॒शून् । \newline
30. प॒शूञ् छुचꣳ॒॒ शुच॑म् प॒शून् प॒शूञ् छुच᳚म् । \newline
31. शुच॒ मन्वनु॒ शुचꣳ॒॒ शुच॒ मनु॑ । \newline
32. अनू दु दन्व नूत् । \newline
33. उथ् सृ॑जति सृज॒ त्युदुथ् सृ॑जति । \newline
34. सृ॒ज॒ति॒ तस्मा॒त् तस्मा᳚थ् सृजति सृजति॒ तस्मा᳚त् । \newline
35. तस्मा᳚थ् स॒माव॑थ् स॒माव॒त् तस्मा॒त् तस्मा᳚थ् स॒माव॑त् । \newline
36. स॒माव॑त् पशू॒नाम् प॑शू॒नाꣳ स॒माव॑थ् स॒माव॑त् पशू॒नाम् । \newline
37. प॒शू॒नाम् प्र॒जाय॑मानानाम् प्र॒जाय॑मानानाम् पशू॒नाम् प॑शू॒नाम् प्र॒जाय॑मानानाम् । \newline
38. प्र॒जाय॑मानाना मार॒ण्या आ॑र॒ण्याः प्र॒जाय॑मानानाम् प्र॒जाय॑मानाना मार॒ण्याः । \newline
39. प्र॒जाय॑मानाना॒मिति॑ प्र - जाय॑मानानाम् । \newline
40. आ॒र॒ण्याः प॒शवः॑ प॒शव॑ आर॒ण्या आ॑र॒ण्याः प॒शवः॑ । \newline
41. प॒शवः॒ कनी॑याꣳसः॒ कनी॑याꣳसः प॒शवः॑ प॒शवः॒ कनी॑याꣳसः । \newline
42. कनी॑याꣳसः शु॒चा शु॒चा कनी॑याꣳसः॒ कनी॑याꣳसः शु॒चा । \newline
43. शु॒चा हि हि शु॒चा शु॒चा हि । \newline
44. ह्यृ॑ता ऋ॒ता हि ह्यृ॑ताः । \newline
45. ऋ॒ताः स॑र्पशी॒र्॒.षꣳ स॑र्पशी॒र्॒.ष मृ॒ता ऋ॒ताः स॑र्पशी॒र्॒.षम् । \newline
46. स॒र्प॒शी॒र्॒.ष मुपोप॑ सर्पशी॒र्॒.षꣳ स॑र्पशी॒र्॒.ष मुप॑ । \newline
47. स॒र्प॒शी॒र्॒.षमिति॑ सर्प - शी॒र्॒.षम् । \newline
48. उप॑ दधाति दधा॒ त्युपोप॑ दधाति । \newline
49. द॒धा॒ति॒ या या द॑धाति दधाति॒ या । \newline
50. यै वैव या यैव । \newline
51. ए॒व स॒र्पे स॒र्प ए॒वैव स॒र्पे । \newline
52. स॒र्पे त्विषि॒ स्त्विषिः॑ स॒र्पे स॒र्पे त्विषिः॑ । \newline
53. त्विषि॒ स्ताम् ताम् त्विषि॒ स्त्विषि॒ स्ताम् । \newline
54. ता मे॒वैव ताम् ता मे॒व । \newline
55. ए॒वावा वै॒वै वाव॑ । \newline
56. अव॑ रुन्धे रु॒न्धे ऽवाव॑ रुन्धे । \newline
57. रु॒न्धे॒ यद् यद् रु॑न्धे रुन्धे॒ यत् । \newline

\textbf{Ghana Paata } \newline

1. द्वि॒पाद॑श्च च द्वि॒पादो᳚ द्वि॒पाद॑श्च॒ चतु॑ष्पाद॒ श्चतु॑ष्पादश्च द्वि॒पादो᳚ द्वि॒पाद॑श्च॒ चतु॑ष्पादः । \newline
2. द्वि॒पाद॒ इति॑ द्वि - पादः॑ । \newline
3. च॒ चतु॑ष्पाद॒ श्चतु॑ष्पादश्च च॒ चतु॑ष्पादश्च च॒ चतु॑ष्पादश्च च॒ चतु॑ष्पादश्च । \newline
4. चतु॑ष्पादश्च च॒ चतु॑ष्पाद॒ श्चतु॑ष्पादश्च॒ ताꣳ स्ताꣳ श्च॒ चतु॑ष्पाद॒ श्चतु॑ष्पादश्च॒ तान् । \newline
5. चतु॑ष्पाद॒ इति॒ चतुः॑ - पा॒दः॒ । \newline
6. च॒ ताꣳ स्ताꣳ श्च॑ च॒ तान्. वै वै ताꣳ श्च॑ च॒ तान्. वै । \newline
7. तान्. वै वै ताꣳ स्तान्. वा ए॒त दे॒तद् वै ताꣳ स्तान्. वा ए॒तत् । \newline
8. वा ए॒त दे॒तद् वै वा ए॒त द॒ग्ना व॒ग्ना वे॒तद् वै वा ए॒त द॒ग्नौ । \newline
9. ए॒त द॒ग्ना व॒ग्ना वे॒त दे॒त द॒ग्नौ प्र प्राग्ना वे॒त दे॒त द॒ग्नौ प्र । \newline
10. अ॒ग्नौ प्र प्राग्ना व॒ग्नौ प्र द॑धाति दधाति॒ प्राग्ना व॒ग्नौ प्र द॑धाति । \newline
11. प्र द॑धाति दधाति॒ प्र प्र द॑धाति॒ यद् यद् द॑धाति॒ प्र प्र द॑धाति॒ यत् । \newline
12. द॒धा॒ति॒ यद् यद् द॑धाति दधाति॒ यत् प॑शुशी॒र्॒.षाणि॑ पशुशी॒र्॒.षाणि॒ यद् द॑धाति दधाति॒ यत् प॑शुशी॒र्॒.षाणि॑ । \newline
13. यत् प॑शुशी॒र्॒.षाणि॑ पशुशी॒र्॒.षाणि॒ यद् यत् प॑शुशी॒र्॒.षा ण्यु॑प॒दधा᳚ त्युप॒दधा॑ति पशुशी॒र्॒.षाणि॒ यद् यत् प॑शुशी॒र्॒.षा ण्यु॑प॒दधा॑ति । \newline
14. प॒शु॒शी॒र्॒.षा ण्यु॑प॒दधा᳚ त्युप॒दधा॑ति पशुशी॒र्॒.षाणि॑ पशुशी॒र्॒.षा ण्यु॑प॒दधा᳚ त्य॒मु म॒मु मु॑प॒दधा॑ति पशुशी॒र्॒.षाणि॑ पशुशी॒र्॒.षा ण्यु॑प॒दधा᳚ त्य॒मुम् । \newline
15. प॒शु॒शी॒र्॒.षाणीति॑ पशु - शी॒र्॒.षाणि॑ । \newline
16. उ॒प॒दधा᳚ त्य॒मु म॒मु मु॑प॒दधा᳚ त्युप॒दधा᳚ त्य॒मु मा॑र॒ण्य मा॑र॒ण्य म॒मु मु॑प॒दधा ᳚त्युप॒दधा᳚ त्य॒मु मा॑र॒ण्यम् । \newline
17. उ॒प॒दधा॒तीत्यु॑प - दधा॑ति । \newline
18. अ॒मु मा॑र॒ण्य मा॑र॒ण्य म॒मु म॒मु मा॑र॒ण्य मन् वन् वा॑र॒ण्य म॒मु म॒मु मा॑र॒ण्य मनु॑ । \newline
19. आ॒र॒ण्य मन् वन् वा॑र॒ण्य मा॑र॒ण्य मनु॑ ते ते॒ अन्वा॑र॒ण्य मा॑र॒ण्य मनु॑ ते । \newline
20. अनु॑ ते ते॒ अन्वनु॑ ते दिशामि दिशामि ते॒ अन्वनु॑ ते दिशामि । \newline
21. ते॒ दि॒शा॒मि॒ दि॒शा॒मि॒ ते॒ ते॒ दि॒शा॒मीतीति॑ दिशामि ते ते दिशा॒मीति॑ । \newline
22. दि॒शा॒मीतीति॑ दिशामि दिशा॒मी त्या॑हा॒हेति॑ दिशामि दिशा॒मीत्या॑ह । \newline
23. इत्या॑हा॒हे तीत्या॑ह ग्रा॒म्येभ्यो᳚ ग्रा॒म्येभ्य॑ आ॒हे तीत्या॑ह ग्रा॒म्येभ्यः॑ । \newline
24. आ॒ह॒ ग्रा॒म्येभ्यो᳚ ग्रा॒म्येभ्य॑ आहाह ग्रा॒म्येभ्य॑ ए॒वैव ग्रा॒म्येभ्य॑ आहाह ग्रा॒म्येभ्य॑ ए॒व । \newline
25. ग्रा॒म्येभ्य॑ ए॒वैव ग्रा॒म्येभ्यो᳚ ग्रा॒म्येभ्य॑ ए॒व प॒शुभ्यः॑ प॒शुभ्य॑ ए॒व ग्रा॒म्येभ्यो᳚ ग्रा॒म्येभ्य॑ ए॒व प॒शुभ्यः॑ । \newline
26. ए॒व प॒शुभ्यः॑ प॒शुभ्य॑ ए॒वैव प॒शुभ्य॑ आर॒ण्या ना॑र॒ण्यान् प॒शुभ्य॑ ए॒वैव प॒शुभ्य॑ आर॒ण्यान् । \newline
27. प॒शुभ्य॑ आर॒ण्या ना॑र॒ण्यान् प॒शुभ्यः॑ प॒शुभ्य॑ आर॒ण्यान् प॒शून् प॒शू ना॑र॒ण्यान् प॒शुभ्यः॑ प॒शुभ्य॑ आर॒ण्यान् प॒शून् । \newline
28. प॒शुभ्य॒ इति॑ प॒शु - भ्यः॒ । \newline
29. आ॒र॒ण्यान् प॒शून् प॒शू ना॑र॒ण्या ना॑र॒ण्यान् प॒शूञ् छुचꣳ॒॒ शुच॑म् प॒शू ना॑र॒ण्या ना॑र॒ण्यान् प॒शूञ् छुच᳚म् । \newline
30. प॒शूञ् छुचꣳ॒॒ शुच॑म् प॒शून् प॒शूञ् छुच॒ मन्वनु॒ शुच॑म् प॒शून् प॒शूञ् छुच॒ मनु॑ । \newline
31. शुच॒ मन्वनु॒ शुचꣳ॒॒ शुच॒ मनू दुदनु॒ शुचꣳ॒॒ शुच॒ मनूत् । \newline
32. अनू दुदन् वनूथ् सृ॑जति सृज॒ त्युदन् वनूथ् सृ॑जति । \newline
33. उथ् सृ॑जति सृज॒ त्युदुथ् सृ॑जति॒ तस्मा॒त् तस्मा᳚थ् सृज॒ त्युदुथ् सृ॑जति॒ तस्मा᳚त् । \newline
34. सृ॒ज॒ति॒ तस्मा॒त् तस्मा᳚थ् सृजति सृजति॒ तस्मा᳚थ् स॒माव॑थ् स॒माव॒त् तस्मा᳚थ् सृजति सृजति॒ तस्मा᳚थ् स॒माव॑त् । \newline
35. तस्मा᳚थ् स॒माव॑थ् स॒माव॒त् तस्मा॒त् तस्मा᳚थ् स॒माव॑त् पशू॒नाम् प॑शू॒नाꣳ स॒माव॒त् तस्मा॒त् तस्मा᳚थ् स॒माव॑त् पशू॒नाम् । \newline
36. स॒माव॑त् पशू॒नाम् प॑शू॒नाꣳ स॒माव॑थ् स॒माव॑त् पशू॒नाम् प्र॒जाय॑मानानाम् प्र॒जाय॑मानानाम् पशू॒नाꣳ स॒माव॑थ् स॒माव॑त् पशू॒नाम् प्र॒जाय॑मानानाम् । \newline
37. प॒शू॒नाम् प्र॒जाय॑मानानाम् प्र॒जाय॑मानानाम् पशू॒नाम् प॑शू॒नाम् प्र॒जाय॑मानाना मार॒ण्या आ॑र॒ण्याः प्र॒जाय॑मानानाम् पशू॒नाम् प॑शू॒नाम् प्र॒जाय॑मानाना मार॒ण्याः । \newline
38. प्र॒जाय॑मानाना मार॒ण्या आ॑र॒ण्याः प्र॒जाय॑मानानाम् प्र॒जाय॑मानाना मार॒ण्याः प॒शवः॑ प॒शव॑ आर॒ण्याः प्र॒जाय॑मानानाम् प्र॒जाय॑मानाना मार॒ण्याः प॒शवः॑ । \newline
39. प्र॒जाय॑मानाना॒मिति॑ प्र - जाय॑मानानाम् । \newline
40. आ॒र॒ण्याः प॒शवः॑ प॒शव॑ आर॒ण्या आ॑र॒ण्याः प॒शवः॒ कनी॑याꣳसः॒ कनी॑याꣳसः प॒शव॑ आर॒ण्या आ॑र॒ण्याः प॒शवः॒ कनी॑याꣳसः । \newline
41. प॒शवः॒ कनी॑याꣳसः॒ कनी॑याꣳसः प॒शवः॑ प॒शवः॒ कनी॑याꣳसः शु॒चा शु॒चा कनी॑याꣳसः प॒शवः॑ प॒शवः॒ कनी॑याꣳसः शु॒चा । \newline
42. कनी॑याꣳसः शु॒चा शु॒चा कनी॑याꣳसः॒ कनी॑याꣳसः शु॒चा हि हि शु॒चा कनी॑याꣳसः॒ कनी॑याꣳसः शु॒चा हि । \newline
43. शु॒चा हि हि शु॒चा शु॒चा ह्यृ॑ता ऋ॒ता हि शु॒चा शु॒चा ह्यृ॑ताः । \newline
44. ह्यृ॑ता ऋ॒ता हि ह्यृ॑ताः स॑र्पशी॒र्॒.षꣳ स॑र्पशी॒र्॒.ष मृ॒ता हि ह्यृ॑ताः स॑र्पशी॒र्॒.षम् । \newline
45. ऋ॒ताः स॑र्पशी॒र्॒.षꣳ स॑र्पशी॒र्॒.ष मृ॒ता ऋ॒ताः स॑र्पशी॒र्॒.ष मुपोप॑ सर्पशी॒र्॒.ष मृ॒ता ऋ॒ताः स॑र्पशी॒र्॒.ष मुप॑ । \newline
46. स॒र्प॒शी॒र्॒.ष मुपोप॑ सर्पशी॒र्॒.षꣳ स॑र्पशी॒र्॒.ष मुप॑ दधाति दधा॒ त्युप॑ सर्पशी॒र्॒.षꣳ स॑र्पशी॒र्॒.ष मुप॑ दधाति । \newline
47. स॒र्प॒शी॒र्॒.षमिति॑ सर्प - शी॒र्॒.षम् । \newline
48. उप॑ दधाति दधा॒ त्युपोप॑ दधाति॒ या या द॑धा॒ त्युपोप॑ दधाति॒ या । \newline
49. द॒धा॒ति॒ या या द॑धाति दधाति॒ यैवैव या द॑धाति दधाति॒ यैव । \newline
50. यैवैव या यैव स॒र्पे स॒र्प ए॒व या यैव स॒र्पे । \newline
51. ए॒व स॒र्पे स॒र्प ए॒वैव स॒र्पे त्विषि॒ स्त्विषिः॑ स॒र्प ए॒वैव स॒र्पे त्विषिः॑ । \newline
52. स॒र्पे त्विषि॒ स्त्विषिः॑ स॒र्पे स॒र्पे त्विषि॒ स्ताम् ताम् त्विषिः॑ स॒र्पे स॒र्पे त्विषि॒ स्ताम् । \newline
53. त्विषि॒ स्ताम् ताम् त्विषि॒ स्त्विषि॒ स्ता मे॒वैव ताम् त्विषि॒ स्त्विषि॒ स्ता मे॒व । \newline
54. ता मे॒वैव ताम् ता मे॒वावा वै॒व ताम् ता मे॒वाव॑ । \newline
55. ए॒वावा वै॒वै वाव॑ रुन्धे रु॒न्धे ऽवै॒वै वाव॑ रुन्धे । \newline
56. अव॑ रुन्धे रु॒न्धे ऽवाव॑ रुन्धे॒ यद् यद् रु॒न्धे ऽवाव॑ रुन्धे॒ यत् । \newline
57. रु॒न्धे॒ यद् यद् रु॑न्धे रुन्धे॒ यथ् स॑मी॒चीनꣳ॑ समी॒चीनं॒ ॅयद् रु॑न्धे रुन्धे॒ यथ् स॑मी॒चीन᳚म् । \newline
\pagebreak
\markright{ TS 5.2.9.6  \hfill https://www.vedavms.in \hfill}

\section{ TS 5.2.9.6 }

\textbf{TS 5.2.9.6 } \newline
\textbf{Samhita Paata} \newline

यथ् स॑मी॒चीनं॑ पशुशी॒र्॒.षैरु॑प द॒द्ध्याद् ग्रा॒म्यान् प॒शून् दꣳशु॑काः स्यु॒र्यद्-वि॑षू॒चीन॑-मार॒ण्यान्. यजु॑रे॒व व॑दे॒दव॒ तां त्विषिꣳ॑ रुन्धे॒ या स॒र्पे न ग्रा॒म्यान् प॒शून्. हि॒नस्ति॒ नाऽऽ*र॒ण्यानथो॒ खलू॑प॒धेय॑मे॒व यदु॑प॒दधा॑ति॒ तेन॒ तां त्विषि॒मव॑ रुन्धे॒ या स॒र्पे यद्-यजु॒र्वद॑ति॒ तेन॑ शा॒न्तं ॥ \newline

\textbf{Pada Paata} \newline

यत् । स॒मी॒चीन᳚म् । प॒शु॒शी॒र्॒.षैरिति॑ पशु - शी॒र्॒.षैः । उ॒प॒द॒द्ध्यादित्यु॑प - द॒द्ध्यात् । ग्रा॒म्यान् । प॒शून् । दꣳशु॑काः । स्युः॒ । यत् । वि॒षू॒चीन᳚म् । आ॒र॒ण्यान् । यजुः॑ । ए॒व । व॒दे॒त् । अवेति॑ । ताम् । त्विषि᳚म् । रु॒न्धे॒ । या । स॒र्पे । न । ग्रा॒म्यान् । प॒शून् । हि॒नस्ति॑ । न । आ॒र॒ण्यान् । अथो॒ इति॑ । खलु॑ । उ॒प॒धेय॒मित्यु॑प - धेय᳚म् । ए॒व । यत् । उ॒प॒दधा॒तीत्यु॑प - दधा॑ति । तेन॑ । ताम् । त्विषि᳚म् । अवेति॑ । रु॒न्धे॒ । या । स॒र्पे । यत् । यजुः॑ । वद॑ति । तेन॑ । शा॒न्तम् ॥  \newline


\textbf{Krama Paata} \newline

यथ् स॑मी॒चीन᳚म् । स॒मी॒चीन॑म् पशुशी॒र्॒.षैः । प॒शु॒शी॒र्॒.षैरु॑पद॒द्ध्यात् । प॒शु॒शी॒र्॒.षैरिति॑ पशु - शी॒र्॒.षैः । उ॒प॒द॒द्ध्याद् ग्रा॒म्यान् । उ॒प॒द॒द्ध्यादित्यु॑प - द॒द्ध्यात् । ग्रा॒म्यान् प॒शून् । प॒शून् दꣳशु॑काः । दꣳशु॑काः स्युः । स्यु॒र् यत् । यद् वि॑षू॒चीन᳚म् । वि॒षू॒चीन॑मार॒ण्यान् । आ॒र॒ण्यान् यजुः॑ । यजु॑रे॒व । ए॒व व॑देत् । व॒दे॒दव॑ । अव॒ ताम् । ताम् त्विषि᳚म् । त्विषिꣳ॑ रुन्धे । रु॒न्धे॒ या । या स॒र्पे । स॒र्पे न । न ग्रा॒म्यान् । ग्रा॒म्यान् प॒शून् । प॒शून् हि॒नस्ति॑ । हि॒नस्ति॒ न । नार॒ण्यान् । आ॒र॒ण्यानथो᳚ । अथो॒ खलु॑ । अथो॒ इत्यथो᳚ । खलू॑प॒धेय᳚म् । उ॒प॒धेय॑मे॒व । उ॒प॒धेय॒मित्यु॑प - धेय᳚म् । ए॒व यत् । यदु॑प॒दधा॑ति । उ॒प॒दधा॑ति॒ तेन॑ । उ॒प॒दधा॒तीत्यु॑प - दधा॑ति । तेन॒ ताम् । ताम् त्विषि᳚म् । त्विषि॒मव॑ । अव॑ रुन्धे । रु॒न्धे॒ या । या स॒र्पे । स॒र्पे यत् । यद् यजुः॑ । यजु॒र् वद॑ति । वद॑ति॒ तेन॑ । तेन॑ शा॒न्तम् । शा॒न्तमिति॑ शा॒न्तम् । \newline

\textbf{Jatai Paata} \newline

1. यथ् स॑मी॒चीनꣳ॑ समी॒चीनं॒ ॅयद् यथ् स॑मी॒चीन᳚म् । \newline
2. स॒मी॒चीन॑म् पशुशी॒र्॒.षैः प॑शुशी॒र्॒.षैः स॑मी॒चीनꣳ॑ समी॒चीन॑म् पशुशी॒र्॒.षैः । \newline
3. प॒शु॒शी॒र्॒.षै रु॑पद॒द्ध्या दु॑पद॒द्ध्यात् प॑शुशी॒र्॒.षैः प॑शुशी॒र्॒.षै रु॑पद॒द्ध्यात् । \newline
4. प॒शु॒शी॒र्॒.षैरिति॑ पशु - शी॒र्॒.षैः । \newline
5. उ॒प॒द॒द्ध्याद् ग्रा॒म्यान् ग्रा॒म्या नु॑पद॒द्ध्या दु॑पद॒द्ध्याद् ग्रा॒म्यान् । \newline
6. उ॒प॒द॒द्ध्यादित्यु॑प - द॒द्ध्यात् । \newline
7. ग्रा॒म्यान् प॒शून् प॒शून् ग्रा॒म्यान् ग्रा॒म्यान् प॒शून् । \newline
8. प॒शून् दꣳशु॑का॒ दꣳशु॑काः प॒शून् प॒शून् दꣳशु॑काः । \newline
9. दꣳशु॑काः स्युः स्यु॒र् दꣳशु॑का॒ दꣳशु॑काः स्युः । \newline
10. स्यु॒र् यद् यथ् स्युः॑ स्यु॒र् यत् । \newline
11. यद् वि॑षू॒चीनं॑ ॅविषू॒चीनं॒ ॅयद् यद् वि॑षू॒चीन᳚म् । \newline
12. वि॒षू॒चीन॑ मार॒ण्या ना॑र॒ण्यान्. वि॑षू॒चीनं॑ ॅविषू॒चीन॑ मार॒ण्यान् । \newline
13. आ॒र॒ण्यान्. यजु॒र् यजु॑रार॒ण्या ना॑र॒ण्यान्. यजुः॑ । \newline
14. यजु॑ रे॒वैव यजु॒र् यजु॑ रे॒व । \newline
15. ए॒व व॑देद् वदे दे॒वैव व॑देत् । \newline
16. व॒दे॒ दवाव॑ वदेद् वदे॒ दव॑ । \newline
17. अव॒ ताम् ता मवाव॒ ताम् । \newline
18. ताम् त्विषि॒म् त्विषि॒म् ताम् ताम् त्विषि᳚म् । \newline
19. त्विषिꣳ॑ रुन्धे रुन्धे॒ त्विषि॒म् त्विषिꣳ॑ रुन्धे । \newline
20. रु॒न्धे॒ या या रु॑न्धे रुन्धे॒ या । \newline
21. या स॒र्पे स॒र्पे या या स॒र्पे । \newline
22. स॒र्पे न न स॒र्पे स॒र्पे न । \newline
23. न ग्रा॒म्यान् ग्रा॒म्यान् न न ग्रा॒म्यान् । \newline
24. ग्रा॒म्यान् प॒शून् प॒शून् ग्रा॒म्यान् ग्रा॒म्यान् प॒शून् । \newline
25. प॒शून्. हि॒नस्ति॑ हि॒नस्ति॑ प॒शून् प॒शून्. हि॒नस्ति॑ । \newline
26. हि॒नस्ति॒ न न हि॒नस्ति॑ हि॒नस्ति॒ न । \newline
27. नार॒ण्या ना॑र॒ण्यान् न नार॒ण्यान् । \newline
28. आ॒र॒ण्या नथो॒ अथो॑ आर॒ण्या ना॑र॒ण्या नथो᳚ । \newline
29. अथो॒ खलु॒ खल्वथो॒ अथो॒ खलु॑ । \newline
30. अथो॒ इत्यथो᳚ । \newline
31. खलू॑प॒धेय॑ मुप॒धेय॒म् खलु॒ खलू॑ प॒धेय᳚म् । \newline
32. उ॒प॒धेय॑ मे॒वैवो प॒धेय॑ मुप॒धेय॑ मे॒व । \newline
33. उ॒प॒धेय॒मित्यु॑प - धेय᳚म् । \newline
34. ए॒व यद् यदे॒ वैव यत् । \newline
35. यदु॑प॒दधा᳚ त्युप॒दधा॑ति॒ यद् यदु॑प॒दधा॑ति । \newline
36. उ॒प॒दधा॑ति॒ तेन॒ तेनो॑ प॒दधा᳚ त्युप॒दधा॑ति॒ तेन॑ । \newline
37. उ॒प॒दधा॒तीत्यु॑प - दधा॑ति । \newline
38. तेन॒ ताम् ताम् तेन॒ तेन॒ ताम् । \newline
39. ताम् त्विषि॒म् त्विषि॒म् ताम् ताम् त्विषि᳚म् । \newline
40. त्विषि॒ मवाव॒ त्विषि॒म् त्विषि॒ मव॑ । \newline
41. अव॑ रुन्धे रु॒न्धे ऽवाव॑ रुन्धे । \newline
42. रु॒न्धे॒ या या रु॑न्धे रुन्धे॒ या । \newline
43. या स॒र्पे स॒र्पे या या स॒र्पे । \newline
44. स॒र्पे यद् यथ् स॒र्पे स॒र्पे यत् । \newline
45. यद् यजु॒र् यजु॒र् यद् यद् यजुः॑ । \newline
46. यजु॒र् वद॑ति॒ वद॑ति॒ यजु॒र् यजु॒र् वद॑ति । \newline
47. वद॑ति॒ तेन॒ तेन॒ वद॑ति॒ वद॑ति॒ तेन॑ । \newline
48. तेन॑ शा॒न्तꣳ शा॒न्तम् तेन॒ तेन॑ शा॒न्तम् । \newline
49. शा॒न्तमिति॑ शा॒न्तम् । \newline

\textbf{Ghana Paata } \newline

1. यथ् स॑मी॒चीनꣳ॑ समी॒चीनं॒ ॅयद् यथ् स॑मी॒चीन॑म् पशुशी॒र्॒.षैः प॑शुशी॒र्॒.षैः स॑मी॒चीनं॒ ॅयद् यथ् स॑मी॒चीन॑म् पशुशी॒र्॒.षैः । \newline
2. स॒मी॒चीन॑म् पशुशी॒र्॒.षैः प॑शुशी॒र्॒.षैः स॑मी॒चीनꣳ॑ समी॒चीन॑म् पशुशी॒र्॒.षै रु॑पद॒द्ध्या दु॑पद॒द्ध्यात् प॑शुशी॒र्॒.षैः स॑मी॒चीनꣳ॑ समी॒चीन॑म् पशुशी॒र्॒.षै रु॑पद॒द्ध्यात् । \newline
3. प॒शु॒शी॒र्॒.षै रु॑पद॒द्ध्या दु॑पद॒द्ध्यात् प॑शुशी॒र्॒.षैः प॑शुशी॒र्॒.षै रु॑पद॒द्ध्याद् ग्रा॒म्यान् ग्रा॒म्या नु॑पद॒द्ध्यात् प॑शुशी॒र्॒.षैः प॑शुशी॒र्॒.षै रु॑पद॒द्ध्याद् ग्रा॒म्यान् । \newline
4. प॒शु॒शी॒र्॒.षैरिति॑ पशु - शी॒र्॒.षैः । \newline
5. उ॒प॒द॒द्ध्याद् ग्रा॒म्यान् ग्रा॒म्या नु॑पद॒द्ध्या दु॑पद॒द्ध्याद् ग्रा॒म्यान् प॒शून् प॒शून् ग्रा॒म्या नु॑पद॒द्ध्या दु॑पद॒द्ध्याद् ग्रा॒म्यान् प॒शून् । \newline
6. उ॒प॒द॒द्ध्यादित्यु॑प - द॒द्ध्यात् । \newline
7. ग्रा॒म्यान् प॒शून् प॒शून् ग्रा॒म्यान् ग्रा॒म्यान् प॒शून् दꣳशु॑का॒ दꣳशु॑काः प॒शून् ग्रा॒म्यान् ग्रा॒म्यान् प॒शून् दꣳशु॑काः । \newline
8. प॒शून् दꣳशु॑का॒ दꣳशु॑काः प॒शून् प॒शून् दꣳशु॑काः स्युः स्यु॒र् दꣳशु॑काः प॒शून् प॒शून् दꣳशु॑काः स्युः । \newline
9. दꣳशु॑काः स्युः स्यु॒र् दꣳशु॑का॒ दꣳशु॑काः स्यु॒र् यद् यथ् स्यु॒र् दꣳशु॑का॒ दꣳशु॑काः स्यु॒र् यत् । \newline
10. स्यु॒र् यद् यथ् स्युः॑ स्यु॒र् यद् वि॑षू॒चीनं॑ ॅविषू॒चीनं॒ ॅयथ् स्युः॑ स्यु॒र् यद् वि॑षू॒चीन᳚म् । \newline
11. यद् वि॑षू॒चीनं॑ ॅविषू॒चीनं॒ ॅयद् यद् वि॑षू॒चीन॑ मार॒ण्या ना॑र॒ण्यान्. वि॑षू॒चीनं॒ ॅयद् यद् वि॑षू॒चीन॑ मार॒ण्यान् । \newline
12. वि॒षू॒चीन॑ मार॒ण्या ना॑र॒ण्यान्. वि॑षू॒चीनं॑ ॅविषू॒चीन॑ मार॒ण्यान्. यजु॒र् यजु॑ रार॒ण्यान्. वि॑षू॒चीनं॑ ॅविषू॒चीन॑ मार॒ण्यान्. यजुः॑ । \newline
13. आ॒र॒ण्यान्. यजु॒र् यजु॑ रार॒ण्या ना॑र॒ण्यान्. यजु॑ रे॒वैव यजु॑ रार॒ण्या ना॑र॒ण्यान्. यजु॑रे॒व । \newline
14. यजु॑ रे॒वैव यजु॒र् यजु॑ रे॒व व॑देद् वदेदे॒व यजु॒र् यजु॑ रे॒व व॑देत् । \newline
15. ए॒व व॑देद् वदे दे॒वैव व॑दे॒ दवाव॑ वदे दे॒वैव व॑दे॒ दव॑ । \newline
16. व॒दे॒ दवाव॑ वदेद् वदे॒ दव॒ ताम् ता मव॑ वदेद् वदे॒ दव॒ ताम् । \newline
17. अव॒ ताम् ता मवाव॒ ताम् त्विषि॒म् त्विषि॒म् ता मवाव॒ ताम् त्विषि᳚म् । \newline
18. ताम् त्विषि॒म् त्विषि॒म् ताम् ताम् त्विषिꣳ॑ रुन्धे रुन्धे॒ त्विषि॒म् ताम् ताम् त्विषिꣳ॑ रुन्धे । \newline
19. त्विषिꣳ॑ रुन्धे रुन्धे॒ त्विषि॒म् त्विषिꣳ॑ रुन्धे॒ या या रु॑न्धे॒ त्विषि॒म् त्विषिꣳ॑ रुन्धे॒ या । \newline
20. रु॒न्धे॒ या या रु॑न्धे रुन्धे॒ या स॒र्पे स॒र्पे या रु॑न्धे रुन्धे॒ या स॒र्पे । \newline
21. या स॒र्पे स॒र्पे या या स॒र्पे न न स॒र्पे या या स॒र्पे न । \newline
22. स॒र्पे न न स॒र्पे स॒र्पे न ग्रा॒म्यान् ग्रा॒म्यान् न स॒र्पे स॒र्पे न ग्रा॒म्यान् । \newline
23. न ग्रा॒म्यान् ग्रा॒म्यान् न न ग्रा॒म्यान् प॒शून् प॒शून् ग्रा॒म्यान् न न ग्रा॒म्यान् प॒शून् । \newline
24. ग्रा॒म्यान् प॒शून् प॒शून् ग्रा॒म्यान् ग्रा॒म्यान् प॒शून्. हि॒नस्ति॑ हि॒नस्ति॑ प॒शून् ग्रा॒म्यान् ग्रा॒म्यान् प॒शून्. हि॒नस्ति॑ । \newline
25. प॒शून्. हि॒नस्ति॑ हि॒नस्ति॑ प॒शून् प॒शून्. हि॒नस्ति॒ न न हि॒नस्ति॑ प॒शून् प॒शून्. हि॒नस्ति॒ न । \newline
26. हि॒नस्ति॒ न न हि॒नस्ति॑ हि॒नस्ति॒ नार॒ण्या ना॑र॒ण्यान् न हि॒नस्ति॑ हि॒नस्ति॒ नार॒ण्यान् । \newline
27. नार॒ण्या ना॑र॒ण्यान् न नार॒ण्या नथो॒ अथो॑ आर॒ण्यान् न नार॒ण्या नथो᳚ । \newline
28. आ॒र॒ण्या नथो॒ अथो॑ आर॒ण्या ना॑र॒ण्या नथो॒ खलु॒ खल्वथो॑ आर॒ण्या ना॑र॒ण्या नथो॒ खलु॑ । \newline
29. अथो॒ खलु॒ खल्वथो॒ अथो॒ खलू॑प॒धेय॑ मुप॒धेय॒म् खल्वथो॒ अथो॒ खलू॑प॒धेय᳚म् । \newline
30. अथो॒ इत्यथो᳚ । \newline
31. खलू॑प॒धेय॑ मुप॒धेय॒म् खलु॒ खलू॑प॒धेय॑ मे॒वै वोप॒धेय॒म् खलु॒ खलू॑प॒धेय॑ मे॒व । \newline
32. उ॒प॒धेय॑ मे॒वै वोप॒धेय॑ मुप॒धेय॑ मे॒व यद् यदे॒ वोप॒धेय॑ मुप॒धेय॑ मे॒व यत् । \newline
33. उ॒प॒धेय॒मित्यु॑प - धेय᳚म् । \newline
34. ए॒व यद् यदे॒वैव यदु॑प॒दधा᳚ त्युप॒दधा॑ति॒ यदे॒वैव यदु॑प॒दधा॑ति । \newline
35. यदु॑प॒दधा᳚ त्युप॒दधा॑ति॒ यद् यदु॑प॒दधा॑ति॒ तेन॒ तेनो॑प॒दधा॑ति॒ यद् यदु॑प॒दधा॑ति॒ तेन॑ । \newline
36. उ॒प॒दधा॑ति॒ तेन॒ तेनो॑प॒दधा᳚ त्युप॒दधा॑ति॒ तेन॒ ताम् ताम् तेनो॑प॒दधा᳚ त्युप॒दधा॑ति॒ तेन॒ ताम् । \newline
37. उ॒प॒दधा॒तीत्यु॑प - दधा॑ति । \newline
38. तेन॒ ताम् ताम् तेन॒ तेन॒ ताम् त्विषि॒म् त्विषि॒म् ताम् तेन॒ तेन॒ ताम् त्विषि᳚म् । \newline
39. ताम् त्विषि॒म् त्विषि॒म् ताम् ताम् त्विषि॒ मवाव॒ त्विषि॒म् ताम् ताम् त्विषि॒ मव॑ । \newline
40. त्विषि॒ मवाव॒ त्विषि॒म् त्विषि॒ मव॑ रुन्धे रु॒न्धे ऽव॒ त्विषि॒म् त्विषि॒ मव॑ रुन्धे । \newline
41. अव॑ रुन्धे रु॒न्धे ऽवाव॑ रुन्धे॒ या या रु॒न्धे ऽवाव॑ रुन्धे॒ या । \newline
42. रु॒न्धे॒ या या रु॑न्धे रुन्धे॒ या स॒र्पे स॒र्पे या रु॑न्धे रुन्धे॒ या स॒र्पे । \newline
43. या स॒र्पे स॒र्पे या या स॒र्पे यद् यथ् स॒र्पे या या स॒र्पे यत् । \newline
44. स॒र्पे यद् यथ् स॒र्पे स॒र्पे यद् यजु॒र् यजु॒र् यथ् स॒र्पे स॒र्पे यद् यजुः॑ । \newline
45. यद् यजु॒र् यजु॒र् यद् यद् यजु॒र् वद॑ति॒ वद॑ति॒ यजु॒र् यद् यद् यजु॒र् वद॑ति । \newline
46. यजु॒र् वद॑ति॒ वद॑ति॒ यजु॒र् यजु॒र् वद॑ति॒ तेन॒ तेन॒ वद॑ति॒ यजु॒र् यजु॒र् वद॑ति॒ तेन॑ । \newline
47. वद॑ति॒ तेन॒ तेन॒ वद॑ति॒ वद॑ति॒ तेन॑ शा॒न्तꣳ शा॒न्तम् तेन॒ वद॑ति॒ वद॑ति॒ तेन॑ शा॒न्तम् । \newline
48. तेन॑ शा॒न्तꣳ शा॒न्तम् तेन॒ तेन॑ शा॒न्तम् । \newline
49. शा॒न्तमिति॑ शा॒न्तम् । \newline
\pagebreak
\markright{ TS 5.2.10.1  \hfill https://www.vedavms.in \hfill}

\section{ TS 5.2.10.1 }

\textbf{TS 5.2.10.1 } \newline
\textbf{Samhita Paata} \newline

प॒शुर्वा ए॒ष यद॒ग्निर्योनिः॒ खलु॒ वा ए॒षा प॒शोर्वि क्रि॑यते॒ यत् प्रा॒चीन॑मैष्ट॒काद्-यजुः॑ क्रि॒यते॒ रेतो॑ऽप॒स्या॑ अप॒स्या॑ उप॑ दधाति॒ योना॑वे॒व रेतो॑ दधाति॒ पञ्चोप॑ दधाति॒ पाङ्क्ताः᳚ प॒शवः॑ प॒शूने॒वास्मै॒ प्रज॑नयति॒ पञ्च॑ दक्षिण॒तो वज्रो॒ वा अ॑प॒स्या॑ वज्रे॑णै॒व य॒ज्ञ्स्य॑ दक्षिण॒तो रक्षाꣳ॒॒स्यप॑ हन्ति॒ पञ्च॑ प॒श्चात् - [  ] \newline

\textbf{Pada Paata} \newline

प॒शुः । वै । ए॒षः । यत् । अ॒ग्निः । योनिः॑ । खलु॑ । वै । ए॒षा । प॒शोः । वीति॑ । क्रि॒य॒ते॒ । यत् । प्रा॒चीन᳚म् । ऐ॒ष्ट॒कात् । यजुः॑ । क्रि॒यते᳚ । रेतः॑ । अ॒प॒स्याः᳚ । अ॒प॒स्याः᳚ । उपेति॑ । द॒धा॒ति॒ । योनौ᳚ । ए॒व । रेतः॑ । द॒धा॒ति॒ । पञ्च॑ । उपेति॑ । द॒धा॒ति॒ । पाङ्क्ताः᳚ । प॒शवः॑ । प॒शून् । ए॒व । अ॒स्मै॒ । प्रेति॑ । ज॒न॒य॒ति॒ । पञ्च॑ । द॒क्षि॒ण॒तः । वज्रः॑ । वै । अ॒प॒स्याः᳚ । वज्रे॑ण । ए॒व । य॒ज्ञ्स्य॑ । द॒क्षि॒ण॒तः । रक्षाꣳ॑सि । अपेति॑ । ह॒न्ति॒ । पञ्च॑ । प॒श्चात् ।  \newline


\textbf{Krama Paata} \newline

प॒शुर् वै । वा ए॒षः । ए॒ष यत् । यद॒ग्निः । अ॒ग्निर् योनिः॑ । योनिः॒ खलु॑ । खलु॒ वै । वा ए॒षा । ए॒षा प॒शोः । प॒शोर् वि । वि क्रि॑यते । क्रि॒य॒ते॒ यत् । यत् प्रा॒चीन᳚म् । प्रा॒चीन॑मैष्ट॒कात् । ऐ॒ष्ट॒काद् यजुः॑ । यजुः॑ क्रि॒यते᳚ । क्रि॒यते॒ रेतः॑ । रेतो॑ऽप॒स्याः᳚ । अ॒प॒स्या॑ अप॒स्याः᳚ । अ॒प॒स्या॑ उप॑ । उप॑ दधाति । द॒धा॒ति॒ योनौ᳚ । योना॑वे॒व । ए॒व रेतः॑ । रेतो॑ दधाति । द॒धा॒ति॒ पञ्च॑ । पञ्चोप॑ । उप॑ दधाति । द॒धा॒ति॒ पाङ्क्ताः᳚ । पाङ्क्ताः᳚ प॒शवः॑ । प॒शवः॑ प॒शून् । प॒शूने॒व । ए॒वास्मै᳚ । अ॒स्मै॒ प्र । प्र ज॑नयति । ज॒न॒य॒ति॒ पञ्च॑ । पञ्च॑ दक्षिण॒तः । द॒क्षि॒ण॒तो वज्रः॑ । वज्रो॒ वै । वा अ॑प॒स्याः᳚ । अ॒प॒स्या॑ वज्रे॑ण । वज्रे॑णै॒व । ए॒व य॒ज्ञ्स्य॑ । य॒ज्ञ्स्य॑ दक्षिण॒तः । द॒क्षि॒ण॒तो रक्षाꣳ॑सि । रक्षाꣳ॒॒स्यप॑ । अप॑ हन्ति । ह॒न्ति॒ पञ्च॑ । पञ्च॑ प॒श्चात् । प॒श्चात् प्राचीः᳚ \newline

\textbf{Jatai Paata} \newline

1. प॒शुर् वै वै प॒शुः प॒शुर् वै । \newline
2. वा ए॒ष ए॒ष वै वा ए॒षः । \newline
3. ए॒ष यद् यदे॒ष ए॒ष यत् । \newline
4. यद॒ग्नि र॒ग्निर् यद् यद॒ग्निः । \newline
5. अ॒ग्निर् योनि॒र् योनि॑ र॒ग्नि र॒ग्निर् योनिः॑ । \newline
6. योनिः॒ खलु॒ खलु॒ योनि॒र् योनिः॒ खलु॑ । \newline
7. खलु॒ वै वै खलु॒ खलु॒ वै । \newline
8. वा ए॒षैषा वै वा ए॒षा । \newline
9. ए॒षा प॒शोः प॒शो रे॒षैषा प॒शोः । \newline
10. प॒शोर् वि वि प॒शोः प॒शोर् वि । \newline
11. वि क्रि॑यते क्रियते॒ वि वि क्रि॑यते । \newline
12. क्रि॒य॒ते॒ यद् यत् क्रि॑यते क्रियते॒ यत् । \newline
13. यत् प्रा॒चीन॑म् प्रा॒चीनं॒ ॅयद् यत् प्रा॒चीन᳚म् । \newline
14. प्रा॒चीन॑ मैष्ट॒का दै᳚ष्ट॒कात् प्रा॒चीन॑म् प्रा॒चीन॑ मैष्ट॒कात् । \newline
15. ऐ॒ष्ट॒काद् यजु॒र् यजु॑ रैष्ट॒का दै᳚ष्ट॒काद् यजुः॑ । \newline
16. यजुः॑ क्रि॒यते᳚ क्रि॒यते॒ यजु॒र् यजुः॑ क्रि॒यते᳚ । \newline
17. क्रि॒यते॒ रेतो॒ रेतः॑ क्रि॒यते᳚ क्रि॒यते॒ रेतः॑ । \newline
18. रेतो॑ ऽप॒स्या॑ अप॒स्या॑ रेतो॒ रेतो॑ ऽप॒स्याः᳚ । \newline
19. अ॒प॒स्या॑ अप॒स्याः᳚ । \newline
20. अ॒प॒स्या॑ उपोपा॑ प॒स्या॑ अप॒स्या॑ उप॑ । \newline
21. उप॑ दधाति दधा॒ त्युपोप॑ दधाति । \newline
22. द॒धा॒ति॒ योनौ॒ योनौ॑ दधाति दधाति॒ योनौ᳚ । \newline
23. योना॑ वे॒वैव योनौ॒ योना॑ वे॒व । \newline
24. ए॒व रेतो॒ रेत॑ ए॒वैव रेतः॑ । \newline
25. रेतो॑ दधाति दधाति॒ रेतो॒ रेतो॑ दधाति । \newline
26. द॒धा॒ति॒ पञ्च॒ पञ्च॑ दधाति दधाति॒ पञ्च॑ । \newline
27. पञ्चोपोप॒ पञ्च॒ पञ्चोप॑ । \newline
28. उप॑ दधाति दधा॒ त्युपोप॑ दधाति । \newline
29. द॒धा॒ति॒ पाङ्क्ताः॒ पाङ्क्ता॑ दधाति दधाति॒ पाङ्क्ताः᳚ । \newline
30. पाङ्क्ताः᳚ प॒शवः॑ प॒शवः॒ पाङ्क्ताः॒ पाङ्क्ताः᳚ प॒शवः॑ । \newline
31. प॒शवः॑ प॒शून् प॒शून् प॒शवः॑ प॒शवः॑ प॒शून् । \newline
32. प॒शू ने॒वैव प॒शून् प॒शू ने॒व । \newline
33. ए॒वास्मा॑ अस्मा ए॒वै वास्मै᳚ । \newline
34. अ॒स्मै॒ प्र प्रास्मा॑ अस्मै॒ प्र । \newline
35. प्र ज॑नयति जनयति॒ प्र प्र ज॑नयति । \newline
36. ज॒न॒य॒ति॒ पञ्च॒ पञ्च॑ जनयति जनयति॒ पञ्च॑ । \newline
37. पञ्च॑ दक्षिण॒तो द॑क्षिण॒तः पञ्च॒ पञ्च॑ दक्षिण॒तः । \newline
38. द॒क्षि॒ण॒तो वज्रो॒ वज्रो॑ दक्षिण॒तो द॑क्षिण॒तो वज्रः॑ । \newline
39. वज्रो॒ वै वै वज्रो॒ वज्रो॒ वै । \newline
40. वा अ॑प॒स्या॑ अप॒स्या॑ वै वा अ॑प॒स्याः᳚ । \newline
41. अ॒प॒स्या॑ वज्रे॑ण॒ वज्रे॑णा प॒स्या॑ अप॒स्या॑ वज्रे॑ण । \newline
42. वज्रे॑ णै॒वैव वज्रे॑ण॒ वज्रे॑णै॒व । \newline
43. ए॒व य॒ज्ञ्स्य॑ य॒ज्ञ् स्यै॒वैव य॒ज्ञ्स्य॑ । \newline
44. य॒ज्ञ्स्य॑ दक्षिण॒तो द॑क्षिण॒तो य॒ज्ञ्स्य॑ य॒ज्ञ्स्य॑ दक्षिण॒तः । \newline
45. द॒क्षि॒ण॒तो रक्षाꣳ॑सि॒ रक्षाꣳ॑सि दक्षिण॒तो द॑क्षिण॒तो रक्षाꣳ॑सि । \newline
46. रक्षाꣳ॒॒ स्यपाप॒ रक्षाꣳ॑सि॒ रक्षाꣳ॒॒ स्यप॑ । \newline
47. अप॑ हन्ति ह॒न्त्य पाप॑ हन्ति । \newline
48. ह॒न्ति॒ पञ्च॒ पञ्च॑ हन्ति हन्ति॒ पञ्च॑ । \newline
49. पञ्च॑ प॒श्चात् प॒श्चात् पञ्च॒ पञ्च॑ प॒श्चात् । \newline
50. प॒श्चात् प्राचीः॒ प्राचीः᳚ प॒श्चात् प॒श्चात् प्राचीः᳚ । \newline

\textbf{Ghana Paata } \newline

1. प॒शुर् वै वै प॒शुः प॒शुर् वा ए॒ष ए॒ष वै प॒शुः प॒शुर् वा ए॒षः । \newline
2. वा ए॒ष ए॒ष वै वा ए॒ष यद् यदे॒ष वै वा ए॒ष यत् । \newline
3. ए॒ष यद् यदे॒ष ए॒ष यद॒ग्नि र॒ग्निर् यदे॒ष ए॒ष यद॒ग्निः । \newline
4. यद॒ग्नि र॒ग्निर् यद् यद॒ग्निर् योनि॒र् योनि॑ र॒ग्निर् यद् यद॒ग्निर् योनिः॑ । \newline
5. अ॒ग्निर् योनि॒र् योनि॑ र॒ग्नि र॒ग्निर् योनिः॒ खलु॒ खलु॒ योनि॑ र॒ग्नि र॒ग्निर् योनिः॒ खलु॑ । \newline
6. योनिः॒ खलु॒ खलु॒ योनि॒र् योनिः॒ खलु॒ वै वै खलु॒ योनि॒र् योनिः॒ खलु॒ वै । \newline
7. खलु॒ वै वै खलु॒ खलु॒ वा ए॒षैषा वै खलु॒ खलु॒ वा ए॒षा । \newline
8. वा ए॒षैषा वै वा ए॒षा प॒शोः प॒शो रे॒षा वै वा ए॒षा प॒शोः । \newline
9. ए॒षा प॒शोः प॒शो रे॒षैषा प॒शोर् वि वि प॒शो रे॒षैषा प॒शोर् वि । \newline
10. प॒शोर् वि वि प॒शोः प॒शोर् वि क्रि॑यते क्रियते॒ वि प॒शोः प॒शोर् वि क्रि॑यते । \newline
11. वि क्रि॑यते क्रियते॒ वि वि क्रि॑यते॒ यद् यत् क्रि॑यते॒ वि वि क्रि॑यते॒ यत् । \newline
12. क्रि॒य॒ते॒ यद् यत् क्रि॑यते क्रियते॒ यत् प्रा॒चीन॑म् प्रा॒चीनं॒ ॅयत् क्रि॑यते क्रियते॒ यत् प्रा॒चीन᳚म् । \newline
13. यत् प्रा॒चीन॑म् प्रा॒चीनं॒ ॅयद् यत् प्रा॒चीन॑ मैष्ट॒का दै᳚ष्ट॒कात् प्रा॒चीनं॒ ॅयद् यत् प्रा॒चीन॑ मैष्ट॒कात् । \newline
14. प्रा॒चीन॑ मैष्ट॒का दै᳚ष्ट॒कात् प्रा॒चीन॑म् प्रा॒चीन॑ मैष्ट॒काद् यजु॒र् यजु॑ रैष्ट॒कात् प्रा॒चीन॑म् प्रा॒चीन॑ मैष्ट॒काद् यजुः॑ । \newline
15. ऐ॒ष्ट॒काद् यजु॒र् यजु॑ रैष्ट॒का दै᳚ष्ट॒काद् यजुः॑ क्रि॒यते᳚ क्रि॒यते॒ यजु॑ रैष्ट॒का दै᳚ष्ट॒काद् यजुः॑ क्रि॒यते᳚ । \newline
16. यजुः॑ क्रि॒यते᳚ क्रि॒यते॒ यजु॒र् यजुः॑ क्रि॒यते॒ रेतो॒ रेतः॑ क्रि॒यते॒ यजु॒र् यजुः॑ क्रि॒यते॒ रेतः॑ । \newline
17. क्रि॒यते॒ रेतो॒ रेतः॑ क्रि॒यते᳚ क्रि॒यते॒ रेतो॑ ऽप॒स्या॑ अप॒स्या॑ रेतः॑ क्रि॒यते᳚ क्रि॒यते॒ रेतो॑ ऽप॒स्याः᳚ । \newline
18. रेतो॑ ऽप॒स्या॑ अप॒स्या॑ रेतो॒ रेतो॑ ऽप॒स्याः᳚ । \newline
19. अ॒प॒स्या॑ अप॒स्याः᳚ । \newline
20. अ॒प॒स्या॑ उपोपा॑प॒स्या॑ अप॒स्या॑ उप॑ दधाति दधा॒ त्युपा॑प॒स्या॑ अप॒स्या॑ उप॑ दधाति । \newline
21. उप॑ दधाति दधा॒ त्युपोप॑ दधाति॒ योनौ॒ योनौ॑ दधा॒ त्युपोप॑ दधाति॒ योनौ᳚ । \newline
22. द॒धा॒ति॒ योनौ॒ योनौ॑ दधाति दधाति॒ योना॑ वे॒वैव योनौ॑ दधाति दधाति॒ योना॑ वे॒व । \newline
23. योना॑ वे॒वैव योनौ॒ योना॑ वे॒व रेतो॒ रेत॑ ए॒व योनौ॒ योना॑ वे॒व रेतः॑ । \newline
24. ए॒व रेतो॒ रेत॑ ए॒वैव रेतो॑ दधाति दधाति॒ रेत॑ ए॒वैव रेतो॑ दधाति । \newline
25. रेतो॑ दधाति दधाति॒ रेतो॒ रेतो॑ दधाति॒ पञ्च॒ पञ्च॑ दधाति॒ रेतो॒ रेतो॑ दधाति॒ पञ्च॑ । \newline
26. द॒धा॒ति॒ पञ्च॒ पञ्च॑ दधाति दधाति॒ पञ्चोपोप॒ पञ्च॑ दधाति दधाति॒ पञ्चोप॑ । \newline
27. पञ्चोपोप॒ पञ्च॒ पञ्चोप॑ दधाति दधा॒ त्युप॒ पञ्च॒ पञ्चोप॑ दधाति । \newline
28. उप॑ दधाति दधा॒ त्युपोप॑ दधाति॒ पाङ्क्ताः॒ पाङ्क्ता॑ दधा॒ त्युपोप॑ दधाति॒ पाङ्क्ताः᳚ । \newline
29. द॒धा॒ति॒ पाङ्क्ताः॒ पाङ्क्ता॑ दधाति दधाति॒ पाङ्क्ताः᳚ प॒शवः॑ प॒शवः॒ पाङ्क्ता॑ दधाति दधाति॒ पाङ्क्ताः᳚ प॒शवः॑ । \newline
30. पाङ्क्ताः᳚ प॒शवः॑ प॒शवः॒ पाङ्क्ताः॒ पाङ्क्ताः᳚ प॒शवः॑ प॒शून् प॒शून् प॒शवः॒ पाङ्क्ताः॒ पाङ्क्ताः᳚ प॒शवः॑ प॒शून् । \newline
31. प॒शवः॑ प॒शून् प॒शून् प॒शवः॑ प॒शवः॑ प॒शू ने॒वैव प॒शून् प॒शवः॑ प॒शवः॑ प॒शू ने॒व । \newline
32. प॒शू ने॒वैव प॒शून् प॒शू ने॒वास्मा॑ अस्मा ए॒व प॒शून् प॒शू ने॒वास्मै᳚ । \newline
33. ए॒वास्मा॑ अस्मा ए॒वैवास्मै॒ प्र प्रास्मा॑ ए॒वैवास्मै॒ प्र । \newline
34. अ॒स्मै॒ प्र प्रास्मा॑ अस्मै॒ प्र ज॑नयति जनयति॒ प्रास्मा॑ अस्मै॒ प्र ज॑नयति । \newline
35. प्र ज॑नयति जनयति॒ प्र प्र ज॑नयति॒ पञ्च॒ पञ्च॑ जनयति॒ प्र प्र ज॑नयति॒ पञ्च॑ । \newline
36. ज॒न॒य॒ति॒ पञ्च॒ पञ्च॑ जनयति जनयति॒ पञ्च॑ दक्षिण॒तो द॑क्षिण॒तः पञ्च॑ जनयति जनयति॒ पञ्च॑ दक्षिण॒तः । \newline
37. पञ्च॑ दक्षिण॒तो द॑क्षिण॒तः पञ्च॒ पञ्च॑ दक्षिण॒तो वज्रो॒ वज्रो॑ दक्षिण॒तः पञ्च॒ पञ्च॑ दक्षिण॒तो वज्रः॑ । \newline
38. द॒क्षि॒ण॒तो वज्रो॒ वज्रो॑ दक्षिण॒तो द॑क्षिण॒तो वज्रो॒ वै वै वज्रो॑ दक्षिण॒तो द॑क्षिण॒तो वज्रो॒ वै । \newline
39. वज्रो॒ वै वै वज्रो॒ वज्रो॒ वा अ॑प॒स्या॑ अप॒स्या॑ वै वज्रो॒ वज्रो॒ वा अ॑प॒स्याः᳚ । \newline
40. वा अ॑प॒स्या॑ अप॒स्या॑ वै वा अ॑प॒स्या॑ वज्रे॑ण॒ वज्रे॑णा प॒स्या॑ वै वा अ॑प॒स्या॑ वज्रे॑ण । \newline
41. अ॒प॒स्या॑ वज्रे॑ण॒ वज्रे॑णा प॒स्या॑ अप॒स्या॑ वज्रे॑णै॒वैव वज्रे॑णा प॒स्या॑ अप॒स्या॑ वज्रे॑णै॒व । \newline
42. वज्रे॑णै॒वैव वज्रे॑ण॒ वज्रे॑णै॒व य॒ज्ञ्स्य॑ य॒ज्ञ्स्यै॒व वज्रे॑ण॒ वज्रे॑णै॒व य॒ज्ञ्स्य॑ । \newline
43. ए॒व य॒ज्ञ्स्य॑ य॒ज्ञ् स्यै॒वैव य॒ज्ञ्स्य॑ दक्षिण॒तो द॑क्षिण॒तो य॒ज्ञ् स्यै॒वैव य॒ज्ञ्स्य॑ दक्षिण॒तः । \newline
44. य॒ज्ञ्स्य॑ दक्षिण॒तो द॑क्षिण॒तो य॒ज्ञ्स्य॑ य॒ज्ञ्स्य॑ दक्षिण॒तो रक्षाꣳ॑सि॒ रक्षाꣳ॑सि दक्षिण॒तो य॒ज्ञ्स्य॑ य॒ज्ञ्स्य॑ दक्षिण॒तो रक्षाꣳ॑सि । \newline
45. द॒क्षि॒ण॒तो रक्षाꣳ॑सि॒ रक्षाꣳ॑सि दक्षिण॒तो द॑क्षिण॒तो रक्षाꣳ॒॒ स्यपाप॒ रक्षाꣳ॑सि दक्षिण॒तो द॑क्षिण॒तो रक्षाꣳ॒॒स्यप॑ । \newline
46. रक्षाꣳ॒॒ स्यपाप॒ रक्षाꣳ॑सि॒ रक्षाꣳ॒॒स्यप॑ हन्ति ह॒न्त्यप॒ रक्षाꣳ॑सि॒ रक्षाꣳ॒॒स्यप॑ हन्ति । \newline
47. अप॑ हन्ति ह॒न्त्यपाप॑ हन्ति॒ पञ्च॒ पञ्च॑ ह॒न्त्यपाप॑ हन्ति॒ पञ्च॑ । \newline
48. ह॒न्ति॒ पञ्च॒ पञ्च॑ हन्ति हन्ति॒ पञ्च॑ प॒श्चात् प॒श्चात् पञ्च॑ हन्ति हन्ति॒ पञ्च॑ प॒श्चात् । \newline
49. पञ्च॑ प॒श्चात् प॒श्चात् पञ्च॒ पञ्च॑ प॒श्चात् प्राचीः॒ प्राचीः᳚ प॒श्चात् पञ्च॒ पञ्च॑ प॒श्चात् प्राचीः᳚ । \newline
50. प॒श्चात् प्राचीः॒ प्राचीः᳚ प॒श्चात् प॒श्चात् प्राची॒ रुपोप॒ प्राचीः᳚ प॒श्चात् प॒श्चात् प्राची॒रुप॑ । \newline
\pagebreak
\markright{ TS 5.2.10.2  \hfill https://www.vedavms.in \hfill}

\section{ TS 5.2.10.2 }

\textbf{TS 5.2.10.2 } \newline
\textbf{Samhita Paata} \newline

प्राची॒रुप॑ दधाति प॒श्चाद्वै प्रा॒चीनꣳ॒॒ रेतो॑ धीयते प॒श्चादे॒वास्मै᳚ प्रा॒चीनꣳ॒॒ रेतो॑ दधाति॒ पञ्च॑ पु॒रस्ता᳚त् प्र॒तीची॒रुप॑ दधाति॒ पञ्च॑ प॒श्चात् प्राची॒स्तस्मा᳚त् प्रा॒चीनꣳ॒॒ रेतो॑ धीयते प्र॒तीचीः᳚ प्र॒जा जा॑यन्ते॒ पञ्चो᳚त्तर॒त श्छ॑न्द॒स्याः᳚ प॒शवो॒ वै छ॑न्द॒स्याः᳚ प॒शूने॒व प्रजा॑ता॒न्थ् स्वमा॒यत॑नम॒भि पर्यू॑हत इ॒यं ॅवा अ॒ग्ने-र॑तिदा॒हाद॑बिभे॒थ् सैता - [  ] \newline

\textbf{Pada Paata} \newline

प्राचीः᳚ । उपेति॑ । द॒धा॒ति॒ । प॒श्चात् । वै । प्रा॒चीन᳚म् । रेतः॑ । धी॒य॒ते॒ । प॒श्चात् । ए॒व । अ॒स्मै॒ । प्रा॒चीन᳚म् । रेतः॑ । द॒धा॒ति॒ । पञ्च॑ । पु॒रस्ता᳚त् । प्र॒तीचीः᳚ । उपेति॑ । द॒धा॒ति॒ । पञ्च॑ । प॒श्चात् । प्राचीः᳚ । तस्मा᳚त् । प्रा॒चीन᳚म् । रेतः॑ । धी॒य॒ते॒ । प्र॒तीचीः᳚ । प्र॒जा इति॑ प्र-जाः । जा॒य॒न्ते॒ । पञ्च॑ । उ॒त्त॒र॒त इत्यु॑त् - त॒र॒तः । छ॒न्द॒स्याः᳚ । प॒शवः॑ । वै । छ॒न्द॒स्याः᳚ । प॒शून् । ए॒व । प्रजा॑ता॒निति॒ प्र - जा॒ता॒न् । स्वम् । आ॒यत॑न॒मित्या᳚ - यत॑नम् । अ॒भि । परीति॑ । ऊ॒ह॒ते॒ । इ॒यम् । वै । अ॒ग्नेः । अ॒ति॒दा॒हादित्य॑ति - दा॒हात् । अ॒बि॒भे॒त् । सा । ए॒ताः ।  \newline


\textbf{Krama Paata} \newline

प्राची॒रुप॑ । उप॑ दधाति । द॒धा॒ति॒ प॒श्चात् । प॒श्चाद् वै । वै प्रा॒चीन᳚म् । प्रा॒चीनꣳ॒॒ रेतः॑ । रेतो॑ धीयते । धी॒य॒ते॒ प॒श्चात् । प॒श्चादे॒व । ए॒वास्मै᳚ । अ॒स्मै॒ प्रा॒चीन᳚म् । प्रा॒चीनꣳ॒॒ रेतः॑ । रेतो॑ दधाति । द॒धा॒ति॒ पञ्च॑ । पञ्च॑ पु॒रस्ता᳚त् । पु॒रस्ता᳚त् प्र॒तीचीः᳚ । प्र॒तीची॒रुप॑ । उप॑ दधाति । द॒धा॒ति॒ पञ्च॑ । पञ्च॑ प॒श्चात् । प॒श्चात् प्राचीः᳚ । प्राची॒स्तस्मा᳚त् । तस्मा᳚त् प्रा॒चीन᳚म् । प्रा॒चीनꣳ॒॒ रेतः॑ । रेतो॑ धीयते । धी॒य॒ते॒ प्र॒तीचीः᳚ । प्र॒तीचीः᳚ प्र॒जाः । प्र॒जा जा॑यन्ते । प्र॒जा इति॑ प्र - जाः । जा॒य॒न्ते॒ पञ्च॑ । पञ्चो᳚त्तर॒तः । उ॒त्त॒र॒तश्छ॑न्द॒स्याः᳚ । उ॒त्त॒र॒त इत्यु॑त् - त॒र॒तः । छ॒न्द॒स्याः᳚ प॒शवः॑ । प॒शवो॒ वै । वै छ॑न्द॒स्याः᳚ । छ॒न्द॒स्याः᳚ प॒शून् । प॒शूने॒व । ए॒व प्रजा॑तान् । प्रजा॑ता॒न्थ् स्वम् । प्रजा॑ता॒निति॒ प्र - जा॒ता॒न्॒ । स्वमा॒यत॑नम् । आ॒यत॑नम॒भि । आ॒यत॑न॒मित्या᳚ - यत॑नम् । अ॒भि परि॑ । पर्यू॑हते । ऊ॒ह॒त॒ इ॒यम् । इ॒यम् ॅवै । वा अ॒ग्नेः । अ॒ग्नेर॑तिदा॒हात् । अ॒ति॒दा॒हाद॑बिभेत् । अ॒ति॒दा॒हादित्य॑ति - दा॒हात् । अ॒बि॒भे॒थ् सा । सैताः । ए॒ता अ॑प॒स्याः᳚ \newline

\textbf{Jatai Paata} \newline

1. प्राची॒ रुपोप॒ प्राचीः॒ प्राची॒ रुप॑ । \newline
2. उप॑ दधाति दधा॒ त्युपोप॑ दधाति । \newline
3. द॒धा॒ति॒ प॒श्चात् प॒श्चाद् द॑धाति दधाति प॒श्चात् । \newline
4. प॒श्चाद् वै वै प॒श्चात् प॒श्चाद् वै । \newline
5. वै प्रा॒चीन॑म् प्रा॒चीनं॒ ॅवै वै प्रा॒चीन᳚म् । \newline
6. प्रा॒चीनꣳ॒॒ रेतो॒ रेतः॑ प्रा॒चीन॑म् प्रा॒चीनꣳ॒॒ रेतः॑ । \newline
7. रेतो॑ धीयते धीयते॒ रेतो॒ रेतो॑ धीयते । \newline
8. धी॒य॒ते॒ प॒श्चात् प॒श्चाद् धी॑यते धीयते प॒श्चात् । \newline
9. प॒श्चा दे॒वैव प॒श्चात् प॒श्चा दे॒व । \newline
10. ए॒वास्मा॑ अस्मा ए॒वै वास्मै᳚ । \newline
11. अ॒स्मै॒ प्रा॒चीन॑म् प्रा॒चीन॑ मस्मा अस्मै प्रा॒चीन᳚म् । \newline
12. प्रा॒चीनꣳ॒॒ रेतो॒ रेतः॑ प्रा॒चीन॑म् प्रा॒चीनꣳ॒॒ रेतः॑ । \newline
13. रेतो॑ दधाति दधाति॒ रेतो॒ रेतो॑ दधाति । \newline
14. द॒धा॒ति॒ पञ्च॒ पञ्च॑ दधाति दधाति॒ पञ्च॑ । \newline
15. पञ्च॑ पु॒रस्ता᳚त् पु॒रस्ता॒त् पञ्च॒ पञ्च॑ पु॒रस्ता᳚त् । \newline
16. पु॒रस्ता᳚त् प्र॒तीचीः᳚ प्र॒तीचीः᳚ पु॒रस्ता᳚त् पु॒रस्ता᳚त् प्र॒तीचीः᳚ । \newline
17. प्र॒तीची॒ रुपोप॑ प्र॒तीचीः᳚ प्र॒तीची॒ रुप॑ । \newline
18. उप॑ दधाति दधा॒ त्युपोप॑ दधाति । \newline
19. द॒धा॒ति॒ पञ्च॒ पञ्च॑ दधाति दधाति॒ पञ्च॑ । \newline
20. पञ्च॑ प॒श्चात् प॒श्चात् पञ्च॒ पञ्च॑ प॒श्चात् । \newline
21. प॒श्चात् प्राचीः॒ प्राचीः᳚ प॒श्चात् प॒श्चात् प्राचीः᳚ । \newline
22. प्राची॒ स्तस्मा॒त् तस्मा॒त् प्राचीः॒ प्राची॒ स्तस्मा᳚त् । \newline
23. तस्मा᳚त् प्रा॒चीन॑म् प्रा॒चीन॒म् तस्मा॒त् तस्मा᳚त् प्रा॒चीन᳚म् । \newline
24. प्रा॒चीनꣳ॒॒ रेतो॒ रेतः॑ प्रा॒चीन॑म् प्रा॒चीनꣳ॒॒ रेतः॑ । \newline
25. रेतो॑ धीयते धीयते॒ रेतो॒ रेतो॑ धीयते । \newline
26. धी॒य॒ते॒ प्र॒तीचीः᳚ प्र॒तीची᳚र् धीयते धीयते प्र॒तीचीः᳚ । \newline
27. प्र॒तीचीः᳚ प्र॒जाः प्र॒जाः प्र॒तीचीः᳚ प्र॒तीचीः᳚ प्र॒जाः । \newline
28. प्र॒जा जा॑यन्ते जायन्ते प्र॒जाः प्र॒जा जा॑यन्ते । \newline
29. प्र॒जा इति॑ प्र - जाः । \newline
30. जा॒य॒न्ते॒ पञ्च॒ पञ्च॑ जायन्ते जायन्ते॒ पञ्च॑ । \newline
31. पञ्चो᳚त्तर॒त उ॑त्तर॒तः पञ्च॒ पञ्चो᳚त्तर॒तः । \newline
32. उ॒त्त॒र॒त श्छ॑न्द॒स्या᳚ श्छन्द॒स्या॑ उत्तर॒त उ॑त्तर॒त श्छ॑न्द॒स्याः᳚ । \newline
33. उ॒त्त॒र॒त इत्यु॑त् - त॒र॒तः । \newline
34. छ॒न्द॒स्याः᳚ प॒शवः॑ प॒शव॑ श्छन्द॒स्या᳚ श्छन्द॒स्याः᳚ प॒शवः॑ । \newline
35. प॒शवो॒ वै वै प॒शवः॑ प॒शवो॒ वै । \newline
36. वै छ॑न्द॒स्या᳚ श्छन्द॒स्या॑ वै वै छ॑न्द॒स्याः᳚ । \newline
37. छ॒न्द॒स्याः᳚ प॒शून् प॒शून् छ॑न्द॒स्या᳚ श्छन्द॒स्याः᳚ प॒शून् । \newline
38. प॒शू ने॒वैव प॒शून् प॒शू ने॒व । \newline
39. ए॒व प्रजा॑ता॒न् प्रजा॑ता ने॒वैव प्रजा॑तान् । \newline
40. प्रजा॑ता॒न् थ्स्वꣳ स्वम् प्रजा॑ता॒न् प्रजा॑ता॒न् थ्स्वम् । \newline
41. प्रजा॑ता॒निति॒ प्र - जा॒ता॒न् । \newline
42. स्व मा॒यत॑न मा॒यत॑नꣳ॒॒ स्वꣳ स्व मा॒यत॑नम् । \newline
43. आ॒यत॑न म॒भ्या᳚(1॒)भ्या॑यत॑न मा॒यत॑न म॒भि । \newline
44. आ॒यत॑न॒मित्या᳚ - यत॑नम् । \newline
45. अ॒भि परि॒ पर्य॒ भ्य॑भि परि॑ । \newline
46. पर्यू॑हत ऊहते॒ परि॒ पर्यू॑हते । \newline
47. ऊ॒ह॒त॒ इ॒य मि॒य मू॑हत ऊहत इ॒यम् । \newline
48. इ॒यं ॅवै वा इ॒य मि॒यं ॅवै । \newline
49. वा अ॒ग्ने र॒ग्नेर् वै वा अ॒ग्नेः । \newline
50. अ॒ग्ने र॑तिदा॒हा द॑तिदा॒हा द॒ग्ने र॒ग्ने र॑तिदा॒हात् । \newline
51. अ॒ति॒दा॒हा द॑बिभे दबिभे दतिदा॒हा द॑तिदा॒हा द॑बिभेत् । \newline
52. अ॒ति॒दा॒हादित्य॑ति - दा॒हात् । \newline
53. अ॒बि॒भे॒थ् सा सा ऽबि॑भे दबिभे॒थ् सा । \newline
54. सैता ए॒ताः सा सैताः । \newline
55. ए॒ता अ॑प॒स्या॑ अप॒स्या॑ ए॒ता ए॒ता अ॑प॒स्याः᳚ । \newline

\textbf{Ghana Paata } \newline

1. प्राची॒ रुपोप॒ प्राचीः॒ प्राची॒रुप॑ दधाति दधा॒ त्युप॒ प्राचीः॒ प्राची॒रुप॑ दधाति । \newline
2. उप॑ दधाति दधा॒ त्युपोप॑ दधाति प॒श्चात् प॒श्चाद् द॑धा॒ त्युपोप॑ दधाति प॒श्चात् । \newline
3. द॒धा॒ति॒ प॒श्चात् प॒श्चाद् द॑धाति दधाति प॒श्चाद् वै वै प॒श्चाद् द॑धाति दधाति प॒श्चाद् वै । \newline
4. प॒श्चाद् वै वै प॒श्चात् प॒श्चाद् वै प्रा॒चीन॑म् प्रा॒चीनं॒ ॅवै प॒श्चात् प॒श्चाद् वै प्रा॒चीन᳚म् । \newline
5. वै प्रा॒चीन॑म् प्रा॒चीनं॒ ॅवै वै प्रा॒चीनꣳ॒॒ रेतो॒ रेतः॑ प्रा॒चीनं॒ ॅवै वै प्रा॒चीनꣳ॒॒ रेतः॑ । \newline
6. प्रा॒चीनꣳ॒॒ रेतो॒ रेतः॑ प्रा॒चीन॑म् प्रा॒चीनꣳ॒॒ रेतो॑ धीयते धीयते॒ रेतः॑ प्रा॒चीन॑म् प्रा॒चीनꣳ॒॒ रेतो॑ धीयते । \newline
7. रेतो॑ धीयते धीयते॒ रेतो॒ रेतो॑ धीयते प॒श्चात् प॒श्चाद् धी॑यते॒ रेतो॒ रेतो॑ धीयते प॒श्चात् । \newline
8. धी॒य॒ते॒ प॒श्चात् प॒श्चाद् धी॑यते धीयते प॒श्चा दे॒वैव प॒श्चाद् धी॑यते धीयते प॒श्चादे॒व । \newline
9. प॒श्चा दे॒वैव प॒श्चात् प॒श्चा दे॒वास्मा॑ अस्मा ए॒व प॒श्चात् प॒श्चा दे॒वास्मै᳚ । \newline
10. ए॒वास्मा॑ अस्मा ए॒वैवास्मै᳚ प्रा॒चीन॑म् प्रा॒चीन॑ मस्मा ए॒वैवास्मै᳚ प्रा॒चीन᳚म् । \newline
11. अ॒स्मै॒ प्रा॒चीन॑म् प्रा॒चीन॑ मस्मा अस्मै प्रा॒चीनꣳ॒॒ रेतो॒ रेतः॑ प्रा॒चीन॑ मस्मा अस्मै प्रा॒चीनꣳ॒॒ रेतः॑ । \newline
12. प्रा॒चीनꣳ॒॒ रेतो॒ रेतः॑ प्रा॒चीन॑म् प्रा॒चीनꣳ॒॒ रेतो॑ दधाति दधाति॒ रेतः॑ प्रा॒चीन॑म् प्रा॒चीनꣳ॒॒ रेतो॑ दधाति । \newline
13. रेतो॑ दधाति दधाति॒ रेतो॒ रेतो॑ दधाति॒ पञ्च॒ पञ्च॑ दधाति॒ रेतो॒ रेतो॑ दधाति॒ पञ्च॑ । \newline
14. द॒धा॒ति॒ पञ्च॒ पञ्च॑ दधाति दधाति॒ पञ्च॑ पु॒रस्ता᳚त् पु॒रस्ता॒त् पञ्च॑ दधाति दधाति॒ पञ्च॑ पु॒रस्ता᳚त् । \newline
15. पञ्च॑ पु॒रस्ता᳚त् पु॒रस्ता॒त् पञ्च॒ पञ्च॑ पु॒रस्ता᳚त् प्र॒तीचीः᳚ प्र॒तीचीः᳚ पु॒रस्ता॒त् पञ्च॒ पञ्च॑ पु॒रस्ता᳚त् प्र॒तीचीः᳚ । \newline
16. पु॒रस्ता᳚त् प्र॒तीचीः᳚ प्र॒तीचीः᳚ पु॒रस्ता᳚त् पु॒रस्ता᳚त् प्र॒तीची॒ रुपोप॑ प्र॒तीचीः᳚ पु॒रस्ता᳚त् पु॒रस्ता᳚त् प्र॒तीची॒रुप॑ । \newline
17. प्र॒तीची॒ रुपोप॑ प्र॒तीचीः᳚ प्र॒तीची॒रुप॑ दधाति दधा॒ त्युप॑ प्र॒तीचीः᳚ प्र॒तीची॒ रुप॑ दधाति । \newline
18. उप॑ दधाति दधा॒ त्युपोप॑ दधाति॒ पञ्च॒ पञ्च॑ दधा॒ त्युपोप॑ दधाति॒ पञ्च॑ । \newline
19. द॒धा॒ति॒ पञ्च॒ पञ्च॑ दधाति दधाति॒ पञ्च॑ प॒श्चात् प॒श्चात् पञ्च॑ दधाति दधाति॒ पञ्च॑ प॒श्चात् । \newline
20. पञ्च॑ प॒श्चात् प॒श्चात् पञ्च॒ पञ्च॑ प॒श्चात् प्राचीः॒ प्राचीः᳚ प॒श्चात् पञ्च॒ पञ्च॑ प॒श्चात् प्राचीः᳚ । \newline
21. प॒श्चात् प्राचीः॒ प्राचीः᳚ प॒श्चात् प॒श्चात् प्राची॒ स्तस्मा॒त् तस्मा॒त् प्राचीः᳚ प॒श्चात् प॒श्चात् प्राची॒ स्तस्मा᳚त् । \newline
22. प्राची॒ स्तस्मा॒त् तस्मा॒त् प्राचीः॒ प्राची॒ स्तस्मा᳚त् प्रा॒चीन॑म् प्रा॒चीन॒म् तस्मा॒त् प्राचीः॒ प्राची॒ स्तस्मा᳚त् प्रा॒चीन᳚म् । \newline
23. तस्मा᳚त् प्रा॒चीन॑म् प्रा॒चीन॒म् तस्मा॒त् तस्मा᳚त् प्रा॒चीनꣳ॒॒ रेतो॒ रेतः॑ प्रा॒चीन॒म् तस्मा॒त् तस्मा᳚त् प्रा॒चीनꣳ॒॒ रेतः॑ । \newline
24. प्रा॒चीनꣳ॒॒ रेतो॒ रेतः॑ प्रा॒चीन॑म् प्रा॒चीनꣳ॒॒ रेतो॑ धीयते धीयते॒ रेतः॑ प्रा॒चीन॑म् प्रा॒चीनꣳ॒॒ रेतो॑ धीयते । \newline
25. रेतो॑ धीयते धीयते॒ रेतो॒ रेतो॑ धीयते प्र॒तीचीः᳚ प्र॒तीची᳚र् धीयते॒ रेतो॒ रेतो॑ धीयते प्र॒तीचीः᳚ । \newline
26. धी॒य॒ते॒ प्र॒तीचीः᳚ प्र॒तीची᳚र् धीयते धीयते प्र॒तीचीः᳚ प्र॒जाः प्र॒जाः प्र॒तीची᳚र् धीयते धीयते प्र॒तीचीः᳚ प्र॒जाः । \newline
27. प्र॒तीचीः᳚ प्र॒जाः प्र॒जाः प्र॒तीचीः᳚ प्र॒तीचीः᳚ प्र॒जा जा॑यन्ते जायन्ते प्र॒जाः प्र॒तीचीः᳚ प्र॒तीचीः᳚ प्र॒जा जा॑यन्ते । \newline
28. प्र॒जा जा॑यन्ते जायन्ते प्र॒जाः प्र॒जा जा॑यन्ते॒ पञ्च॒ पञ्च॑ जायन्ते प्र॒जाः प्र॒जा जा॑यन्ते॒ पञ्च॑ । \newline
29. प्र॒जा इति॑ प्र - जाः । \newline
30. जा॒य॒न्ते॒ पञ्च॒ पञ्च॑ जायन्ते जायन्ते॒ पञ्चो᳚त्तर॒त उ॑त्तर॒तः पञ्च॑ जायन्ते जायन्ते॒ पञ्चो᳚त्तर॒तः । \newline
31. पञ्चो᳚त्तर॒त उ॑त्तर॒तः पञ्च॒ पञ्चो᳚त्तर॒त श्छ॑न्द॒स्या᳚ श्छन्द॒स्या॑ उत्तर॒तः पञ्च॒ पञ्चो᳚त्तर॒त श्छ॑न्द॒स्याः᳚ । \newline
32. उ॒त्त॒र॒त श्छ॑न्द॒स्या᳚ श्छन्द॒स्या॑ उत्तर॒त उ॑त्तर॒त श्छ॑न्द॒स्याः᳚ प॒शवः॑ प॒शव॑ श्छन्द॒स्या॑ उत्तर॒त उ॑त्तर॒त श्छ॑न्द॒स्याः᳚ प॒शवः॑ । \newline
33. उ॒त्त॒र॒त इत्यु॑त् - त॒र॒तः । \newline
34. छ॒न्द॒स्याः᳚ प॒शवः॑ प॒शव॑ श्छन्द॒स्या᳚ श्छन्द॒स्याः᳚ प॒शवो॒ वै वै प॒शव॑ श्छन्द॒स्या᳚ श्छन्द॒स्याः᳚ प॒शवो॒ वै । \newline
35. प॒शवो॒ वै वै प॒शवः॑ प॒शवो॒ वै छ॑न्द॒स्या᳚ श्छन्द॒स्या॑ वै प॒शवः॑ प॒शवो॒ वै छ॑न्द॒स्याः᳚ । \newline
36. वै छ॑न्द॒स्या᳚ श्छन्द॒स्या॑ वै वै छ॑न्द॒स्याः᳚ प॒शून् प॒शून् छ॑न्द॒स्या॑ वै वै छ॑न्द॒स्याः᳚ प॒शून् । \newline
37. छ॒न्द॒स्याः᳚ प॒शून् प॒शून् छ॑न्द॒स्या᳚ श्छन्द॒स्याः᳚ प॒शू ने॒वैव प॒शून् छ॑न्द॒स्या᳚ श्छन्द॒स्याः᳚ प॒शू ने॒व । \newline
38. प॒शू ने॒वैव प॒शून् प॒शू ने॒व प्रजा॑ता॒न् प्रजा॑ता ने॒व प॒शून् प॒शू ने॒व प्रजा॑तान् । \newline
39. ए॒व प्रजा॑ता॒न् प्रजा॑ता ने॒वैव प्रजा॑ता॒न् थ्स्वꣳ स्वम् प्रजा॑ता ने॒वैव प्रजा॑ता॒न् थ्स्वम् । \newline
40. प्रजा॑ता॒न् थ्स्वꣳ स्वम् प्रजा॑ता॒न् प्रजा॑ता॒न् थ्स्व मा॒यत॑न मा॒यत॑नꣳ॒॒ स्वम् प्रजा॑ता॒न् प्रजा॑ता॒न् थ्स्व मा॒यत॑नम् । \newline
41. प्रजा॑ता॒निति॒ प्र - जा॒ता॒न् । \newline
42. स्व मा॒यत॑न मा॒यत॑नꣳ॒॒ स्वꣳ स्व मा॒यत॑न म॒भ्या᳚(1॒)भ्या॑यत॑नꣳ॒॒ स्वꣳ स्व मा॒यत॑न म॒भि । \newline
43. आ॒यत॑न म॒भ्या᳚(1॒)भ्या॑यत॑न मा॒यत॑न म॒भि परि॒ पर्य॒भ्या॑यत॑न मा॒यत॑न म॒भि परि॑ । \newline
44. आ॒यत॑न॒मित्या᳚ - यत॑नम् । \newline
45. अ॒भि परि॒ पर्य॒भ्य॑भि पर्यू॑हत ऊहते॒ पर्य॒भ्य॑भि पर्यू॑हते । \newline
46. पर्यू॑हत ऊहते॒ परि॒ पर्यू॑हत इ॒य मि॒य मू॑हते॒ परि॒ पर्यू॑हत इ॒यम् । \newline
47. ऊ॒ह॒त॒ इ॒य मि॒य मू॑हत ऊहत इ॒यं ॅवै वा इ॒य मू॑हत ऊहत इ॒यं ॅवै । \newline
48. इ॒यं ॅवै वा इ॒य मि॒यं ॅवा अ॒ग्ने र॒ग्नेर् वा इ॒य मि॒यं ॅवा अ॒ग्नेः । \newline
49. वा अ॒ग्ने र॒ग्नेर् वै वा अ॒ग्ने र॑तिदा॒हा द॑तिदा॒हा द॒ग्नेर् वै वा अ॒ग्ने र॑तिदा॒हात् । \newline
50. अ॒ग्ने र॑तिदा॒हा द॑तिदा॒हा द॒ग्ने र॒ग्ने र॑तिदा॒हा द॑बिभे दबिभे दतिदा॒हा द॒ग्ने र॒ग्ने र॑तिदा॒हा द॑बिभेत् । \newline
51. अ॒ति॒दा॒हा द॑बिभे दबिभे दतिदा॒हा द॑तिदा॒हा द॑बिभे॒थ् सा सा ऽबि॑भे दतिदा॒हा द॑तिदा॒हा द॑बिभे॒थ् सा । \newline
52. अ॒ति॒दा॒हादित्य॑ति - दा॒हात् । \newline
53. अ॒बि॒भे॒थ् सा सा ऽबि॑भे दबिभे॒थ् सैता ए॒ताः सा ऽबि॑भे दबिभे॒थ् सैताः । \newline
54. सैता ए॒ताः सा सैता अ॑प॒स्या॑ अप॒स्या॑ ए॒ताः सा सैता अ॑प॒स्याः᳚ । \newline
55. ए॒ता अ॑प॒स्या॑ अप॒स्या॑ ए॒ता ए॒ता अ॑प॒स्या॑ अपश्य दपश्य दप॒स्या॑ ए॒ता ए॒ता अ॑प॒स्या॑ अपश्यत् । \newline
\pagebreak
\markright{ TS 5.2.10.3  \hfill https://www.vedavms.in \hfill}

\section{ TS 5.2.10.3 }

\textbf{TS 5.2.10.3 } \newline
\textbf{Samhita Paata} \newline

अ॑प॒स्या॑ अपश्य॒त् ता उपा॑धत्त॒ ततो॒ वा इ॒मां नात्य॑दह॒द्-यद॑प॒स्या॑ उप॒दधा᳚त्य॒स्या अन॑तिदाहायो॒वाच॑ हे॒यमद॒दिथ् स ब्रह्म॒णाऽन्नं॒ ॅयस्यै॒ता उ॑पधी॒यान्तै॒ य उ॑ चैना ए॒वंॅवे द॒दिति॑ प्राण॒भृत॒ उप॑ दधाति॒ रेत॑स्ये॒व प्रा॒णान् द॑धाति॒ तस्मा॒द्-वद॑न् प्रा॒णन् पश्य॑ञ्छृ॒ण्वन् प॒शुर्जा॑यते॒ ऽयं पु॒रो - [  ] \newline

\textbf{Pada Paata} \newline

अ॒प॒स्याः᳚ । अ॒प॒श्य॒त् । ताः । उपेति॑ । अ॒ध॒त्त॒ । ततः॑ । वै । इ॒माम् । न । अतीति॑ । अ॒द॒ह॒त् । यत् । अ॒प॒स्याः᳚ । उ॒प॒दधा॒तीत्यु॑प - दधा॑ति । अ॒स्याः । अन॑तिदाहा॒येत्यन॑ति - दा॒हा॒य॒ । उ॒वाच॑ । ह॒ । इ॒यम् । अद॑त् । इत् । सः । ब्रह्म॑णा । अन्न᳚म् । यस्य॑ । ए॒ताः । उ॒प॒धी॒यान्ता॒ इत्यु॑प - धी॒यान्तै᳚ । यः । उ॒ । च॒ । ए॒नाः॒ । ए॒वम् । वेद॑त् । इति॑ । प्रा॒ण॒भृत॒ इति॑ प्राण-भृतः॑ । उपेति॑ । द॒धा॒ति॒ । रेत॑सि । ए॒व । प्रा॒णानिति॑ प्र - अ॒नान् । द॒धा॒ति॒ । तस्मा᳚त् । वदन्न्॑ । प्रा॒णन्निति॑ प्र - अ॒नन्न् । पश्यन्न्॑ । शृ॒ण्वन्न् । प॒शुः । जा॒य॒ते॒ । अ॒यम् । पु॒रः ।  \newline


\textbf{Krama Paata} \newline

अ॒प॒स्या॑ अपश्यत् । अ॒प॒श्य॒त् ताः । ता उप॑ । उपा॑धत्त । अ॒ध॒त्त॒ ततः॑ । ततो॒ वै । वा इ॒माम् । इ॒माम् न । नाति॑ । अत्य॑दहत् । अ॒द॒ह॒द् यत् । यद॑प॒स्याः᳚ । अ॒प॒स्या॑ उप॒दधा॑ति । उ॒प॒दधा᳚त्य॒स्याः । उ॒प॒दधा॒तीत्यु॑प - दधा॑ति । अ॒स्या अन॑तिदाहाय । अन॑तिदाहायो॒वाच॑ । अन॑तिदाहा॒येत्यन॑ति - दा॒हा॒य॒ । उ॒वाच॑ ह । हे॒यम् । इ॒यमद॑त् । अद॒दित् । इथ् सः । स ब्रह्म॑णा । ब्रह्म॒णाऽन्न᳚म् । अन्न॒म् ॅयस्य॑ । यस्यै॒ताः । ए॒ता उ॑पधी॒यान्तै᳚ । उ॒प॒धी॒यान्तै॒ यः । उ॒प॒धी॒यान्ता॒ इत्यु॑प - धी॒यान्तै᳚ । य उ॑ । उ॒ च॒ । चै॒नाः॒ । ए॒ना॒ ए॒वम् । ए॒वम् ॅवेद॑त् । वेद॒दिति॑ । इति॑ प्राण॒भृतः॑ । प्रा॒ण॒भृत॒ उप॑ । प्रा॒ण॒भृत॒ इति॑ प्राण - भृतः॑ । उप॑ दधाति । द॒धा॒ति॒ रेत॑सि । रेत॑स्ये॒व । ए॒व प्रा॒णान् । प्रा॒णान् द॑धाति । प्रा॒णानिति॑ प्र - अ॒नान् । द॒धा॒ति॒ तस्मा᳚त् । तस्मा॒द् वदन्न्॑ । वद॑न् प्रा॒णन्न् । प्रा॒णन् पश्यन्न्॑ । प्रा॒णन्निति॑ प्र - अ॒नन्न् । पश्य॑ञ्छृ॒ण्वन्न् । शृ॒ण्वन् प॒शुः । प॒शुर् जा॑यते । जा॒य॒ते॒ऽयम् । अ॒यम् पु॒रः । पु॒रो भुवः॑ \newline

\textbf{Jatai Paata} \newline

1. अ॒प॒स्या॑ अपश्यद पश्य दप॒स्या॑ अप॒स्या॑ अपश्यत् । \newline
2. अ॒प॒श्य॒त् ता स्ता अ॑पश्य दपश्य॒त् ताः । \newline
3. ता उपोप॒ ता स्ता उप॑ । \newline
4. उपा॑ धत्ता ध॒त्तो पोपा॑ धत्त । \newline
5. अ॒ध॒त्त॒ तत॒ स्ततो॑ ऽधत्ता धत्त॒ ततः॑ । \newline
6. ततो॒ वै वै तत॒ स्ततो॒ वै । \newline
7. वा इ॒मा मि॒मां ॅवै वा इ॒माम् । \newline
8. इ॒माम् न ने मा मि॒माम् न । \newline
9. नात्यति॒ न नाति॑ । \newline
10. अत्य॑दह ददह॒ दत्य त्य॑दहत् । \newline
11. अ॒द॒ह॒द् यद् यद॑दह ददह॒द् यत् । \newline
12. यद॑प॒स्या॑ अप॒स्या॑ यद् यद॑प॒स्याः᳚ । \newline
13. अ॒प॒स्या॑ उप॒दधा᳚ त्युप॒दधा᳚ त्यप॒स्या॑ अप॒स्या॑ उप॒दधा॑ति । \newline
14. उ॒प॒दधा᳚ त्य॒स्या अ॒स्या उ॑प॒दधा᳚ त्युप॒दधा᳚ त्य॒स्याः । \newline
15. उ॒प॒दधा॒तीत्यु॑प - दधा॑ति । \newline
16. अ॒स्या अन॑तिदाहा॒या न॑तिदाहा या॒स्या अ॒स्या अन॑तिदाहाय । \newline
17. अन॑तिदाहायो॒ वाचो॒वाचा न॑तिदाहा॒या न॑तिदाहा यो॒वाच॑ । \newline
18. अन॑तिदाहा॒येत्यन॑ति - दा॒हा॒य॒ । \newline
19. उ॒वाच॑ ह हो॒वाचो॒ वाच॑ ह । \newline
20. हे॒य मि॒यꣳ ह॑ हे॒यम् । \newline
21. इ॒य मद॒ दद॑ दि॒य मि॒य मद॑त् । \newline
22. अद॒ दिदिद द॒द द॒दित् । \newline
23. इथ् स सेतिथ् सः । \newline
24. स ब्रह्म॑णा॒ ब्रह्म॑णा॒ स स ब्रह्म॑णा । \newline
25. ब्रह्म॒णा ऽन्न॒ मन्न॒म् ब्रह्म॑णा॒ ब्रह्म॒णा ऽन्न᳚म् । \newline
26. अन्नं॒ ॅयस्य॒ यस्या न्न॒ मन्नं॒ ॅयस्य॑ । \newline
27. यस्यै॒ता ए॒ता यस्य॒ यस्यै॒ताः । \newline
28. ए॒ता उ॑पधी॒यान्ता॑ उपधी॒यान्ता॑ ए॒ता ए॒ता उ॑पधी॒यान्तै᳚ । \newline
29. उ॒प॒धी॒यान्तै॒ यो य उ॑पधी॒यान्ता॑ उपधी॒यान्तै॒ यः । \newline
30. उ॒प॒धी॒यान्ता॒ इत्यु॑प - धी॒यान्तै᳚ । \newline
31. य उ॑ वु॒ यो य उ॑ । \newline
32. उ॒ च॒ चो॒ वु॒ च॒ । \newline
33. चै॒ना॒ ए॒ना॒श्च॒ चै॒नाः॒ । \newline
34. ए॒ना॒ ए॒व मे॒व मे॑ना एना ए॒वम् । \newline
35. ए॒वं ॅवेद॒द् वेद॑ दे॒व मे॒वं ॅवेद॑त् । \newline
36. वेद॒ दितीति॒ वेद॒द् वेद॒ दिति॑ । \newline
37. इति॑ प्राण॒भृतः॑ प्राण॒भृत॒ इतीति॑ प्राण॒भृतः॑ । \newline
38. प्रा॒ण॒भृत॒ उपोप॑ प्राण॒भृतः॑ प्राण॒भृत॒ उप॑ । \newline
39. प्रा॒ण॒भृत॒ इति॑ प्राण - भृतः॑ । \newline
40. उप॑ दधाति दधा॒ त्युपोप॑ दधाति । \newline
41. द॒धा॒ति॒ रेत॑सि॒ रेत॑सि दधाति दधाति॒ रेत॑सि । \newline
42. रेत॑ स्ये॒वैव रेत॑सि॒ रेत॑ स्ये॒व । \newline
43. ए॒व प्रा॒णान् प्रा॒णा ने॒वैव प्रा॒णान् । \newline
44. प्रा॒णान् द॑धाति दधाति प्रा॒णान् प्रा॒णान् द॑धाति । \newline
45. प्रा॒णानिति॑ प्र - अ॒नान् । \newline
46. द॒धा॒ति॒ तस्मा॒त् तस्मा᳚द् दधाति दधाति॒ तस्मा᳚त् । \newline
47. तस्मा॒द् वद॒न्॒. वद॒न् तस्मा॒त् तस्मा॒द् वदन्न्॑ । \newline
48. वद॑न् प्रा॒णन् प्रा॒णन्. वद॒न्॒. वद॑न् प्रा॒णन्न् । \newline
49. प्रा॒णन् पश्य॒न् पश्य॑न् प्रा॒णन् प्रा॒णन् पश्यन्न्॑ । \newline
50. प्रा॒णन्निति॑ प्र - अ॒नन्न् । \newline
51. पश्य॑ञ् छृ॒ण्वञ् छृ॒ण्वन् पश्य॒न् पश्य॑ञ् छृ॒ण्वन्न् । \newline
52. शृ॒ण्वन् प॒शुः प॒शुः शृ॒ण्वञ् छृ॒ण्वन् प॒शुः । \newline
53. प॒शुर् जा॑यते जायते प॒शुः प॒शुर् जा॑यते । \newline
54. जा॒य॒ते॒ ऽय म॒यम् जा॑यते जायते॒ ऽयम् । \newline
55. अ॒यम् पु॒रः पु॒रो॑ ऽय म॒यम् पु॒रः । \newline
56. पु॒रो भुवो॒ भुवः॑ पु॒रः पु॒रो भुवः॑ । \newline

\textbf{Ghana Paata } \newline

1. अ॒प॒स्या॑ अपश्य दपश्य दप॒स्या॑ अप॒स्या॑ अपश्य॒त् ता स्ता अ॑पश्य दप॒स्या॑ अप॒स्या॑ अपश्य॒त् ताः । \newline
2. अ॒प॒श्य॒त् ता स्ता अ॑पश्य दपश्य॒त् ता उपोप॒ ता अ॑पश्य दपश्य॒त् ता उप॑ । \newline
3. ता उपोप॒ ता स्ता उपा॑धत्ता ध॒त्तोप॒ ता स्ता उपा॑धत्त । \newline
4. उपा॑धत्ता ध॒त्तोपो पा॑धत्त॒ तत॒ स्ततो॑ ऽध॒त्तोपो पा॑धत्त॒ ततः॑ । \newline
5. अ॒ध॒त्त॒ तत॒ स्ततो॑ ऽधत्ता धत्त॒ ततो॒ वै वै ततो॑ ऽधत्ता धत्त॒ ततो॒ वै । \newline
6. ततो॒ वै वै तत॒ स्ततो॒ वा इ॒मा मि॒मां ॅवै तत॒ स्ततो॒ वा इ॒माम् । \newline
7. वा इ॒मा मि॒मां ॅवै वा इ॒माम् न नेमां ॅवै वा इ॒माम् न । \newline
8. इ॒माम् न नेमा मि॒माम् नात्यति॒ नेमा मि॒माम् नाति॑ । \newline
9. नात्यति॒ न नात्य॑दह ददह॒दति॒ न नात्य॑दहत् । \newline
10. अत्य॑दह ददह॒द त्य त्य॑दह॒द् यद् यद॑दह॒द त्य त्य॑दह॒द् यत् । \newline
11. अ॒द॒ह॒द् यद् यद॑दह ददह॒द् यद॑प॒स्या॑ अप॒स्या॑ यद॑दह ददह॒द् यद॑प॒स्याः᳚ । \newline
12. यद॑प॒स्या॑ अप॒स्या॑ यद् यद॑प॒स्या॑ उप॒दधा᳚ त्युप॒दधा᳚ त्यप॒स्या॑ यद् यद॑प॒स्या॑ उप॒दधा॑ति । \newline
13. अ॒प॒स्या॑ उप॒दधा᳚ त्युप॒दधा᳚ त्यप॒स्या॑ अप॒स्या॑ उप॒दधा᳚ त्य॒स्या अ॒स्या उ॑प॒दधा᳚ त्यप॒स्या॑ अप॒स्या॑ उप॒दधा᳚ त्य॒स्याः । \newline
14. उ॒प॒दधा᳚ त्य॒स्या अ॒स्या उ॑प॒दधा᳚ त्युप॒दधा᳚ त्य॒स्या अन॑तिदाहा॒या न॑तिदाहाया॒स्या उ॑प॒दधा ᳚त्युप॒दधा᳚ त्य॒स्या अन॑तिदाहाय । \newline
15. उ॒प॒दधा॒तीत्यु॑प - दधा॑ति । \newline
16. अ॒स्या अन॑तिदाहा॒या न॑तिदाहाया॒ स्या अ॒स्या अन॑तिदाहा यो॒वाचो॒ वाचा न॑तिदाहाया॒ स्या अ॒स्या अन॑तिदाहा यो॒वाच॑ । \newline
17. अन॑तिदाहा यो॒वाचो॒ वाचा न॑तिदाहा॒या न॑तिदाहा यो॒वाच॑ ह हो॒वाचा न॑तिदाहा॒या न॑तिदाहा यो॒वाच॑ ह । \newline
18. अन॑तिदाहा॒येत्यन॑ति - दा॒हा॒य॒ । \newline
19. उ॒वाच॑ह हो॒वाचो॒ वाच॑ हे॒य मि॒यꣳ हो॒वाचो॒ वाच॑ हे॒यम् । \newline
20. हे॒य मि॒यꣳ ह॑ हे॒य मद॒ दद॑ दि॒यꣳ ह॑ हे॒य मद॑त् । \newline
21. इ॒य मद॒ दद॑ दि॒य मि॒य मद॒ दिदि दद॑ दि॒य मि॒य मद॒दित् । \newline
22. अद॒ दिदि दद॒ दद॒ दिथ् स स इदद॒ दद॒ दिथ् सः । \newline
23. इथ् स स इदिथ् स ब्रह्म॑णा॒ ब्रह्म॑णा॒ स इदिथ् स ब्रह्म॑णा । \newline
24. स ब्रह्म॑णा॒ ब्रह्म॑णा॒ स स ब्रह्म॒णा ऽन्न॒ मन्न॒म् ब्रह्म॑णा॒ स स ब्रह्म॒णा ऽन्न᳚म् । \newline
25. ब्रह्म॒णा ऽन्न॒ मन्न॒म् ब्रह्म॑णा॒ ब्रह्म॒णा ऽन्नं॒ ॅयस्य॒ यस्यान्न॒म् ब्रह्म॑णा॒ ब्रह्म॒णा ऽन्नं॒ ॅयस्य॑ । \newline
26. अन्नं॒ ॅयस्य॒ यस्यान्न॒ मन्नं॒ ॅयस्यै॒ता ए॒ता यस्यान्न॒ मन्नं॒ ॅयस्यै॒ताः । \newline
27. यस्यै॒ता ए॒ता यस्य॒ यस्यै॒ता उ॑पधी॒यान्ता॑ उपधी॒यान्ता॑ ए॒ता यस्य॒ यस्यै॒ता उ॑पधी॒यान्तै᳚ । \newline
28. ए॒ता उ॑पधी॒यान्ता॑ उपधी॒यान्ता॑ ए॒ता ए॒ता उ॑पधी॒यान्तै॒ यो य उ॑पधी॒यान्ता॑ ए॒ता ए॒ता उ॑पधी॒यान्तै॒ यः । \newline
29. उ॒प॒धी॒यान्तै॒ यो य उ॑पधी॒यान्ता॑ उपधी॒यान्तै॒ य उ॑ वु॒ य उ॑पधी॒यान्ता॑ उपधी॒यान्तै॒ य उ॑ । \newline
30. उ॒प॒धी॒यान्ता॒ इत्यु॑प - धी॒यान्तै᳚ । \newline
31. य उ॑ वु॒ यो य उ॑ च चो॒ यो य उ॑ च । \newline
32. उ॒ च॒ च॒ वु॒ चै॒ना॒ ए॒ना॒श्च॒ वु॒ चै॒नाः॒ । \newline
33. चै॒ना॒ ए॒ना॒श्च॒ चै॒ना॒ ए॒व मे॒व मे॑नाश्च चैना ए॒वम् । \newline
34. ए॒ना॒ ए॒व मे॒व मे॑ना एना ए॒वं ॅवेद॒द् वेद॑दे॒व मे॑ना एना ए॒वं ॅवेद॑त् । \newline
35. ए॒वं ॅवेद॒द् वेद॑दे॒व मे॒वं ॅवेद॒ दितीति॒ वेद॑दे॒व मे॒वं ॅवेद॒दिति॑ । \newline
36. वेद॒ दितीति॒ वेद॒द् वेद॒ दिति॑ प्राण॒भृतः॑ प्राण॒भृत॒ इति॒ वेद॒द् वेद॒दिति॑ प्राण॒भृतः॑ । \newline
37. इति॑ प्राण॒भृतः॑ प्राण॒भृत॒ इतीति॑ प्राण॒भृत॒ उपोप॑ प्राण॒भृत॒ इतीति॑ प्राण॒भृत॒ उप॑ । \newline
38. प्रा॒ण॒भृत॒ उपोप॑ प्राण॒भृतः॑ प्राण॒भृत॒ उप॑ दधाति दधा॒ त्युप॑ प्राण॒भृतः॑ प्राण॒भृत॒ उप॑ दधाति । \newline
39. प्रा॒ण॒भृत॒ इति॑ प्राण - भृतः॑ । \newline
40. उप॑ दधाति दधा॒ त्युपोप॑ दधाति॒ रेत॑सि॒ रेत॑सि दधा॒ त्युपोप॑ दधाति॒ रेत॑सि । \newline
41. द॒धा॒ति॒ रेत॑सि॒ रेत॑सि दधाति दधाति॒ रेत॑ स्ये॒वैव रेत॑सि दधाति दधाति॒ रेत॑स्ये॒व । \newline
42. रेत॑स्ये॒वैव रेत॑सि॒ रेत॑स्ये॒व प्रा॒णान् प्रा॒णा ने॒व रेत॑सि॒ रेत॑स्ये॒व प्रा॒णान् । \newline
43. ए॒व प्रा॒णान् प्रा॒णा ने॒वैव प्रा॒णान् द॑धाति दधाति प्रा॒णा ने॒वैव प्रा॒णान् द॑धाति । \newline
44. प्रा॒णान् द॑धाति दधाति प्रा॒णान् प्रा॒णान् द॑धाति॒ तस्मा॒त् तस्मा᳚द् दधाति प्रा॒णान् प्रा॒णान् द॑धाति॒ तस्मा᳚त् । \newline
45. प्रा॒णानिति॑ प्र - अ॒नान् । \newline
46. द॒धा॒ति॒ तस्मा॒त् तस्मा᳚द् दधाति दधाति॒ तस्मा॒द् वद॒न्॒. वद॒न् तस्मा᳚द् दधाति दधाति॒ तस्मा॒द् वदन्न्॑ । \newline
47. तस्मा॒द् वद॒न्॒. वद॒न् तस्मा॒त् तस्मा॒द् वद॑न् प्रा॒णन् प्रा॒णन्. वद॒न् तस्मा॒त् तस्मा॒द् वद॑न् प्रा॒णन्न् । \newline
48. वद॑न् प्रा॒णन् प्रा॒णन्. वद॒न्॒. वद॑न् प्रा॒णन् पश्य॒न् पश्य॑न् प्रा॒णन्. वद॒न्॒. वद॑न् प्रा॒णन् पश्यन्न्॑ । \newline
49. प्रा॒णन् पश्य॒न् पश्य॑न् प्रा॒णन् प्रा॒णन् पश्य॑ञ् छृ॒ण्वञ् छृ॒ण्वन् पश्य॑न् प्रा॒णन् प्रा॒णन् पश्य॑ञ् छृ॒ण्वन्न् । \newline
50. प्रा॒णन्निति॑ प्र - अ॒नन्न् । \newline
51. पश्य॑ञ् छृ॒ण्वञ् छृ॒ण्वन् पश्य॒न् पश्य॑ञ् छृ॒ण्वन् प॒शुः प॒शुः शृ॒ण्वन् पश्य॒न् पश्य॑ञ् छृ॒ण्वन् प॒शुः । \newline
52. शृ॒ण्वन् प॒शुः प॒शुः शृ॒ण्वञ् छृ॒ण्वन् प॒शुर् जा॑यते जायते प॒शुः शृ॒ण्वञ् छृ॒ण्वन् प॒शुर् जा॑यते । \newline
53. प॒शुर् जा॑यते जायते प॒शुः प॒शुर् जा॑यते॒ ऽय म॒यम् जा॑यते प॒शुः प॒शुर् जा॑यते॒ ऽयम् । \newline
54. जा॒य॒ते॒ ऽय म॒यम् जा॑यते जायते॒ ऽयम् पु॒रः पु॒रो॑ ऽयम् जा॑यते जायते॒ ऽयम् पु॒रः । \newline
55. अ॒यम् पु॒रः पु॒रो॑ ऽय म॒यम् पु॒रो भुवो॒ भुवः॑ पु॒रो॑ ऽय म॒यम् पु॒रो भुवः॑ । \newline
56. पु॒रो भुवो॒ भुवः॑ पु॒रः पु॒रो भुव॒ इतीति॒ भुवः॑ पु॒रः पु॒रो भुव॒ इति॑ । \newline
\pagebreak
\markright{ TS 5.2.10.4  \hfill https://www.vedavms.in \hfill}

\section{ TS 5.2.10.4 }

\textbf{TS 5.2.10.4 } \newline
\textbf{Samhita Paata} \newline

भुव॒ इति॑ पु॒रस्ता॒दुप॑ दधाति प्रा॒णमे॒वैताभि॑-र्दाधारा॒ऽयं द॑क्षि॒णा वि॒श्वक॒र्मेति॑ दक्षिण॒तो मन॑ ए॒वैताभि॑र्दाधारा॒यं प॒श्चाद्-वि॒श्वव्य॑चा॒ इति॑ प॒श्चा-च्चक्षु॑रे॒वैताभि॑-र्दाधारे॒द-मु॑त्त॒राथ् सुव॒रित्यु॑त्तर॒तः श्रोत्र॑मे॒वैताभि॑-र्दाधारे॒यमु॒परि॑ म॒तिरित्यु॒परि॑ष्टा॒द्-वाच॑मे॒वैताभि॑-र्दाधार॒ दश॑द॒शोप॑ दधाति सवीर्य॒त्वाया᳚क्ष्ण॒यो - [  ] \newline

\textbf{Pada Paata} \newline

भुवः॑ । इति॑ । पु॒रस्ता᳚त् । उपेति॑ । द॒धा॒ति॒ । प्रा॒णमिति॑ प्र - अ॒नम् । ए॒व । ए॒ताभिः॑ । दा॒धा॒र॒ । अ॒यम् । द॒क्षि॒णा । वि॒श्वक॒र्मेति॑ वि॒श्व-क॒र्मा॒ । इति॑ । द॒क्षि॒ण॒तः । मनः॑ । ए॒व । ए॒ताभिः॑ । दा॒धा॒र॒ । अ॒यम् । प॒श्चात् । वि॒श्वव्य॑चा॒ इति॑ वि॒श्व - व्य॒चाः॒ । इति॑ । प॒श्चात् । चक्षुः॑ । ए॒व । ए॒ताभिः॑ । दा॒धा॒र॒ । इ॒दम् । उ॒त्त॒रादित्यु॑त् - त॒रात् । सुवः॑ । इति॑ । उ॒त्त॒र॒त इत्यु॑त् - त॒र॒तः । श्रोत्र᳚म् । ए॒व । ए॒ताभिः॑ । दा॒धा॒र॒ । इ॒यम् । उ॒परि॑ । म॒तिः । इति॑ । उ॒परि॑ष्टात् । वाच᳚म् । ए॒व । ए॒ताभिः॑ । दा॒धा॒र॒ । दश॑द॒शेति॒ दश॑-द॒श॒ । उपेति॑ । द॒धा॒ति॒ । स॒वी॒र्य॒त्वायेति॑ सवीर्य-त्वाय॑ । अ॒क्ष्ण॒या ।  \newline


\textbf{Krama Paata} \newline

भुव॒ इति॑ । इति॑ पु॒रस्ता᳚त् । पु॒रस्ता॒दुप॑ । उप॑ दधाति । द॒धा॒ति॒ प्रा॒णम् । प्रा॒णमे॒व । प्रा॒णमिति॑ प्र - अ॒नम् । ए॒वैताभिः॑ । ए॒ताभि॑र् दाधार । दा॒धा॒रा॒यम् । अ॒यम् द॑क्षि॒णा । द॒क्षि॒णा वि॒श्वक॑र्मा । वि॒श्वक॒र्मेति॑ । वि॒श्वक॒र्मेति॑ वि॒श्व - क॒र्मा॒ । इति॑ दक्षिण॒तः । द॒क्षि॒ण॒तो मनः॑ । मन॑ ए॒व । ए॒वैताभिः॑ । ए॒ताभि॑र् दाधार । दा॒धा॒रा॒यम् । अ॒यम् प॒श्चात् । प॒श्चाद् वि॒श्वव्य॑चाः । वि॒श्वव्य॑चा॒ इति॑ । वि॒श्वव्य॑चा॒ इति॑ वि॒श्व - व्य॒चाः॒ । इति॑ प॒श्चात् । प॒श्चाच् चक्षुः॑ । चक्षु॑रे॒व । ए॒वैताभिः॑ । ए॒ताभि॑र् दाधार । दा॒धा॒रे॒दम् । इ॒दमु॑त्त॒रात् । उ॒त्त॒राथ् सुवः॑ । उ॒त्त॒रादित्यु॑त् - त॒रात् । सुव॒रिति॑ । इत्यु॑त्तर॒तः । उ॒त्त॒र॒तः श्रोत्र᳚म् । उ॒त्त॒र॒त इत्यु॑त् - त॒र॒तः । श्रोत्र॑मे॒व । ए॒वैताभिः॑ । ए॒ताभि॑र् दाधार । दा॒धा॒रे॒यम् । इ॒यमु॒परि॑ । उ॒परि॑ म॒तिः । म॒तिरिति॑ । इत्यु॒परि॑ष्टात् । उ॒परि॑ष्टा॒द् वाच᳚म् । वाच॑मे॒व । ए॒वैताभिः॑ । ए॒ताभि॑र् दाधार । दा॒धा॒र॒ दश॑दश । दश॑द॒शोप॑ । दश॑द॒शेति॒ दश॑ - द॒श॒ । 
उप॑ दधाति । द॒धा॒ति॒ स॒वी॒र्य॒त्वाय॑ । स॒वी॒र्य॒त्वाया᳚क्ष्ण॒या । स॒वी॒र्य॒त्वायेति॑ सवीर्य - त्वाय॑ । अ॒क्ष्ण॒योप॑ \newline

\textbf{Jatai Paata} \newline

1. भुव॒ इतीति॒ भुवो॒ भुव॒ इति॑ । \newline
2. इति॑ पु॒रस्ता᳚त् पु॒रस्ता॒ दितीति॑ पु॒रस्ता᳚त् । \newline
3. पु॒रस्ता॒ दुपोप॑ पु॒रस्ता᳚त् पु॒रस्ता॒ दुप॑ । \newline
4. उप॑ दधाति दधा॒ त्युपोप॑ दधाति । \newline
5. द॒धा॒ति॒ प्रा॒णम् प्रा॒णम् द॑धाति दधाति प्रा॒णम् । \newline
6. प्रा॒ण मे॒वैव प्रा॒णम् प्रा॒ण मे॒व । \newline
7. प्रा॒णमिति॑ प्र - अ॒नम् । \newline
8. ए॒वै ताभि॑ रे॒ताभि॑ रे॒वै वैताभिः॑ । \newline
9. ए॒ताभि॑र् दाधार दाधारै॒ ताभि॑ रे॒ताभि॑र् दाधार । \newline
10. दा॒धा॒ रा॒य म॒यम् दा॑धार दाधा रा॒यम् । \newline
11. अ॒यम् द॑क्षि॒णा द॑क्षि॒णा ऽय म॒यम् द॑क्षि॒णा । \newline
12. द॒क्षि॒णा वि॒श्वक॑र्मा वि॒श्वक॑र्मा दक्षि॒णा द॑क्षि॒णा वि॒श्वक॑र्मा । \newline
13. वि॒श्वक॒र् मेतीति॑ वि॒श्वक॑र्मा वि॒श्वक॒र् मेति॑ । \newline
14. वि॒श्वक॒र्मेति॑ वि॒श्व - क॒र्मा॒ । \newline
15. इति॑ दक्षिण॒तो द॑क्षिण॒त इतीति॑ दक्षिण॒तः । \newline
16. द॒क्षि॒ण॒तो मनो॒ मनो॑ दक्षिण॒तो द॑क्षिण॒तो मनः॑ । \newline
17. मन॑ ए॒वैव मनो॒ मन॑ ए॒व । \newline
18. ए॒वैताभि॑ रे॒ताभि॑ रे॒वै वैताभिः॑ । \newline
19. ए॒ताभि॑र् दाधार दाधा रै॒ताभि॑ रे॒ताभि॑र् दाधार । \newline
20. दा॒धा॒ रा॒य म॒यम् दा॑धार दाधा रा॒यम् । \newline
21. अ॒यम् प॒श्चात् प॒श्चा द॒य म॒यम् प॒श्चात् । \newline
22. प॒श्चाद् वि॒श्वव्य॑चा वि॒श्वव्य॑चाः प॒श्चात् प॒श्चाद् वि॒श्वव्य॑चाः । \newline
23. वि॒श्वव्य॑चा॒ इतीति॑ वि॒श्वव्य॑चा वि॒श्वव्य॑चा॒ इति॑ । \newline
24. वि॒श्वव्य॑चा॒ इति॑ वि॒श्व - व्य॒चाः॒ । \newline
25. इति॑ प॒श्चात् प॒श्चा दितीति॑ प॒श्चात् । \newline
26. प॒श्चाच् चक्षु॒ श्चक्षुः॑ प॒श्चात् प॒श्चाच् चक्षुः॑ । \newline
27. चक्षु॑ रे॒वैव चक्षु॒ श्चक्षु॑ रे॒व । \newline
28. ए॒वै ताभि॑ रे॒ताभि॑ रे॒वै वैताभिः॑ । \newline
29. ए॒ताभि॑र् दाधार दाधारै॒ ताभि॑ रे॒ताभि॑र् दाधार । \newline
30. दा॒धा॒रे॒द मि॒दम् दा॑धार दाधारे॒दम् । \newline
31. इ॒द मु॑त्त॒रा दु॑त्त॒रा दि॒द मि॒द मु॑त्त॒रात् । \newline
32. उ॒त्त॒राथ् सुवः॒ सुव॑ रुत्त॒रा दु॑त्त॒राथ् सुवः॑ । \newline
33. उ॒त्त॒रादित्यु॑त् - त॒रात् । \newline
34. सुव॒ रितीति॒ सुवः॒ सुव॒ रिति॑ । \newline
35. इत्यु॑त्तर॒त उ॑त्तर॒त इती त्यु॑त्तर॒तः । \newline
36. उ॒त्त॒र॒तः श्रोत्रꣳ॒॒ श्रोत्र॑ मुत्तर॒त उ॑त्तर॒तः श्रोत्र᳚म् । \newline
37. उ॒त्त॒र॒त इत्यु॑त् - त॒र॒तः । \newline
38. श्रोत्र॑ मे॒वैव श्रोत्रꣳ॒॒ श्रोत्र॑ मे॒व । \newline
39. ए॒वै ताभि॑ रे॒ताभि॑ रे॒वै वैताभिः॑ । \newline
40. ए॒ताभि॑र् दाधार दाधारै॒ ताभि॑ रे॒ताभि॑र् दाधार । \newline
41. दा॒धा॒रे॒य मि॒यम् दा॑धार दाधारे॒यम् । \newline
42. इ॒य मु॒पर्यु॒ परी॒य मि॒य मु॒परि॑ । \newline
43. उ॒परि॑ म॒तिर् म॒ति रु॒पर्यु॒ परि॑ म॒तिः । \newline
44. म॒ति रितीति॑ म॒तिर् म॒ति रिति॑ । \newline
45. इत्यु॒परि॑ष्टा दु॒परि॑ष्टा॒ दिती त्यु॒परि॑ष्टात् । \newline
46. उ॒परि॑ष्टा॒द् वाचं॒ ॅवाच॑ मु॒परि॑ष्टा दु॒परि॑ष्टा॒द् वाच᳚म् । \newline
47. वाच॑ मे॒वैव वाचं॒ ॅवाच॑ मे॒व । \newline
48. ए॒वै ताभि॑ रे॒ताभि॑ रे॒वै वैताभिः॑ । \newline
49. ए॒ताभि॑र् दाधार दाधारै॒ ताभि॑ रे॒ताभि॑र् दाधार । \newline
50. दा॒धा॒र॒ दश॑दश॒ दश॑दश दाधार दाधार॒ दश॑दश । \newline
51. दश॑द॒शो पोप॒ दश॑दश॒ दश॑द॒ शोप॑ । \newline
52. दश॑द॒शेति॒ दश॑ - द॒श॒ । \newline
53. उप॑ दधाति दधा॒ त्युपोप॑ दधाति । \newline
54. द॒धा॒ति॒ स॒वी॒र्य॒त्वाय॑ सवीर्य॒त्वाय॑ दधाति दधाति सवीर्य॒त्वाय॑ । \newline
55. स॒वी॒र्य॒त्वाया᳚ क्ष्ण॒या ऽक्ष्ण॒या स॑वीर्य॒त्वाय॑ सवीर्य॒त्वाया᳚ क्ष्ण॒या । \newline
56. स॒वी॒र्य॒त्वायेति॑ सवीर्य - त्वाय॑ । \newline
57. अ॒क्ष्ण॒यो पोपा᳚ क्ष्ण॒या ऽक्ष्ण॒ योप॑ । \newline

\textbf{Ghana Paata } \newline

1. भुव॒ इतीति॒ भुवो॒ भुव॒ इति॑ पु॒रस्ता᳚त् पु॒रस्ता॒ दिति॒ भुवो॒ भुव॒ इति॑ पु॒रस्ता᳚त् । \newline
2. इति॑ पु॒रस्ता᳚त् पु॒रस्ता॒ दितीति॑ पु॒रस्ता॒ दुपोप॑ पु॒रस्ता॒ दितीति॑ पु॒रस्ता॒ दुप॑ । \newline
3. पु॒रस्ता॒ दुपोप॑ पु॒रस्ता᳚त् पु॒रस्ता॒ दुप॑ दधाति दधा॒ त्युप॑ पु॒रस्ता᳚त् पु॒रस्ता॒ दुप॑ दधाति । \newline
4. उप॑ दधाति दधा॒ त्युपोप॑ दधाति प्रा॒णम् प्रा॒णम् द॑धा॒ त्युपोप॑ दधाति प्रा॒णम् । \newline
5. द॒धा॒ति॒ प्रा॒णम् प्रा॒णम् द॑धाति दधाति प्रा॒ण मे॒वैव प्रा॒णम् द॑धाति दधाति प्रा॒ण मे॒व । \newline
6. प्रा॒ण मे॒वैव प्रा॒णम् प्रा॒ण मे॒वैताभि॑ रे॒ताभि॑ रे॒व प्रा॒णम् प्रा॒ण मे॒वैताभिः॑ । \newline
7. प्रा॒णमिति॑ प्र - अ॒नम् । \newline
8. ए॒वैताभि॑ रे॒ताभि॑ रे॒वैवैताभि॑र् दाधार दाधा रै॒ताभि॑ रे॒वै वैताभि॑र् दाधार । \newline
9. ए॒ताभि॑र् दाधार दाधा रै॒ताभि॑ रे॒ताभि॑र् दाधारा॒य म॒यम् दा॑धा रै॒ताभि॑ रे॒ताभि॑र् दाधारा॒यम् । \newline
10. दा॒धा॒रा॒य म॒यम् दा॑धार दाधारा॒यम् द॑क्षि॒णा द॑क्षि॒णा ऽयम् दा॑धार दाधारा॒यम् द॑क्षि॒णा । \newline
11. अ॒यम् द॑क्षि॒णा द॑क्षि॒णा ऽय म॒यम् द॑क्षि॒णा वि॒श्वक॑र्मा वि॒श्वक॑र्मा दक्षि॒णा ऽय म॒यम् द॑क्षि॒णा वि॒श्वक॑र्मा । \newline
12. द॒क्षि॒णा वि॒श्वक॑र्मा वि॒श्वक॑र्मा दक्षि॒णा द॑क्षि॒णा वि॒श्वक॒र्मेतीति॑ वि॒श्वक॑र्मा दक्षि॒णा द॑क्षि॒णा वि॒श्वक॒र्मेति॑ । \newline
13. वि॒श्वक॒र्मेतीति॑ वि॒श्वक॑र्मा वि॒श्वक॒र्मेति॑ दक्षिण॒तो द॑क्षिण॒त इति॑ वि॒श्वक॑र्मा वि॒श्वक॒र्मेति॑ दक्षिण॒तः । \newline
14. वि॒श्वक॒र्मेति॑ वि॒श्व - क॒र्मा॒ । \newline
15. इति॑ दक्षिण॒तो द॑क्षिण॒त इतीति॑ दक्षिण॒तो मनो॒ मनो॑ दक्षिण॒त इतीति॑ दक्षिण॒तो मनः॑ । \newline
16. द॒क्षि॒ण॒तो मनो॒ मनो॑ दक्षिण॒तो द॑क्षिण॒तो मन॑ ए॒वैव मनो॑ दक्षिण॒तो द॑क्षिण॒तो मन॑ ए॒व । \newline
17. मन॑ ए॒वैव मनो॒ मन॑ ए॒वैताभि॑ रे॒ताभि॑ रे॒व मनो॒ मन॑ ए॒वैताभिः॑ । \newline
18. ए॒वैताभि॑ रे॒ताभि॑ रे॒वै वैताभि॑र् दाधार दाधा रै॒ताभि॑ रे॒वै वैताभि॑र् दाधार । \newline
19. ए॒ताभि॑र् दाधार दाधा रै॒ताभि॑ रे॒ताभि॑र् दाधारा॒य म॒यम् दा॑धा रै॒ताभि॑ रे॒ताभि॑र् दाधारा॒यम् । \newline
20. दा॒धा॒रा॒य म॒यम् दा॑धार दाधारा॒यम् प॒श्चात् प॒श्चा द॒यम् दा॑धार दाधारा॒यम् प॒श्चात् । \newline
21. अ॒यम् प॒श्चात् प॒श्चाद॒य म॒यम् प॒श्चाद् वि॒श्वव्य॑चा वि॒श्वव्य॑चाः प॒श्चाद॒य म॒यम् प॒श्चाद् वि॒श्वव्य॑चाः । \newline
22. प॒श्चाद् वि॒श्वव्य॑चा वि॒श्वव्य॑चाः प॒श्चात् प॒श्चाद् वि॒श्वव्य॑चा॒ इतीति॑ वि॒श्वव्य॑चाः प॒श्चात् प॒श्चाद् वि॒श्वव्य॑चा॒ इति॑ । \newline
23. वि॒श्वव्य॑चा॒ इतीति॑ वि॒श्वव्य॑चा वि॒श्वव्य॑चा॒ इति॑ प॒श्चात् प॒श्चादिति॑ वि॒श्वव्य॑चा वि॒श्वव्य॑चा॒ इति॑ प॒श्चात् । \newline
24. वि॒श्वव्य॑चा॒ इति॑ वि॒श्व - व्य॒चाः॒ । \newline
25. इति॑ प॒श्चात् प॒श्चा दितीति॑ प॒श्चाच् चक्षु॒ श्चक्षुः॑ प॒श्चा दितीति॑ प॒श्चाच् चक्षुः॑ । \newline
26. प॒श्चाच् चक्षु॒ श्चक्षुः॑ प॒श्चात् प॒श्चाच् चक्षु॑ रे॒वैव चक्षुः॑ प॒श्चात् प॒श्चाच् चक्षु॑ रे॒व । \newline
27. चक्षु॑ रे॒वैव चक्षु॒ श्चक्षु॑ रे॒वैताभि॑ रे॒ताभि॑ रे॒व चक्षु॒ श्चक्षु॑ रे॒वैताभिः॑ । \newline
28. ए॒वैताभि॑ रे॒ताभि॑ रे॒वै वैताभि॑र् दाधार दाधा रै॒ताभि॑ रे॒वै वैताभि॑र् दाधार । \newline
29. ए॒ताभि॑र् दाधार दाधा रै॒ताभि॑ रे॒ताभि॑र् दाधारे॒द मि॒दम् दा॑धा रै॒ताभि॑ रे॒ताभि॑र् दाधारे॒दम् । \newline
30. दा॒धा॒रे॒द मि॒दम् दा॑धार दाधारे॒द मु॑त्त॒रा दु॑त्त॒रा दि॒दम् दा॑धार दाधारे॒द मु॑त्त॒रात् । \newline
31. इ॒द मु॑त्त॒रा दु॑त्त॒रा दि॒द मि॒द मु॑त्त॒राथ् सुवः॒ सुव॑ रुत्त॒रा दि॒द मि॒द मु॑त्त॒राथ् सुवः॑ । \newline
32. उ॒त्त॒राथ् सुवः॒ सुव॑ रुत्त॒रा दु॑त्त॒राथ् सुव॒ रितीति॒ सुव॑ रुत्त॒रा दु॑त्त॒राथ् सुव॒ रिति॑ । \newline
33. उ॒त्त॒रादित्यु॑त् - त॒रात् । \newline
34. सुव॒ रितीति॒ सुवः॒ सुव॒ रित्यु॑त्तर॒त उ॑त्तर॒त इति॒ सुवः॒ सुव॒ रित्यु॑त्तर॒तः । \newline
35. इत्यु॑त्तर॒त उ॑त्तर॒त इतीत्यु॑त्तर॒तः श्रोत्रꣳ॒॒ श्रोत्र॑ मुत्तर॒त इतीत्यु॑त्तर॒तः श्रोत्र᳚म् । \newline
36. उ॒त्त॒र॒तः श्रोत्रꣳ॒॒ श्रोत्र॑ मुत्तर॒त उ॑त्तर॒तः श्रोत्र॑ मे॒वैव श्रोत्र॑ मुत्तर॒त उ॑त्तर॒तः श्रोत्र॑ मे॒व । \newline
37. उ॒त्त॒र॒त इत्यु॑त् - त॒र॒तः । \newline
38. श्रोत्र॑ मे॒वैव श्रोत्रꣳ॒॒ श्रोत्र॑ मे॒वैताभि॑ रे॒ताभि॑ रे॒व श्रोत्रꣳ॒॒ श्रोत्र॑ मे॒वैताभिः॑ । \newline
39. ए॒वैताभि॑ रे॒ताभि॑ रे॒वै वैताभि॑र् दाधार दाधा रै॒ताभि॑ रे॒वै वैताभि॑र् दाधार । \newline
40. ए॒ताभि॑र् दाधार दाधा रै॒ताभि॑ रे॒ताभि॑र् दाधारे॒य मि॒यम् दा॑धा रै॒ताभि॑ रे॒ताभि॑र् दाधारे॒यम् । \newline
41. दा॒धा॒रे॒य मि॒यम् दा॑धार दाधारे॒य मु॒पर्यु॒परी॒यम् दा॑धार दाधारे॒य मु॒परि॑ । \newline
42. इ॒य मु॒पर्यु॒प री॒य मि॒य मु॒परि॑ म॒तिर् म॒ति रु॒परी॒य मि॒य मु॒परि॑ म॒तिः । \newline
43. उ॒परि॑ म॒तिर् म॒ति रु॒पर्यु॒परि॑ म॒तिरितीति॑ म॒ति रु॒पर्यु॒परि॑ म॒तिरिति॑ । \newline
44. म॒तिरितीति॑ म॒तिर् म॒तिरि त्यु॒परि॑ष्टा दु॒परि॑ष्टा॒ दिति॑ म॒तिर् म॒तिरि त्यु॒परि॑ष्टात् । \newline
45. इत्यु॒परि॑ष्टा दु॒परि॑ष्टा॒ दिती त्यु॒परि॑ष्टा॒द् वाचं॒ ॅवाच॑ मु॒परि॑ष्टा॒ दिती त्यु॒परि॑ष्टा॒द् वाच᳚म् । \newline
46. उ॒परि॑ष्टा॒द् वाचं॒ ॅवाच॑ मु॒परि॑ष्टा दु॒परि॑ष्टा॒द् वाच॑ मे॒वैव वाच॑ मु॒परि॑ष्टा दु॒परि॑ष्टा॒द् वाच॑ मे॒व । \newline
47. वाच॑ मे॒वैव वाचं॒ ॅवाच॑ मे॒वैताभि॑ रे॒ताभि॑ रे॒व वाचं॒ ॅवाच॑ मे॒वैताभिः॑ । \newline
48. ए॒वैताभि॑ रे॒ताभि॑ रे॒वैवैताभि॑र् दाधार दाधा रै॒ताभि॑ रे॒वै वैताभि॑र् दाधार । \newline
49. ए॒ताभि॑र् दाधार दाधा रै॒ताभि॑ रे॒ताभि॑र् दाधार॒ दश॑दश॒ दश॑दश दाधा रै॒ताभि॑ रे॒ताभि॑र् दाधार॒ दश॑दश । \newline
50. दा॒धा॒र॒ दश॑दश॒ दश॑दश दाधार दाधार॒ दश॑द॒शोपोप॒ दश॑दश दाधार दाधार॒ दश॑द॒शोप॑ । \newline
51. दश॑द॒शो पोप॒ दश॑दश॒ दश॑द॒शोप॑ दधाति दधा॒ त्युप॒ दश॑दश॒ दश॑द॒शोप॑ दधाति । \newline
52. दश॑द॒शेति॒ दश॑ - द॒श॒ । \newline
53. उप॑ दधाति दधा॒ त्युपोप॑ दधाति सवीर्य॒त्वाय॑ सवीर्य॒त्वाय॑ दधा॒ त्युपोप॑ दधाति सवीर्य॒त्वाय॑ । \newline
54. द॒धा॒ति॒ स॒वी॒र्य॒त्वाय॑ सवीर्य॒त्वाय॑ दधाति दधाति सवीर्य॒त्वाया᳚ क्ष्ण॒या ऽक्ष्ण॒या स॑वीर्य॒त्वाय॑ दधाति दधाति सवीर्य॒त्वाया᳚ क्ष्ण॒या । \newline
55. स॒वी॒र्य॒त्वाया᳚ क्ष्ण॒या ऽक्ष्ण॒या स॑वीर्य॒त्वाय॑ सवीर्य॒त्वाया᳚ क्ष्ण॒योपोपा᳚ क्ष्ण॒या स॑वीर्य॒त्वाय॑ सवीर्य॒त्वाया᳚ क्ष्ण॒योप॑ । \newline
56. स॒वी॒र्य॒त्वायेति॑ सवीर्य - त्वाय॑ । \newline
57. अ॒क्ष्ण॒योपोपा᳚ क्ष्ण॒या ऽक्ष्ण॒योप॑ दधाति दधा॒ त्युपा᳚क्ष्ण॒या ऽक्ष्ण॒योप॑ दधाति । \newline
\pagebreak
\markright{ TS 5.2.10.5  \hfill https://www.vedavms.in \hfill}

\section{ TS 5.2.10.5 }

\textbf{TS 5.2.10.5 } \newline
\textbf{Samhita Paata} \newline

-प॑ दधाति॒ तस्मा॑दक्ष्ण॒या प॒शवोऽङ्गा॑नि॒ प्रह॑रन्ति॒ प्रति॑ष्ठित्यै॒ याः प्राची॒स्ताभि॒-र्वसि॑ष्ठ आर्द्ध्नो॒द्या द॑क्षि॒णा ताभि॑र्भ॒रद्वा॑जो॒ याः प्र॒तीची॒स्ताभि॑ र्वि॒श्वामि॑त्रो॒ या उदी॑ची॒स्ताभि॑-र्ज॒मद॑ग्नि॒र्या ऊ॒र्द्ध्वास्ताभि॑-र्वि॒श्वक॑र्मा॒ य ए॒वमे॒तासा॒मृद्धिं॒ ॅवेद॒र्द्ध्नोत्ये॒व य आ॑सामे॒वं ब॒न्धुतां॒ ॅवेद॒ बन्धु॑मान् भवति॒ य आ॑सामे॒वं क्लृप्तिं॒ ॅवेद॒ कल्प॑ते - [  ] \newline

\textbf{Pada Paata} \newline

उपेति॑ । द॒धा॒ति॒ । तस्मा᳚त् । अ॒क्ष्ण॒या । प॒शवः॑ । अङ्गा॑नि । प्रेति॑ । ह॒र॒न्ति॒ । प्रति॑ष्ठित्या॒ इति॒ प्रति॑ - स्थि॒त्यै॒ । याः । प्राचीः᳚ । ताभिः॑ । वसि॑ष्ठः । आ॒द्‌र्ध्नो॒त् । याः । द॒क्षि॒णा । ताभिः॑ । भ॒रद्वा॑जः । याः । प्र॒तीचीः᳚ । ताभिः॑ । वि॒श्वामि॑त्र॒ इति॑ वि॒श्व - मि॒त्रः॒ । याः । उदी॑चीः । ताभिः॑ । ज॒मद॑ग्निः । याः । ऊ॒द्‌र्ध्वाः । ताभिः॑ । वि॒श्वक॒र्मेति॑ वि॒श्व - क॒र्मा॒ । यः । ए॒वम् । ए॒तासा᳚म् । ऋद्धि᳚म् । वेद॑ । ऋ॒द्ध्नोति॑ । ए॒व । यः । आ॒सा॒म् । ए॒वम् । ब॒न्धुता᳚म् । वेद॑ । बन्धु॑मा॒निति॒ बन्धु॑ - मा॒न् । भ॒व॒ति॒ । यः । आ॒सा॒म् । ए॒वम् । क्लृप्ति᳚म् । वेद॑ । कल्प॑ते ।  \newline


\textbf{Krama Paata} \newline

उप॑ दधाति । द॒धा॒ति॒ तस्मा᳚त् । तस्मा॑दक्ष्ण॒या । अ॒क्ष्ण॒या प॒शवः॑ । प॒शवोऽङ्गा॑नि । अङ्गा॑नि॒ प्र । प्र ह॑रन्ति । ह॒र॒न्ति॒ प्रति॑ष्ठित्यै । प्रति॑ष्ठित्यै॒ याः । प्रति॑ष्ठित्या॒ इति॒ प्रति॑ - स्थि॒त्यै॒ । याः प्राचीः᳚ । प्राची॒स्ताभिः॑ । ताभि॒र् वसि॑ष्ठः । वसि॑ष्ठ आर्द्ध्नोत् । आ॒र्द्ध्नो॒द् याः । या द॑क्षि॒णा । द॒क्षि॒णा ताभिः॑ । 
ताभि॑र् भ॒रद्वा॑जः । भ॒रद्वा॑जो॒ याः । याः प्र॒तीचीः᳚ । प्र॒तीची॒स्ताभिः॑ । ताभि॑र् वि॒श्वामि॑त्रः । वि॒श्वामि॑त्रो॒ याः । वि॒श्वामि॑त्र॒ इति॑ वि॒श्व - मि॒त्रः॒ । या उदी॑चीः । उदी॑ची॒स्ताभिः॑ । ताभि॑र् ज॒मद॑ग्निः । ज॒मद॑ग्नि॒र् याः । या ऊ॒र्द्ध्वाः । ऊ॒र्द्ध्वास्ताभिः॑ । ताभि॑र् वि॒श्वक॑र्मा । वि॒श्वक॑र्मा॒ यः । वि॒श्वक॒र्मेति॑ वि॒श्व - क॒र्मा॒ । य ए॒वम् । ए॒वमे॒तासा᳚म् । ए॒तासा॒मृद्धि᳚म् । ऋद्धि॒म् ॅवेद॑ । वेद॒र्.द्ध्नोति॑ । ऋ॒द्ध्नोत्ये॒व । ए॒व यः । य आ॑साम् । आ॒सा॒मे॒वम् । ए॒वम् ब॒न्धुता᳚म् । ब॒न्धुता॒म् ॅवेद॑ । वेद॒ बन्धु॑मान् । बन्धु॑मान् भवति । बन्धु॑मा॒निति॒ बन्धु॑ - मा॒न्॒ । भ॒व॒ति॒ यः । य आ॑साम् । आ॒सा॒मे॒वम् । ए॒वम् क्लृप्ति᳚म् । क्लृप्ति॒म् ॅवेद॑ । वेद॒ कल्प॑ते । कल्प॑तेऽस्मै \newline

\textbf{Jatai Paata} \newline

1. उप॑ दधाति दधा॒ त्युपोप॑ दधाति । \newline
2. द॒धा॒ति॒ तस्मा॒त् तस्मा᳚द् दधाति दधाति॒ तस्मा᳚त् । \newline
3. तस्मा॑ दक्ष्ण॒या ऽक्ष्ण॒या तस्मा॒त् तस्मा॑ दक्ष्ण॒या । \newline
4. अ॒क्ष्ण॒या प॒शवः॑ प॒शवो᳚ ऽक्ष्ण॒या ऽक्ष्ण॒या प॒शवः॑ । \newline
5. प॒शवो ऽङ्गा॒ न्यङ्गा॑नि प॒शवः॑ प॒शवो ऽङ्गा॑नि । \newline
6. अङ्गा॑नि॒ प्र प्राङ्गा॒ न्यङ्गा॑नि॒ प्र । \newline
7. प्र ह॑रन्ति हरन्ति॒ प्र प्र ह॑रन्ति । \newline
8. ह॒र॒न्ति॒ प्रति॑ष्ठित्यै॒ प्रति॑ष्ठित्यै हरन्ति हरन्ति॒ प्रति॑ष्ठित्यै । \newline
9. प्रति॑ष्ठित्यै॒ या याः प्रति॑ष्ठित्यै॒ प्रति॑ष्ठित्यै॒ याः । \newline
10. प्रति॑ष्ठित्या॒ इति॒ प्रति॑ - स्थि॒त्यै॒ । \newline
11. याः प्राचीः॒ प्राची॒र् या याः प्राचीः᳚ । \newline
12. प्राची॒ स्ताभि॒ स्ताभिः॒ प्राचीः॒ प्राची॒ स्ताभिः॑ । \newline
13. ताभि॒र् वसि॑ष्ठो॒ वसि॑ष्ठ॒ स्ताभि॒ स्ताभि॒र् वसि॑ष्ठः । \newline
14. वसि॑ष्ठ आर्द्ध्नो दार्द्ध्नो॒द् वसि॑ष्ठो॒ वसि॑ष्ठ आर्द्ध्नोत् । \newline
15. आ॒र्द्ध्नो॒द् या या आ᳚र्द्ध्नो दार्द्ध्नो॒द् याः । \newline
16. या द॑क्षि॒णा द॑क्षि॒णा या या द॑क्षि॒णा । \newline
17. द॒क्षि॒णा ताभि॒ स्ताभि॑र् दक्षि॒णा द॑क्षि॒णा ताभिः॑ । \newline
18. ताभि॑र् भ॒रद्वा॑जो भ॒रद्वा॑ज॒ स्ताभि॒ स्ताभि॑र् भ॒रद्वा॑जः । \newline
19. भ॒रद्वा॑जो॒ या या भ॒रद्वा॑जो भ॒रद्वा॑जो॒ याः । \newline
20. याः प्र॒तीचीः᳚ प्र॒तीची॒र् या याः प्र॒तीचीः᳚ । \newline
21. प्र॒तीची॒ स्ताभि॒ स्ताभिः॑ प्र॒तीचीः᳚ प्र॒तीची॒ स्ताभिः॑ । \newline
22. ताभि॑र् वि॒श्वामि॑त्रो वि॒श्वामि॑त्र॒ स्ताभि॒ स्ताभि॑र् वि॒श्वामि॑त्रः । \newline
23. वि॒श्वामि॑त्रो॒ या या वि॒श्वामि॑त्रो वि॒श्वामि॑त्रो॒ याः । \newline
24. वि॒श्वामि॑त्र॒ इति॑ वि॒श्व - मि॒त्रः॒ । \newline
25. या उदी॑ची॒ रुदी॑ची॒र् या या उदी॑चीः । \newline
26. उदी॑ची॒ स्ताभि॒ स्ताभि॒ रुदी॑ची॒ रुदी॑ची॒ स्ताभिः॑ । \newline
27. ताभि॑र् ज॒मद॑ग्निर् ज॒मद॑ग्नि॒ स्ताभि॒ स्ताभि॑र् ज॒मद॑ग्निः । \newline
28. ज॒मद॑ग्नि॒र् या या ज॒मद॑ग्निर् ज॒मद॑ग्नि॒र् याः । \newline
29. या ऊ॒र्द्ध्वा ऊ॒र्द्ध्वा या या ऊ॒र्द्ध्वाः । \newline
30. ऊ॒र्द्ध्वा स्ताभि॒ स्ताभि॑ रू॒र्द्ध्वा ऊ॒र्द्ध्वा स्ताभिः॑ । \newline
31. ताभि॑र् वि॒श्वक॑र्मा वि॒श्वक॑र्मा॒ ताभि॒ स्ताभि॑र् वि॒श्वक॑र्मा । \newline
32. वि॒श्वक॑र्मा॒ यो यो वि॒श्वक॑र्मा वि॒श्वक॑र्मा॒ यः । \newline
33. वि॒श्वक॒र्मेति॑ वि॒श्व - क॒र्मा॒ । \newline
34. य ए॒व मे॒वं ॅयो य ए॒वम् । \newline
35. ए॒व मे॒तासा॑ मे॒तासा॑ मे॒व मे॒व मे॒तासा᳚म् । \newline
36. ए॒तासा॒ मृद्धि॒ मृद्धि॑ मे॒तासा॑ मे॒तासा॒ मृद्धि᳚म् । \newline
37. ऋद्धिं॒ ॅवेद॒ वेद र्द्धि॒ मृद्धिं॒ ॅवेद॑ । \newline
38. वेद॒ र्‌द्ध्नोत्यृ॒ द्ध्नोति॒ वेद॒ वेद॒ र्‌द्ध्नोति॑ । \newline
39. ऋ॒द्ध्नो त्ये॒वैव र्‌द्ध्नोत्यृ॒ द्ध्नो त्ये॒व । \newline
40. ए॒व यो य ए॒वैव यः । \newline
41. य आ॑सा मासां॒ ॅयो य आ॑साम् । \newline
42. आ॒सा॒ मे॒व मे॒व मा॑सा मासा मे॒वम् । \newline
43. ए॒वम् ब॒न्धुता᳚म् ब॒न्धुता॑ मे॒व मे॒वम् ब॒न्धुता᳚म् । \newline
44. ब॒न्धुतां॒ ॅवेद॒ वेद॑ ब॒न्धुता᳚म् ब॒न्धुतां॒ ॅवेद॑ । \newline
45. वेद॒ बन्धु॑मा॒न् बन्धु॑मा॒न्॒. वेद॒ वेद॒ बन्धु॑मान् । \newline
46. बन्धु॑मान् भवति भवति॒ बन्धु॑मा॒न् बन्धु॑मान् भवति । \newline
47. बन्धु॑मा॒निति॒ बन्धु॑ - मा॒न् । \newline
48. भ॒व॒ति॒ यो यो भ॑वति भवति॒ यः । \newline
49. य आ॑सा मासां॒ ॅयो य आ॑साम् । \newline
50. आ॒सा॒ मे॒व मे॒व मा॑सा मासा मे॒वम् । \newline
51. ए॒वम् क्लृप्ति॒म् क्लृप्ति॑ मे॒व मे॒वम् क्लृप्ति᳚म् । \newline
52. क्लृप्तिं॒ ॅवेद॒ वेद॒ क्लृप्ति॒म् क्लृप्तिं॒ ॅवेद॑ । \newline
53. वेद॒ कल्प॑ते॒ कल्प॑ते॒ वेद॒ वेद॒ कल्प॑ते । \newline
54. कल्प॑ते ऽस्मा अस्मै॒ कल्प॑ते॒ कल्प॑ते ऽस्मै । \newline

\textbf{Ghana Paata } \newline

1. उप॑ दधाति दधा॒ त्युपोप॑ दधाति॒ तस्मा॒त् तस्मा᳚द् दधा॒ त्युपोप॑ दधाति॒ तस्मा᳚त् । \newline
2. द॒धा॒ति॒ तस्मा॒त् तस्मा᳚द् दधाति दधाति॒ तस्मा॑ दक्ष्ण॒या ऽक्ष्ण॒या तस्मा᳚द् दधाति दधाति॒ तस्मा॑ दक्ष्ण॒या । \newline
3. तस्मा॑ दक्ष्ण॒या ऽक्ष्ण॒या तस्मा॒त् तस्मा॑ दक्ष्ण॒या प॒शवः॑ प॒शवो᳚ ऽक्ष्ण॒या तस्मा॒त् तस्मा॑ दक्ष्ण॒या प॒शवः॑ । \newline
4. अ॒क्ष्ण॒या प॒शवः॑ प॒शवो᳚ ऽक्ष्ण॒या ऽक्ष्ण॒या प॒शवो ऽङ्गा॒ न्यङ्गा॑नि प॒शवो᳚ ऽक्ष्ण॒या ऽक्ष्ण॒या प॒शवो ऽङ्गा॑नि । \newline
5. प॒शवो ऽङ्गा॒ न्यङ्गा॑नि प॒शवः॑ प॒शवो ऽङ्गा॑नि॒ प्र प्राङ्गा॑नि प॒शवः॑ प॒शवो ऽङ्गा॑नि॒ प्र । \newline
6. अङ्गा॑नि॒ प्र प्राङ्गा॒ न्यङ्गा॑नि॒ प्र ह॑रन्ति हरन्ति॒ प्राङ्गा॒ न्यङ्गा॑नि॒ प्र ह॑रन्ति । \newline
7. प्र ह॑रन्ति हरन्ति॒ प्र प्र ह॑रन्ति॒ प्रति॑ष्ठित्यै॒ प्रति॑ष्ठित्यै हरन्ति॒ प्र प्र ह॑रन्ति॒ प्रति॑ष्ठित्यै । \newline
8. ह॒र॒न्ति॒ प्रति॑ष्ठित्यै॒ प्रति॑ष्ठित्यै हरन्ति हरन्ति॒ प्रति॑ष्ठित्यै॒ या याः प्रति॑ष्ठित्यै हरन्ति हरन्ति॒ प्रति॑ष्ठित्यै॒ याः । \newline
9. प्रति॑ष्ठित्यै॒ या याः प्रति॑ष्ठित्यै॒ प्रति॑ष्ठित्यै॒ याः प्राचीः॒ प्राची॒र् याः प्रति॑ष्ठित्यै॒ प्रति॑ष्ठित्यै॒ याः प्राचीः᳚ । \newline
10. प्रति॑ष्ठित्या॒ इति॒ प्रति॑ - स्थि॒त्यै॒ । \newline
11. याः प्राचीः॒ प्राची॒र् या याः प्राची॒ स्ताभि॒ स्ताभिः॒ प्राची॒र् या याः प्राची॒ स्ताभिः॑ । \newline
12. प्राची॒ स्ताभि॒ स्ताभिः॒ प्राचीः॒ प्राची॒ स्ताभि॒र् वसि॑ष्ठो॒ वसि॑ष्ठ॒ स्ताभिः॒ प्राचीः॒ प्राची॒ स्ताभि॒र् वसि॑ष्ठः । \newline
13. ताभि॒र् वसि॑ष्ठो॒ वसि॑ष्ठ॒ स्ताभि॒ स्ताभि॒र् वसि॑ष्ठ आर्द्ध्नो दार्द्ध्नो॒द् वसि॑ष्ठ॒ स्ताभि॒ स्ताभि॒र् वसि॑ष्ठ आर्द्ध्नोत् । \newline
14. वसि॑ष्ठ आर्द्ध्नो दार्द्ध्नो॒द् वसि॑ष्ठो॒ वसि॑ष्ठ आर्द्ध्नो॒द् या या आ᳚र्द्ध्नो॒द् वसि॑ष्ठो॒ वसि॑ष्ठ आर्द्ध्नो॒द् याः । \newline
15. आ॒र्द्ध्नो॒द् या या आ᳚र्द्ध्नो दार्द्ध्नो॒द् या द॑क्षि॒णा द॑क्षि॒णा या आ᳚र्द्ध्नो दार्द्ध्नो॒द् या द॑क्षि॒णा । \newline
16. या द॑क्षि॒णा द॑क्षि॒णा या या द॑क्षि॒णा ताभि॒ स्ताभि॑र् दक्षि॒णा या या द॑क्षि॒णा ताभिः॑ । \newline
17. द॒क्षि॒णा ताभि॒ स्ताभि॑र् दक्षि॒णा द॑क्षि॒णा ताभि॑र् भ॒रद्वा॑जो भ॒रद्वा॑ज॒ स्ताभि॑र् दक्षि॒णा द॑क्षि॒णा ताभि॑र् भ॒रद्वा॑जः । \newline
18. ताभि॑र् भ॒रद्वा॑जो भ॒रद्वा॑ज॒ स्ताभि॒ स्ताभि॑र् भ॒रद्वा॑जो॒ या या भ॒रद्वा॑ज॒ स्ताभि॒ स्ताभि॑र् भ॒रद्वा॑जो॒ याः । \newline
19. भ॒रद्वा॑जो॒ या या भ॒रद्वा॑जो भ॒रद्वा॑जो॒ याः प्र॒तीचीः᳚ प्र॒तीची॒र् या भ॒रद्वा॑जो भ॒रद्वा॑जो॒ याः प्र॒तीचीः᳚ । \newline
20. याः प्र॒तीचीः᳚ प्र॒तीची॒र् या याः प्र॒तीची॒ स्ताभि॒ स्ताभिः॑ प्र॒तीची॒र् या याः प्र॒तीची॒ स्ताभिः॑ । \newline
21. प्र॒तीची॒ स्ताभि॒ स्ताभिः॑ प्र॒तीचीः᳚ प्र॒तीची॒ स्ताभि॑र् वि॒श्वामि॑त्रो वि॒श्वामि॑त्र॒ स्ताभिः॑ प्र॒तीचीः᳚ प्र॒तीची॒ स्ताभि॑र् वि॒श्वामि॑त्रः । \newline
22. ताभि॑र् वि॒श्वामि॑त्रो वि॒श्वामि॑त्र॒ स्ताभि॒ स्ताभि॑र् वि॒श्वामि॑त्रो॒ या या वि॒श्वामि॑त्र॒ स्ताभि॒ स्ताभि॑र् वि॒श्वामि॑त्रो॒ याः । \newline
23. वि॒श्वामि॑त्रो॒ या या वि॒श्वामि॑त्रो वि॒श्वामि॑त्रो॒ या उदी॑ची॒ रुदी॑ची॒र् या वि॒श्वामि॑त्रो वि॒श्वामि॑त्रो॒ या उदी॑चीः । \newline
24. वि॒श्वामि॑त्र॒ इति॑ वि॒श्व - मि॒त्रः॒ । \newline
25. या उदी॑ची॒ रुदी॑ची॒र् या या उदी॑ची॒ स्ताभि॒ स्ताभि॒ रुदी॑ची॒र् या या उदी॑ची॒ स्ताभिः॑ । \newline
26. उदी॑ची॒ स्ताभि॒ स्ताभि॒ रुदी॑ची॒ रुदी॑ची॒ स्ताभि॑र् ज॒मद॑ग्निर् ज॒मद॑ग्नि॒ स्ताभि॒ रुदी॑ची॒ रुदी॑ची॒ स्ताभि॑र् ज॒मद॑ग्निः । \newline
27. ताभि॑र् ज॒मद॑ग्निर् ज॒मद॑ग्नि॒ स्ताभि॒ स्ताभि॑र् ज॒मद॑ग्नि॒र् या या ज॒मद॑ग्नि॒ स्ताभि॒ स्ताभि॑र् ज॒मद॑ग्नि॒र् याः । \newline
28. ज॒मद॑ग्नि॒र् या या ज॒मद॑ग्निर् ज॒मद॑ग्नि॒र् या ऊ॒र्द्ध्वा ऊ॒र्द्ध्वा या ज॒मद॑ग्निर् ज॒मद॑ग्नि॒र् या ऊ॒र्द्ध्वाः । \newline
29. या ऊ॒र्द्ध्वा ऊ॒र्द्ध्वा या या ऊ॒र्द्ध्वा स्ताभि॒ स्ताभि॑ रू॒र्द्ध्वा या या ऊ॒र्द्ध्वा स्ताभिः॑ । \newline
30. ऊ॒र्द्ध्वा स्ताभि॒ स्ताभि॑ रू॒र्द्ध्वा ऊ॒र्द्ध्वा स्ताभि॑र् वि॒श्वक॑र्मा वि॒श्वक॑र्मा॒ ताभि॑ रू॒र्द्ध्वा ऊ॒र्द्ध्वा स्ताभि॑र् वि॒श्वक॑र्मा । \newline
31. ताभि॑र् वि॒श्वक॑र्मा वि॒श्वक॑र्मा॒ ताभि॒ स्ताभि॑र् वि॒श्वक॑र्मा॒ यो यो वि॒श्वक॑र्मा॒ ताभि॒ स्ताभि॑र् वि॒श्वक॑र्मा॒ यः । \newline
32. वि॒श्वक॑र्मा॒ यो यो वि॒श्वक॑र्मा वि॒श्वक॑र्मा॒ य ए॒व मे॒वं ॅयो वि॒श्वक॑र्मा वि॒श्वक॑र्मा॒ य ए॒वम् । \newline
33. वि॒श्वक॒र्मेति॑ वि॒श्व - क॒र्मा॒ । \newline
34. य ए॒व मे॒वं ॅयो य ए॒व मे॒तासा॑ मे॒तासा॑ मे॒वं ॅयो य ए॒व मे॒तासा᳚म् । \newline
35. ए॒व मे॒तासा॑ मे॒तासा॑ मे॒व मे॒व मे॒तासा॒ मृद्धि॒ मृद्धि॑ मे॒तासा॑ मे॒व मे॒व मे॒तासा॒ मृद्धि᳚म् । \newline
36. ए॒तासा॒ मृद्धि॒ मृद्धि॑ मे॒तासा॑ मे॒तासा॒ मृद्धिं॒ ॅवेद॒ वेद र्‌द्धि॑ मे॒तासा॑ मे॒तासा॒ मृद्धिं॒ ॅवेद॑ । \newline
37. ऋद्धिं॒ ॅवेद॒ वेद र्‌द्धि॒ मृद्धिं॒ ॅवेद॒ र्‌द्ध्नो त्यृ॒द्ध्नोति॒ वेद र्‌द्धि॒ मृद्धिं॒ ॅवेद॒ र्‌द्ध्नोति॑ । \newline
38. वेद॒ र्‌द्ध्नो त्यृ॒द्ध्नोति॒ वेद॒ वेद॒ र्‌द्ध्नो त्ये॒वैव र्‌द्ध्नोति॒ वेद॒ वेद॒ र्‌द्ध्नो त्ये॒व । \newline
39. ऋ॒द्ध्नो त्ये॒वैव र्‌द्ध्नो त्यृ॒द्ध्नो त्ये॒व यो य ए॒व र्द्‌ध्नो त्यृ॒द्ध्नो त्ये॒व यः । \newline
40. ए॒व यो य ए॒वैव य आ॑सा मासां॒ ॅय ए॒वैव य आ॑साम् । \newline
41. य आ॑सा मासां॒ ॅयो य आ॑सा मे॒व मे॒व मा॑सां॒ ॅयो य आ॑सा मे॒वम् । \newline
42. आ॒सा॒ मे॒व मे॒व मा॑सा मासा मे॒वम् ब॒न्धुता᳚म् ब॒न्धुता॑ मे॒व मा॑सा मासा मे॒वम् ब॒न्धुता᳚म् । \newline
43. ए॒वम् ब॒न्धुता᳚म् ब॒न्धुता॑ मे॒व मे॒वम् ब॒न्धुतां॒ ॅवेद॒ वेद॑ ब॒न्धुता॑ मे॒व मे॒वम् ब॒न्धुतां॒ ॅवेद॑ । \newline
44. ब॒न्धुतां॒ ॅवेद॒ वेद॑ ब॒न्धुता᳚म् ब॒न्धुतां॒ ॅवेद॒ बन्धु॑मा॒न् बन्धु॑मा॒न्॒. वेद॑ ब॒न्धुता᳚म् ब॒न्धुतां॒ ॅवेद॒ बन्धु॑मान् । \newline
45. वेद॒ बन्धु॑मा॒न् बन्धु॑मा॒न्॒. वेद॒ वेद॒ बन्धु॑मान् भवति भवति॒ बन्धु॑मा॒न्॒. वेद॒ वेद॒ बन्धु॑मान् भवति । \newline
46. बन्धु॑मान् भवति भवति॒ बन्धु॑मा॒न् बन्धु॑मान् भवति॒ यो यो भ॑वति॒ बन्धु॑मा॒न् बन्धु॑मान् भवति॒ यः । \newline
47. बन्धु॑मा॒निति॒ बन्धु॑ - मा॒न् । \newline
48. भ॒व॒ति॒ यो यो भ॑वति भवति॒ य आ॑सा मासां॒ ॅयो भ॑वति भवति॒ य आ॑साम् । \newline
49. य आ॑सा मासां॒ ॅयो य आ॑सा मे॒व मे॒व मा॑सां॒ ॅयो य आ॑सा मे॒वम् । \newline
50. आ॒सा॒ मे॒व मे॒व मा॑सा मासा मे॒वम् क्लृप्ति॒म् क्लृप्ति॑ मे॒व मा॑सा मासा मे॒वम् क्लृप्ति᳚म् । \newline
51. ए॒वम् क्लृप्ति॒म् क्लृप्ति॑ मे॒व मे॒वम् क्लृप्तिं॒ ॅवेद॒ वेद॒ क्लृप्ति॑ मे॒व मे॒वम् क्लृप्तिं॒ ॅवेद॑ । \newline
52. क्लृप्तिं॒ ॅवेद॒ वेद॒ क्लृप्ति॒म् क्लृप्तिं॒ ॅवेद॒ कल्प॑ते॒ कल्प॑ते॒ वेद॒ क्लृप्ति॒म् क्लृप्तिं॒ ॅवेद॒ कल्प॑ते । \newline
53. वेद॒ कल्प॑ते॒ कल्प॑ते॒ वेद॒ वेद॒ कल्प॑ते ऽस्मा अस्मै॒ कल्प॑ते॒ वेद॒ वेद॒ कल्प॑ते ऽस्मै । \newline
54. कल्प॑ते ऽस्मा अस्मै॒ कल्प॑ते॒ कल्प॑ते ऽस्मै॒ यो यो᳚ ऽस्मै॒ कल्प॑ते॒ कल्प॑ते ऽस्मै॒ यः । \newline
\pagebreak
\markright{ TS 5.2.10.6  \hfill https://www.vedavms.in \hfill}

\section{ TS 5.2.10.6 }

\textbf{TS 5.2.10.6 } \newline
\textbf{Samhita Paata} \newline

ऽस्मै॒ य आ॑सामे॒वमा॒यत॑नं॒ ॅवेदा॒ऽऽ*यत॑नवान् भवति॒ य आ॑सामे॒वं प्र॑ति॒ष्ठां ॅवेद॒ प्रत्ये॒व ति॑ष्ठति प्राण॒भृत॑ उप॒धाय॑ सं॒ॅयत॒ उप॑ दधाति प्रा॒णाने॒वा ऽस्मि॑न् धि॒त्वा सं॒ॅयद्भिः॒ संॅय॑च्छति॒ तथ् सं॒ॅयताꣳ॑ संॅय॒त्त्वमथो᳚ प्रा॒ण ए॒वापा॒नं द॑धाति॒ तस्मा᳚त् प्राणापा॒नौ सं च॑रतो॒ विषू॑ची॒रुप॑ दधाति॒ तस्मा॒द्-विष्व॑ञ्चौ प्राणापा॒नौ यद्वा अ॒ग्नेरसं॑ॅयत॒ - [  ] \newline

\textbf{Pada Paata} \newline

अ॒स्मै॒ । यः । आ॒सा॒म् । ए॒वम् । आ॒यत॑न॒मित्या᳚ - यत॑नम् । वेद॑ । आ॒यत॑नवा॒नित्या॒यत॑न - वा॒न् । भ॒व॒ति॒ । यः । आ॒सा॒म् । ए॒वम् । प्र॒ति॒ष्ठामिति॑ प्रति - स्थाम् । वेद॑ । प्रतीति॑ । ए॒व । ति॒ष्ठ॒ति॒ । प्रा॒ण॒भृत॒ इति॑ प्राण - भृतः॑ । उ॒प॒धायेत्यु॑प - धाय॑ । सं॒ॅयत॒ इति॑ सं - यतः॑ । उपेति॑ । द॒धा॒ति॒ । प्रा॒णा॒निति॑ प्र - अ॒नान् । ए॒व । अ॒स्मि॒न्न् । धि॒त्वा । सं॒ॅयद्भि॒रिति॑ सं॒ॅयत् - भिः॒ । समिति॑ । य॒च्छ॒ति॒ । तत् । सं॒ॅयता॒मिति॑ सं - यता᳚म् । सं॒ॅय॒त्त्वमिति॑ संॅयत् - त्वम् । अथो॒ इति॑ । प्रा॒ण इति॑ प्र - अ॒ने । ए॒व । अ॒पा॒नमित्य॑प - अ॒नम् । द॒धा॒ति॒ । तस्मा᳚त् । प्रा॒णा॒पा॒नाविति॑ प्राण - अ॒पा॒नौ । समिति॑ । च॒र॒तः॒ । विषू॑चीः । उपेति॑ । द॒धा॒ति॒ । तस्मा᳚त् । विष्व॑ञ्चौ । प्रा॒णा॒पा॒नाविति॑ प्राण-अ॒पा॒नौ । यत् । वै । अ॒ग्नेः । असं॑ॅयत॒मित्यसं᳚ - य॒त॒म् ।  \newline


\textbf{Krama Paata} \newline

अ॒स्मै॒ यः । य आ॑साम् । आ॒सा॒मे॒वम् । ए॒वमा॒यत॑नम् । आ॒यत॑न॒म् ॅवेद॑ । आ॒यत॑न॒मित्या᳚ - यत॑नम् । वेदा॒यत॑नवान् । आ॒यत॑नवान् भवति । आ॒यत॑नवा॒नित्या॒यत॑न - वा॒न्॒ । भ॒व॒ति॒ यः । य आ॑साम् । आ॒सा॒मे॒वम् । ए॒वम् प्र॑ति॒ष्ठाम् । प्र॒ति॒ष्ठाम् ॅवेद॑ । प्र॒ति॒ष्ठामिति॑ प्रति - स्थाम् । वेद॒ प्रति॑ । प्रत्ये॒व । ए॒व ति॑ष्ठति । ति॒ष्ठ॒ति॒ प्रा॒ण॒भृतः॑ । प्रा॒ण॒भृत॑ उप॒धाय॑ । प्रा॒ण॒भृत॒ इति॑ प्राण - भृतः॑ । उ॒प॒धाय॑ स॒म्ॅयतः॑ । उ॒प॒धायेत्यु॑प - धाय॑ । स॒म्ॅयत॒ उप॑ । स॒म्ॅयत॒ इति॑ सम् - यतः॑ । उप॑ दधाति । द॒धा॒ति॒ प्रा॒णान् । प्रा॒णाने॒व । प्रा॒णानिति॑ प्र - अ॒नान् । ए॒वास्मिन्न्॑ । अ॒स्मि॒न् धि॒त्वा । धि॒त्वा स॒म्ॅयद्भिः॑ । स॒म्ॅयद्भिः॒ सम् । स॒म्ॅयद्भि॒रिति॑ स॒म्ॅयत् - भिः॒ । सम् ॅय॑च्छति । य॒च्छ॒ति॒ तत् । तथ् स॒म्ॅयता᳚म् । स॒म्ॅयताꣳ॑ सम्ॅय॒त्वम् । स॒म्ॅयता॒मिति॑ सम् - यता᳚म् । स॒म्ॅय॒त्वमथो᳚ । स॒म्ॅय॒त्वमिति॑ सम्ॅयत् - त्वम् । अथो᳚ प्रा॒णे । अथो॒ इत्यथो᳚ । प्रा॒ण ए॒व । प्रा॒ण इति॑ प्र - अ॒ने । ए॒वापा॒नम् । अ॒पा॒नम् द॑धाति । अ॒पा॒नमित्य॑प - अ॒नम् । द॒धा॒ति॒ तस्मा᳚त् । तस्मा᳚त् प्राणापा॒नौ । प्रा॒णा॒पा॒नौ सम् । प्रा॒णा॒पा॒नाविति॑ प्राण - अ॒पा॒नौ । सम् च॑रतः । च॒र॒तो॒ विषू॑चीः । विषू॑ची॒रुप॑ । उप॑ दधाति । द॒धा॒ति॒ तस्मा᳚त् । तस्मा॒द् विष्व॑ञ्चौ । विष्व॑ञ्चौ प्राणापा॒नौ । प्रा॒णा॒पा॒नौ यत् । प्रा॒णा॒पा॒नाविति॑ प्राण - अ॒पा॒नौ । यद् वै । वा अ॒ग्नेः । अ॒ग्नेरस॑म्ॅयतम् ( ) । अस॑म्ॅयत॒मसु॑वर्ग्यम् । अस॑म्ॅयत॒मित्यस᳚म् - य॒त॒म् \newline

\textbf{Jatai Paata} \newline

1. अ॒स्मै॒ यो यो᳚ ऽस्मा अस्मै॒ यः । \newline
2. य आ॑सा मासां॒ ॅयो य आ॑साम् । \newline
3. आ॒सा॒ मे॒व मे॒व मा॑सा मासा मे॒वम् । \newline
4. ए॒व मा॒यत॑न मा॒यत॑न मे॒व मे॒व मा॒यत॑नम् । \newline
5. आ॒यत॑नं॒ ॅवेद॒ वेदा॒ यत॑न मा॒यत॑नं॒ ॅवेद॑ । \newline
6. आ॒यत॑न॒मित्या᳚ - यत॑नम् । \newline
7. वेदा॒ यत॑नवा ना॒यत॑नवा॒न्॒. वेद॒ वेदा॒ यत॑नवान् । \newline
8. आ॒यत॑नवान् भवति भवत्या॒ यत॑नवा ना॒यत॑नवान् भवति । \newline
9. आ॒यत॑नवा॒नित्या॒यत॑न - वा॒न् । \newline
10. भ॒व॒ति॒ यो यो भ॑वति भवति॒ यः । \newline
11. य आ॑सा मासां॒ ॅयो य आ॑साम् । \newline
12. आ॒सा॒ मे॒व मे॒व मा॑सा मासा मे॒वम् । \newline
13. ए॒वम् प्र॑ति॒ष्ठाम् प्र॑ति॒ष्ठा मे॒व मे॒वम् प्र॑ति॒ष्ठाम् । \newline
14. प्र॒ति॒ष्ठां ॅवेद॒ वेद॑ प्रति॒ष्ठाम् प्र॑ति॒ष्ठां ॅवेद॑ । \newline
15. प्र॒ति॒ष्ठामिति॑ प्रति - स्थाम् । \newline
16. वेद॒ प्रति॒ प्रति॒ वेद॒ वेद॒ प्रति॑ । \newline
17. प्रत्ये॒ वैव प्रति॒ प्रत्ये॒व । \newline
18. ए॒व ति॑ष्ठति तिष्ठ त्ये॒वैव ति॑ष्ठति । \newline
19. ति॒ष्ठ॒ति॒ प्रा॒ण॒भृतः॑ प्राण॒भृत॑ स्तिष्ठति तिष्ठति प्राण॒भृतः॑ । \newline
20. प्रा॒ण॒भृत॑ उप॒धायो॑ प॒धाय॑ प्राण॒भृतः॑ प्राण॒भृत॑ उप॒धाय॑ । \newline
21. प्रा॒ण॒भृत॒ इति॑ प्राण - भृतः॑ । \newline
22. उ॒प॒धाय॑ सं॒ॅयतः॑ सं॒ॅयत॑ उप॒धायो॑ प॒धाय॑ सं॒ॅयतः॑ । \newline
23. उ॒प॒धायेत्यु॑प - धाय॑ । \newline
24. सं॒ॅयत॒ उपोप॑ सं॒ॅयतः॑ सं॒ॅयत॒ उप॑ । \newline
25. सं॒ॅयत॒ इति॑ सं - यतः॑ । \newline
26. उप॑ दधाति दधा॒ त्युपोप॑ दधाति । \newline
27. द॒धा॒ति॒ प्रा॒णान् प्रा॒णान् द॑धाति दधाति प्रा॒णान् । \newline
28. प्रा॒णा ने॒वैव प्रा॒णान् प्रा॒णा ने॒व । \newline
29. प्रा॒णानिति॑ प्र - अ॒नान् । \newline
30. ए॒वास्मि॑न् नस्मिन् ने॒वै वास्मिन्न्॑ । \newline
31. अ॒स्मि॒न् धि॒त्वा धि॒त्वा ऽस्मि॑न् नस्मिन् धि॒त्वा । \newline
32. धि॒त्वा सं॒ॅयद्भिः॑ सं॒ॅयद्भि॑र् धि॒त्वा धि॒त्वा सं॒ॅयद्भिः॑ । \newline
33. सं॒ॅयद्भिः॒ सꣳ सꣳ सं॒ॅयद्भिः॑ सं॒ॅयद्भिः॒ सम् । \newline
34. सं॒ॅयद्भि॒रिति॑ सं॒ॅयत् - भिः॒ । \newline
35. सं ॅय॑च्छति यच्छति॒ सꣳ सं ॅय॑च्छति । \newline
36. य॒च्छ॒ति॒ तत् तद् य॑च्छति यच्छति॒ तत् । \newline
37. तथ् सं॒ॅयताꣳ॑ सं॒ॅयता॒म् तत् तथ् सं॒ॅयता᳚म् । \newline
38. सं॒ॅयताꣳ॑ संॅय॒त्त्वꣳ सं॑ॅय॒त्त्वꣳ सं॒ॅयताꣳ॑ सं॒ॅयताꣳ॑ संॅय॒त्त्वम् । \newline
39. सं॒ॅयता॒मिति॑ सं - यता᳚म् । \newline
40. सं॒ॅय॒त्त्व मथो॒ अथो॑ संॅय॒त्त्वꣳ सं॑ॅय॒त्त्व मथो᳚ । \newline
41. सं॒ॅय॒त्त्वमिति॑ संॅयत् - त्वम् । \newline
42. अथो᳚ प्रा॒णे प्रा॒णे ऽथो॒ अथो᳚ प्रा॒णे । \newline
43. अथो॒ इत्यथो᳚ । \newline
44. प्रा॒ण ए॒वैव प्रा॒णे प्रा॒ण ए॒व । \newline
45. प्रा॒ण इति॑ प्र - अ॒ने । \newline
46. ए॒वा पा॒न म॑पा॒न मे॒वैवा पा॒नम् । \newline
47. अ॒पा॒नम् द॑धाति दधा त्यपा॒न म॑पा॒नम् द॑धाति । \newline
48. अ॒पा॒नमित्य॑प - अ॒नम् । \newline
49. द॒धा॒ति॒ तस्मा॒त् तस्मा᳚द् दधाति दधाति॒ तस्मा᳚त् । \newline
50. तस्मा᳚त् प्राणापा॒नौ प्रा॑णापा॒नौ तस्मा॒त् तस्मा᳚त् प्राणापा॒नौ । \newline
51. प्रा॒णा॒पा॒नौ सꣳ सम् प्रा॑णापा॒नौ प्रा॑णापा॒नौ सम् । \newline
52. प्रा॒णा॒पा॒नाविति॑ प्राण - अ॒पा॒नौ । \newline
53. सम् च॑रत श्चरतः॒ सꣳ सम् च॑रतः । \newline
54. च॒र॒तो॒ विषू॑ची॒र् विषू॑ची श्चरत श्चरतो॒ विषू॑चीः । \newline
55. विषू॑ची॒ रुपोप॒ विषू॑ची॒र् विषू॑ची॒ रुप॑ । \newline
56. उप॑ दधाति दधा॒ त्युपोप॑ दधाति । \newline
57. द॒धा॒ति॒ तस्मा॒त् तस्मा᳚द् दधाति दधाति॒ तस्मा᳚त् । \newline
58. तस्मा॒द् विष्व॑ञ्चौ॒ विष्व॑ञ्चौ॒ तस्मा॒त् तस्मा॒द् विष्व॑ञ्चौ । \newline
59. विष्व॑ञ्चौ प्राणापा॒नौ प्रा॑णापा॒नौ विष्व॑ञ्चौ॒ विष्व॑ञ्चौ प्राणापा॒नौ । \newline
60. प्रा॒णा॒पा॒नौ यद् यत् प्रा॑णापा॒नौ प्रा॑णापा॒नौ यत् । \newline
61. प्रा॒णा॒पा॒नाविति॑ प्राण - अ॒पा॒नौ । \newline
62. यद् वै वै यद् यद् वै । \newline
63. वा अ॒ग्ने र॒ग्नेर् वै वा अ॒ग्नेः । \newline
64. अ॒ग्ने रसं॑ॅयत॒ मसं॑ॅयत म॒ग्ने र॒ग्ने रसं॑ॅयतम् । \newline
65. असं॑ॅयत॒ मसु॑वर्ग्य॒ मसु॑वर्ग्य॒ मसं॑ॅयत॒ मसं॑ॅयत॒ मसु॑वर्ग्यम् । \newline
66. असं॑ॅयत॒मित्यसं᳚ - य॒त॒म् । \newline

\textbf{Ghana Paata } \newline

1. अ॒स्मै॒ यो यो᳚ ऽस्मा अस्मै॒ य आ॑सा मासां॒ ॅयो᳚ ऽस्मा अस्मै॒ य आ॑साम् । \newline
2. य आ॑सा मासां॒ ॅयो य आ॑सा मे॒व मे॒व मा॑सां॒ ॅयो य आ॑सा मे॒वम् । \newline
3. आ॒सा॒ मे॒व मे॒व मा॑सा मासा मे॒व मा॒यत॑न मा॒यत॑न मे॒व मा॑सा मासा मे॒व मा॒यत॑नम् । \newline
4. ए॒व मा॒यत॑न मा॒यत॑न मे॒व मे॒व मा॒यत॑नं॒ ॅवेद॒ वेदा॒ यत॑न मे॒व मे॒व मा॒यत॑नं॒ ॅवेद॑ । \newline
5. आ॒यत॑नं॒ ॅवेद॒ वेदा॒ यत॑न मा॒यत॑नं॒ ॅवेदा॒ यत॑नवा ना॒यत॑नवा॒न्॒. वेदा॒यत॑न मा॒यत॑नं॒ ॅवेदा॒यत॑नवान् । \newline
6. आ॒यत॑न॒मित्या᳚ - यत॑नम् । \newline
7. वेदा॒यत॑नवा ना॒यत॑नवा॒न्॒. वेद॒ वेदा॒यत॑नवान् भवति भव त्या॒यत॑नवा॒न्॒. वेद॒ वेदा॒यत॑नवान् भवति । \newline
8. आ॒यत॑नवान् भवति भव त्या॒यत॑नवा ना॒यत॑नवान् भवति॒ यो यो भ॑व त्या॒यत॑नवा ना॒यत॑नवान् भवति॒ यः । \newline
9. आ॒यत॑नवा॒नित्या॒यत॑न - वा॒न् । \newline
10. भ॒व॒ति॒ यो यो भ॑वति भवति॒ य आ॑सा मासां॒ ॅयो भ॑वति भवति॒ य आ॑साम् । \newline
11. य आ॑सा मासां॒ ॅयो य आ॑सा मे॒व मे॒व मा॑सां॒ ॅयो य आ॑सा मे॒वम् । \newline
12. आ॒सा॒ मे॒व मे॒व मा॑सा मासा मे॒वम् प्र॑ति॒ष्ठाम् प्र॑ति॒ष्ठा मे॒व मा॑सा मासा मे॒वम् प्र॑ति॒ष्ठाम् । \newline
13. ए॒वम् प्र॑ति॒ष्ठाम् प्र॑ति॒ष्ठा मे॒व मे॒वम् प्र॑ति॒ष्ठां ॅवेद॒ वेद॑ प्रति॒ष्ठा मे॒व मे॒वम् प्र॑ति॒ष्ठां ॅवेद॑ । \newline
14. प्र॒ति॒ष्ठां ॅवेद॒ वेद॑ प्रति॒ष्ठाम् प्र॑ति॒ष्ठां ॅवेद॒ प्रति॒ प्रति॒ वेद॑ प्रति॒ष्ठाम् प्र॑ति॒ष्ठां ॅवेद॒ प्रति॑ । \newline
15. प्र॒ति॒ष्ठामिति॑ प्रति - स्थाम् । \newline
16. वेद॒ प्रति॒ प्रति॒ वेद॒ वेद॒ प्रत्ये॒वैव प्रति॒ वेद॒ वेद॒ प्रत्ये॒व । \newline
17. प्रत्ये॒वैव प्रति॒ प्रत्ये॒व ति॑ष्ठति तिष्ठत्ये॒व प्रति॒ प्रत्ये॒व ति॑ष्ठति । \newline
18. ए॒व ति॑ष्ठति तिष्ठ त्ये॒वैव ति॑ष्ठति प्राण॒भृतः॑ प्राण॒भृत॑ स्तिष्ठ त्ये॒वैव ति॑ष्ठति प्राण॒भृतः॑ । \newline
19. ति॒ष्ठ॒ति॒ प्रा॒ण॒भृतः॑ प्राण॒भृत॑ स्तिष्ठति तिष्ठति प्राण॒भृत॑ उप॒धा यो॑प॒धाय॑ प्राण॒भृत॑ स्तिष्ठति तिष्ठति प्राण॒भृत॑ उप॒धाय॑ । \newline
20. प्रा॒ण॒भृत॑ उप॒धा यो॑प॒धाय॑ प्राण॒भृतः॑ प्राण॒भृत॑ उप॒धाय॑ सं॒ॅयतः॑ सं॒ॅयत॑ उप॒धाय॑ प्राण॒भृतः॑ प्राण॒भृत॑ उप॒धाय॑ सं॒ॅयतः॑ । \newline
21. प्रा॒ण॒भृत॒ इति॑ प्राण - भृतः॑ । \newline
22. उ॒प॒धाय॑ सं॒ॅयतः॑ सं॒ॅयत॑ उप॒धा यो॑प॒धाय॑ सं॒ॅयत॒ उपोप॑ सं॒ॅयत॑ उप॒धा यो॑प॒धाय॑ सं॒ॅयत॒ उप॑ । \newline
23. उ॒प॒धायेत्यु॑प - धाय॑ । \newline
24. सं॒ॅयत॒ उपोप॑ सं॒ॅयतः॑ सं॒ॅयत॒ उप॑ दधाति दधा॒ त्युप॑ सं॒ॅयतः॑ सं॒ॅयत॒ उप॑ दधाति । \newline
25. सं॒ॅयत॒ इति॑ सं - यतः॑ । \newline
26. उप॑ दधाति दधा॒ त्युपोप॑ दधाति प्रा॒णान् प्रा॒णान् द॑धा॒ त्युपोप॑ दधाति प्रा॒णान् । \newline
27. द॒धा॒ति॒ प्रा॒णान् प्रा॒णान् द॑धाति दधाति प्रा॒णा ने॒वैव प्रा॒णान् द॑धाति दधाति प्रा॒णा ने॒व । \newline
28. प्रा॒णा ने॒वैव प्रा॒णान् प्रा॒णा ने॒वास्मि॑न् नस्मिन् ने॒व प्रा॒णान् प्रा॒णा ने॒वास्मिन्न्॑ । \newline
29. प्रा॒णानिति॑ प्र - अ॒नान् । \newline
30. ए॒वास्मि॑न् नस्मिन् ने॒वैवास्मि॑न् धि॒त्वा धि॒त्वा ऽस्मि॑न् ने॒वैवास्मि॑न् धि॒त्वा । \newline
31. अ॒स्मि॒न् धि॒त्वा धि॒त्वा ऽस्मि॑न् नस्मिन् धि॒त्वा सं॒ॅयद्भिः॑ सं॒ॅयद्भि॑र् धि॒त्वा ऽस्मि॑न् नस्मिन् धि॒त्वा सं॒ॅयद्भिः॑ । \newline
32. धि॒त्वा सं॒ॅयद्भिः॑ सं॒ॅयद्भि॑र् धि॒त्वा धि॒त्वा सं॒ॅयद्भिः॒ सꣳ सꣳ सं॒ॅयद्भि॑र् धि॒त्वा धि॒त्वा सं॒ॅयद्भिः॒ सम् । \newline
33. सं॒ॅयद्भिः॒ सꣳ सꣳ सं॒ॅयद्भिः॑ सं॒ॅयद्भिः॒ सं ॅय॑च्छति यच्छति॒ सꣳ सं॒ॅयद्भिः॑ सं॒ॅयद्भिः॒ सं ॅय॑च्छति । \newline
34. सं॒ॅयद्भि॒रिति॑ सं॒ॅयत् - भिः॒ । \newline
35. सं ॅय॑च्छति यच्छति॒ सꣳ सं ॅय॑च्छति॒ तत् तद् य॑च्छति॒ सꣳ सं ॅय॑च्छति॒ तत् । \newline
36. य॒च्छ॒ति॒ तत् तद् य॑च्छति यच्छति॒ तथ् सं॒ॅयताꣳ॑ सं॒ॅयता॒म् तद् य॑च्छति यच्छति॒ तथ् सं॒ॅयता᳚म् । \newline
37. तथ् सं॒ॅयताꣳ॑ सं॒ॅयता॒म् तत् तथ् सं॒ॅयताꣳ॑ संॅय॒त्त्वꣳ सं॑ॅय॒त्त्वꣳ सं॒ॅयता॒म् तत् तथ् सं॒ॅयताꣳ॑ संॅय॒त्त्वम् । \newline
38. सं॒ॅयताꣳ॑ संॅय॒त्त्वꣳ सं॑ॅय॒त्त्वꣳ सं॒ॅयताꣳ॑ सं॒ॅयताꣳ॑ संॅय॒त्त्व मथो॒ अथो॑ संॅय॒त्त्वꣳ सं॒ॅयताꣳ॑ सं॒ॅयताꣳ॑ संॅय॒त्त्व मथो᳚ । \newline
39. सं॒ॅयता॒मिति॑ सं - यता᳚म् । \newline
40. सं॒ॅय॒त्त्व मथो॒ अथो॑ संॅय॒त्त्वꣳ सं॑ॅय॒त्त्व मथो᳚ प्रा॒णे प्रा॒णे ऽथो॑ संॅय॒त्त्वꣳ सं॑ॅय॒त्त्व मथो᳚ प्रा॒णे । \newline
41. सं॒ॅय॒त्त्वमिति॑ संॅयत् - त्वम् । \newline
42. अथो᳚ प्रा॒णे प्रा॒णे ऽथो॒ अथो᳚ प्रा॒ण ए॒वैव प्रा॒णे ऽथो॒ अथो᳚ प्रा॒ण ए॒व । \newline
43. अथो॒ इत्यथो᳚ । \newline
44. प्रा॒ण ए॒वैव प्रा॒णे प्रा॒ण ए॒वापा॒न म॑पा॒न मे॒व प्रा॒णे प्रा॒ण ए॒वापा॒नम् । \newline
45. प्रा॒ण इति॑ प्र - अ॒ने । \newline
46. ए॒वापा॒न म॑पा॒न मे॒वैवापा॒नम् द॑धाति दधा त्यपा॒न मे॒वैवापा॒नम् द॑धाति । \newline
47. अ॒पा॒नम् द॑धाति दधा त्यपा॒न म॑पा॒नम् द॑धाति॒ तस्मा॒त् तस्मा᳚द् दधा त्यपा॒न म॑पा॒नम् द॑धाति॒ तस्मा᳚त् । \newline
48. अ॒पा॒नमित्य॑प - अ॒नम् । \newline
49. द॒धा॒ति॒ तस्मा॒त् तस्मा᳚द् दधाति दधाति॒ तस्मा᳚त् प्राणापा॒नौ प्रा॑णापा॒नौ तस्मा᳚द् दधाति दधाति॒ तस्मा᳚त् प्राणापा॒नौ । \newline
50. तस्मा᳚त् प्राणापा॒नौ प्रा॑णापा॒नौ तस्मा॒त् तस्मा᳚त् प्राणापा॒नौ सꣳ सम् प्रा॑णापा॒नौ तस्मा॒त् तस्मा᳚त् प्राणापा॒नौ सम् । \newline
51. प्रा॒णा॒पा॒नौ सꣳ सम् प्रा॑णापा॒नौ प्रा॑णापा॒नौ सम् च॑रत श्चरतः॒ सम् प्रा॑णापा॒नौ प्रा॑णापा॒नौ सम् च॑रतः । \newline
52. प्रा॒णा॒पा॒नाविति॑ प्राण - अ॒पा॒नौ । \newline
53. सम् च॑रत श्चरतः॒ सꣳ सम् च॑रतो॒ विषू॑ची॒र् विषू॑ची श्चरतः॒ सꣳ सम् च॑रतो॒ विषू॑चीः । \newline
54. च॒र॒तो॒ विषू॑ची॒र् विषू॑ची श्चरत श्चरतो॒ विषू॑ची॒ रुपोप॒ विषू॑ची श्चरत श्चरतो॒ विषू॑ची॒ रुप॑ । \newline
55. विषू॑ची॒ रुपोप॒ विषू॑ची॒र् विषू॑ची॒ रुप॑ दधाति दधा॒ त्युप॒ विषू॑ची॒र् विषू॑ची॒ रुप॑ दधाति । \newline
56. उप॑ दधाति दधा॒ त्युपोप॑ दधाति॒ तस्मा॒त् तस्मा᳚द् दधा॒ त्युपोप॑ दधाति॒ तस्मा᳚त् । \newline
57. द॒धा॒ति॒ तस्मा॒त् तस्मा᳚द् दधाति दधाति॒ तस्मा॒द् विष्व॑ञ्चौ॒ विष्व॑ञ्चौ॒ तस्मा᳚द् दधाति दधाति॒ तस्मा॒द् विष्व॑ञ्चौ । \newline
58. तस्मा॒द् विष्व॑ञ्चौ॒ विष्व॑ञ्चौ॒ तस्मा॒त् तस्मा॒द् विष्व॑ञ्चौ प्राणापा॒नौ प्रा॑णापा॒नौ विष्व॑ञ्चौ॒ तस्मा॒त् तस्मा॒द् विष्व॑ञ्चौ प्राणापा॒नौ । \newline
59. विष्व॑ञ्चौ प्राणापा॒नौ प्रा॑णापा॒नौ विष्व॑ञ्चौ॒ विष्व॑ञ्चौ प्राणापा॒नौ यद् यत् प्रा॑णापा॒नौ विष्व॑ञ्चौ॒ विष्व॑ञ्चौ प्राणापा॒नौ यत् । \newline
60. प्रा॒णा॒पा॒नौ यद् यत् प्रा॑णापा॒नौ प्रा॑णापा॒नौ यद् वै वै यत् प्रा॑णापा॒नौ प्रा॑णापा॒नौ यद् वै । \newline
61. प्रा॒णा॒पा॒नाविति॑ प्राण - अ॒पा॒नौ । \newline
62. यद् वै वै यद् यद् वा अ॒ग्ने र॒ग्नेर् वै यद् यद् वा अ॒ग्नेः । \newline
63. वा अ॒ग्ने र॒ग्नेर् वै वा अ॒ग्ने रसं॑ॅयत॒ मसं॑ॅयत म॒ग्नेर् वै वा अ॒ग्ने रसं॑ॅयतम् । \newline
64. अ॒ग्ने रसं॑ॅयत॒ मसं॑ॅयत म॒ग्ने र॒ग्ने रसं॑ॅयत॒ मसु॑वर्ग्य॒ मसु॑वर्ग्य॒ मसं॑ॅयत म॒ग्ने र॒ग्ने रसं॑ॅयत॒ मसु॑वर्ग्यम् । \newline
65. असं॑ॅयत॒ मसु॑वर्ग्य॒ मसु॑वर्ग्य॒ मसं॑ॅयत॒ मसं॑ॅयत॒ मसु॑वर्ग्य मस्या॒ स्या सु॑वर्ग्य॒ मसं॑ॅयत॒ मसं॑ॅयत॒ मसु॑वर्ग्य मस्य । \newline
66. असं॑ॅयत॒मित्यसं᳚ - य॒त॒म् । \newline
\pagebreak
\markright{ TS 5.2.10.7  \hfill https://www.vedavms.in \hfill}

\section{ TS 5.2.10.7 }

\textbf{TS 5.2.10.7 } \newline
\textbf{Samhita Paata} \newline

-मसु॑वर्ग्यमस्य॒ तथ् सु॑व॒र्ग्यो᳚ऽग्निर्यथ् सं॒ॅयत॑ उप॒ दधा॑ति॒ समे॒वैनं॑ ॅयच्छति उव॒र्ग्य॑मे॒वाक॒ -स्त्र्यवि॒र्वयः॑ कृ॒तमया॑ना॒मित्या॑ह॒ वयो॑भिरे॒वाया॒नव॑ रु॒न्धे ऽयै॒र्वयाꣳ॑सि स॒र्वतो॑ वायु॒मती᳚र्भवन्ति॒ तस्मा॑द॒यꣳ स॒र्वतः॑ पवते ॥ \newline

\textbf{Pada Paata} \newline

असु॑वर्ग्य॒मित्यसु॑वः - ग्य॒म् । अ॒स्य॒ । तत् । सु॒व॒र्ग्य॑ इति॑ सुवः - ग्यः॑ । अ॒ग्निः । यत् । सं॒ॅयत॒ इति॑ सं - यतः॑ । उ॒प॒दधा॒तीत्यु॑प - दधा॑ति । समिति॑ । ए॒व । ए॒न॒म् । य॒च्छ॒ति॒ । सु॒व॒र्ग्य॑मिति॑ सुवः - ग्य᳚म् । ए॒व । अ॒कः॒ । त्र्यवि॒रिति॑ त्रि - अविः॑ । वयः॑ । कृ॒तम् । अया॑नाम् । इति॑ । आ॒ह॒ । वयो॑भि॒रिति॒ वयः॑ - भिः॒ । ए॒व । अयान्॑ । अवेति॑ । रु॒न्धे॒ । अयैः᳚ । वयाꣳ॑सि । स॒र्वतः॑ । वा॒यु॒मती॒रिति॑ वायु - मतीः᳚ । भ॒व॒न्ति॒ । तस्मा᳚त् । अ॒यम् । स॒र्वतः॑ । प॒व॒ते॒ ॥  \newline


\textbf{Krama Paata} \newline

असु॑वर्ग्यमस्य । असु॑वर्ग्य॒मित्य॑सुवः - ग्य॒म् । अ॒स्य॒ तत् । तथ् सु॑व॒र्ग्यः॑ । सु॒व॒र्ग्यो᳚ऽग्निः । सु॒व॒र्ग्य॑ इति॑ सुवः - ग्यः॑ । अ॒ग्निर् यत् । यथ् स॒म्ॅयतः॑ । स॒म्ॅयत॑ उप॒दधा॑ति । स॒म्ॅयत॒ इति॑ सम् - यतः॑ । उ॒प॒दधा॑ति॒ सम् । उ॒प॒दधा॒तीत्यु॑प - दधा॑ति । समे॒व । ए॒वैन᳚म् । ए॒न॒म् ॅय॒च्छ॒ति॒ । य॒च्छ॒ति॒ सु॒व॒र्ग्य᳚म् । सु॒व॒र्ग्य॑मे॒व । सु॒व॒र्ग्य॑मिति॑ सुवः - ग्य᳚म् । ए॒वाकः॑ । अ॒क॒ स्त्र्यविः॑ । त्र्यवि॒र् वयः॑ । त्र्यवि॒रिति॑ त्रि - अविः॑ । वयः॑ कृ॒तम् । कृ॒तमया॑नाम् । अया॑ना॒मिति॑ । इत्या॑ह । आ॒ह॒ वयो॑भिः । वयो॑भिरे॒व । वयो॑भि॒रिति॒ वयः॑ - भिः॒ । ए॒वायान्॑ । अया॒नव॑ । अव॑ रुन्धे । रु॒न्धेऽयैः᳚ । अयै॒र् वयाꣳ॑सि । वयाꣳ॑सि स॒र्वतः॑ । स॒र्वतो॑ वायु॒मतीः᳚ । वा॒यु॒मती᳚र् भवन्ति । वा॒यु॒मती॒रिति॑ वायु - मतीः᳚ । भ॒व॒न्ति॒ तस्मा᳚त् । तस्मा॑द॒यम् । अ॒यꣳ स॒र्वतः॑ । स॒र्वतः॑ पवते । प॒व॒त॒ इति॑ पवते । \newline

\textbf{Jatai Paata} \newline

1. असु॑वर्ग्य मस्या॒स्या सु॑वर्ग्य॒ मसु॑वर्ग्य मस्य । \newline
2. असु॑वर्ग्य॒मित्यसु॑वः - ग्य॒म् । \newline
3. अ॒स्य॒ तत् तद॑स्या स्य॒ तत् । \newline
4. तथ् सु॑व॒र्ग्यः॑ सुव॒र्ग्य॑ स्तत् तथ् सु॑व॒र्ग्यः॑ । \newline
5. सु॒व॒र्ग्यो᳚ ऽग्नि र॒ग्निः सु॑व॒र्ग्यः॑ सुव॒र्ग्यो᳚ ऽग्निः । \newline
6. सु॒व॒र्ग्य॑ इति॑ सुवः - ग्यः॑ । \newline
7. अ॒ग्निर् यद् यद॒ग्नि र॒ग्निर् यत् । \newline
8. यथ् सं॒ॅयतः॑ सं॒ॅयतो॒ यद् यथ् सं॒ॅयतः॑ । \newline
9. सं॒ॅयत॑ उप॒दधा᳚ त्युप॒दधा॑ति सं॒ॅयतः॑ सं॒ॅयत॑ उप॒दधा॑ति । \newline
10. सं॒ॅयत॒ इति॑ सं - यतः॑ । \newline
11. उ॒प॒दधा॑ति॒ सꣳ स मु॑प॒दधा᳚ त्युप॒दधा॑ति॒ सम् । \newline
12. उ॒प॒दधा॒तीत्यु॑प - दधा॑ति । \newline
13. स मे॒वैव सꣳ स मे॒व । \newline
14. ए॒वैन॑ मेन मे॒वै वैन᳚म् । \newline
15. ए॒नं॒ ॅय॒च्छ॒ति॒ य॒च्छ॒ त्ये॒न॒ मे॒नं॒ ॅय॒च्छ॒ति॒ । \newline
16. य॒च्छ॒ति॒ सु॒व॒र्ग्यꣳ॑ सुव॒र्ग्यं॑ ॅयच्छति यच्छति सुव॒र्ग्य᳚म् । \newline
17. सु॒व॒र्ग्य॑ मे॒वैव सु॑व॒र्ग्यꣳ॑ सुव॒र्ग्य॑ मे॒व । \newline
18. सु॒व॒र्ग्य॑मिति॑ सुवः - ग्य᳚म् । \newline
19. ए॒वाक॑ रक रे॒वै वाकः॑ । \newline
20. अ॒क॒ स्त्र्यवि॒ स्त्र्यवि॑ रक रक॒ स्त्र्यविः॑ । \newline
21. त्र्यवि॒र् वयो॒ वय॒ स्त्र्यवि॒ स्त्र्यवि॒र् वयः॑ । \newline
22. त्र्यवि॒रिति॑ त्रि - अविः॑ । \newline
23. वयः॑ कृ॒तम् कृ॒तं ॅवयो॒ वयः॑ कृ॒तम् । \newline
24. कृ॒त मया॑ना॒ मया॑नाम् कृ॒तम् कृ॒त मया॑नाम् । \newline
25. अया॑ना॒ मिती त्यया॑ना॒ मया॑ना॒ मिति॑ । \newline
26. इत्या॑हा॒ हेती त्या॑ह । \newline
27. आ॒ह॒ वयो॑भि॒र् वयो॑भि राहाह॒ वयो॑भिः । \newline
28. वयो॑भि रे॒वैव वयो॑भि॒र् वयो॑भि रे॒व । \newline
29. वयो॑भि॒रिति॒ वयः॑ - भिः॒ । \newline
30. ए॒वाया॒ नया॑ ने॒वैवायान्॑ । \newline
31. अया॒ नवा वाया॒ नया॒ नव॑ । \newline
32. अव॑ रुन्धे रु॒न्धे ऽवाव॑ रुन्धे । \newline
33. रु॒न्धे ऽयै॒ रयै॑ रुन्धे रु॒न्धे ऽयैः᳚ । \newline
34. अयै॒र् वयाꣳ॑सि॒ वयाꣳ॒॒स्य यै॒ रयै॒र् वयाꣳ॑सि । \newline
35. वयाꣳ॑सि स॒र्वतः॑ स॒र्वतो॒ वयाꣳ॑सि॒ वयाꣳ॑सि स॒र्वतः॑ । \newline
36. स॒र्वतो॑ वायु॒मती᳚र् वायु॒मतीः᳚ स॒र्वतः॑ स॒र्वतो॑ वायु॒मतीः᳚ । \newline
37. वा॒यु॒मती᳚र् भवन्ति भवन्ति वायु॒मती᳚र् वायु॒मती᳚र् भवन्ति । \newline
38. वा॒यु॒मती॒रिति॑ वायु - मतीः᳚ । \newline
39. भ॒व॒न्ति॒ तस्मा॒त् तस्मा᳚द् भवन्ति भवन्ति॒ तस्मा᳚त् । \newline
40. तस्मा॑ द॒य म॒यम् तस्मा॒त् तस्मा॑ द॒यम् । \newline
41. अ॒यꣳ स॒र्वतः॑ स॒र्वतो॒ ऽय म॒यꣳ स॒र्वतः॑ । \newline
42. स॒र्वतः॑ पवते पवते स॒र्वतः॑ स॒र्वतः॑ पवते । \newline
43. प॒व॒त॒ इति॑ पवते । \newline

\textbf{Ghana Paata } \newline

1. असु॑वर्ग्य मस्या॒स्या सु॑वर्ग्य॒ मसु॑वर्ग्य मस्य॒ तत् तद॒स्या सु॑वर्ग्य॒ मसु॑वर्ग्य मस्य॒ तत् । \newline
2. असु॑वर्ग्य॒मित्यसु॑वः - ग्य॒म् । \newline
3. अ॒स्य॒ तत् तद॑स्यास्य॒ तथ् सु॑व॒र्ग्यः॑ सुव॒र्ग्य॑ स्तद॑स्यास्य॒ तथ् सु॑व॒र्ग्यः॑ । \newline
4. तथ् सु॑व॒र्ग्यः॑ सुव॒र्ग्य॑ स्तत् तथ् सु॑व॒र्ग्यो᳚ ऽग्नि र॒ग्निः सु॑व॒र्ग्य॑ स्तत् तथ् सु॑व॒र्ग्यो᳚ ऽग्निः । \newline
5. सु॒व॒र्ग्यो᳚ ऽग्नि र॒ग्निः सु॑व॒र्ग्यः॑ सुव॒र्ग्यो᳚ ऽग्निर् यद् यद॒ग्निः सु॑व॒र्ग्यः॑ सुव॒र्ग्यो᳚ ऽग्निर् यत् । \newline
6. सु॒व॒र्ग्य॑ इति॑ सुवः - ग्यः॑ । \newline
7. अ॒ग्निर् यद् यद॒ग्नि र॒ग्निर् यथ् सं॒ॅयतः॑ सं॒ॅयतो॒ यद॒ग्नि र॒ग्निर् यथ् सं॒ॅयतः॑ । \newline
8. यथ् सं॒ॅयतः॑ सं॒ॅयतो॒ यद् यथ् सं॒ॅयत॑ उप॒दधा᳚ त्युप॒दधा॑ति सं॒ॅयतो॒ यद् यथ् सं॒ॅयत॑ उप॒दधा॑ति । \newline
9. सं॒ॅयत॑ उप॒दधा᳚ त्युप॒दधा॑ति सं॒ॅयतः॑ सं॒ॅयत॑ उप॒दधा॑ति॒ सꣳ स मु॑प॒दधा॑ति सं॒ॅयतः॑ सं॒ॅयत॑ उप॒दधा॑ति॒ सम् । \newline
10. सं॒ॅयत॒ इति॑ सं - यतः॑ । \newline
11. उ॒प॒दधा॑ति॒ सꣳ स मु॑प॒दधा᳚ त्युप॒दधा॑ति॒ स मे॒वैव स मु॑प॒दधा᳚ त्युप॒दधा॑ति॒ स मे॒व । \newline
12. उ॒प॒दधा॒तीत्यु॑प - दधा॑ति । \newline
13. स मे॒वैव सꣳ स मे॒वैन॑ मेन मे॒व सꣳ स मे॒वैन᳚म् । \newline
14. ए॒वैन॑ मेन मे॒वैवैनं॑ ॅयच्छति यच्छ त्येन मे॒वैवैनं॑ ॅयच्छति । \newline
15. ए॒नं॒ ॅय॒च्छ॒ति॒ य॒च्छ॒ त्ये॒न॒ मे॒नं॒ ॅय॒च्छ॒ति॒ सु॒व॒र्ग्यꣳ॑ सुव॒र्ग्यं॑ ॅयच्छ त्येन मेनं ॅयच्छति सुव॒र्ग्य᳚म् । \newline
16. य॒च्छ॒ति॒ सु॒व॒र्ग्यꣳ॑ सुव॒र्ग्यं॑ ॅयच्छति यच्छति सुव॒र्ग्य॑ मे॒वैव सु॑व॒र्ग्यं॑ ॅयच्छति यच्छति सुव॒र्ग्य॑ मे॒व । \newline
17. सु॒व॒र्ग्य॑ मे॒वैव सु॑व॒र्ग्यꣳ॑ सुव॒र्ग्य॑ मे॒वाक॑ रक रे॒व सु॑व॒र्ग्यꣳ॑ सुव॒र्ग्य॑ मे॒वाकः॑ । \newline
18. सु॒व॒र्ग्य॑मिति॑ सुवः - ग्य᳚म् । \newline
19. ए॒वाक॑ रक रे॒वैवाक॒ स्त्र्यवि॒ स्त्र्यवि॑रक रे॒वैवाक॒ स्त्र्यविः॑ । \newline
20. अ॒क॒ स्त्र्यवि॒ स्त्र्यवि॑रक रक॒ स्त्र्यवि॒र् वयो॒ वय॒ स्त्र्यवि॑रक रक॒ स्त्र्यवि॒र् वयः॑ । \newline
21. त्र्यवि॒र् वयो॒ वय॒ स्त्र्यवि॒ स्त्र्यवि॒र् वयः॑ कृ॒तम् कृ॒तं ॅवय॒ स्त्र्यवि॒ स्त्र्यवि॒र् वयः॑ कृ॒तम् । \newline
22. त्र्यवि॒रिति॑ त्रि - अविः॑ । \newline
23. वयः॑ कृ॒तम् कृ॒तं ॅवयो॒ वयः॑ कृ॒त मया॑ना॒ मया॑नाम् कृ॒तं ॅवयो॒ वयः॑ कृ॒त मया॑नाम् । \newline
24. कृ॒त मया॑ना॒ मया॑नाम् कृ॒तम् कृ॒त मया॑ना॒ मिती त्यया॑नाम् कृ॒तम् कृ॒त मया॑ना॒ मिति॑ । \newline
25. अया॑ना॒ मिती त्यया॑ना॒ मया॑ना॒ मित्या॑हा॒हे त्यया॑ना॒ मया॑ना॒ मित्या॑ह । \newline
26. इत्या॑हा॒हे तीत्या॑ह॒ वयो॑भि॒र् वयो॑भि रा॒हे तीत्या॑ह॒ वयो॑भिः । \newline
27. आ॒ह॒ वयो॑भि॒र् वयो॑भि राहाह॒ वयो॑भि रे॒वैव वयो॑भि राहाह॒ वयो॑भि रे॒व । \newline
28. वयो॑भि रे॒वैव वयो॑भि॒र् वयो॑भि रे॒वाया॒ नया॑ ने॒व वयो॑भि॒र् वयो॑भि रे॒वायान्॑ । \newline
29. वयो॑भि॒रिति॒ वयः॑ - भिः॒ । \newline
30. ए॒वाया॒ नया॑ ने॒वैवाया॒ नवावाया॑ ने॒वैवाया॒ नव॑ । \newline
31. अया॒ नवावाया॒ नया॒ नव॑ रुन्धे रु॒न्धे ऽवाया॒ नया॒ नव॑ रुन्धे । \newline
32. अव॑ रुन्धे रु॒न्धे ऽवाव॑ रु॒न्धे ऽयै॒ रयै॑ रु॒न्धे ऽवाव॑ रु॒न्धे ऽयैः᳚ । \newline
33. रु॒न्धे ऽयै॒ रयै॑ रुन्धे रु॒न्धे ऽयै॒र् वयाꣳ॑सि॒ वयाꣳ॒॒ स्ययै॑ रुन्धे रु॒न्धे ऽयै॒र् वयाꣳ॑सि । \newline
34. अयै॒र् वयाꣳ॑सि॒ वयाꣳ॒॒ स्ययै॒ रयै॒र् वयाꣳ॑सि स॒र्वतः॑ स॒र्वतो॒ वयाꣳ॒॒ स्ययै॒ रयै॒र् वयाꣳ॑सि स॒र्वतः॑ । \newline
35. वयाꣳ॑सि स॒र्वतः॑ स॒र्वतो॒ वयाꣳ॑सि॒ वयाꣳ॑सि स॒र्वतो॑ वायु॒मती᳚र् वायु॒मतीः᳚ स॒र्वतो॒ वयाꣳ॑सि॒ वयाꣳ॑सि स॒र्वतो॑ वायु॒मतीः᳚ । \newline
36. स॒र्वतो॑ वायु॒मती᳚र् वायु॒मतीः᳚ स॒र्वतः॑ स॒र्वतो॑ वायु॒मती᳚र् भवन्ति भवन्ति वायु॒मतीः᳚ स॒र्वतः॑ स॒र्वतो॑ वायु॒मती᳚र् भवन्ति । \newline
37. वा॒यु॒मती᳚र् भवन्ति भवन्ति वायु॒मती᳚र् वायु॒मती᳚र् भवन्ति॒ तस्मा॒त् तस्मा᳚द् भवन्ति वायु॒मती᳚र् वायु॒मती᳚र् भवन्ति॒ तस्मा᳚त् । \newline
38. वा॒यु॒मती॒रिति॑ वायु - मतीः᳚ । \newline
39. भ॒व॒न्ति॒ तस्मा॒त् तस्मा᳚द् भवन्ति भवन्ति॒ तस्मा॑ द॒य म॒यम् तस्मा᳚द् भवन्ति भवन्ति॒ तस्मा॑ द॒यम् । \newline
40. तस्मा॑द॒य म॒यम् तस्मा॒त् तस्मा॑ द॒यꣳ स॒र्वतः॑ स॒र्वतो॒ ऽयम् तस्मा॒त् तस्मा॑ द॒यꣳ स॒र्वतः॑ । \newline
41. अ॒यꣳ स॒र्वतः॑ स॒र्वतो॒ ऽय म॒यꣳ स॒र्वतः॑ पवते पवते स॒र्वतो॒ ऽय म॒यꣳ स॒र्वतः॑ पवते । \newline
42. स॒र्वतः॑ पवते पवते स॒र्वतः॑ स॒र्वतः॑ पवते । \newline
43. प॒व॒त॒ इति॑ पवते । \newline
\pagebreak
\markright{ TS 5.2.11.1  \hfill https://www.vedavms.in \hfill}

\section{ TS 5.2.11.1 }

\textbf{TS 5.2.11.1 } \newline
\textbf{Samhita Paata} \newline

गा॒य॒त्री त्रि॒ष्टुब् जग॑त्यनु॒ष्टुक् प॒ङ्क्त्या॑ स॒ह । बृ॒ह॒त्यु॑ष्णिहा॑ क॒कुथ् सू॒चीभिः॑ शिम्यन्तु त्वा ॥द्वि॒पदा॒ या चतु॑ष्पदा त्रि॒पदा॒ याच॒ षट्प॑दा । सछ॑न्दा॒ या च॒ विच्छ॑न्दाः सू॒चीभिः॑ शिम्यन्तु त्वा ॥म॒हाना᳚म्नी रे॒वत॑यो॒ विश्वा॒ आशाः᳚ प्र॒सूव॑रीः । मेघ्या॑ वि॒द्युतो॒ वाचः॑ सू॒चीभिः॑ शिम्यन्तु त्वा ॥र॒ज॒ता हरि॑णीः॒ सीसा॒ युजो॑ युज्यन्ते॒ कर्म॑भिः । अश्व॑स्य वा॒जिन॑स्त्व॒चि सू॒चीभिः॑ शिम्यन्तु त्वा ॥ नारी᳚ - [  ] \newline

\textbf{Pada Paata} \newline

गा॒य॒त्री । त्रि॒ष्टुप् । जग॑ती । अ॒नु॒ष्टुगित्य॑नु-स्तुक् । प॒ङ्क्त्या᳚ । स॒ह ॥ बृ॒ह॒ती । उ॒ष्णिहा᳚ । क॒कुत् । सू॒चीभिः॑ । शि॒म्य॒न्तु॒ । त्वा॒ ॥ द्वि॒पदेति॑ द्वि - पदा᳚ । या । चतु॑ष्प॒देति॒ चतुः॑ - प॒दा॒ । त्रि॒पदेति॑ त्रि - पदा᳚ । या । च॒ । षट्प॒देति॒ षट् - प॒दा॒ ॥ सछ॑न्दा॒ इति॒ स - छ॒न्दाः॒ । या । च॒ । विच्छ॑न्दा॒ इति॒ वि - छ॒न्दाः॒ । सू॒चीभिः॑ । शि॒म्य॒न्तु॒ । त्वा॒ ॥ म॒हाना᳚म्नी॒रिति॑ म॒हा - ना॒म्नीः॒ । रे॒वत॑यः । विश्वाः᳚ । आशाः᳚ । प्र॒सूव॑री॒रिति॑ प्र - सूव॑रीः ॥ मेघ्‌याः᳚ । वि॒द्युत॒ इति॑ वि - द्युतः॑ । वाचः॑ । सू॒चीभिः॑ । शि॒म्य॒न्तु॒ । त्वा॒ ॥ र॒ज॒ताः । हरि॑णीः । सीसाः᳚ । युजः॑ । यु॒ज्य॒न्ते॒ । कर्म॑भि॒रिति॒ कर्म॑ -  भिः॒ ॥ अश्व॑स्य । वा॒जिनः॑ । त्व॒चि । सू॒चीभिः॑ । शि॒म्य॒न्तु॒ । त्वा॒ ॥ नारीः᳚ ।  \newline


\textbf{Krama Paata} \newline

गा॒य॒त्री त्रि॒ष्टुप् । त्रि॒ष्टुब् जग॑ती । जग॑त्यनु॒ष्टुक् । अ॒नु॒ष्टुक् प॒ङ्क्त्या᳚ । अ॒नु॒ष्टुगित्य॑नु - स्तुक् । प॒ङ्क्त्या॑ स॒ह । स॒हेति॑ स॒ह ॥ बृ॒ह॒त्यु॑ष्णिहा᳚ । उ॒ष्णिहा॑ क॒कुत् । क॒कुथ् सू॒चीभिः॑ । सू॒चीभिः॑ शिम्यन्तु । शि॒म्य॒न्तु॒ त्वा॒ । त्वेति॑ त्वा ॥ द्वि॒पदा॒ या । द्वि॒पदेति॑ द्वि - पदा᳚ । या चतु॑ष्पदा । चतु॑ष्पदा त्रि॒पदा᳚ । चतु॑ष्प॒देति॒ चतुः॑ - प॒दा॒ । त्रि॒पदा॒ या । त्रि॒पदेति॑ त्रि - पदा᳚ । या च॑ । च॒ षट्प॑दा । षट्प॒देति॒ षट् - प॒दा॒ ॥ सछ॑न्दा॒ या । सछ॑न्दा॒ इति॒ स - छ॒न्दाः॒ । या च॑ । च॒ विच्छ॑न्दाः । विच्छ॑न्दाः सू॒चीभिः॑ । विच्छ॑न्दा॒ इति॒ वि - छ॒न्दाः॒ । सू॒चीभिः॑ शिम्यन्तु । शि॒म्य॒न्तु॒ त्वा॒ । त्वेति॑ त्वा ॥ म॒हाना᳚म्नी रे॒वत॑यः । म॒हाना᳚म्नी॒रिति॑ म॒हा - ना॒म्नीः॒ । रे॒वत॑यो॒ विश्वाः᳚ । विश्वा॒ आशाः᳚ । आशाः᳚ प्र॒सूव॑रीः । प्र॒सूव॑री॒रिति॑ प्र - सूव॑रीः ॥ मेद्ध्या॑ वि॒द्युतः॑ । वि॒द्युतो॒ वाचः॑ । वि॒द्युत॒ इति॑ वि - द्युतः॑ । वाचः॑ सू॒चीभिः॑ । सू॒चीभिः॑ शिम्यन्तु । शि॒म्य॒न्तु॒ त्वा॒ । त्वेति॑ त्वा ॥ र॒ज॒ता हरि॑णीः । हरि॑णीः॒ सीसाः᳚ । सीसा॒ युजः॑ । युजो॑ युज्यन्ते । यु॒ज्य॒न्ते॒ कर्म॑भिः । कर्म॑भि॒रिति॒ कर्म॑ - भिः॒ ॥ अश्व॑स्य वा॒जिनः॑ । वा॒जिन॑स्त्व॒चि । त्व॒चि सू॒चीभिः॑ । सू॒चीभिः॑ शिम्यन्तु । शि॒म्य॒न्तु॒ त्वा॒ । त्वेति॑ त्वा ॥ नारी᳚स्ते ( ) \newline

\textbf{Jatai Paata} \newline

1. गा॒य॒त्री त्रि॒ष्टुप् त्रि॒ष्टुब् गा॑य॒त्री गा॑य॒त्री त्रि॒ष्टुप् । \newline
2. त्रि॒ष्टुब् जग॑ती॒ जग॑ती त्रि॒ष्टुप् त्रि॒ष्टुब् जग॑ती । \newline
3. जग॑त्य नु॒ष्टु ग॑नु॒ष्टुग् जग॑ती॒ जग॑त्य नु॒ष्टुक् । \newline
4. अ॒नु॒ष्टुक् प॒ङ्क्त्या॑ प॒ङ्क्त्या॑ ऽनु॒ष्टु ग॑नु॒ष्टुक् प॒ङ्क्त्या᳚ । \newline
5. अ॒नु॒ष्टुगित्य॑नु - स्तुक् । \newline
6. प॒ङ्क्त्या॑ स॒ह स॒ह प॒ङ्क्त्या॑ प॒ङ्क्त्या॑ स॒ह । \newline
7. स॒हेति॑ स॒ह । \newline
8. बृ॒ह॒ त्यु॑ष्णिहो॒ ष्णिहा॑ बृह॒ती बृ॑ह॒ त्यु॑ष्णिहा᳚ । \newline
9. उ॒ष्णिहा॑ क॒कुत् क॒कु दु॒ष्णिहो॒ ष्णिहा॑ क॒कुत् । \newline
10. क॒कुथ् सू॒चीभिः॑ सू॒चीभिः॑ क॒कुत् क॒कुथ् सू॒चीभिः॑ । \newline
11. सू॒चीभिः॑ शिम्यन्तु शिम्यन्तु सू॒चीभिः॑ सू॒चीभिः॑ शिम्यन्तु । \newline
12. शि॒म्य॒न्तु॒ त्वा॒ त्वा॒ शि॒म्य॒न्तु॒ शि॒म्य॒न्तु॒ त्वा॒ । \newline
13. त्वेति॑ त्वा । \newline
14. द्वि॒पदा॒ या या द्वि॒पदा᳚ द्वि॒पदा॒ या । \newline
15. द्वि॒पदेति॑ द्वि - पदा᳚ । \newline
16. या चतु॑ष्पदा॒ चतु॑ष्पदा॒ या या चतु॑ष्पदा । \newline
17. चतु॑ष्पदा त्रि॒पदा᳚ त्रि॒पदा॒ चतु॑ष्पदा॒ चतु॑ष्पदा त्रि॒पदा᳚ । \newline
18. चतु॑ष्प॒देति॒ चतुः॑ - प॒दा॒ । \newline
19. त्रि॒पदा॒ या या त्रि॒पदा᳚ त्रि॒पदा॒ या । \newline
20. त्रि॒पदेति॑ त्रि - पदा᳚ । \newline
21. या च॑ च॒ या या च॑ । \newline
22. च॒ षट्प॑दा॒ षट्प॑दा च च॒ षट्प॑दा । \newline
23. षट्प॒देति॒ षट् - प॒दा॒ । \newline
24. सछ॑न्दा॒ या या सछ॑न्दाः॒ सछ॑न्दा॒ या । \newline
25. सछ॑न्दा॒ इति॒ स - छ॒न्दाः॒ । \newline
26. या च॑ च॒ या या च॑ । \newline
27. च॒ विच्छ॑न्दा॒ विच्छ॑न्दाश्च च॒ विच्छ॑न्दाः । \newline
28. विच्छ॑न्दाः सू॒चीभिः॑ सू॒चीभि॒र् विच्छ॑न्दा॒ विच्छ॑न्दाः सू॒चीभिः॑ । \newline
29. विच्छ॑न्दा॒ इति॒ वि - छ॒न्दाः॒ । \newline
30. सू॒चीभिः॑ शिम्यन्तु शिम्यन्तु सू॒चीभिः॑ सू॒चीभिः॑ शिम्यन्तु । \newline
31. शि॒म्य॒न्तु॒ त्वा॒ त्वा॒ शि॒म्य॒न्तु॒ शि॒म्य॒न्तु॒ त्वा॒ । \newline
32. त्वेति॑ त्वा । \newline
33. म॒हाना᳚म्नी रे॒वत॑यो रे॒वत॑यो म॒हाना᳚म्नीर् म॒हाना᳚म्नी रे॒वत॑यः । \newline
34. म॒हाना᳚म्नी॒रिति॑ म॒हा - ना॒म्नीः॒ । \newline
35. रे॒वत॑यो॒ विश्वा॒ विश्वा॑ रे॒वत॑यो रे॒वत॑यो॒ विश्वाः᳚ । \newline
36. विश्वा॒ आशा॒ आशा॒ विश्वा॒ विश्वा॒ आशाः᳚ । \newline
37. आशाः᳚ प्र॒सूव॑रीः प्र॒सूव॑री॒ राशा॒ आशाः᳚ प्र॒सूव॑रीः । \newline
38. प्र॒सूव॑री॒रिति॑ प्र - सूव॑रीः । \newline
39. मेघ्या॑ वि॒द्युतो॑ वि॒द्युतो॒ मेघ्या॒ मेघ्या॑ वि॒द्युतः॑ । \newline
40. वि॒द्युतो॒ वाचो॒ वाचो॑ वि॒द्युतो॑ वि॒द्युतो॒ वाचः॑ । \newline
41. वि॒द्युत॒ इति॑ वि - द्युतः॑ । \newline
42. वाचः॑ सू॒चीभिः॑ सू॒चीभि॒र् वाचो॒ वाचः॑ सू॒चीभिः॑ । \newline
43. सू॒चीभिः॑ शिम्यन्तु शिम्यन्तु सू॒चीभिः॑ सू॒चीभिः॑ शिम्यन्तु । \newline
44. शि॒म्य॒न्तु॒ त्वा॒ त्वा॒ शि॒म्य॒न्तु॒ शि॒म्य॒न्तु॒ त्वा॒ । \newline
45. त्वेति॑ त्वा । \newline
46. र॒ज॒ता हरि॑णी॒र्॒. हरि॑णी रज॒ता र॑ज॒ता हरि॑णीः । \newline
47. हरि॑णीः॒ सीसाः॒ सीसा॒ हरि॑णी॒र्॒. हरि॑णीः॒ सीसाः᳚ । \newline
48. सीसा॒ युजो॒ युजः॒ सीसाः॒ सीसा॒ युजः॑ । \newline
49. युजो॑ युज्यन्ते युज्यन्ते॒ युजो॒ युजो॑ युज्यन्ते । \newline
50. यु॒ज्य॒न्ते॒ कर्म॑भिः॒ कर्म॑भिर् युज्यन्ते युज्यन्ते॒ कर्म॑भिः । \newline
51. कर्म॑भि॒रिति॒ कर्म॑ - भिः॒ । \newline
52. अश्व॑स्य वा॒जिनो॑ वा॒जिनो ऽश्व॒स्या श्व॑स्य वा॒जिनः॑ । \newline
53. वा॒जिन॑ स्त्व॒चि त्व॒चि वा॒जिनो॑ वा॒जिन॑ स्त्व॒चि । \newline
54. त्व॒चि सू॒चीभिः॑ सू॒चीभि॑ स्त्व॒चि त्व॒चि सू॒चीभिः॑ । \newline
55. सू॒चीभिः॑ शिम्यन्तु शिम्यन्तु सू॒चीभिः॑ सू॒चीभिः॑ शिम्यन्तु । \newline
56. शि॒म्य॒न्तु॒ त्वा॒ त्वा॒ शि॒म्य॒न्तु॒ शि॒म्य॒न्तु॒ त्वा॒ । \newline
57. त्वेति॑ त्वा । \newline
58. नारी᳚ स्ते ते॒ नारी॒र् नारी᳚ स्ते । \newline

\textbf{Ghana Paata } \newline

1. गा॒य॒त्री त्रि॒ष्टुप् त्रि॒ष्टुब् गा॑य॒त्री गा॑य॒त्री त्रि॒ष्टुब् जग॑ती॒ जग॑ती त्रि॒ष्टुब् गा॑य॒त्री गा॑य॒त्री त्रि॒ष्टुब् जग॑ती । \newline
2. त्रि॒ष्टुब् जग॑ती॒ जग॑ती त्रि॒ष्टुप् त्रि॒ष्टुब् जग॑ त्यनु॒ष्टु ग॑नु॒ष्टुग् जग॑ती त्रि॒ष्टुप् त्रि॒ष्टुब् जग॑ त्यनु॒ष्टुक् । \newline
3. जग॑ त्यनु॒ष्टु ग॑नु॒ष्टुग् जग॑ती॒ जग॑ त्यनु॒ष्टुक् प॒ङ्क्त्या॑ प॒ङ्क्त्या॑ ऽनु॒ष्टुग् जग॑ती॒ जग॑ त्यनु॒ष्टुक् प॒ङ्क्त्या᳚ । \newline
4. अ॒नु॒ष्टुक् प॒ङ्क्त्या॑ प॒ङ्क्त्या॑ ऽनु॒ष्टु ग॑नु॒ष्टुक् प॒ङ्क्त्या॑ स॒ह स॒ह प॒ङ्क्त्या॑ ऽनु॒ष्टु ग॑नु॒ष्टुक् प॒ङ्क्त्या॑ स॒ह । \newline
5. अ॒नु॒ष्टुगित्य॑नु - स्तुक् । \newline
6. प॒ङ्क्त्या॑ स॒ह स॒ह प॒ङ्क्त्या॑ प॒ङ्क्त्या॑ स॒ह । \newline
7. स॒हेति॑ स॒ह । \newline
8. बृ॒ह॒ त्यु॑ष्णि हो॒ष्णिहा॑ बृह॒ती बृ॑ह॒ त्यु॑ष्णिहा॑ क॒कुत् क॒कु दु॒ष्णिहा॑ बृह॒ती बृ॑ह॒ त्यु॑ष्णिहा॑ क॒कुत् । \newline
9. उ॒ष्णिहा॑ क॒कुत् क॒कु दु॒ष्णि हो॒ष्णिहा॑ क॒कुथ् सू॒चीभिः॑ सू॒चीभिः॑ क॒कु दु॒ष्णि हो॒ष्णिहा॑ क॒कुथ् सू॒चीभिः॑ । \newline
10. क॒कुथ् सू॒चीभिः॑ सू॒चीभिः॑ क॒कुत् क॒कुथ् सू॒चीभिः॑ शिम्यन्तु शिम्यन्तु सू॒चीभिः॑ क॒कुत् क॒कुथ् सू॒चीभिः॑ शिम्यन्तु । \newline
11. सू॒चीभिः॑ शिम्यन्तु शिम्यन्तु सू॒चीभिः॑ सू॒चीभिः॑ शिम्यन्तु त्वा त्वा शिम्यन्तु सू॒चीभिः॑ सू॒चीभिः॑ शिम्यन्तु त्वा । \newline
12. शि॒म्य॒न्तु॒ त्वा॒ त्वा॒ शि॒म्य॒न्तु॒ शि॒म्य॒न्तु॒ त्वा॒ । \newline
13. त्वेति॑ त्वा । \newline
14. द्वि॒पदा॒ या या द्वि॒पदा᳚ द्वि॒पदा॒ या चतु॑ष्पदा॒ चतु॑ष्पदा॒ या द्वि॒पदा᳚ द्वि॒पदा॒ या चतु॑ष्पदा । \newline
15. द्वि॒पदेति॑ द्वि - पदा᳚ । \newline
16. या चतु॑ष्पदा॒ चतु॑ष्पदा॒ या या चतु॑ष्पदा त्रि॒पदा᳚ त्रि॒पदा॒ चतु॑ष्पदा॒ या या चतु॑ष्पदा त्रि॒पदा᳚ । \newline
17. चतु॑ष्पदा त्रि॒पदा᳚ त्रि॒पदा॒ चतु॑ष्पदा॒ चतु॑ष्पदा त्रि॒पदा॒ या या त्रि॒पदा॒ चतु॑ष्पदा॒ चतु॑ष्पदा त्रि॒पदा॒ या । \newline
18. चतु॑ष्प॒देति॒ चतुः॑ - प॒दा॒ । \newline
19. त्रि॒पदा॒ या या त्रि॒पदा᳚ त्रि॒पदा॒ या च॑ च॒ या त्रि॒पदा᳚ त्रि॒पदा॒ या च॑ । \newline
20. त्रि॒पदेति॑ त्रि - पदा᳚ । \newline
21. या च॑ च॒ या या च॒ षट्प॑दा॒ षट्प॑दा च॒ या या च॒ षट्प॑दा । \newline
22. च॒ षट्प॑दा॒ षट्प॑दा च च॒ षट्प॑दा । \newline
23. षट्प॒देति॒ षट् - प॒दा॒ । \newline
24. सछ॑न्दा॒ या या सछ॑न्दाः॒ सछ॑न्दा॒ या च॑ च॒ या सछ॑न्दाः॒ सछ॑न्दा॒ या च॑ । \newline
25. सछ॑न्दा॒ इति॒ स - छ॒न्दाः॒ । \newline
26. या च॑ च॒ या या च॒ विच्छ॑न्दा॒ विच्छ॑न्दाश्च॒ या या च॒ विच्छ॑न्दाः । \newline
27. च॒ विच्छ॑न्दा॒ विच्छ॑न्दाश्च च॒ विच्छ॑न्दाः सू॒चीभिः॑ सू॒चीभि॒र् विच्छ॑न्दाश्च च॒ विच्छ॑न्दाः सू॒चीभिः॑ । \newline
28. विच्छ॑न्दाः सू॒चीभिः॑ सू॒चीभि॒र् विच्छ॑न्दा॒ विच्छ॑न्दाः सू॒चीभिः॑ शिम्यन्तु शिम्यन्तु सू॒चीभि॒र् विच्छ॑न्दा॒ विच्छ॑न्दाः सू॒चीभिः॑ शिम्यन्तु । \newline
29. विच्छ॑न्दा॒ इति॒ वि - छ॒न्दाः॒ । \newline
30. सू॒चीभिः॑ शिम्यन्तु शिम्यन्तु सू॒चीभिः॑ सू॒चीभिः॑ शिम्यन्तु त्वा त्वा शिम्यन्तु सू॒चीभिः॑ सू॒चीभिः॑ शिम्यन्तु त्वा । \newline
31. शि॒म्य॒न्तु॒ त्वा॒ त्वा॒ शि॒म्य॒न्तु॒ शि॒म्य॒न्तु॒ त्वा॒ । \newline
32. त्वेति॑ त्वा । \newline
33. म॒हाना᳚म्नी रे॒वत॑यो रे॒वत॑यो म॒हाना᳚म्नीर् म॒हाना᳚म्नी रे॒वत॑यो॒ विश्वा॒ विश्वा॑ रे॒वत॑यो म॒हाना᳚म्नीर् म॒हाना᳚म्नी रे॒वत॑यो॒ विश्वाः᳚ । \newline
34. म॒हाना᳚म्नी॒रिति॑ म॒हा - ना॒म्नीः॒ । \newline
35. रे॒वत॑यो॒ विश्वा॒ विश्वा॑ रे॒वत॑यो रे॒वत॑यो॒ विश्वा॒ आशा॒ आशा॒ विश्वा॑ रे॒वत॑यो रे॒वत॑यो॒ विश्वा॒ आशाः᳚ । \newline
36. विश्वा॒ आशा॒ आशा॒ विश्वा॒ विश्वा॒ आशाः᳚ प्र॒सूव॑रीः प्र॒सूव॑री॒ राशा॒ विश्वा॒ विश्वा॒ आशाः᳚ प्र॒सूव॑रीः । \newline
37. आशाः᳚ प्र॒सूव॑रीः प्र॒सूव॑री॒ राशा॒ आशाः᳚ प्र॒सूव॑रीः । \newline
38. प्र॒सूव॑री॒रिति॑ प्र - सूव॑रीः । \newline
39. मेघ्या॑ वि॒द्युतो॑ वि॒द्युतो॒ मेघ्या॒ मेघ्या॑ वि॒द्युतो॒ वाचो॒ वाचो॑ वि॒द्युतो॒ मेघ्या॒ मेघ्या॑ वि॒द्युतो॒ वाचः॑ । \newline
40. वि॒द्युतो॒ वाचो॒ वाचो॑ वि॒द्युतो॑ वि॒द्युतो॒ वाचः॑ सू॒चीभिः॑ सू॒चीभि॒र् वाचो॑ वि॒द्युतो॑ वि॒द्युतो॒ वाचः॑ सू॒चीभिः॑ । \newline
41. वि॒द्युत॒ इति॑ वि - द्युतः॑ । \newline
42. वाचः॑ सू॒चीभिः॑ सू॒चीभि॒र् वाचो॒ वाचः॑ सू॒चीभिः॑ शिम्यन्तु शिम्यन्तु सू॒चीभि॒र् वाचो॒ वाचः॑ सू॒चीभिः॑ शिम्यन्तु । \newline
43. सू॒चीभिः॑ शिम्यन्तु शिम्यन्तु सू॒चीभिः॑ सू॒चीभिः॑ शिम्यन्तु त्वा त्वा शिम्यन्तु सू॒चीभिः॑ सू॒चीभिः॑ शिम्यन्तु त्वा । \newline
44. शि॒म्य॒न्तु॒ त्वा॒ त्वा॒ शि॒म्य॒न्तु॒ शि॒म्य॒न्तु॒ त्वा॒ । \newline
45. त्वेति॑ त्वा । \newline
46. र॒ज॒ता हरि॑णी॒र्॒. हरि॑णी रज॒ता र॑ज॒ता हरि॑णीः॒ सीसाः॒ सीसा॒ हरि॑णी रज॒ता र॑ज॒ता हरि॑णीः॒ सीसाः᳚ । \newline
47. हरि॑णीः॒ सीसाः॒ सीसा॒ हरि॑णी॒र्॒. हरि॑णीः॒ सीसा॒ युजो॒ युजः॒ सीसा॒ हरि॑णी॒र्॒. हरि॑णीः॒ सीसा॒ युजः॑ । \newline
48. सीसा॒ युजो॒ युजः॒ सीसाः॒ सीसा॒ युजो॑ युज्यन्ते युज्यन्ते॒ युजः॒ सीसाः॒ सीसा॒ युजो॑ युज्यन्ते । \newline
49. युजो॑ युज्यन्ते युज्यन्ते॒ युजो॒ युजो॑ युज्यन्ते॒ कर्म॑भिः॒ कर्म॑भिर् युज्यन्ते॒ युजो॒ युजो॑ युज्यन्ते॒ कर्म॑भिः । \newline
50. यु॒ज्य॒न्ते॒ कर्म॑भिः॒ कर्म॑भिर् युज्यन्ते युज्यन्ते॒ कर्म॑भिः । \newline
51. कर्म॑भि॒रिति॒ कर्म॑ - भिः॒ । \newline
52. अश्व॑स्य वा॒जिनो॑ वा॒जिनो ऽश्व॒स्या श्व॑स्य वा॒जिन॑ स्त्व॒चि त्व॒चि वा॒जिनो ऽश्व॒स्या श्व॑स्य वा॒जिन॑ स्त्व॒चि । \newline
53. वा॒जिन॑ स्त्व॒चि त्व॒चि वा॒जिनो॑ वा॒जिन॑ स्त्व॒चि सू॒चीभिः॑ सू॒चीभि॑ स्त्व॒चि वा॒जिनो॑ वा॒जिन॑ स्त्व॒चि सू॒चीभिः॑ । \newline
54. त्व॒चि सू॒चीभिः॑ सू॒चीभि॑ स्त्व॒चि त्व॒चि सू॒चीभिः॑ शिम्यन्तु शिम्यन्तु सू॒चीभि॑ स्त्व॒चि त्व॒चि सू॒चीभिः॑ शिम्यन्तु । \newline
55. सू॒चीभिः॑ शिम्यन्तु शिम्यन्तु सू॒चीभिः॑ सू॒चीभिः॑ शिम्यन्तु त्वा त्वा शिम्यन्तु सू॒चीभिः॑ सू॒चीभिः॑ शिम्यन्तु त्वा । \newline
56. शि॒म्य॒न्तु॒ त्वा॒ त्वा॒ शि॒म्य॒न्तु॒ शि॒म्य॒न्तु॒ त्वा॒ । \newline
57. त्वेति॑ त्वा । \newline
58. नारी᳚ स्ते ते॒ नारी॒र् नारी᳚ स्ते॒ पत्न॑यः॒ पत्न॑य स्ते॒ नारी॒र् नारी᳚ स्ते॒ पत्न॑यः । \newline
\pagebreak
\markright{ TS 5.2.11.2  \hfill https://www.vedavms.in \hfill}

\section{ TS 5.2.11.2 }

\textbf{TS 5.2.11.2 } \newline
\textbf{Samhita Paata} \newline

-स्ते॒ पत्न॑यो॒ लोम॒ विचि॑न्वन्तु मनी॒षया᳚ । दे॒वानां॒ पत्नी॒र्दिशः॑ सू॒चीभिः॑ शिम्यन्तु त्वा ॥कु॒विद॒ङ्ग यव॑मन्तो॒ यवं॑ चि॒द्यथा॒ दान्त्य॑नुपू॒र्वं ॅवि॒यूय॑ ।इ॒हेहै॑षां कृणुत॒ भोज॑नानि॒ ये ब॒र्॒.हिषो॒ नमो॑वृक्तिं॒ नज॒ग्मुः ॥ \newline

\textbf{Pada Paata} \newline

ते । पत्न॑यः । लोम॑ । वीति॑ । चि॒न्व॒न्तु॒ । म॒नी॒षया᳚ ॥ दे॒वाना᳚म् । पत्नीः᳚ । दिशः॑ । सू॒चीभिः॑ । शि॒म्य॒न्तु॒ । त्वा॒ ॥ कु॒वित् । अ॒ङ्ग । यव॑मन्त॒ इति॒ यव॑ - म॒न्तः॒ । यव᳚म् । चि॒त् । यथा᳚ । दान्ति॑ । अ॒नु॒पू॒र्वमित्य॑नु - पू॒र्वम् । वि॒यूयेति॑ वि - यूय॑ ॥ इ॒हेहेती॒ह - इ॒ह॒ । ए॒षा॒म् । कृ॒णु॒त॒ । भोज॑नानि । ये । ब॒र्॒.हिषः॑ । नमो॑वृक्ति॒मिति॒ नमः॑ - वृ॒क्ति॒म् । न । ज॒ग्मुः ॥  \newline


\textbf{Krama Paata} \newline

ते॒ पत्न॑यः । पत्न॑यो॒ लोम॑ । लोम॒ वि । वि चि॑न्वन्तु । चि॒न्व॒न्तु॒ म॒नी॒षया᳚ । म॒नी॒षयेति॑ मनी॒षया᳚ ॥ दे॒वाना॒म् पत्नीः᳚ । पत्नी॒र् दिशः॑ । दिशः॑ सू॒चीभिः॑ । सू॒चीभिः॑ शिम्यन्तु । शि॒म्य॒न्तु॒ त्वा॒ । त्वेति॑ त्वा ॥ कु॒विद॒ङ्ग । अ॒ङ्ग यव॑मन्तः । यव॑मन्तो॒ यव᳚म् । यव॑मन्त॒ इति॒ यव॑ - म॒न्तः॒ । यव॑म् चि॒त् । चि॒द् यथा᳚ । यथा॒ दान्ति॑ । दान्त्य॑नुपू॒र्वम् । अ॒नु॒पू॒र्वम् ॅवि॒यूय॑ । अ॒नु॒पू॒र्वमित्य॑नु - पू॒र्वम् । वि॒यूयेति॑ वि - यूय॑ ॥ इ॒हेहै॑षाम् । इ॒हेहेती॒ह - इ॒ह॒ । ए॒षा॒म् कृ॒णु॒त॒ । कृ॒णु॒त॒ भोज॑नानि । भोज॑नानि॒ ये । ये ब॒र्॒.हिषः॑ । ब॒र्॒.हिषो॒ नमो॑वृक्तिम् । नमो॑वृक्ति॒म् न । नमो॑वृक्ति॒मिति॒ नमः॑ - वृ॒क्ति॒म् । न ज॒ग्मुः । ज॒ग्मुरिति॑ ज॒ग्मुः । \newline

\textbf{Jatai Paata} \newline

1. ते॒ पत्न॑यः॒ पत्न॑य स्ते ते॒ पत्न॑यः । \newline
2. पत्न॑यो॒ लोम॒ लोम॒ पत्न॑यः॒ पत्न॑यो॒ लोम॑ । \newline
3. लोम॒ वि वि लोम॒ लोम॒ वि । \newline
4. वि चि॑न्वन्तु चिन्वन्तु॒ वि वि चि॑न्वन्तु । \newline
5. चि॒न्व॒न्तु॒ म॒नी॒षया॑ मनी॒षया॑ चिन्वन्तु चिन्वन्तु मनी॒षया᳚ । \newline
6. म॒नी॒षयेति॑ मनी॒षया᳚ । \newline
7. दे॒वाना॒म् पत्नीः॒ पत्नी᳚र् दे॒वाना᳚म् दे॒वाना॒म् पत्नीः᳚ । \newline
8. पत्नी॒र् दिशो॒ दिशः॒ पत्नीः॒ पत्नी॒र् दिशः॑ । \newline
9. दिशः॑ सू॒चीभिः॑ सू॒चीभि॒र् दिशो॒ दिशः॑ सू॒चीभिः॑ । \newline
10. सू॒चीभिः॑ शिम्यन्तु शिम्यन्तु सू॒चीभिः॑ सू॒चीभिः॑ शिम्यन्तु । \newline
11. शि॒म्य॒न्तु॒ त्वा॒ त्वा॒ शि॒म्य॒न्तु॒ शि॒म्य॒न्तु॒ त्वा॒ । \newline
12. त्वेति॑ त्वा । \newline
13. कु॒वि द॒ङ्गा ङ्ग कु॒वित् कु॒वि द॒ङ्ग । \newline
14. अ॒ङ्ग यव॑मन्तो॒ यव॑मन्तो॒ ऽङ्गा ङ्ग यव॑मन्तः । \newline
15. यव॑मन्तो॒ यवं॒ ॅयवं॒ ॅयव॑मन्तो॒ यव॑मन्तो॒ यव᳚म् । \newline
16. यव॑मन्त॒ इति॒ यव॑ - म॒न्तः॒ । \newline
17. यव॑म् चिच् चि॒द् यवं॒ ॅयव॑म् चित् । \newline
18. चि॒द् यथा॒ यथा॑ चिच् चि॒द् यथा᳚ । \newline
19. यथा॒ दान्ति॒ दान्ति॒ यथा॒ यथा॒ दान्ति॑ । \newline
20. दान्त्य॑ नुपू॒र्व म॑नुपू॒र्वम् दान्ति॒ दान्त्य॑ नुपू॒र्वम् । \newline
21. अ॒नु॒पू॒र्वं ॅवि॒यूय॑ वि॒यूया॑ नुपू॒र्व म॑नुपू॒र्वं ॅवि॒यूय॑ । \newline
22. अ॒नु॒पू॒र्वमित्य॑नु - पू॒र्वम् । \newline
23. वि॒यूयेति॑ वि - यूय॑ । \newline
24. इ॒हेहै॑षा मेषा मि॒हेहे॒ हेहै॑षाम् । \newline
25. इ॒हेहेती॒ह - इ॒ह॒ । \newline
26. ए॒षा॒म् कृ॒णु॒त॒ कृ॒णु॒ तै॒षा॒ मे॒षा॒म् कृ॒णु॒त॒ । \newline
27. कृ॒णु॒त॒ भोज॑नानि॒ भोज॑नानि कृणुत कृणुत॒ भोज॑नानि । \newline
28. भोज॑नानि॒ ये ये भोज॑नानि॒ भोज॑नानि॒ ये । \newline
29. ये ब॒र्॒.हिषो॑ ब॒र्॒.हिषो॒ ये ये ब॒र्॒.हिषः॑ । \newline
30. ब॒र्॒.हिषो॒ नमो॑वृक्ति॒न् नमो॑वृक्तिम् ब॒र्॒.हिषो॑ ब॒र्॒.हिषो॒ नमो॑वृक्तिम् । \newline
31. नमो॑वृक्ति॒म् न न नमो॑वृक्ति॒म् नमो॑वृक्ति॒म् न । \newline
32. नमो॑वृक्ति॒मिति॒ नमः॑ - वृ॒क्ति॒म् । \newline
33. न ज॒ग्मुर् ज॒ग्मुर् न न ज॒ग्मुः । \newline
34. ज॒ग्मुरिति॑ ज॒ग्मुः । \newline

\textbf{Ghana Paata } \newline

1. ते॒ पत्न॑यः॒ पत्न॑य स्ते ते॒ पत्न॑यो॒ लोम॒ लोम॒ पत्न॑य स्ते ते॒ पत्न॑यो॒ लोम॑ । \newline
2. पत्न॑यो॒ लोम॒ लोम॒ पत्न॑यः॒ पत्न॑यो॒ लोम॒ वि वि लोम॒ पत्न॑यः॒ पत्न॑यो॒ लोम॒ वि । \newline
3. लोम॒ वि वि लोम॒ लोम॒ वि चि॑न्वन्तु चिन्वन्तु॒ वि लोम॒ लोम॒ वि चि॑न्वन्तु । \newline
4. वि चि॑न्वन्तु चिन्वन्तु॒ वि वि चि॑न्वन्तु मनी॒षया॑ मनी॒षया॑ चिन्वन्तु॒ वि वि चि॑न्वन्तु मनी॒षया᳚ । \newline
5. चि॒न्व॒न्तु॒ म॒नी॒षया॑ मनी॒षया॑ चिन्वन्तु चिन्वन्तु मनी॒षया᳚ । \newline
6. म॒नी॒षयेति॑ मनी॒षया᳚ । \newline
7. दे॒वाना॒म् पत्नीः॒ पत्नी᳚र् दे॒वाना᳚म् दे॒वाना॒म् पत्नी॒र् दिशो॒ दिशः॒ पत्नी᳚र् दे॒वाना᳚म् दे॒वाना॒म् पत्नी॒र् दिशः॑ । \newline
8. पत्नी॒र् दिशो॒ दिशः॒ पत्नीः॒ पत्नी॒र् दिशः॑ सू॒चीभिः॑ सू॒चीभि॒र् दिशः॒ पत्नीः॒ पत्नी॒र् दिशः॑ सू॒चीभिः॑ । \newline
9. दिशः॑ सू॒चीभिः॑ सू॒चीभि॒र् दिशो॒ दिशः॑ सू॒चीभिः॑ शिम्यन्तु शिम्यन्तु सू॒चीभि॒र् दिशो॒ दिशः॑ सू॒चीभिः॑ शिम्यन्तु । \newline
10. सू॒चीभिः॑ शिम्यन्तु शिम्यन्तु सू॒चीभिः॑ सू॒चीभिः॑ शिम्यन्तु त्वा त्वा शिम्यन्तु सू॒चीभिः॑ सू॒चीभिः॑ शिम्यन्तु त्वा । \newline
11. शि॒म्य॒न्तु॒ त्वा॒ त्वा॒ शि॒म्य॒न्तु॒ शि॒म्य॒न्तु॒ त्वा॒ । \newline
12. त्वेति॑ त्वा । \newline
13. कु॒वि द॒ङ्गाङ्ग कु॒वित् कु॒वि द॒ङ्ग यव॑मन्तो॒ यव॑मन्तो॒ ऽङ्ग कु॒वित् कु॒वि द॒ङ्ग यव॑मन्तः । \newline
14. अ॒ङ्ग यव॑मन्तो॒ यव॑मन्तो॒ ऽङ्गाङ्ग यव॑मन्तो॒ यवं॒ ॅयवं॒ ॅयव॑मन्तो॒ ऽङ्गाङ्ग यव॑मन्तो॒ यव᳚म् । \newline
15. यव॑मन्तो॒ यवं॒ ॅयवं॒ ॅयव॑मन्तो॒ यव॑मन्तो॒ यव॑म् चिच् चि॒द् यवं॒ ॅयव॑मन्तो॒ यव॑मन्तो॒ यव॑म् चित् । \newline
16. यव॑मन्त॒ इति॒ यव॑ - म॒न्तः॒ । \newline
17. यव॑म् चिच् चि॒द् यवं॒ ॅयव॑म् चि॒द् यथा॒ यथा॑ चि॒द् यवं॒ ॅयव॑म् चि॒द् यथा᳚ । \newline
18. चि॒द् यथा॒ यथा॑ चिच् चि॒द् यथा॒ दान्ति॒ दान्ति॒ यथा॑ चिच् चि॒द् यथा॒ दान्ति॑ । \newline
19. यथा॒ दान्ति॒ दान्ति॒ यथा॒ यथा॒ दान् त्य॑नुपू॒र्व म॑नुपू॒र्वम् दान्ति॒ यथा॒ यथा॒ दान् त्य॑नुपू॒र्वम् । \newline
20. दान्त्य॑नुपू॒र्व म॑नुपू॒र्वम् दान्ति॒ दान्त्य॑नुपू॒र्वं ॅवि॒यूय॑ वि॒यूया॑ नुपू॒र्वम् दान्ति॒ दान् त्य॑नुपू॒र्वं ॅवि॒यूय॑ । \newline
21. अ॒नु॒पू॒र्वं ॅवि॒यूय॑ वि॒यूया॑ नुपू॒र्व म॑नुपू॒र्वं ॅवि॒यूय॑ । \newline
22. अ॒नु॒पू॒र्वमित्य॑नु - पू॒र्वम् । \newline
23. वि॒यूयेति॑ वि - यूय॑ । \newline
24. इ॒हेहै॑षा मेषा मि॒हेहे॒ हेहै॑षाम् कृणुत कृणुतैषा मि॒हेहे॒ हेहै॑षाम् कृणुत । \newline
25. इ॒हेहेती॒ह - इ॒ह॒ । \newline
26. ए॒षा॒म् कृ॒णु॒त॒ कृ॒णु॒तै॒षा॒ मे॒षा॒म् कृ॒णु॒त॒ भोज॑नानि॒ भोज॑नानि कृणुतैषा मेषाम् कृणुत॒ भोज॑नानि । \newline
27. कृ॒णु॒त॒ भोज॑नानि॒ भोज॑नानि कृणुत कृणुत॒ भोज॑नानि॒ ये ये भोज॑नानि कृणुत कृणुत॒ भोज॑नानि॒ ये । \newline
28. भोज॑नानि॒ ये ये भोज॑नानि॒ भोज॑नानि॒ ये ब॒र्॒.हिषो॑ ब॒र्॒.हिषो॒ ये भोज॑नानि॒ भोज॑नानि॒ ये ब॒र्॒.हिषः॑ । \newline
29. ये ब॒र्॒.हिषो॑ ब॒र्॒.हिषो॒ ये ये ब॒र्॒.हिषो॒ नमो॑वृक्ति॒म् नमो॑वृक्तिम् ब॒र्॒.हिषो॒ ये ये ब॒र्॒.हिषो॒ नमो॑वृक्तिम् । \newline
30. ब॒र्॒.हिषो॒ नमो॑वृक्ति॒म् नमो॑वृक्तिम् ब॒र्॒.हिषो॑ ब॒र्॒.हिषो॒ नमो॑वृक्ति॒म् न न नमो॑वृक्तिम् ब॒र्॒.हिषो॑ ब॒र्॒.हिषो॒ नमो॑वृक्ति॒म् न । \newline
31. नमो॑वृक्ति॒म् न न नमो॑वृक्ति॒म् नमो॑वृक्ति॒म् न ज॒ग्मुर् ज॒ग्मुर् न नमो॑वृक्ति॒म् नमो॑वृक्ति॒म् न ज॒ग्मुः । \newline
32. नमो॑वृक्ति॒मिति॒ नमः॑ - वृ॒क्ति॒म् । \newline
33. न ज॒ग्मुर् ज॒ग्मुर् न न ज॒ग्मुः । \newline
34. ज॒ग्मुरिति॑ ज॒ग्मुः । \newline
\pagebreak
\markright{ TS 5.2.12.1  \hfill https://www.vedavms.in \hfill}

\section{ TS 5.2.12.1 }

\textbf{TS 5.2.12.1 } \newline
\textbf{Samhita Paata} \newline

कस्त्वा᳚ च्छ्यति॒ कस्त्वा॒ वि शा᳚स्ति॒ कस्ते॒ गात्रा॑णि शिम्यति । क उ॑ ते शमि॒ता क॒विः ॥ ऋ॒तव॑स्त ऋतु॒धा परुः॑ शमि॒तारो॒ विशा॑सतु । सं॒ॅव॒थ्स॒रस्य॒ धाय॑सा॒ शिमी॑भिः शिम्यन्तु त्वा ॥दैव्या॑ अद्ध्व॒र्यव॑स्त्वा॒ च्छ्यन्तु॒ वि च॑ शासतु । गात्रा॑णि पर्व॒शस्ते॒ शिमाः᳚ कृण्वन्तु॒ शिम्य॑न्तः ॥अ॒र्द्ध॒मा॒साः परूꣳ॑षि ते॒ मासाः᳚ छ्यन्तु॒ शिम्य॑न्तः । अ॒हो॒रा॒त्राणि॑ म॒रुतो॒ विलि॑ष्टꣳ - [  ] \newline

\textbf{Pada Paata} \newline

कः । त्वा॒ । छ्य॒ति॒ । कः । त्वा॒ । वीति॑ । शा॒स्ति॒ । कः । ते॒ । गात्रा॑णि । शि॒म्य॒ति॒ ॥ कः । उ॒ । ते॒ । श॒मि॒ता । क॒विः ॥ ऋ॒तवः॑ । ते॒ । ऋ॒तु॒धेत्यृ॑तु - धा । परुः॑ । श॒मि॒तारः॑ । वीति॑ । शा॒स॒तु॒ ॥ सं॒ॅव॒थ्स॒रस्येति॑ सं - व॒थ्स॒रस्य॑ । धाय॑सा । शिमी॑भिः । शि॒म्य॒न्तु॒ । त्वा॒ ॥ दैव्याः᳚ । अ॒द्ध्व॒र्यवः॑ । त्वा॒ । छ्यन्तु॑ । वीति॑ । च॒ । शा॒स॒तु॒ ॥ गात्रा॑णि । प॒र्व॒श इति॑ पर्व - शः । ते॒ । शिमाः᳚ । कृ॒ण्व॒न्तु॒ । शिम्य॑न्तः ॥ अ॒द्‌र्ध॒मा॒सा इत्य॑द्‌र्ध - मा॒साः । परूꣳ॑षि । ते॒ । मासाः᳚ । छ्य॒न्तु॒ । शिम्य॑न्तः ॥ अ॒हो॒रा॒त्राणीत्य॑हः - रा॒त्राणि॑ । म॒रुतः॑ । विलि॑ष्ट॒मिति॒ वि - लि॒ष्ट॒म् ।  \newline


\textbf{Krama Paata} \newline

कस्त्वा᳚ । त्वा॒ छ्य॒ति॒ । छ्य॒ति॒ कः । कस्त्वा᳚ । त्वा॒ वि । वि शा᳚स्ति । शा॒स्ति॒ कः । कस्ते᳚ । ते॒ गात्रा॑णि । गात्रा॑णि शिम्यति । शि॒म्य॒तीति॑ शिम्यति ॥ क उ॑ । उ॒ ते॒ । ते॒ श॒मि॒ता । श॒मि॒ता क॒विः । क॒विरिति॑ क॒विः ॥ ऋ॒तव॑स्ते । त॒ ऋ॒तु॒धा । ऋ॒तु॒धा परुः॑ । ऋ॒तु॒धेत्यृ॑तु - धा । परुः॑ शमि॒तारः॑ । श॒मि॒तारो॒ वि । वि शा॑सतु । शा॒स॒त्विति॑ शासतु ॥ स॒म्ॅव॒थ्स॒रस्य॒ धाय॑सा । स॒म्ॅव॒थ्स॒रस्येति॑ सम् - व॒थ्स॒रस्य॑ । धाय॑सा॒ शिमी॑भिः । शिमी॑भिः शिम्यन्तु । शि॒म्य॒न्तु॒ त्वा॒ । त्वेति॑ त्वा ॥ दैव्या॑ अद्ध्व॒र्यवः॑ । अ॒द्ध्व॒र्यव॑स्त्वा । त्वा॒ छ्यन्तु॑ । छ्यन्तु॒ वि । वि च॑ । च॒ शा॒स॒तु॒ । शा॒स॒त्विति॑ शासतु ॥ गात्रा॑णि पर्व॒शः । प॒र्व॒शस्ते᳚ । प॒र्व॒श इति॑ पर्व - शः । ते॒ शिमाः᳚ । शिमाः᳚ कृण्वन्तु । कृ॒ण्व॒न्तु॒ शिम्य॑न्तः । शिम्य॑न्त॒ इति॒ शिम्य॑न्तः ॥ अ॒र्द्ध॒मा॒साः परूꣳ॑षि । अ॒र्द्ध॒मा॒सा इत्य॑र्द्ध - मा॒साः । परूꣳ॑षि ते । ते॒ मासाः᳚ । मासाः᳚ छ्यन्तु । छ्य॒न्तु॒ शिम्य॑न्तः । शिम्य॑न्त॒ इति॒ शिम्य॑न्तः ॥ अ॒हो॒रा॒त्राणि॑ म॒रुतः॑ । अ॒हो॒रा॒त्राणीत्य॑हः - रा॒त्राणि॑ । म॒रुतो॒ विलि॑ष्टम् ( ) । विलि॑ष्टꣳ सूदयन्तु । विलि॑ष्ट॒मिति॒ वि - लि॒ष्ट॒म् \newline

\textbf{Jatai Paata} \newline

1. क स्त्वा᳚ त्वा॒ कः क स्त्वा᳚ । \newline
2. त्वा॒ छ्य॒ति॒ छ्य॒ति॒ त्वा॒ त्वा॒ छ्य॒ति॒ । \newline
3. छ्य॒ति॒ कः क श्छ्य॑ति छ्यति॒ कः । \newline
4. क स्त्वा᳚ त्वा॒ कः क स्त्वा᳚ । \newline
5. त्वा॒ वि वि त्वा᳚ त्वा॒ वि । \newline
6. वि शा᳚स्ति शास्ति॒ वि वि शा᳚स्ति । \newline
7. शा॒स्ति॒ कः कः शा᳚स्ति शास्ति॒ कः । \newline
8. क स्ते॑ ते॒ कः क स्ते᳚ । \newline
9. ते॒ गात्रा॑णि॒ गात्रा॑णि ते ते॒ गात्रा॑णि । \newline
10. गात्रा॑णि शिम्यति शिम्यति॒ गात्रा॑णि॒ गात्रा॑णि शिम्यति । \newline
11. शि॒म्य॒तीति॑ शिम्यति । \newline
12. क उ॑ वु॒ कः क उ॑ । \newline
13. उ॒ ते॒ त॒ उ॒ वु॒ ते॒ । \newline
14. ते॒ श॒मि॒ता श॑मि॒ता ते॑ ते शमि॒ता । \newline
15. श॒मि॒ता क॒विः क॒विः श॑मि॒ता श॑मि॒ता क॒विः । \newline
16. क॒विरिति॑ क॒विः । \newline
17. ऋ॒तव॑ स्ते त ऋ॒तव॑ ऋ॒तव॑ स्ते । \newline
18. त॒ ऋ॒तु॒ध र्‌तु॒धा ते॑ त ऋतु॒धा । \newline
19. ऋ॒तु॒धा परुः॒ परुर्॑. ऋतु॒ध र्‌तु॒धा परुः॑ । \newline
20. ऋ॒तु॒धेत्यृ॑तु - धा । \newline
21. परुः॑ शमि॒तारः॑ शमि॒तारः॒ परुः॒ परुः॑ शमि॒तारः॑ । \newline
22. श॒मि॒तारो॒ वि वि श॑मि॒तारः॑ शमि॒तारो॒ वि । \newline
23. वि शा॑सतु शासतु॒ वि वि शा॑सतु । \newline
24. शा॒स॒त्विति॑ शासतु । \newline
25. सं॒ॅव॒थ्स॒रस्य॒ धाय॑सा॒ धाय॑सा संॅवथ्स॒रस्य॑ संॅवथ्स॒रस्य॒ धाय॑सा । \newline
26. सं॒ॅव॒थ्स॒रस्येति॑ सं - व॒थ्स॒रस्य॑ । \newline
27. धाय॑सा॒ शिमी॑भिः॒ शिमी॑भि॒र् धाय॑सा॒ धाय॑सा॒ शिमी॑भिः । \newline
28. शिमी॑भिः शिम्यन्तु शिम्यन्तु॒ शिमी॑भिः॒ शिमी॑भिः शिम्यन्तु । \newline
29. शि॒म्य॒न्तु॒ त्वा॒ त्वा॒ शि॒म्य॒न्तु॒ शि॒म्य॒न्तु॒ त्वा॒ । \newline
30. त्वेति॑ त्वा । \newline
31. दैव्या॑ अद्ध्व॒र्यवो᳚ ऽद्ध्व॒र्यवो॒ दैव्या॒ दैव्या॑ अद्ध्व॒र्यवः॑ । \newline
32. अ॒द्ध्व॒र्यव॑ स्त्वा त्वा ऽद्ध्व॒र्यवो᳚ ऽद्ध्व॒र्यव॑ स्त्वा । \newline
33. त्वा॒ च्छ्यन्तु॒ छ्यन्तु॑ त्वा त्वा॒ च्छ्यन्तु॑ । \newline
34. छ्यन्तु॒ वि वि च्छ्यन्तु॒ छ्यन्तु॒ वि । \newline
35. वि च॑ च॒ वि वि च॑ । \newline
36. च॒ शा॒स॒तु॒ शा॒स॒तु॒ च॒ च॒ शा॒स॒तु॒ । \newline
37. शा॒स॒त्विति॑ शासतु । \newline
38. गात्रा॑णि पर्व॒शः प॑र्व॒शो गात्रा॑णि॒ गात्रा॑णि पर्व॒शः । \newline
39. प॒र्व॒श स्ते॑ ते पर्व॒शः प॑र्व॒श स्ते᳚ । \newline
40. प॒र्व॒श इति॑ पर्व - शः । \newline
41. ते॒ शिमाः॒ शिमा᳚ स्ते ते॒ शिमाः᳚ । \newline
42. शिमाः᳚ कृण्वन्तु कृण्वन्तु॒ शिमाः॒ शिमाः᳚ कृण्वन्तु । \newline
43. कृ॒ण्व॒न्तु॒ शिम्य॑न्तः॒ शिम्य॑न्तः कृण्वन्तु कृण्वन्तु॒ शिम्य॑न्तः । \newline
44. शिम्य॑न्त॒ इति॒ शिम्य॑न्तः । \newline
45. अ॒र्द्ध॒मा॒साः परूꣳ॑षि॒ परूꣳ॑ष्य र्द्धमा॒सा अ॑र्द्धमा॒साः परूꣳ॑षि । \newline
46. अ॒र्द्ध॒मा॒सा इत्य॑र्द्ध - मा॒साः । \newline
47. परूꣳ॑षि ते ते॒ परूꣳ॑षि॒ परूꣳ॑षि ते । \newline
48. ते॒ मासा॒ मासा᳚ स्ते ते॒ मासाः᳚ । \newline
49. मासा᳚ श्छ्यन्तु छ्यन्तु॒ मासा॒ मासा᳚ श्छ्यन्तु । \newline
50. छ्य॒न्तु॒ शिम्य॑न्तः॒ शिम्य॑न्त श्छ्यन्तु छ्यन्तु॒ शिम्य॑न्तः । \newline
51. शिम्य॑न्त॒ इति॒ शिम्य॑न्तः । \newline
52. अ॒हो॒रा॒त्राणि॑ म॒रुतो॑ म॒रुतो॑ ऽहोरा॒त्राण्य॑ होरा॒त्राणि॑ म॒रुतः॑ । \newline
53. अ॒हो॒रा॒त्राणीत्य॑हः - रा॒त्राणि॑ । \newline
54. म॒रुतो॒ विलि॑ष्टं॒ ॅविलि॑ष्टम् म॒रुतो॑ म॒रुतो॒ विलि॑ष्टम् । \newline
55. विलि॑ष्टꣳ सूदयन्तु सूदयन्तु॒ विलि॑ष्टं॒ ॅविलि॑ष्टꣳ सूदयन्तु । \newline
56. विलि॑ष्ट॒मिति॒ वि - लि॒ष्ट॒म् । \newline

\textbf{Ghana Paata } \newline

1. क स्त्वा᳚ त्वा॒ कः क स्त्वा᳚ छ्यति छ्यति त्वा॒ कः क स्त्वा᳚ छ्यति । \newline
2. त्वा॒ छ्य॒ति॒ छ्य॒ति॒ त्वा॒ त्वा॒ छ्य॒ति॒ कः क श्छ्य॑ति त्वा त्वा छ्यति॒ कः । \newline
3. छ्य॒ति॒ कः क श्छ्य॑ति छ्यति॒ क स्त्वा᳚ त्वा॒ क श्छ्य॑ति छ्यति॒ क स्त्वा᳚ । \newline
4. क स्त्वा᳚ त्वा॒ कः क स्त्वा॒ वि वि त्वा॒ कः क स्त्वा॒ वि । \newline
5. त्वा॒ वि वि त्वा᳚ त्वा॒ वि शा᳚स्ति शास्ति॒ वि त्वा᳚ त्वा॒ वि शा᳚स्ति । \newline
6. वि शा᳚स्ति शास्ति॒ वि वि शा᳚स्ति॒ कः कः शा᳚स्ति॒ वि वि शा᳚स्ति॒ कः । \newline
7. शा॒स्ति॒ कः कः शा᳚स्ति शास्ति॒ क स्ते॑ ते॒ कः शा᳚स्ति शास्ति॒ क स्ते᳚ । \newline
8. क स्ते॑ ते॒ कः क स्ते॒ गात्रा॑णि॒ गात्रा॑णि ते॒ कः क स्ते॒ गात्रा॑णि । \newline
9. ते॒ गात्रा॑णि॒ गात्रा॑णि ते ते॒ गात्रा॑णि शिम्यति शिम्यति॒ गात्रा॑णि ते ते॒ गात्रा॑णि शिम्यति । \newline
10. गात्रा॑णि शिम्यति शिम्यति॒ गात्रा॑णि॒ गात्रा॑णि शिम्यति । \newline
11. शि॒म्य॒तीति॑ शिम्यति । \newline
12. क उ॑ वु॒ कः क उ॑ ते त उ॒ कः क उ॑ ते । \newline
13. उ॒ ते॒ त॒ उ॒ वु॒ ते॒ श॒मि॒ता श॑मि॒ता त॑ उ वु ते शमि॒ता । \newline
14. ते॒ श॒मि॒ता श॑मि॒ता ते॑ ते शमि॒ता क॒विः क॒विः श॑मि॒ता ते॑ ते शमि॒ता क॒विः । \newline
15. श॒मि॒ता क॒विः क॒विः श॑मि॒ता श॑मि॒ता क॒विः । \newline
16. क॒विरिति॑ क॒विः । \newline
17. ऋ॒तव॑ स्ते त ऋ॒तव॑ ऋ॒तव॑ स्त ऋतु॒ध र्‌तु॒धा त॑ ऋ॒तव॑ ऋ॒तव॑ स्त ऋतु॒धा । \newline
18. त॒ ऋ॒तु॒ध र्‌तु॒धा ते॑ त ऋतु॒धा परुः॒ परुर्॑. ऋतु॒धा ते॑ त ऋतु॒धा परुः॑ । \newline
19. ऋ॒तु॒धा परुः॒ परुर्॑. ऋतु॒ध र्‌तु॒धा परुः॑ शमि॒तारः॑ शमि॒तारः॒ परुर्॑. ऋतु॒ध र्‌तु॒धा परुः॑ शमि॒तारः॑ । \newline
20. ऋ॒तु॒धेत्यृ॑तु - धा । \newline
21. परुः॑ शमि॒तारः॑ शमि॒तारः॒ परुः॒ परुः॑ शमि॒तारो॒ वि वि श॑मि॒तारः॒ परुः॒ परुः॑ शमि॒तारो॒ वि । \newline
22. श॒मि॒तारो॒ वि वि श॑मि॒तारः॑ शमि॒तारो॒ वि शा॑सतु शासतु॒ वि श॑मि॒तारः॑ शमि॒तारो॒ वि शा॑सतु । \newline
23. वि शा॑सतु शासतु॒ वि वि शा॑सतु । \newline
24. शा॒स॒त्विति॑ शासतु । \newline
25. सं॒ॅव॒थ्स॒रस्य॒ धाय॑सा॒ धाय॑सा संॅवथ्स॒रस्य॑ संॅवथ्स॒रस्य॒ धाय॑सा॒ शिमी॑भिः॒ शिमी॑भि॒र् धाय॑सा संॅवथ्स॒रस्य॑ संॅवथ्स॒रस्य॒ धाय॑सा॒ शिमी॑भिः । \newline
26. सं॒ॅव॒थ्स॒रस्येति॑ सं - व॒थ्स॒रस्य॑ । \newline
27. धाय॑सा॒ शिमी॑भिः॒ शिमी॑भि॒र् धाय॑सा॒ धाय॑सा॒ शिमी॑भिः शिम्यन्तु शिम्यन्तु॒ शिमी॑भि॒र् धाय॑सा॒ धाय॑सा॒ शिमी॑भिः शिम्यन्तु । \newline
28. शिमी॑भिः शिम्यन्तु शिम्यन्तु॒ शिमी॑भिः॒ शिमी॑भिः शिम्यन्तु त्वा त्वा शिम्यन्तु॒ शिमी॑भिः॒ शिमी॑भिः शिम्यन्तु त्वा । \newline
29. शि॒म्य॒न्तु॒ त्वा॒ त्वा॒ शि॒म्य॒न्तु॒ शि॒म्य॒न्तु॒ त्वा॒ । \newline
30. त्वेति॑ त्वा । \newline
31. दैव्या॑ अद्ध्व॒र्यवो᳚ ऽद्ध्व॒र्यवो॒ दैव्या॒ दैव्या॑ अद्ध्व॒र्यव॑ स्त्वा त्वा ऽद्ध्व॒र्यवो॒ दैव्या॒ दैव्या॑ अद्ध्व॒र्यव॑ स्त्वा । \newline
32. अ॒द्ध्व॒र्यव॑ स्त्वा त्वा ऽद्ध्व॒र्यवो᳚ ऽद्ध्व॒र्यव॑ स्त्वा॒ च्छ्यन्तु॒ छ्यन्तु॑ त्वा ऽद्ध्व॒र्यवो᳚ ऽद्ध्व॒र्यव॑ स्त्वा॒ च्छ्यन्तु॑ । \newline
33. त्वा॒ च्छ्यन्तु॒ छ्यन्तु॑ त्वा त्वा॒ च्छ्यन्तु॒ वि वि च्छ्यन्तु॑ त्वा त्वा॒ च्छ्यन्तु॒ वि । \newline
34. छ्यन्तु॒ वि वि च्छ्यन्तु॒ छ्यन्तु॒ वि च॑ च॒ वि च्छ्यन्तु॒ छ्यन्तु॒ वि च॑ । \newline
35. वि च॑ च॒ वि वि च॑ शासतु शासतु च॒ वि वि च॑ शासतु । \newline
36. च॒ शा॒स॒तु॒ शा॒स॒तु॒ च॒ च॒ शा॒स॒तु॒ । \newline
37. शा॒स॒त्विति॑ शासतु । \newline
38. गात्रा॑णि पर्व॒शः प॑र्व॒शो गात्रा॑णि॒ गात्रा॑णि पर्व॒श स्ते॑ ते पर्व॒शो गात्रा॑णि॒ गात्रा॑णि पर्व॒श स्ते᳚ । \newline
39. प॒र्व॒श स्ते॑ ते पर्व॒शः प॑र्व॒श स्ते॒ शिमाः॒ शिमा᳚ स्ते पर्व॒शः प॑र्व॒श स्ते॒ शिमाः᳚ । \newline
40. प॒र्व॒श इति॑ पर्व - शः । \newline
41. ते॒ शिमाः॒ शिमा᳚ स्ते ते॒ शिमाः᳚ कृण्वन्तु कृण्वन्तु॒ शिमा᳚ स्ते ते॒ शिमाः᳚ कृण्वन्तु । \newline
42. शिमाः᳚ कृण्वन्तु कृण्वन्तु॒ शिमाः॒ शिमाः᳚ कृण्वन्तु॒ शिम्य॑न्तः॒ शिम्य॑न्तः कृण्वन्तु॒ शिमाः॒ शिमाः᳚ कृण्वन्तु॒ शिम्य॑न्तः । \newline
43. कृ॒ण्व॒न्तु॒ शिम्य॑न्तः॒ शिम्य॑न्तः कृण्वन्तु कृण्वन्तु॒ शिम्य॑न्तः । \newline
44. शिम्य॑न्त॒ इति॒ शिम्य॑न्तः । \newline
45. अ॒र्द्ध॒मा॒साः परूꣳ॑षि॒ परूꣳ॑ ष्यर्द्धमा॒सा अ॑र्द्धमा॒साः परूꣳ॑षि ते ते॒ परूꣳ॑ ष्यर्द्धमा॒सा अ॑र्द्धमा॒साः परूꣳ॑षि ते । \newline
46. अ॒र्द्ध॒मा॒सा इत्य॑र्द्ध - मा॒साः । \newline
47. परूꣳ॑षि ते ते॒ परूꣳ॑षि॒ परूꣳ॑षि ते॒ मासा॒ मासा᳚ स्ते॒ परूꣳ॑षि॒ परूꣳ॑षि ते॒ मासाः᳚ । \newline
48. ते॒ मासा॒ मासा᳚ स्ते ते॒ मासा᳚ श्छ्यन्तु छ्यन्तु॒ मासा᳚ स्ते ते॒ मासा᳚ श्छ्यन्तु । \newline
49. मासा᳚ श्छ्यन्तु छ्यन्तु॒ मासा॒ मासा᳚ श्छ्यन्तु॒ शिम्य॑न्तः॒ शिम्य॑न्त श्छ्यन्तु॒ मासा॒ मासा᳚ श्छ्यन्तु॒ शिम्य॑न्तः । \newline
50. छ्य॒न्तु॒ शिम्य॑न्तः॒ शिम्य॑न्त श्छ्यन्तु छ्यन्तु॒ शिम्य॑न्तः । \newline
51. शिम्य॑न्त॒ इति॒ शिम्य॑न्तः । \newline
52. अ॒हो॒रा॒त्राणि॑ म॒रुतो॑ म॒रुतो॑ ऽहोरा॒त्रा ण्य॑होरा॒त्राणि॑ म॒रुतो॒ विलि॑ष्टं॒ ॅविलि॑ष्टम् म॒रुतो॑ ऽहोरा॒त्रा ण्य॑होरा॒त्राणि॑ म॒रुतो॒ विलि॑ष्टम् । \newline
53. अ॒हो॒रा॒त्राणीत्य॑हः - रा॒त्राणि॑ । \newline
54. म॒रुतो॒ विलि॑ष्टं॒ ॅविलि॑ष्टम् म॒रुतो॑ म॒रुतो॒ विलि॑ष्टꣳ सूदयन्तु सूदयन्तु॒ विलि॑ष्टम् म॒रुतो॑ म॒रुतो॒ विलि॑ष्टꣳ सूदयन्तु । \newline
55. विलि॑ष्टꣳ सूदयन्तु सूदयन्तु॒ विलि॑ष्टं॒ ॅविलि॑ष्टꣳ सूदयन्तु ते ते सूदयन्तु॒ विलि॑ष्टं॒ ॅविलि॑ष्टꣳ सूदयन्तु ते । \newline
56. विलि॑ष्ट॒मिति॒ वि - लि॒ष्ट॒म् । \newline
\pagebreak
\markright{ TS 5.2.12.2  \hfill https://www.vedavms.in \hfill}

\section{ TS 5.2.12.2 }

\textbf{TS 5.2.12.2 } \newline
\textbf{Samhita Paata} \newline

सूदयन्तु ते ॥ पृ॒थि॒वी ते॒ ऽन्तरि॑क्षेण वा॒युश्छि॒द्रं भि॑षज्यतु । द्यौस्ते॒ नक्ष॑त्रैः स॒ह रू॒पं कृ॑णोतु साधु॒या ॥ शं ते॒ परे᳚भ्यो॒ गात्रे᳚भ्यः॒ शम॒स्त्वव॑रेभ्यः । शम॒स्थभ्यो॑ म॒ज्जभ्यः॒ शमु॑ ते त॒नुवे॑ भुवत् ॥ \newline

\textbf{Pada Paata} \newline

सू॒द॒य॒न्तु॒ । ते॒ ॥ पृ॒थि॒वी । ते॒ । अ॒न्तरि॑क्षेण । वा॒युः । छि॒द्रम् । भि॒ष॒ज्य॒तु॒ ॥ द्यौः । ते॒ । नक्ष॑त्रैः । स॒ह । रू॒पम् । कृ॒णो॒तु॒ । सा॒धु॒या ॥ शम् । ते॒ । परे᳚भ्यः । गात्रे᳚भ्यः । शम् । अ॒स्तु॒ । अव॑रेभ्यः ॥ शम् । अ॒स्थभ्य॒ इत्य॒स्थ - भ्यः॒ । म॒ज्जभ्य॒ इति॑ म॒ज्ज - भ्यः॒ । शम् । उ॒ । ते॒ । त॒नुवे᳚ । भु॒व॒त् ॥  \newline


\textbf{Krama Paata} \newline

सू॒द॒य॒न्तु॒ ते॒ । त॒ इति॑ ते ॥ पृ॒थि॒वी ते᳚ । ते॒ऽन्तरि॑क्षेण । अ॒न्तरि॑क्षेण वा॒युः । वा॒युश्छि॒द्रम् । छि॒द्रम् भि॑षज्यतु । भि॒ष॒ज्य॒त्विति॑ भिषज्यतु ॥ द्यौस्ते᳚ । ते॒ नक्ष॑त्रैः । नक्ष॑त्रैः स॒ह । स॒ह रू॒पम् । रू॒पम् कृ॑णोतु । कृ॒णो॒तु॒ सा॒धु॒या । सा॒धु॒येति॑ साधु॒या ॥ शम् ते᳚ । ते॒ परे᳚भ्यः । परे᳚भ्यो॒ गात्रे᳚भ्यः । गात्रे᳚भ्यः॒ शम् । शम॑स्तु । अ॒स्त्वव॑रेभ्यः । अव॑रेभ्य॒ इत्यव॑रेभ्यः ॥ शम॒स्थभ्यः॑ । अ॒स्थभ्यो॑ म॒ज्जभ्यः॑ । अ॒स्थभ्य॒ इत्य॒स्थ - भ्यः॒ । म॒ज्जभ्यः॒ शम् । म॒ज्जभ्य॒ इति॑ म॒ज्ज - भ्यः॒ । शमु॑ । उ॒ ते॒ । ते॒ त॒नुवे᳚ । त॒नुवे॑ भुवत् । भु॒व॒दिति॑ भुवत् । \newline

\textbf{Jatai Paata} \newline

1. सू॒द॒य॒न्तु॒ ते॒ ते॒ सू॒द॒य॒न्तु॒ सू॒द॒य॒न्तु॒ ते॒ । \newline
2. त॒ इति॑ ते । \newline
3. पृ॒थि॒वी ते॑ ते पृथि॒वी पृ॑थि॒वी ते᳚ । \newline
4. ते॒ ऽन्तरि॑क्षेणा॒ न्तरि॑क्षेण ते ते॒ ऽन्तरि॑क्षेण । \newline
5. अ॒न्तरि॑क्षेण वा॒युर् वा॒यु र॒न्तरि॑क्षेणा॒ न्तरि॑क्षेण वा॒युः । \newline
6. वा॒यु श्छि॒द्रम् छि॒द्रं ॅवा॒युर् वा॒यु श्छि॒द्रम् । \newline
7. छि॒द्रम् भि॑षज्यतु भिषज्यतु छि॒द्रम् छि॒द्रम् भि॑षज्यतु । \newline
8. भि॒ष॒ज्य॒त्विति॑ भिषज्यतु । \newline
9. द्यौ स्ते॑ ते॒ द्यौर् द्यौ स्ते᳚ । \newline
10. ते॒ नक्ष॑त्रै॒र् नक्ष॑त्रै स्ते ते॒ नक्ष॑त्रैः । \newline
11. नक्ष॑त्रैः स॒ह स॒ह नक्ष॑त्रै॒र् नक्ष॑त्रैः स॒ह । \newline
12. स॒ह रू॒पꣳ रू॒पꣳ स॒ह स॒ह रू॒पम् । \newline
13. रू॒पम् कृ॑णोतु कृणोतु रू॒पꣳ रू॒पम् कृ॑णोतु । \newline
14. कृ॒णो॒तु॒ सा॒धु॒या सा॑धु॒या कृ॑णोतु कृणोतु साधु॒या । \newline
15. सा॒धु॒येति॑ साधु॒या । \newline
16. शम् ते॑ ते॒ शꣳ शम् ते᳚ । \newline
17. ते॒ परे᳚भ्यः॒ परे᳚भ्य स्ते ते॒ परे᳚भ्यः । \newline
18. परे᳚भ्यो॒ गात्रे᳚भ्यो॒ गात्रे᳚भ्यः॒ परे᳚भ्यः॒ परे᳚भ्यो॒ गात्रे᳚भ्यः । \newline
19. गात्रे᳚भ्यः॒ शꣳ शम् गात्रे᳚भ्यो॒ गात्रे᳚भ्यः॒ शम् । \newline
20. श म॑स्त्वस्तु॒ शꣳ श म॑स्तु । \newline
21. अ॒स्त्व व॑रे॒भ्यो ऽव॑रेभ्यो अस्त्व॒स्त्व व॑रेभ्यः । \newline
22. अव॑रेभ्य॒ इत्यव॑रेभ्यः । \newline
23. श म॒स्थभ्यो॒ ऽस्थभ्यः॒ शꣳ श म॒स्थभ्यः॑ । \newline
24. अ॒स्थभ्यो॑ म॒ज्जभ्यो॑ म॒ज्जभ्यो॒ ऽस्थभ्यो॒ ऽस्थभ्यो॑ म॒ज्जभ्यः॑ । \newline
25. अ॒स्थभ्य॒ इत्य॒स्थ - भ्यः॒ । \newline
26. म॒ज्जभ्यः॒ शꣳ शम् म॒ज्जभ्यो॑ म॒ज्जभ्यः॒ शम् । \newline
27. म॒ज्जभ्य॒ इति॑ म॒ज्ज - भ्यः॒ । \newline
28. श मु॑ वु॒ शꣳ श मु॑ । \newline
29. उ॒ ते॒ त॒ उ॒ वु॒ ते॒ । \newline
30. ते॒ त॒नुवे॑ त॒नुवे॑ ते ते त॒नुवे᳚ । \newline
31. त॒नुवे॑ भुवद् भुवत् त॒नुवे॑ त॒नुवे॑ भुवत् । \newline
32. भु॒व॒दिति॑ भुवत् । \newline

\textbf{Ghana Paata } \newline

1. सू॒द॒य॒न्तु॒ ते॒ ते॒ सू॒द॒य॒न्तु॒ सू॒द॒य॒न्तु॒ ते॒ । \newline
2. त॒ इति॑ ते । \newline
3. पृ॒थि॒वी ते॑ ते पृथि॒वी पृ॑थि॒वी ते॒ ऽन्तरि॑क्षेणा॒ न्तरि॑क्षेण ते पृथि॒वी पृ॑थि॒वी ते॒ ऽन्तरि॑क्षेण । \newline
4. ते॒ ऽन्तरि॑क्षेणा॒ न्तरि॑क्षेण ते ते॒ ऽन्तरि॑क्षेण वा॒युर् वा॒यु र॒न्तरि॑क्षेण ते ते॒ ऽन्तरि॑क्षेण वा॒युः । \newline
5. अ॒न्तरि॑क्षेण वा॒युर् वा॒यु र॒न्तरि॑क्षेणा॒ न्तरि॑क्षेण वा॒यु श्छि॒द्रम् छि॒द्रं ॅवा॒यु र॒न्तरि॑क्षेणा॒ न्तरि॑क्षेण वा॒यु श्छि॒द्रम् । \newline
6. वा॒यु श्छि॒द्रम् छि॒द्रं ॅवा॒युर् वा॒यु श्छि॒द्रम् भि॑षज्यतु भिषज्यतु छि॒द्रं ॅवा॒युर् वा॒यु श्छि॒द्रम् भि॑षज्यतु । \newline
7. छि॒द्रम् भि॑षज्यतु भिषज्यतु छि॒द्रम् छि॒द्रम् भि॑षज्यतु । \newline
8. भि॒ष॒ज्य॒त्विति॑ भिषज्यतु । \newline
9. द्यौ स्ते॑ ते॒ द्यौर् द्यौ स्ते॒ नक्ष॑त्रै॒र् नक्ष॑त्रै स्ते॒ द्यौर् द्यौ स्ते॒ नक्ष॑त्रैः । \newline
10. ते॒ नक्ष॑त्रै॒र् नक्ष॑त्रै स्ते ते॒ नक्ष॑त्रैः स॒ह स॒ह नक्ष॑त्रै स्ते ते॒ नक्ष॑त्रैः स॒ह । \newline
11. नक्ष॑त्रैः स॒ह स॒ह नक्ष॑त्रै॒र् नक्ष॑त्रैः स॒ह रू॒पꣳ रू॒पꣳ स॒ह नक्ष॑त्रै॒र् नक्ष॑त्रैः स॒ह रू॒पम् । \newline
12. स॒ह रू॒पꣳ रू॒पꣳ स॒ह स॒ह रू॒पम् कृ॑णोतु कृणोतु रू॒पꣳ स॒ह स॒ह रू॒पम् कृ॑णोतु । \newline
13. रू॒पम् कृ॑णोतु कृणोतु रू॒पꣳ रू॒पम् कृ॑णोतु साधु॒या सा॑धु॒या कृ॑णोतु रू॒पꣳ रू॒पम् कृ॑णोतु साधु॒या । \newline
14. कृ॒णो॒तु॒ सा॒धु॒या सा॑धु॒या कृ॑णोतु कृणोतु साधु॒या । \newline
15. सा॒धु॒येति॑ साधु॒या । \newline
16. शम् ते॑ ते॒ शꣳ शम् ते॒ परे᳚भ्यः॒ परे᳚भ्य स्ते॒ शꣳ शम् ते॒ परे᳚भ्यः । \newline
17. ते॒ परे᳚भ्यः॒ परे᳚भ्य स्ते ते॒ परे᳚भ्यो॒ गात्रे᳚भ्यो॒ गात्रे᳚भ्यः॒ परे᳚भ्य स्ते ते॒ परे᳚भ्यो॒ गात्रे᳚भ्यः । \newline
18. परे᳚भ्यो॒ गात्रे᳚भ्यो॒ गात्रे᳚भ्यः॒ परे᳚भ्यः॒ परे᳚भ्यो॒ गात्रे᳚भ्यः॒ शꣳ शम् गात्रे᳚भ्यः॒ परे᳚भ्यः॒ परे᳚भ्यो॒ गात्रे᳚भ्यः॒ शम् । \newline
19. गात्रे᳚भ्यः॒ शꣳ शम् गात्रे᳚भ्यो॒ गात्रे᳚भ्यः॒ श म॑स्त्वस्तु॒ शम् गात्रे᳚भ्यो॒ गात्रे᳚भ्यः॒ श म॑स्तु । \newline
20. श म॑स्त्वस्तु॒ शꣳ श म॒स्त्वव॑रे॒भ्यो ऽव॑रेभ्यो अस्तु॒ शꣳ श म॒स्त्वव॑रेभ्यः । \newline
21. अ॒स्त्वव॑रे॒भ्यो ऽव॑रेभ्यो अस्त्व॒ स्त्वव॑रेभ्यः । \newline
22. अव॑रेभ्य॒ इत्यव॑रेभ्यः । \newline
23. श म॒स्थभ्यो॒ ऽस्थभ्यः॒ शꣳ श म॒स्थभ्यो॑ म॒ज्जभ्यो॑ म॒ज्जभ्यो॒ ऽस्थभ्यः॒ शꣳ श म॒स्थभ्यो॑ म॒ज्जभ्यः॑ । \newline
24. अ॒स्थभ्यो॑ म॒ज्जभ्यो॑ म॒ज्जभ्यो॒ ऽस्थभ्यो॒ ऽस्थभ्यो॑ म॒ज्जभ्यः॒ शꣳ शम् म॒ज्जभ्यो॒ ऽस्थभ्यो॒ ऽस्थभ्यो॑ म॒ज्जभ्यः॒ शम् । \newline
25. अ॒स्थभ्य॒ इत्य॒स्थ - भ्यः॒ । \newline
26. म॒ज्जभ्यः॒ शꣳ शम् म॒ज्जभ्यो॑ म॒ज्जभ्यः॒ श मु॑ वु॒ शम् म॒ज्जभ्यो॑ म॒ज्जभ्यः॒ श मु॑ । \newline
27. म॒ज्जभ्य॒ इति॑ म॒ज्ज - भ्यः॒ । \newline
28. श मु॑ वु॒ शꣳ श मु॑ ते त उ॒ शꣳ श मु॑ ते । \newline
29. उ॒ ते॒ त॒ उ॒ वु॒ ते॒ त॒नुवे॑ त॒नुवे॑ त उ वु ते त॒नुवे᳚ । \newline
30. ते॒ त॒नुवे॑ त॒नुवे॑ ते ते त॒नुवे॑ भुवद् भुवत् त॒नुवे॑ ते ते त॒नुवे॑ भुवत् । \newline
31. त॒नुवे॑ भुवद् भुवत् त॒नुवे॑ त॒नुवे॑ भुवत् । \newline
32. भु॒व॒दिति॑ भुवत् । \newline
\pagebreak


\end{document}