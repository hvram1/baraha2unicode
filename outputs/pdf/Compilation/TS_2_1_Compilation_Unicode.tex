\documentclass[17pt]{extarticle}
\usepackage{babel}
\usepackage{fontspec}
\usepackage{polyglossia}
\usepackage{extsizes}

\usepackage{color}   %May be necessary if you want to color links
\usepackage{hyperref}
\hypersetup{
    colorlinks=true, %set true if you want colored links
    linktoc=all,     %set to all if you want both sections and subsections linked
    linkcolor=black,  %choose some color if you want links to stand out
}

\setmainlanguage{sanskrit}
\setotherlanguages{english} %% or other languages
\setlength{\parindent}{0pt}
\pagestyle{myheadings}
\newfontfamily\devanagarifont[Script=Devanagari]{AdishilaVedic}
\renewcommand{\theHsection}{\thepart.section.\thesection}

\newcommand{\VAR}[1]{}
\newcommand{\BLOCK}[1]{}




\begin{document}
\begin{titlepage}
    \begin{center}
 
\begin{sanskrit}
    { \Large
    कृष्ण यजुर्वेदीय तैत्तिरीय संहिता,पद,जटा,घन पाठः 
    }
    \\
    \vspace{2.5cm}
    \mbox{ \Large
    2.1      द्वितीयकाण्डे  प्रथमः प्रश्नः - पशुविधानं   }
\end{sanskrit}
\end{center}

\end{titlepage}
\tableofcontents
\phantomsection
\pagebreak

\markright{ TS 2.1.1.1  \hfill https://www.vedavms.in \hfill}

\section{ TS 2.1.1.1 }

\textbf{TS 2.1.1.1 } \newline
\textbf{Samhita Paata} \newline

वा॒य॒व्यꣳ॑ श्वे॒तमा ल॑भेत॒ भूति॑कामॊ  वा॒युर्वै क्षेपि॑ष्ठा दे॒वता॑वा॒युमे॒व स्वेन॑ भाग॒धेये॒नोप॑ धावति॒ स ए॒वैनं॒ भूतिं॑ गमयति॒भव॑त्ये॒वा-ति॑क्षिप्रा दे॒वतेत्या॑हुः॒ सैन॑मीश्व॒रा प्र॒दह॒ इत्ये॒तमे॒व सन्तं॑ ॅवा॒यवे॑ नि॒युत्व॑त॒ आ ल॑भेत नि॒युद्वा अ॑स्य॒ धृति॑र्द्धृ॒त ए॒व भूति॒मुपै॒त्य प्र॑दाहाय॒ भव॑त्ये॒व - [  ] \newline

\textbf{Pada Paata} \newline

वा॒य॒व्य᳚म् । श्वे॒तम् । एति॑ । ल॒भे॒त॒ । भूति॑काम॒ इति॒ भूति॑-का॒मः॒ । वा॒युः । वै । क्षेपि॑ष्ठा । दे॒वता᳚ । वा॒युम् । ए॒व । स्वेन॑ । भा॒ग॒धेये॒नेति॑ भाग - धेये॑न । उपेति॑ । धा॒व॒ति॒ । सः । ए॒व । ए॒न॒म् । भूति᳚म् । ग॒म॒य॒ति॒ । भव॑ति । ए॒व । अति॑क्षि॒प्रेत्यति॑ - क्षि॒प्रा॒ । दे॒वता᳚ । इति॑ । आ॒हुः॒ । सा । ए॒न॒म् । ई॒श्व॒रा । प्र॒दह॒ इति॑ प्र-दहः॑ । इति॑ । ए॒तम् । ए॒व । सन्त᳚म् । वा॒यवे᳚ । नि॒युत्व॑त॒ इति॑ नि-युत्व॑ते । एति॑ । ल॒भे॒त॒ । नि॒युदिति॑ नि - युत् । वै । अ॒स्य॒ । धृतिः॑ । धृ॒तः । ए॒व । भूति᳚म् । उपेति॑ । ए॒ति॒ । अप्र॑दाहा॒येत्यप्र॑ - दा॒हा॒य॒ । भव॑ति । ए॒व ।  \newline


\textbf{Krama Paata} \newline

वा॒य॒व्यꣳ॑ श्वे॒तम् । श्वे॒तमा । आ ल॑भेत । ल॒भे॒त॒ भूति॑कामः । भूति॑कामो वा॒युः । भूति॑काम॒ इति॒ भूति॑ - का॒मः॒ । वा॒युर् वै । वै क्षेपि॑ष्ठा । क्षेपि॑ष्ठा दे॒वता᳚ । दे॒वता॑ वा॒युम् । वा॒युमे॒व । ए॒व स्वेन॑ । स्वेन॑ भाग॒धेये॑न । भा॒ग॒धेये॒नोप॑ । भा॒ग॒धेये॒नेति॑ भाग - धेये॑न । उप॑ धावति । धा॒व॒ति॒ सः । स ए॒व । ए॒वैन᳚म् । ए॒न॒म् भूति᳚म् । भूति॑म् गमयति । ग॒म॒य॒ति॒ भव॑ति । भव॑त्ये॒व । ए॒वाति॑क्षिप्रा । अति॑क्षिप्रा दे॒वता᳚ । अति॑क्षि॒प्रेत्यति॑ - क्षि॒प्रा॒ । दे॒वतेति॑ । इत्या॑हुः । आ॒हुः॒ सा । सैन᳚म् । ए॒न॒मी॒श्व॒रा । ई॒श्व॒रा प्र॒दहः॑ । प्र॒दह॒ इति॑ । प्र॒दह॒ इति॑ प्र - दहः॑ । इत्ये॒तम् । ए॒तमे॒व । ए॒व सन्त᳚म् । सन्तं॑ ॅवा॒यवे᳚ । वा॒यवे॑ नि॒युत्व॑ते । नि॒युत्व॑त॒ आ । नि॒युत्व॑त॒ इति॑ नि - युत्व॑ते । आ ल॑भेत । ल॒भे॒त॒ नि॒युत् । नि॒युद् वै । नि॒युदिति॑ नि - युत् । वा अ॑स्य । अ॒स्य॒ धृतिः॑ । धृति॑र् धृ॒तः । धृ॒त ए॒व । ए॒व भूति᳚म् । भूति॒मुप॑ । उपै॑ति । ए॒त्यप्र॑दाहाय । अप्र॑दाहाय॒ भव॑ति । अप्र॑दाहा॒येत्यप्र॑ - दा॒हा॒य॒ । भव॑त्ये॒व । ए॒व वा॒यवे᳚ \newline

\textbf{Jatai Paata} \newline

1. वा॒य॒व्यꣳ॑ श्वे॒तꣳ श्वे॒तं ॅवा॑य॒व्यं॑ ॅवाय॒व्यꣳ॑ श्वे॒तम् । \newline
2. श्वे॒त मा श्वे॒तꣳ श्वे॒त मा । \newline
3. आ ल॑भेत लभे॒ता ल॑भेत । \newline
4. ल॒भे॒त॒ भूति॑कामो॒ भूति॑कामो लभेत लभेत॒ भूति॑कामः । \newline
5. भूति॑कामो वा॒युर् वा॒युर् भूति॑कामो॒ भूति॑कामो वा॒युः । \newline
6. भूति॑काम॒ इति॒ भूति॑ - का॒मः॒ । \newline
7. वा॒युर् वै वै वा॒युर् वा॒युर् वै । \newline
8. वै क्षेपि॑ष्ठा॒ क्षेपि॑ष्ठा॒ वै वै क्षेपि॑ष्ठा । \newline
9. क्षेपि॑ष्ठा दे॒वता॑ दे॒वता॒ क्षेपि॑ष्ठा॒ क्षेपि॑ष्ठा दे॒वता᳚ । \newline
10. दे॒वता॑ वा॒युं ॅवा॒युम् दे॒वता॑ दे॒वता॑ वा॒युम् । \newline
11. वा॒यु मे॒वैव वा॒युं ॅवा॒यु मे॒व । \newline
12. ए॒व स्वेन॒ स्वेनै॒ वैव स्वेन॑ । \newline
13. स्वेन॑ भाग॒धेये॑न भाग॒धेये॑न॒ स्वेन॒ स्वेन॑ भाग॒धेये॑न । \newline
14. भा॒ग॒धेये॒नोपोप॑ भाग॒धेये॑न भाग॒धेये॒नोप॑ । \newline
15. भा॒ग॒धेये॒नेति॑ भाग - धेये॑न । \newline
16. उप॑ धावति धाव॒ त्युपोप॑ धावति । \newline
17. धा॒व॒ति॒ स स धा॑वति धावति॒ सः । \newline
18. स ए॒वैव स स ए॒व । \newline
19. ए॒वैन॑ मेन मे॒वै वैन᳚म् । \newline
20. ए॒न॒म् भूति॒म् भूति॑ मेन मेन॒म् भूति᳚म् । \newline
21. भूति॑म् गमयति गमयति॒ भूति॒म् भूति॑म् गमयति । \newline
22. ग॒म॒य॒ति॒ भव॑ति॒ भव॑ति गमयति गमयति॒ भव॑ति । \newline
23. भव॑त्ये॒ वैव भव॑ति॒ भव॑त्ये॒व । \newline
24. ए॒वा ति॑क्षि॒प्रा ऽति॑ क्षिप्रै॒वैवा ति॑क्षिप्रा । \newline
25. अति॑क्षिप्रा दे॒वता॑ दे॒वता ऽति॑क्षि॒प्रा ऽति॑क्षिप्रा दे॒वता᳚ । \newline
26. अति॑क्षि॒प्रेत्यति॑ - क्षि॒प्रा॒ । \newline
27. दे॒व तेतीति॑ दे॒वता॑ दे॒वतेति॑ । \newline
28. इत्या॑हु राहु॒ रिती त्या॑हुः । \newline
29. आ॒हुः॒ सा सा ऽऽहु॑ राहुः॒ सा । \newline
30. सैन॑ मेनꣳ॒॒ सा सैन᳚म् । \newline
31. ए॒न॒ मी॒श्व॒ रेश्व॒ रैन॑ मेन मीश्व॒रा । \newline
32. ई॒श्व॒रा प्र॒दहः॑ प्र॒दह॑ ईश्व॒ रेश्व॒रा प्र॒दहः॑ । \newline
33. प्र॒दह॒ इतीति॑ प्र॒दहः॑ प्र॒दह॒ इति॑ । \newline
34. प्र॒दह॒ इति॑ प्र - दहः॑ । \newline
35. इत्ये॒त मे॒त मिती त्ये॒तम् । \newline
36. ए॒त मे॒वैवैत मे॒त मे॒व । \newline
37. ए॒व सन्तꣳ॒॒ सन्त॑ मे॒वैव सन्त᳚म् । \newline
38. सन्तं॑ ॅवा॒यवे॑ वा॒यवे॒ सन्तꣳ॒॒ सन्तं॑ ॅवा॒यवे᳚ । \newline
39. वा॒यवे॑ नि॒युत्व॑ते नि॒युत्व॑ते वा॒यवे॑ वा॒यवे॑ नि॒युत्व॑ते । \newline
40. नि॒युत्व॑त॒ आ नि॒युत्व॑ते नि॒युत्व॑त॒ आ । \newline
41. नि॒युत्व॑त॒ इति॑ नि - युत्व॑ते । \newline
42. आ ल॑भेत लभे॒ता ल॑भेत । \newline
43. ल॒भे॒त॒ नि॒युन् नि॒यु ल्ल॑भेत लभेत नि॒युत् । \newline
44. नि॒युद् वै वै नि॒युन् नि॒युद् वै । \newline
45. नि॒युदिति॑ नि - युत् । \newline
46. वा अ॑स्यास्य॒ वै वा अ॑स्य । \newline
47. अ॒स्य॒ धृति॒र् धृति॑ रस्यास्य॒ धृतिः॑ । \newline
48. धृति॑र् धृ॒तो धृ॒तो धृति॒र् धृति॑र् धृ॒तः । \newline
49. धृ॒त ए॒वैव धृ॒तो धृ॒त ए॒व । \newline
50. ए॒व भूति॒म् भूति॑ मे॒वैव भूति᳚म् । \newline
51. भूति॒ मुपोप॒ भूति॒म् भूति॒ मुप॑ । \newline
52. उपै᳚त्ये॒ त्युपोपै॑ति । \newline
53. ए॒त्य प्र॑दाहा॒या प्र॑दाहायै त्ये॒त्य प्र॑दाहाय । \newline
54. अप्र॑दाहाय॒ भव॑ति॒ भव॒ त्यप्र॑दाहा॒या प्र॑दाहाय॒ भव॑ति । \newline
55. अप्र॑दाहा॒येत्यप्र॑ - दा॒हा॒य॒ । \newline
56. भव॑त्ये॒वैव भव॑ति॒ भव॑त्ये॒व । \newline
57. ए॒व वा॒यवे॑ वा॒यव॑ ए॒वैव वा॒यवे᳚ । \newline

\textbf{Ghana Paata } \newline

1. वा॒य॒व्यꣳ॑ श्वे॒तꣳ श्वे॒तं ॅवा॑य॒व्यं॑ ॅवाय॒व्यꣳ॑ श्वे॒त मा श्वे॒तं ॅवा॑य॒व्यं॑ ॅवाय॒व्यꣳ॑ श्वे॒त मा । \newline
2. श्वे॒त मा श्वे॒तꣳ श्वे॒त मा ल॑भेत लभे॒ता श्वे॒तꣳ श्वे॒त 
मा ल॑भेत । \newline
3. आ ल॑भेत लभे॒ता ल॑भेत॒ भूति॑कामो॒ भूति॑कामो लभे॒ता ल॑भेत॒ भूति॑कामः । \newline
4. ल॒भे॒त॒ भूति॑कामो॒ भूति॑कामो लभेत लभेत॒ भूति॑कामो वा॒युर् वा॒युर् भूति॑कामो लभेत लभेत॒ भूति॑कामो वा॒युः । \newline
5. भूति॑कामो वा॒युर् वा॒युर् भूति॑कामो॒ भूति॑कामो वा॒युर् वै वै वा॒युर् भूति॑कामो॒ भूति॑कामो वा॒युर् वै । \newline
6. भूति॑काम॒ इति॒ भूति॑ - का॒मः॒ । \newline
7. वा॒युर् वै वै वा॒युर् वा॒युर् वै क्षेपि॑ष्ठा॒ क्षेपि॑ष्ठा॒ वै वा॒युर् वा॒युर् वै क्षेपि॑ष्ठा । \newline
8. वै क्षेपि॑ष्ठा॒ क्षेपि॑ष्ठा॒ वै वै क्षेपि॑ष्ठा दे॒वता॑ दे॒वता॒ क्षेपि॑ष्ठा॒ वै वै क्षेपि॑ष्ठा दे॒वता᳚ । \newline
9. क्षेपि॑ष्ठा दे॒वता॑ दे॒वता॒ क्षेपि॑ष्ठा॒ क्षेपि॑ष्ठा दे॒वता॑ वा॒युं ॅवा॒युम् दे॒वता॒ क्षेपि॑ष्ठा॒ क्षेपि॑ष्ठा दे॒वता॑ वा॒युम् । \newline
10. दे॒वता॑ वा॒युं ॅवा॒युम् दे॒वता॑ दे॒वता॑ वा॒यु मे॒वैव वा॒युम् दे॒वता॑ दे॒वता॑ वा॒यु मे॒व । \newline
11. वा॒यु मे॒वैव वा॒युं ॅवा॒यु मे॒व स्वेन॒ स्वेनै॒व वा॒युं ॅवा॒यु मे॒व स्वेन॑ । \newline
12. ए॒व स्वेन॒ स्वेनै॒वैव स्वेन॑ भाग॒धेये॑न भाग॒धेये॑न॒ स्वेनै॒वैव स्वेन॑ भाग॒धेये॑न । \newline
13. स्वेन॑ भाग॒धेये॑न भाग॒धेये॑न॒ स्वेन॒ स्वेन॑ भाग॒धेये॒ नोपोप॑ भाग॒धेये॑न॒ स्वेन॒ स्वेन॑ भाग॒धेये॒नोप॑ । \newline
14. भा॒ग॒धेये॒ नोपोप॑ भाग॒धेये॑न भाग॒धेये॒नोप॑ धावति धाव॒त्युप॑ भाग॒धेये॑न भाग॒धेये॒नोप॑ धावति । \newline
15. भा॒ग॒धेये॒नेति॑ भाग - धेये॑न । \newline
16. उप॑ धावति धाव॒त्युपोप॑ धावति॒ स स धा॑व॒त्युपोप॑ धावति॒ सः । \newline
17. धा॒व॒ति॒ स स धा॑वति धावति॒ स ए॒वैव स धा॑वति धावति॒ स ए॒व । \newline
18. स ए॒वैव स स ए॒वैन॑ मेन मे॒व स स ए॒वैन᳚म् । \newline
19. ए॒वैन॑ मेन मे॒वै वैन॒म् भूति॒म् भूति॑ मेन मे॒वै वैन॒म् भूति᳚म् । \newline
20. ए॒न॒म् भूति॒म् भूति॑ मेन मेन॒म् भूति॑म् गमयति गमयति॒ भूति॑ मेन मेन॒म् भूति॑म् गमयति । \newline
21. भूति॑म् गमयति गमयति॒ भूति॒म् भूति॑म् गमयति॒ भव॑ति॒ भव॑ति गमयति॒ भूति॒म् भूति॑म् गमयति॒ भव॑ति । \newline
22. ग॒म॒य॒ति॒ भव॑ति॒ भव॑ति गमयति गमयति॒ भव॑ त्ये॒वैव भव॑ति गमयति गमयति॒ भव॑त्ये॒व । \newline
23. भव॑ त्ये॒वैव भव॑ति॒ भव॑ त्ये॒वाति॑ क्षि॒प्रा ऽति॑क्षिप्रै॒व भव॑ति॒ भव॑ त्ये॒वाति॑ क्षिप्रा । \newline
24. ए॒वाति॑ क्षि॒प्रा ऽति॑क्षि प्रै॒वैवाति॑ क्षिप्रा दे॒वता॑ दे॒वता ऽति॑क्षि प्रै॒वैवाति॑ क्षिप्रा दे॒वता᳚ । \newline
25. अति॑क्षिप्रा दे॒वता॑ दे॒वता ऽति॑क्षि॒प्रा ऽति॑क्षिप्रा दे॒वतेतीति॑ दे॒वता ऽति॑क्षि॒प्रा ऽति॑क्षिप्रा दे॒वतेति॑ । \newline
26. अति॑क्षि॒प्रेत्यति॑ - क्षि॒प्रा॒ । \newline
27. दे॒वतेतीति॑ दे॒वता॑ दे॒व तेत्या॑हु राहु॒रिति॑ दे॒वता॑ दे॒वतेत्या॑हुः । \newline
28. इत्या॑हु राहु॒ रितीत्या॑हुः॒ सा सा ऽऽहु॒ रितीत्या॑हुः॒ सा । \newline
29. आ॒हुः॒ सा सा ऽऽहु॑राहुः॒ सैन॑ मेनꣳ॒॒ सा ऽऽहु॑राहुः॒ सैन᳚म् । \newline
30. सैन॑ मेनꣳ॒॒ सा सैन॑ मीश्व॒ रेश्व॒ रैनꣳ॒॒ सा सैन॑ मीश्व॒रा । \newline
31. ए॒न॒ मी॒श्व॒ रेश्व॒रैन॑ मेन मीश्व॒रा प्र॒दहः॑ प्र॒दह॑ ईश्व॒रैन॑ मेन मीश्व॒रा प्र॒दहः॑ । \newline
32. ई॒श्व॒रा प्र॒दहः॑ प्र॒दह॑ ईश्व॒रेश्व॒रा प्र॒दह॒ इतीति॑ प्र॒दह॑ ईश्व॒रेश्व॒रा प्र॒दह॒ इति॑ । \newline
33. प्र॒दह॒ इतीति॑ प्र॒दहः॑ प्र॒दह॒ इत्ये॒त मे॒त मिति॑ प्र॒दहः॑ प्र॒दह॒ इत्ये॒तम् । \newline
34. प्र॒दह॒ इति॑ प्र - दहः॑ । \newline
35. इत्ये॒त मे॒त मितीत्ये॒त मे॒वैवैत मितीत्ये॒त मे॒व । \newline
36. ए॒त मे॒वैवैत मे॒त मे॒व सन्तꣳ॒॒ सन्त॑ मे॒वैत मे॒त मे॒व सन्त᳚म् । \newline
37. ए॒व सन्तꣳ॒॒ सन्त॑ मे॒वैव सन्तं॑ ॅवा॒यवे॑ वा॒यवे॒ सन्त॑ मे॒वैव सन्तं॑ ॅवा॒यवे᳚ । \newline
38. सन्तं॑ ॅवा॒यवे॑ वा॒यवे॒ सन्तꣳ॒॒ सन्तं॑ ॅवा॒यवे॑ नि॒युत्व॑ते नि॒युत्व॑ते वा॒यवे॒ सन्तꣳ॒॒ सन्तं॑ ॅवा॒यवे॑ नि॒युत्व॑ते । \newline
39. वा॒यवे॑ नि॒युत्व॑ते नि॒युत्व॑ते वा॒यवे॑ वा॒यवे॑ नि॒युत्व॑त॒ आ नि॒युत्व॑ते वा॒यवे॑ वा॒यवे॑ नि॒युत्व॑त॒ आ । \newline
40. नि॒युत्व॑त॒ आ नि॒युत्व॑ते नि॒युत्व॑त॒ आ ल॑भेत लभे॒ता नि॒युत्व॑ते नि॒युत्व॑त॒ आ ल॑भेत । \newline
41. नि॒युत्व॑त॒ इति॑ नि - युत्व॑ते । \newline
42. आ ल॑भेत लभे॒ता ल॑भेत नि॒युन् नि॒यु ल्ल॑भे॒ता ल॑भेत नि॒युत् । \newline
43. ल॒भे॒त॒ नि॒युन् नि॒यु ल्ल॑भेत लभेत नि॒युद् वै वै नि॒यु ल्ल॑भेत लभेत नि॒युद् वै । \newline
44. नि॒युद् वै वै नि॒युन् नि॒युद् वा अ॑स्यास्य॒ वै नि॒युन् नि॒युद् वा अ॑स्य । \newline
45. नि॒युदिति॑ नि - युत् । \newline
46. वा अ॑स्यास्य॒ वै वा अ॑स्य॒ धृति॒र् धृति॑ रस्य॒ वै वा अ॑स्य॒ धृतिः॑ । \newline
47. अ॒स्य॒ धृति॒र् धृति॑ रस्यास्य॒ धृति॑र् धृ॒तो धृ॒तो धृति॑ रस्यास्य॒ धृति॑र् धृ॒तः । \newline
48. धृति॑र् धृ॒तो धृ॒तो धृति॒र् धृति॑र् धृ॒त ए॒वैव धृ॒तो धृति॒र् धृति॑र् धृ॒त ए॒व । \newline
49. धृ॒त ए॒वैव धृ॒तो धृ॒त ए॒व भूति॒म् भूति॑ मे॒व धृ॒तो धृ॒त ए॒व भूति᳚म् । \newline
50. ए॒व भूति॒म् भूति॑ मे॒वैव भूति॒ मुपोप॒ भूति॑ मे॒वैव भूति॒ मुप॑ । \newline
51. भूति॒ मुपोप॒ भूति॒म् भूति॒ मुपै᳚ त्ये॒त्युप॒ भूति॒म् भूति॒ मुपै॑ति । \newline
52. उपै᳚ त्ये॒त्यु पोपै॒त्य प्र॑दाहा॒या प्र॑दाहायै॒ त्युपोपै॒ त्यप्र॑दाहाय । \newline
53. ए॒त्यप्र॑दाहा॒या प्र॑दाहायै त्ये॒त्यप्र॑दाहाय॒ भव॑ति॒ भव॒ त्यप्र॑दाहायैत्ये॒ त्यप्र॑दाहाय॒ भव॑ति । \newline
54. अप्र॑दाहाय॒ भव॑ति॒ भव॒ त्यप्र॑दाहा॒या प्र॑दाहाय॒ भव॑त्ये॒वैव भव॒ त्यप्र॑दाहा॒या प्र॑दाहाय॒ भव॑त्ये॒व । \newline
55. अप्र॑दाहा॒येत्यप्र॑ - दा॒हा॒य॒ । \newline
56. भव॑त्ये॒वैव भव॑ति॒ भव॑त्ये॒व वा॒यवे॑ वा॒यव॑ ए॒व भव॑ति॒ भव॑त्ये॒व वा॒यवे᳚ । \newline
57. ए॒व वा॒यवे॑ वा॒यव॑ ए॒वैव वा॒यवे॑ नि॒युत्व॑ते नि॒युत्व॑ते वा॒यव॑ ए॒वैव वा॒यवे॑ नि॒युत्व॑ते । \newline
\pagebreak
\markright{ TS 2.1.1.2  \hfill https://www.vedavms.in \hfill}

\section{ TS 2.1.1.2 }

\textbf{TS 2.1.1.2 } \newline
\textbf{Samhita Paata} \newline

वा॒यवे॑ नि॒युत्व॑त॒ आ ल॑भेत॒ ग्राम॑कामो वा॒युर्वा इ॒माः प्र॒जा न॑स्यो॒ता ने॑नीयते वा॒युमे॒व नि॒युत्व॑न्तꣳ॒॒ स्वेन॑ भाग॒धेये॒नोप॑ धावति॒ स ए॒वास्मै᳚ प्र॒जा न॑स्यो॒ता निय॑च्छति ग्रा॒म्ये॑व भ॑वति नि॒युत्व॑ते भवति ध्रु॒वा ए॒वास्मा॒ अन॑पगाः करोति वा॒यवे॑ नि॒युत्व॑त॒ आ ल॑भेत प्र॒जाका॑मः प्रा॒णो वै वा॒युर॑पा॒नो नि॒युत् प्रा॑णापा॒नौ खलु॒ वा ए॒तस्य॑ प्र॒जाया॒-  [  ] \newline

\textbf{Pada Paata} \newline

वा॒यवे᳚ । नि॒युत्व॑त॒ इति॑ नि - युत्व॑ते । एति॑ । ल॒भे॒त॒ । ग्राम॑काम॒ इति॒ ग्राम॑ - का॒मः॒ । वा॒युः । वै । इ॒माः । प्र॒जा इति॑ प्र - जाः । न॒स्यो॒ता इति॑ नसि - ओ॒ताः । ने॒नी॒य॒ते॒ । वा॒युम् । ए॒व । नि॒युत्व॑न्त॒मिति॑ नि-युत्व॑न्तम् । स्वेन॑ । भा॒ग॒धेये॒नेति॑ भाग - धेये॑न । उपेति॑ । धा॒व॒ति॒ । सः । ए॒व । अ॒स्मै॒ । प्र॒जा इति॑ प्र - जाः । न॒स्यो॒ता इति॑ नसि - ओ॒ताः । नीति॑ । य॒च्छ॒ति॒ । ग्रा॒मी । ए॒व । भ॒व॒ति॒ । नि॒युत्व॑त॒ इति॑ नि-युत्व॑ते । भ॒व॒ति॒ । ध्रु॒वाः । ए॒व । अ॒स्मै॒ । अन॑पगा॒ इत्यन॑प - गाः॒ । क॒रो॒ति॒ । वा॒यवे᳚ । नि॒युत्व॑त॒ इति॑ नि - युत्व॑ते । एति॑ । ल॒भे॒त॒ । प्र॒जाका॑म॒ इति॑ प्र॒जा - का॒मः॒ । प्रा॒ण इति॑ प्र- अ॒नः । वै । वा॒युः । अ॒पा॒न इत्य॑प - अ॒नः । नि॒युदिति॑ नि - युत् । प्रा॒णा॒पा॒नाविति॑ प्राण- अ॒पा॒नौ । खलु॑ । वै । ए॒तस्य॑ । प्र॒जाया॒ इति॑ प्र - जायाः᳚ ।  \newline


\textbf{Krama Paata} \newline

वा॒यवे॑ नि॒युत्व॑ते । नि॒युत्व॑त॒ आ । नि॒युत्व॑त॒ इति॑ नि - युत्व॑ते । आ ल॑भेत । ल॒भे॒त॒ ग्राम॑कामः । ग्राम॑कामो वा॒युः । ग्राम॑काम॒ इति॒ ग्राम॑ - का॒मः॒ । वा॒युर् वै । वा इ॒माः । इ॒माः प्र॒जाः । प्र॒जा न॑स्यो॒ताः । प्र॒जा इति॑ प्र - जाः । न॒स्यो॒ता ने॑नीयते । न॒स्यो॒ता इति॑ नसि - ओ॒ताः । ने॒नी॒य॒ते॒ वा॒युम् । वा॒युमे॒व । ए॒व नि॒युत्व॑न्तम् । नि॒युत्व॑न्तꣳ॒॒स्वेन॑ । नि॒युत्व॑न्त॒मिति॑ नि - युत्व॑न्तम् । स्वेन॑ भाग॒धेये॑न । भा॒ग॒धेये॒नोप॑ । भा॒ग॒धेये॒नेति॑ भाग - धेये॑न । उप॑ धावति । धा॒व॒ति॒ सः । स ए॒व । ए॒वास्मै᳚ । अ॒स्मै॒ प्र॒जाः । प्र॒जा न॑स्यो॒ताः । प्र॒जा इति॑ प्र - जाः । न॒स्यो॒ता नि । 
न॒स्यो॒ता इति॑ नसि - ओ॒ताः । नि य॑च्छति । य॒च्छ॒ति॒ ग्रा॒मी । ग्रा॒म्ये॑व । ए॒व भ॑वति । भ॒व॒ति॒ नि॒युत्व॑ते । नि॒युत्व॑ते भवति । नि॒युत्व॑त॒ इति॑ नि - युत्व॑ते । भ॒व॒ति॒ ध्रु॒वाः । ध्रु॒वा ए॒व । ए॒वास्मै᳚ । अ॒स्मा॒ अन॑पगाः । अन॑पगाः करोति । अन॑पगा॒ इत्यन॑प - गाः॒ । क॒रो॒ति॒ वा॒यवे᳚ । वा॒यवे॑ नि॒युत्व॑ते । नि॒युत्व॑त॒ आ । नि॒युत्व॑त॒ इति॑ नि - युत्व॑ते । आ ल॑भेत । ल॒भे॒त॒ प्र॒जाका॑मः । प्र॒जाका॑मः प्रा॒णः । प्र॒जाका॑म॒ इति॑ प्र॒जा - का॒मः॒ । प्रा॒णो वै । प्रा॒ण इति॑ प्र - अ॒नः । वै वा॒युः । वा॒युर॑पा॒नः । अ॒पा॒नो नि॒युत् । अ॒पा॒न इत्य॑प - अ॒नः । नि॒युत् प्रा॑णापा॒नौ । नि॒युदिति॑ नि - युत् । प्रा॒णा॒पा॒नौ खलु॑ । प्रा॒णा॒पा॒नाविति॑ प्राण - अ॒पा॒नौ । खलु॒ वै । वा ए॒तस्य॑ । ए॒तस्य॑ प्र॒जायाः᳚ । प्र॒जाया॒  अप॑ । प्र॒जाया॒ इति॑ प्र - जायाः᳚ \newline

\textbf{Jatai Paata} \newline

1. वा॒यवे॑ नि॒युत्व॑ते नि॒युत्व॑ते वा॒यवे॑ वा॒यवे॑ नि॒युत्व॑ते । \newline
2. नि॒युत्व॑त॒ आ नि॒युत्व॑ते नि॒युत्व॑त॒ आ । \newline
3. नि॒युत्व॑त॒ इति॑ नि - युत्व॑ते । \newline
4. आ ल॑भेत लभे॒ता ल॑भेत । \newline
5. ल॒भे॒त॒ ग्राम॑कामो॒ ग्राम॑कामो लभेत लभेत॒ ग्राम॑कामः । \newline
6. ग्राम॑कामो वा॒युर् वा॒युर् ग्राम॑कामो॒ ग्राम॑कामो वा॒युः । \newline
7. ग्राम॑काम॒ इति॒ ग्राम॑ - का॒मः॒ । \newline
8. वा॒युर् वै वै वा॒युर् वा॒युर् वै । \newline
9. वा इ॒मा इ॒मा वै वा इ॒माः । \newline
10. इ॒माः प्र॒जाः प्र॒जा इ॒मा इ॒माः प्र॒जाः । \newline
11. प्र॒जा न॑स्यो॒ता न॑स्यो॒ताः प्र॒जाः प्र॒जा न॑स्यो॒ताः । \newline
12. प्र॒जा इति॑ प्र - जाः । \newline
13. न॒स्यो॒ता ने॑नीयते नेनीयते नस्यो॒ता न॑स्यो॒ता ने॑नीयते । \newline
14. न॒स्यो॒ता इति॑ नसि - ओ॒ताः । \newline
15. ने॒नी॒य॒ते॒ वा॒युं ॅवा॒युम् ने॑नीयते नेनीयते वा॒युम् । \newline
16. वा॒यु मे॒वैव वा॒युं ॅवा॒यु मे॒व । \newline
17. ए॒व नि॒युत्व॑न्तम् नि॒युत्व॑न्त मे॒वैव नि॒युत्व॑न्तम् । \newline
18. नि॒युत्व॑न्तꣳ॒॒ स्वेन॒ स्वेन॑ नि॒युत्व॑न्तम् नि॒युत्व॑न्तꣳ॒॒ स्वेन॑ । \newline
19. नि॒युत्व॑न्त॒मिति॑ नि - युत्व॑न्तम् । \newline
20. स्वेन॑ भाग॒धेये॑न भाग॒धेये॑न॒ स्वेन॒ स्वेन॑ भाग॒धेये॑न । \newline
21. भा॒ग॒धेये॒नोपोप॑ भाग॒धेये॑न भाग॒धेये॒नोप॑ । \newline
22. भा॒ग॒धेये॒नेति॑ भाग - धेये॑न । \newline
23. उप॑ धावति धाव॒ त्युपोप॑ धावति । \newline
24. धा॒व॒ति॒ स स धा॑वति धावति॒ सः । \newline
25. स ए॒वैव स स ए॒व । \newline
26. ए॒वास्मा॑ अस्मा ए॒वैवास्मै᳚ । \newline
27. अ॒स्मै॒ प्र॒जाः प्र॒जा अ॑स्मा अस्मै प्र॒जाः । \newline
28. प्र॒जा न॑स्यो॒ता न॑स्यो॒ताः प्र॒जाः प्र॒जा न॑स्यो॒ताः । \newline
29. प्र॒जा इति॑ प्र - जाः । \newline
30. न॒स्यो॒ता नि नि न॑स्यो॒ता न॑स्यो॒ता नि । \newline
31. न॒स्यो॒ता इति॑ नसि - ओ॒ताः । \newline
32. नि य॑च्छति यच्छति॒ नि नि य॑च्छति । \newline
33. य॒च्छ॒ति॒ ग्रा॒मी ग्रा॒मी य॑च्छति यच्छति ग्रा॒मी । \newline
34. ग्रा॒म्ये॑वैव ग्रा॒मी ग्रा॒म्ये॑व । \newline
35. ए॒व भ॑वति भव त्ये॒वैव भ॑वति । \newline
36. भ॒व॒ति॒ नि॒युत्व॑ते नि॒युत्व॑ते भवति भवति नि॒युत्व॑ते । \newline
37. नि॒युत्व॑ते भवति भवति नि॒युत्व॑ते नि॒युत्व॑ते भवति । \newline
38. नि॒युत्व॑त॒ इति॑ नि - युत्व॑ते । \newline
39. भ॒व॒ति॒ ध्रु॒वा ध्रु॒वा भ॑वति भवति ध्रु॒वाः । \newline
40. ध्रु॒वा ए॒वैव ध्रु॒वा ध्रु॒वा ए॒व । \newline
41. ए॒वास्मा॑ अस्मा ए॒वैवास्मै᳚ । \newline
42. अ॒स्मा॒ अन॑पगा॒ अन॑पगा अस्मा अस्मा॒ अन॑पगाः । \newline
43. अन॑पगाः करोति करो॒ त्यन॑पगा॒ अन॑पगाः करोति । \newline
44. अन॑पगा॒ इत्यन॑प - गाः॒ । \newline
45. क॒रो॒ति॒ वा॒यवे॑ वा॒यवे॑ करोति करोति वा॒यवे᳚ । \newline
46. वा॒यवे॑ नि॒युत्व॑ते नि॒युत्व॑ते वा॒यवे॑ वा॒यवे॑ नि॒युत्व॑ते । \newline
47. नि॒युत्व॑त॒ आ नि॒युत्व॑ते नि॒युत्व॑त॒ आ । \newline
48. नि॒युत्व॑त॒ इति॑ नि - युत्व॑ते । \newline
49. आ ल॑भेत लभे॒ता ल॑भेत । \newline
50. ल॒भे॒त॒ प्र॒जाका॑मः प्र॒जाका॑मो लभेत लभेत प्र॒जाका॑मः । \newline
51. प्र॒जाका॑मः प्रा॒णः प्रा॒णः प्र॒जाका॑मः प्र॒जाका॑मः प्रा॒णः । \newline
52. प्र॒जाका॑म॒ इति॑ प्र॒जा - का॒मः॒ । \newline
53. प्रा॒णो वै वै प्रा॒णः प्रा॒णो वै । \newline
54. प्रा॒ण इति॑ प्र - अ॒नः । \newline
55. वै वा॒युर् वा॒युर् वै वै वा॒युः । \newline
56. वा॒यु र॑पा॒नो॑ ऽपा॒नो वा॒युर् वा॒यु र॑पा॒नः । \newline
57. अ॒पा॒नो नि॒युन् नि॒यु द॑पा॒नो॑ ऽपा॒नो नि॒युत् । \newline
58. अ॒पा॒न इत्य॑प - अ॒नः । \newline
59. नि॒युत् प्रा॑णापा॒नौ प्रा॑णापा॒नौ नि॒युन् नि॒युत् प्रा॑णापा॒नौ । \newline
60. नि॒युदिति॑ नि - युत् । \newline
61. प्रा॒णा॒पा॒नौ खलु॒ खलु॑ प्राणापा॒नौ प्रा॑णापा॒नौ खलु॑ । \newline
62. प्रा॒णा॒पा॒नाविति॑ प्राण - अ॒पा॒नौ । \newline
63. खलु॒ वै वै खलु॒ खलु॒ वै । \newline
64. वा ए॒त स्यै॒तस्य॒ वै वा ए॒तस्य॑ । \newline
65. ए॒तस्य॑ प्र॒जायाः᳚ प्र॒जाया॑ ए॒त स्यै॒तस्य॑ प्र॒जायाः᳚ । \newline
66. प्र॒जाया॒ अपाप॑ प्र॒जायाः᳚ प्र॒जाया॒ अप॑ । \newline
67. प्र॒जाया॒ इति॑ प्र - जायाः᳚ । \newline

\textbf{Ghana Paata } \newline

1. वा॒यवे॑ नि॒युत्व॑ते नि॒युत्व॑ते वा॒यवे॑ वा॒यवे॑ नि॒युत्व॑त॒ आ नि॒युत्व॑ते वा॒यवे॑ वा॒यवे॑ नि॒युत्व॑त॒ आ । \newline
2. नि॒युत्व॑त॒ आ नि॒युत्व॑ते नि॒युत्व॑त॒ आ ल॑भेत लभे॒ता नि॒युत्व॑ते नि॒युत्व॑त॒ आ ल॑भेत । \newline
3. नि॒युत्व॑त॒ इति॑ नि - युत्व॑ते । \newline
4. आ ल॑भेत लभे॒ता ल॑भेत॒ ग्राम॑कामो॒ ग्राम॑कामो लभे॒ता ल॑भेत॒ ग्राम॑कामः । \newline
5. ल॒भे॒त॒ ग्राम॑कामो॒ ग्राम॑कामो लभेत लभेत॒ ग्राम॑कामो वा॒युर् वा॒युर् ग्राम॑कामो लभेत लभेत॒ ग्राम॑कामो वा॒युः । \newline
6. ग्राम॑कामो वा॒युर् वा॒युर् ग्राम॑कामो॒ ग्राम॑कामो वा॒युर् वै वै वा॒युर् ग्राम॑कामो॒ ग्राम॑कामो वा॒युर् वै । \newline
7. ग्राम॑काम॒ इति॒ ग्राम॑ - का॒मः॒ । \newline
8. वा॒युर् वै वै वा॒युर् वा॒युर् वा इ॒मा इ॒मा वै वा॒युर् वा॒युर् वा इ॒माः । \newline
9. वा इ॒मा इ॒मा वै वा इ॒माः प्र॒जाः प्र॒जा इ॒मा वै वा इ॒माः प्र॒जाः । \newline
10. इ॒माः प्र॒जाः प्र॒जा इ॒मा इ॒माः प्र॒जा न॑स्यो॒ता न॑स्यो॒ताः प्र॒जा इ॒मा इ॒माः प्र॒जा न॑स्यो॒ताः । \newline
11. प्र॒जा न॑स्यो॒ता न॑स्यो॒ताः प्र॒जाः प्र॒जा न॑स्यो॒ता ने॑नीयते नेनीयते नस्यो॒ताः प्र॒जाः प्र॒जा न॑स्यो॒ता ने॑नीयते । \newline
12. प्र॒जा इति॑ प्र - जाः । \newline
13. न॒स्यो॒ता ने॑नीयते नेनीयते नस्यो॒ता न॑स्यो॒ता ने॑नीयते वा॒युं ॅवा॒युन् ने॑नीयते नस्यो॒ता न॑स्यो॒ता ने॑नीयते वा॒युम् । \newline
14. न॒स्यो॒ता इति॑ नसि - ओ॒ताः । \newline
15. ने॒नी॒य॒ते॒ वा॒युं ॅवा॒युन् ने॑नीयते नेनीयते वा॒यु मे॒वैव वा॒युन् ने॑नीयते नेनीयते वा॒यु मे॒व । \newline
16. वा॒यु मे॒वैव वा॒युं ॅवा॒यु मे॒व नि॒युत्व॑न्तन् नि॒युत्व॑न्त मे॒व वा॒युं ॅवा॒यु मे॒व नि॒युत्व॑न्तम् । \newline
17. ए॒व नि॒युत्व॑न्तन् नि॒युत्व॑न्त मे॒वैव नि॒युत्व॑न्तꣳ॒॒ स्वेन॒ स्वेन॑ नि॒युत्व॑न्त मे॒वैव नि॒युत्व॑न्तꣳ॒॒ स्वेन॑ । \newline
18. नि॒युत्व॑न्तꣳ॒॒ स्वेन॒ स्वेन॑ नि॒युत्व॑न्तन् नि॒युत्व॑न्तꣳ॒॒ स्वेन॑ भाग॒धेये॑न भाग॒धेये॑न॒ स्वेन॑ नि॒युत्व॑न्तन् नि॒युत्व॑न्तꣳ॒॒ स्वेन॑ भाग॒धेये॑न । \newline
19. नि॒युत्व॑न्त॒मिति॑ नि - युत्व॑न्तम् । \newline
20. स्वेन॑ भाग॒धेये॑न भाग॒धेये॑न॒ स्वेन॒ स्वेन॑ भाग॒धेये॒नो पोप॑ भाग॒धेये॑न॒ स्वेन॒ स्वेन॑ भाग॒धेये॒नोप॑ । \newline
21. भा॒ग॒धेये॒नो पोप॑ भाग॒धेये॑न भाग॒धेये॒नोप॑ धावति धाव॒त्युप॑ भाग॒धेये॑न भाग॒धेये॒नोप॑ धावति । \newline
22. भा॒ग॒धेये॒नेति॑ भाग - धेये॑न । \newline
23. उप॑ धावति धाव॒ त्युपोप॑ धावति॒ स स धा॑व॒ त्युपोप॑ धावति॒ सः । \newline
24. धा॒व॒ति॒ स स धा॑वति धावति॒ स ए॒वैव स धा॑वति धावति॒ स ए॒व । \newline
25. स ए॒वैव स स ए॒वास्मा॑ अस्मा ए॒व स स ए॒वास्मै᳚ । \newline
26. ए॒वास्मा॑ अस्मा ए॒वैवास्मै᳚ प्र॒जाः प्र॒जा अ॑स्मा ए॒वैवास्मै᳚ प्र॒जाः । \newline
27. अ॒स्मै॒ प्र॒जाः प्र॒जा अ॑स्मा अस्मै प्र॒जा न॑स्यो॒ता न॑स्यो॒ताः प्र॒जा अ॑स्मा अस्मै प्र॒जा न॑स्यो॒ताः । \newline
28. प्र॒जा न॑स्यो॒ता न॑स्यो॒ताः प्र॒जाः प्र॒जा न॑स्यो॒ता नि नि न॑स्यो॒ताः प्र॒जाः प्र॒जा न॑स्यो॒ता नि । \newline
29. प्र॒जा इति॑ प्र - जाः । \newline
30. न॒स्यो॒ता नि नि न॑स्यो॒ता न॑स्यो॒ता नि य॑च्छति यच्छति॒ नि न॑स्यो॒ता न॑स्यो॒ता नि य॑च्छति । \newline
31. न॒स्यो॒ता इति॑ नसि - ओ॒ताः । \newline
32. नि य॑च्छति यच्छति॒ नि नि य॑च्छति ग्रा॒मी ग्रा॒मी य॑च्छति॒ नि नि य॑च्छति ग्रा॒मी । \newline
33. य॒च्छ॒ति॒ ग्रा॒मी ग्रा॒मी य॑च्छति यच्छति ग्रा॒म्ये॑वैव ग्रा॒मी य॑च्छति यच्छति ग्रा॒म्ये॑व । \newline
34. ग्रा॒म्ये॑वैव ग्रा॒मी ग्रा॒म्ये॑व भ॑वति भवत्ये॒व ग्रा॒मी ग्रा॒म्ये॑व भ॑वति । \newline
35. ए॒व भ॑वति भवत्ये॒वैव भ॑वति नि॒युत्व॑ते नि॒युत्व॑ते भवत्ये॒वैव भ॑वति नि॒युत्व॑ते । \newline
36. भ॒व॒ति॒ नि॒युत्व॑ते नि॒युत्व॑ते भवति भवति नि॒युत्व॑ते भवति भवति नि॒युत्व॑ते भवति भवति नि॒युत्व॑ते भवति । \newline
37. नि॒युत्व॑ते भवति भवति नि॒युत्व॑ते नि॒युत्व॑ते भवति ध्रु॒वा ध्रु॒वा भ॑वति नि॒युत्व॑ते नि॒युत्व॑ते भवति ध्रु॒वाः । \newline
38. नि॒युत्व॑त॒ इति॑ नि - युत्व॑ते । \newline
39. भ॒व॒ति॒ ध्रु॒वा ध्रु॒वा भ॑वति भवति ध्रु॒वा ए॒वैव ध्रु॒वा भ॑वति भवति ध्रु॒वा ए॒व । \newline
40. ध्रु॒वा ए॒वैव ध्रु॒वा ध्रु॒वा ए॒वास्मा॑ अस्मा ए॒व ध्रु॒वा ध्रु॒वा ए॒वास्मै᳚ । \newline
41. ए॒वास्मा॑ अस्मा ए॒वैवास्मा॒ अन॑पगा॒ अन॑पगा अस्मा ए॒वैवास्मा॒ अन॑पगाः । \newline
42. अ॒स्मा॒ अन॑पगा॒ अन॑पगा अस्मा अस्मा॒ अन॑पगाः करोति करो॒ त्यन॑पगा अस्मा अस्मा॒ अन॑पगाः करोति । \newline
43. अन॑पगाः करोति करो॒ त्यन॑पगा॒ अन॑पगाः करोति वा॒यवे॑ वा॒यवे॑ करो॒ त्यन॑पगा॒ अन॑पगाः करोति वा॒यवे᳚ । \newline
44. अन॑पगा॒ इत्यन॑प - गाः॒ । \newline
45. क॒रो॒ति॒ वा॒यवे॑ वा॒यवे॑ करोति करोति वा॒यवे॑ नि॒युत्व॑ते नि॒युत्व॑ते वा॒यवे॑ करोति करोति वा॒यवे॑ नि॒युत्व॑ते । \newline
46. वा॒यवे॑ नि॒युत्व॑ते नि॒युत्व॑ते वा॒यवे॑ वा॒यवे॑ नि॒युत्व॑त॒ आ नि॒युत्व॑ते वा॒यवे॑ वा॒यवे॑ नि॒युत्व॑त॒ आ । \newline
47. नि॒युत्व॑त॒ आ नि॒युत्व॑ते नि॒युत्व॑त॒ आ ल॑भेत लभे॒ता नि॒युत्व॑ते नि॒युत्व॑त॒ आ ल॑भेत । \newline
48. नि॒युत्व॑त॒ इति॑ नि - युत्व॑ते । \newline
49. आ ल॑भेत लभे॒ता ल॑भेत प्र॒जाका॑मः प्र॒जाका॑मो लभे॒ता ल॑भेत प्र॒जाका॑मः । \newline
50. ल॒भे॒त॒ प्र॒जाका॑मः प्र॒जाका॑मो लभेत लभेत प्र॒जाका॑मः प्रा॒णः प्रा॒णः प्र॒जाका॑मो लभेत लभेत प्र॒जाका॑मः प्रा॒णः । \newline
51. प्र॒जाका॑मः प्रा॒णः प्रा॒णः प्र॒जाका॑मः प्र॒जाका॑मः प्रा॒णो वै वै प्रा॒णः प्र॒जाका॑मः प्र॒जाका॑मः प्रा॒णो वै । \newline
52. प्र॒जाका॑म॒ इति॑ प्र॒जा - का॒मः॒ । \newline
53. प्रा॒णो वै वै प्रा॒णः प्रा॒णो वै वा॒युर् वा॒युर् वै प्रा॒णः प्रा॒णो वै वा॒युः । \newline
54. प्रा॒ण इति॑ प्र - अ॒नः । \newline
55. वै वा॒युर् वा॒युर् वै वै वा॒यु र॑पा॒नो॑ ऽपा॒नो वा॒युर् वै वै वा॒यु र॑पा॒नः । \newline
56. वा॒यु र॑पा॒नो॑ ऽपा॒नो वा॒युर् वा॒यु र॑पा॒नो नि॒युन् नि॒यु द॑पा॒नो वा॒युर् वा॒यु र॑पा॒नो नि॒युत् । \newline
57. अ॒पा॒नो नि॒युन् नि॒यु द॑पा॒नो॑ ऽपा॒नो नि॒युत् प्रा॑णापा॒नौ प्रा॑णापा॒नौ नि॒यु द॑पा॒नो॑ ऽपा॒नो नि॒युत् प्रा॑णापा॒नौ । \newline
58. अ॒पा॒न इत्य॑प - अ॒नः । \newline
59. नि॒युत् प्रा॑णापा॒नौ प्रा॑णापा॒नौ नि॒युन् नि॒युत् प्रा॑णापा॒नौ खलु॒ खलु॑ प्राणापा॒नौ नि॒युन् नि॒युत् प्रा॑णापा॒नौ खलु॑ । \newline
60. नि॒युदिति॑ नि - युत् । \newline
61. प्रा॒णा॒पा॒नौ खलु॒ खलु॑ प्राणापा॒नौ प्रा॑णापा॒नौ खलु॒ वै वै खलु॑ प्राणापा॒नौ प्रा॑णापा॒नौ खलु॒ वै । \newline
62. प्रा॒णा॒पा॒नाविति॑ प्राण - अ॒पा॒नौ । \newline
63. खलु॒ वै वै खलु॒ खलु॒ वा ए॒त स्यै॒तस्य॒ वै खलु॒ खलु॒ वा ए॒तस्य॑ । \newline
64. वा ए॒त स्यै॒तस्य॒ वै वा ए॒तस्य॑ प्र॒जायाः᳚ प्र॒जाया॑ ए॒तस्य॒ वै वा ए॒तस्य॑ प्र॒जायाः᳚ । \newline
65. ए॒तस्य॑ प्र॒जायाः᳚ प्र॒जाया॑ ए॒त स्यै॒तस्य॑ प्र॒जाया॒ अपाप॑ प्र॒जाया॑ ए॒त स्यै॒तस्य॑ प्र॒जाया॒ अप॑ । \newline
66. प्र॒जाया॒ अपाप॑ प्र॒जायाः᳚ प्र॒जाया॒ अप॑ क्रामतः क्राम॒तो ऽप॑ प्र॒जायाः᳚ प्र॒जाया॒ अप॑ क्रामतः । \newline
67. प्र॒जाया॒ इति॑ प्र - जायाः᳚ । \newline
\pagebreak
\markright{ TS 2.1.1.3  \hfill https://www.vedavms.in \hfill}

\section{ TS 2.1.1.3 }

\textbf{TS 2.1.1.3 } \newline
\textbf{Samhita Paata} \newline

अप॑ क्रामतो॒ योऽलं॑ प्र॒जायै॒ सन् प्र॒जां न वि॒न्दते॑ वा॒युमे॒व नि॒युत्व॑न्तꣳ॒॒  स्वेन॑ भाग॒धेये॒नोप॑ धावति॒ स ए॒वास्मै᳚ प्राणापा॒नाभ्यां᳚ प्र॒जां प्र ज॑नयति वि॒न्दते᳚ प्र॒जां ॅवा॒यवे॑ नि॒युत्व॑त॒ आ ल॑भेत॒ ज्योगा॑मयावी प्रा॒णो वै वा॒युर॑पा॒नो नि॒युत् प्रा॑णापा॒नौ खलु॒ वा ए॒तस्मा॒ दप॑क्रामतो॒ यस्य॒ ज्योगा॒मय॑ति वा॒युमे॒व नि॒युत्व॑न्तꣳ॒॒ स्वेन॑ भाग॒धेये॒नोप॑ - [  ] \newline

\textbf{Pada Paata} \newline

अपेति॑ । क्रा॒म॒तः॒ । यः । अल᳚म् । प्र॒जाया॒ इति॑ प्र - जायै᳚ । सन्न् । प्र॒जामिति॑ प्र -जाम् । न । वि॒न्दते᳚ । वा॒युम् । ए॒व । नि॒युत्व॑न्त॒मिति॑ नि - युत्व॑न्तम् । स्वेन॑ । भा॒ग॒धेये॒नेति॑ भाग-धेये॑न । उपेति॑ । धा॒व॒ति॒ । सः । ए॒व । अ॒स्मै॒ । प्रा॒णा॒पा॒नाभ्या॒मिति॑ प्राण - अ॒पा॒नाभ्या᳚म् । प्र॒जामिति॑ प्र -जाम् । प्रेति॑ । ज॒न॒य॒ति॒ । वि॒न्दते᳚ । प्र॒जामिति॑ प्र -जाम् । वा॒यवे᳚ । नि॒युत्व॑त॒ इति॑ नि-युत्व॑ते । एति॑ । ल॒भे॒त॒ । ज्योगा॑मया॒वीति॒ ज्योक् - आ॒म॒या॒वी॒ । प्रा॒ण इति॑ प्र - अ॒नः । वै । वा॒युः । अ॒पा॒न इत्य॑प - अ॒नः । नि॒युदिति॑ नि - युत् । प्रा॒णा॒पा॒नाविति॑ प्राण-अ॒पा॒नौ । खलु॑ । वै । ए॒तस्मा᳚त् । अपेति॑ । क्रा॒म॒तः॒ । यस्य॑ । ज्योक् । आ॒मय॑ति । वा॒युम् । ए॒व । नि॒युत्व॑न्त॒मिति॑ नि - युत्व॑न्तम् । स्वेन॑ । भा॒ग॒धेये॒नेति॑ भाग-धेये॑न । उपेति॑ ।  \newline


\textbf{Krama Paata} \newline

अप॑ क्रामतः । क्रा॒म॒तो॒ यः । यो ऽल᳚म् । अल॑म् प्र॒जायै᳚ । प्र॒जायै॒ सन्न् । प्र॒जाया॒ इति॑ प्र - जायै᳚ । सन् प्र॒जाम् । प्र॒जाम् न । प्र॒जामिति॑ प्र - जाम् । न वि॒न्दते᳚ । वि॒न्दते॑ वा॒युम् । वा॒युमे॒व । ए॒व नि॒युत्व॑न्तम् । नि॒युत्व॑न्तꣳ॒॒ स्वेन॑ । नि॒युत्व॑न्त॒मिति॑ नि - युत्व॑न्तम् । स्वेन॑ भाग॒धेये॑न । भा॒ग॒धेये॒नोप॑ । भा॒ग॒धेये॒नेति॑ भाग - धेये॑न । उप॑ धावति । धा॒व॒ति॒ सः । स ए॒व । ए॒वास्मै᳚ । अ॒स्मै॒ प्रा॒णा॒पा॒नाभ्या᳚म् । प्रा॒णा॒पा॒नाभ्या᳚म् प्र॒जाम् । प्रा॒णा॒पा॒नाभ्या॒मिति॑ प्राण - अ॒पा॒नाभ्या᳚म् । प्र॒जाम् प्र । प्र॒जामिति॑ प्र - जाम् । प्र ज॑नयति । ज॒न॒य॒ति॒ वि॒न्दते᳚ । वि॒न्दते᳚ प्र॒जाम् । प्र॒जां ॅवा॒यवे᳚ । प्र॒जामिति॑ प्र - जाम् । वा॒यवे॑ नि॒युत्व॑ते । नि॒युत्व॑त॒ आ । नि॒युत्व॑त॒ इति॑ नि - युत्व॑ते । आ ल॑भेत । ल॒भे॒त॒ ज्योगा॑मयावी । ज्योगा॑मयावी प्रा॒णः । ज्योगा॑मया॒वीति॒ ज्योक् - आ॒म॒या॒वी॒ । प्रा॒णो वै । प्रा॒ण इति॑ प्र - अ॒नः । वै वा॒युः । वा॒युर॑पा॒नः । अ॒पा॒नो नि॒युत् । अ॒पा॒न इत्य॑प - अ॒नः । नि॒युत् प्रा॑णापा॒नौ । नि॒युदिति॑ नि - युत् । प्रा॒णा॒पा॒नौ खलु॑ । प्रा॒णा॒पा॒नाविति॑ प्राण - अ॒पा॒नौ । खलु॒ वै । वा ए॒तस्मा᳚त् । ए॒तस्मा॒दप॑ । अप॑ क्रामतः । क्रा॒म॒तो॒ यस्य॑ । यस्य॒ ज्योक् । ज्योगा॒मय॑ति । आ॒मय॑ति वा॒युम् । वा॒युमे॒व । ए॒व नि॒युत्व॑न्तम् । नि॒युत्व॑न्तꣳ॒॒ स्वेन॑ । नि॒युत्व॑न्त॒मिति॑ नि - युत्व॑न्तम् । स्वेन॑ भाग॒धेये॑न । भा॒ग॒धेये॒नोप॑ । भा॒ग॒धेये॒नेति॑ भाग - धेये॑न । उप॑ धावति \newline

\textbf{Jatai Paata} \newline

1. अप॑ क्रामतः क्राम॒तो ऽपाप॑ क्रामतः । \newline
2. क्रा॒म॒तो॒ यो यः क्रा॑मतः क्रामतो॒ यः । \newline
3. यो ऽल॒ मलं॒ ॅयो यो ऽल᳚म् । \newline
4. अल॑म् प्र॒जायै᳚ प्र॒जाया॒ अल॒ मल॑म् प्र॒जायै᳚ । \newline
5. प्र॒जायै॒ सन् थ्सन् प्र॒जायै᳚ प्र॒जायै॒ सन्न् । \newline
6. प्र॒जाया॒ इति॑ प्र - जायै᳚ । \newline
7. सन् प्र॒जाम् प्र॒जाꣳ सन् थ्सन् प्र॒जाम् । \newline
8. प्र॒जाम् न न प्र॒जाम् प्र॒जाम् न । \newline
9. प्र॒जामिति॑ प्र - जाम् । \newline
10. न वि॒न्दते॑ वि॒न्दते॒ न न वि॒न्दते᳚ । \newline
11. वि॒न्दते॑ वा॒युं ॅवा॒युं ॅवि॒न्दते॑ वि॒न्दते॑ वा॒युम् । \newline
12. वा॒यु मे॒वैव वा॒युं ॅवा॒यु मे॒व । \newline
13. ए॒व नि॒युत्व॑न्तम् नि॒युत्व॑न्त मे॒वैव नि॒युत्व॑न्तम् । \newline
14. नि॒युत्व॑न्तꣳ॒॒ स्वेन॒ स्वेन॑ नि॒युत्व॑न्तम् नि॒युत्व॑न्तꣳ॒॒ स्वेन॑ । \newline
15. नि॒युत्व॑न्त॒मिति॑ नि - युत्व॑न्तम् । \newline
16. स्वेन॑ भाग॒धेये॑न भाग॒धेये॑न॒ स्वेन॒ स्वेन॑ भाग॒धेये॑न । \newline
17. भा॒ग॒धेये॒नोपोप॑ भाग॒धेये॑न भाग॒धेये॒नोप॑ । \newline
18. भा॒ग॒धेये॒नेति॑ भाग - धेये॑न । \newline
19. उप॑ धावति धाव॒ त्युपोप॑ धावति । \newline
20. धा॒व॒ति॒ स स धा॑वति धावति॒ सः । \newline
21. स ए॒वैव स स ए॒व । \newline
22. ए॒वास्मा॑ अस्मा ए॒वैवास्मै᳚ । \newline
23. अ॒स्मै॒ प्रा॒णा॒पा॒नाभ्या᳚म् प्राणापा॒नाभ्या॑ मस्मा अस्मै प्राणापा॒नाभ्या᳚म् । \newline
24. प्रा॒णा॒पा॒नाभ्या᳚म् प्र॒जाम् प्र॒जाम् प्रा॑णापा॒नाभ्या᳚म् प्राणापा॒नाभ्या᳚म् प्र॒जाम् । \newline
25. प्रा॒णा॒पा॒नाभ्या॒मिति॑ प्राण - अ॒पा॒नाभ्या᳚म् । \newline
26. प्र॒जाम् प्र प्र प्र॒जाम् प्र॒जाम् प्र । \newline
27. प्र॒जामिति॑ प्र - जाम् । \newline
28. प्र ज॑नयति जनयति॒ प्र प्र ज॑नयति । \newline
29. ज॒न॒य॒ति॒ वि॒न्दते॑ वि॒न्दते॑ जनयति जनयति वि॒न्दते᳚ । \newline
30. वि॒न्दते᳚ प्र॒जाम् प्र॒जां ॅवि॒न्दते॑ वि॒न्दते᳚ प्र॒जाम् । \newline
31. प्र॒जां ॅवा॒यवे॑ वा॒यवे᳚ प्र॒जाम् प्र॒जां ॅवा॒यवे᳚ । \newline
32. प्र॒जामिति॑ प्र - जाम् । \newline
33. वा॒यवे॑ नि॒युत्व॑ते नि॒युत्व॑ते वा॒यवे॑ वा॒यवे॑ नि॒युत्व॑ते । \newline
34. नि॒युत्व॑त॒ आ नि॒युत्व॑ते नि॒युत्व॑त॒ आ । \newline
35. नि॒युत्व॑त॒ इति॑ नि - युत्व॑ते । \newline
36. आ ल॑भेत लभे॒ता ल॑भेत । \newline
37. ल॒भे॒त॒ ज्योगा॑मयावी॒ ज्योगा॑मयावी लभेत लभेत॒ ज्योगा॑मयावी । \newline
38. ज्योगा॑मयावी प्रा॒णः प्रा॒णो ज्योगा॑मयावी॒ ज्योगा॑मयावी प्रा॒णः । \newline
39. ज्योगा॑मया॒वीति॒ ज्योक् - आ॒म॒या॒वी॒ । \newline
40. प्रा॒णो वै वै प्रा॒णः प्रा॒णो वै । \newline
41. प्रा॒ण इति॑ प्र - अ॒नः । \newline
42. वै वा॒युर् वा॒युर् वै वै वा॒युः । \newline
43. वा॒यु र॑पा॒नो॑ ऽपा॒नो वा॒युर् वा॒यु र॑पा॒नः । \newline
44. अ॒पा॒नो नि॒युन् नि॒यु द॑पा॒नो॑ ऽपा॒नो नि॒युत् । \newline
45. अ॒पा॒न इत्य॑प - अ॒नः । \newline
46. नि॒युत् प्रा॑णापा॒नौ प्रा॑णापा॒नौ नि॒युन् नि॒युत् प्रा॑णापा॒नौ । \newline
47. नि॒युदिति॑ नि - युत् । \newline
48. प्रा॒णा॒पा॒नौ खलु॒ खलु॑ प्राणापा॒नौ प्रा॑णापा॒नौ खलु॑ । \newline
49. प्रा॒णा॒पा॒नाविति॑ प्राण - अ॒पा॒नौ । \newline
50. खलु॒ वै वै खलु॒ खलु॒ वै । \newline
51. वा ए॒तस्मा॑ दे॒तस्मा॒द् वै वा ए॒तस्मा᳚त् । \newline
52. ए॒तस्मा॒ दपा पै॒तस्मा॑ दे॒तस्मा॒ दप॑ । \newline
53. अप॑ क्रामतः क्राम॒तो ऽपाप॑ क्रामतः । \newline
54. क्रा॒म॒तो॒ यस्य॒ यस्य॑ क्रामतः क्रामतो॒ यस्य॑ । \newline
55. यस्य॒ ज्योग् ज्योग् यस्य॒ यस्य॒ ज्योक् । \newline
56. ज्योगा॒मय॑ त्या॒मय॑ति॒ ज्योग् ज्योगा॒मय॑ति । \newline
57. आ॒मय॑ति वा॒युं ॅवा॒यु मा॒मय॑ त्या॒मय॑ति वा॒युम् । \newline
58. वा॒यु मे॒वैव वा॒युं ॅवा॒यु मे॒व । \newline
59. ए॒व नि॒युत्व॑न्तम् नि॒युत्व॑न्त मे॒वैव नि॒युत्व॑न्तम् । \newline
60. नि॒युत्व॑न्तꣳ॒॒ स्वेन॒ स्वेन॑ नि॒युत्व॑न्तम् नि॒युत्व॑न्तꣳ॒॒ स्वेन॑ । \newline
61. नि॒युत्व॑न्त॒मिति॑ नि - युत्व॑न्तम् । \newline
62. स्वेन॑ भाग॒धेये॑न भाग॒धेये॑न॒ स्वेन॒ स्वेन॑ भाग॒धेये॑न । \newline
63. भा॒ग॒धेये॒नोपोप॑ भाग॒धेये॑न भाग॒धेये॒नोप॑ । \newline
64. भा॒ग॒धेये॒नेति॑ भाग - धेये॑न । \newline
65. उप॑ धावति धाव॒ त्युपोप॑ धावति । \newline

\textbf{Ghana Paata } \newline

1. अप॑ क्रामतः क्राम॒तो ऽपाप॑ क्रामतो॒ यो यः क्रा॑म॒तो ऽपाप॑ क्रामतो॒ यः । \newline
2. क्रा॒म॒तो॒ यो यः क्रा॑मतः क्रामतो॒ यो ऽल॒ मलं॒ ॅयः क्रा॑मतः क्रामतो॒ यो ऽल᳚म् । \newline
3. यो ऽल॒ मलं॒ ॅयो यो ऽल॑म् प्र॒जायै᳚ प्र॒जाया॒ अलं॒ ॅयो यो ऽल॑म् प्र॒जायै᳚ । \newline
4. अल॑म् प्र॒जायै᳚ प्र॒जाया॒ अल॒ मल॑म् प्र॒जायै॒ सन् थ्सन् प्र॒जाया॒ अल॒ मल॑म् प्र॒जायै॒ सन्न् । \newline
5. प्र॒जायै॒ सन् थ्सन् प्र॒जायै᳚ प्र॒जायै॒ सन् प्र॒जाम् प्र॒जाꣳ सन् प्र॒जायै᳚ प्र॒जायै॒ सन् प्र॒जाम् । \newline
6. प्र॒जाया॒ इति॑ प्र - जायै᳚ । \newline
7. सन् प्र॒जाम् प्र॒जाꣳ सन् थ्सन् प्र॒जान्न न प्र॒जाꣳ सन् थ्सन् प्र॒जान्न । \newline
8. प्र॒जान्न न प्र॒जाम् प्र॒जान्न वि॒न्दते॑ वि॒न्दते॒ न प्र॒जाम् प्र॒जान्न वि॒न्दते᳚ । \newline
9. प्र॒जामिति॑ प्र - जाम् । \newline
10. न वि॒न्दते॑ वि॒न्दते॒ न न वि॒न्दते॑ वा॒युं ॅवा॒युं ॅवि॒न्दते॒ न न वि॒न्दते॑ वा॒युम् । \newline
11. वि॒न्दते॑ वा॒युं ॅवा॒युं ॅवि॒न्दते॑ वि॒न्दते॑ वा॒यु मे॒वैव वा॒युं ॅवि॒न्दते॑ वि॒न्दते॑ वा॒यु मे॒व । \newline
12. वा॒यु मे॒वैव वा॒युं ॅवा॒यु मे॒व नि॒युत्व॑न्तन् नि॒युत्व॑न्त मे॒व वा॒युं ॅवा॒यु मे॒व नि॒युत्व॑न्तम् । \newline
13. ए॒व नि॒युत्व॑न्तन् नि॒युत्व॑न्त मे॒वैव नि॒युत्व॑न्तꣳ॒॒ स्वेन॒ स्वेन॑ नि॒युत्व॑न्त मे॒वैव नि॒युत्व॑न्तꣳ॒॒ स्वेन॑ । \newline
14. नि॒युत्व॑न्तꣳ॒॒ स्वेन॒ स्वेन॑ नि॒युत्व॑न्तन् नि॒युत्व॑न्तꣳ॒॒ स्वेन॑ भाग॒धेये॑न भाग॒धेये॑न॒ स्वेन॑ नि॒युत्व॑न्तन् नि॒युत्व॑न्तꣳ॒॒ स्वेन॑ भाग॒धेये॑न । \newline
15. नि॒युत्व॑न्त॒मिति॑ नि - युत्व॑न्तम् । \newline
16. स्वेन॑ भाग॒धेये॑न भाग॒धेये॑न॒ स्वेन॒ स्वेन॑ भाग॒धेये॒नो पोप॑ भाग॒धेये॑न॒ स्वेन॒ स्वेन॑ भाग॒धेये॒नोप॑ । \newline
17. भा॒ग॒धेये॒नो पोप॑ भाग॒धेये॑न भाग॒धेये॒नोप॑ धावति धाव॒त्युप॑ भाग॒धेये॑न भाग॒धेये॒नोप॑ धावति । \newline
18. भा॒ग॒धेये॒नेति॑ भाग - धेये॑न । \newline
19. उप॑ धावति धाव॒ त्युपोप॑ धावति॒ स स धा॑व॒ त्युपोप॑ धावति॒ सः । \newline
20. धा॒व॒ति॒ स स धा॑वति धावति॒ स ए॒वैव स धा॑वति धावति॒ स ए॒व । \newline
21. स ए॒वैव स स ए॒वास्मा॑ अस्मा ए॒व स स ए॒वास्मै᳚ । \newline
22. ए॒वास्मा॑ अस्मा ए॒वैवास्मै᳚ प्राणापा॒नाभ्या᳚म् प्राणापा॒नाभ्या॑ मस्मा ए॒वैवास्मै᳚ प्राणापा॒नाभ्या᳚म् । \newline
23. अ॒स्मै॒ प्रा॒णा॒पा॒नाभ्या᳚म् प्राणापा॒नाभ्या॑ मस्मा अस्मै प्राणापा॒नाभ्या᳚म् प्र॒जाम् प्र॒जाम् प्रा॑णापा॒नाभ्या॑ मस्मा अस्मै प्राणापा॒नाभ्या᳚म् प्र॒जाम् । \newline
24. प्रा॒णा॒पा॒नाभ्या᳚म् प्र॒जाम् प्र॒जाम् प्रा॑णापा॒नाभ्या᳚म् प्राणापा॒नाभ्या᳚म् प्र॒जाम् प्र प्र प्र॒जाम् प्रा॑णापा॒नाभ्या᳚म् प्राणापा॒नाभ्या᳚म् प्र॒जाम् प्र । \newline
25. प्रा॒णा॒पा॒नाभ्या॒मिति॑ प्राण - अ॒पा॒नाभ्या᳚म् । \newline
26. प्र॒जाम् प्र प्र प्र॒जाम् प्र॒जाम् प्र ज॑नयति जनयति॒ प्र प्र॒जाम् प्र॒जाम् प्र ज॑नयति । \newline
27. प्र॒जामिति॑ प्र - जाम् । \newline
28. प्र ज॑नयति जनयति॒ प्र प्र ज॑नयति वि॒न्दते॑ वि॒न्दते॑ जनयति॒ प्र प्र ज॑नयति वि॒न्दते᳚ । \newline
29. ज॒न॒य॒ति॒ वि॒न्दते॑ वि॒न्दते॑ जनयति जनयति वि॒न्दते᳚ प्र॒जाम् प्र॒जां ॅवि॒न्दते॑ जनयति जनयति वि॒न्दते᳚ प्र॒जाम् । \newline
30. वि॒न्दते᳚ प्र॒जाम् प्र॒जां ॅवि॒न्दते॑ वि॒न्दते᳚ प्र॒जां ॅवा॒यवे॑ वा॒यवे᳚ प्र॒जां ॅवि॒न्दते॑ वि॒न्दते᳚ प्र॒जां ॅवा॒यवे᳚ । \newline
31. प्र॒जां ॅवा॒यवे॑ वा॒यवे᳚ प्र॒जाम् प्र॒जां ॅवा॒यवे॑ नि॒युत्व॑ते नि॒युत्व॑ते वा॒यवे᳚ प्र॒जाम् प्र॒जां ॅवा॒यवे॑ नि॒युत्व॑ते । \newline
32. प्र॒जामिति॑ प्र - जाम् । \newline
33. वा॒यवे॑ नि॒युत्व॑ते नि॒युत्व॑ते वा॒यवे॑ वा॒यवे॑ नि॒युत्व॑त॒ आ नि॒युत्व॑ते वा॒यवे॑ वा॒यवे॑ नि॒युत्व॑त॒ आ । \newline
34. नि॒युत्व॑त॒ आ नि॒युत्व॑ते नि॒युत्व॑त॒ आ ल॑भेत लभे॒ता नि॒युत्व॑ते नि॒युत्व॑त॒ आ ल॑भेत । \newline
35. नि॒युत्व॑त॒ इति॑ नि - युत्व॑ते । \newline
36. आ ल॑भेत लभे॒ता ल॑भेत॒ ज्योगा॑मयावी॒ ज्योगा॑मयावी लभे॒ता ल॑भेत॒ ज्योगा॑मयावी । \newline
37. ल॒भे॒त॒ ज्योगा॑मयावी॒ ज्योगा॑मयावी लभेत लभेत॒ ज्योगा॑मयावी प्रा॒णः प्रा॒णो ज्योगा॑मयावी लभेत लभेत॒ ज्योगा॑मयावी प्रा॒णः । \newline
38. ज्योगा॑मयावी प्रा॒णः प्रा॒णो ज्योगा॑मयावी॒ ज्योगा॑मयावी प्रा॒णो वै वै प्रा॒णो ज्योगा॑मयावी॒ ज्योगा॑मयावी प्रा॒णो वै । \newline
39. ज्योगा॑मया॒वीति॒ ज्योक् - आ॒म॒या॒वी॒ । \newline
40. प्रा॒णो वै वै प्रा॒णः प्रा॒णो वै वा॒युर् वा॒युर् वै प्रा॒णः प्रा॒णो वै वा॒युः । \newline
41. प्रा॒ण इति॑ प्र - अ॒नः । \newline
42. वै वा॒युर् वा॒युर् वै वै वा॒यु र॑पा॒नो॑ ऽपा॒नो वा॒युर् वै वै वा॒यु र॑पा॒नः । \newline
43. वा॒यु र॑पा॒नो॑ ऽपा॒नो वा॒युर् वा॒यु र॑पा॒नो नि॒युन् नि॒युद॑पा॒नो वा॒युर् वा॒यु र॑पा॒नो नि॒युत् । \newline
44. अ॒पा॒नो नि॒युन् नि॒यु द॑पा॒नो॑ ऽपा॒नो नि॒युत् प्रा॑णापा॒नौ प्रा॑णापा॒नौ नि॒यु द॑पा॒नो॑ ऽपा॒नो नि॒युत् प्रा॑णापा॒नौ । \newline
45. अ॒पा॒न इत्य॑प - अ॒नः । \newline
46. नि॒युत् प्रा॑णापा॒नौ प्रा॑णापा॒नौ नि॒युन् नि॒युत् प्रा॑णापा॒नौ खलु॒ खलु॑ प्राणापा॒नौ नि॒युन् नि॒युत् प्रा॑णापा॒नौ खलु॑ । \newline
47. नि॒युदिति॑ नि - युत् । \newline
48. प्रा॒णा॒पा॒नौ खलु॒ खलु॑ प्राणापा॒नौ प्रा॑णापा॒नौ खलु॒ वै वै खलु॑ प्राणापा॒नौ प्रा॑णापा॒नौ खलु॒ वै । \newline
49. प्रा॒णा॒पा॒नाविति॑ प्राण - अ॒पा॒नौ । \newline
50. खलु॒ वै वै खलु॒ खलु॒ वा ए॒तस्मा॑ दे॒तस्मा॒द् वै खलु॒ खलु॒ वा ए॒तस्मा᳚त् । \newline
51. वा ए॒तस्मा॑ दे॒तस्मा॒द् वै वा ए॒तस्मा॒ दपा पै॒तस्मा॒द् वै वा ए॒तस्मा॒ दप॑ । \newline
52. ए॒तस्मा॒ दपा पै॒तस्मा॑ दे॒तस्मा॒ दप॑ क्रामतः क्राम॒तो ऽपै॒तस्मा॑ दे॒तस्मा॒ दप॑ क्रामतः । \newline
53. अप॑ क्रामतः क्राम॒तो ऽपाप॑ क्रामतो॒ यस्य॒ यस्य॑ क्राम॒तो ऽपाप॑ क्रामतो॒ यस्य॑ । \newline
54. क्रा॒म॒तो॒ यस्य॒ यस्य॑ क्रामतः क्रामतो॒ यस्य॒ ज्योग् ज्योग् यस्य॑ क्रामतः क्रामतो॒ यस्य॒ ज्योक् । \newline
55. यस्य॒ ज्योग् ज्योग् यस्य॒ यस्य॒ ज्योगा॒मय॑ त्या॒मय॑ति॒ ज्योग् यस्य॒ यस्य॒ ज्योगा॒मय॑ति । \newline
56. ज्योगा॒मय॑ त्या॒मय॑ति॒ ज्योग् ज्योगा॒मय॑ति वा॒युं ॅवा॒यु मा॒मय॑ति॒ ज्योग् ज्योगा॒मय॑ति वा॒युम् । \newline
57. आ॒मय॑ति वा॒युं ॅवा॒यु मा॒मय॑ त्या॒मय॑ति वा॒यु मे॒वैव वा॒यु मा॒मय॑ त्या॒मय॑ति वा॒यु मे॒व । \newline
58. वा॒यु मे॒वैव वा॒युं ॅवा॒यु मे॒व नि॒युत्व॑न्तन् नि॒युत्व॑न्त मे॒व वा॒युं ॅवा॒यु मे॒व नि॒युत्व॑न्तम् । \newline
59. ए॒व नि॒युत्व॑न्तन् नि॒युत्व॑न्त मे॒वैव नि॒युत्व॑न्तꣳ॒॒ स्वेन॒ स्वेन॑ नि॒युत्व॑न्त मे॒वैव नि॒युत्व॑न्तꣳ॒॒ स्वेन॑ । \newline
60. नि॒युत्व॑न्तꣳ॒॒ स्वेन॒ स्वेन॑ नि॒युत्व॑न्तन् नि॒युत्व॑न्तꣳ॒॒ स्वेन॑ भाग॒धेये॑न भाग॒धेये॑न॒ स्वेन॑ नि॒युत्व॑न्तन् नि॒युत्व॑न्तꣳ॒॒ स्वेन॑ भाग॒धेये॑न । \newline
61. नि॒युत्व॑न्त॒मिति॑ नि - युत्व॑न्तम् । \newline
62. स्वेन॑ भाग॒धेये॑न भाग॒धेये॑न॒ स्वेन॒ स्वेन॑ भाग॒धेये॒नो पोप॑ भाग॒धेये॑न॒ स्वेन॒ स्वेन॑ भाग॒धेये॒नोप॑ । \newline
63. भा॒ग॒धेये॒नो पोप॑ भाग॒धेये॑न भाग॒धेये॒नोप॑ धावति धाव॒त्युप॑ भाग॒धेये॑न भाग॒धेये॒नोप॑ धावति । \newline
64. भा॒ग॒धेये॒नेति॑ भाग - धेये॑न । \newline
65. उप॑ धावति धाव॒ त्युपोप॑ धावति॒ स स धा॑व॒ त्युपोप॑ धावति॒ सः । \newline
\pagebreak
\markright{ TS 2.1.1.4  \hfill https://www.vedavms.in \hfill}

\section{ TS 2.1.1.4 }

\textbf{TS 2.1.1.4 } \newline
\textbf{Samhita Paata} \newline

धावति॒ स ए॒वास्मि॑न् प्राणापा॒नौ द॑धात्यु॒त यदी॒तासु॒र्भव॑ति॒ जीव॑त्ये॒व प्र॒जाप॑ति॒र्वा इ॒दमेक॑ आसी॒थ् सो॑ऽकामयत प्र॒जाः प॒शून्थ् सृ॑जे॒येति॒ स आ॒त्मनो॑ व॒पामुद॑क्खिद॒त् ताम॒ग्नौ प्रागृ॑ह्णा॒त् ततो॒ऽजस्तू॑प॒रः सम॑भव॒त् तꣳ स्वायै॑ दे॒वता॑या॒ आ ऽल॑भत॒ ततो॒ वै स प्र॒जाः प॒शून॑सृजत॒ यः प्र॒जाका॑मः - [  ] \newline

\textbf{Pada Paata} \newline

धा॒व॒ति॒ । सः । ए॒व । अ॒स्मि॒न्न् । प्रा॒णा॒पा॒नाविति॑ प्राण - अ॒पा॒नौ । द॒धा॒ति॒ । उ॒त । यदि॑ । इ॒तासु॒रिती॒त-अ॒सुः॒ । भव॑ति । जीव॑ति । ए॒व । प्र॒जाप॑ति॒रिति॑ प्र॒जा - प॒तिः॒ । वै । इ॒दम् । एकः॑ । आ॒सी॒त् । सः । अ॒का॒म॒य॒त॒ । प्र॒जा इति॑ प्र - जाः । प॒शून् । सृ॒जे॒य॒ । इति॑ । सः । आ॒त्मनः॑ । व॒पाम् । उदिति॑ । अ॒क्खि॒द॒त् । ताम् । अ॒ग्नौ । प्रेति॑ । अ॒गृ॒ह्णा॒त् । ततः॑ । अ॒जः । तू॒प॒रः । समिति॑ । अ॒भ॒व॒त् । तम् । स्वायै᳚ । दे॒वता॑यै । एति॑ । अ॒ल॒भ॒त॒ । ततः॑ । वै । सः । प्र॒जा इति॑ प्र - जाः । प॒शून् । अ॒सृ॒ज॒त॒ । यः । प्र॒जाका॑म॒ इति॑ प्र॒जा - का॒मः॒ ।  \newline


\textbf{Krama Paata} \newline

धा॒व॒ति॒ सः । स ए॒व । ए॒वास्मिन्न्॑ । अ॒स्मि॒न् प्रा॒णा॒पा॒नौ । प्रा॒णा॒पा॒नौ द॑धाति । प्रा॒णा॒पा॒नाविति॑ प्राण - अ॒पा॒नौ । द॒धा॒त्यु॒त । उ॒त यदि॑ । यदी॒तासुः॑ । इ॒तासु॒र् भव॑ति । इ॒तासु॒रिती॒त - अ॒सुः॒ । भव॑ति॒ जीव॑ति । जीव॑त्ये॒व । ए॒व प्र॒जाप॑तिः । प्र॒जाप॑ति॒र् वै । प्र॒जाप॑ति॒रिति॑ प्र॒जा - प॒तिः॒ । वा इ॒दम् । इ॒दमेकः॑ । एक॑ आसीत् । आ॒सी॒थ् सः । सो॑ऽकामयत । अ॒का॒म॒य॒त॒ प्र॒जाः । प्र॒जाः प॒शून् । प्र॒जा इति॑ प्र - जाः । प॒शून्थ् सृ॑जेय । सृ॒जे॒येति॑ । इति॒ सः । स आ॒त्मनः॑ । आ॒त्मनो॑ व॒पाम् । व॒पामुत् । उद॑क्खिदत् । अ॒क्खि॒द॒त् ताम् । ताम॒ग्नौ । अ॒ग्नौ प्र । प्रागृ॑ह्णात् । अ॒गृ॒ह्णा॒त् ततः॑ । ततो॒ऽजः । अ॒जस्तू॑प॒रः । तू॒प॒रः सम् । सम॑भवत् । अ॒भ॒व॒त् तम् । तꣳ स्वायै᳚ । स्वायै॑ दे॒वता॑यै । दे॒वता॑या॒ आ । आऽल॑भत । अ॒ल॒भ॒त॒ ततः॑ । ततो॒ वै । वै सः । स प्र॒जाः । प्र॒जाः प॒शून् । प्र॒जा इति॑ प्र - जाः । प॒शून॑सृजत । अ॒सृ॒ज॒त॒ यः । यः प्र॒जाका॑मः । प्र॒जाका॑मः  प॒शुका॑मः । प्र॒जाका॑म॒ इति॑ प्र॒जा - का॒मः॒ \newline

\textbf{Jatai Paata} \newline

1. धा॒व॒ति॒ स स धा॑वति धावति॒ सः । \newline
2. स ए॒वैव स स ए॒व । \newline
3. ए॒वास्मि॑न् नस्मिन् ने॒वै वास्मिन्न्॑ । \newline
4. अ॒स्मि॒न् प्रा॒णा॒पा॒नौ प्रा॑णापा॒ना व॑स्मिन् नस्मिन् प्राणापा॒नौ । \newline
5. प्रा॒णा॒पा॒नौ द॑धाति दधाति प्राणापा॒नौ प्रा॑णापा॒नौ द॑धाति । \newline
6. प्रा॒णा॒पा॒नाविति॑ प्राण - अ॒पा॒नौ । \newline
7. द॒धा॒ त्यु॒तोत द॑धाति दधात्यु॒त । \newline
8. उ॒त यदि॒ यद्यु॒तोत यदि॑ । \newline
9. यदी॒तासु॑ रि॒तासु॒र् यदि॒ यदी॒तासुः॑ । \newline
10. इ॒तासु॒र् भव॑ति॒ भव॑ती॒ तासु॑ रि॒तासु॒र् भव॑ति । \newline
11. इ॒तासु॒रिती॒त - अ॒सुः॒ । \newline
12. भव॑ति॒ जीव॑ति॒ जीव॑ति॒ भव॑ति॒ भव॑ति॒ जीव॑ति । \newline
13. जीव॑ त्ये॒वैव जीव॑ति॒ जीव॑ त्ये॒व । \newline
14. ए॒व प्र॒जाप॑तिः प्र॒जाप॑ति रे॒वैव प्र॒जाप॑तिः । \newline
15. प्र॒जाप॑ति॒र् वै वै प्र॒जाप॑तिः प्र॒जाप॑ति॒र् वै । \newline
16. प्र॒जाप॑ति॒रिति॑ प्र॒जा - प॒तिः॒ । \newline
17. वा इ॒द मि॒दं ॅवै वा इ॒दम् । \newline
18. इ॒द मेक॒ एक॑ इ॒द मि॒द मेकः॑ । \newline
19. एक॑ आसी दासी॒ देक॒ एक॑ आसीत् । \newline
20. आ॒सी॒थ् स स आ॑सी दासी॒थ् सः । \newline
21. सो॑ ऽकामयता कामयत॒ स सो॑ ऽकामयत । \newline
22. अ॒का॒म॒य॒त॒ प्र॒जाः प्र॒जा अ॑कामयता कामयत प्र॒जाः । \newline
23. प्र॒जाः प॒शून् प॒शून् प्र॒जाः प्र॒जाः प॒शून् । \newline
24. प्र॒जा इति॑ प्र - जाः । \newline
25. प॒शून् थ्सृ॑जेय सृजेय प॒शून् प॒शून् थ्सृ॑जेय । \newline
26. सृ॒जे॒ये तीति॑ सृजेय सृजे॒ये ति॑ । \newline
27. इति॒ स स इतीति॒ सः । \newline
28. स आ॒त्मन॑ आ॒त्मनः॒ स स आ॒त्मनः॑ । \newline
29. आ॒त्मनो॑ व॒पां ॅव॒पा मा॒त्मन॑ आ॒त्मनो॑ व॒पाम् । \newline
30. व॒पा मुदुद् व॒पां ॅव॒पा मुत् । \newline
31. उद॑क्खिद दक्खिद॒ दुदु द॑क्खिदत् । \newline
32. अ॒क्खि॒द॒त् ताम् ता म॑क्खिद दक्खिद॒त् ताम् । \newline
33. ता म॒ग्ना व॒ग्नौ ताम् ता म॒ग्नौ । \newline
34. अ॒ग्नौ प्र प्राग्ना व॒ग्नौ प्र । \newline
35. प्रागृ॑ह्णा दगृह्णा॒त् प्र प्रागृ॑ह्णात् । \newline
36. अ॒गृ॒ह्णा॒त् तत॒ स्ततो॑ ऽगृह्णा दगृह्णा॒त् ततः॑ । \newline
37. ततो॒ ऽजो॑ ऽज स्तत॒ स्ततो॒ ऽजः । \newline
38. अ॒ज स्तू॑प॒र स्तू॑प॒रो᳚(1॒) ऽजो॑ ऽज स्तू॑प॒रः । \newline
39. तू॒प॒रः सꣳ सम् तू॑प॒र स्तू॑प॒रः सम् । \newline
40. स म॑भव दभव॒थ् सꣳ स म॑भवत् । \newline
41. अ॒भ॒व॒त् तम् त म॑भव दभव॒त् तम् । \newline
42. तꣳ स्वायै॒ स्वायै॒ तम् तꣳ स्वायै᳚ । \newline
43. स्वायै॑ दे॒वता॑यै दे॒वता॑यै॒ स्वायै॒ स्वायै॑ दे॒वता॑यै । \newline
44. दे॒वता॑या॒ आ दे॒वता॑यै दे॒वता॑या॒ आ । \newline
45. आ ऽल॑भता लभ॒ता ऽल॑भत । \newline
46. अ॒ल॒भ॒त॒ तत॒ स्ततो॑ ऽलभता लभत॒ ततः॑ । \newline
47. ततो॒ वै वै तत॒ स्ततो॒ वै । \newline
48. वै स स वै वै सः । \newline
49. स प्र॒जाः प्र॒जाः स स प्र॒जाः । \newline
50. प्र॒जाः प॒शून् प॒शून् प्र॒जाः प्र॒जाः प॒शून् । \newline
51. प्र॒जा इति॑ प्र - जाः । \newline
52. प॒शू न॑सृजता सृजत प॒शून् प॒शू न॑सृजत । \newline
53. अ॒सृ॒ज॒त॒ यो यो॑ ऽसृजता सृजत॒ यः । \newline
54. यः प्र॒जाका॑मः प्र॒जाका॑मो॒ यो यः प्र॒जाका॑मः । \newline
55. प्र॒जाका॑मः प॒शुका॑मः प॒शुका॑मः प्र॒जाका॑मः प्र॒जाका॑मः प॒शुका॑मः । \newline
56. प्र॒जाका॑म॒ इति॑ प्र॒जा - का॒मः॒ । \newline

\textbf{Ghana Paata } \newline

1. धा॒व॒ति॒ स स धा॑वति धावति॒ स ए॒वैव स धा॑वति धावति॒ स ए॒व । \newline
2. स ए॒वैव स स ए॒वास्मि॑न् नस्मिन् ने॒व स स ए॒वास्मिन्न्॑ । \newline
3. ए॒वास्मि॑न् नस्मिन् ने॒वैवास्मि॑न् प्राणापा॒नौ प्रा॑णापा॒ना व॑स्मिन् ने॒वैवास्मि॑न् प्राणापा॒नौ । \newline
4. अ॒स्मि॒न् प्रा॒णा॒पा॒नौ प्रा॑णापा॒ना व॑स्मिन् नस्मिन् प्राणापा॒नौ द॑धाति दधाति प्राणापा॒ना व॑स्मिन् नस्मिन् प्राणापा॒नौ द॑धाति । \newline
5. प्रा॒णा॒पा॒नौ द॑धाति दधाति प्राणापा॒नौ प्रा॑णापा॒नौ द॑धा त्यु॒तोत द॑धाति प्राणापा॒नौ प्रा॑णापा॒नौ द॑धात्यु॒त । \newline
6. प्रा॒णा॒पा॒नाविति॑ प्राण - अ॒पा॒नौ । \newline
7. द॒धा॒त्यु॒तोत द॑धाति दधात्यु॒त यदि॒ यद्यु॒त द॑धाति दधात्यु॒त यदि॑ । \newline
8. उ॒त यदि॒ यद्यु॒तोत यदी॒तासु॑ रि॒तासु॒र् यद्यु॒तोत यदी॒तासुः॑ । \newline
9. यदी॒तासु॑ रि॒तासु॒र् यदि॒ यदी॒तासु॒र् भव॑ति॒ भव॑ती॒तासु॒र् यदि॒ यदी॒तासु॒र् भव॑ति । \newline
10. इ॒तासु॒र् भव॑ति॒ भव॑ती॒तासु॑ रि॒तासु॒र् भव॑ति॒ जीव॑ति॒ जीव॑ति॒ भव॑ती॒तासु॑ रि॒तासु॒र् भव॑ति॒ जीव॑ति । \newline
11. इ॒तासु॒रिती॒त - अ॒सुः॒ । \newline
12. भव॑ति॒ जीव॑ति॒ जीव॑ति॒ भव॑ति॒ भव॑ति॒ जीव॑त्ये॒वैव जीव॑ति॒ भव॑ति॒ भव॑ति॒ जीव॑त्ये॒व । \newline
13. जीव॑त्ये॒वैव जीव॑ति॒ जीव॑त्ये॒व प्र॒जाप॑तिः प्र॒जाप॑तिरे॒व जीव॑ति॒ जीव॑त्ये॒व प्र॒जाप॑तिः । \newline
14. ए॒व प्र॒जाप॑तिः प्र॒जाप॑ति रे॒वैव प्र॒जाप॑ति॒र् वै वै प्र॒जाप॑ति रे॒वैव प्र॒जाप॑ति॒र् वै । \newline
15. प्र॒जाप॑ति॒र् वै वै प्र॒जाप॑तिः प्र॒जाप॑ति॒र् वा इ॒द मि॒दं ॅवै प्र॒जाप॑तिः प्र॒जाप॑ति॒र् वा इ॒दम् । \newline
16. प्र॒जाप॑ति॒रिति॑ प्र॒जा - प॒तिः॒ । \newline
17. वा इ॒द मि॒दं ॅवै वा इ॒द मेक॒ एक॑ इ॒दं ॅवै वा इ॒द मेकः॑ । \newline
18. इ॒द मेक॒ एक॑ इ॒द मि॒द मेक॑ आसी दासी॒ देक॑ इ॒द मि॒द मेक॑ आसीत् । \newline
19. एक॑ आसी दासी॒ देक॒ एक॑ आसी॒थ् स स आ॑सी॒ देक॒ एक॑ आसी॒थ् सः । \newline
20. आ॒सी॒थ् स स आ॑सी दासी॒थ् सो॑ ऽकामयता कामयत॒ स आ॑सी दासी॒थ् सो॑ ऽकामयत । \newline
21. सो॑ ऽकामयता कामयत॒ स सो॑ ऽकामयत प्र॒जाः प्र॒जा अ॑कामयत॒ स सो॑ ऽकामयत प्र॒जाः । \newline
22. अ॒का॒म॒य॒त॒ प्र॒जाः प्र॒जा अ॑कामयता कामयत प्र॒जाः प॒शून् प॒शून् प्र॒जा अ॑कामयता कामयत प्र॒जाः प॒शून् । \newline
23. प्र॒जाः प॒शून् प॒शून् प्र॒जाः प्र॒जाः प॒शून् थ्सृ॑जेय सृजेय प॒शून् प्र॒जाः प्र॒जाः प॒शून् थ्सृ॑जेय । \newline
24. प्र॒जा इति॑ प्र - जाः । \newline
25. प॒शून् थ्सृ॑जेय सृजेय प॒शून् प॒शून् थ्सृ॑जे॒ये तीति॑ सृजेय प॒शून् प॒शून् थ्सृ॑जे॒ये ति॑ । \newline
26. सृ॒जे॒ये तीति॑ सृजेय सृजे॒ये ति॒ स स इति॑ सृजेय सृजे॒ये ति॒ सः । \newline
27. इति॒ स स इतीति॒ स आ॒त्मन॑ आ॒त्मनः॒ स इतीति॒ स आ॒त्मनः॑ । \newline
28. स आ॒त्मन॑ आ॒त्मनः॒ स स आ॒त्मनो॑ व॒पां ॅव॒पा मा॒त्मनः॒ स स आ॒त्मनो॑ व॒पाम् । \newline
29. आ॒त्मनो॑ व॒पां ॅव॒पा मा॒त्मन॑ आ॒त्मनो॑ व॒पा मुदुद् व॒पा मा॒त्मन॑ आ॒त्मनो॑ व॒पा मुत् । \newline
30. व॒पा मुदुद् व॒पां ॅव॒पा मुद॑क्खि ददक्खि द॒दुद् व॒पां ॅव॒पा मुद॑क्खिदत् । \newline
31. उद॑क्खि ददक्खि द॒दु दुद॑क्खिद॒त् ताम् ता म॑क्खिद॒ दुदुद॑क्खिद॒त् ताम् । \newline
32. अ॒क्खि॒द॒त् ताम् ता म॑क्खि ददक्खिद॒त् ता म॒ग्ना व॒ग्नौ ता म॑क्खि ददक्खिद॒त् ता म॒ग्नौ । \newline
33. ता म॒ग्ना व॒ग्नौ ताम् ता म॒ग्नौ प्र प्राग्नौ ताम् ता म॒ग्नौ प्र । \newline
34. अ॒ग्नौ प्र प्राग्ना व॒ग्नौ प्रागृ॑ह्णा दगृह्णा॒त् प्राग्ना व॒ग्नौ प्रागृ॑ह्णात् । \newline
35. प्रागृ॑ह्णा दगृह्णा॒त् प्र प्रागृ॑ह्णा॒त् तत॒स्ततो॑ ऽगृह्णा॒त् प्र प्रागृ॑ह्णा॒त् ततः॑ । \newline
36. अ॒गृ॒ह्णा॒त् तत॒स्ततो॑ ऽगृह्णा दगृह्णा॒त् ततो॒ ऽजो॑ ऽजस्ततो॑ ऽगृह्णा दगृह्णा॒त् ततो॒ ऽजः । \newline
37. ततो॒ ऽजो॑ ऽज स्तत॒ स्ततो॒ ऽज स्तू॑प॒र स्तू॑प॒रो॑ ऽज स्तत॒ स्ततो॒ ऽज स्तू॑प॒रः । \newline
38. अ॒ज स्तू॑प॒र स्तू॑प॒रो᳚(1॒) ऽजो॑ ऽज स्तू॑प॒रः सꣳ सम् तू॑प॒रो᳚(1॒) ऽजो॑ ऽज स्तू॑प॒रः सम् । \newline
39. तू॒प॒रः सꣳ सम् तू॑प॒र स्तू॑प॒रः स म॑भव दभव॒थ् सम् तू॑प॒र स्तू॑प॒रः स म॑भवत् । \newline
40. स म॑भव दभव॒थ् सꣳ स म॑भव॒त् तम् त म॑भव॒थ् सꣳ स म॑भव॒त् तम् । \newline
41. अ॒भ॒व॒त् तम् त म॑भव दभव॒त् तꣳ स्वायै॒ स्वायै॒ त म॑भव दभव॒त् तꣳ स्वायै᳚ । \newline
42. तꣳ स्वायै॒ स्वायै॒ तम् तꣳ स्वायै॑ दे॒वता॑यै दे॒वता॑यै॒ स्वायै॒ तम् तꣳ स्वायै॑ दे॒वता॑यै । \newline
43. स्वायै॑ दे॒वता॑यै दे॒वता॑यै॒ स्वायै॒ स्वायै॑ दे॒वता॑या॒ आ दे॒वता॑यै॒ स्वायै॒ स्वायै॑ दे॒वता॑या॒ आ । \newline
44. दे॒वता॑या॒ आ दे॒वता॑यै दे॒वता॑या॒ आ ऽल॑भता लभ॒ता दे॒वता॑यै दे॒वता॑या॒ आ ऽल॑भत । \newline
45. आ ऽल॑भता लभ॒ता ऽल॑भत॒ तत॒ स्ततो॑ ऽलभ॒ता ऽल॑भत॒ ततः॑ । \newline
46. अ॒ल॒भ॒त॒ तत॒स्ततो॑ ऽलभता लभत॒ ततो॒ वै वै ततो॑ ऽलभता लभत॒ ततो॒ वै । \newline
47. ततो॒ वै वै तत॒ स्ततो॒ वै स स वै तत॒ स्ततो॒ वै सः । \newline
48. वै स स वै वै स प्र॒जाः प्र॒जाः स वै वै स प्र॒जाः । \newline
49. स प्र॒जाः प्र॒जाः स स प्र॒जाः प॒शून् प॒शून् प्र॒जाः स स प्र॒जाः प॒शून् । \newline
50. प्र॒जाः प॒शून् प॒शून् प्र॒जाः प्र॒जाः प॒शू न॑सृजता सृजत प॒शून् प्र॒जाः प्र॒जाः प॒शू न॑सृजत । \newline
51. प्र॒जा इति॑ प्र - जाः । \newline
52. प॒शू न॑सृजता सृजत प॒शून् प॒शू न॑सृजत॒ यो यो॑ ऽसृजत प॒शून् प॒शू न॑सृजत॒ यः । \newline
53. अ॒सृ॒ज॒त॒ यो यो॑ ऽसृजता सृजत॒ यः प्र॒जाका॑मः प्र॒जाका॑मो॒ यो॑ ऽसृजता सृजत॒ यः प्र॒जाका॑मः । \newline
54. यः प्र॒जाका॑मः प्र॒जाका॑मो॒ यो यः प्र॒जाका॑मः प॒शुका॑मः प॒शुका॑मः प्र॒जाका॑मो॒ यो यः प्र॒जाका॑मः प॒शुका॑मः । \newline
55. प्र॒जाका॑मः प॒शुका॑मः प॒शुका॑मः प्र॒जाका॑मः प्र॒जाका॑मः प॒शुका॑मः॒ स्याथ् स्यात् प॒शुका॑मः प्र॒जाका॑मः प्र॒जाका॑मः प॒शुका॑मः॒ स्यात् । \newline
56. प्र॒जाका॑म॒ इति॑ प्र॒जा - का॒मः॒ । \newline
\pagebreak
\markright{ TS 2.1.1.5  \hfill https://www.vedavms.in \hfill}

\section{ TS 2.1.1.5 }

\textbf{TS 2.1.1.5 } \newline
\textbf{Samhita Paata} \newline

प॒शुका॑मः॒ स्याथ् स ए॒तं प्रा॑जाप॒त्यम॒जं तू॑प॒रमा ल॑भेत प्र॒जाप॑तिमे॒व स्वेन॑ भाग॒धेये॒नोप॑ धावति॒ स ए॒वास्मै᳚ प्र॒जां प॒शून् प्रज॑नयति॒ यच्छ्‌म॑श्रु॒णस्तत् पुरु॑षाणाꣳ रू॒पं ॅयत् तू॑प॒रस्तदश्वा॑नां॒ ॅयद॒न्यतो॑द॒न् तद्-गवां॒ ॅयदव्या॑ इव श॒फास्तदवी॑नां॒ ॅयद॒जस्त-द॒जाना॑-मे॒ताव॑न्तो॒ वै ग्रा॒म्याः प॒शव॒स्तान् - [  ] \newline

\textbf{Pada Paata} \newline

प॒शुका॑म॒ इति॑ प॒शु - का॒मः॒ । स्यात् । सः । ए॒तम् । प्रा॒जा॒प॒त्यमिति॑ प्राजा - प॒त्यम् । अ॒जम् । तू॒प॒रम् । एति॑ । ल॒भे॒त॒ । प्र॒जाप॑ति॒मिति॑ प्र॒जा - प॒ति॒म् । ए॒व । स्वेन॑ । भा॒ग॒धेये॒नेति॑ भाग - धेये॑न । उपेति॑ । धा॒व॒ति॒ । सः । ए॒व । अ॒स्मै॒ । प्र॒जामिति॑ प्र-जाम् । प॒शून् । प्रेति॑ । ज॒न॒य॒ति॒ । यत् । श्म॒श्रु॒णः । तत् । पुरु॑षाणाम् । रू॒पम् । यत् । तू॒प॒रः । तत् । अश्वा॑नाम् । यत् । अ॒न्यतो॑द॒न्नित्य॒न्यतः॑ - द॒न्न् । तत् । गवा᳚म् । यत् । अव्याः᳚ । इ॒व॒ । श॒फाः । तत् । अवी॑नाम् । यत् । अ॒जः । तत् । अ॒जाना᳚म् । ए॒ताव॑न्तः । वै । ग्रा॒म्याः । प॒शवः॑ । तान् ।  \newline


\textbf{Krama Paata} \newline

प॒शुका॑मः॒ स्यात् । प॒शुका॑म॒ इति॑ प॒शु - का॒मः॒ । स्याथ् सः । स ए॒तम् । ए॒तम् प्रा॑जाप॒त्यम् । प्रा॒जा॒प॒त्यम॒जम् । प्रा॒जा॒प॒त्यमिति॑ प्राजा - प॒त्यम् । अ॒जम् तू॑प॒रम् । तू॒प॒रमा । आ ल॑भेत । ल॒भे॒त॒ प्र॒जाप॑तिम् । प्र॒जाप॑तिमे॒व । प्र॒जाप॑ति॒मिति॑ प्र॒जा - प॒ति॒म् । ए॒व स्वेन॑ । स्वेन॑ भाग॒धेये॑न । भा॒ग॒धेये॒नोप॑ । भा॒ग॒धेये॒नेति॑ भाग - धेये॑न । उप॑ धावति । धा॒व॒ति॒ सः । स ए॒व । ए॒वास्मै᳚ । अ॒स्मै॒ प्र॒जाम् । प्र॒जाम् प॒शून् । प्र॒जामिति॑ प्र - जाम् । प॒शून् प्र । प्र ज॑नयति । ज॒न॒य॒ति॒ यत् । यच्छ्म॑श्रु॒णः । श्म॒श्रु॒णस्तत् । तत् पुरु॑षाणाम् । पुरु॑षाणाꣳ रू॒पम् । रू॒पं ॅयत् । यत् तू॑प॒रः । तू॒प॒रस्तत् । तदश्वा॑नाम् । अश्वा॑नां॒ ॅयत् । यद॒न्यतो॑दन्न् । अ॒न्यतो॑द॒न् तत् । अ॒न्यतो॑द॒न्नित्य॒न्यतः॑ - द॒न्न्॒ । तद् गवा᳚म् । गवां॒ ॅयत् । यदव्याः᳚ । अव्या॑ इव । इ॒व॒ श॒फाः । श॒फास्तत् । तदवी॑नाम् । अवी॑नां॒ ॅयत् । यद॒जः । अ॒जस्तत् । तद॒जाना᳚म् । अ॒जाना॑मे॒ताव॑न्तः । ए॒ताव॑न्तो॒ वै । वै ग्रा॒म्याः । ग्रा॒म्याः प॒शवः॑ । प॒शव॒स्तान् । तान् रू॒पेण॑ \newline

\textbf{Jatai Paata} \newline

1. प॒शुका॑मः॒ स्याथ् स्यात् प॒शुका॑मः प॒शुका॑मः॒ स्यात् । \newline
2. प॒शुका॑म॒ इति॑ प॒शु - का॒मः॒ । \newline
3. स्याथ् स स स्याथ् स्याथ् सः । \newline
4. स ए॒त मे॒तꣳ स स ए॒तम् । \newline
5. ए॒तम् प्रा॑जाप॒त्यम् प्रा॑जाप॒त्य मे॒त मे॒तम् प्रा॑जाप॒त्यम् । \newline
6. प्रा॒जा॒प॒त्य म॒ज म॒जम् प्रा॑जाप॒त्यम् प्रा॑जाप॒त्य म॒जम् । \newline
7. प्रा॒जा॒प॒त्यमिति॑ प्राजा - प॒त्यम् । \newline
8. अ॒जम् तू॑प॒रम् तू॑प॒र म॒ज म॒जम् तू॑प॒रम् । \newline
9. तू॒प॒र मा तू॑प॒रम् तू॑प॒र मा । \newline
10. आ ल॑भेत लभे॒ता ल॑भेत । \newline
11. ल॒भे॒त॒ प्र॒जाप॑तिम् प्र॒जाप॑तिम् ॅलभेत लभेत प्र॒जाप॑तिम् । \newline
12. प्र॒जाप॑ति मे॒वैव प्र॒जाप॑तिम् प्र॒जाप॑ति मे॒व । \newline
13. प्र॒जाप॑ति॒मिति॑ प्र॒जा - प॒ति॒म् । \newline
14. ए॒व स्वेन॒ स्वेनै॒वैव स्वेन॑ । \newline
15. स्वेन॑ भाग॒धेये॑न भाग॒धेये॑न॒ स्वेन॒ स्वेन॑ भाग॒धेये॑न । \newline
16. भा॒ग॒धेये॒नोपोप॑ भाग॒धेये॑न भाग॒धेये॒नोप॑ । \newline
17. भा॒ग॒धेये॒नेति॑ भाग - धेये॑न । \newline
18. उप॑ धावति धाव॒ त्युपोप॑ धावति । \newline
19. धा॒व॒ति॒ स स धा॑वति धावति॒ सः । \newline
20. स ए॒वैव स स ए॒व । \newline
21. ए॒वास्मा॑ अस्मा ए॒वैवास्मै᳚ । \newline
22. अ॒स्मै॒ प्र॒जाम् प्र॒जा म॑स्मा अस्मै प्र॒जाम् । \newline
23. प्र॒जाम् प॒शून् प॒शून् प्र॒जाम् प्र॒जाम् प॒शून् । \newline
24. प्र॒जामिति॑ प्र - जाम् । \newline
25. प॒शून् प्र प्र प॒शून् प॒शून् प्र । \newline
26. प्र ज॑नयति जनयति॒ प्र प्र ज॑नयति । \newline
27. ज॒न॒य॒ति॒ यद् यज् ज॑नयति जनयति॒ यत् । \newline
28. यच् छ्‌म॑श्रु॒णः श्म॑श्रु॒णो यद् यच् छ्‌म॑श्रु॒णः । \newline
29. श्म॒श्रु॒ण स्तत् तच् छ्‌म॑श्रु॒णः श्म॑श्रु॒ण स्तत् । \newline
30. तत् पुरु॑षाणा॒म् पुरु॑षाणा॒म् तत् तत् पुरु॑षाणाम् । \newline
31. पुरु॑षाणाꣳ रू॒पꣳ रू॒पम् पुरु॑षाणा॒म् पुरु॑षाणाꣳ रू॒पम् । \newline
32. रू॒पं ॅयद् यद् रू॒पꣳ रू॒पं ॅयत् । \newline
33. यत् तू॑प॒र स्तू॑प॒रो यद् यत् तू॑प॒रः । \newline
34. तू॒प॒र स्तत् तत् तू॑प॒र स्तू॑प॒र स्तत् । \newline
35. तदश्वा॑ना॒ मश्वा॑ना॒म् तत् तदश्वा॑नाम् । \newline
36. अश्वा॑नां॒ ॅयद् यदश्वा॑ना॒ मश्वा॑नां॒ ॅयत् । \newline
37. यद॒न्यतो॑दन् न॒न्यतो॑द॒न्॒. यद् यद॒न्यतो॑दन्न् । \newline
38. अ॒न्यतो॑द॒न् तत् तद॒न्यतो॑दन् न॒न्यतो॑द॒न् तत् । \newline
39. अ॒न्यतो॑द॒न्नित्य॒न्यतः॑ - द॒न्न् । \newline
40. तद् गवा॒म् गवा॒म् तत् तद् गवा᳚म् । \newline
41. गवां॒ ॅयद् यद् गवा॒म् गवां॒ ॅयत् । \newline
42. यदव्या॒ अव्या॒ यद् यदव्याः᳚ । \newline
43. अव्या॑ इवे॒ वाव्या॒ अव्या॑ इव । \newline
44. इ॒व॒ श॒फाः श॒फा इ॑वे व श॒फाः । \newline
45. श॒फा स्तत् तच्छ॒फाः श॒फा स्तत् । \newline
46. तदवी॑ना॒ मवी॑ना॒म् तत् तदवी॑नाम् । \newline
47. अवी॑नां॒ ॅयद् यदवी॑ना॒ मवी॑नां॒ ॅयत् । \newline
48. यद॒जो॑ ऽजो यद् यद॒जः । \newline
49. अ॒ज स्तत् तद॒जो॑ ऽज स्तत् । \newline
50. तद॒जाना॑ म॒जाना॒म् तत् तद॒जाना᳚म् । \newline
51. अ॒जाना॑ मे॒ताव॑न्त ए॒ताव॑न्तो॒ ऽजाना॑ म॒जाना॑ मे॒ताव॑न्तः । \newline
52. ए॒ताव॑न्तो॒ वै वा ए॒ताव॑न्त ए॒ताव॑न्तो॒ वै । \newline
53. वै ग्रा॒म्या ग्रा॒म्या वै वै ग्रा॒म्याः । \newline
54. ग्रा॒म्याः प॒शवः॑ प॒शवो᳚ ग्रा॒म्या ग्रा॒म्याः प॒शवः॑ । \newline
55. प॒शव॒ स्ताꣳ स्तान् प॒शवः॑ प॒शव॒ स्तान् । \newline
56. तान् रू॒पेण॑ रू॒पेण॒ ताꣳस् तान् रू॒पेण॑ । \newline

\textbf{Ghana Paata } \newline

1. प॒शुका॑मः॒ स्याथ् स्यात् प॒शुका॑मः प॒शुका॑मः॒ स्याथ् स स स्यात् प॒शुका॑मः प॒शुका॑मः॒ स्याथ् सः । \newline
2. प॒शुका॑म॒ इति॑ प॒शु - का॒मः॒ । \newline
3. स्याथ् स स स्याथ् स्याथ् स ए॒त मे॒तꣳ स स्याथ् स्याथ् स ए॒तम् । \newline
4. स ए॒त मे॒तꣳ स स ए॒तम् प्रा॑जाप॒त्यम् प्रा॑जाप॒त्य मे॒तꣳ स स ए॒तम् प्रा॑जाप॒त्यम् । \newline
5. ए॒तम् प्रा॑जाप॒त्यम् प्रा॑जाप॒त्य मे॒त मे॒तम् प्रा॑जाप॒त्य म॒ज म॒जम् प्रा॑जाप॒त्य मे॒त मे॒तम् प्रा॑जाप॒त्य म॒जम् । \newline
6. प्रा॒जा॒प॒त्य म॒ज म॒जम् प्रा॑जाप॒त्यम् प्रा॑जाप॒त्य म॒जम् तू॑प॒रम् तू॑प॒र म॒जम् प्रा॑जाप॒त्यम् प्रा॑जाप॒त्य म॒जम् तू॑प॒रम् । \newline
7. प्रा॒जा॒प॒त्यमिति॑ प्राजा - प॒त्यम् । \newline
8. अ॒जम् तू॑प॒रम् तू॑प॒र म॒ज म॒जम् तू॑प॒र मा तू॑प॒र म॒ज म॒जम् तू॑प॒र मा । \newline
9. तू॒प॒र मा तू॑प॒रम् तू॑प॒र मा ल॑भेत लभे॒ता तू॑प॒रम् तू॑प॒र मा ल॑भेत । \newline
10. आ ल॑भेत लभे॒ता ल॑भेत प्र॒जाप॑तिम् प्र॒जाप॑तिम् ॅलभे॒ता ल॑भेत प्र॒जाप॑तिम् । \newline
11. ल॒भे॒त॒ प्र॒जाप॑तिम् प्र॒जाप॑तिम् ॅलभेत लभेत प्र॒जाप॑ति मे॒वैव प्र॒जाप॑तिम् ॅलभेत लभेत प्र॒जाप॑ति मे॒व । \newline
12. प्र॒जाप॑ति मे॒वैव प्र॒जाप॑तिम् प्र॒जाप॑ति मे॒व स्वेन॒ स्वेनै॒व प्र॒जाप॑तिम् प्र॒जाप॑ति मे॒व स्वेन॑ । \newline
13. प्र॒जाप॑ति॒मिति॑ प्र॒जा - प॒ति॒म् । \newline
14. ए॒व स्वेन॒ स्वेनै॒वैव स्वेन॑ भाग॒धेये॑न भाग॒धेये॑न॒ स्वेनै॒वैव स्वेन॑ भाग॒धेये॑न । \newline
15. स्वेन॑ भाग॒धेये॑न भाग॒धेये॑न॒ स्वेन॒ स्वेन॑ भाग॒धेये॒नो पोप॑ भाग॒धेये॑न॒ स्वेन॒ स्वेन॑ भाग॒धेये॒नोप॑ । \newline
16. भा॒ग॒धेये॒नो पोप॑ भाग॒धेये॑न भाग॒धेये॒नोप॑ धावति धाव॒त्युप॑ भाग॒धेये॑न भाग॒धेये॒नोप॑ धावति । \newline
17. भा॒ग॒धेये॒नेति॑ भाग - धेये॑न । \newline
18. उप॑ धावति धाव॒ त्युपोप॑ धावति॒ स स धा॑व॒ त्युपोप॑ धावति॒ सः । \newline
19. धा॒व॒ति॒ स स धा॑वति धावति॒ स ए॒वैव स धा॑वति धावति॒ स ए॒व । \newline
20. स ए॒वैव स स ए॒वास्मा॑ अस्मा ए॒व स स ए॒वास्मै᳚ । \newline
21. ए॒वास्मा॑ अस्मा ए॒वैवास्मै᳚ प्र॒जाम् प्र॒जा म॑स्मा ए॒वैवास्मै᳚ प्र॒जाम् । \newline
22. अ॒स्मै॒ प्र॒जाम् प्र॒जा म॑स्मा अस्मै प्र॒जाम् प॒शून् प॒शून् प्र॒जा म॑स्मा अस्मै प्र॒जाम् प॒शून् । \newline
23. प्र॒जाम् प॒शून् प॒शून् प्र॒जाम् प्र॒जाम् प॒शून् प्र प्र प॒शून् प्र॒जाम् प्र॒जाम् प॒शून् प्र । \newline
24. प्र॒जामिति॑ प्र - जाम् । \newline
25. प॒शून् प्र प्र प॒शून् प॒शून् प्र ज॑नयति जनयति॒ प्र प॒शून् प॒शून् प्र ज॑नयति । \newline
26. प्र ज॑नयति जनयति॒ प्र प्र ज॑नयति॒ यद् यज् ज॑नयति॒ प्र प्र ज॑नयति॒ यत् । \newline
27. ज॒न॒य॒ति॒ यद् यज् ज॑नयति जनयति॒ यच् छ्‌म॑श्रु॒णः श्म॑श्रु॒णो यज् ज॑नयति जनयति॒ यच् छ्‌म॑श्रु॒णः । \newline
28. यच् छ्‌म॑श्रु॒णः श्म॑श्रु॒णो यद् यच् छ्‌म॑श्रु॒ण स्तत् तच् छ्‌म॑श्रु॒णो यद् यच् छ्‌म॑श्रु॒ण स्तत् । \newline
29. श्म॒श्रु॒ण स्तत् तच् छ्‌म॑श्रु॒णः श्म॑श्रु॒ण स्तत् पुरु॑षाणा॒म् पुरु॑षाणा॒म् तच् छ्‌म॑श्रु॒णः श्म॑श्रु॒ण स्तत् पुरु॑षाणाम् । \newline
30. तत् पुरु॑षाणा॒म् पुरु॑षाणा॒म् तत् तत् पुरु॑षाणाꣳ रू॒पꣳ रू॒पम् पुरु॑षाणा॒म् तत् तत् पुरु॑षाणाꣳ रू॒पम् । \newline
31. पुरु॑षाणाꣳ रू॒पꣳ रू॒पम् पुरु॑षाणा॒म् पुरु॑षाणाꣳ रू॒पं ॅयद् यद् रू॒पम् पुरु॑षाणा॒म् पुरु॑षाणाꣳ रू॒पं ॅयत् । \newline
32. रू॒पं ॅयद् यद् रू॒पꣳ रू॒पं ॅयत् तू॑प॒र स्तू॑प॒रो यद् रू॒पꣳ रू॒पं ॅयत् तू॑प॒रः । \newline
33. यत् तू॑प॒र स्तू॑प॒रो यद् यत् तू॑प॒र स्तत् तत् तू॑प॒रो यद् यत् तू॑प॒र स्तत् । \newline
34. तू॒प॒र स्तत् तत् तू॑प॒र स्तू॑प॒र स्तदश्वा॑ना॒ मश्वा॑ना॒म् तत् तू॑प॒र स्तू॑प॒र स्तदश्वा॑नाम् । \newline
35. तदश्वा॑ना॒ मश्वा॑ना॒म् तत् तदश्वा॑नां॒ ॅयद् यदश्वा॑ना॒म् तत् तदश्वा॑नां॒ ॅयत् । \newline
36. अश्वा॑नां॒ ॅयद् यदश्वा॑ना॒ मश्वा॑नां॒ ॅयद॒न्यतो॑दन् न॒न्यतो॑द॒न्॒. यदश्वा॑ना॒ मश्वा॑नां॒ ॅयद॒न्यतो॑दन्न् । \newline
37. यद॒न्यतो॑दन् न॒न्यतो॑द॒न्॒. यद् यद॒न्यतो॑द॒न् तत् तद॒न्यतो॑द॒न्॒. यद् यद॒न्यतो॑द॒न् तत् । \newline
38. अ॒न्यतो॑द॒न् तत् तद॒न्यतो॑दन् न॒न्यतो॑द॒न् तद् गवा॒म् गवा॒म् तद॒न्यतो॑दन् न॒न्यतो॑द॒न् तद् गवा᳚म् । \newline
39. अ॒न्यतो॑द॒न्नित्य॒न्यतः॑ - द॒न्न् । \newline
40. तद् गवा॒म् गवा॒म् तत् तद् गवां॒ ॅयद् यद् गवा॒म् तत् तद् गवां॒ ॅयत् । \newline
41. गवां॒ ॅयद् यद् गवा॒म् गवां॒ ॅयदव्या॒ अव्या॒ यद् गवा॒म् गवां॒ ॅयदव्याः᳚ । \newline
42. यदव्या॒ अव्या॒ यद् यदव्या॑ इवे॒ वाव्या॒ यद् यदव्या॑ इव । \newline
43. अव्या॑ इवे॒ वाव्या॒ अव्या॑ इव श॒फाः श॒फा इ॒वाव्या॒ अव्या॑ इव श॒फाः । \newline
44. इ॒व॒ श॒फाः श॒फा इ॑वे व श॒फा स्तत् तच् छ॒फा इ॑वे व श॒फा स्तत् । \newline
45. श॒फा स्तत् तच् छ॒फाः श॒फा स्तदवी॑ना॒ मवी॑ना॒म् तच् छ॒फाः श॒फा स्तदवी॑नाम् । \newline
46. तदवी॑ना॒ मवी॑ना॒म् तत् तदवी॑नां॒ ॅयद् यदवी॑ना॒म् तत् तदवी॑नां॒ ॅयत् । \newline
47. अवी॑नां॒ ॅयद् यदवी॑ना॒ मवी॑नां॒ ॅयद॒जो॑ ऽजो यदवी॑ना॒ मवी॑नां॒ ॅयद॒जः । \newline
48. यद॒जो॑ ऽजो यद् यद॒ज स्तत् तद॒जो यद् यद॒ज स्तत् । \newline
49. अ॒ज स्तत् तद॒जो॑ ऽज स्तद॒जाना॑ म॒जाना॒म् तद॒जो॑ ऽज स्तद॒जाना᳚म् । \newline
50. तद॒जाना॑ म॒जाना॒म् तत् तद॒जाना॑ मे॒ताव॑न्त ए॒ताव॑न्तो॒ ऽजाना॒म् तत् तद॒जाना॑ मे॒ताव॑न्तः । \newline
51. अ॒जाना॑ मे॒ताव॑न्त ए॒ताव॑न्तो॒ ऽजाना॑ म॒जाना॑ मे॒ताव॑न्तो॒ वै वा ए॒ताव॑न्तो॒ ऽजाना॑ म॒जाना॑ मे॒ताव॑न्तो॒ वै । \newline
52. ए॒ताव॑न्तो॒ वै वा ए॒ताव॑न्त ए॒ताव॑न्तो॒ वै ग्रा॒म्या ग्रा॒म्या वा ए॒ताव॑न्त ए॒ताव॑न्तो॒ वै ग्रा॒म्याः । \newline
53. वै ग्रा॒म्या ग्रा॒म्या वै वै ग्रा॒म्याः प॒शवः॑ प॒शवो᳚ ग्रा॒म्या वै वै ग्रा॒म्याः प॒शवः॑ । \newline
54. ग्रा॒म्याः प॒शवः॑ प॒शवो᳚ ग्रा॒म्या ग्रा॒म्याः प॒शव॒ स्ताꣳ स्तान् प॒शवो᳚ ग्रा॒म्या ग्रा॒म्याः प॒शव॒ स्तान् । \newline
55. प॒शव॒ स्ताꣳ स्तान् प॒शवः॑ प॒शव॒ स्तान् रू॒पेण॑ रू॒पेण॒ तान् प॒शवः॑ प॒शव॒ स्तान् रू॒पेण॑ । \newline
56. तान् रू॒पेण॑ रू॒पेण॒ ताꣳ स्तान् रू॒पे णै॒वैव रू॒पेण॒ ताꣳ स्तान् रू॒पेणै॒व । \newline
\pagebreak
\markright{ TS 2.1.1.6  \hfill https://www.vedavms.in \hfill}

\section{ TS 2.1.1.6 }

\textbf{TS 2.1.1.6 } \newline
\textbf{Samhita Paata} \newline

रू॒पेणै॒वाव॑ रुन्धे सोमापौ॒ष्णं त्रै॒तमा ल॑भेत प॒शुका॑मो॒द्वौ वा अ॒जायै॒ स्तनौ॒ नानै॒व द्वाव॒भि जाये॑ते॒ ऊर्जं॒ पुष्टिं॑ तृ॒तीयः॑सोमापू॒षणा॑वे॒व स्वेन॑ भाग॒धेये॒नोप॑ धावति॒ तावे॒वास्मै॑ प॒शून् प्रज॑नयतः॒ सोमो॒ वै रे॑तो॒धाः पू॒षा प॑शू॒नां प्र॑जनयि॒ता सोम॑ ए॒वास्मै॒ रेतो॒ दधा॑ति पू॒षा प॒शून् प्र ज॑नय॒त्यौदु॑म्बरो॒ यूपो॑ ( ) भव॒त्यूर्ग्वा उ॑दु॒म्बर॒ ऊर्क् प॒शव॑ ऊ॒र्जैवास्मा॒ ऊर्जं॑ प॒शूनव॑ रुन्धे ॥ \newline

\textbf{Pada Paata} \newline

रू॒पेण॑ । ए॒व । अवेति॑ । रु॒न्धे॒ । सो॒मा॒पौ॒ष्णमिति॑ सोमा - पौ॒ष्णम् । त्रै॒तम् । एति॑ । ल॒भे॒त॒ । प॒शुका॑म॒ इति॑ प॒शु - का॒मः॒ । द्वौ । वै । अ॒जायै᳚ । स्तनौ᳚ । नाना᳚ । ए॒व । द्वौ । अ॒भीति॑ । जाये॑ते॒ इति॑ । ऊर्ज᳚म् । पुष्टि᳚म् । तृ॒तीयः॑ । सो॒मा॒पू॒षणा॒विति॑ सोमा - पू॒षणौ᳚ । ए॒व । स्वेन॑ । भा॒ग॒धेये॒नेति॑ भाग - धेये॑न । उपेति॑ । धा॒व॒ति॒ । तौ । ए॒व । अ॒स्मै॒ । प॒शून् । प्रेति॑ । ज॒न॒य॒तः॒ । सोमः॑ । वै । रे॒तो॒धा इति॑ रेतः - धाः । पू॒षा । प॒शू॒नाम् । प्र॒ज॒न॒यि॒तेति॑ प्र-ज॒न॒यि॒ता । सोमः॑ । ए॒व । अ॒स्मै॒ । रेतः॑ । दधा॑ति । पू॒षा । प॒शून् । प्रेति॑ । ज॒न॒य॒ति॒ । औदु॑म्बरः । यूपः॑ ( ) । भ॒व॒ति॒ । ऊर्क् । वै । उ॒दु॒म्बरः॑ । ऊर्क् । प॒शवः॑ । ऊ॒र्जा । ए॒व । अ॒स्मै॒ । ऊर्ज᳚म् । प॒शून् । अवेति॑ । रु॒न्धे॒ ॥  \newline


\textbf{Krama Paata} \newline

रू॒पेणै॒व । ए॒वाव॑ । अव॑ रुन्धे । रु॒न्धे॒ सो॒मा॒पौ॒ष्णम् । सो॒मा॒पौ॒ष्णम् त्रै॒तम् । सो॒मा॒पौ॒ष्णमिति॑ सोमा - पौ॒ष्णम् । त्रै॒तमा । आ ल॑भेत । ल॒भे॒त॒ प॒शुका॑मः । प॒शुका॑मो॒ द्वौ । प॒शुका॑म॒ इति॑ प॒शु - का॒मः॒ । द्वौ वै । वा अ॒जायै᳚ । अ॒जायै॒ स्तनौ᳚ । स्तनौ॒ नाना᳚ । नानै॒व । ए॒व द्वौ । द्वाव॒भि । अ॒भि जाये॑ते । जाये॑ते॒ ऊर्ज᳚म् । जाये॑ते॒ इति॒ जाये॑ते । ऊर्ज॒म् पुष्टि᳚म् । पुष्टि॑म् तृ॒तीयः॑ । तृ॒तीयः॑ सोमापू॒षणौ᳚ । सो॒मा॒पू॒षणा॑वे॒व । सो॒मा॒पू॒षणा॒विति॑ सोमा - पू॒षणौ᳚ । ए॒व स्वेन॑ । स्वेन॑ भाग॒धेये॑न । भा॒ग॒धेये॒नोप॑ । भा॒ग॒धेये॒नेति॑ भाग - धेये॑न । उप॑ धावति । धा॒व॒ति॒ तौ । तावे॒व । ए॒वास्मै᳚ । अ॒स्मै॒ प॒शून् । प॒शून् प्र । प्र ज॑नयतः । ज॒न॒य॒तः॒ सोमः॑ । सोमो॒ वै । वै रे॑तो॒धाः । रे॒तो॒धाः पू॒षा । रे॒तो॒धा इति॑ रेतः - धाः । पू॒षा प॑शू॒नाम् । प॒शू॒नाम् प्र॑जनयि॒ता । प्र॒ज॒न॒यि॒ता सोमः॑ । प्र॒ज॒न॒यि॒तेति॑ प्र - ज॒न॒यि॒ता । सोम॑ ए॒व । ए॒वास्मै᳚ । अ॒स्मै॒ रेतः॑ । रेतो॒ दधा॑ति । दधा॑ति पू॒षा । पू॒षा प॒शून् । प॒शून् प्र । प्र ज॑नयति । ज॒न॒य॒त्यौदु॑म्बरः । औदु॑म्बरो॒ यूपः॑ ( ) । यूपो॑ भवति । भ॒व॒त्यूर्क् । ऊर्ग्वै । वा उ॑दुं॒बरः॑ । उ॒दु॒म्बर॒ ऊर्क् । ऊर्क् प॒शवः॑ । प॒शव॑ ऊ॒र्जा । ऊ॒र्जैव । ए॒वास्मै᳚ । अ॒स्मा॒ ऊर्ज᳚म् । ऊर्ज॑म् प॒शून् । प॒शूनव॑ । अव॑ रुन्धे । रु॒न्ध॒ इति॑ रुन्धे । \newline

\textbf{Jatai Paata} \newline

1. रू॒पेणै॒वैव रू॒पेण॑ रू॒पेणै॒व । \newline
2. ए॒वावा वै॒वै वाव॑ । \newline
3. अव॑ रुन्धे रु॒न्धे ऽवाव॑ रुन्धे । \newline
4. रु॒न्धे॒ सो॒मा॒पौ॒ष्णꣳ सो॑मापौ॒ष्णꣳ रु॑न्धे रुन्धे सोमापौ॒ष्णम् । \newline
5. सो॒मा॒पौ॒ष्णम् त्रै॒तम् त्रै॒तꣳ सो॑मापौ॒ष्णꣳ सो॑मापौ॒ष्णम् त्रै॒तम् । \newline
6. सो॒मा॒पौ॒ष्णमिति॑ सोमा - पौ॒ष्णम् । \newline
7. त्रै॒त मा त्रै॒तम् त्रै॒त मा । \newline
8. आ ल॑भेत लभे॒ता ल॑भेत । \newline
9. ल॒भे॒त॒ प॒शुका॑मः प॒शुका॑मो लभेत लभेत प॒शुका॑मः । \newline
10. प॒शुका॑मो॒ द्वौ द्वौ प॒शुका॑मः प॒शुका॑मो॒ द्वौ । \newline
11. प॒शुका॑म॒ इति॑ प॒शु - का॒मः॒ । \newline
12. द्वौ वै वै द्वौ द्वौ वै । \newline
13. वा अ॒जाया॑ अ॒जायै॒ वै वा अ॒जायै᳚ । \newline
14. अ॒जायै॒ स्तनौ॒ स्तना॑ व॒जाया॑ अ॒जायै॒ स्तनौ᳚ । \newline
15. स्तनौ॒ नाना॒ नाना॒ स्तनौ॒ स्तनौ॒ नाना᳚ । \newline
16. नानै॒वैव नाना॒ नानै॒व । \newline
17. ए॒व द्वौ द्वा वे॒वैव द्वौ । \newline
18. द्वा व॒भ्य॑भि द्वौ द्वा व॒भि । \newline
19. अ॒भि जाये॑ते॒ जाये॑ते अ॒भ्य॑भि जाये॑ते । \newline
20. जाये॑ते॒ ऊर्ज॒ मूर्ज॒म् जाये॑ते॒ जाये॑ते॒ ऊर्ज᳚म् । \newline
21. जाये॑ते॒ इति॒ जाये॑ते । \newline
22. ऊर्ज॒म् पुष्टि॒म् पुष्टि॒ मूर्ज॒ मूर्ज॒म् पुष्टि᳚म् । \newline
23. पुष्टि॑म् तृ॒तीय॑ स्तृ॒तीयः॒ पुष्टि॒म् पुष्टि॑म् तृ॒तीयः॑ । \newline
24. तृ॒तीयः॑ सोमापू॒षणौ॑ सोमापू॒षणौ॑ तृ॒तीय॑ स्तृ॒तीयः॑ सोमापू॒षणौ᳚ । \newline
25. सो॒मा॒पू॒षणा॑ वे॒वैव सो॑मापू॒षणौ॑ सोमापू॒षणा॑ वे॒व । \newline
26. सो॒मा॒पू॒षणा॒विति॑ सोमा - पू॒षणौ᳚ । \newline
27. ए॒व स्वेन॒ स्वेनै॒ वैव स्वेन॑ । \newline
28. स्वेन॑ भाग॒धेये॑न भाग॒धेये॑न॒ स्वेन॒ स्वेन॑ भाग॒धेये॑न । \newline
29. भा॒ग॒धेये॒नोपोप॑ भाग॒धेये॑न भाग॒धेये॒नोप॑ । \newline
30. भा॒ग॒धेये॒नेति॑ भाग - धेये॑न । \newline
31. उप॑ धावति धाव॒ त्युपोप॑ धावति । \newline
32. धा॒व॒ति॒ तौ तौ धा॑वति धावति॒ तौ । \newline
33. ता वे॒वैव तौ ता वे॒व । \newline
34. ए॒वास्मा॑ अस्मा ए॒वैवास्मै᳚ । \newline
35. अ॒स्मै॒ प॒शून् प॒शू न॑स्मा अस्मै प॒शून् । \newline
36. प॒शून् प्र प्र प॒शून् प॒शून् प्र । \newline
37. प्र ज॑नयतो जनयतः॒ प्र प्र ज॑नयतः । \newline
38. ज॒न॒य॒तः॒ सोमः॒ सोमो॑ जनयतो जनयतः॒ सोमः॑ । \newline
39. सोमो॒ वै वै सोमः॒ सोमो॒ वै । \newline
40. वै रे॑तो॒धा रे॑तो॒धा वै वै रे॑तो॒धाः । \newline
41. रे॒तो॒धाः पू॒षा पू॒षा रे॑तो॒धा रे॑तो॒धाः पू॒षा । \newline
42. रे॒तो॒धा इति॑ रेतः - धाः । \newline
43. पू॒षा प॑शू॒नाम् प॑शू॒नाम् पू॒षा पू॒षा प॑शू॒नाम् । \newline
44. प॒शू॒नाम् प्र॑जनयि॒ता प्र॑जनयि॒ता प॑शू॒नाम् प॑शू॒नाम् प्र॑जनयि॒ता । \newline
45. प्र॒ज॒न॒यि॒ता सोमः॒ सोमः॑ प्रजनयि॒ता प्र॑जनयि॒ता सोमः॑ । \newline
46. प्र॒ज॒न॒यि॒तेति॑ प्र - ज॒न॒यि॒ता । \newline
47. सोम॑ ए॒वैव सोमः॒ सोम॑ ए॒व । \newline
48. ए॒वास्मा॑ अस्मा ए॒वैवास्मै᳚ । \newline
49. अ॒स्मै॒ रेतो॒ रेतो᳚ ऽस्मा अस्मै॒ रेतः॑ । \newline
50. रेतो॒ दधा॑ति॒ दधा॑ति॒ रेतो॒ रेतो॒ दधा॑ति । \newline
51. दधा॑ति पू॒षा पू॒षा दधा॑ति॒ दधा॑ति पू॒षा । \newline
52. पू॒षा प॒शून् प॒शून् पू॒षा पू॒षा प॒शून् । \newline
53. प॒शून् प्र प्र प॒शून् प॒शून् प्र । \newline
54. प्र ज॑नयति जनयति॒ प्र प्र ज॑नयति । \newline
55. ज॒न॒य॒ त्यौदु॑म्बर॒ औदु॑म्बरो जनयति जनय॒ त्यौदु॑म्बरः । \newline
56. औदु॑म्बरो॒ यूपो॒ यूप॒ औदु॑म्बर॒ औदु॑म्बरो॒ यूपः॑ । \newline
57. यूपो॑ भवति भवति॒ यूपो॒ यूपो॑ भवति । \newline
58. भ॒व॒ त्यूर्गूर्ग् भ॑वति भव॒ त्यूर्क् । \newline
59. ऊर्ग् वै वा ऊर्गूर्ग् वै । \newline
60. वा उ॑दु॒म्बर॑ उदु॒म्बरो॒ वै वा उ॑दु॒म्बरः॑ । \newline
61. उ॒दु॒म्बर॒ ऊर्गूर्गु॑दु॒म्बर॑ उदु॒म्बर॒ ऊर्क् । \newline
62. ऊर्क् प॒शवः॑ प॒शव॒ ऊर्गूर्क् प॒शवः॑ । \newline
63. प॒शव॑ ऊ॒र्जोर्जा प॒शवः॑ प॒शव॑ ऊ॒र्जा । \newline
64. ऊ॒र्जै वै वोर्जोर् जैव । \newline
65. ए॒वास्मा॑ अस्मा ए॒वैवास्मै᳚ । \newline
66. अ॒स्मा॒ ऊर्ज॒ मूर्ज॑ मस्मा अस्मा॒ ऊर्ज᳚म् । \newline
67. ऊर्ज॑म् प॒शून् प॒शू नूर्ज॒ मूर्ज॑म् प॒शून् । \newline
68. प॒शू नवाव॑ प॒शून् प॒शू नव॑ । \newline
69. अव॑ रुन्धे रु॒न्धे ऽवाव॑ रुन्धे । \newline
70. रु॒न्ध॒ इति॑ रुन्धे । \newline

\textbf{Ghana Paata } \newline

1. रू॒पे णै॒वैव रू॒पेण॑ रू॒पे णै॒वावा वै॒व रू॒पेण॑ रू॒पे णै॒वाव॑ । \newline
2. ए॒वावा वै॒वैवाव॑ रुन्धे रु॒न्धे ऽवै॒वैवाव॑ रुन्धे । \newline
3. अव॑ रुन्धे रु॒न्धे ऽवाव॑ रुन्धे सोमापौ॒ष्णꣳ सो॑मापौ॒ष्णꣳ रु॒न्धे ऽवाव॑ रुन्धे सोमापौ॒ष्णम् । \newline
4. रु॒न्धे॒ सो॒मा॒पौ॒ष्णꣳ सो॑मापौ॒ष्णꣳ रु॑न्धे रुन्धे सोमापौ॒ष्णम् त्रै॒तम् त्रै॒तꣳ सो॑मापौ॒ष्णꣳ रु॑न्धे रुन्धे सोमापौ॒ष्णम् त्रै॒तम् । \newline
5. सो॒मा॒पौ॒ष्णम् त्रै॒तम् त्रै॒तꣳ सो॑मापौ॒ष्णꣳ सो॑मापौ॒ष्णम् त्रै॒त मा त्रै॒तꣳ सो॑मापौ॒ष्णꣳ सो॑मापौ॒ष्णम् त्रै॒त मा । \newline
6. सो॒मा॒पौ॒ष्णमिति॑ सोमा - पौ॒ष्णम् । \newline
7. त्रै॒त मा त्रै॒तम् त्रै॒त मा ल॑भेत लभे॒ता त्रै॒तम् त्रै॒त मा ल॑भेत । \newline
8. आ ल॑भेत लभे॒ता ल॑भेत प॒शुका॑मः प॒शुका॑मो लभे॒ता ल॑भेत प॒शुका॑मः । \newline
9. ल॒भे॒त॒ प॒शुका॑मः प॒शुका॑मो लभेत लभेत प॒शुका॑मो॒ द्वौ द्वौ प॒शुका॑मो लभेत लभेत प॒शुका॑मो॒ द्वौ । \newline
10. प॒शुका॑मो॒ द्वौ द्वौ प॒शुका॑मः प॒शुका॑मो॒ द्वौ वै वै द्वौ प॒शुका॑मः प॒शुका॑मो॒ द्वौ वै । \newline
11. प॒शुका॑म॒ इति॑ प॒शु - का॒मः॒ । \newline
12. द्वौ वै वै द्वौ द्वौ वा अ॒जाया॑ अ॒जायै॒ वै द्वौ द्वौ वा अ॒जायै᳚ । \newline
13. वा अ॒जाया॑ अ॒जायै॒ वै वा अ॒जायै॒ स्तनौ॒ स्तना॑ व॒जायै॒ वै वा अ॒जायै॒ स्तनौ᳚ । \newline
14. अ॒जायै॒ स्तनौ॒ स्तना॑ व॒जाया॑ अ॒जायै॒ स्तनौ॒ नाना॒ नाना॒ स्तना॑ व॒जाया॑ अ॒जायै॒ स्तनौ॒ नाना᳚ । \newline
15. स्तनौ॒ नाना॒ नाना॒ स्तनौ॒ स्तनौ॒ नानै॒वैव नाना॒ स्तनौ॒ स्तनौ॒ नानै॒व । \newline
16. नानै॒वैव नाना॒ नानै॒व द्वौ द्वा वे॒व नाना॒ नानै॒व द्वौ । \newline
17. ए॒व द्वौ द्वा वे॒वैव द्वा व॒भ्य॑भि द्वा वे॒वैव द्वा व॒भि । \newline
18. द्वा व॒भ्य॑भि द्वौ द्वा व॒भि जाये॑ते॒ जाये॑ते अ॒भि द्वौ द्वा व॒भि जाये॑ते । \newline
19. अ॒भि जाये॑ते॒ जाये॑ते अ॒भ्य॑भि जाये॑ते॒ ऊर्ज॒ मूर्ज॒म् जाये॑ते अ॒भ्य॑भि जाये॑ते॒ ऊर्ज᳚म् । \newline
20. जाये॑ते॒ ऊर्ज॒ मूर्ज॒म् जाये॑ते॒ जाये॑ते॒ ऊर्ज॒म् पुष्टि॒म् पुष्टि॒ मूर्ज॒म् जाये॑ते॒ जाये॑ते॒ ऊर्ज॒म् पुष्टि᳚म् । \newline
21. जाये॑ते॒ इति॒ जाये॑ते । \newline
22. ऊर्ज॒म् पुष्टि॒म् पुष्टि॒ मूर्ज॒ मूर्ज॒म् पुष्टि॑म् तृ॒तीय॑ स्तृ॒तीयः॒ पुष्टि॒ मूर्ज॒ मूर्ज॒म् पुष्टि॑म् तृ॒तीयः॑ । \newline
23. पुष्टि॑म् तृ॒तीय॑ स्तृ॒तीयः॒ पुष्टि॒म् पुष्टि॑म् तृ॒तीयः॑ सोमापू॒षणौ॑ सोमापू॒षणौ॑ तृ॒तीयः॒ पुष्टि॒म् पुष्टि॑म् तृ॒तीयः॑ सोमापू॒षणौ᳚ । \newline
24. तृ॒तीयः॑ सोमापू॒षणौ॑ सोमापू॒षणौ॑ तृ॒तीय॑ स्तृ॒तीयः॑ सोमापू॒षणा॑ वे॒वैव सो॑मापू॒षणौ॑ तृ॒तीय॑ स्तृ॒तीयः॑ सोमापू॒षणा॑ वे॒व । \newline
25. सो॒मा॒पू॒षणा॑ वे॒वैव सो॑मापू॒षणौ॑ सोमापू॒षणा॑ वे॒व स्वेन॒ स्वेनै॒व सो॑मापू॒षणौ॑ सोमापू॒षणा॑ वे॒व स्वेन॑ । \newline
26. सो॒मा॒पू॒षणा॒विति॑ सोमा - पू॒षणौ᳚ । \newline
27. ए॒व स्वेन॒ स्वेनै॒वैव स्वेन॑ भाग॒धेये॑न भाग॒धेये॑न॒ स्वेनै॒वैव स्वेन॑ भाग॒धेये॑न । \newline
28. स्वेन॑ भाग॒धेये॑न भाग॒धेये॑न॒ स्वेन॒ स्वेन॑ भाग॒धेये॒नो पोप॑ भाग॒धेये॑न॒ स्वेन॒ स्वेन॑ भाग॒धेये॒नोप॑ । \newline
29. भा॒ग॒धेये॒नो पोप॑ भाग॒धेये॑न भाग॒धेये॒नोप॑ धावति धाव॒त्युप॑ भाग॒धेये॑न भाग॒धेये॒नोप॑ धावति । \newline
30. भा॒ग॒धेये॒नेति॑ भाग - धेये॑न । \newline
31. उप॑ धावति धाव॒ त्युपोप॑ धावति॒ तौ तौ धा॑व॒ त्युपोप॑ धावति॒ तौ । \newline
32. धा॒व॒ति॒ तौ तौ धा॑वति धावति॒ ता वे॒वैव तौ धा॑वति धावति॒ ता वे॒व । \newline
33. ता वे॒वैव तौ ता वे॒वास्मा॑ अस्मा ए॒व तौ ता वे॒वास्मै᳚ । \newline
34. ए॒वास्मा॑ अस्मा ए॒वैवास्मै॑ प॒शून् प॒शू न॑स्मा ए॒वैवास्मै॑ प॒शून् । \newline
35. अ॒स्मै॒ प॒शून् प॒शू न॑स्मा अस्मै प॒शून् प्र प्र प॒शू न॑स्मा अस्मै प॒शून् प्र । \newline
36. प॒शून् प्र प्र प॒शून् प॒शून् प्र ज॑नयतो जनयतः॒ प्र प॒शून् प॒शून् प्र ज॑नयतः । \newline
37. प्र ज॑नयतो जनयतः॒ प्र प्र ज॑नयतः॒ सोमः॒ सोमो॑ जनयतः॒ प्र प्र ज॑नयतः॒ सोमः॑ । \newline
38. ज॒न॒य॒तः॒ सोमः॒ सोमो॑ जनयतो जनयतः॒ सोमो॒ वै वै सोमो॑ जनयतो जनयतः॒ सोमो॒ वै । \newline
39. सोमो॒ वै वै सोमः॒ सोमो॒ वै रे॑तो॒धा रे॑तो॒धा वै सोमः॒ सोमो॒ वै रे॑तो॒धाः । \newline
40. वै रे॑तो॒धा रे॑तो॒धा वै वै रे॑तो॒धाः पू॒षा पू॒षा रे॑तो॒धा वै वै रे॑तो॒धाः पू॒षा । \newline
41. रे॒तो॒धाः पू॒षा पू॒षा रे॑तो॒धा रे॑तो॒धाः पू॒षा प॑शू॒नाम् प॑शू॒नाम् पू॒षा रे॑तो॒धा रे॑तो॒धाः पू॒षा प॑शू॒नाम् । \newline
42. रे॒तो॒धा इति॑ रेतः - धाः । \newline
43. पू॒षा प॑शू॒नाम् प॑शू॒नाम् पू॒षा पू॒षा प॑शू॒नाम् प्र॑जनयि॒ता प्र॑जनयि॒ता प॑शू॒नाम् पू॒षा पू॒षा प॑शू॒नाम् प्र॑जनयि॒ता । \newline
44. प॒शू॒नाम् प्र॑जनयि॒ता प्र॑जनयि॒ता प॑शू॒नाम् प॑शू॒नाम् प्र॑जनयि॒ता सोमः॒ सोमः॑ प्रजनयि॒ता प॑शू॒नाम् प॑शू॒नाम् प्र॑जनयि॒ता सोमः॑ । \newline
45. प्र॒ज॒न॒यि॒ता सोमः॒ सोमः॑ प्रजनयि॒ता प्र॑जनयि॒ता सोम॑ ए॒वैव सोमः॑ प्रजनयि॒ता प्र॑जनयि॒ता सोम॑ ए॒व । \newline
46. प्र॒ज॒न॒यि॒तेति॑ प्र - ज॒न॒यि॒ता । \newline
47. सोम॑ ए॒वैव सोमः॒ सोम॑ ए॒वास्मा॑ अस्मा ए॒व सोमः॒ सोम॑ ए॒वास्मै᳚ । \newline
48. ए॒वास्मा॑ अस्मा ए॒वैवास्मै॒ रेतो॒ रेतो᳚ ऽस्मा ए॒वैवास्मै॒ रेतः॑ । \newline
49. अ॒स्मै॒ रेतो॒ रेतो᳚ ऽस्मा अस्मै॒ रेतो॒ दधा॑ति॒ दधा॑ति॒ रेतो᳚ ऽस्मा अस्मै॒ रेतो॒ दधा॑ति । \newline
50. रेतो॒ दधा॑ति॒ दधा॑ति॒ रेतो॒ रेतो॒ दधा॑ति पू॒षा पू॒षा दधा॑ति॒ रेतो॒ रेतो॒ दधा॑ति पू॒षा । \newline
51. दधा॑ति पू॒षा पू॒षा दधा॑ति॒ दधा॑ति पू॒षा प॒शून् प॒शून् पू॒षा दधा॑ति॒ दधा॑ति पू॒षा प॒शून् । \newline
52. पू॒षा प॒शून् प॒शून् पू॒षा पू॒षा प॒शून् प्र प्र प॒शून् पू॒षा पू॒षा प॒शून् प्र । \newline
53. प॒शून् प्र प्र प॒शून् प॒शून् प्र ज॑नयति जनयति॒ प्र प॒शून् प॒शून् प्र ज॑नयति । \newline
54. प्र ज॑नयति जनयति॒ प्र प्र ज॑नय॒ त्यौदु॑म्बर॒ औदु॑म्बरो जनयति॒ प्र प्र ज॑नय॒ त्यौदु॑म्बरः । \newline
55. ज॒न॒य॒ त्यौदु॑म्बर॒ औदु॑म्बरो जनयति जनय॒ त्यौदु॑म्बरो॒ यूपो॒ यूप॒ औदु॑म्बरो जनयति जनय॒ त्यौदु॑म्बरो॒ यूपः॑ । \newline
56. औदु॑म्बरो॒ यूपो॒ यूप॒ औदु॑म्बर॒ औदु॑म्बरो॒ यूपो॑ भवति भवति॒ यूप॒ औदु॑म्बर॒ औदु॑म्बरो॒ यूपो॑ भवति । \newline
57. यूपो॑ भवति भवति॒ यूपो॒ यूपो॑ भव॒ त्यूर्गूर्ग् भ॑वति॒ यूपो॒ यूपो॑ भव॒त्यूर्क् । \newline
58. भ॒व॒त्यूर्गूर्ग् भ॑वति भव॒त्यूर्ग् वै वा ऊर्ग् भ॑वति भव॒त्यूर्ग् वै । \newline
59. ऊर्ग् वै वा ऊर्गूर्ग् वा उ॑दु॒म्बर॑ उदु॒म्बरो॒ वा ऊर्गूर्ग् वा उ॑दु॒म्बरः॑ । \newline
60. वा उ॑दु॒म्बर॑ उदु॒म्बरो॒ वै वा उ॑दु॒म्बर॒ ऊर्गूर्गु॑दु॒म्बरो॒ वै वा उ॑दु॒म्बर॒ ऊर्क् । \newline
61. उ॒दु॒म्बर॒ ऊर्गूर्गु॑दु॒म्बर॑ उदु॒म्बर॒ ऊर्क् प॒शवः॑ प॒शव॒ ऊर्गु॑दु॒म्बर॑ उदु॒म्बर॒ ऊर्क् प॒शवः॑ । \newline
62. ऊर्क् प॒शवः॑ प॒शव॒ ऊर्गूर्क् प॒शव॑ ऊ॒र्जोर्जा प॒शव॒ ऊर्गूर्क् प॒शव॑ ऊ॒र्जा । \newline
63. प॒शव॑ ऊ॒र्जोर्जा प॒शवः॑ प॒शव॑ ऊ॒र्जै वैवोर्जा प॒शवः॑ प॒शव॑ ऊ॒र्जैव । \newline
64. ऊ॒र्जैवैवोर्जो र्जैवास्मा॑ अस्मा ए॒वोर् जोर्जैवास्मै᳚ । \newline
65. ए॒वास्मा॑ अस्मा ए॒वैवास्मा॒ ऊर्ज॒ मूर्ज॑ मस्मा ए॒वैवास्मा॒ ऊर्ज᳚म् । \newline
66. अ॒स्मा॒ ऊर्ज॒ मूर्ज॑ मस्मा अस्मा॒ ऊर्ज॑म् प॒शून् प॒शू नूर्ज॑ मस्मा अस्मा॒ ऊर्ज॑म् प॒शून् । \newline
67. ऊर्ज॑म् प॒शून् प॒शू नूर्ज॒ मूर्ज॑म् प॒शू नवाव॑ प॒शू नूर्ज॒ मूर्ज॑म् प॒शू नव॑ । \newline
68. प॒शू नवाव॑ प॒शून् प॒शू नव॑ रुन्धे रु॒न्धे ऽव॑ प॒शून् प॒शू नव॑ रुन्धे । \newline
69. अव॑ रुन्धे रु॒न्धे ऽवाव॑ रुन्धे । \newline
70. रु॒न्ध॒ इति॑ रुन्धे । \newline
\pagebreak
\markright{ TS 2.1.2.1  \hfill https://www.vedavms.in \hfill}

\section{ TS 2.1.2.1 }

\textbf{TS 2.1.2.1 } \newline
\textbf{Samhita Paata} \newline

प्र॒जाप॑तिः प्र॒जा अ॑सृजत॒ ता अ॑स्माथ् सृ॒ष्टाः परा॑चीराय॒न् ता वरु॑णमगच्छ॒न् ता अन्वै॒त् ताः पुन॑रयाचत॒ ता अ॑स्मै॒ न पुन॑रददा॒थ् सो᳚ऽब्रवी॒द्-वरं॑ ॅवृणी॒ष्वाथ॑ मे॒ पुन॑र्दे॒हीति॒ तासां॒ ॅवर॒मा ऽल॑भत॒ स कृ॒ष्ण एक॑शितिपाद-भव॒द्यो वरु॑ण गृहीतः॒ स्याथ् स ए॒तं ॅवा॑रु॒णं कृ॒ष्ण-मेक॑शितिपाद॒मा-ल॑भेत॒ वरु॑ण - [  ] \newline

\textbf{Pada Paata} \newline

प्र॒जाप॑ति॒रिति॑ प्र॒जा - प॒तिः॒ । प्र॒जा इति॑ प्र - जाः । अ॒सृ॒ज॒त॒ । ताः । अ॒स्मा॒त् । सृ॒ष्टाः । परा॑चीः । आ॒य॒न्न् । ताः । वरु॑णम् । अ॒ग॒च्छ॒न्न् । ताः । अन्विति॑ । ऐ॒त् । ताः । पुनः॑ । अ॒या॒च॒त॒ । ताः । अ॒स्मै॒ । न । पुनः॑ । अ॒द॒दा॒त् । सः । अ॒ब्र॒वी॒त् । वर᳚म् । वृ॒णी॒ष्व॒ । अथ॑ । मे॒ । पुनः॑ । दे॒हि॒ । इति॑ । तासा᳚म् । वर᳚म् । एति॑ । अ॒ल॒भ॒त॒ । सः । कृ॒ष्णः । एक॑शितिपा॒दित्येक॑ - शि॒ति॒पा॒त् । अ॒भ॒व॒त् । यः । वरु॑णगृहीत॒ इति॒ वरु॑ण - गृ॒ही॒तः॒ । स्यात् । सः । ए॒तम् । वा॒रु॒णम् । कृ॒ष्णम् । एक॑शितिपाद॒मित्येक॑ - शि॒ति॒पा॒द॒म् । एति॑ । ल॒भे॒त॒ । वरु॑णम् ।  \newline


\textbf{Krama Paata} \newline

प्र॒जाप॑तिः प्र॒जाः । प्र॒जाप॑ति॒रिति॑ प्र॒जा - प॒तिः॒ । प्र॒जा अ॑सृजत । प्र॒जा इति॑ प्र - जाः । अ॒सृ॒ज॒त॒ ताः । ता अ॑स्मात् । अ॒स्मा॒थ् सृ॒ष्टाः । सृ॒ष्टाः परा॑चीः । परा॑चीरायन्न् । आ॒य॒न् ताः । ता वरु॑णम् । वरु॑णमगच्छन्न् । अ॒ग॒च्छ॒न् ताः । ता अनु॑ । अन्वै᳚त् । ऐ॒त् ताः । ताः पुनः॑ । पुन॑रयाचत । अ॒या॒च॒त ताः । ता अ॑स्मै । अ॒स्मै॒ न । न पुनः॑ । पुन॑रददात् । अ॒द॒दा॒थ् सः । सो᳚ऽब्रवीत् । अ॒ब्र॒वी॒द् वर᳚म् । वरं॑ ॅवृणीष्व । वृ॒णी॒ष्वाथ॑ । अथ॑ मे । मे॒ पुनः॑ । पुन॑र् देहि । दे॒हीति॑ । इति॒ तासा᳚म् । तासां॒ ॅवर᳚म् । वर॒मा । आऽल॑भत । अ॒ल॒भ॒त॒ सः । स कृ॒ष्णः । कृ॒ष्ण एक॑शितिपात् । एक॑शिति,पादभवत् । एक॑शितिपा॒दित्येक॑ - शि॒ति॒पा॒त्॒ । अ॒भ॒व॒द् यः । यो वरु॑णगृहीतः । वरु॑णगृहीतः॒ स्यात् । वरु॑णगृहीत॒ इति॒ वरु॑ण - गृ॒ही॒तः॒ । स्याथ् सः । स ए॒तम् । ए॒तं ॅवा॑रु॒णम् । वा॒रु॒णम् कृ॒ष्णम् । कृ॒ष्ण,मेक॑शितिपादम् । एक॑शितिपाद॒मा । एक॑शितिपाद॒मित्येक॑ - शि॒ति॒पा॒द॒म् । आ ल॑भेत । ल॒भे॒त॒ वरु॑णम् । वरु॑णमे॒व \newline

\textbf{Jatai Paata} \newline

1. प्र॒जाप॑तिः प्र॒जाः प्र॒जाः प्र॒जाप॑तिः प्र॒जाप॑तिः प्र॒जाः । \newline
2. प्र॒जाप॑ति॒रिति॑ प्र॒जा - प॒तिः॒ । \newline
3. प्र॒जा अ॑सृजता सृजत प्र॒जाः प्र॒जा अ॑सृजत । \newline
4. प्र॒जा इति॑ प्र - जाः । \newline
5. अ॒सृ॒ज॒त॒ तास्ता अ॑सृजता सृजत॒ ताः । \newline
6. ता अ॑स्मा दस्मा॒त् तास्ता अ॑स्मात् । \newline
7. अ॒स्मा॒थ् सृ॒ष्टाः सृ॒ष्टा अ॑स्मा दस्माथ् सृ॒ष्टाः । \newline
8. सृ॒ष्टाः परा॑चीः॒ परा॑चीः सृ॒ष्टाः सृ॒ष्टाः परा॑चीः । \newline
9. परा॑ची रायन् नाय॒न् परा॑चीः॒ परा॑ची रायन्न् । \newline
10. आ॒य॒न् तास्ता आ॑यन् नाय॒न् ताः । \newline
11. ता वरु॑णं॒ ॅवरु॑ण॒म् तास्ता वरु॑णम् । \newline
12. वरु॑ण मगच्छन् नगच्छ॒न्॒. वरु॑णं॒ ॅवरु॑ण मगच्छन्न् । \newline
13. अ॒ग॒च्छ॒न् तास्ता अ॑गच्छन् नगच्छ॒न् ताः । \newline
14. ता अन्वनु॒ ता स्ता अनु॑ । \newline
15. अन्वै॑ दै॒दन्वन् वै᳚त् । \newline
16. ऐ॒त् तास्ता ऐ॑दै॒त् ताः । \newline
17. ताः पुनः॒ पुन॒ स्ता स्ताः पुनः॑ । \newline
18. पुन॑ रयाचता याचत॒ पुनः॒ पुन॑ रयाचत । \newline
19. अ॒या॒च॒त॒ तास्ता अ॑याचता याचत॒ ताः । \newline
20. ता अ॑स्मा अस्मै॒ तास्ता अ॑स्मै । \newline
21. अ॒स्मै॒ न नास्मा॑ अस्मै॒ न । \newline
22. न पुनः॒ पुन॒र् न न पुनः॑ । \newline
23. पुन॑ रददा दददा॒त् पुनः॒ पुन॑ रददात् । \newline
24. अ॒द॒दा॒थ् स सो॑ ऽददा दददा॒थ् सः । \newline
25. सो᳚ ऽब्रवी दब्रवी॒थ् स सो᳚ ऽब्रवीत् । \newline
26. अ॒ब्र॒वी॒द् वरं॒ ॅवर॑ मब्रवी दब्रवी॒द् वर᳚म् । \newline
27. वरं॑ ॅवृणीष्व वृणीष्व॒ वरं॒ ॅवरं॑ ॅवृणीष्व । \newline
28. वृ॒णी॒ष्वा थाथ॑ वृणीष्व वृणी॒ष्वाथ॑ । \newline
29. अथ॑ मे॒ मे ऽथाथ॑ मे । \newline
30. मे॒ पुनः॒ पुन॑र् मे मे॒ पुनः॑ । \newline
31. पुन॑र् देहि देहि॒ पुनः॒ पुन॑र् देहि । \newline
32. दे॒हीतीति॑ देहि दे॒हीति॑ । \newline
33. इति॒ तासा॒म् तासा॒ मितीति॒ तासा᳚म् । \newline
34. तासां॒ ॅवरं॒ ॅवर॒म् तासा॒म् तासां॒ ॅवर᳚म् । \newline
35. वर॒ मा वरं॒ ॅवर॒ मा । \newline
36. आ ऽल॑भता लभ॒ता ऽल॑भत । \newline
37. अ॒ल॒भ॒त॒ स सो॑ ऽलभता लभत॒ सः । \newline
38. स कृ॒ष्णः कृ॒ष्णः स स कृ॒ष्णः । \newline
39. कृ॒ष्ण एक॑शितिपा॒ देक॑शितिपात् कृ॒ष्णः कृ॒ष्ण एक॑शितिपात् । \newline
40. एक॑शितिपा दभव दभव॒ देक॑शितिपा॒ देक॑शितिपा दभवत् । \newline
41. एक॑शितिपा॒दित्येक॑ - शि॒ति॒पा॒त् । \newline
42. अ॒भ॒व॒द् यो यो॑ ऽभव दभव॒द् यः । \newline
43. यो वरु॑णगृहीतो॒ वरु॑णगृहीतो॒ यो यो वरु॑णगृहीतः । \newline
44. वरु॑णगृहीतः॒ स्याथ् स्याद् वरु॑णगृहीतो॒ वरु॑णगृहीतः॒ स्यात् । \newline
45. वरु॑णगृहीत॒ इति॒ वरु॑ण - गृ॒ही॒तः॒ । \newline
46. स्याथ् स स स्याथ् स्याथ् सः । \newline
47. स ए॒त मे॒तꣳ स स ए॒तम् । \newline
48. ए॒तं ॅवा॑रु॒णं ॅवा॑रु॒ण मे॒त मे॒तं ॅवा॑रु॒णम् । \newline
49. वा॒रु॒णम् कृ॒ष्णम् कृ॒ष्णं ॅवा॑रु॒णं ॅवा॑रु॒णम् कृ॒ष्णम् । \newline
50. कृ॒ष्ण मेक॑शितिपाद॒ मेक॑शितिपादम् कृ॒ष्णम् कृ॒ष्ण मेक॑शितिपादम् । \newline
51. एक॑शितिपाद॒ मैक॑शितिपाद॒ मेक॑शितिपाद॒ मा । \newline
52. एक॑शितिपाद॒मित्येक॑ - शि॒ति॒पा॒द॒म् । \newline
53. आ ल॑भेत लभे॒ता ल॑भेत । \newline
54. ल॒भे॒त॒ वरु॑णं॒ ॅवरु॑णम् ॅलभेत लभेत॒ वरु॑णम् । \newline
55. वरु॑ण मे॒वैव वरु॑णं॒ ॅवरु॑ण मे॒व । \newline

\textbf{Ghana Paata } \newline

1. प्र॒जाप॑तिः प्र॒जाः प्र॒जाः प्र॒जाप॑तिः प्र॒जाप॑तिः प्र॒जा अ॑सृजता सृजत प्र॒जाः प्र॒जाप॑तिः प्र॒जाप॑तिः प्र॒जा अ॑सृजत । \newline
2. प्र॒जाप॑ति॒रिति॑ प्र॒जा - प॒तिः॒ । \newline
3. प्र॒जा अ॑सृजता सृजत प्र॒जाः प्र॒जा अ॑सृजत॒ ता स्ता अ॑सृजत प्र॒जाः प्र॒जा अ॑सृजत॒ ताः । \newline
4. प्र॒जा इति॑ प्र - जाः । \newline
5. अ॒सृ॒ज॒त॒ ता स्ता अ॑सृजता सृजत॒ ता अ॑स्मादस्मा॒त् ता अ॑सृजता सृजत॒ ता अ॑स्मात् । \newline
6. ता अ॑स्मा दस्मा॒त् ता स्ता अ॑स्माथ् सृ॒ष्टाः सृ॒ष्टा अ॑स्मा॒त् ता स्ता अ॑स्माथ् सृ॒ष्टाः । \newline
7. अ॒स्मा॒थ् सृ॒ष्टाः सृ॒ष्टा अ॑स्मा दस्माथ् सृ॒ष्टाः परा॑चीः॒ परा॑चीः सृ॒ष्टा अ॑स्मा दस्माथ् सृ॒ष्टाः परा॑चीः । \newline
8. सृ॒ष्टाः परा॑चीः॒ परा॑चीः सृ॒ष्टाः सृ॒ष्टाः परा॑ची रायन् नाय॒न् परा॑चीः सृ॒ष्टाः सृ॒ष्टाः परा॑ चीरायन्न् । \newline
9. परा॑ची रायन् नाय॒न् परा॑चीः॒ परा॑ची राय॒न् ता स्ता आ॑य॒न् परा॑चीः॒ परा॑ची राय॒न् ताः । \newline
10. आ॒य॒न् ता स्ता आ॑यन् नाय॒न् ता वरु॑णं॒ ॅवरु॑ण॒म् ता आ॑यन् नाय॒न् ता वरु॑णम् । \newline
11. ता वरु॑णं॒ ॅवरु॑ण॒म् ता स्ता वरु॑ण मगच्छन् नगच्छ॒न्॒. वरु॑ण॒म् ता स्ता वरु॑ण मगच्छन्न् । \newline
12. वरु॑ण मगच्छन् नगच्छ॒न्॒. वरु॑णं॒ ॅवरु॑ण मगच्छ॒न् ता स्ता अ॑गच्छ॒न्॒. वरु॑णं॒ ॅवरु॑ण मगच्छ॒न् ताः । \newline
13. अ॒ग॒च्छ॒न् ता स्ता अ॑गच्छन् नगच्छ॒न् ता अन्वनु॒ ता अ॑गच्छन् नगच्छ॒न् ता अनु॑ । \newline
14. ता अन्वनु॒ ता स्ता अन्वै॑ दै॒दनु॒ ता स्ता अन्वै᳚त् । \newline
15. अन्वै॑ दै॒दन्व न्वै॒त् ता स्ता ऐ॒दन्व न्वै॒त् ताः । \newline
16. ऐ॒त् ता स्ता ऐ॑दै॒त् ताः पुनः॒ पुन॒ स्ता ऐ॑दै॒त् ताः पुनः॑ । \newline
17. ताः पुनः॒ पुन॒ स्ता स्ताः पुन॑ रयाचता याचत॒ पुन॒ स्ता स्ताः पुन॑ रयाचत । \newline
18. पुन॑ रयाचता याचत॒ पुनः॒ पुन॑ रयाचत॒ ता स्ता अ॑याचत॒ पुनः॒ पुन॑ रयाचत॒ ताः । \newline
19. अ॒या॒च॒त॒ ता स्ता अ॑याचता याचत॒ ता अ॑स्मा अस्मै॒ ता अ॑याचता याचत॒ ता अ॑स्मै । \newline
20. ता अ॑स्मा अस्मै॒ ता स्ता अ॑स्मै॒ न नास्मै॒ ता स्ता अ॑स्मै॒ न । \newline
21. अ॒स्मै॒ न नास्मा॑ अस्मै॒ न पुनः॒ पुन॒र् नास्मा॑ अस्मै॒ न पुनः॑ । \newline
22. न पुनः॒ पुन॒र् न न पुन॑ रददा दद दा॒त् पुन॒र् न न पुन॑ रददात् । \newline
23. पुन॑ रददा दद दा॒त् पुनः॒ पुन॑ रद दा॒थ् स सो॑ ऽददा॒त् पुनः॒ पुन॑ रद दा॒थ् सः । \newline
24. अ॒द॒ दा॒थ् स सो॑ ऽददा दद दा॒थ् सो᳚ ऽब्रवी दब्रवी॒थ् सो॑ ऽददा दद दा॒थ् सो᳚ ऽब्रवीत् । \newline
25. सो᳚ ऽब्रवी दब्रवी॒थ् स सो᳚ ऽब्रवी॒द् वरं॒ ॅवर॑ मब्रवी॒थ् स सो᳚ ऽब्रवी॒द् वर᳚म् । \newline
26. अ॒ब्र॒वी॒द् वरं॒ ॅवर॑ मब्रवी दब्रवी॒द् वरं॑ ॅवृणीष्व वृणीष्व॒ वर॑ मब्रवी दब्रवी॒द् वरं॑ ॅवृणीष्व । \newline
27. वरं॑ ॅवृणीष्व वृणीष्व॒ वरं॒ ॅवरं॑ ॅवृणी॒ ष्वाथाथ॑ वृणीष्व॒ वरं॒ ॅवरं॑ ॅवृणी॒ ष्वाथ॑ । \newline
28. वृ॒णी॒ ष्वाथाथ॑ वृणीष्व वृणी॒ ष्वाथ॑ मे॒ मे ऽथ॑ वृणीष्व वृणी॒ ष्वाथ॑ मे । \newline
29. अथ॑ मे॒ मे ऽथाथ॑ मे॒ पुनः॒ पुन॒र् मे ऽथाथ॑ मे॒ पुनः॑ । \newline
30. मे॒ पुनः॒ पुन॑र् मे मे॒ पुन॑र् देहि देहि॒ पुन॑र् मे मे॒ पुन॑र् देहि । \newline
31. पुन॑र् देहि देहि॒ पुनः॒ पुन॑र् दे॒हीतीति॑ देहि॒ पुनः॒ पुन॑र् दे॒हीति॑ । \newline
32. दे॒हीतीति॑ देहि दे॒हीति॒ तासा॒म् तासा॒ मिति॑ देहि दे॒हीति॒ तासा᳚म् । \newline
33. इति॒ तासा॒म् तासा॒ मितीति॒ तासां॒ ॅवरं॒ ॅवर॒म् तासा॒ मितीति॒ तासां॒ ॅवर᳚म् । \newline
34. तासां॒ ॅवरं॒ ॅवर॒म् तासा॒म् तासां॒ ॅवर॒ मा वर॒म् तासा॒म् तासां॒ ॅवर॒ मा । \newline
35. वर॒ मा वरं॒ ॅवर॒ मा ऽल॑भता लभ॒ता वरं॒ ॅवर॒ मा ऽल॑भत । \newline
36. आ ऽल॑भता लभ॒ता ऽल॑भत॒ स सो॑ ऽलभ॒ता ऽल॑भत॒ सः । \newline
37. अ॒ल॒भ॒त॒ स सो॑ ऽलभता लभत॒ स कृ॒ष्णः कृ॒ष्णः सो॑ ऽलभता लभत॒ स कृ॒ष्णः । \newline
38. स कृ॒ष्णः कृ॒ष्णः स स कृ॒ष्ण एक॑शितिपा॒ देक॑शितिपात् कृ॒ष्णः स स कृ॒ष्ण एक॑शितिपात् । \newline
39. कृ॒ष्ण एक॑शितिपा॒ देक॑शितिपात् कृ॒ष्णः कृ॒ष्ण एक॑शितिपा दभव दभव॒ देक॑शितिपात् कृ॒ष्णः कृ॒ष्ण एक॑शितिपा दभवत् । \newline
40. एक॑शितिपा दभव दभव॒ देक॑शितिपा॒ देक॑शितिपा दभव॒द् यो यो॑ ऽभव॒ देक॑शितिपा॒ देक॑शितिपा दभव॒द् यः । \newline
41. एक॑शितिपा॒दित्येक॑ - शि॒ति॒पा॒त् । \newline
42. अ॒भ॒व॒द् यो यो॑ ऽभव दभव॒द् यो वरु॑णगृहीतो॒ वरु॑णगृहीतो॒ यो॑ ऽभव दभव॒द् यो वरु॑णगृहीतः । \newline
43. यो वरु॑णगृहीतो॒ वरु॑णगृहीतो॒ यो यो वरु॑णगृहीतः॒ स्याथ् स्याद् वरु॑णगृहीतो॒ यो यो वरु॑णगृहीतः॒ स्यात् । \newline
44. वरु॑णगृहीतः॒ स्याथ् स्याद् वरु॑णगृहीतो॒ वरु॑णगृहीतः॒ स्याथ् स स स्याद् वरु॑णगृहीतो॒ वरु॑णगृहीतः॒ स्याथ् सः । \newline
45. वरु॑णगृहीत॒ इति॒ वरु॑ण - गृ॒ही॒तः॒ । \newline
46. स्याथ् स स स्याथ् स्याथ् स ए॒त मे॒तꣳ स स्याथ् स्याथ् स ए॒तम् । \newline
47. स ए॒त मे॒तꣳ स स ए॒तं ॅवा॑रु॒णं ॅवा॑रु॒ण मे॒तꣳ स स ए॒तं ॅवा॑रु॒णम् । \newline
48. ए॒तं ॅवा॑रु॒णं ॅवा॑रु॒ण मे॒त मे॒तं ॅवा॑रु॒णम् कृ॒ष्णम् कृ॒ष्णं ॅवा॑रु॒ण मे॒त मे॒तं ॅवा॑रु॒णम् कृ॒ष्णम् । \newline
49. वा॒रु॒णम् कृ॒ष्णम् कृ॒ष्णं ॅवा॑रु॒णं ॅवा॑रु॒णम् कृ॒ष्ण मेक॑शितिपाद॒ मेक॑शितिपादम् कृ॒ष्णं ॅवा॑रु॒णं ॅवा॑रु॒णम् कृ॒ष्ण मेक॑शितिपादम् । \newline
50. कृ॒ष्ण मेक॑शितिपाद॒ मेक॑शितिपादम् कृ॒ष्णम् कृ॒ष्ण मेक॑शितिपाद॒ मैक॑शितिपादम् कृ॒ष्णम् कृ॒ष्ण मेक॑शितिपाद॒ मा । \newline
51. एक॑शितिपाद॒ मैक॑शितिपाद॒ मेक॑शितिपाद॒ मा ल॑भेत लभे॒ तैक॑शितिपाद॒ मेक॑शितिपाद॒ मा ल॑भेत । \newline
52. एक॑शितिपाद॒मित्येक॑ - शि॒ति॒पा॒द॒म् । \newline
53. आ ल॑भेत लभे॒ता ल॑भेत॒ वरु॑णं॒ ॅवरु॑णम् ॅलभे॒ता ल॑भेत॒ वरु॑णम् । \newline
54. ल॒भे॒त॒ वरु॑णं॒ ॅवरु॑णम् ॅलभेत लभेत॒ वरु॑ण मे॒वैव वरु॑णम् ॅलभेत लभेत॒ वरु॑ण मे॒व । \newline
55. वरु॑ण मे॒वैव वरु॑णं॒ ॅवरु॑ण मे॒व स्वेन॒ स्वेनै॒व वरु॑णं॒ ॅवरु॑ण मे॒व स्वेन॑ । \newline
\pagebreak
\markright{ TS 2.1.2.2  \hfill https://www.vedavms.in \hfill}

\section{ TS 2.1.2.2 }

\textbf{TS 2.1.2.2 } \newline
\textbf{Samhita Paata} \newline

-मे॒व स्वेन॑ भाग॒धेये॒नोप॑ धावति॒ स ए॒वैनं॑ ॅवरुणपा॒शान् मु॑ञ्चति कृ॒ष्ण एक॑शितिपाद्-भवति वारु॒णो ह्ये॑ष दे॒वत॑या॒ समृ॑द्ध्यै॒ सुव॑र्भानुरासु॒रः सूर्यं॒ तम॑साऽविद्ध्य॒त तस्मै॑दे॒वाः प्राय॑श्चित्ति-मैच्छ॒न् तस्य॒ यत् प्र॑थ॒मं तमो॒ऽपाघ्न॒न्थ्‌सा कृ॒ष्णाऽवि॑रभव॒द्-यद् द्वि॒तीयꣳ॒॒ सा फल्गु॑नी॒ यत तृ॒तीयꣳ॒॒ सा ब॑ल॒क्षी यद॑द्ध्य॒स्था-द॒पाकृ॑न्त॒न्थ्‌सा ऽवि॑र्व॒शा - [  ] \newline

\textbf{Pada Paata} \newline

ए॒व । स्वेन॑ । भा॒ग॒धेये॒नेति॑ भाग - धेये॑न । उपेति॑ । धा॒व॒ति॒ । सः । ए॒व । ए॒न॒म् । व॒रु॒ण॒पा॒शादिति॑ वरुण - पा॒शात् । मु॒ञ्च॒ति॒ । कृ॒ष्णः । एक॑शितिपा॒दित्येक॑ - शि॒ति॒पा॒त् । भ॒व॒ति॒ । वा॒रु॒णः । हि । ए॒षः । दे॒वत॑या । समृ॑द्ध्या॒ इति॒ सम् - ऋ॒द्धयै॒ । सुव॑र्भानु॒रिति॒ सुवः॑ - भा॒नुः॒ । आ॒सु॒रः । सूर्य᳚म् । तम॑सा । अ॒वि॒द्ध्य॒त् । तस्मै᳚ । दे॒वाः । प्राय॑श्चित्तिम् । ऐ॒च्छ॒न्न् । तस्य॑ । यत् । प्र॒थ॒मम् । तमः॑ । अ॒पाघ्न॒न्नित्य॑प - अघ्नन्न्॑ । सा । कृ॒ष्णा । अविः॑ । अ॒भ॒व॒त् । यत् । द्वि॒तीय᳚म् । सा । फल्गु॑नी । यत् । तृ॒तीय᳚म् । सा । ब॒ल॒क्षी । यत् । अ॒द्ध्य॒स्थादित्य॑धि - अ॒स्थात् । अ॒पाकृ॑न्त॒न्नित्य॑प - अकृ॑न्तन्न् । सा । अविः॑ । व॒शा ।  \newline


\textbf{Krama Paata} \newline

ए॒व स्वेन॑ । स्वेन॑ भाग॒धेये॑न । भा॒ग॒धेये॒नोप॑ । भा॒ग॒धेये॒नेति॑ भाग - धेये॑न । उप॑ धावति । धा॒व॒ति॒ सः । स ए॒व । ए॒वैन᳚म् । ए॒नं॒ ॅव॒रु॒ण॒पा॒शात् । व॒रु॒ण॒पा॒शान् मु॑ञ्चति । व॒रु॒ण॒पा॒शादिति॑ वरुण - पा॒शात् । मु॒ञ्च॒ति॒ कृ॒ष्णः । कृ॒ष्ण एक॑शितिपात् । एक॑शितिपाद्,भवति । एक॑शितिपा॒दित्येक॑ - शि॒ति॒पा॒त्॒ । भ॒व॒ति॒ वा॒रु॒णः । वा॒रु॒णो हि । ह्ये॑षः । ए॒ष दे॒वत॑या । दे॒वत॑या॒ समृ॑द्ध्यै । समृ॑द्ध्यै॒ सुव॑र्भानुः । समृ॑द्ध्या॒ इति॒ सं - ऋ॒द्ध्यै॒ । सुव॑र्भानुरासु॒रः । सुव॑र्भानु॒रिति॒ सुवः॑ - भा॒नुः॒ । आ॒सु॒रः सूर्य᳚म् । सूर्य॒म् तम॑सा । तम॑साऽविद्ध्यत् । अ॒वि॒द्ध्य॒त् तस्मै᳚ । तस्मै॑ दे॒वाः । दे॒वाः प्राय॑श्चित्तिम् । प्राय॑श्चित्तिमैच्छन्न् । ऐ॒च्छ॒न् तस्य॑ । तस्य॒ यत् । यत्,प्र॑थ॒मम् । प्र॒थ॒मम् तमः॑ । तमो॒ऽपाघ्नन्न्॑ । अ॒पाघ्न॒न्थ् सा । अ॒पाघ्न॒न्नित्य॑प - अघ्नन्न्॑ । सा कृ॒ष्णा । कृ॒ष्णा ऽविः॑ । अवि॑रभवत् । अ॒भ॒व॒द् यत् । यद् द्वि॒तीय᳚म् । द्वि॒तीयꣳ॒॒ सा । सा फल्गु॑नी । फल्गु॑नी॒ यत् । यत् तृ॒तीय᳚म् । तृ॒तीयꣳ॒॒ सा । सा ब॑ल॒क्षी । ब॒ल॒क्षी यत् । यद॑द्ध्य॒स्थात् । अ॒द्ध्य॒स्थाद॒पाकृ॑न्तन्न् । अ॒द्ध्य॒स्थादित्य॑धि - अ॒स्थात् । अ॒पाकृ॑न्त॒न्थ् सा । अ॒पाकृ॑न्त॒न्नित्य॑प - अकृ॑न्तन्न् । साऽविः॑ । अवि॑र् व॒शा । व॒शा सम् \newline

\textbf{Jatai Paata} \newline

1. ए॒व स्वेन॒ स्वेनै॒ वैव स्वेन॑ । \newline
2. स्वेन॑ भाग॒धेये॑न भाग॒धेये॑न॒ स्वेन॒ स्वेन॑ भाग॒धेये॑न । \newline
3. भा॒ग॒धेये॒नोपोप॑ भाग॒धेये॑न भाग॒धेये॒नोप॑ । \newline
4. भा॒ग॒धेये॒नेति॑ भाग - धेये॑न । \newline
5. उप॑ धावति धाव॒ त्युपोप॑ धावति । \newline
6. धा॒व॒ति॒ स स धा॑वति धावति॒ सः । \newline
7. स ए॒वैव स स ए॒व । \newline
8. ए॒वैन॑ मेन मे॒वै वैन᳚म् । \newline
9. ए॒नं॒ ॅव॒रु॒ण॒पा॒शाद् व॑रुणपा॒शादे॑न मेनं ॅवरुणपा॒शात् । \newline
10. व॒रु॒ण॒पा॒शान् मु॑ञ्चति मुञ्चति वरुणपा॒शाद् व॑रुणपा॒शान् मु॑ञ्चति । \newline
11. व॒रु॒ण॒पा॒शादिति॑ वरुण - पा॒शात् । \newline
12. मु॒ञ्च॒ति॒ कृ॒ष्णः कृ॒ष्णो मु॑ञ्चति मुञ्चति कृ॒ष्णः । \newline
13. कृ॒ष्ण एक॑शितिपा॒ देक॑शितिपात् कृ॒ष्णः कृ॒ष्ण एक॑शितिपात् । \newline
14. एक॑शितिपाद् भवति भव॒ त्येक॑शितिपा॒ देक॑शितिपाद् भवति । \newline
15. एक॑शितिपा॒दित्येक॑ - शि॒ति॒पा॒त् । \newline
16. भ॒व॒ति॒ वा॒रु॒णो वा॑रु॒णो भ॑वति भवति वारु॒णः । \newline
17. वा॒रु॒णो हि हि वा॑रु॒णो वा॑रु॒णो हि । \newline
18. ह्ये॑ष ए॒ष हि ह्ये॑षः । \newline
19. ए॒ष दे॒वत॑या दे॒वत॑यै॒ष ए॒ष दे॒वत॑या । \newline
20. दे॒वत॑या॒ समृ॑द्ध्यै॒ समृ॑द्ध्यै दे॒वत॑या दे॒वत॑या॒ समृ॑द्ध्यै । \newline
21. समृ॑द्ध्यै॒ सुव॑र्भानुः॒ सुव॑र्भानुः॒ समृ॑द्ध्यै॒ समृ॑द्ध्यै॒ सुव॑र्भानुः । \newline
22. समृ॑द्ध्या॒ इति॒ सम् - ऋ॒द्ध्यै॒ । \newline
23. सुव॑र्भानु रासु॒र आ॑सु॒रः सुव॑र्भानुः॒ सुव॑र्भानु रासु॒रः । \newline
24. सुव॑र्भानु॒रिति॒ सुवः॑ - भा॒नुः॒ । \newline
25. आ॒सु॒रः सूर्यꣳ॒॒ सूर्य॑ मासु॒र आ॑सु॒रः सूर्य᳚म् । \newline
26. सूर्य॒म् तम॑सा॒ तम॑सा॒ सूर्यꣳ॒॒ सूर्य॒म् तम॑सा । \newline
27. तम॑सा ऽविद्ध्य दविद्ध्य॒त् तम॑सा॒ तम॑सा ऽविद्ध्यत् । \newline
28. अ॒वि॒द्ध्य॒त् तस्मै॒ तस्मा॑ अविद्ध्य दविद्ध्य॒त् तस्मै᳚ । \newline
29. तस्मै॑ दे॒वा दे॒वा स्तस्मै॒ तस्मै॑ दे॒वाः । \newline
30. दे॒वाः प्राय॑श्चित्ति॒म् प्राय॑श्चित्तिम् दे॒वा दे॒वाः प्राय॑श्चित्तिम् । \newline
31. प्राय॑श्चित्ति मैच्छन् नैच्छ॒न् प्राय॑श्चित्ति॒म् प्राय॑श्चित्ति मैच्छन्न् । \newline
32. ऐ॒च्छ॒न् तस्य॒ तस्यै᳚च्छन् नैच्छ॒न् तस्य॑ । \newline
33. तस्य॒ यद् यत् तस्य॒ तस्य॒ यत् । \newline
34. यत् प्र॑थ॒मम् प्र॑थ॒मं ॅयद् यत् प्र॑थ॒मम् । \newline
35. प्र॒थ॒मम् तम॒स्तमः॑ प्रथ॒मम् प्र॑थ॒मम् तमः॑ । \newline
36. तमो॒ ऽपाघ्न॑न् न॒पाघ्न॒न् तम॒स्तमो॒ ऽपाघ्नन्न्॑ । \newline
37. अ॒पाघ्न॒न् थ्सा सा ऽपाघ्न॑न् न॒पाघ्न॒न् थ्सा । \newline
38. अ॒पाघ्न॒न्नित्य॑प - अघ्नन्न्॑ । \newline
39. सा कृ॒ष्णा कृ॒ष्णा सा सा कृ॒ष्णा । \newline
40. कृ॒ष्णा ऽवि॒रविः॑ कृ॒ष्णा कृ॒ष्णा ऽविः॑ । \newline
41. अवि॑ रभव दभव॒ दवि॒ रवि॑ रभवत् । \newline
42. अ॒भ॒व॒द् यद् यद॑भव दभव॒द् यत् । \newline
43. यद् द्वि॒तीय॑म् द्वि॒तीयं॒ ॅयद् यद् द्वि॒तीय᳚म् । \newline
44. द्वि॒तीयꣳ॒॒ सा सा द्वि॒तीय॑म् द्वि॒तीयꣳ॒॒ सा । \newline
45. सा फल्गु॑नी॒ फल्गु॑नी॒ सा सा फल्गु॑नी । \newline
46. फल्गु॑नी॒ यद् यत् फल्गु॑नी॒ फल्गु॑नी॒ यत् । \newline
47. यत् तृ॒तीय॑म् तृ॒तीयं॒ ॅयद् यत् तृ॒तीय᳚म् । \newline
48. तृ॒तीयꣳ॒॒ सा सा तृ॒तीय॑म् तृ॒तीयꣳ॒॒ सा । \newline
49. सा ब॑ल॒क्षी ब॑ल॒क्षी सा सा ब॑ल॒क्षी । \newline
50. ब॒ल॒क्षी यद् यद् ब॑ल॒क्षी ब॑ल॒क्षी यत् । \newline
51. यद॑द्ध्य॒स्था द॑द्ध्य॒स्थाद् यद् यद॑द्ध्य॒स्थात् । \newline
52. अ॒द्ध्य॒स्था द॒पाकृ॑न्तन् न॒पाकृ॑न्तन् नद्ध्य॒स्था द॑द्ध्य॒स्था द॒पाकृ॑न्तन्न् । \newline
53. अ॒द्ध्य॒स्थादित्य॑धि - अ॒स्थात् । \newline
54. अ॒पाकृ॑न्त॒न् थ्सा सा ऽपाकृ॑न्तन् न॒पाकृ॑न्त॒न् थ्सा । \newline
55. अ॒पाकृ॑न्त॒न्नित्य॑प - अकृ॑न्तन्न् । \newline
56. सा ऽवि॒ रविः॒ सा सा ऽविः॑ । \newline
57. अवि॑र् व॒शा व॒शा ऽवि॒ रवि॑र् व॒शा । \newline
58. व॒शा सꣳ सं ॅव॒शा व॒शा सम् । \newline

\textbf{Ghana Paata } \newline

1. ए॒व स्वेन॒ स्वेनै॒वैव स्वेन॑ भाग॒धेये॑न भाग॒धेये॑न॒ स्वेनै॒वैव स्वेन॑ भाग॒धेये॑न । \newline
2. स्वेन॑ भाग॒धेये॑न भाग॒धेये॑न॒ स्वेन॒ स्वेन॑ भाग॒धेये॒नो पोप॑ भाग॒धेये॑न॒ स्वेन॒ स्वेन॑ भाग॒धेये॒नोप॑ । \newline
3. भा॒ग॒धेये॒नो पोप॑ भाग॒धेये॑न भाग॒धेये॒नोप॑ धावति धाव॒त्युप॑ भाग॒धेये॑न भाग॒धेये॒नोप॑ धावति । \newline
4. भा॒ग॒धेये॒नेति॑ भाग - धेये॑न । \newline
5. उप॑ धावति धाव॒ त्युपोप॑ धावति॒ स स धा॑व॒ त्युपोप॑ धावति॒ सः । \newline
6. धा॒व॒ति॒ स स धा॑वति धावति॒ स ए॒वैव स धा॑वति धावति॒ स ए॒व । \newline
7. स ए॒वैव स स ए॒वैन॑ मेन मे॒व स स ए॒वैन᳚म् । \newline
8. ए॒वैन॑ मेन मे॒वैवैनं॑ ॅवरुणपा॒शाद् व॑रुणपा॒शा दे॑न मे॒वैवैनं॑ ॅवरुणपा॒शात् । \newline
9. ए॒नं॒ ॅव॒रु॒ण॒पा॒शाद् व॑रुणपा॒शा दे॑न मेनं ॅवरुणपा॒शान् मु॑ञ्चति मुञ्चति वरुणपा॒शा दे॑न मेनं ॅवरुणपा॒शान् मु॑ञ्चति । \newline
10. व॒रु॒ण॒पा॒शान् मु॑ञ्चति मुञ्चति वरुणपा॒शाद् व॑रुणपा॒शान् मु॑ञ्चति कृ॒ष्णः कृ॒ष्णो मु॑ञ्चति वरुणपा॒शाद् व॑रुणपा॒शान् मु॑ञ्चति कृ॒ष्णः । \newline
11. व॒रु॒ण॒पा॒शादिति॑ वरुण - पा॒शात् । \newline
12. मु॒ञ्च॒ति॒ कृ॒ष्णः कृ॒ष्णो मु॑ञ्चति मुञ्चति कृ॒ष्ण एक॑शितिपा॒ देक॑शितिपात् कृ॒ष्णो मु॑ञ्चति मुञ्चति कृ॒ष्ण एक॑शितिपात् । \newline
13. कृ॒ष्ण एक॑शितिपा॒ देक॑शितिपात् कृ॒ष्णः कृ॒ष्ण एक॑शितिपाद् भवति भव॒ त्येक॑शितिपात् कृ॒ष्णः कृ॒ष्ण एक॑शितिपाद् भवति । \newline
14. एक॑शितिपाद् भवति भव॒ त्येक॑शितिपा॒ देक॑शितिपाद् भवति वारु॒णो वा॑रु॒णो भ॑व॒ त्येक॑शितिपा॒ देक॑शितिपाद् भवति वारु॒णः । \newline
15. एक॑शितिपा॒दित्येक॑ - शि॒ति॒पा॒त् । \newline
16. भ॒व॒ति॒ वा॒रु॒णो वा॑रु॒णो भ॑वति भवति वारु॒णो हि हि वा॑रु॒णो भ॑वति भवति वारु॒णो हि । \newline
17. वा॒रु॒णो हि हि वा॑रु॒णो वा॑रु॒णो ह्ये॑ष ए॒ष हि वा॑रु॒णो वा॑रु॒णो ह्ये॑षः । \newline
18. ह्ये॑ष ए॒ष हि ह्ये॑ष दे॒वत॑या दे॒वत॑यै॒ष हि ह्ये॑ष दे॒वत॑या । \newline
19. ए॒ष दे॒वत॑या दे॒वत॑यै॒ष ए॒ष दे॒वत॑या॒ समृ॑द्ध्यै॒ समृ॑द्ध्यै दे॒वत॑यै॒ष ए॒ष दे॒वत॑या॒ समृ॑द्ध्यै । \newline
20. दे॒वत॑या॒ समृ॑द्ध्यै॒ समृ॑द्ध्यै दे॒वत॑या दे॒वत॑या॒ समृ॑द्ध्यै॒ सुव॑र्भानुः॒ सुव॑र्भानुः॒ समृ॑द्ध्यै दे॒वत॑या दे॒वत॑या॒ समृ॑द्ध्यै॒ सुव॑र्भानुः । \newline
21. समृ॑द्ध्यै॒ सुव॑र्भानुः॒ सुव॑र्भानुः॒ समृ॑द्ध्यै॒ समृ॑द्ध्यै॒ सुव॑र्भानु रासु॒र आ॑सु॒रः सुव॑र्भानुः॒ समृ॑द्ध्यै॒ समृ॑द्ध्यै॒ सुव॑र्भानु रासु॒रः । \newline
22. समृ॑द्ध्या॒ इति॒ सम् - ऋ॒द्ध्यै॒ । \newline
23. सुव॑र्भानु रासु॒र आ॑सु॒रः सुव॑र्भानुः॒ सुव॑र्भानु रासु॒रः सूर्यꣳ॒॒ सूर्य॑ मासु॒रः सुव॑र्भानुः॒ सुव॑र्भानु रासु॒रः सूर्य᳚म् । \newline
24. सुव॑र्भानु॒रिति॒ सुवः॑ - भा॒नुः॒ । \newline
25. आ॒सु॒रः सूर्यꣳ॒॒ सूर्य॑ मासु॒र आ॑सु॒रः सूर्य॒म् तम॑सा॒ तम॑सा॒ सूर्य॑ मासु॒र आ॑सु॒रः सूर्य॒म् तम॑सा । \newline
26. सूर्य॒म् तम॑सा॒ तम॑सा॒ सूर्यꣳ॒॒ सूर्य॒म् तम॑सा ऽविद्ध्य दविद्ध्य॒त् तम॑सा॒ सूर्यꣳ॒॒ सूर्य॒म् तम॑सा ऽविद्ध्यत् । \newline
27. तम॑सा ऽविद्ध्य दविद्ध्य॒त् तम॑सा॒ तम॑सा ऽविद्ध्य॒त् तस्मै॒ तस्मा॑ अविद्ध्य॒त् तम॑सा॒ तम॑सा ऽविद्ध्य॒त् तस्मै᳚ । \newline
28. अ॒वि॒द्ध्य॒त् तस्मै॒ तस्मा॑ अविद्ध्य दविद्ध्य॒त् तस्मै॑ दे॒वा दे॒वा स्तस्मा॑ अविद्ध्य दविद्ध्य॒त् तस्मै॑ दे॒वाः । \newline
29. तस्मै॑ दे॒वा दे॒वा स्तस्मै॒ तस्मै॑ दे॒वाः प्राय॑श्चित्ति॒म् प्राय॑श्चित्तिम् दे॒वा स्तस्मै॒ तस्मै॑ दे॒वाः प्राय॑श्चित्तिम् । \newline
30. दे॒वाः प्राय॑श्चित्ति॒म् प्राय॑श्चित्तिम् दे॒वा दे॒वाः प्राय॑श्चित्ति मैच्छन् नैच्छ॒न् प्राय॑श्चित्तिम् दे॒वा दे॒वाः प्राय॑श्चित्ति मैच्छन्न् । \newline
31. प्राय॑श्चित्ति मैच्छन् नैच्छ॒न् प्राय॑श्चित्ति॒म् प्राय॑श्चित्ति मैच्छ॒न् तस्य॒ तस्यै᳚च्छ॒न् प्राय॑श्चित्ति॒म् प्राय॑श्चित्ति मैच्छ॒न् तस्य॑ । \newline
32. ऐ॒च्छ॒न् तस्य॒ तस्यै᳚च्छन् नैच्छ॒न् तस्य॒ यद् यत् तस्यै᳚च्छन् नैच्छ॒न् तस्य॒ यत् । \newline
33. तस्य॒ यद् यत् तस्य॒ तस्य॒ यत् प्र॑थ॒मम् प्र॑थ॒मं ॅयत् तस्य॒ तस्य॒ यत् प्र॑थ॒मम् । \newline
34. यत् प्र॑थ॒मम् प्र॑थ॒मं ॅयद् यत् प्र॑थ॒मम् तम॒ स्तमः॑ प्रथ॒मं ॅयद् यत् प्र॑थ॒मम् तमः॑ । \newline
35. प्र॒थ॒मम् तम॒ स्तमः॑ प्रथ॒मम् प्र॑थ॒मम् तमो॒ ऽपाघ्न॑न् न॒पाघ्न॒न् तमः॑ प्रथ॒मम् प्र॑थ॒मम् तमो॒ ऽपाघ्नन्न्॑ । \newline
36. तमो॒ ऽपाघ्न॑न् न॒पाघ्न॒न् तम॒ स्तमो॒ ऽपाघ्न॒न् थ्सा सा ऽपाघ्न॒न् तम॒ स्तमो॒ ऽपाघ्न॒न् थ्सा । \newline
37. अ॒पाघ्न॒न् थ्सा सा ऽपाघ्न॑न् न॒पाघ्न॒न् थ्सा कृ॒ष्णा कृ॒ष्णा सा ऽपाघ्न॑न् न॒पाघ्न॒न् थ्सा कृ॒ष्णा । \newline
38. अ॒पाघ्न॒न्नित्य॑प - अघ्नन्न्॑ । \newline
39. सा कृ॒ष्णा कृ॒ष्णा सा सा कृ॒ष्णा ऽवि॒रविः॑ कृ॒ष्णा सा सा कृ॒ष्णा ऽविः॑ । \newline
40. कृ॒ष्णा ऽवि॒रविः॑ कृ॒ष्णा कृ॒ष्णा ऽवि॑ रभव दभव॒ दविः॑ कृ॒ष्णा कृ॒ष्णा ऽवि॑रभवत् । \newline
41. अवि॑ रभव दभव॒ दवि॒ रवि॑ रभव॒द् यद् यद॑भव॒ दवि॒ रवि॑ रभव॒द् यत् । \newline
42. अ॒भ॒व॒द् यद् यद॑भव दभव॒द् यद् द्वि॒तीय॑म् द्वि॒तीयं॒ ॅयद॑भव दभव॒द् यद् द्वि॒तीय᳚म् । \newline
43. यद् द्वि॒तीय॑म् द्वि॒तीयं॒ ॅयद् यद् द्वि॒तीयꣳ॒॒ सा सा द्वि॒तीयं॒ ॅयद् यद् द्वि॒तीयꣳ॒॒ सा । \newline
44. द्वि॒तीयꣳ॒॒ सा सा द्वि॒तीय॑म् द्वि॒तीयꣳ॒॒ सा फल्गु॑नी॒ फल्गु॑नी॒ सा द्वि॒तीय॑म् द्वि॒तीयꣳ॒॒ सा फल्गु॑नी । \newline
45. सा फल्गु॑नी॒ फल्गु॑नी॒ सा सा फल्गु॑नी॒ यद् यत् फल्गु॑नी॒ सा सा फल्गु॑नी॒ यत् । \newline
46. फल्गु॑नी॒ यद् यत् फल्गु॑नी॒ फल्गु॑नी॒ यत् तृ॒तीय॑म् तृ॒तीयं॒ ॅयत् फल्गु॑नी॒ फल्गु॑नी॒ यत् तृ॒तीय᳚म् । \newline
47. यत् तृ॒तीय॑म् तृ॒तीयं॒ ॅयद् यत् तृ॒तीयꣳ॒॒ सा सा तृ॒तीयं॒ ॅयद् यत् तृ॒तीयꣳ॒॒ सा । \newline
48. तृ॒तीयꣳ॒॒ सा सा तृ॒तीय॑म् तृ॒तीयꣳ॒॒ सा ब॑ल॒क्षी ब॑ल॒क्षी सा तृ॒तीय॑म् तृ॒तीयꣳ॒॒ सा ब॑ल॒क्षी । \newline
49. सा ब॑ल॒क्षी ब॑ल॒क्षी सा सा ब॑ल॒क्षी यद् यद् ब॑ल॒क्षी सा सा ब॑ल॒क्षी यत् । \newline
50. ब॒ल॒क्षी यद् यद् ब॑ल॒क्षी ब॑ल॒क्षी यद॑द्ध्य॒स्था द॑द्ध्य॒स्थाद् यद् ब॑ल॒क्षी ब॑ल॒क्षी यद॑द्ध्य॒स्थात् । \newline
51. यद॑द्ध्य॒स्था द॑द्ध्य॒स्थाद् यद् यद॑द्ध्य॒स्था द॒पाकृ॑न्तन् न॒पाकृ॑न्तन् नद्ध्य॒स्थाद् यद् यद॑द्ध्य॒स्था द॒पाकृ॑न्तन्न् । \newline
52. अ॒द्ध्य॒स्था द॒पाकृ॑न्तन् न॒पाकृ॑न्तन् नद्ध्य॒स्था द॑द्ध्य॒स्था द॒पाकृ॑न्त॒न् थ्सा सा ऽपाकृ॑न्तन् नद्ध्य॒स्था द॑द्ध्य॒स्था द॒पाकृ॑न्त॒न् थ्सा । \newline
53. अ॒द्ध्य॒स्थादित्य॑धि - अ॒स्थात् । \newline
54. अ॒पाकृ॑न्त॒न् थ्सा सा ऽपाकृ॑न्तन् न॒पाकृ॑न्त॒न् थ्सा ऽवि॒रविः॒ सा ऽपाकृ॑न्तन् न॒पाकृ॑न्त॒न् थ्सा ऽविः॑ । \newline
55. अ॒पाकृ॑न्त॒न्नित्य॑प - अकृ॑न्तन्न् । \newline
56. सा ऽवि॒ रविः॒ सा सा ऽवि॑र् व॒शा व॒शा ऽविः॒ सा सा ऽवि॑र् व॒शा । \newline
57. अवि॑र् व॒शा व॒शा ऽवि॒ रवि॑र् व॒शा सꣳ सं ॅव॒शा ऽवि॒ रवि॑र् व॒शा सम् । \newline
58. व॒शा सꣳ सं ॅव॒शा व॒शा स म॑भव दभव॒थ् सं ॅव॒शा व॒शा स म॑भवत् । \newline
\pagebreak
\markright{ TS 2.1.2.3  \hfill https://www.vedavms.in \hfill}

\section{ TS 2.1.2.3 }

\textbf{TS 2.1.2.3 } \newline
\textbf{Samhita Paata} \newline

सम॑भव॒त् ते दे॒वा अ॑ब्रुवन् देवप॒शुर्वा अ॒यꣳ सम॑भू॒त् कस्मा॑ इ॒ममा ल॑फ्स्यामह॒ इत्यथ॒ वै तर्ह्यल्पा॑ पृथि॒व्यासी॒-दजा॑ता॒ ओष॑धय॒स्तामविं॑ ॅव॒शामा॑दि॒त्येभ्यः॒ कामा॒याऽल॑भन्त॒ ततो॒ वा अप्र॑थत पृथि॒व्य-जा॑य॒न्तौष॑धयो॒ यः का॒मये॑त॒ प्रथे॑य प॒शुभिः॒ प्र प्र॒जया॑ जाये॒येति॒ स ए॒तामविं॑ ॅव॒शामा॑दि॒त्येभ्यः॒ कामा॒या - [  ] \newline

\textbf{Pada Paata} \newline

समिति॑ । अ॒भ॒व॒त् । ते । दे॒वाः । अ॒ब्रु॒व॒न्न् । दे॒व॒प॒शुरिति॑ देव-प॒शुः । वै । अ॒यम् । समिति॑ । अ॒भू॒त् । कस्मै᳚ । इ॒मम् । एति॑ । ल॒फ्स्या॒म॒हे॒ । इति॑ । अथ॑ । वै । तर्.हि॑ । अल्पा᳚ । पृ॒थि॒वी । आसी᳚त् । अजा॑ताः । ओष॑धयः । ताम् । अवि᳚म् । व॒शाम् । आ॒दि॒त्येभ्यः॑ । कामा॑य । एति॑ । अ॒ल॒भ॒न्त॒ । ततः॑ । वै । अप्र॑थत । पृ॒थि॒वी । अजा॑यन्त । ओष॑धयः । यः । का॒मये॑त । प्रथे॑य । प॒शुभि॒रिति॑ प॒शु - भिः॒ । प्रेति॑ । प्र॒जयेति॑ प्र - जया᳚ । जा॒ये॒य॒ । इति॑ । सः । ए॒ताम् । अवि᳚म् । व॒शाम् । आ॒दि॒त्येभ्यः॑ । कामा॑य ।  \newline


\textbf{Krama Paata} \newline

सम॑भवत् । अ॒भ॒व॒त् ते । ते दे॒वाः । दे॒वा अ॑ब्रुवन्न् । अ॒ब्रु॒व॒न् दे॒व॒प॒शुः । दे॒व॒प॒शुर्,वै । दे॒व॒प॒शुरिति॑ देव - प॒शुः । वा अ॒यम् । अ॒यꣳ सम् । सम॑भूत् । अ॒भू॒त्,कस्मै᳚ । कस्मा॑ इ॒मम् । इ॒ममा । आ ल॑फ्स्यामहे । ल॒फ्स्या॒म॒ह॒ इति॑ । इत्यथ॑ । अथ॒ वै । वै तर्.हि॑ । तर्.ह्यल्पा᳚ । अल्पा॑ पृथि॒वी । पृ॒थि॒व्यासी᳚त् । आसी॒दजा॑ताः । अजा॑ता॒ ओष॑धयः । ओष॑धय॒स्ताम् । तामवि᳚म् । अविं॑ ॅव॒शाम् । व॒शामा॑दि॒त्येभ्यः॑ । आ॒दि॒त्येभ्यः॒ कामा॑य । कामा॒या । आऽल॑भन्त । अ॒ल॒भ॒न्त॒ ततः॑ । ततो॒ वै । वा अप्र॑थत । अप्र॑थत पृथि॒वी । पृ॒थि॒व्यजा॑यन्त । अजा॑य॒न्तौष॑धयः । ओष॑धयो॒ यः । यः का॒मये॑त । का॒मये॑त॒ प्रथे॑य । प्रथे॑य प॒शुभिः॑ । प॒शुभिः॒ प्र । प॒शुभि॒रिति॑ प॒शु - भिः॒ । प्र प्र॒जया᳚ । प्र॒जया॑ जायेय । प्र॒जयेति॑ प्र - जया᳚ । जा॒ये॒येति॑ । इति॒ सः । स ए॒ताम् । ए॒तामवि᳚म् । अविं॑ ॅव॒शाम् । व॒शामा॑दि॒त्येभ्यः॑ । आ॒दि॒त्येभ्यः॒ कामा॑य । कामा॒या \newline

\textbf{Jatai Paata} \newline

1. स म॑भव दभव॒थ् सꣳ स म॑भवत् । \newline
2. अ॒भ॒व॒त् ते ते॑ ऽभव दभव॒त् ते । \newline
3. ते दे॒वा दे॒वा स्ते ते दे॒वाः । \newline
4. दे॒वा अ॑ब्रुवन् नब्रुवन् दे॒वा दे॒वा अ॑ब्रुवन्न् । \newline
5. अ॒ब्रु॒व॒न् दे॒व॒प॒शुर् दे॑वप॒शु र॑ब्रुवन् नब्रुवन् देवप॒शुः । \newline
6. दे॒व॒प॒शुर् वै वै दे॑वप॒शुर् दे॑वप॒शुर् वै । \newline
7. दे॒व॒प॒शुरिति॑ देव - प॒शुः । \newline
8. वा अ॒य म॒यं ॅवै वा अ॒यम् । \newline
9. अ॒यꣳ सꣳ स म॒य म॒यꣳ सम् । \newline
10. स म॑भू दभू॒थ् सꣳ स म॑भूत् । \newline
11. अ॒भू॒त् कस्मै॒ कस्मा॑ अभू दभू॒त् कस्मै᳚ । \newline
12. कस्मा॑ इ॒म मि॒मम् कस्मै॒ कस्मा॑ इ॒मम् । \newline
13. इ॒म मेम मि॒म मा । \newline
14. आ ल॑फ्स्यामहे लफ्स्यामह॒ आ ल॑फ्स्यामहे । \newline
15. ल॒फ्स्या॒म॒ह॒ इतीति॑ लफ्स्यामहे लफ्स्यामह॒ इति॑ । \newline
16. इत्यथाथे तीत्यथ॑ । \newline
17. अथ॒ वै वा अथाथ॒ वै । \newline
18. वै तर्.हि॒ तर्.हि॒ वै वै तर्.हि॑ । \newline
19. तर्ह्यल्पा ऽल्पा॒ तर्.हि॒ तर्ह्यल्पा᳚ । \newline
20. अल्पा॑ पृथि॒वी पृ॑थि॒व्यल्पा ऽल्पा॑ पृथि॒वी । \newline
21. पृ॒थि॒व्यासी॒ दासी᳚त् पृथि॒वी पृ॑थि॒व्यासी᳚त् । \newline
22. आसी॒ दजा॑ता॒ अजा॑ता॒ आसी॒ दासी॒ दजा॑ताः । \newline
23. अजा॑ता॒ ओष॑धय॒ ओष॑ध॒यो ऽजा॑ता॒ अजा॑ता॒ ओष॑धयः । \newline
24. ओष॑धय॒ स्ताम् ता मोष॑धय॒ ओष॑धय॒ स्ताम् । \newline
25. ता मवि॒ मवि॒म् ताम् ता मवि᳚म् । \newline
26. अविं॑ ॅव॒शां ॅव॒शा मवि॒ मविं॑ ॅव॒शाम् । \newline
27. व॒शा मा॑दि॒त्येभ्य॑ आदि॒त्येभ्यो॑ व॒शां ॅव॒शा मा॑दि॒त्येभ्यः॑ । \newline
28. आ॒दि॒त्येभ्यः॒ कामा॑य॒ कामा॑या दि॒त्येभ्य॑ आदि॒त्येभ्यः॒ कामा॑य । \newline
29. कामा॒या कामा॑य॒ कामा॒या । \newline
30. आ ऽल॑भन्ता लभ॒न्ता ऽल॑भन्त । \newline
31. अ॒ल॒भ॒न्त॒ तत॒ स्ततो॑ ऽलभन्ता लभन्त॒ ततः॑ । \newline
32. ततो॒ वै वै तत॒ स्ततो॒ वै । \newline
33. वा अप्र॑थ॒ता प्र॑थत॒ वै वा अप्र॑थत । \newline
34. अप्र॑थत पृथि॒वी पृ॑थि॒ व्यप्र॑थ॒ता प्र॑थत पृथि॒वी । \newline
35. पृ॒थि॒ व्यजा॑य॒न्ता जा॑यन्त पृथि॒वी पृ॑थि॒ व्यजा॑यन्त । \newline
36. अजा॑य॒ न्तौष॑धय॒ ओष॑ध॒यो ऽजा॑य॒न्ता जा॑य॒ न्तौष॑धयः । \newline
37. ओष॑धयो॒ यो य ओष॑धय॒ ओष॑धयो॒ यः । \newline
38. यः का॒मये॑त का॒मये॑त॒ यो यः का॒मये॑त । \newline
39. का॒मये॑त॒ प्रथे॑य॒ प्रथे॑य का॒मये॑त का॒मये॑त॒ प्रथे॑य । \newline
40. प्रथे॑य प॒शुभिः॑ प॒शुभिः॒ प्रथे॑य॒ प्रथे॑य प॒शुभिः॑ । \newline
41. प॒शुभिः॒ प्र प्र प॒शुभिः॑ प॒शुभिः॒ प्र । \newline
42. प॒शुभि॒रिति॑ प॒शु - भिः॒ । \newline
43. प्र प्र॒जया᳚ प्र॒जया॒ प्र प्र प्र॒जया᳚ । \newline
44. प्र॒जया॑ जायेय जायेय प्र॒जया᳚ प्र॒जया॑ जायेय । \newline
45. प्र॒जयेति॑ प्र - जया᳚ । \newline
46. जा॒ये॒ये तीति॑ जायेय जाये॒ये ति॑ । \newline
47. इति॒ स स इतीति॒ सः । \newline
48. स ए॒ता मे॒ताꣳ स स ए॒ताम् । \newline
49. ए॒ता मवि॒ मवि॑ मे॒ता मे॒ता मवि᳚म् । \newline
50. अविं॑ ॅव॒शां ॅव॒शा मवि॒ मविं॑ ॅव॒शाम् । \newline
51. व॒शा मा॑दि॒त्येभ्य॑ आदि॒त्येभ्यो॑ व॒शां ॅव॒शा मा॑दि॒त्येभ्यः॑ । \newline
52. आ॒दि॒त्येभ्यः॒ कामा॑य॒ कामा॑यादि॒त्येभ्य॑ आदि॒त्येभ्यः॒ कामा॑य । \newline
53. कामा॒या कामा॑य॒ कामा॒या । \newline

\textbf{Ghana Paata } \newline

1. स म॑भव दभव॒थ् सꣳ स म॑भव॒त् ते ते॑ ऽभव॒थ् सꣳ स म॑भव॒त् ते । \newline
2. अ॒भ॒व॒त् ते ते॑ ऽभव दभव॒त् ते दे॒वा दे॒वा स्ते॑ ऽभव दभव॒त् ते दे॒वाः । \newline
3. ते दे॒वा दे॒वा स्ते ते दे॒वा अ॑ब्रुवन् नब्रुवन् दे॒वा स्ते ते दे॒वा अ॑ब्रुवन्न् । \newline
4. दे॒वा अ॑ब्रुवन् नब्रुवन् दे॒वा दे॒वा अ॑ब्रुवन् देवप॒शुर् दे॑वप॒शु र॑ब्रुवन् दे॒वा दे॒वा अ॑ब्रुवन् देवप॒शुः । \newline
5. अ॒ब्रु॒व॒न् दे॒व॒प॒शुर् दे॑वप॒शु र॑ब्रुवन् नब्रुवन् देवप॒शुर् वै वै दे॑वप॒शु र॑ब्रुवन् नब्रुवन् देवप॒शुर् वै । \newline
6. दे॒व॒प॒शुर् वै वै दे॑वप॒शुर् दे॑वप॒शुर् वा अ॒य म॒यं ॅवै दे॑वप॒शुर् दे॑वप॒शुर् वा अ॒यम् । \newline
7. दे॒व॒प॒शुरिति॑ देव - प॒शुः । \newline
8. वा अ॒य म॒यं ॅवै वा अ॒यꣳ सꣳ स म॒यं ॅवै वा अ॒यꣳ सम् । \newline
9. अ॒यꣳ सꣳ स म॒य म॒यꣳ स म॑भू दभू॒थ् स म॒य म॒यꣳ स म॑भूत् । \newline
10. स म॑भू दभू॒थ् सꣳ स म॑भू॒त् कस्मै॒ कस्मा॑ अभू॒थ् सꣳ स म॑भू॒त् कस्मै᳚ । \newline
11. अ॒भू॒त् कस्मै॒ कस्मा॑ अभू दभू॒त् कस्मा॑ इ॒म मि॒मम् कस्मा॑ अभू दभू॒त् कस्मा॑ इ॒मम् । \newline
12. कस्मा॑ इ॒म मि॒मम् कस्मै॒ कस्मा॑ इ॒म मेमम् कस्मै॒ कस्मा॑ इ॒म मा । \newline
13. इ॒म मेम मि॒म मा ल॑फ्स्यामहे लफ्स्यामह॒ एम मि॒म मा ल॑फ्स्यामहे । \newline
14. आ ल॑फ्स्यामहे लफ्स्यामह॒ आ ल॑फ्स्यामह॒ इतीति॑ लफ्स्यामह॒ आ ल॑फ्स्यामह॒ इति॑ । \newline
15. ल॒फ्स्या॒म॒ह॒ इतीति॑ लफ्स्यामहे लफ्स्यामह॒ इत्यथाथे ति॑ लफ्स्यामहे लफ्स्यामह॒ इत्यथ॑ । \newline
16. इत्यथाथे तीत्यथ॒ वै वा अथे तीत्यथ॒ वै । \newline
17. अथ॒ वै वा अथाथ॒ वै तर्.हि॒ तर्.हि॒ वा अथाथ॒ वै तर्.हि॑ । \newline
18. वै तर्.हि॒ तर्.हि॒ वै वै तर्ह्यल्पा ऽल्पा॒ तर्.हि॒ वै वै तर्ह्यल्पा᳚ । \newline
19. तर्ह्यल्पा ऽल्पा॒ तर्.हि॒ तर्ह्यल्पा॑ पृथि॒वी पृ॑थि॒ व्यल्पा॒ तर्.हि॒ तर्ह्यल्पा॑ पृथि॒वी । \newline
20. अल्पा॑ पृथि॒वी पृ॑थि॒ व्यल्पा ऽल्पा॑ पृथि॒ व्यासी॒ दासी᳚त् पृथि॒ व्यल्पा ऽल्पा॑ पृथि॒ व्यासी᳚त् । \newline
21. पृ॒थि॒व्यासी॒ दासी᳚त् पृथि॒वी पृ॑थि॒व्यासी॒ दजा॑ता॒ अजा॑ता॒ आसी᳚त् पृथि॒वी पृ॑थि॒व्यासी॒ दजा॑ताः । \newline
22. आसी॒ दजा॑ता॒ अजा॑ता॒ आसी॒ दासी॒ दजा॑ता॒ ओष॑धय॒ ओष॑ध॒यो ऽजा॑ता॒ आसी॒ दासी॒ दजा॑ता॒ ओष॑धयः । \newline
23. अजा॑ता॒ ओष॑धय॒ ओष॑ध॒यो ऽजा॑ता॒ अजा॑ता॒ ओष॑धय॒ स्ताम् ता मोष॑ध॒यो ऽजा॑ता॒ अजा॑ता॒ ओष॑धय॒ स्ताम् । \newline
24. ओष॑धय॒ स्ताम् ता मोष॑धय॒ ओष॑धय॒ स्ता मवि॒ मवि॒म् ता मोष॑धय॒ ओष॑धय॒ स्ता मवि᳚म् । \newline
25. ता मवि॒ मवि॒म् ताम् ता मविं॑ ॅव॒शां ॅव॒शा मवि॒म् ताम् ता मविं॑ ॅव॒शाम् । \newline
26. अविं॑ ॅव॒शां ॅव॒शा मवि॒ मविं॑ ॅव॒शा मा॑दि॒त्येभ्य॑ आदि॒त्येभ्यो॑ व॒शा मवि॒ मविं॑ ॅव॒शा मा॑दि॒त्येभ्यः॑ । \newline
27. व॒शा मा॑दि॒त्येभ्य॑ आदि॒त्येभ्यो॑ व॒शां ॅव॒शा मा॑दि॒त्येभ्यः॒ कामा॑य॒ कामा॑या दि॒त्येभ्यो॑ व॒शां ॅव॒शा मा॑दि॒त्येभ्यः॒ कामा॑य । \newline
28. आ॒दि॒त्येभ्यः॒ कामा॑य॒ कामा॑या दि॒त्येभ्य॑ आदि॒त्येभ्यः॒ कामा॒या कामा॑या दि॒त्येभ्य॑ आदि॒त्येभ्यः॒ कामा॒या । \newline
29. कामा॒या कामा॑य॒ कामा॒या ऽल॑भन्ता लभ॒न्ता कामा॑य॒ कामा॒या ऽल॑भन्त । \newline
30. आ ऽल॑भन्ता लभ॒न्ता ऽल॑भन्त॒ तत॒ स्ततो॑ ऽलभ॒न्ता ऽल॑भन्त॒ ततः॑ । \newline
31. अ॒ल॒भ॒न्त॒ तत॒ स्ततो॑ ऽलभन्ता लभन्त॒ ततो॒ वै वै ततो॑ ऽलभन्ता लभन्त॒ ततो॒ वै । \newline
32. ततो॒ वै वै तत॒ स्ततो॒ वा अप्र॑थ॒ता प्र॑थत॒ वै तत॒ स्ततो॒ वा अप्र॑थत । \newline
33. वा अप्र॑थ॒ता प्र॑थत॒ वै वा अप्र॑थत पृथि॒वी पृ॑थि॒ व्यप्र॑थत॒ वै वा अप्र॑थत पृथि॒वी । \newline
34. अप्र॑थत पृथि॒वी पृ॑थि॒ व्यप्र॑थ॒ता प्र॑थत पृथि॒ व्यजा॑य॒न्ता जा॑यन्त पृथि॒ व्यप्र॑थ॒ता प्र॑थत पृथि॒ व्यजा॑यन्त । \newline
35. पृ॒थि॒ व्यजा॑य॒न्ता जा॑यन्त पृथि॒वी पृ॑थि॒ व्यजा॑य॒ न्तौष॑धय॒ ओष॑ध॒यो ऽजा॑यन्त पृथि॒वी पृ॑थि॒ व्यजा॑य॒ न्तौष॑धयः । \newline
36. अजा॑य॒न् तौष॑धय॒ ओष॑ध॒यो ऽजा॑य॒न्ता जा॑य॒ न्तौष॑धयो॒ यो य ओष॑ध॒यो ऽजा॑य॒न्ता जा॑य॒ न्तौष॑धयो॒ यः । \newline
37. ओष॑धयो॒ यो य ओष॑धय॒ ओष॑धयो॒ यः का॒मये॑त का॒मये॑त॒ य ओष॑धय॒ ओष॑धयो॒ यः का॒मये॑त । \newline
38. यः का॒मये॑त का॒मये॑त॒ यो यः का॒मये॑त॒ प्रथे॑य॒ प्रथे॑य का॒मये॑त॒ यो यः का॒मये॑त॒ प्रथे॑य । \newline
39. का॒मये॑त॒ प्रथे॑य॒ प्रथे॑य का॒मये॑त का॒मये॑त॒ प्रथे॑य प॒शुभिः॑ प॒शुभिः॒ प्रथे॑य का॒मये॑त का॒मये॑त॒ प्रथे॑य प॒शुभिः॑ । \newline
40. प्रथे॑य प॒शुभिः॑ प॒शुभिः॒ प्रथे॑य॒ प्रथे॑य प॒शुभिः॒ प्र प्र प॒शुभिः॒ प्रथे॑य॒ प्रथे॑य प॒शुभिः॒ प्र । \newline
41. प॒शुभिः॒ प्र प्र प॒शुभिः॑ प॒शुभिः॒ प्र प्र॒जया᳚ प्र॒जया॒ प्र प॒शुभिः॑ प॒शुभिः॒ प्र प्र॒जया᳚ । \newline
42. प॒शुभि॒रिति॑ प॒शु - भिः॒ । \newline
43. प्र प्र॒जया᳚ प्र॒जया॒ प्र प्र प्र॒जया॑ जायेय जायेय प्र॒जया॒ प्र प्र प्र॒जया॑ जायेय । \newline
44. प्र॒जया॑ जायेय जायेय प्र॒जया᳚ प्र॒जया॑ जाये॒ये तीति॑ जायेय प्र॒जया᳚ प्र॒जया॑ जाये॒ये ति॑ । \newline
45. प्र॒जयेति॑ प्र - जया᳚ । \newline
46. जा॒ये॒ये तीति॑ जायेय जाये॒ये ति॒ स स इति॑ जायेय जाये॒ये ति॒ सः । \newline
47. इति॒ स स इतीति॒ स ए॒ता मे॒ताꣳ स इतीति॒ स ए॒ताम् । \newline
48. स ए॒ता मे॒ताꣳ स स ए॒ता मवि॒ मवि॑ मे॒ताꣳ स स ए॒ता मवि᳚म् । \newline
49. ए॒ता मवि॒ मवि॑ मे॒ता मे॒ता मविं॑ ॅव॒शां ॅव॒शा मवि॑ मे॒ता मे॒ता मविं॑ ॅव॒शाम् । \newline
50. अविं॑ ॅव॒शां ॅव॒शा मवि॒ मविं॑ ॅव॒शा मा॑दि॒त्येभ्य॑ आदि॒त्येभ्यो॑ व॒शा मवि॒ मविं॑ ॅव॒शा मा॑दि॒त्येभ्यः॑ । \newline
51. व॒शा मा॑दि॒त्येभ्य॑ आदि॒त्येभ्यो॑ व॒शां ॅव॒शा मा॑दि॒त्येभ्यः॒ कामा॑य॒ कामा॑या दि॒त्येभ्यो॑ व॒शां ॅव॒शा मा॑दि॒त्येभ्यः॒ कामा॑य । \newline
52. आ॒दि॒त्येभ्यः॒ कामा॑य॒ कामा॑या दि॒त्येभ्य॑ आदि॒त्येभ्यः॒ कामा॒या कामा॑या दि॒त्येभ्य॑ आदि॒त्येभ्यः॒ कामा॒या । \newline
53. कामा॒या कामा॑य॒ कामा॒या ल॑भेत लभे॒ता कामा॑य॒ कामा॒या ल॑भेत । \newline
\pagebreak
\markright{ TS 2.1.2.4  \hfill https://www.vedavms.in \hfill}

\section{ TS 2.1.2.4 }

\textbf{TS 2.1.2.4 } \newline
\textbf{Samhita Paata} \newline

ऽऽ*ल॑भेता ऽऽ*दि॒त्याने॒व कामꣳ॒॒ स्वेन॑ भाग॒धेये॒नोप॑ धावति॒ त ए॒वैनं॑ प्र॒थय॑न्ति प॒शुभिः॒ प्र प्र॒जया॑ जनयन्त्य॒-सावा॑दि॒त्यो न व्य॑रोचत॒ तस्मै॑ दे॒वाः प्राय॑श्चित्तिमैच्छ॒न् तस्मा॑ ए॒ता म॒ल्॒.हा आल॑भन्ताऽऽ*ग्ने॒यीं कृ॑ष्णग्री॒वीꣳ सꣳ॑हि॒तामै॒न्द्रीꣳ श्वे॒तां बा॑र्.हस्प॒त्यां ताभि॑रे॒वास्मि॒न् रुच॑मदधु॒र्यो ब्र॑ह्मवर्च॒स-का॑मः॒ स्यात् तस्मा॑ ए॒ता म॒॒ल्॒.हा आ ल॑भेता - [  ] \newline

\textbf{Pada Paata} \newline

एति॑ । ल॒भे॒त॒ । आ॒दि॒त्यान् । ए॒व । काम᳚म् । स्वेन॑ । भा॒ग॒धेये॒नेति॑ भाग - धेये॑न । उपेति॑ । धा॒व॒ति॒ । ते । ए॒व । ए॒न॒म् । प्र॒थय॑न्ति । प॒शुभि॒रिति॑ प॒शु - भिः॒ । प्रेति॑ । प्र॒जयेति॑ प्र - जया᳚ । ज॒न॒य॒न्ति॒ । अ॒सौ । आ॒दि॒त्यः । न । वीति॑ । अ॒रो॒च॒त॒ । तस्मै᳚ । दे॒वाः । प्राय॑श्चित्तिम् । ऐ॒च्छ॒न्न् । तस्मै᳚ । ए॒ताः । म॒ल्॒.हाः । एति॑ । अ॒ल॒भ॒न्त॒ । आ॒ग्ने॒यीम् । कृ॒ष्ण॒ग्री॒वीमिति॑ कृष्ण - ग्री॒वीम् । सꣳ॒॒हि॒तामिति॑ सं - हि॒ताम् । ऐ॒न्द्रीम् । श्वे॒ताम् । बा॒र्.॒ह॒स्प॒त्याम् । ताभिः॑ । ए॒व । अ॒स्मि॒न्न् । रुच᳚म् । अ॒द॒धुः॒ । यः । ब्र॒ह्म॒व॒र्च॒सका॑म॒ इति॑ ब्रह्मवर्च॒स - का॒मः॒ । स्यात् । तस्मै᳚ । ए॒ताः । म॒ल्॒.हाः । एति॑ । ल॒भे॒त॒ ।  \newline


\textbf{Krama Paata} \newline

आ ल॑भेत । ल॒भे॒ता॒दि॒त्यान् । आ॒दि॒त्याने॒व । ए॒व काम᳚म् । कामꣳ॒॒ स्वेन॑ । स्वेन॑ भाग॒धेये॑न । भा॒ग॒धेये॒नोप॑ । भा॒ग॒धेये॒नेति॑ भाग - धेये॑न । उप॑ धावति । धा॒व॒ति॒ ते । त ए॒व । ए॒वैन᳚म् । ए॒नं॒ प्र॒थय॑न्ति । प्र॒थय॑न्ति प॒शुभिः॑ । प॒शुभिः॒ प्र । प॒शुभि॒रिति॑ प॒शु - भिः॒ । प्र प्र॒जया᳚ । प्र॒जया॑ जनयन्ति । प्र॒जयेति॑ प्र - जया᳚ । ज॒न॒य॒न्त्य॒सौ । अ॒सावा॑दि॒त्यः । आ॒दि॒त्यो न । न वि । व्य॑रोचत । अ॒रो॒च॒त॒ तस्मै᳚ । तस्मै॑ दे॒वाः । दे॒वाः प्राय॑श्चित्तिम् । प्राय॑श्चित्तिमैच्छन्न् । ऐ॒च्छ॒न् तस्मै᳚ । तस्मा॑ ए॒ताः । ए॒ता म॒ल्॒.हाः । म॒ल्॒.हा आ । आ ऽल॑भन्त । अ॒ल॒भ॒न्ता॒ग्ने॒यीम् । आ॒ग्ने॒यीम् कृ॑ष्णग्री॒वीम् । कृ॒ष्ण॒ग्री॒वीꣳ सꣳ॑हि॒ताम् । कृ॒ष्ण॒ग्री॒वीमिति॑ कृष्ण - ग्री॒वीम् । सꣳ॒॒हि॒तामै॒न्द्रीम् । सꣳ॒॒हि॒तामिति॑ सं - हि॒ताम् । ऐ॒न्द्रीꣳ श्वे॒ताम् । श्वे॒ताम् बा॑र्.हस्प॒त्याम् । बा॒र्॒.ह॒स्प॒त्याम् ताभिः॑ । ताभि॑रे॒व । ए॒वास्मिन्न्॑ । अ॒स्मि॒न् रुच᳚म् । रुच॑मदधुः । अ॒द॒धु॒र् यः । यो ब्र॑ह्मवर्च॒सका॑मः । ब्र॒ह्म॒व॒र्च॒सका॑मः॒ स्यात् । ब्र॒ह्म॒व॒र्च॒सका॑म॒ इति॑ ब्रह्मवर्च॒स - का॒मः॒ । स्यात् तस्मै᳚ । तस्मा॑ ए॒ताः । ए॒ता म॒ल्॒.हाः । म॒ल्॒.हा आ । आ ल॑भेत । ल॒भे॒ता॒ग्ने॒यीम् \newline

\textbf{Jatai Paata} \newline

1. आ ल॑भेत लभे॒ता ल॑भेत । \newline
2. ल॒भे॒ता॒दि॒त्या ना॑दि॒त्यान् ॅल॑भेत लभेतादि॒त्यान् । \newline
3. आ॒दि॒त्या ने॒वैवादि॒त्या ना॑दि॒त्या ने॒व । \newline
4. ए॒व काम॒म् काम॑ मे॒वैव काम᳚म् । \newline
5. कामꣳ॒॒ स्वेन॒ स्वेन॒ काम॒म् कामꣳ॒॒ स्वेन॑ । \newline
6. स्वेन॑ भाग॒धेये॑न भाग॒धेये॑न॒ स्वेन॒ स्वेन॑ भाग॒धेये॑न । \newline
7. भा॒ग॒धेये॒नोपोप॑ भाग॒धेये॑न भाग॒धेये॒नोप॑ । \newline
8. भा॒ग॒धेये॒नेति॑ भाग - धेये॑न । \newline
9. उप॑ धावति धाव॒ त्युपोप॑ धावति । \newline
10. धा॒व॒ति॒ ते ते धा॑वति धावति॒ ते । \newline
11. त ए॒वैव ते त ए॒व । \newline
12. ए॒वैन॑ मेन मे॒वै वैन᳚म् । \newline
13. ए॒न॒म् प्र॒थय॑न्ति प्र॒थय॑न्त्येन मेनम् प्र॒थय॑न्ति । \newline
14. प्र॒थय॑न्ति प॒शुभिः॑ प॒शुभिः॑ प्र॒थय॑न्ति प्र॒थय॑न्ति प॒शुभिः॑ । \newline
15. प॒शुभिः॒ प्र प्र प॒शुभिः॑ प॒शुभिः॒ प्र । \newline
16. प॒शुभि॒रिति॑ प॒शु - भिः॒ । \newline
17. प्र प्र॒जया᳚ प्र॒जया॒ प्र प्र प्र॒जया᳚ । \newline
18. प्र॒जया॑ जनयन्ति जनयन्ति प्र॒जया᳚ प्र॒जया॑ जनयन्ति । \newline
19. प्र॒जयेति॑ प्र - जया᳚ । \newline
20. ज॒न॒य॒ न्त्य॒सा व॒सौ ज॑नयन्ति जनय न्त्य॒सौ । \newline
21. अ॒सा वा॑दि॒त्य आ॑दि॒त्यो॑ ऽसा व॒सा वा॑दि॒त्यः । \newline
22. आ॒दि॒त्यो न नादि॒त्य आ॑दि॒त्यो न । \newline
23. न वि वि न न वि । \newline
24. व्य॑रोचता रोचत॒ वि व्य॑रोचत । \newline
25. अ॒रो॒च॒त॒ तस्मै॒ तस्मा॑ अरोचता रोचत॒ तस्मै᳚ । \newline
26. तस्मै॑ दे॒वा दे॒वा स्तस्मै॒ तस्मै॑ दे॒वाः । \newline
27. दे॒वाः प्राय॑श्चित्ति॒म् प्राय॑श्चित्तिम् दे॒वा दे॒वाः प्राय॑श्चित्तिम् । \newline
28. प्राय॑श्चित्ति मैच्छन् नैच्छ॒न् प्राय॑श्चित्ति॒म् प्राय॑श्चित्ति मैच्छन्न् । \newline
29. ऐ॒च्छ॒न् तस्मै॒ तस्मा॑ ऐच्छन् नैच्छ॒न् तस्मै᳚ । \newline
30. तस्मा॑ ए॒ता ए॒ता स्तस्मै॒ तस्मा॑ ए॒ताः । \newline
31. ए॒ता म॒ल्॒.हा म॒ल्॒.हा ए॒ता ए॒ता म॒ल्॒.हाः । \newline
32. म॒ल्॒.हा आ म॒ल्॒.हा म॒ल्॒.हा आ । \newline
33. आ ऽल॑भन्ता लभ॒न्ता ऽल॑भन्त । \newline
34. अ॒ल॒भ॒ न्ता॒ग्ने॒यी मा᳚ग्ने॒यी म॑लभन्ता लभन्ता ग्ने॒यीम् । \newline
35. आ॒ग्ने॒यीम् कृ॑ष्णग्री॒वीम् कृ॑ष्णग्री॒वी मा᳚ग्ने॒यी मा᳚ग्ने॒यीम् कृ॑ष्णग्री॒वीम् । \newline
36. कृ॒ष्ण॒ग्री॒वीꣳ सꣳ॑हि॒ताꣳ सꣳ॑हि॒ताम् कृ॑ष्णग्री॒वीम् कृ॑ष्णग्री॒वीꣳ सꣳ॑हि॒ताम् । \newline
37. कृ॒ष्ण॒ग्री॒वीमिति॑ कृष्ण - ग्री॒वीम् । \newline
38. सꣳ॒॒हि॒ता मै॒न्द्री मै॒न्द्रीꣳ सꣳ॑हि॒ताꣳ सꣳ॑हि॒ता मै॒न्द्रीम् । \newline
39. सꣳ॒॒हि॒तामिति॑ सं - हि॒ताम् । \newline
40. ऐ॒न्द्रीꣳ श्वे॒ताꣳ श्वे॒ता मै॒न्द्री मै॒न्द्रीꣳ श्वे॒ताम् । \newline
41. श्वे॒ताम् बा॑र्.हस्प॒त्याम् बा॑र्.हस्प॒त्याꣳ श्वे॒ताꣳ श्वे॒ताम् बा॑र्.हस्प॒त्याम् । \newline
42. बा॒र्॒.ह॒स्प॒त्याम् ताभि॒ स्ताभि॑र् बार्.हस्प॒त्याम् बा॑र्.हस्प॒त्याम् ताभिः॑ । \newline
43. ताभि॑ रे॒वैव ताभि॒ स्ताभि॑ रे॒व । \newline
44. ए॒वास्मि॑न् नस्मिन् ने॒वैवास्मिन्न्॑ । \newline
45. अ॒स्मि॒न् रुचꣳ॒॒ रुच॑ मस्मिन् नस्मि॒न् रुच᳚म् । \newline
46. रुच॑ मदधु रदधू॒ रुचꣳ॒॒ रुच॑ मदधुः । \newline
47. अ॒द॒धु॒र् यो यो॑ ऽदधु रदधु॒र् यः । \newline
48. यो ब्र॑ह्मवर्च॒सका॑मो ब्रह्मवर्च॒सका॑मो॒ यो यो ब्र॑ह्मवर्च॒सका॑मः । \newline
49. ब्र॒ह्म॒व॒र्च॒सका॑मः॒ स्याथ् स्याद् ब्र॑ह्मवर्च॒सका॑मो ब्रह्मवर्च॒सका॑मः॒ स्यात् । \newline
50. ब्र॒ह्म॒व॒र्च॒सका॑म॒ इति॑ ब्रह्मवर्च॒स - का॒मः॒ । \newline
51. स्यात् तस्मै॒ तस्मै॒ स्याथ् स्यात् तस्मै᳚ । \newline
52. तस्मा॑ ए॒ता ए॒ता स्तस्मै॒ तस्मा॑ ए॒ताः । \newline
53. ए॒ता म॒ल्॒.हा म॒ल्॒.हा ए॒ता ए॒ता म॒ल्॒.हाः । \newline
54. म॒ल्॒.हा आ म॒ल्॒.हा म॒ल्॒.हा आ । \newline
55. आ ल॑भेत लभे॒ता ल॑भेत । \newline
56. ल॒भे॒ता॒ग्ने॒यी मा᳚ग्ने॒यीम् ॅल॑भेत लभेताग्ने॒यीम् । \newline

\textbf{Ghana Paata } \newline

1. आ ल॑भेत लभे॒ता ल॑भेतादि॒त्या ना॑दि॒त्यान् ॅल॑भे॒ता ल॑भेतादि॒त्यान् । \newline
2. ल॒भे॒ ता॒दि॒त्या ना॑दि॒त्यान् ॅल॑भेत लभे तादि॒त्या ने॒वै वादि॒त्यान् ॅल॑भेत लभे तादि॒त्या ने॒व । \newline
3. आ॒दि॒त्या ने॒वैवादि॒त्या ना॑दि॒त्या ने॒व काम॒म् काम॑ मे॒वादि॒त्या ना॑दि॒त्या ने॒व काम᳚म् । \newline
4. ए॒व काम॒म् काम॑ मे॒वैव कामꣳ॒॒ स्वेन॒ स्वेन॒ काम॑ मे॒वैव कामꣳ॒॒ स्वेन॑ । \newline
5. कामꣳ॒॒ स्वेन॒ स्वेन॒ काम॒म् कामꣳ॒॒ स्वेन॑ भाग॒धेये॑न भाग॒धेये॑न॒ स्वेन॒ काम॒म् कामꣳ॒॒ स्वेन॑ भाग॒धेये॑न । \newline
6. स्वेन॑ भाग॒धेये॑न भाग॒धेये॑न॒ स्वेन॒ स्वेन॑ भाग॒धेये॒नो पोप॑ भाग॒धेये॑न॒ स्वेन॒ स्वेन॑ भाग॒धेये॒नोप॑ । \newline
7. भा॒ग॒धेये॒नो पोप॑ भाग॒धेये॑न भाग॒धेये॒नोप॑ धावति धाव॒त्युप॑ भाग॒धेये॑न भाग॒धेये॒नोप॑ धावति । \newline
8. भा॒ग॒धेये॒नेति॑ भाग - धेये॑न । \newline
9. उप॑ धावति धाव॒ त्युपोप॑ धावति॒ ते ते धा॑व॒ त्युपोप॑ धावति॒ ते । \newline
10. धा॒व॒ति॒ ते ते धा॑वति धावति॒ त ए॒वैव ते धा॑वति धावति॒ त ए॒व । \newline
11. त ए॒वैव ते त ए॒वैन॑ मेन मे॒व ते त ए॒वैन᳚म् । \newline
12. ए॒वैन॑ मेन मे॒वै वैन॑म् प्र॒थय॑न्ति प्र॒थय॑न्त्येन मे॒वै वैन॑म् प्र॒थय॑न्ति । \newline
13. ए॒न॒म् प्र॒थय॑न्ति प्र॒थय॑न्त्येन मेनम् प्र॒थय॑न्ति प॒शुभिः॑ प॒शुभिः॑ प्र॒थय॑न्त्येन मेनम् प्र॒थय॑न्ति प॒शुभिः॑ । \newline
14. प्र॒थय॑न्ति प॒शुभिः॑ प॒शुभिः॑ प्र॒थय॑न्ति प्र॒थय॑न्ति प॒शुभिः॒ प्र प्र प॒शुभिः॑ प्र॒थय॑न्ति प्र॒थय॑न्ति प॒शुभिः॒ प्र । \newline
15. प॒शुभिः॒ प्र प्र प॒शुभिः॑ प॒शुभिः॒ प्र प्र॒जया᳚ प्र॒जया॒ प्र प॒शुभिः॑ प॒शुभिः॒ प्र प्र॒जया᳚ । \newline
16. प॒शुभि॒रिति॑ प॒शु - भिः॒ । \newline
17. प्र प्र॒जया᳚ प्र॒जया॒ प्र प्र प्र॒जया॑ जनयन्ति जनयन्ति प्र॒जया॒ प्र प्र प्र॒जया॑ जनयन्ति । \newline
18. प्र॒जया॑ जनयन्ति जनयन्ति प्र॒जया᳚ प्र॒जया॑ जनयन्त्य॒सा व॒सौ ज॑नयन्ति प्र॒जया᳚ प्र॒जया॑ जनयन्त्य॒सौ । \newline
19. प्र॒जयेति॑ प्र - जया᳚ । \newline
20. ज॒न॒य॒न्त्य॒सा व॒सौ ज॑नयन्ति जनयन्त्य॒सा वा॑दि॒त्य आ॑दि॒त्यो॑ ऽसौ ज॑नयन्ति जनयन्त्य॒सा वा॑दि॒त्यः । \newline
21. अ॒सा वा॑दि॒त्य आ॑दि॒त्यो॑ ऽसा व॒सा वा॑दि॒त्यो न नादि॒त्यो॑ ऽसा व॒सा वा॑दि॒त्यो न । \newline
22. आ॒दि॒त्यो न नादि॒त्य आ॑दि॒त्यो न वि वि नादि॒त्य आ॑दि॒त्यो न वि । \newline
23. न वि वि न न व्य॑रोचता रोचत॒ वि न न व्य॑रोचत । \newline
24. व्य॑रोचता रोचत॒ वि व्य॑रोचत॒ तस्मै॒ तस्मा॑ अरोचत॒ वि व्य॑रोचत॒ तस्मै᳚ । \newline
25. अ॒रो॒च॒त॒ तस्मै॒ तस्मा॑ अरोचता रोचत॒ तस्मै॑ दे॒वा दे॒वा स्तस्मा॑ अरोचता रोचत॒ तस्मै॑ दे॒वाः । \newline
26. तस्मै॑ दे॒वा दे॒वा स्तस्मै॒ तस्मै॑ दे॒वाः प्राय॑श्चित्ति॒म् प्राय॑श्चित्तिम् दे॒वा स्तस्मै॒ तस्मै॑ दे॒वाः प्राय॑श्चित्तिम् । \newline
27. दे॒वाः प्राय॑श्चित्ति॒म् प्राय॑श्चित्तिम् दे॒वा दे॒वाः प्राय॑श्चित्ति मैच्छन् नैच्छ॒न् प्राय॑श्चित्तिम् दे॒वा दे॒वाः प्राय॑श्चित्ति मैच्छन्न् । \newline
28. प्राय॑श्चित्ति मैच्छन् नैच्छ॒न् प्राय॑श्चित्ति॒म् प्राय॑श्चित्ति मैच्छ॒न् तस्मै॒ तस्मा॑ ऐच्छ॒न् प्राय॑श्चित्ति॒म् प्राय॑श्चित्ति मैच्छ॒न् तस्मै᳚ । \newline
29. ऐ॒च्छ॒न् तस्मै॒ तस्मा॑ ऐच्छन् नैच्छ॒न् तस्मा॑ ए॒ता ए॒ता स्तस्मा॑ ऐच्छन् नैच्छ॒न् तस्मा॑ ए॒ताः । \newline
30. तस्मा॑ ए॒ता ए॒ता स्तस्मै॒ तस्मा॑ ए॒ता म॒ल्॒.हा म॒ल्॒.हा ए॒ता स्तस्मै॒ तस्मा॑ ए॒ता म॒ल्॒.हाः । \newline
31. ए॒ता म॒ल्॒.हा म॒ल्॒.हा ए॒ता ए॒ता म॒ल्॒.हा आ म॒ल्॒.हा ए॒ता ए॒ता म॒ल्॒.हा आ । \newline
32. म॒ल्॒.हा आ म॒ल्॒.हा म॒ल्॒.हा आ ऽल॑भन्ता लभ॒न्ता म॒ल्॒.हा म॒ल्॒.हा आ ऽल॑भन्त । \newline
33. आ ऽल॑भन्ता लभ॒न्ता ऽल॑भन्ताग्ने॒यी मा᳚ग्ने॒यी म॑लभ॒न्ता ऽल॑भन्ताग्ने॒यीम् । \newline
34. अ॒ल॒भ॒न्ता॒ग्ने॒यी मा᳚ग्ने॒यी म॑लभन्ता लभन्ता ग्ने॒यीम् कृ॑ष्णग्री॒वीम् कृ॑ष्णग्री॒वी मा᳚ग्ने॒यी म॑लभन्ता लभन्ता ग्ने॒यीम् कृ॑ष्णग्री॒वीम् । \newline
35. आ॒ग्ने॒यीम् कृ॑ष्णग्री॒वीम् कृ॑ष्णग्री॒वी मा᳚ग्ने॒यी मा᳚ग्ने॒यीम् कृ॑ष्णग्री॒वीꣳ सꣳ॑हि॒ताꣳ सꣳ॑हि॒ताम् कृ॑ष्णग्री॒वी मा᳚ग्ने॒यी मा᳚ग्ने॒यीम् कृ॑ष्णग्री॒वीꣳ सꣳ॑हि॒ताम् । \newline
36. कृ॒ष्ण॒ग्री॒वीꣳ सꣳ॑हि॒ताꣳ सꣳ॑हि॒ताम् कृ॑ष्णग्री॒वीम् कृ॑ष्णग्री॒वीꣳ सꣳ॑हि॒ता मै॒न्द्री मै॒न्द्रीꣳ सꣳ॑हि॒ताम् कृ॑ष्णग्री॒वीम् कृ॑ष्णग्री॒वीꣳ सꣳ॑हि॒ता मै॒न्द्रीम् । \newline
37. कृ॒ष्ण॒ग्री॒वीमिति॑ कृष्ण - ग्री॒वीम् । \newline
38. सꣳ॒॒हि॒ता मै॒न्द्री मै॒न्द्रीꣳ सꣳ॑हि॒ताꣳ सꣳ॑हि॒ता मै॒न्द्रीꣳ श्वे॒ताꣳ श्वे॒ता मै॒न्द्रीꣳ सꣳ॑हि॒ताꣳ सꣳ॑हि॒ता मै॒न्द्रीꣳ श्वे॒ताम् । \newline
39. सꣳ॒॒हि॒तामिति॑ सं - हि॒ताम् । \newline
40. ऐ॒न्द्रीꣳ श्वे॒ताꣳ श्वे॒ता मै॒न्द्री मै॒न्द्रीꣳ श्वे॒ताम् बा॑र्.हस्प॒त्याम् बा॑र्.हस्प॒त्याꣳ श्वे॒ता मै॒न्द्री मै॒न्द्रीꣳ श्वे॒ताम् बा॑र्.हस्प॒त्याम् । \newline
41. श्वे॒ताम् बा॑र्.हस्प॒त्याम् बा॑र्.हस्प॒त्याꣳ श्वे॒ताꣳ श्वे॒ताम् बा॑र्.हस्प॒त्याम् ताभि॒ स्ताभि॑र् बार्.हस्प॒त्याꣳ श्वे॒ताꣳ श्वे॒ताम् बा॑र्.हस्प॒त्याम् ताभिः॑ । \newline
42. बा॒र्॒.ह॒स्प॒त्याम् ताभि॒ स्ताभि॑र् बार्.हस्प॒त्याम् बा॑र्.हस्प॒त्याम् ताभि॑ रे॒वैव ताभि॑र् बार्.हस्प॒त्याम् बा॑र्.हस्प॒त्याम् ताभि॑ रे॒व । \newline
43. ताभि॑ रे॒वैव ताभि॒ स्ताभि॑ रे॒वास्मि॑न् नस्मिन् ने॒व ताभि॒ स्ताभि॑ रे॒वास्मिन्न्॑ । \newline
44. ए॒वास्मि॑न् नस्मिन् ने॒वैवास्मि॒न् रुचꣳ॒॒ रुच॑ मस्मिन् ने॒वैवास्मि॒न् रुच᳚म् । \newline
45. अ॒स्मि॒न् रुचꣳ॒॒ रुच॑ मस्मिन् नस्मि॒न् रुच॑ मदधु रदधू॒ रुच॑ मस्मिन् नस्मि॒न् रुच॑ मदधुः । \newline
46. रुच॑ मदधु रदधू॒ रुचꣳ॒॒ रुच॑ मदधु॒र् यो यो॑ ऽदधू॒ रुचꣳ॒॒ रुच॑ मदधु॒र् यः । \newline
47. अ॒द॒धु॒र् यो यो॑ ऽदधुरदधु॒र् यो ब्र॑ह्मवर्च॒सका॑मो ब्रह्मवर्च॒सका॑मो॒ यो॑ ऽदधु रदधु॒र् यो ब्र॑ह्मवर्च॒सका॑मः । \newline
48. यो ब्र॑ह्मवर्च॒सका॑मो ब्रह्मवर्च॒सका॑मो॒ यो यो ब्र॑ह्मवर्च॒सका॑मः॒ स्याथ् स्याद् ब्र॑ह्मवर्च॒सका॑मो॒ यो यो ब्र॑ह्मवर्च॒सका॑मः॒ स्यात् । \newline
49. ब्र॒ह्म॒व॒र्च॒सका॑मः॒ स्याथ् स्याद् ब्र॑ह्मवर्च॒सका॑मो ब्रह्मवर्च॒सका॑मः॒ स्यात् तस्मै॒ तस्मै॒ स्याद् ब्र॑ह्मवर्च॒सका॑मो ब्रह्मवर्च॒सका॑मः॒ स्यात् तस्मै᳚ । \newline
50. ब्र॒ह्म॒व॒र्च॒सका॑म॒ इति॑ ब्रह्मवर्च॒स - का॒मः॒ । \newline
51. स्यात् तस्मै॒ तस्मै॒ स्याथ् स्यात् तस्मा॑ ए॒ता ए॒ता स्तस्मै॒ स्याथ् स्यात् तस्मा॑ ए॒ताः । \newline
52. तस्मा॑ ए॒ता ए॒ता स्तस्मै॒ तस्मा॑ ए॒ता म॒ल्॒.हा म॒ल्॒.हा ए॒ता स्तस्मै॒ तस्मा॑ ए॒ता म॒ल्॒.हाः । \newline
53. ए॒ता म॒ल्॒.हा म॒ल्॒.हा ए॒ता ए॒ता म॒ल्॒.हा आ म॒ल्॒.हा ए॒ता ए॒ता म॒ल्॒.हा आ । \newline
54. म॒ल्॒.हा आ म॒ल्॒.हा म॒ल्॒.हा आ ल॑भेत लभे॒ता म॒ल्॒.हा म॒ल्॒.हा आ ल॑भेत । \newline
55. आ ल॑भेत लभे॒ता ल॑भेताग्ने॒यी मा᳚ग्ने॒यीम् ॅल॑भे॒ता ल॑भेताग्ने॒यीम् । \newline
56. ल॒भे॒ता॒ग्ने॒यी मा᳚ग्ने॒यीम् ॅल॑भेत लभेताग्ने॒यीम् कृ॑ष्णग्री॒वीम् कृ॑ष्णग्री॒वी मा᳚ग्ने॒यीम् ॅल॑भेत लभेताग्ने॒यीम् कृ॑ष्णग्री॒वीम् । \newline
\pagebreak
\markright{ TS 2.1.2.5  \hfill https://www.vedavms.in \hfill}

\section{ TS 2.1.2.5 }

\textbf{TS 2.1.2.5 } \newline
\textbf{Samhita Paata} \newline

ऽऽ*ग्ने॒यीं कृ॑ष्णग्री॒वीꣳ सꣳ॑हि॒तामै॒न्द्रीꣳ श्वे॒तां बा॑र्.हस्प॒त्यामे॒ता ए॒व दे॒वताः॒ स्वेन॑ भाग॒धेये॒नोप॑ धावति॒ ता ए॒वास्मि॑न् ब्रह्मवर्च॒सं द॑धति ब्रह्मवर्च॒स्ये॑व भ॑वति व॒सन्ता᳚ प्रा॒तरा᳚ग्ने॒यीं कृ॑ष्ण ग्री॒वीमा ल॑भेत ग्री॒ष्मे म॒द्ध्यन्दि॑ने सꣳहि॒तामै॒न्द्रीꣳ श॒रद्य॑परा॒ह्णे श्वे॒तां बा॑र्.हस्प॒त्यां त्रीणि॒ वा आ॑दि॒त्यस्य॒ तेजाꣳ॑सि व॒सन्ता᳚ प्रा॒तर्ग्री॒ष्मे म॒द्ध्यन्दि॑ने श॒रद्य॑परा॒ह्णे याव॑न्त्ये॒व तेजाꣳ॑सि॒ तान्ये॒वा - [  ] \newline

\textbf{Pada Paata} \newline

आ॒ग्ने॒यीम् । कृ॒ष्ण॒ग्री॒वीमिति॑ कृष्ण - ग्री॒वीम् । सꣳ॒॒हि॒तामिति॑ सं - हि॒ताम् । ऐ॒न्द्रीम् । श्वे॒ताम् । बा॒र्.॒ह॒स्प॒त्याम् । ए॒ताः । ए॒व । दे॒वताः᳚ । स्वेन॑ । भा॒ग॒धेये॒नेति॑ भाग-धेये॑न । उपेति॑ । धा॒व॒ति॒ । ताः । ए॒व । अ॒स्मि॒न्न् । ब्र॒ह्म॒व॒र्च॒समिति॑ ब्रह्म - व॒र्च॒सम् । द॒ध॒ति॒ । ब्र॒ह्म॒व॒र्च॒सीति॑ ब्रह्म - व॒र्च॒सी । ए॒व । भ॒व॒ति॒ । व॒सन्ता᳚ । प्रा॒तः । आ॒ग्ने॒यीम् । कृ॒ष्ण॒ग्री॒वीमिति॑ कृष्ण - ग्री॒वीम् । एति॑ । ल॒भे॒त॒ । ग्री॒ष्मे । म॒द्ध्यन्दि॑ने । सꣳ॒॒हि॒तामिति॑ सं - हि॒ताम् । ऐ॒न्द्रीम् । श॒रदि॑ । अ॒प॒रा॒ह्ण इत्य॑पर - अ॒ह्ने । श्वे॒ताम् । बा॒र्.॒ह॒स्प॒त्याम् । त्रीणि॑ । वै । आ॒दि॒त्यस्य॑ । तेजाꣳ॑सि । व॒सन्ता᳚ । प्रा॒तः । ग्री॒ष्मे । म॒द्ध्यन्दि॑ने । श॒रदि॑ । अ॒प॒रा॒ह्ण इत्य॑पर - अ॒ह्ने । याव॑न्ति । ए॒व । तेजाꣳ॑सि । तानि॑ । ए॒व ।  \newline


\textbf{Krama Paata} \newline

आ॒ग्ने॒यीम् कृ॑ष्णग्री॒वीम् । कृ॒ष्ण॒ग्री॒वीꣳ सꣳ॑हि॒ताम् । कृ॒ष्ण॒ग्री॒वीमिति॑ कृष्ण - ग्री॒वीम् । सꣳ॒॒हि॒तामै॒न्द्रीम् । सꣳ॒॒हि॒तामिति॑ सं - हि॒ताम् । ऐ॒न्द्रीꣳ श्वे॒ताम् । श्वे॒ताम् बा॑र्.हस्प॒त्याम् । बा॒र्॒.ह॒स्प॒त्यामे॒ताः । ए॒ता ए॒व । ए॒व दे॒वताः᳚ । दे॒वताः॒ स्वेन॑ । स्वेन॑ भाग॒धेये॑न । भा॒ग॒धेये॒नोप॑ । भा॒ग॒धेये॒नेति॑ भाग - धेये॑न । उप॑ धावति । धा॒व॒ति॒ ताः । ता ए॒व । ए॒वास्मिन्न्॑ । अ॒स्मि॒न् ब्र॒ह्म॒व॒र्च॒सम् । ब्र॒ह्म॒व॒र्च॒सम् द॑धति । ब्र॒ह्म॒व॒र्च॒समिति॑ ब्रह्म - व॒र्च॒सम् । द॒ध॒ति॒ ब्र॒ह्म॒व॒र्च॒सी । ब्र॒ह्म॒व॒र्च॒स्ये॑व । ब्र॒ह्म॒व॒र्च॒सीति॑ ब्रह्म - व॒र्च॒सी । ए॒व भ॑वति । भ॒व॒ति॒ व॒सन्ता᳚ । व॒सन्ता᳚ प्रा॒तः । प्रा॒तरा᳚ग्ने॒यीम् । आ॒ग्ने॒यीम् कृ॑ष्णग्री॒वीम् । कृ॒ष्ण॒ग्री॒वीमा । कृ॒ष्ण॒ग्री॒वीमिति॑ कृष्ण - ग्री॒वीम् । आ ल॑भेत । ल॒भे॒त॒ ग्री॒ष्मे । ग्री॒ष्मे म॒द्ध्यन्दि॑ने । म॒द्ध्यन्दि॑ने सꣳहि॒ताम् । सꣳ॒॒हि॒तामै॒न्द्रीम् । सꣳ॒॒हि॒तामिति॑ सं - हि॒ताम् । ऐ॒न्द्रीꣳ श॒रदि॑ । श॒रद्य॑परा॒ह्णे । अ॒प॒रा॒ह्णे श्वे॒ताम् । अ॒प॒रा॒ह्ण इत्य॑पर - अ॒ह्ने । श्वे॒ताम् बा॑र्.हस्प॒त्याम् । बा॒र्॒.ह॒स्प॒त्याम् त्रीणि॑ । त्रीणि॒ वै । वा आ॑दि॒त्यस्य॑ । आ॒दि॒त्यस्य॒ तेजाꣳ॑सि । तेजाꣳ॑सि व॒सन्ता᳚ । व॒सन्ता᳚ प्रा॒तः । प्रा॒तर्,ग्री॒ष्मे । ग्री॒ष्मे म॒द्ध्यन्दि॑ने । म॒द्ध्यन्दि॑ने श॒रदि॑ । श॒रद्य॑परा॒ह्णे । अ॒प॒रा॒ह्णे याव॑न्ति । अ॒प॒रा॒ह्ण इत्य॑पर - अ॒ह्ने । याव॑न्त्ये॒व । ए॒व तेजाꣳ॑सि । तेजाꣳ॑सि॒ तानि॑ । तान्ये॒व । ए॒वाव॑ \newline

\textbf{Jatai Paata} \newline

1. आ॒ग्ने॒यीम् कृ॑ष्णग्री॒वीम् कृ॑ष्णग्री॒वी मा᳚ग्ने॒यी मा᳚ग्ने॒यीम् कृ॑ष्णग्री॒वीम् । \newline
2. कृ॒ष्ण॒ग्री॒वीꣳ सꣳ॑हि॒ताꣳ सꣳ॑हि॒ताम् कृ॑ष्णग्री॒वीम् कृ॑ष्णग्री॒वीꣳ सꣳ॑हि॒ताम् । \newline
3. कृ॒ष्ण॒ग्री॒वीमिति॑ कृष्ण - ग्री॒वीम् । \newline
4. सꣳ॒॒हि॒ता मै॒न्द्री मै॒न्द्रीꣳ सꣳ॑हि॒ताꣳ सꣳ॑हि॒ता मै॒न्द्रीम् । \newline
5. सꣳ॒॒हि॒तामिति॑ सं - हि॒ताम् । \newline
6. ऐ॒न्द्रीꣳ श्वे॒ताꣳ श्वे॒ता मै॒न्द्री मै॒न्द्रीꣳ श्वे॒ताम् । \newline
7. श्वे॒ताम् बा॑र्.हस्प॒त्याम् बा॑र्.हस्प॒त्याꣳ श्वे॒ताꣳ श्वे॒ताम् बा॑र्.हस्प॒त्याम् । \newline
8. बा॒र्॒.ह॒स्प॒त्या मे॒ता ए॒ता बा॑र्.हस्प॒त्याम् बा॑र्.हस्प॒त्या मे॒ताः । \newline
9. ए॒ता ए॒वैवैता ए॒ता ए॒व । \newline
10. ए॒व दे॒वता॑ दे॒वता॑ ए॒वैव दे॒वताः᳚ । \newline
11. दे॒वताः॒ स्वेन॒ स्वेन॑ दे॒वता॑ दे॒वताः॒ स्वेन॑ । \newline
12. स्वेन॑ भाग॒धेये॑न भाग॒धेये॑न॒ स्वेन॒ स्वेन॑ भाग॒धेये॑न । \newline
13. भा॒ग॒धेये॒नोपोप॑ भाग॒धेये॑न भाग॒धेये॒नोप॑ । \newline
14. भा॒ग॒धेये॒नेति॑ भाग - धेये॑न । \newline
15. उप॑ धावति धाव॒ त्युपोप॑ धावति । \newline
16. धा॒व॒ति॒ ता स्ता धा॑वति धावति॒ ताः । \newline
17. ता ए॒वैव ता स्ता ए॒व । \newline
18. ए॒वास्मि॑न् नस्मिन् ने॒वैवास्मिन्न्॑ । \newline
19. अ॒स्मि॒न् ब्र॒ह्म॒व॒र्च॒सम् ब्र॑ह्मवर्च॒स म॑स्मिन् नस्मिन् ब्रह्मवर्च॒सम् । \newline
20. ब्र॒ह्म॒व॒र्च॒सम् द॑धति दधति ब्रह्मवर्च॒सम् ब्र॑ह्मवर्च॒सम् द॑धति । \newline
21. ब्र॒ह्म॒व॒र्च॒समिति॑ ब्रह्म - व॒र्च॒सम् । \newline
22. द॒ध॒ति॒ ब्र॒ह्म॒व॒र्च॒सी ब्र॑ह्मवर्च॒सी द॑धति दधति ब्रह्मवर्च॒सी । \newline
23. ब्र॒ह्म॒व॒र्च॒ स्ये॑वैव ब्र॑ह्मवर्च॒सी ब्र॑ह्मवर्च॒ स्ये॑व । \newline
24. ब्र॒ह्म॒व॒र्च॒सीति॑ ब्रह्म - व॒र्च॒सी । \newline
25. ए॒व भ॑वति भव त्ये॒वैव भ॑वति । \newline
26. भ॒व॒ति॒ व॒सन्ता॑ व॒सन्ता॑ भवति भवति व॒सन्ता᳚ । \newline
27. व॒सन्ता᳚ प्रा॒तः प्रा॒तर् व॒सन्ता॑ व॒सन्ता᳚ प्रा॒तः । \newline
28. प्रा॒त रा᳚ग्ने॒यी मा᳚ग्ने॒यीम् प्रा॒तः प्रा॒त रा᳚ग्ने॒यीम् । \newline
29. आ॒ग्ने॒यीम् कृ॑ष्णग्री॒वीम् कृ॑ष्णग्री॒वी मा᳚ग्ने॒यी मा᳚ग्ने॒यीम् कृ॑ष्णग्री॒वीम् । \newline
30. कृ॒ष्ण॒ग्री॒वी मा कृ॑ष्णग्री॒वीम् कृ॑ष्णग्री॒वी मा । \newline
31. कृ॒ष्ण॒ग्री॒वीमिति॑ कृष्ण - ग्री॒वीम् । \newline
32. आ ल॑भेत लभे॒ता ल॑भेत । \newline
33. ल॒भे॒त॒ ग्री॒ष्मे ग्री॒ष्मे ल॑भेत लभेत ग्री॒ष्मे । \newline
34. ग्री॒ष्मे म॒द्ध्यन्दि॑ने म॒द्ध्यन्दि॑ने ग्री॒ष्मे ग्री॒ष्मे म॒द्ध्यन्दि॑ने । \newline
35. म॒द्ध्यन्दि॑ने सꣳहि॒ताꣳ सꣳ॑हि॒ताम् म॒द्ध्यन्दि॑ने म॒द्ध्यन्दि॑ने सꣳहि॒ताम् । \newline
36. सꣳ॒॒हि॒ता मै॒न्द्री मै॒न्द्रीꣳ सꣳ॑हि॒ताꣳ सꣳ॑हि॒ता मै॒न्द्रीम् । \newline
37. सꣳ॒॒हि॒तामिति॑ सं - हि॒ताम् । \newline
38. ऐ॒न्द्रीꣳ श॒रदि॑ श॒रद्यै॒न्द्री मै॒न्द्रीꣳ श॒रदि॑ । \newline
39. श॒रद्य॑परा॒ह्णे॑ ऽपरा॒ह्णे श॒रदि॑ श॒रद्य॑परा॒ह्णे । \newline
40. अ॒प॒रा॒ह्णे श्वे॒ताꣳ श्वे॒ता म॑परा॒ह्णे॑ ऽपरा॒ह्णे श्वे॒ताम् । \newline
41. अ॒प॒रा॒ह्ण इत्य॑ पर - अ॒ह्ने । \newline
42. श्वे॒ताम् बा॑र्.हस्प॒त्याम् बा॑र्.हस्प॒त्याꣳ श्वे॒ताꣳ श्वे॒ताम् बा॑र्.हस्प॒त्याम् । \newline
43. बा॒र्॒.ह॒स्प॒त्याम् त्रीणि॒ त्रीणि॑ बार्.हस्प॒त्याम् बा॑र्.हस्प॒त्याम् त्रीणि॑ । \newline
44. त्रीणि॒ वै वै त्रीणि॒ त्रीणि॒ वै । \newline
45. वा आ॑दि॒त्यस्या॑ दि॒त्यस्य॒ वै वा आ॑दि॒त्यस्य॑ । \newline
46. आ॒दि॒त्यस्य॒ तेजाꣳ॑सि॒ तेजाꣳ॑स्या दि॒त्यस्या॑ दि॒त्यस्य॒ तेजाꣳ॑सि । \newline
47. तेजाꣳ॑सि व॒सन्ता॑ व॒सन्ता॒ तेजाꣳ॑सि॒ तेजाꣳ॑सि व॒सन्ता᳚ । \newline
48. व॒सन्ता᳚ प्रा॒तः प्रा॒तर् व॒सन्ता॑ व॒सन्ता᳚ प्रा॒तः । \newline
49. प्रा॒तर् ग्री॒ष्मे ग्री॒ष्मे प्रा॒तः प्रा॒तर् ग्री॒ष्मे । \newline
50. ग्री॒ष्मे म॒द्ध्यन्दि॑ने म॒द्ध्यन्दि॑ने ग्री॒ष्मे ग्री॒ष्मे म॒द्ध्यन्दि॑ने । \newline
51. म॒द्ध्यन्दि॑ने श॒रदि॑ श॒रदि॑ म॒द्ध्यन्दि॑ने म॒द्ध्यन्दि॑ने श॒रदि॑ । \newline
52. श॒रद्य॑परा॒ह्णे॑ ऽपरा॒ह्णे श॒रदि॑ श॒रद्य॑परा॒ह्णे । \newline
53. अ॒प॒रा॒ह्णे याव॑न्ति॒ याव॑ न्त्यपरा॒ह्णे॑ ऽपरा॒ह्णे याव॑न्ति । \newline
54. अ॒प॒रा॒ह्ण इत्य॑पर - अ॒ह्ने । \newline
55. याव॑न्त्ये॒वैव याव॑न्ति॒ याव॑न्त्ये॒व । \newline
56. ए॒व तेजाꣳ॑सि॒ तेजाꣳ॑ स्ये॒वैव तेजाꣳ॑सि । \newline
57. तेजाꣳ॑सि॒ तानि॒ तानि॒ तेजाꣳ॑सि॒ तेजाꣳ॑सि॒ तानि॑ । \newline
58. ता न्ये॒वैव तानि॒ तान्ये॒व । \newline
59. ए॒वावा वै॒वै वाव॑ । \newline

\textbf{Ghana Paata } \newline

1. आ॒ग्ने॒यीम् कृ॑ष्णग्री॒वीम् कृ॑ष्णग्री॒वी मा᳚ग्ने॒यी मा᳚ग्ने॒यीम् कृ॑ष्णग्री॒वीꣳ सꣳ॑हि॒ताꣳ सꣳ॑हि॒ताम् कृ॑ष्णग्री॒वी मा᳚ग्ने॒यी मा᳚ग्ने॒यीम् कृ॑ष्णग्री॒वीꣳ सꣳ॑हि॒ताम् । \newline
2. कृ॒ष्ण॒ग्री॒वीꣳ सꣳ॑हि॒ताꣳ सꣳ॑हि॒ताम् कृ॑ष्णग्री॒वीम् कृ॑ष्णग्री॒वीꣳ सꣳ॑हि॒ता मै॒न्द्री मै॒न्द्रीꣳ सꣳ॑हि॒ताम् कृ॑ष्णग्री॒वीम् कृ॑ष्णग्री॒वीꣳ सꣳ॑हि॒ता मै॒न्द्रीम् । \newline
3. कृ॒ष्ण॒ग्री॒वीमिति॑ कृष्ण - ग्री॒वीम् । \newline
4. सꣳ॒॒हि॒ता मै॒न्द्री मै॒न्द्रीꣳ सꣳ॑हि॒ताꣳ सꣳ॑हि॒ता मै॒न्द्रीꣳ श्वे॒ताꣳ श्वे॒ता मै॒न्द्रीꣳ सꣳ॑हि॒ताꣳ सꣳ॑हि॒ता मै॒न्द्रीꣳ श्वे॒ताम् । \newline
5. सꣳ॒॒हि॒तामिति॑ सं - हि॒ताम् । \newline
6. ऐ॒न्द्रीꣳ श्वे॒ताꣳ श्वे॒ता मै॒न्द्री मै॒न्द्रीꣳ श्वे॒ताम् बा॑र्.हस्प॒त्याम् बा॑र्.हस्प॒त्याꣳ श्वे॒ता मै॒न्द्री मै॒न्द्रीꣳ श्वे॒ताम् बा॑र्.हस्प॒त्याम् । \newline
7. श्वे॒ताम् बा॑र्.हस्प॒त्याम् बा॑र्.हस्प॒त्याꣳ श्वे॒ताꣳ श्वे॒ताम् बा॑र्.हस्प॒त्या मे॒ता ए॒ता बा॑र्.हस्प॒त्याꣳ श्वे॒ताꣳ श्वे॒ताम् बा॑र्.हस्प॒त्या मे॒ताः । \newline
8. बा॒र्॒.ह॒स्प॒त्या मे॒ता ए॒ता बा॑र्.हस्प॒त्याम् बा॑र्.हस्प॒त्या मे॒ता ए॒वैवैता बा॑र्.हस्प॒त्याम् बा॑र्.हस्प॒त्या मे॒ता ए॒व । \newline
9. ए॒ता ए॒वैवैता ए॒ता ए॒व दे॒वता॑ दे॒वता॑ ए॒वैता ए॒ता ए॒व दे॒वताः᳚ । \newline
10. ए॒व दे॒वता॑ दे॒वता॑ ए॒वैव दे॒वताः॒ स्वेन॒ स्वेन॑ दे॒वता॑ ए॒वैव दे॒वताः॒ स्वेन॑ । \newline
11. दे॒वताः॒ स्वेन॒ स्वेन॑ दे॒वता॑ दे॒वताः॒ स्वेन॑ भाग॒धेये॑न भाग॒धेये॑न॒ स्वेन॑ दे॒वता॑ दे॒वताः॒ स्वेन॑ भाग॒धेये॑न । \newline
12. स्वेन॑ भाग॒धेये॑न भाग॒धेये॑न॒ स्वेन॒ स्वेन॑ भाग॒धेये॒नो पोप॑ भाग॒धेये॑न॒ स्वेन॒ स्वेन॑ भाग॒धेये॒नोप॑ । \newline
13. भा॒ग॒धेये॒नो पोप॑ भाग॒धेये॑न भाग॒धेये॒नोप॑ धावति धाव॒त्युप॑ भाग॒धेये॑न भाग॒धेये॒नोप॑ धावति । \newline
14. भा॒ग॒धेये॒नेति॑ भाग - धेये॑न । \newline
15. उप॑ धावति धाव॒ त्युपोप॑ धावति॒ ता स्ता धा॑व॒ त्युपोप॑ धावति॒ ताः । \newline
16. धा॒व॒ति॒ ता स्ता धा॑वति धावति॒ ता ए॒वैव ता धा॑वति धावति॒ ता ए॒व । \newline
17. ता ए॒वैव ता स्ता ए॒वास्मि॑न् नस्मिन् ने॒व ता स्ता ए॒वास्मिन्न्॑ । \newline
18. ए॒वास्मि॑न् नस्मिन् ने॒वैवास्मि॑न् ब्रह्मवर्च॒सम् ब्र॑ह्मवर्च॒स म॑स्मिन् ने॒वैवास्मि॑न् ब्रह्मवर्च॒सम् । \newline
19. अ॒स्मि॒न् ब्र॒ह्म॒व॒र्च॒सम् ब्र॑ह्मवर्च॒स म॑स्मिन् नस्मिन् ब्रह्मवर्च॒सम् द॑धति दधति ब्रह्मवर्च॒स म॑स्मिन् नस्मिन् ब्रह्मवर्च॒सम् द॑धति । \newline
20. ब्र॒ह्म॒व॒र्च॒सम् द॑धति दधति ब्रह्मवर्च॒सम् ब्र॑ह्मवर्च॒सम् द॑धति ब्रह्मवर्च॒सी ब्र॑ह्मवर्च॒सी द॑धति ब्रह्मवर्च॒सम् ब्र॑ह्मवर्च॒सम् द॑धति ब्रह्मवर्च॒सी । \newline
21. ब्र॒ह्म॒व॒र्च॒समिति॑ ब्रह्म - व॒र्च॒सम् । \newline
22. द॒ध॒ति॒ ब्र॒ह्म॒व॒र्च॒सी ब्र॑ह्मवर्च॒सी द॑धति दधति ब्रह्मवर्च॒ स्ये॑वैव ब्र॑ह्मवर्च॒सी द॑धति दधति ब्रह्मवर्च॒स्ये॑व । \newline
23. ब्र॒ह्म॒व॒र्च॒ स्ये॑वैव ब्र॑ह्मवर्च॒सी ब्र॑ह्मवर्च॒स्ये॑व भ॑वति भवत्ये॒व ब्र॑ह्मवर्च॒सी ब्र॑ह्मवर्च॒स्ये॑व भ॑वति । \newline
24. ब्र॒ह्म॒व॒र्च॒सीति॑ ब्रह्म - व॒र्च॒सी । \newline
25. ए॒व भ॑वति भव त्ये॒वैव भ॑वति व॒सन्ता॑ व॒सन्ता॑ भव त्ये॒वैव भ॑वति व॒सन्ता᳚ । \newline
26. भ॒व॒ति॒ व॒सन्ता॑ व॒सन्ता॑ भवति भवति व॒सन्ता᳚ प्रा॒तः प्रा॒तर् व॒सन्ता॑ भवति भवति व॒सन्ता᳚ प्रा॒तः । \newline
27. व॒सन्ता᳚ प्रा॒तः प्रा॒तर् व॒सन्ता॑ व॒सन्ता᳚ प्रा॒त रा᳚ग्ने॒यी मा᳚ग्ने॒यीम् प्रा॒तर् व॒सन्ता॑ व॒सन्ता᳚ प्रा॒त रा᳚ग्ने॒यीम् । \newline
28. प्रा॒त रा᳚ग्ने॒यी मा᳚ग्ने॒यीम् प्रा॒तः प्रा॒त रा᳚ग्ने॒यीम् कृ॑ष्णग्री॒वीम् कृ॑ष्णग्री॒वी मा᳚ग्ने॒यीम् प्रा॒तः प्रा॒त रा᳚ग्ने॒यीम् कृ॑ष्णग्री॒वीम् । \newline
29. आ॒ग्ने॒यीम् कृ॑ष्णग्री॒वीम् कृ॑ष्णग्री॒वी मा᳚ग्ने॒यी मा᳚ग्ने॒यीम् कृ॑ष्णग्री॒वी मा कृ॑ष्णग्री॒वी मा᳚ग्ने॒यी मा᳚ग्ने॒यीम् कृ॑ष्णग्री॒वी मा । \newline
30. कृ॒ष्ण॒ग्री॒वी मा कृ॑ष्णग्री॒वीम् कृ॑ष्णग्री॒वी मा ल॑भेत लभे॒ता कृ॑ष्णग्री॒वीम् कृ॑ष्णग्री॒वी मा ल॑भेत । \newline
31. कृ॒ष्ण॒ग्री॒वीमिति॑ कृष्ण - ग्री॒वीम् । \newline
32. आ ल॑भेत लभे॒ता ल॑भेत ग्री॒ष्मे ग्री॒ष्मे ल॑भे॒ता ल॑भेत ग्री॒ष्मे । \newline
33. ल॒भे॒त॒ ग्री॒ष्मे ग्री॒ष्मे ल॑भेत लभेत ग्री॒ष्मे म॒द्ध्यन्दि॑ने म॒द्ध्यन्दि॑ने ग्री॒ष्मे ल॑भेत लभेत ग्री॒ष्मे म॒द्ध्यन्दि॑ने । \newline
34. ग्री॒ष्मे म॒द्ध्यन्दि॑ने म॒द्ध्यन्दि॑ने ग्री॒ष्मे ग्री॒ष्मे म॒द्ध्यन्दि॑ने सꣳहि॒ताꣳ सꣳ॑हि॒ताम् म॒द्ध्यन्दि॑ने ग्री॒ष्मे ग्री॒ष्मे म॒द्ध्यन्दि॑ने सꣳहि॒ताम् । \newline
35. म॒द्ध्यन्दि॑ने सꣳहि॒ताꣳ सꣳ॑हि॒ताम् म॒द्ध्यन्दि॑ने म॒द्ध्यन्दि॑ने सꣳहि॒ता मै॒न्द्री मै॒न्द्रीꣳ सꣳ॑हि॒ताम् म॒द्ध्यन्दि॑ने म॒द्ध्यन्दि॑ने सꣳहि॒ता मै॒न्द्रीम् । \newline
36. सꣳ॒॒हि॒ता मै॒न्द्री मै॒न्द्रीꣳ सꣳ॑हि॒ताꣳ सꣳ॑हि॒ता मै॒न्द्रीꣳ श॒रदि॑ श॒रद्यै॒न्द्रीꣳ सꣳ॑हि॒ताꣳ सꣳ॑हि॒ता मै॒न्द्रीꣳ श॒रदि॑ । \newline
37. सꣳ॒॒हि॒तामिति॑ सं - हि॒ताम् । \newline
38. ऐ॒न्द्रीꣳ श॒रदि॑ श॒रद्यै॒न्द्री मै॒न्द्रीꣳ श॒रद्य॑परा॒ह्णे॑ ऽपरा॒ह्णे श॒रद्यै॒न्द्री मै॒न्द्रीꣳ श॒रद्य॑परा॒ह्णे । \newline
39. श॒रद्य॑परा॒ह्णे॑ ऽपरा॒ह्णे श॒रदि॑ श॒रद्य॑परा॒ह्णे श्वे॒ताꣳ श्वे॒ता म॑परा॒ह्णे श॒रदि॑ श॒रद्य॑परा॒ह्णे श्वे॒ताम् । \newline
40. अ॒प॒रा॒ह्णे श्वे॒ताꣳ श्वे॒ता म॑परा॒ह्णे॑ ऽपरा॒ह्णे श्वे॒ताम् बा॑र्.हस्प॒त्याम् बा॑र्.हस्प॒त्याꣳ श्वे॒ता म॑परा॒ह्णे॑ ऽपरा॒ह्णे श्वे॒ताम् बा॑र्.हस्प॒त्याम् । \newline
41. अ॒प॒रा॒ह्ण इत्य॑ पर - अ॒ह्ने । \newline
42. श्वे॒ताम् बा॑र्.हस्प॒त्याम् बा॑र्.हस्प॒त्याꣳ श्वे॒ताꣳ श्वे॒ताम् बा॑र्.हस्प॒त्याम् त्रीणि॒ त्रीणि॑ बार्.हस्प॒त्याꣳ श्वे॒ताꣳ श्वे॒ताम् बा॑र्.हस्प॒त्याम् त्रीणि॑ । \newline
43. बा॒र्॒.ह॒स्प॒त्याम् त्रीणि॒ त्रीणि॑ बार्.हस्प॒त्याम् बा॑र्.हस्प॒त्याम् त्रीणि॒ वै वै त्रीणि॑ बार्.हस्प॒त्याम् बा॑र्.हस्प॒त्याम् त्रीणि॒ वै । \newline
44. त्रीणि॒ वै वै त्रीणि॒ त्रीणि॒ वा आ॑दि॒त्यस्या॑ दि॒त्यस्य॒ वै त्रीणि॒ त्रीणि॒ वा आ॑दि॒त्यस्य॑ । \newline
45. वा आ॑दि॒त्यस्या॑ दि॒त्यस्य॒ वै वा आ॑दि॒त्यस्य॒ तेजाꣳ॑सि॒ तेजाꣳ॑ स्यादि॒त्यस्य॒ वै वा आ॑दि॒त्यस्य॒ तेजाꣳ॑सि । \newline
46. आ॒दि॒त्यस्य॒ तेजाꣳ॑सि॒ तेजाꣳ॑ स्यादि॒त्यस्या॑ दि॒त्यस्य॒ तेजाꣳ॑सि व॒सन्ता॑ व॒सन्ता॒ तेजाꣳ॑ स्यादि॒त्यस्या॑ दि॒त्यस्य॒ तेजाꣳ॑सि व॒सन्ता᳚ । \newline
47. तेजाꣳ॑सि व॒सन्ता॑ व॒सन्ता॒ तेजाꣳ॑सि॒ तेजाꣳ॑सि व॒सन्ता᳚ प्रा॒तः प्रा॒तर् व॒सन्ता॒ तेजाꣳ॑सि॒ तेजाꣳ॑सि व॒सन्ता᳚ प्रा॒तः । \newline
48. व॒सन्ता᳚ प्रा॒तः प्रा॒तर् व॒सन्ता॑ व॒सन्ता᳚ प्रा॒तर् ग्री॒ष्मे ग्री॒ष्मे प्रा॒तर् व॒सन्ता॑ व॒सन्ता᳚ प्रा॒तर् ग्री॒ष्मे । \newline
49. प्रा॒तर् ग्री॒ष्मे ग्री॒ष्मे प्रा॒तः प्रा॒तर् ग्री॒ष्मे म॒द्ध्यन्दि॑ने म॒द्ध्यन्दि॑ने ग्री॒ष्मे प्रा॒तः प्रा॒तर् ग्री॒ष्मे म॒द्ध्यन्दि॑ने । \newline
50. ग्री॒ष्मे म॒द्ध्यन्दि॑ने म॒द्ध्यन्दि॑ने ग्री॒ष्मे ग्री॒ष्मे म॒द्ध्यन्दि॑ने श॒रदि॑ श॒रदि॑ म॒द्ध्यन्दि॑ने ग्री॒ष्मे ग्री॒ष्मे म॒द्ध्यन्दि॑ने श॒रदि॑ । \newline
51. म॒द्ध्यन्दि॑ने श॒रदि॑ श॒रदि॑ म॒द्ध्यन्दि॑ने म॒द्ध्यन्दि॑ने श॒रद्य॑परा॒ह्णे॑ ऽपरा॒ह्णे श॒रदि॑ म॒द्ध्यन्दि॑ने म॒द्ध्यन्दि॑ने श॒रद्य॑परा॒ह्णे । \newline
52. श॒रद्य॑परा॒ह्णे॑ ऽपरा॒ह्णे श॒रदि॑ श॒रद्य॑परा॒ह्णे याव॑न्ति॒ याव॑न्त्यपरा॒ह्णे श॒रदि॑ श॒रद्य॑परा॒ह्णे याव॑न्ति । \newline
53. अ॒प॒रा॒ह्णे याव॑न्ति॒ याव॑न्त्यपरा॒ह्णे॑ ऽपरा॒ह्णे याव॑न्त्ये॒वैव याव॑न्त्यपरा॒ह्णे॑ ऽपरा॒ह्णे याव॑न्त्ये॒व । \newline
54. अ॒प॒रा॒ह्ण इत्य॑पर - अ॒ह्ने । \newline
55. याव॑न्त्ये॒वैव याव॑न्ति॒ याव॑न्त्ये॒व तेजाꣳ॑सि॒ तेजाꣳ॑स्ये॒व याव॑न्ति॒ याव॑न्त्ये॒व तेजाꣳ॑सि । \newline
56. ए॒व तेजाꣳ॑सि॒ तेजाꣳ॑ स्ये॒वैव तेजाꣳ॑सि॒ तानि॒ तानि॒ तेजाꣳ॑ स्ये॒वैव तेजाꣳ॑सि॒ तानि॑ । \newline
57. तेजाꣳ॑सि॒ तानि॒ तानि॒ तेजाꣳ॑सि॒ तेजाꣳ॑सि॒ तान्ये॒वैव तानि॒ तेजाꣳ॑सि॒ तेजाꣳ॑सि॒ तान्ये॒व । \newline
58. तान्ये॒वैव तानि॒ तान्ये॒ वावावै॒व तानि॒ तान्ये॒वाव॑ । \newline
59. ए॒वावा वै॒वै वाव॑ रुन्धे रु॒न्धे ऽवै॒वै वाव॑ रुन्धे । \newline
\pagebreak
\markright{ TS 2.1.2.6  \hfill https://www.vedavms.in \hfill}

\section{ TS 2.1.2.6 }

\textbf{TS 2.1.2.6 } \newline
\textbf{Samhita Paata} \newline

-व॑ रुन्धे संॅवथ्स॒रं प॒र्याल॑भ्यन्ते संॅवथ्स॒रो वै ब्र॑ह्मवर्च॒सस्य॑ प्रदा॒ता सं॑ॅवथ्स॒र ए॒वास्मै᳚ ब्रह्मवर्च॒सं प्र य॑च्छति ब्रह्मवर्च॒स्ये॑व भ॑वति ग॒र्भिण॑यो भवन्तीन्द्रि॒यं ॅवै गर्भ॑ इन्द्रि॒यमे॒वास्मि॑न् दधति सारस्व॒तीं मे॒षीमा ल॑भेत॒ य ई᳚श्व॒रो वा॒चो वदि॑तोः॒ सन् वाचं॒ न वदे॒द्-वाग्वै सर॑स्वती॒ सर॑स्वतीमे॒व स्वेन॑ भाग॒धेये॒नोप॑ धावति॒ सैवास्मि॒न्-  [  ] \newline

\textbf{Pada Paata} \newline

अवेति॑ । रु॒न्धे॒ । सं॒ॅव॒थ्स॒रमिति॑ सं - व॒थ्स॒रम् । प॒र्याल॑भ्यन्त॒ इति॑ परि - आल॑भ्यन्ते । सं॒ॅव॒थ्स॒र इति॑ सं-व॒थ्स॒रः । वै । ब्र॒ह्म॒व॒र्च॒सस्येति॑ ब्रह्म - व॒र्च॒सस्य॑ । प्र॒दा॒तेति॑ प्र - दा॒ता । सं॒ॅव॒थ्स॒र इति॑ सं - व॒थ्स॒रः । ए॒व । अ॒स्मै॒ । ब्र॒ह्म॒व॒र्च॒समिति॑ ब्रह्म - व॒र्च॒सम् । प्रेति॑ । य॒च्छ॒ति॒ । ब्र॒ह्म॒व॒र्च॒सीति॑ ब्रह्म - व॒र्च॒सी । ए॒व । भ॒व॒ति॒ । ग॒र्भिण॑यः । भ॒व॒न्ति॒ । इ॒न्द्रि॒यम् । वै । गर्भः॑ । इ॒न्द्रि॒यम् । ए॒व । अ॒स्मि॒न्न् । द॒ध॒ति॒ । सा॒र॒स्व॒तीम् । मे॒षीम् । एति॑ । ल॒भे॒त॒ । यः । ई॒श्व॒रः । वा॒चः । वदि॑तोः । सन्न् । वाच᳚म् । न । वदे᳚त् । वाक् । वै । सर॑स्वती । सर॑स्वतीम् । ए॒व । स्वेन॑ । भा॒ग॒धेये॒नेति॑ भाग - धेये॑न । उपेति॑ । धा॒व॒ति॒ । सा । ए॒व । अ॒स्मि॒न्न् ।  \newline


\textbf{Krama Paata} \newline

अव॑ रुन्धे । रु॒न्धे॒ स॒म्ॅव॒थ्स॒रम् । स॒म्ॅव॒थ्स॒रम् प॒र्याल॑भ्यन्ते । स॒म्ॅव॒थ्स॒रमिति॑ सं - व॒थ्स॒रम् । प॒र्याल॑भ्यन्ते सम्ॅवथ्स॒रः । प॒र्याल॑भ्यन्त॒ इति॑ परि - आल॑भ्यन्ते । स॒म्ॅव॒थ्स॒रो वै । स॒म्ॅव॒थ्स॒र इति॑ सं - व॒थ्स॒रः । वै ब्र॑ह्मवर्च॒सस्य॑ । ब्र॒ह्म॒व॒र्च॒सस्य॑ प्रदा॒ता । ब्र॒ह्म॒व॒र्च॒सस्येति॑ ब्रह्म - व॒र्च॒सस्य॑ । प्र॒दा॒ता स॑म्ॅवथ्स॒रः । प्र॒दा॒तेति॑ प्र - दा॒ता । स॒म्ॅव॒थ्स॒र ए॒व । स॒म्ॅव॒थ्स॒र इति॑ सम् - व॒थ्स॒रः । ए॒वास्मै᳚ । अ॒स्मै॒ ब्र॒ह्म॒व॒र्च॒सम् । ब्र॒ह्म॒व॒र्च॒सम् प्र । ब्र॒ह्म॒व॒र्च॒समिति॑ ब्रह्म - व॒र्च॒सम् । प्र य॑च्छति । य॒च्छ॒ति॒ ब्र॒ह्म॒व॒र्च॒सी । ब्र॒ह्म॒व॒र्च॒स्ये॑व । ब्र॒ह्म॒व॒र्च॒सीति॑ ब्रह्म - व॒र्च॒सी । ए॒व भ॑वति । भ॒व॒ति॒ ग॒र्भिण॑यः । ग॒र्भिण॑यो भवन्ति । भ॒व॒न्ती॒न्द्रि॒यम् । इ॒न्द्रि॒यं ॅवै । वै गर्भः॑ । गर्भ॑ इन्द्रि॒यम् । इ॒न्द्रि॒यमे॒व । ए॒वास्मिन्न्॑ । अ॒स्मि॒न् द॒ध॒॒ति॒ । द॒ध॒ति॒ सा॒र॒स्व॒तीम् । सा॒र॒स्व॒तीम् मे॒षीम् । मे॒षीमा । आ ल॑भेत । ल॒भे॒त॒ यः । य ई᳚श्व॒रः । ई॒श्व॒रो वा॒चः । वा॒चो वदि॑तोः । वदि॑तोः॒ सन्न् । सन् वाच᳚म् । वाच॒म् न । न वदे᳚त् । वदे॒द् वाक् । वाग् वै । वै सर॑स्वती । सर॑स्वती॒ सर॑स्वतीम् । सर॑स्वतीमे॒व । ए॒व स्वेन॑ । स्वेन॑ भाग॒धेये॑न । भा॒ग॒धेये॒नोप॑ । भा॒ग॒धेये॒नेति॑ भाग - धेये॑न । उप॑ धावति । धा॒व॒ति॒ सा । सैव । ए॒वास्मिन्न्॑ । अ॒स्मि॒न् वाच᳚म् \newline

\textbf{Jatai Paata} \newline

1. अव॑ रुन्धे रु॒न्धे ऽवाव॑ रुन्धे । \newline
2. रु॒न्धे॒ सं॒ॅव॒थ्स॒रꣳ सं॑ॅवथ्स॒रꣳ रु॑न्धे रुन्धे संॅवथ्स॒रम् । \newline
3. सं॒ॅव॒थ्स॒रम् प॒र्याल॑भ्यन्ते प॒र्याल॑भ्यन्ते संॅवथ्स॒रꣳ सं॑ॅवथ्स॒रम् प॒र्याल॑भ्यन्ते । \newline
4. सं॒ॅव॒थ्स॒रमिति॑ सं - व॒थ्स॒रम् । \newline
5. प॒र्याल॑भ्यन्ते संॅवथ्स॒रः सं॑ॅवथ्स॒रः प॒र्याल॑भ्यन्ते प॒र्याल॑भ्यन्ते संॅवथ्स॒रः । \newline
6. प॒र्याल॑भ्यन्त॒ इति॑ परि - आल॑भ्यन्ते । \newline
7. सं॒ॅव॒थ्स॒रो वै वै सं॑ॅवथ्स॒रः सं॑ॅवथ्स॒रो वै । \newline
8. सं॒ॅव॒थ्स॒र इति॑ सं - व॒थ्स॒रः । \newline
9. वै ब्र॑ह्मवर्च॒सस्य॑ ब्रह्मवर्च॒सस्य॒ वै वै ब्र॑ह्मवर्च॒सस्य॑ । \newline
10. ब्र॒ह्म॒व॒र्च॒सस्य॑ प्रदा॒ता प्र॑दा॒ता ब्र॑ह्मवर्च॒सस्य॑ ब्रह्मवर्च॒सस्य॑ प्रदा॒ता । \newline
11. ब्र॒ह्म॒व॒र्च॒सस्येति॑ ब्रह्म - व॒र्च॒सस्य॑ । \newline
12. प्र॒दा॒ता सं॑ॅवथ्स॒रः सं॑ॅवथ्स॒रः प्र॑दा॒ता प्र॑दा॒ता सं॑ॅवथ्स॒रः । \newline
13. प्र॒दा॒तेति॑ प्र - दा॒ता । \newline
14. सं॒ॅव॒थ्स॒र ए॒वैव सं॑ॅवथ्स॒रः सं॑ॅवथ्स॒र ए॒व । \newline
15. सं॒ॅव॒थ्स॒र इति॑ सं - व॒थ्स॒रः । \newline
16. ए॒वास्मा॑ अस्मा ए॒वैवास्मै᳚ । \newline
17. अ॒स्मै॒ ब्र॒ह्म॒व॒र्च॒सम् ब्र॑ह्मवर्च॒स म॑स्मा अस्मै ब्रह्मवर्च॒सम् । \newline
18. ब्र॒ह्म॒व॒र्च॒सम् प्र प्र ब्र॑ह्मवर्च॒सम् ब्र॑ह्मवर्च॒सम् प्र । \newline
19. ब्र॒ह्म॒व॒र्च॒समिति॑ ब्रह्म - व॒र्च॒सम् । \newline
20. प्र य॑च्छति यच्छति॒ प्र प्र य॑च्छति । \newline
21. य॒च्छ॒ति॒ ब्र॒ह्म॒व॒र्च॒सी ब्र॑ह्मवर्च॒सी य॑च्छति यच्छति ब्रह्मवर्च॒सी । \newline
22. ब्र॒ह्म॒व॒र्च॒ स्ये॑वैव ब्र॑ह्मवर्च॒सी ब्र॑ह्मवर्च॒ स्ये॑व । \newline
23. ब्र॒ह्म॒व॒र्च॒सीति॑ ब्रह्म - व॒र्च॒सी । \newline
24. ए॒व भ॑वति भव त्ये॒वैव भ॑वति । \newline
25. भ॒व॒ति॒ ग॒र्भिण॑यो ग॒र्भिण॑यो भवति भवति ग॒र्भिण॑यः । \newline
26. ग॒र्भिण॑यो भवन्ति भवन्ति ग॒र्भिण॑यो ग॒र्भिण॑यो भवन्ति । \newline
27. भ॒व॒न्ती॒न्द्रि॒य मि॑न्द्रि॒यम् भ॑वन्ति भवन्तीन्द्रि॒यम् । \newline
28. इ॒न्द्रि॒यं ॅवै वा इ॑न्द्रि॒य मि॑न्द्रि॒यं ॅवै । \newline
29. वै गर्भो॒ गर्भो॒ वै वै गर्भः॑ । \newline
30. गर्भ॑ इन्द्रि॒य मि॑न्द्रि॒यम् गर्भो॒ गर्भ॑ इन्द्रि॒यम् । \newline
31. इ॒न्द्रि॒य मे॒वैवे न्द्रि॒य मि॑न्द्रि॒य मे॒व । \newline
32. ए॒वास्मि॑न् नस्मिन् ने॒वैवास्मिन्न्॑ । \newline
33. अ॒स्मि॒न् द॒ध॒ति॒ द॒ध॒त्य॒स्मि॒न् न॒स्मि॒न् द॒ध॒ति॒ । \newline
34. द॒ध॒ति॒ सा॒र॒स्व॒तीꣳ सा॑रस्व॒तीम् द॑धति दधति सारस्व॒तीम् । \newline
35. सा॒र॒स्व॒तीम् मे॒षीम् मे॒षीꣳ सा॑रस्व॒तीꣳ सा॑रस्व॒तीम् मे॒षीम् । \newline
36. मे॒षी मा मे॒षीम् मे॒षी मा । \newline
37. आ ल॑भेत लभे॒ता ल॑भेत । \newline
38. ल॒भे॒त॒ यो यो ल॑भेत लभेत॒ यः । \newline
39. य ई᳚श्व॒र ई᳚श्व॒रो यो य ई᳚श्व॒रः । \newline
40. ई॒श्व॒रो वा॒चो वा॒च ई᳚श्व॒र ई᳚श्व॒रो वा॒चः । \newline
41. वा॒चो वदि॑तो॒र् वदि॑तोर् वा॒चो वा॒चो वदि॑तोः । \newline
42. वदि॑तोः॒ सन् थ्सन्. वदि॑तो॒र् वदि॑तोः॒ सन्न् । \newline
43. सन्. वाचं॒ ॅवाचꣳ॒॒ सन् थ्सन्. वाच᳚म् । \newline
44. वाच॒म् न न वाचं॒ ॅवाच॒म् न । \newline
45. न वदे॒द् वदे॒न् न न वदे᳚त् । \newline
46. वदे॒द् वाग् वाग् वदे॒द् वदे॒द् वाक् । \newline
47. वाग् वै वै वाग् वाग् वै । \newline
48. वै सर॑स्वती॒ सर॑स्वती॒ वै वै सर॑स्वती । \newline
49. सर॑स्वती॒ सर॑स्वतीꣳ॒॒ सर॑स्वतीꣳ॒॒ सर॑स्वती॒ सर॑स्वती॒ सर॑स्वतीम् । \newline
50. सर॑स्वती मे॒वैव सर॑स्वतीꣳ॒॒ सर॑स्वती मे॒व । \newline
51. ए॒व स्वेन॒ स्वेनै॒वैव स्वेन॑ । \newline
52. स्वेन॑ भाग॒धेये॑न भाग॒धेये॑न॒ स्वेन॒ स्वेन॑ भाग॒धेये॑न । \newline
53. भा॒ग॒धेये॒नोपोप॑ भाग॒धेये॑न भाग॒धेये॒नोप॑ । \newline
54. भा॒ग॒धेये॒नेति॑ भाग - धेये॑न । \newline
55. उप॑ धावति धाव॒ त्युपोप॑ धावति । \newline
56. धा॒व॒ति॒ सा सा धा॑वति धावति॒ सा । \newline
57. सैवैव सा सैव । \newline
58. ए॒वास्मि॑न् नस्मिन् ने॒वैवास्मिन्न्॑ । \newline
59. अ॒स्मि॒न्॒. वाचं॒ ॅवाच॑ मस्मिन् नस्मि॒न्॒. वाच᳚म् । \newline

\textbf{Ghana Paata } \newline

1. अव॑ रुन्धे रु॒न्धे ऽवाव॑ रुन्धे संॅवथ्स॒रꣳ सं॑ॅवथ्स॒रꣳ रु॒न्धे ऽवाव॑ रुन्धे संॅवथ्स॒रम् । \newline
2. रु॒न्धे॒ सं॒ॅव॒थ्स॒रꣳ सं॑ॅवथ्स॒रꣳ रु॑न्धे रुन्धे संॅवथ्स॒रम् प॒र्याल॑भ्यन्ते प॒र्याल॑भ्यन्ते संॅवथ्स॒रꣳ रु॑न्धे रुन्धे संॅवथ्स॒रम् प॒र्याल॑भ्यन्ते । \newline
3. सं॒ॅव॒थ्स॒रम् प॒र्याल॑भ्यन्ते प॒र्याल॑भ्यन्ते संॅवथ्स॒रꣳ सं॑ॅवथ्स॒रम् प॒र्याल॑भ्यन्ते संॅवथ्स॒रः सं॑ॅवथ्स॒रः प॒र्याल॑भ्यन्ते संॅवथ्स॒रꣳ सं॑ॅवथ्स॒रम् प॒र्याल॑भ्यन्ते संॅवथ्स॒रः । \newline
4. सं॒ॅव॒थ्स॒रमिति॑ सं - व॒थ्स॒रम् । \newline
5. प॒र्याल॑भ्यन्ते संॅवथ्स॒रः सं॑ॅवथ्स॒रः प॒र्याल॑भ्यन्ते प॒र्याल॑भ्यन्ते संॅवथ्स॒रो वै वै सं॑ॅवथ्स॒रः प॒र्याल॑भ्यन्ते प॒र्याल॑भ्यन्ते संॅवथ्स॒रो वै । \newline
6. प॒र्याल॑भ्यन्त॒ इति॑ परि - आल॑भ्यन्ते । \newline
7. सं॒ॅव॒थ्स॒रो वै वै सं॑ॅवथ्स॒रः सं॑ॅवथ्स॒रो वै ब्र॑ह्मवर्च॒सस्य॑ ब्रह्मवर्च॒सस्य॒ वै सं॑ॅवथ्स॒रः सं॑ॅवथ्स॒रो वै ब्र॑ह्मवर्च॒सस्य॑ । \newline
8. सं॒ॅव॒थ्स॒र इति॑ सं - व॒थ्स॒रः । \newline
9. वै ब्र॑ह्मवर्च॒सस्य॑ ब्रह्मवर्च॒सस्य॒ वै वै ब्र॑ह्मवर्च॒सस्य॑ प्रदा॒ता प्र॑दा॒ता ब्र॑ह्मवर्च॒सस्य॒ वै वै ब्र॑ह्मवर्च॒सस्य॑ प्रदा॒ता । \newline
10. ब्र॒ह्म॒व॒र्च॒सस्य॑ प्रदा॒ता प्र॑दा॒ता ब्र॑ह्मवर्च॒सस्य॑ ब्रह्मवर्च॒सस्य॑ प्रदा॒ता सं॑ॅवथ्स॒रः सं॑ॅवथ्स॒रः प्र॑दा॒ता ब्र॑ह्मवर्च॒सस्य॑ ब्रह्मवर्च॒सस्य॑ प्रदा॒ता सं॑ॅवथ्स॒रः । \newline
11. ब्र॒ह्म॒व॒र्च॒सस्येति॑ ब्रह्म - व॒र्च॒सस्य॑ । \newline
12. प्र॒दा॒ता सं॑ॅवथ्स॒रः सं॑ॅवथ्स॒रः प्र॑दा॒ता प्र॑दा॒ता सं॑ॅवथ्स॒र ए॒वैव सं॑ॅवथ्स॒रः प्र॑दा॒ता प्र॑दा॒ता सं॑ॅवथ्स॒र ए॒व । \newline
13. प्र॒दा॒तेति॑ प्र - दा॒ता । \newline
14. सं॒ॅव॒थ्स॒र ए॒वैव सं॑ॅवथ्स॒रः सं॑ॅवथ्स॒र ए॒वास्मा॑ अस्मा ए॒व सं॑ॅवथ्स॒रः सं॑ॅवथ्स॒र ए॒वास्मै᳚ । \newline
15. सं॒ॅव॒थ्स॒र इति॑ सं - व॒थ्स॒रः । \newline
16. ए॒वास्मा॑ अस्मा ए॒वैवास्मै᳚ ब्रह्मवर्च॒सम् ब्र॑ह्मवर्च॒स म॑स्मा ए॒वैवास्मै᳚ ब्रह्मवर्च॒सम् । \newline
17. अ॒स्मै॒ ब्र॒ह्म॒व॒र्च॒सम् ब्र॑ह्मवर्च॒स म॑स्मा अस्मै ब्रह्मवर्च॒सम् प्र प्र ब्र॑ह्मवर्च॒स म॑स्मा अस्मै ब्रह्मवर्च॒सम् प्र । \newline
18. ब्र॒ह्म॒व॒र्च॒सम् प्र प्र ब्र॑ह्मवर्च॒सम् ब्र॑ह्मवर्च॒सम् प्र य॑च्छति यच्छति॒ प्र ब्र॑ह्मवर्च॒सम् ब्र॑ह्मवर्च॒सम् प्र य॑च्छति । \newline
19. ब्र॒ह्म॒व॒र्च॒समिति॑ ब्रह्म - व॒र्च॒सम् । \newline
20. प्र य॑च्छति यच्छति॒ प्र प्र य॑च्छति ब्रह्मवर्च॒सी ब्र॑ह्मवर्च॒सी य॑च्छति॒ प्र प्र य॑च्छति ब्रह्मवर्च॒सी । \newline
21. य॒च्छ॒ति॒ ब्र॒ह्म॒व॒र्च॒सी ब्र॑ह्मवर्च॒सी य॑च्छति यच्छति ब्रह्मवर्च॒ स्ये॑वैव ब्र॑ह्मवर्च॒सी य॑च्छति यच्छति ब्रह्मवर्च॒स्ये॑व । \newline
22. ब्र॒ह्म॒व॒र्च॒ स्ये॑वैव ब्र॑ह्मवर्च॒सी ब्र॑ह्मवर्च॒स्ये॑व भ॑वति भवत्ये॒व ब्र॑ह्मवर्च॒सी ब्र॑ह्मवर्च॒स्ये॑व भ॑वति । \newline
23. ब्र॒ह्म॒व॒र्च॒सीति॑ ब्रह्म - व॒र्च॒सी । \newline
24. ए॒व भ॑वति भव त्ये॒वैव भ॑वति ग॒र्भिण॑यो ग॒र्भिण॑यो भव त्ये॒वैव भ॑वति ग॒र्भिण॑यः । \newline
25. भ॒व॒ति॒ ग॒र्भिण॑यो ग॒र्भिण॑यो भवति भवति ग॒र्भिण॑यो भवन्ति भवन्ति ग॒र्भिण॑यो भवति भवति ग॒र्भिण॑यो भवन्ति । \newline
26. ग॒र्भिण॑यो भवन्ति भवन्ति ग॒र्भिण॑यो ग॒र्भिण॑यो भवन्तीन्द्रि॒य मि॑न्द्रि॒यम् भ॑वन्ति ग॒र्भिण॑यो ग॒र्भिण॑यो भवन्तीन्द्रि॒यम् । \newline
27. भ॒व॒न्ती॒न्द्रि॒य मि॑न्द्रि॒यम् भ॑वन्ति भवन्तीन्द्रि॒यं ॅवै वा इ॑न्द्रि॒यम् भ॑वन्ति भवन्तीन्द्रि॒यं ॅवै । \newline
28. इ॒न्द्रि॒यं ॅवै वा इ॑न्द्रि॒य मि॑न्द्रि॒यं ॅवै गर्भो॒ गर्भो॒ वा इ॑न्द्रि॒य मि॑न्द्रि॒यं ॅवै गर्भः॑ । \newline
29. वै गर्भो॒ गर्भो॒ वै वै गर्भ॑ इन्द्रि॒य मि॑न्द्रि॒यम् गर्भो॒ वै वै गर्भ॑ इन्द्रि॒यम् । \newline
30. गर्भ॑ इन्द्रि॒य मि॑न्द्रि॒यम् गर्भो॒ गर्भ॑ इन्द्रि॒य मे॒वैवे न्द्रि॒यम् गर्भो॒ गर्भ॑ इन्द्रि॒य मे॒व । \newline
31. इ॒न्द्रि॒य मे॒वैवे न्द्रि॒य मि॑न्द्रि॒य मे॒वास्मि॑न् नस्मिन् ने॒वे न्द्रि॒य मि॑न्द्रि॒य मे॒वास्मिन्न्॑ । \newline
32. ए॒वास्मि॑न् नस्मिन् ने॒वैवास्मि॑न् दधति दधत्यस्मिन् ने॒वैवास्मि॑न् दधति । \newline
33. अ॒स्मि॒न् द॒ध॒ति॒ द॒ध॒त्य॒स्मि॒न् न॒स्मि॒न् द॒ध॒ति॒ सा॒र॒स्व॒तीꣳ सा॑रस्व॒तीम् द॑धत्यस्मिन् नस्मिन् दधति सारस्व॒तीम् । \newline
34. द॒ध॒ति॒ सा॒र॒स्व॒तीꣳ सा॑रस्व॒तीम् द॑धति दधति सारस्व॒तीम् मे॒षीम् मे॒षीꣳ सा॑रस्व॒तीम् द॑धति दधति सारस्व॒तीम् मे॒षीम् । \newline
35. सा॒र॒स्व॒तीम् मे॒षीम् मे॒षीꣳ सा॑रस्व॒तीꣳ सा॑रस्व॒तीम् मे॒षी मा मे॒षीꣳ सा॑रस्व॒तीꣳ सा॑रस्व॒तीम् मे॒षी मा । \newline
36. मे॒षी मा मे॒षीम् मे॒षी मा ल॑भेत लभे॒ता मे॒षीम् मे॒षी मा ल॑भेत । \newline
37. आ ल॑भेत लभे॒ता ल॑भेत॒ यो यो ल॑भे॒ता ल॑भेत॒ यः । \newline
38. ल॒भे॒त॒ यो यो ल॑भेत लभेत॒ य ई᳚श्व॒र ई᳚श्व॒रो यो ल॑भेत लभेत॒ य ई᳚श्व॒रः । \newline
39. य ई᳚श्व॒र ई᳚श्व॒रो यो य ई᳚श्व॒रो वा॒चो वा॒च ई᳚श्व॒रो यो य ई᳚श्व॒रो वा॒चः । \newline
40. ई॒श्व॒रो वा॒चो वा॒च ई᳚श्व॒र ई᳚श्व॒रो वा॒चो वदि॑तो॒र् वदि॑तोर् वा॒च ई᳚श्व॒र ई᳚श्व॒रो वा॒चो वदि॑तोः । \newline
41. वा॒चो वदि॑तो॒र् वदि॑तोर् वा॒चो वा॒चो वदि॑तोः॒ सन् थ्सन्. वदि॑तोर् वा॒चो वा॒चो वदि॑तोः॒ सन्न् । \newline
42. वदि॑तोः॒ सन् थ्सन्. वदि॑तो॒र् वदि॑तोः॒ सन्. वाचं॒ ॅवाचꣳ॒॒ सन्. वदि॑तो॒र् वदि॑तोः॒ सन्. वाच᳚म् । \newline
43. सन्. वाचं॒ ॅवाचꣳ॒॒ सन् थ्सन्. वाच॒न्न न वाचꣳ॒॒ सन् थ्सन्. वाच॒न्न । \newline
44. वाच॒न्न न वाचं॒ ॅवाच॒न्न वदे॒द् वदे॒न् न वाचं॒ ॅवाच॒न्न वदे᳚त् । \newline
45. न वदे॒द् वदे॒न् न न वदे॒द् वाग् वाग् वदे॒न् न न वदे॒द् वाक् । \newline
46. वदे॒द् वाग् वाग् वदे॒द् वदे॒द् वाग् वै वै वाग् वदे॒द् वदे॒द् वाग् वै । \newline
47. वाग् वै वै वाग् वाग् वै सर॑स्वती॒ सर॑स्वती॒ वै वाग् वाग् वै सर॑स्वती । \newline
48. वै सर॑स्वती॒ सर॑स्वती॒ वै वै सर॑स्वती॒ सर॑स्वतीꣳ॒॒ सर॑स्वतीꣳ॒॒ सर॑स्वती॒ वै वै सर॑स्वती॒ सर॑स्वतीम् । \newline
49. सर॑स्वती॒ सर॑स्वतीꣳ॒॒ सर॑स्वतीꣳ॒॒ सर॑स्वती॒ सर॑स्वती॒ सर॑स्वती मे॒वैव सर॑स्वतीꣳ॒॒ सर॑स्वती॒ सर॑स्वती॒ सर॑स्वती मे॒व । \newline
50. सर॑स्वती मे॒वैव सर॑स्वतीꣳ॒॒ सर॑स्वती मे॒व स्वेन॒ स्वेनै॒व सर॑स्वतीꣳ॒॒ सर॑स्वती मे॒व स्वेन॑ । \newline
51. ए॒व स्वेन॒ स्वेनै॒वैव स्वेन॑ भाग॒धेये॑न भाग॒धेये॑न॒ स्वेनै॒वैव स्वेन॑ भाग॒धेये॑न । \newline
52. स्वेन॑ भाग॒धेये॑न भाग॒धेये॑न॒ स्वेन॒ स्वेन॑ भाग॒धेये॒नो पोप॑ भाग॒धेये॑न॒ स्वेन॒ स्वेन॑ भाग॒धेये॒नोप॑ । \newline
53. भा॒ग॒धेये॒नो पोप॑ भाग॒धेये॑न भाग॒धेये॒नोप॑ धावति धाव॒त्युप॑ भाग॒धेये॑न भाग॒धेये॒नोप॑ धावति । \newline
54. भा॒ग॒धेये॒नेति॑ भाग - धेये॑न । \newline
55. उप॑ धावति धाव॒ त्युपोप॑ धावति॒ सा सा धा॑व॒ त्युपोप॑ धावति॒ सा । \newline
56. धा॒व॒ति॒ सा सा धा॑वति धावति॒ सैवैव सा धा॑वति धावति॒ सैव । \newline
57. सैवैव सा सैवास्मि॑न् नस्मिन् ने॒व सा सैवास्मिन्न्॑ । \newline
58. ए॒वास्मि॑न् नस्मिन् ने॒वैवास्मि॒न्॒. वाचं॒ ॅवाच॑ मस्मिन् ने॒वै वास्मि॒न्॒. वाच᳚म् । \newline
59. अ॒स्मि॒न्॒. वाचं॒ ॅवाच॑ मस्मिन् नस्मि॒न्॒. वाच॑म् दधाति दधाति॒ वाच॑ मस्मिन् नस्मि॒न्॒. वाच॑म् दधाति । \newline
\pagebreak
\markright{ TS 2.1.2.7  \hfill https://www.vedavms.in \hfill}

\section{ TS 2.1.2.7 }

\textbf{TS 2.1.2.7 } \newline
\textbf{Samhita Paata} \newline

वाचं॑ दधाति प्रवदि॒ता वा॒चो भ॑व॒त्यप॑न्नदती भवति॒ तस्मा᳚न् मनु॒ष्याः᳚ सर्वां॒ ॅवाचं॑ ॅवदन्त्याग्ने॒यं कृ॒ष्णग्री॑व॒मा ल॑भेत सौ॒म्यं ब॒भ्रुं ज्योगा॑मयाव्य॒ग्निं ॅवा ए॒तस्य॒ शरी॑रं गच्छति॒ सोमꣳ॒॒ रसो॒ यस्य॒ ज्योगा॒मय॑त्य॒ग्नेरे॒वास्य॒ शरी॑रं निष्क्री॒णाति॒ सोमा॒द् रस॑मु॒त यदी॒तासु॒र्भव॑ति॒ जीव॑त्ये॒व सौ॒म्यं ब॒भ्रुमा ल॑भेताऽऽ*ग्ने॒यं कृ॒ष्णग्री॑वं प्र॒जाका॑मः॒ सोमो॒ - [  ] \newline

\textbf{Pada Paata} \newline

वाच᳚म् । द॒धा॒ति॒ । प्र॒व॒दि॒तेति॑ प्र - व॒दि॒ता । वा॒चः । भ॒व॒ति॒ । अप॑न्नद॒तीत्यप॑न्न - द॒ती॒ । भ॒व॒ति॒ । तस्मा᳚त् । म॒नु॒ष्याः᳚ । सर्वा᳚म् । वाच᳚म् । व॒द॒न्ति॒ । आ॒ग्ने॒यम् । कृ॒ष्णग्री॑व॒मिति॑ कृ॒ष्ण-ग्री॒व॒म् । एति॑ । ल॒भे॒त॒ । सौ॒म्यम् । ब॒भ्रुम् । ज्योगा॑मया॒वीति॒ ज्योक् - आ॒म॒या॒वी॒ । अ॒ग्निम् । वै । ए॒तस्य॑ । शरी॑रम् । ग॒च्छ॒ति॒ । सोम᳚म् । रसः॑ । यस्य॑ । ज्योक् । आ॒मय॑ति । अ॒ग्नेः । ए॒व । अ॒स्य॒ । शरी॑रम् । नि॒ष्क्री॒णातीति॑ निः - क्री॒णाति॑ । सोमा᳚त् । रस᳚म् । उ॒त । यदि॑ । इ॒तासु॒रिती॒त - अ॒सुः॒ । भव॑ति । जीव॑ति । ए॒व । सौ॒म्यम् । ब॒भ्रुम् । एति॑ । ल॒भे॒त॒ । आ॒ग्ने॒यम् । कृ॒ष्णग्री॑व॒मिति॑ कृ॒ष्ण - ग्री॒व॒म् । प्र॒जाका॑म॒ इति॑ प्र॒जा - का॒मः॒ । सोमः॑ ।  \newline


\textbf{Krama Paata} \newline

वाच॑म् दधाति । द॒धा॒ति॒ प्र॒व॒दि॒ता । प्र॒व॒दि॒ता वा॒चः । प्र॒व॒दि॒तेति॑ प्र - व॒दि॒ता । वा॒चो भ॑वति । भ॒व॒त्यप॑न्नदती । अप॑न्नदती भवति । अप॑न्नद॒तीत्यप॑न्न - द॒ती॒ । भ॒व॒ति॒ तस्मा᳚त् । तस्मा᳚न्,मनु॒ष्याः᳚ । म॒नु॒ष्याः᳚ सर्वा᳚म् । सर्वां॒ ॅवाच᳚म् । वाचं॑ ॅवदन्ति । व॒द॒न्त्या॒ग्ने॒यम् । आ॒ग्ने॒यं कृ॒ष्णग्री॑वम् । कृ॒ष्णग्री॑व॒मा । कृ॒ष्णग्री॑व॒मिति॑ कृ॒ष्ण - ग्री॒व॒म् । आ ल॑भेत । ल॒भे॒त॒ सौ॒म्यम् । सौ॒म्यम् ब॒भ्रुम् । ब॒भ्रुम् ज्योगा॑मयावी । ज्योगा॑मयाव्य॒ग्निम् । ज्योगा॑माया॒वीति॒ ज्योक् - आ॒म॒या॒वी॒ । अ॒ग्निं ॅवै । वा ए॒तस्य॑ । ए॒तस्य॒ शरी॑रम् । शरी॑रम् गच्छति । ग॒च्छ॒ति॒ सोम᳚म् । सोमꣳ॒॒ रसः॑ । रसो॒ यस्य॑ । यस्य॒ ज्योक् । ज्योगा॒मय॑ति । आ॒मय॑त्य॒ग्नेः । अ॒ग्नेरे॒व । ए॒वास्य॑ । अ॒स्य॒ शरी॑रम् । शरी॑रम् निष्क्री॒णाति॑ । नि॒ष्क्री॒णाति॒ सोमा᳚त् । नि॒ष्क्री॒णातीति॑ निः - क्री॒णाति॑ । सोमा॒द् रस᳚म् । रस॑मु॒त । उ॒त यदि॑ । यदी॒तासुः॑ । इ॒तासु॒र्,भव॑ति । इ॒तासु॒रिती॒त - अ॒सुः॒ । भव॑ति॒ जीव॑ति । जीव॑त्ये॒व । ए॒व सौ॒म्यम् । सौ॒म्यम् ब॒भ्रुम् । ब॒भ्रुमा । आ ल॑भेत । ल॒भे॒ता॒ग्ने॒यम् । आ॒ग्ने॒यम् कृ॒ष्णग्री॑वम् । कृ॒ष्णग्री॑वम् प्र॒जाका॑मः । कृ॒ष्णग्री॑व॒मिति॑ कृ॒ष्ण - ग्री॒व॒म् । प्र॒जाका॑मः॒ सोमः॑ । प्र॒जाका॑म॒ इति॑ प्र॒जा - का॒मः॒ । सोमो॒ वै \newline

\textbf{Jatai Paata} \newline

1. वाच॑म् दधाति दधाति॒ वाचं॒ ॅवाच॑म् दधाति । \newline
2. द॒धा॒ति॒ प्र॒व॒दि॒ता प्र॑वदि॒ता द॑धाति दधाति प्रवदि॒ता । \newline
3. प्र॒व॒दि॒ता वा॒चो वा॒चः प्र॑वदि॒ता प्र॑वदि॒ता वा॒चः । \newline
4. प्र॒व॒दि॒तेति॑ प्र - व॒दि॒ता । \newline
5. वा॒चो भ॑वति भवति वा॒चो वा॒चो भ॑वति । \newline
6. भ॒व॒ त्यप॑न्नद॒ त्यप॑न्नदती भवति भव॒ त्यप॑न्नदती । \newline
7. अप॑न्नदती भवति भव॒ त्यप॑न्नद॒ त्यप॑न्नदती भवति । \newline
8. अप॑न्नद॒तीत्यप॑न्न - द॒ती॒ । \newline
9. भ॒व॒ति॒ तस्मा॒त् तस्मा᳚द् भवति भवति॒ तस्मा᳚त् । \newline
10. तस्मा᳚न् मनु॒ष्या॑ मनु॒ष्या᳚ स्तस्मा॒त् तस्मा᳚न् मनु॒ष्याः᳚ । \newline
11. म॒नु॒ष्याः᳚ सर्वाꣳ॒॒ सर्वा᳚म् मनु॒ष्या॑ मनु॒ष्याः᳚ सर्वा᳚म् । \newline
12. सर्वां॒ ॅवाचं॒ ॅवाचꣳ॒॒ सर्वाꣳ॒॒ सर्वां॒ ॅवाच᳚म् । \newline
13. वाचं॑ ॅवदन्ति वदन्ति॒ वाचं॒ ॅवाचं॑ ॅवदन्ति । \newline
14. व॒द॒ न्त्या॒ग्ने॒य मा᳚ग्ने॒यं ॅव॑दन्ति वद न्त्याग्ने॒यम् । \newline
15. आ॒ग्ने॒यम् कृ॒ष्णग्री॑वम् कृ॒ष्णग्री॑व माग्ने॒य मा᳚ग्ने॒यम् कृ॒ष्णग्री॑वम् । \newline
16. कृ॒ष्णग्री॑व॒ मा कृ॒ष्णग्री॑वम् कृ॒ष्णग्री॑व॒ मा । \newline
17. कृ॒ष्णग्री॑व॒मिति॑ कृ॒ष्ण - ग्री॒व॒म् । \newline
18. आ ल॑भेत लभे॒ता ल॑भेत । \newline
19. ल॒भे॒त॒ सौ॒म्यꣳ सौ॒म्यम् ॅल॑भेत लभेत सौ॒म्यम् । \newline
20. सौ॒म्यम् ब॒भ्रुम् ब॒भ्रुꣳ सौ॒म्यꣳ सौ॒म्यम् ब॒भ्रुम् । \newline
21. ब॒भ्रुम् ज्योगा॑मयावी॒ ज्योगा॑मयावी ब॒भ्रुम् ब॒भ्रुम् ज्योगा॑मयावी । \newline
22. ज्योगा॑मयाव्य॒ग्नि म॒ग्निम् ज्योगा॑मयावी॒ ज्योगा॑मयाव्य॒ग्निम् । \newline
23. ज्योगा॑मया॒वीति॒ ज्योक् - आ॒म॒या॒वी॒ । \newline
24. अ॒ग्निं ॅवै वा अ॒ग्नि म॒ग्निं ॅवै । \newline
25. वा ए॒त स्यै॒तस्य॒ वै वा ए॒तस्य॑ । \newline
26. ए॒तस्य॒ शरी॑रꣳ॒॒ शरी॑र मे॒त स्यै॒तस्य॒ शरी॑रम् । \newline
27. शरी॑रम् गच्छति गच्छति॒ शरी॑रꣳ॒॒ शरी॑रम् गच्छति । \newline
28. ग॒च्छ॒ति॒ सोमꣳ॒॒ सोम॑म् गच्छति गच्छति॒ सोम᳚म् । \newline
29. सोमꣳ॒॒ रसो॒ रसः॒ सोमꣳ॒॒ सोमꣳ॒॒ रसः॑ । \newline
30. रसो॒ यस्य॒ यस्य॒ रसो॒ रसो॒ यस्य॑ । \newline
31. यस्य॒ ज्योग् ज्योग् यस्य॒ यस्य॒ ज्योक् । \newline
32. ज्योगा॒मय॑ त्या॒मय॑ति॒ ज्योग् ज्योगा॒मय॑ति । \newline
33. आ॒मय॑ त्य॒ग्ने र॒ग्ने रा॒मय॑ त्या॒मय॑ त्य॒ग्नेः । \newline
34. अ॒ग्ने रे॒वैवाग्ने र॒ग्ने रे॒व । \newline
35. ए॒वास्या᳚ स्यै॒वैवास्य॑ । \newline
36. अ॒स्य॒ शरी॑रꣳ॒॒ शरी॑र मस्यास्य॒ शरी॑रम् । \newline
37. शरी॑रम् निष्क्री॒णाति॑ निष्क्री॒णाति॒ शरी॑रꣳ॒॒ शरी॑रम् निष्क्री॒णाति॑ । \newline
38. नि॒ष्क्री॒णाति॒ सोमा॒थ् सोमा᳚न् निष्क्री॒णाति॑ निष्क्री॒णाति॒ सोमा᳚त् । \newline
39. नि॒ष्क्री॒णातीति॑ निः - क्री॒णाति॑ । \newline
40. सोमा॒द् रसꣳ॒॒ रसꣳ॒॒ सोमा॒थ् सोमा॒द् रस᳚म् । \newline
41. रस॑ मु॒तोत रसꣳ॒॒ रस॑ मु॒त । \newline
42. उ॒त यदि॒ यद्यु॒तोत यदि॑ । \newline
43. यदी॒तासु॑ रि॒तासु॒र् यदि॒ यदी॒तासुः॑ । \newline
44. इ॒तासु॒र् भव॑ति॒ भव॑ती॒तासु॑ रि॒तासु॒र् भव॑ति । \newline
45. इ॒तासु॒रिती॒त - अ॒सुः॒ । \newline
46. भव॑ति॒ जीव॑ति॒ जीव॑ति॒ भव॑ति॒ भव॑ति॒ जीव॑ति । \newline
47. जीव॑ त्ये॒वैव जीव॑ति॒ जीव॑ त्ये॒व । \newline
48. ए॒व सौ॒म्यꣳ सौ॒म्य मे॒वैव सौ॒म्यम् । \newline
49. सौ॒म्यम् ब॒भ्रुम् ब॒भ्रुꣳ सौ॒म्यꣳ सौ॒म्यम् ब॒भ्रुम् । \newline
50. ब॒भ्रु मा ब॒भ्रुम् ब॒भ्रु मा । \newline
51. आ ल॑भेत लभे॒ता ल॑भेत । \newline
52. ल॒भे॒ता॒ग्ने॒य मा᳚ग्ने॒यम् ॅल॑भेत लभेताग्ने॒यम् । \newline
53. आ॒ग्ने॒यम् कृ॒ष्णग्री॑वम् कृ॒ष्णग्री॑व माग्ने॒य मा᳚ग्ने॒यम् कृ॒ष्णग्री॑वम् । \newline
54. कृ॒ष्णग्री॑वम् प्र॒जाका॑मः प्र॒जाका॑मः कृ॒ष्णग्री॑वम् कृ॒ष्णग्री॑वम् प्र॒जाका॑मः । \newline
55. कृ॒ष्णग्री॑व॒मिति॑ कृ॒ष्ण - ग्री॒व॒म् । \newline
56. प्र॒जाका॑मः॒ सोमः॒ सोमः॑ प्र॒जाका॑मः प्र॒जाका॑मः॒ सोमः॑ । \newline
57. प्र॒जाका॑म॒ इति॑ प्र॒जा - का॒मः॒ । \newline
58. सोमो॒ वै वै सोमः॒ सोमो॒ वै । \newline

\textbf{Ghana Paata } \newline

1. वाच॑म् दधाति दधाति॒ वाचं॒ ॅवाच॑म् दधाति प्रवदि॒ता प्र॑वदि॒ता द॑धाति॒ वाचं॒ ॅवाच॑म् दधाति प्रवदि॒ता । \newline
2. द॒धा॒ति॒ प्र॒व॒दि॒ता प्र॑वदि॒ता द॑धाति दधाति प्रवदि॒ता वा॒चो वा॒चः प्र॑वदि॒ता द॑धाति दधाति प्रवदि॒ता वा॒चः । \newline
3. प्र॒व॒दि॒ता वा॒चो वा॒चः प्र॑वदि॒ता प्र॑वदि॒ता वा॒चो भ॑वति भवति वा॒चः प्र॑वदि॒ता प्र॑वदि॒ता वा॒चो भ॑वति । \newline
4. प्र॒व॒दि॒तेति॑ प्र - व॒दि॒ता । \newline
5. वा॒चो भ॑वति भवति वा॒चो वा॒चो भ॑व॒ त्यप॑न्नद॒ त्यप॑न्नदती भवति वा॒चो वा॒चो भ॑व॒त्यप॑न्नदती । \newline
6. भ॒व॒ त्यप॑न्नद॒ त्यप॑न्नदती भवति भव॒ त्यप॑न्नदती भवति भव॒ त्यप॑न्नदती भवति भव॒ त्यप॑न्नदती भवति । \newline
7. अप॑न्नदती भवति भव॒ त्यप॑न्नद॒ त्यप॑न्नदती भवति॒ तस्मा॒त् तस्मा᳚द् भव॒ त्यप॑न्नद॒ त्यप॑न्नदती भवति॒ तस्मा᳚त् । \newline
8. अप॑न्नद॒तीत्यप॑न्न - द॒ती॒ । \newline
9. भ॒व॒ति॒ तस्मा॒त् तस्मा᳚द् भवति भवति॒ तस्मा᳚न् मनु॒ष्या॑ मनु॒ष्या᳚ स्तस्मा᳚द् भवति भवति॒ तस्मा᳚न् मनु॒ष्याः᳚ । \newline
10. तस्मा᳚न् मनु॒ष्या॑ मनु॒ष्या᳚ स्तस्मा॒त् तस्मा᳚न् मनु॒ष्याः᳚ सर्वाꣳ॒॒ सर्वा᳚म् मनु॒ष्या᳚ स्तस्मा॒त् तस्मा᳚न् मनु॒ष्याः᳚ सर्वा᳚म् । \newline
11. म॒नु॒ष्याः᳚ सर्वाꣳ॒॒ सर्वा᳚म् मनु॒ष्या॑ मनु॒ष्याः᳚ सर्वां॒ ॅवाचं॒ ॅवाचꣳ॒॒ सर्वा᳚म् मनु॒ष्या॑ मनु॒ष्याः᳚ सर्वां॒ ॅवाच᳚म् । \newline
12. सर्वां॒ ॅवाचं॒ ॅवाचꣳ॒॒ सर्वाꣳ॒॒ सर्वां॒ ॅवाचं॑ ॅवदन्ति वदन्ति॒ वाचꣳ॒॒ सर्वाꣳ॒॒ सर्वां॒ ॅवाचं॑ ॅवदन्ति । \newline
13. वाचं॑ ॅवदन्ति वदन्ति॒ वाचं॒ ॅवाचं॑ ॅवदन्त्याग्ने॒य मा᳚ग्ने॒यं ॅव॑दन्ति॒ वाचं॒ ॅवाचं॑ ॅवदन्त्याग्ने॒यम् । \newline
14. व॒द॒ न्त्या॒ग्ने॒य मा᳚ग्ने॒यं ॅव॑दन्ति वद न्त्याग्ने॒यम् कृ॒ष्णग्री॑वम् कृ॒ष्णग्री॑व माग्ने॒यं ॅव॑दन्ति वदन् त्याग्ने॒यम् कृ॒ष्णग्री॑वम् । \newline
15. आ॒ग्ने॒यम् कृ॒ष्णग्री॑वम् कृ॒ष्णग्री॑व माग्ने॒य मा᳚ग्ने॒यम् कृ॒ष्णग्री॑व॒ मा कृ॒ष्णग्री॑व माग्ने॒य मा᳚ग्ने॒यम् कृ॒ष्णग्री॑व॒ मा । \newline
16. कृ॒ष्णग्री॑व॒ मा कृ॒ष्णग्री॑वम् कृ॒ष्णग्री॑व॒ मा ल॑भेत लभे॒ता कृ॒ष्णग्री॑वम् कृ॒ष्णग्री॑व॒ मा ल॑भेत । \newline
17. कृ॒ष्णग्री॑व॒मिति॑ कृ॒ष्ण - ग्री॒व॒म् । \newline
18. आ ल॑भेत लभे॒ता ल॑भेत सौ॒म्यꣳ सौ॒म्यम् ॅल॑भे॒ता ल॑भेत सौ॒म्यम् । \newline
19. ल॒भे॒त॒ सौ॒म्यꣳ सौ॒म्यम् ॅल॑भेत लभेत सौ॒म्यम् ब॒भ्रुम् ब॒भ्रुꣳ सौ॒म्यम् ॅल॑भेत लभेत सौ॒म्यम् ब॒भ्रुम् । \newline
20. सौ॒म्यम् ब॒भ्रुम् ब॒भ्रुꣳ सौ॒म्यꣳ सौ॒म्यम् ब॒भ्रुम् ज्योगा॑मयावी॒ ज्योगा॑मयावी ब॒भ्रुꣳ सौ॒म्यꣳ सौ॒म्यम् ब॒भ्रुम् ज्योगा॑मयावी । \newline
21. ब॒भ्रुम् ज्योगा॑मयावी॒ ज्योगा॑मयावी ब॒भ्रुम् ब॒भ्रुम् ज्योगा॑मया व्य॒ग्नि म॒ग्निम् ज्योगा॑मयावी ब॒भ्रुम् ब॒भ्रुम् ज्योगा॑मया व्य॒ग्निम् । \newline
22. ज्योगा॑मया व्य॒ग्नि म॒ग्निम् ज्योगा॑मयावी॒ ज्योगा॑मया व्य॒ग्निं ॅवै वा अ॒ग्निम् ज्योगा॑मयावी॒ ज्योगा॑मया व्य॒ग्निं ॅवै । \newline
23. ज्योगा॑मया॒वीति॒ ज्योक् - आ॒म॒या॒वी॒ । \newline
24. अ॒ग्निं ॅवै वा अ॒ग्नि म॒ग्निं ॅवा ए॒त स्यै॒तस्य॒ वा अ॒ग्नि म॒ग्निं ॅवा ए॒तस्य॑ । \newline
25. वा ए॒त स्यै॒तस्य॒ वै वा ए॒तस्य॒ शरी॑रꣳ॒॒ शरी॑र मे॒तस्य॒ वै वा ए॒तस्य॒ शरी॑रम् । \newline
26. ए॒तस्य॒ शरी॑रꣳ॒॒ शरी॑र मे॒त स्यै॒तस्य॒ शरी॑रम् गच्छति गच्छति॒ शरी॑र मे॒त स्यै॒तस्य॒ शरी॑रम् गच्छति । \newline
27. शरी॑रम् गच्छति गच्छति॒ शरी॑रꣳ॒॒ शरी॑रम् गच्छति॒ सोमꣳ॒॒ सोम॑म् गच्छति॒ शरी॑रꣳ॒॒ शरी॑रम् गच्छति॒ सोम᳚म् । \newline
28. ग॒च्छ॒ति॒ सोमꣳ॒॒ सोम॑म् गच्छति गच्छति॒ सोमꣳ॒॒ रसो॒ रसः॒ सोम॑म् गच्छति गच्छति॒ सोमꣳ॒॒ रसः॑ । \newline
29. सोमꣳ॒॒ रसो॒ रसः॒ सोमꣳ॒॒ सोमꣳ॒॒ रसो॒ यस्य॒ यस्य॒ रसः॒ सोमꣳ॒॒ सोमꣳ॒॒ रसो॒ यस्य॑ । \newline
30. रसो॒ यस्य॒ यस्य॒ रसो॒ रसो॒ यस्य॒ ज्योग् ज्योग् यस्य॒ रसो॒ रसो॒ यस्य॒ ज्योक् । \newline
31. यस्य॒ ज्योग् ज्योग् यस्य॒ यस्य॒ ज्योगा॒मय॑ त्या॒मय॑ति॒ ज्योग् यस्य॒ यस्य॒ ज्योगा॒मय॑ति । \newline
32. ज्योगा॒मय॑ त्या॒मय॑ति॒ ज्योग् ज्योगा॒मय॑ त्य॒ग्ने र॒ग्ने रा॒मय॑ति॒ ज्योग् ज्योगा॒मय॑ त्य॒ग्नेः । \newline
33. आ॒मय॑ त्य॒ग्ने र॒ग्ने रा॒मय॑ त्या॒मय॑ त्य॒ग्ने रे॒वै वाग्ने रा॒मय॑ त्या॒मय॑ त्य॒ग्ने रे॒व । \newline
34. अ॒ग्ने रे॒वै वाग्ने र॒ग्ने रे॒वा स्या᳚स्यै॒वाग्ने र॒ग्ने रे॒वास्य॑ । \newline
35. ए॒वास्या᳚ स्यै॒वैवास्य॒ शरी॑रꣳ॒॒ शरी॑र मस्यै॒ वैवास्य॒ शरी॑रम् । \newline
36. अ॒स्य॒ शरी॑रꣳ॒॒ शरी॑र मस्यास्य॒ शरी॑रन् निष्क्री॒णाति॑ निष्क्री॒णाति॒ शरी॑र मस्यास्य॒ शरी॑रन् निष्क्री॒णाति॑ । \newline
37. शरी॑रन् निष्क्री॒णाति॑ निष्क्री॒णाति॒ शरी॑रꣳ॒॒ शरी॑रन् निष्क्री॒णाति॒ सोमा॒थ् सोमा᳚न् निष्क्री॒णाति॒ शरी॑रꣳ॒॒ शरी॑रन् निष्क्री॒णाति॒ सोमा᳚त् । \newline
38. नि॒ष्क्री॒णाति॒ सोमा॒थ् सोमा᳚न् निष्क्री॒णाति॑ निष्क्री॒णाति॒ सोमा॒द् रसꣳ॒॒ रसꣳ॒॒ सोमा᳚न् निष्क्री॒णाति॑ निष्क्री॒णाति॒ सोमा॒द् रस᳚म् । \newline
39. नि॒ष्क्री॒णातीति॑ निः - क्री॒णाति॑ । \newline
40. सोमा॒द् रसꣳ॒॒ रसꣳ॒॒ सोमा॒थ् सोमा॒द् रस॑ मु॒तोत रसꣳ॒॒ सोमा॒थ् सोमा॒द् रस॑ मु॒त । \newline
41. रस॑ मु॒तोत रसꣳ॒॒ रस॑ मु॒त यदि॒ यद्यु॒त रसꣳ॒॒ रस॑ मु॒त यदि॑ । \newline
42. उ॒त यदि॒ यद्यु॒तोत यदी॒तासु॑ रि॒तासु॒र् यद्यु॒तोत यदी॒तासुः॑ । \newline
43. यदी॒तासु॑ रि॒तासु॒र् यदि॒ यदी॒तासु॒र् भव॑ति॒ भव॑ती॒तासु॒र् यदि॒ यदी॒तासु॒र् भव॑ति । \newline
44. इ॒तासु॒र् भव॑ति॒ भव॑ती॒तासु॑ रि॒तासु॒र् भव॑ति॒ जीव॑ति॒ जीव॑ति॒ भव॑ती॒तासु॑ रि॒तासु॒र् भव॑ति॒ जीव॑ति । \newline
45. इ॒तासु॒रिती॒त - अ॒सुः॒ । \newline
46. भव॑ति॒ जीव॑ति॒ जीव॑ति॒ भव॑ति॒ भव॑ति॒ जीव॑त्ये॒वैव जीव॑ति॒ भव॑ति॒ भव॑ति॒ जीव॑त्ये॒व । \newline
47. जीव॑त्ये॒वैव जीव॑ति॒ जीव॑त्ये॒व सौ॒म्यꣳ सौ॒म्य मे॒व जीव॑ति॒ जीव॑त्ये॒व सौ॒म्यम् । \newline
48. ए॒व सौ॒म्यꣳ सौ॒म्य मे॒वैव सौ॒म्यम् ब॒भ्रुम् ब॒भ्रुꣳ सौ॒म्य मे॒वैव सौ॒म्यम् ब॒भ्रुम् । \newline
49. सौ॒म्यम् ब॒भ्रुम् ब॒भ्रुꣳ सौ॒म्यꣳ सौ॒म्यम् ब॒भ्रु मा ब॒भ्रुꣳ सौ॒म्यꣳ सौ॒म्यम् ब॒भ्रु मा । \newline
50. ब॒भ्रु मा ब॒भ्रुम् ब॒भ्रु मा ल॑भेत लभे॒ता ब॒भ्रुम् ब॒भ्रु मा ल॑भेत । \newline
51. आ ल॑भेत लभे॒ता ल॑भेताग्ने॒य मा᳚ग्ने॒यम् ॅल॑भे॒ता ल॑भेताग्ने॒यम् । \newline
52. ल॒भे॒ता॒ग्ने॒य मा᳚ग्ने॒यम् ॅल॑भेत लभेताग्ने॒यम् कृ॒ष्णग्री॑वम् कृ॒ष्णग्री॑व माग्ने॒यम् ॅल॑भेत लभेताग्ने॒यम् कृ॒ष्णग्री॑वम् । \newline
53. आ॒ग्ने॒यम् कृ॒ष्णग्री॑वम् कृ॒ष्णग्री॑व माग्ने॒य मा᳚ग्ने॒यम् कृ॒ष्णग्री॑वम् प्र॒जाका॑मः प्र॒जाका॑मः कृ॒ष्णग्री॑व माग्ने॒य मा᳚ग्ने॒यम् कृ॒ष्णग्री॑वम् प्र॒जाका॑मः । \newline
54. कृ॒ष्णग्री॑वम् प्र॒जाका॑मः प्र॒जाका॑मः कृ॒ष्णग्री॑वम् कृ॒ष्णग्री॑वम् प्र॒जाका॑मः॒ सोमः॒ सोमः॑ प्र॒जाका॑मः कृ॒ष्णग्री॑वम् कृ॒ष्णग्री॑वम् प्र॒जाका॑मः॒ सोमः॑ । \newline
55. कृ॒ष्णग्री॑व॒मिति॑ कृ॒ष्ण - ग्री॒व॒म् । \newline
56. प्र॒जाका॑मः॒ सोमः॒ सोमः॑ प्र॒जाका॑मः प्र॒जाका॑मः॒ सोमो॒ वै वै सोमः॑ प्र॒जाका॑मः प्र॒जाका॑मः॒ सोमो॒ वै । \newline
57. प्र॒जाका॑म॒ इति॑ प्र॒जा - का॒मः॒ । \newline
58. सोमो॒ वै वै सोमः॒ सोमो॒ वै रे॑तो॒धा रे॑तो॒धा वै सोमः॒ सोमो॒ वै रे॑तो॒धाः । \newline
\pagebreak
\markright{ TS 2.1.2.8  \hfill https://www.vedavms.in \hfill}

\section{ TS 2.1.2.8 }

\textbf{TS 2.1.2.8 } \newline
\textbf{Samhita Paata} \newline

वै रे॑तो॒धा अ॒ग्निः प्र॒जानां᳚ प्रजनयि॒ता सोम॑ ए॒वास्मै॒ रेतो॒ दधा᳚त्य॒ग्निः प्र॒जां प्रज॑नयति वि॒न्दते᳚ प्र॒जामा᳚ग्ने॒यं कृ॒ष्णग्री॑व॒मा ल॑भेत सौ॒म्यं ब॒भ्रुं ॅयो ब्रा᳚ह्म॒णो वि॒द्याम॒नूच्य॒ न वि॒रोचे॑त॒ यदा᳚ग्ने॒यो भव॑ति॒ तेज॑ ए॒वास्मि॒न् तेन॑ दधाति॒ यथ् सौ॒म्यो ब्र॑ह्मवर्च॒सं तेन॑ कृ॒ष्णग्री॑व आग्ने॒यो भ॑वति॒ तम॑ ए॒वास्मा॒दप॑ हन्ति श्वे॒तो भ॑वति॒ - [  ] \newline

\textbf{Pada Paata} \newline

वै । रे॒तो॒धा इति॑ रेतः - धाः । अ॒ग्निः । प्र॒जाना॒मिति॑ प्र - जाना᳚म् । प्र॒ज॒न॒यि॒तेति॑ प्र-ज॒न॒यि॒ता । सोमः॑ । ए॒व । अ॒स्मै॒ । रेतः॑ । दधा॑ति । अ॒ग्निः । प्र॒जामिति॑ प्र-जाम् । प्रेति॑ । ज॒न॒य॒ति॒ । वि॒न्दते᳚ । प्र॒जामिति॑ प्र-जाम् । आ॒ग्ने॒यम् । कृ॒ष्णग्री॑व॒मिति॑ कृ॒ष्ण - ग्री॒व॒म् । एति॑ । ल॒भे॒त॒ । सौ॒म्यम् । ब॒भ्रुम् । यः । ब्रा॒ह्म॒णः । वि॒द्याम् । अ॒नूच्येत्य॑नु - उच्य॑ । न । वि॒रोचे॒तेति॑ वि - रोचे॑त । यत् । आ॒ग्ने॒यः । भव॑ति । तेजः॑ । ए॒व । अ॒स्मि॒न्न् । तेन॑ । द॒धा॒ति॒ । यत् । सौ॒म्यः । ब्र॒ह्म॒व॒र्च॒समिति॑ ब्रह्म - व॒र्च॒सम् । तेन॑ । कृ॒ष्णग्री॑व॒ इति॑ कृ॒ष्ण - ग्री॒वः॒ । आ॒ग्ने॒यः । भ॒व॒ति॒ । तमः॑ । ए॒व । अ॒स्मा॒त् । अपेति॑ । ह॒न्ति॒ । श्वे॒तः । भ॒व॒ति॒ ।  \newline


\textbf{Krama Paata} \newline

वै रे॑तो॒धाः । रे॒तो॒धा अ॒ग्निः । रे॒तो॒धा इति॑ रेतः - धाः । अ॒ग्निः प्र॒जाना᳚म् । प्र॒जाना᳚म् प्रजनयि॒ता । प्र॒जाना॒मिति॑ प्र - जाना᳚म् । प्र॒ज॒न॒यि॒ता सोमः॑ । प्र॒ज॒न॒यि॒तेति॑ प्र - ज॒न॒यि॒ता । सोम॑ ए॒व । ए॒वास्मै᳚ । अ॒स्मै॒ रेतः॑ । रेतो॒ दधा॑ति । दधा᳚त्य॒ग्निः । अ॒ग्निः प्र॒जाम् । प्र॒जाम् प्र । प्र॒जामिति॑ प्र - जाम् । प्र ज॑नयति । ज॒न॒य॒ति॒ वि॒न्दते᳚ । वि॒न्दते᳚ प्र॒जाम् । प्र॒जामा᳚ग्ने॒यम् । प्र॒जामिति॑ प्र - जाम् । आ॒ग्ने॒यम् कृ॒ष्णग्री॑वम् । कृ॒ष्णग्री॑व॒मा । कृ॒ष्णग्री॑व॒मिति॑ कृ॒ष्ण - ग्री॒व॒म् । आ ल॑भेत । ल॒भे॒त॒ सौ॒म्यम् । सौ॒म्यम् ब॒भ्रुम् । ब॒भ्रुं ॅयः । यो ब्रा᳚ह्म॒णः । ब्रा॒ह्म॒णो वि॒द्याम् । वि॒द्याम॒नूच्य॑ । अ॒नूच्य॒ न । अ॒नूच्येत्य॑नु - उच्य॑ । न वि॒रोचे॑त । वि॒रोचे॑त॒ यत् । वि॒रोचे॒तेति॑ वि - रोचे॑त । यदा᳚ग्ने॒यः । आ॒ग्ने॒यो भव॑ति । भव॑ति॒ तेजः॑ । तेज॑ ए॒व । ए॒वास्मिन्न्॑ । अ॒स्मि॒न् तेन॑ । तेन॑ दधाति । द॒धा॒ति॒ यत् । यथ् सौ॒म्यः । सौ॒म्यो ब्र॑ह्मवर्च॒सम् । ब्र॒ह्म॒व॒र्च॒सम् तेन॑ । ब्र॒ह्म॒व॒र्च॒समिति॑ ब्रह्म - व॒र्च॒सम् । तेन॑ कृ॒ष्णग्री॑वः । कृ॒ष्णग्री॑व आग्ने॒यः । कृ॒ष्णग्री॑व॒ इति॑ कृ॒ष्ण - ग्री॒वः॒ । आ॒ग्ने॒यो भ॑वति । भ॒व॒ति॒ तमः॑ । तम॑ ए॒व । ए॒वास्मा᳚त् । अ॒स्मा॒दप॑ । अप॑ हन्ति । ह॒न्ति॒ श्वे॒तः । श्वे॒तो भ॑वति ( ) । भ॒व॒ति॒ रुच᳚म् \newline

\textbf{Jatai Paata} \newline

1. वै रे॑तो॒धा रे॑तो॒धा वै वै रे॑तो॒धाः । \newline
2. रे॒तो॒धा अ॒ग्नि र॒ग्नी रे॑तो॒धा रे॑तो॒धा अ॒ग्निः । \newline
3. रे॒तो॒धा इति॑ रेतः - धाः । \newline
4. अ॒ग्निः प्र॒जाना᳚म् प्र॒जाना॑ म॒ग्नि र॒ग्निः प्र॒जाना᳚म् । \newline
5. प्र॒जाना᳚म् प्रजनयि॒ता प्र॑जनयि॒ता प्र॒जाना᳚म् प्र॒जाना᳚म् प्रजनयि॒ता । \newline
6. प्र॒जाना॒मिति॑ प्र - जाना᳚म् । \newline
7. प्र॒ज॒न॒यि॒ता सोमः॒ सोमः॑ प्रजनयि॒ता प्र॑जनयि॒ता सोमः॑ । \newline
8. प्र॒ज॒न॒यि॒तेति॑ प्र - ज॒न॒यि॒ता । \newline
9. सोम॑ ए॒वैव सोमः॒ सोम॑ ए॒व । \newline
10. ए॒वास्मा॑ अस्मा ए॒वैवास्मै᳚ । \newline
11. अ॒स्मै॒ रेतो॒ रेतो᳚ ऽस्मा अस्मै॒ रेतः॑ । \newline
12. रेतो॒ दधा॑ति॒ दधा॑ति॒ रेतो॒ रेतो॒ दधा॑ति । \newline
13. दधा᳚ त्य॒ग्नि र॒ग्निर् दधा॑ति॒ दधा᳚ त्य॒ग्निः । \newline
14. अ॒ग्निः प्र॒जाम् प्र॒जा म॒ग्नि र॒ग्निः प्र॒जाम् । \newline
15. प्र॒जाम् प्र प्र प्र॒जाम् प्र॒जाम् प्र । \newline
16. प्र॒जामिति॑ प्र - जाम् । \newline
17. प्र ज॑नयति जनयति॒ प्र प्र ज॑नयति । \newline
18. ज॒न॒य॒ति॒ वि॒न्दते॑ वि॒न्दते॑ जनयति जनयति वि॒न्दते᳚ । \newline
19. वि॒न्दते᳚ प्र॒जाम् प्र॒जां ॅवि॒न्दते॑ वि॒न्दते᳚ प्र॒जाम् । \newline
20. प्र॒जा मा᳚ग्ने॒य मा᳚ग्ने॒यम् प्र॒जाम् प्र॒जा मा᳚ग्ने॒यम् । \newline
21. प्र॒जामिति॑ प्र - जाम् । \newline
22. आ॒ग्ने॒यम् कृ॒ष्णग्री॑वम् कृ॒ष्णग्री॑व माग्ने॒य मा᳚ग्ने॒यम् कृ॒ष्णग्री॑वम् । \newline
23. कृ॒ष्णग्री॑व॒ मा कृ॒ष्णग्री॑वम् कृ॒ष्णग्री॑व॒ मा । \newline
24. कृ॒ष्णग्री॑व॒मिति॑ कृ॒ष्ण - ग्री॒व॒म् । \newline
25. आ ल॑भेत लभे॒ता ल॑भेत । \newline
26. ल॒भे॒त॒ सौ॒म्यꣳ सौ॒म्यम् ॅल॑भेत लभेत सौ॒म्यम् । \newline
27. सौ॒म्यम् ब॒भ्रुम् ब॒भ्रुꣳ सौ॒म्यꣳ सौ॒म्यम् ब॒भ्रुम् । \newline
28. ब॒भ्रुं ॅयो यो ब॒भ्रुम् ब॒भ्रुं ॅयः । \newline
29. यो ब्रा᳚ह्म॒णो ब्रा᳚ह्म॒णो यो यो ब्रा᳚ह्म॒णः । \newline
30. ब्रा॒ह्म॒णो वि॒द्यां ॅवि॒द्याम् ब्रा᳚ह्म॒णो ब्रा᳚ह्म॒णो वि॒द्याम् । \newline
31. वि॒द्या म॒नूच्या॒ नूच्य॑ वि॒द्यां ॅवि॒द्या म॒नूच्य॑ । \newline
32. अ॒नूच्य॒ न नानूच्या॒ नूच्य॒ न । \newline
33. अ॒नूच्येत्य॑नु - उच्य॑ । \newline
34. न वि॒रोचे॑त वि॒रोचे॑त॒ न न वि॒रोचे॑त । \newline
35. वि॒रोचे॑त॒ यद् यद् वि॒रोचे॑त वि॒रोचे॑त॒ यत् । \newline
36. वि॒रोचे॒तेति॑ वि - रोचे॑त । \newline
37. यदा᳚ग्ने॒य आ᳚ग्ने॒यो यद् यदा᳚ग्ने॒यः । \newline
38. आ॒ग्ने॒यो भव॑ति॒ भव॑ त्याग्ने॒य आ᳚ग्ने॒यो भव॑ति । \newline
39. भव॑ति॒ तेज॒ स्तेजो॒ भव॑ति॒ भव॑ति॒ तेजः॑ । \newline
40. तेज॑ ए॒वैव तेज॒ स्तेज॑ ए॒व । \newline
41. ए॒वास्मि॑न् नस्मिन् ने॒वैवास्मिन्न्॑ । \newline
42. अ॒स्मि॒न् तेन॒ तेना᳚स्मिन् नस्मि॒न् तेन॑ । \newline
43. तेन॑ दधाति दधाति॒ तेन॒ तेन॑ दधाति । \newline
44. द॒धा॒ति॒ यद् यद् द॑धाति दधाति॒ यत् । \newline
45. यथ् सौ॒म्यः सौ॒म्यो यद् यथ् सौ॒म्यः । \newline
46. सौ॒म्यो ब्र॑ह्मवर्च॒सम् ब्र॑ह्मवर्च॒सꣳ सौ॒म्यः सौ॒म्यो ब्र॑ह्मवर्च॒सम् । \newline
47. ब्र॒ह्म॒व॒र्च॒सम् तेन॒ तेन॑ ब्रह्मवर्च॒सम् ब्र॑ह्मवर्च॒सम् तेन॑ । \newline
48. ब्र॒ह्म॒व॒र्च॒समिति॑ ब्रह्म - व॒र्च॒सम् । \newline
49. तेन॑ कृ॒ष्णग्री॑वः कृ॒ष्णग्री॑व॒ स्तेन॒ तेन॑ कृ॒ष्णग्री॑वः । \newline
50. कृ॒ष्णग्री॑व आग्ने॒य आ᳚ग्ने॒यः कृ॒ष्णग्री॑वः कृ॒ष्णग्री॑व आग्ने॒यः । \newline
51. कृ॒ष्णग्री॑व॒ इति॑ कृ॒ष्ण - ग्री॒वः॒ । \newline
52. आ॒ग्ने॒यो भ॑वति भवत्याग्ने॒य आ᳚ग्ने॒यो भ॑वति । \newline
53. भ॒व॒ति॒ तम॒ स्तमो॑ भवति भवति॒ तमः॑ । \newline
54. तम॑ ए॒वैव तम॒ स्तम॑ ए॒व । \newline
55. ए॒वास्मा॑ दस्मा दे॒वैवास्मा᳚त् । \newline
56. अ॒स्मा॒ दपापा᳚स्मा दस्मा॒ दप॑ । \newline
57. अप॑ हन्ति ह॒न्त्यपाप॑ हन्ति । \newline
58. ह॒न्ति॒ श्वे॒तः श्वे॒तो ह॑न्ति हन्ति श्वे॒तः । \newline
59. श्वे॒तो भ॑वति भवति श्वे॒तः श्वे॒तो भ॑वति । \newline
60. भ॒व॒ति॒ रुचꣳ॒॒ रुच॑म् भवति भवति॒ रुच᳚म् । \newline

\textbf{Ghana Paata } \newline

1. वै रे॑तो॒धा रे॑तो॒धा वै वै रे॑तो॒धा अ॒ग्नि र॒ग्नी रे॑तो॒धा वै वै रे॑तो॒धा अ॒ग्निः । \newline
2. रे॒तो॒धा अ॒ग्नि र॒ग्नी रे॑तो॒धा रे॑तो॒धा अ॒ग्निः प्र॒जाना᳚म् प्र॒जाना॑ म॒ग्नी रे॑तो॒धा रे॑तो॒धा अ॒ग्निः प्र॒जाना᳚म् । \newline
3. रे॒तो॒धा इति॑ रेतः - धाः । \newline
4. अ॒ग्निः प्र॒जाना᳚म् प्र॒जाना॑ म॒ग्नि र॒ग्निः प्र॒जाना᳚म् प्रजनयि॒ता प्र॑जनयि॒ता प्र॒जाना॑ म॒ग्नि र॒ग्निः प्र॒जाना᳚म् प्रजनयि॒ता । \newline
5. प्र॒जाना᳚म् प्रजनयि॒ता प्र॑जनयि॒ता प्र॒जाना᳚म् प्र॒जाना᳚म् प्रजनयि॒ता सोमः॒ सोमः॑ प्रजनयि॒ता प्र॒जाना᳚म् प्र॒जाना᳚म् प्रजनयि॒ता सोमः॑ । \newline
6. प्र॒जाना॒मिति॑ प्र - जाना᳚म् । \newline
7. प्र॒ज॒न॒यि॒ता सोमः॒ सोमः॑ प्रजनयि॒ता प्र॑जनयि॒ता सोम॑ ए॒वैव सोमः॑ प्रजनयि॒ता प्र॑जनयि॒ता सोम॑ ए॒व । \newline
8. प्र॒ज॒न॒यि॒तेति॑ प्र - ज॒न॒यि॒ता । \newline
9. सोम॑ ए॒वैव सोमः॒ सोम॑ ए॒वास्मा॑ अस्मा ए॒व सोमः॒ सोम॑ ए॒वास्मै᳚ । \newline
10. ए॒वास्मा॑ अस्मा ए॒वैवास्मै॒ रेतो॒ रेतो᳚ ऽस्मा ए॒वैवास्मै॒ रेतः॑ । \newline
11. अ॒स्मै॒ रेतो॒ रेतो᳚ ऽस्मा अस्मै॒ रेतो॒ दधा॑ति॒ दधा॑ति॒ रेतो᳚ ऽस्मा अस्मै॒ रेतो॒ दधा॑ति । \newline
12. रेतो॒ दधा॑ति॒ दधा॑ति॒ रेतो॒ रेतो॒ दधा᳚ त्य॒ग्नि र॒ग्निर् दधा॑ति॒ रेतो॒ रेतो॒ दधा᳚त्य॒ग्निः । \newline
13. दधा᳚ त्य॒ग्नि र॒ग्निर् दधा॑ति॒ दधा᳚त्य॒ग्निः प्र॒जाम् प्र॒जा म॒ग्निर् दधा॑ति॒ दधा᳚त्य॒ग्निः प्र॒जाम् । \newline
14. अ॒ग्निः प्र॒जाम् प्र॒जा म॒ग्नि र॒ग्निः प्र॒जाम् प्र प्र प्र॒जा म॒ग्नि र॒ग्निः प्र॒जाम् प्र । \newline
15. प्र॒जाम् प्र प्र प्र॒जाम् प्र॒जाम् प्र ज॑नयति जनयति॒ प्र प्र॒जाम् प्र॒जाम् प्र ज॑नयति । \newline
16. प्र॒जामिति॑ प्र - जाम् । \newline
17. प्र ज॑नयति जनयति॒ प्र प्र ज॑नयति वि॒न्दते॑ वि॒न्दते॑ जनयति॒ प्र प्र ज॑नयति वि॒न्दते᳚ । \newline
18. ज॒न॒य॒ति॒ वि॒न्दते॑ वि॒न्दते॑ जनयति जनयति वि॒न्दते᳚ प्र॒जाम् प्र॒जां ॅवि॒न्दते॑ जनयति जनयति वि॒न्दते᳚ प्र॒जाम् । \newline
19. वि॒न्दते᳚ प्र॒जाम् प्र॒जां ॅवि॒न्दते॑ वि॒न्दते᳚ प्र॒जा मा᳚ग्ने॒य मा᳚ग्ने॒यम् प्र॒जां ॅवि॒न्दते॑ वि॒न्दते᳚ प्र॒जा मा᳚ग्ने॒यम् । \newline
20. प्र॒जा मा᳚ग्ने॒य मा᳚ग्ने॒यम् प्र॒जाम् प्र॒जा मा᳚ग्ने॒यम् कृ॒ष्णग्री॑वम् कृ॒ष्णग्री॑व माग्ने॒यम् प्र॒जाम् प्र॒जा मा᳚ग्ने॒यम् कृ॒ष्णग्री॑वम् । \newline
21. प्र॒जामिति॑ प्र - जाम् । \newline
22. आ॒ग्ने॒यम् कृ॒ष्णग्री॑वम् कृ॒ष्णग्री॑व माग्ने॒य मा᳚ग्ने॒यम् कृ॒ष्णग्री॑व॒ मा कृ॒ष्णग्री॑व माग्ने॒य मा᳚ग्ने॒यम् कृ॒ष्णग्री॑व॒ मा । \newline
23. कृ॒ष्णग्री॑व॒ मा कृ॒ष्णग्री॑वम् कृ॒ष्णग्री॑व॒ मा ल॑भेत लभे॒ता कृ॒ष्णग्री॑वम् कृ॒ष्णग्री॑व॒ मा ल॑भेत । \newline
24. कृ॒ष्णग्री॑व॒मिति॑ कृ॒ष्ण - ग्री॒व॒म् । \newline
25. आ ल॑भेत लभे॒ता ल॑भेत सौ॒म्यꣳ सौ॒म्यम् ॅल॑भे॒ता ल॑भेत सौ॒म्यम् । \newline
26. ल॒भे॒त॒ सौ॒म्यꣳ सौ॒म्यम् ॅल॑भेत लभेत सौ॒म्यम् ब॒भ्रुम् ब॒भ्रुꣳ सौ॒म्यम् ॅल॑भेत लभेत सौ॒म्यम् ब॒भ्रुम् । \newline
27. सौ॒म्यम् ब॒भ्रुम् ब॒भ्रुꣳ सौ॒म्यꣳ सौ॒म्यम् ब॒भ्रुं ॅयो यो ब॒भ्रुꣳ सौ॒म्यꣳ सौ॒म्यम् ब॒भ्रुं ॅयः । \newline
28. ब॒भ्रुं ॅयो यो ब॒भ्रुम् ब॒भ्रुं ॅयो ब्रा᳚ह्म॒णो ब्रा᳚ह्म॒णो यो ब॒भ्रुम् ब॒भ्रुं ॅयो ब्रा᳚ह्म॒णः । \newline
29. यो ब्रा᳚ह्म॒णो ब्रा᳚ह्म॒णो यो यो ब्रा᳚ह्म॒णो वि॒द्यां ॅवि॒द्याम् ब्रा᳚ह्म॒णो यो यो ब्रा᳚ह्म॒णो वि॒द्याम् । \newline
30. ब्रा॒ह्म॒णो वि॒द्यां ॅवि॒द्याम् ब्रा᳚ह्म॒णो ब्रा᳚ह्म॒णो वि॒द्या म॒नूच्या॒नूच्य॑ वि॒द्याम् ब्रा᳚ह्म॒णो ब्रा᳚ह्म॒णो वि॒द्या म॒नूच्य॑ । \newline
31. वि॒द्या म॒नूच्या॒नूच्य॑ वि॒द्यां ॅवि॒द्या म॒नूच्य॒ न नानूच्य॑ वि॒द्यां ॅवि॒द्या म॒नूच्य॒ न । \newline
32. अ॒नूच्य॒ न नानूच्या॒ नूच्य॒ न वि॒रोचे॑त वि॒रोचे॑त॒ नानूच्या॒ नूच्य॒ न वि॒रोचे॑त । \newline
33. अ॒नूच्येत्य॑नु - उच्य॑ । \newline
34. न वि॒रोचे॑त वि॒रोचे॑त॒ न न वि॒रोचे॑त॒ यद् यद् वि॒रोचे॑त॒ न न वि॒रोचे॑त॒ यत् । \newline
35. वि॒रोचे॑त॒ यद् यद् वि॒रोचे॑त वि॒रोचे॑त॒ यदा᳚ग्ने॒य आ᳚ग्ने॒यो यद् वि॒रोचे॑त वि॒रोचे॑त॒ यदा᳚ग्ने॒यः । \newline
36. वि॒रोचे॒तेति॑ वि - रोचे॑त । \newline
37. यदा᳚ग्ने॒य आ᳚ग्ने॒यो यद् यदा᳚ग्ने॒यो भव॑ति॒ भव॑त्याग्ने॒यो यद् यदा᳚ग्ने॒यो भव॑ति । \newline
38. आ॒ग्ने॒यो भव॑ति॒ भव॑त्याग्ने॒य आ᳚ग्ने॒यो भव॑ति॒ तेज॒ स्तेजो॒ भव॑त्याग्ने॒य आ᳚ग्ने॒यो भव॑ति॒ तेजः॑ । \newline
39. भव॑ति॒ तेज॒ स्तेजो॒ भव॑ति॒ भव॑ति॒ तेज॑ ए॒वैव तेजो॒ भव॑ति॒ भव॑ति॒ तेज॑ ए॒व । \newline
40. तेज॑ ए॒वैव तेज॒ स्तेज॑ ए॒वास्मि॑न् नस्मिन् ने॒व तेज॒ स्तेज॑ ए॒वास्मिन्न्॑ । \newline
41. ए॒वास्मि॑न् नस्मिन् ने॒वैवास्मि॒न् तेन॒ तेना᳚स्मिन् ने॒वैवास्मि॒न् तेन॑ । \newline
42. अ॒स्मि॒न् तेन॒ तेना᳚स्मिन् नस्मि॒न् तेन॑ दधाति दधाति॒ तेना᳚स्मिन् नस्मि॒न् तेन॑ दधाति । \newline
43. तेन॑ दधाति दधाति॒ तेन॒ तेन॑ दधाति॒ यद् यद् द॑धाति॒ तेन॒ तेन॑ दधाति॒ यत् । \newline
44. द॒धा॒ति॒ यद् यद् द॑धाति दधाति॒ यथ् सौ॒म्यः सौ॒म्यो यद् द॑धाति दधाति॒ यथ् सौ॒म्यः । \newline
45. यथ् सौ॒म्यः सौ॒म्यो यद् यथ् सौ॒म्यो ब्र॑ह्मवर्च॒सम् ब्र॑ह्मवर्च॒सꣳ सौ॒म्यो यद् यथ् सौ॒म्यो ब्र॑ह्मवर्च॒सम् । \newline
46. सौ॒म्यो ब्र॑ह्मवर्च॒सम् ब्र॑ह्मवर्च॒सꣳ सौ॒म्यः सौ॒म्यो ब्र॑ह्मवर्च॒सम् तेन॒ तेन॑ ब्रह्मवर्च॒सꣳ सौ॒म्यः सौ॒म्यो ब्र॑ह्मवर्च॒सम् तेन॑ । \newline
47. ब्र॒ह्म॒व॒र्च॒सम् तेन॒ तेन॑ ब्रह्मवर्च॒सम् ब्र॑ह्मवर्च॒सम् तेन॑ कृ॒ष्णग्री॑वः कृ॒ष्णग्री॑व॒ स्तेन॑ ब्रह्मवर्च॒सम् ब्र॑ह्मवर्च॒सम् तेन॑ कृ॒ष्णग्री॑वः । \newline
48. ब्र॒ह्म॒व॒र्च॒समिति॑ ब्रह्म - व॒र्च॒सम् । \newline
49. तेन॑ कृ॒ष्णग्री॑वः कृ॒ष्णग्री॑व॒ स्तेन॒ तेन॑ कृ॒ष्णग्री॑व आग्ने॒य आ᳚ग्ने॒यः कृ॒ष्णग्री॑व॒ स्तेन॒ तेन॑ कृ॒ष्णग्री॑व आग्ने॒यः । \newline
50. कृ॒ष्णग्री॑व आग्ने॒य आ᳚ग्ने॒यः कृ॒ष्णग्री॑वः कृ॒ष्णग्री॑व आग्ने॒यो भ॑वति भवत्याग्ने॒यः कृ॒ष्णग्री॑वः कृ॒ष्णग्री॑व आग्ने॒यो भ॑वति । \newline
51. कृ॒ष्णग्री॑व॒ इति॑ कृ॒ष्ण - ग्री॒वः॒ । \newline
52. आ॒ग्ने॒यो भ॑वति भवत्याग्ने॒य आ᳚ग्ने॒यो भ॑वति॒ तम॒ स्तमो॑ भवत्याग्ने॒य आ᳚ग्ने॒यो भ॑वति॒ तमः॑ । \newline
53. भ॒व॒ति॒ तम॒ स्तमो॑ भवति भवति॒ तम॑ ए॒वैव तमो॑ भवति भवति॒ तम॑ ए॒व । \newline
54. तम॑ ए॒वैव तम॒ स्तम॑ ए॒वास्मा॑ दस्मा दे॒व तम॒ स्तम॑ ए॒वास्मा᳚त् । \newline
55. ए॒वास्मा॑ दस्मा दे॒वैवास्मा॒ दपापा᳚ स्मा दे॒वैवास्मा॒ दप॑ । \newline
56. अ॒स्मा॒ दपापा᳚ स्मादस्मा॒ दप॑ हन्ति ह॒न्त्यपा᳚ स्मा दस्मा॒ दप॑ हन्ति । \newline
57. अप॑ हन्ति ह॒न्त्यपाप॑ हन्ति श्वे॒तः श्वे॒तो ह॒न्त्यपाप॑ हन्ति श्वे॒तः । \newline
58. ह॒न्ति॒ श्वे॒तः श्वे॒तो ह॑न्ति हन्ति श्वे॒तो भ॑वति भवति श्वे॒तो ह॑न्ति हन्ति श्वे॒तो भ॑वति । \newline
59. श्वे॒तो भ॑वति भवति श्वे॒तः श्वे॒तो भ॑वति॒ रुचꣳ॒॒ रुच॑म् भवति श्वे॒तः श्वे॒तो भ॑वति॒ रुच᳚म् । \newline
60. भ॒व॒ति॒ रुचꣳ॒॒ रुच॑म् भवति भवति॒ रुच॑ मे॒वैव रुच॑म् भवति भवति॒ रुच॑ मे॒व । \newline
\pagebreak
\markright{ TS 2.1.2.9  \hfill https://www.vedavms.in \hfill}

\section{ TS 2.1.2.9 }

\textbf{TS 2.1.2.9 } \newline
\textbf{Samhita Paata} \newline

रुच॑मे॒वास्मि॑न् दधाति ब॒भ्रुः सौ॒म्यो भ॑वति ब्रह्मवर्च॒समे॒वास्मि॒न् त्विषिं॑ दधात्या-ग्ने॒यं कृ॒ष्णग्री॑व॒मा ल॑भेत सौ॒म्यं ब॒भ्रुमा᳚ग्ने॒यं कृ॒ष्णग्री॑वं पुरो॒धायाꣳ॒॒ स्पर्द्ध॑मान आग्ने॒यो वै ब्रा᳚ह्म॒णः सौ॒म्यो रा॑ज॒न्यो॑ऽभितः॑ सौ॒म्यमा᳚ग्ने॒यौ भ॑वत॒-स्तेज॑सै॒व ब्रह्म॑णोभ॒यतो॑ रा॒ष्ट्रं परि॑ गृह्णात्येक॒धा स॒मा वृ॑ङ्क्ते पु॒र ए॑नं दधते ॥ \newline

\textbf{Pada Paata} \newline

रुच᳚म् । ए॒व । अ॒स्मि॒न्न् । द॒धा॒ति॒ । ब॒भ्रुः । सौ॒म्यः । भ॒व॒ति॒ । ब्र॒ह्म॒व॒र्च॒समिति॑ ब्रह्म - व॒र्च॒सम् । ए॒व । अ॒स्मि॒न्न् । त्विषि᳚म् । द॒धा॒ति॒ । आ॒ग्ने॒यम् । कृ॒ष्णग्री॑व॒मिति॑ कृ॒ष्ण - ग्री॒व॒म् । एति॑ । ल॒भे॒त॒ । सौ॒म्यम् । ब॒भ्रुम् । आ॒ग्ने॒यम् । कृ॒ष्णग्री॑व॒मिति॑ कृ॒ष्ण - ग्री॒व॒म् । पु॒रो॒धाया॒मिति॑ पुरः - धाया᳚म् । स्पर्द्ध॑मानः । आ॒ग्ने॒यः । वै । ब्रा॒ह्म॒णः । सौ॒म्यः । रा॒ज॒न्यः॑ । अ॒भितः॑ । सौ॒म्यम् । आ॒ग्ने॒यौ । भ॒व॒तः॒ । तेज॑सा । ए॒व । ब्रह्म॑णा । उ॒भ॒यतः॑ । रा॒ष्ट्रम् । परीति॑ । गृ॒ह्णा॒ति॒ । ए॒क॒धेत्ये॑क - धा । स॒मावृ॑ङ्क्त॒ इति॑ सं - आवृ॑ङ्क्ते । पु॒रः । ए॒न॒म् । द॒ध॒ते॒ ॥  \newline


\textbf{Krama Paata} \newline

रुच॑मे॒व । ए॒वास्मिन्न्॑ । अ॒स्मि॒न् द॒धा॒ति॒ । द॒धा॒ति॒ ब॒भ्रुः । ब॒भ्रुः सौ॒म्यः । सौ॒म्यो भ॑वति । भ॒व॒ति॒ ब्र॒ह्म॒व॒र्च॒सम् । ब्र॒ह्म॒व॒र्च॒समे॒व । ब्र॒ह्म॒व॒र्च॒समिति॑ ब्रह्म - व॒र्च॒सम् । ए॒वास्मिन्न्॑ । अ॒स्मि॒न् त्विषि᳚म् । त्विषि॑म् दधाति । द॒धा॒त्या॒ग्ने॒यम् । आ॒ग्ने॒यम् कृ॒ष्णग्री॑वम् । कृ॒ष्णग्री॑व॒मा । कृ॒ष्णग्री॑व॒मिति॑ कृ॒ष्ण - ग्री॒व॒म् । आ ल॑भेत । ल॒भे॒त॒ सौ॒म्यम् । सौ॒म्यम् ब॒भ्रुम् । ब॒भ्रुमा᳚ग्ने॒यम् । आ॒ग्ने॒यम् कृ॒ष्णग्री॑वम् । कृ॒ष्णग्री॑वम् पुरो॒धाया᳚म् । कृ॒ष्णग्री॑व॒मिति॑ कृ॒ष्ण - ग्री॒व॒म् । पु॒रो॒धायाꣳ॒॒ स्पर्द्ध॑मानः । पु॒रो॒धाया॒मिति॑ पुरः - धाया᳚म् । स्पर्द्ध॑मान आग्ने॒यः । आ॒ग्ने॒यो वै । वै ब्रा᳚ह्म॒णः । ब्रा॒ह्म॒णः सौ॒म्यः । सौ॒म्यो रा॑ज॒न्यः॑ । रा॒ज॒न्यो॑ऽभितः॑ । अ॒भितः॑ सौ॒म्यम् । सौ॒म्यमा᳚ग्ने॒यौ । आ॒ग्ने॒यौ भ॑वतः । भ॒व॒त॒स्तेज॑सा । तेज॑सै॒व । ए॒व ब्रह्म॑णा । ब्रह्म॑णोभ॒यतः॑ । उ॒भ॒यतो॑ रा॒ष्ट्रम् । रा॒ष्ट्रम् परि॑ । परि॑ गृह्णाति । गृ॒ह्णा॒त्ये॒क॒धा । ए॒क॒धा स॒मावृ॑ङ्क्ते । ए॒क॒धेत्ये॑क - धा । स॒मावृ॑ङ्क्ते पु॒रः । स॒मावृ॑ङ्क्त॒ इति॑ सं - आवृ॑ङ्क्ते । पु॒र ए॑नम् । ए॒न॒म् द॒ध॒ते॒ । द॒ध॒त॒ इति॑ दधते । \newline

\textbf{Jatai Paata} \newline

1. रुच॑ मे॒वैव रुचꣳ॒॒ रुच॑ मे॒व । \newline
2. ए॒वास्मि॑न् नस्मिन् ने॒वैवास्मिन्न्॑ । \newline
3. अ॒स्मि॒न् द॒धा॒ति॒ द॒धा॒ त्य॒स्मि॒न् न॒स्मि॒न् द॒धा॒ति॒ । \newline
4. द॒धा॒ति॒ ब॒भ्रुर् ब॒भ्रुर् द॑धाति दधाति ब॒भ्रुः । \newline
5. ब॒भ्रुः सौ॒म्यः सौ॒म्यो ब॒भ्रुर् ब॒भ्रुः सौ॒म्यः । \newline
6. सौ॒म्यो भ॑वति भवति सौ॒म्यः सौ॒म्यो भ॑वति । \newline
7. भ॒व॒ति॒ ब्र॒ह्म॒व॒र्च॒सम् ब्र॑ह्मवर्च॒सम् भ॑वति भवति ब्रह्मवर्च॒सम् । \newline
8. ब्र॒ह्म॒व॒र्च॒स मे॒वैव ब्र॑ह्मवर्च॒सम् ब्र॑ह्मवर्च॒स मे॒व । \newline
9. ब्र॒ह्म॒व॒र्च॒समिति॑ ब्रह्म - व॒र्च॒सम् । \newline
10. ए॒वास्मि॑न् नस्मिन् ने॒वैवास्मिन्न्॑ । \newline
11. अ॒स्मि॒न् त्विषि॒म् त्विषि॑ मस्मिन् नस्मि॒न् त्विषि᳚म् । \newline
12. त्विषि॑म् दधाति दधाति॒ त्विषि॒म् त्विषि॑म् दधाति । \newline
13. द॒धा॒त्या॒ग्ने॒य मा᳚ग्ने॒यम् द॑धाति दधात्याग्ने॒यम् । \newline
14. आ॒ग्ने॒यम् कृ॒ष्णग्री॑वम् कृ॒ष्णग्री॑व माग्ने॒य मा᳚ग्ने॒यम् कृ॒ष्णग्री॑वम् । \newline
15. कृ॒ष्णग्री॑व॒ मा कृ॒ष्णग्री॑वम् कृ॒ष्णग्री॑व॒ मा । \newline
16. कृ॒ष्णग्री॑व॒मिति॑ कृ॒ष्ण - ग्री॒व॒म् । \newline
17. आ ल॑भेत लभे॒ता ल॑भेत । \newline
18. ल॒भे॒त॒ सौ॒म्यꣳ सौ॒म्यम् ॅल॑भेत लभेत सौ॒म्यम् । \newline
19. सौ॒म्यम् ब॒भ्रुम् ब॒भ्रुꣳ सौ॒म्यꣳ सौ॒म्यम् ब॒भ्रुम् । \newline
20. ब॒भ्रु मा᳚ग्ने॒य मा᳚ग्ने॒यम् ब॒भ्रुम् ब॒भ्रु मा᳚ग्ने॒यम् । \newline
21. आ॒ग्ने॒यम् कृ॒ष्णग्री॑वम् कृ॒ष्णग्री॑व माग्ने॒य मा᳚ग्ने॒यम् कृ॒ष्णग्री॑वम् । \newline
22. कृ॒ष्णग्री॑वम् पुरो॒धाया᳚म् पुरो॒धाया᳚म् कृ॒ष्णग्री॑वम् कृ॒ष्णग्री॑वम् पुरो॒धाया᳚म् । \newline
23. कृ॒ष्णग्री॑व॒मिति॑ कृ॒ष्ण - ग्री॒व॒म् । \newline
24. पु॒रो॒धायाꣳ॒॒ स्पर्द्ध॑मानः॒ स्पर्द्ध॑मानः पुरो॒धाया᳚म् पुरो॒धायाꣳ॒॒ स्पर्द्ध॑मानः । \newline
25. पु॒रो॒धाया॒मिति॑ पुरः - धाया᳚म् । \newline
26. स्पर्द्ध॑मान आग्ने॒य आ᳚ग्ने॒यः स्पर्द्ध॑मानः॒ स्पर्द्ध॑मान आग्ने॒यः । \newline
27. आ॒ग्ने॒यो वै वा आ᳚ग्ने॒य आ᳚ग्ने॒यो वै । \newline
28. वै ब्रा᳚ह्म॒णो ब्रा᳚ह्म॒णो वै वै ब्रा᳚ह्म॒णः । \newline
29. ब्रा॒ह्म॒णः सौ॒म्यः सौ॒म्यो ब्रा᳚ह्म॒णो ब्रा᳚ह्म॒णः सौ॒म्यः । \newline
30. सौ॒म्यो रा॑ज॒न्यो॑ राज॒न्यः॑ सौ॒म्यः सौ॒म्यो रा॑ज॒न्यः॑ । \newline
31. रा॒ज॒न्यो॑ ऽभितो॒ ऽभितो॑ राज॒न्यो॑ राज॒न्यो॑ ऽभितः॑ । \newline
32. अ॒भितः॑ सौ॒म्यꣳ सौ॒म्य म॒भितो॒ ऽभितः॑ सौ॒म्यम् । \newline
33. सौ॒म्य मा᳚ग्ने॒या वा᳚ग्ने॒यौ सौ॒म्यꣳ सौ॒म्य मा᳚ग्ने॒यौ । \newline
34. आ॒ग्ने॒यौ भ॑वतो भवत आग्ने॒या वा᳚ग्ने॒यौ भ॑वतः । \newline
35. भ॒व॒त॒ स्तेज॑सा॒ तेज॑सा भवतो भवत॒ स्तेज॑सा । \newline
36. तेज॑सै॒वैव तेज॑सा॒ तेज॑सै॒व । \newline
37. ए॒व ब्रह्म॑णा॒ ब्रह्म॑ णै॒वैव ब्रह्म॑णा । \newline
38. ब्रह्म॑णोभ॒यत॑ उभ॒यतो॒ ब्रह्म॑णा॒ ब्रह्म॑णोभ॒यतः॑ । \newline
39. उ॒भ॒यतो॑ रा॒ष्ट्रꣳ रा॒ष्ट्र मु॑भ॒यत॑ उभ॒यतो॑ रा॒ष्ट्रम् । \newline
40. रा॒ष्ट्रम् परि॒ परि॑ रा॒ष्ट्रꣳ रा॒ष्ट्रम् परि॑ । \newline
41. परि॑ गृह्णाति गृह्णाति॒ परि॒ परि॑ गृह्णाति । \newline
42. गृ॒ह्णा॒ त्ये॒क॒धैक॒धा गृ॑ह्णाति गृह्णा त्येक॒धा । \newline
43. ए॒क॒धा स॒मावृ॑ङ्क्ते स॒मावृ॑ङ्क्त एक॒धैक॒धा स॒मावृ॑ङ्क्ते । \newline
44. ए॒क॒धेत्ये॑क - धा । \newline
45. स॒मावृ॑ङ्क्ते पु॒रः पु॒रः स॒मावृ॑ङ्क्ते स॒मावृ॑ङ्क्ते पु॒रः । \newline
46. स॒मावृ॑ङ्क्त॒ इति॑ सं - आवृ॑ङ्क्ते । \newline
47. पु॒र ए॑न मेनम् पु॒रः पु॒र ए॑नम् । \newline
48. ए॒न॒म् द॒ध॒ते॒ द॒ध॒त॒ ए॒न॒ मे॒न॒म् द॒ध॒ते॒ । \newline
49. द॒ध॒त॒ इति॑ दधते । \newline

\textbf{Ghana Paata } \newline

1. रुच॑ मे॒वैव रुचꣳ॒॒ रुच॑ मे॒वास्मि॑न् नस्मिन् ने॒व रुचꣳ॒॒ रुच॑ मे॒वास्मिन्न्॑ । \newline
2. ए॒वास्मि॑न् नस्मिन् ने॒वै वास्मि॑न् दधाति दधा त्यस्मिन् ने॒वै वास्मि॑न् दधाति । \newline
3. अ॒स्मि॒न् द॒धा॒ति॒ द॒धा॒ त्य॒स्मि॒न् न॒स्मि॒न् द॒धा॒ति॒ ब॒भ्रुर् ब॒भ्रुर् द॑धा त्यस्मिन् नस्मिन् दधाति ब॒भ्रुः । \newline
4. द॒धा॒ति॒ ब॒भ्रुर् ब॒भ्रुर् द॑धाति दधाति ब॒भ्रुः सौ॒म्यः सौ॒म्यो ब॒भ्रुर् द॑धाति दधाति ब॒भ्रुः सौ॒म्यः । \newline
5. ब॒भ्रुः सौ॒म्यः सौ॒म्यो ब॒भ्रुर् ब॒भ्रुः सौ॒म्यो भ॑वति भवति सौ॒म्यो ब॒भ्रुर् ब॒भ्रुः सौ॒म्यो भ॑वति । \newline
6. सौ॒म्यो भ॑वति भवति सौ॒म्यः सौ॒म्यो भ॑वति ब्रह्मवर्च॒सम् ब्र॑ह्मवर्च॒सम् भ॑वति सौ॒म्यः सौ॒म्यो भ॑वति ब्रह्मवर्च॒सम् । \newline
7. भ॒व॒ति॒ ब्र॒ह्म॒व॒र्च॒सम् ब्र॑ह्मवर्च॒सम् भ॑वति भवति ब्रह्मवर्च॒स मे॒वैव ब्र॑ह्मवर्च॒सम् भ॑वति भवति ब्रह्मवर्च॒स मे॒व । \newline
8. ब्र॒ह्म॒व॒र्च॒स मे॒वैव ब्र॑ह्मवर्च॒सम् ब्र॑ह्मवर्च॒स मे॒वास्मि॑न् नस्मिन् ने॒व ब्र॑ह्मवर्च॒सम् ब्र॑ह्मवर्च॒स मे॒वास्मिन्न्॑ । \newline
9. ब्र॒ह्म॒व॒र्च॒समिति॑ ब्रह्म - व॒र्च॒सम् । \newline
10. ए॒वास्मि॑न् नस्मिन् ने॒वै वास्मि॒न् त्विषि॒म् त्विषि॑ मस्मिन् ने॒वै वास्मि॒न् त्विषि᳚म् । \newline
11. अ॒स्मि॒न् त्विषि॒म् त्विषि॑ मस्मिन् नस्मि॒न् त्विषि॑म् दधाति दधाति॒ त्विषि॑ मस्मिन् नस्मि॒न् त्विषि॑म् दधाति । \newline
12. त्विषि॑म् दधाति दधाति॒ त्विषि॒म् त्विषि॑म् दधात्याग्ने॒य मा᳚ग्ने॒यम् द॑धाति॒ त्विषि॒म् त्विषि॑म् दधात्याग्ने॒यम् । \newline
13. द॒धा॒ त्या॒ग्ने॒य मा᳚ग्ने॒यम् द॑धाति दधा त्याग्ने॒यम् कृ॒ष्णग्री॑वम् कृ॒ष्णग्री॑व माग्ने॒यम् द॑धाति दधा त्याग्ने॒यम् कृ॒ष्णग्री॑वम् । \newline
14. आ॒ग्ने॒यम् कृ॒ष्णग्री॑वम् कृ॒ष्णग्री॑व माग्ने॒य मा᳚ग्ने॒यम् कृ॒ष्णग्री॑व॒ मा कृ॒ष्णग्री॑व माग्ने॒य मा᳚ग्ने॒यम् कृ॒ष्णग्री॑व॒ मा । \newline
15. कृ॒ष्णग्री॑व॒ मा कृ॒ष्णग्री॑वम् कृ॒ष्णग्री॑व॒ मा ल॑भेत लभे॒ता कृ॒ष्णग्री॑वम् कृ॒ष्णग्री॑व॒ मा ल॑भेत । \newline
16. कृ॒ष्णग्री॑व॒मिति॑ कृ॒ष्ण - ग्री॒व॒म् । \newline
17. आ ल॑भेत लभे॒ता ल॑भेत सौ॒म्यꣳ सौ॒म्यम् ॅल॑भे॒ता ल॑भेत सौ॒म्यम् । \newline
18. ल॒भे॒त॒ सौ॒म्यꣳ सौ॒म्यम् ॅल॑भेत लभेत सौ॒म्यम् ब॒भ्रुम् ब॒भ्रुꣳ सौ॒म्यम् ॅल॑भेत लभेत सौ॒म्यम् ब॒भ्रुम् । \newline
19. सौ॒म्यम् ब॒भ्रुम् ब॒भ्रुꣳ सौ॒म्यꣳ सौ॒म्यम् ब॒भ्रु मा᳚ग्ने॒य मा᳚ग्ने॒यम् ब॒भ्रुꣳ सौ॒म्यꣳ सौ॒म्यम् ब॒भ्रु मा᳚ग्ने॒यम् । \newline
20. ब॒भ्रु मा᳚ग्ने॒य मा᳚ग्ने॒यम् ब॒भ्रुम् ब॒भ्रु मा᳚ग्ने॒यम् कृ॒ष्णग्री॑वम् कृ॒ष्णग्री॑व माग्ने॒यम् ब॒भ्रुम् ब॒भ्रु मा᳚ग्ने॒यम् कृ॒ष्णग्री॑वम् । \newline
21. आ॒ग्ने॒यम् कृ॒ष्णग्री॑वम् कृ॒ष्णग्री॑व माग्ने॒य मा᳚ग्ने॒यम् कृ॒ष्णग्री॑वम् पुरो॒धाया᳚म् पुरो॒धाया᳚म् कृ॒ष्णग्री॑व माग्ने॒य मा᳚ग्ने॒यम् कृ॒ष्णग्री॑वम् पुरो॒धाया᳚म् । \newline
22. कृ॒ष्णग्री॑वम् पुरो॒धाया᳚म् पुरो॒धाया᳚म् कृ॒ष्णग्री॑वम् कृ॒ष्णग्री॑वम् पुरो॒धायाꣳ॒॒ स्पर्द्ध॑मानः॒ स्पर्द्ध॑मानः पुरो॒धाया᳚म् कृ॒ष्णग्री॑वम् कृ॒ष्णग्री॑वम् पुरो॒धायाꣳ॒॒ स्पर्द्ध॑मानः । \newline
23. कृ॒ष्णग्री॑व॒मिति॑ कृ॒ष्ण - ग्री॒व॒म् । \newline
24. पु॒रो॒धायाꣳ॒॒ स्पर्द्ध॑मानः॒ स्पर्द्ध॑मानः पुरो॒धाया᳚म् पुरो॒धायाꣳ॒॒ स्पर्द्ध॑मान आग्ने॒य आ᳚ग्ने॒यः स्पर्द्ध॑मानः पुरो॒धाया᳚म् पुरो॒धायाꣳ॒॒ स्पर्द्ध॑मान आग्ने॒यः । \newline
25. पु॒रो॒धाया॒मिति॑ पुरः - धाया᳚म् । \newline
26. स्पर्द्ध॑मान आग्ने॒य आ᳚ग्ने॒यः स्पर्द्ध॑मानः॒ स्पर्द्ध॑मान आग्ने॒यो वै वा आ᳚ग्ने॒यः स्पर्द्ध॑मानः॒ स्पर्द्ध॑मान आग्ने॒यो वै । \newline
27. आ॒ग्ने॒यो वै वा आ᳚ग्ने॒य आ᳚ग्ने॒यो वै ब्रा᳚ह्म॒णो ब्रा᳚ह्म॒णो वा आ᳚ग्ने॒य आ᳚ग्ने॒यो वै ब्रा᳚ह्म॒णः । \newline
28. वै ब्रा᳚ह्म॒णो ब्रा᳚ह्म॒णो वै वै ब्रा᳚ह्म॒णः सौ॒म्यः सौ॒म्यो ब्रा᳚ह्म॒णो वै वै ब्रा᳚ह्म॒णः सौ॒म्यः । \newline
29. ब्रा॒ह्म॒णः सौ॒म्यः सौ॒म्यो ब्रा᳚ह्म॒णो ब्रा᳚ह्म॒णः सौ॒म्यो रा॑ज॒न्यो॑ राज॒न्यः॑ सौ॒म्यो ब्रा᳚ह्म॒णो ब्रा᳚ह्म॒णः सौ॒म्यो रा॑ज॒न्यः॑ । \newline
30. सौ॒म्यो रा॑ज॒न्यो॑ राज॒न्यः॑ सौ॒म्यः सौ॒म्यो रा॑ज॒न्यो॑ ऽभितो॒ ऽभितो॑ राज॒न्यः॑ सौ॒म्यः सौ॒म्यो रा॑ज॒न्यो॑ ऽभितः॑ । \newline
31. रा॒ज॒न्यो॑ ऽभितो॒ ऽभितो॑ राज॒न्यो॑ राज॒न्यो॑ ऽभितः॑ सौ॒म्यꣳ सौ॒म्य म॒भितो॑ राज॒न्यो॑ राज॒न्यो॑ ऽभितः॑ सौ॒म्यम् । \newline
32. अ॒भितः॑ सौ॒म्यꣳ सौ॒म्य म॒भितो॒ ऽभितः॑ सौ॒म्य मा᳚ग्ने॒या वा᳚ग्ने॒यौ सौ॒म्य म॒भितो॒ ऽभितः॑ सौ॒म्य मा᳚ग्ने॒यौ । \newline
33. सौ॒म्य मा᳚ग्ने॒या वा᳚ग्ने॒यौ सौ॒म्यꣳ सौ॒म्य मा᳚ग्ने॒यौ भ॑वतो भवत आग्ने॒यौ सौ॒म्यꣳ सौ॒म्य मा᳚ग्ने॒यौ भ॑वतः । \newline
34. आ॒ग्ने॒यौ भ॑वतो भवत आग्ने॒या वा᳚ग्ने॒यौ भ॑वत॒ स्तेज॑सा॒ तेज॑सा भवत आग्ने॒या वा᳚ग्ने॒यौ भ॑वत॒ स्तेज॑सा । \newline
35. भ॒व॒त॒ स्तेज॑सा॒ तेज॑सा भवतो भवत॒ स्तेज॑ सै॒वैव तेज॑सा भवतो भवत॒ स्तेज॑सै॒व । \newline
36. तेज॑ सै॒वैव तेज॑सा॒ तेज॑सै॒व ब्रह्म॑णा॒ ब्रह्म॑णै॒व तेज॑सा॒ तेज॑सै॒व ब्रह्म॑णा । \newline
37. ए॒व ब्रह्म॑णा॒ ब्रह्म॑णै॒वैव ब्रह्म॑णोभ॒यत॑ उभ॒यतो॒ ब्रह्म॑णै॒वैव ब्रह्म॑णोभ॒यतः॑ । \newline
38. ब्रह्म॑णोभ॒यत॑ उभ॒यतो॒ ब्रह्म॑णा॒ ब्रह्म॑णोभ॒यतो॑ रा॒ष्ट्रꣳ रा॒ष्ट्र मु॑भ॒यतो॒ ब्रह्म॑णा॒ ब्रह्म॑णोभ॒यतो॑ रा॒ष्ट्रम् । \newline
39. उ॒भ॒यतो॑ रा॒ष्ट्रꣳ रा॒ष्ट्र मु॑भ॒यत॑ उभ॒यतो॑ रा॒ष्ट्रम् परि॒ परि॑ रा॒ष्ट्र मु॑भ॒यत॑ उभ॒यतो॑ रा॒ष्ट्रम् परि॑ । \newline
40. रा॒ष्ट्रम् परि॒ परि॑ रा॒ष्ट्रꣳ रा॒ष्ट्रम् परि॑ गृह्णाति गृह्णाति॒ परि॑ रा॒ष्ट्रꣳ रा॒ष्ट्रम् परि॑ गृह्णाति । \newline
41. परि॑ गृह्णाति गृह्णाति॒ परि॒ परि॑ गृह्णा त्येक॒ धैक॒धा गृ॑ह्णाति॒ परि॒ परि॑ गृह्णा त्येक॒धा । \newline
42. गृ॒ह्णा॒ त्ये॒क॒ धैक॒धा गृ॑ह्णाति गृह्णा त्येक॒धा स॒मावृ॑ङ्क्ते स॒मावृ॑ङ्क्त एक॒धा गृ॑ह्णाति गृह्णा त्येक॒धा स॒मावृ॑ङ्क्ते । \newline
43. ए॒क॒धा स॒मावृ॑ङ्क्ते स॒मावृ॑ङ्क्त एक॒ धैक॒धा स॒मावृ॑ङ्क्ते पु॒रः पु॒रः स॒मावृ॑ङ्क्त एक॒ धैक॒धा स॒मावृ॑ङ्क्ते पु॒रः । \newline
44. ए॒क॒धेत्ये॑क - धा । \newline
45. स॒मावृ॑ङ्क्ते पु॒रः पु॒रः स॒मावृ॑ङ्क्ते स॒मावृ॑ङ्क्ते पु॒र ए॑न मेनम् पु॒रः स॒मावृ॑ङ्क्ते 
स॒मावृ॑ङ्क्ते पु॒र ए॑नम् । \newline
46. स॒मावृ॑ङ्क्त॒ इति॑ सं - आवृ॑ङ्क्ते । \newline
47. पु॒र ए॑न मेनम् पु॒रः पु॒र ए॑नम् दधते दधत एनम् पु॒रः पु॒र ए॑नम् दधते । \newline
48. ए॒न॒म् द॒ध॒ते॒ द॒ध॒त॒ ए॒न॒ मे॒न॒म् द॒ध॒ते॒ । \newline
49. द॒ध॒त॒ इति॑ दधते । \newline
\pagebreak
\markright{ TS 2.1.3.1  \hfill https://www.vedavms.in \hfill}

\section{ TS 2.1.3.1 }

\textbf{TS 2.1.3.1 } \newline
\textbf{Samhita Paata} \newline

दे॒वा॒सु॒रा ए॒षु लो॒केष्व॑स्पर्द्धन्त॒ स ए॒तं ॅविष्णु॑-र्वाम॒नम॑पश्य॒त् तꣳ स्वायै॑ दे॒वता॑या॒ आऽल॑भत॒ ततो॒ वै स इ॒मां ॅलो॒कान॒भ्य॑जयद्- वैष्ण॒वं ॅवा॑म॒नमा ल॑भेत॒ स्पर्द्ध॑मानो॒ विष्णु॑रे॒व भू॒त्वेमान् ॅलो॒कान॒भि ज॑यति॒ विष॑म॒ आ ल॑भेत॒ विष॑मा इव॒ हीमे लो॒काः समृ॑द्ध्या॒ इन्द्रा॑य मन्यु॒मते॒ मन॑स्वते ल॒लामं॑ प्राशृ॒ङ्गमा ल॑भेत संग्रा॒मे - [  ] \newline

\textbf{Pada Paata} \newline

दे॒वा॒सु॒रा इति॑ देव - अ॒सु॒राः । ए॒षु । लो॒केषु॑ । अ॒स्प॒र्द्ध॒न्त॒ । सः । ए॒तम् । विष्णुः॑ । वा॒म॒नम् । अ॒प॒श्य॒त् । तम् । स्वायै᳚ । दे॒वता॑यै । एति॑ । अ॒ल॒भ॒त॒ । ततः॑ । वै । सः । इ॒मान् । लो॒कान् । अ॒भीति॑ । अ॒ज॒य॒त् । वै॒ष्ण॒वम् । वा॒म॒नम् । एति॑ । ल॒भे॒त॒ । स्पर्द्ध॑मानः । विष्णुः॑ । ए॒व । भू॒त्वा । इ॒मान् । लो॒कान् । अ॒भीति॑ । ज॒य॒ति॒ । विष॑म॒ इति॒ वि - स॒मे॒ । एति॑ । ल॒भे॒त॒ । विष॑मा॒ इति॒ वि - स॒माः॒ । इ॒व॒ । हि । इ॒मे । लो॒काः । समृ॑द्ध्या॒ इति॒ सं - ऋ॒द्ध्यै॒ । इन्द्रा॑य । म॒न्यु॒मत॒ इति॑ मन्यु - मते᳚ । मन॑स्वते । ल॒लाम᳚म् । प्रा॒शृ॒ङ्गम् । एति॑ । ल॒भे॒त॒ । स॒ङ्ग्रा॒म इति॑ सं - ग्रा॒मे ।  \newline


\textbf{Krama Paata} \newline

दे॒वा॒सु॒रा ए॒षु । दे॒वा॒सु॒रा इति॑ देव - अ॒सु॒राः । ए॒षु लो॒केषु॑ । लो॒केष्व॑स्पर्द्धन्त । अ॒स्प॒र्द्ध॒न्त॒ सः । स ए॒तम् । ए॒तं ॅविष्णुः॑ । विष्णु॑र्,वाम॒नम् । वा॒म॒नम॑पश्यत् । अ॒प॒श्य॒त् तम् । तꣳ स्वायै᳚ । स्वायै॑ दे॒वता॑यै । दे॒वता॑या॒ आ । आ ऽल॑भत । अ॒ल॒भ॒त॒ ततः॑ । ततो॒ वै । वै सः । स इ॒मान् । इ॒मान् ॅलो॒कान् । लो॒कान॒भि । अ॒भ्य॑जयत् । अ॒ज॒य॒द् वै॒ष्ण॒वम् । वै॒ष्ण॒वं ॅवा॑म॒नम् । वा॒म॒नमा । आ ल॑भेत । ल॒भे॒त॒ स्पर्द्ध॑मानः । स्पर्द्ध॑मानो॒ विष्णुः॑ । विष्णु॑रे॒व । ए॒व भू॒त्वा । भू॒त्वेमान् । इ॒मान् ॅलो॒कान् । लो॒कान॒भि । अ॒भि ज॑यति । ज॒य॒ति॒ विष॑मे । विष॑म॒ आ । विष॑म॒ इति॒ वि - स॒मे॒ । आ ल॑भेत । ल॒भे॒त॒ विष॑माः । विष॑मा इव । विष॑मा॒ इति॒ वि - स॒माः॒ । इ॒व॒ हि । हीमे । इ॒मे लो॒काः । लो॒काः समृ॑द्ध्यै । समृ॑द्ध्या॒ इन्द्रा॑य । समृ॑द्धा॒ इति॒ सं - ऋ॒द्ध्यै॒ । इन्द्रा॑य मन्यु॒मते᳚ । म॒न्यु॒मते॒ मन॑स्वते । म॒न्यु॒मत॒ इति॑ मन्यु - मते᳚ । मन॑स्वते ल॒लाम᳚म् । ल॒लाम॑म् प्राशृ॒ङ्गम् । प्रा॒शृ॒ङ्गमा । आ ल॑भेत । ल॒भे॒त॒ स॒ङ्ग्रा॒मे । स॒ङ्ग्रा॒मे सम्ॅय॑त्ते । स॒ङ्ग्रा॒म इति॑ सम् - ग्रा॒मे \newline

\textbf{Jatai Paata} \newline

1. दे॒वा॒सु॒रा ए॒ष्वे॑षु दे॑वासु॒रा दे॑वासु॒रा ए॒षु । \newline
2. दे॒वा॒सु॒रा इति॑ देव - अ॒सु॒राः । \newline
3. ए॒षु लो॒केषु॑ लो॒के ष्वे॒ष्वे॑षु लो॒केषु॑ । \newline
4. लो॒के ष्व॑स्पर्द्धन्ता स्पर्द्धन्त लो॒केषु॑ लो॒के ष्व॑स्पर्द्धन्त । \newline
5. अ॒स्प॒र्द्ध॒न्त॒ स सो᳚ ऽस्पर्द्धन्ता स्पर्द्धन्त॒ सः । \newline
6. स ए॒त मे॒तꣳ स स ए॒तम् । \newline
7. ए॒तं ॅविष्णु॒र् विष्णु॑ रे॒त मे॒तं ॅविष्णुः॑ । \newline
8. विष्णु॑र् वाम॒नं ॅवा॑म॒नं ॅविष्णु॒र् विष्णु॑र् वाम॒नम् । \newline
9. वा॒म॒न म॑पश्य दपश्यद् वाम॒नं ॅवा॑म॒न म॑पश्यत् । \newline
10. अ॒प॒श्य॒त् तम् त म॑पश्य दपश्य॒त् तम् । \newline
11. तꣳ स्वायै॒ स्वायै॒ तम् तꣳ स्वायै᳚ । \newline
12. स्वायै॑ दे॒वता॑यै दे॒वता॑यै॒ स्वायै॒ स्वायै॑ दे॒वता॑यै । \newline
13. दे॒वता॑या॒ आ दे॒वता॑यै दे॒वता॑या॒ आ । \newline
14. आ ऽल॑भता लभ॒ता ऽल॑भत । \newline
15. अ॒ल॒भ॒त॒ तत॒ स्ततो॑ ऽलभता लभत॒ ततः॑ । \newline
16. ततो॒ वै वै तत॒ स्ततो॒ वै । \newline
17. वै स स वै वै सः । \newline
18. स इ॒मा नि॒मान् थ्स स इ॒मान् । \newline
19. इ॒मान् ॅलो॒कान् ॅलो॒का नि॒मा नि॒मान् ॅलो॒कान् । \newline
20. लो॒का न॒भ्य॑भि लो॒कान् ॅलो॒का न॒भि । \newline
21. अ॒भ्य॑जय दजय द॒भ्या᳚(1॒)भ्य॑जयत् । \newline
22. अ॒ज॒य॒द् वै॒ष्ण॒वं ॅवै᳚ष्ण॒व म॑जय दजयद् वैष्ण॒वम् । \newline
23. वै॒ष्ण॒वं ॅवा॑म॒नं ॅवा॑म॒नं ॅवै᳚ष्ण॒वं ॅवै᳚ष्ण॒वं ॅवा॑म॒नम् । \newline
24. वा॒म॒न मा वा॑म॒नं ॅवा॑म॒न मा । \newline
25. आ ल॑भेत लभे॒ता ल॑भेत । \newline
26. ल॒भे॒त॒ स्पर्द्ध॑मानः॒ स्पर्द्ध॑मानो लभेत लभेत॒ स्पर्द्ध॑मानः । \newline
27. स्पर्द्ध॑मानो॒ विष्णु॒र् विष्णुः॒ स्पर्द्ध॑मानः॒ स्पर्द्ध॑मानो॒ विष्णुः॑ । \newline
28. विष्णु॑ रे॒वैव विष्णु॒र् विष्णु॑ रे॒व । \newline
29. ए॒व भू॒त्वा भू॒त्वैवैव भू॒त्वा । \newline
30. भू॒त्वेमा नि॒मान् भू॒त्वा भू॒त्वेमान् । \newline
31. इ॒मान् ॅलो॒कान् ॅलो॒का नि॒मा नि॒मान् ॅलो॒कान् । \newline
32. लो॒का न॒भ्य॑भि लो॒कान् ॅलो॒का न॒भि । \newline
33. अ॒भि ज॑यति जय त्य॒भ्य॑भि ज॑यति । \newline
34. ज॒य॒ति॒ विष॑मे॒ विष॑मे जयति जयति॒ विष॑मे । \newline
35. विष॑म॒ आ विष॑मे॒ विष॑म॒ आ । \newline
36. विष॑म॒ इति॒ वि - स॒मे॒ । \newline
37. आ ल॑भेत लभे॒ता ल॑भेत । \newline
38. ल॒भे॒त॒ विष॑मा॒ विष॑मा लभेत लभेत॒ विष॑माः । \newline
39. विष॑मा इवे व॒ विष॑मा॒ विष॑मा इव । \newline
40. विष॑मा॒ इति॒ वि - स॒माः॒ । \newline
41. इ॒व॒ हि हीवे॑ व॒ हि । \newline
42. हीम इ॒मे हि हीमे । \newline
43. इ॒मे लो॒का लो॒का इ॒म इ॒मे लो॒काः । \newline
44. लो॒काः समृ॑द्ध्यै॒ समृ॑द्ध्यै लो॒का लो॒काः समृ॑द्ध्यै । \newline
45. समृ॑द्ध्या॒ इन्द्रा॒ये न्द्रा॑य॒ समृ॑द्ध्यै॒ समृ॑द्ध्या॒ इन्द्रा॑य । \newline
46. समृ॑द्ध्या॒ इति॒ सं - ऋ॒द्ध्यै॒ । \newline
47. इन्द्रा॑य मन्यु॒मते॑ मन्यु॒मत॒ इन्द्रा॒ये न्द्रा॑य मन्यु॒मते᳚ । \newline
48. म॒न्यु॒मते॒ मन॑स्वते॒ मन॑स्वते मन्यु॒मते॑ मन्यु॒मते॒ मन॑स्वते । \newline
49. म॒न्यु॒मत॒ इति॑ मन्यु - मते᳚ । \newline
50. मन॑स्वते ल॒लाम॑म् ॅल॒लाम॒म् मन॑स्वते॒ मन॑स्वते ल॒लाम᳚म् । \newline
51. ल॒लाम॑म् प्राशृ॒ङ्गम् प्रा॑शृ॒ङ्गम् ॅल॒लाम॑म् ॅल॒लाम॑म् प्राशृ॒ङ्गम् । \newline
52. प्रा॒शृ॒ङ्ग मा प्रा॑शृ॒ङ्गम् प्रा॑शृ॒ङ्ग मा । \newline
53. आ ल॑भेत लभे॒ता ल॑भेत । \newline
54. ल॒भे॒त॒ स॒ङ्ग्रा॒मे स॑ङ्ग्रा॒मे ल॑भेत लभेत सङ्ग्रा॒मे । \newline
55. स॒ङ्ग्रा॒मे संॅय॑त्ते॒ संॅय॑त्ते सङ्ग्रा॒मे स॑ङ्ग्रा॒मे संॅय॑त्ते । \newline
56. स॒ङ्ग्रा॒म इति॑ सं - ग्रा॒मे । \newline

\textbf{Ghana Paata } \newline

1. दे॒वा॒सु॒रा ए॒ष्वे॑षु दे॑वासु॒रा दे॑वासु॒रा ए॒षु लो॒केषु॑ लो॒के ष्वे॒षु दे॑वासु॒रा दे॑वासु॒रा ए॒षु लो॒केषु॑ । \newline
2. दे॒वा॒सु॒रा इति॑ देव - अ॒सु॒राः । \newline
3. ए॒षु लो॒केषु॑ लो॒के ष्वे॒ ष्वे॑षु लो॒के ष्व॑ स्पर्द्धन्ता स्पर्द्धन्त लो॒के ष्वे॒ ष्वे॑षु लो॒के ष्व॑ स्पर्द्धन्त । \newline
4. लो॒के ष्व॑स्पर्द्धन्ता स्पर्द्धन्त लो॒केषु॑ लो॒के ष्व॑स्पर्द्धन्त॒ स सो᳚ ऽस्पर्द्धन्त लो॒केषु॑ लो॒के ष्व॑स्पर्द्धन्त॒ सः । \newline
5. अ॒स्प॒र्द्ध॒न्त॒ स सो᳚ ऽस्पर्द्धन्ता स्पर्द्धन्त॒ स ए॒त मे॒तꣳ सो᳚ ऽस्पर्द्धन्ता स्पर्द्धन्त॒ स ए॒तम् । \newline
6. स ए॒त मे॒तꣳ स स ए॒तं ॅविष्णु॒र् विष्णु॑रे॒तꣳ स स ए॒तं ॅविष्णुः॑ । \newline
7. ए॒तं ॅविष्णु॒र् विष्णु॑रे॒त मे॒तं ॅविष्णु॑र् वाम॒नं ॅवा॑म॒नं ॅविष्णु॑रे॒त मे॒तं ॅविष्णु॑र् वाम॒नम् । \newline
8. विष्णु॑र् वाम॒नं ॅवा॑म॒नं ॅविष्णु॒र् विष्णु॑र् वाम॒न म॑पश्य दपश्यद् वाम॒नं ॅविष्णु॒र् विष्णु॑र् वाम॒न म॑पश्यत् । \newline
9. वा॒म॒न म॑पश्य दपश्यद् वाम॒नं ॅवा॑म॒न म॑पश्य॒त् तम् त म॑पश्यद् वाम॒नं ॅवा॑म॒न म॑पश्य॒त् तम् । \newline
10. अ॒प॒श्य॒त् तम् त म॑पश्य दपश्य॒त् तꣳ स्वायै॒ स्वायै॒ त म॑पश्य दपश्य॒त् तꣳ स्वायै᳚ । \newline
11. तꣳ स्वायै॒ स्वायै॒ तम् तꣳ स्वायै॑ दे॒वता॑यै दे॒वता॑यै॒ स्वायै॒ तम् तꣳ स्वायै॑ दे॒वता॑यै । \newline
12. स्वायै॑ दे॒वता॑यै दे॒वता॑यै॒ स्वायै॒ स्वायै॑ दे॒वता॑या॒ आ दे॒वता॑यै॒ स्वायै॒ स्वायै॑ दे॒वता॑या॒ आ । \newline
13. दे॒वता॑या॒ आ दे॒वता॑यै दे॒वता॑या॒ आ ऽल॑भता लभ॒ता दे॒वता॑यै दे॒वता॑या॒ आ ऽल॑भत । \newline
14. आ ऽल॑भता लभ॒ता ऽल॑भत॒ तत॒ स्ततो॑ ऽलभ॒ता ऽल॑भत॒ ततः॑ । \newline
15. अ॒ल॒भ॒त॒ तत॒ स्ततो॑ ऽलभता लभत॒ ततो॒ वै वै ततो॑ ऽलभता लभत॒ ततो॒ वै । \newline
16. ततो॒ वै वै तत॒ स्ततो॒ वै स स वै तत॒ स्ततो॒ वै सः । \newline
17. वै स स वै वै स इ॒मा नि॒मान् थ्स वै वै स इ॒मान् । \newline
18. स इ॒मा नि॒मान् थ्स स इ॒मान् ॅलो॒कान् ॅलो॒का नि॒मान् थ्स स इ॒मान् ॅलो॒कान् । \newline
19. इ॒मान् ॅलो॒कान् ॅलो॒का नि॒मा नि॒मान् ॅलो॒का न॒भ्य॑भि लो॒का नि॒मा नि॒मान् ॅलो॒का न॒भि । \newline
20. लो॒का न॒भ्य॑भि लो॒कान् ॅलो॒का न॒भ्य॑जय दजयद॒भि लो॒कान् ॅलो॒का न॒भ्य॑जयत् । \newline
21. अ॒भ्य॑जय दजयद॒भ्या᳚(1॒) भ्य॑जयद् वैष्ण॒वं ॅवै᳚ष्ण॒व म॑जयद॒भ्या᳚(1॒) भ्य॑जयद् वैष्ण॒वम् । \newline
22. अ॒ज॒य॒द् वै॒ष्ण॒वं ॅवै᳚ष्ण॒व म॑जय दजयद् वैष्ण॒वं ॅवा॑म॒नं ॅवा॑म॒नं ॅवै᳚ष्ण॒व म॑जय दजयद् वैष्ण॒वं ॅवा॑म॒नम् । \newline
23. वै॒ष्ण॒वं ॅवा॑म॒नं ॅवा॑म॒नं ॅवै᳚ष्ण॒वं ॅवै᳚ष्ण॒वं ॅवा॑म॒न मा वा॑म॒नं ॅवै᳚ष्ण॒वं ॅवै᳚ष्ण॒वं ॅवा॑म॒न मा । \newline
24. वा॒म॒न मा वा॑म॒नं ॅवा॑म॒न मा ल॑भेत लभे॒ता वा॑म॒नं ॅवा॑म॒न मा ल॑भेत । \newline
25. आ ल॑भेत लभे॒ता ल॑भेत॒ स्पर्द्ध॑मानः॒ स्पर्द्ध॑मानो लभे॒ता ल॑भेत॒ स्पर्द्ध॑मानः । \newline
26. ल॒भे॒त॒ स्पर्द्ध॑मानः॒ स्पर्द्ध॑मानो लभेत लभेत॒ स्पर्द्ध॑मानो॒ विष्णु॒र् विष्णुः॒ स्पर्द्ध॑मानो लभेत लभेत॒ स्पर्द्ध॑मानो॒ विष्णुः॑ । \newline
27. स्पर्द्ध॑मानो॒ विष्णु॒र् विष्णुः॒ स्पर्द्ध॑मानः॒ स्पर्द्ध॑मानो॒ विष्णु॑ रे॒वैव विष्णुः॒ स्पर्द्ध॑मानः॒ स्पर्द्ध॑मानो॒ विष्णु॑ रे॒व । \newline
28. विष्णु॑ रे॒वैव विष्णु॒र् विष्णु॑ रे॒व भू॒त्वा भू॒त्वैव विष्णु॒र् विष्णु॑ रे॒व भू॒त्वा । \newline
29. ए॒व भू॒त्वा भू॒त्वै वैव भू॒त्वेमा नि॒मान् भू॒त्वै वैव भू॒त्वेमान् । \newline
30. भू॒त्वेमा नि॒मान् भू॒त्वा भू॒त्वेमान् ॅलो॒कान् ॅलो॒का नि॒मान् भू॒त्वा भू॒त्वेमान् ॅलो॒कान् । \newline
31. इ॒मान् ॅलो॒कान् ॅलो॒का नि॒मा नि॒मान् ॅलो॒का न॒भ्य॑भि लो॒का नि॒मा नि॒मान् ॅलो॒का न॒भि । \newline
32. लो॒का न॒भ्य॑भि लो॒कान् ॅलो॒का न॒भि ज॑यति जयत्य॒भि लो॒कान् ॅलो॒का न॒भि ज॑यति । \newline
33. अ॒भि ज॑यति जय त्य॒भ्य॑भि ज॑यति॒ विष॑मे॒ विष॑मे जय त्य॒भ्य॑भि ज॑यति॒ विष॑मे । \newline
34. ज॒य॒ति॒ विष॑मे॒ विष॑मे जयति जयति॒ विष॑म॒ आ विष॑मे जयति जयति॒ विष॑म॒ आ । \newline
35. विष॑म॒ आ विष॑मे॒ विष॑म॒ आ ल॑भेत लभे॒ता विष॑मे॒ विष॑म॒ आ ल॑भेत । \newline
36. विष॑म॒ इति॒ वि - स॒मे॒ । \newline
37. आ ल॑भेत लभे॒ता ल॑भेत॒ विष॑मा॒ विष॑मा लभे॒ता ल॑भेत॒ विष॑माः । \newline
38. ल॒भे॒त॒ विष॑मा॒ विष॑मा लभेत लभेत॒ विष॑मा इवे व॒ विष॑मा लभेत लभेत॒ विष॑मा इव । \newline
39. विष॑मा इवे व॒ विष॑मा॒ विष॑मा इव॒ हि हीव॒ विष॑मा॒ विष॑मा इव॒ हि । \newline
40. विष॑मा॒ इति॒ वि - स॒माः॒ । \newline
41. इ॒व॒ हि हीवे॑ व॒ हीम इ॒मे हीवे॑ व॒ हीमे । \newline
42. हीम इ॒मे हि हीमे लो॒का लो॒का इ॒मे हि हीमे लो॒काः । \newline
43. इ॒मे लो॒का लो॒का इ॒म इ॒मे लो॒काः समृ॑द्ध्यै॒ समृ॑द्ध्यै लो॒का इ॒म इ॒मे लो॒काः समृ॑द्ध्यै । \newline
44. लो॒काः समृ॑द्ध्यै॒ समृ॑द्ध्यै लो॒का लो॒काः समृ॑द्ध्या॒ इन्द्रा॒ये न्द्रा॑य॒ समृ॑द्ध्यै लो॒का लो॒काः समृ॑द्ध्या॒ इन्द्रा॑य । \newline
45. समृ॑द्ध्या॒ इन्द्रा॒ये न्द्रा॑य॒ समृ॑द्ध्यै॒ समृ॑द्ध्या॒ इन्द्रा॑य मन्यु॒मते॑ मन्यु॒मत॒ इन्द्रा॑य॒ समृ॑द्ध्यै॒ समृ॑द्ध्या॒ इन्द्रा॑य मन्यु॒मते᳚ । \newline
46. समृ॑द्ध्या॒ इति॒ सं - ऋ॒द्ध्यै॒ । \newline
47. इन्द्रा॑य मन्यु॒मते॑ मन्यु॒मत॒ इन्द्रा॒ये न्द्रा॑य मन्यु॒मते॒ मन॑स्वते॒ मन॑स्वते मन्यु॒मत॒ इन्द्रा॒ये न्द्रा॑य मन्यु॒मते॒ मन॑स्वते । \newline
48. म॒न्यु॒मते॒ मन॑स्वते॒ मन॑स्वते मन्यु॒मते॑ मन्यु॒मते॒ मन॑स्वते ल॒लाम॑म् ॅल॒लाम॒म् मन॑स्वते मन्यु॒मते॑ मन्यु॒मते॒ मन॑स्वते ल॒लाम᳚म् । \newline
49. म॒न्यु॒मत॒ इति॑ मन्यु - मते᳚ । \newline
50. मन॑स्वते ल॒लाम॑म् ॅल॒लाम॒म् मन॑स्वते॒ मन॑स्वते ल॒लाम॑म् प्राशृ॒ङ्गम् प्रा॑शृ॒ङ्गम् ॅल॒लाम॒म् मन॑स्वते॒ मन॑स्वते ल॒लाम॑म् प्राशृ॒ङ्गम् । \newline
51. ल॒लाम॑म् प्राशृ॒ङ्गम् प्रा॑शृ॒ङ्गम् ॅल॒लाम॑म् ॅल॒लाम॑म् प्राशृ॒ङ्ग मा प्रा॑शृ॒ङ्गम् ॅल॒लाम॑म् ॅल॒लाम॑म् प्राशृ॒ङ्ग मा । \newline
52. प्रा॒शृ॒ङ्ग मा प्रा॑शृ॒ङ्गम् प्रा॑शृ॒ङ्ग मा ल॑भेत लभे॒ता प्रा॑शृ॒ङ्गम् प्रा॑शृ॒ङ्ग मा ल॑भेत । \newline
53. आ ल॑भेत लभे॒ता ल॑भेत सङ्ग्रा॒मे स॑ङ्ग्रा॒मे ल॑भे॒ता ल॑भेत सङ्ग्रा॒मे । \newline
54. ल॒भे॒त॒ स॒ङ्ग्रा॒मे स॑ङ्ग्रा॒मे ल॑भेत लभेत सङ्ग्रा॒मे संॅय॑त्ते॒ संॅय॑त्ते सङ्ग्रा॒मे ल॑भेत लभेत सङ्ग्रा॒मे संॅय॑त्ते । \newline
55. स॒ङ्ग्रा॒मे संॅय॑त्ते॒ संॅय॑त्ते सङ्ग्रा॒मे स॑ङ्ग्रा॒मे संॅय॑त्त इन्द्रि॒येणे᳚ न्द्रि॒येण॒ संॅय॑त्ते सङ्ग्रा॒मे स॑ङ्ग्रा॒मे संॅय॑त्त इन्द्रि॒येण॑ । \newline
56. स॒ङ्ग्रा॒म इति॑ सं - ग्रा॒मे । \newline
\pagebreak
\markright{ TS 2.1.3.2  \hfill https://www.vedavms.in \hfill}

\section{ TS 2.1.3.2 }

\textbf{TS 2.1.3.2 } \newline
\textbf{Samhita Paata} \newline

संॅय॑त्त इन्द्रि॒येण॒ वै म॒न्युना॒ मन॑सा संग्रा॒मं ज॑य॒तीन्द्र॑मे॒व म॑न्यु॒मन्तं॒ मन॑स्वन्तꣳ॒॒ स्वेन॑ भाग॒धेये॒नोप॑ धावति॒ स ए॒वास्मि॑न्निन्द्रि॒यं म॒न्युं मनो॑ दधाति॒ जय॑ति॒ तꣳ स॑ग्रां॒ममिन्द्रा॑य म॒रुत्व॑ते पृश्निस॒क्थमा ल॑भेत॒ ग्राम॑काम॒ इन्द्र॑मे॒व म॒रुत्व॑न्तꣳ॒॒ स्वेन॑ भाग॒धेये॒नोप॑ धावति॒ स ए॒वास्मै॑ स जा॒तान् प्रय॑च्छति ग्रा॒म्ये॑व भ॑वति॒ यदृ॑ष॒भस्तेनै॒ - [  ] \newline

\textbf{Pada Paata} \newline

संॅय॑त्त॒ इति॒ सं - य॒त्ते॒ । इ॒न्द्रि॒येण॑ । वै । म॒न्युना᳚ । मन॑सा । स॒ङ्ग्रा॒ममिति॑ सं - ग्रा॒मम् । ज॒य॒ति॒ । इन्द्र᳚म् । ए॒व । म॒न्यु॒मन्त॒मिति॑ मन्यु - मन्त᳚म् । मन॑स्वन्तम् । स्वेन॑ । भा॒ग॒धेये॒नेति॑ भाग - धेये॑न । उपेति॑ । धा॒व॒ति॒ । सः । ए॒व । अ॒स्मि॒न्न् । इ॒न्द्रि॒यम् । म॒न्युम् । मनः॑ । द॒धा॒ति॒ । जय॑ति । तम् । स॒ङ्ग्रा॒ममिति॑ सं - ग्रा॒मम् । इन्द्रा॑य । म॒रुत्व॑ते । पृ॒श्नि॒स॒क्थमिति॑ पृश्नि - स॒क्थम् । एति॑ । ल॒भे॒त॒ । ग्राम॑काम॒ इति॒ ग्राम॑ - का॒मः॒ । इन्द्र᳚म् । ए॒व । म॒रुत्व॑न्तम् । स्वेन॑ । भा॒ग॒धेये॒नेति॑ भाग - धेये॑न । उपेति॑ । धा॒व॒ति॒ । सः । ए॒व । अ॒स्मै॒ । स॒जा॒तानिति॑ स - जा॒तान् । प्रेति॑ । य॒च्छ॒ति॒ । ग्रा॒मी । ए॒व । भ॒व॒ति॒ । यत् । ऋ॒ष॒भः । तेन॑ ।  \newline


\textbf{Krama Paata} \newline

सम्ॅय॑त्त इन्द्रि॒येण॑ । सम्ॅय॑त्त॒ इति॒ सं - य॒त्ते॒ । इ॒न्द्रि॒येण॒ वै । वै म॒न्युना᳚ । म॒न्युना॒ मन॑सा । मन॑सा सङ्ग्रा॒मम् । स॒ङ्ग्रा॒मम् ज॑यति । स॒ङ्ग्रा॒ममिति॑ सं - ग्रा॒मम् । ज॒य॒तीन्द्र᳚म् । इन्द्र॑मे॒व । ए॒व म॑न्यु॒मन्त᳚म् । म॒न्यु॒मन्त॒म् मन॑स्वन्तम् । म॒न्यु॒मन्त॒मिति॑ मन्यु - मन्त᳚म् । मन॑स्वन्तꣳ॒॒ स्वेन॑ । स्वेन॑ भाग॒धेये॑न । भा॒ग॒धेये॒नोप॑ । भा॒ग॒धेये॒नेति॑ भाग - धेये॑न । उप॑ धावति । धा॒व॒ति॒ सः । स ए॒व । ए॒वास्मिन्न्॑ । अ॒स्मि॒न्नि॒न्द्रि॒यम् । इ॒न्द्रि॒यम् म॒न्युम् । म॒न्युम् मनः॑ । मनो॑ दधाति । द॒धा॒ति॒ जय॑ति । जय॑ति॒ तम् । तꣳ स॑ङ्ग्रा॒मम् । स॒ङ्ग्रा॒ममिन्द्रा॑य । स॒ङ्ग्रा॒ममिति॑ सं - ग्रा॒मम् । इन्द्रा॑य म॒रुत्व॑ते । म॒रुत्व॑ते पृश्ञिस॒क्थम् । पृ॒श्ञि॒स॒क्थमा । पृ॒श्ञि॒स॒क्थमिति॑ पृश्ञि - स॒क्थम् । आ ल॑भेत । ल॒भे॒त॒ ग्राम॑कामः । ग्राम॑काम॒ इन्द्र᳚म् । ग्राम॑काम॒ इति॒ ग्राम॑ - का॒मः॒ । इन्द्र॑मे॒व । ए॒व म॒रुत्व॑न्तम् । म॒रुत्व॑न्तꣳ॒॒ स्वेन॑ । स्वेन॑ भाग॒धेये॑न । भा॒ग॒धेये॒नोप॑ । भा॒ग॒धेये॒नेति॑ भाग - धेये॑न । उप॑ धावति । धा॒व॒ति॒ सः । स ए॒व । ए॒वास्मै᳚ । अ॒स्मै॒ स॒जा॒तान् । स॒जा॒तान् प्र । स॒जा॒तानिति॑ स - जा॒तान् । प्र य॑च्छति । य॒च्छ॒ति॒ ग्रा॒मी । ग्रा॒म्ये॑व । ए॒व भ॑वति । भ॒व॒ति॒ यत् । यदृ॑ष॒भः । ऋ॒ष॒भस्तेन॑ । तेनै॒न्द्रः \newline

\textbf{Jatai Paata} \newline

1. संॅय॑त्त इन्द्रि॒येणे᳚ न्द्रि॒येण॒ संॅय॑त्ते॒ संॅय॑त्त इन्द्रि॒येण॑ । \newline
2. संॅय॑त्त॒ इति॒ सं - य॒त्ते॒ । \newline
3. इ॒न्द्रि॒येण॒ वै वा इ॑न्द्रि॒येणे᳚ न्द्रि॒येण॒ वै । \newline
4. वै म॒न्युना॑ म॒न्युना॒ वै वै म॒न्युना᳚ । \newline
5. म॒न्युना॒ मन॑सा॒ मन॑सा म॒न्युना॑ म॒न्युना॒ मन॑सा । \newline
6. मन॑सा सङ्ग्रा॒मꣳ स॑ङ्ग्रा॒मम् मन॑सा॒ मन॑सा सङ्ग्रा॒मम् । \newline
7. स॒ङ्ग्रा॒मम् ज॑यति जयति सङ्ग्रा॒मꣳ स॑ङ्ग्रा॒मम् ज॑यति । \newline
8. स॒ङ्ग्रा॒ममिति॑ सं - ग्रा॒मम् । \newline
9. ज॒य॒तीन्द्र॒ मिन्द्र॑म् जयति जय॒तीन्द्र᳚म् । \newline
10. इन्द्र॑ मे॒वैवे न्द्र॒ मिन्द्र॑ मे॒व । \newline
11. ए॒व म॑न्यु॒मन्त॑म् मन्यु॒मन्त॑ मे॒वैव म॑न्यु॒मन्त᳚म् । \newline
12. म॒न्यु॒मन्त॒म् मन॑स्वन्त॒म् मन॑स्वन्तम् मन्यु॒मन्त॑म् मन्यु॒मन्त॒म् मन॑स्वन्तम् । \newline
13. म॒न्यु॒मन्त॒मिति॑ मन्यु - मन्त᳚म् । \newline
14. मन॑स्वन्तꣳ॒॒ स्वेन॒ स्वेन॒ मन॑स्वन्त॒म् मन॑स्वन्तꣳ॒॒ स्वेन॑ । \newline
15. स्वेन॑ भाग॒धेये॑न भाग॒धेये॑न॒ स्वेन॒ स्वेन॑ भाग॒धेये॑न । \newline
16. भा॒ग॒धेये॒नोपोप॑ भाग॒धेये॑न भाग॒धेये॒नोप॑ । \newline
17. भा॒ग॒धेये॒नेति॑ भाग - धेये॑न । \newline
18. उप॑ धावति धाव॒ त्युपोप॑ धावति । \newline
19. धा॒व॒ति॒ स स धा॑वति धावति॒ सः । \newline
20. स ए॒वैव स स ए॒व । \newline
21. ए॒वास्मि॑न् नस्मिन् ने॒वैवास्मिन्न्॑ । \newline
22. अ॒स्मि॒न् नि॒न्द्रि॒य मि॑न्द्रि॒य म॑स्मिन् नस्मिन् निन्द्रि॒यम् । \newline
23. इ॒न्द्रि॒यम् म॒न्युम् म॒न्यु मि॑न्द्रि॒य मि॑न्द्रि॒यम् म॒न्युम् । \newline
24. म॒न्युम् मनो॒ मनो॑ म॒न्युम् म॒न्युम् मनः॑ । \newline
25. मनो॑ दधाति दधाति॒ मनो॒ मनो॑ दधाति । \newline
26. द॒धा॒ति॒ जय॑ति॒ जय॑ति दधाति दधाति॒ जय॑ति । \newline
27. जय॑ति॒ तम् तम् जय॑ति॒ जय॑ति॒ तम् । \newline
28. तꣳ स॑ङ्ग्रा॒मꣳ स॑ङ्ग्रा॒मम् तम् तꣳ स॑ङ्ग्रा॒मम् । \newline
29. स॒ङ्ग्रा॒म मिन्द्रा॒ये न्द्रा॑य सङ्ग्रा॒मꣳ स॑ङ्ग्रा॒म मिन्द्रा॑य । \newline
30. स॒ङ्ग्रा॒ममिति॑ सं - ग्रा॒मम् । \newline
31. इन्द्रा॑य म॒रुत्व॑ते म॒रुत्व॑त॒ इन्द्रा॒ये न्द्रा॑य म॒रुत्व॑ते । \newline
32. म॒रुत्व॑ते पृश्ञिस॒क्थम् पृ॑श्ञिस॒क्थम् म॒रुत्व॑ते म॒रुत्व॑ते पृश्ञिस॒क्थम् । \newline
33. पृ॒श्ञि॒स॒क्थ मा पृ॑श्ञिस॒क्थम् पृ॑श्ञिस॒क्थ मा । \newline
34. पृ॒श्ञि॒स॒क्थमिति॑ पृश्ञि - स॒क्थम् । \newline
35. आ ल॑भेत लभे॒ता ल॑भेत । \newline
36. ल॒भे॒त॒ ग्राम॑कामो॒ ग्राम॑कामो लभेत लभेत॒ ग्राम॑कामः । \newline
37. ग्राम॑काम॒ इन्द्र॒ मिन्द्र॒म् ग्राम॑कामो॒ ग्राम॑काम॒ इन्द्र᳚म् । \newline
38. ग्राम॑काम॒ इति॒ ग्राम॑ - का॒मः॒ । \newline
39. इन्द्र॑ मे॒वैवे न्द्र॒ मिन्द्र॑ मे॒व । \newline
40. ए॒व म॒रुत्व॑न्तम् म॒रुत्व॑न्त मे॒वैव म॒रुत्व॑न्तम् । \newline
41. म॒रुत्व॑न्तꣳ॒॒ स्वेन॒ स्वेन॑ म॒रुत्व॑न्तम् म॒रुत्व॑न्तꣳ॒॒ स्वेन॑ । \newline
42. स्वेन॑ भाग॒धेये॑न भाग॒धेये॑न॒ स्वेन॒ स्वेन॑ भाग॒धेये॑न । \newline
43. भा॒ग॒धेये॒नोपोप॑ भाग॒धेये॑न भाग॒धेये॒नोप॑ । \newline
44. भा॒ग॒धेये॒नेति॑ भाग - धेये॑न । \newline
45. उप॑ धावति धाव॒ त्युपोप॑ धावति । \newline
46. धा॒व॒ति॒ स स धा॑वति धावति॒ सः । \newline
47. स ए॒वैव स स ए॒व । \newline
48. ए॒वास्मा॑ अस्मा ए॒वैवास्मै᳚ । \newline
49. अ॒स्मै॒ स॒जा॒तान् थ्स॑जा॒ता न॑स्मा अस्मै सजा॒तान् । \newline
50. स॒जा॒तान् प्र प्र स॑जा॒तान् थ्स॑जा॒तान् प्र । \newline
51. स॒जा॒तानिति॑ स - जा॒तान् । \newline
52. प्र य॑च्छति यच्छति॒ प्र प्र य॑च्छति । \newline
53. य॒च्छ॒ति॒ ग्रा॒मी ग्रा॒मी य॑च्छति यच्छति ग्रा॒मी । \newline
54. ग्रा॒म्ये॑वैव ग्रा॒मी ग्रा॒म्ये॑व । \newline
55. ए॒व भ॑वति भव त्ये॒वैव भ॑वति । \newline
56. भ॒व॒ति॒ यद् यद् भ॑वति भवति॒ यत् । \newline
57. यदृ॑ष॒भ ऋ॑ष॒भो यद् यदृ॑ष॒भः । \newline
58. ऋ॒ष॒भ स्तेन॒ तेन॑ र्.ष॒भ ऋ॑ष॒भ स्तेन॑ । \newline
59. तेनै॒न्द्र ऐ॒न्द्रस् तेन॒ तेनै॒न्द्रः । \newline

\textbf{Ghana Paata } \newline

1. संॅय॑त्त इन्द्रि॒येणे᳚ न्द्रि॒येण॒ संॅय॑त्ते॒ संॅय॑त्त इन्द्रि॒येण॒ वै वा इ॑न्द्रि॒येण॒ संॅय॑त्ते॒ संॅय॑त्त इन्द्रि॒येण॒ वै । \newline
2. संॅय॑त्त॒ इति॒ सं - य॒त्ते॒ । \newline
3. इ॒न्द्रि॒येण॒ वै वा इ॑न्द्रि॒येणे᳚ न्द्रि॒येण॒ वै म॒न्युना॑ म॒न्युना॒ वा इ॑न्द्रि॒येणे᳚ न्द्रि॒येण॒ वै म॒न्युना᳚ । \newline
4. वै म॒न्युना॑ म॒न्युना॒ वै वै म॒न्युना॒ मन॑सा॒ मन॑सा म॒न्युना॒ वै वै म॒न्युना॒ मन॑सा । \newline
5. म॒न्युना॒ मन॑सा॒ मन॑सा म॒न्युना॑ म॒न्युना॒ मन॑सा सङ्ग्रा॒मꣳ स॑ङ्ग्रा॒मम् मन॑सा म॒न्युना॑ म॒न्युना॒ मन॑सा सङ्ग्रा॒मम् । \newline
6. मन॑सा सङ्ग्रा॒मꣳ स॑ङ्ग्रा॒मम् मन॑सा॒ मन॑सा सङ्ग्रा॒मम् ज॑यति जयति सङ्ग्रा॒मम् मन॑सा॒ मन॑सा सङ्ग्रा॒मम् ज॑यति । \newline
7. स॒ङ्ग्रा॒मम् ज॑यति जयति सङ्ग्रा॒मꣳ स॑ङ्ग्रा॒मम् ज॑य॒तीन्द्र॒ मिन्द्र॑म् जयति सङ्ग्रा॒मꣳ स॑ङ्ग्रा॒मम् ज॑य॒तीन्द्र᳚म् । \newline
8. स॒ङ्ग्रा॒ममिति॑ सं - ग्रा॒मम् । \newline
9. ज॒य॒तीन्द्र॒ मिन्द्र॑म् जयति जय॒तीन्द्र॑ मे॒वैवे न्द्र॑म् जयति जय॒तीन्द्र॑ मे॒व । \newline
10. इन्द्र॑ मे॒वैवे न्द्र॒ मिन्द्र॑ मे॒व म॑न्यु॒मन्त॑म् मन्यु॒मन्त॑ मे॒वे न्द्र॒ मिन्द्र॑ मे॒व म॑न्यु॒मन्त᳚म् । \newline
11. ए॒व म॑न्यु॒मन्त॑म् मन्यु॒मन्त॑ मे॒वैव म॑न्यु॒मन्त॒म् मन॑स्वन्त॒म् मन॑स्वन्तम् मन्यु॒मन्त॑ मे॒वैव म॑न्यु॒मन्त॒म् मन॑स्वन्तम् । \newline
12. म॒न्यु॒मन्त॒म् मन॑स्वन्त॒म् मन॑स्वन्तम् मन्यु॒मन्त॑म् मन्यु॒मन्त॒म् मन॑स्वन्तꣳ॒॒ स्वेन॒ स्वेन॒ मन॑स्वन्तम् मन्यु॒मन्त॑म् मन्यु॒मन्त॒म् मन॑स्वन्तꣳ॒॒ स्वेन॑ । \newline
13. म॒न्यु॒मन्त॒मिति॑ मन्यु - मन्त᳚म् । \newline
14. मन॑स्वन्तꣳ॒॒ स्वेन॒ स्वेन॒ मन॑स्वन्त॒म् मन॑स्वन्तꣳ॒॒ स्वेन॑ भाग॒धेये॑न भाग॒धेये॑न॒ स्वेन॒ मन॑स्वन्त॒म् मन॑स्वन्तꣳ॒॒ स्वेन॑ भाग॒धेये॑न । \newline
15. स्वेन॑ भाग॒धेये॑न भाग॒धेये॑न॒ स्वेन॒ स्वेन॑ भाग॒धेये॒नो पोप॑ भाग॒धेये॑न॒ स्वेन॒ स्वेन॑ भाग॒धेये॒नोप॑ । \newline
16. भा॒ग॒धेये॒नो पोप॑ भाग॒धेये॑न भाग॒धेये॒नोप॑ धावति धाव॒त्युप॑ भाग॒धेये॑न भाग॒धेये॒नोप॑ धावति । \newline
17. भा॒ग॒धेये॒नेति॑ भाग - धेये॑न । \newline
18. उप॑ धावति धाव॒ त्युपोप॑ धावति॒ स स धा॑व॒ त्युपोप॑ धावति॒ सः । \newline
19. धा॒व॒ति॒ स स धा॑वति धावति॒ स ए॒वैव स धा॑वति धावति॒ स ए॒व । \newline
20. स ए॒वैव स स ए॒वास्मि॑न् नस्मिन् ने॒व स स ए॒वास्मिन्न्॑ । \newline
21. ए॒वास्मि॑न् नस्मिन् ने॒वैवास्मि॑न् निन्द्रि॒य मि॑न्द्रि॒य म॑स्मिन् ने॒वैवास्मि॑न् निन्द्रि॒यम् । \newline
22. अ॒स्मि॒न् नि॒न्द्रि॒य मि॑न्द्रि॒य म॑स्मिन् नस्मिन् निन्द्रि॒यम् म॒न्युम् म॒न्यु मि॑न्द्रि॒य म॑स्मिन् नस्मिन् निन्द्रि॒यम् म॒न्युम् । \newline
23. इ॒न्द्रि॒यम् म॒न्युम् म॒न्यु मि॑न्द्रि॒य मि॑न्द्रि॒यम् म॒न्युम् मनो॒ मनो॑ म॒न्यु मि॑न्द्रि॒य मि॑न्द्रि॒यम् म॒न्युम् मनः॑ । \newline
24. म॒न्युम् मनो॒ मनो॑ म॒न्युम् म॒न्युम् मनो॑ दधाति दधाति॒ मनो॑ म॒न्युम् म॒न्युम् मनो॑ दधाति । \newline
25. मनो॑ दधाति दधाति॒ मनो॒ मनो॑ दधाति॒ जय॑ति॒ जय॑ति दधाति॒ मनो॒ मनो॑ दधाति॒ जय॑ति । \newline
26. द॒धा॒ति॒ जय॑ति॒ जय॑ति दधाति दधाति॒ जय॑ति॒ तम् तम् जय॑ति दधाति दधाति॒ जय॑ति॒ तम् । \newline
27. जय॑ति॒ तम् तम् जय॑ति॒ जय॑ति॒ तꣳ स॑ङ्ग्रा॒मꣳ स॑ङ्ग्रा॒मम् तम् जय॑ति॒ जय॑ति॒ तꣳ स॑ङ्ग्रा॒मम् । \newline
28. तꣳ स॑ङ्ग्रा॒मꣳ स॑ङ्ग्रा॒मम् तम् तꣳ स॑ङ्ग्रा॒म मिन्द्रा॒ये न्द्रा॑य सङ्ग्रा॒मम् तम् तꣳ स॑ङ्ग्रा॒म मिन्द्रा॑य । \newline
29. स॒ङ्ग्रा॒म मिन्द्रा॒ये न्द्रा॑य सङ्ग्रा॒मꣳ स॑ङ्ग्रा॒म मिन्द्रा॑य म॒रुत्व॑ते म॒रुत्व॑त॒ इन्द्रा॑य सङ्ग्रा॒मꣳ स॑ङ्ग्रा॒म मिन्द्रा॑य म॒रुत्व॑ते । \newline
30. स॒ङ्ग्रा॒ममिति॑ सं - ग्रा॒मम् । \newline
31. इन्द्रा॑य म॒रुत्व॑ते म॒रुत्व॑त॒ इन्द्रा॒ये न्द्रा॑य म॒रुत्व॑ते पृश्ञिस॒क्थम् पृ॑श्ञिस॒क्थम् म॒रुत्व॑त॒ इन्द्रा॒ये न्द्रा॑य म॒रुत्व॑ते पृश्ञिस॒क्थम् । \newline
32. म॒रुत्व॑ते पृश्ञिस॒क्थम् पृ॑श्ञिस॒क्थम् म॒रुत्व॑ते म॒रुत्व॑ते पृश्ञिस॒क्थ मा पृ॑श्ञिस॒क्थम् म॒रुत्व॑ते म॒रुत्व॑ते पृश्ञिस॒क्थ मा । \newline
33. पृ॒श्ञि॒स॒क्थ मा पृ॑श्ञिस॒क्थम् पृ॑श्ञिस॒क्थ मा ल॑भेत लभे॒ता पृ॑श्ञिस॒क्थम् पृ॑श्ञिस॒क्थ मा ल॑भेत । \newline
34. पृ॒श्ञि॒स॒क्थमिति॑ पृश्ञि - स॒क्थम् । \newline
35. आ ल॑भेत लभे॒ता ल॑भेत॒ ग्राम॑कामो॒ ग्राम॑कामो लभे॒ता ल॑भेत॒ ग्राम॑कामः । \newline
36. ल॒भे॒त॒ ग्राम॑कामो॒ ग्राम॑कामो लभेत लभेत॒ ग्राम॑काम॒ इन्द्र॒ मिन्द्र॒म् ग्राम॑कामो लभेत लभेत॒ ग्राम॑काम॒ इन्द्र᳚म् । \newline
37. ग्राम॑काम॒ इन्द्र॒ मिन्द्र॒म् ग्राम॑कामो॒ ग्राम॑काम॒ इन्द्र॑ मे॒वैवे न्द्र॒म् ग्राम॑कामो॒ ग्राम॑काम॒ इन्द्र॑ मे॒व । \newline
38. ग्राम॑काम॒ इति॒ ग्राम॑ - का॒मः॒ । \newline
39. इन्द्र॑ मे॒वैवे न्द्र॒ मिन्द्र॑ मे॒व म॒रुत्व॑न्तम् म॒रुत्व॑न्त मे॒वे न्द्र॒ मिन्द्र॑ मे॒व म॒रुत्व॑न्तम् । \newline
40. ए॒व म॒रुत्व॑न्तम् म॒रुत्व॑न्त मे॒वैव म॒रुत्व॑न्तꣳ॒॒ स्वेन॒ स्वेन॑ म॒रुत्व॑न्त मे॒वैव म॒रुत्व॑न्तꣳ॒॒ स्वेन॑ । \newline
41. म॒रुत्व॑न्तꣳ॒॒ स्वेन॒ स्वेन॑ म॒रुत्व॑न्तम् म॒रुत्व॑न्तꣳ॒॒ स्वेन॑ भाग॒धेये॑न भाग॒धेये॑न॒ स्वेन॑ म॒रुत्व॑न्तम् म॒रुत्व॑न्तꣳ॒॒ स्वेन॑ भाग॒धेये॑न । \newline
42. स्वेन॑ भाग॒धेये॑न भाग॒धेये॑न॒ स्वेन॒ स्वेन॑ भाग॒धेये॒नो पोप॑ भाग॒धेये॑न॒ स्वेन॒ स्वेन॑ भाग॒धेये॒नोप॑ । \newline
43. भा॒ग॒धेये॒नो पोप॑ भाग॒धेये॑न भाग॒धेये॒नोप॑ धावति धाव॒त्युप॑ भाग॒धेये॑न भाग॒धेये॒नोप॑ धावति । \newline
44. भा॒ग॒धेये॒नेति॑ भाग - धेये॑न । \newline
45. उप॑ धावति धाव॒त्युपोप॑ धावति॒ स स धा॑व॒त्युपोप॑ धावति॒ सः । \newline
46. धा॒व॒ति॒ स स धा॑वति धावति॒ स ए॒वैव स धा॑वति धावति॒ स ए॒व । \newline
47. स ए॒वैव स स ए॒वास्मा॑ अस्मा ए॒व स स ए॒वास्मै᳚ । \newline
48. ए॒वास्मा॑ अस्मा ए॒वैवास्मै॑ सजा॒तान् थ्स॑जा॒ता न॑स्मा ए॒वैवास्मै॑ सजा॒तान् । \newline
49. अ॒स्मै॒ स॒जा॒तान् थ्स॑जा॒ता न॑स्मा अस्मै सजा॒तान् प्र प्र स॑जा॒ता न॑स्मा अस्मै सजा॒तान् प्र । \newline
50. स॒जा॒तान् प्र प्र स॑जा॒तान् थ्स॑जा॒तान् प्र य॑च्छति यच्छति॒ प्र स॑जा॒तान् थ्स॑जा॒तान् प्र य॑च्छति । \newline
51. स॒जा॒तानिति॑ स - जा॒तान् । \newline
52. प्र य॑च्छति यच्छति॒ प्र प्र य॑च्छति ग्रा॒मी ग्रा॒मी य॑च्छति॒ प्र प्र य॑च्छति ग्रा॒मी । \newline
53. य॒च्छ॒ति॒ ग्रा॒मी ग्रा॒मी य॑च्छति यच्छति ग्रा॒म्ये॑वैव ग्रा॒मी य॑च्छति यच्छति ग्रा॒म्ये॑व । \newline
54. ग्रा॒म्ये॑वैव ग्रा॒मी ग्रा॒म्ये॑व भ॑वति भवत्ये॒व ग्रा॒मी ग्रा॒म्ये॑व भ॑वति । \newline
55. ए॒व भ॑वति भव त्ये॒वैव भ॑वति॒ यद् यद् भ॑व त्ये॒वैव भ॑वति॒ यत् । \newline
56. भ॒व॒ति॒ यद् यद् भ॑वति भवति॒ यदृ॑ष॒भ ऋ॑ष॒भो यद् भ॑वति भवति॒ यदृ॑ष॒भः । \newline
57. यदृ॑ष॒भ ऋ॑ष॒भो यद् यदृ॑ष॒भ स्तेन॒ तेन॑ र्.ष॒भो यद् यदृ॑ष॒भ स्तेन॑ । \newline
58. ऋ॒ष॒भ स्तेन॒ तेन॑ र्.ष॒भ ऋ॑ष॒भ स्तेनै॒न्द्र ऐ॒न्द्र स्तेन॑ र्.ष॒भ ऋ॑ष॒भ स्तेनै॒न्द्रः । \newline
59. तेनै॒न्द्र ऐ॒न्द्र स्तेन॒ तेनै॒न्द्रो यद् यदै॒न्द्र स्तेन॒ तेनै॒न्द्रो यत् । \newline
\pagebreak
\markright{ TS 2.1.3.3  \hfill https://www.vedavms.in \hfill}

\section{ TS 2.1.3.3 }

\textbf{TS 2.1.3.3 } \newline
\textbf{Samhita Paata} \newline

-न्द्रो यत् पृश्नि॒स्तेन॑ मारु॒तः समृ॑द्ध्यै प॒श्चात् पृ॑श्निस॒क्थो भ॑वति पश्चादन्व-वसा॒यिनी॑मे॒वास्मै॒ विशं॑ करोति सौ॒म्यं ब॒भ्रुमा ल॑भे॒तान्न॑कामः सौ॒म्यं ॅवा अन्नꣳ॒॒ सोम॑मे॒व स्वेन॑ भाग॒धेये॒नोप॑ धावति॒ स ए॒वास्मा॒ अन्नं॒ प्रय॑॑च्छत्यन्ना॒द ए॒व भ॑वति ब॒भ्रुर्भ॑वत्ये॒तद्वा अन्न॑स्य रू॒पꣳ समृ॑द्ध्यै सौ॒म्यं ब॒भ्रुमा ल॑भेत॒ यमलꣳ॑ - [  ] \newline

\textbf{Pada Paata} \newline

ऐ॒न्द्रः । यत् । पृश्निः॑ । तेन॑ । मा॒रु॒तः । समृ॑द्ध्या॒ इति॒ सं-ऋ॒द्ध्यै॒ । प॒श्चात् । पृ॒श्नि॒स॒क्थ इति॑ पृश्नि - स॒क्थः । भ॒व॒ति॒ । प॒श्चा॒द॒न्व॒व॒सा॒यिनी॒मिति॑ पश्चात् - अ॒न्व॒व॒सा॒यिनी᳚म् । ए॒व । अ॒स्मै॒ । विश᳚म् । क॒रो॒ति॒ । सौ॒म्यम् । ब॒भ्रुम् । एति॑ । ल॒भे॒त॒ । अन्न॑काम॒ इत्यन्न॑ - का॒मः॒ । सौ॒म्यम् । वै । अन्न᳚म् । सोम᳚म् । ए॒व । स्वेन॑ । भा॒ग॒धेये॒नेति॑ भाग - धेये॑न । उपेति॑ । धा॒व॒ति॒ । सः । ए॒व । अ॒स्मै॒ । अन्न᳚म् । प्रेति॑ । य॒च्छ॒ति॒ । अ॒न्ना॒द इत्य॑न्न - अ॒दः । ए॒व । भ॒व॒ति॒ । ब॒भ्रुः । भ॒व॒ति॒ । ए॒तत् । वै । अन्न॑स्य । रू॒पम् । समृ॑द्ध्या॒ इति॒ सं - ऋ॒द्ध्यै॒ । सौ॒म्यम् । ब॒भ्रुम् । एति॑ । ल॒भे॒त॒ । यम् । अल᳚म् ।  \newline


\textbf{Krama Paata} \newline

ऐ॒न्द्रो यत् । यत् पृश्ञिः॑ । पृश्ञि॒स्तेन॑ । तेन॑ मारु॒तः । मा॒रु॒तः समृ॑द्ध्यै । समृ॑द्ध्यै प॒श्चात् । समृ॑द्ध्या॒ इति॒ सं - ऋ॒द्ध्यै॒ । प॒श्चात् पृ॑श्ञिस॒क्थः । पृ॒श्ञि॒स॒क्थो भ॑वति । पृ॒श्ञि॒स॒क्थ इति॑ पृश्ञि - स॒क्थः । भ॒व॒ति॒ प॒श्चा॒द॒न्व॒व॒सा॒यिनी᳚म् । प॒श्वा॒द॒न्व॒व॒सा॒यिनी॑मे॒व । प॒श्चा॒द॒न्व॒व॒सा॒यिनी॒मिति॑ पश्चात् - अ॒न्व॒व॒सा॒यिनी᳚म् । ए॒वास्मै᳚ । अ॒स्मै॒ विश᳚म् । विश॑म् करोति । क॒रो॒ति॒ सौ॒म्यम् । सौ॒म्यम् ब॒भ्रुम् । ब॒भ्रुमा । आ ल॑भेत । ल॒भे॒तान्न॑कामः । अन्न॑कामः सौ॒म्यम् । अन्न॑काम॒ इत्यन्न॑ - का॒मः॒ । सौ॒म्यं ॅवै । वा अन्न᳚म् । अन्नꣳ॒॒ सोम᳚म् । सोम॑मे॒व । ए॒व स्वेन॑ । स्वेन॑ भाग॒धेये॑न । भा॒ग॒धेये॒नोप॑ । भा॒ग॒धेये॒नेति॑ भाग - धेये॑न । उप॑ धावति । धा॒व॒ति॒ सः । स ए॒व । ए॒वास्मै᳚ । अ॒स्मा॒ अन्न᳚म् । अन्न॒म् प्र । प्र य॑च्छति । य॒च्छ॒त्य॒न्ना॒दः । अ॒न्ना॒द ए॒व । अ॒न्ना॒द इत्य॑न्न - अ॒दः । ए॒व भ॑वति । भ॒व॒ति॒ ब॒भ्रुः । ब॒भ्रुर्,भ॑वति । भ॒व॒त्ये॒तत् । ए॒तद् वै । वा अन्न॑स्य । अन्न॑स्य रू॒पम् । रू॒पꣳ समृ॑द्ध्यै । समृ॑द्ध्यै सौ॒म्यम् । समृ॑द्ध्या॒ इति॒ सं - ऋ॒द्ध्यै॒ । सौ॒म्यम् ब॒भ्रुम् । ब॒भ्रुमा । आ ल॑भेत । ल॒भे॒त॒ यम् । यमल᳚म् । अलꣳ॑ रा॒ज्याय॑ \newline

\textbf{Jatai Paata} \newline

1. ऐ॒न्द्रो यद् यदै॒न्द्र ऐ॒न्द्रो यत् । \newline
2. यत् पृश्ञिः॒ पृश्ञि॒र् यद् यत् पृश्ञिः॑ । \newline
3. पृश्ञि॒ स्तेन॒ तेन॒ पृश्ञिः॒ पृश्ञि॒ स्तेन॑ । \newline
4. तेन॑ मारु॒तो मा॑रु॒त स्तेन॒ तेन॑ मारु॒तः । \newline
5. मा॒रु॒तः समृ॑द्ध्यै॒ समृ॑द्ध्यै मारु॒तो मा॑रु॒तः समृ॑द्ध्यै । \newline
6. समृ॑द्ध्यै प॒श्चात् प॒श्चाथ् समृ॑द्ध्यै॒ समृ॑द्ध्यै प॒श्चात् । \newline
7. समृ॑द्ध्या॒ इति॒ सं - ऋ॒द्ध्यै॒ । \newline
8. प॒श्चात् पृ॑श्ञिस॒क्थः पृ॑श्ञिस॒क्थः प॒श्चात् प॒श्चात् पृ॑श्ञिस॒क्थः । \newline
9. पृ॒श्ञि॒स॒क्थो भ॑वति भवति पृश्ञिस॒क्थः पृ॑श्ञिस॒क्थो भ॑वति । \newline
10. पृ॒श्ञि॒स॒क्थ इति॑ पृश्ञि - स॒क्थः । \newline
11. भ॒व॒ति॒ प॒श्चा॒द॒न्व॒व॒सा॒यिनी᳚म् पश्चादन्ववसा॒यिनी᳚म् भवति भवति पश्चादन्ववसा॒यिनी᳚म् । \newline
12. प॒श्चा॒द॒न्व॒व॒सा॒यिनी॑ मे॒वैव प॑श्चादन्ववसा॒यिनी᳚म् पश्चादन्ववसा॒यिनी॑ मे॒व । \newline
13. प॒श्चा॒द॒न्व॒व॒सा॒यिनी॒मिति॑ पश्चात् - अ॒न्व॒व॒सा॒यिनी᳚म् । \newline
14. ए॒वास्मा॑ अस्मा ए॒वैवास्मै᳚ । \newline
15. अ॒स्मै॒ विशं॒ ॅविश॑ मस्मा अस्मै॒ विश᳚म् । \newline
16. विश॑म् करोति करोति॒ विशं॒ ॅविश॑म् करोति । \newline
17. क॒रो॒ति॒ सौ॒म्यꣳ सौ॒म्यम् क॑रोति करोति सौ॒म्यम् । \newline
18. सौ॒म्यम् ब॒भ्रुम् ब॒भ्रुꣳ सौ॒म्यꣳ सौ॒म्यम् ब॒भ्रुम् । \newline
19. ब॒भ्रु मा ब॒भ्रुम् ब॒भ्रु मा । \newline
20. आ ल॑भेत लभे॒ता ल॑भेत । \newline
21. ल॒भे॒ता न्न॑का॒मो ऽन्न॑कामो लभेत लभे॒ता न्न॑कामः । \newline
22. अन्न॑कामः सौ॒म्यꣳ सौ॒म्य मन्न॑का॒मो ऽन्न॑कामः सौ॒म्यम् । \newline
23. अन्न॑काम॒इत्यन्न॑ - का॒मः॒ । \newline
24. सौ॒म्यं ॅवै वै सौ॒म्यꣳ सौ॒म्यं ॅवै । \newline
25. वा अन्न॒ मन्नं॒ ॅवै वा अन्न᳚म् । \newline
26. अन्नꣳ॒॒ सोमꣳ॒॒ सोम॒ मन्न॒ मन्नꣳ॒॒ सोम᳚म् । \newline
27. सोम॑ मे॒वैव सोमꣳ॒॒ सोम॑ मे॒व । \newline
28. ए॒व स्वेन॒ स्वेनै॒वैव स्वेन॑ । \newline
29. स्वेन॑ भाग॒धेये॑न भाग॒धेये॑न॒ स्वेन॒ स्वेन॑ भाग॒धेये॑न । \newline
30. भा॒ग॒धेये॒नोपोप॑ भाग॒धेये॑न भाग॒धेये॒नोप॑ । \newline
31. भा॒ग॒धेये॒नेति॑ भाग - धेये॑न । \newline
32. उप॑ धावति धाव॒ त्युपोप॑ धावति । \newline
33. धा॒व॒ति॒ स स धा॑वति धावति॒ सः । \newline
34. स ए॒वैव स स ए॒व । \newline
35. ए॒वास्मा॑ अस्मा ए॒वैवास्मै᳚ । \newline
36. अ॒स्मा॒ अन्न॒ मन्न॑ मस्मा अस्मा॒ अन्न᳚म् । \newline
37. अन्न॒म् प्र प्रान्न॒ मन्न॒म् प्र । \newline
38. प्र य॑च्छति यच्छति॒ प्र प्र य॑च्छति । \newline
39. य॒च्छ॒ त्य॒न्ना॒दो᳚ ऽन्ना॒दो य॑च्छति यच्छ त्यन्ना॒दः । \newline
40. अ॒न्ना॒द ए॒वैवा न्ना॒दो᳚ ऽन्ना॒द ए॒व । \newline
41. अ॒न्ना॒द इत्य॑न्न - अ॒दः । \newline
42. ए॒व भ॑वति भव त्ये॒वैव भ॑वति । \newline
43. भ॒व॒ति॒ ब॒भ्रुर् ब॒भ्रुर् भ॑वति भवति ब॒भ्रुः । \newline
44. ब॒भ्रुर् भ॑वति भवति ब॒भ्रुर् ब॒भ्रुर् भ॑वति । \newline
45. भ॒व॒ त्ये॒त दे॒तद् भ॑वति भव त्ये॒तत् । \newline
46. ए॒तद् वै वा ए॒त दे॒तद् वै । \newline
47. वा अन्न॒स्यान्न॑स्य॒ वै वा अन्न॑स्य । \newline
48. अन्न॑स्य रू॒पꣳ रू॒प मन्न॒स्या न्न॑स्य रू॒पम् । \newline
49. रू॒पꣳ समृ॑द्ध्यै॒ समृ॑द्ध्यै रू॒पꣳ रू॒पꣳ समृ॑द्ध्यै । \newline
50. समृ॑द्ध्यै सौ॒म्यꣳ सौ॒म्यꣳ समृ॑द्ध्यै॒ समृ॑द्ध्यै सौ॒म्यम् । \newline
51. समृ॑द्ध्या॒ इति॒ सं - ऋ॒द्ध्यै॒ । \newline
52. सौ॒म्यम् ब॒भ्रुम् ब॒भ्रुꣳ सौ॒म्यꣳ सौ॒म्यम् ब॒भ्रुम् । \newline
53. ब॒भ्रु मा ब॒भ्रुम् ब॒भ्रु मा । \newline
54. आ ल॑भेत लभे॒ता ल॑भेत । \newline
55. ल॒भे॒त॒ यं ॅयम् ॅल॑भेत लभेत॒ यम् । \newline
56. य मल॒ मलं॒ ॅयं ॅय मल᳚म् । \newline
57. अलꣳ॑ रा॒ज्याय॑ रा॒ज्यायाल॒ मलꣳ॑ रा॒ज्याय॑ । \newline

\textbf{Ghana Paata } \newline

1. ऐ॒न्द्रो यद् यदै॒न्द्र ऐ॒न्द्रो यत् पृश्ञिः॒ पृश्ञि॒र् यदै॒न्द्र ऐ॒न्द्रो यत् पृश्ञिः॑ । \newline
2. यत् पृश्ञिः॒ पृश्ञि॒र् यद् यत् पृश्ञि॒ स्तेन॒ तेन॒ पृश्ञि॒र् यद् यत् पृश्ञि॒ स्तेन॑ । \newline
3. पृश्ञि॒ स्तेन॒ तेन॒ पृश्ञिः॒ पृश्ञि॒ स्तेन॑ मारु॒तो मा॑रु॒त स्तेन॒ पृश्ञिः॒ पृश्ञि॒ स्तेन॑ मारु॒तः । \newline
4. तेन॑ मारु॒तो मा॑रु॒त स्तेन॒ तेन॑ मारु॒तः समृ॑द्ध्यै॒ समृ॑द्ध्यै मारु॒त स्तेन॒ तेन॑ मारु॒तः समृ॑द्ध्यै । \newline
5. मा॒रु॒तः समृ॑द्ध्यै॒ समृ॑द्ध्यै मारु॒तो मा॑रु॒तः समृ॑द्ध्यै प॒श्चात् प॒श्चाथ् समृ॑द्ध्यै मारु॒तो मा॑रु॒तः समृ॑द्ध्यै प॒श्चात् । \newline
6. समृ॑द्ध्यै प॒श्चात् प॒श्चाथ् समृ॑द्ध्यै॒ समृ॑द्ध्यै प॒श्चात् पृ॑श्ञिस॒क्थः पृ॑श्ञिस॒क्थः प॒श्चाथ् समृ॑द्ध्यै॒ समृ॑द्ध्यै प॒श्चात् पृ॑श्ञिस॒क्थः । \newline
7. समृ॑द्ध्या॒ इति॒ सं - ऋ॒द्ध्यै॒ । \newline
8. प॒श्चात् पृ॑श्ञिस॒क्थः पृ॑श्ञिस॒क्थः प॒श्चात् प॒श्चात् पृ॑श्ञिस॒क्थो भ॑वति भवति पृश्ञिस॒क्थः प॒श्चात् प॒श्चात् पृ॑श्ञिस॒क्थो भ॑वति । \newline
9. पृ॒श्ञि॒स॒क्थो भ॑वति भवति पृश्ञिस॒क्थः पृ॑श्ञिस॒क्थो भ॑वति पश्चादन्ववसा॒यिनी᳚म् पश्चादन्ववसा॒यिनी᳚म् भवति पृश्ञिस॒क्थः पृ॑श्ञिस॒क्थो भ॑वति पश्चादन्ववसा॒यिनी᳚म् । \newline
10. पृ॒श्ञि॒स॒क्थ इति॑ पृश्ञि - स॒क्थः । \newline
11. भ॒व॒ति॒ प॒श्चा॒द॒न्व॒व॒सा॒यिनी᳚म् पश्चादन्ववसा॒यिनी᳚म् भवति भवति पश्चादन्ववसा॒यिनी॑ मे॒वैव प॑श्चादन्ववसा॒यिनी᳚म् भवति भवति पश्चादन्ववसा॒यिनी॑ मे॒व । \newline
12. प॒श्चा॒द॒न्व॒व॒सा॒यिनी॑ मे॒वैव प॑श्चादन्ववसा॒यिनी᳚म् पश्चादन्ववसा॒यिनी॑ मे॒वास्मा॑ अस्मा ए॒व प॑श्चादन्ववसा॒यिनी᳚म् पश्चादन्ववसा॒यिनी॑ मे॒वास्मै᳚ । \newline
13. प॒श्चा॒द॒न्व॒व॒सा॒यिनी॒मिति॑ पश्चात् - अ॒न्व॒व॒सा॒यिनी᳚म् । \newline
14. ए॒वास्मा॑ अस्मा ए॒वैवास्मै॒ विशं॒ ॅविश॑ मस्मा ए॒वैवास्मै॒ विश᳚म् । \newline
15. अ॒स्मै॒ विशं॒ ॅविश॑ मस्मा अस्मै॒ विश॑म् करोति करोति॒ विश॑ मस्मा अस्मै॒ विश॑म् करोति । \newline
16. विश॑म् करोति करोति॒ विशं॒ ॅविश॑म् करोति सौ॒म्यꣳ सौ॒म्यम् क॑रोति॒ विशं॒ ॅविश॑म् करोति सौ॒म्यम् । \newline
17. क॒रो॒ति॒ सौ॒म्यꣳ सौ॒म्यम् क॑रोति करोति सौ॒म्यम् ब॒भ्रुम् ब॒भ्रुꣳ सौ॒म्यम् क॑रोति करोति सौ॒म्यम् ब॒भ्रुम् । \newline
18. सौ॒म्यम् ब॒भ्रुम् ब॒भ्रुꣳ सौ॒म्यꣳ सौ॒म्यम् ब॒भ्रु मा ब॒भ्रुꣳ सौ॒म्यꣳ सौ॒म्यम् ब॒भ्रु मा । \newline
19. ब॒भ्रु मा ब॒भ्रुम् ब॒भ्रु मा ल॑भेत लभे॒ता ब॒भ्रुम् ब॒भ्रु मा ल॑भेत । \newline
20. आ ल॑भेत लभे॒ता ल॑भे॒ता न्न॑का॒मो ऽन्न॑कामो लभे॒ता ल॑भे॒ता न्न॑कामः । \newline
21. ल॒भे॒ता न्न॑का॒मो ऽन्न॑कामो लभेत लभे॒ता न्न॑कामः सौ॒म्यꣳ सौ॒म्य मन्न॑कामो लभेत लभे॒ता न्न॑कामः सौ॒म्यम् । \newline
22. अन्न॑कामः सौ॒म्यꣳ सौ॒म्य मन्न॑का॒मो ऽन्न॑कामः सौ॒म्यं ॅवै वै सौ॒म्य मन्न॑का॒मो ऽन्न॑कामः सौ॒म्यं ॅवै । \newline
23. अन्न॑काम॒इत्यन्न॑ - का॒मः॒ । \newline
24. सौ॒म्यं ॅवै वै सौ॒म्यꣳ सौ॒म्यं ॅवा अन्न॒ मन्नं॒ ॅवै सौ॒म्यꣳ सौ॒म्यं ॅवा अन्न᳚म् । \newline
25. वा अन्न॒ मन्नं॒ ॅवै वा अन्नꣳ॒॒ सोमꣳ॒॒ सोम॒ मन्नं॒ ॅवै वा अन्नꣳ॒॒ सोम᳚म् । \newline
26. अन्नꣳ॒॒ सोमꣳ॒॒ सोम॒ मन्न॒ मन्नꣳ॒॒ सोम॑ मे॒वैव सोम॒ मन्न॒ मन्नꣳ॒॒ सोम॑ मे॒व । \newline
27. सोम॑ मे॒वैव सोमꣳ॒॒ सोम॑ मे॒व स्वेन॒ स्वेनै॒व सोमꣳ॒॒ सोम॑ मे॒व स्वेन॑ । \newline
28. ए॒व स्वेन॒ स्वेनै॒वैव स्वेन॑ भाग॒धेये॑न भाग॒धेये॑न॒ स्वेनै॒वैव स्वेन॑ भाग॒धेये॑न । \newline
29. स्वेन॑ भाग॒धेये॑न भाग॒धेये॑न॒ स्वेन॒ स्वेन॑ भाग॒धेये॒नो पोप॑ भाग॒धेये॑न॒ स्वेन॒ स्वेन॑ भाग॒धेये॒नोप॑ । \newline
30. भा॒ग॒धेये॒नो पोप॑ भाग॒धेये॑न भाग॒धेये॒नोप॑ धावति धाव॒त्युप॑ भाग॒धेये॑न भाग॒धेये॒नोप॑ धावति । \newline
31. भा॒ग॒धेये॒नेति॑ भाग - धेये॑न । \newline
32. उप॑ धावति धाव॒ त्युपोप॑ धावति॒ स स धा॑व॒ त्युपोप॑ धावति॒ सः । \newline
33. धा॒व॒ति॒ स स धा॑वति धावति॒ स ए॒वैव स धा॑वति धावति॒ स ए॒व । \newline
34. स ए॒वैव स स ए॒वास्मा॑ अस्मा ए॒व स स ए॒वास्मै᳚ । \newline
35. ए॒वास्मा॑ अस्मा ए॒वैवास्मा॒ अन्न॒ मन्न॑ मस्मा ए॒वैवास्मा॒ अन्न᳚म् । \newline
36. अ॒स्मा॒ अन्न॒ मन्न॑ मस्मा अस्मा॒ अन्न॒म् प्र प्रान्न॑ मस्मा अस्मा॒ अन्न॒म् प्र । \newline
37. अन्न॒म् प्र प्रान्न॒ मन्न॒म् प्र य॑च्छति यच्छति॒ प्रान्न॒ मन्न॒म् प्र य॑च्छति । \newline
38. प्र य॑च्छति यच्छति॒ प्र प्र य॑च्छत्यन्ना॒दो᳚ ऽन्ना॒दो य॑च्छति॒ प्र प्र य॑च्छत्यन्ना॒दः । \newline
39. य॒च्छ॒ त्य॒न्ना॒दो᳚ ऽन्ना॒दो य॑च्छति यच्छ त्यन्ना॒द ए॒वैवा न्ना॒दो य॑च्छति यच्छ त्यन्ना॒द ए॒व । \newline
40. अ॒न्ना॒द ए॒वैवा न्ना॒दो᳚ ऽन्ना॒द ए॒व भ॑वति भवत्ये॒वा न्ना॒दो᳚ ऽन्ना॒द ए॒व भ॑वति । \newline
41. अ॒न्ना॒द इत्य॑न्न - अ॒दः । \newline
42. ए॒व भ॑वति भवत्ये॒वैव भ॑वति ब॒भ्रुर् ब॒भ्रुर् भ॑वत्ये॒वैव भ॑वति ब॒भ्रुः । \newline
43. भ॒व॒ति॒ ब॒भ्रुर् ब॒भ्रुर् भ॑वति भवति ब॒भ्रुर् भ॑वति भवति ब॒भ्रुर् भ॑वति भवति ब॒भ्रुर् भ॑वति । \newline
44. ब॒भ्रुर् भ॑वति भवति ब॒भ्रुर् ब॒भ्रुर् भ॑व त्ये॒तदे॒तद् भ॑वति ब॒भ्रुर् ब॒भ्रुर् भ॑वत्ये॒तत् । \newline
45. भ॒व॒ त्ये॒तदे॒तद् भ॑वति भवत्ये॒तद् वै वा ए॒तद् भ॑वति भवत्ये॒तद् वै । \newline
46. ए॒तद् वै वा ए॒तदे॒तद् वा अन्न॒स्या न्न॑स्य॒ वा ए॒तदे॒तद् वा अन्न॑स्य । \newline
47. वा अन्न॒स्यान्न॑स्य॒ वै वा अन्न॑स्य रू॒पꣳ रू॒प मन्न॑स्य॒ वै वा अन्न॑स्य रू॒पम् । \newline
48. अन्न॑स्य रू॒पꣳ रू॒प मन्न॒स्यान्न॑स्य रू॒पꣳ समृ॑द्ध्यै॒ समृ॑द्ध्यै रू॒प मन्न॒स्यान्न॑स्य रू॒पꣳ समृ॑द्ध्यै । \newline
49. रू॒पꣳ समृ॑द्ध्यै॒ समृ॑द्ध्यै रू॒पꣳ रू॒पꣳ समृ॑द्ध्यै सौ॒म्यꣳ सौ॒म्यꣳ समृ॑द्ध्यै रू॒पꣳ रू॒पꣳ समृ॑द्ध्यै सौ॒म्यम् । \newline
50. समृ॑द्ध्यै सौ॒म्यꣳ सौ॒म्यꣳ समृ॑द्ध्यै॒ समृ॑द्ध्यै सौ॒म्यम् ब॒भ्रुम् ब॒भ्रुꣳ सौ॒म्यꣳ समृ॑द्ध्यै॒ समृ॑द्ध्यै सौ॒म्यम् ब॒भ्रुम् । \newline
51. समृ॑द्ध्या॒ इति॒ सं - ऋ॒द्ध्यै॒ । \newline
52. सौ॒म्यम् ब॒भ्रुम् ब॒भ्रुꣳ सौ॒म्यꣳ सौ॒म्यम् ब॒भ्रु मा ब॒भ्रुꣳ सौ॒म्यꣳ सौ॒म्यम् ब॒भ्रु मा । \newline
53. ब॒भ्रु मा ब॒भ्रुम् ब॒भ्रु मा ल॑भेत लभे॒ता ब॒भ्रुम् ब॒भ्रु मा ल॑भेत । \newline
54. आ ल॑भेत लभे॒ता ल॑भेत॒ यं ॅयम् ॅल॑भे॒ता ल॑भेत॒ यम् । \newline
55. ल॒भे॒त॒ यं ॅयम् ॅल॑भेत लभेत॒ य मल॒ मलं॒ ॅयम् ॅल॑भेत लभेत॒ य मल᳚म् । \newline
56. य मल॒ मलं॒ ॅयं ॅय मलꣳ॑ रा॒ज्याय॑ रा॒ज्यायालं॒ ॅयं ॅय मलꣳ॑ रा॒ज्याय॑ । \newline
57. अलꣳ॑ रा॒ज्याय॑ रा॒ज्यायाल॒ मलꣳ॑ रा॒ज्याय॒ सन्तꣳ॒॒ सन्तꣳ॑ रा॒ज्यायाल॒ मलꣳ॑ रा॒ज्याय॒ सन्त᳚म् । \newline
\pagebreak
\markright{ TS 2.1.3.4  \hfill https://www.vedavms.in \hfill}

\section{ TS 2.1.3.4 }

\textbf{TS 2.1.3.4 } \newline
\textbf{Samhita Paata} \newline

रा॒ज्याय॒ सन्तꣳ॑ रा॒ज्यं नोप॒नमे᳚थ् सौ॒म्यं ॅवै रा॒ज्य सोम॑मे॒व स्वेन॑ भाग॒धेये॒नोप॑ धावति॒ स ए॒वास्मै॑ रा॒ज्यं प्रय॑च्छ॒त्युपै॑नꣳ रा॒ज्यं न॑मति ब॒भ्रुर्भ॑वत्ये॒तद् वै सोम॑स्य रू॒पꣳ समृ॑द्ध्या॒ इन्द्रा॑य वृत्र॒तुरे॑ ल॒लामं॑ प्राशृ॒ङ्गमा ल॑भेत ग॒तश्रीः᳚ प्रति॒ष्ठाका॑मः पा॒प्मान॑मे॒व वृ॒त्रं ती॒र्त्वा प्र॑ति॒ष्ठां ग॑च्छ॒तीन्द्रा॑याभिमाति॒घ्ने ल॒लामं॑ प्राशृ॒ङ्गमा - [  ] \newline

\textbf{Pada Paata} \newline

रा॒ज्याय॑ । सन्त᳚म् । रा॒ज्यम् । न । उ॒प॒नमे॒दित्यु॑प-नमे᳚त् । सौ॒म्यम् । वै । रा॒ज्यम् । सोम᳚म् । ए॒व । स्वेन॑ । भा॒ग॒धेये॒नेति॑ भाग - धेये॑न । उपेति॑ । धा॒व॒ति॒ । सः । ए॒व । अ॒स्मै॒ । रा॒ज्यम् । प्रेति॑ । य॒च्छ॒ति॒ । उपेति॑ । ए॒न॒म् । रा॒ज्यम् । न॒म॒ति॒ । ब॒भ्रुः । भ॒व॒ति॒ । ए॒तत् । वै । सोम॑स्य । रू॒पम् । समृ॑द्ध्या॒ इति॒ सं - ऋ॒द्ध्यै॒ । इन्द्रा॑य । वृ॒त्र॒तुर॒ इति॑ वृत्र - तुरे᳚ । ल॒लाम᳚म् । प्रा॒शृ॒ङ्गम् । एति॑ । ल॒भे॒त॒ । ग॒तश्री॒रिति॑ ग॒त - श्रीः॒ । प्र॒ति॒ष्ठाका॑म॒ इति॑ प्रति॒ष्ठा - का॒मः॒ । पा॒प्मान᳚म् । ए॒व । वृ॒त्रम् । ती॒र्त्वा । प्र॒ति॒ष्ठामिति॑ प्रति - स्थाम् । ग॒च्छ॒ति॒ । इन्द्रा॑य । अ॒भि॒मा॒ति॒घ्न इत्य॑भिमाति - घ्ने । ल॒लाम᳚म् । प्रा॒शृ॒ङ्गम् । एति॑ ।  \newline


\textbf{Krama Paata} \newline

रा॒ज्याय॒ सन्त᳚म् । सन्तꣳ॑ रा॒ज्यम् । रा॒ज्यम् न । नोप॒नमे᳚त् । उ॒प॒नमे᳚थ् सौ॒म्यम् । उ॒प॒नमे॒दित्यु॑प - नमे᳚त् । सौ॒म्यं ॅवै । वै रा॒ज्यम् । रा॒ज्यꣳ सोम᳚म् । सोम॑मे॒व । ए॒व स्वेन॑ । स्वेन॑ भाग॒धेये॑न । भा॒ग॒धेये॒नोप॑ । भा॒ग॒धेये॒नेति॑ भाग - धेये॑न । उप॑ धावति । धा॒व॒ति॒ सः । स ए॒व । ए॒वास्मै᳚ । अ॒स्मै॒ रा॒ज्यम् । रा॒ज्यम् प्र । प्र य॑च्छति । य॒च्छ॒त्युप॑ । उपै॑नम् । ए॒नꣳ॒॒ रा॒ज्यम् । रा॒ज्यम् न॑मति । न॒म॒ति॒ ब॒भ्रुः । ब॒भ्रुर् भ॑वति । भ॒व॒त्ये॒तत् । ए॒तद् वै । वै सोम॑स्य । सोम॑स्य रू॒पम् । रू॒पꣳ समृ॑द्ध्यै । समृ॑द्ध्या॒ इन्द्रा॑य । समृ॑द्ध्या॒ इति॒ सं - ऋ॒द्ध्यै॒ । इन्द्रा॑य वृत्र॒तुरे᳚ । वृ॒त्र॒तुरे॑ ल॒लाम᳚म् । वृ॒त्र॒तुर॒ इति॑ वृत्र - तुरे᳚ । ल॒लाम॑म् प्राशृ॒ङ्गम् । प्रा॒शृ॒ङ्गमा । आ ल॑भेत । ल॒भे॒त॒ ग॒तश्रीः᳚ । ग॒तश्रीः᳚ प्रति॒ष्ठाका॑मः । ग॒तश्री॒रिति॑ ग॒त - श्रीः॒ । प्र॒ति॒ष्ठाका॑मः पा॒प्मान᳚म् । प्र॒ति॒ष्ठाका॑म॒ इति॑ प्रति॒ष्ठा - का॒मः॒ । पा॒प्मान॑मे॒व । ए॒व वृ॒त्रम् । वृ॒त्रम् ती॒र्त्वा । ती॒र्त्वा प्र॑ति॒ष्ठाम् । प्र॒ति॒ष्ठाम् ग॑च्छति । प्र॒ति॒ष्ठामिति॑ प्रति - स्थाम् । ग॒च्छ॒तीन्द्रा॑य । इन्द्रा॑याभिमाति॒घ्ने । अ॒भि॒मा॒ति॒घ्ने ल॒लाम᳚म् । अ॒भि॒॒मा॒ति॒घ्न इत्य॑भिमाति - घ्ने । ल॒लाम॑म् प्राशृ॒ङ्गम् ( ) । प्रा॒शृ॒ङ्गमा । आ ल॑भेत \newline

\textbf{Jatai Paata} \newline

1. रा॒ज्याय॒ सन्तꣳ॒॒ सन्तꣳ॑ रा॒ज्याय॑ रा॒ज्याय॒ सन्त᳚म् । \newline
2. सन्तꣳ॑ रा॒ज्यꣳ रा॒ज्यꣳ सन्तꣳ॒॒ सन्तꣳ॑ रा॒ज्यम् । \newline
3. रा॒ज्यम् न न रा॒ज्यꣳ रा॒ज्यम् न । \newline
4. नोप॒नमे॑ दुप॒नमे॒न् न नोप॒नमे᳚त् । \newline
5. उ॒प॒नमे᳚थ् सौ॒म्यꣳ सौ॒म्य मु॑प॒नमे॑ दुप॒नमे᳚थ् सौ॒म्यम् । \newline
6. उ॒प॒नमे॒दित्यु॑प - नमे᳚त् । \newline
7. सौ॒म्यं ॅवै वै सौ॒म्यꣳ सौ॒म्यं ॅवै । \newline
8. वै रा॒ज्यꣳ रा॒ज्यं ॅवै वै रा॒ज्यम् । \newline
9. रा॒ज्यꣳ सोमꣳ॒॒ सोमꣳ॑ रा॒ज्यꣳ रा॒ज्यꣳ सोम᳚म् । \newline
10. सोम॑ मे॒वैव सोमꣳ॒॒ सोम॑ मे॒व । \newline
11. ए॒व स्वेन॒ स्वेनै॒वैव स्वेन॑ । \newline
12. स्वेन॑ भाग॒धेये॑न भाग॒धेये॑न॒ स्वेन॒ स्वेन॑ भाग॒धेये॑न । \newline
13. भा॒ग॒धेये॒नोपोप॑ भाग॒धेये॑न भाग॒धेये॒नोप॑ । \newline
14. भा॒ग॒धेये॒नेति॑ भाग - धेये॑न । \newline
15. उप॑ धावति धाव॒ त्युपोप॑ धावति । \newline
16. धा॒व॒ति॒ स स धा॑वति धावति॒ सः । \newline
17. स ए॒वैव स स ए॒व । \newline
18. ए॒वास्मा॑ अस्मा ए॒वैवास्मै᳚ । \newline
19. अ॒स्मै॒ रा॒ज्यꣳ रा॒ज्य म॑स्मा अस्मै रा॒ज्यम् । \newline
20. रा॒ज्यम् प्र प्र रा॒ज्यꣳ रा॒ज्यम् प्र । \newline
21. प्र य॑च्छति यच्छति॒ प्र प्र य॑च्छति । \newline
22. य॒च्छ॒ त्युपोप॑ यच्छति यच्छ॒ त्युप॑ । \newline
23. उपै॑न मेन॒ मुपोपै॑नम् । \newline
24. ए॒नꣳ॒॒ रा॒ज्यꣳ रा॒ज्य मे॑न मेनꣳ रा॒ज्यम् । \newline
25. रा॒ज्यम् न॑मति नमति रा॒ज्यꣳ रा॒ज्यम् न॑मति । \newline
26. न॒म॒ति॒ ब॒भ्रुर् ब॒भ्रुर् न॑मति नमति ब॒भ्रुः । \newline
27. ब॒भ्रुर् भ॑वति भवति ब॒भ्रुर् ब॒भ्रुर् भ॑वति । \newline
28. भ॒व॒ त्ये॒त दे॒तद् भ॑वति भव त्ये॒तत् । \newline
29. ए॒तद् वै वा ए॒त दे॒तद् वै । \newline
30. वै सोम॑स्य॒ सोम॑स्य॒ वै वै सोम॑स्य । \newline
31. सोम॑स्य रू॒पꣳ रू॒पꣳ सोम॑स्य॒ सोम॑स्य रू॒पम् । \newline
32. रू॒पꣳ समृ॑द्ध्यै॒ समृ॑द्ध्यै रू॒पꣳ रू॒पꣳ समृ॑द्ध्यै । \newline
33. समृ॑द्ध्या॒ इन्द्रा॒ये न्द्रा॑य॒ समृ॑द्ध्यै॒ समृ॑द्ध्या॒ इन्द्रा॑य । \newline
34. समृ॑द्ध्या॒ इति॒ सं - ऋ॒द्ध्यै॒ । \newline
35. इन्द्रा॑य वृत्र॒तुरे॑ वृत्र॒तुर॒ इन्द्रा॒ये न्द्रा॑य वृत्र॒तुरे᳚ । \newline
36. वृ॒त्र॒तुरे॑ ल॒लाम॑म् ॅल॒लामं॑ ॅवृत्र॒तुरे॑ वृत्र॒तुरे॑ ल॒लाम᳚म् । \newline
37. वृ॒त्र॒तुर॒ इति॑ वृत्र - तुरे᳚ । \newline
38. ल॒लाम॑म् प्राशृ॒ङ्गम् प्रा॑शृ॒ङ्गम् ॅल॒लाम॑म् ॅल॒लाम॑म् प्राशृ॒ङ्गम् । \newline
39. प्रा॒शृ॒ङ्ग मा प्रा॑शृ॒ङ्गम् प्रा॑शृ॒ङ्ग मा । \newline
40. आ ल॑भेत लभे॒ता ल॑भेत । \newline
41. ल॒भे॒त॒ ग॒तश्री᳚र् ग॒तश्री᳚र् लभेत लभेत ग॒तश्रीः᳚ । \newline
42. ग॒तश्रीः᳚ प्रति॒ष्ठाका॑मः प्रति॒ष्ठाका॑मो ग॒तश्री᳚र् ग॒तश्रीः᳚ प्रति॒ष्ठाका॑मः । \newline
43. ग॒तश्री॒रिति॑ ग॒त - श्रीः॒ । \newline
44. प्र॒ति॒ष्ठाका॑मः पा॒प्मान॑म् पा॒प्मान॑म् प्रति॒ष्ठाका॑मः प्रति॒ष्ठाका॑मः पा॒प्मान᳚म् । \newline
45. प्र॒ति॒ष्ठाका॑म॒ इति॑ प्रति॒ष्ठा - का॒मः॒ । \newline
46. पा॒प्मान॑ मे॒वैव पा॒प्मान॑म् पा॒प्मान॑ मे॒व । \newline
47. ए॒व वृ॒त्रं ॅवृ॒त्र मे॒वैव वृ॒त्रम् । \newline
48. वृ॒त्रम् ती॒र्त्वा ती॒र्त्वा वृ॒त्रं ॅवृ॒त्रम् ती॒र्त्वा । \newline
49. ती॒र्त्वा प्र॑ति॒ष्ठाम् प्र॑ति॒ष्ठाम् ती॒र्त्वा ती॒र्त्वा प्र॑ति॒ष्ठाम् । \newline
50. प्र॒ति॒ष्ठाम् ग॑च्छति गच्छति प्रति॒ष्ठाम् प्र॑ति॒ष्ठाम् ग॑च्छति । \newline
51. प्र॒ति॒ष्ठामिति॑ प्रति - स्थाम् । \newline
52. ग॒च्छ॒तीन्द्रा॒ये न्द्रा॑य गच्छति गच्छ॒तीन्द्रा॑य । \newline
53. इन्द्रा॑ याभिमाति॒घ्ने॑ ऽभिमाति॒घ्न इन्द्रा॒ये न्द्रा॑याभिमाति॒घ्ने । \newline
54. अ॒भि॒मा॒ति॒घ्ने ल॒लाम॑म् ॅल॒लाम॑ मभिमाति॒घ्ने॑ ऽभिमाति॒घ्ने ल॒लाम᳚म् । \newline
55. अ॒भि॒मा॒ति॒घ्न इत्य॑भिमाति - घ्ने । \newline
56. ल॒लाम॑म् प्राशृ॒ङ्गम् प्रा॑शृ॒ङ्गम् ॅल॒लाम॑म् ॅल॒लाम॑म् प्राशृ॒ङ्गम् । \newline
57. प्रा॒शृ॒ङ्ग मा प्रा॑शृ॒ङ्गम् प्रा॑शृ॒ङ्ग मा । \newline
58. आ ल॑भेत लभे॒ता ल॑भेत । \newline

\textbf{Ghana Paata } \newline

1. रा॒ज्याय॒ सन्तꣳ॒॒ सन्तꣳ॑ रा॒ज्याय॑ रा॒ज्याय॒ सन्तꣳ॑ रा॒ज्यꣳ रा॒ज्यꣳ सन्तꣳ॑ रा॒ज्याय॑ रा॒ज्याय॒ सन्तꣳ॑ रा॒ज्यम् । \newline
2. सन्तꣳ॑ रा॒ज्यꣳ रा॒ज्यꣳ सन्तꣳ॒॒ सन्तꣳ॑ रा॒ज्यन्न न रा॒ज्यꣳ सन्तꣳ॒॒ सन्तꣳ॑ रा॒ज्यन्न । \newline
3. रा॒ज्यन्न न रा॒ज्यꣳ रा॒ज्यन् नोप॒नमे॑ दुप॒नमे॒न् न रा॒ज्यꣳ रा॒ज्यन् नोप॒नमे᳚त् । \newline
4. नोप॒नमे॑ दुप॒नमे॒न् न नोप॒नमे᳚थ् सौ॒म्यꣳ सौ॒म्य मु॑प॒नमे॒न् न नोप॒नमे᳚थ् सौ॒म्यम् । \newline
5. उ॒प॒नमे᳚थ् सौ॒म्यꣳ सौ॒म्य मु॑प॒नमे॑ दुप॒नमे᳚थ् सौ॒म्यं ॅवै वै सौ॒म्य मु॑प॒नमे॑ दुप॒नमे᳚थ् सौ॒म्यं ॅवै । \newline
6. उ॒प॒नमे॒दित्यु॑प - नमे᳚त् । \newline
7. सौ॒म्यं ॅवै वै सौ॒म्यꣳ सौ॒म्यं ॅवै रा॒ज्यꣳ रा॒ज्यं ॅवै सौ॒म्यꣳ सौ॒म्यं ॅवै रा॒ज्यम् । \newline
8. वै रा॒ज्यꣳ रा॒ज्यं ॅवै वै रा॒ज्यꣳ सोमꣳ॒॒ सोमꣳ॑ रा॒ज्यं ॅवै वै रा॒ज्यꣳ सोम᳚म् । \newline
9. रा॒ज्यꣳ सोमꣳ॒॒ सोमꣳ॑ रा॒ज्यꣳ रा॒ज्यꣳ सोम॑ मे॒वैव सोमꣳ॑ रा॒ज्यꣳ रा॒ज्यꣳ सोम॑ मे॒व । \newline
10. सोम॑ मे॒वैव सोमꣳ॒॒ सोम॑ मे॒व स्वेन॒ स्वेनै॒व सोमꣳ॒॒ सोम॑ मे॒व स्वेन॑ । \newline
11. ए॒व स्वेन॒ स्वेनै॒वैव स्वेन॑ भाग॒धेये॑न भाग॒धेये॑न॒ स्वेनै॒वैव स्वेन॑ भाग॒धेये॑न । \newline
12. स्वेन॑ भाग॒धेये॑न भाग॒धेये॑न॒ स्वेन॒ स्वेन॑ भाग॒धेये॒नो पोप॑ भाग॒धेये॑न॒ स्वेन॒ स्वेन॑ भाग॒धेये॒नोप॑ । \newline
13. भा॒ग॒धेये॒नो पोप॑ भाग॒धेये॑न भाग॒धेये॒नोप॑ धावति धाव॒त्युप॑ भाग॒धेये॑न भाग॒धेये॒नोप॑ धावति । \newline
14. भा॒ग॒धेये॒नेति॑ भाग - धेये॑न । \newline
15. उप॑ धावति धाव॒ त्युपोप॑ धावति॒ स स धा॑व॒ त्युपोप॑ धावति॒ सः । \newline
16. धा॒व॒ति॒ स स धा॑वति धावति॒ स ए॒वैव स धा॑वति धावति॒ स ए॒व । \newline
17. स ए॒वैव स स ए॒वास्मा॑ अस्मा ए॒व स स ए॒वास्मै᳚ । \newline
18. ए॒वास्मा॑ अस्मा ए॒वैवास्मै॑ रा॒ज्यꣳ रा॒ज्य म॑स्मा ए॒वैवास्मै॑ रा॒ज्यम् । \newline
19. अ॒स्मै॒ रा॒ज्यꣳ रा॒ज्य म॑स्मा अस्मै रा॒ज्यम् प्र प्र रा॒ज्य म॑स्मा अस्मै रा॒ज्यम् प्र । \newline
20. रा॒ज्यम् प्र प्र रा॒ज्यꣳ रा॒ज्यम् प्र य॑च्छति यच्छति॒ प्र रा॒ज्यꣳ रा॒ज्यम् प्र य॑च्छति । \newline
21. प्र य॑च्छति यच्छति॒ प्र प्र य॑च्छ॒ त्युपोप॑ यच्छति॒ प्र प्र य॑च्छ॒ त्युप॑ । \newline
22. य॒च्छ॒ त्युपोप॑ यच्छति यच्छ॒ त्युपै॑न मेन॒ मुप॑ यच्छति यच्छ॒ त्युपै॑नम् । \newline
23. उपै॑न मेन॒ मुपोपै॑नꣳ रा॒ज्यꣳ रा॒ज्य मे॑न॒ मुपोपै॑नꣳ रा॒ज्यम् । \newline
24. ए॒नꣳ॒॒ रा॒ज्यꣳ रा॒ज्य मे॑न मेनꣳ रा॒ज्यन् न॑मति नमति रा॒ज्य मे॑न मेनꣳ रा॒ज्यन् न॑मति । \newline
25. रा॒ज्यन् न॑मति नमति रा॒ज्यꣳ रा॒ज्यन् न॑मति ब॒भ्रुर् ब॒भ्रुर् न॑मति रा॒ज्यꣳ रा॒ज्यन् न॑मति ब॒भ्रुः । \newline
26. न॒म॒ति॒ ब॒भ्रुर् ब॒भ्रुर् न॑मति नमति ब॒भ्रुर् भ॑वति भवति ब॒भ्रुर् न॑मति नमति ब॒भ्रुर् भ॑वति । \newline
27. ब॒भ्रुर् भ॑वति भवति ब॒भ्रुर् ब॒भ्रुर् भ॑व त्ये॒तदे॒तद् भ॑वति ब॒भ्रुर् ब॒भ्रुर् भ॑वत्ये॒तत् । \newline
28. भ॒व॒ त्ये॒तदे॒तद् भ॑वति भवत्ये॒तद् वै वा ए॒तद् भ॑वति भवत्ये॒तद् वै । \newline
29. ए॒तद् वै वा ए॒तदे॒तद् वै सोम॑स्य॒ सोम॑स्य॒ वा ए॒तदे॒तद् वै सोम॑स्य । \newline
30. वै सोम॑स्य॒ सोम॑स्य॒ वै वै सोम॑स्य रू॒पꣳ रू॒पꣳ सोम॑स्य॒ वै वै सोम॑स्य रू॒पम् । \newline
31. सोम॑स्य रू॒पꣳ रू॒पꣳ सोम॑स्य॒ सोम॑स्य रू॒पꣳ समृ॑द्ध्यै॒ समृ॑द्ध्यै रू॒पꣳ सोम॑स्य॒ सोम॑स्य रू॒पꣳ समृ॑द्ध्यै । \newline
32. रू॒पꣳ समृ॑द्ध्यै॒ समृ॑द्ध्यै रू॒पꣳ रू॒पꣳ समृ॑द्ध्या॒ इन्द्रा॒ये न्द्रा॑य॒ समृ॑द्ध्यै रू॒पꣳ रू॒पꣳ समृ॑द्ध्या॒ इन्द्रा॑य । \newline
33. समृ॑द्ध्या॒ इन्द्रा॒ये न्द्रा॑य॒ समृ॑द्ध्यै॒ समृ॑द्ध्या॒ इन्द्रा॑य वृत्र॒तुरे॑ वृत्र॒तुर॒ इन्द्रा॑य॒ समृ॑द्ध्यै॒ समृ॑द्ध्या॒ इन्द्रा॑य वृत्र॒तुरे᳚ । \newline
34. समृ॑द्ध्या॒ इति॒ सं - ऋ॒द्ध्यै॒ । \newline
35. इन्द्रा॑य वृत्र॒तुरे॑ वृत्र॒तुर॒ इन्द्रा॒ये न्द्रा॑य वृत्र॒तुरे॑ ल॒लाम॑म् ॅल॒लामं॑ ॅवृत्र॒तुर॒ इन्द्रा॒ये न्द्रा॑य वृत्र॒तुरे॑ ल॒लाम᳚म् । \newline
36. वृ॒त्र॒तुरे॑ ल॒लाम॑म् ॅल॒लामं॑ ॅवृत्र॒तुरे॑ वृत्र॒तुरे॑ ल॒लाम॑म् प्राशृ॒ङ्गम् प्रा॑शृ॒ङ्गम् ॅल॒लामं॑ ॅवृत्र॒तुरे॑ वृत्र॒तुरे॑ ल॒लाम॑म् प्राशृ॒ङ्गम् । \newline
37. वृ॒त्र॒तुर॒ इति॑ वृत्र - तुरे᳚ । \newline
38. ल॒लाम॑म् प्राशृ॒ङ्गम् प्रा॑शृ॒ङ्गम् ॅल॒लाम॑म् ॅल॒लाम॑म् प्राशृ॒ङ्ग मा प्रा॑शृ॒ङ्गम् ॅल॒लाम॑म् ॅल॒लाम॑म् प्राशृ॒ङ्ग मा । \newline
39. प्रा॒शृ॒ङ्ग मा प्रा॑शृ॒ङ्गम् प्रा॑शृ॒ङ्ग मा ल॑भेत लभे॒ता प्रा॑शृ॒ङ्गम् प्रा॑शृ॒ङ्ग मा ल॑भेत । \newline
40. आ ल॑भेत लभे॒ता ल॑भेत ग॒तश्री᳚र् ग॒तश्री᳚र् लभे॒ता ल॑भेत ग॒तश्रीः᳚ । \newline
41. ल॒भे॒त॒ ग॒तश्री᳚र् ग॒तश्री᳚र् लभेत लभेत ग॒तश्रीः᳚ प्रति॒ष्ठाका॑मः प्रति॒ष्ठाका॑मो ग॒तश्री᳚र् लभेत लभेत ग॒तश्रीः᳚ प्रति॒ष्ठाका॑मः । \newline
42. ग॒तश्रीः᳚ प्रति॒ष्ठाका॑मः प्रति॒ष्ठाका॑मो ग॒तश्री᳚र् ग॒तश्रीः᳚ प्रति॒ष्ठाका॑मः पा॒प्मान॑म् पा॒प्मान॑म् प्रति॒ष्ठाका॑मो ग॒तश्री᳚र् ग॒तश्रीः᳚ प्रति॒ष्ठाका॑मः पा॒प्मान᳚म् । \newline
43. ग॒तश्री॒रिति॑ ग॒त - श्रीः॒ । \newline
44. प्र॒ति॒ष्ठाका॑मः पा॒प्मान॑म् पा॒प्मान॑म् प्रति॒ष्ठाका॑मः प्रति॒ष्ठाका॑मः पा॒प्मान॑ मे॒वैव पा॒प्मान॑म् प्रति॒ष्ठाका॑मः प्रति॒ष्ठाका॑मः पा॒प्मान॑ मे॒व । \newline
45. प्र॒ति॒ष्ठाका॑म॒ इति॑ प्रति॒ष्ठा - का॒मः॒ । \newline
46. पा॒प्मान॑ मे॒वैव पा॒प्मान॑म् पा॒प्मान॑ मे॒व वृ॒त्रं ॅवृ॒त्र मे॒व पा॒प्मान॑म् पा॒प्मान॑ मे॒व वृ॒त्रम् । \newline
47. ए॒व वृ॒त्रं ॅवृ॒त्र मे॒वैव वृ॒त्रम् ती॒र्त्वा ती॒र्त्वा वृ॒त्र मे॒वैव वृ॒त्रम् ती॒र्त्वा । \newline
48. वृ॒त्रम् ती॒र्त्वा ती॒र्त्वा वृ॒त्रं ॅवृ॒त्रम् ती॒र्त्वा प्र॑ति॒ष्ठाम् प्र॑ति॒ष्ठाम् ती॒र्त्वा वृ॒त्रं ॅवृ॒त्रम् ती॒र्त्वा प्र॑ति॒ष्ठाम् । \newline
49. ती॒र्त्वा प्र॑ति॒ष्ठाम् प्र॑ति॒ष्ठाम् ती॒र्त्वा ती॒र्त्वा प्र॑ति॒ष्ठाम् ग॑च्छति गच्छति प्रति॒ष्ठाम् ती॒र्त्वा ती॒र्त्वा प्र॑ति॒ष्ठाम् ग॑च्छति । \newline
50. प्र॒ति॒ष्ठाम् ग॑च्छति गच्छति प्रति॒ष्ठाम् प्र॑ति॒ष्ठाम् ग॑च्छ॒तीन्द्रा॒ये न्द्रा॑य गच्छति प्रति॒ष्ठाम् प्र॑ति॒ष्ठाम् ग॑च्छ॒तीन्द्रा॑य । \newline
51. प्र॒ति॒ष्ठामिति॑ प्रति - स्थाम् । \newline
52. ग॒च्छ॒तीन्द्रा॒ये न्द्रा॑य गच्छति गच्छ॒तीन्द्रा॑या भिमाति॒घ्ने॑ ऽभिमाति॒घ्न इन्द्रा॑य गच्छति गच्छ॒तीन्द्रा॑या भिमाति॒घ्ने । \newline
53. इन्द्रा॑या भिमाति॒घ्ने॑ ऽभिमाति॒घ्न इन्द्रा॒ये न्द्रा॑या भिमाति॒घ्ने ल॒लाम॑म् ॅल॒लाम॑ मभिमाति॒घ्न इन्द्रा॒ये न्द्रा॑या भिमाति॒घ्ने ल॒लाम᳚म् । \newline
54. अ॒भि॒मा॒ति॒घ्ने ल॒लाम॑म् ॅल॒लाम॑ मभिमाति॒घ्ने॑ ऽभिमाति॒घ्ने ल॒लाम॑म् प्राशृ॒ङ्गम् प्रा॑शृ॒ङ्गम् ॅल॒लाम॑ मभिमाति॒घ्ने॑ ऽभिमाति॒घ्ने ल॒लाम॑म् प्राशृ॒ङ्गम् । \newline
55. अ॒भि॒मा॒ति॒घ्न इत्य॑भिमाति - घ्ने । \newline
56. ल॒लाम॑म् प्राशृ॒ङ्गम् प्रा॑शृ॒ङ्गम् ॅल॒लाम॑म् ॅल॒लाम॑म् प्राशृ॒ङ्ग मा प्रा॑शृ॒ङ्गम् ॅल॒लाम॑म् ॅल॒लाम॑म् प्राशृ॒ङ्ग मा । \newline
57. प्रा॒शृ॒ङ्ग मा प्रा॑शृ॒ङ्गम् प्रा॑शृ॒ङ्ग मा ल॑भेत लभे॒ता प्रा॑शृ॒ङ्गम् प्रा॑शृ॒ङ्ग मा ल॑भेत । \newline
58. आ ल॑भेत लभे॒ता ल॑भेत॒ यो यो ल॑भे॒ता ल॑भेत॒ यः । \newline
\pagebreak
\markright{ TS 2.1.3.5  \hfill https://www.vedavms.in \hfill}

\section{ TS 2.1.3.5 }

\textbf{TS 2.1.3.5 } \newline
\textbf{Samhita Paata} \newline

ल॑भेत॒ यः पा॒प्मना॑ गृही॒तः स्यात् पा॒प्मा वा अ॒भिमा॑ति॒रिन्द्र॑मे॒वा- भि॑माति॒हनꣳ॒॒ स्वेन॑ भाग॒धेये॒नोप॑ धावति॒ स ए॒वास्मा᳚त् पा॒प्मान॑म॒भिमा॑तिं॒ प्रणु॑दत॒ इन्द्रा॑य व॒ज्रिणे॑ ल॒लामं॑ प्राशृ॒ङ्गमा ल॑भेत॒ यमलꣳ॑ रा॒ज्याय॒ सन्तꣳ॑ रा॒ज्यं नोप॒नमे॒दिन्द्र॑मे॒व व॒ज्रिणꣳ॒॒ स्वेन॑ भाग॒धेये॒नोप॑ धावति॒ स ए॒वास्मै॒ वज्रं॒ प्र य॑च्छति॒ स ए॑नं॒ ( ) ॅवज्रो॒ भूत्या॑ इन्ध॒ उपै॑नꣳ रा॒ज्यं न॑मति ल॒लामः॑ प्राशृ॒ङ्गो भ॑वत्ये॒तद्वै वज्र॑स्य रू॒पꣳ समृ॑द्ध्यै ॥ \newline

\textbf{Pada Paata} \newline

ल॒भे॒त॒ । यः । पा॒प्मना᳚ । गृ॒ही॒तः । स्यात् । पा॒प्मा । वै । अ॒भिमा॑ति॒रित्य॒भि - मा॒तिः॒ । इन्द्र᳚म् । ए॒व । अ॒भि॒मा॒ति॒हन॒मित्य॑भिमाति - हन᳚म् । स्वेन॑ । भा॒ग॒धेये॒नेति॑ भाग - धेये॑न । उपेति॑ । धा॒व॒ति॒ । सः । ए॒व । अ॒स्मा॒त् । पा॒प्मान᳚म् । अ॒भिमा॑ति॒मित्य॒भि - मा॒ति॒म् । प्रेति॑ । नु॒द॒ते॒ । इन्द्रा॑य । व॒ज्रिणे᳚ । ल॒लाम᳚म् । प्रा॒शृ॒ङ्गम् । एति॑ । ल॒भे॒त॒ । यम् । अल᳚म् । रा॒ज्याय॑ । सन्त᳚म् । रा॒ज्यम् । न । उ॒प॒नमे॒दित्यु॑प - नमे᳚त् । इन्द्र᳚म् । ए॒व । व॒ज्रिण᳚म् । स्वेन॑ । भा॒ग॒धेये॒नेति॑ भाग - धेये॑न । उपेति॑ । धा॒व॒ति॒ । सः । ए॒व । अ॒स्मै॒ । वज्र᳚म् । प्रेति॑ । य॒च्छ॒ति॒ । सः । ए॒न॒म् ( ) । वज्रः॑ । भूत्यै᳚ । इ॒न्धे॒ । उपेति॑ । ए॒न॒म् । रा॒ज्यम् । न॒म॒ति॒ । ल॒लामः॑ । प्रा॒शृ॒ङ्गः । भ॒व॒ति॒ । ए॒तत् । वै । वज्र॑स्य । रू॒पम् । समृ॑द्ध्या॒ इति॒ सं - ऋ॒द्ध्यै॒ ॥  \newline


\textbf{Krama Paata} \newline

ल॒भे॒त॒ यः । यः पा॒प्मना᳚ । पा॒प्मना॑ गृही॒तः । गृ॒ही॒तः स्यात् । स्यात्,पा॒प्मा । पा॒प्मा वै । वा अ॒भिमा॑तिः । अ॒भिमा॑ति॒रिन्द्र᳚म् । अ॒भिमा॑ति॒रित्य॒भि - मा॒तिः॒ । इन्द्र॑मे॒व । ए॒वाभि॑माति॒हन᳚म् । अ॒भि॒मा॒ति॒हनꣳ॒॒ स्वेन॑ । अ॒भि॒मा॒ति॒हन॒मित्य॑भिमाति - हन᳚म् । स्वेन॑ भाग॒धेये॑न । भा॒ग॒धेये॒नोप॑ । भा॒ग॒धेये॒नेति॑ भाग - धेये॑न । उप॑ धावति । धा॒व॒ति॒ सः । स ए॒व । ए॒वास्मा᳚त् । अ॒स्मा॒त् पा॒प्मान᳚म् । पा॒प्मान॑म॒भिमा॑तिम् । अ॒भिमा॑ति॒म् प्र । अ॒भिमा॑ति॒मित्य॒भि - मा॒ति॒म् । प्र णु॑दते । नु॒द॒त॒ इन्द्रा॑य । इन्द्रा॑य व॒ज्रिणे᳚ । व॒ज्रिणे॑ ल॒लाम᳚म् । ल॒लाम॑म् प्राशृ॒ङ्गम् । प्रा॒शृ॒ङ्गमा । आ ल॑भेत । ल॒भे॒त॒ यम् । यमल᳚म् । अलꣳ॑ रा॒ज्याय॑ । रा॒ज्याय॒ सन्त᳚म् । सन्तꣳ॑ रा॒ज्यम् । रा॒ज्यम् न । नोप॒नमे᳚त् । उ॒प॒नमे॒दिन्द्र᳚म् । उ॒प॒नमे॒दित्यु॑प - नमे᳚त् । इन्द्र॑मे॒व । ए॒व व॒ज्रिण᳚म् । व॒ज्रिणꣳ॒॒ स्वेन॑ । स्वेन॑ भाग॒धेये॑न । भा॒ग॒धेये॒नोप॑ । भा॒ग॒धेये॒नेति॑ भाग - धेये॑न । उप॑ धावति । धा॒व॒ति॒ सः । स ए॒व । ए॒वास्मै᳚ । अ॒स्मै॒ वज्र᳚म् । वज्र॒म् प्र । प्र य॑च्छति । य॒च्छ॒ति॒ सः । स ए॑नम् ( ) । ए॒नं॒ ॅवज्रः॑ । वज्रो॒ भूत्यै᳚ । भूत्या॑ इन्धे । इ॒न्ध॒ उप॑ । उपै॑नम् । ए॒नꣳ॒॒ रा॒ज्यम् । रा॒ज्यम् न॑मति । न॒म॒ति॒ ल॒लामः॑ । ल॒लामः॑ प्राशृ॒ङ्गः । प्रा॒शृ॒ङ्गो भ॑वति । भ॒व॒त्ये॒तत् । ए॒तद् वै । वै वज्र॑स्य । वज्र॑स्य रू॒पम् । रू॒पꣳ समृ॑द्ध्यै । समृ॑द्ध्या॒ इति॒ सं - ऋ॒द्ध्यै॒ । \newline

\textbf{Jatai Paata} \newline

1. ल॒भे॒त॒ यो यो ल॑भेत लभेत॒ यः । \newline
2. यः पा॒प्मना॑ पा॒प्मना॒ यो यः पा॒प्मना᳚ । \newline
3. पा॒प्मना॑ गृही॒तो गृ॑ही॒तः पा॒प्मना॑ पा॒प्मना॑ गृही॒तः । \newline
4. गृ॒ही॒तः स्याथ् स्याद् गृ॑ही॒तो गृ॑ही॒तः स्यात् । \newline
5. स्यात् पा॒प्मा पा॒प्मा स्याथ् स्यात् पा॒प्मा । \newline
6. पा॒प्मा वै वै पा॒प्मा पा॒प्मा वै । \newline
7. वा अ॒भिमा॑ति र॒भिमा॑ति॒र् वै वा अ॒भिमा॑तिः । \newline
8. अ॒भिमा॑ति॒ रिन्द्र॒ मिन्द्र॑ म॒भिमा॑ति र॒भिमा॑ति॒ रिन्द्र᳚म् । \newline
9. अ॒भिमा॑ति॒रित्य॒भि - मा॒तिः॒ । \newline
10. इन्द्र॑ मे॒वैवे न्द्र॒ मिन्द्र॑ मे॒व । \newline
11. ए॒वा भि॑माति॒हन॑ मभिमाति॒हन॑ मे॒वैवा भि॑माति॒हन᳚म् । \newline
12. अ॒भि॒मा॒ति॒हनꣳ॒॒ स्वेन॒ स्वेना॑ भिमाति॒हन॑ मभिमाति॒हनꣳ॒॒ स्वेन॑ । \newline
13. अ॒भि॒मा॒ति॒हन॒मित्य॑भिमाति - हन᳚म् । \newline
14. स्वेन॑ भाग॒धेये॑न भाग॒धेये॑न॒ स्वेन॒ स्वेन॑ भाग॒धेये॑न । \newline
15. भा॒ग॒धेये॒नोपोप॑ भाग॒धेये॑न भाग॒धेये॒नोप॑ । \newline
16. भा॒ग॒धेये॒नेति॑ भाग - धेये॑न । \newline
17. उप॑ धावति धाव॒ त्युपोप॑ धावति । \newline
18. धा॒व॒ति॒ स स धा॑वति धावति॒ सः । \newline
19. स ए॒वैव स स ए॒व । \newline
20. ए॒वास्मा॑ दस्मा दे॒वैवास्मा᳚त् । \newline
21. अ॒स्मा॒त् पा॒प्मान॑म् पा॒प्मान॑ मस्मा दस्मात् पा॒प्मान᳚म् । \newline
22. पा॒प्मान॑ म॒भिमा॑ति म॒भिमा॑तिम् पा॒प्मान॑म् पा॒प्मान॑ म॒भिमा॑तिम् । \newline
23. अ॒भिमा॑ति॒म् प्र प्राभिमा॑ति म॒भिमा॑ति॒म् प्र । \newline
24. अ॒भिमा॑ति॒मित्य॒भि - मा॒ति॒म् । \newline
25. प्र णु॑दते नुदते॒ प्र प्र णु॑दते । \newline
26. नु॒द॒त॒ इन्द्रा॒ये न्द्रा॑य नुदते नुदत॒ इन्द्रा॑य । \newline
27. इन्द्रा॑य व॒ज्रिणे॑ व॒ज्रिण॒ इन्द्रा॒ये न्द्रा॑य व॒ज्रिणे᳚ । \newline
28. व॒ज्रिणे॑ ल॒लाम॑म् ॅल॒लामं॑ ॅव॒ज्रिणे॑ व॒ज्रिणे॑ ल॒लाम᳚म् । \newline
29. ल॒लाम॑म् प्राशृ॒ङ्गम् प्रा॑शृ॒ङ्गम् ॅल॒लाम॑म् ॅल॒लाम॑म् प्राशृ॒ङ्गम् । \newline
30. प्रा॒शृ॒ङ्ग मा प्रा॑शृ॒ङ्गम् प्रा॑शृ॒ङ्ग मा । \newline
31. आ ल॑भेत लभे॒ता ल॑भेत । \newline
32. ल॒भे॒त॒ यं ॅयम् ॅल॑भेत लभेत॒ यम् । \newline
33. य मल॒ मलं॒ ॅयं ॅय मल᳚म् । \newline
34. अलꣳ॑ रा॒ज्याय॑ रा॒ज्याया ल॒ मलꣳ॑ रा॒ज्याय॑ । \newline
35. रा॒ज्याय॒ सन्तꣳ॒॒ सन्तꣳ॑ रा॒ज्याय॑ रा॒ज्याय॒ सन्त᳚म् । \newline
36. सन्तꣳ॑ रा॒ज्यꣳ रा॒ज्यꣳ सन्तꣳ॒॒ सन्तꣳ॑ रा॒ज्यम् । \newline
37. रा॒ज्यम् न न रा॒ज्यꣳ रा॒ज्यम् न । \newline
38. नोप॒नमे॑ दुप॒नमे॒न् न नोप॒नमे᳚त् । \newline
39. उ॒प॒नमे॒ दिन्द्र॒ मिन्द्र॑ मुप॒नमे॑ दुप॒नमे॒ दिन्द्र᳚म् । \newline
40. उ॒प॒नमे॒दित्यु॑प - नमे᳚त् । \newline
41. इन्द्र॑ मे॒वैवे न्द्र॒ मिन्द्र॑ मे॒व । \newline
42. ए॒व व॒ज्रिणं॑ ॅव॒ज्रिण॑ मे॒वैव व॒ज्रिण᳚म् । \newline
43. व॒ज्रिणꣳ॒॒ स्वेन॒ स्वेन॑ व॒ज्रिणं॑ ॅव॒ज्रिणꣳ॒॒ स्वेन॑ । \newline
44. स्वेन॑ भाग॒धेये॑न भाग॒धेये॑न॒ स्वेन॒ स्वेन॑ भाग॒धेये॑न । \newline
45. भा॒ग॒धेये॒नोपोप॑ भाग॒धेये॑न भाग॒धेये॒नोप॑ । \newline
46. भा॒ग॒धेये॒नेति॑ भाग - धेये॑न । \newline
47. उप॑ धावति धाव॒ त्युपोप॑ धावति । \newline
48. धा॒व॒ति॒ स स धा॑वति धावति॒ सः । \newline
49. स ए॒वैव स स ए॒व । \newline
50. ए॒वास्मा॑ अस्मा ए॒वैवास्मै᳚ । \newline
51. अ॒स्मै॒ वज्रं॒ ॅवज्र॑ मस्मा अस्मै॒ वज्र᳚म् । \newline
52. वज्र॒म् प्र प्र वज्रं॒ ॅवज्र॒म् प्र । \newline
53. प्र य॑च्छति यच्छति॒ प्र प्र य॑च्छति । \newline
54. य॒च्छ॒ति॒ स स य॑च्छति यच्छति॒ सः । \newline
55. स ए॑न मेनꣳ॒॒ स स ए॑नम् । \newline
56. ए॒नं॒ ॅवज्रो॒ वज्र॑ एन मेनं॒ ॅवज्रः॑ । \newline
57. वज्रो॒ भूत्यै॒ भूत्यै॒ वज्रो॒ वज्रो॒ भूत्यै᳚ । \newline
58. भूत्या॑ इन्ध इन्धे॒ भूत्यै॒ भूत्या॑ इन्धे । \newline
59. इ॒न्ध॒ उपोपे᳚ न्ध इन्ध॒ उप॑ । \newline
60. उपै॑न मेन॒ मुपोपै॑नम् । \newline
61. ए॒नꣳ॒॒ रा॒ज्यꣳ रा॒ज्य मे॑न मेनꣳ रा॒ज्यम् । \newline
62. रा॒ज्यम् न॑मति नमति रा॒ज्यꣳ रा॒ज्यम् न॑मति । \newline
63. न॒म॒ति॒ ल॒लामो॑ ल॒लामो॑ नमति नमति ल॒लामः॑ । \newline
64. ल॒लामः॑ प्राशृ॒ङ्गः प्रा॑शृ॒ङ्गो ल॒लामो॑ ल॒लामः॑ प्राशृ॒ङ्गः । \newline
65. प्रा॒शृ॒ङ्गो भ॑वति भवति प्राशृ॒ङ्गः प्रा॑शृ॒ङ्गो भ॑वति । \newline
66. भ॒व॒ त्ये॒त दे॒तद् भ॑वति भव त्ये॒तत् । \newline
67. ए॒तद् वै वा ए॒त दे॒तद् वै । \newline
68. वै वज्र॑स्य॒ वज्र॑स्य॒ वै वै वज्र॑स्य । \newline
69. वज्र॑स्य रू॒पꣳ रू॒पं ॅवज्र॑स्य॒ वज्र॑स्य रू॒पम् । \newline
70. रू॒पꣳ समृ॑द्ध्यै॒ समृ॑द्ध्यै रू॒पꣳ रू॒पꣳ समृ॑द्ध्यै । \newline
71. समृ॑द्ध्या॒ इति॒ सं - ऋ॒द्ध्यै॒ । \newline

\textbf{Ghana Paata } \newline

1. ल॒भे॒त॒ यो यो ल॑भेत लभेत॒ यः पा॒प्मना॑ पा॒प्मना॒ यो ल॑भेत लभेत॒ यः पा॒प्मना᳚ । \newline
2. यः पा॒प्मना॑ पा॒प्मना॒ यो यः पा॒प्मना॑ गृही॒तो गृ॑ही॒तः पा॒प्मना॒ यो यः पा॒प्मना॑ गृही॒तः । \newline
3. पा॒प्मना॑ गृही॒तो गृ॑ही॒तः पा॒प्मना॑ पा॒प्मना॑ गृही॒तः स्याथ् स्याद् गृ॑ही॒तः पा॒प्मना॑ पा॒प्मना॑ गृही॒तः स्यात् । \newline
4. गृ॒ही॒तः स्याथ् स्याद् गृ॑ही॒तो गृ॑ही॒तः स्यात् पा॒प्मा पा॒प्मा स्याद् गृ॑ही॒तो गृ॑ही॒तः स्यात् पा॒प्मा । \newline
5. स्यात् पा॒प्मा पा॒प्मा स्याथ् स्यात् पा॒प्मा वै वै पा॒प्मा स्याथ् स्यात् पा॒प्मा वै । \newline
6. पा॒प्मा वै वै पा॒प्मा पा॒प्मा वा अ॒भिमा॑ति र॒भिमा॑ति॒र् वै पा॒प्मा पा॒प्मा वा अ॒भिमा॑तिः । \newline
7. वा अ॒भिमा॑ति र॒भिमा॑ति॒र् वै वा अ॒भिमा॑ति॒ रिन्द्र॒ मिन्द्र॑ म॒भिमा॑ति॒र् वै वा अ॒भिमा॑ति॒ रिन्द्र᳚म् । \newline
8. अ॒भिमा॑ति॒ रिन्द्र॒ मिन्द्र॑ म॒भिमा॑ति र॒भिमा॑ति॒ रिन्द्र॑ मे॒वैवे न्द्र॑ म॒भिमा॑ति र॒भिमा॑ति॒ रिन्द्र॑ मे॒व । \newline
9. अ॒भिमा॑ति॒रित्य॒भि - मा॒तिः॒ । \newline
10. इन्द्र॑ मे॒वैवे न्द्र॒ मिन्द्र॑ मे॒वा भि॑माति॒हन॑ मभिमाति॒हन॑ मे॒वे न्द्र॒ मिन्द्र॑ मे॒वा भि॑माति॒हन᳚म् । \newline
11. ए॒वा भि॑माति॒हन॑ मभिमाति॒हन॑ मे॒वैवा भि॑माति॒हनꣳ॒॒ स्वेन॒ स्वेना॑भिमाति॒हन॑ मे॒वैवा भि॑माति॒हनꣳ॒॒ स्वेन॑ । \newline
12. अ॒भि॒मा॒ति॒हनꣳ॒॒ स्वेन॒ स्वेना॑ भिमाति॒हन॑ मभिमाति॒हनꣳ॒॒ स्वेन॑ भाग॒धेये॑न भाग॒धेये॑न॒ स्वेना॑ भिमाति॒हन॑ मभिमाति॒हनꣳ॒॒ स्वेन॑ भाग॒धेये॑न । \newline
13. अ॒भि॒मा॒ति॒हन॒मित्य॑भिमाति - हन᳚म् । \newline
14. स्वेन॑ भाग॒धेये॑न भाग॒धेये॑न॒ स्वेन॒ स्वेन॑ भाग॒धेये॒नो पोप॑ भाग॒धेये॑न॒ स्वेन॒ स्वेन॑ भाग॒धेये॒नोप॑ । \newline
15. भा॒ग॒धेये॒नो पोप॑ भाग॒धेये॑न भाग॒धेये॒नोप॑ धावति धाव॒त्युप॑ भाग॒धेये॑न भाग॒धेये॒नोप॑ धावति । \newline
16. भा॒ग॒धेये॒नेति॑ भाग - धेये॑न । \newline
17. उप॑ धावति धाव॒त् युपोप॑ धावति॒ स स धा॑व॒ त्युपोप॑ धावति॒ सः । \newline
18. धा॒व॒ति॒ स स धा॑वति धावति॒ स ए॒वैव स धा॑वति धावति॒ स ए॒व । \newline
19. स ए॒वैव स स ए॒वास्मा॑ दस्मादे॒व स स ए॒वास्मा᳚त् । \newline
20. ए॒वास्मा॑ दस्मा दे॒वैवा स्मा᳚त् पा॒प्मान॑म् पा॒प्मान॑ मस्मा दे॒वैवा स्मा᳚त् पा॒प्मान᳚म् । \newline
21. अ॒स्मा॒त् पा॒प्मान॑म् पा॒प्मान॑ मस्मा दस्मात् पा॒प्मान॑ म॒भिमा॑ति म॒भिमा॑तिम् पा॒प्मान॑ मस्मा दस्मात् पा॒प्मान॑ म॒भिमा॑तिम् । \newline
22. पा॒प्मान॑ म॒भिमा॑ति म॒भिमा॑तिम् पा॒प्मान॑म् पा॒प्मान॑ म॒भिमा॑ति॒म् प्र प्राभिमा॑तिम् पा॒प्मान॑म् पा॒प्मान॑ म॒भिमा॑ति॒म् प्र । \newline
23. अ॒भिमा॑ति॒म् प्र प्राभिमा॑ति म॒भिमा॑ति॒म् प्र णु॑दते नुदते॒ प्राभिमा॑ति म॒भिमा॑ति॒म् प्र णु॑दते । \newline
24. अ॒भिमा॑ति॒मित्य॒भि - मा॒ति॒म् । \newline
25. प्र णु॑दते नुदते॒ प्र प्र णु॑दत॒ इन्द्रा॒ये न्द्रा॑य नुदते॒ प्र प्र णु॑दत॒ इन्द्रा॑य । \newline
26. नु॒द॒त॒ इन्द्रा॒ये न्द्रा॑य नुदते नुदत॒ इन्द्रा॑य व॒ज्रिणे॑ व॒ज्रिण॒ इन्द्रा॑य नुदते नुदत॒ इन्द्रा॑य व॒ज्रिणे᳚ । \newline
27. इन्द्रा॑य व॒ज्रिणे॑ व॒ज्रिण॒ इन्द्रा॒ये न्द्रा॑य व॒ज्रिणे॑ ल॒लाम॑म् ॅल॒लामं॑ ॅव॒ज्रिण॒ इन्द्रा॒ये न्द्रा॑य व॒ज्रिणे॑ ल॒लाम᳚म् । \newline
28. व॒ज्रिणे॑ ल॒लाम॑म् ॅल॒लामं॑ ॅव॒ज्रिणे॑ व॒ज्रिणे॑ ल॒लाम॑म् प्राशृ॒ङ्गम् प्रा॑शृ॒ङ्गम् ॅल॒लामं॑ ॅव॒ज्रिणे॑ व॒ज्रिणे॑ ल॒लाम॑म् प्राशृ॒ङ्गम् । \newline
29. ल॒लाम॑म् प्राशृ॒ङ्गम् प्रा॑शृ॒ङ्गम् ॅल॒लाम॑म् ॅल॒लाम॑म् प्राशृ॒ङ्ग मा प्रा॑शृ॒ङ्गम् ॅल॒लाम॑म् ॅल॒लाम॑म् प्राशृ॒ङ्ग मा । \newline
30. प्रा॒शृ॒ङ्ग मा प्रा॑शृ॒ङ्गम् प्रा॑शृ॒ङ्ग मा ल॑भेत लभे॒ता प्रा॑शृ॒ङ्गम् प्रा॑शृ॒ङ्ग मा ल॑भेत । \newline
31. आ ल॑भेत लभे॒ता ल॑भेत॒ यं ॅयम् ॅल॑भे॒ता ल॑भेत॒ यम् । \newline
32. ल॒भे॒त॒ यं ॅयम् ॅल॑भेत लभेत॒ य मल॒ मलं॒ ॅयम् ॅल॑भेत लभेत॒ य मल᳚म् । \newline
33. य मल॒ मलं॒ ॅयं ॅय मलꣳ॑ रा॒ज्याय॑ रा॒ज्यायालं॒ ॅयं ॅय मलꣳ॑ रा॒ज्याय॑ । \newline
34. अलꣳ॑ रा॒ज्याय॑ रा॒ज्यायाल॒ मलꣳ॑ रा॒ज्याय॒ सन्तꣳ॒॒ सन्तꣳ॑ रा॒ज्यायाल॒ मलꣳ॑ रा॒ज्याय॒ सन्त᳚म् । \newline
35. रा॒ज्याय॒ सन्तꣳ॒॒ सन्तꣳ॑ रा॒ज्याय॑ रा॒ज्याय॒ सन्तꣳ॑ रा॒ज्यꣳ रा॒ज्यꣳ सन्तꣳ॑ रा॒ज्याय॑ रा॒ज्याय॒ सन्तꣳ॑ रा॒ज्यम् । \newline
36. सन्तꣳ॑ रा॒ज्यꣳ रा॒ज्यꣳ सन्तꣳ॒॒ सन्तꣳ॑ रा॒ज्यन्न न रा॒ज्यꣳ सन्तꣳ॒॒ सन्तꣳ॑ रा॒ज्यन्न । \newline
37. रा॒ज्यन्न न रा॒ज्यꣳ रा॒ज्यन् नोप॒नमे॑ दुप॒नमे॒न् न रा॒ज्यꣳ रा॒ज्यन् नोप॒नमे᳚त् । \newline
38. नोप॒नमे॑ दुप॒नमे॒न् न नोप॒नमे॒ दिन्द्र॒ मिन्द्र॑ मुप॒नमे॒न् न नोप॒नमे॒ दिन्द्र᳚म् । \newline
39. उ॒प॒नमे॒ दिन्द्र॒ मिन्द्र॑ मुप॒नमे॑ दुप॒नमे॒ दिन्द्र॑ मे॒वैवे न्द्र॑ मुप॒नमे॑ दुप॒नमे॒ दिन्द्र॑ मे॒व । \newline
40. उ॒प॒नमे॒दित्यु॑प - नमे᳚त् । \newline
41. इन्द्र॑ मे॒वैवे न्द्र॒ मिन्द्र॑ मे॒व व॒ज्रिणं॑ ॅव॒ज्रिण॑ मे॒वे न्द्र॒ मिन्द्र॑ मे॒व व॒ज्रिण᳚म् । \newline
42. ए॒व व॒ज्रिणं॑ ॅव॒ज्रिण॑ मे॒वैव व॒ज्रिणꣳ॒॒ स्वेन॒ स्वेन॑ व॒ज्रिण॑ मे॒वैव व॒ज्रिणꣳ॒॒ स्वेन॑ । \newline
43. व॒ज्रिणꣳ॒॒ स्वेन॒ स्वेन॑ व॒ज्रिणं॑ ॅव॒ज्रिणꣳ॒॒ स्वेन॑ भाग॒धेये॑न भाग॒धेये॑न॒ स्वेन॑ व॒ज्रिणं॑ ॅव॒ज्रिणꣳ॒॒ स्वेन॑ भाग॒धेये॑न । \newline
44. स्वेन॑ भाग॒धेये॑न भाग॒धेये॑न॒ स्वेन॒ स्वेन॑ भाग॒धेये॒नो पोप॑ भाग॒धेये॑न॒ स्वेन॒ स्वेन॑ भाग॒धेये॒नोप॑ । \newline
45. भा॒ग॒धेये॒नो पोप॑ भाग॒धेये॑न भाग॒धेये॒नोप॑ धावति धाव॒त्युप॑ भाग॒धेये॑न भाग॒धेये॒नोप॑ धावति । \newline
46. भा॒ग॒धेये॒नेति॑ भाग - धेये॑न । \newline
47. उप॑ धावति धाव॒ त्युपोप॑ धावति॒ स स धा॑व॒ त्युपोप॑ धावति॒ सः । \newline
48. धा॒व॒ति॒ स स धा॑वति धावति॒ स ए॒वैव स धा॑वति धावति॒ स ए॒व । \newline
49. स ए॒वैव स स ए॒वास्मा॑ अस्मा ए॒व स स ए॒वास्मै᳚ । \newline
50. ए॒वास्मा॑ अस्मा ए॒वैवास्मै॒ वज्रं॒ ॅवज्र॑ मस्मा ए॒वैवास्मै॒ वज्र᳚म् । \newline
51. अ॒स्मै॒ वज्रं॒ ॅवज्र॑ मस्मा अस्मै॒ वज्र॒म् प्र प्र वज्र॑ मस्मा अस्मै॒ वज्र॒म् प्र । \newline
52. वज्र॒म् प्र प्र वज्रं॒ ॅवज्र॒म् प्र य॑च्छति यच्छति॒ प्र वज्रं॒ ॅवज्र॒म् प्र य॑च्छति । \newline
53. प्र य॑च्छति यच्छति॒ प्र प्र य॑च्छति॒ स स य॑च्छति॒ प्र प्र य॑च्छति॒ सः । \newline
54. य॒च्छ॒ति॒ स स य॑च्छति यच्छति॒ स ए॑न मेनꣳ॒॒ स य॑च्छति यच्छति॒ स ए॑नम् । \newline
55. स ए॑न मेनꣳ॒॒ स स ए॑नं॒ ॅवज्रो॒ वज्र॑ एनꣳ॒॒ स स ए॑नं॒ ॅवज्रः॑ । \newline
56. ए॒नं॒ ॅवज्रो॒ वज्र॑ एन मेनं॒ ॅवज्रो॒ भूत्यै॒ भूत्यै॒ वज्र॑ एन मेनं॒ ॅवज्रो॒ भूत्यै᳚ । \newline
57. वज्रो॒ भूत्यै॒ भूत्यै॒ वज्रो॒ वज्रो॒ भूत्या॑ इन्ध इन्धे॒ भूत्यै॒ वज्रो॒ वज्रो॒ भूत्या॑ इन्धे । \newline
58. भूत्या॑ इन्ध इन्धे॒ भूत्यै॒ भूत्या॑ इन्ध॒ उपोपे᳚ न्धे॒ भूत्यै॒ भूत्या॑ इन्ध॒ उप॑ । \newline
59. इ॒न्ध॒ उपोपे᳚ न्ध इन्ध॒ उपै॑न मेन॒ मुपे᳚ न्ध इन्ध॒ उपै॑नम् । \newline
60. उपै॑न मेन॒ मुपोपै॑नꣳ रा॒ज्यꣳ रा॒ज्य मे॑न॒ मुपोपै॑नꣳ रा॒ज्यम् । \newline
61. ए॒नꣳ॒॒ रा॒ज्यꣳ रा॒ज्य मे॑न मेनꣳ रा॒ज्यन् न॑मति नमति रा॒ज्य मे॑न मेनꣳ रा॒ज्यन् न॑मति । \newline
62. रा॒ज्यन् न॑मति नमति रा॒ज्यꣳ रा॒ज्यन् न॑मति ल॒लामो॑ ल॒लामो॑ नमति रा॒ज्यꣳ रा॒ज्यन् न॑मति ल॒लामः॑ । \newline
63. न॒म॒ति॒ ल॒लामो॑ ल॒लामो॑ नमति नमति ल॒लामः॑ प्राशृ॒ङ्गः प्रा॑शृ॒ङ्गो ल॒लामो॑ नमति नमति ल॒लामः॑ प्राशृ॒ङ्गः । \newline
64. ल॒लामः॑ प्राशृ॒ङ्गः प्रा॑शृ॒ङ्गो ल॒लामो॑ ल॒लामः॑ प्राशृ॒ङ्गो भ॑वति भवति प्राशृ॒ङ्गो ल॒लामो॑ ल॒लामः॑ प्राशृ॒ङ्गो भ॑वति । \newline
65. प्रा॒शृ॒ङ्गो भ॑वति भवति प्राशृ॒ङ्गः प्रा॑शृ॒ङ्गो भ॑वत्ये॒त दे॒तद् भ॑वति प्राशृ॒ङ्गः प्रा॑शृ॒ङ्गो भ॑वत्ये॒तत् । \newline
66. भ॒व॒त्ये॒त दे॒तद् भ॑वति भवत्ये॒तद् वै वा ए॒तद् भ॑वति भवत्ये॒तद् वै । \newline
67. ए॒तद् वै वा ए॒तदे॒तद् वै वज्र॑स्य॒ वज्र॑स्य॒ वा ए॒तदे॒तद् वै वज्र॑स्य । \newline
68. वै वज्र॑स्य॒ वज्र॑स्य॒ वै वै वज्र॑स्य रू॒पꣳ रू॒पं ॅवज्र॑स्य॒ वै वै वज्र॑स्य रू॒पम् । \newline
69. वज्र॑स्य रू॒पꣳ रू॒पं ॅवज्र॑स्य॒ वज्र॑स्य रू॒पꣳ समृ॑द्ध्यै॒ समृ॑द्ध्यै रू॒पं ॅवज्र॑स्य॒ वज्र॑स्य रू॒पꣳ समृ॑द्ध्यै । \newline
70. रू॒पꣳ समृ॑द्ध्यै॒ समृ॑द्ध्यै रू॒पꣳ रू॒पꣳ समृ॑द्ध्यै । \newline
71. समृ॑द्ध्या॒ इति॒ सं - ऋ॒द्ध्यै॒ । \newline
\pagebreak
\markright{ TS 2.1.4.1  \hfill https://www.vedavms.in \hfill}

\section{ TS 2.1.4.1 }

\textbf{TS 2.1.4.1 } \newline
\textbf{Samhita Paata} \newline

अ॒सावा॑दि॒त्यो न व्य॑रोचत॒ तस्मै॑ दे॒वाः प्राय॑श्चित्तिमैच्छ॒न् तस्मा॑ ए॒तां दश॑र्.षभा॒माऽल॑भन्त॒ तयै॒वास्मि॒न् रुच॑मदधु॒र्यो ब्र॑ह्मवर्च॒सका॑मः॒ स्यात् तस्मा॑ ए॒तां दश॑र्.षभा॒मा ल॑भेता॒-मुमे॒वाऽऽ*दि॒त्यꣳ स्वेन॑ भाग॒धेये॒नोप॑ धावति॒ स ए॒वास्मि॑न् ब्रह्मवर्च॒सं द॑धाति ब्रह्मवर्च॒स्ये॑व भ॑वति व॒सन्ता᳚ प्रा॒तस्त्रीन् ॅल॒लामा॒ना ल॑भेत ग्री॒ष्मे म॒द्ध्यन्दि॑ने॒ - [  ] \newline

\textbf{Pada Paata} \newline

अ॒सौ । आ॒दि॒त्यः । न । वीति॑ । अ॒रो॒च॒त॒ । तस्मै᳚ । दे॒वाः । प्राय॑श्चित्तिम् । ऐ॒च्छ॒न्न् । तस्मै᳚ । ए॒ताम् । दश॑र्.षभा॒मिति॒ दश॑ - ऋ॒ष॒भा॒म् । एति॑ । अ॒ल॒भ॒न्त॒ । तया᳚ । ए॒व । अ॒स्मि॒न्न् । रुच᳚म् । अ॒द॒धुः॒ । यः । ब्र॒ह्म॒व॒र्च॒सका॑म॒ इति॑ ब्रह्मवर्च॒स - का॒मः॒ । स्यात् । तस्मै᳚ । ए॒ताम् । दश॑र्.षभा॒मिति॒ दश॑ - ऋ॒ष॒भा॒म् । एति॑ । ल॒भे॒त॒ । अ॒मुम् । ए॒व । आ॒दि॒त्यम् । स्वेन॑ । भा॒ग॒धेये॒नेति॑ भाग - धेये॑न । उपेति॑ । धा॒व॒ति॒ । सः । ए॒व । अ॒स्मि॒न्न् । ब्र॒ह्म॒व॒र्च॒समिति॑ ब्रह्म - व॒र्च॒सम् । द॒धा॒ति॒ । ब्र॒ह्म॒व॒र्च॒सीति॑ ब्रह्म - व॒र्च॒सी । ए॒व । भ॒व॒ति॒ । व॒सन्ता᳚ । प्रा॒तः । त्रीन् । ल॒लामान्॑ । एति॑ । ल॒भे॒त॒ । ग्री॒ष्मे । म॒द्ध्यन्दि॑ने ।  \newline


\textbf{Krama Paata} \newline

अ॒सावा॑दि॒त्यः । आ॒दि॒त्यो न । न वि । व्य॑रोचत । अ॒रो॒च॒त॒ तस्मै᳚ । तस्मै॑ दे॒वाः । दे॒वाः प्राय॑श्चित्तिम् । प्राय॑श्चित्तिमैच्छ॒न्न् । ऐ॒च्छ॒न्,तस्मै᳚ । तस्मा॑ ए॒ताम् । ए॒ताम् दश॑र्.षभाम् । दश॑र्.षभा॒मा । दश॑र्.षभा॒मिति॒ दश॑ - ऋ॒ष॒भा॒॒म् । आऽल॑भन्त । अ॒ल॒भ॒न्त॒ तया᳚ । तयै॒व । ए॒वास्मिन्न्॑ । अ॒स्मि॒न् रुच᳚म् । रुच॑मदधुः । अ॒द॒धु॒र् यः । यो ब्र॑ह्मवर्च॒सका॑मः । ब्र॒ह्म॒व॒र्च॒सका॑मः॒ स्यात् । ब्र॒ह्म॒व॒र्च॒सका॑म॒ इति॑ ब्रह्मवर्च॒स - का॒मः॒ । स्यात् तस्मै᳚ । तस्मा॑ ए॒ताम् । ए॒ताम् दश॑र्.षभाम् । दश॑र्.षभा॒मा । दश॑र्.षभा॒मिति॒ दश॑ - ऋ॒ष॒भा॒॒म् । आ ल॑भेत । ल॒भे॒ता॒मुम् । अ॒मुमे॒व । ए॒वादि॒त्यम् । आ॒दि॒त्यꣳ स्वेन॑ । स्वेन॑ भाग॒धेये॑न । भा॒ग॒धेये॒नोप॑ । भा॒ग॒धेये॒नेति॑ भाग - धेये॑न । उप॑ धावति । धा॒व॒ति॒ सः । स ए॒व । ए॒वास्मिन्न्॑ । अ॒स्मि॒न् ब्र॒ह्म॒व॒र्च॒सम् । ब्र॒ह्म॒व॒र्च॒सम् द॑धाति । ब्र॒ह्म॒व॒र्च॒समिति॑ ब्रह्म - व॒र्च॒सम् । द॒धा॒ति॒ ब्र॒ह्म॒व॒र्च॒सी । ब्र॒ह्म॒व॒र्च॒स्ये॑व । ब्र॒ह्म॒व॒र्च॒सीति॑ ब्रह्म - व॒र्च॒सी । ए॒व भ॑वति । भ॒व॒ति॒ व॒सन्ता᳚ । व॒सन्ता᳚ प्रा॒तः । प्रा॒तस्त्रीन् । त्री॒न् ॅल॒लामान्॑ । ल॒लामा॒ना । आ ल॑भेत । ल॒भे॒त॒ ग्री॒ष्मे । ग्री॒ष्मे म॒द्ध्यन्दि॑ने । म॒द्ध्यन्दि॑ने॒ त्रीन् \newline

\textbf{Jatai Paata} \newline

1. अ॒सा वा॑दि॒त्य आ॑दि॒त्यो॑ ऽसा व॒सा वा॑दि॒त्यः । \newline
2. आ॒दि॒त्यो न नादि॒त्य आ॑दि॒त्यो न । \newline
3. न वि वि न न वि । \newline
4. व्य॑रोचता रोचत॒ वि व्य॑रोचत । \newline
5. अ॒रो॒च॒त॒ तस्मै॒ तस्मा॑ अरोचता रोचत॒ तस्मै᳚ । \newline
6. तस्मै॑ दे॒वा दे॒वा स्तस्मै॒ तस्मै॑ दे॒वाः । \newline
7. दे॒वाः प्राय॑श्चित्ति॒म् प्राय॑श्चित्तिम् दे॒वा दे॒वाः प्राय॑श्चित्तिम् । \newline
8. प्राय॑श्चित्ति मैच्छन् नैच्छ॒न् प्राय॑श्चित्ति॒म् प्राय॑श्चित्ति मैच्छन्न् । \newline
9. ऐ॒च्छ॒न् तस्मै॒ तस्मा॑ ऐच्छन् नैच्छ॒न् तस्मै᳚ । \newline
10. तस्मा॑ ए॒ता मे॒ताम् तस्मै॒ तस्मा॑ ए॒ताम् । \newline
11. ए॒ताम् दश॑र्.षभा॒म् दश॑र्.षभा मे॒ता मे॒ताम् दश॑र्.षभाम् । \newline
12. दश॑र्.षभा॒ मा दश॑र्.षभा॒म् दश॑र्.षभा॒ मा । \newline
13. दश॑र्.षभा॒मिति॒ दश॑ - ऋ॒ष॒भा॒म् । \newline
14. आ ऽल॑भन्ता लभ॒न्ता ऽल॑भन्त । \newline
15. अ॒ल॒भ॒न्त॒ तया॒ तया॑ ऽलभन्ता लभन्त॒ तया᳚ । \newline
16. तयै॒वैव तया॒ तयै॒व । \newline
17. ए॒वास्मि॑न् नस्मिन् ने॒वैवास्मिन्न्॑ । \newline
18. अ॒स्मि॒न् रुचꣳ॒॒ रुच॑ मस्मिन् नस्मि॒न् रुच᳚म् । \newline
19. रुच॑ मदधु रदधू॒ रुचꣳ॒॒ रुच॑ मदधुः । \newline
20. अ॒द॒धु॒र् यो यो॑ ऽदधु रदधु॒र् यः । \newline
21. यो ब्र॑ह्मवर्च॒सका॑मो ब्रह्मवर्च॒सका॑मो॒ यो यो ब्र॑ह्मवर्च॒सका॑मः । \newline
22. ब्र॒ह्म॒व॒र्च॒सका॑मः॒ स्याथ् स्याद् ब्र॑ह्मवर्च॒सका॑मो ब्रह्मवर्च॒सका॑मः॒ स्यात् । \newline
23. ब्र॒ह्म॒व॒र्च॒सका॑म॒ इति॑ ब्रह्मवर्च॒स - का॒मः॒ । \newline
24. स्यात् तस्मै॒ तस्मै॒ स्याथ् स्यात् तस्मै᳚ । \newline
25. तस्मा॑ ए॒ता मे॒ताम् तस्मै॒ तस्मा॑ ए॒ताम् । \newline
26. ए॒ताम् दश॑र्.षभा॒म् दश॑र्.षभा मे॒ता मे॒ताम् दश॑र्.षभाम् । \newline
27. दश॑र्.षभा॒ मा दश॑र्.षभा॒म् दश॑र्.षभा॒ मा । \newline
28. दश॑र्.षभा॒मिति॒ दश॑ - ऋ॒ष॒भा॒म् । \newline
29. आ ल॑भेत लभे॒ता ल॑भेत । \newline
30. ल॒भे॒ता॒मु म॒मुम् ॅल॑भेत लभेता॒मुम् । \newline
31. अ॒मु मे॒वैवामु म॒मु मे॒व । \newline
32. ए॒वा दि॒त्य मा॑दि॒त्य मे॒वैवा दि॒त्यम् । \newline
33. आ॒दि॒त्यꣳ स्वेन॒ स्वेना॑दि॒त्य मा॑दि॒त्यꣳ स्वेन॑ । \newline
34. स्वेन॑ भाग॒धेये॑न भाग॒धेये॑न॒ स्वेन॒ स्वेन॑ भाग॒धेये॑न । \newline
35. भा॒ग॒धेये॒नोपोप॑ भाग॒धेये॑न भाग॒धेये॒नोप॑ । \newline
36. भा॒ग॒धेये॒नेति॑ भाग - धेये॑न । \newline
37. उप॑ धावति धाव॒ त्युपोप॑ धावति । \newline
38. धा॒व॒ति॒ स स धा॑वति धावति॒ सः । \newline
39. स ए॒वैव स स ए॒व । \newline
40. ए॒वास्मि॑न् नस्मिन् ने॒वैवास्मिन्न्॑ । \newline
41. अ॒स्मि॒न् ब्र॒ह्म॒व॒र्च॒सम् ब्र॑ह्मवर्च॒स म॑स्मिन् नस्मिन् ब्रह्मवर्च॒सम् । \newline
42. ब्र॒ह्म॒व॒र्च॒सम् द॑धाति दधाति ब्रह्मवर्च॒सम् ब्र॑ह्मवर्च॒सम् द॑धाति । \newline
43. ब्र॒ह्म॒व॒र्च॒समिति॑ ब्रह्म - व॒र्च॒सम् । \newline
44. द॒धा॒ति॒ ब्र॒ह्म॒व॒र्च॒सी ब्र॑ह्मवर्च॒सी द॑धाति दधाति ब्रह्मवर्च॒सी । \newline
45. ब्र॒ह्म॒व॒र्च॒ स्ये॑वैव ब्र॑ह्मवर्च॒सी ब्र॑ह्मवर्च॒ स्ये॑व । \newline
46. ब्र॒ह्म॒व॒र्च॒सीति॑ ब्रह्म - व॒र्च॒सी । \newline
47. ए॒व भ॑वति भव त्ये॒वैव भ॑वति । \newline
48. भ॒व॒ति॒ व॒सन्ता॑ व॒सन्ता॑ भवति भवति व॒सन्ता᳚ । \newline
49. व॒सन्ता᳚ प्रा॒तः प्रा॒तर् व॒सन्ता॑ व॒सन्ता᳚ प्रा॒तः । \newline
50. प्रा॒त स्त्रीꣳ स्त्रीन् प्रा॒तः प्रा॒त स्त्रीन् । \newline
51. त्रीन् ॅल॒लामा᳚न् ॅल॒लामा॒न् त्रीꣳ स्त्रीन् ॅल॒लामान्॑ । \newline
52. ल॒लामा॒ ना ल॒लामा᳚न् ॅल॒लामा॒ ना । \newline
53. आ ल॑भेत लभे॒ता ल॑भेत । \newline
54. ल॒भे॒त॒ ग्री॒ष्मे ग्री॒ष्मे ल॑भेत लभेत ग्री॒ष्मे । \newline
55. ग्री॒ष्मे म॒द्ध्यन्दि॑ने म॒द्ध्यन्दि॑ने ग्री॒ष्मे ग्री॒ष्मे म॒द्ध्यन्दि॑ने । \newline
56. म॒द्ध्यन्दि॑ने॒ त्रीꣳ स्त्रीन् म॒द्ध्यन्दि॑ने म॒द्ध्यन्दि॑ने॒ त्रीन् । \newline

\textbf{Ghana Paata } \newline

1. अ॒सा वा॑दि॒त्य आ॑दि॒त्यो॑ ऽसा व॒सा वा॑दि॒त्यो न नादि॒त्यो॑ ऽसा व॒सा वा॑दि॒त्यो न । \newline
2. आ॒दि॒त्यो न नादि॒त्य आ॑दि॒त्यो न वि वि नादि॒त्य आ॑दि॒त्यो न वि । \newline
3. न वि वि न न व्य॑रोचता रोचत॒ वि न न व्य॑रोचत । \newline
4. व्य॑रोचता रोचत॒ वि व्य॑रोचत॒ तस्मै॒ तस्मा॑ अरोचत॒ वि व्य॑रोचत॒ तस्मै᳚ । \newline
5. अ॒रो॒च॒त॒ तस्मै॒ तस्मा॑ अरोचता रोचत॒ तस्मै॑ दे॒वा दे॒वा स्तस्मा॑ अरोचता रोचत॒ तस्मै॑ दे॒वाः । \newline
6. तस्मै॑ दे॒वा दे॒वा स्तस्मै॒ तस्मै॑ दे॒वाः प्राय॑श्चित्ति॒म् प्राय॑श्चित्तिम् दे॒वा स्तस्मै॒ तस्मै॑ दे॒वाः प्राय॑श्चित्तिम् । \newline
7. दे॒वाः प्राय॑श्चित्ति॒म् प्राय॑श्चित्तिम् दे॒वा दे॒वाः प्राय॑श्चित्ति मैच्छन् नैच्छ॒न् प्राय॑श्चित्तिम् दे॒वा दे॒वाः प्राय॑श्चित्ति मैच्छन्न् । \newline
8. प्राय॑श्चित्ति मैच्छन् नैच्छ॒न् प्राय॑श्चित्ति॒म् प्राय॑श्चित्ति मैच्छ॒न् तस्मै॒ तस्मा॑ ऐच्छ॒न् प्राय॑श्चित्ति॒म् प्राय॑श्चित्ति मैच्छ॒न् तस्मै᳚ । \newline
9. ऐ॒च्छ॒न् तस्मै॒ तस्मा॑ ऐच्छन् नैच्छ॒न् तस्मा॑ ए॒ता मे॒ताम् तस्मा॑ ऐच्छन् नैच्छ॒न् तस्मा॑ ए॒ताम् । \newline
10. तस्मा॑ ए॒ता मे॒ताम् तस्मै॒ तस्मा॑ ए॒ताम् दश॑र्.षभा॒म् दश॑र्.षभा मे॒ताम् तस्मै॒ तस्मा॑ ए॒ताम् दश॑र्.षभाम् । \newline
11. ए॒ताम् दश॑र्.षभा॒म् दश॑र्.षभा मे॒ता मे॒ताम् दश॑र्.षभा॒ मा दश॑र्.षभा मे॒ता मे॒ताम् दश॑र्.षभा॒ मा । \newline
12. दश॑र्.षभा॒ मा दश॑र्.षभा॒म् दश॑र्.षभा॒ मा ऽल॑भन्ता लभ॒न्ता दश॑र्.षभा॒म् दश॑र्.षभा॒ मा ऽल॑भन्त । \newline
13. दश॑र्.षभा॒मिति॒ दश॑ - ऋ॒ष॒भा॒म् । \newline
14. आ ऽल॑भन्ता लभ॒न्ता ऽल॑भन्त॒ तया॒ तया॑ ऽलभ॒न्ता ऽल॑भन्त॒ तया᳚ । \newline
15. अ॒ल॒भ॒न्त॒ तया॒ तया॑ ऽलभन्ता लभन्त॒ तयै॒वैव तया॑ ऽलभन्ता लभन्त॒ तयै॒व । \newline
16. तयै॒वैव तया॒ तयै॒वास्मि॑न् नस्मिन् ने॒व तया॒ तयै॒वास्मिन्न्॑ । \newline
17. ए॒वास्मि॑न् नस्मिन् ने॒वैवास्मि॒न् रुचꣳ॒॒ रुच॑ मस्मिन् ने॒वैवास्मि॒न् रुच᳚म् । \newline
18. अ॒स्मि॒न् रुचꣳ॒॒ रुच॑ मस्मिन् नस्मि॒न् रुच॑ मदधु रदधू॒ रुच॑ मस्मिन् नस्मि॒न् रुच॑ मदधुः । \newline
19. रुच॑ मदधु रदधू॒ रुचꣳ॒॒ रुच॑ मदधु॒र् यो यो॑ ऽदधू॒ रुचꣳ॒॒ रुच॑ मदधु॒र् यः । \newline
20. अ॒द॒धु॒र् यो यो॑ ऽदधुरदधु॒र् यो ब्र॑ह्मवर्च॒सका॑मो ब्रह्मवर्च॒सका॑मो॒ यो॑ ऽदधु रदधु॒र् यो ब्र॑ह्मवर्च॒सका॑मः । \newline
21. यो ब्र॑ह्मवर्च॒सका॑मो ब्रह्मवर्च॒सका॑मो॒ यो यो ब्र॑ह्मवर्च॒सका॑मः॒ स्याथ् स्याद् ब्र॑ह्मवर्च॒सका॑मो॒ यो यो ब्र॑ह्मवर्च॒सका॑मः॒ स्यात् । \newline
22. ब्र॒ह्म॒व॒र्च॒सका॑मः॒ स्याथ् स्याद् ब्र॑ह्मवर्च॒सका॑मो ब्रह्मवर्च॒सका॑मः॒ स्यात् तस्मै॒ तस्मै॒ स्याद् ब्र॑ह्मवर्च॒सका॑मो ब्रह्मवर्च॒सका॑मः॒ स्यात् तस्मै᳚ । \newline
23. ब्र॒ह्म॒व॒र्च॒सका॑म॒ इति॑ ब्रह्मवर्च॒स - का॒मः॒ । \newline
24. स्यात् तस्मै॒ तस्मै॒ स्याथ् स्यात् तस्मा॑ ए॒ता मे॒ताम् तस्मै॒ स्याथ् स्यात् तस्मा॑ ए॒ताम् । \newline
25. तस्मा॑ ए॒ता मे॒ताम् तस्मै॒ तस्मा॑ ए॒ताम् दश॑र्.षभा॒म् दश॑र्.षभा मे॒ताम् तस्मै॒ तस्मा॑ ए॒ताम् दश॑र्.षभाम् । \newline
26. ए॒ताम् दश॑र्.षभा॒म् दश॑र्.षभा मे॒ता मे॒ताम् दश॑र्.षभा॒ मा दश॑र्.षभा मे॒ता मे॒ताम् दश॑र्.षभा॒ मा । \newline
27. दश॑र्.षभा॒ मा दश॑र्.षभा॒म् दश॑र्.षभा॒ मा ल॑भेत लभे॒ता दश॑र्.षभा॒म् दश॑र्.षभा॒ मा ल॑भेत । \newline
28. दश॑र्.षभा॒मिति॒ दश॑ - ऋ॒ष॒भा॒म् । \newline
29. आ ल॑भेत लभे॒ता ल॑भेता॒मु म॒मुम् ॅल॑भे॒ता ल॑भेता॒मुम् । \newline
30. ल॒भे॒ता॒मु म॒मुम् ॅल॑भेत लभेता॒मु मे॒वै वामुम् ॅल॑भेत लभेता॒मु मे॒व । \newline
31. अ॒मु मे॒वैवामु म॒मु मे॒वादि॒त्य मा॑दि॒त्य मे॒वामु म॒मु मे॒वादि॒त्यम् । \newline
32. ए॒वादि॒त्य मा॑दि॒त्य मे॒वै वादि॒त्यꣳ स्वेन॒ स्वेना॑दि॒त्य मे॒वै वादि॒त्यꣳ स्वेन॑ । \newline
33. आ॒दि॒त्यꣳ स्वेन॒ स्वेना॑दि॒त्य मा॑दि॒त्यꣳ स्वेन॑ भाग॒धेये॑न भाग॒धेये॑न॒ स्वेना॑दि॒त्य मा॑दि॒त्यꣳ स्वेन॑ भाग॒धेये॑न । \newline
34. स्वेन॑ भाग॒धेये॑न भाग॒धेये॑न॒ स्वेन॒ स्वेन॑ भाग॒धेये॒नो पोप॑ भाग॒धेये॑न॒ स्वेन॒ स्वेन॑ भाग॒धेये॒नोप॑ । \newline
35. भा॒ग॒धेये॒नोपोप॑ भाग॒धेये॑न भाग॒धेये॒नोप॑ धावति धाव॒त्युप॑ भाग॒धेये॑न भाग॒धेये॒नोप॑ धावति । \newline
36. भा॒ग॒धेये॒नेति॑ भाग - धेये॑न । \newline
37. उप॑ धावति धाव॒ त्युपोप॑ धावति॒ स स धा॑व॒ त्युपोप॑ धावति॒ सः । \newline
38. धा॒व॒ति॒ स स धा॑वति धावति॒ स ए॒वैव स धा॑वति धावति॒ स ए॒व । \newline
39. स ए॒वैव स स ए॒वास्मि॑न् नस्मिन् ने॒व स स ए॒वास्मिन्न्॑ । \newline
40. ए॒वास्मि॑न् नस्मिन् ने॒वैवास्मि॑न् ब्रह्मवर्च॒सम् ब्र॑ह्मवर्च॒स म॑स्मिन् ने॒वैवास्मि॑न् ब्रह्मवर्च॒सम् । \newline
41. अ॒स्मि॒न् ब्र॒ह्म॒व॒र्च॒सम् ब्र॑ह्मवर्च॒स म॑स्मिन् नस्मिन् ब्रह्मवर्च॒सम् द॑धाति दधाति ब्रह्मवर्च॒स म॑स्मिन् नस्मिन् ब्रह्मवर्च॒सम् द॑धाति । \newline
42. ब्र॒ह्म॒व॒र्च॒सम् द॑धाति दधाति ब्रह्मवर्च॒सम् ब्र॑ह्मवर्च॒सम् द॑धाति ब्रह्मवर्च॒सी ब्र॑ह्मवर्च॒सी द॑धाति ब्रह्मवर्च॒सम् ब्र॑ह्मवर्च॒सम् द॑धाति ब्रह्मवर्च॒सी । \newline
43. ब्र॒ह्म॒व॒र्च॒समिति॑ ब्रह्म - व॒र्च॒सम् । \newline
44. द॒धा॒ति॒ ब्र॒ह्म॒व॒र्च॒सी ब्र॑ह्मवर्च॒सी द॑धाति दधाति ब्रह्मवर्च॒ स्ये॑वैव ब्र॑ह्मवर्च॒सी द॑धाति दधाति ब्रह्मवर्च॒स्ये॑व । \newline
45. ब्र॒ह्म॒व॒र्च॒स्ये॑वैव ब्र॑ह्मवर्च॒सी ब्र॑ह्मवर्च॒ स्ये॑व भ॑वति भवत्ये॒व ब्र॑ह्मवर्च॒सी ब्र॑ह्मवर्च॒ स्ये॑व भ॑वति । \newline
46. ब्र॒ह्म॒व॒र्च॒सीति॑ ब्रह्म - व॒र्च॒सी । \newline
47. ए॒व भ॑वति भवत्ये॒वैव भ॑वति व॒सन्ता॑ व॒सन्ता॑ भवत्ये॒वैव भ॑वति व॒सन्ता᳚ । \newline
48. भ॒व॒ति॒ व॒सन्ता॑ व॒सन्ता॑ भवति भवति व॒सन्ता᳚ प्रा॒तः प्रा॒तर् व॒सन्ता॑ भवति भवति व॒सन्ता᳚ प्रा॒तः । \newline
49. व॒सन्ता᳚ प्रा॒तः प्रा॒तर् व॒सन्ता॑ व॒सन्ता᳚ प्रा॒त स्त्रीꣳ स्त्रीन् प्रा॒तर् व॒सन्ता॑ व॒सन्ता᳚ प्रा॒त स्त्रीन् । \newline
50. प्रा॒त स्त्रीꣳ स्त्रीन् प्रा॒तः प्रा॒त स्त्रीन् ॅल॒लामा᳚न् ॅल॒लामा॒न् त्रीन् प्रा॒तः प्रा॒त स्त्रीन् ॅल॒लामान्॑ । \newline
51. त्रीन् ॅल॒लामा᳚न् ॅल॒लामा॒न् त्रीꣳ स्त्रीन् ॅल॒लामा॒ ना ल॒लामा॒न् त्रीꣳ स्त्रीन् ॅल॒लामा॒ ना । \newline
52. ल॒लामा॒ ना ल॒लामा᳚न् ॅल॒लामा॒ ना ल॑भेत लभे॒ता ल॒लामा᳚न् ॅल॒लामा॒ ना ल॑भेत । \newline
53. आ ल॑भेत लभे॒ता ल॑भेत ग्री॒ष्मे ग्री॒ष्मे ल॑भे॒ता ल॑भेत ग्री॒ष्मे । \newline
54. ल॒भे॒त॒ ग्री॒ष्मे ग्री॒ष्मे ल॑भेत लभेत ग्री॒ष्मे म॒द्ध्यन्दि॑ने म॒द्ध्यन्दि॑ने ग्री॒ष्मे ल॑भेत लभेत ग्री॒ष्मे म॒द्ध्यन्दि॑ने । \newline
55. ग्री॒ष्मे म॒द्ध्यन्दि॑ने म॒द्ध्यन्दि॑ने ग्री॒ष्मे ग्री॒ष्मे म॒द्ध्यन्दि॑ने॒ त्रीꣳ स्त्रीन् म॒द्ध्यन्दि॑ने ग्री॒ष्मे ग्री॒ष्मे म॒द्ध्यन्दि॑ने॒ त्रीन् । \newline
56. म॒द्ध्यन्दि॑ने॒ त्रीꣳ स्त्रीन् म॒द्ध्यन्दि॑ने म॒द्ध्यन्दि॑ने॒ त्रीञ् छि॑तिपृ॒ष्ठाञ् छि॑तिपृ॒ष्ठान् त्रीन् म॒द्ध्यन्दि॑ने म॒द्ध्यन्दि॑ने॒ त्रीञ् छि॑तिपृ॒ष्ठान् । \newline
\pagebreak
\markright{ TS 2.1.4.2  \hfill https://www.vedavms.in \hfill}

\section{ TS 2.1.4.2 }

\textbf{TS 2.1.4.2 } \newline
\textbf{Samhita Paata} \newline

त्रीञ्छि॑ति पृ॒ष्ठाञ्छ॒रद्य॑परा॒ह्णे त्रीञ्छि॑ति॒वारा॒न् त्रीणि॒ वा आ॑दि॒त्यस्य॒ तेजाꣳ॑सि व॒सन्ता᳚ प्रा॒तर्ग्री॒ष्मे म॒द्ध्यन्दि॑ने श॒रद्य॑परा॒ह्णे याव॑न्त्ये॒व तेजाꣳ॑सि॒ तान्ये॒वाव॑ रुन्धे॒ त्रय॑स्त्रय॒ आ ल॑भ्यन्ते-ऽभि पू॒र्वमे॒वास्मि॒न् तेजो॑ दधाति संॅवथ्स॒रं प॒र्याल॑भ्यन्ते संॅवथ्स॒रो वै ब्र॑ह्मवर्च॒सस्य॑ प्रदा॒ता सं॑ॅवथ्स॒र ए॒वास्मै᳚ ब्रह्मवर्च॒सं प्र य॑च्छति ब्रह्मवर्च॒स्ये॑व भ॑वति संॅवथ्स॒रस्य॑ प॒रस्ता᳚त् प्राजाप॒त्यं कद्रु॒ - [  ] \newline

\textbf{Pada Paata} \newline

त्रीन् । शि॒ति॒पृ॒ष्ठानिति॑ शिति - पृ॒ष्ठान् । श॒रदि॑ । अ॒प॒रा॒ह्ण इत्य॑पर-अ॒ह्ने । त्रीन् । शि॒ति॒वारा॒निति॑ शिति - वारान्॑ । त्रीणि॑ । वै । आ॒दि॒त्यस्य॑ । तेजाꣳ॑सि । व॒सन्ता᳚ । प्रा॒तः । ग्री॒ष्मे । म॒द्ध्यन्दि॑ने । श॒रदि॑ । अ॒प॒रा॒ह्ण इत्य॑पर-अ॒ह्ने । याव॑न्ति । ए॒व । तेजाꣳ॑सि । तानि॑ । ए॒व । अवेति॑ । रु॒न्धे॒ । त्रय॑स्त्रय॒ इति॒ त्रयः॑-त्र॒यः॒ । एति॑ । ल॒भ्य॒न्ते॒ । अ॒भि॒पू॒र्वमित्य॑भि - पू॒र्वम् । ए॒व । अ॒स्मि॒न्न् । तेजः॑ । द॒धा॒ति॒ । सं॒ॅव॒थ्स॒रमिति॑ सं - व॒थ्स॒रम् । प॒र्याल॑भ्यन्त॒ इति॑ परि-आल॑भ्यन्ते । सं॒ॅव॒थ्स॒र इति॑ सं - व॒थ्स॒रः । वै । ब्र॒ह्म॒व॒र्च॒सस्येति॑ ब्रह्म - व॒र्च॒सस्य॑ । प्र॒दा॒तेति॑ प्र - दा॒ता । सं॒ॅव॒थ्स॒र इति॑ सं - व॒थ्स॒रः । ए॒व । अ॒स्मै॒ । ब्र॒ह्म॒व॒र्च॒समिति॑ ब्रह्म - व॒र्च॒सम् । प्रेति॑ । य॒च्छ॒ति॒ । ब्र॒ह्म॒व॒र्च॒सीति॑ ब्रह्म - व॒र्च॒सी । ए॒व । भ॒व॒ति॒ । सं॒ॅव॒थ्स॒रस्येति॑ सं - व॒थ्स॒रस्य॑ । प॒रस्ता᳚त् । प्रा॒जा॒प॒त्यमिति॑ प्राजा - प॒त्यम् । कद्रु᳚म् ।  \newline


\textbf{Krama Paata} \newline

त्रीञ्छि॑तिपृ॒ष्ठान् । शि॒ति॒पृ॒ष्ठाञ्छ॒रदि॑ । शि॒ति॒पृ॒ष्ठानिति॑ शिति - पृ॒ष्ठान् । श॒रद्य॑परा॒ह्णे । अ॒प॒रा॒ह्णे त्रीन् । अ॒प॒रा॒ह्ण इत्य॑पर - अ॒ह्ने । त्रीञ्छि॑ति॒वारान्॑ । शि॒ति॒वारा॒न् त्रीणि॑ । शि॒ति॒वारा॒निति॑ शिति - वारान्॑ । त्रीणि॒ वै । वा आ॑दि॒त्यस्य॑ । आ॒दि॒त्यस्य॒ तेजाꣳ॑सि । तेजाꣳ॑सि व॒सन्ता᳚ । व॒सन्ता᳚ प्रा॒तः । प्रा॒तर् ग्री॒ष्मे । ग्री॒ष्मे म॒द्ध्यन्दि॑ने । म॒द्ध्यन्दि॑ने श॒रदि॑ । श॒रद्य॑परा॒ह्णे । अ॒प॒रा॒ह्णे याव॑न्ति । अ॒प॒रा॒ह्ण इत्य॑पर - अ॒ह्ने । याव॑न्त्ये॒व । ए॒व तेजाꣳ॑सि । तेजाꣳ॑सि॒ तानि॑ । तान्ये॒व । ए॒वाव॑ । अव॑ रुन्धे । रु॒न्धे॒ त्रय॑स्त्रयः । त्रय॑स्त्रय॒ आ । त्रय॑स्त्रय॒ इति॒ त्रयः॑ - त्र॒यः॒ । आ ल॑भ्यन्ते । ल॒भ्य॒न्ते॒ ऽभि॒पू॒र्वम् । अ॒भि॒पू॒र्वमे॒व । अ॒भि॒॒पू॒र्वमित्य॑भि - पू॒र्वम् । ए॒वास्मिन्न्॑ । अ॒स्मि॒न्,तेजः॑ । तेजो॑ दधाति । द॒धा॒ति॒ स॒म्ॅव॒थ्स॒रम् । स॒म्ॅव॒॒थ्स॒रं प॒र्याल॑भ्यन्ते । स॒म्ॅव॒थ्स॒रमिति॑ सं - व॒थ्स॒रम् । प॒र्याल॑भ्यन्ते सम्ॅवथ्स॒रः । प॒र्याल॑भ्यन्त॒ इति॑ परि - आल॑भ्यन्ते । स॒म्॒ॅव॒थ्स॒रो वै । स॒म्ॅव॒थ्स॒र इति॑ सं - व॒थ्स॒रः । वै ब्र॑ह्मवर्च॒सस्य॑ । ब्र॒ह्म॒व॒र्च॒सस्य॑ प्रदा॒ता । ब॒ह्म॒व॒र्च॒सस्येति॑ ब्रह्म - व॒र्च॒सस्य॑ । प्र॒दा॒ता स॑म्ॅवथ्स॒रः । प्र॒दा॒तेति॑ प्र - दा॒ता । स॒म्ॅव॒थ्स॒र ए॒व । स॒म्ॅव॒थ्स॒र इति॑ सं - व॒थ्स॒रः । ए॒वास्मै᳚ । अ॒स्मै॒ ब्र॒ह्म॒व॒र्च॒सम् । ब्र॒ह्म॒व॒र्च॒सम् प्र । ब्र॒ह्म॒व॒र्च॒समिति॑ ब्रह्म - व॒र्च॒सम् । प्र य॑च्छति । य॒च्छ॒ति॒ ब्र॒ह्म॒व॒र्च॒सी । ब॒॒ह्म॒व॒र्च॒स्ये॑व । ब्र॒ह्म॒व॒र्च॒सीति॑ ब्रह्म - व॒र्च॒सी । ए॒व भ॑वति । भ॒व॒ति॒ स॒म्ॅव॒थ्स॒रस्य॑ । स॒म्ॅव॒थ्स॒रस्य॑ प॒रस्ता᳚त् । स॒म्ॅव॒थ्स॒रस्येति॑ सं - व॒थ्स॒रस्य॑ । प॒रस्ता᳚त्,प्राजाप॒त्यम् । प्रा॒जा॒प॒त्यम् कद्रु᳚म् । प्रा॒जा॒प॒त्यमिति॑ प्राजा - प॒त्यम् । कद्रु॒मा \newline

\textbf{Jatai Paata} \newline

1. त्रीञ् छि॑तिपृ॒ष्ठाञ् छि॑तिपृ॒ष्ठान् त्रीꣳ स्त्रीञ् छि॑तिपृ॒ष्ठान् । \newline
2. शि॒ति॒पृ॒ष्ठाञ् छ॒रदि॑ श॒रदि॑ शितिपृ॒ष्ठाञ् छि॑तिपृ॒ष्ठाञ् छ॒रदि॑ । \newline
3. शि॒ति॒पृ॒ष्ठानिति॑ शिति - पृ॒ष्ठान् । \newline
4. श॒रद्य॑परा॒ह्णे॑ ऽपरा॒ह्णे श॒रदि॑ श॒रद्य॑परा॒ह्णे । \newline
5. अ॒प॒रा॒ह्णे त्रीꣳ स्त्री न॑परा॒ह्णे॑ ऽपरा॒ह्णे त्रीन् । \newline
6. अ॒प॒रा॒ह्ण इत्य॑पर - अ॒ह्ने । \newline
7. त्रीञ् छि॑ति॒वारा᳚ञ् छिति॒वारा॒न् त्रीꣳ स्त्रीञ् छि॑ति॒वारान्॑ । \newline
8. शि॒ति॒वारा॒न् त्रीणि॒ त्रीणि॑ शिति॒वारा᳚ञ् छिति॒वारा॒न् त्रीणि॑ । \newline
9. शि॒ति॒वारा॒निति॑ शिति - वारान्॑ । \newline
10. त्रीणि॒ वै वै त्रीणि॒ त्रीणि॒ वै । \newline
11. वा आ॑दि॒त्यस्या॑ दि॒त्यस्य॒ वै वा आ॑दि॒त्यस्य॑ । \newline
12. आ॒दि॒त्यस्य॒ तेजाꣳ॑सि॒ तेजाꣳ॑ स्यादि॒त्यस्या॑ दि॒त्यस्य॒ तेजाꣳ॑सि । \newline
13. तेजाꣳ॑सि व॒सन्ता॑ व॒सन्ता॒ तेजाꣳ॑सि॒ तेजाꣳ॑सि व॒सन्ता᳚ । \newline
14. व॒सन्ता᳚ प्रा॒तः प्रा॒तर् व॒सन्ता॑ व॒सन्ता᳚ प्रा॒तः । \newline
15. प्रा॒तर् ग्री॒ष्मे ग्री॒ष्मे प्रा॒तः प्रा॒तर् ग्री॒ष्मे । \newline
16. ग्री॒ष्मे म॒द्ध्यन्दि॑ने म॒द्ध्यन्दि॑ने ग्री॒ष्मे ग्री॒ष्मे म॒द्ध्यन्दि॑ने । \newline
17. म॒द्ध्यन्दि॑ने श॒रदि॑ श॒रदि॑ म॒द्ध्यन्दि॑ने म॒द्ध्यन्दि॑ने श॒रदि॑ । \newline
18. श॒रद्य॑परा॒ह्णे॑ ऽपरा॒ह्णे श॒रदि॑ श॒रद्य॑परा॒ह्णे । \newline
19. अ॒प॒रा॒ह्णे याव॑न्ति॒ याव॑ न्त्यपरा॒ह्णे॑ ऽपरा॒ह्णे याव॑न्ति । \newline
20. अ॒प॒रा॒ह्णैत्य॑पर - अ॒ह्ने । \newline
21. याव॑ न्त्ये॒वैव याव॑न्ति॒ याव॑ न्त्ये॒व । \newline
22. ए॒व तेजाꣳ॑सि॒ तेजाꣳ॑ स्ये॒वैव तेजाꣳ॑सि । \newline
23. तेजाꣳ॑सि॒ तानि॒ तानि॒ तेजाꣳ॑सि॒ तेजाꣳ॑सि॒ तानि॑ । \newline
24. ता न्ये॒वैव तानि॒ तान्ये॒व । \newline
25. ए॒वावा वै॒वै वाव॑ । \newline
26. अव॑ रुन्धे रु॒न्धे ऽवाव॑ रुन्धे । \newline
27. रु॒न्धे॒ त्रय॑स्त्रय॒ स्त्रय॑स्त्रयो रुन्धे रुन्धे॒ त्रय॑स्त्रयः । \newline
28. त्रय॑स्त्रय॒ आ त्रय॑स्त्रय॒ स्त्रय॑स्त्रय॒ आ । \newline
29. त्रय॑स्त्रय॒ इति॒ त्रयः॑ - त्र॒यः॒ । \newline
30. आ ल॑भ्यन्ते लभ्यन्त॒ आ ल॑भ्यन्ते । \newline
31. ल॒भ्य॒न्ते॒ ऽभि॒पू॒र्व म॑भिपू॒र्वम् ॅल॑भ्यन्ते लभ्यन्ते ऽभिपू॒र्वम् । \newline
32. अ॒भि॒पू॒र्व मे॒वैवाभि॑पू॒र्व म॑भिपू॒र्व मे॒व । \newline
33. अ॒भि॒पू॒र्वमित्य॑भि - पू॒र्वम् । \newline
34. ए॒वास्मि॑न् नस्मिन् ने॒वैवास्मिन्न्॑ । \newline
35. अ॒स्मि॒न् तेज॒ स्तेजो᳚ ऽस्मिन् नस्मि॒न् तेजः॑ । \newline
36. तेजो॑ दधाति दधाति॒ तेज॒ स्तेजो॑ दधाति । \newline
37. द॒धा॒ति॒ सं॒ॅव॒थ्स॒रꣳ सं॑ॅवथ्स॒रम् द॑धाति दधाति संॅवथ्स॒रम् । \newline
38. सं॒ॅव॒थ्स॒रम् प॒र्याल॑भ्यन्ते प॒र्याल॑भ्यन्ते संॅवथ्स॒रꣳ सं॑ॅवथ्स॒रम् प॒र्याल॑भ्यन्ते । \newline
39. सं॒ॅव॒थ्स॒रमिति॑ सं - व॒थ्स॒रम् । \newline
40. प॒र्याल॑भ्यन्ते संॅवथ्स॒रः सं॑ॅवथ्स॒रः प॒र्याल॑भ्यन्ते प॒र्याल॑भ्यन्ते संॅवथ्स॒रः । \newline
41. प॒र्याल॑भ्यन्त॒ इति॑ परि - आल॑भ्यन्ते । \newline
42. सं॒ॅव॒थ्स॒रो वै वै सं॑ॅवथ्स॒रः सं॑ॅवथ्स॒रो वै । \newline
43. सं॒ॅव॒थ्स॒र इति॑ सं - व॒थ्स॒रः । \newline
44. वै ब्र॑ह्मवर्च॒सस्य॑ ब्रह्मवर्च॒सस्य॒ वै वै ब्र॑ह्मवर्च॒सस्य॑ । \newline
45. ब्र॒ह्म॒व॒र्च॒सस्य॑ प्रदा॒ता प्र॑दा॒ता ब्र॑ह्मवर्च॒सस्य॑ ब्रह्मवर्च॒सस्य॑ प्रदा॒ता । \newline
46. ब्र॒ह्म॒व॒र्च॒सस्येति॑ ब्रह्म - व॒र्च॒सस्य॑ । \newline
47. प्र॒दा॒ता सं॑ॅवथ्स॒रः सं॑ॅवथ्स॒रः प्र॑दा॒ता प्र॑दा॒ता सं॑ॅवथ्स॒रः । \newline
48. प्र॒दा॒तेति॑ प्र - दा॒ता । \newline
49. सं॒ॅव॒थ्स॒र ए॒वैव सं॑ॅवथ्स॒रः सं॑ॅवथ्स॒र ए॒व । \newline
50. सं॒ॅव॒थ्स॒र इति॑ सं - व॒थ्स॒रः । \newline
51. ए॒वास्मा॑ अस्मा ए॒वैवास्मै᳚ । \newline
52. अ॒स्मै॒ ब्र॒ह्म॒व॒र्च॒सम् ब्र॑ह्मवर्च॒स म॑स्मा अस्मै ब्रह्मवर्च॒सम् । \newline
53. ब्र॒ह्म॒व॒र्च॒सम् प्र प्र ब्र॑ह्मवर्च॒सम् ब्र॑ह्मवर्च॒सम् प्र । \newline
54. ब्र॒ह्म॒व॒र्च॒समिति॑ ब्रह्म - व॒र्च॒सम् । \newline
55. प्र य॑च्छति यच्छति॒ प्र प्र य॑च्छति । \newline
56. य॒च्छ॒ति॒ ब्र॒ह्म॒व॒र्च॒सी ब्र॑ह्मवर्च॒सी य॑च्छति यच्छति ब्रह्मवर्च॒सी । \newline
57. ब्र॒ह्म॒व॒र्च॒ स्ये॑वैव ब्र॑ह्मवर्च॒सी ब्र॑ह्मवर्च॒ स्ये॑व । \newline
58. ब्र॒ह्म॒व॒र्च॒सीति॑ ब्रह्म - व॒र्च॒सी । \newline
59. ए॒व भ॑वति भव त्ये॒वैव भ॑वति । \newline
60. भ॒व॒ति॒ सं॒ॅव॒थ्स॒रस्य॑ संॅवथ्स॒रस्य॑ भवति भवति संॅवथ्स॒रस्य॑ । \newline
61. सं॒ॅव॒थ्स॒रस्य॑ प॒रस्ता᳚त् प॒रस्ता᳚थ् संॅवथ्स॒रस्य॑ संॅवथ्स॒रस्य॑ प॒रस्ता᳚त् । \newline
62. सं॒ॅव॒थ्स॒रस्येति॑ सं - व॒थ्स॒रस्य॑ । \newline
63. प॒रस्ता᳚त् प्राजाप॒त्यम् प्रा॑जाप॒त्यम् प॒रस्ता᳚त् प॒रस्ता᳚त् प्राजाप॒त्यम् । \newline
64. प्रा॒जा॒प॒त्यम् कद्रु॒म् कद्रु॑म् प्राजाप॒त्यम् प्रा॑जाप॒त्यम् कद्रु᳚म् । \newline
65. प्रा॒जा॒प॒त्यमिति॑ प्राजा - प॒त्यम् । \newline
66. कद्रु॒ मा कद्रु॒म् कद्रु॒ मा । \newline

\textbf{Ghana Paata } \newline

1. त्रीञ् छि॑तिपृ॒ष्ठाञ् छि॑तिपृ॒ष्ठान् त्रीꣳ स्त्रीञ् छि॑तिपृ॒ष्ठाञ् छ॒रदि॑ श॒रदि॑ शितिपृ॒ष्ठान् त्रीꣳ स्त्रीञ् छि॑तिपृ॒ष्ठाञ् छ॒रदि॑ । \newline
2. शि॒ति॒पृ॒ष्ठाञ् छ॒रदि॑ श॒रदि॑ शितिपृ॒ष्ठाञ् छि॑तिपृ॒ष्ठाञ् छ॒रद्य॑परा॒ह्णे॑ ऽपरा॒ह्णे श॒रदि॑ शितिपृ॒ष्ठाञ् छि॑तिपृ॒ष्ठाञ् छ॒रद्य॑परा॒ह्णे । \newline
3. शि॒ति॒पृ॒ष्ठानिति॑ शिति - पृ॒ष्ठान् । \newline
4. श॒रद्य॑परा॒ह्णे॑ ऽपरा॒ह्णे श॒रदि॑ श॒रद्य॑परा॒ह्णे त्रीꣳ स्त्री न॑परा॒ह्णे श॒रदि॑ श॒रद्य॑परा॒ह्णे त्रीन् । \newline
5. अ॒प॒रा॒ह्णे त्रीꣳ स्त्री न॑परा॒ह्णे॑ ऽपरा॒ह्णे त्रीञ् छि॑ति॒वारा᳚ञ् छिति॒वारा॒न् त्री न॑परा॒ह्णे॑ ऽपरा॒ह्णे त्रीञ् छि॑ति॒वारान्॑ । \newline
6. अ॒प॒रा॒ह्ण इत्य॑पर - अ॒ह्ने । \newline
7. त्रीञ् छि॑ति॒वारा᳚ञ् छिति॒वारा॒न् त्रीꣳ स्त्रीञ् छि॑ति॒वारा॒न् त्रीणि॒ त्रीणि॑ शिति॒वारा॒न् त्रीꣳ स्त्रीञ् छि॑ति॒वारा॒न् त्रीणि॑ । \newline
8. शि॒ति॒वारा॒न् त्रीणि॒ त्रीणि॑ शिति॒वारा᳚ञ् छिति॒वारा॒न् त्रीणि॒ वै वै त्रीणि॑ शिति॒वारा᳚ञ् छिति॒वारा॒न् त्रीणि॒ वै । \newline
9. शि॒ति॒वारा॒निति॑ शिति - वारान्॑ । \newline
10. त्रीणि॒ वै वै त्रीणि॒ त्रीणि॒ वा आ॑दि॒त्यस्या॑ दि॒त्यस्य॒ वै त्रीणि॒ त्रीणि॒ वा आ॑दि॒त्यस्य॑ । \newline
11. वा आ॑दि॒त्यस्या॑ दि॒त्यस्य॒ वै वा आ॑दि॒त्यस्य॒ तेजाꣳ॑सि॒ तेजाꣳ॑ स्यादि॒त्यस्य॒ वै वा आ॑दि॒त्यस्य॒ तेजाꣳ॑सि । \newline
12. आ॒दि॒त्यस्य॒ तेजाꣳ॑सि॒ तेजाꣳ॑ स्यादि॒त्यस्या॑ दि॒त्यस्य॒ तेजाꣳ॑सि व॒सन्ता॑ व॒सन्ता॒ तेजाꣳ॑स्या दि॒त्यस्या॑ दि॒त्यस्य॒ तेजाꣳ॑सि व॒सन्ता᳚ । \newline
13. तेजाꣳ॑सि व॒सन्ता॑ व॒सन्ता॒ तेजाꣳ॑सि॒ तेजाꣳ॑सि व॒सन्ता᳚ प्रा॒तः प्रा॒तर् व॒सन्ता॒ तेजाꣳ॑सि॒ तेजाꣳ॑सि व॒सन्ता᳚ प्रा॒तः । \newline
14. व॒सन्ता᳚ प्रा॒तः प्रा॒तर् व॒सन्ता॑ व॒सन्ता᳚ प्रा॒तर् ग्री॒ष्मे ग्री॒ष्मे प्रा॒तर् व॒सन्ता॑ व॒सन्ता᳚ प्रा॒तर् ग्री॒ष्मे । \newline
15. प्रा॒तर् ग्री॒ष्मे ग्री॒ष्मे प्रा॒तः प्रा॒तर् ग्री॒ष्मे म॒द्ध्यन्दि॑ने म॒द्ध्यन्दि॑ने ग्री॒ष्मे प्रा॒तः प्रा॒तर् ग्री॒ष्मे म॒द्ध्यन्दि॑ने । \newline
16. ग्री॒ष्मे म॒द्ध्यन्दि॑ने म॒द्ध्यन्दि॑ने ग्री॒ष्मे ग्री॒ष्मे म॒द्ध्यन्दि॑ने श॒रदि॑ श॒रदि॑ म॒द्ध्यन्दि॑ने ग्री॒ष्मे ग्री॒ष्मे म॒द्ध्यन्दि॑ने श॒रदि॑ । \newline
17. म॒द्ध्यन्दि॑ने श॒रदि॑ श॒रदि॑ म॒द्ध्यन्दि॑ने म॒द्ध्यन्दि॑ने श॒रद्य॑परा॒ह्णे॑ ऽपरा॒ह्णे श॒रदि॑ म॒द्ध्यन्दि॑ने म॒द्ध्यन्दि॑ने श॒रद्य॑परा॒ह्णे । \newline
18. श॒रद्य॑परा॒ह्णे॑ ऽपरा॒ह्णे श॒रदि॑ श॒रद्य॑परा॒ह्णे याव॑न्ति॒ याव॑न्त्यपरा॒ह्णे श॒रदि॑ श॒रद्य॑परा॒ह्णे याव॑न्ति । \newline
19. अ॒प॒रा॒ह्णे याव॑न्ति॒ याव॑न्त्यपरा॒ह्णे॑ ऽपरा॒ह्णे याव॑न्त्ये॒वैव याव॑न्त्यपरा॒ह्णे॑ ऽपरा॒ह्णे याव॑न्त्ये॒व । \newline
20. अ॒प॒रा॒ह्णैत्य॑पर - अ॒ह्ने । \newline
21. याव॑न्त्ये॒वैव याव॑न्ति॒ याव॑न्त्ये॒व तेजाꣳ॑सि॒ तेजाꣳ॑ स्ये॒व याव॑न्ति॒ याव॑न्त्ये॒व तेजाꣳ॑सि । \newline
22. ए॒व तेजाꣳ॑सि॒ तेजाꣳ॑स्ये॒वैव तेजाꣳ॑सि॒ तानि॒ तानि॒ तेजाꣳ॑ स्ये॒वैव तेजाꣳ॑सि॒ तानि॑ । \newline
23. तेजाꣳ॑सि॒ तानि॒ तानि॒ तेजाꣳ॑सि॒ तेजाꣳ॑सि॒ तान्ये॒वैव तानि॒ तेजाꣳ॑सि॒ तेजाꣳ॑सि॒ तान्ये॒व । \newline
24. तान्ये॒वैव तानि॒ तान्ये॒ वावावै॒व तानि॒ तान्ये॒वाव॑ । \newline
25. ए॒वा वावै॒वै वाव॑ रुन्धे रु॒न्धे ऽवै॒वैवाव॑ रुन्धे । \newline
26. अव॑ रुन्धे रु॒न्धे ऽवाव॑ रुन्धे॒ त्रय॑स्त्रय॒ स्त्रय॑स्त्रयो रु॒न्धे ऽवाव॑ रुन्धे॒ त्रय॑स्त्रयः । \newline
27. रु॒न्धे॒ त्रय॑स्त्रय॒ स्त्रय॑स्त्रयो रुन्धे रुन्धे॒ त्रय॑स्त्रय॒ आ त्रय॑स्त्रयो रुन्धे रुन्धे॒ त्रय॑स्त्रय॒ आ । \newline
28. त्रय॑स्त्रय॒ आ त्रय॑स्त्रय॒ स्त्रय॑स्त्रय॒ आ ल॑भ्यन्ते लभ्यन्त॒ आ त्रय॑स्त्रय॒ स्त्रय॑स्त्रय॒ आ ल॑भ्यन्ते । \newline
29. त्रय॑स्त्रय॒ इति॒ त्रयः॑ - त्र॒यः॒ । \newline
30. आ ल॑भ्यन्ते लभ्यन्त॒ आ ल॑भ्यन्ते ऽभिपू॒र्व म॑भिपू॒र्वम् ॅल॑भ्यन्त॒ आ ल॑भ्यन्ते ऽभिपू॒र्वम् । \newline
31. ल॒भ्य॒न्ते॒ ऽभि॒पू॒र्व म॑भिपू॒र्वम् ॅल॑भ्यन्ते लभ्यन्ते ऽभिपू॒र्व मे॒वैवा भि॑पू॒र्वम् ॅल॑भ्यन्ते लभ्यन्ते ऽभिपू॒र्व मे॒व । \newline
32. अ॒भि॒पू॒र्व मे॒वैवा भि॑पू॒र्व म॑भिपू॒र्व मे॒वास्मि॑न् नस्मिन् ने॒वाभि॑पू॒र्व म॑भिपू॒र्व मे॒वास्मिन्न्॑ । \newline
33. अ॒भि॒पू॒र्वमित्य॑भि - पू॒र्वम् । \newline
34. ए॒वास्मि॑न् नस्मिन् ने॒वैवास्मि॒न् तेज॒ स्तेजो᳚ ऽस्मिन् ने॒वैवास्मि॒न् तेजः॑ । \newline
35. अ॒स्मि॒न् तेज॒ स्तेजो᳚ ऽस्मिन् नस्मि॒न् तेजो॑ दधाति दधाति॒ तेजो᳚ ऽस्मिन् नस्मि॒न् तेजो॑ दधाति । \newline
36. तेजो॑ दधाति दधाति॒ तेज॒ स्तेजो॑ दधाति संॅवथ्स॒रꣳ सं॑ॅवथ्स॒रम् द॑धाति॒ तेज॒ स्तेजो॑ दधाति संॅवथ्स॒रम् । \newline
37. द॒धा॒ति॒ सं॒ॅव॒थ्स॒रꣳ सं॑ॅवथ्स॒रम् द॑धाति दधाति संॅवथ्स॒रम् प॒र्याल॑भ्यन्ते प॒र्याल॑भ्यन्ते संॅवथ्स॒रम् द॑धाति दधाति संॅवथ्स॒रम् प॒र्याल॑भ्यन्ते । \newline
38. सं॒ॅव॒थ्स॒रम् प॒र्याल॑भ्यन्ते प॒र्याल॑भ्यन्ते संॅवथ्स॒रꣳ सं॑ॅवथ्स॒रम् प॒र्याल॑भ्यन्ते संॅवथ्स॒रः सं॑ॅवथ्स॒रः प॒र्याल॑भ्यन्ते संॅवथ्स॒रꣳ सं॑ॅवथ्स॒रम् प॒र्याल॑भ्यन्ते संॅवथ्स॒रः । \newline
39. सं॒ॅव॒थ्स॒रमिति॑ सं - व॒थ्स॒रम् । \newline
40. प॒र्याल॑भ्यन्ते संॅवथ्स॒रः सं॑ॅवथ्स॒रः प॒र्याल॑भ्यन्ते प॒र्याल॑भ्यन्ते संॅवथ्स॒रो वै वै सं॑ॅवथ्स॒रः प॒र्याल॑भ्यन्ते प॒र्याल॑भ्यन्ते संॅवथ्स॒रो वै । \newline
41. प॒र्याल॑भ्यन्त॒ इति॑ परि - आल॑भ्यन्ते । \newline
42. सं॒ॅव॒थ्स॒रो वै वै सं॑ॅवथ्स॒रः सं॑ॅवथ्स॒रो वै ब्र॑ह्मवर्च॒सस्य॑ ब्रह्मवर्च॒सस्य॒ वै सं॑ॅवथ्स॒रः सं॑ॅवथ्स॒रो वै ब्र॑ह्मवर्च॒सस्य॑ । \newline
43. सं॒ॅव॒थ्स॒र इति॑ सं - व॒थ्स॒रः । \newline
44. वै ब्र॑ह्मवर्च॒सस्य॑ ब्रह्मवर्च॒सस्य॒ वै वै ब्र॑ह्मवर्च॒सस्य॑ प्रदा॒ता प्र॑दा॒ता ब्र॑ह्मवर्च॒सस्य॒ वै वै ब्र॑ह्मवर्च॒सस्य॑ प्रदा॒ता । \newline
45. ब्र॒ह्म॒व॒र्च॒सस्य॑ प्रदा॒ता प्र॑दा॒ता ब्र॑ह्मवर्च॒सस्य॑ ब्रह्मवर्च॒सस्य॑ प्रदा॒ता सं॑ॅवथ्स॒रः सं॑ॅवथ्स॒रः प्र॑दा॒ता ब्र॑ह्मवर्च॒सस्य॑ ब्रह्मवर्च॒सस्य॑ प्रदा॒ता सं॑ॅवथ्स॒रः । \newline
46. ब्र॒ह्म॒व॒र्च॒सस्येति॑ ब्रह्म - व॒र्च॒सस्य॑ । \newline
47. प्र॒दा॒ता सं॑ॅवथ्स॒रः सं॑ॅवथ्स॒रः प्र॑दा॒ता प्र॑दा॒ता सं॑ॅवथ्स॒र ए॒वैव सं॑ॅवथ्स॒रः प्र॑दा॒ता प्र॑दा॒ता सं॑ॅवथ्स॒र ए॒व । \newline
48. प्र॒दा॒तेति॑ प्र - दा॒ता । \newline
49. सं॒ॅव॒थ्स॒र ए॒वैव सं॑ॅवथ्स॒रः सं॑ॅवथ्स॒र ए॒वास्मा॑ अस्मा ए॒व सं॑ॅवथ्स॒रः सं॑ॅवथ्स॒र ए॒वास्मै᳚ । \newline
50. सं॒ॅव॒थ्स॒र इति॑ सं - व॒थ्स॒रः । \newline
51. ए॒वास्मा॑ अस्मा ए॒वैवास्मै᳚ ब्रह्मवर्च॒सम् ब्र॑ह्मवर्च॒स म॑स्मा ए॒वैवास्मै᳚ ब्रह्मवर्च॒सम् । \newline
52. अ॒स्मै॒ ब्र॒ह्म॒व॒र्च॒सम् ब्र॑ह्मवर्च॒स म॑स्मा अस्मै ब्रह्मवर्च॒सम् प्र प्र ब्र॑ह्मवर्च॒स म॑स्मा अस्मै ब्रह्मवर्च॒सम् प्र । \newline
53. ब्र॒ह्म॒व॒र्च॒सम् प्र प्र ब्र॑ह्मवर्च॒सम् ब्र॑ह्मवर्च॒सम् प्र य॑च्छति यच्छति॒ प्र ब्र॑ह्मवर्च॒सम् ब्र॑ह्मवर्च॒सम् प्र य॑च्छति । \newline
54. ब्र॒ह्म॒व॒र्च॒समिति॑ ब्रह्म - व॒र्च॒सम् । \newline
55. प्र य॑च्छति यच्छति॒ प्र प्र य॑च्छति ब्रह्मवर्च॒सी ब्र॑ह्मवर्च॒सी य॑च्छति॒ प्र प्र य॑च्छति ब्रह्मवर्च॒सी । \newline
56. य॒च्छ॒ति॒ ब्र॒ह्म॒व॒र्च॒सी ब्र॑ह्मवर्च॒सी य॑च्छति यच्छति ब्रह्मवर्च॒स्ये॑वैव ब्र॑ह्मवर्च॒सी य॑च्छति यच्छति ब्रह्मवर्च॒स्ये॑व । \newline
57. ब्र॒ह्म॒व॒र्च॒स्ये॑वैव ब्र॑ह्मवर्च॒सी ब्र॑ह्मवर्च॒स्ये॑व भ॑वति भवत्ये॒व ब्र॑ह्मवर्च॒सी ब्र॑ह्मवर्च॒स्ये॑व भ॑वति । \newline
58. ब्र॒ह्म॒व॒र्च॒सीति॑ ब्रह्म - व॒र्च॒सी । \newline
59. ए॒व भ॑वति भवत्ये॒वैव भ॑वति संॅवथ्स॒रस्य॑ संॅवथ्स॒रस्य॑ भवत्ये॒वैव भ॑वति संॅवथ्स॒रस्य॑ । \newline
60. भ॒व॒ति॒ सं॒ॅव॒थ्स॒रस्य॑ संॅवथ्स॒रस्य॑ भवति भवति संॅवथ्स॒रस्य॑ प॒रस्ता᳚त् प॒रस्ता᳚थ् संॅवथ्स॒रस्य॑ भवति भवति संॅवथ्स॒रस्य॑ प॒रस्ता᳚त् । \newline
61. सं॒ॅव॒थ्स॒रस्य॑ प॒रस्ता᳚त् प॒रस्ता᳚थ् संॅवथ्स॒रस्य॑ संॅवथ्स॒रस्य॑ प॒रस्ता᳚त् प्राजाप॒त्यम् प्रा॑जाप॒त्यम् प॒रस्ता᳚थ् संॅवथ्स॒रस्य॑ संॅवथ्स॒रस्य॑ प॒रस्ता᳚त् प्राजाप॒त्यम् । \newline
62. सं॒ॅव॒थ्स॒रस्येति॑ सं - व॒थ्स॒रस्य॑ । \newline
63. प॒रस्ता᳚त् प्राजाप॒त्यम् प्रा॑जाप॒त्यम् प॒रस्ता᳚त् प॒रस्ता᳚त् प्राजाप॒त्यम् कद्रु॒म् कद्रु॑म् प्राजाप॒त्यम् प॒रस्ता᳚त् प॒रस्ता᳚त् प्राजाप॒त्यम् कद्रु᳚म् । \newline
64. प्रा॒जा॒प॒त्यम् कद्रु॒म् कद्रु॑म् प्राजाप॒त्यम् प्रा॑जाप॒त्यम् कद्रु॒ मा कद्रु॑म् प्राजाप॒त्यम् प्रा॑जाप॒त्यम् कद्रु॒ मा । \newline
65. प्रा॒जा॒प॒त्यमिति॑ प्राजा - प॒त्यम् । \newline
66. कद्रु॒ मा कद्रु॒म् कद्रु॒ मा ल॑भेत लभे॒ता कद्रु॒म् कद्रु॒ मा ल॑भेत । \newline
\pagebreak
\markright{ TS 2.1.4.3  \hfill https://www.vedavms.in \hfill}

\section{ TS 2.1.4.3 }

\textbf{TS 2.1.4.3 } \newline
\textbf{Samhita Paata} \newline

-मा ल॑भेत प्र॒जाप॑तिः॒ सर्वा॑ दे॒वता॑ दे॒वता᳚स्वे॒व प्रति॑तिष्ठति॒ यदि॑ बिभी॒याद्-दु॒श्चर्मा॑ भविष्या॒मीति॑ सोमापौ॒ष्णꣳ श्या॒ममा ल॑भेत सौ॒म्यो वै दे॒वत॑या॒ पुरु॑षः पौ॒ष्णाः प॒शवः॒ स्वयै॒वास्मै॑ दे॒वत॑या प॒शुभि॒स्त्वचं॑ करोति॒ न दु॒श्चर्मा॑ भवति दे॒वाश्च॒ वै य॒मश्चा॒स्मिन् ॅलो॒के᳚ऽस्पर्द्धन्त॒ स य॒मो दे॒वाना॑मिन्द्रि॒यं ॅवी॒र्य॑मयुवत॒ तद्य॒मस्य॑ - [  ] \newline

\textbf{Pada Paata} \newline

एति॑ । ल॒भे॒त॒ । प्र॒जाप॑ति॒रिति॑ प्र॒जा - प॒तिः॒ । सर्वाः᳚ । दे॒वताः᳚ । दे॒वता॑सु । ए॒व । प्रतीति॑ । ति॒ष्ठ॒ति॒ । यदि॑ । बि॒भी॒यात् । दु॒श्चर्मेति॑ दुः-चर्मा᳚ । भ॒वि॒ष्या॒मि॒ । इति॑ । सो॒मा॒पौ॒ष्णमिति॑ सोमा - पौ॒ष्णम् । श्या॒मम् । एति॑ । ल॒भे॒त॒ । सौ॒म्यः । वै । दे॒वत॑या । पुरु॑षः । पौ॒ष्णाः । प॒शवः॑ । स्वया᳚ । ए॒व । अ॒स्मै॒ । दे॒वत॑या । प॒शुभि॒रिति॑ प॒शु - भिः॒ । त्वच᳚म् । क॒रो॒ति॒ । न । दु॒श्चर्मेति॑ दुः - चर्मा᳚ । भ॒व॒ति॒ । दे॒वाः । च॒ । वै । य॒मः । च॒ । अ॒स्मिन्न् । लो॒के । अ॒स्प॒र्ध॒न्त॒ । सः । य॒मः । दे॒वाना᳚म् । इ॒न्द्रि॒यम् । वी॒र्य᳚म् । अ॒यु॒व॒त॒ । तत् । य॒मस्य॑ ।  \newline


\textbf{Krama Paata} \newline

आ ल॑भेत । ल॒भे॒त॒ प्र॒जाप॑तिः । प्र॒जाप॑तिः॒ सर्वाः᳚ । प्र॒जाप॑ति॒रिति॑ प्र॒जा - प॒तिः॒ । सर्वा॑ दे॒वताः᳚ । दे॒वता॑ दे॒वता॑सु । दे॒वता᳚स्वे॒व । ए॒व प्रति॑ । प्रति॑ तिष्ठति । ति॒ष्ठ॒ति॒ यदि॑ । यदि॑ बिभी॒यात् । बि॒भी॒याद् दु॒श्चर्मा᳚ । दु॒श्चर्मा॑ भविष्यामि । दु॒श्चर्मेति॑ दुः - चर्मा᳚ । भ॒वि॒ष्या॒मीति॑ । इति॑ सोमापौ॒ष्णम् । सो॒मा॒पौ॒ष्णꣳ श्या॒मम् । सो॒मा॒पौ॒ष्णमिति॑ सोमा - पौ॒ष्णम् । श्या॒ममा । आ ल॑भेत । ल॒भे॒त॒ सौ॒म्यः । सौ॒म्यो वै । वै दे॒वत॑या । दे॒वत॑या॒ पुरु॑षः । पुरु॑षः पौ॒ष्णाः । पौ॒ष्णाः प॒शवः॑ । प॒शवः॒ स्वया᳚ । स्वयै॒व । ए॒वास्मै᳚ । अ॒स्मै॒ दे॒वत॑या । दे॒वत॑या प॒शुभिः॑ । प॒शुभि॒ स्त्वच᳚म् । प॒शुभि॒रिति॑ प॒शु - भिः॒ । त्वच॑म् करोति । क॒रो॒ति॒ न । न दु॒श्चर्मा᳚ । दु॒श्चर्मा॑ भवति । दु॒श्चर्मेति॑ दुः - चर्मा᳚ । भ॒व॒ति॒ दे॒वाः । दे॒वाश्च॑ । च॒ वै । वै य॒मः । य॒मश्च॑ । चा॒स्मिन्न् । अ॒स्मिन् ॅलो॒के । लो॒के᳚ऽस्पर्द्धन्त । अ॒स्प॒र्द्ध॒॒न्त॒ सः । स य॒मः । य॒मो दे॒वाना᳚म् । दे॒वाना॑मिन्द्रि॒यम् । इ॒न्द्रि॒यं ॅवी॒र्य᳚म् । वी॒र्य॑मयुवत । अ॒यु॒व॒त॒ तत् । तद् य॒मस्य॑ । य॒मस्य॑ यम॒त्वम् \newline

\textbf{Jatai Paata} \newline

1. आ ल॑भेत लभे॒ता ल॑भेत । \newline
2. ल॒भे॒त॒ प्र॒जाप॑तिः प्र॒जाप॑तिर् लभेत लभेत प्र॒जाप॑तिः । \newline
3. प्र॒जाप॑तिः॒ सर्वाः॒ सर्वाः᳚ प्र॒जाप॑तिः प्र॒जाप॑तिः॒ सर्वाः᳚ । \newline
4. प्र॒जाप॑ति॒रिति॑ प्र॒जा - प॒तिः॒ । \newline
5. सर्वा॑ दे॒वता॑ दे॒वताः॒ सर्वाः॒ सर्वा॑ दे॒वताः᳚ । \newline
6. दे॒वता॑ दे॒वता॑सु दे॒वता॑सु दे॒वता॑ दे॒वता॑ दे॒वता॑सु । \newline
7. दे॒वता᳚ स्वे॒वैव दे॒वता॑सु दे॒वता᳚ स्वे॒व । \newline
8. ए॒व प्रति॒ प्र त्ये॒वैव प्रति॑ । \newline
9. प्रति॑ तिष्ठति तिष्ठति॒ प्रति॒ प्रति॑ तिष्ठति । \newline
10. ति॒ष्ठ॒ति॒ यदि॒ यदि॑ तिष्ठति तिष्ठति॒ यदि॑ । \newline
11. यदि॑ बिभी॒याद् बि॑भी॒याद् यदि॒ यदि॑ बिभी॒यात् । \newline
12. बि॒भी॒याद् दु॒श्चर्मा॑ दु॒श्चर्मा॑ बिभी॒याद् बि॑भी॒याद् दु॒श्चर्मा᳚ । \newline
13. दु॒श्चर्मा॑ भविष्यामि भविष्यामि दु॒श्चर्मा॑ दु॒श्चर्मा॑ भविष्यामि । \newline
14. दु॒श्चर्मेति॑ दुः - चर्मा᳚ । \newline
15. भ॒वि॒ष्या॒ मीतीति॑ भविष्यामि भविष्या॒ मीति॑ । \newline
16. इति॑ सोमापौ॒ष्णꣳ सो॑मापौ॒ष्ण मितीति॑ सोमापौ॒ष्णम् । \newline
17. सो॒मा॒पौ॒ष्णꣳ श्या॒मꣳ श्या॒मꣳ सो॑मापौ॒ष्णꣳ सो॑मापौ॒ष्णꣳ श्या॒मम् । \newline
18. सो॒मा॒पौ॒ष्णमिति॑ सोमा - पौ॒ष्णम् । \newline
19. श्या॒म मा श्या॒मꣳ श्या॒म मा । \newline
20. आ ल॑भेत लभे॒ता ल॑भेत । \newline
21. ल॒भे॒त॒ सौ॒म्यः सौ॒म्यो ल॑भेत लभेत सौ॒म्यः । \newline
22. सौ॒म्यो वै वै सौ॒म्यः सौ॒म्यो वै । \newline
23. वै दे॒वत॑या दे॒वत॑या॒ वै वै दे॒वत॑या । \newline
24. दे॒वत॑या॒ पुरु॑षः॒ पुरु॑षो दे॒वत॑या दे॒वत॑या॒ पुरु॑षः । \newline
25. पुरु॑षः पौ॒ष्णाः पौ॒ष्णाः पुरु॑षः॒ पुरु॑षः पौ॒ष्णाः । \newline
26. पौ॒ष्णाः प॒शवः॑ प॒शवः॑ पौ॒ष्णाः पौ॒ष्णाः प॒शवः॑ । \newline
27. प॒शवः॒ स्वया॒ स्वया॑ प॒शवः॑ प॒शवः॒ स्वया᳚ । \newline
28. स्वयै॒वैव स्वया॒ स्वयै॒व । \newline
29. ए॒वास्मा॑ अस्मा ए॒वैवास्मै᳚ । \newline
30. अ॒स्मै॒ दे॒वत॑या दे॒वत॑या ऽस्मा अस्मै दे॒वत॑या । \newline
31. दे॒वत॑या प॒शुभिः॑ प॒शुभि॑र् दे॒वत॑या दे॒वत॑या प॒शुभिः॑ । \newline
32. प॒शुभि॒ स्त्वच॒म् त्वच॑म् प॒शुभिः॑ प॒शुभि॒ स्त्वच᳚म् । \newline
33. प॒शुभि॒रिति॑ प॒शु - भिः॒ । \newline
34. त्वच॑म् करोति करोति॒ त्वच॒म् त्वच॑म् करोति । \newline
35. क॒रो॒ति॒ न न क॑रोति करोति॒ न । \newline
36. न दु॒श्चर्मा॑ दु॒श्चर्मा॒ न न दु॒श्चर्मा᳚ । \newline
37. दु॒श्चर्मा॑ भवति भवति दु॒श्चर्मा॑ दु॒श्चर्मा॑ भवति । \newline
38. दु॒श्चर्मेति॑ दुः - चर्मा᳚ । \newline
39. भ॒व॒ति॒ दे॒वा दे॒वा भ॑वति भवति दे॒वाः । \newline
40. दे॒वाश्च॑ च दे॒वा दे॒वाश्च॑ । \newline
41. च॒ वै वै च॑ च॒ वै । \newline
42. वै य॒मो य॒मो वै वै य॒मः । \newline
43. य॒मश्च॑ च य॒मो य॒मश्च॑ । \newline
44. चा॒स्मिन् न॒स्मिꣳश्च॑ चा॒स्मिन्न् । \newline
45. अ॒स्मिन् ॅलो॒के लो॒के᳚ ऽस्मिन् न॒स्मिन् ॅलो॒के । \newline
46. लो॒के᳚ ऽस्पर्द्धन्ता स्पर्द्धन्त लो॒के लो॒के᳚ ऽस्पर्द्धन्त । \newline
47. अ॒स्प॒र्द्ध॒न्त॒ स सो᳚ ऽस्पर्द्धन्ता स्पर्द्धन्त॒ सः । \newline
48. स य॒मो य॒मः स स य॒मः । \newline
49. य॒मो दे॒वाना᳚म् दे॒वानां᳚ ॅय॒मो य॒मो दे॒वाना᳚म् । \newline
50. दे॒वाना॑ मिन्द्रि॒य मि॑न्द्रि॒यम् दे॒वाना᳚म् दे॒वाना॑ मिन्द्रि॒यम् । \newline
51. इ॒न्द्रि॒यं ॅवी॒र्यं॑ ॅवी॒र्य॑ मिन्द्रि॒य मि॑न्द्रि॒यं ॅवी॒र्य᳚म् । \newline
52. वी॒र्य॑ मयुवता युवत वी॒र्यं॑ ॅवी॒र्य॑ मयुवत । \newline
53. अ॒यु॒व॒त॒ तत् तद॑युवता युवत॒ तत् । \newline
54. तद् य॒मस्य॑ य॒मस्य॒ तत् तद् य॒मस्य॑ । \newline
55. य॒मस्य॑ यम॒त्वं ॅय॑म॒त्वं ॅय॒मस्य॑ य॒मस्य॑ यम॒त्वम् । \newline

\textbf{Ghana Paata } \newline

1. आ ल॑भेत लभे॒ता ल॑भेत प्र॒जाप॑तिः प्र॒जाप॑तिर् लभे॒ता ल॑भेत प्र॒जाप॑तिः । \newline
2. ल॒भे॒त॒ प्र॒जाप॑तिः प्र॒जाप॑तिर् लभेत लभेत प्र॒जाप॑तिः॒ सर्वाः॒ सर्वाः᳚ प्र॒जाप॑तिर् लभेत लभेत प्र॒जाप॑तिः॒ सर्वाः᳚ । \newline
3. प्र॒जाप॑तिः॒ सर्वाः॒ सर्वाः᳚ प्र॒जाप॑तिः प्र॒जाप॑तिः॒ सर्वा॑ दे॒वता॑ दे॒वताः॒ सर्वाः᳚ प्र॒जाप॑तिः प्र॒जाप॑तिः॒ सर्वा॑ दे॒वताः᳚ । \newline
4. प्र॒जाप॑ति॒रिति॑ प्र॒जा - प॒तिः॒ । \newline
5. सर्वा॑ दे॒वता॑ दे॒वताः॒ सर्वाः॒ सर्वा॑ दे॒वता॑ दे॒वता॑सु दे॒वता॑सु दे॒वताः॒ सर्वाः॒ सर्वा॑ दे॒वता॑ दे॒वता॑सु । \newline
6. दे॒वता॑ दे॒वता॑सु दे॒वता॑सु दे॒वता॑ दे॒वता॑ दे॒वता᳚ स्वे॒वैव दे॒वता॑सु दे॒वता॑ दे॒वता॑ दे॒वता᳚ स्वे॒व । \newline
7. दे॒वता᳚ स्वे॒वैव दे॒वता॑सु दे॒वता᳚स्वे॒व प्रति॒ प्रत्ये॒व दे॒वता॑सु दे॒वता᳚स्वे॒व प्रति॑ । \newline
8. ए॒व प्रति॒ प्रत्ये॒वैव प्रति॑ तिष्ठति तिष्ठति॒ प्रत्ये॒वैव प्रति॑ तिष्ठति । \newline
9. प्रति॑ तिष्ठति तिष्ठति॒ प्रति॒ प्रति॑ तिष्ठति॒ यदि॒ यदि॑ तिष्ठति॒ प्रति॒ प्रति॑ तिष्ठति॒ यदि॑ । \newline
10. ति॒ष्ठ॒ति॒ यदि॒ यदि॑ तिष्ठति तिष्ठति॒ यदि॑ बिभी॒याद् बि॑भी॒याद् यदि॑ तिष्ठति तिष्ठति॒ यदि॑ बिभी॒यात् । \newline
11. यदि॑ बिभी॒याद् बि॑भी॒याद् यदि॒ यदि॑ बिभी॒याद् दु॒श्चर्मा॑ दु॒श्चर्मा॑ बिभी॒याद् यदि॒ यदि॑ बिभी॒याद् दु॒श्चर्मा᳚ । \newline
12. बि॒भी॒याद् दु॒श्चर्मा॑ दु॒श्चर्मा॑ बिभी॒याद् बि॑भी॒याद् दु॒श्चर्मा॑ भविष्यामि भविष्यामि दु॒श्चर्मा॑ बिभी॒याद् बि॑भी॒याद् दु॒श्चर्मा॑ भविष्यामि । \newline
13. दु॒श्चर्मा॑ भविष्यामि भविष्यामि दु॒श्चर्मा॑ दु॒श्चर्मा॑ भविष्या॒मीतीति॑ भविष्यामि दु॒श्चर्मा॑ दु॒श्चर्मा॑ भविष्या॒मीति॑ । \newline
14. दु॒श्चर्मेति॑ दुः - चर्मा᳚ । \newline
15. भ॒वि॒ष्या॒मीतीति॑ भविष्यामि भविष्या॒मीति॑ सोमापौ॒ष्णꣳ सो॑मापौ॒ष्ण मिति॑ भविष्यामि भविष्या॒मीति॑ सोमापौ॒ष्णम् । \newline
16. इति॑ सोमापौ॒ष्णꣳ सो॑मापौ॒ष्ण मितीति॑ सोमापौ॒ष्णꣳ श्या॒मꣳ श्या॒मꣳ सो॑मापौ॒ष्ण मितीति॑ सोमापौ॒ष्णꣳ श्या॒मम् । \newline
17. सो॒मा॒पौ॒ष्णꣳ श्या॒मꣳ श्या॒मꣳ सो॑मापौ॒ष्णꣳ सो॑मापौ॒ष्णꣳ श्या॒म मा श्या॒मꣳ सो॑मापौ॒ष्णꣳ सो॑मापौ॒ष्णꣳ श्या॒म मा । \newline
18. सो॒मा॒पौ॒ष्णमिति॑ सोमा - पौ॒ष्णम् । \newline
19. श्या॒म मा श्या॒मꣳ श्या॒म मा ल॑भेत लभे॒ता श्या॒मꣳ श्या॒म मा ल॑भेत । \newline
20. आ ल॑भेत लभे॒ता ल॑भेत सौ॒म्यः सौ॒म्यो ल॑भे॒ता ल॑भेत सौ॒म्यः । \newline
21. ल॒भे॒त॒ सौ॒म्यः सौ॒म्यो ल॑भेत लभेत सौ॒म्यो वै वै सौ॒म्यो ल॑भेत लभेत सौ॒म्यो वै । \newline
22. सौ॒म्यो वै वै सौ॒म्यः सौ॒म्यो वै दे॒वत॑या दे॒वत॑या॒ वै सौ॒म्यः सौ॒म्यो वै दे॒वत॑या । \newline
23. वै दे॒वत॑या दे॒वत॑या॒ वै वै दे॒वत॑या॒ पुरु॑षः॒ पुरु॑षो दे॒वत॑या॒ वै वै दे॒वत॑या॒ पुरु॑षः । \newline
24. दे॒वत॑या॒ पुरु॑षः॒ पुरु॑षो दे॒वत॑या दे॒वत॑या॒ पुरु॑षः पौ॒ष्णाः पौ॒ष्णाः पुरु॑षो दे॒वत॑या दे॒वत॑या॒ पुरु॑षः पौ॒ष्णाः । \newline
25. पुरु॑षः पौ॒ष्णाः पौ॒ष्णाः पुरु॑षः॒ पुरु॑षः पौ॒ष्णाः प॒शवः॑ प॒शवः॑ पौ॒ष्णाः पुरु॑षः॒ पुरु॑षः पौ॒ष्णाः प॒शवः॑ । \newline
26. पौ॒ष्णाः प॒शवः॑ प॒शवः॑ पौ॒ष्णाः पौ॒ष्णाः प॒शवः॒ स्वया॒ स्वया॑ प॒शवः॑ पौ॒ष्णाः पौ॒ष्णाः प॒शवः॒ स्वया᳚ । \newline
27. प॒शवः॒ स्वया॒ स्वया॑ प॒शवः॑ प॒शवः॒ स्वयै॒वैव स्वया॑ प॒शवः॑ प॒शवः॒ स्वयै॒व । \newline
28. स्वयै॒वैव स्वया॒ स्वयै॒वास्मा॑ अस्मा ए॒व स्वया॒ स्वयै॒वास्मै᳚ । \newline
29. ए॒वास्मा॑ अस्मा ए॒वैवास्मै॑ दे॒वत॑या दे॒वत॑या ऽस्मा ए॒वैवास्मै॑ दे॒वत॑या । \newline
30. अ॒स्मै॒ दे॒वत॑या दे॒वत॑या ऽस्मा अस्मै दे॒वत॑या प॒शुभिः॑ प॒शुभि॑र् दे॒वत॑या ऽस्मा अस्मै दे॒वत॑या प॒शुभिः॑ । \newline
31. दे॒वत॑या प॒शुभिः॑ प॒शुभि॑र् दे॒वत॑या दे॒वत॑या प॒शुभि॒ स्त्वच॒म् त्वच॑म् प॒शुभि॑र् दे॒वत॑या दे॒वत॑या प॒शुभि॒ स्त्वच᳚म् । \newline
32. प॒शुभि॒ स्त्वच॒म् त्वच॑म् प॒शुभिः॑ प॒शुभि॒ स्त्वच॑म् करोति करोति॒ त्वच॑म् प॒शुभिः॑ प॒शुभि॒ स्त्वच॑म् करोति । \newline
33. प॒शुभि॒रिति॑ प॒शु - भिः॒ । \newline
34. त्वच॑म् करोति करोति॒ त्वच॒म् त्वच॑म् करोति॒ न न क॑रोति॒ त्वच॒म् त्वच॑म् करोति॒ न । \newline
35. क॒रो॒ति॒ न न क॑रोति करोति॒ न दु॒श्चर्मा॑ दु॒श्चर्मा॒ न क॑रोति करोति॒ न दु॒श्चर्मा᳚ । \newline
36. न दु॒श्चर्मा॑ दु॒श्चर्मा॒ न न दु॒श्चर्मा॑ भवति भवति दु॒श्चर्मा॒ न न दु॒श्चर्मा॑ भवति । \newline
37. दु॒श्चर्मा॑ भवति भवति दु॒श्चर्मा॑ दु॒श्चर्मा॑ भवति दे॒वा दे॒वा भ॑वति दु॒श्चर्मा॑ दु॒श्चर्मा॑ भवति दे॒वाः । \newline
38. दु॒श्चर्मेति॑ दुः - चर्मा᳚ । \newline
39. भ॒व॒ति॒ दे॒वा दे॒वा भ॑वति भवति दे॒वाश्च॑ च दे॒वा भ॑वति भवति दे॒वाश्च॑ । \newline
40. दे॒वाश्च॑ च दे॒वा दे॒वाश्च॒ वै वै च॑ दे॒वा दे॒वाश्च॒ वै । \newline
41. च॒ वै वै च॑ च॒ वै य॒मो य॒मो वै च॑ च॒ वै य॒मः । \newline
42. वै य॒मो य॒मो वै वै य॒मश्च॑ च य॒मो वै वै य॒मश्च॑ । \newline
43. य॒मश्च॑ च य॒मो य॒मश्चा॒स्मिन् न॒स्मिꣳश्च॑ य॒मो य॒मश्चा॒स्मिन्न् । \newline
44. चा॒स्मिन् न॒स्मिꣳश्च॑ चा॒स्मिन् ॅलो॒के लो॒के᳚ ऽस्मिꣳश्च॑ चा॒स्मिन् ॅलो॒के । \newline
45. अ॒स्मिन् ॅलो॒के लो॒के᳚ ऽस्मिन् न॒स्मिन् ॅलो॒के᳚ ऽस्पर्द्धन्ता स्पर्द्धन्त लो॒के᳚ ऽस्मिन् न॒स्मिन् ॅलो॒के᳚ ऽस्पर्द्धन्त । \newline
46. लो॒के᳚ ऽस्पर्द्धन्ता स्पर्द्धन्त लो॒के लो॒के᳚ ऽस्पर्द्धन्त॒ स सो᳚ ऽस्पर्द्धन्त लो॒के लो॒के᳚ ऽस्पर्द्धन्त॒ सः । \newline
47. अ॒स्प॒र्द्ध॒न्त॒ स सो᳚ ऽस्पर्द्धन्ता स्पर्द्धन्त॒ स य॒मो य॒मः सो᳚ ऽस्पर्द्धन्ता स्पर्द्धन्त॒ स य॒मः । \newline
48. स य॒मो य॒मः स स य॒मो दे॒वाना᳚म् दे॒वानां᳚ ॅय॒मः स स य॒मो दे॒वाना᳚म् । \newline
49. य॒मो दे॒वाना᳚म् दे॒वानां᳚ ॅय॒मो य॒मो दे॒वाना॑ मिन्द्रि॒य मि॑न्द्रि॒यम् दे॒वानां᳚ ॅय॒मो य॒मो दे॒वाना॑ मिन्द्रि॒यम् । \newline
50. दे॒वाना॑ मिन्द्रि॒य मि॑न्द्रि॒यम् दे॒वाना᳚म् दे॒वाना॑ मिन्द्रि॒यं ॅवी॒र्यं॑ ॅवी॒र्य॑ मिन्द्रि॒यम् दे॒वाना᳚म् दे॒वाना॑ मिन्द्रि॒यं ॅवी॒र्य᳚म् । \newline
51. इ॒न्द्रि॒यं ॅवी॒र्यं॑ ॅवी॒र्य॑ मिन्द्रि॒य मि॑न्द्रि॒यं ॅवी॒र्य॑ मयुवता युवत वी॒र्य॑ मिन्द्रि॒य मि॑न्द्रि॒यं ॅवी॒र्य॑ मयुवत । \newline
52. वी॒र्य॑ मयुवता युवत वी॒र्यं॑ ॅवी॒र्य॑ मयुवत॒ तत् तद॑युवत वी॒र्यं॑ ॅवी॒र्य॑ मयुवत॒ तत् । \newline
53. अ॒यु॒व॒त॒ तत् तद॑युवता युवत॒ तद् य॒मस्य॑ य॒मस्य॒ तद॑युवता युवत॒ तद् य॒मस्य॑ । \newline
54. तद् य॒मस्य॑ य॒मस्य॒ तत् तद् य॒मस्य॑ यम॒त्वं ॅय॑म॒त्वं ॅय॒मस्य॒ तत् तद् य॒मस्य॑ यम॒त्वम् । \newline
55. य॒मस्य॑ यम॒त्वं ॅय॑म॒त्वं ॅय॒मस्य॑ य॒मस्य॑ यम॒त्वम् ते ते य॑म॒त्वं ॅय॒मस्य॑ य॒मस्य॑ यम॒त्वम् ते । \newline
\pagebreak
\markright{ TS 2.1.4.4  \hfill https://www.vedavms.in \hfill}

\section{ TS 2.1.4.4 }

\textbf{TS 2.1.4.4 } \newline
\textbf{Samhita Paata} \newline

यम॒त्वं ते दे॒वा अ॑मन्यन्त य॒मो वा इ॒दम॑भू॒द्-यद्-व॒यꣳ स्म इति॒ ते प्र॒जाप॑ति॒मुपा॑धाव॒न्थ् स ए॒तौ प्र॒जाप॑तिरा॒त्मन॑ उक्षव॒शौ निर॑मिमीत॒ ते दे॒वा वै᳚ष्णावरु॒णीं ॅव॒शामा-ऽल॑भन्तै॒न्द्रमु॒क्षाण॒तं ॅवरु॑णेनै॒व ग्रा॑हयि॒त्वा विष्णु॑ना य॒ज्ञेन॒ प्राणु॑दन्तै॒न्द्रेणै॒-वास्ये᳚न्द्रि॒यम॑-वृञ्जत॒ यो भ्रातृ॑व्यवा॒न्थ् स्याथ् स स्पर्द्ध॑मानो वैष्णावरु॒णीं - [  ] \newline

\textbf{Pada Paata} \newline

य॒म॒त्वमिति॑ यम - त्वम् । ते । दे॒वाः । अ॒म॒न्य॒न्त॒ । य॒मः । वै । इ॒दम् । अ॒भू॒त् । यत् । व॒यम् । स्मः । इति॑ । ते । प्र॒जाप॑ति॒मिति॑ प्र॒जा - प॒ति॒म् । उपेति॑ । अ॒धा॒वन्न् । सः । ए॒तौ । प्र॒जाप॑ति॒रिति॑ प्र॒जा - प॒तिः॒ । आ॒त्मनः॑ । उ॒क्ष॒व॒शावित्यु॑क्ष - व॒शौ । निरिति॑ । अ॒मि॒मी॒त॒ । ते । दे॒वाः । वै॒ष्णा॒व॒रु॒णीमिति॑ वैष्णा-व॒रु॒णीम् । व॒शाम् । एति॑ । अ॒ल॒भ॒न्त॒ । ऐ॒न्द्रम् । उ॒क्षाण᳚म् । तम् । वरु॑णेन । ए॒व । ग्रा॒ह॒यि॒त्वा । विष्णु॑ना । य॒ज्ञेन॑ । प्रेति॑ । अ॒नु॒द॒न्त॒ । ऐ॒न्द्रेण॑ । ए॒व । अ॒स्य॒ । इ॒न्द्रि॒यम् । अ॒वृ॒ञ्ज॒त॒ । यः । भ्रातृ॑व्यवा॒निति॒ भ्रातृ॑व्य - वा॒न् । स्यात् । सः । स्पर्द्ध॑मानः । वै॒ष्णा॒व॒रु॒णीमिति॑ वैष्णा - व॒रु॒णीम् ।  \newline


\textbf{Krama Paata} \newline

य॒म॒त्वं ते । य॒म॒त्वमिति॑ यम - त्वम् । ते दे॒वाः । दे॒वा अ॑मन्यन्त । अ॒म॒न्य॒न्त॒ य॒मः । य॒मो वै । वा इ॒दम् । इ॒दम॑भूत् । अ॒भू॒द् यत् । यद् व॒यम् । व॒यꣳ स्मः । स्म इति॑ । इति॒ ते । ते प्र॒जाप॑तिम् । प्र॒जाप॑ति॒मुप॑ । प्र॒जाप॑ति॒मिति॑ प्र॒जा - प॒ति॒म् । उपा॑धावन्न् । अ॒धा॒व॒न्थ् सः । स ए॒तौ । ए॒तौ प्र॒जाप॑तिः । प्र॒जाप॑ति,रा॒त्मनः॑ । प्र॒जाप॑ति॒रिति॑ प्र॒जा - प॒तिः॒ । आ॒त्मन॑ उक्षव॒शौ । उ॒क्ष॒व॒शौ निः । उ॒क्ष॒व॒शावित्यु॑क्ष - व॒शौ । निर॑मिमीत । अ॒मि॒मी॒त॒ ते । ते दे॒वाः । दे॒वा वै᳚ष्णावरु॒णीम् । वै॒ष्णा॒व॒रु॒णीं ॅव॒शाम् । वै॒ष्णा॒व॒रु॒णीमिति॑ वैष्णा - व॒रु॒णीम् । व॒शामा । आ ऽल॑भन्त । अ॒ल॒भ॒न्तै॒न्द्रम् । ऐ॒न्द्रमु॒क्षाण᳚म् । उ॒क्षाणं॒ तम् । तं ॅवरु॑णेन । वरु॑णेनै॒व । ए॒व ग्रा॑हयि॒त्वा । ग्रा॒ह॒यि॒त्वा विष्णु॑ना । विष्णु॑ना य॒ज्ञेन॑ । य॒ज्ञेन॒ प्र । प्राणु॑दन्त । अ॒नु॒द॒न्तै॒न्द्रेण॑ । ऐ॒न्द्रेणै॒व । ए॒वास्य॑ । अ॒स्ये॒न्द्रि॒यम् । इ॒न्द्रि॒यम॑वृञ्चत । अ॒वृ॒ञ्च॒त॒ यः । यो भ्रातृ॑व्यवान् । भातृ॑व्यवा॒न्थ् स्यात् । भातृ॑व्यवा॒निति॒ भातृ॑व्य - वा॒न्॒ । स्याथ् सः । स स्पर्द्ध॑मानः । स्पर्द्ध॑मानो वैष्णावरु॒णीम् । वै॒षा॒॒व॒रु॒णीं ॅव॒शाम् । वै॒ष्णा॒व॒रु॒णीमिति॑ वैष्णा - व॒रु॒णीं \newline

\textbf{Jatai Paata} \newline

1. य॒म॒त्वम् ते ते य॑म॒त्वं ॅय॑म॒त्वम् ते । \newline
2. य॒म॒त्वमिति॑ यम - त्वम् । \newline
3. ते दे॒वा दे॒वा स्ते ते दे॒वाः । \newline
4. दे॒वा अ॑मन्यन्ता मन्यन्त दे॒वा दे॒वा अ॑मन्यन्त । \newline
5. अ॒म॒न्य॒न्त॒ य॒मो य॒मो॑ ऽमन्यन्ता मन्यन्त य॒मः । \newline
6. य॒मो वै वै य॒मो य॒मो वै । \newline
7. वा इ॒द मि॒दं ॅवै वा इ॒दम् । \newline
8. इ॒द म॑भू दभू दि॒द मि॒द म॑भूत् । \newline
9. अ॒भू॒द् यद् यद॑भू दभू॒द् यत् । \newline
10. यद् व॒यं ॅव॒यं ॅयद् यद् व॒यम् । \newline
11. व॒यꣳ स्मः स्मो व॒यं ॅव॒यꣳ स्मः । \newline
12. स्म इतीति॒ स्मः स्म इति॑ । \newline
13. इति॒ ते त इतीति॒ ते । \newline
14. ते प्र॒जाप॑तिम् प्र॒जाप॑ति॒म् ते ते प्र॒जाप॑तिम् । \newline
15. प्र॒जाप॑ति॒ मुपोप॑ प्र॒जाप॑तिम् प्र॒जाप॑ति॒ मुप॑ । \newline
16. प्र॒जाप॑ति॒मिति॑ प्र॒जा - प॒ति॒म् । \newline
17. उपा॑धावन् नधाव॒न् नुपोपा॑धावन्न् । \newline
18. अ॒धा॒व॒न् थ्स सो॑ ऽधावन् नधाव॒न् थ्सः । \newline
19. स ए॒ता वे॒तौ स स ए॒तौ । \newline
20. ए॒तौ प्र॒जाप॑तिः प्र॒जाप॑ति रे॒ता वे॒तौ प्र॒जाप॑तिः । \newline
21. प्र॒जाप॑ति रा॒त्मन॑ आ॒त्मनः॑ प्र॒जाप॑तिः प्र॒जाप॑ति रा॒त्मनः॑ । \newline
22. प्र॒जाप॑ति॒रिति॑ प्र॒जा - प॒तिः॒ । \newline
23. आ॒त्मन॑ उक्षव॒शा वु॑क्षव॒शा वा॒त्मन॑ आ॒त्मन॑ उक्षव॒शौ । \newline
24. उ॒क्ष॒व॒शौ निर् णिरु॑क्षव॒शा वु॑क्षव॒शौ निः । \newline
25. उ॒क्ष॒व॒शावित्यु॑क्ष - व॒शौ । \newline
26. निर॑मिमीता मिमीत॒ निर् णिर॑मिमीत । \newline
27. अ॒मि॒मी॒त॒ ते ते॑ ऽमिमीता मिमीत॒ ते । \newline
28. ते दे॒वा दे॒वा स्ते ते दे॒वाः । \newline
29. दे॒वा वै᳚ष्णावरु॒णीं ॅवै᳚ष्णावरु॒णीम् दे॒वा दे॒वा वै᳚ष्णावरु॒णीम् । \newline
30. वै॒ष्णा॒व॒रु॒णीं ॅव॒शां ॅव॒शां ॅवै᳚ष्णावरु॒णीं ॅवै᳚ष्णावरु॒णीं ॅव॒शाम् । \newline
31. वै॒ष्णा॒व॒रु॒णीमिति॑ वैष्णा - व॒रु॒णीम् । \newline
32. व॒शा मा व॒शां ॅव॒शा मा । \newline
33. आ ऽल॑भन्ता लभ॒न्ता ऽल॑भन्त । \newline
34. अ॒ल॒भ॒न्तै॒न्द्र मै॒न्द्र म॑लभन्ता लभन्तै॒न्द्रम् । \newline
35. ऐ॒न्द्र मु॒क्षाण॑ मु॒क्षाण॑ मै॒न्द्र मै॒न्द्र मु॒क्षाण᳚म् । \newline
36. उ॒क्षाण॒म् तम् त मु॒क्षाण॑ मु॒क्षाण॒म् तम् । \newline
37. तं ॅवरु॑णेन॒ वरु॑णेन॒ तम् तं ॅवरु॑णेन । \newline
38. वरु॑णे नै॒वैव वरु॑णेन॒ वरु॑णे नै॒व । \newline
39. ए॒व ग्रा॑हयि॒त्वा ग्रा॑हयि॒ त्वैवैव ग्रा॑हयि॒त्वा । \newline
40. ग्रा॒ह॒यि॒त्वा विष्णु॑ना॒ विष्णु॑ना ग्राहयि॒त्वा ग्रा॑हयि॒त्वा विष्णु॑ना । \newline
41. विष्णु॑ना य॒ज्ञेन॑ य॒ज्ञेन॒ विष्णु॑ना॒ विष्णु॑ना य॒ज्ञेन॑ । \newline
42. य॒ज्ञेन॒ प्र प्र य॒ज्ञेन॑ य॒ज्ञेन॒ प्र । \newline
43. प्राणु॑दन्ता नुदन्त॒ प्र प्राणु॑दन्त । \newline
44. अ॒नु॒द॒न्तै॒ न्द्रेणै॒न्द्रेणा॑ नुदन्ता नुदन्तै॒न्द्रेण॑ । \newline
45. ऐ॒न्द्रे णै॒वै वैन्द्रे णै॒न्द्रे णै॒व । \newline
46. ए॒वास्या᳚ स्यै॒ वैवास्य॑ । \newline
47. अ॒स्ये॒ न्द्रि॒य मि॑न्द्रि॒य म॑स्यास्ये न्द्रि॒यम् । \newline
48. इ॒न्द्रि॒य म॑वृञ्जता वृञ्जते न्द्रि॒य मि॑न्द्रि॒य म॑वृञ्जत । \newline
49. अ॒वृ॒ञ्ज॒त॒ यो यो॑ ऽवृञ्जता वृञ्जत॒ यः । \newline
50. यो भ्रातृ॑व्यवा॒न् भ्रातृ॑व्यवा॒न्॒. यो यो भ्रातृ॑व्यवान् । \newline
51. भ्रातृ॑व्यवा॒न् थ्स्याथ् स्याद् भ्रातृ॑व्यवा॒न् भ्रातृ॑व्यवा॒न् थ्स्यात् । \newline
52. भ्रातृ॑व्यवा॒निति॒ भ्रातृ॑व्य - वा॒न् । \newline
53. स्याथ् स स स्याथ् स्याथ् सः । \newline
54. स स्पर्द्ध॑मानः॒ स्पर्द्ध॑मानः॒ स स स्पर्द्ध॑मानः । \newline
55. स्पर्द्ध॑मानो वैष्णावरु॒णीं ॅवै᳚ष्णावरु॒णीꣳ स्पर्द्ध॑मानः॒ स्पर्द्ध॑मानो वैष्णावरु॒णीम् । \newline
56. वै॒ष्णा॒व॒रु॒णीं ॅव॒शां ॅव॒शां ॅवै᳚ष्णावरु॒णीं ॅवै᳚ष्णावरु॒णीं ॅव॒शाम् । \newline
57. वै॒ष्णा॒व॒रु॒णीमिति॑ वैष्णा - व॒रु॒णीम् । \newline

\textbf{Ghana Paata } \newline

1. य॒म॒त्वम् ते ते य॑म॒त्वं ॅय॑म॒त्वम् ते दे॒वा दे॒वा स्ते य॑म॒त्वं ॅय॑म॒त्वम् ते दे॒वाः । \newline
2. य॒म॒त्वमिति॑ यम - त्वम् । \newline
3. ते दे॒वा दे॒वा स्ते ते दे॒वा अ॑मन्यन्ता मन्यन्त दे॒वा स्ते ते दे॒वा अ॑मन्यन्त । \newline
4. दे॒वा अ॑मन्यन्ता मन्यन्त दे॒वा दे॒वा अ॑मन्यन्त य॒मो य॒मो॑ ऽमन्यन्त दे॒वा दे॒वा अ॑मन्यन्त य॒मः । \newline
5. अ॒म॒न्य॒न्त॒ य॒मो य॒मो॑ ऽमन्यन्ता मन्यन्त य॒मो वै वै य॒मो॑ ऽमन्यन्ता मन्यन्त य॒मो वै । \newline
6. य॒मो वै वै य॒मो य॒मो वा इ॒द मि॒दं ॅवै य॒मो य॒मो वा इ॒दम् । \newline
7. वा इ॒द मि॒दं ॅवै वा इ॒द म॑भू दभू दि॒दं ॅवै वा इ॒द म॑भूत् । \newline
8. इ॒द म॑भू दभू दि॒द मि॒द म॑भू॒द् यद् यद॑भू दि॒द मि॒द म॑भू॒द् यत् । \newline
9. अ॒भू॒द् यद् यद॑भू दभू॒द् यद् व॒यं ॅव॒यं ॅयद॑भू दभू॒द् यद् व॒यम् । \newline
10. यद् व॒यं ॅव॒यं ॅयद् यद् व॒यꣳ स्मः स्मो व॒यं ॅयद् यद् व॒यꣳ स्मः । \newline
11. व॒यꣳ स्मः स्मो व॒यं ॅव॒यꣳ स्म इतीति॒ स्मो व॒यं ॅव॒यꣳ स्म इति॑ । \newline
12. स्म इतीति॒ स्मः स्म इति॒ ते त इति॒ स्मः स्म इति॒ ते । \newline
13. इति॒ ते त इतीति॒ ते प्र॒जाप॑तिम् प्र॒जाप॑ति॒म् त इतीति॒ ते प्र॒जाप॑तिम् । \newline
14. ते प्र॒जाप॑तिम् प्र॒जाप॑ति॒म् ते ते प्र॒जाप॑ति॒ मुपोप॑ प्र॒जाप॑ति॒म् ते ते प्र॒जाप॑ति॒ मुप॑ । \newline
15. प्र॒जाप॑ति॒ मुपोप॑ प्र॒जाप॑तिम् प्र॒जाप॑ति॒ मुपा॑धावन् नधाव॒न् नुप॑ प्र॒जाप॑तिम् प्र॒जाप॑ति॒ मुपा॑धावन्न् । \newline
16. प्र॒जाप॑ति॒मिति॑ प्र॒जा - प॒ति॒म् । \newline
17. उपा॑धावन् नधाव॒न् नुपोपा॑धाव॒न् थ्स सो॑ ऽधाव॒न् नुपोपा॑धाव॒न् थ्सः । \newline
18. अ॒धा॒व॒न् थ्स सो॑ ऽधावन् नधाव॒न् थ्स ए॒ता वे॒तौ सो॑ ऽधावन् नधाव॒न् थ्स ए॒तौ । \newline
19. स ए॒ता वे॒तौ स स ए॒तौ प्र॒जाप॑तिः प्र॒जाप॑ति रे॒तौ स स ए॒तौ प्र॒जाप॑तिः । \newline
20. ए॒तौ प्र॒जाप॑तिः प्र॒जाप॑ति रे॒ता वे॒तौ प्र॒जाप॑ति रा॒त्मन॑ आ॒त्मनः॑ प्र॒जाप॑तिरे॒ता वे॒तौ प्र॒जाप॑ति रा॒त्मनः॑ । \newline
21. प्र॒जाप॑ति रा॒त्मन॑ आ॒त्मनः॑ प्र॒जाप॑तिः प्र॒जाप॑ति रा॒त्मन॑ उक्षव॒शा वु॑क्षव॒शा वा॒त्मनः॑ प्र॒जाप॑तिः प्र॒जाप॑ति रा॒त्मन॑ उक्षव॒शौ । \newline
22. प्र॒जाप॑ति॒रिति॑ प्र॒जा - प॒तिः॒ । \newline
23. आ॒त्मन॑ उक्षव॒शा वु॑क्षव॒शा वा॒त्मन॑ आ॒त्मन॑ उक्षव॒शौ निर् णिरु॑क्षव॒शा वा॒त्मन॑ आ॒त्मन॑ उक्षव॒शौ निः । \newline
24. उ॒क्ष॒व॒शौ निर् णिरु॑क्षव॒शा वु॑क्षव॒शौ निर॑मिमीता मिमीत॒ निरु॑क्षव॒शा वु॑क्षव॒शौ निर॑मिमीत । \newline
25. उ॒क्ष॒व॒शावित्यु॑क्ष - व॒शौ । \newline
26. निर॑मिमीता मिमीत॒ निर् णिर॑मिमीत॒ ते ते॑ ऽमिमीत॒ निर् णिर॑मिमीत॒ ते । \newline
27. अ॒मि॒मी॒त॒ ते ते॑ ऽमिमीता मिमीत॒ ते दे॒वा दे॒वा स्ते॑ ऽमिमीता मिमीत॒ ते दे॒वाः । \newline
28. ते दे॒वा दे॒वा स्ते ते दे॒वा वै᳚ष्णावरु॒णीं ॅवै᳚ष्णावरु॒णीम् दे॒वा स्ते ते दे॒वा वै᳚ष्णावरु॒णीम् । \newline
29. दे॒वा वै᳚ष्णावरु॒णीं ॅवै᳚ष्णावरु॒णीम् दे॒वा दे॒वा वै᳚ष्णावरु॒णीं ॅव॒शां ॅव॒शां ॅवै᳚ष्णावरु॒णीम् दे॒वा दे॒वा वै᳚ष्णावरु॒णीं ॅव॒शाम् । \newline
30. वै॒ष्णा॒व॒रु॒णीं ॅव॒शां ॅव॒शां ॅवै᳚ष्णावरु॒णीं ॅवै᳚ष्णावरु॒णीं ॅव॒शा मा व॒शां ॅवै᳚ष्णावरु॒णीं ॅवै᳚ष्णावरु॒णीं ॅव॒शा मा । \newline
31. वै॒ष्णा॒व॒रु॒णीमिति॑ वैष्णा - व॒रु॒णीम् । \newline
32. व॒शा मा व॒शां ॅव॒शा मा ऽल॑भन्ता लभ॒न्ता व॒शां ॅव॒शा मा ऽल॑भन्त । \newline
33. आ ऽल॑भन्ता लभ॒न्ता ऽल॑भन्तै॒न्द्र मै॒न्द्र म॑लभ॒न्ता ऽल॑भन्तै॒न्द्रम् । \newline
34. अ॒ल॒भ॒ न्तै॒न्द्र मै॒न्द्र म॑लभन्ता लभन्तै॒न्द्र मु॒क्षाण॑ मु॒क्षाण॑ मै॒न्द्र म॑लभन्ता लभन्तै॒न्द्र मु॒क्षाण᳚म् । \newline
35. ऐ॒न्द्र मु॒क्षाण॑ मु॒क्षाण॑ मै॒न्द्र मै॒न्द्र मु॒क्षाण॒म् तम् त मु॒क्षाण॑ मै॒न्द्र मै॒न्द्र मु॒क्षाण॒म् तम् । \newline
36. उ॒क्षाण॒म् तम् त मु॒क्षाण॑ मु॒क्षाण॒म् तं ॅवरु॑णेन॒ वरु॑णेन॒ त मु॒क्षाण॑ मु॒क्षाण॒म् तं ॅवरु॑णेन । \newline
37. तं ॅवरु॑णेन॒ वरु॑णेन॒ तम् तं ॅवरु॑णेनै॒वैव वरु॑णेन॒ तम् तं ॅवरु॑णेनै॒व । \newline
38. वरु॑णेनै॒वैव वरु॑णेन॒ वरु॑णेनै॒व ग्रा॑हयि॒त्वा ग्रा॑हयि॒त्वैव वरु॑णेन॒ वरु॑णेनै॒व ग्रा॑हयि॒त्वा । \newline
39. ए॒व ग्रा॑हयि॒त्वा ग्रा॑हयि॒त्वैवैव ग्रा॑हयि॒त्वा विष्णु॑ना॒ विष्णु॑ना ग्राहयि॒त्वैवैव ग्रा॑हयि॒त्वा विष्णु॑ना । \newline
40. ग्रा॒ह॒यि॒त्वा विष्णु॑ना॒ विष्णु॑ना ग्राहयि॒त्वा ग्रा॑हयि॒त्वा विष्णु॑ना य॒ज्ञेन॑ य॒ज्ञेन॒ विष्णु॑ना ग्राहयि॒त्वा ग्रा॑हयि॒त्वा विष्णु॑ना य॒ज्ञेन॑ । \newline
41. विष्णु॑ना य॒ज्ञेन॑ य॒ज्ञेन॒ विष्णु॑ना॒ विष्णु॑ना य॒ज्ञेन॒ प्र प्र य॒ज्ञेन॒ विष्णु॑ना॒ विष्णु॑ना य॒ज्ञेन॒ प्र । \newline
42. य॒ज्ञेन॒ प्र प्र य॒ज्ञेन॑ य॒ज्ञेन॒ प्राणु॑दन्ता नुदन्त॒ प्र य॒ज्ञेन॑ य॒ज्ञेन॒ प्राणु॑दन्त । \newline
43. प्राणु॑दन्ता नुदन्त॒ प्र प्राणु॑दन्तै॒न्द्रे णै॒न्द्रेणा॑ नुदन्त॒ प्र प्राणु॑दन्तै॒न्द्रेण॑ । \newline
44. अ॒नु॒द॒ न्तै॒न्द्रेणै॒ न्द्रेणा॑ नुदन्ता नुद न्तै॒न्द्रेणै॒ वैवैन्द्रेणा॑ नुदन्ता नुद न्तै॒न्द्रेणै॒व । \newline
45. ऐ॒न्द्रेणै॒ वैवैन्द्रेणै॒ न्द्रेणै॒वास्या᳚ स्यै॒वैन्द्रेणै॒ न्द्रेणै॒वास्य॑ । \newline
46. ए॒वास्या᳚ स्यै॒वैवास्ये᳚ न्द्रि॒य मि॑न्द्रि॒य म॑स्यै॒वैवास्ये᳚ न्द्रि॒यम् । \newline
47. अ॒स्ये॒ न्द्रि॒य मि॑न्द्रि॒य म॑स्यास्ये न्द्रि॒य म॑वृञ्जता वृञ्जते न्द्रि॒य म॑स्यास्ये न्द्रि॒य म॑वृञ्जत । \newline
48. इ॒न्द्रि॒य म॑वृञ्जता वृञ्जते न्द्रि॒य मि॑न्द्रि॒य म॑वृञ्जत॒ यो यो॑ ऽवृञ्जते न्द्रि॒य मि॑न्द्रि॒य म॑वृञ्जत॒ यः । \newline
49. अ॒वृ॒ञ्ज॒त॒ यो यो॑ ऽवृञ्जता वृञ्जत॒ यो भ्रातृ॑व्यवा॒न् भ्रातृ॑व्यवा॒न्॒. यो॑ ऽवृञ्जता वृञ्जत॒ यो भ्रातृ॑व्यवान् । \newline
50. यो भ्रातृ॑व्यवा॒न् भ्रातृ॑व्यवा॒न्॒. यो यो भ्रातृ॑व्यवा॒न् थ्स्याथ् स्याद् भ्रातृ॑व्यवा॒न्॒. यो यो भ्रातृ॑व्यवा॒न् थ्स्यात् । \newline
51. भ्रातृ॑व्यवा॒न् थ्स्याथ् स्याद् भ्रातृ॑व्यवा॒न् भ्रातृ॑व्यवा॒न् थ्स्याथ् स स स्याद् भ्रातृ॑व्यवा॒न् भ्रातृ॑व्यवा॒न् थ्स्याथ् सः । \newline
52. भ्रातृ॑व्यवा॒निति॒ भ्रातृ॑व्य - वा॒न् । \newline
53. स्याथ् स स स्याथ् स्याथ् स स्पर्द्ध॑मानः॒ स्पर्द्ध॑मानः॒ स स्याथ् स्याथ् स स्पर्द्ध॑मानः । \newline
54. स स्पर्द्ध॑मानः॒ स्पर्द्ध॑मानः॒ स स स्पर्द्ध॑मानो वैष्णावरु॒णीं ॅवै᳚ष्णावरु॒णीꣳ स्पर्द्ध॑मानः॒ स स स्पर्द्ध॑मानो वैष्णावरु॒णीम् । \newline
55. स्पर्द्ध॑मानो वैष्णावरु॒णीं ॅवै᳚ष्णावरु॒णीꣳ स्पर्द्ध॑मानः॒ स्पर्द्ध॑मानो वैष्णावरु॒णीं ॅव॒शां ॅव॒शां ॅवै᳚ष्णावरु॒णीꣳ स्पर्द्ध॑मानः॒ स्पर्द्ध॑मानो वैष्णावरु॒णीं ॅव॒शाम् । \newline
56. वै॒ष्णा॒व॒रु॒णीं ॅव॒शां ॅव॒शां ॅवै᳚ष्णावरु॒णीं ॅवै᳚ष्णावरु॒णीं ॅव॒शा मा व॒शां ॅवै᳚ष्णावरु॒णीं ॅवै᳚ष्णावरु॒णीं ॅव॒शा मा । \newline
57. वै॒ष्णा॒व॒रु॒णीमिति॑ वैष्णा - व॒रु॒णीम् । \newline
\pagebreak
\markright{ TS 2.1.4.5  \hfill https://www.vedavms.in \hfill}

\section{ TS 2.1.4.5 }

\textbf{TS 2.1.4.5 } \newline
\textbf{Samhita Paata} \newline

ॅव॒शामा ल॑भेतै॒न्द्रमु॒क्षाणं॒ ॅवरु॑णेनै॒व भ्रातृ॑व्यं ग्राहयि॒त्वा विष्णु॑ना य॒ज्ञेन॒ प्रणु॑दत ऐ॒न्द्रेणै॒वास्ये᳚न्द्रि॒यं ॅवृ॑ङ्क्ते॒ भव॑त्या॒त्मना॒ परा᳚स्य॒ भ्रातृ॑व्यो भव॒तीन्द्रो॑ वृ॒त्रम॑ह॒न् तं ॅवृ॒त्रो ह॒तः षो॑ड॒शभि॑-र्भो॒गैर॑सिना॒त् तस्य॑ वृ॒त्रस्य॑ शीर्.ष॒तो गाव॒ उदा॑य॒न् ता वै॑दे॒ह्यो॑ऽभव॒न् तासा॑मृष॒भो ज॒घनेऽनूदै॒त् तमिन्द्रो॑ - [  ] \newline

\textbf{Pada Paata} \newline

व॒शाम् । एति॑ । ल॒भे॒त॒ । ऐ॒न्द्रम् । उ॒क्षाण᳚म् । वरु॑णेन । ए॒व । भ्रातृ॑व्यम् । ग्रा॒ह॒यि॒त्वा । विष्णु॑ना । य॒ज्ञेन॑ । प्रेति॑ । नु॒द॒ते॒ । ऐ॒न्द्रेण॑ । ए॒व । अ॒स्य॒ । इ॒न्द्रि॒यम् । वृ॒ङ्क्ते॒ । भव॑ति । आ॒त्मना᳚ । परेति॑ । अ॒स्य॒ । भ्रातृ॑व्यः । भ॒व॒ति॒ । इन्द्रः॑ । वृ॒त्रम् । अ॒ह॒न्न् । तम् । वृ॒त्रः । ह॒तः । षो॒ड॒शभि॒रिति॑ षोड॒श - भिः॒ । भो॒गैः । अ॒सि॒ना॒त् । तस्य॑ । वृ॒त्रस्य॑ । शी॒र्.॒ष॒तः । गावः॑ । उदिति॑ । आ॒य॒न्न् । ताः । वै॒दे॒ह्यः॑ । अ॒भ॒व॒न्न् । तासा᳚म् । ऋ॒ष॒भः । ज॒घने᳚ । अनु॑ । उदिति॑ । ऐ॒त् । तम् । इन्द्रः॑ ।  \newline


\textbf{Krama Paata} \newline

व॒शामा । आ ल॑भेत । ल॒भे॒तै॒न्द्रम् । ऐ॒न्द्रमु॒क्षाण᳚म् । उ॒क्षाणं॒ ॅवरु॑णेन । वरु॑णेनै॒व । ए॒व भ्रातृ॑व्यम् । भ्रातृ॑व्यं ग्राहयि॒त्वा । ग्रा॒॒ह॒यि॒त्वा विष्णु॑ना । विष्णु॑ना य॒ज्ञेन॑ । य॒ज्ञेन॒ प्र । प्र णु॑दते । नु॒द॒त॒ ऐ॒न्द्रेण॑ । ऐ॒न्द्रेणै॒व । ए॒वास्य॑ । अ॒स्ये॒न्द्रि॒यम् । इ॒न्द्रि॒यं ॅवृ॑ङ्क्ते । वृ॒ङ्क्ते॒ भव॑ति । भव॑त्या॒त्मना᳚ । आ॒त्मना॒ परा᳚ । परा᳚ऽस्य । अ॒स्य॒ भ्रातृ॑व्यः । भ्रातृ॑व्यो भवति । भ॒व॒तीन्द्रः॑ । इन्द्रो॑ वृ॒त्रम् । वृ॒त्रम॑हन्न् । अ॒ह॒न् तम् । तं ॅवृ॒त्रः । वृ॒त्रो ह॒तः । ह॒तः षो॑ड॒शभिः॑ । षो॒ड॒शभि॑र् भो॒गैः । षो॒ड॒शभि॒रिति॑ षोड॒श - भिः॒ । भो॒गैर॑सिनात् । अ॒सि॒ना॒त् तस्य॑ । तस्य॑ वृ॒त्रस्य॑ । वृ॒त्रस्य॑ शीर्.ष॒तः । शी॒र्॒.॒ष॒तो गावः॑ । गाव॒ उत् । उदा॑यन्न् । आ॒य॒न् ताः । ता वै॑दे॒ह्यः॑ । वै॒दे॒ह्यो॑ ऽभवन्न् । अ॒भ॒व॒न् तासा᳚म् । तासा॑मृष॒भः । ऋ॒ष॒भो ज॒घने᳚ । ज॒घनेऽनु॑ । अनूत् । उदै᳚त् । ऐ॒त् तम् । तमिन्द्रः॑ । इन्द्रो॑ ऽचायत् \newline

\textbf{Jatai Paata} \newline

1. व॒शा मा व॒शां ॅव॒शा मा । \newline
2. आ ल॑भेत लभे॒ता ल॑भेत । \newline
3. ल॒भे॒तै॒न्द्र मै॒न्द्रम् ॅल॑भेत लभेतै॒न्द्रम् । \newline
4. ऐ॒न्द्र मु॒क्षाण॑ मु॒क्षाण॑ मै॒न्द्र मै॒न्द्र मु॒क्षाण᳚म् । \newline
5. उ॒क्षाणं॒ ॅवरु॑णेन॒ वरु॑णे नो॒क्षाण॑ मु॒क्षाणं॒ ॅवरु॑णेन । \newline
6. वरु॑णे नै॒वैव वरु॑णेन॒ वरु॑णे नै॒व । \newline
7. ए॒व भ्रातृ॑व्य॒म् भ्रातृ॑व्य मे॒वैव भ्रातृ॑व्यम् । \newline
8. भ्रातृ॑व्यम् ग्राहयि॒त्वा ग्रा॑हयि॒त्वा भ्रातृ॑व्य॒म् भ्रातृ॑व्यम् ग्राहयि॒त्वा । \newline
9. ग्रा॒ह॒यि॒त्वा विष्णु॑ना॒ विष्णु॑ना ग्राहयि॒त्वा ग्रा॑हयि॒त्वा विष्णु॑ना । \newline
10. विष्णु॑ना य॒ज्ञेन॑ य॒ज्ञेन॒ विष्णु॑ना॒ विष्णु॑ना य॒ज्ञेन॑ । \newline
11. य॒ज्ञेन॒ प्र प्र य॒ज्ञेन॑ य॒ज्ञेन॒ प्र । \newline
12. प्र णु॑दते नुदते॒ प्र प्र णु॑दते । \newline
13. नु॒द॒त॒ ऐ॒न्द्रे णै॒न्द्रेण॑ नुदते नुदत ऐ॒न्द्रेण॑ । \newline
14. ऐ॒न्द्रे णै॒वै वैन्द्रे णै॒न्द्रे णै॒व । \newline
15. ए॒वास्या᳚ स्यै॒ वैवास्य॑ । \newline
16. अ॒स्ये॒ न्द्रि॒य मि॑न्द्रि॒य म॑स्यास्ये न्द्रि॒यम् । \newline
17. इ॒न्द्रि॒यं ॅवृ॑ङ्क्ते वृङ्क्त इन्द्रि॒य मि॑न्द्रि॒यं ॅवृ॑ङ्क्ते । \newline
18. वृ॒ङ्क्ते॒ भव॑ति॒ भव॑ति वृङ्क्ते वृङ्क्ते॒ भव॑ति । \newline
19. भव॑त्या॒त्मना॒ ऽऽत्मना॒ भव॑ति॒ भव॑त्या॒त्मना᳚ । \newline
20. आ॒त्मना॒ परा॒ परा॒ ऽऽत्मना॒ ऽऽत्मना॒ परा᳚ । \newline
21. परा᳚ ऽस्या स्य॒ परा॒ परा᳚ ऽस्य । \newline
22. अ॒स्य॒ भ्रातृ॑व्यो॒ भ्रातृ॑व्यो ऽस्यास्य॒ भ्रातृ॑व्यः । \newline
23. भ्रातृ॑व्यो भवति भवति॒ भ्रातृ॑व्यो॒ भ्रातृ॑व्यो भवति । \newline
24. भ॒व॒तीन्द्र॒ इन्द्रो॑ भवति भव॒तीन्द्रः॑ । \newline
25. इन्द्रो॑ वृ॒त्रं ॅवृ॒त्र मिन्द्र॒ इन्द्रो॑ वृ॒त्रम् । \newline
26. वृ॒त्र म॑हन् नहन् वृ॒त्रं ॅवृ॒त्र म॑हन्न् । \newline
27. अ॒ह॒न् तम् त म॑हन् नह॒न् तम् । \newline
28. तं ॅवृ॒त्रो वृ॒त्र स्तम् तं ॅवृ॒त्रः । \newline
29. वृ॒त्रो ह॒तो ह॒तो वृ॒त्रो वृ॒त्रो ह॒तः । \newline
30. ह॒त ष्षो॑ड॒शभि॑ ष्षोड॒शभि॑र्. ह॒तो ह॒त ष्षो॑ड॒शभिः॑ । \newline
31. षो॒ड॒शभि॑र् भो॒गैर् भो॒गै ष्षो॑ड॒शभि॑ ष्षोड॒शभि॑र् भो॒गैः । \newline
32. षो॒ड॒शभि॒रिति॑ षोड॒श - भिः॒ । \newline
33. भो॒गै र॑सिना दसिनाद् भो॒गैर् भो॒गै र॑सिनात् । \newline
34. अ॒सि॒ना॒त् तस्य॒ तस्या॑ सिना दसिना॒त् तस्य॑ । \newline
35. तस्य॑ वृ॒त्रस्य॑ वृ॒त्रस्य॒ तस्य॒ तस्य॑ वृ॒त्रस्य॑ । \newline
36. वृ॒त्रस्य॑ शीर्.ष॒तः शी॑र्.ष॒तो वृ॒त्रस्य॑ वृ॒त्रस्य॑ शीर्.ष॒तः । \newline
37. शी॒र्॒.ष॒तो गावो॒ गावः॑ शीर्.ष॒तः शी॑र्.ष॒तो गावः॑ । \newline
38. गाव॒ उदुद् गावो॒ गाव॒ उत् । \newline
39. उदा॑यन् नाय॒न् नुदु दा॑यन्न् । \newline
40. आ॒य॒न् तास्ता आ॑यन् नाय॒न् ताः । \newline
41. ता वै॑दे॒ह्यो॑ वैदे॒ह्य॑ स्ता स्ता वै॑दे॒ह्यः॑ । \newline
42. वै॒दे॒ह्यो॑ ऽभवन् नभवन्. वैदे॒ह्यो॑ वैदे॒ह्यो॑ ऽभवन्न् । \newline
43. अ॒भ॒व॒न् तासा॒म् तासा॑ मभवन् नभव॒न् तासा᳚म् । \newline
44. तासा॑ मृष॒भ ऋ॑ष॒भ स्तासा॒म् तासा॑ मृष॒भः । \newline
45. ऋ॒ष॒भो ज॒घने॑ ज॒घन॑ ऋष॒भ ऋ॑ष॒भो ज॒घने᳚ । \newline
46. ज॒घने ऽन्वनु॑ ज॒घने॑ ज॒घने ऽनु॑ । \newline
47. अनू दु दन्व नूत् । \newline
48. उदै॑ दै॒दुदु दै᳚त् । \newline
49. ऐ॒त् तम् त मै॑दै॒त् तम् । \newline
50. त मिन्द्र॒ इन्द्र॒ स्तम् त मिन्द्रः॑ । \newline
51. इन्द्रो॑ ऽचाय दचाय॒ दिन्द्र॒ इन्द्रो॑ ऽचायत् । \newline

\textbf{Ghana Paata } \newline

1. व॒शा मा व॒शां ॅव॒शा मा ल॑भेत लभे॒ता व॒शां ॅव॒शा मा ल॑भेत । \newline
2. आ ल॑भेत लभे॒ता ल॑भेतै॒न्द्र मै॒न्द्रम् ॅल॑भे॒ता ल॑भेतै॒न्द्रम् । \newline
3. ल॒भे॒तै॒न्द्र मै॒न्द्रम् ॅल॑भेत लभेतै॒न्द्र मु॒क्षाण॑ मु॒क्षाण॑ मै॒न्द्रम् ॅल॑भेत लभेतै॒न्द्र मु॒क्षाण᳚म् । \newline
4. ऐ॒न्द्र मु॒क्षाण॑ मु॒क्षाण॑ मै॒न्द्र मै॒न्द्र मु॒क्षाणं॒ ॅवरु॑णेन॒ वरु॑णेनो॒ क्षाण॑ मै॒न्द्र मै॒न्द्र मु॒क्षाणं॒ ॅवरु॑णेन । \newline
5. उ॒क्षाणं॒ ॅवरु॑णेन॒ वरु॑णेनो॒ क्षाण॑ मु॒क्षाणं॒ ॅवरु॑णेनै॒वैव वरु॑णेनो॒ क्षाण॑ मु॒क्षाणं॒ ॅवरु॑णेनै॒व । \newline
6. वरु॑णेनै॒वैव वरु॑णेन॒ वरु॑णेनै॒व भ्रातृ॑व्य॒म् भ्रातृ॑व्य मे॒व वरु॑णेन॒ वरु॑णेनै॒व भ्रातृ॑व्यम् । \newline
7. ए॒व भ्रातृ॑व्य॒म् भ्रातृ॑व्य मे॒वैव भ्रातृ॑व्यम् ग्राहयि॒त्वा ग्रा॑हयि॒त्वा भ्रातृ॑व्य मे॒वैव भ्रातृ॑व्यम् ग्राहयि॒त्वा । \newline
8. भ्रातृ॑व्यम् ग्राहयि॒त्वा ग्रा॑हयि॒त्वा भ्रातृ॑व्य॒म् भ्रातृ॑व्यम् ग्राहयि॒त्वा विष्णु॑ना॒ विष्णु॑ना ग्राहयि॒त्वा भ्रातृ॑व्य॒म् भ्रातृ॑व्यम् ग्राहयि॒त्वा विष्णु॑ना । \newline
9. ग्रा॒ह॒यि॒त्वा विष्णु॑ना॒ विष्णु॑ना ग्राहयि॒त्वा ग्रा॑हयि॒त्वा विष्णु॑ना य॒ज्ञेन॑ य॒ज्ञेन॒ विष्णु॑ना ग्राहयि॒त्वा ग्रा॑हयि॒त्वा विष्णु॑ना य॒ज्ञेन॑ । \newline
10. विष्णु॑ना य॒ज्ञेन॑ य॒ज्ञेन॒ विष्णु॑ना॒ विष्णु॑ना य॒ज्ञेन॒ प्र प्र य॒ज्ञेन॒ विष्णु॑ना॒ विष्णु॑ना य॒ज्ञेन॒ प्र । \newline
11. य॒ज्ञेन॒ प्र प्र य॒ज्ञेन॑ य॒ज्ञेन॒ प्र णु॑दते नुदते॒ प्र य॒ज्ञेन॑ य॒ज्ञेन॒ प्र णु॑दते । \newline
12. प्र णु॑दते नुदते॒ प्र प्र णु॑दत ऐ॒न्द्रे णै॒न्द्रेण॑ नुदते॒ प्र प्र णु॑दत ऐ॒न्द्रेण॑ । \newline
13. नु॒द॒त॒ ऐ॒न्द्रे णै॒न्द्रेण॑ नुदते नुदत ऐ॒न्द्रे णै॒वैवै न्द्रेण॑ नुदते नुदत ऐ॒न्द्रेणै॒व । \newline
14. ऐ॒न्द्रे णै॒वैवैन्द्रे णै॒न्द्रे णै॒वास्या᳚ स्यै॒वैन्द्रे णै॒न्द्रे णै॒वास्य॑ । \newline
15. ए॒वास्या᳚ स्यै॒वैवास्ये᳚ न्द्रि॒य मि॑न्द्रि॒य म॑स्यै॒ वैवास्ये᳚ न्द्रि॒यम् । \newline
16. अ॒स्ये॒ न्द्रि॒य मि॑न्द्रि॒य म॑स्यास्ये न्द्रि॒यं ॅवृ॑ङ्क्ते वृङ्क्त इन्द्रि॒य म॑स्यास्ये न्द्रि॒यं ॅवृ॑ङ्क्ते । \newline
17. इ॒न्द्रि॒यं ॅवृ॑ङ्क्ते वृङ्क्त इन्द्रि॒य मि॑न्द्रि॒यं ॅवृ॑ङ्क्ते॒ भव॑ति॒ भव॑ति वृङ्क्त इन्द्रि॒य मि॑न्द्रि॒यं ॅवृ॑ङ्क्ते॒ भव॑ति । \newline
18. वृ॒ङ्क्ते॒ भव॑ति॒ भव॑ति वृङ्क्ते वृङ्क्ते॒ भव॑ त्या॒त्मना॒ ऽऽत्मना॒ भव॑ति वृङ्क्ते वृङ्क्ते॒ भव॑ त्या॒त्मना᳚ । \newline
19. भव॑त्या॒त्मना॒ ऽऽत्मना॒ भव॑ति॒ भव॑ त्या॒त्मना॒ परा॒ परा॒ ऽऽत्मना॒ भव॑ति॒ भव॑त्या॒त्मना॒ परा᳚ । \newline
20. आ॒त्मना॒ परा॒ परा॒ ऽऽत्मना॒ ऽऽत्मना॒ परा᳚ ऽस्यास्य॒ परा॒ ऽऽत्मना॒ ऽऽत्मना॒ परा᳚ ऽस्य । \newline
21. परा᳚ ऽस्यास्य॒ परा॒ परा᳚ ऽस्य॒ भ्रातृ॑व्यो॒ भ्रातृ॑व्यो ऽस्य॒ परा॒ परा᳚ ऽस्य॒ भ्रातृ॑व्यः । \newline
22. अ॒स्य॒ भ्रातृ॑व्यो॒ भ्रातृ॑व्यो ऽस्यास्य॒ भ्रातृ॑व्यो भवति भवति॒ भ्रातृ॑व्यो ऽस्यास्य॒ भ्रातृ॑व्यो भवति । \newline
23. भ्रातृ॑व्यो भवति भवति॒ भ्रातृ॑व्यो॒ भ्रातृ॑व्यो भव॒तीन्द्र॒ इन्द्रो॑ भवति॒ भ्रातृ॑व्यो॒ भ्रातृ॑व्यो भव॒तीन्द्रः॑ । \newline
24. भ॒व॒तीन्द्र॒ इन्द्रो॑ भवति भव॒तीन्द्रो॑ वृ॒त्रं ॅवृ॒त्र मिन्द्रो॑ भवति भव॒तीन्द्रो॑ वृ॒त्रम् । \newline
25. इन्द्रो॑ वृ॒त्रं ॅवृ॒त्र मिन्द्र॒ इन्द्रो॑ वृ॒त्र म॑हन् नहन् वृ॒त्र मिन्द्र॒ इन्द्रो॑ वृ॒त्र म॑हन्न् । \newline
26. वृ॒त्र म॑हन् नहन् वृ॒त्रं ॅवृ॒त्र म॑ह॒न् तम् त म॑हन् वृ॒त्रं ॅवृ॒त्र म॑ह॒न् तम् । \newline
27. अ॒ह॒न् तम् त म॑हन् नह॒न् तं ॅवृ॒त्रो वृ॒त्र स्त म॑हन् नह॒न् तं ॅवृ॒त्रः । \newline
28. तं ॅवृ॒त्रो वृ॒त्र स्तम् तं ॅवृ॒त्रो ह॒तो ह॒तो वृ॒त्र स्तम् तं ॅवृ॒त्रो ह॒तः । \newline
29. वृ॒त्रो ह॒तो ह॒तो वृ॒त्रो वृ॒त्रो ह॒त ष्षो॑ड॒शभि॑ ष्षोड॒शभि॑र्. ह॒तो वृ॒त्रो वृ॒त्रो ह॒त ष्षो॑ड॒शभिः॑ । \newline
30. ह॒त ष्षो॑ड॒शभि॑ ष्षोड॒शभि॑र्. ह॒तो ह॒त ष्षो॑ड॒शभि॑र् भो॒गैर् भो॒गै ष्षो॑ड॒शभि॑र्. ह॒तो ह॒त ष्षो॑ड॒शभि॑र् भो॒गैः । \newline
31. षो॒ड॒शभि॑र् भो॒गैर् भो॒गै ष्षो॑ड॒शभि॑ ष्षोड॒शभि॑र् भो॒गै र॑सिना दसिनाद् भो॒गै ष्षो॑ड॒शभि॑ ष्षोड॒शभि॑र् भो॒गै र॑सिनात् । \newline
32. षो॒ड॒शभि॒रिति॑ षोड॒श - भिः॒ । \newline
33. भो॒गै र॑सिना दसिनाद् भो॒गैर् भो॒गै र॑सिना॒त् तस्य॒ तस्या॑सिनाद् भो॒गैर् भो॒गै र॑सिना॒त् तस्य॑ । \newline
34. अ॒सि॒ना॒त् तस्य॒ तस्या॑ सिना दसिना॒त् तस्य॑ वृ॒त्रस्य॑ वृ॒त्रस्य॒ तस्या॑ सिना दसिना॒त् तस्य॑ वृ॒त्रस्य॑ । \newline
35. तस्य॑ वृ॒त्रस्य॑ वृ॒त्रस्य॒ तस्य॒ तस्य॑ वृ॒त्रस्य॑ शीर्.ष॒तः शी॑र्.ष॒तो वृ॒त्रस्य॒ तस्य॒ तस्य॑ वृ॒त्रस्य॑ शीर्.ष॒तः । \newline
36. वृ॒त्रस्य॑ शीर्.ष॒तः शी॑र्.ष॒तो वृ॒त्रस्य॑ वृ॒त्रस्य॑ शीर्.ष॒तो गावो॒ गावः॑ शीर्.ष॒तो वृ॒त्रस्य॑ वृ॒त्रस्य॑ शीर्.ष॒तो गावः॑ । \newline
37. शी॒र्॒.ष॒तो गावो॒ गावः॑ शीर्.ष॒तः शी॑र्.ष॒तो गाव॒ उदुद् गावः॑ शीर्.ष॒तः शी॑र्.ष॒तो गाव॒ उत् । \newline
38. गाव॒ उदुद् गावो॒ गाव॒ उदा॑यन् नाय॒न् नुद् गावो॒ गाव॒ उदा॑यन्न् । \newline
39. उदा॑यन् नाय॒न् नुदु दा॑य॒न् ता स्ता आ॑य॒न् नुदु दा॑य॒न् ताः । \newline
40. आ॒य॒न् ता स्ता आ॑यन् नाय॒न् ता वै॑दे॒ह्यो॑ वैदे॒ह्य॑ स्ता आ॑यन् नाय॒न् ता वै॑दे॒ह्यः॑ । \newline
41. ता वै॑दे॒ह्यो॑ वैदे॒ह्य॑ स्ता स्ता वै॑दे॒ह्यो॑ ऽभवन् नभवन्. वैदे॒ह्य॑ स्ता स्ता वै॑दे॒ह्यो॑ ऽभवन्न् । \newline
42. वै॒दे॒ह्यो॑ ऽभवन् नभवन्. वैदे॒ह्यो॑ वैदे॒ह्यो॑ ऽभव॒न् तासा॒म् तासा॑ मभवन्. वैदे॒ह्यो॑ वैदे॒ह्यो॑ ऽभव॒न् तासा᳚म् । \newline
43. अ॒भ॒व॒न् तासा॒म् तासा॑ मभवन् नभव॒न् तासा॑ मृष॒भ ऋ॑ष॒भ स्तासा॑ मभवन् नभव॒न् तासा॑ मृष॒भः । \newline
44. तासा॑ मृष॒भ ऋ॑ष॒भ स्तासा॒म् तासा॑ मृष॒भो ज॒घने॑ ज॒घन॑ ऋष॒भ स्तासा॒म् तासा॑ मृष॒भो ज॒घने᳚ । \newline
45. ऋ॒ष॒भो ज॒घने॑ ज॒घन॑ ऋष॒भ ऋ॑ष॒भो ज॒घने ऽन्वनु॑ ज॒घन॑ ऋष॒भ ऋ॑ष॒भो ज॒घने ऽनु॑ । \newline
46. ज॒घने ऽन्वनु॑ ज॒घने॑ ज॒घने ऽनूदु दनु॑ ज॒घने॑ ज॒घने ऽनूत् । \newline
47. अनू दुदन्व नूदै॑दै॒ दुद न्व नूदै᳚त् । \newline
48. उदै॑ दै॒ दुदु दै॒त् तम् त मै॒ दुदु दै॒त् तम् । \newline
49. ऐ॒त् तम् त मै॑दै॒त् त मिन्द्र॒ इन्द्र॒ स्त मै॑दै॒त् त मिन्द्रः॑ । \newline
50. त मिन्द्र॒ इन्द्र॒ स्तम् त मिन्द्रो॑ ऽचाय दचाय॒ दिन्द्र॒ स्तम् त मिन्द्रो॑ ऽचायत् । \newline
51. इन्द्रो॑ ऽचाय दचाय॒ दिन्द्र॒ इन्द्रो॑ ऽचाय॒थ् स सो॑ ऽचाय॒ दिन्द्र॒ इन्द्रो॑ ऽचाय॒थ् सः । \newline
\pagebreak
\markright{ TS 2.1.4.6  \hfill https://www.vedavms.in \hfill}

\section{ TS 2.1.4.6 }

\textbf{TS 2.1.4.6 } \newline
\textbf{Samhita Paata} \newline

ऽचाय॒थ् सो॑ऽमन्यत॒ यो वा इ॒ममा॒लभे॑त॒ मुच्ये॑ता॒स्मात् पा॒प्मन॒ इति॒ स आ᳚ग्ने॒यं कृ॒ष्णग्री॑व॒मा ल॑भतै॒न्द्रमृ॑ष॒भं तस्या॒ग्निरे॒व स्वेन॑ भाग॒धेये॒नोप॑ सृतः षोडश॒धा वृ॒त्रस्य॑ भो॒गानप्य॑दहदै॒न्द्रेणे᳚न्द्रि॒य- मा॒त्मन्न॑धत्त॒ यः पा॒प्मना॑ गृही॒तः स्याथ् स आ᳚ग्ने॒यं कृ॒ष्णग्री॑व॒मा ल॑भेतै॒न्द्रमृ॑ष॒भ-म॒ग्निरे॒वास्य॒ स्वेन॑ भाग॒धेये॒नोप॑सृतः - [  ] \newline

\textbf{Pada Paata} \newline

अ॒चा॒य॒त् । सः । अ॒म॒न्य॒त॒ । यः । वै । इ॒मम् । आ॒लभे॒तेत्या᳚ - लभे॑त । मुच्ये॑त । अ॒स्मात् । पा॒प्मनः॑ । इति॑ । सः । आ॒ग्ने॒यम् । कृ॒ष्णग्री॑व॒मिति॑ कृ॒ष्ण - ग्री॒व॒म् । एति॑ । अ॒ल॒भ॒त॒ । ऐ॒न्द्रम् । ऋ॒ष॒भम् । तस्य॑ । अ॒ग्निः । ए॒व । स्वेन॑ । भा॒ग॒धेये॒नेति॑ भाग - धेये॑न । उप॑सृत॒ इत्युप॑ - सृ॒तः॒ । षो॒ड॒श॒धेति॑ षोडश-धा । वृ॒त्रस्य॑ । भो॒गान् । अपीति॑ । अ॒द॒ह॒त् । ऐ॒न्द्रेण॑ । इ॒न्द्रि॒यम् । आ॒त्मन्न् । अ॒ध॒त्त॒ । यः । पा॒प्मना᳚ । गृ॒ही॒तः । स्यात् । सः । आ॒ग्ने॒यम् । कृ॒ष्णग्री॑व॒मिति॑ कृ॒ष्ण - ग्री॒व॒म् । एति॑ । ल॒भे॒त॒ । ऐ॒न्द्रम् । ऋ॒ष॒भम् । अ॒ग्निः । ए॒व । अ॒स्य॒ । स्वेन॑ । भा॒ग॒धेये॒नेति॑ भाग - धेये॑न । उप॑सृत॒ इत्युप॑ - सृ॒तः॒ ।  \newline


\textbf{Krama Paata} \newline

अ॒चा॒य॒थ् सः । सो॑ ऽमन्यत । अ॒म॒न्य॒त॒ यः । यो वै । वा इ॒मम् । इ॒ममा॒लभे॑त । आ॒लभे॑त॒ मुच्ये॑त । आ॒लभे॒तेत्या᳚ - लभे॑त । मुच्ये॑ता॒स्मात् । अ॒स्मात् पा॒प्मनः॑ । पा॒प्मन॒ इति॑ । इति॒ सः । स आ᳚ग्ने॒यम् । आ॒ग्ने॒यं कृ॒ष्णग्री॑वम् । कृ॒ष्णग्री॑व॒मा । कृ॒ष्णग्री॑व॒मिति॑ कृ॒ष्ण - ग्री॒व॒म् । आऽल॑भत । अ॒ल॒भ॒तै॒न्द्रम् । ऐ॒न्द्रमृ॑ष॒भम् । ऋ॒ष॒भं तस्य॑ । तस्या॒ग्निः । अ॒ग्निरे॒व । ए॒व स्वेन॑ । स्वेन॑ भाग॒धेये॑न । भा॒ग॒धेये॒नोप॑सृतः । भा॒ग॒धेये॒नेति॑ भाग - धेये॑न । उप॑सृतः षोडश॒धा । उप॑सृत॒ इत्युप॑ - सृ॒तः॒ । षो॒ड॒श॒धा वृ॒त्रस्य॑ । षो॒ड॒श॒धेति॑ षोडश - धा । वृ॒त्रस्य॑ भो॒गान् । भो॒गानपि॑ । अप्य॑दहत् । अ॒द॒ह॒दै॒न्द्रेण॑ । ऐ॒न्द्रेणे᳚न्द्रि॒यम् । इ॒न्द्रि॒यमा॒त्मन्न् । आ॒त्मन्न॑धत्त । अ॒ध॒त्त॒ यः । यः पा॒प्मना᳚ । पा॒प्मना॑ गृही॒तः । गृ॒ही॒तः स्यात् । स्याथ् सः । स आ᳚ग्ने॒यम् । आ॒ग्ने॒यं कृ॒ष्णग्री॑वम् । कृ॒ष्णग्री॑व॒मा । कृ॒ष्णग्री॑व॒मिति॑ कृ॒ष्ण - ग्री॒व॒म् । आ ल॑भेत । ल॒भे॒तै॒न्द्रम् । ऐ॒न्द्रमृ॑ष॒भम् । ऋ॒ष॒भम॒ग्निः । अ॒ग्निरे॒व । ए॒वास्य॑ । अ॒स्य॒ स्वेन॑ । स्वेन॑ भाग॒धेये॑न । भा॒ग॒धेये॒नोप॑सृतः । भा॒ग॒धेये॒नेति॑ भाग - धेये॑न । उप॑सृतः पा॒प्मान᳚म् । उप॑सृत॒ इत्युप॑ - सृ॒तः॒ \newline

\textbf{Jatai Paata} \newline

1. अ॒चा॒य॒थ् स सो॑ ऽचाय दचाय॒थ् सः । \newline
2. सो॑ ऽमन्यता मन्यत॒ स सो॑ ऽमन्यत । \newline
3. अ॒म॒न्य॒त॒ यो यो॑ ऽमन्यता मन्यत॒ यः । \newline
4. यो वै वै यो यो वै । \newline
5. वा इ॒म मि॒मं ॅवै वा इ॒मम् । \newline
6. इ॒म मा॒लभे॑ता॒ लभे॑ते॒ म मि॒म मा॒लभे॑त । \newline
7. आ॒लभे॑त॒ मुच्ये॑त॒ मुच्ये॑ता॒ लभे॑ता॒ लभे॑त॒ मुच्ये॑त । \newline
8. आ॒लभे॒तेत्या᳚ - लभे॑त । \newline
9. मुच्ये॑ता॒स्मा द॒स्मान् मुच्ये॑त॒ मुच्ये॑ता॒स्मात् । \newline
10. अ॒स्मात् पा॒प्मनः॑ पा॒प्मनो॒ ऽस्मा द॒स्मात् पा॒प्मनः॑ । \newline
11. पा॒प्मन॒ इतीति॑ पा॒प्मनः॑ पा॒प्मन॒ इति॑ । \newline
12. इति॒ स स इतीति॒ सः । \newline
13. स आ᳚ग्ने॒य मा᳚ग्ने॒यꣳ स स आ᳚ग्ने॒यम् । \newline
14. आ॒ग्ने॒यम् कृ॒ष्णग्री॑वम् कृ॒ष्णग्री॑व माग्ने॒य मा᳚ग्ने॒यम् कृ॒ष्णग्री॑वम् । \newline
15. कृ॒ष्णग्री॑व॒ मा कृ॒ष्णग्री॑वम् कृ॒ष्णग्री॑व॒ मा । \newline
16. कृ॒ष्णग्री॑व॒मिति॑ कृ॒ष्ण - ग्री॒व॒म् । \newline
17. आ ऽल॑भता लभ॒ता ऽल॑भत । \newline
18. अ॒ल॒भ॒तै॒न्द्र मै॒न्द्र म॑लभता लभतै॒न्द्रम् । \newline
19. ऐ॒न्द्र मृ॑ष॒भ मृ॑ष॒भ मै॒न्द्र मै॒न्द्र मृ॑ष॒भम् । \newline
20. ऋ॒ष॒भम् तस्य॒ तस्य॑ र्.ष॒भ मृ॑ष॒भम् तस्य॑ । \newline
21. तस्या॒ग्नि र॒ग्नि स्तस्य॒ तस्या॒ग्निः । \newline
22. अ॒ग्नि रे॒वैवाग्नि र॒ग्नि रे॒व । \newline
23. ए॒व स्वेन॒ स्वेनै॒वैव स्वेन॑ । \newline
24. स्वेन॑ भाग॒धेये॑न भाग॒धेये॑न॒ स्वेन॒ स्वेन॑ भाग॒धेये॑न । \newline
25. भा॒ग॒धेये॒नो प॑सृत॒ उप॑सृतो भाग॒धेये॑न भाग॒धेये॒नो प॑सृतः । \newline
26. भा॒ग॒धेये॒नेति॑ भाग - धेये॑न । \newline
27. उप॑सृत ष्षोडश॒धा षो॑डश॒धो प॑सृत॒ उप॑सृत ष्षोडश॒धा । \newline
28. उप॑सृत॒इत्युप॑ - सृ॒तः॒ । \newline
29. षो॒ड॒श॒धा वृ॒त्रस्य॑ वृ॒त्रस्य॑ षोडश॒धा षो॑डश॒धा वृ॒त्रस्य॑ । \newline
30. षो॒ड॒श॒धेति॑ षोडश - धा । \newline
31. वृ॒त्रस्य॑ भो॒गान् भो॒गान् वृ॒त्रस्य॑ वृ॒त्रस्य॑ भो॒गान् । \newline
32. भो॒गा नप्यपि॑ भो॒गान् भो॒गा नपि॑ । \newline
33. अप्य॑दह ददह॒ दप्यप्य॑ दहत् । \newline
34. अ॒द॒ह॒ दै॒न्द्रे णै॒न्द्रेणा॑ दह ददह दै॒न्द्रेण॑ । \newline
35. ऐ॒न्द्रेणे᳚ न्द्रि॒य मि॑न्द्रि॒य मै॒न्द्रे णै॒न्द्रेणे᳚ न्द्रि॒यम् । \newline
36. इ॒न्द्रि॒य मा॒त्मन् ना॒त्मन् नि॑न्द्रि॒य मि॑न्द्रि॒य मा॒त्मन्न् । \newline
37. आ॒त्मन् न॑धत्ता धत्ता॒त्मन् ना॒त्मन् न॑धत्त । \newline
38. अ॒ध॒त्त॒ यो यो॑ ऽधत्ताधत्त॒ यः । \newline
39. यः पा॒प्मना॑ पा॒प्मना॒ यो यः पा॒प्मना᳚ । \newline
40. पा॒प्मना॑ गृही॒तो गृ॑ही॒तः पा॒प्मना॑ पा॒प्मना॑ गृही॒तः । \newline
41. गृ॒ही॒तः स्याथ् स्याद् गृ॑ही॒तो गृ॑ही॒तः स्यात् । \newline
42. स्याथ् स स स्याथ् स्याथ् सः । \newline
43. स आ᳚ग्ने॒य मा᳚ग्ने॒यꣳ स स आ᳚ग्ने॒यम् । \newline
44. आ॒ग्ने॒यम् कृ॒ष्णग्री॑वम् कृ॒ष्णग्री॑व माग्ने॒य मा᳚ग्ने॒यम् कृ॒ष्णग्री॑वम् । \newline
45. कृ॒ष्णग्री॑व॒ मा कृ॒ष्णग्री॑वम् कृ॒ष्णग्री॑व॒ मा । \newline
46. कृ॒ष्णग्री॑व॒मिति॑ कृ॒ष्ण - ग्री॒व॒म् । \newline
47. आ ल॑भेत लभे॒ता ल॑भेत । \newline
48. ल॒भे॒तै॒न्द्र मै॒न्द्रम् ॅल॑भेत लभेतै॒न्द्रम् । \newline
49. ऐ॒न्द्र मृ॑ष॒भ मृ॑ष॒भ मै॒न्द्र मै॒न्द्र मृ॑ष॒भम् । \newline
50. ऋ॒ष॒भ म॒ग्नि र॒ग्निर्. ऋ॑ष॒भ मृ॑ष॒भ म॒ग्निः । \newline
51. अ॒ग्नि रे॒वैवाग्नि र॒ग्नि रे॒व । \newline
52. ए॒वास्या᳚ स्यै॒ वैवास्य॑ । \newline
53. अ॒स्य॒ स्वेन॒ स्वेना᳚ स्यास्य॒ स्वेन॑ । \newline
54. स्वेन॑ भाग॒धेये॑न भाग॒धेये॑न॒ स्वेन॒ स्वेन॑ भाग॒धेये॑न । \newline
55. भा॒ग॒धेये॒नोप॑सृत॒ उप॑सृतो भाग॒धेये॑न भाग॒धेये॒नोप॑सृतः । \newline
56. भा॒ग॒धेये॒नेति॑ भाग - धेये॑न । \newline
57. उप॑सृतः पा॒प्मान॑म् पा॒प्मान॒ मुप॑सृत॒ उप॑सृतः पा॒प्मान᳚म् । \newline
58. उप॑सृत॒ इत्युप॑ - सृ॒तः॒ । \newline

\textbf{Ghana Paata } \newline

1. अ॒चा॒य॒थ् स सो॑ ऽचाय दचायथ् सो॑ ऽमन्यता मन्यत सो॑ ऽचाय दचायथ् सो॑ ऽमन्यत । \newline
2. सो॑ ऽमन्यता मन्यत॒ स सो॑ ऽमन्यत॒ यो यो॑ ऽमन्यत॒ स सो॑ ऽमन्यत॒ यः । \newline
3. अ॒म॒न्य॒त॒ यो यो॑ ऽमन्यता मन्यत॒ यो वै वै यो॑ ऽमन्यता मन्यत॒ यो वै । \newline
4. यो वै वै यो यो वा इ॒म मि॒मं ॅवै यो यो वा इ॒मम् । \newline
5. वा इ॒म मि॒मं ॅवै वा इ॒म मा॒लभे॑ता॒ लभे॑ते॒ मं ॅवै वा इ॒म मा॒लभे॑त । \newline
6. इ॒म मा॒लभे॑ता॒ लभे॑ते॒ म मि॒म मा॒लभे॑त॒ मुच्ये॑त॒ मुच्ये॑ता॒ लभे॑ते॒ म मि॒म मा॒लभे॑त॒ मुच्ये॑त । \newline
7. आ॒लभे॑त॒ मुच्ये॑त॒ मुच्ये॑ता॒ लभे॑ता॒ लभे॑त॒ मुच्ये॑ ता॒स्मा द॒स्मान् मुच्ये॑ता॒ लभे॑ता॒ लभे॑त॒ मुच्ये॑ता॒स्मात् । \newline
8. आ॒लभे॒तेत्या᳚ - लभे॑त । \newline
9. मुच्ये॑ ता॒स्मा द॒स्मान् मुच्ये॑त॒ मुच्ये॑ ता॒स्मात् पा॒प्मनः॑ पा॒प्मनो॒ ऽस्मान् मुच्ये॑त॒ मुच्ये॑ ता॒स्मात् पा॒प्मनः॑ । \newline
10. अ॒स्मात् पा॒प्मनः॑ पा॒प्मनो॒ ऽस्मा द॒स्मात् पा॒प्मन॒ इतीति॑ पा॒प्मनो॒ ऽस्मा द॒स्मात् पा॒प्मन॒ इति॑ । \newline
11. पा॒प्मन॒ इतीति॑ पा॒प्मनः॑ पा॒प्मन॒ इति॒ स स इति॑ पा॒प्मनः॑ पा॒प्मन॒ इति॒ सः । \newline
12. इति॒ स स इतीति॒ स आ᳚ग्ने॒य मा᳚ग्ने॒यꣳ स इतीति॒ स आ᳚ग्ने॒यम् । \newline
13. स आ᳚ग्ने॒य मा᳚ग्ने॒यꣳ स स आ᳚ग्ने॒यम् कृ॒ष्णग्री॑वम् कृ॒ष्णग्री॑व माग्ने॒यꣳ स स आ᳚ग्ने॒यम् कृ॒ष्णग्री॑वम् । \newline
14. आ॒ग्ने॒यम् कृ॒ष्णग्री॑वम् कृ॒ष्णग्री॑व माग्ने॒य मा᳚ग्ने॒यम् कृ॒ष्णग्री॑व॒ मा कृ॒ष्णग्री॑व माग्ने॒य मा᳚ग्ने॒यम् कृ॒ष्णग्री॑व॒ मा । \newline
15. कृ॒ष्णग्री॑व॒ मा कृ॒ष्णग्री॑वम् कृ॒ष्णग्री॑व॒ मा ऽल॑भता लभ॒ता कृ॒ष्णग्री॑वम् कृ॒ष्णग्री॑व॒ मा ऽल॑भत । \newline
16. कृ॒ष्णग्री॑व॒मिति॑ कृ॒ष्ण - ग्री॒व॒म् । \newline
17. आ ऽल॑भता लभ॒ता ऽल॑भतै॒न्द्र मै॒न्द्र म॑लभ॒ता ऽल॑भतै॒न्द्रम् । \newline
18. अ॒ल॒भ॒तै॒न्द्र मै॒न्द्र म॑लभता लभतै॒न्द्र मृ॑ष॒भ मृ॑ष॒भ मै॒न्द्र म॑लभता लभतै॒न्द्र मृ॑ष॒भम् । \newline
19. ऐ॒न्द्र मृ॑ष॒भ मृ॑ष॒भ मै॒न्द्र मै॒न्द्र मृ॑ष॒भम् तस्य॒ तस्य॑ र्.ष॒भ मै॒न्द्र मै॒न्द्र मृ॑ष॒भम् तस्य॑ । \newline
20. ऋ॒ष॒भम् तस्य॒ तस्य॑ र्.ष॒भ मृ॑ष॒भम् तस्या॒ग्नि र॒ग्नि स्तस्य॑ र्.ष॒भ मृ॑ष॒भम् तस्या॒ग्निः । \newline
21. तस्या॒ग्नि र॒ग्नि स्तस्य॒ तस्या॒ग्नि रे॒वैवाग्नि स्तस्य॒ तस्या॒ग्नि रे॒व । \newline
22. अ॒ग्नि रे॒वैवाग्नि र॒ग्नि रे॒व स्वेन॒ स्वेनै॒वाग्नि र॒ग्नि रे॒व स्वेन॑ । \newline
23. ए॒व स्वेन॒ स्वेनै॒वैव स्वेन॑ भाग॒धेये॑न भाग॒धेये॑न॒ स्वेनै॒वैव स्वेन॑ भाग॒धेये॑न । \newline
24. स्वेन॑ भाग॒धेये॑न भाग॒धेये॑न॒ स्वेन॒ स्वेन॑ भाग॒धेये॒ नोप॑सृत॒ उप॑सृतो भाग॒धेये॑न॒ स्वेन॒ स्वेन॑ भाग॒धेये॒ नोप॑सृतः । \newline
25. भा॒ग॒धेये॒ नोप॑सृत॒ उप॑सृतो भाग॒धेये॑न भाग॒धेये॒ नोप॑सृत ष्षोडश॒धा षो॑डश॒धो प॑सृतो भाग॒धेये॑न भाग॒धेये॒ नोप॑सृत ष्षोडश॒धा । \newline
26. भा॒ग॒धेये॒नेति॑ भाग - धेये॑न । \newline
27. उप॑सृत ष्षोडश॒धा षो॑डश॒धो प॑सृत॒ उप॑सृत ष्षोडश॒धा वृ॒त्रस्य॑ वृ॒त्रस्य॑ षोडश॒धो प॑सृत॒ उप॑सृत ष्षोडश॒धा वृ॒त्रस्य॑ । \newline
28. उप॑सृत॒इत्युप॑ - सृ॒तः॒ । \newline
29. षो॒ड॒श॒धा वृ॒त्रस्य॑ वृ॒त्रस्य॑ षोडश॒धा षो॑डश॒धा वृ॒त्रस्य॑ भो॒गान् भो॒गान् वृ॒त्रस्य॑ षोडश॒धा षो॑डश॒धा वृ॒त्रस्य॑ भो॒गान् । \newline
30. षो॒ड॒श॒धेति॑ षोडश - धा । \newline
31. वृ॒त्रस्य॑ भो॒गान् भो॒गान् वृ॒त्रस्य॑ वृ॒त्रस्य॑ भो॒गा नप्यपि॑ भो॒गान् वृ॒त्रस्य॑ वृ॒त्रस्य॑ भो॒गा नपि॑ । \newline
32. भो॒गा नप्यपि॑ भो॒गान् भो॒गा नप्य॑ दह ददह॒ दपि॑ भो॒गान् भो॒गा नप्य॑ दहत् । \newline
33. अप्य॑ दह ददह॒ दप्यप्य॑ दह दै॒न्द्रेणै॒न्द्रेणा॑ दह॒ दप्यप्य॑ दह दै॒न्द्रेण॑ । \newline
34. अ॒द॒ह॒ दै॒न्द्रेणै॒न्द्रेणा॑ दह ददह दै॒न्द्रेणे᳚ न्द्रि॒य मि॑न्द्रि॒य मै॒न्द्रेणा॑ दह ददह दै॒न्द्रेणे᳚ न्द्रि॒यम् । \newline
35. ऐ॒न्द्रेणे᳚ न्द्रि॒य मि॑न्द्रि॒य मै॒न्द्रेणै॒ न्द्रेणे᳚ न्द्रि॒य मा॒त्मन् ना॒त्मन् नि॑न्द्रि॒य मै॒न्द्रेणै॒ न्द्रेणे᳚ न्द्रि॒य मा॒त्मन्न् । \newline
36. इ॒न्द्रि॒य मा॒त्मन् ना॒त्मन् नि॑न्द्रि॒य मि॑न्द्रि॒य मा॒त्मन् न॑धत्ता धत्ता॒त्मन् नि॑न्द्रि॒य मि॑न्द्रि॒य मा॒त्मन् न॑धत्त । \newline
37. आ॒त्मन् न॑धत्ता धत्ता॒त्मन् ना॒त्मन् न॑धत्त॒ यो यो॑ ऽधत्ता॒त्मन् ना॒त्मन् न॑धत्त॒ यः । \newline
38. अ॒ध॒त्त॒ यो यो॑ ऽधत्ताधत्त॒ यः पा॒प्मना॑ पा॒प्मना॒ यो॑ ऽधत्ताधत्त॒ यः पा॒प्मना᳚ । \newline
39. यः पा॒प्मना॑ पा॒प्मना॒ यो यः पा॒प्मना॑ गृही॒तो गृ॑ही॒तः पा॒प्मना॒ यो यः पा॒प्मना॑ गृही॒तः । \newline
40. पा॒प्मना॑ गृही॒तो गृ॑ही॒तः पा॒प्मना॑ पा॒प्मना॑ गृही॒तः स्याथ् स्याद् गृ॑ही॒तः पा॒प्मना॑ पा॒प्मना॑ गृही॒तः स्यात् । \newline
41. गृ॒ही॒तः स्याथ् स्याद् गृ॑ही॒तो गृ॑ही॒तः स्याथ् स स स्याद् गृ॑ही॒तो गृ॑ही॒तः स्याथ् सः । \newline
42. स्याथ् स स स्याथ् स्याथ् स आ᳚ग्ने॒य मा᳚ग्ने॒यꣳ स स्याथ् स्याथ् स आ᳚ग्ने॒यम् । \newline
43. स आ᳚ग्ने॒य मा᳚ग्ने॒यꣳ स स आ᳚ग्ने॒यम् कृ॒ष्णग्री॑वम् कृ॒ष्णग्री॑व माग्ने॒यꣳ स स आ᳚ग्ने॒यम् कृ॒ष्णग्री॑वम् । \newline
44. आ॒ग्ने॒यम् कृ॒ष्णग्री॑वम् कृ॒ष्णग्री॑व माग्ने॒य मा᳚ग्ने॒यम् कृ॒ष्णग्री॑व॒ मा कृ॒ष्णग्री॑व माग्ने॒य मा᳚ग्ने॒यम् कृ॒ष्णग्री॑व॒ मा । \newline
45. कृ॒ष्णग्री॑व॒ मा कृ॒ष्णग्री॑वम् कृ॒ष्णग्री॑व॒ मा ल॑भेत लभे॒ता कृ॒ष्णग्री॑वम् कृ॒ष्णग्री॑व॒ मा ल॑भेत । \newline
46. कृ॒ष्णग्री॑व॒मिति॑ कृ॒ष्ण - ग्री॒व॒म् । \newline
47. आ ल॑भेत लभे॒ता ल॑भेतै॒न्द्र मै॒न्द्रम् ॅल॑भे॒ता ल॑भेतै॒न्द्रम् । \newline
48. ल॒भे॒तै॒न्द्र मै॒न्द्रम् ॅल॑भेत लभेतै॒न्द्र मृ॑ष॒भ मृ॑ष॒भ मै॒न्द्रम् ॅल॑भेत लभेतै॒न्द्र मृ॑ष॒भम् । \newline
49. ऐ॒न्द्र मृ॑ष॒भ मृ॑ष॒भ मै॒न्द्र मै॒न्द्र मृ॑ष॒भ म॒ग्नि र॒ग्निर्. ऋ॑ष॒भ मै॒न्द्र मै॒न्द्र मृ॑ष॒भ म॒ग्निः । \newline
50. ऋ॒ष॒भ म॒ग्नि र॒ग्निर्. ऋ॑ष॒भ मृ॑ष॒भ म॒ग्नि रे॒वैवाग्निर्. ऋ॑ष॒भ मृ॑ष॒भ म॒ग्नि रे॒व । \newline
51. अ॒ग्नि रे॒वैवा ग्नि र॒ग्नि रे॒वास्या᳚ स्यै॒वाग्नि र॒ग्नि रे॒वास्य॑ । \newline
52. ए॒वास्या᳚ स्यै॒वैवास्य॒ स्वेन॒ स्वेना᳚ स्यै॒वैवास्य॒ स्वेन॑ । \newline
53. अ॒स्य॒ स्वेन॒ स्वेना᳚स्यास्य॒ स्वेन॑ भाग॒धेये॑न भाग॒धेये॑न॒ स्वेना᳚स्यास्य॒ स्वेन॑ भाग॒धेये॑न । \newline
54. स्वेन॑ भाग॒धेये॑न भाग॒धेये॑न॒ स्वेन॒ स्वेन॑ भाग॒धेये॒ नोप॑सृत॒ उप॑सृतो भाग॒धेये॑न॒ स्वेन॒ स्वेन॑ भाग॒धेये॒ नोप॑सृतः । \newline
55. भा॒ग॒धेये॒नो प॑सृत॒ उप॑सृतो भाग॒धेये॑न भाग॒धेये॒नो प॑सृतः पा॒प्मान॑म् पा॒प्मान॒ मुप॑सृतो भाग॒धेये॑न भाग॒धेये॒नो प॑सृतः पा॒प्मान᳚म् । \newline
56. भा॒ग॒धेये॒नेति॑ भाग - धेये॑न । \newline
57. उप॑सृतः पा॒प्मान॑म् पा॒प्मान॒ मुप॑सृत॒ उप॑सृतः पा॒प्मान॒ मप्यपि॑ पा॒प्मान॒ मुप॑सृत॒ उप॑सृतः पा॒प्मान॒ मपि॑ । \newline
58. उप॑सृत॒ इत्युप॑ - सृ॒तः॒ । \newline
\pagebreak
\markright{ TS 2.1.4.7  \hfill https://www.vedavms.in \hfill}

\section{ TS 2.1.4.7 }

\textbf{TS 2.1.4.7 } \newline
\textbf{Samhita Paata} \newline

पा॒प्मान॒मपि॑ दहत्यै॒न्द्रेणे᳚न्द्रि॒यमा॒त्मन् ध॑त्ते॒ मुच्य॑ते पा॒प्मनो॒ भव॑त्ये॒व द्या॑वापृथि॒व्यां᳚ धे॒नुमा ल॑भेत॒ ज्योग॑परुद्धो॒ ऽनयो॒र्॒॒.हि वा  ए॒षोऽप्र॑तिष्ठि॒तोऽथै॒ष ज्योगप॑रुद्धो॒ द्यावा॑पृथि॒वी ए॒व स्वेन॑ भाग॒धेये॒नोप॑ धावति॒ ते ए॒वैनं॑ प्रति॒ष्ठां ग॑मयतः॒ प्रत्ये॒व ति॑ष्ठति पर्या॒रिणी॑ भवति पर्या॒रीव॒ ह्ये॑तस्य॑ रा॒ष्ट्रं ॅयो ज्योग॑परुद्धः॒ समृ॑द्ध्यै वाय॒व्यं॑ - [  ] \newline

\textbf{Pada Paata} \newline

पा॒प्मान᳚म् । अपीति॑ । द॒ह॒ति॒ । ऐ॒न्द्रेण॑ । इ॒न्द्रि॒यम् । आ॒त्मन्न् । ध॒त्ते॒ । मुच्य॑ते । पा॒प्मनः॑ । भव॑ति । ए॒व । द्या॒वा॒पृ॒थि॒व्या॑मिति॑ द्यावा - पृ॒थि॒व्या᳚म् । धे॒नुम् । एति॑ । ल॒भे॒त॒ । ज्योग॑परुद्ध॒ इति॒ ज्योक् - अ॒प॒रु॒द्धः॒ । अ॒नयोः᳚ । हि । वै । ए॒षः । अप्र॑तिष्ठित॒ इत्यप्र॑ति - स्थि॒तः॒ । अथ॑ । ए॒षः । ज्योक् । अप॑रुद्ध॒ इत्यप॑ - रु॒द्धः॒ । द्यावा॑पृथि॒वी इति॒ द्यावा᳚ - पृ॒थि॒वी । ए॒व । स्वेन॑ । भा॒ग॒धेये॒नेति॑ भाग - धेये॑न । उपेति॑ । धा॒व॒ति॒ । ते इति॑ । ए॒व । ए॒न॒म् । प्र॒ति॒ष्ठामिति॑ प्रति - स्थाम् । ग॒म॒य॒तः॒ । प्रतीति॑ । ए॒व । ति॒ष्ठ॒ति॒ । प॒र्या॒रिणी᳚ । भ॒व॒ति॒ । प॒र्या॒रि । इ॒व॒ । हि । ए॒तस्य॑ । रा॒ष्ट्रम् । यः । ज्योग॑परुद्ध॒ इति॒ ज्योक् - अ॒प॒रु॒द्धः॒ । समृ॑द्ध्या॒ इति॒ सं - ऋ॒द्ध्यै॒ । वा॒य॒व्य᳚म् ।  \newline


\textbf{Krama Paata} \newline

पा॒प्मान॒मपि॑ । अपि॑ दहति । द॒ह॒त्यै॒न्द्रेण॑ । ऐ॒न्द्रेणे᳚न्द्रि॒यम् । इ॒न्द्रि॒यमा॒त्मन्न् । आ॒त्मन् ध॑त्ते । ध॒त्ते॒ मुच्य॑ते । मुच्य॑ते पा॒प्मनः॑ । पा॒प्मनो॒ भव॑ति । भव॑त्ये॒व । ए॒व द्या॑वापृथि॒व्या᳚म् । द्या॒वा॒पृ॒थि॒व्या᳚म् धे॒नुम् । द्या॒वा॒पृ॒थि॒व्या॑मिति॑ द्यावा - पृ॒थि॒व्या᳚म् । धे॒नुमा । आ ल॑भेत । ल॒भे॒त॒ ज्योग॑परुद्धः । ज्योग॑परुद्धो॒ ऽनयोः᳚ । ज्योग॑परुद्ध॒ इति॒ ज्योक् - अ॒प॒रु॒द्धः॒ । अ॒नयो॒र्॒. हि । हि वै । वा ए॒षः । ए॒षोऽप्र॑तिष्ठितः । अप्र॑तिष्ठि॒तोऽथ॑ । अप्र॑तिष्ठित॒ इत्यप्र॑ति - स्थि॒तः॒ । अथै॒षः । ए॒ष ज्योक् । ज्योगप॑रुद्धः । अप॑रुद्धो॒ द्यावा॑पृथि॒वी । अप॑रुद्ध॒ इत्यप॑ - रु॒द्धः॒ । द्यावा॑पृथि॒वी ए॒व । द्यावा॑पृथि॒वी इति॒ द्यावा᳚ - पृ॒थि॒वी । ए॒व स्वेन॑ । स्वेन॑ भाग॒धेये॑न । भा॒ग॒धेये॒नोप॑ । भा॒ग॒धेये॒नेति॑ भाग - धेये॑न । उप॑ धावति । धा॒व॒ति॒ ते । ते ए॒व । ते इति॒ ते । ए॒वैन᳚म् । ए॒न॒म् प्र॒ति॒ष्ठाम् । प्र॒ति॒ष्ठाम् ग॑मयतः । प्र॒ति॒ष्ठामिति॑ प्रति - स्थाम् । ग॒म॒य॒तः॒ प्रति॑ । प्रत्ये॒व । ए॒व ति॑ष्ठति । ति॒ष्ठ॒ति॒ प॒र्या॒रिणी᳚ । प॒र्या॒रिणी॑ भवति । भ॒व॒ति॒ प॒र्या॒री । प॒र्या॒रीव॑ । इ॒व॒ हि । ह्ये॑तस्य॑ । ए॒तस्य॑ रा॒ष्ट्रम् । रा॒ष्ट्रं ॅयः । यो ज्योग॑परुद्धः । ज्योग॑परुद्धः॒ समृ॑द्ध्यै । ज्योग॑परुद्ध॒ इति॒ ज्योक् - अ॒प॒रु॒द्धः॒ । समृ॑द्ध्यै वाय॒व्य᳚म् ( ) । समृ॑द्ध्या॒ इति॒ सं - ऋ॒द्ध्यै॒ । वा॒य॒व्यं॑ ॅव॒थ्सम् \newline

\textbf{Jatai Paata} \newline

1. पा॒प्मान॒ मप्यपि॑ पा॒प्मान॑म् पा॒प्मान॒ मपि॑ । \newline
2. अपि॑ दहति दह॒ त्यप्यपि॑ दहति । \newline
3. द॒ह॒ त्यै॒न्द्रे णै॒न्द्रेण॑ दहति दह त्यै॒न्द्रेण॑ । \newline
4. ऐ॒न्द्रेणे᳚ न्द्रि॒य मि॑न्द्रि॒य मै॒न्द्रे णै॒न्द्रेणे᳚ न्द्रि॒यम् । \newline
5. इ॒न्द्रि॒य मा॒त्मन् ना॒त्मन् नि॑न्द्रि॒य मि॑न्द्रि॒य मा॒त्मन्न् । \newline
6. आ॒त्मन् ध॑त्ते धत्त आ॒त्मन् ना॒त्मन् ध॑त्ते । \newline
7. ध॒त्ते॒ मुच्य॑ते॒ मुच्य॑ते धत्ते धत्ते॒ मुच्य॑ते । \newline
8. मुच्य॑ते पा॒प्मनः॑ पा॒प्मनो॒ मुच्य॑ते॒ मुच्य॑ते पा॒प्मनः॑ । \newline
9. पा॒प्मनो॒ भव॑ति॒ भव॑ति पा॒प्मनः॑ पा॒प्मनो॒ भव॑ति । \newline
10. भव॑ त्ये॒वैव भव॑ति॒ भव॑ त्ये॒व । \newline
11. ए॒व द्या॑वापृथि॒व्या᳚म् द्यावापृथि॒व्या॑ मे॒वैव द्या॑वापृथि॒व्या᳚म् । \newline
12. द्या॒वा॒पृ॒थि॒व्या᳚म् धे॒नुम् धे॒नुम् द्या॑वापृथि॒व्या᳚म् द्यावापृथि॒व्या᳚म् धे॒नुम् । \newline
13. द्या॒वा॒पृ॒थि॒व्या॑मिति॑ द्यावा - पृ॒थि॒व्या᳚म् । \newline
14. धे॒नु मा धे॒नुम् धे॒नु मा । \newline
15. आ ल॑भेत लभे॒ता ल॑भेत । \newline
16. ल॒भे॒त॒ ज्योग॑परुद्धो॒ ज्योग॑परुद्धो लभेत लभेत॒ ज्योग॑परुद्धः । \newline
17. ज्योग॑परुद्धो॒ ऽनयो॑र॒नयो॒र् ज्योग॑परुद्धो॒ ज्योग॑परुद्धो॒ ऽनयोः᳚ । \newline
18. ज्योग॑परुद्ध॒ इति॒ ज्योक् - अ॒प॒रु॒द्धः॒ । \newline
19. अ॒नयो॒र्॒. हि ह्य॑नयो॑ र॒नयो॒र्॒. हि । \newline
20. हि वै वै हि हि वै । \newline
21. वा ए॒ष ए॒ष वै वा ए॒षः । \newline
22. ए॒षो ऽप्र॑तिष्ठि॒तो ऽप्र॑तिष्ठित ए॒ष ए॒षो ऽप्र॑तिष्ठितः । \newline
23. अप्र॑तिष्ठि॒तो ऽथाथा प्र॑तिष्ठि॒तो ऽप्र॑तिष्ठि॒तो ऽथ॑ । \newline
24. अप्र॑तिष्ठित॒इत्यप्र॑ति - स्थि॒तः॒ । \newline
25. अथै॒ष ए॒षो ऽथा थै॒षः । \newline
26. ए॒ष ज्योग् ज्योगे॒ष ए॒ष ज्योक् । \newline
27. ज्योगप॑रु॒द्धो ऽप॑रुद्धो॒ ज्योग् ज्योगप॑रुद्धः । \newline
28. अप॑रुद्धो॒ द्यावा॑पृथि॒वी द्यावा॑पृथि॒वी अप॑रु॒द्धो ऽप॑रुद्धो॒ द्यावा॑पृथि॒वी । \newline
29. अप॑रुद्ध॒ इत्यप॑ - रु॒द्धः॒ । \newline
30. द्यावा॑पृथि॒वी ए॒वैव द्यावा॑पृथि॒वी द्यावा॑पृथि॒वी ए॒व । \newline
31. द्यावा॑पृथि॒वी इति॒ द्यावा᳚ - पृ॒थि॒वी । \newline
32. ए॒व स्वेन॒ स्वेनै॒वैव स्वेन॑ । \newline
33. स्वेन॑ भाग॒धेये॑न भाग॒धेये॑न॒ स्वेन॒ स्वेन॑ भाग॒धेये॑न । \newline
34. भा॒ग॒धेये॒नोपोप॑ भाग॒धेये॑न भाग॒धेये॒नोप॑ । \newline
35. भा॒ग॒धेये॒नेति॑ भाग - धेये॑न । \newline
36. उप॑ धावति धाव॒ त्युपोप॑ धावति । \newline
37. धा॒व॒ति॒ ते ते धा॑वति धावति॒ ते । \newline
38. ते ए॒वैव ते ते ए॒व । \newline
39. ते इति॒ ते । \newline
40. ए॒वैन॑ मेन मे॒वैवैन᳚म् । \newline
41. ए॒न॒म् प्र॒ति॒ष्ठाम् प्र॑ति॒ष्ठा मे॑न मेनम् प्रति॒ष्ठाम् । \newline
42. प्र॒ति॒ष्ठाम् ग॑मयतो गमयतः प्रति॒ष्ठाम् प्र॑ति॒ष्ठाम् ग॑मयतः । \newline
43. प्र॒ति॒ष्ठामिति॑ प्रति - स्थाम् । \newline
44. ग॒म॒य॒तः॒ प्रति॒ प्रति॑ गमयतो गमयतः॒ प्रति॑ । \newline
45. प्रत्ये॒वैव प्रति॒ प्रत्ये॒व । \newline
46. ए॒व ति॑ष्ठति तिष्ठ त्ये॒वैव ति॑ष्ठति । \newline
47. ति॒ष्ठ॒ति॒ प॒र्या॒रिणी॑ पर्या॒रिणी॑ तिष्ठति तिष्ठति पर्या॒रिणी᳚ । \newline
48. प॒र्या॒रिणी॑ भवति भवति पर्या॒रिणी॑ पर्या॒रिणी॑ भवति । \newline
49. भ॒व॒ति॒ प॒र्या॒रि प॑र्या॒रि भ॑वति भवति पर्या॒रि । \newline
50. प॒र्या॒रीवे॑ व पर्या॒रि प॑र्या॒रीव॑ । \newline
51. इ॒व॒ हि हीवे॑ व॒ हि । \newline
52. ह्ये॑त स्यै॒तस्य॒ हि ह्ये॑तस्य॑ । \newline
53. ए॒तस्य॑ रा॒ष्ट्रꣳ रा॒ष्ट्र मे॒त स्यै॒तस्य॑ रा॒ष्ट्रम् । \newline
54. रा॒ष्ट्रं ॅयो यो रा॒ष्ट्रꣳ रा॒ष्ट्रं ॅयः । \newline
55. यो ज्योग॑परुद्धो॒ ज्योग॑परुद्धो॒ यो यो ज्योग॑परुद्धः । \newline
56. ज्योग॑परुद्धः॒ समृ॑द्ध्यै॒ समृ॑द्ध्यै॒ ज्योग॑परुद्धो॒ ज्योग॑परुद्धः॒ समृ॑द्ध्यै । \newline
57. ज्योग॑परुद्ध॒ इति॒ ज्योक् - अ॒प॒रु॒द्धः॒ । \newline
58. समृ॑द्ध्यै वाय॒व्यं॑ ॅवाय॒व्यꣳ॑ समृ॑द्ध्यै॒ समृ॑द्ध्यै वाय॒व्य᳚म् । \newline
59. समृ॑द्ध्या॒ इति॒ सं - ऋ॒द्ध्यै॒ । \newline
60. वा॒य॒व्यं॑ ॅव॒थ्सं ॅव॒थ्सं ॅवा॑य॒व्यं॑ ॅवाय॒व्यं॑ ॅव॒थ्सम् । \newline

\textbf{Ghana Paata } \newline

1. पा॒प्मान॒ मप्यपि॑ पा॒प्मान॑म् पा॒प्मान॒ मपि॑ दहति दह॒त्यपि॑ पा॒प्मान॑म् पा॒प्मान॒ मपि॑ दहति । \newline
2. अपि॑ दहति दह॒ त्यप्यपि॑ दह त्यै॒न्द्रे णै॒न्द्रेण॑ दह॒त्यप्यपि॑ दह त्यै॒न्द्रेण॑ । \newline
3. द॒ह॒ त्यै॒न्द्रे णै॒न्द्रेण॑ दहति दह त्यै॒न्द्रेणे᳚ न्द्रि॒य मि॑न्द्रि॒य मै॒न्द्रेण॑ दहति दह त्यै॒न्द्रेणे᳚ न्द्रि॒यम् । \newline
4. ऐ॒न्द्रेणे᳚ न्द्रि॒य मि॑न्द्रि॒य मै॒न्द्रे णै॒न्द्रेणे᳚ न्द्रि॒य मा॒त्मन् ना॒त्मन् नि॑न्द्रि॒य मै॒न्द्रे णै॒न्द्रेणे᳚ न्द्रि॒य मा॒त्मन्न् । \newline
5. इ॒न्द्रि॒य मा॒त्मन् ना॒त्मन् नि॑न्द्रि॒य मि॑न्द्रि॒य मा॒त्मन् ध॑त्ते धत्त आ॒त्मन् नि॑न्द्रि॒य मि॑न्द्रि॒य मा॒त्मन् ध॑त्ते । \newline
6. आ॒त्मन् ध॑त्ते धत्त आ॒त्मन् ना॒त्मन् ध॑त्ते॒ मुच्य॑ते॒ मुच्य॑ते धत्त आ॒त्मन् ना॒त्मन् ध॑त्ते॒ मुच्य॑ते । \newline
7. ध॒त्ते॒ मुच्य॑ते॒ मुच्य॑ते धत्ते धत्ते॒ मुच्य॑ते पा॒प्मनः॑ पा॒प्मनो॒ मुच्य॑ते धत्ते धत्ते॒ मुच्य॑ते पा॒प्मनः॑ । \newline
8. मुच्य॑ते पा॒प्मनः॑ पा॒प्मनो॒ मुच्य॑ते॒ मुच्य॑ते पा॒प्मनो॒ भव॑ति॒ भव॑ति पा॒प्मनो॒ मुच्य॑ते॒ मुच्य॑ते पा॒प्मनो॒ भव॑ति । \newline
9. पा॒प्मनो॒ भव॑ति॒ भव॑ति पा॒प्मनः॑ पा॒प्मनो॒ भव॑त्ये॒वैव भव॑ति पा॒प्मनः॑ पा॒प्मनो॒ भव॑त्ये॒व । \newline
10. भव॑त्ये॒वैव भव॑ति॒ भव॑त्ये॒व द्या॑वापृथि॒व्या᳚म् द्यावापृथि॒व्या॑ मे॒व भव॑ति॒ भव॑त्ये॒व द्या॑वापृथि॒व्या᳚म् । \newline
11. ए॒व द्या॑वापृथि॒व्या᳚म् द्यावापृथि॒व्या॑ मे॒वैव द्या॑वापृथि॒व्या᳚म् धे॒नुम् धे॒नुम् द्या॑वापृथि॒व्या॑ मे॒वैव द्या॑वापृथि॒व्या᳚म् धे॒नुम् । \newline
12. द्या॒वा॒पृ॒थि॒व्या᳚म् धे॒नुम् धे॒नुम् द्या॑वापृथि॒व्या᳚म् द्यावापृथि॒व्या᳚म् धे॒नु मा धे॒नुम् द्या॑वापृथि॒व्या᳚म् द्यावापृथि॒व्या᳚म् धे॒नु मा । \newline
13. द्या॒वा॒पृ॒थि॒व्या॑मिति॑ द्यावा - पृ॒थि॒व्या᳚म् । \newline
14. धे॒नु मा धे॒नुम् धे॒नु मा ल॑भेत लभे॒ता धे॒नुम् धे॒नु मा ल॑भेत । \newline
15. आ ल॑भेत लभे॒ता ल॑भेत॒ ज्योग॑परुद्धो॒ ज्योग॑परुद्धो लभे॒ता ल॑भेत॒ ज्योग॑परुद्धः । \newline
16. ल॒भे॒त॒ ज्योग॑परुद्धो॒ ज्योग॑परुद्धो लभेत लभेत॒ ज्योग॑परुद्धो॒ ऽनयो॑ र॒नयो॒र् ज्योग॑परुद्धो लभेत लभेत॒ ज्योग॑परुद्धो॒ ऽनयोः᳚ । \newline
17. ज्योग॑परुद्धो॒ ऽनयो॑ र॒नयो॒र् ज्योग॑परुद्धो॒ ज्योग॑परुद्धो॒ ऽनयो॒र्॒. हि ह्य॑नयो॒र् ज्योग॑परुद्धो॒ ज्योग॑परुद्धो॒ ऽनयो॒र्॒. हि । \newline
18. ज्योग॑परुद्ध॒ इति॒ ज्योक् - अ॒प॒रु॒द्धः॒ । \newline
19. अ॒नयो॒र्॒. हि ह्य॑नयो॑ र॒नयो॒र्॒. हि वै वै ह्य॑नयो॑ र॒नयो॒र्॒. हि वै । \newline
20. हि वै वै हि हि वा ए॒ष ए॒ष वै हि हि वा ए॒षः । \newline
21. वा ए॒ष ए॒ष वै वा ए॒षो ऽप्र॑तिष्ठि॒तो ऽप्र॑तिष्ठित ए॒ष वै वा ए॒षो ऽप्र॑तिष्ठितः । \newline
22. ए॒षो ऽप्र॑तिष्ठि॒तो ऽप्र॑तिष्ठित ए॒ष ए॒षो ऽप्र॑तिष्ठि॒तो ऽथाथा प्र॑तिष्ठित ए॒ष ए॒षो ऽप्र॑तिष्ठि॒तो ऽथ॑ । \newline
23. अप्र॑तिष्ठि॒तो ऽथाथा प्र॑तिष्ठि॒तो ऽप्र॑तिष्ठि॒तो ऽथै॒ष ए॒षो ऽथा प्र॑तिष्ठि॒तो ऽप्र॑तिष्ठि॒तो ऽथै॒षः । \newline
24. अप्र॑तिष्ठित॒इत्यप्र॑ति - स्थि॒तः॒ । \newline
25. अथै॒ष ए॒षो ऽथा थै॒ष ज्योग् ज्योगे॒षो ऽथा थै॒ष ज्योक् । \newline
26. ए॒ष ज्योग् ज्योगे॒ष ए॒ष ज्योगप॑रु॒द्धो ऽप॑रुद्धो॒ ज्योगे॒ष ए॒ष ज्योगप॑रुद्धः । \newline
27. ज्योगप॑रु॒द्धो ऽप॑रुद्धो॒ ज्योग् ज्योगप॑रुद्धो॒ द्यावा॑पृथि॒वी द्यावा॑पृथि॒वी अप॑रुद्धो॒ ज्योग् ज्योगप॑रुद्धो॒ द्यावा॑पृथि॒वी । \newline
28. अप॑रुद्धो॒ द्यावा॑पृथि॒वी द्यावा॑पृथि॒वी अप॑रु॒द्धो ऽप॑रुद्धो॒ द्यावा॑पृथि॒वी ए॒वैव द्यावा॑पृथि॒वी अप॑रु॒द्धो ऽप॑रुद्धो॒ द्यावा॑पृथि॒वी ए॒व । \newline
29. अप॑रुद्ध॒ इत्यप॑ - रु॒द्धः॒ । \newline
30. द्यावा॑पृथि॒वी ए॒वैव द्यावा॑पृथि॒वी द्यावा॑पृथि॒वी ए॒व स्वेन॒ स्वेनै॒व द्यावा॑पृथि॒वी द्यावा॑पृथि॒वी ए॒व स्वेन॑ । \newline
31. द्यावा॑पृथि॒वी इति॒ द्यावा᳚ - पृ॒थि॒वी । \newline
32. ए॒व स्वेन॒ स्वेनै॒वैव स्वेन॑ भाग॒धेये॑न भाग॒धेये॑न॒ स्वेनै॒वैव स्वेन॑ भाग॒धेये॑न । \newline
33. स्वेन॑ भाग॒धेये॑न भाग॒धेये॑न॒ स्वेन॒ स्वेन॑ भाग॒धेये॒नोपोप॑ भाग॒धेये॑न॒ स्वेन॒ स्वेन॑ भाग॒धेये॒नोप॑ । \newline
34. भा॒ग॒धेये॒नोपोप॑ भाग॒धेये॑न भाग॒धेये॒नोप॑ धावति धाव॒त्युप॑ भाग॒धेये॑न भाग॒धेये॒नोप॑ धावति । \newline
35. भा॒ग॒धेये॒नेति॑ भाग - धेये॑न । \newline
36. उप॑ धावति धाव॒ त्युपोप॑ धावति॒ ते ते धा॑व॒ त्युपोप॑ धावति॒ ते । \newline
37. धा॒व॒ति॒ ते ते धा॑वति धावति॒ ते ए॒वैव ते धा॑वति धावति॒ ते ए॒व । \newline
38. ते ए॒वैव ते ते ए॒वैन॑ मेन मे॒व ते ते ए॒वैन᳚म् । \newline
39. ते इति॒ ते । \newline
40. ए॒वैन॑ मेन मे॒वैवैन॑म् प्रति॒ष्ठाम् प्र॑ति॒ष्ठा मे॑न मे॒वैवैन॑म् प्रति॒ष्ठाम् । \newline
41. ए॒न॒म् प्र॒ति॒ष्ठाम् प्र॑ति॒ष्ठा मे॑न मेनम् प्रति॒ष्ठाम् ग॑मयतो गमयतः प्रति॒ष्ठा मे॑न मेनम् प्रति॒ष्ठाम् ग॑मयतः । \newline
42. प्र॒ति॒ष्ठाम् ग॑मयतो गमयतः प्रति॒ष्ठाम् प्र॑ति॒ष्ठाम् ग॑मयतः॒ प्रति॒ प्रति॑ गमयतः प्रति॒ष्ठाम् प्र॑ति॒ष्ठाम् ग॑मयतः॒ प्रति॑ । \newline
43. प्र॒ति॒ष्ठामिति॑ प्रति - स्थाम् । \newline
44. ग॒म॒य॒तः॒ प्रति॒ प्रति॑ गमयतो गमयतः॒ प्रत्ये॒वैव प्रति॑ गमयतो गमयतः॒ प्रत्ये॒व । \newline
45. प्रत्ये॒वैव प्रति॒ प्रत्ये॒व ति॑ष्ठति तिष्ठत्ये॒व प्रति॒ प्रत्ये॒व ति॑ष्ठति । \newline
46. ए॒व ति॑ष्ठति तिष्ठ त्ये॒वैव ति॑ष्ठति पर्या॒रिणी॑ पर्या॒रिणी॑ तिष्ठ त्ये॒वैव ति॑ष्ठति पर्या॒रिणी᳚ । \newline
47. ति॒ष्ठ॒ति॒ प॒र्या॒रिणी॑ पर्या॒रिणी॑ तिष्ठति तिष्ठति पर्या॒रिणी॑ भवति भवति पर्या॒रिणी॑ तिष्ठति तिष्ठति पर्या॒रिणी॑ भवति । \newline
48. प॒र्या॒रिणी॑ भवति भवति पर्या॒रिणी॑ पर्या॒रिणी॑ भवति पर्या॒रि प॑र्या॒रि भ॑वति पर्या॒रिणी॑ पर्या॒रिणी॑ भवति पर्या॒रि । \newline
49. भ॒व॒ति॒ प॒र्या॒रि प॑र्या॒रि भ॑वति भवति पर्या॒रीवे॑ व पर्या॒रि भ॑वति भवति पर्या॒रीव॑ । \newline
50. प॒र्या॒रीवे॑ व पर्या॒रि प॑र्या॒रीव॒ हि हीव॑ पर्या॒रि प॑र्या॒रीव॒ हि । \newline
51. इ॒व॒ हि हीवे॑ व॒ ह्ये॑त स्यै॒तस्य॒ हीवे॑ व॒ ह्ये॑तस्य॑ । \newline
52. ह्ये॑त स्यै॒तस्य॒ हि ह्ये॑तस्य॑ रा॒ष्ट्रꣳ रा॒ष्ट्र मे॒तस्य॒ हि ह्ये॑तस्य॑ रा॒ष्ट्रम् । \newline
53. ए॒तस्य॑ रा॒ष्ट्रꣳ रा॒ष्ट्र मे॒त स्यै॒तस्य॑ रा॒ष्ट्रं ॅयो यो रा॒ष्ट्र मे॒त स्यै॒तस्य॑ रा॒ष्ट्रं ॅयः । \newline
54. रा॒ष्ट्रं ॅयो यो रा॒ष्ट्रꣳ रा॒ष्ट्रं ॅयो ज्योग॑परुद्धो॒ ज्योग॑परुद्धो॒ यो रा॒ष्ट्रꣳ रा॒ष्ट्रं ॅयो ज्योग॑परुद्धः । \newline
55. यो ज्योग॑परुद्धो॒ ज्योग॑परुद्धो॒ यो यो ज्योग॑परुद्धः॒ समृ॑द्ध्यै॒ समृ॑द्ध्यै॒ ज्योग॑परुद्धो॒ यो यो ज्योग॑परुद्धः॒ समृ॑द्ध्यै । \newline
56. ज्योग॑परुद्धः॒ समृ॑द्ध्यै॒ समृ॑द्ध्यै॒ ज्योग॑परुद्धो॒ ज्योग॑परुद्धः॒ समृ॑द्ध्यै वाय॒व्यं॑ ॅवाय॒व्यꣳ॑ समृ॑द्ध्यै॒ ज्योग॑परुद्धो॒ ज्योग॑परुद्धः॒ समृ॑द्ध्यै वाय॒व्य᳚म् । \newline
57. ज्योग॑परुद्ध॒ इति॒ ज्योक् - अ॒प॒रु॒द्धः॒ । \newline
58. समृ॑द्ध्यै वाय॒व्यं॑ ॅवाय॒व्यꣳ॑ समृ॑द्ध्यै॒ समृ॑द्ध्यै वाय॒व्यं॑ ॅव॒थ्सं ॅव॒थ्सं ॅवा॑य॒व्यꣳ॑ समृ॑द्ध्यै॒ समृ॑द्ध्यै वाय॒व्यं॑ ॅव॒थ्सम् । \newline
59. समृ॑द्ध्या॒ इति॒ सं - ऋ॒द्ध्यै॒ । \newline
60. वा॒य॒व्यं॑ ॅव॒थ्सं ॅव॒थ्सं ॅवा॑य॒व्यं॑ ॅवाय॒व्यं॑ ॅव॒थ्स मा व॒थ्सं ॅवा॑य॒व्यं॑ ॅवाय॒व्यं॑ ॅव॒थ्स मा । \newline
\pagebreak
\markright{ TS 2.1.4.8  \hfill https://www.vedavms.in \hfill}

\section{ TS 2.1.4.8 }

\textbf{TS 2.1.4.8 } \newline
\textbf{Samhita Paata} \newline

ॅव॒थ्समा ल॑भेत वा॒युर्वा अ॒नयो᳚र्व॒थ्स इ॒मे वा ए॒तस्मै॑ लो॒का अप॑शुष्का॒ विडप॑शु॒ष्काऽथै॒ष ज्योगप॑रुद्धो वा॒युमे॒व स्वेन॑ भाग॒धेये॒नोप॑ धावति॒ स ए॒वास्मा॑ इ॒मां ॅलो॒कान्. विशं॒ प्रदा॑पयति॒ प्रास्मा॑ इ॒मे लो॒काः स्नु॑वन्तिभुञ्ज॒त्ये॑नं॒ ॅविडुप॑तिष्ठते ॥ \newline

\textbf{Pada Paata} \newline

व॒थ्सम् । एति॑ । ल॒भे॒त॒ । वा॒युः । वै । अ॒नयोः᳚ । व॒थ्सः । इ॒मे । वै । ए॒तस्मै᳚ । लो॒काः । अप॑शुष्का॒ इत्यप॑ - शु॒ष्काः॒ । विट् । अप॑शु॒ष्केत्यप॑ - शु॒ष्का॒ । अथ॑ । ए॒षः । ज्योक् । अप॑रुद्ध॒ इत्यप॑ - रु॒द्धः॒ । वा॒युम् । ए॒व । स्वेन॑ । भा॒ग॒धेये॒नेति॑ भाग - धेये॑न । उपेति॑ । धा॒व॒ति॒ । सः । ए॒व । अ॒स्मै॒ । इ॒मान् । लो॒कान् । विश᳚म् । प्रेति॑ । दा॒प॒य॒ति॒ । प्रेति॑ । अ॒स्मै॒ । इ॒मे । लो॒काः । स्नु॒व॒न्ति॒ । भु॒ञ्ज॒ती । ए॒न॒म् । विट् । उपेति॑ । ति॒ष्ठ॒ते॒ ॥  \newline


\textbf{Krama Paata} \newline

व॒थ्समा । आ ल॑भेत । ल॒भे॒त॒ वा॒युः । वा॒युर् वै । वा अ॒नयोः᳚ । अ॒नयो᳚र् व॒थ्सः । व॒थ्स इ॒मे । इ॒मे वै । वा ए॒तस्मै᳚ । ए॒तस्मै॑ लो॒काः । लो॒का अप॑शुष्काः । अप॑शुष्का॒ विट् । अप॑शुष्का॒ इत्यप॑ - शु॒ष्काः॒ । विडप॑शुष्का । अप॑शु॒ष्काऽथ॑ । अप॑शु॒ष्केत्यप॑ - शु॒ष्का॒ । अथै॒षः । ए॒ष ज्योक् । ज्योगप॑रुद्धः । अप॑रुद्धो वा॒युम् । अप॑रुद्ध॒ इत्यप॑ - रु॒द्धः॒ । वा॒युमे॒व । ए॒व स्वेन॑ । स्वेन॑ भाग॒धेये॑न । भा॒ग॒धेये॒नोप॑ । भा॒ग॒धेये॒नेति॑ भाग - धेये॑न । उप॑ धावति । धा॒व॒ति॒ सः । स ए॒व । ए॒वास्मै᳚ । अ॒स्मा॒ इ॒मान् । इ॒मान् ॅलो॒कान् । लो॒कान्. विश᳚म् । विशं॒ प्र । प्र दा॑पयति । दा॒प॒य॒ति॒ प्र । प्रास्मै᳚ । अ॒स्मा॒ इ॒मे । इ॒मे लो॒काः । लो॒काः स्नु॑वन्ति । स्नु॒व॒न्ति॒ भु॒ञ्च॒ती । भु॒ञ्च॒त्ये॑नम् । ए॒नं॒ ॅविट् । विडुप॑ । उप॑ तिष्ठते । ति॒ष्ठ॒त॒ इति॑ तिष्ठते । \newline

\textbf{Jatai Paata} \newline

1. व॒थ्स मा व॒थ्सं ॅव॒थ्स मा । \newline
2. आ ल॑भेत लभे॒ता ल॑भेत । \newline
3. ल॒भे॒त॒ वा॒युर् वा॒युर् ल॑भेत लभेत वा॒युः । \newline
4. वा॒युर् वै वै वा॒युर् वा॒युर् वै । \newline
5. वा अ॒नयो॑ र॒नयो॒र् वै वा अ॒नयोः᳚ । \newline
6. अ॒नयो᳚र् व॒थ्सो व॒थ्सो॑ ऽनयो॑ र॒नयो᳚र् व॒थ्सः । \newline
7. व॒थ्स इ॒म इ॒मे व॒थ्सो व॒थ्स इ॒मे । \newline
8. इ॒मे वै वा इ॒म इ॒मे वै । \newline
9. वा ए॒तस्मा॑ ए॒तस्मै॒ वै वा ए॒तस्मै᳚ । \newline
10. ए॒तस्मै॑ लो॒का लो॒का ए॒तस्मा॑ ए॒तस्मै॑ लो॒काः । \newline
11. लो॒का अप॑शुष्का॒ अप॑शुष्का लो॒का लो॒का अप॑शुष्काः । \newline
12. अप॑शुष्का॒ विड् विडप॑शुष्का॒ अप॑शुष्का॒ विट् । \newline
13. अप॑शुष्का॒ इत्यप॑ - शु॒ष्काः॒ । \newline
14. विडप॑शु॒ष्का ऽप॑शुष्का॒ विड् विडप॑शुष्का । \newline
15. अप॑शु॒ष्का ऽथाथा प॑शु॒ष्का ऽप॑शु॒ष्का ऽथ॑ । \newline
16. अप॑शु॒ष्केत्यप॑ - शु॒ष्का॒ । \newline
17. अथै॒ष ए॒षो ऽथा थै॒षः । \newline
18. ए॒ष ज्योग् ज्योगे॒ष ए॒ष ज्योक् । \newline
19. ज्योगप॑रु॒द्धो ऽप॑रुद्धो॒ ज्योग् ज्योगप॑रुद्धः । \newline
20. अप॑रुद्धो वा॒युं ॅवा॒यु मप॑रु॒द्धो ऽप॑रुद्धो वा॒युम् । \newline
21. अप॑रुद्ध॒ इत्यप॑ - रु॒द्धः॒ । \newline
22. वा॒यु मे॒वैव वा॒युं ॅवा॒यु मे॒व । \newline
23. ए॒व स्वेन॒ स्वेनै॒ वैव स्वेन॑ । \newline
24. स्वेन॑ भाग॒धेये॑न भाग॒धेये॑न॒ स्वेन॒ स्वेन॑ भाग॒धेये॑न । \newline
25. भा॒ग॒धेये॒नोपोप॑ भाग॒धेये॑न भाग॒धेये॒नोप॑ । \newline
26. भा॒ग॒धेये॒नेति॑ भाग - धेये॑न । \newline
27. उप॑ धावति धाव॒ त्युपोप॑ धावति । \newline
28. धा॒व॒ति॒ स स धा॑वति धावति॒ सः । \newline
29. स ए॒वैव स स ए॒व । \newline
30. ए॒वास्मा॑ अस्मा ए॒वैवास्मै᳚ । \newline
31. अ॒स्मा॒ इ॒मा नि॒मा न॑स्मा अस्मा इ॒मान् । \newline
32. इ॒मान् ॅलो॒कान् ॅलो॒का नि॒मा नि॒मान् ॅलो॒कान् । \newline
33. लो॒कान्. विशं॒ ॅविश॑म् ॅलो॒कान् ॅलो॒कान्. विश᳚म् । \newline
34. विश॒म् प्र प्र विशं॒ ॅविश॒म् प्र । \newline
35. प्र दा॑पयति दापयति॒ प्र प्र दा॑पयति । \newline
36. दा॒प॒य॒ति॒ प्र प्र दा॑पयति दापयति॒ प्र । \newline
37. प्रास्मा॑ अस्मै॒ प्र प्रास्मै᳚ । \newline
38. अ॒स्मा॒ इ॒म इ॒मे᳚ ऽस्मा अस्मा इ॒मे । \newline
39. इ॒मे लो॒का लो॒का इ॒म इ॒मे लो॒काः । \newline
40. लो॒काः स्नु॑वन्ति स्नुवन्ति लो॒का लो॒काः स्नु॑वन्ति । \newline
41. स्नु॒व॒न्ति॒ भु॒ञ्ज॒ती भु॑ञ्ज॒ती स्नु॑वन्ति स्नुवन्ति भुञ्ज॒ती । \newline
42. भु॒ञ्ज॒ त्ये॑न मेनम् भुञ्ज॒ती भु॑ञ्ज॒ त्ये॑नम् । \newline
43. ए॒नं॒ ॅविड् विडे॑न मेनं॒ ॅविट् । \newline
44. विडुपोप॒ विड् विडुप॑ । \newline
45. उप॑ तिष्ठते तिष्ठत॒ उपोप॑ तिष्ठते । \newline
46. ति॒ष्ठ॒त॒ इति॑ तिष्ठते । \newline

\textbf{Ghana Paata } \newline

1. व॒थ्स मा व॒थ्सं ॅव॒थ्स मा ल॑भेत लभे॒ता व॒थ्सं ॅव॒थ्स मा ल॑भेत । \newline
2. आ ल॑भेत लभे॒ता ल॑भेत वा॒युर् वा॒युर् ल॑भे॒ता ल॑भेत वा॒युः । \newline
3. ल॒भे॒त॒ वा॒युर् वा॒युर् ल॑भेत लभेत वा॒युर् वै वै वा॒युर् ल॑भेत लभेत वा॒युर् वै । \newline
4. वा॒युर् वै वै वा॒युर् वा॒युर् वा अ॒नयो॑ र॒नयो॒र् वै वा॒युर् वा॒युर् वा अ॒नयोः᳚ । \newline
5. वा अ॒नयो॑ र॒नयो॒र् वै वा अ॒नयो᳚र् व॒थ्सो व॒थ्सो॑ ऽनयो॒र् वै वा अ॒नयो᳚र् व॒थ्सः । \newline
6. अ॒नयो᳚र् व॒थ्सो व॒थ्सो॑ ऽनयो॑ र॒नयो᳚र् व॒थ्स इ॒म इ॒मे व॒थ्सो॑ ऽनयो॑ र॒नयो᳚र् व॒थ्स इ॒मे । \newline
7. व॒थ्स इ॒म इ॒मे व॒थ्सो व॒थ्स इ॒मे वै वा इ॒मे व॒थ्सो व॒थ्स इ॒मे वै । \newline
8. इ॒मे वै वा इ॒म इ॒मे वा ए॒तस्मा॑ ए॒तस्मै॒ वा इ॒म इ॒मे वा ए॒तस्मै᳚ । \newline
9. वा ए॒तस्मा॑ ए॒तस्मै॒ वै वा ए॒तस्मै॑ लो॒का लो॒का ए॒तस्मै॒ वै वा ए॒तस्मै॑ लो॒काः । \newline
10. ए॒तस्मै॑ लो॒का लो॒का ए॒तस्मा॑ ए॒तस्मै॑ लो॒का अप॑शुष्का॒ अप॑शुष्का लो॒का ए॒तस्मा॑ ए॒तस्मै॑ लो॒का अप॑शुष्काः । \newline
11. लो॒का अप॑शुष्का॒ अप॑शुष्का लो॒का लो॒का अप॑शुष्का॒ विड् विडप॑शुष्का लो॒का लो॒का अप॑शुष्का॒ विट् । \newline
12. अप॑शुष्का॒ विड् विडप॑शुष्का॒ अप॑शुष्का॒ विडप॑शु॒ष्का ऽप॑शुष्का॒ विडप॑शुष्का॒ अप॑शुष्का॒ विडप॑शुष्का । \newline
13. अप॑शुष्का॒ इत्यप॑ - शु॒ष्काः॒ । \newline
14. विडप॑शु॒ष्का ऽप॑शुष्का॒ विड् विडप॑शु॒ष्का ऽथाथा प॑शुष्का॒ विड् विडप॑शु॒ष्का ऽथ॑ । \newline
15. अप॑शु॒ष्का ऽथाथाप॑शु॒ष्का ऽप॑शु॒ष्का ऽथै॒ष ए॒षो ऽथाप॑शु॒ष्का ऽप॑शु॒ष्का ऽथै॒षः । \newline
16. अप॑शु॒ष्केत्यप॑ - शु॒ष्का॒ । \newline
17. अथै॒ष ए॒षो ऽथाथै॒ष ज्योग् ज्योगे॒षो ऽथाथै॒ष ज्योक् । \newline
18. ए॒ष ज्योग् ज्योगे॒ष ए॒ष ज्योगप॑रु॒द्धो ऽप॑रुद्धो॒ ज्योगे॒ष ए॒ष ज्योगप॑रुद्धः । \newline
19. ज्योगप॑रु॒द्धो ऽप॑रुद्धो॒ ज्योग् ज्योगप॑रुद्धो वा॒युं ॅवा॒यु मप॑रुद्धो॒ ज्योग् ज्योगप॑रुद्धो वा॒युम् । \newline
20. अप॑रुद्धो वा॒युं ॅवा॒यु मप॑रु॒द्धो ऽप॑रुद्धो वा॒यु मे॒वैव वा॒यु मप॑रु॒द्धो ऽप॑रुद्धो वा॒यु मे॒व । \newline
21. अप॑रुद्ध॒ इत्यप॑ - रु॒द्धः॒ । \newline
22. वा॒यु मे॒वैव वा॒युं ॅवा॒यु मे॒व स्वेन॒ स्वेनै॒व वा॒युं ॅवा॒यु मे॒व स्वेन॑ । \newline
23. ए॒व स्वेन॒ स्वेनै॒वैव स्वेन॑ भाग॒धेये॑न भाग॒धेये॑न॒ स्वेनै॒वैव स्वेन॑ भाग॒धेये॑न । \newline
24. स्वेन॑ भाग॒धेये॑न भाग॒धेये॑न॒ स्वेन॒ स्वेन॑ भाग॒धेये॒नोपोप॑ भाग॒धेये॑न॒ स्वेन॒ स्वेन॑ भाग॒धेये॒नोप॑ । \newline
25. भा॒ग॒धेये॒नोपोप॑ भाग॒धेये॑न भाग॒धेये॒नोप॑ धावति धाव॒त्युप॑ भाग॒धेये॑न भाग॒धेये॒नोप॑ धावति । \newline
26. भा॒ग॒धेये॒नेति॑ भाग - धेये॑न । \newline
27. उप॑ धावति धाव॒ त्युपोप॑ धावति॒ स स धा॑व॒ त्युपोप॑ धावति॒ सः । \newline
28. धा॒व॒ति॒ स स धा॑वति धावति॒ स ए॒वैव स धा॑वति धावति॒ स ए॒व । \newline
29. स ए॒वैव स स ए॒वास्मा॑ अस्मा ए॒व स स ए॒वास्मै᳚ । \newline
30. ए॒वास्मा॑ अस्मा ए॒वैवास्मा॑ इ॒मा नि॒मा न॑स्मा ए॒वैवास्मा॑ इ॒मान् । \newline
31. अ॒स्मा॒ इ॒मा नि॒मा न॑स्मा अस्मा इ॒मान् ॅलो॒कान् ॅलो॒का नि॒मा न॑स्मा अस्मा इ॒मान् ॅलो॒कान् । \newline
32. इ॒मान् ॅलो॒कान् ॅलो॒का नि॒मा नि॒मान् ॅलो॒कान्. विशं॒ ॅविश॑म् ॅलो॒का नि॒मा नि॒मान् ॅलो॒कान्. विश᳚म् । \newline
33. लो॒कान्. विशं॒ ॅविश॑म् ॅलो॒कान् ॅलो॒कान्. विश॒म् प्र प्र विश॑म् ॅलो॒कान् ॅलो॒कान्. विश॒म् प्र । \newline
34. विश॒म् प्र प्र विशं॒ ॅविश॒म् प्र दा॑पयति दापयति॒ प्र विशं॒ ॅविश॒म् प्र दा॑पयति । \newline
35. प्र दा॑पयति दापयति॒ प्र प्र दा॑पयति॒ प्र प्र दा॑पयति॒ प्र प्र दा॑पयति॒ प्र । \newline
36. दा॒प॒य॒ति॒ प्र प्र दा॑पयति दापयति॒ प्रास्मा॑ अस्मै॒ प्र दा॑पयति दापयति॒ प्रास्मै᳚ । \newline
37. प्रास्मा॑ अस्मै॒ प्र प्रास्मा॑ इ॒म इ॒मे᳚ ऽस्मै॒ प्र प्रास्मा॑ इ॒मे । \newline
38. अ॒स्मा॒ इ॒म इ॒मे᳚ ऽस्मा अस्मा इ॒मे लो॒का लो॒का इ॒मे᳚ ऽस्मा अस्मा इ॒मे लो॒काः । \newline
39. इ॒मे लो॒का लो॒का इ॒म इ॒मे लो॒काः स्नु॑वन्ति स्नुवन्ति लो॒का इ॒म इ॒मे लो॒काः स्नु॑वन्ति । \newline
40. लो॒काः स्नु॑वन्ति स्नुवन्ति लो॒का लो॒काः स्नु॑वन्ति भुञ्ज॒ती भु॑ञ्ज॒ती स्नु॑वन्ति लो॒का लो॒काः स्नु॑वन्ति भुञ्ज॒ती । \newline
41. स्नु॒व॒न्ति॒ भु॒ञ्ज॒ती भु॑ञ्ज॒ती स्नु॑वन्ति स्नुवन्ति भुञ्ज॒त्ये॑न मेनम् भुञ्ज॒ती स्नु॑वन्ति स्नुवन्ति भुञ्ज॒त्ये॑नम् । \newline
42. भु॒ञ्ज॒त्ये॑न मेनम् भुञ्ज॒ती भु॑ञ्ज॒त्ये॑नं॒ ॅविड् विडे॑नम् भुञ्ज॒ती भु॑ञ्ज॒त्ये॑नं॒ ॅविट् । \newline
43. ए॒नं॒ ॅविड् विडे॑न मेनं॒ ॅविडुपोप॒ विडे॑न मेनं॒ ॅविडुप॑ । \newline
44. विडुपोप॒ विड् विडुप॑ तिष्ठते तिष्ठत॒ उप॒ विड् विडुप॑ तिष्ठते । \newline
45. उप॑ तिष्ठते तिष्ठत॒ उपोप॑ तिष्ठते । \newline
46. ति॒ष्ठ॒त॒ इति॑ तिष्ठते । \newline
\pagebreak
\markright{ TS 2.1.5.1  \hfill https://www.vedavms.in \hfill}

\section{ TS 2.1.5.1 }

\textbf{TS 2.1.5.1 } \newline
\textbf{Samhita Paata} \newline

इन्द्रो॑ व॒लस्य॒ बिल॒मपौ᳚र्णो॒थ् स य उ॑त्त॒मः प॒शुरासी॒त् तं पृ॒ष्ठं प्रति॑ स॒गृंह्योद॑क्खिद॒त् तꣳ स॒हस्रं॑ प॒शवोऽनूदा॑य॒न्थ् स उ॑न्न॒तो॑ऽभव॒द्यः प॒शुका॑मः॒ स्याथ् स ए॒तमै॒न्द्रमु॑न्न॒तमा ल॑भे॒तेन्द्र॑मे॒व स्वेन॑ भाग॒धेये॒नोप॑ धावति॒ स ए॒वास्मै॑ प॒शून् प्रय॑च्छति पशु॒माने॒व भ॑वत्युन्न॒तो -  [  ] \newline

\textbf{Pada Paata} \newline

इन्द्रः॑ । व॒लस्य॑ । बिल᳚म् । अपेति॑ । औ॒र्णो॒त् । सः । यः । उ॒त्त॒म इत्यु॑त् - त॒मः । प॒शुः । आसी᳚त् । तम् । पृ॒ष्ठम् । प्रतीति॑ । स॒गृंह्येति॑ सं - गृह्य॑ । उदिति॑ । अ॒क्खि॒द॒त् । तम् । स॒हस्र᳚म् । प॒शवः॑ । अनु॑ । उदिति॑ । आ॒य॒न्न् । सः । उ॒न्न॒त इत्यु॑त् - न॒तः । अ॒भ॒व॒त् । यः । प॒शुका॑म॒ इति॑ प॒शु-का॒मः॒ । स्यात् । सः । ए॒तम् । ऐ॒न्द्रम् । उ॒न्न॒तमित्यु॑त्-न॒तम् । एति॑ । ल॒भे॒त॒ । इन्द्र᳚म् । ए॒व । स्वेन॑ । भा॒ग॒धेये॒नेति॑ भाग - धेये॑न । उपेति॑ । धा॒व॒ति॒ । सः । ए॒व । अ॒स्मै॒ । प॒शून् । प्रेति॑ । य॒च्छ॒ति॒ । प॒शु॒मानिति॑ पशु - मान् । ए॒व । भ॒व॒ति॒ । उ॒न्न॒त इत्यु॑त् - न॒तः ।  \newline


\textbf{Krama Paata} \newline

इन्द्रो॑ व॒लस्य॑ । व॒लस्य॒ बिल᳚म् । बिल॒मप॑ । अपौ᳚र्णोत् । औ॒र्णो॒थ् सः । स यः । य उ॑त्त॒मः । उ॒त्त॒मः प॒शुः । उ॒त्त॒म इत्यु॑त् - त॒मः । प॒शुरासी᳚त् । आसी॒त् तम् । तम् पृ॒ष्ठम् । पृ॒ष्ठम् प्रति॑ । प्रति॑ स॒ङ्गृह्य॑ । स॒ङ्गृह्योत् । स॒ङ्गृह्येति॑ सं - गृह्य॑ । उद॑क्खिदत् । अ॒क्खि॒द॒त् तम् । तꣳ स॒हस्र᳚म् । स॒हस्र॑म् प॒शवः॑ । प॒शवोऽनु॑ । अनूत् । उदा॑यन्न् । आ॒य॒न्थ् सः । स उ॑न्न॒तः । उ॒न्न॒तो॑ऽभवत् । उ॒न्न॒त इत्यु॑त् - न॒तः । अ॒भ॒व॒द् यः । यः प॒शुका॑मः । प॒शुका॑मः॒ स्यात् । प॒शुका॑म॒ इति॑ प॒शु - का॒मः॒ । स्याथ् सः । स ए॒तम् । ए॒तमै॒न्द्रम् । ऐ॒न्द्रमु॑न्न॒तम् । उ॒न्न॒तमा । उ॒न्न॒तमित्यु॑त् - न॒तम् । आ ल॑भेत । ल॒भे॒तेन्द्र᳚म् । इन्द्र॑मे॒व । ए॒व स्वेन॑ । स्वेन॑ भाग॒धेये॑न । भा॒ग॒धेये॒नोप॑ । भा॒ग॒धेये॒नेति॑ भाग - धेये॑न । उप॑ धावति । धा॒व॒ति॒ सः । स ए॒व । ए॒वास्मै᳚ । अ॒स्मै॒ प॒शून् । प॒शून् प्र । प्र य॑च्छति । य॒च्छ॒ति॒ प॒शु॒मान् । प॒शु॒माने॒व । प॒शु॒मानिति॑ पशु - मान् । ए॒व भ॑वति । भ॒व॒त्यु॒न्न॒तः । उ॒न्न॒तो भ॑वति । उ॒न्न॒त इत्यु॑त् - न॒तः \newline

\textbf{Jatai Paata} \newline

1. इन्द्रो॑ व॒लस्य॑ व॒लस्ये न्द्र॒ इन्द्रो॑ व॒लस्य॑ । \newline
2. व॒लस्य॒ बिल॒म् बिलं॑ ॅव॒लस्य॑ व॒लस्य॒ बिल᳚म् । \newline
3. बिल॒ मपाप॒ बिल॒म् बिल॒ मप॑ । \newline
4. अपौ᳚र्णो दौर्णो॒ दपा पौ᳚र्णोत् । \newline
5. औ॒र्णो॒थ् स स औ᳚र्णो दौर्णो॒थ् सः । \newline
6. स यो यः स स यः । \newline
7. य उ॑त्त॒म उ॑त्त॒मो यो य उ॑त्त॒मः । \newline
8. उ॒त्त॒मः प॒शुः प॒शु रु॑त्त॒म उ॑त्त॒मः प॒शुः । \newline
9. उ॒त्त॒म इत्यु॑त् - त॒मः । \newline
10. प॒शु रासी॒ दासी᳚त् प॒शुः प॒शु रासी᳚त् । \newline
11. आसी॒त् तम् त मासी॒ दासी॒त् तम् । \newline
12. तम् पृ॒ष्ठम् पृ॒ष्ठम् तम् तम् पृ॒ष्ठम् । \newline
13. पृ॒ष्ठम् प्रति॒ प्रति॑ पृ॒ष्ठम् पृ॒ष्ठम् प्रति॑ । \newline
14. प्रति॑ स॒ङ्गृह्य॑ स॒ङ्गृह्य॒ प्रति॒ प्रति॑ स॒ङ्गृह्य॑ । \newline
15. स॒ङ्गृह्योदुथ् स॒ङ्गृह्य॑ स॒ङ्गृह्योत् । \newline
16. स॒ङ्गृह्येति॑ सं - गृह्य॑ । \newline
17. उद॑क्खि ददक्खि द॒दुदु द॑क्खिदत् । \newline
18. अ॒क्खि॒द॒त् तम् त म॑क्खि ददक्खिद॒त् तम् । \newline
19. तꣳ स॒हस्रꣳ॑ स॒हस्र॒म् तम् तꣳ स॒हस्र᳚म् । \newline
20. स॒हस्र॑म् प॒शवः॑ प॒शवः॑ स॒हस्रꣳ॑ स॒हस्र॑म् प॒शवः॑ । \newline
21. प॒शवो ऽन्वनु॑ प॒शवः॑ प॒शवो ऽनु॑ । \newline
22. अनू दु दन्वनूत् । \newline
23. उदा॑यन् नाय॒न् नुदु दा॑यन्न् । \newline
24. आ॒य॒न् थ्स स आ॑यन् नाय॒न् थ्सः । \newline
25. स उ॑न्न॒त उ॑न्न॒तः स स उ॑न्न॒तः । \newline
26. उ॒न्न॒तो॑ ऽभव दभव दुन्न॒त उ॑न्न॒तो॑ ऽभवत् । \newline
27. उ॒न्न॒त इत्यु॑त् - न॒तः । \newline
28. अ॒भ॒व॒द् यो यो॑ ऽभव दभव॒द् यः । \newline
29. यः प॒शुका॑मः प॒शुका॑मो॒ यो यः प॒शुका॑मः । \newline
30. प॒शुका॑मः॒ स्याथ् स्यात् प॒शुका॑मः प॒शुका॑मः॒ स्यात् । \newline
31. प॒शुका॑म॒ इति॑ प॒शु - का॒मः॒ । \newline
32. स्याथ् स स स्याथ् स्याथ् सः । \newline
33. स ए॒त मे॒तꣳ स स ए॒तम् । \newline
34. ए॒त मै॒न्द्र मै॒न्द्र मे॒त मे॒त मै॒न्द्रम् । \newline
35. ऐ॒न्द्र मु॑न्न॒त मु॑न्न॒त मै॒न्द्र मै॒न्द्र मु॑न्न॒तम् । \newline
36. उ॒न्न॒त मोन्न॒त मु॑न्न॒त मा । \newline
37. उ॒न्न॒तमित्यु॑त् - न॒तम् । \newline
38. आ ल॑भेत लभे॒ता ल॑भेत । \newline
39. ल॒भे॒ते न्द्र॒ मिन्द्र॑म् ॅलभेत लभे॒ते न्द्र᳚म् । \newline
40. इन्द्र॑ मे॒वैवे न्द्र॒ मिन्द्र॑ मे॒व । \newline
41. ए॒व स्वेन॒ स्वे नै॒वैव स्वेन॑ । \newline
42. स्वेन॑ भाग॒धेये॑न भाग॒धेये॑न॒ स्वेन॒ स्वेन॑ भाग॒धेये॑न । \newline
43. भा॒ग॒धेये॒नोपोप॑ भाग॒धेये॑न भाग॒धेये॒नोप॑ । \newline
44. भा॒ग॒धेये॒नेति॑ भाग - धेये॑न । \newline
45. उप॑ धावति धाव॒ त्युपोप॑ धावति । \newline
46. धा॒व॒ति॒ स स धा॑वति धावति॒ सः । \newline
47. स ए॒वैव स स ए॒व । \newline
48. ए॒वास्मा॑ अस्मा ए॒वैवास्मै᳚ । \newline
49. अ॒स्मै॒ प॒शून् प॒शू न॑स्मा अस्मै प॒शून् । \newline
50. प॒शून् प्र प्र प॒शून् प॒शून् प्र । \newline
51. प्र य॑च्छति यच्छति॒ प्र प्र य॑च्छति । \newline
52. य॒च्छ॒ति॒ प॒शु॒मान् प॑शु॒मान्. य॑च्छति यच्छति पशु॒मान् । \newline
53. प॒शु॒मा ने॒वैव प॑शु॒मान् प॑शु॒मा ने॒व । \newline
54. प॒शु॒मानिति॑ पशु - मान् । \newline
55. ए॒व भ॑वति भव त्ये॒वैव भ॑वति । \newline
56. भ॒व॒ त्यु॒न्न॒त उ॑न्न॒तो भ॑वति भव त्युन्न॒तः । \newline
57. उ॒न्न॒तो भ॑वति भव त्युन्न॒त उ॑न्न॒तो भ॑वति । \newline
58. उ॒न्न॒त इत्यु॑त् - न॒तः । \newline

\textbf{Ghana Paata } \newline

1. इन्द्रो॑ व॒लस्य॑ व॒लस्ये न्द्र॒ इन्द्रो॑ व॒लस्य॒ बिल॒म् बिलं॑ ॅव॒लस्ये न्द्र॒ इन्द्रो॑ व॒लस्य॒ बिल᳚म् । \newline
2. व॒लस्य॒ बिल॒म् बिलं॑ ॅव॒लस्य॑ व॒लस्य॒ बिल॒ मपाप॒ बिलं॑ ॅव॒लस्य॑ व॒लस्य॒ बिल॒ मप॑ । \newline
3. बिल॒ मपाप॒ बिल॒म् बिल॒ मपौ᳚र्णो दौर्णो॒ दप॒ बिल॒म् बिल॒ मपौ᳚र्णोत् । \newline
4. अपौ᳚र्णो दौर्णो॒ दपापौ᳚र्णो॒थ् स स औ᳚र्णो॒ दपापौ᳚र्णो॒थ् सः । \newline
5. औ॒र्णो॒थ् स स औ᳚र्णो दौर्णो॒थ् स यो यः स औ᳚र्णो दौर्णो॒थ् स यः । \newline
6. स यो यः स स य उ॑त्त॒म उ॑त्त॒मो यः स स य उ॑त्त॒मः । \newline
7. य उ॑त्त॒म उ॑त्त॒मो यो य उ॑त्त॒मः प॒शुः प॒शु रु॑त्त॒मो यो य उ॑त्त॒मः प॒शुः । \newline
8. उ॒त्त॒मः प॒शुः प॒शु रु॑त्त॒म उ॑त्त॒मः प॒शु रासी॒ दासी᳚त् प॒शु रु॑त्त॒म उ॑त्त॒मः प॒शु रासी᳚त् । \newline
9. उ॒त्त॒म इत्यु॑त् - त॒मः । \newline
10. प॒शु रासी॒ दासी᳚त् प॒शुः प॒शु रासी॒त् तम् त मासी᳚त् प॒शुः प॒शु रासी॒त् तम् । \newline
11. आसी॒त् तम् त मासी॒ दासी॒त् तम् पृ॒ष्ठम् पृ॒ष्ठम् त मासी॒ दासी॒त् तम् पृ॒ष्ठम् । \newline
12. तम् पृ॒ष्ठम् पृ॒ष्ठम् तम् तम् पृ॒ष्ठम् प्रति॒ प्रति॑ पृ॒ष्ठम् तम् तम् पृ॒ष्ठम् प्रति॑ । \newline
13. पृ॒ष्ठम् प्रति॒ प्रति॑ पृ॒ष्ठम् पृ॒ष्ठम् प्रति॑ स॒ङ्गृह्य॑ स॒ङ्गृह्य॒ प्रति॑ पृ॒ष्ठम् पृ॒ष्ठम् प्रति॑ स॒ङ्गृह्य॑ । \newline
14. प्रति॑ स॒ङ्गृह्य॑ स॒ङ्गृह्य॒ प्रति॒ प्रति॑ स॒ङ्गृह्योदुथ् स॒ङ्गृह्य॒ प्रति॒ प्रति॑ स॒ङ्गृह्योत् । \newline
15. स॒ङ्गृह्यो दुथ् स॒ङ्गृह्य॑ स॒ङ्गृह्यो द॑क्खिद दक्खिद॒ दुथ् स॒ङ्गृह्य॑ स॒ङ्गृह्यो द॑क्खिदत् । \newline
16. स॒ङ्गृह्येति॑ सं - गृह्य॑ । \newline
17. उद॑क्खिद दक्खिद॒ दुदु द॑क्खिद॒त् तम् त म॑क्खिद॒ दुदु द॑क्खिद॒त् तम् । \newline
18. अ॒क्खि॒द॒त् तम् त म॑क्खिद दक्खिद॒त् तꣳ स॒हस्रꣳ॑ स॒हस्र॒म् त म॑क्खिद दक्खिद॒त् तꣳ स॒हस्र᳚म् । \newline
19. तꣳ स॒हस्रꣳ॑ स॒हस्र॒म् तम् तꣳ स॒हस्र॑म् प॒शवः॑ प॒शवः॑ स॒हस्र॒म् तम् तꣳ स॒हस्र॑म् प॒शवः॑ । \newline
20. स॒हस्र॑म् प॒शवः॑ प॒शवः॑ स॒हस्रꣳ॑ स॒हस्र॑म् प॒शवो ऽन्वनु॑ प॒शवः॑ स॒हस्रꣳ॑ स॒हस्र॑म् प॒शवो ऽनु॑ । \newline
21. प॒शवो ऽन्वनु॑ प॒शवः॑ प॒शवो ऽनूदु दनु॑ प॒शवः॑ प॒शवो ऽनूत् । \newline
22. अनू दु दन्व नूदा॑यन् नाय॒न् नुदन्व नूदा॑यन्न् । \newline
23. उदा॑यन् नाय॒न् नुदु दा॑य॒न् थ्स स आ॑य॒न् नुदु दा॑य॒न् थ्सः । \newline
24. आ॒य॒न् थ्स स आ॑यन् नाय॒न् थ्स उ॑न्न॒त उ॑न्न॒तः स आ॑यन् नाय॒न् थ्स उ॑न्न॒तः । \newline
25. स उ॑न्न॒त उ॑न्न॒तः स स उ॑न्न॒तो॑ ऽभव दभव दुन्न॒तः स स उ॑न्न॒तो॑ ऽभवत् । \newline
26. उ॒न्न॒तो॑ ऽभव दभव दुन्न॒त उ॑न्न॒तो॑ ऽभव॒द् यो यो॑ ऽभव दुन्न॒त उ॑न्न॒तो॑ ऽभव॒द् यः । \newline
27. उ॒न्न॒त इत्यु॑त् - न॒तः । \newline
28. अ॒भ॒व॒द् यो यो॑ ऽभव दभव॒द् यः प॒शुका॑मः प॒शुका॑मो॒ यो॑ ऽभव दभव॒द् यः प॒शुका॑मः । \newline
29. यः प॒शुका॑मः प॒शुका॑मो॒ यो यः प॒शुका॑मः॒ स्याथ् स्यात् प॒शुका॑मो॒ यो यः प॒शुका॑मः॒ स्यात् । \newline
30. प॒शुका॑मः॒ स्याथ् स्यात् प॒शुका॑मः प॒शुका॑मः॒ स्याथ् स स स्यात् प॒शुका॑मः प॒शुका॑मः॒ स्याथ् सः । \newline
31. प॒शुका॑म॒ इति॑ प॒शु - का॒मः॒ । \newline
32. स्याथ् स स स्याथ् स्याथ् स ए॒त मे॒तꣳ स स्याथ् स्याथ् स ए॒तम् । \newline
33. स ए॒त मे॒तꣳ स स ए॒त मै॒न्द्र मै॒न्द्र मे॒तꣳ स स ए॒त मै॒न्द्रम् । \newline
34. ए॒त मै॒न्द्र मै॒न्द्र मे॒त मे॒त मै॒न्द्र मु॑न्न॒त मु॑न्न॒त मै॒न्द्र मे॒त मे॒त मै॒न्द्र मु॑न्न॒तम् । \newline
35. ऐ॒न्द्र मु॑न्न॒त मु॑न्न॒त मै॒न्द्र मै॒न्द्र मु॑न्न॒त मोन्न॒त मै॒न्द्र मै॒न्द्र मु॑न्न॒त मा । \newline
36. उ॒न्न॒त मोन्न॒त मु॑न्न॒त मा ल॑भेत लभे॒तोन्न॒त मु॑न्न॒त मा ल॑भेत । \newline
37. उ॒न्न॒तमित्यु॑त् - न॒तम् । \newline
38. आ ल॑भेत लभे॒ता ल॑भे॒ते न्द्र॒ मिन्द्र॑म् ॅलभे॒ता ल॑भे॒ते न्द्र᳚म् । \newline
39. ल॒भे॒ते न्द्र॒ मिन्द्र॑म् ॅलभेत लभे॒ते न्द्र॑ मे॒वैवे न्द्र॑म् ॅलभेत लभे॒ते न्द्र॑ मे॒व । \newline
40. इन्द्र॑ मे॒वैवे न्द्र॒ मिन्द्र॑ मे॒व स्वेन॒ स्वेनै॒वे न्द्र॒ मिन्द्र॑ मे॒व स्वेन॑ । \newline
41. ए॒व स्वेन॒ स्वेनै॒वैव स्वेन॑ भाग॒धेये॑न भाग॒धेये॑न॒ स्वेनै॒वैव स्वेन॑ भाग॒धेये॑न । \newline
42. स्वेन॑ भाग॒धेये॑न भाग॒धेये॑न॒ स्वेन॒ स्वेन॑ भाग॒धेये॒नोपोप॑ भाग॒धेये॑न॒ स्वेन॒ स्वेन॑ भाग॒धेये॒नोप॑ । \newline
43. भा॒ग॒धेये॒नोपोप॑ भाग॒धेये॑न भाग॒धेये॒नोप॑ धावति धाव॒त्युप॑ भाग॒धेये॑न भाग॒धेये॒नोप॑ धावति । \newline
44. भा॒ग॒धेये॒नेति॑ भाग - धेये॑न । \newline
45. उप॑ धावति धाव॒ त्युपोप॑ धावति॒ स स धा॑व॒ त्युपोप॑ धावति॒ सः । \newline
46. धा॒व॒ति॒ स स धा॑वति धावति॒ स ए॒वैव स धा॑वति धावति॒ स ए॒व । \newline
47. स ए॒वैव स स ए॒वास्मा॑ अस्मा ए॒व स स ए॒वास्मै᳚ । \newline
48. ए॒वास्मा॑ अस्मा ए॒वैवास्मै॑ प॒शून् प॒शू न॑स्मा ए॒वैवास्मै॑ प॒शून् । \newline
49. अ॒स्मै॒ प॒शून् प॒शू न॑स्मा अस्मै प॒शून् प्र प्र प॒शू न॑स्मा अस्मै प॒शून् प्र । \newline
50. प॒शून् प्र प्र प॒शून् प॒शून् प्र य॑च्छति यच्छति॒ प्र प॒शून् प॒शून् प्र य॑च्छति । \newline
51. प्र य॑च्छति यच्छति॒ प्र प्र य॑च्छति पशु॒मान् प॑शु॒मान्. य॑च्छति॒ प्र प्र य॑च्छति पशु॒मान् । \newline
52. य॒च्छ॒ति॒ प॒शु॒मान् प॑शु॒मान्. य॑च्छति यच्छति पशु॒मा ने॒वैव प॑शु॒मान्. य॑च्छति यच्छति पशु॒मा ने॒व । \newline
53. प॒शु॒मा ने॒वैव प॑शु॒मान् प॑शु॒मा ने॒व भ॑वति भवत्ये॒व प॑शु॒मान् प॑शु॒मा ने॒व भ॑वति । \newline
54. प॒शु॒मानिति॑ पशु - मान् । \newline
55. ए॒व भ॑वति भव त्ये॒वैव भ॑व त्युन्न॒त उ॑न्न॒तो भ॑व त्ये॒वैव भ॑व त्युन्न॒तः । \newline
56. भ॒व॒ त्यु॒न्न॒त उ॑न्न॒तो भ॑वति भव त्युन्न॒तो भ॑वति भव त्युन्न॒तो भ॑वति भव त्युन्न॒तो भ॑वति । \newline
57. उ॒न्न॒तो भ॑वति भव त्युन्न॒त उ॑न्न॒तो भ॑वति साह॒स्री सा॑ह॒स्री भ॑व त्युन्न॒त उ॑न्न॒तो भ॑वति साह॒स्री । \newline
58. उ॒न्न॒त इत्यु॑त् - न॒तः । \newline
\pagebreak
\markright{ TS 2.1.5.2  \hfill https://www.vedavms.in \hfill}

\section{ TS 2.1.5.2 }

\textbf{TS 2.1.5.2 } \newline
\textbf{Samhita Paata} \newline

भ॑वति साह॒स्री वा ए॒षा ल॒क्ष्मी यदु॑न्न॒तो ल॒क्ष्मियै॒ व प॒शूनव॑ रुन्धे य॒दा स॒हस्रं॑ प॒शून् प्रा᳚प्नु॒यादथ॑ वैष्ण॒वं ॅवा॑म॒नमा ल॑भेतै॒तस्मि॒न्. वै तथ् स॒हस्र॒मद्ध्य॑तिष्ठ॒त् तस्मा॑दे॒ष वा॑म॒नः समी॑षितः प॒शुभ्य॑ ए॒व प्रजा॑तेभ्यः प्रति॒ष्ठां द॑धाति॒ को॑ऽर्.हति स॒हस्रं॑ प॒शून् प्राप्तु॒मित्या॑हु- रहोरा॒त्राण्ये॒व स॒हस्रꣳ॑ स॒म्पाद्याऽऽ*ल॑भेत प॒शवो॒ - [  ] \newline

\textbf{Pada Paata} \newline

भ॒व॒ति॒ । सा॒ह॒स्री । वै । ए॒षा । ल॒क्ष्मी । यत् । उ॒न्न॒त इत्यु॑त् - न॒तः । ल॒क्ष्मिया᳚ । ए॒व । प॒शून् । अवेति॑ । रु॒न्धे॒ । य॒दा । स॒हस्र᳚म् । प॒शून् । प्रा॒प्नु॒यादिति॑ प्र - आ॒प्नु॒यात् । अथ॑ । वै॒ष्ण॒वम् । वा॒म॒नम् । एति॑ । ल॒भे॒त॒ । ए॒तस्मिन्न्॑ । वै । तत् । स॒हस्र᳚म् । अधीति॑ । अ॒ति॒ष्ठ॒त् । तस्मा᳚त् । ए॒षः । वा॒म॒नः । समी॑षित॒ इति॒ सं - ई॒षि॒तः॒ । प॒शुभ्य॒ इति॑ प॒शु - भ्यः॒ । ए॒व । प्रजा॑तेभ्य॒ इति॒ प्र - जा॒ते॒भ्यः॒ । प्र॒ति॒ष्ठामिति॑ प्रति - स्थाम् । द॒धा॒ति॒ । कः । अ॒र्॒.ह॒ति॒ । स॒हस्र᳚म् । प॒शून् । प्राप्तु॒मिति॒ प्र - आ॒प्तु॒म् । इति॑ । आ॒हुः॒ । अ॒हो॒रा॒त्राणीत्य॑हः - रा॒त्राणि॑ । ए॒व । स॒हस्र᳚म् । स॒म्पाद्येति॑ सं - पाद्य॑ । एति॑ । ल॒भे॒त॒ । प॒शवः॑ ।  \newline


\textbf{Krama Paata} \newline

भ॒व॒ति॒ सा॒ह॒स्री । सा॒ह॒स्री वै । वा ए॒षा । ए॒षा ल॒क्ष्मी । ल॒क्ष्मी यत् । यदु॑न्न॒तः । उ॒न्न॒तो ल॒क्ष्मिया᳚ । उ॒न्न॒त इत्यु॑त् - न॒तः । ल॒क्ष्मियै॒व । ए॒व प॒शून् । प॒शूनव॑ । अव॑ रुन्धे । रु॒न्धे॒ य॒दा । य॒दा स॒हस्र᳚म् । स॒हस्र॑म् प॒शून् । प॒शून् प्रा᳚प्नु॒यात् । प्रा॒प्नु॒यादथ॑ । प्रा॒प्नु॒यादिति॑ प्र - आ॒प्नु॒यात् । अथ॑ वैष्ण॒वम् । वै॒ष्ण॒वं ॅवा॑म॒नम् । वा॒म॒नमा । आ ल॑भेत । ल॒भे॒तै॒तस्मिन्न्॑ । ए॒तस्मि॒न् वै । वै तत् । तथ् स॒हस्र᳚म् । स॒हस्र॒मधि॑ । अद्ध्य॑तिष्ठत् । अ॒ति॒ष्ठ॒त् तस्मा᳚त् । तस्मा॑दे॒षः । ए॒ष वा॑म॒नः । वा॒म॒नः समी॑षितः । समी॑षितः प॒शुभ्यः॑ । समी॑षित॒ इति॒ सं - ई॒षि॒तः॒ । प॒शुभ्य॑ ए॒व । प॒शुभ्य॒ इति॑ प॒शु - भ्यः॒ । ए॒व प्रजा॑तेभ्यः । प्रजा॑तेभ्यः प्रति॒ष्ठाम् । प्रजा॑तेभ्य॒ इति॒ प्र - जा॒ते॒भ्यः॒ । प्र॒ति॒ष्ठाम् द॑धाति । प्र॒ति॒ष्ठामिति॑ प्रति - स्थाम् । द॒धा॒ति॒ कः । को॑ऽर्.हति । अ॒र्॒.ह॒ति॒ स॒हस्र᳚म् । स॒हस्र॑म् प॒शून् । प॒शून् प्राप्तु᳚म् । प्राप्तु॒मिति॑ । प्राप्तु॒मिति॒ प्र - आ॒प्तु॒म् । इत्या॑हुः । आ॒हु॒र॒हो॒रा॒त्राणि॑ । अ॒हो॒रा॒त्राण्ये॒व । अ॒हो॒रा॒त्राणीत्य॑हः - रा॒त्राणि॑ । ए॒व स॒हस्र᳚म् । स॒हस्रꣳ॑ स॒म्पाद्य॑ । स॒म्पाद्या । स॒म्पाद्येति॑ सं - पाद्य॑ । आ ल॑भेत । ल॒भे॒त॒ प॒शवः॑ \newline

\textbf{Jatai Paata} \newline

1. भ॒व॒ति॒ सा॒ह॒स्री सा॑ह॒स्री भ॑वति भवति साह॒स्री । \newline
2. सा॒ह॒स्री वै वै सा॑ह॒स्री सा॑ह॒स्री वै । \newline
3. वा ए॒षैषा वै वा ए॒षा । \newline
4. ए॒षा ल॒क्ष्मी ल॒क्ष्म्ये॑षैषा ल॒क्ष्मी । \newline
5. ल॒क्ष्मी यद् यल्ल॒क्ष्मी ल॒क्ष्मी यत् । \newline
6. यदु॑न्न॒त उ॑न्न॒तो यद् यदु॑न्न॒तः । \newline
7. उ॒न्न॒तो ल॒क्ष्मिया॑ ल॒क्ष्मियो᳚न्न॒त उ॑न्न॒तो ल॒क्ष्मिया᳚ । \newline
8. उ॒न्न॒त इत्यु॑त् - न॒तः । \newline
9. ल॒क्ष्मियै॒वैव ल॒क्ष्मिया॑ ल॒क्ष्मियै॒व । \newline
10. ए॒व प॒शून् प॒शू ने॒वैव प॒शून् । \newline
11. प॒शू नवाव॑ प॒शून् प॒शू नव॑ । \newline
12. अव॑ रुन्धे रु॒न्धे ऽवाव॑ रुन्धे । \newline
13. रु॒न्धे॒ य॒दा य॒दा रु॑न्धे रुन्धे य॒दा । \newline
14. य॒दा स॒हस्रꣳ॑ स॒हस्रं॑ ॅय॒दा य॒दा स॒हस्र᳚म् । \newline
15. स॒हस्र॑म् प॒शून् प॒शून् थ्स॒हस्रꣳ॑ स॒हस्र॑म् प॒शून् । \newline
16. प॒शून् प्रा᳚प्नु॒यात् प्रा᳚प्नु॒यात् प॒शून् प॒शून् प्रा᳚प्नु॒यात् । \newline
17. प्रा॒प्नु॒या दथाथ॑ प्राप्नु॒यात् प्रा᳚प्नु॒या दथ॑ । \newline
18. प्रा॒प्नु॒यादिति॑ प्र - आ॒प्नु॒यात् । \newline
19. अथ॑ वैष्ण॒वं ॅवै᳚ष्ण॒व मथाथ॑ वैष्ण॒वम् । \newline
20. वै॒ष्ण॒वं ॅवा॑म॒नं ॅवा॑म॒नं ॅवै᳚ष्ण॒वं ॅवै᳚ष्ण॒वं ॅवा॑म॒नम् । \newline
21. वा॒म॒न मा वा॑म॒नं ॅवा॑म॒न मा । \newline
22. आ ल॑भेत लभे॒ता ल॑भेत । \newline
23. ल॒भे॒ तै॒तस्मि॑न् ने॒तस्मि॑न् ॅलभेत लभे तै॒तस्मिन्न्॑ । \newline
24. ए॒तस्मि॒न् वै वा ए॒तस्मि॑न् ने॒तस्मि॒न् वै । \newline
25. वै तत् तद् वै वै तत् । \newline
26. तथ् स॒हस्रꣳ॑ स॒हस्र॒म् तत् तथ् स॒हस्र᳚म् । \newline
27. स॒हस्र॒ मध्यधि॑ स॒हस्रꣳ॑ स॒हस्र॒ मधि॑ । \newline
28. अध्य॑तिष्ठ दतिष्ठ॒ दध्य ध्य॑तिष्ठत् । \newline
29. अ॒ति॒ष्ठ॒त् तस्मा॒त् तस्मा॑ दतिष्ठ दतिष्ठ॒त् तस्मा᳚त् । \newline
30. तस्मा॑ दे॒ष ए॒ष तस्मा॒त् तस्मा॑ दे॒षः । \newline
31. ए॒ष वा॑म॒नो वा॑म॒न ए॒ष ए॒ष वा॑म॒नः । \newline
32. वा॒म॒नः समी॑षितः॒ समी॑षितो वाम॒नो वा॑म॒नः समी॑षितः । \newline
33. समी॑षितः प॒शुभ्यः॑ प॒शुभ्यः॒ समी॑षितः॒ समी॑षितः प॒शुभ्यः॑ । \newline
34. समी॑षित॒ इति॒ सं - ई॒षि॒तः॒ । \newline
35. प॒शुभ्य॑ ए॒वैव प॒शुभ्यः॑ प॒शुभ्य॑ ए॒व । \newline
36. प॒शुभ्य॒ इति॑ प॒शु - भ्यः॒ । \newline
37. ए॒व प्रजा॑तेभ्यः॒ प्रजा॑तेभ्य ए॒वैव प्रजा॑तेभ्यः । \newline
38. प्रजा॑तेभ्यः प्रति॒ष्ठाम् प्र॑ति॒ष्ठाम् प्रजा॑तेभ्यः॒ प्रजा॑तेभ्यः प्रति॒ष्ठाम् । \newline
39. प्रजा॑तेभ्य॒ इति॒ प्र - जा॒ते॒भ्यः॒ । \newline
40. प्र॒ति॒ष्ठाम् द॑धाति दधाति प्रति॒ष्ठाम् प्र॑ति॒ष्ठाम् द॑धाति । \newline
41. प्र॒ति॒ष्ठामिति॑ प्रति - स्थाम् । \newline
42. द॒धा॒ति॒ कः को द॑धाति दधाति॒ कः । \newline
43. को॑ऽर्. हत्यर्.हति॒ कः को॑ऽर्.हति । \newline
44. अ॒र्॒.ह॒ति॒ स॒हस्रꣳ॑ स॒हस्र॑ मर्.हत्यर्.हति स॒हस्र᳚म् । \newline
45. स॒हस्र॑म् प॒शून् प॒शून् थ्स॒हस्रꣳ॑ स॒हस्र॑म् प॒शून् । \newline
46. प॒शून् प्राप्तु॒म् प्राप्तु॑म् प॒शून् प॒शून् प्राप्तु᳚म् । \newline
47. प्राप्तु॒ मितीति॒ प्राप्तु॒म् प्राप्तु॒ मिति॑ । \newline
48. प्राप्तु॒मिति॒ प्र - आ॒प्तु॒म् । \newline
49. इत्या॑हु राहु॒ रिती त्या॑हुः । \newline
50. आ॒हु॒ र॒हो॒रा॒त्रा ण्य॑होरा॒त्रा ण्या॑हु राहु रहोरा॒त्राणि॑ । \newline
51. अ॒हो॒रा॒त्रा ण्ये॒वैवा हो॑रा॒त्रा ण्य॑होरा॒त्रा ण्ये॒व । \newline
52. अ॒हो॒रा॒त्राणीत्य॑हः - रा॒त्राणि॑ । \newline
53. ए॒व स॒हस्रꣳ॑ स॒हस्र॑ मे॒वैव स॒हस्र᳚म् । \newline
54. स॒हस्रꣳ॑ स॒म्पाद्य॑ स॒म्पाद्य॑ स॒हस्रꣳ॑ स॒हस्रꣳ॑ स॒म्पाद्य॑ । \newline
55. स॒म्पाद्या स॒म्पाद्य॑ स॒म्पाद्या । \newline
56. स॒म्पाद्येति॑ सं - पाद्य॑ । \newline
57. आ ल॑भेत लभे॒ता ल॑भेत । \newline
58. ल॒भे॒त॒ प॒शवः॑ प॒शवो॑ लभेत लभेत प॒शवः॑ । \newline
59. प॒शवो॒ वै वै प॒शवः॑ प॒शवो॒ वै । \newline

\textbf{Ghana Paata } \newline

1. भ॒व॒ति॒ सा॒ह॒स्री सा॑ह॒स्री भ॑वति भवति साह॒स्री वै वै सा॑ह॒स्री भ॑वति भवति साह॒स्री वै । \newline
2. सा॒ह॒स्री वै वै सा॑ह॒स्री सा॑ह॒स्री वा ए॒षैषा वै सा॑ह॒स्री सा॑ह॒स्री वा ए॒षा । \newline
3. वा ए॒षैषा वै वा ए॒षा ल॒क्ष्मी ल॒क्ष्म्ये॑षा वै वा ए॒षा ल॒क्ष्मी । \newline
4. ए॒षा ल॒क्ष्मी ल॒क्ष्म्ये॑षैषा ल॒क्ष्मी यद् यल्ल॒क्ष्म्ये॑ षैषा ल॒क्ष्मी यत् । \newline
5. ल॒क्ष्मी यद् यल्ल॒क्ष्मी ल॒क्ष्मी यदु॑न्न॒त उ॑न्न॒तो यल्ल॒क्ष्मी ल॒क्ष्मी यदु॑न्न॒तः । \newline
6. यदु॑न्न॒त उ॑न्न॒तो यद् यदु॑न्न॒तो ल॒क्ष्मिया॑ ल॒क्ष्मियो᳚ न्न॒तो यद् यदु॑न्न॒तो ल॒क्ष्मिया᳚ । \newline
7. उ॒न्न॒तो ल॒क्ष्मिया॑ ल॒क्ष्मियो᳚ न्न॒त उ॑न्न॒तो ल॒क्ष्मि यै॒वैव ल॒क्ष्मियो᳚न्न॒त उ॑न्न॒तो ल॒क्ष्मियै॒व । \newline
8. उ॒न्न॒त इत्यु॑त् - न॒तः । \newline
9. ल॒क्ष्मियै॒वैव ल॒क्ष्मिया॑ ल॒क्ष्मियै॒व प॒शून् प॒शू ने॒व ल॒क्ष्मिया॑ ल॒क्ष्मियै॒व प॒शून् । \newline
10. ए॒व प॒शून् प॒शू ने॒वैव प॒शू नवाव॑ प॒शू ने॒वैव प॒शू नव॑ । \newline
11. प॒शू नवाव॑ प॒शून् प॒शू नव॑ रुन्धे रु॒न्धे ऽव॑ प॒शून् प॒शू नव॑ रुन्धे । \newline
12. अव॑ रुन्धे रु॒न्धे ऽवाव॑ रुन्धे य॒दा य॒दा रु॒न्धे ऽवाव॑ रुन्धे य॒दा । \newline
13. रु॒न्धे॒ य॒दा य॒दा रु॑न्धे रुन्धे य॒दा स॒हस्रꣳ॑ स॒हस्रं॑ ॅय॒दा रु॑न्धे रुन्धे य॒दा स॒हस्र᳚म् । \newline
14. य॒दा स॒हस्रꣳ॑ स॒हस्रं॑ ॅय॒दा य॒दा स॒हस्र॑म् प॒शून् प॒शून् थ्स॒हस्रं॑ ॅय॒दा य॒दा स॒हस्र॑म् प॒शून् । \newline
15. स॒हस्र॑म् प॒शून् प॒शून् थ्स॒हस्रꣳ॑ स॒हस्र॑म् प॒शून् प्रा᳚प्नु॒यात् प्रा᳚प्नु॒यात् प॒शून् थ्स॒हस्रꣳ॑ स॒हस्र॑म् प॒शून् प्रा᳚प्नु॒यात् । \newline
16. प॒शून् प्रा᳚प्नु॒यात् प्रा᳚प्नु॒यात् प॒शून् प॒शून् प्रा᳚प्नु॒याद थाथ॑ प्राप्नु॒यात् प॒शून् प॒शून् प्रा᳚प्नु॒यादथ॑ । \newline
17. प्रा॒प्नु॒याद थाथ॑ प्राप्नु॒यात् प्रा᳚प्नु॒यादथ॑ वैष्ण॒वं ॅवै᳚ष्ण॒व मथ॑ प्राप्नु॒यात् प्रा᳚प्नु॒यादथ॑ वैष्ण॒वम् । \newline
18. प्रा॒प्नु॒यादिति॑ प्र - आ॒प्नु॒यात् । \newline
19. अथ॑ वैष्ण॒वं ॅवै᳚ष्ण॒व मथाथ॑ वैष्ण॒वं ॅवा॑म॒नं ॅवा॑म॒नं ॅवै᳚ष्ण॒व मथाथ॑ वैष्ण॒वं ॅवा॑म॒नम् । \newline
20. वै॒ष्ण॒वं ॅवा॑म॒नं ॅवा॑म॒नं ॅवै᳚ष्ण॒वं ॅवै᳚ष्ण॒वं ॅवा॑म॒न मा वा॑म॒नं ॅवै᳚ष्ण॒वं ॅवै᳚ष्ण॒वं ॅवा॑म॒न मा । \newline
21. वा॒म॒न मा वा॑म॒नं ॅवा॑म॒न मा ल॑भेत लभे॒ता वा॑म॒नं ॅवा॑म॒न मा ल॑भेत । \newline
22. आ ल॑भेत लभे॒ता ल॑भेतै॒ तस्मि॑न् ने॒तस्मि॑न् ॅलभे॒ता ल॑भेतै॒ तस्मिन्न्॑ । \newline
23. ल॒भे॒तै॒ तस्मि॑न् ने॒तस्मि॑न् ॅलभेत लभेतै॒ तस्मि॒न् वै वा ए॒तस्मि॑न् ॅलभेत लभेतै॒तस्मि॒न् वै । \newline
24. ए॒तस्मि॒न् वै वा ए॒तस्मि॑न् ने॒तस्मि॒न् वै तत् तद् वा ए॒तस्मि॑न् ने॒तस्मि॒न् वै तत् । \newline
25. वै तत् तद् वै वै तथ् स॒हस्रꣳ॑ स॒हस्र॒म् तद् वै वै तथ् स॒हस्र᳚म् । \newline
26. तथ् स॒हस्रꣳ॑ स॒हस्र॒म् तत् तथ् स॒हस्र॒ मध्यधि॑ स॒हस्र॒म् तत् तथ् स॒हस्र॒ मधि॑ । \newline
27. स॒हस्र॒ मध्यधि॑ स॒हस्रꣳ॑ स॒हस्र॒ मध्य॑तिष्ठ दतिष्ठ॒दधि॑ स॒हस्रꣳ॑ स॒हस्र॒ मध्य॑तिष्ठत् । \newline
28. अध्य॑तिष्ठ दतिष्ठ॒ दध्यध्य॑ तिष्ठ॒त् तस्मा॒त् तस्मा॑ दतिष्ठ॒ दध्यध्य॑ तिष्ठ॒त् तस्मा᳚त् । \newline
29. अ॒ति॒ष्ठ॒त् तस्मा॒त् तस्मा॑ दतिष्ठ दतिष्ठ॒त् तस्मा॑दे॒ष ए॒ष तस्मा॑ दतिष्ठ दतिष्ठ॒त् तस्मा॑दे॒षः । \newline
30. तस्मा॑दे॒ष ए॒ष तस्मा॒त् तस्मा॑दे॒ष वा॑म॒नो वा॑म॒न ए॒ष तस्मा॒त् तस्मा॑दे॒ष वा॑म॒नः । \newline
31. ए॒ष वा॑म॒नो वा॑म॒न ए॒ष ए॒ष वा॑म॒नः समी॑षितः॒ समी॑षितो वाम॒न ए॒ष ए॒ष वा॑म॒नः समी॑षितः । \newline
32. वा॒म॒नः समी॑षितः॒ समी॑षितो वाम॒नो वा॑म॒नः समी॑षितः प॒शुभ्यः॑ प॒शुभ्यः॒ समी॑षितो वाम॒नो वा॑म॒नः समी॑षितः प॒शुभ्यः॑ । \newline
33. समी॑षितः प॒शुभ्यः॑ प॒शुभ्यः॒ समी॑षितः॒ समी॑षितः प॒शुभ्य॑ ए॒वैव प॒शुभ्यः॒ समी॑षितः॒ समी॑षितः प॒शुभ्य॑ ए॒व । \newline
34. समी॑षित॒ इति॒ सं - ई॒षि॒तः॒ । \newline
35. प॒शुभ्य॑ ए॒वैव प॒शुभ्यः॑ प॒शुभ्य॑ ए॒व प्रजा॑तेभ्यः॒ प्रजा॑तेभ्य ए॒व प॒शुभ्यः॑ प॒शुभ्य॑ ए॒व प्रजा॑तेभ्यः । \newline
36. प॒शुभ्य॒ इति॑ प॒शु - भ्यः॒ । \newline
37. ए॒व प्रजा॑तेभ्यः॒ प्रजा॑तेभ्य ए॒वैव प्रजा॑तेभ्यः प्रति॒ष्ठाम् प्र॑ति॒ष्ठाम् प्रजा॑तेभ्य ए॒वैव प्रजा॑तेभ्यः प्रति॒ष्ठाम् । \newline
38. प्रजा॑तेभ्यः प्रति॒ष्ठाम् प्र॑ति॒ष्ठाम् प्रजा॑तेभ्यः॒ प्रजा॑तेभ्यः प्रति॒ष्ठाम् द॑धाति दधाति प्रति॒ष्ठाम् प्रजा॑तेभ्यः॒ प्रजा॑तेभ्यः प्रति॒ष्ठाम् द॑धाति । \newline
39. प्रजा॑तेभ्य॒ इति॒ प्र - जा॒ते॒भ्यः॒ । \newline
40. प्र॒ति॒ष्ठाम् द॑धाति दधाति प्रति॒ष्ठाम् प्र॑ति॒ष्ठाम् द॑धाति॒ कः को द॑धाति प्रति॒ष्ठाम् प्र॑ति॒ष्ठाम् द॑धाति॒ कः । \newline
41. प्र॒ति॒ष्ठामिति॑ प्रति - स्थाम् । \newline
42. द॒धा॒ति॒ कः को द॑धाति दधाति॒ को॑ऽर्.ह त्यर्.हति॒ को द॑धाति दधाति॒ को॑ऽर्.हति । \newline
43. को॑ऽर्.ह त्यर्.हति॒ कः को॑ ऽर्.हति स॒हस्रꣳ॑ स॒हस्र॑ मर्.हति॒ कः को॑ ऽर्.हति स॒हस्र᳚म् । \newline
44. अ॒र्॒.ह॒ति॒ स॒हस्रꣳ॑ स॒हस्र॑ मर्.ह त्यर्.हति स॒हस्र॑म् प॒शून् प॒शून् थ्स॒हस्र॑ मर्.ह त्यर्.हति स॒हस्र॑म् प॒शून् । \newline
45. स॒हस्र॑म् प॒शून् प॒शून् थ्स॒हस्रꣳ॑ स॒हस्र॑म् प॒शून् प्राप्तु॒म् प्राप्तु॑म् प॒शून् थ्स॒हस्रꣳ॑ स॒हस्र॑म् प॒शून् प्राप्तु᳚म् । \newline
46. प॒शून् प्राप्तु॒म् प्राप्तु॑म् प॒शून् प॒शून् प्राप्तु॒ मितीति॒ प्राप्तु॑म् प॒शून् प॒शून् प्राप्तु॒ मिति॑ । \newline
47. प्राप्तु॒ मितीति॒ प्राप्तु॒म् प्राप्तु॒ मित्या॑हु राहु॒रिति॒ प्राप्तु॒म् प्राप्तु॒ मित्या॑हुः । \newline
48. प्राप्तु॒मिति॒ प्र - आ॒प्तु॒म् । \newline
49. इत्या॑हु राहु॒रिती त्या॑हु रहोरा॒त्रा ण्य॑होरा॒त्रा ण्या॑हु॒ रितीत्या॑हु रहोरा॒त्राणि॑ । \newline
50. आ॒हु॒ र॒हो॒रा॒त्रा ण्य॑होरा॒त्रा ण्या॑हु राहु रहोरा॒त्रा ण्ये॒वैवा हो॑रा॒त्रा ण्या॑हु राहु रहोरा॒त्राण्ये॒व । \newline
51. अ॒हो॒ रा॒त्रा ण्ये॒वैवा हो॑रा॒त्रा ण्य॑होरा॒त्रा ण्ये॒व स॒हस्रꣳ॑ स॒हस्र॑ मे॒वा हो॑रा॒त्रा ण्य॑होरा॒त्रा ण्ये॒व स॒हस्र᳚म् । \newline
52. अ॒हो॒रा॒त्राणीत्य॑हः - रा॒त्राणि॑ । \newline
53. ए॒व स॒हस्रꣳ॑ स॒हस्र॑ मे॒वैव स॒हस्रꣳ॑ स॒म्पाद्य॑ स॒म्पाद्य॑ स॒हस्र॑ मे॒वैव स॒हस्रꣳ॑ स॒म्पाद्य॑ । \newline
54. स॒हस्रꣳ॑ स॒म्पाद्य॑ स॒म्पाद्य॑ स॒हस्रꣳ॑ स॒हस्रꣳ॑ स॒म्पाद्या स॒म्पाद्य॑ स॒हस्रꣳ॑ स॒हस्रꣳ॑ स॒म्पाद्या । \newline
55. स॒म्पाद्या स॒म्पाद्य॑ स॒म्पाद्या ल॑भेत लभे॒ता स॒म्पाद्य॑ स॒म्पाद्या ल॑भेत । \newline
56. स॒म्पाद्येति॑ सं - पाद्य॑ । \newline
57. आ ल॑भेत लभे॒ता ल॑भेत प॒शवः॑ प॒शवो॑ लभे॒ता ल॑भेत प॒शवः॑ । \newline
58. ल॒भे॒त॒ प॒शवः॑ प॒शवो॑ लभेत लभेत प॒शवो॒ वै वै प॒शवो॑ लभेत लभेत प॒शवो॒ वै । \newline
59. प॒शवो॒ वै वै प॒शवः॑ प॒शवो॒ वा अ॑होरा॒त्रा ण्य॑होरा॒त्राणि॒ वै प॒शवः॑ प॒शवो॒ वा अ॑होरा॒त्राणि॑ । \newline
\pagebreak
\markright{ TS 2.1.5.3  \hfill https://www.vedavms.in \hfill}

\section{ TS 2.1.5.3 }

\textbf{TS 2.1.5.3 } \newline
\textbf{Samhita Paata} \newline

वा अ॑होरा॒त्राणि॑ प॒शूने॒व प्रजा॑तान् प्रति॒ष्ठां ग॑मय॒-त्योष॑धीभ्यो वे॒हत॒मा ल॑भेत प्र॒जाका॑म॒ ओष॑धयो॒ वा ए॒तं प्र॒जायै॒ परि॑बाधन्ते॒ योऽलं॑ प्र॒जायै॒ सन् प्र॒जां न वि॒न्दत॒ ओष॑धयः॒ खलु॒ वा ए॒तस्यै॒ सूतु॒मपि॑ घ्नन्ति॒ या वे॒हद्-भव॒त्योष॑धीरे॒व स्वेन॑ भाग॒धेये॒नोप॑ धावति॒ ता ए॒वास्मै॒ स्वाद्योनेः᳚ प्र॒जां प्र ज॑नयन्ति वि॒न्दते᳚ - [  ] \newline

\textbf{Pada Paata} \newline

वै । अ॒हो॒रा॒त्राणीइत्य॑हः - रा॒त्राणि॑ । प॒शून् । ए॒व । प्रजा॑ता॒निति॒ प्र - जा॒ता॒न् । प्र॒ति॒ष्ठामिति॑ प्रति - स्थाम् । ग॒म॒य॒ति॒ । ओष॑धीभ्य॒ इत्योष॑धि - भ्यः॒ । वे॒हत᳚म् । एति॑ । ल॒भे॒त॒ । प्र॒जाका॑म॒ इति॑ प्र॒जा - का॒मः॒ । ओष॑धयः । वै । ए॒तम् । प्र॒जाया॒ इति॑ प्र - जायै᳚ । परीति॑ । बा॒ध॒न्ते॒ । यः । अल᳚म् । प्र॒जाया॒ इति॑ प्र - जायै᳚ । सन्न् । प्र॒जामिति॑ प्र-जाम् । न । वि॒न्दते᳚ । ओष॑धयः । खलु॑ । वै । ए॒तस्यै᳚ । सूतु᳚म् । अपीति॑ । घ्न॒न्ति॒ । या । वे॒हत् । भव॑ति । ओष॑धीः । ए॒व । स्वेन॑ । भा॒ग॒धेये॒नेति॑ भाग - धेये॑न । उपेति॑ । धा॒व॒ति॒ । ताः । ए॒व । अ॒स्मै॒ । स्वात् । योनेः᳚ । प्र॒जामिति॑ प्र - जाम् । प्रेति॑ । ज॒न॒य॒न्ति॒ । वि॒न्दते᳚ । 31(50)  \newline


\textbf{Krama Paata} \newline

प॒शवो॒ वै । वा अ॑होरा॒त्राणि॑ । अ॒हो॒रा॒त्राणि॑ प॒शून् । अ॒हो॒रा॒त्राणीत्य॑हः - रा॒त्राणि॑ । प॒शूने॒व । ए॒व प्रजा॑तान् । प्रजा॑तान् प्रति॒ष्ठाम् । प्रजा॑ता॒निति॒ प्र - जा॒ता॒न्॒ । प्र॒ति॒ष्ठाम् ग॑मयति । प्र॒ति॒ष्ठामिति॑ प्रति - स्थाम् । ग॒म॒य॒त्योष॑धीभ्यः । ओष॑धीभ्यो वे॒हत᳚म् । ओष॑धीभ्य॒ इत्योष॑धि - भ्यः॒ । वे॒हत॒मा । आ ल॑भेत । ल॒भे॒त॒ प्र॒जाका॑मः । प्र॒जाका॑म॒ ओष॑धयः । प्र॒जाका॑म॒ इति॑ प्र॒जा - का॒मः॒ । ओष॑धयो॒ वै । वा ए॒तम् । ए॒तम् प्र॒जायै᳚ । प्र॒जायै॒ परि॑ । प्र॒जाया॒ इति॑ प्र - जायै᳚ । परि॑ बाधन्ते । बा॒ध॒न्ते॒ यः । योऽल᳚म् । अल॑म् प्र॒जायै᳚ । प्र॒जायै॒ सन्न् । प्र॒जाया॒ इति॑ प्र - जायै᳚ । सन् प्र॒जाम् । प्र॒जाम् न । प्र॒जामिति॑ प्र - जाम् । न वि॒न्दते᳚ । वि॒न्दत॒ ओष॑धयः । ओष॑धयः॒ खलु॑ । खलु॒ वै । वा ए॒तस्यै᳚ । ए॒तस्यै॒ सूतु᳚म् । सूतु॒मपि॑ । अपि॑ घ्नन्ति । घ्न॒न्ति॒ या । या वे॒हत् । वे॒हद् भव॑ति । भव॒त्योष॑धीः । ओष॑धीरे॒व । ए॒व स्वेन॑ । स्वेन॑ भाग॒धेये॑न । भा॒ग॒धेये॒नोप॑ । भा॒ग॒धेये॒नेति॑ भाग - धेये॑न । उप॑ धावति । धा॒व॒ति॒ ताः । ता ए॒व । ए॒वास्मै᳚ । अ॒स्मै॒ स्वात् । स्वाद् योनेः᳚ । योनेः᳚ प्र॒जाम् । प्र॒जाम् प्र । प्र॒जामिति॑ प्र - जाम् । प्र ज॑नयन्ति । ज॒न॒य॒न्ति॒ वि॒न्दते᳚ । वि॒न्दते᳚ प्र॒जाम् \newline

\textbf{Jatai Paata} \newline

1. वा अ॑होरा॒त्रा ण्य॑होरा॒त्राणि॒ वै वा अ॑होरा॒त्राणि॑ । \newline
2. अ॒हो॒रा॒त्राणि॑ प॒शून् प॒शू न॑होरा॒त्रा ण्य॑होरा॒त्राणि॑ प॒शून् । \newline
3. अ॒हो॒रा॒त्राणीत्य॑हः - रा॒त्राणि॑ । \newline
4. प॒शू ने॒वैव प॒शून् प॒शू ने॒व । \newline
5. ए॒व प्रजा॑ता॒न् प्रजा॑ता ने॒वैव प्रजा॑तान् । \newline
6. प्रजा॑तान् प्रति॒ष्ठाम् प्र॑ति॒ष्ठाम् प्रजा॑ता॒न् प्रजा॑तान् प्रति॒ष्ठाम् । \newline
7. प्रजा॑ता॒निति॒ प्र - जा॒ता॒न् । \newline
8. प्र॒ति॒ष्ठाम् ग॑मयति गमयति प्रति॒ष्ठाम् प्र॑ति॒ष्ठाम् ग॑मयति । \newline
9. प्र॒ति॒ष्ठामिति॑ प्रति - स्थाम् । \newline
10. ग॒म॒य॒ त्योष॑धीभ्य॒ ओष॑धीभ्यो गमयति गमय॒ त्योष॑धीभ्यः । \newline
11. ओष॑धीभ्यो वे॒हतं॑ ॅवे॒हत॒ मोष॑धीभ्य॒ ओष॑धीभ्यो वे॒हत᳚म् । \newline
12. ओष॑धीभ्य॒इत्योष॑धि - भ्यः॒ । \newline
13. वे॒हत॒ मा वे॒हतं॑ ॅवे॒हत॒ मा । \newline
14. आ ल॑भेत लभे॒ता ल॑भेत । \newline
15. ल॒भे॒त॒ प्र॒जाका॑मः प्र॒जाका॑मो लभेत लभेत प्र॒जाका॑मः । \newline
16. प्र॒जाका॑म॒ ओष॑धय॒ ओष॑धयः प्र॒जाका॑मः प्र॒जाका॑म॒ ओष॑धयः । \newline
17. प्र॒जाका॑म॒ इति॑ प्र॒जा - का॒मः॒ । \newline
18. ओष॑धयो॒ वै वा ओष॑धय॒ ओष॑धयो॒ वै । \newline
19. वा ए॒त मे॒तं ॅवै वा ए॒तम् । \newline
20. ए॒तम् प्र॒जायै᳚ प्र॒जाया॑ ए॒त मे॒तम् प्र॒जायै᳚ । \newline
21. प्र॒जायै॒ परि॒ परि॑ प्र॒जायै᳚ प्र॒जायै॒ परि॑ । \newline
22. प्र॒जाया॒ इति॑ प्र - जायै᳚ । \newline
23. परि॑ बाधन्ते बाधन्ते॒ परि॒ परि॑ बाधन्ते । \newline
24. बा॒ध॒न्ते॒ यो यो बा॑धन्ते बाधन्ते॒ यः । \newline
25. यो ऽल॒ मलं॒ ॅयो यो ऽल᳚म् । \newline
26. अल॑म् प्र॒जायै᳚ प्र॒जाया॒ अल॒ मल॑म् प्र॒जायै᳚ । \newline
27. प्र॒जायै॒ सन् थ्सन् प्र॒जायै᳚ प्र॒जायै॒ सन्न् । \newline
28. प्र॒जाया॒ इति॑ प्र - जायै᳚ । \newline
29. सन् प्र॒जाम् प्र॒जाꣳ सन् थ्सन् प्र॒जाम् । \newline
30. प्र॒जाम् न न प्र॒जाम् प्र॒जाम् न । \newline
31. प्र॒जामिति॑ प्र - जाम् । \newline
32. न वि॒न्दते॑ वि॒न्दते॒ न न वि॒न्दते᳚ । \newline
33. वि॒न्दत॒ ओष॑धय॒ ओष॑धयो वि॒न्दते॑ वि॒न्दत॒ ओष॑धयः । \newline
34. ओष॑धयः॒ खलु॒ खल्वोष॑धय॒ ओष॑धयः॒ खलु॑ । \newline
35. खलु॒ वै वै खलु॒ खलु॒ वै । \newline
36. वा ए॒तस्या॑ ए॒तस्यै॒ वै वा ए॒तस्यै᳚ । \newline
37. ए॒तस्यै॒ सूतुꣳ॒॒ सूतु॑ मे॒तस्या॑ ए॒तस्यै॒ सूतु᳚म् । \newline
38. सूतु॒ मप्यपि॒ सूतुꣳ॒॒ सूतु॒ मपि॑ । \newline
39. अपि॑ घ्नन्ति घ्न॒न्त्यप्यपि॑ घ्नन्ति । \newline
40. घ्न॒न्ति॒ या या घ्न॑न्ति घ्नन्ति॒ या । \newline
41. या वे॒हद् वे॒हद् या या वे॒हत् । \newline
42. वे॒हद् भव॑ति॒ भव॑ति वे॒हद् वे॒हद् भव॑ति । \newline
43. भव॒ त्योष॑धी॒ रोष॑धी॒र् भव॑ति॒ भव॒ त्योष॑धीः । \newline
44. ओष॑धी रे॒वैवौष॑धी॒ रोष॑धी रे॒व । \newline
45. ए॒व स्वेन॒ स्वेनै॒ वैव स्वेन॑ । \newline
46. स्वेन॑ भाग॒धेये॑न भाग॒धेये॑न॒ स्वेन॒ स्वेन॑ भाग॒धेये॑न । \newline
47. भा॒ग॒धेये॒नोपोप॑ भाग॒धेये॑न भाग॒धेये॒नोप॑ । \newline
48. भा॒ग॒धेये॒नेति॑ भाग - धेये॑न । \newline
49. उप॑ धावति धाव॒ त्युपोप॑ धावति । \newline
50. धा॒व॒ति॒ तास्ता धा॑वति धावति॒ ताः । \newline
51. ता ए॒वैव ता स्ता ए॒व । \newline
52. ए॒वास्मा॑ अस्मा ए॒वैवास्मै᳚ । \newline
53. अ॒स्मै॒ स्वाथ् स्वा द॑स्मा अस्मै॒ स्वात् । \newline
54. स्वाद् योने॒र् योनेः॒ स्वाथ् स्वाद् योनेः᳚ । \newline
55. योनेः᳚ प्र॒जाम् प्र॒जां ॅयोने॒र् योनेः᳚ प्र॒जाम् । \newline
56. प्र॒जाम् प्र प्र प्र॒जाम् प्र॒जाम् प्र । \newline
57. प्र॒जामिति॑ प्र - जाम् । \newline
58. प्र ज॑नयन्ति जनयन्ति॒ प्र प्र ज॑नयन्ति । \newline
59. ज॒न॒य॒न्ति॒ वि॒न्दते॑ वि॒न्दते॑ जनयन्ति जनयन्ति वि॒न्दते᳚ । \newline
60. वि॒न्दते᳚ प्र॒जाम् प्र॒जां ॅवि॒न्दते॑ वि॒न्दते᳚ प्र॒जाम् । \newline

\textbf{Ghana Paata } \newline

1. वा अ॑होरा॒त्रा ण्य॑होरा॒त्राणि॒ वै वा अ॑होरा॒त्राणि॑ प॒शून् प॒शू न॑होरा॒त्राणि॒ वै वा अ॑होरा॒त्राणि॑ प॒शून् । \newline
2. अ॒हो॒रा॒त्राणि॑ प॒शून् प॒शू न॑होरा॒त्रा ण्य॑होरा॒त्राणि॑ प॒शू ने॒वैव प॒शू न॑होरा॒त्रा ण्य॑होरा॒त्राणि॑ प॒शू ने॒व । \newline
3. अ॒हो॒रा॒त्राणीत्य॑हः - रा॒त्राणि॑ । \newline
4. प॒शू ने॒वैव प॒शून् प॒शू ने॒व प्रजा॑ता॒न् प्रजा॑ता ने॒व प॒शून् प॒शू ने॒व प्रजा॑तान् । \newline
5. ए॒व प्रजा॑ता॒न् प्रजा॑ता ने॒वैव प्रजा॑तान् प्रति॒ष्ठाम् प्र॑ति॒ष्ठाम् प्रजा॑ता ने॒वैव प्रजा॑तान् प्रति॒ष्ठाम् । \newline
6. प्रजा॑तान् प्रति॒ष्ठाम् प्र॑ति॒ष्ठाम् प्रजा॑ता॒न् प्रजा॑तान् प्रति॒ष्ठाम् ग॑मयति गमयति प्रति॒ष्ठाम् प्रजा॑ता॒न् प्रजा॑तान् प्रति॒ष्ठाम् ग॑मयति । \newline
7. प्रजा॑ता॒निति॒ प्र - जा॒ता॒न् । \newline
8. प्र॒ति॒ष्ठाम् ग॑मयति गमयति प्रति॒ष्ठाम् प्र॑ति॒ष्ठाम् ग॑मय॒ त्योष॑धीभ्य॒ ओष॑धीभ्यो गमयति प्रति॒ष्ठाम् प्र॑ति॒ष्ठाम् ग॑मय॒ त्योष॑धीभ्यः । \newline
9. प्र॒ति॒ष्ठामिति॑ प्रति - स्थाम् । \newline
10. ग॒म॒य॒ त्योष॑धीभ्य॒ ओष॑धीभ्यो गमयति गमय॒ त्योष॑धीभ्यो वे॒हतं॑ ॅवे॒हत॒ मोष॑धीभ्यो गमयति गमय॒ त्योष॑धीभ्यो वे॒हत᳚म् । \newline
11. ओष॑धीभ्यो वे॒हतं॑ ॅवे॒हत॒ मोष॑धीभ्य॒ ओष॑धीभ्यो वे॒हत॒ मा वे॒हत॒ मोष॑धीभ्य॒ ओष॑धीभ्यो वे॒हत॒ मा । \newline
12. ओष॑धीभ्य॒इत्योष॑धि - भ्यः॒ । \newline
13. वे॒हत॒ मा वे॒हतं॑ ॅवे॒हत॒ मा ल॑भेत लभे॒ता वे॒हतं॑ ॅवे॒हत॒ मा ल॑भेत । \newline
14. आ ल॑भेत लभे॒ता ल॑भेत प्र॒जाका॑मः प्र॒जाका॑मो लभे॒ता ल॑भेत प्र॒जाका॑मः । \newline
15. ल॒भे॒त॒ प्र॒जाका॑मः प्र॒जाका॑मो लभेत लभेत प्र॒जाका॑म॒ ओष॑धय॒ ओष॑धयः प्र॒जाका॑मो लभेत लभेत प्र॒जाका॑म॒ ओष॑धयः । \newline
16. प्र॒जाका॑म॒ ओष॑धय॒ ओष॑धयः प्र॒जाका॑मः प्र॒जाका॑म॒ ओष॑धयो॒ वै वा ओष॑धयः प्र॒जाका॑मः प्र॒जाका॑म॒ ओष॑धयो॒ वै । \newline
17. प्र॒जाका॑म॒ इति॑ प्र॒जा - का॒मः॒ । \newline
18. ओष॑धयो॒ वै वा ओष॑धय॒ ओष॑धयो॒ वा ए॒त मे॒तं ॅवा ओष॑धय॒ ओष॑धयो॒ वा ए॒तम् । \newline
19. वा ए॒त मे॒तं ॅवै वा ए॒तम् प्र॒जायै᳚ प्र॒जाया॑ ए॒तं ॅवै वा ए॒तम् प्र॒जायै᳚ । \newline
20. ए॒तम् प्र॒जायै᳚ प्र॒जाया॑ ए॒त मे॒तम् प्र॒जायै॒ परि॒ परि॑ प्र॒जाया॑ ए॒त मे॒तम् प्र॒जायै॒ परि॑ । \newline
21. प्र॒जायै॒ परि॒ परि॑ प्र॒जायै᳚ प्र॒जायै॒ परि॑ बाधन्ते बाधन्ते॒ परि॑ प्र॒जायै᳚ प्र॒जायै॒ परि॑ बाधन्ते । \newline
22. प्र॒जाया॒ इति॑ प्र - जायै᳚ । \newline
23. परि॑ बाधन्ते बाधन्ते॒ परि॒ परि॑ बाधन्ते॒ यो यो बा॑धन्ते॒ परि॒ परि॑ बाधन्ते॒ यः । \newline
24. बा॒ध॒न्ते॒ यो यो बा॑धन्ते बाधन्ते॒ यो ऽल॒ मलं॒ ॅयो बा॑धन्ते बाधन्ते॒ यो ऽल᳚म् । \newline
25. यो ऽल॒ मलं॒ ॅयो यो ऽल॑म् प्र॒जायै᳚ प्र॒जाया॒ अलं॒ ॅयो यो ऽल॑म् प्र॒जायै᳚ । \newline
26. अल॑म् प्र॒जायै᳚ प्र॒जाया॒ अल॒ मल॑म् प्र॒जायै॒ सन् थ्सन् प्र॒जाया॒ अल॒ मल॑म् प्र॒जायै॒ सन्न् । \newline
27. प्र॒जायै॒ सन् थ्सन् प्र॒जायै᳚ प्र॒जायै॒ सन् प्र॒जाम् प्र॒जाꣳ सन् प्र॒जायै᳚ प्र॒जायै॒ सन् प्र॒जाम् । \newline
28. प्र॒जाया॒ इति॑ प्र - जायै᳚ । \newline
29. सन् प्र॒जाम् प्र॒जाꣳ सन् थ्सन् प्र॒जान्न न प्र॒जाꣳ सन् थ्सन् प्र॒जान्न । \newline
30. प्र॒जान्न न प्र॒जाम् प्र॒जान्न वि॒न्दते॑ वि॒न्दते॒ न प्र॒जाम् प्र॒जान्न वि॒न्दते᳚ । \newline
31. प्र॒जामिति॑ प्र - जाम् । \newline
32. न वि॒न्दते॑ वि॒न्दते॒ न न वि॒न्दत॒ ओष॑धय॒ ओष॑धयो वि॒न्दते॒ न न वि॒न्दत॒ ओष॑धयः । \newline
33. वि॒न्दत॒ ओष॑धय॒ ओष॑धयो वि॒न्दते॑ वि॒न्दत॒ ओष॑धयः॒ खलु॒ खल्वोष॑धयो वि॒न्दते॑ वि॒न्दत॒ ओष॑धयः॒ खलु॑ । \newline
34. ओष॑धयः॒ खलु॒ खल्वोष॑धय॒ ओष॑धयः॒ खलु॒ वै वै खल्वोष॑धय॒ ओष॑धयः॒ खलु॒ वै । \newline
35. खलु॒ वै वै खलु॒ खलु॒ वा ए॒तस्या॑ ए॒तस्यै॒ वै खलु॒ खलु॒ वा ए॒तस्यै᳚ । \newline
36. वा ए॒तस्या॑ ए॒तस्यै॒ वै वा ए॒तस्यै॒ सूतुꣳ॒॒ सूतु॑ मे॒तस्यै॒ वै वा ए॒तस्यै॒ सूतु᳚म् । \newline
37. ए॒तस्यै॒ सूतुꣳ॒॒ सूतु॑ मे॒तस्या॑ ए॒तस्यै॒ सूतु॒ मप्यपि॒ सूतु॑ मे॒तस्या॑ ए॒तस्यै॒ सूतु॒ मपि॑ । \newline
38. सूतु॒ मप्यपि॒ सूतुꣳ॒॒ सूतु॒ मपि॑ घ्नन्ति घ्न॒न्त्यपि॒ सूतुꣳ॒॒ सूतु॒ मपि॑ घ्नन्ति । \newline
39. अपि॑ घ्नन्ति घ्न॒न्त्यप्यपि॑ घ्नन्ति॒ या या घ्न॒न्त्यप्यपि॑ घ्नन्ति॒ या । \newline
40. घ्न॒न्ति॒ या या घ्न॑न्ति घ्नन्ति॒ या वे॒हद् वे॒हद् या घ्न॑न्ति घ्नन्ति॒ या वे॒हत् । \newline
41. या वे॒हद् वे॒हद् या या वे॒हद् भव॑ति॒ भव॑ति वे॒हद् या या वे॒हद् भव॑ति । \newline
42. वे॒हद् भव॑ति॒ भव॑ति वे॒हद् वे॒हद् भव॒ त्योष॑धी॒ रोष॑धी॒र् भव॑ति वे॒हद् वे॒हद् भव॒त्योष॑धीः । \newline
43. भव॒ त्योष॑धी॒ रोष॑धी॒र् भव॑ति॒ भव॒ त्योष॑धी रे॒वै वौष॑धी॒र् भव॑ति॒ भव॒ त्योष॑धी रे॒व । \newline
44. ओष॑धी रे॒वै वौष॑धी॒ रोष॑धी रे॒व स्वेन॒ स्वेनै॒ वौष॑धी॒ रोष॑धी रे॒व स्वेन॑ । \newline
45. ए॒व स्वेन॒ स्वेनै॒वैव स्वेन॑ भाग॒धेये॑न भाग॒धेये॑न॒ स्वेनै॒वैव स्वेन॑ भाग॒धेये॑न । \newline
46. स्वेन॑ भाग॒धेये॑न भाग॒धेये॑न॒ स्वेन॒ स्वेन॑ भाग॒धेये॒नो पोप॑ भाग॒धेये॑न॒ स्वेन॒ स्वेन॑ भाग॒धेये॒नोप॑ । \newline
47. भा॒ग॒धेये॒नो पोप॑ भाग॒धेये॑न भाग॒धेये॒नोप॑ धावति धाव॒त्युप॑ भाग॒धेये॑न भाग॒धेये॒नोप॑ धावति । \newline
48. भा॒ग॒धेये॒नेति॑ भाग - धेये॑न । \newline
49. उप॑ धावति धाव॒ त्युपोप॑ धावति॒ ता स्ता धा॑व॒ त्युपोप॑ धावति॒ ताः । \newline
50. धा॒व॒ति॒ ता स्ता धा॑वति धावति॒ ता ए॒वैव ता धा॑वति धावति॒ ता ए॒व । \newline
51. ता ए॒वैव ता स्ता ए॒वास्मा॑ अस्मा ए॒व ता स्ता ए॒वास्मै᳚ । \newline
52. ए॒वास्मा॑ अस्मा ए॒वैवास्मै॒ स्वाथ् स्वाद॑स्मा ए॒वैवास्मै॒ स्वात् । \newline
53. अ॒स्मै॒ स्वाथ् स्वाद॑स्मा अस्मै॒ स्वाद् योने॒र् योनेः॒ स्वाद॑स्मा अस्मै॒ स्वाद् योनेः᳚ । \newline
54. स्वाद् योने॒र् योनेः॒ स्वाथ् स्वाद् योनेः᳚ प्र॒जाम् प्र॒जां ॅयोनेः॒ स्वाथ् स्वाद् योनेः᳚ प्र॒जाम् । \newline
55. योनेः᳚ प्र॒जाम् प्र॒जां ॅयोने॒र् योनेः᳚ प्र॒जाम् प्र प्र प्र॒जां ॅयोने॒र् योनेः᳚ प्र॒जाम् प्र । \newline
56. प्र॒जाम् प्र प्र प्र॒जाम् प्र॒जाम् प्र ज॑नयन्ति जनयन्ति॒ प्र प्र॒जाम् प्र॒जाम् प्र ज॑नयन्ति । \newline
57. प्र॒जामिति॑ प्र - जाम् । \newline
58. प्र ज॑नयन्ति जनयन्ति॒ प्र प्र ज॑नयन्ति वि॒न्दते॑ वि॒न्दते॑ जनयन्ति॒ प्र प्र ज॑नयन्ति वि॒न्दते᳚ । \newline
59. ज॒न॒य॒न्ति॒ वि॒न्दते॑ वि॒न्दते॑ जनयन्ति जनयन्ति वि॒न्दते᳚ प्र॒जाम् प्र॒जां ॅवि॒न्दते॑ जनयन्ति जनयन्ति वि॒न्दते᳚ प्र॒जाम् । \newline
60. वि॒न्दते᳚ प्र॒जाम् प्र॒जां ॅवि॒न्दते॑ वि॒न्दते᳚ प्र॒जा माप॒ आपः॑ प्र॒जां ॅवि॒न्दते॑ वि॒न्दते᳚ प्र॒जा मापः॑ । \newline
\pagebreak
\markright{ TS 2.1.5.4  \hfill https://www.vedavms.in \hfill}

\section{ TS 2.1.5.4 }

\textbf{TS 2.1.5.4 } \newline
\textbf{Samhita Paata} \newline

प्र॒जामापो॒ वा ओष॑ध॒योऽस॒त् पुरु॑ष॒ आप॑ ए॒वास्मा॒ अस॑तः॒ सद्द॑दति॒ तस्मा॑दाहु॒र्यश्चै॒वं ॅवेद॒ यश्च॒ नाप॒स्त्वावास॑तः॒ सद्द॑द॒ती-त्यै॒न्द्रीꣳ सू॒तव॑शा॒मा ल॑भेत॒ भूति॑का॒मोऽजा॑तो॒ वा ए॒ष योऽलं॒ भूत्यै॒ सन् भूतिं॒ न प्रा॒प्नोतीन्द्रं॒ खलु॒ वा ए॒षा सू॒त्वा व॒शाऽभ॑व॒ - [  ] \newline

\textbf{Pada Paata} \newline

प्र॒जामिति॑ प्र - जाम् । आपः॑ । वै । ओष॑धयः । अस॑त् । पुरु॑षः । आपः॑ । ए॒व । अ॒स्मै॒ । अस॑तः । सत् । द॒द॒ति॒ । तस्मा᳚त् । आ॒हुः॒ । यः । च॒ । ए॒वम् । वेद॑ । यः । च॒ । न । आपः॑ । तु । वाव । अस॑तः । सत् । द॒द॒ति॒ । इति॑ । ऐ॒न्द्रीम् । सू॒तव॑शा॒मिति॑ सू॒त - व॒शा॒म् । एति॑ । ल॒भे॒त॒ । भूति॑काम॒ इति॒ भूति॑ - का॒मः॒ । अजा॑तः । वै । ए॒षः । यः । अल᳚म् । भूत्यै᳚ । सन्न् । भूति᳚म् । न । प्रा॒प्नोतीति॑ प्र - आ॒प्नोति॑ । इन्द्र᳚म् । खलु॑ । वै । ए॒षा । सू॒त्वा । व॒शा । अ॒भ॒व॒त् ।  \newline


\textbf{Krama Paata} \newline

प्र॒जामापः॑ । प्र॒जामिति॑ प्र - जाम् । आपो॒ वै । वा ओष॑धयः । ओष॑ध॒योऽस॑त् । अस॒त् पुरु॑षः । पुरु॑ष॒ आपः॑ । आप॑ ए॒व । ए॒वास्मै᳚ । अ॒स्मा॒ अस॑तः । अस॑तः॒ सत् । सद् द॑दति । द॒द॒ति॒ तस्मा᳚त् । तस्मा॑दाहुः । आ॒हु॒र् यः । यश्च॑ । चै॒वम् । ए॒वं ॅवेद॑ । वेद॒ यः । यश्च॑ । च॒ न । नापः॑ । आप॒स्तु । त्वाव । वावास॑तः । अस॑तः॒ सत् । सद् द॑दति । द॒द॒तीति॑ । इत्यै॒न्द्रीम् । ऐ॒न्द्रीꣳ सू॒तव॑शाम् । सू॒तव॑शा॒मा । सू॒तव॑शा॒मिति॑ सू॒त - व॒शा॒म् । आ ल॑भेत । ल॒भे॒त॒ भूति॑कामः । भूति॑का॒मोऽजा॑तः । भूति॑काम॒ इति॒ भूति॑ - का॒मः॒ । अजा॑तो॒ वै । वा ए॒षः । ए॒ष यः । योऽल᳚म् । अल॒म् भूत्यै᳚ । भूत्यै॒ सन्न् । सन् भूति᳚म् । भूति॒म् न । न प्रा॒प्नोति॑ । प्रा॒प्नोतीन्द्र᳚म् । प्रा॒प्नोतीति॑ प्र - आ॒प्नोति॑ । इन्द्र॒म् खलु॑ । खलु॒ वै । वा ए॒षा । ए॒षा सू॒त्वा । सू॒त्वा व॒शा । व॒शा ऽभ॑वत् । अ॒भ॒व॒दिन्द्र᳚म् \newline

\textbf{Jatai Paata} \newline

1. प्र॒जा माप॒ आपः॑ प्र॒जाम् प्र॒जा मापः॑ । \newline
2. प्र॒जामिति॑ प्र - जाम् । \newline
3. आपो॒ वै वा आप॒ आपो॒ वै । \newline
4. वा ओष॑धय॒ ओष॑धयो॒ वै वा ओष॑धयः । \newline
5. ओष॑ध॒यो ऽस॒ दस॒ दोष॑धय॒ ओष॑ध॒यो ऽस॑त् । \newline
6. अस॒त् पुरु॑षः॒ पुरु॒षो ऽस॒दस॒त् पुरु॑षः । \newline
7. पुरु॑ष॒ आप॒ आपः॒ पुरु॑षः॒ पुरु॑ष॒ आपः॑ । \newline
8. आप॑ ए॒वैवाप॒ आप॑ ए॒व । \newline
9. ए॒वास्मा॑ अस्मा ए॒वैवास्मै᳚ । \newline
10. अ॒स्मा॒ अस॒तो ऽस॑तो ऽस्मा अस्मा॒ अस॑तः । \newline
11. अस॑तः॒ सथ् सदस॒तो ऽस॑तः॒ सत् । \newline
12. सद् द॑दति ददति॒ सथ् सद् द॑दति । \newline
13. द॒द॒ति॒ तस्मा॒त् तस्मा᳚द् ददति ददति॒ तस्मा᳚त् । \newline
14. तस्मा॑ दाहु राहु॒ स्तस्मा॒त् तस्मा॑ दाहुः । \newline
15. आ॒हु॒र् यो य आ॑हु राहु॒र् यः । \newline
16. यश्च॑ च॒ यो यश्च॑ । \newline
17. चै॒व मे॒वम् च॑ चै॒वम् । \newline
18. ए॒वं ॅवेद॒ वेदै॒व मे॒वं ॅवेद॑ । \newline
19. वेद॒ यो यो वेद॒ वेद॒ यः । \newline
20. यश्च॑ च॒ यो यश्च॑ । \newline
21. च॒ न न च॑ च॒ न । \newline
22. नाप॒ आपो॒ न नापः॑ । \newline
23. आप॒ स्तु त्वाप॒ आप॒ स्तु । \newline
24. त्वाव वाव तु त्वाव । \newline
25. वावास॒तो ऽस॑तो॒ वाव वावास॑तः । \newline
26. अस॑तः॒ सथ् सदस॒तो ऽस॑तः॒ सत् । \newline
27. सद् द॑दति ददति॒ सथ् सद् द॑दति । \newline
28. द॒द॒तीतीति॑ ददति दद॒तीति॑ । \newline
29. इत्यै॒न्द्री मै॒न्द्री मिती त्यै॒न्द्रीम् । \newline
30. ऐ॒न्द्रीꣳ सू॒तव॑शाꣳ सू॒तव॑शा मै॒न्द्री मै॒न्द्रीꣳ सू॒तव॑शाम् । \newline
31. सू॒तव॑शा॒ मा सू॒तव॑शाꣳ सू॒तव॑शा॒ मा । \newline
32. सू॒तव॑शा॒मिति॑ सू॒त - व॒शा॒म् । \newline
33. आ ल॑भेत लभे॒ता ल॑भेत । \newline
34. ल॒भे॒त॒ भूति॑कामो॒ भूति॑कामो लभेत लभेत॒ भूति॑कामः । \newline
35. भूति॑का॒मो ऽजा॒तो ऽजा॑तो॒ भूति॑कामो॒ भूति॑का॒मो ऽजा॑तः । \newline
36. भूति॑काम॒ इति॒ भूति॑ - का॒मः॒ । \newline
37. अजा॑तो॒ वै वा अजा॒तो ऽजा॑तो॒ वै । \newline
38. वा ए॒ष ए॒ष वै वा ए॒षः । \newline
39. ए॒ष यो य ए॒ष ए॒ष यः । \newline
40. यो ऽल॒ मलं॒ ॅयो यो ऽल᳚म् । \newline
41. अल॒म् भूत्यै॒ भूत्या॒ अल॒ मल॒म् भूत्यै᳚ । \newline
42. भूत्यै॒ सन् थ्सन् भूत्यै॒ भूत्यै॒ सन्न् । \newline
43. सन् भूति॒म् भूतिꣳ॒॒ सन् थ्सन् भूति᳚म् । \newline
44. भूति॒म् न न भूति॒म् भूति॒म् न । \newline
45. न प्रा॒प्नोति॑ प्रा॒प्नोति॒ न न प्रा॒प्नोति॑ । \newline
46. प्रा॒प्नोतीन्द्र॒ मिन्द्र॑म् प्रा॒प्नोति॑ प्रा॒प्नोतीन्द्र᳚म् । \newline
47. प्रा॒प्नोतीति॑ प्र - आ॒प्नोति॑ । \newline
48. इन्द्र॒म् खलु॒ खल्विन्द्र॒ मिन्द्र॒म् खलु॑ । \newline
49. खलु॒ वै वै खलु॒ खलु॒ वै । \newline
50. वा ए॒षैषा वै वा ए॒षा । \newline
51. ए॒षा सू॒त्वा सू॒त्वैषैषा सू॒त्वा । \newline
52. सू॒त्वा व॒शा व॒शा सू॒त्वा सू॒त्वा व॒शा । \newline
53. व॒शा ऽभ॑व दभवद् व॒शा व॒शा ऽभ॑वत् । \newline
54. अ॒भ॒व॒ दिन्द्र॒ मिन्द्र॑ मभव दभव॒ दिन्द्र᳚म् । \newline

\textbf{Ghana Paata } \newline

1. प्र॒जा माप॒ आपः॑ प्र॒जाम् प्र॒जा मापो॒ वै वा आपः॑ प्र॒जाम् प्र॒जा मापो॒ वै । \newline
2. प्र॒जामिति॑ प्र - जाम् । \newline
3. आपो॒ वै वा आप॒ आपो॒ वा ओष॑धय॒ ओष॑धयो॒ वा आप॒ आपो॒ वा ओष॑धयः । \newline
4. वा ओष॑धय॒ ओष॑धयो॒ वै वा ओष॑ध॒यो ऽस॒ दस॒ दोष॑धयो॒ वै वा ओष॑ध॒यो ऽस॑त् । \newline
5. ओष॑ध॒यो ऽस॒ दस॒ दोष॑धय॒ ओष॑ध॒यो ऽस॒त् पुरु॑षः॒ पुरु॒षो ऽस॒दोष॑धय॒ ओष॑ध॒यो ऽस॒त् पुरु॑षः । \newline
6. अस॒त् पुरु॑षः॒ पुरु॒षो ऽस॒दस॒त् पुरु॑ष॒ आप॒ आपः॒ पुरु॒षो ऽस॒दस॒त् पुरु॑ष॒ आपः॑ । \newline
7. पुरु॑ष॒ आप॒ आपः॒ पुरु॑षः॒ पुरु॑ष॒ आप॑ ए॒वैवापः॒ पुरु॑षः॒ पुरु॑ष॒ आप॑ ए॒व । \newline
8. आप॑ ए॒वैवाप॒ आप॑ ए॒वास्मा॑ अस्मा ए॒वाप॒ आप॑ ए॒वास्मै᳚ । \newline
9. ए॒वास्मा॑ अस्मा ए॒वैवास्मा॒ अस॒तो ऽस॑तो ऽस्मा ए॒वैवास्मा॒ अस॑तः । \newline
10. अ॒स्मा॒ अस॒तो ऽस॑तो ऽस्मा अस्मा॒ अस॑तः॒ सथ् सदस॑तो ऽस्मा अस्मा॒ अस॑तः॒ सत् । \newline
11. अस॑तः॒ सथ् सदस॒तो ऽस॑तः॒ सद् द॑दति ददति॒ सदस॒तो ऽस॑तः॒ सद् द॑दति । \newline
12. सद् द॑दति ददति॒ सथ् सद् द॑दति॒ तस्मा॒त् तस्मा᳚द् ददति॒ सथ् सद् द॑दति॒ तस्मा᳚त् । \newline
13. द॒द॒ति॒ तस्मा॒त् तस्मा᳚द् ददति ददति॒ तस्मा॑ दाहु राहु॒ स्तस्मा᳚द् ददति ददति॒ तस्मा॑ दाहुः । \newline
14. तस्मा॑ दाहु राहु॒ स्तस्मा॒त् तस्मा॑ दाहु॒र् यो य आ॑हु॒ स्तस्मा॒त् तस्मा॑ दाहु॒र् यः । \newline
15. आ॒हु॒र् यो य आ॑हु राहु॒र् यश्च॑ च॒ य आ॑हु राहु॒र् यश्च॑ । \newline
16. यश्च॑ च॒ यो यश्चै॒व मे॒वम् च॒ यो यश्चै॒वम् । \newline
17. चै॒व मे॒वम् च॑ चै॒वं ॅवेद॒ वेदै॒वम् च॑ चै॒वं ॅवेद॑ । \newline
18. ए॒वं ॅवेद॒ वेदै॒व मे॒वं ॅवेद॒ यो यो वेदै॒व मे॒वं ॅवेद॒ यः । \newline
19. वेद॒ यो यो वेद॒ वेद॒ यश्च॑ च॒ यो वेद॒ वेद॒ यश्च॑ । \newline
20. यश्च॑ च॒ यो यश्च॒ न न च॒ यो यश्च॒ न । \newline
21. च॒ न न च॑ च॒ नाप॒ आपो॒ न च॑ च॒ नापः॑ । \newline
22. नाप॒ आपो॒ न नाप॒ स्तु त्वापो॒ न नाप॒ स्तु । \newline
23. आप॒ स्तु त्वाप॒ आप॒ स्त्वाव वाव त्वाप॒ आप॒ स्त्वाव । \newline
24. त्वाव वाव तु त्वावास॒तो ऽस॑तो॒ वाव तु त्वावास॑तः । \newline
25. वावास॒तो ऽस॑तो॒ वाव वावास॑तः॒ सथ् सदस॑तो॒ वाव वावास॑तः॒ सत् । \newline
26. अस॑तः॒ सथ् सदस॒तो ऽस॑तः॒ सद् द॑दति ददति॒ सदस॒तो ऽस॑तः॒ सद् द॑दति । \newline
27. सद् द॑दति ददति॒ सथ् सद् द॑द॒तीतीति॑ ददति॒ सथ् सद् द॑द॒तीति॑ । \newline
28. द॒द॒तीतीति॑ ददति दद॒ती त्यै॒न्द्री मै॒न्द्री मिति॑ ददति दद॒ती त्यै॒न्द्रीम् । \newline
29. इत्यै॒न्द्री मै॒न्द्री मितीत्यै॒न्द्रीꣳ सू॒तव॑शाꣳ सू॒तव॑शा मै॒न्द्री मिती त्यै॒न्द्रीꣳ सू॒तव॑शाम् । \newline
30. ऐ॒न्द्रीꣳ सू॒तव॑शाꣳ सू॒तव॑शा मै॒न्द्री मै॒न्द्रीꣳ सू॒तव॑शा॒ मा सू॒तव॑शा मै॒न्द्री मै॒न्द्रीꣳ सू॒तव॑शा॒ मा । \newline
31. सू॒तव॑शा॒ मा सू॒तव॑शाꣳ सू॒तव॑शा॒ मा ल॑भेत लभे॒ता सू॒तव॑शाꣳ सू॒तव॑शा॒ मा ल॑भेत । \newline
32. सू॒तव॑शा॒मिति॑ सू॒त - व॒शा॒म् । \newline
33. आ ल॑भेत लभे॒ता ल॑भेत॒ भूति॑कामो॒ भूति॑कामो लभे॒ता ल॑भेत॒ भूति॑कामः । \newline
34. ल॒भे॒त॒ भूति॑कामो॒ भूति॑कामो लभेत लभेत॒ भूति॑का॒मो ऽजा॒तो ऽजा॑तो॒ भूति॑कामो लभेत लभेत॒ भूति॑का॒मो ऽजा॑तः । \newline
35. भूति॑का॒मो ऽजा॒तो ऽजा॑तो॒ भूति॑कामो॒ भूति॑का॒मो ऽजा॑तो॒ वै वा अजा॑तो॒ भूति॑कामो॒ भूति॑का॒मो ऽजा॑तो॒ वै । \newline
36. भूति॑काम॒ इति॒ भूति॑ - का॒मः॒ । \newline
37. अजा॑तो॒ वै वा अजा॒तो ऽजा॑तो॒ वा ए॒ष ए॒ष वा अजा॒तो ऽजा॑तो॒ वा ए॒षः । \newline
38. वा ए॒ष ए॒ष वै वा ए॒ष यो य ए॒ष वै वा ए॒ष यः । \newline
39. ए॒ष यो य ए॒ष ए॒ष यो ऽल॒ मलं॒ ॅय ए॒ष ए॒ष यो ऽल᳚म् । \newline
40. यो ऽल॒ मलं॒ ॅयो यो ऽल॒म् भूत्यै॒ भूत्या॒ अलं॒ ॅयो यो ऽल॒म् भूत्यै᳚ । \newline
41. अल॒म् भूत्यै॒ भूत्या॒ अल॒ मल॒म् भूत्यै॒ सन् थ्सन् भूत्या॒ अल॒ मल॒म् भूत्यै॒ सन्न् । \newline
42. भूत्यै॒ सन् थ्सन् भूत्यै॒ भूत्यै॒ सन् भूति॒म् भूतिꣳ॒॒ सन् भूत्यै॒ भूत्यै॒ सन् भूति᳚म् । \newline
43. सन् भूति॒म् भूतिꣳ॒॒ सन् थ्सन् भूति॒न्न न भूतिꣳ॒॒ सन् थ्सन् भूति॒न्न । \newline
44. भूति॒न्न न भूति॒म् भूति॒न्न प्रा॒प्नोति॑ प्रा॒प्नोति॒ न भूति॒म् भूति॒न्न प्रा॒प्नोति॑ । \newline
45. न प्रा॒प्नोति॑ प्रा॒प्नोति॒ न न प्रा॒प्नोतीन्द्र॒ मिन्द्र॑म् प्रा॒प्नोति॒ न न प्रा॒प्नोतीन्द्र᳚म् । \newline
46. प्रा॒प्नोतीन्द्र॒ मिन्द्र॑म् प्रा॒प्नोति॑ प्रा॒प्नोतीन्द्र॒म् खलु॒ खल्विन्द्र॑म् प्रा॒प्नोति॑ प्रा॒प्नोतीन्द्र॒म् खलु॑ । \newline
47. प्रा॒प्नोतीति॑ प्र - आ॒प्नोति॑ । \newline
48. इन्द्र॒म् खलु॒ खल्विन्द्र॒ मिन्द्र॒म् खलु॒ वै वै खल्विन्द्र॒ मिन्द्र॒म् खलु॒ वै । \newline
49. खलु॒ वै वै खलु॒ खलु॒ वा ए॒षैषा वै खलु॒ खलु॒ वा ए॒षा । \newline
50. वा ए॒षैषा वै वा ए॒षा सू॒त्वा सू॒त्वैषा वै वा ए॒षा सू॒त्वा । \newline
51. ए॒षा सू॒त्वा सू॒त्वैषैषा सू॒त्वा व॒शा व॒शा सू॒त्वैषैषा सू॒त्वा व॒शा । \newline
52. सू॒त्वा व॒शा व॒शा सू॒त्वा सू॒त्वा व॒शा ऽभ॑व दभवद् व॒शा सू॒त्वा सू॒त्वा व॒शा ऽभ॑वत् । \newline
53. व॒शा ऽभ॑व दभवद् व॒शा व॒शा ऽभ॑व॒ दिन्द्र॒ मिन्द्र॑ मभवद् व॒शा व॒शा ऽभ॑व॒ दिन्द्र᳚म् । \newline
54. अ॒भ॒व॒ दिन्द्र॒ मिन्द्र॑ मभव दभव॒ दिन्द्र॑ मे॒वैवे न्द्र॑ मभव दभव॒ दिन्द्र॑ मे॒व । \newline
\pagebreak
\markright{ TS 2.1.5.5  \hfill https://www.vedavms.in \hfill}

\section{ TS 2.1.5.5 }

\textbf{TS 2.1.5.5 } \newline
\textbf{Samhita Paata} \newline

-दिन्द्र॑मे॒व स्वेन॑ भाग॒धेये॒नोप॑ धावति॒ स ए॒वैनं॒ भूतिं॑ गमयति॒ भव॑त्ये॒व यꣳ सू॒त्वा व॒शा स्यात् तमै॒न्द्रमे॒वाऽऽ* ल॑भेतै॒तद्वाव तदि॑न्द्रि॒यꣳ सा॒क्षादे॒वेन्द्रि॒यमव॑ रुन्ध ऐन्द्रा॒ग्नं पु॑नरुथ् सृ॒ष्टमा ल॑भेत॒ य आ तृ॒तीया॒त् पुरु॑षा॒थ् सोमं॒ न पिबे॒द्-विच्छि॑न्नो॒ वा ए॒तस्य॑ सोमपी॒थो यो ब्रा᳚ह्म॒णः सन्ना - [  ] \newline

\textbf{Pada Paata} \newline

इन्द्र᳚म् । ए॒व । स्वेन॑ । भा॒ग॒धेये॒नेति॑ भाग - धेये॑न । उपेति॑ । धा॒व॒ति॒ । सः । ए॒व । ए॒न॒म् । भूति᳚म् । ग॒म॒य॒ति॒ । भव॑ति । ए॒व । यम् । सू॒त्वा । व॒शा । स्यात् । तम् । ऐ॒न्द्रम् । ए॒व । एति॑ । ल॒भे॒त॒ । ए॒तत् । वाव । तत् । इ॒न्द्रि॒यम् । सा॒क्षादिति॑ स - अ॒क्षात् । ए॒व । इ॒न्द्रि॒यम् । अवेति॑ । रु॒न्धे॒ । ऐ॒न्द्रा॒ग्नमित्यै᳚न्द्र - अ॒ग्नम् । पु॒न॒रु॒थ्सृ॒ष्टमिति॑ पुनः - उ॒थ्सृ॒ष्टम् । एति॑ । ल॒भे॒त॒ । यः । एति॑ । तृ॒तीया᳚त् । पुरु॑षात् । सोम᳚म् । न । पिबे᳚त् । विच्छि॑न्न॒ इति॒ वि - छि॒न्नः॒ । वै । ए॒तस्य॑ । सो॒म॒पी॒थ इति॑ सोम - पी॒थः । यः । ब्रा॒ह्म॒णः । सन्न् । एति॑ ।  \newline


\textbf{Krama Paata} \newline

इन्द्र॑मे॒व । ए॒व स्वेन॑ । स्वेन॑ भाग॒धेये॑न । भा॒ग॒धेये॒नोप॑ । भा॒ग॒धेये॒नेति॑ भाग - धेये॑न । उप॑ धावति । धा॒व॒ति॒ सः । स ए॒व । ए॒वैन᳚म् । ए॒न॒म् भूति᳚म् । भूति॑म् गमयति । ग॒म॒य॒ति॒ भव॑ति । भव॑त्ये॒व । ए॒व यम् । यꣳ सू॒त्वा । सू॒त्वा व॒शा । व॒शा स्यात् । स्यात् तम् । तमै॒न्द्रम् । ऐ॒न्द्रमे॒व । ए॒वा । आ ल॑भेत । ल॒भे॒तै॒तत् । ए॒तद् वाव । वाव तत् । तदि॑न्द्रि॒यम् । इ॒न्द्रि॒यꣳ सा॒क्षात् । सा॒क्षादे॒व । सा॒क्षादिति॑ स - अ॒क्षात् । ए॒वेन्द्रि॒यम् । इ॒न्द्रि॒यमव॑ । अव॑ रुन्धे । रु॒न्ध॒ ऐ॒न्द्रा॒ग्नम् । ऐ॒न्द्रा॒ग्नम् पु॑नरुथ्सृ॒ष्टम् । ऐ॒न्द्रा॒ग्नमित्यै᳚न्द्र - अ॒ग्नम् । पु॒न॒रु॒थ्सृ॒ष्टमा । पु॒न॒रु॒थ्सृ॒ष्टमिति॑ पुनः - उ॒थ्सृ॒ष्टम् । आ ल॑भेत । ल॒भे॒त॒ यः । य आ । आ तृ॒तीया᳚त् । तृ॒तीया॒त् पुरु॑षात् । पुरु॑षा॒थ् सोम᳚म् । सोम॒म् न । न पिबे᳚त् । पिबे॒द् विच्छि॑न्नः । विच्छि॑न्नो॒ वै । विच्छि॑न्न॒ इति॒ वि - छि॒न्नः॒ । वा ए॒तस्य॑ । ए॒तस्य॑ सोमपी॒थः । सो॒म॒पी॒थो यः । सो॒म॒पी॒थ इति॑ सोम - पी॒थः । यो ब्रा᳚ह्म॒णः । ब्रा॒ह्म॒णः सन्न् । सन्ना । आ तृ॒तीया᳚त् \newline

\textbf{Jatai Paata} \newline

1. इन्द्र॑ मे॒वैवे न्द्र॒ मिन्द्र॑ मे॒व । \newline
2. ए॒व स्वेन॒ स्वेनै॒वैव स्वेन॑ । \newline
3. स्वेन॑ भाग॒धेये॑न भाग॒धेये॑न॒ स्वेन॒ स्वेन॑ भाग॒धेये॑न । \newline
4. भा॒ग॒धेये॒नोपोप॑ भाग॒धेये॑न भाग॒धेये॒नोप॑ । \newline
5. भा॒ग॒धेये॒नेति॑ भाग - धेये॑न । \newline
6. उप॑ धावति धाव॒ त्युपोप॑ धावति । \newline
7. धा॒व॒ति॒ स स धा॑वति धावति॒ सः । \newline
8. स ए॒वैव स स ए॒व । \newline
9. ए॒वैन॑ मेन मे॒वैवैन᳚म् । \newline
10. ए॒न॒म् भूति॒म् भूति॑ मेन मेन॒म् भूति᳚म् । \newline
11. भूति॑म् गमयति गमयति॒ भूति॒म् भूति॑म् गमयति । \newline
12. ग॒म॒य॒ति॒ भव॑ति॒ भव॑ति गमयति गमयति॒ भव॑ति । \newline
13. भव॑त्ये॒वैव भव॑ति॒ भव॑त्ये॒व । \newline
14. ए॒व यं ॅय मे॒वैव यम् । \newline
15. यꣳ सू॒त्वा सू॒त्वा यं ॅयꣳ सू॒त्वा । \newline
16. सू॒त्वा व॒शा व॒शा सू॒त्वा सू॒त्वा व॒शा । \newline
17. व॒शा स्याथ् स्याद् व॒शा व॒शा स्यात् । \newline
18. स्यात् तम् तꣳ स्याथ् स्यात् तम् । \newline
19. त मै॒न्द्र मै॒न्द्रम् तम् त मै॒न्द्रम् । \newline
20. ऐ॒न्द्र मे॒वैवैन्द्र मै॒न्द्र मे॒व । \newline
21. ए॒वैवैवा । \newline
22. आ ल॑भेत लभे॒ता ल॑भेत । \newline
23. ल॒भे॒ तै॒त दे॒त ल्ल॑भेत लभे तै॒तत् । \newline
24. ए॒तद् वाव वावैत दे॒तद् वाव । \newline
25. वाव तत् तद् वाव वाव तत् । \newline
26. तदि॑न्द्रि॒य मि॑न्द्रि॒यम् तत् तदि॑न्द्रि॒यम् । \newline
27. इ॒न्द्रि॒यꣳ सा॒क्षाथ् सा॒क्षादि॑न्द्रि॒य मि॑न्द्रि॒यꣳ सा॒क्षात् । \newline
28. सा॒क्षा दे॒वैव सा॒क्षाथ् सा॒क्षा दे॒व । \newline
29. सा॒क्षादिति॑ स - अ॒क्षात् । \newline
30. ए॒वे न्द्रि॒य मि॑न्द्रि॒य मे॒वैवे न्द्रि॒यम् । \newline
31. इ॒न्द्रि॒य मवावे᳚ न्द्रि॒य मि॑न्द्रि॒य मव॑ । \newline
32. अव॑ रुन्धे रु॒न्धे ऽवाव॑ रुन्धे । \newline
33. रु॒न्ध॒ ऐ॒न्द्रा॒ग्न मै᳚न्द्रा॒ग्नꣳ रु॑न्धे रुन्ध ऐन्द्रा॒ग्नम् । \newline
34. ऐ॒न्द्रा॒ग्नम् पु॑नरुथ्सृ॒ष्टम् पु॑नरुथ्सृ॒ष्ट मै᳚न्द्रा॒ग्न मै᳚न्द्रा॒ग्नम् पु॑नरुथ्सृ॒ष्टम् । \newline
35. ऐ॒न्द्रा॒ग्नमित्यै᳚न्द्र - अ॒ग्नम् । \newline
36. पु॒न॒रु॒थ्सृ॒ष्ट मा पु॑नरुथ्सृ॒ष्टम् पु॑नरुथ्सृ॒ष्ट मा । \newline
37. पु॒न॒रु॒थ्सृ॒ष्टमिति॑ पुनः - उ॒थ्सृ॒ष्टम् । \newline
38. आ ल॑भेत लभे॒ता ल॑भेत । \newline
39. ल॒भे॒त॒ यो यो ल॑भेत लभेत॒ यः । \newline
40. य आ यो य आ । \newline
41. आ तृ॒तीया᳚त् तृ॒तीया॒दा तृ॒तीया᳚त् । \newline
42. तृ॒तीया॒त् पुरु॑षा॒त् पुरु॑षात् तृ॒तीया᳚त् तृ॒तीया॒त् पुरु॑षात् । \newline
43. पुरु॑षा॒थ् सोमꣳ॒॒ सोम॒म् पुरु॑षा॒त् पुरु॑षा॒थ् सोम᳚म् । \newline
44. सोम॒म् न न सोमꣳ॒॒ सोम॒म् न । \newline
45. न पिबे॒त् पिबे॒न् न न पिबे᳚त् । \newline
46. पिबे॒द् विच्छि॑न्नो॒ विच्छि॑न्नः॒ पिबे॒त् पिबे॒द् विच्छि॑न्नः । \newline
47. विच्छि॑न्नो॒ वै वै विच्छि॑न्नो॒ विच्छि॑न्नो॒ वै । \newline
48. विच्छि॑न्न॒ इति॒ वि - छि॒न्नः॒ । \newline
49. वा ए॒त स्यै॒तस्य॒ वै वा ए॒तस्य॑ । \newline
50. ए॒तस्य॑ सोमपी॒थः सो॑मपी॒थ ए॒त स्यै॒तस्य॑ सोमपी॒थः । \newline
51. सो॒म॒पी॒थो यो यः सो॑मपी॒थः सो॑मपी॒थो यः । \newline
52. सो॒म॒पी॒थ इति॑ सोम - पी॒थः । \newline
53. यो ब्रा᳚ह्म॒णो ब्रा᳚ह्म॒णो यो यो ब्रा᳚ह्म॒णः । \newline
54. ब्रा॒ह्म॒णः सन् थ्सन् ब्रा᳚ह्म॒णो ब्रा᳚ह्म॒णः सन्न् । \newline
55. सन् ना सन् थ्सन् ना । \newline
56. आ तृ॒तीया᳚त् तृ॒तीया॒दा तृ॒तीया᳚त् । \newline

\textbf{Ghana Paata } \newline

1. इन्द्र॑ मे॒वैवे न्द्र॒ मिन्द्र॑ मे॒व स्वेन॒ स्वेनै॒वे न्द्र॒ मिन्द्र॑ मे॒व स्वेन॑ । \newline
2. ए॒व स्वेन॒ स्वेनै॒वैव स्वेन॑ भाग॒धेये॑न भाग॒धेये॑न॒ स्वेनै॒वैव स्वेन॑ भाग॒धेये॑न । \newline
3. स्वेन॑ भाग॒धेये॑न भाग॒धेये॑न॒ स्वेन॒ स्वेन॑ भाग॒धेये॒नो पोप॑ भाग॒धेये॑न॒ स्वेन॒ स्वेन॑ भाग॒धेये॒नोप॑ । \newline
4. भा॒ग॒धेये॒नो पोप॑ भाग॒धेये॑न भाग॒धेये॒नोप॑ धावति धाव॒त्युप॑ भाग॒धेये॑न भाग॒धेये॒नोप॑ धावति । \newline
5. भा॒ग॒धेये॒नेति॑ भाग - धेये॑न । \newline
6. उप॑ धावति धाव॒त्यु पोप॑ धावति॒ स स धा॑व॒ त्युपोप॑ धावति॒ सः । \newline
7. धा॒व॒ति॒ स स धा॑वति धावति॒ स ए॒वैव स धा॑वति धावति॒ स ए॒व । \newline
8. स ए॒वैव स स ए॒वैन॑ मेन मे॒व स स ए॒वैन᳚म् । \newline
9. ए॒वैन॑ मेन मे॒वैवैन॒म् भूति॒म् भूति॑ मेन मे॒वैवैन॒म् भूति᳚म् । \newline
10. ए॒न॒म् भूति॒म् भूति॑ मेन मेन॒म् भूति॑म् गमयति गमयति॒ भूति॑ मेन मेन॒म् भूति॑म् गमयति । \newline
11. भूति॑म् गमयति गमयति॒ भूति॒म् भूति॑म् गमयति॒ भव॑ति॒ भव॑ति गमयति॒ भूति॒म् भूति॑म् गमयति॒ भव॑ति । \newline
12. ग॒म॒य॒ति॒ भव॑ति॒ भव॑ति गमयति गमयति॒ भव॑त्ये॒वैव भव॑ति गमयति गमयति॒ भव॑त्ये॒व । \newline
13. भव॑त्ये॒वैव भव॑ति॒ भव॑त्ये॒व यं ॅय मे॒व भव॑ति॒ भव॑त्ये॒व यम् । \newline
14. ए॒व यं ॅय मे॒वैव यꣳ सू॒त्वा सू॒त्वा य मे॒वैव यꣳ सू॒त्वा । \newline
15. यꣳ सू॒त्वा सू॒त्वा यं ॅयꣳ सू॒त्वा व॒शा व॒शा सू॒त्वा यं ॅयꣳ सू॒त्वा व॒शा । \newline
16. सू॒त्वा व॒शा व॒शा सू॒त्वा सू॒त्वा व॒शा स्याथ् स्याद् व॒शा सू॒त्वा सू॒त्वा व॒शा स्यात् । \newline
17. व॒शा स्याथ् स्याद् व॒शा व॒शा स्यात् तम् तꣳ स्याद् व॒शा व॒शा स्यात् तम् । \newline
18. स्यात् तम् तꣳ स्याथ् स्यात् त मै॒न्द्र मै॒न्द्रम् तꣳ स्याथ् स्यात् त मै॒न्द्रम् । \newline
19. त मै॒न्द्र मै॒न्द्रम् तम् त मै॒न्द्र मे॒वैवैन्द्रम् तम् त मै॒न्द्र मे॒व । \newline
20. ऐ॒न्द्र मे॒वैवैन्द्र मै॒न्द्र मे॒वैवैन्द्र मै॒न्द्र मे॒वा । \newline
21. ए॒वैवैवा ल॑भेत लभे॒तैवैवा ल॑भेत । \newline
22. आ ल॑भेत लभे॒ता ल॑भे तै॒त दे॒त ल्ल॑भे॒ता ल॑भेतै॒तत् । \newline
23. ल॒भे॒ तै॒त दे॒त ल्ल॑भेत लभेतै॒तद् वाव वावैत ल्ल॑भेत लभेतै॒तद् वाव । \newline
24. ए॒तद् वाव वावै तदे॒तद् वाव तत् तद् वावै तदे॒तद् वाव तत् । \newline
25. वाव तत् तद् वाव वाव तदि॑न्द्रि॒य मि॑न्द्रि॒यम् तद् वाव वाव तदि॑न्द्रि॒यम् । \newline
26. तदि॑न्द्रि॒य मि॑न्द्रि॒यम् तत् तदि॑न्द्रि॒यꣳ सा॒क्षाथ् सा॒क्षा दि॑न्द्रि॒यम् तत् तदि॑न्द्रि॒यꣳ सा॒क्षात् । \newline
27. इ॒न्द्रि॒यꣳ सा॒क्षाथ् सा॒क्षा दि॑न्द्रि॒य मि॑न्द्रि॒यꣳ सा॒क्षादे॒वैव सा॒क्षा दि॑न्द्रि॒य मि॑न्द्रि॒यꣳ सा॒क्षादे॒व । \newline
28. सा॒क्षादे॒वैव सा॒क्षाथ् सा॒क्षादे॒वे न्द्रि॒य मि॑न्द्रि॒य मे॒व सा॒क्षाथ् सा॒क्षादे॒वे न्द्रि॒यम् । \newline
29. सा॒क्षादिति॑ स - अ॒क्षात् । \newline
30. ए॒वे न्द्रि॒य मि॑न्द्रि॒य मे॒वैवे न्द्रि॒य मवावे᳚ न्द्रि॒य मे॒वैवे न्द्रि॒य मव॑ । \newline
31. इ॒न्द्रि॒य मवावे᳚ न्द्रि॒य मि॑न्द्रि॒य मव॑ रुन्धे रु॒न्धे ऽवे᳚ न्द्रि॒य मि॑न्द्रि॒य मव॑ रुन्धे । \newline
32. अव॑ रुन्धे रु॒न्धे ऽवाव॑ रुन्ध ऐन्द्रा॒ग्न मै᳚न्द्रा॒ग्नꣳ रु॒न्धे ऽवाव॑ रुन्ध ऐन्द्रा॒ग्नम् । \newline
33. रु॒न्ध॒ ऐ॒न्द्रा॒ग्न मै᳚न्द्रा॒ग्नꣳ रु॑न्धे रुन्ध ऐन्द्रा॒ग्नम् पु॑नरुथ्सृ॒ष्टम् पु॑नरुथ्सृ॒ष्ट मै᳚न्द्रा॒ग्नꣳ रु॑न्धे रुन्ध ऐन्द्रा॒ग्नम् पु॑नरुथ्सृ॒ष्टम् । \newline
34. ऐ॒न्द्रा॒ग्नम् पु॑नरुथ्सृ॒ष्टम् पु॑नरुथ्सृ॒ष्ट मै᳚न्द्रा॒ग्न मै᳚न्द्रा॒ग्नम् पु॑नरुथ्सृ॒ष्ट मा पु॑नरुथ्सृ॒ष्ट मै᳚न्द्रा॒ग्न मै᳚न्द्रा॒ग्नम् पु॑नरुथ्सृ॒ष्ट मा । \newline
35. ऐ॒न्द्रा॒ग्नमित्यै᳚न्द्र - अ॒ग्नम् । \newline
36. पु॒न॒रु॒थ्सृ॒ष्ट मा पु॑नरुथ्सृ॒ष्टम् पु॑नरुथ्सृ॒ष्ट मा ल॑भेत लभे॒ता पु॑नरुथ्सृ॒ष्टम् पु॑नरुथ्सृ॒ष्ट मा ल॑भेत । \newline
37. पु॒न॒रु॒थ्सृ॒ष्टमिति॑ पुनः - उ॒थ्सृ॒ष्टम् । \newline
38. आ ल॑भेत लभे॒ता ल॑भेत॒ यो यो ल॑भे॒ता ल॑भेत॒ यः । \newline
39. ल॒भे॒त॒ यो यो ल॑भेत लभेत॒ य आ यो ल॑भेत लभेत॒ य आ । \newline
40. य आ यो य आ तृ॒तीया᳚त् तृ॒तीया॒दा यो य आ तृ॒तीया᳚त् । \newline
41. आ तृ॒तीया᳚त् तृ॒तीया॒दा तृ॒तीया॒त् पुरु॑षा॒त् पुरु॑षात् तृ॒तीया॒दा तृ॒तीया॒त् पुरु॑षात् । \newline
42. तृ॒तीया॒त् पुरु॑षा॒त् पुरु॑षात् तृ॒तीया᳚त् तृ॒तीया॒त् पुरु॑षा॒थ् सोमꣳ॒॒ सोम॒म् पुरु॑षात् तृ॒तीया᳚त् तृ॒तीया॒त् पुरु॑षा॒थ् सोम᳚म् । \newline
43. पुरु॑षा॒थ् सोमꣳ॒॒ सोम॒म् पुरु॑षा॒त् पुरु॑षा॒थ् सोम॒न्न न सोम॒म् पुरु॑षा॒त् पुरु॑षा॒थ् सोम॒न्न । \newline
44. सोम॒न्न न सोमꣳ॒॒ सोम॒न्न पिबे॒त् पिबे॒न् न सोमꣳ॒॒ सोम॒न्न पिबे᳚त् । \newline
45. न पिबे॒त् पिबे॒न् न न पिबे॒द् विच्छि॑न्नो॒ विच्छि॑न्नः॒ पिबे॒न् न न पिबे॒द् विच्छि॑न्नः । \newline
46. पिबे॒द् विच्छि॑न्नो॒ विच्छि॑न्नः॒ पिबे॒त् पिबे॒द् विच्छि॑न्नो॒ वै वै विच्छि॑न्नः॒ पिबे॒त् पिबे॒द् विच्छि॑न्नो॒ वै । \newline
47. विच्छि॑न्नो॒ वै वै विच्छि॑न्नो॒ विच्छि॑न्नो॒ वा ए॒तस्यै॒तस्य॒ वै विच्छि॑न्नो॒ विच्छि॑न्नो॒ वा ए॒तस्य॑ । \newline
48. विच्छि॑न्न॒ इति॒ वि - छि॒न्नः॒ । \newline
49. वा ए॒तस्यै॒तस्य॒ वै वा ए॒तस्य॑ सोमपी॒थः सो॑मपी॒थ ए॒तस्य॒ वै वा ए॒तस्य॑ सोमपी॒थः । \newline
50. ए॒तस्य॑ सोमपी॒थः सो॑मपी॒थ ए॒तस्यै॒तस्य॑ सोमपी॒थो यो यः सो॑मपी॒थ ए॒तस्यै॒तस्य॑ सोमपी॒थो यः । \newline
51. सो॒म॒पी॒थो यो यः सो॑मपी॒थः सो॑मपी॒थो यो ब्रा᳚ह्म॒णो ब्रा᳚ह्म॒णो यः सो॑मपी॒थः सो॑मपी॒थो यो ब्रा᳚ह्म॒णः । \newline
52. सो॒म॒पी॒थ इति॑ सोम - पी॒थः । \newline
53. यो ब्रा᳚ह्म॒णो ब्रा᳚ह्म॒णो यो यो ब्रा᳚ह्म॒णः सन् थ्सन् ब्रा᳚ह्म॒णो यो यो ब्रा᳚ह्म॒णः सन्न् । \newline
54. ब्रा॒ह्म॒णः सन् थ्सन् ब्रा᳚ह्म॒णो ब्रा᳚ह्म॒णः सन् ना सन् ब्रा᳚ह्म॒णो ब्रा᳚ह्म॒णः सन् ना । \newline
55. सन् ना सन् थ्सन् ना तृ॒तीया᳚त् तृ॒तीया॒दा सन् थ्सन् ना तृ॒तीया᳚त् । \newline
56. आ तृ॒तीया᳚त् तृ॒तीया॒दा तृ॒तीया॒त् पुरु॑षा॒त् पुरु॑षात् तृ॒तीया॒दा तृ॒तीया॒त् पुरु॑षात् । \newline
\pagebreak
\markright{ TS 2.1.5.6  \hfill https://www.vedavms.in \hfill}

\section{ TS 2.1.5.6 }

\textbf{TS 2.1.5.6 } \newline
\textbf{Samhita Paata} \newline

तृ॒तीया॒त् पुरु॑षा॒थ् सोमं॒ न पिब॑तीन्द्रा॒ग्नी ए॒व स्वेन॑ भाग॒धेये॒नोप॑ धावति॒ तावे॒वास्मै॑ सोमपी॒थं प्रय॑च्छत॒ उपै॑नꣳ सोमपी॒थो न॑मति॒ यदै॒न्द्रो भव॑तीन्द्रि॒यं ॅवै सो॑मपी॒थ इ॑न्द्रि॒यमे॒व सो॑मपी॒थमव॑ रुन्धे॒ यदा᳚ग्ने॒यो भव॑त्याग्ने॒यो वै ब्रा᳚ह्म॒णः स्वामे॒व दे॒वता॒मनु॒ संत॑नोति पुनरुथ्‌सृ॒ष्टो भ॑वति पुनरुथ्‌सृ॒ष्ट इ॑व॒ ह्ये॑तस्य॑ - [  ] \newline

\textbf{Pada Paata} \newline

तृ॒तीया᳚त् । पुरु॑षात् । सोम᳚म् । न । पिब॑ति । इ॒न्द्रा॒ग्नी इती᳚न्द्र - अ॒ग्नी । ए॒व । स्वेन॑ । भा॒ग॒धेये॒नेति॑ भाग-धेये॑न । उपेति॑ । धा॒व॒ति॒ । तौ । ए॒व । अ॒स्मै॒ । सो॒म॒पी॒थमिति॑ सोम - पी॒थम् । प्रेति॑ । य॒च्छ॒तः॒ । उपेति॑ । ए॒न॒म् । सो॒म॒पी॒थ इति॑ सोम - पी॒थः । न॒म॒ति॒ । यत् । ऐ॒न्द्रः । भव॑ति । इ॒न्द्रि॒यम् । वै । सो॒म॒पी॒थ इति॑ सोम-पी॒थः । इ॒न्द्रि॒यम् । ए॒व । सो॒म॒पी॒थमिति॑ सोम - पी॒थम् । अवेति॑ । रु॒न्धे॒ । यत् । आ॒ग्ने॒यः । भव॑ति । आ॒ग्ने॒यः । वै । ब्रा॒ह्म॒णः । स्वाम् । ए॒व । दे॒वता᳚म् । अनु॑ । समिति॑ । त॒नो॒ति॒ । पु॒न॒रु॒थ्सृ॒ष्ट इति॑ पुनः-उ॒थ्सृ॒ष्टः । भ॒व॒ति॒ । पु॒न॒रु॒थ्सृ॒ष्ट इति॑ पुनः - उ॒थ्सृ॒ष्टः । इ॒व॒ । हि । ए॒तस्य॑ ।  \newline


\textbf{Krama Paata} \newline

तृ॒तीया॒त् पुरु॑षात् । पुरु॑षा॒थ् सोम᳚म् । सोम॒म् न । न पिब॑ति । पिब॑तीन्द्रा॒ग्नी । इ॒न्द्रा॒ग्नी ए॒व । इ॒न्द्रा॒ग्नी इती᳚न्द्र - अ॒ग्नी । ए॒व स्वेन॑ । स्वेन॑ भाग॒धेये॑न । भा॒ग॒धेये॒नोप॑ । भा॒ग॒धेये॒नेति॑ भाग - धेये॑न । उप॑ धावति । धा॒व॒ति॒ तौ । तावे॒व । ए॒वास्मै᳚ । अ॒स्मै॒ सो॒म॒पी॒थम् । सो॒म॒पी॒थम् प्र । सो॒म॒पी॒थमिति॑ सोम - पी॒थम् । प्र य॑च्छतः । य॒च्छ॒त॒ उप॑ । उपै॑नम् । ए॒नꣳ॒॒ सो॒म॒पी॒थः । सो॒म॒पी॒थो न॑मति । सो॒म॒पी॒थ इति॑ सोम - पी॒थः । न॒म॒ति॒ यत् । यदै॒न्द्रः । ऐ॒न्द्रो भव॑ति । भव॑तीन्द्रि॒यम् । इ॒न्द्रि॒यं ॅवै । वै सो॑मपी॒थः । सो॒म॒पी॒थ इ॑न्द्रि॒यम् । सो॒म॒पी॒थ इति॑ सोम - पी॒थः । इ॒न्द्रि॒यमे॒व । ए॒व सो॑मपी॒थम् । सो॒म॒पी॒थमव॑ । सो॒म॒पी॒थमिति॑ सोम - पी॒थम् । अव॑ रुन्धे । रु॒न्धे॒ यत् । यदा᳚ग्ने॒यः । आ॒ग्ने॒यो भव॑ति । भव॑त्याग्ने॒यः । आ॒ग्ने॒यो वै । वै ब्रा᳚ह्म॒णः । ब्रा॒ह्म॒णः स्वाम् । स्वामे॒व । ए॒व दे॒वता᳚म् । दे॒वता॒मनु॑ । अनु॒ सम् । सं त॑नोति । त॒नो॒ति॒ पु॒न॒रु॒थ्सृ॒ष्टः । पु॒न॒रु॒थ्सृ॒ष्टो भ॑वति । पु॒न॒रु॒थ्सृ॒ष्ट इति॑ पुनः - उ॒थ्सृ॒ष्टः । भ॒व॒ति॒ पु॒न॒रु॒थ्सृ॒ष्टः । पु॒न॒रु॒थ्सृ॒ष्ट इ॑व । पु॒न॒रु॒थ्सृ॒ष्ट इति॑ पुनः - उ॒थ्सृ॒ष्टः । इ॒व॒ हि । ह्ये॑तस्य॑ । ए॒तस्य॑ सोमपी॒थः \newline

\textbf{Jatai Paata} \newline

1. तृ॒तीया॒त् पुरु॑षा॒त् पुरु॑षात् तृ॒तीया᳚त् तृ॒तीया॒त् पुरु॑षात् । \newline
2. पुरु॑षा॒थ् सोमꣳ॒॒ सोम॒म् पुरु॑षा॒त् पुरु॑षा॒थ् सोम᳚म् । \newline
3. सोम॒म् न न सोमꣳ॒॒ सोम॒म् न । \newline
4. न पिब॑ति॒ पिब॑ति॒ न न पिब॑ति । \newline
5. पिब॑तीन्द्रा॒ग्नी इ॑न्द्रा॒ग्नी पिब॑ति॒ पिब॑तीन्द्रा॒ग्नी । \newline
6. इ॒न्द्रा॒ग्नी ए॒वैवे न्द्रा॒ग्नी इ॑न्द्रा॒ग्नी ए॒व । \newline
7. इ॒न्द्रा॒ग्नी इती᳚न्द्र - अ॒ग्नी । \newline
8. ए॒व स्वेन॒ स्वेनै॒वैव स्वेन॑ । \newline
9. स्वेन॑ भाग॒धेये॑न भाग॒धेये॑न॒ स्वेन॒ स्वेन॑ भाग॒धेये॑न । \newline
10. भा॒ग॒धेये॒नोपोप॑ भाग॒धेये॑न भाग॒धेये॒नोप॑ । \newline
11. भा॒ग॒धेये॒नेति॑ भाग - धेये॑न । \newline
12. उप॑ धावति धाव॒ त्युपोप॑ धावति । \newline
13. धा॒व॒ति॒ तौ तौ धा॑वति धावति॒ तौ । \newline
14. ता वे॒वैव तौ ता वे॒व । \newline
15. ए॒वास्मा॑ अस्मा ए॒वैवास्मै᳚ । \newline
16. अ॒स्मै॒ सो॒म॒पी॒थꣳ सो॑मपी॒थ म॑स्मा अस्मै सोमपी॒थम् । \newline
17. सो॒म॒पी॒थम् प्र प्र सो॑मपी॒थꣳ सो॑मपी॒थम् प्र । \newline
18. सो॒म॒पी॒थमिति॑ सोम - पी॒थम् । \newline
19. प्र य॑च्छतो यच्छतः॒ प्र प्र य॑च्छतः । \newline
20. य॒च्छ॒त॒ उपोप॑ यच्छतो यच्छत॒ उप॑ । \newline
21. उपै॑न मेन॒ मुपोपै॑नम् । \newline
22. ए॒नꣳ॒॒ सो॒म॒पी॒थः सो॑मपी॒थ ए॑न मेनꣳ सोमपी॒थः । \newline
23. सो॒म॒पी॒थो न॑मति नमति सोमपी॒थः सो॑मपी॒थो न॑मति । \newline
24. सो॒म॒पी॒थ इति॑ सोम - पी॒थः । \newline
25. न॒म॒ति॒ यद् यन् न॑मति नमति॒ यत् । \newline
26. यदै॒न्द्र ऐ॒न्द्रो यद् यदै॒न्द्रः । \newline
27. ऐ॒न्द्रो भव॑ति॒ भव॑ त्यै॒न्द्र ऐ॒न्द्रो भव॑ति । \newline
28. भव॑तीन्द्रि॒य मि॑न्द्रि॒यम् भव॑ति॒ भव॑तीन्द्रि॒यम् । \newline
29. इ॒न्द्रि॒यं ॅवै वा इ॑न्द्रि॒य मि॑न्द्रि॒यं ॅवै । \newline
30. वै सो॑मपी॒थः सो॑मपी॒थो वै वै सो॑मपी॒थः । \newline
31. सो॒म॒पी॒थ इ॑न्द्रि॒य मि॑न्द्रि॒यꣳ सो॑मपी॒थः सो॑मपी॒थ इ॑न्द्रि॒यम् । \newline
32. सो॒म॒पी॒थ इति॑ सोम - पी॒थः । \newline
33. इ॒न्द्रि॒य मे॒वैवे न्द्रि॒य मि॑न्द्रि॒य मे॒व । \newline
34. ए॒व सो॑मपी॒थꣳ सो॑मपी॒थ मे॒वैव सो॑मपी॒थम् । \newline
35. सो॒म॒पी॒थ मवाव॑ सोमपी॒थꣳ सो॑मपी॒थ मव॑ । \newline
36. सो॒म॒पी॒थमिति॑ सोम - पी॒थम् । \newline
37. अव॑ रुन्धे रु॒न्धे ऽवाव॑ रुन्धे । \newline
38. रु॒न्धे॒ यद् यद् रु॑न्धे रुन्धे॒ यत् । \newline
39. यदा᳚ग्ने॒य आ᳚ग्ने॒यो यद् यदा᳚ग्ने॒यः । \newline
40. आ॒ग्ने॒यो भव॑ति॒ भव॑ त्याग्ने॒य आ᳚ग्ने॒यो भव॑ति । \newline
41. भव॑ त्याग्ने॒य आ᳚ग्ने॒यो भव॑ति॒ भव॑ त्याग्ने॒यः । \newline
42. आ॒ग्ने॒यो वै वा आ᳚ग्ने॒य आ᳚ग्ने॒यो वै । \newline
43. वै ब्रा᳚ह्म॒णो ब्रा᳚ह्म॒णो वै वै ब्रा᳚ह्म॒णः । \newline
44. ब्रा॒ह्म॒णः स्वाꣳ स्वाम् ब्रा᳚ह्म॒णो ब्रा᳚ह्म॒णः स्वाम् । \newline
45. स्वा मे॒वैव स्वाꣳ स्वा मे॒व । \newline
46. ए॒व दे॒वता᳚म् दे॒वता॑ मे॒वैव दे॒वता᳚म् । \newline
47. दे॒वता॒ मन्वनु॑ दे॒वता᳚म् दे॒वता॒ मनु॑ । \newline
48. अनु॒ सꣳ स मन्वनु॒ सम् । \newline
49. सम् त॑नोति तनोति॒ सꣳ सम् त॑नोति । \newline
50. त॒नो॒ति॒ पु॒न॒रु॒थ्सृ॒ष्टः पु॑नरुथ्सृ॒ष्ट स्त॑नोति तनोति पुनरुथ्सृ॒ष्टः । \newline
51. पु॒न॒रु॒थ्सृ॒ष्टो भ॑वति भवति पुनरुथ्सृ॒ष्टः पु॑नरुथ्सृ॒ष्टो भ॑वति । \newline
52. पु॒न॒रु॒थ्सृ॒ष्ट इति॑ पुनः - उ॒थ्सृ॒ष्टः । \newline
53. भ॒व॒ति॒ पु॒न॒रु॒थ्सृ॒ष्टः पु॑नरुथ्सृ॒ष्टो भ॑वति भवति पुनरुथ्सृ॒ष्टः । \newline
54. पु॒न॒रु॒थ्सृ॒ष्ट इ॑वे व पुनरुथ्सृ॒ष्टः पु॑नरुथ्सृ॒ष्ट इ॑व । \newline
55. पु॒न॒रु॒थ्सृ॒ष्ट इति॑ पुनः - उ॒थ्सृ॒ष्टः । \newline
56. इ॒व॒ हि हीवे॑ व॒ हि । \newline
57. ह्ये॑त स्यै॒तस्य॒ हि ह्ये॑तस्य॑ । \newline
58. ए॒तस्य॑ सोमपी॒थः सो॑मपी॒थ ए॒तस्यै॒तस्य॑ सोमपी॒थः । \newline

\textbf{Ghana Paata } \newline

1. तृ॒तीया॒त् पुरु॑षा॒त् पुरु॑षात् तृ॒तीया᳚त् तृ॒तीया॒त् पुरु॑षा॒थ् सोमꣳ॒॒ सोम॒म् पुरु॑षात् तृ॒तीया᳚त् तृ॒तीया॒त् पुरु॑षा॒थ् सोम᳚म् । \newline
2. पुरु॑षा॒थ् सोमꣳ॒॒ सोम॒म् पुरु॑षा॒त् पुरु॑षा॒थ् सोम॒न्न न सोम॒म् पुरु॑षा॒त् पुरु॑षा॒थ् सोम॒न्न । \newline
3. सोम॒न्न न सोमꣳ॒॒ सोम॒न्न पिब॑ति॒ पिब॑ति॒ न सोमꣳ॒॒ सोम॒न्न पिब॑ति । \newline
4. न पिब॑ति॒ पिब॑ति॒ न न पिब॑ती न्द्रा॒ग्नी इ॑न्द्रा॒ग्नी पिब॑ति॒ न न पिब॑ती न्द्रा॒ग्नी । \newline
5. पिब॑ती न्द्रा॒ग्नी इ॑न्द्रा॒ग्नी पिब॑ति॒ पिब॑तीन्द्रा॒ग्नी ए॒वैवे न्द्रा॒ग्नी पिब॑ति॒ पिब॑ती न्द्रा॒ग्नी ए॒व । \newline
6. इ॒न्द्रा॒ग्नी ए॒वैवे न्द्रा॒ग्नी इ॑न्द्रा॒ग्नी ए॒व स्वेन॒ स्वेनै॒वे न्द्रा॒ग्नी इ॑न्द्रा॒ग्नी ए॒व स्वेन॑ । \newline
7. इ॒न्द्रा॒ग्नी इती᳚न्द्र - अ॒ग्नी । \newline
8. ए॒व स्वेन॒ स्वेनै॒वैव स्वेन॑ भाग॒धेये॑न भाग॒धेये॑न॒ स्वेनै॒वैव स्वेन॑ भाग॒धेये॑न । \newline
9. स्वेन॑ भाग॒धेये॑न भाग॒धेये॑न॒ स्वेन॒ स्वेन॑ भाग॒धेये॒नो पोप॑ भाग॒धेये॑न॒ स्वेन॒ स्वेन॑ भाग॒धेये॒नोप॑ । \newline
10. भा॒ग॒धेये॒नो पोप॑ भाग॒धेये॑न भाग॒धेये॒नोप॑ धावति धाव॒त्युप॑ भाग॒धेये॑न भाग॒धेये॒नोप॑ धावति । \newline
11. भा॒ग॒धेये॒नेति॑ भाग - धेये॑न । \newline
12. उप॑ धावति धाव॒ त्युपोप॑ धावति॒ तौ तौ धा॑व॒ त्युपोप॑ धावति॒ तौ । \newline
13. धा॒व॒ति॒ तौ तौ धा॑वति धावति॒ ता वे॒वैव तौ धा॑वति धावति॒ ता वे॒व । \newline
14. ता वे॒वैव तौ ता वे॒वास्मा॑ अस्मा ए॒व तौ ता वे॒वास्मै᳚ । \newline
15. ए॒वास्मा॑ अस्मा ए॒वैवास्मै॑ सोमपी॒थꣳ सो॑मपी॒थ म॑स्मा ए॒वैवास्मै॑ सोमपी॒थम् । \newline
16. अ॒स्मै॒ सो॒म॒पी॒थꣳ सो॑मपी॒थ म॑स्मा अस्मै सोमपी॒थम् प्र प्र सो॑मपी॒थ म॑स्मा अस्मै सोमपी॒थम् प्र । \newline
17. सो॒म॒पी॒थम् प्र प्र सो॑मपी॒थꣳ सो॑मपी॒थम् प्र य॑च्छतो यच्छतः॒ प्र सो॑मपी॒थꣳ सो॑मपी॒थम् प्र य॑च्छतः । \newline
18. सो॒म॒पी॒थमिति॑ सोम - पी॒थम् । \newline
19. प्र य॑च्छतो यच्छतः॒ प्र प्र य॑च्छत॒ उपोप॑ यच्छतः॒ प्र प्र य॑च्छत॒ उप॑ । \newline
20. य॒च्छ॒त॒ उपोप॑ यच्छतो यच्छत॒ उपै॑न मेन॒ मुप॑ यच्छतो यच्छत॒ उपै॑नम् । \newline
21. उपै॑न मेन॒ मुपोपै॑नꣳ सोमपी॒थः सो॑मपी॒थ ए॑न॒ मुपोपै॑नꣳ सोमपी॒थः । \newline
22. ए॒नꣳ॒॒ सो॒म॒पी॒थः सो॑मपी॒थ ए॑न मेनꣳ सोमपी॒थो न॑मति नमति सोमपी॒थ ए॑न मेनꣳ सोमपी॒थो न॑मति । \newline
23. सो॒म॒पी॒थो न॑मति नमति सोमपी॒थः सो॑मपी॒थो न॑मति॒ यद् यन् न॑मति सोमपी॒थः सो॑मपी॒थो न॑मति॒ यत् । \newline
24. सो॒म॒पी॒थ इति॑ सोम - पी॒थः । \newline
25. न॒म॒ति॒ यद् यन् न॑मति नमति॒ यदै॒न्द्र ऐ॒न्द्रो यन् न॑मति नमति॒ यदै॒न्द्रः । \newline
26. यदै॒न्द्र ऐ॒न्द्रो यद् यदै॒न्द्रो भव॑ति॒ भव॑त्यै॒न्द्रो यद् यदै॒न्द्रो भव॑ति । \newline
27. ऐ॒न्द्रो भव॑ति॒ भव॑त्यै॒न्द्र ऐ॒न्द्रो भव॑तीन्द्रि॒य मि॑न्द्रि॒यम् भव॑त्यै॒न्द्र ऐ॒न्द्रो भव॑तीन्द्रि॒यम् । \newline
28. भव॑तीन्द्रि॒य मि॑न्द्रि॒यम् भव॑ति॒ भव॑तीन्द्रि॒यं ॅवै वा इ॑न्द्रि॒यम् भव॑ति॒ भव॑तीन्द्रि॒यं ॅवै । \newline
29. इ॒न्द्रि॒यं ॅवै वा इ॑न्द्रि॒य मि॑न्द्रि॒यं ॅवै सो॑मपी॒थः सो॑मपी॒थो वा इ॑न्द्रि॒य मि॑न्द्रि॒यं ॅवै सो॑मपी॒थः । \newline
30. वै सो॑मपी॒थः सो॑मपी॒थो वै वै सो॑मपी॒थ इ॑न्द्रि॒य मि॑न्द्रि॒यꣳ सो॑मपी॒थो वै वै सो॑मपी॒थ इ॑न्द्रि॒यम् । \newline
31. सो॒म॒पी॒थ इ॑न्द्रि॒य मि॑न्द्रि॒यꣳ सो॑मपी॒थः सो॑मपी॒थ इ॑न्द्रि॒य मे॒वैवे न्द्रि॒यꣳ सो॑मपी॒थः सो॑मपी॒थ इ॑न्द्रि॒य मे॒व । \newline
32. सो॒म॒पी॒थ इति॑ सोम - पी॒थः । \newline
33. इ॒न्द्रि॒य मे॒वैवे न्द्रि॒य मि॑न्द्रि॒य मे॒व सो॑मपी॒थꣳ सो॑मपी॒थ मे॒वे न्द्रि॒य मि॑न्द्रि॒य मे॒व सो॑मपी॒थम् । \newline
34. ए॒व सो॑मपी॒थꣳ सो॑मपी॒थ मे॒वैव सो॑मपी॒थ मवाव॑ सोमपी॒थ मे॒वैव सो॑मपी॒थ मव॑ । \newline
35. सो॒म॒पी॒थ मवाव॑ सोमपी॒थꣳ सो॑मपी॒थ मव॑ रुन्धे रु॒न्धे ऽव॑ सोमपी॒थꣳ सो॑मपी॒थ मव॑ रुन्धे । \newline
36. सो॒म॒पी॒थमिति॑ सोम - पी॒थम् । \newline
37. अव॑ रुन्धे रु॒न्धे ऽवाव॑ रुन्धे॒ यद् यद् रु॒न्धे ऽवाव॑ रुन्धे॒ यत् । \newline
38. रु॒न्धे॒ यद् यद् रु॑न्धे रुन्धे॒ यदा᳚ग्ने॒य आ᳚ग्ने॒यो यद् रु॑न्धे रुन्धे॒ यदा᳚ग्ने॒यः । \newline
39. यदा᳚ग्ने॒य आ᳚ग्ने॒यो यद् यदा᳚ग्ने॒यो भव॑ति॒ भव॑ त्याग्ने॒यो यद् यदा᳚ग्ने॒यो भव॑ति । \newline
40. आ॒ग्ने॒यो भव॑ति॒ भव॑ त्याग्ने॒य आ᳚ग्ने॒यो भव॑त्याग्ने॒य आ᳚ग्ने॒यो भव॑त्याग्ने॒य आ᳚ग्ने॒यो भव॑ त्याग्ने॒यः । \newline
41. भव॑ त्याग्ने॒य आ᳚ग्ने॒यो भव॑ति॒ भव॑ त्याग्ने॒यो वै वा आ᳚ग्ने॒यो भव॑ति॒ भव॑ त्याग्ने॒यो वै । \newline
42. आ॒ग्ने॒यो वै वा आ᳚ग्ने॒य आ᳚ग्ने॒यो वै ब्रा᳚ह्म॒णो ब्रा᳚ह्म॒णो वा आ᳚ग्ने॒य आ᳚ग्ने॒यो वै ब्रा᳚ह्म॒णः । \newline
43. वै ब्रा᳚ह्म॒णो ब्रा᳚ह्म॒णो वै वै ब्रा᳚ह्म॒णः स्वाꣳ स्वाम् ब्रा᳚ह्म॒णो वै वै ब्रा᳚ह्म॒णः स्वाम् । \newline
44. ब्रा॒ह्म॒णः स्वाꣳ स्वाम् ब्रा᳚ह्म॒णो ब्रा᳚ह्म॒णः स्वा मे॒वैव स्वाम् ब्रा᳚ह्म॒णो ब्रा᳚ह्म॒णः स्वा मे॒व । \newline
45. स्वा मे॒वैव स्वाꣳ स्वा मे॒व दे॒वता᳚म् दे॒वता॑ मे॒व स्वाꣳ स्वा मे॒व दे॒वता᳚म् । \newline
46. ए॒व दे॒वता᳚म् दे॒वता॑ मे॒वैव दे॒वता॒ मन्वनु॑ दे॒वता॑ मे॒वैव दे॒वता॒ मनु॑ । \newline
47. दे॒वता॒ मन्वनु॑ दे॒वता᳚म् दे॒वता॒ मनु॒ सꣳ स मनु॑ दे॒वता᳚म् दे॒वता॒ मनु॒ सम् । \newline
48. अनु॒ सꣳ स मन्वनु॒ सम् त॑नोति तनोति॒ स मन्वनु॒ सम् त॑नोति । \newline
49. सम् त॑नोति तनोति॒ सꣳ सम् त॑नोति पुनरुथ्सृ॒ष्टः पु॑नरुथ्सृ॒ष्ट स्त॑नोति॒ सꣳ सम् त॑नोति पुनरुथ्सृ॒ष्टः । \newline
50. त॒नो॒ति॒ पु॒न॒रु॒थ्सृ॒ष्टः पु॑नरुथ्सृ॒ष्ट स्त॑नोति तनोति पुनरुथ्सृ॒ष्टो भ॑वति भवति पुनरुथ्सृ॒ष्ट स्त॑नोति तनोति पुनरुथ्सृ॒ष्टो भ॑वति । \newline
51. पु॒न॒रु॒थ्सृ॒ष्टो भ॑वति भवति पुनरुथ्सृ॒ष्टः पु॑नरुथ्सृ॒ष्टो भ॑वति पुनरुथ्सृ॒ष्टः पु॑नरुथ्सृ॒ष्टो भ॑वति पुनरुथ्सृ॒ष्टः पु॑नरुथ्सृ॒ष्टो भ॑वति पुनरुथ्सृ॒ष्टः । \newline
52. पु॒न॒रु॒थ्सृ॒ष्ट इति॑ पुनः - उ॒थ्सृ॒ष्टः । \newline
53. भ॒व॒ति॒ पु॒न॒रु॒थ्सृ॒ष्टः पु॑नरुथ्सृ॒ष्टो भ॑वति भवति पुनरुथ्सृ॒ष्ट इ॑वे व पुनरुथ्सृ॒ष्टो भ॑वति भवति पुनरुथ्सृ॒ष्ट इ॑व । \newline
54. पु॒न॒रु॒थ्सृ॒ष्ट इ॑वे व पुनरुथ्सृ॒ष्टः पु॑नरुथ्सृ॒ष्ट इ॑व॒ हि हीव॑ पुनरुथ्सृ॒ष्टः पु॑नरुथ्सृ॒ष्ट इ॑व॒ हि । \newline
55. पु॒न॒रु॒थ्सृ॒ष्ट इति॑ पुनः - उ॒थ्सृ॒ष्टः । \newline
56. इ॒व॒ हि हीवे॑ व॒ ह्ये॑त स्यै॒तस्य॒ हीवे॑ व॒ ह्ये॑तस्य॑ । \newline
57. ह्ये॑त स्यै॒तस्य॒ हि ह्ये॑तस्य॑ सोमपी॒थः सो॑मपी॒थ ए॒तस्य॒ हि ह्ये॑तस्य॑ सोमपी॒थः । \newline
58. ए॒तस्य॑ सोमपी॒थः सो॑मपी॒थ ए॒तस्यै॒तस्य॑ सोमपी॒थः समृ॑द्ध्यै॒ समृ॑द्ध्यै सोमपी॒थ ए॒तस्यै॒तस्य॑ सोमपी॒थः समृ॑द्ध्यै । \newline
\pagebreak
\markright{ TS 2.1.5.7  \hfill https://www.vedavms.in \hfill}

\section{ TS 2.1.5.7 }

\textbf{TS 2.1.5.7 } \newline
\textbf{Samhita Paata} \newline

सोमपी॒थः समृ॑द्ध्यै ब्राह्मणस्प॒त्यं तू॑प॒रमा ल॑भेता-भि॒चर॒न् ब्रह्म॑ण॒स्पति॑मे॒व स्वेन॑ भाग॒धेये॒नोप॑ धावति॒ तस्मा॑ ए॒वैन॒मा वृ॑श्चति ता॒जगार्ति॒-मार्च्छ॑ति तूप॒रो भ॑वति क्षु॒रप॑वि॒र्वा ए॒षा ल॒क्ष्मी यत् तू॑प॒रः समृ॑द्ध्यै॒ स्फ्यो यूपो॑ भवति॒ वज्रो॒ वै स्फ्यो वज्र॑मे॒वास्मै॒ प्रह॑रति शर॒मयं॑ ब॒र्॒.हिः शृ॒णात्ये॒वैनं॒ ॅवैभी॑दक इ॒द्ध्मो ( ) भि॒नत्त्ये॒वैनं᳚ ॥ \newline

\textbf{Pada Paata} \newline

सो॒म॒पी॒थ इति॑ सोम - पी॒थः । समृ॑द्ध्या॒ इति॒ सं - ऋ॒द्ध्यै॒ । ब्रा॒ह्म॒ण॒स्प॒त्यमिति॑ ब्राह्मणः - प॒त्यम् । तू॒प॒रम् । एति॑ । ल॒भे॒त॒ । अ॒भि॒चर॒न्नित्य॑भि - चरन्न्॑ । ब्रह्म॑णः । पति᳚म् । ए॒व । स्वेन॑ । भा॒ग॒धेये॒नेति॑ भाग - धेये॑न । उपेति॑ । धा॒व॒ति॒ । तस्मै᳚ । ए॒व । ए॒न॒म् । एति॑ । वृ॒श्च॒ति॒ । ता॒जक् । आर्ति᳚म् । एति॑ । ऋ॒च्छ॒ति॒ । तू॒प॒रः । भ॒व॒ति॒ । क्षु॒रप॑वि॒रिति॑ क्षु॒र - प॒विः॒ । वै । ए॒षा । ल॒क्ष्मी । यत् । तू॒प॒रः । समृ॑द्ध्य॒ इति॒ सं - ऋ॒द्ध्यै॒ । स्फ्यः । यूपः॑ । भ॒व॒ति॒ । वज्रः॑ । वै । स्फ्यः । वज्र᳚म् । ए॒व । अ॒स्मै॒ । प्रेति॑ । ह॒र॒ति॒ । श॒र॒मय॒मिति॑ शर - मय᳚म् । ब॒र्॒.हिः । शृ॒णाति॑ । ए॒व । ए॒न॒म् । वैभी॑दकः । इ॒द्ध्मः ( ) । भि॒नत्ति॑ । ए॒व । ए॒न॒म् ॥  \newline


\textbf{Krama Paata} \newline

सो॒म॒पी॒थः समृ॑द्ध्यै । सो॒म॒पी॒थ इति॑ सोम - पी॒थः । समृ॑द्ध्यै ब्राह्मणस्प॒त्यम् । समृ॑द्ध्या॒ इति॒ सं - ऋ॒द्ध्यै॒ । ब्रा॒ह्म॒ण॒स्प॒त्यम् तू॑प॒रम् । ब्रा॒ह्म॒ण॒स्प॒त्यमिति॑ ब्राह्मणः - प॒त्यम् । तू॒प॒रमा । आ ल॑भेत । ल॒भे॒ता॒भि॒चरन्न्॑ । अ॒भि॒चर॒न् ब्रह्म॑णः । अ॒भि॒चर॒न्नित्य॑भि - चरन्न्॑ । ब्रह्म॑ण॒स्पति᳚म् । पति॑मे॒व । ए॒व स्वेन॑ । स्वेन॑ भाग॒धेये॑न । भा॒ग॒धेये॒नोप॑ । भा॒ग॒धेये॒नेति॑ भाग - धेये॑न । उप॑ धावति । धा॒व॒ति॒ तस्मै᳚ । तस्मा॑ ए॒व । ए॒वैन᳚म् । ए॒न॒मा । आ वृ॑श्चति । वृ॒श्च॒ति॒ ता॒जक् । ता॒जगार्ति᳚म् । आर्ति॒मा । आर्च्छ॑ति । ऋ॒च्छ॒ति॒ तू॒प॒रः । तू॒प॒रो भ॑वति । भ॒व॒ति॒ क्षु॒रप॑विः । क्षु॒रप॑वि॒र् वै । क्षु॒रप॑वि॒रिति॑ क्षु॒र - प॒विः॒ । वा ए॒षा । ए॒षा ल॒क्ष्मी । ल॒क्ष्मी यत् । यत् तू॑प॒रः । तू॒प॒रः समृ॑द्ध्यै । समृ॑द्ध्यै॒ स्फ्यः । समृ॑द्ध्या॒ इति॒ सं - ऋ॒द्ध्यै॒ । स्फ्यो यूपः॑ । यूपो॑ भवति । भ॒व॒ति॒ वज्रः॑ । वज्रो॒ वै । वै स्फ्यः । स्फ्यो वज्र᳚म् । वज्र॑मे॒व । ए॒वास्मै᳚ । अ॒स्मै॒ प्र । प्र ह॑रति । ह॒र॒ति॒ श॒र॒मय᳚म् । श॒र॒मय॑म् ब॒र्.॒हिः । श॒र॒मय॒मिति॑ शर - मय᳚म् । ब॒र्॒.हिः शृ॒णाति॑ । शृ॒णात्ये॒व । ए॒वैन᳚म् । ए॒नं॒ ॅवैभी॑दकः । वैभी॑दक इ॒ध्मः ( ) । इ॒ध्मो भि॒नत्ति॑ । भि॒नत्त्ये॒व । ए॒वैन᳚म् । 
ए॒न॒मित्ये॑नम् । \newline

\textbf{Jatai Paata} \newline

1. सो॒म॒पी॒थः समृ॑द्ध्यै॒ समृ॑द्ध्यै सोमपी॒थः सो॑मपी॒थः समृ॑द्ध्यै । \newline
2. सो॒म॒पी॒थ इति॑ सोम - पी॒थः । \newline
3. समृ॑द्ध्यै ब्राह्मणस्प॒त्यम् ब्रा᳚ह्मणस्प॒त्यꣳ समृ॑द्ध्यै॒ समृ॑द्ध्यै ब्राह्मणस्प॒त्यम् । \newline
4. समृ॑द्ध्या॒ इति॒ सं - ऋ॒द्ध्यै॒ । \newline
5. ब्रा॒ह्म॒ण॒स्प॒त्यम् तू॑प॒रम् तू॑प॒रम् ब्रा᳚ह्मणस्प॒त्यम् ब्रा᳚ह्मणस्प॒त्यम् तू॑प॒रम् । \newline
6. ब्रा॒ह्म॒ण॒स्प॒त्यमिति॑ ब्राह्मणः - प॒त्यम् । \newline
7. तू॒प॒र मा तू॑प॒रम् तू॑प॒र मा । \newline
8. आ ल॑भेत लभे॒ता ल॑भेत । \newline
9. ल॒भे॒ता॒ भि॒चर॑न् नभि॒चर॑न् ॅलभेत लभेता भि॒चरन्न्॑ । \newline
10. अ॒भि॒चर॒न् ब्रह्म॑णो॒ ब्रह्म॑णो ऽभि॒चर॑न् नभि॒चर॒न् ब्रह्म॑णः । \newline
11. अ॒भि॒चर॒न्नित्य॑भि - चरन्न्॑ । \newline
12. ब्रह्म॒ण स्पति॒म् पति॒म् ब्रह्म॑णो॒ ब्रह्म॒ण स्पति᳚म् । \newline
13. पति॑ मे॒वैव पति॒म् पति॑ मे॒व । \newline
14. ए॒व स्वेन॒ स्वे नै॒वैव स्वेन॑ । \newline
15. स्वेन॑ भाग॒धेये॑न भाग॒धेये॑न॒ स्वेन॒ स्वेन॑ भाग॒धेये॑न । \newline
16. भा॒ग॒धेये॒नोपोप॑ भाग॒धेये॑न भाग॒धेये॒नोप॑ । \newline
17. भा॒ग॒धेये॒नेति॑ भाग - धेये॑न । \newline
18. उप॑ धावति धाव॒ त्युपोप॑ धावति । \newline
19. धा॒व॒ति॒ तस्मै॒ तस्मै॑ धावति धावति॒ तस्मै᳚ । \newline
20. तस्मा॑ ए॒वैव तस्मै॒ तस्मा॑ ए॒व । \newline
21. ए॒वैन॑ मेन मे॒वैवैन᳚म् । \newline
22. ए॒न॒ मैन॑ मेन॒ मा । \newline
23. आ वृ॑श्चति वृश्च॒त्या वृ॑श्चति । \newline
24. वृ॒श्च॒ति॒ ता॒जक् ता॒जग् वृ॑श्चति वृश्चति ता॒जक् । \newline
25. ता॒जगार्ति॒ मार्ति॑म् ता॒जक् ता॒जगार्ति᳚म् । \newline
26. आर्ति॒ मा ऽऽर्ति॒ मार्ति॒ मा । \newline
27. आर्च्छ॑ त्यृच्छ॒ त्यार्च्छ॑ति । \newline
28. ऋ॒च्छ॒ति॒ तू॒प॒र स्तू॑प॒र ऋ॑च्छ त्यृच्छति तूप॒रः । \newline
29. तू॒प॒रो भ॑वति भवति तूप॒र स्तू॑प॒रो भ॑वति । \newline
30. भ॒व॒ति॒ क्षु॒रप॑विः क्षु॒रप॑विर् भवति भवति क्षु॒रप॑विः । \newline
31. क्षु॒रप॑वि॒र् वै वै क्षु॒रप॑विः क्षु॒रप॑वि॒र् वै । \newline
32. क्षु॒रप॑वि॒रिति॑ क्षु॒र - प॒विः॒ । \newline
33. वा ए॒षैषा वै वा ए॒षा । \newline
34. ए॒षा ल॒क्ष्मी ल॒क्ष्म्ये॑षैषा ल॒क्ष्मी । \newline
35. ल॒क्ष्मी यद् यल्ल॒क्ष्मी ल॒क्ष्मी यत् । \newline
36. यत् तू॑प॒र स्तू॑प॒रो यद् यत् तू॑प॒रः । \newline
37. तू॒प॒रः समृ॑द्ध्यै॒ समृ॑द्ध्यै तूप॒र स्तू॑प॒रः समृ॑द्ध्यै । \newline
38. समृ॑द्ध्यै॒ स्फ्यः स्फ्यः समृ॑द्ध्यै॒ समृ॑द्ध्यै॒ स्फ्यः । \newline
39. समृ॑द्ध्या॒ इति॒ सं - ऋ॒द्ध्यै॒ । \newline
40. स्फ्यो यूपो॒ यूपः॒ स्फ्यः स्फ्यो यूपः॑ । \newline
41. यूपो॑ भवति भवति॒ यूपो॒ यूपो॑ भवति । \newline
42. भ॒व॒ति॒ वज्रो॒ वज्रो॑ भवति भवति॒ वज्रः॑ । \newline
43. वज्रो॒ वै वै वज्रो॒ वज्रो॒ वै । \newline
44. वै स्फ्यः स्फ्यो वै वै स्फ्यः । \newline
45. स्फ्यो वज्रं॒ ॅवज्रꣳ॒॒ स्फ्यः स्फ्यो वज्र᳚म् । \newline
46. वज्र॑ मे॒वैव वज्रं॒ ॅवज्र॑ मे॒व । \newline
47. ए॒वास्मा॑ अस्मा ए॒वैवास्मै᳚ । \newline
48. अ॒स्मै॒ प्र प्रास्मा॑ अस्मै॒ प्र । \newline
49. प्र ह॑रति हरति॒ प्र प्र ह॑रति । \newline
50. ह॒र॒ति॒ श॒र॒मयꣳ॑ शर॒मयꣳ॑ हरति हरति शर॒मय᳚म् । \newline
51. श॒र॒मय॑म् ब॒र्॒.हिर् ब॒र्॒.हिः श॑र॒मयꣳ॑ शर॒मय॑म् ब॒र्॒.हिः । \newline
52. श॒र॒मय॒मिति॑ शर - मय᳚म् । \newline
53. ब॒र्॒.हिः शृ॒णाति॑ शृ॒णाति॑ ब॒र्॒.हिर् ब॒र्॒.हिः शृ॒णाति॑ । \newline
54. शृ॒णा त्ये॒वैव शृ॒णाति॑ शृ॒णा त्ये॒व । \newline
55. ए॒वैन॑ मेन मे॒वैवैन᳚म् । \newline
56. ए॒नं॒ ॅवैभी॑दको॒ वैभी॑दक एन मेनं॒ ॅवैभी॑दकः । \newline
57. वैभी॑दक इ॒द्ध्म इ॒द्ध्मो वैभी॑दको॒ वैभी॑दक इ॒द्ध्मः । \newline
58. इ॒द्ध्मो भि॒नत्ति॑ भि॒नत्ती॒द्ध्म इ॒द्ध्मो भि॒नत्ति॑ । \newline
59. भि॒न त्त्ये॒वैव भि॒नत्ति॑ भि॒न त्त्ये॒व । \newline
60. ए॒वैन॑ मेन मे॒वैवैन᳚म् । \newline
61. ए॒न॒मित्ये॑नम् । \newline

\textbf{Ghana Paata } \newline

1. सो॒म॒पी॒थः समृ॑द्ध्यै॒ समृ॑द्ध्यै सोमपी॒थः सो॑मपी॒थः समृ॑द्ध्यै ब्राह्मणस्प॒त्यम् ब्रा᳚ह्मणस्प॒त्यꣳ समृ॑द्ध्यै सोमपी॒थः सो॑मपी॒थः समृ॑द्ध्यै ब्राह्मणस्प॒त्यम् । \newline
2. सो॒म॒पी॒थ इति॑ सोम - पी॒थः । \newline
3. समृ॑द्ध्यै ब्राह्मणस्प॒त्यम् ब्रा᳚ह्मणस्प॒त्यꣳ समृ॑द्ध्यै॒ समृ॑द्ध्यै ब्राह्मणस्प॒त्यम् तू॑प॒रम् तू॑प॒रम् ब्रा᳚ह्मणस्प॒त्यꣳ समृ॑द्ध्यै॒ समृ॑द्ध्यै ब्राह्मणस्प॒त्यम् तू॑प॒रम् । \newline
4. समृ॑द्ध्या॒ इति॒ सं - ऋ॒द्ध्यै॒ । \newline
5. ब्रा॒ह्म॒ण॒स्प॒त्यम् तू॑प॒रम् तू॑प॒रम् ब्रा᳚ह्मणस्प॒त्यम् ब्रा᳚ह्मणस्प॒त्यम् तू॑प॒र मा तू॑प॒रम् ब्रा᳚ह्मणस्प॒त्यम् ब्रा᳚ह्मणस्प॒त्यम् तू॑प॒र मा । \newline
6. ब्रा॒ह्म॒ण॒स्प॒त्यमिति॑ ब्राह्मणः - प॒त्यम् । \newline
7. तू॒प॒र मा तू॑प॒रम् तू॑प॒र मा ल॑भेत लभे॒ता तू॑प॒रम् तू॑प॒र मा ल॑भेत । \newline
8. आ ल॑भेत लभे॒ता ल॑भेता भि॒चर॑न् नभि॒चर॑न् ॅलभे॒ता ल॑भेता भि॒चरन्न्॑ । \newline
9. ल॒भे॒ता॒ भि॒चर॑न् नभि॒चर॑न् ॅलभेत लभेता भि॒चर॒न् ब्रह्म॑णो॒ ब्रह्म॑णो ऽभि॒चर॑न् ॅलभेत लभेता भि॒चर॒न् ब्रह्म॑णः । \newline
10. अ॒भि॒चर॒न् ब्रह्म॑णो॒ ब्रह्म॑णो ऽभि॒चर॑न् नभि॒चर॒न् ब्रह्म॒ण स्पति॒म् पति॒म् ब्रह्म॑णो ऽभि॒चर॑न् नभि॒चर॒न् ब्रह्म॒ण स्पति᳚म् । \newline
11. अ॒भि॒चर॒न्नित्य॑भि - चरन्न्॑ । \newline
12. ब्रह्म॒ण स्पति॒म् पति॒म् ब्रह्म॑णो॒ ब्रह्म॒ण स्पति॑ मे॒वैव पति॒म् ब्रह्म॑णो॒ ब्रह्म॒ण स्पति॑ मे॒व । \newline
13. पति॑ मे॒वैव पति॒म् पति॑ मे॒व स्वेन॒ स्वेनै॒व पति॒म् पति॑ मे॒व स्वेन॑ । \newline
14. ए॒व स्वेन॒ स्वेनै॒वैव स्वेन॑ भाग॒धेये॑न भाग॒धेये॑न॒ स्वेनै॒वैव स्वेन॑ भाग॒धेये॑न । \newline
15. स्वेन॑ भाग॒धेये॑न भाग॒धेये॑न॒ स्वेन॒ स्वेन॑ भाग॒धेये॒नो पोप॑ भाग॒धेये॑न॒ स्वेन॒ स्वेन॑ भाग॒धेये॒नोप॑ । \newline
16. भा॒ग॒धेये॒नो पोप॑ भाग॒धेये॑न भाग॒धेये॒नोप॑ धावति धाव॒त्युप॑ भाग॒धेये॑न भाग॒धेये॒नोप॑ धावति । \newline
17. भा॒ग॒धेये॒नेति॑ भाग - धेये॑न । \newline
18. उप॑ धावति धाव॒ त्युपोप॑ धावति॒ तस्मै॒ तस्मै॑ धाव॒ त्युपोप॑ धावति॒ तस्मै᳚ । \newline
19. धा॒व॒ति॒ तस्मै॒ तस्मै॑ धावति धावति॒ तस्मा॑ ए॒वैव तस्मै॑ धावति धावति॒ तस्मा॑ ए॒व । \newline
20. तस्मा॑ ए॒वैव तस्मै॒ तस्मा॑ ए॒वैन॑ मेन मे॒व तस्मै॒ तस्मा॑ ए॒वैन᳚म् । \newline
21. ए॒वैन॑ मेन मे॒वैवैन॒ मैन॑ मे॒वैवैन॒ मा । \newline
22. ए॒न॒ मैन॑ मेन॒ मा वृ॑श्चति वृश्च॒त्यैन॑ मेन॒ मा वृ॑श्चति । \newline
23. आ वृ॑श्चति वृश्च॒त्या वृ॑श्चति ता॒जक् ता॒जग् वृ॑श्च॒त्या वृ॑श्चति ता॒जक् । \newline
24. वृ॒श्च॒ति॒ ता॒जक् ता॒जग् वृ॑श्चति वृश्चति ता॒जगार्ति॒ मार्ति॑म् ता॒जग् वृ॑श्चति वृश्चति ता॒जगार्ति᳚म् । \newline
25. ता॒जगार्ति॒ मार्ति॑म् ता॒जक् ता॒जगार्ति॒ मा ऽऽर्ति॑म् ता॒जक् ता॒जगार्ति॒ मा । \newline
26. आर्ति॒ मा ऽऽर्ति॒ मार्ति॒ मार्च्छ॑ त्यृच्छ॒त्या ऽऽर्ति॒ मार्ति॒ मार्च्छ॑ति । \newline
27. आर्च्छ॑ त्यृच्छ॒ त्यार्च्छ॑ति तूप॒र स्तू॑प॒र ऋ॑च्छ ॒त्यार्च्छ॑ति तूप॒रः । \newline
28. ऋ॒च्छ॒ति॒ तू॒प॒र स्तू॑प॒र ऋ॑च्छ त्यृच्छति तूप॒रो भ॑वति भवति तूप॒र ऋ॑च्छ त्यृच्छति तूप॒रो भ॑वति । \newline
29. तू॒प॒रो भ॑वति भवति तूप॒र स्तू॑प॒रो भ॑वति क्षु॒रप॑विः क्षु॒रप॑विर् भवति तूप॒र स्तू॑प॒रो भ॑वति क्षु॒रप॑विः । \newline
30. भ॒व॒ति॒ क्षु॒रप॑विः क्षु॒रप॑विर् भवति भवति क्षु॒रप॑वि॒र् वै वै क्षु॒रप॑विर् भवति भवति क्षु॒रप॑वि॒र् वै । \newline
31. क्षु॒रप॑वि॒र् वै वै क्षु॒रप॑विः क्षु॒रप॑वि॒र् वा ए॒षैषा वै क्षु॒रप॑विः क्षु॒रप॑वि॒र् वा ए॒षा । \newline
32. क्षु॒रप॑वि॒रिति॑ क्षु॒र - प॒विः॒ । \newline
33. वा ए॒षैषा वै वा ए॒षा ल॒क्ष्मी ल॒क्ष्म्ये॑षा वै वा ए॒षा ल॒क्ष्मी । \newline
34. ए॒षा ल॒क्ष्मी ल॒क्ष्म्ये॑षैषा ल॒क्ष्मी यद् यल्ल॒क्ष्म्ये॑षैषा ल॒क्ष्मी यत् । \newline
35. ल॒क्ष्मी यद् यल्ल॒क्ष्मी ल॒क्ष्मी यत् तू॑प॒र स्तू॑प॒रो यल्ल॒क्ष्मी ल॒क्ष्मी यत् तू॑प॒रः । \newline
36. यत् तू॑प॒र स्तू॑प॒रो यद् यत् तू॑प॒रः समृ॑द्ध्यै॒ समृ॑द्ध्यै तूप॒रो यद् यत् तू॑प॒रः समृ॑द्ध्यै । \newline
37. तू॒प॒रः समृ॑द्ध्यै॒ समृ॑द्ध्यै तूप॒र स्तू॑प॒रः समृ॑द्ध्यै॒ स्फ्यः स्फ्यः समृ॑द्ध्यै तूप॒र स्तू॑प॒रः समृ॑द्ध्यै॒ स्फ्यः । \newline
38. समृ॑द्ध्यै॒ स्फ्यः स्फ्यः समृ॑द्ध्यै॒ समृ॑द्ध्यै॒ स्फ्यो यूपो॒ यूपः॒ स्फ्यः समृ॑द्ध्यै॒ समृ॑द्ध्यै॒ स्फ्यो यूपः॑ । \newline
39. समृ॑द्ध्या॒ इति॒ सं - ऋ॒द्ध्यै॒ । \newline
40. स्फ्यो यूपो॒ यूपः॒ स्फ्यः स्फ्यो यूपो॑ भवति भवति॒ यूपः॒ स्फ्यः स्फ्यो यूपो॑ भवति । \newline
41. यूपो॑ भवति भवति॒ यूपो॒ यूपो॑ भवति॒ वज्रो॒ वज्रो॑ भवति॒ यूपो॒ यूपो॑ भवति॒ वज्रः॑ । \newline
42. भ॒व॒ति॒ वज्रो॒ वज्रो॑ भवति भवति॒ वज्रो॒ वै वै वज्रो॑ भवति भवति॒ वज्रो॒ वै । \newline
43. वज्रो॒ वै वै वज्रो॒ वज्रो॒ वै स्फ्यः स्फ्यो वै वज्रो॒ वज्रो॒ वै स्फ्यः । \newline
44. वै स्फ्यः स्फ्यो वै वै स्फ्यो वज्रं॒ ॅवज्रꣳ॒॒ स्फ्यो वै वै स्फ्यो वज्र᳚म् । \newline
45. स्फ्यो वज्रं॒ ॅवज्रꣳ॒॒ स्फ्यः स्फ्यो वज्र॑ मे॒वैव वज्रꣳ॒॒ स्फ्यः स्फ्यो वज्र॑ मे॒व । \newline
46. वज्र॑ मे॒वैव वज्रं॒ ॅवज्र॑ मे॒वास्मा॑ अस्मा ए॒व वज्रं॒ ॅवज्र॑ मे॒वास्मै᳚ । \newline
47. ए॒वास्मा॑ अस्मा ए॒वैवास्मै॒ प्र प्रास्मा॑ ए॒वैवास्मै॒ प्र । \newline
48. अ॒स्मै॒ प्र प्रास्मा॑ अस्मै॒ प्र ह॑रति हरति॒ प्रास्मा॑ अस्मै॒ प्र ह॑रति । \newline
49. प्र ह॑रति हरति॒ प्र प्र ह॑रति शर॒मयꣳ॑ शर॒मयꣳ॑ हरति॒ प्र प्र ह॑रति शर॒मय᳚म् । \newline
50. ह॒र॒ति॒ श॒र॒मयꣳ॑ शर॒मयꣳ॑ हरति हरति शर॒मय॑म् ब॒र्॒.हिर् ब॒र्॒.हिः श॑र॒मयꣳ॑ हरति हरति शर॒मय॑म् ब॒र्॒.हिः । \newline
51. श॒र॒मय॑म् ब॒र्॒.हिर् ब॒र्॒.हिः श॑र॒मयꣳ॑ शर॒मय॑म् ब॒र्॒.हिः शृ॒णाति॑ शृ॒णाति॑ ब॒र्॒.हिः श॑र॒मयꣳ॑ शर॒मय॑म् ब॒र्॒.हिः शृ॒णाति॑ । \newline
52. श॒र॒मय॒मिति॑ शर - मय᳚म् । \newline
53. ब॒र्॒.हिः शृ॒णाति॑ शृ॒णाति॑ ब॒र्॒.हिर् ब॒र्॒.हिः शृ॒णा त्ये॒वैव शृ॒णाति॑ ब॒र्॒.हिर् ब॒र्॒.हिः शृ॒णात्ये॒व । \newline
54. शृ॒णा त्ये॒वैव शृ॒णाति॑ शृ॒णात्ये॒वैन॑ मेन मे॒व शृ॒णाति॑ शृ॒णा त्ये॒वैन᳚म् । \newline
55. ए॒वैन॑ मेन मे॒वैवैनं॒ ॅवैभी॑दको॒ वैभी॑दक एन मे॒वैवैनं॒ ॅवैभी॑दकः । \newline
56. ए॒नं॒ ॅवैभी॑दको॒ वैभी॑दक एन मेनं॒ ॅवैभी॑दक इ॒द्ध्म इ॒द्ध्मो वैभी॑दक एन मेनं॒ ॅवैभी॑दक इ॒द्ध्मः । \newline
57. वैभी॑दक इ॒द्ध्म इ॒द्ध्मो वैभी॑दको॒ वैभी॑दक इ॒द्ध्मो भि॒नत्ति॑ भि॒नत्ती॒द्ध्मो वैभी॑दको॒ वैभी॑दक इ॒द्ध्मो भि॒नत्ति॑ । \newline
58. इ॒द्ध्मो भि॒नत्ति॑ भि॒नत्ती॒द्ध्म इ॒द्ध्मो भि॒नत्त्ये॒वैव भि॒नत्ती॒द्ध्म इ॒द्ध्मो भि॒नत्त्ये॒व । \newline
59. भि॒नत्त्ये॒वैव भि॒नत्ति॑ भि॒नत्त्ये॒वैन॑ मेन मे॒व भि॒नत्ति॑ भि॒नत्त्ये॒वैन᳚म् । \newline
60. ए॒वैन॑ मेन मे॒वैवैन᳚म् । \newline
61. ए॒न॒मित्ये॑नम् । \newline
\pagebreak
\markright{ TS 2.1.6.1  \hfill https://www.vedavms.in \hfill}

\section{ TS 2.1.6.1 }

\textbf{TS 2.1.6.1 } \newline
\textbf{Samhita Paata} \newline

बा॒र्॒.ह॒स्प॒त्यꣳ शि॑तिपृ॒ष्ठमा ल॑भेत॒ ग्राम॑कामो॒ यः का॒मये॑त पृ॒ष्ठꣳ स॑मा॒नानाꣳ॑ स्या॒मिति॒ बृह॒स्पति॑मे॒व स्वेन॑ भाग॒धेये॒नोप॑ धावति॒ स ए॒वैनं॑ पृ॒ष्ठꣳ स॑मा॒नानां᳚ करोति ग्रा॒म्ये॑व भ॑वति शितिपृ॒ष्ठो भ॑वति बार्.हस्प॒त्यो ह्ये॑ष दे॒वत॑या॒ समृ॑द्ध्यै पौ॒ष्णꣳ श्या॒ममा ल॑भे॒तान्न॑का॒मोऽन्नं॒ ॅवै पू॒षा पू॒षण॑मे॒व स्वेन॑ भाग॒धेये॒नोप॑ धावति॒ स ए॒वास्मा॒ - [  ] \newline

\textbf{Pada Paata} \newline

बा॒र्.॒ह॒स्प॒त्यम् । शि॒ति॒पृ॒ष्ठमिति॑ शिति - पृ॒ष्ठम् । एति॑ । ल॒भे॒त॒ । ग्राम॑काम॒ इति॒ ग्राम॑ - का॒मः॒ । यः । का॒मये॑त । पृ॒ष्ठम् । स॒मा॒नाना᳚म् । स्या॒म् । इति॑ । बृह॒स्पति᳚म् । ए॒व । स्वेन॑ । भा॒ग॒धेये॒नेति॑ भाग - धेये॑न । उपेति॑ । धा॒व॒ति॒ । सः । ए॒व । ए॒न॒म् । पृ॒ष्ठम् । स॒मा॒नाना᳚म् । क॒रो॒ति॒ । ग्रा॒मी । ए॒व । भ॒व॒ति॒ । शि॒ति॒पृ॒ष्ठ इति॑ शिति - पृ॒ष्ठः । भ॒व॒ति॒ । बा॒र्.॒ह॒स्प॒त्यः । हि । ए॒षः । दे॒वत॑या । समृ॑द्ध्या॒ इति॒ सं - ऋ॒द्ध्यै॒ । पौ॒ष्णम् । श्या॒मम् । एति॑ । ल॒भे॒त॒ । अन्न॑काम॒ इत्यन्न॑ - का॒मः॒ । अन्न᳚म् । वै । पू॒षा । पू॒षण᳚म् । ए॒व । स्वेन॑ । भा॒ग॒धेये॒नेति॑ भाग - धेये॑न । उपेति॑ । धा॒व॒ति॒ । सः । ए॒व । अ॒स्मै॒ ।  \newline


\textbf{Krama Paata} \newline

बा॒र्॒.ह॒स्प॒त्यꣳ शि॑तिपृ॒ष्ठम् । शि॒ति॒पृ॒ष्ठमा । शि॒ति॒पृ॒ष्ठमिति॑ शिति - पृ॒ष्ठम् । आ ल॑भेत । ल॒भे॒त॒ ग्राम॑कामः । ग्राम॑कामो॒ यः । ग्राम॑काम॒ इति॒ ग्राम॑ - का॒मः॒ । यः का॒मये॑त । का॒मये॑त पृ॒ष्ठम् । पृ॒ष्ठꣳ स॑मा॒नाना᳚म् । स॒मा॒नानाꣳ॑ स्याम् । स्या॒मिति॑ । इति॒ बृह॒स्पति᳚म् । बृह॒स्पति॑मे॒व । ए॒व स्वेन॑ । स्वेन॑ भाग॒धेये॑न । भा॒ग॒धेये॒नोप॑ । भा॒ग॒धेये॒नेति॑ भाग - धेये॑न । उप॑ धावति । धा॒व॒ति॒ सः । स ए॒व । ए॒वैन᳚म् । ए॒न॒म् पृ॒ष्ठम् । पृ॒ष्ठꣳ स॑मा॒नाना᳚म् । स॒मा॒नाना᳚म् करोति । क॒रो॒ति॒ ग्रा॒मी । ग्रा॒म्ये॑व । ए॒व भ॑वति । भ॒व॒ति॒ शि॒ति॒पृ॒ष्ठः । शि॒ति॒पृ॒ष्ठो भ॑वति । शि॒ति॒पृ॒ष्ठ इति॑ शिति - पृ॒ष्ठः । भ॒व॒ति॒ बा॒र्॒.ह॒स्प॒त्यः । बा॒र्॒.ह॒स्प॒त्यो हि । ह्ये॑षः । ए॒ष दे॒वत॑या । दे॒वत॑या॒ समृ॑द्ध्यै । समृ॑द्ध्यै पौ॒ष्णम् । समृ॑द्ध्या॒ इति॒ सं - ऋ॒द्ध्यै॒ । पौ॒ष्णꣳ श्या॒मम् । श्या॒ममा । आ ल॑भेत । ल॒भे॒तान्न॑कामः । अन्न॑का॒मोऽन्न᳚म् । अन्न॑काम॒ इत्यन्न॑ - का॒मः॒ । अन्नं॒ ॅवै । वै पू॒षा । पू॒षा पू॒षण᳚म् । पू॒षण॑मे॒व । ए॒व स्वेन॑ । स्वेन॑ भाग॒धेये॑न । भा॒ग॒धेये॒नोप॑ । भा॒ग॒धेये॒नेति॑ भाग - धेये॑न । उप॑ धावति । धा॒व॒ति॒ सः । स ए॒व । ए॒वास्मै᳚ । अ॒स्मा॒ अन्न᳚म् \newline

\textbf{Jatai Paata} \newline

1. बा॒र्॒.ह॒स्प॒त्यꣳ शि॑तिपृ॒ष्ठꣳ शि॑तिपृ॒ष्ठम् बा॑र्.हस्प॒त्यम् बा॑र्.हस्प॒त्यꣳ शि॑तिपृ॒ष्ठम् । \newline
2. शि॒ति॒पृ॒ष्ठ मा शि॑तिपृ॒ष्ठꣳ शि॑तिपृ॒ष्ठ मा । \newline
3. शि॒ति॒पृ॒ष्ठमिति॑ शिति - पृ॒ष्ठम् । \newline
4. आ ल॑भेत लभे॒ता ल॑भेत । \newline
5. ल॒भे॒त॒ ग्राम॑कामो॒ ग्राम॑कामो लभेत लभेत॒ ग्राम॑कामः । \newline
6. ग्राम॑कामो॒ यो यो ग्राम॑कामो॒ ग्राम॑कामो॒ यः । \newline
7. ग्राम॑काम॒ इति॒ ग्राम॑ - का॒मः॒ । \newline
8. यः का॒मये॑त का॒मये॑त॒ यो यः का॒मये॑त । \newline
9. का॒मये॑त पृ॒ष्ठम् पृ॒ष्ठम् का॒मये॑त का॒मये॑त पृ॒ष्ठम् । \newline
10. पृ॒ष्ठꣳ स॑मा॒नानाꣳ॑ समा॒नाना᳚म् पृ॒ष्ठम् पृ॒ष्ठꣳ स॑मा॒नाना᳚म् । \newline
11. स॒मा॒नानाꣳ॑ स्याꣳ स्याꣳ समा॒नानाꣳ॑ समा॒नानाꣳ॑ स्याम् । \newline
12. स्या॒ मितीति॑ स्याꣳ स्या॒ मिति॑ । \newline
13. इति॒ बृह॒स्पति॒म् बृह॒स्पति॒ मितीति॒ बृह॒स्पति᳚म् । \newline
14. बृह॒स्पति॑ मे॒वैव बृह॒स्पति॒म् बृह॒स्पति॑ मे॒व । \newline
15. ए॒व स्वेन॒ स्वेनै॒वैव स्वेन॑ । \newline
16. स्वेन॑ भाग॒धेये॑न भाग॒धेये॑न॒ स्वेन॒ स्वेन॑ भाग॒धेये॑न । \newline
17. भा॒ग॒धेये॒नोपोप॑ भाग॒धेये॑न भाग॒धेये॒नोप॑ । \newline
18. भा॒ग॒धेये॒नेति॑ भाग - धेये॑न । \newline
19. उप॑ धावति धाव॒ त्युपोप॑ धावति । \newline
20. धा॒व॒ति॒ स स धा॑वति धावति॒ सः । \newline
21. स ए॒वैव स स ए॒व । \newline
22. ए॒वैन॑ मेन मे॒वैवैन᳚म् । \newline
23. ए॒न॒म् पृ॒ष्ठम् पृ॒ष्ठ मे॑न मेनम् पृ॒ष्ठम् । \newline
24. पृ॒ष्ठꣳ स॑मा॒नानाꣳ॑ समा॒नाना᳚म् पृ॒ष्ठम् पृ॒ष्ठꣳ स॑मा॒नाना᳚म् । \newline
25. स॒मा॒नाना᳚म् करोति करोति समा॒नानाꣳ॑ समा॒नाना᳚म् करोति । \newline
26. क॒रो॒ति॒ ग्रा॒मी ग्रा॒मी क॑रोति करोति ग्रा॒मी । \newline
27. ग्रा॒म्ये॑वैव ग्रा॒मी ग्रा॒म्ये॑व । \newline
28. ए॒व भ॑वति भवत्ये॒वैव भ॑वति । \newline
29. भ॒व॒ति॒ शि॒ति॒पृ॒ष्ठः शि॑तिपृ॒ष्ठो भ॑वति भवति शितिपृ॒ष्ठः । \newline
30. शि॒ति॒पृ॒ष्ठो भ॑वति भवति शितिपृ॒ष्ठः शि॑तिपृ॒ष्ठो भ॑वति । \newline
31. शि॒ति॒पृ॒ष्ठ इति॑ शिति - पृ॒ष्ठः । \newline
32. भ॒व॒ति॒ बा॒र्॒.ह॒स्प॒त्यो बा॑र्.हस्प॒त्यो भ॑वति भवति बार्.हस्प॒त्यः । \newline
33. बा॒र्॒.ह॒स्प॒त्यो हि हि बा॑र्.हस्प॒त्यो बा॑र्.हस्प॒त्यो हि । \newline
34. ह्ये॑ष ए॒ष हि ह्ये॑षः । \newline
35. ए॒ष दे॒वत॑या दे॒वत॑यै॒ष ए॒ष दे॒वत॑या । \newline
36. दे॒वत॑या॒ समृ॑द्ध्यै॒ समृ॑द्ध्यै दे॒वत॑या दे॒वत॑या॒ समृ॑द्ध्यै । \newline
37. समृ॑द्ध्यै पौ॒ष्णम् पौ॒ष्णꣳ समृ॑द्ध्यै॒ समृ॑द्ध्यै पौ॒ष्णम् । \newline
38. समृ॑द्ध्या॒ इति॒ सं - ऋ॒द्ध्यै॒ । \newline
39. पौ॒ष्णꣳ श्या॒मꣳ श्या॒मम् पौ॒ष्णम् पौ॒ष्णꣳ श्या॒मम् । \newline
40. श्या॒म मा श्या॒मꣳ श्या॒म मा । \newline
41. आ ल॑भेत लभे॒ता ल॑भेत । \newline
42. ल॒भे॒तान्न॑का॒मो ऽन्न॑कामो लभेत लभे॒तान्न॑कामः । \newline
43. अन्न॑का॒मो ऽन्न॒ मन्न॒ मन्न॑का॒मो ऽन्न॑का॒मो ऽन्न᳚म् । \newline
44. अन्न॑काम॒ इत्यन्न॑ - का॒मः॒ । \newline
45. अन्नं॒ ॅवै वा अन्न॒ मन्नं॒ ॅवै । \newline
46. वै पू॒षा पू॒षा वै वै पू॒षा । \newline
47. पू॒षा पू॒षण॑म् पू॒षण॑म् पू॒षा पू॒षा पू॒षण᳚म् । \newline
48. पू॒षण॑ मे॒वैव पू॒षण॑म् पू॒षण॑ मे॒व । \newline
49. ए॒व स्वेन॒ स्वेनै॒वैव स्वेन॑ । \newline
50. स्वेन॑ भाग॒धेये॑न भाग॒धेये॑न॒ स्वेन॒ स्वेन॑ भाग॒धेये॑न । \newline
51. भा॒ग॒धेये॒नोपोप॑ भाग॒धेये॑न भाग॒धेये॒नोप॑ । \newline
52. भा॒ग॒धेये॒नेति॑ भाग - धेये॑न । \newline
53. उप॑ धावति धाव॒ त्युपोप॑ धावति । \newline
54. धा॒व॒ति॒ स स धा॑वति धावति॒ सः । \newline
55. स ए॒वैव स स ए॒व । \newline
56. ए॒वास्मा॑ अस्मा ए॒वैवास्मै᳚ । \newline
57. अ॒स्मा॒ अन्न॒ मन्न॑ मस्मा अस्मा॒ अन्न᳚म् । \newline

\textbf{Ghana Paata } \newline

1. बा॒र्॒.ह॒स्प॒त्यꣳ शि॑तिपृ॒ष्ठꣳ शि॑तिपृ॒ष्ठम् बा॑र्.हस्प॒त्यम् बा॑र्.हस्प॒त्यꣳ शि॑तिपृ॒ष्ठ मा शि॑तिपृ॒ष्ठम् बा॑र्.हस्प॒त्यम् बा॑र्.हस्प॒त्यꣳ शि॑तिपृ॒ष्ठ मा । \newline
2. शि॒ति॒पृ॒ष्ठ मा शि॑तिपृ॒ष्ठꣳ शि॑तिपृ॒ष्ठ मा ल॑भेत लभे॒ता शि॑तिपृ॒ष्ठꣳ शि॑तिपृ॒ष्ठ मा ल॑भेत । \newline
3. शि॒ति॒पृ॒ष्ठमिति॑ शिति - पृ॒ष्ठम् । \newline
4. आ ल॑भेत लभे॒ता ल॑भेत॒ ग्राम॑कामो॒ ग्राम॑कामो लभे॒ता ल॑भेत॒ ग्राम॑कामः । \newline
5. ल॒भे॒त॒ ग्राम॑कामो॒ ग्राम॑कामो लभेत लभेत॒ ग्राम॑कामो॒ यो यो ग्राम॑कामो लभेत लभेत॒ ग्राम॑कामो॒ यः । \newline
6. ग्राम॑कामो॒ यो यो ग्राम॑कामो॒ ग्राम॑कामो॒ यः का॒मये॑त का॒मये॑त॒ यो ग्राम॑कामो॒ ग्राम॑कामो॒ यः का॒मये॑त । \newline
7. ग्राम॑काम॒ इति॒ ग्राम॑ - का॒मः॒ । \newline
8. यः का॒मये॑त का॒मये॑त॒ यो यः का॒मये॑त पृ॒ष्ठम् पृ॒ष्ठम् का॒मये॑त॒ यो यः का॒मये॑त पृ॒ष्ठम् । \newline
9. का॒मये॑त पृ॒ष्ठम् पृ॒ष्ठम् का॒मये॑त का॒मये॑त पृ॒ष्ठꣳ स॑मा॒नानाꣳ॑ समा॒नाना᳚म् पृ॒ष्ठम् का॒मये॑त का॒मये॑त पृ॒ष्ठꣳ स॑मा॒नाना᳚म् । \newline
10. पृ॒ष्ठꣳ स॑मा॒नानाꣳ॑ समा॒नाना᳚म् पृ॒ष्ठम् पृ॒ष्ठꣳ स॑मा॒नानाꣳ॑ स्याꣳ स्याꣳ समा॒नाना᳚म् पृ॒ष्ठम् पृ॒ष्ठꣳ स॑मा॒नानाꣳ॑ स्याम् । \newline
11. स॒मा॒नानाꣳ॑ स्याꣳ स्याꣳ समा॒नानाꣳ॑ समा॒नानाꣳ॑ स्या॒ मितीति॑ स्याꣳ समा॒नानाꣳ॑ समा॒नानाꣳ॑ स्या॒ मिति॑ । \newline
12. स्या॒ मितीति॑ स्याꣳ स्या॒ मिति॒ बृह॒स्पति॒म् बृह॒स्पति॒ मिति॑ स्याꣳ स्या॒ मिति॒ बृह॒स्पति᳚म् । \newline
13. इति॒ बृह॒स्पति॒म् बृह॒स्पति॒ मितीति॒ बृह॒स्पति॑ मे॒वैव बृह॒स्पति॒ मितीति॒ बृह॒स्पति॑ मे॒व । \newline
14. बृह॒स्पति॑ मे॒वैव बृह॒स्पति॒म् बृह॒स्पति॑ मे॒व स्वेन॒ स्वेनै॒व बृह॒स्पति॒म् बृह॒स्पति॑ मे॒व स्वेन॑ । \newline
15. ए॒व स्वेन॒ स्वेनै॒वैव स्वेन॑ भाग॒धेये॑न भाग॒धेये॑न॒ स्वेनै॒वैव स्वेन॑ भाग॒धेये॑न । \newline
16. स्वेन॑ भाग॒धेये॑न भाग॒धेये॑न॒ स्वेन॒ स्वेन॑ भाग॒धेये॒नो पोप॑ भाग॒धेये॑न॒ स्वेन॒ स्वेन॑ भाग॒धेये॒नोप॑ । \newline
17. भा॒ग॒धेये॒नो पोप॑ भाग॒धेये॑न भाग॒धेये॒नोप॑ धावति धाव॒त्युप॑ भाग॒धेये॑न भाग॒धेये॒नोप॑ धावति । \newline
18. भा॒ग॒धेये॒नेति॑ भाग - धेये॑न । \newline
19. उप॑ धावति धाव॒ त्युपोप॑ धावति॒ स स धा॑व॒ त्युपोप॑ धावति॒ सः । \newline
20. धा॒व॒ति॒ स स धा॑वति धावति॒ स ए॒वैव स धा॑वति धावति॒ स ए॒व । \newline
21. स ए॒वैव स स ए॒वैन॑ मेन मे॒व स स ए॒वैन᳚म् । \newline
22. ए॒वैन॑ मेन मे॒वैवैन॑म् पृ॒ष्ठम् पृ॒ष्ठ मे॑न मे॒वैवैन॑म् पृ॒ष्ठम् । \newline
23. ए॒न॒म् पृ॒ष्ठम् पृ॒ष्ठ मे॑न मेनम् पृ॒ष्ठꣳ स॑मा॒नानाꣳ॑ समा॒नाना᳚म् पृ॒ष्ठ मे॑न मेनम् पृ॒ष्ठꣳ स॑मा॒नाना᳚म् । \newline
24. पृ॒ष्ठꣳ स॑मा॒नानाꣳ॑ समा॒नाना᳚म् पृ॒ष्ठम् पृ॒ष्ठꣳ स॑मा॒नाना᳚म् करोति करोति समा॒नाना᳚म् पृ॒ष्ठम् पृ॒ष्ठꣳ स॑मा॒नाना᳚म् करोति । \newline
25. स॒मा॒नाना᳚म् करोति करोति समा॒नानाꣳ॑ समा॒नाना᳚म् करोति ग्रा॒मी ग्रा॒मी क॑रोति समा॒नानाꣳ॑ समा॒नाना᳚म् करोति ग्रा॒मी । \newline
26. क॒रो॒ति॒ ग्रा॒मी ग्रा॒मी क॑रोति करोति ग्रा॒म्ये॑वैव ग्रा॒मी क॑रोति करोति ग्रा॒म्ये॑व । \newline
27. ग्रा॒म्ये॑वैव ग्रा॒मी ग्रा॒म्ये॑व भ॑वति भवत्ये॒व ग्रा॒मी ग्रा॒म्ये॑व भ॑वति । \newline
28. ए॒व भ॑वति भव त्ये॒वैव भ॑वति शितिपृ॒ष्ठः शि॑तिपृ॒ष्ठो भ॑व त्ये॒वैव भ॑वति शितिपृ॒ष्ठः । \newline
29. भ॒व॒ति॒ शि॒ति॒पृ॒ष्ठः शि॑तिपृ॒ष्ठो भ॑वति भवति शितिपृ॒ष्ठो भ॑वति भवति शितिपृ॒ष्ठो भ॑वति भवति शितिपृ॒ष्ठो भ॑वति । \newline
30. शि॒ति॒पृ॒ष्ठो भ॑वति भवति शितिपृ॒ष्ठः शि॑तिपृ॒ष्ठो भ॑वति बार्.हस्प॒त्यो बा॑र्.हस्प॒त्यो भ॑वति शितिपृ॒ष्ठः शि॑तिपृ॒ष्ठो भ॑वति बार्.हस्प॒त्यः । \newline
31. शि॒ति॒पृ॒ष्ठ इति॑ शिति - पृ॒ष्ठः । \newline
32. भ॒व॒ति॒ बा॒र्॒.ह॒स्प॒त्यो बा॑र्.हस्प॒त्यो भ॑वति भवति बार्.हस्प॒त्यो हि हि बा॑र्.हस्प॒त्यो भ॑वति भवति बार्.हस्प॒त्यो हि । \newline
33. बा॒र्॒.ह॒स्प॒त्यो हि हि बा॑र्.हस्प॒त्यो बा॑र्.हस्प॒त्यो ह्ये॑ष ए॒ष हि बा॑र्.हस्प॒त्यो बा॑र्.हस्प॒त्यो ह्ये॑षः । \newline
34. ह्ये॑ष ए॒ष हि ह्ये॑ष दे॒वत॑या दे॒वत॑यै॒ष हि ह्ये॑ष दे॒वत॑या । \newline
35. ए॒ष दे॒वत॑या दे॒वत॑यै॒ष ए॒ष दे॒वत॑या॒ समृ॑द्ध्यै॒ समृ॑द्ध्यै दे॒वत॑यै॒ष ए॒ष दे॒वत॑या॒ समृ॑द्ध्यै । \newline
36. दे॒वत॑या॒ समृ॑द्ध्यै॒ समृ॑द्ध्यै दे॒वत॑या दे॒वत॑या॒ समृ॑द्ध्यै पौ॒ष्णम् पौ॒ष्णꣳ समृ॑द्ध्यै दे॒वत॑या दे॒वत॑या॒ समृ॑द्ध्यै पौ॒ष्णम् । \newline
37. समृ॑द्ध्यै पौ॒ष्णम् पौ॒ष्णꣳ समृ॑द्ध्यै॒ समृ॑द्ध्यै पौ॒ष्णꣳ श्या॒मꣳ श्या॒मम् पौ॒ष्णꣳ समृ॑द्ध्यै॒ समृ॑द्ध्यै पौ॒ष्णꣳ श्या॒मम् । \newline
38. समृ॑द्ध्या॒ इति॒ सं - ऋ॒द्ध्यै॒ । \newline
39. पौ॒ष्णꣳ श्या॒मꣳ श्या॒मम् पौ॒ष्णम् पौ॒ष्णꣳ श्या॒म मा श्या॒मम् पौ॒ष्णम् पौ॒ष्णꣳ श्या॒म मा । \newline
40. श्या॒म मा श्या॒मꣳ श्या॒म मा ल॑भेत लभे॒ता श्या॒मꣳ श्या॒म मा ल॑भेत । \newline
41. आ ल॑भेत लभे॒ता ल॑भे॒ता न्न॑का॒मो ऽन्न॑कामो लभे॒ता ल॑भे॒ता न्न॑कामः । \newline
42. ल॒भे॒ता न्न॑का॒मो ऽन्न॑कामो लभेत लभे॒ता न्न॑का॒मो ऽन्न॒ मन्न॒ मन्न॑कामो लभेत लभे॒ता न्न॑का॒मो ऽन्न᳚म् । \newline
43. अन्न॑का॒मो ऽन्न॒ मन्न॒ मन्न॑का॒मो ऽन्न॑का॒मो ऽन्नं॒ ॅवै वा अन्न॒ मन्न॑का॒मो ऽन्न॑का॒मो ऽन्नं॒ ॅवै । \newline
44. अन्न॑काम॒इत्यन्न॑ - का॒मः॒ । \newline
45. अन्नं॒ ॅवै वा अन्न॒ मन्नं॒ ॅवै पू॒षा पू॒षा वा अन्न॒ मन्नं॒ ॅवै पू॒षा । \newline
46. वै पू॒षा पू॒षा वै वै पू॒षा पू॒षण॑म् पू॒षण॑म् पू॒षा वै वै पू॒षा पू॒षण᳚म् । \newline
47. पू॒षा पू॒षण॑म् पू॒षण॑म् पू॒षा पू॒षा पू॒षण॑ मे॒वैव पू॒षण॑म् पू॒षा पू॒षा पू॒षण॑ मे॒व । \newline
48. पू॒षण॑ मे॒वैव पू॒षण॑म् पू॒षण॑ मे॒व स्वेन॒ स्वेनै॒व पू॒षण॑म् पू॒षण॑ मे॒व स्वेन॑ । \newline
49. ए॒व स्वेन॒ स्वेनै॒वैव स्वेन॑ भाग॒धेये॑न भाग॒धेये॑न॒ स्वेनै॒वैव स्वेन॑ भाग॒धेये॑न । \newline
50. स्वेन॑ भाग॒धेये॑न भाग॒धेये॑न॒ स्वेन॒ स्वेन॑ भाग॒धेये॒नो पोप॑ भाग॒धेये॑न॒ स्वेन॒ स्वेन॑ भाग॒धेये॒नोप॑ । \newline
51. भा॒ग॒धेये॒नो पोप॑ भाग॒धेये॑न भाग॒धेये॒नोप॑ धावति धाव॒त्युप॑ भाग॒धेये॑न भाग॒धेये॒नोप॑ धावति । \newline
52. भा॒ग॒धेये॒नेति॑ भाग - धेये॑न । \newline
53. उप॑ धावति धाव॒ त्युपोप॑ धावति॒ स स धा॑व॒ त्युपोप॑ धावति॒ सः । \newline
54. धा॒व॒ति॒ स स धा॑वति धावति॒ स ए॒वैव स धा॑वति धावति॒ स ए॒व । \newline
55. स ए॒वैव स स ए॒वास्मा॑ अस्मा ए॒व स स ए॒वास्मै᳚ । \newline
56. ए॒वास्मा॑ अस्मा ए॒वैवास्मा॒ अन्न॒ मन्न॑ मस्मा ए॒वैवास्मा॒ अन्न᳚म् । \newline
57. अ॒स्मा॒ अन्न॒ मन्न॑ मस्मा अस्मा॒ अन्न॒म् प्र प्रान्न॑ मस्मा अस्मा॒ अन्न॒म् प्र । \newline
\pagebreak
\markright{ TS 2.1.6.2  \hfill https://www.vedavms.in \hfill}

\section{ TS 2.1.6.2 }

\textbf{TS 2.1.6.2 } \newline
\textbf{Samhita Paata} \newline

अन्नं॒ प्र य॑च्छत्यन्ना॒द ए॒व भ॑वति श्या॒मो भ॑वत्ये॒तद्वा अन्न॑स्य रू॒पꣳ समृ॑द्ध्यै मारु॒तं पृश्नि॒मा ल॑भे॒ताऽन्न॑-का॒मोऽन्नं॒ ॅवै म॒रुतो॑म॒रुत॑ ए॒व स्वेन॑ भाग॒धेये॒नोप॑ धावति॒ त ए॒वास्मा॒ अन्नं॒ प्रय॑च्छन्त्यन्ना॒द‌ए॒व भ॑वति॒ पृश्नि॑ र्भवत्ये॒तद्वा अन्न॑स्य रू॒पꣳ समृ॑द्ध्या ऐ॒न्द्रम॑रु॒णमा ल॑भेतेन्द्रि॒यका॑म॒ इन्द्र॑मे॒व - [  ] \newline

\textbf{Pada Paata} \newline

अन्न᳚म् । प्रेति॑ । य॒च्छ॒ति॒ । अ॒न्ना॒द इत्य॑न्न - अ॒दः । ए॒व । भ॒व॒ति॒ । श्या॒मः । भ॒व॒ति॒ । ए॒तत् । वै । अन्न॑स्य । रू॒पम् । समृ॑द्ध्या॒ इति॒ सं - ऋ॒द्ध्यै॒ । मा॒रु॒तम् । पृश्नि᳚म् । एति॑ । ल॒भे॒त॒ । अन्न॑काम॒ इत्यन्न॑ - का॒मः॒ । अन्न᳚म् । वै । म॒रुतः॑ । म॒रुतः॑ । ए॒व । स्वेन॑ । भा॒ग॒धेये॒नेति॑ भाग - धेये॑न । उपेति॑ । धा॒व॒ति॒ । ते । ए॒व । अ॒स्मै॒ । अन्न᳚म् । प्रेति॑ । य॒च्छ॒न्ति॒ । अ॒न्ना॒द इत्य॑न्न - अ॒दः । ए॒व । भ॒व॒ति॒ । पृश्निः॑ । भ॒व॒ति॒ । ए॒तत् । वै । अन्न॑स्य । रू॒पम् । समृ॑द्ध्या॒ इति॒ सं - ऋ॒द्ध्यै॒ । ऐ॒न्द्रम् । अ॒रु॒णम् । एति॑ । ल॒भे॒त॒ । इ॒न्द्रि॒यका॑म॒ इती᳚न्द्रि॒य - का॒मः॒ । इन्द्र᳚म् । ए॒व ।  \newline


\textbf{Krama Paata} \newline

अन्न॒म् प्र । प्र य॑च्छति । य॒च्छ॒त्य॒न्ना॒दः । अ॒न्ना॒द ए॒व । अ॒न्ना॒द इत्य॑न्न - अ॒दः । ए॒व भ॑वति । भ॒व॒ति॒ श्या॒मः । श्या॒मो भ॑वति । भ॒व॒त्ये॒तत् । ए॒तद् वै । वा अन्न॑स्य । अन्न॑स्य रू॒पम् । रू॒पꣳ समृ॑द्ध्यै । समृ॑द्ध्यै मारु॒तम् । समृ॑द्ध्या॒ इति॒ सं - ऋ॒द्ध्यै॒ । मा॒रु॒तम् पृश्ञि᳚म् । पृश्ञि॒मा । आ ल॑भेत । ल॒भे॒तान्न॑कामः । अन्न॑का॒मोऽन्न᳚म् । अन्न॑काम॒ इत्यन्न॑ - का॒मः॒ । अन्नं॒ ॅवै । वै म॒रुतः॑ । म॒रुतो॑ म॒रुतः॑ । म॒रुत॑ ए॒व । ए॒व स्वेन॑ । स्वेन॑ भाग॒धेये॑न । भा॒ग॒धेये॒नोप॑ । भा॒ग॒धेये॒नेति॑ भाग - धेये॑न । उप॑ धावति । धा॒व॒ति॒ ते । त ए॒व । ए॒वास्मै᳚ । अ॒स्मा॒ अन्न᳚म् । अन्न॒म् प्र । प्र य॑च्छन्ति । य॒च्छ॒न्त्य॒न्ना॒दः । अ॒न्ना॒द ए॒व । अ॒न्ना॒द इत्य॑न्न - अ॒दः । ए॒व भ॑वति । भ॒व॒ति॒ पृश्ञिः॑ । पृश्ञि॑र् भवति । भ॒व॒त्ये॒तत् । ए॒तद् वै । वा अन्न॑स्य । अन्न॑स्य रू॒पम् । रू॒पꣳ समृ॑द्ध्यै । समृ॑द्ध्या ऐ॒न्द्रम् । समृ॑द्ध्या॒ इति॒ सं - ऋ॒द्ध्यै॒ । ऐ॒न्द्रम॑रु॒णम् । अ॒रु॒णमा । आ ल॑भेत । ल॒भे॒ते॒न्द्रि॒यका॑मः । इ॒न्द्रि॒यका॑म॒ इन्द्र᳚म् । इ॒न्द्रि॒यका॑म॒ इती᳚न्द्रि॒य - का॒मः॒ । इन्द्र॑मे॒व । ए॒व स्वेन॑ \newline

\textbf{Jatai Paata} \newline

1. अन्न॒म् प्र प्रान्न॒ मन्न॒म् प्र । \newline
2. प्र य॑च्छति यच्छति॒ प्र प्र य॑च्छति । \newline
3. य॒च्छ॒ त्य॒न्ना॒दो᳚ ऽन्ना॒दो य॑च्छति यच्छ त्यन्ना॒दः । \newline
4. अ॒न्ना॒द ए॒वैवान्ना॒दो᳚ ऽन्ना॒द ए॒व । \newline
5. अ॒न्ना॒द इत्य॑न्न - अ॒दः । \newline
6. ए॒व भ॑वति भव त्ये॒वैव भ॑वति । \newline
7. भ॒व॒ति॒ श्या॒मः श्या॒मो भ॑वति भवति श्या॒मः । \newline
8. श्या॒मो भ॑वति भवति श्या॒मः श्या॒मो भ॑वति । \newline
9. भ॒व॒ त्ये॒त दे॒तद् भ॑वति भव त्ये॒तत् । \newline
10. ए॒तद् वै वा ए॒त दे॒तद् वै । \newline
11. वा अन्न॒स्यान्न॑स्य॒ वै वा अन्न॑स्य । \newline
12. अन्न॑स्य रू॒पꣳ रू॒प मन्न॒स्या न्न॑स्य रू॒पम् । \newline
13. रू॒पꣳ समृ॑द्ध्यै॒ समृ॑द्ध्यै रू॒पꣳ रू॒पꣳ समृ॑द्ध्यै । \newline
14. समृ॑द्ध्यै मारु॒तम् मा॑रु॒तꣳ समृ॑द्ध्यै॒ समृ॑द्ध्यै मारु॒तम् । \newline
15. समृ॑द्ध्या॒ इति॒ सं - ऋ॒द्ध्यै॒ । \newline
16. मा॒रु॒तम् पृश्ञि॒म् पृश्ञि॑म् मारु॒तम् मा॑रु॒तम् पृश्ञि᳚म् । \newline
17. पृश्ञि॒ मा पृश्ञि॒म् पृश्ञि॒ मा । \newline
18. आ ल॑भेत लभे॒ता ल॑भेत । \newline
19. ल॒भे॒ता न्न॑का॒मो ऽन्न॑कामो लभेत लभे॒ता न्न॑कामः । \newline
20. अन्न॑का॒मो ऽन्न॒ मन्न॒ मन्न॑का॒मो ऽन्न॑का॒मो ऽन्न᳚म् । \newline
21. अन्न॑काम॒इत्यन्न॑ - का॒मः॒ । \newline
22. अन्नं॒ ॅवै वा अन्न॒ मन्नं॒ ॅवै । \newline
23. वै म॒रुतो॑ म॒रुतो॒ वै वै म॒रुतः॑ । \newline
24. म॒रुतो॑ म॒रुतः॑ । \newline
25. म॒रुत॑ ए॒वैव म॒रुतो॑ म॒रुत॑ ए॒व । \newline
26. ए॒व स्वेन॒ स्वेनै॒वैव स्वेन॑ । \newline
27. स्वेन॑ भाग॒धेये॑न भाग॒धेये॑न॒ स्वेन॒ स्वेन॑ भाग॒धेये॑न । \newline
28. भा॒ग॒धेये॒नोपोप॑ भाग॒धेये॑न भाग॒धेये॒नोप॑ । \newline
29. भा॒ग॒धेये॒नेति॑ भाग - धेये॑न । \newline
30. उप॑ धावति धाव॒ त्युपोप॑ धावति । \newline
31. धा॒व॒ति॒ ते ते धा॑वति धावति॒ ते । \newline
32. त ए॒वैव ते त ए॒व । \newline
33. ए॒वास्मा॑ अस्मा ए॒वैवास्मै᳚ । \newline
34. अ॒स्मा॒ अन्न॒ मन्न॑ मस्मा अस्मा॒ अन्न᳚म् । \newline
35. अन्न॒म् प्र प्रान्न॒ मन्न॒म् प्र । \newline
36. प्र य॑च्छन्ति यच्छन्ति॒ प्र प्र य॑च्छन्ति । \newline
37. य॒च्छ॒न् त्य॒न्ना॒दो᳚ ऽन्ना॒दो य॑च्छन्ति यच्छ न्त्यन्ना॒दः । \newline
38. अ॒न्ना॒द ए॒वैवा न्ना॒दो᳚ ऽन्ना॒द ए॒व । \newline
39. अ॒न्ना॒द इत्य॑न्न - अ॒दः । \newline
40. ए॒व भ॑वति भव त्ये॒वैव भ॑वति । \newline
41. भ॒व॒ति॒ पृश्ञिः॒ पृश्ञि॑र् भवति भवति॒ पृश्ञिः॑ । \newline
42. पृश्ञि॑र् भवति भवति॒ पृश्ञिः॒ पृश्ञि॑र् भवति । \newline
43. भ॒व॒ त्ये॒त दे॒तद् भ॑वति भव त्ये॒तत् । \newline
44. ए॒तद् वै वा ए॒त दे॒तद् वै । \newline
45. वा अन्न॒स्या न्न॑स्य॒ वै वा अन्न॑स्य । \newline
46. अन्न॑स्य रू॒पꣳ रू॒प मन्न॒स्या न्न॑स्य रू॒पम् । \newline
47. रू॒पꣳ समृ॑द्ध्यै॒ समृ॑द्ध्यै रू॒पꣳ रू॒पꣳ समृ॑द्ध्यै । \newline
48. समृ॑द्ध्या ऐ॒न्द्र मै॒न्द्रꣳ समृ॑द्ध्यै॒ समृ॑द्ध्या ऐ॒न्द्रम् । \newline
49. समृ॑द्ध्या॒ इति॒ सं - ऋ॒द्ध्यै॒ । \newline
50. ऐ॒न्द्र म॑रु॒ण म॑रु॒ण मै॒न्द्र मै॒न्द्र म॑रु॒णम् । \newline
51. अ॒रु॒ण मा ऽरु॒ण म॑रु॒ण मा । \newline
52. आ ल॑भेत लभे॒ता ल॑भेत । \newline
53. ल॒भे॒ते॒ न्द्रि॒यका॑म इन्द्रि॒यका॑मो लभेत लभेते न्द्रि॒यका॑मः । \newline
54. इ॒न्द्रि॒यका॑म॒ इन्द्र॒ मिन्द्र॑ मिन्द्रि॒यका॑म इन्द्रि॒यका॑म॒ इन्द्र᳚म् । \newline
55. इ॒न्द्रि॒यका॑म॒इती᳚न्द्रि॒य - का॒मः॒ । \newline
56. इन्द्र॑ मे॒वैवे न्द्र॒ मिन्द्र॑ मे॒व । \newline
57. ए॒व स्वेन॒ स्वेनै॒वैव स्वेन॑ । \newline

\textbf{Ghana Paata } \newline

1. अन्न॒म् प्र प्रान्न॒ मन्न॒म् प्र य॑च्छति यच्छति॒ प्रान्न॒ मन्न॒म् प्र य॑च्छति । \newline
2. प्र य॑च्छति यच्छति॒ प्र प्र य॑च्छ त्यन्ना॒दो᳚ ऽन्ना॒दो य॑च्छति॒ प्र प्र य॑च्छ त्यन्ना॒दः । \newline
3. य॒च्छ॒ त्य॒न्ना॒दो᳚ ऽन्ना॒दो य॑च्छति यच्छ त्यन्ना॒द ए॒वैवा न्ना॒दो य॑च्छति यच्छ त्यन्ना॒द ए॒व । \newline
4. अ॒न्ना॒द ए॒वैवा न्ना॒दो᳚ ऽन्ना॒द ए॒व भ॑वति भवत्ये॒वा न्ना॒दो᳚ ऽन्ना॒द ए॒व भ॑वति । \newline
5. अ॒न्ना॒द इत्य॑न्न - अ॒दः । \newline
6. ए॒व भ॑वति भव त्ये॒वैव भ॑वति श्या॒मः श्या॒मो भ॑व त्ये॒वैव भ॑वति श्या॒मः । \newline
7. भ॒व॒ति॒ श्या॒मः श्या॒मो भ॑वति भवति श्या॒मो भ॑वति भवति श्या॒मो भ॑वति भवति श्या॒मो भ॑वति । \newline
8. श्या॒मो भ॑वति भवति श्या॒मः श्या॒मो भ॑व त्ये॒त दे॒तद् भ॑वति श्या॒मः श्या॒मो भ॑व त्ये॒तत् । \newline
9. भ॒व॒ त्ये॒त दे॒तद् भ॑वति भव त्ये॒तद् वै वा ए॒तद् भ॑वति भव त्ये॒तद् वै । \newline
10. ए॒तद् वै वा ए॒त दे॒तद् वा अन्न॒स्या न्न॑स्य॒ वा ए॒त दे॒तद् वा अन्न॑स्य । \newline
11. वा अन्न॒स्या न्न॑स्य॒ वै वा अन्न॑स्य रू॒पꣳ रू॒प मन्न॑स्य॒ वै वा अन्न॑स्य रू॒पम् । \newline
12. अन्न॑स्य रू॒पꣳ रू॒प मन्न॒स्या न्न॑स्य रू॒पꣳ समृ॑द्ध्यै॒ समृ॑द्ध्यै रू॒प मन्न॒स्या न्न॑स्य रू॒पꣳ समृ॑द्ध्यै । \newline
13. रू॒पꣳ समृ॑द्ध्यै॒ समृ॑द्ध्यै रू॒पꣳ रू॒पꣳ समृ॑द्ध्यै मारु॒तम् मा॑रु॒तꣳ समृ॑द्ध्यै रू॒पꣳ रू॒पꣳ समृ॑द्ध्यै मारु॒तम् । \newline
14. समृ॑द्ध्यै मारु॒तम् मा॑रु॒तꣳ समृ॑द्ध्यै॒ समृ॑द्ध्यै मारु॒तम् पृश्ञि॒म् पृश्ञि॑म् मारु॒तꣳ समृ॑द्ध्यै॒ समृ॑द्ध्यै मारु॒तम् पृश्ञि᳚म् । \newline
15. समृ॑द्ध्या॒ इति॒ सं - ऋ॒द्ध्यै॒ । \newline
16. मा॒रु॒तम् पृश्ञि॒म् पृश्ञि॑म् मारु॒तम् मा॑रु॒तम् पृश्ञि॒ मा पृश्ञि॑म् मारु॒तम् मा॑रु॒तम् पृश्ञि॒ मा । \newline
17. पृश्ञि॒ मा पृश्ञि॒म् पृश्ञि॒ मा ल॑भेत लभे॒ता पृश्ञि॒म् पृश्ञि॒ मा ल॑भेत । \newline
18. आ ल॑भेत लभे॒ता ल॑भे॒ता न्न॑का॒मो ऽन्न॑कामो लभे॒ता ल॑भे॒ता न्न॑कामः । \newline
19. ल॒भे॒ता न्न॑का॒मो ऽन्न॑कामो लभेत लभे॒ता न्न॑का॒मो ऽन्न॒ मन्न॒ मन्न॑कामो लभेत लभे॒ता न्न॑का॒मो ऽन्न᳚म् । \newline
20. अन्न॑का॒मो ऽन्न॒ मन्न॒ मन्न॑का॒मो ऽन्न॑का॒मो ऽन्नं॒ ॅवै वा अन्न॒ मन्न॑का॒मो ऽन्न॑का॒मो ऽन्नं॒ ॅवै । \newline
21. अन्न॑काम॒इत्यन्न॑ - का॒मः॒ । \newline
22. अन्नं॒ ॅवै वा अन्न॒ मन्नं॒ ॅवै म॒रुतो॑ म॒रुतो॒ वा अन्न॒ मन्नं॒ ॅवै म॒रुतः॑ । \newline
23. वै म॒रुतो॑ म॒रुतो॒ वै वै म॒रुतः॑ । \newline
24. म॒रुतो॑ म॒रुतः॑ । \newline
25. म॒रुत॑ ए॒वैव म॒रुतो॑ म॒रुत॑ ए॒व स्वेन॒ स्वेनै॒व म॒रुतो॑ म॒रुत॑ ए॒व स्वेन॑ । \newline
26. ए॒व स्वेन॒ स्वेनै॒वैव स्वेन॑ भाग॒धेये॑न भाग॒धेये॑न॒ स्वेनै॒वैव स्वेन॑ भाग॒धेये॑न । \newline
27. स्वेन॑ भाग॒धेये॑न भाग॒धेये॑न॒ स्वेन॒ स्वेन॑ भाग॒धेये॒नो पोप॑ भाग॒धेये॑न॒ स्वेन॒ स्वेन॑ भाग॒धेये॒नोप॑ । \newline
28. भा॒ग॒धेये॒नो पोप॑ भाग॒धेये॑न भाग॒धेये॒नोप॑ धावति धाव॒त्युप॑ भाग॒धेये॑न भाग॒धेये॒नोप॑ धावति । \newline
29. भा॒ग॒धेये॒नेति॑ भाग - धेये॑न । \newline
30. उप॑ धावति धाव॒ त्युपोप॑ धावति॒ ते ते धा॑व॒ त्युपोप॑ धावति॒ ते । \newline
31. धा॒व॒ति॒ ते ते धा॑वति धावति॒ त ए॒वैव ते धा॑वति धावति॒ त ए॒व । \newline
32. त ए॒वैव ते त ए॒वास्मा॑ अस्मा ए॒व ते त ए॒वास्मै᳚ । \newline
33. ए॒वास्मा॑ अस्मा ए॒वैवास्मा॒ अन्न॒ मन्न॑ मस्मा ए॒वैवास्मा॒ अन्न᳚म् । \newline
34. अ॒स्मा॒ अन्न॒ मन्न॑ मस्मा अस्मा॒ अन्न॒म् प्र प्रान्न॑ मस्मा अस्मा॒ अन्न॒म् प्र । \newline
35. अन्न॒म् प्र प्रान्न॒ मन्न॒म् प्र य॑च्छन्ति यच्छन्ति॒ प्रान्न॒ मन्न॒म् प्र य॑च्छन्ति । \newline
36. प्र य॑च्छन्ति यच्छन्ति॒ प्र प्र य॑च्छ न्त्यन्ना॒दो᳚ ऽन्ना॒दो य॑च्छन्ति॒ प्र प्र य॑च्छ न्त्यन्ना॒दः । \newline
37. य॒च्छ॒न् त्य॒न्ना॒दो᳚ ऽन्ना॒दो य॑च्छन्ति यच्छ न्त्यन्ना॒द ए॒वैवान्ना॒दो य॑च्छन्ति यच्छ न्त्यन्ना॒द ए॒व । \newline
38. अ॒न्ना॒द ए॒वैवा न्ना॒दो᳚ ऽन्ना॒द ए॒व भ॑वति भव त्ये॒वा न्ना॒दो᳚ ऽन्ना॒द ए॒व भ॑वति । \newline
39. अ॒न्ना॒द इत्य॑न्न - अ॒दः । \newline
40. ए॒व भ॑वति भव त्ये॒वैव भ॑वति॒ पृश्ञिः॒ पृश्ञि॑र् भव त्ये॒वैव भ॑वति॒ पृश्ञिः॑ । \newline
41. भ॒व॒ति॒ पृश्ञिः॒ पृश्ञि॑र् भवति भवति॒ पृश्ञि॑र् भवति भवति॒ पृश्ञि॑र् भवति भवति॒ पृश्ञि॑र् भवति । \newline
42. पृश्ञि॑र् भवति भवति॒ पृश्ञिः॒ पृश्ञि॑र् भव त्ये॒त दे॒तद् भ॑वति॒ पृश्ञिः॒ पृश्ञि॑र् भव त्ये॒तत् । \newline
43. भ॒व॒ त्ये॒त दे॒तद् भ॑वति भव त्ये॒तद् वै वा ए॒तद् भ॑वति भव त्ये॒तद् वै । \newline
44. ए॒तद् वै वा ए॒त दे॒तद् वा अन्न॒स्या न्न॑स्य॒ वा ए॒त दे॒तद् वा अन्न॑स्य । \newline
45. वा अन्न॒स्या न्न॑स्य॒ वै वा अन्न॑स्य रू॒पꣳ रू॒प मन्न॑स्य॒ वै वा अन्न॑स्य रू॒पम् । \newline
46. अन्न॑स्य रू॒पꣳ रू॒प मन्न॒स्या न्न॑स्य रू॒पꣳ समृ॑द्ध्यै॒ समृ॑द्ध्यै रू॒प मन्न॒स्या न्न॑स्य रू॒पꣳ समृ॑द्ध्यै । \newline
47. रू॒पꣳ समृ॑द्ध्यै॒ समृ॑द्ध्यै रू॒पꣳ रू॒पꣳ समृ॑द्ध्या ऐ॒न्द्र मै॒न्द्रꣳ समृ॑द्ध्यै रू॒पꣳ रू॒पꣳ समृ॑द्ध्या ऐ॒न्द्रम् । \newline
48. समृ॑द्ध्या ऐ॒न्द्र मै॒न्द्रꣳ समृ॑द्ध्यै॒ समृ॑द्ध्या ऐ॒न्द्र म॑रु॒ण म॑रु॒ण मै॒न्द्रꣳ समृ॑द्ध्यै॒ समृ॑द्ध्या ऐ॒न्द्र म॑रु॒णम् । \newline
49. समृ॑द्ध्या॒ इति॒ सं - ऋ॒द्ध्यै॒ । \newline
50. ऐ॒न्द्र म॑रु॒ण म॑रु॒ण मै॒न्द्र मै॒न्द्र म॑रु॒ण मा ऽरु॒ण मै॒न्द्र मै॒न्द्र म॑रु॒ण मा । \newline
51. अ॒रु॒ण मा ऽरु॒ण म॑रु॒ण मा ल॑भेत लभे॒ता ऽरु॒ण म॑रु॒ण मा ल॑भेत । \newline
52. आ ल॑भेत लभे॒ता ल॑भेते न्द्रि॒यका॑म इन्द्रि॒यका॑मो लभे॒ता ल॑भेते न्द्रि॒यका॑मः । \newline
53. ल॒भे॒ते॒ न्द्रि॒यका॑म इन्द्रि॒यका॑मो लभेत लभेते न्द्रि॒यका॑म॒ इन्द्र॒ मिन्द्र॑ मिन्द्रि॒यका॑मो लभेत लभेते न्द्रि॒यका॑म॒ इन्द्र᳚म् । \newline
54. इ॒न्द्रि॒यका॑म॒ इन्द्र॒ मिन्द्र॑ मिन्द्रि॒यका॑म इन्द्रि॒यका॑म॒ इन्द्र॑ मे॒वैवे न्द्र॑ मिन्द्रि॒यका॑म इन्द्रि॒यका॑म॒ इन्द्र॑ मे॒व । \newline
55. इ॒न्द्रि॒यका॑म॒ इती᳚न्द्रि॒य - का॒मः॒ । \newline
56. इन्द्र॑ मे॒वैवे न्द्र॒ मिन्द्र॑ मे॒व स्वेन॒ स्वेनै॒वे न्द्र॒ मिन्द्र॑ मे॒व स्वेन॑ । \newline
57. ए॒व स्वेन॒ स्वेनै॒वैव स्वेन॑ भाग॒धेये॑न भाग॒धेये॑न॒ स्वेनै॒वैव स्वेन॑ भाग॒धेये॑न । \newline
\pagebreak
\markright{ TS 2.1.6.3  \hfill https://www.vedavms.in \hfill}

\section{ TS 2.1.6.3 }

\textbf{TS 2.1.6.3 } \newline
\textbf{Samhita Paata} \newline

स्वेन॑ भाग॒धेये॒नोप॑ धावति॒ स ए॒वास्मि॑न्निन्द्रि॒यं द॑धातीन्द्रिया॒व्ये॑व भ॑वत्यरु॒णो भ्रूमा᳚न् भवत्ये॒तद्वा इन्द्र॑स्य रू॒पꣳ समृ॑द्ध्यै सावि॒त्रमु॑पद्ध्व॒स्तमा ल॑भेत स॒निका॑मः सवि॒ता वै प्र॑स॒वाना॑मीशे सवि॒तार॑मे॒व स्वेन॑ भाग॒धेये॒नोप॑ धावति॒ स ए॒वास्मै॑ स॒निं प्रसु॑वति॒ दान॑कामा अस्मै प्र॒जा भ॑वन्त्युपद्ध्व॒स्तो भ॑वति सावि॒त्रो ह्ये॑ष- [  ] \newline

\textbf{Pada Paata} \newline

स्वेन॑ । भा॒ग॒धेये॒नेति॑ भाग - धेये॑न । उपेति॑ । धा॒व॒ति॒ । सः । ए॒व । अ॒स्मि॒न्न् । इ॒न्द्रि॒यम् । द॒धा॒ति॒ । इ॒न्द्रि॒या॒वी । ए॒व । भ॒व॒ति॒ । अ॒रु॒णः । भ्रूमा॒निति॒ भ्रू - मा॒न् । भ॒व॒ति॒ । ऐ॒तत् । वै । इन्द्र॑स्य । रू॒पम् । समृ॑द्ध्या॒ इति॒ सं - ऋ॒द्ध्यै॒ । सा॒वि॒त्रम् । उ॒प॒द्ध्व॒स्तमित्यु॑प - ध्व॒स्तम् । एति॑ । ल॒भे॒त॒ । स॒निका॑म॒ इति॑ स॒नि - का॒मः॒ । स॒वि॒ता । वै । प्र॒स॒वाना॒मिति॑ प्र-स॒वाना᳚म् । ई॒शे॒ । स॒वि॒तार᳚म् । ए॒व । स्वेन॑ । भा॒ग॒धेये॒नेति॑ भाग - धेये॑न । उपेति॑ । धा॒व॒ति॒ । सः । ए॒व । अ॒स्मै॒ । स॒निम् । प्रेति॑ । सु॒व॒ति॒ । दान॑कामा॒ इति॒ दान॑-का॒माः॒ । अ॒स्मै॒ । प्र॒जा इति॑ प्र-जाः । भ॒व॒न्ति॒ । उ॒प॒द्ध्व॒स्त इत्यु॑प - ध्व॒स्तः । भ॒व॒ति॒ । सा॒वि॒त्रः । हि । ए॒षः ।  \newline


\textbf{Krama Paata} \newline

स्वेन॑ भाग॒धेये॑न । भा॒ग॒धेये॒नोप॑ । भा॒ग॒धेये॒नेति॑ भाग - धेये॑न । उप॑ धावति । धा॒व॒ति॒ सः । स ए॒व । ए॒वास्मिन्न्॑ । अ॒स्मि॒न्नि॒न्द्रि॒यम् । इ॒न्द्रि॒यम् द॑धाति । द॒धा॒ती॒न्द्रि॒या॒वी । इ॒न्द्रि॒या॒व्ये॑व । ए॒व भ॑वति । भ॒व॒त्य॒रु॒णः । अ॒रु॒णो भ्रूमान्॑ । भ्रूमा᳚न्,भवति । भ्रूमा॒निति॒ भ्रू - मा॒न्॒ । भ॒व॒त्ये॒तत् । ए॒तद् वै । वा इन्द्र॑स्य । इन्द्र॑स्य रू॒पम् । रू॒पꣳ समृ॑द्ध्यै । समृ॑द्ध्यै सावि॒त्रम् । समृ॑द्ध्या॒ इति॒ सं - ऋ॒द्ध्यै॒ । सा॒वि॒त्रमु॑पद्ध्व॒स्तम् । उ॒प॒द्ध्व॒स्तमा । उ॒प॒द्ध्व॒स्तमित्यु॑प - ध्व॒स्तम् । आ ल॑भेत । ल॒भे॒त॒ स॒निका॑मः । स॒निका॑मः सवि॒ता । स॒निका॑म॒ इति॑ स॒नि - का॒मः॒ । स॒वि॒ता वै । वै प्र॑स॒वाना᳚म् । प्र॒स॒वाना॑मीशे । प्र॒स॒वाना॒मिति॑ प्र - स॒वाना᳚म् । ई॒शे॒ स॒वि॒तार᳚म् । स॒वि॒तार॑मे॒व । ए॒व स्वेन॑ । स्वेन॑ भाग॒धेये॑न । भा॒ग॒धेये॒नोप॑ । भा॒ग॒धेये॒नेति॑ भाग - धेये॑न । उप॑ धावति । धा॒व॒ति॒ सः । स ए॒व । ए॒वास्मै᳚ । अ॒स्मै॒ स॒निम् । स॒निम् प्र । प्र सु॑वति । सु॒व॒ति॒ दान॑कामाः । दान॑कामा अस्मै । दान॑कामा॒ इति॒ दान॑ - का॒माः॒ । अ॒स्मै॒ प्र॒जाः । प्र॒जा भ॑वन्ति । प्र॒जा इति॑ प्र - जाः । भ॒व॒न्त्यु॒प॒द्ध्व॒स्तः । उ॒प॒द्ध्व॒स्तो भ॑वति । उ॒प॒द्ध्व॒स्त इत्यु॑प - ध्व॒स्तः । भ॒व॒ति॒ सा॒वि॒त्रः । सा॒वि॒त्रो हि । ह्ये॑षः । ए॒ष दे॒वत॑या \newline

\textbf{Jatai Paata} \newline

1. स्वेन॑ भाग॒धेये॑न भाग॒धेये॑न॒ स्वेन॒ स्वेन॑ भाग॒धेये॑न । \newline
2. भा॒ग॒धेये॒नोपोप॑ भाग॒धेये॑न भाग॒धेये॒नोप॑ । \newline
3. भा॒ग॒धेये॒नेति॑ भाग - धेये॑न । \newline
4. उप॑ धावति धाव॒ त्युपोप॑ धावति । \newline
5. धा॒व॒ति॒ स स धा॑वति धावति॒ सः । \newline
6. स ए॒वैव स स ए॒व । \newline
7. ए॒वास्मि॑न् नस्मिन् ने॒वैवास्मिन्न्॑ । \newline
8. अ॒स्मि॒न् नि॒न्द्रि॒य मि॑न्द्रि॒य म॑स्मिन् नस्मिन् निन्द्रि॒यम् । \newline
9. इ॒न्द्रि॒यम् द॑धाति दधातीन्द्रि॒य मि॑न्द्रि॒यम् द॑धाति । \newline
10. द॒धा॒ती॒ न्द्रि॒या॒वी न्द्रि॑या॒वी द॑धाति दधाती न्द्रिया॒वी । \newline
11. इ॒न्द्रि॒या॒ व्ये॑वैवे न्द्रि॑या॒वी न्द्रि॑या॒ व्ये॑व । \newline
12. ए॒व भ॑वति भव त्ये॒वैव भ॑वति । \newline
13. भ॒व॒ त्य॒रु॒णो॑ ऽरु॒णो भ॑वति भव त्यरु॒णः । \newline
14. अ॒रु॒णो भ्रूमा॒न् भ्रूमा॑ नरु॒णो॑ ऽरु॒णो भ्रूमान्॑ । \newline
15. भ्रूमा᳚न् भवति भवति॒ भ्रूमा॒न् भ्रूमा᳚न् भवति । \newline
16. भ्रूमा॒निति॒ भ्रु - मा॒न् । \newline
17. भ॒व॒ त्ये॒त दे॒तद् भ॑वति भव त्ये॒तत् । \newline
18. ए॒तद् वै वा ए॒त दे॒तद् वै । \newline
19. वा इन्द्र॒स्ये न्द्र॑स्य॒ वै वा इन्द्र॑स्य । \newline
20. इन्द्र॑स्य रू॒पꣳ रू॒प मिन्द्र॒स्ये न्द्र॑स्य रू॒पम् । \newline
21. रू॒पꣳ समृ॑द्ध्यै॒ समृ॑द्ध्यै रू॒पꣳ रू॒पꣳ समृ॑द्ध्यै । \newline
22. समृ॑द्ध्यै सावि॒त्रꣳ सा॑वि॒त्रꣳ समृ॑द्ध्यै॒ समृ॑द्ध्यै सावि॒त्रम् । \newline
23. समृ॑द्ध्या॒ इति॒ सं - ऋ॒द्ध्यै॒ । \newline
24. सा॒वि॒त्र मु॑पद्ध्व॒स्त मु॑पद्ध्व॒स्तꣳ सा॑वि॒त्रꣳ सा॑वि॒त्र मु॑पद्ध्व॒स्तम् । \newline
25. उ॒प॒द्ध्व॒स्त मोप॑द्ध्व॒स्त मु॑पद्ध्व॒स्त मा । \newline
26. उ॒प॒द्ध्व॒स्तमित्यु॑प - ध्व॒स्तम् । \newline
27. आ ल॑भेत लभे॒ता ल॑भेत । \newline
28. ल॒भे॒त॒ स॒निका॑मः स॒निका॑मो लभेत लभेत स॒निका॑मः । \newline
29. स॒निका॑मः सवि॒ता स॑वि॒ता स॒निका॑मः स॒निका॑मः सवि॒ता । \newline
30. स॒निका॑म॒ इति॑ स॒नि - का॒मः॒ । \newline
31. स॒वि॒ता वै वै स॑वि॒ता स॑वि॒ता वै । \newline
32. वै प्र॑स॒वाना᳚म् प्रस॒वानां॒ ॅवै वै प्र॑स॒वाना᳚म् । \newline
33. प्र॒स॒वाना॑ मीश ईशे प्रस॒वाना᳚म् प्रस॒वाना॑ मीशे । \newline
34. प्र॒स॒वाना॒मिति॑ प्र - स॒वाना᳚म् । \newline
35. ई॒शे॒ स॒वि॒तारꣳ॑ सवि॒तार॑ मीश ईशे सवि॒तार᳚म् । \newline
36. स॒वि॒तार॑ मे॒वैव स॑वि॒तारꣳ॑ सवि॒तार॑ मे॒व । \newline
37. ए॒व स्वेन॒ स्वेनै॒वैव स्वेन॑ । \newline
38. स्वेन॑ भाग॒धेये॑न भाग॒धेये॑न॒ स्वेन॒ स्वेन॑ भाग॒धेये॑न । \newline
39. भा॒ग॒धेये॒नोपोप॑ भाग॒धेये॑न भाग॒धेये॒नोप॑ । \newline
40. भा॒ग॒धेये॒नेति॑ भाग - धेये॑न । \newline
41. उप॑ धावति धाव॒ त्युपोप॑ धावति । \newline
42. धा॒व॒ति॒ स स धा॑वति धावति॒ सः । \newline
43. स ए॒वैव स स ए॒व । \newline
44. ए॒वास्मा॑ अस्मा ए॒वैवास्मै᳚ । \newline
45. अ॒स्मै॒ स॒निꣳ स॒नि म॑स्मा अस्मै स॒निम् । \newline
46. स॒निम् प्र प्र स॒निꣳ स॒निम् प्र । \newline
47. प्र सु॑वति सुवति॒ प्र प्र सु॑वति । \newline
48. सु॒व॒ति॒ दान॑कामा॒ दान॑कामाः सुवति सुवति॒ दान॑कामाः । \newline
49. दान॑कामा अस्मा अस्मै॒ दान॑कामा॒ दान॑कामा अस्मै । \newline
50. दान॑कामा॒ इति॒ दान॑ - का॒माः॒ । \newline
51. अ॒स्मै॒ प्र॒जाः प्र॒जा अ॑स्मा अस्मै प्र॒जाः । \newline
52. प्र॒जा भ॑वन्ति भवन्ति प्र॒जाः प्र॒जा भ॑वन्ति । \newline
53. प्र॒जा इति॑ प्र - जाः । \newline
54. भ॒व॒ न्त्यु॒प॒द्ध्व॒स्त उ॑पद्ध्व॒स्तो भ॑वन्ति भव न्त्युपद्ध्व॒स्तः । \newline
55. उ॒प॒द्ध्व॒स्तो भ॑वति भव त्युपद्ध्व॒स्त उ॑पद्ध्व॒स्तो भ॑वति । \newline
56. उ॒प॒द्ध्व॒स्त इत्यु॑प - ध्व॒स्तः । \newline
57. भ॒व॒ति॒ सा॒वि॒त्रः सा॑वि॒त्रो भ॑वति भवति सावि॒त्रः । \newline
58. सा॒वि॒त्रो हि हि सा॑वि॒त्रः सा॑वि॒त्रो हि । \newline
59. ह्ये॑ष ए॒ष हि ह्ये॑षः । \newline
60. ए॒ष दे॒वत॑या दे॒वत॑यै॒ष ए॒ष दे॒वत॑या । \newline

\textbf{Ghana Paata } \newline

1. स्वेन॑ भाग॒धेये॑न भाग॒धेये॑न॒ स्वेन॒ स्वेन॑ भाग॒धेये॒नो पोप॑ भाग॒धेये॑न॒ स्वेन॒ स्वेन॑ भाग॒धेये॒नोप॑ । \newline
2. भा॒ग॒धेये॒नो पोप॑ भाग॒धेये॑न भाग॒धेये॒नोप॑ धावति धाव॒त्युप॑ भाग॒धेये॑न भाग॒धेये॒नोप॑ धावति । \newline
3. भा॒ग॒धेये॒नेति॑ भाग - धेये॑न । \newline
4. उप॑ धावति धाव॒ त्युपोप॑ धावति॒ स स धा॑व॒ त्युपोप॑ धावति॒ सः । \newline
5. धा॒व॒ति॒ स स धा॑वति धावति॒ स ए॒वैव स धा॑वति धावति॒ स ए॒व । \newline
6. स ए॒वैव स स ए॒वास्मि॑न् नस्मिन् ने॒व स स ए॒वास्मिन्न्॑ । \newline
7. ए॒वास्मि॑न् नस्मिन् ने॒वैवास्मि॑न् निन्द्रि॒य मि॑न्द्रि॒य म॑स्मिन् ने॒वैवास्मि॑न् निन्द्रि॒यम् । \newline
8. अ॒स्मि॒न् नि॒न्द्रि॒य मि॑न्द्रि॒य म॑स्मिन् नस्मिन् निन्द्रि॒यम् द॑धाति दधातीन्द्रि॒य म॑स्मिन् नस्मिन् निन्द्रि॒यम् द॑धाति । \newline
9. इ॒न्द्रि॒यम् द॑धाति दधाती न्द्रि॒य मि॑न्द्रि॒यम् द॑धाती न्द्रिया॒वी न्द्रि॑या॒वी द॑धाती न्द्रि॒य मि॑न्द्रि॒यम् द॑धाती न्द्रिया॒वी । \newline
10. द॒धा॒ती॒ न्द्रि॒या॒वी न्द्रि॑या॒वी द॑धाति दधाती न्द्रिया॒ व्ये॑वैवे न्द्रि॑या॒वी द॑धाति दधाती न्द्रिया॒ व्ये॑व । \newline
11. इ॒न्द्रि॒या॒ व्ये॑वैवे न्द्रि॑या॒वी न्द्रि॑या॒ व्ये॑व भ॑वति भवत्ये॒वे न्द्रि॑या॒वी न्द्रि॑या॒ व्ये॑व भ॑वति । \newline
12. ए॒व भ॑वति भवत्ये॒वैव भ॑वत्यरु॒णो॑ ऽरु॒णो भ॑वत्ये॒वैव भ॑वत्यरु॒णः । \newline
13. भ॒व॒ त्य॒रु॒णो॑ ऽरु॒णो भ॑वति भव त्यरु॒णो भ्रूमा॒न् भ्रूमा॑ नरु॒णो भ॑वति भव त्यरु॒णो भ्रूमान्॑ । \newline
14. अ॒रु॒णो भ्रूमा॒न् भ्रूमा॑ नरु॒णो॑ ऽरु॒णो भ्रूमा᳚न् भवति भवति॒ भ्रूमा॑ नरु॒णो॑ ऽरु॒णो भ्रूमा᳚न् भवति । \newline
15. भ्रूमा᳚न् भवति भवति॒ भ्रूमा॒न् भ्रूमा᳚न् भव त्ये॒त दे॒तद् भ॑वति॒ भ्रूमा॒न् भ्रूमा᳚न् भव त्ये॒तत् । \newline
16. भ्रूमा॒निति॒ भ्रु - मा॒न् । \newline
17. भ॒व॒ त्ये॒त दे॒तद् भ॑वति भव त्ये॒तद् वै वा ए॒तद् भ॑वति भव त्ये॒तद् वै । \newline
18. ए॒तद् वै वा ए॒त दे॒तद् वा इन्द्र॒स्ये न्द्र॑स्य॒ वा ए॒त दे॒तद् वा इन्द्र॑स्य । \newline
19. वा इन्द्र॒स्ये न्द्र॑स्य॒ वै वा इन्द्र॑स्य रू॒पꣳ रू॒प मिन्द्र॑स्य॒ वै वा इन्द्र॑स्य रू॒पम् । \newline
20. इन्द्र॑स्य रू॒पꣳ रू॒प मिन्द्र॒स्ये न्द्र॑स्य रू॒पꣳ समृ॑द्ध्यै॒ समृ॑द्ध्यै रू॒प मिन्द्र॒स्ये न्द्र॑स्य रू॒पꣳ समृ॑द्ध्यै । \newline
21. रू॒पꣳ समृ॑द्ध्यै॒ समृ॑द्ध्यै रू॒पꣳ रू॒पꣳ समृ॑द्ध्यै सावि॒त्रꣳ सा॑वि॒त्रꣳ समृ॑द्ध्यै रू॒पꣳ रू॒पꣳ समृ॑द्ध्यै सावि॒त्रम् । \newline
22. समृ॑द्ध्यै सावि॒त्रꣳ सा॑वि॒त्रꣳ समृ॑द्ध्यै॒ समृ॑द्ध्यै सावि॒त्र मु॑पद्ध्व॒स्त मु॑पद्ध्व॒स्तꣳ सा॑वि॒त्रꣳ समृ॑द्ध्यै॒ समृ॑द्ध्यै सावि॒त्र मु॑पद्ध्व॒स्तम् । \newline
23. समृ॑द्ध्या॒ इति॒ सं - ऋ॒द्ध्यै॒ । \newline
24. सा॒वि॒त्र मु॑पद्ध्व॒स्त मु॑पद्ध्व॒स्तꣳ सा॑वि॒त्रꣳ सा॑वि॒त्र मु॑पद्ध्व॒स्त मोप॑द्ध्व॒स्तꣳ सा॑वि॒त्रꣳ सा॑वि॒त्र मु॑पद्ध्व॒स्त मा । \newline
25. उ॒प॒द्ध्व॒स्त मोप॑द्ध्व॒स्त मु॑पद्ध्व॒स्त मा ल॑भेत लभे॒तो प॑द्ध्व॒स्त मु॑पद्ध्व॒स्त मा ल॑भेत । \newline
26. उ॒प॒द्ध्व॒स्तमित्यु॑प - ध्व॒स्तम् । \newline
27. आ ल॑भेत लभे॒ता ल॑भेत स॒निका॑मः स॒निका॑मो लभे॒ता ल॑भेत स॒निका॑मः । \newline
28. ल॒भे॒त॒ स॒निका॑मः स॒निका॑मो लभेत लभेत स॒निका॑मः सवि॒ता स॑वि॒ता स॒निका॑मो लभेत लभेत स॒निका॑मः सवि॒ता । \newline
29. स॒निका॑मः सवि॒ता स॑वि॒ता स॒निका॑मः स॒निका॑मः सवि॒ता वै वै स॑वि॒ता स॒निका॑मः स॒निका॑मः सवि॒ता वै । \newline
30. स॒निका॑म॒ इति॑ स॒नि - का॒मः॒ । \newline
31. स॒वि॒ता वै वै स॑वि॒ता स॑वि॒ता वै प्र॑स॒वाना᳚म् प्रस॒वानां॒ ॅवै स॑वि॒ता स॑वि॒ता वै प्र॑स॒वाना᳚म् । \newline
32. वै प्र॑स॒वाना᳚म् प्रस॒वानां॒ ॅवै वै प्र॑स॒वाना॑ मीश ईशे प्रस॒वानां॒ ॅवै वै प्र॑स॒वाना॑ मीशे । \newline
33. प्र॒स॒वाना॑ मीश ईशे प्रस॒वाना᳚म् प्रस॒वाना॑ मीशे सवि॒तारꣳ॑ सवि॒तार॑ मीशे प्रस॒वाना᳚म् प्रस॒वाना॑ मीशे सवि॒तार᳚म् । \newline
34. प्र॒स॒वाना॒मिति॑ प्र - स॒वाना᳚म् । \newline
35. ई॒शे॒ स॒वि॒तारꣳ॑ सवि॒तार॑ मीश ईशे सवि॒तार॑ मे॒वैव स॑वि॒तार॑ मीश ईशे सवि॒तार॑ मे॒व । \newline
36. स॒वि॒तार॑ मे॒वैव स॑वि॒तारꣳ॑ सवि॒तार॑ मे॒व स्वेन॒ स्वेनै॒व स॑वि॒तारꣳ॑ सवि॒तार॑ मे॒व स्वेन॑ । \newline
37. ए॒व स्वेन॒ स्वेनै॒वैव स्वेन॑ भाग॒धेये॑न भाग॒धेये॑न॒ स्वेनै॒वैव स्वेन॑ भाग॒धेये॑न । \newline
38. स्वेन॑ भाग॒धेये॑न भाग॒धेये॑न॒ स्वेन॒ स्वेन॑ भाग॒धेये॒नो पोप॑ भाग॒धेये॑न॒ स्वेन॒ स्वेन॑ भाग॒धेये॒नोप॑ । \newline
39. भा॒ग॒धेये॒नो पोप॑ भाग॒धेये॑न भाग॒धेये॒नोप॑ धावति धाव॒त्युप॑ भाग॒धेये॑न भाग॒धेये॒नोप॑ धावति । \newline
40. भा॒ग॒धेये॒नेति॑ भाग - धेये॑न । \newline
41. उप॑ धावति धाव॒ त्युपोप॑ धावति॒ स स धा॑व॒ त्युपोप॑ धावति॒ सः । \newline
42. धा॒व॒ति॒ स स धा॑वति धावति॒ स ए॒वैव स धा॑वति धावति॒ स ए॒व । \newline
43. स ए॒वैव स स ए॒वास्मा॑ अस्मा ए॒व स स ए॒वास्मै᳚ । \newline
44. ए॒वास्मा॑ अस्मा ए॒वैवास्मै॑ स॒निꣳ स॒नि म॑स्मा ए॒वैवास्मै॑ स॒निम् । \newline
45. अ॒स्मै॒ स॒निꣳ स॒नि म॑स्मा अस्मै स॒निम् प्र प्र स॒नि म॑स्मा अस्मै स॒निम् प्र । \newline
46. स॒निम् प्र प्र स॒निꣳ स॒निम् प्र सु॑वति सुवति॒ प्र स॒निꣳ स॒निम् प्र सु॑वति । \newline
47. प्र सु॑वति सुवति॒ प्र प्र सु॑वति॒ दान॑कामा॒ दान॑कामाः सुवति॒ प्र प्र सु॑वति॒ दान॑कामाः । \newline
48. सु॒व॒ति॒ दान॑कामा॒ दान॑कामाः सुवति सुवति॒ दान॑कामा अस्मा अस्मै॒ दान॑कामाः सुवति सुवति॒ दान॑कामा अस्मै । \newline
49. दान॑कामा अस्मा अस्मै॒ दान॑कामा॒ दान॑कामा अस्मै प्र॒जाः प्र॒जा अ॑स्मै॒ दान॑कामा॒ दान॑कामा अस्मै प्र॒जाः । \newline
50. दान॑कामा॒ इति॒ दान॑ - का॒माः॒ । \newline
51. अ॒स्मै॒ प्र॒जाः प्र॒जा अ॑स्मा अस्मै प्र॒जा भ॑वन्ति भवन्ति प्र॒जा अ॑स्मा अस्मै प्र॒जा भ॑वन्ति । \newline
52. प्र॒जा भ॑वन्ति भवन्ति प्र॒जाः प्र॒जा भ॑व न्त्युपद्ध्व॒स्त उ॑पद्ध्व॒स्तो भ॑वन्ति प्र॒जाः प्र॒जा 
भ॑व न्त्युपद्ध्व॒स्तः । \newline
53. प्र॒जा इति॑ प्र - जाः । \newline
54. भ॒व॒ न्त्यु॒प॒द्ध्व॒स्त उ॑पद्ध्व॒स्तो भ॑वन्ति भवन् त्युपद्ध्व॒स्तो भ॑वति भव त्युपद्ध्व॒स्तो भ॑वन्ति भवन् त्युपद्ध्व॒स्तो भ॑वति । \newline
55. उ॒प॒द्ध्व॒स्तो भ॑वति भव त्युपद्ध्व॒स्त उ॑पद्ध्व॒स्तो भ॑वति सावि॒त्रः सा॑वि॒त्रो भ॑व त्युपद्ध्व॒स्त उ॑पद्ध्व॒स्तो भ॑वति सावि॒त्रः । \newline
56. उ॒प॒द्ध्व॒स्त इत्यु॑प - ध्व॒स्तः । \newline
57. भ॒व॒ति॒ सा॒वि॒त्रः सा॑वि॒त्रो भ॑वति भवति सावि॒त्रो हि हि सा॑वि॒त्रो भ॑वति भवति सावि॒त्रो हि । \newline
58. सा॒वि॒त्रो हि हि सा॑वि॒त्रः सा॑वि॒त्रो ह्ये॑ष ए॒ष हि सा॑वि॒त्रः सा॑वि॒त्रो ह्ये॑षः । \newline
59. ह्ये॑ष ए॒ष हि ह्ये॑ष दे॒वत॑या दे॒वत॑यै॒ष हि ह्ये॑ष दे॒वत॑या । \newline
60. ए॒ष दे॒वत॑या दे॒वत॑यै॒ष ए॒ष दे॒वत॑या॒ समृ॑द्ध्यै॒ समृ॑द्ध्यै दे॒वत॑यै॒ष ए॒ष दे॒वत॑या॒ समृ॑द्ध्यै । \newline
\pagebreak
\markright{ TS 2.1.6.4  \hfill https://www.vedavms.in \hfill}

\section{ TS 2.1.6.4 }

\textbf{TS 2.1.6.4 } \newline
\textbf{Samhita Paata} \newline

दे॒वत॑या॒ समृ॑द्ध्यै वैश्वदे॒वं ब॑हुरू॒पमा ल॑भे॒ताऽन्न॑कामोवैश्वदे॒वं ॅवा अन्नं॒ ॅविश्वा॑ने॒व दे॒वान्थ्-स्वेन॑ भाग॒धेये॒नोप॑ धावति॒त ए॒वास्मा॒ अन्नं॒ प्रय॑च्छन्त्यन्ना॒द ए॒व भ॑वति बहुरू॒पो भ॑वतिबहुरू॒पꣳ ह्यन्नꣳ॒॒ समृ॑द्ध्यै वैश्वदे॒वं ब॑हुरू॒पमा ल॑भेत॒ ग्राम॑कामो वैश्वदे॒वा वै स॑जा॒ता विश्वा॑ने॒व दे॒वान्थ्-स्वेन॑ भाग॒धेये॒नोप॑ धावति॒ त ए॒वास्मै॑ - [  ] \newline

\textbf{Pada Paata} \newline

दे॒वत॑या । समृ॑द्ध्या॒ इति॒ सं - ऋ॒द्ध्यै॒ । वै॒श्व॒दे॒वमिति॑ वैश्व-दे॒वम् । ब॒हु॒रू॒पमिति॑ बहु - रू॒पम् । एति॑ । ल॒भे॒त॒ । अन्न॑काम॒ इत्यन्न॑ - का॒मः॒ । वै॒श्व॒दे॒वमिति॑ वैश्व - दे॒वम् । वै । अन्न᳚म् । विश्वान्॑ । ए॒व । दे॒वान् । स्वेन॑ । भा॒ग॒धेये॒नेति॑ भाग-धेये॑न । उपेति॑ । धा॒व॒ति॒ । ते । ए॒व । अ॒स्मै॒ । अन्न᳚म् । प्रेति॑ । य॒च्छ॒न्ति॒ । अ॒न्ना॒द इत्य॑न्न - अ॒दः । ए॒व । भ॒व॒ति॒ । ब॒हु॒रू॒प इति॑ बहु-रू॒पः । भ॒व॒ति॒ । ब॒हु॒रू॒पमिति॑ बहु - रू॒पम् । हि । अन्न᳚म् । समृ॑द्ध्या॒ इति॒ सं - ऋ॒द्ध्यै॒ । वै॒श्व॒दे॒वमिति॑ वैश्व - दे॒वम् । ब॒हु॒रू॒पमिति॑ बहु - रू॒पम् । एति॑ । ल॒भे॒त॒ । ग्राम॑काम॒ इति॒ ग्राम॑ - का॒मः॒ । वै॒श्व॒दे॒वा इति॑ वैश्व - दे॒वाः । वै । स॒जा॒ता इति॑ स - जा॒ताः । विश्वान्॑ । ए॒व । दे॒वान् । स्वेन॑ । भा॒ग॒धेये॒नेति॑ भाग-धेये॑न । उपेति॑ । धा॒व॒ति॒ । ते । ए॒व । अ॒स्मै॒ ।  \newline


\textbf{Krama Paata} \newline

दे॒वत॑या॒ समृ॑द्ध्यै । समृ॑द्ध्यै वैश्वदे॒वम् । समृ॑द्ध्या॒ इति॒ सं - ऋ॒द्ध्यै॒ । वै॒श्व॒दे॒वम् ब॑हुरू॒पम् । वै॒श्व॒दे॒वमिति॑ वैश्व - दे॒वम् । ब॒हु॒रू॒पमा । ब॒हु॒रू॒पमिति॑ बहु - रू॒पम् । आ ल॑भेत । ल॒भे॒तान्न॑कामः । अन्न॑कामो वैश्वदे॒वम् । अन्न॑काम॒ इत्यन्न॑ - का॒मः॒ । वै॒श्व॒दे॒वं ॅवै । वै॒श्व॒दे॒वमिति॑ वैश्व - दे॒वम् । वा अन्न᳚म् । अन्नं॒ ॅविश्वान्॑ । विश्वा॑ने॒व । ए॒व दे॒वान् । दे॒वान्थ् स्वेन॑ । स्वेन॑ भाग॒धेये॑न । भा॒ग॒धेये॒नोप॑ । भा॒ग॒धेये॒नेति॑ भाग - धेये॑न । उप॑ धावति । धा॒व॒ति॒ ते । त ए॒व । ए॒वास्मै᳚ । अ॒स्मा॒ अन्न᳚म् । अन्न॒म् प्र । प्र य॑च्छन्ति । य॒च्छ॒न्त्य॒न्ना॒दः । अ॒न्ना॒द ए॒व । अ॒न्ना॒द इत्य॑न्न - अ॒दः । ए॒व भ॑वति । भ॒व॒ति॒ ब॒हु॒रू॒पः । ब॒हु॒रू॒पो भ॑वति । ब॒हु॒रू॒प इति॑ बहु - रू॒पः । भ॒व॒ति॒ ब॒हु॒रू॒पम् । ब॒हु॒रू॒पꣳ हि । ब॒हु॒रू॒पमिति॑ बहु - रू॒पम् । ह्यन्न᳚म् । अन्नꣳ॒॒ समृ॑द्ध्यै । समृ॑द्ध्यै वैश्वदे॒वम् । समृ॑द्ध्या॒ इति॒ सं - ऋ॒द्ध्यै॒ । वै॒श्व॒दे॒वम् ब॑हुरू॒पम् । वै॒श्व॒दे॒वमिति॑ वैश्व - दे॒वम् । ब॒हु॒रू॒पमा । ब॒हु॒रू॒पमिति॑ बहु - रू॒पम् । आ ल॑भेत । ल॒भे॒त॒ ग्राम॑कामः । ग्राम॑कामो वैश्वदे॒वाः । ग्रामका॑म॒ इति॒ ग्राम॑ - का॒मः॒ । वै॒श्व॒दे॒वा वै । वै॒श्व॒दे॒वा इति॑ वैश्व - दे॒वाः । वै स॑जा॒ताः । स॒जा॒ता विश्वान्॑ । स॒जा॒ता इति॑ स - जा॒ताः । विश्वा॑ने॒व । ए॒व दे॒वान् । दे॒वान्थ् स्वेन॑ । स्वेन॑ भाग॒धेये॑न । भा॒ग॒धेये॒नोप॑ । भा॒ग॒धेये॒नेति॑ भाग - धेये॑न । उप॑ धावति । धा॒व॒ति॒ ते । त ए॒व । ए॒वास्मै᳚ । अ॒स्मै॒ स॒जा॒तान् \newline

\textbf{Jatai Paata} \newline

1. दे॒वत॑या॒ समृ॑द्ध्यै॒ समृ॑द्ध्यै दे॒वत॑या दे॒वत॑या॒ समृ॑द्ध्यै । \newline
2. समृ॑द्ध्यै वैश्वदे॒वं ॅवै᳚श्वदे॒वꣳ समृ॑द्ध्यै॒ समृ॑द्ध्यै वैश्वदे॒वम् । \newline
3. समृ॑द्ध्या॒ इति॒ सं - ऋ॒द्ध्यै॒ । \newline
4. वै॒श्व॒दे॒वम् ब॑हुरू॒पम् ब॑हुरू॒पं ॅवै᳚श्वदे॒वं ॅवै᳚श्वदे॒वम् ब॑हुरू॒पम् । \newline
5. वै॒श्व॒दे॒वमिति॑ वैश्व - दे॒वम् । \newline
6. ब॒हु॒रू॒प मा ब॑हुरू॒पम् ब॑हुरू॒प मा । \newline
7. ब॒हु॒रू॒पमिति॑ बहु - रू॒पम् । \newline
8. आ ल॑भेत लभे॒ता ल॑भेत । \newline
9. ल॒भे॒तान्न॑का॒मो ऽन्न॑कामो लभेत लभे॒तान्न॑कामः । \newline
10. अन्न॑कामो वैश्वदे॒वं ॅवै᳚श्वदे॒व मन्न॑का॒मो ऽन्न॑कामो वैश्वदे॒वम् । \newline
11. अन्न॑काम॒इत्यन्न॑ - का॒मः॒ । \newline
12. वै॒श्व॒दे॒वं ॅवै वै वै᳚श्वदे॒वं ॅवै᳚श्वदे॒वं ॅवै । \newline
13. वै॒श्व॒दे॒वमिति॑ वैश्व - दे॒वम् । \newline
14. वा अन्न॒ मन्नं॒ ॅवै वा अन्न᳚म् । \newline
15. अन्नं॒ ॅविश्वा॒न्॒. विश्वा॒ नन्न॒ मन्नं॒ ॅविश्वान्॑ । \newline
16. विश्वा॑ ने॒वैव विश्वा॒न्॒. विश्वा॑ ने॒व । \newline
17. ए॒व दे॒वान् दे॒वा ने॒वैव दे॒वान् । \newline
18. दे॒वान् थ्स्वेन॒ स्वेन॑ दे॒वान् दे॒वान् थ्स्वेन॑ । \newline
19. स्वेन॑ भाग॒धेये॑न भाग॒धेये॑न॒ स्वेन॒ स्वेन॑ भाग॒धेये॑न । \newline
20. भा॒ग॒धेये॒नोपोप॑ भाग॒धेये॑न भाग॒धेये॒नोप॑ । \newline
21. भा॒ग॒धेये॒नेति॑ भाग - धेये॑न । \newline
22. उप॑ धावति धाव॒ त्युपोप॑ धावति । \newline
23. धा॒व॒ति॒ ते ते धा॑वति धावति॒ ते । \newline
24. त ए॒वैव ते त ए॒व । \newline
25. ए॒वास्मा॑ अस्मा ए॒वैवास्मै᳚ । \newline
26. अ॒स्मा॒ अन्न॒ मन्न॑ मस्मा अस्मा॒ अन्न᳚म् । \newline
27. अन्न॒म् प्र प्रान्न॒ मन्न॒म् प्र । \newline
28. प्र य॑च्छन्ति यच्छन्ति॒ प्र प्र य॑च्छन्ति । \newline
29. य॒च्छ॒ न्त्य॒न्ना॒दो᳚ ऽन्ना॒दो य॑च्छन्ति यच्छ न्त्यन्ना॒दः । \newline
30. अ॒न्ना॒द ए॒वैवा न्ना॒दो᳚ ऽन्ना॒द ए॒व । \newline
31. अ॒न्ना॒द इत्य॑न्न - अ॒दः । \newline
32. ए॒व भ॑वति भव त्ये॒वैव भ॑वति । \newline
33. भ॒व॒ति॒ ब॒हु॒रू॒पो ब॑हुरू॒पो भ॑वति भवति बहुरू॒पः । \newline
34. ब॒हु॒रू॒पो भ॑वति भवति बहुरू॒पो ब॑हुरू॒पो भ॑वति । \newline
35. ब॒हु॒रू॒प इति॑ बहु - रू॒पः । \newline
36. भ॒व॒ति॒ ब॒हु॒रू॒पम् ब॑हुरू॒पम् भ॑वति भवति बहुरू॒पम् । \newline
37. ब॒हु॒रू॒पꣳ हि हि ब॑हुरू॒पम् ब॑हुरू॒पꣳ हि । \newline
38. ब॒हु॒रू॒पमिति॑ बहु - रू॒पम् । \newline
39. ह्यन्न॒ मन्नꣳ॒॒ हि ह्यन्न᳚म् । \newline
40. अन्नꣳ॒॒ समृ॑द्ध्यै॒ समृ॑द्ध्या॒ अन्न॒ मन्नꣳ॒॒ समृ॑द्ध्यै । \newline
41. समृ॑द्ध्यै वैश्वदे॒वं ॅवै᳚श्वदे॒वꣳ समृ॑द्ध्यै॒ समृ॑द्ध्यै वैश्वदे॒वम् । \newline
42. समृ॑द्ध्या॒ इति॒ सं - ऋ॒द्ध्यै॒ । \newline
43. वै॒श्व॒दे॒वम् ब॑हुरू॒पम् ब॑हुरू॒पं ॅवै᳚श्वदे॒वं ॅवै᳚श्वदे॒वम् ब॑हुरू॒पम् । \newline
44. वै॒श्व॒दे॒वमिति॑ वैश्व - दे॒वम् । \newline
45. ब॒हु॒रू॒प मा ब॑हुरू॒पम् ब॑हुरू॒प मा । \newline
46. ब॒हु॒रू॒पमिति॑ बहु - रू॒पम् । \newline
47. आ ल॑भेत लभे॒ता ल॑भेत । \newline
48. ल॒भे॒त॒ ग्राम॑कामो॒ ग्राम॑कामो लभेत लभेत॒ ग्राम॑कामः । \newline
49. ग्राम॑कामो वैश्वदे॒वा वै᳚श्वदे॒वा ग्राम॑कामो॒ ग्राम॑कामो वैश्वदे॒वाः । \newline
50. ग्राम॑काम॒ इति॒ ग्राम॑ - का॒मः॒ । \newline
51. वै॒श्व॒दे॒वा वै वै वै᳚श्वदे॒वा वै᳚श्वदे॒वा वै । \newline
52. वै॒श्व॒दे॒वा इति॑ वैश्व - दे॒वाः । \newline
53. वै स॑जा॒ताः स॑जा॒ता वै वै स॑जा॒ताः । \newline
54. स॒जा॒ता विश्वा॒न्॒. विश्वा᳚न् थ्सजा॒ताः स॑जा॒ता विश्वान्॑ । \newline
55. स॒जा॒ता इति॑ स - जा॒ताः । \newline
56. विश्वा॑ ने॒वैव विश्वा॒न्॒. विश्वा॑ ने॒व । \newline
57. ए॒व दे॒वान् दे॒वा ने॒वैव दे॒वान् । \newline
58. दे॒वान् थ्स्वेन॒ स्वेन॑ दे॒वान् दे॒वान् थ्स्वेन॑ । \newline
59. स्वेन॑ भाग॒धेये॑न भाग॒धेये॑न॒ स्वेन॒ स्वेन॑ भाग॒धेये॑न । \newline
60. भा॒ग॒धेये॒नोपोप॑ भाग॒धेये॑न भाग॒धेये॒नोप॑ । \newline
61. भा॒ग॒धेये॒नेति॑ भाग - धेये॑न । \newline
62. उप॑ धावति धाव॒ त्युपोप॑ धावति । \newline
63. धा॒व॒ति॒ ते ते धा॑वति धावति॒ ते । \newline
64. त ए॒वैव ते त ए॒व । \newline
65. ए॒वास्मा॑ अस्मा ए॒वैवास्मै᳚ । \newline
66. अ॒स्मै॒ स॒जा॒तान् थ्स॑जा॒ता न॑स्मा अस्मै सजा॒तान् । \newline

\textbf{Ghana Paata } \newline

1. दे॒वत॑या॒ समृ॑द्ध्यै॒ समृ॑द्ध्यै दे॒वत॑या दे॒वत॑या॒ समृ॑द्ध्यै वैश्वदे॒वं ॅवै᳚श्वदे॒वꣳ समृ॑द्ध्यै दे॒वत॑या दे॒वत॑या॒ समृ॑द्ध्यै वैश्वदे॒वम् । \newline
2. समृ॑द्ध्यै वैश्वदे॒वं ॅवै᳚श्वदे॒वꣳ समृ॑द्ध्यै॒ समृ॑द्ध्यै वैश्वदे॒वम् ब॑हुरू॒पम् ब॑हुरू॒पं ॅवै᳚श्वदे॒वꣳ समृ॑द्ध्यै॒ समृ॑द्ध्यै वैश्वदे॒वम् ब॑हुरू॒पम् । \newline
3. समृ॑द्ध्या॒ इति॒ सं - ऋ॒द्ध्यै॒ । \newline
4. वै॒श्व॒दे॒वम् ब॑हुरू॒पम् ब॑हुरू॒पं ॅवै᳚श्वदे॒वं ॅवै᳚श्वदे॒वम् ब॑हुरू॒प मा ब॑हुरू॒पं ॅवै᳚श्वदे॒वं ॅवै᳚श्वदे॒वम् ब॑हुरू॒प मा । \newline
5. वै॒श्व॒दे॒वमिति॑ वैश्व - दे॒वम् । \newline
6. ब॒हु॒रू॒प मा ब॑हुरू॒पम् ब॑हुरू॒प मा ल॑भेत लभे॒ता ब॑हुरू॒पम् ब॑हुरू॒प मा ल॑भेत । \newline
7. ब॒हु॒रू॒पमिति॑ बहु - रू॒पम् । \newline
8. आ ल॑भेत लभे॒ता ल॑भे॒ता न्न॑का॒मो ऽन्न॑कामो लभे॒ता ल॑भे॒ता न्न॑कामः । \newline
9. ल॒भे॒ता न्न॑का॒मो ऽन्न॑कामो लभेत लभे॒ता न्न॑कामो वैश्वदे॒वं ॅवै᳚श्वदे॒व मन्न॑कामो लभेत लभे॒तान्न॑कामो वैश्वदे॒वम् । \newline
10. अन्न॑कामो वैश्वदे॒वं ॅवै᳚श्वदे॒व मन्न॑का॒मो ऽन्न॑कामो वैश्वदे॒वं ॅवै वै वै᳚श्वदे॒व मन्न॑का॒मो ऽन्न॑कामो वैश्वदे॒वं ॅवै । \newline
11. अन्न॑काम॒इत्यन्न॑ - का॒मः॒ । \newline
12. वै॒श्व॒दे॒वं ॅवै वै वै᳚श्वदे॒वं ॅवै᳚श्वदे॒वं ॅवा अन्न॒ मन्नं॒ ॅवै वै᳚श्वदे॒वं ॅवै᳚श्वदे॒वं ॅवा अन्न᳚म् । \newline
13. वै॒श्व॒दे॒वमिति॑ वैश्व - दे॒वम् । \newline
14. वा अन्न॒ मन्नं॒ ॅवै वा अन्नं॒ ॅविश्वा॒न्॒. विश्वा॒ नन्नं॒ ॅवै वा अन्नं॒ ॅविश्वान्॑ । \newline
15. अन्नं॒ ॅविश्वा॒न्॒. विश्वा॒ नन्न॒ मन्नं॒ ॅविश्वा॑ ने॒वैव विश्वा॒ नन्न॒ मन्नं॒ ॅविश्वा॑ ने॒व । \newline
16. विश्वा॑ ने॒वैव विश्वा॒न्॒. विश्वा॑ ने॒व दे॒वान् दे॒वा ने॒व विश्वा॒न्॒. विश्वा॑ ने॒व दे॒वान् । \newline
17. ए॒व दे॒वान् दे॒वा ने॒वैव दे॒वान् थ्स्वेन॒ स्वेन॑ दे॒वा ने॒वैव दे॒वान् थ्स्वेन॑ । \newline
18. दे॒वान् थ्स्वेन॒ स्वेन॑ दे॒वान् दे॒वान् थ्स्वेन॑ भाग॒धेये॑न भाग॒धेये॑न॒ स्वेन॑ दे॒वान् दे॒वान् थ्स्वेन॑ भाग॒धेये॑न । \newline
19. स्वेन॑ भाग॒धेये॑न भाग॒धेये॑न॒ स्वेन॒ स्वेन॑ भाग॒धेये॒नो पोप॑ भाग॒धेये॑न॒ स्वेन॒ स्वेन॑ भाग॒धेये॒नोप॑ । \newline
20. भा॒ग॒धेये॒नो पोप॑ भाग॒धेये॑न भाग॒धेये॒नोप॑ धावति धाव॒त्युप॑ भाग॒धेये॑न भाग॒धेये॒नोप॑ धावति । \newline
21. भा॒ग॒धेये॒नेति॑ भाग - धेये॑न । \newline
22. उप॑ धावति धाव॒ त्युपोप॑ धावति॒ ते ते धा॑व॒ त्युपोप॑ धावति॒ ते । \newline
23. धा॒व॒ति॒ ते ते धा॑वति धावति॒ त ए॒वैव ते धा॑वति धावति॒ त ए॒व । \newline
24. त ए॒वैव ते त ए॒वास्मा॑ अस्मा ए॒व ते त ए॒वास्मै᳚ । \newline
25. ए॒वास्मा॑ अस्मा ए॒वैवास्मा॒ अन्न॒ मन्न॑ मस्मा ए॒वैवास्मा॒ अन्न᳚म् । \newline
26. अ॒स्मा॒ अन्न॒ मन्न॑ मस्मा अस्मा॒ अन्न॒म् प्र प्रान्न॑ मस्मा अस्मा॒ अन्न॒म् प्र । \newline
27. अन्न॒म् प्र प्रान्न॒ मन्न॒म् प्र य॑च्छन्ति यच्छन्ति॒ प्रान्न॒ मन्न॒म् प्र य॑च्छन्ति । \newline
28. प्र य॑च्छन्ति यच्छन्ति॒ प्र प्र य॑च्छ न्त्यन्ना॒दो᳚ ऽन्ना॒दो य॑च्छन्ति॒ प्र प्र य॑च्छ न्त्यन्ना॒दः । \newline
29. य॒च्छ॒ न्त्य॒न्ना॒दो᳚ ऽन्ना॒दो य॑च्छन्ति यच्छ न्त्यन्ना॒द ए॒वैवान्ना॒दो य॑च्छन्ति यच्छ न्त्यन्ना॒द ए॒व । \newline
30. अ॒न्ना॒द ए॒वैवा न्ना॒दो᳚ ऽन्ना॒द ए॒व भ॑वति भवत्ये॒वा न्ना॒दो᳚ ऽन्ना॒द ए॒व भ॑वति । \newline
31. अ॒न्ना॒द इत्य॑न्न - अ॒दः । \newline
32. ए॒व भ॑वति भवत्ये॒वैव भ॑वति बहुरू॒पो ब॑हुरू॒पो भ॑वत्ये॒वैव भ॑वति बहुरू॒पः । \newline
33. भ॒व॒ति॒ ब॒हु॒रू॒पो ब॑हुरू॒पो भ॑वति भवति बहुरू॒पो भ॑वति भवति बहुरू॒पो भ॑वति भवति बहुरू॒पो भ॑वति । \newline
34. ब॒हु॒रू॒पो भ॑वति भवति बहुरू॒पो ब॑हुरू॒पो भ॑वति बहुरू॒पम् ब॑हुरू॒पम् भ॑वति बहुरू॒पो ब॑हुरू॒पो भ॑वति बहुरू॒पम् । \newline
35. ब॒हु॒रू॒प इति॑ बहु - रू॒पः । \newline
36. भ॒व॒ति॒ ब॒हु॒रू॒पम् ब॑हुरू॒पम् भ॑वति भवति बहुरू॒पꣳ हि हि ब॑हुरू॒पम् भ॑वति भवति बहुरू॒पꣳ हि । \newline
37. ब॒हु॒रू॒पꣳ हि हि ब॑हुरू॒पम् ब॑हुरू॒पꣳ ह्यन्न॒ मन्नꣳ॒॒ हि ब॑हुरू॒पम् ब॑हुरू॒पꣳ ह्यन्न᳚म् । \newline
38. ब॒हु॒रू॒पमिति॑ बहु - रू॒पम् । \newline
39. ह्यन्न॒ मन्नꣳ॒॒ हि ह्यन्नꣳ॒॒ समृ॑द्ध्यै॒ समृ॑द्ध्या॒ अन्नꣳ॒॒ हि ह्यन्नꣳ॒॒ समृ॑द्ध्यै । \newline
40. अन्नꣳ॒॒ समृ॑द्ध्यै॒ समृ॑द्ध्या॒ अन्न॒ मन्नꣳ॒॒ समृ॑द्ध्यै वैश्वदे॒वं ॅवै᳚श्वदे॒वꣳ समृ॑द्ध्या॒ अन्न॒ मन्नꣳ॒॒ समृ॑द्ध्यै वैश्वदे॒वम् । \newline
41. समृ॑द्ध्यै वैश्वदे॒वं ॅवै᳚श्वदे॒वꣳ समृ॑द्ध्यै॒ समृ॑द्ध्यै वैश्वदे॒वम् ब॑हुरू॒पम् ब॑हुरू॒पं ॅवै᳚श्वदे॒वꣳ समृ॑द्ध्यै॒ समृ॑द्ध्यै वैश्वदे॒वम् ब॑हुरू॒पम् । \newline
42. समृ॑द्ध्या॒ इति॒ सं - ऋ॒द्ध्यै॒ । \newline
43. वै॒श्व॒दे॒वम् ब॑हुरू॒पम् ब॑हुरू॒पं ॅवै᳚श्वदे॒वं ॅवै᳚श्वदे॒वम् ब॑हुरू॒प मा ब॑हुरू॒पं ॅवै᳚श्वदे॒वं ॅवै᳚श्वदे॒वम् ब॑हुरू॒प मा । \newline
44. वै॒श्व॒दे॒वमिति॑ वैश्व - दे॒वम् । \newline
45. ब॒हु॒रू॒प मा ब॑हुरू॒पम् ब॑हुरू॒प मा ल॑भेत लभे॒ता ब॑हुरू॒पम् ब॑हुरू॒प मा ल॑भेत । \newline
46. ब॒हु॒रू॒पमिति॑ बहु - रू॒पम् । \newline
47. आ ल॑भेत लभे॒ता ल॑भेत॒ ग्राम॑कामो॒ ग्राम॑कामो लभे॒ता ल॑भेत॒ ग्राम॑कामः । \newline
48. ल॒भे॒त॒ ग्राम॑कामो॒ ग्राम॑कामो लभेत लभेत॒ ग्राम॑कामो वैश्वदे॒वा वै᳚श्वदे॒वा ग्राम॑कामो लभेत लभेत॒ ग्राम॑कामो वैश्वदे॒वाः । \newline
49. ग्राम॑कामो वैश्वदे॒वा वै᳚श्वदे॒वा ग्राम॑कामो॒ ग्राम॑कामो वैश्वदे॒वा वै वै वै᳚श्वदे॒वा ग्राम॑कामो॒ ग्राम॑कामो वैश्वदे॒वा वै । \newline
50. ग्राम॑काम॒ इति॒ ग्राम॑ - का॒मः॒ । \newline
51. वै॒श्व॒दे॒वा वै वै वै᳚श्वदे॒वा वै᳚श्वदे॒वा वै स॑जा॒ताः स॑जा॒ता वै वै᳚श्वदे॒वा वै᳚श्वदे॒वा वै स॑जा॒ताः । \newline
52. वै॒श्व॒दे॒वा इति॑ वैश्व - दे॒वाः । \newline
53. वै स॑जा॒ताः स॑जा॒ता वै वै स॑जा॒ता विश्वा॒न्॒. विश्वा᳚न् थ्सजा॒ता वै वै स॑जा॒ता विश्वान्॑ । \newline
54. स॒जा॒ता विश्वा॒न्॒. विश्वा᳚न् थ्सजा॒ताः स॑जा॒ता विश्वा॑ ने॒वैव विश्वा᳚न् थ्सजा॒ताः स॑जा॒ता विश्वा॑ ने॒व । \newline
55. स॒जा॒ता इति॑ स - जा॒ताः । \newline
56. विश्वा॑ ने॒वैव विश्वा॒न्॒. विश्वा॑ ने॒व दे॒वान् दे॒वा ने॒व विश्वा॒न्॒. विश्वा॑ ने॒व दे॒वान् । \newline
57. ए॒व दे॒वान् दे॒वा ने॒वैव दे॒वान् थ्स्वेन॒ स्वेन॑ दे॒वा ने॒वैव दे॒वान् थ्स्वेन॑ । \newline
58. दे॒वान् थ्स्वेन॒ स्वेन॑ दे॒वान् दे॒वान् थ्स्वेन॑ भाग॒धेये॑न भाग॒धेये॑न॒ स्वेन॑ दे॒वान् दे॒वान् थ्स्वेन॑ भाग॒धेये॑न । \newline
59. स्वेन॑ भाग॒धेये॑न भाग॒धेये॑न॒ स्वेन॒ स्वेन॑ भाग॒धेये॒नो पोप॑ भाग॒धेये॑न॒ स्वेन॒ स्वेन॑ भाग॒धेये॒नोप॑ । \newline
60. भा॒ग॒धेये॒नो पोप॑ भाग॒धेये॑न भाग॒धेये॒नोप॑ धावति धाव॒त्युप॑ भाग॒धेये॑न भाग॒धेये॒नोप॑ धावति । \newline
61. भा॒ग॒धेये॒नेति॑ भाग - धेये॑न । \newline
62. उप॑ धावति धाव॒ त्युपोप॑ धावति॒ ते ते धा॑व॒ त्युपोप॑ धावति॒ ते । \newline
63. धा॒व॒ति॒ ते ते धा॑वति धावति॒ त ए॒वैव ते धा॑वति धावति॒ त ए॒व । \newline
64. त ए॒वैव ते त ए॒वास्मा॑ अस्मा ए॒व ते त ए॒वास्मै᳚ । \newline
65. ए॒वास्मा॑ अस्मा ए॒वैवास्मै॑ सजा॒तान् थ्स॑जा॒ता न॑स्मा ए॒वैवास्मै॑ सजा॒तान् । \newline
66. अ॒स्मै॒ स॒जा॒तान् थ्स॑जा॒ता न॑स्मा अस्मै सजा॒तान् प्र प्र स॑जा॒ता न॑स्मा अस्मै सजा॒तान् प्र । \newline
\pagebreak
\markright{ TS 2.1.6.5  \hfill https://www.vedavms.in \hfill}

\section{ TS 2.1.6.5 }

\textbf{TS 2.1.6.5 } \newline
\textbf{Samhita Paata} \newline

सजा॒तान् प्र य॑च्छन्ति ग्रा॒म्ये॑व भ॑वति बहुरू॒पो भ॑वति बहुदेव॒त्यो᳚(1॒) ह्ये॑ष समृ॑द्ध्यै प्राजाप॒त्यं तू॑प॒रमा ल॑भेत॒ यस्याना᳚ज्ञातमिव॒ ज्योगा॒मये᳚त् प्राजाप॒त्यो वै पुरु॑षः प्र॒जाप॑तिः॒ खलु॒ वै तस्य॑ वेद॒ यस्याना᳚ज्ञातमिव॒ ज्योगा॒मय॑ति प्र॒जाप॑तिमे॒व स्वेन॑ भाग॒धेये॒नोप॑ धावति॒ स ए॒वैनं॒ तस्मा॒थ् स्रामा᳚न्-मुञ्चति तूप॒रो भ॑वति प्राजाप॒त्यो ह्ये॑ ( ) -ष दे॒वत॑या॒ समृ॑द्ध्यै ॥ \newline

\textbf{Pada Paata} \newline

स॒जा॒तानिति॑ स-जा॒तान् । प्रेति॑ । य॒च्छ॒न्ति॒ । ग्रा॒मी । ए॒व । भ॒व॒ति॒ । ब॒हु॒रू॒प इति॑ बहु - रू॒पः । भ॒व॒ति॒ । ब॒हु॒दे॒व॒त्य॑ इति॑ बहु-दे॒व॒त्यः॑ । हि । ए॒षः । समृ॑द्ध्या॒ इति॒ सं - ऋ॒द्ध्यै॒ । प्र॒जा॒प॒त्यमिति॑ प्राजा - प॒त्यम् । तू॒प॒रम् । एति॑ । ल॒भे॒त॒ । यस्य॑ । अना᳚ज्ञात॒मित्यना᳚ - ज्ञा॒त॒म् । इ॒व॒ । ज्योक् । आ॒मये᳚त् । प्रा॒जा॒प॒त्य इति॑ प्राजा - प॒त्यः । वै । पुरु॑षः । प्र॒जाप॑ति॒रिति॑ प्र॒जा - प॒तिः॒ । खलु॑ । वै । तस्य॑ । वे॒द॒ । यस्य॑ । अना᳚ज्ञात॒मित्यना᳚ - ज्ञा॒त॒म् । इ॒व॒ । ज्योक् । आ॒मय॑ति । प्र॒जाप॑ति॒मिति॑ प्र॒जा - प॒ति॒म् । ए॒व । स्वेन॑ । भा॒ग॒धेये॒नेति॑ भाग - धेये॑न । उपेति॑ । धा॒व॒ति॒ । सः । ए॒व । ए॒न॒म् । तस्मा᳚त् । स्रामा᳚त् । मु॒ञ्च॒ति॒ । तू॒प॒रः । भ॒व॒ति॒ । प्रा॒जा॒प॒त्य इति॑ प्राजा-प॒त्यः । हि ( ) । ए॒षः । दे॒वत॑या । समृ॑द्ध्या॒ इति॒ सं - ऋ॒द्ध्यै॒ ॥  \newline


\textbf{Krama Paata} \newline

स॒जा॒तान् प्र । स॒जा॒तानिति॑ स - जा॒तान् । प्र य॑च्छन्ति । य॒च्छ॒न्ति॒ ग्रा॒मी । ग्रा॒म्ये॑व । ए॒व भ॑वति । भ॒व॒ति॒ ब॒हु॒रू॒पः । ब॒हु॒रू॒पो भ॑वति । ब॒हु॒रू॒प इति॑ बहु - रू॒पः । भ॒व॒ति॒ ब॒हु॒दे॒व॒त्यः॑ । ब॒हु॒दे॒व॒त्यो॑ हि । ब॒हु॒दे॒व॒त्य॑ इति॑ बहु - दे॒व॒त्यः॑ । ह्ये॑षः । ए॒ष समृ॑द्ध्यै । समृ॑द्ध्यै प्राजाप॒त्यम् । समृ॑द्ध्या॒ इति॒ सं - ऋ॒द्ध्यै॒ । प्रा॒जा॒प॒त्यम् तू॑प॒रम् । प्रा॒जा॒प॒त्यमिति॑ प्राजा - प॒त्यम् । तू॒प॒रमा । आ ल॑भेत । ल॒भे॒त॒ यस्य॑ । यस्याना᳚ज्ञातम् । अना᳚ज्ञातमिव । अना᳚ज्ञात॒मित्यना᳚ - ज्ञा॒त॒॒म् । इ॒व॒ ज्योक् । ज्योगा॒मये᳚त् । आ॒मये᳚त् प्राजाप॒त्यः । प्रा॒जा॒प॒त्यो वै । प्रा॒जा॒प॒त्य इति॑ प्राजा - प॒त्यः । वै पुरु॑षः । पुरु॑षः प्र॒जाप॑तिः । प्र॒जाप॑तिः॒ खलु॑ । प्र॒जाप॑ति॒रिति॑ प्र॒जा - प॒तिः॒ । खलु॒ वै । वै तस्य॑ । तस्य॑ वेद । वे॒द॒ यस्य॑ । यस्याना᳚ज्ञातम् । अना᳚ज्ञातमिव । अना᳚ज्ञात॒मित्यना᳚ - ज्ञा॒त॒॒म् । इ॒व॒ ज्योक् । ज्योगा॒मय॑ति । आ॒मय॑ति प्र॒जाप॑तिम् । प्र॒जाप॑तिमे॒व । प्र॒जाप॑ति॒मिति॑ प्र॒जा - प॒ति॒म् । ए॒व स्वेन॑ । स्वेन॑ भाग॒धेये॑न । भा॒ग॒धेये॒नोप॑ । भा॒ग॒धेये॒नेति॑ भाग - धेये॑न । उप॑ धावति । धा॒व॒ति॒ सः । स ए॒व । ए॒वैन᳚म् । ए॒न॒म् तस्मा᳚त् । तस्मा॒थ् स्रामा᳚त् । स्रामा᳚न् मुञ्चति । मु॒ञ्च॒ति॒ तू॒प॒रः । तू॒प॒रो भ॑वति । भ॒व॒ति॒ प्रा॒जा॒प॒त्यः । प्रा॒जा॒प॒त्यो हि ( ) । प्रा॒जा॒प॒त्य इति॑ प्राजा - प॒त्यः । ह्ये॑षः । ए॒ष दे॒वत॑या । दे॒वत॑या॒ समृ॑द्ध्यै । 
समृ॑द्ध्या॒ इति॒ सं - ऋ॒द्ध्यै॒ । \newline

\textbf{Jatai Paata} \newline

1. स॒जा॒तान् प्र प्र स॑जा॒तान् थ्स॑जा॒तान् प्र । \newline
2. स॒जा॒तानिति॑ स - जा॒तान् । \newline
3. प्र य॑च्छन्ति यच्छन्ति॒ प्र प्र य॑च्छन्ति । \newline
4. य॒च्छ॒न्ति॒ ग्रा॒मी ग्रा॒मी य॑च्छन्ति यच्छन्ति ग्रा॒मी । \newline
5. ग्रा॒म्ये॑वैव ग्रा॒मी ग्रा॒म्ये॑व । \newline
6. ए॒व भ॑वति भव त्ये॒वैव भ॑वति । \newline
7. भ॒व॒ति॒ ब॒हु॒रू॒पो ब॑हुरू॒पो भ॑वति भवति बहुरू॒पः । \newline
8. ब॒हु॒रू॒पो भ॑वति भवति बहुरू॒पो ब॑हुरू॒पो भ॑वति । \newline
9. ब॒हु॒रू॒प इति॑ बहु - रू॒पः । \newline
10. भ॒व॒ति॒ ब॒हु॒दे॒व॒त्यो॑ बहुदेव॒त्यो॑ भवति भवति बहुदेव॒त्यः॑ । \newline
11. ब॒हु॒दे॒व॒त्यो॑ हि हि ब॑हुदेव॒त्यो॑ बहुदेव॒त्यो॑ हि । \newline
12. ब॒हु॒दे॒व॒त्य॑ इति॑ बहु - दे॒व॒त्यः॑ । \newline
13. ह्ये॑ष ए॒ष हि ह्ये॑षः । \newline
14. ए॒ष समृ॑द्ध्यै॒ समृ॑द्ध्या ए॒ष ए॒ष समृ॑द्ध्यै । \newline
15. समृ॑द्ध्यै प्राजाप॒त्यम् प्रा॑जाप॒त्यꣳ समृ॑द्ध्यै॒ समृ॑द्ध्यै प्राजाप॒त्यम् । \newline
16. समृ॑द्ध्या॒ इति॒ सं - ऋ॒द्ध्यै॒ । \newline
17. प्रा॒जा॒प॒त्यम् तू॑प॒रम् तू॑प॒रम् प्रा॑जाप॒त्यम् प्रा॑जाप॒त्यम् तू॑प॒रम् । \newline
18. प्रा॒जा॒प॒त्यमिति॑ प्राजा - प॒त्यम् । \newline
19. तू॒प॒र मा तू॑प॒रम् तू॑प॒र मा । \newline
20. आ ल॑भेत लभे॒ता ल॑भेत । \newline
21. ल॒भे॒त॒ यस्य॒ यस्य॑ लभेत लभेत॒ यस्य॑ । \newline
22. यस्याना᳚ज्ञात॒ मना᳚ज्ञातं॒ ॅयस्य॒ यस्याना᳚ज्ञातम् । \newline
23. अना᳚ज्ञात मिवे॒ वाना᳚ज्ञात॒ मना᳚ज्ञात मिव । \newline
24. अना᳚ज्ञात॒मित्यना᳚ - ज्ञा॒त॒म् । \newline
25. इ॒व॒ ज्योग् ज्योगि॑वे व॒ ज्योक् । \newline
26. ज्योगा॒मये॑ दा॒मये॒ज् ज्योग् ज्योगा॒मये᳚त् । \newline
27. आ॒मये᳚त् प्राजाप॒त्यः प्रा॑जाप॒त्य आ॒मये॑ दा॒मये᳚त् प्राजाप॒त्यः । \newline
28. प्रा॒जा॒प॒त्यो वै वै प्रा॑जाप॒त्यः प्रा॑जाप॒त्यो वै । \newline
29. प्रा॒जा॒प॒त्य इति॑ प्राजा - प॒त्यः । \newline
30. वै पुरु॑षः॒ पुरु॑षो॒ वै वै पुरु॑षः । \newline
31. पुरु॑षः प्र॒जाप॑तिः प्र॒जाप॑तिः॒ पुरु॑षः॒ पुरु॑षः प्र॒जाप॑तिः । \newline
32. प्र॒जाप॑तिः॒ खलु॒ खलु॑ प्र॒जाप॑तिः प्र॒जाप॑तिः॒ खलु॑ । \newline
33. प्र॒जाप॑ति॒रिति॑ प्र॒जा - प॒तिः॒ । \newline
34. खलु॒ वै वै खलु॒ खलु॒ वै । \newline
35. वै तस्य॒ तस्य॒ वै वै तस्य॑ । \newline
36. तस्य॑ वेद वेद॒ तस्य॒ तस्य॑ वेद । \newline
37. वे॒द॒ यस्य॒ यस्य॑ वेद वेद॒ यस्य॑ । \newline
38. यस्याना᳚ज्ञात॒ मना᳚ज्ञातं॒ ॅयस्य॒ यस्याना᳚ज्ञातम् । \newline
39. अना᳚ज्ञात मिवे॒ वाना᳚ज्ञात॒ मना᳚ज्ञात मिव । \newline
40. अना᳚ज्ञात॒मित्यना᳚ - ज्ञा॒त॒म् । \newline
41. इ॒व॒ ज्योग् ज्योगि॑वे व॒ ज्योक् । \newline
42. ज्योगा॒मय॑ त्या॒मय॑ति॒ ज्योग् ज्योगा॒मय॑ति । \newline
43. आ॒मय॑ति प्र॒जाप॑तिम् प्र॒जाप॑ति मा॒मय॑ त्या॒मय॑ति प्र॒जाप॑तिम् । \newline
44. प्र॒जाप॑ति मे॒वैव प्र॒जाप॑तिम् प्र॒जाप॑ति मे॒व । \newline
45. प्र॒जाप॑ति॒मिति॑ प्र॒जा - प॒ति॒म् । \newline
46. ए॒व स्वेन॒ स्वेनै॒वैव स्वेन॑ । \newline
47. स्वेन॑ भाग॒धेये॑न भाग॒धेये॑न॒ स्वेन॒ स्वेन॑ भाग॒धेये॑न । \newline
48. भा॒ग॒धेये॒नोपोप॑ भाग॒धेये॑न भाग॒धेये॒नोप॑ । \newline
49. भा॒ग॒धेये॒नेति॑ भाग - धेये॑न । \newline
50. उप॑ धावति धाव॒ त्युपोप॑ धावति । \newline
51. धा॒व॒ति॒ स स धा॑वति धावति॒ सः । \newline
52. स ए॒वैव स स ए॒व । \newline
53. ए॒वैन॑ मेन मे॒वैवैन᳚म् । \newline
54. ए॒न॒म् तस्मा॒त् तस्मा॑ देन मेन॒म् तस्मा᳚त् । \newline
55. तस्मा॒थ् स्रामा॒थ् स्रामा॒त् तस्मा॒त् तस्मा॒थ् स्रामा᳚त् । \newline
56. स्रामा᳚न् मुञ्चति मुञ्चति॒ स्रामा॒थ् स्रामा᳚न् मुञ्चति । \newline
57. मु॒ञ्च॒ति॒ तू॒प॒र स्तू॑प॒रो मु॑ञ्चति मुञ्चति तूप॒रः । \newline
58. तू॒प॒रो भ॑वति भवति तूप॒र स्तू॑प॒रो भ॑वति । \newline
59. भ॒व॒ति॒ प्रा॒जा॒प॒त्यः प्रा॑जाप॒त्यो भ॑वति भवति प्राजाप॒त्यः । \newline
60. प्रा॒जा॒प॒त्यो हि हि प्रा॑जाप॒त्यः प्रा॑जाप॒त्यो हि । \newline
61. प्रा॒जा॒प॒त्य इति॑ प्राजा - प॒त्यः । \newline
62. ह्ये॑ष ए॒ष हि ह्ये॑षः । \newline
63. ए॒ष दे॒वत॑या दे॒वत॑यै॒ष ए॒ष दे॒वत॑या । \newline
64. दे॒वत॑या॒ समृ॑द्ध्यै॒ समृ॑द्ध्यै दे॒वत॑या दे॒वत॑या॒ समृ॑द्ध्यै । \newline
65. समृ॑द्ध्या॒ इति॒ सं - ऋ॒द्ध्यै॒ । \newline

\textbf{Ghana Paata } \newline

1. स॒जा॒तान् प्र प्र स॑जा॒तान् थ्स॑जा॒तान् प्र य॑च्छन्ति यच्छन्ति॒ प्र स॑जा॒तान् थ्स॑जा॒तान् प्र य॑च्छन्ति । \newline
2. स॒जा॒तानिति॑ स - जा॒तान् । \newline
3. प्र य॑च्छन्ति यच्छन्ति॒ प्र प्र य॑च्छन्ति ग्रा॒मी ग्रा॒मी य॑च्छन्ति॒ प्र प्र य॑च्छन्ति ग्रा॒मी । \newline
4. य॒च्छ॒न्ति॒ ग्रा॒मी ग्रा॒मी य॑च्छन्ति यच्छन्ति ग्रा॒म्ये॑वैव ग्रा॒मी य॑च्छन्ति यच्छन्ति ग्रा॒म्ये॑व । \newline
5. ग्रा॒म्ये॑वैव ग्रा॒मी ग्रा॒म्ये॑व भ॑वति भवत्ये॒व ग्रा॒मी ग्रा॒म्ये॑व भ॑वति । \newline
6. ए॒व भ॑वति भवत्ये॒वैव भ॑वति बहुरू॒पो ब॑हुरू॒पो भ॑वत्ये॒वैव भ॑वति बहुरू॒पः । \newline
7. भ॒व॒ति॒ ब॒हु॒रू॒पो ब॑हुरू॒पो भ॑वति भवति बहुरू॒पो भ॑वति भवति बहुरू॒पो भ॑वति भवति बहुरू॒पो भ॑वति । \newline
8. ब॒हु॒रू॒पो भ॑वति भवति बहुरू॒पो ब॑हुरू॒पो भ॑वति बहुदेव॒त्यो॑ बहुदेव॒त्यो॑ भवति बहुरू॒पो ब॑हुरू॒पो भ॑वति बहुदेव॒त्यः॑ । \newline
9. ब॒हु॒रू॒प इति॑ बहु - रू॒पः । \newline
10. भ॒व॒ति॒ ब॒हु॒दे॒व॒त्यो॑ बहुदेव॒त्यो॑ भवति भवति बहुदेव॒त्यो॑ हि हि ब॑हुदेव॒त्यो॑ भवति भवति बहुदेव॒त्यो॑ हि । \newline
11. ब॒हु॒दे॒व॒त्यो॑ हि हि ब॑हुदेव॒त्यो॑ बहुदेव॒त्यो᳚(1॒) ह्ये॑ष ए॒ष हि ब॑हुदेव॒त्यो॑ बहुदेव॒त्यो᳚(1॒) ह्ये॑षः । \newline
12. ब॒हु॒दे॒व॒त्य॑ इति॑ बहु - दे॒व॒त्यः॑ । \newline
13. ह्ये॑ष ए॒ष हि ह्ये॑ष समृ॑द्ध्यै॒ समृ॑द्ध्या ए॒ष हि ह्ये॑ष समृ॑द्ध्यै । \newline
14. ए॒ष समृ॑द्ध्यै॒ समृ॑द्ध्या ए॒ष ए॒ष समृ॑द्ध्यै प्राजाप॒त्यम् प्रा॑जाप॒त्यꣳ समृ॑द्ध्या ए॒ष ए॒ष समृ॑द्ध्यै प्राजाप॒त्यम् । \newline
15. समृ॑द्ध्यै प्राजाप॒त्यम् प्रा॑जाप॒त्यꣳ समृ॑द्ध्यै॒ समृ॑द्ध्यै प्राजाप॒त्यम् तू॑प॒रम् तू॑प॒रम् प्रा॑जाप॒त्यꣳ समृ॑द्ध्यै॒ समृ॑द्ध्यै प्राजाप॒त्यम् तू॑प॒रम् । \newline
16. समृ॑द्ध्या॒ इति॒ सं - ऋ॒द्ध्यै॒ । \newline
17. प्रा॒जा॒प॒त्यम् तू॑प॒रम् तू॑प॒रम् प्रा॑जाप॒त्यम् प्रा॑जाप॒त्यम् तू॑प॒र मा तू॑प॒रम् प्रा॑जाप॒त्यम् प्रा॑जाप॒त्यम् तू॑प॒र मा । \newline
18. प्रा॒जा॒प॒त्यमिति॑ प्राजा - प॒त्यम् । \newline
19. तू॒प॒र मा तू॑प॒रम् तू॑प॒र मा ल॑भेत लभे॒ता तू॑प॒रम् तू॑प॒र मा ल॑भेत । \newline
20. आ ल॑भेत लभे॒ता ल॑भेत॒ यस्य॒ यस्य॑ लभे॒ता ल॑भेत॒ यस्य॑ । \newline
21. ल॒भे॒त॒ यस्य॒ यस्य॑ लभेत लभेत॒ यस्याना᳚ज्ञात॒ मना᳚ज्ञातं॒ ॅयस्य॑ लभेत लभेत॒ यस्याना᳚ज्ञातम् । \newline
22. यस्याना᳚ज्ञात॒ मना᳚ज्ञातं॒ ॅयस्य॒ यस्याना᳚ज्ञात मिवे॒ वाना᳚ज्ञातं॒ ॅयस्य॒ यस्याना᳚ज्ञात मिव । \newline
23. अना᳚ज्ञात मिवे॒ वाना᳚ज्ञात॒ मना᳚ज्ञात मिव॒ ज्योग् ज्योगि॒वाना᳚ज्ञात॒ मना᳚ज्ञात मिव॒ ज्योक् । \newline
24. अना᳚ज्ञात॒मित्यना᳚ - ज्ञा॒त॒म् । \newline
25. इ॒व॒ ज्योग् ज्योगि॑वे व॒ ज्योगा॒मये॑ दा॒मये॒ज् ज्योगि॑वे व॒ ज्योगा॒मये᳚त् । \newline
26. ज्योगा॒मये॑ दा॒मये॒ज् ज्योग् ज्योगा॒मये᳚त् प्राजाप॒त्यः प्रा॑जाप॒त्य आ॒मये॒ज् ज्योग् ज्योगा॒मये᳚त् प्राजाप॒त्यः । \newline
27. आ॒मये᳚त् प्राजाप॒त्यः प्रा॑जाप॒त्य आ॒मये॑ दा॒मये᳚त् प्राजाप॒त्यो वै वै प्रा॑जाप॒त्य आ॒मये॑ दा॒मये᳚त् प्राजाप॒त्यो वै । \newline
28. प्रा॒जा॒प॒त्यो वै वै प्रा॑जाप॒त्यः प्रा॑जाप॒त्यो वै पुरु॑षः॒ पुरु॑षो॒ वै प्रा॑जाप॒त्यः प्रा॑जाप॒त्यो वै पुरु॑षः । \newline
29. प्रा॒जा॒प॒त्य इति॑ प्राजा - प॒त्यः । \newline
30. वै पुरु॑षः॒ पुरु॑षो॒ वै वै पुरु॑षः प्र॒जाप॑तिः प्र॒जाप॑तिः॒ पुरु॑षो॒ वै वै पुरु॑षः प्र॒जाप॑तिः । \newline
31. पुरु॑षः प्र॒जाप॑तिः प्र॒जाप॑तिः॒ पुरु॑षः॒ पुरु॑षः प्र॒जाप॑तिः॒ खलु॒ खलु॑ प्र॒जाप॑तिः॒ पुरु॑षः॒ पुरु॑षः प्र॒जाप॑तिः॒ खलु॑ । \newline
32. प्र॒जाप॑तिः॒ खलु॒ खलु॑ प्र॒जाप॑तिः प्र॒जाप॑तिः॒ खलु॒ वै वै खलु॑ प्र॒जाप॑तिः प्र॒जाप॑तिः॒ खलु॒ वै । \newline
33. प्र॒जाप॑ति॒रिति॑ प्र॒जा - प॒तिः॒ । \newline
34. खलु॒ वै वै खलु॒ खलु॒ वै तस्य॒ तस्य॒ वै खलु॒ खलु॒ वै तस्य॑ । \newline
35. वै तस्य॒ तस्य॒ वै वै तस्य॑ वेद वेद॒ तस्य॒ वै वै तस्य॑ वेद । \newline
36. तस्य॑ वेद वेद॒ तस्य॒ तस्य॑ वेद॒ यस्य॒ यस्य॑ वेद॒ तस्य॒ तस्य॑ वेद॒ यस्य॑ । \newline
37. वे॒द॒ यस्य॒ यस्य॑ वेद वेद॒ यस्याना᳚ज्ञात॒ मना᳚ज्ञातं॒ ॅयस्य॑ वेद वेद॒ यस्याना᳚ज्ञातम् । \newline
38. यस्याना᳚ज्ञात॒ मना᳚ज्ञातं॒ ॅयस्य॒ यस्याना᳚ज्ञात मिवे॒ वाना᳚ज्ञातं॒ ॅयस्य॒ यस्याना᳚ज्ञात मिव । \newline
39. अना᳚ज्ञात मिवे॒ वाना᳚ज्ञात॒ मना᳚ज्ञात मिव॒ ज्योग् ज्योगि॒वाना᳚ज्ञात॒ मना᳚ज्ञात मिव॒ ज्योक् । \newline
40. अना᳚ज्ञात॒मित्यना᳚ - ज्ञा॒त॒म् । \newline
41. इ॒व॒ ज्योग् ज्योगि॑वे व॒ ज्योगा॒मय॑ त्या॒मय॑ति॒ ज्योगि॑वे व॒ ज्योगा॒मय॑ति । \newline
42. ज्योगा॒मय॑ त्या॒मय॑ति॒ ज्योग् ज्योगा॒मय॑ति प्र॒जाप॑तिम् प्र॒जाप॑ति मा॒मय॑ति॒ ज्योग् ज्योगा॒मय॑ति प्र॒जाप॑तिम् । \newline
43. आ॒मय॑ति प्र॒जाप॑तिम् प्र॒जाप॑ति मा॒मय॑ त्या॒मय॑ति प्र॒जाप॑ति मे॒वैव प्र॒जाप॑ति मा॒मय॑ त्या॒मय॑ति प्र॒जाप॑ति मे॒व । \newline
44. प्र॒जाप॑ति मे॒वैव प्र॒जाप॑तिम् प्र॒जाप॑ति मे॒व स्वेन॒ स्वेनै॒व प्र॒जाप॑तिम् प्र॒जाप॑ति मे॒व स्वेन॑ । \newline
45. प्र॒जाप॑ति॒मिति॑ प्र॒जा - प॒ति॒म् । \newline
46. ए॒व स्वेन॒ स्वेनै॒वैव स्वेन॑ भाग॒धेये॑न भाग॒धेये॑न॒ स्वेनै॒वैव स्वेन॑ भाग॒धेये॑न । \newline
47. स्वेन॑ भाग॒धेये॑न भाग॒धेये॑न॒ स्वेन॒ स्वेन॑ भाग॒धेये॒नो पोप॑ भाग॒धेये॑न॒ स्वेन॒ स्वेन॑ भाग॒धेये॒नोप॑ । \newline
48. भा॒ग॒धेये॒नो पोप॑ भाग॒धेये॑न भाग॒धेये॒नोप॑ धावति धाव॒त्युप॑ भाग॒धेये॑न भाग॒धेये॒नोप॑ धावति । \newline
49. भा॒ग॒धेये॒नेति॑ भाग - धेये॑न । \newline
50. उप॑ धावति धाव॒ त्युपोप॑ धावति॒ स स धा॑व॒ त्युपोप॑ धावति॒ सः । \newline
51. धा॒व॒ति॒ स स धा॑वति धावति॒ स ए॒वैव स धा॑वति धावति॒ स ए॒व । \newline
52. स ए॒वैव स स ए॒वैन॑ मेन मे॒व स स ए॒वैन᳚म् । \newline
53. ए॒वैन॑ मेन मे॒वैवैन॒म् तस्मा॒त् तस्मा॑देन मे॒वैवैन॒म् तस्मा᳚त् । \newline
54. ए॒न॒म् तस्मा॒त् तस्मा॑देन मेन॒म् तस्मा॒थ् स्रामा॒थ् स्रामा॒त् तस्मा॑देन मेन॒म् तस्मा॒थ् स्रामा᳚त् । \newline
55. तस्मा॒थ् स्रामा॒थ् स्रामा॒त् तस्मा॒त् तस्मा॒थ् स्रामा᳚न् मुञ्चति मुञ्चति॒ स्रामा॒त् तस्मा॒त् तस्मा॒थ् स्रामा᳚न् मुञ्चति । \newline
56. स्रामा᳚न् मुञ्चति मुञ्चति॒ स्रामा॒थ् स्रामा᳚न् मुञ्चति तूप॒र स्तू॑प॒रो मु॑ञ्चति॒ स्रामा॒थ् स्रामा᳚न् मुञ्चति तूप॒रः । \newline
57. मु॒ञ्च॒ति॒ तू॒प॒र स्तू॑प॒रो मु॑ञ्चति मुञ्चति तूप॒रो भ॑वति भवति तूप॒रो मु॑ञ्चति मुञ्चति तूप॒रो भ॑वति । \newline
58. तू॒प॒रो भ॑वति भवति तूप॒र स्तू॑प॒रो भ॑वति प्राजाप॒त्यः प्रा॑जाप॒त्यो भ॑वति तूप॒र स्तू॑प॒रो भ॑वति प्राजाप॒त्यः । \newline
59. भ॒व॒ति॒ प्रा॒जा॒प॒त्यः प्रा॑जाप॒त्यो भ॑वति भवति प्राजाप॒त्यो हि हि प्रा॑जाप॒त्यो भ॑वति भवति प्राजाप॒त्यो हि । \newline
60. प्रा॒जा॒प॒त्यो हि हि प्रा॑जाप॒त्यः प्रा॑जाप॒त्यो ह्ये॑ष ए॒ष हि प्रा॑जाप॒त्यः प्रा॑जाप॒त्यो ह्ये॑षः । \newline
61. प्रा॒जा॒प॒त्य इति॑ प्राजा - प॒त्यः । \newline
62. ह्ये॑ष ए॒ष हि ह्ये॑ष दे॒वत॑या दे॒वत॑यै॒ष हि ह्ये॑ष दे॒वत॑या । \newline
63. ए॒ष दे॒वत॑या दे॒वत॑यै॒ष ए॒ष दे॒वत॑या॒ समृ॑द्ध्यै॒ समृ॑द्ध्यै दे॒वत॑यै॒ष ए॒ष दे॒वत॑या॒ समृ॑द्ध्यै । \newline
64. दे॒वत॑या॒ समृ॑द्ध्यै॒ समृ॑द्ध्यै दे॒वत॑या दे॒वत॑या॒ समृ॑द्ध्यै । \newline
65. समृ॑द्ध्या॒ इति॒ सं - ऋ॒द्ध्यै॒ । \newline
\pagebreak
\markright{ TS 2.1.7.1  \hfill https://www.vedavms.in \hfill}

\section{ TS 2.1.7.1 }

\textbf{TS 2.1.7.1 } \newline
\textbf{Samhita Paata} \newline

व॒ष॒ट्का॒रो वै गा॑यत्रि॒यै शिरो᳚ऽच्छिन॒त् तस्यै॒ रसः॒ परा॑पत॒त् तं बृह॒स्पति॒ रुपा॑ऽगृह्णा॒थ्‌सा शि॑तिपृ॒ष्ठा व॒शाऽभ॑व॒द्यो द्वि॒तीयः॑ प॒राऽप॑त॒त् तं मि॒त्रावरु॑णा॒-वुपा॑गृह्णीताꣳ॒॒ सा द्वि॑रू॒पा व॒शाऽभ॑व॒द्-यस्तृ॒तीयः॑ प॒राप॑त॒त् तं ॅविश्वे॑ दे॒वा उपा॑गृह्ण॒न्थ् सा ब॑हुरू॒पा व॒शा भ॑व॒द्य-श्च॑तु॒र्त्थः प॒राप॑त॒थ् स पृ॑थि॒वीं प्राऽवि॑श॒त् तं बृह॒स्पति॑र॒भ्य॑ - [  ] \newline

\textbf{Pada Paata} \newline

व॒ष॒ट्का॒र इति॑ वषट् - का॒रः । वै । गा॒य॒त्रि॒यै । शिरः॑ । अ॒च्छि॒न॒त् । तस्यै᳚ । रसः॑ । परेति॑ । अ॒प॒त॒त् । तम् । बृह॒स्पतिः॑ । उपेति॑ । अ॒गृ॒ह्णा॒त् । सा । शि॒ति॒पृ॒ष्ठेति॑ शिति - प॒ष्ठा । व॒शा । अ॒भ॒व॒त् । यः । द्वि॒तीयः॑ । प॒राप॑त॒दिति॑ परा - अप॑तत् । तम् । मि॒त्रावरु॑णा॒विति॑ मि॒त्रा - वरु॑णौ । उपेति॑ । अ॒गृ॒ह्णी॒ता॒म् । सा । द्वि॒रू॒पेति॑ द्वि - रू॒पा । व॒शा । अ॒भ॒व॒त् । यः । तृ॒तीयः॑ । प॒राप॑त॒दिति॑ परा - अप॑तत् । तम् । विश्वे᳚ । दे॒वाः । उपेति॑ । अ॒गृ॒ह्ण॒न्न् । सा । ब॒हु॒रू॒पेति॑ बहु - रू॒पा । व॒शा । अ॒भ॒व॒त् । यः । च॒तु॒र्थः । प॒राप॑त॒दिति॑ परा - अप॑तत् । सः । पृ॒थि॒वीम् । प्रेति॑ । अ॒वि॒श॒त् । तम् । बृह॒स्पतिः॑ । अ॒भीति॑ ।  \newline


\textbf{Krama Paata} \newline

व॒ष॒ट्का॒रो वै । व॒ष॒ट्का॒र इति॑ वषट् - का॒रः । वै गा॑यत्रि॒यै । गा॒य॒त्रि॒यै शिरः॑ । शिरो᳚ऽच्छिनत् । अ॒च्छि॒न॒त् तस्यै᳚ । तस्यै॒ रसः॑ । रसः॒ परा᳚ । परा॑ऽपतत् । अ॒प॒त॒त् तम् । तम् बृह॒स्पतिः॑ । बृह॒स्पति॒रुप॑ । उपा॑गृह्णात् । अ॒गृ॒ह्णा॒थ् सा । सा शि॑तिपृ॒ष्ठा । शि॒ति॒पृ॒ष्ठा व॒शा । शि॒ति॒पृ॒ष्ठेति॑ शिति - पृ॒ष्ठा । व॒शा ऽभ॑वत् । अ॒भ॒व॒द् यः । यो द्वि॒तीयः॑ । द्वि॒तीयः॑ प॒राप॑तत् । प॒राप॑त॒त् तम् । प॒राप॑त॒दिति॑ परा - अप॑तत् । तम् मि॒त्रावरु॑णौ । मि॒त्रावरु॑णा॒वुप॑ । मि॒त्रावरु॑णा॒विति॑ मि॒त्रा - वरु॑णौ । उपा॑गृह्णीताम् । अ॒गृ॒ह्णी॒ताꣳ॒॒ सा । सा द्वि॑रू॒पा । द्वि॒रू॒पा व॒शा । द्वि॒रू॒पेति॑ द्वि - रू॒पा । व॒शाऽभ॑वत् । अ॒भ॒व॒द् यः । य स्तृ॒तीयः॑ । तृ॒तीयः॑ प॒राप॑तत् । प॒राप॑त॒त् तम् । प॒राप॑त॒दिति॑ परा - अप॑तत् । तं ॅविश्वे᳚ । विश्वे॑ दे॒वाः । दे॒वा उप॑ । उपा॑गृह्णन्न् । अ॒गृ॒ह्ण॒न्थ् सा । सा ब॑हुरू॒पा । ब॒हु॒रू॒पा व॒शा । ब॒हु॒रू॒पेति॑ बहु - रू॒पा । व॒शाऽभ॑वत् । अ॒भ॒व॒द् यः । य श्च॑तु॒र्त्थः । च॒तु॒र्त्थः प॒राप॑तत् । प॒राप॑त॒थ् सः । प॒राप॑त॒दिति॑ परा - अप॑तत् । स पृ॑थि॒वीम् । पृ॒थि॒वीम् प्र । प्रावि॑शत् । अ॒वि॒श॒त् तम् । तम् बृह॒स्पतिः॑ । बृह॒स्पति॑र॒भि । अ॒भ्य॑गृह्णात् \newline

\textbf{Jatai Paata} \newline

1. व॒ष॒ट्का॒रो वै वै व॑षट्का॒रो व॑षट्का॒रो वै । \newline
2. व॒ष॒ट्का॒र इति॑ वषट् - का॒रः । \newline
3. वै गा॑यत्रि॒यै गा॑यत्रि॒यै वै वै गा॑यत्रि॒यै । \newline
4. गा॒य॒त्रि॒यै शिरः॒ शिरो॑ गायत्रि॒यै गा॑यत्रि॒यै शिरः॑ । \newline
5. शिरो᳚ ऽच्छिन दच्छिन॒च् छिरः॒ शिरो᳚ ऽच्छिनत् । \newline
6. अ॒च्छि॒न॒त् तस्यै॒ तस्या॑ अच्छिन दच्छिन॒त् तस्यै᳚ । \newline
7. तस्यै॒ रसो॒ रस॒ स्तस्यै॒ तस्यै॒ रसः॑ । \newline
8. रसः॒ परा॒ परा॒ रसो॒ रसः॒ परा᳚ । \newline
9. परा॑पत दपत॒त् परा॒ परा॑पतत् । \newline
10. अ॒प॒त॒त् तम् त म॑पत दपत॒त् तम् । \newline
11. तम् बृह॒स्पति॒र् बृह॒स्पति॒ स्तम् तम् बृह॒स्पतिः॑ । \newline
12. बृह॒स्पति॒ रुपोप॒ बृह॒स्पति॒र् बृह॒स्पति॒ रुप॑ । \newline
13. उपा॑गृह्णा दगृह्णा॒ दुपोपा॑गृह्णात् । \newline
14. अ॒गृ॒ह्णा॒थ् सा सा ऽगृ॑ह्णा दगृह्णा॒थ् सा । \newline
15. सा शि॑तिपृ॒ष्ठा शि॑तिपृ॒ष्ठा सा सा शि॑तिपृ॒ष्ठा । \newline
16. शि॒ति॒पृ॒ष्ठा व॒शा व॒शा शि॑तिपृ॒ष्ठा शि॑तिपृ॒ष्ठा व॒शा । \newline
17. शि॒ति॒पृ॒ष्ठेति॑ शिति - पृ॒ष्ठा । \newline
18. व॒शा ऽभ॑व दभवद् व॒शा व॒शा ऽभ॑वत् । \newline
19. अ॒भ॒व॒द् यो यो॑ ऽभव दभव॒द् यः । \newline
20. यो द्वि॒तीयो᳚ द्वि॒तीयो॒ यो यो द्वि॒तीयः॑ । \newline
21. द्वि॒तीयः॑ प॒राप॑तत् प॒राप॑तद् द्वि॒तीयो᳚ द्वि॒तीयः॑ प॒राप॑तत् । \newline
22. प॒राप॑त॒त् तम् तम् प॒राप॑तत् प॒राप॑त॒त् तम् । \newline
23. प॒राप॑त॒दिति॑ परा - अप॑तत् । \newline
24. तम् मि॒त्रावरु॑णौ मि॒त्रावरु॑णौ॒ तम् तम् मि॒त्रावरु॑णौ । \newline
25. मि॒त्रावरु॑णा॒ वुपोप॑ मि॒त्रावरु॑णौ मि॒त्रावरु॑णा॒ वुप॑ । \newline
26. मि॒त्रावरु॑णा॒विति॑ मि॒त्रा - वरु॑णौ । \newline
27. उपा॑गृह्णीता मगृह्णीता॒ मुपोपा॑गृह्णीताम् । \newline
28. अ॒गृ॒ह्णी॒ताꣳ॒॒ सा सा ऽगृ॑ह्णीता मगृह्णीताꣳ॒॒ सा । \newline
29. सा द्वि॑रू॒पा द्वि॑रू॒पा सा सा द्वि॑रू॒पा । \newline
30. द्वि॒रू॒पा व॒शा व॒शा द्वि॑रू॒पा द्वि॑रू॒पा व॒शा । \newline
31. द्वि॒रू॒पेति॑ द्वि - रू॒पा । \newline
32. व॒शा ऽभ॑व दभवद् व॒शा व॒शा ऽभ॑वत् । \newline
33. अ॒भ॒व॒द् यो यो॑ ऽभव दभव॒द् यः । \newline
34. यस्तृ॒तीय॑ स्तृ॒तीयो॒ यो यस्तृ॒तीयः॑ । \newline
35. तृ॒तीयः॑ प॒राप॑तत् प॒राप॑तत् तृ॒तीय॑ स्तृ॒तीयः॑ प॒राप॑तत् । \newline
36. प॒राप॑त॒त् तम् तम् प॒राप॑तत् प॒राप॑त॒त् तम् । \newline
37. प॒राप॑त॒दिति॑ परा - अप॑तत् । \newline
38. तं ॅविश्वे॒ विश्वे॒ तम् तं ॅविश्वे᳚ । \newline
39. विश्वे॑ दे॒वा दे॒वा विश्वे॒ विश्वे॑ दे॒वाः । \newline
40. दे॒वा उपोप॑ दे॒वा दे॒वा उप॑ । \newline
41. उपा॑गृह्णन् नगृह्ण॒न् नुपोपा॑गृह्णन्न् । \newline
42. अ॒गृ॒ह्ण॒न् थ्सा सा ऽगृ॑ह्णन् नगृह्ण॒न् थ्सा । \newline
43. सा ब॑हुरू॒पा ब॑हुरू॒पा सा सा ब॑हुरू॒पा । \newline
44. ब॒हु॒रू॒पा व॒शा व॒शा ब॑हुरू॒पा ब॑हुरू॒पा व॒शा । \newline
45. ब॒हु॒रू॒पेति॑ बहु - रू॒पा । \newline
46. व॒शा ऽभ॑व दभवद् व॒शा व॒शा ऽभ॑वत् । \newline
47. अ॒भ॒व॒द् यो यो॑ ऽभव दभव॒द् यः । \newline
48. यश्च॑तु॒र्थ श्च॑तु॒र्थो यो यश्च॑तु॒र्थः । \newline
49. च॒तु॒र्थः प॒राप॑तत् प॒राप॑तच् चतु॒र्थ श्च॑तु॒र्थः प॒राप॑तत् । \newline
50. प॒राप॑त॒थ् स स प॒राप॑तत् प॒राप॑त॒थ् सः । \newline
51. प॒राप॑त॒दिति॑ परा - अप॑तत् । \newline
52. स पृ॑थि॒वीम् पृ॑थि॒वीꣳ स स पृ॑थि॒वीम् । \newline
53. पृ॒थि॒वीम् प्र प्र पृ॑थि॒वीम् पृ॑थि॒वीम् प्र । \newline
54. प्रावि॑श दविश॒त् प्र प्रावि॑शत् । \newline
55. अ॒वि॒श॒त् तम् त म॑विश दविश॒त् तम् । \newline
56. तम् बृह॒स्पति॒र् बृह॒स्पति॒ स्तम् तम् बृह॒स्पतिः॑ । \newline
57. बृह॒स्पति॑ र॒भ्य॑भि बृह॒स्पति॒र् बृह॒स्पति॑ र॒भि । \newline
58. अ॒भ्य॑गृह्णा दगृह्णा द॒भ्या᳚(1॒)भ्य॑गृह्णात् । \newline

\textbf{Ghana Paata } \newline

1. व॒ष॒ट्का॒रो वै वै व॑षट्का॒रो व॑षट्का॒रो वै गा॑यत्रि॒यै गा॑यत्रि॒यै वै व॑षट्का॒रो व॑षट्का॒रो वै गा॑यत्रि॒यै । \newline
2. व॒ष॒ट्का॒र इति॑ वषट् - का॒रः । \newline
3. वै गा॑यत्रि॒यै गा॑यत्रि॒यै वै वै गा॑यत्रि॒यै शिरः॒ शिरो॑ गायत्रि॒यै वै वै गा॑यत्रि॒यै शिरः॑ । \newline
4. गा॒य॒त्रि॒यै शिरः॒ शिरो॑ गायत्रि॒यै गा॑यत्रि॒यै शिरो᳚ ऽच्छिन दच्छिन॒च् छिरो॑ गायत्रि॒यै गा॑यत्रि॒यै शिरो᳚ ऽच्छिनत् । \newline
5. शिरो᳚ ऽच्छिन दच्छिन॒च् छिरः॒ शिरो᳚ ऽच्छिन॒त् तस्यै॒ तस्या॑ अच्छिन॒च् छिरः॒ शिरो᳚ ऽच्छिन॒त् तस्यै᳚ । \newline
6. अ॒च्छि॒न॒त् तस्यै॒ तस्या॑ अच्छिन दच्छिन॒त् तस्यै॒ रसो॒ रस॒ स्तस्या॑ अच्छिन दच्छिन॒त् तस्यै॒ रसः॑ । \newline
7. तस्यै॒ रसो॒ रस॒ स्तस्यै॒ तस्यै॒ रसः॒ परा॒ परा॒ रस॒ स्तस्यै॒ तस्यै॒ रसः॒ परा᳚ । \newline
8. रसः॒ परा॒ परा॒ रसो॒ रसः॒ परा॑पत दपत॒त् परा॒ रसो॒ रसः॒ परा॑पतत् । \newline
9. परा॑पत दपत॒त् परा॒ परा॑पत॒त् तम् त म॑पत॒त् परा॒ परा॑पत॒त् तम् । \newline
10. अ॒प॒त॒त् तम् त म॑पत दपत॒त् तम् बृह॒स्पति॒र् बृह॒स्पति॒ स्त म॑पत दपत॒त् तम् बृह॒स्पतिः॑ । \newline
11. तम् बृह॒स्पति॒र् बृह॒स्पति॒ स्तम् तम् बृह॒स्पति॒ रुपोप॒ बृह॒स्पति॒ स्तम् तम् बृह॒स्पति॒ रुप॑ । \newline
12. बृह॒स्पति॒ रुपोप॒ बृह॒स्पति॒र् बृह॒स्पति॒ रुपा॑ गृह्णा दगृह्णा॒ दुप॒ बृह॒स्पति॒र् बृह॒स्पति॒ रुपा॑गृह्णात् । \newline
13. उपा॑गृह्णा दगृह्णा॒ दुपोपा॑गृह्णा॒थ् सा सा ऽगृ॑ह्णा॒ दुपोपा॑गृह्णा॒थ् सा । \newline
14. अ॒गृ॒ह्णा॒थ् सा सा ऽगृ॑ह्णा दगृह्णा॒थ् सा शि॑तिपृ॒ष्ठा शि॑तिपृ॒ष्ठा सा ऽगृ॑ह्णा दगृह्णा॒थ् सा शि॑तिपृ॒ष्ठा । \newline
15. सा शि॑तिपृ॒ष्ठा शि॑तिपृ॒ष्ठा सा सा शि॑तिपृ॒ष्ठा व॒शा व॒शा शि॑तिपृ॒ष्ठा सा सा शि॑तिपृ॒ष्ठा व॒शा । \newline
16. शि॒ति॒पृ॒ष्ठा व॒शा व॒शा शि॑तिपृ॒ष्ठा शि॑तिपृ॒ष्ठा व॒शा ऽभ॑व दभवद् व॒शा शि॑तिपृ॒ष्ठा शि॑तिपृ॒ष्ठा व॒शा ऽभ॑वत् । \newline
17. शि॒ति॒पृ॒ष्ठेति॑ शिति - पृ॒ष्ठा । \newline
18. व॒शा ऽभ॑व दभवद् व॒शा व॒शा ऽभ॑व॒द् यो यो॑ ऽभवद् व॒शा व॒शा ऽभ॑व॒द् यः । \newline
19. अ॒भ॒व॒द् यो यो॑ ऽभव दभव॒द् यो द्वि॒तीयो᳚ द्वि॒तीयो॒ यो॑ ऽभव दभव॒द् यो द्वि॒तीयः॑ । \newline
20. यो द्वि॒तीयो᳚ द्वि॒तीयो॒ यो यो द्वि॒तीयः॑ प॒राप॑तत् प॒राप॑तद् द्वि॒तीयो॒ यो यो द्वि॒तीयः॑ प॒राप॑तत् । \newline
21. द्वि॒तीयः॑ प॒राप॑तत् प॒राप॑तद् द्वि॒तीयो᳚ द्वि॒तीयः॑ प॒राप॑त॒त् तम् तम् प॒राप॑तद् द्वि॒तीयो᳚ द्वि॒तीयः॑ प॒राप॑त॒त् तम् । \newline
22. प॒राप॑त॒त् तम् तम् प॒राप॑तत् प॒राप॑त॒त् तम् मि॒त्रावरु॑णौ मि॒त्रावरु॑णौ॒ तम् प॒राप॑तत् प॒राप॑त॒त् तम् मि॒त्रावरु॑णौ । \newline
23. प॒राप॑त॒दिति॑ परा - अप॑तत् । \newline
24. तम् मि॒त्रावरु॑णौ मि॒त्रावरु॑णौ॒ तम् तम् मि॒त्रावरु॑णा॒ वुपोप॑ मि॒त्रावरु॑णौ॒ तम् तम् मि॒त्रावरु॑णा॒ वुप॑ । \newline
25. मि॒त्रावरु॑णा॒ वुपोप॑ मि॒त्रावरु॑णौ मि॒त्रावरु॑णा॒ वुपा॑गृह्णीता मगृह्णीता॒ मुप॑ मि॒त्रावरु॑णौ मि॒त्रावरु॑णा॒ वुपा॑गृह्णीताम् । \newline
26. मि॒त्रावरु॑णा॒विति॑ मि॒त्रा - वरु॑णौ । \newline
27. उपा॑गृह्णीता मगृह्णीता॒ मुपोपा॑गृह्णीताꣳ॒॒ सा सा ऽगृ॑ह्णीता॒ मुपोपा॑गृह्णीताꣳ॒॒ सा । \newline
28. अ॒गृ॒ह्णी॒ताꣳ॒॒ सा सा ऽगृ॑ह्णीता मगृह्णीताꣳ॒॒ सा द्वि॑रू॒पा द्वि॑रू॒पा सा ऽगृ॑ह्णीता मगृह्णीताꣳ॒॒ सा द्वि॑रू॒पा । \newline
29. सा द्वि॑रू॒पा द्वि॑रू॒पा सा सा द्वि॑रू॒पा व॒शा व॒शा द्वि॑रू॒पा सा सा द्वि॑रू॒पा व॒शा । \newline
30. द्वि॒रू॒पा व॒शा व॒शा द्वि॑रू॒पा द्वि॑रू॒पा व॒शा ऽभ॑व दभवद् व॒शा द्वि॑रू॒पा द्वि॑रू॒पा व॒शा ऽभ॑वत् । \newline
31. द्वि॒रू॒पेति॑ द्वि - रू॒पा । \newline
32. व॒शा ऽभ॑व दभवद् व॒शा व॒शा ऽभ॑व॒द् यो यो॑ ऽभवद् व॒शा व॒शा ऽभ॑व॒द् यः । \newline
33. अ॒भ॒व॒द् यो यो॑ ऽभव दभव॒द् य स्तृ॒तीय॑ स्तृ॒तीयो॒ यो॑ ऽभव दभव॒द् य स्तृ॒तीयः॑ । \newline
34. य स्तृ॒तीय॑ स्तृ॒तीयो॒ यो य स्तृ॒तीयः॑ प॒राप॑तत् प॒राप॑तत् तृ॒तीयो॒ यो य स्तृ॒तीयः॑ प॒राप॑तत् । \newline
35. तृ॒तीयः॑ प॒राप॑तत् प॒राप॑तत् तृ॒तीय॑ स्तृ॒तीयः॑ प॒राप॑त॒त् तम् तम् प॒राप॑तत् तृ॒तीय॑ स्तृ॒तीयः॑ प॒राप॑त॒त् तम् । \newline
36. प॒राप॑त॒त् तम् तम् प॒राप॑तत् प॒राप॑त॒त् तं ॅविश्वे॒ विश्वे॒ तम् प॒राप॑तत् प॒राप॑त॒त् तं ॅविश्वे᳚ । \newline
37. प॒राप॑त॒दिति॑ परा - अप॑तत् । \newline
38. तं ॅविश्वे॒ विश्वे॒ तम् तं ॅविश्वे॑ दे॒वा दे॒वा विश्वे॒ तम् तं ॅविश्वे॑ दे॒वाः । \newline
39. विश्वे॑ दे॒वा दे॒वा विश्वे॒ विश्वे॑ दे॒वा उपोप॑ दे॒वा विश्वे॒ विश्वे॑ दे॒वा उप॑ । \newline
40. दे॒वा उपोप॑ दे॒वा दे॒वा उपा॑गृह्णन् नगृह्ण॒न् नुप॑ दे॒वा दे॒वा उपा॑गृह्णन्न् । \newline
41. उपा॑गृह्णन् नगृह्ण॒न् नुपोपा॑गृह्ण॒न् थ्सा सा ऽगृ॑ह्ण॒न् नुपोपा॑गृह्ण॒न् थ्सा । \newline
42. अ॒गृ॒ह्ण॒न् थ्सा सा ऽगृ॑ह्णन् नगृह्ण॒न् थ्सा ब॑हुरू॒पा ब॑हुरू॒पा सा ऽगृ॑ह्णन् नगृह्ण॒न् थ्सा ब॑हुरू॒पा । \newline
43. सा ब॑हुरू॒पा ब॑हुरू॒पा सा सा ब॑हुरू॒पा व॒शा व॒शा ब॑हुरू॒पा सा सा ब॑हुरू॒पा व॒शा । \newline
44. ब॒हु॒रू॒पा व॒शा व॒शा ब॑हुरू॒पा ब॑हुरू॒पा व॒शा ऽभ॑व दभवद् व॒शा ब॑हुरू॒पा ब॑हुरू॒पा व॒शा ऽभ॑वत् । \newline
45. ब॒हु॒रू॒पेति॑ बहु - रू॒पा । \newline
46. व॒शा ऽभ॑व दभवद् व॒शा व॒शा ऽभ॑व॒द् यो यो॑ ऽभवद् व॒शा व॒शा ऽभ॑व॒द् यः । \newline
47. अ॒भ॒व॒द् यो यो॑ ऽभव दभव॒द् यश्च॑तु॒र्थ श्च॑तु॒र्थो यो॑ ऽभव दभव॒द् यश्च॑तु॒र्थः । \newline
48. यश्च॑तु॒र्थ श्च॑तु॒र्थो यो यश्च॑तु॒र्थः प॒राप॑तत् प॒राप॑तच् चतु॒र्थो यो यश्च॑तु॒र्थः प॒राप॑तत् । \newline
49. च॒तु॒र्थः प॒राप॑तत् प॒राप॑तच् चतु॒र्थ श्च॑तु॒र्थः प॒राप॑त॒थ् स स प॒राप॑तच् चतु॒र्थ श्च॑तु॒र्थः प॒राप॑त॒थ् सः । \newline
50. प॒राप॑त॒थ् स स प॒राप॑तत् प॒राप॑त॒थ् स पृ॑थि॒वीम् पृ॑थि॒वीꣳ स प॒राप॑तत् प॒राप॑त॒थ् स पृ॑थि॒वीम् । \newline
51. प॒राप॑त॒दिति॑ परा - अप॑तत् । \newline
52. स पृ॑थि॒वीम् पृ॑थि॒वीꣳ स स पृ॑थि॒वीम् प्र प्र पृ॑थि॒वीꣳ स स पृ॑थि॒वीम् प्र । \newline
53. पृ॒थि॒वीम् प्र प्र पृ॑थि॒वीम् पृ॑थि॒वीम् प्रावि॑श दविश॒त् प्र पृ॑थि॒वीम् पृ॑थि॒वीम् प्रावि॑शत् । \newline
54. प्रावि॑श दविश॒त् प्र प्रावि॑श॒त् तम् त म॑विश॒त् प्र प्रावि॑श॒त् तम् । \newline
55. अ॒वि॒श॒त् तम् त म॑विश दविश॒त् तम् बृह॒स्पति॒र् बृह॒स्पति॒ स्त म॑विश दविश॒त् तम् बृह॒स्पतिः॑ । \newline
56. तम् बृह॒स्पति॒र् बृह॒स्पति॒ स्तम् तम् बृह॒स्पति॑ र॒भ्य॑भि बृह॒स्पति॒ स्तम् तम् बृह॒स्पति॑ र॒भि । \newline
57. बृह॒स्पति॑ र॒भ्य॑भि बृह॒स्पति॒र् बृह॒स्पति॑ र॒भ्य॑गृह्णा दगृह्णाद॒भि बृह॒स्पति॒र् बृह॒स्पति॑ र॒भ्य॑गृह्णात् । \newline
58. अ॒भ्य॑गृह्णा दगृह्णा द॒भ्या᳚(1॒)भ्य॑गृह्णा॒ दस्त्व स्त्व॑गृह्णा द॒भ्या᳚(1॒)भ्य॑गृह्णा॒ दस्तु॑ । \newline
\pagebreak
\markright{ TS 2.1.7.2  \hfill https://www.vedavms.in \hfill}

\section{ TS 2.1.7.2 }

\textbf{TS 2.1.7.2 } \newline
\textbf{Samhita Paata} \newline

-गृह्णा॒-दस्त्वे॒वायं भोगा॒येति॒ स उ॑क्षव॒शः सम॑भव॒द्-यल्लोहि॑तं प॒राप॑त॒त् तद्-रु॒द्र उपा॑ऽगृह्णा॒थ् सा रौ॒द्री रोहि॑णी व॒शाऽभ॑वद्- बार्.हस्प॒त्याꣳ शि॑तिपृ॒ष्ठामा ल॑भेत ब्रह्मवर्च॒सका॑मो॒ बृह॒स्पति॑मे॒व स्वेन॑ भाग॒धेये॒नोप॑ धावति॒ स ए॒वास्मि॑न् ब्रह्मवर्च॒सं द॑धाति ब्रह्मवर्च॒स्ये॑व भ॑वति॒ छन्द॑सां॒ ॅवा ए॒ष रसो॒ यद्व॒शा रस॑ इव॒ खलु॒- [  ] \newline

\textbf{Pada Paata} \newline

अ॒गृ॒ह्णा॒त् । अस्तु॑ । ए॒व । अ॒यम् । भोगा॑य । इति॑ । सः । उ॒क्ष॒व॒श इत्यु॑क्ष - व॒शः । समिति॑ । अ॒भ॒व॒त् । यत् । लोहि॑तम् । प॒राप॑त॒दिति॑ परा - अप॑तत् । तत् । रु॒द्रः । उपेति॑ । अ॒गृ॒ह्णा॒त् । सा । रौ॒द्री । रोहि॑णी । व॒शा । अ॒भ॒व॒त् । बा॒र्.॒ह॒स्प॒त्याम् । शि॒ति॒पृ॒ष्ठामिति॑ शिति - पृ॒ष्ठाम् । एति॑ । ल॒भे॒त॒ । ब्र॒ह्म॒व॒र्च॒सका॑म॒ इति॑ ब्रह्मवर्च॒स - का॒मः॒ । बृह॒स्पति᳚म् । ए॒व । स्वेन॑ । भा॒ग॒धेये॒नेति॑ भाग - धेये॑न । उपेति॑ । धा॒व॒ति॒ । सः । ए॒व । अ॒स्मि॒न्न् । ब्र॒ह्म॒व॒र्च॒समिति॑ ब्रह्म - व॒र्च॒सम् । द॒धा॒ति॒ । ब्र॒ह्म॒व॒र्च॒सीति॑ ब्रह्म - व॒र्च॒सी । ए॒व । भ॒व॒ति॒ । छन्द॑साम् । वै । ए॒षः । रसः॑ । यत् । व॒शा । रसः॑ । इ॒व॒ । खलु॑ ।  \newline


\textbf{Krama Paata} \newline

अ॒गृ॒ह्णा॒दस्तु॑ । अस्त्वे॒व । ए॒वायम् । अ॒यम् भोगा॑य । भोगा॒येति॑ । इति॒ सः । स उ॑क्षव॒शः । उ॒क्ष॒व॒शः सम् । उ॒क्ष॒व॒श इत्यु॑क्ष - व॒शः । सम॑भवत् । अ॒भ॒व॒द् यत् । यल् लोहि॑तम् । लोहि॑तम् प॒राप॑तत् । प॒राप॑त॒त् तत् । प॒राप॑त॒दिति॑ परा - अप॑तत् । तद् रु॒द्रः । रु॒द्र उप॑ । उपा॑गृह्णात् । अ॒गृ॒ह्णा॒थ् सा । सा रौ॒द्री । रौ॒द्री रोहि॑णी । रोहि॑णी व॒शा । व॒शाऽभ॑वत् । अ॒भ॒व॒द् बा॒र्॒.ह॒स्प॒त्याम् । बा॒र्॒.ह॒स्प॒त्याꣳ शि॑तिपृ॒ष्ठाम् । शि॒ति॒पृ॒ष्ठामा । शि॒ति॒पृ॒ष्ठामिति॑ शिति - पृ॒ष्ठाम् । आ ल॑भेत । ल॒भे॒त॒ ब्र॒ह्म॒व॒र्च॒सका॑मः । ब्र॒ह्म॒व॒र्च॒सका॑मो॒ बृह॒स्पति᳚म् । ब्र॒ह्म॒व॒र्च॒सका॑म॒ इति॑ ब्रह्मवर्च॒स - का॒मः॒ । बृह॒स्पति॑मे॒व । ए॒व स्वेन॑ । स्वेन॑ भाग॒धेये॑न । भा॒ग॒धेये॒नोप॑ । भा॒ग॒धेये॒नेति॑ भाग - धेये॑न । उप॑ धावति । धा॒व॒ति॒ सः । स ए॒व । ए॒वास्मिन्न्॑ । अ॒स्मि॒न् ब्र॒ह्म॒व॒र्च॒सम् । ब्र॒ह्म॒व॒र्च॒सम् द॑धाति । ब्र॒ह्म॒व॒र्च॒समिति॑ ब्रह्म - व॒र्च॒सम् । द॒धा॒ति॒ ब्र॒ह्म॒व॒र्च॒सी । ब्र॒ह्म॒व॒र्च॒स्ये॑व । ब्र॒ह्म॒व॒र्च॒सीति॑ ब्रह्म - व॒र्च॒सी । ए॒व भ॑वति । भ॒व॒ति॒ छन्द॑साम् । छन्द॑सां॒ ॅवै । वा ए॒षः । ए॒ष रसः॑ । रसो॒ यत् । यद् व॒शा । व॒शा रसः॑ । रस॑ इव । इ॒व॒ खलु॑ । खलु॒ वै \newline

\textbf{Jatai Paata} \newline

1. अ॒गृ॒ह्णा॒ दस्त्वस्त्व॑ गृह्णा दगृह्णा॒ दस्तु॑ । \newline
2. अस्त्वे॒ वैवा स्त्व स्त्वे॒व । \newline
3. ए॒वाय म॒य मे॒वैवायम् । \newline
4. अ॒यम् भोगा॑य॒ भोगा॑या॒य म॒यम् भोगा॑य । \newline
5. भोगा॒ये तीति॒ भोगा॑य॒ भोगा॒ये ति॑ । \newline
6. इति॒ स स इतीति॒ सः । \newline
7. स उ॑क्षव॒श उ॑क्षव॒शः स स उ॑क्षव॒शः । \newline
8. उ॒क्ष॒व॒शः सꣳ स मु॑क्षव॒श उ॑क्षव॒शः सम् । \newline
9. उ॒क्ष॒व॒श इत्यु॑क्ष - व॒शः । \newline
10. स म॑भव दभव॒थ् सꣳ स म॑भवत् । \newline
11. अ॒भ॒व॒द् यद् यद॑भव दभव॒द् यत् । \newline
12. यल्लोहि॑त॒म् ॅलोहि॑तं॒ ॅयद् यल्लोहि॑तम् । \newline
13. लोहि॑तम् प॒राप॑तत् प॒राप॑त॒ ल्लोहि॑त॒म् ॅलोहि॑तम् प॒राप॑तत् । \newline
14. प॒राप॑त॒त् तत् तत् प॒राप॑तत् प॒राप॑त॒त् तत् । \newline
15. प॒राप॑त॒दिति॑ परा - अप॑तत् । \newline
16. तद् रु॒द्रो रु॒द्र स्तत् तद् रु॒द्रः । \newline
17. रु॒द्र उपोप॑ रु॒द्रो रु॒द्र उप॑ । \newline
18. उपा॑गृह्णा दगृह्णा॒ दुपोपा॑गृह्णात् । \newline
19. अ॒गृ॒ह्णा॒थ् सा सा ऽगृ॑ह्णा दगृह्णा॒थ् सा । \newline
20. सा रौ॒द्री रौ॒द्री सा सा रौ॒द्री । \newline
21. रौ॒द्री रोहि॑णी॒ रोहि॑णी रौ॒द्री रौ॒द्री रोहि॑णी । \newline
22. रोहि॑णी व॒शा व॒शा रोहि॑णी॒ रोहि॑णी व॒शा । \newline
23. व॒शा ऽभ॑व दभवद् व॒शा व॒शा ऽभ॑वत् । \newline
24. अ॒भ॒व॒द् बा॒र्॒.ह॒स्प॒त्याम् बा॑र्.हस्प॒त्या म॑भवदभवद् बार्.हस्प॒त्याम् । \newline
25. बा॒र्॒.ह॒स्प॒त्याꣳ शि॑तिपृ॒ष्ठाꣳ शि॑तिपृ॒ष्ठाम् बा॑र्.हस्प॒त्याम् बा॑र्.हस्प॒त्याꣳ शि॑तिपृ॒ष्ठाम् । \newline
26. शि॒ति॒पृ॒ष्ठा मा शि॑तिपृ॒ष्ठाꣳ शि॑तिपृ॒ष्ठा मा । \newline
27. शि॒ति॒पृ॒ष्ठामिति॑ शिति - पृ॒ष्ठाम् । \newline
28. आ ल॑भेत लभे॒ता ल॑भेत । \newline
29. ल॒भे॒त॒ ब्र॒ह्म॒व॒र्च॒सका॑मो ब्रह्मवर्च॒सका॑मो लभेत लभेत ब्रह्मवर्च॒सका॑मः । \newline
30. ब्र॒ह्म॒व॒र्च॒सका॑मो॒ बृह॒स्पति॒म् बृह॒स्पति॑म् ब्रह्मवर्च॒सका॑मो ब्रह्मवर्च॒सका॑मो॒ बृह॒स्पति᳚म् । \newline
31. ब्र॒ह्म॒व॒र्च॒सका॑म॒ इति॑ ब्रह्मवर्च॒स - का॒मः॒ । \newline
32. बृह॒स्पति॑ मे॒वैव बृह॒स्पति॒म् बृह॒स्पति॑ मे॒व । \newline
33. ए॒व स्वेन॒ स्वेनै॒वैव स्वेन॑ । \newline
34. स्वेन॑ भाग॒धेये॑न भाग॒धेये॑न॒ स्वेन॒ स्वेन॑ भाग॒धेये॑न । \newline
35. भा॒ग॒धेये॒नोपोप॑ भाग॒धेये॑न भाग॒धेये॒नोप॑ । \newline
36. भा॒ग॒धेये॒नेति॑ भाग - धेये॑न । \newline
37. उप॑ धावति धाव॒ त्युपोप॑ धावति । \newline
38. धा॒व॒ति॒ स स धा॑वति धावति॒ सः । \newline
39. स ए॒वैव स स ए॒व । \newline
40. ए॒वास्मि॑न् नस्मिन् ने॒वैवास्मिन्न्॑ । \newline
41. अ॒स्मि॒न् ब्र॒ह्म॒व॒र्च॒सम् ब्र॑ह्मवर्च॒स म॑स्मिन् नस्मिन् ब्रह्मवर्च॒सम् । \newline
42. ब्र॒ह्म॒व॒र्च॒सम् द॑धाति दधाति ब्रह्मवर्च॒सम् ब्र॑ह्मवर्च॒सम् द॑धाति । \newline
43. ब्र॒ह्म॒व॒र्च॒समिति॑ ब्रह्म - व॒र्च॒सम् । \newline
44. द॒धा॒ति॒ ब्र॒ह्म॒व॒र्च॒सी ब्र॑ह्मवर्च॒सी द॑धाति दधाति ब्रह्मवर्च॒सी । \newline
45. ब्र॒ह्म॒व॒र्च॒स्ये॑वैव ब्र॑ह्मवर्च॒सी ब्र॑ह्मवर्च॒स्ये॑व । \newline
46. ब्र॒ह्म॒व॒र्च॒सीति॑ ब्रह्म - व॒र्च॒सी । \newline
47. ए॒व भ॑वति भव त्ये॒वैव भ॑वति । \newline
48. भ॒व॒ति॒ छन्द॑सा॒म् छन्द॑साम् भवति भवति॒ छन्द॑साम् । \newline
49. छन्द॑सां॒ ॅवै वै छन्द॑सा॒म् छन्द॑सां॒ ॅवै । \newline
50. वा ए॒ष ए॒ष वै वा ए॒षः । \newline
51. ए॒ष रसो॒ रस॑ ए॒ष ए॒ष रसः॑ । \newline
52. रसो॒ यद् यद् रसो॒ रसो॒ यत् । \newline
53. यद् व॒शा व॒शा यद् यद् व॒शा । \newline
54. व॒शा रसो॒ रसो॑ व॒शा व॒शा रसः॑ । \newline
55. रस॑ इवे व॒ रसो॒ रस॑ इव । \newline
56. इ॒व॒ खलु॒ खल्वि॑वे व॒ खलु॑ । \newline
57. खलु॒ वै वै खलु॒ खलु॒ वै । \newline

\textbf{Ghana Paata } \newline

1. अ॒गृ॒ह्णा॒ दस्त्व स्त्व॑ गृह्णा दगृह्णा॒द स्त्वे॒वैवा स्त्व॑गृह्णा दगृह्णा॒ दस्त्वे॒व । \newline
2. अस्त्वे॒ वैवा स्त्व स्त्वे॒ वाय म॒य मे॒वा स्त्व स्त्वे॒ वायम् । \newline
3. ए॒वाय म॒य मे॒वैवायम् भोगा॑य॒ भोगा॑या॒य मे॒वैवायम् भोगा॑य । \newline
4. अ॒यम् भोगा॑य॒ भोगा॑या॒य म॒यम् भोगा॒ये तीति॒ भोगा॑या॒य म॒यम् भोगा॒ये ति॑ । \newline
5. भोगा॒ये तीति॒ भोगा॑य॒ भोगा॒ये ति॒ स स इति॒ भोगा॑य॒ भोगा॒ये ति॒ सः । \newline
6. इति॒ स स इतीति॒ स उ॑क्षव॒श उ॑क्षव॒शः स इतीति॒ स उ॑क्षव॒शः । \newline
7. स उ॑क्षव॒श उ॑क्षव॒शः स स उ॑क्षव॒शः सꣳ स मु॑क्षव॒शः स स उ॑क्षव॒शः सम् । \newline
8. उ॒क्ष॒व॒शः सꣳ स मु॑क्षव॒श उ॑क्षव॒शः स म॑भव दभव॒थ् स मु॑क्षव॒श उ॑क्षव॒शः स म॑भवत् । \newline
9. उ॒क्ष॒व॒श इत्यु॑क्ष - व॒शः । \newline
10. स म॑भव दभव॒थ् सꣳ स म॑भव॒द् यद् यद॑भव॒थ् सꣳ स म॑भव॒द् यत् । \newline
11. अ॒भ॒व॒द् यद् यद॑भव दभव॒द् य ल्लोहि॑त॒म् ॅलोहि॑तं॒ ॅयद॑भव दभव॒द् य ल्लोहि॑तम् । \newline
12. यल्लोहि॑त॒म् ॅलोहि॑तं॒ ॅयद् यल्लोहि॑तम् प॒राप॑तत् प॒राप॑त॒ ल्लोहि॑तं॒ ॅयद् यल्लोहि॑तम् प॒राप॑तत् । \newline
13. लोहि॑तम् प॒राप॑तत् प॒राप॑त॒ ल्लोहि॑त॒म् ॅलोहि॑तम् प॒राप॑त॒त् तत् तत् प॒राप॑त॒ ल्लोहि॑त॒म् ॅलोहि॑तम् प॒राप॑त॒त् तत् । \newline
14. प॒राप॑त॒त् तत् तत् प॒राप॑तत् प॒राप॑त॒त् तद् रु॒द्रो रु॒द्र स्तत् प॒राप॑तत् प॒राप॑त॒त् तद् रु॒द्रः । \newline
15. प॒राप॑त॒दिति॑ परा - अप॑तत् । \newline
16. तद् रु॒द्रो रु॒द्र स्तत् तद् रु॒द्र उपोप॑ रु॒द्र स्तत् तद् रु॒द्र उप॑ । \newline
17. रु॒द्र उपोप॑ रु॒द्रो रु॒द्र उपा॑गृह्णा दगृह्णा॒ दुप॑ रु॒द्रो रु॒द्र उपा॑गृह्णात् । \newline
18. उपा॑गृह्णा दगृह्णा॒ दुपोपा॑ गृह्णा॒थ् सा सा ऽगृ॑ह्णा॒ दुपोपा॑ गृह्णा॒थ् सा । \newline
19. अ॒गृ॒ह्णा॒थ् सा सा ऽगृ॑ह्णा दगृह्णा॒थ् सा रौ॒द्री रौ॒द्री सा ऽगृ॑ह्णा दगृह्णा॒थ् सा रौ॒द्री । \newline
20. सा रौ॒द्री रौ॒द्री सा सा रौ॒द्री रोहि॑णी॒ रोहि॑णी रौ॒द्री सा सा रौ॒द्री रोहि॑णी । \newline
21. रौ॒द्री रोहि॑णी॒ रोहि॑णी रौ॒द्री रौ॒द्री रोहि॑णी व॒शा व॒शा रोहि॑णी रौ॒द्री रौ॒द्री रोहि॑णी व॒शा । \newline
22. रोहि॑णी व॒शा व॒शा रोहि॑णी॒ रोहि॑णी व॒शा ऽभ॑व दभवद् व॒शा रोहि॑णी॒ रोहि॑णी व॒शा ऽभ॑वत् । \newline
23. व॒शा ऽभ॑व दभवद् व॒शा व॒शा ऽभ॑वद् बार्.हस्प॒त्याम् बा॑र्.हस्प॒त्या म॑भवद् व॒शा व॒शा ऽभ॑वद् बार्.हस्प॒त्याम् । \newline
24. अ॒भ॒व॒द् बा॒र्॒.ह॒स्प॒त्याम् बा॑र्.हस्प॒त्या म॑भवदभवद् बार्.हस्प॒त्याꣳ शि॑तिपृ॒ष्ठाꣳ शि॑तिपृ॒ष्ठाम् बा॑र्.हस्प॒त्या म॑भवदभवद् बार्.हस्प॒त्याꣳ शि॑तिपृ॒ष्ठाम् । \newline
25. बा॒र्॒.ह॒स्प॒त्याꣳ शि॑तिपृ॒ष्ठाꣳ शि॑तिपृ॒ष्ठाम् बा॑र्.हस्प॒त्याम् बा॑र्.हस्प॒त्याꣳ शि॑तिपृ॒ष्ठा मा शि॑तिपृ॒ष्ठाम् बा॑र्.हस्प॒त्याम् बा॑र्.हस्प॒त्याꣳ शि॑तिपृ॒ष्ठा मा । \newline
26. शि॒ति॒पृ॒ष्ठा मा शि॑तिपृ॒ष्ठाꣳ शि॑तिपृ॒ष्ठा मा ल॑भेत लभे॒ता शि॑तिपृ॒ष्ठाꣳ शि॑तिपृ॒ष्ठा मा ल॑भेत । \newline
27. शि॒ति॒पृ॒ष्ठामिति॑ शिति - पृ॒ष्ठाम् । \newline
28. आ ल॑भेत लभे॒ता ल॑भेत ब्रह्मवर्च॒सका॑मो ब्रह्मवर्च॒सका॑मो लभे॒ता ल॑भेत ब्रह्मवर्च॒सका॑मः । \newline
29. ल॒भे॒त॒ ब्र॒ह्म॒व॒र्च॒सका॑मो ब्रह्मवर्च॒सका॑मो लभेत लभेत ब्रह्मवर्च॒सका॑मो॒ बृह॒स्पति॒म् बृह॒स्पति॑म् ब्रह्मवर्च॒सका॑मो लभेत लभेत ब्रह्मवर्च॒सका॑मो॒ बृह॒स्पति᳚म् । \newline
30. ब्र॒ह्म॒व॒र्च॒सका॑मो॒ बृह॒स्पति॒म् बृह॒स्पति॑म् ब्रह्मवर्च॒सका॑मो ब्रह्मवर्च॒सका॑मो॒ बृह॒स्पति॑ मे॒वैव बृह॒स्पति॑म् ब्रह्मवर्च॒सका॑मो ब्रह्मवर्च॒सका॑मो॒ बृह॒स्पति॑ मे॒व । \newline
31. ब्र॒ह्म॒व॒र्च॒सका॑म॒ इति॑ ब्रह्मवर्च॒स - का॒मः॒ । \newline
32. बृह॒स्पति॑ मे॒वैव बृह॒स्पति॒म् बृह॒स्पति॑ मे॒व स्वेन॒ स्वेनै॒व बृह॒स्पति॒म् बृह॒स्पति॑ मे॒व स्वेन॑ । \newline
33. ए॒व स्वेन॒ स्वेनै॒वैव स्वेन॑ भाग॒धेये॑न भाग॒धेये॑न॒ स्वेनै॒वैव स्वेन॑ भाग॒धेये॑न । \newline
34. स्वेन॑ भाग॒धेये॑न भाग॒धेये॑न॒ स्वेन॒ स्वेन॑ भाग॒धेये॒नो पोप॑ भाग॒धेये॑न॒ स्वेन॒ स्वेन॑ भाग॒धेये॒नोप॑ । \newline
35. भा॒ग॒धेये॒नो पोप॑ भाग॒धेये॑न भाग॒धेये॒नोप॑ धावति धाव॒त्युप॑ भाग॒धेये॑न भाग॒धेये॒नोप॑ धावति । \newline
36. भा॒ग॒धेये॒नेति॑ भाग - धेये॑न । \newline
37. उप॑ धावति धाव॒ त्युपोप॑ धावति॒ स स धा॑व॒ त्युपोप॑ धावति॒ सः । \newline
38. धा॒व॒ति॒ स स धा॑वति धावति॒ स ए॒वैव स धा॑वति धावति॒ स ए॒व । \newline
39. स ए॒वैव स स ए॒वास्मि॑न् नस्मिन् ने॒व स स ए॒वास्मिन्न्॑ । \newline
40. ए॒वास्मि॑न् नस्मिन् ने॒वैवास्मि॑न् ब्रह्मवर्च॒सम् ब्र॑ह्मवर्च॒स म॑स्मिन् ने॒वैवास्मि॑न् ब्रह्मवर्च॒सम् । \newline
41. अ॒स्मि॒न् ब्र॒ह्म॒व॒र्च॒सम् ब्र॑ह्मवर्च॒स म॑स्मिन् नस्मिन् ब्रह्मवर्च॒सम् द॑धाति दधाति ब्रह्मवर्च॒स म॑स्मिन् नस्मिन् ब्रह्मवर्च॒सम् द॑धाति । \newline
42. ब्र॒ह्म॒व॒र्च॒सम् द॑धाति दधाति ब्रह्मवर्च॒सम् ब्र॑ह्मवर्च॒सम् द॑धाति ब्रह्मवर्च॒सी ब्र॑ह्मवर्च॒सी द॑धाति ब्रह्मवर्च॒सम् ब्र॑ह्मवर्च॒सम् द॑धाति ब्रह्मवर्च॒सी । \newline
43. ब्र॒ह्म॒व॒र्च॒समिति॑ ब्रह्म - व॒र्च॒सम् । \newline
44. द॒धा॒ति॒ ब्र॒ह्म॒व॒र्च॒सी ब्र॑ह्मवर्च॒सी द॑धाति दधाति ब्रह्मवर्च॒ स्ये॑वैव ब्र॑ह्मवर्च॒सी द॑धाति दधाति ब्रह्मवर्च॒ स्ये॑व । \newline
45. ब्र॒ह्म॒व॒र्च॒ स्ये॑वैव ब्र॑ह्मवर्च॒सी ब्र॑ह्मवर्च॒ स्ये॑व भ॑वति भवत्ये॒व ब्र॑ह्मवर्च॒सी ब्र॑ह्मवर्च॒ स्ये॑व भ॑वति । \newline
46. ब्र॒ह्म॒व॒र्च॒सीति॑ ब्रह्म - व॒र्च॒सी । \newline
47. ए॒व भ॑वति भवत्ये॒वैव भ॑वति॒ छन्द॑सा॒म् छन्द॑साम् भवत्ये॒वैव भ॑वति॒ छन्द॑साम् । \newline
48. भ॒व॒ति॒ छन्द॑सा॒म् छन्द॑साम् भवति भवति॒ छन्द॑सां॒ ॅवै वै छन्द॑साम् भवति भवति॒ छन्द॑सां॒ ॅवै । \newline
49. छन्द॑सां॒ ॅवै वै छन्द॑सा॒म् छन्द॑सां॒ ॅवा ए॒ष ए॒ष वै छन्द॑सा॒म् छन्द॑सां॒ ॅवा ए॒षः । \newline
50. वा ए॒ष ए॒ष वै वा ए॒ष रसो॒ रस॑ ए॒ष वै वा ए॒ष रसः॑ । \newline
51. ए॒ष रसो॒ रस॑ ए॒ष ए॒ष रसो॒ यद् यद् रस॑ ए॒ष ए॒ष रसो॒ यत् । \newline
52. रसो॒ यद् यद् रसो॒ रसो॒ यद् व॒शा व॒शा यद् रसो॒ रसो॒ यद् व॒शा । \newline
53. यद् व॒शा व॒शा यद् यद् व॒शा रसो॒ रसो॑ व॒शा यद् यद् व॒शा रसः॑ । \newline
54. व॒शा रसो॒ रसो॑ व॒शा व॒शा रस॑ इवे व॒ रसो॑ व॒शा व॒शा रस॑ इव । \newline
55. रस॑ इवे व॒ रसो॒ रस॑ इव॒ खलु॒ खल्वि॑व॒ रसो॒ रस॑ इव॒ खलु॑ । \newline
56. इ॒व॒ खलु॒ खल्वि॑वे व॒ खलु॒ वै वै खल्वि॑वे व॒ खलु॒ वै । \newline
57. खलु॒ वै वै खलु॒ खलु॒ वै ब्र॑ह्मवर्च॒सम् ब्र॑ह्मवर्च॒सं ॅवै खलु॒ खलु॒ वै ब्र॑ह्मवर्च॒सम् । \newline
\pagebreak
\markright{ TS 2.1.7.3  \hfill https://www.vedavms.in \hfill}

\section{ TS 2.1.7.3 }

\textbf{TS 2.1.7.3 } \newline
\textbf{Samhita Paata} \newline

वै ब्र॑ह्मवर्च॒सं छन्द॑सामे॒व रसे॑न॒ रसं॑ ब्रह्मवर्च॒समव॑ रुन्धे मैत्रावरु॒णीं द्वि॑रू॒पामा ल॑भेत॒ वृष्टि॑कामो मै॒त्रं ॅवा अह॑र्वारु॒णी रात्रि॑रहोरा॒त्राभ्यां॒ खलु॒ वै प॒र्जन्यो॑ वर्.षति मि॒त्रावरु॑णावे॒व स्वेन॑ भाग॒धेये॒नोप॑ धावति॒ तावे॒वास्मा॑ अहोरा॒त्राभ्यां᳚ प॒र्जन्यं॑ ॅवर्.षयतः॒ छन्द॑सां॒ ॅवा ए॒ष रसो॒ यद्व॒शा रस॑ इव॒ खलु॒ वै वृष्टिः॒ छन्द॑सामे॒व रसे॑न॒ - [  ] \newline

\textbf{Pada Paata} \newline

वै । ब्र॒ह्म॒व॒र्च॒समिति॑ ब्रह्म - व॒र्च॒सम् । छन्द॑साम् । ए॒व । रसे॑न । रस᳚म् । ब्र॒ह्म॒व॒र्च॒समिति॑ ब्रह्म - व॒र्च॒सम् । अवेति॑ । रु॒न्धे॒ । मै॒त्रा॒व॒रु॒णीमिति॑ मैत्रा - व॒रु॒णीम् । द्वि॒रू॒पामिति॑ द्वि - रू॒पाम् । एति॑ । ल॒भे॒त॒ । वृष्टि॑काम॒ इति॒ वृष्टि॑ - का॒मः॒ । मै॒त्रम् । वै । अहः॑ । वा॒रु॒णी । रात्रिः॑ । अ॒हो॒रा॒त्राभ्या॒मित्य॑हः - रा॒त्राभ्या᳚म् । खलु॑ । वै । प॒र्जन्यः॑ । व॒र्.॒ष॒ति॒ । मि॒त्रावरु॑णा॒विति॑ मि॒त्रा - वरु॑णौ । ए॒व । स्वेन॑ । भा॒ग॒धेये॒नेति॑ भाग - धेये॑न । उपेति॑ । धा॒व॒ति॒ । तौ । ए॒व । अ॒स्मै॒ । अ॒हो॒रा॒त्राभ्या॒मित्य॑हः - रा॒त्राभ्या᳚म् । प॒र्जन्य᳚म् । व॒र्.॒ष॒य॒तः॒ । छन्द॑साम् । वै । ए॒षः । रसः॑ । यत् । व॒शा । रसः॑ । इ॒व॒ । खलु॑ । वै । वृष्टिः॑ । छन्द॑साम् । ए॒व । रसे॑न ।  \newline


\textbf{Krama Paata} \newline

वै ब्र॑ह्मवर्च॒सम् । ब्र॒ह्म॒व॒र्च॒सम् छन्द॑साम् । ब्र॒ह्म॒व॒र्च॒समिति॑ ब्रह्म - व॒र्च॒सम् । छन्द॑सामे॒व । ए॒व रसे॑न । रसे॑न॒ रस᳚म् । रस॑म् ब्रह्मवर्च॒सम् । ब्र॒ह्म॒व॒र्च॒समव॑ । ब्र॒ह्म॒व॒र्च॒समिति॑ ब्रह्म - व॒र्च॒सम् । अव॑ रुन्धे । रु॒न्धे॒ मै॒त्रा॒व॒रु॒णीम् । मै॒त्रा॒व॒रु॒णीम् द्वि॑रू॒पाम् । मै॒त्रा॒व॒रु॒णीमिति॑ मैत्रा - व॒रु॒णीम् । द्वि॒रू॒पामा । द्वि॒रू॒पामिति॑ द्वि - रू॒पाम् । आ ल॑भेत । ल॒भे॒त॒ वृष्टि॑कामः । वृष्टि॑कामो मै॒त्रम् । वृष्टि॑काम॒ इति॒ वृष्टि॑ - का॒मः॒ । मै॒त्रं ॅवै । वा अहः॑ । अह॑र् वारु॒णी । वा॒रु॒णी रात्रिः॑ । रात्रि॑रहोरा॒त्राभ्या᳚म् । अ॒हो॒रा॒त्राभ्या॒म् खलु॑ । अ॒हो॒रा॒त्राभ्या॒मित्य॑हः - रा॒त्राभ्या᳚म् । खलु॒ वै । वै प॒र्जन्यः॑ । प॒र्जन्यो॑ वर्.षति । व॒र्॒.ष॒ति॒ मि॒त्रावरु॑णौ । मि॒त्रावरु॑णावे॒व । मि॒त्रावरु॑णा॒विति॑ मि॒त्रा - वरु॑णौ । ए॒व स्वेन॑ । स्वेन॑ भाग॒धेये॑न । भा॒ग॒धेये॒नोप॑ । भा॒ग॒धेये॒नेति॑ भाग - धेये॑न । उप॑ धावति । धा॒व॒ति॒ तौ । तावे॒व । ए॒वास्मै᳚ । अ॒स्मा॒ अ॒हो॒रा॒त्राभ्या᳚म् । अ॒हो॒रा॒त्राभ्या᳚म् प॒र्जन्य᳚म् । अ॒हो॒रा॒त्राभ्या॒मित्य॑हः - रा॒त्राभ्या᳚म् । प॒र्जन्यं॑ ॅवर्.षयतः । व॒र्॒.ष॒य॒त॒ श्छन्द॑साम् । छन्द॑सां॒ ॅवै । वा ए॒षः । ए॒ष रसः॑ । रसो॒ यत् । यद् व॒शा । व॒शा रसः॑ । रस॑ इव । इ॒व॒ खलु॑ । खलु॒ वै । वै वृष्टिः॑ । वृष्टि॒ श्छन्द॑साम् । छन्द॑सामे॒व । ए॒व रसे॑न । रसे॑न॒ रस᳚म् \newline

\textbf{Jatai Paata} \newline

1. वै ब्र॑ह्मवर्च॒सम् ब्र॑ह्मवर्च॒सं ॅवै वै ब्र॑ह्मवर्च॒सम् । \newline
2. ब्र॒ह्म॒व॒र्च॒सम् छन्द॑सा॒म् छन्द॑साम् ब्रह्मवर्च॒सम् ब्र॑ह्मवर्च॒सम् छन्द॑साम् । \newline
3. ब्र॒ह्म॒व॒र्च॒समिति॑ ब्रह्म - व॒र्च॒सम् । \newline
4. छन्द॑सा मे॒वैव छन्द॑सा॒म् छन्द॑सा मे॒व । \newline
5. ए॒व रसे॑न॒ रसे॑ नै॒वैव रसे॑न । \newline
6. रसे॑न॒ रसꣳ॒॒ रसꣳ॒॒ रसे॑न॒ रसे॑न॒ रस᳚म् । \newline
7. रस॑म् ब्रह्मवर्च॒सम् ब्र॑ह्मवर्च॒सꣳ रसꣳ॒॒ रस॑म् ब्रह्मवर्च॒सम् । \newline
8. ब्र॒ह्म॒व॒र्च॒स मवाव॑ ब्रह्मवर्च॒सम् ब्र॑ह्मवर्च॒स मव॑ । \newline
9. ब्र॒ह्म॒व॒र्च॒समिति॑ ब्रह्म - व॒र्च॒सम् । \newline
10. अव॑ रुन्धे रु॒न्धे ऽवाव॑ रुन्धे । \newline
11. रु॒न्धे॒ मै॒त्रा॒व॒रु॒णीम् मै᳚त्रावरु॒णीꣳ रु॑न्धे रुन्धे मैत्रावरु॒णीम् । \newline
12. मै॒त्रा॒व॒रु॒णीम् द्वि॑रू॒पाम् द्वि॑रू॒पाम् मै᳚त्रावरु॒णीम् मै᳚त्रावरु॒णीम् द्वि॑रू॒पाम् । \newline
13. मै॒त्रा॒व॒रु॒णीमिति॑ मैत्रा - व॒रु॒णीम् । \newline
14. द्वि॒रू॒पा मा द्वि॑रू॒पाम् द्वि॑रू॒पा मा । \newline
15. द्वि॒रू॒पामिति॑ द्वि - रू॒पाम् । \newline
16. आ ल॑भेत लभे॒ता ल॑भेत । \newline
17. ल॒भे॒त॒ वृष्टि॑कामो॒ वृष्टि॑कामो लभेत लभेत॒ वृष्टि॑कामः । \newline
18. वृष्टि॑कामो मै॒त्रम् मै॒त्रं ॅवृष्टि॑कामो॒ वृष्टि॑कामो मै॒त्रम् । \newline
19. वृष्टि॑काम॒ इति॒ वृष्टि॑ - का॒मः॒ । \newline
20. मै॒त्रं ॅवै वै मै॒त्रम् मै॒त्रं ॅवै । \newline
21. वा अह॒ रह॒र् वै वा अहः॑ । \newline
22. अह॑र् वारु॒णी वा॑रु॒ण्यह॒ रह॑र् वारु॒णी । \newline
23. वा॒रु॒णी रात्री॒ रात्रि॑र् वारु॒णी वा॑रु॒णी रात्रिः॑ । \newline
24. रात्रि॑रहोरा॒त्राभ्या॑ महोरा॒त्राभ्याꣳ॒॒ रात्री॒ रात्रि॑रहोरा॒त्राभ्या᳚म् । \newline
25. अ॒हो॒रा॒त्राभ्या॒म् खलु॒ खल्व॑होरा॒त्राभ्या॑ महोरा॒त्राभ्या॒म् खलु॑ । \newline
26. अ॒हो॒रा॒त्राभ्या॒मित्य॑हः - रा॒त्राभ्या᳚म् । \newline
27. खलु॒ वै वै खलु॒ खलु॒ वै । \newline
28. वै प॒र्जन्यः॑ प॒र्जन्यो॒ वै वै प॒र्जन्यः॑ । \newline
29. प॒र्जन्यो॑ वर्.षति वर्.षति प॒र्जन्यः॑ प॒र्जन्यो॑ वर्.षति । \newline
30. व॒र्॒.ष॒ति॒ मि॒त्रावरु॑णौ मि॒त्रावरु॑णौ वर्.षति वर्.षति मि॒त्रावरु॑णौ । \newline
31. मि॒त्रावरु॑णा वे॒वैव मि॒त्रावरु॑णौ मि॒त्रावरु॑णा वे॒व । \newline
32. मि॒त्रावरु॑णा॒विति॑ मि॒त्रा - वरु॑णौ । \newline
33. ए॒व स्वेन॒ स्वेनै॒वैव स्वेन॑ । \newline
34. स्वेन॑ भाग॒धेये॑न भाग॒धेये॑न॒ स्वेन॒ स्वेन॑ भाग॒धेये॑न । \newline
35. भा॒ग॒धेये॒नोपोप॑ भाग॒धेये॑न भाग॒धेये॒नोप॑ । \newline
36. भा॒ग॒धेये॒नेति॑ भाग - धेये॑न । \newline
37. उप॑ धावति धाव॒ त्युपोप॑ धावति । \newline
38. धा॒व॒ति॒ तौ तौ धा॑वति धावति॒ तौ । \newline
39. ता वे॒वैव तौ ता वे॒व । \newline
40. ए॒वास्मा॑ अस्मा ए॒वैवास्मै᳚ । \newline
41. अ॒स्मा॒ अ॒हो॒रा॒त्राभ्या॑ महोरा॒त्राभ्या॑ मस्मा अस्मा अहोरा॒त्राभ्या᳚म् । \newline
42. अ॒हो॒रा॒त्राभ्या᳚म् प॒र्जन्य॑म् प॒र्जन्य॑ महोरा॒त्राभ्या॑ महोरा॒त्राभ्या᳚म् प॒र्जन्य᳚म् । \newline
43. अ॒हो॒रा॒त्राभ्या॒मित्य॑हः - रा॒त्राभ्या᳚म् । \newline
44. प॒र्जन्यं॑ ॅवर्.षयतो वर्.षयतः प॒र्जन्य॑म् प॒र्जन्यं॑ ॅवर्.षयतः । \newline
45. व॒र्॒.ष॒य॒त॒ श्छन्द॑सा॒म् छन्द॑सां ॅवर्.षयतो वर्.षयत॒ श्छन्द॑साम् । \newline
46. छन्द॑सां॒ ॅवै वै छन्द॑सा॒म् छन्द॑सां॒ ॅवै । \newline
47. वा ए॒ष ए॒ष वै वा ए॒षः । \newline
48. ए॒ष रसो॒ रस॑ ए॒ष ए॒ष रसः॑ । \newline
49. रसो॒ यद् यद् रसो॒ रसो॒ यत् । \newline
50. यद् व॒शा व॒शा यद् यद् व॒शा । \newline
51. व॒शा रसो॒ रसो॑ व॒शा व॒शा रसः॑ । \newline
52. रस॑ इवे व॒ रसो॒ रस॑ इव । \newline
53. इ॒व॒ खलु॒ खल्वि॑वे व॒ खलु॑ । \newline
54. खलु॒ वै वै खलु॒ खलु॒ वै । \newline
55. वै वृष्टि॒र् वृष्टि॒र् वै वै वृष्टिः॑ । \newline
56. वृष्टि॒ श्छन्द॑सा॒म् छन्द॑सां॒ ॅवृष्टि॒र् वृष्टि॒ श्छन्द॑साम् । \newline
57. छन्द॑सा मे॒वैव छन्द॑सा॒म् छन्द॑सा मे॒व । \newline
58. ए॒व रसे॑न॒ रसे॑नै॒वैव रसे॑न । \newline
59. रसे॑न॒ रसꣳ॒॒ रसꣳ॒॒ रसे॑न॒ रसे॑न॒ रस᳚म् । \newline

\textbf{Ghana Paata } \newline

1. वै ब्र॑ह्मवर्च॒सम् ब्र॑ह्मवर्च॒सं ॅवै वै ब्र॑ह्मवर्च॒सम् छन्द॑सा॒म् छन्द॑साम् ब्रह्मवर्च॒सं ॅवै वै ब्र॑ह्मवर्च॒सम् छन्द॑साम् । \newline
2. ब्र॒ह्म॒व॒र्च॒सम् छन्द॑सा॒म् छन्द॑साम् ब्रह्मवर्च॒सम् ब्र॑ह्मवर्च॒सम् छन्द॑सा मे॒वैव छन्द॑साम् ब्रह्मवर्च॒सम् ब्र॑ह्मवर्च॒सम् छन्द॑सा मे॒व । \newline
3. ब्र॒ह्म॒व॒र्च॒समिति॑ ब्रह्म - व॒र्च॒सम् । \newline
4. छन्द॑सा मे॒वैव छन्द॑सा॒म् छन्द॑सा मे॒व रसे॑न॒ रसे॑नै॒व छन्द॑सा॒म् छन्द॑सा मे॒व रसे॑न । \newline
5. ए॒व रसे॑न॒ रसे॑नै॒वैव रसे॑न॒ रसꣳ॒॒ रसꣳ॒॒ रसे॑नै॒वैव रसे॑न॒ रस᳚म् । \newline
6. रसे॑न॒ रसꣳ॒॒ रसꣳ॒॒ रसे॑न॒ रसे॑न॒ रस॑म् ब्रह्मवर्च॒सम् ब्र॑ह्मवर्च॒सꣳ रसꣳ॒॒ रसे॑न॒ रसे॑न॒ रस॑म् ब्रह्मवर्च॒सम् । \newline
7. रस॑म् ब्रह्मवर्च॒सम् ब्र॑ह्मवर्च॒सꣳ रसꣳ॒॒ रस॑म् ब्रह्मवर्च॒स मवाव॑ ब्रह्मवर्च॒सꣳ रसꣳ॒॒ रस॑म् ब्रह्मवर्च॒स मव॑ । \newline
8. ब्र॒ह्म॒व॒र्च॒स मवाव॑ ब्रह्मवर्च॒सम् ब्र॑ह्मवर्च॒स मव॑ रुन्धे रु॒न्धे ऽव॑ ब्रह्मवर्च॒सम् ब्र॑ह्मवर्च॒स मव॑ रुन्धे । \newline
9. ब्र॒ह्म॒व॒र्च॒समिति॑ ब्रह्म - व॒र्च॒सम् । \newline
10. अव॑ रुन्धे रु॒न्धे ऽवाव॑ रुन्धे मैत्रावरु॒णीम् मै᳚त्रावरु॒णीꣳ रु॒न्धे ऽवाव॑ रुन्धे मैत्रावरु॒णीम् । \newline
11. रु॒न्धे॒ मै॒त्रा॒व॒रु॒णीम् मै᳚त्रावरु॒णीꣳ रु॑न्धे रुन्धे मैत्रावरु॒णीम् द्वि॑रू॒पाम् द्वि॑रू॒पाम् मै᳚त्रावरु॒णीꣳ रु॑न्धे रुन्धे मैत्रावरु॒णीम् द्वि॑रू॒पाम् । \newline
12. मै॒त्रा॒व॒रु॒णीम् द्वि॑रू॒पाम् द्वि॑रू॒पाम् मै᳚त्रावरु॒णीम् मै᳚त्रावरु॒णीम् द्वि॑रू॒पा मा द्वि॑रू॒पाम् मै᳚त्रावरु॒णीम् मै᳚त्रावरु॒णीम् द्वि॑रू॒पा मा । \newline
13. मै॒त्रा॒व॒रु॒णीमिति॑ मैत्रा - व॒रु॒णीम् । \newline
14. द्वि॒रू॒पा मा द्वि॑रू॒पाम् द्वि॑रू॒पा मा ल॑भेत लभे॒ता द्वि॑रू॒पाम् द्वि॑रू॒पा मा ल॑भेत । \newline
15. द्वि॒रू॒पामिति॑ द्वि - रू॒पाम् । \newline
16. आ ल॑भेत लभे॒ता ल॑भेत॒ वृष्टि॑कामो॒ वृष्टि॑कामो लभे॒ता ल॑भेत॒ वृष्टि॑कामः । \newline
17. ल॒भे॒त॒ वृष्टि॑कामो॒ वृष्टि॑कामो लभेत लभेत॒ वृष्टि॑कामो मै॒त्रम् मै॒त्रं ॅवृष्टि॑कामो लभेत लभेत॒ वृष्टि॑कामो मै॒त्रम् । \newline
18. वृष्टि॑कामो मै॒त्रम् मै॒त्रं ॅवृष्टि॑कामो॒ वृष्टि॑कामो मै॒त्रं ॅवै वै मै॒त्रं ॅवृष्टि॑कामो॒ वृष्टि॑कामो मै॒त्रं ॅवै । \newline
19. वृष्टि॑काम॒ इति॒ वृष्टि॑ - का॒मः॒ । \newline
20. मै॒त्रं ॅवै वै मै॒त्रम् मै॒त्रं ॅवा अह॒ रह॒र् वै मै॒त्रम् मै॒त्रं ॅवा अहः॑ । \newline
21. वा अह॒ रह॒र् वै वा अह॑र् वारु॒णी वा॑रु॒ण्यह॒र् वै वा अह॑र् वारु॒णी । \newline
22. अह॑र् वारु॒णी वा॑रु॒ण्यह॒ रह॑र् वारु॒णी रात्री॒ रात्रि॑र् वारु॒ण्यह॒ रह॑र् वारु॒णी रात्रिः॑ । \newline
23. वा॒रु॒णी रात्री॒ रात्रि॑र् वारु॒णी वा॑रु॒णी रात्रि॑ रहोरा॒त्राभ्या॑ महोरा॒त्राभ्याꣳ॒॒ रात्रि॑र् वारु॒णी वा॑रु॒णी रात्रि॑ रहोरा॒त्राभ्या᳚म् । \newline
24. रात्रि॑ रहोरा॒त्राभ्या॑ महोरा॒त्राभ्याꣳ॒॒ रात्री॒ रात्रि॑ रहोरा॒त्राभ्या॒म् खलु॒ खल्व॑होरा॒त्राभ्याꣳ॒॒ रात्री॒ रात्रि॑ रहोरा॒त्राभ्या॒म् खलु॑ । \newline
25. अ॒हो॒रा॒त्राभ्या॒म् खलु॒ खल्व॑होरा॒त्राभ्या॑ महोरा॒त्राभ्या॒म् खलु॒ वै वै खल्व॑होरा॒त्राभ्या॑ महोरा॒त्राभ्या॒म् खलु॒ वै । \newline
26. अ॒हो॒रा॒त्राभ्या॒मित्य॑हः - रा॒त्राभ्या᳚म् । \newline
27. खलु॒ वै वै खलु॒ खलु॒ वै प॒र्जन्यः॑ प॒र्जन्यो॒ वै खलु॒ खलु॒ वै प॒र्जन्यः॑ । \newline
28. वै प॒र्जन्यः॑ प॒र्जन्यो॒ वै वै प॒र्जन्यो॑ वर्.षति वर्.षति प॒र्जन्यो॒ वै वै प॒र्जन्यो॑ वर्.षति । \newline
29. प॒र्जन्यो॑ वर्.षति वर्.षति प॒र्जन्यः॑ प॒र्जन्यो॑ वर्.षति मि॒त्रावरु॑णौ मि॒त्रावरु॑णौ वर्.षति प॒र्जन्यः॑ प॒र्जन्यो॑ वर्.षति मि॒त्रावरु॑णौ । \newline
30. व॒र्॒.ष॒ति॒ मि॒त्रावरु॑णौ मि॒त्रावरु॑णौ वर्.षति वर्.षति मि॒त्रावरु॑णा वे॒वैव मि॒त्रावरु॑णौ वर्.षति वर्.षति मि॒त्रावरु॑णा वे॒व । \newline
31. मि॒त्रावरु॑णा वे॒वैव मि॒त्रावरु॑णौ मि॒त्रावरु॑णा वे॒व स्वेन॒ स्वेनै॒व मि॒त्रावरु॑णौ मि॒त्रावरु॑णा वे॒व स्वेन॑ । \newline
32. मि॒त्रावरु॑णा॒विति॑ मि॒त्रा - वरु॑णौ । \newline
33. ए॒व स्वेन॒ स्वेनै॒वैव स्वेन॑ भाग॒धेये॑न भाग॒धेये॑न॒ स्वेनै॒वैव स्वेन॑ भाग॒धेये॑न । \newline
34. स्वेन॑ भाग॒धेये॑न भाग॒धेये॑न॒ स्वेन॒ स्वेन॑ भाग॒धेये॒नो पोप॑ भाग॒धेये॑न॒ स्वेन॒ स्वेन॑ भाग॒धेये॒नोप॑ । \newline
35. भा॒ग॒धेये॒नो पोप॑ भाग॒धेये॑न भाग॒धेये॒नोप॑ धावति धाव॒त्युप॑ भाग॒धेये॑न भाग॒धेये॒नोप॑ धावति । \newline
36. भा॒ग॒धेये॒नेति॑ भाग - धेये॑न । \newline
37. उप॑ धावति धाव॒ त्युपोप॑ धावति॒ तौ तौ धा॑व॒ त्युपोप॑ धावति॒ तौ । \newline
38. धा॒व॒ति॒ तौ तौ धा॑वति धावति॒ ता वे॒वैव तौ धा॑वति धावति॒ ता वे॒व । \newline
39. ता वे॒वैव तौ ता वे॒वास्मा॑ अस्मा ए॒व तौ ता वे॒वास्मै᳚ । \newline
40. ए॒वास्मा॑ अस्मा ए॒वैवास्मा॑ अहोरा॒त्राभ्या॑ महोरा॒त्राभ्या॑ मस्मा ए॒वैवास्मा॑ अहोरा॒त्राभ्या᳚म् । \newline
41. अ॒स्मा॒ अ॒हो॒रा॒त्राभ्या॑ महोरा॒त्राभ्या॑ मस्मा अस्मा अहोरा॒त्राभ्या᳚म् प॒र्जन्य॑म् प॒र्जन्य॑ महोरा॒त्राभ्या॑ मस्मा अस्मा अहोरा॒त्राभ्या᳚म् प॒र्जन्य᳚म् । \newline
42. अ॒हो॒रा॒त्राभ्या᳚म् प॒र्जन्य॑म् प॒र्जन्य॑ महोरा॒त्राभ्या॑ महोरा॒त्राभ्या᳚म् प॒र्जन्यं॑ ॅवर्.षयतो वर्.षयतः प॒र्जन्य॑ महोरा॒त्राभ्या॑ महोरा॒त्राभ्या᳚म् प॒र्जन्यं॑ ॅवर्.षयतः । \newline
43. अ॒हो॒रा॒त्राभ्या॒मित्य॑हः - रा॒त्राभ्या᳚म् । \newline
44. प॒र्जन्यं॑ ॅवर्.षयतो वर्.षयतः प॒र्जन्य॑म् प॒र्जन्यं॑ ॅवर्.षयत॒ श्छन्द॑सा॒म् छन्द॑सां ॅवर्.षयतः प॒र्जन्य॑म् प॒र्जन्यं॑ ॅवर्.षयत॒ श्छन्द॑साम् । \newline
45. व॒र्॒.ष॒य॒त॒ श्छन्द॑सा॒म् छन्द॑सां ॅवर्.षयतो वर्.षयत॒ श्छन्द॑सां॒ ॅवै वै छन्द॑सां ॅवर्.षयतो वर्.षयत॒ श्छन्द॑सां॒ ॅवै । \newline
46. छन्द॑सां॒ ॅवै वै छन्द॑सा॒म् छन्द॑सां॒ ॅवा ए॒ष ए॒ष वै छन्द॑सा॒म् छन्द॑सां॒ ॅवा ए॒षः । \newline
47. वा ए॒ष ए॒ष वै वा ए॒ष रसो॒ रस॑ ए॒ष वै वा ए॒ष रसः॑ । \newline
48. ए॒ष रसो॒ रस॑ ए॒ष ए॒ष रसो॒ यद् यद् रस॑ ए॒ष ए॒ष रसो॒ यत् । \newline
49. रसो॒ यद् यद् रसो॒ रसो॒ यद् व॒शा व॒शा यद् रसो॒ रसो॒ यद् व॒शा । \newline
50. यद् व॒शा व॒शा यद् यद् व॒शा रसो॒ रसो॑ व॒शा यद् यद् व॒शा रसः॑ । \newline
51. व॒शा रसो॒ रसो॑ व॒शा व॒शा रस॑ इवे व॒ रसो॑ व॒शा व॒शा रस॑ इव । \newline
52. रस॑ इवे व॒ रसो॒ रस॑ इव॒ खलु॒ खल्वि॑व॒ रसो॒ रस॑ इव॒ खलु॑ । \newline
53. इ॒व॒ खलु॒ खल्वि॑वे व॒ खलु॒ वै वै खल्वि॑वे व॒ खलु॒ वै । \newline
54. खलु॒ वै वै खलु॒ खलु॒ वै वृष्टि॒र् वृष्टि॒र् वै खलु॒ खलु॒ वै वृष्टिः॑ । \newline
55. वै वृष्टि॒र् वृष्टि॒र् वै वै वृष्टि॒ श्छन्द॑सा॒म् छन्द॑सां॒ ॅवृष्टि॒र् वै वै वृष्टि॒ श्छन्द॑साम् । \newline
56. वृष्टि॒ श्छन्द॑सा॒म् छन्द॑सां॒ ॅवृष्टि॒र् वृष्टि॒ श्छन्द॑सा मे॒वैव छन्द॑सां॒ ॅवृष्टि॒र् वृष्टि॒ श्छन्द॑सा मे॒व । \newline
57. छन्द॑सा मे॒वैव छन्द॑सा॒म् छन्द॑सा मे॒व रसे॑न॒ रसे॑नै॒व छन्द॑सा॒म् छन्द॑सा मे॒व रसे॑न । \newline
58. ए॒व रसे॑न॒ रसे॑नै॒वैव रसे॑न॒ रसꣳ॒॒ रसꣳ॒॒ रसे॑नै॒वैव रसे॑न॒ रस᳚म् । \newline
59. रसे॑न॒ रसꣳ॒॒ रसꣳ॒॒ रसे॑न॒ रसे॑न॒ रसं॒ ॅवृष्टिं॒ ॅवृष्टिꣳ॒॒ रसꣳ॒॒ रसे॑न॒ रसे॑न॒ रसं॒ ॅवृष्टि᳚म् । \newline
\pagebreak
\markright{ TS 2.1.7.4  \hfill https://www.vedavms.in \hfill}

\section{ TS 2.1.7.4 }

\textbf{TS 2.1.7.4 } \newline
\textbf{Samhita Paata} \newline

रसं॒ ॅवृष्टि॒मव॑ रुन्धे मैत्रावरु॒णीं द्वि॑रू॒पामा ल॑भेत प्र॒जाका॑मो मै॒त्रं ॅवा अह॑र्वारु॒णी रात्रि॑रहोरा॒त्राभ्यां॒ खलु॒ वै प्र॒जाः प्रजा॑यन्ते मि॒त्रावरु॑णावे॒व स्वेन॑ भाग॒धेये॒नोप॑ धावति॒ तावे॒वास्मा॑ अहोरा॒त्राभ्यां᳚ प्र॒जां प्रज॑नयतः॒ छन्द॑सां॒ ॅवा ए॒ष रसो॒ यद्व॒शा रस॑ इव॒ खलु॒ वै प्र॒जा छन्द॑सामे॒व रसे॑न॒ रसं॑ प्र॒जामव॑ - [  ] \newline

\textbf{Pada Paata} \newline

रस᳚म् । वृष्टि᳚म् । अवेति॑ । रु॒न्धे॒ । मै॒त्रा॒व॒रु॒णीमिति॑ मैत्रा - व॒रु॒णीम् । द्वि॒रू॒पामिति॑ द्वि - रू॒पाम् । एति॑ । ल॒भे॒त॒ । प्र॒जाका॑म॒ इति॑ प्र॒जा - का॒मः॒ । मै॒त्रम् । वै । अहः॑ । वा॒रु॒णी । रात्रिः॑ । अ॒हो॒रा॒त्राभ्या॒मित्य॑हः - रा॒त्राभ्या᳚म् । खलु॑ । वै । प्र॒जा इति॑ प्र - जाः । प्रेति॑ । जा॒य॒न्ते॒ । मि॒त्रावरु॑णा॒विति॑ मि॒त्रा-वरु॑णौ । ए॒व । स्वेन॑ । भा॒ग॒धेये॒नेति॑ भाग - धेये॑न । उपेति॑ । धा॒व॒ति॒ । तौ । ए॒व । अ॒स्मै॒ । अ॒हो॒रा॒त्राभ्या॒मित्य॑हः - रा॒त्राभ्या᳚म् । प्र॒जामिति॑ प्र -जाम् । प्रेति॑ । ज॒न॒य॒तः॒ । छन्द॑साम् । वै । ए॒षः । रसः॑ । यत् । व॒शा । रसः॑ । इ॒व॒ । खलु॑ । वै । प्र॒जेति॑ प्र-जा । छन्द॑साम् । ए॒व । रसे॑न । रस᳚म् । प्र॒जामिति॑ प्र -जाम् । अवेति॑ ।  \newline


\textbf{Krama Paata} \newline

रसं॒ ॅवृष्टि᳚म् । वृष्टि॒मव॑ । अव॑ रुन्धे । रु॒न्धे॒ मै॒त्रा॒व॒रु॒णीम् । मै॒त्रा॒व॒रु॒णीम् द्वि॑रू॒पाम् । मै॒त्रा॒व॒रु॒णीमिति॑ मैत्रा - व॒रु॒णीम् । द्वि॒रू॒पामा । द्वि॒रू॒पामिति॑ द्वि - रू॒पाम् । आ ल॑भेत । ल॒भे॒त॒ प्र॒जाका॑मः । प्र॒जाका॑मो मै॒त्रम् । प्र॒जाका॑म॒ इति॑ प्र॒जा - का॒मः॒ । मै॒त्रं ॅवै । वा अहः॑ । अह॑र् वारु॒णी । वा॒रु॒णी रात्रिः॑ । 
रात्रि॑रहोरा॒त्राभ्या᳚म् । अ॒हो॒रा॒त्राभ्या॒म् खलु॑ । अ॒हो॒रा॒त्राभ्या॒मित्य॑हः - रा॒त्राभ्या᳚म् । खलु॒ वै । वै प्र॒जाः । प्र॒जाः प्र । प्र॒जा इति॑ प्र - जाः । प्र जा॑यन्ते । जा॒य॒न्ते॒ मि॒त्रावरु॑णौ । मि॒त्रावरु॑णावे॒व । मि॒त्रावरु॑णा॒विति॑ मि॒त्रा - वरु॑णौ । ए॒व स्वेन॑ । स्वेन॑ भाग॒धेये॑न । भा॒ग॒धेये॒नोप॑ । भा॒ग॒धेये॒नेति॑ भाग - धेये॑न । उप॑ धावति । धा॒व॒ति॒ तौ । तावे॒व । ए॒वास्मै᳚ । अ॒स्मा॒ अ॒हो॒रा॒त्राभ्या᳚म् । अ॒हो॒रा॒त्राभ्या᳚म् प्र॒जाम् । अ॒हो॒रा॒त्राभ्या॒मित्य॑हः - रा॒त्राभ्या᳚म् । प्र॒जाम् प्र । प्र॒जामिति॑ प्र - जाम् । प्र ज॑नयतः । ज॒न॒य॒त॒ श्छन्द॑साम् । छन्द॑सां॒ ॅवै । वा ए॒षः । ए॒ष रसः॑ । रसो॒ यत् । यद् व॒शा । व॒शा रसः॑ । रस॑ इव । इ॒व॒ खलु॑ । खलु॒ वै । वै प्र॒जा । प्र॒जा छन्द॑साम् । प्र॒जेति॑ प्र - जा । छन्द॑सामे॒व । ए॒व रसे॑न । रसे॑न॒ रस᳚म् । रस॑म् प्र॒जाम् । प्र॒जामव॑ । प्र॒जामिति॑ प्र - जाम् । अव॑ रुन्धे \newline

\textbf{Jatai Paata} \newline

1. रसं॒ ॅवृष्टिं॒ ॅवृष्टिꣳ॒॒ रसꣳ॒॒ रसं॒ ॅवृष्टि᳚म् । \newline
2. वृष्टि॒ मवाव॒ वृष्टिं॒ ॅवृष्टि॒ मव॑ । \newline
3. अव॑ रुन्धे रु॒न्धे ऽवाव॑ रुन्धे । \newline
4. रु॒न्धे॒ मै॒त्रा॒व॒रु॒णीम् मै᳚त्रावरु॒णीꣳ रु॑न्धे रुन्धे मैत्रावरु॒णीम् । \newline
5. मै॒त्रा॒व॒रु॒णीम् द्वि॑रू॒पाम् द्वि॑रू॒पाम् मै᳚त्रावरु॒णीम् मै᳚त्रावरु॒णीम् द्वि॑रू॒पाम् । \newline
6. मै॒त्रा॒व॒रु॒णीमिति॑ मैत्रा - व॒रु॒णीम् । \newline
7. द्वि॒रू॒पा मा द्वि॑रू॒पाम् द्वि॑रू॒पा मा । \newline
8. द्वि॒रू॒पामिति॑ द्वि - रू॒पाम् । \newline
9. आ ल॑भेत लभे॒ता ल॑भेत । \newline
10. ल॒भे॒त॒ प्र॒जाका॑मः प्र॒जाका॑मो लभेत लभेत प्र॒जाका॑मः । \newline
11. प्र॒जाका॑मो मै॒त्रम् मै॒त्रम् प्र॒जाका॑मः प्र॒जाका॑मो मै॒त्रम् । \newline
12. प्र॒जाका॑म॒ इति॑ प्र॒जा - का॒मः॒ । \newline
13. मै॒त्रं ॅवै वै मै॒त्रम् मै॒त्रं ॅवै । \newline
14. वा अह॒ रह॒र् वै वा अहः॑ । \newline
15. अह॑र् वारु॒णी वा॑रु॒ण्यह॒ रह॑र् वारु॒णी । \newline
16. वा॒रु॒णी रात्री॒ रात्रि॑र् वारु॒णी वा॑रु॒णी रात्रिः॑ । \newline
17. रात्रि॑ रहोरा॒त्राभ्या॑ महोरा॒त्राभ्याꣳ॒॒ रात्री॒ रात्रि॑ रहोरा॒त्राभ्या᳚म् । \newline
18. अ॒हो॒रा॒त्राभ्या॒म् खलु॒ खल्व॑होरा॒त्राभ्या॑ महोरा॒त्राभ्या॒म् खलु॑ । \newline
19. अ॒हो॒रा॒त्राभ्या॒मित्य॑हः - रा॒त्राभ्या᳚म् । \newline
20. खलु॒ वै वै खलु॒ खलु॒ वै । \newline
21. वै प्र॒जाः प्र॒जा वै वै प्र॒जाः । \newline
22. प्र॒जाः प्र प्र प्र॒जाः प्र॒जाः प्र । \newline
23. प्र॒जा इति॑ प्र - जाः । \newline
24. प्र जा॑यन्ते जायन्ते॒ प्र प्र जा॑यन्ते । \newline
25. जा॒य॒न्ते॒ मि॒त्रावरु॑णौ मि॒त्रावरु॑णौ जायन्ते जायन्ते मि॒त्रावरु॑णौ । \newline
26. मि॒त्रावरु॑णा वे॒वैव मि॒त्रावरु॑णौ मि॒त्रावरु॑णा वे॒व । \newline
27. मि॒त्रावरु॑णा॒विति॑ मि॒त्रा - वरु॑णौ । \newline
28. ए॒व स्वेन॒ स्वेनै॒वैव स्वेन॑ । \newline
29. स्वेन॑ भाग॒धेये॑न भाग॒धेये॑न॒ स्वेन॒ स्वेन॑ भाग॒धेये॑न । \newline
30. भा॒ग॒धेये॒नोपोप॑ भाग॒धेये॑न भाग॒धेये॒नोप॑ । \newline
31. भा॒ग॒धेये॒नेति॑ भाग - धेये॑न । \newline
32. उप॑ धावति धाव॒ त्युपोप॑ धावति । \newline
33. धा॒व॒ति॒ तौ तौ धा॑वति धावति॒ तौ । \newline
34. ता वे॒वैव तौ ता वे॒व । \newline
35. ए॒वास्मा॑ अस्मा ए॒वैवास्मै᳚ । \newline
36. अ॒स्मा॒ अ॒हो॒रा॒त्राभ्या॑ महोरा॒त्राभ्या॑ मस्मा अस्मा अहोरा॒त्राभ्या᳚म् । \newline
37. अ॒हो॒रा॒त्राभ्या᳚म् प्र॒जाम् प्र॒जा म॑होरा॒त्राभ्या॑ महोरा॒त्राभ्या᳚म् प्र॒जाम् । \newline
38. अ॒हो॒रा॒त्राभ्या॒मित्य॑हः - रा॒त्राभ्या᳚म् । \newline
39. प्र॒जाम् प्र प्र प्र॒जाम् प्र॒जाम् प्र । \newline
40. प्र॒जामिति॑ प्र - जाम् । \newline
41. प्र ज॑नयतो जनयतः॒ प्र प्र ज॑नयतः । \newline
42. ज॒न॒य॒त॒ श्छन्द॑सा॒म् छन्द॑साम् जनयतो जनयत॒ श्छन्द॑साम् । \newline
43. छन्द॑सां॒ ॅवै वै छन्द॑सा॒म् छन्द॑सां॒ ॅवै । \newline
44. वा ए॒ष ए॒ष वै वा ए॒षः । \newline
45. ए॒ष रसो॒ रस॑ ए॒ष ए॒ष रसः॑ । \newline
46. रसो॒ यद् यद् रसो॒ रसो॒ यत् । \newline
47. यद् व॒शा व॒शा यद् यद् व॒शा । \newline
48. व॒शा रसो॒ रसो॑ व॒शा व॒शा रसः॑ । \newline
49. रस॑ इवे व॒ रसो॒ रस॑ इव । \newline
50. इ॒व॒ खलु॒ खल्वि॑वे व॒ खलु॑ । \newline
51. खलु॒ वै वै खलु॒ खलु॒ वै । \newline
52. वै प्र॒जा प्र॒जा वै वै प्र॒जा । \newline
53. प्र॒जा छन्द॑सा॒म् छन्द॑साम् प्र॒जा प्र॒जा छन्द॑साम् । \newline
54. प्र॒जेति॑ प्र - जा । \newline
55. छन्द॑सा मे॒वैव छन्द॑सा॒म् छन्द॑सा मे॒व । \newline
56. ए॒व रसे॑न॒ रसे॑नै॒वैव रसे॑न । \newline
57. रसे॑न॒ रसꣳ॒॒ रसꣳ॒॒ रसे॑न॒ रसे॑न॒ रस᳚म् । \newline
58. रस॑म् प्र॒जाम् प्र॒जाꣳ रसꣳ॒॒ रस॑म् प्र॒जाम् । \newline
59. प्र॒जा मवाव॑ प्र॒जाम् प्र॒जा मव॑ । \newline
60. प्र॒जामिति॑ प्र - जाम् । \newline
61. अव॑ रुन्धे रु॒न्धे ऽवाव॑ रुन्धे । \newline

\textbf{Ghana Paata } \newline

1. रसं॒ ॅवृष्टिं॒ ॅवृष्टिꣳ॒॒ रसꣳ॒॒ रसं॒ ॅवृष्टि॒ मवाव॒ वृष्टिꣳ॒॒ रसꣳ॒॒ रसं॒ ॅवृष्टि॒ मव॑ । \newline
2. वृष्टि॒ मवाव॒ वृष्टिं॒ ॅवृष्टि॒ मव॑ रुन्धे रु॒न्धे ऽव॒ वृष्टिं॒ ॅवृष्टि॒ मव॑ रुन्धे । \newline
3. अव॑ रुन्धे रु॒न्धे ऽवाव॑ रुन्धे मैत्रावरु॒णीम् मै᳚त्रावरु॒णीꣳ रु॒न्धे ऽवाव॑ रुन्धे मैत्रावरु॒णीम् । \newline
4. रु॒न्धे॒ मै॒त्रा॒व॒रु॒णीम् मै᳚त्रावरु॒णीꣳ रु॑न्धे रुन्धे मैत्रावरु॒णीम् द्वि॑रू॒पाम् द्वि॑रू॒पाम् मै᳚त्रावरु॒णीꣳ रु॑न्धे रुन्धे मैत्रावरु॒णीम् द्वि॑रू॒पाम् । \newline
5. मै॒त्रा॒व॒रु॒णीम् द्वि॑रू॒पाम् द्वि॑रू॒पाम् मै᳚त्रावरु॒णीम् मै᳚त्रावरु॒णीम् द्वि॑रू॒पा मा द्वि॑रू॒पाम् मै᳚त्रावरु॒णीम् मै᳚त्रावरु॒णीम् द्वि॑रू॒पा मा । \newline
6. मै॒त्रा॒व॒रु॒णीमिति॑ मैत्रा - व॒रु॒णीम् । \newline
7. द्वि॒रू॒पा मा द्वि॑रू॒पाम् द्वि॑रू॒पा मा ल॑भेत लभे॒ता द्वि॑रू॒पाम् द्वि॑रू॒पा मा ल॑भेत । \newline
8. द्वि॒रू॒पामिति॑ द्वि - रू॒पाम् । \newline
9. आ ल॑भेत लभे॒ता ल॑भेत प्र॒जाका॑मः प्र॒जाका॑मो लभे॒ता ल॑भेत प्र॒जाका॑मः । \newline
10. ल॒भे॒त॒ प्र॒जाका॑मः प्र॒जाका॑मो लभेत लभेत प्र॒जाका॑मो मै॒त्रम् मै॒त्रम् प्र॒जाका॑मो लभेत लभेत प्र॒जाका॑मो मै॒त्रम् । \newline
11. प्र॒जाका॑मो मै॒त्रम् मै॒त्रम् प्र॒जाका॑मः प्र॒जाका॑मो मै॒त्रं ॅवै वै मै॒त्रम् प्र॒जाका॑मः प्र॒जाका॑मो मै॒त्रं ॅवै । \newline
12. प्र॒जाका॑म॒ इति॑ प्र॒जा - का॒मः॒ । \newline
13. मै॒त्रं ॅवै वै मै॒त्रम् मै॒त्रं ॅवा अह॒ रह॒र् वै मै॒त्रम् मै॒त्रं ॅवा अहः॑ । \newline
14. वा अह॒ रह॒र् वै वा अह॑र् वारु॒णी वा॑रु॒ण्यह॒र् वै वा अह॑र् वारु॒णी । \newline
15. अह॑र् वारु॒णी वा॑रु॒ण्यह॒ रह॑र् वारु॒णी रात्री॒ रात्रि॑र् वारु॒ण्यह॒ रह॑र् वारु॒णी रात्रिः॑ । \newline
16. वा॒रु॒णी रात्री॒ रात्रि॑र् वारु॒णी वा॑रु॒णी रात्रि॑ रहोरा॒त्राभ्या॑ महोरा॒त्राभ्याꣳ॒॒ रात्रि॑र् वारु॒णी वा॑रु॒णी रात्रि॑ रहोरा॒त्राभ्या᳚म् । \newline
17. रात्रि॑ रहोरा॒त्राभ्या॑ महोरा॒त्राभ्याꣳ॒॒ रात्री॒ रात्रि॑ रहोरा॒त्राभ्या॒म् खलु॒ खल्व॑ होरा॒त्राभ्याꣳ॒॒ रात्री॒ रात्रि॑ रहोरा॒त्राभ्या॒म् खलु॑ । \newline
18. अ॒हो॒रा॒त्राभ्या॒म् खलु॒ खल्व॑ होरा॒त्राभ्या॑ महोरा॒त्राभ्या॒म् खलु॒ वै वै खल्व॑ होरा॒त्राभ्या॑ महोरा॒त्राभ्या॒म् खलु॒ वै । \newline
19. अ॒हो॒रा॒त्राभ्या॒मित्य॑हः - रा॒त्राभ्या᳚म् । \newline
20. खलु॒ वै वै खलु॒ खलु॒ वै प्र॒जाः प्र॒जा वै खलु॒ खलु॒ वै प्र॒जाः । \newline
21. वै प्र॒जाः प्र॒जा वै वै प्र॒जाः प्र प्र प्र॒जा वै वै प्र॒जाः प्र । \newline
22. प्र॒जाः प्र प्र प्र॒जाः प्र॒जाः प्र जा॑यन्ते जायन्ते॒ प्र प्र॒जाः प्र॒जाः प्र जा॑यन्ते । \newline
23. प्र॒जा इति॑ प्र - जाः । \newline
24. प्र जा॑यन्ते जायन्ते॒ प्र प्र जा॑यन्ते मि॒त्रावरु॑णौ मि॒त्रावरु॑णौ जायन्ते॒ प्र प्र जा॑यन्ते मि॒त्रावरु॑णौ । \newline
25. जा॒य॒न्ते॒ मि॒त्रावरु॑णौ मि॒त्रावरु॑णौ जायन्ते जायन्ते मि॒त्रावरु॑णा वे॒वैव मि॒त्रावरु॑णौ जायन्ते जायन्ते मि॒त्रावरु॑णा वे॒व । \newline
26. मि॒त्रावरु॑णा वे॒वैव मि॒त्रावरु॑णौ मि॒त्रावरु॑णा वे॒व स्वेन॒ स्वेनै॒व मि॒त्रावरु॑णौ मि॒त्रावरु॑णा वे॒व स्वेन॑ । \newline
27. मि॒त्रावरु॑णा॒विति॑ मि॒त्रा - वरु॑णौ । \newline
28. ए॒व स्वेन॒ स्वेनै॒वैव स्वेन॑ भाग॒धेये॑न भाग॒धेये॑न॒ स्वेनै॒वैव स्वेन॑ भाग॒धेये॑न । \newline
29. स्वेन॑ भाग॒धेये॑न भाग॒धेये॑न॒ स्वेन॒ स्वेन॑ भाग॒धेये॒नो पोप॑ भाग॒धेये॑न॒ स्वेन॒ स्वेन॑ भाग॒धेये॒नोप॑ । \newline
30. भा॒ग॒धेये॒नो पोप॑ भाग॒धेये॑न भाग॒धेये॒नोप॑ धावति धाव॒त्युप॑ भाग॒धेये॑न भाग॒धेये॒नोप॑ धावति । \newline
31. भा॒ग॒धेये॒नेति॑ भाग - धेये॑न । \newline
32. उप॑ धावति धाव॒ त्युपोप॑ धावति॒ तौ तौ धा॑व॒ त्युपोप॑ धावति॒ तौ । \newline
33. धा॒व॒ति॒ तौ तौ धा॑वति धावति॒ ता वे॒वैव तौ धा॑वति धावति॒ ता वे॒व । \newline
34. ता वे॒वैव तौ ता वे॒वास्मा॑ अस्मा ए॒व तौ ता वे॒वास्मै᳚ । \newline
35. ए॒वास्मा॑ अस्मा ए॒वैवास्मा॑ अहोरा॒त्राभ्या॑ महोरा॒त्राभ्या॑ मस्मा ए॒वैवास्मा॑ अहोरा॒त्राभ्या᳚म् । \newline
36. अ॒स्मा॒ अ॒हो॒रा॒त्राभ्या॑ महोरा॒त्राभ्या॑ मस्मा अस्मा अहोरा॒त्राभ्या᳚म् प्र॒जाम् प्र॒जा म॑होरा॒त्राभ्या॑ मस्मा अस्मा अहोरा॒त्राभ्या᳚म् प्र॒जाम् । \newline
37. अ॒हो॒रा॒त्राभ्या᳚म् प्र॒जाम् प्र॒जा म॑होरा॒त्राभ्या॑ महोरा॒त्राभ्या᳚म् प्र॒जाम् प्र प्र प्र॒जा म॑होरा॒त्राभ्या॑ महोरा॒त्राभ्या᳚म् प्र॒जाम् प्र । \newline
38. अ॒हो॒रा॒त्राभ्या॒मित्य॑हः - रा॒त्राभ्या᳚म् । \newline
39. प्र॒जाम् प्र प्र प्र॒जाम् प्र॒जाम् प्र ज॑नयतो जनयतः॒ प्र प्र॒जाम् प्र॒जाम् प्र ज॑नयतः । \newline
40. प्र॒जामिति॑ प्र - जाम् । \newline
41. प्र ज॑नयतो जनयतः॒ प्र प्र ज॑नयत॒ श्छन्द॑सा॒म् छन्द॑साम् जनयतः॒ प्र प्र ज॑नयत॒ श्छन्द॑साम् । \newline
42. ज॒न॒य॒त॒ श्छन्द॑सा॒म् छन्द॑साम् जनयतो जनयत॒ श्छन्द॑सां॒ ॅवै वै छन्द॑साम् जनयतो जनयत॒ श्छन्द॑सां॒ ॅवै । \newline
43. छन्द॑सां॒ ॅवै वै छन्द॑सा॒म् छन्द॑सां॒ ॅवा ए॒ष ए॒ष वै छन्द॑सा॒म् छन्द॑सां॒ ॅवा ए॒षः । \newline
44. वा ए॒ष ए॒ष वै वा ए॒ष रसो॒ रस॑ ए॒ष वै वा ए॒ष रसः॑ । \newline
45. ए॒ष रसो॒ रस॑ ए॒ष ए॒ष रसो॒ यद् यद् रस॑ ए॒ष ए॒ष रसो॒ यत् । \newline
46. रसो॒ यद् यद् रसो॒ रसो॒ यद् व॒शा व॒शा यद् रसो॒ रसो॒ यद् व॒शा । \newline
47. यद् व॒शा व॒शा यद् यद् व॒शा रसो॒ रसो॑ व॒शा यद् यद् व॒शा रसः॑ । \newline
48. व॒शा रसो॒ रसो॑ व॒शा व॒शा रस॑ इवे व॒ रसो॑ व॒शा व॒शा रस॑ इव । \newline
49. रस॑ इवे व॒ रसो॒ रस॑ इव॒ खलु॒ खल्वि॑व॒ रसो॒ रस॑ इव॒ खलु॑ । \newline
50. इ॒व॒ खलु॒ खल्वि॑वे व॒ खलु॒ वै वै खल्वि॑वे व॒ खलु॒ वै । \newline
51. खलु॒ वै वै खलु॒ खलु॒ वै प्र॒जा प्र॒जा वै खलु॒ खलु॒ वै प्र॒जा । \newline
52. वै प्र॒जा प्र॒जा वै वै प्र॒जा छन्द॑सा॒म् छन्द॑साम् प्र॒जा वै वै प्र॒जा छन्द॑साम् । \newline
53. प्र॒जा छन्द॑सा॒म् छन्द॑साम् प्र॒जा प्र॒जा छन्द॑सा मे॒वैव छन्द॑साम् प्र॒जा प्र॒जा छन्द॑सा मे॒व । \newline
54. प्र॒जेति॑ प्र - जा । \newline
55. छन्द॑सा मे॒वैव छन्द॑सा॒म् छन्द॑सा मे॒व रसे॑न॒ रसे॑नै॒व छन्द॑सा॒म् छन्द॑सा मे॒व रसे॑न । \newline
56. ए॒व रसे॑न॒ रसे॑नै॒वैव रसे॑न॒ रसꣳ॒॒ रसꣳ॒॒ रसे॑नै॒वैव रसे॑न॒ रस᳚म् । \newline
57. रसे॑न॒ रसꣳ॒॒ रसꣳ॒॒ रसे॑न॒ रसे॑न॒ रस॑म् प्र॒जाम् प्र॒जाꣳ रसꣳ॒॒ रसे॑न॒ रसे॑न॒ रस॑म् प्र॒जाम् । \newline
58. रस॑म् प्र॒जाम् प्र॒जाꣳ रसꣳ॒॒ रस॑म् प्र॒जा मवाव॑ प्र॒जाꣳ रसꣳ॒॒ रस॑म् प्र॒जा मव॑ । \newline
59. प्र॒जा मवाव॑ प्र॒जाम् प्र॒जा मव॑ रुन्धे रु॒न्धे ऽव॑ प्र॒जाम् प्र॒जा मव॑ रुन्धे । \newline
60. प्र॒जामिति॑ प्र - जाम् । \newline
61. अव॑ रुन्धे रु॒न्धे ऽवाव॑ रुन्धे वैश्वदे॒वीं ॅवै᳚श्वदे॒वीꣳ रु॒न्धे ऽवाव॑ रुन्धे वैश्वदे॒वीम् । \newline
\pagebreak
\markright{ TS 2.1.7.5  \hfill https://www.vedavms.in \hfill}

\section{ TS 2.1.7.5 }

\textbf{TS 2.1.7.5 } \newline
\textbf{Samhita Paata} \newline

रुन्धे वैश्वदे॒वीं ब॑हुरू॒पामा ल॑भे॒तान्न॑कामो वैश्वदे॒वं ॅवा अन्नं॒ ॅविश्वा॑ने॒व दे॒वान्थ् स्वेन॑ भाग॒धेये॒नोप॑ धावति॒ त ए॒वास्मा॒ अन्नं॒  प्रय॑च्छन्त्यन्ना॒द ए॒व भ॑वति॒ छन्द॑सां॒ ॅवा ए॒ष रसो॒ यद्व॒शा रस॑ इव॒ खलु॒ वा अन्नं॒ छन्द॑सामे॒व रसे॑न॒ रस॒मन्न॒मव॑ रुन्धे वैश्वदे॒वीं ब॑हुरू॒पामा ल॑भेत॒ ग्राम॑कामो वैश्वदे॒वा वै - [  ] \newline

\textbf{Pada Paata} \newline

रु॒न्धे॒ । वै॒श्व॒दे॒वीमिति॑ वैश्व - दे॒वीम् । ब॒हु॒रू॒पामिति॑ बहु - रू॒पाम् । एति॑ । ल॒भे॒त॒ । अन्न॑काम॒ इत्यन्न॑ - का॒मः॒ । वै॒श्व॒दे॒वमिति॑ वैश्व - दे॒वम् । वै । अन्न᳚म् । विश्वान्॑ । ए॒व । दे॒वान् । स्वेन॑ । भा॒ग॒धेये॒नेति॑ भाग - धेये॑न । उपेति॑ । धा॒व॒ति॒ । ते । ए॒व । अ॒स्मै॒ । अन्न᳚म् । प्रेति॑ । य॒च्छ॒न्ति॒ । अ॒न्ना॒द इत्य॑न्न - अ॒दः । ए॒व । भ॒व॒ति॒ । छन्द॑साम् । वै । ए॒षः । रसः॑ । यत् । व॒शा । रसः॑ । इ॒व॒ । खलु॑ । वै । अन्न᳚म् । छन्द॑साम् । ए॒व । रसे॑न । रस᳚म् । अन्न᳚म् । अवेति॑ । रु॒न्धे॒ । वै॒श्व॒दे॒वीमिति॑ वैश्व - दे॒वीम् । ब॒हु॒रू॒पामिति॑ बहु-रू॒पाम् । एति॑ । ल॒भे॒त॒ । ग्राम॑काम॒ इति॒ ग्राम॑ - का॒मः॒ । वै॒श्व॒दे॒वा इति॑ वैश्व - दे॒वाः । वै ।  \newline


\textbf{Krama Paata} \newline

रु॒न्धे॒ वै॒श्व॒दे॒वीम् । वै॒श्व॒दे॒वीम् ब॑हुरू॒पाम् । वै॒श्व॒दे॒वीमिति॑ वैश्व - दे॒वीम् । ब॒हु॒रू॒पामा । ब॒हु॒रू॒पामिति॑ बहु - रू॒पाम् । आ ल॑भेत । ल॒भे॒तान्न॑कामः । अन्न॑कामो वैश्वदे॒वम् । अन्न॑काम॒ इत्यन्न॑ - का॒मः॒ । वै॒श्व॒दे॒वं ॅवै । वै॒श्व॒दे॒वमिति॑ वैश्व - दे॒वम् । वा अन्न᳚म् । अन्नं॒ ॅविश्वान्॑ । विश्वा॑ने॒व । ए॒व दे॒वान् । दे॒वान्थ् स्वेन॑ । स्वेन॑ भाग॒धेये॑न । भा॒ग॒धेये॒नोप॑ । भा॒ग॒धेये॒नेति॑ भाग - धेये॑न । उप॑ धावति । धा॒व॒ति॒ ते । त ए॒व । ए॒वास्मै᳚ । अ॒स्मा॒ अन्न᳚म् । अन्न॒म् प्र । प्र य॑च्छन्ति । य॒च्छ॒न्त्य॒न्ना॒दः । अ॒न्ना॒द ए॒व । अ॒न्ना॒द इत्य॑न्न - अ॒दः । ए॒व भ॑वति । भ॒व॒ति॒ छन्द॑साम् । छन्द॑सां॒ ॅवै । वा ए॒षः । ए॒ष रसः॑ । रसो॒ यत् । यद् व॒शा । व॒शा रसः॑ । रस॑ इव । इ॒व॒ खलु॑ । खलु॒ वै । वा अन्न᳚म् । अन्नं॒ छन्द॑साम् । छन्द॑सामे॒व । ए॒व रसे॑न । रसे॑न॒ रस᳚म् । रस॒मन्न᳚म् । अन्न॒मव॑ । अव॑ रुन्धे । रु॒न्धे॒ वै॒श्व॒दे॒वीम् । वै॒श्व॒दे॒वीं ब॑हुरू॒पाम् । वै॒श्व॒दे॒वीमिति॑ वैश्व - दे॒वीम् । ब॒हु॒रू॒पामा । ब॒हु॒रू॒पामिति॑ बहु - रू॒पाम् । आ ल॑भेत । ल॒भे॒त॒ ग्राम॑कामः । ग्राम॑कामो वैश्वदे॒वाः । ग्राम॑काम॒ इति॒ ग्राम॑ - का॒मः॒ । वै॒श्व॒दे॒वा वै । वै॒श्व॒दे॒वा इति॑ वैश्व - दे॒वाः । वै स॑जा॒ताः \newline

\textbf{Jatai Paata} \newline

1. रु॒न्धे॒ वै॒श्व॒दे॒वीं ॅवै᳚श्वदे॒वीꣳ रु॑न्धे रुन्धे वैश्वदे॒वीम् । \newline
2. वै॒श्व॒दे॒वीम् ब॑हुरू॒पाम् ब॑हुरू॒पां ॅवै᳚श्वदे॒वीं ॅवै᳚श्वदे॒वीम् ब॑हुरू॒पाम् । \newline
3. वै॒श्व॒दे॒वीमिति॑ वैश्व - दे॒वीम् । \newline
4. ब॒हु॒रू॒पा मा ब॑हुरू॒पाम् ब॑हुरू॒पा मा । \newline
5. ब॒हु॒रू॒पामिति॑ बहु - रू॒पाम् । \newline
6. आ ल॑भेत लभे॒ता ल॑भेत । \newline
7. ल॒भे॒ता न्न॑का॒मो ऽन्न॑कामो लभेत लभे॒ता न्न॑कामः । \newline
8. अन्न॑कामो वैश्वदे॒वं ॅवै᳚श्वदे॒व मन्न॑का॒मो ऽन्न॑कामो वैश्वदे॒वम् । \newline
9. अन्न॑काम॒इत्यन्न॑ - का॒मः॒ । \newline
10. वै॒श्व॒दे॒वं ॅवै वै वै᳚श्वदे॒वं ॅवै᳚श्वदे॒वं ॅवै । \newline
11. वै॒श्व॒दे॒वमिति॑ वैश्व - दे॒वम् । \newline
12. वा अन्न॒ मन्नं॒ ॅवै वा अन्न᳚म् । \newline
13. अन्नं॒ ॅविश्वा॒न्॒. विश्वा॒ नन्न॒ मन्नं॒ ॅविश्वान्॑ । \newline
14. विश्वा॑ ने॒वैव विश्वा॒न्॒. विश्वा॑ ने॒व । \newline
15. ए॒व दे॒वान् दे॒वा ने॒वैव दे॒वान् । \newline
16. दे॒वान् थ्स्वेन॒ स्वेन॑ दे॒वान् दे॒वान् थ्स्वेन॑ । \newline
17. स्वेन॑ भाग॒धेये॑न भाग॒धेये॑न॒ स्वेन॒ स्वेन॑ भाग॒धेये॑न । \newline
18. भा॒ग॒धेये॒नोपोप॑ भाग॒धेये॑न भाग॒धेये॒नोप॑ । \newline
19. भा॒ग॒धेये॒नेति॑ भाग - धेये॑न । \newline
20. उप॑ धावति धाव॒ त्युपोप॑ धावति । \newline
21. धा॒व॒ति॒ ते ते धा॑वति धावति॒ ते । \newline
22. त ए॒वैव ते त ए॒व । \newline
23. ए॒वास्मा॑ अस्मा ए॒वैवास्मै᳚ । \newline
24. अ॒स्मा॒ अन्न॒ मन्न॑ मस्मा अस्मा॒ अन्न᳚म् । \newline
25. अन्न॒म् प्र प्रान्न॒ मन्न॒म् प्र । \newline
26. प्र य॑च्छन्ति यच्छन्ति॒ प्र प्र य॑च्छन्ति । \newline
27. य॒च्छ॒ न्त्य॒न्ना॒दो᳚ ऽन्ना॒दो य॑च्छन्ति यच्छ न्त्यन्ना॒दः । \newline
28. अ॒न्ना॒द ए॒वैवा न्ना॒दो᳚ ऽन्ना॒द ए॒व । \newline
29. अ॒न्ना॒द इत्य॑न्न - अ॒दः । \newline
30. ए॒व भ॑वति भव त्ये॒वैव भ॑वति । \newline
31. भ॒व॒ति॒ छन्द॑सा॒म् छन्द॑साम् भवति भवति॒ छन्द॑साम् । \newline
32. छन्द॑सां॒ ॅवै वै छन्द॑सा॒म् छन्द॑सां॒ ॅवै । \newline
33. वा ए॒ष ए॒ष वै वा ए॒षः । \newline
34. ए॒ष रसो॒ रस॑ ए॒ष ए॒ष रसः॑ । \newline
35. रसो॒ यद् यद् रसो॒ रसो॒ यत् । \newline
36. यद् व॒शा व॒शा यद् यद् व॒शा । \newline
37. व॒शा रसो॒ रसो॑ व॒शा व॒शा रसः॑ । \newline
38. रस॑ इवे व॒ रसो॒ रस॑ इव । \newline
39. इ॒व॒ खलु॒ खल्वि॑वे व॒ खलु॑ । \newline
40. खलु॒ वै वै खलु॒ खलु॒ वै । \newline
41. वा अन्न॒ मन्नं॒ ॅवै वा अन्न᳚म् । \newline
42. अन्न॒म् छन्द॑सा॒म् छन्द॑सा॒ मन्न॒ मन्न॒म् छन्द॑साम् । \newline
43. छन्द॑सा मे॒वैव छन्द॑सा॒म् छन्द॑सा मे॒व । \newline
44. ए॒व रसे॑न॒ रसे॑नै॒वैव रसे॑न । \newline
45. रसे॑न॒ रसꣳ॒॒ रसꣳ॒॒ रसे॑न॒ रसे॑न॒ रस᳚म् । \newline
46. रस॒ मन्न॒ मन्नꣳ॒॒ रसꣳ॒॒ रस॒ मन्न᳚म् । \newline
47. अन्न॒ मवावान्न॒ मन्न॒ मव॑ । \newline
48. अव॑ रुन्धे रु॒न्धे ऽवाव॑ रुन्धे । \newline
49. रु॒न्धे॒ वै॒श्व॒दे॒वीं ॅवै᳚श्वदे॒वीꣳ रु॑न्धे रुन्धे वैश्वदे॒वीम् । \newline
50. वै॒श्व॒दे॒वीम् ब॑हुरू॒पाम् ब॑हुरू॒पां ॅवै᳚श्वदे॒वीं ॅवै᳚श्वदे॒वीम् ब॑हुरू॒पाम् । \newline
51. वै॒श्व॒दे॒वीमिति॑ वैश्व - दे॒वीम् । \newline
52. ब॒हु॒रू॒पा मा ब॑हुरू॒पाम् ब॑हुरू॒पा मा । \newline
53. ब॒हु॒रू॒पामिति॑ बहु - रू॒पाम् । \newline
54. आ ल॑भेत लभे॒ता ल॑भेत । \newline
55. ल॒भे॒त॒ ग्राम॑कामो॒ ग्राम॑कामो लभेत लभेत॒ ग्राम॑कामः । \newline
56. ग्राम॑कामो वैश्वदे॒वा वै᳚श्वदे॒वा ग्राम॑कामो॒ ग्राम॑कामो वैश्वदे॒वाः । \newline
57. ग्राम॑काम॒ इति॒ ग्राम॑ - का॒मः॒ । \newline
58. वै॒श्व॒दे॒वा वै वै वै᳚श्वदे॒वा वै᳚श्वदे॒वा वै । \newline
59. वै॒श्व॒दे॒वा इति॑ वैश्व - दे॒वाः । \newline
60. वै स॑जा॒ताः स॑जा॒ता वै वै स॑जा॒ताः । \newline

\textbf{Ghana Paata } \newline

1. रु॒न्धे॒ वै॒श्व॒दे॒वीं ॅवै᳚श्वदे॒वीꣳ रु॑न्धे रुन्धे वैश्वदे॒वीम् ब॑हुरू॒पाम् ब॑हुरू॒पां ॅवै᳚श्वदे॒वीꣳ रु॑न्धे रुन्धे वैश्वदे॒वीम् ब॑हुरू॒पाम् । \newline
2. वै॒श्व॒दे॒वीम् ब॑हुरू॒पाम् ब॑हुरू॒पां ॅवै᳚श्वदे॒वीं ॅवै᳚श्वदे॒वीम् ब॑हुरू॒पा मा ब॑हुरू॒पां ॅवै᳚श्वदे॒वीं ॅवै᳚श्वदे॒वीम् ब॑हुरू॒पा मा । \newline
3. वै॒श्व॒दे॒वीमिति॑ वैश्व - दे॒वीम् । \newline
4. ब॒हु॒रू॒पा मा ब॑हुरू॒पाम् ब॑हुरू॒पा मा ल॑भेत लभे॒ता ब॑हुरू॒पाम् ब॑हुरू॒पा मा ल॑भेत । \newline
5. ब॒हु॒रू॒पामिति॑ बहु - रू॒पाम् । \newline
6. आ ल॑भेत लभे॒ता ल॑भे॒ता न्न॑का॒मो ऽन्न॑कामो लभे॒ता ल॑भे॒ता न्न॑कामः । \newline
7. ल॒भे॒ता न्न॑का॒मो ऽन्न॑कामो लभेत लभे॒ता न्न॑कामो वैश्वदे॒वं ॅवै᳚श्वदे॒व मन्न॑कामो लभेत लभे॒ता न्न॑कामो वैश्वदे॒वम् । \newline
8. अन्न॑कामो वैश्वदे॒वं ॅवै᳚श्वदे॒व मन्न॑का॒मो ऽन्न॑कामो वैश्वदे॒वं ॅवै वै वै᳚श्वदे॒व मन्न॑का॒मो ऽन्न॑कामो वैश्वदे॒वं ॅवै । \newline
9. अन्न॑काम॒इत्यन्न॑ - का॒मः॒ । \newline
10. वै॒श्व॒दे॒वं ॅवै वै वै᳚श्वदे॒वं ॅवै᳚श्वदे॒वं ॅवा अन्न॒ मन्नं॒ ॅवै वै᳚श्वदे॒वं ॅवै᳚श्वदे॒वं ॅवा अन्न᳚म् । \newline
11. वै॒श्व॒दे॒वमिति॑ वैश्व - दे॒वम् । \newline
12. वा अन्न॒ मन्नं॒ ॅवै वा अन्नं॒ ॅविश्वा॒न्॒. विश्वा॒ नन्नं॒ ॅवै वा अन्नं॒ ॅविश्वान्॑ । \newline
13. अन्नं॒ ॅविश्वा॒न्॒. विश्वा॒ नन्न॒ मन्नं॒ ॅविश्वा॑ ने॒वैव विश्वा॒ नन्न॒ मन्नं॒ ॅविश्वा॑ ने॒व । \newline
14. विश्वा॑ ने॒वैव विश्वा॒न्॒. विश्वा॑ ने॒व दे॒वान् दे॒वा ने॒व विश्वा॒न्॒. विश्वा॑ ने॒व दे॒वान् । \newline
15. ए॒व दे॒वान् दे॒वा ने॒वैव दे॒वान् थ्स्वेन॒ स्वेन॑ दे॒वा ने॒वैव दे॒वान् थ्स्वेन॑ । \newline
16. दे॒वान् थ्स्वेन॒ स्वेन॑ दे॒वान् दे॒वान् थ्स्वेन॑ भाग॒धेये॑न भाग॒धेये॑न॒ स्वेन॑ दे॒वान् दे॒वान् थ्स्वेन॑ भाग॒धेये॑न । \newline
17. स्वेन॑ भाग॒धेये॑न भाग॒धेये॑न॒ स्वेन॒ स्वेन॑ भाग॒धेये॒नो पोप॑ भाग॒धेये॑न॒ स्वेन॒ स्वेन॑ भाग॒धेये॒नोप॑ । \newline
18. भा॒ग॒धेये॒नो पोप॑ भाग॒धेये॑न भाग॒धेये॒नोप॑ धावति धाव॒त्युप॑ भाग॒धेये॑न भाग॒धेये॒नोप॑ धावति । \newline
19. भा॒ग॒धेये॒नेति॑ भाग - धेये॑न । \newline
20. उप॑ धावति धाव॒ त्युपोप॑ धावति॒ ते ते धा॑व॒ त्युपोप॑ धावति॒ ते । \newline
21. धा॒व॒ति॒ ते ते धा॑वति धावति॒ त ए॒वैव ते धा॑वति धावति॒ त ए॒व । \newline
22. त ए॒वैव ते त ए॒वास्मा॑ अस्मा ए॒व ते त ए॒वास्मै᳚ । \newline
23. ए॒वास्मा॑ अस्मा ए॒वैवास्मा॒ अन्न॒ मन्न॑ मस्मा ए॒वैवास्मा॒ अन्न᳚म् । \newline
24. अ॒स्मा॒ अन्न॒ मन्न॑ मस्मा अस्मा॒ अन्न॒म् प्र प्रान्न॑ मस्मा अस्मा॒ अन्न॒म् प्र । \newline
25. अन्न॒म् प्र प्रान्न॒ मन्न॒म् प्र य॑च्छन्ति यच्छन्ति॒ प्रान्न॒ मन्न॒म् प्र य॑च्छन्ति । \newline
26. प्र य॑च्छन्ति यच्छन्ति॒ प्र प्र य॑च्छ न्त्यन्ना॒दो᳚ ऽन्ना॒दो य॑च्छन्ति॒ प्र प्र य॑च्छ न्त्यन्ना॒दः । \newline
27. य॒च्छ॒ न्त्य॒न्ना॒दो᳚ ऽन्ना॒दो य॑च्छन्ति यच्छ न्त्यन्ना॒द ए॒वैवान्ना॒दो य॑च्छन्ति यच्छ न्त्यन्ना॒द ए॒व । \newline
28. अ॒न्ना॒द ए॒वैवान्ना॒दो᳚ ऽन्ना॒द ए॒व भ॑वति भवत्ये॒वान्ना॒दो᳚ ऽन्ना॒द ए॒व भ॑वति । \newline
29. अ॒न्ना॒द इत्य॑न्न - अ॒दः । \newline
30. ए॒व भ॑वति भवत्ये॒वैव भ॑वति॒ छन्द॑सा॒म् छन्द॑साम् भवत्ये॒वैव भ॑वति॒ छन्द॑साम् । \newline
31. भ॒व॒ति॒ छन्द॑सा॒म् छन्द॑साम् भवति भवति॒ छन्द॑सां॒ ॅवै वै छन्द॑साम् भवति भवति॒ छन्द॑सां॒ ॅवै । \newline
32. छन्द॑सां॒ ॅवै वै छन्द॑सा॒म् छन्द॑सां॒ ॅवा ए॒ष ए॒ष वै छन्द॑सा॒म् छन्द॑सां॒ ॅवा ए॒षः । \newline
33. वा ए॒ष ए॒ष वै वा ए॒ष रसो॒ रस॑ ए॒ष वै वा ए॒ष रसः॑ । \newline
34. ए॒ष रसो॒ रस॑ ए॒ष ए॒ष रसो॒ यद् यद् रस॑ ए॒ष ए॒ष रसो॒ यत् । \newline
35. रसो॒ यद् यद् रसो॒ रसो॒ यद् व॒शा व॒शा यद् रसो॒ रसो॒ यद् व॒शा । \newline
36. यद् व॒शा व॒शा यद् यद् व॒शा रसो॒ रसो॑ व॒शा यद् यद् व॒शा रसः॑ । \newline
37. व॒शा रसो॒ रसो॑ व॒शा व॒शा रस॑ इवे व॒ रसो॑ व॒शा व॒शा रस॑ इव । \newline
38. रस॑ इवे व॒ रसो॒ रस॑ इव॒ खलु॒ खल्वि॑व॒ रसो॒ रस॑ इव॒ खलु॑ । \newline
39. इ॒व॒ खलु॒ खल्वि॑वे व॒ खलु॒ वै वै खल्वि॑वे व॒ खलु॒ वै । \newline
40. खलु॒ वै वै खलु॒ खलु॒ वा अन्न॒ मन्नं॒ ॅवै खलु॒ खलु॒ वा अन्न᳚म् । \newline
41. वा अन्न॒ मन्नं॒ ॅवै वा अन्न॒म् छन्द॑सा॒म् छन्द॑सा॒ मन्नं॒ ॅवै वा अन्न॒म् छन्द॑साम् । \newline
42. अन्न॒म् छन्द॑सा॒म् छन्द॑सा॒ मन्न॒ मन्न॒म् छन्द॑सा मे॒वैव छन्द॑सा॒ मन्न॒ मन्न॒म् छन्द॑सा मे॒व । \newline
43. छन्द॑सा मे॒वैव छन्द॑सा॒म् छन्द॑सा मे॒व रसे॑न॒ रसे॑नै॒व छन्द॑सा॒म् छन्द॑सा मे॒व रसे॑न । \newline
44. ए॒व रसे॑न॒ रसे॑नै॒वैव रसे॑न॒ रसꣳ॒॒ रसꣳ॒॒ रसे॑नै॒वैव रसे॑न॒ रस᳚म् । \newline
45. रसे॑न॒ रसꣳ॒॒ रसꣳ॒॒ रसे॑न॒ रसे॑न॒ रस॒ मन्न॒ मन्नꣳ॒॒ रसꣳ॒॒ रसे॑न॒ रसे॑न॒ रस॒ मन्न᳚म् । \newline
46. रस॒ मन्न॒ मन्नꣳ॒॒ रसꣳ॒॒ रस॒ मन्न॒ मवावा न्नꣳ॒॒ रसꣳ॒॒ रस॒ मन्न॒ मव॑ । \newline
47. अन्न॒ मवावान्न॒ मन्न॒ मव॑ रुन्धे रु॒न्धे ऽवान्न॒ मन्न॒ मव॑ रुन्धे । \newline
48. अव॑ रुन्धे रु॒न्धे ऽवाव॑ रुन्धे वैश्वदे॒वीं ॅवै᳚श्वदे॒वीꣳ रु॒न्धे ऽवाव॑ रुन्धे वैश्वदे॒वीम् । \newline
49. रु॒न्धे॒ वै॒श्व॒दे॒वीं ॅवै᳚श्वदे॒वीꣳ रु॑न्धे रुन्धे वैश्वदे॒वीम् ब॑हुरू॒पाम् ब॑हुरू॒पां ॅवै᳚श्वदे॒वीꣳ रु॑न्धे रुन्धे वैश्वदे॒वीम् ब॑हुरू॒पाम् । \newline
50. वै॒श्व॒दे॒वीम् ब॑हुरू॒पाम् ब॑हुरू॒पां ॅवै᳚श्वदे॒वीं ॅवै᳚श्वदे॒वीम् ब॑हुरू॒पा मा ब॑हुरू॒पां ॅवै᳚श्वदे॒वीं ॅवै᳚श्वदे॒वीम् ब॑हुरू॒पा मा । \newline
51. वै॒श्व॒दे॒वीमिति॑ वैश्व - दे॒वीम् । \newline
52. ब॒हु॒रू॒पा मा ब॑हुरू॒पाम् ब॑हुरू॒पा मा ल॑भेत लभे॒ता ब॑हुरू॒पाम् ब॑हुरू॒पा मा ल॑भेत । \newline
53. ब॒हु॒रू॒पामिति॑ बहु - रू॒पाम् । \newline
54. आ ल॑भेत लभे॒ता ल॑भेत॒ ग्राम॑कामो॒ ग्राम॑कामो लभे॒ता ल॑भेत॒ ग्राम॑कामः । \newline
55. ल॒भे॒त॒ ग्राम॑कामो॒ ग्राम॑कामो लभेत लभेत॒ ग्राम॑कामो वैश्वदे॒वा वै᳚श्वदे॒वा ग्राम॑कामो लभेत लभेत॒ ग्राम॑कामो वैश्वदे॒वाः । \newline
56. ग्राम॑कामो वैश्वदे॒वा वै᳚श्वदे॒वा ग्राम॑कामो॒ ग्राम॑कामो वैश्वदे॒वा वै वै वै᳚श्वदे॒वा ग्राम॑कामो॒ ग्राम॑कामो वैश्वदे॒वा वै । \newline
57. ग्राम॑काम॒ इति॒ ग्राम॑ - का॒मः॒ । \newline
58. वै॒श्व॒दे॒वा वै वै वै᳚श्वदे॒वा वै᳚श्वदे॒वा वै स॑जा॒ताः स॑जा॒ता वै वै᳚श्वदे॒वा वै᳚श्वदे॒वा वै स॑जा॒ताः । \newline
59. वै॒श्व॒दे॒वा इति॑ वैश्व - दे॒वाः । \newline
60. वै स॑जा॒ताः स॑जा॒ता वै वै स॑जा॒ता विश्वा॒न्॒. विश्वा᳚न् थ्सजा॒ता वै वै स॑जा॒ता विश्वान्॑ । \newline
\pagebreak
\markright{ TS 2.1.7.6  \hfill https://www.vedavms.in \hfill}

\section{ TS 2.1.7.6 }

\textbf{TS 2.1.7.6 } \newline
\textbf{Samhita Paata} \newline

स॑जा॒ता विश्वा॑ने॒व दे॒वान्थ् स्वेन॑ भाग॒धेये॒नोप॑ धावति॒ त ए॒वास्मै॑ सजा॒तान् प्र य॑च्छन्ति ग्रा॒म्ये॑व भ॑वति॒ छन्द॑सां॒ ॅवा ए॒ष रसो॒ यद्व॒शा रस॑ इव॒ खलु॒ वै स॑जा॒ताः छन्द॑सामे॒व रसे॑न॒ रसꣳ॑ सजा॒तानव॑ रुन्धे बार्.हस्प॒त्य- मु॑क्षव॒शमा ल॑भेत ब्रह्मवर्च॒सका॑मो॒ बृह॒स्पति॑मे॒व स्वेन॑ भाग॒धेये॒नोप॑ धावति॒ स ए॒वास्मि॑न् ब्रह्मवर्च॒सं - [  ] \newline

\textbf{Pada Paata} \newline

स॒जा॒ता इति॑ स - जा॒ताः । विश्वान्॑ । ए॒व । दे॒वान् । स्वेन॑ । भा॒ग॒धेये॒नेति॑ भाग - धेये॑न । उपेति॑ । धा॒व॒ति॒ । ते । ए॒व । अ॒स्मै॒ । स॒जा॒तानिति॑ स-जा॒तान् । प्रेति॑ । य॒च्छ॒न्ति॒ । ग्रा॒मी । ए॒व । भ॒व॒ति॒ । छन्द॑साम् । वै । ए॒षः । रसः॑ । यत् । व॒शा । रसः॑ । इ॒व॒ । खलु॑ । वै । स॒जा॒ता इति॑ स - जा॒ताः । छन्द॑साम् । ए॒व । रसे॑न । रस᳚म् । स॒जा॒तानिति॑ स - जा॒तान् । अवेति॑ । रु॒न्धे॒ । बा॒र्.॒ह॒स्प॒त्यम् । उ॒क्ष॒व॒शमित्यु॑क्ष - व॒शम् । एति॑ । ल॒भे॒त॒ । ब्र॒ह्म॒व॒र्च॒सका॑म॒ इति॑ ब्रह्मवर्च॒स - का॒मः॒ । बृह॒स्पति᳚म् । ए॒व । स्वेन॑ । भा॒ग॒धेये॒नेति॑ भाग - धेये॑न । उपेति॑ । धा॒व॒ति॒ । सः । ए॒व । अ॒स्मि॒न्न् । ब्र॒ह्म॒व॒र्च॒समिति॑ ब्रह्म - व॒र्च॒सम् ।  \newline


\textbf{Krama Paata} \newline

स॒जा॒ता विश्वान्॑ । स॒जा॒ता इति॑ स - जा॒ताः । विश्वा॑ने॒व । ए॒व दे॒वान् । दे॒वान्थ् स्वेन॑ । स्वेन॑ भाग॒धेये॑न । भा॒ग॒धेये॒नोप॑ । भा॒ग॒धेये॒नेति॑ भाग - धेये॑न । उप॑ धावति । धा॒व॒ति॒ ते । त ए॒व । ए॒वास्मै᳚ । अ॒स्मै॒ स॒जा॒तान् । स॒जा॒तान् प्र । स॒जा॒तानिति॑ स - जा॒तान् । प्र य॑च्छन्ति । य॒च्छ॒न्ति॒ ग्रा॒मी । ग्रा॒म्ये॑व । ए॒व भ॑वति । भ॒व॒ति॒ छन्द॑साम् । छन्द॑सां॒ ॅवै । वा ए॒षः । ए॒ष रसः॑ । रसो॒ यत् । यद् व॒शा । व॒शा रसः॑ । रस॑ इव । इ॒व॒ खलु॑ । खलु॒ वै । वै स॑जा॒ताः । स॒जा॒ता श्छन्द॑साम् । स॒जा॒ता इति॑ स - जा॒ताः । छन्द॑सामे॒व । ए॒व रसे॑न । रसे॑न॒ रस᳚म् । रसꣳ॑ सजा॒तान् । स॒जा॒तानव॑ । स॒जा॒तानिति॑स - जा॒तान् । अव॑ रुन्धे । रु॒न्धे॒ बा॒र्॒.ह॒स्प॒त्यम् । बा॒र्॒.ह॒स्प॒त्यमु॑क्षव॒शम् । उ॒क्ष॒व॒शमा । उ॒क्ष॒व॒शमित्यु॑क्ष - व॒शम् । आ ल॑भेत । ल॒भे॒त॒ ब्र॒ह्म॒व॒र्च॒सका॑मः । ब्र॒ह्म॒व॒र्च॒सका॑मो॒ बृह॒स्पति᳚म् । ब्र॒ह्म॒व॒र्च॒सका॑म॒ इति॑ ब्रह्मवर्च॒स - का॒मः॒ । बृह॒स्पति॑मे॒व । ए॒व स्वेन॑ । स्वेन॑ भाग॒धेये॑न । भा॒ग॒धेये॒नोप॑ । भा॒ग॒धेये॒नेति॑ भाग - धेये॑न । उप॑ धावति । धा॒व॒ति॒ सः । स ए॒व । ए॒वास्मिन्न्॑ । अ॒स्मि॒न् ब्र॒ह्म॒व॒र्च॒सम् । ब्र॒ह्म॒व॒र्च॒सम् द॑धाति । ब्र॒ह्म॒व॒र्च॒समिति॑ ब्रह्म - व॒र्च॒सम् \newline

\textbf{Jatai Paata} \newline

1. स॒जा॒ता विश्वा॒न्॒. विश्वा᳚न् थ्सजा॒ताः स॑जा॒ता विश्वान्॑ । \newline
2. स॒जा॒ता इति॑ स - जा॒ताः । \newline
3. विश्वा॑ ने॒वैव विश्वा॒न्॒. विश्वा॑ ने॒व । \newline
4. ए॒व दे॒वान् दे॒वा ने॒वैव दे॒वान् । \newline
5. दे॒वान् थ्स्वेन॒ स्वेन॑ दे॒वान् दे॒वान् थ्स्वेन॑ । \newline
6. स्वेन॑ भाग॒धेये॑न भाग॒धेये॑न॒ स्वेन॒ स्वेन॑ भाग॒धेये॑न । \newline
7. भा॒ग॒धेये॒नोपोप॑ भाग॒धेये॑न भाग॒धेये॒नोप॑ । \newline
8. भा॒ग॒धेये॒नेति॑ भाग - धेये॑न । \newline
9. उप॑ धावति धाव॒ त्युपोप॑ धावति । \newline
10. धा॒व॒ति॒ ते ते धा॑वति धावति॒ ते । \newline
11. त ए॒वैव ते त ए॒व । \newline
12. ए॒वास्मा॑ अस्मा ए॒वैवास्मै᳚ । \newline
13. अ॒स्मै॒ स॒जा॒तान् थ्स॑जा॒ता न॑स्मा अस्मै सजा॒तान् । \newline
14. स॒जा॒तान् प्र प्र स॑जा॒तान् थ्स॑जा॒तान् प्र । \newline
15. स॒जा॒तानिति॑ स - जा॒तान् । \newline
16. प्र य॑च्छन्ति यच्छन्ति॒ प्र प्र य॑च्छन्ति । \newline
17. य॒च्छ॒न्ति॒ ग्रा॒मी ग्रा॒मी य॑च्छन्ति यच्छन्ति ग्रा॒मी । \newline
18. ग्रा॒म्ये॑वैव ग्रा॒मी ग्रा॒म्ये॑व । \newline
19. ए॒व भ॑वति भव त्ये॒वैव भ॑वति । \newline
20. भ॒व॒ति॒ छन्द॑सा॒म् छन्द॑साम् भवति भवति॒ छन्द॑साम् । \newline
21. छन्द॑सां॒ ॅवै वै छन्द॑सा॒म् छन्द॑सां॒ ॅवै । \newline
22. वा ए॒ष ए॒ष वै वा ए॒षः । \newline
23. ए॒ष रसो॒ रस॑ ए॒ष ए॒ष रसः॑ । \newline
24. रसो॒ यद् यद् रसो॒ रसो॒ यत् । \newline
25. यद् व॒शा व॒शा यद् यद् व॒शा । \newline
26. व॒शा रसो॒ रसो॑ व॒शा व॒शा रसः॑ । \newline
27. रस॑ इवे व॒ रसो॒ रस॑ इव । \newline
28. इ॒व॒ खलु॒ खल्वि॑वे व॒ खलु॑ । \newline
29. खलु॒ वै वै खलु॒ खलु॒ वै । \newline
30. वै स॑जा॒ताः स॑जा॒ता वै वै स॑जा॒ताः । \newline
31. स॒जा॒ता श्छन्द॑सा॒म् छन्द॑साꣳ सजा॒ताः स॑जा॒ता श्छन्द॑साम् । \newline
32. स॒जा॒ता इति॑ स - जा॒ताः । \newline
33. छन्द॑सा मे॒वैव छन्द॑सा॒म् छन्द॑सा मे॒व । \newline
34. ए॒व रसे॑न॒ रसे॑नै॒वैव रसे॑न । \newline
35. रसे॑न॒ रसꣳ॒॒ रसꣳ॒॒ रसे॑न॒ रसे॑न॒ रस᳚म् । \newline
36. रसꣳ॑ सजा॒तान् थ्स॑जा॒तान् रसꣳ॒॒ रसꣳ॑ सजा॒तान् । \newline
37. स॒जा॒ता नवाव॑ सजा॒तान् थ्स॑जा॒ता नव॑ । \newline
38. स॒जा॒तानिति॑ स - जा॒तान् । \newline
39. अव॑ रुन्धे रु॒न्धे ऽवाव॑ रुन्धे । \newline
40. रु॒न्धे॒ बा॒र्॒.ह॒स्प॒त्यम् बा॑र्.हस्प॒त्यꣳ रु॑न्धे रुन्धे बार्.हस्प॒त्यम् । \newline
41. बा॒र्॒.ह॒स्प॒त्य मु॑क्षव॒श मु॑क्षव॒शम् बा॑र्.हस्प॒त्यम् बा॑र्.हस्प॒त्य मु॑क्षव॒शम् । \newline
42. उ॒क्ष॒व॒श मोक्ष॑व॒श मु॑क्षव॒श मा । \newline
43. उ॒क्ष॒व॒शमित्यु॑क्ष - व॒शम् । \newline
44. आ ल॑भेत लभे॒ता ल॑भेत । \newline
45. ल॒भे॒त॒ ब्र॒ह्म॒व॒र्च॒सका॑मो ब्रह्मवर्च॒सका॑मो लभेत लभेत ब्रह्मवर्च॒सका॑मः । \newline
46. ब्र॒ह्म॒व॒र्च॒सका॑मो॒ बृह॒स्पति॒म् बृह॒स्पति॑म् ब्रह्मवर्च॒सका॑मो ब्रह्मवर्च॒सका॑मो॒ बृह॒स्पति᳚म् । \newline
47. ब्र॒ह्म॒व॒र्च॒सका॑म॒ इति॑ ब्रह्मवर्च॒स - का॒मः॒ । \newline
48. बृह॒स्पति॑ मे॒वैव बृह॒स्पति॒म् बृह॒स्पति॑ मे॒व । \newline
49. ए॒व स्वेन॒ स्वेनै॒वैव स्वेन॑ । \newline
50. स्वेन॑ भाग॒धेये॑न भाग॒धेये॑न॒ स्वेन॒ स्वेन॑ भाग॒धेये॑न । \newline
51. भा॒ग॒धेये॒नोपोप॑ भाग॒धेये॑न भाग॒धेये॒नोप॑ । \newline
52. भा॒ग॒धेये॒नेति॑ भाग - धेये॑न । \newline
53. उप॑ धावति धाव॒ त्युपोप॑ धावति । \newline
54. धा॒व॒ति॒ स स धा॑वति धावति॒ सः । \newline
55. स ए॒वैव स स ए॒व । \newline
56. ए॒वास्मि॑न् नस्मिन् ने॒वैवास्मिन्न्॑ । \newline
57. अ॒स्मि॒न् ब्र॒ह्म॒व॒र्च॒सम् ब्र॑ह्मवर्च॒स म॑स्मिन् नस्मिन् ब्रह्मवर्च॒सम् । \newline
58. ब्र॒ह्म॒व॒र्च॒सम् द॑धाति दधाति ब्रह्मवर्च॒सम् ब्र॑ह्मवर्च॒सम् द॑धाति । \newline
59. ब्र॒ह्म॒व॒र्च॒समिति॑ ब्रह्म - व॒र्च॒सम् । \newline

\textbf{Ghana Paata } \newline

1. स॒जा॒ता विश्वा॒न्॒. विश्वा᳚न् थ्सजा॒ताः स॑जा॒ता विश्वा॑ ने॒वैव विश्वा᳚न् थ्सजा॒ताः स॑जा॒ता विश्वा॑ ने॒व । \newline
2. स॒जा॒ता इति॑ स - जा॒ताः । \newline
3. विश्वा॑ ने॒वैव विश्वा॒न्॒. विश्वा॑ ने॒व दे॒वान् दे॒वा ने॒व विश्वा॒न्॒. विश्वा॑ ने॒व दे॒वान् । \newline
4. ए॒व दे॒वान् दे॒वा ने॒वैव दे॒वान् थ्स्वेन॒ स्वेन॑ दे॒वा ने॒वैव दे॒वान् थ्स्वेन॑ । \newline
5. दे॒वान् थ्स्वेन॒ स्वेन॑ दे॒वान् दे॒वान् थ्स्वेन॑ भाग॒धेये॑न भाग॒धेये॑न॒ स्वेन॑ दे॒वान् दे॒वान् थ्स्वेन॑ भाग॒धेये॑न । \newline
6. स्वेन॑ भाग॒धेये॑न भाग॒धेये॑न॒ स्वेन॒ स्वेन॑ भाग॒धेये॒नो पोप॑ भाग॒धेये॑न॒ स्वेन॒ स्वेन॑ भाग॒धेये॒नोप॑ । \newline
7. भा॒ग॒धेये॒नो पोप॑ भाग॒धेये॑न भाग॒धेये॒नोप॑ धावति धाव॒त्युप॑ भाग॒धेये॑न भाग॒धेये॒नोप॑ धावति । \newline
8. भा॒ग॒धेये॒नेति॑ भाग - धेये॑न । \newline
9. उप॑ धावति धाव॒ त्युपोप॑ धावति॒ ते ते धा॑व॒ त्युपोप॑ धावति॒ ते । \newline
10. धा॒व॒ति॒ ते ते धा॑वति धावति॒ त ए॒वैव ते धा॑वति धावति॒ त ए॒व । \newline
11. त ए॒वैव ते त ए॒वास्मा॑ अस्मा ए॒व ते त ए॒वास्मै᳚ । \newline
12. ए॒वास्मा॑ अस्मा ए॒वैवास्मै॑ सजा॒तान् थ्स॑जा॒ता न॑स्मा ए॒वैवास्मै॑ सजा॒तान् । \newline
13. अ॒स्मै॒ स॒जा॒तान् थ्स॑जा॒ता न॑स्मा अस्मै सजा॒तान् प्र प्र स॑जा॒ता न॑स्मा अस्मै सजा॒तान् प्र । \newline
14. स॒जा॒तान् प्र प्र स॑जा॒तान् थ्स॑जा॒तान् प्र य॑च्छन्ति यच्छन्ति॒ प्र स॑जा॒तान् थ्स॑जा॒तान् प्र य॑च्छन्ति । \newline
15. स॒जा॒तानिति॑ स - जा॒तान् । \newline
16. प्र य॑च्छन्ति यच्छन्ति॒ प्र प्र य॑च्छन्ति ग्रा॒मी ग्रा॒मी य॑च्छन्ति॒ प्र प्र य॑च्छन्ति ग्रा॒मी । \newline
17. य॒च्छ॒न्ति॒ ग्रा॒मी ग्रा॒मी य॑च्छन्ति यच्छन्ति ग्रा॒म्ये॑वैव ग्रा॒मी य॑च्छन्ति यच्छन्ति ग्रा॒म्ये॑व । \newline
18. ग्रा॒म्ये॑वैव ग्रा॒मी ग्रा॒म्ये॑व भ॑वति भवत्ये॒व ग्रा॒मी ग्रा॒म्ये॑व भ॑वति । \newline
19. ए॒व भ॑वति भवत्ये॒वैव भ॑वति॒ छन्द॑सा॒म् छन्द॑साम् भवत्ये॒वैव भ॑वति॒ छन्द॑साम् । \newline
20. भ॒व॒ति॒ छन्द॑सा॒म् छन्द॑साम् भवति भवति॒ छन्द॑सां॒ ॅवै वै छन्द॑साम् भवति भवति॒ छन्द॑सां॒ ॅवै । \newline
21. छन्द॑सां॒ ॅवै वै छन्द॑सा॒म् छन्द॑सां॒ ॅवा ए॒ष ए॒ष वै छन्द॑सा॒म् छन्द॑सां॒ ॅवा ए॒षः । \newline
22. वा ए॒ष ए॒ष वै वा ए॒ष रसो॒ रस॑ ए॒ष वै वा ए॒ष रसः॑ । \newline
23. ए॒ष रसो॒ रस॑ ए॒ष ए॒ष रसो॒ यद् यद् रस॑ ए॒ष ए॒ष रसो॒ यत् । \newline
24. रसो॒ यद् यद् रसो॒ रसो॒ यद् व॒शा व॒शा यद् रसो॒ रसो॒ यद् व॒शा । \newline
25. यद् व॒शा व॒शा यद् यद् व॒शा रसो॒ रसो॑ व॒शा यद् यद् व॒शा रसः॑ । \newline
26. व॒शा रसो॒ रसो॑ व॒शा व॒शा रस॑ इवे व॒ रसो॑ व॒शा व॒शा रस॑ इव । \newline
27. रस॑ इवे व॒ रसो॒ रस॑ इव॒ खलु॒ खल्वि॑व॒ रसो॒ रस॑ इव॒ खलु॑ । \newline
28. इ॒व॒ खलु॒ खल्वि॑वे व॒ खलु॒ वै वै खल्वि॑वे व॒ खलु॒ वै । \newline
29. खलु॒ वै वै खलु॒ खलु॒ वै स॑जा॒ताः स॑जा॒ता वै खलु॒ खलु॒ वै स॑जा॒ताः । \newline
30. वै स॑जा॒ताः स॑जा॒ता वै वै स॑जा॒ता श्छन्द॑सा॒म् छन्द॑साꣳ सजा॒ता वै वै स॑जा॒ता श्छन्द॑साम् । \newline
31. स॒जा॒ता श्छन्द॑सा॒म् छन्द॑साꣳ सजा॒ताः स॑जा॒ता श्छन्द॑सा मे॒वैव छन्द॑साꣳ सजा॒ताः स॑जा॒ता श्छन्द॑सा मे॒व । \newline
32. स॒जा॒ता इति॑ स - जा॒ताः । \newline
33. छन्द॑सा मे॒वैव छन्द॑सा॒म् छन्द॑सा मे॒व रसे॑न॒ रसे॑नै॒व छन्द॑सा॒म् छन्द॑सा मे॒व रसे॑न । \newline
34. ए॒व रसे॑न॒ रसे॑नै॒वैव रसे॑न॒ रसꣳ॒॒ रसꣳ॒॒ रसे॑नै॒वैव रसे॑न॒ रस᳚म् । \newline
35. रसे॑न॒ रसꣳ॒॒ रसꣳ॒॒ रसे॑न॒ रसे॑न॒ रसꣳ॑ सजा॒तान् थ्स॑जा॒तान् रसꣳ॒॒ रसे॑न॒ रसे॑न॒ रसꣳ॑ सजा॒तान् । \newline
36. रसꣳ॑ सजा॒तान् थ्स॑जा॒तान् रसꣳ॒॒ रसꣳ॑ सजा॒ता नवाव॑ सजा॒तान् रसꣳ॒॒ रसꣳ॑ सजा॒ता नव॑ । \newline
37. स॒जा॒ता नवाव॑ सजा॒तान् थ्स॑जा॒ता नव॑ रुन्धे रु॒न्धे ऽव॑ सजा॒तान् थ्स॑जा॒ता नव॑ रुन्धे । \newline
38. स॒जा॒तानिति॑ स - जा॒तान् । \newline
39. अव॑ रुन्धे रु॒न्धे ऽवाव॑ रुन्धे बार्.हस्प॒त्यम् बा॑र्.हस्प॒त्यꣳ रु॒न्धे ऽवाव॑ रुन्धे बार्.हस्प॒त्यम् । \newline
40. रु॒न्धे॒ बा॒र्॒.ह॒स्प॒त्यम् बा॑र्.हस्प॒त्यꣳ रु॑न्धे रुन्धे बार्.हस्प॒त्य मु॑क्षव॒श मु॑क्षव॒शम् बा॑र्.हस्प॒त्यꣳ रु॑न्धे रुन्धे बार्.हस्प॒त्य मु॑क्षव॒शम् । \newline
41. बा॒र्॒.ह॒स्प॒त्य मु॑क्षव॒श मु॑क्षव॒शम् बा॑र्.हस्प॒त्यम् बा॑र्.हस्प॒त्य मु॑क्षव॒श मोक्ष॑व॒शम् बा॑र्.हस्प॒त्यम् बा॑र्.हस्प॒त्य मु॑क्षव॒श मा । \newline
42. उ॒क्ष॒व॒श मोक्ष॑व॒श मु॑क्षव॒श मा ल॑भेत लभे॒तोक्ष॑व॒श मु॑क्षव॒श मा ल॑भेत । \newline
43. उ॒क्ष॒व॒शमित्यु॑क्ष - व॒शम् । \newline
44. आ ल॑भेत लभे॒ता ल॑भेत ब्रह्मवर्च॒सका॑मो ब्रह्मवर्च॒सका॑मो लभे॒ता ल॑भेत ब्रह्मवर्च॒सका॑मः । \newline
45. ल॒भे॒त॒ ब्र॒ह्म॒व॒र्च॒सका॑मो ब्रह्मवर्च॒सका॑मो लभेत लभेत ब्रह्मवर्च॒सका॑मो॒ बृह॒स्पति॒म् बृह॒स्पति॑म् ब्रह्मवर्च॒सका॑मो लभेत लभेत ब्रह्मवर्च॒सका॑मो॒ बृह॒स्पति᳚म् । \newline
46. ब्र॒ह्म॒व॒र्च॒सका॑मो॒ बृह॒स्पति॒म् बृह॒स्पति॑म् ब्रह्मवर्च॒सका॑मो ब्रह्मवर्च॒सका॑मो॒ बृह॒स्पति॑ मे॒वैव बृह॒स्पति॑म् ब्रह्मवर्च॒सका॑मो ब्रह्मवर्च॒सका॑मो॒ बृह॒स्पति॑ मे॒व । \newline
47. ब्र॒ह्म॒व॒र्च॒सका॑म॒ इति॑ ब्रह्मवर्च॒स - का॒मः॒ । \newline
48. बृह॒स्पति॑ मे॒वैव बृह॒स्पति॒म् बृह॒स्पति॑ मे॒व स्वेन॒ स्वेनै॒व बृह॒स्पति॒म् बृह॒स्पति॑ मे॒व स्वेन॑ । \newline
49. ए॒व स्वेन॒ स्वेनै॒वैव स्वेन॑ भाग॒धेये॑न भाग॒धेये॑न॒ स्वेनै॒वैव स्वेन॑ भाग॒धेये॑न । \newline
50. स्वेन॑ भाग॒धेये॑न भाग॒धेये॑न॒ स्वेन॒ स्वेन॑ भाग॒धेये॒नो पोप॑ भाग॒धेये॑न॒ स्वेन॒ स्वेन॑ भाग॒धेये॒नोप॑ । \newline
51. भा॒ग॒धेये॒नो पोप॑ भाग॒धेये॑न भाग॒धेये॒नोप॑ धावति धाव॒त्युप॑ भाग॒धेये॑न भाग॒धेये॒नोप॑ धावति । \newline
52. भा॒ग॒धेये॒नेति॑ भाग - धेये॑न । \newline
53. उप॑ धावति धाव॒ त्युपोप॑ धावति॒ स स धा॑व॒ त्युपोप॑ धावति॒ सः । \newline
54. धा॒व॒ति॒ स स धा॑वति धावति॒ स ए॒वैव स धा॑वति धावति॒ स ए॒व । \newline
55. स ए॒वैव स स ए॒वास्मि॑न् नस्मिन् ने॒व स स ए॒वास्मिन्न्॑ । \newline
56. ए॒वास्मि॑न् नस्मिन् ने॒वैवास्मि॑न् ब्रह्मवर्च॒सम् ब्र॑ह्मवर्च॒स म॑स्मिन् ने॒वैवास्मि॑न् ब्रह्मवर्च॒सम् । \newline
57. अ॒स्मि॒न् ब्र॒ह्म॒व॒र्च॒सम् ब्र॑ह्मवर्च॒स म॑स्मिन् नस्मिन् ब्रह्मवर्च॒सम् द॑धाति दधाति ब्रह्मवर्च॒स म॑स्मिन् नस्मिन् ब्रह्मवर्च॒सम् द॑धाति । \newline
58. ब्र॒ह्म॒व॒र्च॒सम् द॑धाति दधाति ब्रह्मवर्च॒सम् ब्र॑ह्मवर्च॒सम् द॑धाति ब्रह्मवर्च॒सी ब्र॑ह्मवर्च॒सी द॑धाति ब्रह्मवर्च॒सम् ब्र॑ह्मवर्च॒सम् द॑धाति ब्रह्मवर्च॒सी । \newline
59. ब्र॒ह्म॒व॒र्च॒समिति॑ ब्रह्म - व॒र्च॒सम् । \newline
\pagebreak
\markright{ TS 2.1.7.7  \hfill https://www.vedavms.in \hfill}

\section{ TS 2.1.7.7 }

\textbf{TS 2.1.7.7 } \newline
\textbf{Samhita Paata} \newline

द॑धाति ब्रह्मवर्च॒स्ये॑व भ॑वति॒ वशं॒ ॅवा ए॒ष च॑रति॒ यदु॒क्षावश॑ इव॒ खलु॒ वै ब्र॑ह्मवर्च॒सं ॅवशे॑नै॒व वशं॑ ब्रह्मवर्च॒समव॑ रुन्धेरौ॒द्रीꣳरोहि॑णी॒मा ल॑भेताभि॒चर॑न् रु॒द्रमे॒व स्वेन॑ भाग॒धेये॒नोप॑ धावति॒ तस्मा॑ ए॒वैन॒मा वृ॑श्चति ता॒जगार्ति॒मार्च्छ॑ति॒ रोहि॑णी भवति रौ॒द्री ह्ये॑षा दे॒वत॑या॒ समृ॑द्ध्यै॒ स्फ्यो यूपो॑ ( ) भवति॒ वज्रो॒ वै स्फ्यो वज्र॑मे॒वास्मै॒ प्र ह॑रति शर॒मयं॑ ब॒र्॒.हिः शृ॒णात्ये॒वैनं॒ ॅवैभी॑दक इ॒द्ध्मो भि॒नत्त्ये॒वैनं᳚ ॥ \newline

\textbf{Pada Paata} \newline

द॒धा॒ति॒ । ब्र॒ह्म॒व॒र्च॒सीति॑ ब्रह्म - व॒र्च॒सी । ए॒व । भ॒व॒ति॒ । वश᳚म् । वै । ए॒षः । च॒र॒ति॒ । यत् । उ॒क्षा । वशः॑ । इ॒व॒ । खलु॑ । वै । ब्र॒ह्म॒व॒र्च॒समिति॑ ब्रह्म - व॒र्च॒सम् । वशे॑न । ए॒व । वश᳚म् । ब्र॒ह्म॒व॒र्च॒समिति॑ ब्रह्म - व॒र्च॒सम् । अवेति॑ । रु॒न्धे॒ । रौ॒द्रीम् । रोहि॑णीम् । एति॑ । ल॒भे॒त॒ । अ॒भि॒चर॒न्नित्य॑भि-चरन्॑ । रु॒द्रम् । ए॒व । स्वेन॑ । भा॒ग॒धेये॒नेति॑ भाग - धेये॑न । उपेति॑ । धा॒व॒ति॒ । तस्मै᳚ । ए॒व । ए॒न॒म् । एति॑ । वृ॒श्च॒ति॒ । ता॒जक् । आर्ति᳚म् । एति॑ । ऋ॒च्छ॒ति॒ । रोहि॑णी । भ॒व॒ति॒ । रौ॒द्री । हि । ए॒षा । दे॒वत॑या । समृ॑द्ध्या॒ इति॒ सं - ऋ॒द्ध्यै॒ । स्फ्यः । यूपः॑ ( ) । भ॒व॒ति॒ । वज्रः॑ । वै । स्फ्यः । वज्र᳚म् । ए॒व । अ॒स्मै॒ । प्रेति॑ । ह॒र॒ति॒ । श॒र॒मय॒मिति॑ शर - मय᳚म् । ब॒र्॒.हिः । शृ॒णाति॑ । ए॒व । ए॒न॒म् । वैभी॑दकः । इ॒द्ध्मः । भि॒नत्ति॑ । ए॒व । ए॒न॒म् ॥  \newline


\textbf{Krama Paata} \newline

द॒धा॒ति॒ ब्र॒ह्म॒व॒र्च॒सी । ब्र॒ह्म॒व॒र्च॒स्ये॑व । ब्र॒ह्म॒व॒र्च॒सीति॑ ब्रह्म - व॒र्च॒सी । ए॒व भ॑वति । भ॒व॒ति॒ वश᳚म् । वशं॒ ॅवै । वा ए॒षः । ए॒ष च॑रति । च॒र॒ति॒ यत् । यदु॒क्षा । उ॒क्षा वशः॑ । वश॑ इव । इ॒व॒ खलु॑ । खलु॒ वै । वै ब्र॑ह्मवर्च॒सम् । ब्र॒ह्म॒व॒र्च॒सम् ॅवशे॑न । ब्र॒ह्म॒व॒र्च॒समिति॑ ब्रह्म - व॒र्च॒सम् । वशे॑नै॒व । ए॒व वश᳚म् । वश॑म् ब्रह्मवर्च॒सम् । ब्र॒ह्म॒व॒र्च॒समव॑ । ब्र॒ह्म॒व॒र्च॒समिति॑ ब्रह्म - व॒र्च॒सम् । अव॑ रुन्धे । रु॒न्धे॒ रौ॒द्रीम् । रौ॒द्रीꣳ रोहि॑णीम् । रोहि॑णी॒मा । आ ल॑भेत । ल॒भे॒ता॒भि॒चरन्न्॑ । अ॒भि॒चर॑न् रु॒द्रम् । अ॒भि॒चर॒न्नित्य॑भि - चरन्न्॑ । रु॒द्रमे॒व । ए॒व स्वेन॑ । स्वेन॑ भाग॒धेये॑न । भा॒ग॒धेये॒नोप॑ । भा॒ग॒धेये॒नेति॑ भाग - धेये॑न । उप॑ धावति । धा॒व॒ति॒ तस्मै᳚ । तस्मा॑ ए॒व । ए॒वैन᳚म् । ए॒न॒मा । आ वृ॑श्चति । वृ॒श्च॒ति॒ ता॒जक् । ता॒जगार्ति᳚म् । आर्ति॒मा । आर्च्छ॑ति । ऋ॒च्छ॒ति॒ रोहि॑णी । रोहि॑णी भवति । भ॒व॒ति॒ रौ॒द्री । रौ॒द्री हि । ह्ये॑षा । ए॒षा दे॒वत॑या । दे॒वत॑या॒ समृ॑द्ध्यै । समृ॑द्ध्यै॒ स्फ्यः । समृ॑द्ध्या॒ इति॒ सं - ऋ॒द्ध्यै॒ । स्फ्यो यूपः॑ ( ) । यूपो॑ भवति । भ॒व॒ति॒ वज्रः॑ । वज्रो॒ वै । वै स्फ्यः । स्फ्यो वज्र᳚म् । वज्र॑मे॒व । ए॒वास्मै᳚ । अ॒स्मै॒ प्र । प्र ह॑रति । ह॒र॒ति॒ श॒र॒मय᳚म् । श॒र॒मय॑म् ब॒र्॒.हिः । श॒र॒मय॒मिति॑ शर - मय᳚म् । ब॒र्॒.हिः शृ॒णाति॑ । शृ॒णात्ये॒व । ए॒वैन᳚म् । ए॒नं॒ ॅवैभी॑दकः । वैभी॑दक इ॒ध्मः । इ॒ध्मो भि॒नत्ति॑ । भि॒नत्ये॒व । ए॒वैन᳚म् । ए॒न॒मित्ये॑नम् । \newline

\textbf{Jatai Paata} \newline

1. द॒धा॒ति॒ ब्र॒ह्म॒व॒र्च॒सी ब्र॑ह्मवर्च॒सी द॑धाति दधाति ब्रह्मवर्च॒सी । \newline
2. ब्र॒ह्म॒व॒र्च॒स्ये॑वैव ब्र॑ह्मवर्च॒सी ब्र॑ह्मवर्च॒स्ये॑व । \newline
3. ब्र॒ह्म॒व॒र्च॒सीति॑ ब्रह्म - व॒र्च॒सी । \newline
4. ए॒व भ॑वति भव त्ये॒वैव भ॑वति । \newline
5. भ॒व॒ति॒ वशं॒ ॅवश॑म् भवति भवति॒ वश᳚म् । \newline
6. वशं॒ ॅवै वै वशं॒ ॅवशं॒ ॅवै । \newline
7. वा ए॒ष ए॒ष वै वा ए॒षः । \newline
8. ए॒ष च॑रति चरत्ये॒ष ए॒ष च॑रति । \newline
9. च॒र॒ति॒ यद् यच् च॑रति चरति॒ यत् । \newline
10. यदु॒क्षोक्षा यद् यदु॒क्षा । \newline
11. उ॒क्षा वशो॒ वश॑ उ॒क्षोक्षा वशः॑ । \newline
12. वश॑ इवे व॒ वशो॒ वश॑ इव । \newline
13. इ॒व॒ खलु॒ खल्वि॑वे व॒ खलु॑ । \newline
14. खलु॒ वै वै खलु॒ खलु॒ वै । \newline
15. वै ब्र॑ह्मवर्च॒सम् ब्र॑ह्मवर्च॒सं ॅवै वै ब्र॑ह्मवर्च॒सम् । \newline
16. ब्र॒ह्म॒व॒र्च॒सं ॅवशे॑न॒ वशे॑न ब्रह्मवर्च॒सम् ब्र॑ह्मवर्च॒सं ॅवशे॑न । \newline
17. ब्र॒ह्म॒व॒र्च॒समिति॑ ब्रह्म - व॒र्च॒सम् । \newline
18. वशे॑नै॒वैव वशे॑न॒ वशे॑नै॒व । \newline
19. ए॒व वशं॒ ॅवश॑ मे॒वैव वश᳚म् । \newline
20. वश॑म् ब्रह्मवर्च॒सम् ब्र॑ह्मवर्च॒सं ॅवशं॒ ॅवश॑म् ब्रह्मवर्च॒सम् । \newline
21. ब्र॒ह्म॒व॒र्च॒स मवाव॑ ब्रह्मवर्च॒सम् ब्र॑ह्मवर्च॒स मव॑ । \newline
22. ब्र॒ह्म॒व॒र्च॒समिति॑ ब्रह्म - व॒र्च॒सम् । \newline
23. अव॑ रुन्धे रु॒न्धे ऽवाव॑ रुन्धे । \newline
24. रु॒न्धे॒ रौ॒द्रीꣳ रौ॒द्रीꣳ रु॑न्धे रुन्धे रौ॒द्रीम् । \newline
25. रौ॒द्रीꣳ रोहि॑णीꣳ॒॒ रोहि॑णीꣳ रौ॒द्रीꣳ रौ॒द्रीꣳ रोहि॑णीम् । \newline
26. रोहि॑णी॒ मा रोहि॑णीꣳ॒॒ रोहि॑णी॒ मा । \newline
27. आ ल॑भेत लभे॒ता ल॑भेत । \newline
28. ल॒भे॒ता॒ भि॒चर॑न् नभि॒चर॑न् ॅलभेत लभेता भि॒चरन्न्॑ । \newline
29. अ॒भि॒चर॑न् रु॒द्रꣳ रु॒द्र म॑भि॒चर॑न् नभि॒चर॑न् रु॒द्रम् । \newline
30. अ॒भि॒चर॒न्नित्य॑भि - चरन्न्॑ । \newline
31. रु॒द्र मे॒वैव रु॒द्रꣳ रु॒द्र मे॒व । \newline
32. ए॒व स्वेन॒ स्वेनै॒वैव स्वेन॑ । \newline
33. स्वेन॑ भाग॒धेये॑न भाग॒धेये॑न॒ स्वेन॒ स्वेन॑ भाग॒धेये॑न । \newline
34. भा॒ग॒धेये॒नोपोप॑ भाग॒धेये॑न भाग॒धेये॒नोप॑ । \newline
35. भा॒ग॒धेये॒नेति॑ भाग - धेये॑न । \newline
36. उप॑ धावति धाव॒ त्युपोप॑ धावति । \newline
37. धा॒व॒ति॒ तस्मै॒ तस्मै॑ धावति धावति॒ तस्मै᳚ । \newline
38. तस्मा॑ ए॒वैव तस्मै॒ तस्मा॑ ए॒व । \newline
39. ए॒वैन॑ मेन मे॒वैवैन᳚म् । \newline
40. ए॒न॒ मैन॑ मेन॒ मा । \newline
41. आ वृ॑श्चति वृश्च॒त्या वृ॑श्चति । \newline
42. वृ॒श्च॒ति॒ ता॒जक् ता॒जग् वृ॑श्चति वृश्चति ता॒जक् । \newline
43. ता॒जगार्ति॒ मार्ति॑म् ता॒जक् ता॒जगार्ति᳚म् । \newline
44. आर्ति॒ मा ऽऽर्ति॒ मार्ति॒ मा । \newline
45. आर्च्छ॑ त्यृच्छ॒ त्यार्च्छ॑ति । \newline
46. ऋ॒च्छ॒ति॒ रोहि॑णी॒ रोहि॑ण्यृच्छ त्यृच्छति॒ रोहि॑णी । \newline
47. रोहि॑णी भवति भवति॒ रोहि॑णी॒ रोहि॑णी भवति । \newline
48. भ॒व॒ति॒ रौ॒द्री रौ॒द्री भ॑वति भवति रौ॒द्री । \newline
49. रौ॒द्री हि हि रौ॒द्री रौ॒द्री हि । \newline
50. ह्ये॑षैषा हि ह्ये॑षा । \newline
51. ए॒षा दे॒वत॑या दे॒वत॑ यै॒षैषा दे॒वत॑या । \newline
52. दे॒वत॑या॒ समृ॑द्ध्यै॒ समृ॑द्ध्यै दे॒वत॑या दे॒वत॑या॒ समृ॑द्ध्यै । \newline
53. समृ॑द्ध्यै॒ स्फ्यः स्फ्यः समृ॑द्ध्यै॒ समृ॑द्ध्यै॒ स्फ्यः । \newline
54. समृ॑द्ध्या॒ इति॒ सं - ऋ॒द्ध्यै॒ । \newline
55. स्फ्यो यूपो॒ यूपः॒ स्फ्यः स्फ्यो यूपः॑ । \newline
56. यूपो॑ भवति भवति॒ यूपो॒ यूपो॑ भवति । \newline
57. भ॒व॒ति॒ वज्रो॒ वज्रो॑ भवति भवति॒ वज्रः॑ । \newline
58. वज्रो॒ वै वै वज्रो॒ वज्रो॒ वै । \newline
59. वै स्फ्यः स्फ्यो वै वै स्फ्यः । \newline
60. स्फ्यो वज्रं॒ ॅवज्रꣳ॒॒ स्फ्यः स्फ्यो वज्र᳚म् । \newline
61. वज्र॑ मे॒वैव वज्रं॒ ॅवज्र॑ मे॒व । \newline
62. ए॒वास्मा॑ अस्मा ए॒वैवास्मै᳚ । \newline
63. अ॒स्मै॒ प्र प्रास्मा॑ अस्मै॒ प्र । \newline
64. प्र ह॑रति हरति॒ प्र प्र ह॑रति । \newline
65. ह॒र॒ति॒ श॒र॒मयꣳ॑ शर॒मयꣳ॑ हरति हरति शर॒मय᳚म् । \newline
66. श॒र॒मय॑म् ब॒र्॒.हिर् ब॒र्॒.हिः श॑र॒मयꣳ॑ शर॒मय॑म् ब॒र्॒.हिः । \newline
67. श॒र॒मय॒मिति॑ शर - मय᳚म् । \newline
68. ब॒र्॒.हिः शृ॒णाति॑ शृ॒णाति॑ ब॒र्॒.हिर् ब॒र्॒.हिः शृ॒णाति॑ । \newline
69. शृ॒णा त्ये॒वैव शृ॒णाति॑ शृ॒णा त्ये॒व । \newline
70. ए॒वैन॑ मेन मे॒वैवैन᳚म् । \newline
71. ए॒नं॒ ॅवैभी॑दको॒ वैभी॑दक एन मेनं॒ ॅवैभी॑दकः । \newline
72. वैभी॑दक इ॒द्ध्म इ॒द्ध्मो वैभी॑दको॒ वैभी॑दक इ॒द्ध्मः । \newline
73. इ॒द्ध्मो भि॒नत्ति॑ भि॒नत्ती॒द्ध्म इ॒द्ध्मो भि॒नत्ति॑ । \newline
74. भि॒न त्त्ये॒वैव भि॒नत्ति॑ भि॒न त्त्ये॒व । \newline
75. ए॒वैन॑ मेन मे॒वैवैन᳚म् । \newline
76. ए॒न॒मित्ये॑नम् । \newline

\textbf{Ghana Paata } \newline

1. द॒धा॒ति॒ ब्र॒ह्म॒व॒र्च॒सी ब्र॑ह्मवर्च॒सी द॑धाति दधाति ब्रह्मवर्च॒ स्ये॑वैव ब्र॑ह्मवर्च॒सी द॑धाति दधाति ब्रह्मवर्च॒स्ये॑व । \newline
2. ब्र॒ह्म॒व॒र्च॒ स्ये॑वैव ब्र॑ह्मवर्च॒सी ब्र॑ह्मवर्च॒स्ये॑व भ॑वति भवत्ये॒व ब्र॑ह्मवर्च॒सी ब्र॑ह्मवर्च॒ स्ये॑व भ॑वति । \newline
3. ब्र॒ह्म॒व॒र्च॒सीति॑ ब्रह्म - व॒र्च॒सी । \newline
4. ए॒व भ॑वति भवत्ये॒वैव भ॑वति॒ वशं॒ ॅवश॑म् भवत्ये॒वैव भ॑वति॒ वश᳚म् । \newline
5. भ॒व॒ति॒ वशं॒ ॅवश॑म् भवति भवति॒ वशं॒ ॅवै वै वश॑म् भवति भवति॒ वशं॒ ॅवै । \newline
6. वशं॒ ॅवै वै वशं॒ ॅवशं॒ ॅवा ए॒ष ए॒ष वै वशं॒ ॅवशं॒ ॅवा ए॒षः । \newline
7. वा ए॒ष ए॒ष वै वा ए॒ष च॑रति चरत्ये॒ष वै वा ए॒ष च॑रति । \newline
8. ए॒ष च॑रति चरत्ये॒ष ए॒ष च॑रति॒ यद् यच् च॑रत्ये॒ष ए॒ष च॑रति॒ यत् । \newline
9. च॒र॒ति॒ यद् यच् च॑रति चरति॒ यदु॒क्षोक्षा यच् च॑रति चरति॒ यदु॒क्षा । \newline
10. यदु॒क्षोक्षा यद् यदु॒क्षा वशो॒ वश॑ उ॒क्षा यद् यदु॒क्षा वशः॑ । \newline
11. उ॒क्षा वशो॒ वश॑ उ॒क्षोक्षा वश॑ इवे व॒ वश॑ उ॒क्षोक्षा वश॑ इव । \newline
12. वश॑ इवे व॒ वशो॒ वश॑ इव॒ खलु॒ खल्वि॑व॒ वशो॒ वश॑ इव॒ खलु॑ । \newline
13. इ॒व॒ खलु॒ खल्वि॑वे व॒ खलु॒ वै वै खल्वि॑वे व॒ खलु॒ वै । \newline
14. खलु॒ वै वै खलु॒ खलु॒ वै ब्र॑ह्मवर्च॒सम् ब्र॑ह्मवर्च॒सं ॅवै खलु॒ खलु॒ वै ब्र॑ह्मवर्च॒सम् । \newline
15. वै ब्र॑ह्मवर्च॒सम् ब्र॑ह्मवर्च॒सं ॅवै वै ब्र॑ह्मवर्च॒सं ॅवशे॑न॒ वशे॑न ब्रह्मवर्च॒सं ॅवै वै ब्र॑ह्मवर्च॒सं ॅवशे॑न । \newline
16. ब्र॒ह्म॒व॒र्च॒सं ॅवशे॑न॒ वशे॑न ब्रह्मवर्च॒सम् ब्र॑ह्मवर्च॒सं ॅवशे॑नै॒वैव वशे॑न ब्रह्मवर्च॒सम् ब्र॑ह्मवर्च॒सं ॅवशे॑नै॒व । \newline
17. ब्र॒ह्म॒व॒र्च॒समिति॑ ब्रह्म - व॒र्च॒सम् । \newline
18. वशे॑नै॒वैव वशे॑न॒ वशे॑नै॒व वशं॒ ॅवश॑ मे॒व वशे॑न॒ वशे॑नै॒व वश᳚म् । \newline
19. ए॒व वशं॒ ॅवश॑ मे॒वैव वश॑म् ब्रह्मवर्च॒सम् ब्र॑ह्मवर्च॒सं ॅवश॑ मे॒वैव वश॑म् ब्रह्मवर्च॒सम् । \newline
20. वश॑म् ब्रह्मवर्च॒सम् ब्र॑ह्मवर्च॒सं ॅवशं॒ ॅवश॑म् ब्रह्मवर्च॒स मवाव॑ ब्रह्मवर्च॒सं ॅवशं॒ ॅवश॑म् ब्रह्मवर्च॒स मव॑ । \newline
21. ब्र॒ह्म॒व॒र्च॒स मवाव॑ ब्रह्मवर्च॒सम् ब्र॑ह्मवर्च॒स मव॑ रुन्धे रु॒न्धे ऽव॑ ब्रह्मवर्च॒सम् ब्र॑ह्मवर्च॒स मव॑ रुन्धे । \newline
22. ब्र॒ह्म॒व॒र्च॒समिति॑ ब्रह्म - व॒र्च॒सम् । \newline
23. अव॑ रुन्धे रु॒न्धे ऽवाव॑ रुन्धे रौ॒द्रीꣳ रौ॒द्रीꣳ रु॒न्धे ऽवाव॑ रुन्धे रौ॒द्रीम् । \newline
24. रु॒न्धे॒ रौ॒द्रीꣳ रौ॒द्रीꣳ रु॑न्धे रुन्धे रौ॒द्रीꣳ रोहि॑णीꣳ॒॒ रोहि॑णीꣳ रौ॒द्रीꣳ रु॑न्धे रुन्धे रौ॒द्रीꣳ रोहि॑णीम् । \newline
25. रौ॒द्रीꣳ रोहि॑णीꣳ॒॒ रोहि॑णीꣳ रौ॒द्रीꣳ रौ॒द्रीꣳ रोहि॑णी॒ मा रोहि॑णीꣳ रौ॒द्रीꣳ रौ॒द्रीꣳ रोहि॑णी॒ मा । \newline
26. रोहि॑णी॒ मा रोहि॑णीꣳ॒॒ रोहि॑णी॒ मा ल॑भेत लभे॒ता रोहि॑णीꣳ॒॒ रोहि॑णी॒ मा ल॑भेत । \newline
27. आ ल॑भेत लभे॒ता ल॑भेता भि॒चर॑न् नभि॒चर॑न् ॅलभे॒ता ल॑भेता भि॒चरन्न्॑ । \newline
28. ल॒भे॒ता॒ भि॒चर॑न् नभि॒चर॑न् ॅलभेत लभेता भि॒चर॑न् रु॒द्रꣳ रु॒द्र म॑भि॒चर॑न् ॅलभेत लभेता भि॒चर॑न् रु॒द्रम् । \newline
29. अ॒भि॒चर॑न् रु॒द्रꣳ रु॒द्र म॑भि॒चर॑न् नभि॒चर॑न् रु॒द्र मे॒वैव रु॒द्र म॑भि॒चर॑न् नभि॒चर॑न् रु॒द्र मे॒व । \newline
30. अ॒भि॒चर॒न्नित्य॑भि - चरन्न्॑ । \newline
31. रु॒द्र मे॒वैव रु॒द्रꣳ रु॒द्र मे॒व स्वेन॒ स्वेनै॒व रु॒द्रꣳ रु॒द्र मे॒व स्वेन॑ । \newline
32. ए॒व स्वेन॒ स्वेनै॒वैव स्वेन॑ भाग॒धेये॑न भाग॒धेये॑न॒ स्वेनै॒वैव स्वेन॑ भाग॒धेये॑न । \newline
33. स्वेन॑ भाग॒धेये॑न भाग॒धेये॑न॒ स्वेन॒ स्वेन॑ भाग॒धेये॒नो पोप॑ भाग॒धेये॑न॒ स्वेन॒ स्वेन॑ भाग॒धेये॒नोप॑ । \newline
34. भा॒ग॒धेये॒नो पोप॑ भाग॒धेये॑न भाग॒धेये॒नोप॑ धावति धाव॒त्युप॑ भाग॒धेये॑न भाग॒धेये॒नोप॑ धावति । \newline
35. भा॒ग॒धेये॒नेति॑ भाग - धेये॑न । \newline
36. उप॑ धावति धाव॒ त्युपोप॑ धावति॒ तस्मै॒ तस्मै॑ धाव॒ त्युपोप॑ धावति॒ तस्मै᳚ । \newline
37. धा॒व॒ति॒ तस्मै॒ तस्मै॑ धावति धावति॒ तस्मा॑ ए॒वैव तस्मै॑ धावति धावति॒ तस्मा॑ ए॒व । \newline
38. तस्मा॑ ए॒वैव तस्मै॒ तस्मा॑ ए॒वैन॑ मेन मे॒व तस्मै॒ तस्मा॑ ए॒वैन᳚म् । \newline
39. ए॒वैन॑ मेन मे॒वैवैन॒ मैन॑ मे॒वैवैन॒ मा । \newline
40. ए॒न॒ मैन॑ मेन॒ मा वृ॑श्चति वृश्च॒त्यैन॑ मेन॒ मा वृ॑श्चति । \newline
41. आ वृ॑श्चति वृश्च॒त्या वृ॑श्चति ता॒जक् ता॒जग् वृ॑श्च॒त्या वृ॑श्चति ता॒जक् । \newline
42. वृ॒श्च॒ति॒ ता॒जक् ता॒जग् वृ॑श्चति वृश्चति ता॒जगार्ति॒ मार्ति॑म् ता॒जग् वृ॑श्चति वृश्चति ता॒जगार्ति᳚म् । \newline
43. ता॒जगार्ति॒ मार्ति॑म् ता॒जक् ता॒जगार्ति॒ मा ऽऽर्ति॑म् ता॒जक् ता॒जगार्ति॒ मा । \newline
44. आर्ति॒ मा ऽऽर्ति॒ मार्ति॒ मार्च्छ॑ त्यृच्छ॒त्या ऽऽर्ति॒ मार्ति॒ मार्च्छ॑ति । \newline
45. आर्च्छ॑ त्यृच्छ त्यार्च्छति॒ रोहि॑णी॒ रोहि॑ण्यृच्छ त्यार्च्छति॒ रोहि॑णी । \newline
46. ऋ॒च्छ॒ति॒ रोहि॑णी॒ रोहि॑ण्यृच्छ॒ त्यृच्छ॑ति॒ रोहि॑णी भवति भवति॒ रोहि॑ण्यृच्छ॒ त्यृच्छति॒॑ रोहि॑णी भवति । \newline
47. रोहि॑णी भवति भवति॒ रोहि॑णी॒ रोहि॑णी भवति रौ॒द्री रौ॒द्री भ॑वति॒ रोहि॑णी॒ रोहि॑णी भवति रौ॒द्री । \newline
48. भ॒व॒ति॒ रौ॒द्री रौ॒द्री भ॑वति भवति रौ॒द्री हि हि रौ॒द्री भ॑वति भवति रौ॒द्री हि । \newline
49. रौ॒द्री हि हि रौ॒द्री रौ॒द्री ह्ये॑षैषा हि रौ॒द्री रौ॒द्री ह्ये॑षा । \newline
50. ह्ये॑षैषा हि ह्ये॑षा दे॒वत॑या दे॒वत॑यै॒षा हि ह्ये॑षा दे॒वत॑या । \newline
51. ए॒षा दे॒वत॑या दे॒वत॑यै॒षैषा दे॒वत॑या॒ समृ॑द्ध्यै॒ समृ॑द्ध्यै दे॒वत॑यै॒षैषा दे॒वत॑या॒ समृ॑द्ध्यै । \newline
52. दे॒वत॑या॒ समृ॑द्ध्यै॒ समृ॑द्ध्यै दे॒वत॑या दे॒वत॑या॒ समृ॑द्ध्यै॒ स्फ्यः स्फ्यः समृ॑द्ध्यै दे॒वत॑या दे॒वत॑या॒ समृ॑द्ध्यै॒ स्फ्यः । \newline
53. समृ॑द्ध्यै॒ स्फ्यः स्फ्यः समृ॑द्ध्यै॒ समृ॑द्ध्यै॒ स्फ्यो यूपो॒ यूपः॒ स्फ्यः समृ॑द्ध्यै॒ समृ॑द्ध्यै॒ स्फ्यो यूपः॑ । \newline
54. समृ॑द्ध्या॒ इति॒ सं - ऋ॒द्ध्यै॒ । \newline
55. स्फ्यो यूपो॒ यूपः॒ स्फ्यः स्फ्यो यूपो॑ भवति भवति॒ यूपः॒ स्फ्यः स्फ्यो यूपो॑ भवति । \newline
56. यूपो॑ भवति भवति॒ यूपो॒ यूपो॑ भवति॒ वज्रो॒ वज्रो॑ भवति॒ यूपो॒ यूपो॑ भवति॒ वज्रः॑ । \newline
57. भ॒व॒ति॒ वज्रो॒ वज्रो॑ भवति भवति॒ वज्रो॒ वै वै वज्रो॑ भवति भवति॒ वज्रो॒ वै । \newline
58. वज्रो॒ वै वै वज्रो॒ वज्रो॒ वै स्फ्यः स्फ्यो वै वज्रो॒ वज्रो॒ वै स्फ्यः । \newline
59. वै स्फ्यः स्फ्यो वै वै स्फ्यो वज्रं॒ ॅवज्रꣳ॒॒ स्फ्यो वै वै स्फ्यो वज्र᳚म् । \newline
60. स्फ्यो वज्रं॒ ॅवज्रꣳ॒॒ स्फ्यः स्फ्यो वज्र॑ मे॒वैव वज्रꣳ॒॒ स्फ्यः स्फ्यो वज्र॑ मे॒व । \newline
61. वज्र॑ मे॒वैव वज्रं॒ ॅवज्र॑ मे॒वास्मा॑ अस्मा ए॒व वज्रं॒ ॅवज्र॑ मे॒वास्मै᳚ । \newline
62. ए॒वास्मा॑ अस्मा ए॒वैवास्मै॒ प्र प्रास्मा॑ ए॒वैवास्मै॒ प्र । \newline
63. अ॒स्मै॒ प्र प्रास्मा॑ अस्मै॒ प्र ह॑रति हरति॒ प्रास्मा॑ अस्मै॒ प्र ह॑रति । \newline
64. प्र ह॑रति हरति॒ प्र प्र ह॑रति शर॒मयꣳ॑ शर॒मयꣳ॑ हरति॒ प्र प्र ह॑रति शर॒मय᳚म् । \newline
65. ह॒र॒ति॒ श॒र॒मयꣳ॑ शर॒मयꣳ॑ हरति हरति शर॒मय॑म् ब॒र्॒.हिर् ब॒र्॒.हिः श॑र॒मयꣳ॑ हरति हरति शर॒मय॑म् ब॒र्॒.हिः । \newline
66. श॒र॒मय॑म् ब॒र्॒.हिर् ब॒र्॒.हिः श॑र॒मयꣳ॑ शर॒मय॑म् ब॒र्॒.हिः शृ॒णाति॑ शृ॒णाति॑ ब॒र्॒.हिः श॑र॒मयꣳ॑ शर॒मय॑म् ब॒र्॒.हिः शृ॒णाति॑ । \newline
67. श॒र॒मय॒मिति॑ शर - मय᳚म् । \newline
68. ब॒र्॒.हिः शृ॒णाति॑ शृ॒णाति॑ ब॒र्॒.हिर् ब॒र्॒.हिः शृ॒णा त्ये॒वैव शृ॒णाति॑ ब॒र्॒.हिर् ब॒र्॒.हिः शृ॒णात्ये॒व । \newline
69. शृ॒णा त्ये॒वैव शृ॒णाति॑ शृ॒णा त्ये॒वैन॑ मेन मे॒व शृ॒णाति॑ शृ॒णा त्ये॒वैन᳚म् । \newline
70. ए॒वैन॑ मेन मे॒वैवैनं॒ ॅवैभी॑दको॒ वैभी॑दक एन मे॒वैवैनं॒ ॅवैभी॑दकः । \newline
71. ए॒नं॒ ॅवैभी॑दको॒ वैभी॑दक एन मेनं॒ ॅवैभी॑दक इ॒द्ध्म इ॒द्ध्मो वैभी॑दक एन मेनं॒ ॅवैभी॑दक इ॒द्ध्मः । \newline
72. वैभी॑दक इ॒द्ध्म इ॒द्ध्मो वैभी॑दको॒ वैभी॑दक इ॒द्ध्मो भि॒नत्ति॑ भि॒नत्ती॒द्ध्मो वैभी॑दको॒ वैभी॑दक इ॒द्ध्मो भि॒नत्ति॑ । \newline
73. इ॒द्ध्मो भि॒नत्ति॑ भि॒नत्ती॒द्ध्म इ॒द्ध्मो भि॒न त्त्ये॒वैव भि॒नत्ती॒द्ध्म इ॒द्ध्मो भि॒न त्त्ये॒व । \newline
74. भि॒नत्त्ये॒वैव भि॒नत्ति॑ भि॒न त्त्ये॒वैन॑ मेन मे॒व भि॒नत्ति॑ भि॒न त्त्ये॒वैन᳚म् । \newline
75. ए॒वैन॑ मेन मे॒वैवैन᳚म् । \newline
76. ए॒न॒मित्ये॑नम् । \newline
\pagebreak
\markright{ TS 2.1.8.1  \hfill https://www.vedavms.in \hfill}

\section{ TS 2.1.8.1 }

\textbf{TS 2.1.8.1 } \newline
\textbf{Samhita Paata} \newline

अ॒सावा॑दि॒त्यो न व्य॑रोचत॒ तस्मै॑ दे॒वाः प्राय॑श्चित्तिमैच्छ॒न् तस्मा॑ ए॒ताꣳ सौ॒रीꣳ श्वे॒तां ॅव॒शामाऽल॑भन्त॒ तयै॒वास्मि॒न् रुच॑मदधु॒र्यो ब्र॑ह्मवर्च॒सका॑मः॒ स्यात् तस्मा॑ ए॒ताꣳ सौ॒रीꣳ श्वे॒तां ॅव॒शामा ल॑भेता॒मुमे॒वा ऽऽ*दि॒त्यꣳ स्वेन॑ भाग॒धेये॒नोप॑ धावति॒ स ए॒वास्मि॑न् ब्रह्मवर्च॒सं द॑धाति ब्रह्मवर्च॒स्ये॑व भ॑वति बै॒ल्॒.वो यूपो॑ भवत्य॒सौ-  [  ] \newline

\textbf{Pada Paata} \newline

अ॒सौ । आ॒दि॒त्यः । न । वीति॑ । अ॒रो॒च॒त॒ । तस्मै᳚ । दे॒वाः । प्राय॑श्चित्तिम् । ऐ॒च्छ॒न्न् । तस्मै᳚ । ए॒ताम् । सौ॒रीम् । श्वे॒ताम् । व॒शाम् । एति॑ । अ॒ल॒भ॒न्त॒ । तया᳚ । ए॒व । अ॒स्मि॒न्न् । रुच᳚म् । अ॒द॒धुः॒ । यः । ब्र॒ह्म॒व॒र्च॒सका॑म॒ इति॑ ब्रह्मवर्च॒स - का॒मः॒ । स्यात् । तस्मै᳚ । ए॒ताम् । सौ॒रीम् । श्वे॒ताम् । व॒शाम् । एति॑ । ल॒भे॒त॒ । अ॒मुम् । ए॒व । आ॒दि॒त्यम् । स्वेन॑ । भा॒ग॒धेये॒नेति॑ भाग - धेये॑न । उपेति॑ । धा॒व॒ति॒ । सः । ए॒व । अ॒स्मि॒न्न् । ब॒ह्म॒व॒र्च॒समिति॑ ब्रह्म - व॒र्च॒सम् । द॒धा॒ति॒ । ब्र॒ह्म॒व॒र्च॒सीति॑ ब्रह्म - व॒र्च॒सी । ए॒व । भ॒व॒ति॒ । बै॒ल्॒.वः । यूपः॑ । भ॒व॒ति॒ । अ॒सौ ।  \newline


\textbf{Krama Paata} \newline

अ॒सावा॑दि॒त्यः । आ॒दि॒त्यो न । न वि । व्य॑रोचत । अ॒रो॒च॒त॒ तस्मै᳚ । तस्मै॑ दे॒वाः । दे॒वाः प्राय॑श्चित्तिम् । प्राय॑श्चित्तिमैच्छन्न् । ऐ॒च्छ॒न् तस्मै᳚ । तस्मा॑ ए॒ताम् । ए॒ताꣳ सौ॒रीम् । सौ॒रीꣳ श्वे॒ताम् । श्वे॒तां ॅव॒शाम् । व॒शामा । आ ऽल॑भन्त । अ॒ल॒भ॒न्त॒ तया᳚ । तयै॒व । ए॒वास्मिन्न्॑ । अ॒स्मि॒न् रुच᳚म् । रुच॑मदधुः । अ॒द॒धु॒र् यः । यो ब्र॑ह्मवर्च॒सका॑मः । ब्र॒ह्म॒व॒र्च॒सका॑मः॒ स्यात् । ब्र॒ह्म॒व॒र्च॒सका॑म॒ इति॑ ब्रह्मवर्च॒स - का॒मः॒ । स्यात् तस्मै᳚ । तस्मा॑ ए॒ताम् । ए॒ताꣳ सौ॒रीम् । सौ॒रीꣳ श्वे॒ताम् । श्वे॒तां ॅव॒शाम् । व॒शामा । आ ल॑भेत । ल॒भे॒ता॒मुम् । अ॒मुमे॒व । ए॒वादि॒त्यम् । आ॒दि॒त्यꣳ स्वेन॑ । स्वेन॑ भाग॒धेये॑न । भा॒ग॒धेये॒नोप॑ । भा॒ग॒धेये॒नेति॑ भाग - धेये॑न । उप॑ धावति । धा॒व॒ति॒ सः । स ए॒व । ए॒वास्मिन्न्॑ । अ॒स्मि॒न् ब्र॒ह॒व॒र्च॒सम् । ब्र॒ह्म॒व॒र्च॒सम् द॑धाति । ब्र॒ह्म॒व॒र्च॒समिति॑ ब्रह्म - व॒र्च॒सम् । द॒धा॒ति॒ ब्र॒ह्म॒व॒र्च॒सी । ब्र॒ह्म॒व॒र्च॒स्ये॑व । ब्र॒ह्म॒व॒र्च॒सीति॑ ब्रह्म - व॒र्च॒सी । ए॒व भ॑वति । भ॒व॒ति॒ बै॒ल्॒.वः । बै॒ल्॒.वो यूपः॑ । यूपो॑ भवति । भ॒व॒त्य॒सौ । अ॒सौ वै \newline

\textbf{Jatai Paata} \newline

1. अ॒सा वा॑दि॒त्य आ॑दि॒त्यो॑ ऽसा व॒सा वा॑दि॒त्यः । \newline
2. आ॒दि॒त्यो न नादि॒त्य आ॑दि॒त्यो न । \newline
3. न वि वि न न वि । \newline
4. व्य॑रोचता रोचत॒ वि व्य॑रोचत । \newline
5. अ॒रो॒च॒त॒ तस्मै॒ तस्मा॑ अरोचता रोचत॒ तस्मै᳚ । \newline
6. तस्मै॑ दे॒वा दे॒वा स्तस्मै॒ तस्मै॑ दे॒वाः । \newline
7. दे॒वाः प्राय॑श्चित्ति॒म् प्राय॑श्चित्तिम् दे॒वा दे॒वाः प्राय॑श्चित्तिम् । \newline
8. प्राय॑श्चित्ति मैच्छन् नैच्छ॒न् प्राय॑श्चित्ति॒म् प्राय॑श्चित्ति मैच्छन्न् । \newline
9. ऐ॒च्छ॒न् तस्मै॒ तस्मा॑ ऐच्छन् नैच्छ॒न् तस्मै᳚ । \newline
10. तस्मा॑ ए॒ता मे॒ताम् तस्मै॒ तस्मा॑ ए॒ताम् । \newline
11. ए॒ताꣳ सौ॒रीꣳ सौ॒री मे॒ता मे॒ताꣳ सौ॒रीम् । \newline
12. सौ॒रीꣳ श्वे॒ताꣳ श्वे॒ताꣳ सौ॒रीꣳ सौ॒रीꣳ श्वे॒ताम् । \newline
13. श्वे॒तां ॅव॒शां ॅव॒शाꣳ श्वे॒ताꣳ श्वे॒तां ॅव॒शाम् । \newline
14. व॒शा मा व॒शां ॅव॒शा मा । \newline
15. आ ऽल॑भन्ता लभ॒न्ता ऽल॑भन्त । \newline
16. अ॒ल॒भ॒न्त॒ तया॒ तया॑ ऽलभन्ता लभन्त॒ तया᳚ । \newline
17. तयै॒वैव तया॒ तयै॒व । \newline
18. ए॒वास्मि॑न् नस्मिन् ने॒वैवास्मिन्न्॑ । \newline
19. अ॒स्मि॒न् रुचꣳ॒॒ रुच॑ मस्मिन् नस्मि॒न् रुच᳚म् । \newline
20. रुच॑ मदधु रदधू॒ रुचꣳ॒॒ रुच॑ मदधुः । \newline
21. अ॒द॒धु॒र् यो यो॑ ऽदधु रदधु॒र् यः । \newline
22. यो ब्र॑ह्मवर्च॒सका॑मो ब्रह्मवर्च॒सका॑मो॒ यो यो ब्र॑ह्मवर्च॒सका॑मः । \newline
23. ब्र॒ह्म॒व॒र्च॒सका॑मः॒ स्याथ् स्याद् ब्र॑ह्मवर्च॒सका॑मो ब्रह्मवर्च॒सका॑मः॒ स्यात् । \newline
24. ब्र॒ह्म॒व॒र्च॒सका॑म॒ इति॑ ब्रह्मवर्च॒स - का॒मः॒ । \newline
25. स्यात् तस्मै॒ तस्मै॒ स्याथ् स्यात् तस्मै᳚ । \newline
26. तस्मा॑ ए॒ता मे॒ताम् तस्मै॒ तस्मा॑ ए॒ताम् । \newline
27. ए॒ताꣳ सौ॒रीꣳ सौ॒री मे॒ता मे॒ताꣳ सौ॒रीम् । \newline
28. सौ॒रीꣳ श्वे॒ताꣳ श्वे॒ताꣳ सौ॒रीꣳ सौ॒रीꣳ श्वे॒ताम् । \newline
29. श्वे॒तां ॅव॒शां ॅव॒शाꣳ श्वे॒ताꣳ श्वे॒तां ॅव॒शाम् । \newline
30. व॒शा मा व॒शां ॅव॒शा मा । \newline
31. आ ल॑भेत लभे॒ता ल॑भेत । \newline
32. ल॒भे॒ता॒मु म॒मुम् ॅल॑भेत लभेता॒मुम् । \newline
33. अ॒मु मे॒वैवामु म॒मु मे॒व । \newline
34. ए॒वादि॒त्य मा॑दि॒त्य मे॒वैवादि॒त्यम् । \newline
35. आ॒दि॒त्यꣳ स्वेन॒ स्वेना॑दि॒त्य मा॑दि॒त्यꣳ स्वेन॑ । \newline
36. स्वेन॑ भाग॒धेये॑न भाग॒धेये॑न॒ स्वेन॒ स्वेन॑ भाग॒धेये॑न । \newline
37. भा॒ग॒धेये॒नोपोप॑ भाग॒धेये॑न भाग॒धेये॒नोप॑ । \newline
38. भा॒ग॒धेये॒नेति॑ भाग - धेये॑न । \newline
39. उप॑ धावति धाव॒ त्युपोप॑ धावति । \newline
40. धा॒व॒ति॒ स स धा॑वति धावति॒ सः । \newline
41. स ए॒वैव स स ए॒व । \newline
42. ए॒वास्मि॑न् नस्मिन् ने॒वैवास्मिन्न्॑ । \newline
43. अ॒स्मि॒न् ब्र॒ह्म॒व॒र्च॒सम् ब्र॑ह्मवर्च॒स म॑स्मिन् नस्मिन् ब्रह्मवर्च॒सम् । \newline
44. ब्र॒ह्म॒व॒र्च॒सम् द॑धाति दधाति ब्रह्मवर्च॒सम् ब्र॑ह्मवर्च॒सम् द॑धाति । \newline
45. ब्र॒ह्म॒व॒र्च॒समिति॑ ब्रह्म - व॒र्च॒सम् । \newline
46. द॒धा॒ति॒ ब्र॒ह्म॒व॒र्च॒सी ब्र॑ह्मवर्च॒सी द॑धाति दधाति ब्रह्मवर्च॒सी । \newline
47. ब्र॒ह्म॒व॒र्च॒स्ये॑वैव ब्र॑ह्मवर्च॒सी ब्र॑ह्मवर्च॒स्ये॑व । \newline
48. ब्र॒ह्म॒व॒र्च॒सीति॑ ब्रह्म - व॒र्च॒सी । \newline
49. ए॒व भ॑वति भव त्ये॒वैव भ॑वति । \newline
50. भ॒व॒ति॒ बै॒ल्॒.वो बै॒ल्॒.वो भ॑वति भवति बै॒ल्॒.वः । \newline
51. बै॒ल्॒.वो यूपो॒ यूपो॑ बै॒ल्॒.वो बै॒ल्॒.वो यूपः॑ । \newline
52. यूपो॑ भवति भवति॒ यूपो॒ यूपो॑ भवति । \newline
53. भ॒व॒ त्य॒सा व॒सौ भ॑वति भव त्य॒सौ । \newline
54. अ॒सौ वै वा अ॒सा व॒सौ वै । \newline

\textbf{Ghana Paata } \newline

1. अ॒सा वा॑दि॒त्य आ॑दि॒त्यो॑ ऽसा व॒सा वा॑दि॒त्यो न नादि॒त्यो॑ ऽसा व॒सा वा॑दि॒त्यो न । \newline
2. आ॒दि॒त्यो न नादि॒त्य आ॑दि॒त्यो न वि वि नादि॒त्य आ॑दि॒त्यो न वि । \newline
3. न वि वि न न व्य॑रोचता रोचत॒ वि न न व्य॑रोचत । \newline
4. व्य॑रोचता रोचत॒ वि व्य॑रोचत॒ तस्मै॒ तस्मा॑ अरोचत॒ वि व्य॑रोचत॒ तस्मै᳚ । \newline
5. अ॒रो॒च॒त॒ तस्मै॒ तस्मा॑ अरोचता रोचत॒ तस्मै॑ दे॒वा दे॒वा स्तस्मा॑ अरोचता रोचत॒ तस्मै॑ दे॒वाः । \newline
6. तस्मै॑ दे॒वा दे॒वा स्तस्मै॒ तस्मै॑ दे॒वाः प्राय॑श्चित्ति॒म् प्राय॑श्चित्तिम् दे॒वा स्तस्मै॒ तस्मै॑ दे॒वाः प्राय॑श्चित्तिम् । \newline
7. दे॒वाः प्राय॑श्चित्ति॒म् प्राय॑श्चित्तिम् दे॒वा दे॒वाः प्राय॑श्चित्ति मैच्छन् नैच्छ॒न् प्राय॑श्चित्तिम् दे॒वा दे॒वाः प्राय॑श्चित्ति मैच्छन्न् । \newline
8. प्राय॑श्चित्ति मैच्छन् नैच्छ॒न् प्राय॑श्चित्ति॒म् प्राय॑श्चित्ति मैच्छ॒न् तस्मै॒ तस्मा॑ ऐच्छ॒न् प्राय॑श्चित्ति॒म् प्राय॑श्चित्ति मैच्छ॒न् तस्मै᳚ । \newline
9. ऐ॒च्छ॒न् तस्मै॒ तस्मा॑ ऐच्छन् नैच्छ॒न् तस्मा॑ ए॒ता मे॒ताम् तस्मा॑ ऐच्छन् नैच्छ॒न् तस्मा॑ ए॒ताम् । \newline
10. तस्मा॑ ए॒ता मे॒ताम् तस्मै॒ तस्मा॑ ए॒ताꣳ सौ॒रीꣳ सौ॒री मे॒ताम् तस्मै॒ तस्मा॑ ए॒ताꣳ सौ॒रीम् । \newline
11. ए॒ताꣳ सौ॒रीꣳ सौ॒री मे॒ता मे॒ताꣳ सौ॒रीꣳ श्वे॒ताꣳ श्वे॒ताꣳ सौ॒री मे॒ता मे॒ताꣳ सौ॒रीꣳ श्वे॒ताम् । \newline
12. सौ॒रीꣳ श्वे॒ताꣳ श्वे॒ताꣳ सौ॒रीꣳ सौ॒रीꣳ श्वे॒तां ॅव॒शां ॅव॒शाꣳ श्वे॒ताꣳ सौ॒रीꣳ सौ॒रीꣳ श्वे॒तां ॅव॒शाम् । \newline
13. श्वे॒तां ॅव॒शां ॅव॒शाꣳ श्वे॒ताꣳ श्वे॒तां ॅव॒शा मा व॒शाꣳ श्वे॒ताꣳ श्वे॒तां ॅव॒शा मा । \newline
14. व॒शा मा व॒शां ॅव॒शा मा ऽल॑भन्ता लभ॒न्ता व॒शां ॅव॒शा मा ऽल॑भन्त । \newline
15. आ ऽल॑भन्ता लभ॒न्ता ऽल॑भन्त॒ तया॒ तया॑ ऽलभ॒न्ता ऽल॑भन्त॒ तया᳚ । \newline
16. अ॒ल॒भ॒न्त॒ तया॒ तया॑ ऽलभन्ता लभन्त॒ तयै॒वैव तया॑ ऽलभन्ता लभन्त॒ तयै॒व । \newline
17. तयै॒वैव तया॒ तयै॒वास्मि॑न् नस्मिन् ने॒व तया॒ तयै॒वास्मिन्न्॑ । \newline
18. ए॒वास्मि॑न् नस्मिन् ने॒वैवास्मि॒न् रुचꣳ॒॒ रुच॑ मस्मिन् ने॒वैवास्मि॒न् रुच᳚म् । \newline
19. अ॒स्मि॒न् रुचꣳ॒॒ रुच॑ मस्मिन् नस्मि॒न् रुच॑ मदधु रदधू॒ रुच॑ मस्मिन् नस्मि॒न् रुच॑ मदधुः । \newline
20. रुच॑ मदधु रदधू॒ रुचꣳ॒॒ रुच॑ मदधु॒र् यो यो॑ ऽदधू॒ रुचꣳ॒॒ रुच॑ मदधु॒र् यः । \newline
21. अ॒द॒धु॒र् यो यो॑ ऽदधुरदधु॒र् यो ब्र॑ह्मवर्च॒सका॑मो ब्रह्मवर्च॒सका॑मो॒ यो॑ ऽदधुरदधु॒र् यो ब्र॑ह्मवर्च॒सका॑मः । \newline
22. यो ब्र॑ह्मवर्च॒सका॑मो ब्रह्मवर्च॒सका॑मो॒ यो यो ब्र॑ह्मवर्च॒सका॑मः॒ स्याथ् स्याद् ब्र॑ह्मवर्च॒सका॑मो॒ यो यो ब्र॑ह्मवर्च॒सका॑मः॒ स्यात् । \newline
23. ब्र॒ह्म॒व॒र्च॒सका॑मः॒ स्याथ् स्याद् ब्र॑ह्मवर्च॒सका॑मो ब्रह्मवर्च॒सका॑मः॒ स्यात् तस्मै॒ तस्मै॒ स्याद् ब्र॑ह्मवर्च॒सका॑मो ब्रह्मवर्च॒सका॑मः॒ स्यात् तस्मै᳚ । \newline
24. ब्र॒ह्म॒व॒र्च॒सका॑म॒ इति॑ ब्रह्मवर्च॒स - का॒मः॒ । \newline
25. स्यात् तस्मै॒ तस्मै॒ स्याथ् स्यात् तस्मा॑ ए॒ता मे॒ताम् तस्मै॒ स्याथ् स्यात् तस्मा॑ ए॒ताम् । \newline
26. तस्मा॑ ए॒ता मे॒ताम् तस्मै॒ तस्मा॑ ए॒ताꣳ सौ॒रीꣳ सौ॒री मे॒ताम् तस्मै॒ तस्मा॑ ए॒ताꣳ सौ॒रीम् । \newline
27. ए॒ताꣳ सौ॒रीꣳ सौ॒री मे॒ता मे॒ताꣳ सौ॒रीꣳ श्वे॒ताꣳ श्वे॒ताꣳ सौ॒री मे॒ता मे॒ताꣳ सौ॒रीꣳ श्वे॒ताम् । \newline
28. सौ॒रीꣳ श्वे॒ताꣳ श्वे॒ताꣳ सौ॒रीꣳ सौ॒रीꣳ श्वे॒तां ॅव॒शां ॅव॒शाꣳ श्वे॒ताꣳ सौ॒रीꣳ सौ॒रीꣳ श्वे॒तां ॅव॒शाम् । \newline
29. श्वे॒तां ॅव॒शां ॅव॒शाꣳ श्वे॒ताꣳ श्वे॒तां ॅव॒शा मा व॒शाꣳ श्वे॒ताꣳ श्वे॒तां ॅव॒शा मा । \newline
30. व॒शा मा व॒शां ॅव॒शा मा ल॑भेत लभे॒ता व॒शां ॅव॒शा मा ल॑भेत । \newline
31. आ ल॑भेत लभे॒ता ल॑भेता॒मु म॒मुम् ॅल॑भे॒ता ल॑भेता॒मुम् । \newline
32. ल॒भे॒ता॒मु म॒मुम् ॅल॑भेत लभेता॒मु मे॒वैवामुम् ॅल॑भेत लभेता॒मु मे॒व । \newline
33. अ॒मु मे॒वैवामु म॒मु मे॒वादि॒त्य मा॑दि॒त्य मे॒वामु म॒मु मे॒वादि॒त्यम् । \newline
34. ए॒वादि॒त्य मा॑दि॒त्य मे॒वै वादि॒त्यꣳ स्वेन॒ स्वेना॑दि॒त्य मे॒वै वादि॒त्यꣳ स्वेन॑ । \newline
35. आ॒दि॒त्यꣳ स्वेन॒ स्वेना॑दि॒त्य मा॑दि॒त्यꣳ स्वेन॑ भाग॒धेये॑न भाग॒धेये॑न॒ स्वेना॑दि॒त्य मा॑दि॒त्यꣳ स्वेन॑ भाग॒धेये॑न । \newline
36. स्वेन॑ भाग॒धेये॑न भाग॒धेये॑न॒ स्वेन॒ स्वेन॑ भाग॒धेये॒नो पोप॑ भाग॒धेये॑न॒ स्वेन॒ स्वेन॑ भाग॒धेये॒नोप॑ । \newline
37. भा॒ग॒धेये॒नो पोप॑ भाग॒धेये॑न भाग॒धेये॒नोप॑ धावति धाव॒त्युप॑ भाग॒धेये॑न भाग॒धेये॒नोप॑ धावति । \newline
38. भा॒ग॒धेये॒नेति॑ भाग - धेये॑न । \newline
39. उप॑ धावति धाव॒ त्युपोप॑ धावति॒ स स धा॑व॒ त्युपोप॑ धावति॒ सः । \newline
40. धा॒व॒ति॒ स स धा॑वति धावति॒ स ए॒वैव स धा॑वति धावति॒ स ए॒व । \newline
41. स ए॒वैव स स ए॒वास्मि॑न् नस्मिन् ने॒व स स ए॒वास्मिन्न्॑ । \newline
42. ए॒वास्मि॑न् नस्मिन् ने॒वैवास्मि॑न् ब्रह्मवर्च॒सम् ब्र॑ह्मवर्च॒स म॑स्मिन् ने॒वैवास्मि॑न् ब्रह्मवर्च॒सम् । \newline
43. अ॒स्मि॒न् ब्र॒ह्म॒व॒र्च॒सम् ब्र॑ह्मवर्च॒स म॑स्मिन् नस्मिन् ब्रह्मवर्च॒सम् द॑धाति दधाति ब्रह्मवर्च॒स म॑स्मिन् नस्मिन् ब्रह्मवर्च॒सम् द॑धाति । \newline
44. ब्र॒ह्म॒व॒र्च॒सम् द॑धाति दधाति ब्रह्मवर्च॒सम् ब्र॑ह्मवर्च॒सम् द॑धाति ब्रह्मवर्च॒सी ब्र॑ह्मवर्च॒सी द॑धाति ब्रह्मवर्च॒सम् ब्र॑ह्मवर्च॒सम् द॑धाति ब्रह्मवर्च॒सी । \newline
45. ब्र॒ह्म॒व॒र्च॒समिति॑ ब्रह्म - व॒र्च॒सम् । \newline
46. द॒धा॒ति॒ ब्र॒ह्म॒व॒र्च॒सी ब्र॑ह्मवर्च॒सी द॑धाति दधाति ब्रह्मवर्च॒स्ये॑वैव ब्र॑ह्मवर्च॒सी द॑धाति दधाति ब्रह्मवर्च॒स्ये॑व । \newline
47. ब्र॒ह्म॒व॒र्च॒स्ये॑वैव ब्र॑ह्मवर्च॒सी ब्र॑ह्मवर्च॒स्ये॑व भ॑वति भवत्ये॒व ब्र॑ह्मवर्च॒सी ब्र॑ह्मवर्च॒स्ये॑व भ॑वति । \newline
48. ब्र॒ह्म॒व॒र्च॒सीति॑ ब्रह्म - व॒र्च॒सी । \newline
49. ए॒व भ॑वति भवत्ये॒वैव भ॑वति बै॒ल्॒.वो बै॒ल्॒.वो भवत्ये॒वैव भ॑वति बै॒ल्॒.वः । \newline
50. भ॒व॒ति॒ बै॒ल्.॒वो बै॒ल्॒.वो भ॑वति भवति बै॒ल्॒.वो यूपो॒ यूपो॑ बै॒ल्॒.वो भ॑वति भवति बै॒ल्॒.वो यूपः॑ । \newline
51. बै॒ल्॒.वो यूपो॒ यूपो॑ बै॒ल्॒.वो बै॒ल्॒.वो यूपो॑ भवति भवति॒ यूपो॑ बै॒ल्॒.वो बै॒ल्॒.वो यूपो॑ भवति । \newline
52. यूपो॑ भवति भवति॒ यूपो॒ यूपो॑ भवत्य॒सा व॒सौ भ॑वति॒ यूपो॒ यूपो॑ भवत्य॒सौ । \newline
53. भ॒व॒त्य॒सा व॒सौ भ॑वति भवत्य॒सौ वै वा अ॒सौ भ॑वति भवत्य॒सौ वै । \newline
54. अ॒सौ वै वा अ॒सा व॒सौ वा आ॑दि॒त्य आ॑दि॒त्यो वा अ॒सा व॒सौ वा आ॑दि॒त्यः । \newline
\pagebreak
\markright{ TS 2.1.8.2  \hfill https://www.vedavms.in \hfill}

\section{ TS 2.1.8.2 }

\textbf{TS 2.1.8.2 } \newline
\textbf{Samhita Paata} \newline

वा आ॑दि॒त्यो यतोऽजा॑यत॒ ततो॑ बि॒ल्व॑ उद॑तिष्ठ॒थ् सयो᳚न्ये॒व ब्र॑ह्मवर्च॒समव॑ रुन्धे ब्राह्मणस्प॒त्यां ब॑भ्रुक॒र्णीमा ल॑भेता-भि॒चर॑न्. वारु॒णं दश॑कपालं पु॒रस्ता॒न्-निर्व॑पे॒द्-वरु॑णेनै॒व भ्रातृ॑व्यं ग्राहयि॒त्वा ब्रह्म॑णा स्तृणुते बभ्रुक॒र्णी भ॑वत्ये॒तद्वै ब्रह्म॑णो रू॒पꣳ समृ॑द्ध्‌यै॒ स्फ्यो यूपो॑ भवति॒ वज्रो॒ वै स्फ्यो वज्र॑मे॒वास्मै॒ प्र ह॑रति शर॒मयं॑ ब॒र्॒.हिः शृ॒णा - [  ] \newline

\textbf{Pada Paata} \newline

वै । आ॒दि॒त्यः । यतः॑ । अजा॑यत । ततः॑ । बि॒ल्वः॑ । उदिति॑ । अ॒ति॒ष्ठ॒त् । सयो॒नीति॒ स - यो॒नि॒ । ए॒व । ब्र॒ह्म॒व॒र्च॒समिति॑ ब्रह्म - व॒र्च॒सम् । अवेति॑ । रु॒न्धे॒ । ब्रा॒ह्म॒ण॒स्प॒त्यामिति॑ ब्राह्मणः - प॒त्याम् । ब॒भ्रु॒क॒र्णीमिति॑ बभ्रु - क॒र्णीम् । एति॑ । ल॒भे॒त॒ । अ॒भि॒चर॒न्नित्य॑भि - चरन्॑ । वा॒रु॒णम् । दश॑कपाल॒मिति॒ दश॑ - क॒पा॒ल॒म् । पु॒रस्ता᳚त् । निरिति॑ । व॒पे॒त् । वरु॑णेन । ए॒व । भ्रातृ॑व्यम् । ग्रा॒ह॒यि॒त्वा । ब्रह्म॑णा । स्तृ॒णु॒ते॒ । ब॒भ्रु॒क॒र्णीति॑ बभ्रु - क॒र्णी । भ॒व॒ति॒ । ए॒तत् । वै । ब्रह्म॑णः । रू॒पम् । समृ॑द्ध्या॒ इति॒ सं-ऋ॒द्ध्यै॒ । स्फ्यः । यूपः॑ । भ॒व॒ति॒ । वज्रः॑ । वै । स्फ्यः । वज्र᳚म् । ए॒व । अ॒स्मै॒ । प्रेति॑ । ह॒र॒ति॒ । श॒र॒मय॒मिति॑ शर - मय᳚म् । ब॒र्॒.हिः । शृ॒णाति॑ ।  \newline


\textbf{Krama Paata} \newline

वा आ॑दि॒त्यः । आ॒दि॒त्यो यतः॑ । यतो ऽजा॑यत । अजा॑यत॒ ततः॑ । ततो॑ बि॒ल्वः॑ । बि॒ल्व॑ उत् । उद॑तिष्ठत् । अ॒ति॒ष्ठ॒थ् सयो॑नि । सयो᳚न्ये॒व । सयो॒नीति॒ स - यो॒नि॒ । ए॒व ब्र॑ह्मवर्च॒सम् । ब्र॒ह्म॒व॒र्च॒समव॑ । ब्र॒ह्म॒व॒र्च॒समिति॑ ब्रह्म - व॒र्च॒सम् । अव॑ रुन्धे । रु॒न्धे॒ ब्रा॒ह्म॒ण॒स्प॒त्याम् । ब्रा॒ह्म॒ण॒स्प॒त्याम् ब॑भ्रुक॒र्णीम् । ब्रा॒ह्म॒ण॒स्प॒त्यामिति॑ ब्राह्मणः - प॒त्याम् । ब॒भ्रु॒क॒र्णीमा । ब॒भ्रु॒क॒र्णीमिति॑ बभ्रु - क॒र्णीम् । आ ल॑भेत । ल॒भे॒ता॒भि॒चरन्न्॑ । अ॒भि॒चर॑न्. वारु॒णम् । अ॒भि॒चर॒न्नित्य॑भि - चरन्न्॑ । वा॒रु॒णम् दश॑कपालम् । दश॑कपालम् पु॒रस्ता᳚त् । दश॑कपाल॒मिति॒ दश॑ - क॒पा॒ल॒म् । पु॒रस्ता॒न् निः । निर् व॑पेत् । व॒पे॒द् वरु॑णेन । वरु॑णेनै॒व । ए॒व भ्रातृ॑व्यम् । भ्रातृ॑व्यम् ग्राहयि॒त्वा । ग्रा॒ह॒यि॒त्वा ब्रह्म॑णा । ब्रह्म॑णा स्तृणुते । स्तृ॒णु॒ते॒ ब॒भ्रु॒क॒र्णी । ब॒भ्रु॒क॒र्णी भ॑वति । ब॒भ्रु॒क॒र्णीति॑ बभ्रु - क॒र्णी । भ॒व॒त्ये॒तत् । ए॒तद् वै । वै ब्रह्म॑णः । ब्रह्म॑णो रू॒पम् । रू॒पꣳ समृ॑द्ध्यै । समृ॑द्ध्यै॒ स्फ्यः । समृ॑द्ध्या॒ इति॒ सं - ऋ॒द्ध्यै॒ । स्फ्यो यूपः॑ । यूपो॑ भवति । भ॒व॒ति॒ वज्रः॑ । वज्रो॒ वै । वै स्फ्यः । स्फ्यो वज्र᳚म् । वज्र॑मे॒व । ए॒वास्मै᳚ । अ॒स्मै॒ प्र । प्र ह॑रति । ह॒र॒ति॒ श॒र॒मय᳚म् । श॒र॒मय॑म् ब॒र्.॒हिः । श॒र॒मय॒मिति॑ शर - मय᳚म् । ब॒र्॒.हिः शृ॒णाति॑ । शृ॒णात्ये॒व \newline

\textbf{Jatai Paata} \newline

1. वा आ॑दि॒त्य आ॑दि॒त्यो वै वा आ॑दि॒त्यः । \newline
2. आ॒दि॒त्यो यतो॒ यत॑ आदि॒त्य आ॑दि॒त्यो यतः॑ । \newline
3. यतो ऽजा॑य॒ता जा॑यत॒ यतो॒ यतो ऽजा॑यत । \newline
4. अजा॑यत॒ तत॒ स्ततो ऽजा॑य॒ता जा॑यत॒ ततः॑ । \newline
5. ततो॑ बि॒ल्वो॑ बि॒ल्व॑ स्तत॒ स्ततो॑ बि॒ल्वः॑ । \newline
6. बि॒ल्व॑ उदुद् बि॒ल्वो॑ बि॒ल्व॑ उत् । \newline
7. उद॑तिष्ठ दतिष्ठ॒ दुदु द॑तिष्ठत् । \newline
8. अ॒ति॒ष्ठ॒थ् सयो॑नि॒ सयो᳚ न्यतिष्ठ दतिष्ठ॒थ् सयो॑नि । \newline
9. सयो᳚न्ये॒वैव सयो॑नि॒ सयो᳚न्ये॒व । \newline
10. सयो॒नीति॒ स - यो॒नि॒ । \newline
11. ए॒व ब्र॑ह्मवर्च॒सम् ब्र॑ह्मवर्च॒स मे॒वैव ब्र॑ह्मवर्च॒सम् । \newline
12. ब्र॒ह्म॒व॒र्च॒स मवाव॑ ब्रह्मवर्च॒सम् ब्र॑ह्मवर्च॒स मव॑ । \newline
13. ब्र॒ह्म॒व॒र्च॒समिति॑ ब्रह्म - व॒र्च॒सम् । \newline
14. अव॑ रुन्धे रु॒न्धे ऽवाव॑ रुन्धे । \newline
15. रु॒न्धे॒ ब्रा॒ह्म॒ण॒स्प॒त्याम् ब्रा᳚ह्मणस्प॒त्याꣳ रु॑न्धे रुन्धे ब्राह्मणस्प॒त्याम् । \newline
16. ब्रा॒ह्म॒ण॒स्प॒त्याम् ब॑भ्रुक॒र्णीम् ब॑भ्रुक॒र्णीम् ब्रा᳚ह्मणस्प॒त्याम् ब्रा᳚ह्मणस्प॒त्याम् ब॑भ्रुक॒र्णीम् । \newline
17. ब्रा॒ह्म॒ण॒स्प॒त्यामिति॑ ब्राह्मणः - प॒त्याम् । \newline
18. ब॒भ्रु॒क॒र्णी मा ब॑भ्रुक॒र्णीम् ब॑भ्रुक॒र्णी मा । \newline
19. ब॒भ्रु॒क॒र्णीमिति॑ बभ्रु - क॒र्णीम् । \newline
20. आ ल॑भेत लभे॒ता ल॑भेत । \newline
21. ल॒भे॒ता॒ भि॒चर॑न् नभि॒चर॑न् ॅलभेत लभेता भि॒चरन्न्॑ । \newline
22. अ॒भि॒चर॑न् वारु॒णं ॅवा॑रु॒ण म॑भि॒चर॑न् नभि॒चर॑न् वारु॒णम् । \newline
23. अ॒भि॒चर॒न्नित्य॑भि - चरन्न्॑ । \newline
24. वा॒रु॒णम् दश॑कपाल॒म् दश॑कपालं ॅवारु॒णं ॅवा॑रु॒णम् दश॑कपालम् । \newline
25. दश॑कपालम् पु॒रस्ता᳚त् पु॒रस्ता॒द् दश॑कपाल॒म् दश॑कपालम् पु॒रस्ता᳚त् । \newline
26. दश॑कपाल॒मिति॒ दश॑ - क॒पा॒ल॒म् । \newline
27. पु॒रस्ता॒न् निर् णिष् पु॒रस्ता᳚त् पु॒रस्ता॒न् निः । \newline
28. निर् व॑पेद् वपे॒न् निर् णिर् व॑पेत् । \newline
29. व॒पे॒द् वरु॑णेन॒ वरु॑णेन वपेद् वपे॒द् वरु॑णेन । \newline
30. वरु॑णे नै॒वैव वरु॑णेन॒ वरु॑णे नै॒व । \newline
31. ए॒व भ्रातृ॑व्य॒म् भ्रातृ॑व्य मे॒वैव भ्रातृ॑व्यम् । \newline
32. भ्रातृ॑व्यम् ग्राहयि॒त्वा ग्रा॑हयि॒त्वा भ्रातृ॑व्य॒म् भ्रातृ॑व्यम् ग्राहयि॒त्वा । \newline
33. ग्रा॒ह॒यि॒त्वा ब्रह्म॑णा॒ ब्रह्म॑णा ग्राहयि॒त्वा ग्रा॑हयि॒त्वा ब्रह्म॑णा । \newline
34. ब्रह्म॑णा स्तृणुते स्तृणुते॒ ब्रह्म॑णा॒ ब्रह्म॑णा स्तृणुते । \newline
35. स्तृ॒णु॒ते॒ ब॒भ्रु॒क॒र्णी ब॑भ्रुक॒र्णी स्तृ॑णुते स्तृणुते बभ्रुक॒र्णी । \newline
36. ब॒भ्रु॒क॒र्णी भ॑वति भवति बभ्रुक॒र्णी ब॑भ्रुक॒र्णी भ॑वति । \newline
37. ब॒भ्रु॒क॒र्णीति॑ बभ्रु - क॒र्णी । \newline
38. भ॒व॒ त्ये॒त दे॒तद् भ॑वति भव त्ये॒तत् । \newline
39. ए॒तद् वै वा ए॒त दे॒तद् वै । \newline
40. वै ब्रह्म॑णो॒ ब्रह्म॑णो॒ वै वै ब्रह्म॑णः । \newline
41. ब्रह्म॑णो रू॒पꣳ रू॒पम् ब्रह्म॑णो॒ ब्रह्म॑णो रू॒पम् । \newline
42. रू॒पꣳ समृ॑द्ध्यै॒ समृ॑द्ध्यै रू॒पꣳ रू॒पꣳ समृ॑द्ध्यै । \newline
43. समृ॑द्ध्यै॒ स्फ्यः स्फ्यः समृ॑द्ध्यै॒ समृ॑द्ध्यै॒ स्फ्यः । \newline
44. समृ॑द्ध्या॒ इति॒ सं - ऋ॒द्ध्यै॒ । \newline
45. स्फ्यो यूपो॒ यूपः॒ स्फ्यः स्फ्यो यूपः॑ । \newline
46. यूपो॑ भवति भवति॒ यूपो॒ यूपो॑ भवति । \newline
47. भ॒व॒ति॒ वज्रो॒ वज्रो॑ भवति भवति॒ वज्रः॑ । \newline
48. वज्रो॒ वै वै वज्रो॒ वज्रो॒ वै । \newline
49. वै स्फ्यः स्फ्यो वै वै स्फ्यः । \newline
50. स्फ्यो वज्रं॒ ॅवज्रꣳ॒॒ स्फ्यः स्फ्यो वज्र᳚म् । \newline
51. वज्र॑ मे॒वैव वज्रं॒ ॅवज्र॑ मे॒व । \newline
52. ए॒वास्मा॑ अस्मा ए॒वैवास्मै᳚ । \newline
53. अ॒स्मै॒ प्र प्रास्मा॑ अस्मै॒ प्र । \newline
54. प्र ह॑रति हरति॒ प्र प्र ह॑रति । \newline
55. ह॒र॒ति॒ श॒र॒मयꣳ॑ शर॒मयꣳ॑ हरति हरति शर॒मय᳚म् । \newline
56. श॒र॒मय॑म् ब॒र्॒.हिर् ब॒र्॒.हिः श॑र॒मयꣳ॑ शर॒मय॑म् ब॒र्॒.हिः । \newline
57. श॒र॒मय॒मिति॑ शर - मय᳚म् । \newline
58. ब॒र्॒.हिः शृ॒णाति॑ शृ॒णाति॑ ब॒र्॒.हिर् ब॒र्॒.हिः शृ॒णाति॑ । \newline
59. शृ॒णा त्ये॒वैव शृ॒णाति॑ शृ॒णा त्ये॒व । \newline

\textbf{Ghana Paata } \newline

1. वा आ॑दि॒त्य आ॑दि॒त्यो वै वा आ॑दि॒त्यो यतो॒ यत॑ आदि॒त्यो वै वा आ॑दि॒त्यो यतः॑ । \newline
2. आ॒दि॒त्यो यतो॒ यत॑ आदि॒त्य आ॑दि॒त्यो यतो ऽजा॑य॒ता जा॑यत॒ यत॑ आदि॒त्य आ॑दि॒त्यो यतो ऽजा॑यत । \newline
3. यतो ऽजा॑य॒ता जा॑यत॒ यतो॒ यतो ऽजा॑यत॒ तत॒ स्ततो ऽजा॑यत॒ यतो॒ यतो ऽजा॑यत॒ ततः॑ । \newline
4. अजा॑यत॒ तत॒ स्ततो ऽजा॑य॒ता जा॑यत॒ ततो॑ बि॒ल्वो॑ बि॒ल्व॑ स्ततो ऽजा॑य॒ता जा॑यत॒ ततो॑ बि॒ल्वः॑ । \newline
5. ततो॑ बि॒ल्वो॑ बि॒ल्व॑ स्तत॒ स्ततो॑ बि॒ल्व॑ उदुद् बि॒ल्व॑ स्तत॒ स्ततो॑ बि॒ल्व॑ उत् । \newline
6. बि॒ल्व॑ उदुद् बि॒ल्वो॑ बि॒ल्व॑ उद॑तिष्ठ दतिष्ठ॒दुद् बि॒ल्वो॑ बि॒ल्व॑ उद॑तिष्ठत् । \newline
7. उद॑तिष्ठ दतिष्ठ॒दुदु द॑तिष्ठ॒थ् सयो॑नि॒ सयो᳚न्य तिष्ठ॒दुदु द॑तिष्ठ॒थ् सयो॑नि । \newline
8. अ॒ति॒ष्ठ॒थ् सयो॑नि॒ सयो᳚न्यतिष्ठ दतिष्ठ॒थ् सयो᳚न्ये॒वैव सयो᳚न्यतिष्ठ दतिष्ठ॒थ् सयो᳚न्ये॒व । \newline
9. सयो᳚न्ये॒वैव सयो॑नि॒ सयो᳚न्ये॒व ब्र॑ह्मवर्च॒सम् ब्र॑ह्मवर्च॒स मे॒व सयो॑नि॒ सयो᳚न्ये॒व ब्र॑ह्मवर्च॒सम् । \newline
10. सयो॒नीति॒ स - यो॒नि॒ । \newline
11. ए॒व ब्र॑ह्मवर्च॒सम् ब्र॑ह्मवर्च॒स मे॒वैव ब्र॑ह्मवर्च॒स मवाव॑ ब्रह्मवर्च॒स मे॒वैव ब्र॑ह्मवर्च॒स मव॑ । \newline
12. ब्र॒ह्म॒व॒र्च॒स मवाव॑ ब्रह्मवर्च॒सम् ब्र॑ह्मवर्च॒स मव॑ रुन्धे रु॒न्धे ऽव॑ ब्रह्मवर्च॒सम् ब्र॑ह्मवर्च॒स मव॑ रुन्धे । \newline
13. ब्र॒ह्म॒व॒र्च॒समिति॑ ब्रह्म - व॒र्च॒सम् । \newline
14. अव॑ रुन्धे रु॒न्धे ऽवाव॑ रुन्धे ब्राह्मणस्प॒त्याम् ब्रा᳚ह्मणस्प॒त्याꣳ रु॒न्धे ऽवाव॑ रुन्धे ब्राह्मणस्प॒त्याम् । \newline
15. रु॒न्धे॒ ब्रा॒ह्म॒ण॒स्प॒त्याम् ब्रा᳚ह्मणस्प॒त्याꣳ रु॑न्धे रुन्धे ब्राह्मणस्प॒त्याम् ब॑भ्रुक॒र्णीम् ब॑भ्रुक॒र्णीम् ब्रा᳚ह्मणस्प॒त्याꣳ रु॑न्धे रुन्धे ब्राह्मणस्प॒त्याम् ब॑भ्रुक॒र्णीम् । \newline
16. ब्रा॒ह्म॒ण॒स्प॒त्याम् ब॑भ्रुक॒र्णीम् ब॑भ्रुक॒र्णीम् ब्रा᳚ह्मणस्प॒त्याम् ब्रा᳚ह्मणस्प॒त्याम् ब॑भ्रुक॒र्णी मा ब॑भ्रुक॒र्णीम् ब्रा᳚ह्मणस्प॒त्याम् ब्रा᳚ह्मणस्प॒त्याम् ब॑भ्रुक॒र्णी मा । \newline
17. ब्रा॒ह्म॒ण॒स्प॒त्यामिति॑ ब्राह्मणः - प॒त्याम् । \newline
18. ब॒भ्रु॒क॒र्णी मा ब॑भ्रुक॒र्णीम् ब॑भ्रुक॒र्णी मा ल॑भेत लभे॒ता ब॑भ्रुक॒र्णीम् ब॑भ्रुक॒र्णी मा ल॑भेत । \newline
19. ब॒भ्रु॒क॒र्णीमिति॑ बभ्रु - क॒र्णीम् । \newline
20. आ ल॑भेत लभे॒ता ल॑भेताभि॒चर॑न् नभि॒चर॑न् ॅलभे॒ता ल॑भेताभि॒चरन्न्॑ । \newline
21. ल॒भे॒ता॒भि॒चर॑न् नभि॒चर॑न् ॅलभेत लभेताभि॒चर॑न् वारु॒णं ॅवा॑रु॒ण म॑भि॒चर॑न् ॅलभेत लभेताभि॒चर॑न् वारु॒णम् । \newline
22. अ॒भि॒चर॑न् वारु॒णं ॅवा॑रु॒ण म॑भि॒चर॑न् नभि॒चर॑न् वारु॒णम् दश॑कपाल॒म् दश॑कपालं ॅवारु॒ण म॑भि॒चर॑न् नभि॒चर॑न् वारु॒णम् दश॑कपालम् । \newline
23. अ॒भि॒चर॒न्नित्य॑भि - चरन्न्॑ । \newline
24. वा॒रु॒णम् दश॑कपाल॒म् दश॑कपालं ॅवारु॒णं ॅवा॑रु॒णम् दश॑कपालम् पु॒रस्ता᳚त् पु॒रस्ता॒द् दश॑कपालं ॅवारु॒णं ॅवा॑रु॒णम् दश॑कपालम् पु॒रस्ता᳚त् । \newline
25. दश॑कपालम् पु॒रस्ता᳚त् पु॒रस्ता॒द् दश॑कपाल॒म् दश॑कपालम् पु॒रस्ता॒न् निर् णिष् पु॒रस्ता॒द् दश॑कपाल॒म् दश॑कपालम् पु॒रस्ता॒न् निः । \newline
26. दश॑कपाल॒मिति॒ दश॑ - क॒पा॒ल॒म् । \newline
27. पु॒रस्ता॒न् निर् णिष् पु॒रस्ता᳚त् पु॒रस्ता॒न् निर् व॑पेद् वपे॒न् निष् पु॒रस्ता᳚त् पु॒रस्ता॒न् निर् व॑पेत् । \newline
28. निर् व॑पेद् वपे॒न् निर् णिर् व॑पे॒द् वरु॑णेन॒ वरु॑णेन वपे॒न् निर् णिर् व॑पे॒द् वरु॑णेन । \newline
29. व॒पे॒द् वरु॑णेन॒ वरु॑णेन वपेद् वपे॒द् वरु॑णेनै॒वैव वरु॑णेन वपेद् वपे॒द् वरु॑णेनै॒व । \newline
30. वरु॑णेनै॒वैव वरु॑णेन॒ वरु॑णेनै॒व भ्रातृ॑व्य॒म् भ्रातृ॑व्य मे॒व वरु॑णेन॒ वरु॑णेनै॒व भ्रातृ॑व्यम् । \newline
31. ए॒व भ्रातृ॑व्य॒म् भ्रातृ॑व्य मे॒वैव भ्रातृ॑व्यम् ग्राहयि॒त्वा ग्रा॑हयि॒त्वा भ्रातृ॑व्य मे॒वैव भ्रातृ॑व्यम् ग्राहयि॒त्वा । \newline
32. भ्रातृ॑व्यम् ग्राहयि॒त्वा ग्रा॑हयि॒त्वा भ्रातृ॑व्य॒म् भ्रातृ॑व्यम् ग्राहयि॒त्वा ब्रह्म॑णा॒ ब्रह्म॑णा ग्राहयि॒त्वा भ्रातृ॑व्य॒म् भ्रातृ॑व्यम् ग्राहयि॒त्वा ब्रह्म॑णा । \newline
33. ग्रा॒ह॒यि॒त्वा ब्रह्म॑णा॒ ब्रह्म॑णा ग्राहयि॒त्वा ग्रा॑हयि॒त्वा ब्रह्म॑णा स्तृणुते स्तृणुते॒ ब्रह्म॑णा ग्राहयि॒त्वा ग्रा॑हयि॒त्वा ब्रह्म॑णा स्तृणुते । \newline
34. ब्रह्म॑णा स्तृणुते स्तृणुते॒ ब्रह्म॑णा॒ ब्रह्म॑णा स्तृणुते बभ्रुक॒र्णी ब॑भ्रुक॒र्णी स्तृ॑णुते॒ ब्रह्म॑णा॒ ब्रह्म॑णा स्तृणुते बभ्रुक॒र्णी । \newline
35. स्तृ॒णु॒ते॒ ब॒भ्रु॒क॒र्णी ब॑भ्रुक॒र्णी स्तृ॑णुते स्तृणुते बभ्रुक॒र्णी भ॑वति भवति बभ्रुक॒र्णी स्तृ॑णुते स्तृणुते बभ्रुक॒र्णी भ॑वति । \newline
36. ब॒भ्रु॒क॒र्णी भ॑वति भवति बभ्रुक॒र्णी ब॑भ्रुक॒र्णी भ॑वत्ये॒तदे॒तद् भ॑वति बभ्रुक॒र्णी ब॑भ्रुक॒र्णी भ॑वत्ये॒तत् । \newline
37. ब॒भ्रु॒क॒र्णीति॑ बभ्रु - क॒र्णी । \newline
38. भ॒व॒ त्ये॒त दे॒तद् भ॑वति भवत्ये॒तद् वै वा ए॒तद् भ॑वति भवत्ये॒तद् वै । \newline
39. ए॒तद् वै वा ए॒तदे॒तद् वै ब्रह्म॑णो॒ ब्रह्म॑णो॒ वा ए॒तदे॒तद् वै ब्रह्म॑णः । \newline
40. वै ब्रह्म॑णो॒ ब्रह्म॑णो॒ वै वै ब्रह्म॑णो रू॒पꣳ रू॒पम् ब्रह्म॑णो॒ वै वै ब्रह्म॑णो रू॒पम् । \newline
41. ब्रह्म॑णो रू॒पꣳ रू॒पम् ब्रह्म॑णो॒ ब्रह्म॑णो रू॒पꣳ समृ॑द्ध्यै॒ समृ॑द्ध्यै रू॒पम् ब्रह्म॑णो॒ ब्रह्म॑णो रू॒पꣳ समृ॑द्ध्यै । \newline
42. रू॒पꣳ समृ॑द्ध्यै॒ समृ॑द्ध्यै रू॒पꣳ रू॒पꣳ समृ॑द्ध्यै॒ स्फ्यः स्फ्यः समृ॑द्ध्यै रू॒पꣳ रू॒पꣳ समृ॑द्ध्यै॒ स्फ्यः । \newline
43. समृ॑द्ध्यै॒ स्फ्यः स्फ्यः समृ॑द्ध्यै॒ समृ॑द्ध्यै॒ स्फ्यो यूपो॒ यूपः॒ स्फ्यः समृ॑द्ध्यै॒ समृ॑द्ध्यै॒ स्फ्यो यूपः॑ । \newline
44. समृ॑द्ध्या॒ इति॒ सं - ऋ॒द्ध्यै॒ । \newline
45. स्फ्यो यूपो॒ यूपः॒ स्फ्यः स्फ्यो यूपो॑ भवति भवति॒ यूपः॒ स्फ्यः स्फ्यो यूपो॑ भवति । \newline
46. यूपो॑ भवति भवति॒ यूपो॒ यूपो॑ भवति॒ वज्रो॒ वज्रो॑ भवति॒ यूपो॒ यूपो॑ भवति॒ वज्रः॑ । \newline
47. भ॒व॒ति॒ वज्रो॒ वज्रो॑ भवति भवति॒ वज्रो॒ वै वै वज्रो॑ भवति भवति॒ वज्रो॒ वै । \newline
48. वज्रो॒ वै वै वज्रो॒ वज्रो॒ वै स्फ्यः स्फ्यो वै वज्रो॒ वज्रो॒ वै स्फ्यः । \newline
49. वै स्फ्यः स्फ्यो वै वै स्फ्यो वज्रं॒ ॅवज्रꣳ॒॒ स्फ्यो वै वै स्फ्यो वज्र᳚म् । \newline
50. स्फ्यो वज्रं॒ ॅवज्रꣳ॒॒ स्फ्यः स्फ्यो वज्र॑ मे॒वैव वज्रꣳ॒॒ स्फ्यः स्फ्यो वज्र॑ मे॒व । \newline
51. वज्र॑ मे॒वैव वज्रं॒ ॅवज्र॑ मे॒वास्मा॑ अस्मा ए॒व वज्रं॒ ॅवज्र॑ मे॒वास्मै᳚ । \newline
52. ए॒वास्मा॑ अस्मा ए॒वैवास्मै॒ प्र प्रास्मा॑ ए॒वैवास्मै॒ प्र । \newline
53. अ॒स्मै॒ प्र प्रास्मा॑ अस्मै॒ प्र ह॑रति हरति॒ प्रास्मा॑ अस्मै॒ प्र ह॑रति । \newline
54. प्र ह॑रति हरति॒ प्र प्र ह॑रति शर॒मयꣳ॑ शर॒मयꣳ॑ हरति॒ प्र प्र ह॑रति शर॒मय᳚म् । \newline
55. ह॒र॒ति॒ श॒र॒मयꣳ॑ शर॒मयꣳ॑ हरति हरति शर॒मय॑म् ब॒र्॒.हिर् ब॒र्॒.हिः श॑र॒मयꣳ॑ हरति हरति शर॒मय॑म् ब॒र्॒.हिः । \newline
56. श॒र॒मय॑म् ब॒र्॒.हिर् ब॒र्॒.हिः श॑र॒मयꣳ॑ शर॒मय॑म् ब॒र्॒.हिः शृ॒णाति॑ शृ॒णाति॑ ब॒र्॒.हिः श॑र॒मयꣳ॑ शर॒मय॑म् ब॒र्॒.हिः शृ॒णाति॑ । \newline
57. श॒र॒मय॒मिति॑ शर - मय᳚म् । \newline
58. ब॒र्॒.हिः शृ॒णाति॑ शृ॒णाति॑ ब॒र्॒.हिर् ब॒र्॒.हिः शृ॒णात्ये॒वैव शृ॒णाति॑ ब॒र्॒.हिर् ब॒र्॒.हिः शृ॒णात्ये॒व । \newline
59. शृ॒णात्ये॒वैव शृ॒णाति॑ शृ॒णात्ये॒वैन॑ मेन मे॒व शृ॒णाति॑ शृ॒णात्ये॒वैन᳚म् । \newline
\pagebreak
\markright{ TS 2.1.8.3  \hfill https://www.vedavms.in \hfill}

\section{ TS 2.1.8.3 }

\textbf{TS 2.1.8.3 } \newline
\textbf{Samhita Paata} \newline

-त्ये॒वैनं॒ ॅवैभी॑दक इ॒द्ध्मो भि॒नत्त्ये॒वैनं॑ ॅवैष्ण॒वं ॅवा॑म॒नमा ल॑भेत॒ यं ॅय॒ज्ञो नोप॒नमे॒द्-विष्णु॒र्वै य॒ज्ञो विष्णु॑मे॒व स्वेन॑ भाग॒धेये॒नोप॑ धावति॒ स ए॒वास्मै॑ य॒ज्ञ्ं प्र य॑च्छ॒त्युपै॑नं ॅय॒ज्ञो न॑मति वाम॒नो भ॑वति वैष्ण॒वो ह्ये॑ष दे॒वत॑या॒ समृ॑द्ध्यै त्वा॒ष्ट्रं ॅव॑ड॒बमा ल॑भेत प॒शुका॑म॒स्त्वष्टा॒ वै प॑शू॒नां मि॑थु॒नानां᳚ - [  ] \newline

\textbf{Pada Paata} \newline

ए॒व । ए॒न॒म् । वैभी॑दकः । इ॒द्ध्मः । भि॒नत्ति॑ । ए॒व । ए॒न॒म् । वै॒ष्ण॒वम् । वा॒म॒नम् । एति॑ । ल॒भे॒त॒ । यम् । य॒ज्ञ्ः । न । उ॒प॒नमे॒दित्यु॑प-नमे᳚त् । विष्णुः॑ । वै । य॒ज्ञ्ः । विष्णु᳚म् । ए॒व । स्वेन॑ । भा॒ग॒धेये॒नेति॑ भाग - धेये॑न । उपेति॑ । धा॒व॒ति॒ । सः । ए॒व । अ॒स्मै॒ । य॒ज्ञ्म् । प्रेति॑ । य॒च्छ॒ति॒ । उपेति॑ । ए॒न॒म् । य॒ज्ञ्ः । न॒म॒ति॒ । वा॒म॒नः । भ॒व॒ति॒ । वै॒ष्ण॒वः । हि । ए॒षः । दे॒वत॑या । समृ॑द्ध्या॒ इति॒ सं - ऋ॒द्ध्यै॒ । त्वा॒ष्ट्रम् । व॒ड॒बम् । एति॑ । ल॒भे॒त॒ । प॒शुका॑म॒ इति॑ प॒शु - का॒मः॒ । त्वष्टा᳚ । वै । प॒शू॒नाम् । मि॒थु॒नाना᳚म् ।  \newline


\textbf{Krama Paata} \newline

ए॒वैन᳚म् । ए॒नं॒ ॅवैभी॑दकः । वैभी॑दक इ॒ध्मः । इ॒ध्मो भि॒नत्ति॑ । भि॒नत्ये॒व । ए॒वैन᳚म् । ए॒नं॒ ॅवै॒ष्ण॒वम् । वै॒ष्ण॒वं ॅवा॑म॒नम् । वा॒म॒नमा । आ ल॑भेत । ल॒भे॒त॒ यम् । यं ॅय॒ज्ञ्ः । य॒ज्ञो न । नोप॒नमे᳚त् । उ॒प॒नमे॒द् विष्णुः॑ । उ॒प॒नमे॒दित्यु॑प - नमे᳚त् । विष्णु॒र् वै । वै य॒ज्ञ्ः । य॒ज्ञो विष्णु᳚म् । विष्णु॑मे॒व । ए॒वस्वेन॑ । स्वेन॑ भाग॒धेये॑न । भा॒ग॒धेये॒नोप॑ । भा॒ग॒धेये॒नेति॑ भाग - धेये॑न । उप॑ धावति । धा॒व॒ति॒ सः । स ए॒व । ए॒वास्मै᳚ । अ॒स्मै॒ य॒ज्ञ्म् । य॒ज्ञ्म् प्र । प्र य॑च्छति । य॒च्छ॒त्युप॑ । उपै॑नम् । ए॒नं॒ ॅय॒ज्ञ्ः । य॒ज्ञो न॑मति । न॒म॒ति॒ वा॒म॒नः । वा॒म॒नो भ॑वति । भ॒व॒ति॒ वै॒ष्ण॒वः । वै॒ष्ण॒वो हि । ह्ये॑षः । ए॒ष दे॒वत॑या । दे॒वत॑या॒ समृ॑द्ध्यै । समृ॑द्ध्यै त्वा॒ष्ट्रम् । समृ॑द्ध्या॒ इति॒ सं - ऋ॒द्ध्यै॒ । त्वा॒ष्ट्रं ॅव॑ड॒बम् । व॒ड॒बमा । आ ल॑भेत । ल॒भे॒त॒ प॒शुका॑मः । प॒शुका॑म॒,स्त्वष्टा᳚ । प॒शुका॑म॒ इति॑ प॒शु - का॒मः॒ । त्वष्टा॒ वै । वै प॑शू॒नाम् । प॒शू॒नाम् मि॑थु॒नाना᳚म् । मि॒थु॒नानां᳚ प्रजनयि॒ता \newline

\textbf{Jatai Paata} \newline

1. ए॒वैन॑ मेन मे॒वैवैन᳚म् । \newline
2. ए॒नं॒ ॅवैभी॑दको॒ वैभी॑दक एन मेनं॒ ॅवैभी॑दकः । \newline
3. वैभी॑दक इ॒द्ध्म इ॒द्ध्मो वैभी॑दको॒ वैभी॑दक इ॒द्ध्मः । \newline
4. इ॒द्ध्मो भि॒नत्ति॑ भि॒नत्ती॒द्ध्म इ॒द्ध्मो भि॒नत्ति॑ । \newline
5. भि॒न त्त्ये॒वैव भि॒नत्ति॑ भि॒न त्त्ये॒व । \newline
6. ए॒वैन॑ मेन मे॒वैवैन᳚म् । \newline
7. ए॒नं॒ ॅवै॒ष्ण॒वं ॅवै᳚ष्ण॒व मे॑न मेनं ॅवैष्ण॒वम् । \newline
8. वै॒ष्ण॒वं ॅवा॑म॒नं ॅवा॑म॒नं ॅवै᳚ष्ण॒वं ॅवै᳚ष्ण॒वं ॅवा॑म॒नम् । \newline
9. वा॒म॒न मा वा॑म॒नं ॅवा॑म॒न मा । \newline
10. आ ल॑भेत लभे॒ता ल॑भेत । \newline
11. ल॒भे॒त॒ यं ॅयम् ॅल॑भेत लभेत॒ यम् । \newline
12. यं ॅय॒ज्ञो य॒ज्ञो यं ॅयं ॅय॒ज्ञ्ः । \newline
13. य॒ज्ञो न न य॒ज्ञो य॒ज्ञो न । \newline
14. नोप॒नमे॑ दुप॒नमे॒न् न नोप॒नमे᳚त् । \newline
15. उ॒प॒नमे॒द् विष्णु॒र् विष्णु॑ रुप॒नमे॑ दुप॒नमे॒द् विष्णुः॑ । \newline
16. उ॒प॒नमे॒दित्यु॑प - नमे᳚त् । \newline
17. विष्णु॒र् वै वै विष्णु॒र् विष्णु॒र् वै । \newline
18. वै य॒ज्ञो य॒ज्ञो वै वै य॒ज्ञ्ः । \newline
19. य॒ज्ञो विष्णुं॒ ॅविष्णुं॑ ॅय॒ज्ञो य॒ज्ञो विष्णु᳚म् । \newline
20. विष्णु॑ मे॒वैव विष्णुं॒ ॅविष्णु॑ मे॒व । \newline
21. ए॒व स्वेन॒ स्वेनै॒वैव स्वेन॑ । \newline
22. स्वेन॑ भाग॒धेये॑न भाग॒धेये॑न॒ स्वेन॒ स्वेन॑ भाग॒धेये॑न । \newline
23. भा॒ग॒धेये॒नोपोप॑ भाग॒धेये॑न भाग॒धेये॒नोप॑ । \newline
24. भा॒ग॒धेये॒नेति॑ भाग - धेये॑न । \newline
25. उप॑ धावति धाव॒ त्युपोप॑ धावति । \newline
26. धा॒व॒ति॒ स स धा॑वति धावति॒ सः । \newline
27. स ए॒वैव स स ए॒व । \newline
28. ए॒वास्मा॑ अस्मा ए॒वैवास्मै᳚ । \newline
29. अ॒स्मै॒ य॒ज्ञ्ं ॅय॒ज्ञ् म॑स्मा अस्मै य॒ज्ञ्म् । \newline
30. य॒ज्ञ्म् प्र प्र य॒ज्ञ्ं ॅय॒ज्ञ्म् प्र । \newline
31. प्र य॑च्छति यच्छति॒ प्र प्र य॑च्छति । \newline
32. य॒च्छ॒ त्युपोप॑ यच्छति यच्छ॒ त्युप॑ । \newline
33. उपै॑न मेन॒ मुपोपै॑नम् । \newline
34. ए॒नं॒ ॅय॒ज्ञो य॒ज्ञ् ए॑न मेनं ॅय॒ज्ञ्ः । \newline
35. य॒ज्ञो न॑मति नमति य॒ज्ञो य॒ज्ञो न॑मति । \newline
36. न॒म॒ति॒ वा॒म॒नो वा॑म॒नो न॑मति नमति वाम॒नः । \newline
37. वा॒म॒नो भ॑वति भवति वाम॒नो वा॑म॒नो भ॑वति । \newline
38. भ॒व॒ति॒ वै॒ष्ण॒वो वै᳚ष्ण॒वो भ॑वति भवति वैष्ण॒वः । \newline
39. वै॒ष्ण॒वो हि हि वै᳚ष्ण॒वो वै᳚ष्ण॒वो हि । \newline
40. ह्ये॑ष ए॒ष हि ह्ये॑षः । \newline
41. ए॒ष दे॒वत॑या दे॒वत॑यै॒ष ए॒ष दे॒वत॑या । \newline
42. दे॒वत॑या॒ समृ॑द्ध्यै॒ समृ॑द्ध्यै दे॒वत॑या दे॒वत॑या॒ समृ॑द्ध्यै । \newline
43. समृ॑द्ध्यै त्वा॒ष्ट्रम् त्वा॒ष्ट्रꣳ समृ॑द्ध्यै॒ समृ॑द्ध्यै त्वा॒ष्ट्रम् । \newline
44. समृ॑द्ध्या॒ इति॒ सं - ऋ॒द्ध्यै॒ । \newline
45. त्वा॒ष्ट्रं ॅव॑ड॒बं ॅव॑ड॒बम् त्वा॒ष्ट्रम् त्वा॒ष्ट्रं ॅव॑ड॒बम् । \newline
46. व॒ड॒ब मा व॑ड॒बं ॅव॑ड॒ब मा । \newline
47. आ ल॑भेत लभे॒ता ल॑भेत । \newline
48. ल॒भे॒त॒ प॒शुका॑मः प॒शुका॑मो लभेत लभेत प॒शुका॑मः । \newline
49. प॒शुका॑म॒ स्त्वष्टा॒ त्वष्टा॑ प॒शुका॑मः प॒शुका॑म॒ स्त्वष्टा᳚ । \newline
50. प॒शुका॑म॒ इति॑ प॒शु - का॒मः॒ । \newline
51. त्वष्टा॒ वै वै त्वष्टा॒ त्वष्टा॒ वै । \newline
52. वै प॑शू॒नाम् प॑शू॒नां ॅवै वै प॑शू॒नाम् । \newline
53. प॒शू॒नाम् मि॑थु॒नाना᳚म् मिथु॒नाना᳚म् पशू॒नाम् प॑शू॒नाम् मि॑थु॒नाना᳚म् । \newline
54. मि॒थु॒नाना᳚म् प्रजनयि॒ता प्र॑जनयि॒ता मि॑थु॒नाना᳚म् मिथु॒नाना᳚म् प्रजनयि॒ता । \newline

\textbf{Ghana Paata } \newline

1. ए॒वैन॑ मेन मे॒वैवैनं॒ ॅवैभी॑दको॒ वैभी॑दक एन मे॒वैवैनं॒ ॅवैभी॑दकः । \newline
2. ए॒नं॒ ॅवैभी॑दको॒ वैभी॑दक एन मेनं॒ ॅवैभी॑दक इ॒द्ध्म इ॒द्ध्मो वैभी॑दक एन मेनं॒ ॅवैभी॑दक इ॒द्ध्मः । \newline
3. वैभी॑दक इ॒द्ध्म इ॒द्ध्मो वैभी॑दको॒ वैभी॑दक इ॒द्ध्मो भि॒नत्ति॑ भि॒नत्ती॒द्ध्मो वैभी॑दको॒ वैभी॑दक इ॒द्ध्मो भि॒नत्ति॑ । \newline
4. इ॒द्ध्मो भि॒नत्ति॑ भि॒नत्ती॒द्ध्म इ॒द्ध्मो भि॒नत्त्ये॒वैव भि॒नत्ती॒द्ध्म इ॒द्ध्मो भि॒नत्त्ये॒व । \newline
5. भि॒नत्त्ये॒वैव भि॒नत्ति॑ भि॒नत्त्ये॒वैन॑ मेन मे॒व भि॒नत्ति॑ भि॒नत्त्ये॒वैन᳚म् । \newline
6. ए॒वैन॑ मेन मे॒वैवैनं॑ ॅवैष्ण॒वं ॅवै᳚ष्ण॒व मे॑न मे॒वैवैनं॑ ॅवैष्ण॒वम् । \newline
7. ए॒नं॒ ॅवै॒ष्ण॒वं ॅवै᳚ष्ण॒व मे॑न मेनं ॅवैष्ण॒वं ॅवा॑म॒नं ॅवा॑म॒नं ॅवै᳚ष्ण॒व मे॑न मेनं ॅवैष्ण॒वं ॅवा॑म॒नम् । \newline
8. वै॒ष्ण॒वं ॅवा॑म॒नं ॅवा॑म॒नं ॅवै᳚ष्ण॒वं ॅवै᳚ष्ण॒वं ॅवा॑म॒न मा वा॑म॒नं ॅवै᳚ष्ण॒वं ॅवै᳚ष्ण॒वं ॅवा॑म॒न मा । \newline
9. वा॒म॒न मा वा॑म॒नं ॅवा॑म॒न मा ल॑भेत लभे॒ता वा॑म॒नं ॅवा॑म॒न मा ल॑भेत । \newline
10. आ ल॑भेत लभे॒ता ल॑भेत॒ यं ॅयम् ॅल॑भे॒ता ल॑भेत॒ यम् । \newline
11. ल॒भे॒त॒ यं ॅयम् ॅल॑भेत लभेत॒ यं ॅय॒ज्ञो य॒ज्ञो यम् ॅल॑भेत लभेत॒ यं ॅय॒ज्ञ्ः । \newline
12. यं ॅय॒ज्ञो य॒ज्ञो यं ॅयं ॅय॒ज्ञो न न य॒ज्ञो यं ॅयं ॅय॒ज्ञो न । \newline
13. य॒ज्ञो न न य॒ज्ञो य॒ज्ञो नोप॒नमे॑ दुप॒नमे॒न् न य॒ज्ञो य॒ज्ञो नोप॒नमे᳚त् । \newline
14. नोप॒नमे॑ दुप॒नमे॒न् न नोप॒नमे॒द् विष्णु॒र् विष्णु॑ रुप॒नमे॒न् न नोप॒नमे॒द् विष्णुः॑ । \newline
15. उ॒प॒नमे॒द् विष्णु॒र् विष्णु॑ रुप॒नमे॑ दुप॒नमे॒द् विष्णु॒र् वै वै विष्णु॑ रुप॒नमे॑ दुप॒नमे॒द् विष्णु॒र् वै । \newline
16. उ॒प॒नमे॒दित्यु॑प - नमे᳚त् । \newline
17. विष्णु॒र् वै वै विष्णु॒र् विष्णु॒र् वै य॒ज्ञो य॒ज्ञो वै विष्णु॒र् विष्णु॒र् वै य॒ज्ञ्ः । \newline
18. वै य॒ज्ञो य॒ज्ञो वै वै य॒ज्ञो विष्णुं॒ ॅविष्णुं॑ ॅय॒ज्ञो वै वै य॒ज्ञो विष्णु᳚म् । \newline
19. य॒ज्ञो विष्णुं॒ ॅविष्णुं॑ ॅय॒ज्ञो य॒ज्ञो विष्णु॑ मे॒वैव विष्णुं॑ ॅय॒ज्ञो य॒ज्ञो विष्णु॑ मे॒व । \newline
20. विष्णु॑ मे॒वैव विष्णुं॒ ॅविष्णु॑ मे॒व स्वेन॒ स्वेनै॒व विष्णुं॒ ॅविष्णु॑ मे॒व स्वेन॑ । \newline
21. ए॒व स्वेन॒ स्वेनै॒वैव स्वेन॑ भाग॒धेये॑न भाग॒धेये॑न॒ स्वेनै॒वैव स्वेन॑ भाग॒धेये॑न । \newline
22. स्वेन॑ भाग॒धेये॑न भाग॒धेये॑न॒ स्वेन॒ स्वेन॑ भाग॒धेये॒नो पोप॑ भाग॒धेये॑न॒ स्वेन॒ स्वेन॑ भाग॒धेये॒नोप॑ । \newline
23. भा॒ग॒धेये॒नोप् ओप॑ भाग॒धेये॑न भाग॒धेये॒नोप॑ धावति धाव॒त्युप॑ भाग॒धेये॑न भाग॒धेये॒नोप॑ धावति । \newline
24. भा॒ग॒धेये॒नेति॑ भाग - धेये॑न । \newline
25. उप॑ धावति धाव॒ त्युपोप॑ धावति॒ स स धा॑व॒ त्युपोप॑ धावति॒ सः । \newline
26. धा॒व॒ति॒ स स धा॑वति धावति॒ स ए॒वैव स धा॑वति धावति॒ स ए॒व । \newline
27. स ए॒वैव स स ए॒वास्मा॑ अस्मा ए॒व स स ए॒वास्मै᳚ । \newline
28. ए॒वास्मा॑ अस्मा ए॒वैवास्मै॑ य॒ज्ञ्ं ॅय॒ज्ञ् म॑स्मा ए॒वैवास्मै॑ य॒ज्ञ्म् । \newline
29. अ॒स्मै॒ य॒ज्ञ्ं ॅय॒ज्ञ् म॑स्मा अस्मै य॒ज्ञ्म् प्र प्र य॒ज्ञ् म॑स्मा अस्मै य॒ज्ञ्म् प्र । \newline
30. य॒ज्ञ्म् प्र प्र य॒ज्ञ्ं ॅय॒ज्ञ्म् प्र य॑च्छति यच्छति॒ प्र य॒ज्ञ्ं ॅय॒ज्ञ्म् प्र य॑च्छति । \newline
31. प्र य॑च्छति यच्छति॒ प्र प्र य॑च्छ॒त्युपोप॑ यच्छति॒ प्र प्र य॑च्छ॒त्युप॑ । \newline
32. य॒च्छ॒त्युपोप॑ यच्छति यच्छ॒त्युपै॑न मेन॒ मुप॑ यच्छति यच्छ॒त्युपै॑नम् । \newline
33. उपै॑न मेन॒ मुपोपै॑नं ॅय॒ज्ञो य॒ज्ञ् ए॑न॒ मुपोपै॑नं ॅय॒ज्ञ्ः । \newline
34. ए॒नं॒ ॅय॒ज्ञो य॒ज्ञ् ए॑न मेनं ॅय॒ज्ञो न॑मति नमति य॒ज्ञ् ए॑न मेनं ॅय॒ज्ञो न॑मति । \newline
35. य॒ज्ञो न॑मति नमति य॒ज्ञो य॒ज्ञो न॑मति वाम॒नो वा॑म॒नो न॑मति य॒ज्ञो य॒ज्ञो न॑मति वाम॒नः । \newline
36. न॒म॒ति॒ वा॒म॒नो वा॑म॒नो न॑मति नमति वाम॒नो भ॑वति भवति वाम॒नो न॑मति नमति वाम॒नो भ॑वति । \newline
37. वा॒म॒नो भ॑वति भवति वाम॒नो वा॑म॒नो भ॑वति वैष्ण॒वो वै᳚ष्ण॒वो भ॑वति वाम॒नो वा॑म॒नो भ॑वति वैष्ण॒वः । \newline
38. भ॒व॒ति॒ वै॒ष्ण॒वो वै᳚ष्ण॒वो भ॑वति भवति वैष्ण॒वो हि हि वै᳚ष्ण॒वो भ॑वति भवति वैष्ण॒वो हि । \newline
39. वै॒ष्ण॒वो हि हि वै᳚ष्ण॒वो वै᳚ष्ण॒वो ह्ये॑ष ए॒ष हि वै᳚ष्ण॒वो वै᳚ष्ण॒वो ह्ये॑षः । \newline
40. ह्ये॑ष ए॒ष हि ह्ये॑ष दे॒वत॑या दे॒वत॑यै॒ष हि ह्ये॑ष दे॒वत॑या । \newline
41. ए॒ष दे॒वत॑या दे॒वत॑यै॒ष ए॒ष दे॒वत॑या॒ समृ॑द्ध्यै॒ समृ॑द्ध्यै दे॒वत॑यै॒ष ए॒ष दे॒वत॑या॒ समृ॑द्ध्यै । \newline
42. दे॒वत॑या॒ समृ॑द्ध्यै॒ समृ॑द्ध्यै दे॒वत॑या दे॒वत॑या॒ समृ॑द्ध्यै त्वा॒ष्ट्रम् त्वा॒ष्ट्रꣳ समृ॑द्ध्यै दे॒वत॑या दे॒वत॑या॒ समृ॑द्ध्यै त्वा॒ष्ट्रम् । \newline
43. समृ॑द्ध्यै त्वा॒ष्ट्रम् त्वा॒ष्ट्रꣳ समृ॑द्ध्यै॒ समृ॑द्ध्यै त्वा॒ष्ट्रं ॅव॑ड॒बं ॅव॑ड॒बम् त्वा॒ष्ट्रꣳ समृ॑द्ध्यै॒ समृ॑द्ध्यै त्वा॒ष्ट्रं ॅव॑ड॒बम् । \newline
44. समृ॑द्ध्या॒ इति॒ सं - ऋ॒द्ध्यै॒ । \newline
45. त्वा॒ष्ट्रं ॅव॑ड॒बं ॅव॑ड॒बम् त्वा॒ष्ट्रम् त्वा॒ष्ट्रं ॅव॑ड॒ब मा व॑ड॒बम् त्वा॒ष्ट्रम् त्वा॒ष्ट्रं ॅव॑ड॒ब मा । \newline
46. व॒ड॒ब मा व॑ड॒बं ॅव॑ड॒ब मा ल॑भेत लभे॒ता व॑ड॒बं ॅव॑ड॒ब मा ल॑भेत । \newline
47. आ ल॑भेत लभे॒ता ल॑भेत प॒शुका॑मः प॒शुका॑मो लभे॒ता ल॑भेत प॒शुका॑मः । \newline
48. ल॒भे॒त॒ प॒शुका॑मः प॒शुका॑मो लभेत लभेत प॒शुका॑म॒ स्त्वष्टा॒ त्वष्टा॑ प॒शुका॑मो लभेत लभेत प॒शुका॑म॒ स्त्वष्टा᳚ । \newline
49. प॒शुका॑म॒ स्त्वष्टा॒ त्वष्टा॑ प॒शुका॑मः प॒शुका॑म॒ स्त्वष्टा॒ वै वै त्वष्टा॑ प॒शुका॑मः प॒शुका॑म॒ स्त्वष्टा॒ वै । \newline
50. प॒शुका॑म॒ इति॑ प॒शु - का॒मः॒ । \newline
51. त्वष्टा॒ वै वै त्वष्टा॒ त्वष्टा॒ वै प॑शू॒नाम् प॑शू॒नां ॅवै त्वष्टा॒ त्वष्टा॒ वै प॑शू॒नाम् । \newline
52. वै प॑शू॒नाम् प॑शू॒नां ॅवै वै प॑शू॒नाम् मि॑थु॒नाना᳚म् मिथु॒नाना᳚म् पशू॒नां ॅवै वै प॑शू॒नाम् मि॑थु॒नाना᳚म् । \newline
53. प॒शू॒नाम् मि॑थु॒नाना᳚म् मिथु॒नाना᳚म् पशू॒नाम् प॑शू॒नाम् मि॑थु॒नाना᳚म् प्रजनयि॒ता प्र॑जनयि॒ता मि॑थु॒नाना᳚म् पशू॒नाम् प॑शू॒नाम् मि॑थु॒नाना᳚म् प्रजनयि॒ता । \newline
54. मि॒थु॒नाना᳚म् प्रजनयि॒ता प्र॑जनयि॒ता मि॑थु॒नाना᳚म् मिथु॒नाना᳚म् प्रजनयि॒ता त्वष्टा॑र॒म् त्वष्टा॑रम् प्रजनयि॒ता मि॑थु॒नाना᳚म् मिथु॒नाना᳚म् प्रजनयि॒ता त्वष्टा॑रम् । \newline
\pagebreak
\markright{ TS 2.1.8.4  \hfill https://www.vedavms.in \hfill}

\section{ TS 2.1.8.4 }

\textbf{TS 2.1.8.4 } \newline
\textbf{Samhita Paata} \newline

प्रजनयि॒ता त्वष्टा॑रमे॒व स्वेन॑ भाग॒धेये॒नोप॑ धावति॒ स ए॒वास्मै॑ प॒शून् मि॑थु॒नान् प्र ज॑नयति प्र॒जा हि वा ए॒तस्मि॑न् प॒शवः॒ प्रवि॑ष्टा॒ अथै॒ष पुमा॒न्थ्‌सन् व॑ड॒बः सा॒क्षादे॒व प्र॒जां प॒शूनव॑ रुन्धे मै॒त्रꣳ श्वे॒तमा ल॑भेत संग्रा॒मे संॅय॑त्ते सम॒यका॑मो मि॒त्रमे॒व स्वेन॑ भाग॒धेये॒नोप॑ धावति॒ स ए॒वैनं॑ मि॒त्रेण॒ सं न॑यति - [  ] \newline

\textbf{Pada Paata} \newline

प्र॒ज॒न॒यि॒तेति॑ प्र - ज॒न॒यि॒ता । त्वष्टा॑रम् । ए॒व । स्वेन॑ । भा॒ग॒धेये॒नेति॑ भाग - धेये॑न । उपेति॑ । धा॒व॒ति॒ । सः । ए॒व । अ॒स्मै॒ । प॒शून् । मि॒थु॒नान् । प्रेति॑ । ज॒न॒य॒ति॒ । प्र॒जेति॑ प्र - जा । हि । वै । ए॒तस्मिन्न्॑ । प॒शवः॑ । प्रवि॑ष्टा॒ इति॒ प्र - वि॒ष्टाः॒ । अथ॑ । ए॒षः । पुमान्॑ । सन्न् । व॒ड॒बः । सा॒क्षादिति॑ स - अ॒क्षात् । ए॒व । प्र॒जामिति॑ प्र -जाम् । प॒शून् । अवेति॑ । रु॒न्धे॒ । मै॒त्रम् । श्वे॒तम् । एति॑ । ल॒भे॒त॒ । स॒ङ्ग्रा॒म इति॑ सं - ग्रा॒मे । संॅय॑त्त॒ इति॒ सं - य॒त्ते॒ । स॒म॒यका॑म॒ इति॑ सम॒य - का॒मः॒ । मि॒त्रम् । ए॒व । स्वेन॑ । भा॒ग॒धेये॒नेति॑ भाग - धेये॑न । उपेति॑ । धा॒व॒ति॒ । सः । ए॒व । ए॒न॒म् । मि॒त्रेण॑ । समिति॑ । न॒य॒ति॒ ।  \newline


\textbf{Krama Paata} \newline

प्र॒ज॒न॒यि॒ता त्वष्टा॑रम् । प्र॒ज॒न॒यि॒तेति॑ प्र - ज॒न॒यि॒ता । त्वष्टा॑रमे॒व । ए॒व स्वेन॑ । स्वेन॑ भाग॒धेये॑न । भा॒ग॒धेये॒नोप॑ । भा॒ग॒धेये॒नेति॑ भाग - धेये॑न । उप॑ धावति । धा॒व॒ति॒ सः । स ए॒व । ए॒वास्मै᳚ । अ॒स्मै॒ प॒शून् । प॒शुन् मि॑थु॒नान् । मि॒थु॒नान् प्र । प्र ज॑नयति । ज॒न॒य॒ति॒ प्र॒जा । प्र॒जा हि । प्र॒जेति॑ प्र - जा । हि वै । वा ए॒तस्मिन्न्॑ । ए॒तस्मि॑न् प॒शवः॑ । प॒शवः॒ प्रवि॑ष्टाः । प्रवि॑ष्टा॒ अथ॑ । प्रवि॑ष्टा॒ इति॒ प्र - वि॒ष्टाः॒ । अथै॒षः । ए॒ष पुमान्॑ । पुमा॒न्थ् सन्न् । सन् व॑ड॒बः । व॒ड॒बः सा॒क्षात् । सा॒क्षादे॒व । सा॒क्षादिति॑ स - अ॒क्षात् । ए॒व प्र॒जाम् । प्र॒जाम् प॒शून् । प्र॒जामिति॑ प्र - जाम् । प॒शूनव॑ । अव॑ रुन्धे । रु॒न्धे॒ मै॒त्रम् । मै॒त्रꣳ श्वे॒तम् । श्वे॒तमा । आ ल॑भेत । ल॒भे॒त॒ स॒ङ्ग्रा॒मे । स॒ङ्ग्रा॒मे सम्ॅय॑त्ते । स॒ङ्ग्रा॒म इति॑ सं - ग्रा॒मे । सम्ॅय॑त्ते सम॒यका॑मः । सम्ॅय॑त्त॒ इति॒ सं - य॒त्ते॒ । स॒म॒यका॑मो मि॒त्रम् । स॒म॒यका॑म॒ इति॑ सम॒य - का॒मः॒ । मि॒त्रमे॒व । ए॒व स्वेन॑ । स्वेन॑ भाग॒धेये॑न । भा॒ग॒धेये॒नोप॑ । भा॒ग॒धेये॒नेति॑ भाग - धेये॑न । उप॑ धावति । धा॒व॒ति॒ सः । स ए॒व । ए॒वैन᳚म् । ए॒न॒म् मि॒त्रेण॑ । मि॒त्रेण॒ सम् । सम् न॑यति । न॒य॒ति॒ वि॒शा॒लः \newline

\textbf{Jatai Paata} \newline

1. प्र॒ज॒न॒यि॒ता त्वष्टा॑र॒म् त्वष्टा॑रम् प्रजनयि॒ता प्र॑जनयि॒ता त्वष्टा॑रम् । \newline
2. प्र॒ज॒न॒यि॒तेति॑ प्र - ज॒न॒यि॒ता । \newline
3. त्वष्टा॑र मे॒वैव त्वष्टा॑र॒म् त्वष्टा॑र मे॒व । \newline
4. ए॒व स्वेन॒ स्वेनै॒वैव स्वेन॑ । \newline
5. स्वेन॑ भाग॒धेये॑न भाग॒धेये॑न॒ स्वेन॒ स्वेन॑ भाग॒धेये॑न । \newline
6. भा॒ग॒धेये॒नोपोप॑ भाग॒धेये॑न भाग॒धेये॒नोप॑ । \newline
7. भा॒ग॒धेये॒नेति॑ भाग - धेये॑न । \newline
8. उप॑ धावति धाव॒ त्युपोप॑ धावति । \newline
9. धा॒व॒ति॒ स स धा॑वति धावति॒ सः । \newline
10. स ए॒वैव स स ए॒व । \newline
11. ए॒वास्मा॑ अस्मा ए॒वैवास्मै᳚ । \newline
12. अ॒स्मै॒ प॒शून् प॒शू न॑स्मा अस्मै प॒शून् । \newline
13. प॒शून् मि॑थु॒नान् मि॑थु॒नान् प॒शून् प॒शून् मि॑थु॒नान् । \newline
14. मि॒थु॒नान् प्र प्र मि॑थु॒नान् मि॑थु॒नान् प्र । \newline
15. प्र ज॑नयति जनयति॒ प्र प्र ज॑नयति । \newline
16. ज॒न॒य॒ति॒ प्र॒जा प्र॒जा ज॑नयति जनयति प्र॒जा । \newline
17. प्र॒जा हि हि प्र॒जा प्र॒जा हि । \newline
18. प्र॒जेति॑ प्र - जा । \newline
19. हि वै वै हि हि वै । \newline
20. वा ए॒तस्मि॑न् ने॒तस्मि॒न्॒. वै वा ए॒तस्मिन्न्॑ । \newline
21. ए॒तस्मि॑न् प॒शवः॑ प॒शव॑ ए॒तस्मि॑न् ने॒तस्मि॑न् प॒शवः॑ । \newline
22. प॒शवः॒ प्रवि॑ष्टाः॒ प्रवि॑ष्टाः प॒शवः॑ प॒शवः॒ प्रवि॑ष्टाः । \newline
23. प्रवि॑ष्टा॒ अथाथ॒ प्रवि॑ष्टाः॒ प्रवि॑ष्टा॒ अथ॑ । \newline
24. प्रवि॑ष्टा॒ इति॒ प्र - वि॒ष्टाः॒ । \newline
25. अथै॒ष ए॒षो ऽथा थै॒षः । \newline
26. ए॒ष पुमा॒न् पुमा॑ ने॒ष ए॒ष पुमान्॑ । \newline
27. पुमा॒न् थ्सन् थ्सन् पुमा॒न् पुमा॒न् थ्सन्न् । \newline
28. सन् व॑ड॒बो व॑ड॒बः सन् थ्सन् व॑ड॒बः । \newline
29. व॒ड॒बः सा॒क्षाथ् सा॒क्षाद् व॑ड॒बो व॑ड॒बः सा॒क्षात् । \newline
30. सा॒क्षा दे॒वैव सा॒क्षाथ् सा॒क्षा दे॒व । \newline
31. सा॒क्षादिति॑ स - अ॒क्षात् । \newline
32. ए॒व प्र॒जाम् प्र॒जा मे॒वैव प्र॒जाम् । \newline
33. प्र॒जाम् प॒शून् प॒शून् प्र॒जाम् प्र॒जाम् प॒शून् । \newline
34. प्र॒जामिति॑ प्र - जाम् । \newline
35. प॒शू नवाव॑ प॒शून् प॒शू नव॑ । \newline
36. अव॑ रुन्धे रु॒न्धे ऽवाव॑ रुन्धे । \newline
37. रु॒न्धे॒ मै॒त्रम् मै॒त्रꣳ रु॑न्धे रुन्धे मै॒त्रम् । \newline
38. मै॒त्रꣳ श्वे॒तꣳ श्वे॒तम् मै॒त्रम् मै॒त्रꣳ श्वे॒तम् । \newline
39. श्वे॒त मा श्वे॒तꣳ श्वे॒त मा । \newline
40. आ ल॑भेत लभे॒ता ल॑भेत । \newline
41. ल॒भे॒त॒ स॒ङ्ग्रा॒मे स॑ङ्ग्रा॒मे ल॑भेत लभेत सङ्ग्रा॒मे । \newline
42. स॒ङ्ग्रा॒मे संॅय॑त्ते॒ संॅय॑त्ते सङ्ग्रा॒मे स॑ङ्ग्रा॒मे संॅय॑त्ते । \newline
43. स॒ङ्ग्रा॒म इति॑ सं - ग्रा॒मे । \newline
44. संॅय॑त्ते सम॒यका॑मः सम॒यका॑मः॒ संॅय॑त्ते॒ संॅय॑त्ते सम॒यका॑मः । \newline
45. संॅय॑त्त॒ इति॒ सं - य॒त्ते॒ । \newline
46. स॒म॒यका॑मो मि॒त्रम् मि॒त्रꣳ स॑म॒यका॑मः सम॒यका॑मो मि॒त्रम् । \newline
47. स॒म॒यका॑म॒ इति॑ सम॒य - का॒मः॒ । \newline
48. मि॒त्र मे॒वैव मि॒त्रम् मि॒त्र मे॒व । \newline
49. ए॒व स्वेन॒ स्वेनै॒वैव स्वेन॑ । \newline
50. स्वेन॑ भाग॒धेये॑न भाग॒धेये॑न॒ स्वेन॒ स्वेन॑ भाग॒धेये॑न । \newline
51. भा॒ग॒धेये॒नोपोप॑ भाग॒धेये॑न भाग॒धेये॒नोप॑ । \newline
52. भा॒ग॒धेये॒नेति॑ भाग - धेये॑न । \newline
53. उप॑ धावति धाव॒ त्युपोप॑ धावति । \newline
54. धा॒व॒ति॒ स स धा॑वति धावति॒ सः । \newline
55. स ए॒वैव स स ए॒व । \newline
56. ए॒वैन॑ मेन मे॒वैवैन᳚म् । \newline
57. ए॒न॒म् मि॒त्रेण॑ मि॒त्रेणै॑न मेनम् मि॒त्रेण॑ । \newline
58. मि॒त्रेण॒ सꣳ सम् मि॒त्रेण॑ मि॒त्रेण॒ सम् । \newline
59. सम् न॑यति नयति॒ सꣳ सम् न॑यति । \newline
60. न॒य॒ति॒ वि॒शा॒लो वि॑शा॒लो न॑यति नयति विशा॒लः । \newline

\textbf{Ghana Paata } \newline

1. प्र॒ज॒न॒यि॒ता त्वष्टा॑र॒म् त्वष्टा॑रम् प्रजनयि॒ता प्र॑जनयि॒ता त्वष्टा॑र मे॒वैव त्वष्टा॑रम् प्रजनयि॒ता प्र॑जनयि॒ता त्वष्टा॑र मे॒व । \newline
2. प्र॒ज॒न॒यि॒तेति॑ प्र - ज॒न॒यि॒ता । \newline
3. त्वष्टा॑र मे॒वैव त्वष्टा॑र॒म् त्वष्टा॑र मे॒व स्वेन॒ स्वेनै॒व त्वष्टा॑र॒म् त्वष्टा॑र मे॒व स्वेन॑ । \newline
4. ए॒व स्वेन॒ स्वेनै॒वैव स्वेन॑ भाग॒धेये॑न भाग॒धेये॑न॒ स्वेनै॒वैव स्वेन॑ भाग॒धेये॑न । \newline
5. स्वेन॑ भाग॒धेये॑न भाग॒धेये॑न॒ स्वेन॒ स्वेन॑ भाग॒धेये॒नो पोप॑ भाग॒धेये॑न॒ स्वेन॒ स्वेन॑ भाग॒धेये॒नोप॑ । \newline
6. भा॒ग॒धेये॒नो पोप॑ भाग॒धेये॑न भाग॒धेये॒नोप॑ धावति धाव॒त्युप॑ भाग॒धेये॑न भाग॒धेये॒नोप॑ धावति । \newline
7. भा॒ग॒धेये॒नेति॑ भाग - धेये॑न । \newline
8. उप॑ धावति धाव॒त्युपोप॑ धावति॒ स स धा॑व॒त्युपोप॑ धावति॒ सः । \newline
9. धा॒व॒ति॒ स स धा॑वति धावति॒ स ए॒वैव स धा॑वति धावति॒ स ए॒व । \newline
10. स ए॒वैव स स ए॒वास्मा॑ अस्मा ए॒व स स ए॒वास्मै᳚ । \newline
11. ए॒वास्मा॑ अस्मा ए॒वैवास्मै॑ प॒शून् प॒शू न॑स्मा ए॒वैवास्मै॑ प॒शून् । \newline
12. अ॒स्मै॒ प॒शून् प॒शू न॑स्मा अस्मै प॒शून् मि॑थु॒नान् मि॑थु॒नान् प॒शू न॑स्मा अस्मै प॒शून् मि॑थु॒नान् । \newline
13. प॒शून् मि॑थु॒नान् मि॑थु॒नान् प॒शून् प॒शून् मि॑थु॒नान् प्र प्र मि॑थु॒नान् प॒शून् प॒शून् मि॑थु॒नान् प्र । \newline
14. मि॒थु॒नान् प्र प्र मि॑थु॒नान् मि॑थु॒नान् प्र ज॑नयति जनयति॒ प्र मि॑थु॒नान् मि॑थु॒नान् प्र ज॑नयति । \newline
15. प्र ज॑नयति जनयति॒ प्र प्र ज॑नयति प्र॒जा प्र॒जा ज॑नयति॒ प्र प्र ज॑नयति प्र॒जा । \newline
16. ज॒न॒य॒ति॒ प्र॒जा प्र॒जा ज॑नयति जनयति प्र॒जा हि हि प्र॒जा ज॑नयति जनयति प्र॒जा हि । \newline
17. प्र॒जा हि हि प्र॒जा प्र॒जा हि वै वै हि प्र॒जा प्र॒जा हि वै । \newline
18. प्र॒जेति॑ प्र - जा । \newline
19. हि वै वै हि हि वा ए॒तस्मि॑न् ने॒तस्मि॒न्॒. वै हि हि वा ए॒तस्मिन्न्॑ । \newline
20. वा ए॒तस्मि॑न् ने॒तस्मि॒न्॒. वै वा ए॒तस्मि॑न् प॒शवः॑ प॒शव॑ ए॒तस्मि॒न्॒. वै वा ए॒तस्मि॑न् प॒शवः॑ । \newline
21. ए॒तस्मि॑न् प॒शवः॑ प॒शव॑ ए॒तस्मि॑न् ने॒तस्मि॑न् प॒शवः॒ प्रवि॑ष्टाः॒ प्रवि॑ष्टाः प॒शव॑ ए॒तस्मि॑न् ने॒तस्मि॑न् प॒शवः॒ प्रवि॑ष्टाः । \newline
22. प॒शवः॒ प्रवि॑ष्टाः॒ प्रवि॑ष्टाः प॒शवः॑ प॒शवः॒ प्रवि॑ष्टा॒ अथाथ॒ प्रवि॑ष्टाः प॒शवः॑ प॒शवः॒ प्रवि॑ष्टा॒ अथ॑ । \newline
23. प्रवि॑ष्टा॒ अथाथ॒ प्रवि॑ष्टाः॒ प्रवि॑ष्टा॒ अथै॒ष ए॒षो ऽथ॒ प्रवि॑ष्टाः॒ प्रवि॑ष्टा॒ अथै॒षः । \newline
24. प्रवि॑ष्टा॒ इति॒ प्र - वि॒ष्टाः॒ । \newline
25. अथै॒ष ए॒षो ऽथाथै॒ष पुमा॒न् पुमा॑ने॒षो ऽथाथै॒ष पुमान्॑ । \newline
26. ए॒ष पुमा॒न् पुमा॑ ने॒ष ए॒ष पुमा॒न् थ्सन् थ्सन् पुमा॑ ने॒ष ए॒ष पुमा॒न् थ्सन्न् । \newline
27. पुमा॒न् थ्सन् थ्सन् पुमा॒न् पुमा॒न् थ्सन् व॑ड॒बो व॑ड॒बः सन् पुमा॒न् पुमा॒न् थ्सन् व॑ड॒बः । \newline
28. सन्. व॑ड॒बो व॑ड॒बः सन् थ्सन् व॑ड॒बः सा॒क्षाथ् सा॒क्षाद् व॑ड॒बः सन् थ्सन् व॑ड॒बः सा॒क्षात् । \newline
29. व॒ड॒बः सा॒क्षाथ् सा॒क्षाद् व॑ड॒बो व॑ड॒बः सा॒क्षा दे॒वैव सा॒क्षाद् व॑ड॒बो व॑ड॒बः सा॒क्षादे॒व । \newline
30. सा॒क्षा दे॒वैव सा॒क्षाथ् सा॒क्षादे॒व प्र॒जाम् प्र॒जा मे॒व सा॒क्षाथ् सा॒क्षादे॒व प्र॒जाम् । \newline
31. सा॒क्षादिति॑ स - अ॒क्षात् । \newline
32. ए॒व प्र॒जाम् प्र॒जा मे॒वैव प्र॒जाम् प॒शून् प॒शून् प्र॒जा मे॒वैव प्र॒जाम् प॒शून् । \newline
33. प्र॒जाम् प॒शून् प॒शून् प्र॒जाम् प्र॒जाम् प॒शू नवाव॑ प॒शून् प्र॒जाम् प्र॒जाम् प॒शू नव॑ । \newline
34. प्र॒जामिति॑ प्र - जाम् । \newline
35. प॒शू नवाव॑ प॒शून् प॒शू नव॑ रुन्धे रु॒न्धे ऽव॑ प॒शून् प॒शू नव॑ रुन्धे । \newline
36. अव॑ रुन्धे रु॒न्धे ऽवाव॑ रुन्धे मै॒त्रम् मै॒त्रꣳ रु॒न्धे ऽवाव॑ रुन्धे मै॒त्रम् । \newline
37. रु॒न्धे॒ मै॒त्रम् मै॒त्रꣳ रु॑न्धे रुन्धे मै॒त्रꣳ श्वे॒तꣳ श्वे॒तम् मै॒त्रꣳ रु॑न्धे रुन्धे मै॒त्रꣳ श्वे॒तम् । \newline
38. मै॒त्रꣳ श्वे॒तꣳ श्वे॒तम् मै॒त्रम् मै॒त्रꣳ श्वे॒त मा श्वे॒तम् मै॒त्रम् मै॒त्रꣳ श्वे॒त मा । \newline
39. श्वे॒त मा श्वे॒तꣳ श्वे॒त मा ल॑भेत लभे॒ता श्वे॒तꣳ श्वे॒त मा ल॑भेत । \newline
40. आ ल॑भेत लभे॒ता ल॑भेत सङ्ग्रा॒मे स॑ङ्ग्रा॒मे ल॑भे॒ता ल॑भेत सङ्ग्रा॒मे । \newline
41. ल॒भे॒त॒ स॒ङ्ग्रा॒मे स॑ङ्ग्रा॒मे ल॑भेत लभेत सङ्ग्रा॒मे संॅय॑त्ते॒ संॅय॑त्ते सङ्ग्रा॒मे ल॑भेत लभेत सङ्ग्रा॒मे संॅय॑त्ते । \newline
42. स॒ङ्ग्रा॒मे संॅय॑त्ते॒ संॅय॑त्ते सङ्ग्रा॒मे स॑ङ्ग्रा॒मे संॅय॑त्ते सम॒यका॑मः सम॒यका॑मः॒ संॅय॑त्ते सङ्ग्रा॒मे स॑ङ्ग्रा॒मे संॅय॑त्ते सम॒यका॑मः । \newline
43. स॒ङ्ग्रा॒म इति॑ सं - ग्रा॒मे । \newline
44. संॅय॑त्ते सम॒यका॑मः सम॒यका॑मः॒ संॅय॑त्ते॒ संॅय॑त्ते सम॒यका॑मो मि॒त्रम् मि॒त्रꣳ स॑म॒यका॑मः॒ संॅय॑त्ते॒ संॅय॑त्ते सम॒यका॑मो मि॒त्रम् । \newline
45. संॅय॑त्त॒ इति॒ सं - य॒त्ते॒ । \newline
46. स॒म॒यका॑मो मि॒त्रम् मि॒त्रꣳ स॑म॒यका॑मः सम॒यका॑मो मि॒त्र मे॒वैव मि॒त्रꣳ स॑म॒यका॑मः सम॒यका॑मो मि॒त्र मे॒व । \newline
47. स॒म॒यका॑म॒ इति॑ सम॒य - का॒मः॒ । \newline
48. मि॒त्र मे॒वैव मि॒त्रम् मि॒त्र मे॒व स्वेन॒ स्वेनै॒व मि॒त्रम् मि॒त्र मे॒व स्वेन॑ । \newline
49. ए॒व स्वेन॒ स्वेनै॒वैव स्वेन॑ भाग॒धेये॑न भाग॒धेये॑न॒ स्वेनै॒वैव स्वेन॑ भाग॒धेये॑न । \newline
50. स्वेन॑ भाग॒धेये॑न भाग॒धेये॑न॒ स्वेन॒ स्वेन॑ भाग॒धेये॒नो पोप॑ भाग॒धेये॑न॒ स्वेन॒ स्वेन॑ भाग॒धेये॒नोप॑ । \newline
51. भा॒ग॒धेये॒नो पोप॑ भाग॒धेये॑न भाग॒धेये॒नोप॑ धावति धाव॒त्युप॑ भाग॒धेये॑न भाग॒धेये॒नोप॑ धावति । \newline
52. भा॒ग॒धेये॒नेति॑ भाग - धेये॑न । \newline
53. उप॑ धावति धाव॒ त्युपोप॑ धावति॒ स स धा॑व॒ त्युपोप॑ धावति॒ सः । \newline
54. धा॒व॒ति॒ स स धा॑वति धावति॒ स ए॒वैव स धा॑वति धावति॒ स ए॒व । \newline
55. स ए॒वैव स स ए॒वैन॑ मेन मे॒व स स ए॒वैन᳚म् । \newline
56. ए॒वैन॑ मेन मे॒वैवैन॑म् मि॒त्रेण॑ मि॒त्रेणै॑न मे॒वैवैन॑म् मि॒त्रेण॑ । \newline
57. ए॒न॒म् मि॒त्रेण॑ मि॒त्रेणै॑न मेनम् मि॒त्रेण॒ सꣳ सम् मि॒त्रेणै॑न मेनम् मि॒त्रेण॒ सम् । \newline
58. मि॒त्रेण॒ सꣳ सम् मि॒त्रेण॑ मि॒त्रेण॒ सन्न॑यति नयति॒ सम् मि॒त्रेण॑ मि॒त्रेण॒ सन्न॑यति । \newline
59. सन्न॑यति नयति॒ सꣳ सन्न॑यति विशा॒लो वि॑शा॒लो न॑यति॒ सꣳ सन्न॑यति विशा॒लः । \newline
60. न॒य॒ति॒ वि॒शा॒लो वि॑शा॒लो न॑यति नयति विशा॒लो भ॑वति भवति विशा॒लो न॑यति नयति विशा॒लो भ॑वति । \newline
\pagebreak
\markright{ TS 2.1.8.5  \hfill https://www.vedavms.in \hfill}

\section{ TS 2.1.8.5 }

\textbf{TS 2.1.8.5 } \newline
\textbf{Samhita Paata} \newline

विशा॒लो भ॑वति॒ व्यव॑साययत्ये॒वैनं॑ प्राजाप॒त्यं कृ॒ष्णमा ल॑भेत॒ वृष्टि॑कामः प्र॒जाप॑ति॒र्वै वृष्‌ट्या॑ ईशे प्र॒जाप॑तिमे॒व स्वेन॑ भाग॒धेये॒नोप॑ धावति॒ स ए॒वास्मै॑ प॒र्जन्यं॑ ॅवर्.षयति कृ॒ष्णो भ॑वत्ये॒तद्वै वृष्‌ट्यै॑ रू॒पꣳ रू॒पेणै॒व वृष्टि॒मव॑ रुन्धे श॒बलो॑ भवति वि॒द्युत॑मे॒वास्मै॑ जनयि॒त्वा व॑र्.षयत्यवाशृ॒ङ्गो भ॑वति॒ वृष्टि॑मे॒वास्मै॒ नि य॑च्छति ( ) ॥ \newline

\textbf{Pada Paata} \newline

वि॒शा॒ल इति॑ वि - शा॒लः । भ॒व॒ति॒ । व्यव॑सायय॒तीति॑ वि - अव॑साययति । ए॒व । ए॒न॒म् । प्रा॒जा॒प॒त्यमिति॑ प्राजा - प॒त्यम् । कृ॒ष्णम् । एति॑ । ल॒भे॒त॒ । वृष्टि॑काम॒ इति॒ वृष्टि॑ - का॒मः॒ । प्र॒जाप॑ति॒रिति॑ प्र॒जा - प॒तिः॒ । वै । वृष्ट्याः᳚ । ई॒शे॒ । प्र॒जाप॑ति॒मिति॑ प्र॒जा - प॒ति॒म् । ए॒व । स्वेन॑ । भा॒ग॒धेये॒नेति॑ भाग-धेये॑न । उपेति॑ । धा॒व॒ति॒ । सः । ए॒व । अ॒स्मै॒ । प॒र्जन्य᳚म् । व॒र्॒.ष॒य॒ति॒ । कृ॒ष्णः । भ॒व॒ति॒ । ए॒तत् । वै । वृष्‌ट्यै᳚ । रू॒पम् । रू॒पेण॑ । ए॒व । वृष्टि᳚म् । अवेति॑ । रु॒न्धे॒ । श॒बलः॑ । भ॒व॒ति॒ । वि॒द्युत॒मिति॑ वि - द्युत᳚म् । ए॒व । अ॒स्मै॒ । ज॒न॒यि॒त्वा । व॒र्.॒ष॒य॒त॒ । अ॒वा॒शृ॒ङ्गः । भ॒व॒ति॒ । वृष्टि᳚म् । ए॒व । अ॒स्मै॒ । नीति॑ । य॒च्छ॒ति॒ ( ) ॥  \newline


\textbf{Krama Paata} \newline

वि॒शा॒लो भ॑वति । वि॒शा॒ल इति॑ वि - शा॒लः । भ॒व॒ति॒ व्यव॑साययति । व्यव॑साययत्ये॒व । व्यव॑सायय॒तीति॑ वि - अव॑साययति । ए॒वैन᳚म् । ए॒न॒म् प्रा॒जा॒प॒त्यम् । प्रा॒जा॒प॒त्यम् कृ॒ष्णम् । प्रा॒जा॒प॒त्यमिति॑ प्राजा - प॒त्यम् । कृ॒ष्णमा । आ ल॑भेत । ल॒भे॒त॒ वृष्टि॑कामः । वृष्टि॑कामः प्र॒जाप॑तिः । वृष्टि॑काम॒ इति॒ वृष्टि॑ - का॒मः॒ । प्र॒जाप॑ति॒र् वै । प्र॒जाप॑ति॒रिति॑ प्र॒जा - प॒तिः॒ । वै वृष्ट्याः᳚ । वृष्ट्या॑ ईशे । ई॒शे॒ प्र॒जाप॑तिम् । प्र॒जाप॑तिमे॒व । प्र॒जाप॑ति॒मिति॑ प्र॒जा - प॒ति॒म् । ए॒व स्वेन॑ । स्वेन॑ भाग॒धेये॑न । भा॒ग॒धेये॒नोप॑ । भा॒ग॒धेये॒नेति॑ भाग - धेये॑न । उप॑ धावति । धा॒व॒ति॒ सः । स ए॒व । ए॒वास्मै᳚ । अ॒स्मै॒ प॒र्जन्य᳚म् । प॒र्जन्यं॑ ॅवर्.षयति । व॒र्.॒षय॒ति॒ कृ॒ष्णः । कृ॒ष्णो भ॑वति । भ॒व॒त्ये॒तत् । ए॒तद् वै । वै वृष्ट्यै᳚ । वृष्ट्यै॑ रू॒पम् । रू॒पꣳ रू॒पेण॑ । रू॒पेणै॒व । ए॒व वृष्टि᳚म् । वृष्टि॒मव॑ । अव॑ रुन्धे । रु॒न्धे॒ श॒बलः॑ । श॒बलो॑ भवति । भ॒व॒ति॒ वि॒द्युत᳚म् । वि॒द्युत॑मे॒व । वि॒द्युत॒मिति॑ वि - द्युत᳚म् । ए॒वास्मै᳚ । अ॒स्मै॒ ज॒न॒यि॒त्वा । ज॒न॒यि॒त्वा व॑र्.षयति । व॒र्॒.ष॒य॒त्य॒वा॒शृ॒ङ्गः । अ॒वा॒शृ॒ङ्गो भ॑वति । भ॒व॒ति॒ वृष्टि᳚म् । वृष्टि॑मे॒व । ए॒वास्मै᳚ । अ॒स्मै॒ नि । नि य॑च्छति ( ) । य॒च्छ॒तीति॑ यच्छति । \newline

\textbf{Jatai Paata} \newline

1. वि॒शा॒लो भ॑वति भवति विशा॒लो वि॑शा॒लो भ॑वति । \newline
2. वि॒शा॒ल इति॑ वि - शा॒लः । \newline
3. भ॒व॒ति॒ व्यव॑साययति॒ व्यव॑साययति भवति भवति॒ व्यव॑साययति । \newline
4. व्यव॑सायय त्ये॒वैव व्यव॑साययति॒ व्यव॑सायय त्ये॒व । \newline
5. व्यव॑सायय॒तीति॑ वि - अव॑साययति । \newline
6. ए॒वैन॑ मेन मे॒वैवैन᳚म् । \newline
7. ए॒न॒म् प्रा॒जा॒प॒त्यम् प्रा॑जाप॒त्य मे॑न मेनम् प्राजाप॒त्यम् । \newline
8. प्रा॒जा॒प॒त्यम् कृ॒ष्णम् कृ॒ष्णम् प्रा॑जाप॒त्यम् प्रा॑जाप॒त्यम् कृ॒ष्णम् । \newline
9. प्रा॒जा॒प॒त्यमिति॑ प्राजा - प॒त्यम् । \newline
10. कृ॒ष्ण मा कृ॒ष्णम् कृ॒ष्ण मा । \newline
11. आ ल॑भेत लभे॒ता ल॑भेत । \newline
12. ल॒भे॒त॒ वृष्टि॑कामो॒ वृष्टि॑कामो लभेत लभेत॒ वृष्टि॑कामः । \newline
13. वृष्टि॑कामः प्र॒जाप॑तिः प्र॒जाप॑ति॒र् वृष्टि॑कामो॒ वृष्टि॑कामः प्र॒जाप॑तिः । \newline
14. वृष्टि॑काम॒ इति॒ वृष्टि॑ - का॒मः॒ । \newline
15. प्र॒जाप॑ति॒र् वै वै प्र॒जाप॑तिः प्र॒जाप॑ति॒र् वै । \newline
16. प्र॒जाप॑ति॒रिति॑ प्र॒जा - प॒तिः॒ । \newline
17. वै वृष्ट्या॒ वृष्ट्या॒ वै वै वृष्ट्याः᳚ । \newline
18. वृष्ट्या॑ ईश ईशे॒ वृष्ट्या॒ वृष्ट्या॑ ईशे । \newline
19. ई॒शे॒ प्र॒जाप॑तिम् प्र॒जाप॑ति मीश ईशे प्र॒जाप॑तिम् । \newline
20. प्र॒जाप॑ति मे॒वैव प्र॒जाप॑तिम् प्र॒जाप॑ति मे॒व । \newline
21. प्र॒जाप॑ति॒मिति॑ प्र॒जा - प॒ति॒म् । \newline
22. ए॒व स्वेन॒ स्वेनै॒वैव स्वेन॑ । \newline
23. स्वेन॑ भाग॒धेये॑न भाग॒धेये॑न॒ स्वेन॒ स्वेन॑ भाग॒धेये॑न । \newline
24. भा॒ग॒धेये॒नोपोप॑ भाग॒धेये॑न भाग॒धेये॒नोप॑ । \newline
25. भा॒ग॒धेये॒नेति॑ भाग - धेये॑न । \newline
26. उप॑ धावति धाव॒ त्युपोप॑ धावति । \newline
27. धा॒व॒ति॒ स स धा॑वति धावति॒ सः । \newline
28. स ए॒वैव स स ए॒व । \newline
29. ए॒वास्मा॑ अस्मा ए॒वैवास्मै᳚ । \newline
30. अ॒स्मै॒ प॒र्जन्य॑म् प॒र्जन्य॑ मस्मा अस्मै प॒र्जन्य᳚म् । \newline
31. प॒र्जन्यं॑ ॅवर्.षयति वर्.षयति प॒र्जन्य॑म् प॒र्जन्यं॑ ॅवर्.षयति । \newline
32. व॒र्॒.ष॒य॒ति॒ कृ॒ष्णः कृ॒ष्णो व॑र्.षयति वर्.षयति कृ॒ष्णः । \newline
33. कृ॒ष्णो भ॑वति भवति कृ॒ष्णः कृ॒ष्णो भ॑वति । \newline
34. भ॒व॒ त्ये॒त दे॒तद् भ॑वति भव त्ये॒तत् । \newline
35. ए॒तद् वै वा ए॒त दे॒तद् वै । \newline
36. वै वृष्ट्यै॒ वृष्ट्यै॒ वै वै वृष्ट्यै᳚ । \newline
37. वृष्ट्यै॑ रू॒पꣳ रू॒पं ॅवृष्ट्यै॒ वृष्ट्यै॑ रू॒पम् । \newline
38. रू॒पꣳ रू॒पेण॑ रू॒पेण॑ रू॒पꣳ रू॒पꣳ रू॒पेण॑ । \newline
39. रू॒पे णै॒वैव रू॒पेण॑ रू॒पे णै॒व । \newline
40. ए॒व वृष्टिं॒ ॅवृष्टि॑ मे॒वैव वृष्टि᳚म् । \newline
41. वृष्टि॒ मवाव॒ वृष्टिं॒ ॅवृष्टि॒ मव॑ । \newline
42. अव॑ रुन्धे रु॒न्धे ऽवाव॑ रुन्धे । \newline
43. रु॒न्धे॒ श॒बलः॑ श॒बलो॑ रुन्धे रुन्धे श॒बलः॑ । \newline
44. श॒बलो॑ भवति भवति श॒बलः॑ श॒बलो॑ भवति । \newline
45. भ॒व॒ति॒ वि॒द्युतं॑ ॅवि॒द्युत॑म् भवति भवति वि॒द्युत᳚म् । \newline
46. वि॒द्युत॑ मे॒वैव वि॒द्युतं॑ ॅवि॒द्युत॑ मे॒व । \newline
47. वि॒द्युत॒मिति॑ वि - द्युत᳚म् । \newline
48. ए॒वास्मा॑ अस्मा ए॒वैवास्मै᳚ । \newline
49. अ॒स्मै॒ ज॒न॒यि॒त्वा ज॑नयि॒त्वा अ॑स्मा अस्मै जनयि॒त्वा । \newline
50. ज॒न॒यि॒त्वा व॑र्.षयति वर्.षयति जनयि॒त्वा ज॑नयि॒त्वा व॑र्.षयति । \newline
51. व॒र्॒.ष॒य॒ त्य॒वा॒शृ॒ङ्गो॑ ऽवाशृ॒ङ्गो व॑र्.षयति वर्.षय त्यवाशृ॒ङ्गः । \newline
52. अ॒वा॒शृ॒ङ्गो भ॑वति भव त्यवाशृ॒ङ्गो॑ ऽवाशृ॒ङ्गो भ॑वति । \newline
53. भ॒व॒ति॒ वृष्टिं॒ ॅवृष्टि॑म् भवति भवति॒ वृष्टि᳚म् । \newline
54. वृष्टि॑ मे॒वैव वृष्टिं॒ ॅवृष्टि॑ मे॒व । \newline
55. ए॒वास्मा॑ अस्मा ए॒वैवास्मै᳚ । \newline
56. अ॒स्मै॒ नि न्य॑स्मा अस्मै॒ नि । \newline
57. नि य॑च्छति यच्छति॒ नि नि य॑च्छति । \newline
58. य॒च्छ॒तीति॑ यच्छति । \newline

\textbf{Ghana Paata } \newline

1. वि॒शा॒लो भ॑वति भवति विशा॒लो वि॑शा॒लो भ॑वति॒ व्यव॑साययति॒ व्यव॑साययति भवति विशा॒लो वि॑शा॒लो भ॑वति॒ व्यव॑साययति । \newline
2. वि॒शा॒ल इति॑ वि - शा॒लः । \newline
3. भ॒व॒ति॒ व्यव॑साययति॒ व्यव॑साययति भवति भवति॒ व्यव॑सायय त्ये॒वैव व्यव॑साययति भवति भवति॒ व्यव॑सायय त्ये॒व । \newline
4. व्यव॑सायय त्ये॒वैव व्यव॑साययति॒ व्यव॑सायय त्ये॒वैन॑ मेन मे॒व व्यव॑साययति॒ व्यव॑सायय त्ये॒वैन᳚म् । \newline
5. व्यव॑सायय॒तीति॑ वि - अव॑साययति । \newline
6. ए॒वैन॑ मेन मे॒वैवैन॑म् प्राजाप॒त्यम् प्रा॑जाप॒त्य मे॑न मे॒वैवैन॑म् प्राजाप॒त्यम् । \newline
7. ए॒न॒म् प्रा॒जा॒प॒त्यम् प्रा॑जाप॒त्य मे॑न मेनम् प्राजाप॒त्यम् कृ॒ष्णम् कृ॒ष्णम् प्रा॑जाप॒त्य मे॑न मेनम् प्राजाप॒त्यम् कृ॒ष्णम् । \newline
8. प्रा॒जा॒प॒त्यम् कृ॒ष्णम् कृ॒ष्णम् प्रा॑जाप॒त्यम् प्रा॑जाप॒त्यम् कृ॒ष्ण मा कृ॒ष्णम् प्रा॑जाप॒त्यम् प्रा॑जाप॒त्यम् कृ॒ष्ण मा । \newline
9. प्रा॒जा॒प॒त्यमिति॑ प्राजा - प॒त्यम् । \newline
10. कृ॒ष्ण मा कृ॒ष्णम् कृ॒ष्ण मा ल॑भेत लभे॒ता कृ॒ष्णम् कृ॒ष्ण मा ल॑भेत । \newline
11. आ ल॑भेत लभे॒ता ल॑भेत॒ वृष्टि॑कामो॒ वृष्टि॑कामो लभे॒ता ल॑भेत॒ वृष्टि॑कामः । \newline
12. ल॒भे॒त॒ वृष्टि॑कामो॒ वृष्टि॑कामो लभेत लभेत॒ वृष्टि॑कामः प्र॒जाप॑तिः प्र॒जाप॑ति॒र् वृष्टि॑कामो लभेत लभेत॒ वृष्टि॑कामः प्र॒जाप॑तिः । \newline
13. वृष्टि॑कामः प्र॒जाप॑तिः प्र॒जाप॑ति॒र् वृष्टि॑कामो॒ वृष्टि॑कामः प्र॒जाप॑ति॒र् वै वै प्र॒जाप॑ति॒र् वृष्टि॑कामो॒ वृष्टि॑कामः प्र॒जाप॑ति॒र् वै । \newline
14. वृष्टि॑काम॒ इति॒ वृष्टि॑ - का॒मः॒ । \newline
15. प्र॒जाप॑ति॒र् वै वै प्र॒जाप॑तिः प्र॒जाप॑ति॒र् वै वृष्ट्या॒ वृष्ट्या॒ वै प्र॒जाप॑तिः प्र॒जाप॑ति॒र् वै वृष्ट्याः᳚ । \newline
16. प्र॒जाप॑ति॒रिति॑ प्र॒जा - प॒तिः॒ । \newline
17. वै वृष्ट्या॒ वृष्ट्या॒ वै वै वृष्ट्या॑ ईश ईशे॒ वृष्ट्या॒ वै वै वृष्ट्या॑ ईशे । \newline
18. वृष्ट्या॑ ईश ईशे॒ वृष्ट्या॒ वृष्ट्या॑ ईशे प्र॒जाप॑तिम् प्र॒जाप॑ति मीशे॒ वृष्ट्या॒ वृष्ट्या॑ ईशे प्र॒जाप॑तिम् । \newline
19. ई॒शे॒ प्र॒जाप॑तिम् प्र॒जाप॑ति मीश ईशे प्र॒जाप॑ति मे॒वैव प्र॒जाप॑ति मीश ईशे प्र॒जाप॑ति मे॒व । \newline
20. प्र॒जाप॑ति मे॒वैव प्र॒जाप॑तिम् प्र॒जाप॑ति मे॒व स्वेन॒ स्वेनै॒व प्र॒जाप॑तिम् प्र॒जाप॑ति मे॒व स्वेन॑ । \newline
21. प्र॒जाप॑ति॒मिति॑ प्र॒जा - प॒ति॒म् । \newline
22. ए॒व स्वेन॒ स्वेनै॒वैव स्वेन॑ भाग॒धेये॑न भाग॒धेये॑न॒ स्वेनै॒वैव स्वेन॑ भाग॒धेये॑न । \newline
23. स्वेन॑ भाग॒धेये॑न भाग॒धेये॑न॒ स्वेन॒ स्वेन॑ भाग॒धेये॒नो पोप॑ भाग॒धेये॑न॒ स्वेन॒ स्वेन॑ भाग॒धेये॒नोप॑ । \newline
24. भा॒ग॒धेये॒नो पोप॑ भाग॒धेये॑न भाग॒धेये॒नोप॑ धावति धाव॒त्युप॑ भाग॒धेये॑न भाग॒धेये॒नोप॑ धावति । \newline
25. भा॒ग॒धेये॒नेति॑ भाग - धेये॑न । \newline
26. उप॑ धावति धाव॒ त्युपोप॑ धावति॒ स स धा॑व॒ त्युपोप॑ धावति॒ सः । \newline
27. धा॒व॒ति॒ स स धा॑वति धावति॒ स ए॒वैव स धा॑वति धावति॒ स ए॒व । \newline
28. स ए॒वैव स स ए॒वास्मा॑ अस्मा ए॒व स स ए॒वास्मै᳚ । \newline
29. ए॒वास्मा॑ अस्मा ए॒वैवास्मै॑ प॒र्जन्य॑म् प॒र्जन्य॑ मस्मा ए॒वैवास्मै॑ प॒र्जन्य᳚म् । \newline
30. अ॒स्मै॒ प॒र्जन्य॑म् प॒र्जन्य॑ मस्मा अस्मै प॒र्जन्यं॑ ॅवर्.षयति वर्.षयति प॒र्जन्य॑ मस्मा अस्मै प॒र्जन्यं॑ ॅवर्.षयति । \newline
31. प॒र्जन्यं॑ ॅवर्.षयति वर्.षयति प॒र्जन्य॑म् प॒र्जन्यं॑ ॅवर्.षयति कृ॒ष्णः कृ॒ष्णो व॑र्.षयति प॒र्जन्य॑म् प॒र्जन्यं॑ ॅवर्.षयति कृ॒ष्णः । \newline
32. व॒र्॒.ष॒य॒ति॒ कृ॒ष्णः कृ॒ष्णो व॑र्.षयति वर्.षयति कृ॒ष्णो भ॑वति भवति कृ॒ष्णो व॑र्.षयति वर्.षयति कृ॒ष्णो भ॑वति । \newline
33. कृ॒ष्णो भ॑वति भवति कृ॒ष्णः कृ॒ष्णो भ॑व त्ये॒तदे॒तद् भ॑वति कृ॒ष्णः कृ॒ष्णो भ॑वत्ये॒तत् । \newline
34. भ॒व॒ त्ये॒तदे॒तद् भ॑वति भवत्ये॒तद् वै वा ए॒तद् भ॑वति भवत्ये॒तद् वै । \newline
35. ए॒तद् वै वा ए॒तदे॒तद् वै वृष्ट्यै॒ वृष्ट्यै॒ वा ए॒तदे॒तद् वै वृष्ट्यै᳚ । \newline
36. वै वृष्ट्यै॒ वृष्ट्यै॒ वै वै वृष्ट्यै॑ रू॒पꣳ रू॒पं ॅवृष्ट्यै॒ वै वै वृष्ट्यै॑ रू॒पम् । \newline
37. वृष्ट्यै॑ रू॒पꣳ रू॒पं ॅवृष्ट्यै॒ वृष्ट्यै॑ रू॒पꣳ रू॒पेण॑ रू॒पेण॑ रू॒पं ॅवृष्ट्यै॒ वृष्ट्यै॑ रू॒पꣳ रू॒पेण॑ । \newline
38. रू॒पꣳ रू॒पेण॑ रू॒पेण॑ रू॒पꣳ रू॒पꣳ रू॒पेणै॒वैव रू॒पेण॑ रू॒पꣳ रू॒पꣳ रू॒पेणै॒व । \newline
39. रू॒पेणै॒वैव रू॒पेण॑ रू॒पेणै॒व वृष्टिं॒ ॅवृष्टि॑ मे॒व रू॒पेण॑ रू॒पेणै॒व वृष्टि᳚म् । \newline
40. ए॒व वृष्टिं॒ ॅवृष्टि॑ मे॒वैव वृष्टि॒ मवाव॒ वृष्टि॑ मे॒वैव वृष्टि॒ मव॑ । \newline
41. वृष्टि॒ मवाव॒ वृष्टिं॒ ॅवृष्टि॒ मव॑ रुन्धे रु॒न्धे ऽव॒ वृष्टिं॒ ॅवृष्टि॒ मव॑ रुन्धे । \newline
42. अव॑ रुन्धे रु॒न्धे ऽवाव॑ रुन्धे श॒बलः॑ श॒बलो॑ रु॒न्धे ऽवाव॑ रुन्धे श॒बलः॑ । \newline
43. रु॒न्धे॒ श॒बलः॑ श॒बलो॑ रुन्धे रुन्धे श॒बलो॑ भवति भवति श॒बलो॑ रुन्धे रुन्धे श॒बलो॑ भवति । \newline
44. श॒बलो॑ भवति भवति श॒बलः॑ श॒बलो॑ भवति वि॒द्युतं॑ ॅवि॒द्युत॑म् भवति श॒बलः॑ श॒बलो॑ भवति वि॒द्युत᳚म् । \newline
45. भ॒व॒ति॒ वि॒द्युतं॑ ॅवि॒द्युत॑म् भवति भवति वि॒द्युत॑ मे॒वैव वि॒द्युत॑म् भवति भवति वि॒द्युत॑ मे॒व । \newline
46. वि॒द्युत॑ मे॒वैव वि॒द्युतं॑ ॅवि॒द्युत॑ मे॒वास्मा॑ अस्मा ए॒व वि॒द्युतं॑ ॅवि॒द्युत॑ मे॒वास्मै᳚ । \newline
47. वि॒द्युत॒मिति॑ वि - द्युत᳚म् । \newline
48. ए॒वास्मा॑ अस्मा ए॒वैवास्मै॑ जनयि॒त्वा ज॑नयि॒त्वा अ॑स्मा ए॒वैवास्मै॑ जनयि॒त्वा । \newline
49. अ॒स्मै॒ ज॒न॒यि॒त्वा ज॑नयि॒त्वा अ॑स्मा अस्मै जनयि॒त्वा व॑र्.षयति वर्.षयति जनयि॒त्वा अ॑स्मा अस्मै जनयि॒त्वा व॑र्.षयति । \newline
50. ज॒न॒यि॒त्वा व॑र्.षयति वर्.षयति जनयि॒त्वा ज॑नयि॒त्वा व॑र्.षय त्यवाशृ॒ङ्गो॑ ऽवाशृ॒ङ्गो व॑र्.षयति जनयि॒त्वा ज॑नयि॒त्वा व॑र्.षय त्यवाशृ॒ङ्गः । \newline
51. व॒र्॒.ष॒य॒ त्य॒वा॒शृ॒ङ्गो॑ ऽवाशृ॒ङ्गो व॑र्.षयति वर्.षय त्यवाशृ॒ङ्गो भ॑वति भव त्यवाशृ॒ङ्गो व॑र्.षयति वर्.षय त्यवाशृ॒ङ्गो भ॑वति । \newline
52. अ॒वा॒शृ॒ङ्गो भ॑वति भव त्यवाशृ॒ङ्गो॑ ऽवाशृ॒ङ्गो भ॑वति॒ वृष्टिं॒ ॅवृष्टि॑म् भव त्यवाशृ॒ङ्गो॑ ऽवाशृ॒ङ्गो भ॑वति॒ वृष्टि᳚म् । \newline
53. भ॒व॒ति॒ वृष्टिं॒ ॅवृष्टि॑म् भवति भवति॒ वृष्टि॑ मे॒वैव वृष्टि॑म् भवति भवति॒ वृष्टि॑ मे॒व । \newline
54. वृष्टि॑ मे॒वैव वृष्टिं॒ ॅवृष्टि॑ मे॒वास्मा॑ अस्मा ए॒व वृष्टिं॒ ॅवृष्टि॑ मे॒वास्मै᳚ । \newline
55. ए॒वास्मा॑ अस्मा ए॒वैवास्मै॒ नि न्य॑स्मा ए॒वैवास्मै॒ नि । \newline
56. अ॒स्मै॒ नि न्य॑स्मा अस्मै॒ नि य॑च्छति यच्छति॒ न्य॑स्मा अस्मै॒ नि य॑च्छति । \newline
57. नि य॑च्छति यच्छति॒ नि नि य॑च्छति । \newline
58. य॒च्छ॒तीति॑ यच्छति । \newline
\pagebreak
\markright{ TS 2.1.9.1  \hfill https://www.vedavms.in \hfill}

\section{ TS 2.1.9.1 }

\textbf{TS 2.1.9.1 } \newline
\textbf{Samhita Paata} \newline

वरु॑णꣳ सुषुवा॒णम॒न्नाद्यं॒ नोपा॑नम॒थ् स ए॒तां ॅवा॑रु॒णीं कृ॒ष्णां ॅव॒शाम॑पश्य॒त् ताꣳ स्वायै॑ दे॒वता॑या॒ आल॑भत॒ ततो॒ वै तम॒न्नाद्-य॒मुपा॑नम॒द्-यमल॑म॒न्नाद्या॑य॒ सन्त॑म॒न्नाद्यं॒ नोप॒नमे॒थ् स ए॒तां ॅवा॑रु॒णीं कृ॒ष्णां ॅव॒शामा ल॑भेत॒ वरु॑णमे॒व स्वेन॑ भाग॒धेये॒नोप॑ धावति॒ स ए॒वास्मा॒ अन्नं॒ प्र य॑च्छत्यन्ना॒द - [  ] \newline

\textbf{Pada Paata} \newline

वरु॑णम् । सु॒षु॒वा॒णम् । अ॒न्नाद्य॒मित्य॑न्न - अद्य᳚म् । न । उपेति॑ । अ॒न॒म॒त् । सः । ए॒ताम् । वा॒रु॒णीम् । कृ॒ष्णाम् । व॒शाम् । अ॒प॒श्य॒त् । ताम् । स्वायै᳚ । दे॒वता॑यै । एति॑ । अ॒ल॒भ॒त॒ । ततः॑ । वै । तम् । अ॒न्नाद्य॒मित्य॑न्न - अद्य᳚म् । उपेति॑ । अ॒न॒म॒त् । यम् । अल᳚म् । अ॒न्नाद्या॒येत्य॑न्न - अद्या॑य । सन्त᳚म् । अ॒न्नाद्य॒मित्य॑न्न - अद्य᳚म् । न । उ॒प॒नमे॒दित्यु॑प-नमे᳚त् । सः । ए॒ताम् । वा॒रु॒णीम् । कृ॒ष्णाम् । व॒शाम् । एति॑ । ल॒भे॒त॒ । वरु॑णम् । ए॒व । स्वेन॑ । भा॒ग॒धेये॒नेति॑ भाग - धेये॑न । उपेति॑ । धा॒व॒ति॒ । सः । ए॒व । अ॒स्मै॒ । अन्न᳚म् । प्रेति॑ । य॒च्छ॒ति॒ । अ॒न्ना॒द इत्य॑न्न - अ॒दः ।  \newline


\textbf{Krama Paata} \newline

वरु॑णꣳ सुषुवा॒णम् । सु॒षु॒वा॒णम॒न्नाद्य᳚म् । अ॒न्नाद्य॒म् न । अ॒न्नाद्य॒मित्य॑न्न - अद्य᳚म् । नोप॑ । उपा॑नमत् । अ॒न॒म॒थ् सः । स ए॒ताम् । ए॒तां ॅवा॑रु॒णीम् । वा॒रु॒णीम् कृ॒ष्णाम् । कृ॒ष्णां ॅव॒शाम् । व॒शाम॑पश्यत् । अ॒प॒श्य॒त् ताम् । ताꣳ स्वायै᳚ । स्वायै॑ दे॒वता॑यै । दे॒वता॑या॒ आ । आ ऽल॑भत । अ॒ल॒भ॒त॒ ततः॑ । ततो॒ वै । वै तम् । तम॒न्नाद्य᳚म् । अ॒न्नाद्य॒मुप॑ । अ॒न्नाद्य॒मित्य॑न्न - अद्य᳚म् । उपा॑नमत् । अ॒न॒म॒द् यम् । यमल᳚म् । अल॑म॒न्नाद्या॑य । अ॒न्नाद्या॑य॒ सन्त᳚म् । अ॒न्नाद्या॒येत्य॑न्न - अद्या॑य । सन्त॑म॒न्नाद्य᳚म् । अ॒न्नाद्य॒म् न । अ॒न्नाद्य॒मित्य॑न्न - अद्य᳚म् । नोप॒नमे᳚त् । उ॒प॒नमे॒थ् सः । उ॒प॒नमे॒दित्यु॑प - नमे᳚त् । स ए॒ताम् । ए॒तां ॅवा॑रु॒णीम् । वा॒रु॒णीम् कृ॒ष्णाम् । कृ॒ष्णां ॅव॒शाम् । व॒शामा । आ ल॑भेत । ल॒भे॒त॒ वरु॑णम् । वरु॑णमे॒व । ए॒व स्वेन॑ । स्वेन॑ भाग॒धेये॑न । भा॒ग॒धेये॒नोप॑ । भा॒ग॒धेये॒नेति॑ भाग - धेये॑न । उप॑ धावति । धा॒व॒ति॒ सः । स ए॒व । ए॒वास्मै᳚ । अ॒स्मा॒ अन्न᳚म् । अन्न॒म् प्र । प्र य॑च्छति । य॒च्छ॒त्य॒न्ना॒दः । अ॒न्ना॒द ए॒व । अ॒न्ना॒द इत्य॑न्न - अ॒दः \newline

\textbf{Jatai Paata} \newline

1. वरु॑णꣳ सुषुवा॒णꣳ सु॑षुवा॒णं ॅवरु॑णं॒ ॅवरु॑णꣳ सुषुवा॒णम् । \newline
2. सु॒षु॒वा॒ण म॒न्नाद्य॑ म॒न्नाद्यꣳ॑ सुषुवा॒णꣳ सु॑षुवा॒ण म॒न्नाद्य᳚म् । \newline
3. अ॒न्नाद्य॒न्न नान्नाद्य॑ म॒न्नाद्य॒न्न । \newline
4. अ॒न्नाद्य॒मित्य॑न्न - अद्य᳚म् । \newline
5. नोपोप॒ न नोप॑ । \newline
6. उपा॑नम दनम॒ दुपोपा॑नमत् । \newline
7. अ॒न॒म॒थ् स सो॑ ऽनम दनम॒थ् सः । \newline
8. स ए॒ता मे॒ताꣳ स स ए॒ताम् । \newline
9. ए॒तां ॅवा॑रु॒णीं ॅवा॑रु॒णी मे॒ता मे॒तां ॅवा॑रु॒णीम् । \newline
10. वा॒रु॒णीम् कृ॒ष्णाम् कृ॒ष्णां ॅवा॑रु॒णीं ॅवा॑रु॒णीम् कृ॒ष्णाम् । \newline
11. कृ॒ष्णां ॅव॒शां ॅव॒शाम् कृ॒ष्णाम् कृ॒ष्णां ॅव॒शाम् । \newline
12. व॒शा म॑पश्य दपश्यद् व॒शां ॅव॒शा म॑पश्यत् । \newline
13. अ॒प॒श्य॒त् ताम् ता म॑पश्य दपश्य॒त् ताम् । \newline
14. ताꣳ स्वायै॒ स्वायै॒ ताम् ताꣳ स्वायै᳚ । \newline
15. स्वायै॑ दे॒वता॑यै दे॒वता॑यै॒ स्वायै॒ स्वायै॑ दे॒वता॑यै । \newline
16. दे॒वता॑या॒ आ दे॒वता॑यै दे॒वता॑या॒ आ । \newline
17. आ ऽल॑भता लभ॒ता ऽल॑भत । \newline
18. अ॒ल॒भ॒त॒ तत॒ स्ततो॑ ऽलभता लभत॒ ततः॑ । \newline
19. ततो॒ वै वै तत॒ स्ततो॒ वै । \newline
20. वै तम् तं ॅवै वै तम् । \newline
21. त म॒न्नाद्य॑ म॒न्नाद्य॒म् तम् त म॒न्नाद्य᳚म् । \newline
22. अ॒न्नाद्य॒ मुपोपा॒न्नाद्य॑ म॒न्नाद्य॒ मुप॑ । \newline
23. अ॒न्नाद्य॒मित्य॑न्न - अद्य᳚म् । \newline
24. उपा॑नम दनम॒ दुपोपा॑नमत् । \newline
25. अ॒न॒म॒द् यं ॅय म॑नम दनम॒द् यम् । \newline
26. य मल॒ मलं॒ ॅयं ॅय मल᳚म् । \newline
27. अल॑ म॒न्नाद्या॑या॒ न्नाद्या॒या ल॒ मल॑ म॒न्नाद्या॑य । \newline
28. अ॒न्नाद्या॑य॒ सन्तꣳ॒॒ सन्त॑ म॒न्नाद्या॑या॒ न्नाद्या॑य॒ सन्त᳚म् । \newline
29. अ॒न्नाद्या॒येत्य॑न्न - अद्या॑य । \newline
30. सन्त॑ म॒न्नाद्य॑ म॒न्नाद्यꣳ॒॒ सन्तꣳ॒॒ सन्त॑ म॒न्नाद्य᳚म् । \newline
31. अ॒न्नाद्य॒न्न नान्नाद्य॑ म॒न्नाद्य॒न्न । \newline
32. अ॒न्नाद्य॒मित्य॑न्न - अद्य᳚म् । \newline
33. नोप॒नमे॑ दुप॒नमे॒न् न नोप॒नमे᳚त् । \newline
34. उ॒प॒नमे॒थ् स स उ॑प॒नमे॑ दुप॒नमे॒थ् सः । \newline
35. उ॒प॒नमे॒दित्यु॑प - नमे᳚त् । \newline
36. स ए॒ता मे॒ताꣳ स स ए॒ताम् । \newline
37. ए॒तां ॅवा॑रु॒णीं ॅवा॑रु॒णी मे॒ता मे॒तां ॅवा॑रु॒णीम् । \newline
38. वा॒रु॒णीम् कृ॒ष्णाम् कृ॒ष्णां ॅवा॑रु॒णीं ॅवा॑रु॒णीम् कृ॒ष्णाम् । \newline
39. कृ॒ष्णां ॅव॒शां ॅव॒शाम् कृ॒ष्णाम् कृ॒ष्णां ॅव॒शाम् । \newline
40. व॒शा मा व॒शां ॅव॒शा मा । \newline
41. आ ल॑भेत लभे॒ता ल॑भेत । \newline
42. ल॒भे॒त॒ वरु॑णं॒ ॅवरु॑णम् ॅलभेत लभेत॒ वरु॑णम् । \newline
43. वरु॑ण मे॒वैव वरु॑णं॒ ॅवरु॑ण मे॒व । \newline
44. ए॒व स्वेन॒ स्वेनै॒वैव स्वेन॑ । \newline
45. स्वेन॑ भाग॒धेये॑न भाग॒धेये॑न॒ स्वेन॒ स्वेन॑ भाग॒धेये॑न । \newline
46. भा॒ग॒धेये॒नोपोप॑ भाग॒धेये॑न भाग॒धेये॒नोप॑ । \newline
47. भा॒ग॒धेये॒नेति॑ भाग - धेये॑न । \newline
48. उप॑ धावति धाव॒ त्युपोप॑ धावति । \newline
49. धा॒व॒ति॒ स स धा॑वति धावति॒ सः । \newline
50. स ए॒वैव स स ए॒व । \newline
51. ए॒वास्मा॑ अस्मा ए॒वैवास्मै᳚ । \newline
52. अ॒स्मा॒ अन्न॒ मन्न॑ मस्मा अस्मा॒ अन्न᳚म् । \newline
53. अन्न॒म् प्र प्रान्न॒ मन्न॒म् प्र । \newline
54. प्र य॑च्छति यच्छति॒ प्र प्र य॑च्छति । \newline
55. य॒च्छ॒ त्य॒न्ना॒दो᳚ ऽन्ना॒दो य॑च्छति यच्छ त्यन्ना॒दः । \newline
56. अ॒न्ना॒द ए॒वैवा न्ना॒दो᳚ ऽन्ना॒द ए॒व । \newline
57. अ॒न्ना॒द इत्य॑न्न - अ॒दः । \newline

\textbf{Ghana Paata } \newline

1. वरु॑णꣳ सुषुवा॒णꣳ सु॑षुवा॒णं ॅवरु॑णं॒ ॅवरु॑णꣳ सुषुवा॒ण म॒न्नाद्य॑ म॒न्नाद्यꣳ॑ सुषुवा॒णं ॅवरु॑णं॒ ॅवरु॑णꣳ सुषुवा॒ण म॒न्नाद्य᳚म् । \newline
2. सु॒षु॒वा॒ण म॒न्नाद्य॑ म॒न्नाद्यꣳ॑ सुषुवा॒णꣳ सु॑षुवा॒ण म॒न्नाद्य॒न्न नान्नाद्यꣳ॑ सुषुवा॒णꣳ सु॑षुवा॒ण म॒न्नाद्य॒न्न । \newline
3. अ॒न्नाद्य॒न्न नान्नाद्य॑ म॒न्नाद्य॒ न्नोपोप॒ नान्नाद्य॑ म॒न्ना द्य॒न्नोप॑ । \newline
4. अ॒न्नाद्य॒मित्य॑न्न - अद्य᳚म् । \newline
5. नोपोप॒ न नोपा॑नम दनम॒दुप॒ न नोपा॑नमत् । \newline
6. उपा॑नमदन म॒दुपोपा॑ नम॒थ् स सो॑ ऽनम॒ दुपोपा॑नम॒थ् सः । \newline
7. अ॒न॒म॒थ् स सो॑ ऽनम दनम॒थ् स ए॒ता मे॒ताꣳ सो॑ ऽनम दनम॒थ् स ए॒ताम् । \newline
8. स ए॒ता मे॒ताꣳ स स ए॒तां ॅवा॑रु॒णीं ॅवा॑रु॒णी मे॒ताꣳ स स ए॒तां ॅवा॑रु॒णीम् । \newline
9. ए॒तां ॅवा॑रु॒णीं ॅवा॑रु॒णी मे॒ता मे॒तां ॅवा॑रु॒णीम् कृ॒ष्णाम् कृ॒ष्णां ॅवा॑रु॒णी मे॒ता मे॒तां ॅवा॑रु॒णीम् कृ॒ष्णाम् । \newline
10. वा॒रु॒णीम् कृ॒ष्णाम् कृ॒ष्णां ॅवा॑रु॒णीं ॅवा॑रु॒णीम् कृ॒ष्णां ॅव॒शां ॅव॒शाम् कृ॒ष्णां ॅवा॑रु॒णीं ॅवा॑रु॒णीम् कृ॒ष्णां ॅव॒शाम् । \newline
11. कृ॒ष्णां ॅव॒शां ॅव॒शाम् कृ॒ष्णाम् कृ॒ष्णां ॅव॒शा म॑पश्य दपश्यद् व॒शाम् कृ॒ष्णाम् कृ॒ष्णां ॅव॒शा म॑पश्यत् । \newline
12. व॒शा म॑पश्य दपश्यद् व॒शां ॅव॒शा म॑पश्य॒त् ताम् ता म॑पश्यद् व॒शां ॅव॒शा म॑पश्य॒त् ताम् । \newline
13. अ॒प॒श्य॒त् ताम् ता म॑पश्य दपश्य॒त् ताꣳ स्वायै॒ स्वायै॒ ता म॑पश्य दपश्य॒त् ताꣳ स्वायै᳚ । \newline
14. ताꣳ स्वायै॒ स्वायै॒ ताम् ताꣳ स्वायै॑ दे॒वता॑यै दे॒वता॑यै॒ स्वायै॒ ताम् ताꣳ स्वायै॑ दे॒वता॑यै । \newline
15. स्वायै॑ दे॒वता॑यै दे॒वता॑यै॒ स्वायै॒ स्वायै॑ दे॒वता॑या॒ आ दे॒वता॑यै॒ स्वायै॒ स्वायै॑ दे॒वता॑या॒ आ । \newline
16. दे॒वता॑या॒ आ दे॒वता॑यै दे॒वता॑या॒ आ अ॑लभता लभ॒ता दे॒वता॑यै दे॒वता॑या॒ आ अ॑लभत । \newline
17. आ ऽल॑भता लभ॒ता ऽल॑भत॒ तत॒ स्ततो॑ ऽलभ॒ता ऽल॑भत॒ ततः॑ । \newline
18. अ॒ल॒भ॒त॒ तत॒ स्ततो॑ ऽलभता लभत॒ ततो॒ वै वै ततो॑ ऽलभता लभत॒ ततो॒ वै । \newline
19. ततो॒ वै वै तत॒ स्ततो॒ वै तम् तं ॅवै तत॒ स्ततो॒ वै तम् । \newline
20. वै तम् तं ॅवै वै त म॒न्नाद्य॑ म॒न्नाद्य॒म् तं ॅवै वै त म॒न्नाद्य᳚म् । \newline
21. त म॒न्नाद्य॑ म॒न्नाद्य॒म् तम् त म॒न्नाद्य॒ मुपोपा॒ न्नाद्य॒म् तम् त म॒न्नाद्य॒ मुप॑ । \newline
22. अ॒न्नाद्य॒ मुपोपा॒ न्नाद्य॑ म॒न्नाद्य॒ मुपा॑नम दनम॒ दुपा॒न्नाद्य॑ म॒न्नाद्य॒ मुपा॑नमत् । \newline
23. अ॒न्नाद्य॒मित्य॑न्न - अद्य᳚म् । \newline
24. उपा॑नम दनम॒ दुपोपा॑ नम॒द् यं ॅय म॑नम॒ दुपोपा॑ नम॒द् यम् । \newline
25. अ॒न॒म॒द् यं ॅय म॑नम दनम॒द् य मल॒ मलं॒ ॅय म॑नम दनम॒द् य मल᳚म् । \newline
26. य मल॒ मलं॒ ॅयं ॅय मल॑ म॒न्ना द्या॑या॒न्नाद्या॒ यालं॒ ॅयं ॅय मल॑ म॒न्नाद्या॑य । \newline
27. अल॑ म॒न्नाद्या॑या॒ न्नाद्या॒याल॒ मल॑ म॒न्नाद्या॑य॒ सन्तꣳ॒॒ सन्त॑ म॒न्ना द्या॒याल॒ मल॑ म॒न्नाद्या॑य॒ सन्त᳚म् । \newline
28. अ॒न्नाद्या॑य॒ सन्तꣳ॒॒ सन्त॑ म॒न्नाद्या॑ या॒न्नाद्या॑य॒ सन्त॑ म॒न्नाद्य॑ म॒न्नाद्यꣳ॒॒ सन्त॑ म॒न्नाद्या॑या॒ न्नाद्या॑य॒ सन्त॑ म॒न्नाद्य᳚म् । \newline
29. अ॒न्नाद्या॒येत्य॑न्न - अद्या॑य । \newline
30. सन्त॑ म॒न्नाद्य॑ म॒न्नाद्यꣳ॒॒ सन्तꣳ॒॒ सन्त॑ म॒न्नाद्य॒न्न नान्नाद्यꣳ॒॒ सन्तꣳ॒॒ सन्त॑ म॒न्नाद्य॒न्न । \newline
31. अ॒न्नाद्य॒न्न नान्नाद्य॑ म॒न्नाद्य॒न् नोप॒नमे॑ दुप॒नमे॒न् नान्नाद्य॑ म॒न्नाद्य॒न् नोप॒नमे᳚त् । \newline
32. अ॒न्नाद्य॒मित्य॑न्न - अद्य᳚म् । \newline
33. नोप॒नमे॑ दुप॒नमे॒न् न नोप॒नमे॒थ् स स उ॑प॒नमे॒न् न नोप॒नमे॒थ् सः । \newline
34. उ॒प॒नमे॒थ् स स उ॑प॒नमे॑ दुप॒नमे॒थ् स ए॒ता मे॒ताꣳ स उ॑प॒नमे॑ दुप॒नमे॒थ् स ए॒ताम् । \newline
35. उ॒प॒नमे॒दित्यु॑प - नमे᳚त् । \newline
36. स ए॒ता मे॒ताꣳ स स ए॒तां ॅवा॑रु॒णीं ॅवा॑रु॒णी मे॒ताꣳ स स ए॒तां ॅवा॑रु॒णीम् । \newline
37. ए॒तां ॅवा॑रु॒णीं ॅवा॑रु॒णी मे॒ता मे॒तां ॅवा॑रु॒णीम् कृ॒ष्णाम् कृ॒ष्णां ॅवा॑रु॒णी मे॒ता मे॒तां ॅवा॑रु॒णीम् कृ॒ष्णाम् । \newline
38. वा॒रु॒णीम् कृ॒ष्णाम् कृ॒ष्णां ॅवा॑रु॒णीं ॅवा॑रु॒णीम् कृ॒ष्णां ॅव॒शां ॅव॒शाम् कृ॒ष्णां ॅवा॑रु॒णीं ॅवा॑रु॒णीम् कृ॒ष्णां ॅव॒शाम् । \newline
39. कृ॒ष्णां ॅव॒शां ॅव॒शाम् कृ॒ष्णाम् कृ॒ष्णां ॅव॒शा मा व॒शाम् कृ॒ष्णाम् कृ॒ष्णां ॅव॒शा मा । \newline
40. व॒शा मा व॒शां ॅव॒शा मा ल॑भेत लभे॒ता व॒शां ॅव॒शा मा ल॑भेत । \newline
41. आ ल॑भेत लभे॒ता ल॑भेत॒ वरु॑णं॒ ॅवरु॑णम् ॅलभे॒ता ल॑भेत॒ वरु॑णम् । \newline
42. ल॒भे॒त॒ वरु॑णं॒ ॅवरु॑णम् ॅलभेत लभेत॒ वरु॑ण मे॒वैव वरु॑णम् ॅलभेत लभेत॒ वरु॑ण मे॒व । \newline
43. वरु॑ण मे॒वैव वरु॑णं॒ ॅवरु॑ण मे॒व स्वेन॒ स्वेनै॒व वरु॑णं॒ ॅवरु॑ण मे॒व स्वेन॑ । \newline
44. ए॒व स्वेन॒ स्वेनै॒वैव स्वेन॑ भाग॒धेये॑न भाग॒धेये॑न॒ स्वेनै॒वैव स्वेन॑ भाग॒धेये॑न । \newline
45. स्वेन॑ भाग॒धेये॑न भाग॒धेये॑न॒ स्वेन॒ स्वेन॑ भाग॒धेये॒नो पोप॑ भाग॒धेये॑न॒ स्वेन॒ स्वेन॑ भाग॒धेये॒नोप॑ । \newline
46. भा॒ग॒धेये॒नो पोप॑ भाग॒धेये॑न भाग॒धेये॒नोप॑ धावति धाव॒त्युप॑ भाग॒धेये॑न भाग॒धेये॒नोप॑ धावति । \newline
47. भा॒ग॒धेये॒नेति॑ भाग - धेये॑न । \newline
48. उप॑ धावति धाव॒ त्युपोप॑ धावति॒ स स धा॑व॒ त्युपोप॑ धावति॒ सः । \newline
49. धा॒व॒ति॒ स स धा॑वति धावति॒ स ए॒वैव स धा॑वति धावति॒ स ए॒व । \newline
50. स ए॒वैव स स ए॒वास्मा॑ अस्मा ए॒व स स ए॒वास्मै᳚ । \newline
51. ए॒वास्मा॑ अस्मा ए॒वैवास्मा॒ अन्न॒ मन्न॑ मस्मा ए॒वैवास्मा॒ अन्न᳚म् । \newline
52. अ॒स्मा॒ अन्न॒ मन्न॑ मस्मा अस्मा॒ अन्न॒म् प्र प्रान्न॑ मस्मा अस्मा॒ अन्न॒म् प्र । \newline
53. अन्न॒म् प्र प्रान्न॒ मन्न॒म् प्र य॑च्छति यच्छति॒ प्रान्न॒ मन्न॒म् प्र य॑च्छति । \newline
54. प्र य॑च्छति यच्छति॒ प्र प्र य॑च्छ त्यन्ना॒दो᳚ ऽन्ना॒दो य॑च्छति॒ प्र प्र य॑च्छ त्यन्ना॒दः । \newline
55. य॒च्छ॒ त्य॒न्ना॒दो᳚ ऽन्ना॒दो य॑च्छति यच्छ त्यन्ना॒द ए॒वैवान्ना॒दो य॑च्छति यच्छत् यन्ना॒द ए॒व । \newline
56. अ॒न्ना॒द ए॒वैवा न्ना॒दो᳚ ऽन्ना॒द ए॒व भ॑वति भवत्ये॒वा न्ना॒दो᳚ ऽन्ना॒द ए॒व भ॑वति । \newline
57. अ॒न्ना॒द इत्य॑न्न - अ॒दः । \newline
\pagebreak
\markright{ TS 2.1.9.2  \hfill https://www.vedavms.in \hfill}

\section{ TS 2.1.9.2 }

\textbf{TS 2.1.9.2 } \newline
\textbf{Samhita Paata} \newline

ए॒व भ॑वति कृ॒ष्णा भ॑वति वारु॒णी ह्ये॑षा दे॒वत॑या॒ समृ॑द्ध्यै मै॒त्रꣳ श्वे॒तमा ल॑भेत वारु॒णं कृ॒ष्णम॒पां चौष॑धीनां च स॒धांवन्न॑कामो मै॒त्रीर्वा ओष॑धयो वारु॒णीरापो॒ऽपां च॒ खलु॒ वा ओष॑धीनां च॒ रस॒मुप॑ जीवामो मि॒त्रावरु॑णावे॒व स्वेन॑ भाग॒धेये॒नोप॑ धावति॒ तावे॒वास्मा॒ अन्नं॒ प्रय॑च्छतोऽन्ना॒द ए॒व भ॑व - [  ] \newline

\textbf{Pada Paata} \newline

ए॒व । भ॒व॒ति॒ । कृ॒ष्णा । भ॒व॒ति॒ । वा॒रु॒णी । हि । ए॒षा । दे॒वत॑या । समृ॑द्ध्या॒ इति॒ सं - ऋ॒द्ध्यै॒ । मै॒त्रम् । श्वे॒तम् । एति॑ । ल॒भे॒त॒ । वा॒रु॒णम् । कृ॒ष्णम् । अ॒पाम् । च॒ । ओष॑धीनाम् । च॒ । स॒धांविति॑ सं - धौ । अन्न॑काम॒ इत्यन्न॑ - का॒मः॒ । मै॒त्रीः । वै । ओष॑धयः । वा॒रु॒णीः । आपः॑ । अ॒पाम् । च॒ । खलु॑ । वै । ओष॑धीनाम् । च॒ । रस᳚म् । उपेति॑ । जी॒वा॒मः॒ । मि॒त्रावरु॑णा॒विति॑ मि॒त्रा - वरु॑णौ । ए॒व । स्वेन॑ । भा॒ग॒धेये॒नेति॑ भाग - धेये॑न । उपेति॑ । धा॒व॒ति॒ । तौ । ए॒व । अ॒स्मै॒ । अन्न᳚म् । प्रेति॑ । य॒च्छ॒तः॒ । अ॒न्ना॒द इत्य॑न्न - अ॒दः । ए॒व । भ॒व॒ति॒ ।  \newline


\textbf{Krama Paata} \newline

ए॒व भ॑वति । भ॒व॒ति॒ कृ॒ष्णा । कृ॒ष्णा भ॑वति । भ॒व॒ति॒ वा॒रु॒णी । वा॒रु॒णी हि । ह्ये॑षा । ए॒षा दे॒वत॑या । दे॒वत॑या॒ समृ॑द्ध्यै । समृ॑द्ध्यै मै॒त्रम् । समृ॑द्ध्या॒ इति॒ सं - ऋ॒द्ध्यै॒ । मै॒त्रꣳ श्वे॒तम् । श्वे॒तमा । आ ल॑भेत । ल॒भे॒त॒ वा॒रु॒णम् । वा॒रु॒णम् कृ॒ष्णम् । कृ॒ष्णम॒पाम् । अ॒पाम् च॑ । चौष॑धीनाम् । ओष॑धीनाम् च । च॒ स॒न्धौ । स॒न्धावन्न॑कामः । स॒न्धाविति॑ सं - धौ । अन्न॑कामो मै॒त्रीः । अन्न॑काम॒ इत्यन्न॑ - का॒मः॒ । मै॒त्रीर् वै । वा ओष॑धयः । ओष॑धयो वारु॒णीः । वा॒रु॒णीरापः॑ । आपो॒ऽपाम् । अ॒पाम् च॑ । च॒ खलु॑ । खलु॒ वै । वा ओष॑धीनाम् । ओष॑धीनाम् च । च॒ रस᳚म् । रस॒मुप॑ । उप॑ जीवामः । जी॒वा॒मो॒ मि॒त्रावरु॑णौ । मि॒त्रावरु॑णावे॒व । मि॒त्राव॑रुणा॒विति॑ मि॒त्रा - वरु॑णौ । ए॒व स्वेन॑ । स्वेन॑ भाग॒धेये॑न । भा॒ग॒धेये॒नोप॑ । भा॒ग॒धेये॒नेति॑ भाग - धेये॑न । उप॑ धावति । धा॒व॒ति॒ तौ । तावे॒व । ए॒वास्मै᳚ । अ॒स्मा॒ अन्न᳚म् । अन्न॒म् प्र । प्र य॑च्छतः । य॒च्छ॒तो॒ऽन्ना॒दः । अ॒न्ना॒द ए॒व । अ॒न्ना॒द इत्य॑न्न - अ॒दः । ए॒व भ॑वति । भ॒व॒त्य॒पाम् \newline

\textbf{Jatai Paata} \newline

1. ए॒व भ॑वति भव त्ये॒वैव भ॑वति । \newline
2. भ॒व॒ति॒ कृ॒ष्णा कृ॒ष्णा भ॑वति भवति कृ॒ष्णा । \newline
3. कृ॒ष्णा भ॑वति भवति कृ॒ष्णा कृ॒ष्णा भ॑वति । \newline
4. भ॒व॒ति॒ वा॒रु॒णी वा॑रु॒णी भ॑वति भवति वारु॒णी । \newline
5. वा॒रु॒णी हि हि वा॑रु॒णी वा॑रु॒णी हि । \newline
6. ह्ये॑षैषा हि ह्ये॑षा । \newline
7. ए॒षा दे॒वत॑या दे॒वत॑ यै॒षैषा दे॒वत॑या । \newline
8. दे॒वत॑या॒ समृ॑द्ध्यै॒ समृ॑द्ध्यै दे॒वत॑या दे॒वत॑या॒ समृ॑द्ध्यै । \newline
9. समृ॑द्ध्यै मै॒त्रम् मै॒त्रꣳ समृ॑द्ध्यै॒ समृ॑द्ध्यै मै॒त्रम् । \newline
10. समृ॑द्ध्या॒ इति॒ सं - ऋ॒द्ध्यै॒ । \newline
11. मै॒त्रꣳ श्वे॒तꣳ श्वे॒तम् मै॒त्रम् मै॒त्रꣳ श्वे॒तम् । \newline
12. श्वे॒त मा श्वे॒तꣳ श्वे॒त मा । \newline
13. आ ल॑भेत लभे॒ता ल॑भेत । \newline
14. ल॒भे॒त॒ वा॒रु॒णं ॅवा॑रु॒णम् ॅल॑भेत लभेत वारु॒णम् । \newline
15. वा॒रु॒णम् कृ॒ष्णम् कृ॒ष्णं ॅवा॑रु॒णं ॅवा॑रु॒णम् कृ॒ष्णम् । \newline
16. कृ॒ष्ण म॒पा म॒पाम् कृ॒ष्णम् कृ॒ष्ण म॒पाम् । \newline
17. अ॒पाम् च॑ चा॒पा म॒पाम् च॑ । \newline
18. चौष॑धीना॒ मोष॑धीनाम् च॒ चौष॑धीनाम् । \newline
19. ओष॑धीनाम् च॒ चौष॑धीना॒ मोष॑धीनाम् च । \newline
20. च॒ स॒न्धौ स॒न्धौ च॑ च स॒न्धौ । \newline
21. स॒न्धा वन्न॑का॒मो ऽन्न॑कामः स॒न्धौ स॒न्धा वन्न॑कामः । \newline
22. स॒न्धाविति॑ सं - धौ । \newline
23. अन्न॑कामो मै॒त्रीर् मै॒त्री रन्न॑का॒मो ऽन्न॑कामो मै॒त्रीः । \newline
24. अन्न॑काम॒इत्यन्न॑ - का॒मः॒ । \newline
25. मै॒त्रीर् वै वै मै॒त्रीर् मै॒त्रीर् वै । \newline
26. वा ओष॑धय॒ ओष॑धयो॒ वै वा ओष॑धयः । \newline
27. ओष॑धयो वारु॒णीर् वा॑रु॒णी रोष॑धय॒ ओष॑धयो वारु॒णीः । \newline
28. वा॒रु॒णी राप॒ आपो॑ वारु॒णीर् वा॑रु॒णी रापः॑ । \newline
29. आपो॒ ऽपा म॒पा माप॒ आपो॒ ऽपाम् । \newline
30. अ॒पाम् च॑ चा॒पा म॒पाम् च॑ । \newline
31. च॒ खलु॒ खलु॑ च च॒ खलु॑ । \newline
32. खलु॒ वै वै खलु॒ खलु॒ वै । \newline
33. वा ओष॑धीना॒ मोष॑धीनां॒ ॅवै वा ओष॑धीनाम् । \newline
34. ओष॑धीनाम् च॒ चौष॑धीना॒ मोष॑धीनाम् च । \newline
35. च॒ रसꣳ॒॒ रस॑म् च च॒ रस᳚म् । \newline
36. रस॒ मुपोप॒ रसꣳ॒॒ रस॒ मुप॑ । \newline
37. उप॑ जीवामो जीवाम॒ उपोप॑ जीवामः । \newline
38. जी॒वा॒मो॒ मि॒त्रावरु॑णौ मि॒त्रावरु॑णौ जीवामो जीवामो मि॒त्रावरु॑णौ । \newline
39. मि॒त्रावरु॑णा वे॒वैव मि॒त्रावरु॑णौ मि॒त्रावरु॑णा वे॒व । \newline
40. मि॒त्रावरु॑णा॒विति॑ मि॒त्रा - वरु॑णौ । \newline
41. ए॒व स्वेन॒ स्वेनै॒वैव स्वेन॑ । \newline
42. स्वेन॑ भाग॒धेये॑न भाग॒धेये॑न॒ स्वेन॒ स्वेन॑ भाग॒धेये॑न । \newline
43. भा॒ग॒धेये॒नोपोप॑ भाग॒धेये॑न भाग॒धेये॒नोप॑ । \newline
44. भा॒ग॒धेये॒नेति॑ भाग - धेये॑न । \newline
45. उप॑ धावति धाव॒ त्युपोप॑ धावति । \newline
46. धा॒व॒ति॒ तौ तौ धा॑वति धावति॒ तौ । \newline
47. ता वे॒वैव तौ ता वे॒व । \newline
48. ए॒वास्मा॑ अस्मा ए॒वैवास्मै᳚ । \newline
49. अ॒स्मा॒ अन्न॒ मन्न॑ मस्मा अस्मा॒ अन्न᳚म् । \newline
50. अन्न॒म् प्र प्रान्न॒ मन्न॒म् प्र । \newline
51. प्र य॑च्छतो यच्छतः॒ प्र प्र य॑च्छतः । \newline
52. य॒च्छ॒तो॒ ऽन्ना॒दो᳚ ऽन्ना॒दो य॑च्छतो यच्छतो ऽन्ना॒दः । \newline
53. अ॒न्ना॒द ए॒वैवा न्ना॒दो᳚ ऽन्ना॒द ए॒व । \newline
54. अ॒न्ना॒द इत्य॑न्न - अ॒दः । \newline
55. ए॒व भ॑वति भव त्ये॒वैव भ॑वति । \newline
56. भ॒व॒ त्य॒पा म॒पाम् भ॑वति भव त्य॒पाम् । \newline

\textbf{Ghana Paata } \newline

1. ए॒व भ॑वति भवत्ये॒वैव भ॑वति कृ॒ष्णा कृ॒ष्णा भ॑वत्ये॒वैव भ॑वति कृ॒ष्णा । \newline
2. भ॒व॒ति॒ कृ॒ष्णा कृ॒ष्णा भ॑वति भवति कृ॒ष्णा भ॑वति भवति कृ॒ष्णा भ॑वति भवति कृ॒ष्णा भ॑वति । \newline
3. कृ॒ष्णा भ॑वति भवति कृ॒ष्णा कृ॒ष्णा भ॑वति वारु॒णी वा॑रु॒णी भ॑वति कृ॒ष्णा कृ॒ष्णा भ॑वति वारु॒णी । \newline
4. भ॒व॒ति॒ वा॒रु॒णी वा॑रु॒णी भ॑वति भवति वारु॒णी हि हि वा॑रु॒णी भ॑वति भवति वारु॒णी हि । \newline
5. वा॒रु॒णी हि हि वा॑रु॒णी वा॑रु॒णी ह्ये॑षैषा हि वा॑रु॒णी वा॑रु॒णी ह्ये॑षा । \newline
6. ह्ये॑षैषा हि ह्ये॑षा दे॒वत॑या दे॒वत॑यै॒षा हि ह्ये॑षा दे॒वत॑या । \newline
7. ए॒षा दे॒वत॑या दे॒वत॑यै॒षैषा दे॒वत॑या॒ समृ॑द्ध्यै॒ समृ॑द्ध्यै दे॒वत॑यै॒षैषा दे॒वत॑या॒ समृ॑द्ध्यै । \newline
8. दे॒वत॑या॒ समृ॑द्ध्यै॒ समृ॑द्ध्यै दे॒वत॑या दे॒वत॑या॒ समृ॑द्ध्यै मै॒त्रम् मै॒त्रꣳ समृ॑द्ध्यै दे॒वत॑या दे॒वत॑या॒ समृ॑द्ध्यै मै॒त्रम् । \newline
9. समृ॑द्ध्यै मै॒त्रम् मै॒त्रꣳ समृ॑द्ध्यै॒ समृ॑द्ध्यै मै॒त्रꣳ श्वे॒तꣳ श्वे॒तम् मै॒त्रꣳ समृ॑द्ध्यै॒ समृ॑द्ध्यै मै॒त्रꣳ श्वे॒तम् । \newline
10. समृ॑द्ध्या॒ इति॒ सं - ऋ॒द्ध्यै॒ । \newline
11. मै॒त्रꣳ श्वे॒तꣳ श्वे॒तम् मै॒त्रम् मै॒त्रꣳ श्वे॒त मा श्वे॒तम् मै॒त्रम् मै॒त्रꣳ श्वे॒त मा । \newline
12. श्वे॒त मा श्वे॒तꣳ श्वे॒त मा ल॑भेत लभे॒ता श्वे॒तꣳ श्वे॒त मा ल॑भेत । \newline
13. आ ल॑भेत लभे॒ता ल॑भेत वारु॒णं ॅवा॑रु॒णम् ॅल॑भे॒ता ल॑भेत वारु॒णम् । \newline
14. ल॒भे॒त॒ वा॒रु॒णं ॅवा॑रु॒णम् ॅल॑भेत लभेत वारु॒णम् कृ॒ष्णम् कृ॒ष्णं ॅवा॑रु॒णम् ॅल॑भेत लभेत वारु॒णम् कृ॒ष्णम् । \newline
15. वा॒रु॒णम् कृ॒ष्णम् कृ॒ष्णं ॅवा॑रु॒णं ॅवा॑रु॒णम् कृ॒ष्ण म॒पा म॒पाम् कृ॒ष्णं ॅवा॑रु॒णं ॅवा॑रु॒णम् कृ॒ष्ण म॒पाम् । \newline
16. कृ॒ष्ण म॒पा म॒पाम् कृ॒ष्णम् कृ॒ष्ण म॒पाम् च॑ चा॒पाम् कृ॒ष्णम् कृ॒ष्ण म॒पाम् च॑ । \newline
17. अ॒पाम् च॑ चा॒पा म॒पाम् चौष॑धीना॒ मोष॑धीनाम् चा॒पा म॒पाम् चौष॑धीनाम् । \newline
18. चौष॑धीना॒ मोष॑धीनाम् च॒ चौष॑धीनाम् च॒ चौष॑धीनाम् च॒ चौष॑धीनाम् च । \newline
19. ओष॑धीनाम् च॒ चौष॑धीना॒ मोष॑धीनाम् च स॒न्धौ स॒न्धौ चौष॑धीना॒ मोष॑धीनाम् च स॒न्धौ । \newline
20. च॒ स॒न्धौ स॒न्धौ च॑ च स॒न्धा वन्न॑का॒मो ऽन्न॑कामः स॒न्धौ च॑ च स॒न्धा वन्न॑कामः । \newline
21. स॒न्धा वन्न॑का॒मो ऽन्न॑कामः स॒न्धौ स॒न्धा वन्न॑कामो मै॒त्रीर् मै॒त्री रन्न॑कामः स॒न्धौ स॒न्धा वन्न॑कामो मै॒त्रीः । \newline
22. स॒न्धाविति॑ सं - धौ । \newline
23. अन्न॑कामो मै॒त्रीर् मै॒त्री रन्न॑का॒मो ऽन्न॑कामो मै॒त्रीर् वै वै मै॒त्री रन्न॑का॒मो ऽन्न॑कामो मै॒त्रीर् वै । \newline
24. अन्न॑काम॒इत्यन्न॑ - का॒मः॒ । \newline
25. मै॒त्रीर् वै वै मै॒त्रीर् मै॒त्रीर् वा ओष॑धय॒ ओष॑धयो॒ वै मै॒त्रीर् मै॒त्रीर् वा ओष॑धयः । \newline
26. वा ओष॑धय॒ ओष॑धयो॒ वै वा ओष॑धयो वारु॒णीर् वा॑रु॒णी रोष॑धयो॒ वै वा ओष॑धयो वारु॒णीः । \newline
27. ओष॑धयो वारु॒णीर् वा॑रु॒णी रोष॑धय॒ ओष॑धयो वारु॒णी राप॒ आपो॑ वारु॒णी रोष॑धय॒ ओष॑धयो वारु॒णी रापः॑ । \newline
28. वा॒रु॒णी राप॒ आपो॑ वारु॒णीर् वा॑रु॒णी रापो॒ ऽपा म॒पा मापो॑ वारु॒णीर् वा॑रु॒णी रापो॒ ऽपाम् । \newline
29. आपो॒ ऽपा म॒पा माप॒ आपो॒ ऽपाम् च॑ चा॒पा माप॒ आपो॒ ऽपाम् च॑ । \newline
30. अ॒पाम् च॑ चा॒पा म॒पाम् च॒ खलु॒ खलु॑ चा॒पा म॒पाम् च॒ खलु॑ । \newline
31. च॒ खलु॒ खलु॑ च च॒ खलु॒ वै वै खलु॑ च च॒ खलु॒ वै । \newline
32. खलु॒ वै वै खलु॒ खलु॒ वा ओष॑धीना॒ मोष॑धीनां॒ ॅवै खलु॒ खलु॒ वा ओष॑धीनाम् । \newline
33. वा ओष॑धीना॒ मोष॑धीनां॒ ॅवै वा ओष॑धीनाम् च॒ चौष॑धीनां॒ ॅवै वा ओष॑धीनाम् च । \newline
34. ओष॑धीनाम् च॒ चौष॑धीना॒ मोष॑धीनाम् च॒ रसꣳ॒॒ रस॒म् चौष॑धीना॒ मोष॑धीनाम् च॒ रस᳚म् । \newline
35. च॒ रसꣳ॒॒ रस॑म् च च॒ रस॒ मुपोप॒ रस॑म् च च॒ रस॒ मुप॑ । \newline
36. रस॒ मुपोप॒ रसꣳ॒॒ रस॒ मुप॑ जीवामो जीवाम॒ उप॒ रसꣳ॒॒ रस॒ मुप॑ जीवामः । \newline
37. उप॑ जीवामो जीवाम॒ उपोप॑ जीवामो मि॒त्रावरु॑णौ मि॒त्रावरु॑णौ जीवाम॒ उपोप॑ जीवामो मि॒त्रावरु॑णौ । \newline
38. जी॒वा॒मो॒ मि॒त्रावरु॑णौ मि॒त्रावरु॑णौ जीवामो जीवामो मि॒त्रावरु॑णा वे॒वैव मि॒त्रावरु॑णौ जीवामो जीवामो मि॒त्रावरु॑णा वे॒व । \newline
39. मि॒त्रावरु॑णा वे॒वैव मि॒त्रावरु॑णौ मि॒त्रावरु॑णा वे॒व स्वेन॒ स्वेनै॒व मि॒त्रावरु॑णौ मि॒त्रावरु॑णा वे॒व स्वेन॑ । \newline
40. मि॒त्रावरु॑णा॒विति॑ मि॒त्रा - वरु॑णौ । \newline
41. ए॒व स्वेन॒ स्वेनै॒वैव स्वेन॑ भाग॒धेये॑न भाग॒धेये॑न॒ स्वेनै॒वैव स्वेन॑ भाग॒धेये॑न । \newline
42. स्वेन॑ भाग॒धेये॑न भाग॒धेये॑न॒ स्वेन॒ स्वेन॑ भाग॒धेये॒नो पोप॑ भाग॒धेये॑न॒ स्वेन॒ स्वेन॑ भाग॒धेये॒नोप॑ । \newline
43. भा॒ग॒धेये॒नो पोप॑ भाग॒धेये॑न भाग॒धेये॒नोप॑ धावति धाव॒त्युप॑ भाग॒धेये॑न भाग॒धेये॒नोप॑ धावति । \newline
44. भा॒ग॒धेये॒नेति॑ भाग - धेये॑न । \newline
45. उप॑ धावति धाव॒ त्युपोप॑ धावति॒ तौ तौ धा॑व॒ त्युपोप॑ धावति॒ तौ । \newline
46. धा॒व॒ति॒ तौ तौ धा॑वति धावति॒ ता वे॒वैव तौ धा॑वति धावति॒ ता वे॒व । \newline
47. ता वे॒वैव तौ ता वे॒वास्मा॑ अस्मा ए॒व तौ ता वे॒वास्मै᳚ । \newline
48. ए॒वास्मा॑ अस्मा ए॒वैवास्मा॒ अन्न॒ मन्न॑ मस्मा ए॒वैवास्मा॒ अन्न᳚म् । \newline
49. अ॒स्मा॒ अन्न॒ मन्न॑ मस्मा अस्मा॒ अन्न॒म् प्र प्रान्न॑ मस्मा अस्मा॒ अन्न॒म् प्र । \newline
50. अन्न॒म् प्र प्रान्न॒ मन्न॒म् प्र य॑च्छतो यच्छतः॒ प्रान्न॒ मन्न॒म् प्र य॑च्छतः । \newline
51. प्र य॑च्छतो यच्छतः॒ प्र प्र य॑च्छतो ऽन्ना॒दो᳚ ऽन्ना॒दो य॑च्छतः॒ प्र प्र य॑च्छतो ऽन्ना॒दः । \newline
52. य॒च्छ॒तो॒ ऽन्ना॒दो᳚ ऽन्ना॒दो य॑च्छतो यच्छतो ऽन्ना॒द ए॒वैवा न्ना॒दो य॑च्छतो यच्छतो ऽन्ना॒द ए॒व । \newline
53. अ॒न्ना॒द ए॒वैवा न्ना॒दो᳚ ऽन्ना॒द ए॒व भ॑वति भवत्ये॒वा न्ना॒दो᳚ ऽन्ना॒द ए॒व भ॑वति । \newline
54. अ॒न्ना॒द इत्य॑न्न - अ॒दः । \newline
55. ए॒व भ॑वति भव त्ये॒वैव भ॑वत्य॒पा म॒पाम् भ॑व त्ये॒वैव भ॑वत्य॒पाम् । \newline
56. भ॒व॒त्य॒पा म॒पाम् भ॑वति भवत्य॒पाम् च॑ चा॒पाम् भ॑वति भवत्य॒पाम् च॑ । \newline
\pagebreak
\markright{ TS 2.1.9.3  \hfill https://www.vedavms.in \hfill}

\section{ TS 2.1.9.3 }

\textbf{TS 2.1.9.3 } \newline
\textbf{Samhita Paata} \newline

-त्य॒पां चौष॑धीनां च स॒धांवा ल॑भत उ॒भय॒स्या-व॑रुद्ध्यै॒विशा॑खो॒ यूपो॑ भवति॒ द्वे ह्ये॑ते दे॒वते॒ समृ॑द्ध्यै मै॒त्रꣳ श्वे॒तमा ल॑भेत वारु॒णं कृ॒ष्णं ज्योगा॑मयावी॒यन् मै॒त्रो भव॑ति मि॒त्रेणै॒वास्मै॒ वरु॑णꣳ शमयति॒ यद्-वा॑रु॒णः सा॒क्षादे॒वैनं॑ ॅवरुणपा॒शान् मु॑ञ्चत्यु॒त यदी॒तासु॒र्भव॑ति॒ जीव॑त्ये॒व दे॒वा वै पुष्टिं॒ नावि॑न्द॒न् - [  ] \newline

\textbf{Pada Paata} \newline

अ॒पाम् । च॒ । ओष॑धीनाम् । च॒ । स॒धांविति॑ सं-धौ । एति॑ । ल॒भ॒ते॒ । उ॒भय॑स्य । अव॑रुद्ध्या॒ इत्यव॑ - रु॒ध्यै॒ । विशा॑ख॒ इति॒ वि-शा॒खः॒ । यूपः॑ । भ॒व॒ति॒ । द्वे इति॑ । हि । ए॒ते इति॑ । दे॒वते॒ इति॑ । समृ॑द्ध्या॒ इति॒ सं - ऋ॒द्ध्यै॒ । मै॒त्रम् । श्वे॒तम् । एति॑ । ल॒भे॒त॒ । वा॒रु॒णम् । कृ॒ष्णम् । ज्योगा॑मया॒वीति॒ ज्योक् - आ॒म॒या॒वी॒ । यत् । मै॒त्रः । भव॑ति । मि॒त्रेण॑ । ए॒व । अ॒स्मै॒ । वरु॑णम् । श॒म॒य॒ति॒ । यत् । वा॒रु॒णः । सा॒क्षादिति॑ स - अ॒क्षात् । ए॒व । ए॒न॒म् । व॒रु॒ण॒पा॒शादिति॑ वरुण - पा॒शात् । मु॒ञ्च॒ति॒ । उ॒त । यदि॑ । इ॒तासु॒रिती॒त-अ॒सुः॒ । भव॑ति । जीव॑ति । ए॒व । दे॒वाः । वै । पुष्टि᳚म् । न । अ॒वि॒न्द॒न्न् ।  \newline


\textbf{Krama Paata} \newline

अ॒पाम् च॑ । चौष॑धीनाम् । ओष॑धीनाम् च । च॒ स॒न्धौ । स॒न्धावा । स॒न्धाविति॑ सं - धौ । आ ल॑भते । ल॒भ॒त॒ उ॒भय॑स्य । उ॒भय॒स्याव॑रुद्ध्यै । अव॑रुद्ध्यै॒ विशा॑खः । अव॑रुद्धा॒ इत्यव॑ - रु॒द्धै॒ । विशा॑खो॒ यूपः॑ । विशा॑ख॒ इति॒ वि - शा॒खः॒ । यूपो॑ भवति । भ॒व॒ति॒ द्वे । द्वे हि । द्वे इति॒ द्वे । ह्ये॑ते । ए॒ते दे॒वते᳚ । ए॒ते इत्ये॒ते । दे॒वते॒ समृ॑द्ध्यै । दे॒वते॒ इति॑ दे॒वते᳚ । समृ॑द्ध्यै मै॒त्रम् । समृ॑द्ध्या॒ इति॒ सं - ऋ॒द्ध्यै॒ । मै॒त्रꣳ श्वे॒तम् । श्वे॒तमा । आ ल॑भेत । ल॒भे॒त॒ वा॒रु॒णम् । वा॒रु॒णम् कृ॒ष्णम् । कृ॒ष्णम् ज्योगा॑मयावी । ज्योगा॑मयावी॒ यत् । ज्योगा॑मया॒वीति॒ ज्योक् - आ॒म॒या॒वी॒ । यन्मै॒त्रः । मै॒त्रो भव॑ति । भव॑ति मि॒त्रेण॑ । मि॒त्रेणै॒व । ए॒वास्मै᳚ । अ॒स्मै॒ वरु॑णम् । वरु॑णꣳ शमयति । श॒म॒य॒ति॒ यत् । यद् वा॑रु॒णः । वा॒रु॒णः सा॒क्षात् । सा॒क्षादे॒व । सा॒क्षादिति॑ स - अ॒क्षात् । ए॒वैन᳚म् । ए॒नं॒ ॅव॒रु॒ण॒पा॒शात् । व॒रु॒ण॒पा॒शान् मु॑ञ्चति । व॒रु॒ण॒पा॒शादिति॑ वरुण - पा॒शात् । मु॒ञ्च॒त्यु॒त । उ॒त यदि॑ । यदी॒तासुः॑ । इ॒तासु॒र् भव॑ति । इ॒तासु॒रिती॒त - अ॒सुः॒ । भव॑ति॒ जीव॑ति । जीव॑त्ये॒व । ए॒व दे॒वाः । दे॒वा वै । वै पुष्टि᳚म् । पुष्टि॒म् न । नावि॑न्दन्न् ( ) । अ॒वि॒न्द॒न् ताम् \newline

\textbf{Jatai Paata} \newline

1. अ॒पाम् च॑ चा॒पा म॒पाम् च॑ । \newline
2. चौष॑धीना॒ मोष॑धीनाम् च॒ चौष॑धीनाम् । \newline
3. ओष॑धीनाम् च॒ चौष॑धीना॒ मोष॑धीनाम् च । \newline
4. च॒ स॒न्धौ स॒न्धौ च॑ च स॒न्धौ । \newline
5. स॒न्धा वा स॒न्धौ स॒न्धा वा । \newline
6. स॒न्धाविति॑ सं - धौ । \newline
7. आ ल॑भते लभत॒ आ ल॑भते । \newline
8. ल॒भ॒त॒ उ॒भय॑स्यो॒ भय॑स्य लभते लभत उ॒भय॑स्य । \newline
9. उ॒भय॒स्या व॑रुद्ध्या॒ अव॑रुद्ध्या उ॒भय॑स्यो॒ भय॒स्या व॑रुद्ध्यै । \newline
10. अव॑रुद्ध्यै॒ विशा॑खो॒ विशा॒खो ऽव॑रुद्ध्या॒ अव॑रुद्ध्यै॒ विशा॑खः । \newline
11. अव॑रुद्ध्या॒इत्यव॑ - रु॒द्ध्यै॒ । \newline
12. विशा॑खो॒ यूपो॒ यूपो॒ विशा॑खो॒ विशा॑खो॒ यूपः॑ । \newline
13. विशा॑ख॒ इति॒ वि - शा॒खः॒ । \newline
14. यूपो॑ भवति भवति॒ यूपो॒ यूपो॑ भवति । \newline
15. भ॒व॒ति॒ द्वे द्वे भ॑वति भवति॒ द्वे । \newline
16. द्वे हि हि द्वे द्वे हि । \newline
17. द्वे इति॒ द्वे । \newline
18. ह्ये॑ते ए॒ते हि ह्ये॑ते । \newline
19. ए॒ते दे॒वते॑ दे॒वते॑ ए॒ते ए॒ते दे॒वते᳚ । \newline
20. ए॒ते इत्ये॒ते । \newline
21. दे॒वते॒ समृ॑द्ध्यै॒ समृ॑द्ध्यै दे॒वते॑ दे॒वते॒ समृ॑द्ध्यै । \newline
22. दे॒वते॒ इति॑ दे॒वते᳚ । \newline
23. समृ॑द्ध्यै मै॒त्रम् मै॒त्रꣳ समृ॑द्ध्यै॒ समृ॑द्ध्यै मै॒त्रम् । \newline
24. समृ॑द्ध्या॒ इति॒ सं - ऋ॒द्ध्यै॒ । \newline
25. मै॒त्रꣳ श्वे॒तꣳ श्वे॒तम् मै॒त्रम् मै॒त्रꣳ श्वे॒तम् । \newline
26. श्वे॒त मा श्वे॒तꣳ श्वे॒त मा । \newline
27. आ ल॑भेत लभे॒ता ल॑भेत । \newline
28. ल॒भे॒त॒ वा॒रु॒णं ॅवा॑रु॒णम् ॅल॑भेत लभेत वारु॒णम् । \newline
29. वा॒रु॒णम् कृ॒ष्णम् कृ॒ष्णं ॅवा॑रु॒णं ॅवा॑रु॒णम् कृ॒ष्णम् । \newline
30. कृ॒ष्णम् ज्योगा॑मयावी॒ ज्योगा॑मयावी कृ॒ष्णम् कृ॒ष्णम् ज्योगा॑मयावी । \newline
31. ज्योगा॑मयावी॒ यद् यज् ज्योगा॑मयावी॒ ज्योगा॑मयावी॒ यत् । \newline
32. ज्योगा॑मया॒वीति॒ ज्योक् - आ॒म॒या॒वी॒ । \newline
33. यन् मै॒त्रो मै॒त्रो यद् यन् मै॒त्रः । \newline
34. मै॒त्रो भव॑ति॒ भव॑ति मै॒त्रो मै॒त्रो भव॑ति । \newline
35. भव॑ति मि॒त्रेण॑ मि॒त्रेण॒ भव॑ति॒ भव॑ति मि॒त्रेण॑ । \newline
36. मि॒त्रेणै॒वैव मि॒त्रेण॑ मि॒त्रेणै॒व । \newline
37. ए॒वास्मा॑ अस्मा ए॒वैवास्मै᳚ । \newline
38. अ॒स्मै॒ वरु॑णं॒ ॅवरु॑ण मस्मा अस्मै॒ वरु॑णम् । \newline
39. वरु॑णꣳ शमयति शमयति॒ वरु॑णं॒ ॅवरु॑णꣳ शमयति । \newline
40. श॒म॒य॒ति॒ यद् यच् छ॑मयति शमयति॒ यत् । \newline
41. यद् वा॑रु॒णो वा॑रु॒णो यद् यद् वा॑रु॒णः । \newline
42. वा॒रु॒णः सा॒क्षाथ् सा॒क्षाद् वा॑रु॒णो वा॑रु॒णः सा॒क्षात् । \newline
43. सा॒क्षादे॒वैव सा॒क्षाथ् सा॒क्षादे॒व । \newline
44. सा॒क्षादिति॑ स - अ॒क्षात् । \newline
45. ए॒वैन॑ मेन मे॒वैवैन᳚म् । \newline
46. ए॒नं॒ ॅव॒रु॒ण॒पा॒शाद् व॑रुणपा॒शा दे॑न मेनं ॅवरुणपा॒शात् । \newline
47. व॒रु॒ण॒पा॒शान् मु॑ञ्चति मुञ्चति वरुणपा॒शाद् व॑रुणपा॒शान् मु॑ञ्चति । \newline
48. व॒रु॒ण॒पा॒शादिति॑ वरुण - पा॒शात् । \newline
49. मु॒ञ्च॒ त्यु॒तोत मु॑ञ्चति मुञ्च त्यु॒त । \newline
50. उ॒त यदि॒ यद्यु॒तोत यदि॑ । \newline
51. यदी॒तासु॑ रि॒तासु॒र् यदि॒ यदी॒तासुः॑ । \newline
52. इ॒तासु॒र् भव॑ति॒ भव॑ती॒तासु॑ रि॒तासु॒र् भव॑ति । \newline
53. इ॒तासु॒रिती॒त - अ॒सुः॒ । \newline
54. भव॑ति॒ जीव॑ति॒ जीव॑ति॒ भव॑ति॒ भव॑ति॒ जीव॑ति । \newline
55. जीव॑ त्ये॒वैव जीव॑ति॒ जीव॑ त्ये॒व । \newline
56. ए॒व दे॒वा दे॒वा ए॒वैव दे॒वाः । \newline
57. दे॒वा वै वै दे॒वा दे॒वा वै । \newline
58. वै पुष्टि॒म् पुष्टिं॒ ॅवै वै पुष्टि᳚म् । \newline
59. पुष्टि॒म् न न पुष्टि॒म् पुष्टि॒म् न । \newline
60. नावि॑न्दन् नविन्द॒न् न नावि॑न्दन्न् । \newline
61. अ॒वि॒न्द॒न् ताम् ता म॑विन्दन् नविन्द॒न् ताम् । \newline

\textbf{Ghana Paata } \newline

1. अ॒पाम् च॑ चा॒पा म॒पाम् चौष॑धीना॒ मोष॑धीनाम् चा॒पा म॒पाम् चौष॑धीनाम् । \newline
2. चौष॑धीना॒ मोष॑धीनाम् च॒ चौष॑धीनाम् च॒ चौष॑धीनाम् च॒ चौष॑धीनाम् च । \newline
3. ओष॑धीनाम् च॒ चौष॑धीना॒ मोष॑धीनाम् च स॒न्धौ स॒न्धौ चौष॑धीना॒ मोष॑धीनाम् च स॒न्धौ । \newline
4. च॒ स॒न्धौ स॒न्धौ च॑ च स॒न्धा वा स॒न्धौ च॑ च स॒न्धा वा । \newline
5. स॒न्धा वा स॒न्धौ स॒न्धा वा ल॑भते लभत॒ आ स॒न्धौ स॒न्धा वा ल॑भते । \newline
6. स॒न्धाविति॑ सं - धौ । \newline
7. आ ल॑भते लभत॒ आ ल॑भत उ॒भय॑स्यो॒ भय॑स्य लभत॒ आ ल॑भत उ॒भय॑स्य । \newline
8. ल॒भ॒त॒ उ॒भय॑स्यो॒ भय॑स्य लभते लभत उ॒भय॒स्या व॑रुद्ध्या॒ अव॑रुद्ध्या उ॒भय॑स्य लभते लभत उ॒भय॒स्या व॑रुद्ध्यै । \newline
9. उ॒भय॒स्या व॑रुद्ध्या॒ अव॑रुद्ध्या उ॒भय॑स्यो॒ भय॒स्या व॑रुद्ध्यै॒ विशा॑खो॒ विशा॒खो ऽव॑रुद्ध्या उ॒भय॑स्यो॒ भय॒स्या व॑रुद्ध्यै॒ विशा॑खः । \newline
10. अव॑रुद्ध्यै॒ विशा॑खो॒ विशा॒खो ऽव॑रुद्ध्या॒ अव॑रुद्ध्यै॒ विशा॑खो॒ यूपो॒ यूपो॒ विशा॒खो ऽव॑रुद्ध्या॒ अव॑रुद्ध्यै॒ विशा॑खो॒ यूपः॑ । \newline
11. अव॑रुद्ध्या॒इत्यव॑ - रु॒द्ध्यै॒ । \newline
12. विशा॑खो॒ यूपो॒ यूपो॒ विशा॑खो॒ विशा॑खो॒ यूपो॑ भवति भवति॒ यूपो॒ विशा॑खो॒ विशा॑खो॒ यूपो॑ भवति । \newline
13. विशा॑ख॒ इति॒ वि - शा॒खः॒ । \newline
14. यूपो॑ भवति भवति॒ यूपो॒ यूपो॑ भवति॒ द्वे द्वे भ॑वति॒ यूपो॒ यूपो॑ भवति॒ द्वे । \newline
15. भ॒व॒ति॒ द्वे द्वे भ॑वति भवति॒ द्वे हि हि द्वे भ॑वति भवति॒ द्वे हि । \newline
16. द्वे हि हि द्वे द्वे ह्ये॑ते ए॒ते हि द्वे द्वे ह्ये॑ते । \newline
17. द्वे इति॒ द्वे । \newline
18. ह्ये॑ते ए॒ते हि ह्ये॑ते दे॒वते॑ दे॒वते॑ ए॒ते हि ह्ये॑ते दे॒वते᳚ । \newline
19. ए॒ते दे॒वते॑ दे॒वते॑ ए॒ते ए॒ते दे॒वते॒ समृ॑द्ध्यै॒ समृ॑द्ध्यै दे॒वते॑ ए॒ते ए॒ते दे॒वते॒ समृ॑द्ध्यै । \newline
20. ए॒ते इत्ये॒ते । \newline
21. दे॒वते॒ समृ॑द्ध्यै॒ समृ॑द्ध्यै दे॒वते॑ दे॒वते॒ समृ॑द्ध्यै मै॒त्रम् मै॒त्रꣳ समृ॑द्ध्यै दे॒वते॑ दे॒वते॒ समृ॑द्ध्यै मै॒त्रम् । \newline
22. दे॒वते॒ इति॑ दे॒वते᳚ । \newline
23. समृ॑द्ध्यै मै॒त्रम् मै॒त्रꣳ समृ॑द्ध्यै॒ समृ॑द्ध्यै मै॒त्रꣳ श्वे॒तꣳ श्वे॒तम् मै॒त्रꣳ समृ॑द्ध्यै॒ समृ॑द्ध्यै मै॒त्रꣳ श्वे॒तम् । \newline
24. समृ॑द्ध्या॒ इति॒ सं - ऋ॒द्ध्यै॒ । \newline
25. मै॒त्रꣳ श्वे॒तꣳ श्वे॒तम् मै॒त्रम् मै॒त्रꣳ श्वे॒त मा श्वे॒तम् मै॒त्रम् मै॒त्रꣳ श्वे॒त मा । \newline
26. श्वे॒त मा श्वे॒तꣳ श्वे॒त मा ल॑भेत लभे॒ता श्वे॒तꣳ श्वे॒त मा ल॑भेत । \newline
27. आ ल॑भेत लभे॒ता ल॑भेत वारु॒णं ॅवा॑रु॒णम् ॅल॑भे॒ता ल॑भेत वारु॒णम् । \newline
28. ल॒भे॒त॒ वा॒रु॒णं ॅवा॑रु॒णम् ॅल॑भेत लभेत वारु॒णम् कृ॒ष्णम् कृ॒ष्णं ॅवा॑रु॒णम् ॅल॑भेत लभेत वारु॒णम् कृ॒ष्णम् । \newline
29. वा॒रु॒णम् कृ॒ष्णम् कृ॒ष्णं ॅवा॑रु॒णं ॅवा॑रु॒णम् कृ॒ष्णम् ज्योगा॑मयावी॒ ज्योगा॑मयावी कृ॒ष्णं ॅवा॑रु॒णं ॅवा॑रु॒णम् कृ॒ष्णम् ज्योगा॑मयावी । \newline
30. कृ॒ष्णम् ज्योगा॑मयावी॒ ज्योगा॑मयावी कृ॒ष्णम् कृ॒ष्णम् ज्योगा॑मयावी॒ यद् यज् ज्योगा॑मयावी कृ॒ष्णम् कृ॒ष्णम् ज्योगा॑मयावी॒ यत् । \newline
31. ज्योगा॑मयावी॒ यद् यज् ज्योगा॑मयावी॒ ज्योगा॑मयावी॒ यन् मै॒त्रो मै॒त्रो यज् ज्योगा॑मयावी॒ ज्योगा॑मयावी॒ यन् मै॒त्रः । \newline
32. ज्योगा॑मया॒वीति॒ ज्योक् - आ॒म॒या॒वी॒ । \newline
33. यन् मै॒त्रो मै॒त्रो यद् यन् मै॒त्रो भव॑ति॒ भव॑ति मै॒त्रो यद् यन् मै॒त्रो भव॑ति । \newline
34. मै॒त्रो भव॑ति॒ भव॑ति मै॒त्रो मै॒त्रो भव॑ति मि॒त्रेण॑ मि॒त्रेण॒ भव॑ति मै॒त्रो मै॒त्रो भव॑ति मि॒त्रेण॑ । \newline
35. भव॑ति मि॒त्रेण॑ मि॒त्रेण॒ भव॑ति॒ भव॑ति मि॒त्रेणै॒वैव मि॒त्रेण॒ भव॑ति॒ भव॑ति मि॒त्रेणै॒व । \newline
36. मि॒त्रेणै॒वैव मि॒त्रेण॑ मि॒त्रेणै॒वास्मा॑ अस्मा ए॒व मि॒त्रेण॑ मि॒त्रेणै॒वास्मै᳚ । \newline
37. ए॒वास्मा॑ अस्मा ए॒वैवास्मै॒ वरु॑णं॒ ॅवरु॑ण मस्मा ए॒वैवास्मै॒ वरु॑णम् । \newline
38. अ॒स्मै॒ वरु॑णं॒ ॅवरु॑ण मस्मा अस्मै॒ वरु॑णꣳ शमयति शमयति॒ वरु॑ण मस्मा अस्मै॒ वरु॑णꣳ शमयति । \newline
39. वरु॑णꣳ शमयति शमयति॒ वरु॑णं॒ ॅवरु॑णꣳ शमयति॒ यद् यच्छ॑मयति॒ वरु॑णं॒ ॅवरु॑णꣳ शमयति॒ यत् । \newline
40. श॒म॒य॒ति॒ यद् यच्छ॑मयति शमयति॒ यद् वा॑रु॒णो वा॑रु॒णो यच्छ॑मयति शमयति॒ यद् वा॑रु॒णः । \newline
41. यद् वा॑रु॒णो वा॑रु॒णो यद् यद् वा॑रु॒णः सा॒क्षाथ् सा॒क्षाद् वा॑रु॒णो यद् यद् वा॑रु॒णः सा॒क्षात् । \newline
42. वा॒रु॒णः सा॒क्षाथ् सा॒क्षाद् वा॑रु॒णो वा॑रु॒णः सा॒क्षादे॒वैव सा॒क्षाद् वा॑रु॒णो वा॑रु॒णः सा॒क्षादे॒व । \newline
43. सा॒क्षादे॒वैव सा॒क्षाथ् सा॒क्षादे॒वैन॑ मेन मे॒व सा॒क्षाथ् सा॒क्षादे॒वैन᳚म् । \newline
44. सा॒क्षादिति॑ स - अ॒क्षात् । \newline
45. ए॒वैन॑ मेन मे॒वैवैनं॑ ॅवरुणपा॒शाद् व॑रुणपा॒शा दे॑न मे॒वैवैनं॑ ॅवरुणपा॒शात् । \newline
46. ए॒नं॒ ॅव॒रु॒ण॒पा॒शाद् व॑रुणपा॒शा दे॑न मेनं ॅवरुणपा॒शान् मु॑ञ्चति मुञ्चति वरुणपा॒शा दे॑न मेनं ॅवरुणपा॒शान् मु॑ञ्चति । \newline
47. व॒रु॒ण॒पा॒शान् मु॑ञ्चति मुञ्चति वरुणपा॒शाद् व॑रुणपा॒शान् मु॑ञ्च त्यु॒तोत मु॑ञ्चति वरुणपा॒शाद् व॑रुणपा॒शान् मु॑ञ्चत्यु॒त । \newline
48. व॒रु॒ण॒पा॒शादिति॑ वरुण - पा॒शात् । \newline
49. मु॒ञ्च॒ त्यु॒तोत मु॑ञ्चति मुञ्चत्यु॒त यदि॒ यद्यु॒त मु॑ञ्चति मुञ्चत्यु॒त यदि॑ । \newline
50. उ॒त यदि॒ यद्यु॒तोत यदी॒तासु॑ रि॒तासु॒र् यद्यु॒तोत यदी॒तासुः॑ । \newline
51. यदी॒तासु॑ रि॒तासु॒र् यदि॒ यदी॒तासु॒र् भव॑ति॒ भव॑ती॒तासु॒र् यदि॒ यदी॒तासु॒र् भव॑ति । \newline
52. इ॒तासु॒र् भव॑ति॒ भव॑ती॒तासु॑ रि॒तासु॒र् भव॑ति॒ जीव॑ति॒ जीव॑ति॒ भव॑ती॒तासु॑ रि॒तासु॒र् भव॑ति॒ जीव॑ति । \newline
53. इ॒तासु॒रिती॒त - अ॒सुः॒ । \newline
54. भव॑ति॒ जीव॑ति॒ जीव॑ति॒ भव॑ति॒ भव॑ति॒ जीव॑त्ये॒वैव जीव॑ति॒ भव॑ति॒ भव॑ति॒ जीव॑त्ये॒व । \newline
55. जीव॑त्ये॒वैव जीव॑ति॒ जीव॑त्ये॒व दे॒वा दे॒वा ए॒व जीव॑ति॒ जीव॑त्ये॒व दे॒वाः । \newline
56. ए॒व दे॒वा दे॒वा ए॒वैव दे॒वा वै वै दे॒वा ए॒वैव दे॒वा वै । \newline
57. दे॒वा वै वै दे॒वा दे॒वा वै पुष्टि॒म् पुष्टिं॒ ॅवै दे॒वा दे॒वा वै पुष्टि᳚म् । \newline
58. वै पुष्टि॒म् पुष्टिं॒ ॅवै वै पुष्टि॒न्न न पुष्टिं॒ ॅवै वै पुष्टि॒न्न । \newline
59. पुष्टि॒न्न न पुष्टि॒म् पुष्टि॒न् नावि॑न्दन् नविन्द॒न् न पुष्टि॒म् पुष्टि॒न् नावि॑न्दन्न् । \newline
60. नावि॑न्दन् नविन्द॒न् न नावि॑न्द॒न् ताम् ता म॑विन्द॒न् न नावि॑न्द॒न् ताम् । \newline
61. अ॒वि॒न्द॒न् ताम् ता म॑विन्दन् नविन्द॒न् ताम् मि॑थु॒ने मि॑थु॒ने ता म॑विन्दन् नविन्द॒न् ताम् मि॑थु॒ने । \newline
\pagebreak
\markright{ TS 2.1.9.4  \hfill https://www.vedavms.in \hfill}

\section{ TS 2.1.9.4 }

\textbf{TS 2.1.9.4 } \newline
\textbf{Samhita Paata} \newline

तां मि॑थु॒ने॑ ऽपश्य॒न् तस्यां॒ न सम॑राधय॒न्ता-व॒श्विना॑-वब्रूता-मा॒वयो॒र्वा ए॒षा मैतस्यां᳚ ॅवदद्ध्व॒मिति॒ साश्विनो॑रे॒वाभ॑व॒द्यः पुष्टि॑कामः॒ स्याथ् स ए॒तामा᳚श्वि॒नीं ॅय॒मीं ॅव॒शामा ल॑भेता॒ऽश्विना॑वे॒व स्वेन॑ भाग॒धेये॒नोप॑ धावति॒ तावे॒वास्मि॒न् पुष्टिं॑ धत्तः॒ पुष्य॑ति प्र॒जया॑ प॒शुभिः॑ ॥ \newline

\textbf{Pada Paata} \newline

ताम् । मि॒थु॒ने । अ॒प॒श्य॒न्न् । तस्या᳚म् । न । समिति॑ । अ॒रा॒ध॒य॒न्न् । तौ । अ॒श्विनौ᳚ । अ॒ब्रू॒ता॒म् । आ॒वयोः᳚ । वै । ए॒षा । मा । ए॒तस्या᳚म् । व॒द॒द्ध्व॒म् । इति॑ । सा । अ॒श्विनोः᳚ । ए॒व । अ॒भ॒व॒त् । यः । पुष्टि॑काम॒ इति॒ पुष्टि॑ - का॒मः॒ । स्यात् । सः । ए॒ताम् । आ॒श्वि॒नीम् । य॒मीम् । व॒शाम् । एति॑ । ल॒भे॒त॒ । अ॒श्विनौ᳚ । ए॒व । स्वेन॑ । भा॒ग॒धेये॒नेति॑ भाग - धेये॑न । उपेति॑ । धा॒व॒ति॒ । तौ । ए॒व । अ॒स्मि॒न्न् । पुष्टि᳚म् । ध॒त्तः॒ । पुष्य॑ति । प्र॒जयेति॑ प्र - जया᳚ । प॒शुभि॒रिति॑ प॒शु - भिः॒ ॥  \newline


\textbf{Krama Paata} \newline

ताम् मि॑थु॒ने । मि॒थु॒ने॑ ऽपश्यन्न् । अ॒प॒श्य॒न् तस्या᳚म् । तस्या॒म् न । न सम् । सम॑राधयन्न् । अ॒रा॒ध॒य॒न् तौ । ताव॒श्विनौ᳚ । अ॒श्विना॑वब्रूताम् । अ॒ब्रू॒ता॒मा॒वयोः᳚ । आ॒वयो॒र् वै । वा ए॒षा । ए॒षा मा । मैतस्या᳚म् । ए॒तस्यां᳚ ॅवदद्ध्वम् । व॒द॒द्ध्व॒मिति॑ । इति॒ सा । सा ऽश्विनोः᳚ । अ॒श्विनो॑रे॒व । ए॒वाभ॑वत् । अ॒भ॒व॒द् यः । यः पुष्टि॑कामः । पुष्टि॑कामः॒ स्यात् । पुष्टि॑काम॒ इति॒ पुष्टि॑ - का॒मः॒ । स्याथ् सः । स ए॒ताम् । ए॒तामा᳚श्वि॒नीम् । आ॒श्वि॒नीम् ॅय॒मीम् । य॒मीं ॅव॒शाम् । व॒शामा । आ ल॑भेत । ल॒भे॒ता॒श्विनौ᳚ । अ॒श्विना॑वे॒व । ए॒व स्वेन॑ । स्वेन॑ भाग॒धेये॑न । भा॒ग॒धेये॒नोप॑ । भा॒ग॒धेये॒नेति॑ भाग - धेये॑न । उप॑ धावति । धा॒व॒ति॒ तौ । तावे॒व । ए॒वास्मिन्न्॑ । अ॒स्मि॒न् पुष्टि᳚म् । पुष्टि॑म् धत्तः । ध॒त्तः॒ पुष्य॑ति । पुष्य॑ति प्र॒जया᳚ । प्र॒जया॑ प॒शुभिः॑ । प्र॒जयेति॑ प्र - जया᳚ । प॒शुभि॒रिति॑ प॒शु - भिः॒ । \newline

\textbf{Jatai Paata} \newline

1. ताम् मि॑थु॒ने मि॑थु॒ने ताम् ताम् मि॑थु॒ने । \newline
2. मि॒थु॒ने॑ ऽपश्यन् नपश्यन् मिथु॒ने मि॑थु॒ने॑ ऽपश्यन्न् । \newline
3. अ॒प॒श्य॒न् तस्या॒म् तस्या॑ मपश्यन् नपश्य॒न् तस्या᳚म् । \newline
4. तस्या॒म् न न तस्या॒म् तस्या॒म् न । \newline
5. न सꣳ सम् न न सम् । \newline
6. स म॑राधयन् नराधय॒न् थ्सꣳ स म॑राधयन्न् । \newline
7. अ॒रा॒ध॒य॒न् तौ ता व॑राधयन् नराधय॒न् तौ । \newline
8. ता व॒श्विना॑ व॒श्विनौ॒ तौ ता व॒श्विनौ᳚ । \newline
9. अ॒श्विना॑ वब्रूता मब्रूता म॒श्विना॑ व॒श्विना॑ वब्रूताम् । \newline
10. अ॒ब्रू॒ता॒ मा॒वयो॑ रा॒वयो॑ रब्रूता मब्रूता मा॒वयोः᳚ । \newline
11. आ॒वयो॒र् वै वा आ॒वयो॑ रा॒वयो॒र् वै । \newline
12. वा ए॒षैषा वै वा ए॒षा । \newline
13. ए॒षा मा मैषैषा मा । \newline
14. मैतस्या॑ मे॒तस्या॒म् मा मैतस्या᳚म् । \newline
15. ए॒तस्यां᳚ ॅवदद्ध्वं ॅवदद्ध्व मे॒तस्या॑ मे॒तस्यां᳚ ॅवदद्ध्वम् । \newline
16. व॒द॒द्ध्व॒ मितीति॑ वदद्ध्वं ॅवदद्ध्व॒ मिति॑ । \newline
17. इति॒ सा सेतीति॒ सा । \newline
18. सा ऽश्विनो॑ र॒श्विनोः॒ सा सा ऽश्विनोः᳚ । \newline
19. अ॒श्विनो॑ रे॒वै वाश्विनो॑ र॒श्विनो॑ रे॒व । \newline
20. ए॒वाभ॑व दभव दे॒वैवा भ॑वत् । \newline
21. अ॒भ॒व॒द् यो यो॑ ऽभव दभव॒द् यः । \newline
22. यः पुष्टि॑कामः॒ पुष्टि॑कामो॒ यो यः पुष्टि॑कामः । \newline
23. पुष्टि॑कामः॒ स्याथ् स्यात् पुष्टि॑कामः॒ पुष्टि॑कामः॒ स्यात् । \newline
24. पुष्टि॑काम॒ इति॒ पुष्टि॑ - का॒मः॒ । \newline
25. स्याथ् स स स्याथ् स्याथ् सः । \newline
26. स ए॒ता मे॒ताꣳ स स ए॒ताम् । \newline
27. ए॒ता मा᳚श्वि॒नी मा᳚श्वि॒नी मे॒ता मे॒ता मा᳚श्वि॒नीम् । \newline
28. आ॒श्वि॒नीं ॅय॒मीं ॅय॒मी मा᳚श्वि॒नी मा᳚श्वि॒नीं ॅय॒मीम् । \newline
29. य॒मीं ॅव॒शां ॅव॒शां ॅय॒मीं ॅय॒मीं ॅव॒शाम् । \newline
30. व॒शा मा व॒शां ॅव॒शा मा । \newline
31. आ ल॑भेत लभे॒ता ल॑भेत । \newline
32. ल॒भे॒ता॒श्विना॑ व॒श्विनौ॑ लभेत लभेता॒श्विनौ᳚ । \newline
33. अ॒श्विना॑ वे॒वै वाश्विना॑ व॒श्विना॑ वे॒व । \newline
34. ए॒व स्वेन॒ स्वेनै॒वैव स्वेन॑ । \newline
35. स्वेन॑ भाग॒धेये॑न भाग॒धेये॑न॒ स्वेन॒ स्वेन॑ भाग॒धेये॑न । \newline
36. भा॒ग॒धेये॒नोपोप॑ भाग॒धेये॑न भाग॒धेये॒नोप॑ । \newline
37. भा॒ग॒धेये॒नेति॑ भाग - धेये॑न । \newline
38. उप॑ धावति धाव॒ त्युपोप॑ धावति । \newline
39. धा॒व॒ति॒ तौ तौ धा॑वति धावति॒ तौ । \newline
40. ता वे॒वैव तौ ता वे॒व । \newline
41. ए॒वास्मि॑न् नस्मिन् ने॒वैवास्मिन्न्॑ । \newline
42. अ॒स्मि॒न् पुष्टि॒म् पुष्टि॑ मस्मिन् नस्मि॒न् पुष्टि᳚म् । \newline
43. पुष्टि॑म् धत्तो धत्तः॒ पुष्टि॒म् पुष्टि॑म् धत्तः । \newline
44. ध॒त्तः॒ पुष्य॑ति॒ पुष्य॑ति धत्तो धत्तः॒ पुष्य॑ति । \newline
45. पुष्य॑ति प्र॒जया᳚ प्र॒जया॒ पुष्य॑ति॒ पुष्य॑ति प्र॒जया᳚ । \newline
46. प्र॒जया॑ प॒शुभिः॑ प॒शुभिः॑ प्र॒जया᳚ प्र॒जया॑ प॒शुभिः॑ । \newline
47. प्र॒जयेति॑प्र - जया᳚ । \newline
48. प॒शुभि॒रिति॑ प॒शु - भिः॒ । \newline

\textbf{Ghana Paata } \newline

1. ताम् मि॑थु॒ने मि॑थु॒ने ताम् ताम् मि॑थु॒ने॑ ऽपश्यन् नपश्यन् मिथु॒ने ताम् ताम् मि॑थु॒ने॑ ऽपश्यन्न् । \newline
2. मि॒थु॒ने॑ ऽपश्यन् नपश्यन् मिथु॒ने मि॑थु॒ने॑ ऽपश्य॒न् तस्या॒म् तस्या॑ मपश्यन् मिथु॒ने मि॑थु॒ने॑ ऽपश्य॒न् तस्या᳚म् । \newline
3. अ॒प॒श्य॒न् तस्या॒म् तस्या॑ मपश्यन् नपश्य॒न् तस्या॒न्न न तस्या॑ मपश्यन् नपश्य॒न् तस्या॒न्न । \newline
4. तस्या॒न्न न तस्या॒म् तस्या॒न्न सꣳ सन्न तस्या॒म् तस्या॒न्न सम् । \newline
5. न सꣳ सन्न न स म॑राधयन् नराधय॒न् थ्सन्न न स म॑राधयन्न् । \newline
6. स म॑राधयन् नराधय॒न् थ्सꣳ स म॑राधय॒न् तौ ता व॑राधय॒न् थ्सꣳ स म॑राधय॒न् तौ । \newline
7. अ॒रा॒ध॒य॒न् तौ ता व॑राधयन् नराधय॒न् ता व॒श्विना॑ व॒श्विनौ॒ ता व॑राधयन् नराधय॒न् ता व॒श्विनौ᳚ । \newline
8. ता व॒श्विना॑ व॒श्विनौ॒ तौ ता व॒श्विना॑ वब्रूता मब्रूता म॒श्विनौ॒ तौ ता व॒श्विना॑ वब्रूताम् । \newline
9. अ॒श्विना॑ वब्रूता मब्रूता म॒श्विना॑ व॒श्विना॑ वब्रूता मा॒वयो॑ रा॒वयो॑ रब्रूता म॒श्विना॑ व॒श्विना॑ वब्रूता मा॒वयोः᳚ । \newline
10. अ॒ब्रू॒ता॒ मा॒वयो॑ रा॒वयो॑ रब्रूता मब्रूता मा॒वयो॒र् वै वा आ॒वयो॑ रब्रूता मब्रूता मा॒वयो॒र् वै । \newline
11. आ॒वयो॒र् वै वा आ॒वयो॑ रा॒वयो॒र् वा ए॒षैषा वा आ॒वयो॑ रा॒वयो॒र् वा ए॒षा । \newline
12. वा ए॒षैषा वै वा ए॒षा मा मैषा वै वा ए॒षा मा । \newline
13. ए॒षा मा मैषैषा मैतस्या॑ मे॒तस्या॒म् मैषैषा मैतस्या᳚म् । \newline
14. मैतस्या॑ मे॒तस्या॒म् मा मैतस्यां᳚ ॅवदद्ध्वं ॅवदद्ध्व मे॒तस्या॒म् मा मैतस्यां᳚ ॅवदद्ध्वम् । \newline
15. ए॒तस्यां᳚ ॅवदद्ध्वं ॅवदद्ध्व मे॒तस्या॑ मे॒तस्यां᳚ ॅवदद्ध्व॒ मितीति॑ वदद्ध्व मे॒तस्या॑ मे॒तस्यां᳚ ॅवदद्ध्व॒ मिति॑ । \newline
16. व॒द॒द्ध्व॒ मितीति॑ वदद्ध्वं ॅवदद्ध्व॒ मिति॒ सा सेति॑ वदद्ध्वं ॅवदद्ध्व॒ मिति॒ सा । \newline
17. इति॒ सा सेतीति॒ सा ऽश्विनो॑ र॒श्विनोः॒ सेतीति॒ सा ऽश्विनोः᳚ । \newline
18. सा ऽश्विनो॑ र॒श्विनोः॒ सा सा ऽश्विनो॑ रे॒वै वाश्विनोः॒ सा सा ऽश्विनो॑ रे॒व । \newline
19. अ॒श्विनो॑ रे॒वै वाश्विनो॑ र॒श्विनो॑ रे॒वाभ॑व दभव दे॒वाश्विनो॑ र॒श्विनो॑ रे॒वाभ॑वत् । \newline
20. ए॒वाभ॑व दभव दे॒वैवा भ॑व॒द् यो यो॑ ऽभव दे॒वैवा भ॑व॒द् यः । \newline
21. अ॒भ॒व॒द् यो यो॑ ऽभव दभव॒द् यः पुष्टि॑कामः॒ पुष्टि॑कामो॒ यो॑ ऽभव दभव॒द् यः पुष्टि॑कामः । \newline
22. यः पुष्टि॑कामः॒ पुष्टि॑कामो॒ यो यः पुष्टि॑कामः॒ स्याथ् स्यात् पुष्टि॑कामो॒ यो यः पुष्टि॑कामः॒ स्यात् । \newline
23. पुष्टि॑कामः॒ स्याथ् स्यात् पुष्टि॑कामः॒ पुष्टि॑कामः॒ स्याथ् स स स्यात् पुष्टि॑कामः॒ पुष्टि॑कामः॒ स्याथ् सः । \newline
24. पुष्टि॑काम॒ इति॒ पुष्टि॑ - का॒मः॒ । \newline
25. स्याथ् स स स्याथ् स्याथ् स ए॒ता मे॒ताꣳ स स्याथ् स्याथ् स ए॒ताम् । \newline
26. स ए॒ता मे॒ताꣳ स स ए॒ता मा᳚श्वि॒नी मा᳚श्वि॒नी मे॒ताꣳ स स ए॒ता मा᳚श्वि॒नीम् । \newline
27. ए॒ता मा᳚श्वि॒नी मा᳚श्वि॒नी मे॒ता मे॒ता मा᳚श्वि॒नीं ॅय॒मीं ॅय॒मी मा᳚श्वि॒नी मे॒ता मे॒ता मा᳚श्वि॒नीं ॅय॒मीम् । \newline
28. आ॒श्वि॒नीं ॅय॒मीं ॅय॒मी मा᳚श्वि॒नी मा᳚श्वि॒नीं ॅय॒मीं ॅव॒शां ॅव॒शां ॅय॒मी मा᳚श्वि॒नी मा᳚श्वि॒नीं ॅय॒मीं ॅव॒शाम् । \newline
29. य॒मीं ॅव॒शां ॅव॒शां ॅय॒मीं ॅय॒मीं ॅव॒शा मा व॒शां ॅय॒मीं ॅय॒मीं ॅव॒शा मा । \newline
30. व॒शा मा व॒शां ॅव॒शा मा ल॑भेत लभे॒ता व॒शां ॅव॒शा मा ल॑भेत । \newline
31. आ ल॑भेत लभे॒ता ल॑भे ता॒श्विना॑ व॒श्विनौ॑ लभे॒ता ल॑भे ता॒श्विनौ᳚ । \newline
32. ल॒भे॒ ता॒श्विना॑ व॒श्विनौ॑ लभेत लभे ता॒श्विना॑ वे॒वैवाश्विनौ॑ लभेत लभे ता॒श्विना॑ वे॒व । \newline
33. अ॒श्विना॑ वे॒वै वाश्विना॑ व॒श्विना॑ वे॒व स्वेन॒ स्वेनै॒ वाश्विना॑ व॒श्विना॑ वे॒व स्वेन॑ । \newline
34. ए॒व स्वेन॒ स्वेनै॒वैव स्वेन॑ भाग॒धेये॑न भाग॒धेये॑न॒ स्वेनै॒वैव स्वेन॑ भाग॒धेये॑न । \newline
35. स्वेन॑ भाग॒धेये॑न भाग॒धेये॑न॒ स्वेन॒ स्वेन॑ भाग॒धेये॒नो पोप॑ भाग॒धेये॑न॒ स्वेन॒ स्वेन॑ भाग॒धेये॒नोप॑ । \newline
36. भा॒ग॒धेये॒नो पोप॑ भाग॒धेये॑न भाग॒धेये॒नोप॑ धावति धाव॒त्युप॑ भाग॒धेये॑न भाग॒धेये॒नोप॑ धावति । \newline
37. भा॒ग॒धेये॒नेति॑ भाग - धेये॑न । \newline
38. उप॑ धावति धाव॒ त्युपोप॑ धावति॒ तौ तौ धा॑व॒ त्युपोप॑ धावति॒ तौ । \newline
39. धा॒व॒ति॒ तौ तौ धा॑वति धावति॒ ता वे॒वैव तौ धा॑वति धावति॒ ता वे॒व । \newline
40. ता वे॒वैव तौ ता वे॒वास्मि॑न् नस्मिन् ने॒व तौ ता वे॒वास्मिन्न्॑ । \newline
41. ए॒वास्मि॑न् नस्मिन् ने॒वैवास्मि॒न् पुष्टि॒म् पुष्टि॑ मस्मिन् ने॒वैवास्मि॒न् पुष्टि᳚म् । \newline
42. अ॒स्मि॒न् पुष्टि॒म् पुष्टि॑ मस्मिन् नस्मि॒न् पुष्टि॑म् धत्तो धत्तः॒ पुष्टि॑ मस्मिन् नस्मि॒न् पुष्टि॑म् धत्तः । \newline
43. पुष्टि॑म् धत्तो धत्तः॒ पुष्टि॒म् पुष्टि॑म् धत्तः॒ पुष्य॑ति॒ पुष्य॑ति धत्तः॒ पुष्टि॒म् पुष्टि॑म् धत्तः॒ पुष्य॑ति । \newline
44. ध॒त्तः॒ पुष्य॑ति॒ पुष्य॑ति धत्तो धत्तः॒ पुष्य॑ति प्र॒जया᳚ प्र॒जया॒ पुष्य॑ति धत्तो धत्तः॒ पुष्य॑ति प्र॒जया᳚ । \newline
45. पुष्य॑ति प्र॒जया᳚ प्र॒जया॒ पुष्य॑ति॒ पुष्य॑ति प्र॒जया॑ प॒शुभिः॑ प॒शुभिः॑ प्र॒जया॒ पुष्य॑ति॒ पुष्य॑ति प्र॒जया॑ प॒शुभिः॑ । \newline
46. प्र॒जया॑ प॒शुभिः॑ प॒शुभिः॑ प्र॒जया᳚ प्र॒जया॑ प॒शुभिः॑ । \newline
47. प्र॒जयेति॑प्र - जया᳚ । \newline
48. प॒शुभि॒रिति॑ प॒शु - भिः॒ । \newline
\pagebreak
\markright{ TS 2.1.10.1  \hfill https://www.vedavms.in \hfill}

\section{ TS 2.1.10.1 }

\textbf{TS 2.1.10.1 } \newline
\textbf{Samhita Paata} \newline

आ॒श्वि॒नं धू॒म्रल॑लाम॒मा ल॑भेत॒ यो दुर्ब्रा᳚ह्मणः॒ सोमं॒ पिपा॑सेद॒श्विनौ॒ वै दे॒वाना॒-मसो॑मपावास्तां॒ तौ प॒श्चा सो॑मपी॒थं प्राप्नु॑ता-म॒श्विना॑वे॒तस्य॑ दे॒वता॒ यो दुर्ब्रा᳚ह्मणः॒ सोमं॒ पिपा॑सत्य॒श्विना॑वे॒व स्वेन॑ भाग॒धेये॒नोप॑ धावति॒तावे॒वास्मै॑ सोमपी॒थं प्र य॑च्छत॒ उपै॑नꣳ सोमपी॒थो न॑मति॒ यद्-धू॒म्रो भव॑ति धूम्रि॒माण॑-मे॒वास्मा॒दप॑ हन्ति ल॒लामो॑ - [  ] \newline

\textbf{Pada Paata} \newline

आ॒श्वि॒नम् । धू॒म्रल॑लाम॒मिति॑ धू॒म्र - ल॒ला॒म॒म् । एति॑ । ल॒भे॒त॒ । यः । दुर्ब्रा᳚ह्मण॒ इति॒ दुः - ब्रा॒ह्म॒णः॒ । सोम᳚म् । पिपा॑सेत् । अ॒श्विनौ᳚ । वै । दे॒वाना᳚म् । असो॑मपा॒वित्यसो॑म - पौ॒ । आ॒स्ता॒म् । तौ । प॒श्चा । सो॒म॒पी॒थमिति॑ सोम - पी॒थम् । प्रेति॑ । आ॒प्नु॒ता॒म् । अ॒श्विनौ᳚ । ए॒तस्य॑ । दे॒वता᳚ । यः । दुर्ब्रा᳚ह्मण॒ इति॒ दुः- ब्रा॒ह्म॒णः॒ । सोम᳚म् । पिपा॑सति । अ॒श्विनौ᳚ । ए॒व । स्वेन॑ । भा॒ग॒धेये॒नेति॑ भाग - धेये॑न । उपेति॑ । धा॒व॒ति॒ । तौ । ए॒व । अ॒स्मै॒ । सो॒म॒पी॒थमिति॑ सोम-पी॒थम् । प्रेति॑ । य॒च्छ॒तः॒ । उपेति॑ । ए॒न॒म् । सो॒म॒पी॒थ इति॑ सोम - पी॒थः । न॒म॒ति॒ । यत् । धू॒म्रः । भव॑ति । धू॒म्रि॒माण᳚म् । ए॒व । अ॒स्मा॒त् । अपेति॑ । ह॒न्ति॒ । ल॒लामः॑ ।  \newline


\textbf{Krama Paata} \newline

आ॒श्वि॒नम् धू॒म्रल॑लामम् । धू॒म्रल॑लाम॒मा । धू॒म्रल॑लाम॒मिति॑ धू॒म्र - ल॒ला॒म॒म् । आ ल॑भेत । ल॒भे॒त॒ यः । यो दुर्ब्रा᳚ह्मणः । दुर्ब्रा᳚ह्मणः॒ सोम᳚म् । दुर्ब्रा᳚ह्मण॒ इति॒ दुः - ब्रा॒ह्म॒णः॒ । सोम॒म् पिपा॑सेत् । पिपा॑सेद॒श्विनौ᳚ । अ॒श्विनौ॒ वै । वै दे॒वाना᳚म् । दे॒वाना॒मसो॑मपौ । असो॑मपावास्ताम् । असो॑मपा॒वित्यसो॑म - पौ॒ । आ॒स्ता॒म् तौ । तौ प॒श्चा । प॒श्चा सो॑मपी॒थम् । सो॒म॒पी॒थम् प्र । सो॒म॒पी॒थमिति॑ सोम - पी॒थम् । प्राप्नु॑ताम् । आ॒प्नु॒ता॒म॒श्विनौ᳚ । अ॒श्विना॑वे॒तस्य॑ । ए॒तस्य॑ दे॒वता᳚ । दे॒वता॒ यः । यो दुर्ब्रा᳚ह्मणः । दुर्ब्रा᳚ह्मणः॒ सोम᳚म् । दुर्ब्रा᳚ह्मण॒ इति॒ दुः - ब्रा॒ह्म॒णः॒ । सोम॒म् पिपा॑सति । पिपा॑सत्य॒श्विनौ᳚ । अ॒श्विना॑वे॒व । ए॒व स्वेन॑ । स्वेन॑ भाग॒धेये॑न । भा॒ग॒धेये॒नोप॑ । भा॒ग॒धेये॒नेति॑ भाग - धेये॑न । उप॑ धावति । धा॒व॒ति॒ तौ । तावे॒व । ए॒वास्मै᳚ । अ॒स्मै॒ सो॒म॒पी॒थम् । सो॒म॒पी॒थम् प्र । सो॒म॒पी॒थमिति॑ सोम - पी॒थम् । प्र य॑च्छतः । य॒च्छ॒त॒ उप॑ । उपै॑नम् । ए॒नꣳ॒॒ सो॒म॒पी॒थः । सो॒म॒पी॒थो न॑मति । सो॒म॒पी॒थ इति॑ सोम - पी॒थः । न॒म॒ति॒ यत् । यद् धू॒म्रः । धू॒म्रो भव॑ति । भव॑ति धूम्रि॒माण᳚म् । धू॒म्रि॒माण॑मे॒व । ए॒वास्मा᳚त् । अ॒स्मा॒दप॑ । अप॑ हन्ति । ह॒न्ति॒ ल॒लामः॑ । ल॒लामो॑ भवति \newline

\textbf{Jatai Paata} \newline

1. आ॒श्वि॒नम् धू॒म्रल॑लामम् धू॒म्रल॑लाम माश्वि॒न मा᳚श्वि॒नम् धू॒म्रल॑लामम् । \newline
2. धू॒म्रल॑लाम॒ मा धू॒म्रल॑लामम् धू॒म्रल॑लाम॒ मा । \newline
3. धू॒म्रल॑लाम॒मिति॑ धू॒म्र - ल॒ला॒म॒म् । \newline
4. आ ल॑भेत लभे॒ता ल॑भेत । \newline
5. ल॒भे॒त॒ यो यो ल॑भेत लभेत॒ यः । \newline
6. यो दुर्ब्रा᳚ह्मणो॒ दुर्ब्रा᳚ह्मणो॒ यो यो दुर्ब्रा᳚ह्मणः । \newline
7. दुर्ब्रा᳚ह्मणः॒ सोमꣳ॒॒ सोम॒म् दुर्ब्रा᳚ह्मणो॒ दुर्ब्रा᳚ह्मणः॒ सोम᳚म् । \newline
8. दुर्ब्रा᳚ह्मण॒ इति॒ दुः - ब्रा॒ह्म॒णः॒ । \newline
9. सोम॒म् पिपा॑से॒त् पिपा॑से॒थ् सोमꣳ॒॒ सोम॒म् पिपा॑सेत् । \newline
10. पिपा॑से द॒श्विना॑ व॒श्विनौ॒ पिपा॑से॒त् पिपा॑से द॒श्विनौ᳚ । \newline
11. अ॒श्विनौ॒ वै वा अ॒श्विना॑ व॒श्विनौ॒ वै । \newline
12. वै दे॒वाना᳚म् दे॒वानां॒ ॅवै वै दे॒वाना᳚म् । \newline
13. दे॒वाना॒ मसो॑मपा॒ वसो॑मपौ दे॒वाना᳚म् दे॒वाना॒ मसो॑मपौ । \newline
14. असो॑मपा वास्ता मास्ता॒ मसो॑मपा॒ वसो॑मपा वास्ताम् । \newline
15. असो॑मपा॒वित्यसो॑म - पौ॒ । \newline
16. आ॒स्ता॒म् तौ ता वा᳚स्ता मास्ता॒म् तौ । \newline
17. तौ प॒श्चा प॒श्चा तौ तौ प॒श्चा । \newline
18. प॒श्चा सो॑मपी॒थꣳ सो॑मपी॒थम् प॒श्चा प॒श्चा सो॑मपी॒थम् । \newline
19. सो॒म॒पी॒थम् प्र प्र सो॑मपी॒थꣳ सो॑मपी॒थम् प्र । \newline
20. सो॒म॒पी॒थमिति॑ सोम - पी॒थम् । \newline
21. प्राप्नु॑ता माप्नुता॒म् प्र प्राप्नु॑ताम् । \newline
22. आ॒प्नु॒ता॒ म॒श्विना॑ व॒श्विना॑ वाप्नुता माप्नुता म॒श्विनौ᳚ । \newline
23. अ॒श्विना॑ वे॒त स्यै॒त स्या॒श्विना॑ व॒श्विना॑ वे॒तस्य॑ । \newline
24. ए॒तस्य॑ दे॒वता॑ दे॒व तै॒त स्यै॒तस्य॑ दे॒वता᳚ । \newline
25. दे॒वता॒ यो यो दे॒वता॑ दे॒वता॒ यः । \newline
26. यो दुर्ब्रा᳚ह्मणो॒ दुर्ब्रा᳚ह्मणो॒ यो यो दुर्ब्रा᳚ह्मणः । \newline
27. दुर्ब्रा᳚ह्मणः॒ सोमꣳ॒॒ सोम॒म् दुर्ब्रा᳚ह्मणो॒ दुर्ब्रा᳚ह्मणः॒ सोम᳚म् । \newline
28. दुर्ब्रा᳚ह्मण॒ इति॒ दुः - ब्रा॒ह्म॒णः॒ । \newline
29. सोम॒म् पिपा॑सति॒ पिपा॑सति॒ सोमꣳ॒॒ सोम॒म् पिपा॑सति । \newline
30. पिपा॑स त्य॒श्विना॑ व॒श्विनौ॒ पिपा॑सति॒ पिपा॑स त्य॒श्विनौ᳚ । \newline
31. अ॒श्विना॑ वे॒वै वाश्विना॑ व॒श्विना॑ वे॒व । \newline
32. ए॒व स्वेन॒ स्वेनै॒वैव स्वेन॑ । \newline
33. स्वेन॑ भाग॒धेये॑न भाग॒धेये॑न॒ स्वेन॒ स्वेन॑ भाग॒धेये॑न । \newline
34. भा॒ग॒धेये॒नोपोप॑ भाग॒धेये॑न भाग॒धेये॒नोप॑ । \newline
35. भा॒ग॒धेये॒नेति॑ भाग - धेये॑न । \newline
36. उप॑ धावति धाव॒ त्युपोप॑ धावति । \newline
37. धा॒व॒ति॒ तौ तौ धा॑वति धावति॒ तौ । \newline
38. ता वे॒वैव तौ ता वे॒व । \newline
39. ए॒वास्मा॑ अस्मा ए॒वैवास्मै᳚ । \newline
40. अ॒स्मै॒ सो॒म॒पी॒थꣳ सो॑मपी॒थ म॑स्मा अस्मै सोमपी॒थम् । \newline
41. सो॒म॒पी॒थम् प्र प्र सो॑मपी॒थꣳ सो॑मपी॒थम् प्र । \newline
42. सो॒म॒पी॒थमिति॑ सोम - पी॒थम् । \newline
43. प्र य॑च्छतो यच्छतः॒ प्र प्र य॑च्छतः । \newline
44. य॒च्छ॒त॒ उपोप॑ यच्छतो यच्छत॒ उप॑ । \newline
45. उपै॑न मेन॒ मुपोपै॑नम् । \newline
46. ए॒नꣳ॒॒ सो॒म॒पी॒थः सो॑मपी॒थ ए॑न मेनꣳ सोमपी॒थः । \newline
47. सो॒म॒पी॒थो न॑मति नमति सोमपी॒थः सो॑मपी॒थो न॑मति । \newline
48. सो॒म॒पी॒थ इति॑ सोम - पी॒थः । \newline
49. न॒म॒ति॒ यद् यन् न॑मति नमति॒ यत् । \newline
50. यद् धू॒म्रो धू॒म्रो यद् यद् धू॒म्रः । \newline
51. धू॒म्रो भव॑ति॒ भव॑ति धू॒म्रो धू॒म्रो भव॑ति । \newline
52. भव॑ति धूम्रि॒माण॑म् धूम्रि॒माण॒म् भव॑ति॒ भव॑ति धूम्रि॒माण᳚म् । \newline
53. धू॒म्रि॒माण॑ मे॒वैव धू᳚म्रि॒माण॑म् धूम्रि॒माण॑ मे॒व । \newline
54. ए॒वास्मा॑ दस्मा दे॒वैवास्मा᳚त् । \newline
55. अ॒स्मा॒ दपापा᳚ स्मा दस्मा॒ दप॑ । \newline
56. अप॑ हन्ति ह॒न्त्यपाप॑ हन्ति । \newline
57. ह॒न्ति॒ ल॒लामो॑ ल॒लामो॑ हन्ति हन्ति ल॒लामः॑ । \newline
58. ल॒लामो॑ भवति भवति ल॒लामो॑ ल॒लामो॑ भवति । \newline

\textbf{Ghana Paata } \newline

1. आ॒श्वि॒नम् धू॒म्रल॑लामम् धू॒म्रल॑लाम माश्वि॒न मा᳚श्वि॒नम् धू॒म्रल॑लाम॒ मा धू॒म्रल॑लाम माश्वि॒न मा᳚श्वि॒नम् धू॒म्रल॑लाम॒ मा । \newline
2. धू॒म्रल॑लाम॒ मा धू॒म्रल॑लामम् धू॒म्रल॑लाम॒ मा ल॑भेत लभे॒ता धू॒म्रल॑लामम् धू॒म्रल॑लाम॒ मा ल॑भेत । \newline
3. धू॒म्रल॑लाम॒मिति॑ धू॒म्र - ल॒ला॒म॒म् । \newline
4. आ ल॑भेत लभे॒ता ल॑भेत॒ यो यो ल॑भे॒ता ल॑भेत॒ यः । \newline
5. ल॒भे॒त॒ यो यो ल॑भेत लभेत॒ यो दुर्ब्रा᳚ह्मणो॒ दुर्ब्रा᳚ह्मणो॒ यो ल॑भेत लभेत॒ यो दुर्ब्रा᳚ह्मणः । \newline
6. यो दुर्ब्रा᳚ह्मणो॒ दुर्ब्रा᳚ह्मणो॒ यो यो दुर्ब्रा᳚ह्मणः॒ सोमꣳ॒॒ सोम॒म् दुर्ब्रा᳚ह्मणो॒ यो यो दुर्ब्रा᳚ह्मणः॒ सोम᳚म् । \newline
7. दुर्ब्रा᳚ह्मणः॒ सोमꣳ॒॒ सोम॒म् दुर्ब्रा᳚ह्मणो॒ दुर्ब्रा᳚ह्मणः॒ सोम॒म् पिपा॑से॒त् पिपा॑से॒थ् सोम॒म् दुर्ब्रा᳚ह्मणो॒ दुर्ब्रा᳚ह्मणः॒ सोम॒म् पिपा॑सेत् । \newline
8. दुर्ब्रा᳚ह्मण॒ इति॒ दुः - ब्रा॒ह्म॒णः॒ । \newline
9. सोम॒म् पिपा॑से॒त् पिपा॑से॒थ् सोमꣳ॒॒ सोम॒म् पिपा॑से द॒श्विना॑ व॒श्विनौ॒ पिपा॑से॒थ् सोमꣳ॒॒ सोम॒म् पिपा॑से द॒श्विनौ᳚ । \newline
10. पिपा॑से द॒श्विना॑ व॒श्विनौ॒ पिपा॑से॒त् पिपा॑से द॒श्विनौ॒ वै वा अ॒श्विनौ॒ पिपा॑से॒त् पिपा॑से द॒श्विनौ॒ वै । \newline
11. अ॒श्विनौ॒ वै वा अ॒श्विना॑ व॒श्विनौ॒ वै दे॒वाना᳚म् दे॒वानां॒ ॅवा अ॒श्विना॑ व॒श्विनौ॒ वै दे॒वाना᳚म् । \newline
12. वै दे॒वाना᳚म् दे॒वानां॒ ॅवै वै दे॒वाना॒ मसो॑मपा॒ वसो॑मपौ दे॒वानां॒ ॅवै वै दे॒वाना॒ मसो॑मपौ । \newline
13. दे॒वाना॒ मसो॑मपा॒ वसो॑मपौ दे॒वाना᳚म् दे॒वाना॒ मसो॑मपा वास्ता मास्ता॒ मसो॑मपौ दे॒वाना᳚म् दे॒वाना॒ मसो॑मपा वास्ताम् । \newline
14. असो॑मपा वास्ता मास्ता॒ मसो॑मपा॒ वसो॑मपा वास्ता॒म् तौ ता वा᳚स्ता॒ मसो॑मपा॒ वसो॑मपा वास्ता॒म् तौ । \newline
15. असो॑मपा॒वित्यसो॑म - पौ॒ । \newline
16. आ॒स्ता॒म् तौ ता वा᳚स्ता मास्ता॒म् तौ प॒श्चा प॒श्चा ता वा᳚स्ता मास्ता॒म् तौ प॒श्चा । \newline
17. तौ प॒श्चा प॒श्चा तौ तौ प॒श्चा सो॑मपी॒थꣳ सो॑मपी॒थम् प॒श्चा तौ तौ प॒श्चा सो॑मपी॒थम् । \newline
18. प॒श्चा सो॑मपी॒थꣳ सो॑मपी॒थम् प॒श्चा प॒श्चा सो॑मपी॒थम् प्र प्र सो॑मपी॒थम् प॒श्चा प॒श्चा सो॑मपी॒थम् प्र । \newline
19. सो॒म॒पी॒थम् प्र प्र सो॑मपी॒थꣳ सो॑मपी॒थम् प्राप्नु॑ता माप्नुता॒म् प्र सो॑मपी॒थꣳ सो॑मपी॒थम् प्राप्नु॑ताम् । \newline
20. सो॒म॒पी॒थमिति॑ सोम - पी॒थम् । \newline
21. प्राप्नु॑ता माप्नुता॒म् प्र प्राप्नु॑ता म॒श्विना॑ व॒श्विना॑ वाप्नुता॒म् प्र प्राप्नु॑ता म॒श्विनौ᳚ । \newline
22. आ॒प्नु॒ता॒ म॒श्विना॑ व॒श्विना॑ वाप्नुता माप्नुता म॒श्विना॑ वे॒त स्यै॒तस्या॒ श्विना॑ वाप्नुता माप्नुता म॒श्विना॑ वे॒तस्य॑ । \newline
23. अ॒श्विना॑ वे॒त स्यै॒त स्या॒श्विना॑ व॒श्विना॑ वे॒तस्य॑ दे॒वता॑ दे॒व तै॒तस्या॒श्विना॑ व॒श्विना॑ वे॒तस्य॑ दे॒वता᳚ । \newline
24. ए॒तस्य॑ दे॒वता॑ दे॒व तै॒तस्यै॒ तस्य॑ दे॒वता॒ यो यो दे॒व तै॒तस्यै॒ तस्य॑ दे॒वता॒ यः । \newline
25. दे॒वता॒ यो यो दे॒वता॑ दे॒वता॒ यो दुर्ब्रा᳚ह्मणो॒ दुर्ब्रा᳚ह्मणो॒ यो दे॒वता॑ दे॒वता॒ यो दुर्ब्रा᳚ह्मणः । \newline
26. यो दुर्ब्रा᳚ह्मणो॒ दुर्ब्रा᳚ह्मणो॒ यो यो दुर्ब्रा᳚ह्मणः॒ सोमꣳ॒॒ सोम॒म् दुर्ब्रा᳚ह्मणो॒ यो यो दुर्ब्रा᳚ह्मणः॒ सोम᳚म् । \newline
27. दुर्ब्रा᳚ह्मणः॒ सोमꣳ॒॒ सोम॒म् दुर्ब्रा᳚ह्मणो॒ दुर्ब्रा᳚ह्मणः॒ सोम॒म् पिपा॑सति॒ पिपा॑सति॒ सोम॒म् दुर्ब्रा᳚ह्मणो॒ दुर्ब्रा᳚ह्मणः॒ सोम॒म् पिपा॑सति । \newline
28. दुर्ब्रा᳚ह्मण॒ इति॒ दुः - ब्रा॒ह्म॒णः॒ । \newline
29. सोम॒म् पिपा॑सति॒ पिपा॑सति॒ सोमꣳ॒॒ सोम॒म् पिपा॑स त्य॒श्विना॑ व॒श्विनौ॒ पिपा॑सति॒ सोमꣳ॒॒ सोम॒म् पिपा॑स त्य॒श्विनौ᳚ । \newline
30. पिपा॑स त्य॒श्विना॑ व॒श्विनौ॒ पिपा॑सति॒ पिपा॑स त्य॒श्विना॑ वे॒वैवाश्विनौ॒ पिपा॑सति॒ पिपा॑स त्य॒श्विना॑ वे॒व । \newline
31. अ॒श्विना॑ वे॒वै वाश्विना॑ व॒श्विना॑ वे॒व स्वेन॒ स्वेनै॒ वाश्विना॑ व॒श्विना॑ वे॒व स्वेन॑ । \newline
32. ए॒व स्वेन॒ स्वेनै॒वैव स्वेन॑ भाग॒धेये॑न भाग॒धेये॑न॒ स्वेनै॒वैव स्वेन॑ भाग॒धेये॑न । \newline
33. स्वेन॑ भाग॒धेये॑न भाग॒धेये॑न॒ स्वेन॒ स्वेन॑ भाग॒धेये॒नो पोप॑ भाग॒धेये॑न॒ स्वेन॒ स्वेन॑ भाग॒धेये॒नोप॑ । \newline
34. भा॒ग॒धेये॒नो पोप॑ भाग॒धेये॑न भाग॒धेये॒नोप॑ धावति धाव॒त्युप॑ भाग॒धेये॑न भाग॒धेये॒नोप॑ धावति । \newline
35. भा॒ग॒धेये॒नेति॑ भाग - धेये॑न । \newline
36. उप॑ धावति धाव॒ त्युपोप॑ धावति॒ तौ तौ धा॑व॒ त्युपोप॑ धावति॒ तौ । \newline
37. धा॒व॒ति॒ तौ तौ धा॑वति धावति॒ ता वे॒वैव तौ धा॑वति धावति॒ ता वे॒व । \newline
38. ता वे॒वैव तौ ता वे॒वास्मा॑ अस्मा ए॒व तौ ता वे॒वास्मै᳚ । \newline
39. ए॒वास्मा॑ अस्मा ए॒वैवास्मै॑ सोमपी॒थꣳ सो॑मपी॒थ म॑स्मा ए॒वैवास्मै॑ सोमपी॒थम् । \newline
40. अ॒स्मै॒ सो॒म॒पी॒थꣳ सो॑मपी॒थ म॑स्मा अस्मै सोमपी॒थम् प्र प्र सो॑मपी॒थ म॑स्मा अस्मै सोमपी॒थम् प्र । \newline
41. सो॒म॒पी॒थम् प्र प्र सो॑मपी॒थꣳ सो॑मपी॒थम् प्र य॑च्छतो यच्छतः॒ प्र सो॑मपी॒थꣳ सो॑मपी॒थम् प्र य॑च्छतः । \newline
42. सो॒म॒पी॒थमिति॑ सोम - पी॒थम् । \newline
43. प्र य॑च्छतो यच्छतः॒ प्र प्र य॑च्छत॒ उपोप॑ यच्छतः॒ प्र प्र य॑च्छत॒ उप॑ । \newline
44. य॒च्छ॒त॒ उपोप॑ यच्छतो यच्छत॒ उपै॑न मेन॒ मुप॑ यच्छतो यच्छत॒ उपै॑नम् । \newline
45. उपै॑न मेन॒ मुपोपै॑नꣳ सोमपी॒थः सो॑मपी॒थ ए॑न॒ मुपोपै॑नꣳ सोमपी॒थः । \newline
46. ए॒नꣳ॒॒ सो॒म॒पी॒थः सो॑मपी॒थ ए॑न मेनꣳ सोमपी॒थो न॑मति नमति सोमपी॒थ ए॑न मेनꣳ सोमपी॒थो न॑मति । \newline
47. सो॒म॒पी॒थो न॑मति नमति सोमपी॒थः सो॑मपी॒थो न॑मति॒ यद् यन् न॑मति सोमपी॒थः सो॑मपी॒थो न॑मति॒ यत् । \newline
48. सो॒म॒पी॒थ इति॑ सोम - पी॒थः । \newline
49. न॒म॒ति॒ यद् यन् न॑मति नमति॒ यद् धू॒म्रो धू॒म्रो यन् न॑मति नमति॒ यद् धू॒म्रः । \newline
50. यद् धू॒म्रो धू॒म्रो यद् यद् धू॒म्रो भव॑ति॒ भव॑ति धू॒म्रो यद् यद् धू॒म्रो भव॑ति । \newline
51. धू॒म्रो भव॑ति॒ भव॑ति धू॒म्रो धू॒म्रो भव॑ति धूम्रि॒माण॑म् धूम्रि॒माण॒म् भव॑ति धू॒म्रो धू॒म्रो भव॑ति धूम्रि॒माण᳚म् । \newline
52. भव॑ति धूम्रि॒माण॑म् धूम्रि॒माण॒म् भव॑ति॒ भव॑ति धूम्रि॒माण॑ मे॒वैव धू᳚म्रि॒माण॒म् भव॑ति॒ भव॑ति धूम्रि॒माण॑ मे॒व । \newline
53. धू॒म्रि॒माण॑ मे॒वैव धू᳚म्रि॒माण॑म् धूम्रि॒माण॑ मे॒वास्मा॑ दस्मा दे॒व धू᳚म्रि॒माण॑म् धूम्रि॒माण॑ मे॒वास्मा᳚त् । \newline
54. ए॒वास्मा॑ दस्मा दे॒वैवास्मा॒ दपापा᳚स्मा दे॒वैवास्मा॒ दप॑ । \newline
55. अ॒स्मा॒ दपापा᳚स्मा दस्मा॒ दप॑ हन्ति ह॒न्त्यपा᳚स्मा दस्मा॒ दप॑ हन्ति । \newline
56. अप॑ हन्ति ह॒न्त्यपाप॑ हन्ति ल॒लामो॑ ल॒लामो॑ ह॒न्त्यपाप॑ हन्ति ल॒लामः॑ । \newline
57. ह॒न्ति॒ ल॒लामो॑ ल॒लामो॑ हन्ति हन्ति ल॒लामो॑ भवति भवति ल॒लामो॑ हन्ति हन्ति ल॒लामो॑ भवति । \newline
58. ल॒लामो॑ भवति भवति ल॒लामो॑ ल॒लामो॑ भवति मुख॒तो मु॑ख॒तो भ॑वति ल॒लामो॑ ल॒लामो॑ भवति मुख॒तः । \newline
\pagebreak
\markright{ TS 2.1.10.2  \hfill https://www.vedavms.in \hfill}

\section{ TS 2.1.10.2 }

\textbf{TS 2.1.10.2 } \newline
\textbf{Samhita Paata} \newline

भवति मुख॒त ए॒वास्मि॒न् तेजो॑ दधाति वाय॒व्यं॑ गोमृ॒गमा ल॑भेत॒ यमज॑घ्निवाꣳ समभि॒शꣳ से॑यु॒रपू॑ता॒ वा ए॒तं ॅवागृ॑च्छति॒ यमज॑घ्निवाꣳ समभि॒शꣳ स॑न्ति॒ नैष ग्रा॒म्यः प॒शुर्नार॒ण्यो यद्-गो॑मृ॒गो नेवै॒ष ग्रामे॒ नार॑ण्ये॒ यमज॑घ्निवाꣳ समभि॒शꣳ स॑न्ति वा॒युर्वै दे॒वानां᳚ प॒वित्रं॑ ॅवा॒युमे॒व स्वेन॑ भाग॒धेये॒नोप॑ धावति॒ स ए॒वै - [  ] \newline

\textbf{Pada Paata} \newline

भ॒व॒ति॒ । मु॒ख॒तः । ए॒व । अ॒स्मि॒न्न् । तेजः॑ । द॒धा॒ति॒ । वा॒य॒व्य᳚म् । गो॒मृ॒गमिति॑ गो - मृ॒गम् । एति॑ । ल॒भे॒त॒ । यम् । अज॑घ्निवाꣳसम् । अ॒भि॒शꣳसे॑यु॒रित्य॑भि - शꣳसे॑युः । अपू॑ता । वै । ए॒तम् । वाक् । ऋ॒च्छ॒ति॒ । यम् । अज॑घ्निवाꣳसम् । अ॒भि॒शꣳस॒न्तीत्य॑भि - शꣳस॑न्ति । न । ए॒षः । ग्रा॒म्यः । प॒शुः । न । आ॒र॒ण्यः । यत् । गो॒मृ॒ग इति॑ गो - मृ॒गः । न । इ॒व॒ । ए॒षः । ग्रामे᳚ । न । अर॑ण्ये । यम् । अज॑घ्निवाꣳसम् । अ॒भि॒शꣳस॒न्तीत्य॑भि-शꣳस॑न्ति । वा॒युः । वै । दे॒वाना᳚म् । प॒वित्र᳚म् । वा॒युम् । ए॒व । स्वेन॑ । भा॒ग॒धेये॒नेति॑ भाग-धेये॑न । उपेति॑ । धा॒व॒ति॒ । सः । ए॒व ।  \newline


\textbf{Krama Paata} \newline

भ॒व॒ति॒ मु॒ख॒तः । मु॒ख॒त ए॒व । ए॒वास्मिन्न्॑ । अ॒स्मि॒न् तेजः॑ । तेजो॑ दधाति । द॒धा॒ति॒ वा॒य॒व्य᳚म् । वा॒य॒व्य॑म् गोमृ॒गम् । गो॒मृ॒गमा । गो॒मृ॒गमिति॑ गो - मृ॒गम् । आ ल॑भेत । ल॒भे॒त॒ यम् । यमज॑घ्निवाꣳसम् । 
अज॑घ्निवाꣳसमभि॒शꣳसे॑युः । अ॒भि॒शꣳसे॑यु॒रपू॑ता । अ॒भि॒शꣳसे॑यु॒रित्य॑भि - शꣳसे॑युः । अपू॑ता॒ वै । वा ए॒तम् । ए॒तं ॅवाक् । वागृ॑च्छति । ऋ॒च्छ॒ति॒ यम् । यमज॑घ्निवाꣳसम् । अज॑घ्निवाꣳसमभि॒शꣳस॑न्ति । अ॒भि॒शꣳस॑न्ति॒ न । अ॒भि॒शꣳस॒न्तीत्य॑भि - शꣳस॑न्ति । नैषः । ए॒ष ग्रा॒म्यः । ग्रा॒म्यः प॒शुः । प॒शुर् न । नार॒ण्यः । आ॒र॒ण्यो यत् । यद् गो॑मृ॒गः । गो॒मृ॒गो न । गो॒मृ॒ग इति॑ गो - मृ॒गः । नेव॑ । इ॒वै॒षः । ए॒ष ग्रामे᳚ । ग्रामे॒ न । नार॑ण्ये । अर॑ण्ये॒ यम् । यमज॑घ्निवाꣳसम् । अज॑घ्निवाꣳसमभि॒शꣳस॑न्ति । अ॒भि॒शꣳस॑न्ति वा॒युः । अ॒भि॒शꣳस॒न्तीत्य॑भि - शꣳस॑न्ति । वा॒युर् वै । वै दे॒वाना᳚म् । दे॒वाना᳚म् प॒वित्र᳚म् । प॒वित्रं॑ ॅवा॒युम् । वा॒युमे॒व । ए॒व स्वेन॑ । स्वेन॑ भाग॒धेये॑न । भा॒ग॒धेये॒नोप॑ । भा॒ग॒धेये॒नेति॑ भाग - धेये॑न । उप॑ धावति । धा॒व॒ति॒ सः । स ए॒व ( ) । ए॒वैन᳚म् \newline

\textbf{Jatai Paata} \newline

1. भ॒व॒ति॒ मु॒ख॒तो मु॑ख॒तो भ॑वति भवति मुख॒तः । \newline
2. मु॒ख॒त ए॒वैव मु॑ख॒तो मु॑ख॒त ए॒व । \newline
3. ए॒वास्मि॑न् नस्मिन् ने॒वैवास्मिन्न्॑ । \newline
4. अ॒स्मि॒न् तेज॒स्तेजो᳚ ऽस्मिन् नस्मि॒न् तेजः॑ । \newline
5. तेजो॑ दधाति दधाति॒ तेज॒ स्तेजो॑ दधाति । \newline
6. द॒धा॒ति॒ वा॒य॒व्यं॑ ॅवाय॒व्य॑म् दधाति दधाति वाय॒व्य᳚म् । \newline
7. वा॒य॒व्य॑म् गोमृ॒गम् गो॑मृ॒गं ॅवा॑य॒व्यं॑ ॅवाय॒व्य॑म् गोमृ॒गम् । \newline
8. गो॒मृ॒ग मा गो॑मृ॒गम् गो॑मृ॒ग मा । \newline
9. गो॒मृ॒गमिति॑ गो - मृ॒गम् । \newline
10. आ ल॑भेत लभे॒ता ल॑भेत । \newline
11. ल॒भे॒त॒ यं ॅयम् ॅल॑भेत लभेत॒ यम् । \newline
12. य मज॑घ्निवाꣳस॒ मज॑घ्निवाꣳसं॒ ॅयं ॅय मज॑घ्निवाꣳसम् । \newline
13. अज॑घ्निवाꣳस मभि॒शꣳसे॑यु रभि॒शꣳसे॑यु॒ रज॑घ्निवाꣳस॒ मज॑घ्निवाꣳस मभि॒शꣳसे॑युः । \newline
14. अ॒भि॒शꣳसे॑यु॒ रपू॒ता ऽपू॑ता ऽभि॒शꣳसे॑यु रभि॒शꣳसे॑यु॒ रपू॑ता । \newline
15. अ॒भि॒शꣳसे॑यु॒रित्य॑भि - शꣳसे॑युः । \newline
16. अपू॑ता॒ वै वा अपू॒ता ऽपू॑ता॒ वै । \newline
17. वा ए॒त मे॒तं ॅवै वा ए॒तम् । \newline
18. ए॒तं ॅवाग् वागे॒त मे॒तं ॅवाक् । \newline
19. वागृ॑च्छ त्यृच्छति॒ वाग् वागृ॑च्छति । \newline
20. ऋ॒च्छ॒ति॒ यं ॅय मृ॑च्छ त्यृच्छति॒ यम् । \newline
21. य मज॑घ्निवाꣳस॒ मज॑घ्निवाꣳसं॒ ॅयं ॅय मज॑घ्निवाꣳसम् । \newline
22. अज॑घ्निवाꣳस मभि॒शꣳस॑ न्त्यभि॒शꣳस॒ न्त्यज॑घ्निवाꣳस॒ मज॑घ्निवाꣳस मभि॒शꣳस॑न्ति । \newline
23. अ॒भि॒शꣳस॑न्ति॒ न नाभि॒शꣳस॑ न्त्यभि॒शꣳस॑न्ति॒ न । \newline
24. अ॒भि॒शꣳस॒न्तीत्य॑भि - शꣳस॑न्ति । \newline
25. नैष ए॒ष न नैषः । \newline
26. ए॒ष ग्रा॒म्यो ग्रा॒म्य ए॒ष ए॒ष ग्रा॒म्यः । \newline
27. ग्रा॒म्यः प॒शुः प॒शुर् ग्रा॒म्यो ग्रा॒म्यः प॒शुः । \newline
28. प॒शुर् न न प॒शुः प॒शुर् न । \newline
29. नार॒ण्य आ॑र॒ण्यो न नार॒ण्यः । \newline
30. आ॒र॒ण्यो यद् यदा॑र॒ण्य आ॑र॒ण्यो यत् । \newline
31. यद् गो॑मृ॒गो गो॑मृ॒गो यद् यद् गो॑मृ॒गः । \newline
32. गो॒मृ॒गो न न गो॑मृ॒गो गो॑मृ॒गो न । \newline
33. गो॒मृ॒ग इति॑ गो - मृ॒गः । \newline
34. ने वे॑ व॒ न ने व॑ । \newline
35. इ॒वै॒ष ए॒ष इ॑वे वै॒षः । \newline
36. ए॒ष ग्रामे॒ ग्राम॑ ए॒ष ए॒ष ग्रामे᳚ । \newline
37. ग्रामे॒ न न ग्रामे॒ ग्रामे॒ न । \newline
38. नार॒ण्ये ऽर॑ण्ये॒ न नार॑ण्ये । \newline
39. अर॑ण्ये॒ यं ॅय मर॒ण्ये ऽर॑ण्ये॒ यम् । \newline
40. य मज॑घ्निवाꣳस॒ मज॑घ्निवाꣳसं॒ ॅयं ॅय मज॑घ्निवाꣳसम् । \newline
41. अज॑घ्निवाꣳस मभि॒शꣳस॑ न्त्यभि॒शꣳस॒ न्त्यज॑घ्निवाꣳस॒ मज॑घ्निवाꣳस मभि॒शꣳस॑न्ति । \newline
42. अ॒भि॒शꣳस॑न्ति वा॒युर् वा॒यु र॑भि॒शꣳस॑ न्त्यभि॒शꣳस॑न्ति वा॒युः । \newline
43. अ॒भि॒शꣳस॒न्तीत्य॑भि - शꣳस॑न्ति । \newline
44. वा॒युर् वै वै वा॒युर् वा॒युर् वै । \newline
45. वै दे॒वाना᳚म् दे॒वानां॒ ॅवै वै दे॒वाना᳚म् । \newline
46. दे॒वाना᳚म् प॒वित्र॑म् प॒वित्र॑म् दे॒वाना᳚म् दे॒वाना᳚म् प॒वित्र᳚म् । \newline
47. प॒वित्रं॑ ॅवा॒युं ॅवा॒युम् प॒वित्र॑म् प॒वित्रं॑ ॅवा॒युम् । \newline
48. वा॒यु मे॒वैव वा॒युं ॅवा॒यु मे॒व । \newline
49. ए॒व स्वेन॒ स्वेनै॒वैव स्वेन॑ । \newline
50. स्वेन॑ भाग॒धेये॑न भाग॒धेये॑न॒ स्वेन॒ स्वेन॑ भाग॒धेये॑न । \newline
51. भा॒ग॒धेये॒नोपोप॑ भाग॒धेये॑न भाग॒धेये॒नोप॑ । \newline
52. भा॒ग॒धेये॒नेति॑ भाग - धेये॑न । \newline
53. उप॑ धावति धाव॒ त्युपोप॑ धावति । \newline
54. धा॒व॒ति॒ स स धा॑वति धावति॒ सः । \newline
55. स ए॒वैव स स ए॒व । \newline
56. ए॒वैन॑ मेन मे॒वैवैन᳚म् । \newline

\textbf{Ghana Paata } \newline

1. भ॒व॒ति॒ मु॒ख॒तो मु॑ख॒तो भ॑वति भवति मुख॒त ए॒वैव मु॑ख॒तो भ॑वति भवति मुख॒त ए॒व । \newline
2. मु॒ख॒त ए॒वैव मु॑ख॒तो मु॑ख॒त ए॒वास्मि॑न् नस्मिन् ने॒व मु॑ख॒तो मु॑ख॒त ए॒वास्मिन्न्॑ । \newline
3. ए॒वास्मि॑न् नस्मिन् ने॒वैवास्मि॒न् तेज॒ स्तेजो᳚ ऽस्मिन् ने॒वैवास्मि॒न् तेजः॑ । \newline
4. अ॒स्मि॒न् तेज॒ स्तेजो᳚ ऽस्मिन् नस्मि॒न् तेजो॑ दधाति दधाति॒ तेजो᳚ ऽस्मिन् नस्मि॒न् तेजो॑ दधाति । \newline
5. तेजो॑ दधाति दधाति॒ तेज॒ स्तेजो॑ दधाति वाय॒व्यं॑ ॅवाय॒व्य॑म् दधाति॒ तेज॒ स्तेजो॑ दधाति वाय॒व्य᳚म् । \newline
6. द॒धा॒ति॒ वा॒य॒व्यं॑ ॅवाय॒व्य॑म् दधाति दधाति वाय॒व्य॑म् गोमृ॒गम् गो॑मृ॒गं ॅवा॑य॒व्य॑म् दधाति दधाति वाय॒व्य॑म् गोमृ॒गम् । \newline
7. वा॒य॒व्य॑म् गोमृ॒गम् गो॑मृ॒गं ॅवा॑य॒व्यं॑ ॅवाय॒व्य॑म् गोमृ॒ग मा गो॑मृ॒गं ॅवा॑य॒व्यं॑ ॅवाय॒व्य॑म् गोमृ॒ग मा । \newline
8. गो॒मृ॒ग मा गो॑मृ॒गम् गो॑मृ॒ग मा ल॑भेत लभे॒ता गो॑मृ॒गम् गो॑मृ॒ग मा ल॑भेत । \newline
9. गो॒मृ॒गमिति॑ गो - मृ॒गम् । \newline
10. आ ल॑भेत लभे॒ता ल॑भेत॒ यं ॅयम् ॅल॑भे॒ता ल॑भेत॒ यम् । \newline
11. ल॒भे॒त॒ यं ॅयम् ॅल॑भेत लभेत॒ य मज॑घ्निवाꣳस॒ मज॑घ्निवाꣳसं॒ ॅयम् ॅल॑भेत लभेत॒ य मज॑घ्निवाꣳसम् । \newline
12. य मज॑घ्निवाꣳस॒ मज॑घ्निवाꣳसं॒ ॅयं ॅय मज॑घ्निवाꣳस मभि॒शꣳसे॑यु रभि॒शꣳसे॑यु॒ रज॑घ्निवाꣳसं॒ ॅयं ॅय मज॑घ्निवाꣳस मभि॒शꣳसे॑युः । \newline
13. अज॑घ्निवाꣳस मभि॒शꣳसे॑यु रभि॒शꣳसे॑यु॒ रज॑घ्निवाꣳस॒ मज॑घ्निवाꣳस मभि॒शꣳसे॑यु॒ रपू॒ता ऽपू॑ता ऽभि॒शꣳसे॑यु॒ रज॑घ्निवाꣳस॒ मज॑घ्निवाꣳस मभि॒शꣳसे॑यु॒ रपू॑ता । \newline
14. अ॒भि॒शꣳसे॑यु॒ रपू॒ता ऽपू॑ता ऽभि॒शꣳसे॑यु रभि॒शꣳसे॑यु॒ रपू॑ता॒ वै वा अपू॑ता ऽभि॒शꣳसे॑यु रभि॒शꣳसे॑यु॒ रपू॑ता॒ वै । \newline
15. अ॒भि॒शꣳसे॑यु॒रित्य॑भि - शꣳसे॑युः । \newline
16. अपू॑ता॒ वै वा अपू॒ता ऽपू॑ता॒ वा ए॒त मे॒तं ॅवा अपू॒ता ऽपू॑ता॒ वा ए॒तम् । \newline
17. वा ए॒त मे॒तं ॅवै वा ए॒तं ॅवाग् वागे॒तं ॅवै वा ए॒तं ॅवाक् । \newline
18. ए॒तं ॅवाग् वागे॒त मे॒तं ॅवागृ॑च्छ त्यृच्छति॒ वागे॒त मे॒तं ॅवागृ॑च्छति । \newline
19. वागृ॑च्छ त्यृच्छति॒ वाग् वागृ॑च्छति॒ यं ॅय मृ॑च्छति॒ वाग् वागृ॑च्छति॒ यम् । \newline
20. ऋ॒च्छ॒ति॒ यं ॅय मृ॑च्छ त्यृच्छति॒ य मज॑घ्निवाꣳस॒ मज॑घ्निवाꣳसं॒ ॅय मृ॑च्छ त्यृच्छति॒ य मज॑घ्निवाꣳसम् । \newline
21. य मज॑घ्निवाꣳस॒ मज॑घ्निवाꣳसं॒ ॅयं ॅय मज॑घ्निवाꣳस मभि॒शꣳस॑ न्त्यभि॒शꣳस॒ न्त्यज॑घ्निवाꣳसं॒ ॅयं ॅय मज॑घ्निवाꣳस मभि॒शꣳस॑न्ति । \newline
22. अज॑घ्निवाꣳस मभि॒शꣳस॑ न्त्यभि॒शꣳस॒ न्त्यज॑घ्निवाꣳस॒ मज॑घ्निवाꣳस मभि॒शꣳस॑न्ति॒ न नाभि॒शꣳस॒ न्त्यज॑घ्निवाꣳस॒ मज॑घ्निवाꣳस मभि॒शꣳस॑न्ति॒ न । \newline
23. अ॒भि॒शꣳस॑न्ति॒ न नाभि॒शꣳस॑ न्त्यभि॒शꣳस॑न्ति॒ नैष ए॒ष नाभि॒शꣳस॑ न्त्यभि॒शꣳस॑न्ति॒ नैषः । \newline
24. अ॒भि॒शꣳस॒न्तीत्य॑भि - शꣳस॑न्ति । \newline
25. नैष ए॒ष न नैष ग्रा॒म्यो ग्रा॒म्य ए॒ष न नैष ग्रा॒म्यः । \newline
26. ए॒ष ग्रा॒म्यो ग्रा॒म्य ए॒ष ए॒ष ग्रा॒म्यः प॒शुः प॒शुर् ग्रा॒म्य ए॒ष ए॒ष ग्रा॒म्यः प॒शुः । \newline
27. ग्रा॒म्यः प॒शुः प॒शुर् ग्रा॒म्यो ग्रा॒म्यः प॒शुर् न न प॒शुर् ग्रा॒म्यो ग्रा॒म्यः प॒शुर् न । \newline
28. प॒शुर् न न प॒शुः प॒शुर् नार॒ण्य आ॑र॒ण्यो न प॒शुः प॒शुर् नार॒ण्यः । \newline
29. नार॒ण्य आ॑र॒ण्यो न नार॒ण्यो यद् यदा॑र॒ण्यो न नार॒ण्यो यत् । \newline
30. आ॒र॒ण्यो यद् यदा॑र॒ण्य आ॑र॒ण्यो यद् गो॑मृ॒गो गो॑मृ॒गो यदा॑र॒ण्य आ॑र॒ण्यो यद् गो॑मृ॒गः । \newline
31. यद् गो॑मृ॒गो गो॑मृ॒गो यद् यद् गो॑मृ॒गो न न गो॑मृ॒गो यद् यद् गो॑मृ॒गो न । \newline
32. गो॒मृ॒गो न न गो॑मृ॒गो गो॑मृ॒गो ने वे॑ व॒ न गो॑मृ॒गो गो॑मृ॒गो ने व॑ । \newline
33. गो॒मृ॒ग इति॑ गो - मृ॒गः । \newline
34. ने वे॑ व॒ न ने वै॒ष ए॒ष इ॑व॒ न ने वै॒षः । \newline
35. इ॒वै॒ष ए॒ष इ॑वे वै॒ष ग्रामे॒ ग्राम॑ ए॒ष इ॑वे वै॒ष ग्रामे᳚ । \newline
36. ए॒ष ग्रामे॒ ग्राम॑ ए॒ष ए॒ष ग्रामे॒ न न ग्राम॑ ए॒ष ए॒ष ग्रामे॒ न । \newline
37. ग्रामे॒ न न ग्रामे॒ ग्रामे॒ नार॒ण्ये ऽर॑ण्ये॒ न ग्रामे॒ ग्रामे॒ नार॑ण्ये । \newline
38. नार॒ण्ये ऽर॑ण्ये॒ न नार॑ण्ये॒ यं ॅय मर॑ण्ये॒ न नार॑ण्ये॒ यम् । \newline
39. अर॑ण्ये॒ यं ॅय मर॒ण्ये ऽर॑ण्ये॒ य मज॑घ्निवाꣳस॒ मज॑घ्निवाꣳसं॒ ॅय मर॒ण्ये ऽर॑ण्ये॒ य मज॑घ्निवाꣳसम् । \newline
40. य मज॑घ्निवाꣳस॒ मज॑घ्निवाꣳसं॒ ॅयं ॅय मज॑घ्निवाꣳस मभि॒शꣳस॑ न्त्यभि॒शꣳस॒ न्त्यज॑घ्निवाꣳसं॒ ॅयं ॅय मज॑घ्निवाꣳस मभि॒शꣳस॑न्ति । \newline
41. अज॑घ्निवाꣳस मभि॒शꣳस॑ न्त्यभि॒शꣳस॒ न्त्यज॑घ्निवाꣳस॒ मज॑घ्निवाꣳस मभि॒शꣳस॑न्ति वा॒युर् वा॒युर॑भि॒शꣳस॒ न्त्यज॑घ्निवाꣳस॒ मज॑घ्निवाꣳस मभि॒शꣳस॑न्ति वा॒युः । \newline
42. अ॒भि॒शꣳस॑न्ति वा॒युर् वा॒युर॑भि॒शꣳस॑ न्त्यभि॒शꣳस॑न्ति वा॒युर् वै वै वा॒युर॑भि॒शꣳस॑ न्त्यभि॒शꣳस॑न्ति वा॒युर् वै । \newline
43. अ॒भि॒शꣳस॒न्तीत्य॑भि - शꣳस॑न्ति । \newline
44. वा॒युर् वै वै वा॒युर् वा॒युर् वै दे॒वाना᳚म् दे॒वानां॒ ॅवै वा॒युर् वा॒युर् वै दे॒वाना᳚म् । \newline
45. वै दे॒वाना᳚म् दे॒वानां॒ ॅवै वै दे॒वाना᳚म् प॒वित्र॑म् प॒वित्र॑म् दे॒वानां॒ ॅवै वै दे॒वाना᳚म् प॒वित्र᳚म् । \newline
46. दे॒वाना᳚म् प॒वित्र॑म् प॒वित्र॑म् दे॒वाना᳚म् दे॒वाना᳚म् प॒वित्रं॑ ॅवा॒युं ॅवा॒युम् प॒वित्र॑म् दे॒वाना᳚म् दे॒वाना᳚म् प॒वित्रं॑ ॅवा॒युम् । \newline
47. प॒वित्रं॑ ॅवा॒युं ॅवा॒युम् प॒वित्र॑म् प॒वित्रं॑ ॅवा॒यु मे॒वैव वा॒युम् प॒वित्र॑म् प॒वित्रं॑ ॅवा॒यु मे॒व । \newline
48. वा॒यु मे॒वैव वा॒युं ॅवा॒यु मे॒व स्वेन॒ स्वेनै॒व वा॒युं ॅवा॒यु मे॒व स्वेन॑ । \newline
49. ए॒व स्वेन॒ स्वेनै॒वैव स्वेन॑ भाग॒धेये॑न भाग॒धेये॑न॒ स्वेनै॒वैव स्वेन॑ भाग॒धेये॑न । \newline
50. स्वेन॑ भाग॒धेये॑न भाग॒धेये॑न॒ स्वेन॒ स्वेन॑ भाग॒धेये॒नो पोप॑ भाग॒धेये॑न॒ स्वेन॒ स्वेन॑ भाग॒धेये॒नोप॑ । \newline
51. भा॒ग॒धेये॒नो पोप॑ भाग॒धेये॑न भाग॒धेये॒नोप॑ धावति धाव॒त्युप॑ भाग॒धेये॑न भाग॒धेये॒नोप॑ धावति । \newline
52. भा॒ग॒धेये॒नेति॑ भाग - धेये॑न । \newline
53. उप॑ धावति धाव॒ त्युपोप॑ धावति॒ स स धा॑व॒ त्युपोप॑ धावति॒ सः । \newline
54. धा॒व॒ति॒ स स धा॑वति धावति॒ स ए॒वैव स धा॑वति धावति॒ स ए॒व । \newline
55. स ए॒वैव स स ए॒वैन॑ मेन मे॒व स स ए॒वैन᳚म् । \newline
56. ए॒वैन॑ मेन मे॒वैवैन॑म् पवयति पवयत्येन मे॒वैवैन॑म् पवयति । \newline
\pagebreak
\markright{ TS 2.1.10.3  \hfill https://www.vedavms.in \hfill}

\section{ TS 2.1.10.3 }

\textbf{TS 2.1.10.3 } \newline
\textbf{Samhita Paata} \newline

-नं॑ पवयति॒ परा॑ची॒ वा ए॒तस्मै᳚ व्यु॒च्छन्ती॒ व्यु॑च्छति॒ तमः॑ पा॒प्मानं॒ प्रवि॑शति॒ यस्या᳚श्वि॒ने श॒स्यमा॑ने॒ सूर्यो॒ ना*ऽऽविर्भव॑ति सौ॒र्यं ब॑हुरू॒पमा ल॑भेता॒मुमे॒वादि॒त्यꣳ स्वेन॑ भाग॒धेये॒नोप॑ धावति॒ स ए॒वास्मा॒त् तमः॑ पा॒प्मान॒मप॑ हन्ति प्र॒तीच्य॑स्मै व्यु॒च्छन्ती॒ व्यु॑च्छ॒त्यप॒ तमः॑ पा॒प्मानꣳ॑ हते ॥ \newline

\textbf{Pada Paata} \newline

ए॒न॒म् । प॒व॒य॒ति॒ । परा॑ची । वै । ए॒तस्मै᳚ । व्यु॒च्छन्तीति॑ वि-उ॒च्छन्ती᳚ । वीति॑ । उ॒च्छ॒ति॒ । तमः॑ । पा॒प्मान᳚म् । प्रेति॑ । वि॒श॒ति॒ । यस्य॑ । आ॒श्वि॒ने । श॒स्यमा॑ने । सूर्यः॑ । न । आ॒विः । भव॑ति । सौ॒र्यम् । ब॒हु॒रू॒पमिति॑ बहु - रू॒पम् । एति॑ । ल॒भे॒त॒ । अ॒मुम् । ए॒व । आ॒दि॒त्यम् । स्वेन॑ । भा॒ग॒धेये॒नेति॑ भाग - धेये॑न । उपेति॑ । धा॒व॒ति॒ । सः । ए॒व । अ॒स्मा॒त् । तमः॑ । पा॒प्मान᳚म् । अपेति॑ । ह॒न्ति॒ । प्र॒तीची᳚ । अ॒स्मै॒ । व्यु॒च्छन्तीति॑ वि - उच्छन्ती᳚ । वीति॑ । उ॒च्छ॒ति॒ । अपेति॑ । तमः॑ । पा॒प्मान᳚म् । ह॒ते॒ ॥  \newline


\textbf{Krama Paata} \newline

ए॒न॒म् प॒व॒य॒ति॒ । प॒व॒य॒ति॒ परा॑ची । परा॑ची॒ वै । वा ए॒तस्मै᳚ । ए॒तस्मै᳚ व्यु॒च्छन्ती᳚ । व्यु॒च्छन्ती॒ वि । व्यु॒च्छन्तीति॑ वि - उ॒च्छन्ती᳚ । व्यु॑च्छति । उ॒च्छ॒ति॒ तमः॑ । तमः॑ पा॒प्मान᳚म् । पा॒प्मान॒म् प्र । प्र वि॑शति । वि॒श॒ति॒ यस्य॑ । यस्या᳚श्वि॒ने । आ॒श्वि॒ने श॒स्यमा॑ने । श॒स्यमा॑ने॒ सूर्यः॑ । सूर्यो॒ न । नाविः । आ॒विर् भव॑ति । भव॑ति सौ॒र्यम् । सौ॒र्यम् ब॑हुरू॒पम् । ब॒हु॒रू॒पमा । ब॒हु॒रू॒पमिति॑ बहु - रू॒पम् । आ ल॑भेत । ल॒भे॒ता॒मुम् । अ॒मुमे॒व । ए॒वादि॒त्यम् । आ॒दि॒त्यꣳ स्वेन॑ । स्वेन॑ भाग॒धेये॑न । भा॒ग॒धेये॒नोप॑ । भा॒ग॒धेये॒नेति॑ भाग - धेये॑न । उप॑ धावति । धा॒व॒ति॒ सः । स ए॒व । ए॒वास्मा᳚त् । अ॒स्मा॒त् तमः॑ । तमः॑ पा॒प्मान᳚म् । पा॒प्मान॒मप॑ । अप॑ हन्ति । ह॒न्ति॒ प्र॒तीची᳚ । प्र॒तीच्य॑स्मै । अ॒स्मै॒ व्यु॒च्छन्ती᳚ । व्यु॒च्छन्ती॒ वि । व्यु॒च्छन्तीति॑ वि - उ॒च्छन्ती᳚ । व्यु॑च्छति । उ॒च्छ॒त्यप॑ । अप॒ तमः॑ । तमः॑ पा॒प्मान᳚म् । पा॒प्मानꣳ॑ हते । ह॒त॒ इति॑ हते । \newline

\textbf{Jatai Paata} \newline

1. ए॒न॒म् प॒व॒य॒ति॒ प॒व॒य॒त्ये॒न॒ मे॒न॒म् प॒व॒य॒ति॒ । \newline
2. प॒व॒य॒ति॒ परा॑ची॒ परा॑ची पवयति पवयति॒ परा॑ची । \newline
3. परा॑ची॒ वै वै परा॑ची॒ परा॑ची॒ वै । \newline
4. वा ए॒तस्मा॑ ए॒तस्मै॒ वै वा ए॒तस्मै᳚ । \newline
5. ए॒तस्मै᳚ व्यु॒च्छन्ती᳚ व्यु॒च्छ न्त्ये॒तस्मा॑ ए॒तस्मै᳚ व्यु॒च्छन्ती᳚ । \newline
6. व्यु॒च्छन्ती॒ वि वि व्यु॒च्छन्ती᳚ व्यु॒च्छन्ती॒ वि । \newline
7. व्यु॒च्छन्तीति॑ वि - उ॒च्छन्ती᳚ । \newline
8. व्यु॑च्छ त्युच्छति॒ वि व्यु॑च्छति । \newline
9. उ॒च्छ॒ति॒ तम॒स्तम॑ उच्छ त्युच्छति॒ तमः॑ । \newline
10. तमः॑ पा॒प्मान॑म् पा॒प्मान॒म् तम॒स्तमः॑ पा॒प्मान᳚म् । \newline
11. पा॒प्मान॒म् प्र प्र पा॒प्मान॑म् पा॒प्मान॒म् प्र । \newline
12. प्र वि॑शति विशति॒ प्र प्र वि॑शति । \newline
13. वि॒श॒ति॒ यस्य॒ यस्य॑ विशति विशति॒ यस्य॑ । \newline
14. यस्या᳚श्वि॒न आ᳚श्वि॒ने यस्य॒ यस्या᳚श्वि॒ने । \newline
15. आ॒श्वि॒ने श॒स्यमा॑ने श॒स्यमा॑न आश्वि॒न आ᳚श्वि॒ने श॒स्यमा॑ने । \newline
16. श॒स्यमा॑ने॒ सूर्यः॒ सूर्यः॑ श॒स्यमा॑ने श॒स्यमा॑ने॒ सूर्यः॑ । \newline
17. सूर्यो॒ न न सूर्यः॒ सूर्यो॒ न । \newline
18. नावि रा॒विर् न नाविः । \newline
19. आ॒विर् भव॑ति॒ भव॑ त्या॒वि रा॒विर् भव॑ति । \newline
20. भव॑ति सौ॒र्यꣳ सौ॒र्यम् भव॑ति॒ भव॑ति सौ॒र्यम् । \newline
21. सौ॒र्यम् ब॑हुरू॒पम् ब॑हुरू॒पꣳ सौ॒र्यꣳ सौ॒र्यम् ब॑हुरू॒पम् । \newline
22. ब॒हु॒रू॒प मा ब॑हुरू॒पम् ब॑हुरू॒प मा । \newline
23. ब॒हु॒रू॒पमिति॑ बहु - रू॒पम् । \newline
24. आ ल॑भेत लभे॒ता ल॑भेत । \newline
25. ल॒भे॒ता॒मु म॒मुम् ॅल॑भेत लभेता॒मुम् । \newline
26. अ॒मु मे॒वैवामु म॒मु मे॒व । \newline
27. ए॒वादि॒त्य मा॑दि॒त्य मे॒वै वादि॒त्यम् । \newline
28. आ॒दि॒त्यꣳ स्वेन॒ स्वेना॑दि॒त्य मा॑दि॒त्यꣳ स्वेन॑ । \newline
29. स्वेन॑ भाग॒धेये॑न भाग॒धेये॑न॒ स्वेन॒ स्वेन॑ भाग॒धेये॑न । \newline
30. भा॒ग॒धेये॒नोपोप॑ भाग॒धेये॑न भाग॒धेये॒नोप॑ । \newline
31. भा॒ग॒धेये॒नेति॑ भाग - धेये॑न । \newline
32. उप॑ धावति धाव॒ त्युपोप॑ धावति । \newline
33. धा॒व॒ति॒ स स धा॑वति धावति॒ सः । \newline
34. स ए॒वैव स स ए॒व । \newline
35. ए॒वास्मा॑ दस्मा दे॒वैवास्मा᳚त् । \newline
36. अ॒स्मा॒त् तम॒स्तमो᳚ ऽस्मा दस्मा॒त् तमः॑ । \newline
37. तमः॑ पा॒प्मान॑म् पा॒प्मान॒म् तम॒स्तमः॑ पा॒प्मान᳚म् । \newline
38. पा॒प्मान॒ मपाप॑ पा॒प्मान॑म् पा॒प्मान॒ मप॑ । \newline
39. अप॑ हन्ति ह॒न्त्यपाप॑ हन्ति । \newline
40. ह॒न्ति॒ प्र॒तीची᳚ प्र॒तीची॑ हन्ति हन्ति प्र॒तीची᳚ । \newline
41. प्र॒तीच्य॑स्मा अस्मै प्र॒तीची᳚ प्र॒तीच्य॑स्मै । \newline
42. अ॒स्मै॒ व्यु॒च्छन्ती᳚ व्यु॒च्छ न्त्य॑स्मा अस्मै व्यु॒च्छन्ती᳚ । \newline
43. व्यु॒च्छन्ती॒ वि वि व्यु॒च्छन्ती᳚ व्यु॒च्छन्ती॒ वि । \newline
44. व्यु॒च्छन्तीति॑ वि - उ॒च्छन्ती᳚ । \newline
45. व्यु॑च्छ त्युच्छति॒ वि व्यु॑च्छति । \newline
46. उ॒च्छ॒ त्यपापो᳚च्छ त्युच्छ॒ त्यप॑ । \newline
47. अप॒ तम॒ स्तमो ऽपाप॒ तमः॑ । \newline
48. तमः॑ पा॒प्मान॑म् पा॒प्मान॒म् तम॒स्तमः॑ पा॒प्मान᳚म् । \newline
49. पा॒प्मानꣳ॑ हते हते पा॒प्मान॑म् पा॒प्मानꣳ॑ हते । \newline
50. ह॒त॒ इति॑ हते । \newline

\textbf{Ghana Paata } \newline

1. ए॒न॒म् प॒व॒य॒ति॒ प॒व॒य॒त्ये॒न॒ मे॒न॒म् प॒व॒य॒ति॒ परा॑ची॒ परा॑ची पवयत्येन मेनम् पवयति॒ परा॑ची । \newline
2. प॒व॒य॒ति॒ परा॑ची॒ परा॑ची पवयति पवयति॒ परा॑ची॒ वै वै परा॑ची पवयति पवयति॒ परा॑ची॒ वै । \newline
3. परा॑ची॒ वै वै परा॑ची॒ परा॑ची॒ वा ए॒तस्मा॑ ए॒तस्मै॒ वै परा॑ची॒ परा॑ची॒ वा ए॒तस्मै᳚ । \newline
4. वा ए॒तस्मा॑ ए॒तस्मै॒ वै वा ए॒तस्मै᳚ व्यु॒च्छन्ती᳚ व्यु॒च्छ न्त्ये॒तस्मै॒ वै वा ए॒तस्मै᳚ व्यु॒च्छन्ती᳚ । \newline
5. ए॒तस्मै᳚ व्यु॒च्छन्ती᳚ व्यु॒च्छ न्त्ये॒तस्मा॑ ए॒तस्मै᳚ व्यु॒च्छन्ती॒ वि वि व्यु॒च्छ न्त्ये॒तस्मा॑ ए॒तस्मै᳚ व्यु॒च्छन्ती॒ वि । \newline
6. व्यु॒च्छन्ती॒ वि वि व्यु॒च्छन्ती᳚ व्यु॒च्छन्ती॒ व्यु॑च्छ त्युच्छति॒ वि व्यु॒च्छन्ती᳚ व्यु॒च्छन्ती॒ व्यु॑च्छति । \newline
7. व्यु॒च्छन्तीति॑ वि - उ॒च्छन्ती᳚ । \newline
8. व्यु॑च्छ त्युच्छति॒ वि व्यु॑च्छति॒ तम॒ स्तम॑ उच्छति॒ वि व्यु॑च्छति॒ तमः॑ । \newline
9. उ॒च्छ॒ति॒ तम॒ स्तम॑ उच्छ त्युच्छति॒ तमः॑ पा॒प्मान॑म् पा॒प्मान॒म् तम॑ उच्छ त्युच्छति॒ तमः॑ पा॒प्मान᳚म् । \newline
10. तमः॑ पा॒प्मान॑म् पा॒प्मान॒म् तम॒ स्तमः॑ पा॒प्मान॒म् प्र प्र पा॒प्मान॒म् तम॒ स्तमः॑ पा॒प्मान॒म् प्र । \newline
11. पा॒प्मान॒म् प्र प्र पा॒प्मान॑म् पा॒प्मान॒म् प्र वि॑शति विशति॒ प्र पा॒प्मान॑म् पा॒प्मान॒म् प्र वि॑शति । \newline
12. प्र वि॑शति विशति॒ प्र प्र वि॑शति॒ यस्य॒ यस्य॑ विशति॒ प्र प्र वि॑शति॒ यस्य॑ । \newline
13. वि॒श॒ति॒ यस्य॒ यस्य॑ विशति विशति॒ यस्या᳚श्वि॒न आ᳚श्वि॒ने यस्य॑ विशति विशति॒ यस्या᳚श्वि॒ने । \newline
14. यस्या᳚श्वि॒न आ᳚श्वि॒ने यस्य॒ यस्या᳚श्वि॒ने श॒स्यमा॑ने श॒स्यमा॑न आश्वि॒ने यस्य॒ यस्या᳚श्वि॒ने श॒स्यमा॑ने । \newline
15. आ॒श्वि॒ने श॒स्यमा॑ने श॒स्यमा॑न आश्वि॒न आ᳚श्वि॒ने श॒स्यमा॑ने॒ सूर्यः॒ सूर्यः॑ श॒स्यमा॑न आश्वि॒न आ᳚श्वि॒ने श॒स्यमा॑ने॒ सूर्यः॑ । \newline
16. श॒स्यमा॑ने॒ सूर्यः॒ सूर्यः॑ श॒स्यमा॑ने श॒स्यमा॑ने॒ सूर्यो॒ न न सूर्यः॑ श॒स्यमा॑ने श॒स्यमा॑ने॒ सूर्यो॒ न । \newline
17. सूर्यो॒ न न सूर्यः॒ सूर्यो॒ नावि रा॒विर् न सूर्यः॒ सूर्यो॒ नाविः । \newline
18. नावि रा॒विर् न नाविर् भव॑ति॒ भव॑ त्या॒विर् न नाविर् भव॑ति । \newline
19. आ॒विर् भव॑ति॒ भव॑ त्या॒वि रा॒विर् भव॑ति सौ॒र्यꣳ सौ॒र्यम् भव॑ त्या॒वि रा॒विर् भव॑ति सौ॒र्यम् । \newline
20. भव॑ति सौ॒र्यꣳ सौ॒र्यम् भव॑ति॒ भव॑ति सौ॒र्यम् ब॑हुरू॒पम् ब॑हुरू॒पꣳ सौ॒र्यम् भव॑ति॒ भव॑ति सौ॒र्यम् ब॑हुरू॒पम् । \newline
21. सौ॒र्यम् ब॑हुरू॒पम् ब॑हुरू॒पꣳ सौ॒र्यꣳ सौ॒र्यम् ब॑हुरू॒प मा ब॑हुरू॒पꣳ सौ॒र्यꣳ सौ॒र्यम् ब॑हुरू॒प मा । \newline
22. ब॒हु॒रू॒प मा ब॑हुरू॒पम् ब॑हुरू॒प मा ल॑भेत लभे॒ता ब॑हुरू॒पम् ब॑हुरू॒प मा ल॑भेत । \newline
23. ब॒हु॒रू॒पमिति॑ बहु - रू॒पम् । \newline
24. आ ल॑भेत लभे॒ता ल॑भेता॒मु म॒मुम् ॅल॑भे॒ता ल॑भेता॒मुम् । \newline
25. ल॒भे॒ता॒मु म॒मुम् ॅल॑भेत लभेता॒मु मे॒वैवामुम् ॅल॑भेत लभेता॒मु मे॒व । \newline
26. अ॒मु मे॒वैवामु म॒मु मे॒वादि॒त्य मा॑दि॒त्य मे॒वामु म॒मु मे॒वादि॒त्यम् । \newline
27. ए॒वादि॒त्य मा॑दि॒त्य मे॒वैवा दि॒त्यꣳ स्वेन॒ स्वेना॑दि॒त्य मे॒वैवा दि॒त्यꣳ स्वेन॑ । \newline
28. आ॒दि॒त्यꣳ स्वेन॒ स्वेना॑दि॒त्य मा॑दि॒त्यꣳ स्वेन॑ भाग॒धेये॑न भाग॒धेये॑न॒ स्वेना॑दि॒त्य मा॑दि॒त्यꣳ स्वेन॑ भाग॒धेये॑न । \newline
29. स्वेन॑ भाग॒धेये॑न भाग॒धेये॑न॒ स्वेन॒ स्वेन॑ भाग॒धेये॒नो पोप॑ भाग॒धेये॑न॒ स्वेन॒ स्वेन॑ भाग॒धेये॒नोप॑ । \newline
30. भा॒ग॒धेये॒नो पोप॑ भाग॒धेये॑न भाग॒धेये॒नोप॑ धावति धाव॒त्युप॑ भाग॒धेये॑न भाग॒धेये॒नोप॑ धावति । \newline
31. भा॒ग॒धेये॒नेति॑ भाग - धेये॑न । \newline
32. उप॑ धावति धाव॒ त्युपोप॑ धावति॒ स स धा॑व॒ त्युपोप॑ धावति॒ सः । \newline
33. धा॒व॒ति॒ स स धा॑वति धावति॒ स ए॒वैव स धा॑वति धावति॒ स ए॒व । \newline
34. स ए॒वैव स स ए॒वास्मा॑ दस्मा दे॒व स स ए॒वास्मा᳚त् । \newline
35. ए॒वास्मा॑ दस्मा दे॒वै वास्मा॒त् तम॒ स्तमो᳚ ऽस्मा दे॒वै वास्मा॒त् तमः॑ । \newline
36. अ॒स्मा॒त् तम॒ स्तमो᳚ ऽस्मा दस्मा॒त् तमः॑ पा॒प्मान॑म् पा॒प्मान॒म् तमो᳚ ऽस्मा दस्मा॒त् तमः॑ पा॒प्मान᳚म् । \newline
37. तमः॑ पा॒प्मान॑म् पा॒प्मान॒म् तम॒ स्तमः॑ पा॒प्मान॒ मपाप॑ पा॒प्मान॒म् तम॒ स्तमः॑ पा॒प्मान॒ मप॑ । \newline
38. पा॒प्मान॒ मपाप॑ पा॒प्मान॑म् पा॒प्मान॒ मप॑ हन्ति ह॒न्त्यप॑ पा॒प्मान॑म् पा॒प्मान॒ मप॑ हन्ति । \newline
39. अप॑ हन्ति ह॒न्त्यपाप॑ हन्ति प्र॒तीची᳚ प्र॒तीची॑ ह॒न्त्यपाप॑ हन्ति प्र॒तीची᳚ । \newline
40. ह॒न्ति॒ प्र॒तीची᳚ प्र॒तीची॑ हन्ति हन्ति प्र॒तीच्य॑स्मा अस्मै प्र॒तीची॑ हन्ति हन्ति प्र॒तीच्य॑स्मै । \newline
41. प्र॒तीच्य॑स्मा अस्मै प्र॒तीची᳚ प्र॒तीच्य॑स्मै व्यु॒च्छन्ती᳚ व्यु॒च्छ न्त्य॑स्मै प्र॒तीची᳚ प्र॒तीच्य॑स्मै व्यु॒च्छन्ती᳚ । \newline
42. अ॒स्मै॒ व्यु॒च्छन्ती᳚ व्यु॒च्छ न्त्य॑स्मा अस्मै व्यु॒च्छन्ती॒ वि वि व्यु॒च्छ न्त्य॑स्मा अस्मै व्यु॒च्छन्ती॒ वि । \newline
43. व्यु॒च्छन्ती॒ वि वि व्यु॒च्छन्ती᳚ व्यु॒च्छन्ती॒ व्यु॑च्छ त्युच्छति॒ वि व्यु॒च्छन्ती᳚ व्यु॒च्छन्ती॒ व्यु॑च्छति । \newline
44. व्यु॒च्छन्तीति॑ वि - उ॒च्छन्ती᳚ । \newline
45. व्यु॑च्छ त्युच्छति॒ वि व्यु॑च्छ॒ त्यपापो᳚च्छति॒ वि व्यु॑च्छ॒ त्यप॑ । \newline
46. उ॒च्छ॒ त्यपापो᳚च्छ त्युच्छ॒ त्यप॒ तम॒ स्तमो ऽपो᳚च्छ त्युच्छ॒ त्यप॒ तमः॑ । \newline
47. अप॒ तम॒ स्तमो ऽपाप॒ तमः॑ पा॒प्मान॑म् पा॒प्मान॒म् तमो ऽपाप॒ तमः॑ पा॒प्मान᳚म् । \newline
48. तमः॑ पा॒प्मान॑म् पा॒प्मान॒म् तम॒ स्तमः॑ पा॒प्मानꣳ॑ हते हते पा॒प्मान॒म् तम॒ स्तमः॑ पा॒प्मानꣳ॑ हते । \newline
49. पा॒प्मानꣳ॑ हते हते पा॒प्मान॑म् पा॒प्मानꣳ॑ हते । \newline
50. ह॒त॒ इति॑ हते । \newline
\pagebreak
\markright{ TS 2.1.11.1  \hfill https://www.vedavms.in \hfill}

\section{ TS 2.1.11.1 }

\textbf{TS 2.1.11.1 } \newline
\textbf{Samhita Paata} \newline

इन्द्रं॑ ॅवो वि॒श्वत॒स्परी >1, न्द्रं॒ नरो॒ >2, मरु॑तो॒ यद्ध॑ वो दि॒वो >3, या वः॒ शर्म॑ >4 ॥भरे॒ष्विन्द्रꣳ॑ सु॒हवꣳ॑ हवामहे ऽꣳहो॒मुचꣳ॑ सु॒कृतं॒ दैव्यं॒ जनं᳚ । अ॒ग्निं मि॒त्रं ॅवरु॑णꣳ सा॒तये॒ भगं॒ द्यावा॑पृथि॒वी म॒रुतः॑ स्व॒स्तये᳚ ॥ म॒मत्तु॑ नः॒ परि॑ज्मा वस॒र्॒.हा म॒मत्तु॒ वातो॑ अ॒पां ॅवृष॑ण्वान्न् । शि॒शी॒तमि॑न्द्रापर्वता यु॒वं न॒स्तन्नो॒ विश्वे॑ वरिवस्यन्तु दे॒वाः ॥ प्रि॒या वो॒ नाम॑ - [  ] \newline

\textbf{Pada Paata} \newline

इन्द्र᳚म् । वः॒ । वि॒श्वतः॑ । परीति॑ । इन्द्र᳚म् । नरः॑ । मरु॑तः । यत् । ह॒ । वः॒ । दि॒वः । या । वः॒ । शर्म॑ ॥ भरे॑षु । इन्द्र᳚म् । सु॒हव॒मिति॑ सु - हव᳚म् । ह॒वा॒म॒हे॒ । अꣳ॒॒हो॒मुच॒मित्यꣳ॑हः - मुच᳚म् । सु॒कृत॒मिति॑ सु - कृत᳚म् । दैव्य᳚म् । जन᳚म् ॥ अ॒ग्निम् । मि॒त्रम् । वरु॑णम् । सा॒तये᳚ । भग᳚म् । द्यावा॑पृथि॒वी इति॒ द्यावा᳚ - पृ॒थि॒वी । म॒रुतः॑ । स्व॒स्तये᳚ ॥ म॒मत्तु॑ । नः॒ । परि॒ज्मेति॒ परि॑-ज्मा॒ । व॒स॒र्॒.हा । म॒मत्तु॑ । वातः॑ । अ॒पाम् । वृष॑ण्वा॒निति॒ वृषण॑ - वा॒न् ॥ शि॒शी॒तम् । इ॒न्द्रा॒प॒र्व॒तेती᳚न्द्रा - प॒र्व॒ता॒ । यु॒वम् । नः॒ । तत् । नः॒ । विश्वे᳚ । व॒रि॒व॒स्य॒न्तु॒ । दे॒वाः ॥ प्रि॒या । वः॒ । नाम॑ ।  \newline


\textbf{Krama Paata} \newline

इन्द्रं॑ ॅवः । वो॒ वि॒श्वत॑ ः । वि॒श्वत॒स्परि॑ । परीन्द्र᳚म् । इन्द्र॒म् नरः॑ । नरो॒ मरु॑तः । मरु॑तो॒ यत् । यद्ध॑ । ह॒ वः॒ । वो॒ दि॒वः । दि॒वो या । या वः॑ । वः॒ शर्म॑ । शर्मेति॒ शर्म॑ ॥ भरे॒ष्विन्द्र᳚म् । इन्द्रꣳ॑ सु॒हव᳚म् । सु॒हवꣳ॑ हवामहे । सु॒हव॒मिति॑ सु - हव᳚म् । ह॒वा॒म॒हे॒ ऽꣳ॒हो॒मुच᳚म् । अꣳ॒॒हो॒मुचꣳ॑ सु॒कृत᳚म् । अꣳ॒॒हो॒मुच॒मित्यꣳ॑हः - मुच᳚म् । सु॒कृत॒म् दैव्य᳚म् । सु॒कृत॒मिति॑ सु - कृत᳚म् । दैव्य॒म् जन᳚म् । जन॒मिति॒ जन᳚म् ॥ अ॒ग्निम् मि॒त्रम् । मि॒त्रं ॅवरु॑णम् । वरु॑णꣳ सा॒तये᳚ । सा॒तये॒ भग᳚म् । भग॒म् द्यावा॑पृथि॒वी । द्यावा॑पृथि॒वी म॒रुतः॑ । द्यावा॑पृथि॒वी इति॒ द्यावा᳚ - पृ॒थि॒वी । म॒रुतः॑ स्व॒स्तये᳚ । स्व॒स्तय॒ इति॑ स्व॒स्तये᳚ ॥ म॒मत्तु॑ नः । नः॒ परि॑ज्मा । परि॑ज्मा वस॒र्॒.हा । परि॒ज्मेति॒ परि॑ - ज्मा॒ । व॒स॒र्॒.हा म॒मत्तु॑ । म॒मत्तु॒ वातः॑ । वातो॑ अ॒पाम् । अ॒पां ॅवृष॑ण्वान् । वृष॑ण्वा॒निति॒ वृषण्ण्॑ - वा॒न्॒ ॥ शि॒शी॒तमि॑न्द्रापर्वता । इ॒न्द्रा॒प॒र्व॒ता॒ यु॒वम् । इ॒न्द्रा॒प॒र्व॒तेती᳚न्द्रा - प॒र्व॒ता॒ । यु॒वम् नः॑ । न॒,स्तत् । तन्नः॑ । नो॒ विश्वे᳚ । विश्वे॑ वरिवस्यन्तु । व॒रि॒व॒स्य॒न्तु॒ दे॒वाः । दे॒वा इति॑ दे॒वाः ॥ प्रि॒या वः॑ । वो॒ नाम॑ । नाम॑ हुवे \newline

\textbf{Jatai Paata} \newline

1. इन्द्रं॑ ॅवो व॒ इन्द्र॒ मिन्द्रं॑ ॅवः । \newline
2. वो॒ वि॒श्वतो॑ वि॒श्वतो॑ वो वो वि॒श्वतः॑ । \newline
3. वि॒श्वत॒ स्परि॒ परि॑ वि॒श्वतो॑ वि॒श्वत॒ स्परि॑ । \newline
4. परीन्द्र॒ मिन्द्र॒म् परि॒ परीन्द्र᳚म् । \newline
5. इन्द्र॒म् नरो॒ नर॒ इन्द्र॒ मिन्द्र॒म् नरः॑ । \newline
6. नरो॒ मरु॑तो॒ मरु॑तो॒ नरो॒ नरो॒ मरु॑तः । \newline
7. मरु॑तो॒ यद् यन् मरु॑तो॒ मरु॑तो॒ यत् । \newline
8. यद्ध॑ ह॒ यद् यद्ध॑ । \newline
9. ह॒ वो॒ वो॒ ह॒ ह॒ वः॒ । \newline
10. वो॒ दि॒वो दि॒वो वो॑ वो दि॒वः । \newline
11. दि॒वो या या दि॒वो दि॒वो या । \newline
12. या वो॑ वो॒ या या वः॑ । \newline
13. वः॒ शर्म॒ शर्म॑ वो वः॒ शर्म॑ । \newline
14. शर्मेति॒ शर्म॑ । \newline
15. भरे॒ष्विन्द्र॒ मिन्द्र॒म् भरे॑षु॒ भरे॒ष्विन्द्र᳚म् । \newline
16. इन्द्रꣳ॑ सु॒हवꣳ॑ सु॒हव॒ मिन्द्र॒ मिन्द्रꣳ॑ सु॒हव᳚म् । \newline
17. सु॒हवꣳ॑ हवामहे हवामहे सु॒हवꣳ॑ सु॒हवꣳ॑ हवामहे । \newline
18. सु॒हव॒मिति॑ सु - हव᳚म् । \newline
19. ह॒वा॒म॒हे॒ ऽꣳ॒हो॒मुच॑ मꣳहो॒मुचꣳ॑ हवामहे हवामहे ऽꣳहो॒मुच᳚म् । \newline
20. अꣳ॒॒हो॒मुचꣳ॑ सु॒कृतꣳ॑ सु॒कृत॑ मꣳहो॒मुच॑ मꣳहो॒मुचꣳ॑ सु॒कृत᳚म् । \newline
21. अꣳ॒॒हो॒मुच॒मित्यꣳ॑हः - मुच᳚म् । \newline
22. सु॒कृत॒म् दैव्य॒म् दैव्यꣳ॑ सु॒कृतꣳ॑ सु॒कृत॒म् दैव्य᳚म् । \newline
23. सु॒कृत॒मिति॑ सु - कृत᳚म् । \newline
24. दैव्य॒म् जन॒म् जन॒म् दैव्य॒म् दैव्य॒म् जन᳚म् । \newline
25. जन॒मिति॒ जन᳚म् । \newline
26. अ॒ग्निम् मि॒त्रम् मि॒त्र म॒ग्नि म॒ग्निम् मि॒त्रम् । \newline
27. मि॒त्रं ॅवरु॑णं॒ ॅवरु॑णम् मि॒त्रम् मि॒त्रं ॅवरु॑णम् । \newline
28. वरु॑णꣳ सा॒तये॑ सा॒तये॒ वरु॑णं॒ ॅवरु॑णꣳ सा॒तये᳚ । \newline
29. सा॒तये॒ भग॒म् भगꣳ॑ सा॒तये॑ सा॒तये॒ भग᳚म् । \newline
30. भग॒म् द्यावा॑पृथि॒वी द्यावा॑पृथि॒वी भग॒म् भग॒म् द्यावा॑पृथि॒वी । \newline
31. द्यावा॑पृथि॒वी म॒रुतो॑ म॒रुतो॒ द्यावा॑पृथि॒वी द्यावा॑पृथि॒वी म॒रुतः॑ । \newline
32. द्यावा॑पृथि॒वी इति॒ द्यावा᳚ - पृ॒थि॒वी । \newline
33. म॒रुतः॑ स्व॒स्तये᳚ स्व॒स्तये॑ म॒रुतो॑ म॒रुतः॑ स्व॒स्तये᳚ । \newline
34. स्व॒स्तय॒ इति॑ स्व॒स्तये᳚ । \newline
35. म॒मत्तु॑ नो नो म॒मत्तु॑ म॒मत्तु॑ नः । \newline
36. नः॒ परि॑ज्मा॒ परि॑ज्मा नो नः॒ परि॑ज्मा । \newline
37. परि॑ज्मा वस॒र्॒.हा व॑स॒र्॒.हा परि॑ज्मा॒ परि॑ज्मा वस॒र्॒.हा । \newline
38. परि॒ज्मेति॒ परि॑ - ज्मा॒ । \newline
39. व॒स॒र्॒.हा म॒मत्तु॑ म॒मत्तु॑ वस॒र्॒.हा व॑स॒र्॒.हा म॒मत्तु॑ । \newline
40. म॒मत्तु॒ वातो॒ वातो॑ म॒मत्तु॑ म॒मत्तु॒ वातः॑ । \newline
41. वातो॑ अ॒पा म॒पां ॅवातो॒ वातो॑ अ॒पाम् । \newline
42. अ॒पां ॅवृष॑ण्वा॒न् वृष॑ण्वा न॒पा म॒पां ॅवृष॑ण्वान् । \newline
43. वृष॑ण्वा॒निति॒ वृषण्॑ - वा॒न् । \newline
44. शि॒शी॒त मि॑न्द्रापर्वते न्द्रापर्वता शिशी॒तꣳ शि॑शी॒त मि॑न्द्रापर्वता । \newline
45. इ॒न्द्रा॒प॒र्व॒ता॒ यु॒वं ॅयु॒व मि॑न्द्रापर्वते न्द्रापर्वता यु॒वम् । \newline
46. इ॒न्द्रा॒प॒र्व॒तेती᳚न्द्रा - प॒र्व॒ता॒ । \newline
47. यु॒वम् नो॑ नो यु॒वं ॅयु॒वम् नः॑ । \newline
48. न॒ स्तत् तन् नो॑ न॒ स्तत् । \newline
49. तन् नो॑ न॒ स्तत् तन् नः॑ । \newline
50. नो॒ विश्वे॒ विश्वे॑ नो नो॒ विश्वे᳚ । \newline
51. विश्वे॑ वरिवस्यन्तु वरिवस्यन्तु॒ विश्वे॒ विश्वे॑ वरिवस्यन्तु । \newline
52. व॒रि॒व॒स्य॒न्तु॒ दे॒वा दे॒वा व॑रिवस्यन्तु वरिवस्यन्तु दे॒वाः । \newline
53. दे॒वा इति॑ दे॒वाः । \newline
54. प्रि॒या वो॑ वः प्रि॒या प्रि॒या वः॑ । \newline
55. वो॒ नाम॒ नाम॑ वो वो॒ नाम॑ । \newline
56. नाम॑ हुवे हुवे॒ नाम॒ नाम॑ हुवे । \newline

\textbf{Ghana Paata } \newline

1. इन्द्रं॑ ॅवो व॒ इन्द्र॒ मिन्द्रं॑ ॅवो वि॒श्वतो॑ वि॒श्वतो॑ व॒ इन्द्र॒ मिन्द्रं॑ ॅवो वि॒श्वतः॑ । \newline
2. वो॒ वि॒श्वतो॑ वि॒श्वतो॑ वो वो वि॒श्वत॒ स्परि॒ परि॑ वि॒श्वतो॑ वो वो वि॒श्वत॒ स्परि॑ । \newline
3. वि॒श्वत॒ स्परि॒ परि॑ वि॒श्वतो॑ वि॒श्वत॒ स्परीन्द्र॒ मिन्द्र॒म् परि॑ वि॒श्वतो॑ वि॒श्वत॒ स्परीन्द्र᳚म् । \newline
4. परीन्द्र॒ मिन्द्र॒म् परि॒ परीन्द्र॒न् नरो॒ नर॒ इन्द्र॒म् परि॒ परीन्द्र॒न् नरः॑ । \newline
5. इन्द्र॒न्नरो॒ नर॒ इन्द्र॒ मिन्द्र॒न् नरो॒ मरु॑तो॒ मरु॑तो॒ नर॒ इन्द्र॒ मिन्द्र॒न् नरो॒ मरु॑तः । \newline
6. नरो॒ मरु॑तो॒ मरु॑तो॒ नरो॒ नरो॒ मरु॑तो॒ यद् यन् मरु॑तो॒ नरो॒ नरो॒ मरु॑तो॒ यत् । \newline
7. मरु॑तो॒ यद् यन् मरु॑तो॒ मरु॑तो॒ यद्ध॑ ह॒ यन् मरु॑तो॒ मरु॑तो॒ यद्ध॑ । \newline
8. यद्ध॑ ह॒ यद् यद्ध॑ वो वो ह॒ यद् यद्ध॑ वः । \newline
9. ह॒ वो॒ वो॒ ह॒ ह॒ वो॒ दि॒वो दि॒वो वो॑ ह ह वो दि॒वः । \newline
10. वो॒ दि॒वो दि॒वो वो॑ वो दि॒वो या या दि॒वो वो॑ वो दि॒वो या । \newline
11. दि॒वो या या दि॒वो दि॒वो या वो॑ वो॒ या दि॒वो दि॒वो या वः॑ । \newline
12. या वो॑ वो॒ या या वः॒ शर्म॒ शर्म॑ वो॒ या या वः॒ शर्म॑ । \newline
13. वः॒ शर्म॒ शर्म॑ वो वः॒ शर्म॑ । \newline
14. शर्मेति॒ शर्म॑ । \newline
15. भरे॒ष्विन्द्र॒ मिन्द्र॒म् भरे॑षु॒ भरे॒ ष्विन्द्रꣳ॑ सु॒हवꣳ॑ सु॒हव॒ मिन्द्र॒म् भरे॑षु॒ भरे॒ ष्विन्द्रꣳ॑ सु॒हव᳚म् । \newline
16. इन्द्रꣳ॑ सु॒हवꣳ॑ सु॒हव॒ मिन्द्र॒ मिन्द्रꣳ॑ सु॒हवꣳ॑ हवामहे हवामहे सु॒हव॒ मिन्द्र॒ मिन्द्रꣳ॑ सु॒हवꣳ॑ हवामहे । \newline
17. सु॒हवꣳ॑ हवामहे हवामहे सु॒हवꣳ॑ सु॒हवꣳ॑ हवामहे ऽꣳहो॒मुच॑ मꣳहो॒मुचꣳ॑ हवामहे सु॒हवꣳ॑ सु॒हवꣳ॑ हवामहे ऽꣳहो॒मुच᳚म् । \newline
18. सु॒हव॒मिति॑ सु - हव᳚म् । \newline
19. ह॒वा॒म॒हे॒ ऽꣳ॒हो॒मुच॑ मꣳहो॒मुचꣳ॑ हवामहे हवामहे ऽꣳहो॒मुचꣳ॑ सु॒कृतꣳ॑ सु॒कृत॑ मꣳहो॒मुचꣳ॑ हवामहे हवामहे ऽꣳहो॒मुचꣳ॑ सु॒कृत᳚म् । \newline
20. अꣳ॒॒हो॒मुचꣳ॑ सु॒कृतꣳ॑ सु॒कृत॑ मꣳहो॒मुच॑ मꣳहो॒मुचꣳ॑ सु॒कृत॒म् दैव्य॒म् दैव्यꣳ॑ सु॒कृत॑ मꣳहो॒मुच॑ मꣳहो॒मुचꣳ॑ सु॒कृत॒म् दैव्य᳚म् । \newline
21. अꣳ॒॒हो॒मुच॒मित्यꣳ॑हः - मुच᳚म् । \newline
22. सु॒कृत॒म् दैव्य॒म् दैव्यꣳ॑ सु॒कृतꣳ॑ सु॒कृत॒म् दैव्य॒म् जन॒म् जन॒म् दैव्यꣳ॑ सु॒कृतꣳ॑ सु॒कृत॒म् दैव्य॒म् जन᳚म् । \newline
23. सु॒कृत॒मिति॑ सु - कृत᳚म् । \newline
24. दैव्य॒म् जन॒म् जन॒म् दैव्य॒म् दैव्य॒म् जन᳚म् । \newline
25. जन॒मिति॒ जन᳚म् । \newline
26. अ॒ग्निम् मि॒त्रम् मि॒त्र म॒ग्नि म॒ग्निम् मि॒त्रं ॅवरु॑णं॒ ॅवरु॑णम् मि॒त्र म॒ग्नि म॒ग्निम् मि॒त्रं ॅवरु॑णम् । \newline
27. मि॒त्रं ॅवरु॑णं॒ ॅवरु॑णम् मि॒त्रम् मि॒त्रं ॅवरु॑णꣳ सा॒तये॑ सा॒तये॒ वरु॑णम् मि॒त्रम् मि॒त्रं ॅवरु॑णꣳ सा॒तये᳚ । \newline
28. वरु॑णꣳ सा॒तये॑ सा॒तये॒ वरु॑णं॒ ॅवरु॑णꣳ सा॒तये॒ भग॒म् भगꣳ॑ सा॒तये॒ वरु॑णं॒ ॅवरु॑णꣳ सा॒तये॒ भग᳚म् । \newline
29. सा॒तये॒ भग॒म् भगꣳ॑ सा॒तये॑ सा॒तये॒ भग॒म् द्यावा॑पृथि॒वी द्यावा॑पृथि॒वी भगꣳ॑ सा॒तये॑ सा॒तये॒ भग॒म् द्यावा॑पृथि॒वी । \newline
30. भग॒म् द्यावा॑पृथि॒वी द्यावा॑पृथि॒वी भग॒म् भग॒म् द्यावा॑पृथि॒वी म॒रुतो॑ म॒रुतो॒ द्यावा॑पृथि॒वी भग॒म् भग॒म् द्यावा॑पृथि॒वी म॒रुतः॑ । \newline
31. द्यावा॑पृथि॒वी म॒रुतो॑ म॒रुतो॒ द्यावा॑पृथि॒वी द्यावा॑पृथि॒वी म॒रुतः॑ स्व॒स्तये᳚ स्व॒स्तये॑ म॒रुतो॒ द्यावा॑पृथि॒वी द्यावा॑पृथि॒वी म॒रुतः॑ स्व॒स्तये᳚ । \newline
32. द्यावा॑पृथि॒वी इति॒ द्यावा᳚ - पृ॒थि॒वी । \newline
33. म॒रुतः॑ स्व॒स्तये᳚ स्व॒स्तये॑ म॒रुतो॑ म॒रुतः॑ स्व॒स्तये᳚ । \newline
34. स्व॒स्तय॒ इति॑ स्व॒स्तये᳚ । \newline
35. म॒मत्तु॑ नो नो म॒मत्तु॑ म॒मत्तु॑ नः॒ परि॑ज्मा॒ परि॑ज्मा नो म॒मत्तु॑ म॒मत्तु॑ नः॒ परि॑ज्मा । \newline
36. नः॒ परि॑ज्मा॒ परि॑ज्मा नो नः॒ परि॑ज्मा वस॒र्॒.हा व॑स॒र्॒.हा परि॑ज्मा नो नः॒ परि॑ज्मा वस॒र्॒.हा । \newline
37. परि॑ज्मा वस॒र्॒.हा व॑स॒र्॒.हा परि॑ज्मा॒ परि॑ज्मा वस॒र्॒.हा म॒मत्तु॑ म॒मत्तु॑ वस॒र्॒.हा परि॑ज्मा॒ परि॑ज्मा वस॒र्॒.हा म॒मत्तु॑ । \newline
38. परि॒ज्मेति॒ परि॑ - ज्मा॒ । \newline
39. व॒स॒र्॒.हा म॒मत्तु॑ म॒मत्तु॑ वस॒र्॒.हा व॑स॒र्॒.हा म॒मत्तु॒ वातो॒ वातो॑ म॒मत्तु॑ वस॒र्॒.हा व॑स॒र्॒.हा म॒मत्तु॒ वातः॑ । \newline
40. म॒मत्तु॒ वातो॒ वातो॑ म॒मत्तु॑ म॒मत्तु॒ वातो॑ अ॒पा म॒पां ॅवातो॑ म॒मत्तु॑ म॒मत्तु॒ वातो॑ अ॒पाम् । \newline
41. वातो॑ अ॒पा म॒पां ॅवातो॒ वातो॑ अ॒पां ॅवृष॑ण्वा॒न् वृष॑ण्वा न॒पां ॅवातो॒ वातो॑ अ॒पां ॅवृष॑ण्वान् । \newline
42. अ॒पां ॅवृष॑ण्वा॒न् वृष॑ण्वा न॒पा म॒पां ॅवृष॑ण्वान् । \newline
43. वृष॑ण्वा॒निति॒ वृषण्॑ - वा॒न् । \newline
44. शि॒शी॒त मि॑न्द्रापर्वते न्द्रापर्वता शिशी॒तꣳ शि॑शी॒त मि॑न्द्रापर्वता यु॒वं ॅयु॒व मि॑न्द्रापर्वता शिशी॒तꣳ शि॑शी॒त मि॑न्द्रापर्वता यु॒वम् । \newline
45. इ॒न्द्रा॒प॒र्व॒ता॒ यु॒वं ॅयु॒व मि॑न्द्रापर्व तेन्द्रापर्वता यु॒वन्नो॑ नो यु॒व मि॑न्द्रापर्व तेन्द्रापर्वता यु॒वन्नः॑ । \newline
46. इ॒न्द्रा॒प॒र्व॒तेती᳚न्द्रा - प॒र्व॒ता॒ । \newline
47. यु॒वन्नो॑ नो यु॒वं ॅयु॒वन्न॒ स्तत् तन् नो॑ यु॒वं ॅयु॒वन्न॒ स्तत् । \newline
48. न॒ स्तत् तन् नो॑ न॒ स्तन् नो॑ न॒ स्तन् नो॑ न॒ स्तन् नः॑ । \newline
49. तन् नो॑ न॒ स्तत् तन् नो॒ विश्वे॒ विश्वे॑ न॒ स्तत् तन् नो॒ विश्वे᳚ । \newline
50. नो॒ विश्वे॒ विश्वे॑ नो नो॒ विश्वे॑ वरिवस्यन्तु वरिवस्यन्तु॒ विश्वे॑ नो नो॒ विश्वे॑ वरिवस्यन्तु । \newline
51. विश्वे॑ वरिवस्यन्तु वरिवस्यन्तु॒ विश्वे॒ विश्वे॑ वरिवस्यन्तु दे॒वा दे॒वा व॑रिवस्यन्तु॒ विश्वे॒ विश्वे॑ वरिवस्यन्तु दे॒वाः । \newline
52. व॒रि॒व॒स्य॒न्तु॒ दे॒वा दे॒वा व॑रिवस्यन्तु वरिवस्यन्तु दे॒वाः । \newline
53. दे॒वा इति॑ दे॒वाः । \newline
54. प्रि॒या वो॑ वः प्रि॒या प्रि॒या वो॒ नाम॒ नाम॑ वः प्रि॒या प्रि॒या वो॒ नाम॑ । \newline
55. वो॒ नाम॒ नाम॑ वो वो॒ नाम॑ हुवे हुवे॒ नाम॑ वो वो॒ नाम॑ हुवे । \newline
56. नाम॑ हुवे हुवे॒ नाम॒ नाम॑ हुवे तु॒राणा᳚म् तु॒राणाꣳ॑ हुवे॒ नाम॒ नाम॑ हुवे तु॒राणा᳚म् । \newline
\pagebreak
\markright{ TS 2.1.11.2  \hfill https://www.vedavms.in \hfill}

\section{ TS 2.1.11.2 }

\textbf{TS 2.1.11.2 } \newline
\textbf{Samhita Paata} \newline

हुवे तु॒राणां᳚ । आ यत् तृ॒पन्म॑रुतो वावशा॒नाः ॥ श्रि॒यसे॒ कं भा॒नुभिः॒ सं मि॑मिक्षिरे॒ ते र॒श्मिभि॒स्त ऋक्व॑भिः सुखा॒दयः॑ । ते वाशी॑मन्त इ॒ष्मिणो॒ अभी॑रवो वि॒द्रे प्रि॒यस्य॒ मारु॑तस्य॒ धाम्नः॑ ॥ अ॒ग्निः प्र॑थ॒मो वसु॑भिर्नो अव्या॒थ् सोमो॑ रु॒द्रेभि॑र॒भि र॑क्षत॒ त्मना᳚ । इन्द्रो॑ म॒रुद्भि॑र् ऋतु॒धा कृ॑णोत्वादि॒त्यैर्नो॒ वरु॑णः॒ सꣳ शि॑शातु ॥ सं नो॑ दे॒वो वसु॑भिर॒ग्निः सꣳ - [  ] \newline

\textbf{Pada Paata} \newline

हु॒वे॒ । तु॒राणा᳚म् ॥ एति॑ । यत् । तृ॒पत् । म॒रु॒तः॒ । वा॒व॒शा॒नाः ॥ श्रि॒यसे᳚ । कम् । भा॒नुभि॒रिति॑ भा॒नु-भिः॒ । समिति॑ । मि॒मि॒क्षि॒रे॒ । ते । र॒श्मिभि॒रिति॑ र॒श्मि - भिः॒ । ते । ऋक्व॑भि॒रित्यृक्व॑ - भिः॒ । सु॒खा॒दय॒ इति॑ सु - खा॒दयः॑ ॥ ते । वाशी॑मन्त॒ इति॒ वाशि॑ - म॒न्तः॒ । इ॒ष्मिणः॑ । अभी॑रवः । वि॒द्रे । प्रि॒यस्य॑ । मारु॑तस्य । धाम्नः॑ ॥ अ॒ग्निः । प्र॒थ॒मः । वसु॑भि॒रिति॒ वसु॑-भिः॒ । नः॒ । अ॒व्या॒त् । सोमः॑ । रु॒द्रेभिः॑ । अ॒भीति॑ । र॒क्ष॒तु॒ । त्मना᳚ ॥ इन्द्रः॑ । म॒रुद्भि॒रिति॑ म॒रुत् - भिः॒ । ऋ॒तु॒धेत्यृ॑तु - धा । कृ॒णो॒तु॒ । आ॒दि॒त्यैः । नः॒ । वरु॑णः । समिति॑ । शि॒शा॒तु॒ ॥ समिति॑ । नः॒ । दे॒वः । वसु॑भि॒रिति॒ वसु॑ - भिः॒ । अ॒ग्निः । समिति॑ ।  \newline


\textbf{Krama Paata} \newline

हु॒वे॒ तु॒राणा᳚म् । तु॒राणा॒मिति॑ तु॒राणा᳚म् ॥ आ यत् । यत् तृ॒पत् । तृ॒पन् म॑रुतः । म॒रु॒तो॒ वा॒व॒शा॒नाः । वा॒व॒शा॒ना इति॑ वावशा॒नाः ॥ श्रि॒यसे॒ कम् । कम् भा॒नुभिः॑ । भा॒नुभिः॒ सम् । भा॒नुभि॒रिति॑ भा॒नु - भिः॒ । सम् मि॑मिक्षिरे । मि॒मि॒क्षि॒रे॒ ते । ते र॒श्मिभिः॑ । र॒श्मिभि॒स्ते । र॒श्मिभि॒रिति॑ र॒श्मि - भिः॒ । त ऋक्व॑भिः । ऋक्व॑भिः सुखा॒दयः॑ । ऋक्व॑भि॒रित्यृक्व॑ - भिः॒ । सु॒खा॒दय॒ इति॑ सु - खा॒दयः॑ ॥ ते वाशी॑मन्तः । वाशी॑मन्त इ॒ष्मिणः॑ । वाशी॑मन्त॒ इति॒ वाशि॑ - म॒न्तः॒ । इ॒ष्मिणो॒ अभी॑रवः । अभी॑रवो वि॒द्रे । वि॒द्रे प्रि॒यस्य॑ । प्रि॒यस्य॒ मारु॑तस्य । मारु॑तस्य॒ धाम्नः॑ । धाम्न॒ इति॒ धाम्नः॑ ॥ अ॒ग्निः प्र॑थ॒मः । प्र॒थ॒मो वसु॑भिः । वसु॑भिर् नः । वसु॑भि॒रिति॒ वसु॑ - भिः॒ । नो॒ अ॒व्या॒त्॒ । अ॒व्या॒थ् सोमः॑ । सोमो॑ रु॒द्रेभिः॑ । रु॒द्रेभि॑र॒भि । अ॒भि र॑क्षतु । र॒क्ष॒तु॒ त्मना᳚ । त्मनेति॒ त्मना᳚ ॥ इन्द्रो॑ म॒रुद्भिः॑ । म॒रुद्भि॑र्. ऋतु॒धा । म॒रुद्भि॒रिति॑ म॒रुत् - भिः॒ । ऋ॒तु॒धा कृ॑णोतु । ऋ॒तु॒धेत्यृ॑तु - धा । कृ॒णो॒त्वा॒दि॒त्यैः । आ॒दि॒त्यैर् नः॑ । नो॒ वरु॑णः । वरु॑णः॒ सम् । सꣳ शि॑शातु । शि॒शा॒त्विति॑ शिशातु ॥ सम् नः॑ । नो॒ दे॒वः । दे॒वो वसु॑भिः । वसु॑भिर॒ग्निः । वसु॑भि॒रिति॒ वसु॑ - भिः॒ । अ॒ग्निः सम् । सꣳ सोमः॑ \newline

\textbf{Jatai Paata} \newline

1. हु॒वे॒ तु॒राणा᳚म् तु॒राणाꣳ॑ हुवे हुवे तु॒राणा᳚म् । \newline
2. तु॒राणा॒मिति॑ तु॒राणा᳚म् । \newline
3. आ यद् यदा यत् । \newline
4. यत् तृ॒पत् तृ॒पद् यद् यत् तृ॒पत् । \newline
5. तृ॒पन् म॑रुतो मरुत स्तृ॒पत् तृ॒पन् म॑रुतः । \newline
6. म॒रु॒तो॒ वा॒व॒शा॒ना वा॑वशा॒ना म॑रुतो मरुतो वावशा॒नाः । \newline
7. वा॒व॒शा॒ना इति॑ वावशा॒नाः । \newline
8. श्रि॒यसे॒ कम् कꣳ श्रि॒यसे᳚ श्रि॒यसे॒ कम् । \newline
9. कम् भा॒नुभि॑र् भा॒नुभिः॒ कम् कम् भा॒नुभिः॑ । \newline
10. भा॒नुभिः॒ सꣳ सम् भा॒नुभि॑र् भा॒नुभिः॒ सम् । \newline
11. भा॒नुभि॒रिति॑ भा॒नु - भिः॒ । \newline
12. सम् मि॑मिक्षिरे मिमिक्षिरे॒ सꣳ सम् मि॑मिक्षिरे । \newline
13. मि॒मि॒क्षि॒रे॒ ते ते मि॑मिक्षिरे मिमिक्षिरे॒ ते । \newline
14. ते र॒श्मिभी॑ र॒श्मिभि॒ स्ते ते र॒श्मिभिः॑ । \newline
15. र॒श्मिभि॒ स्ते ते र॒श्मिभी॑ र॒श्मिभि॒ स्ते । \newline
16. र॒श्मिभि॒रिति॑ र॒श्मि - भिः॒ । \newline
17. त ऋक्व॑भि॒र्॒. ऋक्व॑भि॒ स्ते त ऋक्व॑भिः । \newline
18. ऋक्व॑भिः सुखा॒दयः॑ सुखा॒दय॒ ऋक्व॑भि॒र्॒. ऋक्व॑भिः सुखा॒दयः॑ । \newline
19. ऋक्व॑भि॒रित्यृक्व॑ - भिः॒ । \newline
20. सु॒खा॒दय॒ इति॑ सु - खा॒दयः॑ । \newline
21. ते वाशी॑मन्तो॒ वाशी॑मन्त॒ स्ते ते वाशी॑मन्तः । \newline
22. वाशी॑मन्त इ॒ष्मिण॑ इ॒ष्मिणो॒ वाशी॑मन्तो॒ वाशी॑मन्त इ॒ष्मिणः॑ । \newline
23. वाशी॑मन्त॒ इति॒ वाशि॑ - म॒न्तः॒ । \newline
24. इ॒ष्मिणो॒ अभी॑रवो॒ अभी॑रव इ॒ष्मिण॑ इ॒ष्मिणो॒ अभी॑रवः । \newline
25. अभी॑रवो वि॒द्रे वि॒द्रे अभी॑रवो॒ अभी॑रवो वि॒द्रे । \newline
26. वि॒द्रे प्रि॒यस्य॑ प्रि॒यस्य॑ वि॒द्रे वि॒द्रे प्रि॒यस्य॑ । \newline
27. प्रि॒यस्य॒ मारु॑तस्य॒ मारु॑तस्य प्रि॒यस्य॑ प्रि॒यस्य॒ मारु॑तस्य । \newline
28. मारु॑तस्य॒ धाम्नो॒ धाम्नो॒ मारु॑तस्य॒ मारु॑तस्य॒ धाम्नः॑ । \newline
29. धाम्न॒ इति॒ धाम्नः॑ । \newline
30. अ॒ग्निः प्र॑थ॒मः प्र॑थ॒मो᳚ ऽग्नि र॒ग्निः प्र॑थ॒मः । \newline
31. प्र॒थ॒मो वसु॑भि॒र् वसु॑भिः प्रथ॒मः प्र॑थ॒मो वसु॑भिः । \newline
32. वसु॑भिर् नो नो॒ वसु॑भि॒र् वसु॑भिर् नः । \newline
33. वसु॑भि॒रिति॒ वसु॑ - भिः॒ । \newline
34. नो॒ अ॒व्या॒ द॒व्या॒न् नो॒ नो॒ अ॒व्या॒त् । \newline
35. अ॒व्या॒थ् सोमः॒ सोमो॑ अव्या दव्या॒थ् सोमः॑ । \newline
36. सोमो॑ रु॒द्रेभी॑ रु॒द्रेभिः॒ सोमः॒ सोमो॑ रु॒द्रेभिः॑ । \newline
37. रु॒द्रेभि॑ र॒भ्य॑भि रु॒द्रेभी॑ रु॒द्रेभि॑ र॒भि । \newline
38. अ॒भि र॑क्षतु रक्ष त्व॒भ्य॑भि र॑क्षतु । \newline
39. र॒क्ष॒तु॒ त्मना॒ त्मना॑ रक्षतु रक्षतु॒ त्मना᳚ । \newline
40. त्मनेति॒ त्मना᳚ । \newline
41. इन्द्रो॑ म॒रुद्भि॑र् म॒रुद्भि॒ रिन्द्र॒ इन्द्रो॑ म॒रुद्भिः॑ । \newline
42. म॒रुद्भि॑र्. ऋतु॒धर्तु॒धा म॒रुद्भि॑र् म॒रुद्भि॑र्. ऋतु॒धा । \newline
43. म॒रुद्भि॒रिति॑ म॒रुत् - भिः॒ । \newline
44. ऋ॒तु॒धा कृ॑णोतु कृणो त्वृतु॒धर्‌तु॒धा कृ॑णोतु । \newline
45. ऋ॒तु॒धेत्यृ॑तु - धा । \newline
46. कृ॒णो॒ त्वा॒दि॒त्यै रा॑दि॒त्यैः कृ॑णोतु कृणो त्वादि॒त्यैः । \newline
47. आ॒दि॒त्यैर् नो॑ न आदि॒त्यै रा॑दि॒त्यैर् नः॑ । \newline
48. नो॒ वरु॑णो॒ वरु॑णो नो नो॒ वरु॑णः । \newline
49. वरु॑णः॒ सꣳ सं ॅवरु॑णो॒ वरु॑णः॒ सम् । \newline
50. सꣳ शि॑शातु शिशातु॒ सꣳ सꣳ शि॑शातु । \newline
51. शि॒शा॒त्विति॑ शिशातु । \newline
52. सम् नो॑ नः॒ सꣳ सम् नः॑ । \newline
53. नो॒ दे॒वो दे॒वो नो॑ नो दे॒वः । \newline
54. दे॒वो वसु॑भि॒र् वसु॑भिर् दे॒वो दे॒वो वसु॑भिः । \newline
55. वसु॑भि र॒ग्नि र॒ग्निर् वसु॑भि॒र् वसु॑भि र॒ग्निः । \newline
56. वसु॑भि॒रिति॒ वसु॑ - भिः॒ । \newline
57. अ॒ग्निः सꣳ स म॒ग्नि र॒ग्निः सम् । \newline
58. सꣳ सोमः॒ सोमः॒ सꣳ सꣳ सोमः॑ । \newline

\textbf{Ghana Paata } \newline

1. हु॒वे॒ तु॒राणा᳚म् तु॒राणाꣳ॑ हुवे हुवे तु॒राणा᳚म् । \newline
2. तु॒राणा॒मिति॑ तु॒राणा᳚म् । \newline
3. आ यद् यदा यत् तृ॒पत् तृ॒पद् यदा यत् तृ॒पत् । \newline
4. यत् तृ॒पत् तृ॒पद् यद् यत् तृ॒पन् म॑रुतो मरुत स्तृ॒पद् यद् यत् तृ॒पन् म॑रुतः । \newline
5. तृ॒पन् म॑रुतो मरुत स्तृ॒पत् तृ॒पन् म॑रुतो वावशा॒ना वा॑वशा॒ना म॑रुत स्तृ॒पत् तृ॒पन् म॑रुतो वावशा॒नाः । \newline
6. म॒रु॒तो॒ वा॒व॒शा॒ना वा॑वशा॒ना म॑रुतो मरुतो वावशा॒नाः । \newline
7. वा॒व॒शा॒ना इति॑ वावशा॒नाः । \newline
8. श्रि॒यसे॒ कम् कꣳ श्रि॒यसे᳚ श्रि॒यसे॒ कम् भा॒नुभि॑र् भा॒नुभिः॒ कꣳ श्रि॒यसे᳚ श्रि॒यसे॒ कम् भा॒नुभिः॑ । \newline
9. कम् भा॒नुभि॑र् भा॒नुभिः॒ कम् कम् भा॒नुभिः॒ सꣳ सम् भा॒नुभिः॒ कम् कम् भा॒नुभिः॒ सम् । \newline
10. भा॒नुभिः॒ सꣳ सम् भा॒नुभि॑र् भा॒नुभिः॒ सम् मि॑मिक्षिरे मिमिक्षिरे॒ सम् भा॒नुभि॑र् भा॒नुभिः॒ सम् मि॑मिक्षिरे । \newline
11. भा॒नुभि॒रिति॑ भा॒नु - भिः॒ । \newline
12. सम् मि॑मिक्षिरे मिमिक्षिरे॒ सꣳ सम् मि॑मिक्षिरे॒ ते ते मि॑मिक्षिरे॒ सꣳ सम् मि॑मिक्षिरे॒ ते । \newline
13. मि॒मि॒क्षि॒रे॒ ते ते मि॑मिक्षिरे मिमिक्षिरे॒ ते र॒श्मिभी॑ र॒श्मिभि॒ स्ते मि॑मिक्षिरे मिमिक्षिरे॒ ते र॒श्मिभिः॑ । \newline
14. ते र॒श्मिभी॑ र॒श्मिभि॒ स्ते ते र॒श्मिभि॒ स्ते ते र॒श्मिभि॒ स्ते ते र॒श्मिभि॒ स्ते । \newline
15. र॒श्मिभि॒ स्ते ते र॒श्मिभी॑ र॒श्मिभि॒ स्त ऋक्व॑भि॒र्॒. ऋक्व॑भि॒ स्ते र॒श्मिभी॑ र॒श्मिभि॒ स्त ऋक्व॑भिः । \newline
16. र॒श्मिभि॒रिति॑ र॒श्मि - भिः॒ । \newline
17. त ऋक्व॑भि॒र्॒. ऋक्व॑भि॒ स्ते त ऋक्व॑भिः सुखा॒दयः॑ सुखा॒दय॒ ऋक्व॑भि॒ स्ते त ऋक्व॑भिः सुखा॒दयः॑ । \newline
18. ऋक्व॑भिः सुखा॒दयः॑ सुखा॒दय॒ ऋक्व॑भि॒र्॒. ऋक्व॑भिः सुखा॒दयः॑ । \newline
19. ऋक्व॑भि॒रित्यृक्व॑ - भिः॒ । \newline
20. सु॒खा॒दय॒ इति॑ सु - खा॒दयः॑ । \newline
21. ते वाशी॑मन्तो॒ वाशी॑मन्त॒ स्ते ते वाशी॑मन्त इ॒ष्मिण॑ इ॒ष्मिणो॒ वाशी॑मन्त॒ स्ते ते वाशी॑मन्त इ॒ष्मिणः॑ । \newline
22. वाशी॑मन्त इ॒ष्मिण॑ इ॒ष्मिणो॒ वाशी॑मन्तो॒ वाशी॑मन्त इ॒ष्मिणो॒ अभी॑रवो॒ अभी॑रव इ॒ष्मिणो॒ वाशी॑मन्तो॒ वाशी॑मन्त इ॒ष्मिणो॒ अभी॑रवः । \newline
23. वाशी॑मन्त॒ इति॒ वाशि॑ - म॒न्तः॒ । \newline
24. इ॒ष्मिणो॒ अभी॑रवो॒ अभी॑रव इ॒ष्मिण॑ इ॒ष्मिणो॒ अभी॑रवो वि॒द्रे वि॒द्रे अभी॑रव इ॒ष्मिण॑ इ॒ष्मिणो॒ अभी॑रवो वि॒द्रे । \newline
25. अभी॑रवो वि॒द्रे वि॒द्रे अभी॑रवो॒ अभी॑रवो वि॒द्रे प्रि॒यस्य॑ प्रि॒यस्य॑ वि॒द्रे अभी॑रवो॒ अभी॑रवो वि॒द्रे प्रि॒यस्य॑ । \newline
26. वि॒द्रे प्रि॒यस्य॑ प्रि॒यस्य॑ वि॒द्रे वि॒द्रे प्रि॒यस्य॒ मारु॑तस्य॒ मारु॑तस्य प्रि॒यस्य॑ वि॒द्रे वि॒द्रे प्रि॒यस्य॒ मारु॑तस्य । \newline
27. प्रि॒यस्य॒ मारु॑तस्य॒ मारु॑तस्य प्रि॒यस्य॑ प्रि॒यस्य॒ मारु॑तस्य॒ धाम्नो॒ धाम्नो॒ मारु॑तस्य प्रि॒यस्य॑ प्रि॒यस्य॒ मारु॑तस्य॒ धाम्नः॑ । \newline
28. मारु॑तस्य॒ धाम्नो॒ धाम्नो॒ मारु॑तस्य॒ मारु॑तस्य॒ धाम्नः॑ । \newline
29. धाम्न॒ इति॒ धाम्नः॑ । \newline
30. अ॒ग्निः प्र॑थ॒मः प्र॑थ॒मो᳚ ऽग्निर॒ग्निः प्र॑थ॒मो वसु॑भि॒र् वसु॑भिः प्रथ॒मो᳚ ऽग्निर॒ग्निः प्र॑थ॒मो वसु॑भिः । \newline
31. प्र॒थ॒मो वसु॑भि॒र् वसु॑भिः प्रथ॒मः प्र॑थ॒मो वसु॑भिर् नो नो॒ वसु॑भिः प्रथ॒मः प्र॑थ॒मो वसु॑भिर् नः । \newline
32. वसु॑भिर् नो नो॒ वसु॑भि॒र् वसु॑भिर् नो अव्या दव्यान् नो॒ वसु॑भि॒र् वसु॑भिर् नो अव्यात् । \newline
33. वसु॑भि॒रिति॒ वसु॑ - भिः॒ । \newline
34. नो॒ अ॒व्या॒ द॒व्या॒न् नो॒ नो॒ अ॒व्या॒थ् सोमः॒ सोमो॑ अव्यान् नो नो अव्या॒थ् सोमः॑ । \newline
35. अ॒व्या॒थ् सोमः॒ सोमो॑ अव्या दव्या॒थ् सोमो॑ रु॒द्रेभी॑ रु॒द्रेभिः॒ सोमो॑ अव्या दव्या॒थ् सोमो॑ रु॒द्रेभिः॑ । \newline
36. सोमो॑ रु॒द्रेभी॑ रु॒द्रेभिः॒ सोमः॒ सोमो॑ रु॒द्रेभि॑ र॒भ्य॑भि रु॒द्रेभिः॒ सोमः॒ सोमो॑ रु॒द्रेभि॑ र॒भि । \newline
37. रु॒द्रेभि॑ र॒भ्य॑भि रु॒द्रेभी॑ रु॒द्रेभि॑ र॒भि र॑क्षतु रक्षत्व॒भि रु॒द्रेभी॑ रु॒द्रेभि॑ र॒भि र॑क्षतु । \newline
38. अ॒भि र॑क्षतु रक्ष त्व॒भ्य॑भि र॑क्षतु॒ त्मना॒ त्मना॑ रक्ष त्व॒भ्य॑भि र॑क्षतु॒ त्मना᳚ । \newline
39. र॒क्ष॒तु॒ त्मना॒ त्मना॑ रक्षतु रक्षतु॒ त्मना᳚ । \newline
40. त्मनेति॒ त्मना᳚ । \newline
41. इन्द्रो॑ म॒रुद्भि॑र् म॒रुद्भि॒ रिन्द्र॒ इन्द्रो॑ म॒रुद्भि॑र्. ऋतु॒धर्तु॒धा म॒रुद्भि॒ रिन्द्र॒ इन्द्रो॑ म॒रुद्भि॑र्. ऋतु॒धा । \newline
42. म॒रुद्भि॑र्. ऋतु॒धर्तु॒धा म॒रुद्भि॑र् म॒रुद्भि॑र्. ऋतु॒धा कृ॑णोत्वृणो त्वृतु॒धा म॒रुद्भि॑र् म॒रुद्भि॑र्. ऋतु॒धा कृ॑णोतु । \newline
43. म॒रुद्भि॒रिति॑ म॒रुत् - भिः॒ । \newline
44. ऋ॒तु॒धा कृ॑णोतु कृणोत्वृतु॒धर्तु॒धा कृ॑णो त्वादि॒त्यै रा॑दि॒त्यैः 
कृ॑णोत्वृतु॒धर्तु॒धा कृ॑णो त्वादि॒त्यैः । \newline
45. ऋ॒तु॒धेत्यृ॑तु - धा । \newline
46. कृ॒णो॒ त्वा॒दि॒त्यै रा॑दि॒त्यैः कृ॑णोतु कृणो त्वादि॒त्यैर् नो॑ न आदि॒त्यैः कृ॑णोतु कृणो त्वादि॒त्यैर् नः॑ । \newline
47. आ॒दि॒त्यैर् नो॑ न आदि॒त्यै रा॑दि॒त्यैर् नो॒ वरु॑णो॒ वरु॑णो न आदि॒त्यै रा॑दि॒त्यैर् नो॒ वरु॑णः । \newline
48. नो॒ वरु॑णो॒ वरु॑णो नो नो॒ वरु॑णः॒ सꣳ सं ॅवरु॑णो नो नो॒ वरु॑णः॒ सम् । \newline
49. वरु॑णः॒ सꣳ सं ॅवरु॑णो॒ वरु॑णः॒ सꣳ शि॑शातु शिशातु॒ सं ॅवरु॑णो॒ वरु॑णः॒ सꣳ शि॑शातु । \newline
50. सꣳ शि॑शातु शिशातु॒ सꣳ सꣳ शि॑शातु । \newline
51. शि॒शा॒त्विति॑ शिशातु । \newline
52. सन्नो॑ नः॒ सꣳ सन्नो॑ दे॒वो दे॒वो नः॒ सꣳ सन्नो॑ दे॒वः । \newline
53. नो॒ दे॒वो दे॒वो नो॑ नो दे॒वो वसु॑भि॒र् वसु॑भिर् दे॒वो नो॑ नो दे॒वो वसु॑भिः । \newline
54. दे॒वो वसु॑भि॒र् वसु॑भिर् दे॒वो दे॒वो वसु॑भि र॒ग्नि र॒ग्निर् वसु॑भिर् दे॒वो दे॒वो वसु॑भि र॒ग्निः । \newline
55. वसु॑भि र॒ग्नि र॒ग्निर् वसु॑भि॒र् वसु॑भि र॒ग्निः सꣳ स म॒ग्निर् वसु॑भि॒र् वसु॑भि र॒ग्निः सम् । \newline
56. वसु॑भि॒रिति॒ वसु॑ - भिः॒ । \newline
57. अ॒ग्निः सꣳ स म॒ग्नि र॒ग्निः सꣳ सोमः॒ सोमः॒ स म॒ग्नि र॒ग्निः सꣳ सोमः॑ । \newline
58. सꣳ सोमः॒ सोमः॒ सꣳ सꣳ सोम॑ स्त॒नूभि॑ स्त॒नूभिः॒ सोमः॒ सꣳ सꣳ सोम॑ स्त॒नूभिः॑ । \newline
\pagebreak
\markright{ TS 2.1.11.3  \hfill https://www.vedavms.in \hfill}

\section{ TS 2.1.11.3 }

\textbf{TS 2.1.11.3 } \newline
\textbf{Samhita Paata} \newline

सोम॑स्त॒नूभी॑ रु॒द्रिया॑भिः । समिन्द्रो॑ म॒रुद्भि॑ र्य॒ज्ञियैः॒ समा॑दि॒त्यैर्नो॒ वरु॑णो अजिज्ञिपत् ॥ यथा॑ऽऽदि॒त्या वसु॑भिः संबभू॒वुर्म॒रुद्भी॑ रु॒द्राः स॒मजा॑नता॒भि । ए॒वा त्रि॑णाम॒न्न-हृ॑णीयमाना॒ विश्वे॑ दे॒वाः सम॑नसो भवन्तु ॥ कुत्रा॑ चि॒द्यस्य॒ समृ॑तौ र॒ण्वा नरो॑ नृ॒षद॑ने । अर्.ह॑न्तश्चि॒द्-यमि॑न्ध॒ते स॑जं॒नय॑न्ति ज॒न्तवः॑ ॥ सं ॅयदि॒षो वना॑महे॒ सꣳ ह॒व्या मानु॑षाणां । उ॒त द्यु॒म्नस्य॒ शव॑स - [  ] \newline

\textbf{Pada Paata} \newline

सोमः॑ । त॒नूभिः॑ । रु॒द्रिया॑भिः ॥ समिति॑ । इन्द्रः॑ । म॒रुद्भि॒रिति॑ म॒रुत् - भिः॒ । य॒ज्ञियैः᳚ । समिति॑ । आ॒दि॒त्यैः । नः॒ । वरु॑णः । अ॒जि॒ज्ञि॒प॒त् ॥ यथा᳚ । आ॒दि॒त्याः । वसु॑भि॒रिति॒ वसु॑ - भिः॒ । स॒म्ब॒भू॒वुरिति॑ सं - ब॒भू॒वुः । म॒रुद्भि॒रिति॑ म॒रुत् - भिः॒ । रु॒द्राः । स॒मजा॑न॒तेति॑ सं - अजा॑नत् । अ॒भि ॥ ए॒वा । त्रि॒णा॒म॒न्निति॑ त्रि - ना॒म॒न्न् । अहृ॑णीयमानाः । विश्वे᳚ । दे॒वाः । सम॑नस॒ इति॒ स - म॒न॒सः॒ । भ॒व॒न्तु॒ ॥ कुत्र॑ । चि॒त्॒ । यस्य॑ । समृ॑ता॒विति॒ सं-ऋ॒तौ॒ । र॒ण्वाः । नरः॑ । नृ॒षद॑न॒ इति॑ नृ-सद॑ने ॥ अर्.ह॑न्तः । चि॒त् । यम् । इ॒न्ध॒ते । स॒ञ्ज॒नय॒न्तीति॑ सं - ज॒नय॑न्ति । ज॒न्तवः॑ ॥ समिति॑ । यत् । इ॒षः । वना॑महे । समिति॑ । ह॒व्या । मानु॑षाणाम् ॥ उ॒त । द्यु॒म्नस्य॑ । शव॑सः ।  \newline


\textbf{Krama Paata} \newline

सोम॑स्त॒नूभिः॑ । त॒नूभी॑ रु॒द्रिया॑भिः । रु॒द्रिया॑भि॒रिति॑ रु॒द्रिया॑भिः ॥ समिन्द्रः॑ । इन्द्रो॑ म॒रुद्भिः॑ । म॒रुद्भि॑र्. य॒ज्ञियैः᳚ । म॒रुद्भि॒रिति॑ म॒रुत् - भिः॒ । य॒ज्ञियैः॒ सम् । समा॑दि॒त्यैः । आ॒दि॒त्यैर् नः॑ । नो॒ वरु॑णः । वरु॑णो अजिज्ञिपत् । अ॒जि॒ज्ञि॒प॒दित्य॑जिज्ञिपत् ॥ यथा॑ऽऽदि॒त्याः । आ॒दि॒त्या वसु॑भिः । वसु॑भिः सम्बभू॒वुः । वसु॑भि॒रिति॒ वसु॑ - भिः॒ । स॒म्ब॒भू॒वुर् म॒रुद्भिः॑ । स॒म्ब॒भू॒वुरिति॑ सं - ब॒भू॒वुः । म॒रुद्भी॑ रु॒द्राः । म॒रुद्भि॒रिति॑ म॒रुत् - भिः॒ । रु॒द्राः स॒मजा॑नत । स॒मजा॑नता॒भि । स॒माजा॑न॒तेति॑ सं - अजा॑नत । अ॒भीत्य॒भि ॥ ए॒वा त्रि॑णामन्न् । त्रि॒णा॒म॒न्नहृ॑णीयमानाः । त्रि॒णा॒म॒न्निति॑ त्रि - ना॒म॒न्न्॒ । अहृ॑णीयमाना॒ विश्वे᳚ । विश्वे॑ दे॒वाः । दे॒वाः सम॑नसः । सम॑नसो भवन्तु । सम॑नस॒ इति॒ स - म॒न॒सः॒ । भ॒व॒न्त्विति॑ भवन्तु ॥ कुत्रा॑ चित् । चि॒द् यस्य॑ । यस्य॒ समृ॑तौ । समृ॑तौ र॒ण्वाः । समृ॑ता॒विति॒ सं - ऋ॒तौ॒ । र॒ण्वा नरः॑ । नरो॑ नृ॒षद॑ने । नृ॒षद॑न॒ इति॑ नृ - सद॑ने ॥ अर्.ह॑न्तश्चित् । चि॒द् यम् । यमि॑न्ध॒ते । इ॒न्ध॒ते स॑ञ्ज॒नय॑न्ति । स॒ञ्ज॒नय॑न्ति ज॒न्तवः॑ । स॒ञ्ज॒नय॒न्तीति॑ सं - ज॒नय॑न्ति । ज॒न्तव॒ इति॑ ज॒न्तवः॑ ॥ सम् ॅयत् । यदि॒षः । इ॒षो वना॑महे । वना॑महे॒ सम् । सꣳ ह॒व्या । ह॒व्या मानु॑षाणाम् । मानु॑षाणा॒मिति॒ मानु॑षाणाम् ॥ उ॒त द्यु॒म्नस्य॑ । द्यु॒म्नस्य॒ शव॑सः । शव॑स ऋ॒तस्य॑ \newline

\textbf{Jatai Paata} \newline

1. सोम॑ स्त॒नूभि॑ स्त॒नूभिः॒ सोमः॒ सोम॑ स्त॒नूभिः॑ । \newline
2. त॒नूभी॑ रु॒द्रिया॑भी रु॒द्रिया॑भि स्त॒नूभि॑ स्त॒नूभी॑ रु॒द्रिया॑भिः । \newline
3. रु॒द्रिया॑भि॒रिति॑ रु॒द्रिया॑भिः । \newline
4. स मिन्द्र॒ इन्द्रः॒ सꣳ स मिन्द्रः॑ । \newline
5. इन्द्रो॑ म॒रुद्भि॑र् म॒रुद्भि॒ रिन्द्र॒ इन्द्रो॑ म॒रुद्भिः॑ । \newline
6. म॒रुद्भि॑र् य॒ज्ञियै᳚र् य॒ज्ञियै᳚र् म॒रुद्भि॑र् म॒रुद्भि॑र् य॒ज्ञियैः᳚ । \newline
7. म॒रुद्भि॒रिति॑ म॒रुत् - भिः॒ । \newline
8. य॒ज्ञियैः॒ सꣳ सं ॅय॒ज्ञियै᳚र् य॒ज्ञियैः॒ सम् । \newline
9. स मा॑दि॒त्यै रा॑दि॒त्यैः सꣳ स मा॑दि॒त्यैः । \newline
10. आ॒दि॒त्यैर् नो॑ न आदि॒त्यै रा॑दि॒त्यैर् नः॑ । \newline
11. नो॒ वरु॑णो॒ वरु॑णो नो नो॒ वरु॑णः । \newline
12. वरु॑णो अजिज्ञिप दजिज्ञिप॒द् वरु॑णो॒ वरु॑णो अजिज्ञिपत् । \newline
13. अ॒जि॒ज्ञि॒प॒दित्य॑जिज्ञिपत् । \newline
14. यथा॑ ऽऽदि॒त्या आ॑दि॒त्या यथा॒ यथा॑ ऽऽदि॒त्याः । \newline
15. आ॒दि॒त्या वसु॑भि॒र् वसु॑भि रादि॒त्या आ॑दि॒त्या वसु॑भिः । \newline
16. वसु॑भिः सम्बभू॒वुः स॑म्बभू॒वुर् वसु॑भि॒र् वसु॑भिः सम्बभू॒वुः । \newline
17. वसु॑भि॒रिति॒ वसु॑ - भिः॒ । \newline
18. स॒म्ब॒भू॒वुर् म॒रुद्भि॑र् म॒रुद्भिः॑ सम्बभू॒वुः स॑म्बभू॒वुर् म॒रुद्भिः॑ । \newline
19. स॒म्ब॒भू॒वुरिति॑ सं - ब॒भू॒वुः । \newline
20. म॒रुद्भी॑ रु॒द्रा रु॒द्रा म॒रुद्भि॑र् म॒रुद्भी॑ रु॒द्राः । \newline
21. म॒रुद्भि॒रिति॑ म॒रुत् - भिः॒ । \newline
22. रु॒द्राः स॒मजा॑नत स॒मजा॑नत रु॒द्रा रु॒द्राः स॒मजा॑नत । \newline
23. स॒मजा॑नता॒ भ्य॑भि स॒मजा॑नत स॒मजा॑नता॒भि । \newline
24. स॒मजा॑न॒तेति॑ सं - अजा॑नत । \newline
25. अ॒भीत्य॒भि । \newline
26. ए॒वा त्रि॑णामन् त्रिणामन् ने॒वैवा त्रि॑णामन्न् । \newline
27. त्रि॒णा॒म॒न् नहृ॑णीयमाना॒ अहृ॑णीयमाना स्त्रिणामन् त्रिणाम॒न् नहृ॑णीयमानाः । \newline
28. त्रि॒णा॒म॒न्निति॑ त्रि - ना॒म॒न्न् । \newline
29. अहृ॑णीयमाना॒ विश्वे॒ विश्वे ऽहृ॑णीयमाना॒ अहृ॑णीयमाना॒ विश्वे᳚ । \newline
30. विश्वे॑ दे॒वा दे॒वा विश्वे॒ विश्वे॑ दे॒वाः । \newline
31. दे॒वाः सम॑नसः॒ सम॑नसो दे॒वा दे॒वाः सम॑नसः । \newline
32. सम॑नसो भवन्तु भवन्तु॒ सम॑नसः॒ सम॑नसो भवन्तु । \newline
33. सम॑नस॒ इति॒ स - म॒न॒सः॒ । \newline
34. भ॒व॒न्त्विति॑ भवन्तु । \newline
35. कुत्रा॑ चिच् चि॒त् कुत्र॒ कुत्रा॑ चित् । \newline
36. चि॒द् यस्य॒ यस्य॑ चिच् चि॒द् यस्य॑ । \newline
37. यस्य॒ समृ॑तौ॒ समृ॑तौ॒ यस्य॒ यस्य॒ समृ॑तौ । \newline
38. समृ॑तौ र॒ण्वा र॒ण्वाः समृ॑तौ॒ समृ॑तौ र॒ण्वाः । \newline
39. समृ॑ता॒विति॒ सं - ऋ॒तौ॒ । \newline
40. र॒ण्वा नरो॒ नरो॑ र॒ण्वा र॒ण्वा नरः॑ । \newline
41. नरो॑ नृ॒षद॑ने नृ॒षद॑ने॒ नरो॒ नरो॑ नृ॒षद॑ने । \newline
42. नृ॒षद॑न॒ इति॑ नृ - सद॑ने । \newline
43. अर्.ह॑न्त श्चिच् चि॒दर्.ह॑न्तो॒ अर्.ह॑न्त श्चित् । \newline
44. चि॒द् यं ॅयम् चि॑च् चि॒द् यम् । \newline
45. य मि॑न्ध॒त इ॑न्ध॒ते यं ॅय मि॑न्ध॒ते । \newline
46. इ॒न्ध॒ते स॑ञ्ज॒नय॑न्ति सञ्ज॒नय॑न्तीन्ध॒त इ॑न्ध॒ते स॑ञ्ज॒नय॑न्ति । \newline
47. स॒ञ्ज॒नय॑न्ति ज॒न्तवो॑ ज॒न्तवः॑ सञ्ज॒नय॑न्ति सञ्ज॒नय॑न्ति ज॒न्तवः॑ । \newline
48. स॒ञ्ज॒नय॒न्तीति॑ सं - ज॒नय॑न्ति । \newline
49. ज॒न्तव॒ इति॑ ज॒न्तवः॑ । \newline
50. सं ॅयद् यथ् सꣳ सं ॅयत् । \newline
51. यदि॒ष इ॒षो यद् यदि॒षः । \newline
52. इ॒षो वना॑महे॒ वना॑मह इ॒ष इ॒षो वना॑महे । \newline
53. वना॑महे॒ सꣳ सं ॅवना॑महे॒ वना॑महे॒ सम् । \newline
54. सꣳ ह॒व्या ह॒व्या सꣳ सꣳ ह॒व्या । \newline
55. ह॒व्या मानु॑षाणा॒म् मानु॑षाणाꣳ ह॒व्या ह॒व्या मानु॑षाणाम् । \newline
56. मानु॑षाणा॒मिति॒ मानु॑षाणाम् । \newline
57. उ॒त द्यु॒म्नस्य॑ द्यु॒म्नस्यो॒तोत द्यु॒म्नस्य॑ । \newline
58. द्यु॒म्नस्य॒ शव॑सः॒ शव॑सो द्यु॒म्नस्य॑ द्यु॒म्नस्य॒ शव॑सः । \newline
59. शव॑स ऋ॒तस्य॒ र्तस्य॒ शव॑सः॒ शव॑स ऋ॒तस्य॑ । \newline

\textbf{Ghana Paata } \newline

1. सोम॑ स्त॒नूभि॑ स्त॒नूभिः॒ सोमः॒ सोम॑ स्त॒नूभी॑ रु॒द्रिया॑भी रु॒द्रिया॑भि स्त॒नूभिः॒ सोमः॒ सोम॑ स्त॒नूभी॑ रु॒द्रिया॑भिः । \newline
2. त॒नूभी॑ रु॒द्रिया॑भी रु॒द्रिया॑भि स्त॒नूभि॑ स्त॒नूभी॑ रु॒द्रिया॑भिः । \newline
3. रु॒द्रिया॑भि॒रिति॑ रु॒द्रिया॑भिः । \newline
4. स मिन्द्र॒ इन्द्रः॒ सꣳ स मिन्द्रो॑ म॒रुद्भि॑र् म॒रुद्भि॒ रिन्द्रः॒ सꣳ स मिन्द्रो॑ म॒रुद्भिः॑ । \newline
5. इन्द्रो॑ म॒रुद्भि॑र् म॒रुद्भि॒ रिन्द्र॒ इन्द्रो॑ म॒रुद्भि॑र् य॒ज्ञियै᳚र् य॒ज्ञियै᳚र् म॒रुद्भि॒ रिन्द्र॒ इन्द्रो॑ म॒रुद्भि॑र् य॒ज्ञियैः᳚ । \newline
6. म॒रुद्भि॑र् य॒ज्ञियै᳚र् य॒ज्ञियै᳚र् म॒रुद्भि॑र् म॒रुद्भि॑र् य॒ज्ञियैः॒ सꣳ सं ॅय॒ज्ञियै᳚र् म॒रुद्भि॑र् म॒रुद्भि॑र् य॒ज्ञियैः॒ सम् । \newline
7. म॒रुद्भि॒रिति॑ म॒रुत् - भिः॒ । \newline
8. य॒ज्ञियैः॒ सꣳ सं ॅय॒ज्ञियै᳚र् य॒ज्ञियैः॒ स मा॑दि॒त्यै रा॑दि॒त्यैः सं ॅय॒ज्ञियै᳚र् य॒ज्ञियैः॒ स मा॑दि॒त्यैः । \newline
9. स मा॑दि॒त्यै रा॑दि॒त्यैः सꣳ स मा॑दि॒त्यैर् नो॑ न आदि॒त्यैः सꣳ स मा॑दि॒त्यैर् नः॑ । \newline
10. आ॒दि॒त्यैर् नो॑ न आदि॒त्यै रा॑दि॒त्यैर् नो॒ वरु॑णो॒ वरु॑णो न आदि॒त्यै रा॑दि॒त्यैर् नो॒ वरु॑णः । \newline
11. नो॒ वरु॑णो॒ वरु॑णो नो नो॒ वरु॑णो अजिज्ञिप दजिज्ञिप॒द् वरु॑णो नो नो॒ वरु॑णो अजिज्ञिपत् । \newline
12. वरु॑णो अजिज्ञिप दजिज्ञिप॒द् वरु॑णो॒ वरु॑णो अजिज्ञिपत् । \newline
13. अ॒जि॒ज्ञि॒प॒दित्य॑जिज्ञिपत् । \newline
14. यथा॑ ऽऽदि॒त्या आ॑दि॒त्या यथा॒ यथा॑ ऽऽदि॒त्या वसु॑भि॒र् वसु॑भि रादि॒त्या यथा॒ यथा॑ ऽऽदि॒त्या वसु॑भिः । \newline
15. आ॒दि॒त्या वसु॑भि॒र् वसु॑भि रादि॒त्या आ॑दि॒त्या वसु॑भिः सम्बभू॒वुः स॑म्बभू॒वुर् वसु॑भि रादि॒त्या आ॑दि॒त्या वसु॑भिः सम्बभू॒वुः । \newline
16. वसु॑भिः सम्बभू॒वुः स॑म्बभू॒वुर् वसु॑भि॒र् वसु॑भिः सम्बभू॒वुर् म॒रुद्भि॑र् म॒रुद्भिः॑ सम्बभू॒वुर् वसु॑भि॒र् वसु॑भिः सम्बभू॒वुर् म॒रुद्भिः॑ । \newline
17. वसु॑भि॒रिति॒ वसु॑ - भिः॒ । \newline
18. स॒म्ब॒भू॒वुर् म॒रुद्भि॑र् म॒रुद्भिः॑ सम्बभू॒वुः स॑म्बभू॒वुर् म॒रुद्भी॑ रु॒द्रा रु॒द्रा म॒रुद्भिः॑ सम्बभू॒वुः स॑म्बभू॒वुर् म॒रुद्भी॑ रु॒द्राः । \newline
19. स॒म्ब॒भू॒वुरिति॑ सं - ब॒भू॒वुः । \newline
20. म॒रुद्भी॑ रु॒द्रा रु॒द्रा म॒रुद्भि॑र् म॒रुद्भी॑ रु॒द्राः स॒मजा॑नत स॒मजा॑नत रु॒द्रा म॒रुद्भि॑र् म॒रुद्भी॑ रु॒द्राः स॒मजा॑नत । \newline
21. म॒रुद्भि॒रिति॑ म॒रुत् - भिः॒ । \newline
22. रु॒द्राः स॒मजा॑नत स॒मजा॑नत रु॒द्रा रु॒द्राः स॒मजा॑नता॒ भ्य॑भि स॒मजा॑नत रु॒द्रा रु॒द्राः स॒मजा॑नता॒भि । \newline
23. स॒मजा॑नता॒ भ्य॑भि स॒मजा॑नत स॒मजा॑नता॒भि । \newline
24. स॒मजा॑न॒तेति॑ सं - अजा॑नत । \newline
25. अ॒भीत्य॒भि । \newline
26. ए॒वा त्रि॑णामन् त्रिणामन् ने॒वैवा त्रि॑णाम॒न् नहृ॑णीयमाना॒ अहृ॑णीयमाना स्त्रिणामन् ने॒वैवा त्रि॑णाम॒न् नहृ॑णीयमानाः । \newline
27. त्रि॒णा॒म॒न् नहृ॑णीयमाना॒ अहृ॑णीयमाना स्त्रिणामन् त्रिणाम॒न् नहृ॑णीयमाना॒ विश्वे॒ विश्वे ऽहृ॑णीयमाना स्त्रिणामन् त्रिणाम॒न् नहृ॑णीयमाना॒ विश्वे᳚ । \newline
28. त्रि॒णा॒म॒न्निति॑ त्रि - ना॒म॒न्न् । \newline
29. अहृ॑णीयमाना॒ विश्वे॒ विश्वे ऽहृ॑णीयमाना॒ अहृ॑णीयमाना॒ विश्वे॑ दे॒वा दे॒वा विश्वे ऽहृ॑णीयमाना॒ अहृ॑णीयमाना॒ विश्वे॑ दे॒वाः । \newline
30. विश्वे॑ दे॒वा दे॒वा विश्वे॒ विश्वे॑ दे॒वाः सम॑नसः॒ सम॑नसो दे॒वा विश्वे॒ विश्वे॑ दे॒वाः सम॑नसः । \newline
31. दे॒वाः सम॑नसः॒ सम॑नसो दे॒वा दे॒वाः सम॑नसो भवन्तु भवन्तु॒ सम॑नसो दे॒वा दे॒वाः सम॑नसो भवन्तु । \newline
32. सम॑नसो भवन्तु भवन्तु॒ सम॑नसः॒ सम॑नसो भवन्तु । \newline
33. सम॑नस॒ इति॒ स - म॒न॒सः॒ । \newline
34. भ॒व॒न्त्विति॑ भवन्तु । \newline
35. कुत्रा॑ चिच् चि॒त् कुत्र॒ कुत्रा॑ चि॒द् यस्य॒ यस्य॑ चि॒त् कुत्र॒ कुत्रा॑ चि॒द् यस्य॑ । \newline
36. चि॒द् यस्य॒ यस्य॑ चिच् चि॒द् यस्य॒ समृ॑तौ॒ समृ॑तौ॒ यस्य॑ चिच् चि॒द् यस्य॒ समृ॑तौ । \newline
37. यस्य॒ समृ॑तौ॒ समृ॑तौ॒ यस्य॒ यस्य॒ समृ॑तौ र॒ण्वा र॒ण्वाः समृ॑तौ॒ यस्य॒ यस्य॒ समृ॑तौ र॒ण्वाः । \newline
38. समृ॑तौ र॒ण्वा र॒ण्वाः समृ॑तौ॒ समृ॑तौ र॒ण्वा नरो॒ नरो॑ र॒ण्वाः समृ॑तौ॒ समृ॑तौ र॒ण्वा नरः॑ । \newline
39. समृ॑ता॒विति॒ सं - ऋ॒तौ॒ । \newline
40. र॒ण्वा नरो॒ नरो॑ र॒ण्वा र॒ण्वा नरो॑ नृ॒षद॑ने नृ॒षद॑ने॒ नरो॑ र॒ण्वा र॒ण्वा नरो॑ नृ॒षद॑ने । \newline
41. नरो॑ नृ॒षद॑ने नृ॒षद॑ने॒ नरो॒ नरो॑ नृ॒षद॑ने । \newline
42. नृ॒षद॑न॒ इति॑ नृ - सद॑ने । \newline
43. अर्.ह॑न्त श्चिच् चि॒दर्.ह॑न्तो॒ अर्.ह॑न्त श्चि॒द् यं ॅयम् चि॒दर्.ह॑न्तो॒ अर्.ह॑न्त श्चि॒द् यम् । \newline
44. चि॒द् यं ॅयम् चि॑च् चि॒द् य मि॑न्ध॒त इ॑न्ध॒ते यम् चि॑च् चि॒द् य मि॑न्ध॒ते । \newline
45. य मि॑न्ध॒त इ॑न्ध॒ते यं ॅय मि॑न्ध॒ते स॑ञ्ज॒नय॑न्ति सञ्ज॒नय॑न् तीन्ध॒ते यं ॅय मि॑न्ध॒ते स॑ञ्ज॒नय॑न्ति । \newline
46. इ॒न्ध॒ते स॑ञ्ज॒नय॑न्ति सञ्ज॒नय॑ न्तीन्ध॒त इ॑न्ध॒ते स॑ञ्ज॒नय॑न्ति ज॒न्तवो॑ ज॒न्तवः॑ सञ्ज॒नय॑ न्तीन्ध॒त इ॑न्ध॒ते स॑ञ्ज॒नय॑न्ति ज॒न्तवः॑ । \newline
47. स॒ञ्ज॒नय॑न्ति ज॒न्तवो॑ ज॒न्तवः॑ सञ्ज॒नय॑न्ति सञ्ज॒नय॑न्ति ज॒न्तवः॑ । \newline
48. स॒ञ्ज॒नय॒न्तीति॑ सं - ज॒नय॑न्ति । \newline
49. ज॒न्तव॒ इति॑ ज॒न्तवः॑ । \newline
50. सं ॅयद् यथ् सꣳ सं ॅयदि॒ष इ॒षो यथ् सꣳ सं ॅयदि॒षः । \newline
51. यदि॒ष इ॒षो यद् यदि॒षो वना॑महे॒ वना॑मह इ॒षो यद् यदि॒षो वना॑महे । \newline
52. इ॒षो वना॑महे॒ वना॑मह इ॒ष इ॒षो वना॑महे॒ सꣳ सं ॅवना॑मह इ॒ष इ॒षो वना॑महे॒ सम् । \newline
53. वना॑महे॒ सꣳ सं ॅवना॑महे॒ वना॑महे॒ सꣳ ह॒व्या ह॒व्या सं ॅवना॑महे॒ वना॑महे॒ सꣳ ह॒व्या । \newline
54. सꣳ ह॒व्या ह॒व्या सꣳ सꣳ ह॒व्या मानु॑षाणा॒म् मानु॑षाणाꣳ ह॒व्या सꣳ सꣳ ह॒व्या मानु॑षाणाम् । \newline
55. ह॒व्या मानु॑षाणा॒म् मानु॑षाणाꣳ ह॒व्या ह॒व्या मानु॑षाणाम् । \newline
56. मानु॑षाणा॒मिति॒ मानु॑षाणाम् । \newline
57. उ॒त द्यु॒म्नस्य॑ द्यु॒म्नस्यो॒तोत द्यु॒म्नस्य॒ शव॑सः॒ शव॑सो द्यु॒म्नस्यो॒तोत द्यु॒म्नस्य॒ शव॑सः । \newline
58. द्यु॒म्नस्य॒ शव॑सः॒ शव॑सो द्यु॒म्नस्य॑ द्यु॒म्नस्य॒ शव॑स ऋ॒तस्य॒ र्तस्य॒ शव॑सो द्यु॒म्नस्य॑ द्यु॒म्नस्य॒ शव॑स ऋ॒तस्य॑ । \newline
59. शव॑स ऋ॒तस्य॒ र्तस्य॒ शव॑सः॒ शव॑स ऋ॒तस्य॑ र॒श्मिꣳ र॒श्मि मृ॒तस्य॒ शव॑सः॒ शव॑स ऋ॒तस्य॑ र॒श्मिम् । \newline
\pagebreak
\markright{ TS 2.1.11.4  \hfill https://www.vedavms.in \hfill}

\section{ TS 2.1.11.4 }

\textbf{TS 2.1.11.4 } \newline
\textbf{Samhita Paata} \newline

ऋ॒तस्य॑ र॒श्मिमा द॑दे ॥ य॒ज्ञो दे॒वाना॒म् प्रत्ये॑ति सु॒म्नमादि॑त्यासो॒ भव॑ता मृड॒यन्तः॑ । आ वो॒ऽर्वाची॑ सुम॒तिर्व॑वृत्यादꣳ॒॒ होश्चि॒द्या व॑रिवो॒वित्त॒राऽस॑त् ॥ शुचि॑र॒पः सू॒यव॑सा अद॑ब्ध॒ उप॑ क्षेति वृ॒द्धव॑याः सु॒वीरः॑ । नकि॒ष्टं घ्न॒न्त्यन्ति॑तो॒ न दू॒राद्य आ॑दि॒त्यानां॒ भव॑ति॒ प्रणी॑तौ ॥ धा॒रय॑न्त आदि॒त्यासो॒ जग॒थ्स्था दे॒वा विश्व॑स्य॒ भुव॑नस्य गो॒पाः । दी॒र्घाधि॑यो॒ रक्ष॑माणा-  [  ] \newline

\textbf{Pada Paata} \newline

ऋ॒तस्य॑ । र॒श्मिम् । एति॑ । द॒दे॒ ॥ य॒ज्ञ्ः । दे॒वाना᳚म् । प्रतीति॑ । ए॒ति॒ । सु॒म्नम् । आदि॑त्यासः । भव॑त । मृ॒ड॒यन्तः॑ ॥ एति॑ । वः॒ । अ॒र्वाची᳚ । सु॒म॒तिरिति॑ सु - म॒तिः । व॒वृ॒त्या॒त् । अꣳ॒॒होः । चि॒त् । या । व॒रि॒वो॒वित्त॒रेति॑ वरिवो॒वित् - त॒रा॒ । अस॑त् ॥ शुचिः॑ । अ॒पः । सू॒यव॑सा॒ इति॑ सु - यव॑साः । अद॑ब्धः । उपेति॑ । क्षे॒ति॒ । वृ॒द्धव॑या॒ इति॑ वृ॒द्ध - व॒याः॒ । सु॒वीर॒ इति॑ सु - वीरः॑ ॥ नकिः॑ । तम् । घ्न॒न्ति॒ । अन्ति॑तः । न । दू॒रात् । यः । आ॒दि॒त्याना᳚म् । भव॑ति । प्रणी॑त॒विति॒ प्र - नी॒तौ॒ ॥ धा॒रय॑न्तः । आ॒दि॒त्यासः॑ । जग॑त् । स्थाः । दे॒वाः । विश्व॑स्य । भुव॑नस्य । गो॒पा इति॑ गो-पाः ॥ दी॒र्घाधि॑य॒ इति॑ दी॒र्घ - धि॒यः॒ । रक्ष॑माणाः ।  \newline


\textbf{Krama Paata} \newline

ऋ॒तस्य॑ र॒श्मिम् । र॒श्मिमा । आ द॑दे । द॒द॒ इति॑ ददे ॥ य॒ज्ञो दे॒वाना᳚म् । दे॒वाना॒म् प्रति॑ । प्रत्ये॑ति । ए॒ति॒ सु॒म्नम् । सु॒म्नमादि॑त्यासः । आदि॑त्यासो॒ भव॑त । भव॑ता मृड॒यन्तः॑ । मृ॒ड॒यन्त॒ इति॑ मृड॒यन्तः॑ ॥ आ वः॑ । वो॒ऽर्वाची᳚ । अ॒र्वाची॑ सुम॒तिः । सु॒म॒तिर् व॑वृत्यात् । सु॒म॒तिरिति॑ सु - म॒तिः । व॒वृ॒त्या॒दꣳ॒॒होः । अꣳ॒॒होश्चि॑त् । चि॒द् या । या व॑रिवो॒वित्त॑रा । व॒रि॒वो॒वित्त॒राऽस॑त् । व॒रि॒वो॒वित्त॒रेति॑ वरिवो॒वित् - त॒रा॒ । अस॒दित्यस॑त् ॥ शुचि॑र॒पः । अ॒पः सू॒यव॑साः । सू॒यव॑सा॒ अद॑ब्धः । सू॒यव॑सा॒ इति॑ सु - यव॑साः । अद॑ब्ध॒ उप॑ । उप॑ क्षेति । क्षे॒ति॒ वृ॒द्धव॑याः । वृ॒द्धव॑याः सु॒वीरः॑ । वृ॒द्धव॑या॒ इति॑ वृ॒द्ध - व॒याः॒ । सु॒वीर॒ इति॑ सु - वीरः॑ ॥ नकि॒ष्टम् । तम्(2) घ्न॑न्ति । घ्न॒न्त्यन्ति॑तः । अन्ति॑तो॒ न । न दू॒रात् । दू॒राद् यः । य आ॑दि॒त्याना᳚म् । आ॒दि॒त्याना॒म् भव॑ति । भव॑ति॒ प्रणी॑तौ । प्रणी॑ता॒विति॒ प्र - नी॒तौ॒ ॥ धा॒रय॑न्त आदि॒त्यासः॑ । आ॒दि॒त्यासो॒ जग॑त् । जग॒थ् स्थाः । स्था दे॒वाः । दे॒वा विश्व॑स्य । विश्व॑स्य॒ भुव॑नस्य । भुव॑नस्य गो॒पाः । गो॒पा इति॑ गो - पाः ॥ दी॒र्घाधि॑यो॒ रक्ष॑माणाः । दी॒र्घाधि॑य॒ इति॑ दी॒र्घ - धि॒यः॒ । रक्ष॑माणा असु॒र्य᳚म् \newline

\textbf{Jatai Paata} \newline

1. ऋ॒तस्य॑ र॒श्मिꣳ र॒श्मि मृ॒तस्य॒ र्तस्य॑ र॒श्मिम् । \newline
2. र॒श्मि मा र॒श्मिꣳ र॒श्मि मा । \newline
3. आ द॑दे दद॒ आ द॑दे । \newline
4. द॒द॒ इति॑ ददे । \newline
5. य॒ज्ञो दे॒वाना᳚म् दे॒वानां᳚ ॅय॒ज्ञो य॒ज्ञो दे॒वाना᳚म् । \newline
6. दे॒वाना॒म् प्रति॒ प्रति॑ दे॒वाना᳚म् दे॒वाना॒म् प्रति॑ । \newline
7. प्रत्ये᳚ त्येति॒ प्रति॒ प्रत्ये॑ति । \newline
8. ए॒ति॒ सु॒म्नꣳ सु॒म्न मे᳚त्येति सु॒म्नम् । \newline
9. सु॒म्न मादि॑त्यास॒ आदि॑त्यासः सु॒म्नꣳ सु॒म्न मादि॑त्यासः । \newline
10. आदि॑त्यासो॒ भव॑त॒ भव॒तादि॑त्यास॒ आदि॑त्यासो॒ भव॑त । \newline
11. भव॑ता मृड॒यन्तो॑ मृड॒यन्तो॒ भव॑त॒ भव॑ता मृड॒यन्तः॑ । \newline
12. मृ॒ड॒यन्त॒ इति॑ मृड॒यन्तः॑ । \newline
13. आ वो॑ व॒ आ वः॑ । \newline
14. वो॒ ऽर्वा च्य॒र्वाची॑ वो वो॒ ऽर्वाची᳚ । \newline
15. अ॒र्वाची॑ सुम॒तिः सु॑म॒ति र॒र्वा च्य॒र्वाची॑ सुम॒तिः । \newline
16. सु॒म॒तिर् व॑वृत्याद् ववृत्याथ् सुम॒तिः सु॑म॒तिर् व॑वृत्यात् । \newline
17. सु॒म॒तिरिति॑ सु - म॒तिः । \newline
18. व॒वृ॒त्या॒ दꣳ॒॒हो रꣳ॒॒होर् व॑वृत्याद् ववृत्या दꣳ॒॒होः । \newline
19. अꣳ॒॒हो श्चि॑च् चि दꣳ॒॒हो रꣳ॒॒होश्चि॑त् । \newline
20. चि॒द् या या चि॑च् चि॒द् या । \newline
21. या व॑रिवो॒वित्त॑रा वरिवो॒वित्त॑रा॒ या या व॑रिवो॒वित्त॑रा । \newline
22. व॒रि॒वो॒वित्त॒रा ऽस॒दस॑द् वरिवो॒वित्त॑रा वरिवो॒वित्त॒रा ऽस॑त् । \newline
23. व॒रि॒वो॒वित्त॒रेति॑ वरिवो॒वित् - त॒रा॒ । \newline
24. अस॒दित्यस॑त् । \newline
25. शुचि॑ र॒पो अ॒पः शुचिः॒ शुचि॑ र॒पः । \newline
26. अ॒पः सू॒यव॑साः सू॒यव॑सा अ॒पो अ॒पः सू॒यव॑साः । \newline
27. सू॒यव॑सा॒ अद॑ब्धो॒ अद॑ब्धः सू॒यव॑साः सू॒यव॑सा॒ अद॑ब्धः । \newline
28. सू॒यव॑सा॒ इति॑ सु - यव॑साः । \newline
29. अद॑ब्ध॒ उपोपाद॑ब्धो॒ अद॑ब्ध॒ उप॑ । \newline
30. उप॑ क्षेति क्षे॒ त्युपोप॑ क्षेति । \newline
31. क्षे॒ति॒ वृ॒द्धव॑या वृ॒द्धव॑याः क्षेति क्षेति वृ॒द्धव॑याः । \newline
32. वृ॒द्धव॑याः सु॒वीरः॑ सु॒वीरो॑ वृ॒द्धव॑या वृ॒द्धव॑याः सु॒वीरः॑ । \newline
33. वृ॒द्धव॑या॒ इति॑ वृ॒द्ध - व॒याः॒ । \newline
34. सु॒वीर॒ इति॑ सु - वीरः॑ । \newline
35. नकि॒ष्टम् तम् नकि॒र् नकि॒ष्टम् । \newline
36. तम्(2) घ्न॑न्ति घ्नन्ति॒ तम् तम्(2) घ्न॑न्ति । \newline
37. घ्न॒न्त्यन्ति॑तो॒ अन्ति॑तो घ्नन्ति घ्न॒न्त्यन्ति॑तः । \newline
38. अन्ति॑तो॒ न नान्ति॑तो॒ अन्ति॑तो॒ न । \newline
39. न दू॒राद् दू॒रान् न न दू॒रात् । \newline
40. दू॒राद् यो यो दू॒राद् दू॒राद् यः । \newline
41. य आ॑दि॒त्याना॑ मादि॒त्यानां॒ ॅयो य आ॑दि॒त्याना᳚म् । \newline
42. आ॒दि॒त्याना॒म् भव॑ति॒ भव॑ त्यादि॒त्याना॑ मादि॒त्याना॒म् भव॑ति । \newline
43. भव॑ति॒ प्रणी॑तौ॒ प्रणी॑तौ॒ भव॑ति॒ भव॑ति॒ प्रणी॑तौ । \newline
44. प्रणी॑ता॒विति॒ प्र - नी॒तौ॒ । \newline
45. धा॒रय॑न्त आदि॒त्यास॑ आदि॒त्यासो॑ धा॒रय॑न्तो धा॒रय॑न्त आदि॒त्यासः॑ । \newline
46. आ॒दि॒त्यासो॒ जग॒ज् जग॑ दादि॒त्यास॑ आदि॒त्यासो॒ जग॑त् । \newline
47. जग॒थ् स्थाः स्था जग॒ज् जग॒थ् स्थाः । \newline
48. स्था दे॒वा दे॒वाः स्थाः स्था दे॒वाः । \newline
49. दे॒वा विश्व॑स्य॒ विश्व॑स्य दे॒वा दे॒वा विश्व॑स्य । \newline
50. विश्व॑स्य॒ भुव॑नस्य॒ भुव॑नस्य॒ विश्व॑स्य॒ विश्व॑स्य॒ भुव॑नस्य । \newline
51. भुव॑नस्य गो॒पा गो॒पा भुव॑नस्य॒ भुव॑नस्य गो॒पाः । \newline
52. गो॒पा इति॑ गो - पाः । \newline
53. दी॒र्घाधि॑यो॒ रक्ष॑माणा॒ रक्ष॑माणा दी॒र्घाधि॑यो दी॒र्घाधि॑यो॒ रक्ष॑माणाः । \newline
54. दी॒र्घाधि॑य॒ इति॑ दी॒र्घ - धि॒यः॒ । \newline
55. रक्ष॑माणा असु॒र्य॑ मसु॒र्यꣳ॑ रक्ष॑माणा॒ रक्ष॑माणा असु॒र्य᳚म् । \newline

\textbf{Ghana Paata } \newline

1. ऋ॒तस्य॑ र॒श्मिꣳ र॒श्मि मृ॒तस्य॒ र्तस्य॑ र॒श्मि मा र॒श्मि मृ॒तस्य॒ र्तस्य॑ र॒श्मि मा । \newline
2. र॒श्मि मा र॒श्मिꣳ र॒श्मि मा द॑दे दद॒ आ र॒श्मिꣳ र॒श्मि मा द॑दे । \newline
3. आ द॑दे दद॒ आ द॑दे । \newline
4. द॒द॒ इति॑ ददे । \newline
5. य॒ज्ञो दे॒वाना᳚म् दे॒वानां᳚ ॅय॒ज्ञो य॒ज्ञो दे॒वाना॒म् प्रति॒ प्रति॑ दे॒वानां᳚ ॅय॒ज्ञो य॒ज्ञो दे॒वाना॒म् प्रति॑ । \newline
6. दे॒वाना॒म् प्रति॒ प्रति॑ दे॒वाना᳚म् दे॒वाना॒म् प्रत्ये᳚त्येति॒ प्रति॑ दे॒वाना᳚म् दे॒वाना॒म् प्रत्ये॑ति । \newline
7. प्रत्ये᳚त्येति॒ प्रति॒ प्रत्ये॑ति सु॒म्नꣳ सु॒म्न मे॑ति॒ प्रति॒ प्रत्ये॑ति सु॒म्नम् । \newline
8. ए॒ति॒ सु॒म्नꣳ सु॒म्न मे᳚त्येति सु॒म्न मादि॑त्यास॒ आदि॑त्यासः सु॒म्न मे᳚त्येति सु॒म्न मादि॑त्यासः । \newline
9. सु॒म्न मादि॑त्यास॒ आदि॑त्यासः सु॒म्नꣳ सु॒म्न मादि॑त्यासो॒ भव॑त॒ भव॒तादि॑त्यासः सु॒म्नꣳ सु॒म्न मादि॑त्यासो॒ भव॑त । \newline
10. आदि॑त्यासो॒ भव॑त॒ भव॒तादि॑त्यास॒ आदि॑त्यासो॒ भव॑ता मृड॒यन्तो॑ मृड॒यन्तो॒ भव॒तादि॑त्यास॒ आदि॑त्यासो॒ भव॑ता मृड॒यन्तः॑ । \newline
11. भव॑ता मृड॒यन्तो॑ मृड॒यन्तो॒ भव॑त॒ भव॑ता मृड॒यन्तः॑ । \newline
12. मृ॒ड॒यन्त॒ इति॑ मृड॒यन्तः॑ । \newline
13. आ वो॑ व॒ आ वो॒ ऽर्वाच्य॒र्वाची॑ व॒ आ वो॒ ऽर्वाची᳚ । \newline
14. वो॒ ऽर्वा च्य॒र्वाची॑ वो वो॒ ऽर्वाची॑ सुम॒तिः सु॑म॒ति र॒र्वाची॑ वो वो॒ ऽर्वाची॑ सुम॒तिः । \newline
15. अ॒र्वाची॑ सुम॒तिः सु॑म॒ति र॒र्वा च्य॒र्वाची॑ सुम॒तिर् व॑वृत्याद् ववृत्याथ् सुम॒ति र॒र्वा च्य॒र्वाची॑ सुम॒तिर् व॑वृत्यात् । \newline
16. सु॒म॒तिर् व॑वृत्याद् ववृत्याथ् सुम॒तिः सु॑म॒तिर् व॑वृत्या दꣳ॒॒हो रꣳ॒॒होर् व॑वृत्याथ् सुम॒तिः सु॑म॒तिर् व॑वृत्यादꣳ॒॒होः । \newline
17. सु॒म॒तिरिति॑ सु - म॒तिः । \newline
18. व॒वृ॒त्या॒ दꣳ॒॒हो रꣳ॒॒होर् व॑वृत्याद् ववृत्या दꣳ॒॒हो श्चि॑च् चिदꣳ॒॒होर् व॑वृत्याद् ववृत्यादꣳ॒॒हो श्चि॑त् । \newline
19. अꣳ॒॒हो श्चि॑च् चिदꣳ॒॒हो रꣳ॒॒हो श्चि॒द् या या चि॑दꣳ॒॒हो रꣳ॒॒होश्चि॒द् या । \newline
20. चि॒द् या या चि॑च् चि॒द् या व॑रिवो॒वित्त॑रा वरिवो॒वित्त॑रा॒ या चि॑च् चि॒द् या व॑रिवो॒वित्त॑रा । \newline
21. या व॑रिवो॒वित्त॑रा वरिवो॒वित्त॑रा॒ या या व॑रिवो॒वित्त॒रा ऽस॒दस॑द् वरिवो॒वित्त॑रा॒ या या व॑रिवो॒वित्त॒रा ऽस॑त् । \newline
22. व॒रि॒वो॒वित्त॒रा ऽस॒दस॑द् वरिवो॒वित्त॑रा वरिवो॒वित्त॒रा ऽस॑त् । \newline
23. व॒रि॒वो॒वित्त॒रेति॑ वरिवो॒वित् - त॒रा॒ । \newline
24. अस॒दित्यस॑त् । \newline
25. शुचि॑र॒पो अ॒पः शुचिः॒ शुचि॑ र॒पः सू॒यव॑साः सू॒यव॑सा अ॒पः शुचिः॒ शुचि॑ र॒पः सू॒यव॑साः । \newline
26. अ॒पः सू॒यव॑साः सू॒यव॑सा अ॒पो अ॒पः सू॒यव॑सा॒ अद॑ब्धो॒ अद॑ब्धः सू॒यव॑सा अ॒पो अ॒पः सू॒यव॑सा॒ अद॑ब्धः । \newline
27. सू॒यव॑सा॒ अद॑ब्धो॒ अद॑ब्धः सू॒यव॑साः सू॒यव॑सा॒ अद॑ब्ध॒ उपोपाद॑ब्धः सू॒यव॑साः सू॒यव॑सा॒ अद॑ब्ध॒ उप॑ । \newline
28. सू॒यव॑सा॒ इति॑ सु - यव॑साः । \newline
29. अद॑ब्ध॒ उपोपाद॑ब्धो॒ अद॑ब्ध॒ उप॑ क्षेति क्षे॒त्युपाद॑ब्धो॒ अद॑ब्ध॒ उप॑ क्षेति । \newline
30. उप॑ क्षेति क्षे॒त्युपोप॑ क्षेति वृ॒द्धव॑या वृ॒द्धव॑याः क्षे॒त्युपोप॑ क्षेति वृ॒द्धव॑याः । \newline
31. क्षे॒ति॒ वृ॒द्धव॑या वृ॒द्धव॑याः क्षेति क्षेति वृ॒द्धव॑याः सु॒वीरः॑ सु॒वीरो॑ वृ॒द्धव॑याः क्षेति क्षेति वृ॒द्धव॑याः सु॒वीरः॑ । \newline
32. वृ॒द्धव॑याः सु॒वीरः॑ सु॒वीरो॑ वृ॒द्धव॑या वृ॒द्धव॑याः सु॒वीरः॑ । \newline
33. वृ॒द्धव॑या॒ इति॑ वृ॒द्ध - व॒याः॒ । \newline
34. सु॒वीर॒ इति॑ सु - वीरः॑ । \newline
35. नकि॒ ष्टम् तम् नकि॒र् नकि॒ ष्टम्(2) घ्न॑न्ति घ्नन्ति॒ तम् नकि॒र् नकि॒ ष्टम्(2) घ्न॑न्ति । \newline
36. तम्(2) घ्न॑न्ति घ्नन्ति॒ तम् तम्(2) घ्न॒ न्त्यन्ति॑तो॒ अन्ति॑तो घ्नन्ति॒ तम् तम्(2) घ्न॒न् त्यन्ति॑तः । \newline
37. घ्न॒न्त्यन्ति॑तो॒ अन्ति॑तो घ्नन्ति घ्न॒न्त्यन्ति॑तो॒ न नान्ति॑तो घ्नन्ति घ्न॒न्त्यन्ति॑तो॒ न । \newline
38. अन्ति॑तो॒ न नान्ति॑तो॒ अन्ति॑तो॒ न दू॒राद् दू॒रान् नान्ति॑तो॒ अन्ति॑तो॒ न दू॒रात् । \newline
39. न दू॒राद् दू॒रान् न न दू॒राद् यो यो दू॒रान् न न दू॒राद् यः । \newline
40. दू॒राद् यो यो दू॒राद् दू॒राद् य आ॑दि॒त्याना॑ मादि॒त्यानां॒ ॅयो दू॒राद् दू॒राद् य आ॑दि॒त्याना᳚म् । \newline
41. य आ॑दि॒त्याना॑ मादि॒त्यानां॒ ॅयो य आ॑दि॒त्याना॒म् भव॑ति॒ भव॑ त्यादि॒त्यानां॒ ॅयो य आ॑दि॒त्याना॒म् भव॑ति । \newline
42. आ॒दि॒त्याना॒म् भव॑ति॒ भव॑ त्यादि॒त्याना॑ मादि॒त्याना॒म् भव॑ति॒ प्रणी॑तौ॒ प्रणी॑तौ॒ भव॑ त्यादि॒त्याना॑ मादि॒त्याना॒म् भव॑ति॒ प्रणी॑तौ । \newline
43. भव॑ति॒ प्रणी॑तौ॒ प्रणी॑तौ॒ भव॑ति॒ भव॑ति॒ प्रणी॑तौ । \newline
44. प्रणी॑ता॒विति॒ प्र - नी॒तौ॒ । \newline
45. धा॒रय॑न्त आदि॒त्यास॑ आदि॒त्यासो॑ धा॒रय॑न्तो धा॒रय॑न्त आदि॒त्यासो॒ जग॒ज् जग॑दादि॒त्यासो॑ धा॒रय॑न्तो धा॒रय॑न्त आदि॒त्यासो॒ जग॑त् । \newline
46. आ॒दि॒त्यासो॒ जग॒ज् जग॑ दादि॒त्यास॑ आदि॒त्यासो॒ जग॒थ् स्थाः स्था जग॑ दादि॒त्यास॑ आदि॒त्यासो॒ जग॒थ् स्थाः । \newline
47. जग॒थ् स्थाः स्था जग॒ज् जग॒थ् स्था दे॒वा दे॒वाः स्था जग॒ज् जग॒थ् स्था दे॒वाः । \newline
48. स्था दे॒वा दे॒वाः स्थाः स्था दे॒वा विश्व॑स्य॒ विश्व॑स्य दे॒वाः स्थाः स्था दे॒वा विश्व॑स्य । \newline
49. दे॒वा विश्व॑स्य॒ विश्व॑स्य दे॒वा दे॒वा विश्व॑स्य॒ भुव॑नस्य॒ भुव॑नस्य॒ विश्व॑स्य दे॒वा दे॒वा विश्व॑स्य॒ भुव॑नस्य । \newline
50. विश्व॑स्य॒ भुव॑नस्य॒ भुव॑नस्य॒ विश्व॑स्य॒ विश्व॑स्य॒ भुव॑नस्य गो॒पा गो॒पा भुव॑नस्य॒ विश्व॑स्य॒ विश्व॑स्य॒ भुव॑नस्य गो॒पाः । \newline
51. भुव॑नस्य गो॒पा गो॒पा भुव॑नस्य॒ भुव॑नस्य गो॒पाः । \newline
52. गो॒पा इति॑ गो - पाः । \newline
53. दी॒र्घाधि॑यो॒ रक्ष॑माणा॒ रक्ष॑माणा दी॒र्घाधि॑यो दी॒र्घाधि॑यो॒ रक्ष॑माणा असु॒र्य॑ मसु॒र्यꣳ॑ रक्ष॑माणा दी॒र्घाधि॑यो दी॒र्घाधि॑यो॒ रक्ष॑माणा असु॒र्य᳚म् । \newline
54. दी॒र्घाधि॑य॒ इति॑ दी॒र्घ - धि॒यः॒ । \newline
55. रक्ष॑माणा असु॒र्य॑ मसु॒र्यꣳ॑ रक्ष॑माणा॒ रक्ष॑माणा असु॒र्य॑ मृ॒तावा॑न ऋ॒तावा॑नो असु॒र्यꣳ॑ रक्ष॑माणा॒ रक्ष॑माणा असु॒र्य॑ मृ॒तावा॑नः । \newline
\pagebreak
\markright{ TS 2.1.11.5  \hfill https://www.vedavms.in \hfill}

\section{ TS 2.1.11.5 }

\textbf{TS 2.1.11.5 } \newline
\textbf{Samhita Paata} \newline

असु॒र्य॑मृ॒तावा॑न॒-श्चय॑माना ऋ॒णानि॑ ॥ ति॒स्रो भूमी᳚र्द्धारय॒न् त्रीꣳ रु॒त द्यून् त्रीणि॑ व्र॒ता वि॒दथे॑ अ॒न्तरे॑षां । ऋ॒तेना॑ऽऽ*दित्या॒ महि॑ वो महि॒त्वं तद॑र्यमन् वरुण मित्र॒ चारु॑ ॥ त्यान्नु क्ष॒त्रियाꣳ॒॒ अव॑ आदि॒त्यान्. या॑चिषामहे । सु॒मृ॒डी॒काꣳ अ॒भिष्ट॑ये ॥ न द॑क्षि॒णा विचि॑किते॒ न स॒व्या न प्रा॒चीन॑मादित्या॒ नोत प॒श्चा । पा॒क्या॑ चिद्वसवो धी॒र्या॑ चिद् - [  ] \newline

\textbf{Pada Paata} \newline

अ॒सु॒र्य᳚म् । ऋ॒तावा॑न॒ इत्यृ॒त - वा॒नः॒ । चय॑मानाः । ऋ॒णानि॑ ॥ ति॒स्रः । भूमीः᳚ । धा॒र॒य॒न्न् । त्रीन् । उ॒त । द्यून् । त्रीणि॑ । व्र॒ता । वि॒दथे᳚ । अ॒न्तः । ए॒षा॒म् ॥ ऋ॒तेन॑ । आ॒दि॒त्याः॒ । महि॑ । वः॒ । म॒हि॒त्वमिति॑ महि - त्वम् । तत् । अ॒र्य॒म॒न्न् । व॒रु॒ण॒ । मि॒त्र॒ । चारु॑ ॥ त्यान् । नु । क्ष॒त्रियान्॑ । अवः॑ । आ॒दि॒त्यान् । या॒चि॒षा॒म॒हे॒ ॥ सु॒मृ॒डी॒कानिति॑ सु - म॒डी॒कान् । अ॒भिष्ट॑ये ॥ न । द॒क्षि॒णा । वीति॑ । चि॒कि॒ते॒ । न । स॒व्या । न । प्रा॒चीन᳚म् । आ॒दि॒त्याः॒ । न । उ॒त । प॒श्चा ॥ पा॒क्या᳚ । चि॒त् । व॒स॒वः॒ । धी॒र्या᳚ । चि॒त् ।  \newline


\textbf{Krama Paata} \newline

अ॒सु॒र्य॑मृ॒तावा॑नः । ऋ॒तावा॑न॒ श्चय॑मानाः । ऋ॒तावा॑न॒ इत्यृ॒त - वा॒नः॒ । चय॑माना ऋ॒णानि॑ । ऋ॒णानीत्यृ॒णानि॑ ॥ ति॒स्रो भूमीः᳚ । भूमी᳚र् धारयन्न् । धा॒र॒य॒न् त्रीन् । त्रीꣳरु॒त । उ॒त द्यून् । द्यून् त्रीणि॑ । त्रीणि॑ व्र॒ता । व्र॒ता वि॒दथे᳚ । वि॒दथे॑ अ॒न्तः । अ॒न्तरे॑षाम् । ए॒षा॒मित्ये॑षाम् ॥ ऋ॒तेना॑दित्याः । आ॒दि॒त्या॒ महि॑ । महि॑ वः । वो॒ म॒हि॒त्वम् । म॒हि॒त्वम् तत् । म॒हि॒त्वमिति॑ महि - त्वम् । तद॑र्यमन्न् । अ॒र्य॒म॒न् व॒रु॒ण॒ । व॒रु॒ण॒ मि॒त्र॒ । मि॒त्र॒ चारु॑ । चार्विति॒ चारु॑ ॥ त्याम् नु । नु क्ष॒त्रियान्॑ । क्ष॒त्रियाꣳ॒॒ अवः॑ । अव॑ आदि॒त्यान् । आ॒दि॒त्यान्. या॑चिषामहे । या॒चि॒षा॒म॒ह॒ इति॑ याचिषामहे ॥ सु॒मृ॒डी॒काꣳ अ॒भिष्ट॑ये । सु॒मृ॒डी॒कानिति॑ सु - मृ॒डी॒कान् । अ॒भिष्ट॑य॒ इत्य॒भिष्ट॑ये ॥ न द॑क्षि॒णा । द॒क्षि॒णा वि । वि चि॑किते । चि॒कि॒ते॒ न । न स॒व्या । स॒व्या न । न प्रा॒चीन᳚म् । प्रा॒चीन॑मादित्याः । आ॒दि॒त्या॒ न । नोत । उ॒त प॒श्चा । प॒श्चेति॑ प॒श्चा ॥ पा॒क्या॑ चित् । चि॒द् व॒स॒वः॒ । व॒स॒वो॒ धी॒र्या᳚ । धी॒र्या॑ चित् ( ) । चि॒द् यु॒ष्मानी॑तः \newline

\textbf{Jatai Paata} \newline

1. अ॒सु॒र्य॑ मृ॒तावा॑न ऋ॒तावा॑नो असु॒र्य॑ मसु॒र्य॑ मृ॒तावा॑नः । \newline
2. ऋ॒तावा॑न॒ श्चय॑माना॒ श्चय॑माना ऋ॒तावा॑न ऋ॒तावा॑न॒ श्चय॑मानाः । \newline
3. ऋ॒तावा॑न॒ इत्यृ॒त - वा॒नः॒ । \newline
4. चय॑माना ऋ॒णा न्यृ॒णानि॒ चय॑माना॒ श्चय॑माना ऋ॒णानि॑ । \newline
5. ऋ॒णानीत्यृ॒णानि॑ । \newline
6. ति॒स्रो भूमी॒र् भूमी᳚ स्ति॒स्र स्ति॒स्रो भूमीः᳚ । \newline
7. भूमी᳚र् धारयन् धारय॒न् भूमी॒र् भूमी᳚र् धारयन्न् । \newline
8. धा॒र॒य॒न् त्रीꣳ स्त्रीन् धा॑रयन् धारय॒न् त्रीन् । \newline
9. त्रीꣳ रु॒तोत त्रीꣳ स्त्रीꣳ रु॒त । \newline
10. उ॒त द्यून् द्यू नु॒तोत द्यून् । \newline
11. द्यून् त्रीणि॒ त्रीणि॒ द्यून् द्यून् त्रीणि॑ । \newline
12. त्रीणि॑ व्र॒ता व्र॒ता त्रीणि॒ त्रीणि॑ व्र॒ता । \newline
13. व्र॒ता वि॒दथे॑ वि॒दथे᳚ व्र॒ता व्र॒ता वि॒दथे᳚ । \newline
14. वि॒दथे॑ अ॒न्त र॒न्तर् वि॒दथे॑ वि॒दथे॑ अ॒न्तः । \newline
15. अ॒न्त रे॑षा मेषा म॒न्त र॒न्त रे॑षाम् । \newline
16. ए॒षा॒मित्ये॑षाम् । \newline
17. ऋ॒तेना॑दित्या आदित्या ऋ॒तेन॒ र्‌तेना॑दित्याः । \newline
18. आ॒दि॒त्या॒ महि॒ मह्या॑दित्या आदित्या॒ महि॑ । \newline
19. महि॑ वो वो॒ महि॒ महि॑ वः । \newline
20. वो॒ म॒हि॒त्वम् म॑हि॒त्वं ॅवो॑ वो महि॒त्वम् । \newline
21. म॒हि॒त्वम् तत् तन् म॑हि॒त्वम् म॑हि॒त्वम् तत् । \newline
22. म॒हि॒त्वमिति॑ महि - त्वम् । \newline
23. तद॑र्यमन् नर्यम॒न् तत् तद॑र्यमन्न् । \newline
24. अ॒र्य॒म॒न्॒. व॒रु॒ण॒ व॒रु॒णा॒र्य॒म॒न् न॒र्य॒म॒न्॒. व॒रु॒ण॒ । \newline
25. व॒रु॒ण॒ मि॒त्र॒ मि॒त्र॒ व॒रु॒ण॒ व॒रु॒ण॒ मि॒त्र॒ । \newline
26. मि॒त्र॒ चारु॒ चारु॑ मित्र मित्र॒ चारु॑ । \newline
27. चार्विति॒ चारु॑ । \newline
28. त्यान् नु नु त्यान् त्यान् नु । \newline
29. नु क्ष॒त्रिया᳚न् क्ष॒त्रिया॒न् नु नु क्ष॒त्रियान्॑ । \newline
30. क्ष॒त्रियाꣳ॒॒ अवो ऽवः॑ क्ष॒त्रिया᳚न् क्ष॒त्रियाꣳ॒॒ अवः॑ । \newline
31. अव॑ आदि॒त्या ना॑दि॒त्या नवो ऽव॑ आदि॒त्यान् । \newline
32. आ॒दि॒त्यान्. या॑चिषामहे याचिषामह आदि॒त्या ना॑दि॒त्यान्. या॑चिषामहे । \newline
33. या॒चि॒षा॒म॒ह॒ इति॑ याचिषामहे । \newline
34. सु॒मृ॒डी॒काꣳ अ॒भिष्ट॑ये अ॒भिष्ट॑ये सुमृडी॒कान् थ्सु॑मृडी॒काꣳ अ॒भिष्ट॑ये । \newline
35. सु॒मृ॒डी॒कानिति॑ सु - मृ॒डी॒कान् । \newline
36. अ॒भिष्ट॑य॒ इत्य॒भिष्ट॑ये । \newline
37. न द॑क्षि॒णा द॑क्षि॒णा न न द॑क्षि॒णा । \newline
38. द॒क्षि॒णा वि वि द॑क्षि॒णा द॑क्षि॒णा वि । \newline
39. वि चि॑किते चिकिते॒ वि वि चि॑किते । \newline
40. चि॒कि॒ते॒ न न चि॑किते चिकिते॒ न । \newline
41. न स॒व्या स॒व्या न न स॒व्या । \newline
42. स॒व्या न न स॒व्या स॒व्या न । \newline
43. न प्रा॒चीन॑म् प्रा॒चीन॒म् न न प्रा॒चीन᳚म् । \newline
44. प्रा॒चीन॑ मादित्या आदित्याः प्रा॒चीन॑म् प्रा॒चीन॑ मादित्याः । \newline
45. आ॒दि॒त्या॒ न नादि॑त्या आदित्या॒ न । \newline
46. नोतोत न नोत । \newline
47. उ॒त प॒श्चा प॒श्चोतोत प॒श्चा । \newline
48. प॒श्चेति॑ प॒श्चा । \newline
49. पा॒क्या॑ चिच् चित् पा॒क्या॑ पा॒क्या॑ चित् । \newline
50. चि॒द् व॒स॒वो॒ व॒स॒व॒ श्चि॒च् चि॒द् व॒स॒वः॒ । \newline
51. व॒स॒वो॒ धी॒र्या॑ धी॒र्या॑ वसवो वसवो धी॒र्या᳚ । \newline
52. धी॒र्या॑ चिच् चिद् धी॒र्या॑ धी॒र्या॑ चित् । \newline
53. चि॒द् यु॒ष्मानी॑तो यु॒ष्मानी॑त श्चिच् चिद् यु॒ष्मानी॑तः । \newline

\textbf{Ghana Paata } \newline

1. अ॒सु॒र्य॑ मृ॒तावा॑न ऋ॒तावा॑नो असु॒र्य॑ मसु॒र्य॑ मृ॒तावा॑न॒ श्चय॑माना॒ श्चय॑माना ऋ॒तावा॑नो असु॒र्य॑ मसु॒र्य॑ मृ॒तावा॑न॒ श्चय॑मानाः । \newline
2. ऋ॒तावा॑न॒ श्चय॑माना॒ श्चय॑माना ऋ॒तावा॑न ऋ॒तावा॑न॒ श्चय॑माना ऋ॒णा न्यृ॒णानि॒ चय॑माना ऋ॒तावा॑न ऋ॒तावा॑न॒ श्चय॑माना ऋ॒णानि॑ । \newline
3. ऋ॒तावा॑न॒ इत्यृ॒त - वा॒नः॒ । \newline
4. चय॑माना ऋ॒णा न्यृ॒णानि॒ चय॑माना॒ श्चय॑माना ऋ॒णानि॑ । \newline
5. ऋ॒णानीत्यृ॒णानि॑ । \newline
6. ति॒स्रो भूमी॒र् भूमी᳚ स्ति॒स्र स्ति॒स्रो भूमी᳚र् धारयन् धारय॒न् भूमी᳚ स्ति॒स्र स्ति॒स्रो भूमी᳚र् धारयन्न् । \newline
7. भूमी᳚र् धारयन् धारय॒न् भूमी॒र् भूमी᳚र् धारय॒न् त्रीꣳ स्त्रीन् धा॑रय॒न् भूमी॒र् भूमी᳚र् धारय॒न् त्रीन् । \newline
8. धा॒र॒य॒न् त्रीꣳ स्त्रीन् धा॑रयन् धारय॒न् त्रीꣳ रु॒तोत त्रीन् धा॑रयन् धारय॒न् त्रीꣳ रु॒त । \newline
9. त्रीꣳ रु॒तोत त्रीꣳ स्त्रीꣳ रु॒त द्यून् द्यू नु॒त त्रीꣳ स्त्रीꣳ रु॒त द्यून् । \newline
10. उ॒त द्यून् द्यू नु॒तोत द्यून् त्रीणि॒ त्रीणि॒ द्यू नु॒तोत द्यून् त्रीणि॑ । \newline
11. द्यून् त्रीणि॒ त्रीणि॒ द्यून् द्यून् त्रीणि॑ व्र॒ता व्र॒ता त्रीणि॒ द्यून् द्यून् त्रीणि॑ व्र॒ता । \newline
12. त्रीणि॑ व्र॒ता व्र॒ता त्रीणि॒ त्रीणि॑ व्र॒ता वि॒दथे॑ वि॒दथे᳚ व्र॒ता त्रीणि॒ त्रीणि॑ व्र॒ता वि॒दथे᳚ । \newline
13. व्र॒ता वि॒दथे॑ वि॒दथे᳚ व्र॒ता व्र॒ता वि॒दथे॑ अ॒न्त र॒न्तर् वि॒दथे᳚ व्र॒ता व्र॒ता वि॒दथे॑ अ॒न्तः । \newline
14. वि॒दथे॑ अ॒न्त र॒न्तर् वि॒दथे॑ वि॒दथे॑ अ॒न्त रे॑षा मेषा म॒न्तर् वि॒दथे॑ वि॒दथे॑ अ॒न्त रे॑षाम् । \newline
15. अ॒न्त रे॑षा मेषा म॒न्त र॒न्त रे॑षाम् । \newline
16. ए॒षा॒मित्ये॑षाम् । \newline
17. ऋ॒तेना॑दित्या आदित्या ऋ॒तेन॒ र्तेना॑दित्या॒ महि॒ मह्या॑दित्या ऋ॒तेन॒ र्तेना॑दित्या॒ महि॑ । \newline
18. आ॒दि॒त्या॒ महि॒ मह्या॑दित्या आदित्या॒ महि॑ वो वो॒ मह्या॑दित्या आदित्या॒ महि॑ वः । \newline
19. महि॑ वो वो॒ महि॒ महि॑ वो महि॒त्वम् म॑हि॒त्वं ॅवो॒ महि॒ महि॑ वो महि॒त्वम् । \newline
20. वो॒ म॒हि॒त्वम् म॑हि॒त्वं ॅवो॑ वो महि॒त्वम् तत् तन् म॑हि॒त्वं ॅवो॑ वो महि॒त्वम् तत् । \newline
21. म॒हि॒त्वम् तत् तन् म॑हि॒त्वम् म॑हि॒त्वम् तद॑र्यमन् नर्यम॒न् तन् म॑हि॒त्वम् म॑हि॒त्वम् तद॑र्यमन्न् । \newline
22. म॒हि॒त्वमिति॑ महि - त्वम् । \newline
23. तद॑र्यमन् नर्यम॒न् तत् तद॑र्यमन्. वरुण वरुणार्यम॒न् तत् तद॑र्यमन्. वरुण । \newline
24. अ॒र्य॒म॒न्॒. व॒रु॒ण॒ व॒रु॒णा॒र्य॒म॒न् न॒र्य॒म॒न्॒. व॒रु॒ण॒ मि॒त्र॒ मि॒त्र॒ व॒रु॒णा॒र्य॒म॒न् न॒र्य॒म॒न्॒. व॒रु॒ण॒ मि॒त्र॒ । \newline
25. व॒रु॒ण॒ मि॒त्र॒ मि॒त्र॒ व॒रु॒ण॒ व॒रु॒ण॒ मि॒त्र॒ चारु॒ चारु॑ मित्र वरुण वरुण मित्र॒ चारु॑ । \newline
26. मि॒त्र॒ चारु॒ चारु॑ मित्र मित्र॒ चारु॑ । \newline
27. चार्विति॒ चारु॑ । \newline
28. त्यान् नु नु त्यान् त्यान् नु क्ष॒त्रिया᳚न् क्ष॒त्रिया॒न् नु त्यान् त्यान् नु क्ष॒त्रियान्॑ । \newline
29. नु क्ष॒त्रिया᳚न् क्ष॒त्रिया॒न् नु नु क्ष॒त्रियाꣳ॒॒ अवो ऽवः॑ क्ष॒त्रिया॒न् नु नु क्ष॒त्रियाꣳ॒॒ अवः॑ । \newline
30. क्ष॒त्रियाꣳ॒॒ अवो ऽवः॑ क्ष॒त्रिया᳚न् क्ष॒त्रियाꣳ॒॒ अव॑ आदि॒त्या ना॑दि॒त्या नवः॑ क्ष॒त्रिया᳚न् क्ष॒त्रियाꣳ॒॒ अव॑ आदि॒त्यान् । \newline
31. अव॑ आदि॒त्या ना॑दि॒त्या नवो ऽव॑ आदि॒त्यान्. या॑चिषामहे याचिषामह आदि॒त्या नवो ऽव॑ आदि॒त्यान्. या॑चिषामहे । \newline
32. आ॒दि॒त्यान्. या॑चिषामहे याचिषामह आदि॒त्या ना॑दि॒त्यान्. या॑चिषामहे । \newline
33. या॒चि॒षा॒म॒ह॒ इति॑ याचिषामहे । \newline
34. सु॒मृ॒डी॒काꣳ अ॒भिष्ट॑ये अ॒भिष्ट॑ये सुमृडी॒कान् थ्सु॑मृडी॒काꣳ अ॒भिष्ट॑ये । \newline
35. सु॒मृ॒डी॒कानिति॑ सु - मृ॒डी॒कान् । \newline
36. अ॒भिष्ट॑य॒ इत्य॒भिष्ट॑ये । \newline
37. न द॑क्षि॒णा द॑क्षि॒णा न न द॑क्षि॒णा वि वि द॑क्षि॒णा न न द॑क्षि॒णा वि । \newline
38. द॒क्षि॒णा वि वि द॑क्षि॒णा द॑क्षि॒णा वि चि॑किते चिकिते॒ वि द॑क्षि॒णा द॑क्षि॒णा वि चि॑किते । \newline
39. वि चि॑किते चिकिते॒ वि वि चि॑किते॒ न न चि॑किते॒ वि वि चि॑किते॒ न । \newline
40. चि॒कि॒ते॒ न न चि॑किते चिकिते॒ न स॒व्या स॒व्या न चि॑किते चिकिते॒ न स॒व्या । \newline
41. न स॒व्या स॒व्या न न स॒व्या न न स॒व्या न न स॒व्या न । \newline
42. स॒व्या न न स॒व्या स॒व्या न प्रा॒चीन॑म् प्रा॒चीन॒न्न स॒व्या स॒व्या न प्रा॒चीन᳚म् । \newline
43. न प्रा॒चीन॑म् प्रा॒चीन॒न्न न प्रा॒चीन॑ मादित्या आदित्याः प्रा॒चीन॒न्न न प्रा॒चीन॑ मादित्याः । \newline
44. प्रा॒चीन॑ मादित्या आदित्याः प्रा॒चीन॑म् प्रा॒चीन॑ मादित्या॒ न नादि॑त्याः प्रा॒चीन॑म् प्रा॒चीन॑ मादित्या॒ न । \newline
45. आ॒दि॒त्या॒ न नादि॑त्या आदित्या॒ नोतोत नादि॑त्या आदित्या॒ नोत । \newline
46. नोतोत न नोत प॒श्चा प॒श्चोत न नोत प॒श्चा । \newline
47. उ॒त प॒श्चा प॒श्चोतोत प॒श्चा । \newline
48. प॒श्चेति॑ प॒श्चा । \newline
49. पा॒क्या॑ चिच् चित् पा॒क्या॑ पा॒क्या॑ चिद् वसवो वसव श्चित् पा॒क्या॑ पा॒क्या॑ चिद् वसवः । \newline
50. चि॒द् व॒स॒वो॒ व॒स॒व॒ श्चि॒च् चि॒द् व॒स॒वो॒ धी॒र्या॑ धी॒र्या॑ वसव श्चिच् चिद् वसवो धी॒र्या᳚ । \newline
51. व॒स॒वो॒ धी॒र्या॑ धी॒र्या॑ वसवो वसवो धी॒र्या॑ चिच् चिद् धी॒र्या॑ वसवो वसवो धी॒र्या॑ चित् । \newline
52. धी॒र्या॑ चिच् चिद् धी॒र्या॑ धी॒र्या॑ चिद् यु॒ष्मानी॑तो यु॒ष्मानी॑त श्चिद् धी॒र्या॑ धी॒र्या॑ चिद् यु॒ष्मानी॑तः । \newline
53. चि॒द् यु॒ष्मानी॑तो यु॒ष्मानी॑त श्चिच् चिद् यु॒ष्मानी॑तो॒ अभ॑य॒ मभ॑यं ॅयु॒ष्मानी॑त श्चिच् चिद् यु॒ष्मानी॑तो॒ अभ॑यम् । \newline
\pagebreak
\markright{ TS 2.1.11.6  \hfill https://www.vedavms.in \hfill}

\section{ TS 2.1.11.6 }

\textbf{TS 2.1.11.6 } \newline
\textbf{Samhita Paata} \newline

-यु॒ष्मानी॑तो॒ अभ॑यं॒ ज्योति॑रश्यां ॥ आ॒दि॒त्याना॒मव॑सा॒ नूत॑नेन सक्षी॒महि॒ शर्म॑णा॒ शन्त॑मेन । अ॒ना॒गा॒स्त्वे अ॑दिति॒त्वे तु॒रास॑ इ॒मं ॅय॒ज्ञ्ं द॑धतु॒ श्रोष॑माणाः ॥ इ॒मं मे॑ वरुण श्रुधी॒ हव॑म॒द्या च॑ मृडय । त्वाम॑व॒स्युरा च॑के ॥ तत्त्वा॑ यामि॒ ब्रह्म॑णा॒ वन्द॑मान॒-स्तदा शा᳚स्ते॒ यज॑मानो ह॒विर्भिः॑ । अहे॑डमानो वरुणे॒ह बो॒द्ध्युरु॑शꣳस॒ मा न॒ आयुः॒ प्रमो॑षीः ॥ \newline

\textbf{Pada Paata} \newline

यु॒ष्मानी॑तः । अभ॑यम् । ज्योतिः॑ । अ॒श्या॒म् ॥ आ॒दि॒त्याना᳚म् । अव॑सा । नूत॑नेन । स॒क्षी॒महि॑ । शर्म॑णा । शन्त॑मे॒नेति॒ शं - त॒मे॒न॒ ॥ अ॒ना॒गा॒स्त्व इत्य॑नागाः-त्वे । अ॒दि॒ति॒त्व इत्य॑दिति - त्वे । तु॒रासः॑ । इ॒मम् । य॒ज्ञ्म् । द॒ध॒तु॒ । श्रोष॑माणाः ॥ इ॒मम् । मे॒ । व॒रु॒ण॒ । श्रु॒ध॒ । हव᳚म् । अ॒द्य । च॒ । मृ॒ड॒य॒ ॥ त्वाम् । अ॒व॒स्युः । एति॑ । च॒के॒ ॥ तत् । त्वा॒ । या॒मि॒ । ब्रह्म॑णा । वन्द॑मानः । तत् । एति॑ । शा॒स्ते॒ । यज॑मानः । ह॒विर्भि॒रिति॑ ह॒विः - भिः॒ ॥ अहे॑डमानः । व॒रु॒ण॒ । इ॒ह । बो॒धि॒ । उरु॑शꣳ॒॒सेत्युरु॑ - शꣳ॒॒स॒ । मा । नः॒ । आयुः॑ । प्रेति॑ । मो॒षीः॒ ॥  \newline


\textbf{Krama Paata} \newline

यु॒ष्मानी॑तो॒ अभ॑यम् । अभ॑य॒म् ज्योतिः॑ । ज्योति॑रश्याम् । अ॒श्या॒मित्य॑श्याम् ॥ आ॒दि॒त्याना॒मव॑सा । अव॑सा॒ नूत॑नेन । नूत॑नेन सक्षी॒महि॑ । स॒क्षी॒महि॒ शर्म॑णा । शर्म॑णा॒ शन्त॑मेन । शन्त॑मे॒नेति॒ शम् - त॒मे॒न॒ ॥ अ॒ना॒गा॒स्त्वे अ॑दिति॒त्वे । अ॒ना॒गा॒स्त्व इत्य॑नागाः - त्वे । अ॒दि॒ति॒त्वे तु॒रासः॑ । अ॒दि॒ति॒त्व इत्य॑दिति - त्वे । तु॒रास॑ इ॒मम् । इ॒मं ॅय॒ज्ञ्म् । य॒ज्ञ्म् द॑धतु । द॒ध॒तु॒ श्रोष॑माणाः । श्रोष॑माणा॒ इति॒ श्रोष॑माणाः ॥ इ॒मं मे᳚ । मे॒ व॒रु॒ण॒ । व॒रु॒ण॒ श्रु॒धि॒ । श्रु॒धी॒ हव᳚म् । हव॑म॒द्य । अ॒द्या च॑ । च॒ मृ॒ड॒य॒ । मृ॒ड॒येति॑ मृडय ॥ त्वाम॑व॒स्युः । अ॒व॒स्युरा । आ च॑के । च॒क॒ इति॑ चके ॥ तत् त्वा᳚ । त्वा॒ या॒मि॒ । या॒मि॒ ब्रह्म॑णा । ब्रह्म॑णा॒ वन्द॑मानः । वन्द॑मान॒स्तत् । तदा । आ शा᳚स्ते । शा॒स्ते॒ यज॑मानः । यज॑मानो ह॒विर्भिः॑ । ह॒विर्भि॒रिति॑ ह॒विः - भिः॒ ॥ अहे॑डमानो वरुण । व॒रु॒णे॒ह । इ॒ह बो॑धि । बो॒ध्युरु॑शꣳस । उरु॑शꣳस॒ मा । उरु॑शꣳ॒॒सेत्युरु॑ - शꣳ॒॒स॒ । मा नः॑ । न॒ आयुः॑ । आयुः॒ प्र । प्र मो॑षीः । मो॒षी॒रिति॑ मोषीः । \newline

\textbf{Jatai Paata} \newline

1. यु॒ष्मानी॑तो॒ अभ॑य॒ मभ॑यं ॅयु॒ष्मानी॑तो यु॒ष्मानी॑तो॒ अभ॑यम् । \newline
2. अभ॑य॒म् ज्योति॒र् ज्योति॒ रभ॑य॒ मभ॑य॒म् ज्योतिः॑ । \newline
3. ज्योति॑ रश्या मश्या॒म् ज्योति॒र् ज्योति॑ रश्याम् । \newline
4. अ॒श्या॒मित्य॑श्याम् । \newline
5. आ॒दि॒त्याना॒ मव॒सा ऽव॑सा ऽऽदि॒त्याना॑ मादि॒त्याना॒ मव॑सा । \newline
6. अव॑सा॒ नूत॑नेन॒ नूत॑ने॒नाव॒सा ऽव॑सा॒ नूत॑नेन । \newline
7. नूत॑नेन सक्षी॒महि॑ सक्षी॒महि॒ नूत॑नेन॒ नूत॑नेन सक्षी॒महि॑ । \newline
8. स॒क्षी॒महि॒ शर्म॑णा॒ शर्म॑णा सक्षी॒महि॑ सक्षी॒महि॒ शर्म॑णा । \newline
9. शर्म॑णा॒ शन्त॑मेन॒ शन्त॑मेन॒ शर्म॑णा॒ शर्म॑णा॒ शन्त॑मेन । \newline
10. शन्त॑मे॒नेति॒ शं - त॒मे॒न॒ । \newline
11. अ॒ना॒गा॒स्त्वे अ॑दिति॒त्वे अ॑दिति॒त्वे॑ ऽनागा॒स्त्वे॑ ऽनागा॒स्त्वे अ॑दिति॒त्वे । \newline
12. अ॒ना॒गा॒स्त्व इत्य॑नागाः - त्वे । \newline
13. अ॒दि॒ति॒त्वे तु॒रास॑ स्तु॒रासो॑ अदिति॒त्वे अ॑दिति॒त्वे तु॒रासः॑ । \newline
14. अ॒दि॒ति॒त्व इत्य॑दिति - त्वे । \newline
15. तु॒रास॑ इ॒म मि॒मम् तु॒रास॑ स्तु॒रास॑ इ॒मम् । \newline
16. इ॒मं ॅय॒ज्ञ्ं ॅय॒ज्ञ् मि॒म मि॒मं ॅय॒ज्ञ्म् । \newline
17. य॒ज्ञ्म् द॑धतु दधतु य॒ज्ञ्ं ॅय॒ज्ञ्म् द॑धतु । \newline
18. द॒ध॒तु॒ श्रोष॑माणाः॒ श्रोष॑माणा दधतु दधतु॒ श्रोष॑माणाः । \newline
19. श्रोष॑माणा॒ इति॒ श्रोष॑माणाः । \newline
20. इ॒मम् मे॑ म इ॒म मि॒मम् मे᳚ । \newline
21. मे॒ व॒रु॒ण॒ व॒रु॒ण॒ मे॒ मे॒ व॒रु॒ण॒ । \newline
22. व॒रु॒ण॒ श्रु॒धि॒ श्रु॒धि॒ व॒रु॒ण॒ व॒रु॒ण॒ श्रु॒धि॒ । \newline
23. श्रु॒धी॒ हवꣳ॒॒ हवꣳ॑ श्रुधि श्रुधी॒ हव᳚म् । \newline
24. हव॑ म॒द्याद्य हवꣳ॒॒ हव॑ म॒द्य । \newline
25. अ॒द्या च॑ चा॒द्याद्या च॑ । \newline
26. च॒ मृ॒ड॒य॒ मृ॒ड॒य॒ च॒ च॒ मृ॒ड॒य॒ । \newline
27. मृ॒ड॒येति॑ मृडय । \newline
28. त्वा म॑व॒स्यु र॑व॒स्यु स्त्वाम् त्वा म॑व॒स्युः । \newline
29. अ॒व॒स्युरा ऽव॒स्यु र॑व॒स्युरा । \newline
30. आ च॑के चक॒ आ च॑के । \newline
31. च॒क॒ इति॑ चके । \newline
32. तत् त्वा᳚ त्वा॒ तत् तत् त्वा᳚ । \newline
33. त्वा॒ या॒मि॒ या॒मि॒ त्वा॒ त्वा॒ या॒मि॒ । \newline
34. या॒मि॒ ब्रह्म॑णा॒ ब्रह्म॑णा यामि यामि॒ ब्रह्म॑णा । \newline
35. ब्रह्म॑णा॒ वन्द॑मानो॒ वन्द॑मानो॒ ब्रह्म॑णा॒ ब्रह्म॑णा॒ वन्द॑मानः । \newline
36. वन्द॑मान॒ स्तत् तद् वन्द॑मानो॒ वन्द॑मान॒ स्तत् । \newline
37. तदा तत् तदा । \newline
38. आ शा᳚स्ते शास्त॒ आ शा᳚स्ते । \newline
39. शा॒स्ते॒ यज॑मानो॒ यज॑मानः शास्ते शास्ते॒ यज॑मानः । \newline
40. यज॑मानो ह॒विर्भि॑र्. ह॒विर्भि॒र् यज॑मानो॒ यज॑मानो ह॒विर्भिः॑ । \newline
41. ह॒विर्भि॒रिति॑ ह॒विः - भिः॒ । \newline
42. अहे॑डमानो वरुण वरु॒णा हे॑डमा॒नो ऽहे॑डमानो वरुण । \newline
43. व॒रु॒णे॒ हे ह व॑रुण वरुणे॒ ह । \newline
44. इ॒ह बो॑धि बोधी॒हे ह बो॑धि । \newline
45. बो॒ध्युरु॑शꣳ॒॒ सोरु॑शꣳस बोधि बो॒ध्युरु॑शꣳस । \newline
46. उरु॑शꣳस॒ मा मोरु॑शꣳ॒॒ सोरु॑शꣳस॒ मा । \newline
47. उरु॑शꣳ॒॒सेत्युरु॑ - शꣳ॒॒स॒ । \newline
48. मा नो॑ नो॒ मा मा नः॑ । \newline
49. न॒ आयु॒ रायु॑र् नो न॒ आयुः॑ । \newline
50. आयुः॒ प्र प्रायु॒ रायुः॒ प्र । \newline
51. प्र मो॑षीर् मोषीः॒ प्र प्र मो॑षीः । \newline
52. मो॒षी॒रिति॑ मोषीः । \newline

\textbf{Ghana Paata } \newline

1. यु॒ष्मानी॑तो॒ अभ॑य॒ मभ॑यं ॅयु॒ष्मानी॑तो यु॒ष्मानी॑तो॒ अभ॑य॒म् ज्योति॒र् ज्योति॒रभ॑यं ॅयु॒ष्मानी॑तो यु॒ष्मानी॑तो॒ अभ॑य॒म् ज्योतिः॑ । \newline
2. अभ॑य॒म् ज्योति॒र् ज्योति॒ रभ॑य॒ मभ॑य॒म् ज्योति॑ रश्या मश्या॒म् ज्योति॒ रभ॑य॒ मभ॑य॒म् ज्योति॑ रश्याम् । \newline
3. ज्योति॑ रश्या मश्या॒म् ज्योति॒र् ज्योति॑ रश्याम् । \newline
4. अ॒श्या॒मित्य॑श्याम् । \newline
5. आ॒दि॒त्याना॒ मव॒सा ऽव॑सा ऽऽदि॒त्याना॑ मादि॒त्याना॒ मव॑सा॒ नूत॑नेन॒ नूत॑ने॒नाव॑सा ऽऽदि॒त्याना॑ मादि॒त्याना॒ मव॑सा॒ नूत॑नेन । \newline
6. अव॑सा॒ नूत॑नेन॒ नूत॑ने॒नाव॒सा ऽव॑सा॒ नूत॑नेन सक्षी॒महि॑ सक्षी॒महि॒ नूत॑ने॒नाव॒सा ऽव॑सा॒ नूत॑नेन सक्षी॒महि॑ । \newline
7. नूत॑नेन सक्षी॒महि॑ सक्षी॒महि॒ नूत॑नेन॒ नूत॑नेन सक्षी॒महि॒ शर्म॑णा॒ शर्म॑णा सक्षी॒महि॒ नूत॑नेन॒ नूत॑नेन सक्षी॒महि॒ शर्म॑णा । \newline
8. स॒क्षी॒महि॒ शर्म॑णा॒ शर्म॑णा सक्षी॒महि॑ सक्षी॒महि॒ शर्म॑णा॒ शन्त॑मेन॒ शन्त॑मेन॒ शर्म॑णा सक्षी॒महि॑ सक्षी॒महि॒ शर्म॑णा॒ शन्त॑मेन । \newline
9. शर्म॑णा॒ शन्त॑मेन॒ शन्त॑मेन॒ शर्म॑णा॒ शर्म॑णा॒ शन्त॑मेन । \newline
10. शन्त॑मे॒नेति॒ शं - त॒मे॒न॒ । \newline
11. अ॒ना॒गा॒स्त्वे अ॑दिति॒त्वे अ॑दिति॒त्वे॑ ऽनागा॒स्त्वे॑ ऽनागा॒स्त्वे अ॑दिति॒त्वे तु॒रास॑ स्तु॒रासो॑ अदिति॒त्वे॑ ऽनागा॒स्त्वे॑ ऽनागा॒स्त्वे अ॑दिति॒त्वे तु॒रासः॑ । \newline
12. अ॒ना॒गा॒स्त्व इत्य॑नागाः - त्वे । \newline
13. अ॒दि॒ति॒त्वे तु॒रास॑ स्तु॒रासो॑ अदिति॒त्वे अ॑दिति॒त्वे तु॒रास॑ इ॒म मि॒मम् तु॒रासो॑ अदिति॒त्वे अ॑दिति॒त्वे तु॒रास॑ इ॒मम् । \newline
14. अ॒दि॒ति॒त्व इत्य॑दिति - त्वे । \newline
15. तु॒रास॑ इ॒म मि॒मम् तु॒रास॑ स्तु॒रास॑ इ॒मं ॅय॒ज्ञ्ं ॅय॒ज्ञ् मि॒मम् तु॒रास॑ स्तु॒रास॑ इ॒मं ॅय॒ज्ञ्म् । \newline
16. इ॒मं ॅय॒ज्ञ्ं ॅय॒ज्ञ् मि॒म मि॒मं ॅय॒ज्ञ्म् द॑धतु दधतु य॒ज्ञ् मि॒म मि॒मं ॅय॒ज्ञ्म् द॑धतु । \newline
17. य॒ज्ञ्म् द॑धतु दधतु य॒ज्ञ्ं ॅय॒ज्ञ्म् द॑धतु॒ श्रोष॑माणाः॒ श्रोष॑माणा दधतु य॒ज्ञ्ं ॅय॒ज्ञ्म् द॑धतु॒ श्रोष॑माणाः । \newline
18. द॒ध॒तु॒ श्रोष॑माणाः॒ श्रोष॑माणा दधतु दधतु॒ श्रोष॑माणाः । \newline
19. श्रोष॑माणा॒ इति॒ श्रोष॑माणाः । \newline
20. इ॒मम् मे॑ म इ॒म मि॒मम् मे॑ वरुण वरुण म इ॒म मि॒मम् मे॑ वरुण । \newline
21. मे॒ व॒रु॒ण॒ व॒रु॒ण॒ मे॒ मे॒ व॒रु॒ण॒ श्रु॒धि॒ श्रु॒धि॒ व॒रु॒ण॒ मे॒ मे॒ व॒रु॒ण॒ श्रु॒धि॒ । \newline
22. व॒रु॒ण॒ श्रु॒धि॒ श्रु॒धि॒ व॒रु॒ण॒ व॒रु॒ण॒ श्रु॒धी॒ हवꣳ॒॒ हवꣳ॑ श्रुधि वरुण वरुण श्रुधी॒ हव᳚म् । \newline
23. श्रु॒धी॒ हवꣳ॒॒ हवꣳ॑ श्रुधि श्रुधी॒ हव॑ म॒द्याद्य हवꣳ॑ श्रुधि श्रुधी॒ हव॑ म॒द्य । \newline
24. हव॑ म॒द्याद्य हवꣳ॒॒ हव॑ म॒द्या च॑ चा॒द्य हवꣳ॒॒ हव॑ म॒द्या च॑ । \newline
25. अ॒द्या च॑ चा॒द्याद्या च॑ मृडय मृडय चा॒द्याद्या च॑ मृडय । \newline
26. च॒ मृ॒ड॒य॒ मृ॒ड॒य॒ च॒ च॒ मृ॒ड॒य॒ । \newline
27. मृ॒ड॒येति॑ मृडय । \newline
28. त्वा म॑व॒स्यु र॑व॒स्यु स्त्वाम् त्वा म॑व॒स्युरा ऽव॒स्यु स्त्वाम् त्वा म॑व॒स्युरा । \newline
29. अ॒व॒स्युरा ऽव॒स्यु र॑व॒स्युरा च॑के चक॒ आ ऽव॒स्यु र॑व॒स्युरा च॑के । \newline
30. आ च॑के चक॒ आ च॑के । \newline
31. च॒क॒ इति॑ चके । \newline
32. तत् त्वा᳚ त्वा॒ तत् तत् त्वा॑ यामि यामि त्वा॒ तत् तत् त्वा॑ यामि । \newline
33. त्वा॒ या॒मि॒ या॒मि॒ त्वा॒ त्वा॒ या॒मि॒ ब्रह्म॑णा॒ ब्रह्म॑णा यामि त्वा त्वा यामि॒ ब्रह्म॑णा । \newline
34. या॒मि॒ ब्रह्म॑णा॒ ब्रह्म॑णा यामि यामि॒ ब्रह्म॑णा॒ वन्द॑मानो॒ वन्द॑मानो॒ ब्रह्म॑णा यामि यामि॒ ब्रह्म॑णा॒ वन्द॑मानः । \newline
35. ब्रह्म॑णा॒ वन्द॑मानो॒ वन्द॑मानो॒ ब्रह्म॑णा॒ ब्रह्म॑णा॒ वन्द॑मान॒ स्तत् तद् वन्द॑मानो॒ ब्रह्म॑णा॒ ब्रह्म॑णा॒ वन्द॑मान॒ स्तत् । \newline
36. वन्द॑मान॒ स्तत् तद् वन्द॑मानो॒ वन्द॑मान॒ स्तदा तद् वन्द॑मानो॒ वन्द॑मान॒ स्तदा । \newline
37. तदा तत् तदा शा᳚स्ते शास्त॒ आ तत् तदा शा᳚स्ते । \newline
38. आ शा᳚स्ते शास्त॒ आ शा᳚स्ते॒ यज॑मानो॒ यज॑मानः शास्त॒ आ शा᳚स्ते॒ यज॑मानः । \newline
39. शा॒स्ते॒ यज॑मानो॒ यज॑मानः शास्ते शास्ते॒ यज॑मानो ह॒विर्भि॑र्. ह॒विर्भि॒र् यज॑मानः शास्ते शास्ते॒ यज॑मानो ह॒विर्भिः॑ । \newline
40. यज॑मानो ह॒विर्भि॑र्. ह॒विर्भि॒र् यज॑मानो॒ यज॑मानो ह॒विर्भिः॑ । \newline
41. ह॒विर्भि॒रिति॑ ह॒विः - भिः॒ । \newline
42. अहे॑डमानो वरुण वरु॒णा हे॑डमा॒नो ऽहे॑डमानो वरुणे॒ हे ह व॑रु॒णा हे॑डमा॒नो ऽहे॑डमानो वरुणे॒ ह । \newline
43. व॒रु॒णे॒ हे ह व॑रुण वरुणे॒ ह बो॑धि बोधी॒ह व॑रुण वरुणे॒ ह बो॑धि । \newline
44. इ॒ह बो॑धि बोधी॒हे ह बो॒ध्युरु॑शꣳ॒॒ सोरु॑शꣳस बोधी॒हे ह बो॒ध्युरु॑शꣳस । \newline
45. बो॒ध्युरु॑शꣳ॒॒ सोरु॑शꣳस बोधि बो॒ध्युरु॑शꣳस॒ मा मोरु॑शꣳस बोधि बो॒ध्युरु॑शꣳस॒ मा । \newline
46. उरु॑शꣳस॒ मा मोरु॑शꣳ॒॒ सोरु॑शꣳस॒ मा नो॑ नो॒ मोरु॑शꣳ॒॒ सोरु॑शꣳस॒ मा नः॑ । \newline
47. उरु॑शꣳ॒॒सेत्युरु॑ - शꣳ॒॒स॒ । \newline
48. मा नो॑ नो॒ मा मा न॒ आयु॒ रायु॑र् नो॒ मा मा न॒ आयुः॑ । \newline
49. न॒ आयु॒ रायु॑र् नो न॒ आयुः॒ प्र प्रायु॑र् नो न॒ आयुः॒ प्र । \newline
50. आयुः॒ प्र प्रायु॒ रायुः॒ प्र मो॑षीर् मोषीः॒ प्रायु॒ रायुः॒ प्र मो॑षीः । \newline
51. प्र मो॑षीर् मोषीः॒ प्र प्र मो॑षीः । \newline
52. मो॒षी॒रिति॑ मोषीः॒ । \newline
\pagebreak


\end{document}