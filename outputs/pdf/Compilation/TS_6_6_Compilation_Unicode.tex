\documentclass[17pt]{extarticle}
\usepackage{babel}
\usepackage{fontspec}
\usepackage{polyglossia}
\usepackage{extsizes}

\usepackage{color}   %May be necessary if you want to color links
\usepackage{hyperref}
\hypersetup{
    colorlinks=true, %set true if you want colored links
    linktoc=all,     %set to all if you want both sections and subsections linked
    linkcolor=black,  %choose some color if you want links to stand out
}

\setmainlanguage{sanskrit}
\setotherlanguages{english} %% or other languages
\setlength{\parindent}{0pt}
\pagestyle{myheadings}
\newfontfamily\devanagarifont[Script=Devanagari]{AdishilaVedic}
\renewcommand{\theHsection}{\thepart.section.\thesection}

\newcommand{\VAR}[1]{}
\newcommand{\BLOCK}[1]{}




\begin{document}
\begin{titlepage}
    \begin{center}
 
\begin{sanskrit}
    { \Large
    कृष्ण यजुर्वेदीय तैत्तिरीय संहिता,पद,जटा,घन पाठः 
    }
    \\
    \vspace{2.5cm}
    \mbox{ \Large
    6.6       षष्ठकाण्डे षष्ठः प्रश्नः - सोममन्त्रब्राह्मणनिरूपणं   }
\end{sanskrit}
\end{center}

\end{titlepage}
\tableofcontents
\phantomsection
\pagebreak

\markright{ TS 6.6.1.1  \hfill https://www.vedavms.in \hfill}

\section{ TS 6.6.1.1 }

\textbf{TS 6.6.1.1 } \newline
\textbf{Samhita Paata} \newline

सु॒व॒र्गाय॒ वा ए॒तानि॑ लो॒काय॑ हूयन्ते॒ यद्-दा᳚क्षि॒णानि॒ द्वाभ्यां॒ गार्.ह॑पत्ये जुहोति द्वि॒पाद्-यज॑मानः॒ प्रति॑ष्ठित्या॒ आग्नी᳚द्ध्रे जुहोत्य॒न्तरि॑क्ष ए॒वाऽऽ*क्र॑मते॒ सदो॒ऽभ्यैति॑ सुव॒र्गमे॒वैनं॑ ॅलो॒कं ग॑मयति सौ॒रीभ्या॑मृ॒ग्भ्यां गार्.ह॑पत्ये जुहोत्य॒मुमे॒वैनं ॅलो॒कꣳ स॒मारो॑हयति॒ नय॑वत्य॒र्चाऽऽ*ग्नी᳚द्ध्रे जुहोति सुव॒र्गस्य॑ लो॒कस्या॒भिनी᳚त्यै॒ दिवं॑ गच्छ॒ सुवः॑ प॒तेति॒ हिर॑ण्यꣳ- [  ] \newline

\textbf{Pada Paata} \newline

सु॒व॒र्गायेति॑ सुवः - गाय॑ । वै । ए॒तानि॑ । लो॒काय॑ । हू॒य॒न्ते॒ । यत् । दा॒क्षि॒णानि॑ । द्वाभ्या᳚म् । गार्.ह॑पत्य॒ इति॒ गार्.ह॑ - प॒त्ये॒ । जु॒हो॒ति॒ । द्वि॒पादिति॑ द्वि - पात् । यज॑मानः । प्रति॑ष्ठित्या॒ इति॒ प्रति॑ - स्थि॒त्यै॒ । आग्नी᳚द्ध्र॒ इत्याग्नि॑ - इ॒द्ध्रे॒ । जु॒हो॒ति॒ । अ॒न्तरि॑क्षे । ए॒व । एति॑ । क्र॒म॒ते॒ । सदः॑ । अ॒भि । एति॑ । ए॒ति॒ । सु॒व॒र्गमिति॑ सुवः - गम् । ए॒व । ए॒न॒म् । लो॒कम् । ग॒म॒य॒ति॒ । सौ॒रीभ्या᳚म् । ऋ॒ग्भ्यामित्यृ॑क् - भ्याम् । गार्.ह॑पत्य॒ इति॒ गार्.ह॑-प॒त्ये॒ । जु॒हो॒ति॒ । अ॒मुम् । ए॒व । ए॒न॒म् । लो॒कम् । स॒मारो॑हय॒तीति॑ सं - आरो॑हयति । नय॑व॒त्येति॒ नय॑ - व॒त्या॒ । ऋ॒चा । आग्नी᳚द्ध्र॒ इत्याग्नि॑ - इ॒द्ध्रे॒ । जु॒हो॒ति॒ । सु॒व॒र्गस्येति॑ सुवः - गस्य॑ । लो॒कस्य॑ । अ॒भिनी᳚त्य॒ इत्य॒भि -नी॒त्यै॒ । दिव᳚म् । ग॒च्छ॒ । सुवः॑ । प॒त॒ । इति॑ । हिर॑ण्यम् ।  \newline




\markright{ TS 6.6.1.2  \hfill https://www.vedavms.in \hfill}

\section{ TS 6.6.1.2 }

\textbf{TS 6.6.1.2 } \newline
\textbf{Samhita Paata} \newline

हु॒त्वोद्-गृ॑ह्णाति सुव॒र्गमे॒वैनं॑ ॅलो॒कं ग॑मयति रू॒पेण॑ वो रू॒पम॒भ्यैमीत्या॑ह रू॒पेण॒ ह्या॑साꣳ रू॒पम॒भ्यैति॒ यद्धिर॑ण्येन तु॒थो वो॑ वि॒श्ववे॑दा॒ वि भ॑ज॒त्वित्या॑ह तु॒थो ह॑ स्म॒ वै वि॒श्ववे॑दा दे॒वानां॒ दक्षि॑णा॒ वि भ॑जति॒ तेनै॒वैना॒ वि भ॑जत्ये॒ तत् ते॑ अग्ने॒ राध॒- [  ] \newline

\textbf{Pada Paata} \newline

हु॒त्वा । उदिति॑ । गृ॒ह्णा॒ति॒ । सु॒व॒र्गमिति॑ सुवः - गम् । ए॒व । ए॒न॒म् । लो॒कम् । ग॒म॒य॒ति॒ । रू॒पेण॑ । वः॒ । रू॒पम् । अ॒भि । एति॑ । ए॒मि॒ । इति॑ । आ॒ह॒ । रू॒पेण॑ । हि । आ॒सा॒म् । रू॒पम् । अ॒भि । एति॑ । एति॑ । यत् । हिर॑ण्येन । तु॒थः । वः॒ । वि॒श्ववे॑दा॒ इति॑ वि॒श्व - वे॒दाः॒ । वीति॑ । भ॒ज॒तु॒ । इति॑ । आ॒ह॒ । तु॒थः । ह॒ । स्म॒ । वै । वि॒श्ववे॑दा॒ इति॑ वि॒श्व - वे॒दाः॒ । दे॒वाना᳚म् । दक्षि॑णाः । वीति॑ । भ॒ज॒ति॒ । तेन॑ । ए॒व । ए॒नाः॒ । वीति॑ । भ॒ज॒ति॒ । ए॒तत् । ते॒ । अ॒ग्ने॒ । राधः॑ ।  \newline




\markright{ TS 6.6.1.3  \hfill https://www.vedavms.in \hfill}

\section{ TS 6.6.1.3 }

\textbf{TS 6.6.1.3 } \newline
\textbf{Samhita Paata} \newline

ऐति॒ सोम॑च्युत॒मित्या॑ह॒ सोम॑च्युतꣳ॒॒ ह्य॑स्य॒ राध॒ ऐति॒ तन्मि॒त्रस्य॑ प॒था न॒येत्या॑ह॒ शान्त्या॑ ऋ॒तस्य॑ प॒था प्रेत॑ च॒न्द्र द॑क्षिणा॒ इत्या॑ह स॒त्यं ॅवा ऋ॒तꣳ स॒त्येनै॒वैना॑ ऋ॒तेन॒ वि भ॑जति य॒ज्ञ्स्य॑ प॒था सु॑वि॒ता नय॑न्ती॒रित्या॑ह य॒ज्ञ्स्य॒ ह्ये॑ताः प॒था यन्ति॒ यद्-दक्षि॑णा ब्राह्म॒णम॒द्य रा᳚द्ध्यास॒- [  ] \newline

\textbf{Pada Paata} \newline

एति॑ । ए॒ति॒ । सोम॑च्युत॒मिति॒ सोम॑ - च्यु॒त॒म् । इति॑ । आ॒ह॒ । सोम॑च्युत॒मिति॒ सोम॑ - च्यु॒त॒म् । हि । अ॒स्य॒ । राधः॑ । एति॑ । एति॑ । तत् । मि॒त्रस्य॑ । प॒था । न॒य॒ । इति॑ । आ॒ह॒ । शान्त्यै᳚ । ऋ॒तस्य॑ । प॒था । प्रेति॑ । इ॒त॒ । च॒न्द्रद॑क्षिणा॒ इति॑ च॒न्द्र - द॒क्षि॒णाः॒ । इति॑ । आ॒ह॒ । स॒त्यम् । वै । ऋ॒तम् । स॒त्येन॑ । ए॒व । ए॒नाः॒ । ऋ॒तेन॑ । वीति॑ । भ॒ज॒ति॒ । य॒ज्ञ्स्य॑ । प॒था । सु॒वि॒ता । नय॑न्तीः । इति॑ । आ॒ह॒ । य॒ज्ञ्स्य॑ । हि । ए॒ताः । प॒था । यन्ति॑ । यत् । दक्षि॑णाः । ब्रा॒ह्म॒णम् । अ॒द्य । रा॒द्ध्या॒स॒म् ।  \newline




\markright{ TS 6.6.1.4  \hfill https://www.vedavms.in \hfill}

\section{ TS 6.6.1.4 }

\textbf{TS 6.6.1.4 } \newline
\textbf{Samhita Paata} \newline

मृषि॑मार्.षे॒यमित्या॑है॒ष वै ब्रा᳚ह्म॒ण ऋषि॑रार्.षे॒यो यः शु॑श्रु॒वान् तस्मा॑दे॒वमा॑ह॒ वि सुवः॒ पश्य॒ व्य॑न्तरि॑क्ष॒मित्या॑ह सुव॒र्गमे॒वैनं॑ ॅलो॒कं ग॑मयति॒ यत॑स्व सद॒स्यै॑रित्या॑ह मित्र॒त्वाया॒स्मद्दा᳚त्रा देव॒त्रा ग॑च्छत॒ मधु॑मतीः प्रदा॒तार॒मा वि॑श॒तेत्या॑ह व॒यमि॒ह प्र॑दा॒तारः॒ स्मो᳚ऽस्मान॒मुत्र॒ मधु॑मती॒रा वि॑श॒तेति॒- [  ] \newline

\textbf{Pada Paata} \newline

ऋषि᳚म् । आ॒र्॒.षे॒यम् । इति॑ । आ॒ह॒ । ए॒षः । वै । ब्रा॒ह्म॒णः । ऋषिः॑ । आ॒र्॒.षे॒यः । यः । शु॒श्रु॒वान् । तस्मा᳚त् । ए॒वम् । आ॒ह॒ । वीति॑ । सुवः॑ । पश्य॑ । वीति॑ । अ॒न्तरि॑क्षम् । इति॑ । आ॒ह॒ । सु॒व॒र्गमिति॑ सुवः-गम् । ए॒व । ए॒न॒म् । लो॒कम् । ग॒म॒य॒ति॒ । यत॑स्व । स॒द॒स्यैः᳚ । इति॑ । आ॒ह॒ । मि॒त्र॒त्वायेति॑ मित्र - त्वाय॑ । अ॒स्मद्दा᳚त्रा॒ इत्य॒स्मत् - दा॒त्राः॒ । दे॒व॒त्रेति॑ देव - त्रा । ग॒च्छ॒त॒ । मधु॑मती॒रिति॒ मधु॑ - म॒तीः॒ । प्र॒दा॒तार॒मिति॑ प्र - दा॒तार᳚म् । एति॑ । वि॒श॒त॒ । इति॑ । आ॒ह॒ । व॒यम् । इ॒ह । प्र॒दा॒तार॒ इति॑ प्र - दा॒तारः॑ । स्मः । अ॒स्मान् । अ॒मुत्र॑ । मधु॑मती॒रिति॒ मधु॑ - म॒तीः॒ । एति॑ । वि॒श॒त॒ । इति॑ ।  \newline




\markright{ TS 6.6.1.5  \hfill https://www.vedavms.in \hfill}

\section{ TS 6.6.1.5 }

\textbf{TS 6.6.1.5 } \newline
\textbf{Samhita Paata} \newline

वावैतदा॑ह॒ हिर॑ण्यं ददाति॒ ज्योति॒र्वै हिर॑ण्यं॒ ज्योति॑रे॒व पु॒रस्ता᳚द्धत्ते सुव॒र्गस्य॑ लो॒कस्यानु॑ख्यात्या अ॒ग्नीधे॑ ददात्य॒ग्निमु॑खाने॒वर्तून् प्री॑णाति ब्र॒ह्मणे॑ ददाति॒ प्रसू᳚त्यै॒ होत्रे॑ ददात्या॒त्मा वा ए॒ष य॒ज्ञ्स्य॒ यद्धोता॒ऽऽत्मान॑मे॒व य॒ज्ञ्स्य॒ दक्षि॑णाभिः॒ सम॑र्द्धयति ॥ \newline

\textbf{Pada Paata} \newline

वाव । ए॒तत् । आ॒ह॒ । हिर॑ण्यम् । द॒दा॒ति॒ । ज्योतिः॑ । वै । हिर॑ण्यम् । ज्योतिः॑ । ए॒व । पु॒रस्ता᳚त् । ध॒त्ते॒ । सु॒व॒र्गस्येति॑ सुवः - गस्य॑ । लो॒कस्य॑ । अनु॑ख्यात्या॒ इत्यनु॑ - ख्या॒त्यै॒ । अ॒ग्नीध॒ इत्य॑ग्नि - इधे᳚ । द॒दा॒ति॒ । अ॒ग्निमु॑खा॒नित्य॒ग्नि - मु॒खा॒न् । ए॒व । ऋ॒तून् । प्री॒णा॒ति॒ । ब्र॒ह्मणे᳚ । द॒दा॒ति॒ । प्रसू᳚त्या॒ इति॒ प्र-सू॒त्यै॒ । होत्रे᳚ । द॒दा॒ति॒ । आ॒त्मा । वै । ए॒षः । य॒ज्ञ्स्य॑ । यत् । होता᳚ । आ॒त्मान᳚म् । ए॒व । य॒ज्ञ्स्य॑ । दक्षि॑णाभिः । समिति॑ । अ॒द्‌र्ध॒य॒ति॒ ॥  \newline




\markright{ TS 6.6.2.1  \hfill https://www.vedavms.in \hfill}

\section{ TS 6.6.2.1 }

\textbf{TS 6.6.2.1 } \newline
\textbf{Samhita Paata} \newline

स॒मि॒ष्ट॒ य॒जूꣳषि॑ जुहोति य॒ज्ञ्स्य॒ समि॑ष्ट्यै॒ यद्वै य॒ज्ञ्स्य॑ क्रू॒रं ॅयद्-विलि॑ष्टं॒ ॅयद॒त्येति॒ यन्नात्येति॒ यद॑तिक॒रोति॒ यन्नापि॑ क॒रोति॒ तदे॒व तैः प्री॑णाति॒ नव॑ जुहोति॒ नव॒ वै पुरु॑षे प्रा॒णाः पुरु॑षेण य॒ज्ञ्ः संमि॑तो॒ यावा॑ने॒व य॒ज्ञ्स्तं प्री॑णाति॒ षड् ऋग्मि॑याणि जुहोति॒ षड्वा ऋ॒तव॑ ऋ॒तूने॒व प्री॑णाति॒ त्रीणि॒ यजूꣳ॑षि॒- [  ] \newline

\textbf{Pada Paata} \newline

स॒मि॒ष्ट॒य॒जूꣳषीति॑ समिष्ट - य॒जूꣳषि॑ । जु॒हो॒ति॒ । य॒ज्ञ्स्य॑ । समि॑ष्ट्या॒ इति॒ सं - इ॒ष्ट्यै॒ । यत् । वै । य॒ज्ञ्स्य॑ । क्रू॒रम् । यत् । विलि॑ष्ट॒मिति॒ वि - लि॒ष्ट॒म् । यत् । अ॒त्येतीत्य॑ति - एति॑ । यत् । न । अ॒त्येतीत्य॑ति - एति॑ । यत् । अ॒ति॒क॒रोतीत्य॑ति - क॒रोति॑ । यत् । न । अपीति॑ । क॒रोति॑ । तत् । ए॒व । तैः । प्री॒णा॒ति॒ । नव॑ । जु॒हो॒ति॒ । नव॑ । वै । पुरु॑षे । प्रा॒णा इति॑ प्र - अ॒नाः । पुरु॑षेण । य॒ज्ञ्ः । सम्मि॑त॒ इति॒ सं - मि॒तः॒ । यावान्॑ । ए॒व । य॒ज्ञ्ः । तम् । प्री॒णा॒ति॒ । षट् । ऋग्मि॑याणि । जु॒हो॒ति॒ । षट् । वै । ऋ॒तवः॑ । ऋ॒तून् । ए॒व । प्री॒णा॒ति॒ । त्रीणि॑ । यजूꣳ॑षि ।  \newline




\markright{ TS 6.6.2.2  \hfill https://www.vedavms.in \hfill}

\section{ TS 6.6.2.2 }

\textbf{TS 6.6.2.2 } \newline
\textbf{Samhita Paata} \newline

त्रय॑ इ॒मे लो॒का इ॒माने॒व लो॒कान् प्री॑णाति॒ यज्ञ्॑ य॒ज्ञ्ं ग॑च्छ य॒ज्ञ्प॑तिं ग॒च्छेत्या॑ह य॒ज्ञ्प॑तिमे॒वैनं॑ गमयति॒ स्वां ॅयोनिं॑ ग॒च्छेत्या॑ह॒ स्वामे॒वैनं॒ ॅयोनिं॑ गमयत्ये॒ष ते॑ य॒ज्ञो य॑ज्ञ्पते स॒हसू᳚क्तवाकः सु॒वीर॒ इत्या॑ह॒ यज॑मान ए॒व वी॒र्यं॑ दधाति वासि॒ष्ठो ह॑ सात्यह॒व्यो दे॑वभा॒गं प॑प्रच्छ॒ यथ् सृञ्ज॑यान् बहुया॒जिनोऽयी॑यजो य॒ज्ञे- [  ] \newline

\textbf{Pada Paata} \newline

त्रयः॑ । इ॒मे । लो॒काः । इ॒मान् । ए॒व । लो॒कान् । प्री॒णा॒ति॒ । यज्ञ्॑ । य॒ज्ञ्म् । ग॒च्छ॒ । य॒ज्ञ्प॑ति॒मिति॑ य॒ज्ञ् - प॒ति॒म् । ग॒च्छ॒ । इति॑ । आ॒ह॒ । य॒ज्ञ्प॑ति॒मिति॑ य॒ज्ञ् - प॒ति॒म् । ए॒व । ए॒न॒म् । ग॒म॒य॒ति॒ । स्वाम् । योनि᳚म् । ग॒च्छ॒ । इति॑ । आ॒ह॒ । स्वाम् । ए॒व । ए॒न॒म् । योनि᳚म् । ग॒म॒य॒ति॒ । ए॒षः । ते॒ । य॒ज्ञ्ः । य॒ज्ञ्॒प॒त॒ इति॑ यज्ञ् - प॒ते॒ । स॒हसू᳚क्तवाक॒ इति॑ स॒हसू᳚क्त - वा॒कः॒ । सु॒वीर॒ इति॑ सु - वीरः॑ । इति॑ । आ॒ह॒ । यज॑माने । ए॒व । वी॒र्य᳚म् । द॒धा॒ति॒ । वा॒सि॒ष्ठः । ह॒ । सा॒त्य॒ह॒व्य इति॑ सात्य - ह॒व्यः । दे॒व॒भा॒गमिति॑ देव - भा॒गम् । प॒प्र॒च्छ॒ । यत् । सृञ्ज॑यान् । ब॒हु॒या॒जिन॒ इति॑ बहु - या॒जिनः॑ । अयी॑यजः । य॒ज्ञे ।  \newline




\markright{ TS 6.6.2.3  \hfill https://www.vedavms.in \hfill}

\section{ TS 6.6.2.3 }

\textbf{TS 6.6.2.3 } \newline
\textbf{Samhita Paata} \newline

य॒ज्ञ्ं प्रत्य॑तिष्ठि॒पा(3) य॒ज्ञ्प॒ता(3)विति॒ स हो॑वाच य॒ज्ञ्प॑ता॒विति॑ स॒त्याद्वै सृञ्ज॑याः॒ परा॑ बभूवु॒रिति॑ होवाच य॒ज्ञे वाव य॒ज्ञ्ः प्र॑ति॒ष्ठाप्य॑ आसी॒द्-यज॑मान॒स्या-प॑राभावा॒येति॒ देवा॑ गातुविदो गा॒तुं ॅवि॒त्त्वा गा॒तु -मि॒तेत्या॑ह य॒ज्ञ् ए॒व य॒ज्ञ्ं प्रति॑ ष्ठापयति॒ यज॑मान॒स्या-प॑राभावाय ॥ \newline

\textbf{Pada Paata} \newline

य॒ज्ञ्म् । प्रतीति॑ । अ॒ति॒ष्ठि॒पा(3)ः । य॒ज्ञ्प॒ता(3)विति॑ य॒ज्ञ्-प॒ता(3)उ । इति॑ । सः । ह॒ । उ॒वा॒च॒ । य॒ज्ञ्प॑ता॒विति॑ य॒ज्ञ्-प॒तौ॒ । इति॑ । स॒त्यात् । वै । सृञ्ज॑याः । परेति॑ । ब॒भू॒वुः॒ । इति॑ । ह॒ । उ॒वा॒च॒ । य॒ज्ञे । वाव । य॒ज्ञ्ः । प्र॒ति॒ष्ठाप्य॒ इति॑ प्रति - स्थाप्यः॑ । आ॒सी॒त् । यज॑मानस्य । अप॑राभावा॒येत्यप॑रा - भा॒वा॒य॒ । इति॑ । देवाः᳚ । गा॒तु॒वि॒द॒ इति॑ गातु - वि॒दः॒ । गा॒तुम् । वि॒त्त्वा । गा॒तुम् । इ॒त॒ । इति॑ । आ॒ह॒ । य॒ज्ञे । ए॒व । य॒ज्ञ्म् । प्रतीति॑ । स्था॒प॒य॒ति॒ । यज॑मानस्य । अप॑राभावा॒येत्यप॑रा- भा॒वा॒य॒ ॥  \newline




\markright{ TS 6.6.3.1  \hfill https://www.vedavms.in \hfill}

\section{ TS 6.6.3.1 }

\textbf{TS 6.6.3.1 } \newline
\textbf{Samhita Paata} \newline

अ॒व॒भृ॒थ॒-य॒जूꣳषि॑ जुहोति॒ यदे॒वार्वा॒चीन॒-मेक॑हायना॒देनः॑ क॒रोति॒ तदे॒व तैरव॑ यजते॒ ऽपो॑ऽवभृ॒थ-मवै᳚त्य॒फ्सु वै वरु॑णः सा॒क्षादे॒व वरु॑ण॒मव॑ यजते॒ वर्त्म॑ना॒ वा अ॒न्वित्य॑ य॒ज्ञ्ꣳ रक्षाꣳ॑सि जिघाꣳसन्ति॒ साम्ना᳚ प्रस्तो॒ताऽन्ववै॑ति॒ साम॒ वै र॑क्षो॒हा रक्ष॑सा॒मप॑हत्यै॒ त्रिर्नि॒धन॒मुपै॑ति॒ त्रय॑ इ॒मे लो॒का ए॒भ्य ए॒व लो॒केभ्यो॒ रक्षाꣳ॒॒- [  ] \newline

\textbf{Pada Paata} \newline

अ॒व॒भृ॒थ॒य॒जूꣳषीत्य॑वभृथ - य॒जूꣳषि॑ । जु॒हो॒ति॒ । यत् । ए॒व । अ॒र्वा॒चीन᳚म् । एक॑हायना॒दित्येक॑ - हा॒य॒ना॒त् । एनः॑ । क॒रोति॑ । तत् । ए॒व । तैः । अवेति॑ । य॒ज॒ते॒ । अ॒पः । अ॒व॒भृ॒थमित्य॑व - भृ॒थम् । अवेति॑ । ए॒ति॒ । अ॒फ्स्वित्य॑प् - सु । वै । वरु॑णः । सा॒क्षादिति॑ स - अ॒क्षात् । ए॒व । वरु॑णम् । अवेति॑ । य॒ज॒ते॒ । वर्त्म॑ना । वै । अ॒न्वित्येत्य॑नु - इत्य॑ । य॒ज्ञ्म् । रक्षाꣳ॑सि । जि॒घाꣳ॒॒स॒न्त॒ । साम्ना᳚ । प्र॒स्तो॒तेति॑ प्र - स्तो॒ता । अ॒न्ववै॒तीत्य॑नु - अवै॑ति । साम॑ । वै । र॒क्षो॒हेति॑ रक्षः - हा । रक्ष॑साम् । अप॑हत्या॒ इत्यप॑ - ह॒त्यै॒ । त्रिः । नि॒धन॒मिति॑ नि - धन᳚म् । उपेति॑ । ए॒ति॒ । त्रयः॑ । इ॒मे । लो॒काः । ए॒भ्यः । ए॒व । लो॒केभ्यः॑ । रक्षाꣳ॑सि ।  \newline




\markright{ TS 6.6.3.2  \hfill https://www.vedavms.in \hfill}

\section{ TS 6.6.3.2 }

\textbf{TS 6.6.3.2 } \newline
\textbf{Samhita Paata} \newline

-स्यप॑ हन्ति॒ पुरु॑षःपुरुषो नि॒धन॒मुपै॑ति॒ पुरु॑षःपुरुषो॒ हि र॑क्ष॒स्वी रक्ष॑सा॒मप॑हत्या उ॒रुꣳ हि राजा॒ वरु॑णश्च॒कारेत्या॑ह॒ प्रति॑ष्ठित्यै श॒तं ते॑ राजन् भि॒षजः॑ स॒हस्र॒मित्या॑ह भेष॒जमे॒वास्मै॑ करोत्य॒भिष्ठि॑तो॒ वरु॑णस्य॒ पाश॒ इत्या॑ह वरुणपा॒शमे॒वाभि ति॑ष्ठति ब॒र्॒.हिर॒भि जु॑हो॒त्याहु॑तीनां॒ प्रति॑ष्ठित्या॒ अथो॑ अग्नि॒वत्ये॒व जु॑हो॒त्यप॑ बर्.हिषः प्रया॒जान्- [  ] \newline

\textbf{Pada Paata} \newline

अपेति॑ । ह॒न्ति॒ । पुरु॑षःपुरुष॒ इति॒ पुरु॑षः - पु॒रु॒षः॒ । नि॒धन॒मिति॑ नि - धन᳚म् । उपेति॑ । ए॒ति॒ । पुरु॑षःपुरुष॒ इति॒ पुरु॑षः - पु॒रु॒षः॒ । हि । र॒क्ष॒स्वी । रक्ष॑साम् । अप॑हत्या॒ इत्यप॑ - ह॒त्यै॒ । उ॒रुम् । हि । राजा᳚ । वरु॑णः । च॒कार॑ । इति॑ । आ॒ह॒ । प्रति॑ष्ठित्या॒ इति॒ प्रति॑-स्थि॒त्यै॒ । श॒तम् । ते॒ । रा॒ज॒न्न् । भि॒षजः॑ । स॒हस्र᳚म् । इति॑ । आ॒ह॒ । भे॒ष॒जम् । ए॒व । अ॒स्मै॒ । क॒रो॒ति॒ । अ॒भिष्ठि॑त॒ इत्य॒भि - स्थि॒तः॒ । वरु॑णस्य । पाशः॑ । इति॑ । आ॒ह॒ । व॒रु॒ण॒पा॒शमिति॑ वरुण - पा॒शम् । ए॒व । अ॒भीति॑ । ति॒ष्ठ॒ति॒ । ब॒र॒.हिः । अ॒भीति॑ । जु॒हो॒ति॒ । आहु॑तीना॒मित्या - हु॒ती॒ना॒म् । प्रति॑ष्ठित्या॒ इति॒ प्रति॑ - स्थि॒त्यै॒ । अथो॒ इति॑ । अ॒ग्नि॒वतीत्य॑ग्नि - वति॑ । ए॒व । जु॒हो॒ति॒ । अप॑बर्.हिष॒ इत्यप॑ - ब॒र॒.हि॒षः॒ । प्र॒या॒जानिति॑ प्र - या॒जान् ।  \newline




\markright{ TS 6.6.3.3  \hfill https://www.vedavms.in \hfill}

\section{ TS 6.6.3.3 }

\textbf{TS 6.6.3.3 } \newline
\textbf{Samhita Paata} \newline

य॑जति प्र॒जा वै ब॒र्॒.हिः प्र॒जा ए॒व व॑रुणपा॒शान्-मु॑ञ्च॒त्याज्य॑भागौ यजति य॒ज्ञ्स्यै॒व चक्षु॑षी॒ नान्तरे॑ति॒ वरु॑णं ॅयजति वरुणपा॒शादे॒वैनं॑ मुञ्चत्य॒ग्नीवरु॑णौ यजति सा॒क्षादे॒वैनं॑ ॅवरुणपा॒शान् मु॑ञ्च॒त्य-प॑बर्.हिषावनूय॒जौ य॑जति प्र॒जा वै ब॒र्॒.हिः प्र॒जा ए॒व व॑रुणपा॒शान्-मु॑ञ्चति च॒तुरः॑ प्रया॒जान्. य॑जति॒ द्वाव॑नूया॒जौ षट्थ् संप॑द्यन्ते॒ षड्वा ऋ॒तव॑- [  ] \newline

\textbf{Pada Paata} \newline

य॒ज॒ति॒ । प्र॒जा इति॑ प्र - जाः । वै । ब॒र्॒.हिः । प्र॒जा इति॑ प्र-जाः । ए॒व । व॒रु॒ण॒पा॒शादिति॑ वरुण - पा॒शात् । मु॒ञ्च॒ति॒ । आज्य॑भागा॒वित्याज्य॑ - भा॒गौ॒ । य॒ज॒ति॒ । य॒ज्ञ्स्य॑ । ए॒व । चक्षु॑षी॒ इति॑ । न । अ॒न्तः । ए॒ति॒ । वरु॑णम् । य॒ज॒ति॒ । व॒रु॒ण॒पा॒शादिति॑ वरुण - पा॒शात् । ए॒व । ए॒न॒म् । मु॒ञ्च॒ति॒ । अ॒ग्नीवरु॑णा॒वित्य॒ग्नी - वरु॑णौ । य॒ज॒ति॒ । सा॒क्षादिति॑ स - अ॒क्षात् । ए॒व । ए॒न॒म् । व॒रु॒ण॒पा॒शादिति॑ वरुण - पा॒शात् । मु॒ञ्च॒ति॒ । अप॑बर्.हिषा॒वित्यप॑ - ब॒र॒.हि॒षौ॒ । अ॒नू॒या॒जावित्य॑नु - या॒जौ । य॒ज॒ति॒ । प्र॒जा इति॑ प्र - जाः । वै । ब॒र॒.हिः । प्र॒जा इति॑ प्र - जाः । ए॒व । व॒रु॒ण॒पा॒शादिति॑ वरुण - पा॒शात् । मु॒ञ्च॒ति॒ । च॒तुरः॑ । प्र॒या॒जानिति॑ प्र - या॒जान् । य॒ज॒ति॒ । द्वौ । अ॒नू॒या॒जावित्य॑नु - या॒जौ । षट् । समिति॑ । प॒द्य॒न्ते॒ । षट् । वै । ऋ॒तवः॑ ।  \newline




\markright{ TS 6.6.3.4  \hfill https://www.vedavms.in \hfill}

\section{ TS 6.6.3.4 }

\textbf{TS 6.6.3.4 } \newline
\textbf{Samhita Paata} \newline

ऋ॒तुष्वे॒व प्रति॑ तिष्ठ॒-त्यव॑भृथ-निचङ्कु॒णेत्या॑ह यथोदि॒तमे॒व वरु॑ण॒मव॑ यजते समु॒द्रे ते॒ हृद॑य-म॒फ्स्व॑न्तरित्या॑ह समु॒द्रे ह्य॑न्तर्वरु॑णः॒ सं त्वा॑ विश॒-न्त्वोष॑धी-रु॒ताऽऽ*प॒ इत्या॑हा॒द्भि-रे॒वैन॒मोष॑धीभिः स॒म्यञ्चं॑ दधाति॒ देवी॑राप ए॒ष वो॒ गर्भ॒ इत्या॑ह यथाय॒जुरे॒वैतत् प॒शवो॒ वै- [  ] \newline

\textbf{Pada Paata} \newline

ऋ॒तुषु॑ । ए॒व । प्रतीति॑ । ति॒ष्ठ॒ति॒ । अव॑भृ॒थेत्यव॑ - भृ॒थ॒ । नि॒च॒ङ्कु॒णेति॑ नि - च॒ङ्कु॒ण॒ । इति॑ । आ॒ह॒ । य॒थो॒दि॒तमिति॑ यथा - उ॒दि॒तम् । ए॒व । वरु॑णम् । अवेति॑ । य॒ज॒ते॒ । स॒मु॒द्रे । ते॒ । हृद॑यम् । अ॒फ्स्वित्य॑प्-सु । अ॒न्तः । इति॑ । आ॒ह॒ । स॒मु॒द्रे । हि । अ॒न्तः । वरु॑णः । समिति॑ । त्वा॒ । वि॒श॒न्तु॒ । ओष॑धीः । उ॒त । आपः॑ । इति॑ । आ॒ह॒ । अ॒द्भिरित्य॑त्-भिः । ए॒व । ए॒न॒म् । ओष॑धीभि॒रित्योष॑धि - भिः॒ । स॒म्यञ्च᳚म् । द॒धा॒ति॒ । देवीः᳚ । आ॒पः॒ । ए॒षः । वः॒ । गर्भः॑ । इति॑ । आ॒ह॒ । य॒था॒य॒जुरिति॑ यथा - य॒जुः । ए॒व । ए॒तत् । प॒शवः॑ । वै ।  \newline




\markright{ TS 6.6.3.5  \hfill https://www.vedavms.in \hfill}

\section{ TS 6.6.3.5 }

\textbf{TS 6.6.3.5 } \newline
\textbf{Samhita Paata} \newline

सोमो॒ यद्-भि॑न्दू॒नां भ॒क्षये᳚त् पशु॒मान्थ्-स्या॒द्-वरु॑ण॒-स्त्वे॑नं गृह्णीया॒द्यन्न भ॒क्षये॑दप॒शुः स्या॒न्नैनं॒ ॅवरु॑णो गृह्णीया-दुप॒स्पृश्य॑मे॒व प॑शु॒मान् भ॑वति॒ नैनं॒ ॅवरु॑णो गृह्णाति॒ प्रति॑युतो॒ वरु॑णस्य॒ पाश॒ इत्या॑ह वरुणपा॒शादे॒व निर्मु॑च्य॒ते ऽप्र॑तीक्ष॒मा य॑न्ति॒ वरु॑णस्या॒न्तर्.हि॑त्या॒ एधो᳚ऽस्येधिषी॒मही-त्या॑ह स॒मिधै॒वाग्निं न॑म॒स्यन्त॑ ( ) उ॒पाय॑न्ति॒ तेजो॑ऽसि॒ तेजो॒ मयि॑ धे॒हीत्या॑ह॒ तेज॑ ए॒वाऽऽत्मन् ध॑त्ते ॥ \newline

\textbf{Pada Paata} \newline

सोमः॑ । यत् । भि॒न्दू॒नाम् । भ॒क्षये᳚त् । प॒शु॒मानिति॑ पशु - मान् । स्या॒त् । वरु॑णः । तु । ए॒न॒म् । गृ॒ह्णी॒या॒त् । यत् । न । भ॒क्षये᳚त् । अ॒प॒शुः । स्या॒त् । न । ए॒न॒म् । वरु॑णः । गृ॒ह्णी॒या॒त् । उ॒प॒स्पृश्य॒मित्यु॑प - स्पृश्य᳚म् । ए॒व । प॒शु॒मानिति॑ पशु - मान् । भ॒व॒ति॒ । न । ए॒न॒म् । वरु॑णः । गृ॒ह्णा॒ति॒ । प्रति॑युत॒ इति॒ प्रति॑ - यु॒तः॒ । वरु॑णस्य । पाशः॑ । इति॑ । आ॒ह॒ । व॒रु॒ण॒पा॒शादिति॑ वरुण - पा॒शात् । ए॒व । निरिति॑ । मु॒च्य॒ते॒ । अप्र॑तीक्ष॒मित्यप्र॑ति - ई॒क्ष॒म् । एति॑ । य॒न्ति॒ । वरु॑णस्य । अ॒न्तर्.हि॑त्या॒ इत्य॒न्तः - हि॒त्यै॒ । एधः॑ । अ॒सि॒ । ए॒धि॒षी॒महि॑ । इति॑ । आ॒ह॒ । स॒मिधेति॑ सम् - इधा᳚ । ए॒व । अ॒ग्निम् । न॒म॒स्यन्तः॑ ( ) । उ॒पाय॒न्तीत्यु॑प - आय॑न्ति । तेजः॑ । अ॒सि॒ । तेजः॑ । मयि॑ । धे॒हि॒ । इति॑ । आ॒ह॒ । तेजः॑ । ए॒व । आ॒त्मन्न् । ध॒त्ते॒ ॥  \newline




\markright{ TS 6.6.4.1  \hfill https://www.vedavms.in \hfill}

\section{ TS 6.6.4.1 }

\textbf{TS 6.6.4.1 } \newline
\textbf{Samhita Paata} \newline

स्फ्येन॒ वेदि॒मुद्ध॑न्ति रथा॒क्षेण॒ वि मि॑मीते॒ यूपं॑ मिनोति त्रि॒वृत॑मे॒व वज्रꣳ॑ स॒भृंत्य॒ भ्रातृ॑व्याय॒ प्र ह॑रति॒ स्तृत्यै॒ यद॑न्तर्वे॒दि मि॑नु॒याद्-दे॑वलो॒कम॒भि ज॑ये॒द्-यद्-ब॑हिर्वे॒दि म॑नुष्य लो॒कं ॅवे᳚द्य॒न्तस्य॑ स॒न्धौ मि॑नोत्यु॒भयो᳚-र्लो॒कयो॑-र॒भिजि॑त्या॒ उप॑रसंमितां मिनुयात् पितृलो॒कका॑मस्य रश॒नस॑मिंतां मनुष्यलो॒कका॑मस्य च॒षाल॑-संमितामिन्द्रि॒य का॑मस्य॒ सर्वा᳚न्थ् स॒मान् प्र॑ति॒ष्ठाका॑मस्य॒ ये त्रयो॑ मद्ध्य॒मास्तान्थ् स॒मान् प॒शुका॑मस्यै॒तान्. वा- [  ] \newline

\textbf{Pada Paata} \newline

स्फ्येन॑ । वेदि᳚म् । उदिति॑ । ह॒न्ति॒ । र॒था॒क्षेणेति॑ रथ-अ॒क्षेण॑ । वीति॑ । मि॒मी॒ते॒ । यूप᳚म् । मि॒नो॒ति॒ । त्रि॒वृत॒मिति॑ त्रि - वृत᳚म् । ए॒व । वज्र᳚म् । स॒भृंत्येति॑ सं - भृत्य॑ । भ्रातृ॑व्याय । प्रेति॑ । ह॒र॒ति॒ । स्तृत्यै᳚ । यत् । अ॒न्त॒र्वे॒दीत्य॑न्तः - वे॒दि । मि॒नु॒यात् । दे॒व॒लो॒कमिति॑ देव - लो॒कम् । अ॒भीति॑ । ज॒ये॒त् । यत् । ब॒हि॒र्वे॒दीति॑ बहिः - वे॒दि । म॒नु॒ष्य॒लो॒कमिति॑ मनुष्य - लो॒कम् । वे॒द्य॒न्तस्येति॑ वेदि - अ॒न्तस्य॑ । स॒न्धाविति॑ सं - धौ । मि॒नो॒ति॒ । उ॒भयोः᳚ । लो॒कयोः᳚ । अ॒भिजि॑त्या॒ इत्य॒भि - जि॒त्यै॒ । उप॑रसम्मिता॒मित्युप॑र - स॒मिं॒ता॒म् । मि॒नु॒या॒त् । पि॒तृ॒लो॒कका॑म॒स्येति॑ पितृलो॒क - का॒म॒स्य॒ । र॒श॒नस॑म्मिता॒मिति॑ रश॒न - स॒मिं॒ता॒म् । म॒नु॒ष्य॒लो॒कका॑म॒स्येति॑ मनुष्यलो॒क - का॒म॒स्य॒ । च॒षाल॑संमिता॒मिति॑ च॒षाल॑ - स॒मिं॒ता॒म् । इ॒न्द्रि॒यका॑म॒स्येती᳚न्द्रि॒य - का॒म॒स्य॒ । सर्वान्॑ । स॒मान् । प्र॒ति॒ष्ठाका॑म॒स्येति॑ प्रति॒ष्ठा - का॒म॒स्य॒ । ये । त्रयः॑ । म॒द्ध्य॒माः । तान् । स॒मान् । प॒शुका॑म॒स्येति॑ प॒शु - का॒म॒स्य॒ । ए॒तान् । वै ।  \newline




\markright{ TS 6.6.4.2  \hfill https://www.vedavms.in \hfill}

\section{ TS 6.6.4.2 }

\textbf{TS 6.6.4.2 } \newline
\textbf{Samhita Paata} \newline

अनु॑ प॒शव॒ उप॑ तिष्ठन्ते पशु॒माने॒व भ॑वति॒ व्यति॑षजे॒दित॑रान् प्र॒जयै॒वैनं॑ प॒शुभि॒र्व्यति॑षजति॒ यं का॒मये॑त प्र॒मायु॑कः स्या॒दिति॑ गर्त॒मितं॒ तस्य॑ मिनुयादुत्तरा॒र्द्ध्यं॑ ॅवर्.षि॑ष्ठ॒मथ॒ ह्रसी॑याꣳसमे॒षा वै ग॑र्त॒मिद्यस्यै॒वं मि॒नोति॑ ता॒जक् प्र मी॑यते दक्षिणा॒र्द्ध्यं॑ ॅवर्.षि॑ष्ठं मिनुयाथ् सुव॒र्गका॑म॒स्याथ॒ ह्रसी॑याꣳस-मा॒क्रम॑णमे॒व तथ् सेतुं॒ ॅयज॑मानः कुरुते सुव॒र्गस्य॑ लो॒कस्य॒ सम॑ष्ट्यै॒- [  ] \newline

\textbf{Pada Paata} \newline

अन्विति॑ । प॒शवः॑ । उपेति॑ । ति॒ष्ठ॒न्ते॒ । प॒शु॒मानिति॑ पशु - मान् । ए॒व । भ॒व॒ति॒ । व्यति॑षजे॒दिति॑ वि - अति॑षजेत् । इत॑रान् । प्र॒जयेति॑ प्र - जया᳚ । ए॒व । ए॒न॒म् । प॒शुभि॒रिति॑ प॒शु - भिः॒ । व्यति॑षज॒तीति॑ वि - अति॑षजति । यम् । का॒मये॑त । प्र॒मायु॑क॒ इति॑ प्र - मायु॑कः । स्या॒त् । इति॑ । ग॒र्त॒मित॒मिति॑ गर्त - मित᳚म् । तस्य॑ । मि॒नु॒या॒त् । उ॒त्त॒रा॒द्‌र्ध्य॑मित्यु॑त्तर - अ॒द्‌र्ध्य᳚म् । वर्.षि॑ष्ठम् । अथ॑ । ह्रसी॑याꣳसम् । ए॒षा । वै । ग॒र्त॒मिदिति॑ गर्त - मित् । यस्य॑ । ए॒वम् । मि॒नोति॑ । ता॒जक् । प्रेति॑ । मी॒य॒ते॒ । द॒क्षि॒णा॒द्‌र्ध्य॑मिति॑ दक्षिण - अ॒द्‌र्ध्य᳚म् । वर्.षि॑ष्ठम् । मि॒नु॒या॒त् । सु॒व॒र्गका॑म॒स्येति॑ सुव॒र्ग - का॒म॒स्य॒ । अथ॑ । ह्रसी॑याꣳसम् । आ॒क्रम॑ण॒मित्या᳚ - क्रम॑णम् । ए॒व । तत् । सेतु᳚म् । यज॑मानः । कु॒रु॒ते॒ । सु॒व॒र्गस्येति॑ सुवः - गस्य॑ । लो॒कस्य॑ । सम॑ष्ट्य॒ इति॒ सं - अ॒ष्ट्यै॒ ।  \newline




\markright{ TS 6.6.4.3  \hfill https://www.vedavms.in \hfill}

\section{ TS 6.6.4.3 }

\textbf{TS 6.6.4.3 } \newline
\textbf{Samhita Paata} \newline

यदेक॑स्मि॒न॒. यूपे॒ द्वे र॑श॒ने प॑रि॒व्यय॑ति॒ तस्मा॒देको॒ द्वे जा॒ये वि॑न्दते॒ यन्नैकाꣳ॑ रश॒नां द्वयो॒र्यूप॑योः परि॒व्यय॑ति॒ तस्मा॒न्नैका॒ द्वौ पती॑ विन्दते॒ यं का॒मये॑त॒ स्त्र्य॑स्य जाये॒तेत्यु॑पा॒न्ते तस्य॒ व्यति॑षजे॒थ् स्त्र्ये॑वास्य॑ जायते॒ यं का॒मये॑त॒ पुमा॑नस्य जाये॒तेत्या॒न्तं तस्य॒ प्र वे᳚ष्टये॒त् पुमा॑ने॒वास्य॑- [  ] \newline

\textbf{Pada Paata} \newline

यत् । एक॑स्मिन्न् । यूपे᳚ । द्वे इति॑ । र॒श॒ने इति॑ । प॒रि॒व्यय॒तीति॑ परि - व्यय॑ति । तस्मा᳚त् । एकः॑ । द्वे इति॑ । जा॒ये इति॑ । वि॒न्द॒ते॒ । यत् । न । एका᳚म् । र॒श॒नाम् । द्वयोः᳚ । यूप॑योः । प॒रि॒व्यय॒तीति॑ परि - व्यय॑ति । तस्मा᳚त् । न । एका᳚ । द्वौ । पती॒ इति॑ । वि॒न्द॒ते॒ । यम् । का॒मये॑त । स्त्री । अ॒स्य॒ । जा॒ये॒त॒ । इति॑ । उ॒पा॒न्त इत्यु॑प - अ॒न्ते । तस्य॑ । व्यति॑षजे॒दिति॑ वि - अति॑षजेत् । स्त्री । ए॒व । अ॒स्य॒ । जा॒य॒ते॒ । यम् । का॒मये॑त । पुमान्॑ । अ॒स्य॒ । जा॒ये॒त॒ । इति॑ । आ॒न्तमित्या᳚ - अ॒न्तम् । तस्य॑ । प्रेति॑ । वे॒ष्ट॒ये॒त् । पुमान्॑ । ए॒व । अ॒स्य॒ ।  \newline




\markright{ TS 6.6.4.4  \hfill https://www.vedavms.in \hfill}

\section{ TS 6.6.4.4 }

\textbf{TS 6.6.4.4 } \newline
\textbf{Samhita Paata} \newline

जाय॒ते ऽसु॑रा॒ वै दे॒वान् द॑क्षिण॒त उपा॑नय॒न् तान् दे॒वा उ॑पश॒येनै॒वापा॑-नुदन्त॒ त-दु॑पश॒यस्यो॑-पशय॒त्वं ॅयद्-द॑क्षिण॒त उ॑पश॒य उ॑प॒शये॒ भ्रातृ॑व्यापनुत्त्यै॒ सर्वे॒ वा अ॒न्ये यूपाः᳚ पशु॒मन्तोऽथो॑पश॒य ए॒वाप॒शुस्तस्य॒ यज॑मानः प॒शुर्यन्न नि॑र्दि॒शेदार्ति॒-मार्च्छे॒द्-यज॑मानो॒ऽसौ ते॑ प॒शुरिति॒ निर्दि॑शे॒द्यं द्वि॒ष्याद्-यमे॒व- [  ] \newline

\textbf{Pada Paata} \newline

जा॒य॒ते॒ । असु॑राः । वै । दे॒वान् । द॒क्षि॒ण॒तः । उपेति॑ । अ॒न॒य॒न्न् । तान् । दे॒वाः । उ॒प॒श॒येनेत्यु॑प - श॒येन॑ । ए॒व । अपेति॑ । अ॒नु॒द॒न्त॒ । तत् । उ॒प॒श॒यस्येत्यु॑प - श॒यस्य॑ । उ॒प॒श॒य॒त्वमित्यु॑पशय - त्वम् । यत् । द॒क्षि॒ण॒तः । उ॒प॒श॒य इत्यु॑प - श॒यः । उ॒प॒शय॒ इत्यु॑प - शये᳚ । भ्रातृ॑व्यापनुत्त्या॒ इति॒ भ्रातृ॑व्य-अ॒प॒नु॒त्त्यै॒ । सर्वे᳚ । वै । अ॒न्ये । यूपाः᳚ । प॒शु॒मन्त॒ इति॑ पशु - मन्तः॑ । अथ॑ । उ॒प॒श॒य इत्यु॑प - श॒यः । ए॒व । अ॒प॒शुः । तस्य॑ । यज॑मानः । प॒शुः । यत् । न । नि॒र्दि॒शेदिति॑ निः - दि॒शेत् । आर्ति᳚म् । एति॑ । ऋ॒च्छे॒त् । यज॑मानः । अ॒सौ । ते॒ । प॒शुः । इति॑ । निरिति॑ । दि॒शे॒त् । यम् । द्वि॒ष्यात् । यम् । ए॒व ।  \newline




\markright{ TS 6.6.4.5  \hfill https://www.vedavms.in \hfill}

\section{ TS 6.6.4.5 }

\textbf{TS 6.6.4.5 } \newline
\textbf{Samhita Paata} \newline

द्वेष्टि॒ तम॑स्मै प॒शुं निर्दि॑शति॒ यदि॒ न द्वि॒ष्यादा॒खुस्ते॑ प॒शुरिति॑ ब्रूया॒न्न ग्रा॒म्यान् प॒शून्. हि॒नस्ति॒ नाऽऽ*र॒ण्यान् प्र॒जाप॑तिः प्र॒जा अ॑सृजत॒ सो᳚ऽन्नाद्ये॑न॒ व्या᳚र्द्ध्यत॒ स ए॒तामे॑काद॒शिनी॑-मपश्य॒त् तया॒ वै सो᳚ऽन्नाद्य॒मवा॑रुन्ध॒ यद्दश॒ यूपा॒ भव॑न्ति॒ दशा᳚क्षरा वि॒राडन्नं॑ ॅवि॒राड् वि॒राजै॒वा-न्नाद्य॒मव॑ रुन्धे॒- [  ] \newline

\textbf{Pada Paata} \newline

द्वेष्टि॑ । तम् । अ॒स्मै॒ । प॒शुम् । निरिति॑ । दि॒श॒ति॒ । यदि॑ । न । द्वि॒ष्यात् । आ॒खुः । ते॒ । प॒शुः । इति॑ । ब्रू॒या॒त् । न । ग्रा॒म्यान् । प॒शून् । हि॒नस्ति॑ । न । आ॒र॒ण्यान् । प्र॒जाप॑ति॒रिति॑ प्र॒जा - प॒तिः॒ । प्र॒जा इति॑ प्र - जाः । अ॒सृ॒ज॒त॒ । सः । अ॒न्नाद्ये॒नेत्य॑न्न - अद्ये॑न । वीति॑ । अ॒द्‌र्ध्य॒त॒ । सः । ए॒ताम् । ए॒का॒द॒शिनी᳚म् । अ॒प॒श्य॒त् । तया᳚ । वै । सः । अ॒न्नाद्य॒मित्य॑न्न - अद्य᳚म् । अवेति॑ । अ॒रु॒न्ध॒ । यत् । दश॑ । यूपाः᳚ । भव॑न्ति । दशा᳚क्ष॒रेति॒ दश॑ - अ॒क्ष॒रा॒ । वि॒राडिति॑ वि - राट् । अन्न᳚म् । वि॒राडिति॑ वि - राट् । वि॒राजेति॑ वि - राजा᳚ । ए॒व । अ॒न्नाद्य॒मित्य॑न्न - अद्य᳚म् । अवेति॑ । रु॒न्धे॒ ।  \newline




\markright{ TS 6.6.4.6  \hfill https://www.vedavms.in \hfill}

\section{ TS 6.6.4.6 }

\textbf{TS 6.6.4.6 } \newline
\textbf{Samhita Paata} \newline

य ए॑काद॒शः स्तन॑ ए॒वास्यै॒ स दु॒ह ए॒वैनां॒ तेन॒ वज्रो॒ वा ए॒षा सं मी॑यते॒ यदे॑काद॒शिनी॒ सेश्व॒रा पु॒रस्ता᳚त् प्र॒त्यञ्चं॑ ॅय॒ज्ञ्ꣳ संम॑र्दितो॒र्यत् पा᳚त्नीव॒तं मि॒नोति॑ य॒ज्ञ्स्य॒ प्रत्युत्त॑ब्ध्यै सय॒त्वाय॑ ॥ \newline

\textbf{Pada Paata} \newline

यः । ए॒का॒द॒शः । स्तनः॑ । ए॒व । अ॒स्यै॒ । सः । दु॒हे । ए॒व । ए॒ना॒म् । तेन॑ । वज्रः॑ । वै । ए॒षा । समिति॑ । मी॒य॒ते॒ । यत् । ए॒का॒द॒शिनी᳚ । सा । ई॒श्व॒रा । पु॒रस्ता᳚त् । प्र॒त्यञ्च᳚म् । य॒ज्ञ्म् । संम॑र्दितो॒रिति॒ सं - म॒र्दि॒तोः॒ । यत् । पा॒त्नी॒व॒तमिति॑ पात्नी - व॒तम् । मि॒नोति॑ । य॒ज्ञ्स्य॑ । प्रतीति॑ । उत्त॑ब्ध्या॒ इत्युत् - स्त॒ब्ध्यै॒ । स॒य॒त्वायेति॑ सय - त्वाय॑ ॥  \newline




\markright{ TS 6.6.5.1  \hfill https://www.vedavms.in \hfill}

\section{ TS 6.6.5.1 }

\textbf{TS 6.6.5.1 } \newline
\textbf{Samhita Paata} \newline

प्र॒जापतिः॑ प्र॒जा अ॑सृजत॒ स रि॑रिचा॒नो॑ऽमन्यत॒ स ए॒तामे॑काद॒शिनी॑-मपश्य॒त् तया॒ वै स आयु॑रिन्द्रि॒यं ॅवी॒र्य॑मा॒त्मन्न॑धत्त प्र॒जा इ॑व॒ खलु॒ वा ए॒ष सृ॑जते॒ यो यज॑ते॒ स ए॒तर्.हि॑ रिरिचा॒न इ॑व॒ यदे॒षैका॑द॒शिनी॒ भव॒त्यायु॑रे॒व तये᳚न्द्रि॒यं ॅवी॒र्यं॑ ॅयज॑मान आ॒त्मन् ध॑त्ते॒ प्रैवाऽऽ*ग्ने॒येन॑ वापयति मिथु॒नꣳ सा॑रस्व॒त्या क॑रोति॒ रेतः॑- [  ] \newline

\textbf{Pada Paata} \newline

प्र॒जाप॑ति॒रिति॑ प्र॒जा - प॒तिः॒ । प्र॒जा इति॑ प्र-जाः । अ॒सृ॒ज॒त॒ । सः । रि॒रि॒चा॒नः । अ॒म॒न्य॒त॒ । सः । ए॒ताम् । ए॒का॒द॒शिनी᳚म् । अ॒प॒श्य॒त् । तया᳚ । वै । सः । आयुः॑ । इ॒न्द्रि॒यम् । वी॒र्य᳚म् । आ॒त्मन्न् । अ॒ध॒त्त॒ । प्र॒जा इति॑ प्र - जाः । इ॒व॒ । खलु॑ । वै । ए॒षः । सृ॒ज॒ते॒ । यः । यज॑ते । सः । ए॒तर्.हि॑ । रि॒रि॒चा॒नः । इ॒व॒ । यत् । ए॒षा । ए॒का॒द॒शिनी᳚ । भव॑ति । आयुः॑ । ए॒व । तया᳚ । इ॒न्द्रि॒यम् । वी॒र्य᳚म् । यज॑मानः । आ॒त्मन्न् । ध॒त्ते॒ । प्रेति॑ । ए॒व । आ॒ग्ने॒येन॑ । वा॒प॒य॒ति॒ । मि॒थु॒नम् । सा॒र॒स्व॒त्या । क॒रो॒ति॒ । रेतः॑ ।  \newline




\markright{ TS 6.6.5.2  \hfill https://www.vedavms.in \hfill}

\section{ TS 6.6.5.2 }

\textbf{TS 6.6.5.2 } \newline
\textbf{Samhita Paata} \newline

सौ॒म्येन॑ दधाति॒ प्र ज॑नयति पौ॒ष्णेन॑ बार्.हस्प॒त्यो भ॑वति॒ ब्रह्म॒ वै दे॒वानां॒ बृह॒स्पति॒र्ब्रह्म॑णै॒वास्मै᳚ प्र॒जाः प्रज॑नयति वैश्वदे॒वो भ॑वति वैश्वदे॒व्यो॑ वै प्र॒जाः प्र॒जा ए॒वास्मै॒ प्रज॑नयती-न्द्रि॒यमे॒वैन्द्रेणाव॑ रुन्धे॒ विशं॑ मारु॒तेनौजो॒ बल॑मैन्द्रा॒ग्नेन॑ प्रस॒वाय॑ सावि॒त्रो नि॑र्वरुण॒त्वाय॑ वारु॒णो म॑द्ध्य॒त ऐ॒न्द्रमा ल॑भते मद्ध्य॒त ए॒वेन्द्रि॒यं ॅयज॑माने दधाति- [  ] \newline

\textbf{Pada Paata} \newline

सौ॒म्येन॑ । द॒धा॒ति॒ । प्रेति॑ । ज॒न॒य॒ति॒ । पौ॒ष्णेन॑ । बा॒र्॒.ह॒स्प॒त्यः । भ॒व॒ति॒ । ब्रह्म॑ । वै । दे॒वाना᳚म् । बृह॒स्पतिः॑ । ब्रह्म॑णा । ए॒व । अ॒स्मै॒ । प्र॒जा इति॑ प्र-जाः । प्रेति॑ । ज॒न॒य॒ति॒ । वै॒श्व॒दे॒व इति॑ वैश्व - दे॒वः । भ॒व॒ति॒ । वै॒श्व॒दे॒व्य॑ इति॑ वैश्व - दे॒व्यः॑ । वै । प्र॒जा इति॑ प्र - जाः । प्र॒जा इति॑ प्र - जाः । ए॒व । अ॒स्मै॒ । प्रेति॑ । ज॒न॒य॒ति॒ । इ॒न्द्रि॒यम् । ए॒व । ऐ॒न्द्रेण॑ । अवेति॑ । रु॒न्धे॒ । विश᳚म् । मा॒रु॒तेन॑ । ओजः॑ । बल᳚म् । ऐ॒न्द्रा॒ग्नेनेत्यै᳚न्द्र - अ॒ग्नेन॑ । प्र॒स॒वायेति॑ प्र - स॒वाय॑ । सा॒वि॒त्रः । नि॒र्व॒रु॒ण॒त्वायेति॑ निर्वरुण - त्वाय॑ । वा॒रु॒णः । म॒द्ध्य॒तः । ऐ॒न्द्रम् । एति॑ । ल॒भ॒ते॒ । म॒द्ध्य॒तः । ए॒व । इ॒न्द्रि॒यम् । यज॑माने । द॒धा॒ति॒ ।  \newline




\markright{ TS 6.6.5.3  \hfill https://www.vedavms.in \hfill}

\section{ TS 6.6.5.3 }

\textbf{TS 6.6.5.3 } \newline
\textbf{Samhita Paata} \newline

पु॒रस्ता॑दै॒न्द्रस्य॑ वैश्वदे॒वमा ल॑भते वैश्वदे॒वं ॅवा अन्न॒मन्न॑मे॒व पु॒रस्ता᳚द्धत्ते॒ तस्मा᳚त् पु॒रस्ता॒दन्न॑मद्यत ऐ॒न्द्रमा॒लभ्य॑ मारु॒तमा ल॑भते॒ विड् वै म॒रुतो॒ विश॑मे॒वास्मा॒ अनु॑ बद्ध्नाति॒ यदि॑ का॒मये॑त॒ योऽव॑गतः॒ सोऽप॑ रुद्ध्यतां॒ ॅयोऽप॑रुद्धः॒ सोऽव॑ गच्छ॒त्वित्यै॒न्द्रस्य॑ लो॒के वा॑रु॒णमा ल॑भेत वारु॒णस्य॑ लो॒क ऐ॒न्द्रं- [  ] \newline

\textbf{Pada Paata} \newline

पु॒रस्ता᳚त् । ऐ॒न्द्रस्य॑ । वै॒श्व॒दे॒वमिति॑ वैश्व - दे॒वम् । एति॑ । ल॒भ॒ते॒ । वै॒श्व॒दे॒वमिति॑ वैश्व - दे॒वम् । वै । अन्न᳚म् । अन्न᳚म् । ए॒व । पु॒रस्ता᳚त् । ध॒त्ते॒ । तस्मा᳚त् । पु॒रस्ता᳚त् । अन्न᳚म् । अ॒द्य॒ते॒ । ऐ॒न्द्रम् । आ॒लभ्येत्या᳚-लभ्य॑ । मा॒रु॒तम् । एति॑ । ल॒भ॒ते॒ । विट् । वै । म॒रुतः॑ । विश᳚म् । ए॒व । अ॒स्मै॒ । अन्विति॑ । ब॒द्ध्ना॒ति॒ । यदि॑ । का॒मये॑त । यः । अव॑गत॒ इत्यव॑ - ग॒तः॒ । सः । अपेति॑ । रु॒द्ध्य॒ता॒म् । यः । अप॑रुद्ध॒ उत्यप॑- रु॒द्धः॒ । सः । अवेति॑ । ग॒च्छ॒तु॒ । इति॑ । ऐ॒न्द्रस्य॑ । लो॒के । वा॒रु॒णम् । एति॑ । ल॒भे॒त॒ । वा॒रु॒णस्य॑ । लो॒के । ऐ॒न्द्रम् ।  \newline




\markright{ TS 6.6.5.4  \hfill https://www.vedavms.in \hfill}

\section{ TS 6.6.5.4 }

\textbf{TS 6.6.5.4 } \newline
\textbf{Samhita Paata} \newline

ॅय ए॒वाव॑गतः॒ सोऽप॑ रुद्ध्यते॒ योऽप॑रुद्धः॒ सोऽव॑ गच्छति॒ यदि॑ का॒मये॑त प्र॒जा मु॑ह्येयु॒रिति॑ प॒शून् व्यति॑षजेत् प्र॒जा ए॒व मो॑हयति॒ यद॑भिवाह॒तो॑ऽपां ॅवा॑रु॒णमा॒लभे॑त प्र॒जा वरु॑णो गृह्णीयाद्-दक्षिण॒त उद॑ञ्च॒मा ल॑भतेऽपवाह॒तो॑ऽपां प्र॒जाना॒-मव॑रुण ग्राहाय ॥ \newline

\textbf{Pada Paata} \newline

यः । ए॒व । अव॑गत॒ इत्यव॑ - ग॒तः॒ । सः । अपेति॑ । रु॒द्ध्य॒ते॒ । यः । अप॑रुद्ध॒ उत्यप॑-रु॒द्धः॒ । सः । अवेति॑ । ग॒च्छ॒ति॒ । यदि॑ । का॒मये॑त । प्र॒जा इति॑ प्र - जाः । मु॒ह्ये॒युः॒ । इति॑ । प॒शून् । व्यति॑षजे॒दिति॑ वि - अति॑षजेत् । प्र॒जा इति॑ प्र - जाः । ए॒व । मो॒ह॒य॒ति॒ । यत् । अ॒भि॒वा॒ह॒त इत्य॑भि - वा॒ह॒तः । अ॒पाम् । वा॒रु॒णम् । आ॒लभे॒तेत्या᳚ - लभे॑त । प्र॒जा इति॑ प्र - जाः । वरु॑णः । गृ॒ह्णी॒या॒त् । द॒क्षि॒ण॒तः । उद॑ञ्चम् । एति॑ । ल॒भ॒ते॒ । अ॒प॒वा॒ह॒त इत्य॑प - वा॒ह॒तः । अ॒पाम् । प्र॒जाना॒मिति॑ प्र - जाना᳚म् । अव॑रुणग्राहा॒येत्यव॑रुण - ग्रा॒हा॒य॒ ॥  \newline




\markright{ TS 6.6.6.1  \hfill https://www.vedavms.in \hfill}

\section{ TS 6.6.6.1 }

\textbf{TS 6.6.6.1 } \newline
\textbf{Samhita Paata} \newline

इन्द्रः॒ पत्नि॑या॒ मनु॑मयाजय॒त् तां पर्य॑ग्निकृता॒-मुद॑सृज॒त् तया॒ मनु॑रार्द्ध्नो॒द्यत् पर्य॑ग्निकृतं पात्नीव॒तमु॑थ् सृ॒जति॒ यामे॒व मनु॒र॒. ऋद्धि॒मार्द्ध्नो॒त् तामे॒व यज॑मान ऋध्नोति य॒ज्ञ्स्य॒ वा अप्र॑तिष्ठिताद्-य॒ज्ञ्ः परा॑ भवति य॒ज्ञ्ं प॑रा॒भव॑न्तं॒ ॅयज॑मा॒नोऽनु॒ परा॑ भवति॒ यदाज्ये॑न पात्नीव॒तꣳ सꣳ॑स्था॒पय॑ति य॒ज्ञ्स्य॒ प्रति॑ष्ठित्यै य॒ज्ञ्ं प्र॑ति॒तिष्ठ॑न्तं॒ ॅयज॑मा॒नोऽनु॒ प्रति॑ तिष्ठती॒ष्टं ॅव॒पया॒- [  ] \newline

\textbf{Pada Paata} \newline

इन्द्रः॑ । पत्नि॑या । मनु᳚म् । अ॒या॒ज॒य॒त् । ताम् । पर्य॑ग्निकृता॒मिति॒ पर्य॑ग्नि - कृ॒ता॒म् । उदिति॑ । अ॒सृ॒ज॒त् । तया᳚ । मनुः॑ । आ॒द्‌र्ध्नो॒त् । यत् । पर्य॑ग्निकृत॒मिति॒ पर्य॑ग्नि-कृ॒त॒म् । पा॒त्नी॒व॒तमिति॑ पात्नी-व॒तम् । उ॒थ्सृ॒जतीत्यु॑त् - सृ॒जति॑ । याम् । ए॒व । मनुः॑ । ऋद्धि᳚म् । आद्‌र्ध्नो᳚त् । ताम् । ए॒व । यज॑मानः । ऋ॒द्ध्नो॒ति॒ । य॒ज्ञ्स्य॑ । वै । अप्र॑तिष्ठिता॒दित्यप्र॑ति - स्थि॒ता॒त् । य॒ज्ञ्ः । परेति॑ । भ॒व॒ति॒ । य॒ज्ञ्म् । प॒रा॒भव॑न्त॒मिति॑ परा - भव॑न्तम् । यज॑मानः । अनु॑ । परेति॑ । भ॒व॒ति॒ । यत् । आज्ये॑न । पा॒त्नी॒व॒तमिति॑ पात्नी - व॒तम् । सꣳ॒॒स्था॒पय॒तीति॑ सं - स्था॒पय॑ति । य॒ज्ञ्स्य॑ । प्रति॑ष्ठित्या॒ इति॒ प्रति॑-स्थि॒त्यै॒ । य॒ज्ञ्म् । प्र॒ति॒तिष्ठ॑न्त॒मिति॑ प्रति - तिष्ठ॑न्तम् । यज॑मानः । अनु॑ । प्रतीति॑ । ति॒ष्ठ॒ति॒ । इ॒ष्टम् । व॒पया᳚ ।  \newline




\markright{ TS 6.6.6.2  \hfill https://www.vedavms.in \hfill}

\section{ TS 6.6.6.2 }

\textbf{TS 6.6.6.2 } \newline
\textbf{Samhita Paata} \newline

भव॒त्यनि॑ष्टं ॅव॒शयाऽथ॑ पात्नीव॒तेन॒ प्र च॑रति ती॒र्त्थ ए॒व प्र च॑र॒त्यथो॑ ए॒तर्ह्ये॒वास्य॒ याम॑स्त्वा॒ष्ट्रो भ॑वति॒ त्वष्टा॒ वै रेत॑सः सि॒क्तस्य॑ रू॒पाणि॒ वि क॑रोति॒ तमे॒व वृषा॑णं॒ पत्नी॒ष्वपि॑ सृजति॒ सो᳚ऽस्मै रू॒पाणि॒ वि क॑रोति ॥ \newline

\textbf{Pada Paata} \newline

भव॑ति । अनि॑ष्टम् । व॒शया᳚ । अथ॑ । पा॒त्नी॒व॒तेनेति॑ पात्नी - व॒तेन॑ । प्रेति॑ । च॒र॒ति॒ । ती॒र्त्थ । ए॒व । प्रेति॑ । च॒र॒ति॒ । अथो॒ इति॑ । ए॒तर्.हि॑ । ए॒व । अ॒स्य॒ । यामः॑ । त्वा॒ष्ट्रः । भ॒व॒ति॒ । त्वष्टा᳚ । वै । रेत॑सः । सि॒क्तस्य॑ । रू॒पाणि॑ । वीति॑ । क॒रो॒ति॒ । तम् । ए॒व । वृषा॑णम् । पत्नी॑षु । अपीति॑ । सृ॒ज॒ति॒ । सः । अ॒स्मै॒ । रू॒पाणि॑ । वीति॑ । क॒रो॒ति॒ ॥  \newline




\markright{ TS 6.6.7.1  \hfill https://www.vedavms.in \hfill}

\section{ TS 6.6.7.1 }

\textbf{TS 6.6.7.1 } \newline
\textbf{Samhita Paata} \newline

घ्नन्ति॒ वा ए॒तथ् सोमं॒ ॅयद॑भिषु॒ण्वन्ति॒ यथ् सौ॒म्यो भव॑ति॒ यथा॑ मृ॒ताया॑नु॒स्तर॑णीं॒ घ्नन्ति॑ ता॒दृगे॒व तद् यदु॑त्तरा॒र्द्धे वा॒ मद्ध्ये॑ वा जुहु॒याद्-दे॒वता᳚भ्यः स॒मदं॑ दद्ध्याद्-दक्षिणा॒र्द्धे जु॑होत्ये॒षा वै पि॑तृ॒णां दिख् स्वाया॑मे॒व दि॒शि पि॒तॄन् नि॒रव॑दयत उद्गा॒तृभ्यो॑ हरन्ति सामदेव॒त्यो॑ वै सौ॒म्यो यदे॒व साम्नः॑ छंबट्कु॒र्वन्ति॒ तस्यै॒व स शान्ति॒रवे᳚- [  ] \newline

\textbf{Pada Paata} \newline

घ्नन्ति॑ । वा । ए॒तत् । सोम᳚म् । यत् । अ॒भि॒षु॒ण्वन्तीत्य॑भि-सु॒न्वन्ति॑ । यत् । सौ॒म्यः । भव॑ति । यथा᳚ । मृ॒ताय॑ । अ॒नु॒स्तर॑णी॒मित्य॑नु - स्तर॑णीम् । घ्नन्ति॑ । ता॒दृक् । ए॒व । तत् । यत् । उ॒त्त॒रा॒द्‌र्ध इत्यु॑त्तर - अ॒द्‌र्धे । वा॒ । मद्ध्ये᳚ । वा॒ । जु॒हु॒यात् । दे॒वता᳚भ्यः । स॒मद॒मिति॑ स - मद᳚म् । द॒द्ध्या॒त् । द॒क्षि॒णा॒द्‌र्ध इति॑ दक्षिण - अ॒द्‌र्धे । जु॒हो॒ति॒ । ए॒षा । वै । पि॒तृ॒णाम् । दिक् । स्वाया᳚म् । ए॒व । दि॒शि । पि॒तॄन् । नि॒रव॑दयत॒ इति॑ निः - अव॑दयते । उ॒द्गा॒तृभ्य॒ इत्यु॑द्गा॒तृ - भ्यः॒ । ह॒र॒न्ति॒ । सा॒म॒दे॒व॒त्य॑ इति॑ साम - दे॒व॒त्यः॑ । वै । सौ॒म्यः । यत् । ए॒व । साम्नः॑ । छ॒बं॒ट्कु॒र्वन्तीति॑ छंबट् - कु॒र्वन्ति॑ । तस्य॑ । ए॒व । सः । शान्तिः॑ । अवेति॑ ।  \newline




\markright{ TS 6.6.7.2  \hfill https://www.vedavms.in \hfill}

\section{ TS 6.6.7.2 }

\textbf{TS 6.6.7.2 } \newline
\textbf{Samhita Paata} \newline

-क्षन्ते प॒वित्रं॒ ॅवै सौ॒म्य आ॒त्मान॑मे॒व प॑वयन्ते॒ य आ॒त्मानं॒ न प॑रि॒पश्ये॑दि॒तासुः॑ स्यादभिद॒दिं कृ॒त्वाऽवे᳚क्षेत॒ तस्मि॒न॒. ह्या᳚त्मानं॑ परि॒पश्य॒त्यथो॑ आ॒त्मान॑मे॒व प॑वयते॒ यो ग॒तम॑नाः॒ स्याथ् सोऽवे᳚क्षेत॒ यन्मे॒ मनः॒ परा॑गतं॒ ॅयद्वा॑ मे॒ अप॑रागतं । राज्ञा॒ सोमे॑न॒ तद्व॒यम॒स्मासु॑ धारयाम॒सीति॒ मन॑ ए॒वात्मन् दा॑धार॒- [  ] \newline

\textbf{Pada Paata} \newline

ई॒क्ष॒न्ते॒ । प॒वित्र᳚म् । वै । सौ॒म्यः । आ॒त्मान᳚म् । ए॒व । प॒व॒य॒न्ते॒ । यः । आ॒त्मान᳚म् । न । प॒रि॒पश्ये॒दिति॑ परि-पश्ये᳚त् । इ॒तासु॒रिती॒त-अ॒सुः॒ । स्या॒त् । अ॒भि॒द॒दिमित्य॑भि - द॒दिम् । कृ॒त्वा । अवेति॑ । ई॒क्षे॒त॒ । तस्मिन्न्॑ । हि । आ॒त्मान᳚म् । प॒रि॒पश्य॒तीति॑ परि - पश्य॑ति । अथो॒ इति॑ । आ॒त्मान᳚म् । ए॒व । प॒व॒य॒ते॒ । यः । ग॒तम॑ना॒ इति॑ ग॒त - म॒नाः॒ । स्यात् । सः । अवेति॑ । ई॒क्षे॒त॒ । यत् । मे॒ । मनः॑ । परा॑गत॒मिति॒ परा᳚-ग॒त॒म् । यत् । वा॒ । मे॒ । अप॑रागत॒मित्यप॑रा-ग॒त॒म् ॥ राज्ञा᳚ । सोमे॑न । तत् । व॒यम् । अ॒स्मासु॑ । धा॒र॒या॒म॒सि॒ । इति॑ । मनः॑ । ए॒व । आ॒त्मन्न् । दा॒धा॒र॒ ।  \newline




\markright{ TS 6.6.7.3  \hfill https://www.vedavms.in \hfill}

\section{ TS 6.6.7.3 }

\textbf{TS 6.6.7.3 } \newline
\textbf{Samhita Paata} \newline

न ग॒तम॑ना भव॒त्यप॒ वै तृ॑तीयसव॒ने य॒ज्ञ्ः क्रा॑मतीजा॒ना-दनी॑जानम॒भ्या᳚-ग्नावैष्ण॒व्यर्चा घृ॒तस्य॑ यजत्य॒ग्निः सर्वा॑ दे॒वता॒ विष्णु॑र्य॒ज्ञो दे॒वता᳚श्चै॒व य॒ज्ञ्ं च॑ दाधारोपाꣳ॒॒शु य॑जति मिथुन॒त्वाय॑ ब्रह्मवा॒दिनो॑ वदन्ति मि॒त्रो य॒ज्ञ्स्य॒ स्वि॑ष्टं ॅयुवते॒ वरु॑णो॒ दुरि॑ष्टं॒ क्व॑ तर्.हि॑ य॒ज्ञ्ः क्व॑ यज॑मानो भव॒तीति॒ यन्मै᳚त्रावरु॒णीं ॅव॒शामा॒लभ॑ते मि॒त्रेणै॒व- [  ] \newline

\textbf{Pada Paata} \newline

न । ग॒तम॑ना॒ इति॑ ग॒त - म॒नाः॒ । भ॒व॒ति॒ । अपेति॑ । वै । तृ॒ती॒य॒स॒व॒न इति॑ तृतीय - स॒व॒ने । य॒ज्ञ्ः । क्रा॒म॒ति॒ । ई॒जा॒नात् । अनी॑जानम् । अ॒भीति॑ । आ॒ग्ना॒वै॒ष्ण॒व्येत्या᳚ग्ना - वै॒ष्ण॒व्या । ऋ॒चा । घृ॒तस्य॑ । य॒ज॒ति॒ । अ॒ग्निः । सर्वाः᳚ । दे॒वताः᳚ । विष्णुः॑ । य॒ज्ञ्ः । दे॒वताः᳚ । च॒ । ए॒व । य॒ज्ञ्म् । च॒ । दा॒धा॒र॒ । उ॒पाꣳ॒॒श्वित्यु॑प - अꣳ॒॒शु । य॒ज॒ति॒ । मि॒थु॒न॒त्वायेति॑ मिथुन - त्वाय॑ । ब्र॒ह्म॒वा॒दिन॒ इति॑ ब्रह्म - वा॒दिनः॑ । व॒द॒न्ति॒ । मि॒त्रः । य॒ज्ञ्स्य॑ । स्वि॑ष्ट॒मिति॒ सु-इ॒ष्ट॒म् । यु॒व॒ते॒ । वरु॑णः । दुरि॑ष्ट॒मिति॒ दुः - इ॒ष्ट॒म् । क्व॑ । तर्.हि॑ । य॒ज्ञ्ः । क्व॑ । यज॑मानः । भ॒व॒ति॒ । इति॑ । यत् । मै॒त्रा॒व॒रु॒णीमिति॑ मैत्रा - व॒रु॒णीम् । व॒शाम् । आ॒लभ॑त॒ इत्या᳚ - लभ॑ते । मि॒त्रेण॑ । ए॒व ।  \newline




\markright{ TS 6.6.7.4  \hfill https://www.vedavms.in \hfill}

\section{ TS 6.6.7.4 }

\textbf{TS 6.6.7.4 } \newline
\textbf{Samhita Paata} \newline

य॒ज्ञ्स्य॒ स्वि॑ष्टꣳ शमयति॒ वरु॑णेन॒ दुरि॑ष्टं॒ नाऽऽ*र्ति॒मार्च्छ॑ति॒ यज॑मानो॒ यथा॒ वै लाङ्ग॑लेनो॒र्वरां᳚ प्रभि॒न्दन्-त्ये॒वमृ॑ख्सा॒मे य॒ज्ञ्ं प्र भि॑न्तो॒ यन्मै᳚त्रावरु॒णीं ॅव॒शामा॒लभ॑ते य॒ज्ञायै॒व प्रभि॑न्नाय म॒त्य॑म॒न्ववा᳚स्यति॒ शान्त्यै॑ या॒तया॑मानि॒ वा ए॒तस्य॒ छन्दाꣳ॑सि॒ य ई॑जा॒नः छन्द॑सामे॒ष रसो॒ यद्-व॒शा यन्मै᳚त्रावरु॒णीं ॅव॒शामा॒लभ॑ते॒ छन्दाꣳ॑स्ये॒व पुन॒रा प्री॑णा॒त्य ( ) या॑तयामत्वा॒याथो॒ छन्द॑स्स्वे॒व रसं॑ दधाति ॥ \newline

\textbf{Pada Paata} \newline

य॒ज्ञ्स्य॑ । स्वि॑ष्ट॒मिति॒ सु - इ॒ष्ट॒म् । श॒म॒य॒ति॒ । वरु॑णेन । दुरि॑ष्ट॒मिति॒ दुः - इ॒ष्ट॒म् । न । आर्ति᳚म् । एति॑ । ऋ॒च्छ॒ति॒ । यज॑मानः । यथा᳚ । वै । लाङ्ग॑लेन । उ॒र्वरा᳚म् । प्र॒भि॒न्दन्तीति॑ प्र - भि॒न्दन्ति॑ । ए॒वम् । ऋ॒ख्सा॒मे इत्यृ॑क् - सा॒मे । य॒ज्ञ्म् । प्रेति॑ । भि॒न्तः॒ । यत् । मै॒त्रा॒व॒रु॒णीमिति॑ मैत्रा - व॒रु॒णीम् । व॒शाम् । आ॒ल॑भत॒ इत्या᳚-लभ॑ते । य॒ज्ञाय॑ । ए॒व । प्रभि॑न्ना॒येति॒ प्र - भि॒न्ना॒य॒ । म॒त्य᳚म् । अ॒न्ववा᳚स्य॒तीत्य॑नु-अवा᳚स्यति । शान्त्यै᳚ । या॒तया॑मा॒नीति॑ या॒त-या॒मा॒नि॒ । वै । ए॒तस्य॑ । छन्दाꣳ॑सि । यः । ई॒जा॒नः । छन्द॑साम् । ए॒षः । रसः॑ । यत् । व॒शा । यत् । मै॒त्रा॒व॒रु॒णीमिति॑ मैत्रा - व॒रु॒णीम् । व॒शाम् । आ॒लभ॑त॒ इत्या᳚ - लभ॑ते । छन्दाꣳ॑सि । ए॒व । पुनः॑ । एति॑ । प्री॒णा॒ति॒ ( ) । अया॑तयामत्वा॒येत्यया॑तयाम - त्वा॒य॒ । अथो॒ इति॑ । छन्द॒स्स्विति॒ छन्दः॑ - सु॒ । ए॒व । रस᳚म् । द॒धा॒ति॒ ॥  \newline




\markright{ TS 6.6.8.1  \hfill https://www.vedavms.in \hfill}

\section{ TS 6.6.8.1 }

\textbf{TS 6.6.8.1 } \newline
\textbf{Samhita Paata} \newline

दे॒वा वा इ॑न्द्रि॒यं ॅवी॒र्यां᳚(1॒) ॅव्य॑भजन्त॒ ततो॒ यद॒त्यशि॑ष्यत॒ तद॑तिग्रा॒ह्या॑ अभव॒न् तद॑तिग्रा॒ह्या॑णा-मतिग्राह्य॒त्वं ॅयद॑तिग्रा॒ह्या॑ गृ॒ह्यन्त॑ इन्द्रि॒यमे॒व तद्-वी॒र्यं॑ ॅयज॑मान आ॒त्मन् ध॑त्ते॒ तेज॑ आग्ने॒येने᳚न्द्रि॒य-मै॒न्द्रेण॑ ब्रह्मवर्च॒सꣳ सौ॒र्येणो॑प॒स्तंभ॑नं॒ ॅवा ए॒तद्-य॒ज्ञ्स्य॒ यद॑तिग्रा॒ह्या᳚श्च॒क्रे पृ॒ष्ठानि॒ यत् पृष्ठ्ये॒ न गृ॑ह्णी॒यात् प्राञ्चं॑ ॅय॒ज्ञ्ं पृ॒ष्ठानि॒ सꣳ शृ॑णीयु॒र्यदु॒क्थ्ये॑- [  ] \newline

\textbf{Pada Paata} \newline

दे॒वा । वै । इ॒न्द्रि॒यम् । वी॒र्या᳚म् । वीति॑ । अ॒भ॒ज॒न्त॒ । ततः॑ । यत् । अ॒त्यशि॑ष्य॒तेत्य॑ति - अशि॑ष्यत । तत् । अ॒ति॒ग्रा॒ह्या॑ इत्य॑ति-ग्रा॒ह्याः᳚ । अ॒भ॒व॒न्न् । तत् । अ॒ति॒ग्रा॒ह्या॑णा॒मित्य॑ति - ग्रा॒ह्या॑णाम् । अ॒ति॒ग्रा॒ह्य॒त्वमित्य॑तिग्राह्य - त्वम् । यत् । अ॒ति॒ग्रा॒ह्या॑ इत्य॑ति-ग्रा॒ह्याः᳚ । गृ॒ह्यन्ते᳚ । इ॒न्द्रि॒यम् । ए॒व । तत् । वी॒र्य᳚म् । यज॑मानः । आ॒त्मन्न् । ध॒त्ते॒ । तेजः॑ । आ॒ग्ने॒येन॑ । इ॒न्द्रि॒यम् । ऐ॒न्द्रेण॑ । ब्र॒ह्म॒व॒र्च॒समिति॑ ब्रह्म - व॒र्च॒सम् । सौ॒र्येण॑ । उ॒प॒स्तंभ॑न॒मित्यु॑प - स्तंभ॑नम् । वै । ए॒तत् । य॒ज्ञ्स्य॑ । यत् । अ॒ति॒ग्रा॒ह्या॑ इत्य॑ति - ग्रा॒ह्याः᳚ । च॒क्रे इति॑ । पृ॒ष्ठानि॑ । यत् । पृष्ठ्ये᳚ । न । गृ॒ह्णी॒यात् । प्राञ्च᳚म् । य॒ज्ञ्म् । पृ॒ष्ठानि॑ । समिति॑ । शृ॒णी॒युः॒ । यत् । उ॒क्थ्ये᳚ ।  \newline




\markright{ TS 6.6.8.2  \hfill https://www.vedavms.in \hfill}

\section{ TS 6.6.8.2 }

\textbf{TS 6.6.8.2 } \newline
\textbf{Samhita Paata} \newline

गृह्णी॒यात् प्र॒त्यञ्चं॑ ॅय॒ज्ञ्म॑तिग्रा॒ह्याः᳚ सꣳ शृ॑णीयुर्विश्व॒जिति॒ सर्व॑पृष्ठे ग्रहीत॒व्या॑ य॒ज्ञ्स्य॑ सवीर्य॒त्वाय॑ प्र॒जाप॑तिर्दे॒वेभ्यो॑ य॒ज्ञान् व्यादि॑श॒थ् स प्रि॒यास्त॒नूरप॒ न्य॑धत्त॒ तद॑तिग्रा॒ह्या॑ अभव॒न् वित॑नु॒स्तस्य॑ य॒ज्ञ् इत्या॑हु॒र्य-स्या॑तिग्रा॒ह्या॑ न गृ॒ह्यन्त॒ इत्यप्य॑ग्निष्टो॒मे ग्र॑हीत॒व्या॑ य॒ज्ञ्स्य॑ सतनु॒त्वाय॑ दे॒वता॒ वै सर्वाः᳚ स॒दृशी॑रास॒न् ता न व्या॒वृत॑-मगच्छ॒न् ते दे॒वा- [  ] \newline

\textbf{Pada Paata} \newline

गृ॒ह्णी॒यात् । प्र॒त्यञ्च᳚म् । य॒ज्ञ्म् । अ॒ति॒ग्रा॒ह्या॑ इत्य॑ति - ग्रा॒ह्याः᳚ । समिति॑ । शृ॒णी॒युः॒ । वि॒श्व॒जितीति॑ विश्व - जिति॑ । सर्व॑पृष्ठ॒ इति॒ सर्व॑ - पृ॒ष्ठे॒ । ग्र॒ही॒त॒व्याः᳚ । य॒ज्ञ्स्य॑ । स॒वी॒र्य॒त्वायेति॑ सवीर्य-त्वाय॑ । प्र॒जाप॑ति॒रिति॑ प्र॒जा - प॒तिः॒ । दे॒वेभ्यः॑ । य॒ज्ञान् । व्यादि॑श॒दिति॑ वि - आदि॑शत् । सः । प्रि॒याः । त॒नूः । अप॑ । नीति॑ । अ॒ध॒त्त॒ । तत् । अ॒ति॒ग्रा॒ह्या॑ इत्य॑ति-ग्रा॒ह्याः᳚ । अ॒भ॒व॒न्न् । वित॑नु॒रिति॒ वि-त॒नुः॒ । तस्य॑ । य॒ज्ञ्ः । इति॑ । आ॒हुः॒ । यस्य॑ । अ॒ति॒ग्रा॒ह्या॑ इत्य॑ति - ग्रा॒ह्याः᳚ । न । गृ॒ह्यन्ते᳚ । इति॑ । अपीति॑ । अ॒ग्नि॒ष्टो॒म इत्य॑ग्नि - स्तो॒मे । ग्र॒ही॒त॒व्याः᳚ । य॒ज्ञ्स्य॑ । स॒त॒नु॒त्वायेति॑ सतनु - त्वाय॑ । दे॒वताः᳚ । वै । सर्वाः᳚ । स॒दृशीः᳚ । आ॒स॒न्न् । ताः । न । व्या॒वृत॒मिति॑ वि-आ॒वृत᳚म् । अ॒ग॒च्छ॒न्न् । ते । दे॒वाः ।  \newline




\markright{ TS 6.6.8.3  \hfill https://www.vedavms.in \hfill}

\section{ TS 6.6.8.3 }

\textbf{TS 6.6.8.3 } \newline
\textbf{Samhita Paata} \newline

ए॒त ए॒तान् ग्रहा॑नपश्य॒न् तान॑गृह्णताऽऽ*ग्ने॒ यम॒ग्निरै॒न्द्रमिन्द्रः॑ सौ॒र्यꣳ सूर्य॒स्ततो॒ वै ते᳚ऽन्याभि॑-र्दे॒वता॑भि-र्व्या॒वृत॑मगच्छ॒न्॒. यस्यै॒वं ॅवि॒दुष॑ ए॒ते ग्रहा॑ गृ॒ह्यन्ते᳚ व्या॒वृत॑मे॒व पा॒प्मना॒ भ्रातृ॑व्येण गच्छती॒मे लो॒का ज्योति॑ष्मन्तः स॒माव॑द्-वीर्याः का॒र्या॑ इत्या॑हुराग्ने॒येना॒स्मिन् ॅलो॒के ज्योति॑र्द्धत्त ऐ॒न्द्रेणा॒न्तरि॑क्ष इन्द्रवा॒यू हि स॒युजौ॑ सौ॒र्येणा॒मुष्मि॑न् ॅलो॒के- [  ] \newline

\textbf{Pada Paata} \newline

ए॒ते । ए॒तान् । ग्रहान्॑ । अ॒प॒श्य॒न्न् । तान् । अ॒गृ॒ह्ण॒त॒ । आ॒ग्ने॒यम् । अ॒ग्निः । ऐ॒न्द्रम् । इन्द्रः॑ । सौ॒र्यम् । सूर्यः॑ । ततः॑ । वै । ते । अ॒न्याभिः॑ । दे॒वता॑भिः । व्या॒वृत॒मिति॑ वि - आ॒वृत᳚म् । अ॒ग॒च्छ॒न्न् । यस्य॑ । ए॒वम् । वि॒दुषः॑ । ए॒ते । ग्रहाः᳚ । गृ॒ह्यन्ते᳚ । व्या॒वृत॒मिति॑ वि-आ॒वृत᳚म् । ए॒व । पा॒प्मना᳚ । भ्रातृ॑व्येण । ग॒च्छ॒ति॒ । इ॒मे । लो॒काः । ज्योति॑ष्मन्तः । स॒माव॑द्वीर्या॒ इति॑ स॒माव॑त् - वी॒र्याः॒ । का॒र्याः᳚ । इति॑ । आ॒हुः॒ । आ॒ग्ने॒येन॑ । अ॒स्मिन्न् । लो॒के । ज्योतिः॑ । ध॒त्ते॒ । ऐ॒न्द्रेण॑ । अ॒न्तरि॑क्षे । इ॒न्द्र॒वा॒यू इती᳚न्द्र - वा॒यू । हि । स॒युजा॒विति॑ स-युजौ᳚ । सौ॒र्येण॑ । अ॒मुष्मिन्न्॑ । लो॒के ।  \newline




\markright{ TS 6.6.8.4  \hfill https://www.vedavms.in \hfill}

\section{ TS 6.6.8.4 }

\textbf{TS 6.6.8.4 } \newline
\textbf{Samhita Paata} \newline

ज्योति॑र्द्धत्ते॒ ज्योति॑ष्मन्तोऽस्मा इ॒मे लो॒का भ॑वन्ति स॒माव॑द्-वीर्यानेनान् कुरुत ए॒तान्. वै ग्रहा᳚न् ब॒बां-वि॒श्वव॑यसा-ववित्तां॒ ताभ्या॑मि॒मे लो॒काः परा᳚ञ्चश्चा॒र्वाञ्च॑श्च॒ प्राभु॒र्यस्यै॒वं ॅवि॒दुष॑ ए॒ते ग्रहा॑ गृ॒ह्यन्ते॒ प्रास्मा॑ इ॒मे लो॒काः परा᳚ञ्चश्चा॒र्वाञ्च॑श्च भान्ति ॥ \newline

\textbf{Pada Paata} \newline

ज्योतिः॑ । ध॒त्ते॒ । ज्योति॑ष्मन्तः । अ॒स्मै॒ । इ॒मे । लो॒काः । भ॒व॒न्ति॒ । स॒माव॑त्वीर्या॒निति॑ स॒माव॑त् - वी॒र्या॒न् । ए॒ना॒न् । कु॒रु॒ते॒ । ए॒तान् । वै । ग्रहान्॑ । ब॒बांवि॒श्वव॑यसा॒विति॑ ब॒बां - वि॒श्वव॑यसौ । अ॒वि॒त्ता॒म् । ताभ्या᳚म् । इ॒मे । लो॒काः । परा᳚ञ्चः । च॒ । अ॒र्वाञ्चः॑ । च॒ । प्रेति॑ । अ॒भुः॒ । यस्य॑ । ए॒वम् । वि॒दुषः॑ । ए॒ते । ग्रहाः᳚ । गृ॒ह्यन्ते᳚ । प्रेति॑ । अ॒स्मै॒ । इ॒मे । लो॒काः । परा᳚ञ्चः । च॒ । अ॒र्वाञ्चः॑ । च॒ । भा॒न्ति॒ ॥  \newline




\markright{ TS 6.6.9.1  \hfill https://www.vedavms.in \hfill}

\section{ TS 6.6.9.1 }

\textbf{TS 6.6.9.1 } \newline
\textbf{Samhita Paata} \newline

दे॒वा वै यद्-य॒ज्ञेऽकु॑र्वत॒ तदसु॑रा अकुर्वत॒ ते दे॒वा अदा᳚भ्ये॒ छन्दाꣳ॑सि॒ सव॑नानि॒ सम॑स्थापय॒न् ततो॑ दे॒वा अभ॑व॒न् पराऽसु॑रा॒ यस्यै॒वं ॅवि॒दुषोऽदा᳚भ्यो गृ॒ह्यते॒ भव॑त्या॒त्मना॒ परा᳚ऽस्य॒ भ्रातृ॑व्यो भवति॒ यद्वै दे॒वा असु॑रा॒-नदा᳚भ्ये॒-नाद॑भ्नुव॒न् तददा᳚भ्यस्या-दाभ्य॒ त्वं ॅय ए॒वं ॅवेद॑ द॒भ्नोत्ये॒व भ्रातृ॑व्यं॒ नैनं॒ भ्रातृ॑व्यो दभ्नोत्ये॒- [  ] \newline

\textbf{Pada Paata} \newline

दे॒वाः । वै । यत् । य॒ज्ञे । अकु॑र्वत । तत् । असु॑राः । अ॒कु॒र्व॒त॒ । ते । दे॒वाः । अदा᳚भ्ये । छन्दाꣳ॑सि । सव॑नानि । समिति॑ । अ॒स्था॒प॒य॒न्न् । ततः॑ । दे॒वाः । अभ॑वन्न् । परेति॑ । असु॑राः । यस्य॑ । ए॒वम् । वि॒दुषः॑ । अदा᳚भ्यः । गृ॒ह्यते᳚ । भव॑ति । आ॒त्मना᳚ । परेति॑ । अ॒स्य॒ । भ्रातृ॑व्यः । भ॒व॒ति॒ । यत् । वै । दे॒वाः । असु॑रान् । अदा᳚भ्येन । अद॑भ्नुवन्न् । तत् । अदा᳚भ्यस्य । अ॒दा॒भ्य॒त्वमित्य॑दाभ्य - त्वम् । यः । ए॒वम् । वेद॑ । द॒भ्नोति॑ । ए॒व । भ्रातृ॑व्यम् । न । ए॒न॒म् । भ्रातृ॑व्यः । द॒भ्नो॒ति॒ ।  \newline




\markright{ TS 6.6.9.2  \hfill https://www.vedavms.in \hfill}

\section{ TS 6.6.9.2 }

\textbf{TS 6.6.9.2 } \newline
\textbf{Samhita Paata} \newline

-षा वै प्र॒जाप॑ते-रतिमो॒क्षिणी॒ नाम॑ त॒नूर्यददा᳚भ्य॒ उप॑नद्धस्य गृह्णा॒त्यति॑मुक्त्या॒ अति॑ पा॒प्मानं॒ भ्रातृ॑व्यं मुच्यते॒ य ए॒वं ॅवेद॒ घ्नन्ति॒ वा ए॒तथ् सोमं॒ ॅयद॑भिषु॒ण्वन्ति॒ सोमे॑ ह॒न्यमा॑ने य॒ज्ञो ह॑न्यते य॒ज्ञे यज॑मानो ब्रह्मवा॒दिनो॑ वदन्ति॒ किं तद्-य॒ज्ञे यज॑मानः कुरुते॒ येन॒ जीवन्᳚थ् सुव॒र्गं ॅलो॒कमेतीति॑ जीवग्र॒हो वा ए॒ष यददा॒भ्यो ऽन॑भिषुतस्य गृह्णाति॒ ( ) जीव॑न्तमे॒वैनꣳ॑ सुव॒र्गं ॅलो॒कं ग॑मयति॒ वि वा ए॒तद्-य॒ज्ञ्ं छि॑न्दन्ति॒ यददा᳚भ्ये सꣳ-स्था॒पय॑-न्त्यꣳ॒॒शूनपि॑ सृजति य॒ज्ञ्स्य॒ सन्त॑त्यै ॥ \newline

\textbf{Pada Paata} \newline

ए॒षा । वै । प्र॒जाप॑ते॒रिति॑ प्र॒जा-प॒तेः॒ । अ॒ति॒मो॒क्षिणीत्य॑ति-मो॒क्षिणी᳚ । नाम॑ । त॒नूः । यत् । अदा᳚भ्यः । उप॑नद्ध॒स्येत्युप॑-न॒द्ध॒स्य॒ । गृ॒ह्णा॒ति॒ । अति॑मुक्त्या॒ इत्यति॑ - मु॒क्त्यै॒ । अतीति॑ । पा॒प्मान᳚म् । भ्रातृ॑व्यम् । मु॒च्य॒ते॒ । यः । ए॒वम् । वेद॑ । घ्नन्ति॑ । वै । ए॒तत् । सोम᳚म् । यत् । अ॒भि॒षु॒ण्वन्तीत्य॑भि - सु॒न्वन्ति॑ । सोमे᳚ । ह॒न्यमा॑ने । य॒ज्ञ्ः । ह॒न्य॒ते॒ । य॒ज्ञे । यज॑मानः । ब्र॒ह्म॒वा॒दिन॒ इति॑ ब्रह्म-वा॒दिनः॑ । व॒द॒न्ति॒ । किम् । तत् । य॒ज्ञे । यज॑मानः । कु॒रु॒ते॒ । येन॑ । जीवन्न्॑ । सु॒व॒र्गमिति॑ सुवः - गम् । लो॒कम् । एति॑ । इति॑ । जी॒व॒ग्र॒ह इति॑ जीव - ग्र॒हः । वै । ए॒षः । यत् । अदा᳚भ्यः । अन॑भिषुत॒स्येत्यन॑भि - सु॒त॒स्य॒ । गृ॒ह्णा॒ति॒ ( ) । जीव॑न्तम् । ए॒व । ए॒न॒म् । सु॒व॒र्गमिति॑ सुवः - गम् । लो॒कम् । ग॒म॒य॒ति॒ । वीति॑ । वै । ए॒तत् । य॒ज्ञ्म् । छि॒न्द॒न्ति॒ । यत् । अदा᳚भ्ये । सꣳ॒॒स्था॒पय॒न्तीति॑ सं - स्था॒पय॑न्ति । अꣳ॒॒शून् । अपीति॑ । स॒ज॒ति॒ । य॒ज्ञ्स्य॑ । सन्त॑त्या॒ इति॒ सं - त॒त्यै॒ ॥  \newline




\markright{ TS 6.6.10.1  \hfill https://www.vedavms.in \hfill}

\section{ TS 6.6.10.1 }

\textbf{TS 6.6.10.1 } \newline
\textbf{Samhita Paata} \newline

दे॒वा वै प्र॒बाहु॒ग्ग्रहा॑-नगृह्णत॒ स ए॒तं प्र॒जाप॑ति-रꣳ॒॒शु-म॑पश्य॒त् तम॑गृह्णीत॒ तेन॒ वै स आ᳚र्द्ध्नो॒द्-यस्यै॒वं ॅवि॒दुषो॒ऽꣳ॒शु-र्गृ॒ह्यत॑ ऋ॒द्ध्नोत्ये॒व स॒कृद॑भिषुतस्य गृह्णाति स॒कृद्धि स तेनाऽऽ*र्द्ध्नो॒न्मन॑सा गृह्णाति॒ मन॑ इव॒ हि प॒जाप॑तिः प्र॒जाप॑ते॒राप्त्या॒ औदु॑बंरेण गृह्णा॒त्यूर्ग्वा उ॑दु॒बंर॒ ऊर्ज॑मे॒वाव॑ रुन्धे॒ चतुः॑स्रक्ति भवति दि॒क्ष्वे॑- [  ] \newline

\textbf{Pada Paata} \newline

दे॒वाः । वै । प्र॒बाहु॒गिति॑ प्र - बाहु॑क् । ग्रहान्॑ । अ॒गृ॒ह्ण॒त॒ । सः । ए॒तम् । प्र॒जाप॑ति॒रिति॑ प्र॒जा - प॒तिः॒ । अꣳ॒॒शुम् । अ॒प॒श्य॒त् । तम् । अ॒गृ॒ह्णी॒त॒ । तेन॑ । वै । सः । आ॒द्‌र्ध्नो॒त् । यस्य॑ । ए॒वम् । वि॒दुषः॑ । अꣳ॒॒शुः । गृ॒ह्यते᳚ । ऋ॒द्ध्नोति॑ । ए॒व । स॒कृद॑भिषुत॒स्येति॑ स॒कृत् - अ॒भि॒षु॒त॒स्य॒ । गृ॒ह्णा॒ति॒ । स॒कृत् । हि । सः । तेन॑ । आद्‌र्ध्नो᳚त् । मन॑सा । गृ॒ह्णा॒ति॒ । मनः॑ । इ॒व॒ । हि । प्र॒जाप॑ति॒रिति॑ प्र॒जा - प॒तिः॒ । प्र॒जाप॑ते॒रिति॑ प्र॒जा - प॒तेः॒ । आप्त्यै᳚ । औदु॑बंरेण । गृ॒ह्णा॒ति॒ । ऊर्क् । वै । उ॒दु॒म्बरः॑ । ऊर्ज᳚म् । ए॒व । अवेति॑ । रु॒न्धे॒ । चतु॑स्स्र॒क्तीति॒ चतुः॑ - स्र॒क्ति॒ । भ॒व॒ति॒ । दि॒क्षु ।  \newline




\markright{ TS 6.6.10.2  \hfill https://www.vedavms.in \hfill}

\section{ TS 6.6.10.2 }

\textbf{TS 6.6.10.2 } \newline
\textbf{Samhita Paata} \newline

-व प्रति॑ तिष्ठति॒ यो वा अꣳ॒॒शोरा॒यत॑नं॒ ॅवेदा॒ऽऽयत॑नवान् भवति वामदे॒व्यमिति॒ साम॒ तद्वा अ॑स्या॒ऽऽ*यत॑नं॒ मन॑सा॒ गाय॑मानो गृह्णात्या॒यत॑नवाने॒व भ॑वति॒ यद॑द्ध्व॒र्युरꣳ॒॒शुं गृ॒ह्णन् नार्द्धये॑दु॒भाभ्यां॒ नर्द्ध्ये॑ताद्ध्व॒र्यवे॑ च॒ यज॑मानाय च॒ यद॒र्द्धये॑-दु॒भाभ्या॑-मृद्ध्ये॒तान॑वानं गृह्णाति॒ सैवास्यर्द्धि॒र॒. हिर॑ण्यम॒भि व्य॑नित्य॒ ( )-मृतं॒ ॅवै हिर॑ण्य॒मायुः॑ प्रा॒ण आयु॑षै॒वामृत॑म॒भि धि॑नोति श॒तमा॑नं भवति श॒तायुः॒ पुरु॑षः श॒तेन्द्रि॑य॒ आयु॑ष्ये॒वेन्द्रि॒ये प्रति॑ तिष्ठति ॥ \newline

\textbf{Pada Paata} \newline

ए॒व । प्रतीति॑ । ति॒ष्ठ॒ति॒ । यः । वै । अꣳ॒॒शोः । आ॒यत॑न॒मित्या᳚-यत॑नम् । वेद॑ । आ॒यत॑नवा॒नित्या॒यत॑न - वा॒न् । भ॒व॒ति॒ । वा॒म॒दे॒व्यमिति॑ वाम - दे॒व्यम् । इति॑ । साम॑ । तत् । वै । अ॒स्य॒ । आ॒यत॑न॒मित्या᳚ - यत॑नम् । मन॑सा । गाय॑मानः । गृ॒ह्णा॒ति॒ । आ॒यत॑नवा॒नित्या॒यत॑न - वा॒न् । ए॒व । भ॒व॒ति॒ । यत् । अ॒द्ध्व॒र्युः । अꣳ॒॒शुम् । गृ॒ह्णन्न् । न । अ॒द्‌र्धये᳚त् । उ॒भाभ्या᳚म् । न । ऋ॒द्ध्ये॒त॒ । अ॒द्ध्व॒र्यवे᳚ । च॒ । यज॑मानाय । च॒ । यत् । अ॒द्‌र्धये᳚त् । उ॒भाभ्या᳚म् । ऋ॒द्ध्ये॒त॒ । अन॑वान॒मित्यन॑व - अ॒न॒म् । गृ॒ह्णा॒ति॒ । सा । ए॒व । अ॒स्य॒ । ऋद्धिः॑ । हिर॑ण्यम् । अ॒भि । वीति॑ । अ॒नि॒ति॒ ( ) । अ॒मृत᳚म् । वै । हिर॑ण्यम् । आयुः॑ । प्रा॒ण इति॑ प्र-अ॒नः । आयु॑षा । ए॒व । अ॒मृत᳚म् । अ॒भीति॑ । धि॒नो॒ति॒ । श॒तमा॑न॒मिति॑ श॒त - मा॒न॒म् । भ॒व॒ति॒ । श॒तायु॒रिति॑ श॒त - आ॒युः॒ । पुरु॑षः । श॒तेन्द्रि॑य॒ इति॑ श॒त-इ॒न्द्रि॒यः॒ । आयु॑षि । ए॒व । इ॒न्द्रि॒ये । प्रतीति॑ । ति॒ष्ठ॒ति॒ ॥  \newline




\markright{ TS 6.6.11.1  \hfill https://www.vedavms.in \hfill}

\section{ TS 6.6.11.1 }

\textbf{TS 6.6.11.1 } \newline
\textbf{Samhita Paata} \newline

प्र॒जाप॑ति-र्दे॒वेभ्यो॑ य॒ज्ञान् व्यादि॑श॒थ् स रि॑रिचा॒नो॑ऽमन्यत॒ स य॒ज्ञानाꣳ॑ षोडश॒धेन्द्रि॒यं ॅवी॒र्य॑मा॒त्मान॑म॒भि सम॑क्खिद॒त् तथ् षो॑ड॒श्य॑भव॒न्न वै षो॑ड॒शी नाम॑ य॒ज्ञो᳚ऽस्ति॒ यद्वाव षो॑ड॒शꣳ स्तो॒त्रꣳ षो॑ड॒शꣳ श॒स्त्रं तेन॑ षोड॒शी तथ् षो॑ड॒शिनः॑ षोडशि॒त्वं ॅयथ् षो॑ड॒शी गृ॒ह्यत॑ इन्द्रि॒यमे॒व तद्-वी॒र्यं॑ ॅयज॑मान आ॒त्मन् ध॑त्ते दे॒वेभ्यो॒ वै सु॑व॒र्गो लो॒को- [  ] \newline

\textbf{Pada Paata} \newline

प्र॒जाप॑ति॒रिति॑ प्र॒जा - प॒तिः॒ । दे॒वेभ्यः॑ । य॒ज्ञान् । व्यादि॑श॒दिति॑ वि - आदि॑शत् । सः । रि॒रि॒चा॒नः । अ॒म॒न्य॒त॒ । सः । य॒ज्ञाना᳚म् । षो॒ड॒श॒धेति॑ षोडश - धा । इ॒न्द्रि॒यम् । वी॒र्य᳚म् । आ॒त्मान᳚म् । अ॒भि । समिति॑ । अ॒क्खि॒द॒त् । तत् । षो॒ड॒शी । अ॒भ॒व॒त् । न । वै । षो॒ड॒शी । नाम॑ । य॒ज्ञ्ः । अ॒स्ति॒ । यत् । वाव । षो॒ड॒शम् । स्तो॒त्रम् । षो॒ड॒शम् । श॒स्त्रम् । तेन॑ । षो॒ड॒शी । तत् । षो॒ड॒शिनः॑ । षो॒ड॒शि॒त्वमिति॑ षोडशि -त्वम् । यत् । षो॒ड॒शी । गृ॒ह्यते᳚ । इ॒न्द्रि॒यम् । ए॒व । तत् । वी॒र्य᳚म् । यज॑मानः । आ॒त्मन्न् । ध॒त्ते॒ । दे॒वेभ्यः॑ । वै । सु॒व॒र्ग इति॑ सुवः - गः । लो॒कः ।  \newline




\markright{ TS 6.6.11.2  \hfill https://www.vedavms.in \hfill}

\section{ TS 6.6.11.2 }

\textbf{TS 6.6.11.2 } \newline
\textbf{Samhita Paata} \newline

न प्राभ॑व॒त् त ए॒तꣳ षो॑ड॒शिन॑मपश्य॒न् तम॑गृह्णत॒ ततो॒ वै तेभ्यः॑ सुव॒र्गो लो॒कः प्राभ॑व॒द्यथ् षो॑ड॒शी गृ॒ह्यते॑ सुव॒र्गस्य॑ लो॒कस्या॒भिजि॑त्या॒ इन्द्रो॒ वै दे॒वाना॑मानुजाव॒र आ॑सी॒थ् स प्र॒जाप॑ति॒मुपा॑धाव॒त् तस्मा॑ ए॒तꣳ षो॑ड॒शिनं॒ प्राय॑च्छ॒त् तम॑गृह्णीत॒ ततो॒ वै सोऽग्रं॑ दे॒वता॑नां॒ पर्यै॒द्-यस्यै॒वं ॅवि॒दुषः॑ षोड॒शी गृ॒ह्यते- [  ] \newline

\textbf{Pada Paata} \newline

न । प्रेति॑ । अ॒भ॒व॒त् । ते । ए॒तम् । षो॒ड॒शिन᳚म् । अ॒प॒श्य॒न्न् । तम् । अ॒गृ॒ह्ण॒त॒ । ततः॑ । वै । तेभ्यः॑ । सु॒व॒र्ग इति॑ सुवः - गः । लो॒कः । प्रेति॑ । अ॒भ॒व॒त् । यत् । षो॒ड॒शी । गृ॒ह्यते᳚ । सु॒व॒र्गस्येति॑ सुवः-गस्य॑ । लो॒कस्य॑ । अ॒भिजि॑त्या॒ इत्य॒भि - जि॒त्यै॒ । इन्द्रः॑ । वै । दे॒वाना᳚म् । आ॒नु॒जा॒व॒र इत्या॑नु - जा॒व॒रः । आ॒सी॒त् । सः । प्र॒जाप॑ति॒मिति॑ प्र॒जा - प॒ति॒म् । उपेति॑ । अ॒धा॒व॒त् । तस्मै᳚ । ए॒तम् । षो॒ड॒शिन᳚म् । प्रेति॑ । अ॒य॒च्छ॒त् । तम् । अ॒गृ॒ह्णी॒त॒ । ततः॑ । वै । सः । अग्र᳚म् । दे॒वता॑नाम् । परीति॑ । ऐ॒त् । यस्य॑ । ए॒वम् । वि॒दुषः॑ । षो॒ड॒शी । गृ॒ह्यते᳚ ।  \newline




\markright{ TS 6.6.11.3  \hfill https://www.vedavms.in \hfill}

\section{ TS 6.6.11.3 }

\textbf{TS 6.6.11.3 } \newline
\textbf{Samhita Paata} \newline

ऽग्र॑मे॒व स॑मा॒नानां॒ पर्ये॑ति प्रातस्सव॒ने गृ॑ह्णाति॒ वज्रो॒ वै षो॑ड॒शी वज्रः॑ प्रातस्सव॒नꣳ स्वादे॒वैनं॒ ॅयोने॒र्निगृ॑ह्णाति॒ सव॑नेसवने॒ऽभि गृ॑ह्णाति॒ सव॑नाथ्सवनादे॒वैनं॒ प्र ज॑नयति तृतीयसव॒ने प॒शुका॑मस्य गृह्णीया॒द्-वज्रो॒ वै षो॑ड॒शी प॒शव॑स्तृतीयसव॒नं ॅवज्रे॑णै॒वास्मै॑ तृतीयसव॒नात् प॒शूनव॑ रुन्धे॒ नोक्थ्ये॑ गृह्णीयात् प्र॒जा वै प॒शव॑ उ॒क्थानि॒ यदु॒क्थ्ये॑- [  ] \newline

\textbf{Pada Paata} \newline

अग्र᳚म् । ए॒व । स॒मा॒नाना᳚म् । परीति॑ । ए॒ति॒ । प्रा॒त॒स्स॒व॒न इति॑ प्रातः - स॒व॒ने । गृ॒ह्णा॒ति॒ । वज्रः॑ । वै । षो॒ड॒शी । वज्रः॑ । प्रा॒त॒स्स॒व॒नमिति॑ प्रातः - स॒व॒नम् । स्वात् । ए॒व । ए॒न॒म् । योनेः᳚ । निरिति॑ । गृ॒ह्णा॒ति॒ । सव॑नेसवन॒ इति॒ सव॑ने - स॒व॒ने॒ । अ॒भीति॑ । गृ॒ह्णा॒ति॒ । सव॑नाथ्सवना॒दिति॒ सव॑नात् - स॒व॒ना॒त् । ए॒व । ए॒न॒म् । प्रेति॑ । ज॒न॒य॒ति॒ । तृ॒ती॒य॒स॒व॒न इति॑ तृतीय - स॒व॒ने । प॒शुका॑म॒स्येति॑ प॒शु - का॒म॒स्य॒ । गृ॒ह्णी॒या॒त् । वज्रः॑ । वै । षो॒ड॒शी । प॒शवः॑ । तृ॒ती॒य॒स॒व॒नमिति॑ तृतीय - स॒व॒नम् । वज्रे॑ण । ए॒व । अ॒स्मै॒ । तृ॒ती॒य॒स॒व॒नादिति॑ तृतीय - स॒व॒नात् । प॒शून् । अवेति॑ । रु॒न्धे॒ । न । उ॒क्थ्ये᳚ । गृ॒ह्णी॒या॒त् । प्र॒जेति॑ प्र - जा । वै । प॒शवः॑ । उ॒क्थानि॑ । यत् । उ॒क्थ्ये᳚ ।  \newline




\markright{ TS 6.6.11.4  \hfill https://www.vedavms.in \hfill}

\section{ TS 6.6.11.4 }

\textbf{TS 6.6.11.4 } \newline
\textbf{Samhita Paata} \newline

गृह्णी॒यात् प्र॒जां प॒शून॑स्य॒ निर्द॑हेदतिरा॒त्रे प॒शुका॑मस्य गृह्णीया॒द्-वज्रो॒ वै षो॑ड॒शी वज्रे॑णै॒वास्मै॑ प॒शून॑व॒रुद्ध्य॒ रात्रि॑-यो॒परि॑ष्टा-च्छमय॒त्यप्य॑ग्निष्टो॒मे रा॑ज॒न्य॑स्य गृह्णीयाद्-व्या॒वृत्का॑मो॒ हि रा॑ज॒न्यो॑ यज॑ते सा॒ह्न ए॒वास्मै॒ वज्रं॑ गृह्णाति॒ स ए॑नं॒ ॅवज्रो॒ भूत्या॑ इन्धे॒ निर्वा॑ दह-त्येकविꣳ॒॒शꣳ स्तो॒त्रं भ॑वति॒ प्रति॑ष्ठित्यै॒ हरि॑वच्छस्यत॒ इन्द्र॑स्य प्रि॒यं धामो- [  ] \newline

\textbf{Pada Paata} \newline

गृ॒ह्णी॒यात् । प्र॒जामिति॑ प्र - जाम् । प॒शून् । अ॒स्य॒ । निरिति॑ । द॒हे॒त् । अ॒ति॒रा॒त्र इत्य॑ति - रा॒त्रे । प॒शुका॑म॒स्येति॑ प॒शु - का॒म॒स्य॒ । गृ॒ह्णी॒या॒त् । वज्रः॑ । वै । षो॒ड॒शी । वज्रे॑ण । ए॒व । अ॒स्मै॒ । प॒शून् । अ॒व॒रुद्ध्येत्य॑व - रुद्ध्य॑ । रात्रि॑या । उ॒परि॑ष्टात् । श॒म॒य॒ति॒ । अपीति॑ । अ॒ग्नि॒ष्टो॒म इत्य॑ग्नि - स्तो॒मे । रा॒ज॒न्य॑स्य । गृ॒ह्णी॒या॒त् । व्या॒वृत्का॑म॒ इति॑ व्या॒वृत् - का॒मः॒ । हि । रा॒ज॒न्यः॑ । यज॑ते । सा॒ह्न इति॑ स - अ॒ह्ने । ए॒व । अ॒स्मै॒ । वज्र᳚म् । गृ॒ह्णा॒ति॒ । सः । ए॒न॒म् । वज्रः॑ । भूत्यै᳚ । इ॒न्धे॒ । निरिति॑ । वा॒ । द॒ह॒ति॒ । ए॒क॒विꣳ॒॒शमित्ये॑क-विꣳ॒॒शम् । स्तो॒त्रम् । भ॒व॒ति॒ । प्रति॑ष्ठित्या॒ इति॒ प्रति॑ - स्थि॒त्यै॒ । हरि॑व॒दिति॒ हरि॑ - व॒त् । श॒स्य॒ते॒ । इन्द्र॑स्य । प्रि॒यम् । धाम॑ ।  \newline




\markright{ TS 6.6.11.5  \hfill https://www.vedavms.in \hfill}

\section{ TS 6.6.11.5 }

\textbf{TS 6.6.11.5 } \newline
\textbf{Samhita Paata} \newline

-पा᳚प्नोति॒ कनी॑याꣳसि॒ वै दे॒वेषु॒ छन्दाꣳ॒॒स्यास॒न्-ज्यायाꣳ॒॒-स्यसु॑रेषु॒ ते दे॒वाः कनी॑यसा॒ छन्द॑सा॒ ज्यायः॒ छन्दो॒ऽभि व्य॑शꣳस॒न् ततो॒ वै तेऽसु॑राणां ॅलो॒कम॑वृञ्जत॒ यत् कनी॑यसा॒ छन्द॑सा॒ ज्यायः॒ छन्दो॒ऽभिवि॒शꣳस॑ति॒ भ्रातृ॑व्यस्यै॒व तल्लो॒कं ॅवृ॑ङ्क्ते॒ षड॒क्षरा॒ण्यति॑ रेचयन्ति॒ षड् वा ऋ॒तव॑ ऋ॒तूने॒व प्री॑णाति च॒त्वारि॒ पूर्वा॒ण्यव॑ कल्पयन्ति॒- [  ] \newline

\textbf{Pada Paata} \newline

उपेति॑ । आ॒प्नो॒ति॒ । कनी॑याꣳसि । वै । दे॒वेषु॑ । छन्दाꣳ॑सि । आसन्न्॑ । ज्यायाꣳ॑सि ।   असु॑रेषु । ते । दे॒वाः । कनी॑यसा । छन्द॑सा । ज्यायः॑ । छन्दः॑ । अ॒भि । वीति॑ । अ॒शꣳ॒॒स॒न्न् । ततः॑ । वै । ते । असु॑राणाम् । लो॒कम् । अ॒वृ॒ञ्ज॒त॒ । यत् । कनी॑यसा । छन्द॑सा । ज्यायः॑ । छन्दः॑ । अ॒भीति॑ । वि॒शꣳस॒तीति॑ वि - शꣳस॑ति । भ्रातृ॑व्यस्य । ए॒व । तत् । लो॒कम् । वृ॒ङ्क्ते॒ । षट् । अ॒क्षरा॑णि । अतीति॑ । रे॒च॒य॒न्ति॒ । षट् । वै । ऋ॒तवः॑ । ऋ॒तून् । ए॒व । प्री॒णा॒ति॒ । च॒त्वारि॑ । पूर्वा॑णि । अवेति॑ । क॒ल्प॒य॒न्ति॒ ।  \newline




\markright{ TS 6.6.11.6  \hfill https://www.vedavms.in \hfill}

\section{ TS 6.6.11.6 }

\textbf{TS 6.6.11.6 } \newline
\textbf{Samhita Paata} \newline

चतु॑ष्पद ए॒व प॒शून॑व रुन्धे॒ द्वे उत्त॑रे द्वि॒पद॑ ए॒वाव॑ रुन्धे ऽनु॒ष्टुभ॑म॒भि सं पा॑दयन्ति॒ वाग्वा अ॑नु॒ष्टुप् तस्मा᳚त् प्रा॒णानां॒ ॅवागु॑त्त॒मा स॑मयाविषि॒ते सूर्ये॑ षोड॒शिनः॑ स्तो॒त्र-मु॒पाक॑रोत्ये॒तस्मि॒न् वै लो॒क इन्द्रो॑ वृ॒त्रम॑हन्थ् सा॒क्षादे॒व वज्रं॒ भ्रातृ॑व्याय॒ प्र ह॑र-त्यरुणपिश॒ङ्गोऽश्वो॒ दक्षि॑णै॒तद्वै वज्र॑स्य रू॒पꣳ समृ॑द्ध्यै ॥ \newline

\textbf{Pada Paata} \newline

चतु॑ष्पद॒ इति॒ चतुः॑-प॒दः॒ । ए॒व । प॒शून् । अवेति॑ । रु॒न्धे॒ । द्वे इति॑ । उत्त॑रे॒ इत्युत् - त॒रे॒ । द्वि॒पद॒ इति॑ द्वि - पदः॑ । ए॒व । अवेति॑ । रु॒न्धे॒ । अ॒नु॒ष्टुभ॒मित्य॑नु - स्तुभ᳚म् । अ॒भि । समिति॑ । पा॒द॒य॒न्ति॒ । वाक् । वै । अ॒नु॒ष्टुबित्य॑नु - स्तुप् । तस्मा᳚त् । प्रा॒णाना॒मिति॑ प्र - अ॒नाना᳚म् । वाक् । उ॒त्त॒मेत्यु॑त् - त॒मा । स॒म॒या॒वि॒षि॒त इति॑ समया - वि॒षि॒ते । सूर्ये᳚ । षो॒ड॒शिनः॑ । स्तो॒त्रम् । उ॒पाक॑रो॒तीत्यु॑प - आक॑रोति । ए॒तस्मिन्न्॑ । वै । लो॒के । इन्द्रः॑ । वृ॒त्रम् । अ॒ह॒न्न् । सा॒क्षादिति॑ स - अ॒क्षात् । ए॒व । वज्र᳚म् । भ्रातृ॑व्याय । प्रेति॑ । ह॒र॒ति॒ । अ॒रु॒ण॒पि॒श॒ङ्ग इत्य॑रुण - पि॒श॒ङ्गः । अश्वः॑ । दक्षि॑णा । ए॒तत् । वै । वज्र॑स्य । रू॒पम् । समृ॑द्ध्या॒ इति॒ सं - ऋ॒द्ध्यै॒ ॥  \newline






\end{document}