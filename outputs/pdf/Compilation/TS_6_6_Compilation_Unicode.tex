\documentclass[17pt]{extarticle}
\usepackage{babel}
\usepackage{fontspec}
\usepackage{polyglossia}
\usepackage{extsizes}

\usepackage{color}   %May be necessary if you want to color links
\usepackage{hyperref}
\hypersetup{
    colorlinks=true, %set true if you want colored links
    linktoc=all,     %set to all if you want both sections and subsections linked
    linkcolor=black,  %choose some color if you want links to stand out
}

\setmainlanguage{sanskrit}
\setotherlanguages{english} %% or other languages
\setlength{\parindent}{0pt}
\pagestyle{myheadings}
\newfontfamily\devanagarifont[Script=Devanagari]{AdishilaVedic}
\renewcommand{\theHsection}{\thepart.section.\thesection}

\newcommand{\VAR}[1]{}
\newcommand{\BLOCK}[1]{}




\begin{document}
\begin{titlepage}
    \begin{center}
 
\begin{sanskrit}
    { \Large
    कृष्ण यजुर्वेदीय तैत्तिरीय संहिता,पद,जटा,घन पाठः 
    }
    \\
    \vspace{2.5cm}
    \mbox{ \Large
    6.6       षष्ठकाण्डे षष्ठः प्रश्नः - सोममन्त्रब्राह्मणनिरूपणं   }
\end{sanskrit}
\end{center}

\end{titlepage}
\tableofcontents
\phantomsection
\pagebreak

\markright{ TS 6.6.1.1  \hfill https://www.vedavms.in \hfill}

\section{ TS 6.6.1.1 }

\textbf{TS 6.6.1.1 } \newline
\textbf{Samhita Paata} \newline

सु॒व॒र्गाय॒ वा ए॒तानि॑ लो॒काय॑ हूयन्ते॒ यद्-दा᳚क्षि॒णानि॒ द्वाभ्यां॒ गार्.ह॑पत्ये जुहोति द्वि॒पाद्-यज॑मानः॒ प्रति॑ष्ठित्या॒ आग्नी᳚द्ध्रे जुहोत्य॒न्तरि॑क्ष ए॒वाऽऽ*क्र॑मते॒ सदो॒ऽभ्यैति॑ सुव॒र्गमे॒वैनं॑ ॅलो॒कं ग॑मयति सौ॒रीभ्या॑मृ॒ग्भ्यां गार्.ह॑पत्ये जुहोत्य॒मुमे॒वैनं ॅलो॒कꣳ स॒मारो॑हयति॒ नय॑वत्य॒र्चाऽऽ*ग्नी᳚द्ध्रे जुहोति सुव॒र्गस्य॑ लो॒कस्या॒भिनी᳚त्यै॒ दिवं॑ गच्छ॒ सुवः॑ प॒तेति॒ हिर॑ण्यꣳ- [  ] \newline

\textbf{Pada Paata} \newline

सु॒व॒र्गायेति॑ सुवः - गाय॑ । वै । ए॒तानि॑ । लो॒काय॑ । हू॒य॒न्ते॒ । यत् । दा॒क्षि॒णानि॑ । द्वाभ्या᳚म् । गार्.ह॑पत्य॒ इति॒ गार्.ह॑ - प॒त्ये॒ । जु॒हो॒ति॒ । द्वि॒पादिति॑ द्वि - पात् । यज॑मानः । प्रति॑ष्ठित्या॒ इति॒ प्रति॑ - स्थि॒त्यै॒ । आग्नी᳚द्ध्र॒ इत्याग्नि॑ - इ॒द्ध्रे॒ । जु॒हो॒ति॒ । अ॒न्तरि॑क्षे । ए॒व । एति॑ । क्र॒म॒ते॒ । सदः॑ । अ॒भि । एति॑ । ए॒ति॒ । सु॒व॒र्गमिति॑ सुवः - गम् । ए॒व । ए॒न॒म् । लो॒कम् । ग॒म॒य॒ति॒ । सौ॒रीभ्या᳚म् । ऋ॒ग्भ्यामित्यृ॑क् - भ्याम् । गार्.ह॑पत्य॒ इति॒ गार्.ह॑-प॒त्ये॒ । जु॒हो॒ति॒ । अ॒मुम् । ए॒व । ए॒न॒म् । लो॒कम् । स॒मारो॑हय॒तीति॑ सं - आरो॑हयति । नय॑व॒त्येति॒ नय॑ - व॒त्या॒ । ऋ॒चा । आग्नी᳚द्ध्र॒ इत्याग्नि॑ - इ॒द्ध्रे॒ । जु॒हो॒ति॒ । सु॒व॒र्गस्येति॑ सुवः - गस्य॑ । लो॒कस्य॑ । अ॒भिनी᳚त्य॒ इत्य॒भि -नी॒त्यै॒ । दिव᳚म् । ग॒च्छ॒ । सुवः॑ । प॒त॒ । इति॑ । हिर॑ण्यम् ।  \newline


\textbf{Krama Paata} \newline

सु॒व॒र्गाय॒ वै । सु॒व॒र्गायेति॑ सुवः - गाय॑ । वा ए॒तानि॑ । ए॒तानि॑ लो॒काय॑ । लो॒काय॑ हूयन्ते । हू॒य॒न्ते॒ यत् । यद् दा᳚क्षि॒णानि॑ । दा॒क्षि॒णानि॒ द्वाभ्या᳚म् । द्वाभ्या॒म् गार्.ह॑पत्ये । गार्.ह॑पत्ये जुहोति । गार्.ह॑पत्य॒ इति॒ गार्.ह॑ - प॒त्ये॒ । जु॒हो॒ति॒ द्वि॒पात् । द्वि॒पाद् यज॑मानः । द्वि॒पादिति॑ द्वि - पात् । यज॑मानः॒ प्रति॑ष्ठित्यै । प्रति॑ष्ठित्या॒ आग्नी᳚द्ध्रे । प्रति॑ष्ठित्या॒ इति॒ प्रति॑ - स्थि॒त्यै॒ । आग्नी᳚द्ध्रे जुहोति । आग्नी᳚द्ध्र॒ इत्याग्नि॑ - इ॒द्ध्रे॒ । जु॒हो॒त्य॒न्तरि॑क्षे । अ॒न्तरि॑क्ष ए॒व । ए॒वा । आ क्र॑मते । क्र॒म॒ते॒ सदः॑ । सदो॒ऽभि । अ॒भ्या । ऐति॑ । ए॒ति॒ सु॒व॒र्गम् । सु॒व॒र्गमे॒व । सु॒व॒र्गमिति॑ सुवः - गम् । ए॒वैन᳚म् । ए॒न॒म् ॅलो॒कम् । लो॒कम् ग॑मयति । ग॒म॒य॒ति॒ सौ॒रीभ्या᳚म् । सौ॒रीभ्या॑मृ॒ग्भ्याम् । ऋ॒ग्भ्याम् गार्.ह॑पत्ये । ऋ॒ग्भ्यामित्यृ॑क् - भ्याम् । गार्.ह॑पत्ये जुहोति । गार्.ह॑पत्य॒ इति॒ गार्.ह॑ - प॒त्ये॒ । जु॒हो॒त्य॒मुम् । अ॒मुमे॒व । ए॒वैन᳚म् । ए॒न॒म् ॅलो॒कम् । लो॒कꣳ स॒मारो॑हयति । स॒मारो॑हयति॒ नय॑वत्या । स॒मारो॑हय॒तीति॑ सम् - आरो॑हयति । नय॑वत्य॒र्चा । नय॑व॒त्येति॒ नय॑ - व॒त्या॒ । ऋ॒चाऽऽग्नी᳚द्ध्रे । आग्नी᳚द्ध्रे जुहोति । आग्नी᳚द्ध्र॒ इत्याग्नि॑ - इ॒द्ध्रे॒ । जु॒हो॒ति॒ सु॒व॒र्गस्य॑ । सु॒व॒र्गस्य॑ लो॒कस्य॑ । सु॒व॒र्गस्येति॑ सुवः - गस्य॑ । लो॒कस्या॒भिनी᳚त्यै । अ॒भिनी᳚त्यै॒ दिव᳚म् । अ॒भिनी᳚त्या॒ इत्य॒भि - नी॒त्यै॒ । दिव॑म् गच्छ । ग॒च्छ॒ सुवः॑ । सुवः॑ पत । प॒तेति॑ । इति॒ हिर॑ण्यम् । हिर॑ण्यꣳ हु॒त्वा \newline

\textbf{Jatai Paata} \newline

1. सु॒व॒र्गाय॒ वै वै सु॑व॒र्गाय॑ सुव॒र्गाय॒ वै । \newline
2. सु॒व॒र्गायेति॑ सुवः - गाय॑ । \newline
3. वा ए॒ता न्ये॒तानि॒ वै वा ए॒तानि॑ । \newline
4. ए॒तानि॑ लो॒काय॑ लो॒का यै॒ता न्ये॒तानि॑ लो॒काय॑ । \newline
5. लो॒काय॑ हूयन्ते हूयन्ते लो॒काय॑ लो॒काय॑ हूयन्ते । \newline
6. हू॒य॒न्ते॒ यद् यद्धू॑यन्ते हूयन्ते॒ यत् । \newline
7. यद् दा᳚क्षि॒णानि॑ दाक्षि॒णानि॒ यद् यद् दा᳚क्षि॒णानि॑ । \newline
8. दा॒क्षि॒णानि॒ द्वाभ्या॒म् द्वाभ्या᳚म् दाक्षि॒णानि॑ दाक्षि॒णानि॒ द्वाभ्या᳚म् । \newline
9. द्वाभ्या॒म् गार्.ह॑पत्ये॒ गार्.ह॑पत्ये॒ द्वाभ्या॒म् द्वाभ्या॒म् गार्.ह॑पत्ये । \newline
10. गार्.ह॑पत्ये जुहोति जुहोति॒ गार्.ह॑पत्ये॒ गार्.ह॑पत्ये जुहोति । \newline
11. गार्.ह॑पत्य॒ इति॒ गार्.ह॑ - प॒त्ये॒ । \newline
12. जु॒हो॒ति॒ द्वि॒पाद् द्वि॒पाज् जु॑होति जुहोति द्वि॒पात् । \newline
13. द्वि॒पाद् यज॑मानो॒ यज॑मानो द्वि॒पाद् द्वि॒पाद् यज॑मानः । \newline
14. द्वि॒पादिति॑ द्वि - पात् । \newline
15. यज॑मानः॒ प्रति॑ष्ठित्यै॒ प्रति॑ष्ठित्यै॒ यज॑मानो॒ यज॑मानः॒ प्रति॑ष्ठित्यै । \newline
16. प्रति॑ष्ठित्या॒ आग्नी᳚द्ध्र॒ आग्नी᳚द्ध्रे॒ प्रति॑ष्ठित्यै॒ प्रति॑ष्ठित्या॒ आग्नी᳚द्ध्रे । \newline
17. प्रति॑ष्ठित्या॒ इति॒ प्रति॑ - स्थि॒त्यै॒ । \newline
18. आग्नी᳚द्ध्रे जुहोति जुहो॒ त्याग्नी᳚द्ध्र॒ आग्नी᳚द्ध्रे जुहोति । \newline
19. आग्नी᳚द्ध्र॒ इत्याग्नि॑ - इ॒द्ध्रे॒ । \newline
20. जु॒हो॒ त्य॒न्तरि॑क्षे॒ ऽन्तरि॑क्षे जुहोति जुहो त्य॒न्तरि॑क्षे । \newline
21. अ॒न्तरि॑क्ष ए॒वैवान्तरि॑क्षे॒ ऽन्तरि॑क्ष ए॒व । \newline
22. ए॒वै वैवा । \newline
23. आ क्र॑मते क्रमत॒ आ क्र॑मते । \newline
24. क्र॒म॒ते॒ सदः॒ सदः॑ क्रमते क्रमते॒ सदः॑ । \newline
25. सदो॒ ऽभ्य॑भि सदः॒ सदो॒ ऽभि । \newline
26. अ॒भ्या ऽभ्य॑भ्या । \newline
27. ऐत्ये॒ त्यैति॑ । \newline
28. ए॒ति॒ सु॒व॒र्गꣳ सु॑व॒र्ग मे᳚त्येति सुव॒र्गम् । \newline
29. सु॒व॒र्ग मे॒वैव सु॑व॒र्गꣳ सु॑व॒र्ग मे॒व । \newline
30. सु॒व॒र्गमिति॑ सुवः - गम् । \newline
31. ए॒वैन॑ मेन मे॒वै वैन᳚म् । \newline
32. ए॒न॒म् ॅलो॒कम् ॅलो॒क मे॑न मेनम् ॅलो॒कम् । \newline
33. लो॒कम् ग॑मयति गमयति लो॒कम् ॅलो॒कम् ग॑मयति । \newline
34. ग॒म॒य॒ति॒ सौ॒रीभ्याꣳ॑ सौ॒रीभ्या᳚म् गमयति गमयति सौ॒रीभ्या᳚म् । \newline
35. सौ॒रीभ्या॑ मृ॒ग्भ्या मृ॒ग्भ्याꣳ सौ॒रीभ्याꣳ॑ सौ॒रीभ्या॑ मृ॒ग्भ्याम् । \newline
36. ऋ॒ग्भ्याम् गार्.ह॑पत्ये॒ गार्.ह॑पत्य ऋ॒ग्भ्या मृ॒ग्भ्याम् गार्.ह॑पत्ये । \newline
37. ऋ॒ग्भ्यामित्यृ॑क् - भ्याम् । \newline
38. गार्.ह॑पत्ये जुहोति जुहोति॒ गार्.ह॑पत्ये॒ गार्.ह॑पत्ये जुहोति । \newline
39. गार्.ह॑पत्य॒ इति॒ गार्.ह॑ - प॒त्ये॒ । \newline
40. जु॒हो॒ त्य॒मु म॒मुम् जु॑होति जुहो त्य॒मुम् । \newline
41. अ॒मु मे॒वै वामु म॒मु मे॒व । \newline
42. ए॒वैन॑ मेन मे॒वै वैन᳚म् । \newline
43. ए॒न॒म् ॅलो॒कम् ॅलो॒क मे॑न मेनम् ॅलो॒कम् । \newline
44. लो॒कꣳ स॒मारो॑हयति स॒मारो॑हयति लो॒कम् ॅलो॒कꣳ स॒मारो॑हयति । \newline
45. स॒मारो॑हयति॒ नय॑वत्या॒ नय॑वत्या स॒मारो॑हयति स॒मारो॑हयति॒ नय॑वत्या । \newline
46. स॒मारो॑हय॒तीति॑ सं - आरो॑हयति । \newline
47. नय॑वत्य॒ र्‌च र्‌चा नय॑वत्या॒ नय॑वत्य॒ र्‌चा । \newline
48. नय॑व॒त्येति॒ नय॑ - व॒त्या॒ । \newline
49. ऋ॒चा ऽऽग्नी᳚द्ध्र॒ आग्नी᳚द्ध्र ऋ॒च र्‌चा ऽऽग्नी᳚द्ध्रे । \newline
50. आग्नी᳚द्ध्रे जुहोति जुहो॒ त्याग्नी᳚द्ध्र॒ आग्नी᳚द्ध्रे जुहोति । \newline
51. आग्नी᳚द्ध्र॒ इत्याग्नि॑ - इ॒द्ध्रे॒ । \newline
52. जु॒हो॒ति॒ सु॒व॒र्गस्य॑ सुव॒र्गस्य॑ जुहोति जुहोति सुव॒र्गस्य॑ । \newline
53. सु॒व॒र्गस्य॑ लो॒कस्य॑ लो॒कस्य॑ सुव॒र्गस्य॑ सुव॒र्गस्य॑ लो॒कस्य॑ । \newline
54. सु॒व॒र्गस्येति॑ सुवः - गस्य॑ । \newline
55. लो॒कस्या॒ भिनी᳚त्या अ॒भिनी᳚त्यै लो॒कस्य॑ लो॒कस्या॒ भिनी᳚त्यै । \newline
56. अ॒भिनी᳚त्यै॒ दिव॒म् दिव॑ म॒भिनी᳚त्या अ॒भिनी᳚त्यै॒ दिव᳚म् । \newline
57. अ॒भिनी᳚त्या॒ इत्य॒भि - नी॒त्यै॒ । \newline
58. दिव॑म् गच्छ गच्छ॒ दिव॒म् दिव॑म् गच्छ । \newline
59. ग॒च्छ॒ सुवः॒ सुव॑र् गच्छ गच्छ॒ सुवः॑ । \newline
60. सुवः॑ पत पत॒ सुवः॒ सुवः॑ पत । \newline
61. प॒तेतीति॑ पत प॒तेति॑ । \newline
62. इति॒ हिर॑ण्यꣳ॒॒ हिर॑ण्य॒ मितीति॒ हिर॑ण्यम् । \newline
63. हिर॑ण्यꣳ हु॒त्वा हु॒त्वा हिर॑ण्यꣳ॒॒ हिर॑ण्यꣳ हु॒त्वा । \newline

\textbf{Ghana Paata } \newline

1. सु॒व॒र्गाय॒ वै वै सु॑व॒र्गाय॑ सुव॒र्गाय॒ वा ए॒ता न्ये॒तानि॒ वै सु॑व॒र्गाय॑ सुव॒र्गाय॒ वा ए॒तानि॑ । \newline
2. सु॒व॒र्गायेति॑ सुवः - गाय॑ । \newline
3. वा ए॒ता न्ये॒तानि॒ वै वा ए॒तानि॑ लो॒काय॑ लो॒का यै॒तानि॒ वै वा ए॒तानि॑ लो॒काय॑ । \newline
4. ए॒तानि॑ लो॒काय॑ लो॒का यै॒ता न्ये॒तानि॑ लो॒काय॑ हूयन्ते हूयन्ते लो॒का यै॒ता न्ये॒तानि॑ लो॒काय॑ हूयन्ते । \newline
5. लो॒काय॑ हूयन्ते हूयन्ते लो॒काय॑ लो॒काय॑ हूयन्ते॒ यद् यद्धू॑यन्ते लो॒काय॑ लो॒काय॑ हूयन्ते॒ यत् । \newline
6. हू॒य॒न्ते॒ यद् यद्धू॑यन्ते हूयन्ते॒ यद् दा᳚क्षि॒णानि॑ दाक्षि॒णानि॒ यद्धू॑यन्ते हूयन्ते॒ यद् दा᳚क्षि॒णानि॑ । \newline
7. यद् दा᳚क्षि॒णानि॑ दाक्षि॒णानि॒ यद् यद् दा᳚क्षि॒णानि॒ द्वाभ्या॒म् द्वाभ्या᳚म् दाक्षि॒णानि॒ यद् यद् दा᳚क्षि॒णानि॒ द्वाभ्या᳚म् । \newline
8. दा॒क्षि॒णानि॒ द्वाभ्या॒म् द्वाभ्या᳚म् दाक्षि॒णानि॑ दाक्षि॒णानि॒ द्वाभ्या॒म् गार्.ह॑पत्ये॒ गार्.ह॑पत्ये॒ द्वाभ्या᳚म् दाक्षि॒णानि॑ दाक्षि॒णानि॒ द्वाभ्या॒म् गार्.ह॑पत्ये । \newline
9. द्वाभ्या॒म् गार्.ह॑पत्ये॒ गार्.ह॑पत्ये॒ द्वाभ्या॒म् द्वाभ्या॒म् गार्.ह॑पत्ये जुहोति जुहोति॒ गार्.ह॑पत्ये॒ द्वाभ्या॒म् द्वाभ्या॒म् गार्.ह॑पत्ये जुहोति । \newline
10. गार्.ह॑पत्ये जुहोति जुहोति॒ गार्.ह॑पत्ये॒ गार्.ह॑पत्ये जुहोति द्वि॒पाद् द्वि॒पाज् जु॑होति॒ गार्.ह॑पत्ये॒ गार्.ह॑पत्ये जुहोति द्वि॒पात् । \newline
11. गार्.ह॑पत्य॒ इति॒ गार्.ह॑ - प॒त्ये॒ । \newline
12. जु॒हो॒ति॒ द्वि॒पाद् द्वि॒पाज् जु॑होति जुहोति द्वि॒पाद् यज॑मानो॒ यज॑मानो द्वि॒पाज् जु॑होति जुहोति द्वि॒पाद् यज॑मानः । \newline
13. द्वि॒पाद् यज॑मानो॒ यज॑मानो द्वि॒पाद् द्वि॒पाद् यज॑मानः॒ प्रति॑ष्ठित्यै॒ प्रति॑ष्ठित्यै॒ यज॑मानो द्वि॒पाद् द्वि॒पाद् यज॑मानः॒ प्रति॑ष्ठित्यै । \newline
14. द्वि॒पादिति॑ द्वि - पात् । \newline
15. यज॑मानः॒ प्रति॑ष्ठित्यै॒ प्रति॑ष्ठित्यै॒ यज॑मानो॒ यज॑मानः॒ प्रति॑ष्ठित्या॒ आग्नी᳚द्ध्र॒ आग्नी᳚द्ध्रे॒ प्रति॑ष्ठित्यै॒ यज॑मानो॒ यज॑मानः॒ प्रति॑ष्ठित्या॒ आग्नी᳚द्ध्रे । \newline
16. प्रति॑ष्ठित्या॒ आग्नी᳚द्ध्र॒ आग्नी᳚द्ध्रे॒ प्रति॑ष्ठित्यै॒ प्रति॑ष्ठित्या॒ आग्नी᳚द्ध्रे जुहोति जुहो॒ त्याग्नी᳚द्ध्रे॒ प्रति॑ष्ठित्यै॒ प्रति॑ष्ठित्या॒ आग्नी᳚द्ध्रे जुहोति । \newline
17. प्रति॑ष्ठित्या॒ इति॒ प्रति॑ - स्थि॒त्यै॒ । \newline
18. आग्नी᳚द्ध्रे जुहोति जुहो॒ त्याग्नी᳚द्ध्र॒ आग्नी᳚द्ध्रे जुहो त्य॒न्तरि॑क्षे॒ ऽन्तरि॑क्षे जुहो॒ त्याग्नी᳚द्ध्र॒ आग्नी᳚द्ध्रे जुहो
त्य॒न्तरि॑क्षे । \newline
19. आग्नी᳚द्ध्र॒ इत्याग्नि॑ - इ॒द्ध्रे॒ । \newline
20. जु॒हो॒ त्य॒न्तरि॑क्षे॒ ऽन्तरि॑क्षे जुहोति जुहो त्य॒न्तरि॑क्ष ए॒वैवा न्तरि॑क्षे जुहोति जुहो त्य॒न्तरि॑क्ष ए॒व । \newline
21. अ॒न्तरि॑क्ष ए॒वैवा न्तरि॑क्षे॒ ऽन्तरि॑क्ष ए॒वैवा न्तरि॑क्षे॒ ऽन्तरि॑क्ष ए॒वा । \newline
22. ए॒वै वैवा क्र॑मते क्रमत॒ ऐवैवा क्र॑मते । \newline
23. आ क्र॑मते क्रमत॒ आ क्र॑मते॒ सदः॒ सदः॑ क्रमत॒ आ क्र॑मते॒ सदः॑ । \newline
24. क्र॒म॒ते॒ सदः॒ सदः॑ क्रमते क्रमते॒ सदो॒ ऽभ्य॑भि सदः॑ क्रमते क्रमते॒ सदो॒ ऽभि । \newline
25. सदो॒ ऽभ्य॑भि सदः॒ सदो॒ ऽभ्या ऽभि सदः॒ सदो॒ ऽभ्या । \newline
26. अ॒भ्या ऽभ्य॑भ्यै त्ये॒त्या ऽभ्य॑भ्यैति॑ । \newline
27. ऐत्ये॒ त्यैति॑ सुव॒र्गꣳ सु॑व॒र्ग मे॒त्यैति॑ सुव॒र्गम् । \newline
28. ए॒ति॒ सु॒व॒र्गꣳ सु॑व॒र्ग मे᳚त्येति सुव॒र्ग मे॒वैव सु॑व॒र्ग मे᳚त्येति सुव॒र्ग मे॒व । \newline
29. सु॒व॒र्ग मे॒वैव सु॑व॒र्गꣳ सु॑व॒र्ग मे॒वैन॑ मेन मे॒व सु॑व॒र्गꣳ सु॑व॒र्ग मे॒वैन᳚म् । \newline
30. सु॒व॒र्गमिति॑ सुवः - गम् । \newline
31. ए॒वैन॑ मेन मे॒वै वैन॑म् ॅलो॒कम् ॅलो॒क मे॑न मे॒वै वैन॑म् ॅलो॒कम् । \newline
32. ए॒न॒म् ॅलो॒कम् ॅलो॒क मे॑न मेनम् ॅलो॒कम् ग॑मयति गमयति लो॒क मे॑न मेनम् ॅलो॒कम् ग॑मयति । \newline
33. लो॒कम् ग॑मयति गमयति लो॒कम् ॅलो॒कम् ग॑मयति सौ॒रीभ्याꣳ॑ सौ॒रीभ्या᳚म् गमयति लो॒कम् ॅलो॒कम् ग॑मयति सौ॒रीभ्या᳚म् । \newline
34. ग॒म॒य॒ति॒ सौ॒रीभ्याꣳ॑ सौ॒रीभ्या᳚म् गमयति गमयति सौ॒रीभ्या॑ मृ॒ग्भ्या मृ॒ग्भ्याꣳ सौ॒रीभ्या᳚म् गमयति गमयति सौ॒रीभ्या॑ मृ॒ग्भ्याम् । \newline
35. सौ॒रीभ्या॑ मृ॒ग्भ्या मृ॒ग्भ्याꣳ सौ॒रीभ्याꣳ॑ सौ॒रीभ्या॑ मृ॒ग्भ्याम् गार्.ह॑पत्ये॒ गार्.ह॑पत्य ऋ॒ग्भ्याꣳ सौ॒रीभ्याꣳ॑ सौ॒रीभ्या॑ मृ॒ग्भ्याम् गार्.ह॑पत्ये । \newline
36. ऋ॒ग्भ्याम् गार्.ह॑पत्ये॒ गार्.ह॑पत्य ऋ॒ग्भ्या मृ॒ग्भ्याम् गार्.ह॑पत्ये जुहोति जुहोति॒ गार्.ह॑पत्य ऋ॒ग्भ्या मृ॒ग्भ्याम् गार्.ह॑पत्ये जुहोति । \newline
37. ऋ॒ग्भ्यामित्यृ॑क् - भ्याम् । \newline
38. गार्.ह॑पत्ये जुहोति जुहोति॒ गार्.ह॑पत्ये॒ गार्.ह॑पत्ये जुहो त्य॒मु म॒मुम् जु॑होति॒ गार्.ह॑पत्ये॒ गार्.ह॑पत्ये जुहो त्य॒मुम् । \newline
39. गार्.ह॑पत्य॒ इति॒ गार्.ह॑ - प॒त्ये॒ । \newline
40. जु॒हो॒ त्य॒मु म॒मुम् जु॑होति जुहो त्य॒मु मे॒वै वामुम् जु॑होति जुहो त्य॒मु मे॒व । \newline
41. अ॒मु मे॒वै वामु म॒मु मे॒वैन॑ मेन मे॒वामु म॒मु मे॒वैन᳚म् । \newline
42. ए॒वैन॑ मेन मे॒वै वैन॑म् ॅलो॒कम् ॅलो॒क मे॑न मे॒वै वैन॑म् ॅलो॒कम् । \newline
43. ए॒न॒म् ॅलो॒कम् ॅलो॒क मे॑न मेनम् ॅलो॒कꣳ स॒मारो॑हयति स॒मारो॑हयति लो॒क मे॑न मेनम् ॅलो॒कꣳ स॒मारो॑हयति । \newline
44. लो॒कꣳ स॒मारो॑हयति स॒मारो॑हयति लो॒कम् ॅलो॒कꣳ स॒मारो॑हयति॒ नय॑वत्या॒ नय॑वत्या स॒मारो॑हयति लो॒कम् ॅलो॒कꣳ स॒मारो॑हयति॒ नय॑वत्या । \newline
45. स॒मारो॑हयति॒ नय॑वत्या॒ नय॑वत्या स॒मारो॑हयति स॒मारो॑हयति॒ नय॑वत्य॒ र्‌च र्‌चा नय॑वत्या स॒मारो॑हयति स॒मारो॑हयति॒ नय॑वत्य॒ र्‌चा । \newline
46. स॒मारो॑हय॒तीति॑ सं - आरो॑हयति । \newline
47. नय॑वत्य॒ र्‌च र्‌चा नय॑वत्या॒ नय॑वत्य॒ र्‌चा ऽऽग्नी᳚द्ध्र॒ आग्नी᳚द्ध्र ऋ॒चा नय॑वत्या॒ नय॑वत्य॒ र्‌चा ऽऽग्नी᳚द्ध्रे । \newline
48. नय॑व॒त्येति॒ नय॑ - व॒त्या॒ । \newline
49. ऋ॒चा ऽऽग्नी᳚द्ध्र॒ आग्नी᳚द्ध्र ऋ॒च र्‌चा ऽऽग्नी᳚द्ध्रे जुहोति जुहो॒ त्याग्नी᳚द्ध्र ऋ॒च र्‌चा ऽऽग्नी᳚द्ध्रे जुहोति । \newline
50. आग्नी᳚द्ध्रे जुहोति जुहो॒ त्याग्नी᳚द्ध्र॒ आग्नी᳚द्ध्रे जुहोति सुव॒र्गस्य॑ सुव॒र्गस्य॑ जुहो॒ त्याग्नी᳚द्ध्र॒ आग्नी᳚द्ध्रे जुहोति सुव॒र्गस्य॑ । \newline
51. आग्नी᳚द्ध्र॒ इत्याग्नि॑ - इ॒द्ध्रे॒ । \newline
52. जु॒हो॒ति॒ सु॒व॒र्गस्य॑ सुव॒र्गस्य॑ जुहोति जुहोति सुव॒र्गस्य॑ लो॒कस्य॑ लो॒कस्य॑ सुव॒र्गस्य॑ जुहोति जुहोति सुव॒र्गस्य॑ लो॒कस्य॑ । \newline
53. सु॒व॒र्गस्य॑ लो॒कस्य॑ लो॒कस्य॑ सुव॒र्गस्य॑ सुव॒र्गस्य॑ लो॒कस्या॒ भिनी᳚त्या अ॒भिनी᳚त्यै लो॒कस्य॑ सुव॒र्गस्य॑ सुव॒र्गस्य॑ लो॒कस्या॒ भिनी᳚त्यै । \newline
54. सु॒व॒र्गस्येति॑ सुवः - गस्य॑ । \newline
55. लो॒कस्या॒ भिनी᳚त्या अ॒भिनी᳚त्यै लो॒कस्य॑ लो॒कस्या॒ भिनी᳚त्यै॒ दिव॒म् दिव॑ म॒भिनी᳚त्यै लो॒कस्य॑ लो॒कस्या॒ भिनी᳚त्यै॒ दिव᳚म् । \newline
56. अ॒भिनी᳚त्यै॒ दिव॒म् दिव॑ म॒भिनी᳚त्या अ॒भिनी᳚त्यै॒ दिव॑म् गच्छ गच्छ॒ दिव॑ म॒भिनी᳚त्या अ॒भिनी᳚त्यै॒ दिव॑म् गच्छ । \newline
57. अ॒भिनी᳚त्या॒ इत्य॒भि - नी॒त्यै॒ । \newline
58. दिव॑म् गच्छ गच्छ॒ दिव॒म् दिव॑म् गच्छ॒ सुवः॒ सुव॑र् गच्छ॒ दिव॒म् दिव॑म् गच्छ॒ सुवः॑ । \newline
59. ग॒च्छ॒ सुवः॒ सुव॑र् गच्छ गच्छ॒ सुवः॑ पत पत॒ सुव॑र् गच्छ गच्छ॒ सुवः॑ पत । \newline
60. सुवः॑ पत पत॒ सुवः॒ सुवः॑ प॒तेतीति॑ पत॒ सुवः॒ सुवः॑ प॒तेति॑ । \newline
61. प॒तेतीति॑ पत प॒तेति॒ हिर॑ण्यꣳ॒॒ हिर॑ण्य॒ मिति॑ पत प॒तेति॒ हिर॑ण्यम् । \newline
62. इति॒ हिर॑ण्यꣳ॒॒ हिर॑ण्य॒ मितीति॒ हिर॑ण्यꣳ हु॒त्वा हु॒त्वा हिर॑ण्य॒ मितीति॒ हिर॑ण्यꣳ हु॒त्वा । \newline
63. हिर॑ण्यꣳ हु॒त्वा हु॒त्वा हिर॑ण्यꣳ॒॒ हिर॑ण्यꣳ हु॒त्वोदु द्धु॒त्वा हिर॑ण्यꣳ॒॒ हिर॑ण्यꣳ हु॒त्वोत् । \newline
\pagebreak
\markright{ TS 6.6.1.2  \hfill https://www.vedavms.in \hfill}

\section{ TS 6.6.1.2 }

\textbf{TS 6.6.1.2 } \newline
\textbf{Samhita Paata} \newline

हु॒त्वोद्-गृ॑ह्णाति सुव॒र्गमे॒वैनं॑ ॅलो॒कं ग॑मयति रू॒पेण॑ वो रू॒पम॒भ्यैमीत्या॑ह रू॒पेण॒ ह्या॑साꣳ रू॒पम॒भ्यैति॒ यद्धिर॑ण्येन तु॒थो वो॑ वि॒श्ववे॑दा॒ वि भ॑ज॒त्वित्या॑ह तु॒थो ह॑ स्म॒ वै वि॒श्ववे॑दा दे॒वानां॒ दक्षि॑णा॒ वि भ॑जति॒ तेनै॒वैना॒ वि भ॑जत्ये॒ तत् ते॑ अग्ने॒ राध॒- [  ] \newline

\textbf{Pada Paata} \newline

हु॒त्वा । उदिति॑ । गृ॒ह्णा॒ति॒ । सु॒व॒र्गमिति॑ सुवः - गम् । ए॒व । ए॒न॒म् । लो॒कम् । ग॒म॒य॒ति॒ । रू॒पेण॑ । वः॒ । रू॒पम् । अ॒भि । एति॑ । ए॒मि॒ । इति॑ । आ॒ह॒ । रू॒पेण॑ । हि । आ॒सा॒म् । रू॒पम् । अ॒भि । एति॑ । एति॑ । यत् । हिर॑ण्येन । तु॒थः । वः॒ । वि॒श्ववे॑दा॒ इति॑ वि॒श्व - वे॒दाः॒ । वीति॑ । भ॒ज॒तु॒ । इति॑ । आ॒ह॒ । तु॒थः । ह॒ । स्म॒ । वै । वि॒श्ववे॑दा॒ इति॑ वि॒श्व - वे॒दाः॒ । दे॒वाना᳚म् । दक्षि॑णाः । वीति॑ । भ॒ज॒ति॒ । तेन॑ । ए॒व । ए॒नाः॒ । वीति॑ । भ॒ज॒ति॒ । ए॒तत् । ते॒ । अ॒ग्ने॒ । राधः॑ ।  \newline


\textbf{Krama Paata} \newline

हु॒त्वोत् । उद् गृ॑ह्णाति । गृ॒ह्णा॒ति॒ सु॒व॒र्गम् । सु॒व॒र्गमे॒व । सु॒व॒र्गमिति॑ सुवः - गम् । ए॒वैन᳚म् । ए॒न॒म् ॅलो॒कम् । लो॒कम् ग॑मयति । ग॒म॒य॒ति॒ रू॒पेण॑ । रू॒पेण॑ वः । वो॒ रू॒पम् । रू॒पम॒भि । अ॒भ्या । ऐमि॑ । ए॒मीति॑ । इत्या॑ह । आ॒ह॒ रू॒पेण॑ । रू॒पेण॒ हि । ह्या॑साम् । आ॒साꣳ॒᳡ रू॒पम् । रू॒पम॒भि । अ॒भ्या । ऐति॑ । एति॒ यत् । यद्‌धिर॑ण्येन । हिर॑ण्येन तु॒थः । तु॒थो वः॑ । वो॒ वि॒श्ववे॑दाः । वि॒श्ववे॑दा॒ वि । वि॒श्ववे॑दा॒ इति॑ वि॒श्व - वे॒दाः॒ । वि भ॑जतु । भ॒ज॒त्विति॑ । इत्या॑ह । आ॒ह॒ तु॒थः । तु॒थो ह॑ । ह॒ स्म॒ । स्म॒ वै । वै वि॒श्ववे॑दाः । वि॒श्ववे॑दा दे॒वाना᳚म् । वि॒श्ववे॑दा॒ इति॑ वि॒श्व - वे॒दाः॒ । दे॒वाना॒म् दक्षि॑णाः । दक्षि॑णा॒ वि । वि भ॑जति । भ॒ज॒ति॒ तेन॑ । तेनै॒व । ए॒वैनाः᳚ । ए॒ना॒ वि । वि भ॑जति । भ॒ज॒त्ये॒तत् । ए॒तत् ते᳚ । ते॒ अ॒ग्ने॒ । अ॒ग्ने॒ राधः॑ । राध॒ आ \newline

\textbf{Jatai Paata} \newline

1. हु॒त्वो दुद्धु॒त्वा हु॒त्वोत् । \newline
2. उद् गृ॑ह्णाति गृह्णा॒ त्युदुद् गृ॑ह्णाति । \newline
3. गृ॒ह्णा॒ति॒ सु॒व॒र्गꣳ सु॑व॒र्गम् गृ॑ह्णाति गृह्णाति सुव॒र्गम् । \newline
4. सु॒व॒र्ग मे॒वैव सु॑व॒र्गꣳ सु॑व॒र्ग मे॒व । \newline
5. सु॒व॒र्गमिति॑ सुवः - गम् । \newline
6. ए॒वैन॑ मेन मे॒वै वैन᳚म् । \newline
7. ए॒न॒म् ॅलो॒कम् ॅलो॒क मे॑न मेनम् ॅलो॒कम् । \newline
8. लो॒कम् ग॑मयति गमयति लो॒कम् ॅलो॒कम् ग॑मयति । \newline
9. ग॒म॒य॒ति॒ रू॒पेण॑ रू॒पेण॑ गमयति गमयति रू॒पेण॑ । \newline
10. रू॒पेण॑ वो वो रू॒पेण॑ रू॒पेण॑ वः । \newline
11. वो॒ रू॒पꣳ रू॒पं ॅवो॑ वो रू॒पम् । \newline
12. रू॒प म॒भ्य॑भि रू॒पꣳ रू॒प म॒भि । \newline
13. अ॒भ्या ऽभ्य॑भ्या । \newline
14. ऐम्ये॒ म्यैमि॑ । \newline
15. ए॒मीती त्ये᳚म्ये॒ मीति॑ । \newline
16. इत्या॑हा॒हे तीत्या॑ह । \newline
17. आ॒ह॒ रू॒पेण॑ रू॒पेणा॑ हाह रू॒पेण॑ । \newline
18. रू॒पेण॒ हि हि रू॒पेण॑ रू॒पेण॒ हि । \newline
19. ह्या॑सा मासाꣳ॒॒ हि ह्या॑साम् । \newline
20. आ॒साꣳ॒॒ रू॒पꣳ रू॒प मा॑सा मासाꣳ रू॒पम् । \newline
21. रू॒प म॒भ्य॑भि रू॒पꣳ रू॒प म॒भि । \newline
22. अ॒भ्या ऽभ्य॑भ्या । \newline
23. ऐत्ये त्यैति॑ । \newline
24. एति॒ यद् यदे त्येति॒ यत् । \newline
25. यद्धिर॑ण्येन॒ हिर॑ण्येन॒ यद् यद्धिर॑ण्येन । \newline
26. हिर॑ण्येन तु॒थ स्तु॒थो हिर॑ण्येन॒ हिर॑ण्येन तु॒थः । \newline
27. तु॒थो वो॑ व  स्तु॒थ स्तु॒थो वः॑ । \newline
28. वो॒ वि॒श्ववे॑दा वि॒श्ववे॑दा वो वो वि॒श्ववे॑दाः । \newline
29. वि॒श्ववे॑दा॒ वि वि वि॒श्ववे॑दा वि॒श्ववे॑दा॒ वि । \newline
30. वि॒श्ववे॑दा॒ इति॑ वि॒श्व - वे॒दाः॒ । \newline
31. वि भ॑जतु भजतु॒ वि वि भ॑जतु । \newline
32. भ॒ज॒त्वितीति॑ भजतु भज॒त्विति॑ । \newline
33. इत्या॑हा॒हे तीत्या॑ह । \newline
34. आ॒ह॒ तु॒थ स्तु॒थ आ॑हाह तु॒थः । \newline
35. तु॒थो ह॑ ह तु॒थ स्तु॒थो ह॑ । \newline
36. ह॒ स्म॒ स्म॒ ह॒ ह॒ स्म॒ । \newline
37. स्म॒ वै वै स्म॑ स्म॒ वै । \newline
38. वै वि॒श्ववे॑दा वि॒श्ववे॑दा॒ वै वै वि॒श्ववे॑दाः । \newline
39. वि॒श्ववे॑दा दे॒वाना᳚म् दे॒वानां᳚ ॅवि॒श्ववे॑दा वि॒श्ववे॑दा दे॒वाना᳚म् । \newline
40. वि॒श्ववे॑दा॒ इति॑ वि॒श्व - वे॒दाः॒ । \newline
41. दे॒वाना॒म् दक्षि॑णा॒ दक्षि॑णा दे॒वाना᳚म् दे॒वाना॒म् दक्षि॑णाः । \newline
42. दक्षि॑णा॒ वि वि दक्षि॑णा॒ दक्षि॑णा॒ वि । \newline
43. वि भ॑जति भजति॒ वि वि भ॑जति । \newline
44. भ॒ज॒ति॒ तेन॒ तेन॑ भजति भजति॒ तेन॑ । \newline
45. तेनै॒ वैव तेन॒ तेनै॒व । \newline
46. ए॒वैना॑ एना ए॒वै वैनाः᳚ । \newline
47. ए॒ना॒ वि व्ये॑ना एना॒ वि । \newline
48. वि भ॑जति भजति॒ वि वि भ॑जति । \newline
49. भ॒ज॒ त्ये॒त दे॒तद् भ॑जति भज त्ये॒तत् । \newline
50. ए॒तत् ते॑ त ए॒त दे॒तत् ते᳚ । \newline
51. ते॒ अ॒ग्ने॒ ऽग्ने॒ ते॒ ते॒ अ॒ग्ने॒ । \newline
52. अ॒ग्ने॒ राधो॒ राधो᳚ ऽग्ने ऽग्ने॒ राधः॑ । \newline
53. राध॒ आ राधो॒ राध॒ आ । \newline

\textbf{Ghana Paata } \newline

1. हु॒त्वोदु द्धु॒त्वा हु॒त्वोद् गृ॑ह्णाति गृह्णा॒ त्युद्धु॒त्वा हु॒त्वोद् गृ॑ह्णाति । \newline
2. उद् गृ॑ह्णाति गृह्णा॒ त्युदुद् गृ॑ह्णाति सुव॒र्गꣳ सु॑व॒र्गम् गृ॑ह्णा॒ त्युदुद् गृ॑ह्णाति सुव॒र्गम् । \newline
3. गृ॒ह्णा॒ति॒ सु॒व॒र्गꣳ सु॑व॒र्गम् गृ॑ह्णाति गृह्णाति सुव॒र्ग मे॒वैव सु॑व॒र्गम् गृ॑ह्णाति गृह्णाति सुव॒र्ग मे॒व । \newline
4. सु॒व॒र्ग मे॒वैव सु॑व॒र्गꣳ सु॑व॒र्ग मे॒वैन॑ मेन मे॒व सु॑व॒र्गꣳ सु॑व॒र्ग मे॒वैन᳚म् । \newline
5. सु॒व॒र्गमिति॑ सुवः - गम् । \newline
6. ए॒वैन॑ मेन मे॒वै वैन॑म् ॅलो॒कम् ॅलो॒क मे॑न मे॒वै वैन॑म् ॅलो॒कम् । \newline
7. ए॒न॒म् ॅलो॒कम् ॅलो॒क मे॑न मेनम् ॅलो॒कम् ग॑मयति गमयति लो॒क मे॑न मेनम् ॅलो॒कम् ग॑मयति । \newline
8. लो॒कम् ग॑मयति गमयति लो॒कम् ॅलो॒कम् ग॑मयति रू॒पेण॑ रू॒पेण॑ गमयति लो॒कम् ॅलो॒कम् ग॑मयति रू॒पेण॑ । \newline
9. ग॒म॒य॒ति॒ रू॒पेण॑ रू॒पेण॑ गमयति गमयति रू॒पेण॑ वो वो रू॒पेण॑ गमयति गमयति रू॒पेण॑ वः । \newline
10. रू॒पेण॑ वो वो रू॒पेण॑ रू॒पेण॑ वो रू॒पꣳ रू॒पं ॅवो॑ रू॒पेण॑ रू॒पेण॑ वो रू॒पम् । \newline
11. वो॒ रू॒पꣳ रू॒पं ॅवो॑ वो रू॒प म॒भ्य॑भि रू॒पं ॅवो॑ वो रू॒प म॒भि । \newline
12. रू॒प म॒भ्य॑भि रू॒पꣳ रू॒प म॒भ्या ऽभि रू॒पꣳ रू॒प म॒भ्या । \newline
13. अ॒भ्या ऽभ्य॑भ्यै म्ये॒म्या ऽभ्य॑भ्यैमि॑ । \newline
14. ऐम्ये॒ म्यैमीती त्ये॒म्यैमीति॑ । \newline
15. ए॒मीती त्ये᳚म्ये॒ मीत्या॑हा॒हे त्ये᳚म्ये॒मीत्या॑ह । \newline
16. इत्या॑हा॒हे तीत्या॑ह रू॒पेण॑ रू॒पेणा॒हे तीत्या॑ह रू॒पेण॑ । \newline
17. आ॒ह॒ रू॒पेण॑ रू॒पेणा॑ हाह रू॒पेण॒ हि हि रू॒पेणा॑ हाह रू॒पेण॒ हि । \newline
18. रू॒पेण॒ हि हि रू॒पेण॑ रू॒पेण॒ ह्या॑सा मासाꣳ॒॒ हि रू॒पेण॑ रू॒पेण॒ ह्या॑साम् । \newline
19. ह्या॑सा मासाꣳ॒॒ हि ह्या॑साꣳ रू॒पꣳ रू॒प मा॑साꣳ॒॒ हि ह्या॑साꣳ रू॒पम् । \newline
20. आ॒साꣳ॒॒ रू॒पꣳ रू॒प मा॑सा मासाꣳ रू॒प म॒भ्य॑भि रू॒प मा॑सा मासाꣳ रू॒प म॒भि । \newline
21. रू॒प म॒भ्य॑भि रू॒पꣳ रू॒प म॒भ्या ऽभि रू॒पꣳ रू॒प म॒भ्या । \newline
22. अ॒भ्या ऽभ्य॑भ्यै त्येत्या ऽभ्य॑भ्यैति॑ । \newline
23. ऐत्ये त्यैति॒ यद् यदे त्यैति॒ यत् । \newline
24. एति॒ यद् यदे त्येति॒ यद्धिर॑ण्येन॒ हिर॑ण्येन॒ यदे त्येति॒ यद्धिर॑ण्येन । \newline
25. यद्धिर॑ण्येन॒ हिर॑ण्येन॒ यद् यद्धिर॑ण्येन तु॒थ स्तु॒थो हिर॑ण्येन॒ यद् यद्धिर॑ण्येन तु॒थः । \newline
26. हिर॑ण्येन तु॒थ स्तु॒थो हिर॑ण्येन॒ हिर॑ण्येन तु॒थो वो॑ व स्तु॒थो हिर॑ण्येन॒ हिर॑ण्येन तु॒थो वः॑ । \newline
27. तु॒थो वो॑ वस्तु॒थ स्तु॒थो वो॑ वि॒श्ववे॑दा वि॒श्ववे॑दा वस्तु॒थ स्तु॒थो वो॑ वि॒श्ववे॑दाः । \newline
28. वो॒ वि॒श्ववे॑दा वि॒श्ववे॑दा वो वो वि॒श्ववे॑दा॒ वि वि वि॒श्ववे॑दा वो वो वि॒श्ववे॑दा॒ वि । \newline
29. वि॒श्ववे॑दा॒ वि वि वि॒श्ववे॑दा वि॒श्ववे॑दा॒ वि भ॑जतु भजतु॒ वि वि॒श्ववे॑दा वि॒श्ववे॑दा॒ वि भ॑जतु । \newline
30. वि॒श्ववे॑दा॒ इति॑ वि॒श्व - वे॒दाः॒ । \newline
31. वि भ॑जतु भजतु॒ वि वि भ॑ज॒ त्वितीति॑ भजतु॒ वि वि भ॑ज॒ त्विति॑ । \newline
32. भ॒ज॒ त्वितीति॑ भजतु भज॒ त्वित्या॑ हा॒हेति॑ भजतु भज॒ त्वित्या॑ह । \newline
33. इत्या॑हा॒हे तीत्या॑ह तु॒थ स्तु॒थ आ॒हे तीत्या॑ह तु॒थः । \newline
34. आ॒ह॒ तु॒थ स्तु॒थ आ॑हाह तु॒थो ह॑ ह तु॒थ आ॑हाह तु॒थो ह॑ । \newline
35. तु॒थो ह॑ ह तु॒थ स्तु॒थो ह॑ स्म स्म ह तु॒थ स्तु॒थो ह॑ स्म । \newline
36. ह॒ स्म॒ स्म॒ ह॒ ह॒ स्म॒ वै वै स्म॑ ह ह स्म॒ वै । \newline
37. स्म॒ वै वै स्म॑ स्म॒ वै वि॒श्ववे॑दा वि॒श्ववे॑दा॒ वै स्म॑ स्म॒ वै वि॒श्ववे॑दाः । \newline
38. वै वि॒श्ववे॑दा वि॒श्ववे॑दा॒ वै वै वि॒श्ववे॑दा दे॒वाना᳚म् दे॒वानां᳚ ॅवि॒श्ववे॑दा॒ वै वै वि॒श्ववे॑दा दे॒वाना᳚म् । \newline
39. वि॒श्ववे॑दा दे॒वाना᳚म् दे॒वानां᳚ ॅवि॒श्ववे॑दा वि॒श्ववे॑दा दे॒वाना॒म् दक्षि॑णा॒ दक्षि॑णा दे॒वानां᳚ ॅवि॒श्ववे॑दा वि॒श्ववे॑दा दे॒वाना॒म् दक्षि॑णाः । \newline
40. वि॒श्ववे॑दा॒ इति॑ वि॒श्व - वे॒दाः॒ । \newline
41. दे॒वाना॒म् दक्षि॑णा॒ दक्षि॑णा दे॒वाना᳚म् दे॒वाना॒म् दक्षि॑णा॒ वि वि दक्षि॑णा दे॒वाना᳚म् दे॒वाना॒म् दक्षि॑णा॒ वि । \newline
42. दक्षि॑णा॒ वि वि दक्षि॑णा॒ दक्षि॑णा॒ वि भ॑जति भजति॒ वि दक्षि॑णा॒ दक्षि॑णा॒ वि भ॑जति । \newline
43. वि भ॑जति भजति॒ वि वि भ॑जति॒ तेन॒ तेन॑ भजति॒ वि वि भ॑जति॒ तेन॑ । \newline
44. भ॒ज॒ति॒ तेन॒ तेन॑ भजति भजति॒ तेनै॒वैव तेन॑ भजति भजति॒ तेनै॒व । \newline
45. तेनै॒वैव तेन॒ तेनै॒वैना॑ एना ए॒व तेन॒ तेनै॒वैनाः᳚ । \newline
46. ए॒वैना॑ एना ए॒वै वैना॒ वि व्ये॑ना ए॒वै वैना॒ वि । \newline
47. ए॒ना॒ वि व्ये॑ना एना॒ वि भ॑जति भजति॒ व्ये॑ना एना॒ वि भ॑जति । \newline
48. वि भ॑जति भजति॒ वि वि भ॑ज त्ये॒त दे॒तद् भ॑जति॒ वि वि भ॑ज त्ये॒तत् । \newline
49. भ॒ज॒ त्ये॒त दे॒तद् भ॑जति भज त्ये॒तत् ते॑ त ए॒तद् भ॑जति भज त्ये॒तत् ते᳚ । \newline
50. ए॒तत् ते॑ त ए॒त दे॒तत् ते॑ अग्ने ऽग्ने त ए॒त दे॒तत् ते॑ अग्ने । \newline
51. ते॒ अ॒ग्ने॒ ऽग्ने॒ ते॒ ते॒ अ॒ग्ने॒ राधो॒ राधो᳚ ऽग्ने ते ते अग्ने॒ राधः॑ । \newline
52. अ॒ग्ने॒ राधो॒ राधो᳚ ऽग्ने ऽग्ने॒ राध॒ आ राधो᳚ ऽग्ने ऽग्ने॒ राध॒ आ । \newline
53. राध॒ आ राधो॒ राध॒ ऐत्ये॒त्या राधो॒ राध॒ ऐति॑ । \newline
\pagebreak
\markright{ TS 6.6.1.3  \hfill https://www.vedavms.in \hfill}

\section{ TS 6.6.1.3 }

\textbf{TS 6.6.1.3 } \newline
\textbf{Samhita Paata} \newline

ऐति॒ सोम॑च्युत॒मित्या॑ह॒ सोम॑च्युतꣳ॒॒ ह्य॑स्य॒ राध॒ ऐति॒ तन्मि॒त्रस्य॑ प॒था न॒येत्या॑ह॒ शान्त्या॑ ऋ॒तस्य॑ प॒था प्रेत॑ च॒न्द्र द॑क्षिणा॒ इत्या॑ह स॒त्यं ॅवा ऋ॒तꣳ स॒त्येनै॒वैना॑ ऋ॒तेन॒ वि भ॑जति य॒ज्ञ्स्य॑ प॒था सु॑वि॒ता नय॑न्ती॒रित्या॑ह य॒ज्ञ्स्य॒ ह्ये॑ताः प॒था यन्ति॒ यद्-दक्षि॑णा ब्राह्म॒णम॒द्य रा᳚द्ध्यास॒- [  ] \newline

\textbf{Pada Paata} \newline

एति॑ । ए॒ति॒ । सोम॑च्युत॒मिति॒ सोम॑ - च्यु॒त॒म् । इति॑ । आ॒ह॒ । सोम॑च्युत॒मिति॒ सोम॑ - च्यु॒त॒म् । हि । अ॒स्य॒ । राधः॑ । एति॑ । एति॑ । तत् । मि॒त्रस्य॑ । प॒था । न॒य॒ । इति॑ । आ॒ह॒ । शान्त्यै᳚ । ऋ॒तस्य॑ । प॒था । प्रेति॑ । इ॒त॒ । च॒न्द्रद॑क्षिणा॒ इति॑ च॒न्द्र - द॒क्षि॒णाः॒ । इति॑ । आ॒ह॒ । स॒त्यम् । वै । ऋ॒तम् । स॒त्येन॑ । ए॒व । ए॒नाः॒ । ऋ॒तेन॑ । वीति॑ । भ॒ज॒ति॒ । य॒ज्ञ्स्य॑ । प॒था । सु॒वि॒ता । नय॑न्तीः । इति॑ । आ॒ह॒ । य॒ज्ञ्स्य॑ । हि । ए॒ताः । प॒था । यन्ति॑ । यत् । दक्षि॑णाः । ब्रा॒ह्म॒णम् । अ॒द्य । रा॒द्ध्या॒स॒म् ।  \newline


\textbf{Krama Paata} \newline

ऐति॑ । ए॒ति॒ सोम॑च्युतम् । सोम॑च्युत॒मिति॑ । सोम॑च्युत॒मिति॒ सोम॑ - च्यु॒त॒म् । इत्या॑ह । आ॒ह॒ सोम॑च्युतम् । सोम॑च्युतꣳ॒॒ हि । सोम॑च्युत॒मिति॒ सोम॑ - च्यु॒त॒म् । ह्य॑स्य । अ॒स्य॒ राधः॑ । राध॒ आ । ऐति॑ । एति॒ तत् । तन् मि॒त्रस्य॑ । मि॒त्रस्य॑ प॒था । प॒था न॑य । न॒येति॑ । इत्या॑ह । आ॒ह॒ शान्त्यै᳚ । शान्त्या॑ ऋ॒तस्य॑ । ऋ॒तस्य॑ प॒था । प॒था प्र । प्रेत॑ । इ॒त॒ च॒न्द्रद॑क्षिणाः । च॒न्द्रद॑क्षिणा॒ इति॑ । च॒न्द्रद॑क्षिणा॒ इति॑ च॒न्द्र - द॒क्षि॒णाः॒ । इत्या॑ह । आ॒ह॒ स॒त्यम् । स॒त्यम् ॅवै । वा ऋ॒तम् । ऋ॒तꣳ स॒त्येन॑ । स॒त्येनै॒व । ए॒वैनाः᳚ । ए॒ना॒ ऋ॒तेन॑ । ऋ॒तेन॒ वि । वि भ॑जति । भ॒ज॒ति॒ य॒ज्ञ्स्य॑ । य॒ज्ञ्स्य॑ प॒था । प॒था सु॑वि॒ता । सु॒वि॒ता नय॑न्तीः । नय॑न्ती॒रिति॑ । इत्या॑ह । आ॒ह॒ य॒ज्ञ्स्य॑ । य॒ज्ञ्स्य॒ हि । ह्ये॑ताः । ए॒ताः प॒था । प॒था यन्ति॑ । यन्ति॒ यत् । यद् दक्षि॑णाः । दक्षि॑णा ब्राह्म॒णम् । ब्रा॒ह्म॒णम॒द्य । अ॒द्य रा᳚द्ध्यासम् । रा॒द्ध्या॒स॒मृषि᳚म् \newline

\textbf{Jatai Paata} \newline

1. ऐत्ये॒ त्यैति॑ । \newline
2. ए॒ति॒ सोम॑च्युतꣳ॒॒ सोम॑च्युत मेत्येति॒ सोम॑च्युतम् । \newline
3. सोम॑च्युत॒ मितीति॒ सोम॑च्युतꣳ॒॒ सोम॑च्युत॒ मिति॑ । \newline
4. सोम॑च्युत॒मिति॒ सोम॑ - च्यु॒त॒म् । \newline
5. इत्या॑हा॒हे तीत्या॑ह । \newline
6. आ॒ह॒ सोम॑च्युतꣳ॒॒ सोम॑च्युत माहाह॒ सोम॑च्युतम् । \newline
7. सोम॑च्युतꣳ॒॒ हि हि सोम॑च्युतꣳ॒॒ सोम॑च्युतꣳ॒॒ हि । \newline
8. सोम॑च्युत॒मिति॒ सोम॑ - च्यु॒त॒म् । \newline
9. ह्य॑स्यास्य॒ हि ह्य॑स्य । \newline
10. अ॒स्य॒ राधो॒ राधो᳚ ऽस्यास्य॒ राधः॑ । \newline
11. राध॒ आ राधो॒ राध॒ आ । \newline
12. ऐत्ये त्यैति॑ । \newline
13. एति॒ तत् तदे त्येति॒ तत् । \newline
14. तन् मि॒त्रस्य॑ मि॒त्रस्य॒ तत् तन् मि॒त्रस्य॑ । \newline
15. मि॒त्रस्य॑ प॒था प॒था मि॒त्रस्य॑ मि॒त्रस्य॑ प॒था । \newline
16. प॒था न॑य नय प॒था प॒था न॑य । \newline
17. न॒येतीति॑ नय न॒येति॑ । \newline
18. इत्या॑हा॒हे तीत्या॑ह । \newline
19. आ॒ह॒ शान्त्यै॒ शान्त्या॑ आहाह॒ शान्त्यै᳚ । \newline
20. शान्त्या॑ ऋ॒तस्य॒ र्‌तस्य॒ शान्त्यै॒ शान्त्या॑ ऋ॒तस्य॑ । \newline
21. ऋ॒तस्य॑ प॒था प॒थ र्‌तस्य॒ र्‌तस्य॑ प॒था । \newline
22. प॒था प्र प्र प॒था प॒था प्र । \newline
23. प्रेते॑त॒ प्र प्रेत॑ । \newline
24. इ॒त॒ च॒न्द्रद॑क्षिणा श्च॒न्द्रद॑क्षिणा इतेत च॒न्द्रद॑क्षिणाः । \newline
25. च॒न्द्रद॑क्षिणा॒ इतीति॑ च॒न्द्रद॑क्षिणा श्च॒न्द्रद॑क्षिणा॒ इति॑ । \newline
26. च॒न्द्रद॑क्षिणा॒ इति॑ च॒न्द्र - द॒क्षि॒णाः॒ । \newline
27. इत्या॑हा॒हे तीत्या॑ह । \newline
28. आ॒ह॒ स॒त्यꣳ स॒त्य मा॑हाह स॒त्यम् । \newline
29. स॒त्यं ॅवै वै स॒त्यꣳ स॒त्यं ॅवै । \newline
30. वा ऋ॒त मृ॒तं ॅवै वा ऋ॒तम् । \newline
31. ऋ॒तꣳ स॒त्येन॑ स॒त्येन॒ र्‌त मृ॒तꣳ स॒त्येन॑ । \newline
32. स॒त्येनै॒वैव स॒त्येन॑ स॒त्येनै॒व । \newline
33. ए॒वैना॑ एना ए॒वै वैनाः᳚ । \newline
34. ए॒ना॒ ऋ॒तेन॒ र्‌तेनै॑ना एना ऋ॒तेन॑ । \newline
35. ऋ॒तेन॒ वि व्यृ॑तेन॒ र्‌तेन॒ वि । \newline
36. वि भ॑जति भजति॒ वि वि भ॑जति । \newline
37. भ॒ज॒ति॒ य॒ज्ञ्स्य॑ य॒ज्ञ्स्य॑ भजति भजति य॒ज्ञ्स्य॑ । \newline
38. य॒ज्ञ्स्य॑ प॒था प॒था य॒ज्ञ्स्य॑ य॒ज्ञ्स्य॑ प॒था । \newline
39. प॒था सु॑वि॒ता सु॑वि॒ता प॒था प॒था सु॑वि॒ता । \newline
40. सु॒वि॒ता नय॑न्ती॒र् नय॑न्तीः सुवि॒ता सु॑वि॒ता नय॑न्तीः । \newline
41. नय॑न्ती॒ रितीति॒ नय॑न्ती॒र् नय॑न्ती॒ रिति॑ । \newline
42. इत्या॑हा॒हे तीत्या॑ह । \newline
43. आ॒ह॒ य॒ज्ञ्स्य॑ य॒ज्ञ् स्या॑हाह य॒ज्ञ्स्य॑ । \newline
44. य॒ज्ञ्स्य॒ हि हि य॒ज्ञ्स्य॑ य॒ज्ञ्स्य॒ हि । \newline
45. ह्ये॑ता ए॒ता हि ह्ये॑ताः । \newline
46. ए॒ताः प॒था प॒थैता ए॒ताः प॒था । \newline
47. प॒था यन्ति॒ यन्ति॑ प॒था प॒था यन्ति॑ । \newline
48. यन्ति॒ यद् यद् यन्ति॒ यन्ति॒ यत् । \newline
49. यद् दक्षि॑णा॒ दक्षि॑णा॒ यद् यद् दक्षि॑णाः । \newline
50. दक्षि॑णा ब्राह्म॒णम् ब्रा᳚ह्म॒णम् दक्षि॑णा॒ दक्षि॑णा ब्राह्म॒णम् । \newline
51. ब्रा॒ह्म॒ण म॒द्याद्य ब्रा᳚ह्म॒णम् ब्रा᳚ह्म॒ण म॒द्य । \newline
52. अ॒द्य रा᳚द्ध्यासꣳ राद्ध्यास म॒द्याद्य रा᳚द्ध्यासम् । \newline
53. रा॒द्ध्या॒स॒ मृषि॒ मृषिꣳ॑ राद्ध्यासꣳ राद्ध्यास॒ मृषि᳚म् । \newline

\textbf{Ghana Paata } \newline

1. ऐत्ये॒ त्यैति॒ सोम॑च्युतꣳ॒॒ सोम॑च्युत मे॒त्यैति॒ सोम॑च्युतम् । \newline
2. ए॒ति॒ सोम॑च्युतꣳ॒॒ सोम॑च्युत मेत्येति॒ सोम॑च्युत॒ मितीति॒ सोम॑च्युत मेत्येति॒ सोम॑च्युत॒ मिति॑ । \newline
3. सोम॑च्युत॒ मितीति॒ सोम॑च्युतꣳ॒॒ सोम॑च्युत॒ मित्या॑ हा॒हेति॒ सोम॑च्युतꣳ॒॒ सोम॑च्युत॒ मित्या॑ह । \newline
4. सोम॑च्युत॒मिति॒ सोम॑ - च्यु॒त॒म् । \newline
5. इत्या॑हा॒हे तीत्या॑ह॒ सोम॑च्युतꣳ॒॒ सोम॑च्युत मा॒हे तीत्या॑ह॒ सोम॑च्युतम् । \newline
6. आ॒ह॒ सोम॑च्युतꣳ॒॒ सोम॑च्युत माहाह॒ सोम॑च्युतꣳ॒॒ हि हि सोम॑च्युत माहाह॒ सोम॑च्युतꣳ॒॒ हि । \newline
7. सोम॑च्युतꣳ॒॒ हि हि सोम॑च्युतꣳ॒॒ सोम॑च्युतꣳ॒॒ ह्य॑स्यास्य॒ हि सोम॑च्युतꣳ॒॒ सोम॑च्युतꣳ॒॒ ह्य॑स्य । \newline
8. सोम॑च्युत॒मिति॒ सोम॑ - च्यु॒त॒म् । \newline
9. ह्य॑स्यास्य॒ हि ह्य॑स्य॒ राधो॒ राधो᳚ ऽस्य॒ हि ह्य॑स्य॒ राधः॑ । \newline
10. अ॒स्य॒ राधो॒ राधो᳚ ऽस्यास्य॒ राध॒ आ राधो᳚ ऽस्यास्य॒ राध॒ आ । \newline
11. राध॒ आ राधो॒ राध॒ ऐत्येत्या राधो॒ राध॒ ऐति॑ । \newline
12. ऐत्ये त्यैति॒ तत् तदे त्यैति॒ तत् । \newline
13. एति॒ तत् तदेत्येति॒ तन् मि॒त्रस्य॑ मि॒त्रस्य॒ तदेत्येति॒ तन् मि॒त्रस्य॑ । \newline
14. तन् मि॒त्रस्य॑ मि॒त्रस्य॒ तत् तन् मि॒त्रस्य॑ प॒था प॒था मि॒त्रस्य॒ तत् तन् मि॒त्रस्य॑ प॒था । \newline
15. मि॒त्रस्य॑ प॒था प॒था मि॒त्रस्य॑ मि॒त्रस्य॑ प॒था न॑य नय प॒था मि॒त्रस्य॑ मि॒त्रस्य॑ प॒था न॑य । \newline
16. प॒था न॑य नय प॒था प॒था न॒येतीति॑ नय प॒था प॒था न॒येति॑ । \newline
17. न॒येतीति॑ नय न॒ये त्या॑हा॒हेति॑ नय न॒ये त्या॑ह । \newline
18. इत्या॑हा॒हे तीत्या॑ह॒ शान्त्यै॒ शान्त्या॑ आ॒हे तीत्या॑ह॒ शान्त्यै᳚ । \newline
19. आ॒ह॒ शान्त्यै॒ शान्त्या॑ आहाह॒ शान्त्या॑ ऋ॒तस्य॒ र्‌तस्य॒ शान्त्या॑ आहाह॒ शान्त्या॑ ऋ॒तस्य॑ । \newline
20. शान्त्या॑ ऋ॒तस्य॒ र्‌तस्य॒ शान्त्यै॒ शान्त्या॑ ऋ॒तस्य॑ प॒था प॒थ र्‌तस्य॒ शान्त्यै॒ शान्त्या॑ ऋ॒तस्य॑ प॒था । \newline
21. ऋ॒तस्य॑ प॒था प॒थ र्‌तस्य॒ र्‌तस्य॑ प॒था प्र प्र प॒थ र्‌तस्य॒ र्‌तस्य॑ प॒था प्र । \newline
22. प॒था प्र प्र प॒था प॒था प्रे ते॑ त॒ प्र प॒था प॒था प्रे त॑ । \newline
23. प्रेते॑त॒ प्र प्रेत॑ च॒न्द्रद॑क्षिणा श्च॒न्द्रद॑क्षिणा इत॒ प्र प्रेत॑ च॒न्द्रद॑क्षिणाः । \newline
24. इ॒त॒ च॒न्द्रद॑क्षिणा श्च॒न्द्रद॑क्षिणा इतेत च॒न्द्रद॑क्षिणा॒ इतीति॑ च॒न्द्रद॑क्षिणा इतेत च॒न्द्रद॑क्षिणा॒ इति॑ । \newline
25. च॒न्द्रद॑क्षिणा॒ इतीति॑ च॒न्द्रद॑क्षिणा श्च॒न्द्रद॑क्षिणा॒ इत्या॑ हा॒हेति॑ च॒न्द्रद॑क्षिणा श्च॒न्द्रद॑क्षिणा॒ इत्या॑ह । \newline
26. च॒न्द्रद॑क्षिणा॒ इति॑ च॒न्द्र - द॒क्षि॒णाः॒ । \newline
27. इत्या॑हा॒हे तीत्या॑ह स॒त्यꣳ स॒त्य मा॒हे तीत्या॑ह स॒त्यम् । \newline
28. आ॒ह॒ स॒त्यꣳ स॒त्य मा॑हाह स॒त्यं ॅवै वै स॒त्य मा॑हाह स॒त्यं ॅवै । \newline
29. स॒त्यं ॅवै वै स॒त्यꣳ स॒त्यं ॅवा ऋ॒त मृ॒तं ॅवै स॒त्यꣳ स॒त्यं ॅवा ऋ॒तम् । \newline
30. वा ऋ॒त मृ॒तं ॅवै वा ऋ॒तꣳ स॒त्येन॑ स॒त्येन॒ र्‌तं ॅवै वा ऋ॒तꣳ स॒त्येन॑ । \newline
31. ऋ॒तꣳ स॒त्येन॑ स॒त्येन॒ र्‌त मृ॒तꣳ स॒त्ये नै॒वैव स॒त्येन॒ र्‌त मृ॒तꣳ स॒त्ये नै॒व । \newline
32. स॒त्ये नै॒वैव स॒त्येन॑ स॒त्ये नै॒वैना॑ एना ए॒व स॒त्येन॑ स॒त्ये नै॒वैनाः᳚ । \newline
33. ए॒वैना॑ एना ए॒वैवैना॑ ऋ॒तेन॒ र्‌तेनै॑ना ए॒वैवैना॑ ऋ॒तेन॑ । \newline
34. ए॒ना॒ ऋ॒तेन॒ र्‌तेनै॑ना एना ऋ॒तेन॒ वि व्यृ॑तेनै॑ना एना ऋ॒तेन॒ वि । \newline
35. ऋ॒तेन॒ वि व्यृ॑तेन॒ र्‌तेन॒ वि भ॑जति भजति॒ व्यृ॑तेन॒ र्‌तेन॒ वि भ॑जति । \newline
36. वि भ॑जति भजति॒ वि वि भ॑जति य॒ज्ञ्स्य॑ य॒ज्ञ्स्य॑ भजति॒ वि वि भ॑जति य॒ज्ञ्स्य॑ । \newline
37. भ॒ज॒ति॒ य॒ज्ञ्स्य॑ य॒ज्ञ्स्य॑ भजति भजति य॒ज्ञ्स्य॑ प॒था प॒था य॒ज्ञ्स्य॑ भजति भजति य॒ज्ञ्स्य॑ प॒था । \newline
38. य॒ज्ञ्स्य॑ प॒था प॒था य॒ज्ञ्स्य॑ य॒ज्ञ्स्य॑ प॒था सु॑वि॒ता सु॑वि॒ता प॒था य॒ज्ञ्स्य॑ य॒ज्ञ्स्य॑ प॒था सु॑वि॒ता । \newline
39. प॒था सु॑वि॒ता सु॑वि॒ता प॒था प॒था सु॑वि॒ता नय॑न्ती॒र् नय॑न्तीः सुवि॒ता प॒था प॒था सु॑वि॒ता नय॑न्तीः । \newline
40. सु॒वि॒ता नय॑न्ती॒र् नय॑न्तीः सुवि॒ता सु॑वि॒ता नय॑न्ती॒ रितीति॒ नय॑न्तीः सुवि॒ता सु॑वि॒ता नय॑न्ती॒ रिति॑ । \newline
41. नय॑न्ती॒ रितीति॒ नय॑न्ती॒र् नय॑न्ती॒ रित्या॑हा॒ हेति॒ नय॑न्ती॒र् नय॑न्ती॒ रित्या॑ह । \newline
42. इत्या॑हा॒हे तीत्या॑ह य॒ज्ञ्स्य॑ य॒ज्ञ्स्या॒हे तीत्या॑ह य॒ज्ञ्स्य॑ । \newline
43. आ॒ह॒ य॒ज्ञ्स्य॑ य॒ज्ञ्स्या॑ हाह य॒ज्ञ्स्य॒ हि हि य॒ज्ञ्स्या॑ हाह य॒ज्ञ्स्य॒ हि । \newline
44. य॒ज्ञ्स्य॒ हि हि य॒ज्ञ्स्य॑ य॒ज्ञ्स्य॒ ह्ये॑ता ए॒ता हि य॒ज्ञ्स्य॑ य॒ज्ञ्स्य॒ ह्ये॑ताः । \newline
45. ह्ये॑ता ए॒ता हि ह्ये॑ताः प॒था प॒थैता हि ह्ये॑ताः प॒था । \newline
46. ए॒ताः प॒था प॒थैता ए॒ताः प॒था यन्ति॒ यन्ति॑ प॒थैता ए॒ताः प॒था यन्ति॑ । \newline
47. प॒था यन्ति॒ यन्ति॑ प॒था प॒था यन्ति॒ यद् यद् यन्ति॑ प॒था प॒था यन्ति॒ यत् । \newline
48. यन्ति॒ यद् यद् यन्ति॒ यन्ति॒ यद् दक्षि॑णा॒ दक्षि॑णा॒ यद् यन्ति॒ यन्ति॒ यद् दक्षि॑णाः । \newline
49. यद् दक्षि॑णा॒ दक्षि॑णा॒ यद् यद् दक्षि॑णा ब्राह्म॒णम् ब्रा᳚ह्म॒णम् दक्षि॑णा॒ यद् यद् दक्षि॑णा ब्राह्म॒णम् । \newline
50. दक्षि॑णा ब्राह्म॒णम् ब्रा᳚ह्म॒णम् दक्षि॑णा॒ दक्षि॑णा ब्राह्म॒ण म॒द्याद्य ब्रा᳚ह्म॒णम् दक्षि॑णा॒ दक्षि॑णा ब्राह्म॒ण म॒द्य । \newline
51. ब्रा॒ह्म॒ण म॒द्याद्य ब्रा᳚ह्म॒णम् ब्रा᳚ह्म॒ण म॒द्य रा᳚द्ध्यासꣳ राद्ध्यास म॒द्य ब्रा᳚ह्म॒णम् ब्रा᳚ह्म॒ण म॒द्य रा᳚द्ध्यासम् । \newline
52. अ॒द्य रा᳚द्ध्यासꣳ राद्ध्यास म॒द्याद्य रा᳚द्ध्यास॒ मृषि॒ मृषिꣳ॑ राद्ध्यास म॒द्याद्य रा᳚द्ध्यास॒ मृषि᳚म् । \newline
53. रा॒द्ध्या॒स॒ मृषि॒ मृषिꣳ॑ राद्ध्यासꣳ राद्ध्यास॒ मृषि॑ मार्.षे॒य मा॑र्.षे॒य मृषिꣳ॑ राद्ध्यासꣳ राद्ध्यास॒ मृषि॑ मार्.षे॒यम् । \newline
\pagebreak
\markright{ TS 6.6.1.4  \hfill https://www.vedavms.in \hfill}

\section{ TS 6.6.1.4 }

\textbf{TS 6.6.1.4 } \newline
\textbf{Samhita Paata} \newline

मृषि॑मार्.षे॒यमित्या॑है॒ष वै ब्रा᳚ह्म॒ण ऋषि॑रार्.षे॒यो यः शु॑श्रु॒वान् तस्मा॑दे॒वमा॑ह॒ वि सुवः॒ पश्य॒ व्य॑न्तरि॑क्ष॒मित्या॑ह सुव॒र्गमे॒वैनं॑ ॅलो॒कं ग॑मयति॒ यत॑स्व सद॒स्यै॑रित्या॑ह मित्र॒त्वाया॒स्मद्दा᳚त्रा देव॒त्रा ग॑च्छत॒ मधु॑मतीः प्रदा॒तार॒मा वि॑श॒तेत्या॑ह व॒यमि॒ह प्र॑दा॒तारः॒ स्मो᳚ऽस्मान॒मुत्र॒ मधु॑मती॒रा वि॑श॒तेति॒- [  ] \newline

\textbf{Pada Paata} \newline

ऋषि᳚म् । आ॒र्॒.षे॒यम् । इति॑ । आ॒ह॒ । ए॒षः । वै । ब्रा॒ह्म॒णः । ऋषिः॑ । आ॒र्॒.षे॒यः । यः । शु॒श्रु॒वान् । तस्मा᳚त् । ए॒वम् । आ॒ह॒ । वीति॑ । सुवः॑ । पश्य॑ । वीति॑ । अ॒न्तरि॑क्षम् । इति॑ । आ॒ह॒ । सु॒व॒र्गमिति॑ सुवः-गम् । ए॒व । ए॒न॒म् । लो॒कम् । ग॒म॒य॒ति॒ । यत॑स्व । स॒द॒स्यैः᳚ । इति॑ । आ॒ह॒ । मि॒त्र॒त्वायेति॑ मित्र - त्वाय॑ । अ॒स्मद्दा᳚त्रा॒ इत्य॒स्मत् - दा॒त्राः॒ । दे॒व॒त्रेति॑ देव - त्रा । ग॒च्छ॒त॒ । मधु॑मती॒रिति॒ मधु॑ - म॒तीः॒ । प्र॒दा॒तार॒मिति॑ प्र - दा॒तार᳚म् । एति॑ । वि॒श॒त॒ । इति॑ । आ॒ह॒ । व॒यम् । इ॒ह । प्र॒दा॒तार॒ इति॑ प्र - दा॒तारः॑ । स्मः । अ॒स्मान् । अ॒मुत्र॑ । मधु॑मती॒रिति॒ मधु॑ - म॒तीः॒ । एति॑ । वि॒श॒त॒ । इति॑ ।  \newline


\textbf{Krama Paata} \newline

ऋषि॑मार्.षे॒यम् । आ॒र्.॒षे॒यमिति॑ । इत्या॑ह । आ॒है॒षः । ए॒ष वै । वै ब्रा᳚ह्म॒णः । ब्रा॒ह्म॒ण ऋषिः॑ । ऋषि॑रार्.षे॒यः । आ॒र्.॒षे॒यो यः । यः शु॑श्रु॒वान् । शु॒श्रु॒वान् तस्मा᳚त् । तस्मा॑दे॒वम् । ए॒वमा॑ह । आ॒ह॒ वि । वि सुवः॑ । सुवः॒ पश्य॑ । पश्य॒ वि । व्य॑न्तरि॑क्षम् । अ॒न्तरि॑क्ष॒मिति॑ । इत्या॑ह । आ॒ह॒ सु॒व॒र्गम् । सु॒व॒र्गमे॒व । सु॒व॒र्गमिति॑ सुवः - गम् । ए॒वैन᳚म् । ए॒न॒म् ॅलो॒कम् । लो॒कम् ग॑मयति । ग॒म॒य॒ति॒ यत॑स्व । यत॑स्व सद॒स्यैः᳚ । स॒द॒स्यै॑रिति॑ । इत्या॑ह । आ॒ह॒ मि॒त्र॒त्वाय॑ । मि॒त्र॒त्वाया॒स्मद्दा᳚त्राः । मि॒त्र॒त्वायेति॑ मित्र - त्वाय॑ । 
अ॒स्मद्‍दा᳚त्रा देव॒त्रा । अ॒स्मद्‍दा᳚त्रा॒ इत्य॒स्मत् - दा॒त्राः॒ । दे॒व॒त्रा ग॑च्छत । दे॒व॒त्रेति॑ देव - त्रा । ग॒च्छ॒त॒ मधु॑मतीः । मधु॑मतीः प्रदा॒तार᳚म् । मधु॑मती॒रिति॒ मधु॑ - म॒तीः॒ । प्र॒दा॒तार॒मा । प्र॒दा॒तार॒मिति॑ प्र - दा॒तार᳚म् । आ वि॑शत । वि॒श॒तेति॑ । इत्या॑ह । आ॒ह॒ व॒यम् । व॒यमि॒ह । इ॒ह प्र॑दा॒तारः॑ । प्र॒दा॒तारः॑ स्मः । प्र॒दा॒तार॒ इति॑ प्र - दा॒तारः॑ । स्मो᳚ऽस्मान् । अ॒स्मान॒मुत्र॑ । अ॒मुत्र॒ मधु॑मतीः । मधु॑मती॒रा । मधु॑मती॒रिति॒ मधु॑ - म॒तीः॒ । आ वि॑शत । वि॒श॒तेति॑ ( ) । इति॒ वाव \newline

\textbf{Jatai Paata} \newline

1. ऋषि॑ मार्.षे॒य मा॑र्.षे॒य मृषि॒ मृषि॑ मार्.षे॒यम् । \newline
2. आ॒र्॒.षे॒य मितीत्या॑र्.षे॒य मा॑र्.षे॒य मिति॑ । \newline
3. इत्या॑हा॒हे तीत्या॑ह । \newline
4. आ॒है॒ष ए॒ष आ॑हा है॒षः । \newline
5. ए॒ष वै वा ए॒ष ए॒ष वै । \newline
6. वै ब्रा᳚ह्म॒णो ब्रा᳚ह्म॒णो वै वै ब्रा᳚ह्म॒णः । \newline
7. ब्रा॒ह्म॒ण ऋषि॒र्॒. ऋषि॑र् ब्राह्म॒णो ब्रा᳚ह्म॒ण ऋषिः॑ । \newline
8. ऋषि॑ रार्.षे॒य आ॑र्.षे॒य ऋषि॒र्॒. ऋषि॑ रार्.षे॒यः । \newline
9. आ॒र्॒.षे॒यो यो य आ॑र्.षे॒य आ॑र्.षे॒यो यः । \newline
10. यः शु॑श्रु॒वाञ् छु॑श्रु॒वान्. यो यः शु॑श्रु॒वान् । \newline
11. शु॒श्रु॒वान् तस्मा॒त् तस्मा᳚च् छुश्रु॒वाञ् छु॑श्रु॒वान् तस्मा᳚त् । \newline
12. तस्मा॑ दे॒व मे॒वम् तस्मा॒त् तस्मा॑ दे॒वम् । \newline
13. ए॒व मा॑हा है॒व मे॒व मा॑ह । \newline
14. आ॒ह॒ वि व्या॑हाह॒ वि । \newline
15. वि सुवः॒ सुव॒र् वि वि सुवः॑ । \newline
16. सुवः॒ पश्य॒ पश्य॒ सुवः॒ सुवः॒ पश्य॑ । \newline
17. पश्य॒ वि वि पश्य॒ पश्य॒ वि । \newline
18. व्य॑न्तरि॑क्ष म॒न्तरि॑क्षं॒ ॅवि व्य॑न्तरि॑क्षम् । \newline
19. अ॒न्तरि॑क्ष॒ मिती त्य॒न्तरि॑क्ष म॒न्तरि॑क्ष॒ मिति॑ । \newline
20. इत्या॑हा॒हे तीत्या॑ह । \newline
21. आ॒ह॒ सु॒व॒र्गꣳ सु॑व॒र्ग मा॑हाह सुव॒र्गम् । \newline
22. सु॒व॒र्ग मे॒वैव सु॑व॒र्गꣳ सु॑व॒र्ग मे॒व । \newline
23. सु॒व॒र्गमिति॑ सुवः - गम् । \newline
24. ए॒वैन॑ मेन मे॒वै वैन᳚म् । \newline
25. ए॒न॒म् ॅलो॒कम् ॅलो॒क मे॑न मेनम् ॅलो॒कम् । \newline
26. लो॒कम् ग॑मयति गमयति लो॒कम् ॅलो॒कम् ग॑मयति । \newline
27. ग॒म॒य॒ति॒ यत॑स्व॒ यत॑स्व गमयति गमयति॒ यत॑स्व । \newline
28. यत॑स्व सद॒स्यैः᳚ सद॒स्यै᳚र् यत॑स्व॒ यत॑स्व सद॒स्यैः᳚ । \newline
29. स॒द॒स्यै॑ रितीति॑ सद॒स्यैः᳚ सद॒स्यै॑ रिति॑ । \newline
30. इत्या॑हा॒हे तीत्या॑ह । \newline
31. आ॒ह॒ मि॒त्र॒त्वाय॑ मित्र॒त्वा या॑हाह मित्र॒त्वाय॑ । \newline
32. मि॒त्र॒त्वा या॒स्मद्दा᳚त्रा अ॒स्मद्दा᳚त्रा मित्र॒त्वाय॑ मित्र॒त्वा या॒स्मद्दा᳚त्राः । \newline
33. मि॒त्र॒त्वायेति॑ मित्र - त्वाय॑ । \newline
34. अ॒स्मद्दा᳚त्रा देव॒त्रा दे॑व॒त्रा ऽस्मद्दा᳚त्रा अ॒स्मद्दा᳚त्रा देव॒त्रा । \newline
35. अ॒स्मद्दा᳚त्रा॒ इत्य॒स्मत् - दा॒त्राः॒ । \newline
36. दे॒व॒त्रा ग॑च्छत गच्छत देव॒त्रा दे॑व॒त्रा ग॑च्छत । \newline
37. दे॒व॒त्रेति॑ देव - त्रा । \newline
38. ग॒च्छ॒त॒ मधु॑मती॒र् मधु॑मतीर् गच्छत गच्छत॒ मधु॑मतीः । \newline
39. मधु॑मतीः प्रदा॒तार॑म् प्रदा॒तार॒म् मधु॑मती॒र् मधु॑मतीः प्रदा॒तार᳚म् । \newline
40. मधु॑मती॒रिति॒ मधु॑ - म॒तीः॒ । \newline
41. प्र॒दा॒तार॒ मा प्र॑दा॒तार॑म् प्रदा॒तार॒ मा । \newline
42. प्र॒दा॒तार॒मिति॑ प्र - दा॒तार᳚म् । \newline
43. आ वि॑शत विश॒ता वि॑शत । \newline
44. वि॒श॒तेतीति॑ विशत विश॒तेति॑ । \newline
45. इत्या॑हा॒हे तीत्या॑ह । \newline
46. आ॒ह॒ व॒यं ॅव॒य मा॑हाह व॒यम् । \newline
47. व॒य मि॒हेह व॒यं ॅव॒य मि॒ह । \newline
48. इ॒ह प्र॑दा॒तारः॑ प्रदा॒तार॑ इ॒हेह प्र॑दा॒तारः॑ । \newline
49. प्र॒दा॒तारः॒ स्मः स्मः प्र॑दा॒तारः॑ प्रदा॒तारः॒ स्मः । \newline
50. प्र॒दा॒तार॒ इति॑ प्र - दा॒तारः॑ । \newline
51. स्मो᳚ ऽस्मा न॒स्मान् थ्स्मः स्मो᳚ ऽस्मान् । \newline
52. अ॒स्मा न॒मुत्रा॒ मुत्रा॒ स्मान॒स्मा न॒मुत्र॑ । \newline
53. अ॒मुत्र॒ मधु॑मती॒र् मधु॑मती र॒मुत्रा॒ मुत्र॒ मधु॑मतीः । \newline
54. मधु॑मती॒रा मधु॑मती॒र् मधु॑मती॒रा । \newline
55. मधु॑मती॒रिति॒ मधु॑ - म॒तीः॒ । \newline
56. आ वि॑शत विश॒ता वि॑शत । \newline
57. वि॒श॒तेतीति॑ विशत विश॒तेति॑ । \newline
58. इति॒ वाव वावे तीति॒ वाव । \newline

\textbf{Ghana Paata } \newline

1. ऋषि॑ मार्.षे॒य मा॑र्.षे॒य मृषि॒ मृषि॑ मार्.षे॒य मिती त्या॑र्.षे॒य मृषि॒ मृषि॑ मार्.षे॒य मिति॑ । \newline
2. आ॒र्॒.षे॒य मिती त्या॑र्.षे॒य मा॑र्.षे॒य मित्या॑हा॒हे त्या॑र्.षे॒य मा॑र्.षे॒य मित्या॑ह । \newline
3. इत्या॑हा॒हे तीत्या॑ है॒ष ए॒ष आ॒हे तीत्या॑ है॒षः । \newline
4. आ॒है॒ष ए॒ष आ॑हा है॒ष वै वा ए॒ष आ॑हा है॒ष वै । \newline
5. ए॒ष वै वा ए॒ष ए॒ष वै ब्रा᳚ह्म॒णो ब्रा᳚ह्म॒णो वा ए॒ष ए॒ष वै ब्रा᳚ह्म॒णः । \newline
6. वै ब्रा᳚ह्म॒णो ब्रा᳚ह्म॒णो वै वै ब्रा᳚ह्म॒ण ऋषि॒र्॒. ऋषि॑र् ब्राह्म॒णो वै वै ब्रा᳚ह्म॒ण ऋषिः॑ । \newline
7. ब्रा॒ह्म॒ण ऋषि॒र्॒. ऋषि॑र् ब्राह्म॒णो ब्रा᳚ह्म॒ण ऋषि॑ रार्.षे॒य आ॑र्.षे॒य ऋषि॑र् ब्राह्म॒णो ब्रा᳚ह्म॒ण ऋषि॑ रार्.षे॒यः । \newline
8. ऋषि॑ रार्.षे॒य आ॑र्.षे॒य ऋषि॒र्॒. ऋषि॑ रार्.षे॒यो यो य आ॑र्.षे॒य ऋषि॒र्॒. ऋषि॑ रार्.षे॒यो यः । \newline
9. आ॒र्॒.षे॒यो यो य आ॑र्.षे॒य आ॑र्.षे॒यो यः शु॑श्रु॒वाञ् छु॑श्रु॒वान्. य आ॑र्.षे॒य आ॑र्.षे॒यो यः शु॑श्रु॒वान् । \newline
10. यः शु॑श्रु॒वाञ् छु॑श्रु॒वान्. यो यः शु॑श्रु॒वान् तस्मा॒त् तस्मा᳚च् छुश्रु॒वान्. यो यः शु॑श्रु॒वान् तस्मा᳚त् । \newline
11. शु॒श्रु॒वान् तस्मा॒त् तस्मा᳚च् छुश्रु॒वाञ् छु॑श्रु॒वान् तस्मा॑ दे॒व मे॒वम् तस्मा᳚च् छुश्रु॒वाञ् छु॑श्रु॒वान् तस्मा॑ दे॒वम् । \newline
12. तस्मा॑ दे॒व मे॒वम् तस्मा॒त् तस्मा॑ दे॒व मा॑हा है॒वम् तस्मा॒त् तस्मा॑ दे॒व मा॑ह । \newline
13. ए॒व मा॑हा है॒व मे॒व मा॑ह॒ वि व्या॑है॒व मे॒व मा॑ह॒ वि । \newline
14. आ॒ह॒ वि व्या॑हाह॒ वि सुवः॒ सुव॒र् व्या॑हाह॒ वि सुवः॑ । \newline
15. वि सुवः॒ सुव॒र् वि वि सुवः॒ पश्य॒ पश्य॒ सुव॒र् वि वि सुवः॒ पश्य॑ । \newline
16. सुवः॒ पश्य॒ पश्य॒ सुवः॒ सुवः॒ पश्य॒ वि वि पश्य॒ सुवः॒ सुवः॒ पश्य॒ वि । \newline
17. पश्य॒ वि वि पश्य॒ पश्य॒ व्य॑न्तरि॑क्ष म॒न्तरि॑क्षं॒ ॅवि पश्य॒ पश्य॒ व्य॑न्तरि॑क्षम् । \newline
18. व्य॑न्तरि॑क्ष म॒न्तरि॑क्षं॒ ॅवि व्य॑न्तरि॑क्ष॒ मिती त्य॒न्तरि॑क्षं॒ ॅवि व्य॑न्तरि॑क्ष॒ मिति॑ । \newline
19. अ॒न्तरि॑क्ष॒ मिती त्य॒न्तरि॑क्ष म॒न्तरि॑क्ष॒ मित्या॑हा॒हे त्य॒न्तरि॑क्ष म॒न्तरि॑क्ष॒ मित्या॑ह । \newline
20. इत्या॑हा॒हे तीत्या॑ह सुव॒र्गꣳ सु॑व॒र्ग मा॒हे तीत्या॑ह सुव॒र्गम् । \newline
21. आ॒ह॒ सु॒व॒र्गꣳ सु॑व॒र्ग मा॑हाह सुव॒र्ग मे॒वैव सु॑व॒र्ग मा॑हाह सुव॒र्ग मे॒व । \newline
22. सु॒व॒र्ग मे॒वैव सु॑व॒र्गꣳ सु॑व॒र्ग मे॒वैन॑ मेन मे॒व सु॑व॒र्गꣳ सु॑व॒र्ग मे॒वैन᳚म् । \newline
23. सु॒व॒र्गमिति॑ सुवः - गम् । \newline
24. ए॒वैन॑ मेन मे॒वै वैन॑म् ॅलो॒कम् ॅलो॒क मे॑न मे॒वै वैन॑म् ॅलो॒कम् । \newline
25. ए॒न॒म् ॅलो॒कम् ॅलो॒क मे॑न मेनम् ॅलो॒कम् ग॑मयति गमयति लो॒क मे॑न मेनम् ॅलो॒कम् ग॑मयति । \newline
26. लो॒कम् ग॑मयति गमयति लो॒कम् ॅलो॒कम् ग॑मयति॒ यत॑स्व॒ यत॑स्व गमयति लो॒कम् ॅलो॒कम् ग॑मयति॒ यत॑स्व । \newline
27. ग॒म॒य॒ति॒ यत॑स्व॒ यत॑स्व गमयति गमयति॒ यत॑स्व सद॒स्यैः᳚ सद॒स्यै᳚र् यत॑स्व गमयति गमयति॒ यत॑स्व सद॒स्यैः᳚ । \newline
28. यत॑स्व सद॒स्यैः᳚ सद॒स्यै᳚र् यत॑स्व॒ यत॑स्व सद॒स्यै॑ रितीति॑ सद॒स्यै᳚र् यत॑स्व॒ यत॑स्व सद॒स्यै॑ रिति॑ । \newline
29. स॒द॒स्यै॑ रितीति॑ सद॒स्यैः᳚ सद॒स्यै॑ रित्या॑हा॒ हेति॑ सद॒स्यैः᳚ सद॒स्यै॑ रित्या॑ह । \newline
30. इत्या॑हा॒हे तीत्या॑ह मित्र॒त्वाय॑ मित्र॒त्वा या॒हेतीत्या॑ह मित्र॒त्वाय॑ । \newline
31. आ॒ह॒ मि॒त्र॒त्वाय॑ मित्र॒त्वाया॑ हाह मित्र॒त्वाया॒ स्मद्दा᳚त्रा अ॒स्मद्दा᳚त्रा मित्र॒त्वाया॑ हाह मित्र॒त्वाया॒ स्मद्दा᳚त्राः । \newline
32. मि॒त्र॒त्वाया॒ स्मद्दा᳚त्रा अ॒स्मद्दा᳚त्रा मित्र॒त्वाय॑ मित्र॒त्वाया॒ स्मद्दा᳚त्रा देव॒त्रा दे॑व॒त्रा ऽस्मद्दा᳚त्रा मित्र॒त्वाय॑ मित्र॒त्वाया॒ स्मद्दा᳚त्रा देव॒त्रा । \newline
33. मि॒त्र॒त्वायेति॑ मित्र - त्वाय॑ । \newline
34. अ॒स्मद्दा᳚त्रा देव॒त्रा दे॑व॒त्रा ऽस्मद्दा᳚त्रा अ॒स्मद्दा᳚त्रा देव॒त्रा ग॑च्छत गच्छत देव॒त्रा ऽस्मद्दा᳚त्रा अ॒स्मद्दा᳚त्रा देव॒त्रा ग॑च्छत । \newline
35. अ॒स्मद्दा᳚त्रा॒ इत्य॒स्मत् - दा॒त्राः॒ । \newline
36. दे॒व॒त्रा ग॑च्छत गच्छत देव॒त्रा दे॑व॒त्रा ग॑च्छत॒ मधु॑मती॒र् मधु॑मतीर् गच्छत देव॒त्रा दे॑व॒त्रा ग॑च्छत॒ मधु॑मतीः । \newline
37. दे॒व॒त्रेति॑ देव - त्रा । \newline
38. ग॒च्छ॒त॒ मधु॑मती॒र् मधु॑मतीर् गच्छत गच्छत॒ मधु॑मतीः प्रदा॒तार॑म् प्रदा॒तार॒म् मधु॑मतीर् गच्छत गच्छत॒ मधु॑मतीः प्रदा॒तार᳚म् । \newline
39. मधु॑मतीः प्रदा॒तार॑म् प्रदा॒तार॒म् मधु॑मती॒र् मधु॑मतीः प्रदा॒तार॒ मा प्र॑दा॒तार॒म् मधु॑मती॒र् मधु॑मतीः प्रदा॒तार॒ मा । \newline
40. मधु॑मती॒रिति॒ मधु॑ - म॒तीः॒ । \newline
41. प्र॒दा॒तार॒ मा प्र॑दा॒तार॑म् प्रदा॒तार॒ मा वि॑शत विश॒ता प्र॑दा॒तार॑म् प्रदा॒तार॒ मा वि॑शत । \newline
42. प्र॒दा॒तार॒मिति॑ प्र - दा॒तार᳚म् । \newline
43. आ वि॑शत विश॒ता वि॑श॒तेतीति॑ विश॒ता वि॑श॒तेति॑ । \newline
44. वि॒श॒तेतीति॑ विशत विश॒ते त्या॑हा॒ हेति॑ विशत विश॒ते त्या॑ह । \newline
45. इत्या॑हा॒हे तीत्या॑ह व॒यं ॅव॒य मा॒हे तीत्या॑ह व॒यम् । \newline
46. आ॒ह॒ व॒यं ॅव॒य मा॑हाह व॒य मि॒हेह व॒य मा॑हाह व॒य मि॒ह । \newline
47. व॒य मि॒हेह व॒यं ॅव॒य मि॒ह प्र॑दा॒तारः॑ प्रदा॒तार॑ इ॒ह व॒यं ॅव॒य मि॒ह प्र॑दा॒तारः॑ । \newline
48. इ॒ह प्र॑दा॒तारः॑ प्रदा॒तार॑ इ॒हेह प्र॑दा॒तारः॒ स्मः स्मः प्र॑दा॒तार॑ इ॒हेह प्र॑दा॒तारः॒ स्मः । \newline
49. प्र॒दा॒तारः॒ स्मः स्मः प्र॑दा॒तारः॑ प्रदा॒तारः॒ स्मो᳚ ऽस्मा न॒स्मान् थ्स्मः प्र॑दा॒तारः॑ प्रदा॒तारः॒ स्मो᳚ ऽस्मान् । \newline
50. प्र॒दा॒तार॒ इति॑ प्र - दा॒तारः॑ । \newline
51. स्मो᳚ ऽस्मा न॒स्मान् थ्स्मः स्मो᳚ ऽस्मा न॒मुत्रा॒ मुत्रा॒ स्मान् थ्स्मः स्मो᳚ ऽस्मा न॒मुत्र॑ । \newline
52. अ॒स्मा न॒मुत्रा॒ मुत्रा॒स्मा न॒स्मा न॒मुत्र॒ मधु॑मती॒र् मधु॑मती र॒मुत्रा॒ स्मा न॒स्मा न॒मुत्र॒ मधु॑मतीः । \newline
53. अ॒मुत्र॒ मधु॑मती॒र् मधु॑मती र॒मुत्रा॒ मुत्र॒ मधु॑मती॒रा मधु॑मती र॒मुत्रा॒ मुत्र॒ मधु॑मती॒रा । \newline
54. मधु॑मती॒रा मधु॑मती॒र् मधु॑मती॒रा वि॑शत विश॒ता मधु॑मती॒र् मधु॑मती॒रा वि॑शत । \newline
55. मधु॑मती॒रिति॒ मधु॑ - म॒तीः॒ । \newline
56. आ वि॑शत विश॒ता वि॑श॒ते तीति॑ विश॒ता वि॑श॒तेति॑ । \newline
57. वि॒श॒तेतीति॑ विशत विश॒तेति॒ वाव वावेति॑ विशत विश॒तेति॒ वाव । \newline
58. इति॒ वाव वावेतीति॒ वावैत दे॒तद् वावेतीति॒ वावैतत् । \newline
\pagebreak
\markright{ TS 6.6.1.5  \hfill https://www.vedavms.in \hfill}

\section{ TS 6.6.1.5 }

\textbf{TS 6.6.1.5 } \newline
\textbf{Samhita Paata} \newline

वावैतदा॑ह॒ हिर॑ण्यं ददाति॒ ज्योति॒र्वै हिर॑ण्यं॒ ज्योति॑रे॒व पु॒रस्ता᳚द्धत्ते सुव॒र्गस्य॑ लो॒कस्यानु॑ख्यात्या अ॒ग्नीधे॑ ददात्य॒ग्निमु॑खाने॒वर्तून् प्री॑णाति ब्र॒ह्मणे॑ ददाति॒ प्रसू᳚त्यै॒ होत्रे॑ ददात्या॒त्मा वा ए॒ष य॒ज्ञ्स्य॒ यद्धोता॒ऽऽत्मान॑मे॒व य॒ज्ञ्स्य॒ दक्षि॑णाभिः॒ सम॑र्द्धयति ॥ \newline

\textbf{Pada Paata} \newline

वाव । ए॒तत् । आ॒ह॒ । हिर॑ण्यम् । द॒दा॒ति॒ । ज्योतिः॑ । वै । हिर॑ण्यम् । ज्योतिः॑ । ए॒व । पु॒रस्ता᳚त् । ध॒त्ते॒ । सु॒व॒र्गस्येति॑ सुवः - गस्य॑ । लो॒कस्य॑ । अनु॑ख्यात्या॒ इत्यनु॑ - ख्या॒त्यै॒ । अ॒ग्नीध॒ इत्य॑ग्नि - इधे᳚ । द॒दा॒ति॒ । अ॒ग्निमु॑खा॒नित्य॒ग्नि - मु॒खा॒न् । ए॒व । ऋ॒तून् । प्री॒णा॒ति॒ । ब्र॒ह्मणे᳚ । द॒दा॒ति॒ । प्रसू᳚त्या॒ इति॒ प्र-सू॒त्यै॒ । होत्रे᳚ । द॒दा॒ति॒ । आ॒त्मा । वै । ए॒षः । य॒ज्ञ्स्य॑ । यत् । होता᳚ । आ॒त्मान᳚म् । ए॒व । य॒ज्ञ्स्य॑ । दक्षि॑णाभिः । समिति॑ । अ॒द्‌र्ध॒य॒ति॒ ॥  \newline


\textbf{Krama Paata} \newline

वावैतत् । ए॒तदा॑ह । आ॒ह॒ हिर॑ण्यम् । हिर॑ण्यम् ददाति । द॒दा॒ति॒ ज्योतिः॑ । ज्योति॒र् वै । वै हिर॑ण्यम् । हिर॑ण्य॒म् ज्योतिः॑ । ज्योति॑रे॒व । ए॒व पु॒रस्ता᳚त् । पु॒रस्ता᳚द् धत्ते । ध॒त्ते॒ सु॒व॒र्गस्य॑ । सु॒व॒र्गस्य॑ लो॒कस्य॑ । सु॒व॒गस्येति॑ सुवः - गस्य॑ । लो॒कस्यानु॑ख्यात्यै । अनु॑ख्यात्या अ॒ग्नीधे᳚ । अनु॑ख्यात्या॒ इत्यनु॑ - ख्या॒त्यै॒ । अ॒ग्नीधे॑ ददाति । अ॒ग्नीध॒ इत्य॑ग्नि - इधे᳚ । द॒दा॒त्य॒ग्निमु॑खान् । अ॒ग्निमु॑खाने॒व । अ॒ग्निमु॑खा॒नित्य॒ग्नि - मु॒खा॒न्॒ । ए॒वर्तून् । ऋ॒तून् प्री॑णाति । प्री॒णा॒ति॒ ब्र॒ह्मणे᳚ । ब्र॒ह्मणे॑ ददाति । द॒दा॒ति॒ प्रसू᳚त्यै । प्रसू᳚त्यै॒ होत्रे᳚ । प्रसू᳚त्या॒ इति॒ प्र - सू॒त्यै॒ । होत्रे॑ ददाति । द॒दा॒त्या॒त्मा । आ॒त्मा वै । वा ए॒षः । ए॒ष य॒ज्ञ्स्य॑ । य॒ज्ञ्स्य॒ यत् । यद्‌धोता᳚ । होता॒ऽऽत्मानम्᳚ । आ॒त्मान॑मे॒व । ए॒व य॒ज्ञ्स्य॑ । य॒ज्ञ्स्य॒ दक्षि॑णाभिः । दक्षि॑णाभिः॒ सम् । सम॑र्द्धयति । अ॒र्द्ध॒य॒तीत्य॑र्द्धयति । \newline

\textbf{Jatai Paata} \newline

1. वावैत दे॒तद् वाव वावैतत् । \newline
2. ए॒त दा॑हा है॒त दे॒त दा॑ह । \newline
3. आ॒ह॒ हिर॑ण्यꣳ॒॒ हिर॑ण्य माहाह॒ हिर॑ण्यम् । \newline
4. हिर॑ण्यम् ददाति ददाति॒ हिर॑ण्यꣳ॒॒ हिर॑ण्यम् ददाति । \newline
5. द॒दा॒ति॒ ज्योति॒र् ज्योति॑र् ददाति ददाति॒ ज्योतिः॑ । \newline
6. ज्योति॒र् वै वै ज्योति॒र् ज्योति॒र् वै । \newline
7. वै हिर॑ण्यꣳ॒॒ हिर॑ण्यं॒ ॅवै वै हिर॑ण्यम् । \newline
8. हिर॑ण्य॒म् ज्योति॒र् ज्योति॒र्॒. हिर॑ण्यꣳ॒॒ हिर॑ण्य॒म् ज्योतिः॑ । \newline
9. ज्योति॑ रे॒वैव ज्योति॒र् ज्योति॑ रे॒व । \newline
10. ए॒व पु॒रस्ता᳚त् पु॒रस्ता॑ दे॒वैव पु॒रस्ता᳚त् । \newline
11. पु॒रस्ता᳚द् धत्ते धत्ते पु॒रस्ता᳚त् पु॒रस्ता᳚द् धत्ते । \newline
12. ध॒त्ते॒ सु॒व॒र्गस्य॑ सुव॒र्गस्य॑ धत्ते धत्ते सुव॒र्गस्य॑ । \newline
13. सु॒व॒र्गस्य॑ लो॒कस्य॑ लो॒कस्य॑ सुव॒र्गस्य॑ सुव॒र्गस्य॑ लो॒कस्य॑ । \newline
14. सु॒व॒र्गस्येति॑ सुवः - गस्य॑ । \newline
15. लो॒कस्यानु॑ ख्यात्या॒ अनु॑ख्यात्यै लो॒कस्य॑ लो॒कस्या नु॑ख्यात्यै । \newline
16. अनु॑ख्यात्या अ॒ग्नीधे॒ ऽग्नीधे ऽनु॑ख्यात्या॒ अनु॑ख्यात्या अ॒ग्नीधे᳚ । \newline
17. अनु॑ख्यात्या॒ इत्यनु॑ - ख्या॒त्यै॒ । \newline
18. अ॒ग्नीधे॑ ददाति ददा त्य॒ग्नीधे॒ ऽग्नीधे॑ ददाति । \newline
19. अ॒ग्नीध॒ इत्य॑ग्नि - इधे᳚ । \newline
20. द॒दा॒ त्य॒ग्निमु॑खा न॒ग्निमु॑खान् ददाति ददा त्य॒ग्निमु॑खान् । \newline
21. अ॒ग्निमु॑खा ने॒वै वाग्निमु॑खा न॒ग्निमु॑खा ने॒व । \newline
22. अ॒ग्निमु॑खा॒नित्य॒ग्नि - मु॒खा॒न् । \newline
23. ए॒व र्‌तू नृ॒तू ने॒वैव र्‌तून् । \newline
24. ऋ॒तून् प्री॑णाति प्रीणा त्यृ॒तू नृ॒तून् प्री॑णाति । \newline
25. प्री॒णा॒ति॒ ब्र॒ह्मणे᳚ ब्र॒ह्मणे᳚ प्रीणाति प्रीणाति ब्र॒ह्मणे᳚ । \newline
26. ब्र॒ह्मणे॑ ददाति ददाति ब्र॒ह्मणे᳚ ब्र॒ह्मणे॑ ददाति । \newline
27. द॒दा॒ति॒ प्रसू᳚त्यै॒ प्रसू᳚त्यै ददाति ददाति॒ प्रसू᳚त्यै । \newline
28. प्रसू᳚त्यै॒ होत्रे॒ होत्रे॒ प्रसू᳚त्यै॒ प्रसू᳚त्यै॒ होत्रे᳚ । \newline
29. प्रसू᳚त्या॒ इति॒ प्र - सू॒त्यै॒ । \newline
30. होत्रे॑ ददाति ददाति॒ होत्रे॒ होत्रे॑ ददाति । \newline
31. द॒दा॒ त्या॒त्मा ऽऽत्मा द॑दाति ददा त्या॒त्मा । \newline
32. आ॒त्मा वै वा आ॒त्मा ऽऽत्मा वै । \newline
33. वा ए॒ष ए॒ष वै वा ए॒षः । \newline
34. ए॒ष य॒ज्ञ्स्य॑ य॒ज्ञ् स्यै॒ष ए॒ष य॒ज्ञ्स्य॑ । \newline
35. य॒ज्ञ्स्य॒ यद् यद् य॒ज्ञ्स्य॑ य॒ज्ञ्स्य॒ यत् । \newline
36. यद्धोता॒ होता॒ यद् यद्धोता᳚ । \newline
37. होता॒ ऽऽत्मान॑ मा॒त्मानꣳ॒॒ होता॒ होता॒ ऽऽत्मान᳚म् । \newline
38. आ॒त्मान॑ मे॒वै वात्मान॑ मा॒त्मान॑ मे॒व । \newline
39. ए॒व य॒ज्ञ्स्य॑ य॒ज्ञ् स्यै॒वैव य॒ज्ञ्स्य॑ । \newline
40. य॒ज्ञ्स्य॒ दक्षि॑णाभि॒र् दक्षि॑णाभिर् य॒ज्ञ्स्य॑ य॒ज्ञ्स्य॒ दक्षि॑णाभिः । \newline
41. दक्षि॑णाभिः॒ सꣳ सम् दक्षि॑णाभि॒र् दक्षि॑णाभिः॒ सम् । \newline
42. स म॑र्द्धय त्यर्द्धयति॒ सꣳ स म॑र्द्धयति । \newline
43. अ॒र्द्ध॒य॒तीत्य॑र्द्धयति । \newline

\textbf{Ghana Paata } \newline

1. वावैत दे॒तद् वाव वावैत दा॑हा है॒तद् वाव वावैतदा॑ह । \newline
2. ए॒त दा॑हा है॒त दे॒त दा॑ह॒ हिर॑ण्यꣳ॒॒ हिर॑ण्य माहै॒त दे॒त दा॑ह॒ हिर॑ण्यम् । \newline
3. आ॒ह॒ हिर॑ण्यꣳ॒॒ हिर॑ण्य माहाह॒ हिर॑ण्यम् ददाति ददाति॒ हिर॑ण्य माहाह॒ हिर॑ण्यम् ददाति । \newline
4. हिर॑ण्यम् ददाति ददाति॒ हिर॑ण्यꣳ॒॒ हिर॑ण्यम् ददाति॒ ज्योति॒र् ज्योति॑र् ददाति॒ हिर॑ण्यꣳ॒॒ हिर॑ण्यम् ददाति॒ ज्योतिः॑ । \newline
5. द॒दा॒ति॒ ज्योति॒र् ज्योति॑र् ददाति ददाति॒ ज्योति॒र् वै वै ज्योति॑र् ददाति ददाति॒ ज्योति॒र् वै । \newline
6. ज्योति॒र् वै वै ज्योति॒र् ज्योति॒र् वै हिर॑ण्यꣳ॒॒ हिर॑ण्यं॒ ॅवै ज्योति॒र् ज्योति॒र् वै हिर॑ण्यम् । \newline
7. वै हिर॑ण्यꣳ॒॒ हिर॑ण्यं॒ ॅवै वै हिर॑ण्य॒म् ज्योति॒र् ज्योति॒र्॒. हिर॑ण्यं॒ ॅवै वै हिर॑ण्य॒म् ज्योतिः॑ । \newline
8. हिर॑ण्य॒म् ज्योति॒र् ज्योति॒र्॒. हिर॑ण्यꣳ॒॒ हिर॑ण्य॒म् ज्योति॑ रे॒वैव ज्योति॒र्॒. हिर॑ण्यꣳ॒॒ हिर॑ण्य॒म् ज्योति॑ रे॒व । \newline
9. ज्योति॑ रे॒वैव ज्योति॒र् ज्योति॑ रे॒व पु॒रस्ता᳚त् पु॒रस्ता॑ दे॒व ज्योति॒र् ज्योति॑ रे॒व पु॒रस्ता᳚त् । \newline
10. ए॒व पु॒रस्ता᳚त् पु॒रस्ता॑ दे॒वैव पु॒रस्ता᳚द् धत्ते धत्ते पु॒रस्ता॑ दे॒वैव पु॒रस्ता᳚द् धत्ते । \newline
11. पु॒रस्ता᳚द् धत्ते धत्ते पु॒रस्ता᳚त् पु॒रस्ता᳚द् धत्ते सुव॒र्गस्य॑ सुव॒र्गस्य॑ धत्ते पु॒रस्ता᳚त् पु॒रस्ता᳚द् धत्ते सुव॒र्गस्य॑ । \newline
12. ध॒त्ते॒ सु॒व॒र्गस्य॑ सुव॒र्गस्य॑ धत्ते धत्ते सुव॒र्गस्य॑ लो॒कस्य॑ लो॒कस्य॑ सुव॒र्गस्य॑ धत्ते धत्ते सुव॒र्गस्य॑ लो॒कस्य॑ । \newline
13. सु॒व॒र्गस्य॑ लो॒कस्य॑ लो॒कस्य॑ सुव॒र्गस्य॑ सुव॒र्गस्य॑ लो॒कस्या नु॑ख्यात्या॒ अनु॑ख्यात्यै लो॒कस्य॑ सुव॒र्गस्य॑ सुव॒र्गस्य॑ लो॒कस्या नु॑ख्यात्यै । \newline
14. सु॒व॒र्गस्येति॑ सुवः - गस्य॑ । \newline
15. लो॒कस्या नु॑ख्यात्या॒ अनु॑ख्यात्यै लो॒कस्य॑ लो॒कस्या नु॑ख्यात्या अ॒ग्नीधे॒ ऽग्नीधे ऽनु॑ख्यात्यै लो॒कस्य॑ लो॒कस्या नु॑ख्यात्या अ॒ग्नीधे᳚ । \newline
16. अनु॑ख्यात्या अ॒ग्नीधे॒ ऽग्नीधे ऽनु॑ख्यात्या॒ अनु॑ख्यात्या अ॒ग्नीधे॑ ददाति ददा त्य॒ग्नीधे ऽनु॑ख्यात्या॒ अनु॑ख्यात्या अ॒ग्नीधे॑ ददाति । \newline
17. अनु॑ख्यात्या॒ इत्यनु॑ - ख्या॒त्यै॒ । \newline
18. अ॒ग्नीधे॑ ददाति ददा त्य॒ग्नीधे॒ ऽग्नीधे॑ ददा त्य॒ग्निमु॑खा न॒ग्निमु॑खान् ददा त्य॒ग्नीधे॒ ऽग्नीधे॑ ददा त्य॒ग्निमु॑खान् । \newline
19. अ॒ग्नीध॒ इत्य॑ग्नि - इधे᳚ । \newline
20. द॒दा॒ त्य॒ग्निमु॑खा न॒ग्निमु॑खान् ददाति ददा त्य॒ग्निमु॑खा ने॒वै वाग्निमु॑खान् ददाति ददा त्य॒ग्निमु॑खा ने॒व । \newline
21. अ॒ग्निमु॑खा ने॒वै वाग्निमु॑खा न॒ग्निमु॑खा ने॒व र्‌तू नृ॒तू ने॒वाग्निमु॑खा न॒ग्निमु॑खा ने॒व र्‌तून् । \newline
22. अ॒ग्निमु॑खा॒नित्य॒ग्नि - मु॒खा॒न् । \newline
23. ए॒व र्‌तू नृ॒तू ने॒वैव र्‌तून् प्री॑णाति प्रीणा त्यृ॒तू ने॒वैव र्‌तून् प्री॑णाति । \newline
24. ऋ॒तून् प्री॑णाति प्रीणा त्यृ॒तू नृ॒तून् प्री॑णाति ब्र॒ह्मणे᳚ ब्र॒ह्मणे᳚ प्रीणा त्यृ॒तू नृ॒तून् प्री॑णाति ब्र॒ह्मणे᳚ । \newline
25. प्री॒णा॒ति॒ ब्र॒ह्मणे᳚ ब्र॒ह्मणे᳚ प्रीणाति प्रीणाति ब्र॒ह्मणे॑ ददाति ददाति ब्र॒ह्मणे᳚ प्रीणाति प्रीणाति ब्र॒ह्मणे॑ ददाति । \newline
26. ब्र॒ह्मणे॑ ददाति ददाति ब्र॒ह्मणे᳚ ब्र॒ह्मणे॑ ददाति॒ प्रसू᳚त्यै॒ प्रसू᳚त्यै ददाति ब्र॒ह्मणे᳚ ब्र॒ह्मणे॑ ददाति॒ प्रसू᳚त्यै । \newline
27. द॒दा॒ति॒ प्रसू᳚त्यै॒ प्रसू᳚त्यै ददाति ददाति॒ प्रसू᳚त्यै॒ होत्रे॒ होत्रे॒ प्रसू᳚त्यै ददाति ददाति॒ प्रसू᳚त्यै॒ होत्रे᳚ । \newline
28. प्रसू᳚त्यै॒ होत्रे॒ होत्रे॒ प्रसू᳚त्यै॒ प्रसू᳚त्यै॒ होत्रे॑ ददाति ददाति॒ होत्रे॒ प्रसू᳚त्यै॒ प्रसू᳚त्यै॒ होत्रे॑ ददाति । \newline
29. प्रसू᳚त्या॒ इति॒ प्र - सू॒त्यै॒ । \newline
30. होत्रे॑ ददाति ददाति॒ होत्रे॒ होत्रे॑ ददा त्या॒त्मा ऽऽत्मा द॑दाति॒ होत्रे॒ होत्रे॑ ददा त्या॒त्मा । \newline
31. द॒दा॒ त्या॒त्मा ऽऽत्मा द॑दाति ददा त्या॒त्मा वै वा आ॒त्मा द॑दाति ददा त्या॒त्मा वै । \newline
32. आ॒त्मा वै वा आ॒त्मा ऽऽत्मा वा ए॒ष ए॒ष वा आ॒त्मा ऽऽत्मा वा ए॒षः । \newline
33. वा ए॒ष ए॒ष वै वा ए॒ष य॒ज्ञ्स्य॑ य॒ज्ञ् स्यै॒ष वै वा ए॒ष य॒ज्ञ्स्य॑ । \newline
34. ए॒ष य॒ज्ञ्स्य॑ य॒ज्ञ् स्यै॒ष ए॒ष य॒ज्ञ्स्य॒ यद् यद् य॒ज्ञ् स्यै॒ष ए॒ष य॒ज्ञ्स्य॒ यत् । \newline
35. य॒ज्ञ्स्य॒ यद् यद् य॒ज्ञ्स्य॑ य॒ज्ञ्स्य॒ यद्धोता॒ होता॒ यद् य॒ज्ञ्स्य॑ य॒ज्ञ्स्य॒ यद्धोता᳚ । \newline
36. यद्धोता॒ होता॒ यद् यद्धोता॒ ऽऽत्मान॑ मा॒त्मानꣳ॒॒ होता॒ यद् यद्धोता॒ ऽऽत्मान᳚म् । \newline
37. होता॒ ऽऽत्मान॑ मा॒त्मानꣳ॒॒ होता॒ होता॒ ऽऽत्मान॑ मे॒वै वात्मानꣳ॒॒ होता॒ होता॒ ऽऽत्मान॑ मे॒व । \newline
38. आ॒त्मान॑ मे॒वै वात्मान॑ मा॒त्मान॑ मे॒व य॒ज्ञ्स्य॑ य॒ज्ञ् स्यै॒वात्मान॑ मा॒त्मान॑ मे॒व य॒ज्ञ्स्य॑ । \newline
39. ए॒व य॒ज्ञ्स्य॑ य॒ज्ञ् स्यै॒वैव य॒ज्ञ्स्य॒ दक्षि॑णाभि॒र् दक्षि॑णाभिर् य॒ज्ञ् स्यै॒वैव य॒ज्ञ्स्य॒ दक्षि॑णाभिः । \newline
40. य॒ज्ञ्स्य॒ दक्षि॑णाभि॒र् दक्षि॑णाभिर् य॒ज्ञ्स्य॑ य॒ज्ञ्स्य॒ दक्षि॑णाभिः॒ सꣳ सम् दक्षि॑णाभिर् य॒ज्ञ्स्य॑ य॒ज्ञ्स्य॒ दक्षि॑णाभिः॒ सम् । \newline
41. दक्षि॑णाभिः॒ सꣳ सम् दक्षि॑णाभि॒र् दक्षि॑णाभिः॒ स म॑र्द्धय त्यर्द्धयति॒ सम् दक्षि॑णाभि॒र् दक्षि॑णाभिः॒ स म॑र्द्धयति । \newline
42. स म॑र्द्धय त्यर्द्धयति॒ सꣳ स म॑र्द्धयति । \newline
43. अ॒र्द्ध॒य॒तीत्य॑र्द्धयति । \newline
\pagebreak
\markright{ TS 6.6.2.1  \hfill https://www.vedavms.in \hfill}

\section{ TS 6.6.2.1 }

\textbf{TS 6.6.2.1 } \newline
\textbf{Samhita Paata} \newline

स॒मि॒ष्ट॒ य॒जूꣳषि॑ जुहोति य॒ज्ञ्स्य॒ समि॑ष्ट्यै॒ यद्वै य॒ज्ञ्स्य॑ क्रू॒रं ॅयद्-विलि॑ष्टं॒ ॅयद॒त्येति॒ यन्नात्येति॒ यद॑तिक॒रोति॒ यन्नापि॑ क॒रोति॒ तदे॒व तैः प्री॑णाति॒ नव॑ जुहोति॒ नव॒ वै पुरु॑षे प्रा॒णाः पुरु॑षेण य॒ज्ञ्ः संमि॑तो॒ यावा॑ने॒व य॒ज्ञ्स्तं प्री॑णाति॒ षड् ऋग्मि॑याणि जुहोति॒ षड्वा ऋ॒तव॑ ऋ॒तूने॒व प्री॑णाति॒ त्रीणि॒ यजूꣳ॑षि॒- [  ] \newline

\textbf{Pada Paata} \newline

स॒मि॒ष्ट॒य॒जूꣳषीति॑ समिष्ट - य॒जूꣳषि॑ । जु॒हो॒ति॒ । य॒ज्ञ्स्य॑ । समि॑ष्ट्या॒ इति॒ सं - इ॒ष्ट्यै॒ । यत् । वै । य॒ज्ञ्स्य॑ । क्रू॒रम् । यत् । विलि॑ष्ट॒मिति॒ वि - लि॒ष्ट॒म् । यत् । अ॒त्येतीत्य॑ति - एति॑ । यत् । न । अ॒त्येतीत्य॑ति - एति॑ । यत् । अ॒ति॒क॒रोतीत्य॑ति - क॒रोति॑ । यत् । न । अपीति॑ । क॒रोति॑ । तत् । ए॒व । तैः । प्री॒णा॒ति॒ । नव॑ । जु॒हो॒ति॒ । नव॑ । वै । पुरु॑षे । प्रा॒णा इति॑ प्र - अ॒नाः । पुरु॑षेण । य॒ज्ञ्ः । सम्मि॑त॒ इति॒ सं - मि॒तः॒ । यावान्॑ । ए॒व । य॒ज्ञ्ः । तम् । प्री॒णा॒ति॒ । षट् । ऋग्मि॑याणि । जु॒हो॒ति॒ । षट् । वै । ऋ॒तवः॑ । ऋ॒तून् । ए॒व । प्री॒णा॒ति॒ । त्रीणि॑ । यजूꣳ॑षि ।  \newline


\textbf{Krama Paata} \newline

स॒मि॒ष्ट॒य॒जूꣳषि॑ जुहोति । स॒मि॒ष्ट॒य॒जूꣳषीति॑ समिष्ट - य॒जूꣳषि॑ । जु॒हो॒ति॒ य॒ज्ञ्स्य॑ । य॒ज्ञ्स्य॒ समि॑ष्ट्‍यै । समि॑ष्ट्‍यै॒ यत् । समि॑ष्ट्‍या॒ इति॒ सम् - इ॒ष्ट्‍यै॒ । यद् वै । वै य॒ज्ञ्स्य॑ । य॒ज्ञ्स्य॑ क्रू॒रम् । क्रू॒रम् ॅयत् । यद् विलि॑ष्टम् । विलि॑ष्ट॒म् ॅयत् । विलि॑ष्ट॒मिति॒ वि - लि॒ष्ट॒म् । यद॒त्येति॑ । अ॒त्येति॒ यत् । अ॒त्येतीत्य॑ति - एति॑ । यन् न । नात्येति॑ । अ॒त्येति॒ यत् । अ॒त्येतीत्य॑ति - एति॑ । यद॑तिक॒रोति॑ । अ॒ति॒क॒रोति॒ यत् । अ॒ति॒क॒रोतीत्य॑ति - क॒रोति॑ । यन् न । नापि॑ । अपि॑ क॒रोति॑ । क॒रोति॒ तत् । तदे॒व । ए॒व तैः । तैः प्री॑णाति । प्री॒णा॒ति॒ नव॑ । नव॑ जुहोति । जु॒हो॒ति॒ नव॑ । नव॒ वै । वै पुरु॑षे । पुरु॑षे प्रा॒णाः । प्रा॒णाः पुरु॑षेण । प्रा॒णा इति॑ प्र - अ॒नाः । पुरु॑षेण य॒ज्ञ्ः । य॒ज्ञ्ः सम्मि॑तः । सम्मि॑तो॒ यावान्॑ । सम्मि॑त॒ इति॒ सम् - मि॒तः॒ । यावा॑ने॒व । ए॒व य॒ज्ञ्ः । य॒ज्ञ्स्तम् । तम् प्री॑णाति । प्री॒णा॒ति॒ षट् । षडृग्मि॑याणि । ऋग्मि॑याणि जुहोति । जु॒हो॒ति॒ षट् । षड् वै । वा ऋ॒तवः॑ । ऋ॒तव॑ ऋ॒तून् । ऋ॒तूने॒व । ए॒व प्री॑णाति । प्री॒णा॒ति॒ त्रीणि॑ । त्रीणि॒ यजूꣳ॑षि । यजूꣳ॑षि॒ त्रयः॑ \newline

\textbf{Jatai Paata} \newline

1. स॒मि॒ष्ट॒य॒जूꣳषि॑ जुहोति जुहोति समिष्टय॒जूꣳषि॑ समिष्टय॒जूꣳषि॑ जुहोति । \newline
2. स॒मि॒ष्ट॒य॒जूꣳषीति॑ समिष्ट - य॒जूꣳषि॑ । \newline
3. जु॒हो॒ति॒ य॒ज्ञ्स्य॑ य॒ज्ञ्स्य॑ जुहोति जुहोति य॒ज्ञ्स्य॑ । \newline
4. य॒ज्ञ्स्य॒ समि॑ष्ट्यै॒ समि॑ष्ट्यै य॒ज्ञ्स्य॑ य॒ज्ञ्स्य॒ समि॑ष्ट्यै । \newline
5. समि॑ष्ट्यै॒ यद् यथ् समि॑ष्ट्यै॒ समि॑ष्ट्यै॒ यत् । \newline
6. समि॑ष्ट्या॒ इति॒ सं - इ॒ष्ट्यै॒ । \newline
7. यद् वै वै यद् यद् वै । \newline
8. वै य॒ज्ञ्स्य॑ य॒ज्ञ्स्य॒ वै वै य॒ज्ञ्स्य॑ । \newline
9. य॒ज्ञ्स्य॑ क्रू॒रम् क्रू॒रं ॅय॒ज्ञ्स्य॑ य॒ज्ञ्स्य॑ क्रू॒रम् । \newline
10. क्रू॒रं ॅयद् यत् क्रू॒रम् क्रू॒रं ॅयत् । \newline
11. यद् विलि॑ष्टं॒ ॅविलि॑ष्टं॒ ॅयद् यद् विलि॑ष्टम् । \newline
12. विलि॑ष्टं॒ ॅयद् यद् विलि॑ष्टं॒ ॅविलि॑ष्टं॒ ॅयत् । \newline
13. विलि॑ष्ट॒मिति॒ वि - लि॒ष्ट॒म् । \newline
14. यद॒त्ये त्य॒त्येति॒ यद् यद॒त्येति॑ । \newline
15. अ॒त्येति॒ यद् यद॒त्ये त्य॒त्येति॒ यत् । \newline
16. अ॒त्येतीत्य॑ति - एति॑ । \newline
17. यन् न न यद् यन् न । \newline
18. नात्येत्य॒ त्येति॒ न नात्येति॑ । \newline
19. अ॒त्येति॒ यद् यद॒त्ये त्य॒त्येति॒ यत् । \newline
20. अ॒त्येतीत्य॑ति - एति॑ । \newline
21. यद॑तिक॒रो त्य॑तिक॒रोति॒ यद् यद॑तिक॒रोति॑ । \newline
22. अ॒ति॒क॒रोति॒ यद् यद॑तिक॒रो त्य॑तिक॒रोति॒ यत् । \newline
23. अ॒ति॒क॒रोतीत्य॑ति - क॒रोति॑ । \newline
24. यन् न न यद् यन् न । \newline
25. नाप्यपि॒ न नापि॑ । \newline
26. अपि॑ क॒रोति॑ क॒रो त्यप्यपि॑ क॒रोति॑ । \newline
27. क॒रोति॒ तत् तत् क॒रोति॑ क॒रोति॒ तत् । \newline
28. तदे॒ वैव तत् तदे॒व । \newline
29. ए॒व तै स्तै रे॒वैव तैः । \newline
30. तैः प्री॑णाति प्रीणाति॒ तै स्तैः प्री॑णाति । \newline
31. प्री॒णा॒ति॒ नव॒ नव॑ प्रीणाति प्रीणाति॒ नव॑ । \newline
32. नव॑ जुहोति जुहोति॒ नव॒ नव॑ जुहोति । \newline
33. जु॒हो॒ति॒ नव॒ नव॑ जुहोति जुहोति॒ नव॑ । \newline
34. नव॒ वै वै नव॒ नव॒ वै । \newline
35. वै पुरु॑षे॒ पुरु॑षे॒ वै वै पुरु॑षे । \newline
36. पुरु॑षे प्रा॒णाः प्रा॒णाः पुरु॑षे॒ पुरु॑षे प्रा॒णाः । \newline
37. प्रा॒णाः पुरु॑षेण॒ पुरु॑षेण प्रा॒णाः प्रा॒णाः पुरु॑षेण । \newline
38. प्रा॒णा इति॑ प्र - अ॒नाः । \newline
39. पुरु॑षेण य॒ज्ञो य॒ज्ञ्ः पुरु॑षेण॒ पुरु॑षेण य॒ज्ञ्ः । \newline
40. य॒ज्ञ्ः सम्मि॑तः॒ सम्मि॑तो य॒ज्ञो य॒ज्ञ्ः सम्मि॑तः । \newline
41. सम्मि॑तो॒ यावा॒न्॒. यावा॒न् थ्सम्मि॑तः॒ सम्मि॑तो॒ यावान्॑ । \newline
42. सम्मि॑त॒ इति॒ सं - मि॒तः॒ । \newline
43. यावा॑ने॒ वैव यावा॒न्॒. यावा॑ने॒व । \newline
44. ए॒व य॒ज्ञो य॒ज्ञ् ए॒वैव य॒ज्ञ्ः । \newline
45. य॒ज्ञ् स्तम् तं ॅय॒ज्ञो य॒ज्ञ् स्तम् । \newline
46. तम् प्री॑णाति प्रीणाति॒ तम् तम् प्री॑णाति । \newline
47. प्री॒णा॒ति॒ षट् थ्षट् प्री॑णाति प्रीणाति॒ षट् । \newline
48. षडृग्मि॑या॒ ण्यृग्मि॑याणि॒ षट् थ्षडृग्मि॑याणि । \newline
49. ऋग्मि॑याणि जुहोति जुहो॒ त्यृग्मि॑या॒ ण्यृग्मि॑याणि जुहोति । \newline
50. जु॒हो॒ति॒ षट् थ्षड् जु॑होति जुहोति॒ षट् । \newline
51. षड् वै वै षट् थ्षड् वै । \newline
52. वा ऋ॒तव॑ ऋ॒तवो॒ वै वा ऋ॒तवः॑ । \newline
53. ऋ॒तव॑ ऋ॒तू नृ॒तू नृ॒तव॑ ऋ॒तव॑ ऋ॒तून् । \newline
54. ऋ॒तू ने॒वैव र्तू नृ॒तू ने॒व । \newline
55. ए॒व प्री॑णाति प्रीणा त्ये॒वैव प्री॑णाति । \newline
56. प्री॒णा॒ति॒ त्रीणि॒ त्रीणि॑ प्रीणाति प्रीणाति॒ त्रीणि॑ । \newline
57. त्रीणि॒ यजूꣳ॑षि॒ यजूꣳ॑षि॒ त्रीणि॒ त्रीणि॒ यजूꣳ॑षि । \newline
58. यजूꣳ॑षि॒ त्रय॒ स्त्रयो॒ यजूꣳ॑षि॒ यजूꣳ॑षि॒ त्रयः॑ । \newline

\textbf{Ghana Paata } \newline

1. स॒मि॒ष्ट॒य॒जूꣳषि॑ जुहोति जुहोति समिष्टय॒जूꣳषि॑ समिष्टय॒जूꣳषि॑ जुहोति य॒ज्ञ्स्य॑ य॒ज्ञ्स्य॑ जुहोति समिष्टय॒जूꣳषि॑ समिष्टय॒जूꣳषि॑ जुहोति य॒ज्ञ्स्य॑ । \newline
2. स॒मि॒ष्ट॒य॒जूꣳषीति॑ समिष्ट - य॒जूꣳषि॑ । \newline
3. जु॒हो॒ति॒ य॒ज्ञ्स्य॑ य॒ज्ञ्स्य॑ जुहोति जुहोति य॒ज्ञ्स्य॒ समि॑ष्ट्यै॒ समि॑ष्ट्यै य॒ज्ञ्स्य॑ जुहोति जुहोति य॒ज्ञ्स्य॒ समि॑ष्ट्यै । \newline
4. य॒ज्ञ्स्य॒ समि॑ष्ट्यै॒ समि॑ष्ट्यै य॒ज्ञ्स्य॑ य॒ज्ञ्स्य॒ समि॑ष्ट्यै॒ यद् यथ् समि॑ष्ट्यै य॒ज्ञ्स्य॑ य॒ज्ञ्स्य॒ समि॑ष्ट्यै॒ यत् । \newline
5. समि॑ष्ट्यै॒ यद् यथ् समि॑ष्ट्यै॒ समि॑ष्ट्यै॒ यद् वै वै यथ् समि॑ष्ट्यै॒ समि॑ष्ट्यै॒ यद् वै । \newline
6. समि॑ष्ट्या॒ इति॒ सं - इ॒ष्ट्यै॒ । \newline
7. यद् वै वै यद् यद् वै य॒ज्ञ्स्य॑ य॒ज्ञ्स्य॒ वै यद् यद् वै य॒ज्ञ्स्य॑ । \newline
8. वै य॒ज्ञ्स्य॑ य॒ज्ञ्स्य॒ वै वै य॒ज्ञ्स्य॑ क्रू॒रम् क्रू॒रं ॅय॒ज्ञ्स्य॒ वै वै य॒ज्ञ्स्य॑ क्रू॒रम् । \newline
9. य॒ज्ञ्स्य॑ क्रू॒रम् क्रू॒रं ॅय॒ज्ञ्स्य॑ य॒ज्ञ्स्य॑ क्रू॒रं ॅयद् यत् क्रू॒रं ॅय॒ज्ञ्स्य॑ य॒ज्ञ्स्य॑ क्रू॒रं ॅयत् । \newline
10. क्रू॒रं ॅयद् यत् क्रू॒रम् क्रू॒रं ॅयद् विलि॑ष्टं॒ ॅविलि॑ष्टं॒ ॅयत् क्रू॒रम् क्रू॒रं ॅयद् विलि॑ष्टम् । \newline
11. यद् विलि॑ष्टं॒ ॅविलि॑ष्टं॒ ॅयद् यद् विलि॑ष्टं॒ ॅयद् यद् विलि॑ष्टं॒ ॅयद् यद् विलि॑ष्टं॒ ॅयत् । \newline
12. विलि॑ष्टं॒ ॅयद् यद् विलि॑ष्टं॒ ॅविलि॑ष्टं॒ ॅयद॒त्ये त्य॒त्येति॒ यद् विलि॑ष्टं॒ ॅविलि॑ष्टं॒ ॅयद॒ त्येति॑ । \newline
13. विलि॑ष्ट॒मिति॒ वि - लि॒ष्ट॒म् । \newline
14. यद॒त्ये त्य॒त्येति॒ यद् यद॒ त्येति॒ यद् यद॒ त्येति॒ यद् यद॒ त्येति॒ यत् । \newline
15. अ॒त्येति॒ यद् यद॒त्ये त्य॒त्येति॒ यन् न न यद॒त्ये त्य॒त्येति॒ यन् न । \newline
16. अ॒त्येतीत्य॑ति - एति॑ । \newline
17. यन् न न यद् यन् नात्ये त्य॒त्येति॒ न यद् यन् नात्येति॑ । \newline
18. नात्ये त्य॒त्येति॒ न नात्येति॒ यद् यद॒ त्येति॒ न नात्येति॒ यत् । \newline
19. अ॒त्येति॒ यद् यद॒त्ये त्य॒त्येति॒ यद॑तिक॒रो त्य॑तिक॒रोति॒ यद॒त्ये त्य॒त्येति॒ यद॑तिक॒रोति॑ । \newline
20. अ॒त्येतीत्य॑ति - एति॑ । \newline
21. यद॑तिक॒रो त्य॑तिक॒रोति॒ यद् यद॑तिक॒रोति॒ यद् यद॑तिक॒रोति॒ यद् यद॑तिक॒रोति॒ यत् । \newline
22. अ॒ति॒क॒रोति॒ यद् यद॑तिक॒रो त्य॑तिक॒रोति॒ यन् न न यद॑तिक॒रो त्य॑तिक॒रोति॒ यन् न । \newline
23. अ॒ति॒क॒रोतीत्य॑ति - क॒रोति॑ । \newline
24. यन् न न यद् यन् नाप्यपि॒ न यद् यन् नापि॑ । \newline
25. नाप्यपि॒ न नापि॑ क॒रोति॑ क॒रो त्यपि॒ न नापि॑ क॒रोति॑ । \newline
26. अपि॑ क॒रोति॑ क॒रो त्यप्यपि॑ क॒रोति॒ तत् तत् क॒रो त्यप्यपि॑ क॒रोति॒ तत् । \newline
27. क॒रोति॒ तत् तत् क॒रोति॑ क॒रोति॒ तदे॒वैव तत् क॒रोति॑ क॒रोति॒ तदे॒व । \newline
28. तदे॒ वैव तत् तदे॒व तै स्तै रे॒व तत् तदे॒व तैः । \newline
29. ए॒व तै स्तै रे॒वैव तैः प्री॑णाति प्रीणाति॒ तै रे॒वैव तैः प्री॑णाति । \newline
30. तैः प्री॑णाति प्रीणाति॒ तै स्तैः प्री॑णाति॒ नव॒ नव॑ प्रीणाति॒ तै स्तैः प्री॑णाति॒ नव॑ । \newline
31. प्री॒णा॒ति॒ नव॒ नव॑ प्रीणाति प्रीणाति॒ नव॑ जुहोति जुहोति॒ नव॑ प्रीणाति प्रीणाति॒ नव॑ जुहोति । \newline
32. नव॑ जुहोति जुहोति॒ नव॒ नव॑ जुहोति॒ नव॒ नव॑ जुहोति॒ नव॒ नव॑ जुहोति॒ नव॑ । \newline
33. जु॒हो॒ति॒ नव॒ नव॑ जुहोति जुहोति॒ नव॒ वै वै नव॑ जुहोति जुहोति॒ नव॒ वै । \newline
34. नव॒ वै वै नव॒ नव॒ वै पुरु॑षे॒ पुरु॑षे॒ वै नव॒ नव॒ वै पुरु॑षे । \newline
35. वै पुरु॑षे॒ पुरु॑षे॒ वै वै पुरु॑षे प्रा॒णाः प्रा॒णाः पुरु॑षे॒ वै वै पुरु॑षे प्रा॒णाः । \newline
36. पुरु॑षे प्रा॒णाः प्रा॒णाः पुरु॑षे॒ पुरु॑षे प्रा॒णाः पुरु॑षेण॒ पुरु॑षेण प्रा॒णाः पुरु॑षे॒ पुरु॑षे प्रा॒णाः पुरु॑षेण । \newline
37. प्रा॒णाः पुरु॑षेण॒ पुरु॑षेण प्रा॒णाः प्रा॒णाः पुरु॑षेण य॒ज्ञो य॒ज्ञ्ः पुरु॑षेण प्रा॒णाः प्रा॒णाः पुरु॑षेण य॒ज्ञ्ः । \newline
38. प्रा॒णा इति॑ प्र - अ॒नाः । \newline
39. पुरु॑षेण य॒ज्ञो य॒ज्ञ्ः पुरु॑षेण॒ पुरु॑षेण य॒ज्ञ्ः सम्मि॑तः॒ सम्मि॑तो य॒ज्ञ्ः पुरु॑षेण॒ पुरु॑षेण य॒ज्ञ्ः सम्मि॑तः । \newline
40. य॒ज्ञ्ः सम्मि॑तः॒ सम्मि॑तो य॒ज्ञो य॒ज्ञ्ः सम्मि॑तो॒ यावा॒न्॒. यावा॒न् थ्सम्मि॑तो य॒ज्ञो य॒ज्ञ्ः सम्मि॑तो॒ यावान्॑ । \newline
41. सम्मि॑तो॒ यावा॒न्॒. यावा॒न् थ्सम्मि॑तः॒ सम्मि॑तो॒ यावा॑ ने॒वैव यावा॒न् थ्सम्मि॑तः॒ सम्मि॑तो॒ यावा॑ ने॒व । \newline
42. सम्मि॑त॒ इति॒ सं - मि॒तः॒ । \newline
43. यावा॑ ने॒वैव यावा॒न्॒. यावा॑ ने॒व य॒ज्ञो य॒ज्ञ् ए॒व यावा॒न्॒. यावा॑ ने॒व य॒ज्ञ्ः । \newline
44. ए॒व य॒ज्ञो य॒ज्ञ् ए॒वैव य॒ज्ञ् स्तम् तं ॅय॒ज्ञ् ए॒वैव य॒ज्ञ् स्तम् । \newline
45. य॒ज्ञ् स्तम् तं ॅय॒ज्ञो य॒ज्ञ् स्तम् प्री॑णाति प्रीणाति॒ तं ॅय॒ज्ञो य॒ज्ञ् स्तम् प्री॑णाति । \newline
46. तम् प्री॑णाति प्रीणाति॒ तम् तम् प्री॑णाति॒ षट् थ्षट् प्री॑णाति॒ तम् तम् प्री॑णाति॒ षट् । \newline
47. प्री॒णा॒ति॒ षट् थ्षट् प्री॑णाति प्रीणाति॒ षडृग्मि॑या॒ ण्यृग्मि॑याणि॒ षट् प्री॑णाति प्रीणाति॒ षडृग्मि॑याणि । \newline
48. षडृग्मि॑या॒ ण्यृग्मि॑याणि॒ षट् थ्षडृग्मि॑याणि जुहोति जुहो॒ त्यृग्मि॑याणि॒ षट् थ्षडृग्मि॑याणि जुहोति । \newline
49. ऋग्मि॑याणि जुहोति जुहो॒ त्यृग्मि॑या॒ ण्यृग्मि॑याणि जुहोति॒ षट् थ्षड् जु॑हो॒ त्यृग्मि॑या॒ ण्यृग्मि॑याणि जुहोति॒ षट् । \newline
50. जु॒हो॒ति॒ षट् थ्षड् जु॑होति जुहोति॒ षड् वै वै षड् जु॑होति जुहोति॒ षड् वै । \newline
51. षड् वै वै षट् थ्षड् वा ऋ॒तव॑ ऋ॒तवो॒ वै षट् थ्षड् वा ऋ॒तवः॑ । \newline
52. वा ऋ॒तव॑ ऋ॒तवो॒ वै वा ऋ॒तव॑ ऋ॒तू नृ॒तू नृ॒तवो॒ वै वा ऋ॒तव॑ ऋ॒तून् । \newline
53. ऋ॒तव॑ ऋ॒तू नृ॒तू नृ॒तव॑ ऋ॒तव॑ ऋ॒तू ने॒वैव र्‌तू नृ॒तव॑ ऋ॒तव॑ ऋ॒तू ने॒व । \newline
54. ऋ॒तू ने॒वैव र्‌तू नृ॒तू ने॒व प्री॑णाति प्रीणा त्ये॒व र्‌तू नृ॒तू ने॒व प्री॑णाति । \newline
55. ए॒व प्री॑णाति प्रीणा त्ये॒वैव प्री॑णाति॒ त्रीणि॒ त्रीणि॑ प्रीणा त्ये॒वैव प्री॑णाति॒ त्रीणि॑ । \newline
56. प्री॒णा॒ति॒ त्रीणि॒ त्रीणि॑ प्रीणाति प्रीणाति॒ त्रीणि॒ यजूꣳ॑षि॒ यजूꣳ॑षि॒ त्रीणि॑ प्रीणाति प्रीणाति॒ त्रीणि॒ यजूꣳ॑षि । \newline
57. त्रीणि॒ यजूꣳ॑षि॒ यजूꣳ॑षि॒ त्रीणि॒ त्रीणि॒ यजूꣳ॑षि॒ त्रय॒ स्त्रयो॒ यजूꣳ॑षि॒ त्रीणि॒ त्रीणि॒ यजूꣳ॑षि॒ त्रयः॑ । \newline
58. यजूꣳ॑षि॒ त्रय॒ स्त्रयो॒ यजूꣳ॑षि॒ यजूꣳ॑षि॒ त्रय॑ इ॒म इ॒मे त्रयो॒ यजूꣳ॑षि॒ यजूꣳ॑षि॒ त्रय॑ इ॒मे । \newline
\pagebreak
\markright{ TS 6.6.2.2  \hfill https://www.vedavms.in \hfill}

\section{ TS 6.6.2.2 }

\textbf{TS 6.6.2.2 } \newline
\textbf{Samhita Paata} \newline

त्रय॑ इ॒मे लो॒का इ॒माने॒व लो॒कान् प्री॑णाति॒ यज्ञ्॑ य॒ज्ञ्ं ग॑च्छ य॒ज्ञ्प॑तिं ग॒च्छेत्या॑ह य॒ज्ञ्प॑तिमे॒वैनं॑ गमयति॒ स्वां ॅयोनिं॑ ग॒च्छेत्या॑ह॒ स्वामे॒वैनं॒ ॅयोनिं॑ गमयत्ये॒ष ते॑ य॒ज्ञो य॑ज्ञ्पते स॒हसू᳚क्तवाकः सु॒वीर॒ इत्या॑ह॒ यज॑मान ए॒व वी॒र्यं॑ दधाति वासि॒ष्ठो ह॑ सात्यह॒व्यो दे॑वभा॒गं प॑प्रच्छ॒ यथ् सृञ्ज॑यान् बहुया॒जिनोऽयी॑यजो य॒ज्ञे- [  ] \newline

\textbf{Pada Paata} \newline

त्रयः॑ । इ॒मे । लो॒काः । इ॒मान् । ए॒व । लो॒कान् । प्री॒णा॒ति॒ । यज्ञ्॑ । य॒ज्ञ्म् । ग॒च्छ॒ । य॒ज्ञ्प॑ति॒मिति॑ य॒ज्ञ् - प॒ति॒म् । ग॒च्छ॒ । इति॑ । आ॒ह॒ । य॒ज्ञ्प॑ति॒मिति॑ य॒ज्ञ् - प॒ति॒म् । ए॒व । ए॒न॒म् । ग॒म॒य॒ति॒ । स्वाम् । योनि᳚म् । ग॒च्छ॒ । इति॑ । आ॒ह॒ । स्वाम् । ए॒व । ए॒न॒म् । योनि᳚म् । ग॒म॒य॒ति॒ । ए॒षः । ते॒ । य॒ज्ञ्ः । य॒ज्ञ्॒प॒त॒ इति॑ यज्ञ् - प॒ते॒ । स॒हसू᳚क्तवाक॒ इति॑ स॒हसू᳚क्त - वा॒कः॒ । सु॒वीर॒ इति॑ सु - वीरः॑ । इति॑ । आ॒ह॒ । यज॑माने । ए॒व । वी॒र्य᳚म् । द॒धा॒ति॒ । वा॒सि॒ष्ठः । ह॒ । सा॒त्य॒ह॒व्य इति॑ सात्य - ह॒व्यः । दे॒व॒भा॒गमिति॑ देव - भा॒गम् । प॒प्र॒च्छ॒ । यत् । सृञ्ज॑यान् । ब॒हु॒या॒जिन॒ इति॑ बहु - या॒जिनः॑ । अयी॑यजः । य॒ज्ञे ।  \newline


\textbf{Krama Paata} \newline

त्रय॑ इ॒मे । इ॒मे लो॒काः । लो॒का इ॒मान् । इ॒माने॒व । ए॒व लो॒कान् । लो॒कान् प्री॑णाति । प्री॒णा॒ति॒ यज्ञ्॑ । यज्ञ्॑ य॒ज्ञ्म् । य॒ज्ञ्म् ग॑च्छ । ग॒च्छ॒ य॒ज्ञ्प॑तिम् । य॒ज्ञ्प॑तिम् गच्छ । य॒ज्ञ्प॑ति॒मिति॑ य॒ज्ञ् - प॒ति॒म् । ग॒च्छेति॑ । इत्या॑ह । आ॒ह॒ य॒ज्ञ्प॑तिम् । य॒ज्ञ्प॑तिमे॒व । य॒ज्ञ्प॑ति॒मिति॑ य॒ज्ञ् - प॒ति॒म् । ए॒वैन᳚म् । ए॒न॒म् ग॒म॒य॒ति॒ । ग॒म॒य॒ति॒ स्वाम् । स्वाम् ॅयोनि᳚म् । योनि॑म् गच्छ । ग॒च्छेति॑ । इत्या॑ह । आ॒ह॒ स्वाम् । स्वामे॒व । ए॒वैन᳚म् । ए॒न॒म् ॅयोनि᳚म् । योनि॑म् गमयति । ग॒म॒य॒त्ये॒षः । ए॒ष ते᳚ । ते॒ य॒ज्ञ्ः । य॒ज्ञो य॑ज्ञ्पते । य॒ज्ञ्॒प॒ते॒ स॒हसू᳚क्तवाकः । य॒ज्ञ्॒प॒त॒ इति॑ यज्ञ् - प॒ते॒ । स॒हसू᳚क्तवाकः सु॒वीरः॑ । स॒हसू᳚क्तवाक॒ इति॑ स॒हसू᳚क्त - वा॒कः॒ । सु॒वीर॒ इति॑ । सु॒वीर॒ इति॑ सु - वीरः॑ । इत्या॑ह । आ॒ह॒ यज॑माने । यज॑मान ए॒व । ए॒व वी॒र्य᳚म् । वी॒र्य॑म् दधाति । द॒धा॒ति॒ वा॒सि॒ष्ठः । वा॒सि॒ष्ठो ह॑ । ह॒ सा॒त्य॒ह॒व्यः । सा॒त्य॒ह॒व्यो दे॑वभा॒गम् । सा॒त्य॒ह॒व्य इति॑ सात्य - ह॒व्यः । दे॒व॒भा॒गम् प॑प्रच्छ । दे॒व॒भा॒गमिति॑ देव - भा॒गम् । प॒प्र॒च्छ॒ यत् । यथ् सृञ्ज॑यान् । सृञ्ज॑यान् बहुया॒जिनः॑ । ब॒हु॒या॒जिनोऽयी॑यजः । ब॒हु॒या॒जिन॒ इति॑ बहु - या॒जिनः॑ । अयी॑यजो य॒ज्ञे ( ) । य॒ज्ञे य॒ज्ञ्म् \newline

\textbf{Jatai Paata} \newline

1. त्रय॑ इ॒म इ॒मे त्रय॒ स्त्रय॑ इ॒मे । \newline
2. इ॒मे लो॒का लो॒का इ॒म इ॒मे लो॒काः । \newline
3. लो॒का इ॒मा नि॒मान् ॅलो॒का लो॒का इ॒मान् । \newline
4. इ॒मा ने॒वैवे मा नि॒मा ने॒व । \newline
5. ए॒व लो॒कान् ॅलो॒का ने॒वैव लो॒कान् । \newline
6. लो॒कान् प्री॑णाति प्रीणाति लो॒कान् ॅलो॒कान् प्री॑णाति । \newline
7. प्री॒णा॒ति॒ यज्ञ्॒ यज्ञ्॑ प्रीणाति प्रीणाति॒ यज्ञ्॑ । \newline
8. यज्ञ्॑ य॒ज्ञ्ं ॅय॒ज्ञ्ं ॅयज्ञ्॒ यज्ञ्॑ य॒ज्ञ्म् । \newline
9. य॒ज्ञ्म् ग॑च्छ गच्छ य॒ज्ञ्ं ॅय॒ज्ञ्म् ग॑च्छ । \newline
10. ग॒च्छ॒ य॒ज्ञ्प॑तिं ॅय॒ज्ञ्प॑तिम् गच्छ गच्छ य॒ज्ञ्प॑तिम् । \newline
11. य॒ज्ञ्प॑तिम् गच्छ गच्छ य॒ज्ञ्प॑तिं ॅय॒ज्ञ्प॑तिम् गच्छ । \newline
12. य॒ज्ञ्प॑ति॒मिति॑ य॒ज्ञ् - प॒ति॒म् । \newline
13. ग॒च्छेतीति॑ गच्छ ग॒च्छेति॑ । \newline
14. इत्या॑हा॒हे तीत्या॑ह । \newline
15. आ॒ह॒ य॒ज्ञ्प॑तिं ॅय॒ज्ञ्प॑ति माहाह य॒ज्ञ्प॑तिम् । \newline
16. य॒ज्ञ्प॑ति मे॒वैव य॒ज्ञ्प॑तिं ॅय॒ज्ञ्प॑ति मे॒व । \newline
17. य॒ज्ञ्प॑ति॒मिति॑ य॒ज्ञ् - प॒ति॒म् । \newline
18. ए॒वैन॑ मेन मे॒वै वैन᳚म् । \newline
19. ए॒न॒म् ग॒म॒य॒ति॒ ग॒म॒य॒ त्ये॒न॒ मे॒न॒म् ग॒म॒य॒ति॒ । \newline
20. ग॒म॒य॒ति॒ स्वाꣳ स्वाम् ग॑मयति गमयति॒ स्वाम् । \newline
21. स्वां ॅयोनिं॒ ॅयोनिꣳ॒॒ स्वाꣳ स्वां ॅयोनि᳚म् । \newline
22. योनि॑म् गच्छ गच्छ॒ योनिं॒ ॅयोनि॑म् गच्छ । \newline
23. ग॒च्छेतीति॑ गच्छ ग॒च्छेति॑ । \newline
24. इत्या॑हा॒हे तीत्या॑ह । \newline
25. आ॒ह॒ स्वाꣳ स्वा मा॑हाह॒ स्वाम् । \newline
26. स्वा मे॒वैव स्वाꣳ स्वा मे॒व । \newline
27. ए॒वैन॑ मेन मे॒वै वैन᳚म् । \newline
28. ए॒नं॒ ॅयोनिं॒ ॅयोनि॑ मेन मेनं॒ ॅयोनि᳚म् । \newline
29. योनि॑म् गमयति गमयति॒ योनिं॒ ॅयोनि॑म् गमयति । \newline
30. ग॒म॒य॒ त्ये॒ष ए॒ष ग॑मयति गमय त्ये॒षः । \newline
31. ए॒ष ते॑ त ए॒ष ए॒ष ते᳚ । \newline
32. ते॒ य॒ज्ञो य॒ज्ञ् स्ते॑ ते य॒ज्ञ्ः । \newline
33. य॒ज्ञो य॑ज्ञ्पते यज्ञ्पते य॒ज्ञो य॒ज्ञो य॑ज्ञ्पते । \newline
34. य॒ज्ञ्॒प॒ते॒ स॒हसू᳚क्तवाकः स॒हसू᳚क्तवाको यज्ञ्पते यज्ञ्पते स॒हसू᳚क्तवाकः । \newline
35. य॒ज्ञ्॒प॒त॒ इति॑ यज्ञ् - प॒ते॒ । \newline
36. स॒हसू᳚क्तवाकः सु॒वीरः॑ सु॒वीरः॑ स॒हसू᳚क्तवाकः स॒हसू᳚क्तवाकः सु॒वीरः॑ । \newline
37. स॒हसू᳚क्तवाक॒ इति॑ स॒हसू᳚क्त - वा॒कः॒ । \newline
38. सु॒वीर॒ इतीति॑ सु॒वीरः॑ सु॒वीर॒ इति॑ । \newline
39. सु॒वीर॒ इति॑ सु - वीरः॑ । \newline
40. इत्या॑हा॒हे तीत्या॑ह । \newline
41. आ॒ह॒ यज॑माने॒ यज॑मान आहाह॒ यज॑माने । \newline
42. यज॑मान ए॒वैव यज॑माने॒ यज॑मान ए॒व । \newline
43. ए॒व वी॒र्यं॑ ॅवी॒र्य॑ मे॒वैव वी॒र्य᳚म् । \newline
44. वी॒र्य॑म् दधाति दधाति वी॒र्यं॑ ॅवी॒र्य॑म् दधाति । \newline
45. द॒धा॒ति॒ वा॒सि॒ष्ठो वा॑सि॒ष्ठो द॑धाति दधाति वासि॒ष्ठः । \newline
46. वा॒सि॒ष्ठो ह॑ ह वासि॒ष्ठो वा॑सि॒ष्ठो ह॑ । \newline
47. ह॒ सा॒त्य॒ह॒व्यः सा᳚त्यह॒व्यो ह॑ ह सात्यह॒व्यः । \newline
48. सा॒त्य॒ह॒व्यो दे॑वभा॒गम् दे॑वभा॒गꣳ सा᳚त्यह॒व्यः सा᳚त्यह॒व्यो दे॑वभा॒गम् । \newline
49. सा॒त्य॒ह॒व्य इति॑ सात्य - ह॒व्यः । \newline
50. दे॒व॒भा॒गम् प॑प्रच्छ पप्रच्छ देवभा॒गम् दे॑वभा॒गम् प॑प्रच्छ । \newline
51. दे॒व॒भा॒गमिति॑ देव - भा॒गम् । \newline
52. प॒प्र॒च्छ॒ यद् यत् प॑प्रच्छ पप्रच्छ॒ यत् । \newline
53. यथ् सृञ्ज॑या॒न् थ्सृञ्ज॑या॒न्॒. यद् यथ् सृञ्ज॑यान् । \newline
54. सृञ्ज॑यान् बहुया॒जिनो॑ बहुया॒जिनः॒ सृञ्ज॑या॒न् थ्सृञ्ज॑यान् बहुया॒जिनः॑ । \newline
55. ब॒हु॒या॒जिनो ऽयी॑य॒जो ऽयी॑यजो बहुया॒जिनो॑ बहुया॒जिनो ऽयी॑यजः । \newline
56. ब॒हु॒या॒जिन॒ इति॑ बहु - या॒जिनः॑ । \newline
57. अयी॑यजो य॒ज्ञे य॒ज्ञे ऽयी॑य॒जो ऽयी॑यजो य॒ज्ञे । \newline
58. य॒ज्ञे य॒ज्ञ्ं ॅय॒ज्ञ्ं ॅय॒ज्ञे य॒ज्ञे य॒ज्ञ्म् । \newline

\textbf{Ghana Paata } \newline

1. त्रय॑ इ॒म इ॒मे त्रय॒ स्त्रय॑ इ॒मे लो॒का लो॒का इ॒मे त्रय॒ स्त्रय॑ इ॒मे लो॒काः । \newline
2. इ॒मे लो॒का लो॒का इ॒म इ॒मे लो॒का इ॒मा नि॒मान् ॅलो॒का इ॒म इ॒मे लो॒का इ॒मान् । \newline
3. लो॒का इ॒मा नि॒मान् ॅलो॒का लो॒का इ॒मा ने॒वैवेमान् ॅलो॒का लो॒का इ॒मा ने॒व । \newline
4. इ॒मा ने॒वैवेमा नि॒मा ने॒व लो॒कान् ॅलो॒का ने॒वेमा नि॒मा ने॒व लो॒कान् । \newline
5. ए॒व लो॒कान् ॅलो॒का ने॒वैव लो॒कान् प्री॑णाति प्रीणाति लो॒का ने॒वैव लो॒कान् प्री॑णाति । \newline
6. लो॒कान् प्री॑णाति प्रीणाति लो॒कान् ॅलो॒कान् प्री॑णाति॒ यज्ञ्॒ यज्ञ्॑ प्रीणाति लो॒कान् ॅलो॒कान् प्री॑णाति॒ यज्ञ्॑ । \newline
7. प्री॒णा॒ति॒ यज्ञ्॒ यज्ञ्॑ प्रीणाति प्रीणाति॒ यज्ञ्॑ य॒ज्ञ्ं ॅय॒ज्ञ्ं ॅयज्ञ्॑ प्रीणाति प्रीणाति॒ यज्ञ्॑ य॒ज्ञ्म् । \newline
8. यज्ञ्॑ य॒ज्ञ्ं ॅय॒ज्ञ्ं ॅयज्ञ्॒ यज्ञ्॑ य॒ज्ञ्म् ग॑च्छ गच्छ य॒ज्ञ्ं ॅयज्ञ्॒ यज्ञ्॑ य॒ज्ञ्म् ग॑च्छ । \newline
9. य॒ज्ञ्म् ग॑च्छ गच्छ य॒ज्ञ्ं ॅय॒ज्ञ्म् ग॑च्छ य॒ज्ञ्प॑तिं ॅय॒ज्ञ्प॑तिम् गच्छ य॒ज्ञ्ं ॅय॒ज्ञ्म् ग॑च्छ य॒ज्ञ्प॑तिम् । \newline
10. ग॒च्छ॒ य॒ज्ञ्प॑तिं ॅय॒ज्ञ्प॑तिम् गच्छ गच्छ य॒ज्ञ्प॑तिम् गच्छ गच्छ य॒ज्ञ्प॑तिम् गच्छ गच्छ य॒ज्ञ्प॑तिम् गच्छ । \newline
11. य॒ज्ञ्प॑तिम् गच्छ गच्छ य॒ज्ञ्प॑तिं ॅय॒ज्ञ्प॑तिम् ग॒च्छेतीति॑ गच्छ य॒ज्ञ्प॑तिं ॅय॒ज्ञ्प॑तिम् ग॒च्छेति॑ । \newline
12. य॒ज्ञ्प॑ति॒मिति॑ य॒ज्ञ् - प॒ति॒म् । \newline
13. ग॒च्छेतीति॑ गच्छ ग॒च्छे त्या॑हा॒ हेति॑ गच्छ ग॒च्छे त्या॑ह । \newline
14. इत्या॑हा॒हे तीत्या॑ह य॒ज्ञ्प॑तिं ॅय॒ज्ञ्प॑ति मा॒हे तीत्या॑ह य॒ज्ञ्प॑तिम् । \newline
15. आ॒ह॒ य॒ज्ञ्प॑तिं ॅय॒ज्ञ्प॑ति माहाह य॒ज्ञ्प॑ति मे॒वैव य॒ज्ञ्प॑ति माहाह य॒ज्ञ्प॑ति मे॒व । \newline
16. य॒ज्ञ्प॑ति मे॒वैव य॒ज्ञ्प॑तिं ॅय॒ज्ञ्प॑ति मे॒वैन॑ मेन मे॒व य॒ज्ञ्प॑तिं ॅय॒ज्ञ्प॑ति मे॒वैन᳚म् । \newline
17. य॒ज्ञ्प॑ति॒मिति॑ य॒ज्ञ् - प॒ति॒म् । \newline
18. ए॒वैन॑ मेन मे॒वै वैन॑म् गमयति गमय त्येन मे॒वै वैन॑म् गमयति । \newline
19. ए॒न॒म् ग॒म॒य॒ति॒ ग॒म॒य॒ त्ये॒न॒ मे॒न॒म् ग॒म॒य॒ति॒ स्वाꣳ स्वाम् ग॑मय त्येन मेनम् गमयति॒ स्वाम् । \newline
20. ग॒म॒य॒ति॒ स्वाꣳ स्वाम् ग॑मयति गमयति॒ स्वां ॅयोनिं॒ ॅयोनिꣳ॒॒ स्वाम् ग॑मयति गमयति॒ स्वां ॅयोनि᳚म् । \newline
21. स्वां ॅयोनिं॒ ॅयोनिꣳ॒॒ स्वाꣳ स्वां ॅयोनि॑म् गच्छ गच्छ॒ योनिꣳ॒॒ स्वाꣳ स्वां ॅयोनि॑म् गच्छ । \newline
22. योनि॑म् गच्छ गच्छ॒ योनिं॒ ॅयोनि॑म् ग॒च्छेतीति॑ गच्छ॒ योनिं॒ ॅयोनि॑म् ग॒च्छेति॑ । \newline
23. ग॒च्छेतीति॑ गच्छ ग॒च्छे त्या॑हा॒हेति॑ गच्छ ग॒च्छे त्या॑ह । \newline
24. इत्या॑हा॒हे तीत्या॑ह॒ स्वाꣳ स्वा मा॒हे तीत्या॑ह॒ स्वाम् । \newline
25. आ॒ह॒ स्वाꣳ स्वा मा॑हाह॒ स्वा मे॒वैव स्वा मा॑हाह॒ स्वा मे॒व । \newline
26. स्वा मे॒वैव स्वाꣳ स्वा मे॒वैन॑ मेन मे॒व स्वाꣳ स्वा मे॒वैन᳚म् । \newline
27. ए॒वैन॑ मेन मे॒वै वैनं॒ ॅयोनिं॒ ॅयोनि॑ मेन मे॒वै वैनं॒ ॅयोनि᳚म् । \newline
28. ए॒नं॒ ॅयोनिं॒ ॅयोनि॑ मेन मेनं॒ ॅयोनि॑म् गमयति गमयति॒ योनि॑ मेन मेनं॒ ॅयोनि॑म् गमयति । \newline
29. योनि॑म् गमयति गमयति॒ योनिं॒ ॅयोनि॑म् गमय त्ये॒ष ए॒ष ग॑मयति॒ योनिं॒ ॅयोनि॑म् गमय त्ये॒षः । \newline
30. ग॒म॒य॒ त्ये॒ष ए॒ष ग॑मयति गमय त्ये॒ष ते॑ त ए॒ष ग॑मयति गमय त्ये॒ष ते᳚ । \newline
31. ए॒ष ते॑ त ए॒ष ए॒ष ते॑ य॒ज्ञो य॒ज्ञ् स्त॑ ए॒ष ए॒ष ते॑ य॒ज्ञ्ः । \newline
32. ते॒ य॒ज्ञो य॒ज्ञ् स्ते॑ ते य॒ज्ञो य॑ज्ञ्पते यज्ञ्पते य॒ज्ञ् स्ते॑ ते य॒ज्ञो य॑ज्ञ्पते । \newline
33. य॒ज्ञो य॑ज्ञ्पते यज्ञ्पते य॒ज्ञो य॒ज्ञो य॑ज्ञ्पते स॒हसू᳚क्तवाकः स॒हसू᳚क्तवाको यज्ञ्पते य॒ज्ञो य॒ज्ञो य॑ज्ञ्पते स॒हसू᳚क्तवाकः । \newline
34. य॒ज्ञ्॒प॒ते॒ स॒हसू᳚क्तवाकः स॒हसू᳚क्तवाको यज्ञ्पते यज्ञ्पते स॒हसू᳚क्तवाकः सु॒वीरः॑ सु॒वीरः॑ स॒हसू᳚क्तवाको यज्ञ्पते यज्ञ्पते स॒हसू᳚क्तवाकः सु॒वीरः॑ । \newline
35. य॒ज्ञ्॒प॒त॒ इति॑ यज्ञ् - प॒ते॒ । \newline
36. स॒हसू᳚क्तवाकः सु॒वीरः॑ सु॒वीरः॑ स॒हसू᳚क्तवाकः स॒हसू᳚क्तवाकः सु॒वीर॒ इतीति॑ सु॒वीरः॑ स॒हसू᳚क्तवाकः स॒हसू᳚क्तवाकः सु॒वीर॒ इति॑ । \newline
37. स॒हसू᳚क्तवाक॒ इति॑ स॒हसू᳚क्त - वा॒कः॒ । \newline
38. सु॒वीर॒ इतीति॑ सु॒वीरः॑ सु॒वीर॒ इत्या॑ हा॒हेति॑ सु॒वीरः॑ सु॒वीर॒ इत्या॑ह । \newline
39. सु॒वीर॒ इति॑ सु - वीरः॑ । \newline
40. इत्या॑हा॒हे तीत्या॑ह॒ यज॑माने॒ यज॑मान आ॒हे तीत्या॑ह॒ यज॑माने । \newline
41. आ॒ह॒ यज॑माने॒ यज॑मान आहाह॒ यज॑मान ए॒वैव यज॑मान आहाह॒ यज॑मान ए॒व । \newline
42. यज॑मान ए॒वैव यज॑माने॒ यज॑मान ए॒व वी॒र्यं॑ ॅवी॒र्य॑ मे॒व यज॑माने॒ यज॑मान ए॒व वी॒र्य᳚म् । \newline
43. ए॒व वी॒र्यं॑ ॅवी॒र्य॑ मे॒वैव वी॒र्य॑म् दधाति दधाति वी॒र्य॑ मे॒वैव वी॒र्य॑म् दधाति । \newline
44. वी॒र्य॑म् दधाति दधाति वी॒र्यं॑ ॅवी॒र्य॑म् दधाति वासि॒ष्ठो वा॑सि॒ष्ठो द॑धाति वी॒र्यं॑ ॅवी॒र्य॑म् दधाति वासि॒ष्ठः । \newline
45. द॒धा॒ति॒ वा॒सि॒ष्ठो वा॑सि॒ष्ठो द॑धाति दधाति वासि॒ष्ठो ह॑ ह वासि॒ष्ठो द॑धाति दधाति वासि॒ष्ठो ह॑ । \newline
46. वा॒सि॒ष्ठो ह॑ ह वासि॒ष्ठो वा॑सि॒ष्ठो ह॑ सात्यह॒व्यः सा᳚त्यह॒व्यो ह॑ वासि॒ष्ठो वा॑सि॒ष्ठो ह॑ सात्यह॒व्यः । \newline
47. ह॒ सा॒त्य॒ह॒व्यः सा᳚त्यह॒व्यो ह॑ ह सात्यह॒व्यो दे॑वभा॒गम् दे॑वभा॒गꣳ सा᳚त्यह॒व्यो ह॑ ह सात्यह॒व्यो दे॑वभा॒गम् । \newline
48. सा॒त्य॒ह॒व्यो दे॑वभा॒गम् दे॑वभा॒गꣳ सा᳚त्यह॒व्यः सा᳚त्यह॒व्यो दे॑वभा॒गम् प॑प्रच्छ पप्रच्छ देवभा॒गꣳ सा᳚त्यह॒व्यः सा᳚त्यह॒व्यो दे॑वभा॒गम् प॑प्रच्छ । \newline
49. सा॒त्य॒ह॒व्य इति॑ सात्य - ह॒व्यः । \newline
50. दे॒व॒भा॒गम् प॑प्रच्छ पप्रच्छ देवभा॒गम् दे॑वभा॒गम् प॑प्रच्छ॒ यद् यत् प॑प्रच्छ देवभा॒गम् दे॑वभा॒गम् प॑प्रच्छ॒ यत् । \newline
51. दे॒व॒भा॒गमिति॑ देव - भा॒गम् । \newline
52. प॒प्र॒च्छ॒ यद् यत् प॑प्रच्छ पप्रच्छ॒ यथ् सृञ्ज॑या॒न् थ्सृञ्ज॑या॒न्॒. यत् प॑प्रच्छ पप्रच्छ॒ यथ् सृञ्ज॑यान् । \newline
53. यथ् सृञ्ज॑या॒न् थ्सृञ्ज॑या॒न्॒. यद् यथ् सृञ्ज॑यान् बहुया॒जिनो॑ बहुया॒जिनः॒ सृञ्ज॑या॒न्॒. यद् यथ् सृञ्ज॑यान् बहुया॒जिनः॑ । \newline
54. सृञ्ज॑यान् बहुया॒जिनो॑ बहुया॒जिनः॒ सृञ्ज॑या॒न् थ्सृञ्ज॑यान् बहुया॒जिनो ऽयी॑य॒जो ऽयी॑यजो बहुया॒जिनः॒ सृञ्ज॑या॒न् थ्सृञ्ज॑यान् बहुया॒जिनो ऽयी॑यजः । \newline
55. ब॒हु॒या॒जिनो ऽयी॑य॒जो ऽयी॑यजो बहुया॒जिनो॑ बहुया॒जिनो ऽयी॑यजो य॒ज्ञे य॒ज्ञे ऽयी॑यजो बहुया॒जिनो॑ बहुया॒जिनो ऽयी॑यजो य॒ज्ञे । \newline
56. ब॒हु॒या॒जिन॒ इति॑ बहु - या॒जिनः॑ । \newline
57. अयी॑यजो य॒ज्ञे य॒ज्ञे ऽयी॑य॒जो ऽयी॑यजो य॒ज्ञे य॒ज्ञ्ं ॅय॒ज्ञ्ं ॅय॒ज्ञे ऽयी॑य॒जो ऽयी॑यजो य॒ज्ञे य॒ज्ञ्म् । \newline
58. य॒ज्ञे य॒ज्ञ्ं ॅय॒ज्ञ्ं ॅय॒ज्ञे य॒ज्ञे य॒ज्ञ्म् प्रति॒ प्रति॑ य॒ज्ञ्ं ॅय॒ज्ञे य॒ज्ञे य॒ज्ञ्म् प्रति॑ । \newline
\pagebreak
\markright{ TS 6.6.2.3  \hfill https://www.vedavms.in \hfill}

\section{ TS 6.6.2.3 }

\textbf{TS 6.6.2.3 } \newline
\textbf{Samhita Paata} \newline

य॒ज्ञ्ं प्रत्य॑तिष्ठि॒पा(3) य॒ज्ञ्प॒ता(3)विति॒ स हो॑वाच य॒ज्ञ्प॑ता॒विति॑ स॒त्याद्वै सृञ्ज॑याः॒ परा॑ बभूवु॒रिति॑ होवाच य॒ज्ञे वाव य॒ज्ञ्ः प्र॑ति॒ष्ठाप्य॑ आसी॒द्-यज॑मान॒स्या-प॑राभावा॒येति॒ देवा॑ गातुविदो गा॒तुं ॅवि॒त्त्वा गा॒तु -मि॒तेत्या॑ह य॒ज्ञ् ए॒व य॒ज्ञ्ं प्रति॑ ष्ठापयति॒ यज॑मान॒स्या-प॑राभावाय ॥ \newline

\textbf{Pada Paata} \newline

य॒ज्ञ्म् । प्रतीति॑ । अ॒ति॒ष्ठि॒पा(3)ः । य॒ज्ञ्प॒ता(3)विति॑ य॒ज्ञ्-प॒ता(3)उ । इति॑ । सः । ह॒ । उ॒वा॒च॒ । य॒ज्ञ्प॑ता॒विति॑ य॒ज्ञ्-प॒तौ॒ । इति॑ । स॒त्यात् । वै । सृञ्ज॑याः । परेति॑ । ब॒भू॒वुः॒ । इति॑ । ह॒ । उ॒वा॒च॒ । य॒ज्ञे । वाव । य॒ज्ञ्ः । प्र॒ति॒ष्ठाप्य॒ इति॑ प्रति - स्थाप्यः॑ । आ॒सी॒त् । यज॑मानस्य । अप॑राभावा॒येत्यप॑रा - भा॒वा॒य॒ । इति॑ । देवाः᳚ । गा॒तु॒वि॒द॒ इति॑ गातु - वि॒दः॒ । गा॒तुम् । वि॒त्त्वा । गा॒तुम् । इ॒त॒ । इति॑ । आ॒ह॒ । य॒ज्ञे । ए॒व । य॒ज्ञ्म् । प्रतीति॑ । स्था॒प॒य॒ति॒ । यज॑मानस्य । अप॑राभावा॒येत्यप॑रा- भा॒वा॒य॒ ॥  \newline


\textbf{Krama Paata} \newline

य॒ज्ञ्म् प्रति॑ । प्रत्य॑तिष्ठि॒पा(3)ः । अ॒ति॒ष्ठि॒पा(3) य॒ज्ञ्प॒ता(3)उ । य॒ज्ञ्प॒ता(3)विति॑ । य॒ज्ञ्प॒ता(3)विति॑ य॒ज्ञ् - प॒ता(3)उ । इति॒ सः । स ह॑ । हो॒वा॒च॒ । उ॒वा॒च॒ य॒ज्ञ्प॑तौ । य॒ज्ञ्प॑ता॒विति॑ । य॒ज्ञ्प॑ता॒विति॑ य॒ज्ञ् - प॒तौ॒ । इति॑ स॒त्यात् । स॒त्याद् वै । वै सृञ्ज॑याः । सृञ्ज॑याः॒ परा᳚ । परा॑ बभूवुः । ब॒भू॒वु॒रिति॑ । इति॑ ह । हो॒वा॒च॒ । उ॒वा॒च॒ य॒ज्ञे । य॒ज्ञे वाव । वाव य॒ज्ञ्ः । य॒ज्ञ्ः प्र॑ति॒ष्ठाप्यः॑ । प्र॒ति॒ष्ठाप्य॑ आसीत् । प्र॒ति॒ष्ठाप्य॒ इति॑ प्रति - स्थाप्यः॑ । आ॒सी॒द् यज॑मानस्य । यज॑मान॒स्याप॑राभावाय । अप॑राभावा॒येति॑ । अप॑राभावा॒येत्यप॑रा - भा॒वा॒य॒ । इति॒ देवाः᳚ । देवा॑ गातुविदः । गा॒तु॒वि॒दो॒ गा॒तुम् । गा॒तु॒वि॒द॒ इति॑ गातु - वि॒दः॒ । गा॒तुम् ॅवि॒त्वा । वि॒त्वा गा॒तुम् । गा॒तुमि॑त । इ॒तेति॑ । इत्या॑ह । आ॒ह॒ य॒ज्ञे । य॒ज्ञ् ए॒व । ए॒व य॒ज्ञ्म् । य॒ज्ञ्म् प्रति॑ । प्रति॑ ष्ठापयति । स्था॒प॒य॒ति॒ यज॑मानस्य । यज॑मान॒स्याप॑राभावाय । अप॑राभावा॒येत्यप॑रा - भा॒वा॒य॒ । \newline

\textbf{Jatai Paata} \newline

1. य॒ज्ञ्म् प्रति॒ प्रति॑ य॒ज्ञ्ं ॅय॒ज्ञ्म् प्रति॑ । \newline
2. प्रत्य॑तिष्ठि॒पा(3) अ॑तिष्ठि॒पा(3)ः प्रति॒ प्रत्य॑तिष्ठि॒पा(3)ः । \newline
3. अ॒ति॒ष्ठि॒पा(3) य॒ज्ञ्प॒ता(3)‌उ य॒ज्ञ्प॒ता(3)‌व॑तिष्ठि॒पा(3) अ॑तिष्ठि॒पा(3) य॒ज्ञ्प॒ता(3)‌उ । \newline
4. य॒ज्ञ्प॒ता(3)‌वितीति॑ य॒ज्ञ्प॒ता(3)‌उ य॒ज्ञ्प॒ता(3)‌विति॑ । \newline
5. य॒ज्ञ्प॒ता(3)विति॑ य॒ज्ञ् - प॒ता(3)उ । \newline
6. इति॒ स स इतीति॒ सः । \newline
7. स ह॑ ह॒ स स ह॑ । \newline
8. हो॒वा॒चो॒ वा॒च॒ ह॒ हो॒वा॒च॒ । \newline
9. उ॒वा॒च॒ य॒ज्ञ्प॑तौ य॒ज्ञ्प॑ता वुवा चोवाच य॒ज्ञ्प॑तौ । \newline
10. य॒ज्ञ्प॑ता॒ वितीति॑ य॒ज्ञ्प॑तौ य॒ज्ञ्प॑ता॒ विति॑ । \newline
11. य॒ज्ञ्प॑ता॒विति॑ य॒ज्ञ् - प॒तौ॒ । \newline
12. इति॑ स॒त्याथ् स॒त्या दितीति॑ स॒त्यात् । \newline
13. स॒त्याद् वै वै स॒त्याथ् स॒त्याद् वै । \newline
14. वै सृञ्ज॑याः॒ सृञ्ज॑या॒ वै वै सृञ्ज॑याः । \newline
15. सृञ्ज॑याः॒ परा॒ परा॒ सृञ्ज॑याः॒ सृञ्ज॑याः॒ परा᳚ । \newline
16. परा॑ बभूवुर् बभूवुः॒ परा॒ परा॑ बभूवुः । \newline
17. ब॒भू॒वु॒ रितीति॑ बभूवुर् बभूवु॒ रिति॑ । \newline
18. इति॑ ह॒ हेतीति॑ ह । \newline
19. हो॒वा॒ चो॒वा॒च॒ ह॒ हो॒वा॒च॒ । \newline
20. उ॒वा॒च॒ य॒ज्ञे य॒ज्ञ् उ॑वा चोवाच य॒ज्ञे । \newline
21. य॒ज्ञे वाव वाव य॒ज्ञे य॒ज्ञे वाव । \newline
22. वाव य॒ज्ञो य॒ज्ञो वाव वाव य॒ज्ञ्ः । \newline
23. य॒ज्ञ्ः प्र॑ति॒ष्ठाप्यः॑ प्रति॒ष्ठाप्यो॑ य॒ज्ञो य॒ज्ञ्ः प्र॑ति॒ष्ठाप्यः॑ । \newline
24. प्र॒ति॒ष्ठाप्य॑ आसीदासीत् प्रति॒ष्ठाप्यः॑ प्रति॒ष्ठाप्य॑ आसीत् । \newline
25. प्र॒ति॒ष्ठाप्य॒ इति॑ प्रति - स्थाप्यः॑ । \newline
26. आ॒सी॒द् यज॑मानस्य॒ यज॑मान स्यासी दासी॒द् यज॑मानस्य । \newline
27. यज॑मान॒स्या प॑राभावा॒या प॑राभावाय॒ यज॑मानस्य॒ यज॑मान॒स्या प॑राभावाय । \newline
28. अप॑राभावा॒ये तीत्यप॑राभावा॒या प॑राभावा॒येति॑ । \newline
29. अप॑राभावा॒येत्यप॑रा - भा॒वा॒य॒ । \newline
30. इति॒ देवा॒ देवा॒ इतीति॒ देवाः᳚ । \newline
31. देवा॑ गातुविदो गातुविदो॒ देवा॒ देवा॑ गातुविदः । \newline
32. गा॒तु॒वि॒दो॒ गा॒तुम् गा॒तुम् गा॑तुविदो गातुविदो गा॒तुम् । \newline
33. गा॒तु॒वि॒द॒ इति॑ गातु - वि॒दः॒ । \newline
34. गा॒तुं ॅवि॒त्त्वा वि॒त्त्वा गा॒तुम् गा॒तुं ॅवि॒त्त्वा । \newline
35. वि॒त्त्वा गा॒तुम् गा॒तुं ॅवि॒त्त्वा वि॒त्त्वा गा॒तुम् । \newline
36. गा॒तु मि॑तेत गा॒तुम् गा॒तु मि॑त । \newline
37. इ॒ते तीती॑ते॒ तेति॑ । \newline
38. इत्या॑हा॒हे तीत्या॑ह । \newline
39. आ॒ह॒ य॒ज्ञे य॒ज्ञ् आ॑हाह य॒ज्ञे । \newline
40. य॒ज्ञ् ए॒वैव य॒ज्ञे य॒ज्ञ् ए॒व । \newline
41. ए॒व य॒ज्ञ्ं ॅय॒ज्ञ् मे॒वैव य॒ज्ञ्म् । \newline
42. य॒ज्ञ्म् प्रति॒ प्रति॑ य॒ज्ञ्ं ॅय॒ज्ञ्म् प्रति॑ । \newline
43. प्रति॑ ष्ठापयति स्थापयति॒ प्रति॒ प्रति॑ ष्ठापयति । \newline
44. स्था॒प॒य॒ति॒ यज॑मानस्य॒ यज॑मानस्य स्थापयति स्थापयति॒ यज॑मानस्य । \newline
45. यज॑मान॒स्या प॑राभावा॒या प॑राभावाय॒ यज॑मानस्य॒ यज॑मान॒स्या प॑राभावाय । \newline
46. अप॑राभावा॒येत्यप॑रा - भा॒वा॒य॒ । \newline

\textbf{Ghana Paata } \newline

1. य॒ज्ञ्म् प्रति॒ प्रति॑ य॒ज्ञ्ं ॅय॒ज्ञ्म् प्रत्य॑तिष्ठि॒पा(3) अ॑तिष्ठि॒पा(3)ः प्रति॑ य॒ज्ञ्ं ॅय॒ज्ञ्म् प्रत्य॑तिष्ठि॒पा(3)ः । \newline
2. प्रत्य॑तिष्ठि॒पा(3) अ॑तिष्ठि॒पा(3)ः प्रति॒ प्रत्य॑तिष्ठि॒पा(3) य॒ज्ञ्प॒ता(3) ‌उ य॒ज्ञ्प॒ता(3) व॑तिष्ठि॒पा(3)ः प्रति॒ प्रत्य॑तिष्ठि॒पा(3) य॒ज्ञ्प॒ता(3) ‌उ । \newline
3. अ॒ति॒ष्ठि॒पा(3) य॒ज्ञ्प॒ता(3) ‌उ य॒ज्ञ्प॒ता(3)‌ व॑तिष्ठि॒पा(3) अ॑तिष्ठि॒पा(3) य॒ज्ञ्प॒ता(3) ‌वितीति॑ य॒ज्ञ्प॒ता(3) ‌व॑तिष्ठि॒पा(3) अ॑तिष्ठि॒पा(3) य॒ज्ञ्प॒ता(3) ‌विति॑ । \newline
4. य॒ज्ञ्प॒ता(3) ‌वितीति॑ य॒ज्ञ्प॒ता(3) ‌उ य॒ज्ञ्प॒ता(3) ‌विति॒ स स इति॑ य॒ज्ञ्प॒ता(3) ‌उ य॒ज्ञ्प॒ता(3) ‌विति॒ सः । \newline
5. य॒ज्ञ्प॒ता(3)विति॑ य॒ज्ञ् - प॒ता(3) उ । \newline
6. इति॒ स स इतीति॒ स ह॑ ह॒ स इतीति॒ स ह॑ । \newline
7. स ह॑ ह॒ स स हो॑वा चोवाच ह॒ स स हो॑वाच । \newline
8. हो॒वा॒ चो॒वा॒च॒ ह॒ हो॒वा॒च॒ य॒ज्ञ्प॑तौ य॒ज्ञ्प॑ता वुवाच ह होवाच य॒ज्ञ्प॑तौ । \newline
9. उ॒वा॒च॒ य॒ज्ञ्प॑तौ य॒ज्ञ्प॑ता वुवा चोवाच य॒ज्ञ्प॑ता॒ वितीति॑ य॒ज्ञ्प॑ता वुवा चोवाच य॒ज्ञ्प॑ता॒ विति॑ । \newline
10. य॒ज्ञ्प॑ता॒ वितीति॑ य॒ज्ञ्प॑तौ य॒ज्ञ्प॑ता॒ विति॑ स॒त्याथ् स॒त्यादिति॑ य॒ज्ञ्प॑तौ य॒ज्ञ्प॑ता॒ विति॑ स॒त्यात् । \newline
11. य॒ज्ञ्प॑ता॒विति॑ य॒ज्ञ् - प॒तौ॒ । \newline
12. इति॑ स॒त्याथ् स॒त्या दितीति॑ स॒त्याद् वै वै स॒त्या दितीति॑ स॒त्याद् वै । \newline
13. स॒त्याद् वै वै स॒त्याथ् स॒त्याद् वै सृञ्ज॑याः॒ सृञ्ज॑या॒ वै स॒त्याथ् स॒त्याद् वै सृञ्ज॑याः । \newline
14. वै सृञ्ज॑याः॒ सृञ्ज॑या॒ वै वै सृञ्ज॑याः॒ परा॒ परा॒ सृञ्ज॑या॒ वै वै सृञ्ज॑याः॒ परा᳚ । \newline
15. सृञ्ज॑याः॒ परा॒ परा॒ सृञ्ज॑याः॒ सृञ्ज॑याः॒ परा॑ बभूवुर् बभूवुः॒ परा॒ सृञ्ज॑याः॒ सृञ्ज॑याः॒ परा॑ बभूवुः । \newline
16. परा॑ बभूवुर् बभूवुः॒ परा॒ परा॑ बभूवु॒ रितीति॑ बभूवुः॒ परा॒ परा॑ बभूवु॒ रिति॑ । \newline
17. ब॒भू॒वु॒ रितीति॑ बभूवुर् बभूवु॒ रिति॑ ह॒ हेति॑ बभूवुर् बभूवु॒ रिति॑ ह । \newline
18. इति॑ ह॒ हेतीति॑ होवाचो वाच॒ हेतीति॑ होवाच । \newline
19. हो॒वा॒चो॒ वा॒च॒ ह॒ हो॒वा॒च॒ य॒ज्ञे य॒ज्ञ् उ॑वाच ह होवाच य॒ज्ञे । \newline
20. उ॒वा॒च॒ य॒ज्ञे य॒ज्ञ् उ॑वाचो वाच य॒ज्ञे वाव वाव य॒ज्ञ् उ॑वाचो वाच य॒ज्ञे वाव । \newline
21. य॒ज्ञे वाव वाव य॒ज्ञे य॒ज्ञे वाव य॒ज्ञो य॒ज्ञो वाव य॒ज्ञे य॒ज्ञे वाव य॒ज्ञ्ः । \newline
22. वाव य॒ज्ञो य॒ज्ञो वाव वाव य॒ज्ञ्ः प्र॑ति॒ष्ठाप्यः॑ प्रति॒ष्ठाप्यो॑ य॒ज्ञो वाव वाव य॒ज्ञ्ः प्र॑ति॒ष्ठाप्यः॑ । \newline
23. य॒ज्ञ्ः प्र॑ति॒ष्ठाप्यः॑ प्रति॒ष्ठाप्यो॑ य॒ज्ञो य॒ज्ञ्ः प्र॑ति॒ष्ठाप्य॑ आसी दासीत् प्रति॒ष्ठाप्यो॑ य॒ज्ञो य॒ज्ञ्ः प्र॑ति॒ष्ठाप्य॑ आसीत् । \newline
24. प्र॒ति॒ष्ठाप्य॑ आसी दासीत् प्रति॒ष्ठाप्यः॑ प्रति॒ष्ठाप्य॑ आसी॒द् यज॑मानस्य॒ यज॑मान स्यासीत् प्रति॒ष्ठाप्यः॑ प्रति॒ष्ठाप्य॑ आसी॒द् यज॑मानस्य । \newline
25. प्र॒ति॒ष्ठाप्य॒ इति॑ प्रति - स्थाप्यः॑ । \newline
26. आ॒सी॒द् यज॑मानस्य॒ यज॑मान स्यासी दासी॒द् यज॑मान॒स्या प॑राभावा॒या प॑राभावाय॒ यज॑मान स्यासी दासी॒द् यज॑मान॒स्या प॑राभावाय । \newline
27. यज॑मान॒स्या प॑राभावा॒या प॑राभावाय॒ यज॑मानस्य॒ यज॑मान॒स्या प॑राभावा॒येती त्यप॑राभावाय॒ यज॑मानस्य॒ यज॑मान॒स्या प॑राभावा॒येति॑ । \newline
28. अप॑राभावा॒ये तीत्यप॑राभावा॒या प॑राभावा॒ येति॒ देवा॒ देवा॒ इत्यप॑राभावा॒या प॑राभावा॒ येति॒ देवाः᳚ । \newline
29. अप॑राभावा॒येत्यप॑रा - भा॒वा॒य॒ । \newline
30. इति॒ देवा॒ देवा॒ इतीति॒ देवा॑ गातुविदो गातुविदो॒ देवा॒ इतीति॒ देवा॑ गातुविदः । \newline
31. देवा॑ गातुविदो गातुविदो॒ देवा॒ देवा॑ गातुविदो गा॒तुम् गा॒तुम् गा॑तुविदो॒ देवा॒ देवा॑ गातुविदो गा॒तुम् । \newline
32. गा॒तु॒वि॒दो॒ गा॒तुम् गा॒तुम् गा॑तुविदो गातुविदो गा॒तुं ॅवि॒त्त्वा वि॒त्त्वा गा॒तुम् गा॑तुविदो गातुविदो गा॒तुं ॅवि॒त्त्वा । \newline
33. गा॒तु॒वि॒द॒ इति॑ गातु - वि॒दः॒ । \newline
34. गा॒तुं ॅवि॒त्त्वा वि॒त्त्वा गा॒तुम् गा॒तुं ॅवि॒त्त्वा गा॒तुम् गा॒तुं ॅवि॒त्त्वा गा॒तुम् गा॒तुं ॅवि॒त्त्वा गा॒तुम् । \newline
35. वि॒त्त्वा गा॒तुम् गा॒तुं ॅवि॒त्त्वा वि॒त्त्वा गा॒तु मि॑तेत गा॒तुं ॅवि॒त्त्वा वि॒त्त्वा गा॒तु मि॑त । \newline
36. गा॒तु मि॑तेत गा॒तुम् गा॒तु मि॒ते तीती॑त गा॒तुम् गा॒तु मि॒तेति॑ । \newline
37. इ॒ते तीती॑ ते॒ते त्या॑हा॒ हेती॑ ते॒ते त्या॑ह । \newline
38. इत्या॑हा॒हे तीत्या॑ह य॒ज्ञे य॒ज्ञ् आ॒हे तीत्या॑ह य॒ज्ञे । \newline
39. आ॒ह॒ य॒ज्ञे य॒ज्ञ् आ॑हाह य॒ज्ञ् ए॒वैव य॒ज्ञ् आ॑हाह य॒ज्ञ् ए॒व । \newline
40. य॒ज्ञ् ए॒वैव य॒ज्ञे य॒ज्ञ् ए॒व य॒ज्ञ्ं ॅय॒ज्ञ् मे॒व य॒ज्ञे य॒ज्ञ् ए॒व य॒ज्ञ्म् । \newline
41. ए॒व य॒ज्ञ्ं ॅय॒ज्ञ् मे॒वैव य॒ज्ञ्म् प्रति॒ प्रति॑ य॒ज्ञ् मे॒वैव य॒ज्ञ्म् प्रति॑ । \newline
42. य॒ज्ञ्म् प्रति॒ प्रति॑ य॒ज्ञ्ं ॅय॒ज्ञ्म् प्रति॑ ष्ठापयति स्थापयति॒ प्रति॑ य॒ज्ञ्ं ॅय॒ज्ञ्म् प्रति॑ ष्ठापयति । \newline
43. प्रति॑ ष्ठापयति स्थापयति॒ प्रति॒ प्रति॑ ष्ठापयति॒ यज॑मानस्य॒ यज॑मानस्य स्थापयति॒ प्रति॒ प्रति॑ ष्ठापयति॒ यज॑मानस्य । \newline
44. स्था॒प॒य॒ति॒ यज॑मानस्य॒ यज॑मानस्य स्थापयति स्थापयति॒ यज॑मान॒स्या प॑राभावा॒या प॑राभावाय॒ यज॑मानस्य स्थापयति स्थापयति॒ यज॑मान॒स्या प॑राभावाय । \newline
45. यज॑मान॒स्या प॑राभावा॒या प॑राभावाय॒ यज॑मानस्य॒ यज॑मान॒स्या प॑राभावाय । \newline
46. अप॑राभावा॒येत्यप॑रा - भा॒वा॒य॒ । \newline
\pagebreak
\markright{ TS 6.6.3.1  \hfill https://www.vedavms.in \hfill}

\section{ TS 6.6.3.1 }

\textbf{TS 6.6.3.1 } \newline
\textbf{Samhita Paata} \newline

अ॒व॒भृ॒थ॒-य॒जूꣳषि॑ जुहोति॒ यदे॒वार्वा॒चीन॒-मेक॑हायना॒देनः॑ क॒रोति॒ तदे॒व तैरव॑ यजते॒ ऽपो॑ऽवभृ॒थ-मवै᳚त्य॒फ्सु वै वरु॑णः सा॒क्षादे॒व वरु॑ण॒मव॑ यजते॒ वर्त्म॑ना॒ वा अ॒न्वित्य॑ य॒ज्ञ्ꣳ रक्षाꣳ॑सि जिघाꣳसन्ति॒ साम्ना᳚ प्रस्तो॒ताऽन्ववै॑ति॒ साम॒ वै र॑क्षो॒हा रक्ष॑सा॒मप॑हत्यै॒ त्रिर्नि॒धन॒मुपै॑ति॒ त्रय॑ इ॒मे लो॒का ए॒भ्य ए॒व लो॒केभ्यो॒ रक्षाꣳ॒॒- [  ] \newline

\textbf{Pada Paata} \newline

अ॒व॒भृ॒थ॒य॒जूꣳषीत्य॑वभृथ - य॒जूꣳषि॑ । जु॒हो॒ति॒ । यत् । ए॒व । अ॒र्वा॒चीन᳚म् । एक॑हायना॒दित्येक॑ - हा॒य॒ना॒त् । एनः॑ । क॒रोति॑ । तत् । ए॒व । तैः । अवेति॑ । य॒ज॒ते॒ । अ॒पः । अ॒व॒भृ॒थमित्य॑व - भृ॒थम् । अवेति॑ । ए॒ति॒ । अ॒फ्स्वित्य॑प् - सु । वै । वरु॑णः । सा॒क्षादिति॑ स - अ॒क्षात् । ए॒व । वरु॑णम् । अवेति॑ । य॒ज॒ते॒ । वर्त्म॑ना । वै । अ॒न्वित्येत्य॑नु - इत्य॑ । य॒ज्ञ्म् । रक्षाꣳ॑सि । जि॒घाꣳ॒॒स॒न्त॒ । साम्ना᳚ । प्र॒स्तो॒तेति॑ प्र - स्तो॒ता । अ॒न्ववै॒तीत्य॑नु - अवै॑ति । साम॑ । वै । र॒क्षो॒हेति॑ रक्षः - हा । रक्ष॑साम् । अप॑हत्या॒ इत्यप॑ - ह॒त्यै॒ । त्रिः । नि॒धन॒मिति॑ नि - धन᳚म् । उपेति॑ । ए॒ति॒ । त्रयः॑ । इ॒मे । लो॒काः । ए॒भ्यः । ए॒व । लो॒केभ्यः॑ । रक्षाꣳ॑सि ।  \newline


\textbf{Krama Paata} \newline

अ॒व॒भृ॒थ॒य॒जूꣳषि॑ जुहोति । अ॒व॒भृ॒थ॒य॒जूꣳषीत्य॑वभृथ - य॒जूꣳषि॑ । जु॒हो॒ति॒ यत् । यदे॒व । ए॒वार्वा॒चीन᳚म् । अ॒र्वा॒चीन॒,मेक॑हायनात् । एक॑हायना॒देनः॑ । एक॑हायना॒दित्येक॑ - हा॒य॒ना॒त्॒ । एनः॑ क॒रोति॑ । क॒रोति॒ तत् । तदे॒व । ए॒व तैः । तैरव॑ । अव॑ यजते । य॒ज॒ते॒ऽपः । अ॒पो॑ऽवभृ॒थम् । अ॒व॒भृ॒थमव॑ । अ॒व॒भृ॒थमित्य॑व - भृ॒थम् । अवै॑ति । ए॒त्य॒फ्सु । अ॒फ्सु वै । अ॒फ्स्वित्य॑प् - सु । वै वरु॑णः । वरु॑णः सा॒क्षात् । सा॒क्षादे॒व । सा॒क्षादिति॑ स - अ॒क्षात् । ए॒व वरु॑णम् । वरु॑ण॒मव॑ । अव॑ यजते । य॒ज॒ते॒ वर्त्म॑ना । वर्त्म॑ना॒ वै । वा अ॒न्वित्य॑ । अ॒न्वित्य॑ य॒ज्ञ्म् । अ॒न्वित्येत्य॑नु - इत्य॑ । य॒ज्ञ्ꣳ रक्षाꣳ॑सि । रक्षाꣳ॑सि जिघाꣳसन्ति । जि॒घाꣳ॒॒स॒न्ति॒ साम्ना᳚ । साम्ना᳚ प्रस्तो॒ता । प्र॒स्तो॒ताऽन्ववै॑ति । प्र॒स्तो॒तेति॑ प्र - स्तो॒ता । अ॒न्ववै॑ति॒ साम॑ । अ॒न्ववै॒तीत्य॑नु - अवै॑ति । साम॒ वै । वै र॑क्षो॒हा । र॒क्षो॒हा रक्ष॑साम् । र॒क्षो॒हेति॑ रक्षः - हा । रक्ष॑सा॒मप॑हत्यै । अप॑हत्यै॒ त्रिः । अप॑हत्या॒ इत्यप॑ - ह॒त्यै॒ । त्रिर् नि॒धन᳚म् । नि॒धन॒मुप॑ । नि॒धन॒मिति॑ नि - धन᳚म् । उपै॑ति । ए॒ति॒ त्रयः॑ । त्रय॑ इ॒मे । इ॒मे लो॒काः । लो॒का ए॒भ्यः । ए॒भ्य ए॒व । ए॒व लो॒केभ्यः॑ । लो॒केभ्यो॒ रक्षाꣳ॑सि । रक्षाꣳ॒॒स्यप॑ \newline

\textbf{Jatai Paata} \newline

1. अ॒व॒भृ॒थ॒य॒जूꣳषि॑ जुहोति जुहो त्यवभृथय॒जूꣳ ष्य॑वभृथय॒जूꣳषि॑ जुहोति । \newline
2. अ॒व॒भृ॒थ॒य॒जूꣳषीत्य॑वभृथ - य॒जूꣳषि॑ । \newline
3. जु॒हो॒ति॒ यद् यज् जु॑होति जुहोति॒ यत् । \newline
4. यदे॒ वैव यद् यदे॒व । \newline
5. ए॒वार्वा॒चीन॑ मर्वा॒चीन॑ मे॒वै वार्वा॒चीन᳚म् । \newline
6. अ॒र्वा॒चीन॒ मेक॑हायना॒ देक॑हायना दर्वा॒चीन॑ मर्वा॒चीन॒ मेक॑हायनात् । \newline
7. एक॑हायना॒ देन॒ एन॒ एक॑हायना॒ देक॑हायना॒ देनः॑ । \newline
8. एक॑हायना॒दित्येक॑ - हा॒य॒ना॒त् । \newline
9. एनः॑ क॒रोति॑ क॒रो त्येन॒ एनः॑ क॒रोति॑ । \newline
10. क॒रोति॒ तत् तत् क॒रोति॑ क॒रोति॒ तत् । \newline
11. तदे॒ वैव तत् तदे॒व । \newline
12. ए॒व तै स्तै रे॒वैव तैः । \newline
13. तैर वाव॒ तै स्तै रव॑ । \newline
14. अव॑ यजते यज॒ते ऽवाव॑ यजते । \newline
15. य॒ज॒ते॒ ऽपो॑ ऽपो य॑जते यजते॒ ऽपः । \newline
16. अ॒पो॑ ऽवभृ॒थ म॑वभृ॒थ म॒पो᳚(1॒) ऽपो॑ ऽवभृ॒थम् । \newline
17. अ॒व॒भृ॒थ मवावा॑ वभृ॒थ म॑वभृ॒थ मव॑ । \newline
18. अ॒व॒भृ॒थमित्य॑व - भृ॒थम् । \newline
19. अवै᳚त्ये॒ त्यवा वै॑ति । \newline
20. ए॒त्य॒ फ्स्वा᳚(1॒) फ्स्वे᳚ त्येत्य॒फ्सु । \newline
21. अ॒फ्सु वै वा अ॒फ्स्व॑फ्सु वै । \newline
22. अ॒फ्स्वित्य॑प् - सु । \newline
23. वै वरु॑णो॒ वरु॑णो॒ वै वै वरु॑णः । \newline
24. वरु॑णः सा॒क्षाथ् सा॒क्षाद् वरु॑णो॒ वरु॑णः सा॒क्षात् । \newline
25. सा॒क्षा दे॒वैव सा॒क्षाथ् सा॒क्षा दे॒व । \newline
26. सा॒क्षादिति॑ स - अ॒क्षात् । \newline
27. ए॒व वरु॑णं॒ ॅवरु॑ण मे॒वैव वरु॑णम् । \newline
28. वरु॑ण॒ मवाव॒ वरु॑णं॒ ॅवरु॑ण॒ मव॑ । \newline
29. अव॑ यजते यज॒ते ऽवाव॑ यजते । \newline
30. य॒ज॒ते॒ वर्त्म॑ना॒ वर्त्म॑ना यजते यजते॒ वर्त्म॑ना । \newline
31. वर्त्म॑ना॒ वै वै वर्त्म॑ना॒ वर्त्म॑ना॒ वै । \newline
32. वा अ॒न्वित्या॒ न्वित्य॒ वै वा अ॒न्वित्य॑ । \newline
33. अ॒न्वित्य॑ य॒ज्ञ्ं ॅय॒ज्ञ् म॒न्वित्या॒ न्वित्य॑ य॒ज्ञ्म् । \newline
34. अ॒न्वित्येत्य॑नु - इत्य॑ । \newline
35. य॒ज्ञ्ꣳ रक्षाꣳ॑सि॒ रक्षाꣳ॑सि य॒ज्ञ्ं ॅय॒ज्ञ्ꣳ रक्षाꣳ॑सि । \newline
36. रक्षाꣳ॑सि जिघाꣳसन्ति जिघाꣳसन्ति॒ रक्षाꣳ॑सि॒ रक्षाꣳ॑सि जिघाꣳसन्ति । \newline
37. जि॒घाꣳ॒॒स॒न्ति॒ साम्ना॒ साम्ना॑ जिघाꣳसन्ति जिघाꣳसन्ति॒ साम्ना᳚ । \newline
38. साम्ना᳚ प्रस्तो॒ता प्र॑स्तो॒ता साम्ना॒ साम्ना᳚ प्रस्तो॒ता । \newline
39. प्र॒स्तो॒ता ऽन्ववै᳚ त्य॒न्ववै॑ति प्रस्तो॒ता प्र॑स्तो॒ता ऽन्ववै॑ति । \newline
40. प्र॒स्तो॒तेति॑ प्र - स्तो॒ता । \newline
41. अ॒न्ववै॑ति॒ साम॒ सामा॒न्ववै᳚ त्य॒न्ववै॑ति॒ साम॑ । \newline
42. अ॒न्ववै॒तीत्य॑नु - अवै॑ति । \newline
43. साम॒ वै वै साम॒ साम॒ वै । \newline
44. वै र॑क्षो॒हा र॑क्षो॒हा वै वै र॑क्षो॒हा । \newline
45. र॒क्षो॒हा रक्ष॑साꣳ॒॒ रक्ष॑साꣳ रक्षो॒हा र॑क्षो॒हा रक्ष॑साम् । \newline
46. र॒क्षो॒हेति॑ रक्षः - हा । \newline
47. रक्ष॑सा॒ मप॑हत्या॒ अप॑हत्यै॒ रक्ष॑साꣳ॒॒ रक्ष॑सा॒ मप॑हत्यै । \newline
48. अप॑हत्यै॒ त्रि स्त्रि रप॑हत्या॒ अप॑हत्यै॒ त्रिः । \newline
49. अप॑हत्या॒ इत्यप॑ - ह॒त्यै॒ । \newline
50. त्रिर् नि॒धन॑न् नि॒धन॒म् त्रि स्त्रिर् नि॒धन᳚म् । \newline
51. नि॒धन॒ मुपोप॑ नि॒धन॑न् नि॒धन॒ मुप॑ । \newline
52. नि॒धन॒मिति॑ नि - धन᳚म् । \newline
53. उपै᳚त्ये॒ त्युपोपै॑ति । \newline
54. ए॒ति॒ त्रय॒ स्त्रय॑ एत्येति॒ त्रयः॑ । \newline
55. त्रय॑ इ॒म इ॒मे त्रय॒ स्त्रय॑ इ॒मे । \newline
56. इ॒मे लो॒का लो॒का इ॒म इ॒मे लो॒काः । \newline
57. लो॒का ए॒भ्य ए॒भ्यो लो॒का लो॒का ए॒भ्यः । \newline
58. ए॒भ्य ए॒वै वैभ्य ए॒भ्य ए॒व । \newline
59. ए॒व लो॒केभ्यो॑ लो॒केभ्य॑ ए॒वैव लो॒केभ्यः॑ । \newline
60. लो॒केभ्यो॒ रक्षाꣳ॑सि॒ रक्षाꣳ॑सि लो॒केभ्यो॑ लो॒केभ्यो॒ रक्षाꣳ॑सि । \newline
61. रक्षाꣳ॒॒ स्यपाप॒ रक्षाꣳ॑सि॒ रक्षाꣳ॒॒ स्यप॑ । \newline

\textbf{Ghana Paata } \newline

1. अ॒व॒भृ॒थ॒य॒जूꣳषि॑ जुहोति जुहो त्यवभृथय॒जूꣳ ष्य॑वभृथय॒जूꣳषि॑ जुहोति॒ यद् यज् जु॑हो
त्यवभृथय॒जूꣳ ष्य॑वभृथय॒जूꣳषि॑ जुहोति॒ यत् । \newline
2. अ॒व॒भृ॒थ॒य॒जूꣳषीत्य॑वभृथ - य॒जूꣳषि॑ । \newline
3. जु॒हो॒ति॒ यद् यज् जु॑होति जुहोति॒ यदे॒ वैव यज् जु॑होति जुहोति॒ यदे॒व । \newline
4. यदे॒वैव यद् यदे॒वार्वा॒चीन॑ मर्वा॒चीन॑ मे॒व यद् यदे॒वार्वा॒चीन᳚म् । \newline
5. ए॒वार्वा॒चीन॑ मर्वा॒चीन॑ मे॒वै वार्वा॒चीन॒ मेक॑हायना॒ देक॑हायना दर्वा॒चीन॑ मे॒वै वार्वा॒चीन॒ मेक॑हायनात् । \newline
6. अ॒र्वा॒चीन॒ मेक॑हायना॒ देक॑हायना दर्वा॒चीन॑ मर्वा॒चीन॒ मेक॑हायना॒ देन॒ एन॒ एक॑हायना दर्वा॒चीन॑ मर्वा॒चीन॒ मेक॑हायना॒ देनः॑ । \newline
7. एक॑हायना॒ देन॒ एन॒ एक॑हायना॒ देक॑हायना॒ देनः॑ क॒रोति॑ क॒रो त्येन॒ एक॑हायना॒ देक॑हायना॒ देनः॑ क॒रोति॑ । \newline
8. एक॑हायना॒दित्येक॑ - हा॒य॒ना॒त् । \newline
9. एनः॑ क॒रोति॑ क॒रो त्येन॒ एनः॑ क॒रोति॒ तत् तत् क॒रो त्येन॒ एनः॑ क॒रोति॒ तत् । \newline
10. क॒रोति॒ तत् तत् क॒रोति॑ क॒रोति॒ तदे॒ वैव तत् क॒रोति॑ क॒रोति॒ तदे॒व । \newline
11. तदे॒ वैव तत् तदे॒व तै स्तै रे॒व तत् तदे॒व तैः । \newline
12. ए॒व तै स्तै रे॒वैव तैर वाव॒ तै रे॒वैव तै रव॑ । \newline
13. तै रवाव॒ तै स्तै रव॑ यजते यज॒ते ऽव॒ तै स्तै रव॑ यजते । \newline
14. अव॑ यजते यज॒ते ऽवाव॑ यजते॒ ऽपो॑ ऽपो य॑ज॒ते ऽवाव॑ यजते॒ ऽपः । \newline
15. य॒ज॒ते॒ ऽपो॑ ऽपो य॑जते यजते॒ ऽपो॑ ऽवभृ॒थ म॑वभृ॒थ म॒पो य॑जते यजते॒ ऽपो॑ ऽवभृ॒थम् । \newline
16. अ॒पो॑ ऽवभृ॒थ म॑वभृ॒थ म॒पो᳚(1॒) ऽपो॑ ऽवभृ॒थ मवा वा॑वभृ॒थ म॒पो᳚(1॒) ऽपो॑ ऽवभृ॒थ मव॑ । \newline
17. अ॒व॒भृ॒थ मवा वा॑वभृ॒थ म॑वभृ॒थ मवै᳚त्ये॒ त्यवा॑वभृ॒थ म॑वभृ॒थ मवै॑ति । \newline
18. अ॒व॒भृ॒थमित्य॑व - भृ॒थम् । \newline
19. अवै᳚ त्ये॒त्य वावै᳚ त्य॒फ्स्वा᳚(1॒)फ्स्वे᳚ त्यवावै᳚ त्य॒फ्सु । \newline
20. ए॒त्य॒फ्स्वा᳚(1॒)फ्स्वे᳚ त्येत्य॒फ्सु वै वा अ॒फ्स्वे᳚ त्येत्य॒फ्सु वै । \newline
21. अ॒फ्सु वै वा अ॒फ्स्व॑फ्सु वै वरु॑णो॒ वरु॑णो॒ वा अ॒फ्स्व॑फ्सु वै वरु॑णः । \newline
22. अ॒फ्स्वित्य॑प् - सु । \newline
23. वै वरु॑णो॒ वरु॑णो॒ वै वै वरु॑णः सा॒क्षाथ् सा॒क्षाद् वरु॑णो॒ वै वै वरु॑णः सा॒क्षात् । \newline
24. वरु॑णः सा॒क्षाथ् सा॒क्षाद् वरु॑णो॒ वरु॑णः सा॒क्षा दे॒वैव सा॒क्षाद् वरु॑णो॒ वरु॑णः सा॒क्षा दे॒व । \newline
25. सा॒क्षा दे॒वैव सा॒क्षाथ् सा॒क्षा दे॒व वरु॑णं॒ ॅवरु॑ण मे॒व सा॒क्षाथ् सा॒क्षा दे॒व वरु॑णम् । \newline
26. सा॒क्षादिति॑ स - अ॒क्षात् । \newline
27. ए॒व वरु॑णं॒ ॅवरु॑ण मे॒वैव वरु॑ण॒ मवाव॒ वरु॑ण मे॒वैव वरु॑ण॒ मव॑ । \newline
28. वरु॑ण॒ मवाव॒ वरु॑णं॒ ॅवरु॑ण॒ मव॑ यजते यज॒ते ऽव॒ वरु॑णं॒ ॅवरु॑ण॒ मव॑ यजते । \newline
29. अव॑ यजते यज॒ते ऽवाव॑ यजते॒ वर्त्म॑ना॒ वर्त्म॑ना यज॒ते ऽवाव॑ यजते॒ वर्त्म॑ना । \newline
30. य॒ज॒ते॒ वर्त्म॑ना॒ वर्त्म॑ना यजते यजते॒ वर्त्म॑ना॒ वै वै वर्त्म॑ना यजते यजते॒ वर्त्म॑ना॒ वै । \newline
31. वर्त्म॑ना॒ वै वै वर्त्म॑ना॒ वर्त्म॑ना॒ वा अ॒न्वित्या॒ न्वित्य॒ वै वर्त्म॑ना॒ वर्त्म॑ना॒ वा अ॒न्वित्य॑ । \newline
32. वा अ॒न्वित्या॒ न्वित्य॒ वै वा अ॒न्वित्य॑ य॒ज्ञ्ं ॅय॒ज्ञ् म॒न्वित्य॒ वै वा अ॒न्वित्य॑ य॒ज्ञ्म् । \newline
33. अ॒न्वित्य॑ य॒ज्ञ्ं ॅय॒ज्ञ् म॒न्वित्या॒ न्वित्य॑ य॒ज्ञ्ꣳ रक्षाꣳ॑सि॒ रक्षाꣳ॑सि य॒ज्ञ् म॒न्वित्या॒ न्वित्य॑ य॒ज्ञ्ꣳ रक्षाꣳ॑सि । \newline
34. अ॒न्वित्येत्य॑नु - इत्य॑ । \newline
35. य॒ज्ञ्ꣳ रक्षाꣳ॑सि॒ रक्षाꣳ॑सि य॒ज्ञ्ं ॅय॒ज्ञ्ꣳ रक्षाꣳ॑सि जिघाꣳसन्ति जिघाꣳसन्ति॒ रक्षाꣳ॑सि य॒ज्ञ्ं ॅय॒ज्ञ्ꣳ रक्षाꣳ॑सि जिघाꣳसन्ति । \newline
36. रक्षाꣳ॑सि जिघाꣳसन्ति जिघाꣳसन्ति॒ रक्षाꣳ॑सि॒ रक्षाꣳ॑सि जिघाꣳसन्ति॒ साम्ना॒ साम्ना॑ जिघाꣳसन्ति॒ रक्षाꣳ॑सि॒ रक्षाꣳ॑सि जिघाꣳसन्ति॒ साम्ना᳚ । \newline
37. जि॒घाꣳ॒॒स॒न्ति॒ साम्ना॒ साम्ना॑ जिघाꣳसन्ति जिघाꣳसन्ति॒ साम्ना᳚ प्रस्तो॒ता प्र॑स्तो॒ता साम्ना॑ जिघाꣳसन्ति जिघाꣳसन्ति॒ साम्ना᳚ प्रस्तो॒ता । \newline
38. साम्ना᳚ प्रस्तो॒ता प्र॑स्तो॒ता साम्ना॒ साम्ना᳚ प्रस्तो॒ता ऽन्ववै᳚ त्य॒न्ववै॑ति प्रस्तो॒ता साम्ना॒ साम्ना᳚ प्रस्तो॒ता ऽन्ववै॑ति । \newline
39. प्र॒स्तो॒ता ऽन्ववै᳚ त्य॒न्ववै॑ति प्रस्तो॒ता प्र॑स्तो॒ता ऽन्ववै॑ति॒ साम॒ सामा॒ न्ववै॑ति प्रस्तो॒ता प्र॑स्तो॒ता ऽन्ववै॑ति॒ साम॑ । \newline
40. प्र॒स्तो॒तेति॑ प्र - स्तो॒ता । \newline
41. अ॒न्व वै॑ति॒ साम॒ सामा॒ न्ववै᳚ त्य॒न्व वै॑ति॒ साम॒ वै वै सामा॒ न्ववै᳚ त्य॒न्व वै॑ति॒ साम॒ वै । \newline
42. अ॒न्ववै॒तीत्य॑नु - अवै॑ति । \newline
43. साम॒ वै वै साम॒ साम॒ वै र॑क्षो॒हा र॑क्षो॒हा वै साम॒ साम॒ वै र॑क्षो॒हा । \newline
44. वै र॑क्षो॒हा र॑क्षो॒हा वै वै र॑क्षो॒हा रक्ष॑साꣳ॒॒ रक्ष॑साꣳ रक्षो॒हा वै वै र॑क्षो॒हा रक्ष॑साम् । \newline
45. र॒क्षो॒हा रक्ष॑साꣳ॒॒ रक्ष॑साꣳ रक्षो॒हा र॑क्षो॒हा रक्ष॑सा॒ मप॑हत्या॒ अप॑हत्यै॒ रक्ष॑साꣳ रक्षो॒हा र॑क्षो॒हा रक्ष॑सा॒ मप॑हत्यै । \newline
46. र॒क्षो॒हेति॑ रक्षः - हा । \newline
47. रक्ष॑सा॒ मप॑हत्या॒ अप॑हत्यै॒ रक्ष॑साꣳ॒॒ रक्ष॑सा॒ मप॑हत्यै॒ त्रि स्त्रि रप॑हत्यै॒ रक्ष॑साꣳ॒॒ रक्ष॑सा॒ मप॑हत्यै॒ त्रिः । \newline
48. अप॑हत्यै॒ त्रि स्त्रि रप॑हत्या॒ अप॑हत्यै॒ त्रिर् नि॒धन॑म् नि॒धन॒म् त्रिरप॑हत्या॒ अप॑हत्यै॒ त्रिर् नि॒धन᳚म् । \newline
49. अप॑हत्या॒ इत्यप॑ - ह॒त्यै॒ । \newline
50. त्रिर् नि॒धन॑म् नि॒धन॒म् त्रि स्त्रिर् नि॒धन॒ मुपोप॑ नि॒धन॒म् त्रि स्त्रिर् नि॒धन॒ मुप॑ । \newline
51. नि॒धन॒ मुपोप॑ नि॒धन॑म् नि॒धन॒ मुपै᳚ त्ये॒त्युप॑ नि॒धन॑म् नि॒धन॒ मुपै॑ति । \newline
52. नि॒धन॒मिति॑ नि - धन᳚म् । \newline
53. उपै᳚ त्ये॒त्युपोपै॑ति॒ त्रय॒ स्त्रय॑ ए॒त्युपोपै॑ति॒ त्रयः॑ । \newline
54. ए॒ति॒ त्रय॒ स्त्रय॑ एत्येति॒ त्रय॑ इ॒म इ॒मे त्रय॑ एत्येति॒ त्रय॑ इ॒मे । \newline
55. त्रय॑ इ॒म इ॒मे त्रय॒ स्त्रय॑ इ॒मे लो॒का लो॒का इ॒मे त्रय॒ स्त्रय॑ इ॒मे लो॒काः । \newline
56. इ॒मे लो॒का लो॒का इ॒म इ॒मे लो॒का ए॒भ्य ए॒भ्यो लो॒का इ॒म इ॒मे लो॒का ए॒भ्यः । \newline
57. लो॒का ए॒भ्य ए॒भ्यो लो॒का लो॒का ए॒भ्य ए॒वै वैभ्यो लो॒का लो॒का ए॒भ्य ए॒व । \newline
58. ए॒भ्य ए॒वै वैभ्य ए॒भ्य ए॒व लो॒केभ्यो॑ लो॒केभ्य॑ ए॒वैभ्य ए॒भ्य ए॒व लो॒केभ्यः॑ । \newline
59. ए॒व लो॒केभ्यो॑ लो॒केभ्य॑ ए॒वैव लो॒केभ्यो॒ रक्षाꣳ॑सि॒ रक्षाꣳ॑सि लो॒केभ्य॑ ए॒वैव लो॒केभ्यो॒ रक्षाꣳ॑सि । \newline
60. लो॒केभ्यो॒ रक्षाꣳ॑सि॒ रक्षाꣳ॑सि लो॒केभ्यो॑ लो॒केभ्यो॒ रक्षाꣳ॒॒ स्यपाप॒ रक्षाꣳ॑सि लो॒केभ्यो॑ लो॒केभ्यो॒ रक्षाꣳ॒॒ स्यप॑ । \newline
61. रक्षाꣳ॒॒ स्यपाप॒ रक्षाꣳ॑सि॒ रक्षाꣳ॒॒ स्यप॑ हन्ति ह॒न्त्यप॒ रक्षाꣳ॑सि॒ रक्षाꣳ॒॒ स्यप॑ हन्ति । \newline
\pagebreak
\markright{ TS 6.6.3.2  \hfill https://www.vedavms.in \hfill}

\section{ TS 6.6.3.2 }

\textbf{TS 6.6.3.2 } \newline
\textbf{Samhita Paata} \newline

-स्यप॑ हन्ति॒ पुरु॑षःपुरुषो नि॒धन॒मुपै॑ति॒ पुरु॑षःपुरुषो॒ हि र॑क्ष॒स्वी रक्ष॑सा॒मप॑हत्या उ॒रुꣳ हि राजा॒ वरु॑णश्च॒कारेत्या॑ह॒ प्रति॑ष्ठित्यै श॒तं ते॑ राजन् भि॒षजः॑ स॒हस्र॒मित्या॑ह भेष॒जमे॒वास्मै॑ करोत्य॒भिष्ठि॑तो॒ वरु॑णस्य॒ पाश॒ इत्या॑ह वरुणपा॒शमे॒वाभि ति॑ष्ठति ब॒र्॒.हिर॒भि जु॑हो॒त्याहु॑तीनां॒ प्रति॑ष्ठित्या॒ अथो॑ अग्नि॒वत्ये॒व जु॑हो॒त्यप॑ बर्.हिषः प्रया॒जान्- [  ] \newline

\textbf{Pada Paata} \newline

अपेति॑ । ह॒न्ति॒ । पुरु॑षःपुरुष॒ इति॒ पुरु॑षः - पु॒रु॒षः॒ । नि॒धन॒मिति॑ नि - धन᳚म् । उपेति॑ । ए॒ति॒ । पुरु॑षःपुरुष॒ इति॒ पुरु॑षः - पु॒रु॒षः॒ । हि । र॒क्ष॒स्वी । रक्ष॑साम् । अप॑हत्या॒ इत्यप॑ - ह॒त्यै॒ । उ॒रुम् । हि । राजा᳚ । वरु॑णः । च॒कार॑ । इति॑ । आ॒ह॒ । प्रति॑ष्ठित्या॒ इति॒ प्रति॑-स्थि॒त्यै॒ । श॒तम् । ते॒ । रा॒ज॒न्न् । भि॒षजः॑ । स॒हस्र᳚म् । इति॑ । आ॒ह॒ । भे॒ष॒जम् । ए॒व । अ॒स्मै॒ । क॒रो॒ति॒ । अ॒भिष्ठि॑त॒ इत्य॒भि - स्थि॒तः॒ । वरु॑णस्य । पाशः॑ । इति॑ । आ॒ह॒ । व॒रु॒ण॒पा॒शमिति॑ वरुण - पा॒शम् । ए॒व । अ॒भीति॑ । ति॒ष्ठ॒ति॒ । ब॒र॒.हिः । अ॒भीति॑ । जु॒हो॒ति॒ । आहु॑तीना॒मित्या - हु॒ती॒ना॒म् । प्रति॑ष्ठित्या॒ इति॒ प्रति॑ - स्थि॒त्यै॒ । अथो॒ इति॑ । अ॒ग्नि॒वतीत्य॑ग्नि - वति॑ । ए॒व । जु॒हो॒ति॒ । अप॑बर्.हिष॒ इत्यप॑ - ब॒र॒.हि॒षः॒ । प्र॒या॒जानिति॑ प्र - या॒जान् ।  \newline


\textbf{Krama Paata} \newline

अप॑ हन्ति । ह॒न्ति॒ पुरु॑षःपुरुषः । पुरु॑षःपुरुषो नि॒धन᳚म् । पुरु॑षःपुरुष॒ इति॒ पुरु॑षः - पु॒रु॒षः॒ । नि॒धन॒मुप॑ । नि॒धन॒मिति॑ नि - धन᳚म् । उपै॑ति । ए॒ति॒ पुरु॑षःपुरुषः । पुरु॑षःपुरुषो॒ हि । पुरु॑षःपुरुष॒ इति॒ पुरु॑षः - पु॒रु॒षः॒ । हि र॑क्ष॒स्वी । र॒क्ष॒स्वी रक्ष॑साम् । रक्ष॑सा॒मप॑हत्यै । अप॑हत्या उ॒रुम् । अप॑हत्या॒ इत्यप॑ - ह॒त्यै॒ । उ॒रुꣳ हि । हि राजा᳚ । राजा॒ वरु॑णः । वरु॑णश्च॒कार॑ । च॒कारेति॑ । इत्या॑ह । आ॒ह॒ प्रति॑ष्ठित्यै । प्रति॑ष्ठित्यै श॒तम् । प्रति॑ष्ठित्या॒ इति॒ प्रति॑ - स्थि॒त्यै॒ । श॒तम् ते᳚ । ते॒ रा॒ज॒न्॒ । रा॒ज॒न् भि॒षजः॑ । भि॒षजः॑ स॒हस्र᳚म् । स॒हस्र॒मिति॑ । इत्या॑ह । आ॒ह॒ भे॒ष॒जम् । भे॒ष॒जमे॒व । ए॒वास्मै᳚ । अ॒स्मै॒ क॒रो॒ति॒ । क॒रो॒त्य॒भिष्ठि॑तः । अ॒भिष्ठि॑तो॒ वरु॑णस्य । अ॒भिष्ठि॑त॒ इत्य॒भि - स्थि॒तः॒ । वरु॑णस्य॒ पाशः॑ । पाश॒ इति॑ । इत्या॑ह । आ॒ह॒ व॒रु॒ण॒पा॒शम् । व॒रु॒ण॒पा॒शमे॒व । व॒रु॒ण॒पा॒शमिति॑ वरुण - पा॒शम् । ए॒वाभि । अ॒भि ति॑ष्ठति । ति॒ष्ठ॒ति॒ ब॒र्.॒हिः । ब॒र्.॒हिर॒भि । अ॒भि जु॑होति । जु॒हो॒त्याहु॑तीनाम् । आहु॑तीना॒म् प्रति॑ष्ठित्यै । आहु॑तीना॒मित्या - हु॒ती॒ना॒म् । प्रति॑ष्ठित्या॒ अथो᳚ । प्रति॑ष्ठित्या॒ इति॒ प्रति॑ - स्थि॒त्यै॒ । अथो॑ अग्नि॒वति॑ । अथो॒ इत्यथो᳚ । अ॒ग्नि॒वत्ये॒व । अ॒ग्नि॒वतीत्य॑ग्नि - वति॑ । ए॒व जु॑होति । जु॒हो॒त्यप॑बर्.हिषः । अप॑बर्.हिषः प्रया॒जान् । अप॑बर्.हिष॒ इत्यप॑ - ब॒र्.॒हि॒षः॒ । प्र॒या॒जान्. य॑जति । प्र॒या॒जानिति॑ प्र - या॒जान् \newline

\textbf{Jatai Paata} \newline

1. अप॑ हन्ति ह॒न्त्यपाप॑ हन्ति । \newline
2. ह॒न्ति॒ पुरु॑षःपुरुषः॒ पुरु॑षःपुरुषो हन्ति हन्ति॒ पुरु॑षःपुरुषः । \newline
3. पुरु॑षःपुरुषो नि॒धन॑न् नि॒धन॒म् पुरु॑षःपुरुषः॒ पुरु॑षःपुरुषो नि॒धन᳚म् । \newline
4. पुरु॑षःपुरुष॒ इति॒ पुरु॑षः - पु॒रु॒षः॒ । \newline
5. नि॒धन॒ मुपोप॑ नि॒धन॑म् नि॒धन॒ मुप॑ । \newline
6. नि॒धन॒मिति॑ नि - धन᳚म् । \newline
7. उपै᳚ त्ये॒त्युपोपै॑ति । \newline
8. ए॒ति॒ पुरु॑षःपुरुषः॒ पुरु॑षःपुरुष एत्येति॒ पुरु॑षःपुरुषः । \newline
9. पुरु॑षःपुरुषो॒ हि हि पुरु॑षःपुरुषः॒ पुरु॑षःपुरुषो॒ हि । \newline
10. पुरु॑षःपुरुष॒ इति॒ पुरु॑षः - पु॒रु॒षः॒ । \newline
11. हि र॑क्ष॒स्वी र॑क्ष॒स्वी हि हि र॑क्ष॒स्वी । \newline
12. र॒क्ष॒स्वी रक्ष॑साꣳ॒॒ रक्ष॑साꣳ रक्ष॒स्वी र॑क्ष॒स्वी रक्ष॑साम् । \newline
13. रक्ष॑सा॒ मप॑हत्या॒ अप॑हत्यै॒ रक्ष॑साꣳ॒॒ रक्ष॑सा॒ मप॑हत्यै । \newline
14. अप॑हत्या उ॒रु मु॒रु मप॑हत्या॒ अप॑हत्या उ॒रुम् । \newline
15. अप॑हत्या॒ इत्यप॑ - ह॒त्यै॒ । \newline
16. उ॒रुꣳ हि ह्यु॑रु मु॒रुꣳ हि । \newline
17. हि राजा॒ राजा॒ हि हि राजा᳚ । \newline
18. राजा॒ वरु॑णो॒ वरु॑णो॒ राजा॒ राजा॒ वरु॑णः । \newline
19. वरु॑ण श्च॒कार॑ च॒कार॒ वरु॑णो॒ वरु॑ण श्च॒कार॑ । \newline
20. च॒कारेतीति॑ च॒कार॑ च॒कारेति॑ । \newline
21. इत्या॑हा॒हे तीत्या॑ह । \newline
22. आ॒ह॒ प्रति॑ष्ठित्यै॒ प्रति॑ष्ठित्या आहाह॒ प्रति॑ष्ठित्यै । \newline
23. प्रति॑ष्ठित्यै श॒तꣳ श॒तम् प्रति॑ष्ठित्यै॒ प्रति॑ष्ठित्यै श॒तम् । \newline
24. प्रति॑ष्ठित्या॒ इति॒ प्रति॑ - स्थि॒त्यै॒ । \newline
25. श॒तम् ते॑ ते श॒तꣳ श॒तम् ते᳚ । \newline
26. ते॒ रा॒ज॒न् रा॒ज॒न् ते॒ ते॒ रा॒ज॒न्न् । \newline
27. रा॒ज॒न् भि॒षजो॑ भि॒षजो॑ राजन् राजन् भि॒षजः॑ । \newline
28. भि॒षजः॑ स॒हस्रꣳ॑ स॒हस्र॑म् भि॒षजो॑ भि॒षजः॑ स॒हस्र᳚म् । \newline
29. स॒हस्र॒ मितीति॑ स॒हस्रꣳ॑ स॒हस्र॒ मिति॑ । \newline
30. इत्या॑हा॒हे तीत्या॑ह । \newline
31. आ॒ह॒ भे॒ष॒जम् भे॑ष॒ज मा॑हाह भेष॒जम् । \newline
32. भे॒ष॒ज मे॒वैव भे॑ष॒जम् भे॑ष॒ज मे॒व । \newline
33. ए॒वास्मा॑ अस्मा ए॒वै वास्मै᳚ । \newline
34. अ॒स्मै॒ क॒रो॒ति॒ क॒रो॒ त्य॒स्मा॒ अ॒स्मै॒ क॒रो॒ति॒ । \newline
35. क॒रो॒ त्य॒भिष्ठि॑तो॒ ऽभिष्ठि॑तः करोति करो त्य॒भिष्ठि॑तः । \newline
36. अ॒भिष्ठि॑तो॒ वरु॑णस्य॒ वरु॑णस्या॒ भिष्ठि॑तो॒ ऽभिष्ठि॑तो॒ वरु॑णस्य । \newline
37. अ॒भिष्ठि॑त॒ इत्य॒भि - स्थि॒तः॒ । \newline
38. वरु॑णस्य॒ पाशः॒ पाशो॒ वरु॑णस्य॒ वरु॑णस्य॒ पाशः॑ । \newline
39. पाश॒ इतीति॒ पाशः॒ पाश॒ इति॑ । \newline
40. इत्या॑हा॒हे तीत्या॑ह । \newline
41. आ॒ह॒ व॒रु॒ण॒पा॒शं ॅव॑रुणपा॒श मा॑हाह वरुणपा॒शम् । \newline
42. व॒रु॒ण॒पा॒श मे॒वैव व॑रुणपा॒शं ॅव॑रुणपा॒श मे॒व । \newline
43. व॒रु॒ण॒पा॒शमिति॑ वरुण - पा॒शम् । \newline
44. ए॒वाभ्या᳚(1॒)भ्ये॑ वैवाभि । \newline
45. अ॒भि ति॑ष्ठति तिष्ठ त्य॒भ्य॑भि ति॑ष्ठति । \newline
46. ति॒ष्ठ॒ति॒ ब॒र्॒.हिर् ब॒र्॒.हि स्ति॑ष्ठति तिष्ठति ब॒र्॒.हिः । \newline
47. ब॒र्॒.हि र॒भ्य॑भि ब॒र्॒.हिर् ब॒र्॒.हि र॒भि । \newline
48. अ॒भि जु॑होति जुहो त्य॒भ्य॑भि जु॑होति । \newline
49. जु॒हो॒ त्याहु॑तीना॒ माहु॑तीनाम् जुहोति जुहो॒ त्याहु॑तीनाम् । \newline
50. आहु॑तीना॒म् प्रति॑ष्ठित्यै॒ प्रति॑ष्ठित्या॒ आहु॑तीना॒ माहु॑तीना॒म् प्रति॑ष्ठित्यै । \newline
51. आहु॑तीना॒मित्या - हु॒ती॒ना॒म् । \newline
52. प्रति॑ष्ठित्या॒ अथो॒ अथो॒ प्रति॑ष्ठित्यै॒ प्रति॑ष्ठित्या॒ अथो᳚ । \newline
53. प्रति॑ष्ठित्या॒ इति॒ प्रति॑ - स्थि॒त्यै॒ । \newline
54. अथो॑ अग्नि॒व त्य॑ग्नि॒व त्यथो॒ अथो॑ अग्नि॒वति॑ । \newline
55. अथो॒ इत्यथो᳚ । \newline
56. अ॒ग्नि॒व त्ये॒वै वाग्नि॒व त्य॑ग्नि॒व त्ये॒व । \newline
57. अ॒ग्नि॒वतीत्य॑ग्नि - वति॑ । \newline
58. ए॒व जु॑होति जुहो त्ये॒वैव जु॑होति । \newline
59. जु॒हो॒ त्यप॑बर्.हि॒षो ऽप॑बर्.हिषो जुहोति जुहो॒ त्यप॑बर्.हिषः । \newline
60. अप॑बर्.हिषः प्रया॒जान् प्र॑या॒जा नप॑बर्.हि॒षो ऽप॑बर्.हिषः प्रया॒जान् । \newline
61. अप॑बर्.हिष॒ इत्यप॑ - ब॒र॒.हि॒षः॒ । \newline
62. प्र॒या॒जान्. य॑जति यजति प्रया॒जान् प्र॑या॒जान्. य॑जति । \newline
63. प्र॒या॒जानिति॑ प्र - या॒जान् । \newline

\textbf{Ghana Paata } \newline

1. अप॑ हन्ति ह॒न्त्य पाप॑ हन्ति॒ पुरु॑षःपुरुषः॒ पुरु॑षःपुरुषो ह॒न्त्य पाप॑ हन्ति॒ पुरु॑षःपुरुषः । \newline
2. ह॒न्ति॒ पुरु॑षःपुरुषः॒ पुरु॑षःपुरुषो हन्ति हन्ति॒ पुरु॑षःपुरुषो नि॒धन॑म् नि॒धन॒म् पुरु॑षःपुरुषो हन्ति हन्ति॒ पुरु॑षःपुरुषो नि॒धन᳚म् । \newline
3. पुरु॑षःपुरुषो नि॒धन॑म् नि॒धन॒म् पुरु॑षःपुरुषः॒ पुरु॑षःपुरुषो नि॒धन॒ मुपोप॑ नि॒धन॒म् पुरु॑षःपुरुषः॒ पुरु॑षःपुरुषो नि॒धन॒ मुप॑ । \newline
4. पुरु॑षःपुरुष॒ इति॒ पुरु॑षः - पु॒रु॒षः॒ । \newline
5. नि॒धन॒ मुपोप॑ नि॒धन॑म् नि॒धन॒ मुपै᳚त्ये॒ त्युप॑ नि॒धन॑म् नि॒धन॒ मुपै॑ति । \newline
6. नि॒धन॒मिति॑ नि - धन᳚म् । \newline
7. उपै᳚त्ये॒ त्युपोपै॑ति॒ पुरु॑षःपुरुषः॒ पुरु॑षःपुरुष ए॒त्युपोपै॑ति॒ पुरु॑षःपुरुषः । \newline
8. ए॒ति॒ पुरु॑षःपुरुषः॒ पुरु॑षःपुरुष एत्येति॒ पुरु॑षःपुरुषो॒ हि हि पुरु॑षःपुरुष एत्येति॒ पुरु॑षःपुरुषो॒ हि । \newline
9. पुरु॑षःपुरुषो॒ हि हि पुरु॑षःपुरुषः॒ पुरु॑षःपुरुषो॒ हि र॑क्ष॒स्वी र॑क्ष॒स्वी हि पुरु॑षःपुरुषः॒ पुरु॑षःपुरुषो॒ हि र॑क्ष॒स्वी । \newline
10. पुरु॑षःपुरुष॒ इति॒ पुरु॑षः - पु॒रु॒षः॒ । \newline
11. हि र॑क्ष॒स्वी र॑क्ष॒स्वी हि हि र॑क्ष॒स्वी रक्ष॑साꣳ॒॒ रक्ष॑साꣳ रक्ष॒स्वी हि हि र॑क्ष॒स्वी रक्ष॑साम् । \newline
12. र॒क्ष॒स्वी रक्ष॑साꣳ॒॒ रक्ष॑साꣳ रक्ष॒स्वी र॑क्ष॒स्वी रक्ष॑सा॒ मप॑हत्या॒ अप॑हत्यै॒ रक्ष॑साꣳ रक्ष॒स्वी र॑क्ष॒स्वी रक्ष॑सा॒ मप॑हत्यै । \newline
13. रक्ष॑सा॒ मप॑हत्या॒ अप॑हत्यै॒ रक्ष॑साꣳ॒॒ रक्ष॑सा॒ मप॑हत्या उ॒रु मु॒रु मप॑हत्यै॒ रक्ष॑साꣳ॒॒ रक्ष॑सा॒ मप॑हत्या उ॒रुम् । \newline
14. अप॑हत्या उ॒रु मु॒रु मप॑हत्या॒ अप॑हत्या उ॒रुꣳ हि ह्यु॑रु मप॑हत्या॒ अप॑हत्या उ॒रुꣳ हि । \newline
15. अप॑हत्या॒ इत्यप॑ - ह॒त्यै॒ । \newline
16. उ॒रुꣳ हि ह्यु॑रु मु॒रुꣳ हि राजा॒ राजा॒ ह्यु॑रु मु॒रुꣳ हि राजा᳚ । \newline
17. हि राजा॒ राजा॒ हि हि राजा॒ वरु॑णो॒ वरु॑णो॒ राजा॒ हि हि राजा॒ वरु॑णः । \newline
18. राजा॒ वरु॑णो॒ वरु॑णो॒ राजा॒ राजा॒ वरु॑ण श्च॒कार॑ च॒कार॒ वरु॑णो॒ राजा॒ राजा॒ वरु॑ण श्च॒कार॑ । \newline
19. वरु॑ण श्च॒कार॑ च॒कार॒ वरु॑णो॒ वरु॑ण श्च॒कारेतीति॑ च॒कार॒ वरु॑णो॒ वरु॑ण श्च॒कारेति॑ । \newline
20. च॒कारेतीति॑ च॒कार॑ च॒कारे त्या॑हा॒हेति॑ च॒कार॑ च॒कारे त्या॑ह । \newline
21. इत्या॑हा॒हे तीत्या॑ह॒ प्रति॑ष्ठित्यै॒ प्रति॑ष्ठित्या आ॒हे तीत्या॑ह॒ प्रति॑ष्ठित्यै । \newline
22. आ॒ह॒ प्रति॑ष्ठित्यै॒ प्रति॑ष्ठित्या आहाह॒ प्रति॑ष्ठित्यै श॒तꣳ श॒तम् प्रति॑ष्ठित्या आहाह॒ प्रति॑ष्ठित्यै श॒तम् । \newline
23. प्रति॑ष्ठित्यै श॒तꣳ श॒तम् प्रति॑ष्ठित्यै॒ प्रति॑ष्ठित्यै श॒तम् ते॑ ते श॒तम् प्रति॑ष्ठित्यै॒ प्रति॑ष्ठित्यै श॒तम् ते᳚ । \newline
24. प्रति॑ष्ठित्या॒ इति॒ प्रति॑ - स्थि॒त्यै॒ । \newline
25. श॒तम् ते॑ ते श॒तꣳ श॒तम् ते॑ राजन् राजन् ते श॒तꣳ श॒तम् ते॑ राजन्न् । \newline
26. ते॒ रा॒ज॒न् रा॒ज॒न् ते॒ ते॒ रा॒ज॒न् भि॒षजो॑ भि॒षजो॑ राजन् ते ते राजन् भि॒षजः॑ । \newline
27. रा॒ज॒न् भि॒षजो॑ भि॒षजो॑ राजन् राजन् भि॒षजः॑ स॒हस्रꣳ॑ स॒हस्र॑म् भि॒षजो॑ राजन् राजन् भि॒षजः॑ स॒हस्र᳚म् । \newline
28. भि॒षजः॑ स॒हस्रꣳ॑ स॒हस्र॑म् भि॒षजो॑ भि॒षजः॑ स॒हस्र॒ मितीति॑ स॒हस्र॑म् भि॒षजो॑ भि॒षजः॑ स॒हस्र॒ मिति॑ । \newline
29. स॒हस्र॒ मितीति॑ स॒हस्रꣳ॑ स॒हस्र॒ मित्या॑हा॒हेति॑ स॒हस्रꣳ॑ स॒हस्र॒ मित्या॑ह । \newline
30. इत्या॑हा॒हे तीत्या॑ह भेष॒जम् भे॑ष॒ज मा॒हे तीत्या॑ह भेष॒जम् । \newline
31. आ॒ह॒ भे॒ष॒जम् भे॑ष॒ज मा॑हाह भेष॒ज मे॒वैव भे॑ष॒ज मा॑हाह भेष॒ज मे॒व । \newline
32. भे॒ष॒ज मे॒वैव भे॑ष॒जम् भे॑ष॒ज मे॒वास्मा॑ अस्मा ए॒व भे॑ष॒जम् भे॑ष॒ज मे॒वास्मै᳚ । \newline
33. ए॒वास्मा॑ अस्मा ए॒वै वास्मै॑ करोति करो त्यस्मा ए॒वै वास्मै॑ करोति । \newline
34. अ॒स्मै॒ क॒रो॒ति॒ क॒रो॒ त्य॒स्मा॒ अ॒स्मै॒ क॒रो॒ त्य॒भिष्ठि॑तो॒ ऽभिष्ठि॑तः करो त्यस्मा अस्मै करो त्य॒भिष्ठि॑तः । \newline
35. क॒रो॒ त्य॒भिष्ठि॑तो॒ ऽभिष्ठि॑तः करोति करो त्य॒भिष्ठि॑तो॒ वरु॑णस्य॒ वरु॑णस्या॒ भिष्ठि॑तः करोति करो त्य॒भिष्ठि॑तो॒ वरु॑णस्य । \newline
36. अ॒भिष्ठि॑तो॒ वरु॑णस्य॒ वरु॑णस्या॒ भिष्ठि॑तो॒ ऽभिष्ठि॑तो॒ वरु॑णस्य॒ पाशः॒ पाशो॒ वरु॑णस्या॒ भिष्ठि॑तो॒ ऽभिष्ठि॑तो॒ वरु॑णस्य॒ पाशः॑ । \newline
37. अ॒भिष्ठि॑त॒ इत्य॒भि - स्थि॒तः॒ । \newline
38. वरु॑णस्य॒ पाशः॒ पाशो॒ वरु॑णस्य॒ वरु॑णस्य॒ पाश॒ इतीति॒ पाशो॒ वरु॑णस्य॒ वरु॑णस्य॒ पाश॒ इति॑ । \newline
39. पाश॒ इतीति॒ पाशः॒ पाश॒ इत्या॑हा॒ हेति॒ पाशः॒ पाश॒ इत्या॑ह । \newline
40. इत्या॑हा॒हे तीत्या॑ह वरुणपा॒शं ॅव॑रुणपा॒श मा॒हे तीत्या॑ह वरुणपा॒शम् । \newline
41. आ॒ह॒ व॒रु॒ण॒पा॒शं ॅव॑रुणपा॒श मा॑हाह वरुणपा॒श मे॒वैव व॑रुणपा॒श मा॑हाह वरुणपा॒श मे॒व । \newline
42. व॒रु॒ण॒पा॒श मे॒वैव व॑रुणपा॒शं ॅव॑रुणपा॒श मे॒वाभ्या᳚(1॒)भ्ये॑व व॑रुणपा॒शं ॅव॑रुणपा॒श मे॒वाभि । \newline
43. व॒रु॒ण॒पा॒शमिति॑ वरुण - पा॒शम् । \newline
44. ए॒वाभ्या᳚(1॒)भ्ये॑ वैवाभि ति॑ष्ठति तिष्ठ त्य॒भ्ये॑ वैवाभि ति॑ष्ठति । \newline
45. अ॒भि ति॑ष्ठति तिष्ठ त्य॒भ्य॑भि ति॑ष्ठति ब॒र्॒.हिर् ब॒र्॒.हि स्ति॑ष्ठ त्य॒भ्य॑भि ति॑ष्ठति ब॒र्॒.हिः । \newline
46. ति॒ष्ठ॒ति॒ ब॒र्॒.हिर् ब॒र्॒.हि स्ति॑ष्ठति तिष्ठति ब॒र्॒.हि र॒भ्य॑भि ब॒र्॒.हि स्ति॑ष्ठति तिष्ठति ब॒र्॒.हि र॒भि । \newline
47. ब॒र्॒.हि र॒भ्य॑भि ब॒र्॒.हिर् ब॒र्॒.हि र॒भि जु॑होति जुहो त्य॒भि ब॒र्॒.हिर् ब॒र्॒.हि र॒भि जु॑होति । \newline
48. अ॒भि जु॑होति जुहो त्य॒भ्य॑भि जु॑हो॒ त्याहु॑तीना॒ माहु॑तीनाम् जुहो त्य॒भ्य॑भि जु॑हो॒ त्याहु॑तीनाम् । \newline
49. जु॒हो॒ त्याहु॑तीना॒ माहु॑तीनाम् जुहोति जुहो॒ त्याहु॑तीना॒म् प्रति॑ष्ठित्यै॒ प्रति॑ष्ठित्या॒ आहु॑तीनाम् जुहोति जुहो॒ त्याहु॑तीना॒म् प्रति॑ष्ठित्यै । \newline
50. आहु॑तीना॒म् प्रति॑ष्ठित्यै॒ प्रति॑ष्ठित्या॒ आहु॑तीना॒ माहु॑तीना॒म् प्रति॑ष्ठित्या॒ अथो॒ अथो॒ प्रति॑ष्ठित्या॒ आहु॑तीना॒ माहु॑तीना॒म् प्रति॑ष्ठित्या॒ अथो᳚ । \newline
51. आहु॑तीना॒मित्या - हु॒ती॒ना॒म् । \newline
52. प्रति॑ष्ठित्या॒ अथो॒ अथो॒ प्रति॑ष्ठित्यै॒ प्रति॑ष्ठित्या॒ अथो॑ अग्नि॒व त्य॑ग्नि॒व त्यथो॒ प्रति॑ष्ठित्यै॒ प्रति॑ष्ठित्या॒ अथो॑ अग्नि॒वति॑ । \newline
53. प्रति॑ष्ठित्या॒ इति॒ प्रति॑ - स्थि॒त्यै॒ । \newline
54. अथो॑ अग्नि॒व त्य॑ग्नि॒व त्यथो॒ अथो॑ अग्नि॒व त्ये॒वै वाग्नि॒व त्यथो॒ अथो॑ अग्नि॒व त्ये॒व । \newline
55. अथो॒ इत्यथो᳚ । \newline
56. अ॒ग्नि॒व त्ये॒वै वाग्नि॒व त्य॑ग्नि॒व त्ये॒व जु॑होति जुहो त्ये॒वाग्नि॒व त्य॑ग्नि॒व त्ये॒व जु॑होति । \newline
57. अ॒ग्नि॒वतीत्य॑ग्नि - वति॑ । \newline
58. ए॒व जु॑होति जुहो त्ये॒वैव जु॑हो॒ त्यप॑बर्.हि॒षो ऽप॑बर्.हिषो जुहो त्ये॒वैव जु॑हो॒ त्यप॑बर्.हिषः । \newline
59. जु॒हो॒ त्यप॑बर्.हि॒षो ऽप॑बर्.हिषो जुहोति जुहो॒ त्यप॑बर्.हिषः प्रया॒जान् प्र॑या॒जा नप॑बर्.हिषो जुहोति जुहो॒ त्यप॑बर्.हिषः प्रया॒जान् । \newline
60. अप॑बर्.हिषः प्रया॒जान् प्र॑या॒जा नप॑बर्.हि॒षो ऽप॑बर्.हिषः प्रया॒जान्. य॑जति यजति प्रया॒जा नप॑बर्.हि॒षो ऽप॑बर्.हिषः प्रया॒जान्. य॑जति । \newline
61. अप॑बर्.हिष॒ इत्यप॑ - ब॒र॒.हि॒षः॒ । \newline
62. प्र॒या॒जान्. य॑जति यजति प्रया॒जान् प्र॑या॒जान्. य॑जति प्र॒जाः प्र॒जा य॑जति प्रया॒जान् प्र॑या॒जान्. य॑जति प्र॒जाः । \newline
63. प्र॒या॒जानिति॑ प्र - या॒जान् । \newline
\pagebreak
\markright{ TS 6.6.3.3  \hfill https://www.vedavms.in \hfill}

\section{ TS 6.6.3.3 }

\textbf{TS 6.6.3.3 } \newline
\textbf{Samhita Paata} \newline

य॑जति प्र॒जा वै ब॒र्॒.हिः प्र॒जा ए॒व व॑रुणपा॒शान्-मु॑ञ्च॒त्याज्य॑भागौ यजति य॒ज्ञ्स्यै॒व चक्षु॑षी॒ नान्तरे॑ति॒ वरु॑णं ॅयजति वरुणपा॒शादे॒वैनं॑ मुञ्चत्य॒ग्नीवरु॑णौ यजति सा॒क्षादे॒वैनं॑ ॅवरुणपा॒शान् मु॑ञ्च॒त्य-प॑बर्.हिषावनूय॒जौ य॑जति प्र॒जा वै ब॒र्॒.हिः प्र॒जा ए॒व व॑रुणपा॒शान्-मु॑ञ्चति च॒तुरः॑ प्रया॒जान्. य॑जति॒ द्वाव॑नूया॒जौ षट्थ् संप॑द्यन्ते॒ षड्वा ऋ॒तव॑- [  ] \newline

\textbf{Pada Paata} \newline

य॒ज॒ति॒ । प्र॒जा इति॑ प्र - जाः । वै । ब॒र्॒.हिः । प्र॒जा इति॑ प्र-जाः । ए॒व । व॒रु॒ण॒पा॒शादिति॑ वरुण - पा॒शात् । मु॒ञ्च॒ति॒ । आज्य॑भागा॒वित्याज्य॑ - भा॒गौ॒ । य॒ज॒ति॒ । य॒ज्ञ्स्य॑ । ए॒व । चक्षु॑षी॒ इति॑ । न । अ॒न्तः । ए॒ति॒ । वरु॑णम् । य॒ज॒ति॒ । व॒रु॒ण॒पा॒शादिति॑ वरुण - पा॒शात् । ए॒व । ए॒न॒म् । मु॒ञ्च॒ति॒ । अ॒ग्नीवरु॑णा॒वित्य॒ग्नी - वरु॑णौ । य॒ज॒ति॒ । सा॒क्षादिति॑ स - अ॒क्षात् । ए॒व । ए॒न॒म् । व॒रु॒ण॒पा॒शादिति॑ वरुण - पा॒शात् । मु॒ञ्च॒ति॒ । अप॑बर्.हिषा॒वित्यप॑ - ब॒र॒.हि॒षौ॒ । अ॒नू॒या॒जावित्य॑नु - या॒जौ । य॒ज॒ति॒ । प्र॒जा इति॑ प्र - जाः । वै । ब॒र॒.हिः । प्र॒जा इति॑ प्र - जाः । ए॒व । व॒रु॒ण॒पा॒शादिति॑ वरुण - पा॒शात् । मु॒ञ्च॒ति॒ । च॒तुरः॑ । प्र॒या॒जानिति॑ प्र - या॒जान् । य॒ज॒ति॒ । द्वौ । अ॒नू॒या॒जावित्य॑नु - या॒जौ । षट् । समिति॑ । प॒द्य॒न्ते॒ । षट् । वै । ऋ॒तवः॑ ।  \newline


\textbf{Krama Paata} \newline

य॒ज॒ति॒ प्र॒जाः । प्र॒जा वै । प्र॒जा इति॑ प्र - जाः । वै ब॒र्.॒हिः । ब॒र्.॒हिः प्र॒जाः । प्र॒जा ए॒व । प्र॒जा इति॑ प्र - जाः । ए॒व व॑रुणपा॒शात् । व॒रु॒ण॒पा॒शान् मु॑ञ्चति । व॒रु॒ण॒पा॒शादिति॑ वरुण - पा॒शात् । मु॒ञ्च॒त्याज्य॑भागौ । आज्य॑भागौ यजति । आज्य॑भागा॒वित्याज्य॑ - भा॒गौ॒ । य॒ज॒ति॒ य॒ज्ञ्स्य॑ । य॒ज्ञ्स्यै॒व । ए॒व चक्षु॑षी । चक्षु॑षी॒ न । चक्षु॑षी॒ इति॒ चक्षु॑षी । नान्तः । अ॒न्तरे॑ति । ए॒ति॒ वरु॑णम् । वरु॑णम् ॅयजति । य॒ज॒ति॒ व॒रु॒ण॒पा॒शात् । व॒रु॒ण॒पा॒शादे॒व । व॒रु॒ण॒पा॒शादिति॑ वरुण - पा॒शात् । ए॒वैन᳚म् । ए॒न॒म् मु॒ञ्च॒ति॒ । मु॒ञ्च॒त्य॒ग्नीवरु॑णौ । अ॒ग्नीवरु॑णौ यजति । अ॒ग्नीवरु॑णा॒वित्य॒ग्नी - वरु॑णौ । य॒ज॒ति॒ सा॒क्षात् । सा॒क्षादे॒व । सा॒क्षादिति॑ स - अ॒क्षात् । ए॒वैन᳚म् । ए॒न॒म् ॅव॒रु॒ण॒पा॒शात् । व॒रु॒ण॒पा॒शान् मु॑ञ्चति । व॒रु॒ण॒पा॒शादिति॑ वरुण - पा॒शात् । मु॒ञ्च॒त्यप॑बर्.हिषौ । अप॑बर्.हिषावनूया॒जौ । अप॑बर्.हिषा॒वित्यप॑ - ब॒र्.॒हि॒षौ॒ । अ॒नू॒या॒जौ य॑जति । अ॒नू॒या॒जावित्य॑नु - या॒जौ । य॒ज॒ति॒ प्र॒जाः । प्र॒जा वै । प्र॒जा इति॑ प्र - जाः । वै ब॒र्.॒हिः । ब॒र्.॒हिः प्र॒जाः । प्र॒जा ए॒व । प्र॒जा इति॑ प्र - जाः । ए॒व व॑रुणपा॒शात् । व॒रु॒ण॒पा॒शान् मु॑ञ्चति । व॒रु॒ण॒पा॒शादिति॑ वरुण - पा॒शात् । मु॒ञ्च॒ति॒ च॒तुरः॑ । च॒तुरः॑ प्रया॒जान् । प्र॒या॒जान्. य॑जति । प्र॒या॒जानिति॑ प्र - या॒जान् । य॒ज॒ति॒ द्वौ । द्वाव॑नूया॒जौ । अ॒नू॒या॒जौ षट् । अ॒नू॒या॒जावित्य॑नु - या॒जौ । षट्थ् सम् । सम् प॑द्यन्ते । प॒द्य॒न्ते॒ षट् । षड् वै । वा ऋ॒तवः॑ । ऋ॒तव॑ ऋ॒तुषु॑ \newline

\textbf{Jatai Paata} \newline

1. य॒ज॒ति॒ प्र॒जाः प्र॒जा य॑जति यजति प्र॒जाः । \newline
2. प्र॒जा वै वै प्र॒जाः प्र॒जा वै । \newline
3. प्र॒जा इति॑ प्र - जाः । \newline
4. वै ब॒र्॒.हिर् ब॒र्॒.हिर् वै वै ब॒र्॒.हिः । \newline
5. ब॒र्॒.हिः प्र॒जाः प्र॒जा ब॒र्॒.हिर् ब॒र्॒.हिः प्र॒जाः । \newline
6. प्र॒जा ए॒वैव प्र॒जाः प्र॒जा ए॒व । \newline
7. प्र॒जा इति॑ प्र - जाः । \newline
8. ए॒व व॑रुणपा॒शाद् व॑रुणपा॒शा दे॒वैव व॑रुणपा॒शात् । \newline
9. व॒रु॒ण॒पा॒शान् मु॑ञ्चति मुञ्चति वरुणपा॒शाद् व॑रुणपा॒शान् मु॑ञ्चति । \newline
10. व॒रु॒ण॒पा॒शादिति॑ वरुण - पा॒शात् । \newline
11. मु॒ञ्च॒ त्याज्य॑भागा॒ वाज्य॑भागौ मुञ्चति मुञ्च॒ त्याज्य॑भागौ । \newline
12. आज्य॑भागौ यजति यज॒ त्याज्य॑भागा॒ वाज्य॑भागौ यजति । \newline
13. आज्य॑भागा॒वित्याज्य॑ - भा॒गौ॒ । \newline
14. य॒ज॒ति॒ य॒ज्ञ्स्य॑ य॒ज्ञ्स्य॑ यजति यजति य॒ज्ञ्स्य॑ । \newline
15. य॒ज्ञ् स्यै॒वैव य॒ज्ञ्स्य॑ य॒ज्ञ् स्यै॒व । \newline
16. ए॒व चक्षु॑षी॒ चक्षु॑षी ए॒वैव चक्षु॑षी । \newline
17. चक्षु॑षी॒ न न चक्षु॑षी॒ चक्षु॑षी॒ न । \newline
18. चक्षु॑षी॒ इति॒ चक्षु॑षी । \newline
19. नान्त र॒न्तर् न नान्तः । \newline
20. अ॒न्त रे᳚त्ये त्य॒न्त र॒न्त रे॑ति । \newline
21. ए॒ति॒ वरु॑णं॒ ॅवरु॑ण मेत्येति॒ वरु॑णम् । \newline
22. वरु॑णं ॅयजति यजति॒ वरु॑णं॒ ॅवरु॑णं ॅयजति । \newline
23. य॒ज॒ति॒ व॒रु॒ण॒पा॒शाद् व॑रुणपा॒शाद् य॑जति यजति वरुणपा॒शात् । \newline
24. व॒रु॒ण॒पा॒शा दे॒वैव व॑रुणपा॒शाद् व॑रुणपा॒शा दे॒व । \newline
25. व॒रु॒ण॒पा॒शादिति॑ वरुण - पा॒शात् । \newline
26. ए॒वैन॑ मेन मे॒वै वैन᳚म् । \newline
27. ए॒न॒म् मु॒ञ्च॒ति॒ मु॒ञ्च॒ त्ये॒न॒ मे॒न॒म् मु॒ञ्च॒ति॒ । \newline
28. मु॒ञ्च॒ त्य॒ग्नीवरु॑णा व॒ग्नीवरु॑णौ मुञ्चति मुञ्च त्य॒ग्नीवरु॑णौ । \newline
29. अ॒ग्नीवरु॑णौ यजति यज त्य॒ग्नीवरु॑णा व॒ग्नीवरु॑णौ यजति । \newline
30. अ॒ग्नीवरु॑णा॒वित्य॒ग्नी - वरु॑णौ । \newline
31. य॒ज॒ति॒ सा॒क्षाथ् सा॒क्षाद् य॑जति यजति सा॒क्षात् । \newline
32. सा॒क्षा दे॒वैव सा॒क्षाथ् सा॒क्षा दे॒व । \newline
33. सा॒क्षादिति॑ स - अ॒क्षात् । \newline
34. ए॒वैन॑ मेन मे॒वै वैन᳚म् । \newline
35. ए॒नं॒ ॅव॒रु॒ण॒पा॒शाद् व॑रुणपा॒शा दे॑न मेनं ॅवरुणपा॒शात् । \newline
36. व॒रु॒ण॒पा॒शान् मु॑ञ्चति मुञ्चति वरुणपा॒शाद् व॑रुणपा॒शान् मु॑ञ्चति । \newline
37. व॒रु॒ण॒पा॒शादिति॑ वरुण - पा॒शात् । \newline
38. मु॒ञ्च॒ त्यप॑बर्.हिषा॒ वप॑बर्.हिषौ मुञ्चति मुञ्च॒ त्यप॑बर्.हिषौ । \newline
39. अप॑बर्.हिषा वनूया॒जा व॑नूया॒जा वप॑बर्.हिषा॒ वप॑बर्.हिषा वनूया॒जौ । \newline
40. अप॑बर्.हिषा॒वित्यप॑ - ब॒र्॒.हि॒षौ॒ । \newline
41. अ॒नू॒या॒जौ य॑जति यज त्यनूया॒जा व॑नूया॒जौ य॑जति । \newline
42. अ॒नू॒या॒जावित्य॑नु - या॒जौ । \newline
43. य॒ज॒ति॒ प्र॒जाः प्र॒जा य॑जति यजति प्र॒जाः । \newline
44. प्र॒जा वै वै प्र॒जाः प्र॒जा वै । \newline
45. प्र॒जा इति॑ प्र - जाः । \newline
46. वै ब॒र्॒.हिर् ब॒र्॒.हिर् वै वै ब॒र्॒.हिः । \newline
47. ब॒र्॒.हिः प्र॒जाः प्र॒जा ब॒र्॒.हिर् ब॒र्॒.हिः प्र॒जाः । \newline
48. प्र॒जा ए॒वैव प्र॒जाः प्र॒जा ए॒व । \newline
49. प्र॒जा इति॑ प्र - जाः । \newline
50. ए॒व व॑रुणपा॒शाद् व॑रुणपा॒शा दे॒वैव व॑रुणपा॒शात् । \newline
51. व॒रु॒ण॒पा॒शान् मु॑ञ्चति मुञ्चति वरुणपा॒शाद् व॑रुणपा॒शान् मु॑ञ्चति । \newline
52. व॒रु॒ण॒पा॒शादिति॑ वरुण - पा॒शात् । \newline
53. मु॒ञ्च॒ति॒ च॒तुर॑ श्च॒तुरो॑ मुञ्चति मुञ्चति च॒तुरः॑ । \newline
54. च॒तुरः॑ प्रया॒जान् प्र॑या॒जाꣳ श्च॒तुर॑ श्च॒तुरः॑ प्रया॒जान् । \newline
55. प्र॒या॒जान्. य॑जति यजति प्रया॒जान् प्र॑या॒जान्. य॑जति । \newline
56. प्र॒या॒जानिति॑ प्र - या॒जान् । \newline
57. य॒ज॒ति॒ द्वौ द्वौ य॑जति यजति॒ द्वौ । \newline
58. द्वा व॑नूया॒जा व॑नूया॒जौ द्वौ द्वा व॑नूया॒जौ । \newline
59. अ॒नू॒या॒जौ षट् थ्षड॑नूया॒जा व॑नूया॒जौ षट् । \newline
60. अ॒नू॒या॒जावित्य॑नु - या॒जौ । \newline
61. षट् थ्सꣳ सꣳ षट् थ्षट् थ्सम् । \newline
62. सम् प॑द्यन्ते पद्यन्ते॒ सꣳ सम् प॑द्यन्ते । \newline
63. प॒द्य॒न्ते॒ षट् थ्षट् प॑द्यन्ते पद्यन्ते॒ षट् । \newline
64. षड् वै वै षट् थ्षड् वै । \newline
65. वा ऋ॒तव॑ ऋ॒तवो॒ वै वा ऋ॒तवः॑ । \newline
66. ऋ॒तव॑ ऋ॒तुष् वृ॒तुष् वृ॒तव॑ ऋ॒तव॑ ऋ॒तुषु॑ । \newline

\textbf{Ghana Paata } \newline

1. य॒ज॒ति॒ प्र॒जाः प्र॒जा य॑जति यजति प्र॒जा वै वै प्र॒जा य॑जति यजति प्र॒जा वै । \newline
2. प्र॒जा वै वै प्र॒जाः प्र॒जा वै ब॒र्॒.हिर् ब॒र्॒.हिर् वै प्र॒जाः प्र॒जा वै ब॒र्॒.हिः । \newline
3. प्र॒जा इति॑ प्र - जाः । \newline
4. वै ब॒र्॒.हिर् ब॒र्॒.हिर् वै वै ब॒र्॒.हिः प्र॒जाः प्र॒जा ब॒र्॒.हिर् वै वै ब॒र्॒.हिः प्र॒जाः । \newline
5. ब॒र्॒.हिः प्र॒जाः प्र॒जा ब॒र्॒.हिर् ब॒र्॒.हिः प्र॒जा ए॒वैव प्र॒जा ब॒र्॒.हिर् ब॒र्॒.हिः प्र॒जा ए॒व । \newline
6. प्र॒जा ए॒वैव प्र॒जाः प्र॒जा ए॒व व॑रुणपा॒शाद् व॑रुणपा॒शा दे॒व प्र॒जाः प्र॒जा ए॒व व॑रुणपा॒शात् । \newline
7. प्र॒जा इति॑ प्र - जाः । \newline
8. ए॒व व॑रुणपा॒शाद् व॑रुणपा॒शा दे॒वैव व॑रुणपा॒शान् मु॑ञ्चति मुञ्चति वरुणपा॒शा दे॒वैव व॑रुणपा॒शान् मु॑ञ्चति । \newline
9. व॒रु॒ण॒पा॒शान् मु॑ञ्चति मुञ्चति वरुणपा॒शाद् व॑रुणपा॒शान् मु॑ञ्च॒ त्याज्य॑भागा॒ वाज्य॑भागौ मुञ्चति वरुणपा॒शाद् व॑रुणपा॒शान् मु॑ञ्च॒ त्याज्य॑भागौ । \newline
10. व॒रु॒ण॒पा॒शादिति॑ वरुण - पा॒शात् । \newline
11. मु॒ञ्च॒ त्याज्य॑भागा॒ वाज्य॑भागौ मुञ्चति मुञ्च॒ त्याज्य॑भागौ यजति यज॒ त्याज्य॑भागौ मुञ्चति मुञ्च॒ त्याज्य॑भागौ यजति । \newline
12. आज्य॑भागौ यजति यज॒ त्याज्य॑भागा॒ वाज्य॑भागौ यजति य॒ज्ञ्स्य॑ य॒ज्ञ्स्य॑ यज॒ त्याज्य॑भागा॒ वाज्य॑भागौ यजति य॒ज्ञ्स्य॑ । \newline
13. आज्य॑भागा॒वित्याज्य॑ - भा॒गौ॒ । \newline
14. य॒ज॒ति॒ य॒ज्ञ्स्य॑ य॒ज्ञ्स्य॑ यजति यजति य॒ज्ञ्स्यै॒वैव य॒ज्ञ्स्य॑ यजति यजति य॒ज्ञ् स्यै॒व । \newline
15. य॒ज्ञ्स्यै॒वैव य॒ज्ञ्स्य॑ य॒ज्ञ्स्यै॒व चक्षु॑षी॒ चक्षु॑षी ए॒व य॒ज्ञ्स्य॑ य॒ज्ञ्स्यै॒व चक्षु॑षी । \newline
16. ए॒व चक्षु॑षी॒ चक्षु॑षी ए॒वैव चक्षु॑षी॒ न न चक्षु॑षी ए॒वैव चक्षु॑षी॒ न । \newline
17. चक्षु॑षी॒ न न चक्षु॑षी॒ चक्षु॑षी॒ नान्त र॒न्तर् न चक्षु॑षी॒ चक्षु॑षी॒ नान्तः । \newline
18. चक्षु॑षी॒ इति॒ चक्षु॑षी । \newline
19. नान्त र॒न्तर् न नान्त रे᳚त्ये त्य॒न्तर् न नान्त रे॑ति । \newline
20. अ॒न्त रे᳚त्ये त्य॒न्त र॒न्त रे॑ति॒ वरु॑णं॒ ॅवरु॑ण मेत्य॒न्त र॒न्त रे॑ति॒ वरु॑णम् । \newline
21. ए॒ति॒ वरु॑णं॒ ॅवरु॑ण मेत्येति॒ वरु॑णं ॅयजति यजति॒ वरु॑ण मेत्येति॒ वरु॑णं ॅयजति । \newline
22. वरु॑णं ॅयजति यजति॒ वरु॑णं॒ ॅवरु॑णं ॅयजति वरुणपा॒शाद् व॑रुणपा॒शाद् य॑जति॒ वरु॑णं॒ ॅवरु॑णं ॅयजति वरुणपा॒शात् । \newline
23. य॒ज॒ति॒ व॒रु॒ण॒पा॒शाद् व॑रुणपा॒शाद् य॑जति यजति वरुणपा॒शा दे॒वैव व॑रुणपा॒शाद् य॑जति यजति वरुणपा॒शा दे॒व । \newline
24. व॒रु॒ण॒पा॒शा दे॒वैव व॑रुणपा॒शाद् व॑रुणपा॒शा दे॒वैन॑ मेन मे॒व व॑रुणपा॒शाद् व॑रुणपा॒शा दे॒वैन᳚म् । \newline
25. व॒रु॒ण॒पा॒शादिति॑ वरुण - पा॒शात् । \newline
26. ए॒वैन॑ मेन मे॒वै वैन॑म् मुञ्चति मुञ्च त्येन मे॒वै वैन॑म् मुञ्चति । \newline
27. ए॒न॒म् मु॒ञ्च॒ति॒ मु॒ञ्च॒ त्ये॒न॒ मे॒न॒म् मु॒ञ्च॒ त्य॒ग्नीवरु॑णा व॒ग्नीवरु॑णौ मुञ्च त्येन मेनम् मुञ्च त्य॒ग्नीवरु॑णौ । \newline
28. मु॒ञ्च॒ त्य॒ग्नीवरु॑णा व॒ग्नीवरु॑णौ मुञ्चति मुञ्च त्य॒ग्नीवरु॑णौ यजति यज त्य॒ग्नीवरु॑णौ मुञ्चति मुञ्च त्य॒ग्नीवरु॑णौ यजति । \newline
29. अ॒ग्नीवरु॑णौ यजति यज त्य॒ग्नीवरु॑णा व॒ग्नीवरु॑णौ यजति सा॒क्षाथ् सा॒क्षाद् य॑ज त्य॒ग्नीवरु॑णा व॒ग्नीवरु॑णौ यजति सा॒क्षात् । \newline
30. अ॒ग्नीवरु॑णा॒वित्य॒ग्नी - वरु॑णौ । \newline
31. य॒ज॒ति॒ सा॒क्षाथ् सा॒क्षाद् य॑जति यजति सा॒क्षा दे॒वैव सा॒क्षाद् य॑जति यजति सा॒क्षा दे॒व । \newline
32. सा॒क्षा दे॒वैव सा॒क्षाथ् सा॒क्षा दे॒वैन॑ मेन मे॒व सा॒क्षाथ् सा॒क्षा दे॒वैन᳚म् । \newline
33. सा॒क्षादिति॑ स - अ॒क्षात् । \newline
34. ए॒वैन॑ मेन मे॒वै वैनं॑ ॅवरुणपा॒शाद् व॑रुणपा॒शा दे॑न मे॒वै वैनं॑ ॅवरुणपा॒शात् । \newline
35. ए॒नं॒ ॅव॒रु॒ण॒पा॒शाद् व॑रुणपा॒शा दे॑न मेनं ॅवरुणपा॒शान् मु॑ञ्चति मुञ्चति वरुणपा॒शा दे॑न मेनं ॅवरुणपा॒शान् मु॑ञ्चति । \newline
36. व॒रु॒ण॒पा॒शान् मु॑ञ्चति मुञ्चति वरुणपा॒शाद् व॑रुणपा॒शान् मु॑ञ्च॒ त्यप॑बर्.हिषा॒ वप॑बर्.हिषौ मुञ्चति वरुणपा॒शाद् व॑रुणपा॒शान् मु॑ञ्च॒ त्यप॑बर्.हिषौ । \newline
37. व॒रु॒ण॒पा॒शादिति॑ वरुण - पा॒शात् । \newline
38. मु॒ञ्च॒ त्यप॑बर्.हिषा॒ वप॑बर्.हिषौ मुञ्चति मुञ्च॒ त्यप॑बर्.हिषा वनूया॒जा व॑नूया॒जा वप॑बर्.हिषौ मुञ्चति मुञ्च॒ त्यप॑बर्.हिषा वनूया॒जौ । \newline
39. अप॑बर्.हिषा वनूया॒जा व॑नूया॒जा वप॑बर्.हिषा॒ वप॑बर्.हिषा वनूया॒जौ य॑जति यज त्यनूया॒जा वप॑बर्.हिषा॒ वप॑बर्.हिषा वनूया॒जौ य॑जति । \newline
40. अप॑बर्.हिषा॒वित्यप॑ - ब॒र्॒.हि॒षौ॒ । \newline
41. अ॒नू॒या॒जौ य॑जति यज त्यनूया॒जा व॑नूया॒जौ य॑जति प्र॒जाः प्र॒जा य॑ज त्यनूया॒जा व॑नूया॒जौ य॑जति प्र॒जाः । \newline
42. अ॒नू॒या॒जावित्य॑नु - या॒जौ । \newline
43. य॒ज॒ति॒ प्र॒जाः प्र॒जा य॑जति यजति प्र॒जा वै वै प्र॒जा य॑जति यजति प्र॒जा वै । \newline
44. प्र॒जा वै वै प्र॒जाः प्र॒जा वै ब॒र्॒.हिर् ब॒र्॒.हिर् वै प्र॒जाः प्र॒जा वै ब॒र्॒.हिः । \newline
45. प्र॒जा इति॑ प्र - जाः । \newline
46. वै ब॒र्॒.हिर् ब॒र्॒.हिर् वै वै ब॒र्॒.हिः प्र॒जाः प्र॒जा ब॒र्॒.हिर् वै वै ब॒र्॒.हिः प्र॒जाः । \newline
47. ब॒र्॒.हिः प्र॒जाः प्र॒जा ब॒र्॒.हिर् ब॒र्॒.हिः प्र॒जा ए॒वैव प्र॒जा ब॒र्॒.हिर् ब॒र्॒.हिः प्र॒जा ए॒व । \newline
48. प्र॒जा ए॒वैव प्र॒जाः प्र॒जा ए॒व व॑रुणपा॒शाद् व॑रुणपा॒शा दे॒व प्र॒जाः प्र॒जा ए॒व व॑रुणपा॒शात् । \newline
49. प्र॒जा इति॑ प्र - जाः । \newline
50. ए॒व व॑रुणपा॒शाद् व॑रुणपा॒शा दे॒वैव व॑रुणपा॒शान् मु॑ञ्चति मुञ्चति वरुणपा॒शा दे॒वैव व॑रुणपा॒शान् मु॑ञ्चति । \newline
51. व॒रु॒ण॒पा॒शान् मु॑ञ्चति मुञ्चति वरुणपा॒शाद् व॑रुणपा॒शान् मु॑ञ्चति च॒तुर॑ श्च॒तुरो॑ मुञ्चति वरुणपा॒शाद् व॑रुणपा॒शान् मु॑ञ्चति च॒तुरः॑ । \newline
52. व॒रु॒ण॒पा॒शादिति॑ वरुण - पा॒शात् । \newline
53. मु॒ञ्च॒ति॒ च॒तुर॑ श्च॒तुरो॑ मुञ्चति मुञ्चति च॒तुरः॑ प्रया॒जान् प्र॑या॒जाꣳ श्च॒तुरो॑ मुञ्चति मुञ्चति च॒तुरः॑ प्रया॒जान् । \newline
54. च॒तुरः॑ प्रया॒जान् प्र॑या॒जाꣳ श्च॒तुर॑ श्च॒तुरः॑ प्रया॒जान्. य॑जति यजति प्रया॒जाꣳ
श्च॒तुर॑ श्च॒तुरः॑ प्रया॒जान्. य॑जति । \newline
55. प्र॒या॒जान्. य॑जति यजति प्रया॒जान् प्र॑या॒जान्. य॑जति॒ द्वौ द्वौ य॑जति प्रया॒जान् प्र॑या॒जान्. य॑जति॒ द्वौ । \newline
56. प्र॒या॒जानिति॑ प्र - या॒जान् । \newline
57. य॒ज॒ति॒ द्वौ द्वौ य॑जति यजति॒ द्वा व॑नूया॒जा व॑नूया॒जौ द्वौ य॑जति यजति॒ द्वा व॑नूया॒जौ । \newline
58. द्वा व॑नूया॒जा व॑नूया॒जौ द्वौ द्वा व॑नूया॒जौ षट् थ्षड॑नूया॒जौ द्वौ द्वा व॑नूया॒जौ षट् । \newline
59. अ॒नू॒या॒जौ षट् थ्षड॑नूया॒जा व॑नूया॒जौ षट् थ्सꣳ सꣳ षड॑नूया॒जा व॑नूया॒जौ षट् थ्सम् । \newline
60. अ॒नू॒या॒जावित्य॑नु - या॒जौ । \newline
61. षट् थ्सꣳ सꣳ षट् थ्षट् थ्सम् प॑द्यन्ते पद्यन्ते॒ सꣳ षट् थ्षट् थ्सम् प॑द्यन्ते । \newline
62. सम् प॑द्यन्ते पद्यन्ते॒ सꣳ सम् प॑द्यन्ते॒ षट् थ्षट् प॑द्यन्ते॒ सꣳ सम् प॑द्यन्ते॒ षट् । \newline
63. प॒द्य॒न्ते॒ षट् थ्षट् प॑द्यन्ते पद्यन्ते॒ षड् वै वै षट् प॑द्यन्ते पद्यन्ते॒ षड् वै । \newline
64. षड् वै वै षट् थ्षड् वा ऋ॒तव॑ ऋ॒तवो॒ वै षट् थ्षड् वा ऋ॒तवः॑ । \newline
65. वा ऋ॒तव॑ ऋ॒तवो॒ वै वा ऋ॒तव॑ ऋ॒तुष् वृ॒तु ष्वृ॒तवो॒ वै वा ऋ॒तव॑ ऋ॒तुषु॑ । \newline
66. ऋ॒तव॑ ऋ॒तुष् वृ॒तुष् वृ॒तव॑ ऋ॒तव॑ ऋ॒तुष् वे॒वैव र्‌तुष् वृ॒तव॑ ऋ॒तव॑ ऋ॒तु ष्वे॒व । \newline
\pagebreak
\markright{ TS 6.6.3.4  \hfill https://www.vedavms.in \hfill}

\section{ TS 6.6.3.4 }

\textbf{TS 6.6.3.4 } \newline
\textbf{Samhita Paata} \newline

ऋ॒तुष्वे॒व प्रति॑ तिष्ठ॒-त्यव॑भृथ-निचङ्कु॒णेत्या॑ह यथोदि॒तमे॒व वरु॑ण॒मव॑ यजते समु॒द्रे ते॒ हृद॑य-म॒फ्स्व॑न्तरित्या॑ह समु॒द्रे ह्य॑न्तर्वरु॑णः॒ सं त्वा॑ विश॒-न्त्वोष॑धी-रु॒ताऽऽ*प॒ इत्या॑हा॒द्भि-रे॒वैन॒मोष॑धीभिः स॒म्यञ्चं॑ दधाति॒ देवी॑राप ए॒ष वो॒ गर्भ॒ इत्या॑ह यथाय॒जुरे॒वैतत् प॒शवो॒ वै- [  ] \newline

\textbf{Pada Paata} \newline

ऋ॒तुषु॑ । ए॒व । प्रतीति॑ । ति॒ष्ठ॒ति॒ । अव॑भृ॒थेत्यव॑ - भृ॒थ॒ । नि॒च॒ङ्कु॒णेति॑ नि - च॒ङ्कु॒ण॒ । इति॑ । आ॒ह॒ । य॒थो॒दि॒तमिति॑ यथा - उ॒दि॒तम् । ए॒व । वरु॑णम् । अवेति॑ । य॒ज॒ते॒ । स॒मु॒द्रे । ते॒ । हृद॑यम् । अ॒फ्स्वित्य॑प्-सु । अ॒न्तः । इति॑ । आ॒ह॒ । स॒मु॒द्रे । हि । अ॒न्तः । वरु॑णः । समिति॑ । त्वा॒ । वि॒श॒न्तु॒ । ओष॑धीः । उ॒त । आपः॑ । इति॑ । आ॒ह॒ । अ॒द्भिरित्य॑त्-भिः । ए॒व । ए॒न॒म् । ओष॑धीभि॒रित्योष॑धि - भिः॒ । स॒म्यञ्च᳚म् । द॒धा॒ति॒ । देवीः᳚ । आ॒पः॒ । ए॒षः । वः॒ । गर्भः॑ । इति॑ । आ॒ह॒ । य॒था॒य॒जुरिति॑ यथा - य॒जुः । ए॒व । ए॒तत् । प॒शवः॑ । वै ।  \newline


\textbf{Krama Paata} \newline

ऋ॒तुष्वे॒व । ए॒व प्रति॑ । प्रति॑ तिष्ठति । ति॒ष्ठ॒त्यव॑भृथ । अव॑भृथ निचङ्कुण । अव॑भृ॒थेत्यव॑ - भृ॒थ॒ । नि॒च॒ङ्‍॒कु॒णेति॑ । नि॒च॒ङ्‍॒कु॒णेति॑ नि - च॒ङ्‍॒कु॒ण॒ । इत्या॑ह । आ॒ह॒ य॒थो॒दि॒तम् । य॒थो॒दि॒तमे॒व । य॒थो॒दि॒तमिति॑ यथा - उ॒दि॒तम् । ए॒व वरु॑णम् । वरु॑ण॒मव॑ । अव॑ यजते । य॒ज॒ते॒ स॒मु॒द्रे । स॒मु॒द्रे ते᳚ । ते॒ हृद॑यम् । हृद॑यम॒फ्सु । अ॒फ्स्व॑न्तः । अ॒फ्स्वित्य॑प् - सु । अ॒न्तरिति॑ । इत्या॑ह । आ॒ह॒ स॒मु॒द्रे । स॒मु॒द्रे हि । ह्य॑न्तः । अ॒न्तर् वरु॑णः । वरु॑णः॒ सम् । सम् त्वा᳚ । त्वा॒ वि॒श॒न्तु॒ । वि॒श॒न्त्वोष॑धीः । ओष॑धीरु॒त । उ॒तापः॑ । आप॒ इति॑ । इत्या॑ह । आ॒हा॒द्‌भिः । अ॒द्‌भिरे॒व । अ॒द्‌भिरित्य॑त् - भिः । ए॒वैन᳚म् । ए॒न॒मोष॑धीभिः । ओष॑धीभिः स॒म्यञ्च᳚म् । ओष॑धीभि॒रत्योष॑धि - भिः॒ । स॒म्यञ्च॑म् दधाति । द॒धा॒ति॒ देवीः᳚ । देवी॑रापः । आ॒प॒ ए॒षः । ए॒ष वः॑ । वो॒ गर्भः॑ । गर्भ॒ इति॑ । इत्या॑ह । आ॒ह॒ य॒था॒य॒जुः । य॒था॒य॒जुरे॒व । य॒था॒य॒जुरिति॑ यथा - य॒जुः । ए॒वैतत् । ए॒तत् प॒शवः॑ । प॒शवो॒ वै । वै सोमः॑ \newline

\textbf{Jatai Paata} \newline

1. ऋ॒तु ष्वे॒वैव र्‌तुष् वृ॒तुष् वे॒व । \newline
2. ए॒व प्रति॒ प्रत्ये॒ वैव प्रति॑ । \newline
3. प्रति॑ तिष्ठति तिष्ठति॒ प्रति॒ प्रति॑ तिष्ठति । \newline
4. ति॒ष्ठ॒ त्यव॑भृ॒था व॑भृथ तिष्ठति तिष्ठ॒ त्यव॑भृथ । \newline
5. अव॑भृथ निचङ्कुण निचङ्कु॒णा व॑भृ॒था व॑भृथ निचङ्कुण । \newline
6. अव॑भृ॒थेत्यव॑ - भृ॒थ॒ । \newline
7. नि॒च॒ङ्कु॒ णेतीति॑ निचङ्कुण निचङ्कु॒णेति॑ । \newline
8. नि॒च॒ङ्कु॒णेति॑ नि - च॒ङ्कु॒ण॒ । \newline
9. इत्या॑हा॒हे तीत्या॑ह । \newline
10. आ॒ह॒ य॒थो॒दि॒तं ॅय॑थोदि॒त मा॑हाह यथोदि॒तम् । \newline
11. य॒थो॒दि॒त मे॒वैव य॑थोदि॒तं ॅय॑थोदि॒त मे॒व । \newline
12. य॒थो॒दि॒तमिति॑ यथा - उ॒दि॒तम् । \newline
13. ए॒व वरु॑णं॒ ॅवरु॑ण मे॒वैव वरु॑णम् । \newline
14. वरु॑ण॒ मवाव॒ वरु॑णं॒ ॅवरु॑ण॒ मव॑ । \newline
15. अव॑ यजते यज॒ते ऽवाव॑ यजते । \newline
16. य॒ज॒ते॒ स॒मु॒द्रे स॑मु॒द्रे य॑जते यजते समु॒द्रे । \newline
17. स॒मु॒द्रे ते॑ ते समु॒द्रे स॑मु॒द्रे ते᳚ । \newline
18. ते॒ हृद॑यꣳ॒॒ हृद॑यम् ते ते॒ हृद॑यम् । \newline
19. हृद॑य म॒फ्स्व॑फ्सु हृद॑यꣳ॒॒ हृद॑य म॒फ्सु । \newline
20. अ॒फ्स्व॑न्त र॒न्त र॒फ्स्वा᳚(1॒) फ्स्व॑न्तः । \newline
21. अ॒फ्स्वित्य॑प् - सु । \newline
22. अ॒न्त रिती त्य॒न्त र॒न्त रिति॑ । \newline
23. इत्या॑हा॒हे तीत्या॑ह । \newline
24. आ॒ह॒ स॒मु॒द्रे स॑मु॒द्र आ॑हाह समु॒द्रे । \newline
25. स॒मु॒द्रे हि हि स॑मु॒द्रे स॑मु॒द्रे हि । \newline
26. ह्य॑न्त र॒न्तर्. हि ह्य॑न्तः । \newline
27. अ॒न्तर् वरु॑णो॒ वरु॑णो॒ ऽन्त र॒न्तर् वरु॑णः । \newline
28. वरु॑णः॒ सꣳ सं ॅवरु॑णो॒ वरु॑णः॒ सम् । \newline
29. सम् त्वा᳚ त्वा॒ सꣳ सम् त्वा᳚ । \newline
30. त्वा॒ वि॒श॒न्तु॒ वि॒श॒न्तु॒ त्वा॒ त्वा॒ वि॒श॒न्तु॒ । \newline
31. वि॒श॒ न्त्वोष॑धी॒ रोष॑धीर् विशन्तु विश॒ न्त्वोष॑धीः । \newline
32. ओष॑धी रु॒तो तौष॑धी॒ रोष॑धी रु॒त । \newline
33. उ॒ताप॒ आप॑ उ॒तो तापः॑ । \newline
34. आप॒ इतीत्याप॒ आप॒ इति॑ । \newline
35. इत्या॑हा॒हे तीत्या॑ह । \newline
36. आ॒हा॒द्भि र॒द्भि रा॑हा हा॒द्भिः । \newline
37. अ॒द्भि रे॒वै वाद्भि र॒द्भि रे॒व । \newline
38. अ॒द्भिरित्य॑त् - भिः । \newline
39. ए॒वैन॑ मेन मे॒वै वैन᳚म् । \newline
40. ए॒न॒ मोष॑धीभि॒ रोष॑धीभि रेन मेन॒ मोष॑धीभिः । \newline
41. ओष॑धीभिः स॒म्यञ्चꣳ॑ स॒म्यञ्च॒ मोष॑धीभि॒ रोष॑धीभिः स॒म्यञ्च᳚म् । \newline
42. ओष॑धीभि॒रित्योष॑धि - भिः॒ । \newline
43. स॒म्यञ्च॑म् दधाति दधाति स॒म्यञ्चꣳ॑ स॒म्यञ्च॑म् दधाति । \newline
44. द॒धा॒ति॒ देवी॒र् देवी᳚र् दधाति दधाति॒ देवीः᳚ । \newline
45. देवी॑ राप आपो॒ देवी॒र् देवी॑ रापः । \newline
46. आ॒प॒ ए॒ष ए॒ष आ॑प आप ए॒षः । \newline
47. ए॒ष वो॑ व ए॒ष ए॒ष वः॑ । \newline
48. वो॒ गर्भो॒ गर्भो॑ वो वो॒ गर्भः॑ । \newline
49. गर्भ॒ इतीति॒ गर्भो॒ गर्भ॒ इति॑ । \newline
50. इत्या॑हा॒हे तीत्या॑ह । \newline
51. आ॒ह॒ य॒था॒य॒जुर् य॑थाय॒जु रा॑हाह यथाय॒जुः । \newline
52. य॒था॒य॒जु रे॒वैव य॑थाय॒जुर् य॑थाय॒जु रे॒व । \newline
53. य॒था॒य॒जुरिति॑ यथा - य॒जुः । \newline
54. ए॒वैत दे॒त दे॒वै वैतत् । \newline
55. ए॒तत् प॒शवः॑ प॒शव॑ ए॒त दे॒तत् प॒शवः॑ । \newline
56. प॒शवो॒ वै वै प॒शवः॑ प॒शवो॒ वै । \newline
57. वै सोमः॒ सोमो॒ वै वै सोमः॑ । \newline

\textbf{Ghana Paata } \newline

1. ऋ॒तुष् वे॒वैव र्‌तुष् वृ॒तु ष्वे॒व प्रति॒ प्रत्ये॒व र्‌तुष् वृ॒तु ष्वे॒व प्रति॑ । \newline
2. ए॒व प्रति॒ प्रत्ये॒ वैव प्रति॑ तिष्ठति तिष्ठति॒ प्रत्ये॒वैव प्रति॑ तिष्ठति । \newline
3. प्रति॑ तिष्ठति तिष्ठति॒ प्रति॒ प्रति॑ तिष्ठ॒ त्यव॑भृ॒था व॑भृथ तिष्ठति॒ प्रति॒ प्रति॑ तिष्ठ॒ त्यव॑भृथ । \newline
4. ति॒ष्ठ॒ त्यव॑भृ॒था व॑भृथ तिष्ठति तिष्ठ॒ त्यव॑भृथ निचङ्कुण निचङ्कु॒णा व॑भृथ तिष्ठति तिष्ठ॒ त्यव॑भृथ निचङ्कुण । \newline
5. अव॑भृथ निचङ्कुण निचङ्कु॒णा व॑भृ॒था व॑भृथ निचङ्कु॒णेतीति॑ निचङ्कु॒णा व॑भृ॒था व॑भृथ निचङ्कु॒णेति॑ । \newline
6. अव॑भृ॒थेत्यव॑ - भृ॒थ॒ । \newline
7. नि॒च॒ङ्कु॒णेतीति॑ निचङ्कुण निचङ्कु॒णे त्या॑हा॒हेति॑ निचङ्कुण निचङ्कु॒णे त्या॑ह । \newline
8. नि॒च॒ङ्कु॒णेति॑ नि - च॒ङ्कु॒ण॒ । \newline
9. इत्या॑हा॒हे तीत्या॑ह यथोदि॒तं ॅय॑थोदि॒त मा॒हे तीत्या॑ह यथोदि॒तम् । \newline
10. आ॒ह॒ य॒थो॒दि॒तं ॅय॑थोदि॒त मा॑हाह यथोदि॒त मे॒वैव य॑थोदि॒त मा॑हाह यथोदि॒त मे॒व । \newline
11. य॒थो॒दि॒त मे॒वैव य॑थोदि॒तं ॅय॑थोदि॒त मे॒व वरु॑णं॒ ॅवरु॑ण मे॒व य॑थोदि॒तं ॅय॑थोदि॒त मे॒व वरु॑णम् । \newline
12. य॒थो॒दि॒तमिति॑ यथा - उ॒दि॒तम् । \newline
13. ए॒व वरु॑णं॒ ॅवरु॑ण मे॒वैव वरु॑ण॒ मवाव॒ वरु॑ण मे॒वैव वरु॑ण॒ मव॑ । \newline
14. वरु॑ण॒ मवाव॒ वरु॑णं॒ ॅवरु॑ण॒ मव॑ यजते यज॒ते ऽव॒ वरु॑णं॒ ॅवरु॑ण॒ मव॑ यजते । \newline
15. अव॑ यजते यज॒ते ऽवाव॑ यजते समु॒द्रे स॑मु॒द्रे य॑ज॒ते ऽवाव॑ यजते समु॒द्रे । \newline
16. य॒ज॒ते॒ स॒मु॒द्रे स॑मु॒द्रे य॑जते यजते समु॒द्रे ते॑ ते समु॒द्रे य॑जते यजते समु॒द्रे ते᳚ । \newline
17. स॒मु॒द्रे ते॑ ते समु॒द्रे स॑मु॒द्रे ते॒ हृद॑यꣳ॒॒ हृद॑यम् ते समु॒द्रे स॑मु॒द्रे ते॒ हृद॑यम् । \newline
18. ते॒ हृद॑यꣳ॒॒ हृद॑यम् ते ते॒ हृद॑य म॒फ्स्व॑फ्सु हृद॑यम् ते ते॒ हृद॑य म॒फ्सु । \newline
19. हृद॑य म॒फ्स्व॑फ्सु हृद॑यꣳ॒॒ हृद॑य म॒फ्स्व॑न्त र॒न्त र॒फ्सु हृद॑यꣳ॒॒ हृद॑य म॒फ्स्व॑न्तः । \newline
20. अ॒फ्स्व॑न्त र॒न्त र॒फ्स्वा᳚(1॒) फ्स्व॑न्त रिती त्य॒न्त र॒फ्स्वा᳚(1॒) फ्स्व॑न्त रिति॑ । \newline
21. अ॒फ्स्वित्य॑प् - सु । \newline
22. अ॒न्त रिती त्य॒न्त र॒न्त रित्या॑हा॒हे त्य॒न्त र॒न्त रित्या॑ह । \newline
23. इत्या॑हा॒हे तीत्या॑ह समु॒द्रे स॑मु॒द्र आ॒हे तीत्या॑ह समु॒द्रे । \newline
24. आ॒ह॒ स॒मु॒द्रे स॑मु॒द्र आ॑हाह समु॒द्रे हि हि स॑मु॒द्र आ॑हाह समु॒द्रे हि । \newline
25. स॒मु॒द्रे हि हि स॑मु॒द्रे स॑मु॒द्रे ह्य॑न्त र॒न्तर्. हि स॑मु॒द्रे स॑मु॒द्रे ह्य॑न्तः । \newline
26. ह्य॑न्त र॒न्तर्. हि ह्य॑न्तर् वरु॑णो॒ वरु॑णो॒ ऽन्तर्. हि ह्य॑न्तर् वरु॑णः । \newline
27. अ॒न्तर् वरु॑णो॒ वरु॑णो॒ ऽन्त र॒न्तर् वरु॑णः॒ सꣳ सं ॅवरु॑णो॒ ऽन्त र॒न्तर् वरु॑णः॒ सम् । \newline
28. वरु॑णः॒ सꣳ सं ॅवरु॑णो॒ वरु॑णः॒ सम् त्वा᳚ त्वा॒ सं ॅवरु॑णो॒ वरु॑णः॒ सम् त्वा᳚ । \newline
29. सम् त्वा᳚ त्वा॒ सꣳ सम् त्वा॑ विशन्तु विशन्तु त्वा॒ सꣳ सम् त्वा॑ विशन्तु । \newline
30. त्वा॒ वि॒श॒न्तु॒ वि॒श॒न्तु॒ त्वा॒ त्वा॒ वि॒श॒ न्त्वोष॑धी॒ रोष॑धीर् विशन्तु त्वा त्वा विश॒ न्त्वोष॑धीः । \newline
31. वि॒श॒ न्त्वोष॑धी॒ रोष॑धीर् विशन्तु विश॒ न्त्वोष॑धी रु॒तोतौष॑धीर् विशन्तु विश॒ न्त्वोष॑धी रु॒त । \newline
32. ओष॑धी रु॒तो तौष॑धी॒ रोष॑धी रु॒ताप॒ आप॑ उ॒तौष॑धी॒ रोष॑धी रु॒तापः॑ । \newline
33. उ॒ताप॒ आप॑ उ॒तोताप॒ इती त्याप॑ उ॒तोताप॒ इति॑ । \newline
34. आप॒ इती त्याप॒ आप॒ इत्या॑हा॒हे त्याप॒ आप॒ इत्या॑ह । \newline
35. इत्या॑हा॒हे तीत्या॑ हा॒द्भि र॒द्भि रा॒हे तीत्या॑ हा॒द्भिः । \newline
36. आ॒हा॒द्भि र॒द्भि रा॑हा हा॒द्भि रे॒वै वाद्भि रा॑हा हा॒द्भि रे॒व । \newline
37. अ॒द्भि रे॒वै वाद्भि र॒द्भि रे॒वैन॑ मेन मे॒वाद्भि र॒द्भि रे॒वैन᳚म् । \newline
38. अ॒द्भिरित्य॑त् - भिः । \newline
39. ए॒वैन॑ मेन मे॒वै वैन॒ मोष॑धीभि॒ रोष॑धीभि रेन मे॒वै वैन॒ मोष॑धीभिः । \newline
40. ए॒न॒ मोष॑धीभि॒ रोष॑धीभि रेन मेन॒ मोष॑धीभिः स॒म्यञ्चꣳ॑ स॒म्यञ्च॒ मोष॑धीभि रेन मेन॒ मोष॑धीभिः स॒म्यञ्च᳚म् । \newline
41. ओष॑धीभिः स॒म्यञ्चꣳ॑ स॒म्यञ्च॒ मोष॑धीभि॒ रोष॑धीभिः स॒म्यञ्च॑म् दधाति दधाति स॒म्यञ्च॒ मोष॑धीभि॒ रोष॑धीभिः स॒म्यञ्च॑म् दधाति । \newline
42. ओष॑धीभि॒रित्योष॑धि - भिः॒ । \newline
43. स॒म्यञ्च॑म् दधाति दधाति स॒म्यञ्चꣳ॑ स॒म्यञ्च॑म् दधाति॒ देवी॒र् देवी᳚र् दधाति स॒म्यञ्चꣳ॑ स॒म्यञ्च॑म् दधाति॒ देवीः᳚ । \newline
44. द॒धा॒ति॒ देवी॒र् देवी᳚र् दधाति दधाति॒ देवी॑ राप आपो॒ देवी᳚र् दधाति दधाति॒ देवी॑ रापः । \newline
45. देवी॑ राप आपो॒ देवी॒र् देवी॑ राप ए॒ष ए॒ष आ॑पो॒ देवी॒र् देवी॑ राप ए॒षः । \newline
46. आ॒प॒ ए॒ष ए॒ष आ॑प आप ए॒ष वो॑ व ए॒ष आ॑प आप ए॒ष वः॑ । \newline
47. ए॒ष वो॑ व ए॒ष ए॒ष वो॒ गर्भो॒ गर्भो॑ व ए॒ष ए॒ष वो॒ गर्भः॑ । \newline
48. वो॒ गर्भो॒ गर्भो॑ वो वो॒ गर्भ॒ इतीति॒ गर्भो॑ वो वो॒ गर्भ॒ इति॑ । \newline
49. गर्भ॒ इतीति॒ गर्भो॒ गर्भ॒ इत्या॑हा॒ हेति॒ गर्भो॒ गर्भ॒ इत्या॑ह । \newline
50. इत्या॑हा॒हे तीत्या॑ह यथाय॒जुर् य॑थाय॒जु रा॒हे तीत्या॑ह यथाय॒जुः । \newline
51. आ॒ह॒ य॒था॒य॒जुर् य॑थाय॒जु रा॑हाह यथाय॒जु रे॒वैव य॑थाय॒जु रा॑हाह यथाय॒जु रे॒व । \newline
52. य॒था॒य॒जु रे॒वैव य॑थाय॒जुर् य॑थाय॒जु रे॒वैत दे॒त दे॒व य॑थाय॒जुर् य॑थाय॒जु रे॒वैतत् । \newline
53. य॒था॒य॒जुरिति॑ यथा - य॒जुः । \newline
54. ए॒वैत दे॒त दे॒वैवैतत् प॒शवः॑ प॒शव॑ ए॒त दे॒वैवैतत् प॒शवः॑ । \newline
55. ए॒तत् प॒शवः॑ प॒शव॑ ए॒त दे॒तत् प॒शवो॒ वै वै प॒शव॑ ए॒त दे॒तत् प॒शवो॒ वै । \newline
56. प॒शवो॒ वै वै प॒शवः॑ प॒शवो॒ वै सोमः॒ सोमो॒ वै प॒शवः॑ प॒शवो॒ वै सोमः॑ । \newline
57. वै सोमः॒ सोमो॒ वै वै सोमो॒ यद् यथ् सोमो॒ वै वै सोमो॒ यत् । \newline
\pagebreak
\markright{ TS 6.6.3.5  \hfill https://www.vedavms.in \hfill}

\section{ TS 6.6.3.5 }

\textbf{TS 6.6.3.5 } \newline
\textbf{Samhita Paata} \newline

सोमो॒ यद्-भि॑न्दू॒नां भ॒क्षये᳚त् पशु॒मान्थ्-स्या॒द्-वरु॑ण॒-स्त्वे॑नं गृह्णीया॒द्यन्न भ॒क्षये॑दप॒शुः स्या॒न्नैनं॒ ॅवरु॑णो गृह्णीया-दुप॒स्पृश्य॑मे॒व प॑शु॒मान् भ॑वति॒ नैनं॒ ॅवरु॑णो गृह्णाति॒ प्रति॑युतो॒ वरु॑णस्य॒ पाश॒ इत्या॑ह वरुणपा॒शादे॒व निर्मु॑च्य॒ते ऽप्र॑तीक्ष॒मा य॑न्ति॒ वरु॑णस्या॒न्तर्.हि॑त्या॒ एधो᳚ऽस्येधिषी॒मही-त्या॑ह स॒मिधै॒वाग्निं न॑म॒स्यन्त॑ ( ) उ॒पाय॑न्ति॒ तेजो॑ऽसि॒ तेजो॒ मयि॑ धे॒हीत्या॑ह॒ तेज॑ ए॒वाऽऽत्मन् ध॑त्ते ॥ \newline

\textbf{Pada Paata} \newline

सोमः॑ । यत् । भि॒न्दू॒नाम् । भ॒क्षये᳚त् । प॒शु॒मानिति॑ पशु - मान् । स्या॒त् । वरु॑णः । तु । ए॒न॒म् । गृ॒ह्णी॒या॒त् । यत् । न । भ॒क्षये᳚त् । अ॒प॒शुः । स्या॒त् । न । ए॒न॒म् । वरु॑णः । गृ॒ह्णी॒या॒त् । उ॒प॒स्पृश्य॒मित्यु॑प - स्पृश्य᳚म् । ए॒व । प॒शु॒मानिति॑ पशु - मान् । भ॒व॒ति॒ । न । ए॒न॒म् । वरु॑णः । गृ॒ह्णा॒ति॒ । प्रति॑युत॒ इति॒ प्रति॑ - यु॒तः॒ । वरु॑णस्य । पाशः॑ । इति॑ । आ॒ह॒ । व॒रु॒ण॒पा॒शादिति॑ वरुण - पा॒शात् । ए॒व । निरिति॑ । मु॒च्य॒ते॒ । अप्र॑तीक्ष॒मित्यप्र॑ति - ई॒क्ष॒म् । एति॑ । य॒न्ति॒ । वरु॑णस्य । अ॒न्तर्.हि॑त्या॒ इत्य॒न्तः - हि॒त्यै॒ । एधः॑ । अ॒सि॒ । ए॒धि॒षी॒महि॑ । इति॑ । आ॒ह॒ । स॒मिधेति॑ सम् - इधा᳚ । ए॒व । अ॒ग्निम् । न॒म॒स्यन्तः॑ ( ) । उ॒पाय॒न्तीत्यु॑प - आय॑न्ति । तेजः॑ । अ॒सि॒ । तेजः॑ । मयि॑ । धे॒हि॒ । इति॑ । आ॒ह॒ । तेजः॑ । ए॒व । आ॒त्मन्न् । ध॒त्ते॒ ॥  \newline


\textbf{Krama Paata} \newline

सोमो॒ यत् । यद् भि॑न्दू॒नाम् । भि॒न्दू॒नाम् भ॒क्षये᳚त् । भ॒क्षये᳚त् पशु॒मान् । प॒शु॒मान्थ् स्या᳚त् । प॒शु॒मानिति॑ पशु - मान् । स्या॒द् वरु॑णः । वरु॑ण॒स्तु । त्वे॑नम् । ए॒न॒म् गृ॒ह्णी॒या॒त्॒ । गृ॒ह्णी॒या॒द् यत् । यन् न । न भ॒क्षये᳚त् । भ॒क्षये॑दप॒शुः । अ॒प॒शुः स्या᳚त् । स्या॒न् न । नैन᳚म् । ए॒न॒म् ॅवरु॑णः । वरु॑णो गृह्णीयात् । गृ॒ह्णी॒या॒दु॒प॒स्पृश्य᳚म् । उ॒प॒स्पृश्य॑मे॒व । उ॒प॒स्पृश्य॒मित्यु॑प - स्पृश्य᳚म् । ए॒व प॑शु॒मान् । प॒शु॒मान् भ॑वति । प॒शु॒मानिति॑ पशु - मान् । भ॒व॒ति॒ न । नैन᳚म् । ए॒न॒म् ॅवरु॑णः । वरु॑णो गृह्णाति । गृ॒ह्णा॒ति॒ प्रति॑युतः । प्रति॑युतो॒ वरु॑णस्य । प्रति॑युत॒ इति॒ प्रति॑ - यु॒तः॒ । वरु॑णस्य॒ पाशः॑ । पाश॒ इति॑ । इत्या॑ह । आ॒ह॒ व॒रु॒ण॒पा॒शात् । व॒रु॒ण॒पा॒शादे॒व । व॒रु॒ण॒पा॒शादिति॑ वरुण - पा॒शात् । ए॒व निः । निर् मु॑च्यते । मु॒च्य॒तेऽप्र॑तीक्षम् । अप्र॑तीक्ष॒मा । अप्र॑तीक्ष॒मित्यप्र॑ति - ई॒क्ष॒म् । आ य॑न्ति । य॒न्ति॒ वरु॑णस्य । वरु॑णस्या॒न्तर्.हि॑त्यै । अ॒न्तर्.हि॑त्या॒ एधः॑ । अ॒न्तर्.हि॑त्या॒ इत्य॒न्तः - हि॒त्यै॒ । एधो॑ऽसि । अ॒स्ये॒धि॒षी॒महि॑ । ए॒धि॒षी॒महीति॑ । इत्या॑ह । आ॒ह॒ स॒मिधा᳚ । स॒मिधै॒व । स॒मिधेति॑ सम् - इधा᳚ । ए॒वाग्निम् । अ॒ग्निम् न॑म॒स्यन्तः॑ ( ) । न॒म॒स्यन्त॑ उ॒पाय॑न्ति । उ॒पाय॑न्ति॒ तेजः॑ । उ॒पाय॒न्तीत्यु॑प - आय॑न्ति । तेजो॑ऽसि । अ॒सि॒ तेजः॑ । तेजो॒ मयि॑ । मयि॑ धेहि । धे॒हीति॑ । इत्या॑ह । आ॒ह॒ तेजः॑ । तेज॑ ए॒व । ए॒वात्मन्न् । आ॒त्मन् ध॑त्ते । ध॒त्त॒ इति॑ धत्ते । \newline

\textbf{Jatai Paata} \newline

1. सोमो॒ यद् यथ् सोमः॒ सोमो॒ यत् । \newline
2. यद् भि॑न्दू॒नाम् भि॑न्दू॒नां ॅयद् यद् भि॑न्दू॒नाम् । \newline
3. भि॒न्दू॒नाम् भ॒क्षये᳚द् भ॒क्षये᳚द् भिन्दू॒नाम् भि॑न्दू॒नाम् भ॒क्षये᳚त् । \newline
4. भ॒क्षये᳚त् पशु॒मान् प॑शु॒मान् भ॒क्षये᳚द् भ॒क्षये᳚त् पशु॒मान् । \newline
5. प॒शु॒मान् थ्स्या᳚थ् स्यात् पशु॒मान् प॑शु॒मान् थ्स्या᳚त् । \newline
6. प॒शु॒मानिति॑ पशु - मान् । \newline
7. स्या॒द् वरु॑णो॒ वरु॑णः स्याथ् स्या॒द् वरु॑णः । \newline
8. वरु॑ण॒ स्तु तु वरु॑णो॒ वरु॑ण॒ स्तु । \newline
9. त्वे॑न मेन॒म् तु त्वे॑नम् । \newline
10. ए॒न॒म् गृ॒ह्णी॒या॒द् गृ॒ह्णी॒या॒ दे॒न॒ मे॒न॒म् गृ॒ह्णी॒या॒त् । \newline
11. गृ॒ह्णी॒या॒द् यद् यद् गृ॑ह्णीयाद् गृह्णीया॒द् यत् । \newline
12. यन् न न यद् यन् न । \newline
13. न भ॒क्षये᳚द् भ॒क्षये॒न् न न भ॒क्षये᳚त् । \newline
14. भ॒क्षये॑ दप॒शु र॑प॒शुर् भ॒क्षये᳚द् भ॒क्षये॑ दप॒शुः । \newline
15. अ॒प॒शुः स्या᳚थ् स्या दप॒शु र॑प॒शुः स्या᳚त् । \newline
16. स्या॒न् न न स्या᳚थ् स्या॒न् न । \newline
17. नैन॑ मेन॒न् न नैन᳚म् । \newline
18. ए॒नं॒ ॅवरु॑णो॒ वरु॑ण एन मेनं॒ ॅवरु॑णः । \newline
19. वरु॑णो गृह्णीयाद् गृह्णीया॒द् वरु॑णो॒ वरु॑णो गृह्णीयात् । \newline
20. गृ॒ह्णी॒या॒ दु॒प॒स्पृश्य॑ मुप॒स्पृश्य॑म् गृह्णीयाद् गृह्णीया दुप॒स्पृश्य᳚म् । \newline
21. उ॒प॒स्पृश्य॑ मे॒वैवो प॒स्पृश्य॑ मुप॒स्पृश्य॑ मे॒व । \newline
22. उ॒प॒स्पृश्य॒मित्यु॑प - स्पृश्य᳚म् । \newline
23. ए॒व प॑शु॒मान् प॑शु॒मा ने॒वैव प॑शु॒मान् । \newline
24. प॒शु॒मान् भ॑वति भवति पशु॒मान् प॑शु॒मान् भ॑वति । \newline
25. प॒शु॒मानिति॑ पशु - मान् । \newline
26. भ॒व॒ति॒ न न भ॑वति भवति॒ न । \newline
27. नैन॑ मेन॒न् न नैन᳚म् । \newline
28. ए॒नं॒ ॅवरु॑णो॒ वरु॑ण एन मेनं॒ ॅवरु॑णः । \newline
29. वरु॑णो गृह्णाति गृह्णाति॒ वरु॑णो॒ वरु॑णो गृह्णाति । \newline
30. गृ॒ह्णा॒ति॒ प्रति॑युतः॒ प्रति॑युतो गृह्णाति गृह्णाति॒ प्रति॑युतः । \newline
31. प्रति॑युतो॒ वरु॑णस्य॒ वरु॑णस्य॒ प्रति॑युतः॒ प्रति॑युतो॒ वरु॑णस्य । \newline
32. प्रति॑युत॒ इति॒ प्रति॑ - यु॒तः॒ । \newline
33. वरु॑णस्य॒ पाशः॒ पाशो॒ वरु॑णस्य॒ वरु॑णस्य॒ पाशः॑ । \newline
34. पाश॒ इतीति॒ पाशः॒ पाश॒ इति॑ । \newline
35. इत्या॑हा॒हे तीत्या॑ह । \newline
36. आ॒ह॒ व॒रु॒ण॒पा॒शाद् व॑रुणपा॒शा दा॑हाह वरुणपा॒शात् । \newline
37. व॒रु॒ण॒पा॒शा दे॒वैव व॑रुणपा॒शाद् व॑रुणपा॒शा दे॒व । \newline
38. व॒रु॒ण॒पा॒शादिति॑ वरुण - पा॒शात् । \newline
39. ए॒व निर् णिरे॒वैव निः । \newline
40. निर् मु॑च्यते मुच्यते॒ निर् णिर् मु॑च्यते । \newline
41. मु॒च्य॒ते ऽप्र॑तीक्ष॒ मप्र॑तीक्षम् मुच्यते मुच्य॒ते ऽप्र॑तीक्षम् । \newline
42. अप्र॑तीक्ष॒ मा ऽप्र॑तीक्ष॒ मप्र॑तीक्ष॒ मा । \newline
43. अप्र॑तीक्ष॒मित्यप्र॑ति - ई॒क्ष॒म् । \newline
44. आ य॑न्ति य॒न्त्या य॑न्ति । \newline
45. य॒न्ति॒ वरु॑णस्य॒ वरु॑णस्य यन्ति यन्ति॒ वरु॑णस्य । \newline
46. वरु॑ण स्या॒न्तर्.हि॑त्या अ॒न्तर्.हि॑त्यै॒ वरु॑णस्य॒ वरु॑ण स्या॒न्तर्.हि॑त्यै । \newline
47. अ॒न्तर्.हि॑त्या॒ एध॒ एधो॒ ऽन्तर्.हि॑त्या अ॒न्तर्.हि॑त्या॒ एधः॑ । \newline
48. अ॒न्तर्.हि॑त्या॒ इत्य॒न्तः - हि॒त्यै॒ । \newline
49. एधो᳚ ऽस्य॒ स्येध॒ एधो॑ ऽसि । \newline
50. अ॒स्ये॒धि॒षी॒म ह्ये॑धिषी॒म ह्य॑स्य स्येधिषी॒महि॑ । \newline
51. ए॒धि॒षी॒म हीती त्ये॑धिषी॒म ह्ये॑धिषी॒म हीति॑ । \newline
52. इत्या॑हा॒हे तीत्या॑ह । \newline
53. आ॒ह॒ स॒मिधा॑ स॒मिधा॑ ऽऽहाह स॒मिधा᳚ । \newline
54. स॒मि धै॒वैव स॒मिधा॑ स॒मिधै॒व । \newline
55. स॒मिधेति॑ सम् - इधा᳚ । \newline
56. ए॒वाग्नि म॒ग्नि मे॒वै वाग्निम् । \newline
57. अ॒ग्निन् न॑म॒स्यन्तो॑ नम॒स्यन्तो॒ ऽग्नि म॒ग्निम् न॑म॒स्यन्तः॑ । \newline
58. न॒म॒स्यन्त॑ उ॒पाय॑ न्त्यु॒पाय॑न्ति नम॒स्यन्तो॑ नम॒स्यन्त॑ उ॒पाय॑न्ति । \newline
59. उ॒पाय॑न्ति॒ तेज॒ स्तेज॑ उ॒पाय॑ न्त्यु॒पाय॑न्ति॒ तेजः॑ । \newline
60. उ॒पाय॒न्तीत्यु॑प - आय॑न्ति । \newline
61. तेजो᳚ ऽस्यसि॒ तेज॒ स्तेजो॑ ऽसि । \newline
62. अ॒सि॒ तेज॒ स्तेजो᳚ ऽस्यसि॒ तेजः॑ । \newline
63. तेजो॒ मयि॒ मयि॒ तेज॒ स्तेजो॒ मयि॑ । \newline
64. मयि॑ धेहि धेहि॒ मयि॒ मयि॑ धेहि । \newline
65. धे॒हीतीति॑ धेहि धे॒हीति॑ । \newline
66. इत्या॑हा॒हे तीत्या॑ह । \newline
67. आ॒ह॒ तेज॒ स्तेज॑ आहाह॒ तेजः॑ । \newline
68. तेज॑ ए॒वैव तेज॒ स्तेज॑ ए॒व । \newline
69. ए॒वात्मन् ना॒त्मन् ने॒वै वात्मन्न् । \newline
70. आ॒त्मन् ध॑त्ते धत्त आ॒त्मन् ना॒त्मन् ध॑त्ते । \newline
71. ध॒त्त॒ इति॑ धत्ते । \newline

\textbf{Ghana Paata } \newline

1. सोमो॒ यद् यथ् सोमः॒ सोमो॒ यद् भि॑न्दू॒नाम् भि॑न्दू॒नां ॅयथ् सोमः॒ सोमो॒ यद् भि॑न्दू॒नाम् । \newline
2. यद् भि॑न्दू॒नाम् भि॑न्दू॒नां ॅयद् यद् भि॑न्दू॒नाम् भ॒क्षये᳚द् भ॒क्षये᳚द् भिन्दू॒नां ॅयद् यद् भि॑न्दू॒नाम् भ॒क्षये᳚त् । \newline
3. भि॒न्दू॒नाम् भ॒क्षये᳚द् भ॒क्षये᳚द् भिन्दू॒नाम् भि॑न्दू॒नाम् भ॒क्षये᳚त् पशु॒मान् प॑शु॒मान् भ॒क्षये᳚द् भिन्दू॒नाम् भि॑न्दू॒नाम् भ॒क्षये᳚त् पशु॒मान् । \newline
4. भ॒क्षये᳚त् पशु॒मान् प॑शु॒मान् भ॒क्षये᳚द् भ॒क्षये᳚त् पशु॒मान् थ्स्या᳚थ् स्यात् पशु॒मान् भ॒क्षये᳚द् भ॒क्षये᳚त् पशु॒मान् थ्स्या᳚त् । \newline
5. प॒शु॒मान् थ्स्या᳚थ् स्यात् पशु॒मान् प॑शु॒मान् थ्स्या॒द् वरु॑णो॒ वरु॑णः स्यात् पशु॒मान् प॑शु॒मान् थ्स्या॒द् वरु॑णः । \newline
6. प॒शु॒मानिति॑ पशु - मान् । \newline
7. स्या॒द् वरु॑णो॒ वरु॑णः स्याथ् स्या॒द् वरु॑ण॒ स्तु तु वरु॑णः स्याथ् स्या॒द् वरु॑ण॒ स्तु । \newline
8. वरु॑ण॒ स्तु तु वरु॑णो॒ वरु॑ण॒ स्त्वे॑न मेन॒म् तु वरु॑णो॒ वरु॑ण॒ स्त्वे॑नम् । \newline
9. त्वे॑न मेन॒म् तु त्वे॑नम् गृह्णीयाद् गृह्णीया देन॒म् तु त्वे॑नम् गृह्णीयात् । \newline
10. ए॒न॒म् गृ॒ह्णी॒या॒द् गृ॒ह्णी॒या॒ दे॒न॒ मे॒न॒म् गृ॒ह्णी॒या॒द् यद् यद् गृ॑ह्णीया देन मेनम् गृह्णीया॒द् यत् । \newline
11. गृ॒ह्णी॒या॒द् यद् यद् गृ॑ह्णीयाद् गृह्णीया॒द् यन् न न यद् गृ॑ह्णीयाद् गृह्णीया॒द् यन् न । \newline
12. यन् न न यद् यन् न भ॒क्षये᳚द् भ॒क्षये॒न् न यद् यन् न भ॒क्षये᳚त् । \newline
13. न भ॒क्षये᳚द् भ॒क्षये॒न् न न भ॒क्षये॑ दप॒शु र॑प॒शुर् भ॒क्षये॒न् न न भ॒क्षये॑ दप॒शुः । \newline
14. भ॒क्षये॑ दप॒शु र॑प॒शुर् भ॒क्षये᳚द् भ॒क्षये॑ दप॒शुः स्या᳚थ् स्या दप॒शुर् भ॒क्षये᳚द् भ॒क्षये॑ दप॒शुः स्या᳚त् । \newline
15. अ॒प॒शुः स्या᳚थ् स्या दप॒शु र॑प॒शुः स्या॒न् न न स्या॑ दप॒शु र॑प॒शुः स्या॒न् न । \newline
16. स्या॒न् न न स्या᳚थ् स्या॒न् नैन॑ मेन॒न् न स्या᳚थ् स्या॒न् नैन᳚म् । \newline
17. नैन॑ मेन॒न् न नैनं॒ ॅवरु॑णो॒ वरु॑ण एन॒न् न नैनं॒ ॅवरु॑णः । \newline
18. ए॒नं॒ ॅवरु॑णो॒ वरु॑ण एन मेनं॒ ॅवरु॑णो गृह्णीयाद् गृह्णीया॒द् वरु॑ण एन मेनं॒ ॅवरु॑णो गृह्णीयात् । \newline
19. वरु॑णो गृह्णीयाद् गृह्णीया॒द् वरु॑णो॒ वरु॑णो गृह्णीया दुप॒स्पृश्य॑ मुप॒स्पृश्य॑म् गृह्णीया॒द् वरु॑णो॒ वरु॑णो गृह्णीया दुप॒स्पृश्य᳚म् । \newline
20. गृ॒ह्णी॒या॒ दु॒प॒स्पृश्य॑ मुप॒स्पृश्य॑म् गृह्णीयाद् गृह्णीया दुप॒स्पृश्य॑ मे॒वै वोप॒स्पृश्य॑म् गृह्णीयाद् गृह्णीया दुप॒स्पृश्य॑ मे॒व । \newline
21. उ॒प॒स्पृश्य॑ मे॒वै वोप॒स्पृश्य॑ मुप॒स्पृश्य॑ मे॒व प॑शु॒मान् प॑शु॒मा ने॒वोप॒स्पृश्य॑ मुप॒स्पृश्य॑ मे॒व प॑शु॒मान् । \newline
22. उ॒प॒स्पृश्य॒मित्यु॑प - स्पृश्य᳚म् । \newline
23. ए॒व प॑शु॒मान् प॑शु॒मा ने॒वैव प॑शु॒मान् भ॑वति भवति पशु॒मा ने॒वैव प॑शु॒मान् भ॑वति । \newline
24. प॒शु॒मान् भ॑वति भवति पशु॒मान् प॑शु॒मान् भ॑वति॒ न न भ॑वति पशु॒मान् प॑शु॒मान् भ॑वति॒ न । \newline
25. प॒शु॒मानिति॑ पशु - मान् । \newline
26. भ॒व॒ति॒ न न भ॑वति भवति॒ नैन॑ मेन॒न् न भ॑वति भवति॒ नैन᳚म् । \newline
27. नैन॑ मेन॒न् न नैनं॒ ॅवरु॑णो॒ वरु॑ण एन॒न् न नैनं॒ ॅवरु॑णः । \newline
28. ए॒नं॒ ॅवरु॑णो॒ वरु॑ण एन मेनं॒ ॅवरु॑णो गृह्णाति गृह्णाति॒ वरु॑ण एन मेनं॒ ॅवरु॑णो गृह्णाति । \newline
29. वरु॑णो गृह्णाति गृह्णाति॒ वरु॑णो॒ वरु॑णो गृह्णाति॒ प्रति॑युतः॒ प्रति॑युतो गृह्णाति॒ वरु॑णो॒ वरु॑णो गृह्णाति॒ प्रति॑युतः । \newline
30. गृ॒ह्णा॒ति॒ प्रति॑युतः॒ प्रति॑युतो गृह्णाति गृह्णाति॒ प्रति॑युतो॒ वरु॑णस्य॒ वरु॑णस्य॒ प्रति॑युतो गृह्णाति गृह्णाति॒ प्रति॑युतो॒ वरु॑णस्य । \newline
31. प्रति॑युतो॒ वरु॑णस्य॒ वरु॑णस्य॒ प्रति॑युतः॒ प्रति॑युतो॒ वरु॑णस्य॒ पाशः॒ पाशो॒ वरु॑णस्य॒ प्रति॑युतः॒ प्रति॑युतो॒ वरु॑णस्य॒ पाशः॑ । \newline
32. प्रति॑युत॒ इति॒ प्रति॑ - यु॒तः॒ । \newline
33. वरु॑णस्य॒ पाशः॒ पाशो॒ वरु॑णस्य॒ वरु॑णस्य॒ पाश॒ इतीति॒ पाशो॒ वरु॑णस्य॒ वरु॑णस्य॒ पाश॒ इति॑ । \newline
34. पाश॒ इतीति॒ पाशः॒ पाश॒ इत्या॑ हा॒हेति॒ पाशः॒ पाश॒ इत्या॑ह । \newline
35. इत्या॑हा॒हे तीत्या॑ह वरुणपा॒शाद् व॑रुणपा॒शा दा॒हे तीत्या॑ह वरुणपा॒शात् । \newline
36. आ॒ह॒ व॒रु॒ण॒पा॒शाद् व॑रुणपा॒शा दा॑हाह वरुणपा॒शा दे॒वैव व॑रुणपा॒शा दा॑हाह वरुणपा॒शा दे॒व । \newline
37. व॒रु॒ण॒पा॒शा दे॒वैव व॑रुणपा॒शाद् व॑रुणपा॒शा दे॒व निर् णिरे॒व व॑रुणपा॒शाद् व॑रुणपा॒शा दे॒व निः । \newline
38. व॒रु॒ण॒पा॒शादिति॑ वरुण - पा॒शात् । \newline
39. ए॒व निर् णिरे॒वैव निर् मु॑च्यते मुच्यते॒ निरे॒वैव निर् मु॑च्यते । \newline
40. निर् मु॑च्यते मुच्यते॒ निर् णिर् मु॑च्य॒ते ऽप्र॑तीक्ष॒ मप्र॑तीक्षम् मुच्यते॒ निर् णिर् मु॑च्य॒ते ऽप्र॑तीक्षम् । \newline
41. मु॒च्य॒ते ऽप्र॑तीक्ष॒ मप्र॑तीक्षम् मुच्यते मुच्य॒ते ऽप्र॑तीक्ष॒ मा ऽप्र॑तीक्षम् मुच्यते मुच्य॒ते ऽप्र॑तीक्ष॒ मा । \newline
42. अप्र॑तीक्ष॒ मा ऽप्र॑तीक्ष॒ मप्र॑तीक्ष॒ मा य॑न्ति य॒न्त्या ऽप्र॑तीक्ष॒ मप्र॑तीक्ष॒ मा य॑न्ति । \newline
43. अप्र॑तीक्ष॒मित्यप्र॑ति - ई॒क्ष॒म् । \newline
44. आ य॑न्ति य॒न्त्या य॑न्ति॒ वरु॑णस्य॒ वरु॑णस्य य॒न्त्या य॑न्ति॒ वरु॑णस्य । \newline
45. य॒न्ति॒ वरु॑णस्य॒ वरु॑णस्य यन्ति यन्ति॒ वरु॑ण स्या॒न्तर्.हि॑त्या अ॒न्तर्.हि॑त्यै॒ वरु॑णस्य यन्ति यन्ति॒ वरु॑ण स्या॒न्तर्.हि॑त्यै । \newline
46. वरु॑ण स्या॒न्तर्.हि॑त्या अ॒न्तर्.हि॑त्यै॒ वरु॑णस्य॒ वरु॑ण स्या॒न्तर्.हि॑त्या॒ एध॒ एधो॒ ऽन्तर्.हि॑त्यै॒ वरु॑णस्य॒ वरु॑ण स्या॒न्तर्.हि॑त्या॒ एधः॑ । \newline
47. अ॒न्तर्.हि॑त्या॒ एध॒ एधो॒ ऽन्तर्.हि॑त्या अ॒न्तर्.हि॑त्या॒ एधो᳚ ऽस्य॒ स्येधो॒ ऽन्तर्.हि॑त्या अ॒न्तर्.हि॑त्या॒ एधो॑ ऽसि । \newline
48. अ॒न्तर्.हि॑त्या॒ इत्य॒न्तः - हि॒त्यै॒ । \newline
49. एधो᳚ ऽस्य॒ स्येध॒ एधो᳚ ऽस्येधिषी॒म ह्ये॑धिषी॒म ह्य॒स्येध॒ एधो᳚ ऽस्येधिषी॒महि॑ । \newline
50. अ॒स्ये॒धि॒षी॒म ह्ये॑धिषी॒मह्य॑ स्यस्येधिषी॒मही तीत्ये॑धिषी॒मह्य॑स्य स्येधिषी॒म हीति॑ । \newline
51. ए॒धि॒षी॒मही तीत्ये॑धिषी॒म ह्ये॑धिषी॒मही त्या॑हा॒हे त्ये॑धिषी॒म ह्ये॑धिषी॒मही त्या॑ह । \newline
52. इत्या॑हा॒हे तीत्या॑ह स॒मिधा॑ स॒मिधा॒ ऽऽहे तीत्या॑ह स॒मिधा᳚ । \newline
53. आ॒ह॒ स॒मिधा॑ स॒मिधा॑ ऽऽहाह स॒मि धै॒वैव स॒मिधा॑ ऽऽहाह स॒मिधै॒व । \newline
54. स॒मिधै॒वैव स॒मिधा॑ स॒मि धै॒वाग्नि म॒ग्नि मे॒व स॒मिधा॑ स॒मि धै॒वाग्निम् । \newline
55. स॒मिधेति॑ सम् - इधा᳚ । \newline
56. ए॒वाग्नि म॒ग्नि मे॒वै वाग्निम् न॑म॒स्यन्तो॑ नम॒स्यन्तो॒ ऽग्नि मे॒वै वाग्निम् न॑म॒स्यन्तः॑ । \newline
57. अ॒ग्निम् न॑म॒स्यन्तो॑ नम॒स्यन्तो॒ ऽग्नि म॒ग्निम् न॑म॒स्यन्त॑ उ॒पाय॑ न्त्यु॒पाय॑न्ति नम॒स्यन्तो॒ ऽग्नि म॒ग्निम् न॑म॒स्यन्त॑ उ॒पाय॑न्ति । \newline
58. न॒म॒स्यन्त॑ उ॒पाय॑ न्त्यु॒पाय॑न्ति नम॒स्यन्तो॑ नम॒स्यन्त॑ उ॒पाय॑न्ति॒ तेज॒ स्तेज॑ उ॒पाय॑न्ति नम॒स्यन्तो॑ नम॒स्यन्त॑ उ॒पाय॑न्ति॒ तेजः॑ । \newline
59. उ॒पाय॑न्ति॒ तेज॒ स्तेज॑ उ॒पाय॑ न्त्यु॒पाय॑न्ति॒ तेजो᳚ ऽस्यसि॒ तेज॑ उ॒पाय॑ न्त्यु॒पाय॑न्ति॒ तेजो॑ ऽसि । \newline
60. उ॒पाय॒न्तीत्यु॑प - आय॑न्ति । \newline
61. तेजो᳚ ऽस्यसि॒ तेज॒ स्तेजो॑ ऽसि॒ तेज॒ स्तेजो॑ ऽसि॒ तेज॒ स्तेजो॑ ऽसि॒ तेजः॑ । \newline
62. अ॒सि॒ तेज॒ स्तेजो᳚ ऽस्यसि॒ तेजो॒ मयि॒ मयि॒ तेजो᳚ ऽस्यसि॒ तेजो॒ मयि॑ । \newline
63. तेजो॒ मयि॒ मयि॒ तेज॒ स्तेजो॒ मयि॑ धेहि धेहि॒ मयि॒ तेज॒ स्तेजो॒ मयि॑ धेहि । \newline
64. मयि॑ धेहि धेहि॒ मयि॒ मयि॑ धे॒हीतीति॑ धेहि॒ मयि॒ मयि॑ धे॒हीति॑ । \newline
65. धे॒हीतीति॑ धेहि धे॒ही त्या॑हा॒हेति॑ धेहि धे॒ही त्या॑ह । \newline
66. इत्या॑हा॒हे तीत्या॑ह॒ तेज॒ स्तेज॑ आ॒हे तीत्या॑ह॒ तेजः॑ । \newline
67. आ॒ह॒ तेज॒ स्तेज॑ आहाह॒ तेज॑ ए॒वैव तेज॑ आहाह॒ तेज॑ ए॒व । \newline
68. तेज॑ ए॒वैव तेज॒ स्तेज॑ ए॒वात्मन् ना॒त्मन् ने॒व तेज॒ स्तेज॑ ए॒वात्मन्न् । \newline
69. ए॒वात्मन् ना॒त्मन् ने॒वै वात्मन् ध॑त्ते धत्त आ॒त्मन् ने॒वै वात्मन् ध॑त्ते । \newline
70. आ॒त्मन् ध॑त्ते धत्त आ॒त्मन् ना॒त्मन् ध॑त्ते । \newline
71. ध॒त्त॒ इति॑ धत्ते । \newline
\pagebreak
\markright{ TS 6.6.4.1  \hfill https://www.vedavms.in \hfill}

\section{ TS 6.6.4.1 }

\textbf{TS 6.6.4.1 } \newline
\textbf{Samhita Paata} \newline

स्फ्येन॒ वेदि॒मुद्ध॑न्ति रथा॒क्षेण॒ वि मि॑मीते॒ यूपं॑ मिनोति त्रि॒वृत॑मे॒व वज्रꣳ॑ स॒भृंत्य॒ भ्रातृ॑व्याय॒ प्र ह॑रति॒ स्तृत्यै॒ यद॑न्तर्वे॒दि मि॑नु॒याद्-दे॑वलो॒कम॒भि ज॑ये॒द्-यद्-ब॑हिर्वे॒दि म॑नुष्य लो॒कं ॅवे᳚द्य॒न्तस्य॑ स॒न्धौ मि॑नोत्यु॒भयो᳚-र्लो॒कयो॑-र॒भिजि॑त्या॒ उप॑रसंमितां मिनुयात् पितृलो॒कका॑मस्य रश॒नस॑मिंतां मनुष्यलो॒कका॑मस्य च॒षाल॑-संमितामिन्द्रि॒य का॑मस्य॒ सर्वा᳚न्थ् स॒मान् प्र॑ति॒ष्ठाका॑मस्य॒ ये त्रयो॑ मद्ध्य॒मास्तान्थ् स॒मान् प॒शुका॑मस्यै॒तान्. वा- [  ] \newline

\textbf{Pada Paata} \newline

स्फ्येन॑ । वेदि᳚म् । उदिति॑ । ह॒न्ति॒ । र॒था॒क्षेणेति॑ रथ-अ॒क्षेण॑ । वीति॑ । मि॒मी॒ते॒ । यूप᳚म् । मि॒नो॒ति॒ । त्रि॒वृत॒मिति॑ त्रि - वृत᳚म् । ए॒व । वज्र᳚म् । स॒भृंत्येति॑ सं - भृत्य॑ । भ्रातृ॑व्याय । प्रेति॑ । ह॒र॒ति॒ । स्तृत्यै᳚ । यत् । अ॒न्त॒र्वे॒दीत्य॑न्तः - वे॒दि । मि॒नु॒यात् । दे॒व॒लो॒कमिति॑ देव - लो॒कम् । अ॒भीति॑ । ज॒ये॒त् । यत् । ब॒हि॒र्वे॒दीति॑ बहिः - वे॒दि । म॒नु॒ष्य॒लो॒कमिति॑ मनुष्य - लो॒कम् । वे॒द्य॒न्तस्येति॑ वेदि - अ॒न्तस्य॑ । स॒न्धाविति॑ सं - धौ । मि॒नो॒ति॒ । उ॒भयोः᳚ । लो॒कयोः᳚ । अ॒भिजि॑त्या॒ इत्य॒भि - जि॒त्यै॒ । उप॑रसम्मिता॒मित्युप॑र - स॒मिं॒ता॒म् । मि॒नु॒या॒त् । पि॒तृ॒लो॒कका॑म॒स्येति॑ पितृलो॒क - का॒म॒स्य॒ । र॒श॒नस॑म्मिता॒मिति॑ रश॒न - स॒मिं॒ता॒म् । म॒नु॒ष्य॒लो॒कका॑म॒स्येति॑ मनुष्यलो॒क - का॒म॒स्य॒ । च॒षाल॑संमिता॒मिति॑ च॒षाल॑ - स॒मिं॒ता॒म् । इ॒न्द्रि॒यका॑म॒स्येती᳚न्द्रि॒य - का॒म॒स्य॒ । सर्वान्॑ । स॒मान् । प्र॒ति॒ष्ठाका॑म॒स्येति॑ प्रति॒ष्ठा - का॒म॒स्य॒ । ये । त्रयः॑ । म॒द्ध्य॒माः । तान् । स॒मान् । प॒शुका॑म॒स्येति॑ प॒शु - का॒म॒स्य॒ । ए॒तान् । वै ।  \newline


\textbf{Krama Paata} \newline

स्फ्येन॒ वेदि᳚म् । वेदि॒मुत् । उद्‌ध॑न्ति । ह॒न्ति॒ र॒था॒क्षेण॑ । र॒था॒क्षेण॒ वि । र॒था॒क्षेणेति॑ रथ - अ॒क्षेण॑ । वि मि॑मीते । मि॒मी॒ते॒ यूप᳚म् । यूप॑म् मिनोति । मि॒नो॒ति॒ त्रि॒वृत᳚म् । त्रि॒वृत॑मे॒व । त्रि॒वृत॒मिति॑ त्रि - वृत᳚म् । ए॒व वज्र᳚म् । वज्रꣳ॑ स॒म्भृत्य॑ । स॒म्भृत्य॒ भ्रातृ॑व्याय । स॒म्भृत्येति॑ सम् - भृत्य॑ । भ्रातृ॑व्याय॒ प्र । प्र ह॑रति । ह॒र॒ति॒ स्तृत्यै᳚ । स्तृत्यै॒ यत् । यद॑न्तर्वे॒दि । अ॒न्त॒र्वे॒दि मि॑नु॒यात् । अ॒न्त॒र्वे॒दीत्य॑न्तः - वे॒दि । मि॒नु॒याद् दे॑वलो॒कम् । दे॒व॒लो॒कम॒भि । दे॒व॒लो॒कमिति॑ देव - लो॒कम् । अ॒भि ज॑येत् । ज॒ये॒द् यत् । यद् ब॑हिर्वे॒दि । ब॒हि॒र्वे॒दि म॑नुष्यलो॒कम् । ब॒हि॒र्वे॒दीति॑ बहिः - वे॒दि । म॒नु॒ष्य॒लो॒कम् ॅवे᳚द्य॒न्तस्य॑ । म॒नु॒ष्य॒लो॒कमिति॑ मनुष्य - लो॒कम् । वे॒द्य॒न्तस्य॑ स॒न्धौ । वे॒द्य॒न्तस्येति॑ वेदि - अ॒न्तस्य॑ । स॒न्धौ मि॑नोति । स॒न्धाविति॑ सम् - धौ । मि॒नो॒त्यु॒भयोः᳚ । उ॒भयो᳚र् लो॒कयोः᳚ । लो॒कयो॑र॒भिजि॑त्यै । अ॒भिजि॑त्या॒ उप॑रसम्मिताम् । अ॒भिजि॑त्या॒ इत्य॒भि - जि॒त्यै॒ । उप॑रसम्मिताम् मिनुयात् । उप॑रसम्मिता॒मित्युप॑र - स॒म्मि॒ता॒म् । मि॒नु॒या॒त् पि॒तृ॒लो॒कका॑मस्य । पि॒तृ॒लो॒कका॑मस्य रश॒नस॑म्मिताम् । पि॒तृ॒लो॒कका॑म॒स्येति॑ पितृलो॒क - का॒म॒स्य॒ । र॒श॒नस॑म्मिताम् मनुष्यलो॒कका॑मस्य । र॒श॒नस॑म्मिता॒मिति॑ रश॒न - स॒म्मि॒ता॒म् । म॒नु॒ष्य॒लो॒कका॑मस्य च॒षाल॑सम्मिताम् । म॒नु॒ष्य॒लो॒कका॑म॒स्येति॑ मनुष्यलो॒क - का॒म॒स्य॒ । च॒षाल॑सम्मितामिन्द्रि॒यका॑मस्य । च॒षाल॑सम्मिता॒मिति॑ च॒षाल॑ - स॒म्मि॒ता॒म् । इ॒न्द्रि॒यका॑मस्य॒ सर्वान्॑ । इ॒न्द्रि॒यका॑म॒स्येती᳚न्द्रि॒य - का॒म॒स्य॒ । सर्वा᳚न्थ् स॒मान् । स॒मान् प्र॑ति॒ष्ठाका॑मस्य । प्र॒ति॒ष्ठाका॑मस्य॒ ये । प्र॒ति॒ष्ठाका॑म॒स्येति॑ प्रति॒ष्ठा - का॒म॒स्य॒ । ये त्रयः॑ । त्रयो॑ मद्ध्य॒माः । म॒द्ध्य॒मास्तान् । तान्थ् स॒मान् । स॒मान् प॒शुका॑मस्य । प॒शुका॑मस्यै॒तान् । प॒शुका॑म॒स्येति॑ प॒शु - का॒म॒स्य॒ । ए॒तान्. वै । वा अनु॑ \newline

\textbf{Jatai Paata} \newline

1. स्फ्येन॒ वेदिं॒ ॅवेदिꣳ॒॒ स्फ्येन॒ स्फ्येन॒ वेदि᳚म् । \newline
2. वेदि॒ मुदुद् वेदिं॒ ॅवेदि॒ मुत् । \newline
3. उद्ध॑न्ति ह॒न्त्यु दुद्ध॑न्ति । \newline
4. ह॒न्ति॒ र॒था॒क्षेण॑ रथा॒क्षेण॑ हन्ति हन्ति रथा॒क्षेण॑ । \newline
5. र॒था॒क्षेण॒ वि वि र॑था॒क्षेण॑ रथा॒क्षेण॒ वि । \newline
6. र॒था॒क्षेणेति॑ रथ - अ॒क्षेण॑ । \newline
7. वि मि॑मीते मिमीते॒ वि वि मि॑मीते । \newline
8. मि॒मी॒ते॒ यूपं॒ ॅयूप॑म् मिमीते मिमीते॒ यूप᳚म् । \newline
9. यूप॑म् मिनोति मिनोति॒ यूपं॒ ॅयूप॑म् मिनोति । \newline
10. मि॒नो॒ति॒ त्रि॒वृत॑म् त्रि॒वृत॑म् मिनोति मिनोति त्रि॒वृत᳚म् । \newline
11. त्रि॒वृत॑ मे॒वैव त्रि॒वृत॑म् त्रि॒वृत॑ मे॒व । \newline
12. त्रि॒वृत॒मिति॑ त्रि - वृत᳚म् । \newline
13. ए॒व वज्रं॒ ॅवज्र॑ मे॒वैव वज्र᳚म् । \newline
14. वज्रꣳ॑ सं॒भृत्य॑ सं॒भृत्य॒ वज्रं॒ ॅवज्रꣳ॑ सं॒भृत्य॑ । \newline
15. सं॒भृत्य॒ भ्रातृ॑व्याय॒ भ्रातृ॑व्याय सं॒भृत्य॑ सं॒भृत्य॒ भ्रातृ॑व्याय । \newline
16. सं॒भृत्येति॑ सं - भृत्य॑ । \newline
17. भ्रातृ॑व्याय॒ प्र प्र भ्रातृ॑व्याय॒ भ्रातृ॑व्याय॒ प्र । \newline
18. प्र ह॑रति हरति॒ प्र प्र ह॑रति । \newline
19. ह॒र॒ति॒ स्तृत्यै॒ स्तृत्यै॑ हरति हरति॒ स्तृत्यै᳚ । \newline
20. स्तृत्यै॒ यद् यथ् स्तृत्यै॒ स्तृत्यै॒ यत् । \newline
21. यद॑न्तर्वे॒ द्य॑न्तर्वे॒दि यद् यद॑न्तर्वे॒दि । \newline
22. अ॒न्त॒र्वे॒दि मि॑नु॒यान् मि॑नु॒या द॑न्तर्वे॒ द्य॑न्तर्वे॒दि मि॑नु॒यात् । \newline
23. अ॒न्त॒र्वे॒दीत्य॑न्तः - वे॒दि । \newline
24. मि॒नु॒याद् दे॑वलो॒कम् दे॑वलो॒कम् मि॑नु॒यान् मि॑नु॒याद् दे॑वलो॒कम् । \newline
25. दे॒व॒लो॒क म॒भ्य॑भि दे॑वलो॒कम् दे॑वलो॒क म॒भि । \newline
26. दे॒व॒लो॒कमिति॑ देव - लो॒कम् । \newline
27. अ॒भि ज॑येज् जये द॒भ्य॑भि ज॑येत् । \newline
28. ज॒ये॒द् यद् यज् ज॑येज् जये॒द् यत् । \newline
29. यद् ब॑हिर्वे॒दि ब॑हिर्वे॒दि यद् यद् ब॑हिर्वे॒दि । \newline
30. ब॒हि॒र्वे॒दि म॑नुष्यलो॒कम् म॑नुष्यलो॒कम् ब॑हिर्वे॒दि ब॑हिर्वे॒दि म॑नुष्यलो॒कम् । \newline
31. ब॒हि॒र्वे॒दीति॑ बहिः - वे॒दि । \newline
32. म॒नु॒ष्य॒लो॒कं ॅवे᳚द्य॒न्तस्य॑ वेद्य॒न्तस्य॑ मनुष्यलो॒कम् म॑नुष्यलो॒कं ॅवे᳚द्य॒न्तस्य॑ । \newline
33. म॒नु॒ष्य॒लो॒कमिति॑ मनुष्य - लो॒कम् । \newline
34. वे॒द्य॒न्तस्य॑ स॒न्धौ स॒न्धौ वे᳚द्य॒न्तस्य॑ वेद्य॒न्तस्य॑ स॒न्धौ । \newline
35. वे॒द्य॒न्तस्येति॑ वेदि - अ॒न्तस्य॑ । \newline
36. स॒न्धौ मि॑नोति मिनोति स॒न्धौ स॒न्धौ मि॑नोति । \newline
37. स॒न्धाविति॑ सं - धौ । \newline
38. मि॒नो॒ त्यु॒भयो॑ रु॒भयो᳚र् मिनोति मिनो त्यु॒भयोः᳚ । \newline
39. उ॒भयो᳚र् लो॒कयो᳚र् लो॒कयो॑ रु॒भयो॑ रु॒भयो᳚र् लो॒कयोः᳚ । \newline
40. लो॒कयो॑ र॒भिजि॑त्या अ॒भिजि॑त्यै लो॒कयो᳚र् लो॒कयो॑ र॒भिजि॑त्यै । \newline
41. अ॒भिजि॑त्या॒ उप॑रसम्मिता॒ मुप॑रसम्मिता म॒भिजि॑त्या अ॒भिजि॑त्या॒ उप॑रसम्मिताम् । \newline
42. अ॒भिजि॑त्या॒ इत्य॒भि - जि॒त्यै॒ । \newline
43. उप॑रसम्मिताम् मिनुयान् मिनुया॒ दुप॑रसम्मिता॒ मुप॑रसम्मिताम् मिनुयात् । \newline
44. उप॑रसम्मिता॒मित्युप॑र - स॒म्मि॒ता॒म् । \newline
45. मि॒नु॒या॒त् पि॒तृ॒लो॒कका॑मस्य पितृलो॒कका॑मस्य मिनुयान् मिनुयात् पितृलो॒कका॑मस्य । \newline
46. पि॒तृ॒लो॒कका॑मस्य रश॒नस॑म्मितातꣳ रश॒नस॑म्मितातम् पितृलो॒कका॑मस्य पितृलो॒कका॑मस्य रश॒नस॑म्मितातम् । \newline
47. पि॒तृ॒लो॒कका॑म॒स्येति॑ पितृलो॒क - का॒म॒स्य॒ । \newline
48. र॒श॒नस॑म्मितातम् मनुष्यलो॒कका॑मस्य मनुष्यलो॒कका॑मस्य रश॒नस॑म्मितातꣳ रश॒नस॑म्मितातम् मनुष्यलो॒कका॑मस्य । \newline
49. र॒श॒नस॑म्मिता॒मिति॑ रश॒न - स॒म्मि॒ता॒म् । \newline
50. म॒नु॒ष्य॒लो॒कका॑मस्य च॒षाल॑सम्मिताम् च॒षाल॑सम्मिताम् मनुष्यलो॒कका॑मस्य मनुष्यलो॒कका॑मस्य च॒षाल॑सम्मिताम् । \newline
51. म॒नु॒ष्य॒लो॒कका॑म॒स्येति॑ मनुष्यलो॒क - का॒म॒स्य॒ । \newline
52. च॒षाल॑सम्मिता मिन्द्रि॒यका॑म स्येन्द्रि॒यका॑मस्य च॒षाल॑सम्मिताम् च॒षाल॑सम्मिता मिन्द्रि॒यका॑मस्य । \newline
53. च॒षाल॑सम्मिता॒मिति॑ च॒षाल॑ - स॒म्मि॒ता॒म् । \newline
54. इ॒न्द्रि॒यका॑मस्य॒ सर्वा॒न् थ्सर्वा॑ निन्द्रि॒यका॑म स्येन्द्रि॒यका॑मस्य॒ सर्वान्॑ । \newline
55. इ॒न्द्रि॒यका॑म॒स्येती᳚न्द्रि॒य - का॒म॒स्य॒ । \newline
56. सर्वा᳚न् थ्स॒मान् थ्स॒मान् थ्सर्वा॒न् थ्सर्वा᳚न् थ्स॒मान् । \newline
57. स॒मान् प्र॑ति॒ष्ठाका॑मस्य प्रति॒ष्ठाका॑मस्य स॒मान् थ्स॒मान् प्र॑ति॒ष्ठाका॑मस्य । \newline
58. प्र॒ति॒ष्ठाका॑मस्य॒ ये ये प्र॑ति॒ष्ठाका॑मस्य प्रति॒ष्ठाका॑मस्य॒ ये । \newline
59. प्र॒ति॒ष्ठाका॑म॒स्येति॑ प्रति॒ष्ठा - का॒म॒स्य॒ । \newline
60. ये त्रय॒ स्त्रयो॒ ये ये त्रयः॑ । \newline
61. त्रयो॑ मद्ध्य॒मा म॑द्ध्य॒मा स्त्रय॒ स्त्रयो॑ मद्ध्य॒माः । \newline
62. म॒द्ध्य॒मा स्ताꣳ स्तान् म॑द्ध्य॒मा म॑द्ध्य॒मा स्तान् । \newline
63. तान् थ्स॒मान् थ्स॒मान् ताꣳ स्तान् थ्स॒मान् । \newline
64. स॒मान् प॒शुका॑मस्य प॒शुका॑मस्य स॒मान् थ्स॒मान् प॒शुका॑मस्य । \newline
65. प॒शुका॑म स्यै॒ता ने॒तान् प॒शुका॑मस्य प॒शुका॑म स्यै॒तान् । \newline
66. प॒शुका॑म॒स्येति॑ प॒शु - का॒म॒स्य॒ । \newline
67. ए॒तान्. वै वा ए॒ता ने॒तान्. वै । \newline
68. वा अन्वनु॒ वै वा अनु॑ । \newline

\textbf{Ghana Paata } \newline

1. स्फ्येन॒ वेदिं॒ ॅवेदिꣳ॒॒ स्फ्येन॒ स्फ्येन॒ वेदि॒ मुदुद् वेदिꣳ॒॒ स्फ्येन॒ स्फ्येन॒ वेदि॒ मुत् । \newline
2. वेदि॒ मुदुद् वेदिं॒ ॅवेदि॒ मुद्ध॑न्ति ह॒न्त्युद् वेदिं॒ ॅवेदि॒ मुद्ध॑न्ति । \newline
3. उद्ध॑न्ति ह॒न्त्यु दुद्ध॑न्ति रथा॒क्षेण॑ रथा॒क्षेण॑ ह॒न्त्यु दुद्ध॑न्ति रथा॒क्षेण॑ । \newline
4. ह॒न्ति॒ र॒था॒क्षेण॑ रथा॒क्षेण॑ हन्ति हन्ति रथा॒क्षेण॒ वि वि र॑था॒क्षेण॑ हन्ति हन्ति रथा॒क्षेण॒ वि । \newline
5. र॒था॒क्षेण॒ वि वि र॑था॒क्षेण॑ रथा॒क्षेण॒ वि मि॑मीते मिमीते॒ वि र॑था॒क्षेण॑ रथा॒क्षेण॒ वि मि॑मीते । \newline
6. र॒था॒क्षेणेति॑ रथ - अ॒क्षेण॑ । \newline
7. वि मि॑मीते मिमीते॒ वि वि मि॑मीते॒ यूपं॒ ॅयूप॑म् मिमीते॒ वि वि मि॑मीते॒ यूप᳚म् । \newline
8. मि॒मी॒ते॒ यूपं॒ ॅयूप॑म् मिमीते मिमीते॒ यूप॑म् मिनोति मिनोति॒ यूप॑म् मिमीते मिमीते॒ यूप॑म् मिनोति । \newline
9. यूप॑म् मिनोति मिनोति॒ यूपं॒ ॅयूप॑म् मिनोति त्रि॒वृत॑म् त्रि॒वृत॑म् मिनोति॒ यूपं॒ ॅयूप॑म् मिनोति त्रि॒वृत᳚म् । \newline
10. मि॒नो॒ति॒ त्रि॒वृत॑म् त्रि॒वृत॑म् मिनोति मिनोति त्रि॒वृत॑ मे॒वैव त्रि॒वृत॑म् मिनोति मिनोति त्रि॒वृत॑ मे॒व । \newline
11. त्रि॒वृत॑ मे॒वैव त्रि॒वृत॑म् त्रि॒वृत॑ मे॒व वज्रं॒ ॅवज्र॑ मे॒व त्रि॒वृत॑म् त्रि॒वृत॑ मे॒व वज्र᳚म् । \newline
12. त्रि॒वृत॒मिति॑ त्रि - वृत᳚म् । \newline
13. ए॒व वज्रं॒ ॅवज्र॑ मे॒वैव वज्रꣳ॑ सं॒भृत्य॑ सं॒भृत्य॒ वज्र॑ मे॒वैव वज्रꣳ॑ सं॒भृत्य॑ । \newline
14. वज्रꣳ॑ सं॒भृत्य॑ सं॒भृत्य॒ वज्रं॒ ॅवज्रꣳ॑ सं॒भृत्य॒ भ्रातृ॑व्याय॒ भ्रातृ॑व्याय सं॒भृत्य॒ वज्रं॒ ॅवज्रꣳ॑ सं॒भृत्य॒ भ्रातृ॑व्याय । \newline
15. सं॒भृत्य॒ भ्रातृ॑व्याय॒ भ्रातृ॑व्याय सं॒भृत्य॑ सं॒भृत्य॒ भ्रातृ॑व्याय॒ प्र प्र भ्रातृ॑व्याय सं॒भृत्य॑ सं॒भृत्य॒ भ्रातृ॑व्याय॒ प्र । \newline
16. सं॒भृत्येति॑ सं - भृत्य॑ । \newline
17. भ्रातृ॑व्याय॒ प्र प्र भ्रातृ॑व्याय॒ भ्रातृ॑व्याय॒ प्र ह॑रति हरति॒ प्र भ्रातृ॑व्याय॒ भ्रातृ॑व्याय॒ प्र ह॑रति । \newline
18. प्र ह॑रति हरति॒ प्र प्र ह॑रति॒ स्तृत्यै॒ स्तृत्यै॑ हरति॒ प्र प्र ह॑रति॒ स्तृत्यै᳚ । \newline
19. ह॒र॒ति॒ स्तृत्यै॒ स्तृत्यै॑ हरति हरति॒ स्तृत्यै॒ यद् यथ् स्तृत्यै॑ हरति हरति॒ स्तृत्यै॒ यत् । \newline
20. स्तृत्यै॒ यद् यथ् स्तृत्यै॒ स्तृत्यै॒ यद॑न्तर्वे॒ द्य॑न्तर्वे॒दि यथ् स्तृत्यै॒ स्तृत्यै॒ यद॑न्तर्वे॒दि । \newline
21. यद॑न्तर्वे॒ द्य॑न्तर्वे॒दि यद् यद॑न्तर्वे॒दि मि॑नु॒यान् मि॑नु॒या द॑न्तर्वे॒दि यद् यद॑न्तर्वे॒दि मि॑नु॒यात् । \newline
22. अ॒न्त॒र्वे॒दि मि॑नु॒यान् मि॑नु॒या द॑न्तर्वे॒ द्य॑न्तर्वे॒दि मि॑नु॒याद् दे॑वलो॒कम् दे॑वलो॒कम् मि॑नु॒या द॑न्तर्वे॒ द्य॑न्तर्वे॒दि मि॑नु॒याद् दे॑वलो॒कम् । \newline
23. अ॒न्त॒र्वे॒दीत्य॑न्तः - वे॒दि । \newline
24. मि॒नु॒याद् दे॑वलो॒कम् दे॑वलो॒कम् मि॑नु॒यान् मि॑नु॒याद् दे॑वलो॒क म॒भ्य॑भि दे॑वलो॒कम् मि॑नु॒यान् मि॑नु॒याद् दे॑वलो॒क म॒भि । \newline
25. दे॒व॒लो॒क म॒भ्य॑भि दे॑वलो॒कम् दे॑वलो॒क म॒भि ज॑येज् जये द॒भि दे॑वलो॒कम् दे॑वलो॒क म॒भि ज॑येत् । \newline
26. दे॒व॒लो॒कमिति॑ देव - लो॒कम् । \newline
27. अ॒भि ज॑येज् जये द॒भ्य॑भि ज॑ये॒द् यद् यज् ज॑ये द॒भ्य॑भि ज॑ये॒द् यत् । \newline
28. ज॒ये॒द् यद् यज् ज॑येज् जये॒द् यद् ब॑हिर्वे॒दि ब॑हिर्वे॒दि यज् ज॑येज् जये॒द् यद् ब॑हिर्वे॒दि । \newline
29. यद् ब॑हिर्वे॒दि ब॑हिर्वे॒दि यद् यद् ब॑हिर्वे॒दि म॑नुष्यलो॒कम् म॑नुष्यलो॒कम् ब॑हिर्वे॒दि यद् यद् ब॑हिर्वे॒दि म॑नुष्यलो॒कम् । \newline
30. ब॒हि॒र्वे॒दि म॑नुष्यलो॒कम् म॑नुष्यलो॒कम् ब॑हिर्वे॒दि ब॑हिर्वे॒दि म॑नुष्यलो॒कं ॅवे᳚द्य॒न्तस्य॑ वेद्य॒न्तस्य॑ मनुष्यलो॒कम् ब॑हिर्वे॒दि ब॑हिर्वे॒दि म॑नुष्यलो॒कं ॅवे᳚द्य॒न्तस्य॑ । \newline
31. ब॒हि॒र्वे॒दीति॑ बहिः - वे॒दि । \newline
32. म॒नु॒ष्य॒लो॒कं ॅवे᳚द्य॒न्तस्य॑ वेद्य॒न्तस्य॑ मनुष्यलो॒कम् म॑नुष्यलो॒कं ॅवे᳚द्य॒न्तस्य॑ स॒न्धौ स॒न्धौ वे᳚द्य॒न्तस्य॑ मनुष्यलो॒कम् म॑नुष्यलो॒कं ॅवे᳚द्य॒न्तस्य॑ स॒न्धौ । \newline
33. म॒नु॒ष्य॒लो॒कमिति॑ मनुष्य - लो॒कम् । \newline
34. वे॒द्य॒न्तस्य॑ स॒न्धौ स॒न्धौ वे᳚द्य॒न्तस्य॑ वेद्य॒न्तस्य॑ स॒न्धौ मि॑नोति मिनोति स॒न्धौ वे᳚द्य॒न्तस्य॑ वेद्य॒न्तस्य॑ स॒न्धौ मि॑नोति । \newline
35. वे॒द्य॒न्तस्येति॑ वेदि - अ॒न्तस्य॑ । \newline
36. स॒न्धौ मि॑नोति मिनोति स॒न्धौ स॒न्धौ मि॑नो त्यु॒भयो॑ रु॒भयो᳚र् मिनोति स॒न्धौ स॒न्धौ मि॑नो त्यु॒भयोः᳚ । \newline
37. स॒न्धाविति॑ सं - धौ । \newline
38. मि॒नो॒ त्यु॒भयो॑ रु॒भयो᳚र् मिनोति मिनो त्यु॒भयो᳚र् लो॒कयो᳚र् लो॒कयो॑ रु॒भयो᳚र् मिनोति मिनो त्यु॒भयो᳚र् लो॒कयोः᳚ । \newline
39. उ॒भयो᳚र् लो॒कयो᳚र् लो॒कयो॑ रु॒भयो॑ रु॒भयो᳚र् लो॒कयो॑ र॒भिजि॑त्या अ॒भिजि॑त्यै लो॒कयो॑ रु॒भयो॑ रु॒भयो᳚र् लो॒कयो॑ र॒भिजि॑त्यै । \newline
40. लो॒कयो॑ र॒भिजि॑त्या अ॒भिजि॑त्यै लो॒कयो᳚र् लो॒कयो॑ र॒भिजि॑त्या॒ उप॑रसम्मिता॒ मुप॑रसम्मिता म॒भिजि॑त्यै लो॒कयो᳚र् लो॒कयो॑ र॒भिजि॑त्या॒ उप॑रसम्मिताम् । \newline
41. अ॒भिजि॑त्या॒ उप॑रसम्मिता॒ मुप॑रसम्मिता म॒भिजि॑त्या अ॒भिजि॑त्या॒ उप॑रसम्मिताम् मिनुयान् मिनुया॒ दुप॑रसम्मिता म॒भिजि॑त्या अ॒भिजि॑त्या॒ उप॑रसम्मिताम् मिनुयात् । \newline
42. अ॒भिजि॑त्या॒ इत्य॒भि - जि॒त्यै॒ । \newline
43. उप॑रसम्मिताम् मिनुयान् मिनुया॒ दुप॑रसम्मिता॒ मुप॑रसम्मिताम् मिनुयात् पितृलो॒कका॑मस्य पितृलो॒कका॑मस्य मिनुया॒ दुप॑रसम्मिता॒ मुप॑रसम्मिताम् मिनुयात् पितृलो॒कका॑मस्य । \newline
44. उप॑रसम्मिता॒मित्युप॑र - स॒म्मि॒ता॒म् । \newline
45. मि॒नु॒या॒त् पि॒तृ॒लो॒कका॑मस्य पितृलो॒कका॑मस्य मिनुयान् मिनुयात् पितृलो॒कका॑मस्य रश॒नस॑म्मितातꣳ रश॒नस॑म्मितातम् पितृलो॒कका॑मस्य मिनुयान् मिनुयात् पितृलो॒कका॑मस्य रश॒नस॑म्मितातम् । \newline
46. पि॒तृ॒लो॒कका॑मस्य रश॒नस॑म्मितातꣳ रश॒नस॑म्मितातम् पितृलो॒कका॑मस्य पितृलो॒कका॑मस्य रश॒नस॑म्मितातम् मनुष्यलो॒कका॑मस्य मनुष्यलो॒कका॑मस्य रश॒नस॑म्मितातम् पितृलो॒कका॑मस्य पितृलो॒कका॑मस्य रश॒नस॑म्मितातम् मनुष्यलो॒कका॑मस्य । \newline
47. पि॒तृ॒लो॒कका॑म॒स्येति॑ पितृलो॒क - का॒म॒स्य॒ । \newline
48. र॒श॒नस॑म्मितातम् मनुष्यलो॒कका॑मस्य मनुष्यलो॒कका॑मस्य रश॒नस॑म्मितातꣳ रश॒नस॑म्मितातम् मनुष्यलो॒कका॑मस्य च॒षाल॑सम्मिताम् च॒षाल॑सम्मिताम् मनुष्यलो॒कका॑मस्य रश॒नस॑म्मितातꣳ रश॒नस॑म्मितातम् मनुष्यलो॒कका॑मस्य च॒षाल॑सम्मिताम् । \newline
49. र॒श॒नस॑म्मिता॒मिति॑ रश॒न - स॒म्मि॒ता॒म् । \newline
50. म॒नु॒ष्य॒लो॒कका॑मस्य च॒षाल॑सम्मिताम् च॒षाल॑सम्मिताम् मनुष्यलो॒कका॑मस्य मनुष्यलो॒कका॑मस्य च॒षाल॑सम्मिता मिन्द्रि॒यका॑म स्येन्द्रि॒यका॑मस्य च॒षाल॑सम्मिताम् मनुष्यलो॒कका॑मस्य मनुष्यलो॒कका॑मस्य च॒षाल॑सम्मिता मिन्द्रि॒यका॑मस्य । \newline
51. म॒नु॒ष्य॒लो॒कका॑म॒स्येति॑ मनुष्यलो॒क - का॒म॒स्य॒ । \newline
52. च॒षाल॑सम्मिता मिन्द्रि॒यका॑म स्येन्द्रि॒यका॑मस्य च॒षाल॑सम्मिताम् च॒षाल॑सम्मिता मिन्द्रि॒यका॑मस्य॒ सर्वा॒न् थ्सर्वा॑ निन्द्रि॒यका॑मस्य च॒षाल॑सम्मिताम् च॒षाल॑सम्मिता मिन्द्रि॒यका॑मस्य॒ सर्वान्॑ । \newline
53. च॒षाल॑सम्मिता॒मिति॑ च॒षाल॑ - स॒म्मि॒ता॒म् । \newline
54. इ॒न्द्रि॒यका॑मस्य॒ सर्वा॒न् थ्सर्वा॑ निन्द्रि॒यका॑म स्येन्द्रि॒यका॑मस्य॒ सर्वा᳚न् थ्स॒मान् थ्स॒मान् थ्सर्वा॑ निन्द्रि॒यका॑म स्येन्द्रि॒यका॑मस्य॒ सर्वा᳚न् थ्स॒मान् । \newline
55. इ॒न्द्रि॒यका॑म॒स्येती᳚न्द्रि॒य - का॒म॒स्य॒ । \newline
56. सर्वा᳚न् थ्स॒मान् थ्स॒मान् थ्सर्वा॒न् थ्सर्वा᳚न् थ्स॒मान् प्र॑ति॒ष्ठाका॑मस्य प्रति॒ष्ठाका॑मस्य स॒मान् थ्सर्वा॒न् थ्सर्वा᳚न् थ्स॒मान् प्र॑ति॒ष्ठाका॑मस्य । \newline
57. स॒मान् प्र॑ति॒ष्ठाका॑मस्य प्रति॒ष्ठाका॑मस्य स॒मान् थ्स॒मान् प्र॑ति॒ष्ठाका॑मस्य॒ ये ये प्र॑ति॒ष्ठाका॑मस्य स॒मान् थ्स॒मान् प्र॑ति॒ष्ठाका॑मस्य॒ ये । \newline
58. प्र॒ति॒ष्ठाका॑मस्य॒ ये ये प्र॑ति॒ष्ठाका॑मस्य प्रति॒ष्ठाका॑मस्य॒ ये त्रय॒ स्त्रयो॒ ये प्र॑ति॒ष्ठाका॑मस्य प्रति॒ष्ठाका॑मस्य॒ ये त्रयः॑ । \newline
59. प्र॒ति॒ष्ठाका॑म॒स्येति॑ प्रति॒ष्ठा - का॒म॒स्य॒ । \newline
60. ये त्रय॒ स्त्रयो॒ ये ये त्रयो॑ मद्ध्य॒मा म॑द्ध्य॒मा स्त्रयो॒ ये ये त्रयो॑ मद्ध्य॒माः । \newline
61. त्रयो॑ मद्ध्य॒मा म॑द्ध्य॒मा स्त्रय॒ स्त्रयो॑ मद्ध्य॒मा स्ताꣳ स्तान् म॑द्ध्य॒मा स्त्रय॒ स्त्रयो॑ मद्ध्य॒मा स्तान् । \newline
62. म॒द्ध्य॒मा स्ताꣳ स्तान् म॑द्ध्य॒मा म॑द्ध्य॒मा स्तान् थ्स॒मान् थ्स॒मान् तान् म॑द्ध्य॒मा म॑द्ध्य॒मा स्तान् थ्स॒मान् । \newline
63. तान् थ्स॒मान् थ्स॒मान् ताꣳ स्तान् थ्स॒मान् प॒शुका॑मस्य प॒शुका॑मस्य स॒मान् ताꣳ स्तान् थ्स॒मान् प॒शुका॑मस्य । \newline
64. स॒मान् प॒शुका॑मस्य प॒शुका॑मस्य स॒मान् थ्स॒मान् प॒शुका॑म स्यै॒ता ने॒तान् प॒शुका॑मस्य स॒मान् थ्स॒मान् प॒शुका॑म स्यै॒तान् । \newline
65. प॒शुका॑म स्यै॒ता ने॒तान् प॒शुका॑मस्य प॒शुका॑म स्यै॒तान्. वै वा ए॒तान् प॒शुका॑मस्य प॒शुका॑म स्यै॒तान्. वै । \newline
66. प॒शुका॑म॒स्येति॑ प॒शु - का॒म॒स्य॒ । \newline
67. ए॒तान्. वै वा ए॒ता ने॒तान्. वा अन्वनु॒ वा ए॒ता ने॒तान्. वा अनु॑ । \newline
68. वा अन्वनु॒ वै वा अनु॑ प॒शवः॑ प॒शवो ऽनु॒ वै वा अनु॑ प॒शवः॑ । \newline
\pagebreak
\markright{ TS 6.6.4.2  \hfill https://www.vedavms.in \hfill}

\section{ TS 6.6.4.2 }

\textbf{TS 6.6.4.2 } \newline
\textbf{Samhita Paata} \newline

अनु॑ प॒शव॒ उप॑ तिष्ठन्ते पशु॒माने॒व भ॑वति॒ व्यति॑षजे॒दित॑रान् प्र॒जयै॒वैनं॑ प॒शुभि॒र्व्यति॑षजति॒ यं का॒मये॑त प्र॒मायु॑कः स्या॒दिति॑ गर्त॒मितं॒ तस्य॑ मिनुयादुत्तरा॒र्द्ध्यं॑ ॅवर्.षि॑ष्ठ॒मथ॒ ह्रसी॑याꣳसमे॒षा वै ग॑र्त॒मिद्यस्यै॒वं मि॒नोति॑ ता॒जक् प्र मी॑यते दक्षिणा॒र्द्ध्यं॑ ॅवर्.षि॑ष्ठं मिनुयाथ् सुव॒र्गका॑म॒स्याथ॒ ह्रसी॑याꣳस-मा॒क्रम॑णमे॒व तथ् सेतुं॒ ॅयज॑मानः कुरुते सुव॒र्गस्य॑ लो॒कस्य॒ सम॑ष्ट्यै॒- [  ] \newline

\textbf{Pada Paata} \newline

अन्विति॑ । प॒शवः॑ । उपेति॑ । ति॒ष्ठ॒न्ते॒ । प॒शु॒मानिति॑ पशु - मान् । ए॒व । भ॒व॒ति॒ । व्यति॑षजे॒दिति॑ वि - अति॑षजेत् । इत॑रान् । प्र॒जयेति॑ प्र - जया᳚ । ए॒व । ए॒न॒म् । प॒शुभि॒रिति॑ प॒शु - भिः॒ । व्यति॑षज॒तीति॑ वि - अति॑षजति । यम् । का॒मये॑त । प्र॒मायु॑क॒ इति॑ प्र - मायु॑कः । स्या॒त् । इति॑ । ग॒र्त॒मित॒मिति॑ गर्त - मित᳚म् । तस्य॑ । मि॒नु॒या॒त् । उ॒त्त॒रा॒द्‌र्ध्य॑मित्यु॑त्तर - अ॒द्‌र्ध्य᳚म् । वर्.षि॑ष्ठम् । अथ॑ । ह्रसी॑याꣳसम् । ए॒षा । वै । ग॒र्त॒मिदिति॑ गर्त - मित् । यस्य॑ । ए॒वम् । मि॒नोति॑ । ता॒जक् । प्रेति॑ । मी॒य॒ते॒ । द॒क्षि॒णा॒द्‌र्ध्य॑मिति॑ दक्षिण - अ॒द्‌र्ध्य᳚म् । वर्.षि॑ष्ठम् । मि॒नु॒या॒त् । सु॒व॒र्गका॑म॒स्येति॑ सुव॒र्ग - का॒म॒स्य॒ । अथ॑ । ह्रसी॑याꣳसम् । आ॒क्रम॑ण॒मित्या᳚ - क्रम॑णम् । ए॒व । तत् । सेतु᳚म् । यज॑मानः । कु॒रु॒ते॒ । सु॒व॒र्गस्येति॑ सुवः - गस्य॑ । लो॒कस्य॑ । सम॑ष्ट्य॒ इति॒ सं - अ॒ष्ट्यै॒ ।  \newline


\textbf{Krama Paata} \newline

अनु॑ प॒शवः॑ । प॒शव॒ उप॑ । उप॑ तिष्ठन्ते । ति॒ष्ठ॒न्ते॒ प॒शु॒मान् । प॒शु॒माने॒व । प॒शु॒मानिति॑ पशु - मान् । ए॒व भ॑वति । भ॒व॒ति॒ व्यति॑षजेत् । व्यति॑षजे॒दित॑रान् । व्यति॑षजे॒दिति॑ वि - अति॑षजेत् । इत॑रान् प्र॒जया᳚ । प्र॒जयै॒व । प्र॒जयेति॑ प्र - जया᳚ । ए॒वैन᳚म् । ए॒न॒म् प॒शुभिः॑ । प॒शुभि॒र् व्यति॑षजति । प॒शुभि॒रिति॑ प॒शु - भिः॒ । व्यति॑षजति॒ यम् । व्यति॑षज॒तीति॑ वि - अति॑षजति । यम् का॒मये॑त । का॒मये॑त प्र॒मायु॑कः । प्र॒मायु॑कः स्यात् । प्र॒मायु॑क॒ इति॑ प्र - मायु॑कः । स्या॒दिति॑ । इति॑ गर्त॒मित᳚म् । ग॒र्त॒मित॒म् तस्य॑ । ग॒र्त॒मित॒मिति॑ गर्त - मित᳚म् । तस्य॑ मिनुयात् । मि॒नु॒या॒दु॒त्त॒रा॒र्द्ध्य᳚म् । उ॒त्त॒रा॒र्द्ध्य॑म् ॅवर्.षि॑ष्ठम् । उ॒त्त॒रा॒र्द्ध्य॑मित्यु॑त्तर - अ॒र्द्ध्य᳚म् । वर्.षि॑ष्ठ॒मथ॑ । अथ॒ ह्रसी॑याꣳसम् । ह्रसी॑याꣳसमे॒षा । ए॒षा वै । वै ग॑र्त॒मित् । ग॒र्त॒मिद् यस्य॑ । ग॒र्त॒मिदिति॑ गर्त - मित् । यस्यै॒वम् । ए॒वम् मि॒नोति॑ । मि॒नोति॑ ता॒जक् । ता॒जक् प्र । प्र मी॑यते । मी॒य॒ते॒ द॒क्षि॒णा॒र्द्ध्य᳚म् । द॒क्षि॒णा॒र्द्ध्य॑म् ॅवर्.षि॑ष्ठम् । द॒क्षि॒णा॒र्द्ध्य॑मिति॑ दक्षिण - अ॒र्द्ध्य᳚म् । वर्.षि॑ष्ठम् मिनुयात् । मि॒नु॒या॒थ् सु॒व॒र्गका॑मस्य । सु॒व॒र्गका॑म॒स्याथ॑ । सु॒व॒र्गका॑म॒स्येति॑ सुव॒र्ग - का॒म॒स्य॒ । अथ॒ ह्रसी॑याꣳसम् । ह्रसी॑याꣳसमा॒क्रम॑णम् । आ॒क्रम॑णमे॒व । आ॒क्रम॑ण॒मित्या᳚ - क्रम॑णम् । ए॒व तत् । तथ् सेतु᳚म् । सेतु॒म् ॅयज॑मानः । यज॑मानः कुरुते । कु॒रु॒ते॒ सु॒व॒र्गस्य॑ । सु॒व॒र्गस्य॑ लो॒कस्य॑ । सु॒व॒र्गस्येति॑ सुवः - गस्य॑ । लो॒कस्य॒ सम॑ष्ट्‍यै । सम॑ष्ट्‍यै॒ यत् । सम॑ष्ट्‍या॒ इति॒ सम् - अ॒ष्ट्‍यै॒ \newline

\textbf{Jatai Paata} \newline

1. अनु॑ प॒शवः॑ प॒शवो ऽन्वनु॑ प॒शवः॑ । \newline
2. प॒शव॒ उपोप॑ प॒शवः॑ प॒शव॒ उप॑ । \newline
3. उप॑ तिष्ठन्ते तिष्ठन्त॒ उपोप॑ तिष्ठन्ते । \newline
4. ति॒ष्ठ॒न्ते॒ प॒शु॒मान् प॑शु॒मान् ति॑ष्ठन्ते तिष्ठन्ते पशु॒मान् । \newline
5. प॒शु॒मा ने॒वैव प॑शु॒मान् प॑शु॒माने॒व । \newline
6. प॒शु॒मानिति॑ पशु - मान् । \newline
7. ए॒व भ॑वति भव त्ये॒वैव भ॑वति । \newline
8. भ॒व॒ति॒ व्यति॑षजे॒द् व्यति॑षजेद् भवति भवति॒ व्यति॑षजेत् । \newline
9. व्यति॑षजे॒ दित॑रा॒ नित॑रा॒न् व्यति॑षजे॒द् व्यति॑षजे॒ दित॑रान् । \newline
10. व्यति॑षजे॒दिति॑ वि - अति॑षजेत् । \newline
11. इत॑रान् प्र॒जया᳚ प्र॒जये त॑रा॒ नित॑रान् प्र॒जया᳚ । \newline
12. प्र॒जयै॒ वैव प्र॒जया᳚ प्र॒ज यै॒व । \newline
13. प्र॒जयेति॑ प्र - जया᳚ । \newline
14. ए॒वैन॑ मेन मे॒वै वैन᳚म् । \newline
15. ए॒न॒म् प॒शुभिः॑ प॒शुभि॑ रेन मेनम् प॒शुभिः॑ । \newline
16. प॒शुभि॒र् व्यति॑षजति॒ व्यति॑षजति प॒शुभिः॑ प॒शुभि॒र् व्यति॑षजति । \newline
17. प॒शुभि॒रिति॑ प॒शु - भिः॒ । \newline
18. व्यति॑षजति॒ यं ॅयं ॅव्यति॑षजति॒ व्यति॑षजति॒ यम् । \newline
19. व्यति॑षज॒तीति॑ वि - अति॑षजति । \newline
20. यम् का॒मये॑त का॒मये॑त॒ यं ॅयम् का॒मये॑त । \newline
21. का॒मये॑त प्र॒मायु॑कः प्र॒मायु॑कः का॒मये॑त का॒मये॑त प्र॒मायु॑कः । \newline
22. प्र॒मायु॑कः स्याथ् स्यात् प्र॒मायु॑कः प्र॒मायु॑कः स्यात् । \newline
23. प्र॒मायु॑क॒ इति॑ प्र - मायु॑कः । \newline
24. स्या॒ दितीति॑ स्याथ् स्या॒ दिति॑ । \newline
25. इति॑ गर्त॒मित॑म् गर्त॒मित॒ मितीति॑ गर्त॒मित᳚म् । \newline
26. ग॒र्त॒मित॒म् तस्य॒ तस्य॑ गर्त॒मित॑म् गर्त॒मित॒म् तस्य॑ । \newline
27. ग॒र्त॒मित॒मिति॑ गर्त - मित᳚म् । \newline
28. तस्य॑ मिनुयान् मिनुया॒त् तस्य॒ तस्य॑ मिनुयात् । \newline
29. मि॒नु॒या॒ दु॒त्त॒रा॒र्द्ध्य॑ मुत्तरा॒र्द्ध्य॑म् मिनुयान् मिनुया दुत्तरा॒र्द्ध्य᳚म् । \newline
30. उ॒त्त॒रा॒र्द्ध्यं॑ ॅवर्.षि॑ष्ठं॒ ॅवर्.षि॑ष्ठ मुत्तरा॒र्द्ध्य॑ मुत्तरा॒र्द्ध्यं॑ ॅवर्.षि॑ष्ठम् । \newline
31. उ॒त्त॒रा॒र्द्ध्य॑मित्यु॑त्तर - अ॒र्द्ध्य᳚म् । \newline
32. वर्.षि॑ष्ठ॒ मथाथ॒ वर्.षि॑ष्ठं॒ ॅवर्.षि॑ष्ठ॒ मथ॑ । \newline
33. अथ॒ ह्रसी॑याꣳसꣳ॒॒ ह्रसी॑याꣳस॒ मथाथ॒ ह्रसी॑याꣳसम् । \newline
34. ह्रसी॑याꣳस मे॒षैषा ह्रसी॑याꣳसꣳ॒॒ ह्रसी॑याꣳस मे॒षा । \newline
35. ए॒षा वै वा ए॒षैषा वै । \newline
36. वै ग॑र्त॒मिद् ग॑र्त॒मिद् वै वै ग॑र्त॒मित् । \newline
37. ग॒र्त॒मिद् यस्य॒ यस्य॑ गर्त॒मिद् ग॑र्त॒मिद् यस्य॑ । \newline
38. ग॒र्त॒मिदिति॑ गर्त - मित् । \newline
39. यस्यै॒व मे॒वं ॅयस्य॒ यस्यै॒वम् । \newline
40. ए॒वम् मि॒नोति॑ मि॒नो त्ये॒व मे॒वम् मि॒नोति॑ । \newline
41. मि॒नोति॑ ता॒जक् ता॒जङ् मि॒नोति॑ मि॒नोति॑ ता॒जक् । \newline
42. ता॒जक् प्र प्र ता॒जक् ता॒जक् प्र । \newline
43. प्र मी॑यते मीयते॒ प्र प्र मी॑यते । \newline
44. मी॒य॒ते॒ द॒क्षि॒णा॒र्द्ध्य॑म् दक्षिणा॒र्द्ध्य॑म् मीयते मीयते दक्षिणा॒र्द्ध्य᳚म् । \newline
45. द॒क्षि॒णा॒र्द्ध्यं॑ ॅवर्.षि॑ष्ठं॒ ॅवर्.षि॑ष्ठम् दक्षिणा॒र्द्ध्य॑म् दक्षिणा॒र्द्ध्यं॑ ॅवर्.षि॑ष्ठम् । \newline
46. द॒क्षि॒णा॒र्द्ध्य॑मिति॑ दक्षिण - अ॒र्द्ध्य᳚म् । \newline
47. वर्.षि॑ष्ठम् मिनुयान् मिनुया॒द् वर्.षि॑ष्ठं॒ ॅवर्.षि॑ष्ठम् मिनुयात् । \newline
48. मि॒नु॒या॒थ् सु॒व॒र्गका॑मस्य सुव॒र्गका॑मस्य मिनुयान् मिनुयाथ् सुव॒र्गका॑मस्य । \newline
49. सु॒व॒र्गका॑म॒ स्याथाथ॑ सुव॒र्गका॑मस्य सुव॒र्गका॑म॒ स्याथ॑ । \newline
50. सु॒व॒र्गका॑म॒स्येति॑ सुव॒र्ग - का॒म॒स्य॒ । \newline
51. अथ॒ ह्रसी॑याꣳसꣳ॒॒ ह्रसी॑याꣳस॒ मथाथ॒ ह्रसी॑याꣳसम् । \newline
52. ह्रसी॑याꣳस मा॒क्रम॑ण मा॒क्रम॑णꣳ॒॒ ह्रसी॑याꣳसꣳ॒॒ ह्रसी॑याꣳस मा॒क्रम॑णम् । \newline
53. आ॒क्रम॑ण मे॒वै वाक्रम॑ण मा॒क्रम॑ण मे॒व । \newline
54. आ॒क्रम॑ण॒मित्या᳚ - क्रम॑णम् । \newline
55. ए॒व तत् तदे॒ वैव तत् । \newline
56. तथ् सेतुꣳ॒॒ सेतु॒म् तत् तथ् सेतु᳚म् । \newline
57. सेतुं॒ ॅयज॑मानो॒ यज॑मानः॒ सेतुꣳ॒॒ सेतुं॒ ॅयज॑मानः । \newline
58. यज॑मानः कुरुते कुरुते॒ यज॑मानो॒ यज॑मानः कुरुते । \newline
59. कु॒रु॒ते॒ सु॒व॒र्गस्य॑ सुव॒र्गस्य॑ कुरुते कुरुते सुव॒र्गस्य॑ । \newline
60. सु॒व॒र्गस्य॑ लो॒कस्य॑ लो॒कस्य॑ सुव॒र्गस्य॑ सुव॒र्गस्य॑ लो॒कस्य॑ । \newline
61. सु॒व॒र्गस्येति॑ सुवः - गस्य॑ । \newline
62. लो॒कस्य॒ सम॑ष्ट्यै॒ सम॑ष्ट्यै लो॒कस्य॑ लो॒कस्य॒ सम॑ष्ट्यै । \newline
63. सम॑ष्ट्यै॒ यद् यथ् सम॑ष्ट्यै॒ सम॑ष्ट्यै॒ यत् । \newline
64. सम॑ष्ट्या॒ इति॒ सं - अ॒ष्ट्यै॒ । \newline

\textbf{Ghana Paata } \newline

1. अनु॑ प॒शवः॑ प॒शवो ऽन्वनु॑ प॒शव॒ उपोप॑ प॒शवो ऽन्वनु॑ प॒शव॒ उप॑ । \newline
2. प॒शव॒ उपोप॑ प॒शवः॑ प॒शव॒ उप॑ तिष्ठन्ते तिष्ठन्त॒ उप॑ प॒शवः॑ प॒शव॒ उप॑ तिष्ठन्ते । \newline
3. उप॑ तिष्ठन्ते तिष्ठन्त॒ उपोप॑ तिष्ठन्ते पशु॒मान् प॑शु॒मान् ति॑ष्ठन्त॒ उपोप॑ तिष्ठन्ते पशु॒मान् । \newline
4. ति॒ष्ठ॒न्ते॒ प॒शु॒मान् प॑शु॒मान् ति॑ष्ठन्ते तिष्ठन्ते पशु॒मा ने॒वैव प॑शु॒मान् ति॑ष्ठन्ते तिष्ठन्ते पशु॒मा ने॒व । \newline
5. प॒शु॒मा ने॒वैव प॑शु॒मान् प॑शु॒मा ने॒व भ॑वति भव त्ये॒व प॑शु॒मान् प॑शु॒मा ने॒व भ॑वति । \newline
6. प॒शु॒मानिति॑ पशु - मान् । \newline
7. ए॒व भ॑वति भव त्ये॒वैव भ॑वति॒ व्यति॑षजे॒द् व्यति॑षजेद् भव त्ये॒वैव भ॑वति॒ व्यति॑षजेत् । \newline
8. भ॒व॒ति॒ व्यति॑षजे॒द् व्यति॑षजेद् भवति भवति॒ व्यति॑षजे॒ दित॑रा॒ नित॑रा॒न् व्यति॑षजेद् भवति भवति॒ व्यति॑षजे॒ दित॑रान् । \newline
9. व्यति॑षजे॒ दित॑रा॒ नित॑रा॒न् व्यति॑षजे॒द् व्यति॑षजे॒ दित॑रान् प्र॒जया᳚ प्र॒ज येत॑रा॒न् व्यति॑षजे॒द् व्यति॑षजे॒ दित॑रान् प्र॒जया᳚ । \newline
10. व्यति॑षजे॒दिति॑ वि - अति॑षजेत् । \newline
11. इत॑रान् प्र॒जया᳚ प्र॒जयेत॑रा॒ नित॑रान् प्र॒जयै॒वैव प्र॒जयेत॑रा॒ नित॑रान् प्र॒ज यै॒व । \newline
12. प्र॒जयै॒ वैव प्र॒जया᳚ प्र॒ज यै॒वैन॑ मेन मे॒व प्र॒जया᳚ प्र॒ज यै॒वैन᳚म् । \newline
13. प्र॒जयेति॑ प्र - जया᳚ । \newline
14. ए॒वैन॑ मेन मे॒वै वैन॑म् प॒शुभिः॑ प॒शुभि॑ रेन मे॒वै वैन॑म् प॒शुभिः॑ । \newline
15. ए॒न॒म् प॒शुभिः॑ प॒शुभि॑ रेन मेनम् प॒शुभि॒र् व्यति॑षजति॒ व्यति॑षजति प॒शुभि॑ रेन मेनम् प॒शुभि॒र् व्यति॑षजति । \newline
16. प॒शुभि॒र् व्यति॑षजति॒ व्यति॑षजति प॒शुभिः॑ प॒शुभि॒र् व्यति॑षजति॒ यं ॅयं ॅव्यति॑षजति प॒शुभिः॑ प॒शुभि॒र् व्यति॑षजति॒ यम् । \newline
17. प॒शुभि॒रिति॑ प॒शु - भिः॒ । \newline
18. व्यति॑षजति॒ यं ॅयं ॅव्यति॑षजति॒ व्यति॑षजति॒ यम् का॒मये॑त का॒मये॑त॒ यं ॅव्यति॑षजति॒ व्यति॑षजति॒ यम् का॒मये॑त । \newline
19. व्यति॑षज॒तीति॑ वि - अति॑षजति । \newline
20. यम् का॒मये॑त का॒मये॑त॒ यं ॅयम् का॒मये॑त प्र॒मायु॑कः प्र॒मायु॑कः का॒मये॑त॒ यं ॅयम् का॒मये॑त प्र॒मायु॑कः । \newline
21. का॒मये॑त प्र॒मायु॑कः प्र॒मायु॑कः का॒मये॑त का॒मये॑त प्र॒मायु॑कः स्याथ् स्यात् प्र॒मायु॑कः का॒मये॑त का॒मये॑त प्र॒मायु॑कः स्यात् । \newline
22. प्र॒मायु॑कः स्याथ् स्यात् प्र॒मायु॑कः प्र॒मायु॑कः स्या॒ दितीति॑ स्यात् प्र॒मायु॑कः प्र॒मायु॑कः स्या॒ दिति॑ । \newline
23. प्र॒मायु॑क॒ इति॑ प्र - मायु॑कः । \newline
24. स्या॒ दितीति॑ स्याथ् स्या॒ दिति॑ गर्त॒मित॑म् गर्त॒मित॒ मिति॑ स्याथ् स्या॒ दिति॑ गर्त॒मित᳚म् । \newline
25. इति॑ गर्त॒मित॑म् गर्त॒मित॒ मितीति॑ गर्त॒मित॒म् तस्य॒ तस्य॑ गर्त॒मित॒ मितीति॑ गर्त॒मित॒म् तस्य॑ । \newline
26. ग॒र्त॒मित॒म् तस्य॒ तस्य॑ गर्त॒मित॑म् गर्त॒मित॒म् तस्य॑ मिनुयान् मिनुया॒त् तस्य॑ गर्त॒मित॑म् गर्त॒मित॒म् तस्य॑ मिनुयात् । \newline
27. ग॒र्त॒मित॒मिति॑ गर्त - मित᳚म् । \newline
28. तस्य॑ मिनुयान् मिनुया॒त् तस्य॒ तस्य॑ मिनुया दुत्तरा॒र्द्ध्य॑ मुत्तरा॒र्द्ध्य॑म् मिनुया॒त् तस्य॒ तस्य॑ मिनुया दुत्तरा॒र्द्ध्य᳚म् । \newline
29. मि॒नु॒या॒ दु॒त्त॒रा॒र्द्ध्य॑ मुत्तरा॒र्द्ध्य॑म् मिनुयान् मिनुया दुत्तरा॒र्द्ध्यं॑ ॅवर्.षि॑ष्ठं॒ ॅवर्.षि॑ष्ठ मुत्तरा॒र्द्ध्य॑म् मिनुयान् मिनुया दुत्तरा॒र्द्ध्यं॑ ॅवर्.षि॑ष्ठम् । \newline
30. उ॒त्त॒रा॒र्द्ध्यं॑ ॅवर्.षि॑ष्ठं॒ ॅवर्.षि॑ष्ठ मुत्तरा॒र्द्ध्य॑ मुत्तरा॒र्द्ध्यं॑ ॅवर्.षि॑ष्ठ॒ मथाथ॒ वर्.षि॑ष्ठ मुत्तरा॒र्द्ध्य॑ मुत्तरा॒र्द्ध्यं॑ ॅवर्.षि॑ष्ठ॒ मथ॑ । \newline
31. उ॒त्त॒रा॒र्द्ध्य॑मित्यु॑त्तर - अ॒र्द्ध्य᳚म् । \newline
32. वर्.षि॑ष्ठ॒ मथाथ॒ वर्.षि॑ष्ठं॒ ॅवर्.षि॑ष्ठ॒ मथ॒ ह्रसी॑याꣳसꣳ॒॒ ह्रसी॑याꣳस॒ मथ॒ वर्.षि॑ष्ठं॒ ॅवर्.षि॑ष्ठ॒ मथ॒ ह्रसी॑याꣳसम् । \newline
33. अथ॒ ह्रसी॑याꣳसꣳ॒॒ ह्रसी॑याꣳस॒ मथाथ॒ ह्रसी॑याꣳस मे॒षैषा ह्रसी॑याꣳस॒ मथाथ॒ ह्रसी॑याꣳस मे॒षा । \newline
34. ह्रसी॑याꣳस मे॒षैषा ह्रसी॑याꣳसꣳ॒॒ ह्रसी॑याꣳस मे॒षा वै वा ए॒षा ह्रसी॑याꣳसꣳ॒॒ ह्रसी॑याꣳस मे॒षा वै । \newline
35. ए॒षा वै वा ए॒षैषा वै ग॑र्त॒मिद् ग॑र्त॒मिद् वा ए॒षैषा वै ग॑र्त॒मित् । \newline
36. वै ग॑र्त॒मिद् ग॑र्त॒मिद् वै वै ग॑र्त॒मिद् यस्य॒ यस्य॑ गर्त॒मिद् वै वै ग॑र्त॒मिद् यस्य॑ । \newline
37. ग॒र्त॒मिद् यस्य॒ यस्य॑ गर्त॒मिद् ग॑र्त॒मिद् यस्यै॒व मे॒वं ॅयस्य॑ गर्त॒मिद् ग॑र्त॒मिद् यस्यै॒वम् । \newline
38. ग॒र्त॒मिदिति॑ गर्त - मित् । \newline
39. यस्यै॒व मे॒वं ॅयस्य॒ यस्यै॒वम् मि॒नोति॑ मि॒नो त्ये॒वं ॅयस्य॒ यस्यै॒वम् मि॒नोति॑ । \newline
40. ए॒वम् मि॒नोति॑ मि॒नो त्ये॒व मे॒वम् मि॒नोति॑ ता॒जक् ता॒जङ् मि॒नो त्ये॒व मे॒वम् मि॒नोति॑ ता॒जक् । \newline
41. मि॒नोति॑ ता॒जक् ता॒जङ् मि॒नोति॑ मि॒नोति॑ ता॒जक् प्र प्र ता॒जङ् मि॒नोति॑ मि॒नोति॑ ता॒जक् प्र । \newline
42. ता॒जक् प्र प्र ता॒जक् ता॒जक् प्र मी॑यते मीयते॒ प्र ता॒जक् ता॒जक् प्र मी॑यते । \newline
43. प्र मी॑यते मीयते॒ प्र प्र मी॑यते दक्षिणा॒र्द्ध्य॑म् दक्षिणा॒र्द्ध्य॑म् मीयते॒ प्र प्र मी॑यते दक्षिणा॒र्द्ध्य᳚म् । \newline
44. मी॒य॒ते॒ द॒क्षि॒णा॒र्द्ध्य॑म् दक्षिणा॒र्द्ध्य॑म् मीयते मीयते दक्षिणा॒र्द्ध्यं॑ ॅवर्.षि॑ष्ठं॒ ॅवर्.षि॑ष्ठम् दक्षिणा॒र्द्ध्य॑म् मीयते मीयते दक्षिणा॒र्द्ध्यं॑ ॅवर्.षि॑ष्ठम् । \newline
45. द॒क्षि॒णा॒र्द्ध्यं॑ ॅवर्.षि॑ष्ठं॒ ॅवर्.षि॑ष्ठम् दक्षिणा॒र्द्ध्य॑म् दक्षिणा॒र्द्ध्यं॑ ॅवर्.षि॑ष्ठम् मिनुयान् मिनुया॒द् वर्.षि॑ष्ठम् दक्षिणा॒र्द्ध्य॑म् दक्षिणा॒र्द्ध्यं॑ ॅवर्.षि॑ष्ठम् मिनुयात् । \newline
46. द॒क्षि॒णा॒र्द्ध्य॑मिति॑ दक्षिण - अ॒र्द्ध्य᳚म् । \newline
47. वर्.षि॑ष्ठम् मिनुयान् मिनुया॒द् वर्.षि॑ष्ठं॒ ॅवर्.षि॑ष्ठम् मिनुयाथ् सुव॒र्गका॑मस्य सुव॒र्गका॑मस्य मिनुया॒द् वर्.षि॑ष्ठं॒ ॅवर्.षि॑ष्ठम् मिनुयाथ् सुव॒र्गका॑मस्य । \newline
48. मि॒नु॒या॒थ् सु॒व॒र्गका॑मस्य सुव॒र्गका॑मस्य मिनुयान् मिनुयाथ् सुव॒र्गका॑म॒स्याथाथ॑ सुव॒र्गका॑मस्य मिनुयान् मिनुयाथ् सुव॒र्गका॑म॒स्याथ॑ । \newline
49. सु॒व॒र्गका॑म॒स्याथाथ॑ सुव॒र्गका॑मस्य सुव॒र्गका॑म॒स्याथ॒ ह्रसी॑याꣳसꣳ॒॒ ह्रसी॑याꣳस॒ मथ॑ सुव॒र्गका॑मस्य सुव॒र्गका॑म॒स्याथ॒ ह्रसी॑याꣳसम् । \newline
50. सु॒व॒र्गका॑म॒स्येति॑ सुव॒र्ग - का॒म॒स्य॒ । \newline
51. अथ॒ ह्रसी॑याꣳसꣳ॒॒ ह्रसी॑याꣳस॒ मथाथ॒ ह्रसी॑याꣳस मा॒क्रम॑ण मा॒क्रम॑णꣳ॒॒ ह्रसी॑याꣳस॒ मथाथ॒ ह्रसी॑याꣳस मा॒क्रम॑णम् । \newline
52. ह्रसी॑याꣳस मा॒क्रम॑ण मा॒क्रम॑णꣳ॒॒ ह्रसी॑याꣳसꣳ॒॒ ह्रसी॑याꣳस मा॒क्रम॑ण मे॒वैवा क्रम॑णꣳ॒॒ ह्रसी॑याꣳसꣳ॒॒ ह्रसी॑याꣳस मा॒क्रम॑ण मे॒व । \newline
53. आ॒क्रम॑ण मे॒वै वाक्रम॑ण मा॒क्रम॑ण मे॒व तत् तदे॒वा क्रम॑ण मा॒क्रम॑ण मे॒व तत् । \newline
54. आ॒क्रम॑ण॒मित्या᳚ - क्रम॑णम् । \newline
55. ए॒व तत् तदे॒ वैव तथ् सेतुꣳ॒॒ सेतु॒म् तदे॒ वैव तथ् सेतु᳚म् । \newline
56. तथ् सेतुꣳ॒॒ सेतु॒म् तत् तथ् सेतुं॒ ॅयज॑मानो॒ यज॑मानः॒ सेतु॒म् तत् तथ् सेतुं॒ ॅयज॑मानः । \newline
57. सेतुं॒ ॅयज॑मानो॒ यज॑मानः॒ सेतुꣳ॒॒ सेतुं॒ ॅयज॑मानः कुरुते कुरुते॒ यज॑मानः॒ सेतुꣳ॒॒ सेतुं॒ ॅयज॑मानः कुरुते । \newline
58. यज॑मानः कुरुते कुरुते॒ यज॑मानो॒ यज॑मानः कुरुते सुव॒र्गस्य॑ सुव॒र्गस्य॑ कुरुते॒ यज॑मानो॒ यज॑मानः कुरुते सुव॒र्गस्य॑ । \newline
59. कु॒रु॒ते॒ सु॒व॒र्गस्य॑ सुव॒र्गस्य॑ कुरुते कुरुते सुव॒र्गस्य॑ लो॒कस्य॑ लो॒कस्य॑ सुव॒र्गस्य॑ कुरुते कुरुते सुव॒र्गस्य॑ लो॒कस्य॑ । \newline
60. सु॒व॒र्गस्य॑ लो॒कस्य॑ लो॒कस्य॑ सुव॒र्गस्य॑ सुव॒र्गस्य॑ लो॒कस्य॒ सम॑ष्ट्यै॒ सम॑ष्ट्यै लो॒कस्य॑ सुव॒र्गस्य॑ सुव॒र्गस्य॑ लो॒कस्य॒ सम॑ष्ट्यै । \newline
61. सु॒व॒र्गस्येति॑ सुवः - गस्य॑ । \newline
62. लो॒कस्य॒ सम॑ष्ट्यै॒ सम॑ष्ट्यै लो॒कस्य॑ लो॒कस्य॒ सम॑ष्ट्यै॒ यद् यथ् सम॑ष्ट्यै लो॒कस्य॑ लो॒कस्य॒ सम॑ष्ट्यै॒ यत् । \newline
63. सम॑ष्ट्यै॒ यद् यथ् सम॑ष्ट्यै॒ सम॑ष्ट्यै॒ यदेक॑स्मि॒न् नेक॑स्मि॒न्॒. यथ् सम॑ष्ट्यै॒ सम॑ष्ट्यै॒ यदेक॑स्मिन्न् । \newline
64. सम॑ष्ट्या॒ इति॒ सं - अ॒ष्ट्यै॒ । \newline
\pagebreak
\markright{ TS 6.6.4.3  \hfill https://www.vedavms.in \hfill}

\section{ TS 6.6.4.3 }

\textbf{TS 6.6.4.3 } \newline
\textbf{Samhita Paata} \newline

यदेक॑स्मि॒न॒. यूपे॒ द्वे र॑श॒ने प॑रि॒व्यय॑ति॒ तस्मा॒देको॒ द्वे जा॒ये वि॑न्दते॒ यन्नैकाꣳ॑ रश॒नां द्वयो॒र्यूप॑योः परि॒व्यय॑ति॒ तस्मा॒न्नैका॒ द्वौ पती॑ विन्दते॒ यं का॒मये॑त॒ स्त्र्य॑स्य जाये॒तेत्यु॑पा॒न्ते तस्य॒ व्यति॑षजे॒थ् स्त्र्ये॑वास्य॑ जायते॒ यं का॒मये॑त॒ पुमा॑नस्य जाये॒तेत्या॒न्तं तस्य॒ प्र वे᳚ष्टये॒त् पुमा॑ने॒वास्य॑- [  ] \newline

\textbf{Pada Paata} \newline

यत् । एक॑स्मिन्न् । यूपे᳚ । द्वे इति॑ । र॒श॒ने इति॑ । प॒रि॒व्यय॒तीति॑ परि - व्यय॑ति । तस्मा᳚त् । एकः॑ । द्वे इति॑ । जा॒ये इति॑ । वि॒न्द॒ते॒ । यत् । न । एका᳚म् । र॒श॒नाम् । द्वयोः᳚ । यूप॑योः । प॒रि॒व्यय॒तीति॑ परि - व्यय॑ति । तस्मा᳚त् । न । एका᳚ । द्वौ । पती॒ इति॑ । वि॒न्द॒ते॒ । यम् । का॒मये॑त । स्त्री । अ॒स्य॒ । जा॒ये॒त॒ । इति॑ । उ॒पा॒न्त इत्यु॑प - अ॒न्ते । तस्य॑ । व्यति॑षजे॒दिति॑ वि - अति॑षजेत् । स्त्री । ए॒व । अ॒स्य॒ । जा॒य॒ते॒ । यम् । का॒मये॑त । पुमान्॑ । अ॒स्य॒ । जा॒ये॒त॒ । इति॑ । आ॒न्तमित्या᳚ - अ॒न्तम् । तस्य॑ । प्रेति॑ । वे॒ष्ट॒ये॒त् । पुमान्॑ । ए॒व । अ॒स्य॒ ।  \newline


\textbf{Krama Paata} \newline

यदेक॑स्मिन्न् । एक॑स्मि॒न्.॒ यूपे᳚ । यूपे॒ द्वे । द्वे र॑श॒ने । द्वे इति॒ द्वे । र॒श॒ने प॑रि॒व्यय॑ति । र॒श॒ने इति॑ रश॒ने । प॒रि॒व्यय॑ति॒ तस्मा᳚त् । प॒रि॒व्यय॒तीति॑ परि - व्यय॑ति । तस्मा॒देकः॑ । एको॒ द्वे । द्वे जा॒ये । द्वे इति॒ द्वे । जा॒ये वि॑न्दते । जा॒ये इति॑ जा॒ये । वि॒न्द॒ते॒ यत् । यन् न । नैका᳚म् । एकाꣳ॑ रश॒नाम् । र॒श॒नाम् द्वयोः᳚ । द्वयो॒र् यूप॑योः । यूप॑योः परि॒व्यय॑ति । प॒रि॒व्यय॑ति॒ तस्मा᳚त् । प॒रि॒व्यय॒तीति॑ परि - व्यय॑ति । तस्मा॒न् न । नैका᳚ । एका॒ द्वौ । द्वौ पती᳚ । पती॑ विन्दते । पती॒ इति॒ पती᳚ । वि॒न्द॒ते॒ यम् । यम् का॒मये॑त । का॒मये॑त॒ स्त्री । स्त्र्य॑स्य । अ॒स्य॒ जा॒ये॒त॒ । जा॒ये॒तेति॑ । इत्यु॑पा॒न्ते । उ॒पा॒न्ते तस्य॑ । उ॒पा॒न्त इत्यु॑प - अ॒न्ते । तस्य॒ व्यति॑षजेत् । व्यति॑षजे॒थ् स्त्री । व्यति॑षजे॒दिति॑ वि - अति॑षजेत् । स्त्र्ये॑व । ए॒वास्य॑ । अ॒स्य॒ जा॒य॒ते॒ । जा॒य॒ते॒ यम् । यम् का॒मये॑त । का॒मये॑त॒ पुमान्॑ । पुमा॑नस्य । अ॒स्य॒ जा॒ये॒त॒ । जा॒ये॒तेति॑ । इत्या॒न्तम् । आ॒न्तम् तस्य॑ । आ॒न्तमित्या᳚ - अ॒न्तम् । तस्य॒ प्र । प्र वे᳚ष्टयेत् । 
वे॒ष्ट॒ये॒त् पुमान्॑ । पुमा॑ने॒व । ए॒वास्य॑ । अ॒स्य॒ जा॒य॒ते॒ \newline

\textbf{Jatai Paata} \newline

1. यदेक॑स्मि॒न् नेक॑स्मि॒न्॒. यद् यदेक॑स्मिन्न् । \newline
2. एक॑स्मि॒न्॒. यूपे॒ यूप॒ एक॑स्मि॒न् नेक॑स्मि॒न्॒. यूपे᳚ । \newline
3. यूपे॒ द्वे द्वे यूपे॒ यूपे॒ द्वे । \newline
4. द्वे र॑श॒ने र॑श॒ने द्वे द्वे र॑श॒ने । \newline
5. द्वे इति॒ द्वे । \newline
6. र॒श॒ने प॑रि॒व्यय॑ति परि॒व्यय॑ति रश॒ने र॑श॒ने प॑रि॒व्यय॑ति । \newline
7. र॒श॒ने इति॑ रश॒ने । \newline
8. प॒रि॒व्यय॑ति॒ तस्मा॒त् तस्मा᳚त् परि॒व्यय॑ति परि॒व्यय॑ति॒ तस्मा᳚त् । \newline
9. प॒रि॒व्यय॒तीति॑ परि - व्यय॑ति । \newline
10. तस्मा॒ देक॒ एक॒ स्तस्मा॒त् तस्मा॒ देकः॑ । \newline
11. एको॒ द्वे द्वे एक॒ एको॒ द्वे । \newline
12. द्वे जा॒ये जा॒ये द्वे द्वे जा॒ये । \newline
13. द्वे इति॒ द्वे । \newline
14. जा॒ये वि॑न्दते विन्दते जा॒ये जा॒ये वि॑न्दते । \newline
15. जा॒ये इति॑ जा॒ये । \newline
16. वि॒न्द॒ते॒ यद् यद् वि॑न्दते विन्दते॒ यत् । \newline
17. यन् न न यद् यन् न । \newline
18. नैका॒ मेका॒न् न नैका᳚म् । \newline
19. एकाꣳ॑ रश॒नाꣳ र॑श॒ना मेका॒ मेकाꣳ॑ रश॒नाम् । \newline
20. र॒श॒नाम् द्वयो॒र् द्वयो॑ रश॒नाꣳ र॑श॒नाम् द्वयोः᳚ । \newline
21. द्वयो॒र् यूप॑यो॒र् यूप॑यो॒र् द्वयो॒र् द्वयो॒र् यूप॑योः । \newline
22. यूप॑योः परि॒व्यय॑ति परि॒व्यय॑ति॒ यूप॑यो॒र् यूप॑योः परि॒व्यय॑ति । \newline
23. प॒रि॒व्यय॑ति॒ तस्मा॒त् तस्मा᳚त् परि॒व्यय॑ति परि॒व्यय॑ति॒ तस्मा᳚त् । \newline
24. प॒रि॒व्यय॒तीति॑ परि - व्यय॑ति । \newline
25. तस्मा॒न् न न तस्मा॒त् तस्मा॒न् न । \newline
26. नैकैका॒ न नैका᳚ । \newline
27. एका॒ द्वौ द्वा वेकैका॒ द्वौ । \newline
28. द्वौ पती॒ पती॒ द्वौ द्वौ पती᳚ । \newline
29. पती॑ विन्दते विन्दते॒ पती॒ पती॑ विन्दते । \newline
30. पती॒ इति॒ पती᳚ । \newline
31. वि॒न्द॒ते॒ यं ॅयं ॅवि॑न्दते विन्दते॒ यम् । \newline
32. यम् का॒मये॑त का॒मये॑त॒ यं ॅयम् का॒मये॑त । \newline
33. का॒मये॑त॒ स्त्री स्त्री का॒मये॑त का॒मये॑त॒ स्त्री । \newline
34. स्त्र्य॑ स्यास्य॒ स्त्री स्त्र्य॑स्य । \newline
35. अ॒स्य॒ जा॒ये॒त॒ जा॒ये॒ता॒ स्या॒स्य॒ जा॒ये॒त॒ । \newline
36. जा॒ये॒तेतीति॑ जायेत जाये॒तेति॑ । \newline
37. इत्यु॑पा॒न्त उ॑पा॒न्त इतीत्यु॑पा॒न्ते । \newline
38. उ॒पा॒न्ते तस्य॒ तस्यो॑पा॒न्त उ॑पा॒न्ते तस्य॑ । \newline
39. उ॒पा॒न्त इत्यु॑प - अ॒न्ते । \newline
40. तस्य॒ व्यति॑षजे॒द् व्यति॑षजे॒त् तस्य॒ तस्य॒ व्यति॑षजेत् । \newline
41. व्यति॑षजे॒थ् स्त्री स्त्री व्यति॑षजे॒द् व्यति॑षजे॒थ् स्त्री । \newline
42. व्यति॑षजे॒दिति॑ वि - अति॑षजेत् । \newline
43. स्त्र्ये॑ वैव स्त्री स्त्र्ये॑व । \newline
44. ए॒वास्या᳚ स्यै॒वै वास्य॑ । \newline
45. अ॒स्य॒ जा॒य॒ते॒ जा॒य॒ते॒ ऽस्या॒स्य॒ जा॒य॒ते॒ । \newline
46. जा॒य॒ते॒ यं ॅयम् जा॑यते जायते॒ यम् । \newline
47. यम् का॒मये॑त का॒मये॑त॒ यं ॅयम् का॒मये॑त । \newline
48. का॒मये॑त॒ पुमा॒न् पुमा᳚न् का॒मये॑त का॒मये॑त॒ पुमान्॑ । \newline
49. पुमा॑ नस्यास्य॒ पुमा॒न् पुमा॑ नस्य । \newline
50. अ॒स्य॒ जा॒ये॒त॒ जा॒ये॒ता॒ स्या॒स्य॒ जा॒ये॒त॒ । \newline
51. जा॒ये॒तेतीति॑ जायेत जाये॒तेति॑ । \newline
52. इत्या॒न्त मा॒न्त मितीत्या॒न्तम् । \newline
53. आ॒न्तम् तस्य॒ तस्या॒न्त मा॒न्तम् तस्य॑ । \newline
54. आ॒न्तमित्या᳚ - अ॒न्तम् । \newline
55. तस्य॒ प्र प्र तस्य॒ तस्य॒ प्र । \newline
56. प्र वे᳚ष्टयेद् वेष्टये॒त् प्र प्र वे᳚ष्टयेत् । \newline
57. वे॒ष्ट॒ये॒त् पुमा॒न् पुमान्॑. वेष्टयेद् वेष्टये॒त् पुमान्॑ । \newline
58. पुमा॑ ने॒वैव पुमा॒न् पुमा॑ ने॒व । \newline
59. ए॒वास्या᳚ स्यै॒वै वास्य॑ । \newline
60. अ॒स्य॒ जा॒य॒ते॒ जा॒य॒ते॒ ऽस्या॒स्य॒ जा॒य॒ते॒ । \newline

\textbf{Ghana Paata } \newline

1. यदेक॑स्मि॒न् नेक॑स्मि॒न्॒. यद् यदेक॑स्मि॒न्॒. यूपे॒ यूप॒ एक॑स्मि॒न्॒. यद् यदेक॑स्मि॒न्॒. यूपे᳚ । \newline
2. एक॑स्मि॒न्॒. यूपे॒ यूप॒ एक॑स्मि॒न् नेक॑स्मि॒न्॒. यूपे॒ द्वे द्वे यूप॒ एक॑स्मि॒न् नेक॑स्मि॒न्॒. यूपे॒ द्वे । \newline
3. यूपे॒ द्वे द्वे यूपे॒ यूपे॒ द्वे र॑श॒ने र॑श॒ने द्वे यूपे॒ यूपे॒ द्वे र॑श॒ने । \newline
4. द्वे र॑श॒ने र॑श॒ने द्वे द्वे र॑श॒ने प॑रि॒व्यय॑ति परि॒व्यय॑ति रश॒ने द्वे द्वे र॑श॒ने प॑रि॒व्यय॑ति । \newline
5. द्वे इति॒ द्वे । \newline
6. र॒श॒ने प॑रि॒व्यय॑ति परि॒व्यय॑ति रश॒ने र॑श॒ने प॑रि॒व्यय॑ति॒ तस्मा॒त् तस्मा᳚त् परि॒व्यय॑ति रश॒ने र॑श॒ने प॑रि॒व्यय॑ति॒ तस्मा᳚त् । \newline
7. र॒श॒ने इति॑ रश॒ने । \newline
8. प॒रि॒व्यय॑ति॒ तस्मा॒त् तस्मा᳚त् परि॒व्यय॑ति परि॒व्यय॑ति॒ तस्मा॒ देक॒ एक॒ स्तस्मा᳚त् परि॒व्यय॑ति परि॒व्यय॑ति॒ तस्मा॒ देकः॑ । \newline
9. प॒रि॒व्यय॒तीति॑ परि - व्यय॑ति । \newline
10. तस्मा॒ देक॒ एक॒ स्तस्मा॒त् तस्मा॒ देको॒ द्वे द्वे एक॒ स्तस्मा॒त् तस्मा॒ देको॒ द्वे । \newline
11. एको॒ द्वे द्वे एक॒ एको॒ द्वे जा॒ये जा॒ये द्वे एक॒ एको॒ द्वे जा॒ये । \newline
12. द्वे जा॒ये जा॒ये द्वे द्वे जा॒ये वि॑न्दते विन्दते जा॒ये द्वे द्वे जा॒ये वि॑न्दते । \newline
13. द्वे इति॒ द्वे । \newline
14. जा॒ये वि॑न्दते विन्दते जा॒ये जा॒ये वि॑न्दते॒ यद् यद् वि॑न्दते जा॒ये जा॒ये वि॑न्दते॒ यत् । \newline
15. जा॒ये इति॑ जा॒ये । \newline
16. वि॒न्द॒ते॒ यद् यद् वि॑न्दते विन्दते॒ यन् न न यद् वि॑न्दते विन्दते॒ यन् न । \newline
17. यन् न न यद् यन् नैका॒ मेका॒न् न यद् यन् नैका᳚म् । \newline
18. नैका॒ मेका॒न् न नैकाꣳ॑ रश॒नाꣳ र॑श॒ना मेका॒न् न नैकाꣳ॑ रश॒नाम् । \newline
19. एकाꣳ॑ रश॒नाꣳ र॑श॒ना मेका॒ मेकाꣳ॑ रश॒नाम् द्वयो॒र् द्वयो॑ रश॒ना मेका॒ मेकाꣳ॑ रश॒नाम् द्वयोः᳚ । \newline
20. र॒श॒नाम् द्वयो॒र् द्वयो॑ रश॒नाꣳ र॑श॒नाम् द्वयो॒र् यूप॑यो॒र् यूप॑यो॒र् द्वयो॑ रश॒नाꣳ र॑श॒नाम् द्वयो॒र् यूप॑योः । \newline
21. द्वयो॒र् यूप॑यो॒र् यूप॑यो॒र् द्वयो॒र् द्वयो॒र् यूप॑योः परि॒व्यय॑ति परि॒व्यय॑ति॒ यूप॑यो॒र् द्वयो॒र् द्वयो॒र् यूप॑योः परि॒व्यय॑ति । \newline
22. यूप॑योः परि॒व्यय॑ति परि॒व्यय॑ति॒ यूप॑यो॒र् यूप॑योः परि॒व्यय॑ति॒ तस्मा॒त् तस्मा᳚त् परि॒व्यय॑ति॒ यूप॑यो॒र् यूप॑योः परि॒व्यय॑ति॒ तस्मा᳚त् । \newline
23. प॒रि॒व्यय॑ति॒ तस्मा॒त् तस्मा᳚त् परि॒व्यय॑ति परि॒व्यय॑ति॒ तस्मा॒न् न न तस्मा᳚त् परि॒व्यय॑ति परि॒व्यय॑ति॒ तस्मा॒न् न । \newline
24. प॒रि॒व्यय॒तीति॑ परि - व्यय॑ति । \newline
25. तस्मा॒न् न न तस्मा॒त् तस्मा॒न् नैकैका॒ न तस्मा॒त् तस्मा॒न् नैका᳚ । \newline
26. नैकैका॒ न नैका॒ द्वौ द्वा वेका॒ न नैका॒ द्वौ । \newline
27. एका॒ द्वौ द्वा वेकैका॒ द्वौ पती॒ पती॒ द्वा वेकैका॒ द्वौ पती᳚ । \newline
28. द्वौ पती॒ पती॒ द्वौ द्वौ पती॑ विन्दते विन्दते॒ पती॒ द्वौ द्वौ पती॑ विन्दते । \newline
29. पती॑ विन्दते विन्दते॒ पती॒ पती॑ विन्दते॒ यं ॅयं ॅवि॑न्दते॒ पती॒ पती॑ विन्दते॒ यम् । \newline
30. पती॒ इति॒ पती᳚ । \newline
31. वि॒न्द॒ते॒ यं ॅयं ॅवि॑न्दते विन्दते॒ यम् का॒मये॑त का॒मये॑त॒ यं ॅवि॑न्दते विन्दते॒ यम् का॒मये॑त । \newline
32. यम् का॒मये॑त का॒मये॑त॒ यं ॅयम् का॒मये॑त॒ स्त्री स्त्री का॒मये॑त॒ यं ॅयम् का॒मये॑त॒ स्त्री । \newline
33. का॒मये॑त॒ स्त्री स्त्री का॒मये॑त का॒मये॑त॒ स्त्र्य॑ स्यास्य॒ स्त्री का॒मये॑त का॒मये॑त॒ स्त्र्य॑स्य । \newline
34. स्त्र्य॑ स्यास्य॒ स्त्री स्त्र्य॑स्य जायेत जायेतास्य॒ स्त्री स्त्र्य॑स्य जायेत । \newline
35. अ॒स्य॒ जा॒ये॒त॒ जा॒ये॒ता॒ स्या॒स्य॒ जा॒ये॒तेतीति॑ जायेता स्यास्य जाये॒तेति॑ । \newline
36. जा॒ये॒तेतीति॑ जायेत जाये॒ते त्यु॑पा॒न्त उ॑पा॒न्त इति॑ जायेत जाये॒ते त्यु॑पा॒न्ते । \newline
37. इत्यु॑पा॒न्त उ॑पा॒न्त इती त्यु॑पा॒न्ते तस्य॒ तस्यो॑ पा॒न्त इती त्यु॑पा॒न्ते तस्य॑ । \newline
38. उ॒पा॒न्ते तस्य॒ तस्यो॑ पा॒न्त उ॑पा॒न्ते तस्य॒ व्यति॑षजे॒द् व्यति॑षजे॒त् तस्यो॑ पा॒न्त उ॑पा॒न्ते तस्य॒ व्यति॑षजेत् । \newline
39. उ॒पा॒न्त इत्यु॑प - अ॒न्ते । \newline
40. तस्य॒ व्यति॑षजे॒द् व्यति॑षजे॒त् तस्य॒ तस्य॒ व्यति॑षजे॒थ् स्त्री स्त्री व्यति॑षजे॒त् तस्य॒ तस्य॒ व्यति॑षजे॒थ् स्त्री । \newline
41. व्यति॑षजे॒थ् स्त्री स्त्री व्यति॑षजे॒द् व्यति॑षजे॒थ् स्त्र्ये॑वैव स्त्री व्यति॑षजे॒द् व्यति॑षजे॒थ् स्त्र्ये॑व । \newline
42. व्यति॑षजे॒दिति॑ वि - अति॑षजेत् । \newline
43. स्त्र्ये॑ वैव स्त्री स्त्र्ये॑ वास्या᳚ स्यै॒व स्त्री स्त्र्ये॑ वास्य॑ । \newline
44. ए॒वास्या᳚ स्यै॒वैवास्य॑ जायते जायते ऽस्यै॒वैवास्य॑ जायते । \newline
45. अ॒स्य॒ जा॒य॒ते॒ जा॒य॒ते॒ ऽस्या॒ स्य॒ जा॒य॒ते॒ यं ॅयम् जा॑यते ऽस्या स्य जायते॒ यम् । \newline
46. जा॒य॒ते॒ यं ॅयम् जा॑यते जायते॒ यम् का॒मये॑त का॒मये॑त॒ यम् जा॑यते जायते॒ यम् का॒मये॑त । \newline
47. यम् का॒मये॑त का॒मये॑त॒ यं ॅयम् का॒मये॑त॒ पुमा॒न् पुमा᳚न् का॒मये॑त॒ यं ॅयम् का॒मये॑त॒ पुमान्॑ । \newline
48. का॒मये॑त॒ पुमा॒न् पुमा᳚न् का॒मये॑त का॒मये॑त॒ पुमा॑ नस्यास्य॒ पुमा᳚न् का॒मये॑त का॒मये॑त॒ पुमा॑ नस्य । \newline
49. पुमा॑ नस्यास्य॒ पुमा॒न् पुमा॑ नस्य जायेत जायेतास्य॒ पुमा॒न् पुमा॑ नस्य जायेत । \newline
50. अ॒स्य॒ जा॒ये॒त॒ जा॒ये॒ता॒ स्या॒स्य॒ जा॒ये॒तेतीति॑ जायेता स्यास्य जाये॒तेति॑ । \newline
51. जा॒ये॒तेतीति॑ जायेत जाये॒ते त्या॒न्त मा॒न्त मिति॑ जायेत जाये॒ते त्या॒न्तम् । \newline
52. इत्या॒न्त मा॒न्त मिती त्या॒न्तम् तस्य॒ तस्या॒न्त मिती त्या॒न्तम् तस्य॑ । \newline
53. आ॒न्तम् तस्य॒ तस्या॒न्त मा॒न्तम् तस्य॒ प्र प्र तस्या॒न्त मा॒न्तम् तस्य॒ प्र । \newline
54. आ॒न्तमित्या᳚ - अ॒न्तम् । \newline
55. तस्य॒ प्र प्र तस्य॒ तस्य॒ प्र वे᳚ष्टयेद् वेष्टये॒त् प्र तस्य॒ तस्य॒ प्र वे᳚ष्टयेत् । \newline
56. प्र वे᳚ष्टयेद् वेष्टये॒त् प्र प्र वे᳚ष्टये॒त् पुमा॒न् पुमान्॑. वेष्टये॒त् प्र प्र वे᳚ष्टये॒त् पुमान्॑ । \newline
57. वे॒ष्ट॒ये॒त् पुमा॒न् पुमान्॑. वेष्टयेद् वेष्टये॒त् पुमा॑ ने॒वैव पुमान्॑. वेष्टयेद् वेष्टये॒त् पुमा॑ ने॒व । \newline
58. पुमा॑ ने॒वैव पुमा॒न् पुमा॑ ने॒वास्या᳚ स्यै॒व पुमा॒न् पुमा॑ ने॒वास्य॑ । \newline
59. ए॒वास्या᳚ स्यै॒वैवास्य॑ जायते जायते ऽस्यै॒वैवास्य॑ जायते । \newline
60. अ॒स्य॒ जा॒य॒ते॒ जा॒य॒ते॒ ऽस्या॒स्य॒ जा॒य॒ते ऽसु॑रा॒ असु॑रा जायते ऽस्यास्य जाय॒ते ऽसु॑राः । \newline
\pagebreak
\markright{ TS 6.6.4.4  \hfill https://www.vedavms.in \hfill}

\section{ TS 6.6.4.4 }

\textbf{TS 6.6.4.4 } \newline
\textbf{Samhita Paata} \newline

जाय॒ते ऽसु॑रा॒ वै दे॒वान् द॑क्षिण॒त उपा॑नय॒न् तान् दे॒वा उ॑पश॒येनै॒वापा॑-नुदन्त॒ त-दु॑पश॒यस्यो॑-पशय॒त्वं ॅयद्-द॑क्षिण॒त उ॑पश॒य उ॑प॒शये॒ भ्रातृ॑व्यापनुत्त्यै॒ सर्वे॒ वा अ॒न्ये यूपाः᳚ पशु॒मन्तोऽथो॑पश॒य ए॒वाप॒शुस्तस्य॒ यज॑मानः प॒शुर्यन्न नि॑र्दि॒शेदार्ति॒-मार्च्छे॒द्-यज॑मानो॒ऽसौ ते॑ प॒शुरिति॒ निर्दि॑शे॒द्यं द्वि॒ष्याद्-यमे॒व- [  ] \newline

\textbf{Pada Paata} \newline

जा॒य॒ते॒ । असु॑राः । वै । दे॒वान् । द॒क्षि॒ण॒तः । उपेति॑ । अ॒न॒य॒न्न् । तान् । दे॒वाः । उ॒प॒श॒येनेत्यु॑प - श॒येन॑ । ए॒व । अपेति॑ । अ॒नु॒द॒न्त॒ । तत् । उ॒प॒श॒यस्येत्यु॑प - श॒यस्य॑ । उ॒प॒श॒य॒त्वमित्यु॑पशय - त्वम् । यत् । द॒क्षि॒ण॒तः । उ॒प॒श॒य इत्यु॑प - श॒यः । उ॒प॒शय॒ इत्यु॑प - शये᳚ । भ्रातृ॑व्यापनुत्त्या॒ इति॒ भ्रातृ॑व्य-अ॒प॒नु॒त्त्यै॒ । सर्वे᳚ । वै । अ॒न्ये । यूपाः᳚ । प॒शु॒मन्त॒ इति॑ पशु - मन्तः॑ । अथ॑ । उ॒प॒श॒य इत्यु॑प - श॒यः । ए॒व । अ॒प॒शुः । तस्य॑ । यज॑मानः । प॒शुः । यत् । न । नि॒र्दि॒शेदिति॑ निः - दि॒शेत् । आर्ति᳚म् । एति॑ । ऋ॒च्छे॒त् । यज॑मानः । अ॒सौ । ते॒ । प॒शुः । इति॑ । निरिति॑ । दि॒शे॒त् । यम् । द्वि॒ष्यात् । यम् । ए॒व ।  \newline


\textbf{Krama Paata} \newline

जा॒य॒तेऽसु॑राः । असु॑रा॒ वै । वै दे॒वान् । दे॒वान् द॑क्षिण॒तः । द॒क्षि॒ण॒त उप॑ । उपा॑नयन्न् । अ॒न॒य॒न् तान् । तान् दे॒वाः । दे॒वा उ॑पश॒येन॑ । उ॒प॒श॒येनै॒व । उ॒प॒श॒येनेत्यु॑प - श॒येन॑ । ए॒वाप॑ । अपा॑नुदन्त । अ॒नु॒द॒न्त॒ तत् । तदु॑पश॒यस्य॑ । उ॒प॒श॒यस्यो॑पशय॒त्वम् । उ॒प॒श॒यस्येत्यु॑प - श॒यस्य॑ । उ॒प॒श॒य॒त्वम् ॅयत् । उ॒प॒श॒य॒त्वमित्यु॑पशय - त्वम् । यद् द॑क्षिण॒तः । द॒क्षि॒ण॒त उ॑पश॒यः । उ॒प॒श॒य उ॑प॒शये᳚ । उ॒प॒श॒य इत्यु॑प - श॒यः । उ॒प॒शये॒ भ्रातृ॑व्यापनुत्यै । उ॒प॒शय॒ इत्यु॑प - शये᳚ । भ्रातृ॑व्यापनुत्यै॒ सर्वे᳚ । भ्रातृ॑व्यापनुत्या॒ इति॒ भ्रातृ॑व्य - अ॒प॒नु॒त्यै॒ । सर्वे॒ वै । वा अ॒न्ये । अ॒न्ये यूपाः᳚ । यूपाः᳚ पशु॒मन्तः॑ । प॒शु॒मन्तोऽथ॑ । प॒शु॒मन्त॒ इति॑ पशु - मन्तः॑ । अथो॑पश॒यः । उ॒प॒श॒य ए॒व । उ॒प॒श॒य इत्यु॑प - श॒यः । ए॒वाप॒शुः । अ॒प॒शुस्तस्य॑ । तस्य॒ यज॑मानः । यज॑मानः प॒शुः । प॒शुर् यत् । यन् न । न नि॑र्दि॒शेत् । नि॒र्दि॒शेदार्ति᳚म् । नि॒र्दि॒शेदिति॑ निः - दि॒शेत् । आर्ति॒मा । आर्च्छे᳚त् । ऋ॒च्छे॒द् यज॑मानः । यज॑मानो॒ऽसौ । अ॒सौ ते᳚ । ते॒ प॒शुः । प॒शुरिति॑ । इति॒ निः । निर् दि॑शेत् । दि॒शे॒द् यम् । यम् द्वि॒ष्यात् । द्वि॒ष्याद् यम् । यमे॒व । ए॒व द्वेष्टि॑ \newline

\textbf{Jatai Paata} \newline

1. जा॒य॒ते ऽसु॑रा॒ असु॑रा जायते जाय॒ते ऽसु॑राः । \newline
2. असु॑रा॒ वै वा असु॑रा॒ असु॑रा॒ वै । \newline
3. वै दे॒वान् दे॒वान्. वै वै दे॒वान् । \newline
4. दे॒वान् द॑क्षिण॒तो द॑क्षिण॒तो दे॒वान् दे॒वान् द॑क्षिण॒तः । \newline
5. द॒क्षि॒ण॒त उपोप॑ दक्षिण॒तो द॑क्षिण॒त उप॑ । \newline
6. उपा॑नयन् ननय॒न् नुपोपा॑नयन्न् । \newline
7. अ॒न॒य॒न् ताꣳ स्तान॑नयन् ननय॒न् तान् । \newline
8. तान् दे॒वा दे॒वा स्ताꣳ स्तान् दे॒वाः । \newline
9. दे॒वा उ॑पश॒ये नो॑पश॒येन॑ दे॒वा दे॒वा उ॑पश॒येन॑ । \newline
10. उ॒प॒श॒ये नै॒वै वोप॑श॒ये नो॑पश॒ये नै॒व । \newline
11. उ॒प॒श॒येनेत्यु॑प - श॒येन॑ । \newline
12. ए॒वापा पै॒वै वाप॑ । \newline
13. अपा॑ नुदन्ता नुद॒न्ता पापा॑ नुदन्त । \newline
14. अ॒नु॒द॒न्त॒ तत् तद॑नुदन्ता नुदन्त॒ तत् । \newline
15. तदु॑पश॒य स्यो॑पश॒यस्य॒ तत् तदु॑पश॒यस्य॑ । \newline
16. उ॒प॒श॒य स्यो॑पशय॒त्व मु॑पशय॒त्व मु॑पश॒य स्यो॑पश॒य स्यो॑पशय॒त्वम् । \newline
17. उ॒प॒श॒यस्येत्यु॑प - श॒यस्य॑ । \newline
18. उ॒प॒श॒य॒त्वं ॅयद् यदु॑पशय॒त्व मु॑पशय॒त्वं ॅयत् । \newline
19. उ॒प॒श॒य॒त्वमित्यु॑पशय - त्वम् । \newline
20. यद् द॑क्षिण॒तो द॑क्षिण॒तो यद् यद् द॑क्षिण॒तः । \newline
21. द॒क्षि॒ण॒त उ॑पश॒य उ॑पश॒यो द॑क्षिण॒तो द॑क्षिण॒त उ॑पश॒यः । \newline
22. उ॒प॒श॒य उ॑प॒शय॑ उप॒शय॑ उपश॒य उ॑पश॒य उ॑प॒शये᳚ । \newline
23. उ॒प॒श॒य इत्यु॑प - श॒यः । \newline
24. उ॒प॒शये॒ भ्रातृ॑व्यापनुत्त्यै॒ भ्रातृ॑व्यापनुत्त्या उप॒शय॑ उप॒शये॒ भ्रातृ॑व्यापनुत्त्यै । \newline
25. उ॒प॒शय॒ इत्यु॑प - शये᳚ । \newline
26. भ्रातृ॑व्यापनुत्त्यै॒ सर्वे॒ सर्वे॒ भ्रातृ॑व्यापनुत्त्यै॒ भ्रातृ॑व्यापनुत्त्यै॒ सर्वे᳚ । \newline
27. भ्रातृ॑व्यापनुत्त्या॒ इति॒ भ्रातृ॑व्य - अ॒प॒नु॒त्त्यै॒ । \newline
28. सर्वे॒ वै वै सर्वे॒ सर्वे॒ वै । \newline
29. वा अ॒न्ये᳚ ऽन्ये वै वा अ॒न्ये । \newline
30. अ॒न्ये यूपा॒ यूपा॑ अ॒न्ये᳚ ऽन्ये यूपाः᳚ । \newline
31. यूपाः᳚ पशु॒मन्तः॑ पशु॒मन्तो॒ यूपा॒ यूपाः᳚ पशु॒मन्तः॑ । \newline
32. प॒शु॒मन्तो ऽथाथ॑ पशु॒मन्तः॑ पशु॒मन्तो ऽथ॑ । \newline
33. प॒शु॒मन्त॒ इति॑ पशु - मन्तः॑ । \newline
34. अथो॑पश॒य उ॑पश॒यो ऽथाथो॑पश॒यः । \newline
35. उ॒प॒श॒य ए॒वैवोप॑श॒य उ॑पश॒य ए॒व । \newline
36. उ॒प॒श॒य इत्यु॑प - श॒यः । \newline
37. ए॒वा प॒शु र॑प॒शु रे॒वै वाप॒शुः । \newline
38. अ॒प॒शु स्तस्य॒ तस्या॑ प॒शु र॑प॒शु स्तस्य॑ । \newline
39. तस्य॒ यज॑मानो॒ यज॑मान॒ स्तस्य॒ तस्य॒ यज॑मानः । \newline
40. यज॑मानः प॒शुः प॒शुर् यज॑मानो॒ यज॑मानः प॒शुः । \newline
41. प॒शुर् यद् यत् प॒शुः प॒शुर् यत् । \newline
42. यन् न न यद् यन् न । \newline
43. न नि॑र्दि॒शेन् नि॑र्दि॒शेन् न न नि॑र्दि॒शेत् । \newline
44. नि॒र्दि॒शे दार्ति॒ मार्ति॑न् निर्दि॒शेन् नि॑र्दि॒शे दार्ति᳚म् । \newline
45. नि॒र्दि॒शेदिति॑ निः - दि॒शेत् । \newline
46. आर्ति॒ मा ऽऽर्ति॒ मार्ति॒ मा । \newline
47. आर्च्छे॑ दृच्छे॒ दार्च्छे᳚त् । \newline
48. ऋ॒च्छे॒द् यज॑मानो॒ यज॑मान ऋच्छे दृच्छे॒द् यज॑मानः । \newline
49. यज॑मानो॒ ऽसा व॒सौ यज॑मानो॒ यज॑मानो॒ ऽसौ । \newline
50. अ॒सौ ते॑ ते॒ ऽसा व॒सौ ते᳚ । \newline
51. ते॒ प॒शुः प॒शु स्ते॑ ते प॒शुः । \newline
52. प॒शु रितीति॑ प॒शुः प॒शु रिति॑ । \newline
53. इति॒ निर् णि रितीति॒ निः । \newline
54. निर् दि॑शेद् दिशे॒न् निर् णिर् दि॑शेत् । \newline
55. दि॒शे॒द् यं ॅयम् दि॑शेद् दिशे॒द् यम् । \newline
56. यम् द्वि॒ष्याद् द्वि॒ष्याद् यं ॅयम् द्वि॒ष्यात् । \newline
57. द्वि॒ष्याद् यं ॅयम् द्वि॒ष्याद् द्वि॒ष्याद् यम् । \newline
58. य मे॒वैव यं ॅय मे॒व । \newline
59. ए॒व द्वेष्टि॒ द्वेष्ट्ये॒ वैव द्वेष्टि॑ । \newline

\textbf{Ghana Paata } \newline

1. जा॒य॒ते ऽसु॑रा॒ असु॑रा जायते जाय॒ते ऽसु॑रा॒ वै वा असु॑रा जायते जाय॒ते ऽसु॑रा॒ वै । \newline
2. असु॑रा॒ वै वा असु॑रा॒ असु॑रा॒ वै दे॒वान् दे॒वान्. वा असु॑रा॒ असु॑रा॒ वै दे॒वान् । \newline
3. वै दे॒वान् दे॒वान्. वै वै दे॒वान् द॑क्षिण॒तो द॑क्षिण॒तो दे॒वान्. वै वै दे॒वान् द॑क्षिण॒तः । \newline
4. दे॒वान् द॑क्षिण॒तो द॑क्षिण॒तो दे॒वान् दे॒वान् द॑क्षिण॒त उपोप॑ दक्षिण॒तो दे॒वान् दे॒वान् द॑क्षिण॒त उप॑ । \newline
5. द॒क्षि॒ण॒त उपोप॑ दक्षिण॒तो द॑क्षिण॒त उपा॑नयन् ननय॒न् नुप॑ दक्षिण॒तो द॑क्षिण॒त उपा॑नयन्न् । \newline
6. उपा॑नयन् ननय॒न् नुपोपा॑नय॒न् ताꣳ स्ता न॑नय॒न् नुपोपा॑नय॒न् तान् । \newline
7. अ॒न॒य॒न् ताꣳ स्तान॑नयन् ननय॒न् तान् दे॒वा दे॒वा स्ता न॑नयन् ननय॒न् तान् दे॒वाः । \newline
8. तान् दे॒वा दे॒वा स्ताꣳ स्तान् दे॒वा उ॑पश॒ये नो॑पश॒येन॑ दे॒वा स्ताꣳ स्तान् दे॒वा उ॑पश॒येन॑ । \newline
9. दे॒वा उ॑पश॒ये नो॑पश॒येन॑ दे॒वा दे॒वा उ॑पश॒ये नै॒वै वोप॑श॒येन॑ दे॒वा दे॒वा उ॑पश॒ येनै॒व । \newline
10. उ॒प॒श॒ये नै॒वै वोप॑श॒ये नो॑पश॒ये नै॒वापा पै॒वोप॑श॒ये नो॑पश॒ये नै॒वाप॑ । \newline
11. उ॒प॒श॒येनेत्यु॑प - श॒येन॑ । \newline
12. ए॒वा पापै॒ वैवा पा॑नुदन्ता नुद॒न्ता पै॒वैवा पा॑नुदन्त । \newline
13. अपा॑ नुदन्ता नुद॒न्ता पापा॑ नुदन्त॒ तत् तद॑नुद॒न्ता पापा॑ नुदन्त॒ तत् । \newline
14. अ॒नु॒द॒न्त॒ तत् तद॑नुदन्ता नुदन्त॒ तदु॑पश॒य स्यो॑पश॒यस्य॒ तद॑नुदन्ता नुदन्त॒ तदु॑पश॒यस्य॑ । \newline
15. तदु॑पश॒य स्यो॑पश॒यस्य॒ तत् तदु॑पश॒य स्यो॑पशय॒त्व मु॑पशय॒त्व मु॑पश॒यस्य॒ तत् तदु॑पश॒य स्यो॑पशय॒त्वम् । \newline
16. उ॒प॒श॒य स्यो॑पशय॒त्व मु॑पशय॒त्व मु॑पश॒य स्यो॑पश॒य स्यो॑पशय॒त्वं ॅयद् यदु॑पशय॒त्व मु॑पश॒य स्यो॑पश॒य स्यो॑पशय॒त्वं ॅयत् । \newline
17. उ॒प॒श॒यस्येत्यु॑प - श॒यस्य॑ । \newline
18. उ॒प॒श॒य॒त्वं ॅयद् यदु॑पशय॒त्व मु॑पशय॒त्वं ॅयद् द॑क्षिण॒तो द॑क्षिण॒तो यदु॑पशय॒त्व मु॑पशय॒त्वं ॅयद् द॑क्षिण॒तः । \newline
19. उ॒प॒श॒य॒त्वमित्यु॑पशय - त्वम् । \newline
20. यद् द॑क्षिण॒तो द॑क्षिण॒तो यद् यद् द॑क्षिण॒त उ॑पश॒य उ॑पश॒यो द॑क्षिण॒तो यद् यद् द॑क्षिण॒त उ॑पश॒यः । \newline
21. द॒क्षि॒ण॒त उ॑पश॒य उ॑पश॒यो द॑क्षिण॒तो द॑क्षिण॒त उ॑पश॒य उ॑प॒शय॑ उप॒शय॑ उपश॒यो द॑क्षिण॒तो द॑क्षिण॒त उ॑पश॒य उ॑प॒शये᳚ । \newline
22. उ॒प॒श॒य उ॑प॒शय॑ उप॒शय॑ उपश॒य उ॑पश॒य उ॑प॒शये॒ भ्रातृ॑व्यापनुत्त्यै॒ भ्रातृ॑व्यापनुत्त्या उप॒शय॑ उपश॒य उ॑पश॒य उ॑प॒शये॒ भ्रातृ॑व्यापनुत्त्यै । \newline
23. उ॒प॒श॒य इत्यु॑प - श॒यः । \newline
24. उ॒प॒शये॒ भ्रातृ॑व्यापनुत्त्यै॒ भ्रातृ॑व्यापनुत्त्या उप॒शय॑ उप॒शये॒ भ्रातृ॑व्यापनुत्त्यै॒ सर्वे॒ सर्वे॒ भ्रातृ॑व्यापनुत्त्या उप॒शय॑ उप॒शये॒ भ्रातृ॑व्यापनुत्त्यै॒ सर्वे᳚ । \newline
25. उ॒प॒शय॒ इत्यु॑प - शये᳚ । \newline
26. भ्रातृ॑व्यापनुत्त्यै॒ सर्वे॒ सर्वे॒ भ्रातृ॑व्यापनुत्त्यै॒ भ्रातृ॑व्यापनुत्त्यै॒ सर्वे॒ वै वै सर्वे॒ भ्रातृ॑व्यापनुत्त्यै॒ भ्रातृ॑व्यापनुत्त्यै॒ सर्वे॒ वै । \newline
27. भ्रातृ॑व्यापनुत्त्या॒ इति॒ भ्रातृ॑व्य - अ॒प॒नु॒त्त्यै॒ । \newline
28. सर्वे॒ वै वै सर्वे॒ सर्वे॒ वा अ॒न्ये᳚ ऽन्ये वै सर्वे॒ सर्वे॒ वा अ॒न्ये । \newline
29. वा अ॒न्ये᳚ ऽन्ये वै वा अ॒न्ये यूपा॒ यूपा॑ अ॒न्ये वै वा अ॒न्ये यूपाः᳚ । \newline
30. अ॒न्ये यूपा॒ यूपा॑ अ॒न्ये᳚ ऽन्ये यूपाः᳚ पशु॒मन्तः॑ पशु॒मन्तो॒ यूपा॑ अ॒न्ये᳚ ऽन्ये यूपाः᳚ पशु॒मन्तः॑ । \newline
31. यूपाः᳚ पशु॒मन्तः॑ पशु॒मन्तो॒ यूपा॒ यूपाः᳚ पशु॒मन्तो ऽथाथ॑ पशु॒मन्तो॒ यूपा॒ यूपाः᳚ पशु॒मन्तो ऽथ॑ । \newline
32. प॒शु॒मन्तो ऽथाथ॑ पशु॒मन्तः॑ पशु॒मन्तो ऽथो॑पश॒य उ॑पश॒यो ऽथ॑ पशु॒मन्तः॑ पशु॒मन्तो ऽथो॑पश॒यः । \newline
33. प॒शु॒मन्त॒ इति॑ पशु - मन्तः॑ । \newline
34. अथो॑ पश॒य उ॑पश॒यो ऽथाथो॑ पश॒य ए॒वैवोप॑श॒यो ऽथाथो॑ पश॒य ए॒व । \newline
35. उ॒प॒श॒य ए॒वैवोप॑श॒य उ॑पश॒य ए॒वा प॒शुर॑ प॒शु रे॒वोप॑श॒य उ॑पश॒य ए॒वाप॒शुः । \newline
36. उ॒प॒श॒य इत्यु॑प - श॒यः । \newline
37. ए॒वा प॒शु र॑प॒शु रे॒वैवा प॒शु स्तस्य॒ तस्या॑ प॒शु रे॒वैवा प॒शु स्तस्य॑ । \newline
38. अ॒प॒शु स्तस्य॒ तस्या॑ प॒शु र॑प॒शु स्तस्य॒ यज॑मानो॒ यज॑मान॒ स्तस्या॑ प॒शु र॑प॒शु स्तस्य॒ यज॑मानः । \newline
39. तस्य॒ यज॑मानो॒ यज॑मान॒ स्तस्य॒ तस्य॒ यज॑मानः प॒शुः प॒शुर् यज॑मान॒ स्तस्य॒ तस्य॒ यज॑मानः प॒शुः । \newline
40. यज॑मानः प॒शुः प॒शुर् यज॑मानो॒ यज॑मानः प॒शुर् यद् यत् प॒शुर् यज॑मानो॒ यज॑मानः प॒शुर् यत् । \newline
41. प॒शुर् यद् यत् प॒शुः प॒शुर् यन् न न यत् प॒शुः प॒शुर् यन् न । \newline
42. यन् न न यद् यन् न नि॑र्दि॒शेन् नि॑र्दि॒शेन् न यद् यन् न नि॑र्दि॒शेत् । \newline
43. न नि॑र्दि॒शेन् नि॑र्दि॒शेन् न न नि॑र्दि॒शे दार्ति॒ मार्ति॑म् निर्दि॒शेन् न न नि॑र्दि॒शे दार्ति᳚म् । \newline
44. नि॒र्दि॒शे दार्ति॒ मार्ति॑म् निर्दि॒शेन् नि॑र्दि॒शे दार्ति॒ मा ऽऽर्ति॑म् निर्दि॒शेन् नि॑र्दि॒शे दार्ति॒ मा । \newline
45. नि॒र्दि॒शेदिति॑ निः - दि॒शेत् । \newline
46. आर्ति॒ मा ऽऽर्ति॒ मार्ति॒ मार्च्छे॑ दृच्छे॒दा ऽऽर्ति॒ मार्ति॒ मार्च्छे᳚त् । \newline
47. आर्च्छे॑ दृच्छे॒ दार्च्छे॒द् यज॑मानो॒ यज॑मान ऋच्छे॒ दार्च्छे॒द् यज॑मानः । \newline
48. ऋ॒च्छे॒द् यज॑मानो॒ यज॑मान ऋच्छे दृच्छे॒द् यज॑मानो॒ ऽसा व॒सौ यज॑मान ऋच्छे दृच्छे॒द् यज॑मानो॒ ऽसौ । \newline
49. यज॑मानो॒ ऽसा व॒सौ यज॑मानो॒ यज॑मानो॒ ऽसौ ते॑ ते॒ ऽसौ यज॑मानो॒ यज॑मानो॒ ऽसौ ते᳚ । \newline
50. अ॒सौ ते॑ ते॒ ऽसा व॒सौ ते॑ प॒शुः प॒शु स्ते॒ ऽसा व॒सौ ते॑ प॒शुः । \newline
51. ते॒ प॒शुः प॒शु स्ते॑ ते प॒शु रितीति॑ प॒शु स्ते॑ ते प॒शु रिति॑ । \newline
52. प॒शु रितीति॑ प॒शुः प॒शु रिति॒ निर् णिरिति॑ प॒शुः प॒शु रिति॒ निः । \newline
53. इति॒ निर् णिरितीति॒ निर् दि॑शेद् दिशे॒न् निरितीति॒ निर् दि॑शेत् । \newline
54. निर् दि॑शेद् दिशे॒न् निर् णिर् दि॑शे॒द् यं ॅयम् दि॑शे॒न् निर् णिर् दि॑शे॒द् यम् । \newline
55. दि॒शे॒द् यं ॅयम् दि॑शेद् दिशे॒द् यम् द्वि॒ष्याद् द्वि॒ष्याद् यम् दि॑शेद् दिशे॒द् यम् द्वि॒ष्यात् । \newline
56. यम् द्वि॒ष्याद् द्वि॒ष्याद् यं ॅयम् द्वि॒ष्याद् यं ॅयम् द्वि॒ष्याद् यं ॅयम् द्वि॒ष्याद् यम् । \newline
57. द्वि॒ष्याद् यं ॅयम् द्वि॒ष्याद् द्वि॒ष्याद् य मे॒वैव यम् द्वि॒ष्याद् द्वि॒ष्याद् य मे॒व । \newline
58. य मे॒वैव यं ॅय मे॒व द्वेष्टि॒ द्वेष्ट्ये॒व यं ॅय मे॒व द्वेष्टि॑ । \newline
59. ए॒व द्वेष्टि॒ द्वेष्ट्ये॒वैव द्वेष्टि॒ तम् तम् द्वेष्ट्ये॒वैव द्वेष्टि॒ तम् । \newline
\pagebreak
\markright{ TS 6.6.4.5  \hfill https://www.vedavms.in \hfill}

\section{ TS 6.6.4.5 }

\textbf{TS 6.6.4.5 } \newline
\textbf{Samhita Paata} \newline

द्वेष्टि॒ तम॑स्मै प॒शुं निर्दि॑शति॒ यदि॒ न द्वि॒ष्यादा॒खुस्ते॑ प॒शुरिति॑ ब्रूया॒न्न ग्रा॒म्यान् प॒शून्. हि॒नस्ति॒ नाऽऽ*र॒ण्यान् प्र॒जाप॑तिः प्र॒जा अ॑सृजत॒ सो᳚ऽन्नाद्ये॑न॒ व्या᳚र्द्ध्यत॒ स ए॒तामे॑काद॒शिनी॑-मपश्य॒त् तया॒ वै सो᳚ऽन्नाद्य॒मवा॑रुन्ध॒ यद्दश॒ यूपा॒ भव॑न्ति॒ दशा᳚क्षरा वि॒राडन्नं॑ ॅवि॒राड् वि॒राजै॒वा-न्नाद्य॒मव॑ रुन्धे॒- [  ] \newline

\textbf{Pada Paata} \newline

द्वेष्टि॑ । तम् । अ॒स्मै॒ । प॒शुम् । निरिति॑ । दि॒श॒ति॒ । यदि॑ । न । द्वि॒ष्यात् । आ॒खुः । ते॒ । प॒शुः । इति॑ । ब्रू॒या॒त् । न । ग्रा॒म्यान् । प॒शून् । हि॒नस्ति॑ । न । आ॒र॒ण्यान् । प्र॒जाप॑ति॒रिति॑ प्र॒जा - प॒तिः॒ । प्र॒जा इति॑ प्र - जाः । अ॒सृ॒ज॒त॒ । सः । अ॒न्नाद्ये॒नेत्य॑न्न - अद्ये॑न । वीति॑ । अ॒द्‌र्ध्य॒त॒ । सः । ए॒ताम् । ए॒का॒द॒शिनी᳚म् । अ॒प॒श्य॒त् । तया᳚ । वै । सः । अ॒न्नाद्य॒मित्य॑न्न - अद्य᳚म् । अवेति॑ । अ॒रु॒न्ध॒ । यत् । दश॑ । यूपाः᳚ । भव॑न्ति । दशा᳚क्ष॒रेति॒ दश॑ - अ॒क्ष॒रा॒ । वि॒राडिति॑ वि - राट् । अन्न᳚म् । वि॒राडिति॑ वि - राट् । वि॒राजेति॑ वि - राजा᳚ । ए॒व । अ॒न्नाद्य॒मित्य॑न्न - अद्य᳚म् । अवेति॑ । रु॒न्धे॒ ।  \newline


\textbf{Krama Paata} \newline

द्वेष्टि॒ तम् । तम॑स्मै । अ॒स्मै॒ प॒शुम् । प॒शुम् निः । निर् दि॑शति । दि॒श॒ति॒ यदि॑ । यदि॒ न । न द्वि॒ष्यात् । द्वि॒ष्यादा॒खुः । आ॒खुस्ते᳚ । ते॒ प॒शुः । प॒शुरिति॑ । इति॑ ब्रूयात् । ब्रू॒या॒न् न । न ग्रा॒म्यान् । ग्रा॒म्यान् प॒शून् । प॒शून्. हि॒नस्ति॑ । हि॒नस्ति॒ न । नार॒ण्यान् । आ॒र॒ण्यान् प्र॒जाप॑तिः । प्र॒जाप॑तिः प्र॒जाः । प्र॒जाप॑ति॒रिति॑ प्र॒जा - प॒तिः॒ । प्र॒जा अ॑सृजत । प्र॒जा इति॑ प्र - जाः । अ॒सृ॒ज॒त॒ सः । सो᳚ऽन्नाद्ये॑न । अ॒न्नाद्ये॑न॒ वि । अ॒न्नाद्ये॒नेत्य॑न्न - अद्ये॑न । व्या᳚र्द्ध्यत । आ॒र्द्ध्य॒त॒ सः । स ए॒ताम् । ए॒तामे॑काद॒शिनी᳚म् । ए॒का॒द॒शिनी॑मपश्यत् । अ॒प॒श्य॒त् तया᳚ । तया॒ वै । वै सः । सो᳚ऽन्नाद्य᳚म् । अ॒न्नाद्य॒मव॑ । अ॒न्नाद्य॒मित्य॑न्न - अद्य᳚म् । अवा॑रुन्ध । अ॒रु॒न्ध॒ यत् । यद् दश॑ । दश॒ यूपाः᳚ । यूपा॒ भव॑न्ति । भव॑न्ति॒ दशा᳚क्षरा । दशा᳚क्षरा वि॒राट् । दशा᳚क्ष॒रेति॒ दश॑ - अ॒क्ष॒रा॒ । वि॒राडन्न᳚म् । वि॒राडिति॑ वि - राट् । अन्न॑म् ॅवि॒राट् । वि॒राड् वि॒राजा᳚ । वि॒राडिति॑ वि - राट् । वि॒राजै॒व । वि॒राजेति॑ वि - राजा᳚ । ए॒वान्नाद्य᳚म् । अ॒न्नाद्य॒मव॑ । अ॒न्नाद्य॒मित्य॑न्न - अद्य᳚म् । अव॑ रुन्धे ( ) । रु॒न्धे॒ यः \newline

\textbf{Jatai Paata} \newline

1. द्वेष्टि॒ तम् तम् द्वेष्टि॒ द्वेष्टि॒ तम् । \newline
2. त म॑स्मा अस्मै॒ तम् त म॑स्मै । \newline
3. अ॒स्मै॒ प॒शुम् प॒शु म॑स्मा अस्मै प॒शुम् । \newline
4. प॒शुन् निर् णिष् प॒शुम् प॒शुन् निः । \newline
5. निर् दि॑शति दिशति॒ निर् णिर् दि॑शति । \newline
6. दि॒श॒ति॒ यदि॒ यदि॑ दिशति दिशति॒ यदि॑ । \newline
7. यदि॒ न न यदि॒ यदि॒ न । \newline
8. न द्वि॒ष्याद् द्वि॒ष्यान् न न द्वि॒ष्यात् । \newline
9. द्वि॒ष्या दा॒खु रा॒खुर् द्वि॒ष्याद् द्वि॒ष्या दा॒खुः । \newline
10. आ॒खु स्ते॑ त आ॒खु रा॒खु स्ते᳚ । \newline
11. ते॒ प॒शुः प॒शु स्ते॑ ते प॒शुः । \newline
12. प॒शु रितीति॑ प॒शुः प॒शु रिति॑ । \newline
13. इति॑ ब्रूयाद् ब्रूया॒ दितीति॑ ब्रूयात् । \newline
14. ब्रू॒या॒न् न न ब्रू॑याद् ब्रूया॒न् न । \newline
15. न ग्रा॒म्यान् ग्रा॒म्यान् न न ग्रा॒म्यान् । \newline
16. ग्रा॒म्यान् प॒शून् प॒शून् ग्रा॒म्यान् ग्रा॒म्यान् प॒शून् । \newline
17. प॒शून्. हि॒नस्ति॑ हि॒नस्ति॑ प॒शून् प॒शून्. हि॒नस्ति॑ । \newline
18. हि॒नस्ति॒ न न हि॒नस्ति॑ हि॒नस्ति॒ न । \newline
19. नार॒ण्या ना॑र॒ण्यान् न नार॒ण्यान् । \newline
20. आ॒र॒ण्यान् प्र॒जाप॑तिः प्र॒जाप॑ति रार॒ण्या ना॑र॒ण्यान् प्र॒जाप॑तिः । \newline
21. प्र॒जाप॑तिः प्र॒जाः प्र॒जाः प्र॒जाप॑तिः प्र॒जाप॑तिः प्र॒जाः । \newline
22. प्र॒जाप॑ति॒रिति॑ प्र॒जा - प॒तिः॒ । \newline
23. प्र॒जा अ॑सृजता सृजत प्र॒जाः प्र॒जा अ॑सृजत । \newline
24. प्र॒जा इति॑ प्र - जाः । \newline
25. अ॒सृ॒ज॒त॒ स सो॑ ऽसृजता सृजत॒ सः । \newline
26. सो᳚ ऽन्नाद्ये॑ना॒ न्नाद्ये॑न॒ स सो᳚ ऽन्नाद्ये॑न । \newline
27. अ॒न्नाद्ये॑न॒ वि व्य॑न्नाद्ये॑ना॒ न्नाद्ये॑न॒ वि । \newline
28. अ॒न्नाद्ये॒नेत्य॑न्न - अद्ये॑न । \newline
29. व्या᳚र्द्ध्यता र्द्ध्यत॒ वि व्या᳚र्द्ध्यत । \newline
30. आ॒र्द्ध्य॒त॒ स स आ᳚र्द्ध्यता र्द्ध्यत॒ सः । \newline
31. स ए॒ता मे॒ताꣳ स स ए॒ताम् । \newline
32. ए॒ता मे॑काद॒शिनी॑ मेकाद॒शिनी॑ मे॒ता मे॒ता मे॑काद॒शिनी᳚म् । \newline
33. ए॒का॒द॒शिनी॑ मपश्य दपश्य देकाद॒शिनी॑ मेकाद॒शिनी॑ मपश्यत् । \newline
34. अ॒प॒श्य॒त् तया॒ तया॑ ऽपश्य दपश्य॒त् तया᳚ । \newline
35. तया॒ वै वै तया॒ तया॒ वै । \newline
36. वै स स वै वै सः । \newline
37. सो᳚ ऽन्नाद्य॑ म॒न्नाद्यꣳ॒॒ स सो᳚ ऽन्नाद्य᳚म् । \newline
38. अ॒न्नाद्य॒ मवा वा॒न्नाद्य॑ म॒न्नाद्य॒ मव॑ । \newline
39. अ॒न्नाद्य॒मित्य॑न्न - अद्य᳚म् । \newline
40. अवा॑ रुन्धा रु॒न्धा वावा॑ रुन्ध । \newline
41. अ॒रु॒न्ध॒ यद् यद॑रुन्धा रुन्ध॒ यत् । \newline
42. यद् दश॒ दश॒ यद् यद् दश॑ । \newline
43. दश॒ यूपा॒ यूपा॒ दश॒ दश॒ यूपाः᳚ । \newline
44. यूपा॒ भव॑न्ति॒ भव॑न्ति॒ यूपा॒ यूपा॒ भव॑न्ति । \newline
45. भव॑न्ति॒ दशा᳚क्षरा॒ दशा᳚क्षरा॒ भव॑न्ति॒ भव॑न्ति॒ दशा᳚क्षरा । \newline
46. दशा᳚क्षरा वि॒राड् वि॒राड् दशा᳚क्षरा॒ दशा᳚क्षरा वि॒राट् । \newline
47. दशा᳚क्ष॒रेति॒ दश॑ - अ॒क्ष॒रा॒ । \newline
48. वि॒रा डन्न॒ मन्नं॑ ॅवि॒राड् वि॒रा डन्न᳚म् । \newline
49. वि॒राडिति॑ वि - राट् । \newline
50. अन्नं॑ ॅवि॒राड् वि॒रा डन्न॒ मन्नं॑ ॅवि॒राट् । \newline
51. वि॒राड् वि॒राजा॑ वि॒राजा॑ वि॒राड् वि॒राड् वि॒राजा᳚ । \newline
52. वि॒राडिति॑ वि - राट् । \newline
53. वि॒राजै॒ वैव वि॒राजा॑ वि॒रा जै॒व । \newline
54. वि॒राजेति॑ वि - राजा᳚ । \newline
55. ए॒वान्नाद्य॑ म॒न्नाद्य॑ मे॒वै वान्नाद्य᳚म् । \newline
56. अ॒न्नाद्य॒ मवा वा॒न्नाद्य॑ म॒न्नाद्य॒ मव॑ । \newline
57. अ॒न्नाद्य॒मित्य॑न्न - अद्य᳚म् । \newline
58. अव॑ रुन्धे रु॒न्धे ऽवाव॑ रुन्धे । \newline
59. रु॒न्धे॒ यो यो रु॑न्धे रुन्धे॒ यः । \newline

\textbf{Ghana Paata } \newline

1. द्वेष्टि॒ तम् तम् द्वेष्टि॒ द्वेष्टि॒ त म॑स्मा अस्मै॒ तम् द्वेष्टि॒ द्वेष्टि॒ त म॑स्मै । \newline
2. त म॑स्मा अस्मै॒ तम् त म॑स्मै प॒शुम् प॒शु म॑स्मै॒ तम् त म॑स्मै प॒शुम् । \newline
3. अ॒स्मै॒ प॒शुम् प॒शु म॑स्मा अस्मै प॒शुन् निर् णिष् प॒शु म॑स्मा अस्मै प॒शुन् निः । \newline
4. प॒शुन् निर् णिष् प॒शुम् प॒शुन् निर् दि॑शति दिशति॒ निष् प॒शुम् प॒शुन् निर् दि॑शति । \newline
5. निर् दि॑शति दिशति॒ निर् णिर् दि॑शति॒ यदि॒ यदि॑ दिशति॒ निर् णिर् दि॑शति॒ यदि॑ । \newline
6. दि॒श॒ति॒ यदि॒ यदि॑ दिशति दिशति॒ यदि॒ न न यदि॑ दिशति दिशति॒ यदि॒ न । \newline
7. यदि॒ न न यदि॒ यदि॒ न द्वि॒ष्याद् द्वि॒ष्यान् न यदि॒ यदि॒ न द्वि॒ष्यात् । \newline
8. न द्वि॒ष्याद् द्वि॒ष्यान् न न द्वि॒ष्या दा॒खु रा॒खुर् द्वि॒ष्यान् न न द्वि॒ष्या दा॒खुः । \newline
9. द्वि॒ष्या दा॒खु रा॒खुर् द्वि॒ष्याद् द्वि॒ष्या दा॒खु स्ते॑ त आ॒खुर् द्वि॒ष्याद् द्वि॒ष्या दा॒खु स्ते᳚ । \newline
10. आ॒खु स्ते॑ त आ॒खु रा॒खु स्ते॑ प॒शुः प॒शु स्त॑ आ॒खु रा॒खु स्ते॑ प॒शुः । \newline
11. ते॒ प॒शुः प॒शु स्ते॑ ते प॒शु रितीति॑ प॒शु स्ते॑ ते प॒शु रिति॑ । \newline
12. प॒शु रितीति॑ प॒शुः प॒शु रिति॑ ब्रूयाद् ब्रूया॒ दिति॑ प॒शुः प॒शु रिति॑ ब्रूयात् । \newline
13. इति॑ ब्रूयाद् ब्रूया॒ दितीति॑ ब्रूया॒न् न न ब्रू॑या॒ दितीति॑ ब्रूया॒न् न । \newline
14. ब्रू॒या॒न् न न ब्रू॑याद् ब्रूया॒न् न ग्रा॒म्यान् ग्रा॒म्यान् न ब्रू॑याद् ब्रूया॒न् न ग्रा॒म्यान् । \newline
15. न ग्रा॒म्यान् ग्रा॒म्यान् न न ग्रा॒म्यान् प॒शून् प॒शून् ग्रा॒म्यान् न न ग्रा॒म्यान् प॒शून् । \newline
16. ग्रा॒म्यान् प॒शून् प॒शून् ग्रा॒म्यान् ग्रा॒म्यान् प॒शून्. हि॒नस्ति॑ हि॒नस्ति॑ प॒शून् ग्रा॒म्यान् ग्रा॒म्यान् प॒शून्. हि॒नस्ति॑ । \newline
17. प॒शून्. हि॒नस्ति॑ हि॒नस्ति॑ प॒शून् प॒शून्. हि॒नस्ति॒ न न हि॒नस्ति॑ प॒शून् प॒शून्. हि॒नस्ति॒ न । \newline
18. हि॒नस्ति॒ न न हि॒नस्ति॑ हि॒नस्ति॒ नार॒ण्या ना॑र॒ण्यान् न हि॒नस्ति॑ हि॒नस्ति॒ नार॒ण्यान् । \newline
19. नार॒ण्या ना॑र॒ण्यान् न नार॒ण्यान् प्र॒जाप॑तिः प्र॒जाप॑ति रार॒ण्यान् न नार॒ण्यान् प्र॒जाप॑तिः । \newline
20. आ॒र॒ण्यान् प्र॒जाप॑तिः प्र॒जाप॑ति रार॒ण्या ना॑र॒ण्यान् प्र॒जाप॑तिः प्र॒जाः प्र॒जाः प्र॒जाप॑ति रार॒ण्या ना॑र॒ण्यान् प्र॒जाप॑तिः प्र॒जाः । \newline
21. प्र॒जाप॑तिः प्र॒जाः प्र॒जाः प्र॒जाप॑तिः प्र॒जाप॑तिः प्र॒जा अ॑सृजता सृजत प्र॒जाः प्र॒जाप॑तिः प्र॒जाप॑तिः प्र॒जा अ॑सृजत । \newline
22. प्र॒जाप॑ति॒रिति॑ प्र॒जा - प॒तिः॒ । \newline
23. प्र॒जा अ॑सृजता सृजत प्र॒जाः प्र॒जा अ॑सृजत॒ स सो॑ ऽसृजत प्र॒जाः प्र॒जा अ॑सृजत॒ सः । \newline
24. प्र॒जा इति॑ प्र - जाः । \newline
25. अ॒सृ॒ज॒त॒ स सो॑ ऽसृजता सृजत॒ सो᳚ ऽन्नाद्ये॑ना॒ न्नाद्ये॑न॒ सो॑ ऽसृजता सृजत॒ सो᳚ ऽन्नाद्ये॑न । \newline
26. सो᳚ ऽन्नाद्ये॑ना॒ न्नाद्ये॑न॒ स सो᳚ ऽन्नाद्ये॑न॒ वि व्य॑न्नाद्ये॑न॒ स सो᳚ ऽन्नाद्ये॑न॒ वि । \newline
27. अ॒न्नाद्ये॑न॒ वि व्य॑न्नाद्ये॑ना॒ न्नाद्ये॑न॒ व्या᳚र्द्ध्यता र्द्ध्यत॒ व्य॑न्नाद्ये॑ना॒ न्नाद्ये॑न॒ व्या᳚र्द्ध्यत । \newline
28. अ॒न्नाद्ये॒नेत्य॑न्न - अद्ये॑न । \newline
29. व्या᳚र्द्ध्यता र्द्ध्यत॒ वि व्या᳚र्द्ध्यत॒ स स आ᳚र्द्ध्यत॒ वि व्या᳚र्द्ध्यत॒ सः । \newline
30. आ॒र्द्ध्य॒त॒ स स आ᳚र्द्ध्यता र्द्ध्यत॒ स ए॒ता मे॒ताꣳ स आ᳚र्द्ध्यता र्द्ध्यत॒ स ए॒ताम् । \newline
31. स ए॒ता मे॒ताꣳ स स ए॒ता मे॑काद॒शिनी॑ मेकाद॒शिनी॑ मे॒ताꣳ स स ए॒ता मे॑काद॒शिनी᳚म् । \newline
32. ए॒ता मे॑काद॒शिनी॑ मेकाद॒शिनी॑ मे॒ता मे॒ता मे॑काद॒शिनी॑ मपश्य दपश्य देकाद॒शिनी॑ मे॒ता मे॒ता मे॑काद॒शिनी॑ मपश्यत् । \newline
33. ए॒का॒द॒शिनी॑ मपश्य दपश्य देकाद॒शिनी॑ मेकाद॒शिनी॑ मपश्य॒त् तया॒ तया॑ ऽपश्य देकाद॒शिनी॑ मेकाद॒शिनी॑ मपश्य॒त् तया᳚ । \newline
34. अ॒प॒श्य॒त् तया॒ तया॑ ऽपश्य दपश्य॒त् तया॒ वै वै तया॑ ऽपश्य दपश्य॒त् तया॒ वै । \newline
35. तया॒ वै वै तया॒ तया॒ वै स स वै तया॒ तया॒ वै सः । \newline
36. वै स स वै वै सो᳚ ऽन्नाद्य॑ म॒न्नाद्यꣳ॒॒ स वै वै सो᳚ ऽन्नाद्य᳚म् । \newline
37. सो᳚ ऽन्नाद्य॑ म॒न्नाद्यꣳ॒॒ स सो᳚ ऽन्नाद्य॒ मवावा॒ न्नाद्यꣳ॒॒ स सो᳚ ऽन्नाद्य॒ मव॑ । \newline
38. अ॒न्नाद्य॒ मवावा॒ न्नाद्य॑ म॒न्नाद्य॒ मवा॑ रुन्धा रु॒न्धावा॒ न्नाद्य॑ म॒न्नाद्य॒ मवा॑ रुन्ध । \newline
39. अ॒न्नाद्य॒मित्य॑न्न - अद्य᳚म् । \newline
40. अवा॑ रुन्धा रु॒न्धा वावा॑ रुन्ध॒ यद् यद॑रु॒न्धा वावा॑ रुन्ध॒ यत् । \newline
41. अ॒रु॒न्ध॒ यद् यद॑रुन्धा रुन्ध॒ यद् दश॒ दश॒ यद॑रुन्धा रुन्ध॒ यद् दश॑ । \newline
42. यद् दश॒ दश॒ यद् यद् दश॒ यूपा॒ यूपा॒ दश॒ यद् यद् दश॒ यूपाः᳚ । \newline
43. दश॒ यूपा॒ यूपा॒ दश॒ दश॒ यूपा॒ भव॑न्ति॒ भव॑न्ति॒ यूपा॒ दश॒ दश॒ यूपा॒ भव॑न्ति । \newline
44. यूपा॒ भव॑न्ति॒ भव॑न्ति॒ यूपा॒ यूपा॒ भव॑न्ति॒ दशा᳚क्षरा॒ दशा᳚क्षरा॒ भव॑न्ति॒ यूपा॒ यूपा॒ भव॑न्ति॒ दशा᳚क्षरा । \newline
45. भव॑न्ति॒ दशा᳚क्षरा॒ दशा᳚क्षरा॒ भव॑न्ति॒ भव॑न्ति॒ दशा᳚क्षरा वि॒राड् वि॒राड् दशा᳚क्षरा॒ भव॑न्ति॒ भव॑न्ति॒ दशा᳚क्षरा वि॒राट् । \newline
46. दशा᳚क्षरा वि॒राड् वि॒राड् दशा᳚क्षरा॒ दशा᳚क्षरा वि॒रा डन्न॒ मन्नं॑ ॅवि॒राड् दशा᳚क्षरा॒ दशा᳚क्षरा वि॒रा डन्न᳚म् । \newline
47. दशा᳚क्ष॒रेति॒ दश॑ - अ॒क्ष॒रा॒ । \newline
48. वि॒रा डन्न॒ मन्नं॑ ॅवि॒राड् वि॒रा डन्नं॑ ॅवि॒राड् वि॒रा डन्नं॑ ॅवि॒राड् वि॒रा डन्नं॑ ॅवि॒राट् । \newline
49. वि॒राडिति॑ वि - राट् । \newline
50. अन्नं॑ ॅवि॒राड् वि॒रा डन्न॒ मन्नं॑ ॅवि॒राड् वि॒राजा॑ वि॒राजा॑ वि॒रा डन्न॒ मन्नं॑ ॅवि॒राड् वि॒राजा᳚ । \newline
51. वि॒राड् वि॒राजा॑ वि॒राजा॑ वि॒राड् वि॒राड् वि॒राजै॒वैव वि॒राजा॑ वि॒राड् वि॒राड् वि॒राजै॒व । \newline
52. वि॒राडिति॑ वि - राट् । \newline
53. वि॒राजै॒वैव वि॒राजा॑ वि॒रा जै॒वान्नाद्य॑ म॒न्नाद्य॑ मे॒व वि॒राजा॑ वि॒रा जै॒वान्नाद्य᳚म् । \newline
54. वि॒राजेति॑ वि - राजा᳚ । \newline
55. ए॒वान्नाद्य॑ म॒न्नाद्य॑ मे॒वैवान्नाद्य॒ मवावा॒न्नाद्य॑ मे॒वैवान्नाद्य॒ मव॑ । \newline
56. अ॒न्नाद्य॒ मवावा॒ न्नाद्य॑ म॒न्नाद्य॒ मव॑ रुन्धे रु॒न्धे ऽवा॒न्नाद्य॑ म॒न्नाद्य॒ मव॑ रुन्धे । \newline
57. अ॒न्नाद्य॒मित्य॑न्न - अद्य᳚म् । \newline
58. अव॑ रुन्धे रु॒न्धे ऽवाव॑ रुन्धे॒ यो यो रु॒न्धे ऽवाव॑ रुन्धे॒ यः । \newline
59. रु॒न्धे॒ यो यो रु॑न्धे रुन्धे॒ य ए॑काद॒श ए॑काद॒शो यो रु॑न्धे रुन्धे॒ य ए॑काद॒शः । \newline
\pagebreak
\markright{ TS 6.6.4.6  \hfill https://www.vedavms.in \hfill}

\section{ TS 6.6.4.6 }

\textbf{TS 6.6.4.6 } \newline
\textbf{Samhita Paata} \newline

य ए॑काद॒शः स्तन॑ ए॒वास्यै॒ स दु॒ह ए॒वैनां॒ तेन॒ वज्रो॒ वा ए॒षा सं मी॑यते॒ यदे॑काद॒शिनी॒ सेश्व॒रा पु॒रस्ता᳚त् प्र॒त्यञ्चं॑ ॅय॒ज्ञ्ꣳ संम॑र्दितो॒र्यत् पा᳚त्नीव॒तं मि॒नोति॑ य॒ज्ञ्स्य॒ प्रत्युत्त॑ब्ध्यै सय॒त्वाय॑ ॥ \newline

\textbf{Pada Paata} \newline

यः । ए॒का॒द॒शः । स्तनः॑ । ए॒व । अ॒स्यै॒ । सः । दु॒हे । ए॒व । ए॒ना॒म् । तेन॑ । वज्रः॑ । वै । ए॒षा । समिति॑ । मी॒य॒ते॒ । यत् । ए॒का॒द॒शिनी᳚ । सा । ई॒श्व॒रा । पु॒रस्ता᳚त् । प्र॒त्यञ्च᳚म् । य॒ज्ञ्म् । संम॑र्दितो॒रिति॒ सं - म॒र्दि॒तोः॒ । यत् । पा॒त्नी॒व॒तमिति॑ पात्नी - व॒तम् । मि॒नोति॑ । य॒ज्ञ्स्य॑ । प्रतीति॑ । उत्त॑ब्ध्या॒ इत्युत् - स्त॒ब्ध्यै॒ । स॒य॒त्वायेति॑ सय - त्वाय॑ ॥  \newline


\textbf{Krama Paata} \newline

य ए॑काद॒शः । ए॒का॒द॒शः स्तनः॑ । स्तन॑ ए॒व । ए॒वास्यै᳚ । अ॒स्यै॒ सः । स दु॒हे । दु॒ह ए॒व । ए॒वैना᳚म् । ए॒ना॒म् तेन॑ । तेन॒ वज्रः॑ । वज्रो॒ वै । वा ए॒षा । ए॒षा सम् । सम् मी॑यते । मी॒य॒ते॒ यत् । यदे॑काद॒शिनी᳚ । ए॒का॒द॒शिनी॒ सा । सेश्व॒रा । ई॒श्व॒रा पु॒रस्ता᳚त् । पु॒रस्ता᳚त् प्र॒त्यञ्च᳚म् । प्र॒त्यञ्च॑म् ॅय॒ज्ञ्म् । य॒ज्ञ्ꣳ सम्म॑र्दितोः । सम्म॑र्दितो॒र् यत् । सम्म॑र्दितो॒रिति॒ सम् - म॒र्दि॒तोः॒ । यत् पा᳚त्नीव॒तम् । पा॒त्नी॒व॒तम् मि॒नोति॑ । पा॒त्नी॒व॒तमिति॑ पात्नी - व॒तम् । मि॒नोति॑ य॒ज्ञ्स्य॑ । य॒ज्ञ्स्य॒ प्रति॑ । प्रत्युत्त॑ब्द्ध्यै । उत्त॑ब्द्ध्यै सय॒त्वाय॑ । उत्त॑ब्द्ध्या॒ इत्युत् - स्त॒ब्द्ध्यै॒ । स॒य॒त्वायेति॑ सय - त्वाय॑ । \newline

\textbf{Jatai Paata} \newline

1. य ए॑काद॒श ए॑काद॒शो यो य ए॑काद॒शः । \newline
2. ए॒का॒द॒शः स्तनः॒ स्तन॑ एकाद॒श ए॑काद॒शः स्तनः॑ । \newline
3. स्तन॑ ए॒वैव स्तनः॒ स्तन॑ ए॒व । \newline
4. ए॒वास्या॑ अस्या ए॒वै वास्यै᳚ । \newline
5. अ॒स्यै॒ स सो᳚ ऽस्या अस्यै॒ सः । \newline
6. स दु॒हे दु॒हे स स दु॒हे । \newline
7. दु॒ह ए॒वैव दु॒हे दु॒ह ए॒व । \newline
8. ए॒वैना॑ मेना मे॒वै वैना᳚म् । \newline
9. ए॒ना॒म् तेन॒ तेनै॑ना मेना॒म् तेन॑ । \newline
10. तेन॒ वज्रो॒ वज्र॒ स्तेन॒ तेन॒ वज्रः॑ । \newline
11. वज्रो॒ वै वै वज्रो॒ वज्रो॒ वै । \newline
12. वा ए॒षैषा वै वा ए॒षा । \newline
13. ए॒षा सꣳ स मे॒षैषा सम् । \newline
14. सम् मी॑यते मीयते॒ सꣳ सम् मी॑यते । \newline
15. मी॒य॒ते॒ यद् यन् मी॑यते मीयते॒ यत् । \newline
16. यदे॑काद॒शि न्ये॑काद॒शिनी॒ यद् यदे॑काद॒शिनी᳚ । \newline
17. ए॒का॒द॒शिनी॒ सा सैका॑द॒शि न्ये॑काद॒शिनी॒ सा । \newline
18. सेश्व॒ रेश्व॒रा सा सेश्व॒रा । \newline
19. ई॒श्व॒रा पु॒रस्ता᳚त् पु॒रस्ता॑ दीश्व॒ रेश्व॒रा पु॒रस्ता᳚त् । \newline
20. पु॒रस्ता᳚त् प्र॒त्यञ्च॑म् प्र॒त्यञ्च॑म् पु॒रस्ता᳚त् पु॒रस्ता᳚त् प्र॒त्यञ्च᳚म् । \newline
21. प्र॒त्यञ्चं॑ ॅय॒ज्ञ्ं ॅय॒ज्ञ्म् प्र॒त्यञ्च॑म् प्र॒त्यञ्चं॑ ॅय॒ज्ञ्म् । \newline
22. य॒ज्ञ्ꣳ सम्म॑र्दितोः॒ सम्म॑र्दितोर् य॒ज्ञ्ं ॅय॒ज्ञ्ꣳ सम्म॑र्दितोः । \newline
23. सम्म॑र्दितो॒र् यद् यथ् सम्म॑र्दितोः॒ सम्म॑र्दितो॒र् यत् । \newline
24. सम्म॑र्दितो॒रिति॒ सं - म॒र्दि॒तोः॒ । \newline
25. यत् पा᳚त्नीव॒तम् पा᳚त्नीव॒तं ॅयद् यत् पा᳚त्नीव॒तम् । \newline
26. पा॒त्नी॒व॒तम् मि॒नोति॑ मि॒नोति॑ पात्नीव॒तम् पा᳚त्नीव॒तम् मि॒नोति॑ । \newline
27. पा॒त्नी॒व॒तमिति॑ पात्नी - व॒तम् । \newline
28. मि॒नोति॑ य॒ज्ञ्स्य॑ य॒ज्ञ्स्य॑ मि॒नोति॑ मि॒नोति॑ य॒ज्ञ्स्य॑ । \newline
29. य॒ज्ञ्स्य॒ प्रति॒ प्रति॑ य॒ज्ञ्स्य॑ य॒ज्ञ्स्य॒ प्रति॑ । \newline
30. प्रत्युत्त॑ब्ध्या॒ उत्त॑ब्ध्यै॒ प्रति॒ प्रत्युत्त॑ब्ध्यै । \newline
31. उत्त॑ब्ध्यै सय॒त्वाय॑ सय॒त्वा योत्त॑ब्ध्या॒ उत्त॑ब्ध्यै सय॒त्वाय॑ । \newline
32. उत्त॑ब्ध्या॒ इत्युत् - स्त॒ब्ध्यै॒ । \newline
33. स॒य॒त्वायेति॑ सय - त्वाय॑ । \newline

\textbf{Ghana Paata } \newline

1. य ए॑काद॒श ए॑काद॒शो यो य ए॑काद॒शः स्तनः॒ स्तन॑ एकाद॒शो यो य ए॑काद॒शः स्तनः॑ । \newline
2. ए॒का॒द॒शः स्तनः॒ स्तन॑ एकाद॒श ए॑काद॒शः स्तन॑ ए॒वैव स्तन॑ एकाद॒श ए॑काद॒शः स्तन॑ ए॒व । \newline
3. स्तन॑ ए॒वैव स्तनः॒ स्तन॑ ए॒वास्या॑ अस्या ए॒व स्तनः॒ स्तन॑ ए॒वास्यै᳚ । \newline
4. ए॒वास्या॑ अस्या ए॒वै वास्यै॒ स सो᳚ ऽस्या ए॒वै वास्यै॒ सः । \newline
5. अ॒स्यै॒ स सो᳚ ऽस्या अस्यै॒ स दु॒हे दु॒हे सो᳚ ऽस्या अस्यै॒ स दु॒हे । \newline
6. स दु॒हे दु॒हे स स दु॒ह ए॒वैव दु॒हे स स दु॒ह ए॒व । \newline
7. दु॒ह ए॒वैव दु॒हे दु॒ह ए॒वैना॑ मेना मे॒व दु॒हे दु॒ह ए॒वैना᳚म् । \newline
8. ए॒वैना॑ मेना मे॒वै वैना॒म् तेन॒ तेनै॑ना मे॒वै वैना॒म् तेन॑ । \newline
9. ए॒ना॒म् तेन॒ तेनै॑ना मेना॒म् तेन॒ वज्रो॒ वज्र॒ स्तेनै॑ना मेना॒म् तेन॒ वज्रः॑ । \newline
10. तेन॒ वज्रो॒ वज्र॒ स्तेन॒ तेन॒ वज्रो॒ वै वै वज्र॒ स्तेन॒ तेन॒ वज्रो॒ वै । \newline
11. वज्रो॒ वै वै वज्रो॒ वज्रो॒ वा ए॒षैषा वै वज्रो॒ वज्रो॒ वा ए॒षा । \newline
12. वा ए॒षैषा वै वा ए॒षा सꣳ स मे॒षा वै वा ए॒षा सम् । \newline
13. ए॒षा सꣳ स मे॒षैषा सम् मी॑यते मीयते॒ स मे॒षैषा सम् मी॑यते । \newline
14. सम् मी॑यते मीयते॒ सꣳ सम् मी॑यते॒ यद् यन् मी॑यते॒ सꣳ सम् मी॑यते॒ यत् । \newline
15. मी॒य॒ते॒ यद् यन् मी॑यते मीयते॒ यदे॑काद॒शि न्ये॑काद॒शिनी॒ यन् मी॑यते मीयते॒ यदे॑काद॒शिनी᳚ । \newline
16. यदे॑काद॒शि न्ये॑काद॒शिनी॒ यद् यदे॑काद॒शिनी॒ सा सैका॑द॒शिनी॒ यद् यदे॑काद॒शिनी॒ सा । \newline
17. ए॒का॒द॒शिनी॒ सा सैका॑द॒शि न्ये॑काद॒शिनी॒ सेश्व॒ रेश्व॒रा सैका॑द॒शि न्ये॑काद॒शिनी॒ सेश्व॒रा । \newline
18. सेश्व॒ रेश्व॒रा सा सेश्व॒रा पु॒रस्ता᳚त् पु॒रस्ता॑ दीश्व॒रा सा सेश्व॒रा पु॒रस्ता᳚त् । \newline
19. ई॒श्व॒रा पु॒रस्ता᳚त् पु॒रस्ता॑ दीश्व॒ रेश्व॒रा पु॒रस्ता᳚त् प्र॒त्यञ्च॑म् प्र॒त्यञ्च॑म् पु॒रस्ता॑ दीश्व॒ रेश्व॒रा पु॒रस्ता᳚त् प्र॒त्यञ्च᳚म् । \newline
20. पु॒रस्ता᳚त् प्र॒त्यञ्च॑म् प्र॒त्यञ्च॑म् पु॒रस्ता᳚त् पु॒रस्ता᳚त् प्र॒त्यञ्चं॑ ॅय॒ज्ञ्ं ॅय॒ज्ञ्म् प्र॒त्यञ्च॑म् पु॒रस्ता᳚त् पु॒रस्ता᳚त् प्र॒त्यञ्चं॑ ॅय॒ज्ञ्म् । \newline
21. प्र॒त्यञ्चं॑ ॅय॒ज्ञ्ं ॅय॒ज्ञ्म् प्र॒त्यञ्च॑म् प्र॒त्यञ्चं॑ ॅय॒ज्ञ्ꣳ सम्म॑र्दितोः॒ सम्म॑र्दितोर् य॒ज्ञ्म् प्र॒त्यञ्च॑म् प्र॒त्यञ्चं॑ ॅय॒ज्ञ्ꣳ सम्म॑र्दितोः । \newline
22. य॒ज्ञ्ꣳ सम्म॑र्दितोः॒ सम्म॑र्दितोर् य॒ज्ञ्ं ॅय॒ज्ञ्ꣳ सम्म॑र्दितो॒र् यद् यथ् सम्म॑र्दितोर् य॒ज्ञ्ं ॅय॒ज्ञ्ꣳ सम्म॑र्दितो॒र् यत् । \newline
23. सम्म॑र्दितो॒र् यद् यथ् सम्म॑र्दितोः॒ सम्म॑र्दितो॒र् यत् पा᳚त्नीव॒तम् पा᳚त्नीव॒तं ॅयथ् सम्म॑र्दितोः॒ सम्म॑र्दितो॒र् यत् पा᳚त्नीव॒तम् । \newline
24. सम्म॑र्दितो॒रिति॒ सं - म॒र्दि॒तोः॒ । \newline
25. यत् पा᳚त्नीव॒तम् पा᳚त्नीव॒तं ॅयद् यत् पा᳚त्नीव॒तम् मि॒नोति॑ मि॒नोति॑ पात्नीव॒तं ॅयद् यत् पा᳚त्नीव॒तम् मि॒नोति॑ । \newline
26. पा॒त्नी॒व॒तम् मि॒नोति॑ मि॒नोति॑ पात्नीव॒तम् पा᳚त्नीव॒तम् मि॒नोति॑ य॒ज्ञ्स्य॑ य॒ज्ञ्स्य॑ मि॒नोति॑ पात्नीव॒तम् पा᳚त्नीव॒तम् मि॒नोति॑ य॒ज्ञ्स्य॑ । \newline
27. पा॒त्नी॒व॒तमिति॑ पात्नी - व॒तम् । \newline
28. मि॒नोति॑ य॒ज्ञ्स्य॑ य॒ज्ञ्स्य॑ मि॒नोति॑ मि॒नोति॑ य॒ज्ञ्स्य॒ प्रति॒ प्रति॑ य॒ज्ञ्स्य॑ मि॒नोति॑ मि॒नोति॑ य॒ज्ञ्स्य॒ प्रति॑ । \newline
29. य॒ज्ञ्स्य॒ प्रति॒ प्रति॑ य॒ज्ञ्स्य॑ य॒ज्ञ्स्य॒ प्रत्युत्त॑ब्ध्या॒ उत्त॑ब्ध्यै॒ प्रति॑ य॒ज्ञ्स्य॑ य॒ज्ञ्स्य॒ 
प्रत्युत्त॑ब्ध्यै । \newline
30. प्रत्युत्त॑ब्ध्या॒ उत्त॑ब्ध्यै॒ प्रति॒ प्रत्युत्त॑ब्ध्यै सय॒त्वाय॑ सय॒त्वा योत्त॑ब्ध्यै॒ प्रति॒ प्रत्युत्त॑ब्ध्यै सय॒त्वाय॑ । \newline
31. उत्त॑ब्ध्यै सय॒त्वाय॑ सय॒त्वा योत्त॑ब्ध्या॒ उत्त॑ब्ध्यै सय॒त्वाय॑ । \newline
32. उत्त॑ब्ध्या॒ इत्युत् - स्त॒ब्ध्यै॒ । \newline
33. स॒य॒त्वायेति॑ सय - त्वाय॑ । \newline
\pagebreak
\markright{ TS 6.6.5.1  \hfill https://www.vedavms.in \hfill}

\section{ TS 6.6.5.1 }

\textbf{TS 6.6.5.1 } \newline
\textbf{Samhita Paata} \newline

प्र॒जापतिः॑ प्र॒जा अ॑सृजत॒ स रि॑रिचा॒नो॑ऽमन्यत॒ स ए॒तामे॑काद॒शिनी॑-मपश्य॒त् तया॒ वै स आयु॑रिन्द्रि॒यं ॅवी॒र्य॑मा॒त्मन्न॑धत्त प्र॒जा इ॑व॒ खलु॒ वा ए॒ष सृ॑जते॒ यो यज॑ते॒ स ए॒तर्.हि॑ रिरिचा॒न इ॑व॒ यदे॒षैका॑द॒शिनी॒ भव॒त्यायु॑रे॒व तये᳚न्द्रि॒यं ॅवी॒र्यं॑ ॅयज॑मान आ॒त्मन् ध॑त्ते॒ प्रैवाऽऽ*ग्ने॒येन॑ वापयति मिथु॒नꣳ सा॑रस्व॒त्या क॑रोति॒ रेतः॑- [  ] \newline

\textbf{Pada Paata} \newline

प्र॒जाप॑ति॒रिति॑ प्र॒जा - प॒तिः॒ । प्र॒जा इति॑ प्र-जाः । अ॒सृ॒ज॒त॒ । सः । रि॒रि॒चा॒नः । अ॒म॒न्य॒त॒ । सः । ए॒ताम् । ए॒का॒द॒शिनी᳚म् । अ॒प॒श्य॒त् । तया᳚ । वै । सः । आयुः॑ । इ॒न्द्रि॒यम् । वी॒र्य᳚म् । आ॒त्मन्न् । अ॒ध॒त्त॒ । प्र॒जा इति॑ प्र - जाः । इ॒व॒ । खलु॑ । वै । ए॒षः । सृ॒ज॒ते॒ । यः । यज॑ते । सः । ए॒तर्.हि॑ । रि॒रि॒चा॒नः । इ॒व॒ । यत् । ए॒षा । ए॒का॒द॒शिनी᳚ । भव॑ति । आयुः॑ । ए॒व । तया᳚ । इ॒न्द्रि॒यम् । वी॒र्य᳚म् । यज॑मानः । आ॒त्मन्न् । ध॒त्ते॒ । प्रेति॑ । ए॒व । आ॒ग्ने॒येन॑ । वा॒प॒य॒ति॒ । मि॒थु॒नम् । सा॒र॒स्व॒त्या । क॒रो॒ति॒ । रेतः॑ ।  \newline


\textbf{Krama Paata} \newline

प्र॒जाप॑तिः प्र॒जाः । प्र॒जाप॑ति॒रिति॑ प्र॒जा - प॒तिः॒ । प्र॒जा अ॑सृजत । प्र॒जा इति॑ प्र - जाः । अ॒सृ॒ज॒त॒ सः । स रि॑रिचा॒नः । रि॒रि॒चा॒नो॑ऽमन्यत । अ॒म॒न्य॒त॒ सः । स ए॒ताम् । ए॒तामे॑काद॒शिनी᳚म् । ए॒का॒द॒शिनी॑मपश्यत् । अ॒प॒श्य॒त् तया᳚ । तया॒ वै । वै सः । स आयुः॑ । आयु॑रिन्द्रि॒यम् । इ॒न्द्रि॒यम् ॅवी॒र्य᳚म् । वी॒र्य॑मा॒त्मन्न् । आ॒त्मन्न॑धत्त । अ॒ध॒त्त॒ प्र॒जाः । प्र॒जा इ॑व । प्र॒जा इति॑ प्र - जाः । इ॒व॒ खलु॑ । खलु॒ वै । वा ए॒षः । ए॒ष सृ॑जते । सृ॒ज॒ते॒ यः । यो यज॑ते । यज॑ते॒ सः । स ए॒तर्.हि॑ । ए॒तर्.हि॑ रिरिचा॒नः । रि॒रि॒चा॒न इ॑व । इ॒व॒ यत् । यदे॒षा । ए॒षैका॑द॒शिनी᳚ । ए॒का॒द॒शिनी॒ भव॑ति । भव॒त्यायुः॑ । आयु॑रे॒व । ए॒व तया᳚ । तये᳚न्द्रि॒यम् । इ॒न्द्रि॒यम् ॅवी॒र्य᳚म् । वी॒र्य॑म् ॅयज॑मानः । यज॑मान आ॒त्मन्न् । आ॒त्मन् ध॑त्ते । ध॒त्ते॒ प्र । प्रैव । ए॒वाग्ने॒येन॑ । आ॒ग्ने॒येन॑ वापयति । वा॒प॒य॒ति॒ मि॒थु॒नम् । मि॒थु॒नꣳ सा॑रस्व॒त्या । सा॒र॒स्व॒त्या क॑रोति । क॒रो॒ति॒ रेतः॑ । रेतः॑ सौ॒म्येन॑ \newline

\textbf{Jatai Paata} \newline

1. प्र॒जाप॑तिः प्र॒जाः प्र॒जाः प्र॒जाप॑तिः प्र॒जाप॑तिः प्र॒जाः । \newline
2. प्र॒जाप॑ति॒रिति॑ प्र॒जा - प॒तिः॒ । \newline
3. प्र॒जा अ॑सृजता सृजत प्र॒जाः प्र॒जा अ॑सृजत । \newline
4. प्र॒जा इति॑ प्र - जाः । \newline
5. अ॒सृ॒ज॒त॒ स सो॑ ऽसृजता सृजत॒ सः । \newline
6. स रि॑रिचा॒नो रि॑रिचा॒नः स स रि॑रिचा॒नः । \newline
7. रि॒रि॒चा॒नो॑ ऽमन्यता मन्यत रिरिचा॒नो रि॑रिचा॒नो॑ ऽमन्यत । \newline
8. अ॒म॒न्य॒त॒ स सो॑ ऽमन्यता मन्यत॒ सः । \newline
9. स ए॒ता मे॒ताꣳ स स ए॒ताम् । \newline
10. ए॒ता मे॑काद॒शिनी॑ मेकाद॒शिनी॑ मे॒ता मे॒ता मे॑काद॒शिनी᳚म् । \newline
11. ए॒का॒द॒शिनी॑ मपश्य दपश्य देकाद॒शिनी॑ मेकाद॒शिनी॑ मपश्यत् । \newline
12. अ॒प॒श्य॒त् तया॒ तया॑ ऽपश्य दपश्य॒त् तया᳚ । \newline
13. तया॒ वै वै तया॒ तया॒ वै । \newline
14. वै स स वै वै सः । \newline
15. स आयु॒ रायुः॒ स स आयुः॑ । \newline
16. आयु॑ रिन्द्रि॒य मि॑न्द्रि॒य मायु॒ रायु॑ रिन्द्रि॒यम् । \newline
17. इ॒न्द्रि॒यं ॅवी॒र्यं॑ ॅवी॒र्य॑ मिन्द्रि॒य मि॑न्द्रि॒यं ॅवी॒र्य᳚म् । \newline
18. वी॒र्य॑ मा॒त्मन् ना॒त्मन्. वी॒र्यं॑ ॅवी॒र्य॑ मा॒त्मन्न् । \newline
19. आ॒त्मन् न॑धत्ता धत्ता॒त्मन् ना॒त्मन् न॑धत्त । \newline
20. अ॒ध॒त्त॒ प्र॒जाः प्र॒जा अ॑धत्ता धत्त प्र॒जाः । \newline
21. प्र॒जा इ॑वेव प्र॒जाः प्र॒जा इ॑व । \newline
22. प्र॒जा इति॑ प्र - जाः । \newline
23. इ॒व॒ खलु॒ खल्वि॑वेव॒ खलु॑ । \newline
24. खलु॒ वै वै खलु॒ खलु॒ वै । \newline
25. वा ए॒ष ए॒ष वै वा ए॒षः । \newline
26. ए॒ष सृ॑जते सृजत ए॒ष ए॒ष सृ॑जते । \newline
27. सृ॒ज॒ते॒ यो यः सृ॑जते सृजते॒ यः । \newline
28. यो यज॑ते॒ यज॑ते॒ यो यो यज॑ते । \newline
29. यज॑ते॒ स स यज॑ते॒ यज॑ते॒ सः । \newline
30. स ए॒तर् ह्ये॒तर्.हि॒ स स ए॒तर्.हि॑ । \newline
31. ए॒तर्.हि॑ रिरिचा॒नो रि॑रिचा॒न ए॒तर् ह्ये॒तर्.हि॑ रिरिचा॒नः । \newline
32. रि॒रि॒चा॒न इ॑वेव रिरिचा॒नो रि॑रिचा॒न इ॑व । \newline
33. इ॒व॒ यद् यदि॑वेव॒ यत् । \newline
34. यदे॒षैषा यद् यदे॒षा । \newline
35. ए॒षैका॑द॒शि न्ये॑काद॒शि न्ये॒षै षैका॑द॒शिनी᳚ । \newline
36. ए॒का॒द॒शिनी॒ भव॑ति॒ भव॑ त्येकाद॒शि न्ये॑काद॒शिनी॒ भव॑ति । \newline
37. भव॒ त्यायु॒ रायु॒र् भव॑ति॒ भव॒ त्यायुः॑ । \newline
38. आयु॑ रे॒वै वायु॒ रायु॑ रे॒व । \newline
39. ए॒व तया॒ तयै॒ वैव तया᳚ । \newline
40. तये᳚न्द्रि॒य मि॑न्द्रि॒यम् तया॒ तये᳚न्द्रि॒यम् । \newline
41. इ॒न्द्रि॒यं ॅवी॒र्यं॑ ॅवी॒र्य॑ मिन्द्रि॒य मि॑न्द्रि॒यं ॅवी॒र्य᳚म् । \newline
42. वी॒र्यं॑ ॅयज॑मानो॒ यज॑मानो वी॒र्यं॑ ॅवी॒र्यं॑ ॅयज॑मानः । \newline
43. यज॑मान आ॒त्मन् ना॒त्मन्. यज॑मानो॒ यज॑मान आ॒त्मन्न् । \newline
44. आ॒त्मन् ध॑त्ते धत्त आ॒त्मन् ना॒त्मन् ध॑त्ते । \newline
45. ध॒त्ते॒ प्र प्र ध॑त्ते धत्ते॒ प्र । \newline
46. प्रैवैव प्र प्रैव । \newline
47. ए॒वा ग्ने॒येना᳚ ग्ने॒ये नै॒वैवा ग्ने॒येन॑ । \newline
48. आ॒ग्ने॒येन॑ वापयति वापय त्याग्ने॒येना᳚ ग्ने॒येन॑ वापयति । \newline
49. वा॒प॒य॒ति॒ मि॒थु॒नम् मि॑थु॒नं ॅवा॑पयति वापयति मिथु॒नम् । \newline
50. मि॒थु॒नꣳ सा॑रस्व॒त्या सा॑रस्व॒त्या मि॑थु॒नम् मि॑थु॒नꣳ सा॑रस्व॒त्या । \newline
51. सा॒र॒स्व॒त्या क॑रोति करोति सारस्व॒त्या सा॑रस्व॒त्या क॑रोति । \newline
52. क॒रो॒ति॒ रेतो॒ रेतः॑ करोति करोति॒ रेतः॑ । \newline
53. रेतः॑ सौ॒म्येन॑ सौ॒म्येन॒ रेतो॒ रेतः॑ सौ॒म्येन॑ । \newline

\textbf{Ghana Paata } \newline

1. प्र॒जाप॑तिः प्र॒जाः प्र॒जाः प्र॒जाप॑तिः प्र॒जाप॑तिः प्र॒जा अ॑सृजता सृजत प्र॒जाः प्र॒जाप॑तिः प्र॒जाप॑तिः प्र॒जा अ॑सृजत । \newline
2. प्र॒जाप॑ति॒रिति॑ प्र॒जा - प॒तिः॒ । \newline
3. प्र॒जा अ॑सृजता सृजत प्र॒जाः प्र॒जा अ॑सृजत॒ स सो॑ ऽसृजत प्र॒जाः प्र॒जा अ॑सृजत॒ सः । \newline
4. प्र॒जा इति॑ प्र - जाः । \newline
5. अ॒सृ॒ज॒त॒ स सो॑ ऽसृजता सृजत॒ स रि॑रिचा॒नो रि॑रिचा॒नः सो॑ ऽसृजता सृजत॒ स रि॑रिचा॒नः । \newline
6. स रि॑रिचा॒नो रि॑रिचा॒नः स स रि॑रिचा॒नो॑ ऽमन्यता मन्यत रिरिचा॒नः स स रि॑रिचा॒नो॑ ऽमन्यत । \newline
7. रि॒रि॒चा॒नो॑ ऽमन्यता मन्यत रिरिचा॒नो रि॑रिचा॒नो॑ ऽमन्यत॒ स सो॑ ऽमन्यत रिरिचा॒नो रि॑रिचा॒नो॑ ऽमन्यत॒ सः । \newline
8. अ॒म॒न्य॒त॒ स सो॑ ऽमन्यता मन्यत॒ स ए॒ता मे॒ताꣳ सो॑ ऽमन्यता मन्यत॒ स ए॒ताम् । \newline
9. स ए॒ता मे॒ताꣳ स स ए॒ता मे॑काद॒शिनी॑ मेकाद॒शिनी॑ मे॒ताꣳ स स ए॒ता मे॑काद॒शिनी᳚म् । \newline
10. ए॒ता मे॑काद॒शिनी॑ मेकाद॒शिनी॑ मे॒ता मे॒ता मे॑काद॒शिनी॑ मपश्य दपश्य देकाद॒शिनी॑ मे॒ता मे॒ता मे॑काद॒शिनी॑ मपश्यत् । \newline
11. ए॒का॒द॒शिनी॑ मपश्य दपश्य देकाद॒शिनी॑ मेकाद॒शिनी॑ मपश्य॒त् तया॒ तया॑ ऽपश्य देकाद॒शिनी॑ मेकाद॒शिनी॑ मपश्य॒त् तया᳚ । \newline
12. अ॒प॒श्य॒त् तया॒ तया॑ ऽपश्य दपश्य॒त् तया॒ वै वै तया॑ ऽपश्य दपश्य॒त् तया॒ वै । \newline
13. तया॒ वै वै तया॒ तया॒ वै स स वै तया॒ तया॒ वै सः । \newline
14. वै स स वै वै स आयु॒ रायुः॒ स वै वै स आयुः॑ । \newline
15. स आयु॒ रायुः॒ स स आयु॑ रिन्द्रि॒य मि॑न्द्रि॒य मायुः॒ स स आयु॑ रिन्द्रि॒यम् । \newline
16. आयु॑ रिन्द्रि॒य मि॑न्द्रि॒य मायु॒ रायु॑ रिन्द्रि॒यं ॅवी॒र्यं॑ ॅवी॒र्य॑ मिन्द्रि॒य मायु॒ रायु॑ रिन्द्रि॒यं ॅवी॒र्य᳚म् । \newline
17. इ॒न्द्रि॒यं ॅवी॒र्यं॑ ॅवी॒र्य॑ मिन्द्रि॒य मि॑न्द्रि॒यं ॅवी॒र्य॑ मा॒त्मन् ना॒त्मन्. वी॒र्य॑ मिन्द्रि॒य मि॑न्द्रि॒यं ॅवी॒र्य॑ मा॒त्मन्न् । \newline
18. वी॒र्य॑ मा॒त्मन् ना॒त्मन्. वी॒र्यं॑ ॅवी॒र्य॑ मा॒त्मन् न॑धत्ता धत्ता॒त्मन्. वी॒र्यं॑ ॅवी॒र्य॑ मा॒त्मन् न॑धत्त । \newline
19. आ॒त्मन् न॑धत्ता धत्ता॒त्मन् ना॒त्मन् न॑धत्त प्र॒जाः प्र॒जा अ॑धत्ता॒त्मन् ना॒त्मन् न॑धत्त प्र॒जाः । \newline
20. अ॒ध॒त्त॒ प्र॒जाः प्र॒जा अ॑धत्ता धत्त प्र॒जा इ॑वेव प्र॒जा अ॑धत्ता धत्त प्र॒जा इ॑व । \newline
21. प्र॒जा इ॑वेव प्र॒जाः प्र॒जा इ॑व॒ खलु॒ खल्वि॑व प्र॒जाः प्र॒जा इ॑व॒ खलु॑ । \newline
22. प्र॒जा इति॑ प्र - जाः । \newline
23. इ॒व॒ खलु॒ खल्वि॑वेव॒ खलु॒ वै वै खल्वि॑वेव॒ खलु॒ वै । \newline
24. खलु॒ वै वै खलु॒ खलु॒ वा ए॒ष ए॒ष वै खलु॒ खलु॒ वा ए॒षः । \newline
25. वा ए॒ष ए॒ष वै वा ए॒ष सृ॑जते सृजत ए॒ष वै वा ए॒ष सृ॑जते । \newline
26. ए॒ष सृ॑जते सृजत ए॒ष ए॒ष सृ॑जते॒ यो यः सृ॑जत ए॒ष ए॒ष सृ॑जते॒ यः । \newline
27. सृ॒ज॒ते॒ यो यः सृ॑जते सृजते॒ यो यज॑ते॒ यज॑ते॒ यः सृ॑जते सृजते॒ यो यज॑ते । \newline
28. यो यज॑ते॒ यज॑ते॒ यो यो यज॑ते॒ स स यज॑ते॒ यो यो यज॑ते॒ सः । \newline
29. यज॑ते॒ स स यज॑ते॒ यज॑ते॒ स ए॒तर् ह्ये॒तर्.हि॒ स यज॑ते॒ यज॑ते॒ स ए॒तर्.हि॑ । \newline
30. स ए॒तर् ह्ये॒तर्.हि॒ स स ए॒तर्.हि॑ रिरिचा॒नो रि॑रिचा॒न ए॒तर्.हि॒ स स ए॒तर्.हि॑ रिरिचा॒नः । \newline
31. ए॒तर्.हि॑ रिरिचा॒नो रि॑रिचा॒न ए॒तर् ह्ये॒तर्.हि॑ रिरिचा॒न इ॑वेव रिरिचा॒न ए॒तर् ह्ये॒तर्.हि॑ रिरिचा॒न इ॑व । \newline
32. रि॒रि॒चा॒न इ॑वेव रिरिचा॒नो रि॑रिचा॒न इ॑व॒ यद् यदि॑व रिरिचा॒नो रि॑रिचा॒न इ॑व॒ यत् । \newline
33. इ॒व॒ यद् यदि॑वेव॒ यदे॒षैषा यदि॑वेव॒ यदे॒षा । \newline
34. यदे॒षैषा यद् यदे॒षैका॑द॒शि न्ये॑काद॒शि न्ये॒षा यद् यदे॒षैका॑द॒शिनी᳚ । \newline
35. ए॒षैका॑द॒शि न्ये॑काद॒शि न्ये॒षै षैका॑द॒शिनी॒ भव॑ति॒ भव॑ त्येकाद॒शिन्ये॒षै षैका॑द॒शिनी॒ भव॑ति । \newline
36. ए॒का॒द॒शिनी॒ भव॑ति॒ भव॑ त्येकाद॒शि न्ये॑काद॒शिनी॒ भव॒ त्यायु॒ रायु॒र् भव॑ त्येकाद॒शि न्ये॑काद॒शिनी॒ भव॒ त्यायुः॑ । \newline
37. भव॒ त्यायु॒ रायु॒र् भव॑ति॒ भव॒ त्यायु॑ रे॒वैवायु॒र् भव॑ति॒ भव॒ त्यायु॑ रे॒व । \newline
38. आयु॑ रे॒वैवायु॒ रायु॑ रे॒व तया॒ तयै॒वायु॒ रायु॑ रे॒व तया᳚ । \newline
39. ए॒व तया॒ तयै॒वैव तये᳚न्द्रि॒य मि॑न्द्रि॒यम् तयै॒वैव तये᳚न्द्रि॒यम् । \newline
40. तये᳚न्द्रि॒य मि॑न्द्रि॒यम् तया॒ तये᳚न्द्रि॒यं ॅवी॒र्यं॑ ॅवी॒र्य॑ मिन्द्रि॒यम् तया॒ तये᳚न्द्रि॒यं ॅवी॒र्य᳚म् । \newline
41. इ॒न्द्रि॒यं ॅवी॒र्यं॑ ॅवी॒र्य॑ मिन्द्रि॒य मि॑न्द्रि॒यं ॅवी॒र्यं॑ ॅयज॑मानो॒ यज॑मानो वी॒र्य॑ मिन्द्रि॒य मि॑न्द्रि॒यं ॅवी॒र्यं॑ ॅयज॑मानः । \newline
42. वी॒र्यं॑ ॅयज॑मानो॒ यज॑मानो वी॒र्यं॑ ॅवी॒र्यं॑ ॅयज॑मान आ॒त्मन् ना॒त्मन्. यज॑मानो वी॒र्यं॑ ॅवी॒र्यं॑ ॅयज॑मान आ॒त्मन्न् । \newline
43. यज॑मान आ॒त्मन् ना॒त्मन्. यज॑मानो॒ यज॑मान आ॒त्मन् ध॑त्ते धत्त आ॒त्मन्. यज॑मानो॒ यज॑मान आ॒त्मन् ध॑त्ते । \newline
44. आ॒त्मन् ध॑त्ते धत्त आ॒त्मन् ना॒त्मन् ध॑त्ते॒ प्र प्र ध॑त्त आ॒त्मन् ना॒त्मन् ध॑त्ते॒ प्र । \newline
45. ध॒त्ते॒ प्र प्र ध॑त्ते धत्ते॒ प्रैवैव प्र ध॑त्ते धत्ते॒ प्रैव । \newline
46. प्रैवैव प्र प्रै वाग्ने॒ येना᳚ग्ने॒ये नै॒व प्र प्रै वाग्ने॒येन॑ । \newline
47. ए॒वाग्ने॒ येना᳚ग्ने॒येनै॒वै वाग्ने॒येन॑ वापयति वापय त्याग्ने॒येनै॒वै वाग्ने॒येन॑ वापयति । \newline
48. आ॒ग्ने॒येन॑ वापयति वापय त्याग्ने॒येना᳚ ग्ने॒येन॑ वापयति मिथु॒नम् मि॑थु॒नं ॅवा॑पय त्याग्ने॒येना᳚ ग्ने॒येन॑ वापयति मिथु॒नम् । \newline
49. वा॒प॒य॒ति॒ मि॒थु॒नम् मि॑थु॒नं ॅवा॑पयति वापयति मिथु॒नꣳ सा॑रस्व॒त्या सा॑रस्व॒त्या मि॑थु॒नं ॅवा॑पयति वापयति मिथु॒नꣳ सा॑रस्व॒त्या । \newline
50. मि॒थु॒नꣳ सा॑रस्व॒त्या सा॑रस्व॒त्या मि॑थु॒नम् मि॑थु॒नꣳ सा॑रस्व॒त्या क॑रोति करोति सारस्व॒त्या मि॑थु॒नम् मि॑थु॒नꣳ सा॑रस्व॒त्या क॑रोति । \newline
51. सा॒र॒स्व॒त्या क॑रोति करोति सारस्व॒त्या सा॑रस्व॒त्या क॑रोति॒ रेतो॒ रेतः॑ करोति सारस्व॒त्या सा॑रस्व॒त्या क॑रोति॒ रेतः॑ । \newline
52. क॒रो॒ति॒ रेतो॒ रेतः॑ करोति करोति॒ रेतः॑ सौ॒म्येन॑ सौ॒म्येन॒ रेतः॑ करोति करोति॒ रेतः॑ सौ॒म्येन॑ । \newline
53. रेतः॑ सौ॒म्येन॑ सौ॒म्येन॒ रेतो॒ रेतः॑ सौ॒म्येन॑ दधाति दधाति सौ॒म्येन॒ रेतो॒ रेतः॑ सौ॒म्येन॑ दधाति । \newline
\pagebreak
\markright{ TS 6.6.5.2  \hfill https://www.vedavms.in \hfill}

\section{ TS 6.6.5.2 }

\textbf{TS 6.6.5.2 } \newline
\textbf{Samhita Paata} \newline

सौ॒म्येन॑ दधाति॒ प्र ज॑नयति पौ॒ष्णेन॑ बार्.हस्प॒त्यो भ॑वति॒ ब्रह्म॒ वै दे॒वानां॒ बृह॒स्पति॒र्ब्रह्म॑णै॒वास्मै᳚ प्र॒जाः प्रज॑नयति वैश्वदे॒वो भ॑वति वैश्वदे॒व्यो॑ वै प्र॒जाः प्र॒जा ए॒वास्मै॒ प्रज॑नयती-न्द्रि॒यमे॒वैन्द्रेणाव॑ रुन्धे॒ विशं॑ मारु॒तेनौजो॒ बल॑मैन्द्रा॒ग्नेन॑ प्रस॒वाय॑ सावि॒त्रो नि॑र्वरुण॒त्वाय॑ वारु॒णो म॑द्ध्य॒त ऐ॒न्द्रमा ल॑भते मद्ध्य॒त ए॒वेन्द्रि॒यं ॅयज॑माने दधाति- [  ] \newline

\textbf{Pada Paata} \newline

सौ॒म्येन॑ । द॒धा॒ति॒ । प्रेति॑ । ज॒न॒य॒ति॒ । पौ॒ष्णेन॑ । बा॒र्॒.ह॒स्प॒त्यः । भ॒व॒ति॒ । ब्रह्म॑ । वै । दे॒वाना᳚म् । बृह॒स्पतिः॑ । ब्रह्म॑णा । ए॒व । अ॒स्मै॒ । प्र॒जा इति॑ प्र-जाः । प्रेति॑ । ज॒न॒य॒ति॒ । वै॒श्व॒दे॒व इति॑ वैश्व - दे॒वः । भ॒व॒ति॒ । वै॒श्व॒दे॒व्य॑ इति॑ वैश्व - दे॒व्यः॑ । वै । प्र॒जा इति॑ प्र - जाः । प्र॒जा इति॑ प्र - जाः । ए॒व । अ॒स्मै॒ । प्रेति॑ । ज॒न॒य॒ति॒ । इ॒न्द्रि॒यम् । ए॒व । ऐ॒न्द्रेण॑ । अवेति॑ । रु॒न्धे॒ । विश᳚म् । मा॒रु॒तेन॑ । ओजः॑ । बल᳚म् । ऐ॒न्द्रा॒ग्नेनेत्यै᳚न्द्र - अ॒ग्नेन॑ । प्र॒स॒वायेति॑ प्र - स॒वाय॑ । सा॒वि॒त्रः । नि॒र्व॒रु॒ण॒त्वायेति॑ निर्वरुण - त्वाय॑ । वा॒रु॒णः । म॒द्ध्य॒तः । ऐ॒न्द्रम् । एति॑ । ल॒भ॒ते॒ । म॒द्ध्य॒तः । ए॒व । इ॒न्द्रि॒यम् । यज॑माने । द॒धा॒ति॒ ।  \newline


\textbf{Krama Paata} \newline

सौ॒म्येन॑ दधाति । द॒धा॒ति॒ प्र । प्र ज॑नयति । ज॒न॒य॒ति॒ पौ॒ष्णेन॑ । पौ॒ष्णेन॑ बार्.हस्प॒त्यः । बा॒र्.॒ह॒स्प॒त्यो भ॑वति । भ॒व॒ति॒ ब्रह्म॑ । ब्रह्म॒ वै । वै दे॒वाना᳚म् । दे॒वाना॒म् बृह॒स्पतिः॑ । बृह॒स्पति॒र् ब्रह्म॑णा । ब्रह्म॑णै॒व । ए॒वास्मै᳚ । अ॒स्मै॒ प्र॒जाः । प्र॒जाः प्र । प्र॒जा इति॑ प्र - जाः । प्र ज॑नयति । ज॒न॒य॒ति॒ वै॒श्व॒दे॒वः । वै॒श्व॒दे॒वो भ॑वति । वै॒श्व॒दे॒व इति॑ वैश्व - दे॒वः । भ॒व॒ति॒ वै॒श्व॒दे॒व्यः॑ । वै॒श्व॒दे॒व्यो॑ वै । वै॒श्व॒दे॒व्य॑ इति॑ वैश्व - दे॒व्यः॑ । वै प्र॒जाः । प्र॒जाः प्र॒जाः । प्र॒जा इति॑ प्र - जाः । प्र॒जा ए॒व । प्र॒जा इति॑ प्र - जाः । ए॒वास्मै᳚ । अ॒स्मै॒ प्र । प्र ज॑नयति । ज॒न॒य॒ती॒न्द्रि॒यम् । इ॒न्द्रि॒यमे॒व । ए॒वैन्द्रेण॑ । ऐ॒न्द्रेणाव॑ । अव॑ रुन्धे । रु॒न्धे॒ विश᳚म् । विश॑म् मारु॒तेन॑ । मा॒रु॒तेनौजः॑ । ओजो॒ बल᳚म् । बल॑मैन्द्रा॒ग्नेन॑ । ऐ॒न्द्रा॒ग्नेन॑ प्रस॒वाय॑ । ऐ॒न्द्रा॒ग्नेनेत्यै᳚न्द्र - अ॒ग्नेन॑ । प्र॒स॒वाय॑ सावि॒त्रः । प्र॒स॒वायेति॑ प्र - स॒वाय॑ । सा॒वि॒त्रो नि॑र्वरुण॒त्वाय॑ । नि॒र्व॒रु॒ण॒त्वाय॑ वारु॒णः । नि॒र्व॒रु॒ण॒त्वायेति॑ निर्वरुण - त्वाय॑ । वा॒रु॒णो म॑द्ध्य॒तः । म॒द्ध्य॒त ऐ॒न्द्रम् । ऐ॒न्द्रमा । आ ल॑भते । ल॒भ॒ते॒ म॒द्ध्य॒तः । म॒द्ध्य॒त ए॒व । ए॒वेन्द्रि॒यम् । इ॒न्द्रि॒यम् ॅयज॑माने । यज॑माने दधाति । द॒धा॒ति॒ पु॒रस्ता᳚त् \newline

\textbf{Jatai Paata} \newline

1. सौ॒म्येन॑ दधाति दधाति सौ॒म्येन॑ सौ॒म्येन॑ दधाति । \newline
2. द॒धा॒ति॒ प्र प्र द॑धाति दधाति॒ प्र । \newline
3. प्र ज॑नयति जनयति॒ प्र प्र ज॑नयति । \newline
4. ज॒न॒य॒ति॒ पौ॒ष्णेन॑ पौ॒ष्णेन॑ जनयति जनयति पौ॒ष्णेन॑ । \newline
5. पौ॒ष्णेन॑ बार्.हस्प॒त्यो बा॑र्.हस्प॒त्यः पौ॒ष्णेन॑ पौ॒ष्णेन॑ बार्.हस्प॒त्यः । \newline
6. बा॒र्॒.ह॒स्प॒त्यो भ॑वति भवति बार्.हस्प॒त्यो बा॑र्.हस्प॒त्यो भ॑वति । \newline
7. भ॒व॒ति॒ ब्रह्म॒ ब्रह्म॑ भवति भवति॒ ब्रह्म॑ । \newline
8. ब्रह्म॒ वै वै ब्रह्म॒ ब्रह्म॒ वै । \newline
9. वै दे॒वाना᳚म् दे॒वानां॒ ॅवै वै दे॒वाना᳚म् । \newline
10. दे॒वाना॒म् बृह॒स्पति॒र् बृह॒स्पति॑र् दे॒वाना᳚म् दे॒वाना॒म् बृह॒स्पतिः॑ । \newline
11. बृह॒स्पति॒र् ब्रह्म॑णा॒ ब्रह्म॑णा॒ बृह॒स्पति॒र् बृह॒स्पति॒र् ब्रह्म॑णा । \newline
12. ब्रह्म॑णै॒वैव ब्रह्म॑णा॒ ब्रह्म॑णै॒व । \newline
13. ए॒वास्मा॑ अस्मा ए॒वै वास्मै᳚ । \newline
14. अ॒स्मै॒ प्र॒जाः प्र॒जा अ॑स्मा अस्मै प्र॒जाः । \newline
15. प्र॒जाः प्र प्र प्र॒जाः प्र॒जाः प्र । \newline
16. प्र॒जा इति॑ प्र - जाः । \newline
17. प्र ज॑नयति जनयति॒ प्र प्र ज॑नयति । \newline
18. ज॒न॒य॒ति॒ वै॒श्व॒दे॒वो वै᳚श्वदे॒वो ज॑नयति जनयति वैश्वदे॒वः । \newline
19. वै॒श्व॒दे॒वो भ॑वति भवति वैश्वदे॒वो वै᳚श्वदे॒वो भ॑वति । \newline
20. वै॒श्व॒दे॒व इति॑ वैश्व - दे॒वः । \newline
21. भ॒व॒ति॒ वै॒श्व॒दे॒व्यो॑ वैश्वदे॒व्यो॑ भवति भवति वैश्वदे॒व्यः॑ । \newline
22. वै॒श्व॒दे॒व्यो॑ वै वै वै᳚श्वदे॒व्यो॑ वैश्वदे॒व्यो॑ वै । \newline
23. वै॒श्व॒दे॒व्य॑ इति॑ वैश्व - दे॒व्यः॑ । \newline
24. वै प्र॒जाः प्र॒जा वै वै प्र॒जाः । \newline
25. प्र॒जाः प्र॒जाः । \newline
26. प्र॒जा इति॑ प्र - जाः । \newline
27. प्र॒जा ए॒वैव प्र॒जाः प्र॒जा ए॒व । \newline
28. प्र॒जा इति॑ प्र - जाः । \newline
29. ए॒वास्मा॑ अस्मा ए॒वै वास्मै᳚ । \newline
30. अ॒स्मै॒ प्र प्रास्मा॑ अस्मै॒ प्र । \newline
31. प्र ज॑नयति जनयति॒ प्र प्र ज॑नयति । \newline
32. ज॒न॒य॒ ती॒न्द्रि॒य मि॑न्द्रि॒यम् ज॑नयति जनय तीन्द्रि॒यम् । \newline
33. इ॒न्द्रि॒य मे॒वै वेन्द्रि॒य मि॑न्द्रि॒य मे॒व । \newline
34. ए॒वैन्द्रे णै॒न्द्रे णै॒वै वैन्द्रेण॑ । \newline
35. ऐ॒न्द्रेणावा वै॒न्द्रे णै॒न्द्रेणाव॑ । \newline
36. अव॑ रुन्धे रु॒न्धे ऽवाव॑ रुन्धे । \newline
37. रु॒न्धे॒ विशं॒ ॅविशꣳ॑ रुन्धे रुन्धे॒ विश᳚म् । \newline
38. विश॑म् मारु॒तेन॑ मारु॒तेन॒ विशं॒ ॅविश॑म् मारु॒तेन॑ । \newline
39. मा॒रु॒ते नौज॒ ओजो॑ मारु॒तेन॑ मारु॒ते नौजः॑ । \newline
40. ओजो॒ बल॒म् बल॒ मोज॒ ओजो॒ बल᳚म् । \newline
41. बल॑ मैन्द्रा॒ग्ने नै᳚न्द्रा॒ग्नेन॒ बल॒म् बल॑ मैन्द्रा॒ग्नेन॑ । \newline
42. ऐ॒न्द्रा॒ग्नेन॑ प्रस॒वाय॑ प्रस॒वा यै᳚न्द्रा॒ग्ने नै᳚न्द्रा॒ग्नेन॑ प्रस॒वाय॑ । \newline
43. ऐ॒न्द्रा॒ग्नेनेत्यै᳚न्द्र - अ॒ग्नेन॑ । \newline
44. प्र॒स॒वाय॑ सावि॒त्रः सा॑वि॒त्रः प्र॑स॒वाय॑ प्रस॒वाय॑ सावि॒त्रः । \newline
45. प्र॒स॒वायेति॑ प्र - स॒वाय॑ । \newline
46. सा॒वि॒त्रो नि॑र्वरुण॒त्वाय॑ निर्वरुण॒त्वाय॑ सावि॒त्रः सा॑वि॒त्रो नि॑र्वरुण॒त्वाय॑ । \newline
47. नि॒र्व॒रु॒ण॒त्वाय॑ वारु॒णो वा॑रु॒णो नि॑र्वरुण॒त्वाय॑ निर्वरुण॒त्वाय॑ वारु॒णः । \newline
48. नि॒र्व॒रु॒ण॒त्वायेति॑ निर्वरुण - त्वाय॑ । \newline
49. वा॒रु॒णो म॑द्ध्य॒तो म॑द्ध्य॒तो वा॑रु॒णो वा॑रु॒णो म॑द्ध्य॒तः । \newline
50. म॒द्ध्य॒त ऐ॒न्द्र मै॒न्द्रम् म॑द्ध्य॒तो म॑द्ध्य॒त ऐ॒न्द्रम् । \newline
51. ऐ॒न्द्र मैन्द्र मै॒न्द्र मा । \newline
52. आ ल॑भते लभत॒ आ ल॑भते । \newline
53. ल॒भ॒ते॒ म॒द्ध्य॒तो म॑द्ध्य॒तो ल॑भते लभते मद्ध्य॒तः । \newline
54. म॒द्ध्य॒त ए॒वैव म॑द्ध्य॒तो म॑द्ध्य॒त ए॒व । \newline
55. ए॒वेन्द्रि॒य मि॑न्द्रि॒य मे॒वैवेन्द्रि॒यम् । \newline
56. इ॒न्द्रि॒यं ॅयज॑माने॒ यज॑मान इन्द्रि॒य मि॑न्द्रि॒यं ॅयज॑माने । \newline
57. यज॑माने दधाति दधाति॒ यज॑माने॒ यज॑माने दधाति । \newline
58. द॒धा॒ति॒ पु॒रस्ता᳚त् पु॒रस्ता᳚द् दधाति दधाति पु॒रस्ता᳚त् । \newline

\textbf{Ghana Paata } \newline

1. सौ॒म्येन॑ दधाति दधाति सौ॒म्येन॑ सौ॒म्येन॑ दधाति॒ प्र प्र द॑धाति सौ॒म्येन॑ सौ॒म्येन॑ दधाति॒ प्र । \newline
2. द॒धा॒ति॒ प्र प्र द॑धाति दधाति॒ प्र ज॑नयति जनयति॒ प्र द॑धाति दधाति॒ प्र ज॑नयति । \newline
3. प्र ज॑नयति जनयति॒ प्र प्र ज॑नयति पौ॒ष्णेन॑ पौ॒ष्णेन॑ जनयति॒ प्र प्र ज॑नयति पौ॒ष्णेन॑ । \newline
4. ज॒न॒य॒ति॒ पौ॒ष्णेन॑ पौ॒ष्णेन॑ जनयति जनयति पौ॒ष्णेन॑ बार्.हस्प॒त्यो बा॑र्.हस्प॒त्यः पौ॒ष्णेन॑ जनयति जनयति पौ॒ष्णेन॑ बार्.हस्प॒त्यः । \newline
5. पौ॒ष्णेन॑ बार्.हस्प॒त्यो बा॑र्.हस्प॒त्यः पौ॒ष्णेन॑ पौ॒ष्णेन॑ बार्.हस्प॒त्यो भ॑वति भवति बार्.हस्प॒त्यः पौ॒ष्णेन॑ पौ॒ष्णेन॑ बार्.हस्प॒त्यो भ॑वति । \newline
6. बा॒र्॒.ह॒स्प॒त्यो भ॑वति भवति बार्.हस्प॒त्यो बा॑र्.हस्प॒त्यो भ॑वति॒ ब्रह्म॒ ब्रह्म॑ भवति बार्.हस्प॒त्यो बा॑र्.हस्प॒त्यो भ॑वति॒ ब्रह्म॑ । \newline
7. भ॒व॒ति॒ ब्रह्म॒ ब्रह्म॑ भवति भवति॒ ब्रह्म॒ वै वै ब्रह्म॑ भवति भवति॒ ब्रह्म॒ वै । \newline
8. ब्रह्म॒ वै वै ब्रह्म॒ ब्रह्म॒ वै दे॒वाना᳚म् दे॒वानां॒ ॅवै ब्रह्म॒ ब्रह्म॒ वै दे॒वाना᳚म् । \newline
9. वै दे॒वाना᳚म् दे॒वानां॒ ॅवै वै दे॒वाना॒म् बृह॒स्पति॒र् बृह॒स्पति॑र् दे॒वानां॒ ॅवै वै दे॒वाना॒म् बृह॒स्पतिः॑ । \newline
10. दे॒वाना॒म् बृह॒स्पति॒र् बृह॒स्पति॑र् दे॒वाना᳚म् दे॒वाना॒म् बृह॒स्पति॒र् ब्रह्म॑णा॒ ब्रह्म॑णा॒ बृह॒स्पति॑र् दे॒वाना᳚म् दे॒वाना॒म् बृह॒स्पति॒र् ब्रह्म॑णा । \newline
11. बृह॒स्पति॒र् ब्रह्म॑णा॒ ब्रह्म॑णा॒ बृह॒स्पति॒र् बृह॒स्पति॒र् ब्रह्म॑णै॒वैव ब्रह्म॑णा॒ बृह॒स्पति॒र् बृह॒स्पति॒र् ब्रह्म॑णै॒व । \newline
12. ब्रह्म॑णै॒वैव ब्रह्म॑णा॒ ब्रह्म॑णै॒वास्मा॑ अस्मा ए॒व ब्रह्म॑णा॒ ब्रह्म॑णै॒वास्मै᳚ । \newline
13. ए॒वास्मा॑ अस्मा ए॒वै वास्मै᳚ प्र॒जाः प्र॒जा अ॑स्मा ए॒वै वास्मै᳚ प्र॒जाः । \newline
14. अ॒स्मै॒ प्र॒जाः प्र॒जा अ॑स्मा अस्मै प्र॒जाः प्र प्र प्र॒जा अ॑स्मा अस्मै प्र॒जाः प्र । \newline
15. प्र॒जाः प्र प्र प्र॒जाः प्र॒जाः प्र ज॑नयति जनयति॒ प्र प्र॒जाः प्र॒जाः प्र ज॑नयति । \newline
16. प्र॒जा इति॑ प्र - जाः । \newline
17. प्र ज॑नयति जनयति॒ प्र प्र ज॑नयति वैश्वदे॒वो वै᳚श्वदे॒वो ज॑नयति॒ प्र प्र ज॑नयति वैश्वदे॒वः । \newline
18. ज॒न॒य॒ति॒ वै॒श्व॒दे॒वो वै᳚श्वदे॒वो ज॑नयति जनयति वैश्वदे॒वो भ॑वति भवति वैश्वदे॒वो ज॑नयति जनयति वैश्वदे॒वो भ॑वति । \newline
19. वै॒श्व॒दे॒वो भ॑वति भवति वैश्वदे॒वो वै᳚श्वदे॒वो भ॑वति वैश्वदे॒व्यो॑ वैश्वदे॒व्यो॑ भवति वैश्वदे॒वो वै᳚श्वदे॒वो भ॑वति वैश्वदे॒व्यः॑ । \newline
20. वै॒श्व॒दे॒व इति॑ वैश्व - दे॒वः । \newline
21. भ॒व॒ति॒ वै॒श्व॒दे॒व्यो॑ वैश्वदे॒व्यो॑ भवति भवति वैश्वदे॒व्यो॑ वै वै वै᳚श्वदे॒व्यो॑ भवति भवति वैश्वदे॒व्यो॑ वै । \newline
22. वै॒श्व॒दे॒व्यो॑ वै वै वै᳚श्वदे॒व्यो॑ वैश्वदे॒व्यो॑ वै प्र॒जाः प्र॒जा वै वै᳚श्वदे॒व्यो॑ वैश्वदे॒व्यो॑ वै प्र॒जाः । \newline
23. वै॒श्व॒दे॒व्य॑ इति॑ वैश्व - दे॒व्यः॑ । \newline
24. वै प्र॒जाः प्र॒जा वै वै प्र॒जाः । \newline
25. प्र॒जाः प्र॒जाः । \newline
26. प्र॒जा इति॑ प्र - जाः । \newline
27. प्र॒जा ए॒वैव प्र॒जाः प्र॒जा ए॒वास्मा॑ अस्मा ए॒व प्र॒जाः प्र॒जा ए॒वास्मै᳚ । \newline
28. प्र॒जा इति॑ प्र - जाः । \newline
29. ए॒वास्मा॑ अस्मा ए॒वैवास्मै॒ प्र प्रास्मा॑ ए॒वैवास्मै॒ प्र । \newline
30. अ॒स्मै॒ प्र प्रास्मा॑ अस्मै॒ प्र ज॑नयति जनयति॒ प्रास्मा॑ अस्मै॒ प्र ज॑नयति । \newline
31. प्र ज॑नयति जनयति॒ प्र प्र ज॑नय तीन्द्रि॒य मि॑न्द्रि॒यम् ज॑नयति॒ प्र प्र ज॑नय तीन्द्रि॒यम् । \newline
32. ज॒न॒य॒ ती॒न्द्रि॒य मि॑न्द्रि॒यम् ज॑नयति जनय तीन्द्रि॒य मे॒वैवेन्द्रि॒यम् ज॑नयति जनय तीन्द्रि॒य मे॒व । \newline
33. इ॒न्द्रि॒य मे॒वैवेन्द्रि॒य मि॑न्द्रि॒य मे॒वैन्द्रे णै॒न्द्रे णै॒वेन्द्रि॒य मि॑न्द्रि॒य मे॒वैन्द्रेण॑ । \newline
34. ए॒वैन्द्रे णै॒न्द्रे णै॒वैवैन्द्रेणा वावै॒न्द्रे णै॒वैवैन्द्रेणाव॑ । \newline
35. ऐ॒न्द्रेणा वावै॒न्द्रे णै॒न्द्रेणाव॑ रुन्धे रु॒न्धे ऽवै॒न्द्रे णै॒न्द्रेणाव॑ रुन्धे । \newline
36. अव॑ रुन्धे रु॒न्धे ऽवाव॑ रुन्धे॒ विशं॒ ॅविशꣳ॑ रु॒न्धे ऽवाव॑ रुन्धे॒ विश᳚म् । \newline
37. रु॒न्धे॒ विशं॒ ॅविशꣳ॑ रुन्धे रुन्धे॒ विश॑म् मारु॒तेन॑ मारु॒तेन॒ विशꣳ॑ रुन्धे रुन्धे॒ विश॑म् मारु॒तेन॑ । \newline
38. विश॑म् मारु॒तेन॑ मारु॒तेन॒ विशं॒ ॅविश॑म् मारु॒ते नौज॒ ओजो॑ मारु॒तेन॒ विशं॒ ॅविश॑म् मारु॒ते नौजः॑ । \newline
39. मा॒रु॒ते नौज॒ ओजो॑ मारु॒तेन॑ मारु॒ते नौजो॒ बल॒म् बल॒ मोजो॑ मारु॒तेन॑ मारु॒ते नौजो॒ बल᳚म् । \newline
40. ओजो॒ बल॒म् बल॒ मोज॒ ओजो॒ बल॑ मैन्द्रा॒ग्ने नै᳚न्द्रा॒ग्नेन॒ बल॒ मोज॒ ओजो॒ बल॑ मैन्द्रा॒ग्नेन॑ । \newline
41. बल॑ मैन्द्रा॒ग्ने नै᳚न्द्रा॒ग्नेन॒ बल॒म् बल॑ मैन्द्रा॒ग्नेन॑ प्रस॒वाय॑ प्रस॒वा यै᳚न्द्रा॒ग्नेन॒ बल॒म् बल॑ मैन्द्रा॒ग्नेन॑ प्रस॒वाय॑ । \newline
42. ऐ॒न्द्रा॒ग्नेन॑ प्रस॒वाय॑ प्रस॒वा यै᳚न्द्रा॒ग्ने नै᳚न्द्रा॒ग्नेन॑ प्रस॒वाय॑ सावि॒त्रः सा॑वि॒त्रः प्र॑स॒वा
यै᳚न्द्रा॒ग्ने नै᳚न्द्रा॒ग्नेन॑ प्रस॒वाय॑ सावि॒त्रः । \newline
43. ऐ॒न्द्रा॒ग्नेनेत्यै᳚न्द्र - अ॒ग्नेन॑ । \newline
44. प्र॒स॒वाय॑ सावि॒त्रः सा॑वि॒त्रः प्र॑स॒वाय॑ प्रस॒वाय॑ सावि॒त्रो नि॑र्वरुण॒त्वाय॑ निर्वरुण॒त्वाय॑ सावि॒त्रः प्र॑स॒वाय॑ प्रस॒वाय॑ सावि॒त्रो नि॑र्वरुण॒त्वाय॑ । \newline
45. प्र॒स॒वायेति॑ प्र - स॒वाय॑ । \newline
46. सा॒वि॒त्रो नि॑र्वरुण॒त्वाय॑ निर्वरुण॒त्वाय॑ सावि॒त्रः सा॑वि॒त्रो नि॑र्वरुण॒त्वाय॑ वारु॒णो वा॑रु॒णो नि॑र्वरुण॒त्वाय॑ सावि॒त्रः सा॑वि॒त्रो नि॑र्वरुण॒त्वाय॑ वारु॒णः । \newline
47. नि॒र्व॒रु॒ण॒त्वाय॑ वारु॒णो वा॑रु॒णो नि॑र्वरुण॒त्वाय॑ निर्वरुण॒त्वाय॑ वारु॒णो म॑द्ध्य॒तो म॑द्ध्य॒तो वा॑रु॒णो नि॑र्वरुण॒त्वाय॑ निर्वरुण॒त्वाय॑ वारु॒णो म॑द्ध्य॒तः । \newline
48. नि॒र्व॒रु॒ण॒त्वायेति॑ निर्वरुण - त्वाय॑ । \newline
49. वा॒रु॒णो म॑द्ध्य॒तो म॑द्ध्य॒तो वा॑रु॒णो वा॑रु॒णो म॑द्ध्य॒त ऐ॒न्द्र मै॒न्द्रम् म॑द्ध्य॒तो वा॑रु॒णो वा॑रु॒णो म॑द्ध्य॒त ऐ॒न्द्रम् । \newline
50. म॒द्ध्य॒त ऐ॒न्द्र मै॒न्द्रम् म॑द्ध्य॒तो म॑द्ध्य॒त ऐ॒न्द्र मैन्द्रम् म॑द्ध्य॒तो म॑द्ध्य॒त ऐ॒न्द्र मा । \newline
51. ऐ॒न्द्र मैन्द्र मै॒न्द्र मा ल॑भते लभत॒ ऐन्द्र मै॒न्द्र मा ल॑भते । \newline
52. आ ल॑भते लभत॒ आ ल॑भते मद्ध्य॒तो म॑द्ध्य॒तो ल॑भत॒ आ ल॑भते मद्ध्य॒तः । \newline
53. ल॒भ॒ते॒ म॒द्ध्य॒तो म॑द्ध्य॒तो ल॑भते लभते मद्ध्य॒त ए॒वैव म॑द्ध्य॒तो ल॑भते लभते मद्ध्य॒त ए॒व । \newline
54. म॒द्ध्य॒त ए॒वैव म॑द्ध्य॒तो म॑द्ध्य॒त ए॒वेन्द्रि॒य मि॑न्द्रि॒य मे॒व म॑द्ध्य॒तो म॑द्ध्य॒त ए॒वेन्द्रि॒यम् । \newline
55. ए॒वेन्द्रि॒य मि॑न्द्रि॒य मे॒वैवेन्द्रि॒यं ॅयज॑माने॒ यज॑मान इन्द्रि॒य मे॒वैवेन्द्रि॒यं ॅयज॑माने । \newline
56. इ॒न्द्रि॒यं ॅयज॑माने॒ यज॑मान इन्द्रि॒य मि॑न्द्रि॒यं ॅयज॑माने दधाति दधाति॒ यज॑मान इन्द्रि॒य मि॑न्द्रि॒यं ॅयज॑माने दधाति । \newline
57. यज॑माने दधाति दधाति॒ यज॑माने॒ यज॑माने दधाति पु॒रस्ता᳚त् पु॒रस्ता᳚द् दधाति॒ यज॑माने॒ यज॑माने दधाति पु॒रस्ता᳚त् । \newline
58. द॒धा॒ति॒ पु॒रस्ता᳚त् पु॒रस्ता᳚द् दधाति दधाति पु॒रस्ता॑ दै॒न्द्र स्यै॒न्द्रस्य॑ पु॒रस्ता᳚द् दधाति दधाति पु॒रस्ता॑ दै॒न्द्रस्य॑ । \newline
\pagebreak
\markright{ TS 6.6.5.3  \hfill https://www.vedavms.in \hfill}

\section{ TS 6.6.5.3 }

\textbf{TS 6.6.5.3 } \newline
\textbf{Samhita Paata} \newline

पु॒रस्ता॑दै॒न्द्रस्य॑ वैश्वदे॒वमा ल॑भते वैश्वदे॒वं ॅवा अन्न॒मन्न॑मे॒व पु॒रस्ता᳚द्धत्ते॒ तस्मा᳚त् पु॒रस्ता॒दन्न॑मद्यत ऐ॒न्द्रमा॒लभ्य॑ मारु॒तमा ल॑भते॒ विड् वै म॒रुतो॒ विश॑मे॒वास्मा॒ अनु॑ बद्ध्नाति॒ यदि॑ का॒मये॑त॒ योऽव॑गतः॒ सोऽप॑ रुद्ध्यतां॒ ॅयोऽप॑रुद्धः॒ सोऽव॑ गच्छ॒त्वित्यै॒न्द्रस्य॑ लो॒के वा॑रु॒णमा ल॑भेत वारु॒णस्य॑ लो॒क ऐ॒न्द्रं- [  ] \newline

\textbf{Pada Paata} \newline

पु॒रस्ता᳚त् । ऐ॒न्द्रस्य॑ । वै॒श्व॒दे॒वमिति॑ वैश्व - दे॒वम् । एति॑ । ल॒भ॒ते॒ । वै॒श्व॒दे॒वमिति॑ वैश्व - दे॒वम् । वै । अन्न᳚म् । अन्न᳚म् । ए॒व । पु॒रस्ता᳚त् । ध॒त्ते॒ । तस्मा᳚त् । पु॒रस्ता᳚त् । अन्न᳚म् । अ॒द्य॒ते॒ । ऐ॒न्द्रम् । आ॒लभ्येत्या᳚-लभ्य॑ । मा॒रु॒तम् । एति॑ । ल॒भ॒ते॒ । विट् । वै । म॒रुतः॑ । विश᳚म् । ए॒व । अ॒स्मै॒ । अन्विति॑ । ब॒द्ध्ना॒ति॒ । यदि॑ । का॒मये॑त । यः । अव॑गत॒ इत्यव॑ - ग॒तः॒ । सः । अपेति॑ । रु॒द्ध्य॒ता॒म् । यः । अप॑रुद्ध॒ उत्यप॑- रु॒द्धः॒ । सः । अवेति॑ । ग॒च्छ॒तु॒ । इति॑ । ऐ॒न्द्रस्य॑ । लो॒के । वा॒रु॒णम् । एति॑ । ल॒भे॒त॒ । वा॒रु॒णस्य॑ । लो॒के । ऐ॒न्द्रम् ।  \newline


\textbf{Krama Paata} \newline

पु॒रस्ता॑दै॒न्द्रस्य॑ । ऐ॒न्द्रस्य॑ वैश्वदे॒वम् । वै॒श्व॒दे॒वमा । वै॒श्व॒दे॒वमिति॑ वैश्व - दे॒वम् । आ ल॑भते । ल॒भ॒ते॒ वै॒श्व॒दे॒वम् । वै॒श्व॒दे॒वम् ॅवै । वै॒श्व॒दे॒वमिति॑ वैश्व - दे॒वम् । वा अन्न᳚म् । अन्न॒मन्न᳚म् । अन्न॑मे॒व । ए॒व पु॒रस्ता᳚त् । पु॒रस्ता᳚द् धत्ते । ध॒त्ते॒ तस्मा᳚त् । तस्मा᳚त् पु॒रस्ता᳚त् । पु॒रस्ता॒दन्न᳚म् । अन्न॑मद्यते । अ॒द्य॒त॒ ऐ॒न्द्रम् । ऐ॒न्द्रमा॒लभ्य॑ । आ॒लभ्य॑ मारु॒तम् । आ॒लभ्येत्या᳚ - लभ्य॑ । मा॒रु॒तमा । आ ल॑भते । ल॒भ॒ते॒ विट् । विड् वै । वै म॒रुतः॑ । म॒रुतो॒ विश᳚म् । विश॑मे॒व । ए॒वास्मै᳚ । अ॒स्मा॒ अनु॑ । अनु॑ बद्ध्नाति । ब॒द्ध्ना॒ति॒ यदि॑ । यदि॑ का॒मये॑त । का॒मये॑त॒ यः । योऽव॑गतः । अव॑गतः॒ सः । अव॑गत॒ इत्यव॑ - ग॒तः॒ । सोऽप॑ । अप॑ रुद्ध्यताम् । रु॒द्ध्य॒ता॒म् ॅयः । योऽप॑रुद्धः । अप॑रुद्धः॒ सः । अप॑रुद्ध॒ इत्यप॑ - रु॒द्धः॒ । सोऽव॑ । अव॑ गच्छतु । ग॒च्छ॒त्विति॑ । इत्यै॒न्द्रस्य॑ । ऐ॒न्द्रस्य॑ लो॒के । लो॒के वा॑रु॒णम् । वा॒रु॒णमा । आ ल॑भेत । ल॒भे॒त॒ वा॒रु॒णस्य॑ । वा॒रु॒णस्य॑ लो॒के । लो॒क ऐ॒न्द्रम् ( ) । ऐ॒न्द्रम् ॅयः \newline

\textbf{Jatai Paata} \newline

1. पु॒रस्ता॑ दै॒न्द्र स्यै॒न्द्रस्य॑ पु॒रस्ता᳚त् पु॒रस्ता॑ दै॒न्द्रस्य॑ । \newline
2. ऐ॒न्द्रस्य॑ वैश्वदे॒वं ॅवै᳚श्वदे॒व मै॒न्द्र स्यै॒न्द्रस्य॑ वैश्वदे॒वम् । \newline
3. वै॒श्व॒दे॒व मा वै᳚श्वदे॒वं ॅवै᳚श्वदे॒व मा । \newline
4. वै॒श्व॒दे॒वमिति॑ वैश्व - दे॒वम् । \newline
5. आ ल॑भते लभत॒ आ ल॑भते । \newline
6. ल॒भ॒ते॒ वै॒श्व॒दे॒वं ॅवै᳚श्वदे॒वम् ॅल॑भते लभते वैश्वदे॒वम् । \newline
7. वै॒श्व॒दे॒वं ॅवै वै वै᳚श्वदे॒वं ॅवै᳚श्वदे॒वं ॅवै । \newline
8. वै॒श्व॒दे॒वमिति॑ वैश्व - दे॒वम् । \newline
9. वा अन्न॒ मन्नं॒ ॅवै वा अन्न᳚म् । \newline
10. अन्न॒ मन्न᳚म् । \newline
11. अन्न॑ मे॒वै वान्न॒ मन्न॑ मे॒व । \newline
12. ए॒व पु॒रस्ता᳚त् पु॒रस्ता॑ दे॒वैव पु॒रस्ता᳚त् । \newline
13. पु॒रस्ता᳚द् धत्ते धत्ते पु॒रस्ता᳚त् पु॒रस्ता᳚द् धत्ते । \newline
14. ध॒त्ते॒ तस्मा॒त् तस्मा᳚द् धत्ते धत्ते॒ तस्मा᳚त् । \newline
15. तस्मा᳚त् पु॒रस्ता᳚त् पु॒रस्ता॒त् तस्मा॒त् तस्मा᳚त् पु॒रस्ता᳚त् । \newline
16. पु॒रस्ता॒ दन्न॒ मन्न॑म् पु॒रस्ता᳚त् पु॒रस्ता॒ दन्न᳚म् । \newline
17. अन्न॑ मद्यते ऽद्य॒ते ऽन्न॒ मन्न॑ मद्यते । \newline
18. अ॒द्य॒त॒ ऐ॒न्द्र मै॒न्द्र म॑द्यते ऽद्यत ऐ॒न्द्रम् । \newline
19. ऐ॒न्द्र मा॒लभ्या॒ लभ्यै॒न्द्र मै॒न्द्र मा॒लभ्य॑ । \newline
20. आ॒लभ्य॑ मारु॒तम् मा॑रु॒त मा॒लभ्या॒ लभ्य॑ मारु॒तम् । \newline
21. आ॒लभ्येत्या᳚ - लभ्य॑ । \newline
22. मा॒रु॒त मा मा॑रु॒तम् मा॑रु॒त मा । \newline
23. आ ल॑भते लभत॒ आ ल॑भते । \newline
24. ल॒भ॒ते॒ विड् विळ् ल॑भते लभते॒ विट् । \newline
25. विड् वै वै विड् विड् वै । \newline
26. वै म॒रुतो॑ म॒रुतो॒ वै वै म॒रुतः॑ । \newline
27. म॒रुतो॒ विशं॒ ॅविश॑म् म॒रुतो॑ म॒रुतो॒ विश᳚म् । \newline
28. विश॑ मे॒वैव विशं॒ ॅविश॑ मे॒व । \newline
29. ए॒वास्मा॑ अस्मा ए॒वै वास्मै᳚ । \newline
30. अ॒स्मा॒ अन्वन् व॑स्मा अस्मा॒ अनु॑ । \newline
31. अनु॑ बद्ध्नाति बद्ध्ना॒ त्यन्वनु॑ बद्ध्नाति । \newline
32. ब॒द्ध्ना॒ति॒ यदि॒ यदि॑ बद्ध्नाति बद्ध्नाति॒ यदि॑ । \newline
33. यदि॑ का॒मये॑त का॒मये॑त॒ यदि॒ यदि॑ का॒मये॑त । \newline
34. का॒मये॑त॒ यो यः का॒मये॑त का॒मये॑त॒ यः । \newline
35. यो ऽव॑ग॒तो ऽव॑गतो॒ यो यो ऽव॑गतः । \newline
36. अव॑गतः॒ स सो ऽव॑ग॒तो ऽव॑गतः॒ सः । \newline
37. अव॑गत॒ इत्यव॑ - ग॒तः॒ । \newline
38. सो ऽपाप॒ स सो ऽप॑ । \newline
39. अप॑ रुद्ध्यताꣳ रुद्ध्यता॒ मपाप॑ रुद्ध्यताम् । \newline
40. रु॒द्ध्य॒तां॒ ॅयो यो रु॑द्ध्यताꣳ रुद्ध्यतां॒ ॅयः । \newline
41. यो ऽप॑रु॒द्धो ऽप॑रुद्धो॒ यो यो ऽप॑रुद्धः । \newline
42. अप॑रुद्धः॒ स सो ऽप॑रु॒द्धो ऽप॑रुद्धः॒ सः । \newline
43. अप॑रुद्ध॒ इत्यप॑ - रु॒द्धः॒ । \newline
44. सो ऽवाव॒ स सो ऽव॑ । \newline
45. अव॑ गच्छतु गच्छ॒ त्ववाव॑ गच्छतु । \newline
46. ग॒च्छ॒ त्वितीति॑ गच्छतु गच्छ॒ त्विति॑ । \newline
47. इत्यै॒न्द्र स्यै॒न्द्र स्येती त्यै॒न्द्रस्य॑ । \newline
48. ऐ॒न्द्रस्य॑ लो॒के लो॒क ऐ॒न्द्र स्यै॒न्द्रस्य॑ लो॒के । \newline
49. लो॒के वा॑रु॒णं ॅवा॑रु॒णम् ॅलो॒के लो॒के वा॑रु॒णम् । \newline
50. वा॒रु॒ण मा वा॑रु॒णं ॅवा॑रु॒ण मा । \newline
51. आ ल॑भेत लभे॒ता ल॑भेत । \newline
52. ल॒भे॒त॒ वा॒रु॒णस्य॑ वारु॒णस्य॑ लभेत लभेत वारु॒णस्य॑ । \newline
53. वा॒रु॒णस्य॑ लो॒के लो॒के वा॑रु॒णस्य॑ वारु॒णस्य॑ लो॒के । \newline
54. लो॒क ऐ॒न्द्र मै॒न्द्रम् ॅलो॒के लो॒क ऐ॒न्द्रम् । \newline
55. ऐ॒न्द्रं ॅयो य ऐ॒न्द्र मै॒न्द्रं ॅयः । \newline

\textbf{Ghana Paata } \newline

1. पु॒रस्ता॑ दै॒न्द्र स्यै॒न्द्रस्य॑ पु॒रस्ता᳚त् पु॒रस्ता॑ दै॒न्द्रस्य॑ वैश्वदे॒वं ॅवै᳚श्वदे॒व मै॒न्द्रस्य॑ पु॒रस्ता᳚त् पु॒रस्ता॑ दै॒न्द्रस्य॑ वैश्वदे॒वम् । \newline
2. ऐ॒न्द्रस्य॑ वैश्वदे॒वं ॅवै᳚श्वदे॒व मै॒न्द्र स्यै॒न्द्रस्य॑ वैश्वदे॒व मा वै᳚श्वदे॒व मै॒न्द्र
स्यै॒न्द्रस्य॑ वैश्वदे॒व मा । \newline
3. वै॒श्व॒दे॒व मा वै᳚श्वदे॒वं ॅवै᳚श्वदे॒व मा ल॑भते लभत॒ आ वै᳚श्वदे॒वं ॅवै᳚श्वदे॒व मा ल॑भते । \newline
4. वै॒श्व॒दे॒वमिति॑ वैश्व - दे॒वम् । \newline
5. आ ल॑भते लभत॒ आ ल॑भते वैश्वदे॒वं ॅवै᳚श्वदे॒वम् ॅल॑भत॒ आ ल॑भते वैश्वदे॒वम् । \newline
6. ल॒भ॒ते॒ वै॒श्व॒दे॒वं ॅवै᳚श्वदे॒वम् ॅल॑भते लभते वैश्वदे॒वं ॅवै वै वै᳚श्वदे॒वम् ॅल॑भते लभते वैश्वदे॒वं ॅवै । \newline
7. वै॒श्व॒दे॒वं ॅवै वै वै᳚श्वदे॒वं ॅवै᳚श्वदे॒वं ॅवा अन्न॒ मन्नं॒ ॅवै वै᳚श्वदे॒वं ॅवै᳚श्वदे॒वं ॅवा अन्न᳚म् । \newline
8. वै॒श्व॒दे॒वमिति॑ वैश्व - दे॒वम् । \newline
9. वा अन्न॒ मन्नं॒ ॅवै वा अन्न᳚म् । \newline
10. अन्न॒ मन्न᳚म् । \newline
11. अन्न॑ मे॒वैवान्न॒ मन्न॑ मे॒व पु॒रस्ता᳚त् पु॒रस्ता॑ दे॒वान्न॒ मन्न॑ मे॒व पु॒रस्ता᳚त् । \newline
12. ए॒व पु॒रस्ता᳚त् पु॒रस्ता॑ दे॒वैव पु॒रस्ता᳚द् धत्ते धत्ते पु॒रस्ता॑ दे॒वैव पु॒रस्ता᳚द् धत्ते । \newline
13. पु॒रस्ता᳚द् धत्ते धत्ते पु॒रस्ता᳚त् पु॒रस्ता᳚द् धत्ते॒ तस्मा॒त् तस्मा᳚द् धत्ते पु॒रस्ता᳚त् पु॒रस्ता᳚द् धत्ते॒ तस्मा᳚त् । \newline
14. ध॒त्ते॒ तस्मा॒त् तस्मा᳚द् धत्ते धत्ते॒ तस्मा᳚त् पु॒रस्ता᳚त् पु॒रस्ता॒त् तस्मा᳚द् धत्ते धत्ते॒ तस्मा᳚त् पु॒रस्ता᳚त् । \newline
15. तस्मा᳚त् पु॒रस्ता᳚त् पु॒रस्ता॒त् तस्मा॒त् तस्मा᳚त् पु॒रस्ता॒ दन्न॒ मन्न॑म् पु॒रस्ता॒त् तस्मा॒त् तस्मा᳚त् पु॒रस्ता॒ दन्न᳚म् । \newline
16. पु॒रस्ता॒ दन्न॒ मन्न॑म् पु॒रस्ता᳚त् पु॒रस्ता॒ दन्न॑ मद्यते ऽद्य॒ते ऽन्न॑म् पु॒रस्ता᳚त् पु॒रस्ता॒ दन्न॑ मद्यते । \newline
17. अन्न॑ मद्यते ऽद्य॒ते ऽन्न॒ मन्न॑ मद्यत ऐ॒न्द्र मै॒न्द्र म॑द्य॒ते ऽन्न॒ मन्न॑ मद्यत ऐ॒न्द्रम् । \newline
18. अ॒द्य॒त॒ ऐ॒न्द्र मै॒न्द्र म॑द्यते ऽद्यत ऐ॒न्द्र मा॒लभ्या॒ लभ्यै॒न्द्र म॑द्यते ऽद्यत ऐ॒न्द्र मा॒लभ्य॑ । \newline
19. ऐ॒न्द्र मा॒लभ्या॒ लभ्यै॒न्द्र मै॒न्द्र मा॒लभ्य॑ मारु॒तम् मा॑रु॒त मा॒लभ्यै॒न्द्र मै॒न्द्र मा॒लभ्य॑ मारु॒तम् । \newline
20. आ॒लभ्य॑ मारु॒तम् मा॑रु॒त मा॒लभ्या॒ लभ्य॑ मारु॒त मा मा॑रु॒त मा॒लभ्या॒ लभ्य॑ मारु॒त मा । \newline
21. आ॒लभ्येत्या᳚ - लभ्य॑ । \newline
22. मा॒रु॒त मा मा॑रु॒तम् मा॑रु॒त मा ल॑भते लभत॒ आ मा॑रु॒तम् मा॑रु॒त मा ल॑भते । \newline
23. आ ल॑भते लभत॒ आ ल॑भते॒ विड् विळ् ल॑भत॒ आ ल॑भते॒ विट् । \newline
24. ल॒भ॒ते॒ विड् विळ् ल॑भते लभते॒ विड् वै वै विळ् ल॑भते लभते॒ विड् वै । \newline
25. विड् वै वै विड् विड् वै म॒रुतो॑ म॒रुतो॒ वै विड् विड् वै म॒रुतः॑ । \newline
26. वै म॒रुतो॑ म॒रुतो॒ वै वै म॒रुतो॒ विशं॒ ॅविश॑म् म॒रुतो॒ वै वै म॒रुतो॒ विश᳚म् । \newline
27. म॒रुतो॒ विशं॒ ॅविश॑म् म॒रुतो॑ म॒रुतो॒ विश॑ मे॒वैव विश॑म् म॒रुतो॑ म॒रुतो॒ विश॑ मे॒व । \newline
28. विश॑ मे॒वैव विशं॒ ॅविश॑ मे॒वास्मा॑ अस्मा ए॒व विशं॒ ॅविश॑ मे॒वास्मै᳚ । \newline
29. ए॒वास्मा॑ अस्मा ए॒वै वास्मा॒ अन्वन् व॑स्मा ए॒वै वास्मा॒ अनु॑ । \newline
30. अ॒स्मा॒ अन्वन् व॑स्मा अस्मा॒ अनु॑ बद्ध्नाति बद्ध्ना॒ त्यन् व॑स्मा अस्मा॒ अनु॑ बद्ध्नाति । \newline
31. अनु॑ बद्ध्नाति बद्ध्ना॒ त्यन् वनु॑ बद्ध्नाति॒ यदि॒ यदि॑ बद्ध्ना॒ त्यन् वनु॑ बद्ध्नाति॒ यदि॑ । \newline
32. ब॒द्ध्ना॒ति॒ यदि॒ यदि॑ बद्ध्नाति बद्ध्नाति॒ यदि॑ का॒मये॑त का॒मये॑त॒ यदि॑ बद्ध्नाति बद्ध्नाति॒ यदि॑ का॒मये॑त । \newline
33. यदि॑ का॒मये॑त का॒मये॑त॒ यदि॒ यदि॑ का॒मये॑त॒ यो यः का॒मये॑त॒ यदि॒ यदि॑ का॒मये॑त॒ यः । \newline
34. का॒मये॑त॒ यो यः का॒मये॑त का॒मये॑त॒ यो ऽव॑ग॒तो ऽव॑गतो॒ यः का॒मये॑त का॒मये॑त॒ यो ऽव॑गतः । \newline
35. यो ऽव॑ग॒तो ऽव॑गतो॒ यो यो ऽव॑गतः॒ स सो ऽव॑गतो॒ यो यो ऽव॑गतः॒ सः । \newline
36. अव॑गतः॒ स सो ऽव॑ग॒तो ऽव॑गतः॒ सो ऽपाप॒ सो ऽव॑ग॒तो ऽव॑गतः॒ सो ऽप॑ । \newline
37. अव॑गत॒ इत्यव॑ - ग॒तः॒ । \newline
38. सो ऽपाप॒ स सो ऽप॑ रुद्ध्यताꣳ रुद्ध्यता॒ मप॒ स सो ऽप॑ रुद्ध्यताम् । \newline
39. अप॑ रुद्ध्यताꣳ रुद्ध्यता॒ मपाप॑ रुद्ध्यतां॒ ॅयो यो रु॑द्ध्यता॒ मपाप॑ रुद्ध्यतां॒ ॅयः । \newline
40. रु॒द्ध्य॒तां॒ ॅयो यो रु॑द्ध्यताꣳ रुद्ध्यतां॒ ॅयो ऽप॑रु॒द्धो ऽप॑रुद्धो॒ यो रु॑द्ध्यताꣳ रुद्ध्यतां॒ ॅयो ऽप॑रुद्धः । \newline
41. यो ऽप॑रु॒द्धो ऽप॑रुद्धो॒ यो यो ऽप॑रुद्धः॒ स सो ऽप॑रुद्धो॒ यो यो ऽप॑रुद्धः॒ सः । \newline
42. अप॑रुद्धः॒ स सो ऽप॑रु॒द्धो ऽप॑रुद्धः॒ सो ऽवाव॒ सो ऽप॑रु॒द्धो ऽप॑रुद्धः॒ सो ऽव॑ । \newline
43. अप॑रुद्ध॒ इत्यप॑ - रु॒द्धः॒ । \newline
44. सो ऽवाव॒ स सो ऽव॑ गच्छतु गच्छ॒ त्वव॒ स सो ऽव॑ गच्छतु । \newline
45. अव॑ गच्छतु गच्छ॒ त्ववाव॑ गच्छ॒ त्वितीति॑ गच्छ॒ त्ववाव॑ गच्छ॒त्विति॑ । \newline
46. ग॒च्छ॒ त्वितीति॑ गच्छतु गच्छ॒ त्वित्यै॒न्द्र स्यै॒न्द्रस्येति॑ गच्छतु गच्छ॒ त्वित्यै॒न्द्रस्य॑ । \newline
47. इत्यै॒न्द्र स्यै॒न्द्रस्येतीत्यै॒ न्द्र स्य॑लो॒के लो॒क ऐ॒न्द्र स्येती त्यै॒न्द्रस्य॑ लो॒के । \newline
48. ऐ॒न्द्रस्य॑ लो॒के लो॒क ऐ॒न्द्र स्यै॒न्द्रस्य॑ लो॒के वा॑रु॒णं ॅवा॑रु॒णम् ॅलो॒क ऐ॒न्द्र स्यै॒न्द्रस्य॑ लो॒के वा॑रु॒णम् । \newline
49. लो॒के वा॑रु॒णं ॅवा॑रु॒णम् ॅलो॒के लो॒के वा॑रु॒ण मा वा॑रु॒णम् ॅलो॒के लो॒के वा॑रु॒ण मा । \newline
50. वा॒रु॒ण मा वा॑रु॒णं ॅवा॑रु॒ण मा ल॑भेत लभे॒ता वा॑रु॒णं ॅवा॑रु॒ण मा ल॑भेत । \newline
51. आ ल॑भेत लभे॒ता ल॑भेत वारु॒णस्य॑ वारु॒णस्य॑ लभे॒ता ल॑भेत वारु॒णस्य॑ । \newline
52. ल॒भे॒त॒ वा॒रु॒णस्य॑ वारु॒णस्य॑ लभेत लभेत वारु॒णस्य॑ लो॒के लो॒के वा॑रु॒णस्य॑ लभेत लभेत वारु॒णस्य॑ लो॒के । \newline
53. वा॒रु॒णस्य॑ लो॒के लो॒के वा॑रु॒णस्य॑ वारु॒णस्य॑ लो॒क ऐ॒न्द्र मै॒न्द्रम् ॅलो॒के वा॑रु॒णस्य॑ वारु॒णस्य॑ लो॒क ऐ॒न्द्रम् । \newline
54. लो॒क ऐ॒न्द्र मै॒न्द्रम् ॅलो॒के लो॒क ऐ॒न्द्रं ॅयो य ऐ॒न्द्रम् ॅलो॒के लो॒क ऐ॒न्द्रं ॅयः । \newline
55. ऐ॒न्द्रं ॅयो य ऐ॒न्द्र मै॒न्द्रं ॅय ए॒वैव य ऐ॒न्द्र मै॒न्द्रं ॅय ए॒व । \newline
\pagebreak
\markright{ TS 6.6.5.4  \hfill https://www.vedavms.in \hfill}

\section{ TS 6.6.5.4 }

\textbf{TS 6.6.5.4 } \newline
\textbf{Samhita Paata} \newline

ॅय ए॒वाव॑गतः॒ सोऽप॑ रुद्ध्यते॒ योऽप॑रुद्धः॒ सोऽव॑ गच्छति॒ यदि॑ का॒मये॑त प्र॒जा मु॑ह्येयु॒रिति॑ प॒शून् व्यति॑षजेत् प्र॒जा ए॒व मो॑हयति॒ यद॑भिवाह॒तो॑ऽपां ॅवा॑रु॒णमा॒लभे॑त प्र॒जा वरु॑णो गृह्णीयाद्-दक्षिण॒त उद॑ञ्च॒मा ल॑भतेऽपवाह॒तो॑ऽपां प्र॒जाना॒-मव॑रुण ग्राहाय ॥ \newline

\textbf{Pada Paata} \newline

यः । ए॒व । अव॑गत॒ इत्यव॑ - ग॒तः॒ । सः । अपेति॑ । रु॒द्ध्य॒ते॒ । यः । अप॑रुद्ध॒ उत्यप॑-रु॒द्धः॒ । सः । अवेति॑ । ग॒च्छ॒ति॒ । यदि॑ । का॒मये॑त । प्र॒जा इति॑ प्र - जाः । मु॒ह्ये॒युः॒ । इति॑ । प॒शून् । व्यति॑षजे॒दिति॑ वि - अति॑षजेत् । प्र॒जा इति॑ प्र - जाः । ए॒व । मो॒ह॒य॒ति॒ । यत् । अ॒भि॒वा॒ह॒त इत्य॑भि - वा॒ह॒तः । अ॒पाम् । वा॒रु॒णम् । आ॒लभे॒तेत्या᳚ - लभे॑त । प्र॒जा इति॑ प्र - जाः । वरु॑णः । गृ॒ह्णी॒या॒त् । द॒क्षि॒ण॒तः । उद॑ञ्चम् । एति॑ । ल॒भ॒ते॒ । अ॒प॒वा॒ह॒त इत्य॑प - वा॒ह॒तः । अ॒पाम् । प्र॒जाना॒मिति॑ प्र - जाना᳚म् । अव॑रुणग्राहा॒येत्यव॑रुण - ग्रा॒हा॒य॒ ॥  \newline


\textbf{Krama Paata} \newline

य ए॒व । ए॒वाव॑गतः । अव॑गतः॒ सः । अव॑गत॒ इत्यव॑ - ग॒तः॒ । सोऽप॑ । अप॑ रुद्ध्यते । रु॒द्ध्य॒ते॒ यः । योऽप॑रुद्धः । अप॑रुद्धः॒ सः । अप॑रुद्ध॒ इत्यप॑ - रु॒द्धः॒ । सोऽव॑ । अव॑ गच्छति । ग॒च्छ॒ति॒ यदि॑ । यदि॑ का॒मये॑त । का॒मये॑त प्र॒जाः । प्र॒जा मु॑ह्येयुः । प्र॒जा इति॑ प्र - जाः । मु॒ह्ये॒यु॒रिति॑ । इति॑ प॒शून् । प॒शून् व्यति॑षजेत् । व्यति॑षजेत् प्र॒जाः । व्यति॑षजे॒दिति॑ वि - अति॑षजेत् । प्र॒जा ए॒व । प्र॒जा इति॑ प्र - जाः । ए॒व मो॑हयति । मो॒ह॒य॒ति॒ यत् । यद॑भिवाह॒तः । अ॒भि॒वा॒ह॒तो॑ऽपाम् । अ॒भि॒वा॒ह॒त इत्य॑भि - वा॒ह॒तः । अ॒पाम् ॅवा॑रु॒णम् । वा॒रु॒णमा॒लभे॑त । आ॒लभे॑त प्र॒जाः । आ॒लभे॒तेत्या᳚ - लभे॑त । प्र॒जा वरु॑णः । प्र॒जा इति॑ प्र - जाः । वरु॑णो गृह्णीयात् । गृ॒ह्णी॒या॒द् द॒क्षि॒ण॒तः । द॒क्षि॒ण॒त उद॑ञ्चम् । उद॑ञ्च॒मा । आ ल॑भते । ल॒भे॒ते॒ऽप॒वा॒ह॒तः । अ॒प॒वा॒ह॒तो॑ऽपाम् । अ॒प॒वा॒ह॒त इत्य॑प - वा॒ह॒तः । अ॒पाम् प्र॒जाना᳚म् । प्र॒जाना॒मव॑रुणग्राहाय । प्र॒जाना॒मिति॑ प्र - जाना᳚म् । अव॑रुणग्राहा॒येत्यव॑रुण - ग्रा॒हा॒य॒ । \newline

\textbf{Jatai Paata} \newline

1. य ए॒वैव यो य ए॒व । \newline
2. ए॒वा व॑ग॒तो ऽव॑गत ए॒वैवा व॑गतः । \newline
3. अव॑गतः॒ स सो ऽव॑ग॒तो ऽव॑गतः॒ सः । \newline
4. अव॑गत॒ इत्यव॑ - ग॒तः॒ । \newline
5. सो ऽपाप॒ स सो ऽप॑ । \newline
6. अप॑ रुद्ध्यते रुद्ध्य॒ते ऽपाप॑ रुद्ध्यते । \newline
7. रु॒द्ध्य॒ते॒ यो यो रु॑द्ध्यते रुद्ध्यते॒ यः । \newline
8. यो ऽप॑रु॒द्धो ऽप॑रुद्धो॒ यो यो ऽप॑रुद्धः । \newline
9. अप॑रुद्धः॒ स सो ऽप॑रु॒द्धो ऽप॑रुद्धः॒ सः । \newline
10. अप॑रुद्ध॒ इत्यप॑ - रु॒द्धः॒ । \newline
11. सो ऽवाव॒ स सो ऽव॑ । \newline
12. अव॑ गच्छति गच्छ॒ त्यवाव॑ गच्छति । \newline
13. ग॒च्छ॒ति॒ यदि॒ यदि॑ गच्छति गच्छति॒ यदि॑ । \newline
14. यदि॑ का॒मये॑त का॒मये॑त॒ यदि॒ यदि॑ का॒मये॑त । \newline
15. का॒मये॑त प्र॒जाः प्र॒जाः का॒मये॑त का॒मये॑त प्र॒जाः । \newline
16. प्र॒जा मु॑ह्येयुर् मुह्येयुः प्र॒जाः प्र॒जा मु॑ह्येयुः । \newline
17. प्र॒जा इति॑ प्र - जाः । \newline
18. मु॒ह्ये॒यु॒ रितीति॑ मुह्येयुर् मुह्येयु॒ रिति॑ । \newline
19. इति॑ प॒शून् प॒शू नितीति॑ प॒शून् । \newline
20. प॒शून् व्यति॑षजे॒द् व्यति॑षजेत् प॒शून् प॒शून् व्यति॑षजेत् । \newline
21. व्यति॑षजेत् प्र॒जाः प्र॒जा व्यति॑षजे॒द् व्यति॑षजेत् प्र॒जाः । \newline
22. व्यति॑षजे॒दिति॑ वि - अति॑षजेत् । \newline
23. प्र॒जा ए॒वैव प्र॒जाः प्र॒जा ए॒व । \newline
24. प्र॒जा इति॑ प्र - जाः । \newline
25. ए॒व मो॑हयति मोहय त्ये॒वैव मो॑हयति । \newline
26. मो॒ह॒य॒ति॒ यद् यन् मो॑हयति मोहयति॒ यत् । \newline
27. यद॑भिवाह॒तो॑ ऽभिवाह॒तो यद् यद॑भिवाह॒तः । \newline
28. अ॒भि॒वा॒ह॒तो॑ ऽपा म॒पा म॑भिवाह॒तो॑ ऽभिवाह॒तो॑ ऽपाम् । \newline
29. अ॒भि॒वा॒ह॒त इत्य॑भि - वा॒ह॒तः । \newline
30. अ॒पां ॅवा॑रु॒णं ॅवा॑रु॒ण म॒पा म॒पां ॅवा॑रु॒णम् । \newline
31. वा॒रु॒ण मा॒लभे॑ता॒ लभे॑त वारु॒णं ॅवा॑रु॒ण मा॒लभे॑त । \newline
32. आ॒लभे॑त प्र॒जाः प्र॒जा आ॒लभे॑ता॒ लभे॑त प्र॒जाः । \newline
33. आ॒लभे॒तेत्या᳚ - लभे॑त । \newline
34. प्र॒जा वरु॑णो॒ वरु॑णः प्र॒जाः प्र॒जा वरु॑णः । \newline
35. प्र॒जा इति॑ प्र - जाः । \newline
36. वरु॑णो गृह्णीयाद् गृह्णीया॒द् वरु॑णो॒ वरु॑णो गृह्णीयात् । \newline
37. गृ॒ह्णी॒या॒द् द॒क्षि॒ण॒तो द॑क्षिण॒तो गृ॑ह्णीयाद् गृह्णीयाद् दक्षिण॒तः । \newline
38. द॒क्षि॒ण॒त उद॑ञ्च॒ मुद॑ञ्चम् दक्षिण॒तो द॑क्षिण॒त उद॑ञ्चम् । \newline
39. उद॑ञ्च॒ मोद॑ञ्च॒ मुद॑ञ्च॒ मा । \newline
40. आ ल॑भते लभत॒ आ ल॑भते । \newline
41. ल॒भ॒ते॒ ऽप॒वा॒ह॒तो॑ ऽपवाह॒तो ल॑भते लभते ऽपवाह॒तः । \newline
42. अ॒प॒वा॒ह॒तो॑ ऽपा म॒पा म॑पवाह॒तो॑ ऽपवाह॒तो॑ ऽपाम् । \newline
43. अ॒प॒वा॒ह॒त इत्य॑प - वा॒ह॒तः । \newline
44. अ॒पाम् प्र॒जाना᳚म् प्र॒जाना॑ म॒पा म॒पाम् प्र॒जाना᳚म् । \newline
45. प्र॒जाना॒ मव॑रुणग्राहा॒या व॑रुणग्राहाय प्र॒जाना᳚म् प्र॒जाना॒ मव॑रुणग्राहाय । \newline
46. प्र॒जाना॒मिति॑ प्र - जाना᳚म् । \newline
47. अव॑रुणग्राहा॒येत्यव॑रुण - ग्रा॒हा॒य॒ । \newline

\textbf{Ghana Paata } \newline

1. य ए॒वैव यो य ए॒वा व॑ग॒तो ऽव॑गत ए॒व यो य ए॒वा व॑गतः । \newline
2. ए॒वा व॑ग॒तो ऽव॑गत ए॒वैवा व॑गतः॒ स सो ऽव॑गत ए॒वैवा व॑गतः॒ सः । \newline
3. अव॑गतः॒ स सो ऽव॑ग॒तो ऽव॑गतः॒ सो ऽपाप॒ सो ऽव॑ग॒तो ऽव॑गतः॒ सो ऽप॑ । \newline
4. अव॑गत॒ इत्यव॑ - ग॒तः॒ । \newline
5. सो ऽपाप॒ स सो ऽप॑ रुद्ध्यते रुद्ध्य॒ते ऽप॒ स सो ऽप॑ रुद्ध्यते । \newline
6. अप॑ रुद्ध्यते रुद्ध्य॒ते ऽपाप॑ रुद्ध्यते॒ यो यो रु॑द्ध्य॒ते ऽपाप॑ रुद्ध्यते॒ यः । \newline
7. रु॒द्ध्य॒ते॒ यो यो रु॑द्ध्यते रुद्ध्यते॒ यो ऽप॑रु॒द्धो ऽप॑रुद्धो॒ यो रु॑द्ध्यते रुद्ध्यते॒ यो ऽप॑रुद्धः । \newline
8. यो ऽप॑रु॒द्धो ऽप॑रुद्धो॒ यो यो ऽप॑रुद्धः॒ स सो ऽप॑रुद्धो॒ यो यो ऽप॑रुद्धः॒ सः । \newline
9. अप॑रुद्धः॒ स सो ऽप॑रु॒द्धो ऽप॑रुद्धः॒ सो ऽवाव॒ सो ऽप॑रु॒द्धो ऽप॑रुद्धः॒ सो ऽव॑ । \newline
10. अप॑रुद्ध॒ इत्यप॑ - रु॒द्धः॒ । \newline
11. सो ऽवाव॒ स सो ऽव॑ गच्छति गच्छ॒ त्यव॒ स सो ऽव॑ गच्छति । \newline
12. अव॑ गच्छति गच्छ॒ त्यवाव॑ गच्छति॒ यदि॒ यदि॑ गच्छ॒ त्यवाव॑ गच्छति॒ यदि॑ । \newline
13. ग॒च्छ॒ति॒ यदि॒ यदि॑ गच्छति गच्छति॒ यदि॑ का॒मये॑त का॒मये॑त॒ यदि॑ गच्छति गच्छति॒ यदि॑ का॒मये॑त । \newline
14. यदि॑ का॒मये॑त का॒मये॑त॒ यदि॒ यदि॑ का॒मये॑त प्र॒जाः प्र॒जाः का॒मये॑त॒ यदि॒ यदि॑ का॒मये॑त प्र॒जाः । \newline
15. का॒मये॑त प्र॒जाः प्र॒जाः का॒मये॑त का॒मये॑त प्र॒जा मु॑ह्येयुर् मुह्येयुः प्र॒जाः का॒मये॑त का॒मये॑त प्र॒जा मु॑ह्येयुः । \newline
16. प्र॒जा मु॑ह्येयुर् मुह्येयुः प्र॒जाः प्र॒जा मु॑ह्येयु॒ रितीति॑ मुह्येयुः प्र॒जाः प्र॒जा मु॑ह्येयु॒ रिति॑ । \newline
17. प्र॒जा इति॑ प्र - जाः । \newline
18. मु॒ह्ये॒यु॒ रितीति॑ मुह्येयुर् मुह्येयु॒ रिति॑ प॒शून् प॒शूनिति॑ मुह्येयुर् मुह्येयु॒ रिति॑ प॒शून् । \newline
19. इति॑ प॒शून् प॒शूनितीति॑ प॒शून् व्यति॑षजे॒द् व्यति॑षजेत् प॒शूनितीति॑ प॒शून् व्यति॑षजेत् । \newline
20. प॒शून् व्यति॑षजे॒द् व्यति॑षजेत् प॒शून् प॒शून् व्यति॑षजेत् प्र॒जाः प्र॒जा व्यति॑षजेत् प॒शून् प॒शून् व्यति॑षजेत् प्र॒जाः । \newline
21. व्यति॑षजेत् प्र॒जाः प्र॒जा व्यति॑षजे॒द् व्यति॑षजेत् प्र॒जा ए॒वैव प्र॒जा व्यति॑षजे॒द् व्यति॑षजेत् प्र॒जा ए॒व । \newline
22. व्यति॑षजे॒दिति॑ वि - अति॑षजेत् । \newline
23. प्र॒जा ए॒वैव प्र॒जाः प्र॒जा ए॒व मो॑हयति मोहय त्ये॒व प्र॒जाः प्र॒जा ए॒व मो॑हयति । \newline
24. प्र॒जा इति॑ प्र - जाः । \newline
25. ए॒व मो॑हयति मोहय त्ये॒वैव मो॑हयति॒ यद् यन् मो॑हय त्ये॒वैव मो॑हयति॒ यत् । \newline
26. मो॒ह॒य॒ति॒ यद् यन् मो॑हयति मोहयति॒ यद॑भिवाह॒तो॑ ऽभिवाह॒तो यन् मो॑हयति मोहयति॒ यद॑भिवाह॒तः । \newline
27. यद॑भिवाह॒तो॑ ऽभिवाह॒तो यद् यद॑भिवाह॒तो॑ ऽपा म॒पा म॑भिवाह॒तो यद् यद॑भिवाह॒तो॑ ऽपाम् । \newline
28. अ॒भि॒वा॒ह॒तो॑ ऽपा म॒पा म॑भिवाह॒तो॑ ऽभिवाह॒तो॑ ऽपां ॅवा॑रु॒णं ॅवा॑रु॒ण म॒पा म॑भिवाह॒तो॑ ऽभिवाह॒तो॑ ऽपां ॅवा॑रु॒णम् । \newline
29. अ॒भि॒वा॒ह॒त इत्य॑भि - वा॒ह॒तः । \newline
30. अ॒पां ॅवा॑रु॒णं ॅवा॑रु॒ण म॒पा म॒पां ॅवा॑रु॒ण मा॒लभे॑ता॒ लभे॑त वारु॒ण म॒पा म॒पां ॅवा॑रु॒ण मा॒लभे॑त । \newline
31. वा॒रु॒ण मा॒लभे॑ता॒ लभे॑त वारु॒णं ॅवा॑रु॒ण मा॒लभे॑त प्र॒जाः प्र॒जा आ॒लभे॑त वारु॒णं ॅवा॑रु॒ण मा॒लभे॑त प्र॒जाः । \newline
32. आ॒लभे॑त प्र॒जाः प्र॒जा आ॒लभे॑ता॒ लभे॑त प्र॒जा वरु॑णो॒ वरु॑णः प्र॒जा आ॒लभे॑ता॒ लभे॑त प्र॒जा वरु॑णः । \newline
33. आ॒लभे॒तेत्या᳚ - लभे॑त । \newline
34. प्र॒जा वरु॑णो॒ वरु॑णः प्र॒जाः प्र॒जा वरु॑णो गृह्णीयाद् गृह्णीया॒द् वरु॑णः प्र॒जाः प्र॒जा वरु॑णो गृह्णीयात् । \newline
35. प्र॒जा इति॑ प्र - जाः । \newline
36. वरु॑णो गृह्णीयाद् गृह्णीया॒द् वरु॑णो॒ वरु॑णो गृह्णीयाद् दक्षिण॒तो द॑क्षिण॒तो गृ॑ह्णीया॒द् वरु॑णो॒ वरु॑णो गृह्णीयाद् दक्षिण॒तः । \newline
37. गृ॒ह्णी॒या॒द् द॒क्षि॒ण॒तो द॑क्षिण॒तो गृ॑ह्णीयाद् गृह्णीयाद् दक्षिण॒त उद॑ञ्च॒ मुद॑ञ्चम् दक्षिण॒तो गृ॑ह्णीयाद् गृह्णीयाद् दक्षिण॒त उद॑ञ्चम् । \newline
38. द॒क्षि॒ण॒त उद॑ञ्च॒ मुद॑ञ्चम् दक्षिण॒तो द॑क्षिण॒त उद॑ञ्च॒ मोद॑ञ्चम् दक्षिण॒तो द॑क्षिण॒त उद॑ञ्च॒ मा । \newline
39. उद॑ञ्च॒ मोद॑ञ्च॒ मुद॑ञ्च॒ मा ल॑भते लभत॒ ओद॑ञ्च॒ मुद॑ञ्च॒ मा ल॑भते । \newline
40. आ ल॑भते लभत॒ आ ल॑भते ऽपवाह॒तो॑ ऽपवाह॒तो ल॑भत॒ आ ल॑भते ऽपवाह॒तः । \newline
41. ल॒भ॒ते॒ ऽप॒वा॒ह॒तो॑ ऽपवाह॒तो ल॑भते लभते ऽपवाह॒तो॑ ऽपा म॒पा म॑पवाह॒तो ल॑भते लभते ऽपवाह॒तो॑ ऽपाम् । \newline
42. अ॒प॒वा॒ह॒तो॑ ऽपा म॒पा म॑पवाह॒तो॑ ऽपवाह॒तो॑ ऽपाम् प्र॒जाना᳚म् प्र॒जाना॑ म॒पा म॑पवाह॒तो॑ ऽपवाह॒तो॑ ऽपाम् प्र॒जाना᳚म् । \newline
43. अ॒प॒वा॒ह॒त इत्य॑प - वा॒ह॒तः । \newline
44. अ॒पाम् प्र॒जाना᳚म् प्र॒जाना॑ म॒पा म॒पाम् प्र॒जाना॒ मव॑रुणग्राहा॒या व॑रुणग्राहाय प्र॒जाना॑ म॒पा म॒पाम् प्र॒जाना॒ मव॑रुणग्राहाय । \newline
45. प्र॒जाना॒ मव॑रुणग्राहा॒या व॑रुणग्राहाय प्र॒जाना᳚म् प्र॒जाना॒ मव॑रुणग्राहाय । \newline
46. प्र॒जाना॒मिति॑ प्र - जाना᳚म् । \newline
47. अव॑रुणग्राहा॒येत्यव॑रुण - ग्रा॒हा॒य॒ । \newline
\pagebreak
\markright{ TS 6.6.6.1  \hfill https://www.vedavms.in \hfill}

\section{ TS 6.6.6.1 }

\textbf{TS 6.6.6.1 } \newline
\textbf{Samhita Paata} \newline

इन्द्रः॒ पत्नि॑या॒ मनु॑मयाजय॒त् तां पर्य॑ग्निकृता॒-मुद॑सृज॒त् तया॒ मनु॑रार्द्ध्नो॒द्यत् पर्य॑ग्निकृतं पात्नीव॒तमु॑थ् सृ॒जति॒ यामे॒व मनु॒र॒. ऋद्धि॒मार्द्ध्नो॒त् तामे॒व यज॑मान ऋध्नोति य॒ज्ञ्स्य॒ वा अप्र॑तिष्ठिताद्-य॒ज्ञ्ः परा॑ भवति य॒ज्ञ्ं प॑रा॒भव॑न्तं॒ ॅयज॑मा॒नोऽनु॒ परा॑ भवति॒ यदाज्ये॑न पात्नीव॒तꣳ सꣳ॑स्था॒पय॑ति य॒ज्ञ्स्य॒ प्रति॑ष्ठित्यै य॒ज्ञ्ं प्र॑ति॒तिष्ठ॑न्तं॒ ॅयज॑मा॒नोऽनु॒ प्रति॑ तिष्ठती॒ष्टं ॅव॒पया॒- [  ] \newline

\textbf{Pada Paata} \newline

इन्द्रः॑ । पत्नि॑या । मनु᳚म् । अ॒या॒ज॒य॒त् । ताम् । पर्य॑ग्निकृता॒मिति॒ पर्य॑ग्नि - कृ॒ता॒म् । उदिति॑ । अ॒सृ॒ज॒त् । तया᳚ । मनुः॑ । आ॒द्‌र्ध्नो॒त् । यत् । पर्य॑ग्निकृत॒मिति॒ पर्य॑ग्नि-कृ॒त॒म् । पा॒त्नी॒व॒तमिति॑ पात्नी-व॒तम् । उ॒थ्सृ॒जतीत्यु॑त् - सृ॒जति॑ । याम् । ए॒व । मनुः॑ । ऋद्धि᳚म् । आद्‌र्ध्नो᳚त् । ताम् । ए॒व । यज॑मानः । ऋ॒द्ध्नो॒ति॒ । य॒ज्ञ्स्य॑ । वै । अप्र॑तिष्ठिता॒दित्यप्र॑ति - स्थि॒ता॒त् । य॒ज्ञ्ः । परेति॑ । भ॒व॒ति॒ । य॒ज्ञ्म् । प॒रा॒भव॑न्त॒मिति॑ परा - भव॑न्तम् । यज॑मानः । अनु॑ । परेति॑ । भ॒व॒ति॒ । यत् । आज्ये॑न । पा॒त्नी॒व॒तमिति॑ पात्नी - व॒तम् । सꣳ॒॒स्था॒पय॒तीति॑ सं - स्था॒पय॑ति । य॒ज्ञ्स्य॑ । प्रति॑ष्ठित्या॒ इति॒ प्रति॑-स्थि॒त्यै॒ । य॒ज्ञ्म् । प्र॒ति॒तिष्ठ॑न्त॒मिति॑ प्रति - तिष्ठ॑न्तम् । यज॑मानः । अनु॑ । प्रतीति॑ । ति॒ष्ठ॒ति॒ । इ॒ष्टम् । व॒पया᳚ ।  \newline


\textbf{Krama Paata} \newline

इन्द्रः॒ पत्नि॑या । पत्नि॑या॒ मनु᳚म् । मनु॑मयाजयत् । अ॒या॒ज॒य॒त् ताम् । ताम् पर्य॑ग्निकृताम् । पर्य॑ग्निकृता॒मुत् । पर्य॑ग्निकृता॒मिति॒ पर्य॑ग्नि - कृ॒ता॒म् । उद॑सृजत् । अ॒सृ॒ज॒त् तया᳚ । तया॒ मनुः॑ । मनु॑रार्द्ध्नोत् । आ॒र्द्ध्नो॒द् यत् । यत् पर्य॑ग्निकृतम् । पर्य॑ग्निकृतम् पात्नीव॒तम् । पर्य॑ग्निकृत॒मिति॒ पर्य॑ग्नि - कृ॒त॒म् । पा॒त्नी॒व॒तमु॑थ्सृ॒जति॑ । पा॒त्नी॒व॒तमिति॑ पात्नी - व॒तम् । उ॒थ्सृ॒जति॒ याम् । उ॒थ्स॒जतीत्यु॑त् - सृ॒जति॑ । यामे॒व । ए॒व मनुः॑ । मनु॒र्.॒ ऋद्धि᳚म् । ऋद्धि॒मार्द्ध्नो᳚त् । आर्द्ध्नो॒त् ताम् । तामे॒व । ए॒व यज॑मानः । यज॑मान ऋद्ध्नोति । ऋ॒ध्नो॒ति॒ य॒ज्ञ्स्य॑ । य॒ज्ञ्स्य॒ वै । वा अप्र॑तिष्ठतात् । अप्र॑तिष्ठताद् य॒ज्ञ्ः । अप्र॑तिष्ठता॒दित्यप्र॑ति - स्थि॒ता॒त्॒ । य॒ज्ञ्ः परा᳚ । परा॑ भवति । भ॒व॒ति॒ य॒ज्ञ्म् । य॒ज्ञ्म् प॑रा॒भव॑न्तम् । प॒रा॒भव॑न्त॒म् ॅयज॑मानः । प॒रा॒भव॑न्त॒मिति॑ परा - भव॑न्तम् । यज॑मा॒नोऽनु॑ । अनु॒ परा᳚ । परा॑ भवति । भ॒व॒ति॒ यत् । यदाज्ये॑न । आज्ये॑न पात्नीव॒तम् । पा॒त्नी॒व॒तꣳ सꣳ॑स्था॒पय॑ति । पा॒त्नी॒व॒तमिति॑ पात्नी - व॒तम् । सꣳ॒॒स्था॒पय॑ति य॒ज्ञ्स्य॑ । सꣳ॒॒स्था॒पय॒तीति॑ सम् - स्था॒पय॑ति । य॒ज्ञ्स्य॒ प्रति॑ष्ठित्यै । प्रति॑ष्ठित्यै य॒ज्ञ्म् । प्रति॑ष्ठित्या॒ इति॒ प्रति॑ - स्थि॒त्यै॒ । य॒ज्ञ्म् प्र॑ति॒तिष्ठ॑न्तम् । प्र॒ति॒तिष्ठ॑न्त॒म् ॅयज॑मानः । प्र॒ति॒तिष्ठ॑न्त॒मिति॑ प्रति - तिष्ठ॑न्तम् । यज॑मा॒नोऽनु॑ । अनु॒ प्रति॑ । प्रति॑ तिष्ठति । ति॒ष्ठ॒ती॒ष्टम् । इ॒ष्टम् ॅव॒पया᳚ ( ) । व॒पया॒ भव॑ति \newline

\textbf{Jatai Paata} \newline

1. इन्द्रः॒ पत्नि॑या॒ पत्नि॒ येन्द्र॒ इन्द्रः॒ पत्नि॑या । \newline
2. पत्नि॑या॒ मनु॒म् मनु॒म् पत्नि॑या॒ पत्नि॑या॒ मनु᳚म् । \newline
3. मनु॑ मयाजय दयाजय॒न् मनु॒म् मनु॑ मयाजयत् । \newline
4. अ॒या॒ज॒य॒त् ताम् ता म॑याजय दयाजय॒त् ताम् । \newline
5. ताम् पर्य॑ग्निकृता॒म् पर्य॑ग्निकृता॒म् ताम् ताम् पर्य॑ग्निकृताम् । \newline
6. पर्य॑ग्निकृता॒ मुदुत् पर्य॑ग्निकृता॒म् पर्य॑ग्निकृता॒ मुत् । \newline
7. पर्य॑ग्निकृता॒मिति॒ पर्य॑ग्नि - कृ॒ता॒म् । \newline
8. उद॑सृज दसृज॒ दुदु द॑सृजत् । \newline
9. अ॒सृ॒ज॒त् तया॒ तया॑ ऽसृज दसृज॒त् तया᳚ । \newline
10. तया॒ मनु॒र् मनु॒ स्तया॒ तया॒ मनुः॑ । \newline
11. मनु॑ रार्द्ध्नो दार्द्ध्नो॒न् मनु॒र् मनु॑ रार्द्ध्नोत् । \newline
12. आ॒र्द्ध्नो॒द् यद् यदा᳚र्द्ध्नो दार्द्ध्नो॒द् यत् । \newline
13. यत् पर्य॑ग्निकृत॒म् पर्य॑ग्निकृतं॒ ॅयद् यत् पर्य॑ग्निकृतम् । \newline
14. पर्य॑ग्निकृतम् पात्नीव॒तम् पा᳚त्नीव॒तम् पर्य॑ग्निकृत॒म् पर्य॑ग्निकृतम् पात्नीव॒तम् । \newline
15. पर्य॑ग्निकृत॒मिति॒ पर्य॑ग्नि - कृ॒त॒म् । \newline
16. पा॒त्नी॒व॒त मु॑थ्सृ॒ज त्यु॑थ्सृ॒जति॑ पात्नीव॒तम् पा᳚त्नीव॒त मु॑थ्सृ॒जति॑ । \newline
17. पा॒त्नी॒व॒तमिति॑ पात्नी - व॒तम् । \newline
18. उ॒थ्सृ॒जति॒ यां ॅया मु॑थ्सृ॒ज त्यु॑थ्सृ॒जति॒ याम् । \newline
19. उ॒थ्सृ॒जतीत्यु॑त् - सृ॒जति॑ । \newline
20. या मे॒वैव यां ॅया मे॒व । \newline
21. ए॒व मनु॒र् मनु॑ रे॒वैव मनुः॑ । \newline
22. मनु॒र्॒. ऋद्धि॒ मृद्धि॒म् मनु॒र् मनु॒र्॒. ऋद्धि᳚म् । \newline
23. ऋद्धि॒ मार्द्ध्नो॒ दार्द्ध्नो॒ दृद्धि॒ मृद्धि॒ मार्द्ध्नो᳚त् । \newline
24. आर्द्ध्नो॒त् ताम् ता मार्द्ध्नो॒ दार्द्ध्नो॒त् ताम् । \newline
25. ता मे॒वैव ताम् ता मे॒व । \newline
26. ए॒व यज॑मानो॒ यज॑मान ए॒वैव यज॑मानः । \newline
27. यज॑मान ऋद्ध्नो त्यृद्ध्नोति॒ यज॑मानो॒ यज॑मान ऋद्ध्नोति । \newline
28. ऋ॒द्ध्नो॒ति॒ य॒ज्ञ्स्य॑ य॒ज्ञ्स्य॑ र्‌द्ध्नो त्यृद्ध्नोति य॒ज्ञ्स्य॑ । \newline
29. य॒ज्ञ्स्य॒ वै वै य॒ज्ञ्स्य॑ य॒ज्ञ्स्य॒ वै । \newline
30. वा अप्र॑तिष्ठिता॒ दप्र॑तिष्ठिता॒द् वै वा अप्र॑तिष्ठितात् । \newline
31. अप्र॑तिष्ठिताद् य॒ज्ञो य॒ज्ञो ऽप्र॑तिष्ठिता॒ दप्र॑तिष्ठिताद् य॒ज्ञ्ः । \newline
32. अप्र॑तिष्ठिता॒दित्यप्र॑ति - स्थि॒ता॒त् । \newline
33. य॒ज्ञ्ः परा॒ परा॑ य॒ज्ञो य॒ज्ञ्ः परा᳚ । \newline
34. परा॑ भवति भवति॒ परा॒ परा॑ भवति । \newline
35. भ॒व॒ति॒ य॒ज्ञ्ं ॅय॒ज्ञ्म् भ॑वति भवति य॒ज्ञ्म् । \newline
36. य॒ज्ञ्म् प॑रा॒भव॑न्तम् परा॒भव॑न्तं ॅय॒ज्ञ्ं ॅय॒ज्ञ्म् प॑रा॒भव॑न्तम् । \newline
37. प॒रा॒भव॑न्तं॒ ॅयज॑मानो॒ यज॑मानः परा॒भव॑न्तम् परा॒भव॑न्तं॒ ॅयज॑मानः । \newline
38. प॒रा॒भव॑न्त॒मिति॑ परा - भव॑न्तम् । \newline
39. यज॑मा॒नो ऽन्वनु॒ यज॑मानो॒ यज॑मा॒नो ऽनु॑ । \newline
40. अनु॒ परा॒ परा ऽन्वनु॒ परा᳚ । \newline
41. परा॑ भवति भवति॒ परा॒ परा॑ भवति । \newline
42. भ॒व॒ति॒ यद् यद् भ॑वति भवति॒ यत् । \newline
43. यदाज्ये॒ना ज्ये॑न॒ यद् यदाज्ये॑न । \newline
44. आज्ये॑न पात्नीव॒तम् पा᳚त्नीव॒त माज्ये॒ना ज्ये॑न पात्नीव॒तम् । \newline
45. पा॒त्नी॒व॒तꣳ सꣳ॑स्था॒पय॑ति सꣳस्था॒पय॑ति पात्नीव॒तम् पा᳚त्नीव॒तꣳ सꣳ॑स्था॒पय॑ति । \newline
46. पा॒त्नी॒व॒तमिति॑ पात्नी - व॒तम् । \newline
47. सꣳ॒॒स्था॒पय॑ति य॒ज्ञ्स्य॑ य॒ज्ञ्स्य॑ सꣳस्था॒पय॑ति सꣳस्था॒पय॑ति य॒ज्ञ्स्य॑ । \newline
48. सꣳ॒॒स्था॒पय॒तीति॑ सं - स्था॒पय॑ति । \newline
49. य॒ज्ञ्स्य॒ प्रति॑ष्ठित्यै॒ प्रति॑ष्ठित्यै य॒ज्ञ्स्य॑ य॒ज्ञ्स्य॒ प्रति॑ष्ठित्यै । \newline
50. प्रति॑ष्ठित्यै य॒ज्ञ्ं ॅय॒ज्ञ्म् प्रति॑ष्ठित्यै॒ प्रति॑ष्ठित्यै य॒ज्ञ्म् । \newline
51. प्रति॑ष्ठित्या॒ इति॒ प्रति॑ - स्थि॒त्यै॒ । \newline
52. य॒ज्ञ्म् प्र॑ति॒तिष्ठ॑न्तम् प्रति॒तिष्ठ॑न्तं ॅय॒ज्ञ्ं ॅय॒ज्ञ्म् प्र॑ति॒तिष्ठ॑न्तम् । \newline
53. प्र॒ति॒तिष्ठ॑न्तं॒ ॅयज॑मानो॒ यज॑मानः प्रति॒तिष्ठ॑न्तम् प्रति॒तिष्ठ॑न्तं॒ ॅयज॑मानः । \newline
54. प्र॒ति॒तिष्ठ॑न्त॒मिति॑ प्रति - तिष्ठ॑न्तम् । \newline
55. यज॑मा॒नो ऽन्वनु॒ यज॑मानो॒ यज॑मा॒नो ऽनु॑ । \newline
56. अनु॒ प्रति॒ प्रत्य न्वनु॒ प्रति॑ । \newline
57. प्रति॑ तिष्ठति तिष्ठति॒ प्रति॒ प्रति॑ तिष्ठति । \newline
58. ति॒ष्ठ॒ती॒ष्ट मि॒ष्टम् ति॑ष्ठति तिष्ठती॒ष्टम् । \newline
59. इ॒ष्टं ॅव॒पया॑ व॒पये॒ष्ट मि॒ष्टं ॅव॒पया᳚ । \newline
60. व॒पया॒ भव॑ति॒ भव॑ति व॒पया॑ व॒पया॒ भव॑ति । \newline

\textbf{Ghana Paata } \newline

1. इन्द्रः॒ पत्नि॑या॒ पत्नि॒येन्द्र॒ इन्द्रः॒ पत्नि॑या॒ मनु॒म् मनु॒म् पत्नि॒येन्द्र॒ इन्द्रः॒ पत्नि॑या॒ मनु᳚म् । \newline
2. पत्नि॑या॒ मनु॒म् मनु॒म् पत्नि॑या॒ पत्नि॑या॒ मनु॑ मयाजय दयाजय॒न् मनु॒म् पत्नि॑या॒ पत्नि॑या॒ मनु॑ मयाजयत् । \newline
3. मनु॑ मयाजय दयाजय॒न् मनु॒म् मनु॑ मयाजय॒त् ताम् ता म॑याजय॒न् मनु॒म् मनु॑ मयाजय॒त् ताम् । \newline
4. अ॒या॒ज॒य॒त् ताम् ता म॑याजय दयाजय॒त् ताम् पर्य॑ग्निकृता॒म् पर्य॑ग्निकृता॒म् ता म॑याजय दयाजय॒त् ताम् पर्य॑ग्निकृताम् । \newline
5. ताम् पर्य॑ग्निकृता॒म् पर्य॑ग्निकृता॒म् ताम् ताम् पर्य॑ग्निकृता॒ मुदुत् पर्य॑ग्निकृता॒म् ताम् ताम् पर्य॑ग्निकृता॒ मुत् । \newline
6. पर्य॑ग्निकृता॒ मुदुत् पर्य॑ग्निकृता॒म् पर्य॑ग्निकृता॒ मुद॑सृज दसृज॒दुत् पर्य॑ग्निकृता॒म् पर्य॑ग्निकृता॒ मुद॑सृजत् । \newline
7. पर्य॑ग्निकृता॒मिति॒ पर्य॑ग्नि - कृ॒ता॒म् । \newline
8. उद॑सृज दसृज॒ दुदु द॑सृज॒त् तया॒ तया॑ ऽसृज॒ दुदु द॑सृज॒त् तया᳚ । \newline
9. अ॒सृ॒ज॒त् तया॒ तया॑ ऽसृज दसृज॒त् तया॒ मनु॒र् मनु॒ स्तया॑ ऽसृज दसृज॒त् तया॒ मनुः॑ । \newline
10. तया॒ मनु॒र् मनु॒ स्तया॒ तया॒ मनु॑ रार्द्ध्नो दार्द्ध्नो॒न् मनु॒ स्तया॒ तया॒ मनु॑ रार्द्ध्नोत् । \newline
11. मनु॑ रार्द्ध्नो दार्द्ध्नो॒न् मनु॒र् मनु॑ रार्द्ध्नो॒द् यद् यदा᳚र्द्ध्नो॒न् मनु॒र् मनु॑ रार्द्ध्नो॒द् यत् । \newline
12. आ॒र्द्ध्नो॒द् यद् यदा᳚र्द्ध्नो दार्द्ध्नो॒द् यत् पर्य॑ग्निकृत॒म् पर्य॑ग्निकृतं॒ ॅयदा᳚र्द्ध्नो दार्द्ध्नो॒द् यत् पर्य॑ग्निकृतम् । \newline
13. यत् पर्य॑ग्निकृत॒म् पर्य॑ग्निकृतं॒ ॅयद् यत् पर्य॑ग्निकृतम् पात्नीव॒तम् पा᳚त्नीव॒तम् पर्य॑ग्निकृतं॒ ॅयद् यत् पर्य॑ग्निकृतम् पात्नीव॒तम् । \newline
14. पर्य॑ग्निकृतम् पात्नीव॒तम् पा᳚त्नीव॒तम् पर्य॑ग्निकृत॒म् पर्य॑ग्निकृतम् पात्नीव॒त मु॑थ्सृ॒ज त्यु॑थ्सृ॒जति॑ पात्नीव॒तम् पर्य॑ग्निकृत॒म् पर्य॑ग्निकृतम् पात्नीव॒त मु॑थ्सृ॒जति॑ । \newline
15. पर्य॑ग्निकृत॒मिति॒ पर्य॑ग्नि - कृ॒त॒म् । \newline
16. पा॒त्नी॒व॒त मु॑थ्सृ॒ज त्यु॑थ्सृ॒जति॑ पात्नीव॒तम् पा᳚त्नीव॒त मु॑थ्सृ॒जति॒ यां ॅया मु॑थ्सृ॒जति॑ पात्नीव॒तम् पा᳚त्नीव॒त मु॑थ्सृ॒जति॒ याम् । \newline
17. पा॒त्नी॒व॒तमिति॑ पात्नी - व॒तम् । \newline
18. उ॒थ्सृ॒जति॒ यां ॅया मु॑थ्सृ॒ज त्यु॑थ्सृ॒जति॒ या मे॒वैव या मु॑थ्सृ॒ज त्यु॑थ्सृ॒जति॒ या मे॒व । \newline
19. उ॒थ्सृ॒जतीत्यु॑त् - सृ॒जति॑ । \newline
20. या मे॒वैव यां ॅया मे॒व मनु॒र् मनु॑ रे॒व यां ॅया मे॒व मनुः॑ । \newline
21. ए॒व मनु॒र् मनु॑ रे॒वैव मनु॒र्॒. ऋद्धि॒ मृद्धि॒म् मनु॑ रे॒वैव मनु॒र्॒. ऋद्धि᳚म् । \newline
22. मनु॒र्॒.ऋद्धि॒ मृद्धि॒म् मनु॒र् मनु॒र्॒. ऋद्धि॒ मार्द्ध्नो॒ दार्द्ध्नो॒ दृद्धि॒म् मनु॒र् मनु॒र्॒. ऋद्धि॒ मार्द्ध्नो᳚त् । \newline
23. ऋद्धि॒ मार्द्ध्नो॒ दार्द्ध्नो॒ दृद्धि॒ मृद्धि॒ मार्द्ध्नो॒त् ताम् ता मार्द्ध्नो॒ दृद्धि॒ मृद्धि॒ मार्द्ध्नो॒त् ताम् । \newline
24. आर्द्ध्नो॒त् ताम् ता मार्द्ध्नो॒ दार्द्ध्नो॒त् ता मे॒वैव ता मार्द्ध्नो॒ दार्द्ध्नो॒त् ता मे॒व । \newline
25. ता मे॒वैव ताम् ता मे॒व यज॑मानो॒ यज॑मान ए॒व ताम् ता मे॒व यज॑मानः । \newline
26. ए॒व यज॑मानो॒ यज॑मान ए॒वैव यज॑मान ऋद्ध्नो त्यृद्ध्नोति॒ यज॑मान ए॒वैव यज॑मान ऋद्ध्नोति । \newline
27. यज॑मान ऋद्ध्नो त्यृद्ध्नोति॒ यज॑मानो॒ यज॑मान ऋद्ध्नोति य॒ज्ञ्स्य॑ य॒ज्ञ्स्य॑ र्‌द्ध्नोति॒ यज॑मानो॒ यज॑मान ऋद्ध्नोति य॒ज्ञ्स्य॑ । \newline
28. ऋ॒द्ध्नो॒ति॒ य॒ज्ञ्स्य॑ य॒ज्ञ्स्य॑ र्‌द्ध्नो त्यृद्ध्नोति य॒ज्ञ्स्य॒ वै वै य॒ज्ञ्स्य॑ र्‌द्ध्नो त्यृद्ध्नोति य॒ज्ञ्स्य॒ वै । \newline
29. य॒ज्ञ्स्य॒ वै वै य॒ज्ञ्स्य॑ य॒ज्ञ्स्य॒ वा अप्र॑तिष्ठिता॒ दप्र॑तिष्ठिता॒द् वै य॒ज्ञ्स्य॑ य॒ज्ञ्स्य॒ वा अप्र॑तिष्ठितात् । \newline
30. वा अप्र॑तिष्ठिता॒ दप्र॑तिष्ठिता॒द् वै वा अप्र॑तिष्ठिताद् य॒ज्ञो य॒ज्ञो ऽप्र॑तिष्ठिता॒द् वै वा अप्र॑तिष्ठिताद् य॒ज्ञ्ः । \newline
31. अप्र॑तिष्ठिताद् य॒ज्ञो य॒ज्ञो ऽप्र॑तिष्ठिता॒ दप्र॑तिष्ठिताद् य॒ज्ञ्ः परा॒ परा॑ य॒ज्ञो ऽप्र॑तिष्ठिता॒ दप्र॑तिष्ठिताद् य॒ज्ञ्ः परा᳚ । \newline
32. अप्र॑तिष्ठिता॒दित्यप्र॑ति - स्थि॒ता॒त् । \newline
33. य॒ज्ञ्ः परा॒ परा॑ य॒ज्ञो य॒ज्ञ्ः परा॑ भवति भवति॒ परा॑ य॒ज्ञो य॒ज्ञ्ः परा॑ भवति । \newline
34. परा॑ भवति भवति॒ परा॒ परा॑ भवति य॒ज्ञ्ं ॅय॒ज्ञ्म् भ॑वति॒ परा॒ परा॑ भवति य॒ज्ञ्म् । \newline
35. भ॒व॒ति॒ य॒ज्ञ्ं ॅय॒ज्ञ्म् भ॑वति भवति य॒ज्ञ्म् प॑रा॒भव॑न्तम् परा॒भव॑न्तं ॅय॒ज्ञ्म् भ॑वति भवति य॒ज्ञ्म् प॑रा॒भव॑न्तम् । \newline
36. य॒ज्ञ्म् प॑रा॒भव॑न्तम् परा॒भव॑न्तं ॅय॒ज्ञ्ं ॅय॒ज्ञ्म् प॑रा॒भव॑न्तं॒ ॅयज॑मानो॒ यज॑मानः परा॒भव॑न्तं ॅय॒ज्ञ्ं ॅय॒ज्ञ्म् प॑रा॒भव॑न्तं॒ ॅयज॑मानः । \newline
37. प॒रा॒भव॑न्तं॒ ॅयज॑मानो॒ यज॑मानः परा॒भव॑न्तम् परा॒भव॑न्तं॒ ॅयज॑मा॒नो ऽन्वनु॒ यज॑मानः परा॒भव॑न्तम् परा॒भव॑न्तं॒ ॅयज॑मा॒नो ऽनु॑ । \newline
38. प॒रा॒भव॑न्त॒मिति॑ परा - भव॑न्तम् । \newline
39. यज॑मा॒नो ऽन्वनु॒ यज॑मानो॒ यज॑मा॒नो ऽनु॒ परा॒ परा ऽनु॒ यज॑मानो॒ यज॑मा॒नो ऽनु॒ परा᳚ । \newline
40. अनु॒ परा॒ परा ऽन्वनु॒ परा॑ भवति भवति॒ परा ऽन्वनु॒ परा॑ भवति । \newline
41. परा॑ भवति भवति॒ परा॒ परा॑ भवति॒ यद् यद् भ॑वति॒ परा॒ परा॑ भवति॒ यत् । \newline
42. भ॒व॒ति॒ यद् यद् भ॑वति भवति॒ यदाज्ये॒ नाज्ये॑न॒ यद् भ॑वति भवति॒ यदाज्ये॑न । \newline
43. यदाज्ये॒ नाज्ये॑न॒ यद् यदाज्ये॑न पात्नीव॒तम् पा᳚त्नीव॒त माज्ये॑न॒ यद् यदाज्ये॑न पात्नीव॒तम् । \newline
44. आज्ये॑न पात्नीव॒तम् पा᳚त्नीव॒त माज्ये॒ नाज्ये॑न पात्नीव॒तꣳ सꣳ॑स्था॒पय॑ति सꣳस्था॒पय॑ति पात्नीव॒त माज्ये॒ नाज्ये॑न पात्नीव॒तꣳ सꣳ॑स्था॒पय॑ति । \newline
45. पा॒त्नी॒व॒तꣳ सꣳ॑स्था॒पय॑ति सꣳस्था॒पय॑ति पात्नीव॒तम् पा᳚त्नीव॒तꣳ सꣳ॑स्था॒पय॑ति य॒ज्ञ्स्य॑ य॒ज्ञ्स्य॑ सꣳस्था॒पय॑ति पात्नीव॒तम् पा᳚त्नीव॒तꣳ सꣳ॑स्था॒पय॑ति य॒ज्ञ्स्य॑ । \newline
46. पा॒त्नी॒व॒तमिति॑ पात्नी - व॒तम् । \newline
47. सꣳ॒॒स्था॒पय॑ति य॒ज्ञ्स्य॑ य॒ज्ञ्स्य॑ सꣳस्था॒पय॑ति सꣳस्था॒पय॑ति य॒ज्ञ्स्य॒ प्रति॑ष्ठित्यै॒ प्रति॑ष्ठित्यै य॒ज्ञ्स्य॑ सꣳस्था॒पय॑ति सꣳस्था॒पय॑ति य॒ज्ञ्स्य॒ प्रति॑ष्ठित्यै । \newline
48. सꣳ॒॒स्था॒पय॒तीति॑ सं - स्था॒पय॑ति । \newline
49. य॒ज्ञ्स्य॒ प्रति॑ष्ठित्यै॒ प्रति॑ष्ठित्यै य॒ज्ञ्स्य॑ य॒ज्ञ्स्य॒ प्रति॑ष्ठित्यै य॒ज्ञ्ं ॅय॒ज्ञ्म् प्रति॑ष्ठित्यै य॒ज्ञ्स्य॑ य॒ज्ञ्स्य॒ प्रति॑ष्ठित्यै य॒ज्ञ्म् । \newline
50. प्रति॑ष्ठित्यै य॒ज्ञ्ं ॅय॒ज्ञ्म् प्रति॑ष्ठित्यै॒ प्रति॑ष्ठित्यै य॒ज्ञ्म् प्र॑ति॒तिष्ठ॑न्तम् प्रति॒तिष्ठ॑न्तं ॅय॒ज्ञ्म् प्रति॑ष्ठित्यै॒ प्रति॑ष्ठित्यै य॒ज्ञ्म् प्र॑ति॒तिष्ठ॑न्तम् । \newline
51. प्रति॑ष्ठित्या॒ इति॒ प्रति॑ - स्थि॒त्यै॒ । \newline
52. य॒ज्ञ्म् प्र॑ति॒तिष्ठ॑न्तम् प्रति॒तिष्ठ॑न्तं ॅय॒ज्ञ्ं ॅय॒ज्ञ्म् प्र॑ति॒तिष्ठ॑न्तं॒ ॅयज॑मानो॒ यज॑मानः प्रति॒तिष्ठ॑न्तं ॅय॒ज्ञ्ं ॅय॒ज्ञ्म् प्र॑ति॒तिष्ठ॑न्तं॒ ॅयज॑मानः । \newline
53. प्र॒ति॒तिष्ठ॑न्तं॒ ॅयज॑मानो॒ यज॑मानः प्रति॒तिष्ठ॑न्तम् प्रति॒तिष्ठ॑न्तं॒ ॅयज॑मा॒नो ऽन्वनु॒ यज॑मानः प्रति॒तिष्ठ॑न्तम् प्रति॒तिष्ठ॑न्तं॒ ॅयज॑मा॒नो ऽनु॑ । \newline
54. प्र॒ति॒तिष्ठ॑न्त॒मिति॑ प्रति - तिष्ठ॑न्तम् । \newline
55. यज॑मा॒नो ऽन्वनु॒ यज॑मानो॒ यज॑मा॒नो ऽनु॒ प्रति॒ प्रत्यनु॒ यज॑मानो॒ यज॑मा॒नो ऽनु॒ प्रति॑ । \newline
56. अनु॒ प्रति॒ प्रत्यन् वनु॒ प्रति॑ तिष्ठति तिष्ठति॒ प्रत्यन् वनु॒ प्रति॑ तिष्ठति । \newline
57. प्रति॑ तिष्ठति तिष्ठति॒ प्रति॒ प्रति॑ तिष्ठती॒ष्ट मि॒ष्टम् ति॑ष्ठति॒ प्रति॒ प्रति॑ तिष्ठती॒ष्टम् । \newline
58. ति॒ष्ठ॒ती॒ष्ट मि॒ष्टम् ति॑ष्ठति तिष्ठती॒ष्टं ॅव॒पया॑ व॒पये॒ष्टम् ति॑ष्ठति तिष्ठती॒ष्टं ॅव॒पया᳚ । \newline
59. इ॒ष्टं ॅव॒पया॑ व॒पये॒ष्ट मि॒ष्टं ॅव॒पया॒ भव॑ति॒ भव॑ति व॒पये॒ष्ट मि॒ष्टं ॅव॒पया॒ भव॑ति । \newline
60. व॒पया॒ भव॑ति॒ भव॑ति व॒पया॑ व॒पया॒ भव॒ त्यनि॑ष्ट॒ मनि॑ष्ट॒म् भव॑ति व॒पया॑ व॒पया॒ भव॒ त्यनि॑ष्टम् । \newline
\pagebreak
\markright{ TS 6.6.6.2  \hfill https://www.vedavms.in \hfill}

\section{ TS 6.6.6.2 }

\textbf{TS 6.6.6.2 } \newline
\textbf{Samhita Paata} \newline

भव॒त्यनि॑ष्टं ॅव॒शयाऽथ॑ पात्नीव॒तेन॒ प्र च॑रति ती॒र्त्थ ए॒व प्र च॑र॒त्यथो॑ ए॒तर्ह्ये॒वास्य॒ याम॑स्त्वा॒ष्ट्रो भ॑वति॒ त्वष्टा॒ वै रेत॑सः सि॒क्तस्य॑ रू॒पाणि॒ वि क॑रोति॒ तमे॒व वृषा॑णं॒ पत्नी॒ष्वपि॑ सृजति॒ सो᳚ऽस्मै रू॒पाणि॒ वि क॑रोति ॥ \newline

\textbf{Pada Paata} \newline

भव॑ति । अनि॑ष्टम् । व॒शया᳚ । अथ॑ । पा॒त्नी॒व॒तेनेति॑ पात्नी - व॒तेन॑ । प्रेति॑ । च॒र॒ति॒ । ती॒र्त्थ । ए॒व । प्रेति॑ । च॒र॒ति॒ । अथो॒ इति॑ । ए॒तर्.हि॑ । ए॒व । अ॒स्य॒ । यामः॑ । त्वा॒ष्ट्रः । भ॒व॒ति॒ । त्वष्टा᳚ । वै । रेत॑सः । सि॒क्तस्य॑ । रू॒पाणि॑ । वीति॑ । क॒रो॒ति॒ । तम् । ए॒व । वृषा॑णम् । पत्नी॑षु । अपीति॑ । सृ॒ज॒ति॒ । सः । अ॒स्मै॒ । रू॒पाणि॑ । वीति॑ । क॒रो॒ति॒ ॥  \newline


\textbf{Krama Paata} \newline

भव॒त्यनि॑ष्टम् । अनि॑ष्टम् ॅव॒शया᳚ । व॒शयाऽथ॑ । अथ॑ पात्नीव॒तेन॑ । पा॒त्नी॒व॒तेन॒ प्र । पा॒त्नी॒व॒तेनेति॑ पात्नी - व॒तेन॑ । प्र च॑रति । च॒र॒ति॒ ती॒र्थे । ती॒र्थ ए॒व । ए॒व प्र । प्र च॑रति । च॒र॒त्यथो᳚ । अथो॑ ए॒तर्.हि॑ । अथो॒ इत्यथो᳚ । ए॒तर्ह्ये॒व । ए॒वास्य॑ । अ॒स्य॒ यामः॑ । याम॑स्त्वा॒ष्ट्रः । त्वा॒ष्ट्रो भ॑वति । भ॒व॒ति॒ त्वष्टा᳚ । त्वष्टा॒ वै । वै रेत॑सः । रेत॑सः सि॒क्तस्य॑ । सि॒क्तस्य॑ रू॒पाणि॑ । रू॒पाणि॒ वि । वि क॑रोति । क॒रो॒ति॒ तम् । तमे॒व । ए॒व वृषा॑णम् । वृषा॑ण॒म् पत्नी॑षु । पत्नी॒ष्वपि॑ । अपि॑ सृजति । सृ॒ज॒ति॒ सः । सो᳚ऽस्मै । अ॒स्मै॒ रू॒पाणि॑ । रू॒पाणि॒ वि । वि क॑रोति । क॒रो॒तीति॑ करोति । \newline

\textbf{Jatai Paata} \newline

1. भव॒ त्यनि॑ष्ट॒ मनि॑ष्ट॒म् भव॑ति॒ भव॒ त्यनि॑ष्टम् । \newline
2. अनि॑ष्टं ॅव॒शया॑ व॒शया ऽनि॑ष्ट॒ मनि॑ष्टं ॅव॒शया᳚ । \newline
3. व॒शया ऽथाथ॑ व॒शया॑ व॒शया ऽथ॑ । \newline
4. अथ॑ पात्नीव॒तेन॑ पात्नीव॒तेना थाथ॑ पात्नीव॒तेन॑ । \newline
5. पा॒त्नी॒व॒तेन॒ प्र प्र पा᳚त्नीव॒तेन॑ पात्नीव॒तेन॒ प्र । \newline
6. पा॒त्नी॒व॒तेनेति॑ पात्नी - व॒तेन॑ । \newline
7. प्र च॑रति चरति॒ प्र प्र च॑रति । \newline
8. च॒र॒ति॒ ती॒र्त्थे ती॒र्त्थे च॑रति चरति ती॒र्त्थे । \newline
9. ती॒र्त्थ ए॒वैव ती॒र्त्थे ती॒र्त्थ ए॒व । \newline
10. ए॒व प्र प्रैवैव प्र । \newline
11. प्र च॑रति चरति॒ प्र प्र च॑रति । \newline
12. च॒र॒ त्यथो॒ अथो॑ चरति चर॒ त्यथो᳚ । \newline
13. अथो॑ ए॒तर् ह्ये॒तर् ह्यथो॒ अथो॑ ए॒तर्.हि॑ । \newline
14. अथो॒ इत्यथो᳚ । \newline
15. ए॒तर् ह्ये॒वै वैतर् ह्ये॒तर् ह्ये॒व । \newline
16. ए॒वास्या᳚ स्यै॒वै वास्य॑ । \newline
17. अ॒स्य॒ यामो॒ यामो᳚ ऽस्यास्य॒ यामः॑ । \newline
18. याम॑ स्त्वा॒ष्ट्र स्त्वा॒ष्ट्रो यामो॒ याम॑ स्त्वा॒ष्ट्रः । \newline
19. त्वा॒ष्ट्रो भ॑वति भवति त्वा॒ष्ट्र स्त्वा॒ष्ट्रो भ॑वति । \newline
20. भ॒व॒ति॒ त्वष्टा॒ त्वष्टा॑ भवति भवति॒ त्वष्टा᳚ । \newline
21. त्वष्टा॒ वै वै त्वष्टा॒ त्वष्टा॒ वै । \newline
22. वै रेत॑सो॒ रेत॑सो॒ वै वै रेत॑सः । \newline
23. रेत॑सः सि॒क्तस्य॑ सि॒क्तस्य॒ रेत॑सो॒ रेत॑सः सि॒क्तस्य॑ । \newline
24. सि॒क्तस्य॑ रू॒पाणि॑ रू॒पाणि॑ सि॒क्तस्य॑ सि॒क्तस्य॑ रू॒पाणि॑ । \newline
25. रू॒पाणि॒ वि वि रू॒पाणि॑ रू॒पाणि॒ वि । \newline
26. वि क॑रोति करोति॒ वि वि क॑रोति । \newline
27. क॒रो॒ति॒ तम् तम् क॑रोति करोति॒ तम् । \newline
28. त मे॒वैव तम् त मे॒व । \newline
29. ए॒व वृषा॑णं॒ ॅवृषा॑ण मे॒वैव वृषा॑णम् । \newline
30. वृषा॑ण॒म् पत्नी॑षु॒ पत्नी॑षु॒ वृषा॑णं॒ ॅवृषा॑ण॒म् पत्नी॑षु । \newline
31. पत्नी॒ष् वप्यपि॒ पत्नी॑षु॒ पत्नी॒ ष्वपि॑ । \newline
32. अपि॑ सृजति सृज॒ त्यप्यपि॑ सृजति । \newline
33. सृ॒ज॒ति॒ स स सृ॑जति सृजति॒ सः । \newline
34. सो᳚ ऽस्मा अस्मै॒ स सो᳚ ऽस्मै । \newline
35. अ॒स्मै॒ रू॒पाणि॑ रू॒पा ण्य॑स्मा अस्मै रू॒पाणि॑ । \newline
36. रू॒पाणि॒ वि वि रू॒पाणि॑ रू॒पाणि॒ वि । \newline
37. वि क॑रोति करोति॒ वि वि क॑रोति । \newline
38. क॒रो॒तीति॑ करोति । \newline

\textbf{Ghana Paata } \newline

1. भव॒ त्यनि॑ष्ट॒ मनि॑ष्ट॒म् भव॑ति॒ भव॒ त्यनि॑ष्टं ॅव॒शया॑ व॒शया ऽनि॑ष्ट॒म् भव॑ति॒ भव॒ त्यनि॑ष्टं ॅव॒शया᳚ । \newline
2. अनि॑ष्टं ॅव॒शया॑ व॒शया ऽनि॑ष्ट॒ मनि॑ष्टं ॅव॒शया ऽथाथ॑ व॒शया ऽनि॑ष्ट॒ मनि॑ष्टं ॅव॒शया ऽथ॑ । \newline
3. व॒शया ऽथाथ॑ व॒शया॑ व॒शया ऽथ॑ पात्नीव॒तेन॑ पात्नीव॒तेनाथ॑ व॒शया॑ व॒शया ऽथ॑ पात्नीव॒तेन॑ । \newline
4. अथ॑ पात्नीव॒तेन॑ पात्नीव॒तेना थाथ॑ पात्नीव॒तेन॒ प्र प्र पा᳚त्नीव॒तेना थाथ॑ पात्नीव॒तेन॒ प्र । \newline
5. पा॒त्नी॒व॒तेन॒ प्र प्र पा᳚त्नीव॒तेन॑ पात्नीव॒तेन॒ प्र च॑रति चरति॒ प्र पा᳚त्नीव॒तेन॑ पात्नीव॒तेन॒ प्र च॑रति । \newline
6. पा॒त्नी॒व॒तेनेति॑ पात्नी - व॒तेन॑ । \newline
7. प्र च॑रति चरति॒ प्र प्र च॑रति ती॒र्त्थे ती॒र्त्थे च॑रति॒ प्र प्र च॑रति ती॒र्त्थे । \newline
8. च॒र॒ति॒ ती॒र्त्थे ती॒र्त्थे च॑रति चरति ती॒र्त्थ ए॒वैव ती॒र्त्थे च॑रति चरति ती॒र्त्थ ए॒व । \newline
9. ती॒र्त्थ ए॒वैव ती॒र्त्थे ती॒र्त्थ ए॒व प्र प्रैव ती॒र्त्थे ती॒र्त्थ ए॒व प्र । \newline
10. ए॒व प्र प्रैवैव प्र च॑रति चरति॒ प्रैवैव प्र च॑रति । \newline
11. प्र च॑रति चरति॒ प्र प्र च॑र॒ त्यथो॒ अथो॑ चरति॒ प्र प्र च॑र॒ त्यथो᳚ । \newline
12. च॒र॒ त्यथो॒ अथो॑ चरति चर॒ त्यथो॑ ए॒तर् ह्ये॒तर् ह्यथो॑ चरति चर॒ त्यथो॑ ए॒तर्.हि॑ । \newline
13. अथो॑ ए॒तर् ह्ये॒तर् ह्यथो॒ अथो॑ ए॒तर् ह्ये॒वै वैतर् ह्यथो॒ अथो॑ ए॒तर् ह्ये॒व । \newline
14. अथो॒ इत्यथो᳚ । \newline
15. ए॒तर् ह्ये॒वै वैतर् ह्ये॒तर् ह्ये॒वा स्या᳚स्यै॒ वैतर् ह्ये॒तर् ह्ये॒वास्य॑ । \newline
16. ए॒वास्या᳚ स्यै॒वैवास्य॒ यामो॒ यामो᳚ ऽस्यै॒वैवास्य॒ यामः॑ । \newline
17. अ॒स्य॒ यामो॒ यामो᳚ ऽस्यास्य॒ याम॑ स्त्वा॒ष्ट्र स्त्वा॒ष्ट्रो यामो᳚ ऽस्यास्य॒ याम॑ स्त्वा॒ष्ट्रः । \newline
18. याम॑ स्त्वा॒ष्ट्र स्त्वा॒ष्ट्रो यामो॒ याम॑ स्त्वा॒ष्ट्रो भ॑वति भवति त्वा॒ष्ट्रो यामो॒ याम॑ स्त्वा॒ष्ट्रो भ॑वति । \newline
19. त्वा॒ष्ट्रो भ॑वति भवति त्वा॒ष्ट्र स्त्वा॒ष्ट्रो भ॑वति॒ त्वष्टा॒ त्वष्टा॑ भवति त्वा॒ष्ट्र स्त्वा॒ष्ट्रो भ॑वति॒ त्वष्टा᳚ । \newline
20. भ॒व॒ति॒ त्वष्टा॒ त्वष्टा॑ भवति भवति॒ त्वष्टा॒ वै वै त्वष्टा॑ भवति भवति॒ त्वष्टा॒ वै । \newline
21. त्वष्टा॒ वै वै त्वष्टा॒ त्वष्टा॒ वै रेत॑सो॒ रेत॑सो॒ वै त्वष्टा॒ त्वष्टा॒ वै रेत॑सः । \newline
22. वै रेत॑सो॒ रेत॑सो॒ वै वै रेत॑सः सि॒क्तस्य॑ सि॒क्तस्य॒ रेत॑सो॒ वै वै रेत॑सः सि॒क्तस्य॑ । \newline
23. रेत॑सः सि॒क्तस्य॑ सि॒क्तस्य॒ रेत॑सो॒ रेत॑सः सि॒क्तस्य॑ रू॒पाणि॑ रू॒पाणि॑ सि॒क्तस्य॒ रेत॑सो॒ रेत॑सः सि॒क्तस्य॑ रू॒पाणि॑ । \newline
24. सि॒क्तस्य॑ रू॒पाणि॑ रू॒पाणि॑ सि॒क्तस्य॑ सि॒क्तस्य॑ रू॒पाणि॒ वि वि रू॒पाणि॑ सि॒क्तस्य॑ सि॒क्तस्य॑ रू॒पाणि॒ वि । \newline
25. रू॒पाणि॒ वि वि रू॒पाणि॑ रू॒पाणि॒ वि क॑रोति करोति॒ वि रू॒पाणि॑ रू॒पाणि॒ वि क॑रोति । \newline
26. वि क॑रोति करोति॒ वि वि क॑रोति॒ तम् तम् क॑रोति॒ वि वि क॑रोति॒ तम् । \newline
27. क॒रो॒ति॒ तम् तम् क॑रोति करोति॒ तमे॒वैव तम् क॑रोति करोति॒ तमे॒व । \newline
28. तमे॒वैव तम् तमे॒व वृषा॑णं॒ ॅवृषा॑ण मे॒व तम् तमे॒व वृषा॑णम् । \newline
29. ए॒व वृषा॑णं॒ ॅवृषा॑ण मे॒वैव वृषा॑ण॒म् पत्नी॑षु॒ पत्नी॑षु॒ वृषा॑ण मे॒वैव वृषा॑ण॒म् पत्नी॑षु । \newline
30. वृषा॑ण॒म् पत्नी॑षु॒ पत्नी॑षु॒ वृषा॑णं॒ ॅवृषा॑ण॒म् पत्नी॒ ष्वप्यपि॒ पत्नी॑षु॒ वृषा॑णं॒ ॅवृषा॑ण॒म् पत्नी॒ ष्वपि॑ । \newline
31. पत्नी॒ ष्वप्यपि॒ पत्नी॑षु॒ पत्नी॒ ष्वपि॑ सृजति सृज॒ त्यपि॒ पत्नी॑षु॒ पत्नी॒ ष्वपि॑ सृजति । \newline
32. अपि॑ सृजति सृज॒ त्यप्यपि॑ सृजति॒ स स सृ॑ज॒ त्यप्यपि॑ सृजति॒ सः । \newline
33. सृ॒ज॒ति॒ स स सृ॑जति सृजति॒ सो᳚ ऽस्मा अस्मै॒ स सृ॑जति सृजति॒ सो᳚ ऽस्मै । \newline
34. सो᳚ ऽस्मा अस्मै॒ स सो᳚ ऽस्मै रू॒पाणि॑ रू॒पा ण्य॑स्मै॒ स सो᳚ ऽस्मै रू॒पाणि॑ । \newline
35. अ॒स्मै॒ रू॒पाणि॑ रू॒पा ण्य॑स्मा अस्मै रू॒पाणि॒ वि वि रू॒पा ण्य॑स्मा अस्मै रू॒पाणि॒ वि । \newline
36. रू॒पाणि॒ वि वि रू॒पाणि॑ रू॒पाणि॒ वि क॑रोति करोति॒ वि रू॒पाणि॑ रू॒पाणि॒ वि क॑रोति । \newline
37. वि क॑रोति करोति॒ वि वि क॑रोति । \newline
38. क॒रो॒तीति॑ करोति । \newline
\pagebreak
\markright{ TS 6.6.7.1  \hfill https://www.vedavms.in \hfill}

\section{ TS 6.6.7.1 }

\textbf{TS 6.6.7.1 } \newline
\textbf{Samhita Paata} \newline

घ्नन्ति॒ वा ए॒तथ् सोमं॒ ॅयद॑भिषु॒ण्वन्ति॒ यथ् सौ॒म्यो भव॑ति॒ यथा॑ मृ॒ताया॑नु॒स्तर॑णीं॒ घ्नन्ति॑ ता॒दृगे॒व तद् यदु॑त्तरा॒र्द्धे वा॒ मद्ध्ये॑ वा जुहु॒याद्-दे॒वता᳚भ्यः स॒मदं॑ दद्ध्याद्-दक्षिणा॒र्द्धे जु॑होत्ये॒षा वै पि॑तृ॒णां दिख् स्वाया॑मे॒व दि॒शि पि॒तॄन् नि॒रव॑दयत उद्गा॒तृभ्यो॑ हरन्ति सामदेव॒त्यो॑ वै सौ॒म्यो यदे॒व साम्नः॑ छंबट्कु॒र्वन्ति॒ तस्यै॒व स शान्ति॒रवे᳚- [  ] \newline

\textbf{Pada Paata} \newline

घ्नन्ति॑ । वा । ए॒तत् । सोम᳚म् । यत् । अ॒भि॒षु॒ण्वन्तीत्य॑भि-सु॒न्वन्ति॑ । यत् । सौ॒म्यः । भव॑ति । यथा᳚ । मृ॒ताय॑ । अ॒नु॒स्तर॑णी॒मित्य॑नु - स्तर॑णीम् । घ्नन्ति॑ । ता॒दृक् । ए॒व । तत् । यत् । उ॒त्त॒रा॒द्‌र्ध इत्यु॑त्तर - अ॒द्‌र्धे । वा॒ । मद्ध्ये᳚ । वा॒ । जु॒हु॒यात् । दे॒वता᳚भ्यः । स॒मद॒मिति॑ स - मद᳚म् । द॒द्ध्या॒त् । द॒क्षि॒णा॒द्‌र्ध इति॑ दक्षिण - अ॒द्‌र्धे । जु॒हो॒ति॒ । ए॒षा । वै । पि॒तृ॒णाम् । दिक् । स्वाया᳚म् । ए॒व । दि॒शि । पि॒तॄन् । नि॒रव॑दयत॒ इति॑ निः - अव॑दयते । उ॒द्गा॒तृभ्य॒ इत्यु॑द्गा॒तृ - भ्यः॒ । ह॒र॒न्ति॒ । सा॒म॒दे॒व॒त्य॑ इति॑ साम - दे॒व॒त्यः॑ । वै । सौ॒म्यः । यत् । ए॒व । साम्नः॑ । छ॒बं॒ट्कु॒र्वन्तीति॑ छंबट् - कु॒र्वन्ति॑ । तस्य॑ । ए॒व । सः । शान्तिः॑ । अवेति॑ ।  \newline


\textbf{Krama Paata} \newline

घ्नन्ति॒ वै । वा ए॒तत् । ए॒तथ् सोम᳚म् । सोम॒म् ॅयत् । यद॑भिषु॒ण्वन्ति॑ । अ॒भि॒षु॒ण्वन्ति॒ यत् । अ॒भि॒षु॒ण्वन्तीत्य॑भि - सु॒न्वन्ति॑ । यथ् सौ॒म्यः । सौ॒म्यो भव॑ति । भव॑ति॒ यथा᳚ । यथा॑ मृ॒ताय॑ । मृ॒ताया॑नु॒स्तर॑णीम् । अ॒नु॒स्तर॑णी॒म्(1) घ्नन्ति॑ । अ॒नु॒स्तर॑णी॒मित्य॑नु - स्तर॑णीम् । घ्नन्ति॑ ता॒दृक् । ता॒दृगे॒व । ए॒व तत् । तद् यत् । यदु॑त्तरा॒र्द्धे । उ॒त्त॒रा॒र्द्धे वा᳚ । उ॒त्त॒रा॒र्द्ध इत्यु॑त्तर - अ॒र्द्धे । वा॒ मद्ध्ये᳚ । मद्ध्ये॑ वा । वा॒ जु॒हु॒यात् । जु॒हु॒याद् दे॒वता᳚भ्यः । दे॒वता᳚भ्यः स॒मद᳚म् । स॒मद॑म् दद्ध्यात् । स॒मद॒मिति॑ स - मद᳚म् । द॒द्ध्या॒द् द॒क्षि॒णा॒र्द्धे । द॒क्षि॒णा॒र्द्धे जु॑होति । द॒क्षि॒णा॒र्द्ध इति॑ दक्षिण - अ॒र्द्धे । जु॒हो॒त्ये॒षा । ए॒षा वै । वै पि॑तृ॒णाम् । पि॒तृ॒णाम् दिक् । दिख् स्वाया᳚म् । स्वाया॑मे॒व । ए॒व दि॒शि । दि॒शि पि॒तॄन् । पि॒तॄन् नि॒रव॑दयते । नि॒रव॑दयत उद्ग्रा॒तृभ्यः॑ । नि॒रव॑दयत॒ इति॑ निः - अव॑दयते । उ॒द्‍गा॒तृभ्यो॑ हरन्ति । उ॒द्‍गा॒तृभ्य॒ इत्यु॑द्‍गा॒तृ - भ्यः॒ । ह॒र॒न्ति॒ सा॒म॒दे॒व॒त्यः॑ । सा॒म॒दे॒व॒त्यो॑ वै । सा॒म॒दे॒व॒त्य॑ इति॑ साम - दे॒व॒त्यः॑ । वै सौ॒म्यः । सौ॒म्यो यत् । यदे॒व । ए॒व साम्नः॑ । साम्न॑श्छम्बट्कु॒र्वन्ति॑ । छ॒म्ब॒ट्कु॒र्वन्ति॒ तस्य॑ । छ॒म्ब॒ट्कु॒र्वन्तीति॑ छम्बट् - कु॒र्वन्ति॑ । तस्यै॒व । ए॒व सः । स शान्तिः॑ । शान्ति॒रव॑ । अवे᳚क्षन्ते \newline

\textbf{Jatai Paata} \newline

1. घ्नन्ति॒ वै वै घ्नन्ति॒ घ्नन्ति॒ वै । \newline
2. वा ए॒त दे॒तद् वै वा ए॒तत् । \newline
3. ए॒तथ् सोमꣳ॒॒ सोम॑ मे॒त दे॒तथ् सोम᳚म् । \newline
4. सोमं॒ ॅयद् यथ् सोमꣳ॒॒ सोमं॒ ॅयत् । \newline
5. यद॑भिषु॒ण्व न्त्य॑भिषु॒ण्वन्ति॒ यद् यद॑भिषु॒ण्वन्ति॑ । \newline
6. अ॒भि॒षु॒ण्वन्ति॒ यद् यद॑भिषु॒ण्व न्त्य॑भिषु॒ण्वन्ति॒ यत् । \newline
7. अ॒भि॒षु॒ण्वन्तीत्य॑भि - सु॒न्वन्ति॑ । \newline
8. यथ् सौ॒म्यः सौ॒म्यो यद् यथ् सौ॒म्यः । \newline
9. सौ॒म्यो भव॑ति॒ भव॑ति सौ॒म्यः सौ॒म्यो भव॑ति । \newline
10. भव॑ति॒ यथा॒ यथा॒ भव॑ति॒ भव॑ति॒ यथा᳚ । \newline
11. यथा॑ मृ॒ताय॑ मृ॒ताय॒ यथा॒ यथा॑ मृ॒ताय॑ । \newline
12. मृ॒ताया॑ नु॒स्तर॑णी मनु॒स्तर॑णीम् मृ॒ताय॑ मृ॒ताया॑ नु॒स्तर॑णीम् । \newline
13. अ॒नु॒स्तर॑णी॒म् घ्नन्ति॒ घ्नन्त्य॑ नु॒स्तर॑णी मनु॒स्तर॑णी॒म् घ्नन्ति॑ । \newline
14. अ॒नु॒स्तर॑णी॒मित्य॑नु - स्तर॑णीम् । \newline
15. घ्नन्ति॑ ता॒दृक् ता॒दृग् घ्नन्ति॒ घ्नन्ति॑ ता॒दृक् । \newline
16. ता॒दृ गे॒वैव ता॒दृक् ता॒दृ गे॒व । \newline
17. ए॒व तत् तदे॒ वैव तत् । \newline
18. तद् यद् यत् तत् तद् यत् । \newline
19. यदु॑त्तरा॒र्द्ध उ॑त्तरा॒र्द्धे यद् यदु॑त्तरा॒र्द्धे । \newline
20. उ॒त्त॒रा॒र्द्धे वा॑ वोत्तरा॒र्द्ध उ॑त्तरा॒र्द्धे वा᳚ । \newline
21. उ॒त्त॒रा॒र्द्ध इत्यु॑त्तर - अ॒र्द्धे । \newline
22. वा॒ मद्ध्ये॒ मद्ध्ये॑ वा वा॒ मद्ध्ये᳚ । \newline
23. मद्ध्ये॑ वा वा॒ मद्ध्ये॒ मद्ध्ये॑ वा । \newline
24. वा॒ जु॒हु॒याज् जु॑हु॒याद् वा॑ वा जुहु॒यात् । \newline
25. जु॒हु॒याद् दे॒वता᳚भ्यो दे॒वता᳚भ्यो जुहु॒याज् जु॑हु॒याद् दे॒वता᳚भ्यः । \newline
26. दे॒वता᳚भ्यः स॒मदꣳ॑ स॒मद॑म् दे॒वता᳚भ्यो दे॒वता᳚भ्यः स॒मद᳚म् । \newline
27. स॒मद॑म् दद्ध्याद् दद्ध्याथ् स॒मदꣳ॑ स॒मद॑म् दद्ध्यात् । \newline
28. स॒मद॒मिति॑ स - मद᳚म् । \newline
29. द॒द्ध्या॒द् द॒क्षि॒णा॒र्द्धे द॑क्षिणा॒र्द्धे द॑द्ध्याद् दद्ध्याद् दक्षिणा॒र्द्धे । \newline
30. द॒क्षि॒णा॒र्द्धे जु॑होति जुहोति दक्षिणा॒र्द्धे द॑क्षिणा॒र्द्धे जु॑होति । \newline
31. द॒क्षि॒णा॒र्द्ध इति॑ दक्षिण - अ॒र्द्धे । \newline
32. जु॒हो॒ त्ये॒षैषा जु॑होति जुहो त्ये॒षा । \newline
33. ए॒षा वै वा ए॒षैषा वै । \newline
34. वै पि॑तृ॒णाम् पि॑तृ॒णां ॅवै वै पि॑तृ॒णाम् । \newline
35. पि॒तृ॒णाम् दिग् दिक् पि॑तृ॒णाम् पि॑तृ॒णाम् दिक् । \newline
36. दिख् स्वायाꣳ॒॒ स्वाया॒म् दिग् दिख् स्वाया᳚म् । \newline
37. स्वाया॑ मे॒वैव स्वायाꣳ॒॒ स्वाया॑ मे॒व । \newline
38. ए॒व दि॒शि दि॒श्ये॑ वैव दि॒शि । \newline
39. दि॒शि पि॒तॄन् पि॒तॄन् दि॒शि दि॒शि पि॒तॄन् । \newline
40. पि॒तॄन् नि॒रव॑दयते नि॒रव॑दयते पि॒तॄन् पि॒तॄन् नि॒रव॑दयते । \newline
41. नि॒रव॑दयत उद्‍गा॒तृभ्य॑ उद्‍गा॒तृभ्यो॑ नि॒रव॑दयते नि॒रव॑दयत उद्‍गा॒तृभ्यः॑ । \newline
42. नि॒रव॑दयत॒ इति॑ निः - अव॑दयते । \newline
43. उ॒द्‍गा॒तृभ्यो॑ हरन्ति हर न्त्युद्‍गा॒तृभ्य॑ उद्‍गा॒तृभ्यो॑ हरन्ति । \newline
44. उ॒द्‍गा॒तृभ्य॒ इत्यु॑द्‍गा॒तृ - भ्यः॒ । \newline
45. ह॒र॒न्ति॒ सा॒म॒दे॒व॒त्यः॑ सामदेव॒त्यो॑ हरन्ति हरन्ति सामदेव॒त्यः॑ । \newline
46. सा॒म॒दे॒व॒त्यो॑ वै वै सा॑मदेव॒त्यः॑ सामदेव॒त्यो॑ वै । \newline
47. सा॒म॒दे॒व॒त्य॑ इति॑ साम - दे॒व॒त्यः॑ । \newline
48. वै सौ॒म्यः सौ॒म्यो वै वै सौ॒म्यः । \newline
49. सौ॒म्यो यद् यथ् सौ॒म्यः सौ॒म्यो यत् । \newline
50. यदे॒वैव यद् यदे॒व । \newline
51. ए॒व साम्नः॒ साम्न॑ ए॒वैव साम्नः॑ । \newline
52. साम्न॑ श्छंबट्कु॒र्वन्ति॑ छंबट्कु॒र्वन्ति॒ साम्नः॒ साम्न॑ श्छंबट्कु॒र्वन्ति॑ । \newline
53. छं॒ब॒ट्कु॒र्वन्ति॒ तस्य॒ तस्य॑ छंबट्कु॒र्वन्ति॑ छंबट्कु॒र्वन्ति॒ तस्य॑ । \newline
54. छं॒ब॒ट्कु॒र्वन्तीति॑ छंबट् - कु॒र्वन्ति॑ । \newline
55. तस्यै॒वैव तस्य॒ तस्यै॒व । \newline
56. ए॒व स स ए॒वैव सः । \newline
57. स शान्तिः॒ शान्तिः॒ स स शान्तिः॑ । \newline
58. शान्ति॒ रवाव॒ शान्तिः॒ शान्ति॒ रव॑ । \newline
59. अवे᳚क्षन्त ईक्ष॒न्ते ऽवावे᳚क्षन्ते । \newline

\textbf{Ghana Paata } \newline

1. घ्नन्ति॒ वै वै घ्नन्ति॒ घ्नन्ति॒ वा ए॒त दे॒तद् वै घ्नन्ति॒ घ्नन्ति॒ वा ए॒तत् । \newline
2. वा ए॒त दे॒तद् वै वा ए॒तथ् सोमꣳ॒॒ सोम॑ मे॒तद् वै वा ए॒तथ् सोम᳚म् । \newline
3. ए॒तथ् सोमꣳ॒॒ सोम॑ मे॒त दे॒तथ् सोमं॒ ॅयद् यथ् सोम॑ मे॒त दे॒तथ् सोमं॒ ॅयत् । \newline
4. सोमं॒ ॅयद् यथ् सोमꣳ॒॒ सोमं॒ ॅयद॑भिषु॒ण्व न्त्य॑भिषु॒ण्वन्ति॒ यथ् सोमꣳ॒॒ सोमं॒ ॅयद॑भिषु॒ण्वन्ति॑ । \newline
5. यद॑भिषु॒ण्व न्त्य॑भिषु॒ण्वन्ति॒ यद् यद॑भिषु॒ण्वन्ति॒ यद् यद॑भिषु॒ण्वन्ति॒ यद् यद॑भिषु॒ण्वन्ति॒ यत् । \newline
6. अ॒भि॒षु॒ण्वन्ति॒ यद् यद॑भिषु॒ण्व न्त्य॑भिषु॒ण्वन्ति॒ यथ् सौ॒म्यः सौ॒म्यो यद॑भिषु॒ण्व न्त्य॑भिषु॒ण्वन्ति॒ यथ् सौ॒म्यः । \newline
7. अ॒भि॒षु॒ण्वन्तीत्य॑भि - सु॒न्वन्ति॑ । \newline
8. यथ् सौ॒म्यः सौ॒म्यो यद् यथ् सौ॒म्यो भव॑ति॒ भव॑ति सौ॒म्यो यद् यथ् सौ॒म्यो भव॑ति । \newline
9. सौ॒म्यो भव॑ति॒ भव॑ति सौ॒म्यः सौ॒म्यो भव॑ति॒ यथा॒ यथा॒ भव॑ति सौ॒म्यः सौ॒म्यो भव॑ति॒ यथा᳚ । \newline
10. भव॑ति॒ यथा॒ यथा॒ भव॑ति॒ भव॑ति॒ यथा॑ मृ॒ताय॑ मृ॒ताय॒ यथा॒ भव॑ति॒ भव॑ति॒ यथा॑ मृ॒ताय॑ । \newline
11. यथा॑ मृ॒ताय॑ मृ॒ताय॒ यथा॒ यथा॑ मृ॒ताया॑ नु॒स्तर॑णी मनु॒स्तर॑णीम् मृ॒ताय॒ यथा॒ यथा॑ मृ॒ताया॑ नु॒स्तर॑णीम् । \newline
12. मृ॒ताया॑ नु॒स्तर॑णी मनु॒स्तर॑णीम् मृ॒ताय॑ मृ॒ताया॑ नु॒स्तर॑णी॒म् घ्नन्ति॒ घ्नन्त्य॑नु॒स्तर॑णीम् मृ॒ताय॑ मृ॒ताया॑ नु॒स्तर॑णी॒म् घ्नन्ति॑ । \newline
13. अ॒नु॒स्तर॑णी॒म् घ्नन्ति॒ घ्नन्त्य॑ नु॒स्तर॑णी मनु॒स्तर॑णी॒म् घ्नन्ति॑ ता॒दृक् ता॒दृग् घ्नन्त्य॑नु॒स्तर॑णी मनु॒स्तर॑णी॒म् घ्नन्ति॑ ता॒दृक् । \newline
14. अ॒नु॒स्तर॑णी॒मित्य॑नु - स्तर॑णीम् । \newline
15. घ्नन्ति॑ ता॒दृक् ता॒दृग् घ्नन्ति॒ घ्नन्ति॑ ता॒दृ गे॒वैव ता॒दृग् घ्नन्ति॒ घ्नन्ति॑ ता॒दृ गे॒व । \newline
16. ता॒दृ गे॒वैव ता॒दृक् ता॒दृ गे॒व तत् तदे॒व ता॒दृक् ता॒दृ गे॒व तत् । \newline
17. ए॒व तत् तदे॒ वैव तद् यद् यत् तदे॒ वैव तद् यत् । \newline
18. तद् यद् यत् तत् तद् यदु॑त्तरा॒र्द्ध उ॑त्तरा॒र्द्धे यत् तत् तद् यदु॑त्तरा॒र्द्धे । \newline
19. यदु॑त्तरा॒र्द्ध उ॑त्तरा॒र्द्धे यद् यदु॑त्तरा॒र्द्धे वा॑ वोत्तरा॒र्द्धे यद् यदु॑त्तरा॒र्द्धे वा᳚ । \newline
20. उ॒त्त॒रा॒र्द्धे वा॑ वोत्तरा॒र्द्ध उ॑त्तरा॒र्द्धे वा॒ मद्ध्ये॒ मद्ध्ये॑ वोत्तरा॒र्द्ध उ॑त्तरा॒र्द्धे वा॒ मद्ध्ये᳚ । \newline
21. उ॒त्त॒रा॒र्द्ध इत्यु॑त्तर - अ॒र्द्धे । \newline
22. वा॒ मद्ध्ये॒ मद्ध्ये॑ वा वा॒ मद्ध्ये॑ वा वा॒ मद्ध्ये॑ वा वा॒ मद्ध्ये॑ वा । \newline
23. मद्ध्ये॑ वा वा॒ मद्ध्ये॒ मद्ध्ये॑ वा जुहु॒याज् जु॑हु॒याद् वा॒ मद्ध्ये॒ मद्ध्ये॑ वा जुहु॒यात् । \newline
24. वा॒ जु॒हु॒याज् जु॑हु॒याद् वा॑ वा जुहु॒याद् दे॒वता᳚भ्यो दे॒वता᳚भ्यो जुहु॒याद् वा॑ वा जुहु॒याद् दे॒वता᳚भ्यः । \newline
25. जु॒हु॒याद् दे॒वता᳚भ्यो दे॒वता᳚भ्यो जुहु॒याज् जु॑हु॒याद् दे॒वता᳚भ्यः स॒मदꣳ॑ स॒मद॑म् दे॒वता᳚भ्यो जुहु॒याज् जु॑हु॒याद् दे॒वता᳚भ्यः स॒मद᳚म् । \newline
26. दे॒वता᳚भ्यः स॒मदꣳ॑ स॒मद॑म् दे॒वता᳚भ्यो दे॒वता᳚भ्यः स॒मद॑म् दद्ध्याद् दद्ध्याथ् स॒मद॑म् दे॒वता᳚भ्यो दे॒वता᳚भ्यः स॒मद॑म् दद्ध्यात् । \newline
27. स॒मद॑म् दद्ध्याद् दद्ध्याथ् स॒मदꣳ॑ स॒मद॑म् दद्ध्याद् दक्षिणा॒र्द्धे द॑क्षिणा॒र्द्धे द॑द्ध्याथ् स॒मदꣳ॑ स॒मद॑म् दद्ध्याद् दक्षिणा॒र्द्धे । \newline
28. स॒मद॒मिति॑ स - मद᳚म् । \newline
29. द॒द्ध्या॒द् द॒क्षि॒णा॒र्द्धे द॑क्षिणा॒र्द्धे द॑द्ध्याद् दद्ध्याद् दक्षिणा॒र्द्धे जु॑होति जुहोति दक्षिणा॒र्द्धे द॑द्ध्याद् दद्ध्याद् दक्षिणा॒र्द्धे जु॑होति । \newline
30. द॒क्षि॒णा॒र्द्धे जु॑होति जुहोति दक्षिणा॒र्द्धे द॑क्षिणा॒र्द्धे जु॑हो त्ये॒षैषा जु॑होति दक्षिणा॒र्द्धे द॑क्षिणा॒र्द्धे जु॑हो त्ये॒षा । \newline
31. द॒क्षि॒णा॒र्द्ध इति॑ दक्षिण - अ॒र्द्धे । \newline
32. जु॒हो॒ त्ये॒षैषा जु॑होति जुहो त्ये॒षा वै वा ए॒षा जु॑होति जुहो त्ये॒षा वै । \newline
33. ए॒षा वै वा ए॒षैषा वै पि॑तृ॒णाम् पि॑तृ॒णां ॅवा ए॒षैषा वै पि॑तृ॒णाम् । \newline
34. वै पि॑तृ॒णाम् पि॑तृ॒णां ॅवै वै पि॑तृ॒णाम् दिग् दिक् पि॑तृ॒णां ॅवै वै पि॑तृ॒णाम् दिक् । \newline
35. पि॒तृ॒णाम् दिग् दिक् पि॑तृ॒णाम् पि॑तृ॒णाम् दिख् स्वायाꣳ॒॒ स्वाया॒म् दिक् पि॑तृ॒णाम् पि॑तृ॒णाम् दिख् स्वाया᳚म् । \newline
36. दिख् स्वायाꣳ॒॒ स्वाया॒म् दिग् दिख् स्वाया॑ मे॒वैव स्वाया॒म् दिग् दिख् स्वाया॑ मे॒व । \newline
37. स्वाया॑ मे॒वैव स्वायाꣳ॒॒ स्वाया॑ मे॒व दि॒शि दि॒श्ये॑व स्वायाꣳ॒॒ स्वाया॑ मे॒व दि॒शि । \newline
38. ए॒व दि॒शि दि॒श्ये॑वैव दि॒शि पि॒तॄन् पि॒तॄन् दि॒श्ये॑वैव दि॒शि पि॒तॄन् । \newline
39. दि॒शि पि॒तॄन् पि॒तॄन् दि॒शि दि॒शि पि॒तॄन् नि॒रव॑दयते नि॒रव॑दयते पि॒तॄन् दि॒शि दि॒शि पि॒तॄन् नि॒रव॑दयते । \newline
40. पि॒तॄन् नि॒रव॑दयते नि॒रव॑दयते पि॒तॄन् पि॒तॄन् नि॒रव॑दयत उद्‌गा॒तृभ्य॑ उद्‌गा॒तृभ्यो॑ नि॒रव॑दयते पि॒तॄन् पि॒तॄन् नि॒रव॑दयत उद्‌गा॒तृभ्यः॑ । \newline
41. नि॒रव॑दयत उद्‌गा॒तृभ्य॑ उद्‌गा॒तृभ्यो॑ नि॒रव॑दयते नि॒रव॑दयत उद्‌गा॒तृभ्यो॑ हरन्ति हर
न्त्युद्‌गा॒तृभ्यो॑ नि॒रव॑दयते नि॒रव॑दयत उद्‌गा॒तृभ्यो॑ हरन्ति । \newline
42. नि॒रव॑दयत॒ इति॑ निः - अव॑दयते । \newline
43. उ॒द्‌गा॒तृभ्यो॑ हरन्ति हर न्त्युद्‌गा॒तृभ्य॑ उद्‌गा॒तृभ्यो॑ हरन्ति सामदेव॒त्यः॑ सामदेव॒त्यो॑ हर
न्त्युद्‌गा॒तृभ्य॑ उद्‌गा॒तृभ्यो॑ हरन्ति सामदेव॒त्यः॑ । \newline
44. उ॒द्‌गा॒तृभ्य॒ इत्यु॑द्‌गा॒तृ - भ्यः॒ । \newline
45. ह॒र॒न्ति॒ सा॒म॒दे॒व॒त्यः॑ सामदेव॒त्यो॑ हरन्ति हरन्ति सामदेव॒त्यो॑ वै वै सा॑मदेव॒त्यो॑ हरन्ति हरन्ति सामदेव॒त्यो॑ वै । \newline
46. सा॒म॒दे॒व॒त्यो॑ वै वै सा॑मदेव॒त्यः॑ सामदेव॒त्यो॑ वै सौ॒म्यः सौ॒म्यो वै सा॑मदेव॒त्यः॑ सामदेव॒त्यो॑ वै सौ॒म्यः । \newline
47. सा॒म॒दे॒व॒त्य॑ इति॑ साम - दे॒व॒त्यः॑ । \newline
48. वै सौ॒म्यः सौ॒म्यो वै वै सौ॒म्यो यद् यथ् सौ॒म्यो वै वै सौ॒म्यो यत् । \newline
49. सौ॒म्यो यद् यथ् सौ॒म्यः सौ॒म्यो यदे॒ वैव यथ् सौ॒म्यः सौ॒म्यो यदे॒व । \newline
50. यदे॒ वैव यद् यदे॒व साम्नः॒ साम्न॑ ए॒व यद् यदे॒व साम्नः॑ । \newline
51. ए॒व साम्नः॒ साम्न॑ ए॒वैव साम्न॑ श्छंबट्कु॒र्वन्ति॑ छंबट्कु॒र्वन्ति॒ साम्न॑ ए॒वैव साम्न॑ श्छंबट्कु॒र्वन्ति॑ । \newline
52. साम्न॑ श्छंबट्कु॒र्वन्ति॑ छंबट्कु॒र्वन्ति॒ साम्नः॒ साम्न॑ श्छंबट्कु॒र्वन्ति॒ तस्य॒ तस्य॑ छंबट्कु॒र्वन्ति॒ साम्नः॒ साम्न॑ श्छंबट्कु॒र्वन्ति॒ तस्य॑ । \newline
53. छं॒ब॒ट्कु॒र्वन्ति॒ तस्य॒ तस्य॑ छंबट्कु॒र्वन्ति॑ छंबट्कु॒र्वन्ति॒ तस्यै॒वैव तस्य॑ छंबट्कु॒र्वन्ति॑ छंबट्कु॒र्वन्ति॒ तस्यै॒व । \newline
54. छं॒ब॒ट्कु॒र्वन्तीति॑ छंबट् - कु॒र्वन्ति॑ । \newline
55. तस्यै॒वैव तस्य॒ तस्यै॒व स स ए॒व तस्य॒ तस्यै॒व सः । \newline
56. ए॒व स स ए॒वैव स शान्तिः॒ शान्तिः॒ स ए॒वैव स शान्तिः॑ । \newline
57. स शान्तिः॒ शान्तिः॒ स स शान्ति॒ रवाव॒ शान्तिः॒ स स शान्ति॒ रव॑ । \newline
58. शान्ति॒ रवाव॒ शान्तिः॒ शान्ति॒ रवे᳚क्षन्त ईक्ष॒न्ते ऽव॒ शान्तिः॒ शान्ति॒ रवे᳚क्षन्ते । \newline
59. अवे᳚क्षन्त ईक्ष॒न्ते ऽवावे᳚क्षन्ते प॒वित्र॑म् प॒वित्र॑ मीक्ष॒न्ते ऽवावे᳚क्षन्ते प॒वित्र᳚म् । \newline
\pagebreak
\markright{ TS 6.6.7.2  \hfill https://www.vedavms.in \hfill}

\section{ TS 6.6.7.2 }

\textbf{TS 6.6.7.2 } \newline
\textbf{Samhita Paata} \newline

-क्षन्ते प॒वित्रं॒ ॅवै सौ॒म्य आ॒त्मान॑मे॒व प॑वयन्ते॒ य आ॒त्मानं॒ न प॑रि॒पश्ये॑दि॒तासुः॑ स्यादभिद॒दिं कृ॒त्वाऽवे᳚क्षेत॒ तस्मि॒न॒. ह्या᳚त्मानं॑ परि॒पश्य॒त्यथो॑ आ॒त्मान॑मे॒व प॑वयते॒ यो ग॒तम॑नाः॒ स्याथ् सोऽवे᳚क्षेत॒ यन्मे॒ मनः॒ परा॑गतं॒ ॅयद्वा॑ मे॒ अप॑रागतं । राज्ञा॒ सोमे॑न॒ तद्व॒यम॒स्मासु॑ धारयाम॒सीति॒ मन॑ ए॒वात्मन् दा॑धार॒- [  ] \newline

\textbf{Pada Paata} \newline

ई॒क्ष॒न्ते॒ । प॒वित्र᳚म् । वै । सौ॒म्यः । आ॒त्मान᳚म् । ए॒व । प॒व॒य॒न्ते॒ । यः । आ॒त्मान᳚म् । न । प॒रि॒पश्ये॒दिति॑ परि-पश्ये᳚त् । इ॒तासु॒रिती॒त-अ॒सुः॒ । स्या॒त् । अ॒भि॒द॒दिमित्य॑भि - द॒दिम् । कृ॒त्वा । अवेति॑ । ई॒क्षे॒त॒ । तस्मिन्न्॑ । हि । आ॒त्मान᳚म् । प॒रि॒पश्य॒तीति॑ परि - पश्य॑ति । अथो॒ इति॑ । आ॒त्मान᳚म् । ए॒व । प॒व॒य॒ते॒ । यः । ग॒तम॑ना॒ इति॑ ग॒त - म॒नाः॒ । स्यात् । सः । अवेति॑ । ई॒क्षे॒त॒ । यत् । मे॒ । मनः॑ । परा॑गत॒मिति॒ परा᳚-ग॒त॒म् । यत् । वा॒ । मे॒ । अप॑रागत॒मित्यप॑रा-ग॒त॒म् ॥ राज्ञा᳚ । सोमे॑न । तत् । व॒यम् । अ॒स्मासु॑ । धा॒र॒या॒म॒सि॒ । इति॑ । मनः॑ । ए॒व । आ॒त्मन्न् । दा॒धा॒र॒ ।  \newline


\textbf{Krama Paata} \newline

ई॒क्ष॒न्ते॒ प॒वित्र᳚म् । प॒वित्र॒म् ॅवै । वै सौ॒म्यः । सौ॒म्य आ॒त्मान᳚म् । आ॒त्मान॑मे॒व । ए॒व प॑वयन्ते । प॒व॒य॒न्ते॒ यः । य आ॒त्मान᳚म् । आ॒त्मान॒म् न । न प॑रि॒पश्ये᳚त् । प॒रि॒पश्ये॑दि॒तासुः॑ । प॒रि॒पश्ये॒दिति॑ परि - पश्ये᳚त् । इ॒तासुः॑ स्यात् । इ॒तासु॒रिती॒त - अ॒सुः॒ । स्या॒द॒भि॒द॒दिम् । अ॒भि॒द॒दिम् कृ॒त्वा । अ॒भि॒द॒दिमित्य॑भि - द॒दिम् । कृ॒त्वाऽव॑ । अवे᳚क्षेतः । ई॒क्षे॒त॒ तस्मिन्न्॑ । तस्मि॒न्.॒ हि । ह्या᳚त्मान᳚म् । आ॒त्मान॑म् परि॒पश्य॑ति । प॒रि॒पश्य॒त्यथो᳚ । प॒रि॒पश्य॒तीति॑ परि - पश्य॑ति । अथो॑ आ॒त्मान᳚म् । अथो॒ इत्यथो᳚ । आ॒त्मान॑मे॒व । ए॒व प॑वयते । प॒व॒य॒ते॒ यः । यो ग॒तम॑नाः । ग॒तम॑नाः॒ स्यात् । ग॒तम॑ना॒ इति॑ ग॒त - म॒नाः॒ । स्याथ् सः । सोऽव॑ । अवे᳚क्षेत । ई॒क्षे॒त॒ यत् । यन् मे᳚ । मे॒ मनः॑ । मनः॒ परा॑गतम् । परा॑गत॒म् ॅयत् । परा॑गत॒मिति॒ परा᳚ - ग॒त॒म् । यद् वा᳚ । वा॒ मे॒ । मे॒ अप॑रागतम् । अप॑रागत॒मित्यप॑रा - ग॒त॒म् ॥ राज्ञा॒ सोमे॑न । सोमे॑न॒ तत् । तद् व॒यम् । व॒यम॒स्मासु॑ । अ॒स्मासु॑ धारयामसि । धा॒रा॒या॒म॒सीति॑ । इति॒ मनः॑ । मन॑ ए॒व । ए॒वात्मन्न् । आ॒त्मन् दा॑धार । दा॒धा॒र॒ न \newline

\textbf{Jatai Paata} \newline

1. ई॒क्ष॒न्ते॒ प॒वित्र॑म् प॒वित्र॑ मीक्षन्त ईक्षन्ते प॒वित्र᳚म् । \newline
2. प॒वित्रं॒ ॅवै वै प॒वित्र॑म् प॒वित्रं॒ ॅवै । \newline
3. वै सौ॒म्यः सौ॒म्यो वै वै सौ॒म्यः । \newline
4. सौ॒म्य आ॒त्मान॑ मा॒त्मानꣳ॑ सौ॒म्यः सौ॒म्य आ॒त्मान᳚म् । \newline
5. आ॒त्मान॑ मे॒वै वात्मान॑ मा॒त्मान॑ मे॒व । \newline
6. ए॒व प॑वयन्ते पवयन्त ए॒वैव प॑वयन्ते । \newline
7. प॒व॒य॒न्ते॒ यो यः प॑वयन्ते पवयन्ते॒ यः । \newline
8. य आ॒त्मान॑ मा॒त्मानं॒ ॅयो य आ॒त्मान᳚म् । \newline
9. आ॒त्मान॒न् न नात्मान॑ मा॒त्मान॒न् न । \newline
10. न प॑रि॒पश्ये᳚त् परि॒पश्ये॒न् न न प॑रि॒पश्ये᳚त् । \newline
11. प॒रि॒पश्ये॑ दि॒तासु॑ रि॒तासुः॑ परि॒पश्ये᳚त् परि॒पश्ये॑ दि॒तासुः॑ । \newline
12. प॒रि॒पश्ये॒दिति॑ परि - पश्ये᳚त् । \newline
13. इ॒तासुः॑ स्याथ् स्या दि॒तासु॑ रि॒तासुः॑ स्यात् । \newline
14. इ॒तासु॒रिती॒त - अ॒सुः॒ । \newline
15. स्या॒ द॒भि॒द॒दि म॑भिद॒दिꣳ स्या᳚थ् स्या दभिद॒दिम् । \newline
16. अ॒भि॒द॒दिम् कृ॒त्वा कृ॒त्वा ऽभि॑द॒दि म॑भिद॒दिम् कृ॒त्वा । \newline
17. अ॒भि॒द॒दिमित्य॑भि - द॒दिम् । \newline
18. कृ॒त्वा ऽवाव॑ कृ॒त्वा कृ॒त्वा ऽव॑ । \newline
19. अवे᳚ क्षेते क्षे॒ता वावे᳚क्षेत । \newline
20. ई॒क्षे॒त॒ तस्मिꣳ॒॒ स्तस्मि॑न् नीक्षेते क्षेत॒ तस्मिन्न्॑ । \newline
21. तस्मि॒न्॒. हि हि तस्मिꣳ॒॒ स्तस्मि॒न्॒. हि । \newline
22. ह्या᳚त्मान॑ मा॒त्मानꣳ॒॒ हि ह्या᳚त्मान᳚म् । \newline
23. आ॒त्मान॑म् परि॒पश्य॑ति परि॒पश्य॑ त्या॒त्मान॑ मा॒त्मान॑म् परि॒पश्य॑ति । \newline
24. प॒रि॒पश्य॒ त्यथो॒ अथो॑ परि॒पश्य॑ति परि॒पश्य॒ त्यथो᳚ । \newline
25. प॒रि॒पश्य॒तीति॑ परि - पश्य॑ति । \newline
26. अथो॑ आ॒त्मान॑ मा॒त्मान॒ मथो॒ अथो॑ आ॒त्मान᳚म् । \newline
27. अथो॒ इत्यथो᳚ । \newline
28. आ॒त्मान॑ मे॒वैवात्मान॑ मा॒त्मान॑ मे॒व । \newline
29. ए॒व प॑वयते पवयत ए॒वैव प॑वयते । \newline
30. प॒व॒य॒ते॒ यो यः प॑वयते पवयते॒ यः । \newline
31. यो ग॒तम॑ना ग॒तम॑ना॒ यो यो ग॒तम॑नाः । \newline
32. ग॒तम॑नाः॒ स्याथ् स्याद् ग॒तम॑ना ग॒तम॑नाः॒ स्यात् । \newline
33. ग॒तम॑ना॒ इति॑ ग॒त - म॒नाः॒ । \newline
34. स्याथ् स स स्याथ् स्याथ् सः । \newline
35. सो ऽवाव॒ स सो ऽव॑ । \newline
36. अवे᳚ क्षेते क्षे॒तावा वे᳚क्षेत । \newline
37. ई॒क्षे॒त॒ यद् यदी᳚ क्षेते क्षेत॒ यत् । \newline
38. यन् मे॑ मे॒ यद् यन् मे᳚ । \newline
39. मे॒ मनो॒ मनो॑ मे मे॒ मनः॑ । \newline
40. मनः॒ परा॑गत॒म् परा॑गत॒म् मनो॒ मनः॒ परा॑गतम् । \newline
41. परा॑गतं॒ ॅयद् यत् परा॑गत॒म् परा॑गतं॒ ॅयत् । \newline
42. परा॑गत॒मिति॒ परा᳚ - ग॒त॒म् । \newline
43. यद् वा॑ वा॒ यद् यद् वा᳚ । \newline
44. वा॒ मे॒ मे॒ वा॒ वा॒ मे॒ । \newline
45. मे॒ अप॑रागत॒ मप॑रागतम् मे मे॒ अप॑रागतम् । \newline
46. अप॑रागत॒मित्यप॑रा - ग॒त॒म् । \newline
47. राज्ञा॒ सोमे॑न॒ सोमे॑न॒ राज्ञा॒ राज्ञा॒ सोमे॑न । \newline
48. सोमे॑न॒ तत् तथ् सोमे॑न॒ सोमे॑न॒ तत् । \newline
49. तद् व॒यं ॅव॒यम् तत् तद् व॒यम् । \newline
50. व॒य म॒स्मा स्व॒स्मासु॑ व॒यं ॅव॒य म॒स्मासु॑ । \newline
51. अ॒स्मासु॑ धारयामसि धारयाम स्य॒स्मा स्व॒स्मासु॑ धारयामसि । \newline
52. धा॒र॒या॒म॒ सीतीति॑ धारयामसि धारयाम॒ सीति॑ । \newline
53. इति॒ मनो॒ मन॒ इतीति॒ मनः॑ । \newline
54. मन॑ ए॒वैव मनो॒ मन॑ ए॒व । \newline
55. ए॒वात्मन् ना॒त्मन् ने॒वै वात्मन्न् । \newline
56. आ॒त्मन् दा॑धार दाधा रा॒त्मन् ना॒त्मन् दा॑धार । \newline
57. दा॒धा॒र॒ न न दा॑धार दाधार॒ न । \newline

\textbf{Ghana Paata } \newline

1. ई॒क्ष॒न्ते॒ प॒वित्र॑म् प॒वित्र॑ मीक्षन्त ईक्षन्ते प॒वित्रं॒ ॅवै वै प॒वित्र॑ मीक्षन्त ईक्षन्ते प॒वित्रं॒ ॅवै । \newline
2. प॒वित्रं॒ ॅवै वै प॒वित्र॑म् प॒वित्रं॒ ॅवै सौ॒म्यः सौ॒म्यो वै प॒वित्र॑म् प॒वित्रं॒ ॅवै सौ॒म्यः । \newline
3. वै सौ॒म्यः सौ॒म्यो वै वै सौ॒म्य आ॒त्मान॑ मा॒त्मानꣳ॑ सौ॒म्यो वै वै सौ॒म्य आ॒त्मान᳚म् । \newline
4. सौ॒म्य आ॒त्मान॑ मा॒त्मानꣳ॑ सौ॒म्यः सौ॒म्य आ॒त्मान॑ मे॒वै वात्मानꣳ॑ सौ॒म्यः सौ॒म्य आ॒त्मान॑ मे॒व । \newline
5. आ॒त्मान॑ मे॒वै वात्मान॑ मा॒त्मान॑ मे॒व प॑वयन्ते पवयन्त ए॒वात्मान॑ मा॒त्मान॑ मे॒व प॑वयन्ते । \newline
6. ए॒व प॑वयन्ते पवयन्त ए॒वैव प॑वयन्ते॒ यो यः प॑वयन्त ए॒वैव प॑वयन्ते॒ यः । \newline
7. प॒व॒य॒न्ते॒ यो यः प॑वयन्ते पवयन्ते॒ य आ॒त्मान॑ मा॒त्मानं॒ ॅयः प॑वयन्ते पवयन्ते॒ य आ॒त्मान᳚म् । \newline
8. य आ॒त्मान॑ मा॒त्मानं॒ ॅयो य आ॒त्मान॒न् न नात्मानं॒ ॅयो य आ॒त्मान॒न् न । \newline
9. आ॒त्मान॒न् न नात्मान॑ मा॒त्मान॒न् न प॑रि॒पश्ये᳚त् परि॒पश्ये॒न् नात्मान॑ मा॒त्मान॒न् न प॑रि॒पश्ये᳚त् । \newline
10. न प॑रि॒पश्ये᳚त् परि॒पश्ये॒न् न न प॑रि॒पश्ये॑ दि॒तासु॑ रि॒तासुः॑ परि॒पश्ये॒न् न न प॑रि॒पश्ये॑ दि॒तासुः॑ । \newline
11. प॒रि॒पश्ये॑ दि॒तासु॑ रि॒तासुः॑ परि॒पश्ये᳚त् परि॒पश्ये॑ दि॒तासुः॑ स्याथ् स्या दि॒तासुः॑ परि॒पश्ये᳚त् परि॒पश्ये॑ दि॒तासुः॑ स्यात् । \newline
12. प॒रि॒पश्ये॒दिति॑ परि - पश्ये᳚त् । \newline
13. इ॒तासुः॑ स्याथ् स्या दि॒तासु॑ रि॒तासुः॑ स्या दभिद॒दि म॑भिद॒दिꣳ स्या॑ दि॒तासु॑ रि॒तासुः॑ स्या दभिद॒दिम् । \newline
14. इ॒तासु॒रिती॒त - अ॒सुः॒ । \newline
15. स्या॒ द॒भि॒द॒दि म॑भिद॒दिꣳ स्या᳚थ् स्या दभिद॒दिम् कृ॒त्वा कृ॒त्वा ऽभि॑द॒दिꣳ स्या᳚थ् स्या दभिद॒दिम् कृ॒त्वा । \newline
16. अ॒भि॒द॒दिम् कृ॒त्वा कृ॒त्वा ऽभि॑द॒दि म॑भिद॒दिम् कृ॒त्वा ऽवाव॑ कृ॒त्वा ऽभि॑द॒दि म॑भिद॒दिम् कृ॒त्वा ऽव॑ । \newline
17. अ॒भि॒द॒दिमित्य॑भि - द॒दिम् । \newline
18. कृ॒त्वा ऽवाव॑ कृ॒त्वा कृ॒त्वा ऽवे᳚क्षे तेक्षे॒ताव॑ कृ॒त्वा कृ॒त्वा ऽवे᳚क्षेत । \newline
19. अवे᳚क्षे तेक्षे॒ता वावे᳚क्षेत॒ तस्मिꣳ॒॒ स्तस्मि॑न् नीक्षे॒ता वावे᳚क्षेत॒ तस्मिन्न्॑ । \newline
20. ई॒क्षे॒त॒ तस्मिꣳ॒॒ स्तस्मि॑न् नीक्षे तेक्षेत॒ तस्मि॒न्॒. हि हि तस्मि॑न् नीक्षे तेक्षेत॒ तस्मि॒न्॒. हि । \newline
21. तस्मि॒न्॒. हि हि तस्मिꣳ॒॒ स्तस्मि॒न्॒. ह्या᳚त्मान॑ मा॒त्मानꣳ॒॒ हि तस्मिꣳ॒॒ स्तस्मि॒न्॒. ह्या᳚त्मान᳚म् । \newline
22. ह्या᳚त्मान॑ मा॒त्मानꣳ॒॒ हि ह्या᳚त्मान॑म् परि॒पश्य॑ति परि॒पश्य॑ त्या॒त्मानꣳ॒॒ हि ह्या᳚त्मान॑म् परि॒पश्य॑ति । \newline
23. आ॒त्मान॑म् परि॒पश्य॑ति परि॒पश्य॑ त्या॒त्मान॑ मा॒त्मान॑म् परि॒पश्य॒ त्यथो॒ अथो॑ परि॒पश्य॑ त्या॒त्मान॑ मा॒त्मान॑म् परि॒पश्य॒ त्यथो᳚ । \newline
24. प॒रि॒पश्य॒ त्यथो॒ अथो॑ परि॒पश्य॑ति परि॒पश्य॒ त्यथो॑ आ॒त्मान॑ मा॒त्मान॒ मथो॑ परि॒पश्य॑ति परि॒पश्य॒ त्यथो॑ आ॒त्मान᳚म् । \newline
25. प॒रि॒पश्य॒तीति॑ परि - पश्य॑ति । \newline
26. अथो॑ आ॒त्मान॑ मा॒त्मान॒ मथो॒ अथो॑ आ॒त्मान॑ मे॒वै वात्मान॒ मथो॒ अथो॑ आ॒त्मान॑ मे॒व । \newline
27. अथो॒ इत्यथो᳚ । \newline
28. आ॒त्मान॑ मे॒वै वात्मान॑ मा॒त्मान॑ मे॒व प॑वयते पवयत ए॒वात्मान॑ मा॒त्मान॑ मे॒व प॑वयते । \newline
29. ए॒व प॑वयते पवयत ए॒वैव प॑वयते॒ यो यः प॑वयत ए॒वैव प॑वयते॒ यः । \newline
30. प॒व॒य॒ते॒ यो यः प॑वयते पवयते॒ यो ग॒तम॑ना ग॒तम॑ना॒ यः प॑वयते पवयते॒ यो ग॒तम॑नाः । \newline
31. यो ग॒तम॑ना ग॒तम॑ना॒ यो यो ग॒तम॑नाः॒ स्याथ् स्याद् ग॒तम॑ना॒ यो यो ग॒तम॑नाः॒ स्यात् । \newline
32. ग॒तम॑नाः॒ स्याथ् स्याद् ग॒तम॑ना ग॒तम॑नाः॒ स्याथ् स स स्याद् ग॒तम॑ना ग॒तम॑नाः॒ स्याथ् सः । \newline
33. ग॒तम॑ना॒ इति॑ ग॒त - म॒नाः॒ । \newline
34. स्याथ् स स स्याथ् स्याथ् सो ऽवाव॒ स स्याथ् स्याथ् सो ऽव॑ । \newline
35. सो ऽवाव॒ स सो ऽवे᳚क्षे तेक्षे॒ताव॒ स सो ऽवे᳚क्षेत । \newline
36. अवे᳚क्षे तेक्षे॒ता वावे᳚क्षेत॒ यद् यदी᳚ क्षे॒ता वावे᳚क्षेत॒ यत् । \newline
37. ई॒क्षे॒त॒ यद् यदी᳚क्षेते क्षेत॒ यन् मे॑ मे॒ यदी᳚क्षेते क्षेत॒ यन् मे᳚ । \newline
38. यन् मे॑ मे॒ यद् यन् मे॒ मनो॒ मनो॑ मे॒ यद् यन् मे॒ मनः॑ । \newline
39. मे॒ मनो॒ मनो॑ मे मे॒ मनः॒ परा॑गत॒म् परा॑गत॒म् मनो॑ मे मे॒ मनः॒ परा॑गतम् । \newline
40. मनः॒ परा॑गत॒म् परा॑गत॒म् मनो॒ मनः॒ परा॑गतं॒ ॅयद् यत् परा॑गत॒म् मनो॒ मनः॒ परा॑गतं॒ ॅयत् । \newline
41. परा॑गतं॒ ॅयद् यत् परा॑गत॒म् परा॑गतं॒ ॅयद् वा॑ वा॒ यत् परा॑गत॒म् परा॑गतं॒ ॅयद् वा᳚ । \newline
42. परा॑गत॒मिति॒ परा᳚ - ग॒त॒म् । \newline
43. यद् वा॑ वा॒ यद् यद् वा॑ मे मे वा॒ यद् यद् वा॑ मे । \newline
44. वा॒ मे॒ मे॒ वा॒ वा॒ मे॒ अप॑रागत॒ मप॑रागतम् मे वा वा मे॒ अप॑रागतम् । \newline
45. मे॒ अप॑रागत॒ मप॑रागतम् मे मे॒ अप॑रागतम् । \newline
46. अप॑रागत॒मित्यप॑रा - ग॒त॒म् । \newline
47. राज्ञा॒ सोमे॑न॒ सोमे॑न॒ राज्ञा॒ राज्ञा॒ सोमे॑न॒ तत् तथ् सोमे॑न॒ राज्ञा॒ राज्ञा॒ सोमे॑न॒ तत् । \newline
48. सोमे॑न॒ तत् तथ् सोमे॑न॒ सोमे॑न॒ तद् व॒यं ॅव॒यम् तथ् सोमे॑न॒ सोमे॑न॒ तद् व॒यम् । \newline
49. तद् व॒यं ॅव॒यम् तत् तद् व॒य म॒स्मा स्व॒स्मासु॑ व॒यम् तत् तद् व॒य म॒स्मासु॑ । \newline
50. व॒य म॒स्मा स्व॒स्मासु॑ व॒यं ॅव॒य म॒स्मासु॑ धारयामसि धारयाम स्य॒स्मासु॑ व॒यं ॅव॒य म॒स्मासु॑ धारयामसि । \newline
51. अ॒स्मासु॑ धारयामसि धारयाम स्य॒स्मा स्व॒स्मासु॑ धारयाम॒सीतीति॑ धारयाम स्य॒स्मा स्व॒स्मासु॑ धारयाम॒ सीति॑ । \newline
52. धा॒र॒या॒म॒सीतीति॑ धारयामसि धारयाम॒सीति॒ मनो॒ मन॒ इति॑ धारयामसि धारयाम॒सीति॒ मनः॑ । \newline
53. इति॒ मनो॒ मन॒ इतीति॒ मन॑ ए॒वैव मन॒ इतीति॒ मन॑ ए॒व । \newline
54. मन॑ ए॒वैव मनो॒ मन॑ ए॒वात्मन् ना॒त्मन् ने॒व मनो॒ मन॑ ए॒वात्मन्न् । \newline
55. ए॒वात्मन् ना॒त्मन् ने॒वैवात्मन् दा॑धार दाधा रा॒त्मन् ने॒वैवात्मन् दा॑धार । \newline
56. आ॒त्मन् दा॑धार दाधा रा॒त्मन् ना॒त्मन् दा॑धार॒ न न दा॑धा रा॒त्मन् ना॒त्मन् दा॑धार॒ न । \newline
57. दा॒धा॒र॒ न न दा॑धार दाधार॒ न ग॒तम॑ना ग॒तम॑ना॒ न दा॑धार दाधार॒ न ग॒तम॑नाः । \newline
\pagebreak
\markright{ TS 6.6.7.3  \hfill https://www.vedavms.in \hfill}

\section{ TS 6.6.7.3 }

\textbf{TS 6.6.7.3 } \newline
\textbf{Samhita Paata} \newline

न ग॒तम॑ना भव॒त्यप॒ वै तृ॑तीयसव॒ने य॒ज्ञ्ः क्रा॑मतीजा॒ना-दनी॑जानम॒भ्या᳚-ग्नावैष्ण॒व्यर्चा घृ॒तस्य॑ यजत्य॒ग्निः सर्वा॑ दे॒वता॒ विष्णु॑र्य॒ज्ञो दे॒वता᳚श्चै॒व य॒ज्ञ्ं च॑ दाधारोपाꣳ॒॒शु य॑जति मिथुन॒त्वाय॑ ब्रह्मवा॒दिनो॑ वदन्ति मि॒त्रो य॒ज्ञ्स्य॒ स्वि॑ष्टं ॅयुवते॒ वरु॑णो॒ दुरि॑ष्टं॒ क्व॑ तर्.हि॑ य॒ज्ञ्ः क्व॑ यज॑मानो भव॒तीति॒ यन्मै᳚त्रावरु॒णीं ॅव॒शामा॒लभ॑ते मि॒त्रेणै॒व- [  ] \newline

\textbf{Pada Paata} \newline

न । ग॒तम॑ना॒ इति॑ ग॒त - म॒नाः॒ । भ॒व॒ति॒ । अपेति॑ । वै । तृ॒ती॒य॒स॒व॒न इति॑ तृतीय - स॒व॒ने । य॒ज्ञ्ः । क्रा॒म॒ति॒ । ई॒जा॒नात् । अनी॑जानम् । अ॒भीति॑ । आ॒ग्ना॒वै॒ष्ण॒व्येत्या᳚ग्ना - वै॒ष्ण॒व्या । ऋ॒चा । घृ॒तस्य॑ । य॒ज॒ति॒ । अ॒ग्निः । सर्वाः᳚ । दे॒वताः᳚ । विष्णुः॑ । य॒ज्ञ्ः । दे॒वताः᳚ । च॒ । ए॒व । य॒ज्ञ्म् । च॒ । दा॒धा॒र॒ । उ॒पाꣳ॒॒श्वित्यु॑प - अꣳ॒॒शु । य॒ज॒ति॒ । मि॒थु॒न॒त्वायेति॑ मिथुन - त्वाय॑ । ब्र॒ह्म॒वा॒दिन॒ इति॑ ब्रह्म - वा॒दिनः॑ । व॒द॒न्ति॒ । मि॒त्रः । य॒ज्ञ्स्य॑ । स्वि॑ष्ट॒मिति॒ सु-इ॒ष्ट॒म् । यु॒व॒ते॒ । वरु॑णः । दुरि॑ष्ट॒मिति॒ दुः - इ॒ष्ट॒म् । क्व॑ । तर्.हि॑ । य॒ज्ञ्ः । क्व॑ । यज॑मानः । भ॒व॒ति॒ । इति॑ । यत् । मै॒त्रा॒व॒रु॒णीमिति॑ मैत्रा - व॒रु॒णीम् । व॒शाम् । आ॒लभ॑त॒ इत्या᳚ - लभ॑ते । मि॒त्रेण॑ । ए॒व ।  \newline


\textbf{Krama Paata} \newline

न ग॒तम॑नाः । ग॒तम॑ना भवति । ग॒तम॑ना॒ इति॑ ग॒त - म॒नाः॒ । भ॒व॒त्यप॑ । अप॒ वै । वै तृ॑तीयसव॒ने । तृ॒ती॒य॒स॒व॒ने य॒ज्ञ्ः । तृ॒ती॒य॒स॒व॒न इति॑ तृतीय - स॒व॒ने । य॒ज्ञ्ः क्रा॑मति । क्रा॒म॒ती॒जा॒नात् । ई॒जा॒नादनी॑जानम् । अनी॑जानम॒भि । अ॒भ्या᳚ग्नावैष्ण॒व्या । आ॒ग्ना॒वै॒ष्ण॒व्यर्चा । आ॒ग्ना॒वै॒ष्ण॒व्येत्या᳚ग्ना - वै॒ष्ण॒व्या । ऋ॒चा घृ॒तस्य॑ । घृ॒तस्य॑ यजति । य॒ज॒त्य॒ग्निः । अ॒ग्निः सर्वाः᳚ । सर्वा॑ दे॒वताः᳚ । दे॒वता॒ विष्णुः॑ । विष्णु॑र् य॒ज्ञ्ः । य॒ज्ञो दे॒वताः᳚ । दे॒वता᳚श्च । चै॒व । ए॒व य॒ज्ञ्म् । य॒ज्ञ्म् च॑ । च॒ दा॒धा॒र॒ । दा॒धा॒रो॒पाꣳ॒॒शु । उ॒पाꣳ॒॒शु य॑जति । उ॒पाꣳ॒॒श्वित्यु॑प - अꣳ॒॒शु । य॒ज॒ति॒ मि॒थु॒न॒त्वाय॑ । मि॒थु॒न॒त्वाय॑ ब्रह्मवा॒दिनः॑ । मि॒थु॒न॒त्वायेति॑ मिथुन - त्वाय॑ । ब्र॒ह्म॒वा॒दिनो॑ वदन्ति । ब्र॒ह्म॒वा॒दिन॒ इति॑ ब्रह्म - वा॒दिनः॑ । व॒द॒न्ति॒ मि॒त्रः । मि॒त्रो य॒ज्ञ्स्य॑ । य॒ज्ञ्स्य॒ स्वि॑ष्टम् । स्वि॑ष्टम् ॅयुवते । स्वि॑ष्ट॒मिति॒ सु - इ॒ष्ट॒म् । यु॒व॒ते॒ वरु॑णः । वरु॑णो॒ दुरि॑ष्टम् । दुरि॑ष्ट॒म् क्व॑ । दुरि॑ष्ट॒मिति॒ दुः - इ॒ष्ट॒म् । क्व॑ तर्.हि॑ । तर्.हि॑ य॒ज्ञ्ः । य॒ज्ञ्ः क्व॑ । क्व॑ यज॑मानः । यज॑मानो भवति । भ॒व॒तीति॑ । इति॒ यत् । यन् मै᳚त्रावरु॒णीम् । मै॒त्रा॒व॒रु॒णीम् ॅव॒शाम् । मै॒त्रा॒व॒रु॒णीमिति॑ मैत्रा - व॒रु॒णीम् । व॒शामा॒लभ॑ते । आ॒लभ॑ते मि॒त्रेण॑ । आ॒लभ॑त॒ इत्या᳚ - लभ॑ते । मि॒त्रेणै॒व । ए॒व य॒ज्ञ्स्य॑ \newline

\textbf{Jatai Paata} \newline

1. न ग॒तम॑ना ग॒तम॑ना॒ न न ग॒तम॑नाः । \newline
2. ग॒तम॑ना भवति भवति ग॒तम॑ना ग॒तम॑ना भवति । \newline
3. ग॒तम॑ना॒ इति॑ ग॒त - म॒नाः॒ । \newline
4. भ॒व॒ त्यपाप॑ भवति भव॒ त्यप॑ । \newline
5. अप॒ वै वा अपाप॒ वै । \newline
6. वै तृ॑तीयसव॒ने तृ॑तीयसव॒ने वै वै तृ॑तीयसव॒ने । \newline
7. तृ॒ती॒य॒स॒व॒ने य॒ज्ञो य॒ज्ञ् स्तृ॑तीयसव॒ने तृ॑तीयसव॒ने य॒ज्ञ्ः । \newline
8. तृ॒ती॒य॒स॒व॒न इति॑ तृतीय - स॒व॒ने । \newline
9. य॒ज्ञ्ः क्रा॑मति क्रामति य॒ज्ञो य॒ज्ञ्ः क्रा॑मति । \newline
10. क्रा॒म॒ ती॒जा॒ना दी॑जा॒नात् क्रा॑मति क्राम तीजा॒नात् । \newline
11. ई॒जा॒ना दनी॑जान॒ मनी॑जान मीजा॒ना दी॑जा॒ना दनी॑जानम् । \newline
12. अनी॑जान म॒भ्य॑भ्य नी॑जान॒ मनी॑जान म॒भि । \newline
13. अ॒भ्या᳚ग्नावैष्ण॒व्या ऽऽग्ना॑वैष्ण॒व्या ऽभ्या᳚(1॒)भ्या᳚ ग्नावैष्ण॒व्या । \newline
14. आ॒ग्ना॒वै॒ष्ण॒व्य र्‌च र्‌चा ऽऽग्ना॑वैष्ण॒व्या ऽऽग्ना॑वैष्ण॒व्य र्‌चा । \newline
15. आ॒ग्ना॒वै॒ष्ण॒व्येत्या᳚ग्ना - वै॒ष्ण॒व्या । \newline
16. ऋ॒चा घृ॒तस्य॑ घृ॒तस्य॒ र्‌च र्‌चा घृ॒तस्य॑ । \newline
17. घृ॒तस्य॑ यजति यजति घृ॒तस्य॑ घृ॒तस्य॑ यजति । \newline
18. य॒ज॒ त्य॒ग्नि र॒ग्निर् य॑जति यज त्य॒ग्निः । \newline
19. अ॒ग्निः सर्वाः॒ सर्वा॑ अ॒ग्नि र॒ग्निः सर्वाः᳚ । \newline
20. सर्वा॑ दे॒वता॑ दे॒वताः॒ सर्वाः॒ सर्वा॑ दे॒वताः᳚ । \newline
21. दे॒वता॒ विष्णु॒र् विष्णु॑र् दे॒वता॑ दे॒वता॒ विष्णुः॑ । \newline
22. विष्णु॑र् य॒ज्ञो य॒ज्ञो विष्णु॒र् विष्णु॑र् य॒ज्ञ्ः । \newline
23. य॒ज्ञो दे॒वता॑ दे॒वता॑ य॒ज्ञो य॒ज्ञो दे॒वताः᳚ । \newline
24. दे॒वता᳚ श्च च दे॒वता॑ दे॒वता᳚ श्च । \newline
25. चै॒वैव च॑ चै॒व । \newline
26. ए॒व य॒ज्ञ्ं ॅय॒ज्ञ् मे॒वैव य॒ज्ञ्म् । \newline
27. य॒ज्ञ्म् च॑ च य॒ज्ञ्ं ॅय॒ज्ञ्म् च॑ । \newline
28. च॒ दा॒धा॒र॒ दा॒धा॒र॒ च॒ च॒ दा॒धा॒र॒ । \newline
29. दा॒धा॒ रो॒पाꣳ॒॒शू॑ पाꣳ॒॒शु दा॑धार दाधा रोपाꣳ॒॒शु । \newline
30. उ॒पाꣳ॒॒शु य॑जति यज त्युपाꣳ॒॒शू॑ पाꣳ॒॒शु य॑जति । \newline
31. उ॒पाꣳ॒॒श्वित्यु॑प - अꣳ॒॒शु । \newline
32. य॒ज॒ति॒ मि॒थु॒न॒त्वाय॑ मिथुन॒त्वाय॑ यजति यजति मिथुन॒त्वाय॑ । \newline
33. मि॒थु॒न॒त्वाय॑ ब्रह्मवा॒दिनो᳚ ब्रह्मवा॒दिनो॑ मिथुन॒त्वाय॑ मिथुन॒त्वाय॑ ब्रह्मवा॒दिनः॑ । \newline
34. मि॒थु॒न॒त्वायेति॑ मिथुन - त्वाय॑ । \newline
35. ब्र॒ह्म॒वा॒दिनो॑ वदन्ति वदन्ति ब्रह्मवा॒दिनो᳚ ब्रह्मवा॒दिनो॑ वदन्ति । \newline
36. ब्र॒ह्म॒वा॒दिन॒ इति॑ ब्रह्म - वा॒दिनः॑ । \newline
37. व॒द॒न्ति॒ मि॒त्रो मि॒त्रो व॑दन्ति वदन्ति मि॒त्रः । \newline
38. मि॒त्रो य॒ज्ञ्स्य॑ य॒ज्ञ्स्य॑ मि॒त्रो मि॒त्रो य॒ज्ञ्स्य॑ । \newline
39. य॒ज्ञ्स्य॒ स्वि॑ष्टꣳ॒॒ स्वि॑ष्टं ॅय॒ज्ञ्स्य॑ य॒ज्ञ्स्य॒ स्वि॑ष्टम् । \newline
40. स्वि॑ष्टं ॅयुवते युवते॒ स्वि॑ष्टꣳ॒॒ स्वि॑ष्टं ॅयुवते । \newline
41. स्वि॑ष्ट॒मिति॒ सु - इ॒ष्ट॒म् । \newline
42. यु॒व॒ते॒ वरु॑णो॒ वरु॑णो युवते युवते॒ वरु॑णः । \newline
43. वरु॑णो॒ दुरि॑ष्ट॒म् दुरि॑ष्टं॒ ॅवरु॑णो॒ वरु॑णो॒ दुरि॑ष्टम् । \newline
44. दुरि॑ष्ट॒म् क्वा᳚(1॒) क्व॑ दुरि॑ष्ट॒म् दुरि॑ष्ट॒म् क्व॑ । \newline
45. दुरि॑ष्ट॒मिति॒ दुः - इ॒ष्ट॒म् । \newline
46. क्व॑ तर्.हि॒ तर्.हि॒ क्वा᳚(1॒) क्व॑ तर्.हि॑ । \newline
47. तर्.हि॑ य॒ज्ञो य॒ज्ञ् स्तर्.हि॒ तर्.हि॑ य॒ज्ञ्ः । \newline
48. य॒ज्ञ्ः क्वा᳚(1॒) क्व॑ य॒ज्ञो य॒ज्ञ्ः क्व॑ । \newline
49. क्व॑ यज॑मानो॒ यज॑मानः॒ क्वा᳚(1॒) क्व॑ यज॑मानः । \newline
50. यज॑मानो भवति भवति॒ यज॑मानो॒ यज॑मानो भवति । \newline
51. भ॒व॒ती तीति॑ भवति भव॒ तीति॑ । \newline
52. इति॒ यद् यदितीति॒ यत् । \newline
53. यन् मै᳚त्रावरु॒णीम् मै᳚त्रावरु॒णीं ॅयद् यन् मै᳚त्रावरु॒णीम् । \newline
54. मै॒त्रा॒व॒रु॒णीं ॅव॒शां ॅव॒शाम् मै᳚त्रावरु॒णीम् मै᳚त्रावरु॒णीं ॅव॒शाम् । \newline
55. मै॒त्रा॒व॒रु॒णीमिति॑ मैत्रा - व॒रु॒णीम् । \newline
56. व॒शा मा॒लभ॑त आ॒लभ॑ते व॒शां ॅव॒शा मा॒लभ॑ते । \newline
57. आ॒लभ॑ते मि॒त्रेण॑ मि॒त्रेणा॒ लभ॑त आ॒लभ॑ते मि॒त्रेण॑ । \newline
58. आ॒लभ॑त॒ इत्या᳚ - लभ॑ते । \newline
59. मि॒त्रे णै॒वैव मि॒त्रेण॑ मि॒त्रे णै॒व । \newline
60. ए॒व य॒ज्ञ्स्य॑ य॒ज्ञ् स्यै॒वैव य॒ज्ञ्स्य॑ । \newline

\textbf{Ghana Paata } \newline

1. न ग॒तम॑ना ग॒तम॑ना॒ न न ग॒तम॑ना भवति भवति ग॒तम॑ना॒ न न ग॒तम॑ना भवति । \newline
2. ग॒तम॑ना भवति भवति ग॒तम॑ना ग॒तम॑ना भव॒ त्यपाप॑ भवति ग॒तम॑ना ग॒तम॑ना भव॒ त्यप॑ । \newline
3. ग॒तम॑ना॒ इति॑ ग॒त - म॒नाः॒ । \newline
4. भ॒व॒ त्यपाप॑ भवति भव॒ त्यप॒ वै वा अप॑ भवति भव॒ त्यप॒ वै । \newline
5. अप॒ वै वा अपाप॒ वै तृ॑तीयसव॒ने तृ॑तीयसव॒ने वा अपाप॒ वै तृ॑तीयसव॒ने । \newline
6. वै तृ॑तीयसव॒ने तृ॑तीयसव॒ने वै वै तृ॑तीयसव॒ने य॒ज्ञो य॒ज्ञ् स्तृ॑तीयसव॒ने वै वै तृ॑तीयसव॒ने य॒ज्ञ्ः । \newline
7. तृ॒ती॒य॒स॒व॒ने य॒ज्ञो य॒ज्ञ् स्तृ॑तीयसव॒ने तृ॑तीयसव॒ने य॒ज्ञ्ः क्रा॑मति क्रामति य॒ज्ञ् स्तृ॑तीयसव॒ने तृ॑तीयसव॒ने य॒ज्ञ्ः क्रा॑मति । \newline
8. तृ॒ती॒य॒स॒व॒न इति॑ तृतीय - स॒व॒ने । \newline
9. य॒ज्ञ्ः क्रा॑मति क्रामति य॒ज्ञो य॒ज्ञ्ः क्रा॑मतीजा॒ना दी॑जा॒नात् क्रा॑मति य॒ज्ञो य॒ज्ञ्ः क्रा॑मतीजा॒नात् । \newline
10. क्रा॒म॒ती॒जा॒ना दी॑जा॒नात् क्रा॑मति क्रामतीजा॒ना दनी॑जान॒ मनी॑जान मीजा॒नात् क्रा॑मति क्रामतीजा॒ना दनी॑जानम् । \newline
11. ई॒जा॒ना दनी॑जान॒ मनी॑जान मीजा॒ना दी॑जा॒ना दनी॑जान म॒भ्य॑भ्य नी॑जान मीजा॒ना दी॑जा॒ना दनी॑जान म॒भि । \newline
12. अनी॑जान म॒भ्य॑भ्य नी॑जान॒ मनी॑जान म॒भ्या᳚ ग्नावैष्ण॒व्या ऽऽग्ना॑वैष्ण॒व्या ऽभ्यनी॑जान॒ मनी॑जान म॒भ्या᳚ ग्नावैष्ण॒व्या । \newline
13. अ॒भ्या᳚ ग्नावैष्ण॒व्या ऽऽग्ना॑वैष्ण॒व्या ऽभ्या᳚(1॒)भ्या᳚ ग्नावैष्ण॒व्य र्‌च र्‌चा ऽऽग्ना॑वैष्ण॒व्या ऽभ्या᳚(1॒)भ्या᳚ ग्नावैष्ण॒व्य र्‌चा । \newline
14. आ॒ग्ना॒वै॒ष्ण॒व्य र्‌च र्‌चा ऽऽग्ना॑वैष्ण॒व्या ऽऽग्ना॑वैष्ण॒व्य र्‌चा घृ॒तस्य॑ घृ॒तस्य॒ र्‌चा ऽऽग्ना॑वैष्ण॒व्या ऽऽग्ना॑वैष्ण॒व्य र्‌चा घृ॒तस्य॑ । \newline
15. आ॒ग्ना॒वै॒ष्ण॒व्येत्या᳚ग्ना - वै॒ष्ण॒व्या । \newline
16. ऋ॒चा घृ॒तस्य॑ घृ॒तस्य॒ र्‌च र्‌चा घृ॒तस्य॑ यजति यजति घृ॒तस्य॒ र्‌च र्‌चा घृ॒तस्य॑ यजति । \newline
17. घृ॒तस्य॑ यजति यजति घृ॒तस्य॑ घृ॒तस्य॑ यज त्य॒ग्नि र॒ग्निर् य॑जति घृ॒तस्य॑ घृ॒तस्य॑ यज त्य॒ग्निः । \newline
18. य॒ज॒ त्य॒ग्नि र॒ग्निर् य॑जति यज त्य॒ग्निः सर्वाः॒ सर्वा॑ अ॒ग्निर् य॑जति यज त्य॒ग्निः सर्वाः᳚ । \newline
19. अ॒ग्निः सर्वाः॒ सर्वा॑ अ॒ग्नि र॒ग्निः सर्वा॑ दे॒वता॑ दे॒वताः॒ सर्वा॑ अ॒ग्नि र॒ग्निः सर्वा॑ दे॒वताः᳚ । \newline
20. सर्वा॑ दे॒वता॑ दे॒वताः॒ सर्वाः॒ सर्वा॑ दे॒वता॒ विष्णु॒र् विष्णु॑र् दे॒वताः॒ सर्वाः॒ सर्वा॑ दे॒वता॒ विष्णुः॑ । \newline
21. दे॒वता॒ विष्णु॒र् विष्णु॑र् दे॒वता॑ दे॒वता॒ विष्णु॑र् य॒ज्ञो य॒ज्ञो विष्णु॑र् दे॒वता॑ दे॒वता॒ विष्णु॑र् य॒ज्ञ्ः । \newline
22. विष्णु॑र् य॒ज्ञो य॒ज्ञो विष्णु॒र् विष्णु॑र् य॒ज्ञो दे॒वता॑ दे॒वता॑ य॒ज्ञो विष्णु॒र् विष्णु॑र् य॒ज्ञो दे॒वताः᳚ । \newline
23. य॒ज्ञो दे॒वता॑ दे॒वता॑ य॒ज्ञो य॒ज्ञो दे॒वता᳚ श्च च दे॒वता॑ य॒ज्ञो य॒ज्ञो दे॒वता᳚ श्च । \newline
24. दे॒वता᳚ श्च च दे॒वता॑ दे॒वता᳚ श्चै॒वैव च॑ दे॒वता॑ दे॒वता᳚ श्चै॒व । \newline
25. चै॒वैव च॑ चै॒व य॒ज्ञ्ं ॅय॒ज्ञ् मे॒व च॑ चै॒व य॒ज्ञ्म् । \newline
26. ए॒व य॒ज्ञ्ं ॅय॒ज्ञ् मे॒वैव य॒ज्ञ्म् च॑ च य॒ज्ञ् मे॒वैव य॒ज्ञ्म् च॑ । \newline
27. य॒ज्ञ्म् च॑ च य॒ज्ञ्ं ॅय॒ज्ञ्म् च॑ दाधार दाधार च य॒ज्ञ्ं ॅय॒ज्ञ्म् च॑ दाधार । \newline
28. च॒ दा॒धा॒र॒ दा॒धा॒र॒ च॒ च॒ दा॒धा॒ रो॒पाꣳ॒॒शू॑ पाꣳ॒॒शु दा॑धार च च दाधा रोपाꣳ॒॒शु । \newline
29. दा॒धा॒ रो॒पाꣳ॒॒शू॑ पाꣳ॒॒शु दा॑धार दाधा रोपाꣳ॒॒शु य॑जति यज त्युपाꣳ॒॒शु दा॑धार दाधा रोपाꣳ॒॒शु य॑जति । \newline
30. उ॒पाꣳ॒॒शु य॑जति यज त्युपाꣳ॒॒शू॑ पाꣳ॒॒शु य॑जति मिथुन॒त्वाय॑ मिथुन॒त्वाय॑ यज त्युपाꣳ॒॒शू॑ पाꣳ॒॒शु य॑जति मिथुन॒त्वाय॑ । \newline
31. उ॒पाꣳ॒॒श्वित्यु॑प - अꣳ॒॒शु । \newline
32. य॒ज॒ति॒ मि॒थु॒न॒त्वाय॑ मिथुन॒त्वाय॑ यजति यजति मिथुन॒त्वाय॑ ब्रह्मवा॒दिनो᳚ ब्रह्मवा॒दिनो॑ मिथुन॒त्वाय॑ यजति यजति मिथुन॒त्वाय॑ ब्रह्मवा॒दिनः॑ । \newline
33. मि॒थु॒न॒त्वाय॑ ब्रह्मवा॒दिनो᳚ ब्रह्मवा॒दिनो॑ मिथुन॒त्वाय॑ मिथुन॒त्वाय॑ ब्रह्मवा॒दिनो॑ वदन्ति वदन्ति ब्रह्मवा॒दिनो॑ मिथुन॒त्वाय॑ मिथुन॒त्वाय॑ ब्रह्मवा॒दिनो॑ वदन्ति । \newline
34. मि॒थु॒न॒त्वायेति॑ मिथुन - त्वाय॑ । \newline
35. ब्र॒ह्म॒वा॒दिनो॑ वदन्ति वदन्ति ब्रह्मवा॒दिनो᳚ ब्रह्मवा॒दिनो॑ वदन्ति मि॒त्रो मि॒त्रो व॑दन्ति ब्रह्मवा॒दिनो᳚ ब्रह्मवा॒दिनो॑ वदन्ति मि॒त्रः । \newline
36. ब्र॒ह्म॒वा॒दिन॒ इति॑ ब्रह्म - वा॒दिनः॑ । \newline
37. व॒द॒न्ति॒ मि॒त्रो मि॒त्रो व॑दन्ति वदन्ति मि॒त्रो य॒ज्ञ्स्य॑ य॒ज्ञ्स्य॑ मि॒त्रो व॑दन्ति वदन्ति मि॒त्रो य॒ज्ञ्स्य॑ । \newline
38. मि॒त्रो य॒ज्ञ्स्य॑ य॒ज्ञ्स्य॑ मि॒त्रो मि॒त्रो य॒ज्ञ्स्य॒ स्वि॑ष्टꣳ॒॒ स्वि॑ष्टं ॅय॒ज्ञ्स्य॑ मि॒त्रो मि॒त्रो य॒ज्ञ्स्य॒ स्वि॑ष्टम् । \newline
39. य॒ज्ञ्स्य॒ स्वि॑ष्टꣳ॒॒ स्वि॑ष्टं ॅय॒ज्ञ्स्य॑ य॒ज्ञ्स्य॒ स्वि॑ष्टं ॅयुवते युवते॒ स्वि॑ष्टं ॅय॒ज्ञ्स्य॑ य॒ज्ञ्स्य॒ स्वि॑ष्टं ॅयुवते । \newline
40. स्वि॑ष्टं ॅयुवते युवते॒ स्वि॑ष्टꣳ॒॒ स्वि॑ष्टं ॅयुवते॒ वरु॑णो॒ वरु॑णो युवते॒ स्वि॑ष्टꣳ॒॒ स्वि॑ष्टं ॅयुवते॒ वरु॑णः । \newline
41. स्वि॑ष्ट॒मिति॒ सु - इ॒ष्ट॒म् । \newline
42. यु॒व॒ते॒ वरु॑णो॒ वरु॑णो युवते युवते॒ वरु॑णो॒ दुरि॑ष्ट॒म् दुरि॑ष्टं॒ ॅवरु॑णो युवते युवते॒ वरु॑णो॒ दुरि॑ष्टम् । \newline
43. वरु॑णो॒ दुरि॑ष्ट॒म् दुरि॑ष्टं॒ ॅवरु॑णो॒ वरु॑णो॒ दुरि॑ष्ट॒म् क्वा᳚(1॒) क्व॑ दुरि॑ष्टं॒ ॅवरु॑णो॒ वरु॑णो॒ दुरि॑ष्ट॒म् क्व॑ । \newline
44. दुरि॑ष्ट॒म् क्वा᳚(1॒) क्व॑ दुरि॑ष्ट॒म् दुरि॑ष्ट॒म् क्व॑ तर्.हि॒ तर्.हि॒ क्व॑ दुरि॑ष्ट॒म् दुरि॑ष्ट॒म् क्व॑ तर्.हि॑ । \newline
45. दुरि॑ष्ट॒मिति॒ दुः - इ॒ष्ट॒म् । \newline
46. क्व॑ तर्.हि॒ तर्.हि॒ क्वा᳚(1॒) क्व॑ तर्.हि॑ य॒ज्ञो य॒ज्ञ् स्तर्.हि॒ क्वा᳚(1॒) क्व॑ तर्.हि॑ य॒ज्ञ्ः । \newline
47. तर्.हि॑ य॒ज्ञो य॒ज्ञ् स्तर्.हि॒ तर्.हि॑ य॒ज्ञ्ः क्वा᳚(1॒) क्व॑ य॒ज्ञ् स्तर्.हि॒ तर्.हि॑ य॒ज्ञ्ः क्व॑ । \newline
48. य॒ज्ञ्ः क्वा᳚(1॒) क्व॑ य॒ज्ञो य॒ज्ञ्ः क्व॑ यज॑मानो॒ यज॑मानः॒ क्व॑ य॒ज्ञो य॒ज्ञ्ः क्व॑ यज॑मानः । \newline
49. क्व॑ यज॑मानो॒ यज॑मानः॒ क्वा᳚(1॒) क्व॑ यज॑मानो भवति भवति॒ यज॑मानः॒ क्वा᳚(1॒) क्व॑ यज॑मानो भवति । \newline
50. यज॑मानो भवति भवति॒ यज॑मानो॒ यज॑मानो भव॒तीतीति॑ भवति॒ यज॑मानो॒ यज॑मानो भव॒तीति॑ । \newline
51. भ॒व॒तीतीति॑ भवति भव॒तीति॒ यद् यदिति॑ भवति भव॒तीति॒ यत् । \newline
52. इति॒ यद् यदितीति॒ यन् मै᳚त्रावरु॒णीम् मै᳚त्रावरु॒णीं ॅयदितीति॒ यन् मै᳚त्रावरु॒णीम् । \newline
53. यन् मै᳚त्रावरु॒णीम् मै᳚त्रावरु॒णीं ॅयद् यन् मै᳚त्रावरु॒णीं ॅव॒शां ॅव॒शाम् मै᳚त्रावरु॒णीं ॅयद् यन् मै᳚त्रावरु॒णीं ॅव॒शाम् । \newline
54. मै॒त्रा॒व॒रु॒णीं ॅव॒शां ॅव॒शाम् मै᳚त्रावरु॒णीम् मै᳚त्रावरु॒णीं ॅव॒शा मा॒लभ॑त आ॒लभ॑ते व॒शाम् मै᳚त्रावरु॒णीम् मै᳚त्रावरु॒णीं ॅव॒शा मा॒लभ॑ते । \newline
55. मै॒त्रा॒व॒रु॒णीमिति॑ मैत्रा - व॒रु॒णीम् । \newline
56. व॒शा मा॒लभ॑त आ॒लभ॑ते व॒शां ॅव॒शा मा॒लभ॑ते मि॒त्रेण॑ मि॒त्रेणा॒ लभ॑ते व॒शां ॅव॒शा मा॒लभ॑ते मि॒त्रेण॑ । \newline
57. आ॒लभ॑ते मि॒त्रेण॑ मि॒त्रेणा॒ लभ॑त आ॒लभ॑ते मि॒त्रेणै॒वैव मि॒त्रेणा॒ लभ॑त आ॒लभ॑ते मि॒त्रेणै॒व । \newline
58. आ॒लभ॑त॒ इत्या᳚ - लभ॑ते । \newline
59. मि॒त्रेणै॒वैव मि॒त्रेण॑ मि॒त्रेणै॒व य॒ज्ञ्स्य॑ य॒ज्ञ्स्यै॒व मि॒त्रेण॑ मि॒त्रेणै॒व य॒ज्ञ्स्य॑ । \newline
60. ए॒व य॒ज्ञ्स्य॑ य॒ज्ञ्स्यै॒वैव य॒ज्ञ्स्य॒ स्वि॑ष्टꣳ॒॒ स्वि॑ष्टं ॅय॒ज्ञ्स्यै॒वैव य॒ज्ञ्स्य॒ स्वि॑ष्टम् । \newline
\pagebreak
\markright{ TS 6.6.7.4  \hfill https://www.vedavms.in \hfill}

\section{ TS 6.6.7.4 }

\textbf{TS 6.6.7.4 } \newline
\textbf{Samhita Paata} \newline

य॒ज्ञ्स्य॒ स्वि॑ष्टꣳ शमयति॒ वरु॑णेन॒ दुरि॑ष्टं॒ नाऽऽ*र्ति॒मार्च्छ॑ति॒ यज॑मानो॒ यथा॒ वै लाङ्ग॑लेनो॒र्वरां᳚ प्रभि॒न्दन्-त्ये॒वमृ॑ख्सा॒मे य॒ज्ञ्ं प्र भि॑न्तो॒ यन्मै᳚त्रावरु॒णीं ॅव॒शामा॒लभ॑ते य॒ज्ञायै॒व प्रभि॑न्नाय म॒त्य॑म॒न्ववा᳚स्यति॒ शान्त्यै॑ या॒तया॑मानि॒ वा ए॒तस्य॒ छन्दाꣳ॑सि॒ य ई॑जा॒नः छन्द॑सामे॒ष रसो॒ यद्-व॒शा यन्मै᳚त्रावरु॒णीं ॅव॒शामा॒लभ॑ते॒ छन्दाꣳ॑स्ये॒व पुन॒रा प्री॑णा॒त्य ( ) या॑तयामत्वा॒याथो॒ छन्द॑स्स्वे॒व रसं॑ दधाति ॥ \newline

\textbf{Pada Paata} \newline

य॒ज्ञ्स्य॑ । स्वि॑ष्ट॒मिति॒ सु - इ॒ष्ट॒म् । श॒म॒य॒ति॒ । वरु॑णेन । दुरि॑ष्ट॒मिति॒ दुः - इ॒ष्ट॒म् । न । आर्ति᳚म् । एति॑ । ऋ॒च्छ॒ति॒ । यज॑मानः । यथा᳚ । वै । लाङ्ग॑लेन । उ॒र्वरा᳚म् । प्र॒भि॒न्दन्तीति॑ प्र - भि॒न्दन्ति॑ । ए॒वम् । ऋ॒ख्सा॒मे इत्यृ॑क् - सा॒मे । य॒ज्ञ्म् । प्रेति॑ । भि॒न्तः॒ । यत् । मै॒त्रा॒व॒रु॒णीमिति॑ मैत्रा - व॒रु॒णीम् । व॒शाम् । आ॒ल॑भत॒ इत्या᳚-लभ॑ते । य॒ज्ञाय॑ । ए॒व । प्रभि॑न्ना॒येति॒ प्र - भि॒न्ना॒य॒ । म॒त्य᳚म् । अ॒न्ववा᳚स्य॒तीत्य॑नु-अवा᳚स्यति । शान्त्यै᳚ । या॒तया॑मा॒नीति॑ या॒त-या॒मा॒नि॒ । वै । ए॒तस्य॑ । छन्दाꣳ॑सि । यः । ई॒जा॒नः । छन्द॑साम् । ए॒षः । रसः॑ । यत् । व॒शा । यत् । मै॒त्रा॒व॒रु॒णीमिति॑ मैत्रा - व॒रु॒णीम् । व॒शाम् । आ॒लभ॑त॒ इत्या᳚ - लभ॑ते । छन्दाꣳ॑सि । ए॒व । पुनः॑ । एति॑ । प्री॒णा॒ति॒ ( ) । अया॑तयामत्वा॒येत्यया॑तयाम - त्वा॒य॒ । अथो॒ इति॑ । छन्द॒स्स्विति॒ छन्दः॑ - सु॒ । ए॒व । रस᳚म् । द॒धा॒ति॒ ॥  \newline


\textbf{Krama Paata} \newline

य॒ज्ञ्स्य॒ स्वि॑ष्टम् । स्वि॑ष्टꣳ शमयति । स्वि॑ष्ट॒मिति॒ सु - इ॒ष्ट॒म् । श॒म॒य॒ति॒ वरु॑णेन । वरु॑णेन॒ दुरि॑ष्टम् । दुरि॑ष्ट॒म् न । दुरि॑ष्ट॒मिति॒ दुः - इ॒ष्ट॒म् । नार्ति᳚म् । आर्ति॒मा । आर्च्छ॑ति । ऋ॒च्छ॒ति॒ यज॑मानः । यज॑मानो॒ यथा᳚ । यथा॒ वै । वै लाङ्‍ग॑लेन । लाङ्‍ग॑लेनो॒र्वरा᳚म् । उ॒र्वरा᳚म् प्रभि॒न्दन्ति॑ । प्र॒भि॒न्दन्त्ये॒वम् । प्र॒भि॒न्दन्तीति॑ प्र - भि॒न्दन्ति॑ । ए॒वमृ॑ख्‌सा॒मे । ऋ॒ख्‌सा॒मे य॒ज्ञ्म् । ऋ॒ख्‌सा॒मे इत्यृ॑क् - सा॒मे । य॒ज्ञ्म् प्र । प्र भि॑न्तः । भि॒न्तो॒ यत् । यन् मै᳚त्रावरु॒णीम् । मै॒त्रा॒व॒रु॒णीम् ॅव॒शाम् । मै॒त्रा॒व॒रु॒णीमिति॑ मैत्रा - व॒रु॒णीम् । व॒शामा॒लभ॑ते । आ॒लभ॑ते य॒ज्ञाय॑ । आ॒लभ॑त॒ इत्या᳚ - लभ॑ते । य॒ज्ञायै॒व । ए॒व प्रभि॑न्नाय । प्रभि॑न्नाय म॒त्य᳚म् । प्रभि॑न्ना॒येति॒ प्र - भि॒न्ना॒य॒ । म॒त्य॑म॒न्ववा᳚स्यति । अ॒न्ववा᳚स्यति॒ शान्त्यै᳚ । अ॒न्ववा᳚स्य॒तीत्य॑नु - अवा᳚स्यति । शान्त्यै॑ या॒तया॑मानि । या॒तया॑मानि॒ वै । या॒तया॑मा॒नीति॑ या॒त - या॒मा॒नि॒ । वा ए॒तस्य॑ । ए॒तस्य॒ छन्दाꣳ॑सि । छन्दाꣳ॑सि॒ यः । य ई॑जा॒नः । ई॒जा॒नश्छन्द॑साम् । छन्द॑सामे॒षः । ए॒ष रसः॑ । रसो॒ यत् । यद् व॒शा । व॒शा यत् । यन् मै᳚त्रावरु॒णीम् । मै॒त्रा॒व॒रु॒णीम् ॅव॒शाम् । मै॒त्रा॒व॒रु॒णीमिति॑ मैत्रा - व॒रु॒णीम् । व॒शामा॒लभ॑ते । आ॒लभ॑ते॒ छन्दाꣳ॑सि । आ॒लभ॑त॒ इत्या᳚ - लभ॑ते । छन्दाꣳ॑स्ये॒व । ए॒व पुनः॑ । पुन॒रा । आ प्री॑णाति ( ) । प्री॒णा॒त्यया॑तयामत्वाय । अया॑तयामत्वा॒याथो᳚ । अया॑तयामत्वा॒ येत्यया॑तयाम - त्वा॒य॒ । अथो॒ छन्द॑स्सु । अथो॒ इत्यथो॑ । छन्द॑स्स्वे॒व । छन्द॒स्स्विति॒ छन्दः॑ - सु॒ । ए॒व रस᳚म् । रस॑म् दधाति । द॒धा॒तीति॑ दधाति । \newline

\textbf{Jatai Paata} \newline

1. य॒ज्ञ्स्य॒ स्वि॑ष्टꣳ॒॒ स्वि॑ष्टं ॅय॒ज्ञ्स्य॑ य॒ज्ञ्स्य॒ स्वि॑ष्टम् । \newline
2. स्वि॑ष्टꣳ शमयति शमयति॒ स्वि॑ष्टꣳ॒॒ स्वि॑ष्टꣳ शमयति । \newline
3. स्वि॑ष्ट॒मिति॒ सु - इ॒ष्ट॒म् । \newline
4. श॒म॒य॒ति॒ वरु॑णेन॒ वरु॑णेन शमयति शमयति॒ वरु॑णेन । \newline
5. वरु॑णेन॒ दुरि॑ष्ट॒म् दुरि॑ष्टं॒ ॅवरु॑णेन॒ वरु॑णेन॒ दुरि॑ष्टम् । \newline
6. दुरि॑ष्ट॒न् न न दुरि॑ष्ट॒म् दुरि॑ष्ट॒न् न । \newline
7. दुरि॑ष्ट॒मिति॒ दुः - इ॒ष्ट॒म् । \newline
8. नार्ति॒ मार्ति॒न् न नार्ति᳚म् । \newline
9. आर्ति॒ मा ऽऽर्ति॒ मार्ति॒ मा । \newline
10. आर्च्छ॑ त्यृच्छ॒ त्यार्च्छ॑ति । \newline
11. ऋ॒च्छ॒ति॒ यज॑मानो॒ यज॑मान ऋच्छ त्यृच्छति॒ यज॑मानः । \newline
12. यज॑मानो॒ यथा॒ यथा॒ यज॑मानो॒ यज॑मानो॒ यथा᳚ । \newline
13. यथा॒ वै वै यथा॒ यथा॒ वै । \newline
14. वै लाङ्ग॑लेन॒ लाङ्ग॑लेन॒ वै वै लाङ्ग॑लेन । \newline
15. लाङ्ग॑ले नो॒र्वरा॑ मु॒र्वरा॒म् ॅलाङ्ग॑लेन॒ लाङ्ग॑ले नो॒र्वरा᳚म् । \newline
16. उ॒र्वरा᳚म् प्रभि॒न्दन्ति॑ प्रभि॒न्दन् त्यु॒र्वरा॑ मु॒र्वरा᳚म् प्रभि॒न्दन्ति॑ । \newline
17. प्र॒भि॒न्द न्त्ये॒व मे॒वम् प्र॑भि॒न्दन्ति॑ प्रभि॒न्द न्त्ये॒वम् । \newline
18. प्र॒भि॒न्दन्तीति॑ प्र - भि॒न्दन्ति॑ । \newline
19. ए॒व मृ॑ख्सा॒मे ऋ॑ख्सा॒मे ए॒व मे॒व मृ॑ख्सा॒मे । \newline
20. ऋ॒ख्सा॒मे य॒ज्ञ्ं ॅय॒ज्ञ् मृ॑ख्सा॒मे ऋ॑ख्सा॒मे य॒ज्ञ्म् । \newline
21. ऋ॒ख्सा॒मे इत्यृ॑क् - सा॒मे । \newline
22. य॒ज्ञ्म् प्र प्र य॒ज्ञ्ं ॅय॒ज्ञ्म् प्र । \newline
23. प्र भि॑न्तो भिन्तः॒ प्र प्र भि॑न्तः । \newline
24. भि॒न्तो॒ यद् यद् भि॑न्तो भिन्तो॒ यत् । \newline
25. यन् मै᳚त्रावरु॒णीम् मै᳚त्रावरु॒णीं ॅयद् यन् मै᳚त्रावरु॒णीम् । \newline
26. मै॒त्रा॒व॒रु॒णीं ॅव॒शां ॅव॒शाम् मै᳚त्रावरु॒णीम् मै᳚त्रावरु॒णीं ॅव॒शाम् । \newline
27. मै॒त्रा॒व॒रु॒णीमिति॑ मैत्रा - व॒रु॒णीम् । \newline
28. व॒शा मा॒लभ॑त आ॒लभ॑ते व॒शां ॅव॒शा मा॒लभ॑ते । \newline
29. आ॒लभ॑ते य॒ज्ञाय॑ य॒ज्ञाया॒ लभ॑त आ॒लभ॑ते य॒ज्ञाय॑ । \newline
30. आ॒लभ॑त॒ इत्या᳚ - लभ॑ते । \newline
31. य॒ज्ञा यै॒वैव य॒ज्ञाय॑ य॒ज्ञा यै॒व । \newline
32. ए॒व प्रभि॑न्नाय॒ प्रभि॑न्ना यै॒वैव प्रभि॑न्नाय । \newline
33. प्रभि॑न्नाय म॒त्य॑म् म॒त्य॑म् प्रभि॑न्नाय॒ प्रभि॑न्नाय म॒त्य᳚म् । \newline
34. प्रभि॑न्ना॒येति॒ प्र - भि॒न्ना॒य॒ । \newline
35. म॒त्य॑ म॒न्ववा᳚स्य त्य॒न्ववा᳚स्यति म॒त्य॑म् म॒त्य॑ म॒न्ववा᳚स्यति । \newline
36. अ॒न्ववा᳚स्यति॒ शान्त्यै॒ शान्त्या॑ अ॒न्ववा᳚स्य त्य॒न्ववा᳚स्यति॒ शान्त्यै᳚ । \newline
37. अ॒न्ववा᳚स्य॒तीत्य॑नु - अवा᳚स्यति । \newline
38. शान्त्यै॑ या॒तया॑मानि या॒तया॑मानि॒ शान्त्यै॒ शान्त्यै॑ या॒तया॑मानि । \newline
39. या॒तया॑मानि॒ वै वै या॒तया॑मानि या॒तया॑मानि॒ वै । \newline
40. या॒तया॑मा॒नीति॑ या॒त - या॒मा॒नि॒ । \newline
41. वा ए॒त स्यै॒तस्य॒ वै वा ए॒तस्य॑ । \newline
42. ए॒तस्य॒ छन्दाꣳ॑सि॒ छन्दाꣳ॑ स्ये॒त स्यै॒तस्य॒ छन्दाꣳ॑सि । \newline
43. छन्दाꣳ॑सि॒ यो यश्छन्दाꣳ॑सि॒ छन्दाꣳ॑सि॒ यः । \newline
44. य ई॑जा॒न ई॑जा॒नो यो य ई॑जा॒नः । \newline
45. ई॒जा॒न श्छन्द॑सा॒म् छन्द॑सा मीजा॒न ई॑जा॒न श्छन्द॑साम् । \newline
46. छन्द॑सा मे॒ष ए॒ष छन्द॑सा॒म् छन्द॑सा मे॒षः । \newline
47. ए॒ष रसो॒ रस॑ ए॒ष ए॒ष रसः॑ । \newline
48. रसो॒ यद् यद् रसो॒ रसो॒ यत् । \newline
49. यद् व॒शा व॒शा यद् यद् व॒शा । \newline
50. व॒शा यद् यद् व॒शा व॒शा यत् । \newline
51. यन् मै᳚त्रावरु॒णीम् मै᳚त्रावरु॒णीं ॅयद् यन् मै᳚त्रावरु॒णीम् । \newline
52. मै॒त्रा॒व॒रु॒णीं ॅव॒शां ॅव॒शाम् मै᳚त्रावरु॒णीम् मै᳚त्रावरु॒णीं ॅव॒शाम् । \newline
53. मै॒त्रा॒व॒रु॒णीमिति॑ मैत्रा - व॒रु॒णीम् । \newline
54. व॒शा मा॒लभ॑त आ॒लभ॑ते व॒शां ॅव॒शा मा॒लभ॑ते । \newline
55. आ॒लभ॑ते॒ छन्दाꣳ॑सि॒ छन्दाꣳ॑स्या॒ लभ॑त आ॒लभ॑ते॒ छन्दाꣳ॑सि । \newline
56. आ॒लभ॑त॒ इत्या᳚ - लभ॑ते । \newline
57. छन्दाꣳ॑ स्ये॒वैव छन्दाꣳ॑सि॒ छन्दाꣳ॑ स्ये॒व । \newline
58. ए॒व पुनः॒ पुन॑ रे॒वैव पुनः॑ । \newline
59. पुन॒ रा पुनः॒ पुन॒ रा । \newline
60. आ प्री॑णाति प्रीणा॒त्या प्री॑णाति । \newline
61. प्री॒णा॒ त्यया॑तयामत्वा॒या या॑तयामत्वाय प्रीणाति प्रीणा॒ त्यया॑तयामत्वाय । \newline
62. अया॑तयामत्वा॒ याथो॒ अथो॒ अया॑तयामत्वा॒या या॑तयामत्वा॒ याथो᳚ । \newline
63. अया॑तयामत्वा॒येत्यया॑तयाम - त्वा॒य॒ । \newline
64. अथो॒ छन्द॑स्सु॒ छन्द॒स्स्वथो॒ अथो॒ छन्द॑स्सु । \newline
65. अथो॒ इत्यथो᳚ । \newline
66. छन्द॑स्स्वे॒वैव छन्द॑स्सु॒ छन्द॑स्स्वे॒व । \newline
67. छन्द॒स्स्विति॒ छन्दः॑ - सु॒ । \newline
68. ए॒व रसꣳ॒॒ रस॑ मे॒वैव रस᳚म् । \newline
69. रस॑म् दधाति दधाति॒ रसꣳ॒॒ रस॑म् दधाति । \newline
70. द॒धा॒तीति॑ दधाति । \newline

\textbf{Ghana Paata } \newline

1. य॒ज्ञ्स्य॒ स्वि॑ष्टꣳ॒॒ स्वि॑ष्टं ॅय॒ज्ञ्स्य॑ य॒ज्ञ्स्य॒ स्वि॑ष्टꣳ शमयति शमयति॒ स्वि॑ष्टं ॅय॒ज्ञ्स्य॑ य॒ज्ञ्स्य॒ स्वि॑ष्टꣳ शमयति । \newline
2. स्वि॑ष्टꣳ शमयति शमयति॒ स्वि॑ष्टꣳ॒॒ स्वि॑ष्टꣳ शमयति॒ वरु॑णेन॒ वरु॑णेन शमयति॒ स्वि॑ष्टꣳ॒॒ स्वि॑ष्टꣳ शमयति॒ वरु॑णेन । \newline
3. स्वि॑ष्ट॒मिति॒ सु - इ॒ष्ट॒म् । \newline
4. श॒म॒य॒ति॒ वरु॑णेन॒ वरु॑णेन शमयति शमयति॒ वरु॑णेन॒ दुरि॑ष्ट॒म् दुरि॑ष्टं॒ ॅवरु॑णेन शमयति शमयति॒ वरु॑णेन॒ दुरि॑ष्टम् । \newline
5. वरु॑णेन॒ दुरि॑ष्ट॒म् दुरि॑ष्टं॒ ॅवरु॑णेन॒ वरु॑णेन॒ दुरि॑ष्ट॒न् न न दुरि॑ष्टं॒ ॅवरु॑णेन॒ वरु॑णेन॒ दुरि॑ष्ट॒न् न । \newline
6. दुरि॑ष्ट॒न् न न दुरि॑ष्ट॒म् दुरि॑ष्ट॒न् नार्ति॒ मार्ति॒न् न दुरि॑ष्ट॒म् दुरि॑ष्ट॒म् नार्ति᳚म् । \newline
7. दुरि॑ष्ट॒मिति॒ दुः - इ॒ष्ट॒म् । \newline
8. नार्ति॒ मार्ति॒न् न नार्ति॒ मा ऽऽर्ति॒न् न नार्ति॒ मा । \newline
9. आर्ति॒ मा ऽऽर्ति॒ मार्ति॒ मार्च्छ॑ त्यृच्छ॒त्या ऽऽर्ति॒ मार्ति॒ मार्च्छ॑ति । \newline
10. आर्च्छ॑ त्यृच्छ॒ त्यार्च्छ॑ति॒ यज॑मानो॒ यज॑मान ऋच्छ॒ त्यार्च्छ॑ति॒ यज॑मानः । \newline
11. ऋ॒च्छ॒ति॒ यज॑मानो॒ यज॑मान ऋच्छ त्यृच्छति॒ यज॑मानो॒ यथा॒ यथा॒ यज॑मान ऋच्छ त्यृच्छति॒ यज॑मानो॒ यथा᳚ । \newline
12. यज॑मानो॒ यथा॒ यथा॒ यज॑मानो॒ यज॑मानो॒ यथा॒ वै वै यथा॒ यज॑मानो॒ यज॑मानो॒ यथा॒ वै । \newline
13. यथा॒ वै वै यथा॒ यथा॒ वै लाङ्ग॑लेन॒ लाङ्ग॑लेन॒ वै यथा॒ यथा॒ वै लाङ्ग॑लेन । \newline
14. वै लाङ्ग॑लेन॒ लाङ्ग॑लेन॒ वै वै लाङ्ग॑ले नो॒र्वरा॑ मु॒र्वरा॒म् ॅलाङ्ग॑लेन॒ वै वै लाङ्ग॑ले नो॒र्वरा᳚म् । \newline
15. लाङ्ग॑ले नो॒र्वरा॑ मु॒र्वरा॒म् ॅलाङ्ग॑लेन॒ लाङ्ग॑ले नो॒र्वरा᳚म् प्रभि॒न्दन्ति॑ प्रभि॒न्द न्त्यु॒र्वरा॒म् ॅलाङ्ग॑लेन॒ लाङ्ग॑ले नो॒र्वरा᳚म् प्रभि॒न्दन्ति॑ । \newline
16. उ॒र्वरा᳚म् प्रभि॒न्दन्ति॑ प्रभि॒न्द न्त्यु॒र्वरा॑ मु॒र्वरा᳚म् प्रभि॒न्द न्त्ये॒व मे॒वम् प्र॑भि॒न्द न्त्यु॒र्वरा॑ मु॒र्वरा᳚म् प्रभि॒न्द न्त्ये॒वम् । \newline
17. प्र॒भि॒न्द न्त्ये॒व मे॒वम् प्र॑भि॒न्दन्ति॑ प्रभि॒न्द न्त्ये॒व मृ॑ख्सा॒मे ऋ॑ख्सा॒मे ए॒वम् प्र॑भि॒न्दन्ति॑ प्रभि॒न्द
न्त्ये॒व मृ॑ख्सा॒मे । \newline
18. प्र॒भि॒न्दन्तीति॑ प्र - भि॒न्दन्ति॑ । \newline
19. ए॒व मृ॑ख्सा॒मे ऋ॑ख्सा॒मे ए॒व मे॒व मृ॑ख्सा॒मे य॒ज्ञ्ं ॅय॒ज्ञ् मृ॑ख्सा॒मे ए॒व मे॒व मृ॑ख्सा॒मे य॒ज्ञ्म् । \newline
20. ऋ॒ख्सा॒मे य॒ज्ञ्ं ॅय॒ज्ञ् मृ॑ख्सा॒मे ऋ॑ख्सा॒मे य॒ज्ञ्म् प्र प्र य॒ज्ञ् मृ॑ख्सा॒मे ऋ॑ख्सा॒मे य॒ज्ञ्म् प्र । \newline
21. ऋ॒ख्सा॒मे इत्यृ॑क् - सा॒मे । \newline
22. य॒ज्ञ्म् प्र प्र य॒ज्ञ्ं ॅय॒ज्ञ्म् प्र भि॑न्तो भिन्तः॒ प्र य॒ज्ञ्ं ॅय॒ज्ञ्म् प्र भि॑न्तः । \newline
23. प्र भि॑न्तो भिन्तः॒ प्र प्र भि॑न्तो॒ यद् यद् भि॑न्तः॒ प्र प्र भि॑न्तो॒ यत् । \newline
24. भि॒न्तो॒ यद् यद् भि॑न्तो भिन्तो॒ यन् मै᳚त्रावरु॒णीम् मै᳚त्रावरु॒णीं ॅयद् भि॑न्तो भिन्तो॒ यन् मै᳚त्रावरु॒णीम् । \newline
25. यन् मै᳚त्रावरु॒णीम् मै᳚त्रावरु॒णीं ॅयद् यन् मै᳚त्रावरु॒णीं ॅव॒शां ॅव॒शाम् मै᳚त्रावरु॒णीं ॅयद् यन् मै᳚त्रावरु॒णीं ॅव॒शाम् । \newline
26. मै॒त्रा॒व॒रु॒णीं ॅव॒शां ॅव॒शाम् मै᳚त्रावरु॒णीम् मै᳚त्रावरु॒णीं ॅव॒शा मा॒लभ॑त आ॒लभ॑ते व॒शाम् मै᳚त्रावरु॒णीम् मै᳚त्रावरु॒णीं ॅव॒शा मा॒लभ॑ते । \newline
27. मै॒त्रा॒व॒रु॒णीमिति॑ मैत्रा - व॒रु॒णीम् । \newline
28. व॒शा मा॒लभ॑त आ॒लभ॑ते व॒शां ॅव॒शा मा॒लभ॑ते य॒ज्ञाय॑ य॒ज्ञाया॒ लभ॑ते व॒शां ॅव॒शा मा॒लभ॑ते य॒ज्ञाय॑ । \newline
29. आ॒लभ॑ते य॒ज्ञाय॑ य॒ज्ञाया॒ लभ॑त आ॒लभ॑ते य॒ज्ञायै॒वैव य॒ज्ञाया॒ लभ॑त आ॒लभ॑ते य॒ज्ञायै॒व । \newline
30. आ॒लभ॑त॒ इत्या᳚ - लभ॑ते । \newline
31. य॒ज्ञायै॒ वैव य॒ज्ञाय॑ य॒ज्ञायै॒व प्रभि॑न्नाय॒ प्रभि॑न्ना यै॒व य॒ज्ञाय॑ य॒ज्ञायै॒व प्रभि॑न्नाय । \newline
32. ए॒व प्रभि॑न्नाय॒ प्रभि॑न्नायै॒वैव प्रभि॑न्नाय म॒त्य॑म् म॒त्य॑म् प्रभि॑न्नायै॒वैव प्रभि॑न्नाय म॒त्य᳚म् । \newline
33. प्रभि॑न्नाय म॒त्य॑म् म॒त्य॑म् प्रभि॑न्नाय॒ प्रभि॑न्नाय म॒त्य॑ म॒न्ववा᳚स्य त्य॒न्ववा᳚स्यति म॒त्य॑म् प्रभि॑न्नाय॒ प्रभि॑न्नाय म॒त्य॑ म॒न्ववा᳚स्यति । \newline
34. प्रभि॑न्ना॒येति॒ प्र - भि॒न्ना॒य॒ । \newline
35. म॒त्य॑ म॒न्ववा᳚स्य त्य॒न्ववा᳚स्यति म॒त्य॑म् म॒त्य॑ म॒न्ववा᳚स्यति॒ शान्त्यै॒ शान्त्या॑ अ॒न्ववा᳚स्यति म॒त्य॑म् म॒त्य॑ म॒न्ववा᳚स्यति॒ शान्त्यै᳚ । \newline
36. अ॒न्ववा᳚स्यति॒ शान्त्यै॒ शान्त्या॑ अ॒न्ववा᳚स्य त्य॒न्ववा᳚स्यति॒ शान्त्यै॑ या॒तया॑मानि या॒तया॑मानि॒ शान्त्या॑ अ॒न्ववा᳚स्य त्य॒न्ववा᳚स्यति॒ शान्त्यै॑ या॒तया॑मानि । \newline
37. अ॒न्ववा᳚स्य॒तीत्य॑नु - अवा᳚स्यति । \newline
38. शान्त्यै॑ या॒तया॑मानि या॒तया॑मानि॒ शान्त्यै॒ शान्त्यै॑ या॒तया॑मानि॒ वै वै या॒तया॑मानि॒ शान्त्यै॒ शान्त्यै॑ या॒तया॑मानि॒ वै । \newline
39. या॒तया॑मानि॒ वै वै या॒तया॑मानि या॒तया॑मानि॒ वा ए॒त स्यै॒तस्य॒ वै या॒तया॑मानि या॒तया॑मानि॒ वा ए॒तस्य॑ । \newline
40. या॒तया॑मा॒नीति॑ या॒त - या॒मा॒नि॒ । \newline
41. वा ए॒त स्यै॒तस्य॒ वै वा ए॒तस्य॒ छन्दाꣳ॑सि॒ छन्दाꣳ॑ स्ये॒तस्य॒ वै वा ए॒तस्य॒ छन्दाꣳ॑सि । \newline
42. ए॒तस्य॒ छन्दाꣳ॑सि॒ छन्दाꣳ॑ स्ये॒त स्यै॒तस्य॒ छन्दाꣳ॑सि॒ यो यश्छन्दाꣳ॑ स्ये॒त स्यै॒तस्य॒ छन्दाꣳ॑सि॒ यः । \newline
43. छन्दाꣳ॑सि॒ यो यश्छन्दाꣳ॑सि॒ छन्दाꣳ॑सि॒ य ई॑जा॒न ई॑जा॒नो यश्छन्दाꣳ॑सि॒ छन्दाꣳ॑सि॒ य ई॑जा॒नः । \newline
44. य ई॑जा॒न ई॑जा॒नो यो य ई॑जा॒न श्छन्द॑सा॒म् छन्द॑सा मीजा॒नो यो य ई॑जा॒न श्छन्द॑साम् । \newline
45. ई॒जा॒न श्छन्द॑सा॒म् छन्द॑सा मीजा॒न ई॑जा॒न श्छन्द॑सा मे॒ष ए॒ष छन्द॑सा मीजा॒न ई॑जा॒न श्छन्द॑सा मे॒षः । \newline
46. छन्द॑सा मे॒ष ए॒ष छन्द॑सा॒म् छन्द॑सा मे॒ष रसो॒ रस॑ ए॒ष छन्द॑सा॒म् छन्द॑सा मे॒ष रसः॑ । \newline
47. ए॒ष रसो॒ रस॑ ए॒ष ए॒ष रसो॒ यद् यद् रस॑ ए॒ष ए॒ष रसो॒ यत् । \newline
48. रसो॒ यद् यद् रसो॒ रसो॒ यद् व॒शा व॒शा यद् रसो॒ रसो॒ यद् व॒शा । \newline
49. यद् व॒शा व॒शा यद् यद् व॒शा यद् यद् व॒शा यद् यद् व॒शा यत् । \newline
50. व॒शा यद् यद् व॒शा व॒शा यन् मै᳚त्रावरु॒णीम् मै᳚त्रावरु॒णीं ॅयद् व॒शा व॒शा यन् मै᳚त्रावरु॒णीम् । \newline
51. यन् मै᳚त्रावरु॒णीम् मै᳚त्रावरु॒णीं ॅयद् यन् मै᳚त्रावरु॒णीं ॅव॒शां ॅव॒शाम् मै᳚त्रावरु॒णीं ॅयद् यन् मै᳚त्रावरु॒णीं ॅव॒शाम् । \newline
52. मै॒त्रा॒व॒रु॒णीं ॅव॒शां ॅव॒शाम् मै᳚त्रावरु॒णीम् मै᳚त्रावरु॒णीं ॅव॒शा मा॒लभ॑त आ॒लभ॑ते व॒शाम् मै᳚त्रावरु॒णीम् मै᳚त्रावरु॒णीं ॅव॒शा मा॒लभ॑ते । \newline
53. मै॒त्रा॒व॒रु॒णीमिति॑ मैत्रा - व॒रु॒णीम् । \newline
54. व॒शा मा॒लभ॑त आ॒लभ॑ते व॒शां ॅव॒शा मा॒लभ॑ते॒ छन्दाꣳ॑सि॒ छन्दाꣳ॑ स्या॒लभ॑ते व॒शां ॅव॒शा मा॒लभ॑ते॒ छन्दाꣳ॑सि । \newline
55. आ॒लभ॑ते॒ छन्दाꣳ॑सि॒ छन्दाꣳ॑ स्या॒लभ॑त आ॒लभ॑ते॒ छन्दाꣳ॑ स्ये॒वैव छन्दाꣳ॑ स्या॒लभ॑त आ॒लभ॑ते॒ छन्दाꣳ॑ स्ये॒व । \newline
56. आ॒लभ॑त॒ इत्या᳚ - लभ॑ते । \newline
57. छन्दाꣳ॑ स्ये॒वैव छन्दाꣳ॑सि॒ छन्दाꣳ॑ स्ये॒व पुनः॒ पुन॑ रे॒व छन्दाꣳ॑सि॒ छन्दाꣳ॑ स्ये॒व पुनः॑ । \newline
58. ए॒व पुनः॒ पुन॑ रे॒वैव पुन॒ रा पुन॑ रे॒वैव पुन॒ रा । \newline
59. पुन॒ रा पुनः॒ पुन॒ रा प्री॑णाति प्रीणा॒त्या पुनः॒ पुन॒ रा प्री॑णाति । \newline
60. आ प्री॑णाति प्रीणा॒त्या प्री॑णा॒ त्यया॑तयामत्वा॒या या॑तयामत्वाय प्रीणा॒त्या प्री॑णा॒ त्यया॑तयामत्वाय । \newline
61. प्री॒णा॒ त्यया॑तयामत्वा॒या या॑तयामत्वाय प्रीणाति प्रीणा॒ त्यया॑तयामत्वा॒ याथो॒ अथो॒ अया॑तयामत्वाय प्रीणाति प्रीणा॒ त्यया॑तयामत्वा॒ याथो᳚ । \newline
62. अया॑तयामत्वा॒ याथो॒ अथो॒ अया॑तयामत्वा॒या या॑तयामत्वा॒याथो॒ छन्द॑स्सु॒ छन्द॒स्स्वथो॒ अया॑तयामत्वा॒या या॑तयामत्वा॒ याथो॒ छन्द॑स्सु । \newline
63. अया॑तयामत्वा॒येत्यया॑तयाम - त्वा॒य॒ । \newline
64. अथो॒ छन्द॑स्सु॒ छन्द॒स्स्वथो॒ अथो॒ छन्द॑स्स्वे॒वैव छन्द॒स्स्वथो॒ अथो॒ छन्द॑स्स्वे॒व । \newline
65. अथो॒ इत्यथो᳚ । \newline
66. छन्द॑स्स्वे॒वैव छन्द॑स्सु॒ छन्द॑स्स्वे॒व रसꣳ॒॒ रस॑ मे॒व छन्द॑स्सु॒ छन्द॑स्स्वे॒व रस᳚म् । \newline
67. छन्द॒स्स्विति॒ छन्दः॑ - सु॒ । \newline
68. ए॒व रसꣳ॒॒ रस॑ मे॒वैव रस॑म् दधाति दधाति॒ रस॑ मे॒वैव रस॑म् दधाति । \newline
69. रस॑म् दधाति दधाति॒ रसꣳ॒॒ रस॑म् दधाति । \newline
70. द॒धा॒तीति॑ दधाति । \newline
\pagebreak
\markright{ TS 6.6.8.1  \hfill https://www.vedavms.in \hfill}

\section{ TS 6.6.8.1 }

\textbf{TS 6.6.8.1 } \newline
\textbf{Samhita Paata} \newline

दे॒वा वा इ॑न्द्रि॒यं ॅवी॒र्यां᳚(1॒) ॅव्य॑भजन्त॒ ततो॒ यद॒त्यशि॑ष्यत॒ तद॑तिग्रा॒ह्या॑ अभव॒न् तद॑तिग्रा॒ह्या॑णा-मतिग्राह्य॒त्वं ॅयद॑तिग्रा॒ह्या॑ गृ॒ह्यन्त॑ इन्द्रि॒यमे॒व तद्-वी॒र्यं॑ ॅयज॑मान आ॒त्मन् ध॑त्ते॒ तेज॑ आग्ने॒येने᳚न्द्रि॒य-मै॒न्द्रेण॑ ब्रह्मवर्च॒सꣳ सौ॒र्येणो॑प॒स्तंभ॑नं॒ ॅवा ए॒तद्-य॒ज्ञ्स्य॒ यद॑तिग्रा॒ह्या᳚श्च॒क्रे पृ॒ष्ठानि॒ यत् पृष्ठ्ये॒ न गृ॑ह्णी॒यात् प्राञ्चं॑ ॅय॒ज्ञ्ं पृ॒ष्ठानि॒ सꣳ शृ॑णीयु॒र्यदु॒क्थ्ये॑- [  ] \newline

\textbf{Pada Paata} \newline

दे॒वा । वै । इ॒न्द्रि॒यम् । वी॒र्या᳚म् । वीति॑ । अ॒भ॒ज॒न्त॒ । ततः॑ । यत् । अ॒त्यशि॑ष्य॒तेत्य॑ति - अशि॑ष्यत । तत् । अ॒ति॒ग्रा॒ह्या॑ इत्य॑ति-ग्रा॒ह्याः᳚ । अ॒भ॒व॒न्न् । तत् । अ॒ति॒ग्रा॒ह्या॑णा॒मित्य॑ति - ग्रा॒ह्या॑णाम् । अ॒ति॒ग्रा॒ह्य॒त्वमित्य॑तिग्राह्य - त्वम् । यत् । अ॒ति॒ग्रा॒ह्या॑ इत्य॑ति-ग्रा॒ह्याः᳚ । गृ॒ह्यन्ते᳚ । इ॒न्द्रि॒यम् । ए॒व । तत् । वी॒र्य᳚म् । यज॑मानः । आ॒त्मन्न् । ध॒त्ते॒ । तेजः॑ । आ॒ग्ने॒येन॑ । इ॒न्द्रि॒यम् । ऐ॒न्द्रेण॑ । ब्र॒ह्म॒व॒र्च॒समिति॑ ब्रह्म - व॒र्च॒सम् । सौ॒र्येण॑ । उ॒प॒स्तंभ॑न॒मित्यु॑प - स्तंभ॑नम् । वै । ए॒तत् । य॒ज्ञ्स्य॑ । यत् । अ॒ति॒ग्रा॒ह्या॑ इत्य॑ति - ग्रा॒ह्याः᳚ । च॒क्रे इति॑ । पृ॒ष्ठानि॑ । यत् । पृष्ठ्ये᳚ । न । गृ॒ह्णी॒यात् । प्राञ्च᳚म् । य॒ज्ञ्म् । पृ॒ष्ठानि॑ । समिति॑ । शृ॒णी॒युः॒ । यत् । उ॒क्थ्ये᳚ ।  \newline


\textbf{Krama Paata} \newline

दे॒वा वै । वा इ॑न्द्रि॒यम् । इ॒न्द्रि॒यम् ॅवी॒र्य᳚म् । वी॒र्य॑म् ॅवि । व्य॑भजन्त । अ॒भ॒ज॒न्त॒ ततः॑ । ततो॒ यत् । यद॒त्यशि॑ष्यत । अ॒त्यशि॑ष्यत॒ तत् । अ॒त्यशि॑ष्य॒तेत्य॑ति - अशि॑ष्यत । तद॑तिग्रा॒ह्याः᳚ । अ॒ति॒ग्रा॒ह्या॑ अभवन्न् । अ॒ति॒ग्रा॒ह्या॑ इत्य॑ति - ग्रा॒ह्याः᳚ । अ॒भ॒व॒न् तत् । तद॑तिग्रा॒ह्या॑णाम् । अ॒ति॒ग्रा॒ह्या॑णा,मतिग्राह्य॒त्वम् । अ॒ति॒ग्रा॒ह्या॑णा॒मित्य॑ति - ग्रा॒ह्या॑णाम् । अ॒ति॒ग्रा॒ह्य॒त्वम् ॅयत् । अ॒ति॒ग्रा॒ह्य॒त्वमित्य॑तिग्राह्य - त्वम् । यद॑तिग्रा॒ह्याः᳚ । अ॒ति॒ग्रा॒ह्या॑ गृ॒ह्यन्ते᳚ । अ॒ति॒ग्रा॒ह्या॑ इत्य॑ति - ग्रा॒ह्याः᳚ । गृ॒ह्यन्त॑ इन्द्रि॒यम् । इ॒न्द्रि॒यमे॒व । ए॒व तत् । तद् वी॒र्य᳚म् । वी॒र्य॑म् ॅयज॑मानः । यज॑मान आ॒त्मन्न् । आ॒त्मन् ध॑त्ते । ध॒त्ते॒ तेजः॑ । तेज॑ आग्ने॒येन॑ । आ॒ग्ने॒येने᳚न्द्रि॒यम् । इ॒न्द्रि॒यमै॒न्द्रेण॑ । ऐ॒न्द्रेण॑ ब्रह्मवर्च॒सम् । ब्र॒ह्म॒व॒र्च॒सꣳ सौ॒र्येण॑ । ब्र॒ह्म॒व॒र्च॒समिति॑ ब्रह्म - व॒र्च॒सम् । सौ॒र्येणो॑प॒स्तम्भ॑नम् । उ॒प॒स्तम्भ॑न॒म् ॅवै । उ॒प॒स्तम्भ॑न॒मित्यु॑प - स्तम्भ॑नम् । वा ए॒तत् । ए॒तद् य॒ज्ञ्स्य॑ । य॒ज्ञ्स्य॒ यत् । यद॑तिग्रा॒ह्याः᳚ । अ॒ति॒ग्रा॒ह्या᳚श्च॒क्रे । अ॒ति॒ग्रा॒ह्या॑ इत्य॑ति - ग्रा॒ह्याः᳚ । च॒क्रे पृ॒ष्ठानि॑ । च॒क्रे इति॑ च॒क्रे । पृ॒ष्ठानि॒ यत् । यत् पृष्ठ्‍ये᳚ । पृष्ठ्‍ये॒ न । न गृ॑ह्णी॒यात् । गृ॒ह्णी॒यात् प्राञ्च᳚म् । प्राञ्च॑म् ॅय॒ज्ञ्म् । य॒ज्ञ्म् पृ॒ष्ठानि॑ । पृ॒ष्ठानि॒ सम् । सꣳ शृ॑णीयुः । शृ॒णी॒यु॒र् यत् । यदु॒क्थ्ये᳚ । उ॒क्थ्ये॑ गृह्णी॒यात् \newline

\textbf{Jatai Paata} \newline

1. दे॒वा वै वै दे॒वा दे॒वा वै । \newline
2. वा इ॑न्द्रि॒य मि॑न्द्रि॒यं ॅवै वा इ॑न्द्रि॒यम् । \newline
3. इ॒न्द्रि॒यं ॅवी॒र्यं॑ ॅवी॒र्य॑ मिन्द्रि॒य मि॑न्द्रि॒यं ॅवी॒र्य᳚म् । \newline
4. वी॒र्यं॑ ॅवि वि वी॒र्यं॑ ॅवी॒र्यं॑ ॅवि । \newline
5. व्य॑भजन्ता भजन्त॒ वि व्य॑भजन्त । \newline
6. अ॒भ॒ज॒न्त॒ तत॒ स्ततो॑ ऽभजन्ता भजन्त॒ ततः॑ । \newline
7. ततो॒ यद् यत् तत॒ स्ततो॒ यत् । \newline
8. यद॒त्यशि॑ष्यता॒ त्यशि॑ष्यत॒ यद् यद॒त्यशि॑ष्यत । \newline
9. अ॒त्यशि॑ष्यत॒ तत् तद॒त्यशि॑ष्यता॒ त्यशि॑ष्यत॒ तत् । \newline
10. अ॒त्यशि॑ष्य॒तेत्य॑ति - अशि॑ष्यत । \newline
11. तद॑तिग्रा॒ह्या॑ अतिग्रा॒ह्या᳚ स्तत् तद॑तिग्रा॒ह्याः᳚ । \newline
12. अ॒ति॒ग्रा॒ह्या॑ अभवन् नभवन् नतिग्रा॒ह्या॑ अतिग्रा॒ह्या॑ अभवन्न् । \newline
13. अ॒ति॒ग्रा॒ह्या॑ इत्य॑ति - ग्रा॒ह्याः᳚ । \newline
14. अ॒भ॒व॒न् तत् तद॑भवन् नभव॒न् तत् । \newline
15. तद॑तिग्रा॒ह्या॑णा मतिग्रा॒ह्या॑णा॒म् तत् तद॑तिग्रा॒ह्या॑णाम् । \newline
16. अ॒ति॒ग्रा॒ह्या॑णा मतिग्राह्य॒त्व म॑तिग्राह्य॒त्व म॑तिग्रा॒ह्या॑णा मतिग्रा॒ह्या॑णा मतिग्राह्य॒त्वम् । \newline
17. अ॒ति॒ग्रा॒ह्या॑णा॒मित्य॑ति - ग्रा॒ह्या॑णाम् । \newline
18. अ॒ति॒ग्रा॒ह्य॒त्वं ॅयद् यद॑तिग्राह्य॒त्व म॑तिग्राह्य॒त्वं ॅयत् । \newline
19. अ॒ति॒ग्रा॒ह्य॒त्वमित्य॑तिग्राह्य - त्वम् । \newline
20. यद॑तिग्रा॒ह्या॑ अतिग्रा॒ह्या॑ यद् यद॑तिग्रा॒ह्याः᳚ । \newline
21. अ॒ति॒ग्रा॒ह्या॑ गृ॒ह्यन्ते॑ गृ॒ह्यन्ते॑ ऽतिग्रा॒ह्या॑ अतिग्रा॒ह्या॑ गृ॒ह्यन्ते᳚ । \newline
22. अ॒ति॒ग्रा॒ह्या॑ इत्य॑ति - ग्रा॒ह्याः᳚ । \newline
23. गृ॒ह्यन्त॑ इन्द्रि॒य मि॑न्द्रि॒यम् गृ॒ह्यन्ते॑ गृ॒ह्यन्त॑ इन्द्रि॒यम् । \newline
24. इ॒न्द्रि॒य मे॒वैवेन्द्रि॒य मि॑न्द्रि॒य मे॒व । \newline
25. ए॒व तत् तदे॒ वैव तत् । \newline
26. तद् वी॒र्यं॑ ॅवी॒र्य॑म् तत् तद् वी॒र्य᳚म् । \newline
27. वी॒र्यं॑ ॅयज॑मानो॒ यज॑मानो वी॒र्यं॑ ॅवी॒र्यं॑ ॅयज॑मानः । \newline
28. यज॑मान आ॒त्मन् ना॒त्मन्. यज॑मानो॒ यज॑मान आ॒त्मन्न् । \newline
29. आ॒त्मन् ध॑त्ते धत्त आ॒त्मन् ना॒त्मन् ध॑त्ते । \newline
30. ध॒त्ते॒ तेज॒ स्तेजो॑ धत्ते धत्ते॒ तेजः॑ । \newline
31. तेज॑ आग्ने॒येना᳚ ग्ने॒येन॒ तेज॒ स्तेज॑ आग्ने॒येन॑ । \newline
32. आ॒ग्ने॒ये ने᳚न्द्रि॒य मि॑न्द्रि॒य मा᳚ग्ने॒येना᳚ ग्ने॒ये ने᳚न्द्रि॒यम् । \newline
33. इ॒न्द्रि॒य मै॒न्द्रे णै॒न्द्रे णे᳚न्द्रि॒य मि॑न्द्रि॒य मै॒न्द्रेण॑ । \newline
34. ऐ॒न्द्रेण॑ ब्रह्मवर्च॒सम् ब्र॑ह्मवर्च॒स मै॒न्द्रे णै॒न्द्रेण॑ ब्रह्मवर्च॒सम् । \newline
35. ब्र॒ह्म॒व॒र्च॒सꣳ सौ॒र्येण॑ सौ॒र्येण॑ ब्रह्मवर्च॒सम् ब्र॑ह्मवर्च॒सꣳ सौ॒र्येण॑ । \newline
36. ब्र॒ह्म॒व॒र्च॒समिति॑ ब्रह्म - व॒र्च॒सम् । \newline
37. सौ॒र्येणो॑ प॒स्तंभ॑न मुप॒स्तंभ॑नꣳ सौ॒र्येण॑ सौ॒र्येणो॑ प॒स्तंभ॑नम् । \newline
38. उ॒प॒स्तंभ॑नं॒ ॅवै वा उ॑प॒स्तंभ॑न मुप॒स्तंभ॑नं॒ ॅवै । \newline
39. उ॒प॒स्तंभ॑न॒मित्यु॑प - स्तंभ॑नम् । \newline
40. वा ए॒त दे॒तद् वै वा ए॒तत् । \newline
41. ए॒तद् य॒ज्ञ्स्य॑ य॒ज्ञ् स्यै॒त दे॒तद् य॒ज्ञ्स्य॑ । \newline
42. य॒ज्ञ्स्य॒ यद् यद् य॒ज्ञ्स्य॑ य॒ज्ञ्स्य॒ यत् । \newline
43. यद॑तिग्रा॒ह्या॑ अतिग्रा॒ह्या॑ यद् यद॑तिग्रा॒ह्याः᳚ । \newline
44. अ॒ति॒ग्रा॒ह्या᳚ श्च॒क्रे च॒क्रे अ॑तिग्रा॒ह्या॑ अतिग्रा॒ह्या᳚ श्च॒क्रे । \newline
45. अ॒ति॒ग्रा॒ह्या॑ इत्य॑ति - ग्रा॒ह्याः᳚ । \newline
46. च॒क्रे पृ॒ष्ठानि॑ पृ॒ष्ठानि॑ च॒क्रे च॒क्रे पृ॒ष्ठानि॑ । \newline
47. च॒क्रे इति॑ च॒क्रे । \newline
48. पृ॒ष्ठानि॒ यद् यत् पृ॒ष्ठानि॑ पृ॒ष्ठानि॒ यत् । \newline
49. यत् पृष्ठ्ये॒ पृष्ठ्ये॒ यद् यत् पृष्ठ्ये᳚ । \newline
50. पृष्ठ्ये॒ न न पृष्ठ्ये॒ पृष्ठ्ये॒ न । \newline
51. न गृ॑ह्णी॒याद् गृ॑ह्णी॒यान् न न गृ॑ह्णी॒यात् । \newline
52. गृ॒ह्णी॒यात् प्राञ्च॒म् प्राञ्च॑म् गृह्णी॒याद् गृ॑ह्णी॒यात् प्राञ्च᳚म् । \newline
53. प्राञ्चं॑ ॅय॒ज्ञ्ं ॅय॒ज्ञ्म् प्राञ्च॒म् प्राञ्चं॑ ॅय॒ज्ञ्म् । \newline
54. य॒ज्ञ्म् पृ॒ष्ठानि॑ पृ॒ष्ठानि॑ य॒ज्ञ्ं ॅय॒ज्ञ्म् पृ॒ष्ठानि॑ । \newline
55. पृ॒ष्ठानि॒ सꣳ सम् पृ॒ष्ठानि॑ पृ॒ष्ठानि॒ सम् । \newline
56. सꣳ शृ॑णीयुः शृणीयुः॒ सꣳ सꣳ शृ॑णीयुः । \newline
57. शृ॒णी॒यु॒र् यद् यच्छृ॑णीयुः शृणीयु॒र् यत् । \newline
58. यदु॒क्थ्य॑ उ॒क्थ्ये॑ यद् यदु॒क्थ्ये᳚ । \newline
59. उ॒क्थ्ये॑ गृह्णी॒याद् गृ॑ह्णी॒या दु॒क्थ्य॑ उ॒क्थ्ये॑ गृह्णी॒यात् । \newline

\textbf{Ghana Paata } \newline

1. दे॒वा वै वै दे॒वा दे॒वा वा इ॑न्द्रि॒य मि॑न्द्रि॒यं ॅवै दे॒वा दे॒वा वा इ॑न्द्रि॒यम् । \newline
2. वा इ॑न्द्रि॒य मि॑न्द्रि॒यं ॅवै वा इ॑न्द्रि॒यं ॅवी॒र्यं॑ ॅवी॒र्य॑ मिन्द्रि॒यं ॅवै वा इ॑न्द्रि॒यं ॅवी॒र्य᳚म् । \newline
3. इ॒न्द्रि॒यं ॅवी॒र्यं॑ ॅवी॒र्य॑ मिन्द्रि॒य मि॑न्द्रि॒यं ॅवी॒र्यं॑ ॅवि वि वी॒र्य॑ मिन्द्रि॒य मि॑न्द्रि॒यं ॅवी॒र्यं॑ ॅवि । \newline
4. वी॒र्यं॑ ॅवि वि वी॒र्यं॑ ॅवी॒र्या᳚(1॒) ॅव्य॑भजन्ता भजन्त॒ वि वी॒र्यं॑ ॅवी॒र्या᳚(1॒) ॅव्य॑भजन्त । \newline
5. व्य॑भजन्ता भजन्त॒ वि व्य॑भजन्त॒ तत॒ स्ततो॑ ऽभजन्त॒ वि व्य॑भजन्त॒ ततः॑ । \newline
6. अ॒भ॒ज॒न्त॒ तत॒ स्ततो॑ ऽभजन्ता भजन्त॒ ततो॒ यद् यत् ततो॑ ऽभजन्ता भजन्त॒ ततो॒ यत् । \newline
7. ततो॒ यद् यत् तत॒स्ततो॒ यद॒ त्यशि॑ष्यता॒ त्यशि॑ष्यत॒ यत् तत॒ स्ततो॒ यद॒ त्यशि॑ष्यत । \newline
8. यद॒ त्यशि॑ष्यता॒ त्यशि॑ष्यत॒ यद् यद॒ त्यशि॑ष्यत॒ तत् तद॒ त्यशि॑ष्यत॒ यद् यद॒ त्यशि॑ष्यत॒ तत् । \newline
9. अ॒त्यशि॑ष्यत॒ तत् तद॒ त्यशि॑ष्यता॒ त्यशि॑ष्यत॒ तद॑तिग्रा॒ह्या॑ अतिग्रा॒ह्या᳚ स्तद॒त्यशि॑ष्यता॒ त्यशि॑ष्यत॒ तद॑तिग्रा॒ह्याः᳚ । \newline
10. अ॒त्यशि॑ष्य॒तेत्य॑ति - अशि॑ष्यत । \newline
11. तद॑तिग्रा॒ह्या॑ अतिग्रा॒ह्या᳚ स्तत् तद॑तिग्रा॒ह्या॑ अभवन् नभवन् नतिग्रा॒ह्या᳚ स्तत् तद॑तिग्रा॒ह्या॑ अभवन्न् । \newline
12. अ॒ति॒ग्रा॒ह्या॑ अभवन् नभवन् नतिग्रा॒ह्या॑ अतिग्रा॒ह्या॑ अभव॒न् तत् तद॑भवन् नतिग्रा॒ह्या॑ अतिग्रा॒ह्या॑ अभव॒न् तत् । \newline
13. अ॒ति॒ग्रा॒ह्या॑ इत्य॑ति - ग्रा॒ह्याः᳚ । \newline
14. अ॒भ॒व॒न् तत् तद॑भवन् नभव॒न् तद॑तिग्रा॒ह्या॑णा मतिग्रा॒ह्या॑णा॒म् तद॑भवन् नभव॒न् तद॑तिग्रा॒ह्या॑णाम् । \newline
15. तद॑तिग्रा॒ह्या॑णा मतिग्रा॒ह्या॑णा॒म् तत् तद॑तिग्रा॒ह्या॑णा मतिग्राह्य॒त्व म॑तिग्राह्य॒त्व म॑तिग्रा॒ह्या॑णा॒म् तत् तद॑तिग्रा॒ह्या॑णा मतिग्राह्य॒त्वम् । \newline
16. अ॒ति॒ग्रा॒ह्या॑णा मतिग्राह्य॒त्व म॑तिग्राह्य॒त्व म॑तिग्रा॒ह्या॑णा मतिग्रा॒ह्या॑णा मतिग्राह्य॒त्वं ॅयद् यद॑तिग्राह्य॒त्व म॑तिग्रा॒ह्या॑णा मतिग्रा॒ह्या॑णा मतिग्राह्य॒त्वं ॅयत् । \newline
17. अ॒ति॒ग्रा॒ह्या॑णा॒मित्य॑ति - ग्रा॒ह्या॑णाम् । \newline
18. अ॒ति॒ग्रा॒ह्य॒त्वं ॅयद् यद॑तिग्राह्य॒त्व म॑तिग्राह्य॒त्वं ॅयद॑तिग्रा॒ह्या॑ अतिग्रा॒ह्या॑ यद॑तिग्राह्य॒त्व म॑तिग्राह्य॒त्वं ॅयद॑तिग्रा॒ह्याः᳚ । \newline
19. अ॒ति॒ग्रा॒ह्य॒त्वमित्य॑तिग्राह्य - त्वम् । \newline
20. यद॑तिग्रा॒ह्या॑ अतिग्रा॒ह्या॑ यद् यद॑तिग्रा॒ह्या॑ गृ॒ह्यन्ते॑ गृ॒ह्यन्ते॑ ऽतिग्रा॒ह्या॑ यद् यद॑तिग्रा॒ह्या॑ गृ॒ह्यन्ते᳚ । \newline
21. अ॒ति॒ग्रा॒ह्या॑ गृ॒ह्यन्ते॑ गृ॒ह्यन्ते॑ ऽतिग्रा॒ह्या॑ अतिग्रा॒ह्या॑ गृ॒ह्यन्त॑ इन्द्रि॒य मि॑न्द्रि॒यम् गृ॒ह्यन्ते॑ ऽतिग्रा॒ह्या॑ अतिग्रा॒ह्या॑ गृ॒ह्यन्त॑ इन्द्रि॒यम् । \newline
22. अ॒ति॒ग्रा॒ह्या॑ इत्य॑ति - ग्रा॒ह्याः᳚ । \newline
23. गृ॒ह्यन्त॑ इन्द्रि॒य मि॑न्द्रि॒यम् गृ॒ह्यन्ते॑ गृ॒ह्यन्त॑ इन्द्रि॒य मे॒वैवेन्द्रि॒यम् गृ॒ह्यन्ते॑ गृ॒ह्यन्त॑ इन्द्रि॒य मे॒व । \newline
24. इ॒न्द्रि॒य मे॒वैवेन्द्रि॒य मि॑न्द्रि॒य मे॒व तत् तदे॒वेन्द्रि॒य मि॑न्द्रि॒य मे॒व तत् । \newline
25. ए॒व तत् तदे॒ वैव तद् वी॒र्यं॑ ॅवी॒र्य॑म् तदे॒ वैव तद् वी॒र्य᳚म् । \newline
26. तद् वी॒र्यं॑ ॅवी॒र्य॑म् तत् तद् वी॒र्यं॑ ॅयज॑मानो॒ यज॑मानो वी॒र्य॑म् तत् तद् वी॒र्यं॑ ॅयज॑मानः । \newline
27. वी॒र्यं॑ ॅयज॑मानो॒ यज॑मानो वी॒र्यं॑ ॅवी॒र्यं॑ ॅयज॑मान आ॒त्मन् ना॒त्मन्. यज॑मानो वी॒र्यं॑ ॅवी॒र्यं॑ ॅयज॑मान आ॒त्मन्न् । \newline
28. यज॑मान आ॒त्मन् ना॒त्मन्. यज॑मानो॒ यज॑मान आ॒त्मन् ध॑त्ते धत्त आ॒त्मन्. यज॑मानो॒ यज॑मान आ॒त्मन् ध॑त्ते । \newline
29. आ॒त्मन् ध॑त्ते धत्त आ॒त्मन् ना॒त्मन् ध॑त्ते॒ तेज॒ स्तेजो॑ धत्त आ॒त्मन् ना॒त्मन् ध॑त्ते॒ तेजः॑ । \newline
30. ध॒त्ते॒ तेज॒ स्तेजो॑ धत्ते धत्ते॒ तेज॑ आग्ने॒येना᳚ ग्ने॒येन॒ तेजो॑ धत्ते धत्ते॒ तेज॑ आग्ने॒येन॑ । \newline
31. तेज॑ आग्ने॒येना᳚ ग्ने॒येन॒ तेज॒ स्तेज॑ आग्ने॒येने᳚न्द्रि॒य मि॑न्द्रि॒य मा᳚ग्ने॒येन॒ तेज॒ स्तेज॑ आग्ने॒येने᳚न्द्रि॒यम् । \newline
32. आ॒ग्ने॒येने᳚न्द्रि॒य मि॑न्द्रि॒य मा᳚ग्ने॒येना᳚ ग्ने॒येने᳚न्द्रि॒य मै॒न्द्रे णै॒न्द्रे णे᳚न्द्रि॒य मा᳚ग्ने॒येना᳚ ग्ने॒ये
ने᳚न्द्रि॒य मै॒न्द्रेण॑ । \newline
33. इ॒न्द्रि॒य मै॒न्द्रे णै॒न्द्रे णे᳚न्द्रि॒य मि॑न्द्रि॒य मै॒न्द्रेण॑ ब्रह्मवर्च॒सम् ब्र॑ह्मवर्च॒स मै॒न्द्रे णे᳚न्द्रि॒य मि॑न्द्रि॒य मै॒न्द्रेण॑ ब्रह्मवर्च॒सम् । \newline
34. ऐ॒न्द्रेण॑ ब्रह्मवर्च॒सम् ब्र॑ह्मवर्च॒स मै॒न्द्रे णै॒न्द्रेण॑ ब्रह्मवर्च॒सꣳ सौ॒र्येण॑ सौ॒र्येण॑ ब्रह्मवर्च॒स मै॒न्द्रे णै॒न्द्रेण॑ ब्रह्मवर्च॒सꣳ सौ॒र्येण॑ । \newline
35. ब्र॒ह्म॒व॒र्च॒सꣳ सौ॒र्येण॑ सौ॒र्येण॑ ब्रह्मवर्च॒सम् ब्र॑ह्मवर्च॒सꣳ सौ॒र्ये णो॑प॒स्तंभ॑न मुप॒स्तंभ॑नꣳ सौ॒र्येण॑ ब्रह्मवर्च॒सम् ब्र॑ह्मवर्च॒सꣳ सौ॒र्ये णो॑प॒स्तंभ॑नम् । \newline
36. ब्र॒ह्म॒व॒र्च॒समिति॑ ब्रह्म - व॒र्च॒सम् । \newline
37. सौ॒र्ये णो॑प॒स्तंभ॑न मुप॒स्तंभ॑नꣳ सौ॒र्येण॑ सौ॒र्ये णो॑प॒स्तंभ॑नं॒ ॅवै वा उ॑प॒स्तंभ॑नꣳ सौ॒र्येण॑ सौ॒र्ये णो॑प॒स्तंभ॑नं॒ ॅवै । \newline
38. उ॒प॒स्तंभ॑नं॒ ॅवै वा उ॑प॒स्तंभ॑न मुप॒स्तंभ॑नं॒ ॅवा ए॒त दे॒तद् वा उ॑प॒स्तंभ॑न मुप॒स्तंभ॑नं॒ ॅवा ए॒तत् । \newline
39. उ॒प॒स्तंभ॑न॒मित्यु॑प - स्तंभ॑नम् । \newline
40. वा ए॒त दे॒तद् वै वा ए॒तद् य॒ज्ञ्स्य॑ य॒ज्ञ् स्यै॒तद् वै वा ए॒तद् य॒ज्ञ्स्य॑ । \newline
41. ए॒तद् य॒ज्ञ्स्य॑ य॒ज्ञ् स्यै॒त दे॒तद् य॒ज्ञ्स्य॒ यद् यद् य॒ज्ञ् स्यै॒त दे॒तद् य॒ज्ञ्स्य॒ यत् । \newline
42. य॒ज्ञ्स्य॒ यद् यद् य॒ज्ञ्स्य॑ य॒ज्ञ्स्य॒ यद॑तिग्रा॒ह्या॑ अतिग्रा॒ह्या॑ यद् य॒ज्ञ्स्य॑ य॒ज्ञ्स्य॒ यद॑तिग्रा॒ह्याः᳚ । \newline
43. यद॑तिग्रा॒ह्या॑ अतिग्रा॒ह्या॑ यद् यद॑तिग्रा॒ह्या᳚ श्च॒क्रे च॒क्रे अ॑तिग्रा॒ह्या॑ यद् यद॑तिग्रा॒ह्या᳚ श्च॒क्रे । \newline
44. अ॒ति॒ग्रा॒ह्या᳚ श्च॒क्रे च॒क्रे अ॑तिग्रा॒ह्या॑ अतिग्रा॒ह्या᳚ श्च॒क्रे पृ॒ष्ठानि॑ पृ॒ष्ठानि॑ च॒क्रे अ॑तिग्रा॒ह्या॑ अतिग्रा॒ह्या᳚ श्च॒क्रे पृ॒ष्ठानि॑ । \newline
45. अ॒ति॒ग्रा॒ह्या॑ इत्य॑ति - ग्रा॒ह्याः᳚ । \newline
46. च॒क्रे पृ॒ष्ठानि॑ पृ॒ष्ठानि॑ च॒क्रे च॒क्रे पृ॒ष्ठानि॒ यद् यत् पृ॒ष्ठानि॑ च॒क्रे च॒क्रे पृ॒ष्ठानि॒ यत् । \newline
47. च॒क्रे इति॑ च॒क्रे । \newline
48. पृ॒ष्ठानि॒ यद् यत् पृ॒ष्ठानि॑ पृ॒ष्ठानि॒ यत् पृष्ठ्ये॒ पृष्ठ्ये॒ यत् पृ॒ष्ठानि॑ पृ॒ष्ठानि॒ यत् पृष्ठ्ये᳚ । \newline
49. यत् पृष्ठ्ये॒ पृष्ठ्ये॒ यद् यत् पृष्ठ्ये॒ न न पृष्ठ्ये॒ यद् यत् पृष्ठ्ये॒ न । \newline
50. पृष्ठ्ये॒ न न पृष्ठ्ये॒ पृष्ठ्ये॒ न गृ॑ह्णी॒याद् गृ॑ह्णी॒यान् न पृष्ठ्ये॒ पृष्ठ्ये॒ न गृ॑ह्णी॒यात् । \newline
51. न गृ॑ह्णी॒याद् गृ॑ह्णी॒यान् न न गृ॑ह्णी॒यात् प्राञ्च॒म् प्राञ्च॑म् गृह्णी॒यान् न न गृ॑ह्णी॒यात् प्राञ्च᳚म् । \newline
52. गृ॒ह्णी॒यात् प्राञ्च॒म् प्राञ्च॑म् गृह्णी॒याद् गृ॑ह्णी॒यात् प्राञ्चं॑ ॅय॒ज्ञ्ं ॅय॒ज्ञ्म् प्राञ्च॑म् गृह्णी॒याद् गृ॑ह्णी॒यात् प्राञ्चं॑ ॅय॒ज्ञ्म् । \newline
53. प्राञ्चं॑ ॅय॒ज्ञ्ं ॅय॒ज्ञ्म् प्राञ्च॒म् प्राञ्चं॑ ॅय॒ज्ञ्म् पृ॒ष्ठानि॑ पृ॒ष्ठानि॑ य॒ज्ञ्म् प्राञ्च॒म् प्राञ्चं॑ ॅय॒ज्ञ्म् पृ॒ष्ठानि॑ । \newline
54. य॒ज्ञ्म् पृ॒ष्ठानि॑ पृ॒ष्ठानि॑ य॒ज्ञ्ं ॅय॒ज्ञ्म् पृ॒ष्ठानि॒ सꣳ सम् पृ॒ष्ठानि॑ य॒ज्ञ्ं ॅय॒ज्ञ्म् पृ॒ष्ठानि॒ सम् । \newline
55. पृ॒ष्ठानि॒ सꣳ सम् पृ॒ष्ठानि॑ पृ॒ष्ठानि॒ सꣳ शृ॑णीयुः शृणीयुः॒ सम् पृ॒ष्ठानि॑ पृ॒ष्ठानि॒ सꣳ शृ॑णीयुः । \newline
56. सꣳ शृ॑णीयुः शृणीयुः॒ सꣳ सꣳ शृ॑णीयु॒र् यद् यच्छृ॑णीयुः॒ सꣳ सꣳ शृ॑णीयु॒र् यत् । \newline
57. शृ॒णी॒यु॒र् यद् यच्छृ॑णीयुः शृणीयु॒र् यदु॒क्थ्य॑ उ॒क्थ्ये॑ यच्छृ॑णीयुः शृणीयु॒र् यदु॒क्थ्ये᳚ । \newline
58. यदु॒क्थ्य॑ उ॒क्थ्ये॑ यद् यदु॒क्थ्ये॑ गृह्णी॒याद् गृ॑ह्णी॒या दु॒क्थ्ये॑ यद् यदु॒क्थ्ये॑ गृह्णी॒यात् । \newline
59. उ॒क्थ्ये॑ गृह्णी॒याद् गृ॑ह्णी॒या दु॒क्थ्य॑ उ॒क्थ्ये॑ गृह्णी॒यात् प्र॒त्यञ्च॑म् प्र॒त्यञ्च॑म् गृह्णी॒या दु॒क्थ्य॑ उ॒क्थ्ये॑ गृह्णी॒यात् प्र॒त्यञ्च᳚म् । \newline
\pagebreak
\markright{ TS 6.6.8.2  \hfill https://www.vedavms.in \hfill}

\section{ TS 6.6.8.2 }

\textbf{TS 6.6.8.2 } \newline
\textbf{Samhita Paata} \newline

गृह्णी॒यात् प्र॒त्यञ्चं॑ ॅय॒ज्ञ्म॑तिग्रा॒ह्याः᳚ सꣳ शृ॑णीयुर्विश्व॒जिति॒ सर्व॑पृष्ठे ग्रहीत॒व्या॑ य॒ज्ञ्स्य॑ सवीर्य॒त्वाय॑ प्र॒जाप॑तिर्दे॒वेभ्यो॑ य॒ज्ञान् व्यादि॑श॒थ् स प्रि॒यास्त॒नूरप॒ न्य॑धत्त॒ तद॑तिग्रा॒ह्या॑ अभव॒न् वित॑नु॒स्तस्य॑ य॒ज्ञ् इत्या॑हु॒र्य-स्या॑तिग्रा॒ह्या॑ न गृ॒ह्यन्त॒ इत्यप्य॑ग्निष्टो॒मे ग्र॑हीत॒व्या॑ य॒ज्ञ्स्य॑ सतनु॒त्वाय॑ दे॒वता॒ वै सर्वाः᳚ स॒दृशी॑रास॒न् ता न व्या॒वृत॑-मगच्छ॒न् ते दे॒वा- [  ] \newline

\textbf{Pada Paata} \newline

गृ॒ह्णी॒यात् । प्र॒त्यञ्च᳚म् । य॒ज्ञ्म् । अ॒ति॒ग्रा॒ह्या॑ इत्य॑ति - ग्रा॒ह्याः᳚ । समिति॑ । शृ॒णी॒युः॒ । वि॒श्व॒जितीति॑ विश्व - जिति॑ । सर्व॑पृष्ठ॒ इति॒ सर्व॑ - पृ॒ष्ठे॒ । ग्र॒ही॒त॒व्याः᳚ । य॒ज्ञ्स्य॑ । स॒वी॒र्य॒त्वायेति॑ सवीर्य-त्वाय॑ । प्र॒जाप॑ति॒रिति॑ प्र॒जा - प॒तिः॒ । दे॒वेभ्यः॑ । य॒ज्ञान् । व्यादि॑श॒दिति॑ वि - आदि॑शत् । सः । प्रि॒याः । त॒नूः । अप॑ । नीति॑ । अ॒ध॒त्त॒ । तत् । अ॒ति॒ग्रा॒ह्या॑ इत्य॑ति-ग्रा॒ह्याः᳚ । अ॒भ॒व॒न्न् । वित॑नु॒रिति॒ वि-त॒नुः॒ । तस्य॑ । य॒ज्ञ्ः । इति॑ । आ॒हुः॒ । यस्य॑ । अ॒ति॒ग्रा॒ह्या॑ इत्य॑ति - ग्रा॒ह्याः᳚ । न । गृ॒ह्यन्ते᳚ । इति॑ । अपीति॑ । अ॒ग्नि॒ष्टो॒म इत्य॑ग्नि - स्तो॒मे । ग्र॒ही॒त॒व्याः᳚ । य॒ज्ञ्स्य॑ । स॒त॒नु॒त्वायेति॑ सतनु - त्वाय॑ । दे॒वताः᳚ । वै । सर्वाः᳚ । स॒दृशीः᳚ । आ॒स॒न्न् । ताः । न । व्या॒वृत॒मिति॑ वि-आ॒वृत᳚म् । अ॒ग॒च्छ॒न्न् । ते । दे॒वाः ।  \newline


\textbf{Krama Paata} \newline

गृ॒ह्णि॒यात् प्र॒त्यञ्च᳚म् । प्र॒त्यञ्च॑म् ॅय॒ज्ञ्म् । य॒ज्ञ्म॑तिग्रा॒ह्याः᳚ । अ॒ति॒ग्रा॒ह्याः᳚ सम् । अ॒ति॒ग्रा॒ह्या॑ इत्य॑ति - ग्रा॒ह्याः᳚ । सꣳ शृ॑णीयुः । शृ॒णी॒यु॒र् वि॒श्व॒जिति॑ । वि॒श्व॒जिति॒ सर्व॑पृष्ठे । वि॒श्व॒जितीति॑ विश्व - जिति॑ । सर्व॑पृष्ठे ग्रहीत॒व्याः᳚ । सर्व॑पृष्ठ॒ इति॒ सर्व॑ - पृ॒ष्ठे॒ । ग्र॒ही॒त॒व्या॑ य॒ज्ञ्स्य॑ । य॒ज्ञ्स्य॑ सवीर्य॒त्वाय॑ । स॒वी॒र्य॒त्वाय॑ प्र॒जाप॑तिः । स॒वी॒र्य॒त्वायेति॑ सवीर्य - त्वाय॑ । प्र॒जाप॑तिर् दे॒वेभ्यः॑ । प्र॒जाप॑ति॒रिति॑ प्र॒जा - प॒तिः॒ । दे॒वेभ्यो॑ य॒ज्ञान् । य॒ज्ञान् व्यादि॑शत् । व्यादि॑श॒थ् सः । व्यादि॑श॒दिति॑ वि - आदि॑शत् । स प्रि॒याः । प्रि॒यास्त॒नूः । त॒नूरप॑ । अप॒ नि । न्य॑धत्त । अ॒ध॒त्त॒ तत् । तद॑तिग्रा॒ह्याः᳚ । अ॒ति॒ग्रा॒ह्या॑ अभवन्न् । अ॒ति॒ग्रा॒ह्या॑ इत्य॑ति - ग्रा॒ह्याः᳚ । अ॒भ॒व॒न् वित॑नुः । वित॑नु॒स्तस्य॑ । वित॑नु॒रिति॒ वि - त॒नुः॒ । तस्य॑ य॒ज्ञ्ः । य॒ज्ञ् इति॑ । इत्या॑हुः । आ॒हु॒र् यस्य॑ । यस्या॑तिग्रा॒ह्याः᳚ । अ॒ति॒ग्रा॒ह्या॑ न । अ॒ति॒ग्रा॒ह्या॑ इत्य॑ति - ग्रा॒ह्याः᳚ । न गृ॒ह्यन्ते᳚ । गृ॒ह्यन्त॒ इति॑ । इत्यपि॑ । अप्य॑ग्निष्टो॒मे । अ॒ग्नि॒ष्टो॒मे ग्र॑हीत॒व्याः᳚ । अ॒ग्नि॒ष्टो॒म इत्य॑ग्नि - स्तो॒मे । ग्र॒ही॒त॒व्या॑ य॒ज्ञ्स्य॑ । य॒ज्ञ्स्य॑ सतनु॒त्वाय॑ । स॒त॒नु॒त्वाय॑ दे॒वताः᳚ । स॒त॒नु॒त्वायेति॑ सतनु - त्वाय॑ । दे॒वता॒ वै । वै सर्वाः᳚ । सर्वाः᳚ स॒दृशीः᳚ । स॒दृशी॑रासन्न् । आ॒स॒न् ताः । ता न । न व्या॒वृत᳚म् । व्या॒वृत॑मगच्छन्न् । व्या॒वृत॒मिति॑ वि - आ॒वृत᳚म् । अ॒ग॒च्छ॒न् ते । ते दे॒वाः । दे॒वा ए॒ते \newline

\textbf{Jatai Paata} \newline

1. गृ॒ह्णी॒यात् प्र॒त्यञ्च॑म् प्र॒त्यञ्च॑म् गृह्णी॒याद् गृ॑ह्णी॒यात् प्र॒त्यञ्च᳚म् । \newline
2. प्र॒त्यञ्चं॑ ॅय॒ज्ञ्ं ॅय॒ज्ञ्म् प्र॒त्यञ्च॑म् प्र॒त्यञ्चं॑ ॅय॒ज्ञ्म् । \newline
3. य॒ज्ञ् म॑तिग्रा॒ह्या॑ अतिग्रा॒ह्या॑ य॒ज्ञ्ं ॅय॒ज्ञ् म॑तिग्रा॒ह्याः᳚ । \newline
4. अ॒ति॒ग्रा॒ह्याः᳚ सꣳ स म॑तिग्रा॒ह्या॑ अतिग्रा॒ह्याः᳚ सम् । \newline
5. अ॒ति॒ग्रा॒ह्या॑ इत्य॑ति - ग्रा॒ह्याः᳚ । \newline
6. सꣳ शृ॑णीयुः शृणीयुः॒ सꣳ सꣳ शृ॑णीयुः । \newline
7. शृ॒णी॒यु॒र् वि॒श्व॒जिति॑ विश्व॒जिति॑ शृणीयुः शृणीयुर् विश्व॒जिति॑ । \newline
8. वि॒श्व॒जिति॒ सर्व॑पृष्ठे॒ सर्व॑पृष्ठे विश्व॒जिति॑ विश्व॒जिति॒ सर्व॑पृष्ठे । \newline
9. वि॒श्व॒जितीति॑ विश्व - जिति॑ । \newline
10. सर्व॑पृष्ठे ग्रहीत॒व्या᳚ ग्रहीत॒व्याः᳚ सर्व॑पृष्ठे॒ सर्व॑पृष्ठे ग्रहीत॒व्याः᳚ । \newline
11. सर्व॑पृष्ठ॒ इति॒ सर्व॑ - पृ॒ष्ठे॒ । \newline
12. ग्र॒ही॒त॒व्या॑ य॒ज्ञ्स्य॑ य॒ज्ञ्स्य॑ ग्रहीत॒व्या᳚ ग्रहीत॒व्या॑ य॒ज्ञ्स्य॑ । \newline
13. य॒ज्ञ्स्य॑ सवीर्य॒त्वाय॑ सवीर्य॒त्वाय॑ य॒ज्ञ्स्य॑ य॒ज्ञ्स्य॑ सवीर्य॒त्वाय॑ । \newline
14. स॒वी॒र्य॒त्वाय॑ प्र॒जाप॑तिः प्र॒जाप॑तिः सवीर्य॒त्वाय॑ सवीर्य॒त्वाय॑ प्र॒जाप॑तिः । \newline
15. स॒वी॒र्य॒त्वायेति॑ सवीर्य - त्वाय॑ । \newline
16. प्र॒जाप॑तिर् दे॒वेभ्यो॑ दे॒वेभ्यः॑ प्र॒जाप॑तिः प्र॒जाप॑तिर् दे॒वेभ्यः॑ । \newline
17. प्र॒जाप॑ति॒रिति॑ प्र॒जा - प॒तिः॒ । \newline
18. दे॒वेभ्यो॑ य॒ज्ञान्. य॒ज्ञान् दे॒वेभ्यो॑ दे॒वेभ्यो॑ य॒ज्ञान् । \newline
19. य॒ज्ञान् व्यादि॑श॒द् व्यादि॑शद् य॒ज्ञान्. य॒ज्ञान् व्यादि॑शत् । \newline
20. व्यादि॑श॒थ् स स व्यादि॑श॒द् व्यादि॑श॒थ् सः । \newline
21. व्यादि॑श॒दिति॑ वि - आदि॑शत् । \newline
22. स प्रि॒याः प्रि॒याः स स प्रि॒याः । \newline
23. प्रि॒या स्त॒नू स्त॒नूः प्रि॒याः प्रि॒या स्त॒नूः । \newline
24. त॒नू रपाप॑ त॒नू स्त॒नू रप॑ । \newline
25. अप॒ नि न्यपाप॒ नि । \newline
26. न्य॑धत्ता धत्त॒ नि न्य॑धत्त । \newline
27. अ॒ध॒त्त॒ तत् तद॑धत्ता धत्त॒ तत् । \newline
28. तद॑तिग्रा॒ह्या॑ अतिग्रा॒ह्या᳚ स्तत् तद॑तिग्रा॒ह्याः᳚ । \newline
29. अ॒ति॒ग्रा॒ह्या॑ अभवन् नभवन् नतिग्रा॒ह्या॑ अतिग्रा॒ह्या॑ अभवन्न् । \newline
30. अ॒ति॒ग्रा॒ह्या॑ इत्य॑ति - ग्रा॒ह्याः᳚ । \newline
31. अ॒भ॒व॒न्॒. वित॑नु॒र् वित॑नु रभवन् नभव॒न्॒. वित॑नुः । \newline
32. वित॑नु॒ स्तस्य॒ तस्य॒ वित॑नु॒र् वित॑नु॒ स्तस्य॑ । \newline
33. वित॑नु॒रिति॒ वि - त॒नुः॒ । \newline
34. तस्य॑ य॒ज्ञो य॒ज्ञ् स्तस्य॒ तस्य॑ य॒ज्ञ्ः । \newline
35. य॒ज्ञ् इतीति॑ य॒ज्ञो य॒ज्ञ् इति॑ । \newline
36. इत्या॑हु राहु॒ रिती त्या॑हुः । \newline
37. आ॒हु॒र् यस्य॒ यस्या॑हु राहु॒र् यस्य॑ । \newline
38. यस्या॑ तिग्रा॒ह्या॑ अतिग्रा॒ह्या॑ यस्य॒ यस्या॑ तिग्रा॒ह्याः᳚ । \newline
39. अ॒ति॒ग्रा॒ह्या॑ न नाति॑ग्रा॒ह्या॑ अतिग्रा॒ह्या॑ न । \newline
40. अ॒ति॒ग्रा॒ह्या॑ इत्य॑ति - ग्रा॒ह्याः᳚ । \newline
41. न गृ॒ह्यन्ते॑ गृ॒ह्यन्ते॒ न न गृ॒ह्यन्ते᳚ । \newline
42. गृ॒ह्यन्त॒ इतीति॑ गृ॒ह्यन्ते॑ गृ॒ह्यन्त॒ इति॑ । \newline
43. इत्य प्यपीती त्यपि॑ । \newline
44. अप्य॑ ग्निष्टो॒मे᳚ ऽग्निष्टो॒मे ऽप्यप्य॑ ग्निष्टो॒मे । \newline
45. अ॒ग्नि॒ष्टो॒मे ग्र॑हीत॒व्या᳚ ग्रहीत॒व्या॑ अग्निष्टो॒मे᳚ ऽग्निष्टो॒मे ग्र॑हीत॒व्याः᳚ । \newline
46. अ॒ग्नि॒ष्टो॒म इत्य॑ग्नि - स्तो॒मे । \newline
47. ग्र॒ही॒त॒व्या॑ य॒ज्ञ्स्य॑ य॒ज्ञ्स्य॑ ग्रहीत॒व्या᳚ ग्रहीत॒व्या॑ य॒ज्ञ्स्य॑ । \newline
48. य॒ज्ञ्स्य॑ सतनु॒त्वाय॑ सतनु॒त्वाय॑ य॒ज्ञ्स्य॑ य॒ज्ञ्स्य॑ सतनु॒त्वाय॑ । \newline
49. स॒त॒नु॒त्वाय॑ दे॒वता॑ दे॒वताः᳚ सतनु॒त्वाय॑ सतनु॒त्वाय॑ दे॒वताः᳚ । \newline
50. स॒त॒नु॒त्वायेति॑ सतनु - त्वाय॑ । \newline
51. दे॒वता॒ वै वै दे॒वता॑ दे॒वता॒ वै । \newline
52. वै सर्वाः॒ सर्वा॒ वै वै सर्वाः᳚ । \newline
53. सर्वाः᳚ स॒दृशीः᳚ स॒दृशीः॒ सर्वाः॒ सर्वाः᳚ स॒दृशीः᳚ । \newline
54. स॒दृशी॑ रासन् नासन् थ्स॒दृशीः᳚ स॒दृशी॑ रासन्न् । \newline
55. आ॒स॒न् ता स्ता आ॑सन् नास॒न् ताः । \newline
56. ता न न ता स्ता न । \newline
57. न व्या॒वृतं॑ ॅव्या॒वृत॒न् न न व्या॒वृत᳚म् । \newline
58. व्या॒वृत॑ मगच्छन् नगच्छन् व्या॒वृतं॑ ॅव्या॒वृत॑ मगच्छन्न् । \newline
59. व्या॒वृत॒मिति॑ वि - आ॒वृत᳚म् । \newline
60. अ॒ग॒च्छ॒न् ते ते॑ ऽगच्छन् नगच्छ॒न् ते । \newline
61. ते दे॒वा दे॒वा स्ते ते दे॒वाः । \newline
62. दे॒वा ए॒त ए॒ते दे॒वा दे॒वा ए॒ते । \newline

\textbf{Ghana Paata } \newline

1. गृ॒ह्णी॒यात् प्र॒त्यञ्च॑म् प्र॒त्यञ्च॑म् गृह्णी॒याद् गृ॑ह्णी॒यात् प्र॒त्यञ्चं॑ ॅय॒ज्ञ्ं ॅय॒ज्ञ्म् प्र॒त्यञ्च॑म् गृह्णी॒याद् गृ॑ह्णी॒यात् प्र॒त्यञ्चं॑ ॅय॒ज्ञ्म् । \newline
2. प्र॒त्यञ्चं॑ ॅय॒ज्ञ्ं ॅय॒ज्ञ्म् प्र॒त्यञ्च॑म् प्र॒त्यञ्चं॑ ॅय॒ज्ञ् म॑तिग्रा॒ह्या॑ अतिग्रा॒ह्या॑ य॒ज्ञ्म् प्र॒त्यञ्च॑म् प्र॒त्यञ्चं॑ ॅय॒ज्ञ् म॑तिग्रा॒ह्याः᳚ । \newline
3. य॒ज्ञ् म॑तिग्रा॒ह्या॑ अतिग्रा॒ह्या॑ य॒ज्ञ्ं ॅय॒ज्ञ् म॑तिग्रा॒ह्याः᳚ सꣳ स म॑तिग्रा॒ह्या॑ य॒ज्ञ्ं ॅय॒ज्ञ् म॑तिग्रा॒ह्याः᳚ सम् । \newline
4. अ॒ति॒ग्रा॒ह्याः᳚ सꣳ स म॑तिग्रा॒ह्या॑ अतिग्रा॒ह्याः᳚ सꣳ शृ॑णीयुः शृणीयुः॒ स म॑तिग्रा॒ह्या॑ अतिग्रा॒ह्याः᳚ सꣳ शृ॑णीयुः । \newline
5. अ॒ति॒ग्रा॒ह्या॑ इत्य॑ति - ग्रा॒ह्याः᳚ । \newline
6. सꣳ शृ॑णीयुः शृणीयुः॒ सꣳ सꣳ शृ॑णीयुर् विश्व॒जिति॑ विश्व॒जिति॑ शृणीयुः॒ सꣳ सꣳ शृ॑णीयुर् विश्व॒जिति॑ । \newline
7. शृ॒णी॒यु॒र् वि॒श्व॒जिति॑ विश्व॒जिति॑ शृणीयुः शृणीयुर् विश्व॒जिति॒ सर्व॑पृष्ठे॒ सर्व॑पृष्ठे विश्व॒जिति॑ शृणीयुः शृणीयुर् विश्व॒जिति॒ सर्व॑पृष्ठे । \newline
8. वि॒श्व॒जिति॒ सर्व॑पृष्ठे॒ सर्व॑पृष्ठे विश्व॒जिति॑ विश्व॒जिति॒ सर्व॑पृष्ठे ग्रहीत॒व्या᳚ ग्रहीत॒व्याः᳚ सर्व॑पृष्ठे विश्व॒जिति॑ विश्व॒जिति॒ सर्व॑पृष्ठे ग्रहीत॒व्याः᳚ । \newline
9. वि॒श्व॒जितीति॑ विश्व - जिति॑ । \newline
10. सर्व॑पृष्ठे ग्रहीत॒व्या᳚ ग्रहीत॒व्याः᳚ सर्व॑पृष्ठे॒ सर्व॑पृष्ठे ग्रहीत॒व्या॑ य॒ज्ञ्स्य॑ य॒ज्ञ्स्य॑ ग्रहीत॒व्याः᳚ सर्व॑पृष्ठे॒ सर्व॑पृष्ठे ग्रहीत॒व्या॑ य॒ज्ञ्स्य॑ । \newline
11. सर्व॑पृष्ठ॒ इति॒ सर्व॑ - पृ॒ष्ठे॒ । \newline
12. ग्र॒ही॒त॒व्या॑ य॒ज्ञ्स्य॑ य॒ज्ञ्स्य॑ ग्रहीत॒व्या᳚ ग्रहीत॒व्या॑ य॒ज्ञ्स्य॑ सवीर्य॒त्वाय॑ सवीर्य॒त्वाय॑ य॒ज्ञ्स्य॑ ग्रहीत॒व्या᳚ ग्रहीत॒व्या॑ य॒ज्ञ्स्य॑ सवीर्य॒त्वाय॑ । \newline
13. य॒ज्ञ्स्य॑ सवीर्य॒त्वाय॑ सवीर्य॒त्वाय॑ य॒ज्ञ्स्य॑ य॒ज्ञ्स्य॑ सवीर्य॒त्वाय॑ प्र॒जाप॑तिः प्र॒जाप॑तिः सवीर्य॒त्वाय॑ य॒ज्ञ्स्य॑ य॒ज्ञ्स्य॑ सवीर्य॒त्वाय॑ प्र॒जाप॑तिः । \newline
14. स॒वी॒र्य॒त्वाय॑ प्र॒जाप॑तिः प्र॒जाप॑तिः सवीर्य॒त्वाय॑ सवीर्य॒त्वाय॑ प्र॒जाप॑तिर् दे॒वेभ्यो॑ दे॒वेभ्यः॑ प्र॒जाप॑तिः सवीर्य॒त्वाय॑ सवीर्य॒त्वाय॑ प्र॒जाप॑तिर् दे॒वेभ्यः॑ । \newline
15. स॒वी॒र्य॒त्वायेति॑ सवीर्य - त्वाय॑ । \newline
16. प्र॒जाप॑तिर् दे॒वेभ्यो॑ दे॒वेभ्यः॑ प्र॒जाप॑तिः प्र॒जाप॑तिर् दे॒वेभ्यो॑ य॒ज्ञान्. य॒ज्ञान् दे॒वेभ्यः॑ प्र॒जाप॑तिः प्र॒जाप॑तिर् दे॒वेभ्यो॑ य॒ज्ञान् । \newline
17. प्र॒जाप॑ति॒रिति॑ प्र॒जा - प॒तिः॒ । \newline
18. दे॒वेभ्यो॑ य॒ज्ञान्. य॒ज्ञान् दे॒वेभ्यो॑ दे॒वेभ्यो॑ य॒ज्ञान् व्यादि॑श॒द् व्यादि॑शद् य॒ज्ञान् दे॒वेभ्यो॑ दे॒वेभ्यो॑ य॒ज्ञान् व्यादि॑शत् । \newline
19. य॒ज्ञान् व्यादि॑श॒द् व्यादि॑शद् य॒ज्ञान्. य॒ज्ञान् व्यादि॑श॒थ् स स व्यादि॑शद् य॒ज्ञान्. य॒ज्ञान् व्यादि॑श॒थ् सः । \newline
20. व्यादि॑श॒थ् स स व्यादि॑श॒द् व्यादि॑श॒थ् स प्रि॒याः प्रि॒याः स व्यादि॑श॒द् व्यादि॑श॒थ् स प्रि॒याः । \newline
21. व्यादि॑श॒दिति॑ वि - आदि॑शत् । \newline
22. स प्रि॒याः प्रि॒याः स स प्रि॒या स्त॒नू स्त॒नूः प्रि॒याः स स प्रि॒या स्त॒नूः । \newline
23. प्रि॒या स्त॒नू स्त॒नूः प्रि॒याः प्रि॒या स्त॒नू रपाप॑ त॒नूः प्रि॒याः प्रि॒या स्त॒नू रप॑ । \newline
24. त॒नू रपाप॑ त॒नू स्त॒नू रप॒ नि न्यप॑ त॒नू स्त॒नू रप॒ नि । \newline
25. अप॒ नि न्यपाप॒ न्य॑धत्ता धत्त॒ न्यपाप॒ न्य॑धत्त । \newline
26. न्य॑धत्ता धत्त॒ नि न्य॑धत्त॒ तत् तद॑धत्त॒ नि न्य॑धत्त॒ तत् । \newline
27. अ॒ध॒त्त॒ तत् तद॑धत्ता धत्त॒ तद॑तिग्रा॒ह्या॑ अतिग्रा॒ह्या᳚ स्तद॑धत्ता धत्त॒ तद॑तिग्रा॒ह्याः᳚ । \newline
28. तद॑तिग्रा॒ह्या॑ अतिग्रा॒ह्या᳚ स्तत् तद॑तिग्रा॒ह्या॑ अभवन् नभवन् नतिग्रा॒ह्या᳚ स्तत् तद॑तिग्रा॒ह्या॑ अभवन्न् । \newline
29. अ॒ति॒ग्रा॒ह्या॑ अभवन् नभवन् नतिग्रा॒ह्या॑ अतिग्रा॒ह्या॑ अभव॒न्॒. वित॑नु॒र् वित॑नु रभवन् नतिग्रा॒ह्या॑ अतिग्रा॒ह्या॑ अभव॒न्॒. वित॑नुः । \newline
30. अ॒ति॒ग्रा॒ह्या॑ इत्य॑ति - ग्रा॒ह्याः᳚ । \newline
31. अ॒भ॒व॒न्॒. वित॑नु॒र् वित॑नु रभवन् नभव॒न्॒. वित॑नु॒ स्तस्य॒ तस्य॒ वित॑नु रभवन् नभव॒न्॒. वित॑नु॒ स्तस्य॑ । \newline
32. वित॑नु॒ स्तस्य॒ तस्य॒ वित॑नु॒र् वित॑नु॒ स्तस्य॑ य॒ज्ञो य॒ज्ञ् स्तस्य॒ वित॑नु॒र् वित॑नु॒ स्तस्य॑ य॒ज्ञ्ः । \newline
33. वित॑नु॒रिति॒ वि - त॒नुः॒ । \newline
34. तस्य॑ य॒ज्ञो य॒ज्ञ् स्तस्य॒ तस्य॑ य॒ज्ञ् इतीति॑ य॒ज्ञ् स्तस्य॒ तस्य॑ य॒ज्ञ् इति॑ । \newline
35. य॒ज्ञ् इतीति॑ य॒ज्ञो य॒ज्ञ् इत्या॑हु राहु॒ रिति॑ य॒ज्ञो य॒ज्ञ् इत्या॑हुः । \newline
36. इत्या॑हु राहु॒ रिती त्या॑हु॒र् यस्य॒ यस्या॑हु॒ रिती त्या॑हु॒र् यस्य॑ । \newline
37. आ॒हु॒र् यस्य॒ यस्या॑हु राहु॒र् यस्या॑ तिग्रा॒ह्या॑ अतिग्रा॒ह्या॑ यस्या॑हु राहु॒र् यस्या॑ तिग्रा॒ह्याः᳚ । \newline
38. यस्या॑ तिग्रा॒ह्या॑ अतिग्रा॒ह्या॑ यस्य॒ यस्या॑ तिग्रा॒ह्या॑ न नाति॑ग्रा॒ह्या॑ यस्य॒ यस्या॑ तिग्रा॒ह्या॑ न । \newline
39. अ॒ति॒ग्रा॒ह्या॑ न नाति॑ग्रा॒ह्या॑ अतिग्रा॒ह्या॑ न गृ॒ह्यन्ते॑ गृ॒ह्यन्ते॒ नाति॑ग्रा॒ह्या॑ अतिग्रा॒ह्या॑ न गृ॒ह्यन्ते᳚ । \newline
40. अ॒ति॒ग्रा॒ह्या॑ इत्य॑ति - ग्रा॒ह्याः᳚ । \newline
41. न गृ॒ह्यन्ते॑ गृ॒ह्यन्ते॒ न न गृ॒ह्यन्त॒ इतीति॑ गृ॒ह्यन्ते॒ न न गृ॒ह्यन्त॒ इति॑ । \newline
42. गृ॒ह्यन्त॒ इतीति॑ गृ॒ह्यन्ते॑ गृ॒ह्यन्त॒ इत्य प्यपीति॑ गृ॒ह्यन्ते॑ गृ॒ह्यन्त॒ इत्यपि॑ । \newline
43. इत्य प्यपीती त्यप्य॑ग्निष्टो॒मे᳚ ऽग्निष्टो॒मे ऽपीती त्यप्य॑ग्निष्टो॒मे । \newline
44. अप्य॑ ग्निष्टो॒मे᳚ ऽग्निष्टो॒मे ऽप्यप्य॑ ग्निष्टो॒मे ग्र॑हीत॒व्या᳚ ग्रहीत॒व्या॑ अग्निष्टो॒मे ऽप्यप्य॑ ग्निष्टो॒मे ग्र॑हीत॒व्याः᳚ । \newline
45. अ॒ग्नि॒ष्टो॒मे ग्र॑हीत॒व्या᳚ ग्रहीत॒व्या॑ अग्निष्टो॒मे᳚ ऽग्निष्टो॒मे ग्र॑हीत॒व्या॑ य॒ज्ञ्स्य॑ य॒ज्ञ्स्य॑ ग्रहीत॒व्या॑ अग्निष्टो॒मे᳚ ऽग्निष्टो॒मे ग्र॑हीत॒व्या॑ य॒ज्ञ्स्य॑ । \newline
46. अ॒ग्नि॒ष्टो॒म इत्य॑ग्नि - स्तो॒मे । \newline
47. ग्र॒ही॒त॒व्या॑ य॒ज्ञ्स्य॑ य॒ज्ञ्स्य॑ ग्रहीत॒व्या᳚ ग्रहीत॒व्या॑ य॒ज्ञ्स्य॑ सतनु॒त्वाय॑ सतनु॒त्वाय॑ य॒ज्ञ्स्य॑ ग्रहीत॒व्या᳚ ग्रहीत॒व्या॑ य॒ज्ञ्स्य॑ सतनु॒त्वाय॑ । \newline
48. य॒ज्ञ्स्य॑ सतनु॒त्वाय॑ सतनु॒त्वाय॑ य॒ज्ञ्स्य॑ य॒ज्ञ्स्य॑ सतनु॒त्वाय॑ दे॒वता॑ दे॒वताः᳚ सतनु॒त्वाय॑ य॒ज्ञ्स्य॑ य॒ज्ञ्स्य॑ सतनु॒त्वाय॑ दे॒वताः᳚ । \newline
49. स॒त॒नु॒त्वाय॑ दे॒वता॑ दे॒वताः᳚ सतनु॒त्वाय॑ सतनु॒त्वाय॑ दे॒वता॒ वै वै दे॒वताः᳚ सतनु॒त्वाय॑ सतनु॒त्वाय॑ दे॒वता॒ वै । \newline
50. स॒त॒नु॒त्वायेति॑ सतनु - त्वाय॑ । \newline
51. दे॒वता॒ वै वै दे॒वता॑ दे॒वता॒ वै सर्वाः॒ सर्वा॒ वै दे॒वता॑ दे॒वता॒ वै सर्वाः᳚ । \newline
52. वै सर्वाः॒ सर्वा॒ वै वै सर्वाः᳚ स॒दृशीः᳚ स॒दृशीः॒ सर्वा॒ वै वै सर्वाः᳚ स॒दृशीः᳚ । \newline
53. सर्वाः᳚ स॒दृशीः᳚ स॒दृशीः॒ सर्वाः॒ सर्वाः᳚ स॒दृशी॑ रासन् नासन् थ्स॒दृशीः॒ सर्वाः॒ सर्वाः᳚ स॒दृशी॑ रासन्न् । \newline
54. स॒दृशी॑ रासन् नासन् थ्स॒दृशीः᳚ स॒दृशी॑ रास॒न् ता स्ता आ॑सन् थ्स॒दृशीः᳚ स॒दृशी॑ रास॒न् ताः । \newline
55. आ॒स॒न् ता स्ता आ॑सन् नास॒न् ता न न ता आ॑सन् नास॒न् ता न । \newline
56. ता न न ता स्ता न व्या॒वृतं॑ ॅव्या॒वृत॒न् न ता स्ता न व्या॒वृत᳚म् । \newline
57. न व्या॒वृतं॑ ॅव्या॒वृत॒न् न न व्या॒वृत॑ मगच्छन् नगच्छन् व्या॒वृत॒न् न न व्या॒वृत॑ मगच्छन्न् । \newline
58. व्या॒वृत॑ मगच्छन् नगच्छन् व्या॒वृतं॑ ॅव्या॒वृत॑ मगच्छ॒न् ते ते॑ ऽगच्छन् व्या॒वृतं॑ ॅव्या॒वृत॑ मगच्छ॒न् ते । \newline
59. व्या॒वृत॒मिति॑ वि - आ॒वृत᳚म् । \newline
60. अ॒ग॒च्छ॒न् ते ते॑ ऽगच्छन् नगच्छ॒न् ते दे॒वा दे॒वा स्ते॑ ऽगच्छन् नगच्छ॒न् ते दे॒वाः । \newline
61. ते दे॒वा दे॒वा स्ते ते दे॒वा ए॒त ए॒ते दे॒वा स्ते ते दे॒वा ए॒ते । \newline
62. दे॒वा ए॒त ए॒ते दे॒वा दे॒वा ए॒त ए॒ता ने॒ता ने॒ते दे॒वा दे॒वा ए॒त ए॒तान् । \newline
\pagebreak
\markright{ TS 6.6.8.3  \hfill https://www.vedavms.in \hfill}

\section{ TS 6.6.8.3 }

\textbf{TS 6.6.8.3 } \newline
\textbf{Samhita Paata} \newline

ए॒त ए॒तान् ग्रहा॑नपश्य॒न् तान॑गृह्णताऽऽ*ग्ने॒ यम॒ग्निरै॒न्द्रमिन्द्रः॑ सौ॒र्यꣳ सूर्य॒स्ततो॒ वै ते᳚ऽन्याभि॑-र्दे॒वता॑भि-र्व्या॒वृत॑मगच्छ॒न्॒. यस्यै॒वं ॅवि॒दुष॑ ए॒ते ग्रहा॑ गृ॒ह्यन्ते᳚ व्या॒वृत॑मे॒व पा॒प्मना॒ भ्रातृ॑व्येण गच्छती॒मे लो॒का ज्योति॑ष्मन्तः स॒माव॑द्-वीर्याः का॒र्या॑ इत्या॑हुराग्ने॒येना॒स्मिन् ॅलो॒के ज्योति॑र्द्धत्त ऐ॒न्द्रेणा॒न्तरि॑क्ष इन्द्रवा॒यू हि स॒युजौ॑ सौ॒र्येणा॒मुष्मि॑न् ॅलो॒के- [  ] \newline

\textbf{Pada Paata} \newline

ए॒ते । ए॒तान् । ग्रहान्॑ । अ॒प॒श्य॒न्न् । तान् । अ॒गृ॒ह्ण॒त॒ । आ॒ग्ने॒यम् । अ॒ग्निः । ऐ॒न्द्रम् । इन्द्रः॑ । सौ॒र्यम् । सूर्यः॑ । ततः॑ । वै । ते । अ॒न्याभिः॑ । दे॒वता॑भिः । व्या॒वृत॒मिति॑ वि - आ॒वृत᳚म् । अ॒ग॒च्छ॒न्न् । यस्य॑ । ए॒वम् । वि॒दुषः॑ । ए॒ते । ग्रहाः᳚ । गृ॒ह्यन्ते᳚ । व्या॒वृत॒मिति॑ वि-आ॒वृत᳚म् । ए॒व । पा॒प्मना᳚ । भ्रातृ॑व्येण । ग॒च्छ॒ति॒ । इ॒मे । लो॒काः । ज्योति॑ष्मन्तः । स॒माव॑द्वीर्या॒ इति॑ स॒माव॑त् - वी॒र्याः॒ । का॒र्याः᳚ । इति॑ । आ॒हुः॒ । आ॒ग्ने॒येन॑ । अ॒स्मिन्न् । लो॒के । ज्योतिः॑ । ध॒त्ते॒ । ऐ॒न्द्रेण॑ । अ॒न्तरि॑क्षे । इ॒न्द्र॒वा॒यू इती᳚न्द्र - वा॒यू । हि । स॒युजा॒विति॑ स-युजौ᳚ । सौ॒र्येण॑ । अ॒मुष्मिन्न्॑ । लो॒के ।  \newline


\textbf{Krama Paata} \newline

ए॒त ए॒तान् । ए॒तान् ग्रहान्॑ । ग्रहा॑नपश्यन्न् । अ॒प॒श्य॒न् तान् । तान॑गृह्णत । अ॒गृ॒ह्ण॒ता॒ग्ने॒यम् । आ॒ग्ने॒यम॒ग्निः । अ॒ग्निरै॒न्द्रम् । ऐ॒न्द्रमिन्द्रः॑ । इन्द्रः॑ सौ॒र्यम् । सौ॒र्यꣳ सूर्यः॑ । सूर्य॒स्ततः॑ । ततो॒ वै । वै ते । ते᳚ऽन्याभिः॑ । अ॒न्याभि॑र् दे॒वता॑भिः । दे॒वता॑भिर् व्या॒वृत᳚म् । व्या॒वृत॑मगच्छन्न् । व्या॒वृत॒मिति॑ वि - आ॒वृत᳚म् । अ॒ग॒च्छ॒न्॒. यस्य॑ । यस्यै॒वम् । ए॒वम् ॅवि॒दुषः॑ । वि॒दुष॑ ए॒ते । ए॒ते ग्रहाः᳚ । ग्रहा॑ गृ॒ह्यन्ते᳚ । गृ॒ह्यन्ते᳚ व्या॒वृत᳚म् । व्या॒वृत॑मे॒व । व्या॒वृत॒मिति॑ वि - आ॒वृत᳚म् । ए॒व पा॒प्मना᳚ । पा॒प्मना॒ भ्रातृ॑व्येण । भ्रातृ॑व्येण गच्छति । ग॒च्छ॒ती॒मे । इ॒मे लो॒काः । लो॒का ज्योति॑ष्मन्तः । ज्योति॑ष्मन्तः स॒माव॑द्वीर्याः । स॒माव॑द्वीर्याः का॒र्याः᳚ । स॒माव॑द्वीर्या॒ इति॑ स॒माव॑त् - वी॒र्याः॒ । का॒र्या॑ इति॑ । इत्या॑हुः । आ॒हु॒रा॒ग्ने॒येन॑ । आ॒ग्ने॒येना॒स्मिन्न् । अ॒स्मिन् ॅलो॒के । लो॒के ज्योतिः॑ । ज्योति॑र् धत्ते । ध॒त्त॒ ऐ॒न्द्रेण॑ । ऐ॒न्द्रेणा॒न्तरि॑क्षे । अ॒न्तरि॑क्ष इन्द्रवा॒यू । इ॒न्द्र॒वा॒यू हि । इ॒न्द्र॒वा॒यू इती᳚न्द्र - वा॒यू । हि स॒युजौ᳚ । स॒युजौ॑ सौ॒र्येण॑ । स॒युजा॒विति॑ स - युजौ᳚ । सौ॒र्येणा॒मुष्मिन्न्॑ । अ॒मुष्मि॑न् ॅलो॒के ( ) । लो॒के ज्योतिः॑ \newline

\textbf{Jatai Paata} \newline

1. ए॒त ए॒ता ने॒ता ने॒त ए॒त ए॒तान् । \newline
2. ए॒तान् ग्रहा॒न् ग्रहा॑ ने॒ता ने॒तान् ग्रहान्॑ । \newline
3. ग्रहा॑ नपश्यन् नपश्य॒न् ग्रहा॒न् ग्रहा॑ नपश्यन्न् । \newline
4. अ॒प॒श्य॒न् ताꣳ स्ता न॑पश्यन् नपश्य॒न् तान् । \newline
5. तान॑गृह्णता गृह्णत॒ ताꣳ स्तान॑गृह्णत । \newline
6. अ॒गृ॒ह्ण॒ता॒ ग्ने॒य मा᳚ग्ने॒य म॑गृह्णता गृह्णता ग्ने॒यम् । \newline
7. आ॒ग्ने॒य म॒ग्नि र॒ग्नि रा᳚ग्ने॒य मा᳚ग्ने॒य म॒ग्निः । \newline
8. अ॒ग्नि रै॒न्द्र मै॒न्द्र म॒ग्नि र॒ग्नि रै॒न्द्रम् । \newline
9. ऐ॒न्द्र मिन्द्र॒ इन्द्र॑ ऐ॒न्द्र मै॒न्द्र मिन्द्रः॑ । \newline
10. इन्द्रः॑ सौ॒र्यꣳ सौ॒र्य मिन्द्र॒ इन्द्रः॑ सौ॒र्यम् । \newline
11. सौ॒र्यꣳ सूर्यः॒ सूर्यः॑ सौ॒र्यꣳ सौ॒र्यꣳ सूर्यः॑ । \newline
12. सूर्य॒ स्तत॒ स्ततः॒ सूर्यः॒ सूर्य॒ स्ततः॑ । \newline
13. ततो॒ वै वै तत॒ स्ततो॒ वै । \newline
14. वै ते ते वै वै ते । \newline
15. ते᳚ ऽन्याभि॑ र॒न्याभि॒ स्ते ते᳚ ऽन्याभिः॑ । \newline
16. अ॒न्याभि॑र् दे॒वता॑भिर् दे॒वता॑भि र॒न्याभि॑ र॒न्याभि॑र् दे॒वता॑भिः । \newline
17. दे॒वता॑भिर् व्या॒वृतं॑ ॅव्या॒वृत॑म् दे॒वता॑भिर् दे॒वता॑भिर् व्या॒वृत᳚म् । \newline
18. व्या॒वृत॑ मगच्छन् नगच्छन् व्या॒वृतं॑ ॅव्या॒वृत॑ मगच्छन्न् । \newline
19. व्या॒वृत॒मिति॑ वि - आ॒वृत᳚म् । \newline
20. अ॒ग॒च्छ॒न्॒. यस्य॒ यस्या॑गच्छन् नगच्छ॒न्॒. यस्य॑ । \newline
21. यस्यै॒व मे॒वं ॅयस्य॒ यस्यै॒वम् । \newline
22. ए॒वं ॅवि॒दुषो॑ वि॒दुष॑ ए॒व मे॒वं ॅवि॒दुषः॑ । \newline
23. वि॒दुष॑ ए॒त ए॒ते वि॒दुषो॑ वि॒दुष॑ ए॒ते । \newline
24. ए॒ते ग्रहा॒ ग्रहा॑ ए॒त ए॒ते ग्रहाः᳚ । \newline
25. ग्रहा॑ गृ॒ह्यन्ते॑ गृ॒ह्यन्ते॒ ग्रहा॒ ग्रहा॑ गृ॒ह्यन्ते᳚ । \newline
26. गृ॒ह्यन्ते᳚ व्या॒वृतं॑ ॅव्या॒वृत॑म् गृ॒ह्यन्ते॑ गृ॒ह्यन्ते᳚ व्या॒वृत᳚म् । \newline
27. व्या॒वृत॑ मे॒वैव व्या॒वृतं॑ ॅव्या॒वृत॑ मे॒व । \newline
28. व्या॒वृत॒मिति॑ वि - आ॒वृत᳚म् । \newline
29. ए॒व पा॒प्मना॑ पा॒प्म नै॒वैव पा॒प्मना᳚ । \newline
30. पा॒प्मना॒ भ्रातृ॑व्येण॒ भ्रातृ॑व्येण पा॒प्मना॑ पा॒प्मना॒ भ्रातृ॑व्येण । \newline
31. भ्रातृ॑व्येण गच्छति गच्छति॒ भ्रातृ॑व्येण॒ भ्रातृ॑व्येण गच्छति । \newline
32. ग॒च्छ॒ ती॒म इ॒मे ग॑च्छति गच्छ ती॒मे । \newline
33. इ॒मे लो॒का लो॒का इ॒म इ॒मे लो॒काः । \newline
34. लो॒का ज्योति॑ष्मन्तो॒ ज्योति॑ष्मन्तो लो॒का लो॒का ज्योति॑ष्मन्तः । \newline
35. ज्योति॑ष्मन्तः स॒माव॑द्वीर्याः स॒माव॑द्वीर्या॒ ज्योति॑ष्मन्तो॒ ज्योति॑ष्मन्तः स॒माव॑द्वीर्याः । \newline
36. स॒माव॑द्वीर्याः का॒र्याः᳚ का॒र्याः᳚ स॒माव॑द्वीर्याः स॒माव॑द्वीर्याः का॒र्याः᳚ । \newline
37. स॒माव॑द्वीर्या॒ इति॑ स॒माव॑त् - वी॒र्याः॒ । \newline
38. का॒र्या॑ इतीति॑ का॒र्याः᳚ का॒र्या॑ इति॑ । \newline
39. इत्या॑हु राहु॒ रिती त्या॑हुः । \newline
40. आ॒हु॒ रा॒ग्ने॒येना᳚ ग्ने॒येना॑हु राहुरा ग्ने॒येन॑ । \newline
41. आ॒ग्ने॒ये ना॒स्मिन् न॒स्मिन् ना᳚ग्ने॒येना᳚ ग्ने॒येना॒स्मिन्न् । \newline
42. अ॒स्मिन् ॅलो॒के लो॒के᳚ ऽस्मिन् न॒स्मिन् ॅलो॒के । \newline
43. लो॒के ज्योति॒र् ज्योति॑र् लो॒के लो॒के ज्योतिः॑ । \newline
44. ज्योति॑र् धत्ते धत्ते॒ ज्योति॒र् ज्योति॑र् धत्ते । \newline
45. ध॒त्त॒ ऐ॒न्द्रे णै॒न्द्रेण॑ धत्ते धत्त ऐ॒न्द्रेण॑ । \newline
46. ऐ॒न्द्रेणा॒ न्तरि॑क्षे॒ ऽन्तरि॑क्ष ऐ॒न्द्रे णै॒न्द्रेणा॒ न्तरि॑क्षे । \newline
47. अ॒न्तरि॑क्ष इन्द्रवा॒यू इ॑न्द्रवा॒यू अ॒न्तरि॑क्षे॒ ऽन्तरि॑क्ष इन्द्रवा॒यू । \newline
48. इ॒न्द्र॒वा॒यू हि हीन्द्र॑वा॒यू इ॑न्द्रवा॒यू हि । \newline
49. इ॒न्द्र॒वा॒यू इती᳚न्द्र - वा॒यू । \newline
50. हि स॒युजौ॑ स॒युजौ॒ हि हि स॒युजौ᳚ । \newline
51. स॒युजौ॑ सौ॒र्येण॑ सौ॒र्येण॑ स॒युजौ॑ स॒युजौ॑ सौ॒र्येण॑ । \newline
52. स॒युजा॒विति॑ स - युजौ᳚ । \newline
53. सौ॒र्येणा॒ मुष्मि॑न् न॒मुष्मिन्᳚ थ्सौ॒र्येण॑ सौ॒र्येणा॒ मुष्मिन्न्॑ । \newline
54. अ॒मुष्मि॑न् ॅलो॒के लो॒के॑ ऽमुष्मि॑न् न॒मुष्मि॑न् ॅलो॒के । \newline
55. लो॒के ज्योति॒र् ज्योति॑र् लो॒के लो॒के ज्योतिः॑ । \newline

\textbf{Ghana Paata } \newline

1. ए॒त ए॒ता ने॒ता ने॒त ए॒त ए॒तान् ग्रहा॒न् ग्रहा॑ ने॒ता ने॒त ए॒त ए॒तान् ग्रहान्॑ । \newline
2. ए॒तान् ग्रहा॒न् ग्रहा॑ ने॒ता ने॒तान् ग्रहा॑ नपश्यन् नपश्य॒न् ग्रहा॑ ने॒ता ने॒तान् ग्रहा॑ नपश्यन्न् । \newline
3. ग्रहा॑ नपश्यन् नपश्य॒न् ग्रहा॒न् ग्रहा॑ नपश्य॒न् ताꣳ स्तान॑पश्य॒न् ग्रहा॒न् ग्रहा॑ नपश्य॒न् तान् । \newline
4. अ॒प॒श्य॒न् ताꣳ स्तान॑पश्यन् नपश्य॒न् तान॑गृह्णता गृह्णत॒ ता न॑पश्यन् नपश्य॒न् तान॑गृह्णत । \newline
5. तान॑गृह्णता गृह्णत॒ ताꣳ स्तान॑गृह्णता ग्ने॒य मा᳚ग्ने॒य म॑गृह्णत॒ ताꣳ स्तान॑गृह्णता ग्ने॒यम् । \newline
6. अ॒गृ॒ह्ण॒ता॒ ग्ने॒य मा᳚ग्ने॒य म॑गृह्णता गृह्णता ग्ने॒य म॒ग्नि र॒ग्नि रा᳚ग्ने॒य म॑गृह्णता गृह्णता ग्ने॒य म॒ग्निः । \newline
7. आ॒ग्ने॒य म॒ग्नि र॒ग्नि रा᳚ग्ने॒य मा᳚ग्ने॒य म॒ग्नि रै॒न्द्र मै॒न्द्र म॒ग्नि रा᳚ग्ने॒य मा᳚ग्ने॒य म॒ग्नि रै॒न्द्रम् । \newline
8. अ॒ग्नि रै॒न्द्र मै॒न्द्र म॒ग्नि र॒ग्नि रै॒न्द्र मिन्द्र॒ इन्द्र॑ ऐ॒न्द्र म॒ग्नि र॒ग्नि रै॒न्द्र मिन्द्रः॑ । \newline
9. ऐ॒न्द्र मिन्द्र॒ इन्द्र॑ ऐ॒न्द्र मै॒न्द्र मिन्द्रः॑ सौ॒र्यꣳ सौ॒र्य मिन्द्र॑ ऐ॒न्द्र मै॒न्द्र मिन्द्रः॑ सौ॒र्यम् । \newline
10. इन्द्रः॑ सौ॒र्यꣳ सौ॒र्य मिन्द्र॒ इन्द्रः॑ सौ॒र्यꣳ सूर्यः॒ सूर्यः॑ सौ॒र्य मिन्द्र॒ इन्द्रः॑ सौ॒र्यꣳ सूर्यः॑ । \newline
11. सौ॒र्यꣳ सूर्यः॒ सूर्यः॑ सौ॒र्यꣳ सौ॒र्यꣳ सूर्य॒ स्तत॒ स्ततः॒ सूर्यः॑ सौ॒र्यꣳ सौ॒र्यꣳ सूर्य॒ स्ततः॑ । \newline
12. सूर्य॒ स्तत॒ स्ततः॒ सूर्यः॒ सूर्य॒ स्ततो॒ वै वै ततः॒ सूर्यः॒ सूर्य॒ स्ततो॒ वै । \newline
13. ततो॒ वै वै तत॒ स्ततो॒ वै ते ते वै तत॒ स्ततो॒ वै ते । \newline
14. वै ते ते वै वै ते᳚ ऽन्याभि॑ र॒न्याभि॒ स्ते वै वै ते᳚ ऽन्याभिः॑ । \newline
15. ते᳚ ऽन्याभि॑ र॒न्याभि॒ स्ते ते᳚ ऽन्याभि॑र् दे॒वता॑भिर् दे॒वता॑भि र॒न्याभि॒ स्ते ते᳚ ऽन्याभि॑र् दे॒वता॑भिः । \newline
16. अ॒न्याभि॑र् दे॒वता॑भिर् दे॒वता॑भि र॒न्याभि॑ र॒न्याभि॑र् दे॒वता॑भिर् व्या॒वृतं॑ ॅव्या॒वृत॑म् दे॒वता॑भि र॒न्याभि॑ र॒न्याभि॑र् दे॒वता॑भिर् व्या॒वृत᳚म् । \newline
17. दे॒वता॑भिर् व्या॒वृतं॑ ॅव्या॒वृत॑म् दे॒वता॑भिर् दे॒वता॑भिर् व्या॒वृत॑ मगच्छन् नगच्छन् व्या॒वृत॑म् दे॒वता॑भिर् दे॒वता॑भिर् व्या॒वृत॑ मगच्छन्न् । \newline
18. व्या॒वृत॑ मगच्छन् नगच्छन् व्या॒वृतं॑ ॅव्या॒वृत॑ मगच्छ॒न्॒. यस्य॒ यस्या॑ गच्छन् व्या॒वृतं॑ ॅव्या॒वृत॑ मगच्छ॒न्॒. यस्य॑ । \newline
19. व्या॒वृत॒मिति॑ वि - आ॒वृत᳚म् । \newline
20. अ॒ग॒च्छ॒न्॒. यस्य॒ यस्या॑ गच्छन् नगच्छ॒न्॒. यस्यै॒व मे॒वं ॅयस्या॑ गच्छन् नगच्छ॒न्॒. यस्यै॒वम् । \newline
21. यस्यै॒व मे॒वं ॅयस्य॒ यस्यै॒वं ॅवि॒दुषो॑ वि॒दुष॑ ए॒वं ॅयस्य॒ यस्यै॒वं ॅवि॒दुषः॑ । \newline
22. ए॒वं ॅवि॒दुषो॑ वि॒दुष॑ ए॒व मे॒वं ॅवि॒दुष॑ ए॒त ए॒ते वि॒दुष॑ ए॒व मे॒वं ॅवि॒दुष॑ ए॒ते । \newline
23. वि॒दुष॑ ए॒त ए॒ते वि॒दुषो॑ वि॒दुष॑ ए॒ते ग्रहा॒ ग्रहा॑ ए॒ते वि॒दुषो॑ वि॒दुष॑ ए॒ते ग्रहाः᳚ । \newline
24. ए॒ते ग्रहा॒ ग्रहा॑ ए॒त ए॒ते ग्रहा॑ गृ॒ह्यन्ते॑ गृ॒ह्यन्ते॒ ग्रहा॑ ए॒त ए॒ते ग्रहा॑ गृ॒ह्यन्ते᳚ । \newline
25. ग्रहा॑ गृ॒ह्यन्ते॑ गृ॒ह्यन्ते॒ ग्रहा॒ ग्रहा॑ गृ॒ह्यन्ते᳚ व्या॒वृतं॑ ॅव्या॒वृत॑म् गृ॒ह्यन्ते॒ ग्रहा॒ ग्रहा॑ गृ॒ह्यन्ते᳚ व्या॒वृत᳚म् । \newline
26. गृ॒ह्यन्ते᳚ व्या॒वृतं॑ ॅव्या॒वृत॑म् गृ॒ह्यन्ते॑ गृ॒ह्यन्ते᳚ व्या॒वृत॑ मे॒वैव व्या॒वृत॑म् गृ॒ह्यन्ते॑ गृ॒ह्यन्ते᳚ व्या॒वृत॑ मे॒व । \newline
27. व्या॒वृत॑ मे॒वैव व्या॒वृतं॑ ॅव्या॒वृत॑ मे॒व पा॒प्मना॑ पा॒प्म नै॒व व्या॒वृतं॑ ॅव्या॒वृत॑ मे॒व पा॒प्मना᳚ । \newline
28. व्या॒वृत॒मिति॑ वि - आ॒वृत᳚म् । \newline
29. ए॒व पा॒प्मना॑ पा॒प्म नै॒वैव पा॒प्मना॒ भ्रातृ॑व्येण॒ भ्रातृ॑व्येण पा॒प्म नै॒वैव पा॒प्मना॒ भ्रातृ॑व्येण । \newline
30. पा॒प्मना॒ भ्रातृ॑व्येण॒ भ्रातृ॑व्येण पा॒प्मना॑ पा॒प्मना॒ भ्रातृ॑व्येण गच्छति गच्छति॒ भ्रातृ॑व्येण पा॒प्मना॑ पा॒प्मना॒ भ्रातृ॑व्येण गच्छति । \newline
31. भ्रातृ॑व्येण गच्छति गच्छति॒ भ्रातृ॑व्येण॒ भ्रातृ॑व्येण गच्छती॒म इ॒मे ग॑च्छति॒ भ्रातृ॑व्येण॒ भ्रातृ॑व्येण गच्छती॒मे । \newline
32. ग॒च्छ॒ती॒म इ॒मे ग॑च्छति गच्छती॒मे लो॒का लो॒का इ॒मे ग॑च्छति गच्छती॒मे लो॒काः । \newline
33. इ॒मे लो॒का लो॒का इ॒म इ॒मे लो॒का ज्योति॑ष्मन्तो॒ ज्योति॑ष्मन्तो लो॒का इ॒म इ॒मे लो॒का ज्योति॑ष्मन्तः । \newline
34. लो॒का ज्योति॑ष्मन्तो॒ ज्योति॑ष्मन्तो लो॒का लो॒का ज्योति॑ष्मन्तः स॒माव॑द्वीर्याः स॒माव॑द्वीर्या॒ ज्योति॑ष्मन्तो लो॒का लो॒का ज्योति॑ष्मन्तः स॒माव॑द्वीर्याः । \newline
35. ज्योति॑ष्मन्तः स॒माव॑द्वीर्याः स॒माव॑द्वीर्या॒ ज्योति॑ष्मन्तो॒ ज्योति॑ष्मन्तः स॒माव॑द्वीर्याः का॒र्याः᳚ का॒र्याः᳚ स॒माव॑द्वीर्या॒ ज्योति॑ष्मन्तो॒ ज्योति॑ष्मन्तः स॒माव॑द्वीर्याः का॒र्याः᳚ । \newline
36. स॒माव॑द्वीर्याः का॒र्याः᳚ का॒र्याः᳚ स॒माव॑द्वीर्याः स॒माव॑द्वीर्याः का॒र्या॑ इतीति॑ का॒र्याः᳚ स॒माव॑द्वीर्याः स॒माव॑द्वीर्याः का॒र्या॑ इति॑ । \newline
37. स॒माव॑द्वीर्या॒ इति॑ स॒माव॑त् - वी॒र्याः॒ । \newline
38. का॒र्या॑ इतीति॑ का॒र्याः᳚ का॒र्या॑ इत्या॑हु राहु॒ रिति॑ का॒र्याः᳚ का॒र्या॑ इत्या॑हुः । \newline
39. इत्या॑हु राहु॒ रिती त्या॑हु राग्ने॒येना᳚ ग्ने॒येना॑ हु॒रितीत्या॑हु राग्ने॒येन॑ । \newline
40. आ॒हु॒ रा॒ग्ने॒येना᳚ ग्ने॒येना॑हु राहु राग्ने॒ये ना॒स्मिन् न॒स्मिन् ना᳚ग्ने॒येना॑हु राहु राग्ने॒ये ना॒स्मिन्न् । \newline
41. आ॒ग्ने॒ये ना॒स्मिन् न॒स्मिन् ना᳚ग्ने॒येना᳚ ग्ने॒ये ना॒स्मिन् ॅलो॒के लो॒के᳚ ऽस्मिन् ना᳚ग्ने॒येना᳚ग्ने॒ये ना॒स्मिन् ॅलो॒के । \newline
42. अ॒स्मिन् ॅलो॒के लो॒के᳚ ऽस्मिन् न॒स्मिन् ॅलो॒के ज्योति॒र् ज्योति॑र् लो॒के᳚ ऽस्मिन् न॒स्मिन् ॅलो॒के ज्योतिः॑ । \newline
43. लो॒के ज्योति॒र् ज्योति॑र् लो॒के लो॒के ज्योति॑र् धत्ते धत्ते॒ ज्योति॑र् लो॒के लो॒के ज्योति॑र् धत्ते । \newline
44. ज्योति॑र् धत्ते धत्ते॒ ज्योति॒र् ज्योति॑र् धत्त ऐ॒न्द्रे णै॒न्द्रेण॑ धत्ते॒ ज्योति॒र् ज्योति॑र् धत्त ऐ॒न्द्रेण॑ । \newline
45. ध॒त्त॒ ऐ॒न्द्रे णै॒न्द्रेण॑ धत्ते धत्त ऐ॒न्द्रेणा॒ न्तरि॑क्षे॒ ऽन्तरि॑क्ष ऐ॒न्द्रेण॑ धत्ते धत्त ऐ॒न्द्रेणा॒ न्तरि॑क्षे । \newline
46. ऐ॒न्द्रेणा॒ न्तरि॑क्षे॒ ऽन्तरि॑क्ष ऐ॒न्द्रे णै॒न्द्रेणा॒ न्तरि॑क्ष इन्द्रवा॒यू इ॑न्द्रवा॒यू अ॒न्तरि॑क्ष ऐ॒न्द्रे
णै॒न्द्रेणा॒ न्तरि॑क्ष इन्द्रवा॒यू । \newline
47. अ॒न्तरि॑क्ष इन्द्रवा॒यू इ॑न्द्रवा॒यू अ॒न्तरि॑क्षे॒ ऽन्तरि॑क्ष इन्द्रवा॒यू हि हीन्द्र॑वा॒यू अ॒न्तरि॑क्षे॒ ऽन्तरि॑क्ष इन्द्रवा॒यू हि । \newline
48. इ॒न्द्र॒वा॒यू हि हीन्द्र॑वा॒यू इ॑न्द्रवा॒यू हि स॒युजौ॑ स॒युजौ॒ हीन्द्र॑वा॒यू इ॑न्द्रवा॒यू हि स॒युजौ᳚ । \newline
49. इ॒न्द्र॒वा॒यू इती᳚न्द्र - वा॒यू । \newline
50. हि स॒युजौ॑ स॒युजौ॒ हि हि स॒युजौ॑ सौ॒र्येण॑ सौ॒र्येण॑ स॒युजौ॒ हि हि स॒युजौ॑ सौ॒र्येण॑ । \newline
51. स॒युजौ॑ सौ॒र्येण॑ सौ॒र्येण॑ स॒युजौ॑ स॒युजौ॑ सौ॒र्येणा॒ मुष्मि॑न् न॒मुष्मिन्᳚ थ्सौ॒र्येण॑ स॒युजौ॑ स॒युजौ॑ सौ॒र्येणा॒ मुष्मिन्न्॑ । \newline
52. स॒युजा॒विति॑ स - युजौ᳚ । \newline
53. सौ॒र्येणा॒ मुष्मि॑न् न॒मुष्मिन्᳚ थ्सौ॒र्येण॑ सौ॒र्येणा॒ मुष्मि॑न् ॅलो॒के लो॒के॑ ऽमुष्मिन्᳚ थ्सौ॒र्येण॑ सौ॒र्येणा॒ मुष्मि॑न् ॅलो॒के । \newline
54. अ॒मुष्मि॑न् ॅलो॒के लो॒के॑ ऽमुष्मि॑न् न॒मुष्मि॑न् ॅलो॒के ज्योति॒र् ज्योति॑र् लो॒के॑ ऽमुष्मि॑न् न॒मुष्मि॑न् ॅलो॒के ज्योतिः॑ । \newline
55. लो॒के ज्योति॒र् ज्योति॑र् लो॒के लो॒के ज्योति॑र् धत्ते धत्ते॒ ज्योति॑र् लो॒के लो॒के ज्योति॑र् धत्ते । \newline
\pagebreak
\markright{ TS 6.6.8.4  \hfill https://www.vedavms.in \hfill}

\section{ TS 6.6.8.4 }

\textbf{TS 6.6.8.4 } \newline
\textbf{Samhita Paata} \newline

ज्योति॑र्द्धत्ते॒ ज्योति॑ष्मन्तोऽस्मा इ॒मे लो॒का भ॑वन्ति स॒माव॑द्-वीर्यानेनान् कुरुत ए॒तान्. वै ग्रहा᳚न् ब॒बां-वि॒श्वव॑यसा-ववित्तां॒ ताभ्या॑मि॒मे लो॒काः परा᳚ञ्चश्चा॒र्वाञ्च॑श्च॒ प्राभु॒र्यस्यै॒वं ॅवि॒दुष॑ ए॒ते ग्रहा॑ गृ॒ह्यन्ते॒ प्रास्मा॑ इ॒मे लो॒काः परा᳚ञ्चश्चा॒र्वाञ्च॑श्च भान्ति ॥ \newline

\textbf{Pada Paata} \newline

ज्योतिः॑ । ध॒त्ते॒ । ज्योति॑ष्मन्तः । अ॒स्मै॒ । इ॒मे । लो॒काः । भ॒व॒न्ति॒ । स॒माव॑त्वीर्या॒निति॑ स॒माव॑त् - वी॒र्या॒न् । ए॒ना॒न् । कु॒रु॒ते॒ । ए॒तान् । वै । ग्रहान्॑ । ब॒बांवि॒श्वव॑यसा॒विति॑ ब॒बां - वि॒श्वव॑यसौ । अ॒वि॒त्ता॒म् । ताभ्या᳚म् । इ॒मे । लो॒काः । परा᳚ञ्चः । च॒ । अ॒र्वाञ्चः॑ । च॒ । प्रेति॑ । अ॒भुः॒ । यस्य॑ । ए॒वम् । वि॒दुषः॑ । ए॒ते । ग्रहाः᳚ । गृ॒ह्यन्ते᳚ । प्रेति॑ । अ॒स्मै॒ । इ॒मे । लो॒काः । परा᳚ञ्चः । च॒ । अ॒र्वाञ्चः॑ । च॒ । भा॒न्ति॒ ॥  \newline


\textbf{Krama Paata} \newline

ज्योति॑र् धत्ते । ध॒त्ते॒ ज्योति॑ष्मन्तः । ज्योति॑ष्मन्तोऽस्मै । अ॒स्मा॒ इ॒मे । इ॒मे लो॒काः । लो॒का भ॑वन्ति । भ॒व॒न्ति॒ स॒माव॑द्वीर्यान् । स॒माव॑द्वीर्यानेनान् । स॒माव॑द्वीर्या॒निति॑ स॒माव॑त् - वी॒र्या॒न्॒ । ए॒ना॒न् कु॒रु॒ते॒ । कु॒रु॒त॒ ए॒तान् । ए॒तान्. वै । वै ग्रहान्॑ । ग्रहा᳚न् ब॒म्बावि॒श्वव॑यसौ । ब॒म्बावि॒श्व,व॑यसाववित्ताम् । ब॒म्बावि॒श्वव॑यसा॒विति॑ ब॒म्बा - वि॒श्वव॑यसौ । अ॒वि॒त्ता॒म् ताभ्या᳚म् । ताभ्या॑मि॒मे । इ॒मे लो॒काः । लो॒काः परा᳚ञ्चः । परा᳚ञ्चश्च । चा॒र्वाञ्चः॑ । अ॒र्वाञ्च॑श्च । च॒ प्र । प्राभुः॑ । अ॒भु॒र् यस्य॑ । यस्यै॒वम् । ए॒वम् ॅवि॒दुषः॑ । वि॒दुष॑ ए॒ते । ए॒ते ग्रहाः᳚ । ग्रहा॑ गृ॒ह्यन्ते᳚ । गृ॒ह्यन्ते॒ प्र । प्रास्मै᳚ । अ॒स्मा॒ इ॒मे । इ॒मे लो॒काः । लो॒काः परा᳚ञ्चः । परा᳚ञ्चश्च । चा॒र्वाञ्चः॑ । अ॒र्वाञ्च॑श्च । च॒ भा॒न्ति॒ । भा॒न्तीति॑ भान्ति । \newline

\textbf{Jatai Paata} \newline

1. ज्योति॑र् धत्ते धत्ते॒ ज्योति॒र् ज्योति॑र् धत्ते । \newline
2. ध॒त्ते॒ ज्योति॑ष्मन्तो॒ ज्योति॑ष्मन्तो धत्ते धत्ते॒ ज्योति॑ष्मन्तः । \newline
3. ज्योति॑ष्मन्तो ऽस्मा अस्मै॒ ज्योति॑ष्मन्तो॒ ज्योति॑ष्मन्तो ऽस्मै । \newline
4. अ॒स्मा॒ इ॒म इ॒मे᳚ ऽस्मा अस्मा इ॒मे । \newline
5. इ॒मे लो॒का लो॒का इ॒म इ॒मे लो॒काः । \newline
6. लो॒का भ॑वन्ति भवन्ति लो॒का लो॒का भ॑वन्ति । \newline
7. भ॒व॒न्ति॒ स॒माव॑द्वीर्यान् थ्स॒माव॑द्वीर्यान् भवन्ति भवन्ति स॒माव॑द्वीर्यान् । \newline
8. स॒माव॑द्वीर्या नेना नेनान् थ्स॒माव॑द्वीर्यान् थ्स॒माव॑द्वीर्या नेनान् । \newline
9. स॒माव॑द्वीर्या॒निति॑ स॒माव॑त् - वी॒र्या॒न् । \newline
10. ए॒ना॒न् कु॒रु॒ते॒ कु॒रु॒त॒ ए॒ना॒ ने॒ना॒न् कु॒रु॒ते॒ । \newline
11. कु॒रु॒त॒ ए॒ता ने॒तान् कु॑रुते कुरुत ए॒तान् । \newline
12. ए॒तान्. वै वा ए॒ता ने॒तान्. वै । \newline
13. वै ग्रहा॒न् ग्रहा॒न्॒. वै वै ग्रहान्॑ । \newline
14. ग्रहा᳚न् बं॒बावि॒श्वव॑यसौ बं॒बावि॒श्वव॑यसौ॒ ग्रहा॒न् ग्रहा᳚न् बं॒बावि॒श्वव॑यसौ । \newline
15. बं॒बावि॒श्वव॑यसा ववित्ता मवित्ताम् बं॒बावि॒श्वव॑यसौ बं॒बावि॒श्वव॑यसा ववित्ताम् । \newline
16. बं॒बावि॒श्वव॑यसा॒विति॑ बं॒बा - वि॒श्वव॑यसौ । \newline
17. अ॒वि॒त्ता॒म् ताभ्या॒म् ताभ्या॑ मवित्ता मवित्ता॒म् ताभ्या᳚म् । \newline
18. ताभ्या॑ मि॒म इ॒मे ताभ्या॒म् ताभ्या॑ मि॒मे । \newline
19. इ॒मे लो॒का लो॒का इ॒म इ॒मे लो॒काः । \newline
20. लो॒काः परा᳚ञ्चः॒ परा᳚ञ्चो लो॒का लो॒काः परा᳚ञ्चः । \newline
21. परा᳚ञ्चश्च च॒ परा᳚ञ्चः॒ परा᳚ञ्चश्च । \newline
22. चा॒र्वाञ्चो॒ ऽर्वाञ्च॑श्च चा॒र्वाञ्चः॑ । \newline
23. अ॒र्वाञ्च॑श्च चा॒र्वाञ्चो॒ ऽर्वाञ्च॑श्च । \newline
24. च॒ प्र प्र च॑ च॒ प्र । \newline
25. प्राभु॑ रभुः॒ प्र प्राभुः॑ । \newline
26. अ॒भु॒र् यस्य॒ यस्या॑भु रभु॒र् यस्य॑ । \newline
27. यस्यै॒व मे॒वं ॅयस्य॒ यस्यै॒वम् । \newline
28. ए॒वं ॅवि॒दुषो॑ वि॒दुष॑ ए॒व मे॒वं ॅवि॒दुषः॑ । \newline
29. वि॒दुष॑ ए॒त ए॒ते वि॒दुषो॑ वि॒दुष॑ ए॒ते । \newline
30. ए॒ते ग्रहा॒ ग्रहा॑ ए॒त ए॒ते ग्रहाः᳚ । \newline
31. ग्रहा॑ गृ॒ह्यन्ते॑ गृ॒ह्यन्ते॒ ग्रहा॒ ग्रहा॑ गृ॒ह्यन्ते᳚ । \newline
32. गृ॒ह्यन्ते॒ प्र प्र गृ॒ह्यन्ते॑ गृ॒ह्यन्ते॒ प्र । \newline
33. प्रास्मा॑ अस्मै॒ प्र प्रास्मै᳚ । \newline
34. अ॒स्मा॒ इ॒म इ॒मे᳚ ऽस्मा अस्मा इ॒मे । \newline
35. इ॒मे लो॒का लो॒का इ॒म इ॒मे लो॒काः । \newline
36. लो॒काः परा᳚ञ्चः॒ परा᳚ञ्चो लो॒का लो॒काः परा᳚ञ्चः । \newline
37. परा᳚ञ्च श्च च॒ परा᳚ञ्चः॒ परा᳚ञ्च श्च । \newline
38. चा॒र्वाञ्चो॒ ऽर्वाञ्च॑श्च चा॒र्वाञ्चः॑ । \newline
39. अ॒र्वाञ्च॑श्च चा॒र्वाञ्चो॒ ऽर्वाञ्च॑श्च । \newline
40. च॒ भा॒न्ति॒ भा॒न्ति॒ च॒ च॒ भा॒न्ति॒ । \newline
41. भा॒न्तीति॑ भान्ति । \newline

\textbf{Ghana Paata } \newline

1. ज्योति॑र् धत्ते धत्ते॒ ज्योति॒र् ज्योति॑र् धत्ते॒ ज्योति॑ष्मन्तो॒ ज्योति॑ष्मन्तो धत्ते॒ ज्योति॒र् ज्योति॑र् धत्ते॒ ज्योति॑ष्मन्तः । \newline
2. ध॒त्ते॒ ज्योति॑ष्मन्तो॒ ज्योति॑ष्मन्तो धत्ते धत्ते॒ ज्योति॑ष्मन्तो ऽस्मा अस्मै॒ ज्योति॑ष्मन्तो धत्ते धत्ते॒ ज्योति॑ष्मन्तो ऽस्मै । \newline
3. ज्योति॑ष्मन्तो ऽस्मा अस्मै॒ ज्योति॑ष्मन्तो॒ ज्योति॑ष्मन्तो ऽस्मा इ॒म इ॒मे᳚ ऽस्मै॒ ज्योति॑ष्मन्तो॒ ज्योति॑ष्मन्तो ऽस्मा इ॒मे । \newline
4. अ॒स्मा॒ इ॒म इ॒मे᳚ ऽस्मा अस्मा इ॒मे लो॒का लो॒का इ॒मे᳚ ऽस्मा अस्मा इ॒मे लो॒काः । \newline
5. इ॒मे लो॒का लो॒का इ॒म इ॒मे लो॒का भ॑वन्ति भवन्ति लो॒का इ॒म इ॒मे लो॒का भ॑वन्ति । \newline
6. लो॒का भ॑वन्ति भवन्ति लो॒का लो॒का भ॑वन्ति स॒माव॑द्वीर्यान् थ्स॒माव॑द्वीर्यान् भवन्ति लो॒का लो॒का भ॑वन्ति स॒माव॑द्वीर्यान् । \newline
7. भ॒व॒न्ति॒ स॒माव॑द्वीर्यान् थ्स॒माव॑द्वीर्यान् भवन्ति भवन्ति स॒माव॑द्वीर्या नेना नेनान् थ्स॒माव॑द्वीर्यान् भवन्ति भवन्ति स॒माव॑द्वीर्या नेनान् । \newline
8. स॒माव॑द्वीर्या नेना नेनान् थ्स॒माव॑द्वीर्यान् थ्स॒माव॑द्वीर्या नेनान् कुरुते कुरुत एनान् थ्स॒माव॑द्वीर्यान् थ्स॒माव॑द्वीर्या नेनान् कुरुते । \newline
9. स॒माव॑द्वीर्या॒निति॑ स॒माव॑त् - वी॒र्या॒न् । \newline
10. ए॒ना॒न् कु॒रु॒ते॒ कु॒रु॒त॒ ए॒ना॒ ने॒ना॒न् कु॒रु॒त॒ ए॒ता ने॒तान् कु॑रुत एना नेनान् कुरुत ए॒तान् । \newline
11. कु॒रु॒त॒ ए॒ता ने॒तान् कु॑रुते कुरुत ए॒तान्. वै वा ए॒तान् कु॑रुते कुरुत ए॒तान्. वै । \newline
12. ए॒तान्. वै वा ए॒ता ने॒तान्. वै ग्रहा॒न् ग्रहा॒न्॒. वा ए॒ता ने॒तान्. वै ग्रहान्॑ । \newline
13. वै ग्रहा॒न् ग्रहा॒न्॒. वै वै ग्रहा᳚न् बं॒बावि॒श्वव॑यसौ बं॒बावि॒श्वव॑यसौ॒ ग्रहा॒न्॒. वै वै ग्रहा᳚न् बं॒बावि॒श्वव॑यसौ । \newline
14. ग्रहा᳚न् बं॒बावि॒श्वव॑यसौ बं॒बावि॒श्वव॑यसौ॒ ग्रहा॒न् ग्रहा᳚न् बं॒बावि॒श्वव॑यसा ववित्ता मवित्ताम् बं॒बावि॒श्वव॑यसौ॒ ग्रहा॒न् ग्रहा᳚न् बं॒बावि॒श्वव॑यसा ववित्ताम् । \newline
15. बं॒बावि॒श्वव॑यसा ववित्ता मवित्ताम् बं॒बावि॒श्वव॑यसौ बं॒बावि॒श्वव॑यसा ववित्ता॒म् ताभ्या॒म् ताभ्या॑ मवित्ताम् बं॒बावि॒श्वव॑यसौ बं॒बावि॒श्वव॑यसा ववित्ता॒म् ताभ्या᳚म् । \newline
16. बं॒बावि॒श्वव॑यसा॒विति॑ बं॒बा - वि॒श्वव॑यसौ । \newline
17. अ॒वि॒त्ता॒म् ताभ्या॒म् ताभ्या॑ मवित्ता मवित्ता॒म् ताभ्या॑ मि॒म इ॒मे ताभ्या॑ मवित्ता मवित्ता॒म् ताभ्या॑ मि॒मे । \newline
18. ताभ्या॑ मि॒म इ॒मे ताभ्या॒म् ताभ्या॑ मि॒मे लो॒का लो॒का इ॒मे ताभ्या॒म् ताभ्या॑ मि॒मे लो॒काः । \newline
19. इ॒मे लो॒का लो॒का इ॒म इ॒मे लो॒काः परा᳚ञ्चः॒ परा᳚ञ्चो लो॒का इ॒म इ॒मे लो॒काः परा᳚ञ्चः । \newline
20. लो॒काः परा᳚ञ्चः॒ परा᳚ञ्चो लो॒का लो॒काः परा᳚ञ्च श्च च॒ परा᳚ञ्चो लो॒का लो॒काः परा᳚ञ्च श्च । \newline
21. परा᳚ञ्च श्च च॒ परा᳚ञ्चः॒ परा᳚ञ्च श्चा॒र्वाञ्चो॒ ऽर्वाञ्च॑ श्च॒ परा᳚ञ्चः॒ परा᳚ञ्च श्चा॒र्वाञ्चः॑ । \newline
22. चा॒र्वाञ्चो॒ ऽर्वाञ्च॑ श्च चा॒र्वाञ्च॑ श्च चा॒र्वाञ्च॑ श्च चा॒र्वाञ्च॑ श्च । \newline
23. अ॒र्वाञ्च॑ श्च चा॒र्वाञ्चो॒ ऽर्वाञ्च॑ श्च॒ प्र प्र चा॒र्वाञ्चो॒ ऽर्वाञ्च॑ श्च॒ प्र । \newline
24. च॒ प्र प्र च॑ च॒ प्राभु॑ रभुः॒ प्र च॑ च॒ प्राभुः॑ । \newline
25. प्राभु॑ रभुः॒ प्र प्राभु॒र् यस्य॒ यस्या॑ भुः॒ प्र प्राभु॒र् यस्य॑ । \newline
26. अ॒भु॒र् यस्य॒ यस्या॑भु रभु॒र् यस्यै॒व मे॒वं ॅयस्या॑भु रभु॒र् यस्यै॒वम् । \newline
27. यस्यै॒व मे॒वं ॅयस्य॒ यस्यै॒वं ॅवि॒दुषो॑ वि॒दुष॑ ए॒वं ॅयस्य॒ यस्यै॒वं ॅवि॒दुषः॑ । \newline
28. ए॒वं ॅवि॒दुषो॑ वि॒दुष॑ ए॒व मे॒वं ॅवि॒दुष॑ ए॒त ए॒ते वि॒दुष॑ ए॒व मे॒वं ॅवि॒दुष॑ ए॒ते । \newline
29. वि॒दुष॑ ए॒त ए॒ते वि॒दुषो॑ वि॒दुष॑ ए॒ते ग्रहा॒ ग्रहा॑ ए॒ते वि॒दुषो॑ वि॒दुष॑ ए॒ते ग्रहाः᳚ । \newline
30. ए॒ते ग्रहा॒ ग्रहा॑ ए॒त ए॒ते ग्रहा॑ गृ॒ह्यन्ते॑ गृ॒ह्यन्ते॒ ग्रहा॑ ए॒त ए॒ते ग्रहा॑ गृ॒ह्यन्ते᳚ । \newline
31. ग्रहा॑ गृ॒ह्यन्ते॑ गृ॒ह्यन्ते॒ ग्रहा॒ ग्रहा॑ गृ॒ह्यन्ते॒ प्र प्र गृ॒ह्यन्ते॒ ग्रहा॒ ग्रहा॑ गृ॒ह्यन्ते॒ प्र । \newline
32. गृ॒ह्यन्ते॒ प्र प्र गृ॒ह्यन्ते॑ गृ॒ह्यन्ते॒ प्रास्मा॑ अस्मै॒ प्र गृ॒ह्यन्ते॑ गृ॒ह्यन्ते॒ प्रास्मै᳚ । \newline
33. प्रास्मा॑ अस्मै॒ प्र प्रास्मा॑ इ॒म इ॒मे᳚ ऽस्मै॒ प्र प्रास्मा॑ इ॒मे । \newline
34. अ॒स्मा॒ इ॒म इ॒मे᳚ ऽस्मा अस्मा इ॒मे लो॒का लो॒का इ॒मे᳚ ऽस्मा अस्मा इ॒मे लो॒काः । \newline
35. इ॒मे लो॒का लो॒का इ॒म इ॒मे लो॒काः परा᳚ञ्चः॒ परा᳚ञ्चो लो॒का इ॒म इ॒मे लो॒काः परा᳚ञ्चः । \newline
36. लो॒काः परा᳚ञ्चः॒ परा᳚ञ्चो लो॒का लो॒काः परा᳚ञ्च श्च च॒ परा᳚ञ्चो लो॒का लो॒काः परा᳚ञ्च श्च । \newline
37. परा᳚ञ्च श्च च॒ परा᳚ञ्चः॒ परा᳚ञ्च श्चा॒र्वाञ्चो॒ ऽर्वाञ्च॑ श्च॒ परा᳚ञ्चः॒ परा᳚ञ्च श्चा॒र्वाञ्चः॑ । \newline
38. चा॒र्वाञ्चो॒ ऽर्वाञ्च॑ श्च चा॒र्वाञ्च॑ श्च चा॒र्वाञ्च॑ श्च चा॒र्वाञ्च॑ श्च । \newline
39. अ॒र्वाञ्च॑ श्च चा॒र्वाञ्चो॒ ऽर्वाञ्च॑ श्च भान्ति भान्ति चा॒र्वाञ्चो॒ ऽर्वाञ्च॑ श्च भान्ति । \newline
40. च॒ भा॒न्ति॒ भा॒न्ति॒ च॒ च॒ भा॒न्ति॒ । \newline
41. भा॒न्तीति॑ भान्ति । \newline
\pagebreak
\markright{ TS 6.6.9.1  \hfill https://www.vedavms.in \hfill}

\section{ TS 6.6.9.1 }

\textbf{TS 6.6.9.1 } \newline
\textbf{Samhita Paata} \newline

दे॒वा वै यद्-य॒ज्ञेऽकु॑र्वत॒ तदसु॑रा अकुर्वत॒ ते दे॒वा अदा᳚भ्ये॒ छन्दाꣳ॑सि॒ सव॑नानि॒ सम॑स्थापय॒न् ततो॑ दे॒वा अभ॑व॒न् पराऽसु॑रा॒ यस्यै॒वं ॅवि॒दुषोऽदा᳚भ्यो गृ॒ह्यते॒ भव॑त्या॒त्मना॒ परा᳚ऽस्य॒ भ्रातृ॑व्यो भवति॒ यद्वै दे॒वा असु॑रा॒-नदा᳚भ्ये॒-नाद॑भ्नुव॒न् तददा᳚भ्यस्या-दाभ्य॒ त्वं ॅय ए॒वं ॅवेद॑ द॒भ्नोत्ये॒व भ्रातृ॑व्यं॒ नैनं॒ भ्रातृ॑व्यो दभ्नोत्ये॒- [  ] \newline

\textbf{Pada Paata} \newline

दे॒वाः । वै । यत् । य॒ज्ञे । अकु॑र्वत । तत् । असु॑राः । अ॒कु॒र्व॒त॒ । ते । दे॒वाः । अदा᳚भ्ये । छन्दाꣳ॑सि । सव॑नानि । समिति॑ । अ॒स्था॒प॒य॒न्न् । ततः॑ । दे॒वाः । अभ॑वन्न् । परेति॑ । असु॑राः । यस्य॑ । ए॒वम् । वि॒दुषः॑ । अदा᳚भ्यः । गृ॒ह्यते᳚ । भव॑ति । आ॒त्मना᳚ । परेति॑ । अ॒स्य॒ । भ्रातृ॑व्यः । भ॒व॒ति॒ । यत् । वै । दे॒वाः । असु॑रान् । अदा᳚भ्येन । अद॑भ्नुवन्न् । तत् । अदा᳚भ्यस्य । अ॒दा॒भ्य॒त्वमित्य॑दाभ्य - त्वम् । यः । ए॒वम् । वेद॑ । द॒भ्नोति॑ । ए॒व । भ्रातृ॑व्यम् । न । ए॒न॒म् । भ्रातृ॑व्यः । द॒भ्नो॒ति॒ ।  \newline


\textbf{Krama Paata} \newline

दे॒वा वै । वै यत् । यद् य॒ज्ञे । य॒ज्ञेऽकु॑र्वत । अकु॑र्वत॒ तत् । तदसु॑राः । असु॑रा अकुर्वत । अ॒कु॒र्व॒त॒ ते । ते दे॒वाः । दे॒वा अदा᳚भ्ये । अदा᳚भ्ये॒ छन्दाꣳ॑सि । छन्दाꣳ॑सि॒ सव॑नानि । सव॑नानि॒ सम् । सम॑स्थापयन्न् । अ॒स्था॒प॒य॒न् ततः॑ । ततो॑ दे॒वाः । दे॒वा अभ॑वन्न् । अभ॑व॒न् परा᳚ । पराऽसु॑राः । असु॑रा॒ यस्य॑ । यस्यै॒वम् । ए॒वम् ॅवि॒दुषः॑ । वि॒दुषोऽदा᳚भ्यः । अदा᳚भ्यो गृ॒ह्यते᳚ । गृ॒ह्यते॒ भव॑ति । भव॑त्या॒त्मना᳚ । आ॒त्मना॒ परा᳚ । परा᳚ऽस्य । अ॒स्य॒ भ्रातृ॑व्यः । भ्रातृ॑व्यो भवति । भ॒व॒ति॒ यत् । यद् वै । वै दे॒वाः । दे॒वा असु॑रान् । असु॑रा॒नदा᳚भ्येन । अदा᳚भ्ये॒नाद॑भ्नुवन्न् । अद॑भ्नुव॒न् तत् । तददा᳚भ्यस्य । अदा᳚भ्यस्यादाभ्य॒त्वम् । अ॒दा॒भ्य॒त्वम् ॅयः । अ॒दा॒भ्य॒त्वमित्य॑दाभ्य - त्वम् । य ए॒वम् । ए॒वम् ॅवेद॑ । वेद॑ द॒भ्नोति॑ । द॒भ्नोत्ये॒व । ए॒व भ्रातृ॑व्यम् । भ्रातृ॑व्य॒म् न । नैन᳚म् । ए॒न॒म् भ्रातृ॑व्यः । भ्रातृ॑व्यो दभ्नोति । द॒भ्नो॒त्ये॒षा \newline

\textbf{Jatai Paata} \newline

1. दे॒वा वै वै दे॒वा दे॒वा वै । \newline
2. वै यद् यद् वै वै यत् । \newline
3. यद् य॒ज्ञे य॒ज्ञे यद् यद् य॒ज्ञे । \newline
4. य॒ज्ञे ऽकु॑र्व॒ता कु॑र्वत य॒ज्ञे य॒ज्ञे ऽकु॑र्वत । \newline
5. अकु॑र्वत॒ तत् तदकु॑र्व॒ता कु॑र्वत॒ तत् । \newline
6. तदसु॑रा॒ असु॑रा॒ स्तत् तदसु॑राः । \newline
7. असु॑रा अकुर्वता कुर्व॒ता सु॑रा॒ असु॑रा अकुर्वत । \newline
8. अ॒कु॒र्व॒त॒ ते ते॑ ऽकुर्वता कुर्वत॒ ते । \newline
9. ते दे॒वा दे॒वा स्ते ते दे॒वाः । \newline
10. दे॒वा अदा॒भ्ये ऽदा᳚भ्ये दे॒वा दे॒वा अदा᳚भ्ये । \newline
11. अदा᳚भ्ये॒ छन्दाꣳ॑सि॒ छन्दाꣳ॒॒ स्यदा॒भ्ये ऽदा᳚भ्ये॒ छन्दाꣳ॑सि । \newline
12. छन्दाꣳ॑सि॒ सव॑नानि॒ सव॑नानि॒ छन्दाꣳ॑सि॒ छन्दाꣳ॑सि॒ सव॑नानि । \newline
13. सव॑नानि॒ सꣳ सꣳ सव॑नानि॒ सव॑नानि॒ सम् । \newline
14. स म॑स्थापयन् नस्थापय॒न् थ्सꣳ स म॑स्थापयन्न् । \newline
15. अ॒स्था॒प॒य॒न् तत॒ स्ततो᳚ ऽस्थापयन् नस्थापय॒न् ततः॑ । \newline
16. ततो॑ दे॒वा दे॒वा स्तत॒ स्ततो॑ दे॒वाः । \newline
17. दे॒वा अभ॑व॒न् नभ॑वन् दे॒वा दे॒वा अभ॑वन्न् । \newline
18. अभ॑व॒न् परा॒ परा ऽभ॑व॒न् नभ॑व॒न् परा᳚ । \newline
19. परा ऽसु॑रा॒ असु॑राः॒ परा॒ परा ऽसु॑राः । \newline
20. असु॑रा॒ यस्य॒ यस्या सु॑रा॒ असु॑रा॒ यस्य॑ । \newline
21. यस्यै॒व मे॒वं ॅयस्य॒ यस्यै॒वम् । \newline
22. ए॒वं ॅवि॒दुषो॑ वि॒दुष॑ ए॒व मे॒वं ॅवि॒दुषः॑ । \newline
23. वि॒दुषो ऽदा॒भ्यो ऽदा᳚भ्यो वि॒दुषो॑ वि॒दुषो ऽदा᳚भ्यः । \newline
24. अदा᳚भ्यो गृ॒ह्यते॑ गृ॒ह्यते ऽदा॒भ्यो ऽदा᳚भ्यो गृ॒ह्यते᳚ । \newline
25. गृ॒ह्यते॒ भव॑ति॒ भव॑ति गृ॒ह्यते॑ गृ॒ह्यते॒ भव॑ति । \newline
26. भव॑ त्या॒त्मना॒ ऽऽत्मना॒ भव॑ति॒ भव॑ त्या॒त्मना᳚ । \newline
27. आ॒त्मना॒ परा॒ परा॒ ऽऽत्मना॒ ऽऽत्मना॒ परा᳚ । \newline
28. परा᳚ ऽस्यास्य॒ परा॒ परा᳚ ऽस्य । \newline
29. अ॒स्य॒ भ्रातृ॑व्यो॒ भ्रातृ॑व्यो ऽस्यास्य॒ भ्रातृ॑व्यः । \newline
30. भ्रातृ॑व्यो भवति भवति॒ भ्रातृ॑व्यो॒ भ्रातृ॑व्यो भवति । \newline
31. भ॒व॒ति॒ यद् यद् भ॑वति भवति॒ यत् । \newline
32. यद् वै वै यद् यद् वै । \newline
33. वै दे॒वा दे॒वा वै वै दे॒वाः । \newline
34. दे॒वा असु॑रा॒ नसु॑रान् दे॒वा दे॒वा असु॑रान् । \newline
35. असु॑रा॒ नदा᳚भ्ये॒ नादा᳚भ्ये॒ना सु॑रा॒ नसु॑रा॒ नदा᳚भ्येन । \newline
36. अदा᳚भ्ये॒ नाद॑भ्नुव॒न् नद॑भ्नुव॒न् नदा᳚भ्ये॒ नादा᳚भ्ये॒ नाद॑भ्नुवन्न् । \newline
37. अद॑भ्नुव॒न् तत् तदद॑भ्नुव॒न् नद॑भ्नुव॒न् तत् । \newline
38. तददा᳚भ्य॒स्या दा᳚भ्यस्य॒ तत् तददा᳚भ्यस्य । \newline
39. अदा᳚भ्यस्या दाभ्य॒त्व म॑दाभ्य॒त्व मदा᳚भ्य॒स्या दा᳚भ्यस्या दाभ्य॒त्वम् । \newline
40. अ॒दा॒भ्य॒त्वं ॅयो यो॑ ऽदाभ्य॒त्व म॑दाभ्य॒त्वं ॅयः । \newline
41. अ॒दा॒भ्य॒त्वमित्य॑दाभ्य - त्वम् । \newline
42. य ए॒व मे॒वं ॅयो य ए॒वम् । \newline
43. ए॒वं ॅवेद॒ वेदै॒व मे॒वं ॅवेद॑ । \newline
44. वेद॑ द॒भ्नोति॑ द॒भ्नोति॒ वेद॒ वेद॑ द॒भ्नोति॑ । \newline
45. द॒भ्नो त्ये॒वैव द॒भ्नोति॑ द॒भ्नो त्ये॒व । \newline
46. ए॒व भ्रातृ॑व्य॒म् भ्रातृ॑व्य मे॒वैव भ्रातृ॑व्यम् । \newline
47. भ्रातृ॑व्य॒न् न न भ्रातृ॑व्य॒म् भ्रातृ॑व्य॒न् न । \newline
48. नैन॑ मेन॒न् न नैन᳚म् । \newline
49. ए॒न॒म् भ्रातृ॑व्यो॒ भ्रातृ॑व्य एन मेन॒म् भ्रातृ॑व्यः । \newline
50. भ्रातृ॑व्यो दभ्नोति दभ्नोति॒ भ्रातृ॑व्यो॒ भ्रातृ॑व्यो दभ्नोति । \newline
51. द॒भ्नो॒ त्ये॒षैषा द॑भ्नोति दभ्नो त्ये॒षा । \newline

\textbf{Ghana Paata } \newline

1. दे॒वा वै वै दे॒वा दे॒वा वै यद् यद् वै दे॒वा दे॒वा वै यत् । \newline
2. वै यद् यद् वै वै यद् य॒ज्ञे य॒ज्ञे यद् वै वै यद् य॒ज्ञे । \newline
3. यद् य॒ज्ञे य॒ज्ञे यद् यद् य॒ज्ञे ऽकु॑र्व॒ता कु॑र्वत य॒ज्ञे यद् यद् य॒ज्ञे ऽकु॑र्वत । \newline
4. य॒ज्ञे ऽकु॑र्व॒ता कु॑र्वत य॒ज्ञे य॒ज्ञे ऽकु॑र्वत॒ तत् तदकु॑र्वत य॒ज्ञे य॒ज्ञे ऽकु॑र्वत॒ तत् । \newline
5. अकु॑र्वत॒ तत् तदकु॑र्व॒ता कु॑र्वत॒ तदसु॑रा॒ असु॑रा॒ स्तदकु॑र्व॒ता कु॑र्वत॒ तदसु॑राः । \newline
6. तदसु॑रा॒ असु॑रा॒ स्तत् तदसु॑रा अकुर्वता कुर्व॒ता सु॑रा॒ स्तत् तदसु॑रा अकुर्वत । \newline
7. असु॑रा अकुर्वता कुर्व॒ता सु॑रा॒ असु॑रा अकुर्वत॒ ते ते॑ ऽकुर्व॒ता सु॑रा॒ असु॑रा अकुर्वत॒ ते । \newline
8. अ॒कु॒र्व॒त॒ ते ते॑ ऽकुर्वता कुर्वत॒ ते दे॒वा दे॒वा स्ते॑ ऽकुर्वता कुर्वत॒ ते दे॒वाः । \newline
9. ते दे॒वा दे॒वा स्ते ते दे॒वा अदा॒भ्ये ऽदा᳚भ्ये दे॒वा स्ते ते दे॒वा अदा᳚भ्ये । \newline
10. दे॒वा अदा॒भ्ये ऽदा᳚भ्ये दे॒वा दे॒वा अदा᳚भ्ये॒ छन्दाꣳ॑सि॒ छन्दाꣳ॒॒ स्यदा᳚भ्ये दे॒वा दे॒वा अदा᳚भ्ये॒ छन्दाꣳ॑सि । \newline
11. अदा᳚भ्ये॒ छन्दाꣳ॑सि॒ छन्दाꣳ॒॒ स्यदा॒भ्ये ऽदा᳚भ्ये॒ छन्दाꣳ॑सि॒ सव॑नानि॒ सव॑नानि॒ छन्दाꣳ॒॒ स्यदा॒भ्ये ऽदा᳚भ्ये॒ छन्दाꣳ॑सि॒ सव॑नानि । \newline
12. छन्दाꣳ॑सि॒ सव॑नानि॒ सव॑नानि॒ छन्दाꣳ॑सि॒ छन्दाꣳ॑सि॒ सव॑नानि॒ सꣳ सꣳ सव॑नानि॒ छन्दाꣳ॑सि॒ छन्दाꣳ॑सि॒ सव॑नानि॒ सम् । \newline
13. सव॑नानि॒ सꣳ सꣳ सव॑नानि॒ सव॑नानि॒ स म॑स्थापयन् नस्थापय॒न् थ्सꣳ सव॑नानि॒ सव॑नानि॒ स म॑स्थापयन्न् । \newline
14. स म॑स्थापयन् नस्थापय॒न् थ्सꣳ स म॑स्थापय॒न् तत॒ स्ततो᳚ ऽस्थापय॒न् थ्सꣳ स म॑स्थापय॒न् ततः॑ । \newline
15. अ॒स्था॒प॒य॒न् तत॒ स्ततो᳚ ऽस्थापयन् नस्थापय॒न् ततो॑ दे॒वा दे॒वा स्ततो᳚ ऽस्थापयन् नस्थापय॒न् ततो॑ दे॒वाः । \newline
16. ततो॑ दे॒वा दे॒वा स्तत॒ स्ततो॑ दे॒वा अभ॑व॒न् नभ॑वन् दे॒वा स्तत॒ स्ततो॑ दे॒वा अभ॑वन्न् । \newline
17. दे॒वा अभ॑व॒न् नभ॑वन् दे॒वा दे॒वा अभ॑व॒न् परा॒ परा ऽभ॑वन् दे॒वा दे॒वा अभ॑व॒न् परा᳚ । \newline
18. अभ॑व॒न् परा॒ परा ऽभ॑व॒न् नभ॑व॒न् परा ऽसु॑रा॒ असु॑राः॒ परा ऽभ॑व॒न् नभ॑व॒न् परा ऽसु॑राः । \newline
19. परा ऽसु॑रा॒ असु॑राः॒ परा॒ परा ऽसु॑रा॒ यस्य॒ यस्या सु॑राः॒ परा॒ परा ऽसु॑रा॒ यस्य॑ । \newline
20. असु॑रा॒ यस्य॒ यस्या सु॑रा॒ असु॑रा॒ यस्यै॒व मे॒वं ॅयस्या सु॑रा॒ असु॑रा॒ यस्यै॒वम् । \newline
21. यस्यै॒व मे॒वं ॅयस्य॒ यस्यै॒वं ॅवि॒दुषो॑ वि॒दुष॑ ए॒वं ॅयस्य॒ यस्यै॒वं ॅवि॒दुषः॑ । \newline
22. ए॒वं ॅवि॒दुषो॑ वि॒दुष॑ ए॒व मे॒वं ॅवि॒दुषो ऽदा॒भ्यो ऽदा᳚भ्यो वि॒दुष॑ ए॒व मे॒वं ॅवि॒दुषो ऽदा᳚भ्यः । \newline
23. वि॒दुषो ऽदा॒भ्यो ऽदा᳚भ्यो वि॒दुषो॑ वि॒दुषो ऽदा᳚भ्यो गृ॒ह्यते॑ गृ॒ह्यते ऽदा᳚भ्यो वि॒दुषो॑ वि॒दुषो ऽदा᳚भ्यो गृ॒ह्यते᳚ । \newline
24. अदा᳚भ्यो गृ॒ह्यते॑ गृ॒ह्यते ऽदा॒भ्यो ऽदा᳚भ्यो गृ॒ह्यते॒ भव॑ति॒ भव॑ति गृ॒ह्यते ऽदा॒भ्यो ऽदा᳚भ्यो गृ॒ह्यते॒ भव॑ति । \newline
25. गृ॒ह्यते॒ भव॑ति॒ भव॑ति गृ॒ह्यते॑ गृ॒ह्यते॒ भव॑ त्या॒त्मना॒ ऽऽत्मना॒ भव॑ति गृ॒ह्यते॑ गृ॒ह्यते॒ भव॑ त्या॒त्मना᳚ । \newline
26. भव॑ त्या॒त्मना॒ ऽऽत्मना॒ भव॑ति॒ भव॑ त्या॒त्मना॒ परा॒ परा॒ ऽऽत्मना॒ भव॑ति॒ भव॑ त्या॒त्मना॒ परा᳚ । \newline
27. आ॒त्मना॒ परा॒ परा॒ ऽऽत्मना॒ ऽऽत्मना॒ परा᳚ ऽस्यास्य॒ परा॒ ऽऽत्मना॒ ऽऽत्मना॒ परा᳚ ऽस्य । \newline
28. परा᳚ ऽस्यास्य॒ परा॒ परा᳚ ऽस्य॒ भ्रातृ॑व्यो॒ भ्रातृ॑व्यो ऽस्य॒ परा॒ परा᳚ ऽस्य॒ भ्रातृ॑व्यः । \newline
29. अ॒स्य॒ भ्रातृ॑व्यो॒ भ्रातृ॑व्यो ऽस्यास्य॒ भ्रातृ॑व्यो भवति भवति॒ भ्रातृ॑व्यो ऽस्यास्य॒ भ्रातृ॑व्यो भवति । \newline
30. भ्रातृ॑व्यो भवति भवति॒ भ्रातृ॑व्यो॒ भ्रातृ॑व्यो भवति॒ यद् यद् भ॑वति॒ भ्रातृ॑व्यो॒ भ्रातृ॑व्यो भवति॒ यत् । \newline
31. भ॒व॒ति॒ यद् यद् भ॑वति भवति॒ यद् वै वै यद् भ॑वति भवति॒ यद् वै । \newline
32. यद् वै वै यद् यद् वै दे॒वा दे॒वा वै यद् यद् वै दे॒वाः । \newline
33. वै दे॒वा दे॒वा वै वै दे॒वा असु॑रा॒ नसु॑रान् दे॒वा वै वै दे॒वा असु॑रान् । \newline
34. दे॒वा असु॑रा॒ नसु॑रान् दे॒वा दे॒वा असु॑रा॒ नदा᳚भ्ये॒ना दा᳚भ्ये॒ना सु॑रान् दे॒वा दे॒वा असु॑रा॒ नदा᳚भ्येन । \newline
35. असु॑रा॒ नदा᳚भ्ये॒ना दा᳚भ्ये॒ना सु॑रा॒ नसु॑रा॒ नदा᳚भ्ये॒ना द॑भ्नुव॒न् नद॑भ्नुव॒न् नदा᳚भ्ये॒ना सु॑रा॒ नसु॑रा॒ नदा᳚भ्ये॒ना द॑भ्नुवन्न् । \newline
36. अदा᳚भ्ये॒ना द॑भ्नुव॒न् नद॑भ्नुव॒न् नदा᳚भ्ये॒ना दा᳚भ्ये॒ना द॑भ्नुव॒न् तत् तदद॑भ्नुव॒न् नदा᳚भ्ये॒ना दा᳚भ्ये॒ना द॑भ्नुव॒न् तत् । \newline
37. अद॑भ्नुव॒न् तत् तदद॑भ्नुव॒न् नद॑भ्नुव॒न् तददा᳚भ्य॒स्या दा᳚भ्यस्य॒ तदद॑भ्नुव॒न् नद॑भ्नुव॒न् 
तददा᳚भ्यस्य । \newline
38. तददा᳚भ्य॒स्या दा᳚भ्यस्य॒ तत् तददा᳚ भ्यस्या दाभ्य॒त्व म॑दाभ्य॒त्व मदा᳚भ्यस्य॒ तत् तददा᳚भ्यस्या दाभ्य॒त्वम् । \newline
39. अदा᳚भ्यस्या दाभ्य॒त्व म॑दाभ्य॒त्व मदा᳚भ्य॒स्या दा᳚भ्यस्या दाभ्य॒त्वं ॅयो यो॑ ऽदाभ्य॒त्व मदा᳚भ्य॒स्या दा᳚भ्यस्या दाभ्य॒त्वं ॅयः । \newline
40. अ॒दा॒भ्य॒त्वं ॅयो यो॑ ऽदाभ्य॒त्व म॑दाभ्य॒त्वं ॅय ए॒व मे॒वं ॅयो॑ ऽदाभ्य॒त्व म॑दाभ्य॒त्वं ॅय ए॒वम् । \newline
41. अ॒दा॒भ्य॒त्वमित्य॑दाभ्य - त्वम् । \newline
42. य ए॒व मे॒वं ॅयो य ए॒वं ॅवेद॒ वेदै॒वं ॅयो य ए॒वं ॅवेद॑ । \newline
43. ए॒वं ॅवेद॒ वेदै॒व मे॒वं ॅवेद॑ द॒भ्नोति॑ द॒भ्नोति॒ वेदै॒व मे॒वं ॅवेद॑ द॒भ्नोति॑ । \newline
44. वेद॑ द॒भ्नोति॑ द॒भ्नोति॒ वेद॒ वेद॑ द॒भ्नो त्ये॒वैव द॒भ्नोति॒ वेद॒ वेद॑ द॒भ्नो त्ये॒व । \newline
45. द॒भ्नो त्ये॒वैव द॒भ्नोति॑ द॒भ्नो त्ये॒व भ्रातृ॑व्य॒म् भ्रातृ॑व्य मे॒व द॒भ्नोति॑ द॒भ्नो त्ये॒व भ्रातृ॑व्यम् । \newline
46. ए॒व भ्रातृ॑व्य॒म् भ्रातृ॑व्य मे॒वैव भ्रातृ॑व्य॒न् न न भ्रातृ॑व्य मे॒वैव भ्रातृ॑व्य॒न् न । \newline
47. भ्रातृ॑व्य॒न् न न भ्रातृ॑व्य॒म् भ्रातृ॑व्य॒न् नैन॑ मेन॒न् न भ्रातृ॑व्य॒म् भ्रातृ॑व्य॒न् नैन᳚म् । \newline
48. नैन॑ मेन॒न् न नैन॒म् भ्रातृ॑व्यो॒ भ्रातृ॑व्य एन॒न् न नैन॒म् भ्रातृ॑व्यः । \newline
49. ए॒न॒म् भ्रातृ॑व्यो॒ भ्रातृ॑व्य एन मेन॒म् भ्रातृ॑व्यो दभ्नोति दभ्नोति॒ भ्रातृ॑व्य एन मेन॒म् भ्रातृ॑व्यो दभ्नोति । \newline
50. भ्रातृ॑व्यो दभ्नोति दभ्नोति॒ भ्रातृ॑व्यो॒ भ्रातृ॑व्यो दभ्नो त्ये॒षैषा द॑भ्नोति॒ भ्रातृ॑व्यो॒ भ्रातृ॑व्यो दभ्नो त्ये॒षा । \newline
51. द॒भ्नो॒ त्ये॒षैषा द॑भ्नोति दभ्नो त्ये॒षा वै वा ए॒षा द॑भ्नोति दभ्नो त्ये॒षा वै । \newline
\pagebreak
\markright{ TS 6.6.9.2  \hfill https://www.vedavms.in \hfill}

\section{ TS 6.6.9.2 }

\textbf{TS 6.6.9.2 } \newline
\textbf{Samhita Paata} \newline

-षा वै प्र॒जाप॑ते-रतिमो॒क्षिणी॒ नाम॑ त॒नूर्यददा᳚भ्य॒ उप॑नद्धस्य गृह्णा॒त्यति॑मुक्त्या॒ अति॑ पा॒प्मानं॒ भ्रातृ॑व्यं मुच्यते॒ य ए॒वं ॅवेद॒ घ्नन्ति॒ वा ए॒तथ् सोमं॒ ॅयद॑भिषु॒ण्वन्ति॒ सोमे॑ ह॒न्यमा॑ने य॒ज्ञो ह॑न्यते य॒ज्ञे यज॑मानो ब्रह्मवा॒दिनो॑ वदन्ति॒ किं तद्-य॒ज्ञे यज॑मानः कुरुते॒ येन॒ जीवन्᳚थ् सुव॒र्गं ॅलो॒कमेतीति॑ जीवग्र॒हो वा ए॒ष यददा॒भ्यो ऽन॑भिषुतस्य गृह्णाति॒ ( ) जीव॑न्तमे॒वैनꣳ॑ सुव॒र्गं ॅलो॒कं ग॑मयति॒ वि वा ए॒तद्-य॒ज्ञ्ं छि॑न्दन्ति॒ यददा᳚भ्ये सꣳ-स्था॒पय॑-न्त्यꣳ॒॒शूनपि॑ सृजति य॒ज्ञ्स्य॒ सन्त॑त्यै ॥ \newline

\textbf{Pada Paata} \newline

ए॒षा । वै । प्र॒जाप॑ते॒रिति॑ प्र॒जा-प॒तेः॒ । अ॒ति॒मो॒क्षिणीत्य॑ति-मो॒क्षिणी᳚ । नाम॑ । त॒नूः । यत् । अदा᳚भ्यः । उप॑नद्ध॒स्येत्युप॑-न॒द्ध॒स्य॒ । गृ॒ह्णा॒ति॒ । अति॑मुक्त्या॒ इत्यति॑ - मु॒क्त्यै॒ । अतीति॑ । पा॒प्मान᳚म् । भ्रातृ॑व्यम् । मु॒च्य॒ते॒ । यः । ए॒वम् । वेद॑ । घ्नन्ति॑ । वै । ए॒तत् । सोम᳚म् । यत् । अ॒भि॒षु॒ण्वन्तीत्य॑भि - सु॒न्वन्ति॑ । सोमे᳚ । ह॒न्यमा॑ने । य॒ज्ञ्ः । ह॒न्य॒ते॒ । य॒ज्ञे । यज॑मानः । ब्र॒ह्म॒वा॒दिन॒ इति॑ ब्रह्म-वा॒दिनः॑ । व॒द॒न्ति॒ । किम् । तत् । य॒ज्ञे । यज॑मानः । कु॒रु॒ते॒ । येन॑ । जीवन्न्॑ । सु॒व॒र्गमिति॑ सुवः - गम् । लो॒कम् । एति॑ । इति॑ । जी॒व॒ग्र॒ह इति॑ जीव - ग्र॒हः । वै । ए॒षः । यत् । अदा᳚भ्यः । अन॑भिषुत॒स्येत्यन॑भि - सु॒त॒स्य॒ । गृ॒ह्णा॒ति॒ ( ) । जीव॑न्तम् । ए॒व । ए॒न॒म् । सु॒व॒र्गमिति॑ सुवः - गम् । लो॒कम् । ग॒म॒य॒ति॒ । वीति॑ । वै । ए॒तत् । य॒ज्ञ्म् । छि॒न्द॒न्ति॒ । यत् । अदा᳚भ्ये । सꣳ॒॒स्था॒पय॒न्तीति॑ सं - स्था॒पय॑न्ति । अꣳ॒॒शून् । अपीति॑ । स॒ज॒ति॒ । य॒ज्ञ्स्य॑ । सन्त॑त्या॒ इति॒ सं - त॒त्यै॒ ॥  \newline


\textbf{Krama Paata} \newline

ए॒षा वै । वै प्र॒जाप॑तेः । प्र॒जाप॑तेरतिमो॒क्षिणी᳚ । प्र॒जाप॑ते॒रिति॑ प्र॒जा - प॒तेः॒ । अ॒ति॒मो॒क्षिणी॒ नाम॑ । अ॒ति॒मो॒क्षिणीत्य॑ति - मो॒क्षिणी᳚ । नाम॑ त॒नूः । त॒नूर् यत् । यददा᳚भ्यः । अदा᳚भ्य॒ उप॑नद्धस्य । उप॑नद्धस्य गृह्णाति । उप॑नद्ध॒स्येत्युप॑ - न॒द्ध॒स्य॒ । गृ॒ह्णा॒त्यति॑मुक्त्यै । अति॑मुक्त्या॒ अति॑ । अति॑मुक्त्या॒ इत्यति॑ - मु॒क्त्यै॒ । अति॑ पा॒प्मान᳚म् । पा॒प्मान॒म् भ्रातृ॑व्यम् । भ्रातृ॑व्यम् मुच्यते । मु॒च्य॒ते॒ यः । य ए॒वम् । ए॒वम् ॅवेद॑ । वेद॒ घ्नन्ति॑ । घ्नन्ति॒ वै । वा ए॒तत् । ए॒तथ् सोम᳚म् । सोम॒म् ॅयत् । यद॑भिषु॒ण्वन्ति॑ । अ॒भि॒षु॒ण्वन्ति॒ सोमे᳚ । अ॒भि॒षु॒ण्वन्तीत्य॑भि - सु॒न्वन्ति॑ । सोमे॑ ह॒न्यमा॑ने । ह॒न्यमा॑ने य॒ज्ञ्ः । य॒ज्ञो ह॑न्यते । ह॒न्य॒ते॒ य॒ज्ञे । य॒ज्ञे यज॑मानः । यज॑मानो ब्रह्मवा॒दिनः॑ । ब्र॒ह्म॒वा॒दिनो॑ वदन्ति । ब्र॒ह्म॒वा॒दिन॒ इति॑ ब्रह्म - वा॒दिनः॑ । व॒द॒न्ति॒ किम् । किम् तत् । तद् य॒ज्ञे । य॒ज्ञे यज॑मानः । यज॑मानः कुरुते । कु॒रु॒ते॒ येन॑ । येन॒ जीवन्न्॑ । जीव᳚न्थ् सुव॒र्गम् । सु॒व॒र्गम् ॅलो॒कम् । सु॒व॒र्गमिति॑ सुवः - गम् । लो॒कमे॑ति । एतीति॑ । इति॑ जीवग्र॒हः । जी॒व॒ग्र॒हो वै । जी॒व॒ग्र॒ह इति॑ जीव - ग्र॒हः । वा ए॒षः । ए॒ष यत् । यददा᳚भ्यः । अदा॒भ्योऽन॑भिषुतस्य । अन॑भिषुतस्य गृह्णाति ( ) । अन॑भिषुत॒स्येत्यन॑भि - सु॒त॒स्य॒ । गृ॒ह्णा॒ति॒ जीव॑न्तम् । जीव॑न्तमे॒व । ए॒वैन᳚म् । ए॒नꣳ॒॒ सु॒व॒र्गम् । सु॒व॒र्गम् ॅलो॒कम् । सु॒व॒र्गमिति॑ सुवः - गम् । लो॒कम् ग॑मयति । ग॒म॒य॒ति॒ वि । वि वै । वा ए॒तत् । ए॒तद् य॒ज्ञ्म् । य॒ज्ञ्म् छि॑न्दन्ति । छि॒न्द॒न्ति॒ यत् । यददा᳚भ्ये । अदा᳚भ्ये सꣳस्था॒पय॑न्ति । सꣳ॒॒स्था॒पय॑न्त्यꣳ॒॒शून् । सꣳ॒॒स्था॒पय॒न्तीति॑ सम् - स्था॒पय॑न्ति । अꣳ॒॒शूनपि॑ । अपि॑ सृजति । सृ॒ज॒ति॒ य॒ज्ञ्स्य॑ । य॒ज्ञ्स्य॒ सन्त॑त्यै । सन्त॑त्या॒ इति॒ सम् - त॒त्यै॒ । \newline

\textbf{Jatai Paata} \newline

1. ए॒षा वै वा ए॒षैषा वै । \newline
2. वै प्र॒जाप॑तेः प्र॒जाप॑ते॒र् वै वै प्र॒जाप॑तेः । \newline
3. प्र॒जाप॑ते रतिमो॒क्षिण्य॑ तिमो॒क्षिणी᳚ प्र॒जाप॑तेः प्र॒जाप॑ते रतिमो॒क्षिणी᳚ । \newline
4. प्र॒जाप॑ते॒रिति॑ प्र॒जा - प॒तेः॒ । \newline
5. अ॒ति॒मो॒क्षिणी॒ नाम॒ नामा॑तिमो॒क्षिण्य॑ तिमो॒क्षिणी॒ नाम॑ । \newline
6. अ॒ति॒मो॒क्षिणीत्य॑ति - मो॒क्षिणी᳚ । \newline
7. नाम॑ त॒नू स्त॒नूर् नाम॒ नाम॑ त॒नूः । \newline
8. त॒नूर् यद् यत् त॒नू स्त॒नूर् यत् । \newline
9. यददा॒भ्यो ऽदा᳚भ्यो॒ यद् यददा᳚भ्यः । \newline
10. अदा᳚भ्य॒ उप॑नद्ध॒स्यो प॑नद्ध॒ स्यादा॒भ्यो ऽदा᳚भ्य॒ उप॑नद्धस्य । \newline
11. उप॑नद्धस्य गृह्णाति गृह्णा॒ त्युप॑नद्ध॒ स्योप॑नद्धस्य गृह्णाति । \newline
12. उप॑नद्ध॒स्येत्युप॑ - न॒द्ध॒स्य॒ । \newline
13. गृ॒ह्णा॒ त्यति॑मुक्त्या॒ अति॑मुक्त्यै गृह्णाति गृह्णा॒ त्यति॑मुक्त्यै । \newline
14. अति॑मुक्त्या॒ अत्यत्य ति॑मुक्त्या॒ अति॑मुक्त्या॒ अति॑ । \newline
15. अति॑मुक्त्या॒ इत्यति॑ - मु॒क्त्यै॒ । \newline
16. अति॑ पा॒प्मान॑म् पा॒प्मान॒ मत्यति॑ पा॒प्मान᳚म् । \newline
17. पा॒प्मान॒म् भ्रातृ॑व्य॒म् भ्रातृ॑व्यम् पा॒प्मान॑म् पा॒प्मान॒म् भ्रातृ॑व्यम् । \newline
18. भ्रातृ॑व्यम् मुच्यते मुच्यते॒ भ्रातृ॑व्य॒म् भ्रातृ॑व्यम् मुच्यते । \newline
19. मु॒च्य॒ते॒ यो यो मु॑च्यते मुच्यते॒ यः । \newline
20. य ए॒व मे॒वं ॅयो य ए॒वम् । \newline
21. ए॒वं ॅवेद॒ वेदै॒व मे॒वं ॅवेद॑ । \newline
22. वेद॒ घ्नन्ति॒ घ्नन्ति॒ वेद॒ वेद॒ घ्नन्ति॑ । \newline
23. घ्नन्ति॒ वै वै घ्नन्ति॒ घ्नन्ति॒ वै । \newline
24. वा ए॒त दे॒तद् वै वा ए॒तत् । \newline
25. ए॒तथ् सोमꣳ॒॒ सोम॑ मे॒त दे॒तथ् सोम᳚म् । \newline
26. सोमं॒ ॅयद् यथ् सोमꣳ॒॒ सोमं॒ ॅयत् । \newline
27. यद॑भिषु॒ण्वन् त्य॑भिषु॒ण्वन्ति॒ यद् यद॑भिषु॒ण्वन्ति॑ । \newline
28. अ॒भि॒षु॒ण्वन्ति॒ सोमे॒ सोमे॑ ऽभिषु॒ण्व न्त्य॑भिषु॒ण्वन्ति॒ सोमे᳚ । \newline
29. अ॒भि॒षु॒ण्वन्तीत्य॑भि - सु॒न्वन्ति॑ । \newline
30. सोमे॑ ह॒न्यमा॑ने ह॒न्यमा॑ने॒ सोमे॒ सोमे॑ ह॒न्यमा॑ने । \newline
31. ह॒न्यमा॑ने य॒ज्ञो य॒ज्ञो ह॒न्यमा॑ने ह॒न्यमा॑ने य॒ज्ञ्ः । \newline
32. य॒ज्ञो ह॑न्यते हन्यते य॒ज्ञो य॒ज्ञो ह॑न्यते । \newline
33. ह॒न्य॒ते॒ य॒ज्ञे य॒ज्ञे ह॑न्यते हन्यते य॒ज्ञे । \newline
34. य॒ज्ञे यज॑मानो॒ यज॑मानो य॒ज्ञे य॒ज्ञे यज॑मानः । \newline
35. यज॑मानो ब्रह्मवा॒दिनो᳚ ब्रह्मवा॒दिनो॒ यज॑मानो॒ यज॑मानो ब्रह्मवा॒दिनः॑ । \newline
36. ब्र॒ह्म॒वा॒दिनो॑ वदन्ति वदन्ति ब्रह्मवा॒दिनो᳚ ब्रह्मवा॒दिनो॑ वदन्ति । \newline
37. ब्र॒ह्म॒वा॒दिन॒ इति॑ ब्रह्म - वा॒दिनः॑ । \newline
38. व॒द॒न्ति॒ किम् किं ॅव॑दन्ति वदन्ति॒ किम् । \newline
39. किम् तत् तत् किम् किम् तत् । \newline
40. तद् य॒ज्ञे य॒ज्ञे तत् तद् य॒ज्ञे । \newline
41. य॒ज्ञे यज॑मानो॒ यज॑मानो य॒ज्ञे य॒ज्ञे यज॑मानः । \newline
42. यज॑मानः कुरुते कुरुते॒ यज॑मानो॒ यज॑मानः कुरुते । \newline
43. कु॒रु॒ते॒ येन॒ येन॑ कुरुते कुरुते॒ येन॑ । \newline
44. येन॒ जीव॒न् जीव॒न्॒. येन॒ येन॒ जीवन्न्॑ । \newline
45. जीवन्᳚ थ्सुव॒र्गꣳ सु॑व॒र्गम् जीव॒न् जीवन्᳚ थ्सुव॒र्गम् । \newline
46. सु॒व॒र्गम् ॅलो॒कम् ॅलो॒कꣳ सु॑व॒र्गꣳ सु॑व॒र्गम् ॅलो॒कम् । \newline
47. सु॒व॒र्गमिति॑ सुवः - गम् । \newline
48. लो॒क मेत्येति॑ लो॒कम् ॅलो॒क मेति॑ । \newline
49. एतीती त्ये त्येतीति॑ । \newline
50. इति॑ जीवग्र॒हो जी॑वग्र॒ह इतीति॑ जीवग्र॒हः । \newline
51. जी॒व॒ग्र॒हो वै वै जी॑वग्र॒हो जी॑वग्र॒हो वै । \newline
52. जी॒व॒ग्र॒ह इति॑ जीव - ग्र॒हः । \newline
53. वा ए॒ष ए॒ष वै वा ए॒षः । \newline
54. ए॒ष यद् यदे॒ष ए॒ष यत् । \newline
55. यददा॒भ्यो ऽदा᳚भ्यो॒ यद् यददा᳚भ्यः । \newline
56. अदा॒भ्यो ऽन॑भिषुत॒स्या न॑भिषुत॒स्या दा॒भ्यो ऽदा॒भ्यो ऽन॑भिषुतस्य । \newline
57. अन॑भिषुतस्य गृह्णाति गृह्णा॒ त्यन॑भिषुत॒स्या न॑भिषुतस्य गृह्णाति । \newline
58. अन॑भिषुत॒स्येत्यन॑भि - सु॒त॒स्य॒ । \newline
59. गृ॒ह्णा॒ति॒ जीव॑न्त॒म् जीव॑न्तम् गृह्णाति गृह्णाति॒ जीव॑न्तम् । \newline
60. जीव॑न्त मे॒वैव जीव॑न्त॒म् जीव॑न्त मे॒व । \newline
61. ए॒वैन॑ मेन मे॒वै वैन᳚म् । \newline
62. ए॒नꣳ॒॒ सु॒व॒र्गꣳ सु॑व॒र्ग मे॑न मेनꣳ सुव॒र्गम् । \newline
63. सु॒व॒र्गम् ॅलो॒कम् ॅलो॒कꣳ सु॑व॒र्गꣳ सु॑व॒र्गम् ॅलो॒कम् । \newline
64. सु॒व॒र्गमिति॑ सुवः - गम् । \newline
65. लो॒कम् ग॑मयति गमयति लो॒कम् ॅलो॒कम् ग॑मयति । \newline
66. ग॒म॒य॒ति॒ वि वि ग॑मयति गमयति॒ वि । \newline
67. वि वै वै वि वि वै । \newline
68. वा ए॒त दे॒तद् वै वा ए॒तत् । \newline
69. ए॒तद् य॒ज्ञ्ं ॅय॒ज्ञ् मे॒त दे॒तद् य॒ज्ञ्म् । \newline
70. य॒ज्ञ्म् छि॑न्दन्ति छिन्दन्ति य॒ज्ञ्ं ॅय॒ज्ञ्म् छि॑न्दन्ति । \newline
71. छि॒न्द॒न्ति॒ यद् यच् छि॑न्दन्ति छिन्दन्ति॒ यत् । \newline
72. यददा॒भ्ये ऽदा᳚भ्ये॒ यद् यददा᳚भ्ये । \newline
73. अदा᳚भ्ये सꣳस्था॒पय॑न्ति सꣳस्था॒पय॒ न्त्यदा॒भ्ये ऽदा᳚भ्ये सꣳस्था॒पय॑न्ति । \newline
74. सꣳ॒॒स्था॒पय॑न् त्यꣳ॒॒शू नꣳ॒॒शून् थ्सꣳ॑स्था॒पय॑न्ति सꣳस्था॒पय॑न् त्यꣳ॒॒शून् । \newline
75. सꣳ॒॒स्था॒पय॒न्तीति॑ सं - स्था॒पय॑न्ति । \newline
76. अꣳ॒॒शू नप्य प्यꣳ॒॒शू नꣳ॒॒शू नपि॑ । \newline
77. अपि॑ सृजति सृज॒ त्यप्यपि॑ सृजति । \newline
78. सृ॒ज॒ति॒ य॒ज्ञ्स्य॑ य॒ज्ञ्स्य॑ सृजति सृजति य॒ज्ञ्स्य॑ । \newline
79. य॒ज्ञ्स्य॒ सन्त॑त्यै॒ सन्त॑त्यै य॒ज्ञ्स्य॑ य॒ज्ञ्स्य॒ सन्त॑त्यै । \newline
80. सन्त॑त्या॒ इति॒ सं - त॒त्यै॒ । \newline

\textbf{Ghana Paata } \newline

1. ए॒षा वै वा ए॒षैषा वै प्र॒जाप॑तेः प्र॒जाप॑ते॒र् वा ए॒षैषा वै प्र॒जाप॑तेः । \newline
2. वै प्र॒जाप॑तेः प्र॒जाप॑ते॒र् वै वै प्र॒जाप॑ते रतिमो॒क्षि ण्य॑तिमो॒क्षिणी᳚ प्र॒जाप॑ते॒र् वै वै प्र॒जाप॑ते रतिमो॒क्षिणी᳚ । \newline
3. प्र॒जाप॑ते रतिमो॒क्षि ण्य॑तिमो॒क्षिणी᳚ प्र॒जाप॑तेः प्र॒जाप॑ते रतिमो॒क्षिणी॒ नाम॒ नामा॑ तिमो॒क्षिणी᳚ प्र॒जाप॑तेः प्र॒जाप॑ते रतिमो॒क्षिणी॒ नाम॑ । \newline
4. प्र॒जाप॑ते॒रिति॑ प्र॒जा - प॒तेः॒ । \newline
5. अ॒ति॒मो॒क्षिणी॒ नाम॒ नामा॑ तिमो॒क्षि ण्य॑तिमो॒क्षिणी॒ नाम॑ त॒नू स्त॒नूर् नामा॑ तिमो॒क्षि ण्य॑तिमो॒क्षिणी॒ नाम॑ त॒नूः । \newline
6. अ॒ति॒मो॒क्षिणीत्य॑ति - मो॒क्षिणी᳚ । \newline
7. नाम॑ त॒नू स्त॒नूर् नाम॒ नाम॑ त॒नूर् यद् यत् त॒नूर् नाम॒ नाम॑ त॒नूर् यत् । \newline
8. त॒नूर् यद् यत् त॒नू स्त॒नूर् यददा॒भ्यो ऽदा᳚भ्यो॒ यत् त॒नू स्त॒नूर् यददा᳚भ्यः । \newline
9. यददा॒भ्यो ऽदा᳚भ्यो॒ यद् यददा᳚भ्य॒ उप॑नद्ध॒ स्योप॑नद्ध॒स्या दा᳚भ्यो॒ यद् यददा᳚भ्य॒ उप॑नद्धस्य । \newline
10. अदा᳚भ्य॒ उप॑नद्ध॒ स्योप॑नद्ध॒स्या दा॒भ्यो ऽदा᳚भ्य॒ उप॑नद्धस्य गृह्णाति गृह्णा॒ त्युप॑नद्ध॒स्या दा॒भ्यो ऽदा᳚भ्य॒ उप॑नद्धस्य गृह्णाति । \newline
11. उप॑नद्धस्य गृह्णाति गृह्णा॒ त्युप॑नद्ध॒ स्योप॑नद्धस्य गृह्णा॒ त्यति॑मुक्त्या॒ अति॑मुक्त्यै गृह्णा॒ त्युप॑नद्ध॒ स्योप॑नद्धस्य गृह्णा॒ त्यति॑मुक्त्यै । \newline
12. उप॑नद्ध॒स्येत्युप॑ - न॒द्ध॒स्य॒ । \newline
13. गृ॒ह्णा॒ त्यति॑मुक्त्या॒ अति॑मुक्त्यै गृह्णाति गृह्णा॒ त्यति॑मुक्त्या॒ अत्य त्यति॑मुक्त्यै गृह्णाति गृह्णा॒ त्यति॑मुक्त्या॒ अति॑ । \newline
14. अति॑मुक्त्या॒ अत्य त्यति॑मुक्त्या॒ अति॑मुक्त्या॒ अति॑ पा॒प्मान॑म् पा॒प्मान॒ मत्यति॑ मुक्त्या॒ अति॑मुक्त्या॒ अति॑ पा॒प्मान᳚म् । \newline
15. अति॑मुक्त्या॒ इत्यति॑ - मु॒क्त्यै॒ । \newline
16. अति॑ पा॒प्मान॑म् पा॒प्मान॒ मत्यति॑ पा॒प्मान॒म् भ्रातृ॑व्य॒म् भ्रातृ॑व्यम् पा॒प्मान॒ मत्यति॑ पा॒प्मान॒म् भ्रातृ॑व्यम् । \newline
17. पा॒प्मान॒म् भ्रातृ॑व्य॒म् भ्रातृ॑व्यम् पा॒प्मान॑म् पा॒प्मान॒म् भ्रातृ॑व्यम् मुच्यते मुच्यते॒ भ्रातृ॑व्यम् पा॒प्मान॑म् पा॒प्मान॒म् भ्रातृ॑व्यम् मुच्यते । \newline
18. भ्रातृ॑व्यम् मुच्यते मुच्यते॒ भ्रातृ॑व्य॒म् भ्रातृ॑व्यम् मुच्यते॒ यो यो मु॑च्यते॒ भ्रातृ॑व्य॒म् भ्रातृ॑व्यम् मुच्यते॒ यः । \newline
19. मु॒च्य॒ते॒ यो यो मु॑च्यते मुच्यते॒ य ए॒व मे॒वं ॅयो मु॑च्यते मुच्यते॒ य ए॒वम् । \newline
20. य ए॒व मे॒वं ॅयो य ए॒वं ॅवेद॒ वेदै॒वं ॅयो य ए॒वं ॅवेद॑ । \newline
21. ए॒वं ॅवेद॒ वेदै॒व मे॒वं ॅवेद॒ घ्नन्ति॒ घ्नन्ति॒ वेदै॒व मे॒वं ॅवेद॒ घ्नन्ति॑ । \newline
22. वेद॒ घ्नन्ति॒ घ्नन्ति॒ वेद॒ वेद॒ घ्नन्ति॒ वै वै घ्नन्ति॒ वेद॒ वेद॒ घ्नन्ति॒ वै । \newline
23. घ्नन्ति॒ वै वै घ्नन्ति॒ घ्नन्ति॒ वा ए॒त दे॒तद् वै घ्नन्ति॒ घ्नन्ति॒ वा ए॒तत् । \newline
24. वा ए॒त दे॒तद् वै वा ए॒तथ् सोमꣳ॒॒ सोम॑ मे॒तद् वै वा ए॒तथ् सोम᳚म् । \newline
25. ए॒तथ् सोमꣳ॒॒ सोम॑ मे॒त दे॒तथ् सोमं॒ ॅयद् यथ् सोम॑ मे॒त दे॒तथ् सोमं॒ ॅयत् । \newline
26. सोमं॒ ॅयद् यथ् सोमꣳ॒॒ सोमं॒ ॅयद॑भिषु॒ण्व न्त्य॑भिषु॒ण्वन्ति॒ यथ् सोमꣳ॒॒ सोमं॒ ॅयद॑भिषु॒ण्वन्ति॑ । \newline
27. यद॑भिषु॒ण्व न्त्य॑भिषु॒ण्वन्ति॒ यद् यद॑भिषु॒ण्वन्ति॒ सोमे॒ सोमे॑ ऽभिषु॒ण्वन्ति॒ यद् यद॑भिषु॒ण्वन्ति॒ सोमे᳚ । \newline
28. अ॒भि॒षु॒ण्वन्ति॒ सोमे॒ सोमे॑ ऽभिषु॒ण्व न्त्य॑भिषु॒ण्वन्ति॒ सोमे॑ ह॒न्यमा॑ने ह॒न्यमा॑ने॒ सोमे॑ ऽभिषु॒ण्व न्त्य॑भिषु॒ण्वन्ति॒ सोमे॑ ह॒न्यमा॑ने । \newline
29. अ॒भि॒षु॒ण्वन्तीत्य॑भि - सु॒न्वन्ति॑ । \newline
30. सोमे॑ ह॒न्यमा॑ने ह॒न्यमा॑ने॒ सोमे॒ सोमे॑ ह॒न्यमा॑ने य॒ज्ञो य॒ज्ञो ह॒न्यमा॑ने॒ सोमे॒ सोमे॑ ह॒न्यमा॑ने य॒ज्ञ्ः । \newline
31. ह॒न्यमा॑ने य॒ज्ञो य॒ज्ञो ह॒न्यमा॑ने ह॒न्यमा॑ने य॒ज्ञो ह॑न्यते हन्यते य॒ज्ञो ह॒न्यमा॑ने ह॒न्यमा॑ने य॒ज्ञो ह॑न्यते । \newline
32. य॒ज्ञो ह॑न्यते हन्यते य॒ज्ञो य॒ज्ञो ह॑न्यते य॒ज्ञे य॒ज्ञे ह॑न्यते य॒ज्ञो य॒ज्ञो ह॑न्यते य॒ज्ञे । \newline
33. ह॒न्य॒ते॒ य॒ज्ञे य॒ज्ञे ह॑न्यते हन्यते य॒ज्ञे यज॑मानो॒ यज॑मानो य॒ज्ञे ह॑न्यते हन्यते य॒ज्ञे यज॑मानः । \newline
34. य॒ज्ञे यज॑मानो॒ यज॑मानो य॒ज्ञे य॒ज्ञे यज॑मानो ब्रह्मवा॒दिनो᳚ ब्रह्मवा॒दिनो॒ यज॑मानो य॒ज्ञे य॒ज्ञे यज॑मानो ब्रह्मवा॒दिनः॑ । \newline
35. यज॑मानो ब्रह्मवा॒दिनो᳚ ब्रह्मवा॒दिनो॒ यज॑मानो॒ यज॑मानो ब्रह्मवा॒दिनो॑ वदन्ति वदन्ति ब्रह्मवा॒दिनो॒ यज॑मानो॒ यज॑मानो ब्रह्मवा॒दिनो॑ वदन्ति । \newline
36. ब्र॒ह्म॒वा॒दिनो॑ वदन्ति वदन्ति ब्रह्मवा॒दिनो᳚ ब्रह्मवा॒दिनो॑ वदन्ति॒ किम् किं ॅव॑दन्ति ब्रह्मवा॒दिनो᳚ ब्रह्मवा॒दिनो॑ वदन्ति॒ किम् । \newline
37. ब्र॒ह्म॒वा॒दिन॒ इति॑ ब्रह्म - वा॒दिनः॑ । \newline
38. व॒द॒न्ति॒ किम् किं ॅव॑दन्ति वदन्ति॒ किम् तत् तत् किं ॅव॑दन्ति वदन्ति॒ किम् तत् । \newline
39. किम् तत् तत् किम् किम् तद् य॒ज्ञे य॒ज्ञे तत् किम् किम् तद् य॒ज्ञे । \newline
40. तद् य॒ज्ञे य॒ज्ञे तत् तद् य॒ज्ञे यज॑मानो॒ यज॑मानो य॒ज्ञे तत् तद् य॒ज्ञे यज॑मानः । \newline
41. य॒ज्ञे यज॑मानो॒ यज॑मानो य॒ज्ञे य॒ज्ञे यज॑मानः कुरुते कुरुते॒ यज॑मानो य॒ज्ञे य॒ज्ञे यज॑मानः कुरुते । \newline
42. यज॑मानः कुरुते कुरुते॒ यज॑मानो॒ यज॑मानः कुरुते॒ येन॒ येन॑ कुरुते॒ यज॑मानो॒ यज॑मानः कुरुते॒ येन॑ । \newline
43. कु॒रु॒ते॒ येन॒ येन॑ कुरुते कुरुते॒ येन॒ जीव॒न् जीव॒न्॒. येन॑ कुरुते कुरुते॒ येन॒ जीवन्न्॑ । \newline
44. येन॒ जीव॒न् जीव॒न्॒. येन॒ येन॒ जीवन्᳚ थ्सुव॒र्गꣳ सु॑व॒र्गम् जीव॒न्॒. येन॒ येन॒ जीवन्᳚ थ्सुव॒र्गम् । \newline
45. जीवन्᳚ थ्सुव॒र्गꣳ सु॑व॒र्गम् जीव॒न् जीवन्᳚ थ्सुव॒र्गम् ॅलो॒कम् ॅलो॒कꣳ सु॑व॒र्गम् जीव॒न् जीवन्᳚ थ्सुव॒र्गम् ॅलो॒कम् । \newline
46. सु॒व॒र्गम् ॅलो॒कम् ॅलो॒कꣳ सु॑व॒र्गꣳ सु॑व॒र्गम् ॅलो॒कम् एत्येति॑ लो॒कꣳ सु॑व॒र्गꣳ 
सु॑व॒र्गम् ॅलो॒क मेति॑ । \newline
47. सु॒व॒र्गमिति॑ सुवः - गम् । \newline
48. लो॒क मेत्येति॑ लो॒कम् ॅलो॒क मेतीती त्येति॑ लो॒कम् ॅलो॒क मेतीति॑ । \newline
49. एतीतीत्ये त्येतीति॑ जीवग्र॒हो जी॑वग्र॒ह इत्ये त्येतीति॑ जीवग्र॒हः । \newline
50. इति॑ जीवग्र॒हो जी॑वग्र॒ह इतीति॑ जीवग्र॒हो वै वै जी॑वग्र॒ह इतीति॑ जीवग्र॒हो वै । \newline
51. जी॒व॒ग्र॒हो वै वै जी॑वग्र॒हो जी॑वग्र॒हो वा ए॒ष ए॒ष वै जी॑वग्र॒हो जी॑वग्र॒हो वा ए॒षः । \newline
52. जी॒व॒ग्र॒ह इति॑ जीव - ग्र॒हः । \newline
53. वा ए॒ष ए॒ष वै वा ए॒ष यद् यदे॒ष वै वा ए॒ष यत् । \newline
54. ए॒ष यद् यदे॒ष ए॒ष यददा॒भ्यो ऽदा᳚भ्यो॒ यदे॒ष ए॒ष यददा᳚भ्यः । \newline
55. यददा॒भ्यो ऽदा᳚भ्यो॒ यद् यददा॒भ्यो ऽन॑भिषुत॒स्या न॑भिषुत॒स्या दा᳚भ्यो॒ यद् यददा॒भ्यो ऽन॑भिषुतस्य । \newline
56. अदा॒भ्यो ऽन॑भिषुत॒स्या न॑भिषुत॒स्या दा॒भ्यो ऽदा॒भ्यो ऽन॑भिषुतस्य गृह्णाति गृह्णा॒ त्यन॑भिषुत॒स्या दा॒भ्यो ऽदा॒भ्यो ऽन॑भिषुतस्य गृह्णाति । \newline
57. अन॑भिषुतस्य गृह्णाति गृह्णा॒ त्यन॑भिषुत॒स्या न॑भिषुतस्य गृह्णाति॒ जीव॑न्त॒म् जीव॑न्तम् गृह्णा॒ त्यन॑भिषुत॒स्या न॑भिषुतस्य गृह्णाति॒ जीव॑न्तम् । \newline
58. अन॑भिषुत॒स्येत्यन॑भि - सु॒त॒स्य॒ । \newline
59. गृ॒ह्णा॒ति॒ जीव॑न्त॒म् जीव॑न्तम् गृह्णाति गृह्णाति॒ जीव॑न्त मे॒वैव जीव॑न्तम् गृह्णाति गृह्णाति॒ जीव॑न्त मे॒व । \newline
60. जीव॑न्त मे॒वैव जीव॑न्त॒म् जीव॑न्त मे॒वैन॑ मेन मे॒व जीव॑न्त॒म् जीव॑न्त मे॒वैन᳚म् । \newline
61. ए॒वैन॑ मेन मे॒वै वैनꣳ॑ सुव॒र्गꣳ सु॑व॒र्ग मे॑न मे॒वै वैनꣳ॑ सुव॒र्गम् । \newline
62. ए॒नꣳ॒॒ सु॒व॒र्गꣳ सु॑व॒र्ग मे॑न मेनꣳ सुव॒र्गम् ॅलो॒कम् ॅलो॒कꣳ सु॑व॒र्ग मे॑न मेनꣳ सुव॒र्गम् ॅलो॒कम् । \newline
63. सु॒व॒र्गम् ॅलो॒कम् ॅलो॒कꣳ सु॑व॒र्गꣳ सु॑व॒र्गम् ॅलो॒कम् ग॑मयति गमयति लो॒कꣳ सु॑व॒र्गꣳ सु॑व॒र्गम् ॅलो॒कम् ग॑मयति । \newline
64. सु॒व॒र्गमिति॑ सुवः - गम् । \newline
65. लो॒कम् ग॑मयति गमयति लो॒कम् ॅलो॒कम् ग॑मयति॒ वि वि ग॑मयति लो॒कम् ॅलो॒कम् ग॑मयति॒ वि । \newline
66. ग॒म॒य॒ति॒ वि वि ग॑मयति गमयति॒ वि वै वै वि ग॑मयति गमयति॒ वि वै । \newline
67. वि वै वै वि वि वा ए॒त दे॒तद् वै वि वि वा ए॒तत् । \newline
68. वा ए॒त दे॒तद् वै वा ए॒तद् य॒ज्ञ्ं ॅय॒ज्ञ् मे॒तद् वै वा ए॒तद् य॒ज्ञ्म् । \newline
69. ए॒तद् य॒ज्ञ्ं ॅय॒ज्ञ् मे॒त दे॒तद् य॒ज्ञ्म् छि॑न्दन्ति छिन्दन्ति य॒ज्ञ् मे॒त दे॒तद् य॒ज्ञ्म् छि॑न्दन्ति । \newline
70. य॒ज्ञ्म् छि॑न्दन्ति छिन्दन्ति य॒ज्ञ्ं ॅय॒ज्ञ्म् छि॑न्दन्ति॒ यद् यच् छि॑न्दन्ति य॒ज्ञ्ं ॅय॒ज्ञ्म् छि॑न्दन्ति॒ यत् । \newline
71. छि॒न्द॒न्ति॒ यद् यच् छि॑न्दन्ति छिन्दन्ति॒ यददा॒भ्ये ऽदा᳚भ्ये॒ यच् छि॑न्दन्ति छिन्दन्ति॒ यददा᳚भ्ये । \newline
72. यददा॒भ्ये ऽदा᳚भ्ये॒ यद् यददा᳚भ्ये सꣳस्था॒पय॑न्ति सꣳस्था॒पय॒ न्त्यदा᳚भ्ये॒ यद् यददा᳚भ्ये सꣳस्था॒पय॑न्ति । \newline
73. अदा᳚भ्ये सꣳस्था॒पय॑न्ति सꣳस्था॒पय॒ न्त्यदा॒भ्ये ऽदा᳚भ्ये सꣳस्था॒पय॑ न्त्यꣳ॒॒शू नꣳ॒॒शून् थ्सꣳ॑स्था॒पय॒ न्त्यदा॒भ्ये ऽदा᳚भ्ये सꣳस्था॒पय॑ न्त्यꣳ॒॒शून् । \newline
74. सꣳ॒॒स्था॒पय॑ न्त्यꣳ॒॒शू नꣳ॒॒शून् थ्सꣳ॑स्था॒पय॑न्ति सꣳस्था॒पय॑ न्त्यꣳ॒॒शू नप्य प्यꣳ॒॒शून् थ्सꣳ॑स्था॒पय॑न्ति सꣳस्था॒पय॑ न्त्यꣳ॒॒शू नपि॑ । \newline
75. सꣳ॒॒स्था॒पय॒न्तीति॑ सं - स्था॒पय॑न्ति । \newline
76. अꣳ॒॒शू नप्य प्यꣳ॒॒शू नꣳ॒॒शू नपि॑ सृजति सृज॒ त्यप्यꣳ॒॒शू नꣳ॒॒शू नपि॑ सृजति । \newline
77. अपि॑ सृजति सृज॒ त्यप्यपि॑ सृजति य॒ज्ञ्स्य॑ य॒ज्ञ्स्य॑ सृज॒ त्यप्यपि॑ सृजति य॒ज्ञ्स्य॑ । \newline
78. सृ॒ज॒ति॒ य॒ज्ञ्स्य॑ य॒ज्ञ्स्य॑ सृजति सृजति य॒ज्ञ्स्य॒ सन्त॑त्यै॒ सन्त॑त्यै य॒ज्ञ्स्य॑ सृजति सृजति य॒ज्ञ्स्य॒ सन्त॑त्यै । \newline
79. य॒ज्ञ्स्य॒ सन्त॑त्यै॒ सन्त॑त्यै य॒ज्ञ्स्य॑ य॒ज्ञ्स्य॒ सन्त॑त्यै । \newline
80. सन्त॑त्या॒ इति॒ सं - त॒त्यै॒ । \newline
\pagebreak
\markright{ TS 6.6.10.1  \hfill https://www.vedavms.in \hfill}

\section{ TS 6.6.10.1 }

\textbf{TS 6.6.10.1 } \newline
\textbf{Samhita Paata} \newline

दे॒वा वै प्र॒बाहु॒ग्ग्रहा॑-नगृह्णत॒ स ए॒तं प्र॒जाप॑ति-रꣳ॒॒शु-म॑पश्य॒त् तम॑गृह्णीत॒ तेन॒ वै स आ᳚र्द्ध्नो॒द्-यस्यै॒वं ॅवि॒दुषो॒ऽꣳ॒शु-र्गृ॒ह्यत॑ ऋ॒द्ध्नोत्ये॒व स॒कृद॑भिषुतस्य गृह्णाति स॒कृद्धि स तेनाऽऽ*र्द्ध्नो॒न्मन॑सा गृह्णाति॒ मन॑ इव॒ हि प॒जाप॑तिः प्र॒जाप॑ते॒राप्त्या॒ औदु॑बंरेण गृह्णा॒त्यूर्ग्वा उ॑दु॒बंर॒ ऊर्ज॑मे॒वाव॑ रुन्धे॒ चतुः॑स्रक्ति भवति दि॒क्ष्वे॑- [  ] \newline

\textbf{Pada Paata} \newline

दे॒वाः । वै । प्र॒बाहु॒गिति॑ प्र - बाहु॑क् । ग्रहान्॑ । अ॒गृ॒ह्ण॒त॒ । सः । ए॒तम् । प्र॒जाप॑ति॒रिति॑ प्र॒जा - प॒तिः॒ । अꣳ॒॒शुम् । अ॒प॒श्य॒त् । तम् । अ॒गृ॒ह्णी॒त॒ । तेन॑ । वै । सः । आ॒द्‌र्ध्नो॒त् । यस्य॑ । ए॒वम् । वि॒दुषः॑ । अꣳ॒॒शुः । गृ॒ह्यते᳚ । ऋ॒द्ध्नोति॑ । ए॒व । स॒कृद॑भिषुत॒स्येति॑ स॒कृत् - अ॒भि॒षु॒त॒स्य॒ । गृ॒ह्णा॒ति॒ । स॒कृत् । हि । सः । तेन॑ । आद्‌र्ध्नो᳚त् । मन॑सा । गृ॒ह्णा॒ति॒ । मनः॑ । इ॒व॒ । हि । प्र॒जाप॑ति॒रिति॑ प्र॒जा - प॒तिः॒ । प्र॒जाप॑ते॒रिति॑ प्र॒जा - प॒तेः॒ । आप्त्यै᳚ । औदु॑बंरेण । गृ॒ह्णा॒ति॒ । ऊर्क् । वै । उ॒दु॒म्बरः॑ । ऊर्ज᳚म् । ए॒व । अवेति॑ । रु॒न्धे॒ । चतु॑स्स्र॒क्तीति॒ चतुः॑ - स्र॒क्ति॒ । भ॒व॒ति॒ । दि॒क्षु ।  \newline


\textbf{Krama Paata} \newline

दे॒वा वै । वै प्र॒बाहु॑क् । प्र॒बाहु॒ग् ग्रहान्॑ । प्र॒बाहु॒गिति॑ प्र - बाहु॑क् । ग्रहा॑नगृह्णत । अ॒गृ॒ह्ण॒त॒ सः । स ए॒तम् । ए॒तम् प्र॒जाप॑तिः । प्र॒जाप॑तिरꣳ॒॒शुम् । प्र॒जाप॑ति॒रिति॑ प्र॒जा - प॒तिः॒ । अꣳ॒॒शुम॑पश्यत् । अ॒प॒श्य॒त् तम् । तम॑गृह्णीत । अ॒गृ॒ह्णी॒त॒ तेन॑ । तेन॒ वै । वै सः । स आ᳚र्द्ध्नोत् । आ॒र्द्ध्नो॒द् यस्य॑ । यस्यै॒वम् । ए॒वम् ॅवि॒दुषः॑ । वि॒दुषो॒ऽꣳ॒शुः । अꣳ॒॒शुर् गृ॒ह्यते᳚ । गृ॒ह्यत॑ ऋ॒द्ध्नोति॑ । ऋ॒द्ध्नोत्ये॒व । ए॒व स॒कृद॑भिषुतस्य । स॒कृद॑भिषुतस्य गृह्णाति । स॒कृद॑भिषुत॒स्येति॑ स॒कृत् - अ॒भि॒षु॒त॒स्य॒ । गृ॒ह्णा॒ति॒ स॒कृत् । स॒कृद्‌धि । हि सः । स तेन॑ । तेनार्द्ध्नो᳚त् । आर्द्ध्नो॒न् मन॑सा । मन॑सा गृह्णाति । गृ॒ह्णा॒ति॒ मनः॑ । मन॑ इव । इ॒व॒ हि । हि प्र॒जाप॑तिः । प्र॒जाप॑तिः प्र॒जाप॑तेः । प्र॒जाप॑ति॒रिति॑ प्र॒जा - प॒तिः॒ । प्र॒जाप॑ते॒राप्त्यै᳚ । प्र॒जाप॑ते॒रिति॑ प्र॒जा - प॒तेः॒ । आप्त्या॒ औदु॑म्बरेण । औदु॑म्बरेण गृह्णाति । गृ॒ह्णा॒त्यूर्क् । ऊर्ग् वै । वा उ॑दु॒म्बरः॑ । उ॒दु॒म्बर॒ ऊर्ज᳚म् । ऊर्ज॑मे॒व । ए॒वाव॑ । अव॑ रुन्धे । रु॒न्धे॒ चतुः॑स्रक्ति । चतुः॑स्रक्ति भवति । चतुः॑स्र॒क्तीति॒ चतुः॑ - स्र॒क्ति॒ । भ॒व॒ति॒ दि॒क्षु । दि॒क्ष्वे॑व \newline

\textbf{Jatai Paata} \newline

1. दे॒वा वै वै दे॒वा दे॒वा वै । \newline
2. वै प्र॒बाहु॑क् प्र॒बाहु॒ग् वै वै प्र॒बाहु॑क् । \newline
3. प्र॒बाहु॒ग् ग्रहा॒न् ग्रहा᳚न् प्र॒बाहु॑क् प्र॒बाहु॒ग् ग्रहान्॑ । \newline
4. प्र॒बाहु॒गिति॑ प्र - बाहु॑क् । \newline
5. ग्रहा॑ नगृह्णता गृह्णत॒ ग्रहा॒न् ग्रहा॑ नगृह्णत । \newline
6. अ॒गृ॒ह्ण॒त॒ स सो॑ ऽगृह्णता गृह्णत॒ सः । \newline
7. स ए॒त मे॒तꣳ स स ए॒तम् । \newline
8. ए॒तम् प्र॒जाप॑तिः प्र॒जाप॑ति रे॒त मे॒तम् प्र॒जाप॑तिः । \newline
9. प्र॒जाप॑ति रꣳ॒॒शु मꣳ॒॒शुम् प्र॒जाप॑तिः प्र॒जाप॑ति रꣳ॒॒शुम् । \newline
10. प्र॒जाप॑ति॒रिति॑ प्र॒जा - प॒तिः॒ । \newline
11. अꣳ॒॒शु म॑पश्य दपश्य दꣳ॒॒शु मꣳ॒॒शु म॑पश्यत् । \newline
12. अ॒प॒श्य॒त् तम् त म॑पश्य दपश्य॒त् तम् । \newline
13. त म॑गृह्णीता गृह्णीत॒ तम् त म॑गृह्णीत । \newline
14. अ॒गृ॒ह्णी॒त॒ तेन॒ तेना॑ गृह्णीता गृह्णीत॒ तेन॑ । \newline
15. तेन॒ वै वै तेन॒ तेन॒ वै । \newline
16. वै स स वै वै सः । \newline
17. स आ᳚र्द्ध्नो दार्द्ध्नो॒थ् स स आ᳚र्द्ध्नोत् । \newline
18. आ॒र्द्ध्नो॒द् यस्य॒ यस्या᳚र्द्ध्नो दार्द्ध्नो॒द् यस्य॑ । \newline
19. यस्यै॒व मे॒वं ॅयस्य॒ यस्यै॒वम् । \newline
20. ए॒वं ॅवि॒दुषो॑ वि॒दुष॑ ए॒व मे॒वं ॅवि॒दुषः॑ । \newline
21. वि॒दुषो॒ ऽꣳ॒शु रꣳ॒॒शुर् वि॒दुषो॑ वि॒दुषो॒ ऽꣳ॒शुः । \newline
22. अꣳ॒॒शुर् गृ॒ह्यते॑ गृ॒ह्यते॒ ऽꣳ॒शु रꣳ॒॒शुर् गृ॒ह्यते᳚ । \newline
23. गृ॒ह्यत॑ ऋ॒द्ध्नो त्यृ॒द्ध्नोति॑ गृ॒ह्यते॑ गृ॒ह्यत॑ ऋ॒द्ध्नोति॑ । \newline
24. ऋ॒द्ध्नो त्ये॒वैव र्‌द्ध्नो त्यृ॒द्ध्नो त्ये॒व । \newline
25. ए॒व स॒कृद॑भिषुतस्य स॒कृद॑भिषुत स्यै॒वैव स॒कृद॑भिषुतस्य । \newline
26. स॒कृद॑भिषुतस्य गृह्णाति गृह्णाति स॒कृद॑भिषुतस्य स॒कृद॑भिषुतस्य गृह्णाति । \newline
27. स॒कृद॑भिषुत॒स्येति॑ स॒कृत् - अ॒भि॒षु॒त॒स्य॒ । \newline
28. गृ॒ह्णा॒ति॒ स॒कृथ् स॒कृद् गृ॑ह्णाति गृह्णाति स॒कृत् । \newline
29. स॒कृद्धि हि स॒कृथ् स॒कृद्धि । \newline
30. हि स स हि हि सः । \newline
31. स तेन॒ तेन॒ स स तेन॑ । \newline
32. तेनार्द्ध्नो॒ दार्द्ध्नो॒त् तेन॒ तेनार्द्ध्नो᳚त् । \newline
33. आर्द्ध्नो॒न् मन॑सा॒ मन॒सा ऽऽर्द्ध्नो॒ दार्द्ध्नो॒न् मन॑सा । \newline
34. मन॑सा गृह्णाति गृह्णाति॒ मन॑सा॒ मन॑सा गृह्णाति । \newline
35. गृ॒ह्णा॒ति॒ मनो॒ मनो॑ गृह्णाति गृह्णाति॒ मनः॑ । \newline
36. मन॑ इवेव॒ मनो॒ मन॑ इव । \newline
37. इ॒व॒ हि हीवे॑व॒ हि । \newline
38. हि प्र॒जाप॑तिः प्र॒जाप॑ति॒र्॒. हि हि प्र॒जाप॑तिः । \newline
39. प्र॒जाप॑तिः प्र॒जाप॑तेः प्र॒जाप॑तेः प्र॒जाप॑तिः प्र॒जाप॑तिः प्र॒जाप॑तेः । \newline
40. प्र॒जाप॑ति॒रिति॑ प्र॒जा - प॒तिः॒ । \newline
41. प्र॒जाप॑ते॒ राप्त्या॒ आप्त्यै᳚ प्र॒जाप॑तेः प्र॒जाप॑ते॒ राप्त्यै᳚ । \newline
42. प्र॒जाप॑ते॒रिति॑ प्र॒जा - प॒तेः॒ । \newline
43. आप्त्या॒ औदुं॑बरे॒ णौदुं॑बरे॒णा प्त्या॒ आप्त्या॒ औदुं॑बरेण । \newline
44. औदुं॑बरेण गृह्णाति गृह्णा॒ त्यौदुं॑बरे॒ णौदुं॑बरेण गृह्णाति । \newline
45. गृ॒ह्णा॒ त्यूर्गूर्ग् गृ॑ह्णाति गृह्णा॒ त्यूर्क् । \newline
46. ऊर्ग् वै वा ऊर्गूर्ग् वै । \newline
47. वा उ॑दु॒म्बर॑ उदु॒म्बरो॒ वै वा उ॑दु॒म्बरः॑ । \newline
48. उ॒दु॒म्बर॒ ऊर्ज॒ मूर्ज॑ मुदु॒म्बर॑ उदु॒म्बर॒ ऊर्ज᳚म् । \newline
49. ऊर्ज॑ मे॒वै वोर्ज॒ मूर्ज॑ मे॒व । \newline
50. ए॒वावा वै॒वै वाव॑ । \newline
51. अव॑ रुन्धे रु॒न्धे ऽवाव॑ रुन्धे । \newline
52. रु॒न्धे॒ चतु॑स्स्रक्ति॒ चतु॑स्स्रक्ति रुन्धे रुन्धे॒ चतु॑स्स्रक्ति । \newline
53. चतु॑स्स्रक्ति भवति भवति॒ चतु॑स्स्रक्ति॒ चतु॑स्स्रक्ति भवति । \newline
54. चतु॑स्स्र॒क्तीति॒ चतुः॑ - स्र॒क्ति॒ । \newline
55. भ॒व॒ति॒ दि॒क्षु दि॒क्षु भ॑वति भवति दि॒क्षु । \newline
56. दि॒क्ष्वे॑ वैव दि॒क्षु दि॒क्ष्वे॑व । \newline

\textbf{Ghana Paata } \newline

1. दे॒वा वै वै दे॒वा दे॒वा वै प्र॒बाहु॑क् प्र॒बाहु॒ग् वै दे॒वा दे॒वा वै प्र॒बाहु॑क् । \newline
2. वै प्र॒बाहु॑क् प्र॒बाहु॒ग् वै वै प्र॒बाहु॒ग् ग्रहा॒न् ग्रहा᳚न् प्र॒बाहु॒ग् वै वै प्र॒बाहु॒ग् ग्रहान्॑ । \newline
3. प्र॒बाहु॒ग् ग्रहा॒न् ग्रहा᳚न् प्र॒बाहु॑क् प्र॒बाहु॒ग् ग्रहा॑ नगृह्णता गृह्णत॒ ग्रहा᳚न् प्र॒बाहु॑क् प्र॒बाहु॒ग् ग्रहा॑ नगृह्णत । \newline
4. प्र॒बाहु॒गिति॑ प्र - बाहु॑क् । \newline
5. ग्रहा॑ नगृह्णता गृह्णत॒ ग्रहा॒न् ग्रहा॑ नगृह्णत॒ स सो॑ ऽगृह्णत॒ ग्रहा॒न् ग्रहा॑ नगृह्णत॒ सः । \newline
6. अ॒गृ॒ह्ण॒त॒ स सो॑ ऽगृह्णता गृह्णत॒ स ए॒त मे॒तꣳ सो॑ ऽगृह्णता गृह्णत॒ स ए॒तम् । \newline
7. स ए॒त मे॒तꣳ स स ए॒तम् प्र॒जाप॑तिः प्र॒जाप॑ति रे॒तꣳ स स ए॒तम् प्र॒जाप॑तिः । \newline
8. ए॒तम् प्र॒जाप॑तिः प्र॒जाप॑ति रे॒त मे॒तम् प्र॒जाप॑ति रꣳ॒॒शु मꣳ॒॒शुम् प्र॒जाप॑ति रे॒त मे॒तम् प्र॒जाप॑ति रꣳ॒॒शुम् । \newline
9. प्र॒जाप॑ति रꣳ॒॒शु मꣳ॒॒शुम् प्र॒जाप॑तिः प्र॒जाप॑ति रꣳ॒॒शु म॑पश्य दपश्य दꣳ॒॒शुम् प्र॒जाप॑तिः प्र॒जाप॑ति रꣳ॒॒शु म॑पश्यत् । \newline
10. प्र॒जाप॑ति॒रिति॑ प्र॒जा - प॒तिः॒ । \newline
11. अꣳ॒॒शु म॑पश्य दपश्य दꣳ॒॒शु मꣳ॒॒शु म॑पश्य॒त् तम् त म॑पश्य दꣳ॒॒शु मꣳ॒॒शु म॑पश्य॒त् तम् । \newline
12. अ॒प॒श्य॒त् तम् त म॑पश्य दपश्य॒त् त म॑गृह्णीता गृह्णीत॒ त म॑पश्य दपश्य॒त् त म॑गृह्णीत । \newline
13. त म॑गृह्णीता गृह्णीत॒ तम् त म॑गृह्णीत॒ तेन॒ तेना॑ गृह्णीत॒ तम् त म॑गृह्णीत॒ तेन॑ । \newline
14. अ॒गृ॒ह्णी॒त॒ तेन॒ तेना॑ गृह्णीता गृह्णीत॒ तेन॒ वै वै तेना॑ गृह्णीता गृह्णीत॒ तेन॒ वै । \newline
15. तेन॒ वै वै तेन॒ तेन॒ वै स स वै तेन॒ तेन॒ वै सः । \newline
16. वै स स वै वै स आ᳚र्द्ध्नो दार्द्ध्नो॒थ् स वै वै स आ᳚र्द्ध्नोत् । \newline
17. स आ᳚र्द्ध्नो दार्द्ध्नो॒थ् स स आ᳚र्द्ध्नो॒द् यस्य॒ यस्या᳚र्द्ध्नो॒थ् स स आ᳚र्द्ध्नो॒द् यस्य॑ । \newline
18. आ॒र्द्ध्नो॒द् यस्य॒ यस्या᳚र्द्ध्नो दार्द्ध्नो॒द् यस्यै॒व मे॒वं ॅयस्या᳚र्द्ध्नो दार्द्ध्नो॒द् यस्यै॒वम् । \newline
19. यस्यै॒व मे॒वं ॅयस्य॒ यस्यै॒वं ॅवि॒दुषो॑ वि॒दुष॑ ए॒वं ॅयस्य॒ यस्यै॒वं ॅवि॒दुषः॑ । \newline
20. ए॒वं ॅवि॒दुषो॑ वि॒दुष॑ ए॒व मे॒वं ॅवि॒दुषो॒ ऽꣳ॒शु रꣳ॒॒शुर् वि॒दुष॑ ए॒व मे॒वं 
ॅवि॒दुषो॒ ऽꣳ॒शुः । \newline
21. वि॒दुषो॒ ऽꣳ॒शु रꣳ॒॒शुर् वि॒दुषो॑ वि॒दुषो॒ ऽꣳ॒शुर् गृ॒ह्यते॑ गृ॒ह्यते॒ ऽꣳ॒शुर् वि॒दुषो॑ 
वि॒दुषो॒ ऽꣳ॒शुर् गृ॒ह्यते᳚ । \newline
22. अꣳ॒॒शुर् गृ॒ह्यते॑ गृ॒ह्यते॒ ऽꣳ॒शु रꣳ॒॒शुर् गृ॒ह्यत॑ ऋ॒द्ध्नो त्यृ॒द्ध्नोति॑ गृ॒ह्यते॒ ऽꣳ॒शु रꣳ॒॒शुर् गृ॒ह्यत॑ ऋ॒द्ध्नोति॑ । \newline
23. गृ॒ह्यत॑ ऋ॒द्ध्नो त्यृ॒द्ध्नोति॑ गृ॒ह्यते॑ गृ॒ह्यत॑ ऋ॒द्ध्नो त्ये॒वैव र्द्ध्नोति॑ गृ॒ह्यते॑ गृ॒ह्यत॑ ऋ॒द्ध्नो त्ये॒व । \newline
24. ऋ॒द्ध्नो त्ये॒वैव र्द्ध्नो त्यृ॒द्ध्नो त्ये॒व स॒कृद॑भिषुतस्य स॒कृद॑भिषुतस्यै॒व र्द्ध्नो त्यृ॒द्ध्नो त्ये॒व स॒कृद॑भिषुतस्य । \newline
25. ए॒व स॒कृद॑भिषुतस्य स॒कृद॑भिषुत स्यै॒वैव स॒कृद॑भिषुतस्य गृह्णाति गृह्णाति स॒कृद॑भिषुत स्यै॒वैव स॒कृद॑भिषुतस्य गृह्णाति । \newline
26. स॒कृद॑भिषुतस्य गृह्णाति गृह्णाति स॒कृद॑भिषुतस्य स॒कृद॑भिषुतस्य गृह्णाति स॒कृथ् स॒कृद् गृ॑ह्णाति स॒कृद॑भिषुतस्य स॒कृद॑भिषुतस्य गृह्णाति स॒कृत् । \newline
27. स॒कृद॑भिषुत॒स्येति॑ स॒कृत् - अ॒भि॒षु॒त॒स्य॒ । \newline
28. गृ॒ह्णा॒ति॒ स॒कृथ् स॒कृद् गृ॑ह्णाति गृह्णाति स॒कृद्धि हि स॒कृद् गृ॑ह्णाति गृह्णाति स॒कृद्धि । \newline
29. स॒कृद्धि हि स॒कृथ् स॒कृद्धि स स हि स॒कृथ् स॒कृद्धि सः । \newline
30. हि स स हि हि स तेन॒ तेन॒ स हि हि स तेन॑ । \newline
31. स तेन॒ तेन॒ स स तेना र्द्ध्नो॒ दार्द्ध्नो॒त् तेन॒ स स तेना र्द्ध्नो᳚त् । \newline
32. तेना र्द्ध्नो॒ दार्द्ध्नो॒त् तेन॒ तेना र्द्ध्नो॒न् मन॑सा॒ मन॒सा ऽऽर्द्ध्नो॒त् तेन॒ तेना र्द्ध्नो॒न् मन॑सा । \newline
33. आर्द्ध्नो॒न् मन॑सा॒ मन॒सा ऽऽर्द्ध्नो॒ दार्द्ध्नो॒न् मन॑सा गृह्णाति गृह्णाति॒ मन॒सा ऽऽर्द्ध्नो॒ दार्द्ध्नो॒न् मन॑सा गृह्णाति । \newline
34. मन॑सा गृह्णाति गृह्णाति॒ मन॑सा॒ मन॑सा गृह्णाति॒ मनो॒ मनो॑ गृह्णाति॒ मन॑सा॒ मन॑सा गृह्णाति॒ मनः॑ । \newline
35. गृ॒ह्णा॒ति॒ मनो॒ मनो॑ गृह्णाति गृह्णाति॒ मन॑ इवेव॒ मनो॑ गृह्णाति गृह्णाति॒ मन॑ इव । \newline
36. मन॑ इवेव॒ मनो॒ मन॑ इव॒ हि हीव॒ मनो॒ मन॑ इव॒ हि । \newline
37. इ॒व॒ हि हीवे॑व॒ हि प्र॒जाप॑तिः प्र॒जाप॑ति॒र्॒. हीवे॑व॒ हि प्र॒जाप॑तिः । \newline
38. हि प्र॒जाप॑तिः प्र॒जाप॑ति॒र्॒. हि हि प्र॒जाप॑तिः प्र॒जाप॑तेः प्र॒जाप॑तेः प्र॒जाप॑ति॒र्॒. हि हि प्र॒जाप॑तिः प्र॒जाप॑तेः । \newline
39. प्र॒जाप॑तिः प्र॒जाप॑तेः प्र॒जाप॑तेः प्र॒जाप॑तिः प्र॒जाप॑तिः प्र॒जाप॑ते॒ राप्त्या॒ आप्त्यै᳚ प्र॒जाप॑तेः प्र॒जाप॑तिः प्र॒जाप॑तिः प्र॒जाप॑ते॒ राप्त्यै᳚ । \newline
40. प्र॒जाप॑ति॒रिति॑ प्र॒जा - प॒तिः॒ । \newline
41. प्र॒जाप॑ते॒ राप्त्या॒ आप्त्यै᳚ प्र॒जाप॑तेः प्र॒जाप॑ते॒ राप्त्या॒ औदुं॑बरे॒ णौदुं॑बरे॒णाप्त्यै᳚ प्र॒जाप॑तेः प्र॒जाप॑ते॒ राप्त्या॒ औदुं॑बरेण । \newline
42. प्र॒जाप॑ते॒रिति॑ प्र॒जा - प॒तेः॒ । \newline
43. आप्त्या॒ औदुं॑बरे॒ णौदुं॑बरे॒ णाप्त्या॒ आप्त्या॒ औदुं॑बरेण गृह्णाति गृह्णा॒ त्यौदुं॑बरे॒ णाप्त्या॒ आप्त्या॒ औदुं॑बरेण गृह्णाति । \newline
44. औदुं॑बरेण गृह्णाति गृह्णा॒ त्यौदुं॑बरे॒ णौदुं॑बरेण गृह्णा॒ त्यूर्गूर्ग् गृ॑ह्णा॒ त्यौदुं॑बरे॒
णौदुं॑बरेण गृह्णा॒ त्यूर्क् । \newline
45. गृ॒ह्णा॒ त्यूर्गूर्ग् गृ॑ह्णाति गृह्णा॒ त्यूर्ग् वै वा ऊर्ग् गृ॑ह्णाति गृह्णा॒ त्यूर्ग् वै । \newline
46. ऊर्ग् वै वा ऊर्गूर्ग् वा उ॑दु॒म्बर॑ उदु॒म्बरो॒ वा ऊर् गूर्ग् वा उ॑दु॒म्बरः॑ । \newline
47. वा उ॑दु॒म्बर॑ उदु॒म्बरो॒ वै वा उ॑दु॒म्बर॒ ऊर्ज॒ मूर्ज॑ मुदु॒म्बरो॒ वै वा उ॑दु॒म्बर॒ ऊर्ज᳚म् । \newline
48. उ॒दु॒म्बर॒ ऊर्ज॒ मूर्ज॑ मुदु॒म्बर॑ उदु॒म्बर॒ ऊर्ज॑ मे॒वैवोर्ज॑ मुदु॒म्बर॑ उदु॒म्बर॒ ऊर्ज॑ मे॒व । \newline
49. ऊर्ज॑ मे॒वै वोर्ज॒ मूर्ज॑ मे॒वा वावै॒वोर्ज॒ मूर्ज॑ मे॒वाव॑ । \newline
50. ए॒वावा वै॒वै वाव॑ रुन्धे रु॒न्धे ऽवै॒वै वाव॑ रुन्धे । \newline
51. अव॑ रुन्धे रु॒न्धे ऽवाव॑ रुन्धे॒ चतु॑स्स्रक्ति॒ चतु॑स्स्रक्ति रु॒न्धे ऽवाव॑ रुन्धे॒ चतु॑स्स्रक्ति । \newline
52. रु॒न्धे॒ चतु॑स्स्रक्ति॒ चतु॑स्स्रक्ति रुन्धे रुन्धे॒ चतु॑स्स्रक्ति भवति भवति॒ चतु॑स्स्रक्ति रुन्धे रुन्धे॒ चतु॑स्स्रक्ति भवति । \newline
53. चतु॑स्स्रक्ति भवति भवति॒ चतु॑स्स्रक्ति॒ चतु॑स्स्रक्ति भवति दि॒क्षु दि॒क्षु भ॑वति॒ चतु॑स्स्रक्ति॒ चतु॑स्स्रक्ति भवति दि॒क्षु । \newline
54. चतु॑स्स्र॒क्तीति॒ चतुः॑ - स्र॒क्ति॒ । \newline
55. भ॒व॒ति॒ दि॒क्षु दि॒क्षु भ॑वति भवति दि॒क्ष्वे॑वैव दि॒क्षु भ॑वति भवति दि॒क्ष्वे॑व । \newline
56. दि॒क्ष्वे॑वैव दि॒क्षु दि॒क्ष्वे॑व प्रति॒ प्रत्ये॒व दि॒क्षु दि॒क्ष्वे॑व प्रति॑ । \newline
\pagebreak
\markright{ TS 6.6.10.2  \hfill https://www.vedavms.in \hfill}

\section{ TS 6.6.10.2 }

\textbf{TS 6.6.10.2 } \newline
\textbf{Samhita Paata} \newline

-व प्रति॑ तिष्ठति॒ यो वा अꣳ॒॒शोरा॒यत॑नं॒ ॅवेदा॒ऽऽयत॑नवान् भवति वामदे॒व्यमिति॒ साम॒ तद्वा अ॑स्या॒ऽऽ*यत॑नं॒ मन॑सा॒ गाय॑मानो गृह्णात्या॒यत॑नवाने॒व भ॑वति॒ यद॑द्ध्व॒र्युरꣳ॒॒शुं गृ॒ह्णन् नार्द्धये॑दु॒भाभ्यां॒ नर्द्ध्ये॑ताद्ध्व॒र्यवे॑ च॒ यज॑मानाय च॒ यद॒र्द्धये॑-दु॒भाभ्या॑-मृद्ध्ये॒तान॑वानं गृह्णाति॒ सैवास्यर्द्धि॒र॒. हिर॑ण्यम॒भि व्य॑नित्य॒ ( )-मृतं॒ ॅवै हिर॑ण्य॒मायुः॑ प्रा॒ण आयु॑षै॒वामृत॑म॒भि धि॑नोति श॒तमा॑नं भवति श॒तायुः॒ पुरु॑षः श॒तेन्द्रि॑य॒ आयु॑ष्ये॒वेन्द्रि॒ये प्रति॑ तिष्ठति ॥ \newline

\textbf{Pada Paata} \newline

ए॒व । प्रतीति॑ । ति॒ष्ठ॒ति॒ । यः । वै । अꣳ॒॒शोः । आ॒यत॑न॒मित्या᳚-यत॑नम् । वेद॑ । आ॒यत॑नवा॒नित्या॒यत॑न - वा॒न् । भ॒व॒ति॒ । वा॒म॒दे॒व्यमिति॑ वाम - दे॒व्यम् । इति॑ । साम॑ । तत् । वै । अ॒स्य॒ । आ॒यत॑न॒मित्या᳚ - यत॑नम् । मन॑सा । गाय॑मानः । गृ॒ह्णा॒ति॒ । आ॒यत॑नवा॒नित्या॒यत॑न - वा॒न् । ए॒व । भ॒व॒ति॒ । यत् । अ॒द्ध्व॒र्युः । अꣳ॒॒शुम् । गृ॒ह्णन्न् । न । अ॒द्‌र्धये᳚त् । उ॒भाभ्या᳚म् । न । ऋ॒द्ध्ये॒त॒ । अ॒द्ध्व॒र्यवे᳚ । च॒ । यज॑मानाय । च॒ । यत् । अ॒द्‌र्धये᳚त् । उ॒भाभ्या᳚म् । ऋ॒द्ध्ये॒त॒ । अन॑वान॒मित्यन॑व - अ॒न॒म् । गृ॒ह्णा॒ति॒ । सा । ए॒व । अ॒स्य॒ । ऋद्धिः॑ । हिर॑ण्यम् । अ॒भि । वीति॑ । अ॒नि॒ति॒ ( ) । अ॒मृत᳚म् । वै । हिर॑ण्यम् । आयुः॑ । प्रा॒ण इति॑ प्र-अ॒नः । आयु॑षा । ए॒व । अ॒मृत᳚म् । अ॒भीति॑ । धि॒नो॒ति॒ । श॒तमा॑न॒मिति॑ श॒त - मा॒न॒म् । भ॒व॒ति॒ । श॒तायु॒रिति॑ श॒त - आ॒युः॒ । पुरु॑षः । श॒तेन्द्रि॑य॒ इति॑ श॒त-इ॒न्द्रि॒यः॒ । आयु॑षि । ए॒व । इ॒न्द्रि॒ये । प्रतीति॑ । ति॒ष्ठ॒ति॒ ॥  \newline


\textbf{Krama Paata} \newline

ए॒व प्रति॑ । प्रति॑ तिष्ठति । ति॒ष्ठ॒ति॒ यः । यो वै । वा अꣳ॒॒शोः । अꣳ॒॒शोरा॒यत॑नम् । आ॒यत॑न॒म् ॅवेद॑ । आ॒यत॑न॒मित्या᳚ - यत॑नम् । वेदा॒यत॑नवान् । आ॒यत॑नवान् भवति । आ॒यत॑नवा॒नित्या॒यत॑न - वा॒न्॒ । भ॒व॒ति॒ वा॒म॒दे॒व्यम् । वा॒म॒दे॒व्यमिति॑ । वा॒म॒दे॒व्यमिति॑ वाम - दे॒व्यम् । इति॒ साम॑ । साम॒ तत् । तद् वै । वा अ॑स्य । अ॒स्या॒यत॑नम् । आ॒यत॑न॒म् मन॑सा । आ॒यत॑न॒मित्या᳚ यत॑नम् । मन॑सा॒ गाय॑मानः । गाय॑मानो गृह्णाति । गृ॒ह्णा॒त्या॒यत॑नवान् । आ॒यत॑नवाने॒व । आ॒यत॑नवा॒नित्या॒यत॑न - वा॒न्॒ । ए॒व भ॑वति । भ॒व॒ति॒ यत् । यद॑द्ध्व॒र्युः । अ॒द्ध्व॒र्युरꣳ॒॒शुम् । अꣳ॒॒शुम् गृ॒ह्णन्न् । गृ॒ह्णन् न । नार्द्धये᳚त् । अ॒र्द्धये॑दु॒भाभ्या᳚म् । उ॒भाभ्या॒म् न । नर्द्ध्ये᳚त । ऋ॒द्ध्ये॒,ता॒द्ध्व॒र्यवे᳚ । अ॒द्ध्व॒र्यवे॑ च । च॒ यज॑मानाय । यज॑मानाय च । च॒ यत् । यद॒र्द्धये᳚त् । अ॒र्द्धये॑दु॒भाभ्या᳚म् । उ॒भाभ्या॑मृद्ध्येत । ऋ॒द्ध्ये॒तान॑वानम् । अन॑वानम् गृह्णाति । अन॑वान॒मित्यन॑व - अ॒न॒म् । गृ॒ह्णा॒ति॒ सा । सैव । ए॒वास्य॑ । अ॒स्यर्द्धिः॑ । ऋद्धि॒र्॒. हिर॑ण्यम् । हिर॑ण्यम॒भि । अ॒भि वि । व्य॑निति ( ) । अ॒नि॒त्य॒मृत᳚म् । अ॒मृत॒म् ॅवै । वै हिर॑ण्यम् । हिर॑ण्य॒मायुः॑ । आयुः॑ प्रा॒णः । प्रा॒ण आयु॑षा । प्रा॒ण इति॑ प्र - अ॒नः । आयु॑षै॒व । ए॒वामृत᳚म् । अ॒मृत॑म॒भि । अ॒भि धि॑नोति । धि॒नो॒ति॒ श॒तमा॑नम् । श॒तमा॑नम् भवति । श॒तमा॑न॒मिति॑ श॒त - मा॒न॒म् । भ॒व॒ति॒ श॒तायुः॑ । श॒तायुः॒ पुरु॑षः । श॒तायु॒रिति॑ श॒त - आ॒युः॒ । पुरु॑षः श॒तेन्द्रि॑यः । श॒तेन्द्रि॑य॒ आयु॑षि । श॒तेन्द्रि॑य॒ इति॑ श॒त - इ॒न्द्रि॒यः॒ । आयु॑ष्ये॒व । ए॒वेन्द्रि॒ये । इ॒न्द्रि॒ये प्रति॑ । प्रति॑ तिष्ठति । ति॒ष्ठ॒तीति॑ तिष्ठति । \newline

\textbf{Jatai Paata} \newline

1. ए॒व प्रति॒ प्रत्ये॒ वैव प्रति॑ । \newline
2. प्रति॑ तिष्ठति तिष्ठति॒ प्रति॒ प्रति॑ तिष्ठति । \newline
3. ति॒ष्ठ॒ति॒ यो यस्ति॑ष्ठति तिष्ठति॒ यः । \newline
4. यो वै वै यो यो वै । \newline
5. वा अꣳ॒॒शो रꣳ॒॒शोर् वै वा अꣳ॒॒शोः । \newline
6. अꣳ॒॒शो रा॒यत॑न मा॒यत॑न मꣳ॒॒शो रꣳ॒॒शो रा॒यत॑नम् । \newline
7. आ॒यत॑नं॒ ॅवेद॒ वेदा॒यत॑न मा॒यत॑नं॒ ॅवेद॑ । \newline
8. आ॒यत॑न॒मित्या᳚ - यत॑नम् । \newline
9. वेदा॒ यत॑नवा ना॒यत॑नवा॒न्॒. वेद॒ वेदा॒ यत॑नवान् । \newline
10. आ॒यत॑नवान् भवति भव त्या॒यत॑नवा ना॒यत॑नवान् भवति । \newline
11. आ॒यत॑नवा॒नित्या॒यत॑न - वा॒न् । \newline
12. भ॒व॒ति॒ वा॒म॒दे॒व्यं ॅवा॑मदे॒व्यम् भ॑वति भवति वामदे॒व्यम् । \newline
13. वा॒म॒दे॒व्य मितीति॑ वामदे॒व्यं ॅवा॑मदे॒व्य मिति॑ । \newline
14. वा॒म॒दे॒व्यमिति॑ वाम - दे॒व्यम् । \newline
15. इति॒ साम॒ सामे तीति॒ साम॑ । \newline
16. साम॒ तत् तथ् साम॒ साम॒ तत् । \newline
17. तद् वै वै तत् तद् वै । \newline
18. वा अ॑स्यास्य॒ वै वा अ॑स्य । \newline
19. अ॒स्या॒ यत॑न मा॒यत॑न मस्यास्या॒ यत॑नम् । \newline
20. आ॒यत॑न॒म् मन॑सा॒ मन॑सा॒ ऽऽयत॑न मा॒यत॑न॒म् मन॑सा । \newline
21. आ॒यत॑न॒मित्या᳚ - यत॑नम् । \newline
22. मन॑सा॒ गाय॑मानो॒ गाय॑मानो॒ मन॑सा॒ मन॑सा॒ गाय॑मानः । \newline
23. गाय॑मानो गृह्णाति गृह्णाति॒ गाय॑मानो॒ गाय॑मानो गृह्णाति । \newline
24. गृ॒ह्णा॒ त्या॒यत॑नवा ना॒यत॑नवान् गृह्णाति गृह्णा त्या॒यत॑नवान् । \newline
25. आ॒यत॑नवा ने॒वैवा यत॑नवा ना॒यत॑न वाने॒व । \newline
26. आ॒यत॑नवा॒नित्या॒यत॑न - वा॒न् । \newline
27. ए॒व भ॑वति भव त्ये॒वैव भ॑वति । \newline
28. भ॒व॒ति॒ यद् यद् भ॑वति भवति॒ यत् । \newline
29. यद॑द्ध्व॒र्यु र॑द्ध्व॒र्युर् यद् यद॑द्ध्व॒र्युः । \newline
30. अ॒द्ध्व॒र्यु रꣳ॒॒शु मꣳ॒॒शु म॑द्ध्व॒र्यु र॑द्ध्व॒र्यु रꣳ॒॒शुम् । \newline
31. अꣳ॒॒शुम् गृ॒ह्णन् गृ॒ह्णन् नꣳ॒॒शु मꣳ॒॒शुम् गृ॒ह्णन्न् । \newline
32. गृ॒ह्णन् न न गृ॒ह्णन् गृ॒ह्णन् न । \newline
33. नार्द्धये॑ द॒र्द्धये॒न् न नार्द्धये᳚त् । \newline
34. अ॒र्द्धये॑ दु॒भाभ्या॑ मु॒भाभ्या॑ म॒र्द्धये॑ द॒र्द्धये॑ दु॒भाभ्या᳚म् । \newline
35. उ॒भाभ्या॒न् न नोभाभ्या॑ मु॒भाभ्या॒न् न । \newline
36. न र्‌द्ध्ये॑त र्‌द्ध्येत॒ न न र्‌द्ध्ये॑त । \newline
37. ऋ॒द्ध्ये॒ता॒ द्ध्व॒र्यवे᳚ ऽद्ध्व॒र्यव॑ ऋद्ध्येत र्‌द्ध्येता द्ध्व॒र्यवे᳚ । \newline
38. अ॒द्ध्व॒र्यवे॑ च चाद्ध्व॒र्यवे᳚ ऽद्ध्व॒र्यवे॑ च । \newline
39. च॒ यज॑मानाय॒ यज॑मानाय च च॒ यज॑मानाय । \newline
40. यज॑मानाय च च॒ यज॑मानाय॒ यज॑मानाय च । \newline
41. च॒ यद् यच् च॑ च॒ यत् । \newline
42. यद॒र्द्धये॑ द॒र्द्धये॒द् यद् यद॒र्द्धये᳚त् । \newline
43. अ॒र्द्धये॑ दु॒भाभ्या॑ मु॒भाभ्या॑ म॒र्द्धये॑ द॒र्द्धये॑ दु॒भाभ्या᳚म् । \newline
44. उ॒भाभ्या॑ मृद्ध्येत र्‌द्ध्ये तो॒भाभ्या॑ मु॒भाभ्या॑ मृद्ध्येत । \newline
45. ऋ॒द्ध्ये॒ता न॑वान॒ मन॑वान मृद्ध्येत र्‌द्ध्ये॒ता न॑वानम् । \newline
46. अन॑वानम् गृह्णाति गृह्णा॒ त्यन॑वान॒ मन॑वानम् गृह्णाति । \newline
47. अन॑वान॒मित्यन॑व - अ॒न॒म् । \newline
48. गृ॒ह्णा॒ति॒ सा सा गृ॑ह्णाति गृह्णाति॒ सा । \newline
49. सैवैव सा सैव । \newline
50. ए॒वास्या᳚ स्यै॒वै वास्य॑ । \newline
51. अ॒स्य र्‌द्धि॒र्॒. ऋद्धि॑र स्या॒स्य र्‌द्धिः॑ । \newline
52. ऋद्धि॒र्॒. हिर॑ण्यꣳ॒॒ हिर॑ण्य॒ मृद्धि॒र्॒. ऋद्धि॒र्॒. हिर॑ण्यम् । \newline
53. हिर॑ण्य म॒भ्य॑भि हिर॑ण्यꣳ॒॒ हिर॑ण्य म॒भि । \newline
54. अ॒भि वि व्या᳚(1॒)भ्य॑भि वि । \newline
55. व्य॑नित्यनिति॒ वि व्य॑निति । \newline
56. अ॒नि॒ त्य॒मृत॑ म॒मृत॑ मनि त्यनि त्य॒मृत᳚म् । \newline
57. अ॒मृतं॒ ॅवै वा अ॒मृत॑ म॒मृतं॒ ॅवै । \newline
58. वै हिर॑ण्यꣳ॒॒ हिर॑ण्यं॒ ॅवै वै हिर॑ण्यम् । \newline
59. हिर॑ण्य॒ मायु॒ रायु॒र्॒. हिर॑ण्यꣳ॒॒ हिर॑ण्य॒ मायुः॑ । \newline
60. आयुः॑ प्रा॒णः प्रा॒ण आयु॒ रायुः॑ प्रा॒णः । \newline
61. प्रा॒ण आयु॒षा ऽऽयु॑षा प्रा॒णः प्रा॒ण आयु॑षा । \newline
62. प्रा॒ण इति॑ प्र - अ॒नः । \newline
63. आयु॑ षै॒वै वायु॒षा ऽऽयु॑ षै॒व । \newline
64. ए॒वा मृत॑ म॒मृत॑ मे॒वै वामृत᳚म् । \newline
65. अ॒मृत॑ म॒भ्या᳚(1॒)भ्य॑मृत॑ म॒मृत॑ म॒भि । \newline
66. अ॒भि धि॑नोति धिनो त्य॒भ्य॑भि धि॑नोति । \newline
67. धि॒नो॒ति॒ श॒तमा॑नꣳ श॒तमा॑नम् धिनोति धिनोति श॒तमा॑नम् । \newline
68. श॒तमा॑नम् भवति भवति श॒तमा॑नꣳ श॒तमा॑नम् भवति । \newline
69. श॒तमा॑न॒मिति॑ श॒त - मा॒न॒म् । \newline
70. भ॒व॒ति॒ श॒तायुः॑ श॒तायु॑र् भवति भवति श॒तायुः॑ । \newline
71. श॒तायुः॒ पुरु॑षः॒ पुरु॑षः श॒तायुः॑ श॒तायुः॒ पुरु॑षः । \newline
72. श॒तायु॒रिति॑ श॒त - आ॒युः॒ । \newline
73. पुरु॑षः श॒तेन्द्रि॑यः श॒तेन्द्रि॑यः॒ पुरु॑षः॒ पुरु॑षः श॒तेन्द्रि॑यः । \newline
74. श॒तेन्द्रि॑य॒ आयु॒ष्या यु॑षि श॒तेन्द्रि॑यः श॒तेन्द्रि॑य॒ आयु॑षि । \newline
75. श॒तेन्द्रि॑य॒ इति॑ श॒त - इ॒न्द्रि॒यः॒ । \newline
76. आयु॑ ष्ये॒वै वायु॒ ष्यायु॑ष्ये॒व । \newline
77. ए॒वेन्द्रि॒य इ॑न्द्रि॒य ए॒वैवेन्द्रि॒ये । \newline
78. इ॒न्द्रि॒ये प्रति॒ प्रती᳚न्द्रि॒य इ॑न्द्रि॒ये प्रति॑ । \newline
79. प्रति॑ तिष्ठति तिष्ठति॒ प्रति॒ प्रति॑ तिष्ठति । \newline
80. ति॒ष्ठ॒तीति॑ तिष्ठति । \newline

\textbf{Ghana Paata } \newline

1. ए॒व प्रति॒ प्रत्ये॒वैव प्रति॑ तिष्ठति तिष्ठति॒ प्रत्ये॒वैव प्रति॑ तिष्ठति । \newline
2. प्रति॑ तिष्ठति तिष्ठति॒ प्रति॒ प्रति॑ तिष्ठति॒ यो यस्ति॑ष्ठति॒ प्रति॒ प्रति॑ तिष्ठति॒ यः । \newline
3. ति॒ष्ठ॒ति॒ यो यस्ति॑ष्ठति तिष्ठति॒ यो वै वै यस्ति॑ष्ठति तिष्ठति॒ यो वै । \newline
4. यो वै वै यो यो वा अꣳ॒॒शो रꣳ॒॒शोर् वै यो यो वा अꣳ॒॒शोः । \newline
5. वा अꣳ॒॒शो रꣳ॒॒शोर् वै वा अꣳ॒॒शो रा॒यत॑न मा॒यत॑न मꣳ॒॒शोर् वै वा अꣳ॒॒शो रा॒यत॑नम् । \newline
6. अꣳ॒॒शो रा॒यत॑न मा॒यत॑न मꣳ॒॒शो रꣳ॒॒शो रा॒यत॑नं॒ ॅवेद॒ वेदा॒यत॑न मꣳ॒॒शो रꣳ॒॒शो रा॒यत॑नं॒ ॅवेद॑ । \newline
7. आ॒यत॑नं॒ ॅवेद॒ वेदा॒ यत॑न मा॒यत॑नं॒ ॅवेदा॒ यत॑नवा ना॒यत॑नवा॒न्॒. वेदा॒ यत॑न मा॒यत॑नं॒ ॅवेदा॒ यत॑नवान् । \newline
8. आ॒यत॑न॒मित्या᳚ - यत॑नम् । \newline
9. वेदा॒ यत॑नवा ना॒यत॑नवा॒न्॒. वेद॒ वेदा॒ यत॑नवान् भवति भव त्या॒यत॑नवा॒न्॒. वेद॒ वेदा॒ यत॑नवान् भवति । \newline
10. आ॒यत॑नवान् भवति भव त्या॒यत॑नवा ना॒यत॑नवान् भवति वामदे॒व्यं ॅवा॑मदे॒व्यम् भ॑व त्या॒यत॑नवा ना॒यत॑नवान् भवति वामदे॒व्यम् । \newline
11. आ॒यत॑नवा॒नित्या॒यत॑न - वा॒न् । \newline
12. भ॒व॒ति॒ वा॒म॒दे॒व्यं ॅवा॑मदे॒व्यम् भ॑वति भवति वामदे॒व्य मितीति॑ वामदे॒व्यम् भ॑वति भवति वामदे॒व्य मिति॑ । \newline
13. वा॒म॒दे॒व्य मितीति॑ वामदे॒व्यं ॅवा॑मदे॒व्य मिति॒ साम॒ सामेति॑ वामदे॒व्यं ॅवा॑मदे॒व्य मिति॒ साम॑ । \newline
14. वा॒म॒दे॒व्यमिति॑ वाम - दे॒व्यम् । \newline
15. इति॒ साम॒ सामे तीति॒ साम॒ तत् तथ् सामे तीति॒ साम॒ तत् । \newline
16. साम॒ तत् तथ् साम॒ साम॒ तद् वै वै तथ् साम॒ साम॒ तद् वै । \newline
17. तद् वै वै तत् तद् वा अ॑स्यास्य॒ वै तत् तद् वा अ॑स्य । \newline
18. वा अ॑स्यास्य॒ वै वा अ॑स्या॒ यत॑न मा॒यत॑न मस्य॒ वै वा अ॑स्या॒ यत॑नम् । \newline
19. अ॒स्या॒ यत॑न मा॒यत॑न मस्यास्या॒ यत॑न॒म् मन॑सा॒ मन॑सा॒ ऽऽयत॑न मस्यास्या॒ यत॑न॒म् मन॑सा । \newline
20. आ॒यत॑न॒म् मन॑सा॒ मन॑सा॒ ऽऽयत॑न मा॒यत॑न॒म् मन॑सा॒ गाय॑मानो॒ गाय॑मानो॒ मन॑सा॒ ऽऽयत॑न मा॒यत॑न॒म् मन॑सा॒ गाय॑मानः । \newline
21. आ॒यत॑न॒मित्या᳚ - यत॑नम् । \newline
22. मन॑सा॒ गाय॑मानो॒ गाय॑मानो॒ मन॑सा॒ मन॑सा॒ गाय॑मानो गृह्णाति गृह्णाति॒ गाय॑मानो॒ मन॑सा॒ मन॑सा॒ गाय॑मानो गृह्णाति । \newline
23. गाय॑मानो गृह्णाति गृह्णाति॒ गाय॑मानो॒ गाय॑मानो गृह्णा त्या॒यत॑नवा ना॒यत॑नवान् गृह्णाति॒ गाय॑मानो॒ गाय॑मानो गृह्णा त्या॒यत॑नवान् । \newline
24. गृ॒ह्णा॒ त्या॒यत॑नवा ना॒यत॑नवान् गृह्णाति गृह्णा त्या॒यत॑नवा ने॒वै वायत॑नवान् गृह्णाति गृह्णा त्या॒यत॑नवा ने॒व । \newline
25. आ॒यत॑नवा ने॒वै वायत॑नवा ना॒यत॑नवा ने॒व भ॑वति भव त्ये॒वायत॑नवा ना॒यत॑नवा ने॒व भ॑वति । \newline
26. आ॒यत॑नवा॒नित्या॒यत॑न - वा॒न् । \newline
27. ए॒व भ॑वति भव त्ये॒वैव भ॑वति॒ यद् यद् भ॑व त्ये॒वैव भ॑वति॒ यत् । \newline
28. भ॒व॒ति॒ यद् यद् भ॑वति भवति॒ यद॑द्ध्व॒र्यु र॑द्ध्व॒र्युर् यद् भ॑वति भवति॒ यद॑द्ध्व॒र्युः । \newline
29. यद॑द्ध्व॒र्यु र॑द्ध्व॒र्युर् यद् यद॑द्ध्व॒र्यु रꣳ॒॒शु मꣳ॒॒शु म॑द्ध्व॒र्युर् यद् यद॑द्ध्व॒र्यु रꣳ॒॒शुम् । \newline
30. अ॒द्ध्व॒र्यु रꣳ॒॒शु मꣳ॒॒शु म॑द्ध्व॒र्यु र॑द्ध्व॒र्यु रꣳ॒॒शुम् गृ॒ह्णन् गृ॒ह्णन् नꣳ॒॒शु म॑द्ध्व॒र्यु र॑द्ध्व॒र्यु रꣳ॒॒शुम् गृ॒ह्णन्न् । \newline
31. अꣳ॒॒शुम् गृ॒ह्णन् गृ॒ह्णन् नꣳ॒॒शु मꣳ॒॒शुम् गृ॒ह्णन् न न गृ॒ह्णन् नꣳ॒॒शु मꣳ॒॒शुम् गृ॒ह्णन् न । \newline
32. गृ॒ह्णन् न न गृ॒ह्णन् गृ॒ह्णन् नार्द्धये॑ द॒र्द्धये॒न् न गृ॒ह्णन् गृ॒ह्णन् नार्द्धये᳚त् । \newline
33. नार्द्धये॑ द॒र्द्धये॒न् न नार्द्धये॑ दु॒भाभ्या॑ मु॒भाभ्या॑ म॒र्द्धये॒न् न नार्द्धये॑ दु॒भाभ्या᳚म् । \newline
34. अ॒र्द्धये॑ दु॒भाभ्या॑ मु॒भाभ्या॑ म॒र्द्धये॑ द॒र्द्धये॑ दु॒भाभ्या॒न् न नोभाभ्या॑ म॒र्द्धये॑ द॒र्द्धये॑ दु॒भाभ्या॒न् न । \newline
35. उ॒भाभ्या॒न् न नोभाभ्या॑ मु॒भाभ्या॒न् न र्द्ध्ये॑त र्द्ध्येत॒ नोभाभ्या॑ मु॒भाभ्या॒न् न र्द्ध्ये॑त । \newline
36. न र्द्ध्ये॑त र्द्ध्येत॒ न न र्द्ध्ये॑ता द्ध्व॒र्यवे᳚ ऽद्ध्व॒र्यव॑ ऋद्ध्येत॒ न न र्द्ध्ये॑ता द्ध्व॒र्यवे᳚ । \newline
37. ऋ॒द्ध्ये॒ता॒ द्ध्व॒र्यवे᳚ ऽद्ध्व॒र्यव॑ ऋद्ध्येत र्द्ध्येता द्ध्व॒र्यवे॑ च चाद्ध्व॒र्यव॑ ऋद्ध्येत र्द्ध्येता द्ध्व॒र्यवे॑ च । \newline
38. अ॒द्ध्व॒र्यवे॑ च चाद्ध्व॒र्यवे᳚ ऽद्ध्व॒र्यवे॑ च॒ यज॑मानाय॒ यज॑मानाय चाद्ध्व॒र्यवे᳚ ऽद्ध्व॒र्यवे॑ च॒ यज॑मानाय । \newline
39. च॒ यज॑मानाय॒ यज॑मानाय च च॒ यज॑मानाय च च॒ यज॑मानाय च च॒ यज॑मानाय च । \newline
40. यज॑मानाय च च॒ यज॑मानाय॒ यज॑मानाय च॒ यद् यच् च॒ यज॑मानाय॒ यज॑मानाय च॒ यत् । \newline
41. च॒ यद् यच् च॑ च॒ यद॒र्द्धये॑ द॒र्द्धये॒द् यच् च॑ च॒ यद॒र्द्धये᳚त् । \newline
42. यद॒र्द्धये॑ द॒र्द्धये॒द् यद् यद॒र्द्धये॑ दु॒भाभ्या॑ मु॒भाभ्या॑ म॒र्द्धये॒द् यद् यद॒र्द्धये॑ दु॒भाभ्या᳚म् । \newline
43. अ॒र्द्धये॑ दु॒भाभ्या॑ मु॒भाभ्या॑ म॒र्द्धये॑ द॒र्द्धये॑ दु॒भाभ्या॑ मृद्ध्येत र्द्ध्ये तो॒भाभ्या॑ म॒र्द्धये॑ द॒र्द्धये॑ दु॒भाभ्या॑ मृद्ध्येत । \newline
44. उ॒भाभ्या॑ मृद्ध्येत र्द्ध्ये तो॒भाभ्या॑ मु॒भाभ्या॑ मृद्ध्ये॒ता न॑वान॒ मन॑वान मृद्ध्येतो॒भाभ्या॑ मु॒भाभ्या॑ मृद्ध्ये॒ता न॑वानम् । \newline
45. ऋ॒द्ध्ये॒ता न॑वान॒ मन॑वान मृद्ध्येत र्द्ध्ये॒ता न॑वानम् गृह्णाति गृह्णा॒ त्यन॑वान मृद्ध्येत र्द्ध्ये॒ता न॑वानम् गृह्णाति । \newline
46. अन॑वानम् गृह्णाति गृह्णा॒ त्यन॑वान॒ मन॑वानम् गृह्णाति॒ सा सा गृ॑ह्णा॒ त्यन॑वान॒ मन॑वानम् गृह्णाति॒ सा । \newline
47. अन॑वान॒मित्यन॑व - अ॒न॒म् । \newline
48. गृ॒ह्णा॒ति॒ सा सा गृ॑ह्णाति गृह्णाति॒ सैवैव सा गृ॑ह्णाति गृह्णाति॒ सैव । \newline
49. सैवैव सा सैवास्या᳚ स्यै॒व सा सैवास्य॑ । \newline
50. ए॒वास्या᳚ स्यै॒वै वास्य र्द्धि॒र्॒. ऋद्धि॑ रस्यै॒वै वास्य र्द्धिः॑ । \newline
51. अ॒स्य र्द्धि॒र्॒. ऋद्धि॑ रस्या॒स्य र्द्धि॒र्॒. हिर॑ण्यꣳ॒॒ हिर॑ण्य॒ मृद्धि॑ रस्या॒स्य र्द्धि॒र्॒. हिर॑ण्यम् । \newline
52. ऋद्धि॒र्॒. हिर॑ण्यꣳ॒॒ हिर॑ण्य॒ मृद्धि॒र्॒. ऋद्धि॒र्॒. हिर॑ण्य म॒भ्य॑भि हिर॑ण्य॒ मृद्धि॒र्॒. ऋद्धि॒र्॒. हिर॑ण्य म॒भि । \newline
53. हिर॑ण्य म॒भ्य॑भि हिर॑ण्यꣳ॒॒ हिर॑ण्य म॒भि वि व्य॑भि हिर॑ण्यꣳ॒॒ हिर॑ण्य म॒भि वि । \newline
54. अ॒भि वि व्या᳚(1॒)भ्य॑भि व्य॑नि त्यनिति॒ व्या᳚(1॒)भ्य॑भि व्य॑निति । \newline
55. व्य॑नि त्यनिति॒ वि व्य॑नि त्य॒मृत॑ म॒मृत॑ मनिति॒ वि व्य॑नि त्य॒मृत᳚म् । \newline
56. अ॒नि॒ त्य॒मृत॑ म॒मृत॑ मनि त्यनि त्य॒मृतं॒ ॅवै वा अ॒मृत॑ मनि त्यनि त्य॒मृतं॒ ॅवै । \newline
57. अ॒मृतं॒ ॅवै वा अ॒मृत॑ म॒मृतं॒ ॅवै हिर॑ण्यꣳ॒॒ हिर॑ण्यं॒ ॅवा अ॒मृत॑ म॒मृतं॒ ॅवै हिर॑ण्यम् । \newline
58. वै हिर॑ण्यꣳ॒॒ हिर॑ण्यं॒ ॅवै वै हिर॑ण्य॒ मायु॒ रायु॒र्॒. हिर॑ण्यं॒ ॅवै वै हिर॑ण्य॒ मायुः॑ । \newline
59. हिर॑ण्य॒ मायु॒ रायु॒र्॒. हिर॑ण्यꣳ॒॒ हिर॑ण्य॒ मायुः॑ प्रा॒णः प्रा॒ण आयु॒र्॒. हिर॑ण्यꣳ॒॒ हिर॑ण्य॒ मायुः॑ प्रा॒णः । \newline
60. आयुः॑ प्रा॒णः प्रा॒ण आयु॒ रायुः॑ प्रा॒ण आयु॒षा ऽऽयु॑षा प्रा॒ण आयु॒ रायुः॑ प्रा॒ण आयु॑षा । \newline
61. प्रा॒ण आयु॒षा ऽऽयु॑षा प्रा॒णः प्रा॒ण आयु॑षै॒वै वायु॑षा प्रा॒णः प्रा॒ण आयु॑षै॒व । \newline
62. प्रा॒ण इति॑ प्र - अ॒नः । \newline
63. आयु॑षै॒वै वायु॒षा ऽऽयु॑षै॒ वामृत॑ म॒मृत॑ मे॒वायु॒षा ऽऽयु॑षै॒ वामृत᳚म् । \newline
64. ए॒वामृत॑ म॒मृत॑ मे॒वै वामृत॑ म॒भ्या᳚(1॒)भ्य॑मृत॑ मे॒वै वामृत॑ म॒भि । \newline
65. अ॒मृत॑ म॒भ्या᳚(1॒)भ्य॑ मृत॑ म॒मृत॑ म॒भि धि॑नोति धिनो त्य॒भ्य॑ मृत॑ म॒मृत॑ म॒भि धि॑नोति । \newline
66. अ॒भि धि॑नोति धिनो त्य॒भ्य॑भि धि॑नोति श॒तमा॑नꣳ श॒तमा॑नम् धिनो त्य॒भ्य॑भि धि॑नोति श॒तमा॑नम् । \newline
67. धि॒नो॒ति॒ श॒तमा॑नꣳ श॒तमा॑नम् धिनोति धिनोति श॒तमा॑नम् भवति भवति श॒तमा॑नम् धिनोति धिनोति श॒तमा॑नम् भवति । \newline
68. श॒तमा॑नम् भवति भवति श॒तमा॑नꣳ श॒तमा॑नम् भवति श॒तायुः॑ श॒तायु॑र् भवति श॒तमा॑नꣳ श॒तमा॑नम् भवति श॒तायुः॑ । \newline
69. श॒तमा॑न॒मिति॑ श॒त - मा॒न॒म् । \newline
70. भ॒व॒ति॒ श॒तायुः॑ श॒तायु॑र् भवति भवति श॒तायुः॒ पुरु॑षः॒ पुरु॑षः श॒तायु॑र् भवति भवति श॒तायुः॒ पुरु॑षः । \newline
71. श॒तायुः॒ पुरु॑षः॒ पुरु॑षः श॒तायुः॑ श॒तायुः॒ पुरु॑षः श॒तेन्द्रि॑यः श॒तेन्द्रि॑यः॒ पुरु॑षः श॒तायुः॑ श॒तायुः॒ पुरु॑षः श॒तेन्द्रि॑यः । \newline
72. श॒तायु॒रिति॑ श॒त - आ॒युः॒ । \newline
73. पुरु॑षः श॒तेन्द्रि॑यः श॒तेन्द्रि॑यः॒ पुरु॑षः॒ पुरु॑षः श॒तेन्द्रि॑य॒ आयु॒ष्यायु॑षि श॒तेन्द्रि॑यः॒ पुरु॑षः॒ पुरु॑षः श॒तेन्द्रि॑य॒ आयु॑षि । \newline
74. श॒तेन्द्रि॑य॒ आयु॒ष्यायु॑षि श॒तेन्द्रि॑यः श॒तेन्द्रि॑य॒ आयु॑ष्ये॒ वैवायु॑षि श॒तेन्द्रि॑यः श॒तेन्द्रि॑य॒ आयु॑ष्ये॒व । \newline
75. श॒तेन्द्रि॑य॒ इति॑ श॒त - इ॒न्द्रि॒यः॒ । \newline
76. आयु॑ष्ये॒ वैवायु॒ ष्यायु॑ ष्ये॒वेन्द्रि॒य इ॑न्द्रि॒य ए॒वायु॒ ष्यायु॑ ष्ये॒वेन्द्रि॒ये । \newline
77. ए॒वेन्द्रि॒य इ॑न्द्रि॒य ए॒वैवेन्द्रि॒ये प्रति॒ प्रती᳚न्द्रि॒य ए॒वैवेन्द्रि॒ये प्रति॑ । \newline
78. इ॒न्द्रि॒ये प्रति॒ प्रती᳚न्द्रि॒य इ॑न्द्रि॒ये प्रति॑ तिष्ठति तिष्ठति॒ प्रती᳚न्द्रि॒य इ॑न्द्रि॒ये प्रति॑ तिष्ठति । \newline
79. प्रति॑ तिष्ठति तिष्ठति॒ प्रति॒ प्रति॑ तिष्ठति । \newline
80. ति॒ष्ठ॒तीति॑ तिष्ठति । \newline
\pagebreak
\markright{ TS 6.6.11.1  \hfill https://www.vedavms.in \hfill}

\section{ TS 6.6.11.1 }

\textbf{TS 6.6.11.1 } \newline
\textbf{Samhita Paata} \newline

प्र॒जाप॑ति-र्दे॒वेभ्यो॑ य॒ज्ञान् व्यादि॑श॒थ् स रि॑रिचा॒नो॑ऽमन्यत॒ स य॒ज्ञानाꣳ॑ षोडश॒धेन्द्रि॒यं ॅवी॒र्य॑मा॒त्मान॑म॒भि सम॑क्खिद॒त् तथ् षो॑ड॒श्य॑भव॒न्न वै षो॑ड॒शी नाम॑ य॒ज्ञो᳚ऽस्ति॒ यद्वाव षो॑ड॒शꣳ स्तो॒त्रꣳ षो॑ड॒शꣳ श॒स्त्रं तेन॑ षोड॒शी तथ् षो॑ड॒शिनः॑ षोडशि॒त्वं ॅयथ् षो॑ड॒शी गृ॒ह्यत॑ इन्द्रि॒यमे॒व तद्-वी॒र्यं॑ ॅयज॑मान आ॒त्मन् ध॑त्ते दे॒वेभ्यो॒ वै सु॑व॒र्गो लो॒को- [  ] \newline

\textbf{Pada Paata} \newline

प्र॒जाप॑ति॒रिति॑ प्र॒जा - प॒तिः॒ । दे॒वेभ्यः॑ । य॒ज्ञान् । व्यादि॑श॒दिति॑ वि - आदि॑शत् । सः । रि॒रि॒चा॒नः । अ॒म॒न्य॒त॒ । सः । य॒ज्ञाना᳚म् । षो॒ड॒श॒धेति॑ षोडश - धा । इ॒न्द्रि॒यम् । वी॒र्य᳚म् । आ॒त्मान᳚म् । अ॒भि । समिति॑ । अ॒क्खि॒द॒त् । तत् । षो॒ड॒शी । अ॒भ॒व॒त् । न । वै । षो॒ड॒शी । नाम॑ । य॒ज्ञ्ः । अ॒स्ति॒ । यत् । वाव । षो॒ड॒शम् । स्तो॒त्रम् । षो॒ड॒शम् । श॒स्त्रम् । तेन॑ । षो॒ड॒शी । तत् । षो॒ड॒शिनः॑ । षो॒ड॒शि॒त्वमिति॑ षोडशि -त्वम् । यत् । षो॒ड॒शी । गृ॒ह्यते᳚ । इ॒न्द्रि॒यम् । ए॒व । तत् । वी॒र्य᳚म् । यज॑मानः । आ॒त्मन्न् । ध॒त्ते॒ । दे॒वेभ्यः॑ । वै । सु॒व॒र्ग इति॑ सुवः - गः । लो॒कः ।  \newline


\textbf{Krama Paata} \newline

प्र॒जाप॑तिर् दे॒वेभ्यः॑ । प्र॒जाप॑ति॒रिति॑ प्र॒जा - प॒तिः॒ । दे॒वेभ्यो॑ य॒ज्ञान् । य॒ज्ञान् व्यादि॑शत् । व्यादि॑श॒थ् सः । व्यादि॑श॒दिति॑ वि - आदि॑शत् । स रि॑रिचा॒नः । रि॒रि॒चा॒नो॑ऽमन्यत । अ॒म॒न्य॒त॒ सः । स य॒ज्ञाना᳚म् । य॒ज्ञानाꣳ॑ षोडश॒धा । षो॒ड॒श॒धेन्द्रि॒यम् । षो॒ड॒श॒धेति॑ षोडश - धा । इ॒न्द्रि॒यम् ॅवी॒र्य᳚म् । वी॒र्य॑मा॒त्मान᳚म् । आ॒त्मान॑म॒भि । अ॒भि सम् । सम॑क्खिदत् । अ॒क्खि॒द॒त् तत् । तथ् षो॑ड॒शी । षो॒ड॒श्य॑भवत् । अ॒भ॒व॒न् न । न वै । वै षो॑ड॒शी । षो॒ड॒शी नाम॑ । नाम॑ य॒ज्ञ्ः । य॒ज्ञो᳚ऽस्ति । अ॒स्ति॒ यत् । यद् वाव । वाव षो॑ड॒शम् । षो॒ड॒शꣳ स्तो॒त्रम् । स्तो॒त्रꣳ षो॑ड॒शम् । षो॒ड॒शꣳ श॒स्त्रम् । श॒स्त्रम् तेन॑ । तेन॑ षोड॒शी । षो॒ड॒शी तत् । तथ् षो॑ड॒शिनः॑ । षो॒ड॒शिनः॑ षोडशि॒त्वम् । षो॒ड॒शि॒त्वम् ॅयत् । षो॒ड॒शि॒त्वमिति॑ षोडशि - त्वम् । यथ् षो॑ड॒शी । षो॒ड॒शी गृ॒ह्यते᳚ । गृ॒ह्यत॑ इन्द्रि॒यम् । इ॒न्द्रि॒यमे॒व । ए॒व तत् । तद् वी॒र्य᳚म् । वी॒र्य॑म् ॅयज॑मानः । यज॑मान आ॒त्मन्न् । आ॒त्मन् ध॑त्ते । ध॒त्ते॒ दे॒वेभ्यः॑ । दे॒वेभ्यो॒ वै । वै सु॑व॒र्गः । सु॒व॒र्गो लो॒कः । सु॒व॒र्ग इति॑ सुवः - गः । लो॒को न \newline

\textbf{Jatai Paata} \newline

1. प्र॒जाप॑तिर् दे॒वेभ्यो॑ दे॒वेभ्यः॑ प्र॒जाप॑तिः प्र॒जाप॑तिर् दे॒वेभ्यः॑ । \newline
2. प्र॒जाप॑ति॒रिति॑ प्र॒जा - प॒तिः॒ । \newline
3. दे॒वेभ्यो॑ य॒ज्ञान्. य॒ज्ञान् दे॒वेभ्यो॑ दे॒वेभ्यो॑ य॒ज्ञान् । \newline
4. य॒ज्ञान् व्यादि॑श॒द् व्यादि॑शद् य॒ज्ञान्. य॒ज्ञान् व्यादि॑शत् । \newline
5. व्यादि॑श॒थ् स स व्यादि॑श॒द् व्यादि॑श॒थ् सः । \newline
6. व्यादि॑श॒दिति॑ वि - आदि॑शत् । \newline
7. स रि॑रिचा॒नो रि॑रिचा॒नः स स रि॑रिचा॒नः । \newline
8. रि॒रि॒चा॒नो॑ ऽमन्यता मन्यत रिरिचा॒नो रि॑रिचा॒नो॑ ऽमन्यत । \newline
9. अ॒म॒न्य॒त॒ स सो॑ ऽमन्यता मन्यत॒ सः । \newline
10. स य॒ज्ञानां᳚ ॅय॒ज्ञानाꣳ॒॒ स स य॒ज्ञाना᳚म् । \newline
11. य॒ज्ञानाꣳ॑ षोडश॒धा षो॑डश॒धा य॒ज्ञानां᳚ ॅय॒ज्ञानाꣳ॑ षोडश॒धा । \newline
12. षो॒ड॒श॒धेन्द्रि॒य मि॑न्द्रि॒यꣳ षो॑डश॒धा षो॑डश॒धेन्द्रि॒यम् । \newline
13. षो॒ड॒श॒धेति॑ षोडश - धा । \newline
14. इ॒न्द्रि॒यं ॅवी॒र्यं॑ ॅवी॒र्य॑ मिन्द्रि॒य मि॑न्द्रि॒यं ॅवी॒र्य᳚म् । \newline
15. वी॒र्य॑ मा॒त्मान॑ मा॒त्मानं॑ ॅवी॒र्यं॑ ॅवी॒र्य॑ मा॒त्मान᳚म् । \newline
16. आ॒त्मान॑ म॒भ्या᳚(1॒)भ्या᳚त्मान॑ मा॒त्मान॑ म॒भि । \newline
17. अ॒भि सꣳ स म॒भ्य॑भि सम् । \newline
18. स म॑क्खिद दक्खिद॒थ् सꣳ स म॑क्खिदत् । \newline
19. अ॒क्खि॒द॒त् तत् तद॑क्खिद दक्खिद॒त् तत् । \newline
20. तथ्षो॑ड॒शी षो॑ड॒शी तत् तथ्षो॑ड॒शी । \newline
21. षो॒ड॒श्य॑ भवद भवथ् षोड॒शी षो॑ड॒श्य॑ भवत् । \newline
22. अ॒भ॒व॒न् न नाभ॑व दभव॒न् न । \newline
23. न वै वै न न वै । \newline
24. वै षो॑ड॒शी षो॑ड॒शी वै वै षो॑ड॒शी । \newline
25. षो॒ड॒शी नाम॒ नाम॑ षोड॒शी षो॑ड॒शी नाम॑ । \newline
26. नाम॑ य॒ज्ञो य॒ज्ञो नाम॒ नाम॑ य॒ज्ञ्ः । \newline
27. य॒ज्ञो᳚ ऽस्त्यस्ति य॒ज्ञो य॒ज्ञो᳚ ऽस्ति । \newline
28. अ॒स्ति॒ यद् यद॑स्त्यस्ति॒ यत् । \newline
29. यद् वाव वाव यद् यद् वाव । \newline
30. वाव षो॑ड॒शꣳ षो॑ड॒शं ॅवाव वाव षो॑ड॒शम् । \newline
31. षो॒ड॒शꣳ स्तो॒त्रꣳ स्तो॒त्रꣳ षो॑ड॒शꣳ षो॑ड॒शꣳ स्तो॒त्रम् । \newline
32. स्तो॒त्रꣳ षो॑ड॒शꣳ षो॑ड॒शꣳ स्तो॒त्रꣳ स्तो॒त्रꣳ षो॑ड॒शम् । \newline
33. षो॒ड॒शꣳ श॒स्त्रꣳ श॒स्त्रꣳ षो॑ड॒शꣳ षो॑ड॒शꣳ श॒स्त्रम् । \newline
34. श॒स्त्रम् तेन॒ तेन॑ श॒स्त्रꣳ श॒स्त्रम् तेन॑ । \newline
35. तेन॑ षोड॒शी षो॑ड॒शी तेन॒ तेन॑ षोड॒शी । \newline
36. षो॒ड॒शी तत् तथ् षो॑ड॒शी षो॑ड॒शी तत् । \newline
37. तथ् षो॑ड॒शिन॑ ष्षोड॒शिन॒ स्तत् तथ् षो॑ड॒शिनः॑ । \newline
38. षो॒ड॒शिन॑ ष्षोडशि॒त्वꣳ षो॑डशि॒त्वꣳ षो॑ड॒शिन॑ ष्षोड॒शिन॑ ष्षोडशि॒त्वम् । \newline
39. षो॒ड॒शि॒त्वं ॅयद् यथ्षो॑डशि॒त्वꣳ षो॑डशि॒त्वं ॅयत् । \newline
40. षो॒ड॒शि॒त्वमिति॑ षोडशि - त्वम् । \newline
41. यथ् षो॑ड॒शी षो॑ड॒शी यद् यथ् षो॑ड॒शी । \newline
42. षो॒ड॒शी गृ॒ह्यते॑ गृ॒ह्यते॑ षोड॒शी षो॑ड॒शी गृ॒ह्यते᳚ । \newline
43. गृ॒ह्यत॑ इन्द्रि॒य मि॑न्द्रि॒यम् गृ॒ह्यते॑ गृ॒ह्यत॑ इन्द्रि॒यम् । \newline
44. इ॒न्द्रि॒य मे॒वैवेन्द्रि॒य मि॑न्द्रि॒य मे॒व । \newline
45. ए॒व तत् तदे॒ वैव तत् । \newline
46. तद् वी॒र्यं॑ ॅवी॒र्य॑म् तत् तद् वी॒र्य᳚म् । \newline
47. वी॒र्यं॑ ॅयज॑मानो॒ यज॑मानो वी॒र्यं॑ ॅवी॒र्यं॑ ॅयज॑मानः । \newline
48. यज॑मान आ॒त्मन् ना॒त्मन्. यज॑मानो॒ यज॑मान आ॒त्मन्न् । \newline
49. आ॒त्मन् ध॑त्ते धत्त आ॒त्मन् ना॒त्मन् ध॑त्ते । \newline
50. ध॒त्ते॒ दे॒वेभ्यो॑ दे॒वेभ्यो॑ धत्ते धत्ते दे॒वेभ्यः॑ । \newline
51. दे॒वेभ्यो॒ वै वै दे॒वेभ्यो॑ दे॒वेभ्यो॒ वै । \newline
52. वै सु॑व॒र्गः सु॑व॒र्गो वै वै सु॑व॒र्गः । \newline
53. सु॒व॒र्गो लो॒को लो॒कः सु॑व॒र्गः सु॑व॒र्गो लो॒कः । \newline
54. सु॒व॒र्ग इति॑ सुवः - गः । \newline
55. लो॒को न न लो॒को लो॒को न । \newline

\textbf{Ghana Paata } \newline

1. प्र॒जाप॑तिर् दे॒वेभ्यो॑ दे॒वेभ्यः॑ प्र॒जाप॑तिः प्र॒जाप॑तिर् दे॒वेभ्यो॑ य॒ज्ञान्. य॒ज्ञान् दे॒वेभ्यः॑ प्र॒जाप॑तिः प्र॒जाप॑तिर् दे॒वेभ्यो॑ य॒ज्ञान् । \newline
2. प्र॒जाप॑ति॒रिति॑ प्र॒जा - प॒तिः॒ । \newline
3. दे॒वेभ्यो॑ य॒ज्ञान्. य॒ज्ञान् दे॒वेभ्यो॑ दे॒वेभ्यो॑ य॒ज्ञान् व्यादि॑श॒द् व्यादि॑शद् य॒ज्ञान् दे॒वेभ्यो॑ दे॒वेभ्यो॑ य॒ज्ञान् व्यादि॑शत् । \newline
4. य॒ज्ञान् व्यादि॑श॒द् व्यादि॑शद् य॒ज्ञान्. य॒ज्ञान् व्यादि॑श॒थ् स स व्यादि॑शद् य॒ज्ञान्. य॒ज्ञान् व्यादि॑श॒थ् सः । \newline
5. व्यादि॑श॒थ् स स व्यादि॑श॒द् व्यादि॑श॒थ् स रि॑रिचा॒नो रि॑रिचा॒नः स व्यादि॑श॒द् व्यादि॑श॒थ् स रि॑रिचा॒नः । \newline
6. व्यादि॑श॒दिति॑ वि - आदि॑शत् । \newline
7. स रि॑रिचा॒नो रि॑रिचा॒नः स स रि॑रिचा॒नो॑ ऽमन्यता मन्यत रिरिचा॒नः स स रि॑रिचा॒नो॑ ऽमन्यत । \newline
8. रि॒रि॒चा॒नो॑ ऽमन्यता मन्यत रिरिचा॒नो रि॑रिचा॒नो॑ ऽमन्यत॒ स सो॑ ऽमन्यत रिरिचा॒नो रि॑रिचा॒नो॑ ऽमन्यत॒ सः । \newline
9. अ॒म॒न्य॒त॒ स सो॑ ऽमन्यता मन्यत॒ स य॒ज्ञानां᳚ ॅय॒ज्ञानाꣳ॒॒ सो॑ ऽमन्यता मन्यत॒ स य॒ज्ञाना᳚म् । \newline
10. स य॒ज्ञानां᳚ ॅय॒ज्ञानाꣳ॒॒ स स य॒ज्ञानाꣳ॑ षोडश॒धा षो॑डश॒धा य॒ज्ञानाꣳ॒॒ स स य॒ज्ञानाꣳ॑ षोडश॒धा । \newline
11. य॒ज्ञानाꣳ॑ षोडश॒धा षो॑डश॒धा य॒ज्ञानां᳚ ॅय॒ज्ञानाꣳ॑ षोडश॒ धेन्द्रि॒य मि॑न्द्रि॒यꣳ षो॑डश॒धा य॒ज्ञानां᳚ ॅय॒ज्ञानाꣳ॑ षोडश॒ धेन्द्रि॒यम् । \newline
12. षो॒ड॒श॒ धेन्द्रि॒य मि॑न्द्रि॒यꣳ षो॑डश॒धा षो॑डश॒ धेन्द्रि॒यं ॅवी॒र्यं॑ ॅवी॒र्य॑ मिन्द्रि॒यꣳ षो॑डश॒धा षो॑डश॒ धेन्द्रि॒यं ॅवी॒र्य᳚म् । \newline
13. षो॒ड॒श॒धेति॑ षोडश - धा । \newline
14. इ॒न्द्रि॒यं ॅवी॒र्यं॑ ॅवी॒र्य॑ मिन्द्रि॒य मि॑न्द्रि॒यं ॅवी॒र्य॑ मा॒त्मान॑ मा॒त्मानं॑ ॅवी॒र्य॑ मिन्द्रि॒य मि॑न्द्रि॒यं ॅवी॒र्य॑ मा॒त्मान᳚म् । \newline
15. वी॒र्य॑ मा॒त्मान॑ मा॒त्मानं॑ ॅवी॒र्यं॑ ॅवी॒र्य॑ मा॒त्मान॑ म॒भ्या᳚(1॒) भ्या᳚त्मानं॑ ॅवी॒र्यं॑ ॅवी॒र्य॑ मा॒त्मान॑ म॒भि । \newline
16. आ॒त्मान॑ म॒भ्या᳚(1॒) भ्या᳚त्मान॑ मा॒त्मान॑ म॒भि सꣳ स म॒भ्या᳚ त्मान॑ मा॒त्मान॑ म॒भि सम् । \newline
17. अ॒भि सꣳ स म॒भ्य॑भि स म॑क्खिद दक्खिद॒थ् स म॒भ्य॑भि स म॑क्खिदत् । \newline
18. स म॑क्खिद दक्खिद॒थ् सꣳ स म॑क्खिद॒त् तत् तद॑क्खिद॒थ् सꣳ स म॑क्खिद॒त् तत् । \newline
19. अ॒क्खि॒द॒त् तत् तद॑क्खिद दक्खिद॒त् तथ्षो॑ड॒शी षो॑ड॒शी तद॑क्खिद दक्खिद॒त् तथ्षो॑ड॒शी । \newline
20. तथ्षो॑ड॒शी षो॑ड॒शी तत् तथ्षो॑ड॒श्य॑भव दभवथ् षोड॒शी तत् तथ्षो॑ड॒श्य॑भवत् । \newline
21. षो॒ड॒ श्य॑भव दभवथ् षोड॒शी षो॑ड॒ श्य॑भव॒न् न नाभ॑वथ् षोड॒शी षो॑ड॒ श्य॑भव॒न् न । \newline
22. अ॒भ॒व॒न् न नाभ॑व दभव॒न् न वै वै नाभ॑व दभव॒न् न वै । \newline
23. न वै वै न न वै षो॑ड॒शी षो॑ड॒शी वै न न वै षो॑ड॒शी । \newline
24. वै षो॑ड॒शी षो॑ड॒शी वै वै षो॑ड॒शी नाम॒ नाम॑ षोड॒शी वै वै षो॑ड॒शी नाम॑ । \newline
25. षो॒ड॒शी नाम॒ नाम॑ षोड॒शी षो॑ड॒शी नाम॑ य॒ज्ञो य॒ज्ञो नाम॑ षोड॒शी षो॑ड॒शी नाम॑ य॒ज्ञ्ः । \newline
26. नाम॑ य॒ज्ञो य॒ज्ञो नाम॒ नाम॑ य॒ज्ञो᳚ ऽस्त्यस्ति य॒ज्ञो नाम॒ नाम॑ य॒ज्ञो᳚ ऽस्ति । \newline
27. य॒ज्ञो᳚ ऽस्त्यस्ति य॒ज्ञो य॒ज्ञो᳚ ऽस्ति॒ यद् यद॑स्ति य॒ज्ञो य॒ज्ञो᳚ ऽस्ति॒ यत् । \newline
28. अ॒स्ति॒ यद् यद॑ स्त्यस्ति॒ यद् वाव वाव यद॑ स्त्यस्ति॒ यद् वाव । \newline
29. यद् वाव वाव यद् यद् वाव षो॑ड॒शꣳ षो॑ड॒शं ॅवाव यद् यद् वाव षो॑ड॒शम् । \newline
30. वाव षो॑ड॒शꣳ षो॑ड॒शं ॅवाव वाव षो॑ड॒शꣳ स्तो॒त्रꣳ स्तो॒त्रꣳ षो॑ड॒शं ॅवाव वाव षो॑ड॒शꣳ स्तो॒त्रम् । \newline
31. षो॒ड॒शꣳ स्तो॒त्रꣳ स्तो॒त्रꣳ षो॑ड॒शꣳ षो॑ड॒शꣳ स्तो॒त्रꣳ षो॑ड॒शꣳ षो॑ड॒शꣳ स्तो॒त्रꣳ षो॑ड॒शꣳ षो॑ड॒शꣳ स्तो॒त्रꣳ षो॑ड॒शम् । \newline
32. स्तो॒त्रꣳ षो॑ड॒शꣳ षो॑ड॒शꣳ स्तो॒त्रꣳ स्तो॒त्रꣳ षो॑ड॒शꣳ श॒स्त्रꣳ श॒स्त्रꣳ षो॑ड॒शꣳ स्तो॒त्रꣳ स्तो॒त्रꣳ षो॑ड॒शꣳ श॒स्त्रम् । \newline
33. षो॒ड॒शꣳ श॒स्त्रꣳ श॒स्त्रꣳ षो॑ड॒शꣳ षो॑ड॒शꣳ श॒स्त्रम् तेन॒ तेन॑ श॒स्त्रꣳ षो॑ड॒शꣳ षो॑ड॒शꣳ श॒स्त्रम् तेन॑ । \newline
34. श॒स्त्रम् तेन॒ तेन॑ श॒स्त्रꣳ श॒स्त्रम् तेन॑ षोड॒शी षो॑ड॒शी तेन॑ श॒स्त्रꣳ श॒स्त्रम् तेन॑ षोड॒शी । \newline
35. तेन॑ षोड॒शी षो॑ड॒शी तेन॒ तेन॑ षोड॒शी तत् तथ् षो॑ड॒शी तेन॒ तेन॑ षोड॒शी तत् । \newline
36. षो॒ड॒शी तत् तथ् षो॑ड॒शी षो॑ड॒शी तथ् षो॑ड॒शिन॑ ष्षोड॒शिन॒ स्तथ् षो॑ड॒शी षो॑ड॒शी तथ् षो॑ड॒शिनः॑ । \newline
37. तथ् षो॑ड॒शिन॑ ष्षोड॒शिन॒ स्तत् तथ् षो॑ड॒शिन॑ ष्षोडशि॒त्वꣳ षो॑डशि॒त्वꣳ षो॑ड॒शिन॒ स्तत् तथ् षो॑ड॒शिन॑ ष्षोडशि॒त्वम् । \newline
38. षो॒ड॒शिन॑ ष्षोडशि॒त्वꣳ षो॑डशि॒त्वꣳ षो॑ड॒शिन॑ ष्षोड॒शिन॑ ष्षोडशि॒त्वं ॅयद् यथ्षो॑डशि॒त्वꣳ षो॑ड॒शिन॑ ष्षोड॒शिन॑ ष्षोडशि॒त्वं ॅयत् । \newline
39. षो॒ड॒शि॒त्वं ॅयद् यथ्षो॑डशि॒त्वꣳ षो॑डशि॒त्वं ॅयथ् षो॑ड॒शी षो॑ड॒शी यथ् षो॑डशि॒त्वꣳ षो॑डशि॒त्वं ॅयथ्षो॑ड॒शी । \newline
40. षो॒ड॒शि॒त्वमिति॑ षोडशि - त्वम् । \newline
41. यथ् षो॑ड॒शी षो॑ड॒शी यद् यथ् षो॑ड॒शी गृ॒ह्यते॑ गृ॒ह्यते॑ षोड॒शी यद् यथ् षो॑ड॒शी गृ॒ह्यते᳚ । \newline
42. षो॒ड॒शी गृ॒ह्यते॑ गृ॒ह्यते॑ षोड॒शी षो॑ड॒शी गृ॒ह्यत॑ इन्द्रि॒य मि॑न्द्रि॒यम् गृ॒ह्यते॑ षोड॒शी षो॑ड॒शी गृ॒ह्यत॑ इन्द्रि॒यम् । \newline
43. गृ॒ह्यत॑ इन्द्रि॒य मि॑न्द्रि॒यम् गृ॒ह्यते॑ गृ॒ह्यत॑ इन्द्रि॒य मे॒वैवेन्द्रि॒यम् गृ॒ह्यते॑ गृ॒ह्यत॑ इन्द्रि॒य मे॒व । \newline
44. इ॒न्द्रि॒य मे॒वैवेन्द्रि॒य मि॑न्द्रि॒य मे॒व तत् तदे॒वेन्द्रि॒य मि॑न्द्रि॒य मे॒व तत् । \newline
45. ए॒व तत् तदे॒ वैव तद् वी॒र्यं॑ ॅवी॒र्य॑म् तदे॒ वैव तद् वी॒र्य᳚म् । \newline
46. तद् वी॒र्यं॑ ॅवी॒र्य॑म् तत् तद् वी॒र्यं॑ ॅयज॑मानो॒ यज॑मानो वी॒र्य॑म् तत् तद् वी॒र्यं॑ ॅयज॑मानः । \newline
47. वी॒र्यं॑ ॅयज॑मानो॒ यज॑मानो वी॒र्यं॑ ॅवी॒र्यं॑ ॅयज॑मान आ॒त्मन् ना॒त्मन्. यज॑मानो वी॒र्यं॑ ॅवी॒र्यं॑ ॅयज॑मान आ॒त्मन्न् । \newline
48. यज॑मान आ॒त्मन् ना॒त्मन्. यज॑मानो॒ यज॑मान आ॒त्मन् ध॑त्ते धत्त आ॒त्मन्. यज॑मानो॒ यज॑मान आ॒त्मन् ध॑त्ते । \newline
49. आ॒त्मन् ध॑त्ते धत्त आ॒त्मन् ना॒त्मन् ध॑त्ते दे॒वेभ्यो॑ दे॒वेभ्यो॑ धत्त आ॒त्मन् ना॒त्मन् ध॑त्ते दे॒वेभ्यः॑ । \newline
50. ध॒त्ते॒ दे॒वेभ्यो॑ दे॒वेभ्यो॑ धत्ते धत्ते दे॒वेभ्यो॒ वै वै दे॒वेभ्यो॑ धत्ते धत्ते दे॒वेभ्यो॒ वै । \newline
51. दे॒वेभ्यो॒ वै वै दे॒वेभ्यो॑ दे॒वेभ्यो॒ वै सु॑व॒र्गः सु॑व॒र्गो वै दे॒वेभ्यो॑ दे॒वेभ्यो॒ वै सु॑व॒र्गः । \newline
52. वै सु॑व॒र्गः सु॑व॒र्गो वै वै सु॑व॒र्गो लो॒को लो॒कः सु॑व॒र्गो वै वै सु॑व॒र्गो लो॒कः । \newline
53. सु॒व॒र्गो लो॒को लो॒कः सु॑व॒र्गः सु॑व॒र्गो लो॒को न न लो॒कः सु॑व॒र्गः सु॑व॒र्गो लो॒को न । \newline
54. सु॒व॒र्ग इति॑ सुवः - गः । \newline
55. लो॒को न न लो॒को लो॒को न प्र प्र ण लो॒को लो॒को न प्र । \newline
\pagebreak
\markright{ TS 6.6.11.2  \hfill https://www.vedavms.in \hfill}

\section{ TS 6.6.11.2 }

\textbf{TS 6.6.11.2 } \newline
\textbf{Samhita Paata} \newline

न प्राभ॑व॒त् त ए॒तꣳ षो॑ड॒शिन॑मपश्य॒न् तम॑गृह्णत॒ ततो॒ वै तेभ्यः॑ सुव॒र्गो लो॒कः प्राभ॑व॒द्यथ् षो॑ड॒शी गृ॒ह्यते॑ सुव॒र्गस्य॑ लो॒कस्या॒भिजि॑त्या॒ इन्द्रो॒ वै दे॒वाना॑मानुजाव॒र आ॑सी॒थ् स प्र॒जाप॑ति॒मुपा॑धाव॒त् तस्मा॑ ए॒तꣳ षो॑ड॒शिनं॒ प्राय॑च्छ॒त् तम॑गृह्णीत॒ ततो॒ वै सोऽग्रं॑ दे॒वता॑नां॒ पर्यै॒द्-यस्यै॒वं ॅवि॒दुषः॑ षोड॒शी गृ॒ह्यते- [  ] \newline

\textbf{Pada Paata} \newline

न । प्रेति॑ । अ॒भ॒व॒त् । ते । ए॒तम् । षो॒ड॒शिन᳚म् । अ॒प॒श्य॒न्न् । तम् । अ॒गृ॒ह्ण॒त॒ । ततः॑ । वै । तेभ्यः॑ । सु॒व॒र्ग इति॑ सुवः - गः । लो॒कः । प्रेति॑ । अ॒भ॒व॒त् । यत् । षो॒ड॒शी । गृ॒ह्यते᳚ । सु॒व॒र्गस्येति॑ सुवः-गस्य॑ । लो॒कस्य॑ । अ॒भिजि॑त्या॒ इत्य॒भि - जि॒त्यै॒ । इन्द्रः॑ । वै । दे॒वाना᳚म् । आ॒नु॒जा॒व॒र इत्या॑नु - जा॒व॒रः । आ॒सी॒त् । सः । प्र॒जाप॑ति॒मिति॑ प्र॒जा - प॒ति॒म् । उपेति॑ । अ॒धा॒व॒त् । तस्मै᳚ । ए॒तम् । षो॒ड॒शिन᳚म् । प्रेति॑ । अ॒य॒च्छ॒त् । तम् । अ॒गृ॒ह्णी॒त॒ । ततः॑ । वै । सः । अग्र᳚म् । दे॒वता॑नाम् । परीति॑ । ऐ॒त् । यस्य॑ । ए॒वम् । वि॒दुषः॑ । षो॒ड॒शी । गृ॒ह्यते᳚ ।  \newline


\textbf{Krama Paata} \newline

न प्र । प्राभ॑वत् । अ॒भ॒व॒त् ते । त ए॒तम् । ए॒तꣳ षो॑ड॒शिन᳚म् । षो॒ड॒शिन॑मपश्यन्न् । अ॒प॒श्य॒न् तम् । तम॑गृह्णत । अ॒गृ॒ह्ण॒त॒ ततः॑ । ततो॒ वै । वै तेभ्यः॑ । तेभ्यः॑ सुव॒र्गः । सु॒व॒र्गो लो॒कः । सु॒व॒र्ग इति॑ सुवः - गः । लो॒कः प्र । प्राभ॑वत् । अ॒भ॒व॒द् यत् । यथ् षो॑ड॒शी । षो॒ड॒शी गृ॒ह्यते᳚ । गृ॒ह्यते॑ सुव॒र्गस्य॑ । सु॒व॒र्गस्य॑ लो॒कस्य॑ । सु॒व॒र्गस्येति॑ सुवः - गस्य॑ । लो॒कस्या॒भिजि॑त्यै । अ॒भिजि॑त्या॒ इन्द्रः॑ । अ॒भिजि॑त्या॒ इत्य॒भि - जि॒त्यै॒ । इन्द्रो॒ वै । वै दे॒वाना᳚म् । दे॒वाना॑मानुजाव॒रः । आ॒नु॒जा॒व॒र आ॑सीत् । आ॒नु॒जा॒व॒र इत्या॑नु - जा॒व॒रः । आ॒सी॒थ् सः । स प्र॒जाप॑तिम् । प्र॒जाप॑ति॒मुप॑ । प्र॒जाप॑ति॒मिति॑ प्र॒जा - प॒ति॒म् । उपा॑धावत् । अ॒धा॒व॒त् तस्मै᳚ । तस्मा॑ ए॒तम् । ए॒तꣳ षो॑ड॒शिन᳚म् । षो॒ड॒शिन॒म् प्र । प्राय॑च्छत् । अ॒य॒च्छ॒त् तम् । तम॑गृह्णीत । अ॒गृ॒ह्णी॒त॒ ततः॑ । ततो॒ वै । वै सः । सोऽग्र᳚म् । अग्र॑म् दे॒वता॑नाम् । दे॒वता॑ना॒म् परि॑ । पर्यै᳚त् । ऐ॒द् यस्य॑ । यस्यै॒वम् । ए॒वम् ॅवि॒दुषः॑ । वि॒दुषः॑ षोड॒शी । षो॒ड॒शी गृ॒ह्यते᳚ । गृ॒ह्यतेऽग्र᳚म् \newline

\textbf{Jatai Paata} \newline

1. न प्र प्र ण न प्र । \newline
2. प्राभ॑व दभव॒त् प्र प्राभ॑वत् । \newline
3. अ॒भ॒व॒त् ते ते॑ ऽभव दभव॒त् ते । \newline
4. त ए॒त मे॒तम् ते त ए॒तम् । \newline
5. ए॒तꣳ षो॑ड॒शिनꣳ॑ षोड॒शिन॑ मे॒त मे॒तꣳ षो॑ड॒शिन᳚म् । \newline
6. षो॒ड॒शिन॑ मपश्यन् नपश्यन् थ्षोड॒शिनꣳ॑ षोड॒शिन॑ मपश्यन्न् । \newline
7. अ॒प॒श्य॒न् तम् त म॑पश्यन् नपश्य॒न् तम् । \newline
8. त म॑गृह्णता गृह्णत॒ तम् त म॑गृह्णत । \newline
9. अ॒गृ॒ह्ण॒त॒ तत॒ स्ततो॑ ऽगृह्णता गृह्णत॒ ततः॑ । \newline
10. ततो॒ वै वै तत॒ स्ततो॒ वै । \newline
11. वै तेभ्य॒ स्तेभ्यो॒ वै वै तेभ्यः॑ । \newline
12. तेभ्यः॑ सुव॒र्गः सु॑व॒र्ग स्तेभ्य॒ स्तेभ्यः॑ सुव॒र्गः । \newline
13. सु॒व॒र्गो लो॒को लो॒कः सु॑व॒र्गः सु॑व॒र्गो लो॒कः । \newline
14. सु॒व॒र्ग इति॑ सुवः - गः । \newline
15. लो॒कः प्र प्र लो॒को लो॒कः प्र । \newline
16. प्राभ॑व दभव॒त् प्र प्राभ॑वत् । \newline
17. अ॒भ॒व॒द् यद् यद॑भव दभव॒द् यत् । \newline
18. यथ्षो॑ड॒शी षो॑ड॒शी यद् यथ्षो॑ड॒शी । \newline
19. षो॒ड॒शी गृ॒ह्यते॑ गृ॒ह्यते॑ षोड॒शी षो॑ड॒शी गृ॒ह्यते᳚ । \newline
20. गृ॒ह्यते॑ सुव॒र्गस्य॑ सुव॒र्गस्य॑ गृ॒ह्यते॑ गृ॒ह्यते॑ सुव॒र्गस्य॑ । \newline
21. सु॒व॒र्गस्य॑ लो॒कस्य॑ लो॒कस्य॑ सुव॒र्गस्य॑ सुव॒र्गस्य॑ लो॒कस्य॑ । \newline
22. सु॒व॒र्गस्येति॑ सुवः - गस्य॑ । \newline
23. लो॒कस्या॒ भिजि॑त्या अ॒भिजि॑त्यै लो॒कस्य॑ लो॒कस्या॒ भिजि॑त्यै । \newline
24. अ॒भिजि॑त्या॒ इन्द्र॒ इन्द्रो॒ ऽभिजि॑त्या अ॒भिजि॑त्या॒ इन्द्रः॑ । \newline
25. अ॒भिजि॑त्या॒ इत्य॒भि - जि॒त्यै॒ । \newline
26. इन्द्रो॒ वै वा इन्द्र॒ इन्द्रो॒ वै । \newline
27. वै दे॒वाना᳚म् दे॒वानां॒ ॅवै वै दे॒वाना᳚म् । \newline
28. दे॒वाना॑ मानुजाव॒र आ॑नुजाव॒रो दे॒वाना᳚म् दे॒वाना॑ मानुजाव॒रः । \newline
29. आ॒नु॒जा॒व॒र आ॑सी दासी दानुजाव॒र आ॑नुजाव॒र आ॑सीत् । \newline
30. आ॒नु॒जा॒व॒र इत्या॑नु - जा॒व॒रः । \newline
31. आ॒सी॒थ् स स आ॑सी दासी॒थ् सः । \newline
32. स प्र॒जाप॑तिम् प्र॒जाप॑तिꣳ॒॒ स स प्र॒जाप॑तिम् । \newline
33. प्र॒जाप॑ति॒ मुपोप॑ प्र॒जाप॑तिम् प्र॒जाप॑ति॒ मुप॑ । \newline
34. प्र॒जाप॑ति॒मिति॑ प्र॒जा - प॒ति॒म् । \newline
35. उपा॑ धावद धाव॒ दुपोपा॑ धावत् । \newline
36. अ॒धा॒व॒त् तस्मै॒ तस्मा॑ अधाव दधाव॒त् तस्मै᳚ । \newline
37. तस्मा॑ ए॒त मे॒तम् तस्मै॒ तस्मा॑ ए॒तम् । \newline
38. ए॒तꣳ षो॑ड॒शिनꣳ॑ षोड॒शिन॑ मे॒त मे॒तꣳ षो॑ड॒शिन᳚म् । \newline
39. षो॒ड॒शिन॒म् प्र प्र षो॑ड॒शिनꣳ॑ षोड॒शिन॒म् प्र । \newline
40. प्राय॑च्छ दयच्छ॒त् प्र प्राय॑च्छत् । \newline
41. अ॒य॒च्छ॒त् तम् त म॑यच्छ दयच्छ॒त् तम् । \newline
42. त म॑गृह्णीता गृह्णीत॒ तम् त म॑गृह्णीत । \newline
43. अ॒गृ॒ह्णी॒त॒ तत॒ स्ततो॑ ऽगृह्णीता गृह्णीत॒ ततः॑ । \newline
44. ततो॒ वै वै तत॒ स्ततो॒ वै । \newline
45. वै स स वै वै सः । \newline
46. सो ऽग्र॒ मग्रꣳ॒॒ स सो ऽग्र᳚म् । \newline
47. अग्र॑म् दे॒वता॑नाम् दे॒वता॑ना॒ मग्र॒ मग्र॑म् दे॒वता॑नाम् । \newline
48. दे॒वता॑ना॒म् परि॒ परि॑ दे॒वता॑नाम् दे॒वता॑ना॒म् परि॑ । \newline
49. पर्यै॑ दै॒त् परि॒ पर्यै᳚त् । \newline
50. ऐ॒द् यस्य॒ यस्यै॑ दै॒द् यस्य॑ । \newline
51. यस्यै॒व मे॒वं ॅयस्य॒ यस्यै॒वम् । \newline
52. ए॒वं ॅवि॒दुषो॑ वि॒दुष॑ ए॒व मे॒वं ॅवि॒दुषः॑ । \newline
53. वि॒दुष॑ ष्षोड॒शी षो॑ड॒शी वि॒दुषो॑ वि॒दुष॑ ष्षोड॒शी । \newline
54. षो॒ड॒शी गृ॒ह्यते॑ गृ॒ह्यते॑ षोड॒शी षो॑ड॒शी गृ॒ह्यते᳚ । \newline
55. गृ॒ह्यते ऽग्र॒ मग्र॑म् गृ॒ह्यते॑ गृ॒ह्यते ऽग्र᳚म् । \newline

\textbf{Ghana Paata } \newline

1. न प्र प्र ण न प्राभ॑व दभव॒त् प्र ण न प्राभ॑वत् । \newline
2. प्राभ॑व दभव॒त् प्र प्राभ॑व॒त् ते ते॑ ऽभव॒त् प्र प्राभ॑व॒त् ते । \newline
3. अ॒भ॒व॒त् ते ते॑ ऽभव दभव॒त् त ए॒त मे॒तम् ते॑ ऽभव दभव॒त् त ए॒तम् । \newline
4. त ए॒त मे॒तम् ते त ए॒तꣳ षो॑ड॒शिनꣳ॑ षोड॒शिन॑ मे॒तम् ते त ए॒तꣳ षो॑ड॒शिन᳚म् । \newline
5. ए॒तꣳ षो॑ड॒शिनꣳ॑ षोड॒शिन॑ मे॒त मे॒तꣳ षो॑ड॒शिन॑ मपश्यन् नपश्यन् थ्षोड॒शिन॑ मे॒त मे॒तꣳ षो॑ड॒शिन॑ मपश्यन्न् । \newline
6. षो॒ड॒शिन॑ मपश्यन् नपश्यन् थ्षोड॒शिनꣳ॑ षोड॒शिन॑ मपश्य॒न् तम् त म॑पश्यन् थ्षोड॒शिनꣳ॑ षोड॒शिन॑ मपश्य॒न् तम् । \newline
7. अ॒प॒श्य॒न् तम् त म॑पश्यन् नपश्य॒न् त म॑गृह्णता गृह्णत॒ त म॑पश्यन् नपश्य॒न् त म॑गृह्णत । \newline
8. त म॑गृह्णता गृह्णत॒ तम् त म॑गृह्णत॒ तत॒ स्ततो॑ ऽगृह्णत॒ तम् त म॑गृह्णत॒ ततः॑ । \newline
9. अ॒गृ॒ह्ण॒त॒ तत॒ स्ततो॑ ऽगृह्णता गृह्णत॒ ततो॒ वै वै ततो॑ ऽगृह्णता गृह्णत॒ ततो॒ वै । \newline
10. ततो॒ वै वै तत॒ स्ततो॒ वै तेभ्य॒ स्तेभ्यो॒ वै तत॒ स्ततो॒ वै तेभ्यः॑ । \newline
11. वै तेभ्य॒ स्तेभ्यो॒ वै वै तेभ्यः॑ सुव॒र्गः सु॑व॒र्ग स्तेभ्यो॒ वै वै तेभ्यः॑ सुव॒र्गः । \newline
12. तेभ्यः॑ सुव॒र्गः सु॑व॒र्ग स्तेभ्य॒ स्तेभ्यः॑ सुव॒र्गो लो॒को लो॒कः सु॑व॒र्ग स्तेभ्य॒ स्तेभ्यः॑ सुव॒र्गो लो॒कः । \newline
13. सु॒व॒र्गो लो॒को लो॒कः सु॑व॒र्गः सु॑व॒र्गो लो॒कः प्र प्र लो॒कः सु॑व॒र्गः सु॑व॒र्गो लो॒कः प्र । \newline
14. सु॒व॒र्ग इति॑ सुवः - गः । \newline
15. लो॒कः प्र प्र लो॒को लो॒कः प्राभ॑व दभव॒त् प्र लो॒को लो॒कः प्राभ॑वत् । \newline
16. प्राभ॑व दभव॒त् प्र प्राभ॑व॒द् यद् यद॑भव॒त् प्र प्राभ॑व॒द् यत् । \newline
17. अ॒भ॒व॒द् यद् यद॑भव दभव॒द् यथ् षो॑ड॒शी षो॑ड॒शी यद॑भव दभव॒द् यथ् षो॑ड॒शी । \newline
18. यथ् षो॑ड॒शी षो॑ड॒शी यद् यथ् षो॑ड॒शी गृ॒ह्यते॑ गृ॒ह्यते॑ षोड॒शी यद् यथ् षो॑ड॒शी गृ॒ह्यते᳚ । \newline
19. षो॒ड॒शी गृ॒ह्यते॑ गृ॒ह्यते॑ षोड॒शी षो॑ड॒शी गृ॒ह्यते॑ सुव॒र्गस्य॑ सुव॒र्गस्य॑ गृ॒ह्यते॑ षोड॒शी षो॑ड॒शी गृ॒ह्यते॑ सुव॒र्गस्य॑ । \newline
20. गृ॒ह्यते॑ सुव॒र्गस्य॑ सुव॒र्गस्य॑ गृ॒ह्यते॑ गृ॒ह्यते॑ सुव॒र्गस्य॑ लो॒कस्य॑ लो॒कस्य॑ सुव॒र्गस्य॑ गृ॒ह्यते॑ गृ॒ह्यते॑ सुव॒र्गस्य॑ लो॒कस्य॑ । \newline
21. सु॒व॒र्गस्य॑ लो॒कस्य॑ लो॒कस्य॑ सुव॒र्गस्य॑ सुव॒र्गस्य॑ लो॒कस्या॒ भिजि॑त्या अ॒भिजि॑त्यै लो॒कस्य॑ सुव॒र्गस्य॑ सुव॒र्गस्य॑ लो॒कस्या॒ भिजि॑त्यै । \newline
22. सु॒व॒र्गस्येति॑ सुवः - गस्य॑ । \newline
23. लो॒कस्या॒ भिजि॑त्या अ॒भिजि॑त्यै लो॒कस्य॑ लो॒कस्या॒ भिजि॑त्या॒ इन्द्र॒ इन्द्रो॒ ऽभिजि॑त्यै लो॒कस्य॑ लो॒कस्या॒ भिजि॑त्या॒ इन्द्रः॑ । \newline
24. अ॒भिजि॑त्या॒ इन्द्र॒ इन्द्रो॒ ऽभिजि॑त्या अ॒भिजि॑त्या॒ इन्द्रो॒ वै वा इन्द्रो॒ ऽभिजि॑त्या अ॒भिजि॑त्या॒ इन्द्रो॒ वै । \newline
25. अ॒भिजि॑त्या॒ इत्य॒भि - जि॒त्यै॒ । \newline
26. इन्द्रो॒ वै वा इन्द्र॒ इन्द्रो॒ वै दे॒वाना᳚म् दे॒वानां॒ ॅवा इन्द्र॒ इन्द्रो॒ वै दे॒वाना᳚म् । \newline
27. वै दे॒वाना᳚म् दे॒वानां॒ ॅवै वै दे॒वाना॑ मानुजाव॒र आ॑नुजाव॒रो दे॒वानां॒ ॅवै वै दे॒वाना॑ मानुजाव॒रः । \newline
28. दे॒वाना॑ मानुजाव॒र आ॑नुजाव॒रो दे॒वाना᳚म् दे॒वाना॑ मानुजाव॒र आ॑सीदा सीदा नुजाव॒रो दे॒वाना᳚म् दे॒वाना॑ मानुजाव॒र आ॑सीत् । \newline
29. आ॒नु॒जा॒व॒र आ॑सी दासी दानुजाव॒र आ॑नुजाव॒र आ॑सी॒थ् स स आ॑सी दानुजाव॒र आ॑नुजाव॒र आ॑सी॒थ् सः । \newline
30. आ॒नु॒जा॒व॒र इत्या॑नु - जा॒व॒रः । \newline
31. आ॒सी॒थ् स स आ॑सी दासी॒थ् स प्र॒जाप॑तिम् प्र॒जाप॑तिꣳ॒॒ स आ॑सी दासी॒थ् स प्र॒जाप॑तिम् । \newline
32. स प्र॒जाप॑तिम् प्र॒जाप॑तिꣳ॒॒ स स प्र॒जाप॑ति॒ मुपोप॑ प्र॒जा॑तिꣳ॒॒ स स प्र॒जाप॑ति॒ मुप॑ । \newline
33. प्र॒जाप॑ति॒ मुपोप॑ प्र॒जाप॑तिम् प्र॒जाप॑ति॒ मुपा॑ धाव दधाव॒ दुप॑ प्र॒जाप॑तिम् प्र॒जाप॑ति॒ मुपा॑धावत् । \newline
34. प्र॒जाप॑ति॒मिति॑ प्र॒जा - प॒ति॒म् । \newline
35. उपा॑धाव दधाव॒ दुपोपा॑ धाव॒त् तस्मै॒ तस्मा॑ अधाव॒ दुपोपा॑ धाव॒त् तस्मै᳚ । \newline
36. अ॒धा॒व॒त् तस्मै॒ तस्मा॑ अधाव दधाव॒त् तस्मा॑ ए॒त मे॒तम् तस्मा॑ अधाव दधाव॒त् तस्मा॑ ए॒तम् । \newline
37. तस्मा॑ ए॒त मे॒तम् तस्मै॒ तस्मा॑ ए॒तꣳ षो॑ड॒शिनꣳ॑ षोड॒शिन॑ मे॒तम् तस्मै॒ तस्मा॑ ए॒तꣳ षो॑ड॒शिन᳚म् । \newline
38. ए॒तꣳ षो॑ड॒शिनꣳ॑ षोड॒शिन॑ मे॒त मे॒तꣳ षो॑ड॒शिन॒म् प्र प्र षो॑ड॒शिन॑ मे॒त मे॒तꣳ षो॑ड॒शिन॒म् प्र । \newline
39. षो॒ड॒शिन॒म् प्र प्र षो॑ड॒शिनꣳ॑ षोड॒शिन॒म् प्राय॑च्छ दयच्छ॒त् प्र षो॑ड॒शिनꣳ॑ षोड॒शिन॒म् प्राय॑च्छत् । \newline
40. प्राय॑च्छ दयच्छ॒त् प्र प्राय॑च्छ॒त् तम् त म॑यच्छ॒त् प्र प्राय॑च्छ॒त् तम् । \newline
41. अ॒य॒च्छ॒त् तम् त म॑यच्छ दयच्छ॒त् त म॑गृह्णीता गृह्णीत॒ त म॑यच्छ दयच्छ॒त् त म॑गृह्णीत । \newline
42. त म॑गृह्णीता गृह्णीत॒ तम् त म॑गृह्णीत॒ तत॒ स्ततो॑ ऽगृह्णीत॒ तम् त म॑गृह्णीत॒ ततः॑ । \newline
43. अ॒गृ॒ह्णी॒त॒ तत॒ स्ततो॑ ऽगृह्णीता गृह्णीत॒ ततो॒ वै वै ततो॑ ऽगृह्णीता गृह्णीत॒ ततो॒ वै । \newline
44. ततो॒ वै वै तत॒ स्ततो॒ वै स स वै तत॒ स्ततो॒ वै सः । \newline
45. वै स स वै वै सो ऽग्र॒ मग्रꣳ॒॒ स वै वै सो ऽग्र᳚म् । \newline
46. सो ऽग्र॒ मग्रꣳ॒॒ स सो ऽग्र॑म् दे॒वता॑नाम् दे॒वता॑ना॒ मग्रꣳ॒॒ स सो ऽग्र॑म् दे॒वता॑नाम् । \newline
47. अग्र॑म् दे॒वता॑नाम् दे॒वता॑ना॒ मग्र॒ मग्र॑म् दे॒वता॑ना॒म् परि॒ परि॑ दे॒वता॑ना॒ मग्र॒ मग्र॑म् दे॒वता॑ना॒म् परि॑ । \newline
48. दे॒वता॑ना॒म् परि॒ परि॑ दे॒वता॑नाम् दे॒वता॑ना॒म् पर्यै॑ दै॒त् परि॑ दे॒वता॑नाम् दे॒वता॑ना॒म् पर्यै᳚त् । \newline
49. पर्यै॑ दै॒त् परि॒ पर्यै॒द् यस्य॒ यस्यै॒त् परि॒ पर्यै॒द् यस्य॑ । \newline
50. ऐ॒द् यस्य॒ यस्यै॑ दै॒द् यस्यै॒व मे॒वं ॅयस्यै॑ दै॒द् यस्यै॒वम् । \newline
51. यस्यै॒व मे॒वं ॅयस्य॒ यस्यै॒वं ॅवि॒दुषो॑ वि॒दुष॑ ए॒वं ॅयस्य॒ यस्यै॒वं ॅवि॒दुषः॑ । \newline
52. ए॒वं ॅवि॒दुषो॑ वि॒दुष॑ ए॒व मे॒वं ॅवि॒दुष॑ ष्षोड॒शी षो॑ड॒शी वि॒दुष॑ ए॒व मे॒वं ॅवि॒दुष॑ ष्षोड॒शी । \newline
53. वि॒दुष॑ ष्षोड॒शी षो॑ड॒शी वि॒दुषो॑ वि॒दुष॑ ष्षोड॒शी गृ॒ह्यते॑ गृ॒ह्यते॑ षोड॒शी वि॒दुषो॑ वि॒दुष॑ ष्षोड॒शी गृ॒ह्यते᳚ । \newline
54. षो॒ड॒शी गृ॒ह्यते॑ गृ॒ह्यते॑ षोड॒शी षो॑ड॒शी गृ॒ह्यते ऽग्र॒ मग्र॑म् गृ॒ह्यते॑ षोड॒शी षो॑ड॒शी गृ॒ह्यते ऽग्र᳚म् । \newline
55. गृ॒ह्यते ऽग्र॒ मग्र॑म् गृ॒ह्यते॑ गृ॒ह्यते ऽग्र॑ मे॒वै वाग्र॑म् गृ॒ह्यते॑ गृ॒ह्यते ऽग्र॑ मे॒व । \newline
\pagebreak
\markright{ TS 6.6.11.3  \hfill https://www.vedavms.in \hfill}

\section{ TS 6.6.11.3 }

\textbf{TS 6.6.11.3 } \newline
\textbf{Samhita Paata} \newline

ऽग्र॑मे॒व स॑मा॒नानां॒ पर्ये॑ति प्रातस्सव॒ने गृ॑ह्णाति॒ वज्रो॒ वै षो॑ड॒शी वज्रः॑ प्रातस्सव॒नꣳ स्वादे॒वैनं॒ ॅयोने॒र्निगृ॑ह्णाति॒ सव॑नेसवने॒ऽभि गृ॑ह्णाति॒ सव॑नाथ्सवनादे॒वैनं॒ प्र ज॑नयति तृतीयसव॒ने प॒शुका॑मस्य गृह्णीया॒द्-वज्रो॒ वै षो॑ड॒शी प॒शव॑स्तृतीयसव॒नं ॅवज्रे॑णै॒वास्मै॑ तृतीयसव॒नात् प॒शूनव॑ रुन्धे॒ नोक्थ्ये॑ गृह्णीयात् प्र॒जा वै प॒शव॑ उ॒क्थानि॒ यदु॒क्थ्ये॑- [  ] \newline

\textbf{Pada Paata} \newline

अग्र᳚म् । ए॒व । स॒मा॒नाना᳚म् । परीति॑ । ए॒ति॒ । प्रा॒त॒स्स॒व॒न इति॑ प्रातः - स॒व॒ने । गृ॒ह्णा॒ति॒ । वज्रः॑ । वै । षो॒ड॒शी । वज्रः॑ । प्रा॒त॒स्स॒व॒नमिति॑ प्रातः - स॒व॒नम् । स्वात् । ए॒व । ए॒न॒म् । योनेः᳚ । निरिति॑ । गृ॒ह्णा॒ति॒ । सव॑नेसवन॒ इति॒ सव॑ने - स॒व॒ने॒ । अ॒भीति॑ । गृ॒ह्णा॒ति॒ । सव॑नाथ्सवना॒दिति॒ सव॑नात् - स॒व॒ना॒त् । ए॒व । ए॒न॒म् । प्रेति॑ । ज॒न॒य॒ति॒ । तृ॒ती॒य॒स॒व॒न इति॑ तृतीय - स॒व॒ने । प॒शुका॑म॒स्येति॑ प॒शु - का॒म॒स्य॒ । गृ॒ह्णी॒या॒त् । वज्रः॑ । वै । षो॒ड॒शी । प॒शवः॑ । तृ॒ती॒य॒स॒व॒नमिति॑ तृतीय - स॒व॒नम् । वज्रे॑ण । ए॒व । अ॒स्मै॒ । तृ॒ती॒य॒स॒व॒नादिति॑ तृतीय - स॒व॒नात् । प॒शून् । अवेति॑ । रु॒न्धे॒ । न । उ॒क्थ्ये᳚ । गृ॒ह्णी॒या॒त् । प्र॒जेति॑ प्र - जा । वै । प॒शवः॑ । उ॒क्थानि॑ । यत् । उ॒क्थ्ये᳚ ।  \newline


\textbf{Krama Paata} \newline

अग्र॑मे॒व । ए॒व स॑मा॒नाना᳚म् । स॒मा॒नाना॒म् परि॑ । पर्ये॑ति । ए॒ति॒ प्रा॒त॒स्स॒व॒ने । प्रा॒त॒स्स॒व॒ने गृ॑ह्णाति । प्रा॒त॒स्स॒व॒न इति॑ प्रातः - स॒व॒ने । गृ॒ह्णा॒ति॒ वज्रः॑ । वज्रो॒ वै । वै षो॑ड॒शी । षो॒ड॒शी वज्रः॑ । वज्रः॑ प्रातस्सव॒नम् । प्रा॒त॒स्स॒व॒नꣳ स्वात् । प्रा॒त॒स्स॒व॒नमिति॑ प्रातः - स॒व॒नम् । स्वादे॒व । ए॒वैन᳚म् । ए॒न॒म् ॅयोनेः᳚ । योने॒र् निः । निर् गृ॑ह्णाति । गृ॒ह्णा॒ति॒ सव॑नेसवने । सव॑नेसवने॒ऽभि । सव॑नेसवन॒ इति॒ सव॑ने - स॒व॒ने॒ । अ॒भि गृ॑ह्णाति । गृ॒ह्णा॒ति॒ सव॑नाथ्सवनात् । सव॑नाथ्सवनादे॒व । सव॑नाथ्सवना॒दिति॒ सव॑नात् - स॒व॒ना॒त्॒ । ए॒वैन᳚म् । ए॒न॒म् प्र । प्र ज॑नयति । ज॒न॒य॒ति॒ तृ॒ती॒य॒स॒व॒ने । तृ॒ती॒य॒स॒व॒ने प॒शुका॑मस्य । तृ॒ती॒य॒स॒व॒न इति॑ तृतीय - स॒व॒ने । प॒शुका॑मस्य गृह्णीयात् । प॒शुका॑म॒स्येति॑ प॒शु - का॒म॒स्य॒ । गृ॒ह्णी॒या॒द् वज्रः॑ । वज्रो॒ वै । वै षो॑ड॒शी । षो॒ड॒शी प॒शवः॑ । प॒शव॑स्तृतीयसव॒नम् । तृ॒ती॒य॒स॒व॒नम् ॅवज्रे॑ण । तृ॒ती॒य॒स॒व॒नमिति॑ तृतीय - स॒व॒नम् । वज्रे॑णै॒व । ए॒वास्मै᳚ । अ॒स्मै॒ तृ॒ती॒य॒स॒व॒नात् । तृ॒ती॒य॒स॒व॒नात् प॒शून् । तृ॒ती॒य॒स॒व॒नादिति॑ तृतीय - स॒व॒नात् । प॒शूनव॑ । अव॑ रुन्धे । रु॒न्धे॒ न । नोक्थ्ये᳚ । उ॒क्थ्ये॑ गृह्णीयात् । गृ॒ह्णी॒या॒त् प्र॒जा । प्र॒जा वै । प्र॒जेति॑ प्र - जा । वै प॒शवः॑ । प॒शव॑ उ॒क्थानि॑ । उ॒क्थानि॒ यत् । यदु॒क्थ्ये᳚ । उ॒क्थ्ये॑ गृह्णी॒यात् \newline

\textbf{Jatai Paata} \newline

1. अग्र॑ मे॒वै वाग्र॒ मग्र॑ मे॒व । \newline
2. ए॒व स॑मा॒नानाꣳ॑ समा॒नाना॑ मे॒वैव स॑मा॒नाना᳚म् । \newline
3. स॒मा॒नाना॒म् परि॒ परि॑ समा॒नानाꣳ॑ समा॒नाना॒म् परि॑ । \newline
4. पर्ये᳚ त्येति॒ परि॒ पर्ये॑ति । \newline
5. ए॒ति॒ प्रा॒त॒स्स॒व॒ने प्रा॑तस्सव॒न ए᳚त्येति प्रातस्सव॒ने । \newline
6. प्रा॒त॒स्स॒व॒ने गृ॑ह्णाति गृह्णाति प्रातस्सव॒ने प्रा॑तस्सव॒ने गृ॑ह्णाति । \newline
7. प्रा॒त॒स्स॒व॒न इति॑ प्रातः - स॒व॒ने । \newline
8. गृ॒ह्णा॒ति॒ वज्रो॒ वज्रो॑ गृह्णाति गृह्णाति॒ वज्रः॑ । \newline
9. वज्रो॒ वै वै वज्रो॒ वज्रो॒ वै । \newline
10. वै षो॑ड॒शी षो॑ड॒शी वै वै षो॑ड॒शी । \newline
11. षो॒ड॒शी वज्रो॒ वज्र॑ ष्षोड॒शी षो॑ड॒शी वज्रः॑ । \newline
12. वज्रः॑ प्रातस्सव॒नम् प्रा॑तस्सव॒नं ॅवज्रो॒ वज्रः॑ प्रातस्सव॒नम् । \newline
13. प्रा॒त॒स्स॒व॒नꣳ स्वाथ् स्वात् प्रा॑तस्सव॒नम् प्रा॑तस्सव॒नꣳ स्वात् । \newline
14. प्रा॒त॒स्स॒व॒नमिति॑ प्रातः - स॒व॒नम् । \newline
15. स्वा दे॒वैव स्वाथ् स्वा दे॒व । \newline
16. ए॒वैन॑ मेन मे॒वै वैन᳚म् । \newline
17. ए॒नं॒ ॅयोने॒र् योने॑ रेन मेनं॒ ॅयोनेः᳚ । \newline
18. योने॒र् निर् णिर् योने॒र् योने॒र् निः । \newline
19. निर् गृ॑ह्णाति गृह्णाति॒ निर् णिर् गृ॑ह्णाति । \newline
20. गृ॒ह्णा॒ति॒ सव॑नेसवने॒ सव॑नेसवने गृह्णाति गृह्णाति॒ सव॑नेसवने । \newline
21. सव॑नेसवने॒ ऽभ्य॑भि सव॑नेसवने॒ सव॑नेसवने॒ ऽभि । \newline
22. सव॑नेसवन॒ इति॒ सव॑ने - स॒व॒ने॒ । \newline
23. अ॒भि गृ॑ह्णाति गृह्णा त्य॒भ्य॑भि गृ॑ह्णाति । \newline
24. गृ॒ह्णा॒ति॒ सव॑नाथ्सवना॒थ् सव॑नाथ्सवनाद् गृह्णाति गृह्णाति॒ सव॑नाथ्सवनात् । \newline
25. सव॑नाथ्सवना दे॒वैव सव॑नाथ्सवना॒थ् सव॑नाथ्सवना दे॒व । \newline
26. सव॑नाथ्सवना॒दिति॒ सव॑नात् - स॒व॒ना॒त् । \newline
27. ए॒वैन॑ मेन मे॒वै वैन᳚म् । \newline
28. ए॒न॒म् प्र प्रैन॑ मेन॒म् प्र । \newline
29. प्र ज॑नयति जनयति॒ प्र प्र ज॑नयति । \newline
30. ज॒न॒य॒ति॒ तृ॒ती॒य॒स॒व॒ने तृ॑तीयसव॒ने ज॑नयति जनयति तृतीयसव॒ने । \newline
31. तृ॒ती॒य॒स॒व॒ने प॒शुका॑मस्य प॒शुका॑मस्य तृतीयसव॒ने तृ॑तीयसव॒ने प॒शुका॑मस्य । \newline
32. तृ॒ती॒य॒स॒व॒न इति॑ तृतीय - स॒व॒ने । \newline
33. प॒शुका॑मस्य गृह्णीयाद् गृह्णीयात् प॒शुका॑मस्य प॒शुका॑मस्य गृह्णीयात् । \newline
34. प॒शुका॑म॒स्येति॑ प॒शु - का॒म॒स्य॒ । \newline
35. गृ॒ह्णी॒या॒द् वज्रो॒ वज्रो॑ गृह्णीयाद् गृह्णीया॒द् वज्रः॑ । \newline
36. वज्रो॒ वै वै वज्रो॒ वज्रो॒ वै । \newline
37. वै षो॑ड॒शी षो॑ड॒शी वै वै षो॑ड॒शी । \newline
38. षो॒ड॒शी प॒शवः॑ प॒शव॑ ष्षोड॒शी षो॑ड॒शी प॒शवः॑ । \newline
39. प॒शव॑ स्तृतीयसव॒नम् तृ॑तीयसव॒नम् प॒शवः॑ प॒शव॑ स्तृतीयसव॒नम् । \newline
40. तृ॒ती॒य॒स॒व॒नं ॅवज्रे॑ण॒ वज्रे॑ण तृतीयसव॒नम् तृ॑तीयसव॒नं ॅवज्रे॑ण । \newline
41. तृ॒ती॒य॒स॒व॒नमिति॑ तृतीय - स॒व॒नम् । \newline
42. वज्रे॑णै॒ वैव वज्रे॑ण॒ वज्रे॑ णै॒व । \newline
43. ए॒वास्मा॑ अस्मा ए॒वै वास्मै᳚ । \newline
44. अ॒स्मै॒ तृ॒ती॒य॒स॒व॒नात् तृ॑तीयसव॒ना द॑स्मा अस्मै तृतीयसव॒नात् । \newline
45. तृ॒ती॒य॒स॒व॒नात् प॒शून् प॒शून् तृ॑तीयसव॒नात् तृ॑तीयसव॒नात् प॒शून् । \newline
46. तृ॒ती॒य॒स॒व॒नादिति॑ तृतीय - स॒व॒नात् । \newline
47. प॒शू नवाव॑ प॒शून् प॒शू नव॑ । \newline
48. अव॑ रुन्धे रु॒न्धे ऽवाव॑ रुन्धे । \newline
49. रु॒न्धे॒ न न रु॑न्धे रुन्धे॒ न । \newline
50. नोक्थ्य॑ उ॒क्थ्ये॑ न नोक्थ्ये᳚ । \newline
51. उ॒क्थ्ये॑ गृह्णीयाद् गृह्णीया दु॒क्थ्य॑ उ॒क्थ्ये॑ गृह्णीयात् । \newline
52. गृ॒ह्णी॒या॒त् प्र॒जा प्र॒जा गृ॑ह्णीयाद् गृह्णीयात् प्र॒जा । \newline
53. प्र॒जा वै वै प्र॒जा प्र॒जा वै । \newline
54. प्र॒जेति॑ प्र - जा । \newline
55. वै प॒शवः॑ प॒शवो॒ वै वै प॒शवः॑ । \newline
56. प॒शव॑ उ॒क्थान् यु॒क्थानि॑ प॒शवः॑ प॒शव॑ उ॒क्थानि॑ । \newline
57. उ॒क्थानि॒ यद् यदु॒क्था न्यु॒क्थानि॒ यत् । \newline
58. यदु॒क्थ्य॑ उ॒क्थ्ये॑ यद् यदु॒क्थ्ये᳚ । \newline
59. उ॒क्थ्ये॑ गृह्णी॒याद् गृ॑ह्णी॒या दु॒क्थ्य॑ उ॒क्थ्ये॑ गृह्णी॒यात् । \newline

\textbf{Ghana Paata } \newline

1. अग्र॑ मे॒वै वाग्र॒ मग्र॑ मे॒व स॑मा॒नानाꣳ॑ समा॒नाना॑ मे॒वाग्र॒ मग्र॑ मे॒व स॑मा॒नाना᳚म् । \newline
2. ए॒व स॑मा॒नानाꣳ॑ समा॒नाना॑ मे॒वैव स॑मा॒नाना॒म् परि॒ परि॑ समा॒नाना॑ मे॒वैव स॑मा॒नाना॒म् परि॑ । \newline
3. स॒मा॒नाना॒म् परि॒ परि॑ समा॒नानाꣳ॑ समा॒नाना॒म् पर्ये᳚ त्येति॒ परि॑ समा॒नानाꣳ॑ समा॒नाना॒म् पर्ये॑ति । \newline
4. पर्ये᳚ त्येति॒ परि॒ पर्ये॑ति प्रातस्सव॒ने प्रा॑तस्सव॒न ए॑ति॒ परि॒ पर्ये॑ति प्रातस्सव॒ने । \newline
5. ए॒ति॒ प्रा॒त॒स्स॒व॒ने प्रा॑तस्सव॒न ए᳚त्येति प्रातस्सव॒ने गृ॑ह्णाति गृह्णाति प्रातस्सव॒न ए᳚त्येति प्रातस्सव॒ने गृ॑ह्णाति । \newline
6. प्रा॒त॒स्स॒व॒ने गृ॑ह्णाति गृह्णाति प्रातस्सव॒ने प्रा॑तस्सव॒ने गृ॑ह्णाति॒ वज्रो॒ वज्रो॑ गृह्णाति प्रातस्सव॒ने प्रा॑तस्सव॒ने गृ॑ह्णाति॒ वज्रः॑ । \newline
7. प्रा॒त॒स्स॒व॒न इति॑ प्रातः - स॒व॒ने । \newline
8. गृ॒ह्णा॒ति॒ वज्रो॒ वज्रो॑ गृह्णाति गृह्णाति॒ वज्रो॒ वै वै वज्रो॑ गृह्णाति गृह्णाति॒ वज्रो॒ वै । \newline
9. वज्रो॒ वै वै वज्रो॒ वज्रो॒ वै षो॑ड॒शी षो॑ड॒शी वै वज्रो॒ वज्रो॒ वै षो॑ड॒शी । \newline
10. वै षो॑ड॒शी षो॑ड॒शी वै वै षो॑ड॒शी वज्रो॒ वज्र॑ ष्षोड॒शी वै वै षो॑ड॒शी वज्रः॑ । \newline
11. षो॒ड॒शी वज्रो॒ वज्र॑ ष्षोड॒शी षो॑ड॒शी वज्रः॑ प्रातस्सव॒नम् प्रा॑तस्सव॒नं ॅवज्र॑ ष्षोड॒शी षो॑ड॒शी वज्रः॑ प्रातस्सव॒नम् । \newline
12. वज्रः॑ प्रातस्सव॒नम् प्रा॑तस्सव॒नं ॅवज्रो॒ वज्रः॑ प्रातस्सव॒नꣳ स्वाथ् स्वात् प्रा॑तस्सव॒नं ॅवज्रो॒ वज्रः॑ प्रातस्सव॒नꣳ स्वात् । \newline
13. प्रा॒त॒स्स॒व॒नꣳ स्वाथ् स्वात् प्रा॑तस्सव॒नम् प्रा॑तस्सव॒नꣳ स्वा दे॒वैव स्वात् प्रा॑तस्सव॒नम् प्रा॑तस्सव॒नꣳ स्वा दे॒व । \newline
14. प्रा॒त॒स्स॒व॒नमिति॑ प्रातः - स॒व॒नम् । \newline
15. स्वा दे॒वैव स्वाथ् स्वा दे॒वैन॑ मेन मे॒व स्वाथ् स्वा दे॒वैन᳚म् । \newline
16. ए॒वैन॑ मेन मे॒वै वैनं॒ ॅयोने॒र् योने॑ रेन मे॒वै वैनं॒ ॅयोनेः᳚ । \newline
17. ए॒नं॒ ॅयोने॒र् योने॑ रेन मेनं॒ ॅयोने॒र् निर् णिर् योने॑ रेन मेनं॒ ॅयोने॒र् निः । \newline
18. योने॒र् निर् णिर् योने॒र् योने॒र् निर् गृ॑ह्णाति गृह्णाति॒ निर् योने॒र् योने॒र् निर् गृ॑ह्णाति । \newline
19. निर् गृ॑ह्णाति गृह्णाति॒ निर् णिर् गृ॑ह्णाति॒ सव॑नेसवने॒ सव॑नेसवने गृह्णाति॒ निर् णिर् गृ॑ह्णाति॒ सव॑नेसवने । \newline
20. गृ॒ह्णा॒ति॒ सव॑नेसवने॒ सव॑नेसवने गृह्णाति गृह्णाति॒ सव॑नेसवने॒ ऽभ्य॑भि सव॑नेसवने गृह्णाति गृह्णाति॒ सव॑नेसवने॒ ऽभि । \newline
21. सव॑नेसवने॒ ऽभ्य॑भि सव॑नेसवने॒ सव॑नेसवने॒ ऽभि गृ॑ह्णाति गृह्णा त्य॒भि सव॑नेसवने॒ सव॑नेसवने॒ ऽभि गृ॑ह्णाति । \newline
22. सव॑नेसवन॒ इति॒ सव॑ने - स॒व॒ने॒ । \newline
23. अ॒भि गृ॑ह्णाति गृह्णा त्य॒भ्य॑भि गृ॑ह्णाति॒ सव॑नाथ्सवना॒थ् सव॑नाथ्सवनाद् गृह्णा त्य॒भ्य॑भि गृ॑ह्णाति॒ सव॑नाथ्सवनात् । \newline
24. गृ॒ह्णा॒ति॒ सव॑नाथ्सवना॒थ् सव॑नाथ्सवनाद् गृह्णाति गृह्णाति॒ सव॑नाथ्सवना दे॒वैव सव॑नाथ्सवनाद् गृह्णाति गृह्णाति॒ सव॑नाथ्सवना दे॒व । \newline
25. सव॑नाथ्सवना दे॒वैव सव॑नाथ्सवना॒थ् सव॑नाथ्सवना दे॒वैन॑ मेन मे॒व सव॑नाथ्सवना॒थ् सव॑नाथ्सवना दे॒वैन᳚म् । \newline
26. सव॑नाथ्सवना॒दिति॒ सव॑नात् - स॒व॒ना॒त् । \newline
27. ए॒वैन॑ मेन मे॒वै वैन॒म् प्र प्रैन॑ मे॒वै वैन॒म् प्र । \newline
28. ए॒न॒म् प्र प्रैन॑ मेन॒म् प्र ज॑नयति जनयति॒ प्रैन॑ मेन॒म् प्र ज॑नयति । \newline
29. प्र ज॑नयति जनयति॒ प्र प्र ज॑नयति तृतीयसव॒ने तृ॑तीयसव॒ने ज॑नयति॒ प्र प्र ज॑नयति तृतीयसव॒ने । \newline
30. ज॒न॒य॒ति॒ तृ॒ती॒य॒स॒व॒ने तृ॑तीयसव॒ने ज॑नयति जनयति तृतीयसव॒ने प॒शुका॑मस्य प॒शुका॑मस्य तृतीयसव॒ने ज॑नयति जनयति तृतीयसव॒ने प॒शुका॑मस्य । \newline
31. तृ॒ती॒य॒स॒व॒ने प॒शुका॑मस्य प॒शुका॑मस्य तृतीयसव॒ने तृ॑तीयसव॒ने प॒शुका॑मस्य गृह्णीयाद् गृह्णीयात् प॒शुका॑मस्य तृतीयसव॒ने तृ॑तीयसव॒ने प॒शुका॑मस्य गृह्णीयात् । \newline
32. तृ॒ती॒य॒स॒व॒न इति॑ तृतीय - स॒व॒ने । \newline
33. प॒शुका॑मस्य गृह्णीयाद् गृह्णीयात् प॒शुका॑मस्य प॒शुका॑मस्य गृह्णीया॒द् वज्रो॒ वज्रो॑ गृह्णीयात् प॒शुका॑मस्य प॒शुका॑मस्य गृह्णीया॒द् वज्रः॑ । \newline
34. प॒शुका॑म॒स्येति॑ प॒शु - का॒म॒स्य॒ । \newline
35. गृ॒ह्णी॒या॒द् वज्रो॒ वज्रो॑ गृह्णीयाद् गृह्णीया॒द् वज्रो॒ वै वै वज्रो॑ गृह्णीयाद् गृह्णीया॒द् वज्रो॒ वै । \newline
36. वज्रो॒ वै वै वज्रो॒ वज्रो॒ वै षो॑ड॒शी षो॑ड॒शी वै वज्रो॒ वज्रो॒ वै षो॑ड॒शी । \newline
37. वै षो॑ड॒शी षो॑ड॒शी वै वै षो॑ड॒शी प॒शवः॑ प॒शव॑ ष्षोड॒शी वै वै षो॑ड॒शी प॒शवः॑ । \newline
38. षो॒ड॒शी प॒शवः॑ प॒शव॑ ष्षोड॒शी षो॑ड॒शी प॒शव॑ स्तृतीयसव॒नम् तृ॑तीयसव॒नम् प॒शव॑ ष्षोड॒शी षो॑ड॒शी प॒शव॑ स्तृतीयसव॒नम् । \newline
39. प॒शव॑ स्तृतीयसव॒नम् तृ॑तीयसव॒नम् प॒शवः॑ प॒शव॑ स्तृतीयसव॒नं ॅवज्रे॑ण॒ वज्रे॑ण तृतीयसव॒नम् प॒शवः॑ प॒शव॑ स्तृतीयसव॒नं ॅवज्रे॑ण । \newline
40. तृ॒ती॒य॒स॒व॒नं ॅवज्रे॑ण॒ वज्रे॑ण तृतीयसव॒नम् तृ॑तीयसव॒नं ॅवज्रे॑णै॒ वैव वज्रे॑ण तृतीयसव॒नम् तृ॑तीयसव॒नं ॅवज्रे॑णै॒व । \newline
41. तृ॒ती॒य॒स॒व॒नमिति॑ तृतीय - स॒व॒नम् । \newline
42. वज्रे॑णै॒वैव वज्रे॑ण॒ वज्रे॑णै॒वास्मा॑ अस्मा ए॒व वज्रे॑ण॒ वज्रे॑णै॒वास्मै᳚ । \newline
43. ए॒वास्मा॑ अस्मा ए॒वै वास्मै॑ तृतीयसव॒नात् तृ॑तीयसव॒ना द॑स्मा ए॒वै वास्मै॑ तृतीयसव॒नात् । \newline
44. अ॒स्मै॒ तृ॒ती॒य॒स॒व॒नात् तृ॑तीयसव॒ना द॑स्मा अस्मै तृतीयसव॒नात् प॒शून् प॒शून् तृ॑तीयसव॒ना द॑स्मा अस्मै तृतीयसव॒नात् प॒शून् । \newline
45. तृ॒ती॒य॒स॒व॒नात् प॒शून् प॒शून् तृ॑तीयसव॒नात् तृ॑तीयसव॒नात् प॒शू नवाव॑ प॒शून् तृ॑तीयसव॒नात् तृ॑तीयसव॒नात् प॒शू नव॑ । \newline
46. तृ॒ती॒य॒स॒व॒नादिति॑ तृतीय - स॒व॒नात् । \newline
47. प॒शू नवाव॑ प॒शून् प॒शू नव॑ रुन्धे रु॒न्धे ऽव॑ प॒शून् प॒शू नव॑ रुन्धे । \newline
48. अव॑ रुन्धे रु॒न्धे ऽवाव॑ रुन्धे॒ न न रु॒न्धे ऽवाव॑ रुन्धे॒ न । \newline
49. रु॒न्धे॒ न न रु॑न्धे रुन्धे॒ नोक्थ्य॑ उ॒क्थ्ये॑ न रु॑न्धे रुन्धे॒ नोक्थ्ये᳚ । \newline
50. नोक्थ्य॑ उ॒क्थ्ये॑ न नोक्थ्ये॑ गृह्णीयाद् गृह्णीया दु॒क्थ्ये॑ न नोक्थ्ये॑ गृह्णीयात् । \newline
51. उ॒क्थ्ये॑ गृह्णीयाद् गृह्णीया दु॒क्थ्य॑ उ॒क्थ्ये॑ गृह्णीयात् प्र॒जा प्र॒जा गृ॑ह्णीया दु॒क्थ्य॑ उ॒क्थ्ये॑ गृह्णीयात् प्र॒जा । \newline
52. गृ॒ह्णी॒या॒त् प्र॒जा प्र॒जा गृ॑ह्णीयाद् गृह्णीयात् प्र॒जा वै वै प्र॒जा गृ॑ह्णीयाद् गृह्णीयात् प्र॒जा वै । \newline
53. प्र॒जा वै वै प्र॒जा प्र॒जा वै प॒शवः॑ प॒शवो॒ वै प्र॒जा प्र॒जा वै प॒शवः॑ । \newline
54. प्र॒जेति॑ प्र - जा । \newline
55. वै प॒शवः॑ प॒शवो॒ वै वै प॒शव॑ उ॒क्था न्यु॒क्थानि॑ प॒शवो॒ वै वै प॒शव॑ उ॒क्थानि॑ । \newline
56. प॒शव॑ उ॒क्था न्यु॒क्थानि॑ प॒शवः॑ प॒शव॑ उ॒क्थानि॒ यद् यदु॒क्थानि॑ प॒शवः॑ प॒शव॑ उ॒क्थानि॒ यत् । \newline
57. उ॒क्थानि॒ यद् यदु॒क्था न्यु॒क्थानि॒ यदु॒क्थ्य॑ उ॒क्थ्ये॑ यदु॒क्था न्यु॒क्थानि॒ यदु॒क्थ्ये᳚ । \newline
58. यदु॒क्थ्य॑ उ॒क्थ्ये॑ यद् यदु॒क्थ्ये॑ गृह्णी॒याद् गृ॑ह्णी॒या दु॒क्थ्ये॑ यद् यदु॒क्थ्ये॑ गृह्णी॒यात् । \newline
59. उ॒क्थ्ये॑ गृह्णी॒याद् गृ॑ह्णी॒या दु॒क्थ्य॑ उ॒क्थ्ये॑ गृह्णी॒यात् प्र॒जाम् प्र॒जाम् गृ॑ह्णी॒या दु॒क्थ्य॑ उ॒क्थ्ये॑ गृह्णी॒यात् प्र॒जाम् । \newline
\pagebreak
\markright{ TS 6.6.11.4  \hfill https://www.vedavms.in \hfill}

\section{ TS 6.6.11.4 }

\textbf{TS 6.6.11.4 } \newline
\textbf{Samhita Paata} \newline

गृह्णी॒यात् प्र॒जां प॒शून॑स्य॒ निर्द॑हेदतिरा॒त्रे प॒शुका॑मस्य गृह्णीया॒द्-वज्रो॒ वै षो॑ड॒शी वज्रे॑णै॒वास्मै॑ प॒शून॑व॒रुद्ध्य॒ रात्रि॑-यो॒परि॑ष्टा-च्छमय॒त्यप्य॑ग्निष्टो॒मे रा॑ज॒न्य॑स्य गृह्णीयाद्-व्या॒वृत्का॑मो॒ हि रा॑ज॒न्यो॑ यज॑ते सा॒ह्न ए॒वास्मै॒ वज्रं॑ गृह्णाति॒ स ए॑नं॒ ॅवज्रो॒ भूत्या॑ इन्धे॒ निर्वा॑ दह-त्येकविꣳ॒॒शꣳ स्तो॒त्रं भ॑वति॒ प्रति॑ष्ठित्यै॒ हरि॑वच्छस्यत॒ इन्द्र॑स्य प्रि॒यं धामो- [  ] \newline

\textbf{Pada Paata} \newline

गृ॒ह्णी॒यात् । प्र॒जामिति॑ प्र - जाम् । प॒शून् । अ॒स्य॒ । निरिति॑ । द॒हे॒त् । अ॒ति॒रा॒त्र इत्य॑ति - रा॒त्रे । प॒शुका॑म॒स्येति॑ प॒शु - का॒म॒स्य॒ । गृ॒ह्णी॒या॒त् । वज्रः॑ । वै । षो॒ड॒शी । वज्रे॑ण । ए॒व । अ॒स्मै॒ । प॒शून् । अ॒व॒रुद्ध्येत्य॑व - रुद्ध्य॑ । रात्रि॑या । उ॒परि॑ष्टात् । श॒म॒य॒ति॒ । अपीति॑ । अ॒ग्नि॒ष्टो॒म इत्य॑ग्नि - स्तो॒मे । रा॒ज॒न्य॑स्य । गृ॒ह्णी॒या॒त् । व्या॒वृत्का॑म॒ इति॑ व्या॒वृत् - का॒मः॒ । हि । रा॒ज॒न्यः॑ । यज॑ते । सा॒ह्न इति॑ स - अ॒ह्ने । ए॒व । अ॒स्मै॒ । वज्र᳚म् । गृ॒ह्णा॒ति॒ । सः । ए॒न॒म् । वज्रः॑ । भूत्यै᳚ । इ॒न्धे॒ । निरिति॑ । वा॒ । द॒ह॒ति॒ । ए॒क॒विꣳ॒॒शमित्ये॑क-विꣳ॒॒शम् । स्तो॒त्रम् । भ॒व॒ति॒ । प्रति॑ष्ठित्या॒ इति॒ प्रति॑ - स्थि॒त्यै॒ । हरि॑व॒दिति॒ हरि॑ - व॒त् । श॒स्य॒ते॒ । इन्द्र॑स्य । प्रि॒यम् । धाम॑ ।  \newline


\textbf{Krama Paata} \newline

गृ॒ह्णी॒यात् प्र॒जाम् । प्र॒जाम् प॒शून् । प्र॒जामिति॑ प्र - जाम् । प॒शून॑स्य । अ॒स्य॒ निः । निर् द॑हेत् । द॒हे॒द॒ति॒रा॒त्रे । अ॒ति॒रा॒त्रे प॒शुका॑मस्य । अ॒ति॒रा॒त्र इत्य॑ति - रा॒त्रे । प॒शुका॑मस्य गृह्णीयात् । प॒शुका॑म॒स्येति॑ प॒शु - का॒म॒स्य॒ । गृ॒ह्णि॒या॒द् वज्रः॑ । वज्रो॒ वै । वै षो॑ड॒शी । षो॒ड॒शी वज्रे॑ण । वज्रे॑णै॒व । ए॒वास्मै᳚ । अ॒स्मै॒ प॒शून् । प॒शून॑व॒रुद्ध्य॑ । अ॒व॒रुद्ध्य॒ रात्रि॑या । अ॒व॒रुद्ध्येत्य॑व - रुद्ध्य॑ । रात्रि॑यो॒परि॑ष्टात् । उ॒परि॑ष्टाच्छमयति । श॒म॒य॒त्यपि॑ । अप्य॑ग्निष्टो॒मे । अ॒ग्नि॒ष्टो॒मे रा॑ज॒न्य॑स्य । अ॒ग्नि॒ष्टो॒म इत्य॑ग्नि - स्तो॒मे । रा॒ज्य॒न्य॑स्य गृह्णीयात् । गृ॒ह्णी॒या॒द् व्या॒वृत्का॑मः । व्या॒वृत्का॑मो॒ हि । व्या॒वृत्का॑म॒ इति॑ व्या॒वृत् - का॒मः॒ । हि रा॑ज॒न्यः॑ । रा॒ज॒न्यो॑ यज॑ते । यज॑ते सा॒ह्ने । सा॒ह्न ए॒व । सा॒ह्न इति॑ स - अ॒ह्ने । ए॒वास्मै᳚ । अ॒स्मै॒ वज्र᳚म् । वज्र॑म् गृह्णाति । गृ॒ह्णा॒ति॒ सः । स ए॑नम् । ए॒न॒म् ॅवज्रः॑ । वज्रो॒ भूत्यै᳚ । भूत्या॑ इन्धे । इ॒न्धे॒ निः । निर् वा᳚ । वा॒ द॒ह॒ति॒ । द॒ह॒त्ये॒क॒विꣳ॒॒शम् । ए॒क॒विꣳ॒॒शꣳ स्तो॒त्रम् । ए॒क॒विꣳ॒॒शमित्ये॑क - विꣳ॒॒शम् । स्तो॒त्रम् भ॑वति । भ॒व॒ति॒ प्रति॑ष्ठित्यै । प्रति॑ष्ठित्यै॒ हरि॑वत् । प्रति॑ष्ठित्या॒ इति॒ प्रति॑ - स्थि॒त्यै॒ । हरि॑वच्छस्यते । हरि॑व॒दिति॒ हरि॑ - व॒त्॒ । श॒स्य॒त॒ इन्द्र॑स्य । इन्द्र॑स्य प्रि॒यम् । प्रि॒यम् धाम॑ । धामोप॑ \newline

\textbf{Jatai Paata} \newline

1. गृ॒ह्णी॒यात् प्र॒जाम् प्र॒जाम् गृ॑ह्णी॒याद् गृ॑ह्णी॒यात् प्र॒जाम् । \newline
2. प्र॒जाम् प॒शून् प॒शून् प्र॒जाम् प्र॒जाम् प॒शून् । \newline
3. प्र॒जामिति॑ प्र - जाम् । \newline
4. प॒शू न॑स्यास्य प॒शून् प॒शू न॑स्य । \newline
5. अ॒स्य॒ निर् णिर॑स्यास्य॒ निः । \newline
6. निर् द॑हेद् दहे॒न् निर् णिर् द॑हेत् । \newline
7. द॒हे॒ द॒ति॒रा॒त्रे॑ ऽतिरा॒त्रे द॑हेद् दहे दतिरा॒त्रे । \newline
8. अ॒ति॒रा॒त्रे प॒शुका॑मस्य प॒शुका॑म स्यातिरा॒त्रे॑ ऽतिरा॒त्रे प॒शुका॑मस्य । \newline
9. अ॒ति॒रा॒त्र इत्य॑ति - रा॒त्रे । \newline
10. प॒शुका॑मस्य गृह्णीयाद् गृह्णीयात् प॒शुका॑मस्य प॒शुका॑मस्य गृह्णीयात् । \newline
11. प॒शुका॑म॒स्येति॑ प॒शु - का॒म॒स्य॒ । \newline
12. गृ॒ह्णी॒या॒द् वज्रो॒ वज्रो॑ गृह्णीयाद् गृह्णीया॒द् वज्रः॑ । \newline
13. वज्रो॒ वै वै वज्रो॒ वज्रो॒ वै । \newline
14. वै षो॑ड॒शी षो॑ड॒शी वै वै षो॑ड॒शी । \newline
15. षो॒ड॒शी वज्रे॑ण॒ वज्रे॑ण षोड॒शी षो॑ड॒शी वज्रे॑ण । \newline
16. वज्रे॑णै॒ वैव वज्रे॑ण॒ वज्रे॑ णै॒व । \newline
17. ए॒वास्मा॑ अस्मा ए॒वै वास्मै᳚ । \newline
18. अ॒स्मै॒ प॒शून् प॒शू न॑स्मा अस्मै प॒शून् । \newline
19. प॒शू न॑व॒रुद्ध्या॑ व॒रुद्ध्य॑ प॒शून् प॒शू न॑व॒रुद्ध्य॑ । \newline
20. अ॒व॒रुद्ध्य॒ रात्रि॑या॒ रात्रि॑या ऽव॒रुद्ध्या॑ व॒रुद्ध्य॒ रात्रि॑या । \newline
21. अ॒व॒रुद्ध्येत्य॑व - रुद्ध्य॑ । \newline
22. रात्रि॑यो॒परि॑ष्टा दु॒परि॑ष्टा॒द् रात्रि॑या॒ रात्रि॑ यो॒परि॑ष्टात् । \newline
23. उ॒परि॑ष्टाच् छमयति शमय त्यु॒परि॑ष्टा दु॒परि॑ष्टाच् छमयति । \newline
24. श॒म॒य॒ त्यप्यपि॑ शमयति शमय॒ त्यपि॑ । \newline
25. अप्य॑ ग्निष्टो॒मे᳚ ऽग्निष्टो॒मे ऽप्यप्य॑ ग्निष्टो॒मे । \newline
26. अ॒ग्नि॒ष्टो॒मे रा॑ज॒न्य॑स्य राज॒न्य॑स्या ग्निष्टो॒मे᳚ ऽग्निष्टो॒मे रा॑ज॒न्य॑स्य । \newline
27. अ॒ग्नि॒ष्टो॒म इत्य॑ग्नि - स्तो॒मे । \newline
28. रा॒ज॒न्य॑स्य गृह्णीयाद् गृह्णीयाद् राज॒न्य॑स्य राज॒न्य॑स्य गृह्णीयात् । \newline
29. गृ॒ह्णी॒या॒द् व्या॒वृत्का॑मो व्या॒वृत्का॑मो गृह्णीयाद् गृह्णीयाद् व्या॒वृत्का॑मः । \newline
30. व्या॒वृत्का॑मो॒ हि हि व्या॒वृत्का॑मो व्या॒वृत्का॑मो॒ हि । \newline
31. व्या॒वृत्का॑म॒ इति॑ व्या॒वृत् - का॒मः॒ । \newline
32. हि रा॑ज॒न्यो॑ राज॒न्यो॑ हि हि रा॑ज॒न्यः॑ । \newline
33. रा॒ज॒न्यो॑ यज॑ते॒ यज॑ते राज॒न्यो॑ राज॒न्यो॑ यज॑ते । \newline
34. यज॑ते सा॒ह्ने सा॒ह्ने यज॑ते॒ यज॑ते सा॒ह्ने । \newline
35. सा॒ह्न ए॒वैव सा॒ह्ने सा॒ह्न ए॒व । \newline
36. सा॒ह्न इति॑ स - अ॒ह्ने । \newline
37. ए॒वास्मा॑ अस्मा ए॒वै वास्मै᳚ । \newline
38. अ॒स्मै॒ वज्रं॒ ॅवज्र॑ मस्मा अस्मै॒ वज्र᳚म् । \newline
39. वज्र॑म् गृह्णाति गृह्णाति॒ वज्रं॒ ॅवज्र॑म् गृह्णाति । \newline
40. गृ॒ह्णा॒ति॒ स स गृ॑ह्णाति गृह्णाति॒ सः । \newline
41. स ए॑न मेनꣳ॒॒ स स ए॑नम् । \newline
42. ए॒नं॒ ॅवज्रो॒ वज्र॑ एन मेनं॒ ॅवज्रः॑ । \newline
43. वज्रो॒ भूत्यै॒ भूत्यै॒ वज्रो॒ वज्रो॒ भूत्यै᳚ । \newline
44. भूत्या॑ इन्ध इन्धे॒ भूत्यै॒ भूत्या॑ इन्धे । \newline
45. इ॒न्धे॒ निर् णिरि॑न्ध इन्धे॒ निः । \newline
46. निर् वा॑ वा॒ निर् णिर् वा᳚ । \newline
47. वा॒ द॒ह॒ति॒ द॒ह॒ति॒ वा॒ वा॒ द॒ह॒ति॒ । \newline
48. द॒ह॒ त्ये॒क॒विꣳ॒॒श मे॑कविꣳ॒॒शम् द॑हति दह त्येकविꣳ॒॒शम् । \newline
49. ए॒क॒विꣳ॒॒शꣳ स्तो॒त्रꣳ स्तो॒त्र मे॑कविꣳ॒॒श मे॑कविꣳ॒॒शꣳ स्तो॒त्रम् । \newline
50. ए॒क॒विꣳ॒॒शमित्ये॑क - विꣳ॒॒शम् । \newline
51. स्तो॒त्रम् भ॑वति भवति स्तो॒त्रꣳ स्तो॒त्रम् भ॑वति । \newline
52. भ॒व॒ति॒ प्रति॑ष्ठित्यै॒ प्रति॑ष्ठित्यै भवति भवति॒ प्रति॑ष्ठित्यै । \newline
53. प्रति॑ष्ठित्यै॒ हरि॑व॒ द्धरि॑व॒त् प्रति॑ष्ठित्यै॒ प्रति॑ष्ठित्यै॒ हरि॑वत् । \newline
54. प्रति॑ष्ठित्या॒ इति॒ प्रति॑ - स्थि॒त्यै॒ । \newline
55. हरि॑वच् छस्यते शस्यते॒ हरि॑व॒ द्धरि॑वच् छस्यते । \newline
56. हरि॑व॒दिति॒ हरि॑ - व॒त् । \newline
57. श॒स्य॒त॒ इन्द्र॒ स्येन्द्र॑स्य शस्यते शस्यत॒ इन्द्र॑स्य । \newline
58. इन्द्र॑स्य प्रि॒यम् प्रि॒य मिन्द्र॒ स्येन्द्र॑स्य प्रि॒यम् । \newline
59. प्रि॒यम् धाम॒ धाम॑ प्रि॒यम् प्रि॒यम् धाम॑ । \newline
60. धामो पोप॒ धाम॒ धामोप॑ । \newline

\textbf{Ghana Paata } \newline

1. गृ॒ह्णी॒यात् प्र॒जाम् प्र॒जाम् गृ॑ह्णी॒याद् गृ॑ह्णी॒यात् प्र॒जाम् प॒शून् प॒शून् प्र॒जाम् गृ॑ह्णी॒याद् गृ॑ह्णी॒यात् प्र॒जाम् प॒शून् । \newline
2. प्र॒जाम् प॒शून् प॒शून् प्र॒जाम् प्र॒जाम् प॒शू न॑स्यास्य प॒शून् प्र॒जाम् प्र॒जाम् प॒शू न॑स्य । \newline
3. प्र॒जामिति॑ प्र - जाम् । \newline
4. प॒शू न॑स्यास्य प॒शून् प॒शू न॑स्य॒ निर् णिर॑स्य प॒शून् प॒शू न॑स्य॒ निः । \newline
5. अ॒स्य॒ निर् णिर॑ स्यास्य॒ निर् द॑हेद् दहे॒न् निर॑ स्यास्य॒ निर् द॑हेत् । \newline
6. निर् द॑हेद् दहे॒न् निर् णिर् द॑हे दतिरा॒त्रे॑ ऽतिरा॒त्रे द॑हे॒न् निर् णिर् द॑हे दतिरा॒त्रे । \newline
7. द॒हे॒ द॒ति॒रा॒त्रे॑ ऽतिरा॒त्रे द॑हेद् दहे दतिरा॒त्रे प॒शुका॑मस्य प॒शुका॑म स्यातिरा॒त्रे द॑हेद् दहे दतिरा॒त्रे प॒शुका॑मस्य । \newline
8. अ॒ति॒रा॒त्रे प॒शुका॑मस्य प॒शुका॑म स्यातिरा॒त्रे॑ ऽतिरा॒त्रे प॒शुका॑मस्य गृह्णीयाद् गृह्णीयात् प॒शुका॑म स्यातिरा॒त्रे॑ ऽतिरा॒त्रे प॒शुका॑मस्य गृह्णीयात् । \newline
9. अ॒ति॒रा॒त्र इत्य॑ति - रा॒त्रे । \newline
10. प॒शुका॑मस्य गृह्णीयाद् गृह्णीयात् प॒शुका॑मस्य प॒शुका॑मस्य गृह्णीया॒द् वज्रो॒ वज्रो॑ गृह्णीयात् प॒शुका॑मस्य प॒शुका॑मस्य गृह्णीया॒द् वज्रः॑ । \newline
11. प॒शुका॑म॒स्येति॑ प॒शु - का॒म॒स्य॒ । \newline
12. गृ॒ह्णी॒या॒द् वज्रो॒ वज्रो॑ गृह्णीयाद् गृह्णीया॒द् वज्रो॒ वै वै वज्रो॑ गृह्णीयाद् गृह्णीया॒द् वज्रो॒ वै । \newline
13. वज्रो॒ वै वै वज्रो॒ वज्रो॒ वै षो॑ड॒शी षो॑ड॒शी वै वज्रो॒ वज्रो॒ वै षो॑ड॒शी । \newline
14. वै षो॑ड॒शी षो॑ड॒शी वै वै षो॑ड॒शी वज्रे॑ण॒ वज्रे॑ण षोड॒शी वै वै षो॑ड॒शी वज्रे॑ण । \newline
15. षो॒ड॒शी वज्रे॑ण॒ वज्रे॑ण षोड॒शी षो॑ड॒शी वज्रे॑णै॒वैव वज्रे॑ण षोड॒शी षो॑ड॒शी वज्रे॑णै॒व । \newline
16. वज्रे॑णै॒वैव वज्रे॑ण॒ वज्रे॑ णै॒वास्मा॑ अस्मा ए॒व वज्रे॑ण॒ वज्रे॑णै॒वास्मै᳚ । \newline
17. ए॒वास्मा॑ अस्मा ए॒वैवास्मै॑ प॒शून् प॒शू न॑स्मा ए॒वैवास्मै॑ प॒शून् । \newline
18. अ॒स्मै॒ प॒शून् प॒शू न॑स्मा अस्मै प॒शू न॑व॒रुद्ध्या॑ व॒रुद्ध्य॑ प॒शू न॑स्मा अस्मै प॒शू न॑व॒रुद्ध्य॑ । \newline
19. प॒शू न॑व॒रुद्ध्या॑ व॒रुद्ध्य॑ प॒शून् प॒शू न॑व॒रुद्ध्य॒ रात्रि॑या॒ रात्रि॑या ऽव॒रुद्ध्य॑ प॒शून् प॒शू
न॑व॒रुद्ध्य॒ रात्रि॑या । \newline
20. अ॒व॒रुद्ध्य॒ रात्रि॑या॒ रात्रि॑या ऽव॒रुद्ध्या॑ व॒रुद्ध्य॒ रात्रि॑ यो॒परि॑ष्टा दु॒परि॑ष्टा॒द् रात्रि॑या ऽव॒रुद्ध्या॑ व॒रुद्ध्य॒ रात्रि॑ यो॒परि॑ष्टात् । \newline
21. अ॒व॒रुद्ध्येत्य॑व - रुद्ध्य॑ । \newline
22. रात्रि॑ यो॒परि॑ष्टा दु॒परि॑ष्टा॒द् रात्रि॑या॒ रात्रि॑ यो॒परि॑ष्टाच् छमयति शमय त्यु॒परि॑ष्टा॒द् रात्रि॑या॒ रात्रि॑
यो॒परि॑ष्टाच् छमयति । \newline
23. उ॒परि॑ष्टाच् छमयति शमय त्यु॒परि॑ष्टा दु॒परि॑ष्टाच् छमय॒ त्यप्यपि॑ शमय त्यु॒परि॑ष्टा दु॒परि॑ष्टाच् छमय॒ त्यपि॑ । \newline
24. श॒म॒य॒ त्यप्यपि॑ शमयति शमय॒ त्यप्य॑ ग्निष्टो॒मे᳚ ऽग्निष्टो॒मे ऽपि॑ शमयति शमय॒ त्यप्य॑ ग्निष्टो॒मे । \newline
25. अप्य॑ ग्निष्टो॒मे᳚ ऽग्निष्टो॒मे ऽप्यप्य॑ ग्निष्टो॒मे रा॑ज॒न्य॑स्य राज॒न्य॑स्या ग्निष्टो॒मे ऽप्यप्य॑ ग्निष्टो॒मे रा॑ज॒न्य॑स्य । \newline
26. अ॒ग्नि॒ष्टो॒मे रा॑ज॒न्य॑स्य राज॒न्य॑स्या ग्निष्टो॒मे᳚ ऽग्निष्टो॒मे रा॑ज॒न्य॑स्य गृह्णीयाद् गृह्णीयाद् राज॒न्य॑स्या ग्निष्टो॒मे᳚ ऽग्निष्टो॒मे रा॑ज॒न्य॑स्य गृह्णीयात् । \newline
27. अ॒ग्नि॒ष्टो॒म इत्य॑ग्नि - स्तो॒मे । \newline
28. रा॒ज॒न्य॑स्य गृह्णीयाद् गृह्णीयाद् राज॒न्य॑स्य राज॒न्य॑स्य गृह्णीयाद् व्या॒वृत्का॑मो व्या॒वृत्का॑मो गृह्णीयाद् राज॒न्य॑स्य राज॒न्य॑स्य गृह्णीयाद् व्या॒वृत्का॑मः । \newline
29. गृ॒ह्णी॒या॒द् व्या॒वृत्का॑मो व्या॒वृत्का॑मो गृह्णीयाद् गृह्णीयाद् व्या॒वृत्का॑मो॒ हि हि व्या॒वृत्का॑मो गृह्णीयाद् गृह्णीयाद् व्या॒वृत्का॑मो॒ हि । \newline
30. व्या॒वृत्का॑मो॒ हि हि व्या॒वृत्का॑मो व्या॒वृत्का॑मो॒ हि रा॑ज॒न्यो॑ राज॒न्यो॑ हि व्या॒वृत्का॑मो व्या॒वृत्का॑मो॒ हि रा॑ज॒न्यः॑ । \newline
31. व्या॒वृत्का॑म॒ इति॑ व्या॒वृत् - का॒मः॒ । \newline
32. हि रा॑ज॒न्यो॑ राज॒न्यो॑ हि हि रा॑ज॒न्यो॑ यज॑ते॒ यज॑ते राज॒न्यो॑ हि हि रा॑ज॒न्यो॑ यज॑ते । \newline
33. रा॒ज॒न्यो॑ यज॑ते॒ यज॑ते राज॒न्यो॑ राज॒न्यो॑ यज॑ते सा॒ह्ने सा॒ह्ने यज॑ते राज॒न्यो॑ राज॒न्यो॑ यज॑ते सा॒ह्ने । \newline
34. यज॑ते सा॒ह्ने सा॒ह्ने यज॑ते॒ यज॑ते सा॒ह्न ए॒वैव सा॒ह्ने यज॑ते॒ यज॑ते सा॒ह्न ए॒व । \newline
35. सा॒ह्न ए॒वैव सा॒ह्ने सा॒ह्न ए॒वास्मा॑ अस्मा ए॒व सा॒ह्ने सा॒ह्न ए॒वास्मै᳚ । \newline
36. सा॒ह्न इति॑ स - अ॒ह्ने । \newline
37. ए॒वास्मा॑ अस्मा ए॒वैवास्मै॒ वज्रं॒ ॅवज्र॑ मस्मा ए॒वैवास्मै॒ वज्र᳚म् । \newline
38. अ॒स्मै॒ वज्रं॒ ॅवज्र॑ मस्मा अस्मै॒ वज्र॑म् गृह्णाति गृह्णाति॒ वज्र॑ मस्मा अस्मै॒ वज्र॑म् गृह्णाति । \newline
39. वज्र॑म् गृह्णाति गृह्णाति॒ वज्रं॒ ॅवज्र॑म् गृह्णाति॒ स स गृ॑ह्णाति॒ वज्रं॒ ॅवज्र॑म् गृह्णाति॒ सः । \newline
40. गृ॒ह्णा॒ति॒ स स गृ॑ह्णाति गृह्णाति॒ स ए॑न मेनꣳ॒॒ स गृ॑ह्णाति गृह्णाति॒ स ए॑नम् । \newline
41. स ए॑न मेनꣳ॒॒ स स ए॑नं॒ ॅवज्रो॒ वज्र॑ एनꣳ॒॒ स स ए॑नं॒ ॅवज्रः॑ । \newline
42. ए॒नं॒ ॅवज्रो॒ वज्र॑ एन मेनं॒ ॅवज्रो॒ भूत्यै॒ भूत्यै॒ वज्र॑ एन मेनं॒ ॅवज्रो॒ भूत्यै᳚ । \newline
43. वज्रो॒ भूत्यै॒ भूत्यै॒ वज्रो॒ वज्रो॒ भूत्या॑ इन्ध इन्धे॒ भूत्यै॒ वज्रो॒ वज्रो॒ भूत्या॑ इन्धे । \newline
44. भूत्या॑ इन्ध इन्धे॒ भूत्यै॒ भूत्या॑ इन्धे॒ निर् णिरि॑न्धे॒ भूत्यै॒ भूत्या॑ इन्धे॒ निः । \newline
45. इ॒न्धे॒ निर् णिरि॑न्ध इन्धे॒ निर् वा॑ वा॒ निरि॑न्ध इन्धे॒ निर् वा᳚ । \newline
46. निर् वा॑ वा॒ निर् णिर् वा॑ दहति दहति वा॒ निर् णिर् वा॑ दहति । \newline
47. वा॒ द॒ह॒ति॒ द॒ह॒ति॒ वा॒ वा॒ द॒ह॒ त्ये॒क॒विꣳ॒॒श मे॑कविꣳ॒॒शम् द॑हति वा वा दह त्येकविꣳ॒॒शम् । \newline
48. द॒ह॒ त्ये॒क॒विꣳ॒॒श मे॑कविꣳ॒॒शम् द॑हति दह त्येकविꣳ॒॒शꣳ स्तो॒त्रꣳ स्तो॒त्र मे॑कविꣳ॒॒शम् द॑हति दह त्येकविꣳ॒॒शꣳ स्तो॒त्रम् । \newline
49. ए॒क॒विꣳ॒॒शꣳ स्तो॒त्रꣳ स्तो॒त्र मे॑कविꣳ॒॒श मे॑कविꣳ॒॒शꣳ स्तो॒त्रम् भ॑वति भवति स्तो॒त्र मे॑कविꣳ॒॒श मे॑कविꣳ॒॒शꣳ स्तो॒त्रम् भ॑वति । \newline
50. ए॒क॒विꣳ॒॒शमित्ये॑क - विꣳ॒॒शम् । \newline
51. स्तो॒त्रम् भ॑वति भवति स्तो॒त्रꣳ स्तो॒त्रम् भ॑वति॒ प्रति॑ष्ठित्यै॒ प्रति॑ष्ठित्यै भवति स्तो॒त्रꣳ स्तो॒त्रम् भ॑वति॒ प्रति॑ष्ठित्यै । \newline
52. भ॒व॒ति॒ प्रति॑ष्ठित्यै॒ प्रति॑ष्ठित्यै भवति भवति॒ प्रति॑ष्ठित्यै॒ हरि॑व॒ द्धरि॑व॒त् प्रति॑ष्ठित्यै भवति भवति॒ प्रति॑ष्ठित्यै॒ हरि॑वत् । \newline
53. प्रति॑ष्ठित्यै॒ हरि॑व॒ द्धरि॑व॒त् प्रति॑ष्ठित्यै॒ प्रति॑ष्ठित्यै॒ हरि॑वच् छस्यते शस्यते॒ हरि॑व॒त् प्रति॑ष्ठित्यै॒ प्रति॑ष्ठित्यै॒ हरि॑वच् छस्यते । \newline
54. प्रति॑ष्ठित्या॒ इति॒ प्रति॑ - स्थि॒त्यै॒ । \newline
55. हरि॑वच् छस्यते शस्यते॒ हरि॑व॒ द्धरि॑वच् छस्यत॒ इन्द्र॒ स्येन्द्र॑स्य शस्यते॒ हरि॑व॒ द्धरि॑वच् छस्यत॒ इन्द्र॑स्य । \newline
56. हरि॑व॒दिति॒ हरि॑ - व॒त् । \newline
57. श॒स्य॒त॒ इन्द्र॒ स्येन्द्र॑स्य शस्यते शस्यत॒ इन्द्र॑स्य प्रि॒यम् प्रि॒य मिन्द्र॑स्य शस्यते शस्यत॒ इन्द्र॑स्य प्रि॒यम् । \newline
58. इन्द्र॑स्य प्रि॒यम् प्रि॒य मिन्द्र॒ स्येन्द्र॑स्य प्रि॒यम् धाम॒ धाम॑ प्रि॒य मिन्द्र॒ स्येन्द्र॑स्य प्रि॒यम् धाम॑ । \newline
59. प्रि॒यम् धाम॒ धाम॑ प्रि॒यम् प्रि॒यम् धामो पोप॒ धाम॑ प्रि॒यम् प्रि॒यम् धामोप॑ । \newline
60. धामो पोप॒ धाम॒ धामोपा᳚प्नो त्याप्नो॒ त्युप॒ धाम॒ धामोपा᳚प्नोति । \newline
\pagebreak
\markright{ TS 6.6.11.5  \hfill https://www.vedavms.in \hfill}

\section{ TS 6.6.11.5 }

\textbf{TS 6.6.11.5 } \newline
\textbf{Samhita Paata} \newline

-पा᳚प्नोति॒ कनी॑याꣳसि॒ वै दे॒वेषु॒ छन्दाꣳ॒॒स्यास॒न्-ज्यायाꣳ॒॒-स्यसु॑रेषु॒ ते दे॒वाः कनी॑यसा॒ छन्द॑सा॒ ज्यायः॒ छन्दो॒ऽभि व्य॑शꣳस॒न् ततो॒ वै तेऽसु॑राणां ॅलो॒कम॑वृञ्जत॒ यत् कनी॑यसा॒ छन्द॑सा॒ ज्यायः॒ छन्दो॒ऽभिवि॒शꣳस॑ति॒ भ्रातृ॑व्यस्यै॒व तल्लो॒कं ॅवृ॑ङ्क्ते॒ षड॒क्षरा॒ण्यति॑ रेचयन्ति॒ षड् वा ऋ॒तव॑ ऋ॒तूने॒व प्री॑णाति च॒त्वारि॒ पूर्वा॒ण्यव॑ कल्पयन्ति॒- [  ] \newline

\textbf{Pada Paata} \newline

उपेति॑ । आ॒प्नो॒ति॒ । कनी॑याꣳसि । वै । दे॒वेषु॑ । छन्दाꣳ॑सि । आसन्न्॑ । ज्यायाꣳ॑सि ।   असु॑रेषु । ते । दे॒वाः । कनी॑यसा । छन्द॑सा । ज्यायः॑ । छन्दः॑ । अ॒भि । वीति॑ । अ॒शꣳ॒॒स॒न्न् । ततः॑ । वै । ते । असु॑राणाम् । लो॒कम् । अ॒वृ॒ञ्ज॒त॒ । यत् । कनी॑यसा । छन्द॑सा । ज्यायः॑ । छन्दः॑ । अ॒भीति॑ । वि॒शꣳस॒तीति॑ वि - शꣳस॑ति । भ्रातृ॑व्यस्य । ए॒व । तत् । लो॒कम् । वृ॒ङ्क्ते॒ । षट् । अ॒क्षरा॑णि । अतीति॑ । रे॒च॒य॒न्ति॒ । षट् । वै । ऋ॒तवः॑ । ऋ॒तून् । ए॒व । प्री॒णा॒ति॒ । च॒त्वारि॑ । पूर्वा॑णि । अवेति॑ । क॒ल्प॒य॒न्ति॒ ।  \newline


\textbf{Krama Paata} \newline

उपा᳚प्नोति । आ॒प्नो॒ति॒ कनी॑याꣳसि । कनी॑याꣳसि॒ वै । वै दे॒वेषु॑ । दे॒वेषु॒ छन्दाꣳ॑सि । छन्दाꣳ॒॒स्यासन्न्॑ । आस॒ञ्ज्यायाꣳ॑सि । ज्यायाꣳ॒॒स्यसु॑रेषु । असु॑रेषु॒ ते । ते दे॒वाः । दे॒वाः कनी॑यसा । कनी॑यसा॒ छन्द॑सा । छन्द॑सा॒ ज्यायः॑ । ज्याय॒श्छन्दः॑ । छन्दो॒ऽभि । अ॒भि वि । व्य॑शꣳसन्न् । अ॒शꣳ॒॒स॒न् ततः॑ । ततो॒ वै । वै ते । तेऽसु॑राणाम् । असु॑राणाम् ॅलो॒कम् । लो॒कम॑वृञ्जत । अ॒वृ॒ञ्ज॒त॒ यत् । यत् कनी॑यसा । कनी॑यसा॒ छन्द॑सा । छन्द॑सा॒ ज्यायः॑ । ज्याय॒श्छन्दः॑ । छन्दो॒ऽभिः । अ॒भिर् वि॒शꣳस॑ति । वि॒शꣳस॑ति॒ भ्रातृ॑व्यस्य । वि॒शꣳस॒तीति॑ वि - शꣳस॑ति । भ्रातृ॑व्यस्यै॒व । ए॒व तत् । तल्लो॒कम् । लो॒कम् ॅवृ॑ङ्‍क्ते । वृ॒ङ्‍क्ते॒ षट् । षड॒क्षरा॑णि । अ॒क्षरा॒ण्यति॑ । अति॑ रेचयन्ति । रे॒च॒य॒न्ति॒ षट् । षड् वै । वा ऋ॒तवः॑ । ऋ॒तव॑ ऋ॒तून् । ऋ॒तूने॒व । ए॒व प्री॑णाति । प्री॒णा॒ति॒ च॒त्वारि॑ । च॒त्वारि॒ पूर्वा॑णि । पूर्वा॒ण्यव॑ । अव॑ कल्पयन्ति ( ) । क॒ल्प॒य॒न्ति॒ चतु॑ष्पदः \newline

\textbf{Jatai Paata} \newline

1. उपा᳚प्नो त्याप्नो॒ त्युपोपा᳚प्नोति । \newline
2. आ॒प्नो॒ति॒ कनी॑याꣳसि॒ कनी॑याꣳ स्याप्नो त्याप्नोति॒ कनी॑याꣳसि । \newline
3. कनी॑याꣳसि॒ वै वै कनी॑याꣳसि॒ कनी॑याꣳसि॒ वै । \newline
4. वै दे॒वेषु॑ दे॒वेषु॒ वै वै दे॒वेषु॑ । \newline
5. दे॒वेषु॒ छन्दाꣳ॑सि॒ छन्दाꣳ॑सि दे॒वेषु॑ दे॒वेषु॒ छन्दाꣳ॑सि । \newline
6. छन्दाꣳ॒॒ स्यास॒न् नास॒न् छन्दाꣳ॑सि॒ छन्दाꣳ॒॒ स्यासन्न्॑ । \newline
7. आस॒न् ज्यायाꣳ॑सि॒ ज्यायाꣳ॒॒ स्यास॒न् नास॒न् ज्यायाꣳ॑सि । \newline
8. ज्यायाꣳ॒॒ स्यसु॑रे॒ष्व सु॑रेषु॒ ज्यायाꣳ॑सि॒ ज्यायाꣳ॒॒ स्यसु॑रेषु । \newline
9. असु॑रेषु॒ ते ते ऽसु॑रे॒ष् वसु॑रेषु॒ ते । \newline
10. ते दे॒वा दे॒वा स्ते ते दे॒वाः । \newline
11. दे॒वाः कनी॑यसा॒ कनी॑यसा दे॒वा दे॒वाः कनी॑यसा । \newline
12. कनी॑यसा॒ छन्द॑सा॒ छन्द॑सा॒ कनी॑यसा॒ कनी॑यसा॒ छन्द॑सा । \newline
13. छन्द॑सा॒ ज्यायो॒ ज्याय॒ श्छन्द॑सा॒ छन्द॑सा॒ ज्यायः॑ । \newline
14. ज्याय॒ श्छन्द॒ श्छन्दो॒ ज्यायो॒ ज्याय॒ श्छन्दः॑ । \newline
15. छन्दो॒ ऽभ्य॑भि च्छन्द॒ श्छन्दो॒ ऽभि । \newline
16. अ॒भि वि व्या᳚(1॒)भ्य॑भि वि । \newline
17. व्य॑शꣳसन् नशꣳस॒न्॒. वि व्य॑शꣳसन्न् । \newline
18. अ॒शꣳ॒॒स॒न् तत॒ स्ततो॑ ऽशꣳसन् नशꣳस॒न् ततः॑ । \newline
19. ततो॒ वै वै तत॒ स्ततो॒ वै । \newline
20. वै ते ते वै वै ते । \newline
21. ते ऽसु॑राणा॒ मसु॑राणा॒म् ते ते ऽसु॑राणाम् । \newline
22. असु॑राणाम् ॅलो॒कम् ॅलो॒क मसु॑राणा॒ मसु॑राणाम् ॅलो॒कम् । \newline
23. लो॒क म॑वृञ्जता वृञ्जत लो॒कम् ॅलो॒क म॑वृञ्जत । \newline
24. अ॒वृ॒ञ्ज॒त॒ यद् यद॑वृञ्जता वृञ्जत॒ यत् । \newline
25. यत् कनी॑यसा॒ कनी॑यसा॒ यद् यत् कनी॑यसा । \newline
26. कनी॑यसा॒ छन्द॑सा॒ छन्द॑सा॒ कनी॑यसा॒ कनी॑यसा॒ छन्द॑सा । \newline
27. छन्द॑सा॒ ज्यायो॒ ज्याय॒ श्छन्द॑सा॒ छन्द॑सा॒ ज्यायः॑ । \newline
28. ज्याय॒ श्छन्द॒ श्छन्दो॒ ज्यायो॒ ज्याय॒ श्छन्दः॑ । \newline
29. छन्दो॒ ऽभ्य॑भि च्छन्द॒ श्छन्दो॒ ऽभि । \newline
30. अ॒भि वि॒शꣳस॑ति वि॒शꣳस॑ त्य॒भ्य॑भि वि॒शꣳस॑ति । \newline
31. वि॒शꣳस॑ति॒ भ्रातृ॑व्यस्य॒ भ्रातृ॑व्यस्य वि॒शꣳस॑ति वि॒शꣳस॑ति॒ भ्रातृ॑व्यस्य । \newline
32. वि॒शꣳस॒तीति॑ वि - शꣳस॑ति । \newline
33. भ्रातृ॑व्य स्यै॒वैव भ्रातृ॑व्यस्य॒ भ्रातृ॑व्य स्यै॒व । \newline
34. ए॒व तत् तदे॒वैव तत् । \newline
35. तल्लो॒कम् ॅलो॒कम् तत् तल्लो॒कम् । \newline
36. लो॒कं ॅवृ॑ङ्क्ते वृङ्क्ते लो॒कम् ॅलो॒कं ॅवृ॑ङ्क्ते । \newline
37. वृ॒ङ्क्ते॒ षट् थ्षड् वृ॑ङ्क्ते वृङ्क्ते॒ षट् । \newline
38. षड॒क्षरा᳚ण्य॒ क्षरा॑णि॒ षट् थ्षड॒क्षरा॑णि । \newline
39. अ॒क्षरा॒ ण्यत्य त्य॒क्षरा᳚ण्य॒ क्षरा॒ ण्यति॑ । \newline
40. अति॑ रेचयन्ति रेचय॒ न्त्यत्यति॑ रेचयन्ति । \newline
41. रे॒च॒य॒न्ति॒ षट् थ्षड् रे॑चयन्ति रेचयन्ति॒ षट् । \newline
42. षड् वै वै षट् थ्षड् वै । \newline
43. वा ऋ॒तव॑ ऋ॒तवो॒ वै वा ऋ॒तवः॑ । \newline
44. ऋ॒तव॑ ऋ॒तू नृ॒तू नृ॒तव॑ ऋ॒तव॑ ऋ॒तून् । \newline
45. ऋ॒तू ने॒वैव र्‌तू नृ॒तू ने॒व । \newline
46. ए॒व प्री॑णाति प्रीणा त्ये॒वैव प्री॑णाति । \newline
47. प्री॒णा॒ति॒ च॒त्वारि॑ च॒त्वारि॑ प्रीणाति प्रीणाति च॒त्वारि॑ । \newline
48. च॒त्वारि॒ पूर्वा॑णि॒ पूर्वा॑णि च॒त्वारि॑ च॒त्वारि॒ पूर्वा॑णि । \newline
49. पूर्वा॒ ण्यवाव॒ पूर्वा॑णि॒ पूर्वा॒ ण्यव॑ । \newline
50. अव॑ कल्पयन्ति कल्पय॒ न्त्यवाव॑ कल्पयन्ति । \newline
51. क॒ल्प॒य॒न्ति॒ चतु॑ष्पद॒ श्चतु॑ष्पदः कल्पयन्ति कल्पयन्ति॒ चतु॑ष्पदः । \newline

\textbf{Ghana Paata } \newline

1. उपा᳚प्नो त्याप्नो॒ त्युपोपा᳚प्नोति॒ कनी॑याꣳसि॒ कनी॑याꣳ स्याप्नो॒ त्युपोपा᳚प्नोति॒ कनी॑याꣳसि । \newline
2. आ॒प्नो॒ति॒ कनी॑याꣳसि॒ कनी॑याꣳ स्याप्नो त्याप्नोति॒ कनी॑याꣳसि॒ वै वै कनी॑याꣳ स्याप्नो त्याप्नोति॒ कनी॑याꣳसि॒ वै । \newline
3. कनी॑याꣳसि॒ वै वै कनी॑याꣳसि॒ कनी॑याꣳसि॒ वै दे॒वेषु॑ दे॒वेषु॒ वै कनी॑याꣳसि॒ कनी॑याꣳसि॒ वै दे॒वेषु॑ । \newline
4. वै दे॒वेषु॑ दे॒वेषु॒ वै वै दे॒वेषु॒ छन्दाꣳ॑सि॒ छन्दाꣳ॑सि दे॒वेषु॒ वै वै दे॒वेषु॒ छन्दाꣳ॑सि । \newline
5. दे॒वेषु॒ छन्दाꣳ॑सि॒ छन्दाꣳ॑सि दे॒वेषु॑ दे॒वेषु॒ छन्दाꣳ॒॒ स्यास॒न् नास॒न् छन्दाꣳ॑सि दे॒वेषु॑ दे॒वेषु॒ छन्दाꣳ॒॒ स्यासन्न्॑ । \newline
6. छन्दाꣳ॒॒ स्यास॒न् नास॒न् छन्दाꣳ॑सि॒ छन्दाꣳ॒॒ स्यास॒न् ज्यायाꣳ॑सि॒ ज्यायाꣳ॒॒ स्यास॒न् छन्दाꣳ॑सि॒ छन्दाꣳ॒॒ स्यास॒न् ज्यायाꣳ॑सि । \newline
7. आस॒न् ज्यायाꣳ॑सि॒ ज्यायाꣳ॒॒ स्यास॒न् नास॒न् ज्यायाꣳ॒॒ स्यसु॑रे॒ ष्वसु॑रेषु॒ ज्यायाꣳ॒॒ स्यास॒न् नास॒न् ज्यायाꣳ॒॒ स्यसु॑रेषु । \newline
8. ज्यायाꣳ॒॒ स्यसु॑रे॒ ष्वसु॑रेषु॒ ज्यायाꣳ॑सि॒ ज्यायाꣳ॒॒ स्यसु॑रेषु॒ ते ते ऽसु॑रेषु॒ ज्यायाꣳ॑सि॒ ज्यायाꣳ॒॒ स्यसु॑रेषु॒ ते । \newline
9. असु॑रेषु॒ ते ते ऽसु॑रे॒ ष्वसु॑रेषु॒ ते दे॒वा दे॒वा स्ते ऽसु॑रे॒ष् वसु॑रेषु॒ ते दे॒वाः । \newline
10. ते दे॒वा दे॒वा स्ते ते दे॒वाः कनी॑यसा॒ कनी॑यसा दे॒वा स्ते ते दे॒वाः कनी॑यसा । \newline
11. दे॒वाः कनी॑यसा॒ कनी॑यसा दे॒वा दे॒वाः कनी॑यसा॒ छन्द॑सा॒ छन्द॑सा॒ कनी॑यसा दे॒वा दे॒वाः कनी॑यसा॒ छन्द॑सा । \newline
12. कनी॑यसा॒ छन्द॑सा॒ छन्द॑सा॒ कनी॑यसा॒ कनी॑यसा॒ छन्द॑सा॒ ज्यायो॒ ज्याय॒ श्छन्द॑सा॒ कनी॑यसा॒ कनी॑यसा॒ छन्द॑सा॒ ज्यायः॑ । \newline
13. छन्द॑सा॒ ज्यायो॒ ज्याय॒ श्छन्द॑सा॒ छन्द॑सा॒ ज्याय॒ श्छन्द॒ श्छन्दो॒ ज्याय॒ श्छन्द॑सा॒ छन्द॑सा॒ ज्याय॒ श्छन्दः॑ । \newline
14. ज्याय॒ श्छन्द॒ श्छन्दो॒ ज्यायो॒ ज्याय॒ श्छन्दो॒ ऽभ्य॑भि च्छन्दो॒ ज्यायो॒ ज्याय॒ श्छन्दो॒ ऽभि । \newline
15. छन्दोः॒ ऽभ्य॑भि च्छन्द॒ श्छन्दो॒ भि वि व्य॑भि च्छन्द॒ श्छन्दो॒ ऽभि वि । \newline
16. अ॒भि वि व्या᳚(1॒)भ्य॑भि व्य॑शꣳसन् नशꣳस॒न् व्या᳚(1॒)भ्य॑भि व्य॑शꣳसन्न् । \newline
17. व्य॑शꣳसन् नशꣳस॒न्॒. वि व्य॑शꣳस॒न् तत॒ स्ततो॑ ऽशꣳस॒न्॒. वि व्य॑शꣳस॒न् ततः॑ । \newline
18. अ॒शꣳ॒॒स॒न् तत॒ स्ततो॑ ऽशꣳसन् नशꣳस॒न् ततो॒ वै वै ततो॑ ऽशꣳसन् नशꣳस॒न् ततो॒ वै । \newline
19. ततो॒ वै वै तत॒ स्ततो॒ वै ते ते वै तत॒ स्ततो॒ वै ते । \newline
20. वै ते ते वै वै ते ऽसु॑राणा॒ मसु॑राणा॒म् ते वै वै ते ऽसु॑राणाम् । \newline
21. ते ऽसु॑राणा॒ मसु॑राणा॒म् ते ते ऽसु॑राणाम् ॅलो॒कम् ॅलो॒क मसु॑राणा॒म् ते ते ऽसु॑राणाम् ॅलो॒कम् । \newline
22. असु॑राणाम् ॅलो॒कम् ॅलो॒क मसु॑राणा॒ मसु॑राणाम् ॅलो॒क म॑वृञ्जता वृञ्जत लो॒क मसु॑राणा॒ मसु॑राणाम् ॅलो॒क म॑वृञ्जत । \newline
23. लो॒क म॑वृञ्जता वृञ्जत लो॒कम् ॅलो॒क म॑वृञ्जत॒ यद् यद॑वृञ्जत लो॒कम् ॅलो॒क म॑वृञ्जत॒ यत् । \newline
24. अ॒वृ॒ञ्ज॒त॒ यद् यद॑वृञ्जता वृञ्जत॒ यत् कनी॑यसा॒ कनी॑यसा॒ यद॑वृञ्जता वृञ्जत॒ यत् कनी॑यसा । \newline
25. यत् कनी॑यसा॒ कनी॑यसा॒ यद् यत् कनी॑यसा॒ छन्द॑सा॒ छन्द॑सा॒ कनी॑यसा॒ यद् यत् कनी॑यसा॒ छन्द॑सा । \newline
26. कनी॑यसा॒ छन्द॑सा॒ छन्द॑सा॒ कनी॑यसा॒ कनी॑यसा॒ छन्द॑सा॒ ज्यायो॒ ज्याय॒ श्छन्द॑सा॒ कनी॑यसा॒ कनी॑यसा॒ छन्द॑सा॒ ज्यायः॑ । \newline
27. छन्द॑सा॒ ज्यायो॒ ज्याय॒ श्छन्द॑सा॒ छन्द॑सा॒ ज्याय॒ श्छन्द॒ श्छन्दो॒ ज्याय॒ श्छन्द॑सा॒ छन्द॑सा॒ ज्याय॒ श्छन्दः॑ । \newline
28. ज्याय॒ श्छन्द॒ श्छन्दो॒ ज्यायो॒ ज्याय॒ श्छन्दो॒ ऽभ्य॑भि च्छन्दो॒ ज्यायो॒ ज्याय॒ श्छन्दो॒ ऽभि । \newline
29. छन्दो॒ ऽभ्य॑भि च्छन्द॒ श्छन्दो॒ ऽभि वि॒शꣳस॑ति वि॒शꣳस॑ त्य॒भि च्छन्द॒ श्छन्दो॒ ऽभि वि॒शꣳस॑ति । \newline
30. अ॒भि वि॒शꣳस॑ति वि॒शꣳस॑ त्य॒भ्य॑भि वि॒शꣳस॑ति॒ भ्रातृ॑व्यस्य॒ भ्रातृ॑व्यस्य वि॒शꣳस॑ त्य॒भ्य॑भि वि॒शꣳस॑ति॒ भ्रातृ॑व्यस्य । \newline
31. वि॒शꣳस॑ति॒ भ्रातृ॑व्यस्य॒ भ्रातृ॑व्यस्य वि॒शꣳस॑ति वि॒शꣳस॑ति॒ भ्रातृ॑व्यस्यै॒वैव भ्रातृ॑व्यस्य वि॒शꣳस॑ति वि॒शꣳस॑ति॒ भ्रातृ॑व्यस्यै॒व । \newline
32. वि॒शꣳस॒तीति॑ वि - शꣳस॑ति । \newline
33. भ्रातृ॑व्यस्यै॒वैव भ्रातृ॑व्यस्य॒ भ्रातृ॑व्य स्यै॒व तत् तदे॒व भ्रातृ॑व्यस्य॒ भ्रातृ॑व्य स्यै॒व तत् । \newline
34. ए॒व तत् तदे॒वैव तल्लो॒कम् ॅलो॒कम् तदे॒ वैव तल्लो॒कम् । \newline
35. तल्लो॒कम् ॅलो॒कम् तत् तल्लो॒कं ॅवृ॑ङ्क्ते वृङ्क्ते लो॒कम् तत् तल्लो॒कं ॅवृ॑ङ्क्ते । \newline
36. लो॒कं ॅवृ॑ङ्क्ते वृङ्क्ते लो॒कम् ॅलो॒कं ॅवृ॑ङ्क्ते॒ षट् थ्षड् वृ॑ङ्क्ते लो॒कम् ॅलो॒कं ॅवृ॑ङ्क्ते॒ षट् । \newline
37. वृ॒ङ्क्ते॒ षट् थ्षड् वृ॑ङ्क्ते वृङ्क्ते॒ षड॒क्षरा᳚ ण्य॒क्षरा॑णि॒ षड् वृ॑ङ्क्ते वृङ्क्ते॒ 
षड॒क्षरा॑णि । \newline
38. षड॒क्षरा᳚ ण्य॒क्षरा॑णि॒ षट् थ्षड॒क्षरा॒ ण्यत्य त्य॒क्षरा॑णि॒ षट् थ्षड॒क्षरा॒ ण्यति॑ । \newline
39. अ॒क्षरा॒ ण्यत्य त्य॒क्षरा᳚ ण्य॒क्षरा॒ ण्यति॑ रेचयन्ति रेचय॒ न्त्यत्य॒क्षरा᳚ ण्य॒क्षरा॒ ण्यति॑ रेचयन्ति । \newline
40. अति॑ रेचयन्ति रेचय॒ न्त्यत्यति॑ रेचयन्ति॒ षट् थ्षड् रे॑चय॒ न्त्यत्यति॑ रेचयन्ति॒ षट् । \newline
41. रे॒च॒य॒न्ति॒ षट् थ्षड् रे॑चयन्ति रेचयन्ति॒ षड् वै वै षड् रे॑चयन्ति रेचयन्ति॒ षड् वै । \newline
42. षड् वै वै षट् थ्षड् वा ऋ॒तव॑ ऋ॒तवो॒ वै षट् थ्षड् वा ऋ॒तवः॑ । \newline
43. वा ऋ॒तव॑ ऋ॒तवो॒ वै वा ऋ॒तव॑ ऋ॒तू नृ॒तू नृ॒तवो॒ वै वा ऋ॒तव॑ ऋ॒तून् । \newline
44. ऋ॒तव॑ ऋ॒तू नृ॒तू नृ॒तव॑ ऋ॒तव॑ ऋ॒तू ने॒वैव र्‌तू नृ॒तव॑ ऋ॒तव॑ ऋ॒तू ने॒व । \newline
45. ऋ॒तू ने॒वैव र्‌तू नृ॒तू ने॒व प्री॑णाति प्रीणा त्ये॒व र्‌तू नृ॒तू ने॒व प्री॑णाति । \newline
46. ए॒व प्री॑णाति प्रीणा त्ये॒वैव प्री॑णाति च॒त्वारि॑ च॒त्वारि॑ प्रीणा त्ये॒वैव प्री॑णाति च॒त्वारि॑ । \newline
47. प्री॒णा॒ति॒ च॒त्वारि॑ च॒त्वारि॑ प्रीणाति प्रीणाति च॒त्वारि॒ पूर्वा॑णि॒ पूर्वा॑णि च॒त्वारि॑ प्रीणाति प्रीणाति च॒त्वारि॒ पूर्वा॑णि । \newline
48. च॒त्वारि॒ पूर्वा॑णि॒ पूर्वा॑णि च॒त्वारि॑ च॒त्वारि॒ पूर्वा॒ ण्यवाव॒ पूर्वा॑णि च॒त्वारि॑ च॒त्वारि॒ पूर्वा॒ ण्यव॑ । \newline
49. पूर्वा॒ ण्यवाव॒ पूर्वा॑णि॒ पूर्वा॒ ण्यव॑ कल्पयन्ति कल्पय॒ न्त्यव॒ पूर्वा॑णि॒ पूर्वा॒ ण्यव॑ कल्पयन्ति । \newline
50. अव॑ कल्पयन्ति कल्पय॒ न्त्यवाव॑ कल्पयन्ति॒ चतु॑ष्पद॒ श्चतु॑ष्पदः कल्पय॒ न्त्यवाव॑ कल्पयन्ति॒ चतु॑ष्पदः । \newline
51. क॒ल्प॒य॒न्ति॒ चतु॑ष्पद॒ श्चतु॑ष्पदः कल्पयन्ति कल्पयन्ति॒ चतु॑ष्पद ए॒वैव चतु॑ष्पदः कल्पयन्ति कल्पयन्ति॒ चतु॑ष्पद ए॒व । \newline
\pagebreak
\markright{ TS 6.6.11.6  \hfill https://www.vedavms.in \hfill}

\section{ TS 6.6.11.6 }

\textbf{TS 6.6.11.6 } \newline
\textbf{Samhita Paata} \newline

चतु॑ष्पद ए॒व प॒शून॑व रुन्धे॒ द्वे उत्त॑रे द्वि॒पद॑ ए॒वाव॑ रुन्धे ऽनु॒ष्टुभ॑म॒भि सं पा॑दयन्ति॒ वाग्वा अ॑नु॒ष्टुप् तस्मा᳚त् प्रा॒णानां॒ ॅवागु॑त्त॒मा स॑मयाविषि॒ते सूर्ये॑ षोड॒शिनः॑ स्तो॒त्र-मु॒पाक॑रोत्ये॒तस्मि॒न् वै लो॒क इन्द्रो॑ वृ॒त्रम॑हन्थ् सा॒क्षादे॒व वज्रं॒ भ्रातृ॑व्याय॒ प्र ह॑र-त्यरुणपिश॒ङ्गोऽश्वो॒ दक्षि॑णै॒तद्वै वज्र॑स्य रू॒पꣳ समृ॑द्ध्यै ॥ \newline

\textbf{Pada Paata} \newline

चतु॑ष्पद॒ इति॒ चतुः॑-प॒दः॒ । ए॒व । प॒शून् । अवेति॑ । रु॒न्धे॒ । द्वे इति॑ । उत्त॑रे॒ इत्युत् - त॒रे॒ । द्वि॒पद॒ इति॑ द्वि - पदः॑ । ए॒व । अवेति॑ । रु॒न्धे॒ । अ॒नु॒ष्टुभ॒मित्य॑नु - स्तुभ᳚म् । अ॒भि । समिति॑ । पा॒द॒य॒न्ति॒ । वाक् । वै । अ॒नु॒ष्टुबित्य॑नु - स्तुप् । तस्मा᳚त् । प्रा॒णाना॒मिति॑ प्र - अ॒नाना᳚म् । वाक् । उ॒त्त॒मेत्यु॑त् - त॒मा । स॒म॒या॒वि॒षि॒त इति॑ समया - वि॒षि॒ते । सूर्ये᳚ । षो॒ड॒शिनः॑ । स्तो॒त्रम् । उ॒पाक॑रो॒तीत्यु॑प - आक॑रोति । ए॒तस्मिन्न्॑ । वै । लो॒के । इन्द्रः॑ । वृ॒त्रम् । अ॒ह॒न्न् । सा॒क्षादिति॑ स - अ॒क्षात् । ए॒व । वज्र᳚म् । भ्रातृ॑व्याय । प्रेति॑ । ह॒र॒ति॒ । अ॒रु॒ण॒पि॒श॒ङ्ग इत्य॑रुण - पि॒श॒ङ्गः । अश्वः॑ । दक्षि॑णा । ए॒तत् । वै । वज्र॑स्य । रू॒पम् । समृ॑द्ध्या॒ इति॒ सं - ऋ॒द्ध्यै॒ ॥  \newline


\textbf{Krama Paata} \newline

चतु॑ष्पद ए॒व । चतु॑ष्पद॒ इति॒ चतुः॑ - प॒दः॒ । ए॒व प॒शून् । प॒शूनव॑ । अव॑ रुन्धे । रु॒न्धे॒ द्वे । द्वे उत्त॑रे । द्वे इति॒ द्वे । उत्त॑रे द्वि॒पदः॑ । उत्त॑रे॒ इत्युत् - त॒रे॒ । द्वि॒पद॑ ए॒व । द्वि॒पद॒ इति॑ द्वि - पदः॑ । ए॒वाव॑ । अव॑ रुन्धे । रु॒न्धे॒ऽनु॒ष्टुभ᳚म् । अ॒नु॒ष्टुभ॑म॒भि । अ॒नु॒ष्टुभ॒मित्य॑नु - स्तुभ᳚म् । अ॒भि सम् । सम् पा॑दयन्ति । पा॒द॒य॒न्ति॒ वाक् । वाग्वै । वा अ॑नु॒ष्टुप् । अ॒नु॒ष्टुप् तस्मा᳚त् । अ॒नु॒ष्टुबित्य॑नु - स्तुप् । तस्मा᳚त् प्रा॒णाना᳚म् । प्रा॒णाना॒म् ॅवाक् । प्रा॒णाना॒मिति॑ प्र - अ॒नाना᳚म् । वागु॑त्त॒मा । उ॒त्त॒मा स॑मयाविषि॒ते । उ॒त्त॒मेत्यु॑त् - त॒मा । स॒म॒या॒वि॒षि॒ते सूर्ये᳚ । स॒म॒या॒वि॒षि॒त इति॑ समया - वि॒षि॒ते । सूर्ये॑ षोड॒शिनः॑ । षो॒ड॒शिनः॑ स्तो॒त्रम् । स्तो॒त्रमु॒पाक॑रोति । उ॒पाक॑रोत्ये॒तस्मिन्न्॑ । उ॒पाक॑रो॒तीत्यु॑प - आक॑रोति । ए॒तस्मि॒न् वै । वै लो॒के । लो॒क इन्द्रः॑ । इन्द्रो॑ वृ॒त्रम् । वृ॒त्रम॑हन्न् । अ॒ह॒न्थ् सा॒क्षात् । सा॒क्षादे॒व । सा॒क्षादिति॑ स - अ॒क्षात् । ए॒व वज्र᳚म् । वज्र॒म् भ्रातृ॑व्याय । भ्रातृ॑व्याय॒ प्र । प्र ह॑रति । ह॒र॒त्य॒रु॒ण॒पि॒श॒ङ्‍गः । अ॒रु॒ण॒पि॒श॒ङ्‍गोऽश्वः॑ । अ॒रु॒ण॒पि॒श॒ङ्‍ग इत्य॑रुण - पि॒श॒ङ्‍गः । अश्वो॒ दक्षि॑णा । दक्षि॑णै॒तत् । ए॒तद् वै । वै वज्र॑स्य । वज्र॑स्य रू॒पम् । रू॒पꣳ समृ॑द्ध्यै । समृ॑द्ध्या॒ इति॒ सम् - ऋ॒द्ध्यै॒ । \newline

\textbf{Jatai Paata} \newline

1. चतु॑ष्पद ए॒वैव चतु॑ष्पद॒ श्चतु॑ष्पद ए॒व । \newline
2. चतु॑ष्पद॒ इति॒ चतुः॑ - प॒दः॒ । \newline
3. ए॒व प॒शून् प॒शू ने॒वैव प॒शून् । \newline
4. प॒शू नवाव॑ प॒शून् प॒शू नव॑ । \newline
5. अव॑ रुन्धे रु॒न्धे ऽवाव॑ रुन्धे । \newline
6. रु॒न्धे॒ द्वे द्वे रु॑न्धे रुन्धे॒ द्वे । \newline
7. द्वे उत्त॑रे॒ उत्त॑रे॒ द्वे द्वे उत्त॑रे । \newline
8. द्वे इति॒ द्वे । \newline
9. उत्त॑रे द्वि॒पदो᳚ द्वि॒पद॒ उत्त॑रे॒ उत्त॑रे द्वि॒पदः॑ । \newline
10. उत्त॑रे॒ इत्युत् - त॒रे॒ । \newline
11. द्वि॒पद॑ ए॒वैव द्वि॒पदो᳚ द्वि॒पद॑ ए॒व । \newline
12. द्वि॒पद॒ इति॑ द्वि - पदः॑ । \newline
13. ए॒वावा वै॒वै वाव॑ । \newline
14. अव॑ रुन्धे रु॒न्धे ऽवाव॑ रुन्धे । \newline
15. रु॒न्धे॒ ऽनु॒ष्टुभ॑ मनु॒ष्टुभꣳ॑ रुन्धे रुन्धे ऽनु॒ष्टुभ᳚म् । \newline
16. अ॒नु॒ष्टुभ॑ म॒भ्या᳚(1॒)भ्य॑ नु॒ष्टुभ॑ मनु॒ष्टुभ॑ म॒भि । \newline
17. अ॒नु॒ष्टुभ॒मित्य॑नु - स्तुभ᳚म् । \newline
18. अ॒भि सꣳ स म॒भ्य॑भि सम् । \newline
19. सम् पा॑दयन्ति पादयन्ति॒ सꣳ सम् पा॑दयन्ति । \newline
20. पा॒द॒य॒न्ति॒ वाग् वाक् पा॑दयन्ति पादयन्ति॒ वाक् । \newline
21. वाग् वै वै वाग् वाग् वै । \newline
22. वा अ॑नु॒ष्टु ब॑नु॒ष्टुब् वै वा अ॑नु॒ष्टुप् । \newline
23. अ॒नु॒ष्टुप् तस्मा॒त् तस्मा॑ दनु॒ष्टु ब॑नु॒ष्टुप् तस्मा᳚त् । \newline
24. अ॒नु॒ष्टुबित्य॑नु - स्तुप् । \newline
25. तस्मा᳚त् प्रा॒णाना᳚म् प्रा॒णाना॒म् तस्मा॒त् तस्मा᳚त् प्रा॒णाना᳚म् । \newline
26. प्रा॒णानां॒ ॅवाग् वाक् प्रा॒णाना᳚म् प्रा॒णानां॒ ॅवाक् । \newline
27. प्रा॒णाना॒मिति॑ प्र - अ॒नाना᳚म् । \newline
28. वागु॑त्त॒ मोत्त॒मा वाग् वागु॑त्त॒मा । \newline
29. उ॒त्त॒मा स॑मयाविषि॒ते स॑मयाविषि॒त उ॑त्त॒ मोत्त॒मा स॑मयाविषि॒ते । \newline
30. उ॒त्त॒मेत्यु॑त् - त॒मा । \newline
31. स॒म॒या॒वि॒षि॒ते सूर्ये॒ सूर्ये॑ समयाविषि॒ते स॑मयाविषि॒ते सूर्ये᳚ । \newline
32. स॒म॒या॒वि॒षि॒त इति॑ समया - वि॒षि॒ते । \newline
33. सूर्ये॑ षोड॒शिन॑ ष्षोड॒शिनः॒ सूर्ये॒ सूर्ये॑ षोड॒शिनः॑ । \newline
34. षो॒ड॒शिनः॑ स्तो॒त्रꣳ स्तो॒त्रꣳ षो॑ड॒शिन॑ ष्षोड॒शिनः॑ स्तो॒त्रम् । \newline
35. स्तो॒त्र मु॒पाक॑रो त्यु॒पाक॑रोति स्तो॒त्रꣳ स्तो॒त्र मु॒पाक॑रोति । \newline
36. उ॒पाक॑रो त्ये॒तस्मि॑न् ने॒तस्मि॑न् नु॒पाक॑रो त्यु॒पाक॑रो त्ये॒तस्मिन्न्॑ । \newline
37. उ॒पाक॑रो॒तीत्यु॑प - आक॑रोति । \newline
38. ए॒तस्मि॒न्॒. वै वा ए॒तस्मि॑न् ने॒तस्मि॒न्॒. वै । \newline
39. वै लो॒के लो॒के वै वै लो॒के । \newline
40. लो॒क इन्द्र॒ इन्द्रो॑ लो॒के लो॒क इन्द्रः॑ । \newline
41. इन्द्रो॑ वृ॒त्रं ॅवृ॒त्र मिन्द्र॒ इन्द्रो॑ वृ॒त्रम् । \newline
42. वृ॒त्र म॑हन् नहन् वृ॒त्रं ॅवृ॒त्र म॑हन्न् । \newline
43. अ॒ह॒न् थ्सा॒क्षाथ् सा॒क्षा द॑हन् नहन् थ्सा॒क्षात् । \newline
44. सा॒क्षा दे॒वैव सा॒क्षाथ् सा॒क्षा दे॒व । \newline
45. सा॒क्षादिति॑ स - अ॒क्षात् । \newline
46. ए॒व वज्रं॒ ॅवज्र॑ मे॒वैव वज्र᳚म् । \newline
47. वज्र॒म् भ्रातृ॑व्याय॒ भ्रातृ॑व्याय॒ वज्रं॒ ॅवज्र॒म् भ्रातृ॑व्याय । \newline
48. भ्रातृ॑व्याय॒ प्र प्र भ्रातृ॑व्याय॒ भ्रातृ॑व्याय॒ प्र । \newline
49. प्र ह॑रति हरति॒ प्र प्र ह॑रति । \newline
50. ह॒र॒ त्य॒रु॒ण॒पि॒श॒ङ्गो॑ ऽरुणपिश॒ङ्गो ह॑रति हर त्यरुणपिश॒ङ्गः । \newline
51. अ॒रु॒ण॒पि॒श॒ङ्गो ऽश्वो ऽश्वो॑ ऽरुणपिश॒ङ्गो॑ ऽरुणपिश॒ङ्गो ऽश्वः॑ । \newline
52. अ॒रु॒ण॒पि॒श॒ङ्ग इत्य॑रुण - पि॒श॒ङ्गः । \newline
53. अश्वो॒ दक्षि॑णा॒ दक्षि॒णा ऽश्वो ऽश्वो॒ दक्षि॑णा । \newline
54. दक्षि॑णै॒त दे॒तद् दक्षि॑णा॒ दक्षि॑ णै॒तत् । \newline
55. ए॒तद् वै वा ए॒त दे॒तद् वै । \newline
56. वै वज्र॑स्य॒ वज्र॑स्य॒ वै वै वज्र॑स्य । \newline
57. वज्र॑स्य रू॒पꣳ रू॒पं ॅवज्र॑स्य॒ वज्र॑स्य रू॒पम् । \newline
58. रू॒पꣳ समृ॑द्ध्यै॒ समृ॑द्ध्यै रू॒पꣳ रू॒पꣳ समृ॑द्ध्यै । \newline
59. समृ॑द्ध्या॒ इति॒ सं - ऋ॒द्ध्यै॒ । \newline

\textbf{Ghana Paata } \newline

1. चतु॑ष्पद ए॒वैव चतु॑ष्पद॒ श्चतु॑ष्पद ए॒व प॒शून् प॒शू ने॒व चतु॑ष्पद॒ श्चतु॑ष्पद ए॒व प॒शून् । \newline
2. चतु॑ष्पद॒ इति॒ चतुः॑ - प॒दः॒ । \newline
3. ए॒व प॒शून् प॒शू ने॒वैव प॒शू नवाव॑ प॒शू ने॒वैव प॒शू नव॑ । \newline
4. प॒शू नवाव॑ प॒शून् प॒शू नव॑ रुन्धे रु॒न्धे ऽव॑ प॒शून् प॒शू नव॑ रुन्धे । \newline
5. अव॑ रुन्धे रु॒न्धे ऽवाव॑ रुन्धे॒ द्वे द्वे रु॒न्धे ऽवाव॑ रुन्धे॒ द्वे । \newline
6. रु॒न्धे॒ द्वे द्वे रु॑न्धे रुन्धे॒ द्वे उत्त॑रे॒ उत्त॑रे॒ द्वे रु॑न्धे रुन्धे॒ द्वे उत्त॑रे । \newline
7. द्वे उत्त॑रे॒ उत्त॑रे॒ द्वे द्वे उत्त॑रे द्वि॒पदो᳚ द्वि॒पद॒ उत्त॑रे॒ द्वे द्वे उत्त॑रे द्वि॒पदः॑ । \newline
8. द्वे इति॒ द्वे । \newline
9. उत्त॑रे द्वि॒पदो᳚ द्वि॒पद॒ उत्त॑रे॒ उत्त॑रे द्वि॒पद॑ ए॒वैव द्वि॒पद॒ उत्त॑रे॒ उत्त॑रे द्वि॒पद॑ ए॒व । \newline
10. उत्त॑रे॒ इत्युत् - त॒रे॒ । \newline
11. द्वि॒पद॑ ए॒वैव द्वि॒पदो᳚ द्वि॒पद॑ ए॒वावा वै॒व द्वि॒पदो᳚ द्वि॒पद॑ ए॒वाव॑ । \newline
12. द्वि॒पद॒ इति॑ द्वि - पदः॑ । \newline
13. ए॒वावा वै॒वै वाव॑ रुन्धे रु॒न्धे ऽवै॒वै वाव॑ रुन्धे । \newline
14. अव॑ रुन्धे रु॒न्धे ऽवाव॑ रुन्धे ऽनु॒ष्टुभ॑ मनु॒ष्टुभꣳ॑ रु॒न्धे ऽवाव॑ रुन्धे ऽनु॒ष्टुभ᳚म् । \newline
15. रु॒न्धे॒ ऽनु॒ष्टुभ॑ मनु॒ष्टुभꣳ॑ रुन्धे रुन्धे ऽनु॒ष्टुभ॑ म॒भ्या᳚(1॒)भ्य॑नु॒ष्टुभꣳ॑ रुन्धे रुन्धे ऽनु॒ष्टुभ॑ म॒भि । \newline
16. अ॒नु॒ष्टुभ॑ म॒भ्या᳚(1॒)भ्य॑ नु॒ष्टुभ॑ मनु॒ष्टुभ॑ म॒भि सꣳ स म॒भ्य॑नु॒ष्टुभ॑ मनु॒ष्टुभ॑ म॒भि सम् । \newline
17. अ॒नु॒ष्टुभ॒मित्य॑नु - स्तुभ᳚म् । \newline
18. अ॒भि सꣳ स म॒भ्य॑भि सम् पा॑दयन्ति पादयन्ति॒ स म॒भ्य॑भि सम् पा॑दयन्ति । \newline
19. सम् पा॑दयन्ति पादयन्ति॒ सꣳ सम् पा॑दयन्ति॒ वाग् वाक् पा॑दयन्ति॒ सꣳ सम् पा॑दयन्ति॒ वाक् । \newline
20. पा॒द॒य॒न्ति॒ वाग् वाक् पा॑दयन्ति पादयन्ति॒ वाग् वै वै वाक् पा॑दयन्ति पादयन्ति॒ वाग् वै । \newline
21. वाग् वै वै वाग् वाग् वा अ॑नु॒ष्टु ब॑नु॒ष्टुब् वै वाग् वाग् वा अ॑नु॒ष्टुप् । \newline
22. वा अ॑नु॒ष्टु ब॑नु॒ष्टुब् वै वा अ॑नु॒ष्टुप् तस्मा॒त् तस्मा॑ दनु॒ष्टुब् वै वा अ॑नु॒ष्टुप् तस्मा᳚त् । \newline
23. अ॒नु॒ष्टुप् तस्मा॒त् तस्मा॑ दनु॒ष्टु ब॑नु॒ष्टुप् तस्मा᳚त् प्रा॒णाना᳚म् प्रा॒णाना॒म् तस्मा॑ दनु॒ष्टु ब॑नु॒ष्टुप् तस्मा᳚त् प्रा॒णाना᳚म् । \newline
24. अ॒नु॒ष्टुबित्य॑नु - स्तुप् । \newline
25. तस्मा᳚त् प्रा॒णाना᳚म् प्रा॒णाना॒म् तस्मा॒त् तस्मा᳚त् प्रा॒णानां॒ ॅवाग् वाक् प्रा॒णाना॒म् तस्मा॒त् तस्मा᳚त् प्रा॒णानां॒ ॅवाक् । \newline
26. प्रा॒णानां॒ ॅवाग् वाक् प्रा॒णाना᳚म् प्रा॒णानां॒ ॅवागु॑त्त॒ मोत्त॒मा वाक् प्रा॒णाना᳚म् प्रा॒णानां॒ ॅवागु॑त्त॒मा । \newline
27. प्रा॒णाना॒मिति॑ प्र - अ॒नाना᳚म् । \newline
28. वागु॑त्त॒ मोत्त॒मा वाग् वागु॑त्त॒मा स॑मयाविषि॒ते स॑मयाविषि॒त उ॑त्त॒मा वाग् वागु॑त्त॒मा स॑मयाविषि॒ते । \newline
29. उ॒त्त॒मा स॑मयाविषि॒ते स॑मयाविषि॒त उ॑त्त॒मोत्त॒मा स॑मयाविषि॒ते सूर्ये॒ सूर्ये॑ समयाविषि॒त उ॑त्त॒मोत्त॒मा स॑मयाविषि॒ते सूर्ये᳚ । \newline
30. उ॒त्त॒मेत्यु॑त् - त॒मा । \newline
31. स॒म॒या॒वि॒षि॒ते सूर्ये॒ सूर्ये॑ समयाविषि॒ते स॑मयाविषि॒ते सूर्ये॑ षोड॒शिन॑ ष्षोड॒शिनः॒ सूर्ये॑ समयाविषि॒ते स॑मयाविषि॒ते सूर्ये॑ षोड॒शिनः॑ । \newline
32. स॒म॒या॒वि॒षि॒त इति॑ समया - वि॒षि॒ते । \newline
33. सूर्ये॑ षोड॒शिन॑ ष्षोड॒शिनः॒ सूर्ये॒ सूर्ये॑ षोड॒शिनः॑ स्तो॒त्रꣳ स्तो॒त्रꣳ षो॑ड॒शिनः॒ सूर्ये॒ सूर्ये॑ षोड॒शिनः॑ स्तो॒त्रम् । \newline
34. षो॒ड॒शिनः॑ स्तो॒त्रꣳ स्तो॒त्रꣳ षो॑ड॒शिन॑ ष्षोड॒शिनः॑ स्तो॒त्र मु॒पाक॑रो त्यु॒पाक॑रोति स्तो॒त्रꣳ षो॑ड॒शिन॑ ष्षोड॒शिनः॑ स्तो॒त्र मु॒पाक॑रोति । \newline
35. स्तो॒त्र मु॒पाक॑रो त्यु॒पाक॑रोति स्तो॒त्रꣳ स्तो॒त्र मु॒पाक॑रो त्ये॒तस्मि॑न् ने॒तस्मि॑न् नु॒पाक॑रोति स्तो॒त्रꣳ स्तो॒त्र मु॒पाक॑रो त्ये॒तस्मिन्न्॑ । \newline
36. उ॒पाक॑रो त्ये॒तस्मि॑न् ने॒तस्मि॑न् नु॒पाक॑रो त्यु॒पाक॑रो त्ये॒तस्मि॒न्॒. वै वा ए॒तस्मि॑न् नु॒पाक॑रो त्यु॒पाक॑रो त्ये॒तस्मि॒न्॒. वै । \newline
37. उ॒पाक॑रो॒तीत्यु॑प - आक॑रोति । \newline
38. ए॒तस्मि॒न्॒. वै वा ए॒तस्मि॑न् ने॒तस्मि॒न्॒. वै लो॒के लो॒के वा ए॒तस्मि॑न् ने॒तस्मि॒न्॒. वै लो॒के । \newline
39. वै लो॒के लो॒के वै वै लो॒क इन्द्र॒ इन्द्रो॑ लो॒के वै वै लो॒क इन्द्रः॑ । \newline
40. लो॒क इन्द्र॒ इन्द्रो॑ लो॒के लो॒क इन्द्रो॑ वृ॒त्रं ॅवृ॒त्र मिन्द्रो॑ लो॒के लो॒क इन्द्रो॑ वृ॒त्रम् । \newline
41. इन्द्रो॑ वृ॒त्रं ॅवृ॒त्र मिन्द्र॒ इन्द्रो॑ वृ॒त्र म॑हन् नहन् वृ॒त्र मिन्द्र॒ इन्द्रो॑ वृ॒त्र म॑हन्न् । \newline
42. वृ॒त्र म॑हन् नहन् वृ॒त्रं ॅवृ॒त्र म॑हन् थ्सा॒क्षाथ् सा॒क्षा द॑हन् वृ॒त्रं ॅवृ॒त्र म॑हन् थ्सा॒क्षात् । \newline
43. अ॒ह॒न् थ्सा॒क्षाथ् सा॒क्षा द॑हन् नहन् थ्सा॒क्षा दे॒वैव सा॒क्षा द॑हन् नहन् थ्सा॒क्षा दे॒व । \newline
44. सा॒क्षा दे॒वैव सा॒क्षाथ् सा॒क्षा दे॒व वज्रं॒ ॅवज्र॑ मे॒व सा॒क्षाथ् सा॒क्षा दे॒व वज्र᳚म् । \newline
45. सा॒क्षादिति॑ स - अ॒क्षात् । \newline
46. ए॒व वज्रं॒ ॅवज्र॑ मे॒वैव वज्र॒म् भ्रातृ॑व्याय॒ भ्रातृ॑व्याय॒ वज्र॑ मे॒वैव वज्र॒म् भ्रातृ॑व्याय । \newline
47. वज्र॒म् भ्रातृ॑व्याय॒ भ्रातृ॑व्याय॒ वज्रं॒ ॅवज्र॒म् भ्रातृ॑व्याय॒ प्र प्र भ्रातृ॑व्याय॒ वज्रं॒ ॅवज्र॒म् भ्रातृ॑व्याय॒ प्र । \newline
48. भ्रातृ॑व्याय॒ प्र प्र भ्रातृ॑व्याय॒ भ्रातृ॑व्याय॒ प्र ह॑रति हरति॒ प्र भ्रातृ॑व्याय॒ भ्रातृ॑व्याय॒ प्र ह॑रति । \newline
49. प्र ह॑रति हरति॒ प्र प्र ह॑र त्यरुणपिश॒ङ्गो॑ ऽरुणपिश॒ङ्गो ह॑रति॒ प्र प्र ह॑र त्यरुणपिश॒ङ्गः । \newline
50. ह॒र॒ त्य॒रु॒ण॒पि॒श॒ङ्गो॑ ऽरुणपिश॒ङ्गो ह॑रति हर त्यरुणपिश॒ङ्गो ऽश्वो ऽश्वो॑ ऽरुणपिश॒ङ्गो ह॑रति हर त्यरुणपिश॒ङ्गो ऽश्वः॑ । \newline
51. अ॒रु॒ण॒पि॒श॒ङ्गो ऽश्वो ऽश्वो॑ ऽरुणपिश॒ङ्गो॑ ऽरुणपिश॒ङ्गो ऽश्वो॒ दक्षि॑णा॒ दक्षि॒णा ऽश्वो॑ ऽरुणपिश॒ङ्गो॑ ऽरुणपिश॒ङ्गो ऽश्वो॒ दक्षि॑णा । \newline
52. अ॒रु॒ण॒पि॒श॒ङ्ग इत्य॑रुण - पि॒श॒ङ्गः । \newline
53. अश्वो॒ दक्षि॑णा॒ दक्षि॒णा ऽश्वो ऽश्वो॒ दक्षि॑ णै॒त दे॒तद् दक्षि॒णा ऽश्वो ऽश्वो॒ दक्षि॑णै॒तत् । \newline
54. दक्षि॑ णै॒त दे॒तद् दक्षि॑णा॒ दक्षि॑ णै॒तद् वै वा ए॒तद् दक्षि॑णा॒ दक्षि॑ णै॒तद् वै । \newline
55. ए॒तद् वै वा ए॒त दे॒तद् वै वज्र॑स्य॒ वज्र॑स्य॒ वा ए॒त दे॒तद् वै वज्र॑स्य । \newline
56. वै वज्र॑स्य॒ वज्र॑स्य॒ वै वै वज्र॑स्य रू॒पꣳ रू॒पं ॅवज्र॑स्य॒ वै वै वज्र॑स्य रू॒पम् । \newline
57. वज्र॑स्य रू॒पꣳ रू॒पं ॅवज्र॑स्य॒ वज्र॑स्य रू॒पꣳ समृ॑द्ध्यै॒ समृ॑द्ध्यै रू॒पं ॅवज्र॑स्य॒ वज्र॑स्य रू॒पꣳ समृ॑द्ध्यै । \newline
58. रू॒पꣳ समृ॑द्ध्यै॒ समृ॑द्ध्यै रू॒पꣳ रू॒पꣳ समृ॑द्ध्यै । \newline
59. समृ॑द्ध्या॒ इति॒ सं - ऋ॒द्ध्यै॒ । \newline
\pagebreak


\end{document}