\documentclass[17pt]{extarticle}
\usepackage{babel}
\usepackage{fontspec}
\usepackage{polyglossia}
\usepackage{extsizes}

\usepackage{color}   %May be necessary if you want to color links
\usepackage{hyperref}
\hypersetup{
    colorlinks=true, %set true if you want colored links
    linktoc=all,     %set to all if you want both sections and subsections linked
    linkcolor=black,  %choose some color if you want links to stand out
}

\setmainlanguage{sanskrit}
\setotherlanguages{english} %% or other languages
\setlength{\parindent}{0pt}
\pagestyle{myheadings}
\newfontfamily\devanagarifont[Script=Devanagari]{AdishilaVedic}
\renewcommand{\theHsection}{\thepart.section.\thesection}

\newcommand{\VAR}[1]{}
\newcommand{\BLOCK}[1]{}




\begin{document}
\begin{titlepage}
    \begin{center}
 
\begin{sanskrit}
    { \Large
    कृष्ण यजुर्वेदीय तैत्तिरीय संहिता,पद,जटा,घन पाठः 
    }
    \\
    \vspace{2.5cm}
    \mbox{ \Large
    6.3      षष्ठकाण्डे तृतीयः प्रश्नः - सोममन्त्रब्राह्मणनिरूपणं   }
\end{sanskrit}
\end{center}

\end{titlepage}
\tableofcontents
\phantomsection
\pagebreak

\markright{ TS 6.3.1.1  \hfill https://www.vedavms.in \hfill}

\section{ TS 6.3.1.1 }

\textbf{TS 6.3.1.1 } \newline
\textbf{Samhita Paata} \newline

चात्वा॑ला॒द्-धिष्णि॑या॒नुप॑ वपति॒ योनि॒र्वै य॒ज्ञ्स्य॒ चात्वा॑लं ॅय॒ज्ञ्स्य॑ सयोनि॒त्वाय॑ दे॒वा वै य॒ज्ञ्ं परा॑ऽजयन्त॒ तमाग्नी᳚द्ध्रा॒त् पुन॒रपा॑जयन्ने॒तद्वै य॒ज्ञ्स्या-प॑राजितं॒ ॅयदाग्नी᳚द्ध्रं॒ ॅयदाग्नी᳚द्ध्रा॒द्धिष्णि॑यान्. वि॒हर॑ति॒ यदे॒व  य॒ज्ञ्स्या-प॑राजितं॒ तत॑ ए॒वैनं॒ पुन॑स्तनुते परा॒जित्ये॑व॒ खलु॒ वा ए॒ते य॑न्ति॒ ये ब॑हिष्पवमा॒नꣳ सर्प॑न्ति बहिष्पवमा॒ने स्तु॒त- [  ] \newline

\textbf{Pada Paata} \newline

चात्वा॑लात् । धिष्णि॑यान् । उपेति॑ । व॒प॒ति॒ । योनिः॑ । वै । य॒ज्ञ्स्य॑ । चात्वा॑लम् । य॒ज्ञ्स्य॑ । स॒यो॒नि॒त्वायेति॑ सयोनि - त्वाय॑ । दे॒वाः । वै । य॒ज्ञ्म् । परेति॑ । अ॒ज॒य॒न्त॒ । तम् । आग्नी᳚᳚द्ध्रा॒दित्याग्नि॑ - इ॒द्ध्रा॒त् । पुनः॑ । अपेति॑ । अ॒ज॒य॒न्न् । ए॒तत् । वै । य॒ज्ञ्स्य॑ । अप॑राजित॒मित्यप॑रा - जि॒त॒म् । यत् । आग्नी᳚द्ध्र॒मित्याग्नि॑ - इ॒द्ध्र॒म् । यत् । आग्नी᳚द्ध्रा॒दित्याग्नि॑ - इ॒द्ध्रा॒त् । धिष्णि॑यान् । वि॒हर॒तीति॑ वि-हर॑ति । यत् । ए॒व । य॒ज्ञ्स्य॑ । अप॑राजित॒मित्यप॑रा - जि॒त॒म् । ततः॑ । ए॒व । ए॒न॒म् । पुनः॑ । त॒नु॒ते॒ । प॒रा॒जित्येति॑ परा-जित्य॑ । इ॒व॒ । खलु॑ । वै । ए॒ते । य॒न्ति॒ । ये । ब॒हि॒ष्प॒व॒मा॒नमिति॑ बहिः - प॒व॒मा॒नम् । सर्प॑न्ति । ब॒हि॒ष्प॒व॒मा॒न इति॑ बहिः - प॒व॒मा॒ने । स्तु॒ते ।  \newline


\textbf{Krama Paata} \newline

चात्वा॑ला॒द् धिष्णि॑यान् । धिष्णि॑या॒नुप॑ । उप॑ वपति । व॒प॒ति॒ योनिः॑ । योनि॒र् वै । वै य॒ज्ञ्स्य॑ । य॒ज्ञ्स्य॒ चात्वा॑लम् । चात्वा॑लम् ॅय॒ज्ञ्स्य॑ । य॒ज्ञ्स्य॑ सयोनि॒त्वाय॑ । स॒यो॒नि॒त्वाय॑ दे॒वाः । स॒यो॒नि॒त्वायेति॑ सयोनि - त्वाय॑ । दे॒वा वै । वै य॒ज्ञ्म् । य॒ज्ञ्म् परा᳚ । परा॑ऽजयन्त । अ॒ज॒य॒न्त॒ तम् । तमाग्नी᳚द्ध्रात् । आग्नी᳚द्ध्रा॒त् पुनः॑ । आग्नी᳚द्ध्रा॒दित्याग्नि॑ - इ॒द्ध्रा॒त्॒ । पुन॒रप॑ । अपा॑जयन्न् । अ॒ज॒य॒न्ने॒तत् । ए॒तद् वै । वै य॒ज्ञ्स्य॑ । य॒ज्ञ्स्याप॑राजितम् । अप॑राजित॒म् ॅयत् । अप॑राजित॒मित्यप॑रा - जि॒त॒म् । यदाग्नी᳚द्ध्रम् । आग्नी᳚द्ध्र॒म् ॅयत् । आग्नी᳚द्ध्र॒मित्याग्नि॑ - इ॒द्ध्र॒म् । यदाग्नी᳚द्ध्रात् । आग्नी᳚द्ध्रा॒द् धिष्णि॑यान् । आग्नी᳚द्ध्रा॒दित्याग्नि॑ - इ॒द्ध्रा॒त्॒ । धिष्णि॑यान्. वि॒हर॑ति । वि॒हर॑ति॒ यत् । वि॒हर॒तीति॑ वि - हर॑ति । यदे॒व । ए॒व य॒ज्ञ्स्य॑ । य॒ज्ञ्स्याप॑राजितम् । अप॑राजित॒म् ततः॑ । अप॑राजित॒मित्यप॑रा - जि॒त॒म् । तत॑ ए॒व । ए॒वैन᳚म् । ए॒न॒म् पुनः॑ । पुन॑स्तनुते । त॒नु॒ते॒ प॒रा॒जित्य॑ । प॒रा॒जित्ये॑व । प॒रा॒जित्येति॑ परा - जित्य॑ । इ॒व॒ खलु॑ । खलु॒ वै । वा ए॒ते । ए॒ते य॑न्ति । य॒न्ति॒ ये । ये ब॑हिष्पवमा॒नम् । ब॒हि॒ष्प॒व॒मा॒नꣳ सर्प॑न्ति । ब॒हि॒ष्प॒व॒मा॒नमिति॑ बहिः - प॒व॒मा॒नम् । सर्प॑न्ति बहिष्पवमा॒ने । ब॒हि॒ष्प॒व॒मा॒ने स्तु॒ते । ब॒हि॒ष्प॒व॒मा॒न इति॑ बहिः - प॒व॒मा॒ने । स्तु॒त आ॑ह \newline

\textbf{Jatai Paata} \newline

1. चात्वा॑ला॒द् धिष्णि॑या॒न् धिष्णि॑याꣳ॒॒ श्चात्वा॑ला॒च् चात्वा॑ला॒द् धिष्णि॑यान् । \newline
2. धिष्णि॑या॒ नुपोप॒ धिष्णि॑या॒न् धिष्णि॑या॒नुप॑ । \newline
3. उप॑ वपति वप॒ त्युपोप॑ वपति । \newline
4. व॒प॒ति॒ योनि॒र् योनि॑र् वपति वपति॒ योनिः॑ । \newline
5. योनि॒र् वै वै योनि॒र् योनि॒र् वै । \newline
6. वै य॒ज्ञ्स्य॑ य॒ज्ञ्स्य॒ वै वै य॒ज्ञ्स्य॑ । \newline
7. य॒ज्ञ्स्य॒ चात्वा॑ल॒म् चात्वा॑लं ॅय॒ज्ञ्स्य॑ य॒ज्ञ्स्य॒ चात्वा॑लम् । \newline
8. चात्वा॑लं ॅय॒ज्ञ्स्य॑ य॒ज्ञ्स्य॒ चात्वा॑ल॒म् चात्वा॑लं ॅय॒ज्ञ्स्य॑ । \newline
9. य॒ज्ञ्स्य॑ सयोनि॒त्वाय॑ सयोनि॒त्वाय॑ य॒ज्ञ्स्य॑ य॒ज्ञ्स्य॑ सयोनि॒त्वाय॑ । \newline
10. स॒यो॒नि॒त्वाय॑ दे॒वा दे॒वाः स॑योनि॒त्वाय॑ सयोनि॒त्वाय॑ दे॒वाः । \newline
11. स॒यो॒नि॒त्वायेति॑ सयोनि - त्वाय॑ । \newline
12. दे॒वा वै वै दे॒वा दे॒वा वै । \newline
13. वै य॒ज्ञ्ं ॅय॒ज्ञ्ं ॅवै वै य॒ज्ञ्म् । \newline
14. य॒ज्ञ्म् परा॒ परा॑ य॒ज्ञ्ं ॅय॒ज्ञ्म् परा᳚ । \newline
15. परा॑ ऽजयन्ता जयन्त॒ परा॒ परा॑ ऽजयन्त । \newline
16. अ॒ज॒य॒न्त॒ तम् त म॑जयन्ता जयन्त॒ तम् । \newline
17. त माग्नी᳚द्ध्रा॒ दाग्नी᳚द्ध्रा॒त् तम् त माग्नी᳚द्ध्रात् । \newline
18. आग्नी᳚द्ध्रा॒त् पुनः॒ पुन॒ राग्नी᳚द्ध्रा॒ दाग्नी᳚द्ध्रा॒त् पुनः॑ । \newline
19. आग्नी᳚द्ध्रा॒दित्याग्नि॑ - इ॒द्ध्रा॒त् । \newline
20. पुन॒ रपाप॒ पुनः॒ पुन॒ रप॑ । \newline
21. अपा॑ जयन् नजय॒न् नपापा॑ जयन्न् । \newline
22. अ॒ज॒य॒न् ने॒त दे॒त द॑जयन् नजयन् ने॒तत् । \newline
23. ए॒तद् वै वा ए॒त दे॒तद् वै । \newline
24. वै य॒ज्ञ्स्य॑ य॒ज्ञ्स्य॒ वै वै य॒ज्ञ्स्य॑ । \newline
25. य॒ज्ञ्स्या प॑राजित॒ मप॑राजितं ॅय॒ज्ञ्स्य॑ य॒ज्ञ्स्या प॑राजितम् । \newline
26. अप॑राजितं॒ ॅयद् यदप॑राजित॒ मप॑राजितं॒ ॅयत् । \newline
27. अप॑राजित॒मित्यप॑रा - जि॒त॒म् । \newline
28. यदाग्नी᳚द्ध्र॒ माग्नी᳚द्ध्रं॒ ॅयद् यदाग्नी᳚द्ध्रम् । \newline
29. आग्नी᳚द्ध्रं॒ ॅयद् यदाग्नी᳚द्ध्र॒ माग्नी᳚द्ध्रं॒ ॅयत् । \newline
30. आग्नी᳚द्ध्र॒मित्याग्नि॑ - इ॒द्ध्र॒म् । \newline
31. यदाग्नी᳚द्ध्रा॒ दाग्नी᳚द्ध्रा॒द् यद् यदाग्नी᳚द्ध्रात् । \newline
32. आग्नी᳚द्ध्रा॒द् धिष्णि॑या॒न् धिष्णि॑या॒ नाग्नी᳚द्ध्रा॒ दाग्नी᳚द्ध्रा॒द् धिष्णि॑यान् । \newline
33. आग्नी᳚द्ध्रा॒दित्याग्नि॑ - इ॒द्ध्रा॒त् । \newline
34. धिष्णि॑यान्. वि॒हर॑ति वि॒हर॑ति॒ धिष्णि॑या॒न् धिष्णि॑यान्. वि॒हर॑ति । \newline
35. वि॒हर॑ति॒ यद् यद् वि॒हर॑ति वि॒हर॑ति॒ यत् । \newline
36. वि॒हर॒तीति॑ वि - हर॑ति । \newline
37. यदे॒ वैव यद् यदे॒व । \newline
38. ए॒व य॒ज्ञ्स्य॑ य॒ज्ञ्स्यै॒वैव य॒ज्ञ्स्य॑ । \newline
39. य॒ज्ञ्स्या प॑राजित॒ मप॑राजितं ॅय॒ज्ञ्स्य॑ य॒ज्ञ्स्या प॑राजितम् । \newline
40. अप॑राजित॒म् तत॒ स्ततो ऽप॑राजित॒ मप॑राजित॒म् ततः॑ । \newline
41. अप॑राजित॒मित्यप॑रा - जि॒त॒म् । \newline
42. तत॑ ए॒वैव तत॒ स्तत॑ ए॒व । \newline
43. ए॒वैन॑ मेन मे॒वै वैन᳚म् । \newline
44. ए॒न॒म् पुनः॒ पुन॑ रेन मेन॒म् पुनः॑ । \newline
45. पुन॑ स्तनुते तनुते॒ पुनः॒ पुन॑ स्तनुते । \newline
46. त॒नु॒ते॒ प॒रा॒जित्य॑ परा॒जित्य॑ तनुते तनुते परा॒जित्य॑ । \newline
47. प॒रा॒जि त्ये॑वेव परा॒जित्य॑ परा॒जित्ये॑व । \newline
48. प॒रा॒जित्येति॑ परा - जित्य॑ । \newline
49. इ॒व॒ खलु॒ खल्वि॑वेव॒ खलु॑ । \newline
50. खलु॒ वै वै खलु॒ खलु॒ वै । \newline
51. वा ए॒त ए॒ते वै वा ए॒ते । \newline
52. ए॒ते य॑न्ति यन्त्ये॒त ए॒ते य॑न्ति । \newline
53. य॒न्ति॒ ये ये य॑न्ति यन्ति॒ ये । \newline
54. ये ब॑हिष्पवमा॒नम् ब॑हिष्पवमा॒नं ॅये ये ब॑हिष्पवमा॒नम् । \newline
55. ब॒हि॒ष्प॒व॒मा॒नꣳ सर्प॑न्ति॒ सर्प॑न्ति बहिष्पवमा॒नम् ब॑हिष्पवमा॒नꣳ सर्प॑न्ति । \newline
56. ब॒हि॒ष्प॒व॒मा॒नमिति॑ बहिः - प॒व॒मा॒नम् । \newline
57. सर्प॑न्ति बहिष्पवमा॒ने ब॑हिष्पवमा॒ने सर्प॑न्ति॒ सर्प॑न्ति बहिष्पवमा॒ने । \newline
58. ब॒हि॒ष्प॒व॒मा॒ने स्तु॒ते स्तु॒ते ब॑हिष्पवमा॒ने ब॑हिष्पवमा॒ने स्तु॒ते । \newline
59. ब॒हि॒ष्प॒व॒मा॒न इति॑ बहिः - प॒व॒मा॒ने । \newline
60. स्तु॒त आ॑हाह स्तु॒ते स्तु॒त आ॑ह । \newline

\textbf{Ghana Paata } \newline

1. चात्वा॑ला॒द् धिष्णि॑या॒न् धिष्णि॑याꣳ॒॒ श्चात्वा॑ला॒च् चात्वा॑ला॒द् धिष्णि॑या॒ नुपोप॒ धिष्णि॑याꣳ॒॒ श्चात्वा॑ला॒च् चात्वा॑ला॒द् धिष्णि॑या॒ नुप॑ । \newline
2. धिष्णि॑या॒ नुपोप॒ धिष्णि॑या॒न् धिष्णि॑या॒ नुप॑ वपति वप॒ त्युप॒ धिष्णि॑या॒न् धिष्णि॑या॒ नुप॑ वपति । \newline
3. उप॑ वपति वप॒ त्युपोप॑ वपति॒ योनि॒र् योनि॑र् वप॒ त्युपोप॑ वपति॒ योनिः॑ । \newline
4. व॒प॒ति॒ योनि॒र् योनि॑र् वपति वपति॒ योनि॒र् वै वै योनि॑र् वपति वपति॒ योनि॒र् वै । \newline
5. योनि॒र् वै वै योनि॒र् योनि॒र् वै य॒ज्ञ्स्य॑ य॒ज्ञ्स्य॒ वै योनि॒र् योनि॒र् वै य॒ज्ञ्स्य॑ । \newline
6. वै य॒ज्ञ्स्य॑ य॒ज्ञ्स्य॒ वै वै य॒ज्ञ्स्य॒ चात्वा॑ल॒म् चात्वा॑लं ॅय॒ज्ञ्स्य॒ वै वै य॒ज्ञ्स्य॒ चात्वा॑लम् । \newline
7. य॒ज्ञ्स्य॒ चात्वा॑ल॒म् चात्वा॑लं ॅय॒ज्ञ्स्य॑ य॒ज्ञ्स्य॒ चात्वा॑लं ॅय॒ज्ञ्स्य॑ य॒ज्ञ्स्य॒ चात्वा॑लं ॅय॒ज्ञ्स्य॑ य॒ज्ञ्स्य॒ चात्वा॑लं ॅय॒ज्ञ्स्य॑ । \newline
8. चात्वा॑लं ॅय॒ज्ञ्स्य॑ य॒ज्ञ्स्य॒ चात्वा॑ल॒म् चात्वा॑लं ॅय॒ज्ञ्स्य॑ सयोनि॒त्वाय॑ सयोनि॒त्वाय॑ य॒ज्ञ्स्य॒ चात्वा॑ल॒म् चात्वा॑लं ॅय॒ज्ञ्स्य॑ सयोनि॒त्वाय॑ । \newline
9. य॒ज्ञ्स्य॑ सयोनि॒त्वाय॑ सयोनि॒त्वाय॑ य॒ज्ञ्स्य॑ य॒ज्ञ्स्य॑ सयोनि॒त्वाय॑ दे॒वा दे॒वाः स॑योनि॒त्वाय॑ य॒ज्ञ्स्य॑ य॒ज्ञ्स्य॑ सयोनि॒त्वाय॑ दे॒वाः । \newline
10. स॒यो॒नि॒त्वाय॑ दे॒वा दे॒वाः स॑योनि॒त्वाय॑ सयोनि॒त्वाय॑ दे॒वा वै वै दे॒वाः स॑योनि॒त्वाय॑ सयोनि॒त्वाय॑ दे॒वा वै । \newline
11. स॒यो॒नि॒त्वायेति॑ सयोनि - त्वाय॑ । \newline
12. दे॒वा वै वै दे॒वा दे॒वा वै य॒ज्ञ्ं ॅय॒ज्ञ्ं ॅवै दे॒वा दे॒वा वै य॒ज्ञ्म् । \newline
13. वै य॒ज्ञ्ं ॅय॒ज्ञ्ं ॅवै वै य॒ज्ञ्म् परा॒ परा॑ य॒ज्ञ्ं ॅवै वै य॒ज्ञ्म् परा᳚ । \newline
14. य॒ज्ञ्म् परा॒ परा॑ य॒ज्ञ्ं ॅय॒ज्ञ्म् परा॑ ऽजयन्ता जयन्त॒ परा॑ य॒ज्ञ्ं ॅय॒ज्ञ्म् परा॑ ऽजयन्त । \newline
15. परा॑ ऽजयन्ता जयन्त॒ परा॒ परा॑ ऽजयन्त॒ तम् त म॑जयन्त॒ परा॒ परा॑ ऽजयन्त॒ तम् । \newline
16. अ॒ज॒य॒न्त॒ तम् त म॑जयन्ता जयन्त॒ त माग्नी᳚द्ध्रा॒ दाग्नी᳚द्ध्रा॒त् त म॑जयन्ता जयन्त॒ तमाग्नी᳚द्ध्रात् । \newline
17. तमाग्नी᳚द्ध्रा॒ दाग्नी᳚द्ध्रा॒त् तम् तमाग्नी᳚द्ध्रा॒त् पुनः॒ पुन॒ राग्नी᳚द्ध्रा॒त् तम् तमाग्नी᳚द्ध्रा॒त् पुनः॑ । \newline
18. आग्नी᳚द्ध्रा॒त् पुनः॒ पुन॒ राग्नी᳚द्ध्रा॒ दाग्नी᳚द्ध्रा॒त् पुन॒ रपाप॒ पुन॒ राग्नी᳚द्ध्रा॒ दाग्नी᳚द्ध्रा॒त् पुन॒ रप॑ । \newline
19. आग्नी᳚᳚द्ध्रा॒दित्याग्नि॑ - इ॒द्ध्रा॒त् । \newline
20. पुन॒ रपाप॒ पुनः॒ पुन॒ रपा॑जयन् नजय॒न् नप॒ पुनः॒ पुन॒ रपा॑ जयन्न् । \newline
21. अपा॑जयन् नजय॒न् नपापा॑ जयन् ने॒त दे॒त द॑जय॒न् नपापा॑ जयन् ने॒तत् । \newline
22. अ॒ज॒य॒न् ने॒त दे॒त द॑जयन् नजयन् ने॒तद् वै वा ए॒त द॑जयन् नजयन् ने॒तद् वै । \newline
23. ए॒तद् वै वा ए॒त दे॒तद् वै य॒ज्ञ्स्य॑ य॒ज्ञ्स्य॒ वा ए॒त दे॒तद् वै य॒ज्ञ्स्य॑ । \newline
24. वै य॒ज्ञ्स्य॑ य॒ज्ञ्स्य॒ वै वै य॒ज्ञ्स्या प॑राजित॒ मप॑राजितं ॅय॒ज्ञ्स्य॒ वै वै य॒ज्ञ्स्या प॑राजितम् । \newline
25. य॒ज्ञ्स्या प॑राजित॒ मप॑राजितं ॅय॒ज्ञ्स्य॑ य॒ज्ञ्स्या प॑राजितं॒ ॅयद् यदप॑राजितं ॅय॒ज्ञ्स्य॑ य॒ज्ञ्स्या प॑राजितं॒ ॅयत् । \newline
26. अप॑राजितं॒ ॅयद् यदप॑राजित॒ मप॑राजितं॒ ॅयदाग्नी᳚द्ध्र॒ माग्नी᳚द्ध्रं॒ ॅयदप॑राजित॒ मप॑राजितं॒ ॅयदाग्नी᳚द्ध्रम् । \newline
27. अप॑राजित॒मित्यप॑रा - जि॒त॒म् । \newline
28. यदाग्नी᳚द्ध्र॒ माग्नी᳚द्ध्रं॒ ॅयद् यदाग्नी᳚द्ध्रं॒ ॅयद् यदाग्नी᳚द्ध्रं॒ ॅयद् यदाग्नी᳚द्ध्रं॒ ॅयत् । \newline
29. आग्नी᳚द्ध्रं॒ ॅयद् यदाग्नी᳚द्ध्र॒ माग्नी᳚द्ध्रं॒ ॅयदाग्नी᳚द्ध्रा॒ दाग्नी᳚द्ध्रा॒द् यदाग्नी᳚द्ध्र॒ माग्नी᳚द्ध्रं॒ ॅयदाग्नी᳚द्ध्रात् । \newline
30. आग्नी᳚द्ध्र॒मित्याग्नि॑ - इ॒द्ध्र॒म् । \newline
31. यदाग्नी᳚द्ध्रा॒ दाग्नी᳚द्ध्रा॒द् यद् यदाग्नी᳚द्ध्रा॒द् धिष्णि॑या॒न् धिष्णि॑या॒ नाग्नी᳚द्ध्रा॒द् यद् यदाग्नी᳚द्ध्रा॒द् धिष्णि॑यान् । \newline
32. आग्नी᳚द्ध्रा॒द् धिष्णि॑या॒न् धिष्णि॑या॒ नाग्नी᳚द्ध्रा॒ दाग्नी᳚द्ध्रा॒द् धिष्णि॑यान्. वि॒हर॑ति वि॒हर॑ति॒ धिष्णि॑या॒ नाग्नी᳚द्ध्रा॒ दाग्नी᳚द्ध्रा॒द् धिष्णि॑यान्. वि॒हर॑ति । \newline
33. आग्नी᳚द्ध्रा॒दित्याग्नि॑ - इ॒द्ध्रा॒त् । \newline
34. धिष्णि॑यान्. वि॒हर॑ति वि॒हर॑ति॒ धिष्णि॑या॒न् धिष्णि॑यान्. वि॒हर॑ति॒ यद् यद् वि॒हर॑ति॒ धिष्णि॑या॒न् धिष्णि॑यान्. वि॒हर॑ति॒ यत् । \newline
35. वि॒हर॑ति॒ यद् यद् वि॒हर॑ति वि॒हर॑ति॒ यदे॒वैव यद् वि॒हर॑ति वि॒हर॑ति॒ यदे॒व । \newline
36. वि॒हर॒तीति॑ वि - हर॑ति । \newline
37. यदे॒वैव यद् यदे॒व य॒ज्ञ्स्य॑ य॒ज्ञ् स्यै॒व यद् यदे॒व य॒ज्ञ्स्य॑ । \newline
38. ए॒व य॒ज्ञ्स्य॑ य॒ज्ञ् स्यै॒वैव य॒ज्ञ्स्या प॑राजित॒ मप॑राजितं ॅय॒ज्ञ् स्यै॒वैव य॒ज्ञ्स्या प॑राजितम् । \newline
39. य॒ज्ञ्स्या प॑राजित॒ मप॑राजितं ॅय॒ज्ञ्स्य॑ य॒ज्ञ्स्या प॑राजित॒म् तत॒ स्ततो ऽप॑राजितं ॅय॒ज्ञ्स्य॑ य॒ज्ञ्स्या प॑राजित॒म् ततः॑ । \newline
40. अप॑राजित॒म् तत॒ स्ततो ऽप॑राजित॒ मप॑राजित॒म् तत॑ ए॒वैव ततो ऽप॑राजित॒ मप॑राजित॒म् तत॑ ए॒व । \newline
41. अप॑राजित॒मित्यप॑रा - जि॒त॒म् । \newline
42. तत॑ ए॒वैव तत॒ स्तत॑ ए॒वैन॑ मेन मे॒व तत॒ स्तत॑ ए॒वैन᳚म् । \newline
43. ए॒वैन॑ मेन मे॒वै वैन॒म् पुनः॒ पुन॑ रेन मे॒वै वैन॒म् पुनः॑ । \newline
44. ए॒न॒म् पुनः॒ पुन॑ रेन मेन॒म् पुन॑ स्तनुते तनुते॒ पुन॑ रेन मेन॒म् पुन॑ स्तनुते । \newline
45. पुन॑ स्तनुते तनुते॒ पुनः॒ पुन॑ स्तनुते परा॒जित्य॑ परा॒जित्य॑ तनुते॒ पुनः॒ पुन॑ स्तनुते परा॒जित्य॑ । \newline
46. त॒नु॒ते॒ प॒रा॒जित्य॑ परा॒जित्य॑ तनुते तनुते परा॒जि त्ये॑वेव परा॒जित्य॑ तनुते तनुते परा॒जि त्ये॑व । \newline
47. प॒रा॒जि त्ये॑वेव परा॒जित्य॑ परा॒जि त्ये॑व॒ खलु॒ खल्वि॑व परा॒जित्य॑ परा॒जि त्ये॑व॒ खलु॑ । \newline
48. प॒रा॒जित्येति॑ परा - जित्य॑ । \newline
49. इ॒व॒ खलु॒ खल्वि॑वेव॒ खलु॒ वै वै खल्वि॑वे व॒ खलु॒ वै । \newline
50. खलु॒ वै वै खलु॒ खलु॒ वा ए॒त ए॒ते वै खलु॒ खलु॒ वा ए॒ते । \newline
51. वा ए॒त ए॒ते वै वा ए॒ते य॑न्ति यन्त्ये॒ते वै वा ए॒ते य॑न्ति । \newline
52. ए॒ते य॑न्ति यन्त्ये॒त ए॒ते य॑न्ति॒ ये ये य॑न्त्ये॒त ए॒ते य॑न्ति॒ ये । \newline
53. य॒न्ति॒ ये ये य॑न्ति यन्ति॒ ये ब॑हिष्पवमा॒नम् ब॑हिष्पवमा॒नं ॅये य॑न्ति यन्ति॒ ये ब॑हिष्पवमा॒नम् । \newline
54. ये ब॑हिष्पवमा॒नम् ब॑हिष्पवमा॒नं ॅये ये ब॑हिष्पवमा॒नꣳ सर्प॑न्ति॒ सर्प॑न्ति बहिष्पवमा॒नं ॅये ये ब॑हिष्पवमा॒नꣳ सर्प॑न्ति । \newline
55. ब॒हि॒ष्प॒व॒मा॒नꣳ सर्प॑न्ति॒ सर्प॑न्ति बहिष्पवमा॒नम् ब॑हिष्पवमा॒नꣳ सर्प॑न्ति बहिष्पवमा॒ने ब॑हिष्पवमा॒ने सर्प॑न्ति बहिष्पवमा॒नम् ब॑हिष्पवमा॒नꣳ सर्प॑न्ति बहिष्पवमा॒ने । \newline
56. ब॒हि॒ष्प॒व॒मा॒नमिति॑ बहिः - प॒व॒मा॒नम् । \newline
57. सर्प॑न्ति बहिष्पवमा॒ने ब॑हिष्पवमा॒ने सर्प॑न्ति॒ सर्प॑न्ति बहिष्पवमा॒ने स्तु॒ते स्तु॒ते ब॑हिष्पवमा॒ने सर्प॑न्ति॒ सर्प॑न्ति बहिष्पवमा॒ने स्तु॒ते । \newline
58. ब॒हि॒ष्प॒व॒मा॒ने स्तु॒ते स्तु॒ते ब॑हिष्पवमा॒ने ब॑हिष्पवमा॒ने स्तु॒त आ॑हाह स्तु॒ते ब॑हिष्पवमा॒ने ब॑हिष्पवमा॒ने स्तु॒त आ॑ह । \newline
59. ब॒हि॒ष्प॒व॒मा॒न इति॑ बहिः - प॒व॒मा॒ने । \newline
60. स्तु॒त आ॑हाह स्तु॒ते स्तु॒त आ॒हाग्नी॒ दग्नी॑ दाह स्तु॒ते स्तु॒त आ॒हाग्नी᳚त् । \newline
\pagebreak
\markright{ TS 6.3.1.2  \hfill https://www.vedavms.in \hfill}

\section{ TS 6.3.1.2 }

\textbf{TS 6.3.1.2 } \newline
\textbf{Samhita Paata} \newline

आ॒हाग्नी॑द॒ग्नीन्. वि ह॑र ब॒र्॒.हिः स्तृ॑णाहि पुरो॒डाशाꣳ॒॒ अलं॑ कु॒र्विति॑ य॒ज्ञ्मे॒वाप॒जित्य॒ पुन॑स्तन्वा॒ना य॒न्त्यङ्गा॑रै॒र्द्वे सव॑ने॒ वि ह॑रति श॒लाका॑भि-स्तृ॒तीयꣳ॑ सशुक्र॒त्वायाथो॒ सं भ॑रत्ये॒वैन॒द्धिष्णि॑या॒ वा अ॒मुष्मि॑न् ॅलो॒के सोम॑मरक्ष॒न् तेभ्योऽधि॒ सोम॒माऽह॑र॒न् त म॑न्व॒वाय॒न्तं पर्य॑विश॒न्॒. य ए॒वं ॅवेद॑ वि॒न्दते॑- [  ] \newline

\textbf{Pada Paata} \newline

आ॒ह॒ । अग्नी॒दित्यग्नि॑ - इ॒त् । अ॒ग्नीन् । वीति॑ । ह॒र॒ । ब॒र॒.हिः । स्तृ॒णा॒हि॒ । पु॒रो॒डाशान्॑ । अल᳚म् । कु॒रु॒ । इति॑ । य॒ज्ञ्म् । ए॒व । अ॒प॒जित्येत्य॑प-जित्य॑ । पुनः॑ । त॒न्वा॒नाः । य॒न्ति॒ । अङ्गा॑रैः । द्वे इति॑ । सव॑ने॒ इति॑ । वीति॑ । ह॒र॒ति॒ । श॒लाका॑भिः । तृ॒तीय᳚म् । स॒शु॒क्र॒त्वायेति॑ सशुक्र - त्वाय॑ । अथो॒ इति॑ । समिति॑ । भ॒र॒ति॒ । ए॒व । ए॒न॒त् । धिष्णि॑याः । वै । अ॒मुष्मिन्न्॑ । लो॒के । सोम᳚म् । अ॒र॒क्ष॒न्न् । तेभ्यः॑ । अधीति॑ । सोम᳚म् । एति॑ । अ॒ह॒र॒न्न् । तम् । अ॒न्व॒वाय॒न्नित्य॑नु-अ॒वायन्न्॑ । तम् । परीति॑ । अ॒वि॒श॒न्न् । य । ए॒वम् । वेद॑ । वि॒न्दते᳚ ।  \newline


\textbf{Krama Paata} \newline

आ॒हाग्नी᳚त् । अग्नी॑द॒ग्नीत् । अग्नी॒दित्यग्नि॑ - इ॒त्॒ । अ॒ग्नीन्. वि । वि ह॑र । ह॒र॒ ब॒र्.॒हिः । ब॒र्.॒हि स्तृ॑णाहि । स्तृ॒णा॒हि॒ पु॒रो॒डाशान्॑ । पु॒रो॒डाशाꣳ॒॒ अल᳚म् । अल॑म् कुरु । कु॒र्विति॑ । इति॑ य॒ज्ञ्म् । य॒ज्ञ्मे॒व । ए॒वाप॒जित्य॑ । अ॒प॒जित्य॒ पुनः॑ । अ॒प॒जित्येत्य॑प - जित्य॑ । पुन॑स्तन्वा॒नाः । त॒न्वा॒ना य॑न्ति । य॒न्त्यङ्‍गा॑रैः । अङ्‍गा॑रै॒र् द्वे । द्वे सव॑ने । द्वे इति॒ द्वे । सव॑ने॒ वि । सव॑ने॒ इति॒ सव॑ने । वि ह॑रति । ह॒र॒ति॒ श॒लाका॑भिः । श॒लाका॑भिस्तृ॒तीय᳚म् । तृ॒तीयꣳ॑ सशुक्र॒त्वाय॑ । स॒शु॒क्र॒त्वायाथो᳚ । स॒शु॒क्र॒त्वायेति॑ सशुक्र - त्वाय॑ । अथो॒ सम् । अथो॒ इत्यथो᳚ । सम् भ॑रति । भ॒र॒त्ये॒व । ए॒वैन॑त् । ए॒न॒द् धिष्णि॑याः । धिष्णि॑या॒ वै । वा अ॒मुष्मिन्न्॑ । अ॒मुष्मि॑न् ॅलो॒के । लो॒के सोम᳚म् । सोम॑मरक्षन्न् । अ॒र॒क्ष॒न् तेभ्यः॑ । तेभ्योऽधि॑ । अधि॒ सोम᳚म् । सोम॒मा । आऽह॑रन्न् । 
अ॒ह॒र॒न् तम् । तम॑न्व॒वायन्न्॑ । अ॒न्व॒वाय॒न् तम् । अ॒न्व॒वाय॒न्नित्य॑नु - अ॒वायन्न्॑ । तम् परि॑ । पर्य॑विशन्न् । अ॒वि॒श॒न्.॒ यः । य ए॒वम् । ए॒वम् ॅवेद॑ । वेद॑ वि॒न्दते᳚ । वि॒न्दते॑ परिवे॒ष्टार᳚म् \newline

\textbf{Jatai Paata} \newline

1. आ॒हाग्नी॒ दग्नी॑ दाहा॒ हाग्नी᳚त् । \newline
2. अग्नी॑ द॒ग्नी न॒ग्नी नग्नी॒ दग्नी॑ द॒ग्नीन् । \newline
3. अग्नी॒दित्यग्नि॑ - इ॒त् । \newline
4. अ॒ग्नीन्. वि व्य॑ग्नी न॒ग्नीन्. वि । \newline
5. वि ह॑र हर॒ वि वि ह॑र । \newline
6. ह॒र॒ ब॒र्॒.हिर् ब॒र्॒.हिर्. ह॑र हर ब॒र्॒.हिः । \newline
7. ब॒र्॒.हिः स्तृ॑णाहि स्तृणाहि ब॒र्॒.हिर् ब॒र्॒.हिः स्तृ॑णाहि । \newline
8. स्तृ॒णा॒हि॒ पु॒रो॒डाशा᳚न् पुरो॒डाशा᳚न् थ्स्तृणाहि स्तृणाहि पुरो॒डाशान्॑ । \newline
9. पु॒रो॒डाशाꣳ॒॒ अल॒ मल॑म् पुरो॒डाशा᳚न् पुरो॒डाशाꣳ॒॒ अल᳚म् । \newline
10. अल॑म् कुरु कु॒र्वल॒ मल॑म् कुरु । \newline
11. कु॒र्वि तीति॑ कुरु कु॒र्विति॑ । \newline
12. इति॑ य॒ज्ञ्ं ॅय॒ज्ञ् मितीति॑ य॒ज्ञ्म् । \newline
13. य॒ज्ञ् मे॒वैव य॒ज्ञ्ं ॅय॒ज्ञ् मे॒व । \newline
14. ए॒वा प॒जित्या॑ प॒जि त्यै॒वैवा प॒जित्य॑ । \newline
15. अ॒प॒जित्य॒ पुनः॒ पुन॑ रप॒जित्या॑ प॒जित्य॒ पुनः॑ । \newline
16. अ॒प॒जित्येत्य॑प - जित्य॑ । \newline
17. पुन॑ स्तन्वा॒ना स्त॑न्वा॒नाः पुनः॒ पुन॑ स्तन्वा॒नाः । \newline
18. त॒न्वा॒ना य॑न्ति यन्ति तन्वा॒ना स्त॑न्वा॒ना य॑न्ति । \newline
19. य॒न्त्यङ्गा॑रै॒ रङ्गा॑रैर् यन्ति य॒न्त्यङ्गा॑रैः । \newline
20. अङ्गा॑रै॒र् द्वे द्वे अङ्गा॑रै॒ रङ्गा॑रै॒र् द्वे । \newline
21. द्वे सव॑ने॒ सव॑ने॒ द्वे द्वे सव॑ने । \newline
22. द्वे इति॒ द्वे । \newline
23. सव॑ने॒ वि वि सव॑ने॒ सव॑ने॒ वि । \newline
24. सव॑ने॒ इति॒ सव॑ने । \newline
25. वि ह॑रति हरति॒ वि वि ह॑रति । \newline
26. ह॒र॒ति॒ श॒लाका॑भिः श॒लाका॑भिर्. हरति हरति श॒लाका॑भिः । \newline
27. श॒लाका॑भि स्तृ॒तीय॑म् तृ॒तीयꣳ॑ श॒लाका॑भिः श॒लाका॑भि स्तृ॒तीय᳚म् । \newline
28. तृ॒तीयꣳ॑ सशुक्र॒त्वाय॑ सशुक्र॒त्वाय॑ तृ॒तीय॑म् तृ॒तीयꣳ॑ सशुक्र॒त्वाय॑ । \newline
29. स॒शु॒क्र॒त्वा याथो॒ अथो॑ सशुक्र॒त्वाय॑ सशुक्र॒त्वा याथो᳚ । \newline
30. स॒शु॒क्र॒त्वायेति॑ सशुक्र - त्वाय॑ । \newline
31. अथो॒ सꣳ स मथो॒ अथो॒ सम् । \newline
32. अथो॒ इत्यथो᳚ । \newline
33. सम् भ॑रति भरति॒ सꣳ सम् भ॑रति । \newline
34. भ॒र॒ त्ये॒वैव भ॑रति भर त्ये॒व । \newline
35. ए॒वैन॑ देन दे॒वै वैन॑त् । \newline
36. ए॒न॒द् धिष्णि॑या॒ धिष्णि॑या एन देन॒द् धिष्णि॑याः । \newline
37. धिष्णि॑या॒ वै वै धिष्णि॑या॒ धिष्णि॑या॒ वै । \newline
38. वा अ॒मुष्मि॑न् न॒मुष्मि॒न्॒. वै वा अ॒मुष्मिन्न्॑ । \newline
39. अ॒मुष्मि॑न् ॅलो॒के लो॒के॑ ऽमुष्मि॑न् न॒मुष्मि॑न् ॅलो॒के । \newline
40. लो॒के सोमꣳ॒॒ सोम॑म् ॅलो॒के लो॒के सोम᳚म् । \newline
41. सोम॑ मरक्षन् नरक्ष॒न् थ्सोमꣳ॒॒ सोम॑ मरक्षन्न् । \newline
42. अ॒र॒क्ष॒न् तेभ्य॒ स्तेभ्यो॑ ऽरक्षन् नरक्ष॒न् तेभ्यः॑ । \newline
43. तेभ्यो ऽध्यधि॒ तेभ्य॒ स्तेभ्यो ऽधि॑ । \newline
44. अधि॒ सोमꣳ॒॒ सोम॒ मध्यधि॒ सोम᳚म् । \newline
45. सोम॒ मा सोमꣳ॒॒ सोम॒ मा । \newline
46. आ ऽह॑रन् नहर॒न्ना ऽह॑रन्न् । \newline
47. अ॒ह॒र॒न् तम् त म॑हरन् नहर॒न् तम् । \newline
48. त म॑न्व॒वाय॑न् नन्व॒वाय॒न् तम् त म॑न्व॒वायन्न्॑ । \newline
49. अ॒न्व॒वाय॒न् तम् त म॑न्व॒वाय॑न् नन्व॒वाय॒न् तम् । \newline
50. अ॒न्व॒वाय॒न्नित्य॑नु - अ॒वायन्न्॑ । \newline
51. तम् परि॒ परि॒ तम् तम् परि॑ । \newline
52. पर्य॑विशन् नविश॒न् परि॒ पर्य॑विशन्न् । \newline
53. अ॒वि॒श॒न्॒. यो यो॑ ऽविशन् नविश॒न्॒. यः । \newline
54. य एव मे॒वं ॅयो य एवम् । \newline
55. ए॒वं ॅवेद॒ वेदै॒व मे॒वं ॅवेद॑ । \newline
56. वेद॑ वि॒न्दते॑ वि॒न्दते॒ वेद॒ वेद॑ वि॒न्दते᳚ । \newline
57. वि॒न्दते॑ परिवे॒ष्टार॑म् परिवे॒ष्टारं॑ ॅवि॒न्दते॑ वि॒न्दते॑ परिवे॒ष्टार᳚म् । \newline

\textbf{Ghana Paata } \newline

1. आ॒हाग्नी॒ दग्नी॑ दाहा॒ हाग्नी॑ द॒ग्नी न॒ग्नी नग्नी॑ दाहा॒हाग्नी॑ द॒ग्नीन् । \newline
2. अग्नी॑ द॒ग्नी न॒ग्नी नग्नी॒ दग्नी॑ द॒ग्नीन्. वि व्य॑ग्नी नग्नी॒ दग्नी॑ द॒ग्नीन्. वि । \newline
3. अग्नी॒दित्यग्नि॑ - इ॒त् । \newline
4. अ॒ग्नीन्. वि व्य॑ग्नी न॒ग्नीन्. वि ह॑र हर॒ व्य॑ग्नी न॒ग्नीन्. वि ह॑र । \newline
5. वि ह॑र हर॒ वि वि ह॑र ब॒र्॒.हिर् ब॒र्॒.हिर्. ह॑र॒ वि वि ह॑र ब॒र्॒.हिः । \newline
6. ह॒र॒ ब॒र्॒.हिर् ब॒र्॒.हिर्. ह॑र हर ब॒र्॒.हिः स्तृ॑णाहि स्तृणाहि ब॒र्॒.हिर्. ह॑र हर ब॒र्॒.हिः स्तृ॑णाहि । \newline
7. ब॒र्॒.हिः स्तृ॑णाहि स्तृणाहि ब॒र्॒.हिर् ब॒र्॒.हिः स्तृ॑णाहि पुरो॒डाशा᳚न् पुरो॒डाशा᳚न् थ्स्तृणाहि ब॒र्॒.हिर् ब॒र्॒.हिः स्तृ॑णाहि पुरो॒डाशान्॑ । \newline
8. स्तृ॒णा॒हि॒ पु॒रो॒डाशा᳚न् पुरो॒डाशा᳚न् थ्स्तृणाहि स्तृणाहि पुरो॒डाशाꣳ॒॒ अल॒ मल॑म् पुरो॒डाशा᳚न् थ्स्तृणाहि स्तृणाहि पुरो॒डाशाꣳ॒॒ अल᳚म् । \newline
9. पु॒रो॒डाशाꣳ॒॒ अल॒ मल॑म् पुरो॒डाशा᳚न् पुरो॒डाशाꣳ॒॒ अल॑म् कुरु कु॒र्वल॑म् पुरो॒डाशा᳚न् पुरो॒डाशाꣳ॒॒ अल॑म् कुरु । \newline
10. अल॑म् कुरु कु॒र्वल॒ मल॑म् कु॒र्वितीति॑ कु॒र्वल॒ मल॑म् कु॒र्विति॑ । \newline
11. कु॒र्वितीति॑ कुरु कु॒र्विति॑ य॒ज्ञ्ं ॅय॒ज्ञ् मिति॑ कुरु कु॒र्विति॑ य॒ज्ञ्म् । \newline
12. इति॑ य॒ज्ञ्ं ॅय॒ज्ञ् मितीति॑ य॒ज्ञ् मे॒वैव य॒ज्ञ् मितीति॑ य॒ज्ञ् मे॒व । \newline
13. य॒ज्ञ् मे॒वैव य॒ज्ञ्ं ॅय॒ज्ञ् मे॒वा प॒जित्या॑ प॒जित्यै॒व य॒ज्ञ्ं ॅय॒ज्ञ् मे॒वा प॒जित्य॑ । \newline
14. ए॒वा प॒जित्या॑ प॒जित्यै॒वैवा प॒जित्य॒ पुनः॒ पुन॑ रप॒जित्यै॒वैवा प॒जित्य॒ पुनः॑ । \newline
15. अ॒प॒जित्य॒ पुनः॒ पुन॑ रप॒जित्या॑ प॒जित्य॒ पुन॑ स्तन्वा॒ना स्त॑न्वा॒नाः पुन॑ रप॒जित्या॑ प॒जित्य॒ पुन॑ स्तन्वा॒नाः । \newline
16. अ॒प॒जित्येत्य॑प - जित्य॑ । \newline
17. पुन॑ स्तन्वा॒ना स्त॑न्वा॒नाः पुनः॒ पुन॑ स्तन्वा॒ना य॑न्ति यन्ति तन्वा॒नाः पुनः॒ पुन॑ स्तन्वा॒ना य॑न्ति । \newline
18. त॒न्वा॒ना य॑न्ति यन्ति तन्वा॒ना स्त॑न्वा॒ना य॒न्त्यङ्गा॑रै॒ रङ्गा॑रैर् यन्ति तन्वा॒ना स्त॑न्वा॒ना य॒न्त्यङ्गा॑रैः । \newline
19. य॒न्त्यङ्गा॑रै॒ रङ्गा॑रैर् यन्ति य॒न्त्यङ्गा॑रै॒र् द्वे द्वे अङ्गा॑रैर् यन्ति य॒न्त्यङ्गा॑रै॒र् द्वे । \newline
20. अङ्गा॑रै॒र् द्वे द्वे अङ्गा॑रै॒ रङ्गा॑रै॒र् द्वे सव॑ने॒ सव॑ने॒ द्वे अङ्गा॑रै॒ रङ्गा॑रै॒र् द्वे सव॑ने । \newline
21. द्वे सव॑ने॒ सव॑ने॒ द्वे द्वे सव॑ने॒ वि वि सव॑ने॒ द्वे द्वे सव॑ने॒ वि । \newline
22. द्वे इति॒ द्वे । \newline
23. सव॑ने॒ वि वि सव॑ने॒ सव॑ने॒ वि ह॑रति हरति॒ वि सव॑ने॒ सव॑ने॒ वि ह॑रति । \newline
24. सव॑ने॒ इति॒ सव॑ने । \newline
25. वि ह॑रति हरति॒ वि वि ह॑रति श॒लाका॑भिः श॒लाका॑भिर्. हरति॒ वि वि ह॑रति श॒लाका॑भिः । \newline
26. ह॒र॒ति॒ श॒लाका॑भिः श॒लाका॑भिर्. हरति हरति श॒लाका॑भि स्तृ॒तीय॑म् तृ॒तीयꣳ॑ श॒लाका॑भिर्. हरति हरति श॒लाका॑भि स्तृ॒तीय᳚म् । \newline
27. श॒लाका॑भि स्तृ॒तीय॑म् तृ॒तीयꣳ॑ श॒लाका॑भिः श॒लाका॑भि स्तृ॒तीयꣳ॑ सशुक्र॒त्वाय॑ सशुक्र॒त्वाय॑ तृ॒तीयꣳ॑ श॒लाका॑भिः श॒लाका॑भि स्तृ॒तीयꣳ॑ सशुक्र॒त्वाय॑ । \newline
28. तृ॒तीयꣳ॑ सशुक्र॒त्वाय॑ सशुक्र॒त्वाय॑ तृ॒तीय॑म् तृ॒तीयꣳ॑ सशुक्र॒त्वा याथो॒ अथो॑ सशुक्र॒त्वाय॑ तृ॒तीय॑म् तृ॒तीयꣳ॑ सशुक्र॒त्वा याथो᳚ । \newline
29. स॒शु॒क्र॒त्वा याथो॒ अथो॑ सशुक्र॒त्वाय॑ सशुक्र॒त्वा याथो॒ सꣳ स मथो॑ सशुक्र॒त्वाय॑ सशुक्र॒त्वा याथो॒ सम् । \newline
30. स॒शु॒क्र॒त्वायेति॑ सशुक्र - त्वाय॑ । \newline
31. अथो॒ सꣳ समथो॒ अथो॒ सम् भ॑रति भरति॒ समथो॒ अथो॒ सम् भ॑रति । \newline
32. अथो॒ इत्यथो᳚ । \newline
33. सम् भ॑रति भरति॒ सꣳ सम् भ॑र त्ये॒वैव भ॑रति॒ सꣳ सम् भ॑र त्ये॒व । \newline
34. भ॒र॒ त्ये॒वैव भ॑रति भर त्ये॒वैन॑ देन दे॒व भ॑रति भर त्ये॒वैन॑त् । \newline
35. ए॒वैन॑ देन दे॒वै वैन॒द् धिष्णि॑या॒ धिष्णि॑या एन दे॒वै वैन॒द् धिष्णि॑याः । \newline
36. ए॒न॒द् धिष्णि॑या॒ धिष्णि॑या एन देन॒द् धिष्णि॑या॒ वै वै धिष्णि॑या एन देन॒द् धिष्णि॑या॒ वै । \newline
37. धिष्णि॑या॒ वै वै धिष्णि॑या॒ धिष्णि॑या॒ वा अ॒मुष्मि॑न् न॒मुष्मि॒न्॒. वै धिष्णि॑या॒ धिष्णि॑या॒ वा अ॒मुष्मिन्न्॑ । \newline
38. वा अ॒मुष्मि॑न् न॒मुष्मि॒न्॒. वै वा अ॒मुष्मि॑न् ॅलो॒के लो॒के॑ ऽमुष्मि॒न्॒. वै वा अ॒मुष्मि॑न् ॅलो॒के । \newline
39. अ॒मुष्मि॑न् ॅलो॒के लो॒के॑ ऽमुष्मि॑न् न॒मुष्मि॑न् ॅलो॒के सोमꣳ॒॒ सोम॑म् ॅलो॒के॑ ऽमुष्मि॑न् न॒मुष्मि॑न् ॅलो॒के सोम᳚म् । \newline
40. लो॒के सोमꣳ॒॒ सोम॑म् ॅलो॒के लो॒के सोम॑ मरक्षन् नरक्ष॒न् थ्सोम॑म् ॅलो॒के लो॒के सोम॑ मरक्षन्न् । \newline
41. सोम॑ मरक्षन् नरक्ष॒न् थ्सोमꣳ॒॒ सोम॑ मरक्ष॒न् तेभ्य॒ स्तेभ्यो॑ ऽरक्ष॒न् थ्सोमꣳ॒॒ सोम॑ मरक्ष॒न् तेभ्यः॑ । \newline
42. अ॒र॒क्ष॒न् तेभ्य॒ स्तेभ्यो॑ ऽरक्षन् नरक्ष॒न् तेभ्यो ऽध्यधि॒ तेभ्यो॑ ऽरक्षन् नरक्ष॒न् तेभ्यो ऽधि॑ । \newline
43. तेभ्यो ऽध्यधि॒ तेभ्य॒ स्तेभ्यो ऽधि॒ सोमꣳ॒॒ सोम॒ मधि॒ तेभ्य॒ स्तेभ्यो ऽधि॒ सोम᳚म् । \newline
44. अधि॒ सोमꣳ॒॒ सोम॒ मध्यधि॒ सोम॒ मा सोम॒ मध्यधि॒ सोम॒ मा । \newline
45. सोम॒ मा सोमꣳ॒॒ सोम॒ मा ऽह॑रन् नहर॒न् ना सोमꣳ॒॒ सोम॒ मा ऽह॑रन्न् । \newline
46. आ ऽह॑रन् नहर॒न् ना ऽह॑र॒न् तम् त म॑हर॒न् ना ऽह॑र॒न् तम् । \newline
47. अ॒ह॒र॒न् तम् त म॑हरन् नहर॒न् तम॑न्व॒वाय॑न् नन्व॒वाय॒न् त म॑हरन् नहर॒न् 
तम॑न्व॒वायन्न्॑ । \newline
48. तम॑न्व॒वाय॑न् नन्व॒वाय॒न् तम् तम॑न्व॒वाय॒न् तम् त म॑न्व॒वाय॒न् तम् तम॑न्व॒वाय॒न् तम् । \newline
49. अ॒न्व॒वाय॒न् तम् तम॑न्व॒वाय॑न् नन्व॒वाय॒न् तम् परि॒ परि॒ तम॑न्व॒वाय॑न् नन्व॒वाय॒न् तम् परि॑ । \newline
50. अ॒न्व॒वाय॒न्नित्य॑नु - अ॒वायन्न्॑ । \newline
51. तम् परि॒ परि॒ तम् तम् पर्य॑विशन् नविश॒न् परि॒ तम् तम् पर्य॑विशन्न् । \newline
52. पर्य॑विशन् नविश॒न् परि॒ पर्य॑विश॒न्॒. यो यो॑ विश॒न् परि॒ पर्य॑विश॒न्॒. यः । \newline
53. अ॒वि॒श॒न्॒. यो यो॑ ऽविशन् नविश॒न्॒. य एव मे॒वं ॅयो॑ ऽविशन् नविश॒न्॒. य एवम् । \newline
54. य एव मे॒वं ॅयो य एवं ॅवेद॒ वेदै॒वं ॅयो य एवं ॅवेद॑ । \newline
55. ए॒वं ॅवेद॒ वेदै॒व मे॒वं ॅवेद॑ वि॒न्दते॑ वि॒न्दते॒ वेदै॒व मे॒वं ॅवेद॑ वि॒न्दते᳚ । \newline
56. वेद॑ वि॒न्दते॑ वि॒न्दते॒ वेद॒ वेद॑ वि॒न्दते॑ परिवे॒ष्टार॑म् परिवे॒ष्टारं॑ ॅवि॒न्दते॒ वेद॒ वेद॑ वि॒न्दते॑ परिवे॒ष्टार᳚म् । \newline
57. वि॒न्दते॑ परिवे॒ष्टार॑म् परिवे॒ष्टारं॑ ॅवि॒न्दते॑ वि॒न्दते॑ परिवे॒ष्टार॒म् ते ते प॑रिवे॒ष्टारं॑ ॅवि॒न्दते॑ वि॒न्दते॑ परिवे॒ष्टार॒म् ते । \newline
\pagebreak
\markright{ TS 6.3.1.3  \hfill https://www.vedavms.in \hfill}

\section{ TS 6.3.1.3 }

\textbf{TS 6.3.1.3 } \newline
\textbf{Samhita Paata} \newline

परिवे॒ष्टारं॒ ते सो॑मपी॒थेन॒ व्या᳚र्द्ध्यन्त॒ ते दे॒वेषु॑ सोमपी॒थमै᳚च्छन्त॒ तान् दे॒वा अ॑ब्रुव॒न् द्वेद्वे॒ नाम॑नी कुरुद्ध्व॒मथ॒ प्र वा॒ऽऽफ्स्यथ॒ न वेत्य॒ग्नयो॒ वा अथ॒ धिष्णि॑या॒स्तस्मा᳚द् द्वि॒नामा᳚ ब्राह्म॒णोऽर्द्धु॑क॒स्तेषां॒ ॅये नेदि॑ष्ठं प॒र्यवि॑श॒न् ते सो॑मपी॒थं प्राऽ*प्नु॑वन्नाहव॒नीय॑ आग्नी॒द्ध्रीयो॑ हो॒त्रीयो॑ मार्जा॒लीय॒स्तस्मा॒त् तेषु॑ जुह्वत्यति॒हाय॒ वष॑ट् करोति॒ वि ह्ये॑- [  ] \newline

\textbf{Pada Paata} \newline

प॒रि॒वे॒ष्टार॒मिति॑ परि - वे॒ष्टार᳚म् । ते । सो॒म॒पी॒थेनेति॑ सोम - पी॒थेन॑ । वीति॑ । आ॒द्‌र्ध्य॒न्त॒ । ते । दे॒वेषु॑ । सो॒म॒पी॒थमिति॑ सोम - पी॒थम् । ऐ॒च्छ॒न्त॒ । तान् । दे॒वाः । अ॒ब्रु॒व॒न्न् । द्वे द्वे॒ इति॒ द्वे-द्वे॒ । नाम॑नी॒ इति॑ । कु॒रु॒द्ध्व॒म् । अथ॑ । प्रेति॑ । वा॒ । आ॒फ्स्यथ॑ । न । वा॒ । इति॑ । अ॒ग्नयः॑ । वै । अथ॑ । धिष्णि॑याः । तस्मा᳚त् । द्वि॒नामेति॑ द्वि - नामा᳚ । ब्रा॒ह्म॒णः । अद्‌र्धु॑कः । तेषा᳚म् । ये । नेदि॑ष्ठम् । प॒र्यवि॑श॒निति॑ परि - अवि॑शन्न् । ते । सो॒म॒पी॒थमिति॑ सोम - पी॒थम् । प्रेति॑ । आ॒प्नु॒व॒न्न् । आ॒ह॒व॒नीय॒ इत्या᳚ - ह॒व॒नीयः॑ । आ॒ग्नी॒द्ध्रीय॒ इत्या᳚ग्नि - इ॒ध्रीयः॑ । हो॒त्रीयः॑ । मा॒र्जा॒लीयः॑ । तस्मा᳚त् । तेषु॑ । जु॒ह्व॒ति॒ । अ॒ति॒हायेत्य॑ति-हाय॑ । वष॑ट् । क॒रो॒ति॒ । वीति॑ । हि ।  \newline


\textbf{Krama Paata} \newline

प॒रि॒वे॒ष्टार॒म् ते । प॒रि॒वे॒ष्टार॒मिति॑ परि - वे॒ष्टार᳚म् । ते सो॑मपी॒थेन॑ । सो॒म॒पी॒थेन॒ वि । सो॒म॒पी॒थेनेति॑ सोम - पी॒थेन॑ । व्या᳚र्द्ध्यन्त । आ॒र्द्ध्य॒न्त॒ ते । ते दे॒वेषु॑ । दे॒वेषु॑ सोमपी॒थम् । सो॒म॒पी॒थमै᳚च्छन्त । सो॒म॒पी॒थमिति॑ सोम - पी॒थम् । ऐ॒च्छ॒न्त॒ तान् । तान् दे॒वाः । दे॒वा अ॑ब्रुवन्न् । अ॒ब्रु॒व॒न् द्वेद्वे᳚ । द्वेद्वे॒ नाम॑नी । द्वेद्वे॒ इति॒ द्वे - द्वे॒ । नाम॑नी कुरुद्ध्वम् । नाम॑नी॒ इति॒ नाम॑नी । कु॒रु॒द्ध्व॒मथ॑ । अथ॒ प्र । प्रवा᳚ । वा॒ऽऽफ्स्यथ॑ । आ॒फ्स्यथ॒ न । न वा᳚ । वेति॑ । इत्य॒ग्नयः॑ । अ॒ग्नयो॒ वै । वा अथ॑ । अथ॒ धिष्णि॑याः । धिष्णि॑या॒स्तस्मा᳚त् । तस्मा᳚द् द्वि॒नामा᳚ । द्वि॒नामा᳚ ब्राह्म॒णः । द्वि॒नामेति॑ द्वि - नामा᳚ । ब्रा॒ह्म॒णोऽर्द्धु॑कः । अर्द्धु॑क॒स्तेषा᳚म् । तेषा॒म् ॅये । ये नेदि॑ष्ठम् । नेदि॑ष्ठम् प॒र्यवि॑शन्न् । प॒र्यवि॑श॒न् ते । प॒र्यवि॑श॒न्निति॑ परि - अवि॑शन्न् । ते सो॑मपी॒थम् । सो॒म॒पी॒थम् प्र । सो॒म॒पी॒थमिति॑ सोम - पी॒थम् । प्राप्नु॑वन्न् । आ॒प्नु॒व॒न्ना॒ह॒व॒नीयः॑ । आ॒ह॒व॒नीय॑ आग्नी॒द्ध्रीयः॑ । आ॒ह॒व॒नीय॒ इत्या᳚ - ह॒व॒नीयः॑ । आ॒ग्नी॒द्ध्रीयो॑ हो॒त्रीयः॑ । आ॒ग्नी॒द्ध्रीय॒ इत्या᳚ग्नि - इ॒द्ध्रीयः॑ । हो॒त्रीयो॑ मार्जा॒लीयः॑ । मा॒र्जा॒लीय॒स्तस्मा᳚त् । तस्मा॒त् तेषु॑ । तेषु॑ जुह्वति । जु॒ह्व॒त्य॒ति॒हाय॑ । अ॒ति॒हाय॒ वष॑ट् । अ॒ति॒हायेत्य॑ति - हाय॑ । वष॑ट् करोति । क॒रो॒ति॒ वि । वि हि । ह्ये॑ते \newline

\textbf{Jatai Paata} \newline

1. प॒रि॒वे॒ष्टार॒म् ते ते प॑रिवे॒ष्टार॑म् परिवे॒ष्टार॒म् ते । \newline
2. प॒रि॒वे॒ष्टार॒मिति॑ परि - वे॒ष्टार᳚म् । \newline
3. ते सो॑मपी॒थेन॑ सोमपी॒थेन॒ ते ते सो॑मपी॒थेन॑ । \newline
4. सो॒म॒पी॒थेन॒ वि वि सो॑मपी॒थेन॑ सोमपी॒थेन॒ वि । \newline
5. सो॒म॒पी॒थेनेति॑ सोम - पी॒थेन॑ । \newline
6. व्या᳚र्द्ध्यन्ता र्द्ध्यन्त॒ वि व्या᳚र्द्ध्यन्त । \newline
7. आ॒र्द्ध्य॒न्त॒ ते त आ᳚र्द्ध्यन्ता र्द्ध्यन्त॒ ते । \newline
8. ते दे॒वेषु॑ दे॒वेषु॒ ते ते दे॒वेषु॑ । \newline
9. दे॒वेषु॑ सोमपी॒थꣳ सो॑मपी॒थम् दे॒वेषु॑ दे॒वेषु॑ सोमपी॒थम् । \newline
10. सो॒म॒पी॒थ मै᳚च्छ न्तैच्छन्त सोमपी॒थꣳ सो॑मपी॒थ मै᳚च्छन्त । \newline
11. सो॒म॒पी॒थमिति॑ सोम - पी॒थम् । \newline
12. ऐ॒च्छ॒न्त॒ ताꣳ स्तानै᳚ च्छ न्तैच्छन्त॒ तान् । \newline
13. तान् दे॒वा दे॒वा स्ताꣳ स्तान् दे॒वाः । \newline
14. दे॒वा अ॑ब्रुवन् नब्रुवन् दे॒वा दे॒वा अ॑ब्रुवन्न् । \newline
15. अ॒ब्रु॒व॒न् द्वेद्वे॒ द्वेद्वे॑ अब्रुवन् नब्रुव॒न् द्वेद्वे᳚ । \newline
16. द्वेद्वे॒ नाम॑नी॒ नाम॑नी॒ द्वेद्वे॒ द्वेद्वे॒ नाम॑नी । \newline
17. द्वे‌द्वे॒ इति॒ द्वे - द्वे॒ । \newline
18. नाम॑नी कुरुद्ध्वम् कुरुद्ध्व॒म् नाम॑नी॒ नाम॑नी कुरुद्ध्वम् । \newline
19. नाम॑नी॒ इति॒ नाम॑नी । \newline
20. कु॒रु॒द्ध्व॒ मथाथ॑ कुरुद्ध्वम् कुरुद्ध्व॒ मथ॑ । \newline
21. अथ॒ प्र प्राथाथ॒ प्र । \newline
22. प्र वा॑ वा॒ प्र प्र वा᳚ । \newline
23. वा॒ ऽऽफ्स्यथा॒ फ्स्यथ॑ वा वा॒ ऽऽफ्स्यथ॑ । \newline
24. आ॒फ्स्यथ॒ न नाफ्स्यथा॒ फ्स्यथ॒ न । \newline
25. न वा॑ वा॒ न न वा᳚ । \newline
26. वेतीति॑ वा॒ वेति॑ । \newline
27. इत्य॒ग्नयो॒ ऽग्नय॒ इती त्य॒ग्नयः॑ । \newline
28. अ॒ग्नयो॒ वै वा अ॒ग्नयो॒ ऽग्नयो॒ वै । \newline
29. वा अथाथ॒ वै वा अथ॑ । \newline
30. अथ॒ धिष्णि॑या॒ धिष्णि॑या॒ अथाथ॒ धिष्णि॑याः । \newline
31. धिष्णि॑या॒ स्तस्मा॒त् तस्मा॒द् धिष्णि॑या॒ धिष्णि॑या॒ स्तस्मा᳚त् । \newline
32. तस्मा᳚द् द्वि॒नामा᳚ द्वि॒नामा॒ तस्मा॒त् तस्मा᳚द् द्वि॒नामा᳚ । \newline
33. द्वि॒नामा᳚ ब्राह्म॒णो ब्रा᳚ह्म॒णो द्वि॒नामा᳚ द्वि॒नामा᳚ ब्राह्म॒णः । \newline
34. द्वि॒नामेति॑ द्वि - नामा᳚ । \newline
35. ब्रा॒ह्म॒णो ऽर्द्धु॒को ऽर्द्धु॑को ब्राह्म॒णो ब्रा᳚ह्म॒णो ऽर्द्धु॑कः । \newline
36. अर्द्धु॑क॒ स्तेषा॒म् तेषा॒ मर्द्धु॒को ऽर्द्धु॑क॒ स्तेषा᳚म् । \newline
37. तेषां॒ ॅये ये तेषा॒म् तेषां॒ ॅये । \newline
38. ये नेदि॑ष्ठ॒म् नेदि॑ष्ठं॒ ॅये ये नेदि॑ष्ठम् । \newline
39. नेदि॑ष्ठम् प॒र्यवि॑शन् प॒र्यवि॑श॒न् नेदि॑ष्ठ॒न् नेदि॑ष्ठम् प॒र्यवि॑शन्न् । \newline
40. प॒र्यवि॑श॒न् ते ते प॒र्यवि॑शन् प॒र्यवि॑श॒न् ते । \newline
41. प॒र्यवि॑श॒न्निति॑ परि - अवि॑शन्न् । \newline
42. ते सो॑मपी॒थꣳ सो॑मपी॒थम् ते ते सो॑मपी॒थम् । \newline
43. सो॒म॒पी॒थम् प्र प्र सो॑मपी॒थꣳ सो॑मपी॒थम् प्र । \newline
44. सो॒म॒पी॒थमिति॑ सोम - पी॒थम् । \newline
45. प्राप्नु॑वन् नाप्नुव॒न् प्र प्राप्नु॑वन्न् । \newline
46. आ॒प्नु॒व॒न् ना॒ह॒व॒नीय॑ आहव॒नीय॑ आप्नुवन् नाप्नुवन् नाहव॒नीयः॑ । \newline
47. आ॒ह॒व॒नीय॑ आग्नी॒द्ध्रीय॑ आग्नी॒द्ध्रीय॑ आहव॒नीय॑ आहव॒नीय॑ आग्नी॒द्ध्रीयः॑ । \newline
48. आ॒ह॒व॒नीय॒ इत्या᳚ - ह॒व॒नीयः॑ । \newline
49. आ॒ग्नी॒द्ध्रीयो॑ हो॒त्रीयो॑ हो॒त्रीय॑ आग्नी॒द्ध्रीय॑ आग्नी॒द्ध्रीयो॑ हो॒त्रीयः॑ । \newline
50. आ॒ग्नी॒द्ध्रीय॒ इत्या᳚ग्नि - इ॒ध्रीयः॑ । \newline
51. हो॒त्रीयो॑ मार्जा॒लीयो॑ मार्जा॒लीयो॑ हो॒त्रीयो॑ हो॒त्रीयो॑ मार्जा॒लीयः॑ । \newline
52. मा॒र्जा॒लीय॒ स्तस्मा॒त् तस्मा᳚न् मार्जा॒लीयो॑ मार्जा॒लीय॒ स्तस्मा᳚त् । \newline
53. तस्मा॒त् तेषु॒ तेषु॒ तस्मा॒त् तस्मा॒त् तेषु॑ । \newline
54. तेषु॑ जुह्वति जुह्वति॒ तेषु॒ तेषु॑ जुह्वति । \newline
55. जु॒ह्व॒ त्य॒ति॒हाया॑ ति॒हाय॑ जुह्वति जुह्व त्यति॒हाय॑ । \newline
56. अ॒ति॒हाय॒ वष॒ड् वष॑ डति॒हाया॑ ति॒हाय॒ वष॑ट् । \newline
57. अ॒ति॒हायेत्य॑ति - हाय॑ । \newline
58. वष॑ट् करोति करोति॒ वष॒ड् वष॑ट् करोति । \newline
59. क॒रो॒ति॒ वि वि क॑रोति करोति॒ वि । \newline
60. वि हि हि वि वि हि । \newline
61. ह्ये॑त ए॒ते हि ह्ये॑ते । \newline

\textbf{Ghana Paata } \newline

1. प॒रि॒वे॒ष्टार॒म् ते ते प॑रिवे॒ष्टार॑म् परिवे॒ष्टार॒म् ते सो॑मपी॒थेन॑ सोमपी॒थेन॒ ते प॑रिवे॒ष्टार॑म् परिवे॒ष्टार॒म् ते सो॑मपी॒थेन॑ । \newline
2. प॒रि॒वे॒ष्टार॒मिति॑ परि - वे॒ष्टार᳚म् । \newline
3. ते सो॑मपी॒थेन॑ सोमपी॒थेन॒ ते ते सो॑मपी॒थेन॒ वि वि सो॑मपी॒थेन॒ ते ते सो॑मपी॒थेन॒ वि । \newline
4. सो॒म॒पी॒थेन॒ वि वि सो॑मपी॒थेन॑ सोमपी॒थेन॒ व्या᳚र्द्ध्यन्ता र्द्ध्यन्त॒ वि सो॑मपी॒थेन॑ सोमपी॒थेन॒ व्या᳚र्द्ध्यन्त । \newline
5. सो॒म॒पी॒थेनेति॑ सोम - पी॒थेन॑ । \newline
6. व्या᳚र्द्ध्यन्ता र्द्ध्यन्त॒ वि व्या᳚र्द्ध्यन्त॒ ते त आ᳚र्द्ध्यन्त॒ वि व्या᳚र्द्ध्यन्त॒ ते । \newline
7. आ॒र्द्ध्य॒न्त॒ ते त आ᳚र्द्ध्यन्ता र्द्ध्यन्त॒ ते दे॒वेषु॑ दे॒वेषु॒ त आ᳚र्द्ध्यन्ता र्द्ध्यन्त॒ ते दे॒वेषु॑ । \newline
8. ते दे॒वेषु॑ दे॒वेषु॒ ते ते दे॒वेषु॑ सोमपी॒थꣳ सो॑मपी॒थम् दे॒वेषु॒ ते ते दे॒वेषु॑ सोमपी॒थम् । \newline
9. दे॒वेषु॑ सोमपी॒थꣳ सो॑मपी॒थम् दे॒वेषु॑ दे॒वेषु॑ सोमपी॒थ मै᳚च्छ न्तैच्छन्त सोमपी॒थम् दे॒वेषु॑ दे॒वेषु॑ सोमपी॒थ मै᳚च्छन्त । \newline
10. सो॒म॒पी॒थ मै᳚च्छ न्तैच्छन्त सोमपी॒थꣳ सो॑मपी॒थ मै᳚च्छन्त॒ ताꣳ स्तानै᳚च्छन्त सोमपी॒थꣳ सो॑मपी॒थ मै᳚च्छन्त॒ तान् । \newline
11. सो॒म॒पी॒थमिति॑ सोम - पी॒थम् । \newline
12. ऐ॒च्छ॒न्त॒ ताꣳ स्ता नै᳚च्छ न्तैच्छन्त॒ तान् दे॒वा दे॒वा स्ता नै᳚च्छ न्तैच्छन्त॒ तान् दे॒वाः । \newline
13. तान् दे॒वा दे॒वा स्ताꣳ स्तान् दे॒वा अ॑ब्रुवन् नब्रुवन् दे॒वा स्ताꣳ स्तान् दे॒वा अ॑ब्रुवन्न् । \newline
14. दे॒वा अ॑ब्रुवन् नब्रुवन् दे॒वा दे॒वा अ॑ब्रुव॒न् द्वेद्वे॒ द्वेद्वे॑ अब्रुवन् दे॒वा दे॒वा अ॑ब्रुव॒न् द्वेद्वे᳚ । \newline
15. अ॒ब्रु॒व॒न् द्वेद्वे॒ द्वेद्वे॑ अब्रुवन् नब्रुव॒न् द्वेद्वे॒ नाम॑नी॒ नाम॑नी॒ द्वेद्वे॑ अब्रुवन् नब्रुव॒न् द्वेद्वे॒ नाम॑नी । \newline
16. द्वेद्वे॒ नाम॑नी॒ नाम॑नी॒ द्वेद्वे॒ द्वेद्वे॒ नाम॑नी कुरुद्ध्वम् कुरुद्ध्व॒म् नाम॑नी॒ द्वेद्वे॒ द्वेद्वे॒ नाम॑नी कुरुद्ध्वम् । \newline
17. द्वेद्वे॒ इति॒ द्वे - द्वे॒ । \newline
18. नाम॑नी कुरुद्ध्वम् कुरुद्ध्व॒म् नाम॑नी॒ नाम॑नी कुरुद्ध्व॒ मथाथ॑ कुरुद्ध्व॒म् नाम॑नी॒ नाम॑नी कुरुद्ध्व॒ मथ॑ । \newline
19. नाम॑नी॒ इति॒ नाम॑नी । \newline
20. कु॒रु॒द्ध्व॒ मथाथ॑ कुरुद्ध्वम् कुरुद्ध्व॒ मथ॒ प्र प्राथ॑ कुरुद्ध्वम् कुरुद्ध्व॒ मथ॒ प्र । \newline
21. अथ॒ प्र प्राथाथ॒ प्र वा॑ वा॒ प्राथाथ॒ प्र वा᳚ । \newline
22. प्र वा॑ वा॒ प्र प्र वा॒ ऽऽफ्स्यथा॒ फ्स्यथ॑ वा॒ प्र प्र वा॒ ऽऽफ्स्यथ॑ । \newline
23. वा॒ ऽऽफ्स्यथा॒ फ्स्यथ॑ वा वा॒ ऽऽफ्स्यथ॒ न नाफ्स्यथ॑ वा वा॒ ऽऽफ्स्यथ॒ न । \newline
24. आ॒फ्स्यथ॒ न नाफ्स्यथा॒ फ्स्यथ॒ न वा॑ वा॒ नाफ्स्यथा॒ फ्स्यथ॒ न वा᳚ । \newline
25. न वा॑ वा॒ न न वेतीति॑ वा॒ न न वेति॑ । \newline
26. वेतीति॑ वा॒ वेत्य॒ग्नयो॒ ऽग्नय॒ इति॑ वा॒ वेत्य॒ग्नयः॑ । \newline
27. इत्य॒ग्नयो॒ ऽग्नय॒ इती त्य॒ग्नयो॒ वै वा अ॒ग्नय॒ इती त्य॒ग्नयो॒ वै । \newline
28. अ॒ग्नयो॒ वै वा अ॒ग्नयो॒ ऽग्नयो॒ वा अथाथ॒ वा अ॒ग्नयो॒ ऽग्नयो॒ वा अथ॑ । \newline
29. वा अथाथ॒ वै वा अथ॒ धिष्णि॑या॒ धिष्णि॑या॒ अथ॒ वै वा अथ॒ धिष्णि॑याः । \newline
30. अथ॒ धिष्णि॑या॒ धिष्णि॑या॒ अथाथ॒ धिष्णि॑या॒ स्तस्मा॒त् तस्मा॒द् धिष्णि॑या॒ अथाथ॒ धिष्णि॑या॒ स्तस्मा᳚त् । \newline
31. धिष्णि॑या॒ स्तस्मा॒त् तस्मा॒द् धिष्णि॑या॒ धिष्णि॑या॒ स्तस्मा᳚द् द्वि॒नामा᳚ द्वि॒नामा॒ तस्मा॒द् धिष्णि॑या॒ धिष्णि॑या॒ स्तस्मा᳚द् द्वि॒नामा᳚ । \newline
32. तस्मा᳚द् द्वि॒नामा᳚ द्वि॒नामा॒ तस्मा॒त् तस्मा᳚द् द्वि॒नामा᳚ ब्राह्म॒णो ब्रा᳚ह्म॒णो द्वि॒नामा॒ तस्मा॒त् तस्मा᳚द् द्वि॒नामा᳚ ब्राह्म॒णः । \newline
33. द्वि॒नामा᳚ ब्राह्म॒णो ब्रा᳚ह्म॒णो द्वि॒नामा᳚ द्वि॒नामा᳚ ब्राह्म॒णो ऽर्द्धु॒को ऽर्द्धु॑को ब्राह्म॒णो द्वि॒नामा᳚ द्वि॒नामा᳚ ब्राह्म॒णो ऽर्द्धु॑कः । \newline
34. द्वि॒नामेति॑ द्वि - नामा᳚ । \newline
35. ब्रा॒ह्म॒णो ऽर्द्धु॒को ऽर्द्धु॑को ब्राह्म॒णो ब्रा᳚ह्म॒णो ऽर्द्धु॑क॒ स्तेषा॒म् तेषा॒ मर्द्धु॑को ब्राह्म॒णो ब्रा᳚ह्म॒णो ऽर्द्धु॑क॒ स्तेषा᳚म् । \newline
36. अर्द्धु॑क॒ स्तेषा॒म् तेषा॒ मर्द्धु॒को ऽर्द्धु॑क॒ स्तेषां॒ ॅये ये तेषा॒ मर्द्धु॒को ऽर्द्धु॑क॒ स्तेषां॒ ॅये । \newline
37. तेषां॒ ॅये ये तेषा॒म् तेषां॒ ॅये नेदि॑ष्ठ॒म् नेदि॑ष्ठं॒ ॅये तेषा॒म् तेषां॒ ॅये नेदि॑ष्ठम् । \newline
38. ये नेदि॑ष्ठ॒म् नेदि॑ष्ठं॒ ॅये ये नेदि॑ष्ठम् प॒र्यवि॑शन् प॒र्यवि॑श॒न् नेदि॑ष्ठं॒ ॅये ये नेदि॑ष्ठम् प॒र्यवि॑शन्न् । \newline
39. नेदि॑ष्ठम् प॒र्यवि॑शन् प॒र्यवि॑श॒न् नेदि॑ष्ठ॒म् नेदि॑ष्ठम् प॒र्यवि॑श॒न् ते ते प॒र्यवि॑श॒न् नेदि॑ष्ठ॒म् नेदि॑ष्ठम् प॒र्यवि॑श॒न् ते । \newline
40. प॒र्यवि॑श॒न् ते ते प॒र्यवि॑शन् प॒र्यवि॑श॒न् ते सो॑मपी॒थꣳ सो॑मपी॒थम् ते प॒र्यवि॑शन् प॒र्यवि॑श॒न् ते सो॑मपी॒थम् । \newline
41. प॒र्यवि॑श॒न्निति॑ परि - अवि॑शन्न् । \newline
42. ते सो॑मपी॒थꣳ सो॑मपी॒थम् ते ते सो॑मपी॒थम् प्र प्र सो॑मपी॒थम् ते ते सो॑मपी॒थम् प्र । \newline
43. सो॒म॒पी॒थम् प्र प्र सो॑मपी॒थꣳ सो॑मपी॒थम् प्राप्नु॑वन् नाप्नुव॒न् प्र सो॑मपी॒थꣳ सो॑मपी॒थम् प्राप्नु॑वन्न् । \newline
44. सो॒म॒पी॒थमिति॑ सोम - पी॒थम् । \newline
45. प्राप्नु॑वन् नाप्नुव॒न् प्र प्राप्नु॑वन् नाहव॒नीय॑ आहव॒नीय॑ आप्नुव॒न् प्र प्राप्नु॑वन् नाहव॒नीयः॑ । \newline
46. आ॒प्नु॒व॒न् ना॒ह॒व॒नीय॑ आहव॒नीय॑ आप्नुवन् नाप्नुवन् नाहव॒नीय॑ आग्नी॒द्ध्रीय॑ आग्नी॒द्ध्रीय॑ आहव॒नीय॑ आप्नुवन् नाप्नुवन् नाहव॒नीय॑ आग्नी॒द्ध्रीयः॑ । \newline
47. आ॒ह॒व॒नीय॑ आग्नी॒द्ध्रीय॑ आग्नी॒द्ध्रीय॑ आहव॒नीय॑ आहव॒नीय॑ आग्नी॒द्ध्रीयो॑ हो॒त्रीयो॑ हो॒त्रीय॑ आग्नी॒द्ध्रीय॑ आहव॒नीय॑ आहव॒नीय॑ आग्नी॒द्ध्रीयो॑ हो॒त्रीयः॑ । \newline
48. आ॒ह॒व॒नीय॒ इत्या᳚ - ह॒व॒नीयः॑ । \newline
49. आ॒ग्नी॒द्ध्रीयो॑ हो॒त्रीयो॑ हो॒त्रीय॑ आग्नी॒द्ध्रीय॑ आग्नी॒द्ध्रीयो॑ हो॒त्रीयो॑ मार्जा॒लीयो॑ मार्जा॒लीयो॑ हो॒त्रीय॑ आग्नी॒द्ध्रीय॑ आग्नी॒द्ध्रीयो॑ हो॒त्रीयो॑ मार्जा॒लीयः॑ । \newline
50. आ॒ग्नी॒द्ध्रीय॒ इत्या᳚ग्नि - इ॒द्ध्रीयः॑ । \newline
51. हो॒त्रीयो॑ मार्जा॒लीयो॑ मार्जा॒लीयो॑ हो॒त्रीयो॑ हो॒त्रीयो॑ मार्जा॒लीय॒ स्तस्मा॒त् तस्मा᳚न् मार्जा॒लीयो॑ हो॒त्रीयो॑ हो॒त्रीयो॑ मार्जा॒लीय॒ स्तस्मा᳚त् । \newline
52. मा॒र्जा॒लीय॒ स्तस्मा॒त् तस्मा᳚न् मार्जा॒लीयो॑ मार्जा॒लीय॒ स्तस्मा॒त् तेषु॒ तेषु॒ तस्मा᳚न् मार्जा॒लीयो॑ मार्जा॒लीय॒ स्तस्मा॒त् तेषु॑ । \newline
53. तस्मा॒त् तेषु॒ तेषु॒ तस्मा॒त् तस्मा॒त् तेषु॑ जुह्वति जुह्वति॒ तेषु॒ तस्मा॒त् तस्मा॒त् तेषु॑ जुह्वति । \newline
54. तेषु॑ जुह्वति जुह्वति॒ तेषु॒ तेषु॑ जुह्व त्यति॒हाया॑ ति॒हाय॑ जुह्वति॒ तेषु॒ तेषु॑ जुह्व त्यति॒हाय॑ । \newline
55. जु॒ह्व॒ त्य॒ति॒हाया॑ ति॒हाय॑ जुह्वति जुह्व त्यति॒हाय॒ वष॒ड् वष॑ डति॒हाय॑ जुह्वति जुह्व त्यति॒हाय॒ वष॑ट् । \newline
56. अ॒ति॒हाय॒ वष॒ड् वष॑ डति॒हाया॑ ति॒हाय॒ वष॑ट् करोति करोति॒ वष॑डति॒हा या॑ति॒हाय॒ वष॑ट् करोति । \newline
57. अ॒ति॒हायेत्य॑ति - हाय॑ । \newline
58. वष॑ट् करोति करोति॒ वष॒ड् वष॑ट् करोति॒ वि वि क॑रोति॒ वष॒ड् वष॑ट् करोति॒ वि । \newline
59. क॒रो॒ति॒ वि वि क॑रोति करोति॒ वि हि हि वि क॑रोति करोति॒ वि हि । \newline
60. वि हि हि वि वि ह्ये॑त ए॒ते हि वि वि ह्ये॑ते । \newline
61. ह्ये॑त ए॒ते हि ह्ये॑ते सो॑मपी॒थेन॑ सोमपी॒थे नै॒ते हि ह्ये॑ते सो॑मपी॒थेन॑ । \newline
\pagebreak
\markright{ TS 6.3.1.4  \hfill https://www.vedavms.in \hfill}

\section{ TS 6.3.1.4 }

\textbf{TS 6.3.1.4 } \newline
\textbf{Samhita Paata} \newline

-ते सो॑मपी॒थेनाऽऽ*र्द्ध्य॑न्त दे॒वा वै याः प्राची॒-राहु॑ती॒-रजु॑हवु॒र्ये पु॒रस्ता॒दसु॑रा॒ आस॒न् ताꣳ स्ताभिः॒ प्राणु॑दन्त॒ याः प्र॒तीची॒र्ये प॒श्चादसु॑रा॒ आस॒न् ताꣳस्ताभि॒-रपा॑नुदन्त॒ प्राची॑र॒न्या आहु॑तयो हू॒यन्ते᳚ प्र॒त्यङ्ङासी॑नो॒ धिष्णि॑या॒न्. व्याघा॑रयति प॒श्चाच्चै॒व पुरस्ता᳚च्च॒ यज॑मानो॒ भ्रातृ॑व्या॒न् प्र णु॑दते॒ तस्मा॒त् परा॑चीः प्र॒जाः प्र वी॑यन्ते प्र॒तीची᳚- [  ] \newline

\textbf{Pada Paata} \newline

ए॒ते । सो॒म॒पी॒थेनेति॑ सोम - पी॒थेन॑ । आद्‌र्ध्य॑न्त । दे॒वाः । वै । याः । प्राचीः᳚ । आहु॑ती॒रित्या - हु॒तीः॒ । अजु॑हवुः । ये । पु॒रस्ता᳚त् । असु॑राः । आसन्न्॑ । तान् । ताभिः॑ । प्रेति॑ । अ॒नु॒द॒न्त॒ । याः । प्र॒तीचीः᳚ । ये । प॒श्चात् । असु॑राः । आसन्न्॑ । तान् । ताभिः॑ । अपेति॑ । अ॒नु॒द॒न्त॒ । प्राचीः᳚ । अ॒न्याः । आहु॑तय॒ इत्या - हु॒त॒यः॒ । हू॒यन्ते᳚ । प्र॒त्यङ् । आसी॑नः । धिष्णि॑यान् । व्याघा॑रय॒तीति॑ वि - आघा॑रयति । प॒श्चात् । च॒ । ए॒व । पु॒रस्ता᳚त् । च॒ । यज॑मानः । भ्रातृ॑व्यान् । प्रेति॑ । नु॒द॒ते॒ । तस्मा᳚त् । परा॑चीः । प्र॒जा इति॑ प्र-जाः । प्रेति॑ । वी॒य॒न्ते॒ । प्र॒तीचीः᳚ ।  \newline


\textbf{Krama Paata} \newline

ए॒ते सो॑मपी॒थेन॑ । सो॒म॒पी॒थेनार्द्ध्य॑न्त । सो॒म॒पी॒थेनेति॑ सोम - पी॒थेन॑ । आर्द्ध्य॑न्त दे॒वाः । दे॒वा वै । वै याः । याः प्राचीः᳚ । प्राची॒राहु॑तीः । आहु॑ती॒रजु॑हवुः । आहु॑ती॒रित्या - हु॒तीः॒ । अजु॑हवु॒र् ये । ये पु॒रस्ता᳚त् । पु॒रस्ता॒दसु॑राः । असु॑रा॒ आसन्न्॑ । आस॒न् तान् । ताꣳस्ताभिः॑ । ताभिः॒ प्र । प्राणु॑दन्त । अ॒नु॒द॒न्त॒ याः । याः प्र॒तीचीः᳚ । प्र॒तीची॒र् ये । ये प॒श्चात् । प॒श्चादसु॑राः । असु॑रा॒ आसन्न्॑ । आस॒न् तान् । ताꣳस्ताभिः॑ । ताभि॒रप॑ । अपा॑नुदन्त । अ॒नु॒द॒न्त॒ प्राचीः᳚ । प्राची॑र॒न्याः । अ॒न्या आहु॑तयः । आहु॑तयो हू॒यन्ते᳚ । आहु॑तय॒ इत्या - हु॒त॒यः॒ । हू॒यन्ते᳚ प्र॒त्यङ्‍ङ् । प्र॒त्यङ्‍ङासी॑नः । आसी॑नो॒ धिष्णि॑यान् । धिष्णि॑या॒न् व्याघा॑रयति । व्याघा॑रयति प॒श्चात् । व्याघा॑रय॒तीति॑ वि - आघा॑रयति । प॒श्चाच् च॑ । चै॒व । ए॒व पु॒रस्ता᳚त् । पु॒रस्ता᳚च् च । च॒ यज॑मानः । यज॑मानो॒ भ्रातृ॑व्यान् । भातृ॑व्या॒न् प्र । प्र णु॑दते । नु॒द॒ते॒ तस्मा᳚त् । तस्मा॒त् परा॑चीः । परा॑चीः प्र॒जाः । प्र॒जाः प्र । प्र॒जा इति॑ प्र - जाः । प्र वी॑यन्ते । वी॒य॒न्ते॒ प्र॒तीचीः᳚ । प्र॒तीची᳚र् जायन्ते \newline

\textbf{Jatai Paata} \newline

1. ए॒ते सो॑मपी॒थेन॑ सोमपी॒थेनै॒त ए॒ते सो॑मपी॒थेन॑ । \newline
2. सो॒म॒पी॒थेना र्द्ध्य॒न्ता र्द्ध्य॑न्त सोमपी॒थेन॑ सोमपी॒थेना र्द्ध्य॑न्त । \newline
3. सो॒म॒पी॒थेनेति॑ सोम - पी॒थेन॑ । \newline
4. आर्द्ध्य॑न्त दे॒वा दे॒वा आर्द्ध्य॒न्ता र्द्ध्य॑न्त दे॒वाः । \newline
5. दे॒वा वै वै दे॒वा दे॒वा वै । \newline
6. वै या या वै वै याः । \newline
7. याः प्राचीः॒ प्राची॒र् या याः प्राचीः᳚ । \newline
8. प्राची॒ राहु॑ती॒ राहु॑तीः॒ प्राचीः॒ प्राची॒ राहु॑तीः । \newline
9. आहु॑ती॒ रजु॑हवु॒ रजु॑हवु॒ राहु॑ती॒ राहु॑ती॒ रजु॑हवुः । \newline
10. आहु॑ती॒रित्या - हु॒तीः॒ । \newline
11. अजु॑हवु॒र् ये ये ऽजु॑हवु॒ रजु॑हवु॒र् ये । \newline
12. ये पु॒रस्ता᳚त् पु॒रस्ता॒द् ये ये पु॒रस्ता᳚त् । \newline
13. पु॒रस्ता॒ दसु॑रा॒ असु॑राः पु॒रस्ता᳚त् पु॒रस्ता॒ दसु॑राः । \newline
14. असु॑रा॒ आस॒न् नास॒न् नसु॑रा॒ असु॑रा॒ आसन्न्॑ । \newline
15. आस॒न् ताꣳ स्ता नास॒न् नास॒न् तान् । \newline
16. ताꣳ स्ताभि॒ स्ताभि॒ स्ताꣳ स्ताꣳ स्ताभिः॑ । \newline
17. ताभिः॒ प्र प्र ताभि॒ स्ताभिः॒ प्र । \newline
18. प्राणु॑दन्ता नुदन्त॒ प्र प्राणु॑दन्त । \newline
19. अ॒नु॒द॒न्त॒ या या अ॑नुदन्ता नुदन्त॒ याः । \newline
20. याः प्र॒तीचीः᳚ प्र॒तीची॒र् या याः प्र॒तीचीः᳚ । \newline
21. प्र॒तीची॒र् ये ये प्र॒तीचीः᳚ प्र॒तीची॒र् ये । \newline
22. ये प॒श्चात् प॒श्चाद् ये ये प॒श्चात् । \newline
23. प॒श्चा दसु॑रा॒ असु॑राः प॒श्चात् प॒श्चा दसु॑राः । \newline
24. असु॑रा॒ आस॒न् नास॒न् नसु॑रा॒ असु॑रा॒ आसन्न्॑ । \newline
25. आस॒न् ताꣳ स्ता नास॒न् नास॒न् तान् । \newline
26. ताꣳ स्ताभि॒ स्ताभि॒ स्ताꣳ स्ताꣳ स्ताभिः॑ । \newline
27. ताभि॒ रपाप॒ ताभि॒ स्ताभि॒ रप॑ । \newline
28. अपा॑नुदन्ता नुद॒न्ता पापा॑ नुदन्त । \newline
29. अ॒नु॒द॒न्त॒ प्राचीः॒ प्राची॑ रनुदन्ता नुदन्त॒ प्राचीः᳚ । \newline
30. प्राची॑ र॒न्या अ॒न्याः प्राचीः॒ प्राची॑ र॒न्याः । \newline
31. अ॒न्या आहु॑तय॒ आहु॑तयो॒ ऽन्या अ॒न्या आहु॑तयः । \newline
32. आहु॑तयो हू॒यन्ते॑ हू॒यन्त॒ आहु॑तय॒ आहु॑तयो हू॒यन्ते᳚ । \newline
33. आहु॑तय॒ इत्या - हु॒त॒यः॒ । \newline
34. हू॒यन्ते᳚ प्र॒त्यङ् प्र॒त्यङ्. हू॒यन्ते॑ हू॒यन्ते᳚ प्र॒त्यङ् । \newline
35. प्र॒त्य ङासी॑न॒ आसी॑नः प्र॒त्यङ् प्र॒त्य ङासी॑नः । \newline
36. आसी॑नो॒ धिष्णि॑या॒न् धिष्णि॑या॒ नासी॑न॒ आसी॑नो॒ धिष्णि॑यान् । \newline
37. धिष्णि॑या॒न् व्याघा॑रयति॒ व्याघा॑रयति॒ धिष्णि॑या॒न् धिष्णि॑या॒न् व्याघा॑रयति । \newline
38. व्याघा॑रयति प॒श्चात् प॒श्चाद् व्याघा॑रयति॒ व्याघा॑रयति प॒श्चात् । \newline
39. व्याघा॑रय॒तीति॑ वि - आघा॑रयति । \newline
40. प॒श्चाच् च॑ च प॒श्चात् प॒श्चाच् च॑ । \newline
41. चै॒वैव च॑ चै॒व । \newline
42. ए॒व पु॒रस्ता᳚त् पु॒रस्ता॑ दे॒वैव पु॒रस्ता᳚त् । \newline
43. पु॒रस्ता᳚च् च च पु॒रस्ता᳚त् पु॒रस्ता᳚च् च । \newline
44. च॒ यज॑मानो॒ यज॑मानश्च च॒ यज॑मानः । \newline
45. यज॑मानो॒ भ्रातृ॑व्या॒न् भ्रातृ॑व्या॒न्॒. यज॑मानो॒ यज॑मानो॒ भ्रातृ॑व्यान् । \newline
46. भ्रातृ॑व्या॒न् प्र प्र भ्रातृ॑व्या॒न् भ्रातृ॑व्या॒न् प्र । \newline
47. प्र णु॑दते नुदते॒ प्र प्र णु॑दते । \newline
48. नु॒द॒ते॒ तस्मा॒त् तस्मा᳚न् नुदते नुदते॒ तस्मा᳚त् । \newline
49. तस्मा॒त् परा॑चीः॒ परा॑ची॒ स्तस्मा॒त् तस्मा॒त् परा॑चीः । \newline
50. परा॑चीः प्र॒जाः प्र॒जाः परा॑चीः॒ परा॑चीः प्र॒जाः । \newline
51. प्र॒जाः प्र प्र प्र॒जाः प्र॒जाः प्र । \newline
52. प्र॒जा इति॑ प्र - जाः । \newline
53. प्र वी॑यन्ते वीयन्ते॒ प्र प्र वी॑यन्ते । \newline
54. वी॒य॒न्ते॒ प्र॒तीचीः᳚ प्र॒तीची᳚र् वीयन्ते वीयन्ते प्र॒तीचीः᳚ । \newline
55. प्र॒तीची᳚र् जायन्ते जायन्ते प्र॒तीचीः᳚ प्र॒तीची᳚र् जायन्ते । \newline

\textbf{Ghana Paata } \newline

1. ए॒ते सो॑मपी॒थेन॑ सोमपी॒थे नै॒त ए॒ते सो॑मपी॒थेना र्द्ध्य॒न्ता र्द्ध्य॑न्त सोमपी॒थे नै॒त ए॒ते सो॑मपी॒थेना र्द्ध्य॑न्त । \newline
2. सो॒म॒पी॒थेना र्द्ध्य॒न्ता र्द्ध्य॑न्त सोमपी॒थेन॑ सोमपी॒थेना र्द्ध्य॑न्त दे॒वा दे॒वा आर्द्ध्य॑न्त सोमपी॒थेन॑ सोमपी॒थेना र्द्ध्य॑न्त दे॒वाः । \newline
3. सो॒म॒पी॒थेनेति॑ सोम - पी॒थेन॑ । \newline
4. आर्द्ध्य॑न्त दे॒वा दे॒वा आर्द्ध्य॒न्ता र्द्ध्य॑न्त दे॒वा वै वै दे॒वा आर्द्ध्य॒न्ता र्द्ध्य॑न्त दे॒वा वै । \newline
5. दे॒वा वै वै दे॒वा दे॒वा वै या या वै दे॒वा दे॒वा वै याः । \newline
6. वै या या वै वै याः प्राचीः॒ प्राची॒र् या वै वै याः प्राचीः᳚ । \newline
7. याः प्राचीः॒ प्राची॒र् या याः प्राची॒ राहु॑ती॒ राहु॑तीः॒ प्राची॒र् या याः प्राची॒ राहु॑तीः । \newline
8. प्राची॒ राहु॑ती॒ राहु॑तीः॒ प्राचीः॒ प्राची॒ राहु॑ती॒ रजु॑हवु॒ रजु॑हवु॒ राहु॑तीः॒ प्राचीः॒ प्राची॒ राहु॑ती॒ रजु॑हवुः । \newline
9. आहु॑ती॒ रजु॑हवु॒ रजु॑हवु॒ राहु॑ती॒ राहु॑ती॒ रजु॑हवु॒र् ये ये ऽजु॑हवु॒ राहु॑ती॒ राहु॑ती॒ रजु॑हवु॒र् ये । \newline
10. आहु॑ती॒रित्या - हु॒तीः॒ । \newline
11. अजु॑हवु॒र् ये ये ऽजु॑हवु॒ रजु॑हवु॒र् ये पु॒रस्ता᳚त् पु॒रस्ता॒द् ये ऽजु॑हवु॒ रजु॑हवु॒र् ये पु॒रस्ता᳚त् । \newline
12. ये पु॒रस्ता᳚त् पु॒रस्ता॒द् ये ये पु॒रस्ता॒ दसु॑रा॒ असु॑राः पु॒रस्ता॒द् ये ये पु॒रस्ता॒ दसु॑राः । \newline
13. पु॒रस्ता॒ दसु॑रा॒ असु॑राः पु॒रस्ता᳚त् पु॒रस्ता॒ दसु॑रा॒ आस॒न् नास॒न् नसु॑राः पु॒रस्ता᳚त् पु॒रस्ता॒ दसु॑रा॒ आसन्न्॑ । \newline
14. असु॑रा॒ आस॒न् नास॒न् नसु॑रा॒ असु॑रा॒ आस॒न् ताꣳ स्ता नास॒न् नसु॑रा॒ असु॑रा॒ आस॒न् तान् । \newline
15. आस॒न् ताꣳ स्ता नास॒न् नास॒न् ताꣳ स्ताभि॒ स्ताभि॒ स्ता नास॒न् नास॒न् ताꣳ स्ताभिः॑ । \newline
16. ताꣳ स्ताभि॒ स्ताभि॒ स्ताꣳ स्ताꣳ स्ताभिः॒ प्र प्र ताभि॒ स्ताꣳ स्ताꣳ स्ताभिः॒ प्र । \newline
17. ताभिः॒ प्र प्र ताभि॒ स्ताभिः॒ प्राणु॑दन्ता नुदन्त॒ प्र ताभि॒ स्ताभिः॒ प्राणु॑दन्त । \newline
18. प्राणु॑दन्ता नुदन्त॒ प्र प्राणु॑दन्त॒ या या अ॑नुदन्त॒ प्र प्राणु॑दन्त॒ याः । \newline
19. अ॒नु॒द॒न्त॒ या या अ॑नुदन्ता नुदन्त॒ याः प्र॒तीचीः᳚ प्र॒तीची॒र् या अ॑नुदन्ता नुदन्त॒ याः प्र॒तीचीः᳚ । \newline
20. याः प्र॒तीचीः᳚ प्र॒तीची॒र् या याः प्र॒तीची॒र् ये ये प्र॒तीची॒र् या याः प्र॒तीची॒र् ये । \newline
21. प्र॒तीची॒र् ये ये प्र॒तीचीः᳚ प्र॒तीची॒र् ये प॒श्चात् प॒श्चाद् ये प्र॒तीचीः᳚ प्र॒तीची॒र् ये प॒श्चात् । \newline
22. ये प॒श्चात् प॒श्चाद् ये ये प॒श्चा दसु॑रा॒ असु॑राः प॒श्चाद् ये ये प॒श्चा दसु॑राः । \newline
23. प॒श्चा दसु॑रा॒ असु॑राः प॒श्चात् प॒श्चा दसु॑रा॒ आस॒न् नास॒न् नसु॑राः प॒श्चात् प॒श्चा दसु॑रा॒ आसन्न्॑ । \newline
24. असु॑रा॒ आस॒न् नास॒न् नसु॑रा॒ असु॑रा॒ आस॒न् ताꣳ स्ता नास॒न् नसु॑रा॒ असु॑रा॒ आस॒न् तान् । \newline
25. आस॒न् ताꣳ स्ता नास॒न् नास॒न् ताꣳ स्ताभि॒ स्ताभि॒ स्ता नास॒न् नास॒न् ताꣳ स्ताभिः॑ । \newline
26. ताꣳ स्ताभि॒ स्ताभि॒ स्ताꣳ स्ताꣳ स्ताभि॒ रपाप॒ ताभि॒ स्ताꣳ स्ताꣳ स्ताभि॒ रप॑ । \newline
27. ताभि॒ रपाप॒ ताभि॒ स्ताभि॒ रपा॑ नुदन्ता नुद॒न्ताप॒ ताभि॒ स्ताभि॒ रपा॑नुदन्त । \newline
28. अपा॑नुदन्ता नुद॒न्ता पापा॑नुदन्त॒ प्राचीः॒ प्राची॑ रनुद॒न्ता पापा॑ नुदन्त॒ प्राचीः᳚ । \newline
29. अ॒नु॒द॒न्त॒ प्राचीः॒ प्राची॑ रनुदन्ता नुदन्त॒ प्राची॑ र॒न्या अ॒न्याः प्राची॑ रनुदन्ता नुदन्त॒ प्राची॑ र॒न्याः । \newline
30. प्राची॑ र॒न्या अ॒न्याः प्राचीः॒ प्राची॑ र॒न्या आहु॑तय॒ आहु॑तयो॒ ऽन्याः प्राचीः॒ प्राची॑ र॒न्या आहु॑तयः । \newline
31. अ॒न्या आहु॑तय॒ आहु॑तयो॒ ऽन्या अ॒न्या आहु॑तयो हू॒यन्ते॑ हू॒यन्त॒ आहु॑तयो॒ ऽन्या अ॒न्या आहु॑तयो हू॒यन्ते᳚ । \newline
32. आहु॑तयो हू॒यन्ते॑ हू॒यन्त॒ आहु॑तय॒ आहु॑तयो हू॒यन्ते᳚ प्र॒त्यङ् प्र॒त्यङ्. हू॒यन्त॒ आहु॑तय॒ आहु॑तयो हू॒यन्ते᳚ प्र॒त्यङ् । \newline
33. आहु॑तय॒ इत्या - हु॒त॒यः॒ । \newline
34. हू॒यन्ते᳚ प्र॒त्यङ् प्र॒त्यङ्. हू॒यन्ते॑ हू॒यन्ते᳚ प्र॒त्य ङासी॑न॒ आसी॑नः प्र॒त्यङ्. हू॒यन्ते॑ हू॒यन्ते᳚ प्र॒त्य ङासी॑नः । \newline
35. प्र॒त्य ङासी॑न॒ आसी॑नः प्र॒त्यङ् प्र॒त्य ङासी॑नो॒ धिष्णि॑या॒न् धिष्णि॑या॒ नासी॑नः प्र॒त्यङ् प्र॒त्य ङासी॑नो॒ धिष्णि॑यान् । \newline
36. आसी॑नो॒ धिष्णि॑या॒न् धिष्णि॑या॒ नासी॑न॒ आसी॑नो॒ धिष्णि॑या॒न् व्याघा॑रयति॒ व्याघा॑रयति॒ धिष्णि॑या॒ नासी॑न॒ आसी॑नो॒ धिष्णि॑या॒न् व्याघा॑रयति । \newline
37. धिष्णि॑या॒न् व्याघा॑रयति॒ व्याघा॑रयति॒ धिष्णि॑या॒न् धिष्णि॑या॒न् व्याघा॑रयति प॒श्चात् प॒श्चाद् व्याघा॑रयति॒ धिष्णि॑या॒न् धिष्णि॑या॒न् व्याघा॑रयति प॒श्चात् । \newline
38. व्याघा॑रयति प॒श्चात् प॒श्चाद् व्याघा॑रयति॒ व्याघा॑रयति प॒श्चाच् च॑ च प॒श्चाद् व्याघा॑रयति॒ व्याघा॑रयति प॒श्चाच् च॑ । \newline
39. व्याघा॑रय॒तीति॑ वि - आघा॑रयति । \newline
40. प॒श्चाच् च॑ च प॒श्चात् प॒श्चाच् चै॒वैव च॑ प॒श्चात् प॒श्चाच् चै॒व । \newline
41. चै॒वैव च॑ चै॒व पु॒रस्ता᳚त् पु॒रस्ता॑ दे॒व च॑ चै॒व पु॒रस्ता᳚त् । \newline
42. ए॒व पु॒रस्ता᳚त् पु॒रस्ता॑ दे॒वैव पु॒रस्ता᳚च् च च पु॒रस्ता॑ दे॒वैव पु॒रस्ता᳚च् च । \newline
43. पु॒रस्ता᳚च् च च पु॒रस्ता᳚त् पु॒रस्ता᳚च् च॒ यज॑मानो॒ यज॑मान श्च पु॒रस्ता᳚त् पु॒रस्ता᳚च् च॒ यज॑मानः । \newline
44. च॒ यज॑मानो॒ यज॑मान श्च च॒ यज॑मानो॒ भ्रातृ॑व्या॒न् भ्रातृ॑व्या॒न्॒. यज॑मान श्च च॒ यज॑मानो॒ भ्रातृ॑व्यान् । \newline
45. यज॑मानो॒ भ्रातृ॑व्या॒न् भ्रातृ॑व्या॒न्॒. यज॑मानो॒ यज॑मानो॒ भ्रातृ॑व्या॒न् प्र प्र भ्रातृ॑व्या॒न्॒. यज॑मानो॒ यज॑मानो॒ भ्रातृ॑व्या॒न् प्र । \newline
46. भ्रातृ॑व्या॒न् प्र प्र भ्रातृ॑व्या॒न् भ्रातृ॑व्या॒न् प्र णु॑दते नुदते॒ प्र भ्रातृ॑व्या॒न् भ्रातृ॑व्या॒न् प्र णु॑दते । \newline
47. प्र णु॑दते नुदते॒ प्र प्र णु॑दते॒ तस्मा॒त् तस्मा᳚न् नुदते॒ प्र प्र णु॑दते॒ तस्मा᳚त् । \newline
48. नु॒द॒ते॒ तस्मा॒त् तस्मा᳚न् नुदते नुदते॒ तस्मा॒त् परा॑चीः॒ परा॑ची॒ स्तस्मा᳚न् नुदते नुदते॒ तस्मा॒त् परा॑चीः । \newline
49. तस्मा॒त् परा॑चीः॒ परा॑ची॒ स्तस्मा॒त् तस्मा॒त् परा॑चीः प्र॒जाः प्र॒जाः परा॑ची॒ स्तस्मा॒त् तस्मा॒त् परा॑चीः प्र॒जाः । \newline
50. परा॑चीः प्र॒जाः प्र॒जाः परा॑चीः॒ परा॑चीः प्र॒जाः प्र प्र प्र॒जाः परा॑चीः॒ परा॑चीः प्र॒जाः प्र । \newline
51. प्र॒जाः प्र प्र प्र॒जाः प्र॒जाः प्र वी॑यन्ते वीयन्ते॒ प्र प्र॒जाः प्र॒जाः प्र वी॑यन्ते । \newline
52. प्र॒जा इति॑ प्र - जाः । \newline
53. प्र वी॑यन्ते वीयन्ते॒ प्र प्र वी॑यन्ते प्र॒तीचीः᳚ प्र॒तीची᳚र् वीयन्ते॒ प्र प्र वी॑यन्ते प्र॒तीचीः᳚ । \newline
54. वी॒य॒न्ते॒ प्र॒तीचीः᳚ प्र॒तीची᳚र् वीयन्ते वीयन्ते प्र॒तीची᳚र् जायन्ते जायन्ते प्र॒तीची᳚र् वीयन्ते वीयन्ते प्र॒तीची᳚र् जायन्ते । \newline
55. प्र॒तीची᳚र् जायन्ते जायन्ते प्र॒तीचीः᳚ प्र॒तीची᳚र् जायन्ते प्रा॒णाः प्रा॒णा जा॑यन्ते प्र॒तीचीः᳚ प्र॒तीची᳚र् जायन्ते प्रा॒णाः । \newline
\pagebreak
\markright{ TS 6.3.1.5  \hfill https://www.vedavms.in \hfill}

\section{ TS 6.3.1.5 }

\textbf{TS 6.3.1.5 } \newline
\textbf{Samhita Paata} \newline

र्जायन्ते प्रा॒णा वा ए॒ते यद्धिष्णि॑या॒ यद॑द्ध्व॒र्युः प्र॒त्यङ् धिष्णि॑या-नति॒सर्पे᳚त् प्रा॒णान्थ् संक॑र्.षेत् प्र॒मायु॑कः स्या॒न्नाभि॒र्वा ए॒षा य॒ज्ञ्स्य॒ यद्धोतो॒र्द्ध्वः खलु॒ वै नाभ्यै᳚ प्रा॒णोऽवा॑ङपा॒नो यद॑ध्व॒र्युः प्र॒त्यङ् होता॑रमति॒सर्पे॑दपा॒ने प्रा॒णं द॑ध्यात् प्र॒मायु॑कः स्या॒न्नाद्ध्व॒र्युरुप॑ गाये॒द् वाग्वी᳚र्यो॒ वा अ॑द्ध्व॒र्यु-र्यद॑द्ध्व॒र्युरु॑प॒-गाये॑दुद्-गा॒त्रे- [  ] \newline

\textbf{Pada Paata} \newline

जा॒य॒न्ते॒ । प्रा॒णा इति॑ प्र-अ॒नाः । वै । ए॒ते । यत् । धिष्णि॑याः । यत् । अ॒द्ध्व॒र्युः । प्र॒त्यङ् । धिष्णि॑यान् । अ॒ति॒सर्पे॒दित्य॑ति - सर्पे᳚त् । प्रा॒णानिति॑ प्र - अ॒नान् । समिति॑ । क॒र्॒.षे॒त् । प्र॒मायु॑क॒ इति॑ प्र - मायु॑कः । स्या॒त् । नाभिः॑ । वै । ए॒षा । य॒ज्ञ्स्य॑ । यत् । होता᳚ । ऊ॒द्‌र्ध्वः । खलु॑ । वै । नाभ्यै᳚ । प्रा॒ण इति॑ प्र - अ॒नः । अवाङ्॑ । अ॒पा॒न इत्य॑प - अ॒नः । यत् । अ॒द्ध्व॒र्युः । प्र॒त्यङ् । होता॑रम् । अ॒ति॒सर्पे॒दित्य॑ति - सर्पे᳚त् । अ॒पा॒न इत्य॑प - अ॒ने । प्रा॒णमिति॑ प्र - अ॒नम् । द॒द्ध्या॒त् । प्र॒मायु॑क॒ इति॑ प्र - मायु॑कः । स्या॒त् । न । अ॒द्ध्व॒र्युः । उपेति॑ । गा॒ये॒त् । वाग्वी᳚र्य॒ इति॒ वाक् - वी॒र्यः॒ । वै । अ॒द्ध्व॒र्युः । यत् । अ॒द्ध्व॒र्युः । उ॒प॒गाये॒दित्यु॑प - गाये᳚त् । उ॒द्गा॒त्र इत्यु॑त् - गा॒त्रे ।  \newline


\textbf{Krama Paata} \newline

जा॒य॒न्ते॒ प्रा॒णाः । प्रा॒णा वै । प्रा॒णा इति॑ प्र - अ॒नाः । वा ए॒ते । ए॒ते यत् । यद् धिष्णि॑याः । धिष्णि॑या॒ यत् । यद॑द्ध्व॒र्युः । अ॒द्ध्व॒र्युः प्र॒त्यङ्‍ङ् । प्र॒त्यङ् धिष्णि॑यान् । धिष्णि॑यानति॒सर्पे᳚त् । अ॒ति॒सर्पे᳚त् प्रा॒णान् । अ॒ति॒सर्पे॒दित्य॑ति - सर्पे᳚त् । प्रा॒णान्थ् सम् । प्रा॒णानिति॑ प्र - अ॒नान् । सम् क॑र्.षेत् । क॒र्.॒षे॒त् प्र॒मायु॑कः । प्र॒मायु॑कः स्यात् । प्र॒मायु॑क॒ इति॑ प्र - मायु॑कः । स्या॒न् नाभिः॑ । नाभि॒र् वै । वा ए॒षा । ए॒षा य॒ज्ञ्स्य॑ । य॒ज्ञ्स्य॒ यत् । यद्‌धोता᳚ । होतो॒र्द्ध्वः । ऊ॒र्द्ध्वः खलु॑ । खलु॒ वै । वै नाभ्यै᳚ । नाभ्यै᳚ प्रा॒णः । प्रा॒णोऽवाङ्॑ । प्रा॒ण इति॑ प्र - अ॒नः । अवा॑ङपा॒नः । अ॒पा॒नो यत् । अ॒पा॒न इत्य॑प - अ॒नः । यद॑द्ध्व॒र्युः । अ॒द्ध्व॒र्युः प्र॒त्यङ्‍ङ् । प्र॒त्यङ् होता॑रम् । होता॑र,मति॒सर्पे᳚त् । अ॒ति॒सर्पे॑दपा॒ने । अ॒ति॒सर्पे॒दित्य॑ति - सर्पे᳚त् । अ॒पा॒ने प्रा॒णम् । अ॒पा॒न इत्य॑प - अ॒ने । प्रा॒णम् द॑द्ध्यात् । प्रा॒णमिति॑ प्र - अ॒नम् । द॒द्ध्या॒त् प्र॒मायु॑कः । प्र॒मायु॑कः स्यात् । प्र॒मायु॑क॒ इति॑ प्र - मायु॑कः । स्या॒न् न । नाद्ध्व॒र्युः । अ॒द्ध्व॒र्युरुप॑ । उप॑ गायेत् । गा॒ये॒द् वाग्वी᳚र्यः । वाग्वी᳚र्यो॒ वै । वाग्वी᳚र्य॒ इति॒ वाक् - वी॒र्यः॒ । वा अ॑द्ध्व॒र्युः । अ॒द्ध्व॒र्युर् यत् । यद॑द्ध्व॒र्युः । अ॒द्ध्व॒र्युरु॑प॒गाये᳚त् । उ॒प॒गाये॑दुद्‍गा॒त्रे । उ॒प॒गाये॒दित्यु॑प - गाये᳚त् । उ॒द्‍गा॒त्रे वाच᳚म् । उ॒द्‍गा॒त्र इत्यु॑त् - गा॒त्रे \newline

\textbf{Jatai Paata} \newline

1. जा॒य॒न्ते॒ प्रा॒णाः प्रा॒णा जा॑यन्ते जायन्ते प्रा॒णाः । \newline
2. प्रा॒णा वै वै प्रा॒णाः प्रा॒णा वै । \newline
3. प्रा॒णा इति॑ प्र - अ॒नाः । \newline
4. वा ए॒त ए॒ते वै वा ए॒ते । \newline
5. ए॒ते यद् यदे॒त ए॒ते यत् । \newline
6. यद् धिष्णि॑या॒ धिष्णि॑या॒ यद् यद् धिष्णि॑याः । \newline
7. धिष्णि॑या॒ यद् यद् धिष्णि॑या॒ धिष्णि॑या॒ यत् । \newline
8. यद॑द्ध्व॒र्यु र॑द्ध्व॒र्युर् यद् यद॑द्ध्व॒र्युः । \newline
9. अ॒द्ध्व॒र्युः प्र॒त्यङ् प्र॒त्यङ् ङ॑द्ध्व॒र्यु र॑द्ध्व॒र्युः प्र॒त्यङ् । \newline
10. प्र॒त्यङ् धिष्णि॑या॒न् धिष्णि॑यान् प्र॒त्यङ् प्र॒त्यङ् धिष्णि॑यान् । \newline
11. धिष्णि॑या नति॒सर्पे॑ दति॒सर्पे॒द् धिष्णि॑या॒न् धिष्णि॑या नति॒सर्पे᳚त् । \newline
12. अ॒ति॒सर्पे᳚त् प्रा॒णान् प्रा॒णा न॑ति॒सर्पे॑ दति॒सर्पे᳚त् प्रा॒णान् । \newline
13. अ॒ति॒सर्पे॒दित्य॑ति - सर्पे᳚त् । \newline
14. प्रा॒णान् थ्सꣳ सम् प्रा॒णान् प्रा॒णान् थ्सम् । \newline
15. प्रा॒णानिति॑ प्र - अ॒नान् । \newline
16. सम् क॑र्.षेत् कर्.षे॒थ् सꣳ सम् क॑र्.षेत् । \newline
17. क॒र्॒.षे॒त् प्र॒मायु॑कः प्र॒मायु॑कः कर्.षेत् कर्.षेत् प्र॒मायु॑कः । \newline
18. प्र॒मायु॑कः स्याथ् स्यात् प्र॒मायु॑कः प्र॒मायु॑कः स्यात् । \newline
19. प्र॒मायु॑क॒ इति॑ प्र - मायु॑कः । \newline
20. स्या॒न् नाभि॒र् नाभिः॑ स्याथ् स्या॒न् नाभिः॑ । \newline
21. नाभि॒र् वै वै नाभि॒र् नाभि॒र् वै । \newline
22. वा ए॒षैषा वै वा ए॒षा । \newline
23. ए॒षा य॒ज्ञ्स्य॑ य॒ज्ञ् स्यै॒षैषा य॒ज्ञ्स्य॑ । \newline
24. य॒ज्ञ्स्य॒ यद् यद् य॒ज्ञ्स्य॑ य॒ज्ञ्स्य॒ यत् । \newline
25. यद्धोता॒ होता॒ यद् यद्धोता᳚ । \newline
26. होतो॒र्द्ध्व ऊ॒र्द्ध्वो होता॒ होतो॒र्द्ध्वः । \newline
27. ऊ॒र्द्ध्वः खलु॒ खलू॒र्द्ध्व ऊ॒र्द्ध्वः खलु॑ । \newline
28. खलु॒ वै वै खलु॒ खलु॒ वै । \newline
29. वै नाभ्यै॒ नाभ्यै॒ वै वै नाभ्यै᳚ । \newline
30. नाभ्यै᳚ प्रा॒णः प्रा॒णो नाभ्यै॒ नाभ्यै᳚ प्रा॒णः । \newline
31. प्रा॒णो ऽवा॒ ङवा᳚ङ् प्रा॒णः प्रा॒णो ऽवाङ्॑ । \newline
32. प्रा॒ण इति॑ प्र - अ॒नः । \newline
33. अवा॑ ङपा॒नो॑ ऽपा॒नो ऽवा॒ ङवा॑ ङपा॒नः । \newline
34. अ॒पा॒नो यद् यद॑पा॒नो॑ ऽपा॒नो यत् । \newline
35. अ॒पा॒न इत्य॑प - अ॒नः । \newline
36. यद॑द्ध्व॒र्यु र॑द्ध्व॒र्युर् यद् यद॑द्ध्व॒र्युः । \newline
37. अ॒द्ध्व॒र्युः प्र॒त्यङ् प्र॒त्यङ् ङ॑द्ध्व॒र्यु र॑द्ध्व॒र्युः प्र॒त्यङ् । \newline
38. प्र॒त्यङ्. होता॑रꣳ॒॒ होता॑रम् प्र॒त्यङ् प्र॒त्यङ्. होता॑रम् । \newline
39. होता॑र मति॒सर्पे॑ दति॒सर्पे॒ द्धोता॑रꣳ॒॒ होता॑र मति॒सर्पे᳚त् । \newline
40. अ॒ति॒सर्पे॑ दपा॒ने॑ ऽपा॒ने॑ ऽति॒सर्पे॑ दति॒सर्पे॑ दपा॒ने । \newline
41. अ॒ति॒सर्पे॒दित्य॑ति - सर्पे᳚त् । \newline
42. अ॒पा॒ने प्रा॒णम् प्रा॒ण म॑पा॒ने॑ ऽपा॒ने प्रा॒णम् । \newline
43. अ॒पा॒न इत्य॑प - अ॒ने । \newline
44. प्रा॒णम् द॑द्ध्याद् दद्ध्यात् प्रा॒णम् प्रा॒णम् द॑द्ध्यात् । \newline
45. प्रा॒णमिति॑ प्र - अ॒नम् । \newline
46. द॒द्ध्या॒त् प्र॒मायु॑कः प्र॒मायु॑को दद्ध्याद् दद्ध्यात् प्र॒मायु॑कः । \newline
47. प्र॒मायु॑कः स्याथ् स्यात् प्र॒मायु॑कः प्र॒मायु॑कः स्यात् । \newline
48. प्र॒मायु॑क॒ इति॑ प्र - मायु॑कः । \newline
49. स्या॒न् न न स्या᳚थ् स्या॒न् न । \newline
50. नाद्ध्व॒र्यु र॑द्ध्व॒र्युर् न नाद्ध्व॒र्युः । \newline
51. अ॒द्ध्व॒र्यु रुपोपा᳚ द्ध्व॒र्यु र॑द्ध्व॒र्यु रुप॑ । \newline
52. उप॑ गायेद् गाये॒ दुपोप॑ गायेत् । \newline
53. गा॒ये॒द् वाग्वी᳚र्यो॒ वाग्वी᳚र्यो गायेद् गाये॒द् वाग्वी᳚र्यः । \newline
54. वाग्वी᳚र्यो॒ वै वै वाग्वी᳚र्यो॒ वाग्वी᳚र्यो॒ वै । \newline
55. वाग्वी᳚र्य॒ इति॒ वाक् - वी॒र्यः॒ । \newline
56. वा अ॑द्ध्व॒र्यु र॑द्ध्व॒र्युर् वै वा अ॑द्ध्व॒र्युः । \newline
57. अ॒द्ध्व॒र्युर् यद् यद॑द्ध्व॒र्यु र॑द्ध्व॒र्युर् यत् । \newline
58. यद॑द्ध्व॒र्यु र॑द्ध्व॒र्युर् यद् यद॑द्ध्व॒र्युः । \newline
59. अ॒द्ध्व॒र्यु रु॑प॒गाये॑ दुप॒गाये॑ दद्ध्व॒र्यु र॑द्ध्व॒र्यु रु॑प॒गाये᳚त् । \newline
60. उ॒प॒गाये॑ दुद्‍गा॒त्र उ॑द्‍गा॒त्र उ॑प॒गाये॑ दुप॒गाये॑ दुद्‍गा॒त्रे । \newline
61. उ॒प॒गाये॒दित्यु॑प - गाये᳚त् । \newline
62. उ॒द्‍गा॒त्रे वाचं॒ ॅवाच॑ मुद्‍गा॒त्र उ॑द्‍गा॒त्रे वाच᳚म् । \newline
63. उ॒द्‍गा॒त्र इत्यु॑त् - गा॒त्रे । \newline

\textbf{Ghana Paata } \newline

1. जा॒य॒न्ते॒ प्रा॒णाः प्रा॒णा जा॑यन्ते जायन्ते प्रा॒णा वै वै प्रा॒णा जा॑यन्ते जायन्ते प्रा॒णा वै । \newline
2. प्रा॒णा वै वै प्रा॒णाः प्रा॒णा वा ए॒त ए॒ते वै प्रा॒णाः प्रा॒णा वा ए॒ते । \newline
3. प्रा॒णा इति॑ प्र - अ॒नाः । \newline
4. वा ए॒त ए॒ते वै वा ए॒ते यद् यदे॒ते वै वा ए॒ते यत् । \newline
5. ए॒ते यद् यदे॒त ए॒ते यद् धिष्णि॑या॒ धिष्णि॑या॒ यदे॒त ए॒ते यद् धिष्णि॑याः । \newline
6. यद् धिष्णि॑या॒ धिष्णि॑या॒ यद् यद् धिष्णि॑या॒ यद् यद् धिष्णि॑या॒ यद् यद् धिष्णि॑या॒ यत् । \newline
7. धिष्णि॑या॒ यद् यद् धिष्णि॑या॒ धिष्णि॑या॒ यद॑द्ध्व॒र्यु र॑द्ध्व॒र्युर् यद् धिष्णि॑या॒ धिष्णि॑या॒ यद॑द्ध्व॒र्युः । \newline
8. यद॑द्ध्व॒र्यु र॑द्ध्व॒र्युर् यद् यद॑द्ध्व॒र्युः प्र॒त्यङ् प्र॒त्यङ् ङ॑द्ध्व॒र्युर् यद् यद॑द्ध्व॒र्युः प्र॒त्यङ् । \newline
9. अ॒द्ध्व॒र्युः प्र॒त्यङ् प्र॒त्यङ् ङ॑द्ध्व॒र्यु र॑द्ध्व॒र्युः प्र॒त्यङ् धिष्णि॑या॒न् धिष्णि॑यान् प्र॒त्यङ् ङ॑द्ध्व॒र्यु र॑द्ध्व॒र्युः प्र॒त्यङ् धिष्णि॑यान् । \newline
10. प्र॒त्यङ् धिष्णि॑या॒न् धिष्णि॑यान् प्र॒त्यङ् प्र॒त्यङ् धिष्णि॑या नति॒सर्पे॑ दति॒सर्पे॒द् धिष्णि॑यान् प्र॒त्यङ् प्र॒त्यङ् धिष्णि॑या नति॒सर्पे᳚त् । \newline
11. धिष्णि॑या नति॒सर्पे॑ दति॒सर्पे॒द् धिष्णि॑या॒न् धिष्णि॑या नति॒सर्पे᳚त् प्रा॒णान् प्रा॒णा न॑ति॒सर्पे॒द् धिष्णि॑या॒न् धिष्णि॑या नति॒सर्पे᳚त् प्रा॒णान् । \newline
12. अ॒ति॒सर्पे᳚त् प्रा॒णान् प्रा॒णा न॑ति॒सर्पे॑ दति॒सर्पे᳚त् प्रा॒णान् थ्सꣳ सम् प्रा॒णा न॑ति॒सर्पे॑ दति॒सर्पे᳚त् प्रा॒णान् थ्सम् । \newline
13. अ॒ति॒सर्पे॒दित्य॑ति - सर्पे᳚त् । \newline
14. प्रा॒णान् थ्सꣳ सम् प्रा॒णान् प्रा॒णान् थ्सम् क॑र्.षेत् कर्.षे॒थ् सम् प्रा॒णान् प्रा॒णान् थ्सम् क॑र्.षेत् । \newline
15. प्रा॒णानिति॑ प्र - अ॒नान् । \newline
16. सम् क॑र्.षेत् कर्.षे॒थ् सꣳ सम् क॑र्.षेत् प्र॒मायु॑कः प्र॒मायु॑कः कर्.षे॒थ् सꣳ सम् क॑र्.षेत् प्र॒मायु॑कः । \newline
17. क॒र्॒.षे॒त् प्र॒मायु॑कः प्र॒मायु॑कः कर्.षेत् कर्.षेत् प्र॒मायु॑कः स्याथ् स्यात् प्र॒मायु॑कः कर्.षेत् कर्.षेत् प्र॒मायु॑कः स्यात् । \newline
18. प्र॒मायु॑कः स्याथ् स्यात् प्र॒मायु॑कः प्र॒मायु॑कः स्या॒न् नाभि॒र् नाभिः॑ स्यात् प्र॒मायु॑कः प्र॒मायु॑कः स्या॒न् नाभिः॑ । \newline
19. प्र॒मायु॑क॒ इति॑ प्र - मायु॑कः । \newline
20. स्या॒न् नाभि॒र् नाभिः॑ स्याथ् स्या॒न् नाभि॒र् वै वै नाभिः॑ स्याथ् स्या॒न् नाभि॒र् वै । \newline
21. नाभि॒र् वै वै नाभि॒र् नाभि॒र् वा ए॒षैषा वै नाभि॒र् नाभि॒र् वा ए॒षा । \newline
22. वा ए॒षैषा वै वा ए॒षा य॒ज्ञ्स्य॑ य॒ज्ञ् स्यै॒षा वै वा ए॒षा य॒ज्ञ्स्य॑ । \newline
23. ए॒षा य॒ज्ञ्स्य॑ य॒ज्ञ् स्यै॒षैषा य॒ज्ञ्स्य॒ यद् यद् य॒ज्ञ् स्यै॒षैषा य॒ज्ञ्स्य॒ यत् । \newline
24. य॒ज्ञ्स्य॒ यद् यद् य॒ज्ञ्स्य॑ य॒ज्ञ्स्य॒ यद्धोता॒ होता॒ यद् य॒ज्ञ्स्य॑ य॒ज्ञ्स्य॒ यद्धोता᳚ । \newline
25. यद्धोता॒ होता॒ यद् यद्धोतो॒र्द्ध्व ऊ॒र्द्ध्वो होता॒ यद् यद्धोतो॒र्द्ध्वः । \newline
26. होतो॒र्द्ध्व ऊ॒र्द्ध्वो होता॒ होतो॒र्द्ध्वः खलु॒ खलू॒र्द्ध्वो होता॒ होतो॒र्द्ध्वः खलु॑ । \newline
27. ऊ॒र्द्ध्वः खलु॒ खलू॒र्द्ध्व ऊ॒र्द्ध्वः खलु॒ वै वै खलू॒र्द्ध्व ऊ॒र्द्ध्वः खलु॒ वै । \newline
28. खलु॒ वै वै खलु॒ खलु॒ वै नाभ्यै॒ नाभ्यै॒ वै खलु॒ खलु॒ वै नाभ्यै᳚ । \newline
29. वै नाभ्यै॒ नाभ्यै॒ वै वै नाभ्यै᳚ प्रा॒णः प्रा॒णो नाभ्यै॒ वै वै नाभ्यै᳚ प्रा॒णः । \newline
30. नाभ्यै᳚ प्रा॒णः प्रा॒णो नाभ्यै॒ नाभ्यै᳚ प्रा॒णो ऽवा॒ ङवा᳚ङ् प्रा॒णो नाभ्यै॒ नाभ्यै᳚ प्रा॒णो ऽवाङ्॑ । \newline
31. प्रा॒णो ऽवा॒ ङवा᳚ङ् प्रा॒णः प्रा॒णो ऽवा॑ ङपा॒नो॑ ऽपा॒नो ऽवा᳚ङ् प्रा॒णः प्रा॒णो ऽवा॑ ङपा॒नः । \newline
32. प्रा॒ण इति॑ प्र - अ॒नः । \newline
33. अवा॑ ङपा॒नो॑ ऽपा॒नो ऽवा॒ ङवा॑ ङपा॒नो यद् यद॑पा॒नो ऽवा॒ ङवा॑ ङपा॒नो यत् । \newline
34. अ॒पा॒नो यद् यद॑पा॒नो॑ ऽपा॒नो यद॑द्ध्व॒र्यु र॑द्ध्व॒र्युर् यद॑पा॒नो॑ ऽपा॒नो यद॑द्ध्व॒र्युः । \newline
35. अ॒पा॒न इत्य॑प - अ॒नः । \newline
36. यद॑द्ध्व॒र्यु र॑द्ध्व॒र्युर् यद् यद॑द्ध्व॒र्युः प्र॒त्यङ् प्र॒त्यङ् ङ॑द्ध्व॒र्युर् यद् यद॑द्ध्व॒र्युः प्र॒त्यङ् । \newline
37. अ॒द्ध्व॒र्युः प्र॒त्यङ् प्र॒त्यङ् ङ॑द्ध्व॒र्यु र॑द्ध्व॒र्युः प्र॒त्यङ् होता॑रꣳ॒॒ होता॑रम् प्र॒त्यङ् ङ॑द्ध्व॒र्यु र॑द्ध्व॒र्युः प्र॒त्यङ्. होता॑रम् । \newline
38. प्र॒त्यङ्. होता॑रꣳ॒॒ होता॑रम् प्र॒त्यङ् प्र॒त्यङ्. होता॑र मति॒सर्पे॑ दति॒सर्पे॒ द्धोता॑रम् प्र॒त्यङ् प्र॒त्यङ्. होता॑र मति॒सर्पे᳚त् । \newline
39. होता॑र मति॒सर्पे॑ दति॒सर्पे॒ द्धोता॑रꣳ॒॒ होता॑र मति॒सर्पे॑ दपा॒ने॑ ऽपा॒ने॑ ऽति॒सर्पे॒ द्धोता॑रꣳ॒॒ होता॑र मति॒सर्पे॑-दपा॒ने । \newline
40. अ॒ति॒सर्पे॑ दपा॒ने॑ ऽपा॒ने॑ ऽति॒सर्पे॑ दति॒सर्पे॑ दपा॒ने प्रा॒णम् प्रा॒ण म॑पा॒ने॑ ऽति॒सर्पे॑ दति॒सर्पे॑ दपा॒ने प्रा॒णम् । \newline
41. अ॒ति॒सर्पे॒दित्य॑ति - सर्पे᳚त् । \newline
42. अ॒पा॒ने प्रा॒णम् प्रा॒ण म॑पा॒ने॑ ऽपा॒ने प्रा॒णम् द॑द्ध्याद् दद्ध्यात् प्रा॒ण म॑पा॒ने॑ ऽपा॒ने प्रा॒णम् द॑द्ध्यात् । \newline
43. अ॒पा॒न इत्य॑प - अ॒ने । \newline
44. प्रा॒णम् द॑द्ध्याद् दद्ध्यात् प्रा॒णम् प्रा॒णम् द॑द्ध्यात् प्र॒मायु॑कः प्र॒मायु॑को दद्ध्यात् प्रा॒णम् प्रा॒णम् द॑द्ध्यात् प्र॒मायु॑कः । \newline
45. प्रा॒णमिति॑ प्र - अ॒नम् । \newline
46. द॒द्ध्या॒त् प्र॒मायु॑कः प्र॒मायु॑को दद्ध्याद् दद्ध्यात् प्र॒मायु॑कः स्याथ् स्यात् प्र॒मायु॑को दद्ध्याद् दद्ध्यात् प्र॒मायु॑कः स्यात् । \newline
47. प्र॒मायु॑कः स्याथ् स्यात् प्र॒मायु॑कः प्र॒मायु॑कः स्या॒न् न न स्या᳚त् प्र॒मायु॑कः प्र॒मायु॑कः स्या॒न् न । \newline
48. प्र॒मायु॑क॒ इति॑ प्र - मायु॑कः । \newline
49. स्या॒न् न न स्या᳚थ् स्या॒न् नाद्ध्व॒र्यु र॑द्ध्व॒र्युर् न स्या᳚थ् स्या॒न् नाद्ध्व॒र्युः । \newline
50. नाद्ध्व॒र्यु र॑द्ध्व॒र्युर् न नाद्ध्व॒र्यु रुपोपा᳚ द्ध्व॒र्युर् न नाद्ध्व॒र्यु रुप॑ । \newline
51. अ॒द्ध्व॒र्यु रुपोपा᳚ द्ध्व॒र्यु र॑द्ध्व॒र्यु रुप॑ गायेद् गाये॒ दुपा᳚द्ध्व॒र्यु र॑द्ध्व॒र्यु रुप॑ गायेत् । \newline
52. उप॑ गायेद् गाये॒ दुपोप॑ गाये॒द् वाग्वी᳚र्यो॒ वाग्वी᳚र्यो गाये॒ दुपोप॑ गाये॒द् वाग्वी᳚र्यः । \newline
53. गा॒ये॒द् वाग्वी᳚र्यो॒ वाग्वी᳚र्यो गायेद् गाये॒द् वाग्वी᳚र्यो॒ वै वै वाग्वी᳚र्यो गायेद् गाये॒द् वाग्वी᳚र्यो॒ वै । \newline
54. वाग्वी᳚र्यो॒ वै वै वाग्वी᳚र्यो॒ वाग्वी᳚र्यो॒ वा अ॑द्ध्व॒र्यु र॑द्ध्व॒र्युर् वै वाग्वी᳚र्यो॒ वाग्वी᳚र्यो॒ वा अ॑द्ध्व॒र्युः । \newline
55. वाग्वी᳚र्य॒ इति॒ वाक् - वी॒र्यः॒ । \newline
56. वा अ॑द्ध्व॒र्यु र॑द्ध्व॒र्युर् वै वा अ॑द्ध्व॒र्युर् यद् यद॑द्ध्व॒र्युर् वै वा अ॑द्ध्व॒र्युर् यत् । \newline
57. अ॒द्ध्व॒र्युर् यद् यद॑द्ध्व॒र्यु र॑द्ध्व॒र्युर् यद॑द्ध्व॒र्यु र॑द्ध्व॒र्युर् यद॑द्ध्व॒र्यु र॑द्ध्व॒र्युर् यद॑द्ध्व॒र्युः । \newline
58. यद॑द्ध्व॒र्यु र॑द्ध्व॒र्युर् यद् यद॑द्ध्व॒र्यु रु॑प॒गाये॑ दुप॒गाये॑ दद्ध्व॒र्युर् यद् यद॑द्ध्व॒र्यु रु॑प॒गाये᳚त् । \newline
59. अ॒द्ध्व॒र्यु रु॑प॒गाये॑ दुप॒गाये॑ दद्ध्व॒र्यु र॑द्ध्व॒र्यु रु॑प॒गाये॑ दुद्‍गा॒त्र उ॑द्‍गा॒त्र उ॑प॒गाये॑ दद्ध्व॒र्यु र॑द्ध्व॒र्यु रु॑प॒गाये॑ दुद्‍गा॒त्रे । \newline
60. उ॒प॒गाये॑ दुद्‍गा॒त्र उ॑द्‍गा॒त्र उ॑प॒गाये॑ दुप॒गाये॑ दुद्‍गा॒त्रे वाचं॒ ॅवाच॑ मुद्‍गा॒त्र उ॑प॒गाये॑ दुप॒गाये॑ दुद्‍गा॒त्रे वाच᳚म् । \newline
61. उ॒प॒गाये॒दित्यु॑प - गाये᳚त् । \newline
62. उ॒द्‍गा॒त्रे वाचं॒ ॅवाच॑ मुद्‍गा॒त्र उ॑द्‍गा॒त्रे वाचꣳ॒॒ सꣳ सं ॅवाच॑ मुद्‍गा॒त्र उ॑द्‍गा॒त्रे वाचꣳ॒॒ सम् । \newline
63. उ॒द्‍गा॒त्र इत्यु॑त् - गा॒त्रे । \newline
\pagebreak
\markright{ TS 6.3.1.6  \hfill https://www.vedavms.in \hfill}

\section{ TS 6.3.1.6 }

\textbf{TS 6.3.1.6 } \newline
\textbf{Samhita Paata} \newline

वाचꣳ॒॒ सं प्र य॑च्छे-दुप॒दासु॑काऽस्य॒ वाख् स्या᳚द्ब्रह्मवा॒दिनो॑ वदन्ति॒ नासꣳ॑स्थिते॒ सोमे᳚ऽद्ध्व॒र्युः प्र॒त्यङ्ख् सदोऽती॑या॒दथ॑ क॒था दा᳚क्षि॒णानि॒ होतु॑मेति॒ यामो॒ हि स तेषां॒ कस्मा॒ अह॑ दे॒वा यामं॒ ॅवाऽया॑मं॒ ॅवाऽनु॑ ज्ञास्य॒न्तीत्यु-त्त॑रे॒णाऽऽ*ग्नी᳚द्ध्रं प॒रीत्य॑ जुहोति दाक्षि॒णानि॒ न प्रा॒णान्थ्सं क॑र्.षति॒ न्य॑न्ये धिष्णि॑या उ॒प्यन्ते॒ ( ) नान्ये यान् नि॒वप॑ति॒ तेन॒ तान् प्री॑णाति॒ यान् ननि॒वप॑ति॒ यद॑नुदि॒शति॒ तेन॒ तान् ॥ \newline

\textbf{Pada Paata} \newline

वाच᳚म् । सम् । प्रेति॑ । य॒च्छे॒त् । उ॒प॒दासु॒केत्यु॑प-दासु॑का । अ॒स्य॒ । वाक् । स्या॒त् । ब्र॒ह्म॒वा॒दिन॒ इति॑ ब्रह्म - वा॒दिनः॑ । व॒द॒न्ति॒ । न । असꣳ॑स्थित॒ इत्यसं᳚ - स्थि॒ते॒ । सोमे᳚ । अ॒द्ध्व॒र्युः । प्र॒त्यङ् । सदः॑ । अतीति॑ । इ॒या॒त् । अथ॑ । क॒था । दा॒क्षि॒णानि॑ । होतु᳚म् । ए॒ति॒ । यामः॑ । हि । सः । तेषा᳚म् । कस्मै᳚ । अह॑ । दे॒वाः । याम᳚म् । वा॒ । अया॑मम् । वा॒ । अन्विति॑ । ज्ञा॒स्य॒न्ति॒ । इति॑ । उत्त॑रे॒णेत्युत्- त॒रे॒ण॒ । आग्नी᳚द्ध्र॒मित्याग्नि॑ - इ॒द्ध्र॒म् । प॒रीत्येति॑ परि - इत्य॑ । जु॒हो॒ति॒ । दा॒क्षि॒णानि॑ । न । प्रा॒णानिति॑ प्र - अ॒नान् । समिति॑ । क॒र्॒.ष॒ति॒ । नीति॑ । अ॒न्ये । धिष्णि॑याः । उ॒प्यन्ते᳚ ( ) । न । अ॒न्ये । यान् । नि॒वप॒तीति॑ नि - वप॑ति । तेन॑ । तान् । प्री॒णा॒ति॒ । यान् । न । नि॒वप॒तीति॑ नि - वप॑ति । यत् । अ॒नु॒दि॒शतीत्य॑नु - दि॒शति॑ । तेन॑ । तान् ॥  \newline


\textbf{Krama Paata} \newline

वाचꣳ॒॒ सम् । सम् प्र । प्र य॑च्छेत् । य॒च्छे॒दु॒प॒दासु॑का । उ॒प॒दासु॑काऽस्य । उ॒प॒दासु॒केत्यु॑प - दासु॑का । अ॒स्य॒ वाक् । वाख् स्या᳚त् । स्या॒द् ब्र॒ह्म॒वा॒दिनः॑ । ब्र॒ह्म॒वा॒दिनो॑ वदन्ति । ब्र॒ह्म॒वा॒दिन॒ इति॑ ब्रह्म - वा॒दिनः॑ । व॒द॒न्ति॒ न । नासꣳ॑स्थिते । असꣳ॑स्थिते॒ सोमे᳚ । असꣳ॑स्थित॒ इत्यस᳚म् - स्थि॒ते॒ । सोमे᳚ऽद्ध्व॒र्युः । अ॒द्ध्व॒र्युः प्र॒त्यङ्‍ङ् । प्र॒त्यङ्‍ख् सदः॑ । सदोऽति॑ । अती॑यात् । इ॒या॒दथ॑ । अथ॑ क॒था । क॒था दा᳚क्षि॒णानि॑ । दा॒क्षि॒णानि॒ होतु᳚म् । होतु॑मेति । ए॒ति॒ यामः॑ । यामो॒ हि । हि सः । स तेषा᳚म् । तेषा॒म् कस्मै᳚ । कस्मा॒ अह॑ । अह॑ दे॒वाः । दे॒वा याम᳚म् । याम॑म् ॅवा । वाऽया॑मम् । अया॑मम् ॅवा । वाऽनु॑ । अनु॑ ज्ञास्यन्ति । ज्ञा॒स्य॒न्तीति॑ । इत्युत्त॑रेण । उत्त॑रे॒णाग्नी᳚द्ध्रम् । उत्त॑रे॒णेत्युत् - त॒रे॒ण॒ । आग्नी᳚द्ध्रम् प॒रीत्य॑ । आग्नी᳚द्ध्र॒मित्याग्नि॑ - इ॒द्ध्र॒म् । प॒रीत्य॑ जुहोति । प॒रीत्येति॑ परि - इत्य॑ । जु॒हो॒ति॒ दा॒क्षि॒णानि॑ । दा॒क्षि॒णानि॒ न । न प्रा॒णान् । प्रा॒णान्थ् सम् । प्रा॒णानिति॑ प्र - अ॒नान् । सम् क॑र्.षति । क॒र्.॒ष॒ति॒ नि । न्य॑न्ये । अ॒न्ये धिष्णि॑याः । धिष्णि॑या उ॒प्यन्ते᳚ ( ) । उ॒प्यन्ते॒ न । नान्ये । अ॒न्ये यान् । यान् नि॒वप॑ति । नि॒वप॑ति॒ तेन॑ । नि॒वप॒तीति॑ नि - वप॑ति । तेन॒ तान् । तान् प्री॑णाति । प्री॒णा॒ति॒ यान् । यान् न । न नि॒वप॑ति । नि॒वप॑ति॒ यत् । नि॒वप॒तीति॑ नि - वप॑ति । यद॑नुदि॒शति॑ । अ॒नु॒दि॒शति॒ तेन॑ । अ॒नु॒दि॒शतीत्य॑नु - दि॒शति॑ । तेन॒ तान् । तानिति॒ तान् । \newline

\textbf{Jatai Paata} \newline

1. वाचꣳ॒॒ सꣳ सं ॅवाचं॒ ॅवाचꣳ॒॒ सम् । \newline
2. सम् प्र प्र सꣳ सम् प्र । \newline
3. प्र य॑च्छेद् यच्छे॒त् प्र प्र य॑च्छेत् । \newline
4. य॒च्छे॒ दु॒प॒दासु॑ कोप॒दासु॑का यच्छेद् यच्छे दुप॒दासु॑का । \newline
5. उ॒प॒दासु॑का ऽस्यास्यो प॒दासु॑ कोप॒दासु॑का ऽस्य । \newline
6. उ॒प॒दासु॒केत्यु॑प - दासु॑का । \newline
7. अ॒स्य॒ वाग् वाग॑ स्यास्य॒ वाक् । \newline
8. वाख् स्या᳚थ् स्या॒द् वाग् वाख् स्या᳚त् । \newline
9. स्या॒द् ब्र॒ह्म॒वा॒दिनो᳚ ब्रह्मवा॒दिनः॑ स्याथ् स्याद् ब्रह्मवा॒दिनः॑ । \newline
10. ब्र॒ह्म॒वा॒दिनो॑ वदन्ति वदन्ति ब्रह्मवा॒दिनो᳚ ब्रह्मवा॒दिनो॑ वदन्ति । \newline
11. ब्र॒ह्म॒वा॒दिन॒ इति॑ ब्रह्म - वा॒दिनः॑ । \newline
12. व॒द॒न्ति॒ न न व॑दन्ति वदन्ति॒ न । \newline
13. नासꣳ॑स्थि॒ते ऽसꣳ॑स्थिते॒ न नासꣳ॑स्थिते । \newline
14. असꣳ॑स्थिते॒ सोमे॒ सोमे ऽसꣳ॑स्थि॒ते ऽसꣳ॑स्थिते॒ सोमे᳚ । \newline
15. असꣳ॑स्थित॒ इत्यसं᳚ - स्थि॒ते॒ । \newline
16. सोमे᳚ ऽद्ध्व॒र्यु र॑द्ध्व॒र्युः सोमे॒ सोमे᳚ ऽद्ध्व॒र्युः । \newline
17. अ॒द्ध्व॒र्युः प्र॒त्यङ् प्र॒त्यङ् ङ॑द्ध्व॒र्यु र॑द्ध्व॒र्युः प्र॒त्यङ् । \newline
18. प्र॒त्यङ् ख्सदः॒ सदः॑ प्र॒त्यङ् प्र॒त्यङ् ख्सदः॑ । \newline
19. सदो ऽत्यति॒ सदः॒ सदो ऽति॑ । \newline
20. अती॑या दिया॒ दत्यती॑यात् । \newline
21. इ॒या॒ दथाथे॑ यादिया॒ दथ॑ । \newline
22. अथ॑ क॒था क॒था ऽथाथ॑ क॒था । \newline
23. क॒था दा᳚क्षि॒णानि॑ दाक्षि॒णानि॑ क॒था क॒था दा᳚क्षि॒णानि॑ । \newline
24. दा॒क्षि॒णानि॒ होतुꣳ॒॒ होतु॑म् दाक्षि॒णानि॑ दाक्षि॒णानि॒ होतु᳚म् । \newline
25. होतु॑ मेत्येति॒ होतुꣳ॒॒ होतु॑ मेति । \newline
26. ए॒ति॒ यामो॒ याम॑ एत्येति॒ यामः॑ । \newline
27. यामो॒ हि हि यामो॒ यामो॒ हि । \newline
28. हि स स हि हि सः । \newline
29. स तेषा॒म् तेषाꣳ॒॒ स स तेषा᳚म् । \newline
30. तेषा॒म् कस्मै॒ कस्मै॒ तेषा॒म् तेषा॒म् कस्मै᳚ । \newline
31. कस्मा॒ अहाह॒ कस्मै॒ कस्मा॒ अह॑ । \newline
32. अह॑ दे॒वा दे॒वा अहाह॑ दे॒वाः । \newline
33. दे॒वा यामं॒ ॅयाम॑म् दे॒वा दे॒वा याम᳚म् । \newline
34. यामं॑ ॅवा वा॒ यामं॒ ॅयामं॑ ॅवा । \newline
35. वा ऽया॑म॒ मया॑मं ॅवा॒ वा ऽया॑मम् । \newline
36. अया॑मं ॅवा॒ वा ऽया॑म॒ मया॑मं ॅवा । \newline
37. वा ऽन्वनु॑ वा॒ वा ऽनु॑ । \newline
38. अनु॑ ज्ञास्यन्ति ज्ञास्य॒ न्त्यन्वनु॑ ज्ञास्यन्ति । \newline
39. ज्ञा॒स्य॒न्ती तीति॑ ज्ञास्यन्ति ज्ञास्य॒न्तीति॑ । \newline
40. इत्युत्त॑रे॒ णोत्त॑रे॒णे तीत्युत्त॑रेण । \newline
41. उत्त॑रे॒णा ग्नी᳚द्ध्र॒ माग्नी᳚द्ध्र॒ मुत्त॑रे॒ णोत्त॑रे॒णा ग्नी᳚द्ध्रम् । \newline
42. उत्त॑रे॒णेत्युत् - त॒रे॒ण॒ । \newline
43. आग्नी᳚द्ध्रम् प॒रीत्य॑ प॒रीत्या ग्नी᳚द्ध्र॒ माग्नी᳚द्ध्रम् प॒रीत्य॑ । \newline
44. आग्नी᳚द्ध्र॒मित्याग्नि॑ - इ॒द्ध्र॒म् । \newline
45. प॒रीत्य॑ जुहोति जुहोति प॒रीत्य॑ प॒रीत्य॑ जुहोति । \newline
46. प॒रीत्येति॑ परि - इत्य॑ । \newline
47. जु॒हो॒ति॒ दा॒क्षि॒णानि॑ दाक्षि॒णानि॑ जुहोति जुहोति दाक्षि॒णानि॑ । \newline
48. दा॒क्षि॒णानि॒ न न दा᳚क्षि॒णानि॑ दाक्षि॒णानि॒ न । \newline
49. न प्रा॒णान् प्रा॒णान् न न प्रा॒णान् । \newline
50. प्रा॒णान् थ्सꣳ सम् प्रा॒णान् प्रा॒णान् थ्सम् । \newline
51. प्रा॒णानिति॑ प्र - अ॒नान् । \newline
52. सम् क॑र्.षति कर्.षति॒ सꣳ सम् क॑र्.षति । \newline
53. क॒र्॒.ष॒ति॒ नि नि क॑र्.षति कर्.षति॒ नि । \newline
54. न्या᳚(1॒)न्ये᳚ ऽन्ये नि न्य॑न्ये । \newline
55. अ॒न्ये धिष्णि॑या॒ धिष्णि॑या अ॒न्ये᳚ ऽन्ये धिष्णि॑याः । \newline
56. धिष्णि॑या उ॒प्यन्त॑ उ॒प्यन्ते॒ धिष्णि॑या॒ धिष्णि॑या उ॒प्यन्ते᳚ । \newline
57. उ॒प्यन्ते॒ न नोप्यन्त॑ उ॒प्यन्ते॒ न । \newline
58. नान्ये᳚ ऽन्ये न नान्ये । \newline
59. अ॒न्ये यान्. यान॒न्ये᳚ ऽन्ये यान् । \newline
60. यान् नि॒वप॑ति नि॒वप॑ति॒ यान्. यान् नि॒वप॑ति । \newline
61. नि॒वप॑ति॒ तेन॒ तेन॑ नि॒वप॑ति नि॒वप॑ति॒ तेन॑ । \newline
62. नि॒वप॒तीति॑ नि - वप॑ति । \newline
63. तेन॒ ताꣳ स्ताꣳ स्तेन॒ तेन॒ तान् । \newline
64. तान् प्री॑णाति प्रीणाति॒ ताꣳ स्तान् प्री॑णाति । \newline
65. प्री॒णा॒ति॒ यान्. यान् प्री॑णाति प्रीणाति॒ यान् । \newline
66. यान् न न यान्. यान् न । \newline
67. न नि॒वप॑ति नि॒वप॑ति॒ न न नि॒वप॑ति । \newline
68. नि॒वप॑ति॒ यद् यन् नि॒वप॑ति नि॒वप॑ति॒ यत् । \newline
69. नि॒वप॒तीति॑ नि - वप॑ति । \newline
70. यद॑नुदि॒श त्य॑नुदि॒शति॒ यद् यद॑नुदि॒शति॑ । \newline
71. अ॒नु॒दि॒शति॒ तेन॒ तेना॑ नुदि॒श त्य॑नुदि॒शति॒ तेन॑ । \newline
72. अ॒नु॒दि॒शतीत्य॑नु - दि॒शति॑ । \newline
73. तेन॒ ताꣳ स्ताꣳ स्तेन॒ तेन॒ तान् । \newline
74. तानिति॒ तान् । \newline

\textbf{Ghana Paata } \newline

1. वाचꣳ॒॒ सꣳ सं ॅवाचं॒ ॅवाचꣳ॒॒ सम् प्र प्र सं ॅवाचं॒ ॅवाचꣳ॒॒ सम् प्र । \newline
2. सम् प्र प्र सꣳ सम् प्र य॑च्छेद् यच्छे॒त् प्र सꣳ सम् प्र य॑च्छेत् । \newline
3. प्र य॑च्छेद् यच्छे॒त् प्र प्र य॑च्छे दुप॒दासु॑ कोप॒दासु॑का यच्छे॒त् प्र प्र य॑च्छे दुप॒दासु॑का । \newline
4. य॒च्छे॒ दु॒प॒दासु॑ कोप॒दासु॑का यच्छेद् यच्छे दुप॒दासु॑का ऽस्या स्योप॒दासु॑का यच्छेद् यच्छे दुप॒दासु॑का ऽस्य । \newline
5. उ॒प॒दासु॑का ऽस्या स्योप॒दासु॑ कोप॒दासु॑का ऽस्य॒ वाग् वाग॑स्योप॒दासु॑ कोप॒दासु॑का ऽस्य॒ वाक् । \newline
6. उ॒प॒दासु॒केत्यु॑प - दासु॑का । \newline
7. अ॒स्य॒ वाग् वाग॑ स्यास्य॒ वाख् स्या᳚थ् स्या॒द् वाग॑ स्यास्य॒ वाख् स्या᳚त् । \newline
8. वाख् स्या᳚थ् स्या॒द् वाग् वाख् स्या᳚द् ब्रह्मवा॒दिनो᳚ ब्रह्मवा॒दिनः॑ स्या॒द् वाग् वाख् स्या᳚द् ब्रह्मवा॒दिनः॑ । \newline
9. स्या॒द् ब्र॒ह्म॒वा॒दिनो᳚ ब्रह्मवा॒दिनः॑ स्याथ् स्याद् ब्रह्मवा॒दिनो॑ वदन्ति वदन्ति ब्रह्मवा॒दिनः॑ स्याथ् स्याद् ब्रह्मवा॒दिनो॑ वदन्ति । \newline
10. ब्र॒ह्म॒वा॒दिनो॑ वदन्ति वदन्ति ब्रह्मवा॒दिनो᳚ ब्रह्मवा॒दिनो॑ वदन्ति॒ न न व॑दन्ति ब्रह्मवा॒दिनो᳚ ब्रह्मवा॒दिनो॑ वदन्ति॒ न । \newline
11. ब्र॒ह्म॒वा॒दिन॒ इति॑ ब्रह्म - वा॒दिनः॑ । \newline
12. व॒द॒न्ति॒ न न व॑दन्ति वदन्ति॒ नासꣳ॑स्थि॒ते ऽसꣳ॑स्थिते॒ न व॑दन्ति वदन्ति॒ नासꣳ॑स्थिते । \newline
13. नासꣳ॑स्थि॒ते ऽसꣳ॑स्थिते॒ न नासꣳ॑स्थिते॒ सोमे॒ सोमे ऽसꣳ॑स्थिते॒ न नासꣳ॑स्थिते॒ सोमे᳚ । \newline
14. असꣳ॑स्थिते॒ सोमे॒ सोमे ऽसꣳ॑स्थि॒ते ऽसꣳ॑स्थिते॒ सोमे᳚ ऽद्ध्व॒र्यु र॑द्ध्व॒र्युः सोमे ऽसꣳ॑स्थि॒ते ऽसꣳ॑स्थिते॒ सोमे᳚ ऽद्ध्व॒र्युः । \newline
15. असꣳ॑स्थित॒ इत्यसं᳚ - स्थि॒ते॒ । \newline
16. सोमे᳚ ऽद्ध्व॒र्यु र॑द्ध्व॒र्युः सोमे॒ सोमे᳚ ऽद्ध्व॒र्युः प्र॒त्यङ् प्र॒त्यङ् ङ॑द्ध्व॒र्युः सोमे॒ सोमे᳚ ऽद्ध्व॒र्युः प्र॒त्यङ् । \newline
17. अ॒द्ध्व॒र्युः प्र॒त्यङ् प्र॒त्यङ् ङ॑द्ध्व॒र्यु र॑द्ध्व॒र्युः प्र॒त्यङ् ख्सदः॒ सदः॑ प्र॒त्यङ् ङ॑द्ध्व॒र्यु र॑द्ध्व॒र्युः प्र॒त्यङ् ख्सदः॑ । \newline
18. प्र॒त्यङ् ख्सदः॒ सदः॑ प्र॒त्यङ् प्र॒त्यङ् ख्सदो ऽत्यति॒ सदः॑ प्र॒त्यङ् प्र॒त्यङ् ख्सदो ऽति॑ । \newline
19. सदो ऽत्यति॒ सदः॒ सदो ऽती॑या दिया॒ दति॒ सदः॒ सदो ऽती॑यात् । \newline
20. अती॑या दिया॒ दत्यती॑या॒ दथा थे॑या॒ दत्यती॑या॒ दथ॑ । \newline
21. इ॒या॒ दथा थे॑यादिया॒ दथ॑ क॒था क॒था ऽथे॑या दिया॒ दथ॑ क॒था । \newline
22. अथ॑ क॒था क॒था ऽथाथ॑ क॒था दा᳚क्षि॒णानि॑ दाक्षि॒णानि॑ क॒था ऽथाथ॑ क॒था दा᳚क्षि॒णानि॑ । \newline
23. क॒था दा᳚क्षि॒णानि॑ दाक्षि॒णानि॑ क॒था क॒था दा᳚क्षि॒णानि॒ होतुꣳ॒॒ होतु॑म् दाक्षि॒णानि॑ क॒था क॒था दा᳚क्षि॒णानि॒ होतु᳚म् । \newline
24. दा॒क्षि॒णानि॒ होतुꣳ॒॒ होतु॑म् दाक्षि॒णानि॑ दाक्षि॒णानि॒ होतु॑ मेत्येति॒ होतु॑म् दाक्षि॒णानि॑ दाक्षि॒णानि॒ होतु॑ मेति । \newline
25. होतु॑ मेत्येति॒ होतुꣳ॒॒ होतु॑ मेति॒ यामो॒ याम॑ एति॒ होतुꣳ॒॒ होतु॑ मेति॒ यामः॑ । \newline
26. ए॒ति॒ यामो॒ याम॑ एत्येति॒ यामो॒ हि हि याम॑ एत्येति॒ यामो॒ हि । \newline
27. यामो॒ हि हि यामो॒ यामो॒ हि स स हि यामो॒ यामो॒ हि सः । \newline
28. हि स स हि हि स तेषा॒म् तेषाꣳ॒॒ स हि हि स तेषा᳚म् । \newline
29. स तेषा॒म् तेषाꣳ॒॒ स स तेषा॒म् कस्मै॒ कस्मै॒ तेषाꣳ॒॒ स स तेषा॒म् कस्मै᳚ । \newline
30. तेषा॒म् कस्मै॒ कस्मै॒ तेषा॒म् तेषा॒म् कस्मा॒ अहाह॒ कस्मै॒ तेषा॒म् तेषा॒म् कस्मा॒ अह॑ । \newline
31. कस्मा॒ अहाह॒ कस्मै॒ कस्मा॒ अह॑ दे॒वा दे॒वा अह॒ कस्मै॒ कस्मा॒ अह॑ दे॒वाः । \newline
32. अह॑ दे॒वा दे॒वा अहाह॑ दे॒वा यामं॒ ॅयाम॑म् दे॒वा अहाह॑ दे॒वा याम᳚म् । \newline
33. दे॒वा यामं॒ ॅयाम॑म् दे॒वा दे॒वा यामं॑ ॅवा वा॒ याम॑म् दे॒वा दे॒वा यामं॑ ॅवा । \newline
34. यामं॑ ॅवा वा॒ यामं॒ ॅयामं॒ ॅवा ऽया॑म॒ मया॑मं ॅवा॒ यामं॒ ॅयामं॒ ॅवा ऽया॑मम् । \newline
35. वा ऽया॑म॒ मया॑मं ॅवा॒ वा ऽया॑मं ॅवा॒ वा ऽया॑मं ॅवा॒ वा ऽया॑मं ॅवा । \newline
36. अया॑मं ॅवा॒ वा ऽया॑म॒ मया॑मं॒ ॅवा ऽन्वनु॒ वा ऽया॑म॒ मया॑मं॒ ॅवा ऽनु॑ । \newline
37. वा ऽन्वनु॑ वा॒ वा ऽनु॑ ज्ञास्यन्ति ज्ञास्य॒ न्त्यनु॑ वा॒ वा ऽनु॑ ज्ञास्यन्ति । \newline
38. अनु॑ ज्ञास्यन्ति ज्ञास्य॒ न्त्यन्वनु॑ ज्ञास्य॒न्ती तीति॑ ज्ञास्य॒ न्त्यन्वनु॑ ज्ञास्य॒न्तीति॑ । \newline
39. ज्ञा॒स्य॒न्ती तीति॑ ज्ञास्यन्ति ज्ञास्य॒न्ती त्युत्त॑रे॒ णोत्त॑रे॒णेति॑ ज्ञास्यन्ति ज्ञास्य॒न्ती त्युत्त॑रेण । \newline
40. इत्युत्त॑रे॒ णोत्त॑रे॒णे तीत्युत्त॑रे॒णा ग्नी᳚द्ध्र॒ माग्नी᳚द्ध्र॒ मुत्त॑रे॒णे तीत्युत्त॑रे॒णा ग्नी᳚द्ध्रम् । \newline
41. उत्त॑रे॒णा ग्नी᳚द्ध्र॒ माग्नी᳚द्ध्र॒ मुत्त॑रे॒ णोत्त॑रे॒णा ग्नी᳚द्ध्रम् प॒रीत्य॑ प॒रीत्या ग्नी᳚द्ध्र॒ मुत्त॑रे॒
णोत्त॑रे॒णा ग्नी᳚द्ध्रम् प॒रीत्य॑ । \newline
42. उत्त॑रे॒णेत्युत् - त॒रे॒ण॒ । \newline
43. आग्नी᳚द्ध्रम् प॒रीत्य॑ प॒रीत्या ग्नी᳚द्ध्र॒ माग्नी᳚द्ध्रम् प॒रीत्य॑ जुहोति जुहोति प॒रीत्या ग्नी᳚द्ध्र॒ माग्नी᳚द्ध्रम् प॒रीत्य॑ जुहोति । \newline
44. आग्नी᳚द्ध्र॒मित्याग्नि॑ - इ॒द्ध्र॒म् । \newline
45. प॒रीत्य॑ जुहोति जुहोति प॒रीत्य॑ प॒रीत्य॑ जुहोति दाक्षि॒णानि॑ दाक्षि॒णानि॑ जुहोति प॒रीत्य॑ प॒रीत्य॑ जुहोति दाक्षि॒णानि॑ । \newline
46. प॒रीत्येति॑ परि - इत्य॑ । \newline
47. जु॒हो॒ति॒ दा॒क्षि॒णानि॑ दाक्षि॒णानि॑ जुहोति जुहोति दाक्षि॒णानि॒ न न दा᳚क्षि॒णानि॑ जुहोति जुहोति दाक्षि॒णानि॒ न । \newline
48. दा॒क्षि॒णानि॒ न न दा᳚क्षि॒णानि॑ दाक्षि॒णानि॒ न प्रा॒णान् प्रा॒णान् न दा᳚क्षि॒णानि॑ दाक्षि॒णानि॒ न प्रा॒णान् । \newline
49. न प्रा॒णान् प्रा॒णान् न न प्रा॒णान् थ्सꣳ सम् प्रा॒णान् न न प्रा॒णान् थ्सम् । \newline
50. प्रा॒णान् थ्सꣳ सम् प्रा॒णान् प्रा॒णान् थ्सम् क॑र्.षति कर्.षति॒ सम् प्रा॒णान् प्रा॒णान् थ्सम् क॑र्.षति । \newline
51. प्रा॒णानिति॑ प्र - अ॒नान् । \newline
52. सम् क॑र्.षति कर्.षति॒ सꣳ सम् क॑र्.षति॒ नि नि क॑र्.षति॒ सꣳ सम् क॑र्.षति॒ नि । \newline
53. क॒र्॒.ष॒ति॒ नि नि क॑र्.षति कर्.षति॒ न्या᳚(1॒)न्ये᳚ ऽन्ये नि क॑र्.षति कर्.षति॒ न्य॑न्ये । \newline
54. न्या᳚(1॒)न्ये᳚ ऽन्ये नि न्य॑न्ये धिष्णि॑या॒ धिष्णि॑या अ॒न्ये नि न्य॑न्ये धिष्णि॑याः । \newline
55. अ॒न्ये धिष्णि॑या॒ धिष्णि॑या अ॒न्ये᳚ ऽन्ये धिष्णि॑या उ॒प्यन्त॑ उ॒प्यन्ते॒ धिष्णि॑या अ॒न्ये᳚ ऽन्ये धिष्णि॑या उ॒प्यन्ते᳚ । \newline
56. धिष्णि॑या उ॒प्यन्त॑ उ॒प्यन्ते॒ धिष्णि॑या॒ धिष्णि॑या उ॒प्यन्ते॒ न नोप्यन्ते॒ धिष्णि॑या॒ धिष्णि॑या उ॒प्यन्ते॒ न । \newline
57. उ॒प्यन्ते॒ न नोप्यन्त॑ उ॒प्यन्ते॒ नान्ये᳚ ऽन्ये नोप्यन्त॑ उ॒प्यन्ते॒ नान्ये । \newline
58. नान्ये᳚ ऽन्ये न नान्ये यान्. यान॒न्ये न नान्ये यान् । \newline
59. अ॒न्ये यान्. यान॒न्ये᳚ ऽन्ये यान् नि॒वप॑ति नि॒वप॑ति॒ यान॒न्ये᳚ ऽन्ये यान् नि॒वप॑ति । \newline
60. यान् नि॒वप॑ति नि॒वप॑ति॒ यान्. यान् नि॒वप॑ति॒ तेन॒ तेन॑ नि॒वप॑ति॒ यान्. यान् नि॒वप॑ति॒ तेन॑ । \newline
61. नि॒वप॑ति॒ तेन॒ तेन॑ नि॒वप॑ति नि॒वप॑ति॒ तेन॒ ताꣳ स्ताꣳ स्तेन॑ नि॒वप॑ति नि॒वप॑ति॒ तेन॒ तान् । \newline
62. नि॒वप॒तीति॑ नि - वप॑ति । \newline
63. तेन॒ ताꣳ स्ताꣳ स्तेन॒ तेन॒ तान् प्री॑णाति प्रीणाति॒ ताꣳ स्तेन॒ तेन॒ तान् प्री॑णाति । \newline
64. तान् प्री॑णाति प्रीणाति॒ ताꣳ स्तान् प्री॑णाति॒ यान्. यान् प्री॑णाति॒ ताꣳ स्तान् प्री॑णाति॒ यान् । \newline
65. प्री॒णा॒ति॒ यान्. यान् प्री॑णाति प्रीणाति॒ यान् न न यान् प्री॑णाति प्रीणाति॒ यान् न । \newline
66. यान् न न यान्. यान् न नि॒वप॑ति नि॒वप॑ति॒ न यान्. यान् न नि॒वप॑ति । \newline
67. न नि॒वप॑ति नि॒वप॑ति॒ न न नि॒वप॑ति॒ यद् यन् नि॒वप॑ति॒ न न नि॒वप॑ति॒ यत् । \newline
68. नि॒वप॑ति॒ यद् यन् नि॒वप॑ति नि॒वप॑ति॒ यद॑नुदि॒श त्य॑नुदि॒शति॒ यन् नि॒वप॑ति नि॒वप॑ति॒ यद॑नुदि॒शति॑ । \newline
69. नि॒वप॒तीति॑ नि - वप॑ति । \newline
70. यद॑नुदि॒श त्य॑नुदि॒शति॒ यद् यद॑नुदि॒शति॒ तेन॒ तेना॑ नुदि॒शति॒ यद् यद॑नुदि॒शति॒ तेन॑ । \newline
71. अ॒नु॒दि॒शति॒ तेन॒ तेना॑ नुदि॒श त्य॑नुदि॒शति॒ तेन॒ ताꣳ स्ताꣳ स्तेना॑ नुदि॒श त्य॑नुदि॒शति॒ तेन॒ तान् । \newline
72. अ॒नु॒दि॒शतीत्य॑नु - दि॒शति॑ । \newline
73. तेन॒ ताꣳ स्ताꣳ स्तेन॒ तेन॒ तान् । \newline
74. तानिति॒ तान् । \newline
\pagebreak
\markright{ TS 6.3.2.1  \hfill https://www.vedavms.in \hfill}

\section{ TS 6.3.2.1 }

\textbf{TS 6.3.2.1 } \newline
\textbf{Samhita Paata} \newline

सु॒व॒र्गाय॒ वा ए॒तानि॑ लो॒काय॑ हूयन्ते॒ यद् वै॑सर्ज॒नानि॒ द्वाभ्यां॒ गार्.ह॑पत्ये जुहोति द्वि॒पाद् यज॑मानः॒ प्रति॑ष्ठित्या॒ आग्नी᳚द्ध्रे जुहोत्य॒न्तरि॑क्ष ए॒वाऽऽ*क्र॑मत आहव॒नीये॑ जुहोति सुव॒र्गमे॒वैनं॑ ॅलो॒कं ग॑मयति दे॒वान्. वै सु॑व॒र्गं ॅलो॒कं ॅय॒तो रक्षाꣳ॑स्य जिघाꣳस॒न्ते सोमे॑न॒ राज्ञा॒ रक्षाꣳ॑-स्यप॒हत्या॒प्तु-मा॒त्मानं॑ कृ॒त्वा सु॑व॒र्गं ॅलो॒कमा॑य॒न् रक्ष॑सा॒-मनु॑पलाभा॒या ऽऽ*त्तः॒ सोमो॑ भव॒त्यथ॑- [  ] \newline

\textbf{Pada Paata} \newline

सु॒व॒र्गायेति॑ सुवः - गाय॑ । वै । ए॒तानि॑ । लो॒काय॑ । हू॒य॒न्ते॒ । यत् । वै॒स॒र्ज॒नानि॑ । द्वाभ्या᳚म् । गार्.ह॑पत्य॒ इति॒ गार्.ह॑ - प॒त्ये॒ । जु॒हो॒ति॒ । द्वि॒पादिति॑ द्वि - पात् । यज॑मानः । प्रति॑ष्ठित्या॒ इति॒ प्रति॑ - स्थि॒त्यै॒ । आग्नी᳚द्ध्र॒ इत्याग्नि॑-इ॒द्ध्रे॒ । जु॒हो॒ति॒ । अ॒न्तरि॑क्षे । ए॒व । एति॑ । क्र॒म॒ते॒ । आ॒ह॒व॒नीय॒ इत्या᳚ - ह॒व॒नीये᳚ । जु॒हो॒ति॒ । सु॒व॒र्गमिति॑ सुवः - गम् । ए॒व । ए॒न॒म् । लो॒कम् । ग॒म॒य॒ति॒ । दे॒वान् । वै । सु॒व॒र्गमिति॑ सुवः - गम् । लो॒कम् । य॒तः । रक्षाꣳ॑सि । अ॒जि॒घाꣳ॒॒स॒न्न् । ते । सोमे॑न । राज्ञा᳚ । रक्षाꣳ॑सि । अ॒प॒हत्येत्य॑प - हत्य॑ । अ॒प्तुम् । आ॒त्मान᳚म् । कृ॒त्वा । सु॒व॒र्गमिति॑ सुवः - गम् । लो॒कम् । आ॒य॒न्न् । रक्ष॑साम् । अनु॑पलाभा॒येत्यनु॑प - ला॒भा॒य॒ । आत्तः॑ । सोमः॑ । भ॒व॒ति॒ । अथ॑ ।  \newline


\textbf{Krama Paata} \newline

सु॒व॒र्गाय॒ वै । सु॒व॒र्गायेति॑ सुवः - गाय॑ । वा ए॒तानि॑ । ए॒तानि॑ लो॒काय॑ । लो॒काय॑ हूयन्ते । हू॒य॒न्ते॒ यत् । यद् वै॑सर्ज॒नानि॑ । वै॒स॒र्ज॒नानि॒ द्वाभ्या᳚म् । द्वाभ्या॒म् गार्.ह॑पत्ये । गार्.॑हपत्ये जुहोति । गार्.ह॑पत्य॒ इति॒ गार्.ह॑ - प॒त्ये॒ । जु॒हो॒ति॒ द्वि॒पात् । द्वि॒पाद् यज॑मानः । द्वि॒पादिति॑ द्वि - पात् । यज॑मानः॒ प्रति॑ष्ठित्यै । प्रति॑ष्ठित्या॒ आग्नी᳚द्ध्रे । प्रति॑ष्ठित्या॒ इति॒ प्रति॑ - स्थि॒त्यै॒ । आग्नी᳚द्ध्रे जुहोति । आग्नी᳚द्ध्र॒ इत्याग्नि॑ - इ॒द्ध्रे॒ । जु॒हो॒त्य॒न्तरि॑क्षे । अ॒न्तरि॑क्ष ए॒व । ए॒वा । आ क्र॑मते । क्र॒म॒त॒ आ॒ह॒व॒नीये᳚ । आ॒ह॒व॒नीये॑ जुहोति । आ॒ह॒व॒नीय॒ इत्या᳚ - ह॒व॒नीये᳚ । जु॒हो॒ति॒ सु॒व॒र्गम् । सु॒व॒र्गमे॒व । सु॒व॒र्गमिति॑ सुवः - गम् । ए॒वैन᳚म् । ए॒न॒म् ॅलो॒कम् । लो॒कम् ग॑मयति । ग॒म॒य॒ति॒ दे॒वान् । दे॒वान्. वै । वै सु॑व॒र्गम् । सु॒व॒र्गम् ॅलो॒कम् । सु॒व॒र्गमिति॑ सुवः - गम् । लो॒कम् ॅय॒तः । य॒तो रक्षाꣳ॑सि । रक्षाꣳ॑स्यजिघाꣳसन्न् । अ॒जि॒घाꣳ॒॒स॒न् ते । ते सोमे॑न । सोमे॑न॒ राज्ञा᳚ । राज्ञा॒ रक्षाꣳ॑सि । रक्षाꣳ॑स्यप॒हत्य॑ । अ॒प॒हत्या॒प्तुम् । अ॒प॒हत्येत्य॑प - हत्य॑ । अ॒प्तुमा॒त्मान᳚म् । आ॒त्मान॑म् कृ॒त्वा । कृ॒त्वा सु॑व॒र्गम् । सु॒व॒र्गम् ॅलो॒कम् । सु॒व॒र्गमिति॑ सुवः - गम् । लो॒कमा॑यन्न् । आ॒य॒न् रक्ष॑साम् । रक्ष॑सा॒मनु॑पलाभाय । अनु॑पलाभा॒यात्तः॑ । अनु॑पलाभा॒येत्यनु॑प - ला॒भा॒य॒ । आत्तः॒ सोमः॑ । सोमो॑ भवति । भ॒व॒त्यथ॑ । अथ॑ वैसर्ज॒नानि॑ \newline

\textbf{Jatai Paata} \newline

1. सु॒व॒र्गाय॒ वै वै सु॑व॒र्गाय॑ सुव॒र्गाय॒ वै । \newline
2. सु॒व॒र्गायेति॑ सुवः - गाय॑ । \newline
3. वा ए॒ता न्ये॒तानि॒ वै वा ए॒तानि॑ । \newline
4. ए॒तानि॑ लो॒काय॑ लो॒कायै॒ तान्ये॒तानि॑ लो॒काय॑ । \newline
5. लो॒काय॑ हूयन्ते हूयन्ते लो॒काय॑ लो॒काय॑ हूयन्ते । \newline
6. हू॒य॒न्ते॒ यद् यद्धू॑यन्ते हूयन्ते॒ यत् । \newline
7. यद् वै॑सर्ज॒नानि॑ वैसर्ज॒नानि॒ यद् यद् वै॑सर्ज॒नानि॑ । \newline
8. वै॒स॒र्ज॒नानि॒ द्वाभ्या॒म् द्वाभ्यां᳚ ॅवैसर्ज॒नानि॑ वैसर्ज॒नानि॒ द्वाभ्या᳚म् । \newline
9. द्वाभ्या॒म् गार्.ह॑पत्ये॒ गार्.ह॑पत्ये॒ द्वाभ्या॒म् द्वाभ्या॒म् गार्.ह॑पत्ये । \newline
10. गार्.ह॑पत्ये जुहोति जुहोति॒ गार्.ह॑पत्ये॒ गार्.ह॑पत्ये जुहोति । \newline
11. गार्.ह॑पत्य॒ इति॒ गार्.ह॑ - प॒त्ये॒ । \newline
12. जु॒हो॒ति॒ द्वि॒पाद् द्वि॒पाज् जु॑होति जुहोति द्वि॒पात् । \newline
13. द्वि॒पाद् यज॑मानो॒ यज॑मानो द्वि॒पाद् द्वि॒पाद् यज॑मानः । \newline
14. द्वि॒पादिति॑ द्वि - पात् । \newline
15. यज॑मानः॒ प्रति॑ष्ठित्यै॒ प्रति॑ष्ठित्यै॒ यज॑मानो॒ यज॑मानः॒ प्रति॑ष्ठित्यै । \newline
16. प्रति॑ष्ठित्या॒ आग्नी᳚द्ध्र॒ आग्नी᳚द्ध्रे॒ प्रति॑ष्ठित्यै॒ प्रति॑ष्ठित्या॒ आग्नी᳚द्ध्रे । \newline
17. प्रति॑ष्ठित्या॒ इति॒ प्रति॑ - स्थि॒त्यै॒ । \newline
18. आग्नी᳚द्ध्रे जुहोति जुहो॒ त्याग्नी᳚द्ध्र॒ आग्नी᳚द्ध्रे जुहोति । \newline
19. आग्नी᳚द्ध्र॒ इत्याग्नि॑ - इ॒द्ध्रे॒ । \newline
20. जु॒हो॒ त्य॒न्तरि॑क्षे॒ ऽन्तरि॑क्षे जुहोति जुहो त्य॒न्तरि॑क्षे । \newline
21. अ॒न्तरि॑क्ष ए॒वैवा न्तरि॑क्षे॒ ऽन्तरि॑क्ष ए॒व । \newline
22. ए॒वैवैवा । \newline
23. आ क्र॑मते क्रमत॒ आ क्र॑मते । \newline
24. क्र॒म॒त॒ आ॒ह॒व॒नीय॑ आहव॒नीये᳚ क्रमते क्रमत आहव॒नीये᳚ । \newline
25. आ॒ह॒व॒नीये॑ जुहोति जुहो त्याहव॒नीय॑ आहव॒नीये॑ जुहोति । \newline
26. आ॒ह॒व॒नीय॒ इत्या᳚ - ह॒व॒नीये᳚ । \newline
27. जु॒हो॒ति॒ सु॒व॒र्गꣳ सु॑व॒र्गम् जु॑होति जुहोति सुव॒र्गम् । \newline
28. सु॒व॒र्ग मे॒वैव सु॑व॒र्गꣳ सु॑व॒र्ग मे॒व । \newline
29. सु॒व॒र्गमिति॑ सुवः - गम् । \newline
30. ए॒वैन॑ मेन मे॒वै वैन᳚म् । \newline
31. ए॒न॒म् ॅलो॒कम् ॅलो॒क मे॑न मेनम् ॅलो॒कम् । \newline
32. लो॒कम् ग॑मयति गमयति लो॒कम् ॅलो॒कम् ग॑मयति । \newline
33. ग॒म॒य॒ति॒ दे॒वान् दे॒वान् ग॑मयति गमयति दे॒वान् । \newline
34. दे॒वान्. वै वै दे॒वान् दे॒वान्. वै । \newline
35. वै सु॑व॒र्गꣳ सु॑व॒र्गं ॅवै वै सु॑व॒र्गम् । \newline
36. सु॒व॒र्गम् ॅलो॒कम् ॅलो॒कꣳ सु॑व॒र्गꣳ सु॑व॒र्गम् ॅलो॒कम् । \newline
37. सु॒व॒र्गमिति॑ सुवः - गम् । \newline
38. लो॒कं ॅय॒तो य॒तो लो॒कम् ॅलो॒कं ॅय॒तः । \newline
39. य॒तो रक्षाꣳ॑सि॒ रक्षाꣳ॑सि य॒तो य॒तो रक्षाꣳ॑सि । \newline
40. रक्षाꣳ॑ स्यजिघाꣳसन् नजिघाꣳस॒न् रक्षाꣳ॑सि॒ रक्षाꣳ॑ स्यजिघाꣳसन्न् । \newline
41. अ॒जि॒घाꣳ॒॒स॒न् ते ते॑ ऽजिघाꣳसन् नजिघाꣳस॒न् ते । \newline
42. ते सोमे॑न॒ सोमे॑न॒ ते ते सोमे॑न । \newline
43. सोमे॑न॒ राज्ञा॒ राज्ञा॒ सोमे॑न॒ सोमे॑न॒ राज्ञा᳚ । \newline
44. राज्ञा॒ रक्षाꣳ॑सि॒ रक्षाꣳ॑सि॒ राज्ञा॒ राज्ञा॒ रक्षाꣳ॑सि । \newline
45. रक्षाꣳ॑ स्यप॒हत्या॑ प॒हत्य॒ रक्षाꣳ॑सि॒ रक्षाꣳ॑ स्यप॒हत्य॑ । \newline
46. अ॒प॒ह त्या॒प्तु म॒प्तु म॑प॒ह त्या॑प॒ह त्या॒प्तुम् । \newline
47. अ॒प॒हत्येत्य॑प - हत्य॑ । \newline
48. अ॒प्तु मा॒त्मान॑ मा॒त्मान॑ म॒प्तु म॒प्तु मा॒त्मान᳚म् । \newline
49. आ॒त्मान॑म् कृ॒त्वा कृ॒त्वा ऽऽत्मान॑ मा॒त्मान॑म् कृ॒त्वा । \newline
50. कृ॒त्वा सु॑व॒र्गꣳ सु॑व॒र्गम् कृ॒त्वा कृ॒त्वा सु॑व॒र्गम् । \newline
51. सु॒व॒र्गम् ॅलो॒कम् ॅलो॒कꣳ सु॑व॒र्गꣳ सु॑व॒र्गम् ॅलो॒कम् । \newline
52. सु॒व॒र्गमिति॑ सुवः - गम् । \newline
53. लो॒क मा॑यन् नायन् ॅलो॒कम् ॅलो॒क मा॑यन्न् । \newline
54. आ॒य॒न् रक्ष॑साꣳ॒॒ रक्ष॑सा मायन् नाय॒न् रक्ष॑साम् । \newline
55. रक्ष॑सा॒ मनु॑पलाभा॒या नु॑पलाभाय॒ रक्ष॑साꣳ॒॒ रक्ष॑सा॒ मनु॑पलाभाय । \newline
56. अनु॑पलाभा॒यात्त॒ आत्तो ऽनु॑पलाभा॒या नु॑पलाभा॒यात्तः॑ । \newline
57. अनु॑पलाभा॒येत्यनु॑प - ला॒भा॒य॒ । \newline
58. आत्तः॒ सोमः॒ सोम॒ आत्त॒ आत्तः॒ सोमः॑ । \newline
59. सोमो॑ भवति भवति॒ सोमः॒ सोमो॑ भवति । \newline
60. भ॒व॒ त्यथाथ॑ भवति भव॒ त्यथ॑ । \newline
61. अथ॑ वैसर्ज॒नानि॑ वैसर्ज॒ना न्यथाथ॑ वैसर्ज॒नानि॑ । \newline

\textbf{Ghana Paata } \newline

1. सु॒व॒र्गाय॒ वै वै सु॑व॒र्गाय॑ सुव॒र्गाय॒ वा ए॒ता न्ये॒तानि॒ वै सु॑व॒र्गाय॑ सुव॒र्गाय॒ वा ए॒तानि॑ । \newline
2. सु॒व॒र्गायेति॑ सुवः - गाय॑ । \newline
3. वा ए॒ता न्ये॒तानि॒ वै वा ए॒तानि॑ लो॒काय॑ लो॒का यै॒तानि॒ वै वा ए॒तानि॑ लो॒काय॑ । \newline
4. ए॒तानि॑ लो॒काय॑ लो॒का यै॒ता न्ये॒तानि॑ लो॒काय॑ हूयन्ते हूयन्ते लो॒का यै॒ता न्ये॒तानि॑ लो॒काय॑ हूयन्ते । \newline
5. लो॒काय॑ हूयन्ते हूयन्ते लो॒काय॑ लो॒काय॑ हूयन्ते॒ यद् यद्धू॑यन्ते लो॒काय॑ लो॒काय॑ हूयन्ते॒ यत् । \newline
6. हू॒य॒न्ते॒ यद् यद्धू॑यन्ते हूयन्ते॒ यद् वै॑सर्ज॒नानि॑ वैसर्ज॒नानि॒ यद्धू॑यन्ते हूयन्ते॒ यद् वै॑सर्ज॒नानि॑ । \newline
7. यद् वै॑सर्ज॒नानि॑ वैसर्ज॒नानि॒ यद् यद् वै॑सर्ज॒नानि॒ द्वाभ्या॒म् द्वाभ्यां᳚ ॅवैसर्ज॒नानि॒ यद् यद् वै॑सर्ज॒नानि॒ द्वाभ्या᳚म् । \newline
8. वै॒स॒र्ज॒नानि॒ द्वाभ्या॒म् द्वाभ्यां᳚ ॅवैसर्ज॒नानि॑ वैसर्ज॒नानि॒ द्वाभ्या॒म् गार्.ह॑पत्ये॒ गार्.ह॑पत्ये॒ द्वाभ्यां᳚ ॅवैसर्ज॒नानि॑ वैसर्ज॒नानि॒ द्वाभ्या॒म् गार्.ह॑पत्ये । \newline
9. द्वाभ्या॒म् गार्.ह॑पत्ये॒ गार्.ह॑पत्ये॒ द्वाभ्या॒म् द्वाभ्या॒म् गार्.ह॑पत्ये जुहोति जुहोति॒ गार्.ह॑पत्ये॒ द्वाभ्या॒म् द्वाभ्या॒म् गार्.ह॑पत्ये जुहोति । \newline
10. गार्.ह॑पत्ये जुहोति जुहोति॒ गार्.ह॑पत्ये॒ गार्.ह॑पत्ये जुहोति द्वि॒पाद् द्वि॒पाज् जु॑होति॒ गार्.ह॑पत्ये॒ गार्.ह॑पत्ये जुहोति द्वि॒पात् । \newline
11. गार्.ह॑पत्य॒ इति॒ गार्.ह॑ - प॒त्ये॒ । \newline
12. जु॒हो॒ति॒ द्वि॒पाद् द्वि॒पाज् जु॑होति जुहोति द्वि॒पाद् यज॑मानो॒ यज॑मानो द्वि॒पाज् जु॑होति जुहोति द्वि॒पाद् यज॑मानः । \newline
13. द्वि॒पाद् यज॑मानो॒ यज॑मानो द्वि॒पाद् द्वि॒पाद् यज॑मानः॒ प्रति॑ष्ठित्यै॒ प्रति॑ष्ठित्यै॒ यज॑मानो द्वि॒पाद् द्वि॒पाद् यज॑मानः॒ प्रति॑ष्ठित्यै । \newline
14. द्वि॒पादिति॑ द्वि - पात् । \newline
15. यज॑मानः॒ प्रति॑ष्ठित्यै॒ प्रति॑ष्ठित्यै॒ यज॑मानो॒ यज॑मानः॒ प्रति॑ष्ठित्या॒ आग्नी᳚द्ध्र॒ आग्नी᳚द्ध्रे॒ प्रति॑ष्ठित्यै॒ यज॑मानो॒ यज॑मानः॒ प्रति॑ष्ठित्या॒ आग्नी᳚द्ध्रे । \newline
16. प्रति॑ष्ठित्या॒ आग्नी᳚द्ध्र॒ आग्नी᳚द्ध्रे॒ प्रति॑ष्ठित्यै॒ प्रति॑ष्ठित्या॒ आग्नी᳚द्ध्रे जुहोति जुहो॒ त्याग्नी᳚द्ध्रे॒ प्रति॑ष्ठित्यै॒ प्रति॑ष्ठित्या॒ आग्नी᳚द्ध्रे जुहोति । \newline
17. प्रति॑ष्ठित्या॒ इति॒ प्रति॑ - स्थि॒त्यै॒ । \newline
18. आग्नी᳚द्ध्रे जुहोति जुहो॒ त्याग्नी᳚द्ध्र॒ आग्नी᳚द्ध्रे जुहो त्य॒न्तरि॑क्षे॒ ऽन्तरि॑क्षे जुहो॒ त्याग्नी᳚द्ध्र॒ आग्नी᳚द्ध्रे जुहो त्य॒न्तरि॑क्षे । \newline
19. आग्नी᳚द्ध्र॒ इत्याग्नि॑ - इ॒द्ध्रे॒ । \newline
20. जु॒हो॒ त्य॒न्तरि॑क्षे॒ ऽन्तरि॑क्षे जुहोति जुहो त्य॒न्तरि॑क्ष ए॒वै वान्तरि॑क्षे जुहोति जुहो त्य॒न्तरि॑क्ष ए॒व । \newline
21. अ॒न्तरि॑क्ष ए॒वै वान्तरि॑क्षे॒ ऽन्तरि॑क्ष ए॒वै वान्तरि॑क्षे॒ ऽन्तरि॑क्ष ए॒वा । \newline
22. ए॒वै वैवा क्र॑मते क्रमत॒ ऐवैवा क्र॑मते । \newline
23. आ क्र॑मते क्रमत॒ आ क्र॑मत आहव॒नीय॑ आहव॒नीये᳚ क्रमत॒ आ क्र॑मत आहव॒नीये᳚ । \newline
24. क्र॒म॒त॒ आ॒ह॒व॒नीय॑ आहव॒नीये᳚ क्रमते क्रमत आहव॒नीये॑ जुहोति जुहो त्याहव॒नीये᳚ क्रमते क्रमत आहव॒नीये॑ जुहोति । \newline
25. आ॒ह॒व॒नीये॑ जुहोति जुहो त्याहव॒नीय॑ आहव॒नीये॑ जुहोति सुव॒र्गꣳ सु॑व॒र्गम् जु॑हो त्याहव॒नीय॑ आहव॒नीये॑ जुहोति सुव॒र्गम् । \newline
26. आ॒ह॒व॒नीय॒ इत्या᳚ - ह॒व॒नीये᳚ । \newline
27. जु॒हो॒ति॒ सु॒व॒र्गꣳ सु॑व॒र्गम् जु॑होति जुहोति सुव॒र्ग मे॒वैव सु॑व॒र्गम् जु॑होति जुहोति सुव॒र्ग मे॒व । \newline
28. सु॒व॒र्ग मे॒वैव सु॑व॒र्गꣳ सु॑व॒र्ग मे॒वैन॑ मेन मे॒व सु॑व॒र्गꣳ सु॑व॒र्ग मे॒वैन᳚म् । \newline
29. सु॒व॒र्गमिति॑ सुवः - गम् । \newline
30. ए॒वैन॑ मेन मे॒वै वैन॑म् ॅलो॒कम् ॅलो॒क मे॑न मे॒वै वैन॑म् ॅलो॒कम् । \newline
31. ए॒न॒म् ॅलो॒कम् ॅलो॒क मे॑न मेनम् ॅलो॒कम् ग॑मयति गमयति लो॒क मे॑न मेनम् ॅलो॒कम् ग॑मयति । \newline
32. लो॒कम् ग॑मयति गमयति लो॒कम् ॅलो॒कम् ग॑मयति दे॒वान् दे॒वान् ग॑मयति लो॒कम् ॅलो॒कम् ग॑मयति दे॒वान् । \newline
33. ग॒म॒य॒ति॒ दे॒वान् दे॒वान् ग॑मयति गमयति दे॒वान्. वै वै दे॒वान् ग॑मयति गमयति दे॒वान्. वै । \newline
34. दे॒वान्. वै वै दे॒वान् दे॒वान्. वै सु॑व॒र्गꣳ सु॑व॒र्गं ॅवै दे॒वान् दे॒वान्. वै सु॑व॒र्गम् । \newline
35. वै सु॑व॒र्गꣳ सु॑व॒र्गं ॅवै वै सु॑व॒र्गम् ॅलो॒कम् ॅलो॒कꣳ सु॑व॒र्गं ॅवै वै सु॑व॒र्गम् ॅलो॒कम् । \newline
36. सु॒व॒र्गम् ॅलो॒कम् ॅलो॒कꣳ सु॑व॒र्गꣳ सु॑व॒र्गम् ॅलो॒कं ॅय॒तो य॒तो लो॒कꣳ सु॑व॒र्गꣳ सु॑व॒र्गम् ॅलो॒कं ॅय॒तः । \newline
37. सु॒व॒र्गमिति॑ सुवः - गम् । \newline
38. लो॒कं ॅय॒तो य॒तो लो॒कम् ॅलो॒कं ॅय॒तो रक्षाꣳ॑सि॒ रक्षाꣳ॑सि य॒तो लो॒कम् ॅलो॒कं ॅय॒तो रक्षाꣳ॑सि । \newline
39. य॒तो रक्षाꣳ॑सि॒ रक्षाꣳ॑सि य॒तो य॒तो रक्षाꣳ॑ स्यजिघाꣳसन् नजिघाꣳस॒न् रक्षाꣳ॑सि य॒तो य॒तो रक्षाꣳ॑ स्यजिघाꣳसन्न् । \newline
40. रक्षाꣳ॑ स्यजिघाꣳसन् नजिघाꣳस॒न् रक्षाꣳ॑सि॒ रक्षाꣳ॑ स्यजिघाꣳस॒न् ते ते॑ ऽजिघाꣳस॒न् रक्षाꣳ॑सि॒ रक्षाꣳ॑ स्यजिघाꣳस॒न् ते । \newline
41. अ॒जि॒घाꣳ॒॒स॒न् ते ते॑ ऽजिघाꣳसन् नजिघाꣳस॒न् ते सोमे॑न॒ सोमे॑न॒ ते॑ ऽजिघाꣳसन् नजिघाꣳस॒न् ते सोमे॑न । \newline
42. ते सोमे॑न॒ सोमे॑न॒ ते ते सोमे॑न॒ राज्ञा॒ राज्ञा॒ सोमे॑न॒ ते ते सोमे॑न॒ राज्ञा᳚ । \newline
43. सोमे॑न॒ राज्ञा॒ राज्ञा॒ सोमे॑न॒ सोमे॑न॒ राज्ञा॒ रक्षाꣳ॑सि॒ रक्षाꣳ॑सि॒ राज्ञा॒ सोमे॑न॒ सोमे॑न॒ राज्ञा॒ रक्षाꣳ॑सि । \newline
44. राज्ञा॒ रक्षाꣳ॑सि॒ रक्षाꣳ॑सि॒ राज्ञा॒ राज्ञा॒ रक्षाꣳ॑ स्यप॒हत्या॑ प॒हत्य॒ रक्षाꣳ॑सि॒ राज्ञा॒ राज्ञा॒ रक्षाꣳ॑ स्यप॒हत्य॑ । \newline
45. रक्षाꣳ॑ स्यप॒हत्या॑ प॒हत्य॒ रक्षाꣳ॑सि॒ रक्षाꣳ॑ स्यप॒हत्या॒ प्तु म॒प्तु म॑प॒हत्य॒ रक्षाꣳ॑सि॒ रक्षाꣳ॑ स्यप॒हत्या॒प्तुम् । \newline
46. अ॒प॒हत्या॒ प्तु म॒प्तु म॑प॒हत्या॑ प॒हत्या॒ प्तु मा॒त्मान॑ मा॒त्मान॑ म॒प्तु म॑प॒हत्या॑ प॒हत्या॒ प्तु मा॒त्मान᳚म् । \newline
47. अ॒प॒हत्येत्य॑प - हत्य॑ । \newline
48. अ॒प्तु मा॒त्मान॑ मा॒त्मान॑ म॒प्तु म॒प्तु मा॒त्मान॑म् कृ॒त्वा कृ॒त्वा ऽऽत्मान॑ म॒प्तु म॒प्तु मा॒त्मान॑म् कृ॒त्वा । \newline
49. आ॒त्मान॑म् कृ॒त्वा कृ॒त्वा ऽऽत्मान॑ मा॒त्मान॑म् कृ॒त्वा सु॑व॒र्गꣳ सु॑व॒र्गम् कृ॒त्वा ऽऽत्मान॑ मा॒त्मान॑म् कृ॒त्वा सु॑व॒र्गम् । \newline
50. कृ॒त्वा सु॑व॒र्गꣳ सु॑व॒र्गम् कृ॒त्वा कृ॒त्वा सु॑व॒र्गम् ॅलो॒कम् ॅलो॒कꣳ सु॑व॒र्गम् कृ॒त्वा कृ॒त्वा सु॑व॒र्गम् ॅलो॒कम् । \newline
51. सु॒व॒र्गम् ॅलो॒कम् ॅलो॒कꣳ सु॑व॒र्गꣳ सु॑व॒र्गम् ॅलो॒क मा॑यन् नायन् ॅलो॒कꣳ सु॑व॒र्गꣳ सु॑व॒र्गम् ॅलो॒क मा॑यन्न् । \newline
52. सु॒व॒र्गमिति॑ सुवः - गम् । \newline
53. लो॒क मा॑यन् नायन् ॅलो॒कम् ॅलो॒क मा॑य॒न् रक्ष॑साꣳ॒॒ रक्ष॑सा मायन् ॅलो॒कम् ॅलो॒क मा॑य॒न् रक्ष॑साम् । \newline
54. आ॒य॒न् रक्ष॑साꣳ॒॒ रक्ष॑सा मायन् नाय॒न् रक्ष॑सा॒ मनु॑पलाभा॒या नु॑पलाभाय॒ रक्ष॑सा मायन् नाय॒न् रक्ष॑सा॒ मनु॑पलाभाय । \newline
55. रक्ष॑सा॒ मनु॑पलाभा॒या नु॑पलाभाय॒ रक्ष॑साꣳ॒॒ रक्ष॑सा॒ मनु॑पलाभा॒ यात्त॒ आत्तो ऽनु॑पलाभाय॒ रक्ष॑साꣳ॒॒ रक्ष॑सा॒ मनु॑पलाभा॒ यात्तः॑ । \newline
56. अनु॑पलाभा॒ यात्त॒ आत्तो ऽनु॑पलाभा॒या नु॑पलाभा॒यात्तः॒ सोमः॒ सोम॒ आत्तो ऽनु॑पलाभा॒या नु॑पलाभा॒यात्तः॒ सोमः॑ । \newline
57. अनु॑पलाभा॒येत्यनु॑प - ला॒भा॒य॒ । \newline
58. आत्तः॒ सोमः॒ सोम॒ आत्त॒ आत्तः॒ सोमो॑ भवति भवति॒ सोम॒ आत्त॒ आत्तः॒ सोमो॑ भवति । \newline
59. सोमो॑ भवति भवति॒ सोमः॒ सोमो॑ भव॒ त्यथाथ॑ भवति॒ सोमः॒ सोमो॑ भव॒ त्यथ॑ । \newline
60. भ॒व॒ त्यथाथ॑ भवति भव॒ त्यथ॑ वैसर्ज॒नानि॑ वैसर्ज॒ना न्यथ॑ भवति भव॒ त्यथ॑ वैसर्ज॒नानि॑ । \newline
61. अथ॑ वैसर्ज॒नानि॑ वैसर्ज॒ना न्यथाथ॑ वैसर्ज॒नानि॑ जुहोति जुहोति वैसर्ज॒ना न्यथाथ॑ वैसर्ज॒नानि॑ जुहोति । \newline
\pagebreak
\markright{ TS 6.3.2.2  \hfill https://www.vedavms.in \hfill}

\section{ TS 6.3.2.2 }

\textbf{TS 6.3.2.2 } \newline
\textbf{Samhita Paata} \newline

वैसर्ज॒नानि॑ जुहोति॒ रक्ष॑सा॒मप॑हत्यै॒ त्वꣳ सो॑म तनू॒कृद्भ्य॒ इत्या॑ह तनू॒कृद्ध्य॑ष द्वेषो᳚भ्यो॒ऽन्यकृ॑तेभ्य॒ इत्या॑हा॒न्यकृ॑तानि॒ हि रक्षाꣳ॑स्यु॒रु य॒न्ताऽसि॒ वरू॑थ॒मित्या॑हो॒रु ण॑स्कृ॒धीति॒ वावैतदा॑ह जुषा॒णो अ॒प्तुराज्य॑स्य वे॒त्वित्या॑हा॒प्तुमे॒व यज॑मानं कृ॒त्वा सु॑व॒र्गं ॅलो॒कं ग॑मयति॒ रक्ष॑सा॒-मनु॑पलाभा॒याऽऽ* सोमं॑ ददत॒- [  ] \newline

\textbf{Pada Paata} \newline

वै॒स॒र्ज॒नानि॑ । जु॒हो॒ति॒ । रक्ष॑साम् । अप॑हत्या॒ इत्यप॑ - ह॒त्यै॒ । त्वम् । सो॒म॒ । त॒नू॒कृद्भ्य॒ इति॑ तनू॒कृत् - भ्यः॒ । इति॑ । आ॒ह॒ । त॒नू॒कृदिति॑ तनू - कृत् । हि । ए॒षः । द्वेषो᳚भ्य॒ इति॒ द्वेषः॑ - भ्यः॒ । अ॒न्यकृ॑तेभ्य॒ इत्य॒न्य - कृ॒ते॒भ्यः॒ । इति॑ । आ॒ह॒ । अ॒न्यकृ॑ता॒नीत्य॒न्य - कृ॒ता॒नि॒ । हि । रक्षाꣳ॑सि । उ॒रु । य॒न्ता । अ॒सि॒ । वरू॑थम् । इति॑ । आ॒ह॒ । उ॒रु । नः॒ । कृ॒धि॒ । इति॑ । वाव । ए॒तत् । आ॒ह॒ । जु॒षा॒णः । अ॒प्तुः । आज्य॑स्य । वे॒तु॒ । इति॑ । आ॒ह॒ । अ॒प्तुम् । ए॒व । यज॑मानम् । कृ॒त्वा । सु॒व॒र्गमिति॑ सुवः - गम् । लो॒कम् । ग॒म॒य॒ति॒ । रक्ष॑साम् । अनु॑पलाभा॒येत्यनु॑प - ला॒भा॒य॒ । एति॑ । सोम᳚म् । द॒द॒ते॒ ।  \newline


\textbf{Krama Paata} \newline

वै॒स॒र्ज॒नानि॑ जुहोति । जु॒हो॒ति॒ रक्ष॑साम् । रक्ष॑सा॒मप॑हत्यै । अप॑हत्यै॒ त्वम् । अप॑हत्या॒ इत्यप॑ - ह॒त्यै॒ । त्वꣳ सो॑म । सो॒म॒ त॒नू॒कृद्भ्यः॑ । त॒नू॒कृद्भ्य॒ इति॑ । त॒नू॒कृद्भ्य॒ इति॑ तनू॒कृत् - भ्यः॒ । इत्या॑ह । आ॒ह॒ त॒नू॒कृत् । त॒नू॒कृद्‌धि । त॒नू॒कृदिति॑ तनू - कृत् । ह्ये॑षः । ए॒ष द्वेषो᳚भ्यः । द्वेषो᳚भ्यो॒ऽन्यकृ॑तेभ्यः । द्वेषो᳚भ्य॒ इति॒ द्वेषः॑ - भ्यः॒ । अ॒न्यकृ॑तेभ्य॒ इति॑ । अ॒न्यकृ॑तेभ्य॒ इत्य॒न्य - कृ॒ते॒भ्यः॒ । इत्या॑ह । आ॒हा॒न्यकृ॑तानि । अ॒न्यकृ॑तानि॒ हि । अ॒न्यकृ॑ता॒नीत्य॒न्य - कृ॒ता॒नि॒ । हि रक्षाꣳ॑सि । रक्षाꣳ॑स्यु॒रु । उ॒रु य॒न्ता । य॒न्ताऽसि॑ । अ॒सि॒ वरू॑थम् । वरू॑थ॒मिति॑ । इत्या॑ह । आ॒हो॒रु । उ॒रु णः॑ । न॒स्कृ॒धि॒ । कृ॒धीति॑ । इति॒ वाव । वावैतत् । ए॒तदा॑ह । आ॒ह॒ जु॒षा॒णः । जु॒षा॒णो अ॒प्तुः । अ॒प्तुराज्य॑स्य । आज्य॑स्य वेतु । वे॒त्विति॑ । इत्या॑ह । आ॒हा॒प्तुम् । अ॒प्तुमे॒व । ए॒व यज॑मानम् । यज॑मानम् कृ॒त्वा । कृ॒त्वा सु॑व॒र्गम् । सु॒व॒र्गम् ॅलो॒कम् । सु॒व॒र्गमिति॑ सुवः - गम् । लो॒कम् ग॑मयति । ग॒म॒य॒ति॒ रक्ष॑साम् । रक्ष॑सा॒मनु॑पलाभाय । अनु॑पलाभा॒या । अनु॑पलाभा॒येत्यनु॑प - ला॒भा॒य॒ । आ सोम᳚म् । सोम॑म् ददते । द॒द॒त॒ आ \newline

\textbf{Jatai Paata} \newline

1. वै॒स॒र्ज॒नानि॑ जुहोति जुहोति वैसर्ज॒नानि॑ वैसर्ज॒नानि॑ जुहोति । \newline
2. जु॒हो॒ति॒ रक्ष॑साꣳ॒॒ रक्ष॑साम् जुहोति जुहोति॒ रक्ष॑साम् । \newline
3. रक्ष॑सा॒ मप॑हत्या॒ अप॑हत्यै॒ रक्ष॑साꣳ॒॒ रक्ष॑सा॒ मप॑हत्यै । \newline
4. अप॑हत्यै॒ त्वम् त्व मप॑हत्या॒ अप॑हत्यै॒ त्वम् । \newline
5. अप॑हत्या॒ इत्यप॑ - ह॒त्यै॒ । \newline
6. त्वꣳ सो॑म सोम॒ त्वम् त्वꣳ सो॑म । \newline
7. सो॒म॒ त॒नू॒कृद्भ्य॑ स्तनू॒कृद्भ्यः॑ सोम सोम तनू॒कृद्भ्यः॑ । \newline
8. त॒नू॒कृद्भ्य॒ इतीति॑ तनू॒कृद्भ्य॑ स्तनू॒कृद्भ्य॒ इति॑ । \newline
9. त॒नू॒कृद्भ्य॒ इति॑ तनू॒कृत् - भ्यः॒ । \newline
10. इत्या॑हा॒हे तीत्या॑ह । \newline
11. आ॒ह॒ त॒नू॒कृत् त॑नू॒कृ दा॑हाह तनू॒कृत् । \newline
12. त॒नू॒कृद्धि हि त॑नू॒कृत् त॑नू॒कृद्धि । \newline
13. त॒नू॒कृदिति॑ तनू - कृत् । \newline
14. ह्ये॑ष ए॒ष हि ह्ये॑षः । \newline
15. ए॒ष द्वेषो᳚भ्यो॒ द्वेषो᳚भ्य ए॒ष ए॒ष द्वेषो᳚भ्यः । \newline
16. द्वेषो᳚भ्यो॒ ऽन्यकृ॑तेभ्यो॒ ऽन्यकृ॑तेभ्यो॒ द्वेषो᳚भ्यो॒ द्वेषो᳚भ्यो॒ ऽन्यकृ॑तेभ्यः । \newline
17. द्वेषो᳚भ्य॒ इति॒ द्वेषः॑ - भ्यः॒ । \newline
18. अ॒न्यकृ॑तेभ्य॒ इती त्य॒न्यकृ॑तेभ्यो॒ ऽन्यकृ॑तेभ्य॒ इति॑ । \newline
19. अ॒न्यकृ॑तेभ्य॒ इत्य॒न्य - कृ॒ते॒भ्यः॒ । \newline
20. इत्या॑हा॒हे तीत्या॑ह । \newline
21. आ॒हा॒ न्यकृ॑ता न्य॒न्यकृ॑ता न्याहाहा॒ न्यकृ॑तानि । \newline
22. अ॒न्यकृ॑तानि॒ हि ह्य॑न्यकृ॑ता न्य॒ न्यकृ॑तानि॒ हि । \newline
23. अ॒न्यकृ॑ता॒नीत्य॒न्य - कृ॒ता॒नि॒ । \newline
24. हि रक्षाꣳ॑सि॒ रक्षाꣳ॑सि॒ हि हि रक्षाꣳ॑सि । \newline
25. रक्षाꣳ॑ स्यु॒ रू॑रु रक्षाꣳ॑सि॒ रक्षाꣳ॑ स्यु॒रु । \newline
26. उ॒रु य॒न्ता य॒न्तो रू॑रु य॒न्ता । \newline
27. य॒न्ता ऽस्य॑सि य॒न्ता य॒न्ता ऽसि॑ । \newline
28. अ॒सि॒ वरू॑थं॒ ॅवरू॑थ मस्यसि॒ वरू॑थम् । \newline
29. वरू॑थ॒ मितीति॒ वरू॑थं॒ ॅवरू॑थ॒ मिति॑ । \newline
30. इत्या॑हा॒हे तीत्या॑ह । \newline
31. आ॒हो॒रू᳚(1॒)र्वा॑हा हो॒रु । \newline
32. उ॒रु णो॑ न उ॒रू॑रु णः॑ । \newline
33. न॒ स्कृ॒धि॒ कृ॒धि॒ नो॒ न॒ स्कृ॒धि॒ । \newline
34. कृ॒धी तीति॑ कृधि कृ॒धीति॑ । \newline
35. इति॒ वाव वावे तीति॒ वाव । \newline
36. वावैत दे॒तद् वाव वावैतत् । \newline
37. ए॒त दा॑हा है॒त दे॒त दा॑ह । \newline
38. आ॒ह॒ जु॒षा॒णो जु॑षा॒ण आ॑हाह जुषा॒णः । \newline
39. जु॒षा॒णो अ॒प्तु र॒प्तुर् जु॑षा॒णो जु॑षा॒णो अ॒प्तुः । \newline
40. अ॒प्तु राज्य॒स्याज्य॑ स्या॒प्तु र॒प्तु राज्य॑स्य । \newline
41. आज्य॑स्य वेतु वे॒त्वाज्य॒ स्याज्य॑स्य वेतु । \newline
42. वे॒त्वितीति॑ वेतु वे॒त्विति॑ । \newline
43. इत्या॑हा॒हे तीत्या॑ह । \newline
44. आ॒हा॒प्तु म॒प्तु मा॑हा हा॒प्तुम् । \newline
45. अ॒प्तु मे॒वै वाप्तु म॒प्तु मे॒व । \newline
46. ए॒व यज॑मानं॒ ॅयज॑मान मे॒वैव यज॑मानम् । \newline
47. यज॑मानम् कृ॒त्वा कृ॒त्वा यज॑मानं॒ ॅयज॑मानम् कृ॒त्वा । \newline
48. कृ॒त्वा सु॑व॒र्गꣳ सु॑व॒र्गम् कृ॒त्वा कृ॒त्वा सु॑व॒र्गम् । \newline
49. सु॒व॒र्गम् ॅलो॒कम् ॅलो॒कꣳ सु॑व॒र्गꣳ सु॑व॒र्गम् ॅलो॒कम् । \newline
50. सु॒व॒र्गमिति॑ सुवः - गम् । \newline
51. लो॒कम् ग॑मयति गमयति लो॒कम् ॅलो॒कम् ग॑मयति । \newline
52. ग॒म॒य॒ति॒ रक्ष॑साꣳ॒॒ रक्ष॑साम् गमयति गमयति॒ रक्ष॑साम् । \newline
53. रक्ष॑सा॒ मनु॑पलाभा॒या नु॑पलाभाय॒ रक्ष॑साꣳ॒॒ रक्ष॑सा॒ मनु॑पलाभाय । \newline
54. अनु॑पलाभा॒या ऽनु॑पलाभा॒या नु॑पलाभा॒या । \newline
55. अनु॑पलाभा॒येत्यनु॑प - ला॒भा॒य॒ । \newline
56. आ सोमꣳ॒॒ सोम॒ मा सोम᳚म् । \newline
57. सोम॑म् ददते ददते॒ सोमꣳ॒॒ सोम॑म् ददते । \newline
58. द॒द॒त॒ आ द॑दते ददत॒ आ । \newline

\textbf{Ghana Paata } \newline

1. वै॒स॒र्ज॒नानि॑ जुहोति जुहोति वैसर्ज॒नानि॑ वैसर्ज॒नानि॑ जुहोति॒ रक्ष॑साꣳ॒॒ रक्ष॑साम् जुहोति वैसर्ज॒नानि॑ वैसर्ज॒नानि॑ जुहोति॒ रक्ष॑साम् । \newline
2. जु॒हो॒ति॒ रक्ष॑साꣳ॒॒ रक्ष॑साम् जुहोति जुहोति॒ रक्ष॑सा॒ मप॑हत्या॒ अप॑हत्यै॒ रक्ष॑साम् जुहोति जुहोति॒ रक्ष॑सा॒ मप॑हत्यै । \newline
3. रक्ष॑सा॒ मप॑हत्या॒ अप॑हत्यै॒ रक्ष॑साꣳ॒॒ रक्ष॑सा॒ मप॑हत्यै॒ त्वम् त्व मप॑हत्यै॒ रक्ष॑साꣳ॒॒ रक्ष॑सा॒ मप॑हत्यै॒ त्वम् । \newline
4. अप॑हत्यै॒ त्वम् त्व मप॑हत्या॒ अप॑हत्यै॒ त्वꣳ सो॑म सोम॒ त्व मप॑हत्या॒ अप॑हत्यै॒ त्वꣳ सो॑म । \newline
5. अप॑हत्या॒ इत्यप॑ - ह॒त्यै॒ । \newline
6. त्वꣳ सो॑म सोम॒ त्वम् त्वꣳ सो॑म तनू॒कृद्भ्य॑ स्तनू॒कृद्भ्यः॑ सोम॒ त्वम् त्वꣳ सो॑म तनू॒कृद्भ्यः॑ । \newline
7. सो॒म॒ त॒नू॒कृद्भ्य॑ स्तनू॒कृद्भ्यः॑ सोम सोम तनू॒कृद्भ्य॒ इतीति॑ तनू॒कृद्भ्यः॑ सोम सोम तनू॒कृद्भ्य॒ इति॑ । \newline
8. त॒नू॒कृद्भ्य॒ इतीति॑ तनू॒कृद्भ्य॑ स्तनू॒कृद्भ्य॒ इत्या॑हा॒हेति॑ तनू॒कृद्भ्य॑ स्तनू॒कृद्भ्य॒ इत्या॑ह । \newline
9. त॒नू॒कृद्भ्य॒ इति॑ तनू॒कृत् - भ्यः॒ । \newline
10. इत्या॑हा॒हे तीत्या॑ह तनू॒कृत् त॑नू॒कृ दा॒हेतीत्या॑ह तनू॒कृत् । \newline
11. आ॒ह॒ त॒नू॒कृत् त॑नू॒कृ दा॑हाह तनू॒कृद्धि हि त॑नू॒कृ दा॑हाह तनू॒कृद्धि । \newline
12. त॒नू॒कृद्धि हि त॑नू॒कृत् त॑नू॒कृ द्ध्ये॑ष ए॒ष हि त॑नू॒कृत् त॑नू॒कृ द्ध्ये॑षः । \newline
13. त॒नू॒कृदिति॑ तनू - कृत् । \newline
14. ह्ये॑ष ए॒ष हि ह्ये॑ष द्वेषो᳚भ्यो॒ द्वेषो᳚भ्य ए॒ष हि ह्ये॑ष द्वेषो᳚भ्यः । \newline
15. ए॒ष द्वेषो᳚भ्यो॒ द्वेषो᳚भ्य ए॒ष ए॒ष द्वेषो᳚भ्यो॒ ऽन्यकृ॑तेभ्यो॒ ऽन्यकृ॑तेभ्यो॒ द्वेषो᳚भ्य ए॒ष ए॒ष द्वेषो᳚भ्यो॒ ऽन्यकृ॑तेभ्यः । \newline
16. द्वेषो᳚भ्यो॒ ऽन्यकृ॑तेभ्यो॒ ऽन्यकृ॑तेभ्यो॒ द्वेषो᳚भ्यो॒ द्वेषो᳚भ्यो॒ ऽन्यकृ॑तेभ्य॒ इती त्य॒न्यकृ॑तेभ्यो॒ द्वेषो᳚भ्यो॒ द्वेषो᳚भ्यो॒ ऽन्यकृ॑तेभ्य॒ इति॑ । \newline
17. द्वेषो᳚भ्य॒ इति॒ द्वेषः॑ - भ्यः॒ । \newline
18. अ॒न्यकृ॑तेभ्य॒ इती त्य॒न्यकृ॑तेभ्यो॒ ऽन्यकृ॑तेभ्य॒ इत्या॑हा॒हे त्य॒न्यकृ॑तेभ्यो॒ ऽन्यकृ॑तेभ्य॒ इत्या॑ह । \newline
19. अ॒न्यकृ॑तेभ्य॒ इत्य॒न्य - कृ॒ते॒भ्यः॒ । \newline
20. इत्या॑हा॒हे तीत्या॑हा॒ न्यकृ॑ता न्य॒न्यकृ॑ता न्या॒हेती त्या॑हा॒ न्यकृ॑तानि । \newline
21. आ॒हा॒ न्यकृ॑ता न्य॒न्यकृ॑ता न्याहाहा॒ न्यकृ॑तानि॒ हि ह्य॑न्यकृ॑ता न्याहाहा॒ न्यकृ॑तानि॒ हि । \newline
22. अ॒न्यकृ॑तानि॒ हि ह्य॑न्यकृ॑ता न्य॒न्यकृ॑तानि॒ हि रक्षाꣳ॑सि॒ रक्षाꣳ॑सि॒ ह्य॑न्यकृ॑ता न्य॒न्यकृ॑तानि॒ हि रक्षाꣳ॑सि । \newline
23. अ॒न्यकृ॑ता॒नीत्य॒न्य - कृ॒ता॒नि॒ । \newline
24. हि रक्षाꣳ॑सि॒ रक्षाꣳ॑सि॒ हि हि रक्षाꣳ॑ स्यु॒रू॑रु रक्षाꣳ॑सि॒ हि हि रक्षाꣳ॑ स्यु॒रु । \newline
25. रक्षाꣳ॑ स्यु॒रू॑रु रक्षाꣳ॑सि॒ रक्षाꣳ॑ स्यु॒रु य॒न्ता य॒न्तोरु रक्षाꣳ॑सि॒ रक्षाꣳ॑ स्यु॒रु य॒न्ता । \newline
26. उ॒रु य॒न्ता य॒न्तो रू॑रु य॒न्ता ऽस्य॑सि य॒न्तो रू॑रु य॒न्ता ऽसि॑ । \newline
27. य॒न्ता ऽस्य॑सि य॒न्ता य॒न्ता ऽसि॒ वरू॑थं॒ ॅवरू॑थ मसि य॒न्ता य॒न्ता ऽसि॒ वरू॑थम् । \newline
28. अ॒सि॒ वरू॑थं॒ ॅवरू॑थ मस्यसि॒ वरू॑थ॒ मितीति॒ वरू॑थ मस्यसि॒ वरू॑थ॒ मिति॑ । \newline
29. वरू॑थ॒ मितीति॒ वरू॑थं॒ ॅवरू॑थ॒ मित्या॑हा॒हेति॒ वरू॑थं॒ ॅवरू॑थ॒ मित्या॑ह । \newline
30. इत्या॑हा॒हे तीत्या॑ हो॒रू᳚(1॒) र्वा॑हे तीत्या॑ हो॒रु । \newline
31. आ॒हो॒रू᳚(1॒) र्वा॑हा हो॒रु णो॑ न उ॒र्वा॑हा हो॒रु णः॑ । \newline
32. उ॒रु णो॑ न उ॒रू॑रु ण॑ स्कृधि कृधि न उ॒रू॑रु ण॑ स्कृधि । \newline
33. न॒ स्कृ॒धि॒ कृ॒धि॒ नो॒ न॒ स्कृ॒धी तीति॑ कृधि नो न स्कृ॒धीति॑ । \newline
34. कृ॒धी तीति॑ कृधि कृ॒धीति॒ वाव वावेति॑ कृधि कृ॒धीति॒ वाव । \newline
35. इति॒ वाव वावे तीति॒ वावैत दे॒तद् वावे तीति॒ वावैतत् । \newline
36. वावैत दे॒तद् वाव वावै तदा॑हा है॒तद् वाव वावै तदा॑ह । \newline
37. ए॒त दा॑हा है॒त दे॒त दा॑ह जुषा॒णो जु॑षा॒ण आ॑है॒त दे॒त दा॑ह जुषा॒णः । \newline
38. आ॒ह॒ जु॒षा॒णो जु॑षा॒ण आ॑हाह जुषा॒णो अ॒प्तु र॒प्तुर् जु॑षा॒ण आ॑हाह जुषा॒णो अ॒प्तुः । \newline
39. जु॒षा॒णो अ॒प्तु र॒प्तुर् जु॑षा॒णो जु॑षा॒णो अ॒प्तु राज्य॒स्या ज्य॑स्या॒प्तुर् जु॑षा॒णो जु॑षा॒णो अ॒प्तु राज्य॑स्य । \newline
40. अ॒प्तु राज्य॒स्या ज्य॑स्या॒प्तु र॒प्तु राज्य॑स्य वेतु वे॒त्वा ज्य॑स्या॒प्तु र॒प्तु राज्य॑स्य वेतु । \newline
41. आज्य॑स्य वेतु वे॒त्वा ज्य॒स्या ज्य॑स्य वे॒त्वि तीति॑ वे॒त्वा ज्य॒स्या ज्य॑स्य वे॒त्विति॑ । \newline
42. वे॒त्वि तीति॑ वेतु वे॒त्वि त्या॑हा॒हेति॑ वेतु वे॒त्वि त्या॑ह । \newline
43. इत्या॑हा॒हे तीत्या॑ हा॒प्तु म॒प्तु मा॒हे तीत्या॑ हा॒प्तुम् । \newline
44. आ॒हा॒प्तु म॒प्तु मा॑हा हा॒प्तु मे॒वै वाप्तु मा॑हा हा॒प्तु मे॒व । \newline
45. अ॒प्तु मे॒वै वाप्तु म॒प्तु मे॒व यज॑मानं॒ ॅयज॑मान मे॒वाप्तु म॒प्तु मे॒व यज॑मानम् । \newline
46. ए॒व यज॑मानं॒ ॅयज॑मान मे॒वैव यज॑मानम् कृ॒त्वा कृ॒त्वा यज॑मान मे॒वैव यज॑मानम् कृ॒त्वा । \newline
47. यज॑मानम् कृ॒त्वा कृ॒त्वा यज॑मानं॒ ॅयज॑मानम् कृ॒त्वा सु॑व॒र्गꣳ सु॑व॒र्गम् कृ॒त्वा यज॑मानं॒ ॅयज॑मानम् कृ॒त्वा सु॑व॒र्गम् । \newline
48. कृ॒त्वा सु॑व॒र्गꣳ सु॑व॒र्गम् कृ॒त्वा कृ॒त्वा सु॑व॒र्गम् ॅलो॒कम् ॅलो॒कꣳ सु॑व॒र्गम् कृ॒त्वा कृ॒त्वा सु॑व॒र्गम् ॅलो॒कम् । \newline
49. सु॒व॒र्गम् ॅलो॒कम् ॅलो॒कꣳ सु॑व॒र्गꣳ सु॑व॒र्गम् ॅलो॒कम् ग॑मयति गमयति लो॒कꣳ सु॑व॒र्गꣳ सु॑व॒र्गम् ॅलो॒कम् ग॑मयति । \newline
50. सु॒व॒र्गमिति॑ सुवः - गम् । \newline
51. लो॒कम् ग॑मयति गमयति लो॒कम् ॅलो॒कम् ग॑मयति॒ रक्ष॑साꣳ॒॒ रक्ष॑साम् गमयति लो॒कम् ॅलो॒कम् ग॑मयति॒ रक्ष॑साम् । \newline
52. ग॒म॒य॒ति॒ रक्ष॑साꣳ॒॒ रक्ष॑साम् गमयति गमयति॒ रक्ष॑सा॒ मनु॑पलाभा॒या नु॑पलाभाय॒ रक्ष॑साम् गमयति गमयति॒ रक्ष॑सा॒ मनु॑पलाभाय । \newline
53. रक्ष॑सा॒ मनु॑पलाभा॒या नु॑पलाभाय॒ रक्ष॑साꣳ॒॒ रक्ष॑सा॒ मनु॑पलाभा॒या ऽनु॑पलाभाय॒ रक्ष॑साꣳ॒॒ रक्ष॑सा॒ मनु॑पलाभा॒या । \newline
54. अनु॑पलाभा॒या ऽनु॑पलाभा॒या नु॑पलाभा॒या सोमꣳ॒॒ सोम॒ मा ऽनु॑पलाभा॒या नु॑पलाभा॒या सोम᳚म् । \newline
55. अनु॑पलाभा॒येत्यनु॑प - ला॒भा॒य॒ । \newline
56. आ सोमꣳ॒॒ सोम॒ मा सोम॑म् ददते ददते॒ सोम॒ मा सोम॑म् ददते । \newline
57. सोम॑म् ददते ददते॒ सोमꣳ॒॒ सोम॑म् ददत॒ आ द॑दते॒ सोमꣳ॒॒ सोम॑म् ददत॒ आ । \newline
58. द॒द॒त॒ आ द॑दते ददत॒ आ ग्राव्.ण्णो॒ ग्राव्.ण्ण॒ आ द॑दते ददत॒ आ ग्राव्.ण्णः॑ । \newline
\pagebreak
\markright{ TS 6.3.2.3  \hfill https://www.vedavms.in \hfill}

\section{ TS 6.3.2.3 }

\textbf{TS 6.3.2.3 } \newline
\textbf{Samhita Paata} \newline

आ ग्राव्ण्ण॒ आ वा॑य॒व्या᳚न्या द्रो॑णकल॒शमुत् पत्नी॒मा न॑य॒न्त्यन्वनाꣳ॑सि॒ प्र व॑र्तयन्ति॒ याव॑दे॒वास्यास्ति॒ तेन॑ स॒ह सु॑व॒र्गं ॅलो॒कमे॑ति॒ नय॑वत्य॒र्चाऽऽग्नी᳚द्ध्रे जुहोति सुव॒र्गस्य॑ लो॒कस्या॒भिनी᳚त्यै॒ ग्राव्ण्णो॑ वाय॒व्या॑नि द्रोणकल॒शमाग्नी᳚द्ध्र॒ उप॑ वासयति॒ वि ह्ये॑नं॒ तैर्गृ॒ह्णते॒ यथ् स॒होप॑वा॒सये॑-दपुवा॒येत॑ सौ॒म्यर्चा प्र पा॑दयति॒ स्वयै॒- [  ] \newline

\textbf{Pada Paata} \newline

एति॑ । ग्राव्‌ण्णः॑ । एति॑ । वा॒य॒व्या॑नि । एति॑ । द्रो॒ण॒क॒ल॒शमिति॑ द्रोण - क॒ल॒शम् । उदिति॑ । पत्नी᳚म् । एति॑ । न॒य॒न्ति॒ । अन्विति॑ । अनाꣳ॑सि । प्रेति॑ । व॒र्त॒य॒न्ति॒ । याव॑त् । ए॒व । अ॒स्य॒ । अस्ति॑ । तेन॑ । स॒ह । सु॒व॒र्गमिति॑ सुवः - गम् । लो॒कम् । ए॒ति॒ । नय॑व॒त्येति॒ नय॑ - व॒त्या॒ । ऋ॒चा । आग्नी᳚द्ध्र॒ इत्याग्नि॑ - इ॒द्ध्रे॒ । जु॒हो॒ति॒ । सु॒व॒र्गस्येति॑ सुवः - गस्य॑ । लो॒कस्य॑ । अ॒भिनी᳚त्या॒ इत्य॒भि-नी॒त्यै॒ । ग्राव्‌ण्णः॑ । वा॒य॒व्या॑नि । द्रो॒ण॒क॒ल॒शमिति॑ द्रोण-क॒ल॒शम् । आग्नी᳚द्ध्र॒ इत्याग्नि॑ - इ॒द्ध्रे॒ । उपेति॑ । वा॒स॒य॒ति॒ । वीति॑ । हि । ए॒न॒म् । तैः । गृ॒ह्णते᳚ । यत् । स॒ह । उ॒प॒वा॒सये॒दित्यु॑प - वा॒सये᳚त् । अ॒पु॒वा॒येत॑ । सौ॒म्य । ऋ॒चा । प्रेति॑ । पा॒द॒य॒ति॒ । स्वया᳚ ।  \newline


\textbf{Krama Paata} \newline

आ ग्राव्.ण्णः॑ । ग्राव्.ण्ण॒ आ । आ वा॑य॒व्या॑नि । वा॒य॒व्या᳚न्या । आ द्रो॑णकल॒शम् । द्रो॒ण॒क॒ल॒शमुत् । द्रो॒ण॒क॒ल॒शमिति॑ द्रोण - क॒ल॒शम् । उत् पत्नी᳚म् । पत्नी॒मा । आ न॑यन्ति । न॒य॒न्त्यनु॑ । अन्वनाꣳ॑सि । अनाꣳ॑सि॒ प्र । प्र व॑र्तयन्ति । व॒र्त॒य॒न्ति॒ याव॑त् । याव॑दे॒व । ए॒वास्य॑ । अ॒स्यास्ति॑ । अस्ति॒ तेन॑ । तेन॑ स॒ह । स॒ह सु॑व॒र्गम् । सु॒व॒र्गम् ॅलो॒कम् । सु॒व॒र्गमिति॑ सुवः - गम् । लो॒कमे॑ति । ए॒ति॒ नय॑वत्या । नय॑वत्य॒र्चा । नय॑व॒त्येति॒ नय॑ - व॒त्या॒ । ऋ॒चाऽऽग्नी᳚द्ध्रे । आग्नी᳚द्ध्रे जुहोति । आग्नी᳚द्ध्र॒ इत्याग्नि॑ - इ॒द्ध्रे॒ । जु॒हो॒ति॒ सु॒व॒र्गस्य॑ । सु॒व॒र्गस्य॑ लो॒कस्य॑ । सु॒व॒र्गस्येति॑ सुवः - गस्य॑ । लो॒कस्या॒भिनी᳚त्यै । अ॒भिनी᳚त्यै॒ ग्राव्.ण्णः॑ । अ॒भिनी᳚त्या॒ इत्य॒भि - नी॒त्यै॒ । ग्राव्.ण्णो॑ वाय॒व्या॑नि । वा॒य॒व्या॑नि द्रोणकल॒शम् । द्रो॒ण॒क॒ल॒शमाग्नी᳚द्ध्रे । द्रो॒ण॒क॒ल॒शमिति॑ द्रोण - क॒ल॒शम् । आग्नी᳚द्ध्र॒ उप॑ । आग्नी᳚द्ध्र॒ इत्याग्नि॑ - इ॒द्ध्रे॒ । उप॑ वासयति । वा॒स॒य॒ति॒ वि । वि हि । ह्ये॑नम् । ए॒न॒म् तैः । तैर् गृ॒ह्णते᳚ । गृ॒ह्णते॒ यत् । यथ् स॒ह । स॒होप॑वा॒सये᳚त् । उ॒प॒वा॒सये॑दपुवा॒येत॑ । उ॒प॒वा॒सये॒दित्यु॑प - वा॒सये᳚त् । अ॒पु॒वा॒येत॑ सौ॒म्या । सौ॒म्यर्चा । ऋ॒चा प्र । प्र पा॑दयति । पा॒द॒य॒ति॒ स्वया᳚ । स्वयै॒व । ए॒वैन᳚म् \newline

\textbf{Jatai Paata} \newline

1. आ ग्राव्.ण्णो॒ ग्राव्.ण्ण॒ आ ग्राव्.ण्णः॑ । \newline
2. ग्राव्.ण्ण॒ आ ग्राव्.ण्णो॒ ग्राव्.ण्ण॒ आ । \newline
3. आ वा॑य॒व्या॑नि वाय॒व्या᳚न्या वा॑य॒व्या॑नि । \newline
4. वा॒य॒व्या᳚न्या वा॑य॒व्या॑नि वाय॒व्या᳚न्या । \newline
5. आ द्रो॑णकल॒शम् द्रो॑णकल॒श मा द्रो॑णकल॒शम् । \newline
6. द्रो॒ण॒क॒ल॒श मुदुद् द्रो॑णकल॒शम् द्रो॑णकल॒श मुत् । \newline
7. द्रो॒ण॒क॒ल॒शमिति॑ द्रोण - क॒ल॒शम् । \newline
8. उत् पत्नी॒म् पत्नी॒ मुदुत् पत्नी᳚म् । \newline
9. पत्नी॒ मा पत्नी॒म् पत्नी॒ मा । \newline
10. आ न॑यन्ति नय॒न्त्या न॑यन्ति । \newline
11. न॒य॒न् त्यन्वनु॑ नयन्ति नय॒न् त्यनु॑ । \newline
12. अन्वनाꣳ॒॒ स्यनाꣳ॒॒ स्यन् वन् वनाꣳ॑सि । \newline
13. अनाꣳ॑सि॒ प्र प्राणाꣳ॒॒ स्यनाꣳ॑सि॒ प्र । \newline
14. प्र व॑र्तयन्ति वर्तयन्ति॒ प्र प्र व॑र्तयन्ति । \newline
15. व॒र्त॒य॒न्ति॒ याव॒द् याव॑द् वर्तयन्ति वर्तयन्ति॒ याव॑त् । \newline
16. याव॑ दे॒वैव याव॒द् याव॑ दे॒व । \newline
17. ए॒वास्या᳚ स्यै॒वै वास्य॑ । \newline
18. अ॒स्या स्त्यस्त्य॑ स्या॒ स्यास्ति॑ । \newline
19. अस्ति॒ तेन॒ तेना स्त्यस्ति॒ तेन॑ । \newline
20. तेन॑ स॒ह स॒ह तेन॒ तेन॑ स॒ह । \newline
21. स॒ह सु॑व॒र्गꣳ सु॑व॒र्गꣳ स॒ह स॒ह सु॑व॒र्गम् । \newline
22. सु॒व॒र्गम् ॅलो॒कम् ॅलो॒कꣳ सु॑व॒र्गꣳ सु॑व॒र्गम् ॅलो॒कम् । \newline
23. सु॒व॒र्गमिति॑ सुवः - गम् । \newline
24. लो॒क मे᳚त्येति लो॒कम् ॅलो॒क मे॑ति । \newline
25. ए॒ति॒ नय॑वत्या॒ नय॑व त्यैत्येति॒ नय॑वत्या । \newline
26. नय॑वत्य॒ र्‌च र्‌चा नय॑वत्या॒ नय॑वत्य॒ र्‌चा । \newline
27. नय॑व॒त्येति॒ नय॑ - व॒त्या॒ । \newline
28. ऋ॒चा ऽऽग्नी᳚द्ध्र॒ आग्नी᳚द्ध्र ऋ॒च र्‌चा ऽऽग्नी᳚द्ध्रे । \newline
29. आग्नी᳚द्ध्रे जुहोति जुहो॒ त्याग्नी᳚द्ध्र॒ आग्नी᳚द्ध्रे जुहोति । \newline
30. आग्नी᳚द्ध्र॒ इत्याग्नि॑ - इ॒द्ध्रे॒ । \newline
31. जु॒हो॒ति॒ सु॒व॒र्गस्य॑ सुव॒र्गस्य॑ जुहोति जुहोति सुव॒र्गस्य॑ । \newline
32. सु॒व॒र्गस्य॑ लो॒कस्य॑ लो॒कस्य॑ सुव॒र्गस्य॑ सुव॒र्गस्य॑ लो॒कस्य॑ । \newline
33. सु॒व॒र्गस्येति॑ सुवः - गस्य॑ । \newline
34. लो॒कस्या॒ भिनी᳚त्या अ॒भिनी᳚त्यै लो॒कस्य॑ लो॒कस्या॒ भिनी᳚त्यै । \newline
35. अ॒भिनी᳚त्यै॒ ग्राव्.ण्णो॒ ग्राव्.ण्णो॒ ऽभिनी᳚त्या अ॒भिनी᳚त्यै॒ ग्राव्.ण्णः॑ । \newline
36. अ॒भिनी᳚त्या॒ इत्य॒भि - नी॒त्यै॒ । \newline
37. ग्राव्.ण्णो॑ वाय॒व्या॑नि वाय॒व्या॑नि॒ ग्राव्.ण्णो॒ ग्राव्.ण्णो॑ वाय॒व्या॑नि । \newline
38. वा॒य॒व्या॑नि द्रोणकल॒शम् द्रो॑णकल॒शं ॅवा॑य॒व्या॑नि वाय॒व्या॑नि द्रोणकल॒शम् । \newline
39. द्रो॒ण॒क॒ल॒श माग्नी᳚द्ध्र॒ आग्नी᳚द्ध्रे द्रोणकल॒शम् द्रो॑णकल॒श माग्नी᳚द्ध्रे । \newline
40. द्रो॒ण॒क॒ल॒शमिति॑ द्रोण - क॒ल॒शम् । \newline
41. आग्नी᳚द्ध्र॒ उपोपा ग्नी᳚द्ध्र॒ आग्नी᳚द्ध्र॒ उप॑ । \newline
42. आग्नी᳚द्ध्र॒ इत्याग्नि॑ - इ॒द्ध्रे॒ । \newline
43. उप॑ वासयति वासय॒ त्युपोप॑ वासयति । \newline
44. वा॒स॒य॒ति॒ वि वि वा॑सयति वासयति॒ वि । \newline
45. वि हि हि वि वि हि । \newline
46. ह्ये॑न मेनꣳ॒॒ हि ह्ये॑नम् । \newline
47. ए॒न॒म् तै स्तै रे॑न मेन॒म् तैः । \newline
48. तैर् गृ॒ह्णते॑ गृ॒ह्णते॒ तै स्तैर् गृ॒ह्णते᳚ । \newline
49. गृ॒ह्णते॒ यद् यद् गृ॒ह्णते॑ गृ॒ह्णते॒ यत् । \newline
50. यथ् स॒ह स॒ह यद् यथ् स॒ह । \newline
51. स॒होप॑वा॒सये॑ दुपवा॒सये᳚थ् स॒ह स॒होप॑वा॒सये᳚त् । \newline
52. उ॒प॒वा॒सये॑ दपुवा॒येता॑ पुवा॒येतो॑ पवा॒सये॑ दुपवा॒सये॑ दपुवा॒येत॑ । \newline
53. उ॒प॒वा॒सये॒दित्यु॑प - वा॒सये᳚त् । \newline
54. अ॒पु॒वा॒येत॑ सौ॒म्या सौ॒म्या ऽपु॑वा॒येता॑ पुवा॒येत॑ सौ॒म्या । \newline
55. सौ॒म्य र्‌च र्‌चा सौ॒म्या सौ॒म्य र्‌चा । \newline
56. ऋ॒चा प्र प्रा र्‌च र्‌चा प्र । \newline
57. प्र पा॑दयति पादयति॒ प्र प्र पा॑दयति । \newline
58. पा॒द॒य॒ति॒ स्वया॒ स्वया॑ पादयति पादयति॒ स्वया᳚ । \newline
59. स्वयै॒वैव स्वया॒ स्वयै॒व । \newline

\textbf{Ghana Paata } \newline

1. आ ग्राव्.ण्णो॒ ग्राव्.ण्ण॒ आ ग्राव्.ण्ण॒ आ ग्राव्.ण्ण॒ आ ग्राव्.ण्ण॒ आ । \newline
2. ग्राव्.ण्ण॒ आ ग्राव्.ण्णो॒ ग्राव्.ण्ण॒ आ वा॑य॒व्या॑नि वाय॒व्या᳚न्या ग्राव्.ण्णो॒ ग्राव्.ण्ण॒ आ वा॑य॒व्या॑नि । \newline
3. आ वा॑य॒व्या॑नि वाय॒व्या᳚न्या वा॑य॒व्या᳚न्या वा॑य॒व्या᳚न्या वा॑य॒व्या᳚न्या । \newline
4. वा॒य॒व्या᳚न्या वा॑य॒व्या॑नि वाय॒व्या᳚न्या द्रो॑णकल॒शम् द्रो॑णकल॒श मा वा॑य॒व्या॑नि वाय॒व्या᳚न्या द्रो॑णकल॒शम् । \newline
5. आ द्रो॑णकल॒शम् द्रो॑णकल॒श मा द्रो॑णकल॒श मुदुद् द्रो॑णकल॒श मा द्रो॑णकल॒श मुत् । \newline
6. द्रो॒ण॒क॒ल॒श मुदुद् द्रो॑णकल॒शम् द्रो॑णकल॒श मुत् पत्नी॒म् पत्नी॒ मुद् द्रो॑णकल॒शम् द्रो॑णकल॒श मुत् पत्नी᳚म् । \newline
7. द्रो॒ण॒क॒ल॒शमिति॑ द्रोण - क॒ल॒शम् । \newline
8. उत् पत्नी॒म् पत्नी॒ मुदुत् पत्नी॒ मा पत्नी॒ मुदुत् पत्नी॒ मा । \newline
9. पत्नी॒ मा पत्नी॒म् पत्नी॒ मा न॑यन्ति नय॒न्त्या पत्नी॒म् पत्नी॒ मा न॑यन्ति । \newline
10. आ न॑यन्ति नय॒न्त्या न॑य॒ न्त्यन्वनु॑ नय॒न्त्या न॑य॒ न्त्यनु॑ । \newline
11. न॒य॒ न्त्यन्वनु॑ नयन्ति नय॒ न्त्य न्वनाꣳ॒॒ स्यनाꣳ॒॒ स्यनु॑ नयन्ति नय॒ न्त्य न्वनाꣳ॑सि । \newline
12. अन्वनाꣳ॒॒ स्यनाꣳ॒॒ स्यन् वन् वनाꣳ॑सि॒ प्र प्राणाꣳ॒॒ स्यन् वन् वनाꣳ॑सि॒ प्र । \newline
13. अनाꣳ॑सि॒ प्र प्राणाꣳ॒॒ स्यनाꣳ॑सि॒ प्र व॑र्तयन्ति वर्तयन्ति॒ प्राणाꣳ॒॒ स्यनाꣳ॑सि॒ प्र व॑र्तयन्ति । \newline
14. प्र व॑र्तयन्ति वर्तयन्ति॒ प्र प्र व॑र्तयन्ति॒ याव॒द् याव॑द् वर्तयन्ति॒ प्र प्र व॑र्तयन्ति॒ याव॑त् । \newline
15. व॒र्त॒य॒न्ति॒ याव॒द् याव॑द् वर्तयन्ति वर्तयन्ति॒ याव॑दे॒ वैव याव॑द् वर्तयन्ति वर्तयन्ति॒ याव॑ दे॒व । \newline
16. याव॑ दे॒वैव याव॒द् याव॑ दे॒वा स्या᳚स्यै॒व याव॒द् याव॑ दे॒वास्य॑ । \newline
17. ए॒वास्या᳚ स्यै॒वैवास्या स्त्य स्त्य॑स्यै॒वैवा स्यास्ति॑ । \newline
18. अ॒स्या स्त्यस्त्य॑ स्या॒ स्यास्ति॒ तेन॒ तेना स्त्य॑स्या॒ स्यास्ति॒ तेन॑ । \newline
19. अस्ति॒ तेन॒ तेना स्त्यस्ति॒ तेन॑ स॒ह स॒ह तेना स्त्यस्ति॒ तेन॑ स॒ह । \newline
20. तेन॑ स॒ह स॒ह तेन॒ तेन॑ स॒ह सु॑व॒र्गꣳ सु॑व॒र्गꣳ स॒ह तेन॒ तेन॑ स॒ह सु॑व॒र्गम् । \newline
21. स॒ह सु॑व॒र्गꣳ सु॑व॒र्गꣳ स॒ह स॒ह सु॑व॒र्गम् ॅलो॒कम् ॅलो॒कꣳ सु॑व॒र्गꣳ स॒ह स॒ह सु॑व॒र्गम् ॅलो॒कम् । \newline
22. सु॒व॒र्गम् ॅलो॒कम् ॅलो॒कꣳ सु॑व॒र्गꣳ सु॑व॒र्गम् ॅलो॒क मे᳚त्येति लो॒कꣳ सु॑व॒र्गꣳ सु॑व॒र्गम् ॅलो॒क मे॑ति । \newline
23. सु॒व॒र्गमिति॑ सुवः - गम् । \newline
24. लो॒क मे᳚त्येति लो॒कम् ॅलो॒क मे॑ति॒ नय॑वत्या॒ नय॑वत्यैति लो॒कम् ॅलो॒क मे॑ति॒ नय॑वत्या । \newline
25. ए॒ति॒ नय॑वत्या॒ नय॑व त्यैत्येति॒ नय॑वत्य॒ र्‌च र्‌चा नय॑व त्यैत्येति॒ नय॑वत्य॒ र्‌चा । \newline
26. नय॑वत्य॒ र्‌च र्‌चा नय॑वत्या॒ नय॑वत्य॒ र्‌चा ऽऽग्नी᳚द्ध्र॒ आग्नी᳚द्ध्र ऋ॒चा नय॑वत्या॒ नय॑वत्य॒ र्‌चा ऽऽग्नी᳚द्ध्रे । \newline
27. नय॑व॒त्येति॒ नय॑ - व॒त्या॒ । \newline
28. ऋ॒चा ऽऽग्नी᳚द्ध्र॒ आग्नी᳚द्ध्र ऋ॒च र्‌चा ऽऽग्नी᳚द्ध्रे जुहोति जुहो॒ त्याग्नी᳚द्ध्र ऋ॒च र्‌चा ऽऽग्नी᳚द्ध्रे जुहोति । \newline
29. आग्नी᳚द्ध्रे जुहोति जुहो॒ त्याग्नी᳚द्ध्र॒ आग्नी᳚द्ध्रे जुहोति सुव॒र्गस्य॑ सुव॒र्गस्य॑ जुहो॒ त्याग्नी᳚द्ध्र॒ आग्नी᳚द्ध्रे जुहोति सुव॒र्गस्य॑ । \newline
30. आग्नी᳚द्ध्र॒ इत्याग्नि॑ - इ॒द्ध्रे॒ । \newline
31. जु॒हो॒ति॒ सु॒व॒र्गस्य॑ सुव॒र्गस्य॑ जुहोति जुहोति सुव॒र्गस्य॑ लो॒कस्य॑ लो॒कस्य॑ सुव॒र्गस्य॑ जुहोति जुहोति सुव॒र्गस्य॑ लो॒कस्य॑ । \newline
32. सु॒व॒र्गस्य॑ लो॒कस्य॑ लो॒कस्य॑ सुव॒र्गस्य॑ सुव॒र्गस्य॑ लो॒कस्या॒ भिनी᳚त्या अ॒भिनी᳚त्यै लो॒कस्य॑ सुव॒र्गस्य॑ सुव॒र्गस्य॑ लो॒कस्या॒ भिनी᳚त्यै । \newline
33. सु॒व॒र्गस्येति॑ सुवः - गस्य॑ । \newline
34. लो॒कस्या॒ भिनी᳚त्या अ॒भिनी᳚त्यै लो॒कस्य॑ लो॒कस्या॒ भिनी᳚त्यै॒ ग्राव्.ण्णो॒ ग्राव्.ण्णो॒ ऽभिनी᳚त्यै लो॒कस्य॑ लो॒कस्या॒ भिनी᳚त्यै॒ ग्राव्.ण्णः॑ । \newline
35. अ॒भिनी᳚त्यै॒ ग्राव्.ण्णो॒ ग्राव्.ण्णो॒ ऽभिनी᳚त्या अ॒भिनी᳚त्यै॒ ग्राव्.ण्णो॑ वाय॒व्या॑नि वाय॒व्या॑नि॒ ग्राव्.ण्णो॒ ऽभिनी᳚त्या अ॒भिनी᳚त्यै॒ ग्राव्.ण्णो॑ वाय॒व्या॑नि । \newline
36. अ॒भिनी᳚त्या॒ इत्य॒भि - नी॒त्यै॒ । \newline
37. ग्राव्.ण्णो॑ वाय॒व्या॑नि वाय॒व्या॑नि॒ ग्राव्.ण्णो॒ ग्राव्.ण्णो॑ वाय॒व्या॑नि द्रोणकल॒शम् द्रो॑णकल॒शं ॅवा॑य॒व्या॑नि॒ ग्राव्.ण्णो॒ ग्राव्.ण्णो॑ वाय॒व्या॑नि द्रोणकल॒शम् । \newline
38. वा॒य॒व्या॑नि द्रोणकल॒शम् द्रो॑णकल॒शं ॅवा॑य॒व्या॑नि वाय॒व्या॑नि द्रोणकल॒श माग्नी᳚द्ध्र॒ आग्नी᳚द्ध्रे द्रोणकल॒शं ॅवा॑य॒व्या॑नि वाय॒व्या॑नि द्रोणकल॒श माग्नी᳚द्ध्रे । \newline
39. द्रो॒ण॒क॒ल॒श माग्नी᳚द्ध्र॒ आग्नी᳚द्ध्रे द्रोणकल॒शम् द्रो॑णकल॒श माग्नी᳚द्ध्र॒ उपोपाग्नी᳚द्ध्रे द्रोणकल॒शम् द्रो॑णकल॒श माग्नी᳚द्ध्र॒ उप॑ । \newline
40. द्रो॒ण॒क॒ल॒शमिति॑ द्रोण - क॒ल॒शम् । \newline
41. आग्नी᳚द्ध्र॒ उपोपा ग्नी᳚द्ध्र॒ आग्नी᳚द्ध्र॒ उप॑ वासयति वासय॒ त्युपाग्नी᳚द्ध्र॒ आग्नी᳚द्ध्र॒ उप॑ वासयति । \newline
42. आग्नी᳚द्ध्र॒ इत्याग्नि॑ - इ॒द्ध्रे॒ । \newline
43. उप॑ वासयति वासय॒ त्युपोप॑ वासयति॒ वि वि वा॑सय॒ त्युपोप॑ वासयति॒ वि । \newline
44. वा॒स॒य॒ति॒ वि वि वा॑सयति वासयति॒ वि हि हि वि वा॑सयति वासयति॒ वि हि । \newline
45. वि हि हि वि वि ह्ये॑न मेनꣳ॒॒ हि वि वि ह्ये॑नम् । \newline
46. ह्ये॑न मेनꣳ॒॒ हि ह्ये॑न॒म् तै स्तै रे॑नꣳ॒॒ हि ह्ये॑न॒म् तैः । \newline
47. ए॒न॒म् तै स्तै रे॑न मेन॒म् तैर् गृ॒ह्णते॑ गृ॒ह्णते॒ तै रे॑न मेन॒म् तैर् गृ॒ह्णते᳚ । \newline
48. तैर् गृ॒ह्णते॑ गृ॒ह्णते॒ तै स्तैर् गृ॒ह्णते॒ यद् यद् गृ॒ह्णते॒ तै स्तैर् गृ॒ह्णते॒ यत् । \newline
49. गृ॒ह्णते॒ यद् यद् गृ॒ह्णते॑ गृ॒ह्णते॒ यथ् स॒ह स॒ह यद् गृ॒ह्णते॑ गृ॒ह्णते॒ यथ् स॒ह । \newline
50. यथ् स॒ह स॒ह यद् यथ् स॒हो प॑वा॒सये॑ दुपवा॒सये᳚थ् स॒ह यद् यथ् स॒हो प॑वा॒सये᳚त् । \newline
51. स॒होप॑वा॒सये॑ दुपवा॒सये᳚थ् स॒ह स॒होप॑वा॒सये॑ दपुवा॒येता॑ पुवा॒येतो॑ पवा॒सये᳚थ् स॒ह 
स॒होप॑वा॒सये॑ दपुवा॒येत॑ । \newline
52. उ॒प॒वा॒सये॑ दपुवा॒येता॑ पुवा॒येतो॑ पवा॒सये॑ दुपवा॒सये॑ दपुवा॒येत॑ सौ॒म्या सौ॒म्या ऽपु॑वा॒येतो॑ पवा॒सये॑ दुपवा॒सये॑ दपुवा॒येत॑ सौ॒म्या । \newline
53. उ॒प॒वा॒सये॒दित्यु॑प - वा॒सये᳚त् । \newline
54. अ॒पु॒वा॒येत॑ सौ॒म्या सौ॒म्या ऽपु॑वा॒येता॑ पुवा॒येत॑ सौ॒म्य र्‌च र्‌चा सौ॒म्या ऽपु॑वा॒येता॑ पुवा॒येत॑ सौ॒म्य र्‌चा । \newline
55. सौ॒म्य र्‌च र्‌चा सौ॒म्या सौ॒म्य र्‌चा प्र प्रा र्‌चा सौ॒म्या सौ॒म्य र्‌चा प्र । \newline
56. ऋ॒चा प्र प्रा र्‌च र्‌चा प्र पा॑दयति पादयति॒ प्रा र्‌च र्‌चा प्र पा॑दयति । \newline
57. प्र पा॑दयति पादयति॒ प्र प्र पा॑दयति॒ स्वया॒ स्वया॑ पादयति॒ प्र प्र पा॑दयति॒ स्वया᳚ । \newline
58. पा॒द॒य॒ति॒ स्वया॒ स्वया॑ पादयति पादयति॒ स्वयै॒वैव स्वया॑ पादयति पादयति॒ स्वयै॒व । \newline
59. स्वयै॒वैव स्वया॒ स्वयै॒वैन॑ मेन मे॒व स्वया॒ स्वयै॒वैन᳚म् । \newline
\pagebreak
\markright{ TS 6.3.2.4  \hfill https://www.vedavms.in \hfill}

\section{ TS 6.3.2.4 }

\textbf{TS 6.3.2.4 } \newline
\textbf{Samhita Paata} \newline

-वैनं॑ दे॒वत॑या॒ प्र पा॑दय॒त्यदि॑त्याः॒ सदो॒ऽस्यदि॑त्याः॒ सद॒ आ सी॒देत्या॑ह यथाय॒जुरे॒वैतद्-यज॑मानो॒ वा ए॒तस्य॑ पु॒रा गो॒प्ता भ॑वत्ये॒ष वो॑ देव सवितः॒ सोम॒ इत्या॑ह सवि॒तृप्र॑सूत ए॒वैनं॑ दे॒वता᳚भ्यः॒ सं प्रय॑च्छत्ये॒तत् त्वꣳ सो॑म दे॒वो दे॒वानुपा॑गा॒ इत्या॑ह दे॒वो ह्ये॑ष सन्- [  ] \newline

\textbf{Pada Paata} \newline

ए॒व । ए॒न॒म् । दे॒वत॑या । प्रेति॑ । पा॒द॒य॒ति॒ । अदि॑त्याः । सदः॑ । अ॒सि॒ । अदि॑त्याः । सदः॑ । एति॑ । सी॒द॒ । इति॑ । आ॒ह॒ । य॒था॒य॒जुरिति॑ यथा-य॒जुः । ए॒व । ए॒तत् । यज॑मानः । वै । ए॒तस्य॑ । पु॒रा । गो॒प्ता । भ॒व॒ति॒ । ए॒षः । वः॒ । दे॒व॒ । स॒वि॒तः॒ । सोमः॑ । इति॑ । आ॒ह॒ । स॒वि॒तृप्र॑सूत॒ इति॑ सवि॒तृ-प्र॒सू॒तः॒ । ए॒व । ए॒न॒म् । दे॒वता᳚भ्यः । सम् । प्रेति॑ । य॒च्छ॒ति॒ । ए॒तत् । त्वम् । सो॒म॒ । दे॒वः । दे॒वान् । उपेति॑ । अ॒गाः॒ । इति॑ । आ॒ह॒ । दे॒वः । हि । ए॒षः । सन्न् ।  \newline


\textbf{Krama Paata} \newline

ए॒न॒म् दे॒वत॑या । दे॒वत॑या॒ प्र । प्र पा॑दयति । पा॒द॒य॒त्यदि॑त्याः । अदि॑त्याः॒ सदः॑ । सदो॑ऽसि । अ॒स्यदि॑त्याः । अदि॑त्याः॒ सदः॑ । सद॒ आ । आ सी॑द । सी॒देति॑ । इत्या॑ह । आ॒ह॒ य॒था॒य॒जुः । य॒था॒य॒जुरे॒व । य॒था॒य॒जुरिति॑ यथा - य॒जुः । ए॒वैतत् । ए॒तद् यज॑मानः । यज॑मानो॒ वै । वा ए॒तस्य॑ । ए॒तस्य॑ पु॒रा । पु॒रा गो॒प्ता । गो॒प्ता भ॑वति । भ॒व॒त्ये॒षः । ए॒ष वः॑ । वो॒ दे॒व॒ । दे॒व॒ स॒वि॒तः॒ । स॒वि॒तः॒ सोमः॑ । सोम॒ इति॑ । इत्या॑ह । आ॒ह॒ स॒वि॒तृप्र॑सूतः । स॒वि॒तृप्र॑सूत ए॒व । स॒वि॒तृप्र॑सूत॒ इति॑ सवि॒तृ - प्र॒सू॒तः॒ । ए॒वैन᳚म् । ए॒न॒म् दे॒वता᳚भ्यः । दे॒वता᳚भ्यः॒ सम् । सम् प्र । प्र य॑च्छति । य॒च्छ॒त्ये॒तत् । ए॒तत् त्वम् । त्वꣳ सो॑म । सो॒म॒ दे॒वः । दे॒वो दे॒वान् । दे॒वानुप॑ । उपा॑गाः । अ॒गा॒ इति॑ । इत्या॑ह । आ॒ह॒ दे॒वः । दे॒वो हि । ह्ये॑षः । ए॒ष सन्न् । सन् दे॒वान् \newline

\textbf{Jatai Paata} \newline

1. ए॒वैन॑ मेन मे॒वै वैन᳚म् । \newline
2. ए॒न॒म् दे॒वत॑या दे॒वत॑यैन मेनम् दे॒वत॑या । \newline
3. दे॒वत॑या॒ प्र प्र दे॒वत॑या दे॒वत॑या॒ प्र । \newline
4. प्र पा॑दयति पादयति॒ प्र प्र पा॑दयति । \newline
5. पा॒द॒य॒ त्यदि॑त्या॒ अदि॑त्याः पादयति पादय॒ त्यदि॑त्याः । \newline
6. अदि॑त्याः॒ सदः॒ सदो ऽदि॑त्या॒ अदि॑त्याः॒ सदः॑ । \newline
7. सदो᳚ ऽस्यसि॒ सदः॒ सदो॑ ऽसि । \newline
8. अ॒स्य दि॑त्या॒ अदि॑त्या अस्य॒स्य दि॑त्याः । \newline
9. अदि॑त्याः॒ सदः॒ सदो ऽदि॑त्या॒ अदि॑त्याः॒ सदः॑ । \newline
10. सद॒ आ सदः॒ सद॒ आ । \newline
11. आ सी॑द सी॒दा सी॑द । \newline
12. सी॒दे तीति॑ सीद सी॒देति॑ । \newline
13. इत्या॑हा॒हे तीत्या॑ह । \newline
14. आ॒ह॒ य॒था॒य॒जुर् य॑थाय॒जु रा॑हाह यथाय॒जुः । \newline
15. य॒था॒य॒जु रे॒वैव य॑थाय॒जुर् य॑थाय॒जु रे॒व । \newline
16. य॒था॒य॒जुरिति॑ यथा - य॒जुः । \newline
17. ए॒वैत दे॒त दे॒वै वैतत् । \newline
18. ए॒तद् यज॑मानो॒ यज॑मान ए॒त दे॒तद् यज॑मानः । \newline
19. यज॑मानो॒ वै वै यज॑मानो॒ यज॑मानो॒ वै । \newline
20. वा ए॒त स्यै॒तस्य॒ वै वा ए॒तस्य॑ । \newline
21. ए॒तस्य॑ पु॒रा पु॒रैत स्यै॒तस्य॑ पु॒रा । \newline
22. पु॒रा गो॒प्ता गो॒प्ता पु॒रा पु॒रा गो॒प्ता । \newline
23. गो॒प्ता भ॑वति भवति गो॒प्ता गो॒प्ता भ॑वति । \newline
24. भ॒व॒ त्ये॒ष ए॒ष भ॑वति भव त्ये॒षः । \newline
25. ए॒ष वो॑ व ए॒ष ए॒ष वः॑ । \newline
26. वो॒ दे॒व॒ दे॒व॒ वो॒ वो॒ दे॒व॒ । \newline
27. दे॒व॒ स॒वि॒तः॒ स॒वि॒त॒र् दे॒व॒ दे॒व॒ स॒वि॒तः॒ । \newline
28. स॒वि॒तः॒ सोमः॒ सोमः॑ सवितः सवितः॒ सोमः॑ । \newline
29. सोम॒ इतीति॒ सोमः॒ सोम॒ इति॑ । \newline
30. इत्या॑हा॒हे तीत्या॑ह । \newline
31. आ॒ह॒ स॒वि॒तृप्र॑सूतः सवि॒तृप्र॑सूत आहाह सवि॒तृप्र॑सूतः । \newline
32. स॒वि॒तृप्र॑सूत ए॒वैव स॑वि॒तृप्र॑सूतः सवि॒तृप्र॑सूत ए॒व । \newline
33. स॒वि॒तृप्र॑सूत॒ इति॑ सवि॒तृ - प्र॒सू॒तः॒ । \newline
34. ए॒वैन॑ मेन मे॒वै वैन᳚म् । \newline
35. ए॒न॒म् दे॒वता᳚भ्यो दे॒वता᳚भ्य एन मेनम् दे॒वता᳚भ्यः । \newline
36. दे॒वता᳚भ्यः॒ सꣳ सम् दे॒वता᳚भ्यो दे॒वता᳚भ्यः॒ सम् । \newline
37. सम् प्र प्र सꣳ सम् प्र । \newline
38. प्र य॑च्छति यच्छति॒ प्र प्र य॑च्छति । \newline
39. य॒च्छ॒ त्ये॒त दे॒तद् य॑च्छति यच्छ त्ये॒तत् । \newline
40. ए॒तत् त्वम् त्व मे॒त दे॒तत् त्वम् । \newline
41. त्वꣳ सो॑म सोम॒ त्वम् त्वꣳ सो॑म । \newline
42. सो॒म॒ दे॒वो दे॒वः सो॑म सोम दे॒वः । \newline
43. दे॒वो दे॒वान् दे॒वान् दे॒वो दे॒वो दे॒वान् । \newline
44. दे॒वानुपोप॑ दे॒वान् दे॒वानुप॑ । \newline
45. उपा॑गा अगा॒ उपोपा॑गाः । \newline
46. अ॒गा॒ इती त्य॑गा अगा॒ इति॑ । \newline
47. इत्या॑हा॒हे तीत्या॑ह । \newline
48. आ॒ह॒ दे॒वो दे॒व आ॑हाह दे॒वः । \newline
49. दे॒वो हि हि दे॒वो दे॒वो हि । \newline
50. ह्ये॑ष ए॒ष हि ह्ये॑षः । \newline
51. ए॒ष सन् थ्सन् ने॒ष ए॒ष सन्न् । \newline
52. सन् दे॒वान् दे॒वान् थ्सन् थ्सन् दे॒वान् । \newline

\textbf{Ghana Paata } \newline

1. ए॒वैन॑ मेन मे॒वैवैन॑म् दे॒वत॑या दे॒वत॑ यैन मे॒वैवैन॑म् दे॒वत॑या । \newline
2. ए॒न॒म् दे॒वत॑या दे॒वत॑ यैन मेनम् दे॒वत॑या॒ प्र प्र दे॒वत॑ यैन मेनम् दे॒वत॑या॒ प्र । \newline
3. दे॒वत॑या॒ प्र प्र दे॒वत॑या दे॒वत॑या॒ प्र पा॑दयति पादयति॒ प्र दे॒वत॑या दे॒वत॑या॒ प्र पा॑दयति । \newline
4. प्र पा॑दयति पादयति॒ प्र प्र पा॑दय॒ त्यदि॑त्या॒ अदि॑त्याः पादयति॒ प्र प्र पा॑दय॒ त्यदि॑त्याः । \newline
5. पा॒द॒य॒ त्यदि॑त्या॒ अदि॑त्याः पादयति पादय॒ त्यदि॑त्याः॒ सदः॒ सदो ऽदि॑त्याः पादयति पादय॒ त्यदि॑त्याः॒ सदः॑ । \newline
6. अदि॑त्याः॒ सदः॒ सदो ऽदि॑त्या॒ अदि॑त्याः॒ सदो᳚ ऽस्यसि॒ सदो ऽदि॑त्या॒ अदि॑त्याः॒ सदो॑ ऽसि । \newline
7. सदो᳚ ऽस्यसि॒ सदः॒ सदो॒ ऽस्य दि॑त्या॒ अदि॑त्या असि॒ सदः॒ सदो॒ ऽस्य दि॑त्याः । \newline
8. अ॒स्य दि॑त्या॒ अदि॑त्या अस्य॒स्य दि॑त्याः॒ सदः॒ सदो ऽदि॑त्या अस्य॒स्य दि॑त्याः॒ सदः॑ । \newline
9. अदि॑त्याः॒ सदः॒ सदो ऽदि॑त्या॒ अदि॑त्याः॒ सद॒ आ सदो ऽदि॑त्या॒ अदि॑त्याः॒ सद॒ आ । \newline
10. सद॒ आ सदः॒ सद॒ आ सी॑द सी॒दा सदः॒ सद॒ आ सी॑द । \newline
11. आ सी॑द सी॒दा सी॒दे तीति॑ सी॒दा सी॒देति॑ । \newline
12. सी॒दे तीति॑ सीद सी॒दे त्या॑हा॒ हेति॑ सीद सी॒दे त्या॑ह । \newline
13. इत्या॑हा॒हे तीत्या॑ह यथाय॒जुर् य॑थाय॒जु रा॒हे तीत्या॑ह यथाय॒जुः । \newline
14. आ॒ह॒ य॒था॒य॒जुर् य॑थाय॒जु रा॑हाह यथाय॒जु रे॒वैव य॑थाय॒जु रा॑हाह यथाय॒जु रे॒व । \newline
15. य॒था॒य॒जु रे॒वैव य॑थाय॒जुर् य॑थाय॒जु रे॒वैत दे॒त दे॒व य॑थाय॒जुर् य॑थाय॒जु रे॒वैतत् । \newline
16. य॒था॒य॒जुरिति॑ यथा - य॒जुः । \newline
17. ए॒वैत दे॒त दे॒वै वैतद् यज॑मानो॒ यज॑मान ए॒त दे॒वै वैतद् यज॑मानः । \newline
18. ए॒तद् यज॑मानो॒ यज॑मान ए॒त दे॒तद् यज॑मानो॒ वै वै यज॑मान ए॒त दे॒तद् यज॑मानो॒ वै । \newline
19. यज॑मानो॒ वै वै यज॑मानो॒ यज॑मानो॒ वा ए॒त स्यै॒तस्य॒ वै यज॑मानो॒ यज॑मानो॒ वा ए॒तस्य॑ । \newline
20. वा ए॒त स्यै॒तस्य॒ वै वा ए॒तस्य॑ पु॒रा पु॒रैतस्य॒ वै वा ए॒तस्य॑ पु॒रा । \newline
21. ए॒तस्य॑ पु॒रा पु॒रै तस्यै॒तस्य॑ पु॒रा गो॒प्ता गो॒प्ता पु॒रै तस्यै॒तस्य॑ पु॒रा गो॒प्ता । \newline
22. पु॒रा गो॒प्ता गो॒प्ता पु॒रा पु॒रा गो॒प्ता भ॑वति भवति गो॒प्ता पु॒रा पु॒रा गो॒प्ता भ॑वति । \newline
23. गो॒प्ता भ॑वति भवति गो॒प्ता गो॒प्ता भ॑व त्ये॒ष ए॒ष भ॑वति गो॒प्ता गो॒प्ता भ॑व त्ये॒षः । \newline
24. भ॒व॒ त्ये॒ष ए॒ष भ॑वति भव त्ये॒ष वो॑ व ए॒ष भ॑वति भव त्ये॒ष वः॑ । \newline
25. ए॒ष वो॑ व ए॒ष ए॒ष वो॑ देव देव व ए॒ष ए॒ष वो॑ देव । \newline
26. वो॒ दे॒व॒ दे॒व॒ वो॒ वो॒ दे॒व॒ स॒वि॒तः॒ स॒वि॒त॒र् दे॒व॒ वो॒ वो॒ दे॒व॒ स॒वि॒तः॒ । \newline
27. दे॒व॒ स॒वि॒तः॒ स॒वि॒त॒र् दे॒व॒ दे॒व॒ स॒वि॒तः॒ सोमः॒ सोमः॑ सवितर् देव देव सवितः॒ सोमः॑ । \newline
28. स॒वि॒तः॒ सोमः॒ सोमः॑ सवितः सवितः॒ सोम॒ इतीति॒ सोमः॑ सवितः सवितः॒ सोम॒ इति॑ । \newline
29. सोम॒ इतीति॒ सोमः॒ सोम॒ इत्या॑हा॒ हेति॒ सोमः॒ सोम॒ इत्या॑ह । \newline
30. इत्या॑हा॒हे तीत्या॑ह सवि॒तृप्र॑सूतः सवि॒तृप्र॑सूत आ॒हेती त्या॑ह सवि॒तृप्र॑सूतः । \newline
31. आ॒ह॒ स॒वि॒तृप्र॑सूतः सवि॒तृप्र॑सूत आहाह सवि॒तृप्र॑सूत ए॒वैव स॑वि॒तृप्र॑सूत आहाह सवि॒तृप्र॑सूत ए॒व । \newline
32. स॒वि॒तृप्र॑सूत ए॒वैव स॑वि॒तृप्र॑सूतः सवि॒तृप्र॑सूत ए॒वैन॑ मेन मे॒व स॑वि॒तृप्र॑सूतः सवि॒तृप्र॑सूत ए॒वैन᳚म् । \newline
33. स॒वि॒तृप्र॑सूत॒ इति॑ सवि॒तृ - प्र॒सू॒तः॒ । \newline
34. ए॒वैन॑ मेन मे॒वै वैन॑म् दे॒वता᳚भ्यो दे॒वता᳚भ्य एन मे॒वै वैन॑म् दे॒वता᳚भ्यः । \newline
35. ए॒न॒म् दे॒वता᳚भ्यो दे॒वता᳚भ्य एन मेनम् दे॒वता᳚भ्यः॒ सꣳ सम् दे॒वता᳚भ्य एन मेनम् दे॒वता᳚भ्यः॒ सम् । \newline
36. दे॒वता᳚भ्यः॒ सꣳ सम् दे॒वता᳚भ्यो दे॒वता᳚भ्यः॒ सम् प्र प्र सम् दे॒वता᳚भ्यो दे॒वता᳚भ्यः॒ सम् प्र । \newline
37. सम् प्र प्र सꣳ सम् प्र य॑च्छति यच्छति॒ प्र सꣳ सम् प्र य॑च्छति । \newline
38. प्र य॑च्छति यच्छति॒ प्र प्र य॑च्छ त्ये॒त दे॒तद् य॑च्छति॒ प्र प्र य॑च्छ त्ये॒तत् । \newline
39. य॒च्छ॒ त्ये॒त दे॒तद् य॑च्छति यच्छ त्ये॒तत् त्वम् त्व मे॒तद् य॑च्छति यच्छ त्ये॒तत् त्वम् । \newline
40. ए॒तत् त्वम् त्व मे॒त दे॒तत् त्वꣳ सो॑म सोम॒ त्व मे॒त दे॒तत् त्वꣳ सो॑म । \newline
41. त्वꣳ सो॑म सोम॒ त्वम् त्वꣳ सो॑म दे॒वो दे॒वः सो॑म॒ त्वम् त्वꣳ सो॑म दे॒वः । \newline
42. सो॒म॒ दे॒वो दे॒वः सो॑म सोम दे॒वो दे॒वान् दे॒वान् दे॒वः सो॑म सोम दे॒वो दे॒वान् । \newline
43. दे॒वो दे॒वान् दे॒वान् दे॒वो दे॒वो दे॒वा नुपोप॑ दे॒वान् दे॒वो दे॒वो दे॒वा नुप॑ । \newline
44. दे॒वा नुपोप॑ दे॒वान् दे॒वा नुपा॑गा अगा॒ उप॑ दे॒वान् दे॒वा नुपा॑गाः । \newline
45. उपा॑गा अगा॒ उपोपा॑गा॒ इतीत्य॑गा॒ उपोपा॑गा॒ इति॑ । \newline
46. अ॒गा॒ इतीत्य॑गा अगा॒ इत्या॑हा॒हे त्य॑गा अगा॒ इत्या॑ह । \newline
47. इत्या॑हा॒हे तीत्या॑ह दे॒वो दे॒व आ॒हे तीत्या॑ह दे॒वः । \newline
48. आ॒ह॒ दे॒वो दे॒व आ॑हाह दे॒वो हि हि दे॒व आ॑हाह दे॒वो हि । \newline
49. दे॒वो हि हि दे॒वो दे॒वो ह्ये॑ष ए॒ष हि दे॒वो दे॒वो ह्ये॑षः । \newline
50. ह्ये॑ष ए॒ष हि ह्ये॑ष सन् थ्सन् ने॒ष हि ह्ये॑ष सन्न् । \newline
51. ए॒ष सन् थ्सन् ने॒ष ए॒ष सन् दे॒वान् दे॒वान् थ्सन् ने॒ष ए॒ष सन् दे॒वान् । \newline
52. सन् दे॒वान् दे॒वान् थ्सन् थ्सन् दे॒वा नु॒पै त्यु॒पैति॑ दे॒वान् थ्सन् थ्सन् दे॒वा नु॒पैति॑ । \newline
\pagebreak
\markright{ TS 6.3.2.5  \hfill https://www.vedavms.in \hfill}

\section{ TS 6.3.2.5 }

\textbf{TS 6.3.2.5 } \newline
\textbf{Samhita Paata} \newline

दे॒वानु॒पैती॒दम॒हं म॑नु॒ष्यो॑ मनु॒ष्या॑नित्या॑ह मनु॒ष्यो᳚(1॒) ह्ये॑ष सन् म॑नु॒ष्या॑नु॒पैति॒ यदे॒तद् यजु॒र्न ब्रू॒यादप्र॑जा अप॒शुर्यज॑मानः स्याथ् स॒ह प्र॒जया॑ सह रा॒यस्पोषे॒णेत्या॑ह प्र॒जयै॒व प॒शुभिः॑ स॒हेमं ॅलो॒कमु॒पाव॑र्तते॒ नमो॑ दे॒वेभ्य॒ इत्या॑ह नमस्का॒रो हि दे॒वानाꣳ॑ स्व॒धा पि॒तृभ्य॒ इत्या॑ह स्वधाका॒रो हि- [  ] \newline

\textbf{Pada Paata} \newline

दे॒वान् । उ॒पैतीत्यु॑प - एति॑ । इ॒दम् । अ॒हम् । म॒नु॒ष्यः॑ । म॒नु॒ष्यान्॑ । इति॑ । आ॒ह॒ । म॒नु॒ष्यः॑ । हि । ए॒षः । सन्न् । म॒नु॒ष्यान्॑ । उ॒पैतीत्यु॑प - एति॑ । यत् । ए॒तत् । यजुः॑ । न । ब्रू॒यात् । अप्र॑जा॒ इत्यप्र॑ - जाः॒ । अ॒प॒शुः । यज॑मानः । स्या॒त् । स॒ह । प्र॒जयेति॑ प्र - जया᳚ । स॒ह । रा॒यः । पोषे॑ण । इति॑ । आ॒ह॒ । प्र॒जयेति॑ प्र - जया᳚ । ए॒व । प॒शुभि॒रिति॑ प॒शु-भिः॒ । स॒ह । इ॒मम् । लो॒कम् । उ॒पाव॑र्तत॒ इत्यु॑प - आव॑र्तते । नमः॑ । दे॒वेभ्यः॑ । इति॑ । आ॒ह॒ । न॒म॒स्का॒र इति॑ नमः - का॒रः । हि । दे॒वाना᳚म् । स्व॒धेति॑ स्व - धा । पि॒तृभ्य॒ इति॑ पि॒तृ - भ्यः॒ । इति॑ । आ॒ह॒ । स्व॒धा॒का॒र इति॑ स्वधा - का॒रः । हि ।  \newline


\textbf{Krama Paata} \newline

दे॒वानु॒पैति॑ । उ॒पैती॒दम् । उ॒पैतीत्यु॑प - एति॑ । इ॒दम॒हम् । अ॒हम् म॑नु॒ष्यः॑ । म॒नु॒ष्यो॑ मनु॒ष्यान्॑ । म॒नु॒ष्या॑निति॑ । इत्या॑ह । आ॒ह॒ म॒नु॒ष्यः॑ । म॒नु॒ष्यो॑ हि । ह्ये॑षः । ए॒ष सन्न् । सन् म॑नु॒ष्यान्॑ । म॒नु॒ष्या॑नु॒पैति॑ । उ॒पैति॒ यत् । उ॒पैतीत्यु॑प - एति॑ । यदे॒तत् । ए॒तद् यजुः॑ । यजु॒र् न । न ब्रू॒यात् । ब्रू॒यादप्र॑जाः । अप्र॑जा अप॒शुः । अप्र॑जा॒ इत्यप्र॑ - जाः॒ । अ॒प॒शुर् यज॑मानः । यज॑मानः स्यात् । स्या॒थ् स॒ह । स॒ह प्र॒जया᳚ । प्र॒जया॑ स॒ह । प्र॒जयेति॑ प्र - जया᳚ । स॒ह रा॒यः । रा॒यस्पोषे॑ण । पोषे॒णेति॑ । इत्या॑ह । आ॒ह॒ प्र॒जया᳚ । प्र॒जयै॒व । प्र॒जयेति॑ प्र - जया᳚ । ए॒व प॒शुभिः॑ । प॒शुभिः॑ स॒ह । प॒शुभि॒रिति॑ प॒शु - भिः॒ । स॒हेमम् । इ॒मम् ॅलो॒कम् । लो॒कमु॒पाव॑र्तते । उ॒पाव॑र्तते॒ नमः॑ । उ॒पाव॑र्तत॒ इत्यु॑प - आव॑र्तते । नमो॑ दे॒वेभ्यः॑ । दे॒वेभ्य॒ इति॑ । इत्या॑ह । आ॒ह॒ न॒म॒स्का॒रः । न॒म॒स्का॒रो हि । न॒म॒स्का॒र इति॑ नमः - का॒रः । हि दे॒वाना᳚म् । दे॒वानाꣳ॑ स्व॒धा । स्व॒धा पि॒तृभ्यः॑ । स्व॒धेति॑ स्व - धा । पि॒तृभ्य॒ इति॑ । पि॒तृभ्य॒ इति॑ पि॒तृ - भ्यः॒ । इत्या॑ह । आ॒ह॒ स्व॒धा॒का॒रः । स्व॒धा॒का॒रो हि । स्व॒धा॒का॒र इति॑ स्वधा - का॒रः । हि पि॑तृ॒णाम् \newline

\textbf{Jatai Paata} \newline

1. दे॒वा नु॒पै त्यु॒पैति॑ दे॒वान् दे॒वानु॒पैति॑ । \newline
2. उ॒पैती॒द मि॒द मु॒पै त्यु॒पैती॒दम् । \newline
3. उ॒पैतीत्यु॑प - एति॑ । \newline
4. इ॒द म॒ह म॒ह मि॒द मि॒द म॒हम् । \newline
5. अ॒हम् म॑नु॒ष्यो॑ मनु॒ष्यो॑ ऽह म॒हम् म॑नु॒ष्यः॑ । \newline
6. म॒नु॒ष्यो॑ मनु॒ष्या᳚न् मनु॒ष्या᳚न् मनु॒ष्यो॑ मनु॒ष्यो॑ मनु॒ष्यान्॑ । \newline
7. म॒नु॒ष्या॑नि तीति॑ मनु॒ष्या᳚न् मनु॒ष्या॑निति॑ । \newline
8. इत्या॑हा॒हे तीत्या॑ह । \newline
9. आ॒ह॒ म॒नु॒ष्यो॑ मनु॒ष्य॑ आहाह मनु॒ष्यः॑ । \newline
10. म॒नु॒ष्यो॑ हि हि म॑नु॒ष्यो॑ मनु॒ष्यो॑ हि । \newline
11. ह्ये॑ष ए॒ष हि ह्ये॑षः । \newline
12. ए॒ष सन् थ्सन् ने॒ष ए॒ष सन्न् । \newline
13. सन् म॑नु॒ष्या᳚न् मनु॒ष्या᳚न् थ्सन् थ्सन् म॑नु॒ष्यान्॑ । \newline
14. म॒नु॒ष्या॑ नु॒पै त्यु॒पैति॑ मनु॒ष्या᳚न् मनु॒ष्या॑ नु॒पैति॑ । \newline
15. उ॒पैति॒ यद् यदु॒पै त्यु॒पैति॒ यत् । \newline
16. उ॒पैतीत्यु॑प - एति॑ । \newline
17. यदे॒त दे॒तद् यद् यदे॒तत् । \newline
18. ए॒तद् यजु॒र् यजु॑ रे॒त दे॒तद् यजुः॑ । \newline
19. यजु॒र् न न यजु॒र् यजु॒र् न । \newline
20. न ब्रू॒याद् ब्रू॒यान् न न ब्रू॒यात् । \newline
21. ब्रू॒या दप्र॑जा॒ अप्र॑जा ब्रू॒याद् ब्रू॒या दप्र॑जाः । \newline
22. अप्र॑जा अप॒शु र॑प॒शु रप्र॑जा॒ अप्र॑जा अप॒शुः । \newline
23. अप्र॑जा॒ इत्यप्र॑ - जाः॒ । \newline
24. अ॒प॒शुर् यज॑मानो॒ यज॑मानो ऽप॒शु र॑प॒शुर् यज॑मानः । \newline
25. यज॑मानः स्याथ् स्या॒द् यज॑मानो॒ यज॑मानः स्यात् । \newline
26. स्या॒थ् स॒ह स॒ह स्या᳚थ् स्याथ् स॒ह । \newline
27. स॒ह प्र॒जया᳚ प्र॒जया॑ स॒ह स॒ह प्र॒जया᳚ । \newline
28. प्र॒जया॑ स॒ह स॒ह प्र॒जया᳚ प्र॒जया॑ स॒ह । \newline
29. प्र॒जयेति॑ प्र - जया᳚ । \newline
30. स॒ह रा॒यो रा॒यः स॒ह स॒ह रा॒यः । \newline
31. रा॒य स्पोषे॑ण॒ पोषे॑ण रा॒यो रा॒य स्पोषे॑ण । \newline
32. पोषे॒णे तीति॒ पोषे॑ण॒ पोषे॒णेति॑ । \newline
33. इत्या॑हा॒हे तीत्या॑ह । \newline
34. आ॒ह॒ प्र॒जया᳚ प्र॒जया॑ ऽऽहाह प्र॒जया᳚ । \newline
35. प्र॒जयै॒वैव प्र॒जया᳚ प्र॒जयै॒व । \newline
36. प्र॒जयेति॑ प्र - जया᳚ । \newline
37. ए॒व प॒शुभिः॑ प॒शुभि॑ रे॒वैव प॒शुभिः॑ । \newline
38. प॒शुभिः॑ स॒ह स॒ह प॒शुभिः॑ प॒शुभिः॑ स॒ह । \newline
39. प॒शुभि॒रिति॑ प॒शु - भिः॒ । \newline
40. स॒हेम मि॒मꣳ स॒ह स॒हेमम् । \newline
41. इ॒मम् ॅलो॒कम् ॅलो॒क मि॒म मि॒मम् ॅलो॒कम् । \newline
42. लो॒क मु॒पाव॑र्तत उ॒पाव॑र्तते लो॒कम् ॅलो॒क मु॒पाव॑र्तते । \newline
43. उ॒पाव॑र्तते॒ नमो॒ नम॑ उ॒पाव॑र्तत उ॒पाव॑र्तते॒ नमः॑ । \newline
44. उ॒पाव॑र्तत॒ इत्यु॑प - आव॑र्तते । \newline
45. नमो॑ दे॒वेभ्यो॑ दे॒वेभ्यो॒ नमो॒ नमो॑ दे॒वेभ्यः॑ । \newline
46. दे॒वेभ्य॒ इतीति॑ दे॒वेभ्यो॑ दे॒वेभ्य॒ इति॑ । \newline
47. इत्या॑हा॒हे तीत्या॑ह । \newline
48. आ॒ह॒ न॒म॒स्का॒रो न॑मस्का॒र आ॑हाह नमस्का॒रः । \newline
49. न॒म॒स्का॒रो हि हि न॑मस्का॒रो न॑मस्का॒रो हि । \newline
50. न॒म॒स्का॒र इति॑ नमः - का॒रः । \newline
51. हि दे॒वाना᳚म् दे॒वानाꣳ॒॒ हि हि दे॒वाना᳚म् । \newline
52. दे॒वानाꣳ॑ स्व॒धा स्व॒धा दे॒वाना᳚म् दे॒वानाꣳ॑ स्व॒धा । \newline
53. स्व॒धा पि॒तृभ्यः॑ पि॒तृभ्यः॑ स्व॒धा स्व॒धा पि॒तृभ्यः॑ । \newline
54. स्व॒धेति॑ स्व - धा । \newline
55. पि॒तृभ्य॒ इतीति॑ पि॒तृभ्यः॑ पि॒तृभ्य॒ इति॑ । \newline
56. पि॒तृभ्य॒ इति॑ पि॒तृ - भ्यः॒ । \newline
57. इत्या॑हा॒हे तीत्या॑ह । \newline
58. आ॒ह॒ स्व॒धा॒का॒रः स्व॑धाका॒र आ॑हाह स्वधाका॒रः । \newline
59. स्व॒धा॒का॒रो हि हि स्व॑धाका॒रः स्व॑धाका॒रो हि । \newline
60. स्व॒धा॒का॒र इति॑ स्वधा - का॒रः । \newline
61. हि पि॑तृ॒णाम् पि॑तृ॒णाꣳ हि हि पि॑तृ॒णाम् । \newline

\textbf{Ghana Paata } \newline

1. दे॒वा नु॒पै त्यु॒पैति॑ दे॒वान् दे॒वा नु॒पैती॒द मि॒द मु॒पैति॑ दे॒वान् दे॒वा नु॒पैती॒दम् । \newline
2. उ॒पैती॒द मि॒द मु॒पै त्यु॒पैती॒द म॒ह म॒ह मि॒द मु॒पै त्यु॒पैती॒द म॒हम् । \newline
3. उ॒पैतीत्यु॑प - एति॑ । \newline
4. इ॒द म॒ह म॒ह मि॒द मि॒द म॒हम् म॑नु॒ष्यो॑ मनु॒ष्यो॑ ऽह मि॒द मि॒द म॒हम् म॑नु॒ष्यः॑ । \newline
5. अ॒हम् म॑नु॒ष्यो॑ मनु॒ष्यो॑ ऽह म॒हम् म॑नु॒ष्यो॑ मनु॒ष्या᳚न् मनु॒ष्या᳚न् मनु॒ष्यो॑ ऽह म॒हम् म॑नु॒ष्यो॑ मनु॒ष्यान्॑ । \newline
6. म॒नु॒ष्यो॑ मनु॒ष्या᳚न् मनु॒ष्या᳚न् मनु॒ष्यो॑ मनु॒ष्यो॑ मनु॒ष्या॑ नितीति॑ मनु॒ष्या᳚न् मनु॒ष्यो॑ मनु॒ष्यो॑ मनु॒ष्या॑ निति॑ । \newline
7. म॒नु॒ष्या॑ नितीति॑ मनु॒ष्या᳚न् मनु॒ष्या॑ नित्या॑हा॒ हेति॑ मनु॒ष्या᳚न् मनु॒ष्या॑ नित्या॑ह । \newline
8. इत्या॑हा॒हे तीत्या॑ह मनु॒ष्यो॑ मनु॒ष्य॑ आ॒हे तीत्या॑ह मनु॒ष्यः॑ । \newline
9. आ॒ह॒ म॒नु॒ष्यो॑ मनु॒ष्य॑ आहाह मनु॒ष्यो॑ हि हि म॑नु॒ष्य॑ आहाह मनु॒ष्यो॑ हि । \newline
10. म॒नु॒ष्यो॑ हि हि म॑नु॒ष्यो॑ मनु॒ष्यो᳚(1॒) ह्ये॑ष ए॒ष हि म॑नु॒ष्यो॑ मनु॒ष्यो᳚(1॒) ह्ये॑षः । \newline
11. ह्ये॑ष ए॒ष हि ह्ये॑ष सन् थ्सन् ने॒ष हि ह्ये॑ष सन्न् । \newline
12. ए॒ष सन् थ्सन् ने॒ष ए॒ष सन् म॑नु॒ष्या᳚न् मनु॒ष्या᳚न् थ्सन् ने॒ष ए॒ष सन् म॑नु॒ष्यान्॑ । \newline
13. सन् म॑नु॒ष्या᳚न् मनु॒ष्या᳚न् थ्सन् थ्सन् म॑नु॒ष्या॑ नु॒पै त्यु॒पैति॑ मनु॒ष्या᳚न् थ्सन् थ्सन् म॑नु॒ष्या॑ नु॒पैति॑ । \newline
14. म॒नु॒ष्या॑ नु॒पै त्यु॒पैति॑ मनु॒ष्या᳚न् मनु॒ष्या॑ नु॒पैति॒ यद् यदु॒पैति॑ मनु॒ष्या᳚न् मनु॒ष्या॑ नु॒पैति॒ यत् । \newline
15. उ॒पैति॒ यद् यदु॒पै त्यु॒पैति॒ यदे॒त दे॒तद् यदु॒पै त्यु॒पैति॒ यदे॒तत् । \newline
16. उ॒पैतीत्यु॑प - एति॑ । \newline
17. यदे॒त दे॒तद् यद् यदे॒तद् यजु॒र् यजु॑ रे॒तद् यद् यदे॒तद् यजुः॑ । \newline
18. ए॒तद् यजु॒र् यजु॑ रे॒त दे॒तद् यजु॒र् न न यजु॑ रे॒त दे॒तद् यजु॒र् न । \newline
19. यजु॒र् न न यजु॒र् यजु॒र् न ब्रू॒याद् ब्रू॒यान् न यजु॒र् यजु॒र् न ब्रू॒यात् । \newline
20. न ब्रू॒याद् ब्रू॒यान् न न ब्रू॒या दप्र॑जा॒ अप्र॑जा ब्रू॒यान् न न ब्रू॒या दप्र॑जाः । \newline
21. ब्रू॒या दप्र॑जा॒ अप्र॑जा ब्रू॒याद् ब्रू॒या दप्र॑जा अप॒शु र॑प॒शु रप्र॑जा ब्रू॒याद् ब्रू॒या दप्र॑जा अप॒शुः । \newline
22. अप्र॑जा अप॒शु र॑प॒शु रप्र॑जा॒ अप्र॑जा अप॒शुर् यज॑मानो॒ यज॑मानो ऽप॒शु रप्र॑जा॒ अप्र॑जा अप॒शुर् यज॑मानः । \newline
23. अप्र॑जा॒ इत्यप्र॑ - जाः॒ । \newline
24. अ॒प॒शुर् यज॑मानो॒ यज॑मानो ऽप॒शु र॑प॒शुर् यज॑मानः स्याथ् स्या॒द् यज॑मानो ऽप॒शु र॑प॒शुर् यज॑मानः स्यात् । \newline
25. यज॑मानः स्याथ् स्या॒द् यज॑मानो॒ यज॑मानः स्याथ् स॒ह स॒ह स्या॒द् यज॑मानो॒ यज॑मानः स्याथ् स॒ह । \newline
26. स्या॒थ् स॒ह स॒ह स्या᳚थ् स्याथ् स॒ह प्र॒जया᳚ प्र॒जया॑ स॒ह स्या᳚थ् स्याथ् स॒ह प्र॒जया᳚ । \newline
27. स॒ह प्र॒जया᳚ प्र॒जया॑ स॒ह स॒ह प्र॒जया॑ स॒ह स॒ह प्र॒जया॑ स॒ह स॒ह प्र॒जया॑ स॒ह । \newline
28. प्र॒जया॑ स॒ह स॒ह प्र॒जया᳚ प्र॒जया॑ स॒ह रा॒यो रा॒यः स॒ह प्र॒जया᳚ प्र॒जया॑ स॒ह रा॒यः । \newline
29. प्र॒जयेति॑ प्र - जया᳚ । \newline
30. स॒ह रा॒यो रा॒यः स॒ह स॒ह रा॒य स्पोषे॑ण॒ पोषे॑ण रा॒यः स॒ह स॒ह रा॒य स्पोषे॑ण । \newline
31. रा॒य स्पोषे॑ण॒ पोषे॑ण रा॒यो रा॒य स्पोषे॒णे तीति॒ पोषे॑ण रा॒यो रा॒य स्पोषे॒णेति॑ । \newline
32. पोषे॒णे तीति॒ पोषे॑ण॒ पोषे॒णे त्या॑हा॒ हेति॒ पोषे॑ण॒ पोषे॒णे त्या॑ह । \newline
33. इत्या॑हा॒हे तीत्या॑ह प्र॒जया᳚ प्र॒जया॒ ऽऽहे तीत्या॑ह प्र॒जया᳚ । \newline
34. आ॒ह॒ प्र॒जया᳚ प्र॒जया॑ ऽऽहाह प्र॒जयै॒वैव प्र॒जया॑ ऽऽहाह प्र॒जयै॒व । \newline
35. प्र॒जयै॒ वैव प्र॒जया᳚ प्र॒जयै॒व प॒शुभिः॑ प॒शुभि॑ रे॒व प्र॒जया᳚ प्र॒जयै॒व प॒शुभिः॑ । \newline
36. प्र॒जयेति॑ प्र - जया᳚ । \newline
37. ए॒व प॒शुभिः॑ प॒शुभि॑ रे॒वैव प॒शुभिः॑ स॒ह स॒ह प॒शुभि॑ रे॒वैव प॒शुभिः॑ स॒ह । \newline
38. प॒शुभिः॑ स॒ह स॒ह प॒शुभिः॑ प॒शुभिः॑ स॒हेम मि॒मꣳ स॒ह प॒शुभिः॑ प॒शुभिः॑ स॒हेमम् । \newline
39. प॒शुभि॒रिति॑ प॒शु - भिः॒ । \newline
40. स॒हेम मि॒मꣳ स॒ह स॒हेमम् ॅलो॒कम् ॅलो॒क मि॒मꣳ स॒ह स॒हेमम् ॅलो॒कम् । \newline
41. इ॒मम् ॅलो॒कम् ॅलो॒क मि॒म मि॒मम् ॅलो॒क मु॒पाव॑र्तत उ॒पाव॑र्तते लो॒क मि॒म मि॒मम् ॅलो॒क मु॒पाव॑र्तते । \newline
42. लो॒क मु॒पाव॑र्तत उ॒पाव॑र्तते लो॒कम् ॅलो॒क मु॒पाव॑र्तते॒ नमो॒ नम॑ उ॒पाव॑र्तते लो॒कम् ॅलो॒क मु॒पाव॑र्तते॒ नमः॑ । \newline
43. उ॒पाव॑र्तते॒ नमो॒ नम॑ उ॒पाव॑र्तत उ॒पाव॑र्तते॒ नमो॑ दे॒वेभ्यो॑ दे॒वेभ्यो॒ नम॑ उ॒पाव॑र्तत उ॒पाव॑र्तते॒ नमो॑ दे॒वेभ्यः॑ । \newline
44. उ॒पाव॑र्तत॒ इत्यु॑प - आव॑र्तते । \newline
45. नमो॑ दे॒वेभ्यो॑ दे॒वेभ्यो॒ नमो॒ नमो॑ दे॒वेभ्य॒ इतीति॑ दे॒वेभ्यो॒ नमो॒ नमो॑ दे॒वेभ्य॒ इति॑ । \newline
46. दे॒वेभ्य॒ इतीति॑ दे॒वेभ्यो॑ दे॒वेभ्य॒ इत्या॑हा॒ हेति॑ दे॒वेभ्यो॑ दे॒वेभ्य॒ इत्या॑ह । \newline
47. इत्या॑हा॒हे तीत्या॑ह नमस्का॒रो न॑मस्का॒र आ॒हे तीत्या॑ह नमस्का॒रः । \newline
48. आ॒ह॒ न॒म॒स्का॒रो न॑मस्का॒र आ॑हाह नमस्का॒रो हि हि न॑मस्का॒र आ॑हाह नमस्का॒रो हि । \newline
49. न॒म॒स्का॒रो हि हि न॑मस्का॒रो न॑मस्का॒रो हि दे॒वाना᳚म् दे॒वानाꣳ॒॒ हि न॑मस्का॒रो न॑मस्का॒रो हि दे॒वाना᳚म् । \newline
50. न॒म॒स्का॒र इति॑ नमः - का॒रः । \newline
51. हि दे॒वाना᳚म् दे॒वानाꣳ॒॒ हि हि दे॒वानाꣳ॑ स्व॒धा स्व॒धा दे॒वानाꣳ॒॒ हि हि दे॒वानाꣳ॑ स्व॒धा । \newline
52. दे॒वानाꣳ॑ स्व॒धा स्व॒धा दे॒वाना᳚म् दे॒वानाꣳ॑ स्व॒धा पि॒तृभ्यः॑ पि॒तृभ्यः॑ स्व॒धा दे॒वाना᳚म् दे॒वानाꣳ॑ स्व॒धा पि॒तृभ्यः॑ । \newline
53. स्व॒धा पि॒तृभ्यः॑ पि॒तृभ्यः॑ स्व॒धा स्व॒धा पि॒तृभ्य॒ इतीति॑ पि॒तृभ्यः॑ स्व॒धा स्व॒धा पि॒तृभ्य॒ इति॑ । \newline
54. स्व॒धेति॑ स्व - धा । \newline
55. पि॒तृभ्य॒ इतीति॑ पि॒तृभ्यः॑ पि॒तृभ्य॒ इत्या॑हा॒ हेति॑ पि॒तृभ्यः॑ पि॒तृभ्य॒ इत्या॑ह । \newline
56. पि॒तृभ्य॒ इति॑ पि॒तृ - भ्यः॒ । \newline
57. इत्या॑हा॒हे तीत्या॑ह स्वधाका॒रः स्व॑धाका॒र आ॒हे तीत्या॑ह स्वधाका॒रः । \newline
58. आ॒ह॒ स्व॒धा॒का॒रः स्व॑धाका॒र आ॑हाह स्वधाका॒रो हि हि स्व॑धाका॒र आ॑हाह स्वधाका॒रो हि । \newline
59. स्व॒धा॒का॒रो हि हि स्व॑धाका॒रः स्व॑धाका॒रो हि पि॑तृ॒णाम् पि॑तृ॒णाꣳ हि स्व॑धाका॒रः स्व॑धाका॒रो हि पि॑तृ॒णाम् । \newline
60. स्व॒धा॒का॒र इति॑ स्वधा - का॒रः । \newline
61. हि पि॑तृ॒णाम् पि॑तृ॒णाꣳ हि हि पि॑तृ॒णा मि॒द मि॒दम् पि॑तृ॒णाꣳ हि हि पि॑तृ॒णा मि॒दम् । \newline
\pagebreak
\markright{ TS 6.3.2.6  \hfill https://www.vedavms.in \hfill}

\section{ TS 6.3.2.6 }

\textbf{TS 6.3.2.6 } \newline
\textbf{Samhita Paata} \newline

पि॑तृ॒णामि॒दम॒हं निर्वरु॑णस्य॒ पाशा॒दित्या॑ह वरुणपा॒शादे॒व निर्मु॑च्य॒ते ऽग्ने᳚ व्रतपत आ॒त्मनः॒ पूर्वा॑ त॒नूरा॒देयेत्या॑हुः॒ को हि तद्वेद॒ यद् वसी॑या॒न्थ् स्वे वशे॑ भू॒ते पुन॑र्वा॒ ददा॑ति॒ न वेति॒ ग्रावा॑णो॒ वै सोम॑स्य॒ राज्ञो॑ मलिम्लुसे॒ना य ए॒वं ॅवि॒द्वान् ग्राव्ण्ण॒॒ आग्नी᳚द्ध्र उपवा॒सय॑ति॒ नैनं॑ मलिम्लुसे॒ना वि॑न्दति ( ) ॥ \newline

\textbf{Pada Paata} \newline

पि॒तृ॒णाम् । इ॒दम् । अ॒हम् । निरिति॑ । वरु॑णस्य । पाशा᳚त् । इति॑ । आ॒ह॒ । व॒रु॒ण॒पा॒शादिति॑ वरुण - पा॒शात् । ए॒व । निरिति॑ । मु॒च्य॒ते॒ । अग्ने᳚ । व्र॒त॒प॒त॒ इति॑ व्रत - प॒ते॒ । आ॒त्मनः॑ । पूर्वा᳚ । त॒नूः । आ॒देयेत्या᳚ - देया᳚ । इति॑ । आ॒हुः॒ । कः । हि । तत् । वेद॑ । यत् । वसी॑यान् । स्वे । वशे᳚ । भू॒ते । पुनः॑ । वा॒ । ददा॑ति । न । वा॒ । इति॑ । ग्रावा॑णः । वै । सोम॑स्य । राज्ञ्ः॑ । म॒लि॒म्लु॒से॒नेति॑ मलिम्लु - से॒ना । यः । ए॒वम् । वि॒द्वान् । ग्राव्‌ण्णः॑  । आग्नी᳚द्ध्र॒ इत्याग्नि॑ - इ॒द्ध्रे॒ । उ॒प॒वा॒सय॒तीत्यु॑प - वा॒सय॑ति । न । ए॒न॒म् । म॒लि॒म्लु॒से॒नेति॑ मलिम्लु - से॒ना । वि॒न्द॒ति॒ ( ) ॥  \newline


\textbf{Krama Paata} \newline

पि॒तृ॒णामि॒दम् । इ॒दम॒हम् । अ॒हम् निः । निर् वरु॑णस्य । वरु॑णस्य॒ पाशा᳚त् । पाशा॒दिति॑ । इत्या॑ह । आ॒ह॒ व॒रु॒ण॒पा॒शात् । व॒रु॒ण॒पा॒शादे॒व । व॒रु॒ण॒पा॒शादिति॑ वरुण - पा॒शात् । ए॒व निः । निर् मु॑च्यते । मु॒च्य॒तेऽग्ने᳚ । अग्ने᳚ व्रतपते । व्र॒त॒प॒त॒ आ॒त्मनः॑ । व्र॒त॒प॒त॒ इति॑ व्रत - प॒ते॒ । आ॒त्मनः॒ पूर्वा᳚ । पूर्वा॑ त॒नूः । त॒नूरा॒देया᳚ । आ॒देयेति॑ । आ॒देयेत्या᳚ - देया᳚ । इत्या॑हुः । आ॒हुः॒ कः । को हि । हि तत् । तद् वेद॑ । वेद॒ यत् । यद् वसी॑यान् । वसी॑या॒न्थ् स्वे । स्वे वशे᳚ । वशे॑ भू॒ते । भू॒ते पुनः॑ । पुन॑र् वा । वा॒ ददा॑ति । ददा॑ति॒ न । न वा᳚ । वेति॑ । इति॒ ग्रावा॑णः । ग्रावा॑णो॒ वै । वै सोम॑स्य । सोम॑स्य॒ राज्ञ्ः॑ । राज्ञो॑ मलिम्लुसे॒ना । म॒लि॒म्लु॒से॒ना यः । म॒लि॒म्लु॒से॒नेति॑ मलिम्लु - से॒ना । य ए॒वम् । ए॒वम् ॅवि॒द्वान् । वि॒द्वान् ग्राव्.ण्णः॑ । ग्राव्.ण्ण॒ आग्नी᳚द्ध्रे । आग्नी᳚द्ध्र उपवा॒सय॑ति । आग्नी᳚द्ध्र॒ इत्याग्नि॑ - इ॒द्ध्रे॒ । उ॒प॒वा॒सय॑ति॒ न । उ॒प॒वा॒सय॒तीत्यु॑प - वा॒सय॑ति । नैन᳚म् । ए॒न॒म् म॒लि॒म्लु॒से॒ना । म॒लि॒म्लु॒से॒ना वि॑न्दति ( ) । म॒लि॒म्लु॒से॒नेति॑ मलिम्लु - से॒ना । वि॒न्द॒तीति॑ विन्दति । \newline

\textbf{Jatai Paata} \newline

1. पि॒तृ॒णा मि॒द मि॒दम् पि॑तृ॒णाम् पि॑तृ॒णा मि॒दम् । \newline
2. इ॒द म॒ह म॒ह मि॒द मि॒द म॒हम् । \newline
3. अ॒हन् निर् णिर॒ह म॒हन् निः । \newline
4. निर् वरु॑णस्य॒ वरु॑णस्य॒ निर् णिर् वरु॑णस्य । \newline
5. वरु॑णस्य॒ पाशा॒त् पाशा॒द् वरु॑णस्य॒ वरु॑णस्य॒ पाशा᳚त् । \newline
6. पाशा॒दि तीति॒ पाशा॒त् पाशा॒ दिति॑ । \newline
7. इत्या॑हा॒हे तीत्या॑ह । \newline
8. आ॒ह॒ व॒रु॒ण॒पा॒शाद् व॑रुणपा॒शा दा॑हाह वरुणपा॒शात् । \newline
9. व॒रु॒ण॒पा॒शा दे॒वैव व॑रुणपा॒शाद् व॑रुणपा॒शा दे॒व । \newline
10. व॒रु॒ण॒पा॒शादिति॑ वरुण - पा॒शात् । \newline
11. ए॒व निर् णिरे॒ वैव निः । \newline
12. निर् मु॑च्यते मुच्यते॒ निर् णिर् मु॑च्यते । \newline
13. मु॒च्य॒ते ऽग्ने ऽग्ने॑ मुच्यते मुच्य॒ते ऽग्ने᳚ । \newline
14. अग्ने᳚ व्रतपते व्रतप॒ते ऽग्ने ऽग्ने᳚ व्रतपते । \newline
15. व्र॒त॒प॒त॒ आ॒त्मन॑ आ॒त्मनो᳚ व्रतपते व्रतपत आ॒त्मनः॑ । \newline
16. व्र॒त॒प॒त॒ इति॑ व्रत - प॒ते॒ । \newline
17. आ॒त्मनः॒ पूर्वा॒ पूर्वा॒ ऽऽत्मन॑ आ॒त्मनः॒ पूर्वा᳚ । \newline
18. पूर्वा॑ त॒नू स्त॒नूः पूर्वा॒ पूर्वा॑ त॒नूः । \newline
19. त॒नू रा॒देया॒ ऽऽदेया॑ त॒नू स्त॒नू रा॒देया᳚ । \newline
20. आ॒देयेती त्या॒देया॒ ऽऽदेयेति॑ । \newline
21. आ॒देयेत्या᳚ - देया᳚ । \newline
22. इत्या॑हु राहु॒रि तीत्या॑हुः । \newline
23. आ॒हुः॒ कः क आ॑हु राहुः॒ कः । \newline
24. को हि हि कः को हि । \newline
25. हि तत् तद्धि हि तत् । \newline
26. तद् वेद॒ वेद॒ तत् तद् वेद॑ । \newline
27. वेद॒ यद् यद् वेद॒ वेद॒ यत् । \newline
28. यद् वसी॑या॒न्॒. वसी॑या॒न्॒. यद् यद् वसी॑यान् । \newline
29. वसी॑या॒न् थ्स्वे स्वे वसी॑या॒न्॒. वसी॑या॒न् थ्स्वे । \newline
30. स्वे वशे॒ वशे॒ स्वे स्वे वशे᳚ । \newline
31. वशे॑ भू॒ते भू॒ते वशे॒ वशे॑ भू॒ते । \newline
32. भू॒ते पुनः॒ पुन॑र् भू॒ते भू॒ते पुनः॑ । \newline
33. पुन॑र् वा वा॒ पुनः॒ पुन॑र् वा । \newline
34. वा॒ ददा॑ति॒ ददा॑ति वा वा॒ ददा॑ति । \newline
35. ददा॑ति॒ न न ददा॑ति॒ ददा॑ति॒ न । \newline
36. न वा॑ वा॒ न न वा᳚ । \newline
37. वेतीति॑ वा॒ वेति॑ । \newline
38. इति॒ ग्रावा॑णो॒ ग्रावा॑ण॒ इतीति॒ ग्रावा॑णः । \newline
39. ग्रावा॑णो॒ वै वै ग्रावा॑णो॒ ग्रावा॑णो॒ वै । \newline
40. वै सोम॑स्य॒ सोम॑स्य॒ वै वै सोम॑स्य । \newline
41. सोम॑स्य॒ राज्ञो॒ राज्ञ्ः॒ सोम॑स्य॒ सोम॑स्य॒ राज्ञ्ः॑ । \newline
42. राज्ञो॑ मलिम्लुसे॒ना म॑लिम्लुसे॒ना राज्ञो॒ राज्ञो॑ मलिम्लुसे॒ना । \newline
43. म॒लि॒म्लु॒से॒ना यो यो म॑लिम्लुसे॒ना म॑लिम्लुसे॒ना यः । \newline
44. म॒लि॒म्लु॒से॒नेति॑ मलिम्लु - से॒ना । \newline
45. य ए॒व मे॒वं ॅयो य ए॒वम् । \newline
46. ए॒वं ॅवि॒द्वान्. वि॒द्वा ने॒व मे॒वं ॅवि॒द्वान् । \newline
47. वि॒द्वान् ग्राव्.ण्णो॒ ग्राव्.ण्णो॑ वि॒द्वान्. वि॒द्वान् ग्राव्.ण्णः॑ । \newline
48. ग्राव्.ण्ण॒ आग्नी᳚द्ध्र॒ आग्नी᳚द्ध्रे॒ ग्राव्.ण्णो॒ ग्राव्.ण्ण॒ आग्नी᳚द्ध्रे । \newline
49. आग्नी᳚द्ध्र उपवा॒सय॑ त्युपवा॒सय॒ त्याग्नी᳚द्ध्र॒ आग्नी᳚द्ध्र उपवा॒सय॑ति । \newline
50. आग्नी᳚द्ध्र॒ इत्याग्नि॑ - इ॒द्ध्रे॒ । \newline
51. उ॒प॒वा॒सय॑ति॒ न नोप॑वा॒सय॑ त्युपवा॒सय॑ति॒ न । \newline
52. उ॒प॒वा॒सय॒तीत्यु॑प - वा॒सय॑ति । \newline
53. नैन॑ मेन॒न्न नैन᳚म् । \newline
54. ए॒न॒म् म॒लि॒म्लु॒से॒ना म॑लिम्लुसे॒नैन॑ मेनम् मलिम्लुसे॒ना । \newline
55. म॒लि॒म्लु॒से॒ना वि॑न्दति विन्दति मलिम्लुसे॒ना म॑लिम्लुसे॒ना वि॑न्दति । \newline
56. म॒लि॒म्लु॒से॒नेति॑ मलिम्लु - से॒ना । \newline
57. वि॒न्द॒तीति॑ विन्दति । \newline

\textbf{Ghana Paata } \newline

1. पि॒तृ॒णा मि॒द मि॒दम् पि॑तृ॒णाम् पि॑तृ॒णा मि॒द म॒ह म॒ह मि॒दम् पि॑तृ॒णाम् पि॑तृ॒णा मि॒द म॒हम् । \newline
2. इ॒द म॒ह म॒ह मि॒द मि॒द म॒हन् निर् णिर॒ह मि॒द मि॒द म॒हन् निः । \newline
3. अ॒हन् निर् णिर॒ह म॒हन् निर् वरु॑णस्य॒ वरु॑णस्य॒ निर॒ह म॒हन् निर् वरु॑णस्य । \newline
4. निर् वरु॑णस्य॒ वरु॑णस्य॒ निर् णिर् वरु॑णस्य॒ पाशा॒त् पाशा॒द् वरु॑णस्य॒ निर् णिर् वरु॑णस्य॒ पाशा᳚त् । \newline
5. वरु॑णस्य॒ पाशा॒त् पाशा॒द् वरु॑णस्य॒ वरु॑णस्य॒ पाशा॒दि तीति॒ पाशा॒द् वरु॑णस्य॒ वरु॑णस्य॒ पाशा॒दिति॑ । \newline
6. पाशा॒दि तीति॒ पाशा॒त् पाशा॒दि त्या॑हा॒ हेति॒ पाशा॒त् पाशा॒दि त्या॑ह । \newline
7. इत्या॑हा॒हे तीत्या॑ह वरुणपा॒शाद् व॑रुणपा॒शा दा॒हे तीत्या॑ह वरुणपा॒शात् । \newline
8. आ॒ह॒ व॒रु॒ण॒पा॒शाद् व॑रुणपा॒शा दा॑हाह वरुणपा॒शा दे॒वैव व॑रुणपा॒शा दा॑हाह वरुणपा॒शा दे॒व । \newline
9. व॒रु॒ण॒पा॒शा दे॒वैव व॑रुणपा॒शाद् व॑रुणपा॒शा दे॒व निर् णिरे॒व व॑रुणपा॒शाद् व॑रुणपा॒शा दे॒व निः । \newline
10. व॒रु॒ण॒पा॒शादिति॑ वरुण - पा॒शात् । \newline
11. ए॒व निर् णिरे॒ वैव निर् मु॑च्यते मुच्यते॒ निरे॒वैव निर् मु॑च्यते । \newline
12. निर् मु॑च्यते मुच्यते॒ निर् णिर् मु॑च्य॒ते ऽग्ने ऽग्ने॑ मुच्यते॒ निर् णिर् मु॑च्य॒ते ऽग्ने᳚ । \newline
13. मु॒च्य॒ते ऽग्ने ऽग्ने॑ मुच्यते मुच्य॒ते ऽग्ने᳚ व्रतपते व्रतप॒ते ऽग्ने॑ मुच्यते मुच्य॒ते ऽग्ने᳚ व्रतपते । \newline
14. अग्ने᳚ व्रतपते व्रतप॒ते ऽग्ने ऽग्ने᳚ व्रतपत आ॒त्मन॑ आ॒त्मनो᳚ व्रतप॒ते ऽग्ने ऽग्ने᳚ व्रतपत आ॒त्मनः॑ । \newline
15. व्र॒त॒प॒त॒ आ॒त्मन॑ आ॒त्मनो᳚ व्रतपते व्रतपत आ॒त्मनः॒ पूर्वा॒ पूर्वा॒ ऽऽत्मनो᳚ व्रतपते व्रतपत आ॒त्मनः॒ पूर्वा᳚ । \newline
16. व्र॒त॒प॒त॒ इति॑ व्रत - प॒ते॒ । \newline
17. आ॒त्मनः॒ पूर्वा॒ पूर्वा॒ ऽऽत्मन॑ आ॒त्मनः॒ पूर्वा॑ त॒नू स्त॒नूः पूर्वा॒ ऽऽत्मन॑ आ॒त्मनः॒ पूर्वा॑ त॒नूः । \newline
18. पूर्वा॑ त॒नू स्त॒नूः पूर्वा॒ पूर्वा॑ त॒नू रा॒देया॒ ऽऽदेया॑ त॒नूः पूर्वा॒ पूर्वा॑ त॒नू रा॒देया᳚ । \newline
19. त॒नू रा॒देया॒ ऽऽदेया॑ त॒नू स्त॒नू रा॒देयेती त्या॒देया॑ त॒नू स्त॒नू रा॒देयेति॑ । \newline
20. आ॒देयेती त्या॒देया॒ ऽऽदेये त्या॑हु राहु॒ रित्या॒देया॒ ऽऽदेये त्या॑हुः । \newline
21. आ॒देयेत्या᳚ - देया᳚ । \newline
22. इत्या॑हु राहु॒ रिती त्या॑हुः॒ कः क आ॑हु॒ रिती त्या॑हुः॒ कः । \newline
23. आ॒हुः॒ कः क आ॑हु राहुः॒ को हि हि क आ॑हु राहुः॒ को हि । \newline
24. को हि हि कः को हि तत् तद्धि कः को हि तत् । \newline
25. हि तत् तद्धि हि तद् वेद॒ वेद॒ तद्धि हि तद् वेद॑ । \newline
26. तद् वेद॒ वेद॒ तत् तद् वेद॒ यद् यद् वेद॒ तत् तद् वेद॒ यत् । \newline
27. वेद॒ यद् यद् वेद॒ वेद॒ यद् वसी॑या॒न्॒. वसी॑या॒न्॒. यद् वेद॒ वेद॒ यद् वसी॑यान् । \newline
28. यद् वसी॑या॒न्॒. वसी॑या॒न्॒. यद् यद् वसी॑या॒न् थ्स्वे स्वे वसी॑या॒न्॒. यद् यद् वसी॑या॒न् थ्स्वे । \newline
29. वसी॑या॒न् थ्स्वे स्वे वसी॑या॒न्॒. वसी॑या॒न् थ्स्वे वशे॒ वशे॒ स्वे वसी॑या॒न्॒. वसी॑या॒न् थ्स्वे वशे᳚ । \newline
30. स्वे वशे॒ वशे॒ स्वे स्वे वशे॑ भू॒ते भू॒ते वशे॒ स्वे स्वे वशे॑ भू॒ते । \newline
31. वशे॑ भू॒ते भू॒ते वशे॒ वशे॑ भू॒ते पुनः॒ पुन॑र् भू॒ते वशे॒ वशे॑ भू॒ते पुनः॑ । \newline
32. भू॒ते पुनः॒ पुन॑र् भू॒ते भू॒ते पुन॑र् वा वा॒ पुन॑र् भू॒ते भू॒ते पुन॑र् वा । \newline
33. पुन॑र् वा वा॒ पुनः॒ पुन॑र् वा॒ ददा॑ति॒ ददा॑ति वा॒ पुनः॒ पुन॑र् वा॒ ददा॑ति । \newline
34. वा॒ ददा॑ति॒ ददा॑ति वा वा॒ ददा॑ति॒ न न ददा॑ति वा वा॒ ददा॑ति॒ न । \newline
35. ददा॑ति॒ न न ददा॑ति॒ ददा॑ति॒ न वा॑ वा॒ न ददा॑ति॒ ददा॑ति॒ न वा᳚ । \newline
36. न वा॑ वा॒ न न वेतीति॑ वा॒ न न वेति॑ । \newline
37. वेतीति॑ वा॒ वेति॒ ग्रावा॑णो॒ ग्रावा॑ण॒ इति॑ वा॒ वेति॒ ग्रावा॑णः । \newline
38. इति॒ ग्रावा॑णो॒ ग्रावा॑ण॒ इतीति॒ ग्रावा॑णो॒ वै वै ग्रावा॑ण॒ इतीति॒ ग्रावा॑णो॒ वै । \newline
39. ग्रावा॑णो॒ वै वै ग्रावा॑णो॒ ग्रावा॑णो॒ वै सोम॑स्य॒ सोम॑स्य॒ वै ग्रावा॑णो॒ ग्रावा॑णो॒ वै सोम॑स्य । \newline
40. वै सोम॑स्य॒ सोम॑स्य॒ वै वै सोम॑स्य॒ राज्ञो॒ राज्ञ्ः॒ सोम॑स्य॒ वै वै सोम॑स्य॒ राज्ञ्ः॑ । \newline
41. सोम॑स्य॒ राज्ञो॒ राज्ञ्ः॒ सोम॑स्य॒ सोम॑स्य॒ राज्ञो॑ मलिम्लुसे॒ना म॑लिम्लुसे॒ना राज्ञ्ः॒ सोम॑स्य॒ सोम॑स्य॒ राज्ञो॑ मलिम्लुसे॒ना । \newline
42. राज्ञो॑ मलिम्लुसे॒ना म॑लिम्लुसे॒ना राज्ञो॒ राज्ञो॑ मलिम्लुसे॒ना यो यो म॑लिम्लुसे॒ना राज्ञो॒ राज्ञो॑ मलिम्लुसे॒ना यः । \newline
43. म॒लि॒म्लु॒से॒ना यो यो म॑लिम्लुसे॒ना म॑लिम्लुसे॒ना य ए॒व मे॒वं ॅयो म॑लिम्लुसे॒ना म॑लिम्लुसे॒ना य ए॒वम् । \newline
44. म॒लि॒म्लु॒से॒नेति॑ मलिम्लु - से॒ना । \newline
45. य ए॒व मे॒वं ॅयो य ए॒वं ॅवि॒द्वान्. वि॒द्वा ने॒वं ॅयो य ए॒वं ॅवि॒द्वान् । \newline
46. ए॒वं ॅवि॒द्वान्. वि॒द्वा ने॒व मे॒वं ॅवि॒द्वान् ग्राव्.ण्णो॒ ग्राव्.ण्णो॑ वि॒द्वा ने॒व मे॒वं ॅवि॒द्वान् ग्राव्.ण्णः॑ । \newline
47. वि॒द्वान् ग्राव्.ण्णो॒ ग्राव्.ण्णो॑ वि॒द्वान्. वि॒द्वान् ग्राव्.ण्ण॒ आग्नी᳚द्ध्र॒ आग्नी᳚द्ध्रे॒ ग्राव्.ण्णो॑ वि॒द्वान्. वि॒द्वान् ग्राव्.ण्ण॒ आग्नी᳚द्ध्रे । \newline
48. ग्राव्.ण्ण॒ आग्नी᳚द्ध्र॒ आग्नी᳚द्ध्रे॒ ग्राव्.ण्णो॒ ग्राव्.ण्ण॒ आग्नी᳚द्ध्र उपवा॒सय॑ त्युपवा॒सय॒ त्याग्नी᳚द्ध्रे॒ ग्राव्.ण्णो॒ ग्राव्.ण्ण॒ आग्नी᳚द्ध्र उपवा॒सय॑ति । \newline
49. आग्नी᳚द्ध्र उपवा॒सय॑ त्युपवा॒सय॒ त्याग्नी᳚द्ध्र॒ आग्नी᳚द्ध्र उपवा॒सय॑ति॒ न नोप॑वा॒सय॒ त्याग्नी᳚द्ध्र॒ आग्नी᳚द्ध्र उपवा॒सय॑ति॒ न । \newline
50. आग्नी᳚द्ध्र॒ इत्याग्नि॑ - इ॒द्ध्रे॒ । \newline
51. उ॒प॒वा॒सय॑ति॒ न नोप॑वा॒सय॑ त्युपवा॒सय॑ति॒ नैन॑ मेन॒न् नो प॑वा॒सय॑ त्युपवा॒सय॑ति॒ नैन᳚म् । \newline
52. उ॒प॒वा॒सय॒तीत्यु॑प - वा॒सय॑ति । \newline
53. नैन॑ मेन॒न् न नैन॑म् मलिम्लुसे॒ना म॑लिम्लुसे॒ नैन॒न् न नैन॑म् मलिम्लुसे॒ना । \newline
54. ए॒न॒म् म॒लि॒म्लु॒से॒ना म॑लिम्लुसे॒नैन॑ मेनम् मलिम्लुसे॒ना वि॑न्दति विन्दति मलिम्लुसे॒नैन॑ मेनम् मलिम्लुसे॒ना वि॑न्दति । \newline
55. म॒लि॒म्लु॒से॒ना वि॑न्दति विन्दति मलिम्लुसे॒ना म॑लिम्लुसे॒ना वि॑न्दति । \newline
56. म॒लि॒म्लु॒से॒नेति॑ मलिम्लु - से॒ना । \newline
57. वि॒न्द॒तीति॑ विन्दति । \newline
\pagebreak
\markright{ TS 6.3.3.1  \hfill https://www.vedavms.in \hfill}

\section{ TS 6.3.3.1 }

\textbf{TS 6.3.3.1 } \newline
\textbf{Samhita Paata} \newline

वै॒ष्ण॒व्यर्चा हु॒त्वा यूप॒मच्छै॑ति वैष्ण॒वो वै दे॒वत॑या॒ यूपः॒ स्वयै॒वैनं॑ दे॒वत॒या ऽच्छै॒त्यत्य॒न्यानगां॒ नान्या-नुपा॑गा॒मित्या॒हाति॒ ह्य॑न्यानेति॒ नान्या-नु॒पैत्य॒र्वाक्त्वा॒ परै॑रविदं प॒रोऽव॑रै॒रित्या॑हा॒र्वाग्घ्ये॑नं॒ परै᳚र्वि॒न्दति॑ प॒रोऽव॑रै॒स्तं त्वा॑ जुषे- [  ] \newline

\textbf{Pada Paata} \newline

वै॒ष्ण॒व्या । ऋ॒चा । हु॒त्वा । यूप᳚म् । अच्छ॑ । ए॒ति॒ । वै॒ष्ण॒वः । वै । दे॒वत॑या । यूपः॑ । स्वया᳚ । ए॒व । ए॒न॒म् । दे॒वत॑या । अच्छ॑ । ए॒ति॒ । अतीति॑ । अ॒न्यान् । अगा᳚म् । न । अ॒न्यान् । उपेति॑ । अ॒गा॒म् । इति॑ । आ॒ह॒ । अतीति॑ । हि । अ॒न्यान् । एति॑ । न । अ॒न्यान् । उ॒पैतीत्यु॑प-एति॑ । अ॒र्वाक् । त्वा॒ । परैः᳚ । अ॒वि॒द॒म् । प॒रः । अव॑रैः । इति॑ । आ॒ह॒ । अ॒र्वाक् । हि । ए॒न॒म् । परैः᳚ । वि॒न्दति॑ । प॒रः । अव॑रैः । तम् । त्वा॒ । जु॒षे॒ ।  \newline


\textbf{Krama Paata} \newline

वै॒ष्ण॒व्यर्चा । ऋ॒चा हु॒त्वा । हु॒त्वा यूप᳚म् । यूप॒मच्छ॑ । अच्छै॑ति । ए॒ति॒ वै॒ष्ण॒वः । वै॒ष्ण॒वो वै । वै दे॒वत॑या । दे॒वत॑या॒ यूपः॑ । यूपः॒ स्वया᳚ । स्वयै॒व । ए॒वैन᳚म् । ए॒न॒म् दे॒वत॑या । दे॒वत॒याऽच्छ॑ । अच्छै॑ति । ए॒त्यति॑ । अत्य॒न्यान् । अ॒न्यानगा᳚म् । अगा॒म् न । नान्यान् । अ॒न्यानुप॑ । उपा॑गाम् । अ॒गा॒मिति॑ । इत्या॑ह । आ॒हाति॑ । अति॒ हि । ह्य॑न्यान् । अ॒न्यानेति॑ । एति॒ न । नान्यान् । अ॒न्यानु॒पैति॑ । उ॒पैत्य॒र्वाक् । उ॒पैतीत्यु॑प - एति॑ । अ॒र्वाक् त्वा᳚ । त्वा॒ परैः᳚ । परै॑रविदम् । अ॒वि॒द॒म् प॒रः । प॒रोऽव॑रैः । अव॑रै॒रिति॑ । इत्या॑ह । आ॒हा॒र्वाक् । अ॒र्वाग्‌घि । ह्ये॑नम् । ए॒न॒म् परैः᳚ । परै᳚र् वि॒न्दति॑ । वि॒न्दति॑ प॒रः । प॒रोऽव॑रैः । अव॑रै॒स्तम् । तम् त्वा᳚ । त्वा॒ जु॒षे॒ । जु॒षे॒ वै॒ष्ण॒वम् \newline

\textbf{Jatai Paata} \newline

1. वै॒ष्ण॒व्य र्‌च र्‌चा वै᳚ष्ण॒व्या वै᳚ष्ण॒व्य र्‌चा । \newline
2. ऋ॒चा हु॒त्वा हु॒त्व र्‌च र्‌चा हु॒त्वा । \newline
3. हु॒त्वा यूपं॒ ॅयूपꣳ॑ हु॒त्वा हु॒त्वा यूप᳚म् । \newline
4. यूप॒ मच्छा च्छ॒ यूपं॒ ॅयूप॒ मच्छ॑ । \newline
5. अच्छै᳚ त्ये॒ त्यच्छा च्छै॑ति । \newline
6. ए॒ति॒ वै॒ष्ण॒वो वै᳚ष्ण॒व ए᳚त्येति वैष्ण॒वः । \newline
7. वै॒ष्ण॒वो वै वै वै᳚ष्ण॒वो वै᳚ष्ण॒वो वै । \newline
8. वै दे॒वत॑या दे॒वत॑या॒ वै वै दे॒वत॑या । \newline
9. दे॒वत॑या॒ यूपो॒ यूपो॑ दे॒वत॑या दे॒वत॑या॒ यूपः॑ । \newline
10. यूपः॒ स्वया॒ स्वया॒ यूपो॒ यूपः॒ स्वया᳚ । \newline
11. स्वयै॒वैव स्वया॒ स्वयै॒व । \newline
12. ए॒वैन॑ मेन मे॒वै वैन᳚म् । \newline
13. ए॒न॒म् दे॒वत॑या दे॒वत॑यैन मेनम् दे॒वत॑या । \newline
14. दे॒वत॒या ऽच्छाच्छ॑ दे॒वत॑या दे॒वत॒या ऽच्छ॑ । \newline
15. अच्छै᳚ त्ये॒ त्यच्छा च्छै॑ति । \newline
16. ए॒त्य त्यत्ये᳚ त्ये॒ त्यति॑ । \newline
17. अत्य॒न्या न॒न्या नत्य त्य॒न्यान् । \newline
18. अ॒न्या नगा॒ मगा॑ म॒न्या न॒न्या नगा᳚म् । \newline
19. अगा॒न्न नागा॒ मगा॒न्न । \newline
20. नान्या न॒न्यान् न नान्यान् । \newline
21. अ॒न्या नुपो पा॒न्या न॒न्या नुप॑ । \newline
22. उपा॑गा मगा॒ मुपो पा॑गाम् । \newline
23. अ॒गा॒ मिती त्य॑गा मगा॒ मिति॑ । \newline
24. इत्या॑हा॒हे तीत्या॑ह । \newline
25. आ॒हात्य त्या॑हा॒ हाति॑ । \newline
26. अति॒ हि ह्य त्यति॒ हि । \newline
27. ह्य॑न्या न॒न्यान्. हि ह्य॑न्यान् । \newline
28. अ॒न्या नेत्ये त्य॒न्या न॒न्या नेति॑ । \newline
29. एति॒ न नैत्येति॒ न । \newline
30. नान्या न॒न्यान् न नान्यान् । \newline
31. अ॒न्या नु॒पै त्यु॒पै त्य॒न्या न॒न्या नु॒पैति॑ । \newline
32. उ॒पैत्य॒ र्वा ग॒र्वा गु॒पै त्यु॒पै त्य॒र्वाक् । \newline
33. उ॒पैतीत्यु॑प - एति॑ । \newline
34. अ॒र्वाक् त्वा᳚ त्वा॒ ऽर्वा ग॒र्वाक् त्वा᳚ । \newline
35. त्वा॒ परैः॒ परै᳚ स्त्वा त्वा॒ परैः᳚ । \newline
36. परै॑ रविद मविद॒म् परैः॒ परै॑ रविदम् । \newline
37. अ॒वि॒द॒म् प॒रः प॒रो॑ ऽविद मविदम् प॒रः । \newline
38. प॒रो ऽव॑रै॒ रव॑रैः प॒रः प॒रो ऽव॑रैः । \newline
39. अव॑रै॒ रिती त्यव॑रै॒ रव॑रै॒ रिति॑ । \newline
40. इत्या॑हा॒हे तीत्या॑ह । \newline
41. आ॒हा॒ र्वा ग॒र्वा गा॑हा हा॒र्वाक् । \newline
42. अ॒र्वा ग्घि ह्य॑ र्वाग॒ र्वा ग्घि । \newline
43. ह्ये॑न मेनꣳ॒॒ हि ह्ये॑नम् । \newline
44. ए॒न॒म् परैः॒ परै॑ रेन मेन॒म् परैः᳚ । \newline
45. परै᳚र् वि॒न्दति॑ वि॒न्दति॒ परैः॒ परै᳚र् वि॒न्दति॑ । \newline
46. वि॒न्दति॑ प॒रः प॒रो वि॒न्दति॑ वि॒न्दति॑ प॒रः । \newline
47. प॒रो ऽव॑रै॒ रव॑रैः प॒रः प॒रो ऽव॑रैः । \newline
48. अव॑रै॒ स्तम् त मव॑रै॒ रव॑रै॒ स्तम् । \newline
49. तम् त्वा᳚ त्वा॒ तम् तम् त्वा᳚ । \newline
50. त्वा॒ जु॒षे॒ जु॒षे॒ त्वा॒ त्वा॒ जु॒षे॒ । \newline
51. जु॒षे॒ वै॒ष्ण॒वं ॅवै᳚ष्ण॒वम् जु॑षे जुषे वैष्ण॒वम् । \newline

\textbf{Ghana Paata } \newline

1. वै॒ष्ण॒व्य र्‌च र्‌चा वै᳚ष्ण॒व्या वै᳚ष्ण॒व्य र्‌चा हु॒त्वा हु॒त्व र्‌चा वै᳚ष्ण॒व्या वै᳚ष्ण॒व्य र्‌चा हु॒त्वा । \newline
2. ऋ॒चा हु॒त्वा हु॒त्व र्‌च र्‌चा हु॒त्वा यूपं॒ ॅयूपꣳ॑ हु॒त्व र्‌च र्‌चा हु॒त्वा यूप᳚म् । \newline
3. हु॒त्वा यूपं॒ ॅयूपꣳ॑ हु॒त्वा हु॒त्वा यूप॒ मच्छाच्छ॒ यूपꣳ॑ हु॒त्वा हु॒त्वा यूप॒ मच्छ॑ । \newline
4. यूप॒ मच्छाच्छ॒ यूपं॒ ॅयूप॒ मच्छै᳚ त्ये॒त्यच्छ॒ यूपं॒ ॅयूप॒ मच्छै॑ति । \newline
5. अच्छै᳚ त्ये॒त्यच्छा च्छै॑ति वैष्ण॒वो वै᳚ष्ण॒व ए॒त्यच्छा च्छै॑ति वैष्ण॒वः । \newline
6. ए॒ति॒ वै॒ष्ण॒वो वै᳚ष्ण॒व ए᳚त्येति वैष्ण॒वो वै वै वै᳚ष्ण॒व ए᳚त्येति वैष्ण॒वो वै । \newline
7. वै॒ष्ण॒वो वै वै वै᳚ष्ण॒वो वै᳚ष्ण॒वो वै दे॒वत॑या दे॒वत॑या॒ वै वै᳚ष्ण॒वो वै᳚ष्ण॒वो वै दे॒वत॑या । \newline
8. वै दे॒वत॑या दे॒वत॑या॒ वै वै दे॒वत॑या॒ यूपो॒ यूपो॑ दे॒वत॑या॒ वै वै दे॒वत॑या॒ यूपः॑ । \newline
9. दे॒वत॑या॒ यूपो॒ यूपो॑ दे॒वत॑या दे॒वत॑या॒ यूपः॒ स्वया॒ स्वया॒ यूपो॑ दे॒वत॑या दे॒वत॑या॒ यूपः॒ स्वया᳚ । \newline
10. यूपः॒ स्वया॒ स्वया॒ यूपो॒ यूपः॒ स्वयै॒ वैव स्वया॒ यूपो॒ यूपः॒ स्वयै॒व । \newline
11. स्वयै॒वैव स्वया॒ स्वयै॒वैन॑ मेन मे॒व स्वया॒ स्वयै॒वैन᳚म् । \newline
12. ए॒वैन॑ मेन मे॒वै वैन॑म् दे॒वत॑या दे॒वत॑ यैन मे॒वै वैन॑म् दे॒वत॑या । \newline
13. ए॒न॒म् दे॒वत॑या दे॒वत॑ यैन मेनम् दे॒वत॒या ऽच्छाच्छ॑ दे॒वत॑ यैन मेनम् दे॒वत॒या ऽच्छ॑ । \newline
14. दे॒वत॒या ऽच्छाच्छ॑ दे॒वत॑या दे॒वत॒या ऽच्छै᳚ त्ये॒ त्यच्छ॑ दे॒वत॑या दे॒वत॒या ऽच्छै॑ति । \newline
15. अच्छै᳚ त्ये॒ त्य च्छाच्छै॒ त्यत्य त्ये॒ त्यच्छा च्छै॒ त्यति॑ । \newline
16. ए॒त्य त्यत्ये᳚ त्ये॒त्य त्य॒न्या न॒न्या नत्ये᳚ त्ये॒त्य त्य॒न्यान् । \newline
17. अत्य॒न्या न॒न्या नत्य त्य॒न्या नगा॒ मगा॑ म॒न्या नत्य त्य॒न्या नगा᳚म् । \newline
18. अ॒न्या नगा॒ मगा॑ म॒न्या न॒न्या नगा॒न्न नागा॑ म॒न्या न॒न्या नगा॒न्न । \newline
19. अगा॒न्न नागा॒ मगा॒न् नान्या न॒न्यान् नागा॒ मगा॒न् नान्यान् । \newline
20. नान्या न॒न्यान् न नान्या नुपो पा॒न्यान् न नान्या नुप॑ । \newline
21. अ॒न्या नुपो पा॒न्या न॒न्या नुपा॑गा मगा॒ मुपा॒न्या न॒न्या नुपा॑गाम् । \newline
22. उपा॑गा मगा॒ मुपो पा॑गा॒ मिती त्य॑गा॒ मुपो पा॑गा॒ मिति॑ । \newline
23. अ॒गा॒ मिती त्य॑गा मगा॒ मित्या॑ हा॒हे त्य॑गा मगा॒ मित्या॑ह । \newline
24. इत्या॑हा॒हे तीत्या॒हा त्य त्या॒हे तीत्या॒ हाति॑ । \newline
25. आ॒हात्य त्या॑हा॒हाति॒ हि ह्यत्या॑ हा॒हाति॒ हि । \newline
26. अति॒ हि ह्यत्यति॒ ह्य॑न्या न॒न्यान्. ह्यत्यति॒ ह्य॑न्यान् । \newline
27. ह्य॑न्या न॒न्यान्. हि ह्य॑न्या नेत्ये त्य॒न्यान्. हि ह्य॑न्या नेति॑ । \newline
28. अ॒न्या नेत्ये त्य॒न्या न॒न्या नेति॒ न नैत्य॒न्या न॒न्या नेति॒ न । \newline
29. एति॒ न नैत्येति॒ नान्या न॒न्यान् नैत्येति॒ नान्यान् । \newline
30. नान्या न॒न्यान् न नान्या नु॒पै त्यु॒पै त्य॒न्यान् न नान्या नु॒पैति॑ । \newline
31. अ॒न्या नु॒पै त्यु॒पै त्य॒न्या न॒न्या नु॒पै त्य॒र्वा ग॒र्वा गु॒पै त्य॒न्या न॒न्या नु॒पै त्य॒र्वाक् । \newline
32. उ॒पै त्य॒र्वा ग॒र्वा गु॒पै त्यु॒पै त्य॒र्वाक् त्वा᳚ त्वा॒ ऽर्वा गु॒पै त्यु॒पै त्य॒र्वाक् त्वा᳚ । \newline
33. उ॒पैतीत्यु॑प - एति॑ । \newline
34. अ॒र्वाक् त्वा᳚ त्वा॒ ऽर्वा ग॒र्वाक् त्वा॒ परैः॒ परै᳚ स्त्वा॒ ऽर्वा ग॒र्वाक् त्वा॒ परैः᳚ । \newline
35. त्वा॒ परैः॒ परै᳚ स्त्वा त्वा॒ परै॑ रविद मविद॒म् परै᳚ स्त्वा त्वा॒ परै॑ रविदम् । \newline
36. परै॑ रविद मविद॒म् परैः॒ परै॑ रविदम् प॒रः प॒रो॑ ऽविद॒म् परैः॒ परै॑ रविदम् प॒रः । \newline
37. अ॒वि॒द॒म् प॒रः प॒रो॑ ऽविद मविदम् प॒रो ऽव॑रै॒ रव॑रैः प॒रो॑ ऽविद मविदम् प॒रो ऽव॑रैः । \newline
38. प॒रो ऽव॑रै॒ रव॑रैः प॒रः प॒रो ऽव॑रै॒ रिती त्यव॑रैः प॒रः प॒रो ऽव॑रै॒ रिति॑ । \newline
39. अव॑रै॒ रितीत्य व॑रै॒ रव॑रै॒ रित्या॑हा॒हे त्यव॑रै॒ रव॑रै॒ रित्या॑ह । \newline
40. इत्या॑हा॒हे तीत्या॑ हा॒र्वा ग॒र्वा गा॒हे तीत्या॑ हा॒र्वाक् । \newline
41. आ॒हा॒र्वा ग॒र्वा गा॑हा हा॒र्वाग्घि ह्य॑र्वा गा॑हा हा॒र्वाग्घि । \newline
42. अ॒र्वाग्घि ह्य॑र्वा ग॒र्वाग् घ्ये॑न मेनꣳ॒॒ ह्य॑र्वा ग॒र्वाग् घ्ये॑नम् । \newline
43. ह्ये॑न मेनꣳ॒॒ हि ह्ये॑न॒म् परैः॒ परै॑ रेनꣳ॒॒ हि ह्ये॑न॒म् परैः᳚ । \newline
44. ए॒न॒म् परैः॒ परै॑ रेन मेन॒म् परै᳚र् वि॒न्दति॑ वि॒न्दति॒ परै॑ रेन मेन॒म् परै᳚र् वि॒न्दति॑ । \newline
45. परै᳚र् वि॒न्दति॑ वि॒न्दति॒ परैः॒ परै᳚र् वि॒न्दति॑ प॒रः प॒रो वि॒न्दति॒ परैः॒ परै᳚र् वि॒न्दति॑ प॒रः । \newline
46. वि॒न्दति॑ प॒रः प॒रो वि॒न्दति॑ वि॒न्दति॑ प॒रो ऽव॑रै॒ रव॑रैः प॒रो वि॒न्दति॑ वि॒न्दति॑ प॒रो ऽव॑रैः । \newline
47. प॒रो ऽव॑रै॒ रव॑रैः प॒रः प॒रो ऽव॑रै॒ स्तम् त मव॑रैः प॒रः प॒रो ऽव॑रै॒ स्तम् । \newline
48. अव॑रै॒ स्तम् त मव॑रै॒ रव॑रै॒ स्तम् त्वा᳚ त्वा॒ त मव॑रै॒ रव॑रै॒ स्तम् त्वा᳚ । \newline
49. तम् त्वा᳚ त्वा॒ तम् तम् त्वा॑ जुषे जुषे त्वा॒ तम् तम् त्वा॑ जुषे । \newline
50. त्वा॒ जु॒षे॒ जु॒षे॒ त्वा॒ त्वा॒ जु॒षे॒ वै॒ष्ण॒वं ॅवै᳚ष्ण॒वम् जु॑षे त्वा त्वा जुषे वैष्ण॒वम् । \newline
51. जु॒षे॒ वै॒ष्ण॒वं ॅवै᳚ष्ण॒वम् जु॑षे जुषे वैष्ण॒वम् दे॑वय॒ज्यायै॑ देवय॒ज्यायै॑ वैष्ण॒वम् जु॑षे जुषे वैष्ण॒वम् दे॑वय॒ज्यायै᳚ । \newline
\pagebreak
\markright{ TS 6.3.3.2  \hfill https://www.vedavms.in \hfill}

\section{ TS 6.3.3.2 }

\textbf{TS 6.3.3.2 } \newline
\textbf{Samhita Paata} \newline

वैष्ण॒वं दे॑वय॒ज्याया॒ इत्या॑ह देवय॒ज्यायै॒ ह्ये॑नं जु॒षते॑ दे॒वस्त्वा॑ सवि॒ता मद्ध्वा॑न॒त्तिवत्या॑ह॒ तेज॑सै॒वैन॑-मन॒क्त्योष॑धे॒ त्राय॑स्वैनꣳ॒॒ स्वधि॑ते॒ मैनꣳ॑ हिꣳसी॒रित्या॑ह॒ वज्रो॒ वै स्वधि॑तिः॒ शान्त्यै॒ स्वधि॑तेर्वृ॒क्षस्य॒ बिभ्य॑तः प्रथ॒मेन॒ शक॑लेन स॒ह तेजः॒ परा॑ पतति॒ यः प्र॑थ॒मः शक॑लः परा॒पते॒त् तमप्या ह॑रे॒थ् सते॑जस- [  ] \newline

\textbf{Pada Paata} \newline

वै॒ष्ण॒वम् । दे॒व॒य॒ज्याया॒ इति॑ देव - य॒ज्यायै᳚ । इति॑ । आ॒ह॒ । दे॒व॒य॒ज्याया॒ इति॑ देव-य॒ज्यायै᳚ । हि । ए॒न॒म् । जु॒षते᳚ । दे॒वः । त्वा॒ । स॒वि॒ता । मद्ध्वा᳚ । अ॒न॒क्तु॒ । इति॑ । आ॒ह॒ । तेज॑सा । ए॒व । ए॒न॒म् । अ॒न॒क्ति॒ । ओष॑धे । त्राय॑स्व । ए॒न॒म् । स्वधि॑त॒ इति॒ स्व-धि॒ते॒ । मा । ए॒न॒म् । हिꣳ॒॒सीः॒ । इति॑ । आ॒ह॒ । वज्रः॑ । वै । स्वधि॑ति॒रिति॒ स्व - धि॒तिः॒ । शान्त्यै᳚ । स्वधि॑ते॒रिति॒ स्व - धि॒तेः॒ । वृ॒क्षस्य॑ । बिभ्य॑तः । प्र॒थ॒मेन॑ । शक॑लेन । स॒ह । तेजः॑ । परेति॑ । प॒त॒ति॒ । यः । प्र॒थ॒मः । शक॑लः । प॒रा॒पते॒दिति॑ परा - पते᳚त् । तम् । अपि॑ । एति॑ । ह॒रे॒त् । सते॑जस॒मिति॒ स - ते॒ज॒स॒म् ।  \newline


\textbf{Krama Paata} \newline

वै॒ष्ण॒वम् दे॑वय॒ज्यायै᳚ । दे॒व॒य॒ज्याया॒ इति॑ । दे॒व॒य॒ज्याया॒ इति॑ देव - य॒ज्यायै᳚ । इत्या॑ह । आ॒ह॒ दे॒व॒य॒ज्यायै᳚ । दे॒व॒य॒ज्यायै॒ हि । दे॒व॒य॒ज्याया॒ इति॑ देव - य॒ज्यायै᳚ । ह्ये॑नम् । ए॒न॒म् जु॒षते᳚ । जु॒षते॑ दे॒वः । दे॒वस्त्वा᳚ । त्वा॒ स॒वि॒ता । स॒वि॒ता मद्ध्वा᳚ । मद्ध्वा॑ऽनक्तु । अ॒न॒क्त्विति॑ । इत्या॑ह । आ॒ह॒ तेज॑सा । तेज॑सै॒व । ए॒वैन᳚म् । ए॒न॒म॒न॒क्ति॒ । अ॒न॒क्त्योष॑धे । ओष॑धे॒ त्राय॑स्व । त्राय॑स्वैनम् । ए॒नꣳ॒॒ स्वधि॑ते । स्वधि॑ते॒ मा । स्वधि॑त॒ इति॒ स्व - धि॒ते॒ । मैन᳚म् । ए॒नꣳ॒॒ हिꣳ॒॒सीः॒ । हिꣳ॒॒सी॒रिति॑ । इत्या॑ह । आ॒ह॒ वज्रः॑ । वज्रो॒ वै । वै स्वधि॑तिः । स्वधि॑तिः॒ शान्त्यै᳚ । स्वधि॑ति॒रिति॒ स्व - धि॒तिः॒ । शान्त्यै॒ स्वधि॑तेः । स्वधि॑तेर् वृ॒क्षस्य॑ । स्वधि॑ते॒रिति॒ स्व - धि॒तेः॒ । वृ॒क्षस्य॒ बिभ्य॑तः । बिभ्य॑तः प्रथ॒मेन॑ । प्र॒थ॒मेन॒ शक॑लेन । शक॑लेन स॒ह । स॒ह तेजः॑ । तेजः॒ परा᳚ । परा॑ पतति । प॒त॒ति॒ यः । यः प्र॑थ॒मः । प्र॒थ॒मः शक॑लः । शक॑लः परा॒पते᳚त् । प॒रा॒पते॒त् तम् । प॒रा॒पते॒दिति॑ परा - पते᳚त् । तमपि॑ । अप्या । आ ह॑रेत् । ह॒रे॒थ् सते॑जसम् । सते॑जसमे॒व । सते॑जस॒मिति॒ स - ते॒ज॒स॒म् \newline

\textbf{Jatai Paata} \newline

1. वै॒ष्ण॒वम् दे॑वय॒ज्यायै॑ देवय॒ज्यायै॑ वैष्ण॒वं ॅवै᳚ष्ण॒वम् दे॑वय॒ज्यायै᳚ । \newline
2. दे॒व॒य॒ज्याया॒ इतीति॑ देवय॒ज्यायै॑ देवय॒ज्याया॒ इति॑ । \newline
3. दे॒व॒य॒ज्याया॒ इति॑ देव - य॒ज्यायै᳚ । \newline
4. इत्या॑हा॒हे तीत्या॑ह । \newline
5. आ॒ह॒ दे॒व॒य॒ज्यायै॑ देवय॒ज्याया॑ आहाह देवय॒ज्यायै᳚ । \newline
6. दे॒व॒य॒ज्यायै॒ हि हि दे॑वय॒ज्यायै॑ देवय॒ज्यायै॒ हि । \newline
7. दे॒व॒य॒ज्याया॒ इति॑ देव - य॒ज्यायै᳚ । \newline
8. ह्ये॑न मेनꣳ॒॒ हि ह्ये॑नम् । \newline
9. ए॒न॒म् जु॒षते॑ जु॒षत॑ एन मेनम् जु॒षते᳚ । \newline
10. जु॒षते॑ दे॒वो दे॒वो जु॒षते॑ जु॒षते॑ दे॒वः । \newline
11. दे॒व स्त्वा᳚ त्वा दे॒वो दे॒व स्त्वा᳚ । \newline
12. त्वा॒ स॒वि॒ता स॑वि॒ता त्वा᳚ त्वा सवि॒ता । \newline
13. स॒वि॒ता मद्ध्वा॒ मद्ध्वा॑ सवि॒ता स॑वि॒ता मद्ध्वा᳚ । \newline
14. मद्ध्वा॑ ऽनक् त्वनक्तु॒ मद्ध्वा॒ मद्ध्वा॑ ऽनक्तु । \newline
15. अ॒न॒क् त्वितीत्य॑नक् त्वन॒क् त्विति॑ । \newline
16. इत्या॑हा॒हे तीत्या॑ह । \newline
17. आ॒ह॒ तेज॑सा॒ तेज॑सा ऽऽहाह॒ तेज॑सा । \newline
18. तेज॑सै॒ वैव तेज॑सा॒ तेज॑सै॒व । \newline
19. ए॒वैन॑ मेन मे॒वै वैन᳚म् । \newline
20. ए॒न॒ म॒न॒क् त्य॒न॒क् त्ये॒न॒ मे॒न॒ म॒न॒क्ति॒ । \newline
21. अ॒न॒क् त्योष॑ध॒ ओष॑धे ऽनक् त्यन॒क् त्योष॑धे । \newline
22. ओष॑धे॒ त्राय॑स्व॒ त्राय॒स्वौष॑ध॒ ओष॑धे॒ त्राय॑स्व । \newline
23. त्राय॑स्वैन मेन॒म् त्राय॑स्व॒ त्राय॑स्वैनम् । \newline
24. ए॒नꣳ॒॒ स्वधि॑ते॒ स्वधि॑त एन मेनꣳ॒॒ स्वधि॑ते । \newline
25. स्वधि॑ते॒ मा मा स्वधि॑ते॒ स्वधि॑ते॒ मा । \newline
26. स्वधि॑त॒ इति॒ स्व - धि॒ते॒ । \newline
27. मैन॑ मेन॒म् मा मैन᳚म् । \newline
28. ए॒नꣳ॒॒ हिꣳ॒॒सी॒र्॒. हिꣳ॒॒सी॒ रे॒न॒ मे॒नꣳ॒॒ हिꣳ॒॒सीः॒ । \newline
29. हिꣳ॒॒सी॒रि तीति॑ हिꣳसीर्. हिꣳसी॒ रिति॑ । \newline
30. इत्या॑हा॒हे तीत्या॑ह । \newline
31. आ॒ह॒ वज्रो॒ वज्र॑ आहाह॒ वज्रः॑ । \newline
32. वज्रो॒ वै वै वज्रो॒ वज्रो॒ वै । \newline
33. वै स्वधि॑तिः॒ स्वधि॑ति॒र् वै वै स्वधि॑तिः । \newline
34. स्वधि॑तिः॒ शान्त्यै॒ शान्त्यै॒ स्वधि॑तिः॒ स्वधि॑तिः॒ शान्त्यै᳚ । \newline
35. स्वधि॑ति॒रिति॒ स्व - धि॒तिः॒ । \newline
36. शान्त्यै॒ स्वधि॑तेः॒ स्वधि॑तेः॒ शान्त्यै॒ शान्त्यै॒ स्वधि॑तेः । \newline
37. स्वधि॑तेर् वृ॒क्षस्य॑ वृ॒क्षस्य॒ स्वधि॑तेः॒ स्वधि॑तेर् वृ॒क्षस्य॑ । \newline
38. स्वधि॑ते॒रिति॒ स्व - धि॒तेः॒ । \newline
39. वृ॒क्षस्य॒ बिभ्य॑तो॒ बिभ्य॑तो वृ॒क्षस्य॑ वृ॒क्षस्य॒ बिभ्य॑तः । \newline
40. बिभ्य॑तः प्रथ॒मेन॑ प्रथ॒मेन॒ बिभ्य॑तो॒ बिभ्य॑तः प्रथ॒मेन॑ । \newline
41. प्र॒थ॒मेन॒ शक॑लेन॒ शक॑लेन प्रथ॒मेन॑ प्रथ॒मेन॒ शक॑लेन । \newline
42. शक॑लेन स॒ह स॒ह शक॑लेन॒ शक॑लेन स॒ह । \newline
43. स॒ह तेज॒ स्तेजः॑ स॒ह स॒ह तेजः॑ । \newline
44. तेजः॒ परा॒ परा॒ तेज॒ स्तेजः॒ परा᳚ । \newline
45. परा॑ पतति पतति॒ परा॒ परा॑ पतति । \newline
46. प॒त॒ति॒ यो यः प॑तति पतति॒ यः । \newline
47. यः प्र॑थ॒मः प्र॑थ॒मो यो यः प्र॑थ॒मः । \newline
48. प्र॒थ॒मः शक॑लः॒ शक॑लः प्रथ॒मः प्र॑थ॒मः शक॑लः । \newline
49. शक॑लः परा॒पते᳚त् परा॒पते॒ च्छक॑लः॒ शक॑लः परा॒पते᳚त् । \newline
50. प॒रा॒पते॒त् तम् तम् प॑रा॒पते᳚त् परा॒पते॒त् तम् । \newline
51. प॒रा॒पते॒दिति॑ परा - पते᳚त् । \newline
52. त मप्यपि॒ तम् त मपि॑ । \newline
53. अप्या ऽप्यप्या । \newline
54. आ ह॑रे द्धरे॒दा ह॑रेत् । \newline
55. ह॒रे॒थ् सते॑जसꣳ॒॒ सते॑जसꣳ हरे द्धरे॒थ् सते॑जसम् । \newline
56. सते॑जस मे॒वैव सते॑जसꣳ॒॒ सते॑जस मे॒व । \newline
57. सते॑जस॒मिति॒ स - ते॒ज॒स॒म् । \newline

\textbf{Ghana Paata } \newline

1. वै॒ष्ण॒वम् दे॑वय॒ज्यायै॑ देवय॒ज्यायै॑ वैष्ण॒वं ॅवै᳚ष्ण॒वम् दे॑वय॒ज्याया॒ इतीति॑ देवय॒ज्यायै॑ वैष्ण॒वं ॅवै᳚ष्ण॒वम् दे॑वय॒ज्याया॒ इति॑ । \newline
2. दे॒व॒य॒ज्याया॒ इतीति॑ देवय॒ज्यायै॑ देवय॒ज्याया॒ इत्या॑हा॒ हेति॑ देवय॒ज्यायै॑ देवय॒ज्याया॒ इत्या॑ह । \newline
3. दे॒व॒य॒ज्याया॒ इति॑ देव - य॒ज्यायै᳚ । \newline
4. इत्या॑हा॒हे तीत्या॑ह देवय॒ज्यायै॑ देवय॒ज्याया॑ आ॒हे तीत्या॑ह देवय॒ज्यायै᳚ । \newline
5. आ॒ह॒ दे॒व॒य॒ज्यायै॑ देवय॒ज्याया॑ आहाह देवय॒ज्यायै॒ हि हि दे॑वय॒ज्याया॑ आहाह देवय॒ज्यायै॒ हि । \newline
6. दे॒व॒य॒ज्यायै॒ हि हि दे॑वय॒ज्यायै॑ देवय॒ज्यायै॒ ह्ये॑न मेनꣳ॒॒ हि दे॑वय॒ज्यायै॑ देवय॒ज्यायै॒ ह्ये॑नम् । \newline
7. दे॒व॒य॒ज्याया॒ इति॑ देव - य॒ज्यायै᳚ । \newline
8. ह्ये॑न मेनꣳ॒॒ हि ह्ये॑नम् जु॒षते॑ जु॒षत॑ एनꣳ॒॒ हि ह्ये॑नम् जु॒षते᳚ । \newline
9. ए॒न॒म् जु॒षते॑ जु॒षत॑ एन मेनम् जु॒षते॑ दे॒वो दे॒वो जु॒षत॑ एन मेनम् जु॒षते॑ दे॒वः । \newline
10. जु॒षते॑ दे॒वो दे॒वो जु॒षते॑ जु॒षते॑ दे॒व स्त्वा᳚ त्वा दे॒वो जु॒षते॑ जु॒षते॑ दे॒व स्त्वा᳚ । \newline
11. दे॒व स्त्वा᳚ त्वा दे॒वो दे॒व स्त्वा॑ सवि॒ता स॑वि॒ता त्वा॑ दे॒वो दे॒व स्त्वा॑ सवि॒ता । \newline
12. त्वा॒ स॒वि॒ता स॑वि॒ता त्वा᳚ त्वा सवि॒ता मद्ध्वा॒ मद्ध्वा॑ सवि॒ता त्वा᳚ त्वा सवि॒ता मद्ध्वा᳚ । \newline
13. स॒वि॒ता मद्ध्वा॒ मद्ध्वा॑ सवि॒ता स॑वि॒ता मद्ध्वा॑ ऽनक् त्वनक्तु॒ मद्ध्वा॑ सवि॒ता स॑वि॒ता मद्ध्वा॑ ऽनक्तु । \newline
14. मद्ध्वा॑ ऽनक् त्वनक्तु॒ मद्ध्वा॒ मद्ध्वा॑ ऽन॒क्त्विती त्य॑नक्तु॒ मद्ध्वा॒ मद्ध्वा॑ ऽन॒क्त्विति॑ । \newline
15. अ॒न॒क् त्विती त्य॑नक् त्वन॒क् त्वित्या॑हा॒हे त्य॑नक् त्वन॒क् त्वित्या॑ह । \newline
16. इत्या॑हा॒हे तीत्या॑ह॒ तेज॑सा॒ तेज॑सा॒ ऽऽहे तीत्या॑ह॒ तेज॑सा । \newline
17. आ॒ह॒ तेज॑सा॒ तेज॑सा ऽऽहाह॒ तेज॑सै॒वैव तेज॑सा ऽऽहाह॒ तेज॑सै॒व । \newline
18. तेज॑सै॒ वैव तेज॑सा॒ तेज॑सै॒ वैन॑ मेन मे॒व तेज॑सा॒ तेज॑सै॒ वैन᳚म् । \newline
19. ए॒वैन॑ मेन मे॒वै वैन॑ मनक् त्यनक् त्येन मे॒वै वैन॑ मनक्ति । \newline
20. ए॒न॒ म॒न॒क् त्य॒न॒क् त्ये॒न॒ मे॒न॒ म॒न॒क् त्योष॑ध॒ ओष॑धे ऽनक्त्येन मेन मन॒क् त्योष॑धे । \newline
21. अ॒न॒क् त्योष॑ध॒ ओष॑धे ऽनक् त्यन॒क् त्योष॑धे॒ त्राय॑स्व॒ त्राय॒ स्वौष॑धे ऽनक् त्यन॒क् त्योष॑धे॒ त्राय॑स्व । \newline
22. ओष॑धे॒ त्राय॑स्व॒ त्राय॒ स्वौष॑ध॒ ओष॑धे॒ त्राय॑स्वैन मेन॒म् त्राय॒ स्वौष॑ध॒ ओष॑धे॒ त्राय॑स्वैनम् । \newline
23. त्राय॑स्वैन मेन॒म् त्राय॑स्व॒ त्राय॑स्वैनꣳ॒॒ स्वधि॑ते॒ स्वधि॑त एन॒म् त्राय॑स्व॒ त्राय॑स्वैनꣳ॒॒ स्वधि॑ते । \newline
24. ए॒नꣳ॒॒ स्वधि॑ते॒ स्वधि॑त एन मेनꣳ॒॒ स्वधि॑ते॒ मा मा स्वधि॑त एन मेनꣳ॒॒ स्वधि॑ते॒ मा । \newline
25. स्वधि॑ते॒ मा मा स्वधि॑ते॒ स्वधि॑ते॒ मैन॑ मेन॒म् मा स्वधि॑ते॒ स्वधि॑ते॒ मैन᳚म् । \newline
26. स्वधि॑त॒ इति॒ स्व - धि॒ते॒ । \newline
27. मैन॑ मेन॒म् मा मैनꣳ॑ हिꣳसीर्. हिꣳसी रेन॒म् मा मैनꣳ॑ हिꣳसीः । \newline
28. ए॒नꣳ॒॒ हिꣳ॒॒सी॒र्॒. हिꣳ॒॒सी॒ रे॒न॒ मे॒नꣳ॒॒ हिꣳ॒॒सी॒रि तीति॑ हिꣳसी रेन मेनꣳ हिꣳसी॒ रिति॑ । \newline
29. हिꣳ॒॒सी॒रि तीति॑ हिꣳसीर्. हिꣳसी॒रि त्या॑हा॒ हेति॑ हिꣳसीर्. हिꣳसी॒रि त्या॑ह । \newline
30. इत्या॑हा॒हे तीत्या॑ह॒ वज्रो॒ वज्र॑ आ॒हे तीत्या॑ह॒ वज्रः॑ । \newline
31. आ॒ह॒ वज्रो॒ वज्र॑ आहाह॒ वज्रो॒ वै वै वज्र॑ आहाह॒ वज्रो॒ वै । \newline
32. वज्रो॒ वै वै वज्रो॒ वज्रो॒ वै स्वधि॑तिः॒ स्वधि॑ति॒र् वै वज्रो॒ वज्रो॒ वै स्वधि॑तिः । \newline
33. वै स्वधि॑तिः॒ स्वधि॑ति॒र् वै वै स्वधि॑तिः॒ शान्त्यै॒ शान्त्यै॒ स्वधि॑ति॒र् वै वै स्वधि॑तिः॒ शान्त्यै᳚ । \newline
34. स्वधि॑तिः॒ शान्त्यै॒ शान्त्यै॒ स्वधि॑तिः॒ स्वधि॑तिः॒ शान्त्यै॒ स्वधि॑तेः॒ स्वधि॑तेः॒ शान्त्यै॒ स्वधि॑तिः॒ स्वधि॑तिः॒ शान्त्यै॒ स्वधि॑तेः । \newline
35. स्वधि॑ति॒रिति॒ स्व - धि॒तिः॒ । \newline
36. शान्त्यै॒ स्वधि॑तेः॒ स्वधि॑तेः॒ शान्त्यै॒ शान्त्यै॒ स्वधि॑तेर् वृ॒क्षस्य॑ वृ॒क्षस्य॒ स्वधि॑तेः॒ शान्त्यै॒ शान्त्यै॒ स्वधि॑तेर् वृ॒क्षस्य॑ । \newline
37. स्वधि॑तेर् वृ॒क्षस्य॑ वृ॒क्षस्य॒ स्वधि॑तेः॒ स्वधि॑तेर् वृ॒क्षस्य॒ बिभ्य॑तो॒ बिभ्य॑तो वृ॒क्षस्य॒ स्वधि॑तेः॒ स्वधि॑तेर् वृ॒क्षस्य॒ बिभ्य॑तः । \newline
38. स्वधि॑ते॒रिति॒ स्व - धि॒तेः॒ । \newline
39. वृ॒क्षस्य॒ बिभ्य॑तो॒ बिभ्य॑तो वृ॒क्षस्य॑ वृ॒क्षस्य॒ बिभ्य॑तः प्रथ॒मेन॑ प्रथ॒मेन॒ बिभ्य॑तो वृ॒क्षस्य॑ वृ॒क्षस्य॒ बिभ्य॑तः प्रथ॒मेन॑ । \newline
40. बिभ्य॑तः प्रथ॒मेन॑ प्रथ॒मेन॒ बिभ्य॑तो॒ बिभ्य॑तः प्रथ॒मेन॒ शक॑लेन॒ शक॑लेन प्रथ॒मेन॒ बिभ्य॑तो॒ बिभ्य॑तः प्रथ॒मेन॒ शक॑लेन । \newline
41. प्र॒थ॒मेन॒ शक॑लेन॒ शक॑लेन प्रथ॒मेन॑ प्रथ॒मेन॒ शक॑लेन स॒ह स॒ह शक॑लेन प्रथ॒मेन॑ प्रथ॒मेन॒ शक॑लेन स॒ह । \newline
42. शक॑लेन स॒ह स॒ह शक॑लेन॒ शक॑लेन स॒ह तेज॒ स्तेजः॑ स॒ह शक॑लेन॒ शक॑लेन स॒ह तेजः॑ । \newline
43. स॒ह तेज॒ स्तेजः॑ स॒ह स॒ह तेजः॒ परा॒ परा॒ तेजः॑ स॒ह स॒ह तेजः॒ परा᳚ । \newline
44. तेजः॒ परा॒ परा॒ तेज॒ स्तेजः॒ परा॑ पतति पतति॒ परा॒ तेज॒ स्तेजः॒ परा॑ पतति । \newline
45. परा॑ पतति पतति॒ परा॒ परा॑ पतति॒ यो यः प॑तति॒ परा॒ परा॑ पतति॒ यः । \newline
46. प॒त॒ति॒ यो यः प॑तति पतति॒ यः प्र॑थ॒मः प्र॑थ॒मो यः प॑तति पतति॒ यः प्र॑थ॒मः । \newline
47. यः प्र॑थ॒मः प्र॑थ॒मो यो यः प्र॑थ॒मः शक॑लः॒ शक॑लः प्रथ॒मो यो यः प्र॑थ॒मः शक॑लः । \newline
48. प्र॒थ॒मः शक॑लः॒ शक॑लः प्रथ॒मः प्र॑थ॒मः शक॑लः परा॒पते᳚त् परा॒पते॒च् छक॑लः प्रथ॒मः प्र॑थ॒मः शक॑लः परा॒पते᳚त् । \newline
49. शक॑लः परा॒पते᳚त् परा॒पते॒च् छक॑लः॒ शक॑लः परा॒पते॒त् तम् तम् प॑रा॒पते॒च् छक॑लः॒ शक॑लः परा॒पते॒त् तम् । \newline
50. प॒रा॒पते॒त् तम् तम् प॑रा॒पते᳚त् परा॒पते॒त् त मप्यपि॒ तम् प॑रा॒पते᳚त् परा॒पते॒त् त मपि॑ । \newline
51. प॒रा॒पते॒दिति॑ परा - पते᳚त् । \newline
52. त मप्यपि॒ तम् त मप्या ऽपि॒ तम् त मप्या । \newline
53. अप्या ऽप्यप्या ह॑रे द्धरे॒दा ऽप्यप्या ह॑रेत् । \newline
54. आ ह॑रे द्धरे॒दा ह॑रे॒थ् सते॑जसꣳ॒॒ सते॑जसꣳ हरे॒दा ह॑रे॒थ् सते॑जसम् । \newline
55. ह॒रे॒थ् सते॑जसꣳ॒॒ सते॑जसꣳ हरे द्धरे॒थ् सते॑जस मे॒वैव सते॑जसꣳ हरे द्धरे॒थ् सते॑जस मे॒व । \newline
56. सते॑जस मे॒वैव सते॑जसꣳ॒॒ सते॑जस मे॒वैन॑ मेन मे॒व सते॑जसꣳ॒॒ सते॑जस मे॒वैन᳚म् । \newline
57. सते॑जस॒मिति॒ स - ते॒ज॒स॒म् । \newline
\pagebreak
\markright{ TS 6.3.3.3  \hfill https://www.vedavms.in \hfill}

\section{ TS 6.3.3.3 }

\textbf{TS 6.3.3.3 } \newline
\textbf{Samhita Paata} \newline

-मे॒वैन॒मा ह॑रती॒मे वै लो॒का यूपा᳚त् प्रय॒तो बि॑भ्यति॒ दिव॒मग्रे॑ण॒ मा ले॑खीर॒न्तरि॑क्षं॒ मद्ध्ये॑न॒ मा हिꣳ॑सी॒रित्या॑है॒भ्य ए॒वैनं॑ ॅलो॒केभ्यः॑ शमयति॒ वन॑स्पते श॒तव॑ल्.शो॒ वि रो॒हेत्या॒व्रश्च॑ने जुहोति॒ तस्मा॑-दा॒व्रश्च॑नाद्-वृ॒क्षाणां॒ भूयाꣳ॑स॒ उत्ति॑ष्ठन्ति स॒हस्र॑वल्.शा॒ वि व॒यꣳ रु॑हे॒मेत्या॑हा॒- ऽऽ*शिष॑मे॒वैतामा शा॒स्ते ऽन॑क्षसङ्गं- [  ] \newline

\textbf{Pada Paata} \newline

ए॒व । ए॒न॒म् । एति॑ । ह॒र॒ति॒ । इ॒मे । वै । लो॒काः । यूपा᳚त् । प्र॒य॒त इति॑ प्र - य॒तः । बि॒भ्य॒ति॒ । दिव᳚म् । अग्रे॑ण । मा । ले॒खीः॒ । अ॒न्तरि॑क्षम् । मद्ध्ये॑न । मा । हिꣳ॒॒सीः॒ । इति॑ । आ॒ह॒ । ए॒भ्यः । ए॒व । ए॒न॒म् । लो॒केभ्यः॑ । श॒म॒य॒ति॒ । वन॑स्पते । श॒तव॑ल्.श॒ इति॑ श॒त - व॒ल॒.शः॒ । वीति॑ । रो॒ह॒ । इति॑ । आ॒व्रश्च॑न॒ इत्या᳚ - व्रश्च॑ने । जु॒हो॒ति॒ । तस्मा᳚त् । आ॒व्रश्च॑ना॒दित्या᳚ - व्रश्च॑नात् । वृ॒क्षाणा᳚म् । भूयाꣳ॑सः । उदिति॑ । ति॒ष्ठ॒न्ति॒ । स॒हस्र॑वल्.शा॒ इति॑ स॒हस्र॑ - व॒ल॒.शाः॒ । वीति॑ । व॒यम् । रु॒हे॒म॒ । इति॑ । आ॒ह॒ । आ॒शिष॒मित्या᳚-शिष᳚म् । ए॒व । ए॒ताम् । एति॑ । शा॒स्ते॒ । अन॑क्षसङ्ग॒मित्यन॑क्ष - स॒ङ्गम् ।  \newline


\textbf{Krama Paata} \newline

ए॒वैन᳚म् । ए॒न॒मा । आ ह॑रति । ह॒र॒ती॒मे । इ॒मे वै । वै लो॒काः । लो॒का यूपा᳚त् । यूपा᳚त् प्रय॒तः । प्र॒य॒तो बि॑भ्यति । प्र॒य॒त इति॑ प्र - य॒तः । बि॒भ्य॒ति॒ दिव᳚म् । दिव॒मग्रे॑ण । अग्रे॑ण॒ मा । मा ले॑खीः । ले॒खी॒र॒न्तरि॑क्षम् । अ॒न्तरि॑क्ष॒म् मद्ध्ये॑न । मद्ध्ये॑न॒ मा । मा हिꣳ॑सीः । हिꣳ॒॒सी॒रिति॑ । इत्या॑ह । आ॒है॒भ्यः । ए॒भ्य ए॒व । ए॒वैन᳚म् । ए॒न॒म् ॅलो॒केभ्यः॑ । लो॒केभ्यः॑ शमयति । श॒म॒य॒ति॒ वन॑स्पते । वन॑स्पते श॒तव॑ल्.शः । श॒तव॑ल्.शो॒ वि । श॒तव॑ल्.श॒ इति॑ श॒त - व॒ल्.॒शः॒ । वि रो॑ह । रो॒हेति॑ । इत्या॒व्रश्च॑ने । आ॒व्रश्च॑ने जुहोति । आ॒व्रश्च॑न॒ इत्या᳚ - व्रश्च॑ने । जु॒हो॒ति॒ तस्मा᳚त् । तस्मा॑दा॒व्रश्च॑नात् । आ॒व्रश्च॑नाद् वृ॒क्षाणा᳚म् । आ॒व्रश्च॑ना॒दित्या᳚ - व्रश्च॑नात् । वृ॒क्षाणा॒म् भूयाꣳ॑सः । भूयाꣳ॑स॒ उत् । उत् ति॑ष्ठन्ति । ति॒ष्ठ॒न्ति॒ स॒हस्र॑वल्.शाः । स॒हस्र॑वल्.शा॒ वि । स॒हस्र॑वल्.शा॒ इति॑ स॒हस्र॑ - व॒ल्॒.शाः॒ । वि व॒यम् । व॒यꣳ रु॑हेम । रु॒हे॒मेति॑ । इत्या॑ह । आ॒हा॒शिष᳚म् । आ॒शिष॑मे॒व । आ॒शिष॒मित्या᳚ - शिष᳚म् । ए॒वैताम् । ए॒तामा । आ शा᳚स्ते । शा॒स्तेऽन॑क्षसङ्‍गम् । अन॑क्षसङ्‍गम् ॅवृश्चेत् । अन॑क्षसङ्‍ग॒मित्यन॑क्ष - स॒ङ्‍ग॒म् \newline

\textbf{Jatai Paata} \newline

1. ए॒वैन॑ मेन मे॒वै वैन᳚म् । \newline
2. ए॒न॒ मैन॑ मेन॒ मा । \newline
3. आ ह॑रति हर॒त्या ह॑रति । \newline
4. ह॒र॒ ती॒म इ॒मे ह॑रति हर ती॒मे । \newline
5. इ॒मे वै वा इ॒म इ॒मे वै । \newline
6. वै लो॒का लो॒का वै वै लो॒काः । \newline
7. लो॒का यूपा॒द् यूपा᳚ ल्लो॒का लो॒का यूपा᳚त् । \newline
8. यूपा᳚त् प्रय॒तः प्र॑य॒तो यूपा॒द् यूपा᳚त् प्रय॒तः । \newline
9. प्र॒य॒तो बि॑भ्यति बिभ्यति प्रय॒तः प्र॑य॒तो बि॑भ्यति । \newline
10. प्र॒य॒त इति॑ प्र - य॒तः । \newline
11. बि॒भ्य॒ति॒ दिव॒म् दिव॑म् बिभ्यति बिभ्यति॒ दिव᳚म् । \newline
12. दिव॒ मग्रे॒णा ग्रे॑ण॒ दिव॒म् दिव॒ मग्रे॑ण । \newline
13. अग्रे॑ण॒ मा मा ऽग्रे॒णा ग्रे॑ण॒ मा । \newline
14. मा ले॑खीर् लेखी॒र् मा मा ले॑खीः । \newline
15. ले॒खी॒ र॒न्तरि॑क्ष म॒न्तरि॑क्षम् ॅलेखीर् लेखी र॒न्तरि॑क्षम् । \newline
16. अ॒न्तरि॑क्ष॒म् मद्ध्ये॑न॒ मद्ध्ये॑ना॒ न्तरि॑क्ष म॒न्तरि॑क्ष॒म् मद्ध्ये॑न । \newline
17. मद्ध्ये॑न॒ मा मा मद्ध्ये॑न॒ मद्ध्ये॑न॒ मा । \newline
18. मा हिꣳ॑सीर्. हिꣳसी॒र् मा मा हिꣳ॑सीः । \newline
19. हिꣳ॒॒सी॒रि तीति॑ हिꣳसीर्. हिꣳसी॒रिति॑ । \newline
20. इत्या॑हा॒हे तीत्या॑ह । \newline
21. आ॒है॒भ्य ए॒भ्य आ॑हा है॒भ्यः । \newline
22. ए॒भ्य ए॒वै वैभ्य ए॒भ्य ए॒व । \newline
23. ए॒वैन॑ मेन मे॒वै वैन᳚म् । \newline
24. ए॒न॒म् ॅलो॒केभ्यो॑ लो॒केभ्य॑ एन मेनम् ॅलो॒केभ्यः॑ । \newline
25. लो॒केभ्यः॑ शमयति शमयति लो॒केभ्यो॑ लो॒केभ्यः॑ शमयति । \newline
26. श॒म॒य॒ति॒ वन॑स्पते॒ वन॑स्पते शमयति शमयति॒ वन॑स्पते । \newline
27. वन॑स्पते श॒तव॑ल्.शः श॒तव॑ल्.शो॒ वन॑स्पते॒ वन॑स्पते श॒तव॑ल्.शः । \newline
28. श॒तव॑ल्.शो॒ वि वि श॒तव॑ल्.शः श॒तव॑ल्.शो॒ वि । \newline
29. श॒तव॑ल्.श॒ इति॑ श॒त - व॒ल्॒.शः॒ । \newline
30. वि रो॑ह रोह॒ वि वि रो॑ह । \newline
31. रो॒हे तीति॑ रोह रो॒हेति॑ । \newline
32. इत्या॒व्रश्च॑न आ॒व्रश्च॑न॒ इती त्या॒व्रश्च॑ने । \newline
33. आ॒व्रश्च॑ने जुहोति जुहो त्या॒व्रश्च॑न आ॒व्रश्च॑ने जुहोति । \newline
34. आ॒व्रश्च॑न॒ इत्या᳚ - व्रश्च॑ने । \newline
35. जु॒हो॒ति॒ तस्मा॒त् तस्मा᳚ज् जुहोति जुहोति॒ तस्मा᳚त् । \newline
36. तस्मा॑ दा॒व्रश्च॑ना दा॒व्रश्च॑ना॒त् तस्मा॒त् तस्मा॑ दा॒व्रश्च॑नात् । \newline
37. आ॒व्रश्च॑नाद् वृ॒क्षाणां᳚ ॅवृ॒क्षाणा॑ मा॒व्रश्च॑ना दा॒व्रश्च॑नाद् वृ॒क्षाणा᳚म् । \newline
38. आ॒व्रश्च॑ना॒दित्या᳚ - व्रश्च॑नात् । \newline
39. वृ॒क्षाणा॒म् भूयाꣳ॑सो॒ भूयाꣳ॑सो वृ॒क्षाणां᳚ ॅवृ॒क्षाणा॒म् भूयाꣳ॑सः । \newline
40. भूयाꣳ॑स॒ उदुद् भूयाꣳ॑सो॒ भूयाꣳ॑स॒ उत् । \newline
41. उत् ति॑ष्ठन्ति तिष्ठ॒न् त्युदुत् ति॑ष्ठन्ति । \newline
42. ति॒ष्ठ॒न्ति॒ स॒हस्र॑वल्.शाः स॒हस्र॑वल्.शा स्तिष्ठन्ति तिष्ठन्ति स॒हस्र॑वल्.शाः । \newline
43. स॒हस्र॑वल्.शा॒ वि वि स॒हस्र॑वल्.शाः स॒हस्र॑वल्.शा॒ वि । \newline
44. स॒हस्र॑वल्.शा॒ इति॑ स॒हस्र॑ - व॒ल्॒.शाः॒ । \newline
45. वि व॒यं ॅव॒यं ॅवि वि व॒यम् । \newline
46. व॒यꣳ रु॑हेम रुहेम व॒यं ॅव॒यꣳ रु॑हेम । \newline
47. रु॒हे॒मे तीति॑ रुहेम रुहे॒मेति॑ । \newline
48. इत्या॑हा॒हे तीत्या॑ह । \newline
49. आ॒हा॒ शिष॑ मा॒शिष॑ माहा हा॒शिष᳚म् । \newline
50. आ॒शिष॑ मे॒वै वाशिष॑ मा॒शिष॑ मे॒व । \newline
51. आ॒शिष॒मित्या᳚ - शिष᳚म् । \newline
52. ए॒वैता मे॒ता मे॒वै वैताम् । \newline
53. ए॒ता मैता मे॒ता मा । \newline
54. आ शा᳚स्ते शास्त॒ आ शा᳚स्ते । \newline
55. शा॒स्ते ऽन॑क्षसङ्ग॒ मन॑क्षसङ्गꣳ शास्ते शा॒स्ते ऽन॑क्षसङ्गम् । \newline
56. अन॑क्षसङ्गं ॅवृश्चेद् वृश्चे॒ दन॑क्षसङ्ग॒ मन॑क्षसङ्गं ॅवृश्चेत् । \newline
57. अन॑क्षसङ्ग॒मित्यन॑क्ष - स॒ङ्ग॒म् । \newline

\textbf{Ghana Paata } \newline

1. ए॒वैन॑ मेन मे॒वै वैन॒ मैन॑ मे॒वै वैन॒ मा । \newline
2. ए॒न॒ मैन॑ मेन॒ मा ह॑रति हर॒ त्यैन॑ मेन॒ मा ह॑रति । \newline
3. आ ह॑रति हर॒त्या ह॑र ती॒म इ॒मे ह॑र॒त्या ह॑र ती॒मे । \newline
4. ह॒र॒ ती॒म इ॒मे ह॑रति हर ती॒मे वै वा इ॒मे ह॑रति हर ती॒मे वै । \newline
5. इ॒मे वै वा इ॒म इ॒मे वै लो॒का लो॒का वा इ॒म इ॒मे वै लो॒काः । \newline
6. वै लो॒का लो॒का वै वै लो॒का यूपा॒द् यूपा᳚ ल्लो॒का वै वै लो॒का यूपा᳚त् । \newline
7. लो॒का यूपा॒द् यूपा᳚ ल्लो॒का लो॒का यूपा᳚त् प्रय॒तः प्र॑य॒तो यूपा᳚ ल्लो॒का लो॒का यूपा᳚त् प्रय॒तः । \newline
8. यूपा᳚त् प्रय॒तः प्र॑य॒तो यूपा॒द् यूपा᳚त् प्रय॒तो बि॑भ्यति बिभ्यति प्रय॒तो यूपा॒द् यूपा᳚त् प्रय॒तो बि॑भ्यति । \newline
9. प्र॒य॒तो बि॑भ्यति बिभ्यति प्रय॒तः प्र॑य॒तो बि॑भ्यति॒ दिव॒म् दिव॑म् बिभ्यति प्रय॒तः प्र॑य॒तो बि॑भ्यति॒ दिव᳚म् । \newline
10. प्र॒य॒त इति॑ प्र - य॒तः । \newline
11. बि॒भ्य॒ति॒ दिव॒म् दिव॑म् बिभ्यति बिभ्यति॒ दिव॒ मग्रे॒णा ग्रे॑ण॒ दिव॑म् बिभ्यति बिभ्यति॒ दिव॒ मग्रे॑ण । \newline
12. दिव॒ मग्रे॒णा ग्रे॑ण॒ दिव॒म् दिव॒ मग्रे॑ण॒ मा मा ऽग्रे॑ण॒ दिव॒म् दिव॒ मग्रे॑ण॒ मा । \newline
13. अग्रे॑ण॒ मा मा ऽग्रे॒णा ग्रे॑ण॒ मा ले॑खीर् लेखी॒र् मा ऽग्रे॒णा ग्रे॑ण॒ मा ले॑खीः । \newline
14. मा ले॑खीर् लेखी॒र् मा मा ले॑खी र॒न्तरि॑क्ष म॒न्तरि॑क्षम् ॅलेखी॒र् मा मा ले॑खी र॒न्तरि॑क्षम् । \newline
15. ले॒खी॒ र॒न्तरि॑क्ष म॒न्तरि॑क्षम् ॅलेखीर् लेखी र॒न्तरि॑क्ष॒म् मद्ध्ये॑न॒ मद्ध्ये॑ना॒ न्तरि॑क्षम् ॅलेखीर् लेखी र॒न्तरि॑क्ष॒म् मद्ध्ये॑न । \newline
16. अ॒न्तरि॑क्ष॒म् मद्ध्ये॑न॒ मद्ध्ये॑ना॒ न्तरि॑क्ष म॒न्तरि॑क्ष॒म् मद्ध्ये॑न॒ मा मा मद्ध्ये॑ना॒ न्तरि॑क्ष म॒न्तरि॑क्ष॒म् मद्ध्ये॑न॒ मा । \newline
17. मद्ध्ये॑न॒ मा मा मद्ध्ये॑न॒ मद्ध्ये॑न॒ मा हिꣳ॑सीर्. हिꣳसी॒र् मा मद्ध्ये॑न॒ मद्ध्ये॑न॒ मा हिꣳ॑सीः । \newline
18. मा हिꣳ॑सीर्. हिꣳसी॒र् मा मा हिꣳ॑सी॒ रितीति॑ हिꣳसी॒र् मा मा हिꣳ॑सी॒ रिति॑ । \newline
19. हिꣳ॒॒सी॒रि तीति॑ हिꣳसीर्. हिꣳसी॒रि त्या॑हा॒ हेति॑ हिꣳसीर्. हिꣳसी॒रि त्या॑ह । \newline
20. इत्या॑हा॒हे तीत्या॑ है॒भ्य ए॒भ्य आ॒हे तीत्या॑ है॒भ्यः । \newline
21. आ॒है॒भ्य ए॒भ्य आ॑हा है॒भ्य ए॒वै वैभ्य आ॑हा है॒भ्य ए॒व । \newline
22. ए॒भ्य ए॒वै वैभ्य ए॒भ्य ए॒वैन॑ मेन मे॒वैभ्य ए॒भ्य ए॒वैन᳚म् । \newline
23. ए॒वैन॑ मेन मे॒वै वैन॑म् ॅलो॒केभ्यो॑ लो॒केभ्य॑ एन मे॒वै वैन॑म् ॅलो॒केभ्यः॑ । \newline
24. ए॒न॒म् ॅलो॒केभ्यो॑ लो॒केभ्य॑ एन मेनम् ॅलो॒केभ्यः॑ शमयति शमयति लो॒केभ्य॑ एन मेनम् ॅलो॒केभ्यः॑ शमयति । \newline
25. लो॒केभ्यः॑ शमयति शमयति लो॒केभ्यो॑ लो॒केभ्यः॑ शमयति॒ वन॑स्पते॒ वन॑स्पते शमयति लो॒केभ्यो॑ लो॒केभ्यः॑ शमयति॒ वन॑स्पते । \newline
26. श॒म॒य॒ति॒ वन॑स्पते॒ वन॑स्पते शमयति शमयति॒ वन॑स्पते श॒तव॑ल्.शः श॒तव॑ल्.शो॒ वन॑स्पते शमयति शमयति॒ वन॑स्पते श॒तव॑ल्.शः । \newline
27. वन॑स्पते श॒तव॑ल्.शः श॒तव॑ल्.शो॒ वन॑स्पते॒ वन॑स्पते श॒तव॑ल्.शो॒ वि वि श॒तव॑ल्.शो॒ वन॑स्पते॒ वन॑स्पते श॒तव॑ल्.शो॒ वि । \newline
28. श॒तव॑ल्.शो॒ वि वि श॒तव॑ल्.शः श॒तव॑ल्.शो॒ वि रो॑ह रोह॒ वि श॒तव॑ल्.शः श॒तव॑ल्.शो॒ वि रो॑ह । \newline
29. श॒तव॑ल्.श॒ इति॑ श॒त - व॒ल्॒.शः॒ । \newline
30. वि रो॑ह रोह॒ वि वि रो॒हे तीति॑ रोह॒ वि वि रो॒हेति॑ । \newline
31. रो॒हे तीति॑ रोह रो॒हे त्या॒व्रश्च॑न आ॒व्रश्च॑न॒ इति॑ रोह रो॒हे त्या॒व्रश्च॑ने । \newline
32. इत्या॒व्रश्च॑न आ॒व्रश्च॑न॒ इती त्या॒व्रश्च॑ने जुहोति जुहो त्या॒व्रश्च॑न॒ इती त्या॒व्रश्च॑ने जुहोति । \newline
33. आ॒व्रश्च॑ने जुहोति जुहो त्या॒व्रश्च॑न आ॒व्रश्च॑ने जुहोति॒ तस्मा॒त् तस्मा᳚ज् जुहो त्या॒व्रश्च॑न आ॒व्रश्च॑ने जुहोति॒ तस्मा᳚त् । \newline
34. आ॒व्रश्च॑न॒ इत्या᳚ - व्रश्च॑ने । \newline
35. जु॒हो॒ति॒ तस्मा॒त् तस्मा᳚ज् जुहोति जुहोति॒ तस्मा॑ दा॒व्रश्च॑ना दा॒व्रश्च॑ना॒त् तस्मा᳚ज् जुहोति जुहोति॒ तस्मा॑ दा॒व्रश्च॑नात् । \newline
36. तस्मा॑ दा॒व्रश्च॑ना दा॒व्रश्च॑ना॒त् तस्मा॒त् तस्मा॑ दा॒व्रश्च॑नाद् वृ॒क्षाणां᳚ ॅवृ॒क्षाणा॑ मा॒व्रश्च॑ना॒त् तस्मा॒त् तस्मा॑ दा॒व्रश्च॑नाद् वृ॒क्षाणा᳚म् । \newline
37. आ॒व्रश्च॑नाद् वृ॒क्षाणां᳚ ॅवृ॒क्षाणा॑ मा॒व्रश्च॑ना दा॒व्रश्च॑नाद् वृ॒क्षाणा॒म् भूयाꣳ॑सो॒ भूयाꣳ॑सो वृ॒क्षाणा॑ मा॒व्रश्च॑ना दा॒व्रश्च॑नाद् वृ॒क्षाणा॒म् भूयाꣳ॑सः । \newline
38. आ॒व्रश्च॑ना॒दित्या᳚ - व्रश्च॑नात् । \newline
39. वृ॒क्षाणा॒म् भूयाꣳ॑सो॒ भूयाꣳ॑सो वृ॒क्षाणां᳚ ॅवृ॒क्षाणा॒म् भूयाꣳ॑स॒ उदुद् भूयाꣳ॑सो वृ॒क्षाणां᳚ ॅवृ॒क्षाणा॒म् भूयाꣳ॑स॒ उत् । \newline
40. भूयाꣳ॑स॒ उदुद् भूयाꣳ॑सो॒ भूयाꣳ॑स॒ उत् ति॑ष्ठन्ति तिष्ठ॒ न्त्युद् भूयाꣳ॑सो॒ भूयाꣳ॑स॒ उत् ति॑ष्ठन्ति । \newline
41. उत् ति॑ष्ठन्ति तिष्ठ॒ न्त्युदुत् ति॑ष्ठन्ति स॒हस्र॑वल्.शाः स॒हस्र॑वल्.शा स्तिष्ठ॒ न्त्युदुत् ति॑ष्ठन्ति स॒हस्र॑वल्.शाः । \newline
42. ति॒ष्ठ॒न्ति॒ स॒हस्र॑वल्.शाः स॒हस्र॑वल्.शा स्तिष्ठन्ति तिष्ठन्ति स॒हस्र॑वल्.शा॒ वि वि स॒हस्र॑वल्.शा स्तिष्ठन्ति तिष्ठन्ति स॒हस्र॑वल्.शा॒ वि । \newline
43. स॒हस्र॑वल्.शा॒ वि वि स॒हस्र॑वल्.शाः स॒हस्र॑वल्.शा॒ वि व॒यं ॅव॒यं ॅवि स॒हस्र॑वल्.शाः स॒हस्र॑वल्.शा॒ वि व॒यम् । \newline
44. स॒हस्र॑वल्.शा॒ इति॑ स॒हस्र॑ - व॒ल्॒.शाः॒ । \newline
45. वि व॒यं ॅव॒यं ॅवि वि व॒यꣳ रु॑हेम रुहेम व॒यं ॅवि वि व॒यꣳ रु॑हेम । \newline
46. व॒यꣳ रु॑हेम रुहेम व॒यं ॅव॒यꣳ रु॑हे॒मे तीति॑ रुहेम व॒यं ॅव॒यꣳ रु॑हे॒मेति॑ । \newline
47. रु॒हे॒मे तीति॑ रुहेम रुहे॒मे त्या॑हा॒ हेति॑ रुहेम रुहे॒मे त्या॑ह । \newline
48. इत्या॑हा॒हे तीत्या॑ हा॒शिष॑ मा॒शिष॑ मा॒हे तीत्या॑ हा॒शिष᳚म् । \newline
49. आ॒हा॒शिष॑ मा॒शिष॑ माहा हा॒शिष॑ मे॒वै वाशिष॑ माहा हा॒शिष॑ मे॒व । \newline
50. आ॒शिष॑ मे॒वै वाशिष॑ मा॒शिष॑ मे॒वैता मे॒ता मे॒वाशिष॑ मा॒शिष॑ मे॒वैताम् । \newline
51. आ॒शिष॒मित्या᳚ - शिष᳚म् । \newline
52. ए॒वैता मे॒ता मे॒वै वैता मैता मे॒वै वैता मा । \newline
53. ए॒ता मैता मे॒ता मा शा᳚स्ते शास्त॒ ऐता मे॒ता मा शा᳚स्ते । \newline
54. आ शा᳚स्ते शास्त॒ आ शा॒स्ते ऽन॑क्षसङ्ग॒ मन॑क्षसङ्गꣳ शास्त॒ आ शा॒स्ते ऽन॑क्षसङ्गम् । \newline
55. शा॒स्ते ऽन॑क्षसङ्ग॒ मन॑क्षसङ्गꣳ शास्ते शा॒स्ते ऽन॑क्षसङ्गं ॅवृश्चेद् वृश्चे॒ दन॑क्षसङ्गꣳ शास्ते शा॒स्ते ऽन॑क्षसङ्गं ॅवृश्चेत् । \newline
56. अन॑क्षसङ्गं ॅवृश्चेद् वृश्चे॒ दन॑क्षसङ्ग॒ मन॑क्षसङ्गं ॅवृश्चे॒द् यद् यद् वृ॑श्चे॒ दन॑क्षसङ्ग॒ मन॑क्षसङ्गं ॅवृश्चे॒द् यत् । \newline
57. अन॑क्षसङ्ग॒मित्यन॑क्ष - स॒ङ्ग॒म् । \newline
\pagebreak
\markright{ TS 6.3.3.4  \hfill https://www.vedavms.in \hfill}

\section{ TS 6.3.3.4 }

\textbf{TS 6.3.3.4 } \newline
\textbf{Samhita Paata} \newline

ॅवृश्चे॒द्-यद॑क्षस॒ङ्गं ॅवृ॒श्चेद॑धई॒षं ॅयज॑मानस्य प्र॒मायु॑कꣳ स्या॒द्यं का॒मये॒ताप्र॑तिष्ठितः स्या॒दित्या॑रो॒हं तस्म॑ वृश्चेदे॒ष वै वन॒स्पती॑ना॒-मप्र॑तिष्ठि॒तोऽप्र॑तिष्ठित ए॒व भ॑वति॒ यं का॒मये॑ताप॒शुः स्या॒दित्य॑प॒र्णं तस्मै॒ शुष्का᳚ग्रं ॅवृश्चेदे॒ष वै वन॒स्पती॑ना-मपश॒व्यो॑ऽप॒शुरे॒व भ॑वति॒ यं का॒मये॑त पशु॒मान्थ् स्या॒दिति॑ बहुप॒र्णं तस्मै॑ बहुशा॒खं ॅवृ॑श्चेदे॒ष वै- [  ] \newline

\textbf{Pada Paata} \newline

वृ॒श्चे॒त् । यत् । अ॒क्ष॒स॒ङ्गमित्य॑क्ष - स॒ङ्गम् । वृ॒श्चेत् । अ॒ध॒ई॒षमित्य॑धः - ई॒षम् । यज॑मानस्य । प्र॒मायु॑क॒मिति॑ प्र-मायु॑कम् । स्या॒त् । यम् । का॒मये॑त । अप्र॑तिष्ठित॒ इत्यप्र॑ति - स्थि॒तः॒ । स्या॒त् । इति॑ । आ॒रो॒हमित्या᳚ - रो॒हम् । तस्मै᳚ । वृ॒श्चे॒त् । ए॒षः । वै । वन॒स्पती॑नाम् । अप्र॑तिष्ठित॒ इत्यप्र॑ति - स्थि॒तः॒ । अप्र॑तिष्ठित॒ इत्यप्र॑ति - स्थि॒तः॒ । ए॒व । भ॒व॒ति॒ । यम् । का॒मये॑त । अ॒प॒शुः । स्या॒त् । इति॑ । अ॒प॒र्णम् । तस्मै᳚ । शुष्का᳚ग्र॒मिति॒ शुष्क॑ - अ॒ग्र॒म् । वृ॒श्चे॒त् । ए॒षः । वै । वन॒स्पती॑नाम् । अ॒प॒श॒व्यः । अ॒प॒शुः । ए॒व । भ॒व॒ति॒ । यम् । का॒मये॑त । प॒शु॒मानिति॑ पशु - मान् । स्या॒त् । इति॑ । ब॒हु॒प॒र्णमिति॑ बहु - प॒र्णम् । तस्मै᳚ । ब॒हु॒शा॒खमिति॑ बहु - शा॒खम् । वृ॒श्चे॒त् । ए॒षः । वै ।  \newline


\textbf{Krama Paata} \newline

वृ॒श्चे॒द् यत् । यद॑क्षस॒ङ्‍गम् । अ॒क्ष॒स॒ङ्‍गम् ॅवृ॒श्चेत् । अ॒क्ष॒स॒ङ्‍गमित्य॑क्ष - स॒ङ्‍गम् । वृ॒श्चेद॑धई॒षम् । अ॒ध॒ई॒षम् ॅयज॑मानस्य । अ॒ध॒ई॒षमित्य॑धः - ई॒षम् । यज॑मानस्य प्र॒मायु॑कम् । प्र॒मायु॑कꣳ स्यात् । प्र॒मायु॑क॒मिति॑ प्र - मायु॑कम् । स्या॒द् यम् । यम् का॒मये॑त । का॒मये॒ताप्र॑तिष्ठितः । अप्र॑तिष्ठितः स्यात् । अप्र॑तिष्ठित॒ इत्यप्र॑ति - स्थि॒तः॒ । स्या॒दिति॑ । इत्या॑रो॒हम् । आ॒रो॒हम् तस्मै᳚ । आ॒रो॒हमित्या᳚ - रो॒हम् । तस्मै॑ वृश्चेत् । वृ॒श्चे॒दे॒षः । ए॒ष वै । वै वन॒स्पती॑नाम् । वन॒स्पती॑ना॒मप्र॑तिष्ठितः । अप्र॑तिष्ठि॒तोऽप्र॑तिष्ठितः । अप्र॑तिष्ठित॒ इत्यप्र॑ति - स्थि॒तः॒ । अप्र॑तिष्ठित ए॒व । अप्र॑तिष्ठित॒ इत्यप्र॑ति - स्थि॒तः॒ । ए॒व भ॑वति । भ॒व॒ति॒ यम् । यम् का॒मये॑त । का॒मये॑ताप॒शुः । अ॒प॒शुः स्या᳚त् । स्या॒दिति॑ । इत्य॑प॒र्णम् । अ॒प॒र्णम् तस्मै᳚ । तस्मै॒ शुष्का᳚ग्रम् । शुष्का᳚ग्रम् ॅवृश्चेत् । शुष्का᳚ग्र॒मिति॒ शुष्क॑ - अ॒ग्र॒म् । वृ॒श्चे॒दे॒षः । ए॒ष वै । वै वन॒स्पती॑नाम् । वन॒स्पती॑नामपश॒व्यः । अ॒प॒श॒व्यो॑ऽप॒शुः । अ॒प॒शुरे॒व । ए॒व भ॑वति । भ॒व॒ति॒ यम् । यम् का॒मये॑त । का॒मये॑त पशु॒मान् । प॒शु॒मान्थ् स्या᳚त् । प॒शु॒मानिति॑ पशु - मान् । स्या॒दिति॑ । इति॑ बहुप॒र्णम् । ब॒हु॒प॒र्णम् तस्मै᳚ । ब॒हु॒प॒र्णमिति॑ बहु - प॒र्णम् । तस्मै॑ बहुशा॒खम् । ब॒हु॒शा॒खम् ॅवृ॑श्चेत् । ब॒हु॒शा॒खमिति॑ बहु - शा॒खम् । वृ॒श्चे॒दे॒षः । ए॒ष वै । वै वन॒स्पती॑नाम् \newline

\textbf{Jatai Paata} \newline

1. वृ॒श्चे॒द् यद् यद् वृ॑श्चेद् वृश्चे॒द् यत् । \newline
2. यद॑क्षस॒ङ्ग म॑क्षस॒ङ्गं ॅयद् यद॑क्षस॒ङ्गम् । \newline
3. अ॒क्ष॒स॒ङ्गं ॅवृ॒श्चेद् वृ॒श्चे द॑क्षस॒ङ्ग म॑क्षस॒ङ्गं ॅवृ॒श्चेत् । \newline
4. अ॒क्ष॒स॒ङ्गमित्य॑क्ष - स॒ङ्गम् । \newline
5. वृ॒श्चेद॑ धई॒ष म॑धई॒षं ॅवृ॒श्चेद् वृ॒श्चे द॑धई॒षम् । \newline
6. अ॒ध॒ई॒षं ॅयज॑मानस्य॒ यज॑मानस्या धई॒ष म॑धई॒षं ॅयज॑मानस्य । \newline
7. अ॒ध॒ई॒षमित्य॑धः - ई॒षम् । \newline
8. यज॑मानस्य प्र॒मायु॑कम् प्र॒मायु॑कं॒ ॅयज॑मानस्य॒ यज॑मानस्य प्र॒मायु॑कम् । \newline
9. प्र॒मायु॑कꣳ स्याथ् स्यात् प्र॒मायु॑कम् प्र॒मायु॑कꣳ स्यात् । \newline
10. प्र॒मायु॑क॒मिति॑ प्र - मायु॑कम् । \newline
11. स्या॒द् यं ॅयꣳ स्या᳚थ् स्या॒द् यम् । \newline
12. यम् का॒मये॑त का॒मये॑त॒ यं ॅयम् का॒मये॑त । \newline
13. का॒मये॒ता प्र॑तिष्ठि॒तो ऽप्र॑तिष्ठितः का॒मये॑त का॒मये॒ता प्र॑तिष्ठितः । \newline
14. अप्र॑तिष्ठितः स्याथ् स्या॒ दप्र॑तिष्ठि॒तो ऽप्र॑तिष्ठितः स्यात् । \newline
15. अप्र॑तिष्ठित॒ इत्यप्र॑ति - स्थि॒तः॒ । \newline
16. स्या॒दि तीति॑ स्याथ् स्या॒ दिति॑ । \newline
17. इत्या॑ रो॒ह मा॑रो॒ह मिती त्या॑रो॒हम् । \newline
18. आ॒रो॒हम् तस्मै॒ तस्मा॑ आरो॒ह मा॑रो॒हम् तस्मै᳚ । \newline
19. आ॒रो॒हमित्या᳚ - रो॒हम् । \newline
20. तस्मै॑ वृश्चेद् वृश्चे॒त् तस्मै॒ तस्मै॑ वृश्चेत् । \newline
21. वृ॒श्चे॒ दे॒ष ए॒ष वृ॑श्चेद् वृश्चे दे॒षः । \newline
22. ए॒ष वै वा ए॒ष ए॒ष वै । \newline
23. वै वन॒स्पती॑नां॒ ॅवन॒स्पती॑नां॒ ॅवै वै वन॒स्पती॑नाम् । \newline
24. वन॒स्पती॑ना॒ मप्र॑तिष्ठि॒तो ऽप्र॑तिष्ठितो॒ वन॒स्पती॑नां॒ ॅवन॒स्पती॑ना॒ मप्र॑तिष्ठितः । \newline
25. अप्र॑तिष्ठि॒तो ऽप्र॑तिष्ठितः । \newline
26. अप्र॑तिष्ठित॒ इत्यप्र॑ति - स्थि॒तः॒ । \newline
27. अप्र॑तिष्ठित ए॒वैवाप्र॑तिष्ठि॒तो ऽप्र॑तिष्ठित ए॒व । \newline
28. अप्र॑तिष्ठित॒ इत्यप्र॑ति - स्थि॒तः॒ । \newline
29. ए॒व भ॑वति भव त्ये॒वैव भ॑वति । \newline
30. भ॒व॒ति॒ यं ॅयम् भ॑वति भवति॒ यम् । \newline
31. यम् का॒मये॑त का॒मये॑त॒ यं ॅयम् का॒मये॑त । \newline
32. का॒मये॑ता प॒शु र॑प॒शुः का॒मये॑त का॒मये॑ता प॒शुः । \newline
33. अ॒प॒शुः स्या᳚थ् स्या दप॒शु र॑प॒शुः स्या᳚त् । \newline
34. स्या॒दि तीति॑ स्याथ् स्या॒ दिति॑ । \newline
35. इत्य॑प॒र्ण म॑प॒र्ण मिती त्य॑प॒र्णम् । \newline
36. अ॒प॒र्णम् तस्मै॒ तस्मा॑ अप॒र्ण म॑प॒र्णम् तस्मै᳚ । \newline
37. तस्मै॒ शुष्का᳚ग्रꣳ॒॒ शुष्का᳚ग्र॒म् तस्मै॒ तस्मै॒ शुष्का᳚ग्रम् । \newline
38. शुष्का᳚ग्रं ॅवृश्चेद् वृश्चे॒ च्छुष्का᳚ग्रꣳ॒॒ शुष्का᳚ग्रं ॅवृश्चेत् । \newline
39. शुष्का᳚ग्र॒मिति॒ शुष्क॑ - अ॒ग्र॒म् । \newline
40. वृ॒श्चे॒ दे॒ष ए॒ष वृ॑श्चेद् वृश्चे दे॒षः । \newline
41. ए॒ष वै वा ए॒ष ए॒ष वै । \newline
42. वै वन॒स्पती॑नां॒ ॅवन॒स्पती॑नां॒ ॅवै वै वन॒स्पती॑नाम् । \newline
43. वन॒स्पती॑ना मपश॒व्यो॑ ऽपश॒व्यो वन॒स्पती॑नां॒ ॅवन॒स्पती॑ना मपश॒व्यः । \newline
44. अ॒प॒श॒व्यो॑ ऽप॒शु र॑प॒शु र॑पश॒व्यो॑ ऽपश॒व्यो॑ ऽप॒शुः । \newline
45. अ॒प॒शु रे॒वै वाप॒शु र॑प॒शु रे॒व । \newline
46. ए॒व भ॑वति भव त्ये॒वैव भ॑वति । \newline
47. भ॒व॒ति॒ यं ॅयम् भ॑वति भवति॒ यम् । \newline
48. यम् का॒मये॑त का॒मये॑त॒ यं ॅयम् का॒मये॑त । \newline
49. का॒मये॑त पशु॒मान् प॑शु॒मान् का॒मये॑त का॒मये॑त पशु॒मान् । \newline
50. प॒शु॒मान् थ्स्या᳚थ् स्यात् पशु॒मान् प॑शु॒मान् थ्स्या᳚त् । \newline
51. प॒शु॒मानिति॑ पशु - मान् । \newline
52. स्या॒दि तीति॑ स्याथ् स्या॒दिति॑ । \newline
53. इति॑ बहुप॒र्णम् ब॑हुप॒र्ण मितीति॑ बहुप॒र्णम् । \newline
54. ब॒हु॒प॒र्णम् तस्मै॒ तस्मै॑ बहुप॒र्णम् ब॑हुप॒र्णम् तस्मै᳚ । \newline
55. ब॒हु॒प॒र्णमिति॑ बहु - प॒र्णम् । \newline
56. तस्मै॑ बहुशा॒खम् ब॑हुशा॒खम् तस्मै॒ तस्मै॑ बहुशा॒खम् । \newline
57. ब॒हु॒शा॒खं ॅवृ॑श्चेद् वृश्चेद् बहुशा॒खम् ब॑हुशा॒खं ॅवृ॑श्चेत् । \newline
58. ब॒हु॒शा॒खमिति॑ बहु - शा॒खम् । \newline
59. वृ॒श्चे॒ दे॒ष ए॒ष वृ॑श्चेद् वृश्चे दे॒षः । \newline
60. ए॒ष वै वा ए॒ष ए॒ष वै । \newline
61. वै वन॒स्पती॑नां॒ ॅवन॒स्पती॑नां॒ ॅवै वै वन॒स्पती॑नाम् । \newline

\textbf{Ghana Paata } \newline

1. वृ॒श्चे॒द् यद् यद् वृ॑श्चेद् वृश्चे॒द् यद॑क्षस॒ङ्ग म॑क्षस॒ङ्गं ॅयद् वृ॑श्चेद् वृश्चे॒द् यद॑क्षस॒ङ्गम् । \newline
2. यद॑क्षस॒ङ्ग म॑क्षस॒ङ्गं ॅयद् यद॑क्षस॒ङ्गं ॅवृ॒श्चेद् वृ॒श्चे द॑क्षस॒ङ्गं ॅयद् यद॑क्षस॒ङ्गं ॅवृ॒श्चेत् । \newline
3. अ॒क्ष॒स॒ङ्गं ॅवृ॒श्चेद् वृ॒श्चे द॑क्षस॒ङ्ग म॑क्षस॒ङ्गं ॅवृ॒श्चे द॑धई॒ष म॑धई॒षं ॅवृ॒श्चे द॑क्षस॒ङ्ग म॑क्षस॒ङ्गं ॅवृ॒श्चे द॑धई॒षम् । \newline
4. अ॒क्ष॒स॒ङ्गमित्य॑क्ष - स॒ङ्गम् । \newline
5. वृ॒श्चे द॑धई॒ष म॑धई॒षं ॅवृ॒श्चेद् वृ॒श्चे द॑धई॒षं ॅयज॑मानस्य॒ यज॑मानस्या धई॒षं ॅवृ॒श्चेद् वृ॒श्चे द॑धई॒षं ॅयज॑मानस्य । \newline
6. अ॒ध॒ई॒षं ॅयज॑मानस्य॒ यज॑मानस्या धई॒ष म॑धई॒षं ॅयज॑मानस्य प्र॒मायु॑कम् प्र॒मायु॑कं॒ ॅयज॑मानस्या धई॒ष म॑धई॒षं ॅयज॑मानस्य प्र॒मायु॑कम् । \newline
7. अ॒ध॒ई॒षमित्य॑धः - ई॒षम् । \newline
8. यज॑मानस्य प्र॒मायु॑कम् प्र॒मायु॑कं॒ ॅयज॑मानस्य॒ यज॑मानस्य प्र॒मायु॑कꣳ स्याथ् स्यात् प्र॒मायु॑कं॒ ॅयज॑मानस्य॒ यज॑मानस्य प्र॒मायु॑कꣳ स्यात् । \newline
9. प्र॒मायु॑कꣳ स्याथ् स्यात् प्र॒मायु॑कम् प्र॒मायु॑कꣳ स्या॒द् यं ॅयꣳ स्या᳚त् प्र॒मायु॑कम् प्र॒मायु॑कꣳ स्या॒द् यम् । \newline
10. प्र॒मायु॑क॒मिति॑ प्र - मायु॑कम् । \newline
11. स्या॒द् यं ॅयꣳ स्या᳚थ् स्या॒द् यम् का॒मये॑त का॒मये॑त॒ यꣳ स्या᳚थ् स्या॒द् यम् का॒मये॑त । \newline
12. यम् का॒मये॑त का॒मये॑त॒ यं ॅयम् का॒मये॒ता प्र॑तिष्ठि॒तो ऽप्र॑तिष्ठितः का॒मये॑त॒ यं ॅयम् का॒मये॒ता प्र॑तिष्ठितः । \newline
13. का॒मये॒ता प्र॑तिष्ठि॒तो ऽप्र॑तिष्ठितः का॒मये॑त का॒मये॒ता प्र॑तिष्ठितः स्याथ् स्या॒ दप्र॑तिष्ठितः का॒मये॑त का॒मये॒ता प्र॑तिष्ठितः स्यात् । \newline
14. अप्र॑तिष्ठितः स्याथ् स्या॒ दप्र॑तिष्ठि॒तो ऽप्र॑तिष्ठितः स्या॒दितीति॑ स्या॒ दप्र॑तिष्ठि॒तो ऽप्र॑तिष्ठितः स्या॒दिति॑ । \newline
15. अप्र॑तिष्ठित॒ इत्यप्र॑ति - स्थि॒तः॒ । \newline
16. स्या॒ दितीति॑ स्याथ् स्या॒दि त्या॑रो॒ह मा॑रो॒ह मिति॑ स्याथ् स्या॒दि त्या॑रो॒हम् । \newline
17. इत्या॑रो॒ह मा॑रो॒ह मिती त्या॑रो॒हम् तस्मै॒ तस्मा॑ आरो॒ह मिती त्या॑रो॒हम् तस्मै᳚ । \newline
18. आ॒रो॒हम् तस्मै॒ तस्मा॑ आरो॒ह मा॑रो॒हम् तस्मै॑ वृश्चेद् वृश्चे॒त् तस्मा॑ आरो॒ह मा॑रो॒हम् तस्मै॑ वृश्चेत् । \newline
19. आ॒रो॒हमित्या᳚ - रो॒हम् । \newline
20. तस्मै॑ वृश्चेद् वृश्चे॒त् तस्मै॒ तस्मै॑ वृश्चे दे॒ष ए॒ष वृ॑श्चे॒त् तस्मै॒ तस्मै॑ वृश्चे दे॒षः । \newline
21. वृ॒श्चे॒ दे॒ष ए॒ष वृ॑श्चेद् वृश्चे दे॒ष वै वा ए॒ष वृ॑श्चेद् वृश्चे दे॒ष वै । \newline
22. ए॒ष वै वा ए॒ष ए॒ष वै वन॒स्पती॑नां॒ ॅवन॒स्पती॑नां॒ ॅवा ए॒ष ए॒ष वै वन॒स्पती॑नाम् । \newline
23. वै वन॒स्पती॑नां॒ ॅवन॒स्पती॑नां॒ ॅवै वै वन॒स्पती॑ना॒ मप्र॑तिष्ठि॒तो ऽप्र॑तिष्ठितो॒ वन॒स्पती॑नां॒ ॅवै वै वन॒स्पती॑ना॒ मप्र॑तिष्ठितः । \newline
24. वन॒स्पती॑ना॒ मप्र॑तिष्ठि॒तो ऽप्र॑तिष्ठितो॒ वन॒स्पती॑नां॒ ॅवन॒स्पती॑ना॒ मप्र॑तिष्ठितः । \newline
25. अप्र॑तिष्ठि॒तो ऽप्र॑तिष्ठितः । \newline
26. अप्र॑तिष्ठित॒ इत्यप्र॑ति - स्थि॒तः॒ । \newline
27. अप्र॑तिष्ठित ए॒वैवा प्र॑तिष्ठि॒तो ऽप्र॑तिष्ठित ए॒व भ॑वति भव त्ये॒वा प्र॑तिष्ठि॒तो ऽप्र॑तिष्ठित ए॒व भ॑वति । \newline
28. अप्र॑तिष्ठित॒ इत्यप्र॑ति - स्थि॒तः॒ । \newline
29. ए॒व भ॑वति भव त्ये॒वैव भ॑वति॒ यं ॅयम् भ॑व त्ये॒वैव भ॑वति॒ यम् । \newline
30. भ॒व॒ति॒ यं ॅयम् भ॑वति भवति॒ यम् का॒मये॑त का॒मये॑त॒ यम् भ॑वति भवति॒ यम् का॒मये॑त । \newline
31. यम् का॒मये॑त का॒मये॑त॒ यं ॅयम् का॒मये॑ता प॒शु र॑प॒शुः का॒मये॑त॒ यं ॅयम् का॒मये॑ता प॒शुः । \newline
32. का॒मये॑ता प॒शु र॑प॒शुः का॒मये॑त का॒मये॑ता प॒शुः स्या᳚थ् स्या दप॒शुः का॒मये॑त का॒मये॑ता प॒शुः स्या᳚त् । \newline
33. अ॒प॒शुः स्या᳚थ् स्या दप॒शु र॑प॒शुः स्या॒ दितीति॑ स्या दप॒शु र॑प॒शुः स्या॒ दिति॑ । \newline
34. स्या॒ दितीति॑ स्याथ् स्या॒दि त्य॑प॒र्ण म॑प॒र्ण मिति॑ स्याथ् स्या॒दि त्य॑प॒र्णम् । \newline
35. इत्य॑प॒र्ण म॑प॒र्ण मिती त्य॑प॒र्णम् तस्मै॒ तस्मा॑ अप॒र्ण मितीत्य॑ प॒र्णम् तस्मै᳚ । \newline
36. अ॒प॒र्णम् तस्मै॒ तस्मा॑ अप॒र्ण म॑प॒र्णम् तस्मै॒ शुष्का᳚ग्रꣳ॒॒ शुष्का᳚ग्र॒म् तस्मा॑ अप॒र्ण म॑प॒र्णम् तस्मै॒ शुष्का᳚ग्रम् । \newline
37. तस्मै॒ शुष्का᳚ग्रꣳ॒॒ शुष्का᳚ग्र॒म् तस्मै॒ तस्मै॒ शुष्का᳚ग्रं ॅवृश्चेद् वृश्चे॒ च्छुष्का᳚ग्र॒म् तस्मै॒ तस्मै॒ शुष्का᳚ग्रं ॅवृश्चेत् । \newline
38. शुष्का᳚ग्रं ॅवृश्चेद् वृश्चे॒च् छुष्का᳚ग्रꣳ॒॒ शुष्का᳚ग्रं ॅवृश्चे दे॒ष ए॒ष वृ॑श्चे॒च् छुष्का᳚ग्रꣳ॒॒ शुष्का᳚ग्रं ॅवृश्चे दे॒षः । \newline
39. शुष्का᳚ग्र॒मिति॒ शुष्क॑ - अ॒ग्र॒म् । \newline
40. वृ॒श्चे॒ दे॒ष ए॒ष वृ॑श्चेद् वृश्चे दे॒ष वै वा ए॒ष वृ॑श्चेद् वृश्चे दे॒ष वै । \newline
41. ए॒ष वै वा ए॒ष ए॒ष वै वन॒स्पती॑नां॒ ॅवन॒स्पती॑नां॒ ॅवा ए॒ष ए॒ष वै वन॒स्पती॑नाम् । \newline
42. वै वन॒स्पती॑नां॒ ॅवन॒स्पती॑नां॒ ॅवै वै वन॒स्पती॑ना मपश॒व्यो॑ ऽपश॒व्यो वन॒स्पती॑नां॒ ॅवै वै वन॒स्पती॑ना मपश॒व्यः । \newline
43. वन॒स्पती॑ना मपश॒व्यो॑ ऽपश॒व्यो वन॒स्पती॑नां॒ ॅवन॒स्पती॑ना मपश॒व्यो॑ ऽप॒शु र॑प॒शु र॑पश॒व्यो वन॒स्पती॑नां॒ ॅवन॒स्पती॑ना मपश॒व्यो॑ ऽप॒शुः । \newline
44. अ॒प॒श॒व्यो॑ ऽप॒शु र॑प॒शु र॑पश॒व्यो॑ ऽपश॒व्यो॑ ऽप॒शु रे॒वैवा प॒शु र॑पश॒व्यो॑ ऽपश॒व्यो॑ ऽप॒शु रे॒व । \newline
45. अ॒प॒शु रे॒वैवा प॒शु र॑प॒शु रे॒व भ॑वति भव त्ये॒वा प॒शु र॑प॒शु रे॒व भ॑वति । \newline
46. ए॒व भ॑वति भव त्ये॒वैव भ॑वति॒ यं ॅयम् भ॑व त्ये॒वैव भ॑वति॒ यम् । \newline
47. भ॒व॒ति॒ यं ॅयम् भ॑वति भवति॒ यम् का॒मये॑त का॒मये॑त॒ यम् भ॑वति भवति॒ यम् का॒मये॑त । \newline
48. यम् का॒मये॑त का॒मये॑त॒ यं ॅयम् का॒मये॑त पशु॒मान् प॑शु॒मान् का॒मये॑त॒ यं ॅयम् का॒मये॑त पशु॒मान् । \newline
49. का॒मये॑त पशु॒मान् प॑शु॒मान् का॒मये॑त का॒मये॑त पशु॒मान् थ्स्या᳚थ् स्यात् पशु॒मान् का॒मये॑त का॒मये॑त पशु॒मान् थ्स्या᳚त् । \newline
50. प॒शु॒मान् थ्स्या᳚थ् स्यात् पशु॒मान् प॑शु॒मान् थ्स्या॒ दितीति॑ स्यात् पशु॒मान् प॑शु॒मान् थ्स्या॒दिति॑ । \newline
51. प॒शु॒मानिति॑ पशु - मान् । \newline
52. स्या॒ दितीति॑ स्याथ् स्या॒दिति॑ बहुप॒र्णम् ब॑हुप॒र्ण मिति॑ स्याथ् स्या॒दिति॑ बहुप॒र्णम् । \newline
53. इति॑ बहुप॒र्णम् ब॑हुप॒र्ण मितीति॑ बहुप॒र्णम् तस्मै॒ तस्मै॑ बहुप॒र्ण मितीति॑ बहुप॒र्णम् तस्मै᳚ । \newline
54. ब॒हु॒प॒र्णम् तस्मै॒ तस्मै॑ बहुप॒र्णम् ब॑हुप॒र्णम् तस्मै॑ बहुशा॒खम् ब॑हुशा॒खम् तस्मै॑ बहुप॒र्णम् ब॑हुप॒र्णम् तस्मै॑ बहुशा॒खम् । \newline
55. ब॒हु॒प॒र्णमिति॑ बहु - प॒र्णम् । \newline
56. तस्मै॑ बहुशा॒खम् ब॑हुशा॒खम् तस्मै॒ तस्मै॑ बहुशा॒खं ॅवृ॑श्चेद् वृश्चेद् बहुशा॒खम् तस्मै॒ तस्मै॑ बहुशा॒खं ॅवृ॑श्चेत् । \newline
57. ब॒हु॒शा॒खं ॅवृ॑श्चेद् वृश्चेद् बहुशा॒खम् ब॑हुशा॒खं ॅवृ॑श्चे दे॒ष ए॒ष वृ॑श्चेद् बहुशा॒खम् ब॑हुशा॒खं ॅवृ॑श्चे दे॒षः । \newline
58. ब॒हु॒शा॒खमिति॑ बहु - शा॒खम् । \newline
59. वृ॒श्चे॒ दे॒ष ए॒ष वृ॑श्चेद् वृश्चे दे॒ष वै वा ए॒ष वृ॑श्चेद् वृश्चे दे॒ष वै । \newline
60. ए॒ष वै वा ए॒ष ए॒ष वै वन॒स्पती॑नां॒ ॅवन॒स्पती॑नां॒ ॅवा ए॒ष ए॒ष वै वन॒स्पती॑नाम् । \newline
61. वै वन॒स्पती॑नां॒ ॅवन॒स्पती॑नां॒ ॅवै वै वन॒स्पती॑नाम् पश॒व्यः॑ पश॒व्यो॑ वन॒स्पती॑नां॒ ॅवै वै वन॒स्पती॑नाम् पश॒व्यः॑ । \newline
\pagebreak
\markright{ TS 6.3.3.5  \hfill https://www.vedavms.in \hfill}

\section{ TS 6.3.3.5 }

\textbf{TS 6.3.3.5 } \newline
\textbf{Samhita Paata} \newline

वन॒स्पती॑नां पश॒व्यः॑ पशु॒माने॒व भ॑वति॒ प्रति॑ष्ठितं ॅवृश्चेत् प्रति॒ष्ठाका॑मस्यै॒ष वै वन॒स्पती॑नां॒ प्रति॑ष्ठितो॒ यः स॒मे भूम्यै॒ स्वाद्योने॑ रू॒ढः प्रत्ये॒व ति॑ष्ठति॒ यः प्र॒त्यङ्ङुप॑नत॒स्तं ॅवृ॑श्चे॒थ् स हि मेध॑म॒भ्युप॑नतः॒ पञ्चा॑रत्निं॒ तस्मै॑ वृश्चे॒द्यं का॒मये॒तोपै॑न॒मुत्त॑रो य॒ज्ञो न॑मे॒दिति॒ पञ्चा᳚क्षरा प॒ङ्क्तिः पाङ्क्तो॑ य॒ज्ञ् उपै॑न॒मुत्त॑रो य॒ज्ञो- [  ] \newline

\textbf{Pada Paata} \newline

वन॒स्पती॑नाम् । प॒श॒व्यः॑ । प॒शु॒मानिति॑ पशु - मान् । ए॒व । भ॒व॒ति॒ । प्रति॑ष्ठित॒मिति॒ प्रति॑ - स्थि॒त॒म् । वृ॒श्चे॒त् । प्र॒ति॒ष्ठाका॑म॒स्येति॑ प्रति॒ष्ठा- का॒म॒स्य॒ । ए॒षः । वै । वन॒स्पती॑नाम् । प्रति॑ष्ठित॒ इति॒ प्रति॑ - स्थि॒तः॒ । यः । स॒मे । भूम्यै᳚ । स्वात् । योनेः᳚ । रू॒ढः । प्रतीति॑ । ए॒व । ति॒ष्ठ॒ति॒ । यः । प्र॒त्यङ् । उप॑नत॒ इत्युप॑-न॒तः॒ । तम् । वृ॒श्चे॒त् । सः । हि । मेध᳚म् । अ॒भीति॑ । उप॑नत॒ इत्युप॑ - न॒तः॒ । पञ्चा॑रत्नि॒मिति॒ पञ्च॑- अ॒र॒त्नि॒म् । तस्मै᳚ । वृ॒श्चे॒त् । यम् । का॒मये॑त । उपेति॑ । ए॒न॒म् । उत्त॑र॒ इत्युत् - त॒रः॒ । य॒ज्ञ्ः । न॒मे॒त् । इति॑ । पञ्चा᳚क्ष॒रेति॒ पञ्च॑-आ॒क्ष॒रा॒ । प॒ङ्क्तिः । पाङ्क्तः॑ । य॒ज्ञ्ः । उपेति॑ । ए॒न॒म् । उत्त॑र॒ इत्युत् - त॒रः॒ । य॒ज्ञ्ः ।  \newline


\textbf{Krama Paata} \newline

वन॒स्पती॑नाम् पश॒व्यः॑ । प॒श॒व्यः॑ पशु॒मान् । प॒शु॒माने॒व । प॒शु॒मानिति॑ पशु - मान् । ए॒व भ॑वति । भ॒व॒ति॒ प्रति॑ष्ठितम् । प्रति॑ष्ठितम् ॅवृश्चेत् । प्रति॑ष्ठित॒मिति॒ प्रति॑ - स्थि॒त॒म् । वृ॒श्चे॒त् प्र॒ति॒ष्ठाका॑मस्य । प्र॒ति॒ष्ठाका॑मस्यै॒षः । प्र॒ति॒ष्ठाका॑म॒स्येति॑ प्रति॒ष्ठा - का॒म॒स्य॒ । ए॒ष वै । वै वन॒स्पती॑नाम् । वन॒स्पती॑ना॒म् प्रति॑ष्ठितः । प्रति॑ष्ठितो॒ यः । प्रति॑ष्ठित॒ इति॒ प्रति॑ - स्थि॒तः॒ । यः स॒मे । स॒मे भूम्यै᳚ । भूम्यै॒ स्वात् । स्वाद् योनेः᳚ । योने॑ रू॒ढः । रू॒ढः प्रति॑ । प्रत्ये॒व । ए॒व ति॑ष्ठति । ति॒ष्ठ॒ति॒ यः । यः प्र॒त्यङ्‍ङ् । प्र॒त्यङ्‍ङुप॑नतः । उप॑नत॒स्तम् । उप॑नत॒ इत्युप॑ - न॒तः॒ । तम् ॅवृ॑श्चेत् । वृ॒श्चे॒थ् सः । स हि । हि मेध᳚म् । मेध॑म॒भि । अ॒भ्युप॑नतः । उप॑नतः॒ पञ्चा॑रत्निम् । उप॑नत॒ इत्युप॑ - न॒तः॒ । पञ्चा॑रत्नि॒म् तस्मै᳚ । पञ्चा॑रत्नि॒मिति॒ पञ्च॑ - अ॒र॒त्नि॒म् । तस्मै॑ वृश्चेत् । वृ॒श्चे॒द् यम् । यम् का॒मये॑त । का॒मये॒तोप॑ । उपै॑नम् । ए॒न॒मुत्त॑रः । उत्त॑रो य॒ज्ञ्ः । उत्त॑र॒ इत्युत् - त॒रः॒ । य॒ज्ञो न॑मेत् । न॒मे॒दिति॑ । इति॒ पञ्चा᳚क्षरा । पञ्चा᳚क्षरा प॒ङ्‍क्तिः । पञ्चा᳚क्ष॒रेति॒ पञ्च॑ - अ॒क्ष॒रा॒ । प॒ङ्‍क्तिः पाङ्‍क्तः॑ । पाङ्‍क्तो॑ य॒ज्ञ्ः । य॒ज्ञ् उप॑ । उपै॑नम् । ए॒न॒मुत्त॑रः । उत्त॑रो य॒ज्ञ्ः । उत्त॑र॒ इत्युत् - त॒रः॒ । य॒ज्ञो न॑मति \newline

\textbf{Jatai Paata} \newline

1. वन॒स्पती॑नाम् पश॒व्यः॑ पश॒व्यो॑ वन॒स्पती॑नां॒ ॅवन॒स्पती॑नाम् पश॒व्यः॑ । \newline
2. प॒श॒व्यः॑ पशु॒मान् प॑शु॒मान् प॑श॒व्यः॑ पश॒व्यः॑ पशु॒मान् । \newline
3. प॒शु॒माने॒ वैव प॑शु॒मान् प॑शु॒माने॒व । \newline
4. प॒शु॒मानिति॑ पशु - मान् । \newline
5. ए॒व भ॑वति भव त्ये॒वैव भ॑वति । \newline
6. भ॒व॒ति॒ प्रति॑ष्ठित॒म् प्रति॑ष्ठितम् भवति भवति॒ प्रति॑ष्ठितम् । \newline
7. प्रति॑ष्ठितं ॅवृश्चेद् वृश्चे॒त् प्रति॑ष्ठित॒म् प्रति॑ष्ठितं ॅवृश्चेत् । \newline
8. प्रति॑ष्ठित॒मिति॒ प्रति॑ - स्थि॒त॒म् । \newline
9. वृ॒श्चे॒त् प्र॒ति॒ष्ठाका॑मस्य प्रति॒ष्ठाका॑मस्य वृश्चेद् वृश्चेत् प्रति॒ष्ठाका॑मस्य । \newline
10. प्र॒ति॒ष्ठाका॑म स्यै॒ष ए॒ष प्र॑ति॒ष्ठाका॑मस्य प्रति॒ष्ठाका॑म स्यै॒षः । \newline
11. प्र॒ति॒ष्ठाका॑म॒स्येति॑ प्रति॒ष्ठा - का॒म॒स्य॒ । \newline
12. ए॒ष वै वा ए॒ष ए॒ष वै । \newline
13. वै वन॒स्पती॑नां॒ ॅवन॒स्पती॑नां॒ ॅवै वै वन॒स्पती॑नाम् । \newline
14. वन॒स्पती॑ना॒म् प्रति॑ष्ठितः॒ प्रति॑ष्ठितो॒ वन॒स्पती॑नां॒ ॅवन॒स्पती॑ना॒म् प्रति॑ष्ठितः । \newline
15. प्रति॑ष्ठितो॒ यो यः प्रति॑ष्ठितः॒ प्रति॑ष्ठितो॒ यः । \newline
16. प्रति॑ष्ठित॒ इति॒ प्रति॑ - स्थि॒तः॒ । \newline
17. यः स॒मे स॒मे यो यः स॒मे । \newline
18. स॒मे भूम्यै॒ भूम्यै॑ स॒मे स॒मे भूम्यै᳚ । \newline
19. भूम्यै॒ स्वाथ् स्वाद् भूम्यै॒ भूम्यै॒ स्वात् । \newline
20. स्वाद् योने॒र् योनेः॒ स्वाथ् स्वाद् योनेः᳚ । \newline
21. योने॑ रू॒ढो रू॒ढो योने॒र् योने॑ रू॒ढः । \newline
22. रू॒ढः प्रति॒ प्रति॑ रू॒ढो रू॒ढः प्रति॑ । \newline
23. प्रत्ये॒वैव प्रति॒ प्रत्ये॒व । \newline
24. ए॒व ति॑ष्ठति तिष्ठ त्ये॒वैव ति॑ष्ठति । \newline
25. ति॒ष्ठ॒ति॒ यो यस्ति॑ष्ठति तिष्ठति॒ यः । \newline
26. यः प्र॒त्यङ् प्र॒त्यङ्. यो यः प्र॒त्यङ् । \newline
27. प्र॒त्यङ् ङुप॑नत॒ उप॑नतः प्र॒त्यङ् प्र॒त्यङ् ङुप॑नतः । \newline
28. उप॑नत॒ स्तम् त मुप॑नत॒ उप॑नत॒ स्तम् । \newline
29. उप॑नत॒ इत्युप॑ - न॒तः॒ । \newline
30. तं ॅवृ॑श्चेद् वृश्चे॒त् तम् तं ॅवृ॑श्चेत् । \newline
31. वृ॒श्चे॒थ् स स वृ॑श्चेद् वृश्चे॒थ् सः । \newline
32. स हि हि स स हि । \newline
33. हि मेध॒म् मेधꣳ॒॒ हि हि मेध᳚म् । \newline
34. मेध॑ म॒भ्य॑भि मेध॒म् मेध॑ म॒भि । \newline
35. अ॒भ्युप॑नत॒ उप॑नतो॒ ऽभ्य॑ भ्युप॑नतः । \newline
36. उप॑नतः॒ पञ्चा॑रत्नि॒म् पञ्चा॑रत्नि॒ मुप॑नत॒ उप॑नतः॒ पञ्चा॑रत्निम् । \newline
37. उप॑नत॒ इत्युप॑ - न॒तः॒ । \newline
38. पञ्चा॑रत्नि॒म् तस्मै॒ तस्मै॒ पञ्चा॑रत्नि॒म् पञ्चा॑रत्नि॒म् तस्मै᳚ । \newline
39. पञ्चा॑रत्नि॒मिति॒ पञ्च॑ - अ॒र॒त्नि॒म् । \newline
40. तस्मै॑ वृश्चेद् वृश्चे॒त् तस्मै॒ तस्मै॑ वृश्चेत् । \newline
41. वृ॒श्चे॒द् यं ॅयं ॅवृ॑श्चेद् वृश्चे॒द् यम् । \newline
42. यम् का॒मये॑त का॒मये॑त॒ यं ॅयम् का॒मये॑त । \newline
43. का॒मये॒तोपोप॑ का॒मये॑त का॒मये॒तोप॑ । \newline
44. उपै॑न मेन॒ मुपो पै॑नम् । \newline
45. ए॒न॒ मुत्त॑र॒ उत्त॑र एन मेन॒ मुत्त॑रः । \newline
46. उत्त॑रो य॒ज्ञो य॒ज्ञ् उत्त॑र॒ उत्त॑रो य॒ज्ञ्ः । \newline
47. उत्त॑र॒ इत्युत् - त॒रः॒ । \newline
48. य॒ज्ञो न॑मेन् नमेद् य॒ज्ञो य॒ज्ञो न॑मेत् । \newline
49. न॒मे॒ दितीति॑ नमेन् नमे॒ दिति॑ । \newline
50. इति॒ पञ्चा᳚क्षरा॒ पञ्चा᳚क्ष॒ रेतीति॒ पञ्चा᳚क्षरा । \newline
51. पञ्चा᳚क्षरा प॒ङ्क्तिः प॒ङ्क्तिः पञ्चा᳚क्षरा॒ पञ्चा᳚क्षरा प॒ङ्क्तिः । \newline
52. पञ्चा᳚क्ष॒रेति॒ पञ्च॑ - अ॒क्ष॒रा॒ । \newline
53. प॒ङ्क्तिः पाङ्क्तः॒ पाङ्क्तः॑ प॒ङ्क्तिः प॒ङ्क्तिः पाङ्क्तः॑ । \newline
54. पाङ्क्तो॑ य॒ज्ञो य॒ज्ञ्ः पाङ्क्तः॒ पाङ्क्तो॑ य॒ज्ञ्ः । \newline
55. य॒ज्ञ् उपोप॑ य॒ज्ञो य॒ज्ञ् उप॑ । \newline
56. उपै॑न मेन॒ मुपो पै॑नम् । \newline
57. ए॒न॒ मुत्त॑र॒ उत्त॑र एन मेन॒ मुत्त॑रः । \newline
58. उत्त॑रो य॒ज्ञो य॒ज्ञ् उत्त॑र॒ उत्त॑रो य॒ज्ञ्ः । \newline
59. उत्त॑र॒ इत्युत् - त॒रः॒ । \newline
60. य॒ज्ञो न॑मति नमति य॒ज्ञो य॒ज्ञो न॑मति । \newline

\textbf{Ghana Paata } \newline

1. वन॒स्पती॑नाम् पश॒व्यः॑ पश॒व्यो॑ वन॒स्पती॑नां॒ ॅवन॒स्पती॑नाम् पश॒व्यः॑ पशु॒मान् प॑शु॒मान् प॑श॒व्यो॑ वन॒स्पती॑नां॒ ॅवन॒स्पती॑नाम् पश॒व्यः॑ पशु॒मान् । \newline
2. प॒श॒व्यः॑ पशु॒मान् प॑शु॒मान् प॑श॒व्यः॑ पश॒व्यः॑ पशु॒मा ने॒वैव प॑शु॒मान् प॑श॒व्यः॑ पश॒व्यः॑ पशु॒मा ने॒व । \newline
3. प॒शु॒मा ने॒वैव प॑शु॒मान् प॑शु॒मा ने॒व भ॑वति भव त्ये॒व प॑शु॒मान् प॑शु॒मा ने॒व भ॑वति । \newline
4. प॒शु॒मानिति॑ पशु - मान् । \newline
5. ए॒व भ॑वति भव त्ये॒वैव भ॑वति॒ प्रति॑ष्ठित॒म् प्रति॑ष्ठितम् भव त्ये॒वैव भ॑वति॒ प्रति॑ष्ठितम् । \newline
6. भ॒व॒ति॒ प्रति॑ष्ठित॒म् प्रति॑ष्ठितम् भवति भवति॒ प्रति॑ष्ठितं ॅवृश्चेद् वृश्चे॒त् प्रति॑ष्ठितम् भवति भवति॒ प्रति॑ष्ठितं ॅवृश्चेत् । \newline
7. प्रति॑ष्ठितं ॅवृश्चेद् वृश्चे॒त् प्रति॑ष्ठित॒म् प्रति॑ष्ठितं ॅवृश्चेत् प्रति॒ष्ठाका॑मस्य प्रति॒ष्ठाका॑मस्य वृश्चे॒त् प्रति॑ष्ठित॒म् प्रति॑ष्ठितं ॅवृश्चेत् प्रति॒ष्ठाका॑मस्य । \newline
8. प्रति॑ष्ठित॒मिति॒ प्रति॑ - स्थि॒त॒म् । \newline
9. वृ॒श्चे॒त् प्र॒ति॒ष्ठाका॑मस्य प्रति॒ष्ठाका॑मस्य वृश्चेद् वृश्चेत् प्रति॒ष्ठाका॑म स्यै॒ष ए॒ष प्र॑ति॒ष्ठाका॑मस्य वृश्चेद् वृश्चेत् प्रति॒ष्ठाका॑म स्यै॒षः । \newline
10. प्र॒ति॒ष्ठाका॑म स्यै॒ष ए॒ष प्र॑ति॒ष्ठाका॑मस्य प्रति॒ष्ठाका॑म स्यै॒ष वै वा ए॒ष प्र॑ति॒ष्ठाका॑मस्य प्रति॒ष्ठाका॑म स्यै॒ष वै । \newline
11. प्र॒ति॒ष्ठाका॑म॒स्येति॑ प्रति॒ष्ठा - का॒म॒स्य॒ । \newline
12. ए॒ष वै वा ए॒ष ए॒ष वै वन॒स्पती॑नां॒ ॅवन॒स्पती॑नां॒ ॅवा ए॒ष ए॒ष वै वन॒स्पती॑नाम् । \newline
13. वै वन॒स्पती॑नां॒ ॅवन॒स्पती॑नां॒ ॅवै वै वन॒स्पती॑ना॒म् प्रति॑ष्ठितः॒ प्रति॑ष्ठितो॒ वन॒स्पती॑नां॒ ॅवै वै वन॒स्पती॑ना॒म् प्रति॑ष्ठितः । \newline
14. वन॒स्पती॑ना॒म् प्रति॑ष्ठितः॒ प्रति॑ष्ठितो॒ वन॒स्पती॑नां॒ ॅवन॒स्पती॑ना॒म् प्रति॑ष्ठितो॒ यो यः प्रति॑ष्ठितो॒ वन॒स्पती॑नां॒ ॅवन॒स्पती॑ना॒म् प्रति॑ष्ठितो॒ यः । \newline
15. प्रति॑ष्ठितो॒ यो यः प्रति॑ष्ठितः॒ प्रति॑ष्ठितो॒ यः स॒मे स॒मे यः प्रति॑ष्ठितः॒ प्रति॑ष्ठितो॒ यः स॒मे । \newline
16. प्रति॑ष्ठित॒ इति॒ प्रति॑ - स्थि॒तः॒ । \newline
17. यः स॒मे स॒मे यो यः स॒मे भूम्यै॒ भूम्यै॑ स॒मे यो यः स॒मे भूम्यै᳚ । \newline
18. स॒मे भूम्यै॒ भूम्यै॑ स॒मे स॒मे भूम्यै॒ स्वाथ् स्वाद् भूम्यै॑ स॒मे स॒मे भूम्यै॒ स्वात् । \newline
19. भूम्यै॒ स्वाथ् स्वाद् भूम्यै॒ भूम्यै॒ स्वाद् योने॒र् योनेः॒ स्वाद् भूम्यै॒ भूम्यै॒ स्वाद् योनेः᳚ । \newline
20. स्वाद् योने॒र् योनेः॒ स्वाथ् स्वाद् योने॑ रू॒ढो रू॒ढो योनेः॒ स्वाथ् स्वाद् योने॑ रू॒ढः । \newline
21. योने॑ रू॒ढो रू॒ढो योने॒र् योने॑ रू॒ढः प्रति॒ प्रति॑ रू॒ढो योने॒र् योने॑ रू॒ढः प्रति॑ । \newline
22. रू॒ढः प्रति॒ प्रति॑ रू॒ढो रू॒ढः प्रत्ये॒ वैव प्रति॑ रू॒ढो रू॒ढः प्रत्ये॒व । \newline
23. प्रत्ये॒ वैव प्रति॒ प्रत्ये॒व ति॑ष्ठति तिष्ठ त्ये॒व प्रति॒ प्रत्ये॒व ति॑ष्ठति । \newline
24. ए॒व ति॑ष्ठति तिष्ठ त्ये॒वैव ति॑ष्ठति॒ यो यस्ति॑ष्ठ त्ये॒वैव ति॑ष्ठति॒ यः । \newline
25. ति॒ष्ठ॒ति॒ यो यस्ति॑ष्ठति तिष्ठति॒ यः प्र॒त्यङ् प्र॒त्यङ्. यस्ति॑ष्ठति तिष्ठति॒ यः प्र॒त्यङ् । \newline
26. यः प्र॒त्यङ् प्र॒त्यङ्. यो यः प्र॒त्यङ् ङुप॑नत॒ उप॑नतः प्र॒त्यङ्. यो यः प्र॒त्यङ् ङुप॑नतः । \newline
27. प्र॒त्यङ् ङुप॑नत॒ उप॑नतः प्र॒त्यङ् प्र॒त्यङ् ङुप॑नत॒ स्तम् त मुप॑नतः प्र॒त्यङ् प्र॒त्यङ् 
ङुप॑नत॒ स्तम् । \newline
28. उप॑नत॒ स्तम् त मुप॑नत॒ उप॑नत॒ स्तं ॅवृ॑श्चेद् वृश्चे॒त् त मुप॑नत॒ उप॑नत॒ स्तं ॅवृ॑श्चेत् । \newline
29. उप॑नत॒ इत्युप॑ - न॒तः॒ । \newline
30. तं ॅवृ॑श्चेद् वृश्चे॒त् तम् तं ॅवृ॑श्चे॒थ् स स वृ॑श्चे॒त् तम् तं ॅवृ॑श्चे॒थ् सः । \newline
31. वृ॒श्चे॒थ् स स वृ॑श्चेद् वृश्चे॒थ् स हि हि स वृ॑श्चेद् वृश्चे॒थ् स हि । \newline
32. स हि हि स स हि मेध॒म् मेधꣳ॒॒ हि स स हि मेध᳚म् । \newline
33. हि मेध॒म् मेधꣳ॒॒ हि हि मेध॑ म॒भ्य॑भि मेधꣳ॒॒ हि हि मेध॑ म॒भि । \newline
34. मेध॑ म॒भ्य॑भि मेध॒म् मेध॑ म॒भ्युप॑नत॒ उप॑नतो॒ ऽभि मेध॒म् मेध॑ म॒भ्युप॑नतः । \newline
35. अ॒भ्युप॑नत॒ उप॑नतो॒ ऽभ्य॑ भ्युप॑नतः॒ पञ्चा॑रत्नि॒म् पञ्चा॑रत्नि॒ मुप॑नतो॒ ऽभ्य॑भ्युप॑नतः॒ पञ्चा॑रत्निम् । \newline
36. उप॑नतः॒ पञ्चा॑रत्नि॒म् पञ्चा॑रत्नि॒ मुप॑नत॒ उप॑नतः॒ पञ्चा॑रत्नि॒म् तस्मै॒ तस्मै॒ पञ्चा॑रत्नि॒ मुप॑नत॒ उप॑नतः॒ पञ्चा॑रत्नि॒म् तस्मै᳚ । \newline
37. उप॑नत॒ इत्युप॑ - न॒तः॒ । \newline
38. पञ्चा॑रत्नि॒म् तस्मै॒ तस्मै॒ पञ्चा॑रत्नि॒म् पञ्चा॑रत्नि॒म् तस्मै॑ वृश्चेद् वृश्चे॒त् तस्मै॒ पञ्चा॑रत्नि॒म् पञ्चा॑रत्नि॒म् तस्मै॑ वृश्चेत् । \newline
39. पञ्चा॑रत्नि॒मिति॒ पञ्च॑ - अ॒र॒त्नि॒म् । \newline
40. तस्मै॑ वृश्चेद् वृश्चे॒त् तस्मै॒ तस्मै॑ वृश्चे॒द् यं ॅयं ॅवृ॑श्चे॒त् तस्मै॒ तस्मै॑ वृश्चे॒द् यम् । \newline
41. वृ॒श्चे॒द् यं ॅयं ॅवृ॑श्चेद् वृश्चे॒द् यम् का॒मये॑त का॒मये॑त॒ यं ॅवृ॑श्चेद् वृश्चे॒द् यम् का॒मये॑त । \newline
42. यम् का॒मये॑त का॒मये॑त॒ यं ॅयम् का॒मये॒तोपोप॑ का॒मये॑त॒ यं ॅयम् का॒मये॒तोप॑ । \newline
43. का॒मये॒तो पोप॑ का॒मये॑त का॒मये॒ तोपै॑न मेन॒ मुप॑ का॒मये॑त का॒मये॒ तोपै॑नम् । \newline
44. उपै॑न मेन॒ मुपो पै॑न॒ मुत्त॑र॒ उत्त॑र एन॒ मुपो पै॑न॒ मुत्त॑रः । \newline
45. ए॒न॒ मुत्त॑र॒ उत्त॑र एन मेन॒ मुत्त॑रो य॒ज्ञो य॒ज्ञ् उत्त॑र एन मेन॒ मुत्त॑रो य॒ज्ञ्ः । \newline
46. उत्त॑रो य॒ज्ञो य॒ज्ञ् उत्त॑र॒ उत्त॑रो य॒ज्ञो न॑मेन् नमेद् य॒ज्ञ् उत्त॑र॒ उत्त॑रो य॒ज्ञो न॑मेत् । \newline
47. उत्त॑र॒ इत्युत् - त॒रः॒ । \newline
48. य॒ज्ञो न॑मेन् नमेद् य॒ज्ञो य॒ज्ञो न॑मे॒ दितीति॑ नमेद् य॒ज्ञो य॒ज्ञो न॑मे॒ दिति॑ । \newline
49. न॒मे॒ दितीति॑ नमेन् नमे॒ दिति॒ पञ्चा᳚क्षरा॒ पञ्चा᳚क्ष॒ रेति॑ नमेन् नमे॒ दिति॒ पञ्चा᳚क्षरा । \newline
50. इति॒ पञ्चा᳚क्षरा॒ पञ्चा᳚क्ष॒रे तीति॒ पञ्चा᳚क्षरा प॒ङ्क्तिः प॒ङ्क्तिः पञ्चा᳚क्ष॒रे तीति॒ पञ्चा᳚क्षरा प॒ङ्क्तिः । \newline
51. पञ्चा᳚क्षरा प॒ङ्क्तिः प॒ङ्क्तिः पञ्चा᳚क्षरा॒ पञ्चा᳚क्षरा प॒ङ्क्तिः पाङ्क्तः॒ पाङ्क्तः॑ प॒ङ्क्तिः पञ्चा᳚क्षरा॒ पञ्चा᳚क्षरा प॒ङ्क्तिः पाङ्क्तः॑ । \newline
52. पञ्चा᳚क्ष॒रेति॒ पञ्च॑ - अ॒क्ष॒रा॒ । \newline
53. प॒ङ्क्तिः पाङ्क्तः॒ पाङ्क्तः॑ प॒ङ्क्तिः प॒ङ्क्तिः पाङ्क्तो॑ य॒ज्ञो य॒ज्ञ्ः पाङ्क्तः॑ प॒ङ्क्तिः प॒ङ्क्तिः पाङ्क्तो॑ य॒ज्ञ्ः । \newline
54. पाङ्क्तो॑ य॒ज्ञो य॒ज्ञ्ः पाङ्क्तः॒ पाङ्क्तो॑ य॒ज्ञ् उपोप॑ य॒ज्ञ्ः पाङ्क्तः॒ पाङ्क्तो॑ य॒ज्ञ् उप॑ । \newline
55. य॒ज्ञ् उपोप॑ य॒ज्ञो य॒ज्ञ् उपै॑न मेन॒ मुप॑ य॒ज्ञो य॒ज्ञ् उपै॑नम् । \newline
56. उपै॑न मेन॒ मुपो पै॑न॒ मुत्त॑र॒ उत्त॑र एन॒ मुपो पै॑न॒ मुत्त॑रः । \newline
57. ए॒न॒ मुत्त॑र॒ उत्त॑र एन मेन॒ मुत्त॑रो य॒ज्ञो य॒ज्ञ् उत्त॑र एन मेन॒ मुत्त॑रो य॒ज्ञ्ः । \newline
58. उत्त॑रो य॒ज्ञो य॒ज्ञ् उत्त॑र॒ उत्त॑रो य॒ज्ञो न॑मति नमति य॒ज्ञ् उत्त॑र॒ उत्त॑रो य॒ज्ञो न॑मति । \newline
59. उत्त॑र॒ इत्युत् - त॒रः॒ । \newline
60. य॒ज्ञो न॑मति नमति य॒ज्ञो य॒ज्ञो न॑मति॒ षड॑रत्निꣳ॒॒ षड॑रत्निम् नमति य॒ज्ञो य॒ज्ञो न॑मति॒ षड॑रत्निम् । \newline
\pagebreak
\markright{ TS 6.3.3.6  \hfill https://www.vedavms.in \hfill}

\section{ TS 6.3.3.6 }

\textbf{TS 6.3.3.6 } \newline
\textbf{Samhita Paata} \newline

न॑मति॒ षड॑रत्निं प्रति॒ष्ठाका॑मस्य॒ षड्वा ऋ॒तव॑ ऋ॒तुष्वे॒व प्रति॑ तिष्ठति स॒प्तार॑त्निं प॒शुका॑मस्य स॒प्तप॑दा॒ शक्व॑री प॒शवः॒ शक्व॑री प॒शूने॒वाव॑ रुन्धे॒ नवा॑रत्निं॒ तेज॑स्कामस्य त्रि॒वृता॒ स्तोमे॑न॒ सम्मि॑तं॒ तेज॑स्त्रि॒वृत् ते॑ज॒स्व्ये॑व भ॑व॒-त्येका॑दशारत्नि-मिन्द्रि॒यका॑म॒-स्यैका॑दशाक्षरा त्रि॒ष्टुगि॑न्द्रि॒यं त्रि॒ष्टुगि॑न्द्रिया॒व्ये॑व भ॑वति॒ पञ्च॑दशारत्निं॒ भ्रातृ॑व्यवतः पञ्चद॒शो वज्रो॒ भ्रातृ॑व्याभिभूत्यै स॒प्तद॑शारत्निं प्र॒जाका॑मस्य सप्तद॒शः प्र॒जाप॑तिः प्र॒जाप॑ते॒राप्त्या॒ ( ) एक॑विꣳशत्यरत्निं प्रति॒ष्ठाका॑म-स्यैकविꣳ॒॒शः स्तोमा॑नां प्रति॒ष्ठा प्रति॑ष्ठित्या अ॒ष्टाश्रि॑र्भव-त्य॒ष्टाक्ष॑रा गाय॒त्री तेजो॑ गाय॒त्री गा॑य॒त्री य॑ज्ञ्मु॒खं तेज॑सै॒व गा॑यत्रि॒या य॑ज्ञ्मु॒खेन॒ संमि॑तः ॥ \newline

\textbf{Pada Paata} \newline

न॒म॒ति॒ । षड॑रत्नि॒मिति॒ षट् - अ॒र॒त्नि॒म् । प्र॒ति॒ष्ठाका॑म॒स्येति॑ प्रति॒ष्ठा- का॒म॒स्य॒ । षट् । वै । ऋ॒तवः॑ । ऋ॒तुषु॑ । ए॒व । प्रतीति॑ । ति॒ष्ठ॒ति॒ । स॒प्तार॑त्नि॒मिति॑ स॒प्त - अ॒र॒त्नि॒म् । प॒शुका॑म॒स्येति॑ प॒शु - का॒म॒स्य॒ । स॒प्तप॒देति॑ स॒प्त-प॒दा॒ । शक्व॑री । प॒शवः॑ । शक्व॑री । प॒शून् । ए॒व । अवेति॑ । रु॒न्धे॒ । नवा॑रत्नि॒मिति॒ नव॑ - अ॒र॒त्नि॒म् । तेज॑स्काम॒स्येति॒ तेजः॑ - का॒म॒स्य॒ । त्रि॒वृतेति॑ त्रि - वृता᳚ । स्तोमे॑न । सम्मि॑त॒मिति॒ सं - मि॒त॒म् । तेजः॑ । त्रि॒वृदिति॑ त्रि - वृत् । ते॒ज॒स्वी । ए॒व । भ॒व॒ति॒ । एका॑दशारत्नि॒मित्येका॑दश-अ॒र॒त्नि॒म् । इ॒न्द्रि॒यका॑म॒स्येती᳚न्द्रि॒य-का॒म॒स्य॒ । एका॑दशाक्ष॒रेत्येका॑दश - अ॒क्ष॒रा॒ । त्रि॒ष्टुक् । इ॒न्द्रि॒यम् । त्रि॒ष्टुक् । इ॒न्द्रि॒या॒वी । ए॒व । भ॒व॒ति॒ । पञ्च॑दशारत्नि॒मिति॒ पञ्च॑दश - अ॒र॒त्नि॒म् । भ्रातृ॑व्यवत॒ इति॒ भ्रातृ॑व्य - व॒तः॒ । प॒ञ्च॒द॒श इति॑ पञ्च - द॒शः । वज्रः॑ । भ्रातृ॑व्याभिभूत्या॒ इति॒ भ्रातृ॑व्य - अ॒भि॒भू॒त्यै॒ । स॒प्तद॑शारत्नि॒मिति॑ स॒प्तद॑श - अ॒र॒त्नि॒म् । प्र॒जाका॑म॒स्येति॑ प्र॒जा - का॒म॒स्य॒ । स॒प्त॒द॒श इति॑ सप्त - द॒शः । प्र॒जाप॑ति॒रिति॑ प्र॒जा-प॒तिः॒ । प्र॒जाप॑ते॒रिति॑ प्र॒जा - प॒तेः॒ । आप्त्यै᳚ ( ) । एक॑विꣳशत्यरत्नि॒मित्येक॑विꣳशति - अ॒र॒त्नि॒म् । प्र॒ति॒ष्ठाका॑म॒स्येति॑ प्रति॒ष्ठा - का॒म॒स्य॒ । ए॒क॒विꣳ॒॒श इत्ये॑क - विꣳ॒॒शः । स्तोमा॑नाम् । प्र॒ति॒ष्ठेति॑ प्रति - स्था । प्रति॑ष्ठित्या॒ इति॒ प्रति॑ - स्थि॒त्यै॒ । अ॒ष्टाश्रि॒रित्य॒ष्टा - अ॒श्रिः॒ । भ॒व॒ति॒ । अ॒ष्टाक्ष॒रेत्य॒ष्टा - अ॒क्ष॒रा॒ । गा॒य॒त्री । तेजः॑ । गा॒य॒त्री । गा॒य॒त्री । य॒ज्ञ्॒मु॒खमिति॑ यज्ञ् - मु॒खम् । तेज॑सा । ए॒व । गा॒य॒त्रि॒या । य॒ज्ञ्॒मु॒खेनेति॑ यज्ञ् - मु॒खेन॑ । संमि॑त॒ इति॒ सं - मि॒तः॒ ॥  \newline


\textbf{Krama Paata} \newline

न॒म॒ति॒ षड॑रत्निम् । षड॑रत्निम् प्रति॒ष्ठाका॑मस्य । षड॑रत्नि॒मिति॒ षट् - अ॒र॒त्नि॒म् । प्र॒ति॒ष्ठाका॑मस्य॒ षट् । प्र॒ति॒ष्ठाका॑म॒स्येति॑ प्रति॒ष्ठा - का॒म॒स्य॒ । षड् वै । वा ऋ॒तवः॑ । ऋ॒तव॑ ऋ॒तुषु॑ । ऋ॒तुष्वे॒व । ए॒व प्रति॑ । प्रति॑ तिष्ठति । ति॒ष्ठ॒ति॒ स॒प्तार॑त्निम् । स॒प्तार॑त्निम् प॒शुका॑मस्य । स॒प्तार॑त्नि॒मिति॑ स॒प्त - अ॒र॒त्नि॒म् । प॒शुका॑मस्य स॒प्तप॑दा । प॒शुका॑म॒स्येति॑ प॒शु - का॒म॒स्य॒ । स॒प्तप॑दा॒ शक्व॑री । स॒प्तप॒देति॑ स॒प्त - प॒दा॒ । शक्व॑री प॒शवः॑ । प॒शवः॒ शक्व॑री । शक्व॑री प॒शून् । प॒शूने॒व । ए॒वाव॑ । अव॑ रुन्धे । रु॒न्धे॒ नवा॑रत्निम् । नवा॑रत्नि॒म् तेज॑स्कामस्य । नवा॑रत्नि॒मिति॒ नव॑ - अ॒र॒त्नि॒म् । तेज॑स्कामस्य त्रि॒वृता᳚ । तेज॑स्काम॒स्येति॒ तेजः॑ - का॒म॒स्य॒ । त्रि॒वृता॒ स्तोमे॑न । त्रि॒वृतेति॑ त्रि - वृता᳚ । स्तोमे॑न॒ सम्मि॑तम् । सम्मि॑त॒म् तेजः॑ । सम्मि॑त॒मिति॒ सम् - मि॒त॒म् । तेज॑स्त्रि॒वृत् । त्रि॒वृत् ते॑ज॒स्वी । त्रि॒वृदिति॑ त्रि - वृत् । ते॒ज॒स्व्ये॑व । ए॒व भ॑वति । भ॒व॒त्येका॑दशारत्निम् । एका॑दशारत्निमिन्द्रि॒यका॑मस्य । एका॑दशारत्नि॒.मित्येका॑श - अ॒र॒त्नि॒म् । इ॒न्द्रि॒यका॑म॒स्यैका॑दशाक्षारा । इ॒न्द्रि॒यका॑म॒स्येती᳚न्द्रि॒य - का॒म॒स्य॒ । एका॑दशाक्षरा त्रि॒ष्टुक् । एका॑दशाक्ष॒रेत्येका॑दश - अ॒क्ष॒रा॒ । त्रि॒ष्टुगि॑न्द्रि॒यम् । इ॒न्द्रि॒यम् त्रि॒ष्टुक् । त्रि॒ष्टुगि॑न्द्रिया॒वी । इ॒न्द्रि॒या॒व्ये॑व । ए॒व भ॑वति । भ॒व॒ति॒ पञ्च॑दशारत्निम् । पञ्च॑दशारत्नि॒म् भ्रातृ॑व्यवतः । पञ्च॑दशारत्नि॒मिति॒ पञ्च॑दश - अ॒र॒त्नि॒म् । भ्रातृ॑व्यवतः पञ्चद॒शः । भ्रातृ॑व्यवत॒ इति॒ भ्रातृ॑व्य - व॒तः॒ । प॒ञ्च॒द॒शो वज्रः॑ । प॒ञ्च॒द॒श इति॑ पञ्च - द॒शः । वज्रो॒ भ्रातृ॑व्याभिभूत्यै । भ्रातृ॑व्याभिभूत्यै स॒प्तद॑शारत्निम् । भ्रातृ॑व्याभिभूत्या॒ इति॒ भ्रातृ॑व्य - अ॒भि॒भू॒त्यै॒ । स॒प्तद॑शारत्निम् प्र॒जाका॑मस्य । स॒प्तद॑शारत्नि॒मिति॑ स॒प्तद॑श - अ॒र॒त्नि॒म् । प्र॒जाका॑मस्य सप्तद॒शः । प्र॒जाका॑म॒स्येति॑ प्र॒जा - का॒म॒स्य॒ । स॒प्त॒द॒शः प्र॒जाप॑तिः । स॒प्त॒द॒श इति॑ सप्त - द॒शः । प्र॒जाप॑तिः प्र॒जाप॑तेः । प्र॒जाप॑ति॒रिति॑ प्र॒जा - प॒तिः॒ । प्र॒जाप॑ते॒राप्त्यै᳚ ( ) । प्र॒जाप॑ते॒रिति॑ प्र॒जा - प॒तेः॒ । आप्त्या॒ एक॑विꣳशत्यरत्निम् । एक॑विꣳशत्यरत्निम् प्रति॒ष्ठाका॑मस्य । एक॑विꣳशत्यरत्नि॒ मित्येक॑विꣳशति - अ॒र॒त्नि॒म् । प्र॒ति॒ष्ठाका॑मस्यैकविꣳ॒॒शः । प्र॒ति॒ष्ठाका॑म॒स्येति॑ प्रति॒ष्ठा - का॒म॒स्य॒ । ए॒क॒विꣳ॒॒शः स्तोमा॑नाम् । ए॒क॒विꣳ॒॒श इत्ये॑क - विꣳ॒॒शः । स्तोमा॑नाम् प्रति॒ष्ठा । प्र॒ति॒ष्ठा प्रति॑ष्ठित्यै । प्र॒ति॒ष्ठेति॑ प्रति - स्था । प्रति॑ष्ठित्या अ॒ष्टाश्रिः॑ । प्रति॑ष्ठित्या॒ इति॒ प्रति॑ - स्थि॒त्यै॒ । अ॒ष्टाश्रि॑र् भवति । अ॒ष्टाश्रि॒रित्य॒ष्टा - अ॒श्रिः॒ । भ॒व॒त्य॒ष्टाक्ष॑रा । अ॒ष्टाक्ष॑रा गाय॒त्री । अ॒ष्टाक्ष॒रेत्य॒ष्टा - अ॒क्ष॒रा॒ । गा॒य॒त्री तेजः॑ । तेजो॑ गाय॒त्री । गा॒य॒त्री गा॑य॒त्री । गा॒य॒त्री य॑ज्ञ्मु॒खम् । य॒ज्ञ्॒मु॒खम् तेज॑सा । य॒ज्ञ्॒मु॒खमिति॑ यज्ञ् - मु॒खम् । तेज॑सै॒व । ए॒व गा॑यत्रि॒या । गा॒य॒त्रि॒या य॑ज्ञ्मु॒खेन॑ । य॒ज्ञ्॒मु॒खेन॒ सम्मि॑तः । य॒ज्ञ्॒मु॒खेनेति॑ यज्ञ् - मु॒खेन॑ । सम्मि॑त॒ इति॒ सम् - मि॒तः॒ । \newline

\textbf{Jatai Paata} \newline

1. न॒म॒ति॒ षड॑रत्निꣳ॒॒ षड॑रत्निन् नमति नमति॒ षड॑रत्निम् । \newline
2. षड॑रत्निम् प्रति॒ष्ठाका॑मस्य प्रति॒ष्ठाका॑मस्य॒ षड॑रत्निꣳ॒॒ षड॑रत्निम् प्रति॒ष्ठाका॑मस्य । \newline
3. षड॑रत्नि॒मिति॒ षट् - अ॒र॒त्नि॒म् । \newline
4. प्र॒ति॒ष्ठाका॑मस्य॒ षट् थ्षट् प्र॑ति॒ष्ठाका॑मस्य प्रति॒ष्ठाका॑मस्य॒ षट् । \newline
5. प्र॒ति॒ष्ठाका॑म॒स्येति॑ प्रति॒ष्ठा - का॒म॒स्य॒ । \newline
6. षड् वै वै षट् थ्षड् वै । \newline
7. वा ऋ॒तव॑ ऋ॒तवो॒ वै वा ऋ॒तवः॑ । \newline
8. ऋ॒तव॑ ऋ॒तुष् वृ॒तुष् वृ॒तव॑ ऋ॒तव॑ ऋ॒तुषु॑ । \newline
9. ऋ॒तुष् वे॒वैव र्‌तुष् वृ॒तु ष्वे॒व । \newline
10. ए॒व प्रति॒ प्रत्ये॒वैव प्रति॑ । \newline
11. प्रति॑ तिष्ठति तिष्ठति॒ प्रति॒ प्रति॑ तिष्ठति । \newline
12. ति॒ष्ठ॒ति॒ स॒प्तार॑त्निꣳ स॒प्तार॑त्निम् तिष्ठति तिष्ठति स॒प्तार॑त्निम् । \newline
13. स॒प्तार॑त्निम् प॒शुका॑मस्य प॒शुका॑मस्य स॒प्तार॑त्निꣳ स॒प्तार॑त्निम् प॒शुका॑मस्य । \newline
14. स॒प्तार॑त्नि॒मिति॑ स॒प्त - अ॒र॒त्नि॒म् । \newline
15. प॒शुका॑मस्य स॒प्तप॑दा स॒प्तप॑दा प॒शुका॑मस्य प॒शुका॑मस्य स॒प्तप॑दा । \newline
16. प॒शुका॑म॒स्येति॑ प॒शु - का॒म॒स्य॒ । \newline
17. स॒प्तप॑दा॒ शक्व॑री॒ शक्व॑री स॒प्तप॑दा स॒प्तप॑दा॒ शक्व॑री । \newline
18. स॒प्तप॒देति॑ स॒प्त - प॒दा॒ । \newline
19. शक्व॑री प॒शवः॑ प॒शवः॒ शक्व॑री॒ शक्व॑री प॒शवः॑ । \newline
20. प॒शवः॒ शक्व॑री॒ शक्व॑री प॒शवः॑ प॒शवः॒ शक्व॑री । \newline
21. शक्व॑री प॒शून् प॒शूञ् छक्व॑री॒ शक्व॑री प॒शून् । \newline
22. प॒शूने॒ वैव प॒शून् प॒शूने॒व । \newline
23. ए॒वावा वै॒वै वाव॑ । \newline
24. अव॑ रुन्धे रु॒न्धे ऽवाव॑ रुन्धे । \newline
25. रु॒न्धे॒ नवा॑रत्नि॒म् नवा॑रत्निꣳ रुन्धे रुन्धे॒ नवा॑रत्निम् । \newline
26. नवा॑रत्नि॒म् तेज॑स्कामस्य॒ तेज॑स्कामस्य॒ नवा॑रत्नि॒म् नवा॑रत्नि॒म् तेज॑स्कामस्य । \newline
27. नवा॑रत्नि॒मिति॒ नव॑ - अ॒र॒त्नि॒म् । \newline
28. तेज॑स्कामस्य त्रि॒वृता᳚ त्रि॒वृता॒ तेज॑स्कामस्य॒ तेज॑स्कामस्य त्रि॒वृता᳚ । \newline
29. तेज॑स्काम॒स्येति॒ तेजः॑ - का॒म॒स्य॒ । \newline
30. त्रि॒वृता॒ स्तोमे॑न॒ स्तोमे॑न त्रि॒वृता᳚ त्रि॒वृता॒ स्तोमे॑न । \newline
31. त्रि॒वृतेति॑ त्रि - वृता᳚ । \newline
32. स्तोमे॑न॒ सम्मि॑तꣳ॒॒ सम्मि॑तꣳ॒॒ स्तोमे॑न॒ स्तोमे॑न॒ सम्मि॑तम् । \newline
33. सम्मि॑त॒म् तेज॒ स्तेजः॒ सम्मि॑तꣳ॒॒ सम्मि॑त॒म् तेजः॑ । \newline
34. सम्मि॑त॒मिति॒ सं - मि॒त॒म् । \newline
35. तेज॑ स्त्रि॒वृत् त्रि॒वृत् तेज॒ स्तेज॑ स्त्रि॒वृत् । \newline
36. त्रि॒वृत् ते॑ज॒स्वी ते॑ज॒स्वी त्रि॒वृत् त्रि॒वृत् ते॑ज॒स्वी । \newline
37. त्रि॒वृदिति॑ त्रि - वृत् । \newline
38. ते॒ज॒ स्व्ये॑वैव ते॑ज॒स्वी ते॑ज॒स्व्ये॑व । \newline
39. ए॒व भ॑वति भव त्ये॒वैव भ॑वति । \newline
40. भ॒व॒ त्येका॑दशारत्नि॒ मेका॑दशारत्निम् भवति भव॒ त्येका॑दशारत्निम् । \newline
41. एका॑दशारत्नि मिन्द्रि॒यका॑म स्येन्द्रि॒यका॑म॒स्यै का॑दशारत्नि॒ मेका॑दशारत्नि मिन्द्रि॒यका॑मस्य । \newline
42. एका॑दशारत्नि॒मित्येका॑दश - अ॒र॒त्नि॒म् । \newline
43. इ॒न्द्रि॒यका॑म॒ स्यैका॑दशाक्ष॒रै का॑दशाक्ष रेन्द्रि॒यका॑म स्येन्द्रि॒यका॑म॒ स्यैका॑दशाक्षरा । \newline
44. इ॒न्द्रि॒यका॑म॒स्येती᳚न्द्रि॒य - का॒म॒स्य॒ । \newline
45. एका॑दशाक्षरा त्रि॒ष्टुक् त्रि॒ष्टु गेका॑दशाक्ष॒ रैका॑दशाक्षरा त्रि॒ष्टुक् । \newline
46. एका॑दशाक्ष॒रेत्येका॑दश - अ॒क्ष॒रा॒ । \newline
47. त्रि॒ष्टु गि॑न्द्रि॒य मि॑न्द्रि॒यम् त्रि॒ष्टुक् त्रि॒ष्टु गि॑न्द्रि॒यम् । \newline
48. इ॒न्द्रि॒यम् त्रि॒ष्टुक् त्रि॒ष्टु गि॑न्द्रि॒य मि॑न्द्रि॒यम् त्रि॒ष्टुक् । \newline
49. त्रि॒ष्टु गि॑न्द्रिया॒वी न्द्रि॑या॒वी त्रि॒ष्टुक् त्रि॒ष्टु गि॑न्द्रिया॒वी । \newline
50. इ॒न्द्रि॒या॒व्ये॑ वैवेन्द्रि॑या॒वी न्द्रि॑या॒ व्ये॑व । \newline
51. ए॒व भ॑वति भव त्ये॒वैव भ॑वति । \newline
52. भ॒व॒ति॒ पञ्च॑दशारत्नि॒म् पञ्च॑दशारत्निम् भवति भवति॒ पञ्च॑दशारत्निम् । \newline
53. पञ्च॑दशारत्नि॒म् भ्रातृ॑व्यवतो॒ भ्रातृ॑व्यवतः॒ पञ्च॑दशारत्नि॒म् पञ्च॑दशारत्नि॒म् भ्रातृ॑व्यवतः । \newline
54. पञ्च॑दशारत्नि॒मिति॒ पञ्च॑दश - अ॒र॒त्नि॒म् । \newline
55. भ्रातृ॑व्यवतः पञ्चद॒शः प॑ञ्चद॒शो भ्रातृ॑व्यवतो॒ भ्रातृ॑व्यवतः पञ्चद॒शः । \newline
56. भ्रातृ॑व्यवत॒ इति॒ भ्रातृ॑व्य - व॒तः॒ । \newline
57. प॒ञ्च॒द॒शो वज्रो॒ वज्रः॑ पञ्चद॒शः प॑ञ्चद॒शो वज्रः॑ । \newline
58. प॒ञ्च॒द॒श इति॑ पञ्च - द॒शः । \newline
59. वज्रो॒ भ्रातृ॑व्याभिभूत्यै॒ भ्रातृ॑व्याभिभूत्यै॒ वज्रो॒ वज्रो॒ भ्रातृ॑व्याभिभूत्यै । \newline
60. भ्रातृ॑व्याभिभूत्यै स॒प्तद॑शारत्निꣳ स॒प्तद॑शारत्नि॒म् भ्रातृ॑व्याभिभूत्यै॒ भ्रातृ॑व्याभिभूत्यै स॒प्तद॑शारत्निम् । \newline
61. भ्रातृ॑व्याभिभूत्या॒ इति॒ भ्रातृ॑व्य - अ॒भि॒भू॒त्यै॒ । \newline
62. स॒प्तद॑शारत्निम् प्र॒जाका॑मस्य प्र॒जाका॑मस्य स॒प्तद॑शारत्निꣳ स॒प्तद॑शारत्निम् प्र॒जाका॑मस्य । \newline
63. स॒प्तद॑शारत्नि॒मिति॑ स॒प्तद॑श - अ॒र॒त्नि॒म् । \newline
64. प्र॒जाका॑मस्य सप्तद॒शः स॑प्तद॒शः प्र॒जाका॑मस्य प्र॒जाका॑मस्य सप्तद॒शः । \newline
65. प्र॒जाका॑म॒स्येति॑ प्र॒जा - का॒म॒स्य॒ । \newline
66. स॒प्त॒द॒शः प्र॒जाप॑तिः प्र॒जाप॑तिः सप्तद॒शः स॑प्तद॒शः प्र॒जाप॑तिः । \newline
67. स॒प्त॒द॒श इति॑ सप्त - द॒शः । \newline
68. प्र॒जाप॑तिः प्र॒जाप॑तेः प्र॒जाप॑तेः प्र॒जाप॑तिः प्र॒जाप॑तिः प्र॒जाप॑तेः । \newline
69. प्र॒जाप॑ति॒रिति॑ प्र॒जा - प॒तिः॒ । \newline
70. प्र॒जाप॑ते॒ राप्त्या॒ आप्त्यै᳚ प्र॒जाप॑तेः प्र॒जाप॑ते॒ राप्त्यै᳚ । \newline
71. प्र॒जाप॑ते॒रिति॑ प्र॒जा - प॒तेः॒ । \newline
72. आप्त्या॒ एक॑विꣳशत्यरत्नि॒ मेक॑विꣳशत्यरत्नि॒ माप्त्या॒ आप्त्या॒ एक॑विꣳशत्यरत्निम् । \newline
73. एक॑विꣳशत्यरत्निम् प्रति॒ष्ठाका॑मस्य प्रति॒ष्ठाका॑म॒ स्यैक॑विꣳशत्यरत्नि॒ मेक॑विꣳशत्यरत्निम् प्रति॒ष्ठाका॑मस्य । \newline
74. एक॑विꣳशत्यरत्नि॒मित्येक॑विꣳशति - अ॒र॒त्नि॒म् । \newline
75. प्र॒ति॒ष्ठाका॑म स्यैकविꣳ॒॒श ए॑कविꣳ॒॒शः प्र॑ति॒ष्ठाका॑मस्य प्रति॒ष्ठाका॑म स्यैकविꣳ॒॒शः । \newline
76. प्र॒ति॒ष्ठाका॑म॒स्येति॑ प्रति॒ष्ठा - का॒म॒स्य॒ । \newline
77. ए॒क॒विꣳ॒॒शः स्तोमा॑नाꣳ॒॒ स्तोमा॑ना मेकविꣳ॒॒श ए॑कविꣳ॒॒शः स्तोमा॑नाम् । \newline
78. ए॒क॒विꣳ॒॒श इत्ये॑क - विꣳ॒॒शः । \newline
79. स्तोमा॑नाम् प्रति॒ष्ठा प्र॑ति॒ष्ठा स्तोमा॑नाꣳ॒॒ स्तोमा॑नाम् प्रति॒ष्ठा । \newline
80. प्र॒ति॒ष्ठा प्रति॑ष्ठित्यै॒ प्रति॑ष्ठित्यै प्रति॒ष्ठा प्र॑ति॒ष्ठा प्रति॑ष्ठित्यै । \newline
81. प्र॒ति॒ष्ठेति॑ प्रति - स्था । \newline
82. प्रति॑ष्ठित्या अ॒ष्टाश्रि॑ र॒ष्टाश्रिः॒ प्रति॑ष्ठित्यै॒ प्रति॑ष्ठित्या अ॒ष्टाश्रिः॑ । \newline
83. प्रति॑ष्ठित्या॒ इति॒ प्रति॑ - स्थि॒त्यै॒ । \newline
84. अ॒ष्टाश्रि॑र् भवति भव त्य॒ष्टाश्रि॑ र॒ष्टाश्रि॑र् भवति । \newline
85. अ॒ष्टाश्रि॒रित्य॒ष्टा - अ॒श्रिः॒ । \newline
86. भ॒व॒ त्य॒ष्टाक्ष॑रा॒ ऽष्टाक्ष॑रा भवति भव त्य॒ष्टाक्ष॑रा । \newline
87. अ॒ष्टाक्ष॑रा गाय॒त्री गा॑य॒ त्र्य॑ष्टाक्ष॑रा॒ ऽष्टाक्ष॑रा गाय॒त्री । \newline
88. अ॒ष्टाक्ष॒रेत्य॒ष्टा - अ॒क्ष॒रा॒ । \newline
89. गा॒य॒त्री तेज॒ स्तेजो॑ गाय॒त्री गा॑य॒त्री तेजः॑ । \newline
90. तेजो॑ गाय॒त्री गा॑य॒त्री तेज॒ स्तेजो॑ गाय॒त्री । \newline
91. गा॒य॒त्री गा॑य॒त्री । \newline
92. गा॒य॒त्री य॑ज्ञ्मु॒खं ॅय॑ज्ञ्मु॒खम् गा॑य॒त्री गा॑य॒त्री य॑ज्ञ्मु॒खम् । \newline
93. य॒ज्ञ्॒मु॒खम् तेज॑सा॒ तेज॑सा यज्ञ्मु॒खं ॅय॑ज्ञ्मु॒खम् तेज॑सा । \newline
94. य॒ज्ञ्॒मु॒खमिति॑ यज्ञ् - मु॒खम् । \newline
95. तेज॑सै॒वैव तेज॑सा॒ तेज॑सै॒व । \newline
96. ए॒व गा॑यत्रि॒या गा॑यत्रि॒यैवैव गा॑यत्रि॒या । \newline
97. गा॒य॒त्रि॒या य॑ज्ञ्मु॒खेन॑ यज्ञ्मु॒खेन॑ गायत्रि॒या गा॑यत्रि॒या य॑ज्ञ्मु॒खेन॑ । \newline
98. य॒ज्ञ्॒मु॒खेन॒ सम्मि॑तः॒ सम्मि॑तो यज्ञ्मु॒खेन॑ यज्ञ्मु॒खेन॒ सम्मि॑तः । \newline
99. य॒ज्ञ्॒मु॒खेनेति॑ यज्ञ् - मु॒खेन॑ । \newline
100. सम्मि॑त॒ इति॒ सं - मि॒तः॒ । \newline

\textbf{Ghana Paata } \newline

1. न॒म॒ति॒ षड॑रत्निꣳ॒॒ षड॑रत्निम् नमति नमति॒ षड॑रत्निम् प्रति॒ष्ठाका॑मस्य प्रति॒ष्ठाका॑मस्य॒ षड॑रत्निम् नमति नमति॒ षड॑रत्निम् प्रति॒ष्ठाका॑मस्य । \newline
2. षड॑रत्निम् प्रति॒ष्ठाका॑मस्य प्रति॒ष्ठाका॑मस्य॒ षड॑रत्निꣳ॒॒ षड॑रत्निम् प्रति॒ष्ठाका॑मस्य॒ षट् थ्षट् प्र॑ति॒ष्ठाका॑मस्य॒ षड॑रत्निꣳ॒॒ षड॑रत्निम् प्रति॒ष्ठाका॑मस्य॒ षट् । \newline
3. षड॑रत्नि॒मिति॒ षट् - अ॒र॒त्नि॒म् । \newline
4. प्र॒ति॒ष्ठाका॑मस्य॒ षट् थ्षट् प्र॑ति॒ष्ठाका॑मस्य प्रति॒ष्ठाका॑मस्य॒ षड् वै वै षट् प्र॑ति॒ष्ठाका॑मस्य प्रति॒ष्ठाका॑मस्य॒ षड् वै । \newline
5. प्र॒ति॒ष्ठाका॑म॒स्येति॑ प्रति॒ष्ठा - का॒म॒स्य॒ । \newline
6. षड् वै वै षट् थ्षड् वा ऋ॒तव॑ ऋ॒तवो॒ वै षट् थ्षड् वा ऋ॒तवः॑ । \newline
7. वा ऋ॒तव॑ ऋ॒तवो॒ वै वा ऋ॒तव॑ ऋ॒तुष् वृ॒तुष् वृ॒तवो॒ वै वा ऋ॒तव॑ ऋ॒तुषु॑ । \newline
8. ऋ॒तव॑ ऋ॒तुष् वृ॒तु ष्वृ॒तव॑ ऋ॒तव॑ ऋ॒तु ष्वे॒वैव र्‌तुष् वृ॒तव॑ ऋ॒तव॑ ऋ॒तु ष्वे॒व । \newline
9. ऋ॒तुष्वे॒ वैव र्‌तुष् वृ॒तु ष्वे॒व प्रति॒ प्रत्ये॒व र्‌तुष् वृ॒तु ष्वे॒व प्रति॑ । \newline
10. ए॒व प्रति॒ प्रत्ये॒ वैव प्रति॑ तिष्ठति तिष्ठति॒ प्रत्ये॒ वैव प्रति॑ तिष्ठति । \newline
11. प्रति॑ तिष्ठति तिष्ठति॒ प्रति॒ प्रति॑ तिष्ठति स॒प्तार॑त्निꣳ स॒प्तार॑त्निम् तिष्ठति॒ प्रति॒ प्रति॑ तिष्ठति स॒प्तार॑त्निम् । \newline
12. ति॒ष्ठ॒ति॒ स॒प्तार॑त्निꣳ स॒प्तार॑त्निम् तिष्ठति तिष्ठति स॒प्तार॑त्निम् प॒शुका॑मस्य प॒शुका॑मस्य स॒प्तार॑त्निम् तिष्ठति तिष्ठति स॒प्तार॑त्निम् प॒शुका॑मस्य । \newline
13. स॒प्तार॑त्निम् प॒शुका॑मस्य प॒शुका॑मस्य स॒प्तार॑त्निꣳ स॒प्तार॑त्निम् प॒शुका॑मस्य स॒प्तप॑दा स॒प्तप॑दा प॒शुका॑मस्य स॒प्तार॑त्निꣳ स॒प्तार॑त्निम् प॒शुका॑मस्य स॒प्तप॑दा । \newline
14. स॒प्तार॑त्नि॒मिति॑ स॒प्त - अ॒र॒त्नि॒म् । \newline
15. प॒शुका॑मस्य स॒प्तप॑दा स॒प्तप॑दा प॒शुका॑मस्य प॒शुका॑मस्य स॒प्तप॑दा॒ शक्व॑री॒ शक्व॑री स॒प्तप॑दा प॒शुका॑मस्य प॒शुका॑मस्य स॒प्तप॑दा॒ शक्व॑री । \newline
16. प॒शुका॑म॒स्येति॑ प॒शु - का॒म॒स्य॒ । \newline
17. स॒प्तप॑दा॒ शक्व॑री॒ शक्व॑री स॒प्तप॑दा स॒प्तप॑दा॒ शक्व॑री प॒शवः॑ प॒शवः॒ शक्व॑री स॒प्तप॑दा स॒प्तप॑दा॒ शक्व॑री प॒शवः॑ । \newline
18. स॒प्तप॒देति॑ स॒प्त - प॒दा॒ । \newline
19. शक्व॑री प॒शवः॑ प॒शवः॒ शक्व॑री॒ शक्व॑री प॒शवः॒ शक्व॑री॒ शक्व॑री प॒शवः॒ शक्व॑री॒ शक्व॑री प॒शवः॒ शक्व॑री । \newline
20. प॒शवः॒ शक्व॑री॒ शक्व॑री प॒शवः॑ प॒शवः॒ शक्व॑री प॒शून् प॒शूञ् छक्व॑री प॒शवः॑ प॒शवः॒ शक्व॑री प॒शून् । \newline
21. शक्व॑री प॒शून् प॒शूञ् छक्व॑री॒ शक्व॑री प॒शूने॒ वैव प॒शूञ् छक्व॑री॒ शक्व॑री प॒शूने॒व । \newline
22. प॒शूने॒ वैव प॒शून् प॒शूने॒ वावा वै॒व प॒शून् प॒शूने॒ वाव॑ । \newline
23. ए॒वावा वै॒वै वाव॑ रुन्धे रु॒न्धे ऽवै॒वै वाव॑ रुन्धे । \newline
24. अव॑ रुन्धे रु॒न्धे ऽवाव॑ रुन्धे॒ नवा॑रत्नि॒म् नवा॑रत्निꣳ रु॒न्धे ऽवाव॑ रुन्धे॒ नवा॑रत्निम् । \newline
25. रु॒न्धे॒ नवा॑रत्नि॒म् नवा॑रत्निꣳ रुन्धे रुन्धे॒ नवा॑रत्नि॒म् तेज॑स्कामस्य॒ तेज॑स्कामस्य॒ नवा॑रत्निꣳ रुन्धे रुन्धे॒ नवा॑रत्नि॒म् तेज॑स्कामस्य । \newline
26. नवा॑रत्नि॒म् तेज॑स्कामस्य॒ तेज॑स्कामस्य॒ नवा॑रत्नि॒म् नवा॑रत्नि॒म् तेज॑स्कामस्य त्रि॒वृता᳚ त्रि॒वृता॒ तेज॑स्कामस्य॒ नवा॑रत्नि॒म् नवा॑रत्नि॒म् तेज॑स्कामस्य त्रि॒वृता᳚ । \newline
27. नवा॑रत्नि॒मिति॒ नव॑ - अ॒र॒त्नि॒म् । \newline
28. तेज॑स्कामस्य त्रि॒वृता᳚ त्रि॒वृता॒ तेज॑स्कामस्य॒ तेज॑स्कामस्य त्रि॒वृता॒ स्तोमे॑न॒ स्तोमे॑न त्रि॒वृता॒ तेज॑स्कामस्य॒ तेज॑स्कामस्य त्रि॒वृता॒ स्तोमे॑न । \newline
29. तेज॑स्काम॒स्येति॒ तेजः॑ - का॒म॒स्य॒ । \newline
30. त्रि॒वृता॒ स्तोमे॑न॒ स्तोमे॑न त्रि॒वृता᳚ त्रि॒वृता॒ स्तोमे॑न॒ सम्मि॑तꣳ॒॒ सम्मि॑तꣳ॒॒ स्तोमे॑न त्रि॒वृता᳚ त्रि॒वृता॒ स्तोमे॑न॒ सम्मि॑तम् । \newline
31. त्रि॒वृतेति॑ त्रि - वृता᳚ । \newline
32. स्तोमे॑न॒ सम्मि॑तꣳ॒॒ सम्मि॑तꣳ॒॒ स्तोमे॑न॒ स्तोमे॑न॒ सम्मि॑त॒म् तेज॒ स्तेजः॒ सम्मि॑तꣳ॒॒ स्तोमे॑न॒ स्तोमे॑न॒ सम्मि॑त॒म् तेजः॑ । \newline
33. सम्मि॑त॒म् तेज॒ स्तेजः॒ सम्मि॑तꣳ॒॒ सम्मि॑त॒म् तेज॑ स्त्रि॒वृत् त्रि॒वृत् तेजः॒ सम्मि॑तꣳ॒॒ सम्मि॑त॒म् तेज॑ स्त्रि॒वृत् । \newline
34. सम्मि॑त॒मिति॒ सं - मि॒त॒म् । \newline
35. तेज॑ स्त्रि॒वृत् त्रि॒वृत् तेज॒ स्तेज॑ स्त्रि॒वृत् ते॑ज॒स्वी ते॑ज॒स्वी त्रि॒वृत् तेज॒ स्तेज॑ स्त्रि॒वृत् ते॑ज॒स्वी । \newline
36. त्रि॒वृत् ते॑ज॒स्वी ते॑ज॒स्वी त्रि॒वृत् त्रि॒वृत् ते॑ज॒स्व्ये॑ वैव ते॑ज॒स्वी त्रि॒वृत् त्रि॒वृत् ते॑ज॒ स्व्ये॑व । \newline
37. त्रि॒वृदिति॑ त्रि - वृत् । \newline
38. ते॒ज॒स्व्ये॑ वैव ते॑ज॒स्वी ते॑ज॒ स्व्ये॑व भ॑वति भव त्ये॒व ते॑ज॒स्वी ते॑ज॒ स्व्ये॑व भ॑वति । \newline
39. ए॒व भ॑वति भव त्ये॒वैव भ॑व॒ त्येका॑दशारत्नि॒ मेका॑दशारत्निम् भव त्ये॒वैव भ॑व॒ त्येका॑दशारत्निम् । \newline
40. भ॒व॒ त्येका॑दशारत्नि॒ मेका॑दशारत्निम् भवति भव॒ त्येका॑दशारत्नि मिन्द्रि॒यका॑म स्येन्द्रि॒यका॑म॒
स्यैका॑दशारत्निम् भवति भव॒ त्येका॑दशारत्नि मिन्द्रि॒यका॑मस्य । \newline
41. एका॑दशारत्नि मिन्द्रि॒यका॑म स्येन्द्रि॒यका॑म॒ स्यैका॑दशारत्नि॒ मेका॑दशारत्नि मिन्द्रि॒यका॑म॒
स्यैका॑दशाक्ष॒ रैका॑दशाक्ष रेन्द्रि॒यका॑म॒ स्यैका॑दशारत्नि॒ मेका॑दशारत्नि मिन्द्रि॒यका॑म॒
स्यैका॑दशाक्षरा । \newline
42. एका॑दशारत्नि॒मित्येका॑दश - अ॒र॒त्नि॒म् । \newline
43. इ॒न्द्रि॒यका॑म॒ स्यैका॑दशाक्ष॒ रैका॑दशाक्ष रेन्द्रि॒यका॑म स्येन्द्रि॒यका॑म॒ स्यैका॑दशाक्षरा त्रि॒ष्टुक् त्रि॒ष्टु गेका॑दशाक्ष रेन्द्रि॒यका॑म स्येन्द्रि॒यका॑म॒ स्यैका॑दशाक्षरा त्रि॒ष्टुक् । \newline
44. इ॒न्द्रि॒यका॑म॒स्येती᳚न्द्रि॒य - का॒म॒स्य॒ । \newline
45. एका॑दशाक्षरा त्रि॒ष्टुक् त्रि॒ष्टु गेका॑दशाक्ष॒ रैका॑दशाक्षरा त्रि॒ष्टु गि॑न्द्रि॒य मि॑न्द्रि॒यम् त्रि॒ष्टु गेका॑दशाक्ष॒ रैका॑दशाक्षरा त्रि॒ष्टु गि॑न्द्रि॒यम् । \newline
46. एका॑दशाक्ष॒रेत्येका॑दश - अ॒क्ष॒रा॒ । \newline
47. त्रि॒ष्टु गि॑न्द्रि॒य मि॑न्द्रि॒यम् त्रि॒ष्टुक् त्रि॒ष्टु गि॑न्द्रि॒यम् त्रि॒ष्टुक् त्रि॒ष्टु गि॑न्द्रि॒यम् त्रि॒ष्टुक् त्रि॒ष्टु
गि॑न्द्रि॒यम् त्रि॒ष्टुक् । \newline
48. इ॒न्द्रि॒यम् त्रि॒ष्टुक् त्रि॒ष्टु गि॑न्द्रि॒य मि॑न्द्रि॒यम् त्रि॒ष्टु गि॑न्द्रिया॒वी न्द्रि॑या॒वी त्रि॒ष्टु गि॑न्द्रि॒य मि॑न्द्रि॒यम् त्रि॒ष्टु गि॑न्द्रिया॒वी । \newline
49. त्रि॒ष्टु गि॑न्द्रिया॒वी न्द्रि॑या॒वी त्रि॒ष्टुक् त्रि॒ष्टु गि॑न्द्रिया॒व्ये॑ वैवेन्द्रि॑या॒वी त्रि॒ष्टुक् त्रि॒ष्टुगि॑न्द्रिया॒ व्ये॑व । \newline
50. इ॒न्द्रि॒या॒व्ये॑ वैवेन्द्रि॑या॒ वीन्द्रि॑या॒ व्ये॑व भ॑वति भव त्ये॒वेन्द्रि॑या॒ वीन्द्रि॑या॒ व्ये॑व भ॑वति । \newline
51. ए॒व भ॑वति भव त्ये॒वैव भ॑वति॒ पञ्च॑दशारत्नि॒म् पञ्च॑दशारत्निम् भव त्ये॒वैव भ॑वति॒ पञ्च॑दशारत्निम् । \newline
52. भ॒व॒ति॒ पञ्च॑दशारत्नि॒म् पञ्च॑दशारत्निम् भवति भवति॒ पञ्च॑दशारत्नि॒म् भ्रातृ॑व्यवतो॒ भ्रातृ॑व्यवतः॒ पञ्च॑दशारत्निम् भवति भवति॒ पञ्च॑दशारत्नि॒म् भ्रातृ॑व्यवतः । \newline
53. पञ्च॑दशारत्नि॒म् भ्रातृ॑व्यवतो॒ भ्रातृ॑व्यवतः॒ पञ्च॑दशारत्नि॒म् पञ्च॑दशारत्नि॒म् भ्रातृ॑व्यवतः पञ्चद॒शः प॑ञ्चद॒शो भ्रातृ॑व्यवतः॒ पञ्च॑दशारत्नि॒म् पञ्च॑दशारत्नि॒म् भ्रातृ॑व्यवतः पञ्चद॒शः । \newline
54. पञ्च॑दशारत्नि॒मिति॒ पञ्च॑दश - अ॒र॒त्नि॒म् । \newline
55. भ्रातृ॑व्यवतः पञ्चद॒शः प॑ञ्चद॒शो भ्रातृ॑व्यवतो॒ भ्रातृ॑व्यवतः पञ्चद॒शो वज्रो॒ वज्रः॑ पञ्चद॒शो भ्रातृ॑व्यवतो॒ भ्रातृ॑व्यवतः पञ्चद॒शो वज्रः॑ । \newline
56. भ्रातृ॑व्यवत॒ इति॒ भ्रातृ॑व्य - व॒तः॒ । \newline
57. प॒ञ्च॒द॒शो वज्रो॒ वज्रः॑ पञ्चद॒शः प॑ञ्चद॒शो वज्रो॒ भ्रातृ॑व्याभिभूत्यै॒ भ्रातृ॑व्याभिभूत्यै॒ वज्रः॑ पञ्चद॒शः प॑ञ्चद॒शो वज्रो॒ भ्रातृ॑व्याभिभूत्यै । \newline
58. प॒ञ्च॒द॒श इति॑ पञ्च - द॒शः । \newline
59. वज्रो॒ भ्रातृ॑व्याभिभूत्यै॒ भ्रातृ॑व्याभिभूत्यै॒ वज्रो॒ वज्रो॒ भ्रातृ॑व्याभिभूत्यै स॒प्तद॑शारत्निꣳ स॒प्तद॑शारत्नि॒म् भ्रातृ॑व्याभिभूत्यै॒ वज्रो॒ वज्रो॒ भ्रातृ॑व्याभिभूत्यै स॒प्तद॑शारत्निम् । \newline
60. भ्रातृ॑व्याभिभूत्यै स॒प्तद॑शारत्निꣳ स॒प्तद॑शारत्नि॒म् भ्रातृ॑व्याभिभूत्यै॒ भ्रातृ॑व्याभिभूत्यै स॒प्तद॑शारत्निम् प्र॒जाका॑मस्य प्र॒जाका॑मस्य स॒प्तद॑शारत्नि॒म् भ्रातृ॑व्याभिभूत्यै॒ भ्रातृ॑व्याभिभूत्यै स॒प्तद॑शारत्निम् प्र॒जाका॑मस्य । \newline
61. भ्रातृ॑व्याभिभूत्या॒ इति॒ भ्रातृ॑व्य - अ॒भि॒भू॒त्यै॒ । \newline
62. स॒प्तद॑शारत्निम् प्र॒जाका॑मस्य प्र॒जाका॑मस्य स॒प्तद॑शारत्निꣳ स॒प्तद॑शारत्निम् प्र॒जाका॑मस्य सप्तद॒शः स॑प्तद॒शः प्र॒जाका॑मस्य स॒प्तद॑शारत्निꣳ स॒प्तद॑शारत्निम् प्र॒जाका॑मस्य सप्तद॒शः । \newline
63. स॒प्तद॑शारत्नि॒मिति॑ स॒प्तद॑श - अ॒र॒त्नि॒म् । \newline
64. प्र॒जाका॑मस्य सप्तद॒शः स॑प्तद॒शः प्र॒जाका॑मस्य प्र॒जाका॑मस्य सप्तद॒शः प्र॒जाप॑तिः प्र॒जाप॑तिः सप्तद॒शः प्र॒जाका॑मस्य प्र॒जाका॑मस्य सप्तद॒शः प्र॒जाप॑तिः । \newline
65. प्र॒जाका॑म॒स्येति॑ प्र॒जा - का॒म॒स्य॒ । \newline
66. स॒प्त॒द॒शः प्र॒जाप॑तिः प्र॒जाप॑तिः सप्तद॒शः स॑प्तद॒शः प्र॒जाप॑तिः प्र॒जाप॑तेः प्र॒जाप॑तेः प्र॒जाप॑तिः सप्तद॒शः स॑प्तद॒शः प्र॒जाप॑तिः प्र॒जाप॑तेः । \newline
67. स॒प्त॒द॒श इति॑ सप्त - द॒शः । \newline
68. प्र॒जाप॑तिः प्र॒जाप॑तेः प्र॒जाप॑तेः प्र॒जाप॑तिः प्र॒जाप॑तिः प्र॒जाप॑ते॒ राप्त्या॒ आप्त्यै᳚ प्र॒जाप॑तेः प्र॒जाप॑तिः प्र॒जाप॑तिः प्र॒जाप॑ते॒ राप्त्यै᳚ । \newline
69. प्र॒जाप॑ति॒रिति॑ प्र॒जा - प॒तिः॒ । \newline
70. प्र॒जाप॑ते॒ राप्त्या॒ आप्त्यै᳚ प्र॒जाप॑तेः प्र॒जाप॑ते॒ राप्त्या॒ एक॑विꣳशत्यरत्नि॒ मेक॑विꣳशत्यरत्नि॒ माप्त्यै᳚ प्र॒जाप॑तेः प्र॒जाप॑ते॒ राप्त्या॒ एक॑विꣳशत्यरत्निम् । \newline
71. प्र॒जाप॑ते॒रिति॑ प्र॒जा - प॒तेः॒ । \newline
72. आप्त्या॒ एक॑विꣳशत्यरत्नि॒ मेक॑विꣳशत्यरत्नि॒ माप्त्या॒ आप्त्या॒ एक॑विꣳशत्यरत्निम् प्रति॒ष्ठाका॑मस्य प्रति॒ष्ठाका॑म॒ स्यैक॑विꣳशत्यरत्नि॒ माप्त्या॒ आप्त्या॒ एक॑विꣳशत्यरत्निम् प्रति॒ष्ठाका॑मस्य । \newline
73. एक॑विꣳशत्यरत्निम् प्रति॒ष्ठाका॑मस्य प्रति॒ष्ठाका॑म॒ स्यैक॑विꣳशत्यरत्नि॒ मेक॑विꣳशत्यरत्निम् प्रति॒ष्ठाका॑म स्यैकविꣳ॒॒श ए॑कविꣳ॒॒शः प्र॑ति॒ष्ठाका॑म॒ स्यैक॑विꣳशत्यरत्नि॒ मेक॑विꣳशत्यरत्निम् प्रति॒ष्ठाका॑म स्यैकविꣳ॒॒शः । \newline
74. एक॑विꣳशत्यरत्नि॒मित्येक॑विꣳशति - अ॒र॒त्नि॒म् । \newline
75. प्र॒ति॒ष्ठाका॑म स्यैकविꣳ॒॒श ए॑कविꣳ॒॒शः प्र॑ति॒ष्ठाका॑मस्य प्रति॒ष्ठाका॑म स्यैकविꣳ॒॒शः स्तोमा॑नाꣳ॒॒ स्तोमा॑ना मेकविꣳ॒॒शः प्र॑ति॒ष्ठाका॑मस्य प्रति॒ष्ठाका॑म स्यैकविꣳ॒॒शः स्तोमा॑नाम् । \newline
76. प्र॒ति॒ष्ठाका॑म॒स्येति॑ प्रति॒ष्ठा - का॒म॒स्य॒ । \newline
77. ए॒क॒विꣳ॒॒शः स्तोमा॑नाꣳ॒॒ स्तोमा॑ना मेकविꣳ॒॒श ए॑कविꣳ॒॒शः स्तोमा॑नाम् प्रति॒ष्ठा प्र॑ति॒ष्ठा स्तोमा॑ना मेकविꣳ॒॒श ए॑कविꣳ॒॒शः स्तोमा॑नाम् प्रति॒ष्ठा । \newline
78. ए॒क॒विꣳ॒॒श इत्ये॑क - विꣳ॒॒शः । \newline
79. स्तोमा॑नाम् प्रति॒ष्ठा प्र॑ति॒ष्ठा स्तोमा॑नाꣳ॒॒ स्तोमा॑नाम् प्रति॒ष्ठा प्रति॑ष्ठित्यै॒ प्रति॑ष्ठित्यै प्रति॒ष्ठा स्तोमा॑नाꣳ॒॒ स्तोमा॑नाम् प्रति॒ष्ठा प्रति॑ष्ठित्यै । \newline
80. प्र॒ति॒ष्ठा प्रति॑ष्ठित्यै॒ प्रति॑ष्ठित्यै प्रति॒ष्ठा प्र॑ति॒ष्ठा प्रति॑ष्ठित्या अ॒ष्टाश्रि॑ र॒ष्टाश्रिः॒ प्रति॑ष्ठित्यै प्रति॒ष्ठा प्र॑ति॒ष्ठा प्रति॑ष्ठित्या अ॒ष्टाश्रिः॑ । \newline
81. प्र॒ति॒ष्ठेति॑ प्रति - स्था । \newline
82. प्रति॑ष्ठित्या अ॒ष्टाश्रि॑ र॒ष्टाश्रिः॒ प्रति॑ष्ठित्यै॒ प्रति॑ष्ठित्या अ॒ष्टाश्रि॑र् भवति भव त्य॒ष्टाश्रिः॒ प्रति॑ष्ठित्यै॒ प्रति॑ष्ठित्या अ॒ष्टाश्रि॑र् भवति । \newline
83. प्रति॑ष्ठित्या॒ इति॒ प्रति॑ - स्थि॒त्यै॒ । \newline
84. अ॒ष्टाश्रि॑र् भवति भव त्य॒ष्टाश्रि॑ र॒ष्टाश्रि॑र् भव त्य॒ष्टाक्ष॑रा॒ ऽष्टाक्ष॑रा भव त्य॒ष्टाश्रि॑ र॒ष्टाश्रि॑र् भव त्य॒ष्टाक्ष॑रा । \newline
85. अ॒ष्टाश्रि॒रित्य॒ष्टा - अ॒श्रिः॒ । \newline
86. भ॒व॒ त्य॒ष्टाक्ष॑रा॒ ऽष्टाक्ष॑रा भवति भव त्य॒ष्टाक्ष॑रा गाय॒त्री गा॑य॒ त्र्य॑ष्टाक्ष॑रा भवति भव त्य॒ष्टाक्ष॑रा गाय॒त्री । \newline
87. अ॒ष्टाक्ष॑रा गाय॒त्री गा॑य॒ त्र्य॑ष्टाक्ष॑रा॒ ऽष्टाक्ष॑रा गाय॒त्री तेज॒ स्तेजो॑ गाय॒ त्र्य॑ष्टाक्ष॑रा॒ ऽष्टाक्ष॑रा गाय॒त्री तेजः॑ । \newline
88. अ॒ष्टाक्ष॒रेत्य॒ष्टा - अ॒क्ष॒रा॒ । \newline
89. गा॒य॒त्री तेज॒ स्तेजो॑ गाय॒त्री गा॑य॒त्री तेजो॑ गाय॒त्री गा॑य॒त्री तेजो॑ गाय॒त्री गा॑य॒त्री तेजो॑ गाय॒त्री । \newline
90. तेजो॑ गाय॒त्री गा॑य॒त्री तेज॒ स्तेजो॑ गाय॒त्री । \newline
91. गा॒य॒त्री गा॑य॒त्री । \newline
92. गा॒य॒त्री य॑ज्ञ्मु॒खं ॅय॑ज्ञ्मु॒खम् गा॑य॒त्री गा॑य॒त्री य॑ज्ञ्मु॒खम् तेज॑सा॒ तेज॑सा यज्ञ्मु॒खम् गा॑य॒त्री गा॑य॒त्री य॑ज्ञ्मु॒खम् तेज॑सा । \newline
93. य॒ज्ञ्॒मु॒खम् तेज॑सा॒ तेज॑सा यज्ञ्मु॒खं ॅय॑ज्ञ्मु॒खम् तेज॑सै॒वैव तेज॑सा यज्ञ्मु॒खं ॅय॑ज्ञ्मु॒खम् तेज॑सै॒व । \newline
94. य॒ज्ञ्॒मु॒खमिति॑ यज्ञ् - मु॒खम् । \newline
95. तेज॑सै॒वैव तेज॑सा॒ तेज॑सै॒व गा॑यत्रि॒या गा॑यत्रि॒यैव तेज॑सा॒ तेज॑सै॒व गा॑यत्रि॒या । \newline
96. ए॒व गा॑यत्रि॒या गा॑यत्रि॒यै वैव गा॑यत्रि॒या य॑ज्ञ्मु॒खेन॑ यज्ञ्मु॒खेन॑ गायत्रि॒यै वैव गा॑यत्रि॒या य॑ज्ञ्मु॒खेन॑ । \newline
97. गा॒य॒त्रि॒या य॑ज्ञ्मु॒खेन॑ यज्ञ्मु॒खेन॑ गायत्रि॒या गा॑यत्रि॒या य॑ज्ञ्मु॒खेन॒ सम्मि॑तः॒ सम्मि॑तो यज्ञ्मु॒खेन॑ गायत्रि॒या गा॑यत्रि॒या य॑ज्ञ्मु॒खेन॒ सम्मि॑तः । \newline
98. य॒ज्ञ्॒मु॒खेन॒ सम्मि॑तः॒ सम्मि॑तो यज्ञ्मु॒खेन॑ यज्ञ्मु॒खेन॒ सम्मि॑तः । \newline
99. य॒ज्ञ्॒मु॒खेनेति॑ यज्ञ् - मु॒खेन॑ । \newline
100. सम्मि॑त॒ इति॒ सं - मि॒तः॒ । \newline
\pagebreak
\markright{ TS 6.3.4.1  \hfill https://www.vedavms.in \hfill}

\section{ TS 6.3.4.1 }

\textbf{TS 6.3.4.1 } \newline
\textbf{Samhita Paata} \newline

पृ॒थि॒व्यै त्वा॒ऽन्तरि॑क्षाय त्वा दि॒वे त्वेत्या॑है॒भ्य ए॒वैनं॑ ॅलो॒केभ्यः॒ प्रोक्ष॑ति॒ परा᳚ञ्चं॒ प्रोक्ष॑ति॒ परा॑ङिव॒ हि सु॑व॒र्गो लो॒कः क्रू॒रमि॑व॒ वा ए॒तत् क॑रोति॒ यत् खन॑त्य॒पोऽव॑ नयति॒ शान्त्यै॒ यव॑मती॒रव॑ नय॒त्यूर्ग्वै यवो॒ यज॑मानेन॒ यूपः॒ संमि॑तो॒ यावा॑ने॒व यज॑मान॒-स्ताव॑ती-मे॒वास्मि॒-न्नूर्जं॑ दधाति- [  ] \newline

\textbf{Pada Paata} \newline

पृ॒थि॒व्यै । त्वा॒ । अ॒न्तरि॑क्षाय । त्वा॒ । दि॒वे । त्वा॒ । इति॑ । आ॒ह॒ । ए॒भ्यः । ए॒व । ए॒न॒म् । लो॒केभ्यः॑ । प्रेति॑ । उ॒क्ष॒ति॒ । परा᳚ञ्चम् । प्रेति॑ । उ॒क्ष॒ति॒ । पराङ्॑ । इ॒व॒ । हि । सु॒व॒र्ग इति॑ सुवः-गः । लो॒कः । क्रू॒रम् । इ॒व॒ । वै । ए॒तत् । क॒रो॒ति॒ । यत् । खन॑ति । अ॒पः । अवेति॑ । न॒य॒ति॒ । शान्त्यै᳚ । यव॑मती॒रिति॒ यव॑-म॒तीः॒ । अवेति॑ । न॒य॒ति॒ । ऊर्क् । वै । यवः॑ । यज॑मानेन । यूपः॑ । संमि॑त॒ इति॒ सं - मि॒तः॒ । यावान्॑ । ए॒व । यज॑मानः । ताव॑तीम् । ए॒व । अ॒स्मि॒न्न् । ऊर्ज᳚म् । द॒धा॒ति॒ ।  \newline


\textbf{Krama Paata} \newline

पृ॒थि॒व्यै त्वा᳚ । त्वा॒ऽन्तरि॑क्षाय । अ॒न्तरि॑क्षाय त्वा । त्वा॒ दि॒वे । दि॒वे त्वा᳚ । त्वेति॑ । इत्या॑ह । आ॒है॒भ्यः । ए॒भ्य ए॒व । ए॒वैन᳚म् । ए॒न॒म् ॅलो॒केभ्यः॑ । लो॒केभ्यः॒ प्र । प्रोक्ष॑ति । उ॒क्ष॒ति॒ परा᳚ञ्चम् । परा᳚ञ्च॒म् प्र । प्रोक्ष॑ति । उ॒क्ष॒ति॒ पराङ्॑ । परा॑ङ्‍ङिव । इ॒व॒ हि । हि सु॑व॒र्गः । सु॒व॒र्गो लो॒कः । सु॒व॒र्ग इति॑ सुवः - गः । लो॒कः क्रू॒रम् । क्रू॒रमि॑व । इ॒व॒ वै । वा ए॒तत् । ए॒तत् क॑रोति । क॒रो॒ति॒ यत् । यत् खन॑ति । खन॑त्य॒पः । अ॒पोऽव॑ । अव॑ नयति । न॒य॒ति॒ शान्त्यै᳚ । शान्त्यै॒ यव॑मतीः । यव॑मती॒रव॑ । यव॑मती॒रिति॒ यव॑ - म॒तीः॒ । अव॑ नयति । न॒य॒त्यूर्क् । ऊर्ग् वै । वै यवः॑ । यवो॒ यज॑मानेन । यज॑मानेन॒ यूपः॑ । यूपः॒ सम्मि॑तः । सम्मि॑तो॒ यावान्॑ । सम्मि॑त॒ इति॒ सम् - मि॒तः॒ । यावा॑ने॒व । ए॒व यज॑मानः । यज॑मान॒स्ताव॑तीम् । ताव॑तीमे॒व । ए॒वास्मिन्न्॑ । अ॒स्मि॒न्नूर्ज᳚म् । ऊर्ज॑म् दधाति । द॒धा॒ति॒ पि॒तृ॒णाम् \newline

\textbf{Jatai Paata} \newline

1. पृ॒थि॒व्यै त्वा᳚ त्वा पृथि॒व्यै पृ॑थि॒व्यै त्वा᳚ । \newline
2. त्वा॒ ऽन्तरि॑क्षाया॒ न्तरि॑क्षाय त्वा त्वा॒ ऽन्तरि॑क्षाय । \newline
3. अ॒न्तरि॑क्षाय त्वा त्वा॒ ऽन्तरि॑क्षाया॒ न्तरि॑क्षाय त्वा । \newline
4. त्वा॒ दि॒वे दि॒वे त्वा᳚ त्वा दि॒वे । \newline
5. दि॒वे त्वा᳚ त्वा दि॒वे दि॒वे त्वा᳚ । \newline
6. त्वे तीति॑ त्वा॒ त्वेति॑ । \newline
7. इत्या॑हा॒हे तीत्या॑ह । \newline
8. आ॒है॒भ्य ए॒भ्य आ॑हा है॒भ्यः । \newline
9. ए॒भ्य ए॒वै वैभ्य ए॒भ्य ए॒व । \newline
10. ए॒वैन॑ मेन मे॒वै वैन᳚म् । \newline
11. ए॒न॒म् ॅलो॒केभ्यो॑ लो॒केभ्य॑ एन मेनम् ॅलो॒केभ्यः॑ । \newline
12. लो॒केभ्यः॒ प्र प्र लो॒केभ्यो॑ लो॒केभ्यः॒ प्र । \newline
13. प्रोक्ष॑ त्युक्षति॒ प्र प्रोक्ष॑ति । \newline
14. उ॒क्ष॒ति॒ परा᳚ञ्च॒म् परा᳚ञ्च मुक्ष त्युक्षति॒ परा᳚ञ्चम् । \newline
15. परा᳚ञ्च॒म् प्र प्र परा᳚ञ्च॒म् परा᳚ञ्च॒म् प्र । \newline
16. प्रोक्ष॑ त्युक्षति॒ प्र प्रोक्ष॑ति । \newline
17. उ॒क्ष॒ति॒ परा॒ङ् परा॑ ङुक्ष त्युक्षति॒ पराङ्॑ । \newline
18. परा॑ ङिवेव॒ परा॒ङ् परा॑ ङिव । \newline
19. इ॒व॒ हि हीवे॑व॒ हि । \newline
20. हि सु॑व॒र्गः सु॑व॒र्गो हि हि सु॑व॒र्गः । \newline
21. सु॒व॒र्गो लो॒को लो॒कः सु॑व॒र्गः सु॑व॒र्गो लो॒कः । \newline
22. सु॒व॒र्ग इति॑ सुवः - गः । \newline
23. लो॒कः क्रू॒रम् क्रू॒रम् ॅलो॒को लो॒कः क्रू॒रम् । \newline
24. क्रू॒र मि॑वेव क्रू॒रम् क्रू॒र मि॑व । \newline
25. इ॒व॒ वै वा इ॑वेव॒ वै । \newline
26. वा ए॒त दे॒तद् वै वा ए॒तत् । \newline
27. ए॒तत् क॑रोति करो त्ये॒त दे॒तत् क॑रोति । \newline
28. क॒रो॒ति॒ यद् यत् क॑रोति करोति॒ यत् । \newline
29. यत् खन॑ति॒ खन॑ति॒ यद् यत् खन॑ति । \newline
30. खन॑ त्य॒पो॑ ऽपः खन॑ति॒ खन॑ त्य॒पः । \newline
31. अ॒पो ऽवा वा॒पो॑ ऽपो ऽव॑ । \newline
32. अव॑ नयति नय॒ त्यवाव॑ नयति । \newline
33. न॒य॒ति॒ शान्त्यै॒ शान्त्यै॑ नयति नयति॒ शान्त्यै᳚ । \newline
34. शान्त्यै॒ यव॑मती॒र् यव॑मतीः॒ शान्त्यै॒ शान्त्यै॒ यव॑मतीः । \newline
35. यव॑मती॒ रवाव॒ यव॑मती॒र् यव॑मती॒ रव॑ । \newline
36. यव॑मती॒रिति॒ यव॑ - म॒तीः॒ । \newline
37. अव॑ नयति नय॒ त्यवाव॑ नयति । \newline
38. न॒य॒ त्यूर् गूर्ङ् न॑यति नय॒ त्यूर्क् । \newline
39. ऊर्ग् वै वा ऊर् गूर्ग् वै । \newline
40. वै यवो॒ यवो॒ वै वै यवः॑ । \newline
41. यवो॒ यज॑मानेन॒ यज॑मानेन॒ यवो॒ यवो॒ यज॑मानेन । \newline
42. यज॑मानेन॒ यूपो॒ यूपो॒ यज॑मानेन॒ यज॑मानेन॒ यूपः॑ । \newline
43. यूपः॒ सम्मि॑तः॒ सम्मि॑तो॒ यूपो॒ यूपः॒ सम्मि॑तः । \newline
44. सम्मि॑तो॒ यावा॒न्॒. यावा॒न् थ्सम्मि॑तः॒ सम्मि॑तो॒ यावान्॑ । \newline
45. सम्मि॑त॒ इति॒ सं - मि॒तः॒ । \newline
46. यावा॑ने॒ वैव यावा॒न्॒. यावा॑ने॒व । \newline
47. ए॒व यज॑मानो॒ यज॑मान ए॒वैव यज॑मानः । \newline
48. यज॑मान॒ स्ताव॑ती॒म् ताव॑तीं॒ ॅयज॑मानो॒ यज॑मान॒ स्ताव॑तीम् । \newline
49. ताव॑ती मे॒वैव ताव॑ती॒म् ताव॑ती मे॒व । \newline
50. ए॒वास्मि॑न् नस्मिन् ने॒वै वास्मिन्न्॑ । \newline
51. अ॒स्मि॒न् नूर्ज॒ मूर्ज॑ मस्मिन् नस्मि॒न् नूर्ज᳚म् । \newline
52. ऊर्ज॑म् दधाति दधा॒ त्यूर्ज॒ मूर्ज॑म् दधाति । \newline
53. द॒धा॒ति॒ पि॒तृ॒णाम् पि॑तृ॒णाम् द॑धाति दधाति पितृ॒णाम् । \newline

\textbf{Ghana Paata } \newline

1. पृ॒थि॒व्यै त्वा᳚ त्वा पृथि॒व्यै पृ॑थि॒व्यै त्वा॒ ऽन्तरि॑क्षाया॒ न्तरि॑क्षाय त्वा पृथि॒व्यै पृ॑थि॒व्यै त्वा॒ ऽन्तरि॑क्षाय । \newline
2. त्वा॒ ऽन्तरि॑क्षाया॒ न्तरि॑क्षाय त्वा त्वा॒ ऽन्तरि॑क्षाय त्वा त्वा॒ ऽन्तरि॑क्षाय त्वा त्वा॒ ऽन्तरि॑क्षाय त्वा । \newline
3. अ॒न्तरि॑क्षाय त्वा त्वा॒ ऽन्तरि॑क्षाया॒ न्तरि॑क्षाय त्वा दि॒वे दि॒वे त्वा॒ ऽन्तरि॑क्षाया॒ न्तरि॑क्षाय त्वा दि॒वे । \newline
4. त्वा॒ दि॒वे दि॒वे त्वा᳚ त्वा दि॒वे त्वा᳚ त्वा दि॒वे त्वा᳚ त्वा दि॒वे त्वा᳚ । \newline
5. दि॒वे त्वा᳚ त्वा दि॒वे दि॒वे त्वेतीति॑ त्वा दि॒वे दि॒वे त्वेति॑ । \newline
6. त्वेतीति॑ त्वा॒ त्वेत्या॑ हा॒हेति॑ त्वा॒ त्वेत्या॑ह । \newline
7. इत्या॑हा॒हे तीत्या॑ है॒भ्य ए॒भ्य आ॒हे तीत्या॑ है॒भ्यः । \newline
8. आ॒है॒भ्य ए॒भ्य आ॑हा है॒भ्य ए॒वै वैभ्य आ॑हा है॒भ्य ए॒व । \newline
9. ए॒भ्य ए॒वै वैभ्य ए॒भ्य ए॒वैन॑ मेन मे॒वैभ्य ए॒भ्य ए॒वैन᳚म् । \newline
10. ए॒वैन॑ मेन मे॒वै वैन॑म् ॅलो॒केभ्यो॑ लो॒केभ्य॑ एन मे॒वै वैन॑म् ॅलो॒केभ्यः॑ । \newline
11. ए॒न॒म् ॅलो॒केभ्यो॑ लो॒केभ्य॑ एन मेनम् ॅलो॒केभ्यः॒ प्र प्र लो॒केभ्य॑ एन मेनम् ॅलो॒केभ्यः॒ प्र । \newline
12. लो॒केभ्यः॒ प्र प्र लो॒केभ्यो॑ लो॒केभ्यः॒ प्रोक्ष॑ त्युक्षति॒ प्र लो॒केभ्यो॑ लो॒केभ्यः॒ प्रोक्ष॑ति । \newline
13. प्रोक्ष॑ त्युक्षति॒ प्र प्रोक्ष॑ति॒ परा᳚ञ्च॒म् परा᳚ञ्च मुक्षति॒ प्र प्रोक्ष॑ति॒ परा᳚ञ्चम् । \newline
14. उ॒क्ष॒ति॒ परा᳚ञ्च॒म् परा᳚ञ्च मुक्ष त्युक्षति॒ परा᳚ञ्च॒म् प्र प्र परा᳚ञ्च मुक्ष त्युक्षति॒ परा᳚ञ्च॒म् प्र । \newline
15. परा᳚ञ्च॒म् प्र प्र परा᳚ञ्च॒म् परा᳚ञ्च॒म् प्रोक्ष॑ त्युक्षति॒ प्र परा᳚ञ्च॒म् परा᳚ञ्च॒म् प्रोक्ष॑ति । \newline
16. प्रोक्ष॑ त्युक्षति॒ प्र प्रोक्ष॑ति॒ परा॒ङ् परा॑ ङुक्षति॒ प्र प्रोक्ष॑ति॒ पराङ्॑ । \newline
17. उ॒क्ष॒ति॒ परा॒ङ् परा॑ ङुक्ष त्युक्षति॒ परा॑ ङिवेव॒ परा॑ ङुक्ष त्युक्षति॒ परा॑ ङिव । \newline
18. परा॑ ङिवेव॒ परा॒ङ् परा॑ ङिव॒ हि हीव॒ परा॒ङ् परा॑ ङिव॒ हि । \newline
19. इ॒व॒ हि हीवे॑व॒ हि सु॑व॒र्गः सु॑व॒र्गो हीवे॑व॒ हि सु॑व॒र्गः । \newline
20. हि सु॑व॒र्गः सु॑व॒र्गो हि हि सु॑व॒र्गो लो॒को लो॒कः सु॑व॒र्गो हि हि सु॑व॒र्गो लो॒कः । \newline
21. सु॒व॒र्गो लो॒को लो॒कः सु॑व॒र्गः सु॑व॒र्गो लो॒कः क्रू॒रम् क्रू॒रम् ॅलो॒कः सु॑व॒र्गः सु॑व॒र्गो लो॒कः क्रू॒रम् । \newline
22. सु॒व॒र्ग इति॑ सुवः - गः । \newline
23. लो॒कः क्रू॒रम् क्रू॒रम् ॅलो॒को लो॒कः क्रू॒र मि॑वेव क्रू॒रम् ॅलो॒को लो॒कः क्रू॒र मि॑व । \newline
24. क्रू॒र मि॑वेव क्रू॒रम् क्रू॒र मि॑व॒ वै वा इ॑व क्रू॒रम् क्रू॒र मि॑व॒ वै । \newline
25. इ॒व॒ वै वा इ॑वेव॒ वा ए॒त दे॒तद् वा इ॑वेव॒ वा ए॒तत् । \newline
26. वा ए॒त दे॒तद् वै वा ए॒तत् क॑रोति करो त्ये॒तद् वै वा ए॒तत् क॑रोति । \newline
27. ए॒तत् क॑रोति करो त्ये॒त दे॒तत् क॑रोति॒ यद् यत् क॑रो त्ये॒त दे॒तत् क॑रोति॒ यत् । \newline
28. क॒रो॒ति॒ यद् यत् क॑रोति करोति॒ यत् खन॑ति॒ खन॑ति॒ यत् क॑रोति करोति॒ यत् खन॑ति । \newline
29. यत् खन॑ति॒ खन॑ति॒ यद् यत् खन॑ त्य॒पो॑ ऽपः खन॑ति॒ यद् यत् खन॑ त्य॒पः । \newline
30. खन॑ त्य॒पो॑ ऽपः खन॑ति॒ खन॑ त्य॒पो ऽवावा॒पः खन॑ति॒ खन॑ त्य॒पो ऽव॑ । \newline
31. अ॒पो ऽवावा॒पो॑ ऽपो ऽव॑ नयति नय॒ त्यवा॒पो॑ ऽपो ऽव॑ नयति । \newline
32. अव॑ नयति नय॒ त्यवाव॑ नयति॒ शान्त्यै॒ शान्त्यै॑ नय॒ त्यवाव॑ नयति॒ शान्त्यै᳚ । \newline
33. न॒य॒ति॒ शान्त्यै॒ शान्त्यै॑ नयति नयति॒ शान्त्यै॒ यव॑मती॒र् यव॑मतीः॒ शान्त्यै॑ नयति नयति॒ शान्त्यै॒ यव॑मतीः । \newline
34. शान्त्यै॒ यव॑मती॒र् यव॑मतीः॒ शान्त्यै॒ शान्त्यै॒ यव॑मती॒ रवाव॒ यव॑मतीः॒ शान्त्यै॒ शान्त्यै॒ यव॑मती॒ रव॑ । \newline
35. यव॑मती॒ रवाव॒ यव॑मती॒र् यव॑मती॒ रव॑ नयति नय॒ त्यव॒ यव॑मती॒र् यव॑मती॒ रव॑ नयति । \newline
36. यव॑मती॒रिति॒ यव॑ - म॒तीः॒ । \newline
37. अव॑ नयति नय॒ त्यवाव॑ नय॒ त्यूर् गूर्ङ् न॑य॒ त्यवाव॑ नय॒ त्यूर्क् । \newline
38. न॒य॒ त्यूर् गूर्ङ् न॑यति नय॒ त्यूर्ग् वै वा ऊर्ङ् न॑यति नय॒ त्यूर्ग् वै । \newline
39. ऊर्ग् वै वा ऊर् गूर्ग् वै यवो॒ यवो॒ वा ऊर् गूर्ग् वै यवः॑ । \newline
40. वै यवो॒ यवो॒ वै वै यवो॒ यज॑मानेन॒ यज॑मानेन॒ यवो॒ वै वै यवो॒ यज॑मानेन । \newline
41. यवो॒ यज॑मानेन॒ यज॑मानेन॒ यवो॒ यवो॒ यज॑मानेन॒ यूपो॒ यूपो॒ यज॑मानेन॒ यवो॒ यवो॒ यज॑मानेन॒ यूपः॑ । \newline
42. यज॑मानेन॒ यूपो॒ यूपो॒ यज॑मानेन॒ यज॑मानेन॒ यूपः॒ सम्मि॑तः॒ सम्मि॑तो॒ यूपो॒ यज॑मानेन॒ यज॑मानेन॒ यूपः॒ सम्मि॑तः । \newline
43. यूपः॒ सम्मि॑तः॒ सम्मि॑तो॒ यूपो॒ यूपः॒ सम्मि॑तो॒ यावा॒न्॒. यावा॒न् थ्सम्मि॑तो॒ यूपो॒ यूपः॒ सम्मि॑तो॒ यावान्॑ । \newline
44. सम्मि॑तो॒ यावा॒न्॒. यावा॒न् थ्सम्मि॑तः॒ सम्मि॑तो॒ यावा॑ने॒वैव यावा॒न् थ्सम्मि॑तः॒ सम्मि॑तो॒ यावा॑ ने॒व । \newline
45. सम्मि॑त॒ इति॒ सं - मि॒तः॒ । \newline
46. यावा॑ने॒ वैव यावा॒न्॒. यावा॑ने॒व यज॑मानो॒ यज॑मान ए॒व यावा॒न्॒. यावा॑ने॒व यज॑मानः । \newline
47. ए॒व यज॑मानो॒ यज॑मान ए॒वैव यज॑मान॒ स्ताव॑ती॒म् ताव॑तीं॒ ॅयज॑मान ए॒वैव यज॑मान॒ स्ताव॑तीम् । \newline
48. यज॑मान॒ स्ताव॑ती॒म् ताव॑तीं॒ ॅयज॑मानो॒ यज॑मान॒ स्ताव॑ती मे॒वैव ताव॑तीं॒ ॅयज॑मानो॒ यज॑मान॒ स्ताव॑ती मे॒व । \newline
49. ताव॑ती मे॒वैव ताव॑ती॒म् ताव॑ती मे॒वास्मि॑न् नस्मिन् ने॒व ताव॑ती॒म् ताव॑ती मे॒वास्मिन्न्॑ । \newline
50. ए॒वास्मि॑न् नस्मिन् ने॒वै वास्मि॒न् नूर्ज॒ मूर्ज॑ मस्मिन् ने॒वै वास्मि॒न् नूर्ज᳚म् । \newline
51. अ॒स्मि॒न् नूर्ज॒ मूर्ज॑ मस्मिन् नस्मि॒न् नूर्ज॑म् दधाति दधा॒ त्यूर्ज॑ मस्मिन् नस्मि॒न् नूर्ज॑म् दधाति । \newline
52. ऊर्ज॑म् दधाति दधा॒ त्यूर्ज॒ मूर्ज॑म् दधाति पितृ॒णाम् पि॑तृ॒णाम् द॑धा॒ त्यूर्ज॒ मूर्ज॑म् दधाति पितृ॒णाम् । \newline
53. द॒धा॒ति॒ पि॒तृ॒णाम् पि॑तृ॒णाम् द॑धाति दधाति पितृ॒णाꣳ सद॑नꣳ॒॒ सद॑नम् पितृ॒णाम् द॑धाति दधाति पितृ॒णाꣳ सद॑नम् । \newline
\pagebreak
\markright{ TS 6.3.4.2  \hfill https://www.vedavms.in \hfill}

\section{ TS 6.3.4.2 }

\textbf{TS 6.3.4.2 } \newline
\textbf{Samhita Paata} \newline

पितृ॒णाꣳ सद॑नम॒सीति॑ ब॒र्॒.हिरव॑ स्तृणाति पितृदेव॒त्या᳚(1॒)ꣳ॒ ह्ये॑तद्-यन्निखा॑तं॒ ॅयद्-ब॒र्॒.हिरन॑वस्तीर्य मिनु॒यात् पि॑तृदेव॒त्यो॑ निखा॑तः स्याद् ब॒र्॒.हिर॑व॒स्तीर्य॑ मिनोत्य॒स्यामे॒वैनं॑ मिनोति यूपशक॒लमवा᳚स्यति॒ सते॑जसमे॒वैनं॑ मिनोति दे॒वस्त्वा॑ सवि॒ता मद्ध्वा॑ऽन॒क्त्वित्या॑ह॒ तेज॑सै॒वैन॑मनक्ति सुपिप्प॒लाभ्य॒-स्त्वौष॑धीभ्य॒ इति॑ च॒षालं॒ प्रति॑- [  ] \newline

\textbf{Pada Paata} \newline

पि॒तृ॒णाम् । सद॑नम् । अ॒सि॒ । इति॑ । ब॒र्॒.हिः । अवेति॑ । स्तृ॒णा॒ति॒ । पि॒तृ॒दे॒व॒त्य॑मिति॑ पितृ - दे॒व॒त्य᳚म् । हि । ए॒तत् । यत् । निखा॑त॒मिति॒ नि - खा॒त॒म् । यत् । ब॒र॒.हिः । अन॑वस्ती॒र्येत्यन॑व - स्ती॒र्य॒ । मि॒नु॒यात् । पि॒तृ॒दे॒वत्य॑ इति॑ पितृ-दे॒व॒त्यः॑ । निखा॑त॒ इति॒ नि-खा॒तः॒ । स्या॒त् । ब॒र॒.हिः । अ॒व॒स्तीर्येत्य॑व - स्तीर्य॑ । मि॒नो॒ति॒ । अ॒स्याम् । ए॒व । ए॒न॒म् । मि॒नो॒ति॒ । यू॒प॒श॒क॒लमिति॑ यूप - श॒क॒लम् । अवेति॑ । अ॒स्य॒ति॒ । सते॑जस॒मिति॒ स - ते॒ज॒स॒म् । ए॒व । ए॒न॒म् । मि॒नो॒ति॒ । दे॒वः । त्वा॒ । स॒वि॒ता । मद्ध्वा᳚ । अ॒न॒क्तु॒ । इति॑ । आ॒ह॒ । तेज॑सा । ए॒व । ए॒न॒म् । अ॒न॒क्ति॒ । सु॒पि॒प्प॒लाभ्य॒ इति॑ सु-पि॒प्प॒लाभ्यः॑ । त्वा॒ । ओष॑धीभ्य॒ इत्योष॑धि - भ्यः॒ । इति॑ । च॒षाल᳚म् । प्रतीति॑ ।  \newline


\textbf{Krama Paata} \newline

पि॒तृ॒णाꣳ सद॑नम् । सद॑नमसि । अ॒सीति॑ । इति॑ ब॒र्.॒हिः । ब॒र्.॒हिरव॑ । अव॑ स्तृणाति । स्तृ॒णा॒ति॒ पि॒तृ॒दे॒व॒त्य᳚म् । पि॒तृ॒दे॒व॒त्यꣳ॑ हि । पि॒तृ॒दे॒व॒त्य॑मिति॑ पितृ - दे॒व॒त्य᳚म् । ह्ये॑तत् । ए॒तद् यत् । यन् निखा॑तम् । निखा॑त॒म् ॅयत् । निखा॑त॒मिति॒ नि - खा॒त॒म् । यद् ब॒र्.॒हिः । ब॒र्.॒हिरन॑वस्तीर्य । अन॑वस्तीर्य मिनु॒यात् । अन॑वस्ती॒र्येत्यन॑व - स्ती॒र्य॒ । मि॒नु॒यात् पि॑तृदेव॒त्यः॑ । पि॒तृ॒दे॒व॒त्यो॑ निखा॑तः । पि॒तृ॒दे॒व॒त्य॑ इति॑ पितृ - दे॒व॒त्यः॑ । निखा॑तः स्यात् । निखा॑त॒ इति॒ नि - खा॒तः॒ । स्या॒द् ब॒र्.॒हिः । ब॒र्.॒हिर॑व॒स्तीर्य॑ । अ॒व॒स्तीर्य॑ मिनोति । अ॒व॒स्तीर्येत्य॑व - स्तीर्य॑ । मि॒नो॒त्य॒स्याम् । अ॒स्यामे॒व । ए॒वैन᳚म् । ए॒न॒म् मि॒नो॒ति॒ । मि॒नो॒ति॒ यू॒प॒श॒क॒लम् । यू॒प॒श॒क॒लमव॑ । यू॒प॒श॒क॒लमिति॑ यूप - श॒क॒लम् । अवा᳚स्यति । अ॒स्य॒ति॒ सते॑जसम् । सते॑जसमे॒व । सते॑जस॒मिति॒ स - ते॒ज॒स॒म् । ए॒वैन᳚म् । ए॒न॒म् मि॒नो॒ति॒ । मि॒नो॒ति॒ दे॒वः । दे॒वस्त्वा᳚ । त्वा॒ स॒वि॒ता । स॒वि॒ता मद्ध्वा᳚ । मद्ध्वा॑ऽनक्तु । अ॒न॒क्त्विति॑ । इत्या॑ह । आ॒ह॒ तेज॑सा । तेज॑सै॒व । ए॒वैन᳚म् । ए॒न॒म॒न॒क्ति॒ । अ॒न॒क्ति॒ सु॒पि॒प्प॒लाभ्यः॑ । सु॒पि॒प्प॒लाभ्य॑स्त्वा । सु॒पि॒प्प॒लाभ्य॒ इति॑ सु - पि॒प्प॒लाभ्यः॑ । त्वौष॑धीभ्यः । ओष॑धीभ्य॒ इति॑ । ओष॑धीभ्य॒ इत्योष॑धि - भ्यः॒ । इति॑ च॒षाल᳚म् । च॒षाल॒म् प्रति॑ । प्रति॑ मुञ्चति \newline

\textbf{Jatai Paata} \newline

1. पि॒तृ॒णाꣳ सद॑नꣳ॒॒ सद॑नम् पितृ॒णाम् पि॑तृ॒णाꣳ सद॑नम् । \newline
2. सद॑न मस्यसि॒ सद॑नꣳ॒॒ सद॑न मसि । \newline
3. अ॒सी तीत्य॑ स्य॒सीति॑ । \newline
4. इति॑ ब॒र्॒.हिर् ब॒र्॒.हि रितीति॑ ब॒र्॒.हिः । \newline
5. ब॒र्॒.हि रवाव॑ ब॒र्॒.हिर् ब॒र्॒.हि रव॑ । \newline
6. अव॑ स्तृणाति स्तृणा॒ त्यवाव॑ स्तृणाति । \newline
7. स्तृ॒णा॒ति॒ पि॒तृ॒दे॒व॒त्य॑म् पितृदेव॒त्यꣳ॑ स्तृणाति स्तृणाति पितृदेव॒त्य᳚म् । \newline
8. पि॒तृ॒दे॒व॒त्यꣳ॑ हि हि पि॑तृदेव॒त्य॑म् पितृदेव॒त्यꣳ॑ हि । \newline
9. पि॒तृ॒दे॒व॒त्य॑मिति॑ पितृ - दे॒व॒त्य᳚म् । \newline
10. ह्ये॑त दे॒तद्धि ह्ये॑तत् । \newline
11. ए॒तद् यद् यदे॒त दे॒तद् यत् । \newline
12. यन् निखा॑त॒म् निखा॑तं॒ ॅयद् यन् निखा॑तम् । \newline
13. निखा॑तं॒ ॅयद् यन् निखा॑त॒म् निखा॑तं॒ ॅयत् । \newline
14. निखा॑त॒मिति॒ नि - खा॒त॒म् । \newline
15. यद् ब॒र्॒.हिर् ब॒र्॒.हिर् यद् यद् ब॒र्॒.हिः । \newline
16. ब॒र्॒.हि रन॑वस्ती॒र्या न॑वस्तीर्य ब॒र्॒.हिर् ब॒र्॒.हि रन॑वस्तीर्य । \newline
17. अन॑वस्तीर्य मिनु॒यान् मि॑नु॒या दन॑वस्ती॒र्या न॑वस्तीर्य मिनु॒यात् । \newline
18. अन॑वस्ती॒र्येत्यन॑व - स्ती॒र्य॒ । \newline
19. मि॒नु॒यात् पि॑तृदेव॒त्यः॑ पितृदेव॒त्यो॑ मिनु॒यान् मि॑नु॒यात् पि॑तृदेव॒त्यः॑ । \newline
20. पि॒तृ॒दे॒व॒त्यो॑ निखा॑तो॒ निखा॑तः पितृदेव॒त्यः॑ पितृदेव॒त्यो॑ निखा॑तः । \newline
21. पि॒तृ॒दे॒व॒त्य॑ इति॑ पितृ - दे॒व॒त्यः॑ । \newline
22. निखा॑तः स्याथ् स्या॒न् निखा॑तो॒ निखा॑तः स्यात् । \newline
23. निखा॑त॒ इति॒ नि - खा॒तः॒ । \newline
24. स्या॒द् ब॒र्॒.हिर् ब॒र्॒.हिः स्या᳚थ् स्याद् ब॒र्॒.हिः । \newline
25. ब॒र्॒.हि र॑व॒स्तीर्या॑ व॒स्तीर्य॑ ब॒र्॒.हिर् ब॒र्॒.हि र॑व॒स्तीर्य॑ । \newline
26. अ॒व॒स्तीर्य॑ मिनोति मिनो त्यव॒स्तीर्या॑ व॒स्तीर्य॑ मिनोति । \newline
27. अ॒व॒स्तीर्येत्य॑व - स्तीर्य॑ । \newline
28. मि॒नो॒ त्य॒स्या म॒स्याम् मि॑नोति मिनो त्य॒स्याम् । \newline
29. अ॒स्या मे॒वै वास्या म॒स्या मे॒व । \newline
30. ए॒वैन॑ मेन मे॒वै वैन᳚म् । \newline
31. ए॒न॒म् मि॒नो॒ति॒ मि॒नो॒ त्ये॒न॒ मे॒न॒म् मि॒नो॒ति॒ । \newline
32. मि॒नो॒ति॒ यू॒प॒श॒क॒लं ॅयू॑पशक॒लम् मि॑नोति मिनोति यूपशक॒लम् । \newline
33. यू॒प॒श॒क॒ल मवाव॑ यूपशक॒लं ॅयू॑पशक॒ल मव॑ । \newline
34. यू॒प॒श॒क॒लमिति॑ यूप - श॒क॒लम् । \newline
35. अवा᳚स्य त्यस्य॒ त्यवा वा᳚स्यति । \newline
36. अ॒स्य॒ति॒ सते॑जसꣳ॒॒ सते॑जस मस्य त्यस्यति॒ सते॑जसम् । \newline
37. सते॑जस मे॒वैव सते॑जसꣳ॒॒ सते॑जस मे॒व । \newline
38. सते॑जस॒मिति॒ स - ते॒ज॒स॒म् । \newline
39. ए॒वैन॑ मेन मे॒वै वैन᳚म् । \newline
40. ए॒न॒म् मि॒नो॒ति॒ मि॒नो॒ त्ये॒न॒ मे॒न॒म् मि॒नो॒ति॒ । \newline
41. मि॒नो॒ति॒ दे॒वो दे॒वो मि॑नोति मिनोति दे॒वः । \newline
42. दे॒व स्त्वा᳚ त्वा दे॒वो दे॒व स्त्वा᳚ । \newline
43. त्वा॒ स॒वि॒ता स॑वि॒ता त्वा᳚ त्वा सवि॒ता । \newline
44. स॒वि॒ता मद्ध्वा॒ मद्ध्वा॑ सवि॒ता स॑वि॒ता मद्ध्वा᳚ । \newline
45. मद्ध्वा॑ ऽनक् त्वनक्तु॒ मद्ध्वा॒ मद्ध्वा॑ ऽनक्तु । \newline
46. अ॒न॒क् त्वितीत्य॑ नक् त्वन॒क्त्विति॑ । \newline
47. इत्या॑हा॒हे तीत्या॑ह । \newline
48. आ॒ह॒ तेज॑सा॒ तेज॑सा ऽऽहाह॒ तेज॑सा । \newline
49. तेज॑सै॒वैव तेज॑सा॒ तेज॑सै॒व । \newline
50. ए॒वैन॑ मेन मे॒वै वैन᳚म् । \newline
51. ए॒न॒ म॒न॒क् त्य॒न॒क् त्ये॒न॒ मे॒न॒ म॒न॒क्ति॒ । \newline
52. अ॒न॒क्ति॒ सु॒पि॒प्प॒लाभ्यः॑ सुपिप्प॒लाभ्यो॑ ऽनक् त्यनक्ति सुपिप्प॒लाभ्यः॑ । \newline
53. सु॒पि॒प्प॒लाभ्य॑ स्त्वा त्वा सुपिप्प॒लाभ्यः॑ सुपिप्प॒लाभ्य॑ स्त्वा । \newline
54. सु॒पि॒प्प॒लाभ्य॒ इति॑ सु - पि॒प्प॒लाभ्यः॑ । \newline
55. त्वौष॑धीभ्य॒ ओष॑धीभ्य स्त्वा॒ त्वौष॑धीभ्यः । \newline
56. ओष॑धीभ्य॒ इतीत्योष॑धीभ्य॒ ओष॑धीभ्य॒ इति॑ । \newline
57. ओष॑धीभ्य॒ इत्योष॑धि - भ्यः॒ । \newline
58. इति॑ च॒षाल॑म् च॒षाल॒ मितीति॑ च॒षाल᳚म् । \newline
59. च॒षाल॒म् प्रति॒ प्रति॑ च॒षाल॑म् च॒षाल॒म् प्रति॑ । \newline
60. प्रति॑ मुञ्चति मुञ्चति॒ प्रति॒ प्रति॑ मुञ्चति । \newline

\textbf{Ghana Paata } \newline

1. पि॒तृ॒णाꣳ सद॑नꣳ॒॒ सद॑नम् पितृ॒णाम् पि॑तृ॒णाꣳ सद॑न मस्यसि॒ सद॑नम् पितृ॒णाम् पि॑तृ॒णाꣳ सद॑न मसि । \newline
2. सद॑न मस्यसि॒ सद॑नꣳ॒॒ सद॑न म॒सीती त्य॑सि॒ सद॑नꣳ॒॒ सद॑न म॒सीति॑ । \newline
3. अ॒सीती त्य॑स्य॒ सीति॑ ब॒र्॒.हिर् ब॒र्॒.हि रित्य॑स्य॒ सीति॑ ब॒र्॒.हिः । \newline
4. इति॑ ब॒र्॒.हिर् ब॒र्॒.हिरि तीति॑ ब॒र्॒.हि रवाव॑ ब॒र्॒.हिरि तीति॑ ब॒र्॒.हिरव॑ । \newline
5. ब॒र्॒.हि रवाव॑ ब॒र्॒.हिर् ब॒र्॒.हि रव॑ स्तृणाति स्तृणा॒ त्यव॑ ब॒र्॒.हिर् ब॒र्॒.हि रव॑ स्तृणाति । \newline
6. अव॑ स्तृणाति स्तृणा॒ त्यवाव॑ स्तृणाति पितृदेव॒त्य॑म् पितृदेव॒त्यꣳ॑ स्तृणा॒ त्यवाव॑ स्तृणाति पितृदेव॒त्य᳚म् । \newline
7. स्तृ॒णा॒ति॒ पि॒तृ॒दे॒व॒त्य॑म् पितृदेव॒त्यꣳ॑ स्तृणाति स्तृणाति पितृदेव॒त्यꣳ॑ हि हि पि॑तृदेव॒त्यꣳ॑ स्तृणाति स्तृणाति पितृदेव॒त्यꣳ॑ हि । \newline
8. पि॒तृ॒दे॒व॒त्यꣳ॑ हि हि पि॑तृदेव॒त्य॑म् पितृदेव॒त्या᳚[(अ1॒) (ग्ग्॒)] ह्ये॑त दे॒तद्धि पि॑तृदेव॒त्य॑म् पितृदेव॒त्या᳚[(अ1॒)(ग्ग्॒)] ह्ये॑तत् । \newline
9. पि॒तृ॒दे॒व॒त्य॑मिति॑ पितृ - दे॒व॒त्य᳚म् । \newline
10. ह्ये॑त दे॒त द्धि ह्ये॑तद् यद् यदे॒त द्धि ह्ये॑तद् यत् । \newline
11. ए॒तद् यद् यदे॒त दे॒तद् यन् निखा॑त॒म् निखा॑तं॒ ॅयदे॒त दे॒तद् यन् निखा॑तम् । \newline
12. यन् निखा॑त॒म् निखा॑तं॒ ॅयद् यन् निखा॑तं॒ ॅयद् यन् निखा॑तं॒ ॅयद् यन् निखा॑तं॒ ॅयत् । \newline
13. निखा॑तं॒ ॅयद् यन् निखा॑त॒म् निखा॑तं॒ ॅयद् ब॒र्॒.हिर् ब॒र्॒.हिर् यन् निखा॑त॒म् निखा॑तं॒ ॅयद् ब॒र्॒.हिः । \newline
14. निखा॑त॒मिति॒ नि - खा॒त॒म् । \newline
15. यद् ब॒र्॒.हिर् ब॒र्॒.हिर् यद् यद् ब॒र्॒.हि रन॑वस्ती॒र्या न॑वस्तीर्य ब॒र्॒.हिर् यद् यद् ब॒र्॒.हि रन॑वस्तीर्य । \newline
16. ब॒र्॒.हि रन॑वस्ती॒र्या न॑वस्तीर्य ब॒र्॒.हिर् ब॒र्॒.हि रन॑वस्तीर्य मिनु॒यान् मि॑नु॒या दन॑वस्तीर्य ब॒र्॒.हिर् ब॒र्॒.हि रन॑वस्तीर्य मिनु॒यात् । \newline
17. अन॑वस्तीर्य मिनु॒यान् मि॑नु॒या दन॑वस्ती॒र्या न॑वस्तीर्य मिनु॒यात् पि॑तृदेव॒त्यः॑ पितृदेव॒त्यो॑ मिनु॒या दन॑वस्ती॒र्या न॑वस्तीर्य मिनु॒यात् पि॑तृदेव॒त्यः॑ । \newline
18. अन॑वस्ती॒र्येत्यन॑व - स्ती॒र्य॒ । \newline
19. मि॒नु॒यात् पि॑तृदेव॒त्यः॑ पितृदेव॒त्यो॑ मिनु॒यान् मि॑नु॒यात् पि॑तृदेव॒त्यो॑ निखा॑तो॒ निखा॑तः पितृदेव॒त्यो॑ मिनु॒यान् मि॑नु॒यात् पि॑तृदेव॒त्यो॑ निखा॑तः । \newline
20. पि॒तृ॒दे॒व॒त्यो॑ निखा॑तो॒ निखा॑तः पितृदेव॒त्यः॑ पितृदेव॒त्यो॑ निखा॑तः स्याथ् स्या॒न् निखा॑तः पितृदेव॒त्यः॑ पितृदेव॒त्यो॑ निखा॑तः स्यात् । \newline
21. पि॒तृ॒दे॒व॒त्य॑ इति॑ पितृ - दे॒व॒त्यः॑ । \newline
22. निखा॑तः स्याथ् स्या॒न् निखा॑तो॒ निखा॑तः स्याद् ब॒र्॒.हिर् ब॒र्॒.हिः स्या॒न् निखा॑तो॒ निखा॑तः स्याद् ब॒र्॒.हिः । \newline
23. निखा॑त॒ इति॒ नि - खा॒तः॒ । \newline
24. स्या॒द् ब॒र्॒.हिर् ब॒र्॒.हिः स्या᳚थ् स्याद् ब॒र्॒.हि र॑व॒स्तीर्या॑ व॒स्तीर्य॑ ब॒र्॒.हिः स्या᳚थ् स्याद् ब॒र्॒.हि र॑व॒स्तीर्य॑ । \newline
25. ब॒र्॒.हि र॑व॒स्तीर्या॑ व॒स्तीर्य॑ ब॒र्॒.हिर् ब॒र्॒.हि र॑व॒स्तीर्य॑ मिनोति मिनो त्यव॒स्तीर्य॑ ब॒र्॒.हिर् ब॒र्॒.हि र॑व॒स्तीर्य॑ मिनोति । \newline
26. अ॒व॒स्तीर्य॑ मिनोति मिनो त्यव॒स्तीर्या॑ व॒स्तीर्य॑ मिनो त्य॒स्या म॒स्याम् मि॑नो त्यव॒स्तीर्या॑ व॒स्तीर्य॑ मिनो त्य॒स्याम् । \newline
27. अ॒व॒स्तीर्येत्य॑व - स्तीर्य॑ । \newline
28. मि॒नो॒ त्य॒स्या म॒स्याम् मि॑नोति मिनो त्य॒स्या मे॒वै वास्याम् मि॑नोति मिनो त्य॒स्या मे॒व । \newline
29. अ॒स्या मे॒वै वास्या म॒स्या मे॒वैन॑ मेन मे॒वास्या म॒स्या मे॒वैन᳚म् । \newline
30. ए॒वैन॑ मेन मे॒वै वैन॑म् मिनोति मिनो त्येन मे॒वै वैन॑म् मिनोति । \newline
31. ए॒न॒म् मि॒नो॒ति॒ मि॒नो॒ त्ये॒न॒ मे॒न॒म् मि॒नो॒ति॒ यू॒प॒श॒क॒लं ॅयू॑पशक॒लम् मि॑नो त्येन मेनम् मिनोति यूपशक॒लम् । \newline
32. मि॒नो॒ति॒ यू॒प॒श॒क॒लं ॅयू॑पशक॒लम् मि॑नोति मिनोति यूपशक॒ल मवाव॑ यूपशक॒लम् मि॑नोति मिनोति यूपशक॒ल मव॑ । \newline
33. यू॒प॒श॒क॒ल मवाव॑ यूपशक॒लं ॅयू॑पशक॒ल मवा᳚ स्यत्य स्य॒त्यव॑ यूपशक॒लं ॅयू॑पशक॒ल मवा᳚स्यति । \newline
34. यू॒प॒श॒क॒लमिति॑ यूप - श॒क॒लम् । \newline
35. अवा᳚ स्यत्य स्य॒त्यवावा᳚ स्यति॒ सते॑जसꣳ॒॒ सते॑जस मस्य॒त्यवावा᳚ स्यति॒ सते॑जसम् । \newline
36. अ॒स्य॒ति॒ सते॑जसꣳ॒॒ सते॑जस मस्य त्यस्यति॒ सते॑जस मे॒वैव सते॑जस मस्य त्यस्यति॒ सते॑जस मे॒व । \newline
37. सते॑जस मे॒वैव सते॑जसꣳ॒॒ सते॑जस मे॒वैन॑ मेन मे॒व सते॑जसꣳ॒॒ सते॑जस मे॒वैन᳚म् । \newline
38. सते॑जस॒मिति॒ स - ते॒ज॒स॒म् । \newline
39. ए॒वैन॑ मेन मे॒वै वैन॑म् मिनोति मिनो त्येन मे॒वै वैन॑म् मिनोति । \newline
40. ए॒न॒म् मि॒नो॒ति॒ मि॒नो॒ त्ये॒न॒ मे॒न॒म् मि॒नो॒ति॒ दे॒वो दे॒वो मि॑नो त्येन मेनम् मिनोति दे॒वः । \newline
41. मि॒नो॒ति॒ दे॒वो दे॒वो मि॑नोति मिनोति दे॒व स्त्वा᳚ त्वा दे॒वो मि॑नोति मिनोति दे॒व स्त्वा᳚ । \newline
42. दे॒व स्त्वा᳚ त्वा दे॒वो दे॒व स्त्वा॑ सवि॒ता स॑वि॒ता त्वा॑ दे॒वो दे॒व स्त्वा॑ सवि॒ता । \newline
43. त्वा॒ स॒वि॒ता स॑वि॒ता त्वा᳚ त्वा सवि॒ता मद्ध्वा॒ मद्ध्वा॑ सवि॒ता त्वा᳚ त्वा सवि॒ता मद्ध्वा᳚ । \newline
44. स॒वि॒ता मद्ध्वा॒ मद्ध्वा॑ सवि॒ता स॑वि॒ता मद्ध्वा॑ ऽनक् त्वनक्तु॒ मद्ध्वा॑ सवि॒ता स॑वि॒ता मद्ध्वा॑ ऽनक्तु । \newline
45. मद्ध्वा॑ ऽनक् त्वनक्तु॒ मद्ध्वा॒ मद्ध्वा॑ ऽन॒क्त्विती त्य॑नक्तु॒ मद्ध्वा॒ मद्ध्वा॑ ऽन॒क्त्विति॑ । \newline
46. अ॒न॒क् त्विती त्य॑नक्त्व न॒क् त्वित्या॑हा॒हे त्य॑नक् त्वन॒क्त्वि त्या॑ह । \newline
47. इत्या॑हा॒हे तीत्या॑ह॒ तेज॑सा॒ तेज॑सा॒ ऽऽहे तीत्या॑ह॒ तेज॑सा । \newline
48. आ॒ह॒ तेज॑सा॒ तेज॑सा ऽऽहाह॒ तेज॑सै॒ वैव तेज॑सा ऽऽहाह॒ तेज॑ सै॒व । \newline
49. तेज॑सै॒ वैव तेज॑सा॒ तेज॑सै॒ वैन॑ मेन मे॒व तेज॑सा॒ तेज॑सै॒ वैन᳚म् । \newline
50. ए॒वैन॑ मेन मे॒वै वैन॑ मनक् त्यनक् त्येन मे॒वै वैन॑ मनक्ति । \newline
51. ए॒न॒ म॒न॒क् त्य॒न॒क् त्ये॒न॒ मे॒न॒ म॒न॒क्ति॒ सु॒पि॒प्प॒लाभ्यः॑ सुपिप्प॒लाभ्यो॑ ऽनक्त्येन मेन मनक्ति सुपिप्प॒लाभ्यः॑ । \newline
52. अ॒न॒क्ति॒ सु॒पि॒प्प॒लाभ्यः॑ सुपिप्प॒लाभ्यो॑ ऽनक् त्यनक्ति सुपिप्प॒लाभ्य॑ स्त्वा त्वा सुपिप्प॒लाभ्यो॑ ऽनक्त्यनक्ति सुपिप्प॒लाभ्य॑ स्त्वा । \newline
53. सु॒पि॒प्प॒लाभ्य॑ स्त्वा त्वा सुपिप्प॒लाभ्यः॑ सुपिप्प॒लाभ्य॒ स्त्वौष॑धीभ्य॒ ओष॑धीभ्य स्त्वा सुपिप्प॒लाभ्यः॑ सुपिप्प॒लाभ्य॒ स्त्वौष॑धीभ्यः । \newline
54. सु॒पि॒प्प॒लाभ्य॒ इति॑ सु - पि॒प्प॒लाभ्यः॑ । \newline
55. त्वौष॑धीभ्य॒ ओष॑धीभ्य स्त्वा॒ त्वौष॑धीभ्य॒ इती त्योष॑धीभ्य स्त्वा॒ त्वौष॑धीभ्य॒ इति॑ । \newline
56. ओष॑धीभ्य॒ इती त्योष॑धीभ्य॒ ओष॑धीभ्य॒ इति॑ च॒षाल॑म् च॒षाल॒ मित्योष॑धीभ्य॒ ओष॑धीभ्य॒ इति॑ च॒षाल᳚म् । \newline
57. ओष॑धीभ्य॒ इत्योष॑धि - भ्यः॒ । \newline
58. इति॑ च॒षाल॑म् च॒षाल॒ मितीति॑ च॒षाल॒म् प्रति॒ प्रति॑ च॒षाल॒ मितीति॑ च॒षाल॒म् प्रति॑ । \newline
59. च॒षाल॒म् प्रति॒ प्रति॑ च॒षाल॑म् च॒षाल॒म् प्रति॑ मुञ्चति मुञ्चति॒ प्रति॑ च॒षाल॑म् च॒षाल॒म् प्रति॑ मुञ्चति । \newline
60. प्रति॑ मुञ्चति मुञ्चति॒ प्रति॒ प्रति॑ मुञ्चति॒ तस्मा॒त् तस्मा᳚न् मुञ्चति॒ प्रति॒ प्रति॑ मुञ्चति॒ तस्मा᳚त् । \newline
\pagebreak
\markright{ TS 6.3.4.3  \hfill https://www.vedavms.in \hfill}

\section{ TS 6.3.4.3 }

\textbf{TS 6.3.4.3 } \newline
\textbf{Samhita Paata} \newline

मुञ्चति॒ तस्मा᳚च्छीर्.ष॒त ओष॑धयः॒ फलं॑ गृह्णन्त्य॒नक्ति॒ तेजो॒ वा आज्यं॒ ॅयज॑मानेनाग्नि॒ष्ठाऽश्रिः॒ संमि॑ता॒ यद॑ग्नि॒ष्ठा-मश्रि॑म॒नक्ति॒ यज॑मानमे॒व तेज॑सा ऽनक्त्या॒न्त-म॑नक्त्या॒न्तमे॒व यज॑मानं॒ तेज॑सानक्ति स॒र्वतः॒ परि॑ मृश॒त्यप॑रिवर्ग-मे॒वास्मि॒न् तेजो॑ दधा॒त्युद् दिवꣳ॑ स्तभा॒नाऽन्तरि॑क्षं पृ॒णेत्या॑है॒षां ॅलो॒कानां॒ ॅविधृ॑त्यै वैष्ण॒व्यर्चा- [  ] \newline

\textbf{Pada Paata} \newline

मु॒ञ्च॒ति॒ । तस्मा᳚त् । शी॒र्॒.ष॒तः । ओष॑धयः । फल᳚म् । गृ॒ह्ण॒न्ति॒ । अ॒नक्ति॑ । तेजः॑ । वै । आज्य᳚म् । यज॑मानेन । अ॒ग्नि॒ष्ठेत्य॑ग्नि-स्था । अश्रिः॑ । संमि॒तेति॒ सं - मि॒ता॒ । यत् । अ॒ग्नि॒ष्ठामित्य॑ग्नि- स्थाम् । अश्रि᳚म् । अ॒न॒क्ति॒ । यज॑मानम् । ए॒व । तेज॑सा । अ॒न॒क्ति॒ । आ॒न्तमित्या᳚ - अ॒न्तम् । अ॒न॒क्ति॒ । आ॒न्तमित्या᳚ - अ॒न्तम् । ए॒व । यज॑मानम् । तेज॑सा । अ॒न॒क्ति॒ । स॒र्वतः॑ । परीति॑ । मृ॒श॒ति॒ । अप॑रिवर्ग॒मित्यप॑रि - व॒र्ग॒म् । ए॒व । अ॒स्मि॒न्न् । तेजः॑ । द॒धा॒ति॒ । उदिति॑ । दिव᳚म् । स्त॒भा॒न॒ । एति॑ । अ॒न्तरि॑क्षम् । पृ॒ण॒ । इति॑ । आ॒ह॒ । ए॒षाम् । लो॒काना᳚म् । विधृ॑त्या॒ इति॒ वि - धृ॒त्यै॒ । वै॒ष्ण॒व्या । ऋ॒चा ।  \newline


\textbf{Krama Paata} \newline

मु॒ञ्च॒ति॒ तस्मा᳚त् । तस्मा᳚च्छीर्.ष॒तः । शी॒र्.॒ष॒त ओष॑धयः । ओष॑धयः॒ फल᳚म् । फल॑म् गृह्णन्ति । गृ॒ह्ण॒न्त्य॒नक्ति॑ । अ॒नक्ति॒ तेजः॑ । तेजो॒ वै । वा आज्य᳚म् । आज्य॒म् ॅयज॑मानेन । यज॑मानेनाग्नि॒ष्ठा । अ॒ग्नि॒ष्ठाऽश्रिः॑ । अ॒ग्नि॒ष्ठेत्य॑ग्नि - स्था । अश्रिः॒ सम्मि॑ता । सम्मि॑ता॒ यत् । सम्मि॒तेति॒ सम् - मि॒ता॒ । यद॑ग्नि॒ष्ठाम् । अ॒ग्नि॒ष्ठामश्रि᳚म् । अ॒ग्नि॒ष्ठामित्य॑ग्नि - स्थाम् । अश्रि॑म॒नक्ति॑ । अ॒नक्ति॒ यज॑मानम् । यज॑मानमे॒व । ए॒व तेज॑सा । तेज॑साऽनक्ति । अ॒न॒क्त्या॒न्तम् । आ॒न्तम॑नक्ति । आ॒न्तमित्या᳚ - अ॒न्तम् । अ॒न॒क्त्या॒न्तम् । आ॒न्तमे॒व । आ॒न्तमित्या᳚ - अ॒न्तम् । ए॒व यज॑मानम् । यज॑मान॒म् तेज॑सा । तेज॑साऽनक्ति । अ॒न॒क्ति॒ स॒र्वतः॑ । स॒र्वतः॒ परि॑ । परि॑ मृशति । मृ॒श॒त्यप॑रिवर्गम् । अप॑रिवर्गमे॒व । अप॑रिवर्ग॒मित्यप॑रि - व॒र्ग॒म् । ए॒वास्मिन्न्॑ । अ॒स्मि॒न् तेजः॑ । तेजो॑ दधाति । द॒धा॒त्युत् । उद् दिव᳚म् । दिवꣳ॑ स्तभा॒न । स्त॒भा॒ना । आऽन्तरि॑क्षम् । अ॒न्तरि॑क्षम् पृण । पृ॒णेति॑ । इत्या॑ह । आ॒है॒षाम् । ए॒षाम् ॅलो॒काना᳚म् । लो॒काना॒म् ॅविधृ॑त्यै । विधृ॑त्यै वैष्ण॒व्या । विधृ॑त्या॒ इति॒ वि - धृ॒त्यै॒ । वै॒ष्ण॒व्यर्चा । ऋ॒चा क॑ल्पयति \newline

\textbf{Jatai Paata} \newline

1. मु॒ञ्च॒ति॒ तस्मा॒त् तस्मा᳚न् मुञ्चति मुञ्चति॒ तस्मा᳚त् । \newline
2. तस्मा᳚ च्छीर्.ष॒तः शी॑र्.ष॒त स्तस्मा॒त् तस्मा᳚ च्छीर्.ष॒तः । \newline
3. शी॒र्॒.ष॒त ओष॑धय॒ ओष॑धयः शीर्.ष॒तः शी॑र्.ष॒त ओष॑धयः । \newline
4. ओष॑धयः॒ फल॒म् फल॒ मोष॑धय॒ ओष॑धयः॒ फल᳚म् । \newline
5. फल॑म् गृह्णन्ति गृह्णन्ति॒ फल॒म् फल॑म् गृह्णन्ति । \newline
6. गृ॒ह्ण॒न् त्य॒नक् त्य॒नक्ति॑ गृह्णन्ति गृह्णन् त्य॒नक्ति॑ । \newline
7. अ॒नक्ति॒ तेज॒ स्तेजो॒ ऽनक् त्य॒नक्ति॒ तेजः॑ । \newline
8. तेजो॒ वै वै तेज॒ स्तेजो॒ वै । \newline
9. वा आज्य॒ माज्यं॒ ॅवै वा आज्य᳚म् । \newline
10. आज्यं॒ ॅयज॑मानेन॒ यज॑माने॒ नाज्य॒ माज्यं॒ ॅयज॑मानेन । \newline
11. यज॑मानेना ग्नि॒ष्ठा ऽग्नि॒ष्ठा यज॑मानेन॒ यज॑मानेना ग्नि॒ष्ठा । \newline
12. अ॒ग्नि॒ष्ठा ऽश्रि॒ रश्रि॑ रग्नि॒ष्ठा ऽग्नि॒ष्ठा ऽश्रिः॑ । \newline
13. अ॒ग्नि॒ष्ठेत्य॑ग्नि - स्था । \newline
14. अश्रिः॒ सम्मि॑ता॒ सम्मि॒ता ऽश्रि॒ रश्रिः॒ सम्मि॑ता । \newline
15. सम्मि॑ता॒ यद् यथ् सम्मि॑ता॒ सम्मि॑ता॒ यत् । \newline
16. सम्मि॒तेति॒ सं - मि॒ता॒ । \newline
17. यद॑ग्नि॒ष्ठा म॑ग्नि॒ष्ठां ॅयद् यद॑ग्नि॒ष्ठाम् । \newline
18. अ॒ग्नि॒ष्ठा मश्रि॒ मश्रि॑ मग्नि॒ष्ठा म॑ग्नि॒ष्ठा मश्रि᳚म् । \newline
19. अ॒ग्नि॒ष्ठामित्य॑ग्नि - स्थाम् । \newline
20. अश्रि॑ म॒नक् त्य॒नक् त्यश्रि॒ मश्रि॑ म॒नक्ति॑ । \newline
21. अ॒नक्ति॒ यज॑मानं॒ ॅयज॑मान म॒नक् त्य॒नक्ति॒ यज॑मानम् । \newline
22. यज॑मान मे॒वैव यज॑मानं॒ ॅयज॑मान मे॒व । \newline
23. ए॒व तेज॑सा॒ तेज॑सै॒वैव तेज॑सा । \newline
24. तेज॑सा ऽनक् त्यनक्ति॒ तेज॑सा॒ तेज॑सा ऽनक्ति । \newline
25. अ॒न॒क् त्या॒न्त मा॒न्त म॑नक् त्यनक् त्या॒न्तम् । \newline
26. आ॒न्त म॑नक् त्यनक् त्या॒न्त मा॒न्त म॑नक्ति । \newline
27. आ॒न्तमित्या᳚ - अ॒न्तम् । \newline
28. अ॒न॒क् त्या॒न्त मा॒न्त म॑नक् त्यनक् त्या॒न्तम् । \newline
29. आ॒न्त मे॒वै वान्त मा॒न्त मे॒व । \newline
30. आ॒न्तमित्या᳚ - अ॒न्तम् । \newline
31. ए॒व यज॑मानं॒ ॅयज॑मान मे॒वैव यज॑मानम् । \newline
32. यज॑मान॒म् तेज॑सा॒ तेज॑सा॒ यज॑मानं॒ ॅयज॑मान॒म् तेज॑सा । \newline
33. तेज॑सा ऽनक् त्यनक्ति॒ तेज॑सा॒ तेज॑सा ऽनक्ति । \newline
34. अ॒न॒क्ति॒ स॒र्वतः॑ स॒र्वतो॑ ऽनक् त्यनक्ति स॒र्वतः॑ । \newline
35. स॒र्वतः॒ परि॒ परि॑ स॒र्वतः॑ स॒र्वतः॒ परि॑ । \newline
36. परि॑ मृशति मृशति॒ परि॒ परि॑ मृशति । \newline
37. मृ॒श॒ त्यप॑रिवर्ग॒ मप॑रिवर्गम् मृशति मृश॒ त्यप॑रिवर्गम् । \newline
38. अप॑रिवर्ग मे॒वै वाप॑रिवर्ग॒ मप॑रिवर्ग मे॒व । \newline
39. अप॑रिवर्ग॒मित्यप॑रि - व॒र्ग॒म् । \newline
40. ए॒वास्मि॑न् नस्मिन् ने॒वै वास्मिन्न्॑ । \newline
41. अ॒स्मि॒न् तेज॒ स्तेजो᳚ ऽस्मिन् नस्मि॒न् तेजः॑ । \newline
42. तेजो॑ दधाति दधाति॒ तेज॒ स्तेजो॑ दधाति । \newline
43. द॒धा॒ त्युदुद् द॑धाति दधा॒ त्युत् । \newline
44. उद् दिव॒म् दिव॒ मुदुद् दिव᳚म् । \newline
45. दिवꣳ॑ स्तभान स्तभान॒ दिव॒म् दिवꣳ॑ स्तभान । \newline
46. स्त॒भा॒ना स्त॑भान स्तभा॒ना । \newline
47. आ ऽन्तरि॑क्ष म॒न्तरि॑क्ष॒ मा ऽन्तरि॑क्षम् । \newline
48. अ॒न्तरि॑क्षम् पृण पृणा॒न्तरि॑क्ष म॒न्तरि॑क्षम् पृण । \newline
49. पृ॒णे तीति॑ पृण पृ॒णेति॑ । \newline
50. इत्या॑हा॒हे तीत्या॑ह । \newline
51. आ॒है॒षा मे॒षा मा॑हा है॒षाम् । \newline
52. ए॒षाम् ॅलो॒काना᳚म् ॅलो॒काना॑ मे॒षा मे॒षाम् ॅलो॒काना᳚म् । \newline
53. लो॒कानां॒ ॅविधृ॑त्यै॒ विधृ॑त्यै लो॒काना᳚म् ॅलो॒कानां॒ ॅविधृ॑त्यै । \newline
54. विधृ॑त्यै वैष्ण॒व्या वै᳚ष्ण॒व्या विधृ॑त्यै॒ विधृ॑त्यै वैष्ण॒व्या । \newline
55. विधृ॑त्या॒ इति॒ वि - धृ॒त्यै॒ । \newline
56. वै॒ष्ण॒व्य र्‌च र्‌चा वै᳚ष्ण॒व्या वै᳚ष्ण॒व्य र्‌चा । \newline
57. ऋ॒चा क॑ल्पयति कल्पय त्यृ॒च र्‌चा क॑ल्पयति । \newline

\textbf{Ghana Paata } \newline

1. मु॒ञ्च॒ति॒ तस्मा॒त् तस्मा᳚न् मुञ्चति मुञ्चति॒ तस्मा᳚च् छीर्.ष॒तः शी॑र्.ष॒त स्तस्मा᳚न् मुञ्चति मुञ्चति॒ तस्मा᳚च् छीर्.ष॒तः । \newline
2. तस्मा᳚च् छीर्.ष॒तः शी॑र्.ष॒त स्तस्मा॒त् तस्मा᳚च् छीर्.ष॒त ओष॑धय॒ ओष॑धयः शीर्.ष॒त स्तस्मा॒त् तस्मा᳚च् छीर्.ष॒त ओष॑धयः । \newline
3. शी॒र्॒.ष॒त ओष॑धय॒ ओष॑धयः शीर्.ष॒तः शी॑र्.ष॒त ओष॑धयः॒ फल॒म् फल॒ मोष॑धयः शीर्.ष॒तः शी॑र्.ष॒त ओष॑धयः॒ फल᳚म् । \newline
4. ओष॑धयः॒ फल॒म् फल॒ मोष॑धय॒ ओष॑धयः॒ फल॑म् गृह्णन्ति गृह्णन्ति॒ फल॒ मोष॑धय॒ ओष॑धयः॒ फल॑म् गृह्णन्ति । \newline
5. फल॑म् गृह्णन्ति गृह्णन्ति॒ फल॒म् फल॑म् गृह्णन् त्य॒नक् त्य॒नक्ति॑ गृह्णन्ति॒ फल॒म् फल॑म् गृह्णन् त्य॒नक्ति॑ । \newline
6. गृ॒ह्ण॒न् त्य॒नक् त्य॒नक्ति॑ गृह्णन्ति गृह्णन् त्य॒नक्ति॒ तेज॒ स्तेजो॒ ऽनक्ति॑ गृह्णन्ति गृह्णन् त्य॒नक्ति॒ तेजः॑ । \newline
7. अ॒नक्ति॒ तेज॒ स्तेजो॒ ऽनक् त्य॒नक्ति॒ तेजो॒ वै वै तेजो॒ ऽनक् त्य॒नक्ति॒ तेजो॒ वै । \newline
8. तेजो॒ वै वै तेज॒ स्तेजो॒ वा आज्य॒ माज्यं॒ ॅवै तेज॒ स्तेजो॒ वा आज्य᳚म् । \newline
9. वा आज्य॒ माज्यं॒ ॅवै वा आज्यं॒ ॅयज॑मानेन॒ यज॑माने॒ नाज्यं॒ ॅवै वा आज्यं॒ ॅयज॑मानेन । \newline
10. आज्यं॒ ॅयज॑मानेन॒ यज॑माने॒ नाज्य॒ माज्यं॒ ॅयज॑माने नाग्नि॒ष्ठा ऽग्नि॒ष्ठा यज॑माने॒ नाज्य॒ माज्यं॒ ॅयज॑माने नाग्नि॒ष्ठा । \newline
11. यज॑माने नाग्नि॒ष्ठा ऽग्नि॒ष्ठा यज॑मानेन॒ यज॑माने नाग्नि॒ष्ठा ऽश्रि॒ रश्रि॑ रग्नि॒ष्ठा यज॑मानेन॒ यज॑माने नाग्नि॒ष्ठा ऽश्रिः॑ । \newline
12. अ॒ग्नि॒ष्ठा ऽश्रि॒ रश्रि॑ रग्नि॒ष्ठा ऽग्नि॒ष्ठा ऽश्रिः॒ सम्मि॑ता॒ सम्मि॒ता ऽश्रि॑ रग्नि॒ष्ठा ऽग्नि॒ष्ठा ऽश्रिः॒ सम्मि॑ता । \newline
13. अ॒ग्नि॒ष्ठेत्य॑ग्नि - स्था । \newline
14. अश्रिः॒ सम्मि॑ता॒ सम्मि॒ता ऽश्रि॒ रश्रिः॒ सम्मि॑ता॒ यद् यथ् सम्मि॒ता ऽश्रि॒ रश्रिः॒ सम्मि॑ता॒ यत् । \newline
15. सम्मि॑ता॒ यद् यथ् सम्मि॑ता॒ सम्मि॑ता॒ यद॑ग्नि॒ष्ठा म॑ग्नि॒ष्ठां ॅयथ् सम्मि॑ता॒ सम्मि॑ता॒ यद॑ग्नि॒ष्ठाम् । \newline
16. सम्मि॒तेति॒ सं - मि॒ता॒ । \newline
17. यद॑ग्नि॒ष्ठा म॑ग्नि॒ष्ठां ॅयद् यद॑ग्नि॒ष्ठा मश्रि॒ मश्रि॑ मग्नि॒ष्ठां ॅयद् यद॑ग्नि॒ष्ठा मश्रि᳚म् । \newline
18. अ॒ग्नि॒ष्ठा मश्रि॒ मश्रि॑ मग्नि॒ष्ठा म॑ग्नि॒ष्ठा मश्रि॑ म॒नक् त्य॒न क्त्यश्रि॑ मग्नि॒ष्ठा म॑ग्नि॒ष्ठा मश्रि॑ म॒नक्ति॑ । \newline
19. अ॒ग्नि॒ष्ठामित्य॑ग्नि - स्थाम् । \newline
20. अश्रि॑ म॒नक् त्य॒नक् त्यश्रि॒ मश्रि॑ म॒नक्ति॒ यज॑मानं॒ ॅयज॑मान म॒नक् त्यश्रि॒ मश्रि॑ म॒नक्ति॒ यज॑मानम् । \newline
21. अ॒नक्ति॒ यज॑मानं॒ ॅयज॑मान म॒नक् त्य॒नक्ति॒ यज॑मान मे॒वैव यज॑मान म॒नक् त्य॒नक्ति॒ यज॑मान मे॒व । \newline
22. यज॑मान मे॒वैव यज॑मानं॒ ॅयज॑मान मे॒व तेज॑सा॒ तेज॑सै॒व यज॑मानं॒ ॅयज॑मान मे॒व तेज॑सा । \newline
23. ए॒व तेज॑सा॒ तेज॑सै॒वैव तेज॑सा ऽनक् त्यनक्ति॒ तेज॑सै॒वैव तेज॑सा ऽनक्ति । \newline
24. तेज॑सा ऽनक् त्यनक्ति॒ तेज॑सा॒ तेज॑सा ऽनक् त्या॒न्त मा॒न्त म॑नक्ति॒ तेज॑सा॒ तेज॑सा ऽनक् त्या॒न्तम् । \newline
25. अ॒न॒क् त्या॒न्त मा॒न्त म॑नक् त्यनक् त्या॒न्त म॑नक् त्यनक् त्या॒न्त म॑नक् त्यनक् त्या॒न्त म॑नक्ति । \newline
26. आ॒न्त म॑नक् त्यनक् त्या॒न्त मा॒न्त म॑नक् त्या॒न्त मा॒न्त म॑नक् त्या॒न्त मा॒न्त म॑नक् त्या॒न्तम् । \newline
27. आ॒न्तमित्या᳚ - अ॒न्तम् । \newline
28. अ॒न॒क् त्या॒न्त मा॒न्त म॑नक् त्यनक् त्या॒न्त मे॒वै वान्त म॑नक् त्यनक् त्या॒न्त मे॒व । \newline
29. आ॒न्त मे॒वै वान्त मा॒न्त मे॒व यज॑मानं॒ ॅयज॑मान मे॒वान्त मा॒न्त मे॒व यज॑मानम् । \newline
30. आ॒न्तमित्या᳚ - अ॒न्तम् । \newline
31. ए॒व यज॑मानं॒ ॅयज॑मान मे॒वैव यज॑मान॒म् तेज॑सा॒ तेज॑सा॒ यज॑मान मे॒वैव यज॑मान॒म् तेज॑सा । \newline
32. यज॑मान॒म् तेज॑सा॒ तेज॑सा॒ यज॑मानं॒ ॅयज॑मान॒म् तेज॑सा ऽनक् त्यनक्ति॒ तेज॑सा॒ यज॑मानं॒ ॅयज॑मान॒म् तेज॑सा ऽनक्ति । \newline
33. तेज॑सा ऽनक् त्यनक्ति॒ तेज॑सा॒ तेज॑सा ऽनक्ति स॒र्वतः॑ स॒र्वतो॑ ऽनक्ति॒ तेज॑सा॒ तेज॑सा ऽनक्ति स॒र्वतः॑ । \newline
34. अ॒न॒क्ति॒ स॒र्वतः॑ स॒र्वतो॑ ऽनक् त्यनक्ति स॒र्वतः॒ परि॒ परि॑ स॒र्वतो॑ ऽनक् त्यनक्ति स॒र्वतः॒ परि॑ । \newline
35. स॒र्वतः॒ परि॒ परि॑ स॒र्वतः॑ स॒र्वतः॒ परि॑ मृशति मृशति॒ परि॑ स॒र्वतः॑ स॒र्वतः॒ परि॑ मृशति । \newline
36. परि॑ मृशति मृशति॒ परि॒ परि॑ मृश॒ त्यप॑रिवर्ग॒ मप॑रिवर्गम् मृशति॒ परि॒ परि॑ मृश॒ त्यप॑रिवर्गम् । \newline
37. मृ॒श॒ त्यप॑रिवर्ग॒ मप॑रिवर्गम् मृशति मृश॒ त्यप॑रिवर्ग मे॒वैवा प॑रिवर्गम् मृशति मृश॒ त्यप॑रिवर्ग मे॒व । \newline
38. अप॑रिवर्ग मे॒वैवा प॑रिवर्ग॒ मप॑रिवर्ग मे॒वास्मि॑न् नस्मिन् ने॒वा प॑रिवर्ग॒ मप॑रिवर्ग मे॒वास्मिन्न्॑ । \newline
39. अप॑रिवर्ग॒मित्यप॑रि - व॒र्ग॒म् । \newline
40. ए॒वास्मि॑न् नस्मिन् ने॒वै वास्मि॒न् तेज॒ स्तेजो᳚ ऽस्मिन् ने॒वै वास्मि॒न् तेजः॑ । \newline
41. अ॒स्मि॒न् तेज॒ स्तेजो᳚ ऽस्मिन् नस्मि॒न् तेजो॑ दधाति दधाति॒ तेजो᳚ ऽस्मिन् नस्मि॒न् तेजो॑ दधाति । \newline
42. तेजो॑ दधाति दधाति॒ तेज॒ स्तेजो॑ दधा॒ त्युदुद् द॑धाति॒ तेज॒ स्तेजो॑ दधा॒ त्युत् । \newline
43. द॒धा॒ त्युदुद् द॑धाति दधा॒ त्युद् दिव॒म् दिव॒ मुद् द॑धाति दधा॒ त्युद् दिव᳚म् । \newline
44. उद् दिव॒म् दिव॒ मुदुद् दिवꣳ॑ स्तभान स्तभान॒ दिव॒ मुदुद् दिवꣳ॑ स्तभान । \newline
45. दिवꣳ॑ स्तभान स्तभान॒ दिव॒म् दिवꣳ॑ स्तभा॒ना स्त॑भान॒ दिव॒म् दिवꣳ॑ स्तभा॒ना । \newline
46. स्त॒भा॒ना स्त॑भान स्तभा॒ना ऽन्तरि॑क्ष म॒न्तरि॑क्ष॒ मा स्त॑भान स्तभा॒ना ऽन्तरि॑क्षम् । \newline
47. आ ऽन्तरि॑क्ष म॒न्तरि॑क्ष॒ मा ऽन्तरि॑क्षम् पृण पृणा॒ न्तरि॑क्ष॒ मा ऽन्तरि॑क्षम् पृण । \newline
48. अ॒न्तरि॑क्षम् पृण पृणा॒ न्तरि॑क्ष म॒न्तरि॑क्षम् पृ॒णे तीति॑ पृणा॒ न्तरि॑क्ष म॒न्तरि॑क्षम् पृ॒णेति॑ । \newline
49. पृ॒णे तीति॑ पृण पृ॒णे त्या॑हा॒ हेति॑ पृण पृ॒णे त्या॑ह । \newline
50. इत्या॑हा॒हे तीत्या॑ है॒षा मे॒षा मा॒हे तीत्या॑ है॒षाम् । \newline
51. आ॒है॒षा मे॒षा मा॑हा है॒षाम् ॅलो॒काना᳚म् ॅलो॒काना॑ मे॒षा मा॑हा है॒षाम् ॅलो॒काना᳚म् । \newline
52. ए॒षाम् ॅलो॒काना᳚म् ॅलो॒काना॑ मे॒षा मे॒षाम् ॅलो॒कानां॒ ॅविधृ॑त्यै॒ विधृ॑त्यै लो॒काना॑ मे॒षा मे॒षाम् ॅलो॒कानां॒ ॅविधृ॑त्यै । \newline
53. लो॒कानां॒ ॅविधृ॑त्यै॒ विधृ॑त्यै लो॒काना᳚म् ॅलो॒कानां॒ ॅविधृ॑त्यै वैष्ण॒व्या वै᳚ष्ण॒व्या विधृ॑त्यै लो॒काना᳚म् ॅलो॒कानां॒ ॅविधृ॑त्यै वैष्ण॒व्या । \newline
54. विधृ॑त्यै वैष्ण॒व्या वै᳚ष्ण॒व्या विधृ॑त्यै॒ विधृ॑त्यै वैष्ण॒व्य र्‌च र्‌चा वै᳚ष्ण॒व्या विधृ॑त्यै॒ विधृ॑त्यै वैष्ण॒व्य र्‌चा । \newline
55. विधृ॑त्या॒ इति॒ वि - धृ॒त्यै॒ । \newline
56. वै॒ष्ण॒व्य र्‌च र्‌चा वै᳚ष्ण॒व्या वै᳚ष्ण॒व्य र्‌चा क॑ल्पयति कल्पय त्यृ॒चा वै᳚ष्ण॒व्या वै᳚ष्ण॒व्य र्‌चा क॑ल्पयति । \newline
57. ऋ॒चा क॑ल्पयति कल्पय त्यृ॒च र्‌चा क॑ल्पयति वैष्ण॒वो वै᳚ष्ण॒वः क॑ल्पय त्यृ॒च र्‌चा क॑ल्पयति वैष्ण॒वः । \newline
\pagebreak
\markright{ TS 6.3.4.4  \hfill https://www.vedavms.in \hfill}

\section{ TS 6.3.4.4 }

\textbf{TS 6.3.4.4 } \newline
\textbf{Samhita Paata} \newline

क॑ल्पयति वैष्ण॒वो वै दे॒वत॑या॒ यूपः॒ स्वयै॒वैनं॑ दे॒वत॑या कल्पयति॒ द्वाभ्यां᳚ कल्पयति द्वि॒पाद् यज॑मानः॒ प्रति॑ष्ठित्यै॒ यं का॒मये॑त॒ तेज॑सैनं दे॒वता॑भिरिन्द्रि॒येण॒ व्य॑र्द्धयेय॒-मित्य॑ग्नि॒ष्ठां तस्याश्रि॑-माहव॒नीया॑दि॒त्थं ॅवे॒त्थं ॅवाऽति॑ नावये॒त् तेज॑सै॒वैनं॑ दे॒वता॑भिरिन्द्रि॒येण॒ व्य॑र्द्धयति॒ यं का॒मये॑त॒ तेज॑सैनं दे॒वता॑भिरिन्द्रि॒येण॒ सम॑र्द्धयेय॒मित्य॑- [  ] \newline

\textbf{Pada Paata} \newline

क॒ल्प॒य॒ति॒ । वै॒ष्ण॒वः । वै । दे॒वत॑या । यूपः॑ । स्वया᳚ । ए॒व । ए॒न॒म् । दे॒वत॑या । क॒ल्प॒य॒ति॒ । द्वाभ्या᳚म् । क॒ल्प॒य॒ति॒ । द्वि॒पादिति॑ द्वि - पात् । यज॑मानः । प्रति॑ष्ठित्या॒ इति॒ प्रति॑ - स्थि॒त्यै॒ । यम् । का॒मये॑त । तेज॑सा । ए॒न॒म् । दे॒वता॑भिः । इ॒न्द्रि॒येण॑ । वीति॑ । अ॒द्‌र्ध॒ये॒य॒म् । इति॑ । अ॒ग्नि॒ष्ठामित्य॑ग्नि - स्थाम् । तस्य॑ । अश्रि᳚म् । आ॒ह॒व॒नीया॒दित्या᳚-ह॒व॒नीया᳚त् । इ॒त्थम् । वा॒ । इ॒त्थम् । वा॒ । अतीति॑ । ना॒व॒ये॒त् । तेज॑सा । ए॒व । ए॒न॒म् । दे॒वता॑भिः । इ॒न्द्रि॒येण॑ । वीति॑ । अ॒द्‌र्ध॒य॒ति॒ । यम् । का॒मये॑त । तेज॑सा । ए॒न॒म् । दे॒वता॑भिः । इ॒न्द्रि॒येण॑ । समिति॑ । अ॒द्‌र्ध॒ये॒य॒म् । इति॑ ।  \newline


\textbf{Krama Paata} \newline

क॒ल्प॒य॒ति॒ वै॒ष्ण॒वः । वै॒ष्ण॒वो वै । वै दे॒वत॑या । दे॒वत॑या॒ यूपः॑ । यूपः॒ स्वया᳚ । स्वयै॒व । ए॒वैन᳚म् । ए॒न॒म् दे॒वत॑या । दे॒वत॑या कल्पयति । क॒ल्प॒य॒ति॒ द्वाभ्या᳚म् । द्वाभ्या᳚म् कल्पयति । क॒ल्प॒य॒ति॒ द्वि॒पात् । द्वि॒पाद् यज॑मानः । द्वि॒पादिति॑ द्वि - पात् । यज॑मानः॒ प्रति॑ष्ठित्यै । प्रति॑ष्ठित्यै॒ यम् । प्रति॑ष्ठित्या॒ इति॒ प्रति॑ - स्थि॒त्यै॒ । यम् का॒मये॑त । का॒मये॑त॒ तेज॑सा । तेज॑सैनम् । ए॒न॒म् दे॒वता॑भिः । दे॒वता॑भिरिन्द्रि॒येण॑ । इ॒न्द्रि॒येण॒ वि । व्य॑र्द्धयेयम् । अ॒र्द्ध॒ये॒य॒मिति॑ । इत्य॑ग्नि॒ष्ठाम् । अ॒ग्नि॒ष्ठाम् तस्य॑ । अ॒ग्नि॒ष्ठामित्य॑ग्नि - स्थाम् । तस्याश्रि᳚म् । अश्रि॑माहव॒नीया᳚त् । आ॒ह॒व॒नीया॑दि॒त्थम् । आ॒ह॒व॒नीया॒दित्या᳚ - ह॒व॒नीया᳚त् । इ॒त्थम् ॅवा᳚ । वे॒त्थम् । इ॒त्थम् ॅवा᳚ । वाऽति॑ । अति॑ नावयेत् । ना॒व॒ये॒त् तेज॑सा । तेज॑सै॒व । ए॒वैन᳚म् । ए॒न॒म् दे॒वता॑भिः । दे॒वता॑भिरिन्द्रि॒येण॑ । इ॒न्द्रि॒येण॒ वि । व्य॑र्द्धयति । अ॒र्द्ध॒य॒ति॒ यम् । यम् का॒मये॑त । का॒मये॑त॒ तेज॑सा । तेज॑सैनम् । ए॒न॒म् दे॒वता॑भिः । दे॒वता॑भिरिन्द्रि॒येण॑ । इ॒न्द्रि॒येण॒ सम् । सम॑र्द्धयेयम् । अ॒र्द्ध॒ये॒य॒मिति॑ । इत्य॑ग्नि॒ष्ठाम् \newline

\textbf{Jatai Paata} \newline

1. क॒ल्प॒य॒ति॒ वै॒ष्ण॒वो वै᳚ष्ण॒वः क॑ल्पयति कल्पयति वैष्ण॒वः । \newline
2. वै॒ष्ण॒वो वै वै वै᳚ष्ण॒वो वै᳚ष्ण॒वो वै । \newline
3. वै दे॒वत॑या दे॒वत॑या॒ वै वै दे॒वत॑या । \newline
4. दे॒वत॑या॒ यूपो॒ यूपो॑ दे॒वत॑या दे॒वत॑या॒ यूपः॑ । \newline
5. यूपः॒ स्वया॒ स्वया॒ यूपो॒ यूपः॒ स्वया᳚ । \newline
6. स्वयै॒ वैव स्वया॒ स्वयै॒व । \newline
7. ए॒वैन॑ मेन मे॒वै वैन᳚म् । \newline
8. ए॒न॒म् दे॒वत॑या दे॒वत॑यैन मेनम् दे॒वत॑या । \newline
9. दे॒वत॑या कल्पयति कल्पयति दे॒वत॑या दे॒वत॑या कल्पयति । \newline
10. क॒ल्प॒य॒ति॒ द्वाभ्या॒म् द्वाभ्या᳚म् कल्पयति कल्पयति॒ द्वाभ्या᳚म् । \newline
11. द्वाभ्या᳚म् कल्पयति कल्पयति॒ द्वाभ्या॒म् द्वाभ्या᳚म् कल्पयति । \newline
12. क॒ल्प॒य॒ति॒ द्वि॒पाद् द्वि॒पात् क॑ल्पयति कल्पयति द्वि॒पात् । \newline
13. द्वि॒पाद् यज॑मानो॒ यज॑मानो द्वि॒पाद् द्वि॒पाद् यज॑मानः । \newline
14. द्वि॒पादिति॑ द्वि - पात् । \newline
15. यज॑मानः॒ प्रति॑ष्ठित्यै॒ प्रति॑ष्ठित्यै॒ यज॑मानो॒ यज॑मानः॒ प्रति॑ष्ठित्यै । \newline
16. प्रति॑ष्ठित्यै॒ यं ॅयम् प्रति॑ष्ठित्यै॒ प्रति॑ष्ठित्यै॒ यम् । \newline
17. प्रति॑ष्ठित्या॒ इति॒ प्रति॑ - स्थि॒त्यै॒ । \newline
18. यम् का॒मये॑त का॒मये॑त॒ यं ॅयम् का॒मये॑त । \newline
19. का॒मये॑त॒ तेज॑सा॒ तेज॑सा का॒मये॑त का॒मये॑त॒ तेज॑सा । \newline
20. तेज॑सैन मेन॒म् तेज॑सा॒ तेज॑सैनम् । \newline
21. ए॒न॒म् दे॒वता॑भिर् दे॒वता॑भि रेन मेनम् दे॒वता॑भिः । \newline
22. दे॒वता॑भि रिन्द्रि॒ये णे᳚न्द्रि॒येण॑ दे॒वता॑भिर् दे॒वता॑भि रिन्द्रि॒येण॑ । \newline
23. इ॒न्द्रि॒येण॒ वि वीन्द्रि॒ये णे᳚न्द्रि॒येण॒ वि । \newline
24. व्य॑र्द्धयेय मर्द्धयेयं॒ ॅवि व्य॑र्द्धयेयम् । \newline
25. अ॒र्द्ध॒ये॒य॒ मिती त्य॑र्द्धयेय मर्द्धयेय॒ मिति॑ । \newline
26. इत्य॑ग्नि॒ष्ठा म॑ग्नि॒ष्ठा मिती त्य॑ग्नि॒ष्ठाम् । \newline
27. अ॒ग्नि॒ष्ठाम् तस्य॒ तस्या᳚ ग्नि॒ष्ठा म॑ग्नि॒ष्ठाम् तस्य॑ । \newline
28. अ॒ग्नि॒ष्ठामित्य॑ग्नि - स्थाम् । \newline
29. तस्याश्रि॒ मश्रि॒म् तस्य॒ तस्याश्रि᳚म् । \newline
30. अश्रि॑ माहव॒नीया॑ दाहव॒नीया॒ दश्रि॒ मश्रि॑ माहव॒नीया᳚त् । \newline
31. आ॒ह॒व॒नीया॑ दि॒त्थ मि॒त्थ मा॑हव॒नीया॑ दाहव॒नीया॑ दि॒त्थम् । \newline
32. आ॒ह॒व॒नीया॒दित्या᳚ - ह॒व॒नीया᳚त् । \newline
33. इ॒त्थं ॅवा॑ वे॒त्थ मि॒त्थं ॅवा᳚ । \newline
34. वे॒त्थ मि॒त्थं ॅवा॑ वे॒त्थम् । \newline
35. इ॒त्थं ॅवा॑ वे॒त्थ मि॒त्थं ॅवा᳚ । \newline
36. वा ऽत्यति॑ वा॒ वा ऽति॑ । \newline
37. अति॑ नावयेन् नावये॒ दत्यति॑ नावयेत् । \newline
38. ना॒व॒ये॒त् तेज॑सा॒ तेज॑सा नावयेन् नावये॒त् तेज॑सा । \newline
39. तेज॑सै॒वैव तेज॑सा॒ तेज॑सै॒व । \newline
40. ए॒वैन॑ मेन मे॒वै वैन᳚म् । \newline
41. ए॒न॒म् दे॒वता॑भिर् दे॒वता॑ भिरेन मेनम् दे॒वता॑भिः । \newline
42. दे॒वता॑भि रिन्द्रि॒ये णे᳚न्द्रि॒येण॑ दे॒वता॑भिर् दे॒वता॑भि रिन्द्रि॒येण॑ । \newline
43. इ॒न्द्रि॒येण॒ वि वीन्द्रि॒ये णे᳚न्द्रि॒येण॒ वि । \newline
44. व्य॑र्द्धय त्यर्द्धयति॒ वि व्य॑र्द्धयति । \newline
45. अ॒र्द्ध॒य॒ति॒ यं ॅय म॑र्द्धय त्यर्द्धयति॒ यम् । \newline
46. यम् का॒मये॑त का॒मये॑त॒ यं ॅयम् का॒मये॑त । \newline
47. का॒मये॑त॒ तेज॑सा॒ तेज॑सा का॒मये॑त का॒मये॑त॒ तेज॑सा । \newline
48. तेज॑सैन मेन॒म् तेज॑सा॒ तेज॑सैनम् । \newline
49. ए॒न॒म् दे॒वता॑भिर् दे॒वता॑भि रेन मेनम् दे॒वता॑भिः । \newline
50. दे॒वता॑भि रिन्द्रि॒ये णे᳚न्द्रि॒येण॑ दे॒वता॑भिर् दे॒वता॑भि रिन्द्रि॒येण॑ । \newline
51. इ॒न्द्रि॒येण॒ सꣳ स मि॑न्द्रि॒ये णे᳚न्द्रि॒येण॒ सम् । \newline
52. स म॑र्द्धयेय मर्द्धयेयꣳ॒॒ सꣳ स म॑र्द्धयेयम् । \newline
53. अ॒र्द्ध॒ये॒य॒ मिती त्य॑र्द्धयेय मर्द्धयेय॒ मिति॑ । \newline
54. इत्य॑ग्नि॒ष्ठा म॑ग्नि॒ष्ठा मिती त्य॑ग्नि॒ष्ठाम् । \newline

\textbf{Ghana Paata } \newline

1. क॒ल्प॒य॒ति॒ वै॒ष्ण॒वो वै᳚ष्ण॒वः क॑ल्पयति कल्पयति वैष्ण॒वो वै वै वै᳚ष्ण॒वः क॑ल्पयति कल्पयति वैष्ण॒वो वै । \newline
2. वै॒ष्ण॒वो वै वै वै᳚ष्ण॒वो वै᳚ष्ण॒वो वै दे॒वत॑या दे॒वत॑या॒ वै वै᳚ष्ण॒वो वै᳚ष्ण॒वो वै दे॒वत॑या । \newline
3. वै दे॒वत॑या दे॒वत॑या॒ वै वै दे॒वत॑या॒ यूपो॒ यूपो॑ दे॒वत॑या॒ वै वै दे॒वत॑या॒ यूपः॑ । \newline
4. दे॒वत॑या॒ यूपो॒ यूपो॑ दे॒वत॑या दे॒वत॑या॒ यूपः॒ स्वया॒ स्वया॒ यूपो॑ दे॒वत॑या दे॒वत॑या॒ यूपः॒ स्वया᳚ । \newline
5. यूपः॒ स्वया॒ स्वया॒ यूपो॒ यूपः॒ स्वयै॒ वैव स्वया॒ यूपो॒ यूपः॒ स्वयै॒व । \newline
6. स्वयै॒ वैव स्वया॒ स्वयै॒ वैन॑ मेन मे॒व स्वया॒ स्वयै॒ वैन᳚म् । \newline
7. ए॒वैन॑ मेन मे॒वै वैन॑म् दे॒वत॑या दे॒वत॑यैन मे॒वै वैन॑म् दे॒वत॑या । \newline
8. ए॒न॒म् दे॒वत॑या दे॒वत॑यैन मेनम् दे॒वत॑या कल्पयति कल्पयति दे॒वत॑यैन मेनम् दे॒वत॑या कल्पयति । \newline
9. दे॒वत॑या कल्पयति कल्पयति दे॒वत॑या दे॒वत॑या कल्पयति॒ द्वाभ्या॒म् द्वाभ्या᳚म् कल्पयति दे॒वत॑या दे॒वत॑या कल्पयति॒ द्वाभ्या᳚म् । \newline
10. क॒ल्प॒य॒ति॒ द्वाभ्या॒म् द्वाभ्या᳚म् कल्पयति कल्पयति॒ द्वाभ्या᳚म् कल्पयति कल्पयति॒ द्वाभ्या᳚म् कल्पयति कल्पयति॒ द्वाभ्या᳚म् कल्पयति । \newline
11. द्वाभ्या᳚म् कल्पयति कल्पयति॒ द्वाभ्या॒म् द्वाभ्या᳚म् कल्पयति द्वि॒पाद् द्वि॒पात् क॑ल्पयति॒ द्वाभ्या॒म् द्वाभ्या᳚म् कल्पयति द्वि॒पात् । \newline
12. क॒ल्प॒य॒ति॒ द्वि॒पाद् द्वि॒पात् क॑ल्पयति कल्पयति द्वि॒पाद् यज॑मानो॒ यज॑मानो द्वि॒पात् क॑ल्पयति कल्पयति द्वि॒पाद् यज॑मानः । \newline
13. द्वि॒पाद् यज॑मानो॒ यज॑मानो द्वि॒पाद् द्वि॒पाद् यज॑मानः॒ प्रति॑ष्ठित्यै॒ प्रति॑ष्ठित्यै॒ यज॑मानो द्वि॒पाद् द्वि॒पाद् यज॑मानः॒ प्रति॑ष्ठित्यै । \newline
14. द्वि॒पादिति॑ द्वि - पात् । \newline
15. यज॑मानः॒ प्रति॑ष्ठित्यै॒ प्रति॑ष्ठित्यै॒ यज॑मानो॒ यज॑मानः॒ प्रति॑ष्ठित्यै॒ यं ॅयम् प्रति॑ष्ठित्यै॒ यज॑मानो॒ यज॑मानः॒ प्रति॑ष्ठित्यै॒ यम् । \newline
16. प्रति॑ष्ठित्यै॒ यं ॅयम् प्रति॑ष्ठित्यै॒ प्रति॑ष्ठित्यै॒ यम् का॒मये॑त का॒मये॑त॒ यम् प्रति॑ष्ठित्यै॒ प्रति॑ष्ठित्यै॒ यम् का॒मये॑त । \newline
17. प्रति॑ष्ठित्या॒ इति॒ प्रति॑ - स्थि॒त्यै॒ । \newline
18. यम् का॒मये॑त का॒मये॑त॒ यं ॅयम् का॒मये॑त॒ तेज॑सा॒ तेज॑सा का॒मये॑त॒ यं ॅयम् का॒मये॑त॒ तेज॑सा । \newline
19. का॒मये॑त॒ तेज॑सा॒ तेज॑सा का॒मये॑त का॒मये॑त॒ तेज॑सैन मेन॒म् तेज॑सा का॒मये॑त का॒मये॑त॒ तेज॑सैनम् । \newline
20. तेज॑सैन मेन॒म् तेज॑सा॒ तेज॑सैनम् दे॒वता॑भिर् दे॒वता॑भि रेन॒म् तेज॑सा॒ तेज॑सैनम् दे॒वता॑भिः । \newline
21. ए॒न॒म् दे॒वता॑भिर् दे॒वता॑भि रेन मेनम् दे॒वता॑भि रिन्द्रि॒ये णे᳚न्द्रि॒येण॑ दे॒वता॑भि रेन मेनम् दे॒वता॑भि रिन्द्रि॒येण॑ । \newline
22. दे॒वता॑भि रिन्द्रि॒ये णे᳚न्द्रि॒येण॑ दे॒वता॑भिर् दे॒वता॑भि रिन्द्रि॒येण॒ वि वीन्द्रि॒येण॑ दे॒वता॑भिर् दे॒वता॑भि रिन्द्रि॒येण॒ वि । \newline
23. इ॒न्द्रि॒येण॒ वि वीन्द्रि॒ये णे᳚न्द्रि॒येण॒ व्य॑र्द्धयेय मर्द्धयेयं॒ ॅवीन्द्रि॒ये णे᳚न्द्रि॒येण॒ व्य॑र्द्धयेयम् । \newline
24. व्य॑र्द्धयेय मर्द्धयेयं॒ ॅवि व्य॑र्द्धयेय॒ मिती त्य॑र्द्धयेयं॒ ॅवि व्य॑र्द्धयेय॒ मिति॑ । \newline
25. अ॒र्द्ध॒ये॒य॒ मिती त्य॑र्द्धयेय मर्द्धयेय॒ मित्य॑ग्नि॒ष्ठा म॑ग्नि॒ष्ठा मित्य॑र्द्धयेय मर्द्धयेय॒ मित्य॑ग्नि॒ष्ठाम् । \newline
26. इत्य॑ग्नि॒ष्ठा म॑ग्नि॒ष्ठा मिती त्य॑ग्नि॒ष्ठाम् तस्य॒ तस्या᳚ ग्नि॒ष्ठा मिती त्य॑ग्नि॒ष्ठाम् तस्य॑ । \newline
27. अ॒ग्नि॒ष्ठाम् तस्य॒ तस्या᳚ ग्नि॒ष्ठा म॑ग्नि॒ष्ठाम् तस्याश्रि॒ मश्रि॒म् तस्या᳚ ग्नि॒ष्ठा म॑ग्नि॒ष्ठाम् तस्याश्रि᳚म् । \newline
28. अ॒ग्नि॒ष्ठामित्य॑ग्नि - स्थाम् । \newline
29. तस्याश्रि॒ मश्रि॒म् तस्य॒ तस्याश्रि॑ माहव॒नीया॑ दाहव॒नीया॒ दश्रि॒म् तस्य॒ तस्याश्रि॑ माहव॒नीया᳚त् । \newline
30. अश्रि॑ माहव॒नीया॑ दाहव॒नीया॒ दश्रि॒ मश्रि॑ माहव॒नीया॑ दि॒त्थ मि॒त्थ मा॑हव॒नीया॒ दश्रि॒ मश्रि॑ माहव॒नीया॑ दि॒त्थम् । \newline
31. आ॒ह॒व॒नीया॑ दि॒त्थ मि॒त्थ मा॑हव॒नीया॑ दाहव॒नीया॑ दि॒त्थं ॅवा॑ वे॒त्थ मा॑हव॒नीया॑ दाहव॒नीया॑ दि॒त्थं ॅवा᳚ । \newline
32. आ॒ह॒व॒नीया॒दित्या᳚ - ह॒व॒नीया᳚त् । \newline
33. इ॒त्थं ॅवा॑ वे॒त्थ मि॒त्थं ॅवे॒त्थ मि॒त्थं ॅवे॒त्थ मि॒त्थं ॅवे॒त्थम् । \newline
34. वे॒त्थ मि॒त्थं ॅवा॑ वे॒त्थं ॅवा॑ वे॒त्थं ॅवा॑ वे॒त्थं ॅवा᳚ । \newline
35. इ॒त्थं ॅवा॑ वे॒त्थ मि॒त्थं ॅवा ऽत्यति॑ वे॒त्थ मि॒त्थं ॅवा ऽति॑ । \newline
36. वा ऽत्यति॑ वा॒ वा ऽति॑ नावयेन् नावये॒ दति॑ वा॒ वा ऽति॑ नावयेत् । \newline
37. अति॑ नावयेन् नावये॒ दत्यति॑ नावये॒त् तेज॑सा॒ तेज॑सा नावये॒ दत्यति॑ नावये॒त् तेज॑सा । \newline
38. ना॒व॒ये॒त् तेज॑सा॒ तेज॑सा नावयेन् नावये॒त् तेज॑सै॒ वैव तेज॑सा नावयेन् नावये॒त् तेज॑सै॒व । \newline
39. तेज॑सै॒ वैव तेज॑सा॒ तेज॑सै॒ वैन॑ मेन मे॒व तेज॑सा॒ तेज॑सै॒ वैन᳚म् । \newline
40. ए॒वैन॑ मेन मे॒वै वैन॑म् दे॒वता॑भिर् दे॒वता॑भि रेन मे॒वै वैन॑म् दे॒वता॑भिः । \newline
41. ए॒न॒म् दे॒वता॑भिर् दे॒वता॑भि रेन मेनम् दे॒वता॑भि रिन्द्रि॒ये णे᳚न्द्रि॒येण॑ दे॒वता॑भि रेन मेनम् दे॒वता॑भि रिन्द्रि॒येण॑ । \newline
42. दे॒वता॑भि रिन्द्रि॒ये णे᳚न्द्रि॒येण॑ दे॒वता॑भिर् दे॒वता॑भि रिन्द्रि॒येण॒ वि वीन्द्रि॒येण॑ दे॒वता॑भिर् दे॒वता॑भि रिन्द्रि॒येण॒ वि । \newline
43. इ॒न्द्रि॒येण॒ वि वीन्द्रि॒ये णे᳚न्द्रि॒येण॒ व्य॑र्द्धय त्यर्द्धयति॒ वीन्द्रि॒ये णे᳚न्द्रि॒येण॒ व्य॑र्द्धयति । \newline
44. व्य॑र्द्धय त्यर्द्धयति॒ वि व्य॑र्द्धयति॒ यं ॅय म॑र्द्धयति॒ वि व्य॑र्द्धयति॒ यम् । \newline
45. अ॒र्द्ध॒य॒ति॒ यं ॅय म॑र्द्धय त्यर्द्धयति॒ यम् का॒मये॑त का॒मये॑त॒ य म॑र्द्धय त्यर्द्धयति॒ यम् का॒मये॑त । \newline
46. यम् का॒मये॑त का॒मये॑त॒ यं ॅयम् का॒मये॑त॒ तेज॑सा॒ तेज॑सा का॒मये॑त॒ यं ॅयम् का॒मये॑त॒ तेज॑सा । \newline
47. का॒मये॑त॒ तेज॑सा॒ तेज॑सा का॒मये॑त का॒मये॑त॒ तेज॑सैन मेन॒म् तेज॑सा का॒मये॑त का॒मये॑त॒ तेज॑सैनम् । \newline
48. तेज॑सैन मेन॒म् तेज॑सा॒ तेज॑सैनम् दे॒वता॑भिर् दे॒वता॑भि रेन॒म् तेज॑सा॒ तेज॑सैनम् दे॒वता॑भिः । \newline
49. ए॒न॒म् दे॒वता॑भिर् दे॒वता॑भि रेन मेनम् दे॒वता॑भि रिन्द्रि॒ये णे᳚न्द्रि॒येण॑ दे॒वता॑भि रेन मेनम् दे॒वता॑भि रिन्द्रि॒येण॑ । \newline
50. दे॒वता॑भि रिन्द्रि॒ये णे᳚न्द्रि॒येण॑ दे॒वता॑भिर् दे॒वता॑भि रिन्द्रि॒येण॒ सꣳ स मि॑न्द्रि॒येण॑ दे॒वता॑भिर् दे॒वता॑भि रिन्द्रि॒येण॒ सम् । \newline
51. इ॒न्द्रि॒येण॒ सꣳ स मि॑न्द्रि॒ये णे᳚न्द्रि॒येण॒ स म॑र्द्धयेय मर्द्धयेयꣳ॒॒ स मि॑न्द्रि॒ये णे᳚न्द्रि॒येण॒ स म॑र्द्धयेयम् । \newline
52. स म॑र्द्धयेय मर्द्धयेयꣳ॒॒ सꣳ स म॑र्द्धयेय॒ मिती त्य॑र्द्धयेयꣳ॒॒ सꣳ स म॑र्द्धयेय॒ मिति॑ । \newline
53. अ॒र्द्ध॒ये॒य॒ मिती त्य॑र्द्धयेय मर्द्धयेय॒ मित्य॑ग्नि॒ष्ठा म॑ग्नि॒ष्ठा मित्य॑र्द्धयेय मर्द्धयेय॒ मित्य॑ग्नि॒ष्ठाम् । \newline
54. इत्य॑ग्नि॒ष्ठा म॑ग्नि॒ष्ठा मिती त्य॑ग्नि॒ष्ठाम् तस्य॒ तस्या᳚ ग्नि॒ष्ठा मिती त्य॑ग्नि॒ष्ठाम् तस्य॑ । \newline
\pagebreak
\markright{ TS 6.3.4.5  \hfill https://www.vedavms.in \hfill}

\section{ TS 6.3.4.5 }

\textbf{TS 6.3.4.5 } \newline
\textbf{Samhita Paata} \newline

-ग्नि॒ष्ठां तस्याश्रि॑माहव॒नीये॑न॒ सं मि॑नुया॒त् तेज॑सै॒वैनं॑ दे॒वता॑भिरिन्द्रि॒येण॒ सम॑र्द्धयति ब्रह्म॒वनिं॑ त्वा क्षत्र॒वनि॒मित्या॑ह यथाय॒जुरे॒वैतत् परि॑ व्यय॒त्यूर्ग्वै र॑श॒ना यज॑मानेन॒ यूपः॒ संमि॑तो॒ यज॑मानमे॒वोर्जा सम॑र्द्धयति नाभिद॒घ्ने परि॑ व्ययति नाभिद॒घ्न ए॒वास्मा॒ ऊर्जं॑ दधाति॒ तस्मा᳚न्नाभिद॒घ्न ऊ॒र्जा भु॑ञ्जते॒ यं का॒मये॑तो॒र्जैनं॒- [  ] \newline

\textbf{Pada Paata} \newline

अ॒ग्नि॒ष्ठामित्य॑ग्नि - स्थाम् । तस्य॑ । अश्रि᳚म् । आ॒ह॒व॒नीये॒नेत्या᳚ - ह॒व॒नीये॑न । समिति॑ । मि॒नु॒या॒त् । तेज॑सा । ए॒व । ए॒न॒म् । दे॒वता॑भिः । इ॒न्द्रि॒येण॑ । समिति॑ । अ॒द्‌र्ध॒य॒ति॒ । ब्र॒ह्म॒वनि॒मिति॑ ब्रह्म - वनि᳚म् । त्वा॒ । क्ष॒त्र॒वनि॒मिति॑ क्षत्र-वनि᳚म् । इति॑ । आ॒ह॒ । य॒था॒य॒जुरिति॑ यथा-य॒जुः । ए॒व । ए॒तत् । परीति॑ । व्य॒य॒ति॒ । ऊर्क् । वै । र॒श॒ना । यज॑मानेन । यूपः॑ । संमि॑त॒ इति॒ सं-मि॒तः॒ । यज॑मानम् । ए॒व । ऊ॒र्जा । समिति॑ । अ॒द्‌र्ध॒य॒ति॒ । ना॒भि॒द॒घ्न इति॑ नाभि - द॒घ्ने । परीति॑ । व्य॒य॒ति॒ । ना॒भि॒द॒घ्न इति॑ नाभि - द॒घ्ने । ए॒व । अ॒स्मै॒ । ऊर्ज᳚म् । द॒धा॒ति॒ । तस्मा᳚त् । ना॒भि॒द॒घ्न इति॑ नाभि - द॒घ्ने । ऊ॒र्जा । भु॒ञ्ज॒ते॒ । यम् । का॒मये॑त । ऊ॒र्जा । ए॒न॒म् ।  \newline


\textbf{Krama Paata} \newline

अ॒ग्नि॒ष्ठाम् तस्य॑ । अ॒ग्नि॒ष्ठामित्य॑ग्नि - स्थाम् । तस्याश्रि᳚म् । अश्रि॑माहव॒नीये॑न । आ॒ह॒व॒नीये॑न॒ सम् । आ॒ह॒व॒नीये॒नेत्या᳚ - ह॒व॒नीये॑न । सम् मि॑नुयात् । मि॒नु॒या॒त् तेज॑सा । तेज॑सै॒व । ए॒वैन᳚म् । ए॒न॒म् दे॒वता॑भिः । दे॒वता॑भिरिन्द्रि॒येण॑ । इ॒न्द्रि॒येण॒ सम् । सम॑र्द्धयति । अ॒र्द्ध॒य॒ति॒ ब्र॒ह्म॒वनि᳚म् । ब्र॒ह्म॒वनि॑म् त्वा । ब्र॒ह्म॒वनि॒मिति॑ ब्रह्म - वनि᳚म् । त्वा॒ क्ष॒त्र॒वनि᳚म् । क्ष॒त्र॒वनि॒मिति॑ । क्ष॒त्र॒वनि॒मिति॑ क्षत्र - वनि᳚म् । इत्या॑ह । आ॒ह॒ य॒था॒य॒जुः । य॒था॒य॒जुरे॒व । य॒था॒य॒जुरिति॑ यथा - य॒जुः । ए॒वैतत् । ए॒तत् परि॑ । परि॑ व्ययति । व्य॒य॒त्यूर्क् । ऊर्ग् वै । वै र॑श॒ना । र॒श॒ना यज॑मानेन । यज॑मानेन॒ यूपः॑ । यूपः॒ सम्मि॑तः । सम्मि॑तो॒ यज॑मानम् । सम्मि॑त॒ इति॒ सम् - मि॒तः॒ । यज॑मानमे॒व । ए॒वोर्जा । ऊ॒र्जा सम् । सम॑र्द्धयति । अ॒र्द्ध॒य॒ति॒ ना॒भि॒द॒घ्ने । ना॒भि॒द॒घ्ने परि॑ । ना॒भि॒द॒घ्न इति॑ नाभि - द॒घ्ने । परि॑ व्ययति । व्य॒य॒ति॒ ना॒भि॒द॒घ्ने । ना॒भि॒द॒घ्न ए॒व । ना॒भि॒द॒घ्न इति॑ नाभि - द॒घ्ने । ए॒वास्मै᳚ । अस्मा॒ ऊर्ज᳚म् । ऊर्ज॑म् दधाति । द॒धा॒ति॒ तस्मा᳚त् । तस्मा᳚न् नाभिद॒घ्ने । ना॒भि॒द॒घ्न ऊ॒र्जा । ना॒भि॒द॒घ्न इति॑ नाभि - द॒घ्ने । ऊ॒र्जा भु॑ञ्जते । भु॒ञ्ज॒ते॒ यम् । यम् का॒मये॑त । का॒मये॑तो॒र्जा । ऊ॒र्जैन᳚म् । ए॒न॒म् ॅवि \newline

\textbf{Jatai Paata} \newline

1. अ॒ग्नि॒ष्ठाम् तस्य॒ तस्या᳚ ग्नि॒ष्ठा म॑ग्नि॒ष्ठाम् तस्य॑ । \newline
2. अ॒ग्नि॒ष्ठामित्य॑ग्नि - स्थाम् । \newline
3. तस्याश्रि॒ मश्रि॒म् तस्य॒ तस्याश्रि᳚म् । \newline
4. अश्रि॑ माहव॒नीये॑ना हव॒नीये॒नाश्रि॒ मश्रि॑ माहव॒नीये॑न । \newline
5. आ॒ह॒व॒नीये॑न॒ सꣳ स मा॑हव॒नीये॑ना हव॒नीये॑न॒ सम् । \newline
6. आ॒ह॒व॒नीये॒नेत्या᳚ - ह॒व॒नीये॑न । \newline
7. सम् मि॑नुयान् मिनुया॒थ् सꣳ सम् मि॑नुयात् । \newline
8. मि॒नु॒या॒त् तेज॑सा॒ तेज॑सा मिनुयान् मिनुया॒त् तेज॑सा । \newline
9. तेज॑सै॒ वैव तेज॑सा॒ तेज॑सै॒व । \newline
10. ए॒वैन॑ मेन मे॒वै वैन᳚म् । \newline
11. ए॒न॒म् दे॒वता॑भिर् दे॒वता॑भि रेन मेनम् दे॒वता॑भिः । \newline
12. दे॒वता॑भि रिन्द्रि॒ये णे᳚न्द्रि॒येण॑ दे॒वता॑भिर् दे॒वता॑भि रिन्द्रि॒येण॑ । \newline
13. इ॒न्द्रि॒येण॒ सꣳ स मि॑न्द्रि॒ये णे᳚न्द्रि॒येण॒ सम् । \newline
14. स म॑र्द्धय त्यर्द्धयति॒ सꣳ स म॑र्द्धयति । \newline
15. अ॒र्द्ध॒य॒ति॒ ब्र॒ह्म॒वनि॑म् ब्रह्म॒वनि॑ मर्द्धय त्यर्द्धयति ब्रह्म॒वनि᳚म् । \newline
16. ब्र॒ह्म॒वनि॑म् त्वा त्वा ब्रह्म॒वनि॑म् ब्रह्म॒वनि॑म् त्वा । \newline
17. ब्र॒ह्म॒वनि॒मिति॑ ब्रह्म - वनि᳚म् । \newline
18. त्वा॒ क्ष॒त्र॒वनि॑म् क्षत्र॒वनि॑म् त्वा त्वा क्षत्र॒वनि᳚म् । \newline
19. क्ष॒त्र॒वनि॒ मितीति॑ क्षत्र॒वनि॑म् क्षत्र॒वनि॒ मिति॑ । \newline
20. क्ष॒त्र॒वनि॒मिति॑ क्षत्र - वनि᳚म् । \newline
21. इत्या॑हा॒हे तीत्या॑ह । \newline
22. आ॒ह॒ य॒था॒य॒जुर् य॑थाय॒जु रा॑हाह यथाय॒जुः । \newline
23. य॒था॒य॒जु रे॒वैव य॑थाय॒जुर् य॑थाय॒जु रे॒व । \newline
24. य॒था॒य॒जुरिति॑ यथा - य॒जुः । \newline
25. ए॒वैत दे॒त दे॒वै वैतत् । \newline
26. ए॒तत् परि॒ पर्ये॒त दे॒तत् परि॑ । \newline
27. परि॑ व्ययति व्ययति॒ परि॒ परि॑ व्ययति । \newline
28. व्य॒य॒ त्यूर् गूर्ग् व्य॑यति व्यय॒ त्यूर्क् । \newline
29. ऊर्ग् वै वा ऊर् गूर्ग् वै । \newline
30. वै र॑श॒ना र॑श॒ना वै वै र॑श॒ना । \newline
31. र॒श॒ना यज॑मानेन॒ यज॑मानेन रश॒ना र॑श॒ना यज॑मानेन । \newline
32. यज॑मानेन॒ यूपो॒ यूपो॒ यज॑मानेन॒ यज॑मानेन॒ यूपः॑ । \newline
33. यूपः॒ सम्मि॑तः॒ सम्मि॑तो॒ यूपो॒ यूपः॒ सम्मि॑तः । \newline
34. सम्मि॑तो॒ यज॑मानं॒ ॅयज॑मानꣳ॒॒ सम्मि॑तः॒ सम्मि॑तो॒ यज॑मानम् । \newline
35. सम्मि॑त॒ इति॒ सं - मि॒तः॒ । \newline
36. यज॑मान मे॒वैव यज॑मानं॒ ॅयज॑मान मे॒व । \newline
37. ए॒वोर् जोर् जैवैवोर्जा । \newline
38. ऊ॒र्जा सꣳ स मू॒र् जोर्जा सम् । \newline
39. स म॑र्द्धय त्यर्द्धयति॒ सꣳ स म॑र्द्धयति । \newline
40. अ॒र्द्ध॒य॒ति॒ ना॒भि॒द॒घ्ने ना॑भिद॒घ्ने᳚ ऽर्द्धय त्यर्द्धयति नाभिद॒घ्ने । \newline
41. ना॒भि॒द॒घ्ने परि॒ परि॑ णाभिद॒घ्णे ना॑भिद॒घ्ने परि॑ । \newline
42. ना॒भि॒द॒घ्न इति॑ नाभि - द॒घ्ने । \newline
43. परि॑ व्ययति व्ययति॒ परि॒ परि॑ व्ययति । \newline
44. व्य॒य॒ति॒ ना॒भि॒द॒घ्ने ना॑भिद॒घ्ने व्य॑यति व्ययति नाभिद॒घ्ने । \newline
45. ना॒भि॒द॒घ्न ए॒वैव ना॑भिद॒घ्ने ना॑भिद॒घ्न ए॒व । \newline
46. ना॒भि॒द॒घ्न इति॑ नाभि - द॒घ्ने । \newline
47. ए॒वास्मा॑ अस्मा ए॒वै वास्मै᳚ । \newline
48. अ॒स्मा॒ ऊर्ज॒ मूर्ज॑ मस्मा अस्मा॒ ऊर्ज᳚म् । \newline
49. ऊर्ज॑म् दधाति दधा॒ त्यूर्ज॒ मूर्ज॑म् दधाति । \newline
50. द॒धा॒ति॒ तस्मा॒त् तस्मा᳚द् दधाति दधाति॒ तस्मा᳚त् । \newline
51. तस्मा᳚न् नाभिद॒घ्ने ना॑भिद॒घ्ने तस्मा॒त् तस्मा᳚न् नाभिद॒घ्ने । \newline
52. ना॒भि॒द॒घ्न ऊ॒र् जोर्जा ना॑भिद॒घ्ने ना॑भिद॒घ्न ऊ॒र्जा । \newline
53. ना॒भि॒द॒घ्न इति॑ नाभि - द॒घ्ने । \newline
54. ऊ॒र्जा भु॑ञ्जते भुञ्जत ऊ॒र् जोर्जा भु॑ञ्जते । \newline
55. भु॒ञ्ज॒ते॒ यं ॅयम् भु॑ञ्जते भुञ्जते॒ यम् । \newline
56. यम् का॒मये॑त का॒मये॑त॒ यं ॅयम् का॒मये॑त । \newline
57. का॒मये॑ तो॒र् जोर्जा का॒मये॑त का॒मये॑ तो॒र्जा । \newline
58. ऊ॒र् जैन॑ मेन मू॒र् जोर् जैन᳚म् । \newline
59. ए॒नं॒ ॅवि व्ये॑न मेनं॒ ॅवि । \newline

\textbf{Ghana Paata } \newline

1. अ॒ग्नि॒ष्ठाम् तस्य॒ तस्या᳚ ग्नि॒ष्ठा म॑ग्नि॒ष्ठाम् तस्याश्रि॒ मश्रि॒म् तस्या᳚ ग्नि॒ष्ठा म॑ग्नि॒ष्ठाम् तस्याश्रि᳚म् । \newline
2. अ॒ग्नि॒ष्ठामित्य॑ग्नि - स्थाम् । \newline
3. तस्याश्रि॒ मश्रि॒म् तस्य॒ तस्याश्रि॑ माहव॒नीये॑ना हव॒नीये॒ना श्रि॒म् तस्य॒ तस्याश्रि॑ माहव॒नीये॑न । \newline
4. अश्रि॑ माहव॒नीये॑ना हव॒नीये॒ना श्रि॒ मश्रि॑ माहव॒नीये॑न॒ सꣳ स मा॑हव॒नीये॒ना श्रि॒ मश्रि॑ माहव॒नीये॑न॒ सम् । \newline
5. आ॒ह॒व॒नीये॑न॒ सꣳ स मा॑हव॒नीये॑ना हव॒नीये॑न॒ सम् मि॑नुयान् मिनुया॒थ् स मा॑हव॒नीये॑ना हव॒नीये॑न॒ सम् मि॑नुयात् । \newline
6. आ॒ह॒व॒नीये॒नेत्या᳚ - ह॒व॒नीये॑न । \newline
7. सम् मि॑नुयान् मिनुया॒थ् सꣳ सम् मि॑नुया॒त् तेज॑सा॒ तेज॑सा मिनुया॒थ् सꣳ सम् मि॑नुया॒त् तेज॑सा । \newline
8. मि॒नु॒या॒त् तेज॑सा॒ तेज॑सा मिनुयान् मिनुया॒त् तेज॑सै॒ वैव तेज॑सा मिनुयान् मिनुया॒त् तेज॑सै॒व । \newline
9. तेज॑सै॒वैव तेज॑सा॒ तेज॑सै॒ वैन॑ मेन मे॒व तेज॑सा॒ तेज॑सै॒ वैन᳚म् । \newline
10. ए॒वैन॑ मेन मे॒वै वैन॑म् दे॒वता॑भिर् दे॒वता॑भि रेन मे॒वै वैन॑म् दे॒वता॑भिः । \newline
11. ए॒न॒म् दे॒वता॑भिर् दे॒वता॑भि रेन मेनम् दे॒वता॑भि रिन्द्रि॒ये णे᳚न्द्रि॒येण॑ दे॒वता॑भि रेन मेनम् दे॒वता॑भि रिन्द्रि॒येण॑ । \newline
12. दे॒वता॑भि रिन्द्रि॒ये णे᳚न्द्रि॒येण॑ दे॒वता॑भिर् दे॒वता॑भि रिन्द्रि॒येण॒ सꣳ स मि॑न्द्रि॒येण॑ दे॒वता॑भिर् दे॒वता॑भि रिन्द्रि॒येण॒ सम् । \newline
13. इ॒न्द्रि॒येण॒ सꣳ स मि॑न्द्रि॒ये णे᳚न्द्रि॒येण॒ स म॑र्द्धय त्यर्द्धयति॒ स मि॑न्द्रि॒ये णे᳚न्द्रि॒येण॒ स म॑र्द्धयति । \newline
14. स म॑र्द्धय त्यर्द्धयति॒ सꣳ स म॑र्द्धयति ब्रह्म॒वनि॑म् ब्रह्म॒वनि॑ मर्द्धयति॒ सꣳ स म॑र्द्धयति ब्रह्म॒वनि᳚म् । \newline
15. अ॒र्द्ध॒य॒ति॒ ब्र॒ह्म॒वनि॑म् ब्रह्म॒वनि॑ मर्द्धय त्यर्द्धयति ब्रह्म॒वनि॑म् त्वा त्वा ब्रह्म॒वनि॑ मर्द्धय त्यर्द्धयति ब्रह्म॒वनि॑म् त्वा । \newline
16. ब्र॒ह्म॒वनि॑म् त्वा त्वा ब्रह्म॒वनि॑म् ब्रह्म॒वनि॑म् त्वा क्षत्र॒वनि॑म् क्षत्र॒वनि॑म् त्वा ब्रह्म॒वनि॑म् ब्रह्म॒वनि॑म् त्वा क्षत्र॒वनि᳚म् । \newline
17. ब्र॒ह्म॒वनि॒मिति॑ ब्रह्म - वनि᳚म् । \newline
18. त्वा॒ क्ष॒त्र॒वनि॑म् क्षत्र॒वनि॑म् त्वा त्वा क्षत्र॒वनि॒ मितीति॑ क्षत्र॒वनि॑म् त्वा त्वा क्षत्र॒वनि॒ मिति॑ । \newline
19. क्ष॒त्र॒वनि॒ मितीति॑ क्षत्र॒वनि॑म् क्षत्र॒वनि॒ मित्या॑हा॒ हेति॑ क्षत्र॒वनि॑म् क्षत्र॒वनि॒ मित्या॑ह । \newline
20. क्ष॒त्र॒वनि॒मिति॑ क्षत्र - वनि᳚म् । \newline
21. इत्या॑हा॒हे तीत्या॑ह यथाय॒जुर् य॑थाय॒जु रा॒हे तीत्या॑ह यथाय॒जुः । \newline
22. आ॒ह॒ य॒था॒य॒जुर् य॑थाय॒जु रा॑हाह यथाय॒जु रे॒वैव य॑थाय॒जु रा॑हाह यथाय॒जु रे॒व । \newline
23. य॒था॒य॒जु रे॒वैव य॑थाय॒जुर् य॑थाय॒जु रे॒वैत दे॒त दे॒व य॑थाय॒जुर् य॑थाय॒जु रे॒वैतत् । \newline
24. य॒था॒य॒जुरिति॑ यथा - य॒जुः । \newline
25. ए॒वैत दे॒त दे॒वै वैतत् परि॒ पर्ये॒त दे॒वै वैतत् परि॑ । \newline
26. ए॒तत् परि॒ पर्ये॒त दे॒तत् परि॑ व्ययति व्ययति॒ पर्ये॒त दे॒तत् परि॑ व्ययति । \newline
27. परि॑ व्ययति व्ययति॒ परि॒ परि॑ व्यय॒ त्यूर् गूर्ग् व्य॑यति॒ परि॒ परि॑ व्यय॒ त्यूर्क् । \newline
28. व्य॒य॒ त्यूर् गूर्ग् व्य॑यति व्यय॒ त्यूर्ग् वै वा ऊर्ग् व्य॑यति व्यय॒ त्यूर्ग् वै । \newline
29. ऊर्ग् वै वा ऊर् गूर्ग् वै र॑श॒ना र॑श॒ना वा ऊर् गूर्ग् वै र॑श॒ना । \newline
30. वै र॑श॒ना र॑श॒ना वै वै र॑श॒ना यज॑मानेन॒ यज॑मानेन रश॒ना वै वै र॑श॒ना यज॑मानेन । \newline
31. र॒श॒ना यज॑मानेन॒ यज॑मानेन रश॒ना र॑श॒ना यज॑मानेन॒ यूपो॒ यूपो॒ यज॑मानेन रश॒ना र॑श॒ना यज॑मानेन॒ यूपः॑ । \newline
32. यज॑मानेन॒ यूपो॒ यूपो॒ यज॑मानेन॒ यज॑मानेन॒ यूपः॒ सम्मि॑तः॒ सम्मि॑तो॒ यूपो॒ यज॑मानेन॒ यज॑मानेन॒ यूपः॒ सम्मि॑तः । \newline
33. यूपः॒ सम्मि॑तः॒ सम्मि॑तो॒ यूपो॒ यूपः॒ सम्मि॑तो॒ यज॑मानं॒ ॅयज॑मानꣳ॒॒ सम्मि॑तो॒ यूपो॒ यूपः॒ सम्मि॑तो॒ यज॑मानम् । \newline
34. सम्मि॑तो॒ यज॑मानं॒ ॅयज॑मानꣳ॒॒ सम्मि॑तः॒ सम्मि॑तो॒ यज॑मान मे॒वैव यज॑मानꣳ॒॒ सम्मि॑तः॒ सम्मि॑तो॒ यज॑मान मे॒व । \newline
35. सम्मि॑त॒ इति॒ सं - मि॒तः॒ । \newline
36. यज॑मान मे॒वैव यज॑मानं॒ ॅयज॑मान मे॒वोर्जोर्जैव यज॑मानं॒ ॅयज॑मान मे॒वोर्जा । \newline
37. ए॒वोर्जोर्जै वैवोर्जा सꣳ स मू॒र्जै वैवोर्जा सम् । \newline
38. ऊ॒र्जा सꣳ स मू॒र्जोर्जा स म॑र्द्धय त्यर्द्धयति॒ स मू॒र्जोर्जा स म॑र्द्धयति । \newline
39. स म॑र्द्धय त्यर्द्धयति॒ सꣳ स म॑र्द्धयति नाभिद॒घ्ने ना॑भिद॒घ्ने᳚ ऽर्द्धयति॒ सꣳ स म॑र्द्धयति नाभिद॒घ्ने । \newline
40. अ॒र्द्ध॒य॒ति॒ ना॒भि॒द॒घ्ने ना॑भिद॒घ्ने᳚ ऽर्द्धय त्यर्द्धयति नाभिद॒घ्ने परि॒ परि॑ णाभिद॒घ्णे᳚ ऽर्द्धय त्यर्द्धयति नाभिद॒घ्ने परि॑ । \newline
41. ना॒भि॒द॒घ्ने परि॒ परि॑ णाभिद॒घ्णे ना॑भिद॒घ्ने परि॑ व्ययति व्ययति॒ परि॑ णाभिद॒घ्णे ना॑भिद॒घ्ने परि॑ व्ययति । \newline
42. ना॒भि॒द॒घ्न इति॑ नाभि - द॒घ्ने । \newline
43. परि॑ व्ययति व्ययति॒ परि॒ परि॑ व्ययति नाभिद॒घ्ने ना॑भिद॒घ्ने व्य॑यति॒ परि॒ परि॑ व्ययति नाभिद॒घ्ने । \newline
44. व्य॒य॒ति॒ ना॒भि॒द॒घ्ने ना॑भिद॒घ्ने व्य॑यति व्ययति नाभिद॒घ्न ए॒वैव ना॑भिद॒घ्ने व्य॑यति व्ययति नाभिद॒घ्न ए॒व । \newline
45. ना॒भि॒द॒घ्न ए॒वैव ना॑भिद॒घ्ने ना॑भिद॒घ्न ए॒वास्मा॑ अस्मा ए॒व ना॑भिद॒घ्ने ना॑भिद॒घ्न ए॒वास्मै᳚ । \newline
46. ना॒भि॒द॒घ्न इति॑ नाभि - द॒घ्ने । \newline
47. ए॒वास्मा॑ अस्मा ए॒वै वास्मा॒ ऊर्ज॒ मूर्ज॑ मस्मा ए॒वै वास्मा॒ ऊर्ज᳚म् । \newline
48. अ॒स्मा॒ ऊर्ज॒ मूर्ज॑ मस्मा अस्मा॒ ऊर्ज॑म् दधाति दधा॒ त्यूर्ज॑ मस्मा अस्मा॒ ऊर्ज॑म् दधाति । \newline
49. ऊर्ज॑म् दधाति दधा॒ त्यूर्ज॒ मूर्ज॑म् दधाति॒ तस्मा॒त् तस्मा᳚द् दधा॒ त्यूर्ज॒ मूर्ज॑म् दधाति॒ तस्मा᳚त् । \newline
50. द॒धा॒ति॒ तस्मा॒त् तस्मा᳚द् दधाति दधाति॒ तस्मा᳚न् नाभिद॒घ्ने ना॑भिद॒घ्ने तस्मा᳚द् दधाति दधाति॒ तस्मा᳚न् नाभिद॒घ्ने । \newline
51. तस्मा᳚न् नाभिद॒घ्ने ना॑भिद॒घ्ने तस्मा॒त् तस्मा᳚न् नाभिद॒घ्न ऊ॒र्जोर्जा ना॑भिद॒घ्ने तस्मा॒त् तस्मा᳚न् नाभिद॒घ्न ऊ॒र्जा । \newline
52. ना॒भि॒द॒घ्न ऊ॒र्जोर्जा ना॑भिद॒घ्ने ना॑भिद॒घ्न ऊ॒र्जा भु॑ञ्जते भुञ्जत ऊ॒र्जा ना॑भिद॒घ्ने ना॑भिद॒घ्न ऊ॒र्जा भु॑ञ्जते । \newline
53. ना॒भि॒द॒घ्न इति॑ नाभि - द॒घ्ने । \newline
54. ऊ॒र्जा भु॑ञ्जते भुञ्जत ऊ॒र्जोर्जा भु॑ञ्जते॒ यं ॅयम् भु॑ञ्जत ऊ॒र्जोर्जा भु॑ञ्जते॒ यम् । \newline
55. भु॒ञ्ज॒ते॒ यं ॅयम् भु॑ञ्जते भुञ्जते॒ यम् का॒मये॑त का॒मये॑त॒ यम् भु॑ञ्जते भुञ्जते॒ यम् का॒मये॑त । \newline
56. यम् का॒मये॑त का॒मये॑त॒ यं ॅयम् का॒मये॑ तो॒र्जोर्जा का॒मये॑त॒ यं ॅयम् का॒मये॑तो॒र्जा । \newline
57. का॒मये॑ तो॒र्जोर्जा का॒मये॑त का॒मये॑ तो॒र्जैन॑ मेन मू॒र्जा का॒मये॑त का॒मये॑ तो॒र्जैन᳚म् । \newline
58. ऊ॒र्जैन॑ मेन मू॒र्जोर्जैनं॒ ॅवि व्ये॑न मू॒र्जोर्जैनं॒ ॅवि । \newline
59. ए॒नं॒ ॅवि व्ये॑न मेनं॒ ॅव्य॑र्द्धयेय मर्द्धयेयं॒ ॅव्ये॑न मेनं॒ ॅव्य॑र्द्धयेयम् । \newline
\pagebreak
\markright{ TS 6.3.4.6  \hfill https://www.vedavms.in \hfill}

\section{ TS 6.3.4.6 }

\textbf{TS 6.3.4.6 } \newline
\textbf{Samhita Paata} \newline

ॅव्य॑र्द्धयेय॒-मित्यू॒र्द्ध्वां ॅवा॒ तस्यावा॑चीं॒ ॅवाऽवो॑हेदू॒र्जैवैनं॒ ॅव्य॑र्द्धयति॒ यदि॑ का॒मये॑त॒ वर्.षु॑कः प॒र्जन्यः॑ स्या॒दित्य-वा॑ची॒मवो॑हे॒द् वृष्टि॑मे॒व नि य॑च्छति॒ यदि॑ का॒मये॒ताव॑र्.षुकः स्या॒दित्यू॒र्द्ध्वामुदू॑हे॒द् वृष्टि॑मे॒वोद् य॑च्छति पितृ॒णां निखा॑तं मनु॒ष्या॑णामू॒र्द्ध्वं निखा॑ता॒दा र॑श॒नाया॒ ओष॑धीनाꣳ रश॒ना विश्वे॑षां- [  ] \newline

\textbf{Pada Paata} \newline

वीति॑ । अ॒द्‌र्ध॒ये॒य॒म् । इति॑ । ऊ॒द्‌र्ध्वाम् । वा॒ । तस्य॑ । अवा॑चीम् । वा॒ । अवेति॑ । ऊ॒हे॒त् । ऊ॒र्जा । ए॒व । ए॒न॒म् । वीति॑ । अ॒द्‌र्ध॒य॒ति॒ । यदि॑ । का॒मये॑त । वर्.षु॑कः । प॒र्जन्यः॑ । स्या॒त् । इति॑ । अवा॑चीम् । अवेति॑ । ऊ॒हे॒त् । वृष्टि᳚म् । ए॒व । नीति॑ । य॒च्छ॒ति॒ । यदि॑ । का॒मये॑त । अव॑र्.षुकः । स्या॒त् । इति॑ । ऊ॒द्‌र्ध्वाम् । उदिति॑ । ऊ॒हे॒त् । वृष्टि᳚म् । ए॒व । उदिति॑ । य॒च्छ॒ति॒ । पि॒तृ॒णाम् । निखा॑त॒मिति॒ नि - खा॒त॒म् । म॒नु॒ष्या॑णाम् । ऊ॒द्‌र्ध्वम् । निखा॑ता॒दिति॒ नि - खा॒ता॒त् । एति॑ । र॒श॒नायाः᳚ । ओष॑धीनाम् । र॒श॒ना । विश्वे॑षां ।  \newline


\textbf{Krama Paata} \newline

व्य॑र्द्धयेयम् । अ॒र्द्ध॒ये॒य॒मिति॑ । इत्यू॒र्द्ध्वाम् । ऊ॒र्द्ध्वाम् ॅवा᳚ । वा॒ तस्य॑ । तस्यावा॑चीम् । अवा॑चीम् ॅवा । वाऽव॑ । अवो॑हेत् । ऊ॒हे॒दू॒र्जा । ऊ॒र्जैव । ए॒वैन᳚म् । ए॒न॒म् ॅवि । व्य॑र्द्धयति । अ॒र्द्ध॒य॒ति॒ यदि॑ । यदि॑ का॒मये॑त । का॒मये॑त॒ वर्.षु॑कः । वर्.षु॑कः प॒र्जन्यः॑ । प॒र्जन्यः॑ स्यात् । स्या॒दिति॑ । इत्यवा॑चीम् । अवा॑ची॒मव॑ । अवो॑हेत् । ऊ॒हे॒द् वृष्टि᳚म् । वृष्टि॑मे॒व । ए॒व नि । नि य॑च्छति । य॒च्छ॒ति॒ यदि॑ । यदि॑ का॒मये॑त । का॒मये॒ताव॑र्.षुकः । अव॑र्.षुकः स्यात् । स्या॒दिति॑ । इत्यू॒र्द्ध्वाम् । ऊ॒र्द्ध्वामुत् । उदू॑हेत् । ऊ॒हे॒द् वृष्टि᳚म् । वृष्टि॑मे॒व । ए॒वोत् । उद् य॑च्छति । य॒च्छ॒ति॒ पि॒तृ॒णाम् । पि॒तृ॒णाम् निखा॑तम् । निखा॑तम् मनु॒ष्या॑णाम् । निखा॑त॒मिति॒ नि - खा॒त॒म् । म॒नु॒ष्या॑णामू॒र्द्ध्वम् । ऊ॒र्द्ध्वम् निखा॑तात् । निखा॑ता॒दा । निखा॑ता॒दिति॒ नि - खा॒ता॒त्॒ । आ र॑श॒नायाः᳚ । र॒श॒नाया॒ ओष॑धीनाम् । ओष॑धीनाꣳ रश॒ना । र॒श॒ना विश्वे॑षाम् । विश्वे॑षाम् दे॒वाना᳚म् \newline

\textbf{Jatai Paata} \newline

1. व्य॑र्द्धयेय मर्द्धयेयं॒ ॅवि व्य॑र्द्धयेयम् । \newline
2. अ॒र्द्ध॒ये॒य॒ मिती त्य॑र्द्धयेय मर्द्धयेय॒ मिति॑ । \newline
3. इत्यू॒र्द्ध्वा मू॒र्द्ध्वा मिती त्यू॒र्द्ध्वाम् । \newline
4. ऊ॒र्द्ध्वां ॅवा॑ वो॒र्द्ध्वा मू॒र्द्ध्वां ॅवा᳚ । \newline
5. वा॒ तस्य॒ तस्य॑ वा वा॒ तस्य॑ । \newline
6. तस्या वा॑ची॒ मवा॑ची॒म् तस्य॒ तस्या वा॑चीम् । \newline
7. अवा॑चीं ॅवा॒ वा ऽवा॑ची॒ मवा॑चीं ॅवा । \newline
8. वा ऽवाव॑ वा॒ वा ऽव॑ । \newline
9. अवो॑हे दूहे॒ दवा वो॑हेत् । \newline
10. ऊ॒हे॒ दू॒र् जोर् जोहे॑ दूहे दू॒र्जा । \newline
11. ऊ॒र् जैवै वोर् जोर् जैव । \newline
12. ए॒वैन॑ मेन मे॒वै वैन᳚म् । \newline
13. ए॒नं॒ ॅवि व्ये॑न मेनं॒ ॅवि । \newline
14. व्य॑र्द्धय त्यर्द्धयति॒ वि व्य॑र्द्धयति । \newline
15. अ॒र्द्ध॒य॒ति॒ यदि॒ यद्य॑र्द्धय त्यर्द्धयति॒ यदि॑ । \newline
16. यदि॑ का॒मये॑त का॒मये॑त॒ यदि॒ यदि॑ का॒मये॑त । \newline
17. का॒मये॑त॒ वर्.षु॑को॒ वर्.षु॑कः का॒मये॑त का॒मये॑त॒ वर्.षु॑कः । \newline
18. वर्.षु॑कः प॒र्जन्यः॑ प॒र्जन्यो॒ वर्.षु॑को॒ वर्.षु॑कः प॒र्जन्यः॑ । \newline
19. प॒र्जन्यः॑ स्याथ् स्यात् प॒र्जन्यः॑ प॒र्जन्यः॑ स्यात् । \newline
20. स्या॒दि तीति॑ स्याथ् स्या॒ दिति॑ । \newline
21. इत्यवा॑ची॒ मवा॑ची॒ मिती त्यवा॑चीम् । \newline
22. अवा॑ची॒ मवावा वा॑ची॒ मवा॑ची॒ मव॑ । \newline
23. अवो॑हे दूहे॒ दवा वो॑हेत् । \newline
24. ऊ॒हे॒द् वृष्टिं॒ ॅवृष्टि॑ मूहे दूहे॒द् वृष्टि᳚म् । \newline
25. वृष्टि॑ मे॒वैव वृष्टिं॒ ॅवृष्टि॑ मे॒व । \newline
26. ए॒व नि न्ये॑वैव नि । \newline
27. नि य॑च्छति यच्छति॒ नि नि य॑च्छति । \newline
28. य॒च्छ॒ति॒ यदि॒ यदि॑ यच्छति यच्छति॒ यदि॑ । \newline
29. यदि॑ का॒मये॑त का॒मये॑त॒ यदि॒ यदि॑ का॒मये॑त । \newline
30. का॒मये॒ता व॑र्.षु॒को ऽव॑र्.षुकः का॒मये॑त का॒मये॒ता व॑र्.षुकः । \newline
31. अव॑र्.षुकः स्याथ् स्या॒ दव॑र्.षु॒को ऽव॑र्.षुकः स्यात् । \newline
32. स्या॒दि तीति॑ स्याथ् स्या॒ दिति॑ । \newline
33. इत्यू॒र्द्ध्वा मू॒र्द्ध्वा मिती त्यू॒र्द्ध्वाम् । \newline
34. ऊ॒र्द्ध्वा मुदु दू॒र्द्ध्वा मू॒र्द्ध्वा मुत् । \newline
35. उदू॑हे दूहे॒ दुदु दू॑हेत् । \newline
36. ऊ॒हे॒द् वृष्टिं॒ ॅवृष्टि॑ मूहे दूहे॒द् वृष्टि᳚म् । \newline
37. वृष्टि॑ मे॒वैव वृष्टिं॒ ॅवृष्टि॑ मे॒व । \newline
38. ए॒वो दुदे॒ वैवोत् । \newline
39. उद् य॑च्छति यच्छ॒ त्युदुद् य॑च्छति । \newline
40. य॒च्छ॒ति॒ पि॒तृ॒णाम् पि॑तृ॒णां ॅय॑च्छति यच्छति पितृ॒णाम् । \newline
41. पि॒तृ॒णाम् निखा॑त॒म् निखा॑तम् पितृ॒णाम् पि॑तृ॒णाम् निखा॑तम् । \newline
42. निखा॑तम् मनु॒ष्या॑णाम् मनु॒ष्या॑णा॒म् निखा॑त॒म् निखा॑तम् मनु॒ष्या॑णाम् । \newline
43. निखा॑त॒मिति॒ नि - खा॒त॒म् । \newline
44. म॒नु॒ष्या॑णा मू॒र्द्ध्व मू॒र्द्ध्वम् म॑नु॒ष्या॑णाम् मनु॒ष्या॑णा मू॒र्द्ध्वम् । \newline
45. ऊ॒र्द्ध्वन् निखा॑ता॒न् निखा॑ता दू॒र्द्ध्व मू॒र्द्ध्वन् निखा॑तात् । \newline
46. निखा॑ता॒दा निखा॑ता॒न् निखा॑ता॒दा । \newline
47. निखा॑ता॒दिति॒ नि - खा॒ता॒त् । \newline
48. आ र॑श॒नाया॑ रश॒नाया॒ आ र॑श॒नायाः᳚ । \newline
49. र॒श॒नाया॒ ओष॑धीना॒ मोष॑धीनाꣳ रश॒नाया॑ रश॒नाया॒ ओष॑धीनाम् । \newline
50. ओष॑धीनाꣳ रश॒ना र॑श॒ नौष॑धीना॒ मोष॑धीनाꣳ रश॒ना । \newline
51. र॒श॒ना विश्वे॑षां॒ विश्वे॑षाꣳ रश॒ना र॑श॒ना विश्वे॑षां । \newline
52. विश्वे॑षां दे॒वाना᳚म् दे॒वानां॒ ॅविश्वे॑षां॒ विश्वे॑षां दे॒वाना᳚म् । \newline

\textbf{Ghana Paata } \newline

1. व्य॑र्द्धयेय मर्द्धयेयं॒ ॅवि व्य॑र्द्धयेय॒ मिती त्य॑र्द्धयेयं॒ ॅवि व्य॑र्द्धयेय॒ मिति॑ । \newline
2. अ॒र्द्ध॒ये॒य॒ मिती त्य॑र्द्धयेय मर्द्धयेय॒ मित्यू॒र्द्ध्वा मू॒र्द्ध्वा मित्य॑र्द्धयेय मर्द्धयेय॒ मित्यू॒र्द्ध्वाम् । \newline
3. इत्यू॒र्द्ध्वा मू॒र्द्ध्वा मिती त्यू॒र्द्ध्वां ॅवा॑ वो॒र्द्ध्वा मिती त्यू॒र्द्ध्वां ॅवा᳚ । \newline
4. ऊ॒र्द्ध्वां ॅवा॑ वो॒र्द्ध्वा मू॒र्द्ध्वां ॅवा॒ तस्य॒ तस्य॑ वो॒र्द्ध्वा मू॒र्द्ध्वां ॅवा॒ तस्य॑ । \newline
5. वा॒ तस्य॒ तस्य॑ वा वा॒ तस्या वा॑ची॒ मवा॑ची॒म् तस्य॑ वा वा॒ तस्या वा॑चीम् । \newline
6. तस्या वा॑ची॒ मवा॑ची॒म् तस्य॒ तस्या वा॑चीं ॅवा॒ वा ऽवा॑ची॒म् तस्य॒ तस्या वा॑चीं ॅवा । \newline
7. अवा॑चीं ॅवा॒ वा ऽवा॑ची॒ मवा॑चीं॒ ॅवा ऽवाव॒ वा ऽवा॑ची॒ मवा॑चीं॒ ॅवा ऽव॑ । \newline
8. वा ऽवाव॑ वा॒ वा ऽवो॑हे दूहे॒ दव॑ वा॒ वा ऽवो॑हेत् । \newline
9. अवो॑हे दूहे॒ दवावो॑हे दू॒र्जोर्जोहे॒ दवावो॑हे दू॒र्जा । \newline
10. ऊ॒हे॒ दू॒र्जोर्जोहे॑ दूहे दू॒र्जै वैवोर्जोहे॑ दूहे दू॒र्जैव । \newline
11. ऊ॒र्जै वैवोर्जोर्जै वैन॑ मेन मे॒वोर्जोर्जै वैन᳚म् । \newline
12. ए॒वैन॑ मेन मे॒वै वैनं॒ ॅवि व्ये॑न मे॒वै वैनं॒ ॅवि । \newline
13. ए॒नं॒ ॅवि व्ये॑न मेनं॒ ॅव्य॑र्द्धय त्यर्द्धयति॒ व्ये॑न मेनं॒ ॅव्य॑र्द्धयति । \newline
14. व्य॑र्द्धय त्यर्द्धयति॒ वि व्य॑र्द्धयति॒ यदि॒ यद्य॑र्द्धयति॒ वि व्य॑र्द्धयति॒ यदि॑ । \newline
15. अ॒र्द्ध॒य॒ति॒ यदि॒ यद्य॑र्द्धय त्यर्द्धयति॒ यदि॑ का॒मये॑त का॒मये॑त॒ यद्य॑र्द्धय त्यर्द्धयति॒ यदि॑ का॒मये॑त । \newline
16. यदि॑ का॒मये॑त का॒मये॑त॒ यदि॒ यदि॑ का॒मये॑त॒ वर्.षु॑को॒ वर्.षु॑कः का॒मये॑त॒ यदि॒ यदि॑ का॒मये॑त॒ वर्.षु॑कः । \newline
17. का॒मये॑त॒ वर्.षु॑को॒ वर्.षु॑कः का॒मये॑त का॒मये॑त॒ वर्.षु॑कः प॒र्जन्यः॑ प॒र्जन्यो॒ वर्.षु॑कः का॒मये॑त का॒मये॑त॒ वर्.षु॑कः प॒र्जन्यः॑ । \newline
18. वर्.षु॑कः प॒र्जन्यः॑ प॒र्जन्यो॒ वर्.षु॑को॒ वर्.षु॑कः प॒र्जन्यः॑ स्याथ् स्यात् प॒र्जन्यो॒ वर्.षु॑को॒ वर्.षु॑कः प॒र्जन्यः॑ स्यात् । \newline
19. प॒र्जन्यः॑ स्याथ् स्यात् प॒र्जन्यः॑ प॒र्जन्यः॑ स्या॒ दितीति॑ स्यात् प॒र्जन्यः॑ प॒र्जन्यः॑ स्या॒ दिति॑ । \newline
20. स्या॒ दितीति॑ स्याथ् स्या॒ दित्यवा॑ची॒ मवा॑ची॒ मिति॑ स्याथ् स्या॒ दित्यवा॑चीम् । \newline
21. इत्यवा॑ची॒ मवा॑ची॒ मिती त्यवा॑ची॒ मवावा वा॑ची॒ मिती त्यवा॑ची॒ मव॑ । \newline
22. अवा॑ची॒ मवावा वा॑ची॒ मवा॑ची॒ मवो॑हे दूहे॒ दवा वा॑ची॒ मवा॑ची॒ मवो॑हेत् । \newline
23. अवो॑हे दूहे॒ दवावो॑हे॒द् वृष्टिं॒ ॅवृष्टि॑ मूहे॒ दवावो॑हे॒द् वृष्टि᳚म् । \newline
24. ऊ॒हे॒द् वृष्टिं॒ ॅवृष्टि॑ मूहे दूहे॒द् वृष्टि॑ मे॒वैव वृष्टि॑ मूहे दूहे॒द् वृष्टि॑ मे॒व । \newline
25. वृष्टि॑ मे॒वैव वृष्टिं॒ ॅवृष्टि॑ मे॒व नि न्ये॑व वृष्टिं॒ ॅवृष्टि॑ मे॒व नि । \newline
26. ए॒व नि न्ये॑वैव नि य॑च्छति यच्छति॒ न्ये॑वैव नि य॑च्छति । \newline
27. नि य॑च्छति यच्छति॒ नि नि य॑च्छति॒ यदि॒ यदि॑ यच्छति॒ नि नि य॑च्छति॒ यदि॑ । \newline
28. य॒च्छ॒ति॒ यदि॒ यदि॑ यच्छति यच्छति॒ यदि॑ का॒मये॑त का॒मये॑त॒ यदि॑ यच्छति यच्छति॒ यदि॑ का॒मये॑त । \newline
29. यदि॑ का॒मये॑त का॒मये॑त॒ यदि॒ यदि॑ का॒मये॒ता व॑र्.षु॒को ऽव॑र्.षुकः का॒मये॑त॒ यदि॒ यदि॑ का॒मये॒ता व॑र्.षुकः । \newline
30. का॒मये॒ता व॑र्.षु॒को ऽव॑र्.षुकः का॒मये॑त का॒मये॒ता व॑र्.षुकः स्याथ् स्या॒ दव॑र्.षुकः का॒मये॑त का॒मये॒ता व॑र्.षुकः स्यात् । \newline
31. अव॑र्.षुकः स्याथ् स्या॒ दव॑र्.षु॒को ऽव॑र्.षुकः स्या॒ दितीति॑ स्या॒ दव॑र्.षु॒को ऽव॑र्.षुकः स्या॒दिति॑ । \newline
32. स्या॒ दितीति॑ स्याथ् स्या॒ दित्यू॒र्द्ध्वा मू॒र्द्ध्वा मिति॑ स्याथ् स्या॒ दित्यू॒र्द्ध्वाम् । \newline
33. इत्यू॒र्द्ध्वा मू॒र्द्ध्वा मिती त्यू॒र्द्ध्वा मुदु दू॒र्द्ध्वा मिती त्यू॒र्द्ध्वा मुत् । \newline
34. ऊ॒र्द्ध्वा मुदु दू॒र्द्ध्वा मू॒र्द्ध्वा मुदू॑हे दूहे॒ दुदू॒र्द्ध्वा मू॒र्द्ध्वा मुदू॑हेत् । \newline
35. उदू॑हे दूहे॒ दुदु दू॑हे॒द् वृष्टिं॒ ॅवृष्टि॑ मूहे॒ दुदु दू॑हे॒द् वृष्टि᳚म् । \newline
36. ऊ॒हे॒द् वृष्टिं॒ ॅवृष्टि॑ मूहे दूहे॒द् वृष्टि॑ मे॒वैव वृष्टि॑ मूहे दूहे॒द् वृष्टि॑ मे॒व । \newline
37. वृष्टि॑ मे॒वैव वृष्टिं॒ ॅवृष्टि॑ मे॒वोदु दे॒व वृष्टिं॒ ॅवृष्टि॑ मे॒वोत् । \newline
38. ए॒वोदु दे॒वैवोद् य॑च्छति यच्छ॒ त्यु दे॒वैवोद् य॑च्छति । \newline
39. उद् य॑च्छति यच्छ॒ त्युदुद् य॑च्छति पितृ॒णाम् पि॑तृ॒णां ॅय॑च्छ॒ त्युदुद् य॑च्छति पितृ॒णाम् । \newline
40. य॒च्छ॒ति॒ पि॒तृ॒णाम् पि॑तृ॒णां ॅय॑च्छति यच्छति पितृ॒णाम् निखा॑त॒म् निखा॑तम् पितृ॒णां ॅय॑च्छति यच्छति पितृ॒णाम् निखा॑तम् । \newline
41. पि॒तृ॒णाम् निखा॑त॒म् निखा॑तम् पितृ॒णाम् पि॑तृ॒णाम् निखा॑तम् मनु॒ष्या॑णाम् मनु॒ष्या॑णा॒म् निखा॑तम् पितृ॒णाम् पि॑तृ॒णाम् निखा॑तम् मनु॒ष्या॑णाम् । \newline
42. निखा॑तम् मनु॒ष्या॑णाम् मनु॒ष्या॑णा॒म् निखा॑त॒म् निखा॑तम् मनु॒ष्या॑णा मू॒र्द्ध्व मू॒र्द्ध्वम् म॑नु॒ष्या॑णा॒म् निखा॑त॒म् निखा॑तम् मनु॒ष्या॑णा मू॒र्द्ध्वम् । \newline
43. निखा॑त॒मिति॒ नि - खा॒त॒म् । \newline
44. म॒नु॒ष्या॑णा मू॒र्द्ध्व मू॒र्द्ध्वम् म॑नु॒ष्या॑णाम् मनु॒ष्या॑णा मू॒र्द्ध्वम् निखा॑ता॒न् निखा॑ता दू॒र्द्ध्वम् म॑नु॒ष्या॑णाम् मनु॒ष्या॑णा मू॒र्द्ध्वम् निखा॑तात् । \newline
45. ऊ॒र्द्ध्वम् निखा॑ता॒न् निखा॑ता दू॒र्द्ध्व मू॒र्द्ध्वम् निखा॑ता॒दा निखा॑ता दू॒र्द्ध्व मू॒र्द्ध्वम् निखा॑ता॒दा । \newline
46. निखा॑ता॒दा निखा॑ता॒न् निखा॑ता॒दा र॑श॒नाया॑ रश॒नाया॒ आ निखा॑ता॒न् निखा॑ता॒दा र॑श॒नायाः᳚ । \newline
47. निखा॑ता॒दिति॒ नि - खा॒ता॒त् । \newline
48. आ र॑श॒नाया॑ रश॒नाया॒ आ र॑श॒नाया॒ ओष॑धीना॒ मोष॑धीनाꣳ रश॒नाया॒ आ र॑श॒नाया॒ ओष॑धीनाम् । \newline
49. र॒श॒नाया॒ ओष॑धीना॒ मोष॑धीनाꣳ रश॒नाया॑ रश॒नाया॒ ओष॑धीनाꣳ रश॒ना र॑श॒ नौष॑धीनाꣳ रश॒नाया॑ रश॒नाया॒ ओष॑धीनाꣳ रश॒ना । \newline
50. ओष॑धीनाꣳ रश॒ना र॑श॒ नौष॑धीना॒ मोष॑धीनाꣳ रश॒ना विश्वे॑षां॒ विश्वे॑षाꣳ रश॒
नौष॑धीना॒ मोष॑धीनाꣳ रश॒ना विश्वे॑षां । \newline
51. र॒श॒ना विश्वे॑षां॒ विश्वे॑षाꣳ रश॒ना र॑श॒ना विश्वे॑षां दे॒वाना᳚म् दे॒वानां॒ ॅविश्वे॑षाꣳ रश॒ना र॑श॒ना विश्वे॑षां दे॒वाना᳚म् । \newline
52. विश्वे॑षां दे॒वाना᳚म् दे॒वानां॒ ॅविश्वे॑षां॒ विश्वे॑षां दे॒वाना॑ मू॒र्द्ध्व मू॒र्द्ध्वम् दे॒वानां॒ ॅविश्वे॑षां॒ विश्वे॑षां दे॒वाना॑ मू॒र्द्ध्वम् । \newline
\pagebreak
\markright{ TS 6.3.4.7  \hfill https://www.vedavms.in \hfill}

\section{ TS 6.3.4.7 }

\textbf{TS 6.3.4.7 } \newline
\textbf{Samhita Paata} \newline

दे॒वाना॑मू॒र्द्ध्वꣳ र॑श॒नाया॒ आ च॒षाला॒दिन्द्र॑स्य च॒षालꣳ॑ सा॒द्ध्याना॒मति॑रिक्तꣳ॒॒ स वा ए॒ष स॑र्वदेव॒त्यो॑ यद्यूपो॒ यद्यूपं॑ मि॒नोति॒ सर्वा॑ ए॒व दे॒वताः᳚ प्रीणाति य॒ज्ञेन॒ वै दे॒वाः सु॑व॒र्गं ॅलो॒कमा॑य॒न् ते॑ऽमन्यन्त मनु॒ष्या॑ नो॒ऽन्वाभ॑विष्य॒न्तीति॒ ते यूपे॑न योपयि॒त्वा सु॑व॒र्गं ॅलो॒कमा॑य॒न् तमृष॑यो॒ यूपे॑नै॒वानु॒ प्राजा॑न॒न् तद् यूप॑स्य यूप॒त्वं- [  ] \newline

\textbf{Pada Paata} \newline

दे॒वाना᳚म् । ऊ॒द्‌र्ध्वम् । र॒श॒नायाः᳚ । एति॑ । च॒षाला᳚त् । इन्द्र॑स्य । च॒षाल᳚म् । सा॒द्ध्याना᳚म् । अति॑रिक्त॒मित्यति॑ - रि॒क्त॒म् । सः । वै । ए॒षः । स॒र्व॒दे॒व॒त्य॑ इति॑ सर्व-दे॒व॒त्यः॑ । यत् । यूपः॑ । यत् । यूप᳚म् । मि॒नोति॑ । सर्वाः᳚ । ए॒व । दे॒वताः᳚ । प्री॒णा॒ति॒ । य॒ज्ञेन॑ । वै । दे॒वाः । सु॒व॒र्गमिति॑ सुवः-गम् । लो॒कम् । आ॒य॒न्न् । ते । अ॒म॒न्य॒न्त॒ । म॒नु॒ष्याः᳚ । नः॒ । अ॒न्वाभ॑विष्य॒न्तीत्य॑नु-आभ॑विष्यन्ति । इति॑ । ते । यूपे॑न । यो॒प॒यि॒त्वा । सु॒व॒र्गमिति॑ सुवः-गम् । लो॒कम् । आ॒य॒न्न् । तम् । ऋष॑यः । यूपे॑न । ए॒व । अनु॑ । प्रेति॑ । अ॒जा॒न॒न्न् । तत् । यूप॑स्य । यू॒प॒त्वमिति॑ यूप- त्वम् ।  \newline


\textbf{Krama Paata} \newline

दे॒वाना॑मू॒र्द्ध्वम् । ऊ॒र्द्ध्वꣳ र॑श॒नायाः᳚ । र॒श॒नाया॒ आ । आ च॒षाला᳚त् । च॒षाला॒दिन्द्र॑स्य । इन्द्र॑स्य च॒षाल᳚म् । च॒षालꣳ॑ सा॒द्ध्याना᳚म् । सा॒द्ध्याना॒मति॑रिक्तम् । अति॑रिक्तꣳ॒॒ सः । अति॑रिक्त॒मित्यति॑ - रि॒क्त॒म् । स वै । वा ए॒षः । ए॒ष स॑र्वदेव॒त्यः॑ । स॒र्व॒दे॒व॒त्यो॑ यत् । स॒र्व॒दे॒व॒त्य॑ इति॑ सर्व - दे॒व॒त्यः॑ । यद् यूपः॑ । यूपो॒ यत् । यद् यूप᳚म् । यूप॑म् मि॒नोति॑ । मि॒नोति॒ सर्वाः᳚ । सर्वा॑ ए॒व । ए॒व दे॒वताः᳚ । दे॒वताः᳚ प्रीणाति । प्री॒णा॒ति॒ य॒ज्ञेन॑ । य॒ज्ञेन॒ वै । वै दे॒वाः । दे॒वाः सु॑व॒र्गम् । सु॒व॒र्गम् ॅलो॒कम् । सु॒व॒र्गमिति॑ सुवः - गम् । लो॒कमा॑यन्न् । आ॒य॒न् ते । ते॑ऽमन्यन्त । अ॒म॒न्य॒न्त॒ म॒नु॒ष्याः᳚ । म॒नु॒ष्या॑ नः । नो॒ऽन्वाभ॑विष्यन्ति । अ॒न्वाभ॑विष्य॒न्तीति॑ । अ॒न्वाभ॑विष्य॒न्तीत्य॑नु - आभ॑विष्यन्ति । इति॒ ते । ते यूपे॑न । यूपे॑न योपयि॒त्वा । यो॒प॒यि॒त्वा सु॑व॒र्गम् । सु॒व॒र्गम् ॅलो॒कम् । सु॒व॒र्गमिति॑ सुवः - गम् । लो॒कमा॑यन्न् । आ॒य॒न् तम् । तमृष॑यः । ऋष॑यो॒ यूपे॑न । यूपे॑नै॒व । ए॒वानु॑ । अनु॒ प्र । प्राजा॑नन्न् । अ॒जा॒न॒न् तत् । तद् यूप॑स्य । यूप॑स्य यूप॒त्वम् । यू॒प॒त्वम् ॅयत् । यू॒प॒त्वमिति॑ यूप - त्वम् \newline

\textbf{Jatai Paata} \newline

1. दे॒वाना॑ मू॒र्द्ध्व मू॒र्द्ध्वम् दे॒वाना᳚म् दे॒वाना॑ मू॒र्द्ध्वम् । \newline
2. ऊ॒र्द्ध्वꣳ र॑श॒नाया॑ रश॒नाया॑ ऊ॒र्द्ध्व मू॒र्द्ध्वꣳ र॑श॒नायाः᳚ । \newline
3. र॒श॒नाया॒ आ र॑श॒नाया॑ रश॒नाया॒ आ । \newline
4. आ च॒षाला᳚च् च॒षाला॒दा च॒षाला᳚त् । \newline
5. च॒षाला॒ दिन्द्र॒ स्येन्द्र॑स्य च॒षाला᳚च् च॒षाला॒ दिन्द्र॑स्य । \newline
6. इन्द्र॑स्य च॒षाल॑म् च॒षाल॒ मिन्द्र॒ स्येन्द्र॑स्य च॒षाल᳚म् । \newline
7. च॒षालꣳ॑ सा॒द्ध्यानाꣳ॑ सा॒द्ध्याना᳚म् च॒षाल॑म् च॒षालꣳ॑ सा॒द्ध्याना᳚म् । \newline
8. सा॒द्ध्याना॒ मति॑रिक्त॒ मति॑रिक्तꣳ सा॒द्ध्यानाꣳ॑ सा॒द्ध्याना॒ मति॑रिक्तम् । \newline
9. अति॑रिक्तꣳ॒॒ स सो ऽति॑रिक्त॒ मति॑रिक्तꣳ॒॒ सः । \newline
10. अति॑रिक्त॒मित्यति॑ - रि॒क्त॒म् । \newline
11. स वै वै स स वै । \newline
12. वा ए॒ष ए॒ष वै वा ए॒षः । \newline
13. ए॒ष स॑र्वदेव॒त्यः॑ सर्वदेव॒त्य॑ ए॒ष ए॒ष स॑र्वदेव॒त्यः॑ । \newline
14. स॒र्व॒दे॒व॒त्यो॑ यद् यथ् स॑र्वदेव॒त्यः॑ सर्वदेव॒त्यो॑ यत् । \newline
15. स॒र्व॒दे॒व॒त्य॑ इति॑ सर्व - दे॒व॒त्यः॑ । \newline
16. यद् यूपो॒ यूपो॒ यद् यद् यूपः॑ । \newline
17. यूपो॒ यद् यद् यूपो॒ यूपो॒ यत् । \newline
18. यद् यूपं॒ ॅयूपं॒ ॅयद् यद् यूप᳚म् । \newline
19. यूप॑म् मि॒नोति॑ मि॒नोति॒ यूपं॒ ॅयूप॑म् मि॒नोति॑ । \newline
20. मि॒नोति॒ सर्वाः॒ सर्वा॑ मि॒नोति॑ मि॒नोति॒ सर्वाः᳚ । \newline
21. सर्वा॑ ए॒वैव सर्वाः॒ सर्वा॑ ए॒व । \newline
22. ए॒व दे॒वता॑ दे॒वता॑ ए॒वैव दे॒वताः᳚ । \newline
23. दे॒वताः᳚ प्रीणाति प्रीणाति दे॒वता॑ दे॒वताः᳚ प्रीणाति । \newline
24. प्री॒णा॒ति॒ य॒ज्ञेन॑ य॒ज्ञेन॑ प्रीणाति प्रीणाति य॒ज्ञेन॑ । \newline
25. य॒ज्ञेन॒ वै वै य॒ज्ञेन॑ य॒ज्ञेन॒ वै । \newline
26. वै दे॒वा दे॒वा वै वै दे॒वाः । \newline
27. दे॒वाः सु॑व॒र्गꣳ सु॑व॒र्गम् दे॒वा दे॒वाः सु॑व॒र्गम् । \newline
28. सु॒व॒र्गम् ॅलो॒कम् ॅलो॒कꣳ सु॑व॒र्गꣳ सु॑व॒र्गम् ॅलो॒कम् । \newline
29. सु॒व॒र्गमिति॑ सुवः - गम् । \newline
30. लो॒क मा॑यन् नायन् ॅलो॒कम् ॅलो॒क मा॑यन्न् । \newline
31. आ॒य॒न् ते त आ॑यन् नाय॒न् ते । \newline
32. ते॑ ऽमन्यन्ता मन्यन्त॒ ते ते॑ ऽमन्यन्त । \newline
33. अ॒म॒न्य॒न्त॒ म॒नु॒ष्या॑ मनु॒ष्या॑ अमन्यन्ता मन्यन्त मनु॒ष्याः᳚ । \newline
34. म॒नु॒ष्या॑ नो नो मनु॒ष्या॑ मनु॒ष्या॑ नः । \newline
35. नो॒ ऽन्वाभ॑विष्य न्त्य॒न्वाभ॑विष्यन्ति नो नो॒ ऽन्वाभ॑विष्यन्ति । \newline
36. अ॒न्वाभ॑विष्य॒न्तीती त्य॒न्वाभ॑विष्य न्त्य॒न्वाभ॑विष्य॒न्तीति॑ । \newline
37. अ॒न्वाभ॑विष्य॒न्तीत्य॑नु - आभ॑विष्यन्ति । \newline
38. इति॒ ते त इतीति॒ ते । \newline
39. ते यूपे॑न॒ यूपे॑न॒ ते ते यूपे॑न । \newline
40. यूपे॑न योपयि॒त्वा यो॑पयि॒त्वा यूपे॑न॒ यूपे॑न योपयि॒त्वा । \newline
41. यो॒प॒यि॒त्वा सु॑व॒र्गꣳ सु॑व॒र्गं ॅयो॑पयि॒त्वा यो॑पयि॒त्वा सु॑व॒र्गम् । \newline
42. सु॒व॒र्गम् ॅलो॒कम् ॅलो॒कꣳ सु॑व॒र्गꣳ सु॑व॒र्गम् ॅलो॒कम् । \newline
43. सु॒व॒र्गमिति॑ सुवः - गम् । \newline
44. लो॒क मा॑यन् नायन् ॅलो॒कम् ॅलो॒क मा॑यन्न् । \newline
45. आ॒य॒न् तम् त मा॑यन् नाय॒न् तम् । \newline
46. त मृष॑य॒ ऋष॑य॒ स्तम् त मृष॑यः । \newline
47. ऋष॑यो॒ यूपे॑न॒ यूपे॒न र्.ष॑य॒ ऋष॑यो॒ यूपे॑न । \newline
48. यूपे॑नै॒ वैव यूपे॑न॒ यूपे॑नै॒व । \newline
49. ए॒वान् वन् वे॒वै वानु॑ । \newline
50. अनु॒ प्र प्राण्वनु॒ प्र । \newline
51. प्राजा॑नन् नजान॒न् प्र प्राजा॑नन्न् । \newline
52. अ॒जा॒न॒न् तत् तद॑जानन् नजान॒न् तत् । \newline
53. तद् यूप॑स्य॒ यूप॑स्य॒ तत् तद् यूप॑स्य । \newline
54. यूप॑स्य यूप॒त्वं ॅयू॑प॒त्वं ॅयूप॑स्य॒ यूप॑स्य यूप॒त्वम् । \newline
55. यू॒प॒त्वं ॅयद् यद् यू॑प॒त्वं ॅयू॑प॒त्वं ॅयत् । \newline
56. यू॒प॒त्वमिति॑ यूप - त्वम् । \newline

\textbf{Ghana Paata } \newline

1. दे॒वाना॑ मू॒र्द्ध्व मू॒र्द्ध्वम् दे॒वाना᳚म् दे॒वाना॑ मू॒र्द्ध्वꣳ र॑श॒नाया॑ रश॒नाया॑ ऊ॒र्द्ध्वम् दे॒वाना᳚म् दे॒वाना॑ मू॒र्द्ध्वꣳ र॑श॒नायाः᳚ । \newline
2. ऊ॒र्द्ध्वꣳ र॑श॒नाया॑ रश॒नाया॑ ऊ॒र्द्ध्व मू॒र्द्ध्वꣳ र॑श॒नाया॒ आ र॑श॒नाया॑ ऊ॒र्द्ध्व मू॒र्द्ध्वꣳ र॑श॒नाया॒ आ । \newline
3. र॒श॒नाया॒ आ र॑श॒नाया॑ रश॒नाया॒ आ च॒षाला᳚च् च॒षाला॒दा र॑श॒नाया॑ रश॒नाया॒ आ च॒षाला᳚त् । \newline
4. आ च॒षाला᳚च् च॒षाला॒दा च॒षाला॒ दिन्द्र॒ स्येन्द्र॑स्य च॒षाला॒दा च॒षाला॒ दिन्द्र॑स्य । \newline
5. च॒षाला॒ दिन्द्र॒ स्येन्द्र॑स्य च॒षाला᳚च् च॒षाला॒ दिन्द्र॑स्य च॒षाल॑म् च॒षाल॒ मिन्द्र॑स्य च॒षाला᳚च् च॒षाला॒ दिन्द्र॑स्य च॒षाल᳚म् । \newline
6. इन्द्र॑स्य च॒षाल॑म् च॒षाल॒ मिन्द्र॒ स्येन्द्र॑स्य च॒षालꣳ॑ सा॒द्ध्यानाꣳ॑ सा॒द्ध्याना᳚म् च॒षाल॒ मिन्द्र॒ स्येन्द्र॑स्य च॒षालꣳ॑ सा॒द्ध्याना᳚म् । \newline
7. च॒षालꣳ॑ सा॒द्ध्यानाꣳ॑ सा॒द्ध्याना᳚म् च॒षाल॑म् च॒षालꣳ॑ सा॒द्ध्याना॒ मति॑रिक्त॒ मति॑रिक्तꣳ सा॒द्ध्याना᳚म् च॒षाल॑म् च॒षालꣳ॑ सा॒द्ध्याना॒ मति॑रिक्तम् । \newline
8. सा॒द्ध्याना॒ मति॑रिक्त॒ मति॑रिक्तꣳ सा॒द्ध्यानाꣳ॑ सा॒द्ध्याना॒ मति॑रिक्तꣳ॒॒ स सो ऽति॑रिक्तꣳ सा॒द्ध्यानाꣳ॑ सा॒द्ध्याना॒ मति॑रिक्तꣳ॒॒ सः । \newline
9. अति॑रिक्तꣳ॒॒ स सो ऽति॑रिक्त॒ मति॑रिक्तꣳ॒॒ स वै वै सो ऽति॑रिक्त॒ मति॑रिक्तꣳ॒॒ स वै । \newline
10. अति॑रिक्त॒मित्यति॑ - रि॒क्त॒म् । \newline
11. स वै वै स स वा ए॒ष ए॒ष वै स स वा ए॒षः । \newline
12. वा ए॒ष ए॒ष वै वा ए॒ष स॑र्वदेव॒त्यः॑ सर्वदेव॒त्य॑ ए॒ष वै वा ए॒ष स॑र्वदेव॒त्यः॑ । \newline
13. ए॒ष स॑र्वदेव॒त्यः॑ सर्वदेव॒त्य॑ ए॒ष ए॒ष स॑र्वदेव॒त्यो॑ यद् यथ् स॑र्वदेव॒त्य॑ ए॒ष ए॒ष स॑र्वदेव॒त्यो॑ यत् । \newline
14. स॒र्व॒दे॒व॒त्यो॑ यद् यथ् स॑र्वदेव॒त्यः॑ सर्वदेव॒त्यो॑ यद् यूपो॒ यूपो॒ यथ् स॑र्वदेव॒त्यः॑ सर्वदेव॒त्यो॑ यद् यूपः॑ । \newline
15. स॒र्व॒दे॒व॒त्य॑ इति॑ सर्व - दे॒व॒त्यः॑ । \newline
16. यद् यूपो॒ यूपो॒ यद् यद् यूपो॒ यद् यद् यूपो॒ यद् यद् यूपो॒ यत् । \newline
17. यूपो॒ यद् यद् यूपो॒ यूपो॒ यद् यूपं॒ ॅयूपं॒ ॅयद् यूपो॒ यूपो॒ यद् यूप᳚म् । \newline
18. यद् यूपं॒ ॅयूपं॒ ॅयद् यद् यूप॑म् मि॒नोति॑ मि॒नोति॒ यूपं॒ ॅयद् यद् यूप॑म् मि॒नोति॑ । \newline
19. यूप॑म् मि॒नोति॑ मि॒नोति॒ यूपं॒ ॅयूप॑म् मि॒नोति॒ सर्वाः॒ सर्वा॑ मि॒नोति॒ यूपं॒ ॅयूप॑म् मि॒नोति॒ सर्वाः᳚ । \newline
20. मि॒नोति॒ सर्वाः॒ सर्वा॑ मि॒नोति॑ मि॒नोति॒ सर्वा॑ ए॒वैव सर्वा॑ मि॒नोति॑ मि॒नोति॒ सर्वा॑ ए॒व । \newline
21. सर्वा॑ ए॒वैव सर्वाः॒ सर्वा॑ ए॒व दे॒वता॑ दे॒वता॑ ए॒व सर्वाः॒ सर्वा॑ ए॒व दे॒वताः᳚ । \newline
22. ए॒व दे॒वता॑ दे॒वता॑ ए॒वैव दे॒वताः᳚ प्रीणाति प्रीणाति दे॒वता॑ ए॒वैव दे॒वताः᳚ प्रीणाति । \newline
23. दे॒वताः᳚ प्रीणाति प्रीणाति दे॒वता॑ दे॒वताः᳚ प्रीणाति य॒ज्ञेन॑ य॒ज्ञेन॑ प्रीणाति दे॒वता॑ दे॒वताः᳚ प्रीणाति य॒ज्ञेन॑ । \newline
24. प्री॒णा॒ति॒ य॒ज्ञेन॑ य॒ज्ञेन॑ प्रीणाति प्रीणाति य॒ज्ञेन॒ वै वै य॒ज्ञेन॑ प्रीणाति प्रीणाति य॒ज्ञेन॒ वै । \newline
25. य॒ज्ञेन॒ वै वै य॒ज्ञेन॑ य॒ज्ञेन॒ वै दे॒वा दे॒वा वै य॒ज्ञेन॑ य॒ज्ञेन॒ वै दे॒वाः । \newline
26. वै दे॒वा दे॒वा वै वै दे॒वाः सु॑व॒र्गꣳ सु॑व॒र्गम् दे॒वा वै वै दे॒वाः सु॑व॒र्गम् । \newline
27. दे॒वाः सु॑व॒र्गꣳ सु॑व॒र्गम् दे॒वा दे॒वाः सु॑व॒र्गम् ॅलो॒कम् ॅलो॒कꣳ सु॑व॒र्गम् दे॒वा दे॒वाः सु॑व॒र्गम् ॅलो॒कम् । \newline
28. सु॒व॒र्गम् ॅलो॒कम् ॅलो॒कꣳ सु॑व॒र्गꣳ सु॑व॒र्गम् ॅलो॒क मा॑यन् नायन् ॅलो॒कꣳ सु॑व॒र्गꣳ सु॑व॒र्गम् ॅलो॒क मा॑यन्न् । \newline
29. सु॒व॒र्गमिति॑ सुवः - गम् । \newline
30. लो॒क मा॑यन् नायन् ॅलो॒कम् ॅलो॒क मा॑य॒न् ते त आ॑यन् ॅलो॒कम् ॅलो॒क मा॑य॒न् ते । \newline
31. आ॒य॒न् ते त आ॑यन् नाय॒न् ते॑ ऽमन्यन्ता मन्यन्त॒ त आ॑यन् नाय॒न् ते॑ ऽमन्यन्त । \newline
32. ते॑ ऽमन्यन्ता मन्यन्त॒ ते ते॑ ऽमन्यन्त मनु॒ष्या॑ मनु॒ष्या॑ अमन्यन्त॒ ते ते॑ ऽमन्यन्त मनु॒ष्याः᳚ । \newline
33. अ॒म॒न्य॒न्त॒ म॒नु॒ष्या॑ मनु॒ष्या॑ अमन्यन्ता मन्यन्त मनु॒ष्या॑ नो नो मनु॒ष्या॑ अमन्यन्ता मन्यन्त मनु॒ष्या॑ नः । \newline
34. म॒नु॒ष्या॑ नो नो मनु॒ष्या॑ मनु॒ष्या॑ नो॒ ऽन्वाभ॑विष्य न्त्य॒न्वाभ॑विष्यन्ति नो मनु॒ष्या॑ मनु॒ष्या॑ नो॒ ऽन्वाभ॑विष्यन्ति । \newline
35. नो॒ ऽन्वाभ॑विष्य न्त्य॒न्वाभ॑विष्यन्ति नो नो॒ ऽन्वाभ॑विष्य॒न्तीती त्य॒न्वाभ॑विष्यन्ति नो नो॒ ऽन्वाभ॑विष्य॒न्तीति॑ । \newline
36. अ॒न्वाभ॑विष्य॒न्तीती त्य॒न्वाभ॑विष्य न्त्य॒न्वाभ॑विष्य॒न्तीति॒ ते त इत्य॒न्वाभ॑विष्य न्त्य॒न्वाभ॑विष्य॒न्तीति॒ ते । \newline
37. अ॒न्वाभ॑विष्य॒न्तीत्य॑नु - आभ॑विष्यन्ति । \newline
38. इति॒ ते त इतीति॒ ते यूपे॑न॒ यूपे॑न॒ त इतीति॒ ते यूपे॑न । \newline
39. ते यूपे॑न॒ यूपे॑न॒ ते ते यूपे॑न योपयि॒त्वा यो॑पयि॒त्वा यूपे॑न॒ ते ते यूपे॑न योपयि॒त्वा । \newline
40. यूपे॑न योपयि॒त्वा यो॑पयि॒त्वा यूपे॑न॒ यूपे॑न योपयि॒त्वा सु॑व॒र्गꣳ सु॑व॒र्गं ॅयो॑पयि॒त्वा यूपे॑न॒ यूपे॑न योपयि॒त्वा सु॑व॒र्गम् । \newline
41. यो॒प॒यि॒त्वा सु॑व॒र्गꣳ सु॑व॒र्गं ॅयो॑पयि॒त्वा यो॑पयि॒त्वा सु॑व॒र्गम् ॅलो॒कम् ॅलो॒कꣳ सु॑व॒र्गं ॅयो॑पयि॒त्वा यो॑पयि॒त्वा सु॑व॒र्गम् ॅलो॒कम् । \newline
42. सु॒व॒र्गम् ॅलो॒कम् ॅलो॒कꣳ सु॑व॒र्गꣳ सु॑व॒र्गम् ॅलो॒क मा॑यन् नायन् ॅलो॒कꣳ सु॑व॒र्गꣳ सु॑व॒र्गम् ॅलो॒क मा॑यन्न् । \newline
43. सु॒व॒र्गमिति॑ सुवः - गम् । \newline
44. लो॒क मा॑यन् नायन् ॅलो॒कम् ॅलो॒क मा॑य॒न् तम् त मा॑यन् ॅलो॒कम् ॅलो॒क मा॑य॒न् तम् । \newline
45. आ॒य॒न् तम् त मा॑यन् नाय॒न् त मृष॑य॒ ऋष॑य॒ स्त मा॑यन् नाय॒न् त मृष॑यः । \newline
46. त मृष॑य॒ ऋष॑य॒ स्तम् त मृष॑यो॒ यूपे॑न॒ यूपे॒न र्.ष॑य॒ स्तम् त मृष॑यो॒ यूपे॑न । \newline
47. ऋष॑यो॒ यूपे॑न॒ यूपे॒न र्.ष॑य॒ ऋष॑यो॒ यूपे॑नै॒ वैव यूपे॒न र्.ष॑य॒ ऋष॑यो॒ यूपे॑नै॒व । \newline
48. यूपे॑नै॒वैव यूपे॑न॒ यूपे॑नै॒ वान् वन् वे॒व यूपे॑न॒ यूपे॑नै॒ वानु॑ । \newline
49. ए॒वान् वन् वे॒वै वानु॒ प्र प्राण्वे॒वै वानु॒ प्र । \newline
50. अनु॒ प्र प्राण्वनु॒ प्राजा॑नन् नजान॒न् प्राण्वनु॒ प्राजा॑नन्न् । \newline
51. प्राजा॑नन् नजान॒न् प्र प्राजा॑न॒न् तत् तद॑जान॒न् प्र प्राजा॑न॒न् तत् । \newline
52. अ॒जा॒न॒न् तत् तद॑जानन् नजान॒न् तद् यूप॑स्य॒ यूप॑स्य॒ तद॑जानन् नजान॒न् तद् यूप॑स्य । \newline
53. तद् यूप॑स्य॒ यूप॑स्य॒ तत् तद् यूप॑स्य यूप॒त्वं ॅयू॑प॒त्वं ॅयूप॑स्य॒ तत् तद् यूप॑स्य यूप॒त्वम् । \newline
54. यूप॑स्य यूप॒त्वं ॅयू॑प॒त्वं ॅयूप॑स्य॒ यूप॑स्य यूप॒त्वं ॅयद् यद् यू॑प॒त्वं ॅयूप॑स्य॒ यूप॑स्य यूप॒त्वं ॅयत् । \newline
55. यू॒प॒त्वं ॅयद् यद् यू॑प॒त्वं ॅयू॑प॒त्वं ॅयद् यूपं॒ ॅयूपं॒ ॅयद् यू॑प॒त्वं ॅयू॑प॒त्वं ॅयद् यूप᳚म् । \newline
56. यू॒प॒त्वमिति॑ यूप - त्वम् । \newline
\pagebreak
\markright{ TS 6.3.4.8  \hfill https://www.vedavms.in \hfill}

\section{ TS 6.3.4.8 }

\textbf{TS 6.3.4.8 } \newline
\textbf{Samhita Paata} \newline

ॅयद् यूपं॑ मि॒नोति॑ सुव॒र्गस्य॑ लो॒कस्य॒ प्रज्ञा᳚त्यै पु॒रस्ता᳚न् मिनोति पु॒रस्ता॒द्धि य॒ज्ञ्स्य॑ प्रज्ञा॒यते प्र॑ज्ञातꣳ॒॒ हि तद् यदति॑पन्न आ॒हुरि॒दं का॒र्य॑मासी॒दिति॑ सा॒द्ध्या वै दे॒वा य॒ज्ञ्मत्य॑मन्यन्त॒ तान्. य॒ज्ञो नास्पृ॑श॒त् तान्. यद् य॒ज्ञ्स्याति॑रिक्त॒मासी॒त् तद॑स्पृश॒दति॑रिक्तं॒ ॅवा ए॒तद् य॒ज्ञ्स्य॒ यद॒ग्नाव॒ग्निं म॑थि॒त्वा प्र॒हर॒त्यति॑रिक्तमे॒त- [  ] \newline

\textbf{Pada Paata} \newline

यत् । यूप᳚म् । मि॒नोति॑ । सु॒व॒र्गस्येति॑ सुवः - गस्य॑ । लो॒कस्य॑ । प्रज्ञा᳚त्या॒ इति॒ प्र - ज्ञा॒त्यै॒ । पु॒रस्ता᳚त् । मि॒नो॒ति॒ । पु॒रस्ता᳚त् । हि । य॒ज्ञ्स्य॑ । प्र॒ज्ञा॒यत॒ इति॑ प्र - ज्ञा॒यते᳚ । अप्र॑ज्ञात॒मित्यप्र॑ - ज्ञा॒त॒म् । हि । तत् । यत् । अति॑पन्न॒ इत्यति॑ - प॒न्ने॒ । आ॒हुः । इ॒दम् । का॒र्य᳚म् । आ॒सी॒त् । इति॑ । सा॒द्ध्याः । वै । दे॒वाः । य॒ज्ञ्म् । अतीति॑ । अ॒म॒न्य॒न्त॒ । तान् । य॒ज्ञ्ः । न । अ॒स्पृ॒श॒त् । तान् । यत् । य॒ज्ञ्स्य॑ । अति॑रिक्त॒मित्यति॑ - रि॒क्त॒म् । आसी᳚त् । तत् । अ॒स्पृ॒श॒त् । अति॑रिक्त॒मित्यति॑ - रि॒क्त॒म् । वै । ए॒तत् । य॒ज्ञ्स्य॑ । यत् । अ॒ग्नौ । अ॒ग्निम् । म॒थि॒त्वा । प्र॒हर॒तीति॑ प्र - हर॑ति । अति॑रिक्त॒मित्यति॑ - रि॒क्त॒म् । ए॒तत् ।  \newline


\textbf{Krama Paata} \newline

यद् यूप᳚म् । यूप॑म् मि॒नोति॑ । मि॒नोति॑ सुव॒र्गस्य॑ । सु॒व॒र्गस्य॑ लो॒कस्य॑ । सु॒व॒र्गस्येति॑ सुवः - गस्य॑ । लो॒कस्य॒ प्रज्ञा᳚त्यै । प्रज्ञा᳚त्यै पु॒रस्ता᳚त् । प्रज्ञा᳚त्या॒ इति॒ प्र - ज्ञा॒त्यै॒ । पु॒रस्ता᳚त् मिनोति । मि॒नो॒ति॒ पु॒रस्ता᳚त् । पु॒रस्ता॒द्‌धि । हि य॒ज्ञ्स्य॑ । य॒ज्ञ्स्य॑ प्रज्ञा॒यते᳚ । प्र॒ज्ञा॒यते ऽप्र॑ज्ञातम् । प्र॒ज्ञा॒यत॒ इति॑ प्र - ज्ञा॒यते᳚ । अप्र॑ज्ञातꣳ॒॒ हि । अप्र॑ज्ञात॒मित्यप्र॑ - ज्ञा॒त॒म् । हि तत् । तद् यत् । यदति॑पन्ने । अति॑पन्न आ॒हुः । अति॑पन्न॒ इत्यति॑ - प॒न्ने॒ । आ॒हुरि॒दम् । इ॒दम् का॒र्य᳚म् । का॒र्य॑मासीत् । आ॒सी॒दिति॑ । इति॑ सा॒द्ध्याः । सा॒द्ध्या वै । वै दे॒वाः । दे॒वा य॒ज्ञ्म् । य॒ज्ञ्मति॑ । अत्य॑मन्यन्त । अ॒म॒न्य॒न्त॒ तान् । तान्. य॒ज्ञ्ः । य॒ज्ञो न । नास्पृ॑शत् । अ॒स्पृ॒श॒त् तान् । तान्. यत् । यद् य॒ज्ञ्स्य॑ । य॒ज्ञ्स्याति॑रिक्तम् । अति॑रिक्त॒मासी᳚त् । अति॑रिक्त॒मित्यति॑ - रि॒क्त॒म् । आसी॒त् तत् । तद॑स्पृशत् । अ॒स्पृ॒श॒दति॑रिक्तम् । अति॑रिक्त॒म् ॅवै । अति॑रिक्त॒मित्यति॑ - रि॒क्त॒म् । वा ए॒तत् । ए॒तद् य॒ज्ञ्स्य॑ । य॒ज्ञ्स्य॒ यत् । यद॒ग्नौ । अ॒ग्नाव॒ग्निम् । अ॒ग्निम् म॑थि॒त्वा । म॒थि॒त्वा प्र॒हर॑ति । प्र॒हर॒त्यति॑रिक्तम् । प्र॒हर॒तीति॑ प्र - हर॑ति । अति॑रिक्तमे॒तत् ( ) । अति॑रिक्त॒मित्यति॑ - रि॒क्त॒म् । ए॒तद् यूप॑स्य \newline

\textbf{Jatai Paata} \newline

1. यद् यूपं॒ ॅयूपं॒ ॅयद् यद् यूप᳚म् । \newline
2. यूप॑म् मि॒नोति॑ मि॒नोति॒ यूपं॒ ॅयूप॑म् मि॒नोति॑ । \newline
3. मि॒नोति॑ सुव॒र्गस्य॑ सुव॒र्गस्य॑ मि॒नोति॑ मि॒नोति॑ सुव॒र्गस्य॑ । \newline
4. सु॒व॒र्गस्य॑ लो॒कस्य॑ लो॒कस्य॑ सुव॒र्गस्य॑ सुव॒र्गस्य॑ लो॒कस्य॑ । \newline
5. सु॒व॒र्गस्येति॑ सुवः - गस्य॑ । \newline
6. लो॒कस्य॒ प्रज्ञा᳚त्यै॒ प्रज्ञा᳚त्यै लो॒कस्य॑ लो॒कस्य॒ प्रज्ञा᳚त्यै । \newline
7. प्रज्ञा᳚त्यै पु॒रस्ता᳚त् पु॒रस्ता॒त् प्रज्ञा᳚त्यै॒ प्रज्ञा᳚त्यै पु॒रस्ता᳚त् । \newline
8. प्रज्ञा᳚त्या॒ इति॒ प्र - ज्ञा॒त्यै॒ । \newline
9. पु॒रस्ता᳚न् मिनोति मिनोति पु॒रस्ता᳚त् पु॒रस्ता᳚न् मिनोति । \newline
10. मि॒नो॒ति॒ पु॒रस्ता᳚त् पु॒रस्ता᳚न् मिनोति मिनोति पु॒रस्ता᳚त् । \newline
11. पु॒रस्ता॒द्धि हि पु॒रस्ता᳚त् पु॒रस्ता॒द्धि । \newline
12. हि य॒ज्ञ्स्य॑ य॒ज्ञ्स्य॒ हि हि य॒ज्ञ्स्य॑ । \newline
13. य॒ज्ञ्स्य॑ प्रज्ञा॒यते᳚ प्रज्ञा॒यते॑ य॒ज्ञ्स्य॑ य॒ज्ञ्स्य॑ प्रज्ञा॒यते᳚ । \newline
14. प्र॒ज्ञा॒यते ऽप्र॑ज्ञात॒ मप्र॑ज्ञातम् प्रज्ञा॒यते᳚ प्रज्ञा॒यते ऽप्र॑ज्ञातम् । \newline
15. प्र॒ज्ञा॒यत॒ इति॑ प्र - ज्ञा॒यते᳚ । \newline
16. अप्र॑ज्ञातꣳ॒॒ हि ह्यप्र॑ज्ञात॒ मप्र॑ज्ञातꣳ॒॒ हि । \newline
17. अप्र॑ज्ञात॒मित्यप्र॑ - ज्ञा॒त॒म् । \newline
18. हि तत् तद्धि हि तत् । \newline
19. तद् यद् यत् तत् तद् यत् । \newline
20. यदति॑प॒न्ने ऽति॑पन्ने॒ यद् यदति॑पन्ने । \newline
21. अति॑पन्न आ॒हु रा॒हु रति॑प॒न्ने ऽति॑पन्न आ॒हुः । \newline
22. अति॑पन्न॒ इत्यति॑ - प॒न्ने॒ । \newline
23. आ॒हु रि॒द मि॒द मा॒हु रा॒हु रि॒दम् । \newline
24. इ॒दम् का॒र्य॑म् का॒र्य॑ मि॒द मि॒दम् का॒र्य᳚म् । \newline
25. का॒र्य॑ मासी दासीत् का॒र्य॑म् का॒र्य॑ मासीत् । \newline
26. आ॒सी॒दि तीत्या॑सी दासी॒ दिति॑ । \newline
27. इति॑ सा॒द्ध्याः सा॒द्ध्या इतीति॑ सा॒द्ध्याः । \newline
28. सा॒द्ध्या वै वै सा॒द्ध्याः सा॒द्ध्या वै । \newline
29. वै दे॒वा दे॒वा वै वै दे॒वाः । \newline
30. दे॒वा य॒ज्ञ्ं ॅय॒ज्ञ्म् दे॒वा दे॒वा य॒ज्ञ्म् । \newline
31. य॒ज्ञ् मत्यति॑ य॒ज्ञ्ं ॅय॒ज्ञ् मति॑ । \newline
32. अत्य॑ मन्यन्ता मन्य॒न्ता त्यत्य॑ मन्यन्त । \newline
33. अ॒म॒न्य॒न्त॒ ताꣳ स्ता न॑मन्यन्ता मन्यन्त॒ तान् । \newline
34. तान्. य॒ज्ञो य॒ज्ञ् स्ताꣳ स्तान्. य॒ज्ञ्ः । \newline
35. य॒ज्ञो न न य॒ज्ञो य॒ज्ञो न । \newline
36. नास्पृ॑श दस्पृश॒न् न नास्पृ॑शत् । \newline
37. अ॒स्पृ॒श॒त् ताꣳ स्ता न॑स्पृश दस्पृश॒त् तान् । \newline
38. तान्. यद् यत् ताꣳ स्तान्. यत् । \newline
39. यद् य॒ज्ञ्स्य॑ य॒ज्ञ्स्य॒ यद् यद् य॒ज्ञ्स्य॑ । \newline
40. य॒ज्ञ्स्या ति॑रिक्त॒ मति॑रिक्तं ॅय॒ज्ञ्स्य॑ य॒ज्ञ्स्या ति॑रिक्तम् । \newline
41. अति॑रिक्त॒ मासी॒ दासी॒ दति॑रिक्त॒ मति॑रिक्त॒ मासी᳚त् । \newline
42. अति॑रिक्त॒मित्यति॑ - रि॒क्त॒म् । \newline
43. आसी॒त् तत् तदासी॒ दासी॒त् तत् । \newline
44. तद॑स्पृश दस्पृश॒त् तत् तद॑स्पृशत् । \newline
45. अ॒स्पृ॒श॒ दति॑रिक्त॒ मति॑रिक्त मस्पृश दस्पृश॒ दति॑रिक्तम् । \newline
46. अति॑रिक्तं॒ ॅवै वा अति॑रिक्त॒ मति॑रिक्तं॒ ॅवै । \newline
47. अति॑रिक्त॒मित्यति॑ - रि॒क्त॒म् । \newline
48. वा ए॒त दे॒तद् वै वा ए॒तत् । \newline
49. ए॒तद् य॒ज्ञ्स्य॑ य॒ज्ञ्स्यै॒त दे॒तद् य॒ज्ञ्स्य॑ । \newline
50. य॒ज्ञ्स्य॒ यद् यद् य॒ज्ञ्स्य॑ य॒ज्ञ्स्य॒ यत् । \newline
51. यद॒ग्ना व॒ग्नौ यद् यद॒ग्नौ । \newline
52. अ॒ग्ना व॒ग्नि म॒ग्नि म॒ग्ना व॒ग्ना व॒ग्निम् । \newline
53. अ॒ग्निम् म॑थि॒त्वा म॑थि॒त्वा ऽग्नि म॒ग्निम् म॑थि॒त्वा । \newline
54. म॒थि॒त्वा प्र॒हर॑ति प्र॒हर॑ति मथि॒त्वा म॑थि॒त्वा प्र॒हर॑ति । \newline
55. प्र॒हर॒ त्यति॑रिक्त॒ मति॑रिक्तम् प्र॒हर॑ति प्र॒हर॒ त्यति॑रिक्तम् । \newline
56. प्र॒हर॒तीति॑ प्र - हर॑ति । \newline
57. अति॑रिक्त मे॒त दे॒त दति॑रिक्त॒ मति॑रिक्त मे॒तत् । \newline
58. अति॑रिक्त॒मित्यति॑ - रि॒क्त॒म् । \newline
59. ए॒तद् यूप॑स्य॒ यूप॑ स्यै॒त दे॒तद् यूप॑स्य । \newline

\textbf{Ghana Paata } \newline

1. यद् यूपं॒ ॅयूपं॒ ॅयद् यद् यूप॑म् मि॒नोति॑ मि॒नोति॒ यूपं॒ ॅयद् यद् यूप॑म् मि॒नोति॑ । \newline
2. यूप॑म् मि॒नोति॑ मि॒नोति॒ यूपं॒ ॅयूप॑म् मि॒नोति॑ सुव॒र्गस्य॑ सुव॒र्गस्य॑ मि॒नोति॒ यूपं॒ ॅयूप॑म् मि॒नोति॑ सुव॒र्गस्य॑ । \newline
3. मि॒नोति॑ सुव॒र्गस्य॑ सुव॒र्गस्य॑ मि॒नोति॑ मि॒नोति॑ सुव॒र्गस्य॑ लो॒कस्य॑ लो॒कस्य॑ सुव॒र्गस्य॑ मि॒नोति॑ मि॒नोति॑ सुव॒र्गस्य॑ लो॒कस्य॑ । \newline
4. सु॒व॒र्गस्य॑ लो॒कस्य॑ लो॒कस्य॑ सुव॒र्गस्य॑ सुव॒र्गस्य॑ लो॒कस्य॒ प्रज्ञा᳚त्यै॒ प्रज्ञा᳚त्यै लो॒कस्य॑ सुव॒र्गस्य॑ सुव॒र्गस्य॑ लो॒कस्य॒ प्रज्ञा᳚त्यै । \newline
5. सु॒व॒र्गस्येति॑ सुवः - गस्य॑ । \newline
6. लो॒कस्य॒ प्रज्ञा᳚त्यै॒ प्रज्ञा᳚त्यै लो॒कस्य॑ लो॒कस्य॒ प्रज्ञा᳚त्यै पु॒रस्ता᳚त् पु॒रस्ता॒त् प्रज्ञा᳚त्यै लो॒कस्य॑ लो॒कस्य॒ प्रज्ञा᳚त्यै पु॒रस्ता᳚त् । \newline
7. प्रज्ञा᳚त्यै पु॒रस्ता᳚त् पु॒रस्ता॒त् प्रज्ञा᳚त्यै॒ प्रज्ञा᳚त्यै पु॒रस्ता᳚न् मिनोति मिनोति पु॒रस्ता॒त् प्रज्ञा᳚त्यै॒ प्रज्ञा᳚त्यै पु॒रस्ता᳚न् मिनोति । \newline
8. प्रज्ञा᳚त्या॒ इति॒ प्र - ज्ञा॒त्यै॒ । \newline
9. पु॒रस्ता᳚न् मिनोति मिनोति पु॒रस्ता᳚त् पु॒रस्ता᳚न् मिनोति पु॒रस्ता᳚त् पु॒रस्ता᳚न् मिनोति पु॒रस्ता᳚त् पु॒रस्ता᳚न् मिनोति पु॒रस्ता᳚त् । \newline
10. मि॒नो॒ति॒ पु॒रस्ता᳚त् पु॒रस्ता᳚न् मिनोति मिनोति पु॒रस्ता॒द्धि हि पु॒रस्ता᳚न् मिनोति मिनोति पु॒रस्ता॒द्धि । \newline
11. पु॒रस्ता॒द्धि हि पु॒रस्ता᳚त् पु॒रस्ता॒द्धि य॒ज्ञ्स्य॑ य॒ज्ञ्स्य॒ हि पु॒रस्ता᳚त् पु॒रस्ता॒द्धि य॒ज्ञ्स्य॑ । \newline
12. हि य॒ज्ञ्स्य॑ य॒ज्ञ्स्य॒ हि हि य॒ज्ञ्स्य॑ प्रज्ञा॒यते᳚ प्रज्ञा॒यते॑ य॒ज्ञ्स्य॒ हि हि य॒ज्ञ्स्य॑ प्रज्ञा॒यते᳚ । \newline
13. य॒ज्ञ्स्य॑ प्रज्ञा॒यते᳚ प्रज्ञा॒यते॑ य॒ज्ञ्स्य॑ य॒ज्ञ्स्य॑ प्रज्ञा॒यते ऽप्र॑ज्ञात॒ मप्र॑ज्ञातम् प्रज्ञा॒यते॑ य॒ज्ञ्स्य॑ य॒ज्ञ्स्य॑ प्रज्ञा॒यते ऽप्र॑ज्ञातम् । \newline
14. प्र॒ज्ञा॒यते ऽप्र॑ज्ञात॒ मप्र॑ज्ञातम् प्रज्ञा॒यते᳚ प्रज्ञा॒यते ऽप्र॑ज्ञातꣳ॒॒ हि ह्यप्र॑ज्ञातम् प्रज्ञा॒यते᳚ प्रज्ञा॒यते ऽप्र॑ज्ञातꣳ॒॒ हि । \newline
15. प्र॒ज्ञा॒यत॒ इति॑ प्र - ज्ञा॒यते᳚ । \newline
16. अप्र॑ज्ञातꣳ॒॒ हि ह्यप्र॑ज्ञात॒ मप्र॑ज्ञातꣳ॒॒ हि तत् तद्ध्यप्र॑ज्ञात॒ मप्र॑ज्ञातꣳ॒॒ हि तत् । \newline
17. अप्र॑ज्ञात॒मित्यप्र॑ - ज्ञा॒त॒म् । \newline
18. हि तत् तद्धि हि तद् यद् यत् तद्धि हि तद् यत् । \newline
19. तद् यद् यत् तत् तद् यदति॑प॒न्ने ऽति॑पन्ने॒ यत् तत् तद् यदति॑पन्ने । \newline
20. यदति॑प॒न्ने ऽति॑पन्ने॒ यद् यदति॑पन्न आ॒हु रा॒हु रति॑पन्ने॒ यद् यदति॑पन्न आ॒हुः । \newline
21. अति॑पन्न आ॒हु रा॒हु रति॑प॒न्ने ऽति॑पन्न आ॒हु रि॒द मि॒द मा॒हु रति॑प॒न्ने ऽति॑पन्न आ॒हु रि॒दम् । \newline
22. अति॑पन्न॒ इत्यति॑ - प॒न्ने॒ । \newline
23. आ॒हु रि॒द मि॒द मा॒हु रा॒हु रि॒दम् का॒र्य॑म् का॒र्य॑ मि॒द मा॒हु रा॒हु रि॒दम् का॒र्य᳚म् । \newline
24. इ॒दम् का॒र्य॑म् का॒र्य॑ मि॒द मि॒दम् का॒र्य॑ मासी दासीत् का॒र्य॑ मि॒द मि॒दम् का॒र्य॑ मासीत् । \newline
25. का॒र्य॑ मासी दासीत् का॒र्य॑म् का॒र्य॑ मासी॒ दिती त्या॑सीत् का॒र्य॑म् का॒र्य॑ मासी॒ दिति॑ । \newline
26. आ॒सी॒दिती त्या॑सी दासी॒ दिति॑ सा॒द्ध्याः सा॒द्ध्या इत्या॑सी दासी॒ दिति॑ सा॒द्ध्याः । \newline
27. इति॑ सा॒द्ध्याः सा॒द्ध्या इतीति॑ सा॒द्ध्या वै वै सा॒द्ध्या इतीति॑ सा॒द्ध्या वै । \newline
28. सा॒द्ध्या वै वै सा॒द्ध्याः सा॒द्ध्या वै दे॒वा दे॒वा वै सा॒द्ध्याः सा॒द्ध्या वै दे॒वाः । \newline
29. वै दे॒वा दे॒वा वै वै दे॒वा य॒ज्ञ्ं ॅय॒ज्ञ्म् दे॒वा वै वै दे॒वा य॒ज्ञ्म् । \newline
30. दे॒वा य॒ज्ञ्ं ॅय॒ज्ञ्म् दे॒वा दे॒वा य॒ज्ञ् मत्यति॑ य॒ज्ञ्म् दे॒वा दे॒वा य॒ज्ञ् मति॑ । \newline
31. य॒ज्ञ् मत्यति॑ य॒ज्ञ्ं ॅय॒ज्ञ् मत्य॑मन्यन्ता मन्य॒न्ताति॑ य॒ज्ञ्ं ॅय॒ज्ञ् मत्य॑मन्यन्त । \newline
32. अत्य॑मन्यन्ता मन्य॒न्ता त्य त्य॑मन्यन्त॒ ताꣳ स्तान॑मन्य॒न्ता त्य त्य॑मन्यन्त॒ तान् । \newline
33. अ॒म॒न्य॒न्त॒ ताꣳ स्तान॑मन्यन्ता मन्यन्त॒ तान्. य॒ज्ञो य॒ज्ञ् स्तान॑मन्यन्ता मन्यन्त॒ तान्. य॒ज्ञ्ः । \newline
34. तान्. य॒ज्ञो य॒ज्ञ् स्ताꣳ स्तान्. य॒ज्ञो न न य॒ज्ञ् स्ताꣳ स्तान्. य॒ज्ञो न । \newline
35. य॒ज्ञो न न य॒ज्ञो य॒ज्ञो नास्पृ॑श दस्पृश॒न् न य॒ज्ञो य॒ज्ञो नास्पृ॑शत् । \newline
36. नास्पृ॑श दस्पृश॒न् न नास्पृ॑श॒त् ताꣳ स्ता न॑स्पृश॒न् न नास्पृ॑श॒त् तान् । \newline
37. अ॒स्पृ॒श॒त् ताꣳ स्तान॑स्पृश दस्पृश॒त् तान्. यद् यत् तान॑स्पृश दस्पृश॒त् तान्. यत् । \newline
38. तान्. यद् यत् ताꣳ स्तान्. यद् य॒ज्ञ्स्य॑ य॒ज्ञ्स्य॒ यत् ताꣳ स्तान्. यद् य॒ज्ञ्स्य॑ । \newline
39. यद् य॒ज्ञ्स्य॑ य॒ज्ञ्स्य॒ यद् यद् य॒ज्ञ्स्या ति॑रिक्त॒ मति॑रिक्तं ॅय॒ज्ञ्स्य॒ यद् यद् य॒ज्ञ्स्या ति॑रिक्तम् । \newline
40. य॒ज्ञ्स्या ति॑रिक्त॒ मति॑रिक्तं ॅय॒ज्ञ्स्य॑ य॒ज्ञ्स्या ति॑रिक्त॒ मासी॒ दासी॒ दति॑रिक्तं ॅय॒ज्ञ्स्य॑ य॒ज्ञ्स्या ति॑रिक्त॒ मासी᳚त् । \newline
41. अति॑रिक्त॒ मासी॒ दासी॒ दति॑रिक्त॒ मति॑रिक्त॒ मासी॒त् तत् तदासी॒ दति॑रिक्त॒ मति॑रिक्त॒ मासी॒त् तत् । \newline
42. अति॑रिक्त॒मित्यति॑ - रि॒क्त॒म् । \newline
43. आसी॒त् तत् तदासी॒ दासी॒त् तद॑स्पृश दस्पृश॒त् तदासी॒ दासी॒त् तद॑स्पृशत् । \newline
44. तद॑स्पृश दस्पृश॒त् तत् तद॑स्पृश॒ दति॑रिक्त॒ मति॑रिक्त मस्पृश॒त् तत् तद॑स्पृश॒ दति॑रिक्तम् । \newline
45. अ॒स्पृ॒श॒ दति॑रिक्त॒ मति॑रिक्त मस्पृश दस्पृश॒ दति॑रिक्तं॒ ॅवै वा अति॑रिक्त मस्पृश दस्पृश॒ दति॑रिक्तं॒ ॅवै । \newline
46. अति॑रिक्तं॒ ॅवै वा अति॑रिक्त॒ मति॑रिक्तं॒ ॅवा ए॒त दे॒तद् वा अति॑रिक्त॒ मति॑रिक्तं॒ ॅवा ए॒तत् । \newline
47. अति॑रिक्त॒मित्यति॑ - रि॒क्त॒म् । \newline
48. वा ए॒त दे॒तद् वै वा ए॒तद् य॒ज्ञ्स्य॑ य॒ज्ञ्स्यै॒तद् वै वा ए॒तद् य॒ज्ञ्स्य॑ । \newline
49. ए॒तद् य॒ज्ञ्स्य॑ य॒ज्ञ्स्यै॒त दे॒तद् य॒ज्ञ्स्य॒ यद् यद् य॒ज्ञ्स्यै॒ तदे॒तद् य॒ज्ञ्स्य॒ यत् । \newline
50. य॒ज्ञ्स्य॒ यद् यद् य॒ज्ञ्स्य॑ य॒ज्ञ्स्य॒ यद॒ग्ना व॒ग्नौ यद् य॒ज्ञ्स्य॑ य॒ज्ञ्स्य॒ यद॒ग्नौ । \newline
51. यद॒ग्ना व॒ग्नौ यद् यद॒ग्ना व॒ग्नि म॒ग्नि म॒ग्नौ यद् यद॒ग्ना व॒ग्निम् । \newline
52. अ॒ग्ना व॒ग्नि म॒ग्नि म॒ग्ना व॒ग्ना व॒ग्निम् म॑थि॒त्वा म॑थि॒त्वा ऽग्नि म॒ग्ना व॒ग्ना व॒ग्निम् म॑थि॒त्वा । \newline
53. अ॒ग्निम् म॑थि॒त्वा म॑थि॒त्वा ऽग्नि म॒ग्निम् म॑थि॒त्वा प्र॒हर॑ति प्र॒हर॑ति मथि॒त्वा ऽग्नि म॒ग्निम् म॑थि॒त्वा प्र॒हर॑ति । \newline
54. म॒थि॒त्वा प्र॒हर॑ति प्र॒हर॑ति मथि॒त्वा म॑थि॒त्वा प्र॒हर॒ त्यति॑रिक्त॒ मति॑रिक्तम् प्र॒हर॑ति मथि॒त्वा म॑थि॒त्वा प्र॒हर॒ त्यति॑रिक्तम् । \newline
55. प्र॒हर॒ त्यति॑रिक्त॒ मति॑रिक्तम् प्र॒हर॑ति प्र॒हर॒ त्यति॑रिक्त मे॒त दे॒त दति॑रिक्तम् प्र॒हर॑ति प्र॒हर॒ त्यति॑रिक्त मे॒तत् । \newline
56. प्र॒हर॒तीति॑ प्र - हर॑ति । \newline
57. अति॑रिक्त मे॒त दे॒त दति॑रिक्त॒ मति॑रिक्त मे॒तद् यूप॑स्य॒ यूप॑स्यै॒त दति॑रिक्त॒ मति॑रिक्त मे॒तद् यूप॑स्य । \newline
58. अति॑रिक्त॒मित्यति॑ - रि॒क्त॒म् । \newline
59. ए॒तद् यूप॑स्य॒ यूप॑स्यै॒त दे॒तद् यूप॑स्य॒ यद् यद् यूप॑स्यै॒त दे॒तद् यूप॑स्य॒ यत् । \newline
\pagebreak
\markright{ TS 6.3.4.9  \hfill https://www.vedavms.in \hfill}

\section{ TS 6.3.4.9 }

\textbf{TS 6.3.4.9 } \newline
\textbf{Samhita Paata} \newline

-द्यूप॑स्य॒ यदू॒र्द्ध्वं च॒षाला॒त् तेषां॒ तद् भा॑ग॒धेयं॒ ताने॒व तेन॑ प्रीणाति दे॒वा वै सꣳस्थि॑ते॒ सोमे॒ प्र स्रुचोऽह॑र॒न् प्र यूपं॒ ते॑ऽमन्यन्त यज्ञ्वेश॒सं ॅवा इ॒दं कु॑र्म॒ इति॒ ते प्र॑स्त॒रꣳ स्रु॒चां नि॒ष्क्रय॑ण-मपश्य॒न्थ् स्वरुं॒ ॅयूप॑स्य॒ सꣳस्थि॑ते॒ सोमे॒ प्र प्र॑स्त॒रꣳ हर॑ति जु॒होति॒ स्वरु॒मय॑ज्ञ्वेशसाय ॥ \newline

\textbf{Pada Paata} \newline

यूप॑स्य । यत् । ऊ॒द्‌र्ध्वम् । च॒षाला᳚त् । तेषा᳚म् । तत् । भा॒ग॒धेय॒मिति॑ भाग - धेय᳚म् । तान् । ए॒व । तेन॑ । प्री॒णा॒ति॒ । दे॒वाः । वै । सꣳस्थि॑त॒ इति॒ सं-स्थि॒ते॒ । सोमे᳚ । प्रेति॑ । स्रुचः॑ । अह॑रन्न् । प्रेति॑ । यूप᳚म् । ते । अ॒म॒न्य॒न्त॒ । य॒ज्ञ्॒वे॒श॒समिति॑ यज्ञ् - वे॒श॒स॒म् । वै । इ॒दम् । कु॒र्मः॒ । इति॑ । ते । प्र॒स्त॒रमिति॑ प्र - स्त॒रम् । स्रु॒चाम् । नि॒ष्क्रय॑ण॒मिति॑ निः - क्रय॑णम् । अ॒प॒श्य॒न्न् । स्वरु᳚म् । यूप॑स्य । सꣳस्थि॑त॒ इति॒ सं - स्थि॒ते॒ । सोमे᳚ । प्रेति॑ । प्र॒स्त॒रमिति॑ प्र-स्त॒रम् । हर॑ति । जु॒होति॑ । स्वरु᳚म् । अय॑ज्ञ्वेशसा॒येत्यय॑ज्ञ्-वे॒श॒सा॒य॒ ॥  \newline


\textbf{Krama Paata} \newline

यूप॑स्य॒ यत् । यदू॒र्द्ध्वम् । ऊ॒र्द्ध्वम् च॒षाला᳚त् । च॒षाला॒त् तेषा᳚म् । तेषा॒म् तत् । तद् भा॑ग॒धेय᳚म् । भा॒ग॒धेय॒म् तान् । भा॒ग॒धेय॒मिति॑ भाग - धेय᳚म् । ताने॒व । ए॒व तेन॑ । तेन॑ प्रीणाति । प्री॒णा॒ति॒ दे॒वाः । दे॒वा वै । वै सꣳस्थि॑ते । सꣳस्थि॑ते॒ सोमे᳚ । सꣳस्थि॑त॒ इति॒ सम् - स्थि॒ते॒ । सोमे॒ प्र । प्र स्रुचः॑ । स्रुचोऽह॑रन्न् । अह॑र॒न् प्र । प्र यूप᳚म् । यूप॒म् ते । ते॑ऽमन्यन्त । अ॒म॒न्य॒न्त॒ य॒ज्ञ्॒वे॒श॒सम् । य॒ज्ञ्॒वे॒श॒सम् ॅवै । य॒ज्ञ्॒वे॒श॒समिति॑ यज्ञ् - वे॒श॒सम् । वा इ॒दम् । इ॒दम् कु॑र्मः । कु॒र्म॒ इति॑ । इति॒ ते । ते प्र॑स्त॒रम् । प्र॒स्त॒रꣳ स्रु॒चाम् । प्र॒स्त॒रमिति॑ प्र - स्त॒रम् । स्रु॒चाम् नि॒ष्क्रय॑णम् । नि॒ष्क्रय॑णमपश्यत् । नि॒ष्क्रय॑ण॒मिति॑ निः - क्रय॑णम् । अ॒प॒श्य॒न्थ् स्वरु᳚म् । स्वरु॒म् ॅयूप॑स्य । यूप॑स्य॒ सꣳस्थि॑ते । सꣳस्थि॑ते॒ सोमे᳚ । सꣳस्थि॑त॒ इति॒ सम् - स्थि॒ते॒ । सोमे॒ प्र । प्र प्र॑स्त॒रम् । प्र॒स्त॒रꣳ हर॑ति । प्र॒स्त॒रमिति॑ प्र - स्त॒रम् । हर॑ति जु॒होति॑ । जु॒होति॒ स्वरु᳚म् । स्वरु॒मय॑ज्ञ्वेशसाय । अय॑ज्ञ्वेशसा॒येत्यय॑ज्ञ् - वे॒श॒सा॒य॒ । \newline

\textbf{Jatai Paata} \newline

1. यूप॑स्य॒ यद् यद् यूप॑स्य॒ यूप॑स्य॒ यत् । \newline
2. यदू॒र्द्ध्व मू॒र्द्ध्वं ॅयद् यदू॒र्द्ध्वम् । \newline
3. ऊ॒र्द्ध्वम् च॒षाला᳚च् च॒षाला॑ दू॒र्द्ध्व मू॒र्द्ध्वम् च॒षाला᳚त् । \newline
4. च॒षाला॒त् तेषा॒म् तेषा᳚म् च॒षाला᳚च् च॒षाला॒त् तेषा᳚म् । \newline
5. तेषा॒म् तत् तत् तेषा॒म् तेषा॒म् तत् । \newline
6. तद् भा॑ग॒धेय॑म् भाग॒धेय॒म् तत् तद् भा॑ग॒धेय᳚म् । \newline
7. भा॒ग॒धेय॒म् ताꣳ स्तान् भा॑ग॒धेय॑म् भाग॒धेय॒म् तान् । \newline
8. भा॒ग॒धेय॒मिति॑ भाग - धेय᳚म् । \newline
9. ताने॒ वैव ताꣳ स्ताने॒व । \newline
10. ए॒व तेन॒ तेनै॒ वैव तेन॑ । \newline
11. तेन॑ प्रीणाति प्रीणाति॒ तेन॒ तेन॑ प्रीणाति । \newline
12. प्री॒णा॒ति॒ दे॒वा दे॒वाः प्री॑णाति प्रीणाति दे॒वाः । \newline
13. दे॒वा वै वै दे॒वा दे॒वा वै । \newline
14. वै सꣳस्थि॑ते॒ सꣳस्थि॑ते॒ वै वै सꣳस्थि॑ते । \newline
15. सꣳस्थि॑ते॒ सोमे॒ सोमे॒ सꣳस्थि॑ते॒ सꣳस्थि॑ते॒ सोमे᳚ । \newline
16. सꣳस्थि॑त॒ इति॒ सं - स्थि॒ते॒ । \newline
17. सोमे॒ प्र प्र सोमे॒ सोमे॒ प्र । \newline
18. प्र स्रुचः॒ स्रुचः॒ प्र प्र स्रुचः॑ । \newline
19. स्रुचो ऽह॑र॒न् नह॑र॒न् थ्स्रुचः॒ स्रुचो ऽह॑रन्न् । \newline
20. अह॑र॒न् प्र प्राह॑र॒न् नह॑र॒न् प्र । \newline
21. प्र यूपं॒ ॅयूप॒म् प्र प्र यूप᳚म् । \newline
22. यूप॒म् ते ते यूपं॒ ॅयूप॒म् ते । \newline
23. ते॑ ऽमन्यन्ता मन्यन्त॒ ते ते॑ ऽमन्यन्त । \newline
24. अ॒म॒न्य॒न्त॒ य॒ज्ञ्॒वे॒श॒सं॒ ॅय॒ज्ञ्॒वे॒श॒स॒ म॒म॒न्य॒न्ता॒ म॒न्य॒न्त॒ य॒ज्ञ्॒वे॒श॒स॒म् । \newline
25. य॒ज्ञ्॒वे॒श॒सं॒ ॅवै वै य॑ज्ञ्वेशसं ॅयज्ञ्वेशसं॒ ॅवै । \newline
26. य॒ज्ञ्॒वे॒श॒समिति॑ यज्ञ् - वे॒श॒स॒म् । \newline
27. वा इ॒द मि॒दं ॅवै वा इ॒दम् । \newline
28. इ॒दम् कु॑र्मः कुर्म इ॒द मि॒दम् कु॑र्मः । \newline
29. कु॒र्म॒ इतीति॑ कुर्मः कुर्म॒ इति॑ । \newline
30. इति॒ ते त इतीति॒ ते । \newline
31. ते प्र॑स्त॒रम् प्र॑स्त॒रम् ते ते प्र॑स्त॒रम् । \newline
32. प्र॒स्त॒रꣳ स्रु॒चाꣳ स्रु॒चाम् प्र॑स्त॒रम् प्र॑स्त॒रꣳ स्रु॒चाम् । \newline
33. प्र॒स्त॒रमिति॑ प्र - स्त॒रम् । \newline
34. स्रु॒चाम् नि॒ष्क्रय॑णम् नि॒ष्क्रय॑णꣳ स्रु॒चाꣳ स्रु॒चाम् नि॒ष्क्रय॑णम् । \newline
35. नि॒ष्क्रय॑ण मपश्यन् नपश्यन् नि॒ष्क्रय॑णम् नि॒ष्क्रय॑ण मपश्यन्न् । \newline
36. नि॒ष्क्रय॑ण॒मिति॑ निः - क्रय॑णम् । \newline
37. अ॒प॒श्य॒न् थ्स्वरुꣳ॒॒ स्वरु॑ मपश्यन् नपश्य॒न् थ्स्वरु᳚म् । \newline
38. स्वरुं॒ ॅयूप॑स्य॒ यूप॑स्य॒ स्वरुꣳ॒॒ स्वरुं॒ ॅयूप॑स्य । \newline
39. यूप॑स्य॒ सꣳस्थि॑ते॒ सꣳस्थि॑ते॒ यूप॑स्य॒ यूप॑स्य॒ सꣳस्थि॑ते । \newline
40. सꣳस्थि॑ते॒ सोमे॒ सोमे॒ सꣳस्थि॑ते॒ सꣳस्थि॑ते॒ सोमे᳚ । \newline
41. सꣳस्थि॑त॒ इति॒ सं - स्थि॒ते॒ । \newline
42. सोमे॒ प्र प्र सोमे॒ सोमे॒ प्र । \newline
43. प्र प्र॑स्त॒रम् प्र॑स्त॒रम् प्र प्र प्र॑स्त॒रम् । \newline
44. प्र॒स्त॒रꣳ हर॑ति॒ हर॑ति प्रस्त॒रम् प्र॑स्त॒रꣳ हर॑ति । \newline
45. प्र॒स्त॒रमिति॑ प्र - स्त॒रम् । \newline
46. हर॑ति जु॒होति॑ जु॒होति॒ हर॑ति॒ हर॑ति जु॒होति॑ । \newline
47. जु॒होति॒ स्वरुꣳ॒॒ स्वरु॑म् जु॒होति॑ जु॒होति॒ स्वरु᳚म् । \newline
48. स्वरु॒ मय॑ज्ञ्वेशसा॒या य॑ज्ञ्वेशसाय॒ स्वरुꣳ॒॒ स्वरु॒ मय॑ज्ञ्वेशसाय । \newline
49. अय॑ज्ञ्वेशसा॒येत्यय॑ज्ञ् - वे॒श॒सा॒य॒ । \newline

\textbf{Ghana Paata } \newline

1. यूप॑स्य॒ यद् यद् यूप॑स्य॒ यूप॑स्य॒ यदू॒र्द्ध्व मू॒र्द्ध्वं ॅयद् यूप॑स्य॒ यूप॑स्य॒ यदू॒र्द्ध्वम् । \newline
2. यदू॒र्द्ध्व मू॒र्द्ध्वं ॅयद् यदू॒र्द्ध्वम् च॒षाला᳚च् च॒षाला॑ दू॒र्द्ध्वं ॅयद् यदू॒र्द्ध्वम् च॒षाला᳚त् । \newline
3. ऊ॒र्द्ध्वम् च॒षाला᳚च् च॒षाला॑ दू॒र्द्ध्व मू॒र्द्ध्वम् च॒षाला॒त् तेषा॒म् तेषा᳚म् च॒षाला॑ दू॒र्द्ध्व मू॒र्द्ध्वम् च॒षाला॒त् तेषा᳚म् । \newline
4. च॒षाला॒त् तेषा॒म् तेषा᳚म् च॒षाला᳚च् च॒षाला॒त् तेषा॒म् तत् तत् तेषा᳚म् च॒षाला᳚च् च॒षाला॒त् तेषा॒म् तत् । \newline
5. तेषा॒म् तत् तत् तेषा॒म् तेषा॒म् तद् भा॑ग॒धेय॑म् भाग॒धेय॒म् तत् तेषा॒म् तेषा॒म् तद् भा॑ग॒धेय᳚म् । \newline
6. तद् भा॑ग॒धेय॑म् भाग॒धेय॒म् तत् तद् भा॑ग॒धेय॒म् ताꣳ स्तान् भा॑ग॒धेय॒म् तत् तद् भा॑ग॒धेय॒म् तान् । \newline
7. भा॒ग॒धेय॒म् ताꣳ स्तान् भा॑ग॒धेय॑म् भाग॒धेय॒म् ताने॒ वैव तान् भा॑ग॒धेय॑म् भाग॒धेय॒म् ताने॒व । \newline
8. भा॒ग॒धेय॒मिति॑ भाग - धेय᳚म् । \newline
9. ताने॒ वैव ताꣳ स्ताने॒व तेन॒ तेनै॒व ताꣳ स्ताने॒व तेन॑ । \newline
10. ए॒व तेन॒ तेनै॒ वैव तेन॑ प्रीणाति प्रीणाति॒ तेनै॒ वैव तेन॑ प्रीणाति । \newline
11. तेन॑ प्रीणाति प्रीणाति॒ तेन॒ तेन॑ प्रीणाति दे॒वा दे॒वाः प्री॑णाति॒ तेन॒ तेन॑ प्रीणाति दे॒वाः । \newline
12. प्री॒णा॒ति॒ दे॒वा दे॒वाः प्री॑णाति प्रीणाति दे॒वा वै वै दे॒वाः प्री॑णाति प्रीणाति दे॒वा वै । \newline
13. दे॒वा वै वै दे॒वा दे॒वा वै सꣳस्थि॑ते॒ सꣳस्थि॑ते॒ वै दे॒वा दे॒वा वै सꣳस्थि॑ते । \newline
14. वै सꣳस्थि॑ते॒ सꣳस्थि॑ते॒ वै वै सꣳस्थि॑ते॒ सोमे॒ सोमे॒ सꣳस्थि॑ते॒ वै वै सꣳस्थि॑ते॒ सोमे᳚ । \newline
15. सꣳस्थि॑ते॒ सोमे॒ सोमे॒ सꣳस्थि॑ते॒ सꣳस्थि॑ते॒ सोमे॒ प्र प्र सोमे॒ सꣳस्थि॑ते॒ सꣳस्थि॑ते॒ सोमे॒ प्र । \newline
16. सꣳस्थि॑त॒ इति॒ सं - स्थि॒ते॒ । \newline
17. सोमे॒ प्र प्र सोमे॒ सोमे॒ प्र स्रुचः॒ स्रुचः॒ प्र सोमे॒ सोमे॒ प्र स्रुचः॑ । \newline
18. प्र स्रुचः॒ स्रुचः॒ प्र प्र स्रुचो ऽह॑र॒न् नह॑र॒न् थ्स्रुचः॒ प्र प्र स्रुचो ऽह॑रन्न् । \newline
19. स्रुचो ऽह॑र॒न् नह॑र॒न् थ्स्रुचः॒ स्रुचो ऽह॑र॒न् प्र प्राह॑र॒न् थ्स्रुचः॒ स्रुचो ऽह॑र॒न् प्र । \newline
20. अह॑र॒न् प्र प्राह॑र॒न् नह॑र॒न् प्र यूपं॒ ॅयूप॒म् प्राह॑र॒न् नह॑र॒न् प्र यूप᳚म् । \newline
21. प्र यूपं॒ ॅयूप॒म् प्र प्र यूप॒म् ते ते यूप॒म् प्र प्र यूप॒म् ते । \newline
22. यूप॒म् ते ते यूपं॒ ॅयूप॒म् ते॑ ऽमन्यन्ता मन्यन्त॒ ते यूपं॒ ॅयूप॒म् ते॑ ऽमन्यन्त । \newline
23. ते॑ ऽमन्यन्ता मन्यन्त॒ ते ते॑ ऽमन्यन्त यज्ञ्वेशसं ॅयज्ञ्वेशस ममन्यन्त॒ ते ते॑ ऽमन्यन्त यज्ञ्वेशसम् । \newline
24. अ॒म॒न्य॒न्त॒ य॒ज्ञ्॒वे॒श॒सं॒ ॅय॒ज्ञ्॒वे॒श॒स॒ म॒म॒न्य॒न्ता॒ म॒न्य॒न्त॒ य॒ज्ञ्॒वे॒श॒सं॒ ॅवै वै य॑ज्ञ्वेशस ममन्यन्ता मन्यन्त यज्ञ्वेशसं॒ ॅवै । \newline
25. य॒ज्ञ्॒वे॒श॒सं॒ ॅवै वै य॑ज्ञ्वेशसं ॅयज्ञ्वेशसं॒ ॅवा इ॒द मि॒दं ॅवै य॑ज्ञ्वेशसं ॅयज्ञ्वेशसं॒ ॅवा इ॒दम् । \newline
26. य॒ज्ञ्॒वे॒श॒समिति॑ यज्ञ् - वे॒श॒स॒म् । \newline
27. वा इ॒द मि॒दं ॅवै वा इ॒दम् कु॑र्मः कुर्म इ॒दं ॅवै वा इ॒दम् कु॑र्मः । \newline
28. इ॒दम् कु॑र्मः कुर्म इ॒द मि॒दम् कु॑र्म॒ इतीति॑ कुर्म इ॒द मि॒दम् कु॑र्म॒ इति॑ । \newline
29. कु॒र्म॒ इतीति॑ कुर्मः कुर्म॒ इति॒ ते त इति॑ कुर्मः कुर्म॒ इति॒ ते । \newline
30. इति॒ ते त इतीति॒ ते प्र॑स्त॒रम् प्र॑स्त॒रम् त इतीति॒ ते प्र॑स्त॒रम् । \newline
31. ते प्र॑स्त॒रम् प्र॑स्त॒रम् ते ते प्र॑स्त॒रꣳ स्रु॒चाꣳ स्रु॒चाम् प्र॑स्त॒रम् ते ते प्र॑स्त॒रꣳ स्रु॒चाम् । \newline
32. प्र॒स्त॒रꣳ स्रु॒चाꣳ स्रु॒चाम् प्र॑स्त॒रम् प्र॑स्त॒रꣳ स्रु॒चाम् नि॒ष्क्रय॑णम् नि॒ष्क्रय॑णꣳ स्रु॒चाम् प्र॑स्त॒रम् प्र॑स्त॒रꣳ स्रु॒चाम् नि॒ष्क्रय॑णम् । \newline
33. प्र॒स्त॒रमिति॑ प्र - स्त॒रम् । \newline
34. स्रु॒चाम् नि॒ष्क्रय॑णम् नि॒ष्क्रय॑णꣳ स्रु॒चाꣳ स्रु॒चाम् नि॒ष्क्रय॑ण मपश्यन् नपश्यन् नि॒ष्क्रय॑णꣳ स्रु॒चाꣳ स्रु॒चाम् नि॒ष्क्रय॑ण मपश्यन्न् । \newline
35. नि॒ष्क्रय॑ण मपश्यन् नपश्यन् नि॒ष्क्रय॑णम् नि॒ष्क्रय॑ण मपश्य॒न् थ्स्वरुꣳ॒॒ स्वरु॑ मपश्यन् नि॒ष्क्रय॑णम् नि॒ष्क्रय॑ण मपश्य॒न् थ्स्वरु᳚म् । \newline
36. नि॒ष्क्रय॑ण॒मिति॑ निः - क्रय॑णम् । \newline
37. अ॒प॒श्य॒न् थ्स्वरुꣳ॒॒ स्वरु॑ मपश्यन् नपश्य॒न् थ्स्वरुं॒ ॅयूप॑स्य॒ यूप॑स्य॒ स्वरु॑ मपश्यन् नपश्य॒न् थ्स्वरुं॒ ॅयूप॑स्य । \newline
38. स्वरुं॒ ॅयूप॑स्य॒ यूप॑स्य॒ स्वरुꣳ॒॒ स्वरुं॒ ॅयूप॑स्य॒ सꣳस्थि॑ते॒ सꣳस्थि॑ते॒ यूप॑स्य॒ स्वरुꣳ॒॒ स्वरुं॒ ॅयूप॑स्य॒ सꣳस्थि॑ते । \newline
39. यूप॑स्य॒ सꣳस्थि॑ते॒ सꣳस्थि॑ते॒ यूप॑स्य॒ यूप॑स्य॒ सꣳस्थि॑ते॒ सोमे॒ सोमे॒ सꣳस्थि॑ते॒ यूप॑स्य॒ यूप॑स्य॒ सꣳस्थि॑ते॒ सोमे᳚ । \newline
40. सꣳस्थि॑ते॒ सोमे॒ सोमे॒ सꣳस्थि॑ते॒ सꣳस्थि॑ते॒ सोमे॒ प्र प्र सोमे॒ सꣳस्थि॑ते॒ सꣳस्थि॑ते॒ सोमे॒ प्र । \newline
41. सꣳस्थि॑त॒ इति॒ सं - स्थि॒ते॒ । \newline
42. सोमे॒ प्र प्र सोमे॒ सोमे॒ प्र प्र॑स्त॒रम् प्र॑स्त॒रम् प्र सोमे॒ सोमे॒ प्र प्र॑स्त॒रम् । \newline
43. प्र प्र॑स्त॒रम् प्र॑स्त॒रम् प्र प्र प्र॑स्त॒रꣳ हर॑ति॒ हर॑ति प्रस्त॒रम् प्र प्र प्र॑स्त॒रꣳ हर॑ति । \newline
44. प्र॒स्त॒रꣳ हर॑ति॒ हर॑ति प्रस्त॒रम् प्र॑स्त॒रꣳ हर॑ति जु॒होति॑ जु॒होति॒ हर॑ति प्रस्त॒रम् प्र॑स्त॒रꣳ हर॑ति जु॒होति॑ । \newline
45. प्र॒स्त॒रमिति॑ प्र - स्त॒रम् । \newline
46. हर॑ति जु॒होति॑ जु॒होति॒ हर॑ति॒ हर॑ति जु॒होति॒ स्वरुꣳ॒॒ स्वरु॑म् जु॒होति॒ हर॑ति॒ हर॑ति जु॒होति॒ स्वरु᳚म् । \newline
47. जु॒होति॒ स्वरुꣳ॒॒ स्वरु॑म् जु॒होति॑ जु॒होति॒ स्वरु॒ मय॑ज्ञ्वेशसा॒या य॑ज्ञ्वेशसाय॒ स्वरु॑म् जु॒होति॑ जु॒होति॒ स्वरु॒ मय॑ज्ञ्वेशसाय । \newline
48. स्वरु॒ मय॑ज्ञ्वेशसा॒या य॑ज्ञ्वेशसाय॒ स्वरुꣳ॒॒ स्वरु॒ मय॑ज्ञ्वेशसाय । \newline
49. अय॑ज्ञ्वेशसा॒येत्यय॑ज्ञ् - वे॒श॒सा॒य॒ । \newline
\pagebreak
\markright{ TS 6.3.5.1  \hfill https://www.vedavms.in \hfill}

\section{ TS 6.3.5.1 }

\textbf{TS 6.3.5.1 } \newline
\textbf{Samhita Paata} \newline

सा॒द्ध्या वै दे॒वा अ॒स्मिॅल्लो॒क आ॑स॒न् नान्यत् किं॑ च॒न मि॒षत् ते᳚ऽग्निमे॒वाग्नये॒ मेधा॒या ऽल॑भन्त॒ न ह्य॑न्यदा॑ल॒भ्यं॑-मवि॑न्द॒न् ततो॒ वा इ॒माः प्र॒जाः प्राजा॑यन्त॒ यद॒ग्नाव॒ग्निं म॑थि॒त्वा प्र॒हर॑ति प्र॒जानां᳚ प्र॒जन॑नाय रु॒द्रो वा ए॒ष यद॒ग्निर्यज॑मानः प॒शुर्यत् प॒शुमा॒लभ्या॒ग्निं मन्थे᳚द् रु॒द्राय॒ यज॑मान॒- [  ] \newline

\textbf{Pada Paata} \newline

सा॒द्ध्याः । वै । दे॒वाः । अ॒स्मिन्न् । लो॒के । आ॒स॒न्न् । न । अ॒न्यत् । किम् । च॒न । मि॒षत् । ते । अ॒ग्निम् । ए॒व । अ॒ग्नये᳚ । मेधा॑य । एति॑ । अ॒ल॒भ॒न्त॒ । न । हि । अ॒न्यत् । आ॒ल॒भ्यं॑मित्या᳚-ल॒भ्यं᳚म् । अवि॑न्दन्न् । ततः॑ । वै । इ॒माः । प्र॒जा इति॑ प्र - जाः । प्रेति॑ । अ॒जा॒य॒न्त॒ । यत् । अ॒ग्नौ । अ॒ग्निम् । म॒थि॒त्वा । प्र॒हर॒तीति॑ प्र - हर॑ति । प्र॒जाना॒मिति॑ प्र - जाना᳚म् । प्र॒जन॑ना॒येति॑ प्र-जन॑नाय । रु॒द्रः । वै । ए॒षः । यत् । अ॒ग्निः । यज॑मानः । प॒शुः । यत् । प॒शुम् । आ॒लभ्येत्या᳚ - लभ्य॑ । अ॒ग्निम् । मन्थे᳚त् । रु॒द्राय॑ । यज॑मानम् ।  \newline


\textbf{Krama Paata} \newline

सा॒द्ध्या वै । वै दे॒वाः । दे॒वा अ॒स्मिन्न् । अ॒स्मिन् ॅलो॒के । लो॒क आ॑सन्न् । आ॒स॒न् न । नान्यत् । अ॒न्यत् किम् । किम् च॒न । च॒न मि॒षत् । मि॒षत् ते । ते᳚ऽग्निम् । अ॒ग्निमे॒व । ए॒वाग्नये᳚ । अ॒ग्नये॒ मेधा॑य । मेधा॒या । आऽल॑भन्त । अ॒ल॒भ॒न्त॒ न । न हि । ह्य॑न्यत् । अ॒न्यदा॑ल॒म्भ्य᳚म् । आ॒ल॒म्भ्य॑मवि॑न्दन्न् । आ॒ल॒म्भ्य॑मित्या᳚ - ल॒म्भ्य᳚म् । अवि॑न्द॒न् ततः॑ । ततो॒ वै । वा इ॒माः । इ॒माः प्र॒जाः । प्र॒जाः प्र । प्र॒जा इति॑ प्र - जाः । प्राजा॑यन्त । अ॒जा॒य॒न्त॒ यत् । यद॒ग्नौ । अ॒ग्नाव॒ग्निम् । अ॒ग्निम् म॑थि॒त्वा । म॒थि॒त्वा प्र॒हर॑ति । प्र॒हर॑ति प्र॒जाना᳚म् । प्र॒हर॒तीति॑ प्र - हर॑ति । प्र॒जाना᳚म् प्र॒जन॑नाय । प्र॒जाना॒मिति॑ प्र - जाना᳚म् । प्र॒जन॑नाय रु॒द्रः । प्र॒जन॑ना॒येति॑ प्र - जन॑नाय । रु॒द्रो वै । वा ए॒षः । ए॒ष यत् । यद॒ग्निः । अ॒ग्निर् यज॑मानः । यज॑मानः प॒शुः । प॒शुर् यत् । यत् प॒शुम् । प॒शुमा॒लभ्य॑ । आ॒लभ्या॒ग्निम् । आ॒लभ्येत्या᳚ - लभ्य॑ । अ॒ग्निम् मन्थे᳚त् । मन्थे᳚द् रु॒द्राय॑ । रु॒द्राय॒ यज॑मानम् । यज॑मान॒मपि॑ \newline

\textbf{Jatai Paata} \newline

1. सा॒द्ध्या वै वै सा॒द्ध्याः सा॒द्ध्या वै । \newline
2. वै दे॒वा दे॒वा वै वै दे॒वाः । \newline
3. दे॒वा अ॒स्मिन् न॒स्मिन् दे॒वा दे॒वा अ॒स्मिन्न् । \newline
4. अ॒स्मिन् ॅलो॒के लो॒के᳚ ऽस्मिन् न॒स्मिन् ॅलो॒के । \newline
5. लो॒क आ॑सन् नासन् ॅलो॒के लो॒क आ॑सन्न् । \newline
6. आ॒स॒न् न नास॑न् नास॒न् न । \newline
7. नान्य द॒न्यन् न नान्यत् । \newline
8. अ॒न्यत् किम् कि म॒न्य द॒न्यत् किम् । \newline
9. किम् च॒न च॒न किम् किम् च॒न । \newline
10. च॒न मि॒षन् मि॒षच् च॒न च॒न मि॒षत् । \newline
11. मि॒षत् ते ते मि॒षन् मि॒षत् ते । \newline
12. ते᳚ ऽग्नि म॒ग्निम् ते ते᳚ ऽग्निम् । \newline
13. अ॒ग्नि मे॒वै वाग्नि म॒ग्नि मे॒व । \newline
14. ए॒वाग्नये॒ ऽग्नय॑ ए॒वै वाग्नये᳚ । \newline
15. अ॒ग्नये॒ मेधा॑य॒ मेधा॑या॒ ग्नये॒ ऽग्नये॒ मेधा॑य । \newline
16. मेधा॒या मेधा॑य॒ मेधा॒या । \newline
17. आ ऽल॑भन्ता लभ॒न्ता ऽल॑भन्त । \newline
18. अ॒ल॒भ॒न्त॒ न नाल॑भन्ता लभन्त॒ न । \newline
19. न हि हि न न हि । \newline
20. ह्य॑न्य द॒न्य द्धि ह्य॑न्यत् । \newline
21. अ॒न्य दा॑लं॒भ्य॑ मालं॒भ्य॑ म॒न्य द॒न्यदा॑ लं॒भ्य᳚म् । \newline
22. आ॒लं॒भ्य॑ मवि॑न्द॒न् नवि॑न्दन् नालं॒भ्य॑ मालं॒भ्य॑ मवि॑न्दन्न् । \newline
23. आ॒लं॒भ्य॑मित्या᳚ - लं॒भ्य᳚म् । \newline
24. अवि॑न्द॒न् तत॒ स्ततो ऽवि॑न्द॒न् नवि॑न्द॒न् ततः॑ । \newline
25. ततो॒ वै वै तत॒ स्ततो॒ वै । \newline
26. वा इ॒मा इ॒मा वै वा इ॒माः । \newline
27. इ॒माः प्र॒जाः प्र॒जा इ॒मा इ॒माः प्र॒जाः । \newline
28. प्र॒जाः प्र प्र प्र॒जाः प्र॒जाः प्र । \newline
29. प्र॒जा इति॑ प्र - जाः । \newline
30. प्राजा॑यन्ता जायन्त॒ प्र प्राजा॑यन्त । \newline
31. अ॒जा॒य॒न्त॒ यद् यद॑जायन्ता जायन्त॒ यत् । \newline
32. यद॒ग्ना व॒ग्नौ यद् यद॒ग्नौ । \newline
33. अ॒ग्ना व॒ग्नि म॒ग्नि म॒ग्ना व॒ग्ना व॒ग्निम् । \newline
34. अ॒ग्निम् म॑थि॒त्वा म॑थि॒त्वा ऽग्नि म॒ग्निम् म॑थि॒त्वा । \newline
35. म॒थि॒त्वा प्र॒हर॑ति प्र॒हर॑ति मथि॒त्वा म॑थि॒त्वा प्र॒हर॑ति । \newline
36. प्र॒हर॑ति प्र॒जाना᳚म् प्र॒जाना᳚म् प्र॒हर॑ति प्र॒हर॑ति प्र॒जाना᳚म् । \newline
37. प्र॒हर॒तीति॑ प्र - हर॑ति । \newline
38. प्र॒जाना᳚म् प्र॒जन॑नाय प्र॒जन॑नाय प्र॒जाना᳚म् प्र॒जाना᳚म् प्र॒जन॑नाय । \newline
39. प्र॒जाना॒मिति॑ प्र - जाना᳚म् । \newline
40. प्र॒जन॑नाय रु॒द्रो रु॒द्रः प्र॒जन॑नाय प्र॒जन॑नाय रु॒द्रः । \newline
41. प्र॒जन॑ना॒येति॑ प्र - जान॑नाय । \newline
42. रु॒द्रो वै वै रु॒द्रो रु॒द्रो वै । \newline
43. वा ए॒ष ए॒ष वै वा ए॒षः । \newline
44. ए॒ष यद् यदे॒ष ए॒ष यत् । \newline
45. यद॒ग्नि र॒ग्निर् यद् यद॒ग्निः । \newline
46. अ॒ग्निर् यज॑मानो॒ यज॑मानो॒ ऽग्नि र॒ग्निर् यज॑मानः । \newline
47. यज॑मानः प॒शुः प॒शुर् यज॑मानो॒ यज॑मानः प॒शुः । \newline
48. प॒शुर् यद् यत् प॒शुः प॒शुर् यत् । \newline
49. यत् प॒शुम् प॒शुं ॅयद् यत् प॒शुम् । \newline
50. प॒शु मा॒लभ्या॒ लभ्य॑ प॒शुम् प॒शु मा॒लभ्य॑ । \newline
51. आ॒लभ्या॒ग्नि म॒ग्नि मा॒लभ्या॒ लभ्या॒ग्निम् । \newline
52. आ॒लभ्येत्या᳚ - लभ्य॑ । \newline
53. अ॒ग्निम् मन्थे॒न् मन्थे॑ द॒ग्नि म॒ग्निम् मन्थे᳚त् । \newline
54. मन्थे᳚द् रु॒द्राय॑ रु॒द्राय॒ मन्थे॒न् मन्थे᳚द् रु॒द्राय॑ । \newline
55. रु॒द्राय॒ यज॑मानं॒ ॅयज॑मानꣳ रु॒द्राय॑ रु॒द्राय॒ यज॑मानम् । \newline
56. यज॑मान॒ मप्यपि॒ यज॑मानं॒ ॅयज॑मान॒ मपि॑ । \newline

\textbf{Ghana Paata } \newline

1. सा॒द्ध्या वै वै सा॒द्ध्याः सा॒द्ध्या वै दे॒वा दे॒वा वै सा॒द्ध्याः सा॒द्ध्या वै दे॒वाः । \newline
2. वै दे॒वा दे॒वा वै वै दे॒वा अ॒स्मिन् न॒स्मिन् दे॒वा वै वै दे॒वा अ॒स्मिन्न् । \newline
3. दे॒वा अ॒स्मिन् न॒स्मिन् दे॒वा दे॒वा अ॒स्मिन् ॅलो॒के लो॒के᳚ ऽस्मिन् दे॒वा दे॒वा अ॒स्मिन् ॅलो॒के । \newline
4. अ॒स्मिन् ॅलो॒के लो॒के᳚ ऽस्मिन् न॒स्मिन् ॅलो॒क आ॑सन् नासन् ॅलो॒के᳚ ऽस्मिन् न॒स्मिन् ॅलो॒क आ॑सन्न् । \newline
5. लो॒क आ॑सन् नासन् ॅलो॒के लो॒क आ॑स॒न् न नास॑न् ॅलो॒के लो॒क आ॑स॒न् न । \newline
6. आ॒स॒न् न नास॑न् नास॒न् नान्य द॒न्यन् नास॑न् नास॒न् नान्यत् । \newline
7. नान्य द॒न्यन् न नान्यत् किम् कि म॒न्यन् न नान्यत् किम् । \newline
8. अ॒न्यत् किम् कि म॒न्य द॒न्यत् किम् च॒न च॒न कि म॒न्य द॒न्यत् किम् च॒न । \newline
9. किम् च॒न च॒न किम् किम् च॒न मि॒षन् मि॒षच् च॒न किम् किम् च॒न मि॒षत् । \newline
10. च॒न मि॒षन् मि॒षच् च॒न च॒न मि॒षत् ते ते मि॒षच् च॒न च॒न मि॒षत् ते । \newline
11. मि॒षत् ते ते मि॒षन् मि॒षत् ते᳚ ऽग्नि म॒ग्निम् ते मि॒षन् मि॒षत् ते᳚ ऽग्निम् । \newline
12. ते᳚ ऽग्नि म॒ग्निम् ते ते᳚ ऽग्नि मे॒वै वाग्निम् ते ते᳚ ऽग्नि मे॒व । \newline
13. अ॒ग्नि मे॒वै वाग्नि म॒ग्नि मे॒वाग्नये॒ ऽग्नय॑ ए॒वाग्नि म॒ग्नि मे॒वाग्नये᳚ । \newline
14. ए॒वाग्नये॒ ऽग्नय॑ ए॒वै वाग्नये॒ मेधा॑य॒ मेधा॑ या॒ग्नय॑ ए॒वैवाग्नये॒ मेधा॑य । \newline
15. अ॒ग्नये॒ मेधा॑य॒ मेधा॑ या॒ग्नये॒ ऽग्नये॒ मेधा॒या मेधा॑या॒ ग्नये॒ ऽग्नये॒ मेधा॒या । \newline
16. मेधा॒या मेधा॑य॒ मेधा॒या ऽल॑भन्ता लभ॒न्ता मेधा॑य॒ मेधा॒या ऽल॑भन्त । \newline
17. आ ऽल॑भन्ता लभ॒न्ता ऽल॑भन्त॒ न नाल॑भ॒न्ता ऽल॑भन्त॒ न । \newline
18. अ॒ल॒भ॒न्त॒ न नाल॑भन्ता लभन्त॒ न हि हि नाल॑भन्ता लभन्त॒ न हि । \newline
19. न हि हि न न ह्य॑न्य द॒न्य द्धि न न ह्य॑न्यत् । \newline
20. ह्य॑न्य द॒न्य द्धि ह्य॑न्य दा॑लं॒भ्य॑ मालं॒भ्य॑ म॒न्यद्धि ह्य॑न्य दा॑लं॒भ्य᳚म् । \newline
21. अ॒न्य दा॑लं॒भ्य॑ मालं॒भ्य॑ म॒न्य द॒न्य दा॑लं॒भ्य॑ मवि॑न्द॒न् नवि॑न्दन् नालं॒भ्य॑ म॒न्य द॒न्य दा॑लं॒भ्य॑ मवि॑न्दन्न् । \newline
22. आ॒लं॒भ्य॑ मवि॑न्द॒न् नवि॑न्दन् नालं॒भ्य॑ मालं॒भ्य॑ मवि॑न्द॒न् तत॒ स्ततो ऽवि॑न्दन् नालं॒भ्य॑ मालं॒भ्य॑ मवि॑न्द॒न् ततः॑ । \newline
23. आ॒लं॒भ्य॑मित्या᳚ - लं॒भ्य᳚म् । \newline
24. अवि॑न्द॒न् तत॒ स्ततो ऽवि॑न्द॒न् नवि॑न्द॒न् ततो॒ वै वै ततो ऽवि॑न्द॒न् नवि॑न्द॒न् ततो॒ वै । \newline
25. ततो॒ वै वै तत॒ स्ततो॒ वा इ॒मा इ॒मा वै तत॒ स्ततो॒ वा इ॒माः । \newline
26. वा इ॒मा इ॒मा वै वा इ॒माः प्र॒जाः प्र॒जा इ॒मा वै वा इ॒माः प्र॒जाः । \newline
27. इ॒माः प्र॒जाः प्र॒जा इ॒मा इ॒माः प्र॒जाः प्र प्र प्र॒जा इ॒मा इ॒माः प्र॒जाः प्र । \newline
28. प्र॒जाः प्र प्र प्र॒जाः प्र॒जाः प्राजा॑यन्ता जायन्त॒ प्र प्र॒जाः प्र॒जाः प्राजा॑यन्त । \newline
29. प्र॒जा इति॑ प्र - जाः । \newline
30. प्राजा॑यन्ता जायन्त॒ प्र प्राजा॑यन्त॒ यद् यद॑जायन्त॒ प्र प्राजा॑यन्त॒ यत् । \newline
31. अ॒जा॒य॒न्त॒ यद् यद॑जायन्ता जायन्त॒ यद॒ग्ना व॒ग्नौ यद॑जायन्ता जायन्त॒ यद॒ग्नौ । \newline
32. यद॒ग्ना व॒ग्नौ यद् यद॒ग्ना व॒ग्नि म॒ग्नि म॒ग्नौ यद् यद॒ग्ना व॒ग्निम् । \newline
33. अ॒ग्ना व॒ग्नि म॒ग्नि म॒ग्ना व॒ग्ना व॒ग्निम् म॑थि॒त्वा म॑थि॒त्वा ऽग्नि म॒ग्ना व॒ग्ना व॒ग्निम् म॑थि॒त्वा । \newline
34. अ॒ग्निम् म॑थि॒त्वा म॑थि॒त्वा ऽग्नि म॒ग्निम् म॑थि॒त्वा प्र॒हर॑ति प्र॒हर॑ति मथि॒त्वा ऽग्नि म॒ग्निम् म॑थि॒त्वा प्र॒हर॑ति । \newline
35. म॒थि॒त्वा प्र॒हर॑ति प्र॒हर॑ति मथि॒त्वा म॑थि॒त्वा प्र॒हर॑ति प्र॒जाना᳚म् प्र॒जाना᳚म् प्र॒हर॑ति मथि॒त्वा म॑थि॒त्वा प्र॒हर॑ति प्र॒जाना᳚म् । \newline
36. प्र॒हर॑ति प्र॒जाना᳚म् प्र॒जाना᳚म् प्र॒हर॑ति प्र॒हर॑ति प्र॒जाना᳚म् प्र॒जन॑नाय प्र॒जन॑नाय प्र॒जाना᳚म् प्र॒हर॑ति प्र॒हर॑ति प्र॒जाना᳚म् प्र॒जन॑नाय । \newline
37. प्र॒हर॒तीति॑ प्र - हर॑ति । \newline
38. प्र॒जाना᳚म् प्र॒जन॑नाय प्र॒जन॑नाय प्र॒जाना᳚म् प्र॒जाना᳚म् प्र॒जन॑नाय रु॒द्रो रु॒द्रः प्र॒जन॑नाय प्र॒जाना᳚म् प्र॒जाना᳚म् प्र॒जन॑नाय रु॒द्रः । \newline
39. प्र॒जाना॒मिति॑ प्र - जाना᳚म् । \newline
40. प्र॒जन॑नाय रु॒द्रो रु॒द्रः प्र॒जन॑नाय प्र॒जन॑नाय रु॒द्रो वै वै रु॒द्रः प्र॒जन॑नाय प्र॒जन॑नाय रु॒द्रो वै । \newline
41. प्र॒जन॑ना॒येति॑ प्र - जान॑नाय । \newline
42. रु॒द्रो वै वै रु॒द्रो रु॒द्रो वा ए॒ष ए॒ष वै रु॒द्रो रु॒द्रो वा ए॒षः । \newline
43. वा ए॒ष ए॒ष वै वा ए॒ष यद् यदे॒ष वै वा ए॒ष यत् । \newline
44. ए॒ष यद् यदे॒ष ए॒ष यद॒ग्नि र॒ग्निर् यदे॒ष ए॒ष यद॒ग्निः । \newline
45. यद॒ग्नि र॒ग्निर् यद् यद॒ग्निर् यज॑मानो॒ यज॑मानो॒ ऽग्निर् यद् यद॒ग्निर् यज॑मानः । \newline
46. अ॒ग्निर् यज॑मानो॒ यज॑मानो॒ ऽग्नि र॒ग्निर् यज॑मानः प॒शुः प॒शुर् यज॑मानो॒ ऽग्नि र॒ग्निर् यज॑मानः प॒शुः । \newline
47. यज॑मानः प॒शुः प॒शुर् यज॑मानो॒ यज॑मानः प॒शुर् यद् यत् प॒शुर् यज॑मानो॒ यज॑मानः प॒शुर् यत् । \newline
48. प॒शुर् यद् यत् प॒शुः प॒शुर् यत् प॒शुम् प॒शुं ॅयत् प॒शुः प॒शुर् यत् प॒शुम् । \newline
49. यत् प॒शुम् प॒शुं ॅयद् यत् प॒शु मा॒लभ्या॒ लभ्य॑ प॒शुं ॅयद् यत् प॒शु मा॒लभ्य॑ । \newline
50. प॒शु मा॒लभ्या॒ लभ्य॑ प॒शुम् प॒शु मा॒लभ्या॒ग्नि म॒ग्नि मा॒लभ्य॑ प॒शुम् प॒शु मा॒लभ्या॒ग्निम् । \newline
51. आ॒लभ्या॒ग्नि म॒ग्नि मा॒लभ्या॒ लभ्या॒ग्निम् मन्थे॒न् मन्थे॑द॒ग्नि मा॒लभ्या॒ लभ्या॒ग्निम् मन्थे᳚त् । \newline
52. आ॒लभ्येत्या᳚ - लभ्य॑ । \newline
53. अ॒ग्निम् मन्थे॒न् मन्थे॑ द॒ग्नि म॒ग्निम् मन्थे᳚द् रु॒द्राय॑ रु॒द्राय॒ मन्थे॑ द॒ग्नि म॒ग्निम् मन्थे᳚द् रु॒द्राय॑ । \newline
54. मन्थे᳚द् रु॒द्राय॑ रु॒द्राय॒ मन्थे॒न् मन्थे᳚द् रु॒द्राय॒ यज॑मानं॒ ॅयज॑मानꣳ रु॒द्राय॒ मन्थे॒न् मन्थे᳚द् रु॒द्राय॒ यज॑मानम् । \newline
55. रु॒द्राय॒ यज॑मानं॒ ॅयज॑मानꣳ रु॒द्राय॑ रु॒द्राय॒ यज॑मान॒ मप्यपि॒ यज॑मानꣳ रु॒द्राय॑ रु॒द्राय॒ यज॑मान॒ मपि॑ । \newline
56. यज॑मान॒ मप्यपि॒ यज॑मानं॒ ॅयज॑मान॒ मपि॑ दद्ध्याद् दद्ध्या॒ दपि॒ यज॑मानं॒ ॅयज॑मान॒ मपि॑ दद्ध्यात् । \newline
\pagebreak
\markright{ TS 6.3.5.2  \hfill https://www.vedavms.in \hfill}

\section{ TS 6.3.5.2 }

\textbf{TS 6.3.5.2 } \newline
\textbf{Samhita Paata} \newline

-मपि॑ दद्ध्यात् प्र॒मायु॑कः स्या॒दथो॒ खल्वा॑हुर॒ग्निः सर्वा॑ दे॒वता॑ ह॒विरे॒तद्यत् प॒शुरिति॒ यत् प॒शुमा॒लभ्या॒ग्निं मन्थ॑ति ह॒व्यायै॒वाऽऽ*स॑न्नाय॒ सर्वा॑ दे॒वता॑ जनय-त्युपा॒कृत्यै॒व मन्थ्य॒-स्तन्नेवाऽऽ*ल॑ब्धं॒ नेवाना॑लब्ध-म॒ग्ने-र्ज॒नित्र॑-म॒सीत्या॑हा॒ग्नेर्ह्ये॑त-ज्ज॒नित्रं॒ ॅवृष॑णौ स्थ॒ इत्या॑ह॒ वृष॑णौ॒- [  ] \newline

\textbf{Pada Paata} \newline

अपीति॑ । द॒द्ध्या॒त् । प्र॒मायु॑क॒ इति॑ प्र-मायु॑कः । स्या॒त् । अथो॒ इति॑ । खलु॑ । आ॒हुः॒ । अ॒ग्निः । सर्वाः᳚ । दे॒वताः᳚ । ह॒विः । ए॒तत् । यत् । प॒शुः । इति॑ । यत् । प॒शुम् । आ॒लभ्येत्या᳚-लभ्य॑ । अ॒ग्निम् । मन्थ॑ति । ह॒व्याय॑ । ए॒व । आस॑न्ना॒येत्या-स॒न्ना॒य॒ । सर्वाः᳚ । दे॒वताः᳚ । ज॒न॒य॒ति॒ । उ॒पा॒कृत्येत्यु॑प - आ॒कृत्य॑ । ए॒व । मन्थ्यः॑ । तत् । न । इ॒व॒ । आल॑ब्ध॒मित्या - ल॒ब्ध॒म् । न । इ॒व॒ । आल॑ब्ध॒मित्यना᳚ - ल॒ब्ध॒म् । अ॒ग्नेः । ज॒नित्र᳚म् । अ॒सि॒ । इति॑ । आ॒ह॒ । अ॒ग्नेः । हि । ए॒तत् । ज॒नित्र᳚म् । वृष॑णौ । स्थः॒ । इति॑ । आ॒ह॒ । वृष॑णौ ।  \newline


\textbf{Krama Paata} \newline

अपि॑ दद्ध्यात् । द॒द्ध्या॒त् प्र॒मायु॑कः । प्र॒मायु॑कः स्यात् । प्र॒मायु॑क॒ इति॑ प्र - मायु॑कः । स्या॒दथो᳚ । अथो॒ खलु॑ । अथो॒ इत्यथो᳚ । खल्वा॑हुः । आ॒हु॒र॒ग्निः । अ॒ग्निः सर्वाः᳚ । सर्वा॑ दे॒वताः᳚ । दे॒वता॑ ह॒विः । ह॒विरे॒तत् । ए॒तद् यत् । यत् प॒शुः । प॒शुरिति॑ । इति॒ यत् । यत् प॒शुम् । प॒शुमा॒लभ्य॑ । आ॒लभ्या॒ग्निम् । आ॒लभ्येत्या᳚ - लभ्य॑ । अ॒ग्निम् मन्थ॑ति । मन्थ॑ति ह॒व्याय॑ । ह॒व्यायै॒व । ए॒वास॑न्नाय । आस॑न्नाय॒ सर्वाः᳚ । आस॑न्ना॒येत्या - स॒न्ना॒य॒ । सर्वा॑ दे॒वताः᳚ । दे॒वता॑ जनयति । ज॒न॒य॒त्यु॒पा॒कृत्य॑ । उ॒पा॒कृत्यै॒व । उ॒पा॒कृत्येत्यु॑प - आ॒कृत्य॑ । ए॒व मन्थ्यः॑ । मन्थ्य॒स्तत् । तन् न । नेव॑ । इ॒वाल॑ब्धम् । आल॑ब्ध॒म् न । आल॑ब्ध॒मित्या - ल॒ब्ध॒म् । नेव॑ । इ॒वाना॑लब्धम् । अना॑लब्धम॒ग्नेः । अना॑लब्ध॒मित्यना᳚ - ल॒ब्ध॒म् । अ॒ग्नेर् ज॒नित्र᳚म् । ज॒नित्र॑मसि । अ॒सीति॑ । इत्या॑ह । आ॒हा॒ग्नेः । अ॒ग्नेर्. हि । ह्ये॑तत् । ए॒तज् ज॒नित्र᳚म् । ज॒नित्र॒म् ॅवृष॑णौ । वृष॑णौ स्थः । स्थ॒ इति॑ । इत्या॑ह । आ॒ह॒ वृष॑णौ । वृष॑णौ॒ हि \newline

\textbf{Jatai Paata} \newline

1. अपि॑ दद्ध्याद् दद्ध्या॒ दप्यपि॑ दद्ध्यात् । \newline
2. द॒द्ध्या॒त् प्र॒मायु॑कः प्र॒मायु॑को दद्ध्याद् दद्ध्यात् प्र॒मायु॑कः । \newline
3. प्र॒मायु॑कः स्याथ् स्यात् प्र॒मायु॑कः प्र॒मायु॑कः स्यात् । \newline
4. प्र॒मायु॑क॒ इति॑ प्र - मायु॑कः । \newline
5. स्या॒ दथो॒ अथो᳚ स्याथ् स्या॒ दथो᳚ । \newline
6. अथो॒ खलु॒ खल्वथो॒ अथो॒ खलु॑ । \newline
7. अथो॒ इत्यथो᳚ । \newline
8. खल्वा॑हु राहुः॒ खलु॒ खल्वा॑हुः । \newline
9. आ॒हु॒ र॒ग्नि र॒ग्नि रा॑हु राहु र॒ग्निः । \newline
10. अ॒ग्निः सर्वाः॒ सर्वा॑ अ॒ग्नि र॒ग्निः सर्वाः᳚ । \newline
11. सर्वा॑ दे॒वता॑ दे॒वताः॒ सर्वाः॒ सर्वा॑ दे॒वताः᳚ । \newline
12. दे॒वता॑ ह॒विर्. ह॒विर् दे॒वता॑ दे॒वता॑ ह॒विः । \newline
13. ह॒वि रे॒त दे॒त द्ध॒विर्. ह॒वि रे॒तत् । \newline
14. ए॒तद् यद् यदे॒त दे॒तद् यत् । \newline
15. यत् प॒शुः प॒शुर् यद् यत् प॒शुः । \newline
16. प॒शुरि तीति॑ प॒शुः प॒शु रिति॑ । \newline
17. इति॒ यद् यदि तीति॒ यत् । \newline
18. यत् प॒शुम् प॒शुं ॅयद् यत् प॒शुम् । \newline
19. प॒शु मा॒लभ्या॒ लभ्य॑ प॒शुम् प॒शु मा॒लभ्य॑ । \newline
20. आ॒लभ्या॒ग्नि म॒ग्नि मा॒लभ्या॒ लभ्या॒ग्निम् । \newline
21. आ॒लभ्येत्या᳚ - लभ्य॑ । \newline
22. अ॒ग्निम् मन्थ॑ति॒ मन्थ॑ त्य॒ग्नि म॒ग्निम् मन्थ॑ति । \newline
23. मन्थ॑ति ह॒व्याय॑ ह॒व्याय॒ मन्थ॑ति॒ मन्थ॑ति ह॒व्याय॑ । \newline
24. ह॒व्यायै॒ वैव ह॒व्याय॑ ह॒व्यायै॒व । \newline
25. ए॒वा स॑न्ना॒या स॑न्ना यै॒वैवा स॑न्नाय । \newline
26. आस॑न्नाय॒ सर्वाः॒ सर्वा॒ आस॑न्ना॒या स॑न्नाय॒ सर्वाः᳚ । \newline
27. आस॑न्ना॒येत्या - स॒न्ना॒य॒ । \newline
28. सर्वा॑ दे॒वता॑ दे॒वताः॒ सर्वाः॒ सर्वा॑ दे॒वताः᳚ । \newline
29. दे॒वता॑ जनयति जनयति दे॒वता॑ दे॒वता॑ जनयति । \newline
30. ज॒न॒य॒ त्यु॒पा॒कृ त्यो॑पा॒कृत्य॑ जनयति जनय त्युपा॒कृत्य॑ । \newline
31. उ॒पा॒कृत्यै॒ वैवोपा॒कृ त्यो॑पा॒कृत्यै॒व । \newline
32. उ॒पा॒कृत्येत्यु॑प - आ॒कृत्य॑ । \newline
33. ए॒व मन्थ्यो॒ मन्थ्य॑ ए॒वैव मन्थ्यः॑ । \newline
34. मन्थ्य॒ स्तत् तन् मन्थ्यो॒ मन्थ्य॒ स्तत् । \newline
35. तन् न न तत् तन् न । \newline
36. नेवे॑व॒ न नेव॑ । \newline
37. इ॒वाल॑ब्ध॒ माल॑ब्ध मिवे॒ वाल॑ब्धम् । \newline
38. आल॑ब्ध॒न् न नाल॑ब्ध॒ माल॑ब्ध॒न् न । \newline
39. आल॑ब्ध॒मित्या - ल॒ब्ध॒म् । \newline
40. नेवे॑व॒ न नेव॑ । \newline
41. इ॒वाना॑लब्ध॒ मना॑लब्ध मिवे॒ वाना॑लब्धम् । \newline
42. अना॑लब्ध म॒ग्ने र॒ग्ने रना॑लब्ध॒ मना॑लब्ध म॒ग्नेः । \newline
43. अना॑लब्ध॒मित्यना᳚ - ल॒ब्ध॒म् । \newline
44. अ॒ग्नेर् ज॒नित्र॑म् ज॒नित्र॑ म॒ग्ने र॒ग्नेर् ज॒नित्र᳚म् । \newline
45. ज॒नित्र॑ मस्यसि ज॒नित्र॑म् ज॒नित्र॑ मसि । \newline
46. अ॒सी तीत्य॑ स्य॒सीति॑ । \newline
47. इत्या॑हा॒हे तीत्या॑ह । \newline
48. आ॒हा॒ग्ने र॒ग्ने रा॑हा हा॒ग्नेः । \newline
49. अ॒ग्नेर्. हि ह्य॑ग्ने र॒ग्नेर्. हि । \newline
50. ह्ये॑त दे॒तद्धि ह्ये॑तत् । \newline
51. ए॒तज् ज॒नित्र॑म् ज॒नित्र॑ मे॒त दे॒तज् ज॒नित्र᳚म् । \newline
52. ज॒नित्रं॒ ॅवृष॑णौ॒ वृष॑णौ ज॒नित्र॑म् ज॒नित्रं॒ ॅवृष॑णौ । \newline
53. वृष॑णौ स्थः स्थो॒ वृष॑णौ॒ वृष॑णौ स्थः । \newline
54. स्थ॒ इतीति॑ स्थः स्थ॒ इति॑ । \newline
55. इत्या॑हा॒हे तीत्या॑ह । \newline
56. आ॒ह॒ वृष॑णौ॒ वृष॑णा वाहाह॒ वृष॑णौ । \newline
57. वृष॑णौ॒ हि हि वृष॑णौ॒ वृष॑णौ॒ हि । \newline

\textbf{Ghana Paata } \newline

1. अपि॑ दद्ध्याद् दद्ध्या॒ दप्यपि॑ दद्ध्यात् प्र॒मायु॑कः प्र॒मायु॑को दद्ध्या॒ दप्यपि॑ दद्ध्यात् प्र॒मायु॑कः । \newline
2. द॒द्ध्या॒त् प्र॒मायु॑कः प्र॒मायु॑को दद्ध्याद् दद्ध्यात् प्र॒मायु॑कः स्याथ् स्यात् प्र॒मायु॑को दद्ध्याद् दद्ध्यात् प्र॒मायु॑कः स्यात् । \newline
3. प्र॒मायु॑कः स्याथ् स्यात् प्र॒मायु॑कः प्र॒मायु॑कः स्या॒ दथो॒ अथो᳚ स्यात् प्र॒मायु॑कः प्र॒मायु॑कः स्या॒दथो᳚ । \newline
4. प्र॒मायु॑क॒ इति॑ प्र - मायु॑कः । \newline
5. स्या॒ दथो॒ अथो᳚ स्याथ् स्या॒ दथो॒ खलु॒ खल्वथो᳚ स्याथ् स्या॒ दथो॒ खलु॑ । \newline
6. अथो॒ खलु॒ खल्वथो॒ अथो॒ खल्वा॑हु राहुः॒ खल्वथो॒ अथो॒ खल्वा॑हुः । \newline
7. अथो॒ इत्यथो᳚ । \newline
8. खल्वा॑हु राहुः॒ खलु॒ खल्वा॑हु र॒ग्नि र॒ग्नि रा॑हुः॒ खलु॒ खल्वा॑हु र॒ग्निः । \newline
9. आ॒हु॒ र॒ग्नि र॒ग्नि रा॑हु राहु र॒ग्निः सर्वाः॒ सर्वा॑ अ॒ग्नि रा॑हु राहु र॒ग्निः सर्वाः᳚ । \newline
10. अ॒ग्निः सर्वाः॒ सर्वा॑ अ॒ग्नि र॒ग्निः सर्वा॑ दे॒वता॑ दे॒वताः॒ सर्वा॑ अ॒ग्नि र॒ग्निः सर्वा॑ दे॒वताः᳚ । \newline
11. सर्वा॑ दे॒वता॑ दे॒वताः॒ सर्वाः॒ सर्वा॑ दे॒वता॑ ह॒विर्. ह॒विर् दे॒वताः॒ सर्वाः॒ सर्वा॑ दे॒वता॑ ह॒विः । \newline
12. दे॒वता॑ ह॒विर्. ह॒विर् दे॒वता॑ दे॒वता॑ ह॒वि रे॒त दे॒त द्ध॒विर् दे॒वता॑ दे॒वता॑ ह॒वि रे॒तत् । \newline
13. ह॒वि रे॒त दे॒त द्ध॒विर्. ह॒वि रे॒तद् यद् यदे॒त द्ध॒विर्. ह॒वि रे॒तद् यत् । \newline
14. ए॒तद् यद् यदे॒त दे॒तद् यत् प॒शुः प॒शुर् यदे॒त दे॒तद् यत् प॒शुः । \newline
15. यत् प॒शुः प॒शुर् यद् यत् प॒शुरि तीति॑ प॒शुर् यद् यत् प॒शु रिति॑ । \newline
16. प॒शुरि तीति॑ प॒शुः प॒शु रिति॒ यद् यदिति॑ प॒शुः प॒शु रिति॒ यत् । \newline
17. इति॒ यद् यदितीति॒ यत् प॒शुम् प॒शुं ॅयदितीति॒ यत् प॒शुम् । \newline
18. यत् प॒शुम् प॒शुं ॅयद् यत् प॒शु मा॒लभ्या॒ लभ्य॑ प॒शुं ॅयद् यत् प॒शु मा॒लभ्य॑ । \newline
19. प॒शु मा॒लभ्या॒ लभ्य॑ प॒शुम् प॒शु मा॒लभ्या॒ग्नि म॒ग्नि मा॒लभ्य॑ प॒शुम् प॒शु मा॒लभ्या॒ग्निम् । \newline
20. आ॒लभ्या॒ग्नि म॒ग्नि मा॒लभ्या॒ लभ्या॒ग्निम् मन्थ॑ति॒ मन्थ॑ त्य॒ग्नि मा॒लभ्या॒ लभ्या॒ग्निम् मन्थ॑ति । \newline
21. आ॒लभ्येत्या᳚ - लभ्य॑ । \newline
22. अ॒ग्निम् मन्थ॑ति॒ मन्थ॑ त्य॒ग्नि म॒ग्निम् मन्थ॑ति ह॒व्याय॑ ह॒व्याय॒ मन्थ॑ त्य॒ग्नि म॒ग्निम् मन्थ॑ति ह॒व्याय॑ । \newline
23. मन्थ॑ति ह॒व्याय॑ ह॒व्याय॒ मन्थ॑ति॒ मन्थ॑ति ह॒व्यायै॒ वैव ह॒व्याय॒ मन्थ॑ति॒ मन्थ॑ति ह॒व्यायै॒व । \newline
24. ह॒व्यायै॒ वैव ह॒व्याय॑ ह॒व्यायै॒वा स॑न्ना॒या स॑न्नायै॒व ह॒व्याय॑ ह॒व्या यै॒वा स॑न्नाय । \newline
25. ए॒वा स॑न्ना॒या स॑न्ना यै॒वै वास॑न्नाय॒ सर्वाः॒ सर्वा॒ आस॑न्ना यै॒वै वास॑न्नाय॒ सर्वाः᳚ । \newline
26. आस॑न्नाय॒ सर्वाः॒ सर्वा॒ आस॑न्ना॒या स॑न्नाय॒ सर्वा॑ दे॒वता॑ दे॒वताः॒ सर्वा॒ आस॑न्ना॒या स॑न्नाय॒ सर्वा॑ दे॒वताः᳚ । \newline
27. आस॑न्ना॒येत्या - स॒न्ना॒य॒ । \newline
28. सर्वा॑ दे॒वता॑ दे॒वताः॒ सर्वाः॒ सर्वा॑ दे॒वता॑ जनयति जनयति दे॒वताः॒ सर्वाः॒ सर्वा॑ दे॒वता॑ जनयति । \newline
29. दे॒वता॑ जनयति जनयति दे॒वता॑ दे॒वता॑ जनय त्युपा॒कृ त्यो॑पा॒कृत्य॑ जनयति दे॒वता॑ दे॒वता॑ जनय त्युपा॒कृत्य॑ । \newline
30. ज॒न॒य॒ त्यु॒पा॒कृत्यो॑ पा॒कृत्य॑ जनयति जनय त्युपा॒कृ त्यै॒वैवो पा॒कृत्य॑ जनयति जनय त्युपा॒कृत्यै॒व । \newline
31. उ॒पा॒कृ त्यै॒वै वोपा॒कृ त्यो॑पा॒कृ त्यै॒व मन्थ्यो॒ मन्थ्य॑ ए॒वोपा॒कृ त्यो॑पा॒कृ त्यै॒व मन्थ्यः॑ । \newline
32. उ॒पा॒कृत्येत्यु॑प - आ॒कृत्य॑ । \newline
33. ए॒व मन्थ्यो॒ मन्थ्य॑ ए॒वैव मन्थ्य॒ स्तत् तन् मन्थ्य॑ ए॒वैव मन्थ्य॒ स्तत् । \newline
34. मन्थ्य॒ स्तत् तन् मन्थ्यो॒ मन्थ्य॒ स्तन् न न तन् मन्थ्यो॒ मन्थ्य॒ स्तन् न । \newline
35. तन् न न तत् तन् नेवे॑व॒ न तत् तन् नेव॑ । \newline
36. नेवे॑व॒ न नेवाल॑ब्ध॒ माल॑ब्ध मिव॒ न नेवाल॑ब्धम् । \newline
37. इ॒वाल॑ब्ध॒ माल॑ब्ध मिवे॒ वाल॑ब्ध॒न्न नाल॑ब्ध मिवे॒ वाल॑ब्ध॒न्न । \newline
38. आल॑ब्ध॒न्न नाल॑ब्ध॒ माल॑ब्ध॒न्ने वे॑ व॒ नाल॑ब्ध॒ माल॑ब्ध॒न्ने व॑ । \newline
39. आल॑ब्ध॒मित्या - ल॒ब्ध॒म् । \newline
40. नेवे॑व॒ न ने वाना॑लब्ध॒ मना॑लब्ध मिव॒ न नेवाना॑लब्धम् । \newline
41. इ॒वा ना॑लब्ध॒ मना॑लब्ध मिवे॒ वाना॑लब्ध म॒ग्ने र॒ग्ने रना॑लब्ध मिवे॒ वाना॑लब्ध म॒ग्नेः । \newline
42. अना॑लब्ध म॒ग्ने र॒ग्ने रना॑लब्ध॒ मना॑लब्ध म॒ग्नेर् ज॒नित्र॑म् ज॒नित्र॑ म॒ग्ने रना॑लब्ध॒ मना॑लब्ध म॒ग्नेर् ज॒नित्र᳚म् । \newline
43. अना॑लब्ध॒मित्यना᳚ - ल॒ब्ध॒म् । \newline
44. अ॒ग्नेर् ज॒नित्र॑म् ज॒नित्र॑ म॒ग्ने र॒ग्नेर् ज॒नित्र॑ मस्यसि ज॒नित्र॑ म॒ग्ने र॒ग्नेर् ज॒नित्र॑ मसि । \newline
45. ज॒नित्र॑ मस्यसि ज॒नित्र॑म् ज॒नित्र॑ म॒सीती त्य॑सि ज॒नित्र॑म् ज॒नित्र॑ म॒सीति॑ । \newline
46. अ॒सीती त्य॑स्य॒ सीत्या॑ हा॒हे त्य॑स्य॒ सीत्या॑ह । \newline
47. इत्या॑हा॒हे तीत्या॑हा॒ग्ने र॒ग्ने रा॒हे तीत्या॑ हा॒ग्नेः । \newline
48. आ॒हा॒ग्ने र॒ग्ने रा॑हा हा॒ग्नेर्. हि ह्य॑ग्ने रा॑हा हा॒ग्नेर्. हि । \newline
49. अ॒ग्नेर्. हि ह्य॑ग्ने र॒ग्नेर् ह्ये॑त दे॒तद्ध्य॑ ग्ने र॒ग्नेर् ह्ये॑तत् । \newline
50. ह्ये॑त दे॒त द्धि ह्ये॑तज् ज॒नित्र॑म् ज॒नित्र॑ मे॒त द्धि ह्ये॑तज् ज॒नित्र᳚म् । \newline
51. ए॒तज् ज॒नित्र॑म् ज॒नित्र॑ मे॒त दे॒तज् ज॒नित्रं॒ ॅवृष॑णौ॒ वृष॑णौ ज॒नित्र॑ मे॒त दे॒तज् ज॒नित्रं॒ ॅवृष॑णौ । \newline
52. ज॒नित्रं॒ ॅवृष॑णौ॒ वृष॑णौ ज॒नित्र॑म् ज॒नित्रं॒ ॅवृष॑णौ स्थः स्थो॒ वृष॑णौ ज॒नित्र॑म् ज॒नित्रं॒ ॅवृष॑णौ स्थः । \newline
53. वृष॑णौ स्थः स्थो॒ वृष॑णौ॒ वृष॑णौ स्थ॒ इतीति॑ स्थो॒ वृष॑णौ॒ वृष॑णौ स्थ॒ इति॑ । \newline
54. स्थ॒ इतीति॑ स्थः स्थ॒ इत्या॑हा॒ हेति॑ स्थः स्थ॒ इत्या॑ह । \newline
55. इत्या॑हा॒हे तीत्या॑ह॒ वृष॑णौ॒ वृष॑णा वा॒हे तीत्या॑ह॒ वृष॑णौ । \newline
56. आ॒ह॒ वृष॑णौ॒ वृष॑णा वाहाह॒ वृष॑णौ॒ हि हि वृष॑णा वाहाह॒ वृष॑णौ॒ हि । \newline
57. वृष॑णौ॒ हि हि वृष॑णौ॒ वृष॑णौ॒ ह्ये॑ता वे॒तौ हि वृष॑णौ॒ वृष॑णौ॒ ह्ये॑तौ । \newline
\pagebreak
\markright{ TS 6.3.5.3  \hfill https://www.vedavms.in \hfill}

\section{ TS 6.3.5.3 }

\textbf{TS 6.3.5.3 } \newline
\textbf{Samhita Paata} \newline

ह्ये॑ता-वु॒र्वश्य॑स्या॒यु-र॒सीत्या॑ह मिथुन॒त्वाय॑ घृ॒तेना॒क्ते वृष॑णं दधाथा॒मित्या॑ह॒ वृष॑णꣳ॒॒ ह्ये॑ते दधा॑ते॒ ये अ॒ग्निं गा॑य॒त्रं छन्दोऽनु॒ प्र जा॑य॒स्वेत्या॑ह॒ छन्दो॑भिरे॒वैनं॒ प्र ज॑नयत्य॒ग्नये॑ म॒थ्यमा॑ना॒यानु॑ ब्रू॒हीत्या॑ह सावि॒त्रीमृच॒मन्वा॑ह सवि॒तृप्र॑सूत ए॒वैनं॑ मन्थति जा॒तायानु॑ ब्रूहि- [  ] \newline

\textbf{Pada Paata} \newline

हि । ए॒तौ । उ॒र्वशी᳚ । अ॒सि॒ । आ॒युः । अ॒सि॒ । इति॑ । आ॒ह॒ । मि॒थु॒न॒त्वायेति॑ मिथुन - त्वाय॑ । घृ॒तेन॑ । अ॒क्ते इति॑ । वृष॑णम् । द॒धा॒था॒म् । इति॑ । आ॒ह॒ । वृष॑णम् । हि । ए॒ते इति॑ । दधा॑ते॒ इति॑ । ये इति॑ । अ॒ग्निम् । गा॒य॒त्रम् । छन्दः॑ । अनु॑ । प्रेति॑ । जा॒य॒स्व॒ । इति॑ । आ॒ह॒ । छन्दो॑भि॒रिति॒ छन्दः॑ - भिः॒ । ए॒व । ए॒न॒म् । प्रेति॑ । ज॒न॒य॒ति॒ । अ॒ग्नये᳚ । म॒थ्यमा॑नाय । अन्विति॑ । ब्रू॒हि॒ । इति॑ । आ॒ह॒ । सा॒वि॒त्रीम् । ऋच᳚म् । अन्विति॑ । आ॒ह॒ । स॒वि॒तृप्र॑सूत॒ इति॑ सवि॒तृ-प्र॒सू॒तः॒ । ए॒व । ए॒न॒म् । म॒न्थ॒ति॒ । जा॒ताय॑ । अन्विति॑ । ब्रू॒हि॒ ।  \newline


\textbf{Krama Paata} \newline

ह्ये॑तौ । ए॒तावु॒र्वशी᳚ । उ॒र्वश्य॑सि । अ॒स्या॒युः । आ॒युर॑सि । अ॒सीति॑ । इत्या॑ह । आ॒ह॒ मि॒थु॒न॒त्वाय॑ । मि॒थु॒न॒त्वाय॑ घृ॒तेन॑ । मि॒थु॒न॒त्वायेति॑ मिथुन - त्वाय॑ । घृ॒तेना॒क्ते । अ॒क्ते वृष॑णम् । अ॒क्ते इत्य॒क्ते । वृष॑णम् दधाथाम् । द॒धा॒था॒मिति॑ । इत्या॑ह । आ॒ह॒ वृष॑णम् । वृष॑णꣳ॒॒ हि । ह्ये॑ते । ए॒ते दधा॑ते । ए॒ते इत्ये॒ते । दधा॑ते॒ ये । दधा॑ते॒ इति॒ दधा॑ते । ये अ॒ग्निम् । ये इति॒ ये । अ॒ग्निम् गा॑य॒त्रम् । गा॒य॒त्रम् छन्दः॑ । छन्दोऽनु॑ । अनु॒ प्र । प्र जा॑यस्व । जा॒य॒स्वेति॑ । इत्या॑ह । आ॒ह॒ छन्दो॑भिः । छन्दो॑भिरे॒व । छन्दो॑भि॒रिति॒ छन्दः॑ - भिः॒ । ए॒वैन᳚म् । ए॒न॒म् प्र । प्र ज॑नयति । ज॒न॒य॒त्य॒ग्नये᳚ । अ॒ग्नये॑ म॒थ्यमा॑नाय । म॒थ्यमा॑ना॒यानु॑ । अनु॑ ब्रूहि । ब्रू॒हीति॑ । इत्या॑ह । आ॒ह॒ सा॒वि॒त्रीम् । सा॒वि॒त्रीमृच᳚म् । ऋच॒मनु॑ । अन्वा॑ह । आ॒ह॒ स॒वि॒तृप्र॑सूतः । स॒वि॒तृप्र॑सूत ए॒व । स॒वि॒तृप्र॑सूत॒ इति॑ सवि॒तृ - प्र॒सू॒तः॒ । ए॒वैन᳚म् । ए॒न॒म् म॒न्थ॒ति॒ । म॒न्थ॒ति॒ जा॒ताय॑ । जा॒तायानु॑ । अनु॑ ब्रूहि । ब्रू॒हि॒ प्र॒ह्रि॒यमा॑णाय \newline

\textbf{Jatai Paata} \newline

1. ह्ये॑ता वे॒तौ हि ह्ये॑तौ । \newline
2. ए॒ता वु॒र्व श्यु॒र्व श्ये॒ता वे॒ता वु॒र्वशी᳚ । \newline
3. उ॒र्वश्य॑ स्य स्यु॒र्व श्यु॒र्व श्य॑सि । \newline
4. अ॒स्या॒यु रा॒यु र॑स्य स्या॒युः । \newline
5. आ॒यु र॑स्य स्या॒यु रा॒यु र॑सि । \newline
6. अ॒सी तीत्य॑ स्य॒सीति॑ । \newline
7. इत्या॑हा॒हे तीत्या॑ह । \newline
8. आ॒ह॒ मि॒थु॒न॒त्वाय॑ मिथुन॒त्वाया॑ हाह मिथुन॒त्वाय॑ । \newline
9. मि॒थु॒न॒त्वाय॑ घृ॒तेन॑ घृ॒तेन॑ मिथुन॒त्वाय॑ मिथुन॒त्वाय॑ घृ॒तेन॑ । \newline
10. मि॒थु॒न॒त्वायेति॑ मिथुन - त्वाय॑ । \newline
11. घृ॒तेना॒क्ते अ॒क्ते घृ॒तेन॑ घृ॒तेना॒क्ते । \newline
12. अ॒क्ते वृष॑णं॒ ॅवृष॑ण म॒क्ते अ॒क्ते वृष॑णम् । \newline
13. अ॒क्ते इत्य॒क्ते । \newline
14. वृष॑णम् दधाथाम् दधाथां॒ ॅवृष॑णं॒ ॅवृष॑णम् दधाथाम् । \newline
15. द॒धा॒था॒ मितीति॑ दधाथाम् दधाथा॒ मिति॑ । \newline
16. इत्या॑हा॒हे तीत्या॑ह । \newline
17. आ॒ह॒ वृष॑णं॒ ॅवृष॑ण माहाह॒ वृष॑णम् । \newline
18. वृष॑णꣳ॒॒ हि हि वृष॑णं॒ ॅवृष॑णꣳ॒॒ हि । \newline
19. ह्ये॑ते ए॒ते हि ह्ये॑ते । \newline
20. ए॒ते दधा॑ते॒ दधा॑ते ए॒ते ए॒ते दधा॑ते । \newline
21. ए॒ते इत्ये॒ते । \newline
22. दधा॑ते॒ ये ये दधा॑ते॒ दधा॑ते॒ ये । \newline
23. दधा॑ते॒ इति॒ दधा॑ते । \newline
24. ये अ॒ग्नि म॒ग्निं ॅये ये अ॒ग्निम् । \newline
25. ये इति॒ ये । \newline
26. अ॒ग्निम् गा॑य॒त्रम् गा॑य॒त्र म॒ग्नि म॒ग्निम् गा॑य॒त्रम् । \newline
27. गा॒य॒त्रम् छन्द॒ श्छन्दो॑ गाय॒त्रम् गा॑य॒त्रम् छन्दः॑ । \newline
28. छन्दो ऽन्वनु॒ छन्द॒ श्छन्दो ऽनु॑ । \newline
29. अनु॒ प्र प्राण्वनु॒ प्र । \newline
30. प्र जा॑यस्व जायस्व॒ प्र प्र जा॑यस्व । \newline
31. जा॒य॒स्वे तीति॑ जायस्व जाय॒स्वेति॑ । \newline
32. इत्या॑हा॒हे तीत्या॑ह । \newline
33. आ॒ह॒ छन्दो॑भि॒ श्छन्दो॑भि राहाह॒ छन्दो॑भिः । \newline
34. छन्दो॑भि रे॒वैव छन्दो॑भि॒ श्छन्दो॑भि रे॒व । \newline
35. छन्दो॑भि॒रिति॒ छन्दः॑ - भिः॒ । \newline
36. ए॒वैन॑ मेन मे॒वै वैन᳚म् । \newline
37. ए॒न॒म् प्र प्रैन॑ मेन॒म् प्र । \newline
38. प्र ज॑नयति जनयति॒ प्र प्र ज॑नयति । \newline
39. ज॒न॒य॒ त्य॒ग्नये॒ ऽग्नये॑ जनयति जनय त्य॒ग्नये᳚ । \newline
40. अ॒ग्नये॑ म॒थ्यमा॑नाय म॒थ्यमा॑ना या॒ग्नये॒ ऽग्नये॑ म॒थ्यमा॑नाय । \newline
41. म॒थ्यमा॑ना॒ यान्वनु॑ म॒थ्यमा॑नाय म॒थ्यमा॑ना॒ यानु॑ । \newline
42. अनु॑ ब्रूहि ब्रू॒ह्यन्वनु॑ ब्रूहि । \newline
43. ब्रू॒ही तीति॑ ब्रूहि ब्रू॒हीति॑ । \newline
44. इत्या॑हा॒हे तीत्या॑ह । \newline
45. आ॒ह॒ सा॒वि॒त्रीꣳ सा॑वि॒त्री मा॑हाह सावि॒त्रीम् । \newline
46. सा॒वि॒त्री मृच॒ मृचꣳ॑ सावि॒त्रीꣳ सा॑वि॒त्री मृच᳚म् । \newline
47. ऋच॒ मन्वन् वृच॒ मृच॒ मनु॑ । \newline
48. अन्वा॑ हा॒हान् वन् वा॑ह । \newline
49. आ॒ह॒ स॒वि॒तृप्र॑सूतः सवि॒तृप्र॑सूत आहाह सवि॒तृप्र॑सूतः । \newline
50. स॒वि॒तृप्र॑सूत ए॒वैव स॑वि॒तृप्र॑सूतः सवि॒तृप्र॑सूत ए॒व । \newline
51. स॒वि॒तृप्र॑सूत॒ इति॑ सवि॒तृ - प्र॒सू॒तः॒ । \newline
52. ए॒वैन॑ मेन मे॒वै वैन᳚म् । \newline
53. ए॒न॒म् म॒न्थ॒ति॒ म॒न्थ॒ त्ये॒न॒ मे॒न॒म् म॒न्थ॒ति॒ । \newline
54. म॒न्थ॒ति॒ जा॒ताय॑ जा॒ताय॑ मन्थति मन्थति जा॒ताय॑ । \newline
55. जा॒ताया न्वनु॑ जा॒ताय॑ जा॒ता यानु॑ । \newline
56. अनु॑ ब्रूहि ब्रू॒ह्य न्वनु॑ ब्रूहि । \newline
57. ब्रू॒हि॒ प्र॒ह्रि॒यमा॑णाय प्रह्रि॒यमा॑णाय ब्रूहि ब्रूहि प्रह्रि॒यमा॑णाय । \newline

\textbf{Ghana Paata } \newline

1. ह्ये॑ता वे॒तौ हि ह्ये॑ता वु॒र्व श्यु॒र्वश्ये॒तौ हि ह्ये॑ता वु॒र्वशी᳚ । \newline
2. ए॒ता वु॒र्व श्यु॒र्व श्ये॒ता वे॒ता वु॒र्वश्य॑ स्य स्यु॒र्व श्ये॒ता वे॒ता वु॒र्वश्य॑सि । \newline
3. उ॒र्वश्य॑ स्य स्यु॒र्व श्यु॒र्व श्य॑स्या॒यु रा॒यु र॑स्यु॒र्व श्यु॒र्व श्य॑स्या॒युः । \newline
4. अ॒स्या॒यु रा॒यु र॑स्य स्या॒यु र॑स्य स्या॒यु र॑स्य स्या॒युर॑सि । \newline
5. आ॒यु र॑स्य स्या॒यु रा॒यु र॒सीती त्य॑स्या॒यु रा॒यु र॒सीति॑ । \newline
6. अ॒सीती त्य॑स्य॒ सीत्या॑हा॒हे त्य॑स्य॒ सीत्या॑ह । \newline
7. इत्या॑हा॒हे तीत्या॑ह मिथुन॒त्वाय॑ मिथुन॒त्वाया॒हे तीत्या॑ह मिथुन॒त्वाय॑ । \newline
8. आ॒ह॒ मि॒थु॒न॒त्वाय॑ मिथुन॒त्वाया॑ हाह मिथुन॒त्वाय॑ घृ॒तेन॑ घृ॒तेन॑ मिथुन॒त्वाया॑ हाह मिथुन॒त्वाय॑ घृ॒तेन॑ । \newline
9. मि॒थु॒न॒त्वाय॑ घृ॒तेन॑ घृ॒तेन॑ मिथुन॒त्वाय॑ मिथुन॒त्वाय॑ घृ॒तेना॒क्ते अ॒क्ते घृ॒तेन॑ मिथुन॒त्वाय॑ मिथुन॒त्वाय॑ घृ॒तेना॒क्ते । \newline
10. मि॒थु॒न॒त्वायेति॑ मिथुन - त्वाय॑ । \newline
11. घृ॒तेना॒क्ते अ॒क्ते घृ॒तेन॑ घृ॒तेना॒क्ते वृष॑णं॒ ॅवृष॑ण म॒क्ते घृ॒तेन॑ घृ॒तेना॒क्ते वृष॑णम् । \newline
12. अ॒क्ते वृष॑णं॒ ॅवृष॑ण म॒क्ते अ॒क्ते वृष॑णम् दधाथाम् दधाथां॒ ॅवृष॑ण म॒क्ते अ॒क्ते वृष॑णम् दधाथाम् । \newline
13. अ॒क्ते इत्य॒क्ते । \newline
14. वृष॑णम् दधाथाम् दधाथां॒ ॅवृष॑णं॒ ॅवृष॑णम् दधाथा॒ मितीति॑ दधाथां॒ ॅवृष॑णं॒ ॅवृष॑णम् दधाथा॒ मिति॑ । \newline
15. द॒धा॒था॒ मितीति॑ दधाथाम् दधाथा॒ मित्या॑हा॒ हेति॑ दधाथाम् दधाथा॒ मित्या॑ह । \newline
16. इत्या॑हा॒हे तीत्या॑ह॒ वृष॑णं॒ ॅवृष॑ण मा॒हे तीत्या॑ह॒ वृष॑णम् । \newline
17. आ॒ह॒ वृष॑णं॒ ॅवृष॑ण माहाह॒ वृष॑णꣳ॒॒ हि हि वृष॑ण माहाह॒ वृष॑णꣳ॒॒ हि । \newline
18. वृष॑णꣳ॒॒ हि हि वृष॑णं॒ ॅवृष॑णꣳ॒॒ ह्ये॑ते ए॒ते हि वृष॑णं॒ ॅवृष॑णꣳ॒॒ ह्ये॑ते । \newline
19. ह्ये॑ते ए॒ते हि ह्ये॑ते दधा॑ते॒ दधा॑ते ए॒ते हि ह्ये॑ते दधा॑ते । \newline
20. ए॒ते दधा॑ते॒ दधा॑ते ए॒ते ए॒ते दधा॑ते॒ ये ये दधा॑ते ए॒ते ए॒ते दधा॑ते॒ ये । \newline
21. ए॒ते इत्ये॒ते । \newline
22. दधा॑ते॒ ये ये दधा॑ते॒ दधा॑ते॒ ये अ॒ग्नि म॒ग्निं ॅये दधा॑ते॒ दधा॑ते॒ ये अ॒ग्निम् । \newline
23. दधा॑ते॒ इति॒ दधा॑ते । \newline
24. ये अ॒ग्नि म॒ग्निं ॅये ये अ॒ग्निम् गा॑य॒त्रम् गा॑य॒त्र म॒ग्निं ॅये ये अ॒ग्निम् गा॑य॒त्रम् । \newline
25. ये इति॒ ये । \newline
26. अ॒ग्निम् गा॑य॒त्रम् गा॑य॒त्र म॒ग्नि म॒ग्निम् गा॑य॒त्रम् छन्द॒ श्छन्दो॑ गाय॒त्र म॒ग्नि म॒ग्निम् गा॑य॒त्रम् छन्दः॑ । \newline
27. गा॒य॒त्रम् छन्द॒ श्छन्दो॑ गाय॒त्रम् गा॑य॒त्रम् छन्दो ऽन्वनु॒ च्छन्दो॑ गाय॒त्रम् गा॑य॒त्रम् छन्दो ऽनु॑ । \newline
28. छन्दो ऽन्वनु॒ च्छन्द॒ श्छन्दो ऽनु॒ प्र प्राणु॒ च्छन्द॒ श्छन्दो ऽनु॒ प्र । \newline
29. अनु॒ प्र प्राण्वनु॒ प्र जा॑यस्व जायस्व॒ प्राण्वनु॒ प्र जा॑यस्व । \newline
30. प्र जा॑यस्व जायस्व॒ प्र प्र जा॑य॒स्वे तीति॑ जायस्व॒ प्र प्र जा॑य॒स्वेति॑ । \newline
31. जा॒य॒स्वे तीति॑ जायस्व जाय॒स्वे त्या॑हा॒हेति॑ जायस्व जाय॒स्वे त्या॑ह । \newline
32. इत्या॑हा॒हे तीत्या॑ह॒ छन्दो॑भि॒ श्छन्दो॑भि रा॒हे तीत्या॑ह॒ छन्दो॑भिः । \newline
33. आ॒ह॒ छन्दो॑भि॒ श्छन्दो॑भि राहाह॒ छन्दो॑भि रे॒वैव छन्दो॑भि राहाह॒ छन्दो॑भि रे॒व । \newline
34. छन्दो॑भि रे॒वैव छन्दो॑भि॒ श्छन्दो॑भि रे॒वैन॑ मेन मे॒व छन्दो॑भि॒ श्छन्दो॑भि रे॒वैन᳚म् । \newline
35. छन्दो॑भि॒रिति॒ छन्दः॑ - भिः॒ । \newline
36. ए॒वैन॑ मेन मे॒वै वैन॒म् प्र प्रैन॑ मे॒वै वैन॒म् प्र । \newline
37. ए॒न॒म् प्र प्रैन॑ मेन॒म् प्र ज॑नयति जनयति॒ प्रैन॑ मेन॒म् प्र ज॑नयति । \newline
38. प्र ज॑नयति जनयति॒ प्र प्र ज॑नय त्य॒ग्नये॒ ऽग्नये॑ जनयति॒ प्र प्र ज॑नय त्य॒ग्नये᳚ । \newline
39. ज॒न॒य॒ त्य॒ग्नये॒ ऽग्नये॑ जनयति जनय त्य॒ग्नये॑ म॒थ्यमा॑नाय म॒थ्यमा॑ना या॒ग्नये॑ जनयति जनय त्य॒ग्नये॑ म॒थ्यमा॑नाय । \newline
40. अ॒ग्नये॑ म॒थ्यमा॑नाय म॒थ्यमा॑ना या॒ग्नये॒ ऽग्नये॑ म॒थ्यमा॑ना॒ यान्वनु॑ म॒थ्यमा॑ना या॒ग्नये॒ ऽग्नये॑ म॒थ्यमा॑ना॒ यानु॑ । \newline
41. म॒थ्यमा॑ना॒ यान्वनु॑ म॒थ्यमा॑नाय म॒थ्यमा॑ना॒ यानु॑ ब्रूहि ब्रू॒ह्यनु॑ म॒थ्यमा॑नाय म॒थ्यमा॑ना॒ यानु॑ ब्रूहि । \newline
42. अनु॑ ब्रूहि ब्रू॒ह्यन् वनु॑ ब्रू॒ही तीति॑ ब्रू॒ह्यन् वनु॑ ब्रू॒हीति॑ । \newline
43. ब्रू॒ही तीति॑ ब्रूहि ब्रू॒हीत्या॑ हा॒हेति॑ ब्रूहि ब्रू॒ही त्या॑ह । \newline
44. इत्या॑हा॒हे तीत्या॑ह सावि॒त्रीꣳ सा॑वि॒त्री मा॒हे तीत्या॑ह सावि॒त्रीम् । \newline
45. आ॒ह॒ सा॒वि॒त्रीꣳ सा॑वि॒त्री मा॑हाह सावि॒त्री मृच॒ मृचꣳ॑ सावि॒त्री मा॑हाह सावि॒त्री मृच᳚म् । \newline
46. सा॒वि॒त्री मृच॒ मृचꣳ॑ सावि॒त्रीꣳ सा॑वि॒त्री मृच॒ मन्वन् वृचꣳ॑ सावि॒त्रीꣳ सा॑वि॒त्री मृच॒ मनु॑ । \newline
47. ऋच॒ मन्वन् वृच॒ मृच॒ मन्वा॑ हा॒हान् वृच॒ मृच॒ मन्वा॑ह । \newline
48. अन्वा॑हा॒हान् वन् वा॑ह सवि॒तृप्र॑सूतः सवि॒तृप्र॑सूत आ॒हान् वन् वा॑ह सवि॒तृप्र॑सूतः । \newline
49. आ॒ह॒ स॒वि॒तृप्र॑सूतः सवि॒तृप्र॑सूत आहाह सवि॒तृप्र॑सूत ए॒वैव स॑वि॒तृप्र॑सूत आहाह सवि॒तृप्र॑सूत ए॒व । \newline
50. स॒वि॒तृप्र॑सूत ए॒वैव स॑वि॒तृप्र॑सूतः सवि॒तृप्र॑सूत ए॒वैन॑ मेन मे॒व स॑वि॒तृप्र॑सूतः सवि॒तृप्र॑सूत ए॒वैन᳚म् । \newline
51. स॒वि॒तृप्र॑सूत॒ इति॑ सवि॒तृ - प्र॒सू॒तः॒ । \newline
52. ए॒वैन॑ मेन मे॒वै वैन॑म् मन्थति मन्थ त्येन मे॒वै वैन॑म् मन्थति । \newline
53. ए॒न॒म् म॒न्थ॒ति॒ म॒न्थ॒ त्ये॒न॒ मे॒न॒म् म॒न्थ॒ति॒ जा॒ताय॑ जा॒ताय॑ मन्थ त्येन मेनम् मन्थति जा॒ताय॑ । \newline
54. म॒न्थ॒ति॒ जा॒ताय॑ जा॒ताय॑ मन्थति मन्थति जा॒तायान् वनु॑ जा॒ताय॑ मन्थति मन्थति जा॒तायानु॑ । \newline
55. जा॒तायान् वनु॑ जा॒ताय॑ जा॒तायानु॑ ब्रूहि ब्रू॒ह्यनु॑ जा॒ताय॑ जा॒तायानु॑ ब्रूहि । \newline
56. अनु॑ ब्रूहि ब्रू॒ह्यन् वनु॑ ब्रूहि प्रह्रि॒यमा॑णाय प्रह्रि॒यमा॑णाय ब्रू॒ह्यन् वनु॑ ब्रूहि प्रह्रि॒यमा॑णाय । \newline
57. ब्रू॒हि॒ प्र॒ह्रि॒यमा॑णाय प्रह्रि॒यमा॑णाय ब्रूहि ब्रूहि प्रह्रि॒यमा॑णा॒ यान् वनु॑ प्रह्रि॒यमा॑णाय ब्रूहि ब्रूहि प्रह्रि॒यमा॑णा॒यानु॑ । \newline
\pagebreak
\markright{ TS 6.3.5.4  \hfill https://www.vedavms.in \hfill}

\section{ TS 6.3.5.4 }

\textbf{TS 6.3.5.4 } \newline
\textbf{Samhita Paata} \newline

प्रह्रि॒यमा॑णा॒यानु॑ ब्रू॒हीत्या॑ह॒ काण्डे॑काण्ड ए॒वैनं॑ क्रि॒यमा॑णे॒ सम॑र्द्धयति गाय॒त्रीः सर्वा॒ अन्वा॑ह गाय॒त्रछ॑न्दा॒ वा अ॒ग्निः स्वेनै॒वैनं॒ छन्द॑सा॒ सम॑र्द्धयत्य॒ग्निः पु॒रा भव॑त्य॒ग्निं म॑थि॒त्वा प्र ह॑रति॒ तौ स॒भंव॑न्तौ॒ यज॑मानम॒भि सं भ॑वतो॒ भव॑तं नः॒ सम॑नसा॒वित्या॑ह॒ शान्त्यै᳚ प्र॒हृत्य॑ जुहोति जा॒तायै॒वास्मा॒ अन्न॒मपि॑ ( ) दधा॒त्याज्ये॑न जुहोत्ये॒तद्वा अ॒ग्नेः प्रि॒यं धाम॒ यदाज्यं॑ प्रि॒येणै॒वैनं॒ धाम्ना॒ सम॑र्द्धय॒त्यथो॒ तेज॑सा ॥ \newline

\textbf{Pada Paata} \newline

प्र॒ह्रि॒यमा॑णा॒येति॑ प्र - ह्रि॒यमा॑णाय । अन्विति॑ । ब्रू॒हि॒ । इति॑ । आ॒ह॒ । काण्डे॑काण्ड॒ इति॒ काण्डे᳚ - का॒ण्डे॒ । ए॒व । ए॒न॒म् । क्रि॒यमा॑णे । समिति॑ । अ॒द्‌र्ध॒य॒ति॒ । गा॒य॒त्रीः । सर्वाः᳚ । अन्विति॑ । आ॒ह॒ । गा॒य॒त्रछ॑न्दा॒ इति॑ गाय॒त्र-छ॒न्दाः॒ । वै । अ॒ग्निः । स्वेन॑ । ए॒व । ए॒न॒म् । छन्द॑सा । समिति॑ । अ॒द्‌र्ध॒य॒ति॒ । अ॒ग्निः । पु॒रा । भव॑ति । अ॒ग्निम् । म॒थि॒त्वा । प्रेति॑ । ह॒र॒ति॒ । तौ । स॒भंव॑न्ता॒विति॑ सं - भव॑न्तौ । यज॑मानम् । अ॒भि । समिति॑ । भ॒व॒तः॒ । भव॑तम् । नः॒ । सम॑नसा॒विति॒ स - म॒न॒सौ॒ । इति॑ । आ॒ह॒ । शान्त्यै᳚ । प्र॒हृत्येति॑ प्र-हृत्य॑ । जु॒हो॒ति॒ । जा॒ताय॑ । ए॒व । अ॒स्मै॒ । अन्न᳚म् । अपीति॑ ( ) । द॒धा॒ति॒ । आज्ये॑न । जु॒हो॒ति॒ । ए॒तत् । वै । अ॒ग्नेः । प्रि॒यम् । धाम॑ । यत् । आज्य᳚म् । प्रि॒येण॑ । ए॒व । ए॒न॒म् । धाम्ना᳚ । समिति॑ । अ॒द्‌र्ध॒य॒ति॒ । अथो॒ इति॑ । तेज॑सा ॥  \newline


\textbf{Krama Paata} \newline

प्र॒ह्रि॒यमा॑णा॒यानु॑ । प्र॒ह्रि॒यमा॑णा॒येति॑ प्र - ह्रि॒यमा॑णाय । अनु॑ ब्रूहि । ब्रू॒हीति॑ । इत्या॑ह । आ॒ह॒ काण्डे॑काण्डे । काण्डे॑काण्ड ए॒व । काण्डे॑काण्ड॒ इति॒ काण्डे᳚ - का॒ण्डे॒ । ए॒वैन᳚म् । ए॒न॒म् क्रि॒यमा॑णे । क्रि॒यमा॑णे॒ सम् । सम॑र्द्धयति । अ॒र्द्ध॒य॒ति॒ गा॒य॒त्रीः । गा॒य॒त्रीः सर्वाः᳚ । सर्वा॒ अनु॑ । अन्वा॑ह । आ॒ह॒ गा॒य॒त्रछ॑न्दाः । गा॒य॒त्रछ॑न्दा॒ वै । गा॒य॒त्रछ॑न्दा॒ इति॑ गाय॒त्र - छ॒न्दाः॒ । वा अ॒ग्निः । अ॒ग्निः स्वेन॑ । स्वेनै॒व । ए॒वैन᳚म् । ए॒न॒म् छन्द॑सा । छन्द॑सा॒ सम् । सम॑र्द्धयति । अ॒र्द्ध॒य॒त्य॒ग्निः । अ॒ग्निः पु॒रा । पु॒रा भव॑ति । भव॑त्य॒ग्निम् । अ॒ग्निम् म॑थि॒त्वा । म॒थि॒त्वा प्र । प्र ह॑रति । ह॒र॒ति॒ तौ । तौ स॒म्भव॑न्तौ । स॒म्भ॑वन्तौ॒ यज॑मानम् । स॒म्भव॑न्ता॒विति॑ सम् - भव॑न्तौ । यज॑मानम॒भि । अ॒भि सम् । सम् भ॑वतः । भ॒व॒तो॒ भव॑तम् । भव॑तम् नः । नः॒ सम॑नसौ । सम॑नसा॒विति॑ । सम॑नसा॒विति॒ स - म॒न॒सौ॒ । इत्या॑ह । आ॒ह॒ शान्त्यै᳚ । शान्त्यै᳚ प्र॒हृत्य॑ । प्र॒हृत्य॑ जुहोति । प्र॒हृत्येति॑ प्र - हृत्य॑ । जु॒हो॒ति॒ जा॒ताय॑ । जा॒तायै॒व । ए॒वास्मै᳚ । अ॒स्मा॒ अन्न᳚म् । अन्न॒मपि॑ ( ) । अपि॑ दधाति । द॒धा॒त्याज्ये॑न । आज्ये॑न जुहोति । जु॒हो॒त्ये॒तत् । ए॒तद् वै । वा अ॒ग्नेः । अ॒ग्नेः प्रि॒यम् । प्रि॒यम् धाम॑ । धाम॒ यत् । यदाज्य᳚म् । आज्य॑म् प्रि॒येण॑ । प्रि॒येणै॒व । ए॒वैन᳚म् । ए॒न॒म् धाम्ना᳚ । धाम्ना॒ सम् । सम॑र्द्धयति । अ॒र्द्ध॒य॒त्यथो᳚ । अथो॒ तेज॑सा । अथो॒ इत्यथो᳚ । तेज॒सेति॒ तेज॑सा । \newline

\textbf{Jatai Paata} \newline

1. प्र॒ह्रि॒यमा॑णा॒या न्वनु॑ प्रह्रि॒यमा॑णाय प्रह्रि॒यमा॑णा॒ यानु॑ । \newline
2. प्र॒ह्रि॒यमा॑णा॒येति॑ प्र - ह्रि॒यमा॑णाय । \newline
3. अनु॑ ब्रूहि ब्रू॒ह्यन् वनु॑ ब्रूहि । \newline
4. ब्रू॒ही तीति॑ ब्रूहि ब्रू॒हीति॑ । \newline
5. इत्या॑हा॒हे तीत्या॑ह । \newline
6. आ॒ह॒ काण्डे॑काण्डे॒ काण्डे॑काण्ड आहाह॒ काण्डे॑काण्डे । \newline
7. काण्डे॑काण्ड ए॒वैव काण्डे॑काण्डे॒ काण्डे॑काण्ड ए॒व । \newline
8. काण्डे॑काण्ड॒ इति॒ काण्डे᳚ - का॒ण्डे॒ । \newline
9. ए॒वैन॑ मेन मे॒वै वैन᳚म् । \newline
10. ए॒न॒म् क्रि॒यमा॑णे क्रि॒यमा॑ण एन मेनम् क्रि॒यमा॑णे । \newline
11. क्रि॒यमा॑णे॒ सꣳ सम् क्रि॒यमा॑णे क्रि॒यमा॑णे॒ सम् । \newline
12. स म॑र्द्धय त्यर्द्धयति॒ सꣳ स म॑र्द्धयति । \newline
13. अ॒र्द्ध॒य॒ति॒ गा॒य॒त्रीर् गा॑य॒त्री र॑र्द्धय त्यर्द्धयति गाय॒त्रीः । \newline
14. गा॒य॒त्रीः सर्वाः॒ सर्वा॑ गाय॒त्रीर् गा॑य॒त्रीः सर्वाः᳚ । \newline
15. सर्वा॒ अन्वनु॒ सर्वाः॒ सर्वा॒ अनु॑ । \newline
16. अन्वा॑ हा॒हा न्वन् वा॑ह । \newline
17. आ॒ह॒ गा॒य॒त्रछ॑न्दा गाय॒त्रछ॑न्दा आहाह गाय॒त्रछ॑न्दाः । \newline
18. गा॒य॒त्रछ॑न्दा॒ वै वै गा॑य॒त्रछ॑न्दा गाय॒त्रछ॑न्दा॒ वै । \newline
19. गा॒य॒त्रछ॑न्दा॒ इति॑ गाय॒त्र - छ॒न्दाः॒ । \newline
20. वा अ॒ग्नि र॒ग्निर् वै वा अ॒ग्निः । \newline
21. अ॒ग्निः स्वेन॒ स्वेना॒ग्नि र॒ग्निः स्वेन॑ । \newline
22. स्वेनै॒ वैव स्वेन॒ स्वेनै॒व । \newline
23. ए॒वैन॑ मेन मे॒वै वैन᳚म् । \newline
24. ए॒न॒म् छन्द॑सा॒ छन्द॑सैन मेन॒म् छन्द॑सा । \newline
25. छन्द॑सा॒ सꣳ सम् छन्द॑सा॒ छन्द॑सा॒ सम् । \newline
26. स म॑र्द्धय त्यर्द्धयति॒ सꣳ स म॑र्द्धयति । \newline
27. अ॒र्द्ध॒य॒ त्य॒ग्नि र॒ग्नि र॑र्द्धय त्यर्द्धय त्य॒ग्निः । \newline
28. अ॒ग्निः पु॒रा पु॒रा ऽग्नि र॒ग्निः पु॒रा । \newline
29. पु॒रा भव॑ति॒ भव॑ति पु॒रा पु॒रा भव॑ति । \newline
30. भव॑ त्य॒ग्नि म॒ग्निम् भव॑ति॒ भव॑ त्य॒ग्निम् । \newline
31. अ॒ग्निम् म॑थि॒त्वा म॑थि॒त्वा ऽग्नि म॒ग्निम् म॑थि॒त्वा । \newline
32. म॒थि॒त्वा प्र प्र म॑थि॒त्वा म॑थि॒त्वा प्र । \newline
33. प्र ह॑रति हरति॒ प्र प्र ह॑रति । \newline
34. ह॒र॒ति॒ तौ तौ ह॑रति हरति॒ तौ । \newline
35. तौ सं॒भव॑न्तौ सं॒भव॑न्तौ॒ तौ तौ सं॒भव॑न्तौ । \newline
36. सं॒भव॑न्तौ॒ यज॑मानं॒ ॅयज॑मानꣳ सं॒भव॑न्तौ सं॒भव॑न्तौ॒ यज॑मानम् । \newline
37. सं॒भव॑न्ता॒विति॑ सं - भव॑न्तौ । \newline
38. यज॑मान म॒भ्य॑भि यज॑मानं॒ ॅयज॑मान म॒भि । \newline
39. अ॒भि सꣳ स म॒भ्य॑भि सम् । \newline
40. सम् भ॑वतो भवतः॒ सꣳ सम् भ॑वतः । \newline
41. भ॒व॒तो॒ भव॑त॒म् भव॑तम् भवतो भवतो॒ भव॑तम् । \newline
42. भव॑तन् नो नो॒ भव॑त॒म् भव॑तन् नः । \newline
43. नः॒ सम॑नसौ॒ सम॑नसौ नो नः॒ सम॑नसौ । \newline
44. सम॑नसा॒ वितीति॒ सम॑नसौ॒ सम॑नसा॒ विति॑ । \newline
45. सम॑नसा॒विति॒ स - म॒न॒सौ॒ । \newline
46. इत्या॑हा॒हे तीत्या॑ह । \newline
47. आ॒ह॒ शान्त्यै॒ शान्त्या॑ आहाह॒ शान्त्यै᳚ । \newline
48. शान्त्यै᳚ प्र॒हृत्य॑ प्र॒हृत्य॒ शान्त्यै॒ शान्त्यै᳚ प्र॒हृत्य॑ । \newline
49. प्र॒हृत्य॑ जुहोति जुहोति प्र॒हृत्य॑ प्र॒हृत्य॑ जुहोति । \newline
50. प्र॒हृत्येति॑ प्र - हृत्य॑ । \newline
51. जु॒हो॒ति॒ जा॒ताय॑ जा॒ताय॑ जुहोति जुहोति जा॒ताय॑ । \newline
52. जा॒तायै॒ वैव जा॒ताय॑ जा॒तायै॒व । \newline
53. ए॒वास्मा॑ अस्मा ए॒वै वास्मै᳚ । \newline
54. अ॒स्मा॒ अन्न॒ मन्न॑ मस्मा अस्मा॒ अन्न᳚म् । \newline
55. अन्न॒ मप्य प्यन्न॒ मन्न॒ मपि॑ । \newline
56. अपि॑ दधाति दधा॒ त्यप्यपि॑ दधाति । \newline
57. द॒धा॒ त्याज्ये॒ना ज्ये॑न दधाति दधा॒ त्याज्ये॑न । \newline
58. आज्ये॑न जुहोति जुहो॒ त्याज्ये॒ना ज्ये॑न जुहोति । \newline
59. जु॒हो॒ त्ये॒त दे॒तज् जु॑होति जुहो त्ये॒तत् । \newline
60. ए॒तद् वै वा ए॒त दे॒तद् वै । \newline
61. वा अ॒ग्ने र॒ग्नेर् वै वा अ॒ग्नेः । \newline
62. अ॒ग्नेः प्रि॒यम् प्रि॒य म॒ग्ने र॒ग्नेः प्रि॒यम् । \newline
63. प्रि॒यम् धाम॒ धाम॑ प्रि॒यम् प्रि॒यम् धाम॑ । \newline
64. धाम॒ यद् यद् धाम॒ धाम॒ यत् । \newline
65. यदाज्य॒ माज्यं॒ ॅयद् यदाज्य᳚म् । \newline
66. आज्य॑म् प्रि॒येण॑ प्रि॒येणाज्य॒ माज्य॑म् प्रि॒येण॑ । \newline
67. प्रि॒येणै॒ वैव प्रि॒येण॑ प्रि॒येणै॒व । \newline
68. ए॒वैन॑ मेन मे॒वै वैन᳚म् । \newline
69. ए॒न॒म् धाम्ना॒ धाम्नै॑न मेन॒म् धाम्ना᳚ । \newline
70. धाम्ना॒ सꣳ सम् धाम्ना॒ धाम्ना॒ सम् । \newline
71. स म॑र्द्धय त्यर्द्धयति॒ सꣳ स म॑र्द्धयति । \newline
72. अ॒र्द्ध॒य॒ त्यथो॒ अथो॑ अर्द्धय त्यर्द्धय॒ त्यथो᳚ । \newline
73. अथो॒ तेज॑सा॒ तेज॒सा ऽथो॒ अथो॒ तेज॑सा । \newline
74. अथो॒ इत्यथो᳚ । \newline
75. तेज॒सेति॒ तेज॑सा । \newline

\textbf{Ghana Paata } \newline

1. प्र॒ह्रि॒यमा॑णा॒ यान् वनु॑ प्रह्रि॒यमा॑णाय प्रह्रि॒यमा॑णा॒यानु॑ ब्रूहि ब्रू॒ह्यनु॑ प्रह्रि॒यमा॑णाय प्रह्रि॒यमा॑णा॒यानु॑ ब्रूहि । \newline
2. प्र॒ह्रि॒यमा॑णा॒येति॑ प्र - ह्रि॒यमा॑णाय । \newline
3. अनु॑ ब्रूहि ब्रू॒ह्यन् वनु॑ ब्रू॒ही तीति॑ ब्रू॒ह्यन् वनु॑ ब्रू॒हीति॑ । \newline
4. ब्रू॒हीतीति॑ ब्रूहि ब्रू॒हीत्या॑ हा॒हेति॑ ब्रूहि ब्रू॒ही त्या॑ह । \newline
5. इत्या॑हा॒हे तीत्या॑ह॒ काण्डे॑काण्डे॒ काण्डे॑काण्ड आ॒हे तीत्या॑ह॒ काण्डे॑काण्डे । \newline
6. आ॒ह॒ काण्डे॑काण्डे॒ काण्डे॑काण्ड आहाह॒ काण्डे॑काण्ड ए॒वैव काण्डे॑काण्ड आहाह॒ काण्डे॑काण्ड ए॒व । \newline
7. काण्डे॑काण्ड ए॒वैव काण्डे॑काण्डे॒ काण्डे॑काण्ड ए॒वैन॑ मेन मे॒व काण्डे॑काण्डे॒ काण्डे॑काण्ड ए॒वैन᳚म् । \newline
8. काण्डे॑काण्ड॒ इति॒ काण्डे᳚ - का॒ण्डे॒ । \newline
9. ए॒वैन॑ मेन मे॒वै वैन॑म् क्रि॒यमा॑णे क्रि॒यमा॑ण एन मे॒वै वैन॑म् क्रि॒यमा॑णे । \newline
10. ए॒न॒म् क्रि॒यमा॑णे क्रि॒यमा॑ण एन मेनम् क्रि॒यमा॑णे॒ सꣳ सम् क्रि॒यमा॑ण एन मेनम् क्रि॒यमा॑णे॒ सम् । \newline
11. क्रि॒यमा॑णे॒ सꣳ सम् क्रि॒यमा॑णे क्रि॒यमा॑णे॒ स म॑र्द्धय त्यर्द्धयति॒ सम् क्रि॒यमा॑णे क्रि॒यमा॑णे॒ स म॑र्द्धयति । \newline
12. स म॑र्द्धय त्यर्द्धयति॒ सꣳ स म॑र्द्धयति गाय॒त्रीर् गा॑य॒त्री र॑र्द्धयति॒ सꣳ स म॑र्द्धयति गाय॒त्रीः । \newline
13. अ॒र्द्ध॒य॒ति॒ गा॒य॒त्रीर् गा॑य॒त्री र॑र्द्धय त्यर्द्धयति गाय॒त्रीः सर्वाः॒ सर्वा॑ गाय॒त्री र॑र्द्धय त्यर्द्धयति गाय॒त्रीः सर्वाः᳚ । \newline
14. गा॒य॒त्रीः सर्वाः॒ सर्वा॑ गाय॒त्रीर् गा॑य॒त्रीः सर्वा॒ अन्वनु॒ सर्वा॑ गाय॒त्रीर् गा॑य॒त्रीः सर्वा॒ अनु॑ । \newline
15. सर्वा॒ अन्वनु॒ सर्वाः॒ सर्वा॒ अन्वा॑हा॒ हानु॒ सर्वाः॒ सर्वा॒ अन्वा॑ह । \newline
16. अन्वा॑हा॒ हान् वन् वा॑ह गाय॒त्रछ॑न्दा गाय॒त्रछ॑न्दा आ॒हान् वन् वा॑ह गाय॒त्रछ॑न्दाः । \newline
17. आ॒ह॒ गा॒य॒त्रछ॑न्दा गाय॒त्रछ॑न्दा आहाह गाय॒त्रछ॑न्दा॒ वै वै गा॑य॒त्रछ॑न्दा आहाह गाय॒त्रछ॑न्दा॒ वै । \newline
18. गा॒य॒त्रछ॑न्दा॒ वै वै गा॑य॒त्रछ॑न्दा गाय॒त्रछ॑न्दा॒ वा अ॒ग्नि र॒ग्निर् वै गा॑य॒त्रछ॑न्दा गाय॒त्रछ॑न्दा॒ वा अ॒ग्निः । \newline
19. गा॒य॒त्रछ॑न्दा॒ इति॑ गाय॒त्र - छ॒न्दाः॒ । \newline
20. वा अ॒ग्नि र॒ग्निर् वै वा अ॒ग्निः स्वेन॒ स्वेना॒ग्निर् वै वा अ॒ग्निः स्वेन॑ । \newline
21. अ॒ग्निः स्वेन॒ स्वेना॒ग्नि र॒ग्निः स्वेनै॒ वैव स्वेना॒ग्नि र॒ग्निः स्वेनै॒व । \newline
22. स्वेनै॒ वैव स्वेन॒ स्वेनै॒ वैन॑ मेन मे॒व स्वेन॒ स्वेनै॒ वैन᳚म् । \newline
23. ए॒वैन॑ मेन मे॒वै वैन॒म् छन्द॑सा॒ छन्द॑सैन मे॒वै वैन॒म् छन्द॑सा । \newline
24. ए॒न॒म् छन्द॑सा॒ छन्द॑सैन मेन॒म् छन्द॑सा॒ सꣳ सम् छन्द॑सैन मेन॒म् छन्द॑सा॒ सम् । \newline
25. छन्द॑सा॒ सꣳ सम् छन्द॑सा॒ छन्द॑सा॒ स म॑र्द्धय त्यर्द्धयति॒ सम् छन्द॑सा॒ छन्द॑सा॒ स म॑र्द्धयति । \newline
26. स म॑र्द्धय त्यर्द्धयति॒ सꣳ स म॑र्द्धय त्य॒ग्नि र॒ग्नि र॑र्द्धयति॒ सꣳ स म॑र्द्धय त्य॒ग्निः । \newline
27. अ॒र्द्ध॒य॒ त्य॒ग्नि र॒ग्नि र॑र्द्धय त्यर्द्धय त्य॒ग्निः पु॒रा पु॒रा ऽग्नि र॑र्द्धय त्यर्द्धय त्य॒ग्निः पु॒रा । \newline
28. अ॒ग्निः पु॒रा पु॒रा ऽग्नि र॒ग्निः पु॒रा भव॑ति॒ भव॑ति पु॒रा ऽग्नि र॒ग्निः पु॒रा भव॑ति । \newline
29. पु॒रा भव॑ति॒ भव॑ति पु॒रा पु॒रा भव॑ त्य॒ग्नि म॒ग्निम् भव॑ति पु॒रा पु॒रा भव॑ त्य॒ग्निम् । \newline
30. भव॑ त्य॒ग्नि म॒ग्निम् भव॑ति॒ भव॑ त्य॒ग्निम् म॑थि॒त्वा म॑थि॒त्वा ऽग्निम् भव॑ति॒ भव॑ त्य॒ग्निम् म॑थि॒त्वा । \newline
31. अ॒ग्निम् म॑थि॒त्वा म॑थि॒त्वा ऽग्नि म॒ग्निम् म॑थि॒त्वा प्र प्र म॑थि॒त्वा ऽग्नि म॒ग्निम् म॑थि॒त्वा प्र । \newline
32. म॒थि॒त्वा प्र प्र म॑थि॒त्वा म॑थि॒त्वा प्र ह॑रति हरति॒ प्र म॑थि॒त्वा म॑थि॒त्वा प्र ह॑रति । \newline
33. प्र ह॑रति हरति॒ प्र प्र ह॑रति॒ तौ तौ ह॑रति॒ प्र प्र ह॑रति॒ तौ । \newline
34. ह॒र॒ति॒ तौ तौ ह॑रति हरति॒ तौ सं॒भव॑न्तौ सं॒भव॑न्तौ॒ तौ ह॑रति हरति॒ तौ सं॒भव॑न्तौ । \newline
35. तौ सं॒भव॑न्तौ सं॒भव॑न्तौ॒ तौ तौ सं॒भव॑न्तौ॒ यज॑मानं॒ ॅयज॑मानꣳ सं॒भव॑न्तौ॒ तौ तौ सं॒भव॑न्तौ॒ यज॑मानम् । \newline
36. सं॒भव॑न्तौ॒ यज॑मानं॒ ॅयज॑मानꣳ सं॒भव॑न्तौ सं॒भव॑न्तौ॒ यज॑मान म॒भ्य॑भि यज॑मानꣳ सं॒भव॑न्तौ सं॒भव॑न्तौ॒ यज॑मान म॒भि । \newline
37. सं॒भव॑न्ता॒विति॑ सं - भव॑न्तौ । \newline
38. यज॑मान म॒भ्य॑भि यज॑मानं॒ ॅयज॑मान म॒भि सꣳ स म॒भि यज॑मानं॒ ॅयज॑मान म॒भि सम् । \newline
39. अ॒भि सꣳ स म॒भ्य॑भि सम् भ॑वतो भवतः॒ स म॒भ्य॑भि सम् भ॑वतः । \newline
40. सम् भ॑वतो भवतः॒ सꣳ सम् भ॑वतो॒ भव॑त॒म् भव॑तम् भवतः॒ सꣳ सम् भ॑वतो॒ भव॑तम् । \newline
41. भ॒व॒तो॒ भव॑त॒म् भव॑तम् भवतो भवतो॒ भव॑तन् नो नो॒ भव॑तम् भवतो भवतो॒ भव॑तन् नः । \newline
42. भव॑तन् नो नो॒ भव॑त॒म् भव॑तन् नः॒ सम॑नसौ॒ सम॑नसौ नो॒ भव॑त॒म् भव॑तन् नः॒ सम॑नसौ । \newline
43. नः॒ सम॑नसौ॒ सम॑नसौ नो नः॒ सम॑नसा॒ वितीति॒ सम॑नसौ नो नः॒ सम॑नसा॒ विति॑ । \newline
44. सम॑नसा॒ वितीति॒ सम॑नसौ॒ सम॑नसा॒ वित्या॑हा॒ हेति॒ सम॑नसौ॒ सम॑नसा॒ वित्या॑ह । \newline
45. सम॑नसा॒विति॒ स - म॒न॒सौ॒ । \newline
46. इत्या॑हा॒हे तीत्या॑ह॒ शान्त्यै॒ शान्त्या॑ आ॒हे तीत्या॑ह॒ शान्त्यै᳚ । \newline
47. आ॒ह॒ शान्त्यै॒ शान्त्या॑ आहाह॒ शान्त्यै᳚ प्र॒हृत्य॑ प्र॒हृत्य॒ शान्त्या॑ आहाह॒ शान्त्यै᳚ प्र॒हृत्य॑ । \newline
48. शान्त्यै᳚ प्र॒हृत्य॑ प्र॒हृत्य॒ शान्त्यै॒ शान्त्यै᳚ प्र॒हृत्य॑ जुहोति जुहोति प्र॒हृत्य॒ शान्त्यै॒ शान्त्यै᳚ प्र॒हृत्य॑ जुहोति । \newline
49. प्र॒हृत्य॑ जुहोति जुहोति प्र॒हृत्य॑ प्र॒हृत्य॑ जुहोति जा॒ताय॑ जा॒ताय॑ जुहोति प्र॒हृत्य॑ प्र॒हृत्य॑ जुहोति जा॒ताय॑ । \newline
50. प्र॒हृत्येति॑ प्र - हृत्य॑ । \newline
51. जु॒हो॒ति॒ जा॒ताय॑ जा॒ताय॑ जुहोति जुहोति जा॒तायै॒ वैव जा॒ताय॑ जुहोति जुहोति जा॒तायै॒व । \newline
52. जा॒तायै॒ वैव जा॒ताय॑ जा॒तायै॒ वास्मा॑ अस्मा ए॒व जा॒ताय॑ जा॒तायै॒ वास्मै᳚ । \newline
53. ए॒वास्मा॑ अस्मा ए॒वै वास्मा॒ अन्न॒ मन्न॑ मस्मा ए॒वै वास्मा॒ अन्न᳚म् । \newline
54. अ॒स्मा॒ अन्न॒ मन्न॑ मस्मा अस्मा॒ अन्न॒ मप्य प्यन्न॑ मस्मा अस्मा॒ अन्न॒ मपि॑ । \newline
55. अन्न॒ मप्य प्यन्न॒ मन्न॒ मपि॑ दधाति दधा॒ त्यप्यन्न॒ मन्न॒ मपि॑ दधाति । \newline
56. अपि॑ दधाति दधा॒ त्यप्यपि॑ दधा॒ त्याज्ये॒ना ज्ये॑न दधा॒ त्यप्यपि॑ दधा॒ त्याज्ये॑न । \newline
57. द॒धा॒ त्याज्ये॒ना ज्ये॑न दधाति दधा॒ त्याज्ये॑न जुहोति जुहो॒ त्याज्ये॑न दधाति दधा॒ त्याज्ये॑न जुहोति । \newline
58. आज्ये॑न जुहोति जुहो॒ त्याज्ये॒ना ज्ये॑न जुहो त्ये॒त दे॒तज् जु॑हो॒ त्याज्ये॒ना ज्ये॑न जुहो त्ये॒तत् । \newline
59. जु॒हो॒ त्ये॒त दे॒तज् जु॑होति जुहो त्ये॒तद् वै वा ए॒तज् जु॑होति जुहो त्ये॒तद् वै । \newline
60. ए॒तद् वै वा ए॒त दे॒तद् वा अ॒ग्ने र॒ग्नेर् वा ए॒त दे॒तद् वा अ॒ग्नेः । \newline
61. वा अ॒ग्ने र॒ग्नेर् वै वा अ॒ग्नेः प्रि॒यम् प्रि॒य म॒ग्नेर् वै वा अ॒ग्नेः प्रि॒यम् । \newline
62. अ॒ग्नेः प्रि॒यम् प्रि॒य म॒ग्ने र॒ग्नेः प्रि॒यम् धाम॒ धाम॑ प्रि॒य म॒ग्ने र॒ग्नेः प्रि॒यम् धाम॑ । \newline
63. प्रि॒यम् धाम॒ धाम॑ प्रि॒यम् प्रि॒यम् धाम॒ यद् यद् धाम॑ प्रि॒यम् प्रि॒यम् धाम॒ यत् । \newline
64. धाम॒ यद् यद् धाम॒ धाम॒ यदाज्य॒ माज्यं॒ ॅयद् धाम॒ धाम॒ यदाज्य᳚म् । \newline
65. यदाज्य॒ माज्यं॒ ॅयद् यदाज्य॑म् प्रि॒येण॑ प्रि॒येणा ज्यं॒ ॅयद् यदाज्य॑म् प्रि॒येण॑ । \newline
66. आज्य॑म् प्रि॒येण॑ प्रि॒येणा ज्य॒ माज्य॑म् प्रि॒येणै॒ वैव प्रि॒येणा ज्य॒ माज्य॑म् प्रि॒येणै॒व । \newline
67. प्रि॒येणै॒ वैव प्रि॒येण॑ प्रि॒ये णै॒वैन॑ मेन मे॒व प्रि॒येण॑ प्रि॒ये णै॒वैन᳚म् । \newline
68. ए॒वैन॑ मेन मे॒वै वैन॒म् धाम्ना॒ धाम्नै॑न मे॒वै वैन॒म् धाम्ना᳚ । \newline
69. ए॒न॒म् धाम्ना॒ धाम्नै॑न मेन॒म् धाम्ना॒ सꣳ सम् धाम्नै॑न मेन॒म् धाम्ना॒ सम् । \newline
70. धाम्ना॒ सꣳ सम् धाम्ना॒ धाम्ना॒ स म॑र्द्धय त्यर्द्धयति॒ सम् धाम्ना॒ धाम्ना॒ स म॑र्द्धयति । \newline
71. स म॑र्द्धय त्यर्द्धयति॒ सꣳ स म॑र्द्धय॒ त्यथो॒ अथो॑ अर्द्धयति॒ सꣳ स म॑र्द्धय॒ त्यथो᳚ । \newline
72. अ॒र्द्ध॒य॒ त्यथो॒ अथो॑ अर्द्धय त्यर्द्धय॒ त्यथो॒ तेज॑सा॒ तेज॒सा ऽथो॑ अर्द्धय त्यर्द्धय॒ त्यथो॒ तेज॑सा । \newline
73. अथो॒ तेज॑सा॒ तेज॒सा ऽथो॒ अथो॒ तेज॑सा । \newline
74. अथो॒ इत्यथो᳚ । \newline
75. तेज॒सेति॒ तेज॑सा । \newline
\pagebreak
\markright{ TS 6.3.6.1  \hfill https://www.vedavms.in \hfill}

\section{ TS 6.3.6.1 }

\textbf{TS 6.3.6.1 } \newline
\textbf{Samhita Paata} \newline

इ॒षे त्वेति॑ ब॒र्॒.हिरा द॑त्त इ॒च्छत॑ इव॒ ह्ये॑ष यो यज॑त उप॒वीर॒सीत्या॒होप॒ ह्ये॑नानाक॒रोत्युपो॑ दे॒वान् दैवी॒र्विशः॒ प्रागु॒रित्या॑ह॒ दैवी॒र्ह्ये॑ता विशः॑ स॒तीर्दे॒वानु॑प॒यन्ति॒ वह्नी॑रु॒शिज॒ इत्या॑ह॒र्त्विजो॒ वै वह्न॑य उ॒शिज॒-स्तस्मा॑दे॒वमा॑ह॒ बृह॑स्पते धा॒रया॒ वसू॒नीत्या॑- [  ] \newline

\textbf{Pada Paata} \newline

इ॒षे । त्वा॒ । इति॑ । ब॒र्॒.हिः । एति॑ । द॒त्ते॒ । इ॒च्छते᳚ । इ॒व॒ । हि । ए॒षः । यः । यज॑ते । उ॒प॒वीरित्यु॑प - वीः । अ॒सि॒ । इति॑ । आ॒ह॒ । उपेति॑ । हि । ए॒ना॒न् । आ॒क॒रोतीत्या᳚ - क॒रोति॑ । उपो॒ इति॑ । दे॒वान् । दैवीः᳚ । विशः॑ । प्रेति॑ । अ॒गुः॒ । इति॑ । आ॒ह॒ । दैवीः᳚ । हि । ए॒ताः । विशः॑ । स॒तीः । दे॒वान् । उ॒प॒यन्तीत्यु॑प - यन्ति॑ । वह्नीः᳚ । उ॒शिजः॑ । इति॑ । आ॒ह॒ । ऋ॒त्विजः॑ । वै । वह्न॑यः । उ॒शिजः॑ । तस्मा᳚त् । ए॒वम् । आ॒ह॒ । बृह॑स्पते । धा॒रय॑ । वसू॑नि । इति॑ ।  \newline


\textbf{Krama Paata} \newline

इ॒षे त्वा᳚ । त्वेति॑ । इति॑ ब॒र्.॒हिः । ब॒र्.॒हिरा । आ द॑त्ते । द॒त्त॒ इ॒च्छते᳚ । इ॒च्छत॑ इव । इ॒व॒ हि । ह्ये॑षः । ए॒ष यः । यो यज॑ते । यज॑त उप॒वीः । उ॒प॒वीर॑सि । उ॒प॒वीरित्यु॑प - वीः । अ॒सीति॑ । इत्या॑ह । आ॒होप॑ । उप॒ हि । ह्ये॑नान् । ए॒ना॒ना॒क॒रोति॑ । आ॒क॒रोत्युपो᳚ । आ॒क॒रोतीत्या᳚ - क॒रोति॑ । उपो॑ दे॒वान् । उ॒पो इत्युपो᳚ । दे॒वान् दैवीः᳚ । दैवी॒र् विशः॑ । विशः॒ प्र । प्रागुः॑ । अ॒गु॒रिति॑ । इत्या॑ह । आ॒ह॒ दैवीः᳚ । दैवी॒र्.॒ हि । ह्ये॑ताः । ए॒ता विशः॑ । विशः॑ स॒तीः । स॒तीर् दे॒वान् । दे॒वानु॑प॒यन्ति॑ । उ॒प॒यन्ति॒ वह्नीः᳚ । उ॒प॒यन्तीत्यु॑प - यन्ति॑ । वह्नी॑रु॒शिजः॑ । उ॒शिज॒ इति॑ । इत्या॑ह । आ॒ह॒र्त्विजः॑ । ऋ॒त्विजो॒ वै । वै वह्न॑यः । वह्न॑य उ॒शिजः॑ । उ॒शिज॒स्तस्मा᳚त् । तस्मा॑दे॒वम् । ए॒वमा॑ह । आ॒ह॒ बृह॑स्पते । बृह॑स्पते धा॒रय॑ । धा॒रया॒ वसू॑नि । वसू॒नीति॑ । इत्या॑ह \newline

\textbf{Jatai Paata} \newline

1. इ॒षे त्वा᳚ त्वे॒ष इ॒षे त्वा᳚ । \newline
2. त्वेतीति॑ त्वा॒ त्वेति॑ । \newline
3. इति॑ ब॒र्॒.हिर् ब॒र्॒.हि रितीति॑ ब॒र्॒.हिः । \newline
4. ब॒र्॒.हिरा ब॒र्॒.हिर् ब॒र्॒.हिरा । \newline
5. आ द॑त्ते दत्त॒ आ द॑त्ते । \newline
6. द॒त्त॒ इ॒च्छत॑ इ॒च्छते॑ दत्ते दत्त इ॒च्छते᳚ । \newline
7. इ॒च्छत॑ इवे वे॒च्छत॑ इ॒च्छत॑ इव । \newline
8. इ॒व॒ हि हीवे॑व॒ हि । \newline
9. ह्ये॑ष ए॒ष हि ह्ये॑षः । \newline
10. ए॒ष यो य ए॒ष ए॒ष यः । \newline
11. यो यज॑ते॒ यज॑ते॒ यो यो यज॑ते । \newline
12. यज॑त उप॒वी रु॑प॒वीर् यज॑ते॒ यज॑त उप॒वीः । \newline
13. उ॒प॒वी र॑स्यस्यु प॒वी रु॑प॒वी र॑सि । \newline
14. उ॒प॒वीरित्यु॑प - वीः । \newline
15. अ॒सी तीत्य॑स्य॒ सीति॑ । \newline
16. इत्या॑हा॒हे तीत्या॑ह । \newline
17. आ॒हो पोपा॑ हा॒होप॑ । \newline
18. उप॒ हि ह्युपोप॒ हि । \newline
19. ह्ये॑ना नेना॒न्॒. हि ह्ये॑नान् । \newline
20. ए॒ना॒ ना॒क॒रो त्या॑क॒रो त्ये॑नानेना नाक॒रोति॑ । \newline
21. आ॒क॒रो त्युपो॒ उपो॑ आक॒रो त्या॑क॒रो त्युपो᳚ । \newline
22. आ॒क॒रोतीत्या᳚ - क॒रोति॑ । \newline
23. उपो॑ दे॒वान् दे॒वा नुपो॒ उपो॑ दे॒वान् । \newline
24. उपो॒ इत्युपो᳚ । \newline
25. दे॒वान् दैवी॒र् दैवी᳚र् दे॒वान् दे॒वान् दैवीः᳚ । \newline
26. दैवी॒र् विशो॒ विशो॒ दैवी॒र् दैवी॒र् विशः॑ । \newline
27. विशः॒ प्र प्र विशो॒ विशः॒ प्र । \newline
28. प्रागु॑ रगुः॒ प्र प्रागुः॑ । \newline
29. अ॒गु॒ रिती त्य॑गु रगु॒ रिति॑ । \newline
30. इत्या॑हा॒हे तीत्या॑ह । \newline
31. आ॒ह॒ दैवी॒र् दैवी॑ राहाह॒ दैवीः᳚ । \newline
32. दैवी॒र्॒. हि हि दैवी॒र् दैवी॒र्॒. हि । \newline
33. ह्ये॑ता ए॒ता हि ह्ये॑ताः । \newline
34. ए॒ता विशो॒ विश॑ ए॒ता ए॒ता विशः॑ । \newline
35. विशः॑ स॒तीः स॒तीर् विशो॒ विशः॑ स॒तीः । \newline
36. स॒तीर् दे॒वान् दे॒वान् थ्स॒तीः स॒तीर् दे॒वान् । \newline
37. दे॒वा नु॑प॒य न्त्यु॑प॒यन्ति॑ दे॒वान् दे॒वा नु॑प॒यन्ति॑ । \newline
38. उ॒प॒यन्ति॒ वह्नी॒र् वह्नी॑ रुप॒य न्त्यु॑प॒यन्ति॒ वह्नीः᳚ । \newline
39. उ॒प॒यन्तीत्यु॑प - यन्ति॑ । \newline
40. वह्नी॑ रु॒शिज॑ उ॒शिजो॒ वह्नी॒र् वह्नी॑ रु॒शिजः॑ । \newline
41. उ॒शिज॒ इती त्यु॒शिज॑ उ॒शिज॒ इति॑ । \newline
42. इत्या॑हा॒हे तीत्या॑ह । \newline
43. आ॒ह॒ र्‌त्विज॑ ऋ॒त्विज॑ आहाह॒ र्‌त्विजः॑ । \newline
44. ऋ॒त्विजो॒ वै वा ऋ॒त्विज॑ ऋ॒त्विजो॒ वै । \newline
45. वै वह्न॑यो॒ वह्न॑यो॒ वै वै वह्न॑यः । \newline
46. वह्न॑य उ॒शिज॑ उ॒शिजो॒ वह्न॑यो॒ वह्न॑य उ॒शिजः॑ । \newline
47. उ॒शिज॒ स्तस्मा॒त् तस्मा॑ दु॒शिज॑ उ॒शिज॒ स्तस्मा᳚त् । \newline
48. तस्मा॑ दे॒व मे॒वम् तस्मा॒त् तस्मा॑ दे॒वम् । \newline
49. ए॒व मा॑हा है॒व मे॒व मा॑ह । \newline
50. आ॒ह॒ बृह॑स्पते॒ बृह॑स्पत आहाह॒ बृह॑स्पते । \newline
51. बृह॑स्पते धा॒रय॑ धा॒रय॒ बृह॑स्पते॒ बृह॑स्पते धा॒रय॑ । \newline
52. धा॒रया॒ वसू॑नि॒ वसू॑नि धा॒रय॑ धा॒रया॒ वसू॑नि । \newline
53. वसू॒नी तीति॒ वसू॑नि॒ वसू॒ नीति॑ । \newline
54. इत्या॑हा॒हे तीत्या॑ह । \newline

\textbf{Ghana Paata } \newline

1. इ॒षे त्वा᳚ त्वे॒ष इ॒षे त्वेतीति॑ त्वे॒ष इ॒षे त्वेति॑ । \newline
2. त्वेतीति॑ त्वा॒ त्वेति॑ ब॒र्॒.हिर् ब॒र्॒.हि रिति॑ त्वा॒ त्वेति॑ ब॒र्॒.हिः । \newline
3. इति॑ ब॒र्॒.हिर् ब॒र्॒.हिरि तीति॑ ब॒र्॒.हिरा ब॒र्॒.हिरि तीति॑ ब॒र्॒.हिरा । \newline
4. ब॒र्॒.हिरा ब॒र्॒.हिर् ब॒र्॒.हिरा द॑त्ते दत्त॒ आ ब॒र्॒.हिर् ब॒र्॒.हिरा द॑त्ते । \newline
5. आ द॑त्ते दत्त॒ आ द॑त्त इ॒च्छत॑ इ॒च्छते॑ दत्त॒ आ द॑त्त इ॒च्छते᳚ । \newline
6. द॒त्त॒ इ॒च्छत॑ इ॒च्छते॑ दत्ते दत्त इ॒च्छत॑ इवे वे॒च्छते॑ दत्ते दत्त इ॒च्छत॑ इव । \newline
7. इ॒च्छत॑ इवे वे॒च्छत॑ इ॒च्छत॑ इव॒ हि हीवे॒ च्छत॑ इ॒च्छत॑ इव॒ हि । \newline
8. इ॒व॒ हि हीवे॑व॒ ह्ये॑ष ए॒ष हीवे॑व॒ ह्ये॑षः । \newline
9. ह्ये॑ष ए॒ष हि ह्ये॑ष यो य ए॒ष हि ह्ये॑ष यः । \newline
10. ए॒ष यो य ए॒ष ए॒ष यो यज॑ते॒ यज॑ते॒ य ए॒ष ए॒ष यो यज॑ते । \newline
11. यो यज॑ते॒ यज॑ते॒ यो यो यज॑त उप॒वी रु॑प॒वीर् यज॑ते॒ यो यो यज॑त उप॒वीः । \newline
12. यज॑त उप॒वी रु॑प॒वीर् यज॑ते॒ यज॑त उप॒वी र॑स्यस्यु प॒वीर् यज॑ते॒ यज॑त उप॒वी र॑सि । \newline
13. उ॒प॒वी र॑स्य स्युप॒वी रु॑प॒वी र॒सी तीत्य॑स्युप॒वी रु॑प॒वी र॒सीति॑ । \newline
14. उ॒प॒वीरित्यु॑प - वीः । \newline
15. अ॒सीती त्य॑स्य॒ सीत्या॑हा॒हे त्य॑स्य॒ सीत्या॑ह । \newline
16. इत्या॑हा॒हे तीत्या॒हो पोपा॒हे तीत्या॒ होप॑ । \newline
17. आ॒हो पोपा॑हा॒होप॒ हि ह्युपा॑हा॒ होप॒ हि । \newline
18. उप॒ हि ह्युपोप॒ ह्ये॑ना नेना॒न् ह्युपोप॒ ह्ये॑नान् । \newline
19. ह्ये॑ना नेना॒न्॒. हि ह्ये॑ना नाक॒रो त्या॑क॒रो त्ये॑ना॒न्॒. हि ह्ये॑ना नाक॒रोति॑ । \newline
20. ए॒ना॒ ना॒क॒रो त्या॑क॒रो त्ये॑ना नेना नाक॒रो त्युपो॒ उपो॑ आक॒रो त्ये॑ना नेना नाक॒रो त्युपो᳚ । \newline
21. आ॒क॒रो त्युपो॒ उपो॑ आक॒रो त्या॑क॒रो त्युपो॑ दे॒वान् दे॒वा नुपो॑ आक॒रो त्या॑क॒रो त्युपो॑ दे॒वान् । \newline
22. आ॒क॒रोतीत्या᳚ - क॒रोति॑ । \newline
23. उपो॑ दे॒वान् दे॒वा नुपो॒ उपो॑ दे॒वान् दैवी॒र् दैवी᳚र् दे॒वा नुपो॒ उपो॑ दे॒वान् दैवीः᳚ । \newline
24. उपो॒ इत्युपो᳚ । \newline
25. दे॒वान् दैवी॒र् दैवी᳚र् दे॒वान् दे॒वान् दैवी॒र् विशो॒ विशो॒ दैवी᳚र् दे॒वान् दे॒वान् दैवी॒र् विशः॑ । \newline
26. दैवी॒र् विशो॒ विशो॒ दैवी॒र् दैवी॒र् विशः॒ प्र प्र विशो॒ दैवी॒र् दैवी॒र् विशः॒ प्र । \newline
27. विशः॒ प्र प्र विशो॒ विशः॒ प्रागु॑ रगुः॒ प्र विशो॒ विशः॒ प्रागुः॑ । \newline
28. प्रागु॑ रगुः॒ प्र प्रागु॒ रिती त्य॑गुः॒ प्र प्रागु॒ रिति॑ । \newline
29. अ॒गु॒ रितीत्य॑गु रगु॒ रित्या॑हा॒हे त्य॑गु रगु॒रि त्या॑ह । \newline
30. इत्या॑हा॒हे तीत्या॑ह॒ दैवी॒र् दैवी॑ रा॒हे तीत्या॑ह॒ दैवीः᳚ । \newline
31. आ॒ह॒ दैवी॒र् दैवी॑रा हाह॒ दैवी॒र्॒. हि हि दैवी॑रा हाह॒ दैवी॒र्॒. हि । \newline
32. दैवी॒र्॒. हि हि दैवी॒र् दैवी॒र् ह्ये॑ता ए॒ता हि दैवी॒र् दैवी॒र् ह्ये॑ताः । \newline
33. ह्ये॑ता ए॒ता हि ह्ये॑ता विशो॒ विश॑ ए॒ता हि ह्ये॑ता विशः॑ । \newline
34. ए॒ता विशो॒ विश॑ ए॒ता ए॒ता विशः॑ स॒तीः स॒तीर् विश॑ ए॒ता ए॒ता विशः॑ स॒तीः । \newline
35. विशः॑ स॒तीः स॒तीर् विशो॒ विशः॑ स॒तीर् दे॒वान् दे॒वान् थ्स॒तीर् विशो॒ विशः॑ स॒तीर् दे॒वान् । \newline
36. स॒तीर् दे॒वान् दे॒वान् थ्स॒तीः स॒तीर् दे॒वा नु॑प॒य न्त्यु॑प॒यन्ति॑ दे॒वान् थ्स॒तीः स॒तीर् दे॒वा नु॑प॒यन्ति॑ । \newline
37. दे॒वा नु॑प॒य न्त्यु॑प॒यन्ति॑ दे॒वान् दे॒वा नु॑प॒यन्ति॒ वह्नी॒र् वह्नी॑ रुप॒यन्ति॑ दे॒वान् दे॒वा नु॑प॒यन्ति॒ वह्नीः᳚ । \newline
38. उ॒प॒यन्ति॒ वह्नी॒र् वह्नी॑ रुप॒य न्त्यु॑प॒यन्ति॒ वह्नी॑ रु॒शिज॑ उ॒शिजो॒ वह्नी॑ रुप॒य न्त्यु॑प॒यन्ति॒ वह्नी॑ रु॒शिजः॑ । \newline
39. उ॒प॒यन्तीत्यु॑प - यन्ति॑ । \newline
40. वह्नी॑ रु॒शिज॑ उ॒शिजो॒ वह्नी॒र् वह्नी॑ रु॒शिज॒ इती त्यु॒शिजो॒ वह्नी॒र् वह्नी॑ रु॒शिज॒ इति॑ । \newline
41. उ॒शिज॒ इती त्यु॒शिज॑ उ॒शिज॒ इत्या॑हा॒हे त्यु॒शिज॑ उ॒शिज॒ इत्या॑ह । \newline
42. इत्या॑हा॒हे तीत्या॑ह॒ र्‌त्विज॑ ऋ॒त्विज॑ आ॒हे तीत्या॑ह॒ र्‌त्विजः॑ । \newline
43. आ॒ह॒ र्‌त्विज॑ ऋ॒त्विज॑ आहाह॒ र्‌त्विजो॒ वै वा ऋ॒त्विज॑ आहाह॒ र्‌त्विजो॒ वै । \newline
44. ऋ॒त्विजो॒ वै वा ऋ॒त्विज॑ ऋ॒त्विजो॒ वै वह्न॑यो॒ वह्न॑यो॒ वा ऋ॒त्विज॑ ऋ॒त्विजो॒ वै वह्न॑यः । \newline
45. वै वह्न॑यो॒ वह्न॑यो॒ वै वै वह्न॑य उ॒शिज॑ उ॒शिजो॒ वह्न॑यो॒ वै वै वह्न॑य उ॒शिजः॑ । \newline
46. वह्न॑य उ॒शिज॑ उ॒शिजो॒ वह्न॑यो॒ वह्न॑य उ॒शिज॒ स्तस्मा॒त् तस्मा॑ दु॒शिजो॒ वह्न॑यो॒ वह्न॑य उ॒शिज॒ स्तस्मा᳚त् । \newline
47. उ॒शिज॒ स्तस्मा॒त् तस्मा॑ दु॒शिज॑ उ॒शिज॒ स्तस्मा॑ दे॒व मे॒वम् तस्मा॑ दु॒शिज॑ उ॒शिज॒ स्तस्मा॑ दे॒वम् । \newline
48. तस्मा॑ दे॒व मे॒वम् तस्मा॒त् तस्मा॑ दे॒व मा॑हा है॒वम् तस्मा॒त् तस्मा॑ दे॒व मा॑ह । \newline
49. ए॒व मा॑हा है॒व मे॒व मा॑ह॒ बृह॑स्पते॒ बृह॑स्पत आहै॒व मे॒व मा॑ह॒ बृह॑स्पते । \newline
50. आ॒ह॒ बृह॑स्पते॒ बृह॑स्पत आहाह॒ बृह॑स्पते धा॒रय॑ धा॒रय॒ बृह॑स्पत आहाह॒ बृह॑स्पते धा॒रय॑ । \newline
51. बृह॑स्पते धा॒रय॑ धा॒रय॒ बृह॑स्पते॒ बृह॑स्पते धा॒रया॒ वसू॑नि॒ वसू॑नि धा॒रय॒ बृह॑स्पते॒ बृह॑स्पते धा॒रया॒ वसू॑नि । \newline
52. धा॒रया॒ वसू॑नि॒ वसू॑नि धा॒रय॑ धा॒रया॒ वसू॒नी तीति॒ वसू॑नि धा॒रय॑ धा॒रया॒ वसू॒नीति॑ । \newline
53. वसू॒नी तीति॒ वसू॑नि॒ वसू॒नी त्या॑हा॒हेति॒ वसू॑नि॒ वसू॒नी त्या॑ह । \newline
54. इत्या॑हा॒हे तीत्या॑ह॒ ब्रह्म॒ ब्रह्मा॒हे तीत्या॑ह॒ ब्रह्म॑ । \newline
\pagebreak
\markright{ TS 6.3.6.2  \hfill https://www.vedavms.in \hfill}

\section{ TS 6.3.6.2 }

\textbf{TS 6.3.6.2 } \newline
\textbf{Samhita Paata} \newline

-ह॒ ब्रह्म॒ वै दे॒वानां॒ बृह॒स्पति॒ र्ब्रह्म॑णै॒वास्मै॑ प॒शूनव॑ रुन्धे ह॒व्या ते᳚ स्वदन्ता॒मित्या॑ह स्व॒दय॑त्ये॒वैना॒न् देव॑ त्वष्ट॒र्वसु॑ र॒ण्वेत्या॑ह॒ त्वष्टा॒ वै प॑शू॒नां मि॑थु॒नानाꣳ॑ रूप॒कृद् रू॒पमे॒व प॒शुषु॑ दधाति॒ रेव॑ती॒ रम॑द्ध्व॒मित्या॑ह प॒शवो॒ वै रे॒वतीः᳚ प॒शूने॒वास्मै॑ रमयति दे॒वस्य॑ त्वा सवि॒तुः प्र॑स॒व इति॑- [  ] \newline

\textbf{Pada Paata} \newline

आ॒ह॒ । ब्रह्म॑ । वै । दे॒वाना᳚म् । बृह॒स्पतिः॑ । ब्रह्म॑णा । ए॒व । अ॒स्मै॒ । प॒शून् । अवेति॑ । रु॒न्धे॒ । ह॒व्या । ते॒ । स्व॒द॒न्ता॒म् । इति॑ । आ॒ह॒ । स्व॒दय॑ति । ए॒व । ए॒ना॒न् । देव॑ । त्व॒ष्टः॒ । वसु॑ । र॒ण्व॒ । इति॑ । आ॒ह॒ । त्वष्टा᳚ । वै । प॒शू॒नाम् । मि॒थु॒नाना᳚म् । रू॒प॒कृदिति॑ रूप - कृत् । रू॒पम् । ए॒व । प॒शुषु॑ । द॒धा॒ति॒ । रेव॑तीः । रम॑द्ध्वम् । इति॑ । आ॒ह॒ । प॒शवः॑ । वै । रे॒वतीः᳚ । प॒शून् । ए॒व । अ॒स्मै॒ । र॒म॒य॒ति॒ । दे॒वस्य॑ । त्वा॒ । स॒वि॒तुः । प॒स॒व इति॑ प्र - स॒वे । इति॑ ।  \newline


\textbf{Krama Paata} \newline

आ॒ह॒ ब्रह्म॑ । ब्रह्म॒ वै । वै दे॒वाना᳚म् । दे॒वाना॒म् बृह॒स्पतिः॑ । बृह॒स्पति॒र् ब्रह्म॑णा । ब्रह्म॑णै॒व । ए॒वास्मै᳚ । अ॒स्मै॒ प॒शून् । प॒शूनव॑ । अव॑ रुन्धे । रु॒न्धे॒ ह॒व्या । ह॒व्या ते᳚ । ते॒ स्व॒द॒न्ता॒म् । स्व॒द॒न्ता॒मिति॑ । इत्या॑ह । आ॒ह॒ स्व॒दय॑ति । स्व॒दय॑त्यै॒व । ए॒वैनान्॑ । ए॒ना॒न् देव॑ । देव॑ त्वष्टः । त्व॒ष्ट॒र् वसु॑ । वसु॑ रण्व । र॒ण्वेति॑ । इत्या॑ह । आ॒ह॒ त्वष्टा᳚ । त्वष्टा॒ वै । वै प॑शू॒नाम् । प॒शू॒नाम् मि॑थु॒नाना᳚म् । मि॒थु॒नानाꣳ॑ रूप॒कृत् । रू॒प॒कृद् रू॒पम् । रू॒प॒कृदिति॑ रूप - कृत् । रू॒पमे॒व । ए॒व प॒शुषु॑ । प॒शुषु॑ दधाति । द॒धा॒ति॒ रेव॑तीः । रेव॑ती॒ रम॑द्ध्वम् । रम॑द्ध्व॒मिति॑ । इत्या॑ह । आ॒ह॒ प॒शवः॑ । प॒शवो॒ वै । वै रे॒वतीः᳚ । रे॒वतीः᳚ प॒शून् । प॒शूने॒व । ए॒वास्मै᳚ । अ॒स्मै॒ र॒म॒य॒ति॒ । र॒म॒य॒ति॒ दे॒वस्य॑ । दे॒वस्य॑ त्वा । त्वा॒ स॒वि॒तुः । स॒वि॒तुः प्र॑स॒वे । प्र॒स॒व इति॑ । प्र॒स॒व इति॑ प्र - स॒वे । इति॑ रश॒नाम् \newline

\textbf{Jatai Paata} \newline

1. आ॒ह॒ ब्रह्म॒ ब्रह्मा॑हाह॒ ब्रह्म॑ । \newline
2. ब्रह्म॒ वै वै ब्रह्म॒ ब्रह्म॒ वै । \newline
3. वै दे॒वाना᳚म् दे॒वानां॒ ॅवै वै दे॒वाना᳚म् । \newline
4. दे॒वाना॒म् बृह॒स्पति॒र् बृह॒स्पति॑र् दे॒वाना᳚म् दे॒वाना॒म् बृह॒स्पतिः॑ । \newline
5. बृह॒स्पति॒र् ब्रह्म॑णा॒ ब्रह्म॑णा॒ बृह॒स्पति॒र् बृह॒स्पति॒र् ब्रह्म॑णा । \newline
6. ब्रह्म॑णै॒ वैव ब्रह्म॑णा॒ ब्रह्म॑णै॒व । \newline
7. ए॒वास्मा॑ अस्मा ए॒वै वास्मै᳚ । \newline
8. अ॒स्मै॒ प॒शून् प॒शू न॑स्मा अस्मै प॒शून् । \newline
9. प॒शून वाव॑ प॒शून् प॒शूनव॑ । \newline
10. अव॑ रुन्धे रु॒न्धे ऽवाव॑ रुन्धे । \newline
11. रु॒न्धे॒ ह॒व्या ह॒व्या रु॑न्धे रुन्धे ह॒व्या । \newline
12. ह॒व्या ते॑ ते ह॒व्या ह॒व्या ते᳚ । \newline
13. ते॒ स्व॒द॒न्ताꣳ॒॒ स्व॒द॒न्ता॒म् ते॒ ते॒ स्व॒द॒न्ता॒म् । \newline
14. स्व॒द॒न्ता॒ मितीति॑ स्वदन्ताꣳ स्वदन्ता॒ मिति॑ । \newline
15. इत्या॑हा॒हे तीत्या॑ह । \newline
16. आ॒ह॒ स्व॒दय॑ति स्व॒दय॑ त्याहाह स्व॒दय॑ति । \newline
17. स्व॒दय॑ त्ये॒वैव स्व॒दय॑ति स्व॒दय॑ त्ये॒व । \newline
18. ए॒वैना॑ नेना ने॒वै वैनान्॑ । \newline
19. ए॒ना॒न् देव॒ देवै॑ना नेना॒न् देव॑ । \newline
20. देव॑ त्वष्ट स्त्वष्ट॒र् देव॒ देव॑ त्वष्टः । \newline
21. त्व॒ष्ट॒र् वसु॒ वसु॑ त्वष्ट स्त्वष्ट॒र् वसु॑ । \newline
22. वसु॑ रण्व रण्व॒ वसु॒ वसु॑ रण्व । \newline
23. र॒ण्वे तीति॑ रण्व र॒ण्वेति॑ । \newline
24. इत्या॑हा॒हे तीत्या॑ह । \newline
25. आ॒ह॒ त्वष्टा॒ त्वष्टा॑ ऽऽहाह॒ त्वष्टा᳚ । \newline
26. त्वष्टा॒ वै वै त्वष्टा॒ त्वष्टा॒ वै । \newline
27. वै प॑शू॒नाम् प॑शू॒नां ॅवै वै प॑शू॒नाम् । \newline
28. प॒शू॒नाम् मि॑थु॒नाना᳚म् मिथु॒नाना᳚म् पशू॒नाम् प॑शू॒नाम् मि॑थु॒नाना᳚म् । \newline
29. मि॒थु॒नानाꣳ॑ रूप॒कृद् रू॑प॒कृन् मि॑थु॒नाना᳚म् मिथु॒नानाꣳ॑ रूप॒कृत् । \newline
30. रू॒प॒कृद् रू॒पꣳ रू॒पꣳ रू॑प॒कृद् रू॑प॒कृद् रू॒पम् । \newline
31. रू॒प॒कृदिति॑ रूप - कृत् । \newline
32. रू॒प मे॒वैव रू॒पꣳ रू॒प मे॒व । \newline
33. ए॒व प॒शुषु॑ प॒शु ष्वे॒वैव प॒शुषु॑ । \newline
34. प॒शुषु॑ दधाति दधाति प॒शुषु॑ प॒शुषु॑ दधाति । \newline
35. द॒धा॒ति॒ रेव॑ती॒ रेव॑तीर् दधाति दधाति॒ रेव॑तीः । \newline
36. रेव॑ती॒ रम॑द्ध्वꣳ॒॒ रम॑द्ध्वꣳ॒॒ रेव॑ती॒ रेव॑ती॒ रम॑द्ध्वम् । \newline
37. रम॑द्ध्व॒ मितीति॒ रम॑द्ध्वꣳ॒॒ रम॑द्ध्व॒ मिति॑ । \newline
38. इत्या॑हा॒हे तीत्या॑ह । \newline
39. आ॒ह॒ प॒शवः॑ प॒शव॑ आहाह प॒शवः॑ । \newline
40. प॒शवो॒ वै वै प॒शवः॑ प॒शवो॒ वै । \newline
41. वै रे॒वती॑ रे॒वती॒र् वै वै रे॒वतीः᳚ । \newline
42. रे॒वतीः᳚ प॒शून् प॒शून् रे॒वती॑ रे॒वतीः᳚ प॒शून् । \newline
43. प॒शूने॒ वैव प॒शून् प॒शूने॒व । \newline
44. ए॒वास्मा॑ अस्मा ए॒वै वास्मै᳚ । \newline
45. अ॒स्मै॒ र॒म॒य॒ति॒ र॒म॒य॒ त्य॒स्मा॒ अ॒स्मै॒ र॒म॒य॒ति॒ । \newline
46. र॒म॒य॒ति॒ दे॒वस्य॑ दे॒वस्य॑ रमयति रमयति दे॒वस्य॑ । \newline
47. दे॒वस्य॑ त्वा त्वा दे॒वस्य॑ दे॒वस्य॑ त्वा । \newline
48. त्वा॒ स॒वि॒तुः स॑वि॒तु स्त्वा᳚ त्वा सवि॒तुः । \newline
49. स॒वि॒तुः प्र॑स॒वे प्र॑स॒वे स॑वि॒तुः स॑वि॒तुः प्र॑स॒वे । \newline
50. प्र॒स॒व इतीति॑ प्रस॒वे प्र॑स॒व इति॑ । \newline
51. प्र॒स॒व इति॑ प्र - स॒वे । \newline
52. इति॑ रश॒नाꣳ र॑श॒ना मितीति॑ रश॒नाम् । \newline

\textbf{Ghana Paata } \newline

1. आ॒ह॒ ब्रह्म॒ ब्रह्मा॑ हाह॒ ब्रह्म॒ वै वै ब्रह्मा॑ हाह॒ ब्रह्म॒ वै । \newline
2. ब्रह्म॒ वै वै ब्रह्म॒ ब्रह्म॒ वै दे॒वाना᳚म् दे॒वानां॒ ॅवै ब्रह्म॒ ब्रह्म॒ वै दे॒वाना᳚म् । \newline
3. वै दे॒वाना᳚म् दे॒वानां॒ ॅवै वै दे॒वाना॒म् बृह॒स्पति॒र् बृह॒स्पति॑र् दे॒वानां॒ ॅवै वै दे॒वाना॒म् बृह॒स्पतिः॑ । \newline
4. दे॒वाना॒म् बृह॒स्पति॒र् बृह॒स्पति॑र् दे॒वाना᳚म् दे॒वाना॒म् बृह॒स्पति॒र् ब्रह्म॑णा॒ ब्रह्म॑णा॒ बृह॒स्पति॑र् दे॒वाना᳚म् दे॒वाना॒म् बृह॒स्पति॒र् ब्रह्म॑णा । \newline
5. बृह॒स्पति॒र् ब्रह्म॑णा॒ ब्रह्म॑णा॒ बृह॒स्पति॒र् बृह॒स्पति॒र् ब्रह्म॑ णै॒वैव ब्रह्म॑णा॒ बृह॒स्पति॒र् बृह॒स्पति॒र् ब्रह्म॑णै॒व । \newline
6. ब्रह्म॑ णै॒वैव ब्रह्म॑णा॒ ब्रह्म॑णै॒ वास्मा॑ अस्मा ए॒व ब्रह्म॑णा॒ ब्रह्म॑ णै॒वास्मै᳚ । \newline
7. ए॒वास्मा॑ अस्मा ए॒वै वास्मै॑ प॒शून् प॒शून॑स्मा ए॒वै वास्मै॑ प॒शून् । \newline
8. अ॒स्मै॒ प॒शून् प॒शून॑स्मा अस्मै प॒शून वाव॑ प॒शून॑स्मा अस्मै प॒शूनव॑ । \newline
9. प॒शून वाव॑ प॒शून् प॒शूनव॑ रुन्धे रु॒न्धे ऽव॑ प॒शून् प॒शूनव॑ रुन्धे । \newline
10. अव॑ रुन्धे रु॒न्धे ऽवाव॑ रुन्धे ह॒व्या ह॒व्या रु॒न्धे ऽवाव॑ रुन्धे ह॒व्या । \newline
11. रु॒न्धे॒ ह॒व्या ह॒व्या रु॑न्धे रुन्धे ह॒व्या ते॑ ते ह॒व्या रु॑न्धे रुन्धे ह॒व्या ते᳚ । \newline
12. ह॒व्या ते॑ ते ह॒व्या ह॒व्या ते᳚ स्वदन्ताꣳ स्वदन्ताम् ते ह॒व्या ह॒व्या ते᳚ स्वदन्ताम् । \newline
13. ते॒ स्व॒द॒न्ताꣳ॒॒ स्व॒द॒न्ता॒म् ते॒ ते॒ स्व॒द॒न्ता॒ मितीति॑ स्वदन्ताम् ते ते स्वदन्ता॒ मिति॑ । \newline
14. स्व॒द॒न्ता॒ मितीति॑ स्वदन्ताꣳ स्वदन्ता॒ मित्या॑हा॒ हेति॑ स्वदन्ताꣳ स्वदन्ता॒ मित्या॑ह । \newline
15. इत्या॑हा॒हे तीत्या॑ह स्व॒दय॑ति स्व॒दय॑ त्या॒हे तीत्या॑ह स्व॒दय॑ति । \newline
16. आ॒ह॒ स्व॒दय॑ति स्व॒दय॑ त्याहाह स्व॒दय॑ त्ये॒वैव स्व॒दय॑ त्याहाह स्व॒दय॑ त्ये॒व । \newline
17. स्व॒दय॑ त्ये॒वैव स्व॒दय॑ति स्व॒दय॑ त्ये॒वैना॑ नेना ने॒व स्व॒दय॑ति स्व॒दय॑ त्ये॒वैनान्॑ । \newline
18. ए॒वैना॑ नेना ने॒वै वैना॒न् देव॒ देवै॑ना ने॒वै वैना॒न् देव॑ । \newline
19. ए॒ना॒न् देव॒ देवै॑ना नेना॒न् देव॑ त्वष्ट स्त्वष्ट॒र् देवै॑ना नेना॒न् देव॑ त्वष्टः । \newline
20. देव॑ त्वष्ट स्त्वष्ट॒र् देव॒ देव॑ त्वष्ट॒र् वसु॒ वसु॑ त्वष्ट॒र् देव॒ देव॑ त्वष्ट॒र् वसु॑ । \newline
21. त्व॒ष्ट॒र् वसु॒ वसु॑ त्वष्ट स्त्वष्ट॒र् वसु॑ रण्व रण्व॒ वसु॑ त्वष्ट स्त्वष्ट॒र् वसु॑ रण्व । \newline
22. वसु॑ रण्व रण्व॒ वसु॒ वसु॑ र॒ण्वे तीति॑ रण्व॒ वसु॒ वसु॑ र॒ण्वेति॑ । \newline
23. र॒ण्वे तीति॑ रण्व र॒ण्वे त्या॑हा॒हेति॑ रण्व र॒ण्वे त्या॑ह । \newline
24. इत्या॑हा॒हे तीत्या॑ह॒ त्वष्टा॒ त्वष्टा॒ ऽऽहे तीत्या॑ह॒ त्वष्टा᳚ । \newline
25. आ॒ह॒ त्वष्टा॒ त्वष्टा॑ ऽऽहाह॒ त्वष्टा॒ वै वै त्वष्टा॑ ऽऽहाह॒ त्वष्टा॒ वै । \newline
26. त्वष्टा॒ वै वै त्वष्टा॒ त्वष्टा॒ वै प॑शू॒नाम् प॑शू॒नां ॅवै त्वष्टा॒ त्वष्टा॒ वै प॑शू॒नाम् । \newline
27. वै प॑शू॒नाम् प॑शू॒नां ॅवै वै प॑शू॒नाम् मि॑थु॒नाना᳚म् मिथु॒नाना᳚म् पशू॒नां ॅवै वै प॑शू॒नाम् मि॑थु॒नाना᳚म् । \newline
28. प॒शू॒नाम् मि॑थु॒नाना᳚म् मिथु॒नाना᳚म् पशू॒नाम् प॑शू॒नाम् मि॑थु॒नानाꣳ॑ रूप॒कृद् रू॑प॒कृन् मि॑थु॒नाना᳚म् पशू॒नाम् प॑शू॒नाम् मि॑थु॒नानाꣳ॑ रूप॒कृत् । \newline
29. मि॒थु॒नानाꣳ॑ रूप॒कृद् रू॑प॒कृन् मि॑थु॒नाना᳚म् मिथु॒नानाꣳ॑ रूप॒कृद् रू॒पꣳ रू॒पꣳ रू॑प॒कृन् मि॑थु॒नाना᳚म् मिथु॒नानाꣳ॑ रूप॒कृद् रू॒पम् । \newline
30. रू॒प॒कृद् रू॒पꣳ रू॒पꣳ रू॑प॒कृद् रू॑प॒कृद् रू॒प मे॒वैव रू॒पꣳ रू॑प॒कृद् रू॑प॒कृद् रू॒प मे॒व । \newline
31. रू॒प॒कृदिति॑ रूप - कृत् । \newline
32. रू॒प मे॒वैव रू॒पꣳ रू॒प मे॒व प॒शुषु॑ प॒शुष्वे॒व रू॒पꣳ रू॒प मे॒व प॒शुषु॑ । \newline
33. ए॒व प॒शुषु॑ प॒शुष्वे॒ वैव प॒शुषु॑ दधाति दधाति प॒शुष्वे॒ वैव प॒शुषु॑ दधाति । \newline
34. प॒शुषु॑ दधाति दधाति प॒शुषु॑ प॒शुषु॑ दधाति॒ रेव॑ती॒ रेव॑तीर् दधाति प॒शुषु॑ प॒शुषु॑ दधाति॒ रेव॑तीः । \newline
35. द॒धा॒ति॒ रेव॑ती॒ रेव॑तीर् दधाति दधाति॒ रेव॑ती॒ रम॑द्ध्वꣳ॒॒ रम॑द्ध्वꣳ॒॒ रेव॑तीर् दधाति दधाति॒ रेव॑ती॒ रम॑द्ध्वम् । \newline
36. रेव॑ती॒ रम॑द्ध्वꣳ॒॒ रम॑द्ध्वꣳ॒॒ रेव॑ती॒ रेव॑ती॒ रम॑द्ध्व॒ मितीति॒ रम॑द्ध्वꣳ॒॒ रेव॑ती॒ रेव॑ती॒ रम॑द्ध्व॒ मिति॑ । \newline
37. रम॑द्ध्व॒ मितीति॒ रम॑द्ध्वꣳ॒॒ रम॑द्ध्व॒ मित्या॑हा॒ हेति॒ रम॑द्ध्वꣳ॒॒ रम॑द्ध्व॒ मित्या॑ह । \newline
38. इत्या॑हा॒हे तीत्या॑ह प॒शवः॑ प॒शव॑ आ॒हे तीत्या॑ह प॒शवः॑ । \newline
39. आ॒ह॒ प॒शवः॑ प॒शव॑ आहाह प॒शवो॒ वै वै प॒शव॑ आहाह प॒शवो॒ वै । \newline
40. प॒शवो॒ वै वै प॒शवः॑ प॒शवो॒ वै रे॒वती॑ रे॒वती॒र् वै प॒शवः॑ प॒शवो॒ वै रे॒वतीः᳚ । \newline
41. वै रे॒वती॑ रे॒वती॒र् वै वै रे॒वतीः᳚ प॒शून् प॒शून् रे॒वती॒र् वै वै रे॒वतीः᳚ प॒शून् । \newline
42. रे॒वतीः᳚ प॒शून् प॒शून् रे॒वती॑ रे॒वतीः᳚ प॒शूने॒ वैव प॒शून् रे॒वती॑ रे॒वतीः᳚ प॒शूने॒व । \newline
43. प॒शूने॒ वैव प॒शून् प॒शूने॒ वास्मा॑ अस्मा ए॒व प॒शून् प॒शूने॒ वास्मै᳚ । \newline
44. ए॒वास्मा॑ अस्मा ए॒वै वास्मै॑ रमयति रमय त्यस्मा ए॒वै वास्मै॑ रमयति । \newline
45. अ॒स्मै॒ र॒म॒य॒ति॒ र॒म॒य॒ त्य॒स्मा॒ अ॒स्मै॒ र॒म॒य॒ति॒ दे॒वस्य॑ दे॒वस्य॑ रमय त्यस्मा अस्मै रमयति दे॒वस्य॑ । \newline
46. र॒म॒य॒ति॒ दे॒वस्य॑ दे॒वस्य॑ रमयति रमयति दे॒वस्य॑ त्वा त्वा दे॒वस्य॑ रमयति रमयति दे॒वस्य॑ त्वा । \newline
47. दे॒वस्य॑ त्वा त्वा दे॒वस्य॑ दे॒वस्य॑ त्वा सवि॒तुः स॑वि॒तु स्त्वा॑ दे॒वस्य॑ दे॒वस्य॑ त्वा सवि॒तुः । \newline
48. त्वा॒ स॒वि॒तुः स॑वि॒तु स्त्वा᳚ त्वा सवि॒तुः प्र॑स॒वे प्र॑स॒वे स॑वि॒तु स्त्वा᳚ त्वा सवि॒तुः प्र॑स॒वे । \newline
49. स॒वि॒तुः प्र॑स॒वे प्र॑स॒वे स॑वि॒तुः स॑वि॒तुः प्र॑स॒व इतीति॑ प्रस॒वे स॑वि॒तुः स॑वि॒तुः प्र॑स॒व इति॑ । \newline
50. प्र॒स॒व इतीति॑ प्रस॒वे प्र॑स॒व इति॑ रश॒नाꣳ र॑श॒ना मिति॑ प्रस॒वे प्र॑स॒व इति॑ रश॒नाम् । \newline
51. प्र॒स॒व इति॑ प्र - स॒वे । \newline
52. इति॑ रश॒नाꣳ र॑श॒ना मितीति॑ रश॒ना मा र॑श॒ना मितीति॑ रश॒ना मा । \newline
\pagebreak
\markright{ TS 6.3.6.3  \hfill https://www.vedavms.in \hfill}

\section{ TS 6.3.6.3 }

\textbf{TS 6.3.6.3 } \newline
\textbf{Samhita Paata} \newline

रश॒नामा द॑त्ते॒ प्रसू᳚त्या अ॒श्विनो᳚र्बा॒हुभ्या॒-मित्या॑हा॒श्विनौ॒ हि दे॒वाना॑मद्ध्व॒र्यू आस्तां᳚ पू॒ष्णो हस्ता᳚भ्या॒मित्या॑ह॒ यत्या॑ ऋ॒तस्य॑ त्वा देवहविः॒ पाशे॒नाऽऽ* र॑भ॒ इत्या॑ह स॒त्यं ॅवा ऋ॒तꣳ स॒त्येनै॒वैन॑मृ॒तेना ऽऽ*र॑भते ऽक्ष्ण॒या परि॑ हरति॒ वद्ध्यꣳ॒॒ हि प्र॒त्यञ्चं॑ प्रति मु॒ञ्चन्ति॒ व्यावृ॑त्त्यै॒ धर्.षा॒ मानु॑षा॒निति॒ नि यु॑नक्ति॒ धृत्या॑ अ॒द्भ्य- [  ] \newline

\textbf{Pada Paata} \newline

र॒श॒नाम् । एति॑ । द॒त्ते॒ । प्रसू᳚त्या॒ इति॒ प्र - सू॒त्यै॒ । अ॒श्विनोः᳚ । बा॒हुभ्या॒मिति॑ बा॒हु-भ्या॒म् । इति॑ । आ॒ह॒ । अ॒श्विनौ᳚ । हि । दे॒वाना᳚म् । अ॒द्ध्व॒र्यू इति॑ । आस्ता᳚म् । पू॒ष्णः । हस्ता᳚भ्याम् । इति॑ । आ॒ह॒ । यत्यै᳚ । ऋ॒तस्य॑ । त्वा॒ । दे॒व॒ह॒वि॒रिति॑ देव - ह॒विः॒ । पाशे॑न । एति॑ । र॒भे॒ । इति॑ । आ॒ह॒ । स॒त्यम् । वै । ऋ॒तम् । स॒त्येन॑ । ए॒व । ए॒न॒म् । ऋ॒तेन॑ । एति॑ । र॒भ॒ते॒ । अ॒क्ष्ण॒या । परीति॑ । ह॒र॒ति॒ । वद्ध्य᳚म् । हि । प्र॒त्यञ्च᳚म् । प्र॒ति॒मु॒ञ्चन्तीति॑ प्रति - मु॒ञ्चन्ति॑ । व्यावृ॑त्त्या॒ इति॑ वि - आवृ॑त्त्यै । धर्.ष॑ । मानु॑षान् । इति॑ । नीति॑ । यु॒न॒क्ति॒ । धृत्यै᳚ । अ॒द्भ्य इत्य॑त् - भ्यः ।  \newline


\textbf{Krama Paata} \newline

र॒श॒नामा । आ द॑त्ते । द॒त्ते॒ प्रसू᳚त्यै । प्रसू᳚त्या अ॒श्विनोः᳚ । प्रसू᳚त्या॒ इति॒ प्र - सू॒त्यै॒ । अ॒श्विनो᳚र् बा॒हुभ्या᳚म् । बा॒हुभ्या॒मिति॑ । बा॒हुभ्या॒मिति॑ बा॒हु - भ्या॒म् । इत्या॑ह । आ॒हा॒श्विनौ᳚ । अ॒श्विनौ॒ हि । हि दे॒वाना᳚म् । दे॒वाना॑मद्ध्व॒र्यू । अ॒द्ध्व॒र्यू आस्ता᳚म् । अ॒द्ध्व॒र्यू इत्य॑द्ध्व॒र्यू । आस्ता᳚म् पू॒ष्णः । पू॒ष्णो हस्ता᳚भ्याम् । हस्ता᳚भ्या॒मिति॑ । इत्या॑ह । आ॒ह॒ यत्यै᳚ । यत्या॑ ऋ॒तस्य॑ । ऋ॒तस्य॑ त्वा । त्वा॒ दे॒व॒ह॒विः॒ । दे॒व॒ह॒विः॒ पाशे॑न । दे॒व॒ह॒वि॒रिति॑ देव - ह॒विः॒ । पाशे॒ना । आ र॑भे । र॒भ॒ इति॑ । इत्या॑ह । आ॒ह॒ स॒त्यम् । स॒त्यम् ॅवै । वा ऋ॒तम् । ऋ॒तꣳ स॒त्येन॑ । स॒त्येनै॒व । ए॒वैन᳚म् । ए॒न॒मृ॒तेन॑ । ऋ॒तेना । आ र॑भते । र॒भ॒ते॒ऽक्ष्ण॒या । अ॒क्ष्ण॒या परि॑ । परि॑ हरति । ह॒र॒ति॒ वद्ध्य᳚म् । वद्ध्यꣳ॒॒ हि । हि प्र॒त्यञ्च᳚म् । प्र॒त्यञ्च॑म् प्रतिमु॒ञ्चन्ति॑ । प्र॒ति॒मु॒ञ्चन्ति॒ व्यावृ॑त्त्यै । प्र॒ति॒मु॒ञ्चन्तीति॑ प्रति - मु॒ञ्चन्ति॑ । व्यावृ॑त्त्यै॒ धर्.ष॑ । व्यावृ॑त्त्या॒ इति॑ वि - आवृ॑त्त्यै । धर्.षा॒ मानु॑षान् । मानु॑षा॒निति॑ । इति॒ नि । नि यु॑नक्ति । यु॒न॒क्ति॒ धृत्यै᳚ । धृत्या॑ अ॒द्भ्यः । अ॒द्भ्यस्त्वा᳚ । अ॒द्‍भ्य इत्य॑त् - भ्यः \newline

\textbf{Jatai Paata} \newline

1. र॒श॒ना मा र॑श॒नाꣳ र॑श॒ना मा । \newline
2. आ द॑त्ते दत्त॒ आ द॑त्ते । \newline
3. द॒त्ते॒ प्रसू᳚त्यै॒ प्रसू᳚त्यै दत्ते दत्ते॒ प्रसू᳚त्यै । \newline
4. प्रसू᳚त्या अ॒श्विनो॑ र॒श्विनोः॒ प्रसू᳚त्यै॒ प्रसू᳚त्या अ॒श्विनोः᳚ । \newline
5. प्रसू᳚त्या॒ इति॒ प्र - सू॒त्यै॒ । \newline
6. अ॒श्विनो᳚र् बा॒हुभ्या᳚म् बा॒हुभ्या॑ म॒श्विनो॑ र॒श्विनो᳚र् बा॒हुभ्या᳚म् । \newline
7. बा॒हुभ्या॒ मितीति॑ बा॒हुभ्या᳚म् बा॒हुभ्या॒ मिति॑ । \newline
8. बा॒हुभ्या॒मिति॑ बा॒हु - भ्या॒म् । \newline
9. इत्या॑हा॒हे तीत्या॑ह । \newline
10. आ॒हा॒श्विना॑ व॒श्विना॑ वाहा हा॒श्विनौ᳚ । \newline
11. अ॒श्विनौ॒ हि ह्य॑श्विना॑ व॒श्विनौ॒ हि । \newline
12. हि दे॒वाना᳚म् दे॒वानाꣳ॒॒ हि हि दे॒वाना᳚म् । \newline
13. दे॒वाना॑ मद्ध्व॒र्यू अ॑द्ध्व॒र्यू दे॒वाना᳚म् दे॒वाना॑ मद्ध्व॒र्यू । \newline
14. अ॒द्ध्व॒र्यू आस्ता॒ मास्ता॑ मद्ध्व॒र्यू अ॑द्ध्व॒र्यू आस्ता᳚म् । \newline
15. अ॒द्ध्व॒र्यू इत्य॑द्ध्व॒र्यू । \newline
16. आस्ता᳚म् पू॒ष्णः पू॒ष्ण आस्ता॒ मास्ता᳚म् पू॒ष्णः । \newline
17. पू॒ष्णो हस्ता᳚भ्याꣳ॒॒ हस्ता᳚भ्याम् पू॒ष्णः पू॒ष्णो हस्ता᳚भ्याम् । \newline
18. हस्ता᳚भ्या॒ मितीति॒ हस्ता᳚भ्याꣳ॒॒ हस्ता᳚भ्या॒ मिति॑ । \newline
19. इत्या॑हा॒हे तीत्या॑ह । \newline
20. आ॒ह॒ यत्यै॒ यत्या॑ आहाह॒ यत्यै᳚ । \newline
21. यत्या॑ ऋ॒तस्य॒ र्‌तस्य॒ यत्यै॒ यत्या॑ ऋ॒तस्य॑ । \newline
22. ऋ॒तस्य॑ त्वा त्व॒ र्तस्य॒ र्‌तस्य॑ त्वा । \newline
23. त्वा॒ दे॒व॒ह॒वि॒र् दे॒व॒ह॒वि॒ स्त्वा॒ त्वा॒ दे॒व॒ह॒विः॒ । \newline
24. दे॒व॒ह॒विः॒ पाशे॑न॒ पाशे॑न देवहविर् देवहविः॒ पाशे॑न । \newline
25. दे॒व॒ह॒वि॒रिति॑ देव - ह॒विः॒ । \newline
26. पाशे॒ना पाशे॑न॒ पाशे॒ना । \newline
27. आ र॑भे रभ॒ आ र॑भे । \newline
28. र॒भ॒ इतीति॑ रभे रभ॒ इति॑ । \newline
29. इत्या॑हा॒हे तीत्या॑ह । \newline
30. आ॒ह॒ स॒त्यꣳ स॒त्य मा॑हाह स॒त्यम् । \newline
31. स॒त्यं ॅवै वै स॒त्यꣳ स॒त्यं ॅवै । \newline
32. वा ऋ॒त मृ॒तं ॅवै वा ऋ॒तम् । \newline
33. ऋ॒तꣳ स॒त्येन॑ स॒त्येन॒ र्‌त मृ॒तꣳ स॒त्येन॑ । \newline
34. स॒त्येनै॒ वैव स॒त्येन॑ स॒त्येनै॒व । \newline
35. ए॒वैन॑ मेन मे॒वै वैन᳚म् । \newline
36. ए॒न॒ मृ॒तेन॒ र्‌तेनै॑न मेन मृ॒तेन॑ । \newline
37. ऋ॒तेना॒ र्‌तेन॒ र्‌तेना । \newline
38. आ र॑भते रभत॒ आ र॑भते । \newline
39. र॒भ॒ते॒ ऽक्ष्ण॒या ऽक्ष्ण॒या र॑भते रभते ऽक्ष्ण॒या । \newline
40. अ॒क्ष्ण॒या परि॒ पर्य॑क्ष्ण॒या ऽक्ष्ण॒या परि॑ । \newline
41. परि॑ हरति हरति॒ परि॒ परि॑ हरति । \newline
42. ह॒र॒ति॒ वद्ध्यं॒ ॅवद्ध्यꣳ॑ हरति हरति॒ वद्ध्य᳚म् । \newline
43. वद्ध्यꣳ॒॒ हि हि वद्ध्यं॒ ॅवद्ध्यꣳ॒॒ हि । \newline
44. हि प्र॒त्यञ्च॑म् प्र॒त्यञ्चꣳ॒॒ हि हि प्र॒त्यञ्च᳚म् । \newline
45. प्र॒त्यञ्च॑म् प्रतिमु॒ञ्चन्ति॑ प्रतिमु॒ञ्चन्ति॑ प्र॒त्यञ्च॑म् प्र॒त्यञ्च॑म् प्रतिमु॒ञ्चन्ति॑ । \newline
46. प्र॒ति॒मु॒ञ्चन्ति॒ व्यावृ॑त्त्यै॒ व्यावृ॑त्त्यै प्रतिमु॒ञ्चन्ति॑ प्रतिमु॒ञ्चन्ति॒ व्यावृ॑त्त्यै । \newline
47. प्र॒ति॒मु॒ञ्चन्तीति॑ प्रति - मु॒ञ्चन्ति॑ । \newline
48. व्यावृ॑त्त्यै॒ धर्.ष॒ धर्.ष॒ व्यावृ॑त्त्यै॒ व्यावृ॑त्त्यै॒ धर्.ष॑ । \newline
49. व्यावृ॑त्त्या॒ इति॑ वि - आवृ॑त्त्यै । \newline
50. धर्.षा॒ मानु॑षा॒न् मानु॑षा॒न् धर्.ष॒ धर्.षा॒ मानु॑षान् । \newline
51. मानु॑षा॒नि तीति॒ मानु॑षा॒न् मानु॑षा॒ निति॑ । \newline
52. इति॒ नि नीतीति॒ नि । \newline
53. नि यु॑नक्ति युनक्ति॒ नि नि यु॑नक्ति । \newline
54. यु॒न॒क्ति॒ धृत्यै॒ धृत्यै॑ युनक्ति युनक्ति॒ धृत्यै᳚ । \newline
55. धृत्या॑ अ॒द्भ्यो᳚ ऽद्भ्यो धृत्यै॒ धृत्या॑ अ॒द्भ्यः । \newline
56. अ॒द्भ्य स्त्वा᳚ त्वा॒ ऽद्भ्यो᳚ ऽद्भ्य स्त्वा᳚ । \newline
57. अ॒द्भ्य इत्य॑त् - भ्यः । \newline

\textbf{Ghana Paata } \newline

1. र॒श॒ना मा र॑श॒नाꣳ र॑श॒ना मा द॑त्ते दत्त॒ आ र॑श॒नाꣳ र॑श॒ना मा द॑त्ते । \newline
2. आ द॑त्ते दत्त॒ आ द॑त्ते॒ प्रसू᳚त्यै॒ प्रसू᳚त्यै दत्त॒ आ द॑त्ते॒ प्रसू᳚त्यै । \newline
3. द॒त्ते॒ प्रसू᳚त्यै॒ प्रसू᳚त्यै दत्ते दत्ते॒ प्रसू᳚त्या अ॒श्विनो॑ र॒श्विनोः॒ प्रसू᳚त्यै दत्ते दत्ते॒ प्रसू᳚त्या अ॒श्विनोः᳚ । \newline
4. प्रसू᳚त्या अ॒श्विनो॑ र॒श्विनोः॒ प्रसू᳚त्यै॒ प्रसू᳚त्या अ॒श्विनो᳚र् बा॒हुभ्या᳚म् बा॒हुभ्या॑ म॒श्विनोः॒ प्रसू᳚त्यै॒ प्रसू᳚त्या अ॒श्विनो᳚र् बा॒हुभ्या᳚म् । \newline
5. प्रसू᳚त्या॒ इति॒ प्र - सू॒त्यै॒ । \newline
6. अ॒श्विनो᳚र् बा॒हुभ्या᳚म् बा॒हुभ्या॑ म॒श्विनो॑ र॒श्विनो᳚र् बा॒हुभ्या॒ मितीति॑ बा॒हुभ्या॑ म॒श्विनो॑ र॒श्विनो᳚र् बा॒हुभ्या॒ मिति॑ । \newline
7. बा॒हुभ्या॒ मितीति॑ बा॒हुभ्या᳚म् बा॒हुभ्या॒ मित्या॑हा॒हेति॑ बा॒हुभ्या᳚म् बा॒हुभ्या॒ मित्या॑ह । \newline
8. बा॒हुभ्या॒मिति॑ बा॒हु - भ्या॒म् । \newline
9. इत्या॑हा॒हे तीत्या॑ हा॒श्विना॑ व॒श्विना॑ वा॒हे तीत्या॑ हा॒श्विनौ᳚ । \newline
10. आ॒हा॒श्विना॑ व॒श्विना॑ वाहा हा॒श्विनौ॒ हि ह्य॑श्विना॑ वाहा हा॒श्विनौ॒ हि । \newline
11. अ॒श्विनौ॒ हि ह्य॑श्विना॑ व॒श्विनौ॒ हि दे॒वाना᳚म् दे॒वानाꣳ॒॒ ह्य॑श्विना॑ व॒श्विनौ॒ हि दे॒वाना᳚म् । \newline
12. हि दे॒वाना᳚म् दे॒वानाꣳ॒॒ हि हि दे॒वाना॑ मद्ध्व॒र्यू अ॑द्ध्व॒र्यू दे॒वानाꣳ॒॒ हि हि दे॒वाना॑ मद्ध्व॒र्यू । \newline
13. दे॒वाना॑ मद्ध्व॒र्यू अ॑द्ध्व॒र्यू दे॒वाना᳚म् दे॒वाना॑ मद्ध्व॒र्यू आस्ता॒ मास्ता॑ मद्ध्व॒र्यू दे॒वाना᳚म् दे॒वाना॑ मद्ध्व॒र्यू आस्ता᳚म् । \newline
14. अ॒द्ध्व॒र्यू आस्ता॒ मास्ता॑ मद्ध्व॒र्यू अ॑द्ध्व॒र्यू आस्ता᳚म् पू॒ष्णः पू॒ष्ण आस्ता॑ मद्ध्व॒र्यू अ॑द्ध्व॒र्यू आस्ता᳚म् पू॒ष्णः । \newline
15. अ॒द्ध्व॒र्यू इत्य॑द्ध्व॒र्यू । \newline
16. आस्ता᳚म् पू॒ष्णः पू॒ष्ण आस्ता॒ मास्ता᳚म् पू॒ष्णो हस्ता᳚भ्याꣳ॒॒ हस्ता᳚भ्याम् पू॒ष्ण आस्ता॒ मास्ता᳚म् पू॒ष्णो हस्ता᳚भ्याम् । \newline
17. पू॒ष्णो हस्ता᳚भ्याꣳ॒॒ हस्ता᳚भ्याम् पू॒ष्णः पू॒ष्णो हस्ता᳚भ्या॒ मितीति॒ हस्ता᳚भ्याम् पू॒ष्णः पू॒ष्णो हस्ता᳚भ्या॒ मिति॑ । \newline
18. हस्ता᳚भ्या॒ मितीति॒ हस्ता᳚भ्याꣳ॒॒ हस्ता᳚भ्या॒ मित्या॑हा॒ हेति॒ हस्ता᳚भ्याꣳ॒॒ हस्ता᳚भ्या॒ मित्या॑ह । \newline
19. इत्या॑हा॒हे तीत्या॑ह॒ यत्यै॒ यत्या॑ आ॒हे तीत्या॑ह॒ यत्यै᳚ । \newline
20. आ॒ह॒ यत्यै॒ यत्या॑ आहाह॒ यत्या॑ ऋ॒तस्य॒ र्‌तस्य॒ यत्या॑ आहाह॒ यत्या॑ ऋ॒तस्य॑ । \newline
21. यत्या॑ ऋ॒तस्य॒ र्‌तस्य॒ यत्यै॒ यत्या॑ ऋ॒तस्य॑ त्वा त्व॒ र्‌तस्य॒ यत्यै॒ यत्या॑ ऋ॒तस्य॑ त्वा । \newline
22. ऋ॒तस्य॑ त्वा त्व॒ र्‌तस्य॒ र्‌तस्य॑ त्वा देवहविर् देवहवि स्त्व॒ र्‌तस्य॒ र्‌तस्य॑ त्वा देवहविः । \newline
23. त्वा॒ दे॒व॒ह॒वि॒र् दे॒व॒ह॒वि॒ स्त्वा॒ त्वा॒ दे॒व॒ह॒विः॒ पाशे॑न॒ पाशे॑न देवहवि स्त्वा त्वा देवहविः॒ पाशे॑न । \newline
24. दे॒व॒ह॒विः॒ पाशे॑न॒ पाशे॑न देवहविर् देवहविः॒ पाशे॒ना पाशे॑न देवहविर् देवहविः॒ पाशे॒ना । \newline
25. दे॒व॒ह॒वि॒रिति॑ देव - ह॒विः॒ । \newline
26. पाशे॒ना पाशे॑न॒ पाशे॒ना र॑भे रभ॒ आ पाशे॑न॒ पाशे॒ना र॑भे । \newline
27. आ र॑भे रभ॒ आ र॑भ॒ इतीति॑ रभ॒ आ र॑भ॒ इति॑ । \newline
28. र॒भ॒ इतीति॑ रभे रभ॒ इत्या॑हा॒ हेति॑ रभे रभ॒ इत्या॑ह । \newline
29. इत्या॑हा॒हे तीत्या॑ह स॒त्यꣳ स॒त्य मा॒हे तीत्या॑ह स॒त्यम् । \newline
30. आ॒ह॒ स॒त्यꣳ स॒त्य मा॑हाह स॒त्यं ॅवै वै स॒त्य मा॑हाह स॒त्यं ॅवै । \newline
31. स॒त्यं ॅवै वै स॒त्यꣳ स॒त्यं ॅवा ऋ॒त मृ॒तं ॅवै स॒त्यꣳ स॒त्यं ॅवा ऋ॒तम् । \newline
32. वा ऋ॒त मृ॒तं ॅवै वा ऋ॒तꣳ स॒त्येन॑ स॒त्येन॒ र्‌‍तं ॅवै वा ऋ॒तꣳ स॒त्येन॑ । \newline
33. ऋ॒तꣳ स॒त्येन॑ स॒त्येन॒ र्‌त मृ॒तꣳ स॒त्येनै॒ वैव स॒त्येन॒ र्‌त मृ॒तꣳ स॒त्येनै॒व । \newline
34. स॒त्येनै॒ वैव स॒त्येन॑ स॒त्ये नै॒वैन॑ मेन मे॒व स॒त्येन॑ स॒त्येनै॒ वैन᳚म् । \newline
35. ए॒वैन॑ मेन मे॒वै वैन॑ मृ॒तेन॒ र्‌तेनै॑न मे॒वै वैन॑ मृ॒तेन॑ । \newline
36. ए॒न॒ मृ॒तेन॒ र्‌तेनै॑न मेन मृ॒तेना॒ र्‌तेनै॑न मेन मृ॒तेना । \newline
37. ऋ॒तेना॒ र्‌तेन॒ र्‌तेना र॑भते रभत॒ आ र्‌तेन॒ र्‌तेना र॑भते । \newline
38. आ र॑भते रभत॒ आ र॑भते ऽक्ष्ण॒या ऽक्ष्ण॒या र॑भत॒ आ र॑भते ऽक्ष्ण॒या । \newline
39. र॒भ॒ते॒ ऽक्ष्ण॒या ऽक्ष्ण॒या र॑भते रभते ऽक्ष्ण॒या परि॒ पर्य॑क्ष्ण॒या र॑भते रभते ऽक्ष्ण॒या परि॑ । \newline
40. अ॒क्ष्ण॒या परि॒ पर्य॑क्ष्ण॒या ऽक्ष्ण॒या परि॑ हरति हरति॒ पर्य॑क्ष्ण॒या ऽक्ष्ण॒या परि॑ हरति । \newline
41. परि॑ हरति हरति॒ परि॒ परि॑ हरति॒ वद्ध्यं॒ ॅवद्ध्यꣳ॑ हरति॒ परि॒ परि॑ हरति॒ वद्ध्य᳚म् । \newline
42. ह॒र॒ति॒ वद्ध्यं॒ ॅवद्ध्यꣳ॑ हरति हरति॒ वद्ध्यꣳ॒॒ हि हि वद्ध्यꣳ॑ हरति हरति॒ वद्ध्यꣳ॒॒ हि । \newline
43. वद्ध्यꣳ॒॒ हि हि वद्ध्यं॒ ॅवद्ध्यꣳ॒॒ हि प्र॒त्यञ्च॑म् प्र॒त्यञ्चꣳ॒॒ हि वद्ध्यं॒ ॅवद्ध्यꣳ॒॒ हि प्र॒त्यञ्च᳚म् । \newline
44. हि प्र॒त्यञ्च॑म् प्र॒त्यञ्चꣳ॒॒ हि हि प्र॒त्यञ्च॑म् प्रतिमु॒ञ्चन्ति॑ प्रतिमु॒ञ्चन्ति॑ प्र॒त्यञ्चꣳ॒॒ हि हि प्र॒त्यञ्च॑म् प्रतिमु॒ञ्चन्ति॑ । \newline
45. प्र॒त्यञ्च॑म् प्रतिमु॒ञ्चन्ति॑ प्रतिमु॒ञ्चन्ति॑ प्र॒त्यञ्च॑म् प्र॒त्यञ्च॑म् प्रतिमु॒ञ्चन्ति॒ व्यावृ॑त्त्यै॒ व्यावृ॑त्त्यै प्रतिमु॒ञ्चन्ति॑ प्र॒त्यञ्च॑म् प्र॒त्यञ्च॑म् प्रतिमु॒ञ्चन्ति॒ व्यावृ॑त्त्यै । \newline
46. प्र॒ति॒मु॒ञ्चन्ति॒ व्यावृ॑त्त्यै॒ व्यावृ॑त्त्यै प्रतिमु॒ञ्चन्ति॑ प्रतिमु॒ञ्चन्ति॒ व्यावृ॑त्त्यै॒ धर्.ष॒ धर्.ष॒ व्यावृ॑त्त्यै प्रतिमु॒ञ्चन्ति॑ प्रतिमु॒ञ्चन्ति॒ व्यावृ॑त्त्यै॒ धर्.ष॑ । \newline
47. प्र॒ति॒मु॒ञ्चन्तीति॑ प्रति - मु॒ञ्चन्ति॑ । \newline
48. व्यावृ॑त्त्यै॒ धर्.ष॒ धर्.ष॒ व्यावृ॑त्त्यै॒ व्यावृ॑त्त्यै॒ धर्.षा॒ मानु॑षा॒न् मानु॑षा॒न् धर्.ष॒ व्यावृ॑त्त्यै॒ व्यावृ॑त्त्यै॒ धर्.षा॒ मानु॑षान् । \newline
49. व्यावृ॑त्त्या॒ इति॑ वि - आवृ॑त्त्यै । \newline
50. धर्.षा॒ मानु॑षा॒न् मानु॑षा॒न् धर्.ष॒ धर्.षा॒ मानु॑षा॒नि तीति॒ मानु॑षा॒न् धर्.ष॒ धर्.षा॒ मानु॑षा॒निति॑ । \newline
51. मानु॑षा॒नि तीति॒ मानु॑षा॒न् मानु॑षा॒निति॒ नि नीति॒ मानु॑षा॒न् मानु॑षा॒निति॒ नि । \newline
52. इति॒ नि नीतीति॒ नि यु॑नक्ति युनक्ति॒ नीतीति॒ नि यु॑नक्ति । \newline
53. नि यु॑नक्ति युनक्ति॒ नि नि यु॑नक्ति॒ धृत्यै॒ धृत्यै॑ युनक्ति॒ नि नि यु॑नक्ति॒ धृत्यै᳚ । \newline
54. यु॒न॒क्ति॒ धृत्यै॒ धृत्यै॑ युनक्ति युनक्ति॒ धृत्या॑ अ॒द्भ्यो᳚ ऽद्भ्यो धृत्यै॑ युनक्ति युनक्ति॒ धृत्या॑ अ॒द्भ्यः । \newline
55. धृत्या॑ अ॒द्भ्यो᳚ ऽद्भ्यो धृत्यै॒ धृत्या॑ अ॒द्भ्य स्त्वा᳚ त्वा॒ ऽद्भ्यो धृत्यै॒ धृत्या॑ अ॒द्भ्य स्त्वा᳚ । \newline
56. अ॒द्भ्य स्त्वा᳚ त्वा॒ ऽद्भ्यो᳚ ऽद्भ्य स्त्वौष॑धीभ्य॒ ओष॑धीभ्य स्त्वा॒ ऽद्भ्यो᳚ ऽद्भ्य स्त्वौष॑धीभ्यः । \newline
57. अ॒द्भ्य इत्य॑त् - भ्यः । \newline
\pagebreak
\markright{ TS 6.3.6.4  \hfill https://www.vedavms.in \hfill}

\section{ TS 6.3.6.4 }

\textbf{TS 6.3.6.4 } \newline
\textbf{Samhita Paata} \newline

स्त्वौष॑धीभ्यः॒ प्रोक्षा॒मीत्या॑हा॒द्भ्यो ह्ये॑ष ओष॑धीभ्यः स॒भंव॑ति॒ यत् प॒शुर॒पां पे॒रुर॒सीत्या॑है॒ष ह्य॑पां पा॒ता यो मेधा॑याऽऽ* र॒भ्यते᳚ स्वा॒त्तं चि॒थ् सदे॑वꣳ ह॒व्यमापो॑ देवीः॒ स्वद॑तैन॒मित्या॑ह स्व॒दय॑त्ये॒वैन॑-मु॒परि॑ष्टा॒त् प्रोक्ष॑त्यु॒परि॑ष्टादे॒वैनं॒ मेद्ध्यं॑ करोति पा॒यय॑त्यन्तर॒त ए॒वैनं॒ ( ) मेद्ध्यं॑ करोत्य॒धस्ता॒दुपो᳚क्षति स॒र्वत॑ ए॒वैनं॒ मेद्ध्यं॑ करोति ॥ \newline

\textbf{Pada Paata} \newline

त्वा॒ । ओष॑धीभ्य॒ इत्योष॑धि - भ्यः॒ । प्रेति॑ । उ॒क्षा॒मि॒ । इति॑ । आ॒ह॒ । अ॒द्भ्य इत्य॑त् - भ्यः । हि । ए॒षः । ओष॑धीभ्य॒ इत्योष॑धि - भ्यः॒ । स॒भंव॒तीति॑ सं - भव॑ति । यत् । प॒शुः । अ॒पाम् । पे॒रुः । अ॒सि॒ । इति॑ । आ॒ह॒ । ए॒षः । हि । अ॒पाम् । पा॒ता । यः । मेधा॑य । आ॒र॒भ्यत॒ इत्या᳚-र॒भ्यते᳚ । स्वा॒त्तम् । चि॒त् । सदे॑व॒मिति॒ स - दे॒व॒म् । ह॒व्यम् । आपः॑ । दे॒वीः॒ । स्वद॑त । ए॒न॒म् । इति॑ । आ॒ह॒ । स्व॒दय॑ति । ए॒व । ए॒न॒म् । उ॒परि॑ष्टात् । प्रेति॑ । उ॒क्ष॒ति॒ । उ॒परि॑ष्टात् । ए॒व । ए॒न॒म् । मेद्ध्य᳚म् । क॒रो॒ति॒ । पा॒यय॑ति । अ॒न्त॒र॒तः । ए॒व । ए॒न॒म् ( ) । मेद्ध्य᳚म् । क॒रो॒ति॒ । अ॒धस्ता᳚त् । उपेति॑ । उ॒क्ष॒ति॒ । स॒र्वतः॑ । ए॒व । ए॒न॒म् । मेद्ध्य᳚म् । क॒रो॒ति॒ ॥  \newline


\textbf{Krama Paata} \newline

त्वौष॑धीभ्यः । ओष॑धीभ्यः॒ प्र । ओष॑धीभ्य॒ इत्योष॑धि - भ्यः॒ । प्रोक्षा॑मि । उ॒क्षा॒मीति॑ । इत्या॑ह । आ॒हा॒द्भ्यः । अ॒द्भ्यो हि । अ॒द्भ्य इत्य॑त् - भ्यः । ह्ये॑षः । ए॒ष ओष॑धीभ्यः । ओष॑धीभ्यः स॒म्भव॑ति । ओष॑धीभ्य॒ इत्योष॑धि - भ्यः॒ । स॒म्भव॑ति॒ यत् । स॒म्भव॒तीति॑ सम् - भव॑ति । यत् प॒शुः । प॒शुर॒पाम् । अ॒पाम् पे॒रुः । पे॒रुर॑सि । अ॒सीति॑ । इत्या॑ह । आ॒है॒षः । ए॒ष हि । ह्य॑पाम् । अ॒पाम् पा॒ता । पा॒ता यः । यो मेधा॑य । मेधा॑यार॒भ्यते᳚ । आ॒र॒भ्यते᳚ स्वा॒त्तम् । आ॒र॒भ्यत॒ इत्या᳚ - र॒भ्यते᳚ । स्वा॒त्तम् चि॑त् । चि॒थ् सदे॑वम् । सदे॑वꣳ ह॒व्यम् । सदे॑व॒मिति॒ स - दे॒व॒म् । ह॒व्यमापः॑ । आपो॑ देवीः । दे॒वीः॒ स्वद॑त । स्वद॑तैनम् । ए॒न॒मिति॑ । इत्या॑ह । आ॒ह॒ स्व॒दय॑ति । स्व॒दय॑त्ये॒व । ए॒वैन᳚म् । ए॒न॒मु॒परि॑ष्टात् । उ॒परि॑ष्टा॒त् प्र । प्रोक्ष॑ति । उ॒क्ष॒त्यु॒परि॑ष्टात् । उ॒परि॑ष्टादे॒व । ए॒वैन᳚म् । ए॒न॒म् मेद्ध्य᳚म् । मेद्ध्य॑म् करोति । क॒रो॒ति॒ पा॒यय॑ति । पा॒यय॑त्यन्तर॒तः । अ॒न्त॒र॒त ए॒व । ए॒वैन᳚म् ( ) । ए॒न॒म् मेद्ध्य᳚म् । मेद्ध्य॑म् करोति । क॒रो॒त्य॒धस्ता᳚त् । अ॒धस्ता॒दुप॑ । उपो᳚क्षति । उ॒क्ष॒ति॒ स॒र्वतः॑ । स॒र्वत॑ ए॒व । ए॒वैन᳚म् । ए॒न॒म् मेद्ध्य᳚म् । मेद्ध्य॑म् करोति । क॒रो॒तीति॑ करोति । \newline

\textbf{Jatai Paata} \newline

1. त्वौष॑धीभ्य॒ ओष॑धीभ्य स्त्वा॒ त्वौष॑धीभ्यः । \newline
2. ओष॑धीभ्यः॒ प्र प्रौष॑धीभ्य॒ ओष॑धीभ्यः॒ प्र । \newline
3. ओष॑धीभ्य॒ इत्योष॑धि - भ्यः॒ । \newline
4. प्रोक्षा᳚ म्युक्षामि॒ प्र प्रोक्षा॑मि । \newline
5. उ॒क्षा॒ मीती त्यु॑क्षा म्युक्षा॒ मीति॑ । \newline
6. इत्या॑हा॒हे तीत्या॑ह । \newline
7. आ॒हा॒द्भ्यो᳚ ऽद्भ्य आ॑हा हा॒द्भ्यः । \newline
8. अ॒द्भ्यो हि ह्या᳚(1॒)द्भ्यो᳚ ऽद्भ्यो हि । \newline
9. अ॒द्भ्य इत्य॑त् - भ्यः । \newline
10. ह्ये॑ष ए॒ष हि ह्ये॑षः । \newline
11. ए॒ष ओष॑धीभ्य॒ ओष॑धीभ्य ए॒ष ए॒ष ओष॑धीभ्यः । \newline
12. ओष॑धीभ्यः सं॒भव॑ति सं॒भव॒ त्योष॑धीभ्य॒ ओष॑धीभ्यः सं॒भव॑ति । \newline
13. ओष॑धीभ्य॒ इत्योष॑धि - भ्यः॒ । \newline
14. सं॒भव॑ति॒ यद् यथ् सं॒भव॑ति सं॒भव॑ति॒ यत् । \newline
15. सं॒भव॒तीति॑ सं - भव॑ति । \newline
16. यत् प॒शुः प॒शुर् यद् यत् प॒शुः । \newline
17. प॒शु र॒पा म॒पाम् प॒शुः प॒शु र॒पाम् । \newline
18. अ॒पाम् पे॒रुः पे॒रु र॒पा म॒पाम् पे॒रुः । \newline
19. पे॒रु र॑स्यसि पे॒रुः पे॒रु र॑सि । \newline
20. अ॒सीती त्य॑स्य॒ सीति॑ । \newline
21. इत्या॑हा॒हे तीत्या॑ह । \newline
22. आ॒है॒ष ए॒ष आ॑हा है॒षः । \newline
23. ए॒ष हि ह्ये॑ष ए॒ष हि । \newline
24. ह्य॑पा म॒पाꣳ हि ह्य॑पाम् । \newline
25. अ॒पाम् पा॒ता पा॒ता ऽपा म॒पाम् पा॒ता । \newline
26. पा॒ता यो यः पा॒ता पा॒ता यः । \newline
27. यो मेधा॑य॒ मेधा॑य॒ यो यो मेधा॑य । \newline
28. मेधा॑या र॒भ्यत॑ आर॒भ्यते॒ मेधा॑य॒ मेधा॑या र॒भ्यते᳚ । \newline
29. आ॒र॒भ्यते᳚ स्वा॒त्तꣳ स्वा॒त्त मा॑र॒भ्यत॑ आर॒भ्यते᳚ स्वा॒त्तम् । \newline
30. आ॒र॒भ्यत॒ इत्या᳚ - र॒भ्यते᳚ । \newline
31. स्वा॒त्तम् चि॑च् चिथ् स्वा॒त्तꣳ स्वा॒त्तम् चि॑त् । \newline
32. चि॒थ् सदे॑वꣳ॒॒ सदे॑वम् चिच् चि॒थ् सदे॑वम् । \newline
33. सदे॑वꣳ ह॒व्यꣳ ह॒व्यꣳ सदे॑वꣳ॒॒ सदे॑वꣳ ह॒व्यम् । \newline
34. सदे॑व॒मिति॒ स - दे॒व॒म् । \newline
35. ह॒व्य माप॒ आपो॑ ह॒व्यꣳ ह॒व्य मापः॑ । \newline
36. आपो॑ देवीर् देवी॒ राप॒ आपो॑ देवीः । \newline
37. दे॒वीः॒ स्वद॑त॒ स्वद॑त देवीर् देवीः॒ स्वद॑त । \newline
38. स्वद॑तैन मेनꣳ॒॒ स्वद॑त॒ स्वद॑तैनम् । \newline
39. ए॒न॒ मिती त्ये॑न मेन॒ मिति॑ । \newline
40. इत्या॑हा॒हे तीत्या॑ह । \newline
41. आ॒ह॒ स्व॒दय॑ति स्व॒दय॑ त्याहाह स्व॒दय॑ति । \newline
42. स्व॒दय॑ त्ये॒वैव स्व॒दय॑ति स्व॒दय॑ त्ये॒व । \newline
43. ए॒वैन॑ मेन मे॒वै वैन᳚म् । \newline
44. ए॒न॒ मु॒परि॑ष्टा दु॒परि॑ष्टा देन मेन मु॒परि॑ष्टात् । \newline
45. उ॒परि॑ष्टा॒त् प्र प्रोपरि॑ष्टा दु॒परि॑ष्टा॒त् प्र । \newline
46. प्रोक्ष॑ त्युक्षति॒ प्र प्रोक्ष॑ति । \newline
47. उ॒क्ष॒ त्यु॒परि॑ष्टा दु॒परि॑ष्टा दुक्ष त्युक्ष त्यु॒परि॑ष्टात् । \newline
48. उ॒परि॑ष्टा दे॒वैवोपरि॑ष्टा दु॒परि॑ष्टा दे॒व । \newline
49. ए॒वैन॑ मेन मे॒वै वैन᳚म् । \newline
50. ए॒न॒म् मेद्ध्य॒म् मेद्ध्य॑ मेन मेन॒म् मेद्ध्य᳚म् । \newline
51. मेद्ध्य॑म् करोति करोति॒ मेद्ध्य॒म् मेद्ध्य॑म् करोति । \newline
52. क॒रो॒ति॒ पा॒यय॑ति पा॒यय॑ति करोति करोति पा॒यय॑ति । \newline
53. पा॒यय॑ त्यन्तर॒तो᳚ ऽन्तर॒तः पा॒यय॑ति पा॒यय॑ त्यन्तर॒तः । \newline
54. अ॒न्त॒र॒त ए॒वैवा न्त॑र॒तो᳚ ऽन्तर॒त ए॒व । \newline
55. ए॒वैन॑ मेन मे॒वै वैन᳚म् । \newline
56. ए॒न॒म् मेद्ध्य॒म् मेद्ध्य॑ मेन मेन॒म् मेद्ध्य᳚म् । \newline
57. मेद्ध्य॑म् करोति करोति॒ मेद्ध्य॒म् मेद्ध्य॑म् करोति । \newline
58. क॒रो॒ त्य॒धस्ता॑ द॒धस्ता᳚त् करोति करो त्य॒धस्ता᳚त् । \newline
59. अ॒धस्ता॒ दुपोपा॒ धस्ता॑ द॒धस्ता॒ दुप॑ । \newline
60. उपो᳚क्ष त्युक्ष॒ त्युपो पो᳚क्षति । \newline
61. उ॒क्ष॒ति॒ स॒र्वतः॑ स॒र्वत॑ उक्ष त्युक्षति स॒र्वतः॑ । \newline
62. स॒र्वत॑ ए॒वैव स॒र्वतः॑ स॒र्वत॑ ए॒व । \newline
63. ए॒वैन॑ मेन मे॒वै वैन᳚म् । \newline
64. ए॒न॒म् मेद्ध्य॒म् मेद्ध्य॑ मेन मेन॒म् मेद्ध्य᳚म् । \newline
65. मेद्ध्य॑म् करोति करोति॒ मेद्ध्य॒म् मेद्ध्य॑म् करोति । \newline
66. क॒रो॒तीति॑ करोति । \newline

\textbf{Ghana Paata } \newline

1. त्वौष॑धीभ्य॒ ओष॑धीभ्य स्त्वा॒ त्वौष॑धीभ्यः॒ प्र प्रौ ष॑धीभ्य स्त्वा॒ त्वौष॑धीभ्यः॒ प्र । \newline
2. ओष॑धीभ्यः॒ प्र प्रौष॑धीभ्य॒ ओष॑धीभ्यः॒ प्रोक्षा᳚ म्युक्षामि॒ प्रौष॑धीभ्य॒ ओष॑धीभ्यः॒ प्रोक्षा॑मि । \newline
3. ओष॑धीभ्य॒ इत्योष॑धि - भ्यः॒ । \newline
4. प्रोक्षा᳚ म्युक्षामि॒ प्र प्रोक्षा॒ मीती त्यु॑क्षामि॒ प्र प्रोक्षा॒ मीति॑ । \newline
5. उ॒क्षा॒ मीती त्यु॑क्षा म्युक्षा॒मी त्या॑हा॒हे त्यु॑क्षा म्युक्षा॒ मीत्या॑ह । \newline
6. इत्या॑हा॒हे तीत्या॑हा॒द्भ्यो᳚ ऽद्भ्य आ॒हे तीत्या॑हा॒द्भ्यः । \newline
7. आ॒हा॒द्भ्यो᳚ ऽद्भ्य आ॑हा हा॒द्भ्यो हि ह्य॑द्भ्य आ॑हा हा॒द्भ्यो हि । \newline
8. अ॒द्भ्यो हि ह्या᳚(1॒)द्भ्यो᳚ ऽद्भ्यो ह्ये॑ष ए॒ष ह्या᳚(1॒)द्भ्यो᳚ ऽद्भ्यो ह्ये॑षः । \newline
9. अ॒द्भ्य इत्य॑त् - भ्यः । \newline
10. ह्ये॑ष ए॒ष हि ह्ये॑ष ओष॑धीभ्य॒ ओष॑धीभ्य ए॒ष हि ह्ये॑ष ओष॑धीभ्यः । \newline
11. ए॒ष ओष॑धीभ्य॒ ओष॑धीभ्य ए॒ष ए॒ष ओष॑धीभ्यः सं॒भव॑ति सं॒भव॒ त्योष॑धीभ्य ए॒ष ए॒ष ओष॑धीभ्यः सं॒भव॑ति । \newline
12. ओष॑धीभ्यः सं॒भव॑ति सं॒भव॒ त्योष॑धीभ्य॒ ओष॑धीभ्यः सं॒भव॑ति॒ यद् यथ् सं॒भव॒ त्योष॑धीभ्य॒ ओष॑धीभ्यः सं॒भव॑ति॒ यत् । \newline
13. ओष॑धीभ्य॒ इत्योष॑धि - भ्यः॒ । \newline
14. सं॒भव॑ति॒ यद् यथ् सं॒भव॑ति सं॒भव॑ति॒ यत् प॒शुः प॒शुर् यथ् सं॒भव॑ति सं॒भव॑ति॒ यत् प॒शुः । \newline
15. सं॒भव॒तीति॑ सं - भव॑ति । \newline
16. यत् प॒शुः प॒शुर् यद् यत् प॒शु र॒पा म॒पाम् प॒शुर् यद् यत् प॒शु र॒पाम् । \newline
17. प॒शु र॒पा म॒पाम् प॒शुः प॒शु र॒पाम् पे॒रुः पे॒रु र॒पाम् प॒शुः प॒शु र॒पाम् पे॒रुः । \newline
18. अ॒पाम् पे॒रुः पे॒रु र॒पा म॒पाम् पे॒रु र॑स्यसि पे॒रु र॒पा म॒पाम् पे॒रु र॑सि । \newline
19. पे॒रु र॑स्यसि पे॒रुः पे॒रु र॒सीती त्य॑सि पे॒रुः पे॒रु र॒सीति॑ । \newline
20. अ॒सीती त्य॑स्य॒ सीत्या॑ हा॒हे त्य॑स्य॒ सीत्या॑ह । \newline
21. इत्या॑हा॒हे तीत्या॑ है॒ष ए॒ष आ॒हे तीत्या॑ है॒षः । \newline
22. आ॒है॒ष ए॒ष आ॑हा है॒ष हि ह्ये॑ष आ॑हा है॒ष हि । \newline
23. ए॒ष हि ह्ये॑ष ए॒ष ह्य॑पा म॒पाꣳ ह्ये॑ष ए॒ष ह्य॑पाम् । \newline
24. ह्य॑पा म॒पाꣳ हि ह्य॑पाम् पा॒ता पा॒ता ऽपाꣳ हि ह्य॑पाम् पा॒ता । \newline
25. अ॒पाम् पा॒ता पा॒ता ऽपा म॒पाम् पा॒ता यो यः पा॒ता ऽपा म॒पाम् पा॒ता यः । \newline
26. पा॒ता यो यः पा॒ता पा॒ता यो मेधा॑य॒ मेधा॑य॒ यः पा॒ता पा॒ता यो मेधा॑य । \newline
27. यो मेधा॑य॒ मेधा॑य॒ यो यो मेधा॑या र॒भ्यत॑ आर॒भ्यते॒ मेधा॑य॒ यो यो मेधा॑या र॒भ्यते᳚ । \newline
28. मेधा॑या र॒भ्यत॑ आर॒भ्यते॒ मेधा॑य॒ मेधा॑या र॒भ्यते᳚ स्वा॒त्तꣳ स्वा॒त्त मा॑र॒भ्यते॒ मेधा॑य॒ मेधा॑या र॒भ्यते᳚ स्वा॒त्तम् । \newline
29. आ॒र॒भ्यते᳚ स्वा॒त्तꣳ स्वा॒त्त मा॑र॒भ्यत॑ आर॒भ्यते᳚ स्वा॒त्तम् चि॑च् चिथ् स्वा॒त्त मा॑र॒भ्यत॑ आर॒भ्यते᳚ स्वा॒त्तम् चि॑त् । \newline
30. आ॒र॒भ्यत॒ इत्या᳚ - र॒भ्यते᳚ । \newline
31. स्वा॒त्तम् चि॑च् चिथ् स्वा॒त्तꣳ स्वा॒त्तम् चि॒थ् सदे॑वꣳ॒॒ सदे॑वम् चिथ् स्वा॒त्तꣳ स्वा॒त्तम् चि॒थ् सदे॑वम् । \newline
32. चि॒थ् सदे॑वꣳ॒॒ सदे॑वम् चिच् चि॒थ् सदे॑वꣳ ह॒व्यꣳ ह॒व्यꣳ सदे॑वम् चिच् चि॒थ् सदे॑वꣳ ह॒व्यम् । \newline
33. सदे॑वꣳ ह॒व्यꣳ ह॒व्यꣳ सदे॑वꣳ॒॒ सदे॑वꣳ ह॒व्य माप॒ आपो॑ ह॒व्यꣳ सदे॑वꣳ॒॒ सदे॑वꣳ ह॒व्य मापः॑ । \newline
34. सदे॑व॒मिति॒ स - दे॒व॒म् । \newline
35. ह॒व्य माप॒ आपो॑ ह॒व्यꣳ ह॒व्य मापो॑ देवीर् देवी॒ रापो॑ ह॒व्यꣳ ह॒व्य मापो॑ देवीः । \newline
36. आपो॑ देवीर् देवी॒ राप॒ आपो॑ देवीः॒ स्वद॑त॒ स्वद॑त देवी॒ राप॒ आपो॑ देवीः॒ स्वद॑त । \newline
37. दे॒वीः॒ स्वद॑त॒ स्वद॑त देवीर् देवीः॒ स्वद॑तैन मेनꣳ॒॒ स्वद॑त देवीर् देवीः॒ स्वद॑तैनम् । \newline
38. स्वद॑तैन मेनꣳ॒॒ स्वद॑त॒ स्वद॑तैन॒ मिती त्ये॑नꣳ॒॒ स्वद॑त॒ स्वद॑तैन॒ मिति॑ । \newline
39. ए॒न॒ मिती त्ये॑न मेन॒ मित्या॑ हा॒हे त्ये॑न मेन॒ मित्या॑ह । \newline
40. इत्या॑हा॒हे तीत्या॑ह स्व॒दय॑ति स्व॒दय॑ त्या॒हे तीत्या॑ह स्व॒दय॑ति । \newline
41. आ॒ह॒ स्व॒दय॑ति स्व॒दय॑ त्याहाह स्व॒दय॑ त्ये॒वैव स्व॒दय॑ त्याहाह स्व॒दय॑ त्ये॒व । \newline
42. स्व॒दय॑ त्ये॒वैव स्व॒दय॑ति स्व॒दय॑ त्ये॒वैन॑ मेन मे॒व स्व॒दय॑ति स्व॒दय॑ त्ये॒वैन᳚म् । \newline
43. ए॒वैन॑ मेन मे॒वै वैन॑ मु॒परि॑ष्टा दु॒परि॑ष्टा देन मे॒वै वैन॑ मु॒परि॑ष्टात् । \newline
44. ए॒न॒ मु॒परि॑ष्टा दु॒परि॑ष्टा देन मेन मु॒परि॑ष्टा॒त् प्र प्रोपरि॑ष्टा देन मेन मु॒परि॑ष्टा॒त् प्र । \newline
45. उ॒परि॑ष्टा॒त् प्र प्रोपरि॑ष्टा दु॒परि॑ष्टा॒त् प्रोक्ष॑ त्युक्षति॒ प्रोपरि॑ष्टा दु॒परि॑ष्टा॒त् प्रोक्ष॑ति । \newline
46. प्रोक्ष॑ त्युक्षति॒ प्र प्रोक्ष॑ त्यु॒परि॑ष्टा दु॒परि॑ष्टा दुक्षति॒ प्र प्रोक्ष॑ त्यु॒परि॑ष्टात् । \newline
47. उ॒क्ष॒ त्यु॒परि॑ष्टा दु॒परि॑ष्टा दुक्ष त्युक्ष त्यु॒परि॑ष्टा दे॒वैवो परि॑ष्टा दुक्ष त्युक्ष त्यु॒परि॑ष्टा दे॒व । \newline
48. उ॒परि॑ष्टा दे॒वै वोपरि॑ष्टा दु॒परि॑ष्टा दे॒वैन॑ मेन मे॒वो परि॑ष्टा दु॒परि॑ष्टा दे॒वैन᳚म् । \newline
49. ए॒वैन॑ मेन मे॒वै वैन॒म् मेद्ध्य॒म् मेद्ध्य॑ मेन मे॒वै वैन॒म् मेद्ध्य᳚म् । \newline
50. ए॒न॒म् मेद्ध्य॒म् मेद्ध्य॑ मेन मेन॒म् मेद्ध्य॑म् करोति करोति॒ मेद्ध्य॑ मेन मेन॒म् मेद्ध्य॑म् करोति । \newline
51. मेद्ध्य॑म् करोति करोति॒ मेद्ध्य॒म् मेद्ध्य॑म् करोति पा॒यय॑ति पा॒यय॑ति करोति॒ मेद्ध्य॒म् मेद्ध्य॑म् करोति पा॒यय॑ति । \newline
52. क॒रो॒ति॒ पा॒यय॑ति पा॒यय॑ति करोति करोति पा॒यय॑ त्यन्तर॒तो᳚ ऽन्तर॒तः पा॒यय॑ति करोति करोति पा॒यय॑ त्यन्तर॒तः । \newline
53. पा॒यय॑ त्यन्तर॒तो᳚ ऽन्तर॒तः पा॒यय॑ति पा॒यय॑ त्यन्तर॒त ए॒वैवा न्त॑र॒तः पा॒यय॑ति पा॒यय॑ त्यन्तर॒त ए॒व । \newline
54. अ॒न्त॒र॒त ए॒वैवा न्त॑र॒तो᳚ ऽन्तर॒त ए॒वैन॑ मेन मे॒वा न्त॑र॒तो᳚ ऽन्तर॒त ए॒वैन᳚म् । \newline
55. ए॒वैन॑ मेन मे॒वै वैन॒म् मेद्ध्य॒म् मेद्ध्य॑ मेन मे॒वै वैन॒म् मेद्ध्य᳚म् । \newline
56. ए॒न॒म् मेद्ध्य॒म् मेद्ध्य॑ मेन मेन॒म् मेद्ध्य॑म् करोति करोति॒ मेद्ध्य॑ मेन मेन॒म् मेद्ध्य॑म् करोति । \newline
57. मेद्ध्य॑म् करोति करोति॒ मेद्ध्य॒म् मेद्ध्य॑म् करो त्य॒धस्ता॑ द॒धस्ता᳚त् करोति॒ मेद्ध्य॒म् मेद्ध्य॑म् करो त्य॒धस्ता᳚त् । \newline
58. क॒रो॒ त्य॒धस्ता॑ द॒धस्ता᳚त् करोति करो त्य॒धस्ता॒ दुपोपा॒ धस्ता᳚त् करोति करो त्य॒धस्ता॒ दुप॑ । \newline
59. अ॒धस्ता॒ दुपोपा॒ धस्ता॑ द॒धस्ता॒ दुपो᳚क्ष त्युक्ष॒ त्युपा॒धस्ता॑ द॒धस्ता॒ दुपो᳚क्षति । \newline
60. उपो᳚क्ष त्युक्ष॒ त्युपो पो᳚क्षति स॒र्वतः॑ स॒र्वत॑ उक्ष॒ त्युपो पो᳚क्षति स॒र्वतः॑ । \newline
61. उ॒क्ष॒ति॒ स॒र्वतः॑ स॒र्वत॑ उक्ष त्युक्षति स॒र्वत॑ ए॒वैव स॒र्वत॑ उक्ष त्युक्षति स॒र्वत॑ ए॒व । \newline
62. स॒र्वत॑ ए॒वैव स॒र्वतः॑ स॒र्वत॑ ए॒वैन॑ मेन मे॒व स॒र्वतः॑ स॒र्वत॑ ए॒वैन᳚म् । \newline
63. ए॒वैन॑ मेन मे॒वै वैन॒म् मेद्ध्य॒म् मेद्ध्य॑ मेन मे॒वै वैन॒म् मेद्ध्य᳚म् । \newline
64. ए॒न॒म् मेद्ध्य॒म् मेद्ध्य॑ मेन मेन॒म् मेद्ध्य॑म् करोति करोति॒ मेद्ध्य॑ मेन मेन॒म् मेद्ध्य॑म् करोति । \newline
65. मेद्ध्य॑म् करोति करोति॒ मेद्ध्य॒म् मेद्ध्य॑म् करोति । \newline
66. क॒रो॒तीति॑ करोति । \newline
\pagebreak
\markright{ TS 6.3.7.1  \hfill https://www.vedavms.in \hfill}

\section{ TS 6.3.7.1 }

\textbf{TS 6.3.7.1 } \newline
\textbf{Samhita Paata} \newline

अ॒ग्निना॒ वै होत्रा॑ दे॒वा असु॑रा-न॒भ्य॑भव-न्न॒ग्नये॑ समि॒द्ध्यमा॑ना॒यानु॑ ब्रू॒हीत्या॑ह॒ भ्रातृ॑व्याभिभूत्यै स॒प्तद॑श सामिधे॒नीरन्वा॑ह सप्तद॒शः प्र॒जाप॑तिः प्र॒जाप॑ते॒राप्त्यै॑ स॒प्तद॒शान्वा॑ह॒ द्वाद॑श॒ मासाः॒ पञ्च॒र्तवः॒ स सं॑ॅवथ्स॒रः सं॑ॅवथ्स॒रं प्र॒जा अनु॒ प्रजा॑यन्ते प्र॒जानां᳚ प्र॒जन॑नाय दे॒वा वै सा॑मिधे॒नीर॒नूच्य॑ य॒ज्ञ्ं नान्व॑पश्य॒न्थ् स प्र॒जाप॑ति-स्तू॒ष्णीमा॑घा॒र- [  ] \newline

\textbf{Pada Paata} \newline

अ॒ग्निना᳚ । वै । होत्रा᳚ । दे॒वाः । असु॑रान् । अ॒भीति॑ । अ॒भ॒व॒न्न् । अ॒ग्नये᳚ । स॒मि॒द्ध्यमा॑ना॒येति॑ सं - इ॒द्ध्यमा॑नाय । अन्विति॑ । ब्रू॒हि॒ । इति॑ । आ॒ह॒ । भ्रातृ॑व्याभिभूत्या॒ इति॒ भ्रातृ॑व्य-अ॒भि॒भू॒त्यै॒ । स॒प्तद॒शेति॑ स॒प्त - द॒श॒ । सा॒मि॒धे॒नीरिति॑ सां - इ॒धे॒नीः । अन्विति॑ । आ॒ह॒ । स॒प्त॒द॒श इति॑ सप्त - द॒शः । प्र॒जाप॑ति॒रिति॑ प्र॒जा - प॒तिः॒ । प्र॒जाप॑ते॒रिति॑ प्र॒जा - प॒तेः॒ । आप्त्यै᳚ । स॒प्तद॒शेति॑ स॒प्त - द॒श॒ । अन्विति॑ । आ॒ह॒ । द्वाद॑श । मासाः᳚ । पञ्च॑ । ऋ॒तवः॑ । सः । सं॒ॅव॒थ्स॒र इति॑ सं - व॒थ्स॒रः । सं॒ॅव॒थ्स॒रमिति॑ सं - व॒थ्स॒रम् । प्र॒जा इति॑ प्र - जाः । अनु॑ । प्रेति॑ । जा॒य॒न्ते॒ । प्र॒जाना॒मिति॑ प्र-जाना᳚म् । प्र॒जन॑ना॒येति॑ प्र-जन॑नाय । दे॒वाः । वै । सा॒मि॒धे॒नीरिति॑ सां-इ॒धे॒नीः । अ॒नूच्येत्य॑नु - उच्य॑ । य॒ज्ञ्म् । न । अन्विति॑ । अ॒प॒श्य॒न्न् । सः । प्र॒जाप॑ति॒रिति॑ प्र॒जा-प॒तिः॒ । तू॒ष्णीम् । आ॒घा॒रमित्या᳚-घा॒रम् ।  \newline


\textbf{Krama Paata} \newline

अ॒ग्निना॒ वै । वै होत्रा᳚ । होत्रा॑ दे॒वाः । दे॒वा असु॑रान् । असु॑रान॒भि । अ॒भ्य॑भवन्न् । अ॒भ॒व॒न्न॒ग्नये᳚ । अ॒ग्नये॑ समि॒द्ध्यमा॑नाय । स॒मि॒द्ध्यमा॑ना॒यानु॑ । स॒मि॒द्ध्यमा॑ना॒येति॑ सम् - इ॒द्ध्यमा॑नाय । अनु॑ ब्रूहि । ब्रू॒हीति॑ । इत्या॑ह । आ॒ह॒ भ्रातृ॑व्याभिभूत्यै । भ्रातृ॑व्याभिभूत्यै स॒प्तद॑श । भ्रातृ॑व्याभिभूत्या॒ इति॒ भ्रातृ॑व्य - अ॒भि॒भू॒त्यै॒ । स॒प्तद॑श सामिधे॒नीः । स॒प्तद॒शेति॑ स॒प्त - द॒श॒ । सा॒मि॒धे॒नीरनु॑ । सा॒मि॒धे॒नीरिति॑ साम् - इ॒धे॒नीः । अन्वा॑ह । आ॒ह॒ स॒प्त॒द॒शः । स॒प्त॒द॒शः प्र॒जाप॑तिः । स॒प्त॒द॒श इति॑ सप्त - द॒शः । प्र॒जाप॑तिः प्र॒जाप॑तेः । प्र॒जाप॑ति॒रिति॑ प्र॒जा - प॒तिः॒ । प्र॒जाप॑ते॒राप्त्यै᳚ । प्र॒जाप॑ते॒रिति॑ प्र॒जा - प॒तेः॒ । आप्त्यै॑ स॒प्तद॑श । स॒प्तद॒शानु॑ । स॒प्तद॒शेति॑ स॒प्त - द॒श॒ । अन्वा॑ह । आ॒ह॒ द्वाद॑श । द्वाद॑श॒ मासाः᳚ । मासाः॒ पञ्च॑ । पञ्च॒र्तवः॑ । ऋ॒तवः॒ सः । 
स स॑म्ॅवथ्स॒रः । स॒म्ॅव॒थ्स॒रः स॑म्ॅवथ्स॒रम् । स॒म्ॅव॒थ्स॒र इति॑ सम् - व॒थ्स॒रः । स॒म्ॅव॒थ्स॒रम् प्र॒जाः । स॒म्ॅव॒थ्स॒रमिति॑ सम् - व॒थ्स॒रम् । प्र॒जा अनु॑ । प्र॒जा इति॑ प्र - जाः । अनु॒ प्र । प्र जा॑यन्ते । जा॒य॒न्ते॒ प्र॒जाना᳚म् । प्र॒जाना᳚म् प्र॒जन॑नाय । प्र॒जाना॒मिति॑ प्र - जाना᳚म् । प्र॒जन॑नाय दे॒वाः । प्र॒जन॑ना॒येति॑ प्र - जन॑नाय । दे॒वा वै । वै सा॑मिधे॒नीः । सा॒मि॒धे॒नीर॒नूच्य॑ । सा॒मि॒धे॒नीरिति॑ साम् - इ॒धे॒नीः । अ॒नूच्य॑ य॒ज्ञ्म् । अ॒नूच्येत्य॑नु - उच्य॑ । य॒ज्ञ्म् न । नानु॑ । अन्व॑पश्यन्न् । अ॒प॒श्य॒न्थ् सः । स प्र॒जाप॑तिः । प्र॒जाप॑तिस्तू॒ष्णीम् । प्र॒जाप॑ति॒रिति॑ प्र॒जा - प॒तिः॒ । तू॒ष्णीमा॑घा॒रम् । आ॒घा॒रमा । आ॒घा॒रमित्या᳚ - घा॒रम् \newline

\textbf{Jatai Paata} \newline

1. अ॒ग्निना॒ वै वा अ॒ग्निना॒ ऽग्निना॒ वै । \newline
2. वै होत्रा॒ होत्रा॒ वै वै होत्रा᳚ । \newline
3. होत्रा॑ दे॒वा दे॒वा होत्रा॒ होत्रा॑ दे॒वाः । \newline
4. दे॒वा असु॑रा॒ नसु॑रान् दे॒वा दे॒वा असु॑रान् । \newline
5. असु॑रान॒ भ्य॑भ्य सु॑रा॒ नसु॑रा न॒भि । \newline
6. अ॒भ्य॑भवन् नभवन् न॒भ्या᳚(1॒)भ्य॑भवन्न् । \newline
7. अ॒भ॒व॒न् न॒ग्नये॒ ऽग्नये॑ ऽभवन् नभवन् न॒ग्नये᳚ । \newline
8. अ॒ग्नये॑ समि॒द्ध्यमा॑नाय समि॒द्ध्यमा॑ना या॒ग्नये॒ ऽग्नये॑ समि॒द्ध्यमा॑नाय । \newline
9. स॒मि॒द्ध्यमा॑ना॒ यान्वनु॑ समि॒द्ध्यमा॑नाय समि॒द्ध्यमा॑ना॒ यानु॑ । \newline
10. स॒मि॒द्ध्यमा॑ना॒येति॑ सं - इ॒द्ध्यमा॑नाय । \newline
11. अनु॑ ब्रूहि ब्रू॒ह्य न्वनु॑ ब्रूहि । \newline
12. ब्रू॒ही तीति॑ ब्रूहि ब्रू॒हीति॑ । \newline
13. इत्या॑हा॒हे तीत्या॑ह । \newline
14. आ॒ह॒ भ्रातृ॑व्याभिभूत्यै॒ भ्रातृ॑व्याभिभूत्या आहाह॒ भ्रातृ॑व्याभिभूत्यै । \newline
15. भ्रातृ॑व्याभिभूत्यै स॒प्तद॑श स॒प्तद॑श॒ भ्रातृ॑व्याभिभूत्यै॒ भ्रातृ॑व्याभिभूत्यै स॒प्तद॑श । \newline
16. भ्रातृ॑व्याभिभूत्या॒ इति॒ भ्रातृ॑व्य - अ॒भि॒भू॒त्यै॒ । \newline
17. स॒प्तद॑श सामिधे॒नीः सा॑मिधे॒नीः स॒प्तद॑श स॒प्तद॑श सामिधे॒नीः । \newline
18. स॒प्तद॒शेति॑ स॒प्त - द॒श॒ । \newline
19. सा॒मि॒धे॒नी रन्वनु॑ सामिधे॒नीः सा॑मिधे॒नी रनु॑ । \newline
20. सा॒मि॒धे॒नीरिति॑ सां - इ॒धे॒नीः । \newline
21. अन्वा॑ हा॒हान् वन् वा॑ह । \newline
22. आ॒ह॒ स॒प्त॒द॒शः स॑प्तद॒श आ॑हाह सप्तद॒शः । \newline
23. स॒प्त॒द॒शः प्र॒जाप॑तिः प्र॒जाप॑तिः सप्तद॒शः स॑प्तद॒शः प्र॒जाप॑तिः । \newline
24. स॒प्त॒द॒श इति॑ सप्त - द॒शः । \newline
25. प्र॒जाप॑तिः प्र॒जाप॑तेः प्र॒जाप॑तेः प्र॒जाप॑तिः प्र॒जाप॑तिः प्र॒जाप॑तेः । \newline
26. प्र॒जाप॑ति॒रिति॑ प्र॒जा - प॒तिः॒ । \newline
27. प्र॒जाप॑ते॒ राप्त्या॒ आप्त्यै᳚ प्र॒जाप॑तेः प्र॒जाप॑ते॒ राप्त्यै᳚ । \newline
28. प्र॒जाप॑ते॒रिति॑ प्र॒जा - प॒तेः॒ । \newline
29. आप्त्यै॑ स॒प्तद॑श स॒प्तद॒शा प्त्या॒ आप्त्यै॑ स॒प्तद॑श । \newline
30. स॒प्तद॒शान् वनु॑ स॒प्तद॑श स॒प्तद॒शानु॑ । \newline
31. स॒प्तद॒शेति॑ स॒प्त - द॒श॒ । \newline
32. अन्वा॑ हा॒हान् वन् वा॑ह । \newline
33. आ॒ह॒ द्वाद॑श॒ द्वाद॑शा हाह॒ द्वाद॑श । \newline
34. द्वाद॑श॒ मासा॒ मासा॒ द्वाद॑श॒ द्वाद॑श॒ मासाः᳚ । \newline
35. मासाः॒ पञ्च॒ पञ्च॒ मासा॒ मासाः॒ पञ्च॑ । \newline
36. पञ्च॒ र्‌तव॑ ऋ॒तवः॒ पञ्च॒ पञ्च॒ र्‌तवः॑ । \newline
37. ऋ॒तवः॒ स स ऋ॒तव॑ ऋ॒तवः॒ सः । \newline
38. स सं॑ॅवथ्स॒रः सं॑ॅवथ्स॒रः स स सं॑ॅवथ्स॒रः । \newline
39. सं॒ॅव॒थ्स॒रः सं॑ॅवथ्स॒रꣳ सं॑ॅवथ्स॒रꣳ सं॑ॅवथ्स॒रः सं॑ॅवथ्स॒रः सं॑ॅवथ्स॒रम् । \newline
40. सं॒ॅव॒थ्स॒र इति॑ सं - व॒थ्स॒रः । \newline
41. सं॒ॅव॒थ्स॒रम् प्र॒जाः प्र॒जाः सं॑ॅवथ्स॒रꣳ सं॑ॅवथ्स॒रम् प्र॒जाः । \newline
42. सं॒ॅव॒थ्स॒रमिति॑ सं - व॒थ्स॒रम् । \newline
43. प्र॒जा अन्वनु॑ प्र॒जाः प्र॒जा अनु॑ । \newline
44. प्र॒जा इति॑ प्र - जाः । \newline
45. अनु॒ प्र प्राण्वनु॒ प्र । \newline
46. प्र जा॑यन्ते जायन्ते॒ प्र प्र जा॑यन्ते । \newline
47. जा॒य॒न्ते॒ प्र॒जाना᳚म् प्र॒जाना᳚म् जायन्ते जायन्ते प्र॒जाना᳚म् । \newline
48. प्र॒जाना᳚म् प्र॒जन॑नाय प्र॒जन॑नाय प्र॒जाना᳚म् प्र॒जाना᳚म् प्र॒जन॑नाय । \newline
49. प्र॒जाना॒मिति॑ प्र - जाना᳚म् । \newline
50. प्र॒जन॑नाय दे॒वा दे॒वाः प्र॒जन॑नाय प्र॒जन॑नाय दे॒वाः । \newline
51. प्र॒जन॑ना॒येति॑ प्र - जन॑नाय । \newline
52. दे॒वा वै वै दे॒वा दे॒वा वै । \newline
53. वै सा॑मिधे॒नीः सा॑मिधे॒नीर् वै वै सा॑मिधे॒नीः । \newline
54. सा॒मि॒धे॒नी र॒नूच्या॒ नूच्य॑ सामिधे॒नीः सा॑मिधे॒नी र॒नूच्य॑ । \newline
55. सा॒मि॒धे॒नीरिति॑ सां - इ॒धे॒नीः । \newline
56. अ॒नूच्य॑ य॒ज्ञ्ं ॅय॒ज्ञ् म॒नूच्या॒ नूच्य॑ य॒ज्ञ्म् । \newline
57. अ॒नूच्येत्य॑नु - उच्य॑ । \newline
58. य॒ज्ञ्न्न न य॒ज्ञ्ं ॅय॒ज्ञ्न्न । \newline
59. नान् वनु॒ न नानु॑ । \newline
60. अन्व॑पश्यन् नपश्य॒न् नन् वन् व॑पश्यन्न् । \newline
61. अ॒प॒श्य॒न् थ्स सो॑ ऽपश्यन् नपश्य॒न् थ्सः । \newline
62. स प्र॒जाप॑तिः प्र॒जाप॑तिः॒ स स प्र॒जाप॑तिः । \newline
63. प्र॒जाप॑ति स्तू॒ष्णीम् तू॒ष्णीम् प्र॒जाप॑तिः प्र॒जाप॑ति स्तू॒ष्णीम् । \newline
64. प्र॒जाप॑ति॒रिति॑ प्र॒जा - प॒तिः॒ । \newline
65. तू॒ष्णी मा॑घा॒र मा॑घा॒रम् तू॒ष्णीम् तू॒ष्णी मा॑घा॒रम् । \newline
66. आ॒घा॒र मा ऽऽघा॒र मा॑घा॒र मा । \newline
67. आ॒घा॒रमित्या᳚ - घा॒रम् । \newline

\textbf{Ghana Paata } \newline

1. अ॒ग्निना॒ वै वा अ॒ग्निना॒ ऽग्निना॒ वै होत्रा॒ होत्रा॒ वा अ॒ग्निना॒ ऽग्निना॒ वै होत्रा᳚ । \newline
2. वै होत्रा॒ होत्रा॒ वै वै होत्रा॑ दे॒वा दे॒वा होत्रा॒ वै वै होत्रा॑ दे॒वाः । \newline
3. होत्रा॑ दे॒वा दे॒वा होत्रा॒ होत्रा॑ दे॒वा असु॑रा॒ नसु॑रान् दे॒वा होत्रा॒ होत्रा॑ दे॒वा असु॑रान् । \newline
4. दे॒वा असु॑रा॒ नसु॑रान् दे॒वा दे॒वा असु॑रान॒ भ्य॑भ्यसु॑रान् दे॒वा दे॒वा असु॑रान॒भि । \newline
5. असु॑रा न॒भ्य॑ भ्यसु॑रा॒ नसु॑रा न॒भ्य॑भवन् नभवन् न॒भ्यसु॑रा॒ नसु॑रा न॒भ्य॑भवन्न् । \newline
6. अ॒भ्य॑भवन् नभवन् न॒भ्या᳚(1॒)भ्य॑भवन् न॒ग्नये॒ ऽग्नये॑ ऽभवन् न॒भ्या᳚(1॒)भ्य॑भवन् न॒ग्नये᳚ । \newline
7. अ॒भ॒व॒न् न॒ग्नये॒ ऽग्नये॑ ऽभवन् नभवन् न॒ग्नये॑ समि॒द्ध्यमा॑नाय समि॒द्ध्यमा॑ना या॒ग्नये॑ ऽभवन् नभवन् न॒ग्नये॑ समि॒द्ध्यमा॑नाय । \newline
8. अ॒ग्नये॑ समि॒द्ध्यमा॑नाय समि॒द्ध्यमा॑ना या॒ग्नये॒ ऽग्नये॑ समि॒द्ध्यमा॑ना॒ यान्वनु॑ समि॒द्ध्यमा॑ना या॒ग्नये॒ ऽग्नये॑ समि॒द्ध्यमा॑ना॒ यानु॑ । \newline
9. स॒मि॒द्ध्यमा॑ना॒ यान्वनु॑ समि॒द्ध्यमा॑नाय समि॒द्ध्यमा॑ना॒ यानु॑ ब्रूहि ब्रू॒ह्यनु॑ समि॒द्ध्यमा॑नाय समि॒द्ध्यमा॑ना॒ यानु॑ ब्रूहि । \newline
10. स॒मि॒द्ध्यमा॑ना॒येति॑ सं - इ॒द्ध्यमा॑नाय । \newline
11. अनु॑ ब्रूहि ब्रू॒ह्यन् वनु॑ ब्रू॒ही तीति॑ ब्रू॒ह्यन् वनु॑ ब्रू॒हीति॑ । \newline
12. ब्रू॒ही तीति॑ ब्रूहि ब्रू॒हीत्या॑ हा॒हेति॑ ब्रूहि ब्रू॒ही त्या॑ह । \newline
13. इत्या॑हा॒हे तीत्या॑ह॒ भ्रातृ॑व्याभिभूत्यै॒ भ्रातृ॑व्याभिभूत्या आ॒हे तीत्या॑ह॒ भ्रातृ॑व्याभिभूत्यै । \newline
14. आ॒ह॒ भ्रातृ॑व्याभिभूत्यै॒ भ्रातृ॑व्याभिभूत्या आहाह॒ भ्रातृ॑व्याभिभूत्यै स॒प्तद॑श स॒प्तद॑श॒ भ्रातृ॑व्याभिभूत्या आहाह॒ भ्रातृ॑व्याभिभूत्यै स॒प्तद॑श । \newline
15. भ्रातृ॑व्याभिभूत्यै स॒प्तद॑श स॒प्तद॑श॒ भ्रातृ॑व्याभिभूत्यै॒ भ्रातृ॑व्याभिभूत्यै स॒प्तद॑श सामिधे॒नीः सा॑मिधे॒नीः स॒प्तद॑श॒ भ्रातृ॑व्याभिभूत्यै॒ भ्रातृ॑व्याभिभूत्यै स॒प्तद॑श सामिधे॒नीः । \newline
16. भ्रातृ॑व्याभिभूत्या॒ इति॒ भ्रातृ॑व्य - अ॒भि॒भू॒त्यै॒ । \newline
17. स॒प्तद॑श सामिधे॒नीः सा॑मिधे॒नीः स॒प्तद॑श स॒प्तद॑श सामिधे॒नी रन्वनु॑ सामिधे॒नीः स॒प्तद॑श स॒प्तद॑श सामिधे॒नी रनु॑ । \newline
18. स॒प्तद॒शेति॑ स॒प्त - द॒श॒ । \newline
19. सा॒मि॒धे॒नी रन्वनु॑ सामिधे॒नीः सा॑मिधे॒नी रन्वा॑हा॒ हानु॑ सामिधे॒नीः सा॑मिधे॒नी रन्वा॑ह । \newline
20. सा॒मि॒धे॒नीरिति॑ सां - इ॒धे॒नीः । \newline
21. अन्वा॑ हा॒हान् वन् वा॑ह सप्तद॒शः स॑प्तद॒श आ॒हान् वन् वा॑ह सप्तद॒शः । \newline
22. आ॒ह॒ स॒प्त॒द॒शः स॑प्तद॒श आ॑हाह सप्तद॒शः प्र॒जाप॑तिः प्र॒जाप॑तिः सप्तद॒श आ॑हाह सप्तद॒शः प्र॒जाप॑तिः । \newline
23. स॒प्त॒द॒शः प्र॒जाप॑तिः प्र॒जाप॑तिः सप्तद॒शः स॑प्तद॒शः प्र॒जाप॑तिः प्र॒जाप॑तेः प्र॒जाप॑तेः प्र॒जाप॑तिः सप्तद॒शः स॑प्तद॒शः प्र॒जाप॑तिः प्र॒जाप॑तेः । \newline
24. स॒प्त॒द॒श इति॑ सप्त - द॒शः । \newline
25. प्र॒जाप॑तिः प्र॒जाप॑तेः प्र॒जाप॑तेः प्र॒जाप॑तिः प्र॒जाप॑तिः प्र॒जाप॑ते॒ राप्त्या॒ आप्त्यै᳚ प्र॒जाप॑तेः प्र॒जाप॑तिः प्र॒जाप॑तिः प्र॒जाप॑ते॒ राप्त्यै᳚ । \newline
26. प्र॒जाप॑ति॒रिति॑ प्र॒जा - प॒तिः॒ । \newline
27. प्र॒जाप॑ते॒ राप्त्या॒ आप्त्यै᳚ प्र॒जाप॑तेः प्र॒जाप॑ते॒ राप्त्यै॑ स॒प्तद॑श स॒प्तद॒ शाप्त्यै᳚ प्र॒जाप॑तेः प्र॒जाप॑ते॒ राप्त्यै॑ स॒प्तद॑श । \newline
28. प्र॒जाप॑ते॒रिति॑ प्र॒जा - प॒तेः॒ । \newline
29. आप्त्यै॑ स॒प्तद॑श स॒प्तद॒ शाप्त्या॒ आप्त्यै॑ स॒प्तद॒ शान्वनु॑ स॒प्तद॒ शाप्त्या॒ आप्त्यै॑ स॒प्तद॒ शानु॑ । \newline
30. स॒प्तद॒ शान्वनु॑ स॒प्तद॑श स॒प्तद॒ शान् वा॑हा॒ हानु॑ स॒प्तद॑श स॒प्तद॒ शान्वा॑ह । \newline
31. स॒प्तद॒शेति॑ स॒प्त - द॒श॒ । \newline
32. अन्वा॑ हा॒हान् वन् वा॑ह॒ द्वाद॑श॒ द्वाद॑शा॒हान् वन् वा॑ह॒ द्वाद॑श । \newline
33. आ॒ह॒ द्वाद॑श॒ द्वाद॑शा हाह॒ द्वाद॑श॒ मासा॒ मासा॒ द्वाद॑शा हाह॒ द्वाद॑श॒ मासाः᳚ । \newline
34. द्वाद॑श॒ मासा॒ मासा॒ द्वाद॑श॒ द्वाद॑श॒ मासाः॒ पञ्च॒ पञ्च॒ मासा॒ द्वाद॑श॒ द्वाद॑श॒ मासाः॒ पञ्च॑ । \newline
35. मासाः॒ पञ्च॒ पञ्च॒ मासा॒ मासाः॒ पञ्च॒ र्‌तव॑ ऋ॒तवः॒ पञ्च॒ मासा॒ मासाः॒ पञ्च॒ र्‌तवः॑ । \newline
36. पञ्च॒ र्‌तव॑ ऋ॒तवः॒ पञ्च॒ पञ्च॒ र्‌तवः॒ स स ऋ॒तवः॒ पञ्च॒ पञ्च॒ र्‌तवः॒ सः । \newline
37. ऋ॒तवः॒ स स ऋ॒तव॑ ऋ॒तवः॒ स सं॑ॅवथ्स॒रः सं॑ॅवथ्स॒रः स ऋ॒तव॑ ऋ॒तवः॒ स सं॑ॅवथ्स॒रः । \newline
38. स सं॑ॅवथ्स॒रः सं॑ॅवथ्स॒रः स स सं॑ॅवथ्स॒रः सं॑ॅवथ्स॒रꣳ सं॑ॅवथ्स॒रꣳ सं॑ॅवथ्स॒रः स स सं॑ॅवथ्स॒रः सं॑ॅवथ्स॒रम् । \newline
39. सं॒ॅव॒थ्स॒रः सं॑ॅवथ्स॒रꣳ सं॑ॅवथ्स॒रꣳ सं॑ॅवथ्स॒रः सं॑ॅवथ्स॒रः सं॑ॅवथ्स॒रम् प्र॒जाः प्र॒जाः सं॑ॅवथ्स॒रꣳ सं॑ॅवथ्स॒रः सं॑ॅवथ्स॒रः सं॑ॅवथ्स॒रम् प्र॒जाः । \newline
40. सं॒ॅव॒थ्स॒र इति॑ सं - व॒थ्स॒रः । \newline
41. सं॒ॅव॒थ्स॒रम् प्र॒जाः प्र॒जाः सं॑ॅवथ्स॒रꣳ सं॑ॅवथ्स॒रम् प्र॒जा अन्वनु॑ प्र॒जाः सं॑ॅवथ्स॒रꣳ सं॑ॅवथ्स॒रम् प्र॒जा अनु॑ । \newline
42. सं॒ॅव॒थ्स॒रमिति॑ सं - व॒थ्स॒रम् । \newline
43. प्र॒जा अन्वनु॑ प्र॒जाः प्र॒जा अनु॒ प्र प्राणु॑ प्र॒जाः प्र॒जा अनु॒ प्र । \newline
44. प्र॒जा इति॑ प्र - जाः । \newline
45. अनु॒ प्र प्राण्वनु॒ प्र जा॑यन्ते जायन्ते॒ प्राण्वनु॒ प्र जा॑यन्ते । \newline
46. प्र जा॑यन्ते जायन्ते॒ प्र प्र जा॑यन्ते प्र॒जाना᳚म् प्र॒जाना᳚म् जायन्ते॒ प्र प्र जा॑यन्ते प्र॒जाना᳚म् । \newline
47. जा॒य॒न्ते॒ प्र॒जाना᳚म् प्र॒जाना᳚म् जायन्ते जायन्ते प्र॒जाना᳚म् प्र॒जन॑नाय प्र॒जन॑नाय प्र॒जाना᳚म् जायन्ते जायन्ते प्र॒जाना᳚म् प्र॒जन॑नाय । \newline
48. प्र॒जाना᳚म् प्र॒जन॑नाय प्र॒जन॑नाय प्र॒जाना᳚म् प्र॒जाना᳚म् प्र॒जन॑नाय दे॒वा दे॒वाः प्र॒जन॑नाय प्र॒जाना᳚म् प्र॒जाना᳚म् प्र॒जन॑नाय दे॒वाः । \newline
49. प्र॒जाना॒मिति॑ प्र - जाना᳚म् । \newline
50. प्र॒जन॑नाय दे॒वा दे॒वाः प्र॒जन॑नाय प्र॒जन॑नाय दे॒वा वै वै दे॒वाः प्र॒जन॑नाय प्र॒जन॑नाय दे॒वा वै । \newline
51. प्र॒जन॑ना॒येति॑ प्र - जन॑नाय । \newline
52. दे॒वा वै वै दे॒वा दे॒वा वै सा॑मिधे॒नीः सा॑मिधे॒नीर् वै दे॒वा दे॒वा वै सा॑मिधे॒नीः । \newline
53. वै सा॑मिधे॒नीः सा॑मिधे॒नीर् वै वै सा॑मिधे॒नी र॒नूच्या॒ नूच्य॑ सामिधे॒नीर् वै वै सा॑मिधे॒नी र॒नूच्य॑ । \newline
54. सा॒मि॒धे॒नी र॒नूच्या॒ नूच्य॑ सामिधे॒नीः सा॑मिधे॒नी र॒नूच्य॑ य॒ज्ञ्ं ॅय॒ज्ञ् म॒नूच्य॑ सामिधे॒नीः सा॑मिधे॒नी र॒नूच्य॑ य॒ज्ञ्म् । \newline
55. सा॒मि॒धे॒नीरिति॑ सां - इ॒धे॒नीः । \newline
56. अ॒नूच्य॑ य॒ज्ञ्ं ॅय॒ज्ञ् म॒नूच्या॒ नूच्य॑ य॒ज्ञ्न् न न य॒ज्ञ् म॒नूच्या॒ नूच्य॑ य॒ज्ञ्न् न । \newline
57. अ॒नूच्येत्य॑नु - उच्य॑ । \newline
58. य॒ज्ञ्न् न न य॒ज्ञ्ं ॅय॒ज्ञ्न् नान्वनु॒ न य॒ज्ञ्ं ॅय॒ज्ञ्न् नानु॑ । \newline
59. नान्वनु॒ न नान् व॑पश्यन् नपश्य॒न् ननु॒ न नान् व॑पश्यन्न् । \newline
60. अन्व॑पश्यन् नपश्य॒न् नन् वन् व॑पश्य॒न् थ्स सो॑ ऽपश्य॒न् नन् वन् व॑पश्य॒न् थ्सः । \newline
61. अ॒प॒श्य॒न् थ्स सो॑ ऽपश्यन् नपश्य॒न् थ्स प्र॒जाप॑तिः प्र॒जाप॑तिः॒ सो॑ ऽपश्यन् नपश्य॒न् थ्स प्र॒जाप॑तिः । \newline
62. स प्र॒जाप॑तिः प्र॒जाप॑तिः॒ स स प्र॒जाप॑ति स्तू॒ष्णीम् तू॒ष्णीम् प्र॒जाप॑तिः॒ स स प्र॒जाप॑ति स्तू॒ष्णीम् । \newline
63. प्र॒जाप॑ति स्तू॒ष्णीम् तू॒ष्णीम् प्र॒जाप॑तिः प्र॒जाप॑ति स्तू॒ष्णी मा॑घा॒र मा॑घा॒रम् तू॒ष्णीम् प्र॒जाप॑तिः प्र॒जाप॑ति स्तू॒ष्णी मा॑घा॒रम् । \newline
64. प्र॒जाप॑ति॒रिति॑ प्र॒जा - प॒तिः॒ । \newline
65. तू॒ष्णी मा॑घा॒र मा॑घा॒रम् तू॒ष्णीम् तू॒ष्णी मा॑घा॒र मा ऽऽघा॒रम् तू॒ष्णीम् तू॒ष्णी मा॑घा॒र मा । \newline
66. आ॒घा॒र मा ऽऽघा॒र मा॑घा॒र मा ऽघा॑रय दघारय॒दा ऽऽघा॒र मा॑घा॒र मा ऽघा॑रयत् । \newline
67. आ॒घा॒रमित्या᳚ - घा॒रम् । \newline
\pagebreak
\markright{ TS 6.3.7.2  \hfill https://www.vedavms.in \hfill}

\section{ TS 6.3.7.2 }

\textbf{TS 6.3.7.2 } \newline
\textbf{Samhita Paata} \newline

-मा ऽघा॑रय॒त् ततो॒ वै दे॒वा य॒ज्ञ्मन्व॑पश्य॒न्॒. यत् तू॒ष्णी-मा॑घा॒र-मा॑घा॒रय॑ति य॒ज्ञ्स्यानु॑ख्यात्या॒ असु॑रेषु॒ वै य॒ज्ञ् आ॑सी॒त् तं दे॒वास्तू᳚ष्णीꣳ हो॒मेना॑वृञ्जत॒ यत् तू॒ष्णी-मा॑घा॒र-मा॑घा॒रय॑ति॒ भ्रातृ॑व्यस्यै॒ व तद् य॒ज्ञ्ं ॅवृ॑ङ्क्ते परि॒धी॑न्थ् सं मा᳚र्ष्टि पु॒नात्ये॒वैना॒न् त्रिस्त्रिः॒ सं मा᳚र्ष्टि॒ त्र्या॑वृ॒द्धि य॒ज्ञोऽथो॒ रक्ष॑सा॒मप॑हत्यै॒ द्वाद॑श॒ सं प॑द्यन्ते॒ द्वाद॑श॒- [  ] \newline

\textbf{Pada Paata} \newline

एति॑ । अ॒घा॒र॒य॒त् । ततः॑ । वै । दे॒वाः । य॒ज्ञ्म् । अन्विति॑ । अ॒प॒श्य॒न्न् । यत् । तू॒ष्णीम् । आ॒घा॒रमित्या᳚ - घा॒रम् । आ॒घा॒रय॒तीत्या᳚ - घा॒रय॑ति । य॒ज्ञ्स्य॑ । अनु॑ख्यात्या॒ इत्यनु॑ - ख्या॒त्यै॒ । असु॑रेषु । वै । य॒ज्ञ्ः । आ॒सी॒त् । तम् । दे॒वाः । तू॒ष्णीꣳ॒॒हो॒मेनेति॑ तूष्णीम्-हो॒मेन॑ । अ॒वृ॒ञ्ज॒त॒ । यत् । तू॒ष्णीम् । आ॒घा॒रमित्या᳚ - घा॒रम् । आ॒घा॒रय॒तीत्या᳚ - घा॒रय॑ति । भ्रातृ॑व्यस्य । ए॒व । तत् । य॒ज्ञ्म् । वृ॒ङ्क्ते॒ । प॒रि॒धीनिति॑ परि - धीन् । समिति॑ । मा॒र्ष्टि॒ । पु॒नाति॑ । ए॒व । ए॒ना॒न् । त्रिस्त्रि॒रिति॒ त्रिः - त्रिः॒ । समिति॑ । मा॒र्ष्टि॒ । त्र्या॑वृ॒दिति॒ त्रि - आ॒वृ॒त् । हि । य॒ज्ञ्ः । अथो॒ इति॑ । रक्ष॑साम् । अप॑हत्या॒ इत्यप॑ - ह॒त्यै॒ । द्वाद॑श । समिति॑ । प॒द्य॒न्ते॒ । द्वाद॑श ।  \newline


\textbf{Krama Paata} \newline

आऽघा॑रयत् । अ॒घा॒र॒य॒त् ततः॑ । ततो॒ वै । वै दे॒वाः । दे॒वा य॒ज्ञ्म् । य॒ज्ञ्मनु॑ । अन्व॑पश्यन्न् । अ॒प॒श्य॒न्॒. यत् । यत् तू॒ष्णीम् । तू॒ष्णीमा॑घा॒रम् । आ॒घा॒रमा॑घा॒रय॑ति । आ॒घा॒रमित्या᳚ - घा॒रम् । आ॒घा॒रय॑ति य॒ज्ञ्स्य॑ । आ॒घा॒रय॒तीत्या᳚ - घा॒रय॑ति । य॒ज्ञ्स्यानु॑ख्यात्यै । अनु॑ख्यात्या॒ असु॑रेषु । अनु॑ख्यात्या॒ इत्यनु॑ - ख्या॒त्यै॒ । असु॑रेषु॒ वै । वै य॒ज्ञ्ः । य॒ज्ञ् आ॑सीत् । आ॒सी॒त् तम् । तम् दे॒वाः । दे॒वास्तू᳚ष्णीꣳहो॒मेन॑ । तू॒ष्णीꣳ॒॒हो॒मेना॑वृञ्जत । तू॒ष्णीꣳ॒॒हो॒मेनेति॑ तूष्णीम् - हो॒मेन॑ । अ॒वृ॒ञ्ज॒त॒ यत् । यत् तू॒ष्णीम् । तू॒ष्णीमा॑घा॒रम् । आ॒घा॒रमा॑घा॒रय॑ति । आ॒घा॒रमित्या᳚ - घा॒रम् । आ॒घा॒रय॑ति॒ भ्रातृ॑व्यस्य । आ॒घा॒रय॒तीत्या᳚ - घा॒रय॑ति । भ्रातृ॑व्यस्यै॒व । ए॒व तत् । तद् य॒ज्ञ्म् । य॒ज्ञ्म् ॅवृ॑ङ्‍क्ते । वृ॒ङ्‍क्ते॒ प॒रि॒धीन् । प॒रि॒धीन्थ् सम् । प॒रि॒धीनिति॑ परि - धीन् । सम् मा᳚र्ष्टि । मा॒र्ष्टि॒ पु॒नाति॑ । पु॒नात्ये॒व । ए॒वैनान्॑ । ए॒ना॒न् त्रिस्त्रिः॑ । त्रिस्त्रिः॒ सम् । त्रिस्त्रि॒रिति॒ त्रिः - त्रिः॒ । सम् मा᳚र्ष्टि । मा॒र्ष्टि॒ त्र्या॑वृत् । त्र्या॑वृ॒द्‌धि । त्र्या॑वृ॒दिति॒ त्रि - आ॒वृ॒त्॒ । हि य॒ज्ञ्ः । य॒ज्ञोऽथो᳚ । अथो॒ रक्ष॑साम् । अथो॒ इत्यथो᳚ । रक्ष॑सा॒मप॑हत्यै । अप॑हत्यै॒ द्वाद॑श । अप॑हत्या॒ इत्यप॑ - ह॒त्यै॒ । द्वाद॑श॒ सम् । सम् प॑द्यन्ते । प॒द्य॒न्ते॒ द्वाद॑श । द्वाद॑श॒ मासाः᳚ \newline

\textbf{Jatai Paata} \newline

1. आ ऽघा॑रय दघारय॒दा ऽघा॑रयत् । \newline
2. अ॒घा॒र॒य॒त् तत॒ स्ततो॑ ऽघारय दघारय॒त् ततः॑ । \newline
3. ततो॒ वै वै तत॒ स्ततो॒ वै । \newline
4. वै दे॒वा दे॒वा वै वै दे॒वाः । \newline
5. दे॒वा य॒ज्ञ्ं ॅय॒ज्ञ्म् दे॒वा दे॒वा य॒ज्ञ्म् । \newline
6. य॒ज्ञ् मन् वनु॑ य॒ज्ञ्ं ॅय॒ज्ञ् मनु॑ । \newline
7. अन्व॑पश्यन् नपश्य॒न् नन् वन् व॑पश्यन्न् । \newline
8. अ॒प॒श्य॒न्॒. यद् यद॑पश्यन् नपश्य॒न्॒. यत् । \newline
9. यत् तू॒ष्णीम् तू॒ष्णीं ॅयद् यत् तू॒ष्णीम् । \newline
10. तू॒ष्णी मा॑घा॒र मा॑घा॒रम् तू॒ष्णीम् तू॒ष्णी मा॑घा॒रम् । \newline
11. आ॒घा॒र मा॑घा॒रय॑ त्याघा॒रय॑ त्याघा॒र मा॑घा॒र मा॑घा॒रय॑ति । \newline
12. आ॒घा॒रमित्या᳚ - घा॒रम् । \newline
13. आ॒घा॒रय॑ति य॒ज्ञ्स्य॑ य॒ज्ञ्स्या॑ घा॒रय॑त्या घा॒रय॑ति य॒ज्ञ्स्य॑ । \newline
14. आ॒घा॒रय॒तीत्या᳚ - घा॒रय॑ति । \newline
15. य॒ज्ञ्स्यानु॑ ख्यात्या॒ अनु॑ख्यात्यै य॒ज्ञ्स्य॑ य॒ज्ञ्स्यानु॑ ख्यात्यै । \newline
16. अनु॑ख्यात्या॒ असु॑रे॒ ष्वसु॑रे॒ ष्वनु॑ख्यात्या॒ अनु॑ख्यात्या॒ असु॑रेषु । \newline
17. अनु॑ख्यात्या॒ इत्यनु॑ - ख्या॒त्यै॒ । \newline
18. असु॑रेषु॒ वै वा असु॑रे॒ ष्वसु॑रेषु॒ वै । \newline
19. वै य॒ज्ञो य॒ज्ञो वै वै य॒ज्ञ्ः । \newline
20. य॒ज्ञ् आ॑सी दासीद् य॒ज्ञो य॒ज्ञ् आ॑सीत् । \newline
21. आ॒सी॒त् तम् त मा॑सी दासी॒त् तम् । \newline
22. तम् दे॒वा दे॒वा स्तम् तम् दे॒वाः । \newline
23. दे॒वा स्तू᳚ष्णीꣳहो॒मेन॑ तूष्णीꣳहो॒मेन॑ दे॒वा दे॒वा स्तू᳚ष्णीꣳहो॒मेन॑ । \newline
24. तू॒ष्णीꣳ॒॒हो॒मेना॑ वृञ्जता वृञ्जत तूष्णीꣳहो॒मेन॑ तूष्णीꣳहो॒मेना॑ वृञ्जत । \newline
25. तू॒ष्णीꣳ॒॒हो॒मेनेति॑ तूष्णीम् - हो॒मेन॑ । \newline
26. अ॒वृ॒ञ्ज॒त॒ यद् यद॑वृञ्जता वृञ्जत॒ यत् । \newline
27. यत् तू॒ष्णीम् तू॒ष्णीं ॅयद् यत् तू॒ष्णीम् । \newline
28. तू॒ष्णी मा॑घा॒र मा॑घा॒रम् तू॒ष्णीम् तू॒ष्णी मा॑घा॒रम् । \newline
29. आ॒घा॒र मा॑घा॒रय॑ त्याघा॒रय॑ त्याघा॒र मा॑घा॒र मा॑घा॒रय॑ति । \newline
30. आ॒घा॒रमित्या᳚ - घा॒रम् । \newline
31. आ॒घा॒रय॑ति॒ भ्रातृ॑व्यस्य॒ भ्रातृ॑व्यस्या घा॒रय॑ त्याघा॒रय॑ति॒ भ्रातृ॑व्यस्य । \newline
32. आ॒घा॒रय॒तीत्या᳚ - घा॒रय॑ति । \newline
33. भ्रातृ॑व्यस्यै॒ वैव भ्रातृ॑व्यस्य॒ भ्रातृ॑व्यस्यै॒व । \newline
34. ए॒व तत् तदे॒ वैव तत् । \newline
35. तद् य॒ज्ञ्ं ॅय॒ज्ञ्म् तत् तद् य॒ज्ञ्म् । \newline
36. य॒ज्ञ्ं ॅवृ॑ङ्क्ते वृङ्क्ते य॒ज्ञ्ं ॅय॒ज्ञ्ं ॅवृ॑ङ्क्ते । \newline
37. वृ॒ङ्क्ते॒ प॒रि॒धीन् प॑रि॒धीन्. वृ॑ङ्क्ते वृङ्क्ते परि॒धीन् । \newline
38. प॒रि॒धीन् थ्सꣳ सम् प॑रि॒धीन् प॑रि॒धीन् थ्सम् । \newline
39. प॒रि॒धीनिति॑ परि - धीन् । \newline
40. सम् मा᳚र्ष्टि मार्ष्टि॒ सꣳ सम् मा᳚र्ष्टि । \newline
41. मा॒र्ष्टि॒ पु॒नाति॑ पु॒नाति॑ मार्ष्टि मार्ष्टि पु॒नाति॑ । \newline
42. पु॒ना त्ये॒वैव पु॒नाति॑ पु॒ना त्ये॒व । \newline
43. ए॒वैना॑ नेना ने॒वै वैनान्॑ । \newline
44. ए॒ना॒न् त्रिस्त्रि॒ स्त्रिस्त्रि॑ रेना नेना॒न् त्रिस्त्रिः॑ । \newline
45. त्रिस्त्रिः॒ सꣳ सम् त्रिस्त्रि॒ स्त्रिस्त्रिः॒ सम् । \newline
46. त्रिस्त्रि॒रिति॒ त्रिः - त्रिः॒ । \newline
47. सम् मा᳚र्ष्टि मार्ष्टि॒ सꣳ सम् मा᳚र्ष्टि । \newline
48. मा॒र्ष्टि॒ त्र्या॑वृ॒त् त्र्या॑वृन् मार्ष्टि मार्ष्टि॒ त्र्या॑वृत् । \newline
49. त्र्या॑वृ॒द्धि हि त्र्या॑वृ॒त् त्र्या॑वृ॒द्धि । \newline
50. त्र्या॑वृ॒दिति॒ त्रि - आ॒वृ॒त् । \newline
51. हि य॒ज्ञो य॒ज्ञो हि हि य॒ज्ञ्ः । \newline
52. य॒ज्ञो ऽथो॒ अथो॑ य॒ज्ञो य॒ज्ञो ऽथो᳚ । \newline
53. अथो॒ रक्ष॑साꣳ॒॒ रक्ष॑सा॒ मथो॒ अथो॒ रक्ष॑साम् । \newline
54. अथो॒ इत्यथो᳚ । \newline
55. रक्ष॑सा॒ मप॑हत्या॒ अप॑हत्यै॒ रक्ष॑साꣳ॒॒ रक्ष॑सा॒ मप॑हत्यै । \newline
56. अप॑हत्यै॒ द्वाद॑श॒ द्वाद॒शा प॑हत्या॒ अप॑हत्यै॒ द्वाद॑श । \newline
57. अप॑हत्या॒ इत्यप॑ - ह॒त्यै॒ । \newline
58. द्वाद॑श॒ सꣳ सम् द्वाद॑श॒ द्वाद॑श॒ सम् । \newline
59. सम् प॑द्यन्ते पद्यन्ते॒ सꣳ सम् प॑द्यन्ते । \newline
60. प॒द्य॒न्ते॒ द्वाद॑श॒ द्वाद॑श पद्यन्ते पद्यन्ते॒ द्वाद॑श । \newline
61. द्वाद॑श॒ मासा॒ मासा॒ द्वाद॑श॒ द्वाद॑श॒ मासाः᳚ । \newline

\textbf{Ghana Paata } \newline

1. आ ऽघा॑रय दघारय॒दा ऽघा॑रय॒त् तत॒ स्ततो॑ ऽघारय॒दा ऽघा॑रय॒त् ततः॑ । \newline
2. अ॒घा॒र॒य॒त् तत॒ स्ततो॑ ऽघारय दघारय॒त् ततो॒ वै वै ततो॑ ऽघारय दघारय॒त् ततो॒ वै । \newline
3. ततो॒ वै वै तत॒ स्ततो॒ वै दे॒वा दे॒वा वै तत॒ स्ततो॒ वै दे॒वाः । \newline
4. वै दे॒वा दे॒वा वै वै दे॒वा य॒ज्ञ्ं ॅय॒ज्ञ्म् दे॒वा वै वै दे॒वा य॒ज्ञ्म् । \newline
5. दे॒वा य॒ज्ञ्ं ॅय॒ज्ञ्म् दे॒वा दे॒वा य॒ज्ञ् मन्वनु॑ य॒ज्ञ्म् दे॒वा दे॒वा य॒ज्ञ् मनु॑ । \newline
6. य॒ज्ञ् मन्वनु॑ य॒ज्ञ्ं ॅय॒ज्ञ् मन्व॑पश्यन् नपश्य॒न् ननु॑ य॒ज्ञ्ं ॅय॒ज्ञ् मन्व॑पश्यन्न् । \newline
7. अन्व॑पश्यन् नपश्य॒न् नन् वन् व॑पश्य॒न्॒. यद् यद॑पश्य॒न् नन् वन् व॑पश्य॒न्॒. यत् । \newline
8. अ॒प॒श्य॒न्॒. यद् यद॑पश्यन् नपश्य॒न्॒. यत् तू॒ष्णीम् तू॒ष्णीं ॅयद॑पश्यन् नपश्य॒न्॒. यत् तू॒ष्णीम् । \newline
9. यत् तू॒ष्णीम् तू॒ष्णीं ॅयद् यत् तू॒ष्णी मा॑घा॒र मा॑घा॒रम् तू॒ष्णीं ॅयद् यत् तू॒ष्णी मा॑घा॒रम् । \newline
10. तू॒ष्णी मा॑घा॒र मा॑घा॒रम् तू॒ष्णीम् तू॒ष्णी मा॑घा॒र मा॑घा॒रय॑ त्याघा॒रय॑ त्याघा॒रम् तू॒ष्णीम् तू॒ष्णी मा॑घा॒र मा॑घा॒रय॑ति । \newline
11. आ॒घा॒र मा॑घा॒रय॑ त्याघा॒रय॑ त्याघा॒र मा॑घा॒र मा॑घा॒रय॑ति य॒ज्ञ्स्य॑ य॒ज्ञ्स्या॑ घा॒रय॑ त्याघा॒र मा॑घा॒र मा॑घा॒रय॑ति य॒ज्ञ्स्य॑ । \newline
12. आ॒घा॒रमित्या᳚ - घा॒रम् । \newline
13. आ॒घा॒रय॑ति य॒ज्ञ्स्य॑ य॒ज्ञ्स्या॑ घा॒रय॑ त्याघा॒रय॑ति य॒ज्ञ्स्या नु॑ख्यात्या॒ अनु॑ख्यात्यै य॒ज्ञ्स्या॑ घा॒रय॑ त्याघा॒रय॑ति य॒ज्ञ्स्या नु॑ख्यात्यै । \newline
14. आ॒घा॒रय॒तीत्या᳚ - घा॒रय॑ति । \newline
15. य॒ज्ञ्स्या नु॑ख्यात्या॒ अनु॑ख्यात्यै य॒ज्ञ्स्य॑ य॒ज्ञ्स्या नु॑ख्यात्या॒ असु॑रे॒ ष्वसु॑रे॒ ष्वनु॑ख्यात्यै य॒ज्ञ्स्य॑ य॒ज्ञ्स्या नु॑ख्यात्या॒ असु॑रेषु । \newline
16. अनु॑ख्यात्या॒ असु॑रे॒ ष्वसु॑रे॒ ष्वनु॑ख्यात्या॒ अनु॑ख्यात्या॒ असु॑रेषु॒ वै वा असु॑रे॒ ष्वनु॑ख्यात्या॒ अनु॑ख्यात्या॒ असु॑रेषु॒ वै । \newline
17. अनु॑ख्यात्या॒ इत्यनु॑ - ख्या॒त्यै॒ । \newline
18. असु॑रेषु॒ वै वा असु॑रे॒ ष्वसु॑रेषु॒ वै य॒ज्ञो य॒ज्ञो वा असु॑रे॒ ष्वसु॑रेषु॒ वै य॒ज्ञ्ः । \newline
19. वै य॒ज्ञो य॒ज्ञो वै वै य॒ज्ञ् आ॑सी दासीद् य॒ज्ञो वै वै य॒ज्ञ् आ॑सीत् । \newline
20. य॒ज्ञ् आ॑सी दासीद् य॒ज्ञो य॒ज्ञ् आ॑सी॒त् तम् त मा॑सीद् य॒ज्ञो य॒ज्ञ् आ॑सी॒त् तम् । \newline
21. आ॒सी॒त् तम् त मा॑सी दासी॒त् तम् दे॒वा दे॒वा स्त मा॑सी दासी॒त् तम् दे॒वाः । \newline
22. तम् दे॒वा दे॒वा स्तम् तम् दे॒वा स्तू᳚ष्णीꣳहो॒मेन॑ तूष्णीꣳहो॒मेन॑ दे॒वा स्तम् तम् दे॒वा 
स्तू᳚ष्णीꣳहो॒मेन॑ । \newline
23. दे॒वा स्तू᳚ष्णीꣳहो॒मेन॑ तूष्णीꣳहो॒मेन॑ दे॒वा दे॒वा स्तू᳚ष्णीꣳहो॒मेना॑ वृञ्जता वृञ्जत तूष्णीꣳहो॒मेन॑ दे॒वा दे॒वा स्तू᳚ष्णीꣳहो॒मेना॑ वृञ्जत । \newline
24. तू॒ष्णीꣳ॒॒हो॒मेना॑ वृञ्जता वृञ्जत तूष्णीꣳहो॒मेन॑ तूष्णीꣳहो॒मेना॑ वृञ्जत॒ यद् यद॑वृञ्जत तूष्णीꣳहो॒मेन॑ तूष्णीꣳहो॒मेना॑ वृञ्जत॒ यत् । \newline
25. तू॒ष्णीꣳ॒॒हो॒मेनेति॑ तूष्णीम् - हो॒मेन॑ । \newline
26. अ॒वृ॒ञ्ज॒त॒ यद् यद॑वृञ्जता वृञ्जत॒ यत् तू॒ष्णीम् तू॒ष्णीं ॅयद॑वृञ्जता वृञ्जत॒ यत् तू॒ष्णीम् । \newline
27. यत् तू॒ष्णीम् तू॒ष्णीं ॅयद् यत् तू॒ष्णी मा॑घा॒र मा॑घा॒रम् तू॒ष्णीं ॅयद् यत् तू॒ष्णी मा॑घा॒रम् । \newline
28. तू॒ष्णी मा॑घा॒र मा॑घा॒रम् तू॒ष्णीम् तू॒ष्णी मा॑घा॒र मा॑घा॒रय॑ त्याघा॒रय॑ त्याघा॒रम् तू॒ष्णीम् तू॒ष्णी मा॑घा॒र मा॑घा॒रय॑ति । \newline
29. आ॒घा॒र मा॑घा॒रय॑ त्याघा॒रय॑ त्याघा॒र मा॑घा॒र मा॑घा॒रय॑ति॒ भ्रातृ॑व्यस्य॒ भ्रातृ॑व्यस्या घा॒रय॑ त्याघा॒र मा॑घा॒र मा॑घा॒रय॑ति॒ भ्रातृ॑व्यस्य । \newline
30. आ॒घा॒रमित्या᳚ - घा॒रम् । \newline
31. आ॒घा॒रय॑ति॒ भ्रातृ॑व्यस्य॒ भ्रातृ॑व्यस्या घा॒रय॑ त्याघा॒रय॑ति॒ भ्रातृ॑व्य स्यै॒वैव भ्रातृ॑व्यस्या घा॒रय॑ त्याघा॒रय॑ति॒ भ्रातृ॑व्यस्यै॒व । \newline
32. आ॒घा॒रय॒तीत्या᳚ - घा॒रय॑ति । \newline
33. भ्रातृ॑व्यस्यै॒ वैव भ्रातृ॑व्यस्य॒ भ्रातृ॑व्यस्यै॒व तत् तदे॒व भ्रातृ॑व्यस्य॒ भ्रातृ॑व्यस्यै॒व तत् । \newline
34. ए॒व तत् तदे॒ वैव तद् य॒ज्ञ्ं ॅय॒ज्ञ्म् तदे॒ वैव तद् य॒ज्ञ्म् । \newline
35. तद् य॒ज्ञ्ं ॅय॒ज्ञ्म् तत् तद् य॒ज्ञ्ं ॅवृ॑ङ्क्ते वृङ्क्ते य॒ज्ञ्म् तत् तद् य॒ज्ञ्ं ॅवृ॑ङ्क्ते । \newline
36. य॒ज्ञ्ं ॅवृ॑ङ्क्ते वृङ्क्ते य॒ज्ञ्ं ॅय॒ज्ञ्ं ॅवृ॑ङ्क्ते परि॒धीन् प॑रि॒धीन् वृ॑ङ्क्ते य॒ज्ञ्ं ॅय॒ज्ञ्ं ॅवृ॑ङ्क्ते परि॒धीन् । \newline
37. वृ॒ङ्क्ते॒ प॒रि॒धीन् प॑रि॒धीन् वृ॑ङ्क्ते वृङ्क्ते परि॒धीन् थ्सꣳ सम् प॑रि॒धीन् वृ॑ङ्क्ते वृङ्क्ते परि॒धीन् थ्सम् । \newline
38. प॒रि॒धीन् थ्सꣳ सम् प॑रि॒धीन् प॑रि॒धीन् थ्सम् मा᳚र्ष्टि मार्ष्टि॒ सम् प॑रि॒धीन् प॑रि॒धीन् थ्सम् मा᳚र्ष्टि । \newline
39. प॒रि॒धीनिति॑ परि - धीन् । \newline
40. सम् मा᳚र्ष्टि मार्ष्टि॒ सꣳ सम् मा᳚र्ष्टि पु॒नाति॑ पु॒नाति॑ मार्ष्टि॒ सꣳ सम् मा᳚र्ष्टि पु॒नाति॑ । \newline
41. मा॒र्ष्टि॒ पु॒नाति॑ पु॒नाति॑ मार्ष्टि मार्ष्टि पु॒ना त्ये॒वैव पु॒नाति॑ मार्ष्टि मार्ष्टि पु॒ना त्ये॒व । \newline
42. पु॒ना त्ये॒वैव पु॒नाति॑ पु॒ना त्ये॒वैना॑ नेना ने॒व पु॒नाति॑ पु॒ना त्ये॒वैनान्॑ । \newline
43. ए॒वैना॑ नेना ने॒वै वैना॒न् त्रिस्त्रि॒ स्त्रिस्त्रि॑ रेना ने॒वै वैना॒न् त्रिस्त्रिः॑ । \newline
44. ए॒ना॒न् त्रिस्त्रि॒ स्त्रिस्त्रि॑ रेना नेना॒न् त्रिस्त्रिः॒ सꣳ सम् त्रिस्त्रि॑ रेना नेना॒न् त्रिस्त्रिः॒ सम् । \newline
45. त्रिस्त्रिः॒ सꣳ सम् त्रिस्त्रि॒ स्त्रिस्त्रिः॒ सम् मा᳚र्ष्टि मार्ष्टि॒ सम् त्रिस्त्रि॒ स्त्रिस्त्रिः॒ सम् मा᳚र्ष्टि । \newline
46. त्रिस्त्रि॒रिति॒ त्रिः - त्रिः॒ । \newline
47. सम् मा᳚र्ष्टि मार्ष्टि॒ सꣳ सम् मा᳚र्ष्टि॒ त्र्या॑वृ॒त् त्र्या॑वृन् मार्ष्टि॒ सꣳ सम् मा᳚र्ष्टि॒ त्र्या॑वृत् । \newline
48. मा॒र्ष्टि॒ त्र्या॑वृ॒त् त्र्या॑वृन् मार्ष्टि मार्ष्टि॒ त्र्या॑वृ॒द्धि हि त्र्या॑वृन् मार्ष्टि मार्ष्टि॒ त्र्या॑वृ॒द्धि । \newline
49. त्र्या॑वृ॒द्धि हि त्र्या॑वृ॒त् त्र्या॑वृ॒द्धि य॒ज्ञो य॒ज्ञो हि त्र्या॑वृ॒त् त्र्या॑वृ॒द्धि य॒ज्ञ्ः । \newline
50. त्र्या॑वृ॒दिति॒ त्रि - आ॒वृ॒त् । \newline
51. हि य॒ज्ञो य॒ज्ञो हि हि य॒ज्ञो ऽथो॒ अथो॑ य॒ज्ञो हि हि य॒ज्ञो ऽथो᳚ । \newline
52. य॒ज्ञो ऽथो॒ अथो॑ य॒ज्ञो य॒ज्ञो ऽथो॒ रक्ष॑साꣳ॒॒ रक्ष॑सा॒ मथो॑ य॒ज्ञो य॒ज्ञो ऽथो॒ रक्ष॑साम् । \newline
53. अथो॒ रक्ष॑साꣳ॒॒ रक्ष॑सा॒ मथो॒ अथो॒ रक्ष॑सा॒ मप॑हत्या॒ अप॑हत्यै॒ रक्ष॑सा॒ मथो॒ अथो॒ रक्ष॑सा॒ मप॑हत्यै । \newline
54. अथो॒ इत्यथो᳚ । \newline
55. रक्ष॑सा॒ मप॑हत्या॒ अप॑हत्यै॒ रक्ष॑साꣳ॒॒ रक्ष॑सा॒ मप॑हत्यै॒ द्वाद॑श॒ द्वाद॒शा प॑हत्यै॒ रक्ष॑साꣳ॒॒ रक्ष॑सा॒ मप॑हत्यै॒ द्वाद॑श । \newline
56. अप॑हत्यै॒ द्वाद॑श॒ द्वाद॒शा प॑हत्या॒ अप॑हत्यै॒ द्वाद॑श॒ सꣳ सम् द्वाद॒शा प॑हत्या॒ अप॑हत्यै॒ द्वाद॑श॒ सम् । \newline
57. अप॑हत्या॒ इत्यप॑ - ह॒त्यै॒ । \newline
58. द्वाद॑श॒ सꣳ सम् द्वाद॑श॒ द्वाद॑श॒ सम् प॑द्यन्ते पद्यन्ते॒ सम् द्वाद॑श॒ द्वाद॑श॒ सम् प॑द्यन्ते । \newline
59. सम् प॑द्यन्ते पद्यन्ते॒ सꣳ सम् प॑द्यन्ते॒ द्वाद॑श॒ द्वाद॑श पद्यन्ते॒ सꣳ सम् प॑द्यन्ते॒ द्वाद॑श । \newline
60. प॒द्य॒न्ते॒ द्वाद॑श॒ द्वाद॑श पद्यन्ते पद्यन्ते॒ द्वाद॑श॒ मासा॒ मासा॒ द्वाद॑श पद्यन्ते पद्यन्ते॒ द्वाद॑श॒ मासाः᳚ । \newline
61. द्वाद॑श॒ मासा॒ मासा॒ द्वाद॑श॒ द्वाद॑श॒ मासाः᳚ संॅवथ्स॒रः सं॑ॅवथ्स॒रो मासा॒ द्वाद॑श॒ द्वाद॑श॒ मासाः᳚ संॅवथ्स॒रः । \newline
\pagebreak
\markright{ TS 6.3.7.3  \hfill https://www.vedavms.in \hfill}

\section{ TS 6.3.7.3 }

\textbf{TS 6.3.7.3 } \newline
\textbf{Samhita Paata} \newline

मासाः᳚ संॅवथ्स॒रः सं॑ॅवथ्स॒रमे॒व प्री॑णा॒त्यथो॑ संॅवथ्स॒रमे॒वास्मा॒ उप॑ दधाति सुव॒र्गस्य॑ लो॒कस्य॒ सम॑ष्ट्यै॒ शिरो॒ वा ए॒तद् य॒ज्ञ्स्य॒ यदा॑घा॒रो᳚ऽग्निः सर्वा॑ दे॒वता॒ यदा॑घा॒र-मा॑घा॒रय॑ति शीर्.ष॒त ए॒व य॒ज्ञ्स्य॒ यज॑मानः॒ सर्वा॑ दे॒वता॒ अव॑ रुन्धे॒ शिरो॒ वा ए॒तद् य॒ज्ञ्स्य॒ यदा॑घा॒र आ॒त्मा प॒शुरा॑घा॒रमा॒घार्य॑ प॒शुꣳ सम॑नक्त्या॒त्मन्ने॒व य॒ज्ञ्स्य॒- [  ] \newline

\textbf{Pada Paata} \newline

मासाः᳚ । सं॒ॅव॒थ्स॒र इति॑ सं-व॒थ्स॒रः । सं॒ॅव॒थ्स॒रमिति॑ सं-व॒थ्स॒रम् । ए॒व । प्री॒णा॒ति॒ । अथो॒ इति॑ । सं॒ॅव॒थ्स॒रमिति॑ सं - व॒थ्स॒रम् । ए॒व । अ॒स्मै॒ । उपेति॑ । द॒धा॒ति॒ । सु॒व॒र्गस्येति॑ सुवः - गस्य॑ । लो॒कस्य॑ । सम॑ष्ट्या॒ इति॒ सं - अ॒ष्ट्यै॒ । शिरः॑ । वै । ए॒तत् । य॒ज्ञ्स्य॑ । यत् । आ॒घा॒र इत्या᳚ - घा॒रः । अ॒ग्निः । सर्वाः᳚ । दे॒वताः᳚ । यत् । आ॒घा॒रमित्या᳚-घा॒रम् । आ॒घा॒रय॒तीत्या᳚-घा॒रय॑ति । शी॒र्॒.ष॒तः । ए॒व । य॒ज्ञ्स्य॑ । यज॑मानः । सर्वाः᳚ । दे॒वताः᳚ । अवेति॑ । रु॒न्धे॒ । शिरः॑ । वै । ए॒तत् । य॒ज्ञ्स्य॑ । यत् । आ॒घा॒र इत्या᳚ - घा॒रः । आ॒त्मा । प॒शुः । आ॒घा॒रमित्या᳚ - घा॒रम् । आ॒घार्येत्या᳚ - घार्य॑ । प॒शुम् । समिति॑ । अ॒न॒क्ति॒ । आ॒त्मन्न् । ए॒व । य॒ज्ञ्स्य॑ ।  \newline


\textbf{Krama Paata} \newline

मासाः᳚ सम्ॅवथ्स॒रः । स॒म्ॅव॒थ्स॒रः स॑म्ॅवथ्स॒रम् । स॒म्ॅव॒थ्स॒र इति॑ सम् - व॒थ्स॒रः । स॒म्ॅव॒थ्स॒रमे॒व । स॒म्ॅव॒थ्स॒रमिति॑ सम् - व॒थ्स॒रम् । ए॒व प्री॑णाति । प्री॒णा॒त्यथो᳚ । अथो॑ सम्ॅवथ्स॒रम् । अथो॒ इत्यथो᳚ । स॒म्ॅव॒थ्स॒रमे॒व । स॒म्ॅव॒थ्स॒रमिति॑ सम् - व॒थ्स॒रम् । ए॒वास्मै᳚ । अ॒स्मा॒ उप॑ । उप॑ दधाति । द॒धा॒ति॒ सु॒व॒र्गस्य॑ । सु॒व॒र्गस्य॑ लो॒कस्य॑ । सु॒व॒र्गस्येति॑ सुवः - गस्य॑ । लो॒कस्य॒ सम॑ष्ट्‍यै । सम॑ष्ट्‍यै॒ शिरः॑ । सम॑ष्ट्‍या॒ इति॒ सम् - अ॒ष्ट्‍यै॒ । शिरो॒ वै । वा ए॒तत् । ए॒तद् य॒ज्ञ्स्य॑ । य॒ज्ञ्स्य॒ यत् । यदा॑घा॒रः । आ॒घा॒रो᳚ऽग्निः । आ॒घा॒र इत्या᳚ - घा॒रः । अ॒ग्निः सर्वाः᳚ । सर्वा॑ दे॒वताः᳚ । दे॒वता॒ यत् । यदा॑घा॒रम् । आ॒घा॒रमा॑घा॒रय॑ति । आ॒घा॒रमित्या᳚ - घा॒रम् । आ॒घा॒रय॑ति शीर्.ष॒तः । आ॒घा॒रय॒तीत्या᳚ - घा॒रय॑ति । शी॒र्.॒ष॒त ए॒व । ए॒व य॒ज्ञ्स्य॑ । य॒ज्ञ्स्य॒ यज॑मानः । यज॑मानः॒ सर्वाः᳚ । सर्वा॑ दे॒वताः᳚ । दे॒वता॒ अव॑ । अव॑ रुन्धे । रु॒न्धे॒ शिरः॑ । शिरो॒ वै । वा ए॒तत् । ए॒तद् य॒ज्ञ्स्य॑ । य॒ज्ञ्स्य॒ यत् । यदा॑घा॒रः । आ॒घा॒र आ॒त्मा । आ॒घा॒र इत्या᳚ - घा॒रः । आ॒त्मा प॒शुः । प॒शुरा॑घा॒रम् । आ॒घा॒रमा॒घार्य॑ । आ॒घा॒रमित्या᳚ - घा॒रम् । आ॒घार्य॑ प॒शुम् । आ॒घार्येत्या᳚ - घार्य॑ । प॒शुꣳ सम् । सम॑नक्ति । अ॒न॒क्त्या॒त्मन्न् । आ॒त्मन्ने॒व । ए॒व य॒ज्ञ्स्य॑ । य॒ज्ञ्स्य॒ शिरः॑ \newline

\textbf{Jatai Paata} \newline

1. मासाः᳚ संॅवथ्स॒रः सं॑ॅवथ्स॒रो मासा॒ मासाः᳚ संॅवथ्स॒रः । \newline
2. सं॒ॅव॒थ्स॒रः सं॑ॅवथ्स॒रꣳ सं॑ॅवथ्स॒रꣳ सं॑ॅवथ्स॒रः सं॑ॅवथ्स॒रः सं॑ॅवथ्स॒रम् । \newline
3. सं॒ॅव॒थ्स॒र इति॑ सं - व॒थ्स॒रः । \newline
4. सं॒ॅव॒थ्स॒र मे॒वैव सं॑ॅवथ्स॒रꣳ सं॑ॅवथ्स॒र मे॒व । \newline
5. सं॒ॅव॒थ्स॒रमिति॑ सं - व॒थ्स॒रम् । \newline
6. ए॒व प्री॑णाति प्रीणा त्ये॒वैव प्री॑णाति । \newline
7. प्री॒णा॒ त्यथो॒ अथो᳚ प्रीणाति प्रीणा॒ त्यथो᳚ । \newline
8. अथो॑ संॅवथ्स॒रꣳ सं॑ॅवथ्स॒र मथो॒ अथो॑ संॅवथ्स॒रम् । \newline
9. अथो॒ इत्यथो᳚ । \newline
10. सं॒ॅव॒थ्स॒र मे॒वैव सं॑ॅवथ्स॒रꣳ सं॑ॅवथ्स॒र मे॒व । \newline
11. सं॒ॅव॒थ्स॒रमिति॑ सं - व॒थ्स॒रम् । \newline
12. ए॒वास्मा॑ अस्मा ए॒वै वास्मै᳚ । \newline
13. अ॒स्मा॒ उपोपा᳚स्मा अस्मा॒ उप॑ । \newline
14. उप॑ दधाति दधा॒ त्युपोप॑ दधाति । \newline
15. द॒धा॒ति॒ सु॒व॒र्गस्य॑ सुव॒र्गस्य॑ दधाति दधाति सुव॒र्गस्य॑ । \newline
16. सु॒व॒र्गस्य॑ लो॒कस्य॑ लो॒कस्य॑ सुव॒र्गस्य॑ सुव॒र्गस्य॑ लो॒कस्य॑ । \newline
17. सु॒व॒र्गस्येति॑ सुवः - गस्य॑ । \newline
18. लो॒कस्य॒ सम॑ष्ट्यै॒ सम॑ष्ट्यै लो॒कस्य॑ लो॒कस्य॒ सम॑ष्ट्यै । \newline
19. सम॑ष्ट्यै॒ शिरः॒ शिरः॒ सम॑ष्ट्यै॒ सम॑ष्ट्यै॒ शिरः॑ । \newline
20. सम॑ष्ट्या॒ इति॒ सं - अ॒ष्ट्यै॒ । \newline
21. शिरो॒ वै वै शिरः॒ शिरो॒ वै । \newline
22. वा ए॒त दे॒तद् वै वा ए॒तत् । \newline
23. ए॒तद् य॒ज्ञ्स्य॑ य॒ज्ञ् स्यै॒त दे॒तद् य॒ज्ञ्स्य॑ । \newline
24. य॒ज्ञ्स्य॒ यद् यद् य॒ज्ञ्स्य॑ य॒ज्ञ्स्य॒ यत् । \newline
25. यदा॑घा॒र आ॑घा॒रो यद् यदा॑घा॒रः । \newline
26. आ॒घा॒रो᳚ ऽग्नि र॒ग्नि रा॑घा॒र आ॑घा॒रो᳚ ऽग्निः । \newline
27. आ॒घा॒र इत्या᳚ - घा॒रः । \newline
28. अ॒ग्निः सर्वाः॒ सर्वा॑ अ॒ग्नि र॒ग्निः सर्वाः᳚ । \newline
29. सर्वा॑ दे॒वता॑ दे॒वताः॒ सर्वाः॒ सर्वा॑ दे॒वताः᳚ । \newline
30. दे॒वता॒ यद् यद् दे॒वता॑ दे॒वता॒ यत् । \newline
31. यदा॑घा॒र मा॑घा॒रं ॅयद् यदा॑घा॒रम् । \newline
32. आ॒घा॒र मा॑घा॒रय॑ त्याघा॒रय॑ त्याघा॒र मा॑घा॒र मा॑घा॒रय॑ति । \newline
33. आ॒घा॒रमित्या᳚ - घा॒रम् । \newline
34. आ॒घा॒रय॑ति शीर्.ष॒तः शी॑र्.ष॒त आ॑घा॒रय॑ त्याघा॒रय॑ति शीर्.ष॒तः । \newline
35. आ॒घा॒रय॒तीत्या᳚ - घा॒रय॑ति । \newline
36. शी॒र्॒.ष॒त ए॒वैव शी॑र्.ष॒तः शी॑र्.ष॒त ए॒व । \newline
37. ए॒व य॒ज्ञ्स्य॑ य॒ज्ञ्स्यै॒ वैव य॒ज्ञ्स्य॑ । \newline
38. य॒ज्ञ्स्य॒ यज॑मानो॒ यज॑मानो य॒ज्ञ्स्य॑ य॒ज्ञ्स्य॒ यज॑मानः । \newline
39. यज॑मानः॒ सर्वाः॒ सर्वा॒ यज॑मानो॒ यज॑मानः॒ सर्वाः᳚ । \newline
40. सर्वा॑ दे॒वता॑ दे॒वताः॒ सर्वाः॒ सर्वा॑ दे॒वताः᳚ । \newline
41. दे॒वता॒ अवाव॑ दे॒वता॑ दे॒वता॒ अव॑ । \newline
42. अव॑ रुन्धे रु॒न्धे ऽवाव॑ रुन्धे । \newline
43. रु॒न्धे॒ शिरः॒ शिरो॑ रुन्धे रुन्धे॒ शिरः॑ । \newline
44. शिरो॒ वै वै शिरः॒ शिरो॒ वै । \newline
45. वा ए॒त दे॒तद् वै वा ए॒तत् । \newline
46. ए॒तद् य॒ज्ञ्स्य॑ य॒ज्ञ् स्यै॒त दे॒तद् य॒ज्ञ्स्य॑ । \newline
47. य॒ज्ञ्स्य॒ यद् यद् य॒ज्ञ्स्य॑ य॒ज्ञ्स्य॒ यत् । \newline
48. यदा॑घा॒र आ॑घा॒रो यद् यदा॑घा॒रः । \newline
49. आ॒घा॒र आ॒त्मा ऽऽत्मा ऽऽघा॒र आ॑घा॒र आ॒त्मा । \newline
50. आ॒घा॒र इत्या᳚ - घा॒रः । \newline
51. आ॒त्मा प॒शुः प॒शु रा॒त्मा ऽऽत्मा प॒शुः । \newline
52. प॒शु रा॑घा॒र मा॑घा॒रम् प॒शुः प॒शु रा॑घा॒रम् । \newline
53. आ॒घा॒र मा॒घार्या॒ घार्या॑ घा॒र मा॑घा॒र मा॒घार्य॑ । \newline
54. आ॒घा॒रमित्या᳚ - घा॒रम् । \newline
55. आ॒घार्य॑ प॒शुम् प॒शु मा॒घार्या॒ घार्य॑ प॒शुम् । \newline
56. आ॒घार्येत्या᳚ - घार्य॑ । \newline
57. प॒शुꣳ सꣳ सम् प॒शुम् प॒शुꣳ सम् । \newline
58. स म॑नक् त्यनक्ति॒ सꣳ स म॑नक्ति । \newline
59. अ॒न॒क् त्या॒त्मन् ना॒त्मन् न॑नक् त्यनक् त्या॒त्मन्न् । \newline
60. आ॒त्मन् ने॒वै वात्मन् ना॒त्मन् ने॒व । \newline
61. ए॒व य॒ज्ञ्स्य॑ य॒ज्ञ्स्यै॒ वैव य॒ज्ञ्स्य॑ । \newline
62. य॒ज्ञ्स्य॒ शिरः॒ शिरो॑ य॒ज्ञ्स्य॑ य॒ज्ञ्स्य॒ शिरः॑ । \newline

\textbf{Ghana Paata } \newline

1. मासाः᳚ संॅवथ्स॒रः सं॑ॅवथ्स॒रो मासा॒ मासाः᳚ संॅवथ्स॒रः सं॑ॅवथ्स॒रꣳ सं॑ॅवथ्स॒रꣳ सं॑ॅवथ्स॒रो मासा॒ मासाः᳚ संॅवथ्स॒रः सं॑ॅवथ्स॒रम् । \newline
2. सं॒ॅव॒थ्स॒रः सं॑ॅवथ्स॒रꣳ सं॑ॅवथ्स॒रꣳ सं॑ॅवथ्स॒रः सं॑ॅवथ्स॒रः सं॑ॅवथ्स॒र मे॒वैव सं॑ॅवथ्स॒रꣳ सं॑ॅवथ्स॒रः सं॑ॅवथ्स॒रः सं॑ॅवथ्स॒र मे॒व । \newline
3. सं॒ॅव॒थ्स॒र इति॑ सं - व॒थ्स॒रः । \newline
4. सं॒ॅव॒थ्स॒र मे॒वैव सं॑ॅवथ्स॒रꣳ सं॑ॅवथ्स॒र मे॒व प्री॑णाति प्रीणा त्ये॒व सं॑ॅवथ्स॒रꣳ सं॑ॅवथ्स॒र मे॒व प्री॑णाति । \newline
5. सं॒ॅव॒थ्स॒रमिति॑ सं - व॒थ्स॒रम् । \newline
6. ए॒व प्री॑णाति प्रीणा त्ये॒वैव प्री॑णा॒ त्यथो॒ अथो᳚ प्रीणा त्ये॒वैव प्री॑णा॒ त्यथो᳚ । \newline
7. प्री॒णा॒ त्यथो॒ अथो᳚ प्रीणाति प्रीणा॒ त्यथो॑ संॅवथ्स॒रꣳ सं॑ॅवथ्स॒र मथो᳚ प्रीणाति प्रीणा॒ त्यथो॑ संॅवथ्स॒रम् । \newline
8. अथो॑ संॅवथ्स॒रꣳ सं॑ॅवथ्स॒र मथो॒ अथो॑ संॅवथ्स॒र मे॒वैव सं॑ॅवथ्स॒र मथो॒ अथो॑ संॅवथ्स॒र मे॒व । \newline
9. अथो॒ इत्यथो᳚ । \newline
10. सं॒ॅव॒थ्स॒र मे॒वैव सं॑ॅवथ्स॒रꣳ सं॑ॅवथ्स॒र मे॒वास्मा॑ अस्मा ए॒व सं॑ॅवथ्स॒रꣳ सं॑ॅवथ्स॒र मे॒वास्मै᳚ । \newline
11. सं॒ॅव॒थ्स॒रमिति॑ सं - व॒थ्स॒रम् । \newline
12. ए॒वास्मा॑ अस्मा ए॒वै वास्मा॒ उपोपा᳚स्मा ए॒वै वास्मा॒ उप॑ । \newline
13. अ॒स्मा॒ उपोपा᳚स्मा अस्मा॒ उप॑ दधाति दधा॒ त्युपा᳚स्मा अस्मा॒ उप॑ दधाति । \newline
14. उप॑ दधाति दधा॒ त्युपोप॑ दधाति सुव॒र्गस्य॑ सुव॒र्गस्य॑ दधा॒ त्युपोप॑ दधाति सुव॒र्गस्य॑ । \newline
15. द॒धा॒ति॒ सु॒व॒र्गस्य॑ सुव॒र्गस्य॑ दधाति दधाति सुव॒र्गस्य॑ लो॒कस्य॑ लो॒कस्य॑ सुव॒र्गस्य॑ दधाति दधाति सुव॒र्गस्य॑ लो॒कस्य॑ । \newline
16. सु॒व॒र्गस्य॑ लो॒कस्य॑ लो॒कस्य॑ सुव॒र्गस्य॑ सुव॒र्गस्य॑ लो॒कस्य॒ सम॑ष्ट्यै॒ सम॑ष्ट्यै लो॒कस्य॑ सुव॒र्गस्य॑ सुव॒र्गस्य॑ लो॒कस्य॒ सम॑ष्ट्यै । \newline
17. सु॒व॒र्गस्येति॑ सुवः - गस्य॑ । \newline
18. लो॒कस्य॒ सम॑ष्ट्यै॒ सम॑ष्ट्यै लो॒कस्य॑ लो॒कस्य॒ सम॑ष्ट्यै॒ शिरः॒ शिरः॒ सम॑ष्ट्यै लो॒कस्य॑ लो॒कस्य॒ सम॑ष्ट्यै॒ शिरः॑ । \newline
19. सम॑ष्ट्यै॒ शिरः॒ शिरः॒ सम॑ष्ट्यै॒ सम॑ष्ट्यै॒ शिरो॒ वै वै शिरः॒ सम॑ष्ट्यै॒ सम॑ष्ट्यै॒ शिरो॒ वै । \newline
20. सम॑ष्ट्या॒ इति॒ सं - अ॒ष्ट्यै॒ । \newline
21. शिरो॒ वै वै शिरः॒ शिरो॒ वा ए॒त दे॒तद् वै शिरः॒ शिरो॒ वा ए॒तत् । \newline
22. वा ए॒त दे॒तद् वै वा ए॒तद् य॒ज्ञ्स्य॑ य॒ज्ञ् स्यै॒तद् वै वा ए॒तद् य॒ज्ञ्स्य॑ । \newline
23. ए॒तद् य॒ज्ञ्स्य॑ य॒ज्ञ्स्यै॒त दे॒तद् य॒ज्ञ्स्य॒ यद् यद् य॒ज्ञ्स्यै॒त दे॒तद् य॒ज्ञ्स्य॒ यत् । \newline
24. य॒ज्ञ्स्य॒ यद् यद् य॒ज्ञ्स्य॑ य॒ज्ञ्स्य॒ यदा॑घा॒र आ॑घा॒रो यद् य॒ज्ञ्स्य॑ य॒ज्ञ्स्य॒ यदा॑घा॒रः । \newline
25. यदा॑घा॒र आ॑घा॒रो यद् यदा॑घा॒रो᳚ ऽग्नि र॒ग्नि रा॑घा॒रो यद् यदा॑घा॒रो᳚ ऽग्निः । \newline
26. आ॒घा॒रो᳚ ऽग्नि र॒ग्नि रा॑घा॒र आ॑घा॒रो᳚ ऽग्निः सर्वाः॒ सर्वा॑ अ॒ग्नि रा॑घा॒र आ॑घा॒रो᳚ ऽग्निः सर्वाः᳚ । \newline
27. आ॒घा॒र इत्या᳚ - घा॒रः । \newline
28. अ॒ग्निः सर्वाः॒ सर्वा॑ अ॒ग्नि र॒ग्निः सर्वा॑ दे॒वता॑ दे॒वताः॒ सर्वा॑ अ॒ग्नि र॒ग्निः सर्वा॑ दे॒वताः᳚ । \newline
29. सर्वा॑ दे॒वता॑ दे॒वताः॒ सर्वाः॒ सर्वा॑ दे॒वता॒ यद् यद् दे॒वताः॒ सर्वाः॒ सर्वा॑ दे॒वता॒ यत् । \newline
30. दे॒वता॒ यद् यद् दे॒वता॑ दे॒वता॒ यदा॑घा॒र मा॑घा॒रं ॅयद् दे॒वता॑ दे॒वता॒ यदा॑घा॒रम् । \newline
31. यदा॑घा॒र मा॑घा॒रं ॅयद् यदा॑घा॒र मा॑घा॒रय॑ त्याघा॒रय॑ त्याघा॒रं ॅयद् यदा॑घा॒र मा॑घा॒रय॑ति । \newline
32. आ॒घा॒र मा॑घा॒रय॑ त्याघा॒रय॑ त्याघा॒र मा॑घा॒र मा॑घा॒रय॑ति शीर्.ष॒तः शी॑र्.ष॒त आ॑घा॒रय॑ त्याघा॒र मा॑घा॒र मा॑घा॒रय॑ति शीर्.ष॒तः । \newline
33. आ॒घा॒रमित्या᳚ - घा॒रम् । \newline
34. आ॒घा॒रय॑ति शीर्.ष॒तः शी॑र्.ष॒त आ॑घा॒रय॑ त्याघा॒रय॑ति शीर्.ष॒त ए॒वैव शी॑र्.ष॒त आ॑घा॒रय॑ त्याघा॒रय॑ति शीर्.ष॒त ए॒व । \newline
35. आ॒घा॒रय॒तीत्या᳚ - घा॒रय॑ति । \newline
36. शी॒र्॒.ष॒त ए॒वैव शी॑र्.ष॒तः शी॑र्.ष॒त ए॒व य॒ज्ञ्स्य॑ य॒ज्ञ्स्यै॒व शी॑र्.ष॒तः शी॑र्.ष॒त ए॒व य॒ज्ञ्स्य॑ । \newline
37. ए॒व य॒ज्ञ्स्य॑ य॒ज्ञ्स्यै॒वैव य॒ज्ञ्स्य॒ यज॑मानो॒ यज॑मानो य॒ज्ञ्स्यै॒वैव य॒ज्ञ्स्य॒ यज॑मानः । \newline
38. य॒ज्ञ्स्य॒ यज॑मानो॒ यज॑मानो य॒ज्ञ्स्य॑ य॒ज्ञ्स्य॒ यज॑मानः॒ सर्वाः॒ सर्वा॒ यज॑मानो य॒ज्ञ्स्य॑ य॒ज्ञ्स्य॒ यज॑मानः॒ सर्वाः᳚ । \newline
39. यज॑मानः॒ सर्वाः॒ सर्वा॒ यज॑मानो॒ यज॑मानः॒ सर्वा॑ दे॒वता॑ दे॒वताः॒ सर्वा॒ यज॑मानो॒ यज॑मानः॒ सर्वा॑ दे॒वताः᳚ । \newline
40. सर्वा॑ दे॒वता॑ दे॒वताः॒ सर्वाः॒ सर्वा॑ दे॒वता॒ अवाव॑ दे॒वताः॒ सर्वाः॒ सर्वा॑ दे॒वता॒ अव॑ । \newline
41. दे॒वता॒ अवाव॑ दे॒वता॑ दे॒वता॒ अव॑ रुन्धे रु॒न्धे ऽव॑ दे॒वता॑ दे॒वता॒ अव॑ रुन्धे । \newline
42. अव॑ रुन्धे रु॒न्धे ऽवाव॑ रुन्धे॒ शिरः॒ शिरो॑ रु॒न्धे ऽवाव॑ रुन्धे॒ शिरः॑ । \newline
43. रु॒न्धे॒ शिरः॒ शिरो॑ रुन्धे रुन्धे॒ शिरो॒ वै वै शिरो॑ रुन्धे रुन्धे॒ शिरो॒ वै । \newline
44. शिरो॒ वै वै शिरः॒ शिरो॒ वा ए॒त दे॒तद् वै शिरः॒ शिरो॒ वा ए॒तत् । \newline
45. वा ए॒त दे॒तद् वै वा ए॒तद् य॒ज्ञ्स्य॑ य॒ज्ञ्स्यै॒तद् वै वा ए॒तद् य॒ज्ञ्स्य॑ । \newline
46. ए॒तद् य॒ज्ञ्स्य॑ य॒ज्ञ्स्यै॒त दे॒तद् य॒ज्ञ्स्य॒ यद् यद् य॒ज्ञ्स्यै॒त दे॒तद् य॒ज्ञ्स्य॒ यत् । \newline
47. य॒ज्ञ्स्य॒ यद् यद् य॒ज्ञ्स्य॑ य॒ज्ञ्स्य॒ यदा॑घा॒र आ॑घा॒रो यद् य॒ज्ञ्स्य॑ य॒ज्ञ्स्य॒ यदा॑घा॒रः । \newline
48. यदा॑घा॒र आ॑घा॒रो यद् यदा॑घा॒र आ॒त्मा ऽऽत्मा ऽऽघा॒रो यद् यदा॑घा॒र आ॒त्मा । \newline
49. आ॒घा॒र आ॒त्मा ऽऽत्मा ऽऽघा॒र आ॑घा॒र आ॒त्मा प॒शुः प॒शु रा॒त्मा ऽऽघा॒र आ॑घा॒र आ॒त्मा प॒शुः । \newline
50. आ॒घा॒र इत्या᳚ - घा॒रः । \newline
51. आ॒त्मा प॒शुः प॒शु रा॒त्मा ऽऽत्मा प॒शु रा॑घा॒र मा॑घा॒रम् प॒शु रा॒त्मा ऽऽत्मा प॒शु रा॑घा॒रम् । \newline
52. प॒शु रा॑घा॒र मा॑घा॒रम् प॒शुः प॒शु रा॑घा॒र मा॒घार्या॒ घार्या॑ घा॒रम् प॒शुः प॒शु रा॑घा॒र मा॒घार्य॑ । \newline
53. आ॒घा॒र मा॒घार्या॒ घार्या॑ घा॒र मा॑घा॒र मा॒घार्य॑ प॒शुम् प॒शु मा॒घार्या॑ घा॒र मा॑घा॒र मा॒घार्य॑ प॒शुम् । \newline
54. आ॒घा॒रमित्या᳚ - घा॒रम् । \newline
55. आ॒घार्य॑ प॒शुम् प॒शु मा॒घार्या॒ घार्य॑ प॒शुꣳ सꣳ सम् प॒शु मा॒घार्या॒ घार्य॑ प॒शुꣳ सम् । \newline
56. आ॒घार्येत्या᳚ - घार्य॑ । \newline
57. प॒शुꣳ सꣳ सम् प॒शुम् प॒शुꣳ स म॑नक् त्यनक्ति॒ सम् प॒शुम् प॒शुꣳ स म॑नक्ति । \newline
58. स म॑नक् त्यनक्ति॒ सꣳ स म॑नक् त्या॒त्मन् ना॒त्मन् न॑नक्ति॒ सꣳ स म॑नक् त्या॒त्मन्न् । \newline
59. अ॒न॒क् त्या॒त्मन् ना॒त्मन् न॑नक् त्यनक् त्या॒त्मन् ने॒वै वात्मन् न॑नक् त्यनक् त्या॒त्मन्ने॒व । \newline
60. आ॒त्मन् ने॒वै वात्मन् ना॒त्मन् ने॒व य॒ज्ञ्स्य॑ य॒ज्ञ्स्यै॒ वात्मन् ना॒त्मन् ने॒व य॒ज्ञ्स्य॑ । \newline
61. ए॒व य॒ज्ञ्स्य॑ य॒ज्ञ्स्यै॒ वैव य॒ज्ञ्स्य॒ शिरः॒ शिरो॑ य॒ज्ञ्स्यै॒ वैव य॒ज्ञ्स्य॒ शिरः॑ । \newline
62. य॒ज्ञ्स्य॒ शिरः॒ शिरो॑ य॒ज्ञ्स्य॑ य॒ज्ञ्स्य॒ शिरः॒ प्रति॒ प्रति॒ शिरो॑ य॒ज्ञ्स्य॑ य॒ज्ञ्स्य॒ शिरः॒ प्रति॑ । \newline
\pagebreak
\markright{ TS 6.3.7.4  \hfill https://www.vedavms.in \hfill}

\section{ TS 6.3.7.4 }

\textbf{TS 6.3.7.4 } \newline
\textbf{Samhita Paata} \newline

शिरः॒ प्रति॑ दधाति॒ सं ते᳚ प्रा॒णो वा॒युना॑ गच्छता॒मित्या॑ह वायुदेव॒त्यो॑ वै प्रा॒णो वा॒यावे॒वास्य॑ प्रा॒णं जु॑होति॒ सं ॅयज॑त्रै॒रङ्गा॑नि॒ सं ॅय॒ज्ञ्प॑तिरा॒शिषेत्या॑ह य॒ज्ञ्प॑तिमे॒वास्या॒ऽऽ*शिषं॑ गमयति वि॒श्वरू॑पो॒ वै त्वा॒ष्ट्र उ॒परि॑ष्टात् प॒शुम॒भ्य॑वमी॒त् तस्मा॑-दु॒परि॑ष्टात् प॒शोर्नाव॑ द्यन्ति॒ यदु॒परि॑ष्टात् प॒शुꣳ स॑म॒नक्ति॒ मेद्ध्य॑मे॒वै- [  ] \newline

\textbf{Pada Paata} \newline

शिरः॑ । प्रतीति॑ । द॒धा॒ति॒ । समिति॑ । ते॒ । प्रा॒ण इति॑ प्र - अ॒नः । वा॒युना᳚ । ग॒च्छ॒ता॒म् । इति॑ । आ॒ह॒ । वा॒यु॒दे॒व॒त्य॑ इति॑ वायु-दे॒व॒त्यः॑ । वै । प्रा॒ण इति॑ प्र - अ॒नः । वा॒यौ । ए॒व । अ॒स्य॒ । प्रा॒णमिति॑ प्र - अ॒नम् । जु॒हो॒ति॒ । समिति॑ । यज॑त्रैः । अङ्गा॑नि । समिति॑ । य॒ज्ञ्प॑ति॒रिति॑ य॒ज्ञ् - प॒तिः॒ । आ॒शिषेत्या᳚ - शिषा᳚ । इति॑ । आ॒ह॒ । य॒ज्ञ्प॑ति॒मिति॑ य॒ज्ञ् - प॒ति॒म् । ए॒व । अ॒स्य॒ । आ॒शिष॒मित्या᳚-शिष᳚म् । ग॒म॒य॒ति॒ । वि॒श्वरू॑प॒ इति॑ वि॒श्व - रू॒पः॒ । वै । त्वा॒ष्ट्रः । उ॒परि॑ष्टात् । प॒शुम् । अ॒भीति॑ । अ॒व॒मी॒त् । तस्मा᳚त् । उ॒परि॑ष्टात् । प॒शोः । न । अवेति॑ । द्य॒न्ति॒ । यत् । उ॒परि॑ष्टात् । प॒शुम् । स॒म॒नक्तीति॑ सं - अ॒नक्ति॑ । मेद्ध्य᳚म् । ए॒व ।  \newline


\textbf{Krama Paata} \newline

शिरः॒ प्रति॑ । प्रति॑ दधाति । द॒धा॒ति॒ सम् । सम् ते᳚ । ते॒ प्रा॒णः । प्रा॒णो वा॒युना᳚ । प्रा॒ण इति॑ प्र - अ॒नः । वा॒युना॑ गच्छताम् । ग॒च्छ॒ता॒मिति॑ । इत्या॑ह । आ॒ह॒ वा॒यु॒दे॒व॒त्यः॑ । वा॒यु॒दे॒व॒त्यो॑ वै । वा॒यु॒दे॒व॒त्य॑ इति॑ वायु - दे॒व॒त्यः॑ । वै प्रा॒णः । प्रा॒णो वा॒यौ । प्रा॒ण इति॑ प्र - अ॒नः । वा॒यावे॒व । ए॒वास्य॑ । अ॒स्य॒ प्रा॒णम् । प्रा॒णम् जु॑होति । प्रा॒णमिति॑ प्र - अ॒नम् । जु॒हो॒ति॒ सम् । सम् ॅयज॑त्रैः । यज॑त्रै॒रङ्‍गा॑नि । अङ्‍गा॑नि॒ सम् । सम् ॅय॒ज्ञ्प॑तिः । य॒ज्ञ्प॑तिरा॒शिषा᳚ । य॒ज्ञ्प॑ति॒रिति॑ य॒ज्ञ् - प॒तिः॒ । आ॒शिषेति॑ । आ॒शिषेत्या᳚ - शिषा᳚ । इत्या॑ह । आ॒ह॒ य॒ज्ञ्प॑तिम् । य॒ज्ञ्प॑तिमे॒व । य॒ज्ञ्प॑ति॒मिति॑ य॒ज्ञ् - प॒ति॒म् । ए॒वास्य॑ । अ॒स्या॒शिष᳚म् । आ॒शिष॑म् गमयति । आ॒शिष॒मित्या᳚ - शिष᳚म् । ग॒म॒य॒ति॒ वि॒श्वरू॑पः । वि॒श्वरू॑पो॒ वै । वि॒श्वरू॑प॒ इति॑ वि॒श्व - रू॒पः॒ । वै त्वा॒ष्ट्रः । त्वा॒ष्ट्र उ॒परि॑ष्टात् । उ॒परि॑ष्टात् प॒शुम् । प॒शुम॒भि । अ॒भ्य॑वमीत् । अ॒म॒वी॒त् तस्मा᳚त् । तस्मा॑दु॒परि॑ष्टात् । उ॒परि॑ष्टात् प॒शोः । प॒शोर् न । नाव॑ । अव॑ द्यन्ति । द्य॒न्ति॒ यत् । यदु॒परि॑ष्टात् । उ॒परि॑ष्टात् प॒शुम् । प॒शुꣳ स॑म॒नक्ति॑ । स॒म॒नक्ति॒ मेद्ध्य᳚म् । स॒म॒नक्तीति॑ सम् - अ॒नक्ति॑ । मेद्ध्य॑मे॒व । ए॒वैन᳚म् \newline

\textbf{Jatai Paata} \newline

1. शिरः॒ प्रति॒ प्रति॒ शिरः॒ शिरः॒ प्रति॑ । \newline
2. प्रति॑ दधाति दधाति॒ प्रति॒ प्रति॑ दधाति । \newline
3. द॒धा॒ति॒ सꣳ सम् द॑धाति दधाति॒ सम् । \newline
4. सम् ते॑ ते॒ सꣳ सम् ते᳚ । \newline
5. ते॒ प्रा॒णः प्रा॒ण स्ते॑ ते प्रा॒णः । \newline
6. प्रा॒णो वा॒युना॑ वा॒युना᳚ प्रा॒णः प्रा॒णो वा॒युना᳚ । \newline
7. प्रा॒ण इति॑ प्र - अ॒नः । \newline
8. वा॒युना॑ गच्छताम् गच्छतां ॅवा॒युना॑ वा॒युना॑ गच्छताम् । \newline
9. ग॒च्छ॒ता॒ मितीति॑ गच्छताम् गच्छता॒ मिति॑ । \newline
10. इत्या॑हा॒हे तीत्या॑ह । \newline
11. आ॒ह॒ वा॒यु॒दे॒व॒त्यो॑ वायुदेव॒त्य॑ आहाह वायुदेव॒त्यः॑ । \newline
12. वा॒यु॒दे॒व॒त्यो॑ वै वै वा॑युदेव॒त्यो॑ वायुदेव॒त्यो॑ वै । \newline
13. वा॒यु॒दे॒व॒त्य॑ इति॑ वायु - दे॒व॒त्यः॑ । \newline
14. वै प्रा॒णः प्रा॒णो वै वै प्रा॒णः । \newline
15. प्रा॒णो वा॒यौ वा॒यौ प्रा॒णः प्रा॒णो वा॒यौ । \newline
16. प्रा॒ण इति॑ प्र - अ॒नः । \newline
17. वा॒या वे॒वैव वा॒यौ वा॒या वे॒व । \newline
18. ए॒वास्या᳚ स्यै॒वै वास्य॑ । \newline
19. अ॒स्य॒ प्रा॒णम् प्रा॒ण म॑स्यास्य प्रा॒णम् । \newline
20. प्रा॒णम् जु॑होति जुहोति प्रा॒णम् प्रा॒णम् जु॑होति । \newline
21. प्रा॒णमिति॑ प्र - अ॒नम् । \newline
22. जु॒हो॒ति॒ सꣳ सम् जु॑होति जुहोति॒ सम् । \newline
23. सं ॅयज॑त्रै॒र् यज॑त्रैः॒ सꣳ सं ॅयज॑त्रैः । \newline
24. यज॑त्रै॒ रङ्गा॒ न्यङ्गा॑नि॒ यज॑त्रै॒र् यज॑त्रै॒ रङ्गा॑नि । \newline
25. अङ्गा॑नि॒ सꣳ स मङ्गा॒ न्यङ्गा॑नि॒ सम् । \newline
26. सं ॅय॒ज्ञ्प॑तिर् य॒ज्ञ्प॑तिः॒ सꣳ सं ॅय॒ज्ञ्प॑तिः । \newline
27. य॒ज्ञ्प॑ति रा॒शिषा॒ ऽऽशिषा॑ य॒ज्ञ्प॑तिर् य॒ज्ञ्प॑ति रा॒शिषा᳚ । \newline
28. य॒ज्ञ्प॑ति॒रिति॑ य॒ज्ञ् - प॒तिः॒ । \newline
29. आ॒शिषेती त्या॒शिषा॒ ऽऽशिषेति॑ । \newline
30. आ॒शिषेत्या᳚ - शिषा᳚ । \newline
31. इत्या॑हा॒हे तीत्या॑ह । \newline
32. आ॒ह॒ य॒ज्ञ्प॑तिं ॅय॒ज्ञ्प॑ति माहाह य॒ज्ञ्प॑तिम् । \newline
33. य॒ज्ञ्प॑ति मे॒वैव य॒ज्ञ्प॑तिं ॅय॒ज्ञ्प॑ति मे॒व । \newline
34. य॒ज्ञ्प॑ति॒मिति॑ य॒ज्ञ् - प॒ति॒म् । \newline
35. ए॒वास्या᳚ स्यै॒वै वास्य॑ । \newline
36. अ॒स्या॒ शिष॑ मा॒शिष॑ मस्या स्या॒शिष᳚म् । \newline
37. आ॒शिष॑म् गमयति गमय त्या॒शिष॑ मा॒शिष॑म् गमयति । \newline
38. आ॒शिष॒मित्या᳚ - शिष᳚म् । \newline
39. ग॒म॒य॒ति॒ वि॒श्वरू॑पो वि॒श्वरू॑पो गमयति गमयति वि॒श्वरू॑पः । \newline
40. वि॒श्वरू॑पो॒ वै वै वि॒श्वरू॑पो वि॒श्वरू॑पो॒ वै । \newline
41. वि॒श्वरू॑प॒ इति॑ वि॒श्व - रू॒पः॒ । \newline
42. वै त्वा॒ष्ट्र स्त्वा॒ष्ट्रो वै वै त्वा॒ष्ट्रः । \newline
43. त्वा॒ष्ट्र उ॒परि॑ष्टा दु॒परि॑ष्टात् त्वा॒ष्ट्र स्त्वा॒ष्ट्र उ॒परि॑ष्टात् । \newline
44. उ॒परि॑ष्टात् प॒शुम् प॒शु मु॒परि॑ष्टा दु॒परि॑ष्टात् प॒शुम् । \newline
45. प॒शु म॒भ्य॑भि प॒शुम् प॒शु म॒भि । \newline
46. अ॒भ्य॑वमी दवमी द॒भ्या᳚(1॒)भ्य॑ वमीत् । \newline
47. अ॒व॒मी॒त् तस्मा॒त् तस्मा॑ दवमी दवमी॒त् तस्मा᳚त् । \newline
48. तस्मा॑ दु॒परि॑ष्टा दु॒परि॑ष्टा॒त् तस्मा॒त् तस्मा॑ दु॒परि॑ष्टात् । \newline
49. उ॒परि॑ष्टात् प॒शोः प॒शो रु॒परि॑ष्टा दु॒परि॑ष्टात् प॒शोः । \newline
50. प॒शोर् न न प॒शोः प॒शोर् न । \newline
51. नावाव॒ न नाव॑ । \newline
52. अव॑ द्यन्ति द्य॒न्त्यवाव॑ द्यन्ति । \newline
53. द्य॒न्ति॒ यद् यद् द्य॑न्ति द्यन्ति॒ यत् । \newline
54. यदु॒परि॑ष्टा दु॒परि॑ष्टा॒द् यद् यदु॒परि॑ष्टात् । \newline
55. उ॒परि॑ष्टात् प॒शुम् प॒शु मु॒परि॑ष्टा दु॒परि॑ष्टात् प॒शुम् । \newline
56. प॒शुꣳ स॑म॒नक्ति॑ सम॒नक्ति॑ प॒शुम् प॒शुꣳ स॑म॒नक्ति॑ । \newline
57. स॒म॒नक्ति॒ मेद्ध्य॒म् मेद्ध्यꣳ॑ सम॒नक्ति॑ सम॒नक्ति॒ मेद्ध्य᳚म् । \newline
58. स॒म॒नक्तीति॑ सं - अ॒नक्ति॑ । \newline
59. मेद्ध्य॑ मे॒वैव मेद्ध्य॒म् मेद्ध्य॑ मे॒व । \newline
60. ए॒वैन॑ मेन मे॒वै वैन᳚म् । \newline

\textbf{Ghana Paata } \newline

1. शिरः॒ प्रति॒ प्रति॒ शिरः॒ शिरः॒ प्रति॑ दधाति दधाति॒ प्रति॒ शिरः॒ शिरः॒ प्रति॑ दधाति । \newline
2. प्रति॑ दधाति दधाति॒ प्रति॒ प्रति॑ दधाति॒ सꣳ सम् द॑धाति॒ प्रति॒ प्रति॑ दधाति॒ सम् । \newline
3. द॒धा॒ति॒ सꣳ सम् द॑धाति दधाति॒ सम् ते॑ ते॒ सम् द॑धाति दधाति॒ सम् ते᳚ । \newline
4. सम् ते॑ ते॒ सꣳ सम् ते᳚ प्रा॒णः प्रा॒ण स्ते॒ सꣳ सम् ते᳚ प्रा॒णः । \newline
5. ते॒ प्रा॒णः प्रा॒ण स्ते॑ ते प्रा॒णो वा॒युना॑ वा॒युना᳚ प्रा॒ण स्ते॑ ते प्रा॒णो वा॒युना᳚ । \newline
6. प्रा॒णो वा॒युना॑ वा॒युना᳚ प्रा॒णः प्रा॒णो वा॒युना॑ गच्छताम् गच्छतां ॅवा॒युना᳚ प्रा॒णः प्रा॒णो वा॒युना॑ गच्छताम् । \newline
7. प्रा॒ण इति॑ प्र - अ॒नः । \newline
8. वा॒युना॑ गच्छताम् गच्छतां ॅवा॒युना॑ वा॒युना॑ गच्छता॒ मितीति॑ गच्छतां ॅवा॒युना॑ वा॒युना॑ गच्छता॒ मिति॑ । \newline
9. ग॒च्छ॒ता॒ मितीति॑ गच्छताम् गच्छता॒ मित्या॑हा॒ हेति॑ गच्छताम् गच्छता॒ मित्या॑ह । \newline
10. इत्या॑हा॒हे तीत्या॑ह वायुदेव॒त्यो॑ वायुदेव॒त्य॑ आ॒हे तीत्या॑ह वायुदेव॒त्यः॑ । \newline
11. आ॒ह॒ वा॒यु॒दे॒व॒त्यो॑ वायुदेव॒त्य॑ आहाह वायुदेव॒त्यो॑ वै वै वा॑युदेव॒त्य॑ आहाह वायुदेव॒त्यो॑ वै । \newline
12. वा॒यु॒दे॒व॒त्यो॑ वै वै वा॑युदेव॒त्यो॑ वायुदेव॒त्यो॑ वै प्रा॒णः प्रा॒णो वै वा॑युदेव॒त्यो॑ वायुदेव॒त्यो॑ वै प्रा॒णः । \newline
13. वा॒यु॒दे॒व॒त्य॑ इति॑ वायु - दे॒व॒त्यः॑ । \newline
14. वै प्रा॒णः प्रा॒णो वै वै प्रा॒णो वा॒यौ वा॒यौ प्रा॒णो वै वै प्रा॒णो वा॒यौ । \newline
15. प्रा॒णो वा॒यौ वा॒यौ प्रा॒णः प्रा॒णो वा॒या वे॒वैव वा॒यौ प्रा॒णः प्रा॒णो वा॒या वे॒व । \newline
16. प्रा॒ण इति॑ प्र - अ॒नः । \newline
17. वा॒या वे॒वैव वा॒यौ वा॒या वे॒वास्या᳚ स्यै॒व वा॒यौ वा॒या वे॒वास्य॑ । \newline
18. ए॒वास्या᳚ स्यै॒वै वास्य॑ प्रा॒णम् प्रा॒ण म॑स्यै॒वै वास्य॑ प्रा॒णम् । \newline
19. अ॒स्य॒ प्रा॒णम् प्रा॒ण म॑स्यास्य प्रा॒णम् जु॑होति जुहोति प्रा॒ण म॑स्यास्य प्रा॒णम् जु॑होति । \newline
20. प्रा॒णम् जु॑होति जुहोति प्रा॒णम् प्रा॒णम् जु॑होति॒ सꣳ सम् जु॑होति प्रा॒णम् प्रा॒णम् जु॑होति॒ सम् । \newline
21. प्रा॒णमिति॑ प्र - अ॒नम् । \newline
22. जु॒हो॒ति॒ सꣳ सम् जु॑होति जुहोति॒ सं ॅयज॑त्रै॒र् यज॑त्रैः॒ सम् जु॑होति जुहोति॒ सं ॅयज॑त्रैः । \newline
23. सं ॅयज॑त्रै॒र् यज॑त्रैः॒ सꣳ सं ॅयज॑त्रै॒ रङ्गा॒ न्यङ्गा॑नि॒ यज॑त्रैः॒ सꣳ सं ॅयज॑त्रै॒ रङ्गा॑नि । \newline
24. यज॑त्रै॒ रङ्गा॒ न्यङ्गा॑नि॒ यज॑त्रै॒र् यज॑त्रै॒ रङ्गा॑नि॒ सꣳ स मङ्गा॑नि॒ यज॑त्रै॒र् यज॑त्रै॒ रङ्गा॑नि॒ सम् । \newline
25. अङ्गा॑नि॒ सꣳ स मङ्गा॒ न्यङ्गा॑नि॒ सं ॅय॒ज्ञ्प॑तिर् य॒ज्ञ्प॑तिः॒ स मङ्गा॒ न्यङ्गा॑नि॒ सं ॅय॒ज्ञ्प॑तिः । \newline
26. सं ॅय॒ज्ञ्प॑तिर् य॒ज्ञ्प॑तिः॒ सꣳ सं ॅय॒ज्ञ्प॑ति रा॒शिषा॒ ऽऽशिषा॑ य॒ज्ञ्प॑तिः॒ सꣳ सं ॅय॒ज्ञ्प॑ति रा॒शिषा᳚ । \newline
27. य॒ज्ञ्प॑ति रा॒शिषा॒ ऽऽशिषा॑ य॒ज्ञ्प॑तिर् य॒ज्ञ्प॑ति रा॒शिषेती त्या॒शिषा॑ य॒ज्ञ्प॑तिर् य॒ज्ञ्प॑ति रा॒शिषेति॑ । \newline
28. य॒ज्ञ्प॑ति॒रिति॑ य॒ज्ञ् - प॒तिः॒ । \newline
29. आ॒शिषेती त्या॒शिषा॒ ऽऽशिषे त्या॑हा॒हे त्या॒शिषा॒ ऽऽशिषे त्या॑ह । \newline
30. आ॒शिषेत्या᳚ - शिषा᳚ । \newline
31. इत्या॑हा॒हे तीत्या॑ह य॒ज्ञ्प॑तिं ॅय॒ज्ञ्प॑ति मा॒हे तीत्या॑ह य॒ज्ञ्प॑तिम् । \newline
32. आ॒ह॒ य॒ज्ञ्प॑तिं ॅय॒ज्ञ्प॑ति माहाह य॒ज्ञ्प॑ति मे॒वैव य॒ज्ञ्प॑ति माहाह य॒ज्ञ्प॑ति मे॒व । \newline
33. य॒ज्ञ्प॑ति मे॒वैव य॒ज्ञ्प॑तिं ॅय॒ज्ञ्प॑ति मे॒वास्या᳚ स्यै॒व य॒ज्ञ्प॑तिं ॅय॒ज्ञ्प॑ति मे॒वास्य॑ । \newline
34. य॒ज्ञ्प॑ति॒मिति॑ य॒ज्ञ् - प॒ति॒म् । \newline
35. ए॒वास्या᳚ स्यै॒वै वास्या॒ शिष॑ मा॒शिष॑ मस्यै॒ वैवा स्या॒शिष᳚म् । \newline
36. अ॒स्या॒शिष॑ मा॒शिष॑ मस्या स्या॒शिष॑म् गमयति गमय त्या॒शिष॑ मस्या स्या॒शिष॑म् गमयति । \newline
37. आ॒शिष॑म् गमयति गमय त्या॒शिष॑ मा॒शिष॑म् गमयति वि॒श्वरू॑पो वि॒श्वरू॑पो गमय त्या॒शिष॑ मा॒शिष॑म् गमयति वि॒श्वरू॑पः । \newline
38. आ॒शिष॒मित्या᳚ - शिष᳚म् । \newline
39. ग॒म॒य॒ति॒ वि॒श्वरू॑पो वि॒श्वरू॑पो गमयति गमयति वि॒श्वरू॑पो॒ वै वै वि॒श्वरू॑पो गमयति गमयति वि॒श्वरू॑पो॒ वै । \newline
40. वि॒श्वरू॑पो॒ वै वै वि॒श्वरू॑पो वि॒श्वरू॑पो॒ वै त्वा॒ष्ट्र स्त्वा॒ष्ट्रो वै वि॒श्वरू॑पो वि॒श्वरू॑पो॒ वै त्वा॒ष्ट्रः । \newline
41. वि॒श्वरू॑प॒ इति॑ वि॒श्व - रू॒पः॒ । \newline
42. वै त्वा॒ष्ट्र स्त्वा॒ष्ट्रो वै वै त्वा॒ष्ट्र उ॒परि॑ष्टा दु॒परि॑ष्टात् त्वा॒ष्ट्रो वै वै त्वा॒ष्ट्र उ॒परि॑ष्टात् । \newline
43. त्वा॒ष्ट्र उ॒परि॑ष्टा दु॒परि॑ष्टात् त्वा॒ष्ट्र स्त्वा॒ष्ट्र उ॒परि॑ष्टात् प॒शुम् प॒शु मु॒परि॑ष्टात् त्वा॒ष्ट्र स्त्वा॒ष्ट्र उ॒परि॑ष्टात् प॒शुम् । \newline
44. उ॒परि॑ष्टात् प॒शुम् प॒शु मु॒परि॑ष्टा दु॒परि॑ष्टात् प॒शु म॒भ्य॑भि प॒शु मु॒परि॑ष्टा दु॒परि॑ष्टात् प॒शु म॒भि । \newline
45. प॒शु म॒भ्य॑भि प॒शुम् प॒शु म॒भ्य॑ वमी दवमी द॒भि प॒शुम् प॒शु म॒भ्य॑वमीत् । \newline
46. अ॒भ्य॑ वमी दवमी द॒भ्या᳚(1॒)भ्य॑वमी॒त् तस्मा॒त् तस्मा॑ दवमी द॒भ्या᳚(1॒)भ्य॑वमी॒त् तस्मा᳚त् । \newline
47. अ॒व॒मी॒त् तस्मा॒त् तस्मा॑ दवमी दवमी॒त् तस्मा॑ दु॒परि॑ष्टा दु॒परि॑ष्टा॒त् तस्मा॑ दवमी दवमी॒त् तस्मा॑ दु॒परि॑ष्टात् । \newline
48. तस्मा॑ दु॒परि॑ष्टा दु॒परि॑ष्टा॒त् तस्मा॒त् तस्मा॑ दु॒परि॑ष्टात् प॒शोः प॒शो रु॒परि॑ष्टा॒त् तस्मा॒त् तस्मा॑ दु॒परि॑ष्टात् प॒शोः । \newline
49. उ॒परि॑ष्टात् प॒शोः प॒शो रु॒परि॑ष्टा दु॒परि॑ष्टात् प॒शोर् न न प॒शो रु॒परि॑ष्टा दु॒परि॑ष्टात् प॒शोर् न । \newline
50. प॒शोर् न न प॒शोः प॒शोर् नावाव॒ न प॒शोः प॒शोर् नाव॑ । \newline
51. नावाव॒ न नाव॑ द्यन्ति द्य॒न्त्यव॒ न नाव॑ द्यन्ति । \newline
52. अव॑ द्यन्ति द्य॒न्त्यवाव॑ द्यन्ति॒ यद् यद् द्य॒न्त्यवाव॑ द्यन्ति॒ यत् । \newline
53. द्य॒न्ति॒ यद् यद् द्य॑न्ति द्यन्ति॒ यदु॒परि॑ष्टा दु॒परि॑ष्टा॒द् यद् द्य॑न्ति द्यन्ति॒ यदु॒परि॑ष्टात् । \newline
54. यदु॒परि॑ष्टा दु॒परि॑ष्टा॒द् यद् यदु॒परि॑ष्टात् प॒शुम् प॒शु मु॒परि॑ष्टा॒द् यद् यदु॒परि॑ष्टात् प॒शुम् । \newline
55. उ॒परि॑ष्टात् प॒शुम् प॒शु मु॒परि॑ष्टा दु॒परि॑ष्टात् प॒शुꣳ स॑म॒नक्ति॑ सम॒नक्ति॑ प॒शु मु॒परि॑ष्टा दु॒परि॑ष्टात् प॒शुꣳ स॑म॒नक्ति॑ । \newline
56. प॒शुꣳ स॑म॒नक्ति॑ सम॒नक्ति॑ प॒शुम् प॒शुꣳ स॑म॒नक्ति॒ मेद्ध्य॒म् मेद्ध्यꣳ॑ सम॒नक्ति॑ प॒शुम् प॒शुꣳ स॑म॒नक्ति॒ मेद्ध्य᳚म् । \newline
57. स॒म॒नक्ति॒ मेद्ध्य॒म् मेद्ध्यꣳ॑ सम॒नक्ति॑ सम॒नक्ति॒ मेद्ध्य॑ मे॒वैव मेद्ध्यꣳ॑ सम॒नक्ति॑ सम॒नक्ति॒ मेद्ध्य॑ मे॒व । \newline
58. स॒म॒नक्तीति॑ सं - अ॒नक्ति॑ । \newline
59. मेद्ध्य॑ मे॒वैव मेद्ध्य॒म् मेद्ध्य॑ मे॒वैन॑ मेन मे॒व मेद्ध्य॒म् मेद्ध्य॑ मे॒वैन᳚म् । \newline
60. ए॒वैन॑ मेन मे॒वै वैन॑म् करोति करो त्येन मे॒वै वैन॑म् करोति । \newline
\pagebreak
\markright{ TS 6.3.7.5  \hfill https://www.vedavms.in \hfill}

\section{ TS 6.3.7.5 }

\textbf{TS 6.3.7.5 } \newline
\textbf{Samhita Paata} \newline

-नं॑ करोत्यृ॒त्विजो॑ वृणीते॒ छन्दाꣳ॑स्ये॒व वृ॑णीते स॒प्त वृ॑णीते स॒प्त ग्रा॒म्याः प॒शवः॑ स॒प्ताऽऽ*र॒ण्याः स॒प्त छन्दाꣳ॑स्यु॒भय॒स्या व॑रुद्ध्या॒ एका॑दश प्रया॒जान्. य॑जति॒ दश॒ वै प॒शोः प्रा॒णा आ॒त्मैका॑द॒शो यावा॑ने॒व प॒शुस्तं प्र य॑जति व॒पामेकः॒ परि॑ शय आ॒त्मैवाऽऽ*त्मानं॒ परि॑ शये॒ वज्रो॒ वै स्वधि॑ति॒र्वज्रो॑ यूपशक॒लो घृ॒तं खलु॒ वै ( )  दे॒वा वज्रं॑ कृ॒त्वा सोम॑मघ्नन् घृ॒तेना॒क्तौ प॒शुं त्रा॑येथा॒मित्या॑ह॒ वज्रे॑णै॒वैनं॒ ॅवशे॑ कृ॒त्वाऽऽल॑भते ॥ \newline

\textbf{Pada Paata} \newline

ए॒न॒म् । क॒रो॒ति॒ । ऋ॒त्विजः॑ । वृ॒णी॒ते॒ । छन्दाꣳ॑सि । ए॒व । वृ॒णी॒ते॒ । स॒प्त । वृ॒णी॒ते॒ । स॒प्त । ग्रा॒म्याः । प॒शवः॑ । स॒प्त । आ॒र॒ण्याः । स॒प्त । छन्दाꣳ॑सि । उ॒भय॑स्य । अव॑रुद्ध्या॒ इत्यव॑ - रु॒द्ध्यै॒ । एका॑दश । प्र॒या॒जानिति॑ प्र - या॒जान् । य॒ज॒ति॒ । दश॑ । वै । प॒शोः । प्रा॒णा इति॑ प्र-अ॒नाः । आ॒त्मा । ए॒का॒द॒शः । यावान्॑ । ए॒व । प॒शुः । तम् । प्रेति॑ । य॒ज॒ति॒ । व॒पाम् । एकः॑ । परीति॑ । श॒ये॒ । आ॒त्मा । ए॒व । आ॒त्मान᳚म् । परीति॑ । श॒ये॒ । वज्रः॑ । वै । स्वधि॑ति॒रिति॒ स्व - धि॒तिः॒ । वज्रः॑ । यू॒प॒श॒क॒ल इति॑ यूप - श॒क॒लः । घृ॒तम् । खलु॑ । वै ( ) । दे॒वाः । वज्र᳚म् । कृ॒त्वा । सोम᳚म् । अ॒घ्न॒न्न् । घृ॒तेन॑ । अ॒क्तौ । प॒शुम् । त्रा॒ये॒था॒म् । इति॑ । आ॒ह॒ । वज्रे॑ण । ए॒व । ए॒न॒म् । वशे᳚ । कृ॒त्वा । एति॑ । ल॒भ॒ते॒ ॥  \newline


\textbf{Krama Paata} \newline

ए॒न॒म् क॒रो॒ति॒ । क॒रो॒त्यृ॒त्विजः॑ । ऋ॒त्विजो॑ वृणीते । वृ॒णी॒ते॒ छन्दाꣳ॑सि । छन्दाꣳ॑स्ये॒व । ए॒व वृ॑णीते । वृ॒णी॒ते॒ स॒प्त । स॒प्त वृ॑णीते । वृ॒णी॒ते॒ स॒प्त । स॒प्त ग्रा॒म्याः । ग्रा॒म्याः प॒शवः॑ । प॒शवः॑ स॒प्त । स॒प्तार॒ण्याः । आ॒र॒ण्याः स॒प्त । स॒प्तछन्दाꣳ॑सि । छन्दाꣳ॑स्यु॒भय॑स्य । उ॒भय॒स्याव॑रुद्ध्यै । अव॑रुद्ध्या॒ एका॑दश । अव॑रुद्ध्या॒ इत्यव॑ - रु॒द्ध्यै॒ । एका॑दश प्रया॒जान् । प्र॒या॒जान्. य॑जति । प्र॒या॒जानिति॑ प्र - या॒जान् । य॒ज॒ति॒ दश॑ । दश॒ वै । वै प॒शोः । प॒शोः प्रा॒णाः । प्रा॒णा आ॒त्मा । प्रा॒णा इति॑ प्र - अ॒नाः । आ॒त्मैका॑द॒शः । ए॒का॒द॒शो यावान्॑ । यावा॑ने॒व । ए॒व प॒शुः । प॒शुस्तम् । तम् प्र । प्र य॑जति । य॒ज॒ति॒ व॒पाम् । व॒पामेकः॑ । एकः॒ परि॑ । परि॑ शये । श॒य॒ आ॒त्मा । आ॒त्मैव । ए॒वात्मान᳚म् । आ॒त्मान॒म् परि॑ । परि॑ शये । श॒ये॒ वज्रः॑ । वज्रो॒ वै । वै स्वधि॑तिः । स्वधि॑ति॒र् वज्रः॑ । स्वधि॑ति॒रिति॒ स्व - धि॒तिः॒ । वज्रो॑ यूपशक॒लः । यू॒प॒श॒क॒लो घृ॒तम् । यू॒प॒श॒क॒ल इति॑ यूप - श॒क॒लः । घृ॒तम् खलु॑ । खलु॒ वै ( ) । वै दे॒वाः । दे॒वा वज्र᳚म् । वज्र॑म् कृ॒त्वा । कृ॒त्वा सोम᳚म् । सोम॑मघ्नन्न् । अ॒घ्न॒न् घृ॒तेन॑ । घृ॒तेना॒क्तौ । अ॒क्तौ प॒शुम् । प॒शुम् त्रा॑येथाम् । त्रा॒ये॒था॒मिति॑ । इत्या॑ह । आ॒ह॒ वज्रे॑ण । वज्रे॑णै॒व । ए॒वैन᳚म् । ए॒न॒म् ॅवशे᳚ । वशे॑ कृ॒त्वा । कृ॒त्वा ऽऽ ल॑भते । आ ल॑भते । ल॒भ॒त॒ इति॑ लभते । \newline

\textbf{Jatai Paata} \newline

1. ए॒न॒म् क॒रो॒ति॒ क॒रो॒ त्ये॒न॒ मे॒न॒म् क॒रो॒ति॒ । \newline
2. क॒रो॒ त्यृ॒त्विज॑ ऋ॒त्विजः॑ करोति करो त्यृ॒त्विजः॑ । \newline
3. ऋ॒त्विजो॑ वृणीते वृणीत ऋ॒त्विज॑ ऋ॒त्विजो॑ वृणीते । \newline
4. वृ॒णी॒ते॒ छन्दाꣳ॑सि॒ छन्दाꣳ॑सि वृणीते वृणीते॒ छन्दाꣳ॑सि । \newline
5. छन्दाꣳ॑ स्ये॒वैव छन्दाꣳ॑सि॒ छन्दाꣳ॑ स्ये॒व । \newline
6. ए॒व वृ॑णीते वृणीत ए॒वैव वृ॑णीते । \newline
7. वृ॒णी॒ते॒ स॒प्त स॒प्त वृ॑णीते वृणीते स॒प्त । \newline
8. स॒प्त वृ॑णीते वृणीते स॒प्त स॒प्त वृ॑णीते । \newline
9. वृ॒णी॒ते॒ स॒प्त स॒प्त वृ॑णीते वृणीते स॒प्त । \newline
10. स॒प्त ग्रा॒म्या ग्रा॒म्याः स॒प्त स॒प्त ग्रा॒म्याः । \newline
11. ग्रा॒म्याः प॒शवः॑ प॒शवो᳚ ग्रा॒म्या ग्रा॒म्याः प॒शवः॑ । \newline
12. प॒शवः॑ स॒प्त स॒प्त प॒शवः॑ प॒शवः॑ स॒प्त । \newline
13. स॒प्तार॒ण्या आ॑र॒ण्याः स॒प्त स॒प्तार॒ण्याः । \newline
14. आ॒र॒ण्याः स॒प्त स॒प्तार॒ण्या आ॑र॒ण्याः स॒प्त । \newline
15. स॒प्त छन्दाꣳ॑सि॒ छन्दाꣳ॑सि स॒प्त स॒प्त छन्दाꣳ॑सि । \newline
16. छन्दाꣳ॑ स्यु॒भय॑ स्यो॒भय॑स्य॒ छन्दाꣳ॑सि॒ छन्दाꣳ॑ स्यु॒भय॑स्य । \newline
17. उ॒भय॒स्या व॑रुद्ध्या॒ अव॑रुद्ध्या उ॒भय॑ स्यो॒भय॒स्या व॑रुद्ध्यै । \newline
18. अव॑रुद्ध्या॒ एका॑द॒ शैका॑द॒शा व॑रुद्ध्या॒ अव॑रुद्ध्या॒ एका॑दश । \newline
19. अव॑रुद्ध्या॒ इत्यव॑ - रु॒द्ध्यै॒ । \newline
20. एका॑दश प्रया॒जान् प्र॑या॒जा नेका॑द॒ शैका॑दश प्रया॒जान् । \newline
21. प्र॒या॒जान्. य॑जति यजति प्रया॒जान् प्र॑या॒जान्. य॑जति । \newline
22. प्र॒या॒जानिति॑ प्र - या॒जान् । \newline
23. य॒ज॒ति॒ दश॒ दश॑ यजति यजति॒ दश॑ । \newline
24. दश॒ वै वै दश॒ दश॒ वै । \newline
25. वै प॒शोः प॒शोर् वै वै प॒शोः । \newline
26. प॒शोः प्रा॒णाः प्रा॒णाः प॒शोः प॒शोः प्रा॒णाः । \newline
27. प्रा॒णा आ॒त्मा ऽऽत्मा प्रा॒णाः प्रा॒णा आ॒त्मा । \newline
28. प्रा॒णा इति॑ प्र - अ॒नाः । \newline
29. आ॒त्मैका॑द॒श ए॑काद॒श आ॒त्मा ऽऽत्मैका॑द॒शः । \newline
30. ए॒का॒द॒शो यावा॒न्॒. यावा॑ नेकाद॒श ए॑काद॒शो यावान्॑ । \newline
31. यावा॑ने॒ वैव यावा॒न्॒. यावा॑ने॒व । \newline
32. ए॒व प॒शुः प॒शु रे॒वैव प॒शुः । \newline
33. प॒शु स्तम् तम् प॒शुः प॒शु स्तम् । \newline
34. तम् प्र प्र तम् तम् प्र । \newline
35. प्र य॑जति यजति॒ प्र प्र य॑जति । \newline
36. य॒ज॒ति॒ व॒पां ॅव॒पां ॅय॑जति यजति व॒पाम् । \newline
37. व॒पा मेक॒ एको॑ व॒पां ॅव॒पा मेकः॑ । \newline
38. एकः॒ परि॒ पर्येक॒ एकः॒ परि॑ । \newline
39. परि॑ शये शये॒ परि॒ परि॑ शये । \newline
40. श॒य॒ आ॒त्मा ऽऽत्मा श॑ये शय आ॒त्मा । \newline
41. आ॒त्मैवै वात्मा ऽऽत्मैव । \newline
42. ए॒वात्मान॑ मा॒त्मान॑ मे॒वै वात्मान᳚म् । \newline
43. आ॒त्मान॒म् परि॒ पर्या॒त्मान॑ मा॒त्मान॒म् परि॑ । \newline
44. परि॑ शये शये॒ परि॒ परि॑ शये । \newline
45. श॒ये॒ वज्रो॒ वज्रः॑ शये शये॒ वज्रः॑ । \newline
46. वज्रो॒ वै वै वज्रो॒ वज्रो॒ वै । \newline
47. वै स्वधि॑तिः॒ स्वधि॑ति॒र् वै वै स्वधि॑तिः । \newline
48. स्वधि॑ति॒र् वज्रो॒ वज्रः॒ स्वधि॑तिः॒ स्वधि॑ति॒र् वज्रः॑ । \newline
49. स्वधि॑ति॒रिति॒ स्व - धि॒तिः॒ । \newline
50. वज्रो॑ यूपशक॒लो यू॑पशक॒लो वज्रो॒ वज्रो॑ यूपशक॒लः । \newline
51. यू॒प॒श॒क॒लो घृ॒तम् घृ॒तं ॅयू॑पशक॒लो यू॑पशक॒लो घृ॒तम् । \newline
52. यू॒प॒श॒क॒ल इति॑ यूप - श॒क॒लः । \newline
53. घृ॒तम् खलु॒ खलु॑ घृ॒तम् घृ॒तम् खलु॑ । \newline
54. खलु॒ वै वै खलु॒ खलु॒ वै । \newline
55. वै दे॒वा दे॒वा वै वै दे॒वाः । \newline
56. दे॒वा वज्रं॒ ॅवज्र॑म् दे॒वा दे॒वा वज्र᳚म् । \newline
57. वज्र॑म् कृ॒त्वा कृ॒त्वा वज्रं॒ ॅवज्र॑म् कृ॒त्वा । \newline
58. कृ॒त्वा सोमꣳ॒॒ सोम॑म् कृ॒त्वा कृ॒त्वा सोम᳚म् । \newline
59. सोम॑ मघ्नन् नघ्न॒न् थ्सोमꣳ॒॒ सोम॑ मघ्नन्न् । \newline
60. अ॒घ्न॒न् घृ॒तेन॑ घृ॒तेना᳚ घ्नन् नघ्नन् घृ॒तेन॑ । \newline
61. घृ॒ते ना॒क्ता व॒क्तौ घृ॒तेन॑ घृ॒ते ना॒क्तौ । \newline
62. अ॒क्तौ प॒शुम् प॒शु म॒क्ता व॒क्तौ प॒शुम् । \newline
63. प॒शुम् त्रा॑येथाम् त्रायेथाम् प॒शुम् प॒शुम् त्रा॑येथाम् । \newline
64. त्रा॒ये॒था॒ मितीति॑ त्रायेथाम् त्रायेथा॒ मिति॑ । \newline
65. इत्या॑हा॒हे तीत्या॑ह । \newline
66. आ॒ह॒ वज्रे॑ण॒ वज्रे॑णाहाह॒ वज्रे॑ण । \newline
67. वज्रे॑णै॒ वैव वज्रे॑ण॒ वज्रे॑णै॒व । \newline
68. ए॒वैन॑ मेन मे॒वै वैन᳚म् । \newline
69. ए॒नं॒ ॅवशे॒ वश॑ एन मेनं॒ ॅवशे᳚ । \newline
70. वशे॑ कृ॒त्वा कृ॒त्वा वशे॒ वशे॑ कृ॒त्वा । \newline
71. कृ॒त्वा ऽऽल॑भते लभत॒ आ कृ॒त्वा कृ॒त्वा ऽऽल॑भते । \newline
72. आ ल॑भते लभत॒ आ ल॑भते । \newline
73. ल॒भ॒त॒ इति॑ लभते । \newline

\textbf{Ghana Paata } \newline

1. ए॒न॒म् क॒रो॒ति॒ क॒रो॒ त्ये॒न॒ मे॒न॒म् क॒रो॒ त्यृ॒त्विज॑ ऋ॒त्विजः॑ करो त्येन मेनम् करो त्यृ॒त्विजः॑ । \newline
2. क॒रो॒ त्यृ॒त्विज॑ ऋ॒त्विजः॑ करोति करो त्यृ॒त्विजो॑ वृणीते वृणीत ऋ॒त्विजः॑ करोति करो त्यृ॒त्विजो॑ वृणीते । \newline
3. ऋ॒त्विजो॑ वृणीते वृणीत ऋ॒त्विज॑ ऋ॒त्विजो॑ वृणीते॒ छन्दाꣳ॑सि॒ छन्दाꣳ॑सि वृणीत ऋ॒त्विज॑ ऋ॒त्विजो॑ वृणीते॒ छन्दाꣳ॑सि । \newline
4. वृ॒णी॒ते॒ छन्दाꣳ॑सि॒ छन्दाꣳ॑सि वृणीते वृणीते॒ छन्दाꣳ॑ स्ये॒वैव छन्दाꣳ॑सि वृणीते वृणीते॒ छन्दाꣳ॑स्ये॒व । \newline
5. छन्दाꣳ॑स्ये॒ वैव छन्दाꣳ॑सि॒ छन्दाꣳ॑स्ये॒व वृ॑णीते वृणीत ए॒व छन्दाꣳ॑सि॒ छन्दाꣳ॑स्ये॒व वृ॑णीते । \newline
6. ए॒व वृ॑णीते वृणीत ए॒वैव वृ॑णीते स॒प्त स॒प्त वृ॑णीत ए॒वैव वृ॑णीते स॒प्त । \newline
7. वृ॒णी॒ते॒ स॒प्त स॒प्त वृ॑णीते वृणीते स॒प्त वृ॑णीते वृणीते स॒प्त वृ॑णीते वृणीते स॒प्त वृ॑णीते । \newline
8. स॒प्त वृ॑णीते वृणीते स॒प्त स॒प्त वृ॑णीते स॒प्त स॒प्त वृ॑णीते स॒प्त स॒प्त वृ॑णीते स॒प्त । \newline
9. वृ॒णी॒ते॒ स॒प्त स॒प्त वृ॑णीते वृणीते स॒प्त ग्रा॒म्या ग्रा॒म्याः स॒प्त वृ॑णीते वृणीते स॒प्त ग्रा॒म्याः । \newline
10. स॒प्त ग्रा॒म्या ग्रा॒म्याः स॒प्त स॒प्त ग्रा॒म्याः प॒शवः॑ प॒शवो᳚ ग्रा॒म्याः स॒प्त स॒प्त ग्रा॒म्याः प॒शवः॑ । \newline
11. ग्रा॒म्याः प॒शवः॑ प॒शवो᳚ ग्रा॒म्या ग्रा॒म्याः प॒शवः॑ स॒प्त स॒प्त प॒शवो᳚ ग्रा॒म्या ग्रा॒म्याः प॒शवः॑ स॒प्त । \newline
12. प॒शवः॑ स॒प्त स॒प्त प॒शवः॑ प॒शवः॑ स॒प्तार॒ण्या आ॑र॒ण्याः स॒प्त प॒शवः॑ प॒शवः॑ स॒प्तार॒ण्याः । \newline
13. स॒प्तार॒ण्या आ॑र॒ण्याः स॒प्त स॒प्तार॒ण्याः स॒प्त स॒प्तार॒ण्याः स॒प्त स॒प्तार॒ण्याः स॒प्त । \newline
14. आ॒र॒ण्याः स॒प्त स॒प्तार॒ण्या आ॑र॒ण्याः स॒प्त छन्दाꣳ॑सि॒ छन्दाꣳ॑सि स॒प्तार॒ण्या आ॑र॒ण्याः स॒प्त छन्दाꣳ॑सि । \newline
15. स॒प्त छन्दाꣳ॑सि॒ छन्दाꣳ॑सि स॒प्त स॒प्त छन्दाꣳ॑ स्यु॒भय॑ स्यो॒भय॑स्य॒ छन्दाꣳ॑सि स॒प्त स॒प्त छन्दाꣳ॑ स्यु॒भय॑स्य । \newline
16. छन्दाꣳ॑ स्यु॒भय॑ स्यो॒भय॑स्य॒ छन्दाꣳ॑सि॒ छन्दाꣳ॑ स्यु॒भय॒स्या व॑रुद्ध्या॒ अव॑रुद्ध्या उ॒भय॑स्य॒ छन्दाꣳ॑सि॒ छन्दाꣳ॑ स्यु॒भय॒ स्याव॑रुद्ध्यै । \newline
17. उ॒भय॒स्या व॑रुद्ध्या॒ अव॑रुद्ध्या उ॒भय॑ स्यो॒भय॒ स्याव॑रुद्ध्या॒ एका॑द॒ शैका॑द॒शा व॑रुद्ध्या उ॒भय॑ स्यो॒भय॒ स्याव॑रुद्ध्या॒ एका॑दश । \newline
18. अव॑रुद्ध्या॒ एका॑द॒ शैका॑द॒शा व॑रुद्ध्या॒ अव॑रुद्ध्या॒ एका॑दश प्रया॒जान् प्र॑या॒जा नेका॑द॒शा व॑रुद्ध्या॒ अव॑रुद्ध्या॒ एका॑दश प्रया॒जान् । \newline
19. अव॑रुद्ध्या॒ इत्यव॑ - रु॒द्ध्यै॒ । \newline
20. एका॑दश प्रया॒जान् प्र॑या॒जा नेका॑द॒ शैका॑दश प्रया॒जान्. य॑जति यजति प्रया॒जा नेका॑द॒ शैका॑दश प्रया॒जान्. य॑जति । \newline
21. प्र॒या॒जान्. य॑जति यजति प्रया॒जान् प्र॑या॒जान्. य॑जति॒ दश॒ दश॑ यजति प्रया॒जान् प्र॑या॒जान्. य॑जति॒ दश॑ । \newline
22. प्र॒या॒जानिति॑ प्र - या॒जान् । \newline
23. य॒ज॒ति॒ दश॒ दश॑ यजति यजति॒ दश॒ वै वै दश॑ यजति यजति॒ दश॒ वै । \newline
24. दश॒ वै वै दश॒ दश॒ वै प॒शोः प॒शोर् वै दश॒ दश॒ वै प॒शोः । \newline
25. वै प॒शोः प॒शोर् वै वै प॒शोः प्रा॒णाः प्रा॒णाः प॒शोर् वै वै प॒शोः प्रा॒णाः । \newline
26. प॒शोः प्रा॒णाः प्रा॒णाः प॒शोः प॒शोः प्रा॒णा आ॒त्मा ऽऽत्मा प्रा॒णाः प॒शोः प॒शोः प्रा॒णा आ॒त्मा । \newline
27. प्रा॒णा आ॒त्मा ऽऽत्मा प्रा॒णाः प्रा॒णा आ॒त्मैका॑द॒श ए॑काद॒श आ॒त्मा प्रा॒णाः प्रा॒णा आ॒त्मैका॑द॒शः । \newline
28. प्रा॒णा इति॑ प्र - अ॒नाः । \newline
29. आ॒त्मैका॑द॒श ए॑काद॒श आ॒त्मा ऽऽत्मैका॑द॒शो यावा॒न्॒. यावा॑नेकाद॒श आ॒त्मा ऽऽत्मैका॑द॒शो यावान्॑ । \newline
30. ए॒का॒द॒शो यावा॒न्॒. यावा॑नेकाद॒श ए॑काद॒शो यावा॑ने॒ वैव यावा॑नेकाद॒श ए॑काद॒शो यावा॑ने॒व । \newline
31. यावा॑ने॒ वैव यावा॒न्॒. यावा॑ने॒व प॒शुः प॒शु रे॒व यावा॒न्॒. यावा॑ने॒व प॒शुः । \newline
32. ए॒व प॒शुः प॒शु रे॒वैव प॒शु स्तम् तम् प॒शु रे॒वैव प॒शु स्तम् । \newline
33. प॒शु स्तम् तम् प॒शुः प॒शु स्तम् प्र प्र तम् प॒शुः प॒शु स्तम् प्र । \newline
34. तम् प्र प्र तम् तम् प्र य॑जति यजति॒ प्र तम् तम् प्र य॑जति । \newline
35. प्र य॑जति यजति॒ प्र प्र य॑जति व॒पां ॅव॒पां ॅय॑जति॒ प्र प्र य॑जति व॒पाम् । \newline
36. य॒ज॒ति॒ व॒पां ॅव॒पां ॅय॑जति यजति व॒पा मेक॒ एको॑ व॒पां ॅय॑जति यजति व॒पा मेकः॑ । \newline
37. व॒पा मेक॒ एको॑ व॒पां ॅव॒पा मेकः॒ परि॒ पर्येको॑ व॒पां ॅव॒पा मेकः॒ परि॑ । \newline
38. एकः॒ परि॒ पर्येक॒ एकः॒ परि॑ शये शये॒ पर्येक॒ एकः॒ परि॑ शये । \newline
39. परि॑ शये शये॒ परि॒ परि॑ शय आ॒त्मा ऽऽत्मा श॑ये॒ परि॒ परि॑ शय आ॒त्मा । \newline
40. श॒य॒ आ॒त्मा ऽऽत्मा श॑ये शय आ॒त्मैवै वात्मा श॑ये शय आ॒त्मैव । \newline
41. आ॒त्मैवै वात्मा ऽऽत्मै वात्मान॑ मा॒त्मान॑ मे॒वात्मा ऽऽत्मै वात्मान᳚म् । \newline
42. ए॒वात्मान॑ मा॒त्मान॑ मे॒वै वात्मान॒म् परि॒ पर्या॒त्मान॑ मे॒वै वात्मान॒म् परि॑ । \newline
43. आ॒त्मान॒म् परि॒ पर्या॒त्मान॑ मा॒त्मान॒म् परि॑ शये शये॒ पर्या॒त्मान॑ मा॒त्मान॒म् परि॑ शये । \newline
44. परि॑ शये शये॒ परि॒ परि॑ शये॒ वज्रो॒ वज्रः॑ शये॒ परि॒ परि॑ शये॒ वज्रः॑ । \newline
45. श॒ये॒ वज्रो॒ वज्रः॑ शये शये॒ वज्रो॒ वै वै वज्रः॑ शये शये॒ वज्रो॒ वै । \newline
46. वज्रो॒ वै वै वज्रो॒ वज्रो॒ वै स्वधि॑तिः॒ स्वधि॑ति॒र् वै वज्रो॒ वज्रो॒ वै स्वधि॑तिः । \newline
47. वै स्वधि॑तिः॒ स्वधि॑ति॒र् वै वै स्वधि॑ति॒र् वज्रो॒ वज्रः॒ स्वधि॑ति॒र् वै वै स्वधि॑ति॒र् वज्रः॑ । \newline
48. स्वधि॑ति॒र् वज्रो॒ वज्रः॒ स्वधि॑तिः॒ स्वधि॑ति॒र् वज्रो॑ यूपशक॒लो यू॑पशक॒लो वज्रः॒ स्वधि॑तिः॒ स्वधि॑ति॒र् वज्रो॑ यूपशक॒लः । \newline
49. स्वधि॑ति॒रिति॒ स्व - धि॒तिः॒ । \newline
50. वज्रो॑ यूपशक॒लो यू॑पशक॒लो वज्रो॒ वज्रो॑ यूपशक॒लो घृ॒तम् घृ॒तं ॅयू॑पशक॒लो वज्रो॒ वज्रो॑ यूपशक॒लो घृ॒तम् । \newline
51. यू॒प॒श॒क॒लो घृ॒तम् घृ॒तं ॅयू॑पशक॒लो यू॑पशक॒लो घृ॒तम् खलु॒ खलु॑ घृ॒तं ॅयू॑पशक॒लो यू॑पशक॒लो घृ॒तम् खलु॑ । \newline
52. यू॒प॒श॒क॒ल इति॑ यूप - श॒क॒लः । \newline
53. घृ॒तम् खलु॒ खलु॑ घृ॒तम् घृ॒तम् खलु॒ वै वै खलु॑ घृ॒तम् घृ॒तम् खलु॒ वै । \newline
54. खलु॒ वै वै खलु॒ खलु॒ वै दे॒वा दे॒वा वै खलु॒ खलु॒ वै दे॒वाः । \newline
55. वै दे॒वा दे॒वा वै वै दे॒वा वज्रं॒ ॅवज्र॑म् दे॒वा वै वै दे॒वा वज्र᳚म् । \newline
56. दे॒वा वज्रं॒ ॅवज्र॑म् दे॒वा दे॒वा वज्र॑म् कृ॒त्वा कृ॒त्वा वज्र॑म् दे॒वा दे॒वा वज्र॑म् कृ॒त्वा । \newline
57. वज्र॑म् कृ॒त्वा कृ॒त्वा वज्रं॒ ॅवज्र॑म् कृ॒त्वा सोमꣳ॒॒ सोम॑म् कृ॒त्वा वज्रं॒ ॅवज्र॑म् कृ॒त्वा सोम᳚म् । \newline
58. कृ॒त्वा सोमꣳ॒॒ सोम॑म् कृ॒त्वा कृ॒त्वा सोम॑ मघ्नन् नघ्न॒न् थ्सोम॑म् कृ॒त्वा कृ॒त्वा सोम॑ मघ्नन्न् । \newline
59. सोम॑ मघ्नन् नघ्न॒न् थ्सोमꣳ॒॒ सोम॑ मघ्नन् घृ॒तेन॑ घृ॒तेना᳚ घ्न॒न् थ्सोमꣳ॒॒ सोम॑ मघ्नन् घृ॒तेन॑ । \newline
60. अ॒घ्न॒न् घृ॒तेन॑ घृ॒तेना᳚ घ्नन् नघ्नन् घृ॒ते ना॒क्ता व॒क्तौ घृ॒तेना᳚ घ्नन् नघ्नन् घृ॒ते ना॒क्तौ । \newline
61. घृ॒ते ना॒क्ता व॒क्तौ घृ॒तेन॑ घृ॒ते ना॒क्तौ प॒शुम् प॒शु म॒क्तौ घृ॒तेन॑ घृ॒ते ना॒क्तौ प॒शुम् । \newline
62. अ॒क्तौ प॒शुम् प॒शु म॒क्ता व॒क्तौ प॒शुम् त्रा॑येथाम् त्रायेथाम् प॒शु म॒क्ता व॒क्तौ प॒शुम् त्रा॑येथाम् । \newline
63. प॒शुम् त्रा॑येथाम् त्रायेथाम् प॒शुम् प॒शुम् त्रा॑येथा॒ मितीति॑ त्रायेथाम् प॒शुम् प॒शुम् त्रा॑येथा॒ मिति॑ । \newline
64. त्रा॒ये॒था॒ मितीति॑ त्रायेथाम् त्रायेथा॒ मित्या॑हा॒हेति॑ त्रायेथाम् त्रायेथा॒ मित्या॑ह । \newline
65. इत्या॑हा॒हे तीत्या॑ह॒ वज्रे॑ण॒ वज्रे॑णा॒हे तीत्या॑ह॒ वज्रे॑ण । \newline
66. आ॒ह॒ वज्रे॑ण॒ वज्रे॑णाहाह॒ वज्रे॑णै॒वैव वज्रे॑णा हाह॒ वज्रे॑णै॒व । \newline
67. वज्रे॑णै॒वैव वज्रे॑ण॒ वज्रे॑णै॒ वैन॑ मेन मे॒व वज्रे॑ण॒ वज्रे॑ णै॒वैन᳚म् । \newline
68. ए॒वैन॑ मेन मे॒वै वैनं॒ ॅवशे॒ वश॑ एन मे॒वै वैनं॒ ॅवशे᳚ । \newline
69. ए॒नं॒ ॅवशे॒ वश॑ एन मेनं॒ ॅवशे॑ कृ॒त्वा कृ॒त्वा वश॑ एन मेनं॒ ॅवशे॑ कृ॒त्वा । \newline
70. वशे॑ कृ॒त्वा कृ॒त्वा वशे॒ वशे॑ कृ॒त्वा ऽऽल॑भते लभत॒ आ कृ॒त्वा वशे॒ वशे॑ कृ॒त्वा ऽऽल॑भते । \newline
71. कृ॒त्वा ऽऽल॑भते लभत॒ आ कृ॒त्वा कृ॒त्वा ऽऽल॑भते । \newline
72. आ ल॑भते लभत॒ आ ल॑भते । \newline
73. ल॒भ॒त॒ इति॑ लभते । \newline
\pagebreak
\markright{ TS 6.3.8.1  \hfill https://www.vedavms.in \hfill}

\section{ TS 6.3.8.1 }

\textbf{TS 6.3.8.1 } \newline
\textbf{Samhita Paata} \newline

पर्य॑ग्नि करोति सर्व॒हुत॑मे॒वैनं॑ करो॒त्य-स्क॑न्दा॒या-स्क॑न्नꣳ॒॒ हि तद् यद् धु॒तस्य॒ स्कन्द॑ति॒ त्रिः पर्य॑ग्नि करोति॒ त्र्या॑वृ॒द्धि य॒ज्ञोऽथो॒ रक्ष॑सा॒मप॑हत्यै ब्रह्मवा॒दिनो॑ वदन्त्यन्वा॒रभ्यः॑ प॒शू(3)र्नान्वा॒रभ्या(3)इति॑ मृ॒त्यवे॒ वा ए॒ष नी॑यते॒ यत् प॒शुस्तं ॅयद॑न्वा॒रभे॑त प्र॒मायु॑को॒ यज॑मानः स्या॒दथो॒ खल्वा॑हुः सुव॒र्गाय॒ वा ए॒ष लो॒काय॑ नीयते॒ यत्- [  ] \newline

\textbf{Pada Paata} \newline

पर्य॒ग्नीति॒ परि॑ - अ॒ग्नि॒ । क॒रो॒ति॒ । स॒र्व॒हुत॒मिति॑ सर्व-हुत᳚म् । ए॒व । ए॒न॒म् । क॒रो॒ति॒ । अस्क॑न्दाय । अस्क॑न्नम् । हि । तत् । यत् । हु॒तस्य॑ । स्कन्द॑ति । त्रिः । पर्य॒ग्नीति॒ परि॑ - अ॒ग्नि॒ । क॒रो॒ति॒ । त्र्या॑वृ॒दिति॒ त्रि - आ॒वृ॒त् । हि । य॒ज्ञ्ः । अथो॒ इति॑ । रक्ष॑साम् । अप॑हत्या॒ इत्यप॑ - ह॒त्यै॒ । ब्र॒ह्म॒वा॒दिन॒ इति॑ ब्रह्म-वा॒दिनः॑ । व॒द॒न्ति॒ । अ॒न्वा॒रभ्य॒ इत्य॑नु - आ॒रभ्यः॑ । प॒शू(3)ः । न । अ॒न्वा॒रभ्या(3) इत्य॑नु -आ॒रभ्या(3)ः । इति॑ । मृ॒त्यवे᳚ । वै । ए॒षः । नी॒य॒ते॒ । यत् ।  प॒शुः । तम् । यत् । अ॒न्वा॒रभे॒तेत्य॑नु-आ॒रभे॑त । प्र॒मायु॑क॒ इति॑ प्र-मायु॑कः । यज॑मानः । स्या॒त् । अथो॒ इति॑ । खलु॑ । आ॒हुः॒ । सु॒व॒र्गायेति॑ सुवः - गाय॑ । वै । ए॒षः । लो॒काय॑ । नी॒य॒ते॒ । यत् ।  \newline


\textbf{Krama Paata} \newline

पर्य॑ग्नि करोति । पर्य॒ग्नीति॒ परि॑ - अ॒ग्नि॒ । क॒रो॒ति॒ स॒र्व॒हुत᳚म् । स॒र्व॒हुत॑मे॒व । स॒र्व॒हुत॒मिति॑ सर्व - हुत᳚म् । ए॒वैन᳚म् । ए॒न॒म् क॒रो॒ति॒ । क॒रो॒त्यस्क॑न्दाय । अस्क॑न्दा॒यास्क॑न्नम् । अस्क॑न्नꣳ॒॒ हि । हि तत् । तद् यत् । यद्‍धु॒तस्य॑ । हु॒तस्य॒ स्कन्द॑ति । स्कन्द॑ति॒ त्रिः । त्रिः पर्य॑ग्नि । पर्य॑ग्नि करोति । पर्य॒ग्नीति॒ परि॑ - अ॒ग्नि॒ । क॒रो॒ति॒ त्र्या॑वृत् । त्र्या॑वृ॒द्‌धि । त्र्या॑वृ॒दिति॒ त्रि - आ॒वृ॒त्॒ । हि य॒ज्ञ्ः । य॒ज्ञोऽथो᳚ । अथो॒ रक्ष॑साम् । अथो॒ इत्यथो᳚ । रक्ष॑सा॒मप॑हत्यै । अप॑हत्यै ब्रह्मवा॒दिनः॑ । अप॑हत्या॒ इत्यप॑ - ह॒त्यै॒ । ब्र॒ह्म॒वा॒दिनो॑ वदन्ति । ब्र॒ह्म॒वा॒दिन॒ इति॑ ब्रह्म - वा॒दिनः॑ । व॒द॒न्त्य॒न्वा॒रभ्यः॑ । अ॒न्वा॒रभ्यः॑ प॒शू(3)ः । अ॒न्वा॒रभ्य॒ इत्य॑नु - आ॒रभ्यः॑ । प॒शू(3)र् न । नान्वा॒रभ्या(3)ः । अ॒न्वा॒रभ्या(3) इति॑ । अ॒न्वा॒रभ्या(3) इत्य॑नु - आ॒रभ्या(3)ः । इति॑ मृ॒त्यवे᳚ । मृ॒त्यवे॒ वै । वा ए॒षः । ए॒ष नी॑यते । नी॒य॒ते॒ यत् । यत् प॒शुः । प॒शुस्तम् । तम् ॅयत् । यद॑न्वा॒रभे॑त । अ॒न्वा॒रभे॑त प्र॒मायु॑कः । अ॒न्वा॒रभे॒तेत्य॑नु - आ॒रभे॑त । प्र॒मायु॑को॒ यज॑मानः । प्र॒मायु॑क॒ इति॑ प्र - मायु॑कः । यज॑मानः स्यात् । स्या॒दथो᳚ । अथो॒ खलु॑ । अथो॒ इत्यथो᳚ । खल्वा॑हुः । आ॒हुः॒ सु॒व॒र्गाय॑ । सु॒व॒र्गाय॒ वै । सु॒व॒र्गायेति॑ सुवः - गाय॑ । वा ए॒षः । ए॒ष लो॒काय॑ । लो॒काय॑ नीयते । नी॒य॒ते॒ यत् । यत् प॒शुः \newline

\textbf{Jatai Paata} \newline

1. पर्य॑ग्नि करोति करोति॒ पर्य॑ग्नि॒ पर्य॑ग्नि करोति । \newline
2. पर्य॒ग्नीति॒ परि॑ - अ॒ग्नि॒ । \newline
3. क॒रो॒ति॒ स॒र्व॒हुतꣳ॑ सर्व॒हुत॑म् करोति करोति सर्व॒हुत᳚म् । \newline
4. स॒र्व॒हुत॑ मे॒वैव स॑र्व॒हुतꣳ॑ सर्व॒हुत॑ मे॒व । \newline
5. स॒र्व॒हुत॒मिति॑ सर्व - हुत᳚म् । \newline
6. ए॒वैन॑ मेन मे॒वै वैन᳚म् । \newline
7. ए॒न॒म् क॒रो॒ति॒ क॒रो॒ त्ये॒न॒ मे॒न॒म् क॒रो॒ति॒ । \newline
8. क॒रो॒ त्यस्क॑न्दा॒या स्क॑न्दाय करोति करो॒ त्यस्क॑न्दाय । \newline
9. अस्क॑न्दा॒या स्क॑न्न॒ मस्क॑न्न॒ मस्क॑न्दा॒या स्क॑न्दा॒या स्क॑न्नम् । \newline
10. अस्क॑न्नꣳ॒॒ हि ह्यस्क॑न्न॒ मस्क॑न्नꣳ॒॒ हि । \newline
11. हि तत् तद्धि हि तत् । \newline
12. तद् यद् यत् तत् तद् यत् । \newline
13. यद्धु॒तस्य॑ हु॒तस्य॒ यद् यद्धु॒तस्य॑ । \newline
14. हु॒तस्य॒ स्कन्द॑ति॒ स्कन्द॑ति हु॒तस्य॑ हु॒तस्य॒ स्कन्द॑ति । \newline
15. स्कन्द॑ति॒ त्रि स्त्रिः स्कन्द॑ति॒ स्कन्द॑ति॒ त्रिः । \newline
16. त्रिः पर्य॑ग्नि॒ पर्य॑ग्नि॒ त्रि स्त्रिः पर्य॑ग्नि । \newline
17. पर्य॑ग्नि करोति करोति॒ पर्य॑ग्नि॒ पर्य॑ग्नि करोति । \newline
18. पर्य॒ग्नीति॒ परि॑ - अ॒ग्नि॒ । \newline
19. क॒रो॒ति॒ त्र्या॑वृ॒त् त्र्या॑वृत् करोति करोति॒ त्र्या॑वृत् । \newline
20. त्र्या॑वृ॒द्धि हि त्र्या॑वृ॒त् त्र्या॑वृ॒द्धि । \newline
21. त्र्या॑वृ॒दिति॒ त्रि - आ॒वृ॒त् । \newline
22. हि य॒ज्ञो य॒ज्ञो हि हि य॒ज्ञ्ः । \newline
23. य॒ज्ञो ऽथो॒ अथो॑ य॒ज्ञो य॒ज्ञो ऽथो᳚ । \newline
24. अथो॒ रक्ष॑साꣳ॒॒ रक्ष॑सा॒ मथो॒ अथो॒ रक्ष॑साम् । \newline
25. अथो॒ इत्यथो᳚ । \newline
26. रक्ष॑सा॒ मप॑हत्या॒ अप॑हत्यै॒ रक्ष॑साꣳ॒॒ रक्ष॑सा॒ मप॑हत्यै । \newline
27. अप॑हत्यै ब्रह्मवा॒दिनो᳚ ब्रह्मवा॒दिनो ऽप॑हत्या॒ अप॑हत्यै ब्रह्मवा॒दिनः॑ । \newline
28. अप॑हत्या॒ इत्यप॑ - ह॒त्यै॒ । \newline
29. ब्र॒ह्म॒वा॒दिनो॑ वदन्ति वदन्ति ब्रह्मवा॒दिनो᳚ ब्रह्मवा॒दिनो॑ वदन्ति । \newline
30. ब्र॒ह्म॒वा॒दिन॒ इति॑ ब्रह्म - वा॒दिनः॑ । \newline
31. व॒द॒ न्त्य॒न्वा॒रभ्यो᳚ ऽन्वा॒रभ्यो॑ वदन्ति वद न्त्यन्वा॒रभ्यः॑ । \newline
32. अ॒न्वा॒रभ्यः॑ प॒शू(3)ः प॒शू(3) र॑न्वा॒रभ्यो᳚ ऽन्वा॒रभ्यः॑ प॒शू(3)ः । \newline
33. अ॒न्वा॒रभ्य॒ इत्य॑नु - आ॒रभ्यः॑ । \newline
34. प॒शू(3)र् न न प॒शू(3)ः प॒शू(3)र् न । \newline
35. नान्वा॒रभ्या(3) अ॑न्वा॒रभ्या(3) न नान्वा॒रभ्या(3)ः । \newline
36. अ॒न्वा॒रभ्या(3) इतीत्य॑ न्वा॒रभ्या(3) अ॑न्वा॒रभ्या(3) इति॑ । \newline
37. अ॒न्वा॒रभ्या(3) इत्य॑नु - आ॒रभ्या(3)ः । \newline
38. इति॑ मृ॒त्यवे॑ मृ॒त्यव॒ इतीति॑ मृ॒त्यवे᳚ । \newline
39. मृ॒त्यवे॒ वै वै मृ॒त्यवे॑ मृ॒त्यवे॒ वै । \newline
40. वा ए॒ष ए॒ष वै वा ए॒षः । \newline
41. ए॒ष नी॑यते नीयत ए॒ष ए॒ष नी॑यते । \newline
42. नी॒य॒ते॒ यद् यन् नी॑यते नीयते॒ यत् । \newline
43. यत् प॒शुः प॒शुर् यद् यत् प॒शुः । \newline
44. प॒शु स्तम् तम् प॒शुः प॒शु स्तम् । \newline
45. तं ॅयद् यत् तम् तं ॅयत् । \newline
46. यद॑न्वा॒रभे॑ता न्वा॒रभे॑त॒ यद् यद॑न्वा॒रभे॑त । \newline
47. अ॒न्वा॒रभे॑त प्र॒मायु॑कः प्र॒मायु॑को ऽन्वा॒रभे॑ता न्वा॒रभे॑त प्र॒मायु॑कः । \newline
48. अ॒न्वा॒रभे॒तेत्य॑नु - आ॒रभे॑त । \newline
49. प्र॒मायु॑को॒ यज॑मानो॒ यज॑मानः प्र॒मायु॑कः प्र॒मायु॑को॒ यज॑मानः । \newline
50. प्र॒मायु॑क॒ इति॑ प्र - मायु॑कः । \newline
51. यज॑मानः स्याथ् स्या॒द् यज॑मानो॒ यज॑मानः स्यात् । \newline
52. स्या॒ दथो॒ अथो᳚ स्याथ् स्या॒ दथो᳚ । \newline
53. अथो॒ खलु॒ खल्वथो॒ अथो॒ खलु॑ । \newline
54. अथो॒ इत्यथो᳚ । \newline
55. खल्वा॑हु राहुः॒ खलु॒ खल्वा॑हुः । \newline
56. आ॒हुः॒ सु॒व॒र्गाय॑ सुव॒र्गा या॑हु राहुः सुव॒र्गाय॑ । \newline
57. सु॒व॒र्गाय॒ वै वै सु॑व॒र्गाय॑ सुव॒र्गाय॒ वै । \newline
58. सु॒व॒र्गायेति॑ सुवः - गाय॑ । \newline
59. वा ए॒ष ए॒ष वै वा ए॒षः । \newline
60. ए॒ष लो॒काय॑ लो॒का यै॒ष ए॒ष लो॒काय॑ । \newline
61. लो॒काय॑ नीयते नीयते लो॒काय॑ लो॒काय॑ नीयते । \newline
62. नी॒य॒ते॒ यद् यन् नी॑यते नीयते॒ यत् । \newline
63. यत् प॒शुः प॒शुर् यद् यत् प॒शुः । \newline

\textbf{Ghana Paata } \newline

1. पर्य॑ग्नि करोति करोति॒ पर्य॑ग्नि॒ पर्य॑ग्नि करोति सर्व॒हुतꣳ॑ सर्व॒हुत॑म् करोति॒ पर्य॑ग्नि॒ पर्य॑ग्नि करोति सर्व॒हुत᳚म् । \newline
2. पर्य॒ग्नीति॒ परि॑ - अ॒ग्नि॒ । \newline
3. क॒रो॒ति॒ स॒र्व॒हुतꣳ॑ सर्व॒हुत॑म् करोति करोति सर्व॒हुत॑ मे॒वैव स॑र्व॒हुत॑म् करोति करोति सर्व॒हुत॑ मे॒व । \newline
4. स॒र्व॒हुत॑ मे॒वैव स॑र्व॒हुतꣳ॑ सर्व॒हुत॑ मे॒वैन॑ मेन मे॒व स॑र्व॒हुतꣳ॑ सर्व॒हुत॑ मे॒वैन᳚म् । \newline
5. स॒र्व॒हुत॒मिति॑ सर्व - हुत᳚म् । \newline
6. ए॒वैन॑ मेन मे॒वै वैन॑म् करोति करो त्येन मे॒वै वैन॑म् करोति । \newline
7. ए॒न॒म् क॒रो॒ति॒ क॒रो॒ त्ये॒न॒ मे॒न॒म् क॒रो॒ त्यस्क॑न्दा॒या स्क॑न्दाय करो त्येन मेनम् करो॒ त्यस्क॑न्दाय । \newline
8. क॒रो॒ त्यस्क॑न्दा॒या स्क॑न्दाय करोति करो॒ त्यस्क॑न्दा॒या स्क॑न्न॒ मस्क॑न्न॒ मस्क॑न्दाय करोति करो॒ त्यस्क॑न्दा॒या स्क॑न्नम् । \newline
9. अस्क॑न्दा॒या स्क॑न्न॒ मस्क॑न्न॒ मस्क॑न्दा॒या स्क॑न्दा॒या स्क॑न्नꣳ॒॒ हि ह्यस्क॑न्न॒ मस्क॑न्दा॒या स्क॑न्दा॒या स्क॑न्नꣳ॒॒ हि । \newline
10. अस्क॑न्नꣳ॒॒ हि ह्यस्क॑न्न॒ मस्क॑न्नꣳ॒॒ हि तत् तद्ध्यस्क॑न्न॒ मस्क॑न्नꣳ॒॒ हि तत् । \newline
11. हि तत् तद्धि हि तद् यद् यत् तद्धि हि तद् यत् । \newline
12. तद् यद् यत् तत् तद् यद्धु॒तस्य॑ हु॒तस्य॒ यत् तत् तद् यद्धु॒तस्य॑ । \newline
13. यद्धु॒तस्य॑ हु॒तस्य॒ यद् यद्धु॒तस्य॒ स्कन्द॑ति॒ स्कन्द॑ति हु॒तस्य॒ यद् यद्धु॒तस्य॒ स्कन्द॑ति । \newline
14. हु॒तस्य॒ स्कन्द॑ति॒ स्कन्द॑ति हु॒तस्य॑ हु॒तस्य॒ स्कन्द॑ति॒ त्रि स्त्रिः स्कन्द॑ति हु॒तस्य॑ हु॒तस्य॒ स्कन्द॑ति॒ त्रिः । \newline
15. स्कन्द॑ति॒ त्रि स्त्रिः स्कन्द॑ति॒ स्कन्द॑ति॒ त्रिः पर्य॑ग्नि॒ पर्य॑ग्नि॒ त्रिः स्कन्द॑ति॒ स्कन्द॑ति॒ त्रिः पर्य॑ग्नि । \newline
16. त्रिः पर्य॑ग्नि॒ पर्य॑ग्नि॒ त्रि स्त्रिः पर्य॑ग्नि करोति करोति॒ पर्य॑ग्नि॒ त्रि स्त्रिः पर्य॑ग्नि करोति । \newline
17. पर्य॑ग्नि करोति करोति॒ पर्य॑ग्नि॒ पर्य॑ग्नि करोति॒ त्र्या॑वृ॒त् त्र्या॑वृत् करोति॒ पर्य॑ग्नि॒ पर्य॑ग्नि करोति॒ त्र्या॑वृत् । \newline
18. पर्य॒ग्नीति॒ परि॑ - अ॒ग्नि॒ । \newline
19. क॒रो॒ति॒ त्र्या॑वृ॒त् त्र्या॑वृत् करोति करोति॒ त्र्या॑वृ॒द्धि हि त्र्या॑वृत् करोति करोति॒ त्र्या॑वृ॒द्धि । \newline
20. त्र्या॑वृ॒द्धि हि त्र्या॑वृ॒त् त्र्या॑वृ॒द्धि य॒ज्ञो य॒ज्ञो हि त्र्या॑वृ॒त् त्र्या॑वृ॒द्धि य॒ज्ञ्ः । \newline
21. त्र्या॑वृ॒दिति॒ त्रि - आ॒वृ॒त् । \newline
22. हि य॒ज्ञो य॒ज्ञो हि हि य॒ज्ञो ऽथो॒ अथो॑ य॒ज्ञो हि हि य॒ज्ञो ऽथो᳚ । \newline
23. य॒ज्ञो ऽथो॒ अथो॑ य॒ज्ञो य॒ज्ञो ऽथो॒ रक्ष॑साꣳ॒॒ रक्ष॑सा॒ मथो॑ य॒ज्ञो य॒ज्ञो ऽथो॒ रक्ष॑साम् । \newline
24. अथो॒ रक्ष॑साꣳ॒॒ रक्ष॑सा॒ मथो॒ अथो॒ रक्ष॑सा॒ मप॑हत्या॒ अप॑हत्यै॒ रक्ष॑सा॒ मथो॒ अथो॒ रक्ष॑सा॒ मप॑हत्यै । \newline
25. अथो॒ इत्यथो᳚ । \newline
26. रक्ष॑सा॒ मप॑हत्या॒ अप॑हत्यै॒ रक्ष॑साꣳ॒॒ रक्ष॑सा॒ मप॑हत्यै ब्रह्मवा॒दिनो᳚ ब्रह्मवा॒दिनो ऽप॑हत्यै॒ रक्ष॑साꣳ॒॒ रक्ष॑सा॒ मप॑हत्यै ब्रह्मवा॒दिनः॑ । \newline
27. अप॑हत्यै ब्रह्मवा॒दिनो᳚ ब्रह्मवा॒दिनो ऽप॑हत्या॒ अप॑हत्यै ब्रह्मवा॒दिनो॑ वदन्ति वदन्ति ब्रह्मवा॒दिनो ऽप॑हत्या॒ अप॑हत्यै ब्रह्मवा॒दिनो॑ वदन्ति । \newline
28. अप॑हत्या॒ इत्यप॑ - ह॒त्यै॒ । \newline
29. ब्र॒ह्म॒वा॒दिनो॑ वदन्ति वदन्ति ब्रह्मवा॒दिनो᳚ ब्रह्मवा॒दिनो॑ वद न्त्यन्वा॒रभ्यो᳚ ऽन्वा॒रभ्यो॑ वदन्ति ब्रह्मवा॒दिनो᳚ ब्रह्मवा॒दिनो॑ वद न्त्यन्वा॒रभ्यः॑ । \newline
30. ब्र॒ह्म॒वा॒दिन॒ इति॑ ब्रह्म - वा॒दिनः॑ । \newline
31. व॒द॒न्त्य॒ न्वा॒रभ्यो᳚ ऽन्वा॒रभ्यो॑ वदन्ति वदन्त्य न्वा॒रभ्यः॑ प॒शू(3)ः प॒शू(3) र॑न्वा॒रभ्यो॑ वदन्ति वदन्त्य न्वा॒रभ्यः॑ प॒शू(3)ः । \newline
32. अ॒न्वा॒रभ्यः॑ प॒शू(3)ः प॒शू(3) र॑न्वा॒रभ्यो᳚ ऽन्वा॒रभ्यः॑ प॒शू(3)र् न न प॒शू(3) र॑न्वा॒रभ्यो᳚ ऽन्वा॒रभ्यः॑ प॒शू(3)र् न । \newline
33. अ॒न्वा॒रभ्य॒ इत्य॑नु - आ॒रभ्यः॑ । \newline
34. प॒शू(3)र् न न प॒शू(3)ः प॒शू(3)र् नान्वा॒रभ्या(3) अ॑न्वा॒रभ्या(3) न प॒शू(3)ः प॒शू(3)र् नान्वा॒रभ्या(3)ः । \newline
35. नान्वा॒रभ्या(3) अ॑न्वा॒रभ्या(3) न नान्वा॒रभ्या(3) इती त्य॑न्वा॒रभ्या(3) न नान्वा॒रभ्या(3) इति॑ । \newline
36. अ॒न्वा॒रभ्या(3) इती त्य॑न्वा॒रभ्या(3) अ॑न्वा॒रभ्या(3) इति॑ मृ॒त्यवे॑ मृ॒त्यव॒ इत्य॑न्वा॒रभ्या(3) अ॑न्वा॒रभ्या(3) इति॑ मृ॒त्यवे᳚ । \newline
37. अ॒न्वा॒रभ्या(3) इत्य॑नु - आ॒रभ्या(3)ः । \newline
38. इति॑ मृ॒त्यवे॑ मृ॒त्यव॒ इतीति॑ मृ॒त्यवे॒ वै वै मृ॒त्यव॒ इतीति॑ मृ॒त्यवे॒ वै । \newline
39. मृ॒त्यवे॒ वै वै मृ॒त्यवे॑ मृ॒त्यवे॒ वा ए॒ष ए॒ष वै मृ॒त्यवे॑ मृ॒त्यवे॒ वा ए॒षः । \newline
40. वा ए॒ष ए॒ष वै वा ए॒ष नी॑यते नीयत ए॒ष वै वा ए॒ष नी॑यते । \newline
41. ए॒ष नी॑यते नीयत ए॒ष ए॒ष नी॑यते॒ यद् यन् नी॑यत ए॒ष ए॒ष नी॑यते॒ यत् । \newline
42. नी॒य॒ते॒ यद् यन् नी॑यते नीयते॒ यत् प॒शुः प॒शुर् यन् नी॑यते नीयते॒ यत् प॒शुः । \newline
43. यत् प॒शुः प॒शुर् यद् यत् प॒शु स्तम् तम् प॒शुर् यद् यत् प॒शु स्तम् । \newline
44. प॒शु स्तम् तम् प॒शुः प॒शु स्तं ॅयद् यत् तम् प॒शुः प॒शु स्तं ॅयत् । \newline
45. तं ॅयद् यत् तम् तं ॅयद॑न्वा॒रभे॑ता न्वा॒रभे॑त॒ यत् तम् तं ॅयद॑न्वा॒रभे॑त । \newline
46. यद॑न्वा॒रभे॑ता न्वा॒रभे॑त॒ यद् यद॑न्वा॒रभे॑त प्र॒मायु॑कः प्र॒मायु॑को ऽन्वा॒रभे॑त॒ यद् यद॑न्वा॒रभे॑त प्र॒मायु॑कः । \newline
47. अ॒न्वा॒रभे॑त प्र॒मायु॑कः प्र॒मायु॑को ऽन्वा॒रभे॑ता न्वा॒रभे॑त प्र॒मायु॑को॒ यज॑मानो॒ यज॑मानः प्र॒मायु॑को ऽन्वा॒रभे॑ता न्वा॒रभे॑त प्र॒मायु॑को॒ यज॑मानः । \newline
48. अ॒न्वा॒रभे॒तेत्य॑नु - आ॒रभे॑त । \newline
49. प्र॒मायु॑को॒ यज॑मानो॒ यज॑मानः प्र॒मायु॑कः प्र॒मायु॑को॒ यज॑मानः स्याथ् स्या॒द् यज॑मानः प्र॒मायु॑कः प्र॒मायु॑को॒ यज॑मानः स्यात् । \newline
50. प्र॒मायु॑क॒ इति॑ प्र - मायु॑कः । \newline
51. यज॑मानः स्याथ् स्या॒द् यज॑मानो॒ यज॑मानः स्या॒ दथो॒ अथो᳚ स्या॒द् यज॑मानो॒ यज॑मानः स्या॒ दथो᳚ । \newline
52. स्या॒ दथो॒ अथो᳚ स्याथ् स्या॒ दथो॒ खलु॒ खल्वथो᳚ स्याथ् स्या॒ दथो॒ खलु॑ । \newline
53. अथो॒ खलु॒ खल्वथो॒ अथो॒ खल्वा॑हु राहुः॒ खल्वथो॒ अथो॒ खल्वा॑हुः । \newline
54. अथो॒ इत्यथो᳚ । \newline
55. खल्वा॑हु राहुः॒ खलु॒ खल्वा॑हुः सुव॒र्गाय॑ सुव॒र्गा या॑हुः॒ खलु॒ खल्वा॑हुः सुव॒र्गाय॑ । \newline
56. आ॒हुः॒ सु॒व॒र्गाय॑ सुव॒र्गा या॑हु राहुः सुव॒र्गाय॒ वै वै सु॑व॒र्गा या॑हु राहुः सुव॒र्गाय॒ वै । \newline
57. सु॒व॒र्गाय॒ वै वै सु॑व॒र्गाय॑ सुव॒र्गाय॒ वा ए॒ष ए॒ष वै सु॑व॒र्गाय॑ सुव॒र्गाय॒ वा ए॒षः । \newline
58. सु॒व॒र्गायेति॑ सुवः - गाय॑ । \newline
59. वा ए॒ष ए॒ष वै वा ए॒ष लो॒काय॑ लो॒कायै॒ष वै वा ए॒ष लो॒काय॑ । \newline
60. ए॒ष लो॒काय॑ लो॒कायै॒ष ए॒ष लो॒काय॑ नीयते नीयते लो॒कायै॒ष ए॒ष लो॒काय॑ नीयते । \newline
61. लो॒काय॑ नीयते नीयते लो॒काय॑ लो॒काय॑ नीयते॒ यद् यन् नी॑यते लो॒काय॑ लो॒काय॑ नीयते॒ यत् । \newline
62. नी॒य॒ते॒ यद् यन् नी॑यते नीयते॒ यत् प॒शुः प॒शुर् यन् नी॑यते नीयते॒ यत् प॒शुः । \newline
63. यत् प॒शुः प॒शुर् यद् यत् प॒शुरि तीति॑ प॒शुर् यद् यत् प॒शुरिति॑ । \newline
\pagebreak
\markright{ TS 6.3.8.2  \hfill https://www.vedavms.in \hfill}

\section{ TS 6.3.8.2 }

\textbf{TS 6.3.8.2 } \newline
\textbf{Samhita Paata} \newline

प॒शुरिति॒ यन्नान्वा॒रभे॑त सुव॒र्गाल्लो॒काद् यज॑मानो हीयेत वपा॒श्रप॑णीभ्या-म॒न्वार॑भते॒ तन्नेवा॒न्वार॑ब्धं॒ नेवान॑न्वारब्ध॒मुप॒ प्रेष्य॑ होतर्.ह॒व्या दे॒वेभ्य॒ इत्या॑हेषि॒तꣳ हि कर्म॑ क्रि॒यते॒ रेव॑तीर्य॒ज्ञ्प॑तिं प्रिय॒धा ऽऽवि॑श॒तेत्या॑ह यथाय॒जुरे॒वैतद॒ग्निना॑ पु॒रस्ता॑देति॒ रक्ष॑सा॒मप॑हत्यै पृथि॒व्याः स॒पृंचः॑ पा॒हीति॑ ब॒र्॒.हि- [  ] \newline

\textbf{Pada Paata} \newline

प॒शुः । इति॑ । यत् । न । अ॒न्वा॒रभे॒तेत्य॑नु - आ॒रभे॑त । सु॒व॒र्गादिति॑ सुवः - गात् । लो॒कात् । यज॑मानः । ही॒ये॒त॒ । व॒पा॒श्रप॑णीभ्या॒मिति॑ वपा - श्रप॑णीभ्याम् । अन्वार॑भत॒ इत्य॑नु - आर॑भते । तत् । न । इ॒व॒ । अ॒न्वार॑ब्ध॒मित्य॑नु-आर॑ब्धम् । न । इ॒व॒ । अन॑न्वारब्ध॒मित्यन॑नु-आ॒र॒ब्ध॒म् । उप॑ । प्रेति॑ । इ॒ष्य॒ । हो॒तः॒ । ह॒व्या । दे॒वेभ्यः॑ । इति॑ । आ॒ह॒ । इ॒षि॒तम् । हि । कर्म॑ । क्रि॒यते᳚ । रेव॑तीः । य॒ज्ञ्प॑ति॒मिति॑ य॒ज्ञ् - प॒ति॒म् । प्रि॒य॒धेति॑ प्रिय - धा । एति॑ । वि॒श॒त॒ । इति॑ । आ॒ह॒ । य॒था॒य॒जुरिति॑ यथा - य॒जुः । ए॒व । ए॒तत् । अ॒ग्निना᳚ । पु॒रस्ता᳚त् । ए॒ति॒ । रक्ष॑साम् । अप॑हत्या॒ इत्यप॑ - ह॒त्यै॒ । पृ॒थि॒व्याः । स॒पृंच॒ इति॑ सं - पृचः॑ । पा॒हि॒ । इति॑ । ब॒र्॒.हिः ।  \newline


\textbf{Krama Paata} \newline

प॒शुरिति॑ । इति॒ यत् । यन् न । नान्वा॒रभे॑त । अ॒न्वा॒रभे॑त सुव॒र्गात् । अ॒न्वा॒रभे॒तेत्य॑नु - आ॒रभे॑त । सु॒व॒र्गाल्लो॒कात् । सु॒व॒र्गादिति॑ सुवः - गात् । लो॒काद् यज॑मानः । यज॑मानो हीयेत । ही॒ये॒त॒ व॒पा॒श्रप॑णीभ्याम् । व॒पा॒श्रप॑णीभ्याम॒न्वार॑भते । व॒पा॒श्रप॑णीभ्या॒मिति॑ वपा - श्रप॑णीभ्याम् । अ॒न्वार॑भते॒ तत् । अ॒न्वारभ॑त॒ इत्य॑नु - आर॑भते । तन् न । नेव॑ । इ॒वान्वार॑ब्धम् । अ॒न्वार॑ब्ध॒म् न । अ॒न्वार॑ब्ध॒मित्य॑नु - आर॑ब्धम् । नेव॑ । इ॒वान॑न्वारब्धम् । अन॑न्वारब्ध॒मुप॑ । अन॑न्वारब्ध॒मित्यन॑नु - आ॒र॒ब्ध॒म् । उप॒ प्र । प्रेष्य॑ । इ॒ष्य॒ हो॒तः॒ । हो॒त॒र्.॒ ह॒व्या । ह॒व्या दे॒वेभ्यः॑ । दे॒वेभ्य॒ इति॑ । इत्या॑ह । आ॒हे॒षि॒तम् । इ॒षि॒तꣳ हि । हि कर्म॑ । कर्म॑ क्रि॒यते᳚ । क्रि॒यते॒ रेव॑तीः । रेव॑तीर् य॒ज्ञ्प॑तिम् । य॒ज्ञ्प॑तिम् प्रिय॒धा । य॒ज्ञ्प॑ति॒मिति॑ य॒ज्ञ् - प॒ति॒म् । प्रि॒य॒धा ऽऽ वि॑शत । प्रि॒य॒धेति॑ प्रिय - धा । आ वि॑शत । वि॒श॒तेति॑ । इत्या॑ह । आ॒ह॒ य॒था॒य॒जुः । य॒था॒य॒जुरे॒व । य॒था॒य॒जुरिति॑ यथा - य॒जुः । ए॒वैतत् । ए॒तद॒ग्निना᳚ । अ॒ग्निना॑ पु॒रस्ता᳚त् । पु॒रस्ता॑देति । ए॒ति॒ रक्ष॑साम् । रक्ष॑सा॒मप॑हत्यै । अप॑हत्यै पृथि॒व्याः । अप॑हत्या॒ इत्यप॑ - ह॒त्यै॒ । पृ॒थि॒व्याः स॒म्पृचः॑ । स॒म्पृचः॑ पाहि । स॒म्पृच॒ इति॑ सम् - पृचः॑ । पा॒हीति॑ । इति॑ ब॒र्.॒हिः । ब॒र्.॒हिरुप॑ \newline

\textbf{Jatai Paata} \newline

1. प॒शु रितीति॑ प॒शुः प॒शु रिति॑ । \newline
2. इति॒ यद् यदितीति॒ यत् । \newline
3. यन् न न यद् यन् न । \newline
4. नान्वा॒रभे॑ता न्वा॒रभे॑त॒ न नान्वा॒रभे॑त । \newline
5. अ॒न्वा॒रभे॑त सुव॒र्गाथ् सु॑व॒र्गा द॑न्वा॒रभे॑ता न्वा॒रभे॑त सुव॒र्गात् । \newline
6. अ॒न्वा॒रभे॒तेत्य॑नु - आ॒रभे॑त । \newline
7. सु॒व॒र्गा ल्लो॒का ल्लो॒काथ् सु॑व॒र्गाथ् सु॑व॒र्गा ल्लो॒कात् । \newline
8. सु॒व॒र्गादिति॑ सुवः - गात् । \newline
9. लो॒काद् यज॑मानो॒ यज॑मानो लो॒का ल्लो॒काद् यज॑मानः । \newline
10. यज॑मानो हीयेत हीयेत॒ यज॑मानो॒ यज॑मानो हीयेत । \newline
11. ही॒ये॒त॒ व॒पा॒श्रप॑णीभ्यां ॅवपा॒श्रप॑णीभ्याꣳ हीयेत हीयेत वपा॒श्रप॑णीभ्याम् । \newline
12. व॒पा॒श्रप॑णीभ्या म॒न्वार॑भते॒ ऽन्वार॑भते वपा॒श्रप॑णीभ्यां ॅवपा॒श्रप॑णीभ्या म॒न्वार॑भते । \newline
13. व॒पा॒श्रप॑णीभ्या॒मिति॑ वपा - श्रप॑णीभ्याम् । \newline
14. अ॒न्वार॑भते॒ तत् तद॒न्वार॑भते॒ ऽन्वार॑भते॒ तत् । \newline
15. अ॒न्वार॑भत॒ इत्य॑नु - आर॑भते । \newline
16. तन् न न तत् तन् न । \newline
17. नेवे॑व॒ न नेव॑ । \newline
18. इ॒वा॒न् वार॑ब्ध म॒न् वार॑ब्ध मिवे वा॒न् वार॑ब्धम् । \newline
19. अ॒न् वार॑ब्ध॒न्न नान् वार॑ब्ध म॒न् वार॑ब्ध॒न्न । \newline
20. अ॒न्वार॑ब्ध॒मित्य॑नु - आर॑ब्धम् । \newline
21. नेवे॑व॒ न नेव॑ । \newline
22. इ॒वान॑न् वारब्ध॒ मन॑न् वारब्ध मिवे॒ वान॑न् वारब्धम् । \newline
23. अन॑न् वारब्ध॒ मुपोपा न॑न् वारब्ध॒ मन॑न् वारब्ध॒ मुप॑ । \newline
24. अन॑न्वारब्ध॒मित्यन॑नु - आ॒र॒ब्ध॒म् । \newline
25. उप॒ प्र प्रोपोप॒ प्र । \newline
26. प्रेष्ये᳚ष्य॒ प्र प्रेष्य॑ । \newline
27. इ॒ष्य॒ हो॒त॒र्॒. हो॒त॒ रि॒ष्ये॒ष्य॒ हो॒तः॒ । \newline
28. हो॒त॒र्॒. ह॒व्या ह॒व्या हो॑तर्. होतर्. ह॒व्या । \newline
29. ह॒व्या दे॒वेभ्यो॑ दे॒वेभ्यो॑ ह॒व्या ह॒व्या दे॒वेभ्यः॑ । \newline
30. दे॒वेभ्य॒ इतीति॑ दे॒वेभ्यो॑ दे॒वेभ्य॒ इति॑ । \newline
31. इत्या॑हा॒हे तीत्या॑ह । \newline
32. आ॒हे॒षि॒त मि॑षि॒त मा॑हाहेषि॒तम् । \newline
33. इ॒षि॒तꣳ हि हीषि॒त मि॑षि॒तꣳ हि । \newline
34. हि कर्म॒ कर्म॒ हि हि कर्म॑ । \newline
35. कर्म॑ क्रि॒यते᳚ क्रि॒यते॒ कर्म॒ कर्म॑ क्रि॒यते᳚ । \newline
36. क्रि॒यते॒ रेव॑ती॒ रेव॑तीः क्रि॒यते᳚ क्रि॒यते॒ रेव॑तीः । \newline
37. रेव॑तीर् य॒ज्ञ्प॑तिं ॅय॒ज्ञ्प॑तिꣳ॒॒ रेव॑ती॒ रेव॑तीर् य॒ज्ञ्प॑तिम् । \newline
38. य॒ज्ञ्प॑तिम् प्रिय॒धा प्रि॑य॒धा य॒ज्ञ्प॑तिं ॅय॒ज्ञ्प॑तिम् प्रिय॒धा । \newline
39. य॒ज्ञ्प॑ति॒मिति॑ य॒ज्ञ् - प॒ति॒म् । \newline
40. प्रि॒य॒धा ऽऽवि॑शत विश॒ता प्रि॑य॒धा प्रि॑य॒धा ऽऽवि॑शत । \newline
41. प्रि॒य॒धेति॑ प्रिय - धा । \newline
42. आ वि॑शत विश॒ता वि॑शत । \newline
43. वि॒श॒ते तीति॑ विशत विश॒तेति॑ । \newline
44. इत्या॑हा॒हे तीत्या॑ह । \newline
45. आ॒ह॒ य॒था॒य॒जुर् य॑थाय॒जु रा॑हाह यथाय॒जुः । \newline
46. य॒था॒य॒जु रे॒वैव य॑थाय॒जुर् य॑थाय॒जु रे॒व । \newline
47. य॒था॒य॒जुरिति॑ यथा - य॒जुः । \newline
48. ए॒वैत दे॒त दे॒वै वैतत् । \newline
49. ए॒त द॒ग्निना॒ ऽग्निनै॒त दे॒त द॒ग्निना᳚ । \newline
50. अ॒ग्निना॑ पु॒रस्ता᳚त् पु॒रस्ता॑ द॒ग्निना॒ ऽग्निना॑ पु॒रस्ता᳚त् । \newline
51. पु॒रस्ता॑ देत्येति पु॒रस्ता᳚त् पु॒रस्ता॑ देति । \newline
52. ए॒ति॒ रक्ष॑साꣳ॒॒ रक्ष॑सा मेत्येति॒ रक्ष॑साम् । \newline
53. रक्ष॑सा॒ मप॑हत्या॒ अप॑हत्यै॒ रक्ष॑साꣳ॒॒ रक्ष॑सा॒ मप॑हत्यै । \newline
54. अप॑हत्यै पृथि॒व्याः पृ॑थि॒व्या अप॑हत्या॒ अप॑हत्यै पृथि॒व्याः । \newline
55. अप॑हत्या॒ इत्यप॑ - ह॒त्यै॒ । \newline
56. पृ॒थि॒व्याः सं॒पृचः॑ सं॒पृचः॑ पृथि॒व्याः पृ॑थि॒व्याः सं॒पृचः॑ । \newline
57. सं॒पृचः॑ पाहि पाहि सं॒पृचः॑ सं॒पृचः॑ पाहि । \newline
58. सं॒पृच॒ इति॑ सं - पृचः॑ । \newline
59. पा॒ही तीति॑ पाहि पा॒हीति॑ । \newline
60. इति॑ ब॒र्॒.हिर् ब॒र्॒.हि रितीति॑ ब॒र्॒.हिः । \newline
61. ब॒र्॒.हि रुपोप॑ ब॒र्॒.हिर् ब॒र्॒.हि रुप॑ । \newline

\textbf{Ghana Paata } \newline

1. प॒शुरि तीति॑ प॒शुः प॒शुरिति॒ यद् यदिति॑ प॒शुः प॒शुरिति॒ यत् । \newline
2. इति॒ यद् यदितीति॒ यन् न न यदितीति॒ यन् न । \newline
3. यन् न न यद् यन् नान्वा॒रभे॑ता न्वा॒रभे॑त॒ न यद् यन् नान्वा॒रभे॑त । \newline
4. नान्वा॒रभे॑ता न्वा॒रभे॑त॒ न नान्वा॒रभे॑त सुव॒र्गाथ् सु॑व॒र्गा द॑न्वा॒रभे॑त॒ न नान्वा॒रभे॑त सुव॒र्गात् । \newline
5. अ॒न्वा॒रभे॑त सुव॒र्गाथ् सु॑व॒र्गा द॑न्वा॒रभे॑ता न्वा॒रभे॑त सुव॒र्गा ल्लो॒का ल्लो॒काथ् सु॑व॒र्गा द॑न्वा॒रभे॑ता न्वा॒रभे॑त सुव॒र्गा ल्लो॒कात् । \newline
6. अ॒न्वा॒रभे॒तेत्य॑नु - आ॒रभे॑त । \newline
7. सु॒व॒र्गा ल्लो॒का ल्लो॒काथ् सु॑व॒र्गाथ् सु॑व॒र्गा ल्लो॒काद् यज॑मानो॒ यज॑मानो लो॒काथ् सु॑व॒र्गाथ् सु॑व॒र्गा ल्लो॒काद् यज॑मानः । \newline
8. सु॒व॒र्गादिति॑ सुवः - गात् । \newline
9. लो॒काद् यज॑मानो॒ यज॑मानो लो॒का ल्लो॒काद् यज॑मानो हीयेत हीयेत॒ यज॑मानो लो॒का ल्लो॒काद् यज॑मानो हीयेत । \newline
10. यज॑मानो हीयेत हीयेत॒ यज॑मानो॒ यज॑मानो हीयेत वपा॒श्रप॑णीभ्यां ॅवपा॒श्रप॑णीभ्याꣳ हीयेत॒ यज॑मानो॒ यज॑मानो हीयेत वपा॒श्रप॑णीभ्याम् । \newline
11. ही॒ये॒त॒ व॒पा॒श्रप॑णीभ्यां ॅवपा॒श्रप॑णीभ्याꣳ हीयेत हीयेत वपा॒श्रप॑णीभ्या म॒न्वार॑भते॒ ऽन्वार॑भते वपा॒श्रप॑णीभ्याꣳ हीयेत हीयेत वपा॒श्रप॑णीभ्या म॒न्वार॑भते । \newline
12. व॒पा॒श्रप॑णीभ्या म॒न्वार॑भते॒ ऽन्वार॑भते वपा॒श्रप॑णीभ्यां ॅवपा॒श्रप॑णीभ्या म॒न्वार॑भते॒ तत् तद॒न्वार॑भते वपा॒श्रप॑णीभ्यां ॅवपा॒श्रप॑णीभ्या म॒न्वार॑भते॒ तत् । \newline
13. व॒पा॒श्रप॑णीभ्या॒मिति॑ वपा - श्रप॑णीभ्याम् । \newline
14. अ॒न्वार॑भते॒ तत् तद॒न्वार॑भते॒ ऽन्वार॑भते॒ तन् न न तद॒न्वार॑भते॒ ऽन्वार॑भते॒ तन् न । \newline
15. अ॒न्वार॑भत॒ इत्य॑नु - आर॑भते । \newline
16. तन् न न तत् तन् नेवे॑व॒ न तत् तन् नेव॑ । \newline
17. नेवे॑ व॒ न नेवा॒न्वार॑ब्ध म॒न्वार॑ब्ध मिव॒ न नेवा॒न्वार॑ब्धम् । \newline
18. इ॒वा॒न्वार॑ब्ध म॒न्वार॑ब्ध मिवे वा॒न्वार॑ब्ध॒न् न नान्वार॑ब्ध मिवे वा॒न्वार॑ब्ध॒न् न । \newline
19. अ॒न्वार॑ब्ध॒न् न नान्वार॑ब्ध म॒न्वार॑ब्ध॒न् नेवे॑व॒ नान्वार॑ब्ध म॒न्वार॑ब्ध॒न् नेव॑ । \newline
20. अ॒न्वार॑ब्ध॒मित्य॑नु - आर॑ब्धम् । \newline
21. नेवे॑ व॒ न नेवान॑न्वारब्ध॒ मन॑न्वारब्ध मिव॒ न नेवान॑न्वारब्धम् । \newline
22. इ॒वान॑न्वारब्ध॒ मन॑न्वारब्ध मिवे॒ वान॑न्वारब्ध॒ मुपोपान॑न्वारब्ध मिवे॒ वान॑न्वारब्ध॒ मुप॑ । \newline
23. अन॑न्वारब्ध॒ मुपोपा न॑न्वारब्ध॒ मन॑न्वारब्ध॒ मुप॒ प्र प्रोपा न॑न्वारब्ध॒ मन॑न्वारब्ध॒ मुप॒ प्र । \newline
24. अन॑न्वारब्ध॒मित्यन॑नु - आ॒र॒ब्ध॒म् । \newline
25. उप॒ प्र प्रोपोप॒ प्रेष्ये᳚ ष्य॒ प्रोपोप॒ प्रेष्य॑ । \newline
26. प्रेष्ये᳚ ष्य॒ प्र प्रेष्य॑ होतर्. होतरिष्य॒ प्र प्रेष्य॑ होतः । \newline
27. इ॒ष्य॒ हो॒त॒र्.॒ हो॒त॒ रि॒ष्ये॒ ष्य॒ हो॒त॒र्.॒ ह॒व्या ह॒व्या हो॑तरिष्ये ष्य होतर्. ह॒व्या । \newline
28. हो॒त॒र्.॒ ह॒व्या ह॒व्या हो॑तर् होतर्. ह॒व्या दे॒वेभ्यो॑ दे॒वेभ्यो॑ ह॒व्या हो॑तर्. होतर् ह॒व्या दे॒वेभ्यः॑ । \newline
29. ह॒व्या दे॒वेभ्यो॑ दे॒वेभ्यो॑ ह॒व्या ह॒व्या दे॒वेभ्य॒ इतीति॑ दे॒वेभ्यो॑ ह॒व्या ह॒व्या दे॒वेभ्य॒ इति॑ । \newline
30. दे॒वेभ्य॒ इतीति॑ दे॒वेभ्यो॑ दे॒वेभ्य॒ इत्या॑हा॒ हेति॑ दे॒वेभ्यो॑ दे॒वेभ्य॒ इत्या॑ह । \newline
31. इत्या॑हा॒हे तीत्या॑ हेषि॒त मि॑षि॒त मा॒हे तीत्या॑ हेषि॒तम् । \newline
32. आ॒हे॒षि॒त मि॑षि॒त मा॑हा हेषि॒तꣳ हि हीषि॒त मा॑हा हेषि॒तꣳ हि । \newline
33. इ॒षि॒तꣳ हि हीषि॒त मि॑षि॒तꣳ हि कर्म॒ कर्म॒ हीषि॒त मि॑षि॒तꣳ हि कर्म॑ । \newline
34. हि कर्म॒ कर्म॒ हि हि कर्म॑ क्रि॒यते᳚ क्रि॒यते॒ कर्म॒ हि हि कर्म॑ क्रि॒यते᳚ । \newline
35. कर्म॑ क्रि॒यते᳚ क्रि॒यते॒ कर्म॒ कर्म॑ क्रि॒यते॒ रेव॑ती॒ रेव॑तीः क्रि॒यते॒ कर्म॒ कर्म॑ क्रि॒यते॒ रेव॑तीः । \newline
36. क्रि॒यते॒ रेव॑ती॒ रेव॑तीः क्रि॒यते᳚ क्रि॒यते॒ रेव॑तीर् य॒ज्ञ्प॑तिं ॅय॒ज्ञ्प॑तिꣳ॒॒ रेव॑तीः क्रि॒यते᳚ क्रि॒यते॒ रेव॑तीर् य॒ज्ञ्प॑तिम् । \newline
37. रेव॑तीर् य॒ज्ञ्प॑तिं ॅय॒ज्ञ्प॑तिꣳ॒॒ रेव॑ती॒ रेव॑तीर् य॒ज्ञ्प॑तिम् प्रिय॒धा प्रि॑य॒धा य॒ज्ञ्प॑तिꣳ॒॒ रेव॑ती॒ रेव॑तीर् य॒ज्ञ्प॑तिम् प्रिय॒धा । \newline
38. य॒ज्ञ्प॑तिम् प्रिय॒धा प्रि॑य॒धा य॒ज्ञ्प॑तिं ॅय॒ज्ञ्प॑तिम् प्रिय॒धा ऽऽवि॑शत विश॒ता प्रि॑य॒धा य॒ज्ञ्प॑तिं ॅय॒ज्ञ्प॑तिम् प्रिय॒धा ऽऽवि॑शत । \newline
39. य॒ज्ञ्प॑ति॒मिति॑ य॒ज्ञ् - प॒ति॒म् । \newline
40. प्रि॒य॒धा ऽऽवि॑शत विश॒ता प्रि॑य॒धा प्रि॑य॒धा ऽऽवि॑श॒ते तीति॑ विश॒ता प्रि॑य॒धा प्रि॑य॒धा ऽऽवि॑श॒तेति॑ । \newline
41. प्रि॒य॒धेति॑ प्रिय - धा । \newline
42. आ वि॑शत विश॒ता वि॑श॒ते तीति॑ विश॒ता वि॑श॒तेति॑ । \newline
43. वि॒श॒ते तीति॑ विशत विश॒ते त्या॑हा॒ हेति॑ विशत विश॒ते त्या॑ह । \newline
44. इत्या॑हा॒हे तीत्या॑ह यथाय॒जुर् य॑थाय॒जु रा॒हे तीत्या॑ह यथाय॒जुः । \newline
45. आ॒ह॒ य॒था॒य॒जुर् य॑थाय॒जु रा॑हाह यथाय॒जु रे॒वैव य॑थाय॒जु रा॑हाह यथाय॒जु रे॒व । \newline
46. य॒था॒य॒जु रे॒वैव य॑थाय॒जुर् य॑थाय॒जु रे॒वैत दे॒त दे॒व य॑थाय॒जुर् य॑थाय॒जु रे॒वैतत् । \newline
47. य॒था॒य॒जुरिति॑ यथा - य॒जुः । \newline
48. ए॒वैत दे॒त दे॒वैवैत द॒ग्निना॒ ऽग्निनै॒त दे॒वैवैत द॒ग्निना᳚ । \newline
49. ए॒त द॒ग्निना॒ ऽग्निनै॒त दे॒त द॒ग्निना॑ पु॒रस्ता᳚त् पु॒रस्ता॑ द॒ग्नि नै॒त दे॒त द॒ग्निना॑ पु॒रस्ता᳚त् । \newline
50. अ॒ग्निना॑ पु॒रस्ता᳚त् पु॒रस्ता॑ द॒ग्निना॒ ऽग्निना॑ पु॒रस्ता॑ देत्येति पु॒रस्ता॑ द॒ग्निना॒ ऽग्निना॑ पु॒रस्ता॑ देति । \newline
51. पु॒रस्ता॑ देत्येति पु॒रस्ता᳚त् पु॒रस्ता॑ देति॒ रक्ष॑साꣳ॒॒ रक्ष॑सा मेति पु॒रस्ता᳚त् पु॒रस्ता॑ देति॒ रक्ष॑साम् । \newline
52. ए॒ति॒ रक्ष॑साꣳ॒॒ रक्ष॑सा मेत्येति॒ रक्ष॑सा॒ मप॑हत्या॒ अप॑हत्यै॒ रक्ष॑सा मेत्येति॒ रक्ष॑सा॒ मप॑हत्यै । \newline
53. रक्ष॑सा॒ मप॑हत्या॒ अप॑हत्यै॒ रक्ष॑साꣳ॒॒ रक्ष॑सा॒ मप॑हत्यै पृथि॒व्याः पृ॑थि॒व्या अप॑हत्यै॒ रक्ष॑साꣳ॒॒ रक्ष॑सा॒ मप॑हत्यै पृथि॒व्याः । \newline
54. अप॑हत्यै पृथि॒व्याः पृ॑थि॒व्या अप॑हत्या॒ अप॑हत्यै पृथि॒व्याः सं॒पृचः॑ सं॒पृचः॑ पृथि॒व्या अप॑हत्या॒ अप॑हत्यै पृथि॒व्याः सं॒पृचः॑ । \newline
55. अप॑हत्या॒ इत्यप॑ - ह॒त्यै॒ । \newline
56. पृ॒थि॒व्याः सं॒पृचः॑ सं॒पृचः॑ पृथि॒व्याः पृ॑थि॒व्याः सं॒पृचः॑ पाहि पाहि सं॒पृचः॑ पृथि॒व्याः पृ॑थि॒व्याः सं॒पृचः॑ पाहि । \newline
57. सं॒पृचः॑ पाहि पाहि सं॒पृचः॑ सं॒पृचः॑ पा॒ही तीति॑ पाहि सं॒पृचः॑ सं॒पृचः॑ पा॒हीति॑ । \newline
58. सं॒पृच॒ इति॑ सं - पृचः॑ । \newline
59. पा॒ही तीति॑ पाहि पा॒हीति॑ ब॒र्॒.हिर् ब॒र्॒.हिरिति॑ पाहि पा॒हीति॑ ब॒र्॒.हिः । \newline
60. इति॑ ब॒र्॒.हिर् ब॒र्॒.हि रितीति॑ ब॒र्॒.हि रुपोप॑ ब॒र्॒.हि रितीति॑ ब॒र्॒.हि रुप॑ । \newline
61. ब॒र्॒.हि रुपोप॑ ब॒र्॒.हिर् ब॒र्॒.हि रुपा᳚स्य त्यस्य॒ त्युप॑ ब॒र्॒.हिर् ब॒र्॒.हि रुपा᳚स्यति । \newline
\pagebreak
\markright{ TS 6.3.8.3  \hfill https://www.vedavms.in \hfill}

\section{ TS 6.3.8.3 }

\textbf{TS 6.3.8.3 } \newline
\textbf{Samhita Paata} \newline

-रुपा᳚स्य॒त्य-स्क॑न्दा॒या-स्क॑न्नꣳ॒॒ हि तद् यद् ब॒र्॒.हिषि॒ स्कन्द॒त्यथो॑ बर्.हि॒षद॑मे॒वैनं॑ करोति॒ परा॒ङा व॑र्ततेऽद्ध्व॒र्युः प॒शोः स᳚ज्ञ्ं॒प्यमा॑नात् प॒शुभ्य॑ ए॒व तन्नि ह्नु॑त आ॒त्मनोऽना᳚व्रस्काय॒ गच्छ॑ति॒ श्रियं॒ प्र प॒शूना᳚प्नोति॒ य ए॒वं ॅवेद॑ प॒श्चाल्लो॑का॒ वा ए॒षा प्राच्यु॒दानी॑यते॒ यत् पत्नी॒ नम॑स्त आता॒नेत्या॑हाऽऽ*दि॒त्यस्य॒ वै र॒श्मय॑- [  ] \newline

\textbf{Pada Paata} \newline

उपेति॑ । अ॒स्य॒ति॒ । अस्क॑न्दाय । अस्क॑न्नम् । हि । तत् । यत् । ब॒र्॒.हिषि॑ । स्कन्द॑ति । अथो॒ इति॑ । ब॒र्॒.हि॒षद॒मिति॑ बर्.हि - सद᳚म् । ए॒व । ए॒न॒म् । क॒रो॒ति॒ । पराङ्॑ । एति॑ । व॒र्त॒ते॒ । अ॒द्ध्व॒र्युः । प॒शोः । स॒ज्ञ्ं॒प्यमा॑ना॒दिति॑ सं - ज्ञ्॒प्यमा॑नात् । प॒शुभ्य॒ इति॑ प॒शु - भ्यः॒ । ए॒व । तत् । नीति॑ । ह्नु॒ते॒ । आ॒त्मनः॑ । अना᳚व्रस्का॒येत्यना᳚-व्र॒स्का॒य॒ । गच्छ॑ति । श्रिय᳚म् । प्रेति॑ । प॒शून् । आ॒प्नो॒ति॒ । यः । ए॒वम् । वेद॑ । प॒श्चाल्लो॒केति॑ प॒श्चात् - लो॒का॒ । वै । ए॒षा । प्राची᳚ । उ॒दानी॑यत॒ इत्यु॑त् - आनी॑यते । यत् । पत्नी᳚ । नमः॑ । ते॒ । आ॒ता॒नेत्या᳚ - ता॒न॒ । इति॑ । आ॒ह॒ । आ॒दि॒त्यस्य॑ । वै । र॒श्मयः॑ ।  \newline


\textbf{Krama Paata} \newline

उपा᳚स्यति । अ॒स्य॒त्यस्क॑न्दाय । अस्क॑न्दा॒यास्क॑न्नम् । अस्क॑न्नꣳ॒॒ हि । हि तत् । तद् यत् । यद् ब॒र्.॒हिषि॑ । ब॒र्.॒हिषि॒ स्कन्द॑ति । स्कन्द॒त्यथो᳚ । अथो॑ बर्.हि॒षद᳚म् । अथो॒ इत्यथो᳚ । ब॒र्.॒हि॒षद॑मे॒व । ब॒र्.॒हि॒षद॒मिति॑ बर्.हि - सद᳚म् । ए॒वैन᳚म् । ए॒न॒म् क॒रो॒ति॒ । क॒रो॒ति॒ पराङ्॑ । परा॒ङा । आ व॑र्तते । व॒र्त॒ते॒ऽद्ध्व॒र्युः । अ॒द्ध्व॒र्युः प॒शोः । प॒शोः स᳚म्(2)ज्ञ्॒प्यमा॑नात् । स॒म्(2)ज्ञ्॒प्यमा॑नात् प॒शुभ्यः॑ । स॒म्(2)ज्ञ्॒प्यमा॑ना॒दिति॑ सम् - ज्ञ्॒प्यमा॑नात् । प॒शुभ्य॑ ए॒व । प॒शुभ्य॒ इति॑ प॒शु - भ्यः॒ । ए॒व तत् । तन् नि । नि ह्नु॑ते । ह्नु॒त॒ आ॒त्मनः॑ । आ॒त्मनोऽना᳚व्रस्काय । अना᳚व्रस्काय॒ गच्छ॑ति । अना᳚व्रस्का॒येत्यना᳚ - व्र॒स्का॒य॒ । गच्छ॑ति॒ श्रिय᳚म् । श्रिय॒म् प्र । प्र प॒शून् । प॒शूना᳚प्नोति । आ॒प्नो॒ति॒ यः । य ए॒वम् । ए॒वम् ॅवेद॑ । वेद॑ प॒श्चाल्लो॑का । प॒श्चाल्लो॑का॒ वै । प॒श्चाल्लो॒केति॑ प॒श्चात् - लो॒का॒ । वा ए॒षा । ए॒षा प्राची᳚ । प्राच्यु॒दानी॑यते । उ॒दानी॑यते॒ यत् । उ॒दानी॑यत॒ इत्यु॑त् - आनी॑यते । यत् पत्नी᳚ । पत्नी॒ नमः॑ । नम॑स्ते । त॒ आ॒ता॒न । आ॒ता॒नेति॑ । आ॒ता॒नेत्या᳚ - ता॒न । इत्या॑ह । आ॒हा॒दि॒त्यस्य॑ । आ॒दि॒त्यस्य॒ वै । वै र॒श्मयः॑ ( ) । र॒श्मय॑ आता॒नाः \newline

\textbf{Jatai Paata} \newline

1. उपा᳚स्य त्यस्य॒ त्युपोपा᳚ स्यति । \newline
2. अ॒स्य॒ त्यस्क॑न्दा॒या स्क॑न्दाया स्य त्यस्य॒ त्यस्क॑न्दाय । \newline
3. अस्क॑न्दा॒या स्क॑न्न॒ मस्क॑न्न॒ मस्क॑न्दा॒या स्क॑न्दा॒या स्क॑न्नम् । \newline
4. अस्क॑न्नꣳ॒॒ हि ह्यस्क॑न्न॒ मस्क॑न्नꣳ॒॒ हि । \newline
5. हि तत् तद्धि हि तत् । \newline
6. तद् यद् यत् तत् तद् यत् । \newline
7. यद् ब॒र्॒.हिषि॑ ब॒र्॒.हिषि॒ यद् यद् ब॒र्॒.हिषि॑ । \newline
8. ब॒र्॒.हिषि॒ स्कन्द॑ति॒ स्कन्द॑ति ब॒र्॒.हिषि॑ ब॒र्॒.हिषि॒ स्कन्द॑ति । \newline
9. स्कन्द॒ त्यथो॒ अथो॒ स्कन्द॑ति॒ स्कन्द॒ त्यथो᳚ । \newline
10. अथो॑ बर्.हि॒षद॑म् बर्.हि॒षद॒ मथो॒ अथो॑ बर्.हि॒षद᳚म् । \newline
11. अथो॒ इत्यथो᳚ । \newline
12. ब॒र्॒.हि॒षद॑ मे॒वैव ब॑र्.हि॒षद॑म् बर्.हि॒षद॑ मे॒व । \newline
13. ब॒र्॒.हि॒षद॒मिति॑ बर्.हि - सद᳚म् । \newline
14. ए॒वैन॑ मेन मे॒वै वैन᳚म् । \newline
15. ए॒न॒म् क॒रो॒ति॒ क॒रो॒ त्ये॒न॒ मे॒न॒म् क॒रो॒ति॒ । \newline
16. क॒रो॒ति॒ परा॒ङ् परा᳚ङ् करोति करोति॒ पराङ्॑ । \newline
17. परा॒ङ् आ परा॒ङ् परा॒ङ् आ । \newline
18. आ व॑र्तते वर्तत॒ आ व॑र्तते । \newline
19. व॒र्त॒ते॒ ऽद्ध्व॒र्यु र॑द्ध्व॒र्युर् व॑र्तते वर्तते ऽद्ध्व॒र्युः । \newline
20. अ॒द्ध्व॒र्युः प॒शोः प॒शो र॑द्ध्व॒र्यु र॑द्ध्व॒र्युः प॒शोः । \newline
21. प॒शोः सं᳚.ज्ञ्॒प्यमा॑नाथ् सं.ज्ञ्॒प्यमा॑नात् प॒शोः प॒शोः सं᳚.ज्ञ्॒प्यमा॑नात् । \newline
22. सं॒.ज्ञ्॒प्यमा॑नात् प॒शुभ्यः॑ प॒शुभ्यः॑ सं.ज्ञ्॒प्यमा॑नाथ् सं.ज्ञ्॒प्यमा॑नात् प॒शुभ्यः॑ । \newline
23. सं॒.ज्ञ्॒प्यमा॑ना॒दिति॑ सं. - ज्ञ्॒प्यमा॑नात् । \newline
24. प॒शुभ्य॑ ए॒वैव प॒शुभ्यः॑ प॒शुभ्य॑ ए॒व । \newline
25. प॒शुभ्य॒ इति॑ प॒शु - भ्यः॒ । \newline
26. ए॒व तत् तदे॒ वैव तत् । \newline
27. तन् नि नि तत् तन् नि । \newline
28. नि ह्नु॑ते ह्नुते॒ नि नि ह्नु॑ते । \newline
29. ह्नु॒त॒ आ॒त्मन॑ आ॒त्मनो᳚ ह्नुते ह्नुत आ॒त्मनः॑ । \newline
30. आ॒त्मनो ऽना᳚व्रस्का॒या ना᳚व्रस्काया॒ त्मन॑ आ॒त्मनो ऽना᳚व्रस्काय । \newline
31. अना᳚व्रस्काय॒ गच्छ॑ति॒ गच्छ॒ त्यना᳚व्रस्का॒या ना᳚व्रस्काय॒ गच्छ॑ति । \newline
32. अना᳚व्रस्का॒येत्यना᳚ - व्र॒स्का॒य॒ । \newline
33. गच्छ॑ति॒ श्रियꣳ॒॒ श्रिय॒म् गच्छ॑ति॒ गच्छ॑ति॒ श्रिय᳚म् । \newline
34. श्रिय॒म् प्र प्र श्रियꣳ॒॒ श्रिय॒म् प्र । \newline
35. प्र प॒शून् प॒शून् प्र प्र प॒शून् । \newline
36. प॒शूना᳚प्नो त्याप्नोति प॒शून् प॒शूना᳚प्नोति । \newline
37. आ॒प्नो॒ति॒ यो य आ᳚प्नो त्याप्नोति॒ यः । \newline
38. य ए॒व मे॒वं ॅयो य ए॒वम् । \newline
39. ए॒वं ॅवेद॒ वेदै॒व मे॒वं ॅवेद॑ । \newline
40. वेद॑ प॒श्चाल्लो॑का प॒श्चाल्लो॑का॒ वेद॒ वेद॑ प॒श्चाल्लो॑का । \newline
41. प॒श्चाल्लो॑का॒ वै वै प॒श्चाल्लो॑का प॒श्चाल्लो॑का॒ वै । \newline
42. प॒श्चाल्लो॒केति॑ प॒श्चात् - लो॒का॒ । \newline
43. वा ए॒षैषा वै वा ए॒षा । \newline
44. ए॒षा प्राची॒ प्राच्ये॒षैषा प्राची᳚ । \newline
45. प्राच्यु॒दानी॑यत उ॒दानी॑यते॒ प्राची॒ प्राच्यु॒दानी॑यते । \newline
46. उ॒दानी॑यते॒ यद् यदु॒दानी॑यत उ॒दानी॑यते॒ यत् । \newline
47. उ॒दानी॑यत॒ इत्यु॑त् - आनी॑यते । \newline
48. यत् पत्नी॒ पत्नी॒ यद् यत् पत्नी᳚ । \newline
49. पत्नी॒ नमो॒ नमः॒ पत्नी॒ पत्नी॒ नमः॑ । \newline
50. नम॑स्ते ते॒ नमो॒ नम॑स्ते । \newline
51. त॒ आ॒ता॒ना॒ ता॒न॒ ते॒ त॒ आ॒ता॒न॒ । \newline
52. आ॒ता॒ने तीत्या॑ता नाता॒ नेति॑ । \newline
53. आ॒ता॒नेत्या᳚ - ता॒न॒ । \newline
54. इत्या॑हा॒हे तीत्या॑ह । \newline
55. आ॒हा॒ दि॒त्यस्या॑ दि॒त्यस्या॑ हाहा दि॒त्यस्य॑ । \newline
56. आ॒दि॒त्यस्य॒ वै वा आ॑दि॒त्यस्या॑ दि॒त्यस्य॒ वै । \newline
57. वै र॒श्मयो॑ र॒श्मयो॒ वै वै र॒श्मयः॑ । \newline
58. र॒श्मय॑ आता॒ना आ॑ता॒ना र॒श्मयो॑ र॒श्मय॑ आता॒नाः । \newline

\textbf{Ghana Paata } \newline

1. उपा᳚स्य त्यस्य॒ त्युपोपा᳚स्य॒ त्यस्क॑न्दा॒या स्क॑न्दाया स्य॒त्युपोपा᳚स्य॒ त्यस्क॑न्दाय । \newline
2. अ॒स्य॒ त्यस्क॑न्दा॒या स्क॑न्दा यास्य त्यस्य॒ त्यस्क॑न्दा॒या स्क॑न्न॒ मस्क॑न्न॒ मस्क॑न्दा यास्य त्यस्य॒ त्यस्क॑न्दा॒या स्क॑न्नम् । \newline
3. अस्क॑न्दा॒या स्क॑न्न॒ मस्क॑न्न॒ मस्क॑न्दा॒या स्क॑न्दा॒या स्क॑न्नꣳ॒॒ हि ह्यस्क॑न्न॒ मस्क॑न्दा॒या स्क॑न्दा॒या स्क॑न्नꣳ॒॒ हि । \newline
4. अस्क॑न्नꣳ॒॒ हि ह्यस्क॑न्न॒ मस्क॑न्नꣳ॒॒ हि तत् तद्ध्यस्क॑न्न॒ मस्क॑न्नꣳ॒॒ हि तत् । \newline
5. हि तत् तद्धि हि तद् यद् यत् तद्धि हि तद् यत् । \newline
6. तद् यद् यत् तत् तद् यद् ब॒र्॒.हिषि॑ ब॒र्॒.हिषि॒ यत् तत् तद् यद् ब॒र्॒.हिषि॑ । \newline
7. यद् ब॒र्॒.हिषि॑ ब॒र्॒.हिषि॒ यद् यद् ब॒र्॒.हिषि॒ स्कन्द॑ति॒ स्कन्द॑ति ब॒र्॒.हिषि॒ यद् यद् ब॒र्॒.हिषि॒ स्कन्द॑ति । \newline
8. ब॒र्॒.हिषि॒ स्कन्द॑ति॒ स्कन्द॑ति ब॒र्॒.हिषि॑ ब॒र्॒.हिषि॒ स्कन्द॒ त्यथो॒ अथो॒ स्कन्द॑ति ब॒र्॒.हिषि॑ ब॒र्॒.हिषि॒ स्कन्द॒ त्यथो᳚ । \newline
9. स्कन्द॒ त्यथो॒ अथो॒ स्कन्द॑ति॒ स्कन्द॒ त्यथो॑ बर्.हि॒षद॑म् बर्.हि॒षद॒ मथो॒ स्कन्द॑ति॒ स्कन्द॒ त्यथो॑ बर्.हि॒षद᳚म् । \newline
10. अथो॑ बर्.हि॒षद॑म् बर्.हि॒षद॒ मथो॒ अथो॑ बर्.हि॒षद॑ मे॒वैव ब॑र्.हि॒षद॒ मथो॒ अथो॑ बर्.हि॒षद॑ मे॒व । \newline
11. अथो॒ इत्यथो᳚ । \newline
12. ब॒र्॒.हि॒षद॑ मे॒वैव ब॑र्.हि॒षद॑म् बर्.हि॒षद॑ मे॒वैन॑ मेन मे॒व ब॑र्.हि॒षद॑म् बर्.हि॒षद॑ मे॒वैन᳚म् । \newline
13. ब॒र्॒.हि॒षद॒मिति॑ बर्.हि - सद᳚म् । \newline
14. ए॒वैन॑ मेन मे॒वै वैन॑म् करोति करो त्येन मे॒वै वैन॑म् करोति । \newline
15. ए॒न॒म् क॒रो॒ति॒ क॒रो॒ त्ये॒न॒ मे॒न॒म् क॒रो॒ति॒ परा॒ङ् परा᳚ङ् करो त्येन मेनम् करोति॒ पराङ्॑ । \newline
16. क॒रो॒ति॒ परा॒ङ् परा᳚ङ् करोति करोति॒ परा॒ ङा परा᳚ङ् करोति करोति॒ परा॒ ङा । \newline
17. परा॒ ङा परा॒ङ् परा॒ ङा व॑र्तते वर्तत॒ आ परा॒ङ् परा॒ ङा व॑र्तते । \newline
18. आ व॑र्तते वर्तत॒ आ व॑र्तते ऽद्ध्व॒र्यु र॑द्ध्व॒र्युर् व॑र्तत॒ आ व॑र्तते ऽद्ध्व॒र्युः । \newline
19. व॒र्त॒ते॒ ऽद्ध्व॒र्यु र॑द्ध्व॒र्युर् व॑र्तते वर्तते ऽद्ध्व॒र्युः प॒शोः प॒शो र॑द्ध्व॒र्युर् व॑र्तते वर्तते ऽद्ध्व॒र्युः प॒शोः । \newline
20. अ॒द्ध्व॒र्युः प॒शोः प॒शो र॑द्ध्व॒र्यु र॑द्ध्व॒र्युः प॒शोः सं᳚.ज्ञ्॒प्यमा॑नाथ् सं.ज्ञ्॒प्यमा॑नात् प॒शो र॑द्ध्व॒र्यु र॑द्ध्व॒र्युः प॒शोः सं᳚.ज्ञ्॒प्यमा॑नात् । \newline
21. प॒शोः सं᳚.ज्ञ्॒प्यमा॑नाथ् सं.ज्ञ्॒प्यमा॑नात् प॒शोः प॒शोः सं᳚.ज्ञ्॒प्यमा॑नात् प॒शुभ्यः॑ प॒शुभ्यः॑ सं.ज्ञ्॒प्यमा॑नात् प॒शोः प॒शोः सं᳚.ज्ञ्॒प्यमा॑नात् प॒शुभ्यः॑ । \newline
22. सं॒.ज्ञ्॒प्यमा॑नात् प॒शुभ्यः॑ प॒शुभ्यः॑ सं.ज्ञ्॒प्यमा॑नाथ् सं.ज्ञ्॒प्यमा॑नात् प॒शुभ्य॑ ए॒वैव प॒शुभ्यः॑ सं.ज्ञ्॒प्यमा॑नाथ् सं.ज्ञ्॒प्यमा॑नात् प॒शुभ्य॑ ए॒व । \newline
23. सं॒.ज्ञ्॒प्यमा॑ना॒दिति॑ सं. - ज्ञ्॒प्यमा॑नात् । \newline
24. प॒शुभ्य॑ ए॒वैव प॒शुभ्यः॑ प॒शुभ्य॑ ए॒व तत् तदे॒व प॒शुभ्यः॑ प॒शुभ्य॑ ए॒व तत् । \newline
25. प॒शुभ्य॒ इति॑ प॒शु - भ्यः॒ । \newline
26. ए॒व तत् तदे॒ वैव तन् नि नि तदे॒ वैव तन् नि । \newline
27. तन् नि नि तत् तन् नि ह्नु॑ते ह्नुते॒ नि तत् तन् नि ह्नु॑ते । \newline
28. नि ह्नु॑ते ह्नुते॒ नि नि ह्नु॑त आ॒त्मन॑ आ॒त्मनो᳚ ह्नुते॒ नि नि ह्नु॑त आ॒त्मनः॑ । \newline
29. ह्नु॒त॒ आ॒त्मन॑ आ॒त्मनो᳚ ह्नुते ह्नुत आ॒त्मनो ऽना᳚व्रस्का॒या ना᳚व्रस्काया॒ त्मनो᳚ ह्नुते ह्नुत आ॒त्मनो ऽना᳚व्रस्काय । \newline
30. आ॒त्मनो ऽना᳚व्रस्का॒या ना᳚व्रस्का या॒त्मन॑ आ॒त्मनो ऽना᳚व्रस्काय॒ गच्छ॑ति॒ गच्छ॒ त्यना᳚व्रस्का या॒त्मन॑ आ॒त्मनो ऽना᳚व्रस्काय॒ गच्छ॑ति । \newline
31. अना᳚व्रस्काय॒ गच्छ॑ति॒ गच्छ॒ त्यना᳚व्रस्का॒या ना᳚व्रस्काय॒ गच्छ॑ति॒ श्रियꣳ॒॒ श्रिय॒म् गच्छ॒ त्यना᳚व्रस्का॒या ना᳚व्रस्काय॒ गच्छ॑ति॒ श्रिय᳚म् । \newline
32. अना᳚व्रस्का॒येत्यना᳚ - व्र॒स्का॒य॒ । \newline
33. गच्छ॑ति॒ श्रियꣳ॒॒ श्रिय॒म् गच्छ॑ति॒ गच्छ॑ति॒ श्रिय॒म् प्र प्र श्रिय॒म् गच्छ॑ति॒ गच्छ॑ति॒ श्रिय॒म् प्र । \newline
34. श्रिय॒म् प्र प्र श्रियꣳ॒॒ श्रिय॒म् प्र प॒शून् प॒शून् प्र श्रियꣳ॒॒ श्रिय॒म् प्र प॒शून् । \newline
35. प्र प॒शून् प॒शून् प्र प्र प॒शूना᳚प्नो त्याप्नोति प॒शून् प्र प्र प॒शूना᳚प्नोति । \newline
36. प॒शूना᳚प्नो त्याप्नोति प॒शून् प॒शूना᳚प्नोति॒ यो य आ᳚प्नोति प॒शून् प॒शूना᳚प्नोति॒ यः । \newline
37. आ॒प्नो॒ति॒ यो य आ᳚प्नो त्याप्नोति॒ य ए॒व मे॒वं ॅय आ᳚प्नो त्याप्नोति॒ य ए॒वम् । \newline
38. य ए॒व मे॒वं ॅयो य ए॒वं ॅवेद॒ वेदै॒वं ॅयो य ए॒वं ॅवेद॑ । \newline
39. ए॒वं ॅवेद॒ वेदै॒व मे॒वं ॅवेद॑ प॒श्चाल्लो॑का प॒श्चाल्लो॑का॒ वेदै॒व मे॒वं ॅवेद॑ प॒श्चाल्लो॑का । \newline
40. वेद॑ प॒श्चाल्लो॑का प॒श्चाल्लो॑का॒ वेद॒ वेद॑ प॒श्चाल्लो॑का॒ वै वै प॒श्चाल्लो॑का॒ वेद॒ वेद॑ प॒श्चाल्लो॑का॒ वै । \newline
41. प॒श्चाल्लो॑का॒ वै वै प॒श्चाल्लो॑का प॒श्चाल्लो॑का॒ वा ए॒षैषा वै प॒श्चाल्लो॑का प॒श्चाल्लो॑का॒ वा ए॒षा । \newline
42. प॒श्चाल्लो॒केति॑ प॒श्चात् - लो॒का॒ । \newline
43. वा ए॒षैषा वै वा ए॒षा प्राची॒ प्राच्ये॒षा वै वा ए॒षा प्राची᳚ । \newline
44. ए॒षा प्राची॒ प्राच्ये॒ षैषा प्राच्यु॒दानी॑यत उ॒दानी॑यते॒ प्राच्ये॒ षैषा प्राच्यु॒दानी॑यते । \newline
45. प्राच्यु॒दानी॑यत उ॒दानी॑यते॒ प्राची॒ प्राच्यु॒दानी॑यते॒ यद् यदु॒दानी॑यते॒ प्राची॒ प्राच्यु॒दानी॑यते॒ यत् । \newline
46. उ॒दानी॑यते॒ यद् यदु॒दानी॑यत उ॒दानी॑यते॒ यत् पत्नी॒ पत्नी॒ यदु॒दानी॑यत उ॒दानी॑यते॒ यत् पत्नी᳚ । \newline
47. उ॒दानी॑यत॒ इत्यु॑त् - आनी॑यते । \newline
48. यत् पत्नी॒ पत्नी॒ यद् यत् पत्नी॒ नमो॒ नमः॒ पत्नी॒ यद् यत् पत्नी॒ नमः॑ । \newline
49. पत्नी॒ नमो॒ नमः॒ पत्नी॒ पत्नी॒ नम॑ स्ते ते॒ नमः॒ पत्नी॒ पत्नी॒ नम॑ स्ते । \newline
50. नम॑ स्ते ते॒ नमो॒ नम॑ स्त आताना तान ते॒ नमो॒ नम॑ स्त आतान । \newline
51. त॒ आ॒ता॒ना॒ ता॒न॒ ते॒ त॒ आ॒ता॒ने तीत्या॑तान ते त आता॒नेति॑ । \newline
52. आ॒ता॒ने तीत्या॑ताना ता॒ने त्या॑हा॒हे त्या॑ताना ता॒ने त्या॑ह । \newline
53. आ॒ता॒नेत्या᳚ - ता॒न॒ । \newline
54. इत्या॑हा॒हे तीत्या॑हादि॒त्यस्या॑ दि॒त्यस्या॒ हेतीत्या॑ हादि॒त्यस्य॑ । \newline
55. आ॒हा॒दि॒त्यस्या॑ दि॒त्यस्या॑ हाहादि॒त्यस्य॒ वै वा आ॑दि॒त्यस्या॑ हाहादि॒त्यस्य॒ वै । \newline
56. आ॒दि॒त्यस्य॒ वै वा आ॑दि॒त्यस्या॑ दि॒त्यस्य॒ वै र॒श्मयो॑ र॒श्मयो॒ वा आ॑दि॒त्यस्या॑ दि॒त्यस्य॒ वै र॒श्मयः॑ । \newline
57. वै र॒श्मयो॑ र॒श्मयो॒ वै वै र॒श्मय॑ आता॒ना आ॑ता॒ना र॒श्मयो॒ वै वै र॒श्मय॑ आता॒नाः । \newline
58. र॒श्मय॑ आता॒ना आ॑ता॒ना र॒श्मयो॑ र॒श्मय॑ आता॒ना स्तेभ्य॒ स्तेभ्य॑ आता॒ना र॒श्मयो॑ र॒श्मय॑ आता॒ना स्तेभ्यः॑ । \newline
\pagebreak
\markright{ TS 6.3.8.4  \hfill https://www.vedavms.in \hfill}

\section{ TS 6.3.8.4 }

\textbf{TS 6.3.8.4 } \newline
\textbf{Samhita Paata} \newline

आता॒नास्तेभ्य॑ ए॒व नम॑स्करोत्यन॒र्वा प्रेहीत्या॑ह॒ भ्रातृ॑व्यो॒ वा अर्वा॒ भ्रातृ॑व्यापनुत्त्यै घृ॒तस्य॑ कु॒ल्यामनु॑ स॒ह प्र॒जया॑ स॒ह रा॒यस्पोषे॒णे-त्या॑हा॒ ऽऽ*शिष॑मे॒वैतामा शा᳚स्त॒ आपो॑ देवीः शुद्धायुव॒ इत्या॑ह यथाय॒जुरे॒वैतत् ॥ \newline

\textbf{Pada Paata} \newline

आ॒ता॒ना इत्या᳚ - ता॒नाः । तेभ्यः॑ । ए॒व । नमः॑ । क॒रो॒ति॒ । अ॒न॒र्वा । प्रेति॑ । इ॒हि॒ । इति॑ । आ॒ह॒ । भ्रातृ॑व्यः । वै । अर्वा᳚ । भ्रातृ॑व्यापनुत्त्या॒ इति॒ भ्रातृ॑व्य - अ॒प॒नु॒त्त्यै॒ । घृ॒तस्य॑ । कु॒ल्याम् । अन्विति॑ । स॒ह । प्र॒जयेति॑ प्र - जया᳚ । स॒ह । रा॒यः । पोषे॑ण । इति॑ । आ॒ह॒ । आ॒शिष॒मित्या᳚ - शिष᳚म् । ए॒व । ए॒ताम् । एति॑ । शा॒स्ते॒ । आपः॑ ।दे॒वीः॒ । शु॒द्धा॒यु॒व॒ इति॑ शुद्ध - यु॒वः॒ । इति॑ । आ॒ह॒ । य॒था॒य॒जुरिति॑ यथा - य॒जुः । ए॒व । ए॒तत् ॥  \newline


\textbf{Krama Paata} \newline

आ॒ता॒नास्तेभ्यः॑ । आ॒ता॒ना इत्या᳚ - ता॒नाः । तेभ्य॑ ए॒व । ए॒व नमः॑ । नम॑स्करोति । क॒रो॒त्य॒न॒र्वा । अ॒न॒र्वा प्र । प्रेहि॑ । इ॒हीति॑ । इत्या॑ह । आ॒ह॒ भ्रातृ॑व्यः । भ्रातृ॑व्यो॒ वै । वा अर्वा᳚ । अर्वा॒ भ्रातृ॑व्यापनुत्त्यै । भ्रातृ॑व्यापनुत्त्यै घृ॒तस्य॑ । भ्रातृ॑व्यापनुत्त्या॒ इति॒ भ्रातृ॑व्य - अ॒प॒नु॒त्त्यै॒ । घृ॒तस्य॑ कु॒ल्याम् । कु॒ल्यामनु॑ । अनु॑ स॒ह । स॒ह प्र॒जया᳚ । प्र॒जया॑ स॒ह । प्र॒जयेति॑ प्र - जया᳚ । स॒ह रा॒यः । रा॒यस्पोषे॑ण । पोषे॒णेति॑ । इत्या॑ह । आ॒हा॒शिष᳚म् । आ॒शिष॑मे॒व । आ॒शिष॒मित्या᳚ - शिष᳚म् । ए॒वैताम् । ए॒तामा । आ शा᳚स्ते । शा॒स्त॒ आपः॑ । आपो॑ देवीः । दे॒वीः॒ शु॒द्धा॒यु॒वः॒ । शु॒द्धा॒यु॒व॒ इति॑ । शु॒द्धा॒यु॒व॒ इति॑ शुद्ध - यु॒वः॒ । इत्या॑ह । आ॒ह॒ य॒था॒य॒जुः । य॒था॒य॒जुरे॒व । य॒था॒य॒जुरिति॑ यथा - य॒जुः । ए॒वैतत् । ए॒तदित्ये॒तत् । \newline

\textbf{Jatai Paata} \newline

1. आ॒ता॒ना स्तेभ्य॒ स्तेभ्य॑ आता॒ना आ॑ता॒ना स्तेभ्यः॑ । \newline
2. आ॒ता॒ना इत्या᳚ - ता॒नाः । \newline
3. तेभ्य॑ ए॒वैव तेभ्य॒ स्तेभ्य॑ ए॒व । \newline
4. ए॒व नमो॒ नम॑ ए॒वैव नमः॑ । \newline
5. नम॑ स्करोति करोति॒ नमो॒ नम॑ स्करोति । \newline
6. क॒रो॒ त्य॒न॒र्वा ऽन॒र्वा क॑रोति करो त्यन॒र्वा । \newline
7. अ॒न॒र्वा प्र प्राण॒र्वा ऽन॒र्वा प्र । \newline
8. प्रेही॑हि॒ प्र प्रेहि॑ । \newline
9. इ॒हीतीती॑ ही॒हीति॑ । \newline
10. इत्या॑हा॒हे तीत्या॑ह । \newline
11. आ॒ह॒ भ्रातृ॑व्यो॒ भ्रातृ॑व्य आहाह॒ भ्रातृ॑व्यः । \newline
12. भ्रातृ॑व्यो॒ वै वै भ्रातृ॑व्यो॒ भ्रातृ॑व्यो॒ वै । \newline
13. वा अर्वा ऽर्वा॒ वै वा अर्वा᳚ । \newline
14. अर्वा॒ भ्रातृ॑व्यापनुत्त्यै॒ भ्रातृ॑व्यापनुत्त्या॒ अर्वा ऽर्वा॒ भ्रातृ॑व्यापनुत्त्यै । \newline
15. भ्रातृ॑व्यापनुत्त्यै घृ॒तस्य॑ घृ॒तस्य॒ भ्रातृ॑व्यापनुत्त्यै॒ भ्रातृ॑व्यापनुत्त्यै घृ॒तस्य॑ । \newline
16. भ्रातृ॑व्यापनुत्त्या॒ इति॒ भ्रातृ॑व्य - अ॒प॒नु॒त्त्यै॒ । \newline
17. घृ॒तस्य॑ कु॒ल्याम् कु॒ल्याम् घृ॒तस्य॑ घृ॒तस्य॑ कु॒ल्याम् । \newline
18. कु॒ल्या मन्वनु॑ कु॒ल्याम् कु॒ल्या मनु॑ । \newline
19. अनु॑ स॒ह स॒हान् वनु॑ स॒ह । \newline
20. स॒ह प्र॒जया᳚ प्र॒जया॑ स॒ह स॒ह प्र॒जया᳚ । \newline
21. प्र॒जया॑ स॒ह स॒ह प्र॒जया᳚ प्र॒जया॑ स॒ह । \newline
22. प्र॒जयेति॑ प्र - जया᳚ । \newline
23. स॒ह रा॒यो रा॒यः स॒ह स॒ह रा॒यः । \newline
24. रा॒य स्पोषे॑ण॒ पोषे॑ण रा॒यो रा॒य स्पोषे॑ण । \newline
25. पोषे॒णे तीति॒ पोषे॑ण॒ पोषे॒णेति॑ । \newline
26. इत्या॑हा॒हे तीत्या॑ह । \newline
27. आ॒हा॒शिष॑ मा॒शिष॑ माहा हा॒शिष᳚म् । \newline
28. आ॒शिष॑ मे॒वै वाशिष॑ मा॒शिष॑ मे॒व । \newline
29. आ॒शिष॒मित्या᳚ - शिष᳚म् । \newline
30. ए॒वैता मे॒ता मे॒वै वैताम् । \newline
31. ए॒ता मैता मे॒ता मा । \newline
32. आ शा᳚स्ते शास्त॒ आ शा᳚स्ते । \newline
33. शा॒स्त॒ आप॒ आपः॑ शास्ते शास्त॒ आपः॑ । \newline
34. आपो॑ देवीर् देवी॒ राप॒ आपो॑ देवीः । \newline
35. दे॒वीः॒ शु॒द्धा॒यु॒वः॒ शु॒द्धा॒यु॒वो॒ दे॒वी॒र् दे॒वीः॒ शु॒द्धा॒यु॒वः॒ । \newline
36. शु॒द्धा॒यु॒व॒ इतीति॑ शुद्धायुवः शुद्धायुव॒ इति॑ । \newline
37. शु॒द्धा॒यु॒व॒ इति॑ शुद्ध - यु॒वः॒ । \newline
38. इत्या॑हा॒हे तीत्या॑ह । \newline
39. आ॒ह॒ य॒था॒य॒जुर् य॑थाय॒जु रा॑हाह यथाय॒जुः । \newline
40. य॒था॒य॒जु रे॒वैव य॑थाय॒जुर् य॑थाय॒जु रे॒व । \newline
41. य॒था॒य॒जुरिति॑ यथा - य॒जुः । \newline
42. ए॒वैत दे॒त दे॒वै वैतत् । \newline
43. ए॒तदित्ये॒तत् । \newline

\textbf{Ghana Paata } \newline

1. आ॒ता॒ना स्तेभ्य॒ स्तेभ्य॑ आता॒ना आ॑ता॒ना स्तेभ्य॑ ए॒वैव तेभ्य॑ आता॒ना आ॑ता॒ना स्तेभ्य॑ ए॒व । \newline
2. आ॒ता॒ना इत्या᳚ - ता॒नाः । \newline
3. तेभ्य॑ ए॒वैव तेभ्य॒ स्तेभ्य॑ ए॒व नमो॒ नम॑ ए॒व तेभ्य॒ स्तेभ्य॑ ए॒व नमः॑ । \newline
4. ए॒व नमो॒ नम॑ ए॒वैव नम॑ स्करोति करोति॒ नम॑ ए॒वैव नम॑ स्करोति । \newline
5. नम॑ स्करोति करोति॒ नमो॒ नम॑ स्करो त्यन॒र्वा ऽन॒र्वा क॑रोति॒ नमो॒ नम॑ स्करो त्यन॒र्वा । \newline
6. क॒रो॒ त्य॒न॒र्वा ऽन॒र्वा क॑रोति करो त्यन॒र्वा प्र प्राण॒र्वा क॑रोति करो त्यन॒र्वा प्र । \newline
7. अ॒न॒र्वा प्र प्राण॒र्वा ऽन॒र्वा प्रेही॑हि॒ प्राण॒र्वा ऽन॒र्वा प्रेहि॑ । \newline
8. प्रेही॑हि॒ प्र प्रेहीती ती॑हि॒ प्र प्रेहीति॑ । \newline
9. इ॒हीती ती॑ही॒ हीत्या॑हा॒हे ती॑ही॒ हीत्या॑ह । \newline
10. इत्या॑हा॒हे तीत्या॑ह॒ भ्रातृ॑व्यो॒ भ्रातृ॑व्य आ॒हे तीत्या॑ह॒ भ्रातृ॑व्यः । \newline
11. आ॒ह॒ भ्रातृ॑व्यो॒ भ्रातृ॑व्य आहाह॒ भ्रातृ॑व्यो॒ वै वै भ्रातृ॑व्य आहाह॒ भ्रातृ॑व्यो॒ वै । \newline
12. भ्रातृ॑व्यो॒ वै वै भ्रातृ॑व्यो॒ भ्रातृ॑व्यो॒ वा अर्वा ऽर्वा॒ वै भ्रातृ॑व्यो॒ भ्रातृ॑व्यो॒ वा अर्वा᳚ । \newline
13. वा अर्वा ऽर्वा॒ वै वा अर्वा॒ भ्रातृ॑व्यापनुत्त्यै॒ भ्रातृ॑व्यापनुत्त्या॒ अर्वा॒ वै वा अर्वा॒ भ्रातृ॑व्यापनुत्त्यै । \newline
14. अर्वा॒ भ्रातृ॑व्यापनुत्त्यै॒ भ्रातृ॑व्यापनुत्त्या॒ अर्वा ऽर्वा॒ भ्रातृ॑व्यापनुत्त्यै घृ॒तस्य॑ घृ॒तस्य॒ भ्रातृ॑व्यापनुत्त्या॒ अर्वा ऽर्वा॒ भ्रातृ॑व्यापनुत्त्यै घृ॒तस्य॑ । \newline
15. भ्रातृ॑व्यापनुत्त्यै घृ॒तस्य॑ घृ॒तस्य॒ भ्रातृ॑व्यापनुत्त्यै॒ भ्रातृ॑व्यापनुत्त्यै घृ॒तस्य॑ कु॒ल्याम् कु॒ल्याम् घृ॒तस्य॒ भ्रातृ॑व्यापनुत्त्यै॒ भ्रातृ॑व्यापनुत्त्यै घृ॒तस्य॑ कु॒ल्याम् । \newline
16. भ्रातृ॑व्यापनुत्त्या॒ इति॒ भ्रातृ॑व्य - अ॒प॒नु॒त्त्यै॒ । \newline
17. घृ॒तस्य॑ कु॒ल्याम् कु॒ल्याम् घृ॒तस्य॑ घृ॒तस्य॑ कु॒ल्या मन्वनु॑ कु॒ल्याम् घृ॒तस्य॑ घृ॒तस्य॑ कु॒ल्या मनु॑ । \newline
18. कु॒ल्या मन्वनु॑ कु॒ल्याम् कु॒ल्या मनु॑ स॒ह स॒हानु॑ कु॒ल्याम् कु॒ल्या मनु॑ स॒ह । \newline
19. अनु॑ स॒ह स॒हान्वनु॑ स॒ह प्र॒जया᳚ प्र॒जया॑ स॒हान्वनु॑ स॒ह प्र॒जया᳚ । \newline
20. स॒ह प्र॒जया᳚ प्र॒जया॑ स॒ह स॒ह प्र॒जया॑ स॒ह स॒ह प्र॒जया॑ स॒ह स॒ह प्र॒जया॑ स॒ह । \newline
21. प्र॒जया॑ स॒ह स॒ह प्र॒जया᳚ प्र॒जया॑ स॒ह रा॒यो रा॒यः स॒ह प्र॒जया᳚ प्र॒जया॑ स॒ह रा॒यः । \newline
22. प्र॒जयेति॑ प्र - जया᳚ । \newline
23. स॒ह रा॒यो रा॒यः स॒ह स॒ह रा॒य स्पोषे॑ण॒ पोषे॑ण रा॒यः स॒ह स॒ह रा॒य स्पोषे॑ण । \newline
24. रा॒य स्पोषे॑ण॒ पोषे॑ण रा॒यो रा॒य स्पोषे॒णे तीति॒ पोषे॑ण रा॒यो रा॒य स्पोषे॒णेति॑ । \newline
25. पोषे॒णे तीति॒ पोषे॑ण॒ पोषे॒णे त्या॑हा॒ हेति॒ पोषे॑ण॒ पोषे॒णे त्या॑ह । \newline
26. इत्या॑हा॒हे तीत्या॑ हा॒शिष॑ मा॒शिष॑ मा॒हे तीत्या॑ हा॒शिष᳚म् । \newline
27. आ॒हा॒शिष॑ मा॒शिष॑ माहा हा॒शिष॑ मे॒वै वाशिष॑ माहा हा॒शिष॑ मे॒व । \newline
28. आ॒शिष॑ मे॒वै वाशिष॑ मा॒शिष॑ मे॒वैता मे॒ता मे॒वाशिष॑ मा॒शिष॑ मे॒वैताम् । \newline
29. आ॒शिष॒मित्या᳚ - शिष᳚म् । \newline
30. ए॒वैता मे॒ता मे॒वै वैता मैता मे॒वै वैता मा । \newline
31. ए॒ता मैता मे॒ता मा शा᳚स्ते शास्त॒ ऐता मे॒ता मा शा᳚स्ते । \newline
32. आ शा᳚स्ते शास्त॒ आ शा᳚स्त॒ आप॒ आपः॑ शास्त॒ आ शा᳚स्त॒ आपः॑ । \newline
33. शा॒स्त॒ आप॒ आपः॑ शास्ते शास्त॒ आपो॑ देवीर् देवी॒ रापः॑ शास्ते शास्त॒ आपो॑ देवीः । \newline
34. आपो॑ देवीर् देवी॒ राप॒ आपो॑ देवीः शुद्धायुवः शुद्धायुवो देवी॒ राप॒ आपो॑ देवीः शुद्धायुवः । \newline
35. दे॒वीः॒ शु॒द्धा॒यु॒वः॒ शु॒द्धा॒यु॒वो॒ दे॒वी॒र् दे॒वीः॒ शु॒द्धा॒यु॒व॒ इतीति॑ शुद्धायुवो देवीर् देवीः शुद्धायुव॒ इति॑ । \newline
36. शु॒द्धा॒यु॒व॒ इतीति॑ शुद्धायुवः शुद्धायुव॒ इत्या॑हा॒ हेति॑ शुद्धायुवः शुद्धायुव॒ इत्या॑ह । \newline
37. शु॒द्धा॒यु॒व॒ इति॑ शुद्ध - यु॒वः॒ । \newline
38. इत्या॑हा॒हे तीत्या॑ह यथाय॒जुर् य॑थाय॒जु रा॒हे तीत्या॑ह यथाय॒जुः । \newline
39. आ॒ह॒ य॒था॒य॒जुर् य॑थाय॒जु रा॑हाह यथाय॒जु रे॒वैव य॑थाय॒जु रा॑हाह यथाय॒जु रे॒व । \newline
40. य॒था॒य॒जु रे॒वैव य॑थाय॒जुर् य॑थाय॒जु रे॒वैत दे॒त दे॒व य॑थाय॒जुर् य॑थाय॒जु रे॒वैतत् । \newline
41. य॒था॒य॒जुरिति॑ यथा - य॒जुः । \newline
42. ए॒वैत दे॒त दे॒वै वैतत् । \newline
43. ए॒तदित्ये॒तत् । \newline
\pagebreak
\markright{ TS 6.3.9.1  \hfill https://www.vedavms.in \hfill}

\section{ TS 6.3.9.1 }

\textbf{TS 6.3.9.1 } \newline
\textbf{Samhita Paata} \newline

प॒शोर्वा आल॑ब्धस्य प्रा॒णाञ्छुगृ॑च्छति॒ वाक्त॒ आ प्या॑यतां प्रा॒णस्त॒ आ प्या॑यता॒मित्या॑ह प्रा॒णेभ्य॑ ए॒वास्य॒ शुचꣳ॑ शमयति॒ सा प्रा॒णेभ्योऽधि॑ पृथि॒वीꣳ शुक् प्र वि॑शति॒ शमहो᳚भ्या॒मिति॒ नि न॑यत्यहोरा॒त्राभ्या॑मे॒व पृ॑थि॒व्यै शुचꣳ॑ शमय॒त्योष॑धे॒ त्रा॑यस्वैनꣳ॒॒ स्वधि॑ते॒ मैनꣳ॑ हिꣳसी॒रित्या॑ह॒ वज्रो॒॒ वै स्वधि॑तिः॒- [  ] \newline

\textbf{Pada Paata} \newline

प॒शोः । वै । आल॑ब्ध॒स्येत्या - ल॒ब्ध॒स्य॒ । प्रा॒णानिति॑ प्र - अ॒नान् । शुक् । ऋ॒च्छ॒ति॒ । वाक् । ते॒ । एति॑ । प्या॒य॒ता॒म् । प्रा॒ण इति॑ प्र - अ॒नः । ते॒ । एति॑ । प्या॒य॒ता॒म् । इति॑ । आ॒ह॒ । प्रा॒णेभ्य॒ इति॑ प्र - अ॒नेभ्यः॑ । ए॒व । अ॒स्य॒ । शुच᳚म् । श॒म॒य॒ति॒ । सा । प्रा॒णेभ्य॒ इति॑ प्र - अ॒नेभ्यः॑ । अधीति॑ । पृ॒थि॒वीम् । शुक् । प्रेति॑ । वि॒श॒ति॒ । शम् । अहो᳚भ्या॒मित्यहः॑ - भ्या॒म् । इति॑ । नीति॑ । न॒य॒ति॒ । अ॒हो॒रा॒त्राभ्या॒मित्य॑हः-रा॒त्राभ्या᳚म् । ए॒व । पृ॒थि॒व्यै । शुच᳚म् । श॒म॒य॒ति॒ । ओष॑धे । त्राय॑स्व । ए॒न॒म् । स्वधि॑त॒ इति॒ स्व - धि॒ते॒ । मा । ए॒न॒म् । हिꣳ॒॒सीः॒ । इति॑ । आ॒ह॒ । वज्रः॑ । वै । स्वधि॑ति॒रिति॒ स्व - धि॒तिः॒ ।  \newline


\textbf{Krama Paata} \newline

प॒शोर् वै । वा आल॑ब्धस्य । आल॑ब्धस्य प्रा॒णान् । आल॑ब्ध॒स्येत्या - ल॒ब्ध॒स्य॒ । प्रा॒णान्छुक् । प्रा॒णनिति॑ प्र - अ॒नान् । शुगृ॑च्छति । ऋ॒च्छ॒ति॒ वाक् । वाक् ते᳚ । त॒ आ । आ प्या॑यताम् । प्या॒य॒ता॒म् प्रा॒णः । प्रा॒णस्ते᳚ । प्रा॒ण इति॑ प्र - अ॒नः । त॒ आ । आ प्या॑यताम् । प्या॒य॒ता॒मिति॑ । इत्या॑ह । आ॒ह॒ प्रा॒णेभ्यः॑ । प्रा॒णेभ्य॑ ए॒व । प्रा॒णेभ्य॒ इति॑ प्र - अ॒नेभ्यः॑ । ए॒वास्य॑ । अ॒स्य॒ शुच᳚म् । शुचꣳ॑ शमयति । श॒म॒य॒ति॒ सा । सा प्रा॒णेभ्यः॑ । प्रा॒णेभ्योऽधि॑ । प्रा॒णेभ्य॒ इति॑ प्र - अ॒नेभ्यः॑ । अधि॑ पृथि॒वीम् । पृ॒थि॒वीꣳ शुक् । शुक् प्र । प्र वि॑शति । वि॒श॒ति॒ शम् । शमहो᳚भ्याम् । अहो᳚भ्या॒मिति॑ । अहो᳚भ्या॒मित्यहः॑ - भ्या॒म् । इति॒ नि । नि न॑यति । न॒य॒त्य॒हो॒रा॒त्राभ्या᳚म् । अ॒हो॒रा॒त्राभ्या॑मे॒व । अ॒हो॒रा॒त्राभ्या॒मित्य॑हः - रा॒त्राभ्या᳚म् । ए॒व पृ॑थि॒व्यै । पृ॒थि॒व्यै शुच᳚म् । शुचꣳ॑ शमयति । श॒म॒य॒त्योष॑धे । ओष॑धे॒ त्राय॑स्व । त्राय॑स्वैनम् । ए॒नꣳ॒ स्वधि॑ते । स्वधि॑ते॒ मा । स्वधि॑त॒ इति॒ स्व - धि॒ते॒ । मैन᳚म् । ए॒नꣳ॒॒ हिꣳ॒॒सीः॒ । हिꣳ॒॒सी॒रिति॑ । इत्या॑ह । आ॒ह॒ वज्रः॑ । वज्रो॒ वै । वै स्वधि॑तिः । स्वधि॑तिः॒ शान्त्यै᳚ । स्वधि॑ति॒रिति॒ स्व - धि॒तिः॒ \newline

\textbf{Jatai Paata} \newline

1. प॒शोर् वै वै प॒शोः प॒शोर् वै । \newline
2. वा आल॑ब्ध॒स्या ल॑ब्धस्य॒ वै वा आल॑ब्धस्य । \newline
3. आल॑ब्धस्य प्रा॒णान् प्रा॒णा नाल॑ब्ध॒स्या ल॑ब्धस्य प्रा॒णान् । \newline
4. आल॑ब्ध॒स्येत्या - ल॒ब्ध॒स्य॒ । \newline
5. प्रा॒णाञ् छुक् छुक् प्रा॒णान् प्रा॒णाञ् छुक् । \newline
6. प्रा॒णानिति॑ प्र - अ॒नान् । \newline
7. शुगृ॑च्छ त्यृच्छति॒ शुक् छुगृ॑च्छति । \newline
8. ऋ॒च्छ॒ति॒ वाग् वागृ॑च्छ त्यृच्छति॒ वाक् । \newline
9. वाक् ते॑ ते॒ वाग् वाक् ते᳚ । \newline
10. त॒ आ ते॑ त॒ आ । \newline
11. आ प्या॑यताम् प्यायता॒ मा प्या॑यताम् । \newline
12. प्या॒य॒ता॒म् प्रा॒णः प्रा॒णः प्या॑यताम् प्यायताम् प्रा॒णः । \newline
13. प्रा॒ण स्ते॑ ते प्रा॒णः प्रा॒ण स्ते᳚ । \newline
14. प्रा॒ण इति॑ प्र - अ॒नः । \newline
15. त॒ आ ते॑ त॒ आ । \newline
16. आ प्या॑यताम् प्यायता॒ मा प्या॑यताम् । \newline
17. प्या॒य॒ता॒ मितीति॑ प्यायताम् प्यायता॒ मिति॑ । \newline
18. इत्या॑हा॒हे तीत्या॑ह । \newline
19. आ॒ह॒ प्रा॒णेभ्यः॑ प्रा॒णेभ्य॑ आहाह प्रा॒णेभ्यः॑ । \newline
20. प्रा॒णेभ्य॑ ए॒वैव प्रा॒णेभ्यः॑ प्रा॒णेभ्य॑ ए॒व । \newline
21. प्रा॒णेभ्य॒ इति॑ प्र - अ॒नेभ्यः॑ । \newline
22. ए॒वास्या᳚ स्यै॒वै वास्य॑ । \newline
23. अ॒स्य॒ शुचꣳ॒॒ शुच॑ मस्यास्य॒ शुच᳚म् । \newline
24. शुचꣳ॑ शमयति शमयति॒ शुचꣳ॒॒ शुचꣳ॑ शमयति । \newline
25. श॒म॒य॒ति॒ सा सा श॑मयति शमयति॒ सा । \newline
26. सा प्रा॒णेभ्यः॑ प्रा॒णेभ्यः॒ सा सा प्रा॒णेभ्यः॑ । \newline
27. प्रा॒णेभ्यो ऽध्यधि॑ प्रा॒णेभ्यः॑ प्रा॒णेभ्यो ऽधि॑ । \newline
28. प्रा॒णेभ्य॒ इति॑ प्र - अ॒नेभ्यः॑ । \newline
29. अधि॑ पृथि॒वीम् पृ॑थि॒वी मध्यधि॑ पृथि॒वीम् । \newline
30. पृ॒थि॒वीꣳ शुक् छुक् पृ॑थि॒वीम् पृ॑थि॒वीꣳ शुक् । \newline
31. शुक् प्र प्र शुक् छुक् प्र । \newline
32. प्र वि॑शति विशति॒ प्र प्र वि॑शति । \newline
33. वि॒श॒ति॒ शꣳ शं ॅवि॑शति विशति॒ शम् । \newline
34. श महो᳚भ्या॒ महो᳚भ्याꣳ॒॒ शꣳ श महो᳚भ्याम् । \newline
35. अहो᳚भ्या॒ मिती त्यहो᳚भ्या॒ महो᳚भ्या॒ मिति॑ । \newline
36. अहो᳚भ्या॒मित्यहः॑ - भ्या॒म् । \newline
37. इति॒ नि नीतीति॒ नि । \newline
38. नि न॑यति नयति॒ नि नि न॑यति । \newline
39. न॒य॒ त्य॒हो॒रा॒त्राभ्या॑ महोरा॒त्राभ्या᳚न् नयति नय त्यहोरा॒त्राभ्या᳚म् । \newline
40. अ॒हो॒रा॒त्राभ्या॑ मे॒वै वाहो॑रा॒त्राभ्या॑ महोरा॒त्राभ्या॑ मे॒व । \newline
41. अ॒हो॒रा॒त्राभ्या॒मित्य॑हः - रा॒त्राभ्या᳚म् । \newline
42. ए॒व पृ॑थि॒व्यै पृ॑थि॒व्या ए॒वैव पृ॑थि॒व्यै । \newline
43. पृ॒थि॒व्यै शुचꣳ॒॒ शुच॑म् पृथि॒व्यै पृ॑थि॒व्यै शुच᳚म् । \newline
44. शुचꣳ॑ शमयति शमयति॒ शुचꣳ॒॒ शुचꣳ॑ शमयति । \newline
45. श॒म॒य॒ त्योष॑ध॒ ओष॑धे शमयति शमय॒ त्योष॑धे । \newline
46. ओष॑धे॒ त्राय॑स्व॒ त्राय॒स्वौष॑ध॒ ओष॑धे॒ त्राय॑स्व । \newline
47. त्राय॑ स्वैन मेन॒म् त्राय॑स्व॒ त्राय॑ स्वैनम् । \newline
48. ए॒नꣳ॒॒ स्वधि॑ते॒ स्वधि॑त एन मेनꣳ॒॒ स्वधि॑ते । \newline
49. स्वधि॑ते॒ मा मा स्वधि॑ते॒ स्वधि॑ते॒ मा । \newline
50. स्वधि॑त॒ इति॒ स्व - धि॒ते॒ । \newline
51. मैन॑ मेन॒म् मा मैन᳚म् । \newline
52. ए॒नꣳ॒॒ हिꣳ॒॒सी॒र्॒. हिꣳ॒॒सी॒ रे॒न॒ मे॒नꣳ॒॒ हिꣳ॒॒सीः॒ । \newline
53. हिꣳ॒॒सी॒ रितीति॑ हिꣳसीर्. हिꣳसी॒ रिति॑ । \newline
54. इत्या॑हा॒हे तीत्या॑ह । \newline
55. आ॒ह॒ वज्रो॒ वज्र॑ आहाह॒ वज्रः॑ । \newline
56. वज्रो॒ वै वै वज्रो॒ वज्रो॒ वै । \newline
57. वै स्वधि॑तिः॒ स्वधि॑ति॒र् वै वै स्वधि॑तिः । \newline
58. स्वधि॑तिः॒ शान्त्यै॒ शान्त्यै॒ स्वधि॑तिः॒ स्वधि॑तिः॒ शान्त्यै᳚ । \newline
59. स्वधि॑ति॒रिति॒ स्व - धि॒तिः॒ । \newline

\textbf{Ghana Paata } \newline

1. प॒शोर् वै वै प॒शोः प॒शोर् वा आल॑ब्ध॒स्या ल॑ब्धस्य॒ वै प॒शोः प॒शोर् वा आल॑ब्धस्य । \newline
2. वा आल॑ब्ध॒स्या ल॑ब्धस्य॒ वै वा आल॑ब्धस्य प्रा॒णान् प्रा॒णाना ल॑ब्धस्य॒ वै वा आल॑ब्धस्य प्रा॒णान् । \newline
3. आल॑ब्धस्य प्रा॒णान् प्रा॒णाना ल॑ब्ध॒स्या ल॑ब्धस्य प्रा॒णाञ् छुक् छुक् प्रा॒णा नाल॑ब्ध॒स्या ल॑ब्धस्य प्रा॒णाञ् छुक् । \newline
4. आल॑ब्ध॒स्येत्या - ल॒ब्ध॒स्य॒ । \newline
5. प्रा॒णाञ् छुक् छुक् प्रा॒णान् प्रा॒णाञ् छुगृ॑च्छ त्यृच्छति॒ शुक् प्रा॒णान् प्रा॒णाञ् छुगृ॑च्छति । \newline
6. प्रा॒णानिति॑ प्र - अ॒नान् । \newline
7. शुगृ॑च्छ त्यृच्छति॒ शुक् छुगृ॑च्छति॒ वाग् वागृ॑च्छति॒ शुक् छुगृ॑च्छति॒ वाक् । \newline
8. ऋ॒च्छ॒ति॒ वाग् वागृ॑च्छ त्यृच्छति॒ वाक् ते॑ ते॒ वागृ॑च्छ त्यृच्छति॒ वाक् ते᳚ । \newline
9. वाक् ते॑ ते॒ वाग् वाक् त॒ आ ते॒ वाग् वाक् त॒ आ । \newline
10. त॒ आ ते॑ त॒ आ प्या॑यताम् प्यायता॒ मा ते॑ त॒ आ प्या॑यताम् । \newline
11. आ प्या॑यताम् प्यायता॒ मा प्या॑यताम् प्रा॒णः प्रा॒णः प्या॑यता॒ मा प्या॑यताम् प्रा॒णः । \newline
12. प्या॒य॒ता॒म् प्रा॒णः प्रा॒णः प्या॑यताम् प्यायताम् प्रा॒ण स्ते॑ ते प्रा॒णः प्या॑यताम् प्यायताम् प्रा॒ण स्ते᳚ । \newline
13. प्रा॒ण स्ते॑ ते प्रा॒णः प्रा॒ण स्त॒ आ ते᳚ प्रा॒णः प्रा॒ण स्त॒ आ । \newline
14. प्रा॒ण इति॑ प्र - अ॒नः । \newline
15. त॒ आ ते॑ त॒ आ प्या॑यताम् प्यायता॒ मा ते॑ त॒ आ प्या॑यताम् । \newline
16. आ प्या॑यताम् प्यायता॒ मा प्या॑यता॒ मितीति॑ प्यायता॒ मा प्या॑यता॒ मिति॑ । \newline
17. प्या॒य॒ता॒ मितीति॑ प्यायताम् प्यायता॒ मित्या॑हा॒ हेति॑ प्यायताम् प्यायता॒ मित्या॑ह । \newline
18. इत्या॑हा॒हे तीत्या॑ह प्रा॒णेभ्यः॑ प्रा॒णेभ्य॑ आ॒हे तीत्या॑ह प्रा॒णेभ्यः॑ । \newline
19. आ॒ह॒ प्रा॒णेभ्यः॑ प्रा॒णेभ्य॑ आहाह प्रा॒णेभ्य॑ ए॒वैव प्रा॒णेभ्य॑ आहाह प्रा॒णेभ्य॑ ए॒व । \newline
20. प्रा॒णेभ्य॑ ए॒वैव प्रा॒णेभ्यः॑ प्रा॒णेभ्य॑ ए॒वास्या᳚ स्यै॒व प्रा॒णेभ्यः॑ प्रा॒णेभ्य॑ ए॒वास्य॑ । \newline
21. प्रा॒णेभ्य॒ इति॑ प्र - अ॒नेभ्यः॑ । \newline
22. ए॒वास्या᳚ स्यै॒वै वास्य॒ शुचꣳ॒॒ शुच॑ मस्यै॒वै वास्य॒ शुच᳚म् । \newline
23. अ॒स्य॒ शुचꣳ॒॒ शुच॑ मस्यास्य॒ शुचꣳ॑ शमयति शमयति॒ शुच॑ मस्यास्य॒ शुचꣳ॑ शमयति । \newline
24. शुचꣳ॑ शमयति शमयति॒ शुचꣳ॒॒ शुचꣳ॑ शमयति॒ सा सा श॑मयति॒ शुचꣳ॒॒ शुचꣳ॑ शमयति॒ सा । \newline
25. श॒म॒य॒ति॒ सा सा श॑मयति शमयति॒ सा प्रा॒णेभ्यः॑ प्रा॒णेभ्यः॒ सा श॑मयति शमयति॒ सा प्रा॒णेभ्यः॑ । \newline
26. सा प्रा॒णेभ्यः॑ प्रा॒णेभ्यः॒ सा सा प्रा॒णेभ्यो ऽध्यधि॑ प्रा॒णेभ्यः॒ सा सा प्रा॒णेभ्यो ऽधि॑ । \newline
27. प्रा॒णेभ्यो ऽध्यधि॑ प्रा॒णेभ्यः॑ प्रा॒णेभ्यो ऽधि॑ पृथि॒वीम् पृ॑थि॒वी मधि॑ प्रा॒णेभ्यः॑ प्रा॒णेभ्यो ऽधि॑ पृथि॒वीम् । \newline
28. प्रा॒णेभ्य॒ इति॑ प्र - अ॒नेभ्यः॑ । \newline
29. अधि॑ पृथि॒वीम् पृ॑थि॒वी मध्यधि॑ पृथि॒वीꣳ शुक् छुक् पृ॑थि॒वी मध्यधि॑ पृथि॒वीꣳ शुक् । \newline
30. पृ॒थि॒वीꣳ शुक् छुक् पृ॑थि॒वीम् पृ॑थि॒वीꣳ शुक् प्र प्र शुक् पृ॑थि॒वीम् पृ॑थि॒वीꣳ शुक् प्र । \newline
31. शुक् प्र प्र शुक् छुक् प्र वि॑शति विशति॒ प्र शुक् छुक् प्र वि॑शति । \newline
32. प्र वि॑शति विशति॒ प्र प्र वि॑शति॒ शꣳ शं ॅवि॑शति॒ प्र प्र वि॑शति॒ शम् । \newline
33. वि॒श॒ति॒ शꣳ शं ॅवि॑शति विशति॒ श महो᳚भ्या॒ महो᳚भ्याꣳ॒॒ शं ॅवि॑शति विशति॒ श महो᳚भ्याम् । \newline
34. श महो᳚भ्या॒ महो᳚भ्याꣳ॒॒ शꣳ श महो᳚भ्या॒ मिती त्यहो᳚भ्याꣳ॒॒ शꣳ श महो᳚भ्या॒ मिति॑ । \newline
35. अहो᳚भ्या॒ मिती त्यहो᳚भ्या॒ महो᳚भ्या॒ मिति॒ नि नीत्यहो᳚भ्या॒ महो᳚भ्या॒ मिति॒ नि । \newline
36. अहो᳚भ्या॒मित्यहः॑ - भ्या॒म् । \newline
37. इति॒ नि नीतीति॒ नि न॑यति नयति॒ नीतीति॒ नि न॑यति । \newline
38. नि न॑यति नयति॒ नि नि न॑य त्यहोरा॒त्राभ्या॑ महोरा॒त्राभ्या᳚म् नयति॒ नि नि न॑य त्यहोरा॒त्राभ्या᳚म् । \newline
39. न॒य॒ त्य॒हो॒रा॒त्राभ्या॑ महोरा॒त्राभ्या᳚म् नयति नय त्यहोरा॒त्राभ्या॑ मे॒वै वाहो॑रा॒त्राभ्या᳚म् नयति नय त्यहोरा॒त्राभ्या॑ मे॒व । \newline
40. अ॒हो॒रा॒त्राभ्या॑ मे॒वै वाहो॑रा॒त्राभ्या॑ महोरा॒त्राभ्या॑ मे॒व पृ॑थि॒व्यै पृ॑थि॒व्या ए॒वाहो॑रा॒त्राभ्या॑ महोरा॒त्राभ्या॑ मे॒व पृ॑थि॒व्यै । \newline
41. अ॒हो॒रा॒त्राभ्या॒मित्य॑हः - रा॒त्राभ्या᳚म् । \newline
42. ए॒व पृ॑थि॒व्यै पृ॑थि॒व्या ए॒वैव पृ॑थि॒व्यै शुचꣳ॒॒ शुच॑म् पृथि॒व्या ए॒वैव पृ॑थि॒व्यै शुच᳚म् । \newline
43. पृ॒थि॒व्यै शुचꣳ॒॒ शुच॑म् पृथि॒व्यै पृ॑थि॒व्यै शुचꣳ॑ शमयति शमयति॒ शुच॑म् पृथि॒व्यै पृ॑थि॒व्यै शुचꣳ॑ शमयति । \newline
44. शुचꣳ॑ शमयति शमयति॒ शुचꣳ॒॒ शुचꣳ॑ शमय॒ त्योष॑ध॒ ओष॑धे शमयति॒ शुचꣳ॒॒ शुचꣳ॑ शमय॒ त्योष॑धे । \newline
45. श॒म॒य॒ त्योष॑ध॒ ओष॑धे शमयति शमय॒ त्योष॑धे॒ त्राय॑स्व॒ त्राय॒ स्वौष॑धे शमयति शमय॒ त्योष॑धे॒ त्राय॑स्व । \newline
46. ओष॑धे॒ त्राय॑स्व॒ त्राय॒ स्वौष॑ध॒ ओष॑धे॒ त्राय॑स्वैन मेन॒म् त्राय॒ स्वौष॑ध॒ ओष॑धे॒ त्राय॑स्वैनम् । \newline
47. त्राय॑स्वैन मेन॒म् त्राय॑स्व॒ त्राय॑स्वैनꣳ॒॒ स्वधि॑ते॒ स्वधि॑त एन॒म् त्राय॑स्व॒ त्राय॑स्वैनꣳ॒॒ स्वधि॑ते । \newline
48. ए॒नꣳ॒॒ स्वधि॑ते॒ स्वधि॑त एन मेनꣳ॒॒ स्वधि॑ते॒ मा मा स्वधि॑त एन मेनꣳ॒॒ स्वधि॑ते॒ मा । \newline
49. स्वधि॑ते॒ मा मा स्वधि॑ते॒ स्वधि॑ते॒ मैन॑ मेन॒म् मा स्वधि॑ते॒ स्वधि॑ते॒ मैन᳚म् । \newline
50. स्वधि॑त॒ इति॒ स्व - धि॒ते॒ । \newline
51. मैन॑ मेन॒म् मा मैनꣳ॑ हिꣳसीर्. हिꣳसी रेन॒म् मा मैनꣳ॑ हिꣳसीः । \newline
52. ए॒नꣳ॒॒ हिꣳ॒॒सी॒र्॒. हिꣳ॒॒सी॒ रे॒न॒ मे॒नꣳ॒॒ हिꣳ॒॒सी॒ रितीति॑ हिꣳसी रेन मेनꣳ हिꣳसी॒ रिति॑ । \newline
53. हिꣳ॒॒सी॒ रितीति॑ हिꣳसीर्. हिꣳसी॒रि त्या॑हा॒ हेति॑ हिꣳसीर्. हिꣳसी॒ रित्या॑ह । \newline
54. इत्या॑हा॒हे तीत्या॑ह॒ वज्रो॒ वज्र॑ आ॒हे तीत्या॑ह॒ वज्रः॑ । \newline
55. आ॒ह॒ वज्रो॒ वज्र॑ आहाह॒ वज्रो॒ वै वै वज्र॑ आहाह॒ वज्रो॒ वै । \newline
56. वज्रो॒ वै वै वज्रो॒ वज्रो॒ वै स्वधि॑तिः॒ स्वधि॑ति॒र् वै वज्रो॒ वज्रो॒ वै स्वधि॑तिः । \newline
57. वै स्वधि॑तिः॒ स्वधि॑ति॒र् वै वै स्वधि॑तिः॒ शान्त्यै॒ शान्त्यै॒ स्वधि॑ति॒र् वै वै स्वधि॑तिः॒ शान्त्यै᳚ । \newline
58. स्वधि॑तिः॒ शान्त्यै॒ शान्त्यै॒ स्वधि॑तिः॒ स्वधि॑तिः॒ शान्त्यै॑ पार्श्व॒तः पा᳚र्श्व॒तः शान्त्यै॒ स्वधि॑तिः॒ स्वधि॑तिः॒ शान्त्यै॑ पार्श्व॒तः । \newline
59. स्वधि॑ति॒रिति॒ स्व - धि॒तिः॒ । \newline
\pagebreak
\markright{ TS 6.3.9.2  \hfill https://www.vedavms.in \hfill}

\section{ TS 6.3.9.2 }

\textbf{TS 6.3.9.2 } \newline
\textbf{Samhita Paata} \newline

शान्त्यै॑ पार्श्व॒त आ च्छ्य॑ति मद्ध्य॒तो हि म॑नु॒ष्या॑ आ॒ च्छ्यन्ति॑ तिर॒श्चीन॒मा च्छ्य॑त्यनू॒चीनꣳ॒॒ हि म॑नु॒ष्या॑ आ॒च्छ्यन्ति॒ व्यावृ॑त्त्यै॒ रक्ष॑सां भा॒गो॑ऽसीति॑ स्थविम॒तो ब॒र्॒.हिर॒क्त्वाऽपा᳚स्यत्य॒स्नैव रक्षाꣳ॑सि नि॒रव॑दयत इ॒दम॒हꣳ रक्षो॑ऽध॒मं तमो॑ नयामि॒ यो᳚ऽस्मान् द्वेष्टि॒ यं च॑ व॒यं द्वि॒ष्म इत्या॑ह॒ द्वौ वाव पुरु॑षौ॒ यं चै॒व- [  ] \newline

\textbf{Pada Paata} \newline

शान्त्यै᳚ । पा॒र्श्व॒तः । एति॑ । छ्‌य॒ति॒ । म॒द्ध्य॒तः । हि । म॒नु॒ष्याः᳚ । आ॒च्छ्यन्तीत्या᳚-छ्यन्ति॑ । ति॒र॒श्चीन᳚म् । एति॑ । छ्य॒ति॒ । अ॒नू॒चीन᳚म् । हि । म॒नु॒ष्याः᳚ । आ॒च्छ्यन्तीत्या᳚ - छ्यन्ति॑ । व्यावृ॑त्त्या॒ इति॑ वि - आवृ॑त्त्यै । रक्ष॑साम् । भा॒गः । अ॒सि॒ । इति॑ । स्थ॒वि॒म॒तः । ब॒र्॒.हिः । अ॒क्त्वा । अपेति॑ । अ॒स्य॒ति॒ । अ॒स्ना । ए॒व । रक्षाꣳ॑सि । नि॒रव॑दयत॒ इति॑ निः - अव॑दयते । इ॒दम् । अ॒हम् । रक्षः॑ । अ॒ध॒मम् । तमः॑ । न॒या॒मि॒ । यः । अ॒स्मान् । द्वेष्टि॑ । यम् । च॒ । व॒यम् । द्वि॒ष्मः । इति॑ । आ॒ह॒ । द्वौ । वाव । पुरु॑षौ । यम् । च॒ । ए॒व ।  \newline


\textbf{Krama Paata} \newline

शान्त्यै॑ पार्श्व॒तः । पा॒र्श्व॒त आ । आच्छ्‌य॑ति । छ्‌य॒ति॒ म॒द्ध्य॒तः । म॒द्ध्य॒तो हि । हि म॑नु॒ष्याः᳚ । म॒नु॒ष्या॑ आ॒च्छ्‌यन्ति॑ । आ॒च्छ्‌यन्ति॑ तिर॒श्चीन᳚म् । आ॒च्छ्‌यन्तीत्या᳚ - छ्‌यन्ति॑ । ति॒र॒श्चीन॒मा । आच्छ्‌य॑ति । छ्‌य॒त्य॒नू॒चीन᳚म् । अ॒नू॒चीनꣳ॒॒ हि । हि म॑नु॒ष्याः᳚ । म॒नु॒ष्या॑ आ॒च्छ्‌यन्ति॑ । आ॒च्छ्‌यन्ति॒ व्यावृ॑त्त्यै । आ॒च्छ्‌यन्तीत्या᳚ - छ्‌यन्ति॑ । व्यावृ॑त्त्यै॒ रक्ष॑साम् । व्यावृ॑त्त्या॒ इति॑ वि - आवृ॑त्त्यै । रक्ष॑साम् भा॒गः । भा॒गो॑ऽसि । अ॒सीति॑ । इति॑ स्थविम॒तः । स्थ॒वि॒म॒तो ब॒र्.॒हिः । ब॒र्.॒हिर॒क्त्वा । अ॒क्त्वाऽप॑ । अपा᳚स्यति । अ॒स्य॒त्य॒स्ना । अ॒स्नैव । ए॒व रक्षाꣳ॑सि । रक्षाꣳ॑सि नि॒रव॑दयते । नि॒रव॑दयत इ॒दम् । नि॒रव॑दयत॒ इति॑ निः - अव॑दयते । इ॒दम॒हम् । अ॒हꣳ रक्षः॑ । रक्षो॑ऽध॒मम् । अ॒ध॒मम् तमः॑ । तमो॑ नयामि । न॒या॒मि॒ यः । यो᳚ऽस्मान् । अ॒स्मान् द्वेष्टि॑ । द्वेष्टि॒ यम् । यम् च॑ । च॒ व॒यम् । व॒यम् द्वि॒ष्मः । द्वि॒ष्म इति॑ । इत्या॑ह । आ॒ह॒ द्वौ । द्वौ वाव । वाव पुरु॑षौ । पुरु॑षौ॒ यम् । यम् च॑ । चै॒व । ए॒व द्वेष्टि॑ \newline

\textbf{Jatai Paata} \newline

1. शान्त्यै॑ पार्श्व॒तः पा᳚र्श्व॒तः शान्त्यै॒ शान्त्यै॑ पार्श्व॒तः । \newline
2. पा॒र्श्व॒त आ पा᳚र्श्व॒तः पा᳚र्श्व॒त आ । \newline
3. आ च्छ्य॑ति छ्यति॒दा च्छ्य॑ति । \newline
4. छ्य॒ति॒ म॒द्ध्य॒तो म॑द्ध्य॒त श्छ्य॑ति छ्यति मद्ध्य॒तः । \newline
5. म॒द्ध्य॒तो हि हि म॑द्ध्य॒तो म॑द्ध्य॒तो हि । \newline
6. हि म॑नु॒ष्या॑ मनु॒ष्या॑ हि हि म॑नु॒ष्याः᳚ । \newline
7. म॒नु॒ष्या॑ आ॒च्छ्य न्त्या॒च्छ्यन्ति॑ मनु॒ष्या॑ मनु॒ष्या॑ आ॒च्छ्यन्ति॑ । \newline
8. आ॒च्छ्यन्ति॑ तिर॒श्चीन॑म् तिर॒श्चीन॑ मा॒च्छ्य न्त्या॒च्छ्यन्ति॑ तिर॒श्चीन᳚म् । \newline
9. आ॒च्छ्यन्तीत्या᳚ - छ्यन्ति॑ । \newline
10. ति॒र॒श्चीन॒ मा ति॑र॒श्चीन॑म् तिर॒श्चीन॒ मा । \newline
11. आ च्छ्य॑ति छ्य॒त्या च्छ्य॑ति । \newline
12. छ्य॒ त्य॒नू॒चीन॑ मनू॒चीन॑म् छ्यति छ्य त्यनू॒चीन᳚म् । \newline
13. अ॒नू॒चीनꣳ॒॒ हि ह्य॑नू॒चीन॑ मनू॒चीनꣳ॒॒ हि । \newline
14. हि म॑नु॒ष्या॑ मनु॒ष्या॑ हि हि म॑नु॒ष्याः᳚ । \newline
15. म॒नु॒ष्या॑ आ॒च्छ्य न्त्या॒च्छ्यन्ति॑ मनु॒ष्या॑ मनु॒ष्या॑ आ॒च्छ्यन्ति॑ । \newline
16. आ॒च्छ्यन्ति॒ व्यावृ॑त्त्यै॒ व्यावृ॑त्त्या आ॒च्छ्य न्त्या॒च्छ्यन्ति॒ व्यावृ॑त्त्यै । \newline
17. आ॒च्छ्यन्तीत्या᳚ - छ्यन्ति॑ । \newline
18. व्यावृ॑त्त्यै॒ रक्ष॑साꣳ॒॒ रक्ष॑सां॒ ॅव्यावृ॑त्त्यै॒ व्यावृ॑त्त्यै॒ रक्ष॑साम् । \newline
19. व्यावृ॑त्त्या॒ इति॑ वि - आवृ॑त्त्यै । \newline
20. रक्ष॑साम् भा॒गो भा॒गो रक्ष॑साꣳ॒॒ रक्ष॑साम् भा॒गः । \newline
21. भा॒गो᳚ ऽस्यसि भा॒गो भा॒गो॑ ऽसि । \newline
22. अ॒सी तीत्य॑ स्य॒ सीति॑ । \newline
23. इति॑ स्थविम॒तः स्थ॑विम॒त इतीति॑ स्थविम॒तः । \newline
24. स्थ॒वि॒म॒तो ब॒र्॒.हिर् ब॒र्॒.हिः स्थ॑विम॒तः स्थ॑विम॒तो ब॒र्॒.हिः । \newline
25. ब॒र्॒.हि र॒क्त्वा ऽक्त्वा ब॒र्॒.हिर् ब॒र्॒.हि र॒क्त्वा । \newline
26. अ॒क्त्वा ऽपापा॒ क्त्वा ऽक्त्वा ऽप॑ । \newline
27. अपा᳚ स्यत्य स्य॒त्य पापा᳚ स्यति । \newline
28. अ॒स्य॒ त्य॒स्ना ऽस्ना ऽस्य॑ त्यस्य त्य॒स्ना । \newline
29. अ॒स्नैवै वास्ना ऽस्नैव । \newline
30. ए॒व रक्षाꣳ॑सि॒ रक्षाꣳ॑ स्ये॒वैव रक्षाꣳ॑सि । \newline
31. रक्षाꣳ॑सि नि॒रव॑दयते नि॒रव॑दयते॒ रक्षाꣳ॑सि॒ रक्षाꣳ॑सि नि॒रव॑दयते । \newline
32. नि॒रव॑दयत इ॒द मि॒दम् नि॒रव॑दयते नि॒रव॑दयत इ॒दम् । \newline
33. नि॒रव॑दयत॒ इति॑ निः - अव॑दयते । \newline
34. इ॒द म॒ह म॒ह मि॒द मि॒द म॒हम् । \newline
35. अ॒हꣳ रक्षो॒ रक्षो॒ ऽह म॒हꣳ रक्षः॑ । \newline
36. रक्षो॑ ऽध॒म म॑ध॒मꣳ रक्षो॒ रक्षो॑ ऽध॒मम् । \newline
37. अ॒ध॒मम् तम॒ स्तमो॑ ऽध॒म म॑ध॒मम् तमः॑ । \newline
38. तमो॑ नयामि नयामि॒ तम॒ स्तमो॑ नयामि । \newline
39. न॒या॒मि॒ यो यो न॑यामि नयामि॒ यः । \newline
40. यो᳚ ऽस्मा न॒स्मान्. यो यो᳚ ऽस्मान् । \newline
41. अ॒स्मान् द्वेष्टि॒ द्वेष्ट्य॒ स्मान॒ स्मान् द्वेष्टि॑ । \newline
42. द्वेष्टि॒ यं ॅयम् द्वेष्टि॒ द्वेष्टि॒ यम् । \newline
43. यम् च॑ च॒ यं ॅयम् च॑ । \newline
44. च॒ व॒यं ॅव॒यम् च॑ च व॒यम् । \newline
45. व॒यम् द्वि॒ष्मो द्वि॒ष्मो व॒यं ॅव॒यम् द्वि॒ष्मः । \newline
46. द्वि॒ष्म इतीति॑ द्वि॒ष्मो द्वि॒ष्म इति॑ । \newline
47. इत्या॑हा॒हे तीत्या॑ह । \newline
48. आ॒ह॒ द्वौ द्वा वा॑हाह॒ द्वौ । \newline
49. द्वौ वाव वाव द्वौ द्वौ वाव । \newline
50. वाव पुरु॑षौ॒ पुरु॑षौ॒ वाव वाव पुरु॑षौ । \newline
51. पुरु॑षौ॒ यं ॅयम् पुरु॑षौ॒ पुरु॑षौ॒ यम् । \newline
52. यम् च॑ च॒ यं ॅयम् च॑ । \newline
53. चै॒वैव च॑ चै॒व । \newline
54. ए॒व द्वेष्टि॒ द्वेष्ट्ये॒ वैव द्वेष्टि॑ । \newline

\textbf{Ghana Paata } \newline

1. शान्त्यै॑ पार्श्व॒तः पा᳚र्श्व॒तः शान्त्यै॒ शान्त्यै॑ पार्श्व॒त आ पा᳚र्श्व॒तः शान्त्यै॒ शान्त्यै॑ पार्श्व॒त आ । \newline
2. पा॒र्श्व॒त आ पा᳚र्श्व॒तः पा᳚र्श्व॒त आ च्छ्य॑ति छ्यति॒दा पा᳚र्श्व॒तः पा᳚र्श्व॒त आ च्छ्य॑ति । \newline
3. आ च्छ्य॑ति छ्यति॒दा च्छ्य॑ति मद्ध्य॒तो म॑द्ध्य॒त श्छ्य॑ति॒दा च्छ्य॑ति मद्ध्य॒तः । \newline
4. छ्य॒ति॒ म॒द्ध्य॒तो म॑द्ध्य॒त श्छ्य॑ति छ्यति मद्ध्य॒तो हि हि म॑द्ध्य॒त श्छ्य॑ति छ्यति मद्ध्य॒तो हि । \newline
5. म॒द्ध्य॒तो हि हि म॑द्ध्य॒तो म॑द्ध्य॒तो हि म॑नु॒ष्या॑ मनु॒ष्या॑ हि म॑द्ध्य॒तो म॑द्ध्य॒तो हि म॑नु॒ष्याः᳚ । \newline
6. हि म॑नु॒ष्या॑ मनु॒ष्या॑ हि हि म॑नु॒ष्या॑ आ॒च्छ्य न्त्या॒च्छ्यन्ति॑ मनु॒ष्या॑ हि हि म॑नु॒ष्या॑ आ॒च्छ्यन्ति॑ । \newline
7. म॒नु॒ष्या॑ आ॒च्छ्य न्त्या॒च्छ्यन्ति॑ मनु॒ष्या॑ मनु॒ष्या॑ आ॒च्छ्यन्ति॑ तिर॒श्चीन॑म् तिर॒श्चीन॑ मा॒च्छ्यन्ति॑ मनु॒ष्या॑ मनु॒ष्या॑ आ॒च्छ्यन्ति॑ तिर॒श्चीन᳚म् । \newline
8. आ॒च्छ्यन्ति॑ तिर॒श्चीन॑म् तिर॒श्चीन॑ मा॒च्छ्य न्त्या॒च्छ्यन्ति॑ तिर॒श्चीन॒ मा ति॑र॒श्चीन॑ मा॒च्छ्य न्त्या॒च्छ्यन्ति॑ तिर॒श्चीन॒ मा । \newline
9. आ॒च्छ्यन्तीत्या᳚ - छ्यन्ति॑ । \newline
10. ति॒र॒श्चीन॒ मा ति॑र॒श्चीन॑म् तिर॒श्चीन॒ मा च्छ्य॑ति छ्य॒त्या ति॑र॒श्चीन॑म् तिर॒श्चीन॒ मा च्छ्य॑ति । \newline
11. आ च्छ्य॑ति छ्य॒त्या च्छ्य॑ त्यनू॒चीन॑ मनू॒चीन॑म् छ्य॒त्या च्छ्य॑ त्यनू॒चीन᳚म् । \newline
12. छ्य॒ त्य॒नू॒चीन॑ मनू॒चीन॑म् छ्यति छ्यत्यनू॒चीनꣳ॒॒ हि ह्य॑नू॒चीन॑म् छ्यति छ्यत्यनू॒चीनꣳ॒॒ हि । \newline
13. अ॒नू॒चीनꣳ॒॒ हि ह्य॑नू॒चीन॑ मनू॒चीनꣳ॒॒ हि म॑नु॒ष्या॑ मनु॒ष्या᳚(1॒) ह्य॑नू॒चीन॑ मनू॒चीनꣳ॒॒ हि म॑नु॒ष्याः᳚ । \newline
14. हि म॑नु॒ष्या॑ मनु॒ष्या॑ हि हि म॑नु॒ष्या॑ आ॒च्छ्य न्त्या॒च्छ्यन्ति॑ मनु॒ष्या॑ हि हि म॑नु॒ष्या॑ आ॒च्छ्यन्ति॑ । \newline
15. म॒नु॒ष्या॑ आ॒च्छ्य न्त्या॒च्छ्यन्ति॑ मनु॒ष्या॑ मनु॒ष्या॑ आ॒च्छ्यन्ति॒ व्यावृ॑त्त्यै॒ व्यावृ॑त्त्या आ॒च्छ्यन्ति॑ मनु॒ष्या॑ मनु॒ष्या॑ आ॒च्छ्यन्ति॒ व्यावृ॑त्त्यै । \newline
16. आ॒च्छ्यन्ति॒ व्यावृ॑त्त्यै॒ व्यावृ॑त्त्या आ॒च्छ्य न्त्या॒च्छ्यन्ति॒ व्यावृ॑त्त्यै॒ रक्ष॑साꣳ॒॒ रक्ष॑सां॒ ॅव्यावृ॑त्त्या आ॒च्छ्य न्त्या॒च्छ्यन्ति॒ व्यावृ॑त्त्यै॒ रक्ष॑साम् । \newline
17. आ॒च्छ्यन्तीत्या᳚ - छ्यन्ति॑ । \newline
18. व्यावृ॑त्त्यै॒ रक्ष॑साꣳ॒॒ रक्ष॑सां॒ ॅव्यावृ॑त्त्यै॒ व्यावृ॑त्त्यै॒ रक्ष॑साम् भा॒गो भा॒गो रक्ष॑सां॒ ॅव्यावृ॑त्त्यै॒ व्यावृ॑त्त्यै॒ रक्ष॑साम् भा॒गः । \newline
19. व्यावृ॑त्त्या॒ इति॑ वि - आवृ॑त्त्यै । \newline
20. रक्ष॑साम् भा॒गो भा॒गो रक्ष॑साꣳ॒॒ रक्ष॑साम् भा॒गो᳚ ऽस्यसि भा॒गो रक्ष॑साꣳ॒॒ रक्ष॑साम् भा॒गो॑ ऽसि । \newline
21. भा॒गो᳚ ऽस्यसि भा॒गो भा॒गो॑ ऽसीती त्य॑सि भा॒गो भा॒गो॑ ऽसीति॑ । \newline
22. अ॒सी तीत्य॑स्य॒ सीति॑ स्थविम॒तः स्थ॑विम॒त इत्य॑स्य॒सीति॑ स्थविम॒तः । \newline
23. इति॑ स्थविम॒तः स्थ॑विम॒त इतीति॑ स्थविम॒तो ब॒र्॒.हिर् ब॒र्॒.हिः स्थ॑विम॒त इतीति॑ स्थविम॒तो ब॒र्॒.हिः । \newline
24. स्थ॒वि॒म॒तो ब॒र्॒.हिर् ब॒र्॒.हिः स्थ॑विम॒तः स्थ॑विम॒तो ब॒र्॒.हि र॒क्त्वा ऽक्त्वा ब॒र्॒.हिः स्थ॑विम॒तः स्थ॑विम॒तो ब॒र्॒.हि र॒क्त्वा । \newline
25. ब॒र्॒.हि र॒क्त्वा ऽक्त्वा ब॒र्॒.हिर् ब॒र्॒.हि र॒क्त्वा ऽपापा॒क्त्वा ब॒र्॒.हिर् ब॒र्॒.हि र॒क्त्वा ऽप॑ । \newline
26. अ॒क्त्वा ऽपापा॒क्त्वा ऽक्त्वा ऽपा᳚स्य त्यस्य॒ त्यपा॒क्त्वा ऽक्त्वा ऽपा᳚स्यति । \newline
27. अपा᳚स्य त्यस्य॒ त्यपापा᳚ स्यत्य॒स्ना ऽस्ना ऽस्य॒त्य पापा᳚ स्यत्य॒स्ना । \newline
28. अ॒स्य॒ त्य॒स्ना ऽस्ना ऽस्य॑त्यस्य त्य॒स्नैवै वास्ना ऽस्य॑त्यस्य त्य॒स्नैव । \newline
29. अ॒स्नैवै वास्ना ऽस्नैव रक्षाꣳ॑सि॒ रक्षाꣳ॑ स्ये॒वास्ना ऽस्नैव रक्षाꣳ॑सि । \newline
30. ए॒व रक्षाꣳ॑सि॒ रक्षाꣳ॑ स्ये॒वैव रक्षाꣳ॑सि नि॒रव॑दयते नि॒रव॑दयते॒ रक्षाꣳ॑ स्ये॒वैव रक्षाꣳ॑सि नि॒रव॑दयते । \newline
31. रक्षाꣳ॑सि नि॒रव॑दयते नि॒रव॑दयते॒ रक्षाꣳ॑सि॒ रक्षाꣳ॑सि नि॒रव॑दयत इ॒द मि॒दम् नि॒रव॑दयते॒ रक्षाꣳ॑सि॒ रक्षाꣳ॑सि नि॒रव॑दयत इ॒दम् । \newline
32. नि॒रव॑दयत इ॒द मि॒दम् नि॒रव॑दयते नि॒रव॑दयत इ॒द म॒ह म॒ह मि॒दम् नि॒रव॑दयते नि॒रव॑दयत इ॒द म॒हम् । \newline
33. नि॒रव॑दयत॒ इति॑ निः - अव॑दयते । \newline
34. इ॒द म॒ह म॒ह मि॒द मि॒द म॒हꣳ रक्षो॒ रक्षो॒ ऽह मि॒द मि॒द म॒हꣳ रक्षः॑ । \newline
35. अ॒हꣳ रक्षो॒ रक्षो॒ ऽह म॒हꣳ रक्षो॑ ऽध॒म म॑ध॒मꣳ रक्षो॒ ऽह म॒हꣳ रक्षो॑ ऽध॒मम् । \newline
36. रक्षो॑ ऽध॒म म॑ध॒मꣳ रक्षो॒ रक्षो॑ ऽध॒मम् तम॒ स्तमो॑ ऽध॒मꣳ रक्षो॒ रक्षो॑ ऽध॒मम् तमः॑ । \newline
37. अ॒ध॒मम् तम॒ स्तमो॑ ऽध॒म म॑ध॒मम् तमो॑ नयामि नयामि॒ तमो॑ ऽध॒म म॑ध॒मम् तमो॑ नयामि । \newline
38. तमो॑ नयामि नयामि॒ तम॒ स्तमो॑ नयामि॒ यो यो न॑यामि॒ तम॒ स्तमो॑ नयामि॒ यः । \newline
39. न॒या॒मि॒ यो यो न॑यामि नयामि॒ यो᳚ ऽस्मा न॒स्मान्. यो न॑यामि नयामि॒ यो᳚ ऽस्मान् । \newline
40. यो᳚ ऽस्मा न॒स्मान्. यो यो᳚ ऽस्मान् द्वेष्टि॒ द्वेष्ट्य॒स्मान्. यो यो᳚ ऽस्मान् द्वेष्टि॑ । \newline
41. अ॒स्मान् द्वेष्टि॒ द्वेष्ट्य॒स्मा न॒स्मान् द्वेष्टि॒ यं ॅयम् द्वेष्ट्य॒स्मा न॒स्मान् द्वेष्टि॒ यम् । \newline
42. द्वेष्टि॒ यं ॅयम् द्वेष्टि॒ द्वेष्टि॒ यम् च॑ च॒ यम् द्वेष्टि॒ द्वेष्टि॒ यम् च॑ । \newline
43. यम् च॑ च॒ यं ॅयम् च॑ व॒यं ॅव॒यम् च॒ यं ॅयम् च॑ व॒यम् । \newline
44. च॒ व॒यं ॅव॒यम् च॑ च व॒यम् द्वि॒ष्मो द्वि॒ष्मो व॒यम् च॑ च व॒यम् द्वि॒ष्मः । \newline
45. व॒यम् द्वि॒ष्मो द्वि॒ष्मो व॒यं ॅव॒यम् द्वि॒ष्म इतीति॑ द्वि॒ष्मो व॒यं ॅव॒यम् द्वि॒ष्म इति॑ । \newline
46. द्वि॒ष्म इतीति॑ द्वि॒ष्मो द्वि॒ष्म इत्या॑हा॒ हेति॑ द्वि॒ष्मो द्वि॒ष्म इत्या॑ह । \newline
47. इत्या॑हा॒हे तीत्या॑ह॒ द्वौ द्वा वा॒हे तीत्या॑ह॒ द्वौ । \newline
48. आ॒ह॒ द्वौ द्वा वा॑हाह॒ द्वौ वाव वाव द्वा वा॑हाह॒ द्वौ वाव । \newline
49. द्वौ वाव वाव द्वौ द्वौ वाव पुरु॑षौ॒ पुरु॑षौ॒ वाव द्वौ द्वौ वाव पुरु॑षौ । \newline
50. वाव पुरु॑षौ॒ पुरु॑षौ॒ वाव वाव पुरु॑षौ॒ यं ॅयम् पुरु॑षौ॒ वाव वाव पुरु॑षौ॒ यम् । \newline
51. पुरु॑षौ॒ यं ॅयम् पुरु॑षौ॒ पुरु॑षौ॒ यम् च॑ च॒ यम् पुरु॑षौ॒ पुरु॑षौ॒ यम् च॑ । \newline
52. यम् च॑ च॒ यं ॅयम् चै॒वैव च॒ यं ॅयम् चै॒व । \newline
53. चै॒वैव च॑ चै॒व द्वेष्टि॒ द्वेष्ट्ये॒व च॑ चै॒व द्वेष्टि॑ । \newline
54. ए॒व द्वेष्टि॒ द्वेष्ट्ये॒वैव द्वेष्टि॒ यो यो द्वेष्ट्ये॒वैव द्वेष्टि॒ यः । \newline
\pagebreak
\markright{ TS 6.3.9.3  \hfill https://www.vedavms.in \hfill}

\section{ TS 6.3.9.3 }

\textbf{TS 6.3.9.3 } \newline
\textbf{Samhita Paata} \newline

द्वेष्टि॒ यश्चै॑नं॒ द्वेष्टि॒ तावु॒भाव॑ध॒मं तमो॑ नयती॒षे त्वेति॑ व॒पामुत्खि॑दती॒च्छत॑ इव॒ ह्ये॑ष यो यज॑ते॒ यदु॑पतृ॒न्द्याद्-रु॒द्रो᳚ऽस्य प॒शून् घातु॑कः स्या॒द् यन्नोप॑तृ॒न्द्या-दय॑ता स्या-द॒न्ययो॑पतृ॒णत्त्य॒न्यया॒ न धृत्यै॑ घृ॒तेन॑ द्यावापृथिवी॒ प्रोर्ण्वा॑था॒मित्या॑ह॒ द्यावा॑पृथि॒वी ए॒व रसे॑नान॒क्त्यच्छि॑न्नो॒- [  ] \newline

\textbf{Pada Paata} \newline

द्वेष्टि॑ । यः । च॒ । ए॒न॒म् । द्वेष्टि॑ । तौ । उ॒भौ । अ॒ध॒मम् । तमः॑ । न॒य॒ति॒ । इ॒षे । त्वा॒ । इति॑ । व॒पाम् । उदिति॑ । खि॒द॒ति॒ । इ॒च्छते᳚ । इ॒व॒ । हि । ए॒षः । यः । यज॑ते । यत् । उ॒प॒तृ॒न्द्यादित्यु॑प - तृ॒न्द्यात् । रु॒द्रः । अ॒स्य॒ । प॒शून् । घातु॑कः । स्या॒त् । यत् । न । उ॒प॒तृ॒न्द्यादित्यु॑प - तृ॒न्द्यात् । अय॑ता । स्या॒त् । अ॒न्यया᳚ । उ॒प॒तृ॒णत्तीत्यु॑प - तृ॒णत्ति॑ । अ॒न्यया᳚ । न । धृत्यै᳚ । घृ॒तेन॑ । द्या॒वा॒पृ॒थि॒वी॒ इति॑ द्यावा-पृ॒थि॒वी॒ । प्रेति॑ । ऊ॒र्ण्वा॒था॒म् । इति॑ । आ॒ह॒ । द्यावा॑पृथि॒वी इति॒ द्यावा᳚ - पृ॒थि॒वी । ए॒व । रसे॑न । अ॒न॒क्ति॒ । अच्छि॑न्नः ।  \newline


\textbf{Krama Paata} \newline

द्वेष्टि॒ यः । यश्च॑ । चै॒न॒म् । ए॒न॒म् द्वेष्टि॑ । द्वेष्टि॒ तौ । तावु॒भौ । उ॒भाव॑ध॒मम् । अ॒ध॒मम् तमः॑ । तमो॑ नयति । न॒य॒ती॒षे । इ॒षे त्वा᳚ । त्वेति॑ । इति॑ व॒पाम् । व॒पामुत् । उत् खि॑दति । खि॒द॒ती॒च्छते᳚ । इ॒च्छत॑ इव । इ॒व॒ हि । ह्ये॑षः । ए॒ष यः । यो यज॑ते । यज॑ते॒ यत् । यदु॑पतृ॒न्द्यात् । उ॒प॒तृ॒न्द्याद् रु॒द्रः । उ॒प॒तृ॒न्द्यादित्यु॑प - तृ॒न्द्यात् । रु॒द्रो᳚ऽस्य । अ॒स्य॒ प॒शून् । प॒शून् घातु॑कः । घातु॑कः स्यात् । स्या॒द् यत् । यन् न । नोप॑तृ॒न्द्यात् । उ॒प॒तृ॒न्द्यादय॑ता । उ॒प॒तृ॒न्द्यादित्यु॑प - तृ॒न्द्यात् । अय॑ता स्यात् । स्या॒द॒न्यया᳚ । अ॒न्ययोप॑तृ॒णत्ति॑ । उ॒प॒तृ॒णत्य॒न्यया᳚ । उ॒प॒तृ॒णत्तीत्यु॑प - तृ॒णत्ति॑ । अ॒न्यया॒ न । न धृत्यै᳚ । धृत्यै॑ घृ॒तेन॑ । घृ॒तेन॑ द्यावापृथिवी । द्या॒वा॒पृ॒थि॒वी॒ प्र । द्या॒वा॒पृ॒थि॒वी॒ इति॑ द्यावा - पृ॒थि॒वी॒ । प्रोर्ण्वा॑थाम् । ऊ॒र्ण्वा॒था॒मिति॑ । इत्या॑ह । आ॒ह॒ द्यावा॑पृथि॒वी । द्यावा॑पृथि॒वी ए॒व । द्यावा॑पृथि॒वी इति॒ द्यावा᳚ - पृ॒थि॒वी । ए॒व रसे॑न । रसे॑नानक्ति । अ॒न॒क्त्यच्छि॑न्नः । अच्छि॑न्नो॒ रायः॑ \newline

\textbf{Jatai Paata} \newline

1. द्वेष्टि॒ यो यो द्वेष्टि॒ द्वेष्टि॒ यः । \newline
2. यश्च॑ च॒ यो यश्च॑ । \newline
3. चै॒न॒ मे॒न॒म् च॒ चै॒न॒म् । \newline
4. ए॒न॒म् द्वेष्टि॒ द्वेष्ट्ये॑न मेन॒म् द्वेष्टि॑ । \newline
5. द्वेष्टि॒ तौ तौ द्वेष्टि॒ द्वेष्टि॒ तौ । \newline
6. ता वु॒भा वु॒भौ तौ ता वु॒भौ । \newline
7. उ॒भा व॑ध॒म म॑ध॒म मु॒भा वु॒भा व॑ध॒मम् । \newline
8. अ॒ध॒मम् तम॒ स्तमो॑ ऽध॒म म॑ध॒मम् तमः॑ । \newline
9. तमो॑ नयति नयति॒ तम॒ स्तमो॑ नयति । \newline
10. न॒य॒ ती॒ष इ॒षे न॑यति नय ती॒षे । \newline
11. इ॒षे त्वा᳚ त्वे॒ष इ॒षे त्वा᳚ । \newline
12. त्वेतीति॑ त्वा॒ त्वेति॑ । \newline
13. इति॑ व॒पां ॅव॒पा मितीति॑ व॒पाम् । \newline
14. व॒पा मुदुद् व॒पां ॅव॒पा मुत् । \newline
15. उत् खि॑दति खिद॒ त्युदुत् खि॑दति । \newline
16. खि॒द॒ ती॒च्छत॑ इ॒च्छते॑ खिदति खिद ती॒च्छते᳚ । \newline
17. इ॒च्छत॑ इवे वे॒च्छत॑ इ॒च्छत॑ इव । \newline
18. इ॒व॒ हि हीवे॑व॒ हि । \newline
19. ह्ये॑ष ए॒ष हि ह्ये॑षः । \newline
20. ए॒ष यो य ए॒ष ए॒ष यः । \newline
21. यो यज॑ते॒ यज॑ते॒ यो यो यज॑ते । \newline
22. यज॑ते॒ यद् यद् यज॑ते॒ यज॑ते॒ यत् । \newline
23. यदु॑पतृ॒न्द्या दु॑पतृ॒न्द्याद् यद् यदु॑पतृ॒न्द्यात् । \newline
24. उ॒प॒तृ॒न्द्याद् रु॒द्रो रु॒द्र उ॑पतृ॒न्द्या दु॑पतृ॒न्द्याद् रु॒द्रः । \newline
25. उ॒प॒तृ॒न्द्यादित्यु॑प - तृ॒न्द्यात् । \newline
26. रु॒द्रो᳚ ऽस्यास्य रु॒द्रो रु॒द्रो᳚ ऽस्य । \newline
27. अ॒स्य॒ प॒शून् प॒शून॑ स्यास्य प॒शून् । \newline
28. प॒शून् घातु॑को॒ घातु॑कः प॒शून् प॒शून् घातु॑कः । \newline
29. घातु॑कः स्याथ् स्या॒द् घातु॑को॒ घातु॑कः स्यात् । \newline
30. स्या॒द् यद् यथ् स्या᳚थ् स्या॒द् यत् । \newline
31. यन् न न यद् यन् न । \newline
32. नोप॑तृ॒न्द्या दु॑पतृ॒न्द्यान् न नोप॑तृ॒न्द्यात् । \newline
33. उ॒प॒तृ॒न्द्या दय॒ता ऽय॑तोपतृ॒न्द्या दु॑पतृ॒न्द्या दय॑ता । \newline
34. उ॒प॒तृ॒न्द्यादित्यु॑प - तृ॒न्द्यात् । \newline
35. अय॑ता स्याथ् स्या॒ दय॒ता ऽय॑ता स्यात् । \newline
36. स्या॒ द॒न्यया॒ ऽन्यया᳚ स्याथ् स्या द॒न्यया᳚ । \newline
37. अ॒न्ययो॑पतृ॒ण त्त्यु॑पतृ॒ण त्त्य॒न्यया॒ ऽन्ययो॑पतृ॒णत्ति॑ । \newline
38. उ॒प॒तृ॒ण त्त्य॒न्यया॒ ऽन्ययो॑पतृ॒ण त्त्यु॑पतृ॒ण त्त्य॒न्यया᳚ । \newline
39. उ॒प॒तृ॒णत्तीत्यु॑प - तृ॒णत्ति॑ । \newline
40. अ॒न्यया॒ न नान्यया॒ ऽन्यया॒ न । \newline
41. न धृत्यै॒ धृत्यै॒ न न धृत्यै᳚ । \newline
42. धृत्यै॑ घृ॒तेन॑ घृ॒तेन॒ धृत्यै॒ धृत्यै॑ घृ॒तेन॑ । \newline
43. घृ॒तेन॑ द्यावापृथिवी द्यावापृथिवी घृ॒तेन॑ घृ॒तेन॑ द्यावापृथिवी । \newline
44. द्या॒वा॒पृ॒थि॒वी॒ प्र प्र द्या॑वापृथिवी द्यावापृथिवी॒ प्र । \newline
45. द्या॒वा॒पृ॒थि॒वी॒ इति॑ द्यावा - पृ॒थि॒वी॒ । \newline
46. प्रोर्ण्वा॑था मूर्ण्वाथा॒म् प्र प्रोर्ण्वा॑थाम् । \newline
47. ऊ॒र्ण्वा॒था॒ मिती त्यू᳚र्ण्वाथा मूर्ण्वाथा॒ मिति॑ । \newline
48. इत्या॑हा॒हे तीत्या॑ह । \newline
49. आ॒ह॒ द्यावा॑पृथि॒वी द्यावा॑पृथि॒वी आ॑हाह॒ द्यावा॑पृथि॒वी । \newline
50. द्यावा॑पृथि॒वी ए॒वैव द्यावा॑पृथि॒वी द्यावा॑पृथि॒वी ए॒व । \newline
51. द्यावा॑पृथि॒वी इति॒ द्यावा᳚ - पृ॒थि॒वी । \newline
52. ए॒व रसे॑न॒ रसे॑नै॒ वैव रसे॑न । \newline
53. रसे॑ना नक् त्यनक्ति॒ रसे॑न॒ रसे॑नानक्ति । \newline
54. अ॒न॒क् त्यच्छि॒न्नो ऽच्छि॑न्नो ऽनक् त्यन॒क् त्यच्छि॑न्नः । \newline
55. अच्छि॑न्नो॒ रायो॒ रायो ऽच्छि॒न्नो ऽच्छि॑न्नो॒ रायः॑ । \newline

\textbf{Ghana Paata } \newline

1. द्वेष्टि॒ यो यो द्वेष्टि॒ द्वेष्टि॒ यश्च॑ च॒ यो द्वेष्टि॒ द्वेष्टि॒ यश्च॑ । \newline
2. यश्च॑ च॒ यो यश्चै॑न मेनम् च॒ यो यश्चै॑नम् । \newline
3. चै॒न॒ मे॒न॒म् च॒ चै॒न॒म् द्वेष्टि॒ द्वेष्ट्ये॑नम् च चैन॒म् द्वेष्टि॑ । \newline
4. ए॒न॒म् द्वेष्टि॒ द्वेष्ट्ये॑न मेन॒म् द्वेष्टि॒ तौ तौ द्वेष्ट्ये॑न मेन॒म् द्वेष्टि॒ तौ । \newline
5. द्वेष्टि॒ तौ तौ द्वेष्टि॒ द्वेष्टि॒ ता वु॒भा वु॒भौ तौ द्वेष्टि॒ द्वेष्टि॒ ता वु॒भौ । \newline
6. ता वु॒भा वु॒भौ तौ ता वु॒भा व॑ध॒म म॑ध॒म मु॒भौ तौ ता वु॒भा व॑ध॒मम् । \newline
7. उ॒भा व॑ध॒म म॑ध॒म मु॒भा वु॒भा व॑ध॒मम् तम॒ स्तमो॑ ऽध॒म मु॒भा वु॒भा व॑ध॒मम् तमः॑ । \newline
8. अ॒ध॒मम् तम॒ स्तमो॑ ऽध॒म म॑ध॒मम् तमो॑ नयति नयति॒ तमो॑ ऽध॒म म॑ध॒मम् तमो॑ नयति । \newline
9. तमो॑ नयति नयति॒ तम॒ स्तमो॑ नयती॒ष इ॒षे न॑यति॒ तम॒ स्तमो॑ नयती॒षे । \newline
10. न॒य॒ती॒ष इ॒षे न॑यति नयती॒षे त्वा᳚ त्वे॒षे न॑यति नयती॒षे त्वा᳚ । \newline
11. इ॒षे त्वा᳚ त्वे॒ष इ॒षे त्वेतीति॑ त्वे॒ष इ॒षे त्वेति॑ । \newline
12. त्वेतीति॑ त्वा॒ त्वेति॑ व॒पां ॅव॒पा मिति॑ त्वा॒ त्वेति॑ व॒पाम् । \newline
13. इति॑ व॒पां ॅव॒पा मितीति॑ व॒पा मुदुद् व॒पा मितीति॑ व॒पा मुत् । \newline
14. व॒पा मुदुद् व॒पां ॅव॒पा मुत् खि॑दति खिद॒ त्युद् व॒पां ॅव॒पा मुत् खि॑दति । \newline
15. उत् खि॑दति खिद॒ त्युदुत् खि॑दती॒च्छत॑ इ॒च्छते॑ खिद॒ त्युदुत् खि॑दती॒च्छते᳚ । \newline
16. खि॒द॒ती॒च्छत॑ इ॒च्छते॑ खिदति खिदती॒च्छत॑ इवेवे॒च्छते॑ खिदति खिदती॒च्छत॑ इव । \newline
17. इ॒च्छत॑ इवेवे॒च्छत॑ इ॒च्छत॑ इव॒ हि हीवे॒च्छत॑ इ॒च्छत॑ इव॒ हि । \newline
18. इ॒व॒ हि हीवे॑व॒ ह्ये॑ष ए॒ष हीवे॑व॒ ह्ये॑षः । \newline
19. ह्ये॑ष ए॒ष हि ह्ये॑ष यो य ए॒ष हि ह्ये॑ष यः । \newline
20. ए॒ष यो य ए॒ष ए॒ष यो यज॑ते॒ यज॑ते॒ य ए॒ष ए॒ष यो यज॑ते । \newline
21. यो यज॑ते॒ यज॑ते॒ यो यो यज॑ते॒ यद् यद् यज॑ते॒ यो यो यज॑ते॒ यत् । \newline
22. यज॑ते॒ यद् यद् यज॑ते॒ यज॑ते॒ यदु॑पतृ॒न्द्या दु॑पतृ॒न्द्याद् यद् यज॑ते॒ यज॑ते॒ यदु॑पतृ॒न्द्यात् । \newline
23. यदु॑पतृ॒न्द्या दु॑पतृ॒न्द्याद् यद् यदु॑पतृ॒न्द्याद् रु॒द्रो रु॒द्र उ॑पतृ॒न्द्याद् यद् यदु॑पतृ॒न्द्याद् रु॒द्रः । \newline
24. उ॒प॒तृ॒न्द्याद् रु॒द्रो रु॒द्र उ॑पतृ॒न्द्या दु॑पतृ॒न्द्याद् रु॒द्रो᳚ ऽस्यास्य रु॒द्र उ॑पतृ॒न्द्या दु॑पतृ॒न्द्याद् रु॒द्रो᳚ ऽस्य । \newline
25. उ॒प॒तृ॒न्द्यादित्यु॑प - तृ॒न्द्यात् । \newline
26. रु॒द्रो᳚ ऽस्यास्य रु॒द्रो रु॒द्रो᳚ ऽस्य प॒शून् प॒शू न॑स्य रु॒द्रो रु॒द्रो᳚ ऽस्य प॒शून् । \newline
27. अ॒स्य॒ प॒शून् प॒शून॑ स्यास्य प॒शून् घातु॑को॒ घातु॑कः प॒शून॑ स्यास्य प॒शून् घातु॑कः । \newline
28. प॒शून् घातु॑को॒ घातु॑कः प॒शून् प॒शून् घातु॑कः स्याथ् स्या॒द् घातु॑कः प॒शून् प॒शून् घातु॑कः स्यात् । \newline
29. घातु॑कः स्याथ् स्या॒द् घातु॑को॒ घातु॑कः स्या॒द् यद् यथ् स्या॒द् घातु॑को॒ घातु॑कः स्या॒द् यत् । \newline
30. स्या॒द् यद् यथ् स्या᳚थ् स्या॒द् यन् न न यथ् स्या᳚थ् स्या॒द् यन् न । \newline
31. यन् न न यद् यन् नोप॑तृ॒न्द्या दु॑पतृ॒न्द्यान् न यद् यन् नोप॑तृ॒न्द्यात् । \newline
32. नोप॑तृ॒न्द्या दु॑पतृ॒न्द्यान् न नोप॑तृ॒न्द्या दय॒ता ऽय॑तोपतृ॒न्द्यान् न नोप॑तृ॒न्द्या दय॑ता । \newline
33. उ॒प॒तृ॒न्द्या दय॒ता ऽय॑तोपतृ॒न्द्या दु॑पतृ॒न्द्या दय॑ता स्याथ् स्या॒ दय॑तोपतृ॒न्द्या दु॑पतृ॒न्द्या दय॑ता स्यात् । \newline
34. उ॒प॒तृ॒न्द्यादित्यु॑प - तृ॒न्द्यात् । \newline
35. अय॑ता स्याथ् स्या॒ दय॒ता ऽय॑ता स्या द॒न्यया॒ ऽन्यया᳚ स्या॒ दय॒ता ऽय॑ता स्या द॒न्यया᳚ । \newline
36. स्या॒ द॒न्यया॒ ऽन्यया᳚ स्याथ् स्या द॒न्ययो॑ पतृ॒ण त्त्यु॑पतृ॒ण त्त्य॒न्यया᳚ स्याथ् स्याद॒न्ययो॑ पतृ॒णत्ति॑ । \newline
37. अ॒न्य यो॑पतृ॒ण त्त्यु॑पतृ॒ण त्त्य॒न्यया॒ ऽन्ययो॑पतृ॒ण त्त्य॒न्यया॒ ऽन्ययो॑ पतृ॒ण त्त्य॒न्यया॒ ऽन्ययो॑पतृ॒ण त्त्य॒न्यया᳚ । \newline
38. उ॒प॒तृ॒ण त्त्य॒न्यया॒ ऽन्ययो॑ पतृ॒ण त्त्यु॑पतृ॒ण त्त्य॒न्यया॒ न नान्ययो॑ पतृ॒ण त्त्यु॑पतृ॒णत् त्य॒न्यया॒ न । \newline
39. उ॒प॒तृ॒णत्तीत्यु॑प - तृ॒णत्ति॑ । \newline
40. अ॒न्यया॒ न नान्यया॒ ऽन्यया॒ न धृत्यै॒ धृत्यै॒ नान्यया॒ ऽन्यया॒ न धृत्यै᳚ । \newline
41. न धृत्यै॒ धृत्यै॒ न न धृत्यै॑ घृ॒तेन॑ घृ॒तेन॒ धृत्यै॒ न न धृत्यै॑ घृ॒तेन॑ । \newline
42. धृत्यै॑ घृ॒तेन॑ घृ॒तेन॒ धृत्यै॒ धृत्यै॑ घृ॒तेन॑ द्यावापृथिवी द्यावापृथिवी घृ॒तेन॒ धृत्यै॒ धृत्यै॑ घृ॒तेन॑ द्यावापृथिवी । \newline
43. घृ॒तेन॑ द्यावापृथिवी द्यावापृथिवी घृ॒तेन॑ घृ॒तेन॑ द्यावापृथिवी॒ प्र प्र द्या॑वापृथिवी घृ॒तेन॑ घृ॒तेन॑ द्यावापृथिवी॒ प्र । \newline
44. द्या॒वा॒पृ॒थि॒वी॒ प्र प्र द्या॑वापृथिवी द्यावापृथिवी॒ प्रोर्ण्वा॑था मूर्ण्वाथा॒म् प्र द्या॑वापृथिवी द्यावापृथिवी॒ प्रोर्ण्वा॑थाम् । \newline
45. द्या॒वा॒पृ॒थि॒वी॒ इति॑ द्यावा - पृ॒थि॒वी॒ । \newline
46. प्रोर्ण्वा॑था मूर्ण्वाथा॒म् प्र प्रोर्ण्वा॑था॒ मिती त्यू᳚र्ण्वाथा॒म् प्र प्रोर्ण्वा॑था॒ मिति॑ । \newline
47. ऊ॒र्ण्वा॒था॒ मितीत्यू᳚र्ण्वाथा मूर्ण्वाथा॒ मित्या॑हा॒हे त्यू᳚र्ण्वाथा मूर्ण्वाथा॒ मित्या॑ह । \newline
48. इत्या॑हा॒हे तीत्या॑ह॒ द्यावा॑पृथि॒वी द्यावा॑पृथि॒वी आ॒हे तीत्या॑ह॒ द्यावा॑पृथि॒वी । \newline
49. आ॒ह॒ द्यावा॑पृथि॒वी द्यावा॑पृथि॒वी आ॑हाह॒ द्यावा॑पृथि॒वी ए॒वैव द्यावा॑पृथि॒वी आ॑हाह॒ द्यावा॑पृथि॒वी ए॒व । \newline
50. द्यावा॑पृथि॒वी ए॒वैव द्यावा॑पृथि॒वी द्यावा॑पृथि॒वी ए॒व रसे॑न॒ रसे॑नै॒व द्यावा॑पृथि॒वी द्यावा॑पृथि॒वी ए॒व रसे॑न । \newline
51. द्यावा॑पृथि॒वी इति॒ द्यावा᳚ - पृ॒थि॒वी । \newline
52. ए॒व रसे॑न॒ रसे॑नै॒ वैव रसे॑नानक् त्यनक्ति॒ रसे॑नै॒ वैव रसे॑नानक्ति । \newline
53. रसे॑नानक् त्यनक्ति॒ रसे॑न॒ रसे॑नान॒क् त्यच्छि॒न्नो ऽच्छि॑न्नो ऽनक्ति॒ रसे॑न॒ रसे॑नान॒क् त्यच्छि॑न्नः । \newline
54. अ॒न॒क् त्यच्छि॒न्नो ऽच्छि॑न्नो ऽनक्त्यन॒क् त्यच्छि॑न्नो॒ रायो॒ रायो ऽच्छि॑न्नो ऽनक्त्यन॒क् त्यच्छि॑न्नो॒ रायः॑ । \newline
55. अच्छि॑न्नो॒ रायो॒ रायो ऽच्छि॒न्नो ऽच्छि॑न्नो॒ रायः॑ सु॒वीरः॑ सु॒वीरो॒ रायो ऽच्छि॒न्नो ऽच्छि॑न्नो॒ रायः॑ सु॒वीरः॑ । \newline
\pagebreak
\markright{ TS 6.3.9.4  \hfill https://www.vedavms.in \hfill}

\section{ TS 6.3.9.4 }

\textbf{TS 6.3.9.4 } \newline
\textbf{Samhita Paata} \newline

रायः॑ सु॒वीर॒ इत्या॑ह यथाय॒जुरे॒वैतत् क्रू॒रमि॑व॒ वा ए॒तत् क॑रोति॒ यद् व॒पा-मु॑त्खि॒द-त्यु॒र्व॑न्तरि॑क्ष॒-मन्वि॒हीत्या॑ह॒ शान्त्यै॒ प्र वा ए॒षो᳚ऽस्माल्लो॒काच्च्य॑वते॒ यः प॒शुं मृ॒त्यवे॑ नी॒यमा॑नमन्वा॒रभ॑ते वपा॒श्रप॑णी॒ पुन॑र॒न्वार॑भते॒ऽस्मिन्ने॒व लो॒के प्रति॑ तिष्ठत्य॒ग्निना॑ पु॒रस्ता॑देति॒ रक्ष॑सा॒मप॑हत्या॒ अथो॑ दे॒वता॑ ए॒व ह॒व्येना- [  ] \newline

\textbf{Pada Paata} \newline

रायः॑ । सु॒वीर॒ इति॑ सु - वीरः॑ । इति॑ । आ॒ह॒ । य॒था॒य॒जुरिति॑ यथा - य॒जुः । ए॒व । ए॒तत् । क्रू॒रम् । इ॒व॒ । वै । ए॒तत् । क॒रो॒ति॒ । यत् । व॒पाम् । उ॒त्खि॒दतीत्यु॑त् - खि॒दति॑ । उ॒रु । अ॒न्तरि॑क्षम् । अन्विति॑ । इ॒हि॒ । इति॑ । आ॒ह॒ । शान्त्यै᳚ । प्रेति॑ । वै । ए॒षः । अ॒स्मात् । लो॒कात् । च्य॒व॒ते॒ । यः । प॒शुम् । मृ॒त्यवे᳚ । नी॒यमा॑नम् । अ॒न्वा॒रभ॑त॒ इत्य॑नु - आ॒रभ॑ते । व॒पा॒श्रप॑णी॒ इति॑ वपा - श्रप॑णी । पुनः॑ । अ॒न्वार॑भत॒ इत्य॑नु - आर॑भते । अ॒स्मिन्न् । ए॒व । लो॒के । प्रतीति॑ । ति॒ष्ठ॒ति॒ । अ॒ग्निना᳚ । पु॒रस्ता᳚त् । ए॒ति॒ । रक्ष॑साम् । अप॑हत्या॒ इत्यप॑-ह॒त्यै॒ । अथो॒ इति॑ । दे॒वताः᳚ । ए॒व । ह॒व्येन॑ ।  \newline


\textbf{Krama Paata} \newline

रायः॑ सु॒वीरः॑ । सु॒वीर॒ इति॑ । सु॒वीर॒ इति॑ सु - वीरः॑ । इत्या॑ह । आ॒ह॒ य॒था॒य॒जुः । य॒था॒य॒जुरे॒व । य॒था॒य॒जुरिति॑ यथा - य॒जुः । ए॒वैतत् । ए॒तत् क्रू॒रम् । क्रू॒रमि॑व । इ॒व॒ वै । वा ए॒तत् । ए॒तत् क॑रोति । क॒रो॒ति॒ यत् । यद् व॒पाम् । व॒पामु॑त्खि॒दति॑ । उ॒त्खि॒दत्यु॒रु । उ॒त्खि॒दतीत्यु॑त् - खि॒दति॑ । उ॒र्व॑न्तरि॑क्षम् । अ॒न्तरि॑क्ष॒मनु॑ । अन्वि॑हि । इ॒हीति॑ । इत्या॑ह । आ॒ह॒ शान्त्यै᳚ । शान्त्यै॒ प्र । प्र वै । वा ए॒षः । ए॒षो᳚ऽस्मात् । अ॒स्माल्लो॒कात् । लो॒काच् च्य॑वते । च्य॒व॒ते॒ यः । यः प॒शुम् । प॒शुम् मृ॒त्यवे᳚ । मृ॒त्यवे॑ नी॒यमा॑नम् । नी॒यमा॑नमन्वा॒रभ॑ते । अ॒न्वा॒रभ॑ते वपा॒श्रप॑णी । अ॒न्वा॒रभ॑त॒ इत्य॑नु - आ॒रभ॑ते । व॒पा॒श्रप॑णी॒ पुनः॑ । व॒पा॒श्रप॑णी॒ इति॑ वपा - श्रप॑णी । पुन॑र॒न्वार॑भते । अ॒न्वार॑भते॒ऽस्मिन्न् । अ॒न्वार॑भत॒ इत्य॑नु - आर॑भते । अ॒स्मिन्ने॒व । ए॒व लो॒के । लो॒के प्रति॑ । प्रति॑ तिष्ठति । ति॒ष्ठ॒त्य॒ग्निना᳚ । अ॒ग्निना॑ पु॒रस्ता᳚त् । पु॒रस्ता॑देति । ए॒ति॒ रक्ष॑साम् । रक्ष॑सा॒मप॑हत्यै । अप॑हत्या॒ अथो᳚ । अप॑हत्या॒ इत्यप॑ - ह॒त्यै॒ । अथो॑ दे॒वताः᳚ । अथो॒ इत्यथो᳚ । दे॒वता॑ ए॒व । ए॒व ह॒व्येन॑ । ह॒व्येनानु॑ \newline

\textbf{Jatai Paata} \newline

1. रायः॑ सु॒वीरः॑ सु॒वीरो॒ रायो॒ रायः॑ सु॒वीरः॑ । \newline
2. सु॒वीर॒ इतीति॑ सु॒वीरः॑ सु॒वीर॒ इति॑ । \newline
3. सु॒वीर॒ इति॑ सु - वीरः॑ । \newline
4. इत्या॑हा॒हे तीत्या॑ह । \newline
5. आ॒ह॒ य॒था॒य॒जुर् य॑थाय॒जु रा॑हाह यथाय॒जुः । \newline
6. य॒था॒य॒जु रे॒वैव य॑थाय॒जुर् य॑थाय॒जु रे॒व । \newline
7. य॒था॒य॒जुरिति॑ यथा - य॒जुः । \newline
8. ए॒वैत दे॒त दे॒वै वैतत् । \newline
9. ए॒तत् क्रू॒रम् क्रू॒र मे॒त दे॒तत् क्रू॒रम् । \newline
10. क्रू॒र मि॑वेव क्रू॒रम् क्रू॒र मि॑व । \newline
11. इ॒व॒ वै वा इ॑वेव॒ वै । \newline
12. वा ए॒त दे॒तद् वै वा ए॒तत् । \newline
13. ए॒तत् क॑रोति करो त्ये॒त दे॒तत् क॑रोति । \newline
14. क॒रो॒ति॒ यद् यत् क॑रोति करोति॒ यत् । \newline
15. यद् व॒पां ॅव॒पां ॅयद् यद् व॒पाम् । \newline
16. व॒पा मु॑त्खि॒द त्यु॑त्खि॒दति॑ व॒पां ॅव॒पा मु॑त्खि॒दति॑ । \newline
17. उ॒त्खि॒द त्यु॒रू᳚(1॒) रू᳚त्खि॒द त्यु॑त्खि॒द त्यु॒रु । \newline
18. उ॒त्खि॒दतीत्यु॑त् - खि॒दति॑ । \newline
19. उ॒र्व॑न्तरि॑क्ष म॒न्तरि॑क्ष मु॒रू᳚(1॒)र्व॑न्तरि॑क्षम् । \newline
20. अ॒न्तरि॑क्ष॒ मन्वन् व॒न्तरि॑क्ष म॒न्तरि॑क्ष॒ मनु॑ । \newline
21. अन्वि॑ ही॒ह्यन् वन्वि॑हि । \newline
22. इ॒ही तीती॑ही॒ हीति॑ । \newline
23. इत्या॑हा॒हे तीत्या॑ह । \newline
24. आ॒ह॒ शान्त्यै॒ शान्त्या॑ आहाह॒ शान्त्यै᳚ । \newline
25. शान्त्यै॒ प्र प्र शान्त्यै॒ शान्त्यै॒ प्र । \newline
26. प्र वै वै प्र प्र वै । \newline
27. वा ए॒ष ए॒ष वै वा ए॒षः । \newline
28. ए॒षो᳚ ऽस्माद॒स्मा दे॒ष ए॒षो᳚ ऽस्मात् । \newline
29. अ॒स्मा ल्लो॒का ल्लो॒का द॒स्मा द॒स्मा ल्लो॒कात् । \newline
30. लो॒काच् च्य॑वते च्यवते लो॒का ल्लो॒काच् च्य॑वते । \newline
31. च्य॒व॒ते॒ यो यश्च्य॑वते च्यवते॒ यः । \newline
32. यः प॒शुम् प॒शुं ॅयो यः प॒शुम् । \newline
33. प॒शुम् मृ॒त्यवे॑ मृ॒त्यवे॑ प॒शुम् प॒शुम् मृ॒त्यवे᳚ । \newline
34. मृ॒त्यवे॑ नी॒यमा॑नम् नी॒यमा॑नम् मृ॒त्यवे॑ मृ॒त्यवे॑ नी॒यमा॑नम् । \newline
35. नी॒यमा॑न मन्वा॒रभ॑ते ऽन्वा॒रभ॑ते नी॒यमा॑नम् नी॒यमा॑न मन्वा॒रभ॑ते । \newline
36. अ॒न्वा॒रभ॑ते वपा॒श्रप॑णी वपा॒श्रप॑णी अन्वा॒रभ॑ते ऽन्वा॒रभ॑ते वपा॒श्रप॑णी । \newline
37. अ॒न्वा॒रभ॑त॒ इत्य॑नु - आ॒रभ॑ते । \newline
38. व॒पा॒श्रप॑णी॒ पुनः॒ पुन॑र् वपा॒श्रप॑णी वपा॒श्रप॑णी॒ पुनः॑ । \newline
39. व॒पा॒श्रप॑णी॒ इति॑ वपा - श्रप॑णी । \newline
40. पुन॑ र॒न्वार॑भते॒ ऽन्वार॑भते॒ पुनः॒ पुन॑ र॒न्वार॑भते । \newline
41. अ॒न्वार॑भते॒ ऽस्मिन् न॒स्मिन्न॒ न्वार॑भते॒ ऽन्वार॑भते॒ ऽस्मिन्न् । \newline
42. अ॒न्वार॑भत॒ इत्य॑नु - आर॑भते । \newline
43. अ॒स्मिन् ने॒वै वास्मिन् न॒स्मिन् ने॒व । \newline
44. ए॒व लो॒के लो॒क ए॒वैव लो॒के । \newline
45. लो॒के प्रति॒ प्रति॑ लो॒के लो॒के प्रति॑ । \newline
46. प्रति॑ तिष्ठति तिष्ठति॒ प्रति॒ प्रति॑ तिष्ठति । \newline
47. ति॒ष्ठ॒ त्य॒ग्निना॒ ऽग्निना॑ तिष्ठति तिष्ठ त्य॒ग्निना᳚ । \newline
48. अ॒ग्निना॑ पु॒रस्ता᳚त् पु॒रस्ता॑ द॒ग्निना॒ ऽग्निना॑ पु॒रस्ता᳚त् । \newline
49. पु॒रस्ता॑ देत्येति पु॒रस्ता᳚त् पु॒रस्ता॑ देति । \newline
50. ए॒ति॒ रक्ष॑साꣳ॒॒ रक्ष॑सा मेत्येति॒ रक्ष॑साम् । \newline
51. रक्ष॑सा॒ मप॑हत्या॒ अप॑हत्यै॒ रक्ष॑साꣳ॒॒ रक्ष॑सा॒ मप॑हत्यै । \newline
52. अप॑हत्या॒ अथो॒ अथो॒ अप॑हत्या॒ अप॑हत्या॒ अथो᳚ । \newline
53. अप॑हत्या॒ इत्यप॑ - ह॒त्यै॒ । \newline
54. अथो॑ दे॒वता॑ दे॒वता॒ अथो॒ अथो॑ दे॒वताः᳚ । \newline
55. अथो॒ इत्यथो᳚ । \newline
56. दे॒वता॑ ए॒वैव दे॒वता॑ दे॒वता॑ ए॒व । \newline
57. ए॒व ह॒व्येन॑ ह॒व्येनै॒ वैव ह॒व्येन॑ । \newline
58. ह॒व्येनान् वनु॑ ह॒व्येन॑ ह॒व्ये नानु॑ । \newline

\textbf{Ghana Paata } \newline

1. रायः॑ सु॒वीरः॑ सु॒वीरो॒ रायो॒ रायः॑ सु॒वीर॒ इतीति॑ सु॒वीरो॒ रायो॒ रायः॑ सु॒वीर॒ इति॑ । \newline
2. सु॒वीर॒ इतीति॑ सु॒वीरः॑ सु॒वीर॒ इत्या॑हा॒ हेति॑ सु॒वीरः॑ सु॒वीर॒ इत्या॑ह । \newline
3. सु॒वीर॒ इति॑ सु - वीरः॑ । \newline
4. इत्या॑हा॒हे तीत्या॑ह यथाय॒जुर् य॑थाय॒जु रा॒हे तीत्या॑ह यथाय॒जुः । \newline
5. आ॒ह॒ य॒था॒य॒जुर् य॑थाय॒जु रा॑हाह यथाय॒जु रे॒वैव य॑थाय॒जु रा॑हाह यथाय॒जु रे॒व । \newline
6. य॒था॒य॒जु रे॒वैव य॑थाय॒जुर् य॑थाय॒जु रे॒वैत दे॒त दे॒व य॑थाय॒जुर् य॑थाय॒जु रे॒वैतत् । \newline
7. य॒था॒य॒जुरिति॑ यथा - य॒जुः । \newline
8. ए॒वैत दे॒त दे॒वै वैतत् क्रू॒रम् क्रू॒र मे॒त दे॒वै वैतत् क्रू॒रम् । \newline
9. ए॒तत् क्रू॒रम् क्रू॒र मे॒त दे॒तत् क्रू॒र मि॑वेव क्रू॒र मे॒त दे॒तत् क्रू॒र मि॑व । \newline
10. क्रू॒र मि॑वेव क्रू॒रम् क्रू॒र मि॑व॒ वै वा इ॑व क्रू॒रम् क्रू॒र मि॑व॒ वै । \newline
11. इ॒व॒ वै वा इ॑वेव॒ वा ए॒त दे॒तद् वा इ॑वेव॒ वा ए॒तत् । \newline
12. वा ए॒त दे॒तद् वै वा ए॒तत् क॑रोति करो त्ये॒तद् वै वा ए॒तत् क॑रोति । \newline
13. ए॒तत् क॑रोति करो त्ये॒त दे॒तत् क॑रोति॒ यद् यत् क॑रो त्ये॒त दे॒तत् क॑रोति॒ यत् । \newline
14. क॒रो॒ति॒ यद् यत् क॑रोति करोति॒ यद् व॒पां ॅव॒पां ॅयत् क॑रोति करोति॒ यद् व॒पाम् । \newline
15. यद् व॒पां ॅव॒पां ॅयद् यद् व॒पा मु॑त्खि॒द त्यु॑त्खि॒दति॑ व॒पां ॅयद् यद् व॒पा मु॑त्खि॒दति॑ । \newline
16. व॒पा मु॑त्खि॒द त्यु॑त्खि॒दति॑ व॒पां ॅव॒पा मु॑त्खि॒द त्यु॒रू᳚(1॒) रू᳚त्खि॒दति॑ व॒पां ॅव॒पा मु॑त्खि॒द त्यु॒रु । \newline
17. उ॒त्खि॒द त्यु॒रू᳚(1॒) रू᳚त्खि॒द त्यु॑त्खि॒द त्यु॒र्व॑न्तरि॑क्ष म॒न्तरि॑क्ष मु॒रू᳚त्खि॒द त्यु॑त्खि॒द त्यु॒र्व॑न्तरि॑क्षम् । \newline
18. उ॒त्खि॒दतीत्यु॑त् - खि॒दति॑ । \newline
19. उ॒र्व॑न्तरि॑क्ष म॒न्तरि॑क्ष मु॒रू᳚(1॒)र्व॑न्तरि॑क्ष॒ मन्वन् व॒न्तरि॑क्ष मु॒रू᳚(1॒)र्व॑न्तरि॑क्ष॒ मनु॑ । \newline
20. अ॒न्तरि॑क्ष॒ मन् वन् व॒न्तरि॑क्ष म॒न्तरि॑क्ष॒ मन्वि॑ ही॒ह्य न्व॒न्तरि॑क्ष म॒न्तरि॑क्ष॒ मन्वि॑हि । \newline
21. अन्वि॑ ही॒ह्यन् वन् वि॒हीतीती॒ ह्यन् वन् वि॒हीति॑ । \newline
22. इ॒हीती ती॑ही॒ही त्या॑हा॒हे ती॑ही॒ हीत्या॑ह । \newline
23. इत्या॑हा॒हे तीत्या॑ह॒ शान्त्यै॒ शान्त्या॑ आ॒हे तीत्या॑ह॒ शान्त्यै᳚ । \newline
24. आ॒ह॒ शान्त्यै॒ शान्त्या॑ आहाह॒ शान्त्यै॒ प्र प्र शान्त्या॑ आहाह॒ शान्त्यै॒ प्र । \newline
25. शान्त्यै॒ प्र प्र शान्त्यै॒ शान्त्यै॒ प्र वै वै प्र शान्त्यै॒ शान्त्यै॒ प्र वै । \newline
26. प्र वै वै प्र प्र वा ए॒ष ए॒ष वै प्र प्र वा ए॒षः । \newline
27. वा ए॒ष ए॒ष वै वा ए॒षो᳚ ऽस्मा द॒स्मा दे॒ष वै वा ए॒षो᳚ ऽस्मात् । \newline
28. ए॒षो᳚ ऽस्मा द॒स्मा दे॒ष ए॒षो᳚ ऽस्मा ल्लो॒का ल्लो॒का द॒स्मा दे॒ष ए॒षो᳚ ऽस्मा ल्लो॒कात् । \newline
29. अ॒स्मा ल्लो॒का ल्लो॒का द॒स्मा द॒स्मा ल्लो॒काच् च्य॑वते च्यवते लो॒का द॒स्मा द॒स्मा ल्लो॒काच् च्य॑वते । \newline
30. लो॒काच् च्य॑वते च्यवते लो॒का ल्लो॒काच् च्य॑वते॒ यो यश्च्य॑वते लो॒का ल्लो॒काच् च्य॑वते॒ यः । \newline
31. च्य॒व॒ते॒ यो यश्च्य॑वते च्यवते॒ यः प॒शुम् प॒शुं ॅयश्च्य॑वते च्यवते॒ यः प॒शुम् । \newline
32. यः प॒शुम् प॒शुं ॅयो यः प॒शुम् मृ॒त्यवे॑ मृ॒त्यवे॑ प॒शुं ॅयो यः प॒शुम् मृ॒त्यवे᳚ । \newline
33. प॒शुम् मृ॒त्यवे॑ मृ॒त्यवे॑ प॒शुम् प॒शुम् मृ॒त्यवे॑ नी॒यमा॑नम् नी॒यमा॑नम् मृ॒त्यवे॑ प॒शुम् प॒शुम् मृ॒त्यवे॑ नी॒यमा॑नम् । \newline
34. मृ॒त्यवे॑ नी॒यमा॑नम् नी॒यमा॑नम् मृ॒त्यवे॑ मृ॒त्यवे॑ नी॒यमा॑न मन्वा॒रभ॑ते ऽन्वा॒रभ॑ते नी॒यमा॑नम् मृ॒त्यवे॑ मृ॒त्यवे॑ नी॒यमा॑न मन्वा॒रभ॑ते । \newline
35. नी॒यमा॑न मन्वा॒रभ॑ते ऽन्वा॒रभ॑ते नी॒यमा॑नम् नी॒यमा॑न मन्वा॒रभ॑ते वपा॒श्रप॑णी वपा॒श्रप॑णी अन्वा॒रभ॑ते नी॒यमा॑नम् नी॒यमा॑न मन्वा॒रभ॑ते वपा॒श्रप॑णी । \newline
36. अ॒न्वा॒रभ॑ते वपा॒श्रप॑णी वपा॒श्रप॑णी अन्वा॒रभ॑ते ऽन्वा॒रभ॑ते वपा॒श्रप॑णी॒ पुनः॒ पुन॑र् वपा॒श्रप॑णी अन्वा॒रभ॑ते ऽन्वा॒रभ॑ते वपा॒श्रप॑णी॒ पुनः॑ । \newline
37. अ॒न्वा॒रभ॑त॒ इत्य॑नु - आ॒रभ॑ते । \newline
38. व॒पा॒श्रप॑णी॒ पुनः॒ पुन॑र् वपा॒श्रप॑णी वपा॒श्रप॑णी॒ पुन॑ र॒न्वार॑भते॒ ऽन्वार॑भते॒ पुन॑र् वपा॒श्रप॑णी वपा॒श्रप॑णी॒ पुन॑ र॒न्वार॑भते । \newline
39. व॒पा॒श्रप॑णी॒ इति॑ वपा - श्रप॑णी । \newline
40. पुन॑ र॒न्वार॑भते॒ ऽन्वार॑भते॒ पुनः॒ पुन॑ र॒न्वार॑भते॒ ऽस्मिन् न॒स्मिन् न॒न्वार॑भते॒ पुनः॒ पुन॑ र॒न्वार॑भते॒ ऽस्मिन्न् । \newline
41. अ॒न्वार॑भते॒ ऽस्मिन् न॒स्मिन् न॒न्वार॑भते॒ ऽन्वार॑भते॒ ऽस्मिन् ने॒वै वास्मिन् न॒न्वार॑भते॒ ऽन्वार॑भते॒ ऽस्मिन् ने॒व । \newline
42. अ॒न्वार॑भत॒ इत्य॑नु - आर॑भते । \newline
43. अ॒स्मिन् ने॒वै वास्मिन् न॒स्मिन् ने॒व लो॒के लो॒क ए॒वास्मिन् न॒स्मिन् ने॒व लो॒के । \newline
44. ए॒व लो॒के लो॒क ए॒वैव लो॒के प्रति॒ प्रति॑ लो॒क ए॒वैव लो॒के प्रति॑ । \newline
45. लो॒के प्रति॒ प्रति॑ लो॒के लो॒के प्रति॑ तिष्ठति तिष्ठति॒ प्रति॑ लो॒के लो॒के प्रति॑ तिष्ठति । \newline
46. प्रति॑ तिष्ठति तिष्ठति॒ प्रति॒ प्रति॑ तिष्ठ त्य॒ग्निना॒ ऽग्निना॑ तिष्ठति॒ प्रति॒ प्रति॑ तिष्ठ त्य॒ग्निना᳚ । \newline
47. ति॒ष्ठ॒ त्य॒ग्निना॒ ऽग्निना॑ तिष्ठति तिष्ठ त्य॒ग्निना॑ पु॒रस्ता᳚त् पु॒रस्ता॑ द॒ग्निना॑ तिष्ठति तिष्ठ त्य॒ग्निना॑ पु॒रस्ता᳚त् । \newline
48. अ॒ग्निना॑ पु॒रस्ता᳚त् पु॒रस्ता॑ द॒ग्निना॒ ऽग्निना॑ पु॒रस्ता॑ देत्येति पु॒रस्ता॑ द॒ग्निना॒ ऽग्निना॑ पु॒रस्ता॑ देति । \newline
49. पु॒रस्ता॑ देत्येति पु॒रस्ता᳚त् पु॒रस्ता॑ देति॒ रक्ष॑साꣳ॒॒ रक्ष॑सा मेति पु॒रस्ता᳚त् पु॒रस्ता॑ देति॒ रक्ष॑साम् । \newline
50. ए॒ति॒ रक्ष॑साꣳ॒॒ रक्ष॑सा मेत्येति॒ रक्ष॑सा॒ मप॑हत्या॒ अप॑हत्यै॒ रक्ष॑सा मेत्येति॒ रक्ष॑सा॒ मप॑हत्यै । \newline
51. रक्ष॑सा॒ मप॑हत्या॒ अप॑हत्यै॒ रक्ष॑साꣳ॒॒ रक्ष॑सा॒ मप॑हत्या॒ अथो॒ अथो॒ अप॑हत्यै॒ रक्ष॑साꣳ॒॒ रक्ष॑सा॒ मप॑हत्या॒ अथो᳚ । \newline
52. अप॑हत्या॒ अथो॒ अथो॒ अप॑हत्या॒ अप॑हत्या॒ अथो॑ दे॒वता॑ दे॒वता॒ अथो॒ अप॑हत्या॒ अप॑हत्या॒ अथो॑ दे॒वताः᳚ । \newline
53. अप॑हत्या॒ इत्यप॑ - ह॒त्यै॒ । \newline
54. अथो॑ दे॒वता॑ दे॒वता॒ अथो॒ अथो॑ दे॒वता॑ ए॒वैव दे॒वता॒ अथो॒ अथो॑ दे॒वता॑ ए॒व । \newline
55. अथो॒ इत्यथो᳚ । \newline
56. दे॒वता॑ ए॒वैव दे॒वता॑ दे॒वता॑ ए॒व ह॒व्येन॑ ह॒व्येनै॒व दे॒वता॑ दे॒वता॑ ए॒व ह॒व्येन॑ । \newline
57. ए॒व ह॒व्येन॑ ह॒व्येनै॒ वैव ह॒व्येना न्वनु॑ ह॒व्येनै॒ वैव ह॒व्येनानु॑ । \newline
58. ह॒व्येना न्वनु॑ ह॒व्येन॑ ह॒व्येनान् वे᳚त्ये॒ त्यनु॑ ह॒व्येन॑ ह॒व्येना न्वे॑ति । \newline
\pagebreak
\markright{ TS 6.3.9.5  \hfill https://www.vedavms.in \hfill}

\section{ TS 6.3.9.5 }

\textbf{TS 6.3.9.5 } \newline
\textbf{Samhita Paata} \newline

-न्वे॑ति॒ नान्त॒ममङ्गा॑र॒मति॑ हरे॒द् यद॑न्त॒ममङ्गा॑रमति॒ हरे᳚द्-दे॒वता॒ अति॑ मन्येत॒ वायो॒ वीहि॑ स्तो॒काना॒मित्या॑ह॒ तस्मा॒द् विभ॑क्ताः स्तो॒का अव॑ पद्य॒न्तेऽग्रं॒ ॅवा ए॒तत् प॑शू॒नां ॅयद् व॒पाऽग्र॒मोष॑धीनां ब॒र्॒.हिरग्रे॑णै॒वाग्रꣳ॒॒ सम॑र्द्धय॒त्यथो॒ ओष॑धीष्वे॒व प॒शून् प्रति॑ष्ठापयति॒ स्वाहा॑कृतीभ्यः॒ प्रेष्येत्या॑ह- [  ] \newline

\textbf{Pada Paata} \newline

अन्विति॑ । ए॒ति॒ । न । अ॒न्त॒मम् । अङ्गा॑रम् । अतीति॑ । ह॒रे॒त् । यत् । अ॒न्त॒मम् । अङ्गा॑रम् । अ॒ति॒हरे॒दित्य॑ति - हरे᳚त् । दे॒वताः᳚ । अतीति॑ । म॒न्ये॒त॒ । वायो॒ इति॑ । वीति॑ । इ॒हि॒ । स्तो॒काना᳚म् । इति॑ । आ॒ह॒ । तस्मा᳚त् । विभ॑क्ता॒ इति॒ वि - भ॒क्ताः॒ । स्तो॒काः । अवेति॑ । प॒द्य॒न्ते॒ । अग्र᳚म् । वै । ए॒तत् । प॒शू॒नाम् । यत् । व॒पा । अग्र᳚म् । ओष॑धीनाम् । ब॒र्॒.हिः । अग्रे॑ण । ए॒व । अग्र᳚म् । समिति॑ । अ॒द्‌र्ध॒य॒ति॒ । अथो॒ इति॑ । ओष॑धीषु । ए॒व । प॒शून् । प्रतीति॑ । स्था॒प॒य॒ति॒ । स्वाहा॑कृतीभ्य॒ इति॒ स्वाहा॑कृति - भ्यः॒ । प्रेति॑ । इ॒ष्य॒ । इति॑ । आ॒ह॒ ।  \newline


\textbf{Krama Paata} \newline

अन्वे॑ति । ए॒ति॒ न । नान्त॒मम् । अ॒न्त॒ममङ्‍गा॑रम् । अङ्‍गा॑र॒मति॑ । अति॑ हरेत् । ह॒रे॒द् यत् । यद॑न्त॒मम् । अ॒न्त॒ममङ्‍गा॑रम् । अङ्‍गा॑रमति॒हरे᳚त् । अ॒ति॒हरे᳚द् दे॒वताः᳚ । अ॒ति॒हरे॒दित्य॑ति - हरे᳚त् । दे॒वता॒ अति॑ । अति॑ मन्येत । म॒न्ये॒त॒ वायो᳚ । वायो॒ वि । वायो॒ इति॒ वायो᳚ । वीहि॑ । इ॒हि॒ स्तो॒काना᳚म् । स्तो॒काना॒मिति॑ । इत्या॑ह । आ॒ह॒ तस्मा᳚त् । तस्मा॒द् विभ॑क्ताः । विभ॑क्ताः स्तो॒काः । विभ॑क्ता॒ इति॒ वि - भ॒क्ताः॒ । स्तो॒का अव॑ । अव॑ पद्यन्ते । प॒द्य॒न्तेऽग्र᳚म् । अग्र॒म् ॅवै । वा ए॒तत् । ए॒तत् प॑शू॒नाम् । प॒शू॒नाम् ॅयत् । यद् व॒पा । व॒पाऽग्र᳚म् । अग्र॒मोष॑धीनाम् । ओष॑धीनाम् ब॒र्.॒हिः । ब॒र्.॒हिरग्रे॑ण । अग्रे॑णै॒व । ए॒वाग्र᳚म् । अग्रꣳ॒॒ सम् । सम॑र्द्धयति । अ॒र्द्ध॒य॒त्यथो᳚ । अथो॒ ओष॑धीषु । अथो॒ इत्यथो᳚ । ओष॑धीष्वे॒व । ए॒व प॒शून् । प॒शून् प्रति॑ । प्रति॑ ष्ठापयति । स्था॒प॒य॒ति॒ स्वाहा॑कृतीभ्यः । स्वाहा॑कृतीभ्यः॒ प्र । स्वाहा॑कृतीभ्य॒ इति॒ स्वाहा॑कृति - भ्यः॒ । प्रेष्य॑ । इ॒ष्येति॑ । इत्या॑ह ( ) । आ॒ह॒ य॒ज्ञ्स्य॑ \newline

\textbf{Jatai Paata} \newline

1. अन्वे᳚त्ये॒ त्यन् वन् वे॑ति । \newline
2. ए॒ति॒ न नैत्ये॑ति॒ न । \newline
3. नान्त॒म म॑न्त॒मन् न नान्त॒मम् । \newline
4. अ॒न्त॒म मङ्गा॑र॒ मङ्गा॑र मन्त॒म म॑न्त॒म मङ्गा॑रम् । \newline
5. अङ्गा॑र॒ मत्य त्यङ्गा॑र॒ मङ्गा॑र॒ मति॑ । \newline
6. अति॑ हरे द्धरे॒ दत्यति॑ हरेत् । \newline
7. ह॒रे॒द् यद् य द्ध॑रे द्धरे॒द् यत् । \newline
8. यद॑न्त॒म म॑न्त॒मं ॅयद् यद॑न्त॒मम् । \newline
9. अ॒न्त॒म मङ्गा॑र॒ मङ्गा॑र मन्त॒म म॑न्त॒म मङ्गा॑रम् । \newline
10. अङ्गा॑र मति॒हरे॑ दति॒हरे॒ दङ्गा॑र॒ मङ्गा॑र मति॒हरे᳚त् । \newline
11. अ॒ति॒हरे᳚द् दे॒वता॑ दे॒वता॑ अति॒हरे॑ दति॒हरे᳚द् दे॒वताः᳚ । \newline
12. अ॒ति॒हरे॒दित्य॑ति - हरे᳚त् । \newline
13. दे॒वता॒ अत्यति॑ दे॒वता॑ दे॒वता॒ अति॑ । \newline
14. अति॑ मन्येत मन्ये॒ तात्यति॑ मन्येत । \newline
15. म॒न्ये॒त॒ वायो॒ वायो॑ मन्येत मन्येत॒ वायो᳚ । \newline
16. वायो॒ वि वि वायो॒ वायो॒ वि । \newline
17. वायो॒ इति॒ वायो᳚ । \newline
18. वीही॑हि॒ वि वीहि॑ । \newline
19. इ॒हि॒ स्तो॒कानाꣳ॑ स्तो॒काना॑ मिहीहि स्तो॒काना᳚म् । \newline
20. स्तो॒काना॒ मितीति॑ स्तो॒कानाꣳ॑ स्तो॒काना॒ मिति॑ । \newline
21. इत्या॑हा॒हे तीत्या॑ह । \newline
22. आ॒ह॒ तस्मा॒त् तस्मा॑दा हाह॒ तस्मा᳚त् । \newline
23. तस्मा॒द् विभ॑क्ता॒ विभ॑क्ता॒ स्तस्मा॒त् तस्मा॒द् विभ॑क्ताः । \newline
24. विभ॑क्ताः स्तो॒काः स्तो॒का विभ॑क्ता॒ विभ॑क्ताः स्तो॒काः । \newline
25. विभ॑क्ता॒ इति॒ वि - भ॒क्ताः॒ । \newline
26. स्तो॒का अवाव॑ स्तो॒काः स्तो॒का अव॑ । \newline
27. अव॑ पद्यन्ते पद्य॒न्ते ऽवाव॑ पद्यन्ते । \newline
28. प॒द्य॒न्ते ऽग्र॒ मग्र॑म् पद्यन्ते पद्य॒न्ते ऽग्र᳚म् । \newline
29. अग्रं॒ ॅवै वा अग्र॒ मग्रं॒ ॅवै । \newline
30. वा ए॒त दे॒तद् वै वा ए॒तत् । \newline
31. ए॒तत् प॑शू॒नाम् प॑शू॒ना मे॒त दे॒तत् प॑शू॒नाम् । \newline
32. प॒शू॒नां ॅयद् यत् प॑शू॒नाम् प॑शू॒नां ॅयत् । \newline
33. यद् व॒पा व॒पा यद् यद् व॒पा । \newline
34. व॒पा ऽग्र॒ मग्रं॑ ॅव॒पा व॒पा ऽग्र᳚म् । \newline
35. अग्र॒ मोष॑धीना॒ मोष॑धीना॒ मग्र॒ मग्र॒ मोष॑धीनाम् । \newline
36. ओष॑धीनाम् ब॒र्॒.हिर् ब॒र्॒.हि रोष॑धीना॒ मोष॑धीनाम् ब॒र्॒.हिः । \newline
37. ब॒र्॒.हि रग्रे॒णा ग्रे॑ण ब॒र्॒.हिर् ब॒र्॒.हि रग्रे॑ण । \newline
38. अग्रे॑णै॒ वैवा ग्रे॒णा ग्रे॑णै॒व । \newline
39. ए॒वाग्र॒ मग्र॑ मे॒वै वाग्र᳚म् । \newline
40. अग्रꣳ॒॒ सꣳ स मग्र॒ मग्रꣳ॒॒ सम् । \newline
41. स म॑र्द्धय त्यर्द्धयति॒ सꣳ स म॑र्द्धयति । \newline
42. अ॒र्द्ध॒य॒ त्यथो॒ अथो॑ अर्द्धय त्यर्द्धय॒ त्यथो᳚ । \newline
43. अथो॒ ओष॑धी॒ ष्वोष॑धी॒ ष्वथो॒ अथो॒ ओष॑धीषु । \newline
44. अथो॒ इत्यथो᳚ । \newline
45. ओष॑धी ष्वे॒वै वौष॑धी॒ ष्वोष॑धी ष्वे॒व । \newline
46. ए॒व प॒शून् प॒शूने॒ वैव प॒शून् । \newline
47. प॒शून् प्रति॒ प्रति॑ प॒शून् प॒शून् प्रति॑ । \newline
48. प्रति॑ ष्ठापयति स्थापयति॒ प्रति॒ प्रति॑ ष्ठापयति । \newline
49. स्था॒प॒य॒ति॒ स्वाहा॑कृतीभ्यः॒ स्वाहा॑कृतीभ्यः स्थापयति स्थापयति॒ स्वाहा॑कृतीभ्यः । \newline
50. स्वाहा॑कृतीभ्यः॒ प्र प्र स्वाहा॑कृतीभ्यः॒ स्वाहा॑कृतीभ्यः॒ प्र । \newline
51. स्वाहा॑कृतीभ्य॒ इति॒ स्वाहा॑कृति - भ्यः॒ । \newline
52. प्रेष्ये᳚ ष्य॒ प्र प्रेष्य॑ । \newline
53. इ॒ष्ये तीती᳚ष्ये॒ ष्येति॑ । \newline
54. इत्या॑हा॒हे तीत्या॑ह । \newline
55. आ॒ह॒ य॒ज्ञ्स्य॑ य॒ज्ञ्स्या॑ हाह य॒ज्ञ्स्य॑ । \newline

\textbf{Ghana Paata } \newline

1. अन्वे᳚त्ये॒ त्यन् वन् वे॑ति॒ न नैत्यन् वन् वे॑ति॒ न । \newline
2. ए॒ति॒ न नैत्ये॑ति॒ नान्त॒म म॑न्त॒मन् नैत्ये॑ति॒ नान्त॒मम् । \newline
3. नान्त॒म म॑न्त॒मन् न नान्त॒म मङ्गा॑र॒ मङ्गा॑र मन्त॒मन् न नान्त॒म मङ्गा॑रम् । \newline
4. अ॒न्त॒म मङ्गा॑र॒ मङ्गा॑र मन्त॒म म॑न्त॒म मङ्गा॑र॒ मत्य त्यङ्गा॑र मन्त॒म म॑न्त॒म मङ्गा॑र॒ मति॑ । \newline
5. अङ्गा॑र॒ मत्य त्यङ्गा॑र॒ मङ्गा॑र॒ मति॑ हरे द्धरे॒ दत्यङ्गा॑र॒ मङ्गा॑र॒ मति॑ हरेत् । \newline
6. अति॑ हरे द्धरे॒ दत्यति॑ हरे॒द् यद् यद्ध॑ रे॒दत्यति॑ हरे॒द् यत् । \newline
7. ह॒रे॒द् यद् यद्ध॑रे द्धरे॒द् यद॑न्त॒म म॑न्त॒मं ॅयद्ध॑रे द्धरे॒द् यद॑न्त॒मम् । \newline
8. यद॑न्त॒म म॑न्त॒मं ॅयद् यद॑न्त॒म मङ्गा॑र॒ मङ्गा॑र मन्त॒मं ॅयद् यद॑न्त॒म मङ्गा॑रम् । \newline
9. अ॒न्त॒म मङ्गा॑र॒ मङ्गा॑र मन्त॒म म॑न्त॒म मङ्गा॑र मति॒हरे॑ दति॒हरे॒ दङ्गा॑र मन्त॒म म॑न्त॒म मङ्गा॑र मति॒हरे᳚त् । \newline
10. अङ्गा॑र मति॒हरे॑ दति॒हरे॒ दङ्गा॑र॒ मङ्गा॑र मति॒हरे᳚द् दे॒वता॑ दे॒वता॑ अति॒हरे॒ दङ्गा॑र॒ मङ्गा॑र मति॒हरे᳚द् दे॒वताः᳚ । \newline
11. अ॒ति॒हरे᳚द् दे॒वता॑ दे॒वता॑ अति॒हरे॑ दति॒हरे᳚द् दे॒वता॒ अत्यति॑ दे॒वता॑ अति॒हरे॑ दति॒हरे᳚द् दे॒वता॒ अति॑ । \newline
12. अ॒ति॒हरे॒दित्य॑ति - हरे᳚त् । \newline
13. दे॒वता॒ अत्यति॑ दे॒वता॑ दे॒वता॒ अति॑ मन्येत मन्ये॒ताति॑ दे॒वता॑ दे॒वता॒ अति॑ मन्येत । \newline
14. अति॑ मन्येत मन्ये॒ता त्यति॑ मन्येत॒ वायो॒ वायो॑ मन्ये॒ता त्यति॑ मन्येत॒ वायो᳚ । \newline
15. म॒न्ये॒त॒ वायो॒ वायो॑ मन्येत मन्येत॒ वायो॒ वि वि वायो॑ मन्येत मन्येत॒ वायो॒ वि । \newline
16. वायो॒ वि वि वायो॒ वायो॒ वीही॑हि॒ वि वायो॒ वायो॒ वीहि॑ । \newline
17. वायो॒ इति॒ वायो᳚ । \newline
18. वीही॑हि॒ वि वीहि॑ स्तो॒कानाꣳ॑ स्तो॒काना॑ मिहि॒ वि वीहि॑ स्तो॒काना᳚म् । \newline
19. इ॒हि॒ स्तो॒कानाꣳ॑ स्तो॒काना॑ मिहीहि स्तो॒काना॒ मितीति॑ स्तो॒काना॑ मिहीहि स्तो॒काना॒ मिति॑ । \newline
20. स्तो॒काना॒ मितीति॑ स्तो॒कानाꣳ॑ स्तो॒काना॒ मित्या॑हा॒ हेति॑ स्तो॒कानाꣳ॑ स्तो॒काना॒ मित्या॑ह । \newline
21. इत्या॑हा॒हे तीत्या॑ह॒ तस्मा॒त् तस्मा॑ दा॒हे तीत्या॑ह॒ तस्मा᳚त् । \newline
22. आ॒ह॒ तस्मा॒त् तस्मा॑ दाहाह॒ तस्मा॒द् विभ॑क्ता॒ विभ॑क्ता॒ स्तस्मा॑ दाहाह॒ तस्मा॒द् विभ॑क्ताः । \newline
23. तस्मा॒द् विभ॑क्ता॒ विभ॑क्ता॒ स्तस्मा॒त् तस्मा॒द् विभ॑क्ताः स्तो॒काः स्तो॒का विभ॑क्ता॒ स्तस्मा॒त् तस्मा॒द् विभ॑क्ताः स्तो॒काः । \newline
24. विभ॑क्ताः स्तो॒काः स्तो॒का विभ॑क्ता॒ विभ॑क्ताः स्तो॒का अवाव॑ स्तो॒का विभ॑क्ता॒ विभ॑क्ताः स्तो॒का अव॑ । \newline
25. विभ॑क्ता॒ इति॒ वि - भ॒क्ताः॒ । \newline
26. स्तो॒का अवाव॑ स्तो॒काः स्तो॒का अव॑ पद्यन्ते पद्य॒न्ते ऽव॑ स्तो॒काः स्तो॒का अव॑ पद्यन्ते । \newline
27. अव॑ पद्यन्ते पद्य॒न्ते ऽवाव॑ पद्य॒न्ते ऽग्र॒ मग्र॑म् पद्य॒न्ते ऽवाव॑ पद्य॒न्ते ऽग्र᳚म् । \newline
28. प॒द्य॒न्ते ऽग्र॒ मग्र॑म् पद्यन्ते पद्य॒न्ते ऽग्रं॒ ॅवै वा अग्र॑म् पद्यन्ते पद्य॒न्ते ऽग्रं॒ ॅवै । \newline
29. अग्रं॒ ॅवै वा अग्र॒ मग्रं॒ ॅवा ए॒त दे॒तद् वा अग्र॒ मग्रं॒ ॅवा ए॒तत् । \newline
30. वा ए॒त दे॒तद् वै वा ए॒तत् प॑शू॒नाम् प॑शू॒ना मे॒तद् वै वा ए॒तत् प॑शू॒नाम् । \newline
31. ए॒तत् प॑शू॒नाम् प॑शू॒ना मे॒त दे॒तत् प॑शू॒नां ॅयद् यत् प॑शू॒ना मे॒त दे॒तत् प॑शू॒नां ॅयत् । \newline
32. प॒शू॒नां ॅयद् यत् प॑शू॒नाम् प॑शू॒नां ॅयद् व॒पा व॒पा यत् प॑शू॒नाम् प॑शू॒नां ॅयद् व॒पा । \newline
33. यद् व॒पा व॒पा यद् यद् व॒पा ऽग्र॒ मग्रं॑ ॅव॒पा यद् यद् व॒पा ऽग्र᳚म् । \newline
34. व॒पा ऽग्र॒ मग्रं॑ ॅव॒पा व॒पा ऽग्र॒ मोष॑धीना॒ मोष॑धीना॒ मग्रं॑ ॅव॒पा व॒पा ऽग्र॒ मोष॑धीनाम् । \newline
35. अग्र॒ मोष॑धीना॒ मोष॑धीना॒ मग्र॒ मग्र॒ मोष॑धीनाम् ब॒र्॒.हिर् ब॒र्॒.हि रोष॑धीना॒ मग्र॒ मग्र॒ मोष॑धीनाम् ब॒र्॒.हिः । \newline
36. ओष॑धीनाम् ब॒र्॒.हिर् ब॒र्॒.हि रोष॑धीना॒ मोष॑धीनाम् ब॒र्॒.हि रग्रे॒णा ग्रे॑ण ब॒र्॒.हि रोष॑धीना॒ मोष॑धीनाम् ब॒र्॒.हि रग्रे॑ण । \newline
37. ब॒र्॒.हि रग्रे॒णा ग्रे॑ण ब॒र्॒.हिर् ब॒र्॒.हि रग्रे॑ णै॒वै वाग्रे॑ण ब॒र्॒.हिर् ब॒र्॒.हि रग्रे॑णै॒व । \newline
38. अग्रे॑णै॒वै वाग्रे॒ णाग्रे॑णै॒ वाग्र॒ मग्र॑ मे॒वा ग्रे॒णा ग्रे॑णै॒वाग्र᳚म् । \newline
39. ए॒वाग्र॒ मग्र॑ मे॒वै वाग्रꣳ॒॒ सꣳ स मग्र॑ मे॒वै वाग्रꣳ॒॒ सम् । \newline
40. अग्रꣳ॒॒ सꣳ स मग्र॒ मग्रꣳ॒॒ स म॑र्द्धय त्यर्द्धयति॒ स मग्र॒ मग्रꣳ॒॒ स म॑र्द्धयति । \newline
41. स म॑र्द्धय त्यर्द्धयति॒ सꣳ स म॑र्द्धय॒ त्यथो॒ अथो॑ अर्द्धयति॒ सꣳ स म॑र्द्धय॒ त्यथो᳚ । \newline
42. अ॒र्द्ध॒य॒ त्यथो॒ अथो॑ अर्द्धय त्यर्द्धय॒ त्यथो॒ ओष॑धी॒ ष्वोष॑धी॒ ष्वथो॑ अर्द्धय त्यर्द्धय॒ त्यथो॒ ओष॑धीषु । \newline
43. अथो॒ ओष॑धी॒ ष्वोष॑धी॒ ष्वथो॒ अथो॒ ओष॑धी ष्वे॒वैवौष॑धी॒ ष्वथो॒ अथो॒ ओष॑धी ष्वे॒व । \newline
44. अथो॒ इत्यथो᳚ । \newline
45. ओष॑धी ष्वे॒वैवौष॑धी॒ ष्वोष॑धी ष्वे॒व प॒शून् प॒शूने॒ वौष॑धी॒ ष्वोष॑धी ष्वे॒व प॒शून् । \newline
46. ए॒व प॒शून् प॒शूने॒ वैव प॒शून् प्रति॒ प्रति॑ प॒शूने॒ वैव प॒शून् प्रति॑ । \newline
47. प॒शून् प्रति॒ प्रति॑ प॒शून् प॒शून् प्रति॑ ष्ठापयति स्थापयति॒ प्रति॑ प॒शून् प॒शून् प्रति॑ ष्ठापयति । \newline
48. प्रति॑ ष्ठापयति स्थापयति॒ प्रति॒ प्रति॑ ष्ठापयति॒ स्वाहा॑कृतीभ्यः॒ स्वाहा॑कृतीभ्यः स्थापयति॒ प्रति॒ प्रति॑ ष्ठापयति॒ स्वाहा॑कृतीभ्यः । \newline
49. स्था॒प॒य॒ति॒ स्वाहा॑कृतीभ्यः॒ स्वाहा॑कृतीभ्यः स्थापयति स्थापयति॒ स्वाहा॑कृतीभ्यः॒ प्र प्र स्वाहा॑कृतीभ्यः स्थापयति स्थापयति॒ स्वाहा॑कृतीभ्यः॒ प्र । \newline
50. स्वाहा॑कृतीभ्यः॒ प्र प्र स्वाहा॑कृतीभ्यः॒ स्वाहा॑कृतीभ्यः॒ प्रेष्ये᳚ ष्य॒ प्र स्वाहा॑कृतीभ्यः॒ स्वाहा॑कृतीभ्यः॒ प्रेष्य॑ । \newline
51. स्वाहा॑कृतीभ्य॒ इति॒ स्वाहा॑कृति - भ्यः॒ । \newline
52. प्रेष्ये᳚ ष्य॒ प्र प्रेष्ये तीती᳚ष्य॒ प्र प्रेष्येति॑ । \newline
53. इ॒ष्ये तीती᳚ष्ये॒ ष्ये त्या॑हा॒हे ती᳚ष्ये॒ ष्ये त्या॑ह । \newline
54. इत्या॑हा॒हे तीत्या॑ह य॒ज्ञ्स्य॑ य॒ज्ञ्स्या॒हे तीत्या॑ह य॒ज्ञ्स्य॑ । \newline
55. आ॒ह॒ य॒ज्ञ्स्य॑ य॒ज्ञ्स्या॑ हाह य॒ज्ञ्स्य॒ समि॑ष्ट्यै॒ समि॑ष्ट्यै य॒ज्ञ्स्या॑ हाह य॒ज्ञ्स्य॒ समि॑ष्ट्यै । \newline
\pagebreak
\markright{ TS 6.3.9.6  \hfill https://www.vedavms.in \hfill}

\section{ TS 6.3.9.6 }

\textbf{TS 6.3.9.6 } \newline
\textbf{Samhita Paata} \newline

य॒ज्ञ्स्य॒ समि॑ष्ट्यै प्राणापा॒नौ वा ए॒तौ प॑शू॒नां ॅयत् पृ॑षदा॒ज्यमा॒त्मा व॒पा पृ॑षदा॒ज्यम॑भि॒घार्य॑ व॒पाम॒भि घा॑रयत्या॒त्मन्ने॒व प॑शू॒नां प्रा॑णापा॒नौ द॑धाति॒ स्वाहो॒र्द्ध्वन॑भसं मारु॒तं ग॑च्छत॒मित्या॑हो॒र्द्ध्वन॑भा ह स्म॒ वै मा॑रु॒तो दे॒वानां᳚ ॅवपा॒श्रप॑णी॒ प्र ह॑रति॒ तेनै॒वैने॒ प्र ह॑रति॒ विषू॑ची॒ प्र ह॑रति॒ तस्मा॒द् विष्व॑ञ्चौ प्राणापा॒नौ ॥ \newline

\textbf{Pada Paata} \newline

य॒ज्ञ्स्य॑ । समि॑ष्ट्या॒ इति॒ सं-इ॒ष्ट्यै॒ । प्रा॒णा॒पा॒नाविति॑ प्राण-अ॒पा॒नौ । वै । ए॒तौ । प॒शू॒नाम् । यत् । पृ॒ष॒दा॒ज्यमिति॑ पृषत् - आ॒ज्यम् । आ॒त्मा । व॒पा । पृ॒ष॒दा॒ज्यमिति॑ पृषत् - आ॒ज्यम् । अ॒भि॒घार्येत्य॑भि - घार्य॑ । व॒पाम् । अ॒भीति॑ । घा॒र॒य॒ति॒ । आ॒त्मन् । ए॒व । प॒शू॒नाम् । प्रा॒णा॒पा॒नाविति॑ प्राण - अ॒पा॒नौ । द॒धा॒ति॒ । स्वाहा᳚ । ऊ॒द्‌र्ध्वन॑भस॒मित्यू॒द्‌र्ध्व - न॒भ॒स॒म् । मा॒रु॒तम् । ग॒च्छ॒त॒म् । इति॑ । आ॒ह॒ । ऊ॒द्‌र्ध्वन॑भा॒ इत्यू॒द्‌र्ध्व - न॒भाः॒ । ह॒ । स्म॒ । वै । मा॒रु॒तः । दे॒वाना᳚म् । व॒पा॒श्रप॑णी॒ इति॑ वपा - श्रप॑णी । प्रेति॑ । ह॒र॒ति॒ । तेन॑ । ए॒व । ए॒ने॒ इति॑ । प्रेति॑ । ह॒र॒ति॒ । विषू॑ची॒ इति॑ । प्रेति॑ । ह॒र॒ति॒ । तस्मा᳚त् । विष्व॑ञ्चौ । प्रा॒णा॒पा॒नाविति॑ प्राण - अ॒पा॒नौ ॥  \newline


\textbf{Krama Paata} \newline

य॒ज्ञ्स्य॒ समि॑ष्ट्‍यै । समि॑ष्ट्‍यै प्राणापा॒नौ । समि॑ष्ट्‍या॒ इति॒ सम् - इ॒ष्ट्‍यै॒ । प्रा॒णा॒पा॒नौ वै । प्रा॒णा॒पा॒नाविति॑ प्राण - अ॒पा॒नौ । वा ए॒तौ । ए॒तौ प॑शू॒नाम् । प॒शू॒नाम् ॅयत् । यत् पृ॑षदा॒ज्यम् । पृ॒ष॒दा॒ज्यमा॒त्मा । पृ॒ष॒दा॒ज्यमिति॑ पृषत् - आ॒ज्यम् । आ॒त्मा व॒पा । व॒पा पृ॑षदा॒ज्यम् । पृ॒ष॒दा॒ज्यम॑भि॒घार्य॑ । पृ॒ष॒दा॒ज्यमिति॑ पृषत् - आ॒ज्यम् । अ॒भि॒घार्य॑ व॒पाम् । अ॒भि॒घार्येत्य॑भि - घार्य॑ । व॒पाम॒भि । अ॒भि घा॑रयति । घा॒र॒य॒त्या॒त्मन्न् । आ॒त्मन्ने॒व । ए॒व प॑शू॒नाम् । प॒शू॒नाम् प्रा॑णापा॒नौ । प्रा॒णा॒पा॒नौ द॑धाति । प्रा॒णा॒पा॒नाविति॑ प्राण - अ॒पा॒नौ । द॒धा॒ति॒ स्वाहा᳚ । स्वाहो॒र्द्ध्वन॑भसम् । ऊ॒र्द्ध्वन॑भसम् मारु॒तम् । ऊ॒र्द्ध्वन॑भस॒मित्यू॒र्द्ध्व - न॒भ॒स॒म् । मा॒रु॒तम् ग॑च्छतम् । ग॒च्छ॒त॒मिति॑ । इत्या॑ह । आ॒हो॒र्द्ध्वन॑भाः । ऊ॒र्द्ध्वन॑भा ह । ऊ॒र्द्ध्वन॑भा॒ इत्यू॒र्द्ध्व - न॒भाः॒ । ह॒ स्म॒ । स्म॒ वै । वै मा॑रु॒तः । मा॒रु॒तो दे॒वाना᳚म् । दे॒वाना᳚म् ॅवपा॒श्रप॑णी । व॒पा॒श्रप॑णी॒ प्र । व॒पा॒श्रप॑णी॒ इति॑ वपा - श्रप॑णी । प्र ह॑रति । ह॒र॒ति॒ तेन॑ । तेनै॒व । ए॒वैने᳚ । ए॒ने॒ प्र । ए॒ने॒ इत्ये॑ने । प्र ह॑रति । ह॒र॒ति॒ विषू॑ची । विषू॑ची॒ प्र । विषू॑ची॒ इति॒ विषू॑ची । प्र ह॑रति । ह॒र॒ति॒ तस्मा᳚त् । तस्मा॒द् विष्व॑ञ्चौ । विष्व॑ञ्चौ प्राणापा॒नौ । प्रा॒णा॒पा॒नाविति॑ प्राण - अ॒पा॒नौ । \newline

\textbf{Jatai Paata} \newline

1. य॒ज्ञ्स्य॒ समि॑ष्ट्यै॒ समि॑ष्ट्यै य॒ज्ञ्स्य॑ य॒ज्ञ्स्य॒ समि॑ष्ट्यै । \newline
2. समि॑ष्ट्यै प्राणापा॒नौ प्रा॑णापा॒नौ समि॑ष्ट्यै॒ समि॑ष्ट्यै प्राणापा॒नौ । \newline
3. समि॑ष्ट्या॒ इति॒ सं - इ॒ष्ट्यै॒ । \newline
4. प्रा॒णा॒पा॒नौ वै वै प्रा॑णापा॒नौ प्रा॑णापा॒नौ वै । \newline
5. प्रा॒णा॒पा॒नाविति॑ प्राण - अ॒पा॒नौ । \newline
6. वा ए॒ता वे॒तौ वै वा ए॒तौ । \newline
7. ए॒तौ प॑शू॒नाम् प॑शू॒ना मे॒ता वे॒तौ प॑शू॒नाम् । \newline
8. प॒शू॒नां ॅयद् यत् प॑शू॒नाम् प॑शू॒नां ॅयत् । \newline
9. यत् पृ॑षदा॒ज्यम् पृ॑षदा॒ज्यं ॅयद् यत् पृ॑षदा॒ज्यम् । \newline
10. पृ॒ष॒दा॒ज्य मा॒त्मा ऽऽत्मा पृ॑षदा॒ज्यम् पृ॑षदा॒ज्य मा॒त्मा । \newline
11. पृ॒ष॒दा॒ज्यमिति॑ पृषत् - आ॒ज्यम् । \newline
12. आ॒त्मा व॒पा व॒पा ऽऽत्मा ऽऽत्मा व॒पा । \newline
13. व॒पा पृ॑षदा॒ज्यम् पृ॑षदा॒ज्यं ॅव॒पा व॒पा पृ॑षदा॒ज्यम् । \newline
14. पृ॒ष॒दा॒ज्य म॑भि॒घार्या॑ भि॒घार्य॑ पृषदा॒ज्यम् पृ॑षदा॒ज्य म॑भि॒घार्य॑ । \newline
15. पृ॒ष॒दा॒ज्यमिति॑ पृषत् - आ॒ज्यम् । \newline
16. अ॒भि॒घार्य॑ व॒पां ॅव॒पा म॑भि॒घार्या॑ भि॒घार्य॑ व॒पाम् । \newline
17. अ॒भि॒घार्येत्य॑भि - घार्य॑ । \newline
18. व॒पा म॒भ्य॑भि व॒पां ॅव॒पा म॒भि । \newline
19. अ॒भि घा॑रयति घारय त्य॒भ्य॑भि घा॑रयति । \newline
20. घा॒र॒य॒ त्या॒त्म ना॒त्मन् घा॑रयति घारय त्या॒त्मन् । \newline
21. आ॒त्म ने॒वै वात्म ना॒त्म ने॒व । \newline
22. ए॒व प॑शू॒नाम् प॑शू॒ना मे॒वैव प॑शू॒नाम् । \newline
23. प॒शू॒नाम् प्रा॑णापा॒नौ प्रा॑णापा॒नौ प॑शू॒नाम् प॑शू॒नाम् प्रा॑णापा॒नौ । \newline
24. प्रा॒णा॒पा॒नौ द॑धाति दधाति प्राणापा॒नौ प्रा॑णापा॒नौ द॑धाति । \newline
25. प्रा॒णा॒पा॒नाविति॑ प्राण - अ॒पा॒नौ । \newline
26. द॒धा॒ति॒ स्वाहा॒ स्वाहा॑ दधाति दधाति॒ स्वाहा᳚ । \newline
27. स्वाहो॒र्द्ध्वन॑भस मू॒र्द्ध्वन॑भसꣳ॒॒ स्वाहा॒ स्वाहो॒र्द्ध्वन॑भसम् । \newline
28. ऊ॒र्द्ध्वन॑भसम् मारु॒तम् मा॑रु॒त मू॒र्द्ध्वन॑भस मू॒र्द्ध्वन॑भसम् मारु॒तम् । \newline
29. ऊ॒र्द्ध्वन॑भस॒मित्यू॒र्द्ध्व - न॒भ॒स॒म् । \newline
30. मा॒रु॒तम् ग॑च्छतम् गच्छतम् मारु॒तम् मा॑रु॒तम् ग॑च्छतम् । \newline
31. ग॒च्छ॒त॒ मितीति॑ गच्छतम् गच्छत॒ मिति॑ । \newline
32. इत्या॑हा॒हे तीत्या॑ह । \newline
33. आ॒हो॒र्द्ध्वन॑भा ऊ॒र्द्ध्वन॑भा आहा हो॒र्द्ध्वन॑भाः । \newline
34. ऊ॒र्द्ध्वन॑भा ह हो॒र्द्ध्वन॑भा ऊ॒र्द्ध्वन॑भा ह । \newline
35. ऊ॒र्द्ध्वन॑भा॒ इत्यू॒र्द्ध्व - न॒भाः॒ । \newline
36. ह॒ स्म॒ स्म॒ ह॒ ह॒ स्म॒ । \newline
37. स्म॒ वै वै स्म॑ स्म॒ वै । \newline
38. वै मा॑रु॒तो मा॑रु॒तो वै वै मा॑रु॒तः । \newline
39. मा॒रु॒तो दे॒वाना᳚म् दे॒वाना᳚म् मारु॒तो मा॑रु॒तो दे॒वाना᳚म् । \newline
40. दे॒वानां᳚ ॅवपा॒श्रप॑णी वपा॒श्रप॑णी दे॒वाना᳚म् दे॒वानां᳚ ॅवपा॒श्रप॑णी । \newline
41. व॒पा॒श्रप॑णी॒ प्र प्र व॑पा॒श्रप॑णी वपा॒श्रप॑णी॒ प्र । \newline
42. व॒पा॒श्रप॑णी॒ इति॑ वपा - श्रप॑णी । \newline
43. प्र ह॑रति हरति॒ प्र प्र ह॑रति । \newline
44. ह॒र॒ति॒ तेन॒ तेन॑ हरति हरति॒ तेन॑ । \newline
45. तेनै॒ वैव तेन॒ तेनै॒व । \newline
46. ए॒वैने॑ एने ए॒वै वैने᳚ । \newline
47. ए॒ने॒ प्र प्रैने॑ एने॒ प्र । \newline
48. ए॒ने॒ इत्ये॑ने । \newline
49. प्र ह॑रति हरति॒ प्र प्र ह॑रति । \newline
50. ह॒र॒ति॒ विषू॑ची॒ विषू॑ची हरति हरति॒ विषू॑ची । \newline
51. विषू॑ची॒ प्र प्र विषू॑ची॒ विषू॑ची॒ प्र । \newline
52. विषू॑ची॒ इति॒ विषू॑ची । \newline
53. प्र ह॑रति हरति॒ प्र प्र ह॑रति । \newline
54. ह॒र॒ति॒ तस्मा॒त् तस्मा᳚ द्धरति हरति॒ तस्मा᳚त् । \newline
55. तस्मा॒द् विष्व॑ञ्चौ॒ विष्व॑ञ्चौ॒ तस्मा॒त् तस्मा॒द् विष्व॑ञ्चौ । \newline
56. विष्व॑ञ्चौ प्राणापा॒नौ प्रा॑णापा॒नौ विष्व॑ञ्चौ॒ विष्व॑ञ्चौ प्राणापा॒नौ । \newline
57. प्रा॒णा॒पा॒नाविति॑ प्राण - अ॒पा॒नौ । \newline

\textbf{Ghana Paata } \newline

1. य॒ज्ञ्स्य॒ समि॑ष्ट्यै॒ समि॑ष्ट्यै य॒ज्ञ्स्य॑ य॒ज्ञ्स्य॒ समि॑ष्ट्यै प्राणापा॒नौ प्रा॑णापा॒नौ समि॑ष्ट्यै य॒ज्ञ्स्य॑ य॒ज्ञ्स्य॒ समि॑ष्ट्यै प्राणापा॒नौ । \newline
2. समि॑ष्ट्यै प्राणापा॒नौ प्रा॑णापा॒नौ समि॑ष्ट्यै॒ समि॑ष्ट्यै प्राणापा॒नौ वै वै प्रा॑णापा॒नौ समि॑ष्ट्यै॒ समि॑ष्ट्यै प्राणापा॒नौ वै । \newline
3. समि॑ष्ट्या॒ इति॒ सं - इ॒ष्ट्यै॒ । \newline
4. प्रा॒णा॒पा॒नौ वै वै प्रा॑णापा॒नौ प्रा॑णापा॒नौ वा ए॒ता वे॒तौ वै प्रा॑णापा॒नौ प्रा॑णापा॒नौ वा ए॒तौ । \newline
5. प्रा॒णा॒पा॒नाविति॑ प्राण - अ॒पा॒नौ । \newline
6. वा ए॒ता वे॒तौ वै वा ए॒तौ प॑शू॒नाम् प॑शू॒ना मे॒तौ वै वा ए॒तौ प॑शू॒नाम् । \newline
7. ए॒तौ प॑शू॒नाम् प॑शू॒ना मे॒ता वे॒तौ प॑शू॒नां ॅयद् यत् प॑शू॒ना मे॒ता वे॒तौ प॑शू॒नां ॅयत् । \newline
8. प॒शू॒नां ॅयद् यत् प॑शू॒नाम् प॑शू॒नां ॅयत् पृ॑षदा॒ज्यम् पृ॑षदा॒ज्यं ॅयत् प॑शू॒नाम् प॑शू॒नां ॅयत् पृ॑षदा॒ज्यम् । \newline
9. यत् पृ॑षदा॒ज्यम् पृ॑षदा॒ज्यं ॅयद् यत् पृ॑षदा॒ज्य मा॒त्मा ऽऽत्मा पृ॑षदा॒ज्यं ॅयद् यत् पृ॑षदा॒ज्य मा॒त्मा । \newline
10. पृ॒ष॒दा॒ज्य मा॒त्मा ऽऽत्मा पृ॑षदा॒ज्यम् पृ॑षदा॒ज्य मा॒त्मा व॒पा व॒पा ऽऽत्मा पृ॑षदा॒ज्यम् पृ॑षदा॒ज्य मा॒त्मा व॒पा । \newline
11. पृ॒ष॒दा॒ज्यमिति॑ पृषत् - आ॒ज्यम् । \newline
12. आ॒त्मा व॒पा व॒पा ऽऽत्मा ऽऽत्मा व॒पा पृ॑षदा॒ज्यम् पृ॑षदा॒ज्यं ॅव॒पा ऽऽत्मा ऽऽत्मा व॒पा पृ॑षदा॒ज्यम् । \newline
13. व॒पा पृ॑षदा॒ज्यम् पृ॑षदा॒ज्यं ॅव॒पा व॒पा पृ॑षदा॒ज्य म॑भि॒घार्या॑ भि॒घार्य॑ पृषदा॒ज्यं ॅव॒पा व॒पा पृ॑षदा॒ज्य म॑भि॒घार्य॑ । \newline
14. पृ॒ष॒दा॒ज्य म॑भि॒घार्या॑ भि॒घार्य॑ पृषदा॒ज्यम् पृ॑षदा॒ज्य म॑भि॒घार्य॑ व॒पां ॅव॒पा म॑भि॒घार्य॑ पृषदा॒ज्यम् पृ॑षदा॒ज्य म॑भि॒घार्य॑ व॒पाम् । \newline
15. पृ॒ष॒दा॒ज्यमिति॑ पृषत् - आ॒ज्यम् । \newline
16. अ॒भि॒घार्य॑ व॒पां ॅव॒पा म॑भि॒घार्या॑ भि॒घार्य॑ व॒पा म॒भ्य॑भि व॒पा म॑भि॒घार्या॑ भि॒घार्य॑ व॒पा म॒भि । \newline
17. अ॒भि॒घार्येत्य॑भि - घार्य॑ । \newline
18. व॒पा म॒भ्य॑भि व॒पां ॅव॒पा म॒भि घा॑रयति घारय त्य॒भि व॒पां ॅव॒पा म॒भि घा॑रयति । \newline
19. अ॒भि घा॑रयति घारय त्य॒भ्य॑भि घा॑रय त्या॒त्म ना॒त्मन् घा॑रय त्य॒भ्य॑भि घा॑रय त्या॒त्मन् । \newline
20. घा॒र॒य॒ त्या॒त्म ना॒त्मन् घा॑रयति घारय त्या॒त्म ने॒वैवात्मन् घा॑रयति घारय त्या॒त्म ने॒व । \newline
21. आ॒त्म ने॒वैवात्म ना॒त्म ने॒व प॑शू॒नाम् प॑शू॒ना मे॒वात्म ना॒त्म ने॒व प॑शू॒नाम् । \newline
22. ए॒व प॑शू॒नाम् प॑शू॒ना मे॒वैव प॑शू॒नाम् प्रा॑णापा॒नौ प्रा॑णापा॒नौ प॑शू॒ना मे॒वैव प॑शू॒नाम् प्रा॑णापा॒नौ । \newline
23. प॒शू॒नाम् प्रा॑णापा॒नौ प्रा॑णापा॒नौ प॑शू॒नाम् प॑शू॒नाम् प्रा॑णापा॒नौ द॑धाति दधाति प्राणापा॒नौ प॑शू॒नाम् प॑शू॒नाम् प्रा॑णापा॒नौ द॑धाति । \newline
24. प्रा॒णा॒पा॒नौ द॑धाति दधाति प्राणापा॒नौ प्रा॑णापा॒नौ द॑धाति॒ स्वाहा॒ स्वाहा॑ दधाति प्राणापा॒नौ प्रा॑णापा॒नौ द॑धाति॒ स्वाहा᳚ । \newline
25. प्रा॒णा॒पा॒नाविति॑ प्राण - अ॒पा॒नौ । \newline
26. द॒धा॒ति॒ स्वाहा॒ स्वाहा॑ दधाति दधाति॒ स्वाहो॒र्द्ध्वन॑भस मू॒र्द्ध्वन॑भसꣳ॒॒ स्वाहा॑ दधाति दधाति॒ स्वाहो॒र्द्ध्वन॑भसम् । \newline
27. स्वाहो॒र्द्ध्वन॑भस मू॒र्द्ध्वन॑भसꣳ॒॒ स्वाहा॒ स्वाहो॒र्द्ध्वन॑भसम् मारु॒तम् मा॑रु॒त मू॒र्द्ध्वन॑भसꣳ॒॒ स्वाहा॒ स्वाहो॒र्द्ध्वन॑भसम् मारु॒तम् । \newline
28. ऊ॒र्द्ध्वन॑भसम् मारु॒तम् मा॑रु॒त मू॒र्द्ध्वन॑भस मू॒र्द्ध्वन॑भसम् मारु॒तम् ग॑च्छतम् गच्छतम् मारु॒त मू॒र्द्ध्वन॑भस मू॒र्द्ध्वन॑भसम् मारु॒तम् ग॑च्छतम् । \newline
29. ऊ॒र्द्ध्वन॑भस॒मित्यू॒र्द्ध्व - न॒भ॒स॒म् । \newline
30. मा॒रु॒तम् ग॑च्छतम् गच्छतम् मारु॒तम् मा॑रु॒तम् ग॑च्छत॒ मितीति॑ गच्छतम् मारु॒तम् मा॑रु॒तम् ग॑च्छत॒ मिति॑ । \newline
31. ग॒च्छ॒त॒ मितीति॑ गच्छतम् गच्छत॒ मित्या॑हा॒ हेति॑ गच्छतम् गच्छत॒ मित्या॑ह । \newline
32. इत्या॑हा॒हे तीत्या॑हो॒र्द्ध्वन॑भा ऊ॒र्द्ध्वन॑भा आ॒हे तीत्या॑हो॒र्द्ध्वन॑भाः । \newline
33. आ॒हो॒र्द्ध्वन॑भा ऊ॒र्द्ध्वन॑भा आहाहो॒र्द्ध्वन॑भा हहो॒र्द्ध्वन॑भा आहाहो॒र्द्ध्वन॑भा ह । \newline
34. ऊ॒र्द्ध्वन॑भा हहो॒र्द्ध्वन॑भा ऊ॒र्द्ध्वन॑भा ह स्म स्म हो॒र्द्ध्वन॑भा ऊ॒र्द्ध्वन॑भा ह स्म । \newline
35. ऊ॒र्द्ध्वन॑भा॒ इत्यू॒र्द्ध्व - न॒भाः॒ । \newline
36. ह॒ स्म॒ स्म॒ ह॒ ह॒ स्म॒ वै वै स्म॑ ह ह स्म॒ वै । \newline
37. स्म॒ वै वै स्म॑ स्म॒ वै मा॑रु॒तो मा॑रु॒तो वै स्म॑ स्म॒ वै मा॑रु॒तः । \newline
38. वै मा॑रु॒तो मा॑रु॒तो वै वै मा॑रु॒तो दे॒वाना᳚म् दे॒वाना᳚म् मारु॒तो वै वै मा॑रु॒तो दे॒वाना᳚म् । \newline
39. मा॒रु॒तो दे॒वाना᳚म् दे॒वाना᳚म् मारु॒तो मा॑रु॒तो दे॒वानां᳚ ॅवपा॒श्रप॑णी वपा॒श्रप॑णी दे॒वाना᳚म् मारु॒तो मा॑रु॒तो दे॒वानां᳚ ॅवपा॒श्रप॑णी । \newline
40. दे॒वानां᳚ ॅवपा॒श्रप॑णी वपा॒श्रप॑णी दे॒वाना᳚म् दे॒वानां᳚ ॅवपा॒श्रप॑णी॒ प्र प्र व॑पा॒श्रप॑णी दे॒वाना᳚म् दे॒वानां᳚ ॅवपा॒श्रप॑णी॒ प्र । \newline
41. व॒पा॒श्रप॑णी॒ प्र प्र व॑पा॒श्रप॑णी वपा॒श्रप॑णी॒ प्र ह॑रति हरति॒ प्र व॑पा॒श्रप॑णी वपा॒श्रप॑णी॒ प्र ह॑रति । \newline
42. व॒पा॒श्रप॑णी॒ इति॑ वपा - श्रप॑णी । \newline
43. प्र ह॑रति हरति॒ प्र प्र ह॑रति॒ तेन॒ तेन॑ हरति॒ प्र प्र ह॑रति॒ तेन॑ । \newline
44. ह॒र॒ति॒ तेन॒ तेन॑ हरति हरति॒ तेनै॒ वैव तेन॑ हरति हरति॒ तेनै॒व । \newline
45. तेनै॒ वैव तेन॒ तेनै॒ वैने॑ एने ए॒व तेन॒ तेनै॒ वैने᳚ । \newline
46. ए॒वैने॑ एने ए॒वै वैने॒ प्र प्रैने॑ ए॒वै वैने॒ प्र । \newline
47. ए॒ने॒ प्र प्रैने॑ एने॒ प्र ह॑रति हरति॒ प्रैने॑ एने॒ प्र ह॑रति । \newline
48. ए॒ने॒ इत्ये॑ने । \newline
49. प्र ह॑रति हरति॒ प्र प्र ह॑रति॒ विषू॑ची॒ विषू॑ची हरति॒ प्र प्र ह॑रति॒ विषू॑ची । \newline
50. ह॒र॒ति॒ विषू॑ची॒ विषू॑ची हरति हरति॒ विषू॑ची॒ प्र प्र विषू॑ची हरति हरति॒ विषू॑ची॒ प्र । \newline
51. विषू॑ची॒ प्र प्र विषू॑ची॒ विषू॑ची॒ प्र ह॑रति हरति॒ प्र विषू॑ची॒ विषू॑ची॒ प्र ह॑रति । \newline
52. विषू॑ची॒ इति॒ विषू॑ची । \newline
53. प्र ह॑रति हरति॒ प्र प्र ह॑रति॒ तस्मा॒त् तस्मा᳚ द्धरति॒ प्र प्र ह॑रति॒ तस्मा᳚त् । \newline
54. ह॒र॒ति॒ तस्मा॒त् तस्मा᳚ द्धरति हरति॒ तस्मा॒द् विष्व॑ञ्चौ॒ विष्व॑ञ्चौ॒ तस्मा᳚ द्धरति हरति॒ तस्मा॒द् विष्व॑ञ्चौ । \newline
55. तस्मा॒द् विष्व॑ञ्चौ॒ विष्व॑ञ्चौ॒ तस्मा॒त् तस्मा॒द् विष्व॑ञ्चौ प्राणापा॒नौ प्रा॑णापा॒नौ विष्व॑ञ्चौ॒ तस्मा॒त् तस्मा॒द् विष्व॑ञ्चौ प्राणापा॒नौ । \newline
56. विष्व॑ञ्चौ प्राणापा॒नौ प्रा॑णापा॒नौ विष्व॑ञ्चौ॒ विष्व॑ञ्चौ प्राणापा॒नौ । \newline
57. प्रा॒णा॒पा॒नाविति॑ प्राण - अ॒पा॒नौ । \newline
\pagebreak
\markright{ TS 6.3.10.1  \hfill https://www.vedavms.in \hfill}

\section{ TS 6.3.10.1 }

\textbf{TS 6.3.10.1 } \newline
\textbf{Samhita Paata} \newline

प॒शुमा॒लभ्य॑ पुरो॒डाशं॒ निर्व॑पति॒ समे॑धमे॒वैन॒मा ल॑भते व॒पया᳚ प्र॒चर्य॑ पुरो॒डाशे॑न॒ प्र च॑र॒त्यूर्ग्वै पु॑रो॒डाश॒ ऊर्ज॑मे॒व प॑शू॒नां म॑द्ध्य॒तो द॑धा॒त्यथो॑ प॒शोरे॒व छि॒द्रमपि॑ दधाति पृषदा॒ज्यस्यो॑प॒हत्य॒ त्रिः पृ॑च्छति शृ॒तꣳ ह॒वी(3)ः श॑मित॒रिति॒ त्रिष॑त्या॒ हि दे॒वा योऽशृ॑तꣳ शृ॒तमाह॒ स एन॑सा प्राणापा॒नौ वा ए॒तौ प॑शू॒नां- [  ] \newline

\textbf{Pada Paata} \newline

प॒शुम् । आ॒लभ्येत्या᳚ - लभ्य॑ । पु॒रो॒डाश᳚म् । निरिति॑ । व॒प॒ति॒ । समे॑ध॒मिति॒ स - मे॒ध॒म् । ए॒व । ए॒न॒म् । एति॑ । ल॒भ॒ते॒ । व॒पया᳚ । प्र॒चर्येति॑ प्र-चर्य॑ । पु॒रो॒डाशे॑न । प्रेति॑ । च॒र॒ति॒ । ऊर्क् । वै । पु॒रो॒डाशः॑ । ऊर्ज᳚म् । ए॒व । प॒शू॒नाम् । म॒द्ध्य॒तः । द॒धा॒ति॒ । अथो॒ इति॑ । प॒शोः । ए॒व । छि॒द्रम् । अपीति॑ । द॒धा॒ति॒ । पृ॒ष॒दा॒ज्यस्येति॑ पृषत् - आ॒ज्यस्य॑ । उ॒प॒हत्येत्यु॑प - हत्य॑ । त्रिः । पृ॒च्छ॒ति॒ । शृ॒तम् । ह॒वी(3)ः । श॒मि॒तः॒ । इति॑ । त्रिष॑त्या॒ इति॒ त्रि - स॒त्याः॒ । हि । दे॒वाः । यः । अशृ॑तम् । शृ॒तम् । आह॑ । सः । एन॑सा । प्रा॒णा॒पा॒नाविति॑ प्राण - अ॒पा॒नौ । वै । ए॒तौ । प॒शू॒नां ।  \newline


\textbf{Krama Paata} \newline

प॒शुमा॒लभ्य॑ । आ॒लभ्य॑ पुरो॒डाश᳚म् । आ॒लभ्येत्या᳚ - लभ्य॑ । पु॒रो॒डाश॒म् निः । निर् व॑पति । व॒प॒ति॒ समे॑धम् । समे॑धमे॒व । समे॑ध॒मिति॒ स - मे॒ध॒म् । ए॒वैन᳚म् । ए॒न॒मा । आ ल॑भते । ल॒भ॒ते॒ व॒पया᳚ । व॒पया᳚ प्र॒चर्य॑ । प्र॒चर्य॑ पुरो॒डाशे॑न । प्र॒चर्येति॑ प्र - चर्य॑ । पु॒रो॒डाशे॑न॒ प्र । प्र च॑रति । च॒र॒त्यूर्क् । ऊर्ग् वै । वै पु॑रो॒डाशः॑ । पु॒रो॒डाश॒ ऊर्ज᳚म् । ऊर्ज॑मे॒व । ए॒व प॑शू॒नाम् । प॒शू॒नाम् म॑द्ध्य॒तः । म॒द्ध्य॒तो द॑धाति । द॒धा॒त्यथो᳚ । अथो॑ प॒शोः । अथो॒ इत्यथो᳚ । प॒शोरे॒व । ए॒व छि॒द्रम् । छि॒द्रमपि॑ । अपि॑ दधाति । द॒धा॒ति॒ पृ॒ष॒दा॒ज्यस्य॑ । पृ॒ष॒दा॒ज्यस्यो॑प॒हत्य॑ । पृ॒ष॒दा॒ज्यस्येति॑ पृषत् - आ॒ज्यस्य॑ । उ॒प॒हत्य॒ त्रिः । उ॒प॒हत्येत्यु॑प - हत्य॑ । त्रिः पृ॑च्छति । पृ॒च्छ॒ति॒ शृ॒तम् । शृ॒तꣳ ह॒वी(3)ः । ह॒वी(3)ः श॑मितः । श॒मि॒त॒रिति॑ । इति॒ त्रिष॑त्याः । त्रिष॑त्या॒ हि । त्रिष॑त्या॒ इति॒ त्रि - स॒त्याः॒ । हि दे॒वाः । दे॒वा यः । योऽशृ॑तम् । अशृ॑तꣳ शृ॒तम् । शृ॒तमाह॑ । आह॒ सः । स एन॑सा । एन॑सा प्राणापा॒नौ । प्रा॒णा॒पा॒नौ वै । प्रा॒णा॒पा॒नाविति॑ प्राण - अ॒पा॒नौ । वा ए॒तौ । ए॒तौ प॑शू॒नाम् । प॒शू॒नाम् ॅयत् \newline

\textbf{Jatai Paata} \newline

1. प॒शु मा॒लभ्या॒ लभ्य॑ प॒शुम् प॒शु मा॒लभ्य॑ । \newline
2. आ॒लभ्य॑ पुरो॒डाश॑म् पुरो॒डाश॑ मा॒लभ्या॒ लभ्य॑ पुरो॒डाश᳚म् । \newline
3. आ॒लभ्येत्या᳚ - लभ्य॑ । \newline
4. पु॒रो॒डाश॒न् निर् णिष् पु॑रो॒डाश॑म् पुरो॒डाश॒न् निः । \newline
5. निर् व॑पति वपति॒ निर् णिर् व॑पति । \newline
6. व॒प॒ति॒ समे॑धꣳ॒॒ समे॑धं ॅवपति वपति॒ समे॑धम् । \newline
7. समे॑ध मे॒वैव समे॑धꣳ॒॒ समे॑ध मे॒व । \newline
8. समे॑ध॒मिति॒ स - मे॒ध॒म् । \newline
9. ए॒वैन॑ मेन मे॒वै वैन᳚म् । \newline
10. ए॒न॒ मैन॑ मेन॒ मा । \newline
11. आ ल॑भते लभत॒ आ ल॑भते । \newline
12. ल॒भ॒ते॒ व॒पया॑ व॒पया॑ लभते लभते व॒पया᳚ । \newline
13. व॒पया᳚ प्र॒चर्य॑ प्र॒चर्य॑ व॒पया॑ व॒पया᳚ प्र॒चर्य॑ । \newline
14. प्र॒चर्य॑ पुरो॒डाशे॑न पुरो॒डाशे॑न प्र॒चर्य॑ प्र॒चर्य॑ पुरो॒डाशे॑न । \newline
15. प्र॒चर्येति॑ प्र - चर्य॑ । \newline
16. पु॒रो॒डाशे॑न॒ प्र प्र पु॑रो॒डाशे॑न पुरो॒डाशे॑न॒ प्र । \newline
17. प्र च॑रति चरति॒ प्र प्र च॑रति । \newline
18. च॒र॒ त्यूर् गूर्क् च॑रति चर॒ त्यूर्क् । \newline
19. ऊर्ग् वै वा ऊर् गूर्ग् वै । \newline
20. वै पु॑रो॒डाशः॑ पुरो॒डाशो॒ वै वै पु॑रो॒डाशः॑ । \newline
21. पु॒रो॒डाश॒ ऊर्ज॒ मूर्ज॑म् पुरो॒डाशः॑ पुरो॒डाश॒ ऊर्ज᳚म् । \newline
22. ऊर्ज॑ मे॒वै वोर्ज॒ मूर्ज॑ मे॒व । \newline
23. ए॒व प॑शू॒नाम् प॑शू॒ना मे॒वैव प॑शू॒नाम् । \newline
24. प॒शू॒नाम् म॑द्ध्य॒तो म॑द्ध्य॒तः प॑शू॒नाम् प॑शू॒नाम् म॑द्ध्य॒तः । \newline
25. म॒द्ध्य॒तो द॑धाति दधाति मद्ध्य॒तो म॑द्ध्य॒तो द॑धाति । \newline
26. द॒धा॒ त्यथो॒ अथो॑ दधाति दधा॒ त्यथो᳚ । \newline
27. अथो॑ प॒शोः प॒शो रथो॒ अथो॑ प॒शोः । \newline
28. अथो॒ इत्यथो᳚ । \newline
29. प॒शो रे॒वैव प॒शोः प॒शो रे॒व । \newline
30. ए॒व छि॒द्रम् छि॒द्र मे॒वैव छि॒द्रम् । \newline
31. छि॒द्र मप्यपि॑ च्छि॒द्रम् छि॒द्र मपि॑ । \newline
32. अपि॑ दधाति दधा॒ त्यप्यपि॑ दधाति । \newline
33. द॒धा॒ति॒ पृ॒ष॒दा॒ज्यस्य॑ पृषदा॒ज्यस्य॑ दधाति दधाति पृषदा॒ज्यस्य॑ । \newline
34. पृ॒ष॒दा॒ज्य स्यो॑प॒ह त्यो॑प॒हत्य॑ पृषदा॒ज्यस्य॑ पृषदा॒ज्य स्यो॑प॒हत्य॑ । \newline
35. पृ॒ष॒दा॒ज्यस्येति॑ पृषत् - आ॒ज्यस्य॑ । \newline
36. उ॒प॒हत्य॒ त्रि स्त्रि रु॑प॒हत्यो॑ प॒हत्य॒ त्रिः । \newline
37. उ॒प॒हत्येत्यु॑प - हत्य॑ । \newline
38. त्रिः पृ॑च्छति पृच्छति॒ त्रि स्त्रिः पृ॑च्छति । \newline
39. पृ॒च्छ॒ति॒ शृ॒तꣳ शृ॒तम् पृ॑च्छति पृच्छति शृ॒तम् । \newline
40. शृ॒तꣳ ह॒वी(3)र्. ह॒वी(3)ः शृ॒तꣳ शृ॒तꣳ ह॒वी(3)ः । \newline
41. ह॒वी(3)ः श॑मितः शमितर्. ह॒वी(3)र्. ह॒वी(3)ः श॑मितः । \newline
42. श॒मि॒त॒ रितीति॑ शमितः शमित॒ रिति॑ । \newline
43. इति॒ त्रिष॑त्या॒ स्त्रिष॑त्या॒ इतीति॒ त्रिष॑त्याः । \newline
44. त्रिष॑त्या॒ हि हि त्रिष॑त्या॒ स्त्रिष॑त्या॒ हि । \newline
45. त्रिष॑त्या॒ इति॒ त्रि - स॒त्याः॒ । \newline
46. हि दे॒वा दे॒वा हि हि दे॒वाः । \newline
47. दे॒वा यो यो दे॒वा दे॒वा यः । \newline
48. यो ऽशृ॑त॒ मशृ॑तं॒ ॅयो यो ऽशृ॑तम् । \newline
49. अशृ॑तꣳ शृ॒तꣳ शृ॒त मशृ॑त॒ मशृ॑तꣳ शृ॒तम् । \newline
50. शृ॒त माहाह॑ शृ॒तꣳ शृ॒त माह॑ । \newline
51. आह॒ स स आहाह॒ सः । \newline
52. स एन॒ सैन॑सा॒ स स एन॑सा । \newline
53. एन॑सा प्राणापा॒नौ प्रा॑णापा॒ना वेन॒ सैन॑सा प्राणापा॒नौ । \newline
54. प्रा॒णा॒पा॒नौ वै वै प्रा॑णापा॒नौ प्रा॑णापा॒नौ वै । \newline
55. प्रा॒णा॒पा॒नाविति॑ प्राण - अ॒पा॒नौ । \newline
56. वा ए॒ता वे॒तौ वै वा ए॒तौ । \newline
57. ए॒तौ प॑शू॒नां प॑शू॒ना मे॒ता वे॒तौ प॑शू॒नां । \newline
58. प॒शू॒नां ॅयद् यत् प॑शू॒नां प॑शू॒नां ॅयत् । \newline

\textbf{Ghana Paata } \newline

1. प॒शु मा॒लभ्या॒ लभ्य॑ प॒शुम् प॒शु मा॒लभ्य॑ पुरो॒डाश॑म् पुरो॒डाश॑ मा॒लभ्य॑ प॒शुम् प॒शु मा॒लभ्य॑ पुरो॒डाश᳚म् । \newline
2. आ॒लभ्य॑ पुरो॒डाश॑म् पुरो॒डाश॑ मा॒लभ्या॒ लभ्य॑ पुरो॒डाश॒न् निर् णिष् पु॑रो॒डाश॑ मा॒लभ्या॒ लभ्य॑ पुरो॒डाश॒न् निः । \newline
3. आ॒लभ्येत्या᳚ - लभ्य॑ । \newline
4. पु॒रो॒डाश॒न् निर् णिष् पु॑रो॒डाश॑म् पुरो॒डाश॒न् निर् व॑पति वपति॒ निष् पु॑रो॒डाश॑म् पुरो॒डाश॒न् निर् व॑पति । \newline
5. निर् व॑पति वपति॒ निर् णिर् व॑पति॒ समे॑धꣳ॒॒ समे॑धं ॅवपति॒ निर् णिर् व॑पति॒ समे॑धम् । \newline
6. व॒प॒ति॒ समे॑धꣳ॒॒ समे॑धं ॅवपति वपति॒ समे॑ध मे॒वैव समे॑धं ॅवपति वपति॒ समे॑ध मे॒व । \newline
7. समे॑ध मे॒वैव समे॑धꣳ॒॒ समे॑ध मे॒वैन॑ मेन मे॒व समे॑धꣳ॒॒ समे॑ध मे॒वैन᳚म् । \newline
8. समे॑ध॒मिति॒ स - मे॒ध॒म् । \newline
9. ए॒वैन॑ मेन मे॒वै वैन॒ मैन॑ मे॒वै वैन॒ मा । \newline
10. ए॒न॒ मैन॑ मेन॒ मा ल॑भते लभत॒ ऐन॑ मेन॒ मा ल॑भते । \newline
11. आ ल॑भते लभत॒ आ ल॑भते व॒पया॑ व॒पया॑ लभत॒ आ ल॑भते व॒पया᳚ । \newline
12. ल॒भ॒ते॒ व॒पया॑ व॒पया॑ लभते लभते व॒पया᳚ प्र॒चर्य॑ प्र॒चर्य॑ व॒पया॑ लभते लभते व॒पया᳚ प्र॒चर्य॑ । \newline
13. व॒पया᳚ प्र॒चर्य॑ प्र॒चर्य॑ व॒पया॑ व॒पया᳚ प्र॒चर्य॑ पुरो॒डाशे॑न पुरो॒डाशे॑न प्र॒चर्य॑ व॒पया॑ व॒पया᳚ प्र॒चर्य॑ पुरो॒डाशे॑न । \newline
14. प्र॒चर्य॑ पुरो॒डाशे॑न पुरो॒डाशे॑न प्र॒चर्य॑ प्र॒चर्य॑ पुरो॒डाशे॑न॒ प्र प्र पु॑रो॒डाशे॑न प्र॒चर्य॑ प्र॒चर्य॑ पुरो॒डाशे॑न॒ प्र । \newline
15. प्र॒चर्येति॑ प्र - चर्य॑ । \newline
16. पु॒रो॒डाशे॑न॒ प्र प्र पु॑रो॒डाशे॑न पुरो॒डाशे॑न॒ प्र च॑रति चरति॒ प्र पु॑रो॒डाशे॑न पुरो॒डाशे॑न॒ प्र च॑रति । \newline
17. प्र च॑रति चरति॒ प्र प्र च॑र॒ त्यूर् गूर्क् च॑रति॒ प्र प्र च॑र॒ त्यूर्क् । \newline
18. च॒र॒ त्यूर् गूर्क् च॑रति चर॒ त्यूर्ग् वै वा ऊर्क् च॑रति चर॒ त्यूर्ग् वै । \newline
19. ऊर्ग् वै वा ऊर्गूर्ग् वै पु॑रो॒डाशः॑ पुरो॒डाशो॒ वा ऊर् गूर्ग् वै पु॑रो॒डाशः॑ । \newline
20. वै पु॑रो॒डाशः॑ पुरो॒डाशो॒ वै वै पु॑रो॒डाश॒ ऊर्ज॒ मूर्ज॑म् पुरो॒डाशो॒ वै वै पु॑रो॒डाश॒ ऊर्ज᳚म् । \newline
21. पु॒रो॒डाश॒ ऊर्ज॒ मूर्ज॑म् पुरो॒डाशः॑ पुरो॒डाश॒ ऊर्ज॑ मे॒वै वोर्ज॑म् पुरो॒डाशः॑ पुरो॒डाश॒ ऊर्ज॑ मे॒व । \newline
22. ऊर्ज॑ मे॒वै वोर्ज॒ मूर्ज॑ मे॒व प॑शू॒नाम् प॑शू॒ना मे॒वोर्ज॒ मूर्ज॑ मे॒व प॑शू॒नाम् । \newline
23. ए॒व प॑शू॒नाम् प॑शू॒ना मे॒वैव प॑शू॒नाम् म॑द्ध्य॒तो म॑द्ध्य॒तः प॑शू॒ना मे॒वैव प॑शू॒नाम् म॑द्ध्य॒तः । \newline
24. प॒शू॒नाम् म॑द्ध्य॒तो म॑द्ध्य॒तः प॑शू॒नाम् प॑शू॒नाम् म॑द्ध्य॒तो द॑धाति दधाति मद्ध्य॒तः प॑शू॒नाम् प॑शू॒नाम् म॑द्ध्य॒तो द॑धाति । \newline
25. म॒द्ध्य॒तो द॑धाति दधाति मद्ध्य॒तो म॑द्ध्य॒तो द॑धा॒ त्यथो॒ अथो॑ दधाति मद्ध्य॒तो म॑द्ध्य॒तो द॑धा॒ त्यथो᳚ । \newline
26. द॒धा॒ त्यथो॒ अथो॑ दधाति दधा॒ त्यथो॑ प॒शोः प॒शो रथो॑ दधाति दधा॒ त्यथो॑ प॒शोः । \newline
27. अथो॑ प॒शोः प॒शो रथो॒ अथो॑ प॒शो रे॒वैव प॒शो रथो॒ अथो॑ प॒शो रे॒व । \newline
28. अथो॒ इत्यथो᳚ । \newline
29. प॒शो रे॒वैव प॒शोः प॒शो रे॒व छि॒द्रम् छि॒द्र मे॒व प॒शोः प॒शो रे॒व छि॒द्रम् । \newline
30. ए॒व छि॒द्रम् छि॒द्र मे॒वैव छि॒द्र मप्यपि॑ च्छि॒द्र मे॒वैव छि॒द्र मपि॑ । \newline
31. छि॒द्र मप्यपि॑ च्छि॒द्रम् छि॒द्र मपि॑ दधाति दधा॒ त्यपि॑ च्छि॒द्रम् छि॒द्र मपि॑ दधाति । \newline
32. अपि॑ दधाति दधा॒ त्यप्यपि॑ दधाति पृषदा॒ज्यस्य॑ पृषदा॒ज्यस्य॑ दधा॒ त्यप्यपि॑ दधाति पृषदा॒ज्यस्य॑ । \newline
33. द॒धा॒ति॒ पृ॒ष॒दा॒ज्यस्य॑ पृषदा॒ज्यस्य॑ दधाति दधाति पृषदा॒ज्यस्यो॑ प॒हत्यो॑ प॒हत्य॑ पृषदा॒ज्यस्य॑ दधाति दधाति पृषदा॒ज्य स्यो॑प॒हत्य॑ । \newline
34. पृ॒ष॒दा॒ज्य स्यो॑प॒हत्यो॑ प॒हत्य॑ पृषदा॒ज्यस्य॑ पृषदा॒ज्य स्यो॑प॒हत्य॒ त्रि स्त्रि रु॑प॒हत्य॑ पृषदा॒ज्यस्य॑ पृषदा॒ज्य स्यो॑प॒हत्य॒ त्रिः । \newline
35. पृ॒ष॒दा॒ज्यस्येति॑ पृषत् - आ॒ज्यस्य॑ । \newline
36. उ॒प॒हत्य॒ त्रि स्त्रि रु॑प॒हत्यो॑ प॒हत्य॒ त्रिः पृ॑च्छति पृच्छति॒ त्रिरु॑ प॒ह त्यो॑प॒हत्य॒ त्रिः पृ॑च्छति । \newline
37. उ॒प॒हत्येत्यु॑प - हत्य॑ । \newline
38. त्रिः पृ॑च्छति पृच्छति॒ त्रि स्त्रिः पृ॑च्छति शृ॒तꣳ शृ॒तम् पृ॑च्छति॒ त्रि स्त्रिः पृ॑च्छति शृ॒तम् । \newline
39. पृ॒च्छ॒ति॒ शृ॒तꣳ शृ॒तम् पृ॑च्छति पृच्छति शृ॒तꣳ ह॒वी(3)र्. ह॒वी(3)ः शृ॒तम् पृ॑च्छति पृच्छति शृ॒तꣳ ह॒वी(3)ः । \newline
40. शृ॒तꣳ ह॒वी(3)र्. ह॒वी(3)ः शृ॒तꣳ शृ॒तꣳ ह॒वी(3)ः श॑मितः शमितर्. ह॒वी(3)ः शृ॒तꣳ शृ॒तꣳ ह॒वी(3)ः श॑मितः । \newline
41. ह॒वी(3)ः श॑मितः शमितर्. ह॒वी(3)र्. ह॒वी(3)ः श॑मित॒ रितीति॑ शमितर्. ह॒वी(3)र्. ह॒वी(3)ः श॑मित॒ रिति॑ । \newline
42. श॒मि॒त॒ रितीति॑ शमितः शमित॒ रिति॒ त्रिष॑त्या॒ स्त्रिष॑त्या॒ इति॑ शमितः शमित॒ रिति॒ त्रिष॑त्याः । \newline
43. इति॒ त्रिष॑त्या॒ स्त्रिष॑त्या॒ इतीति॒ त्रिष॑त्या॒ हि हि त्रिष॑त्या॒ इतीति॒ त्रिष॑त्या॒ हि । \newline
44. त्रिष॑त्या॒ हि हि त्रिष॑त्या॒ स्त्रिष॑त्या॒ हि दे॒वा दे॒वा हि त्रिष॑त्या॒ स्त्रिष॑त्या॒ हि दे॒वाः । \newline
45. त्रिष॑त्या॒ इति॒ त्रि - स॒त्याः॒ । \newline
46. हि दे॒वा दे॒वा हि हि दे॒वा यो यो दे॒वा हि हि दे॒वा यः । \newline
47. दे॒वा यो यो दे॒वा दे॒वा यो ऽशृ॑त॒ मशृ॑तं॒ ॅयो दे॒वा दे॒वा यो ऽशृ॑तम् । \newline
48. यो ऽशृ॑त॒ मशृ॑तं॒ ॅयो यो ऽशृ॑तꣳ शृ॒तꣳ शृ॒त मशृ॑तं॒ ॅयो यो ऽशृ॑तꣳ शृ॒तम् । \newline
49. अशृ॑तꣳ शृ॒तꣳ शृ॒त मशृ॑त॒ मशृ॑तꣳ शृ॒त माहाह॑ शृ॒त मशृ॑त॒ मशृ॑तꣳ शृ॒त माह॑ । \newline
50. शृ॒त माहाह॑ शृ॒तꣳ शृ॒त माह॒ स स आह॑ शृ॒तꣳ शृ॒त माह॒ सः । \newline
51. आह॒ स स आहाह॒ स एन॒ सैन॑सा॒ स आहाह॒ स एन॑सा । \newline
52. स एन॒ सैन॑सा॒ स स एन॑सा प्राणापा॒नौ प्रा॑णापा॒ना वेन॑सा॒ स स एन॑सा प्राणापा॒नौ । \newline
53. एन॑सा प्राणापा॒नौ प्रा॑णापा॒ना वेन॒सैन॑सा प्राणापा॒नौ वै वै प्रा॑णापा॒ना वेन॒सैन॑सा प्राणापा॒नौ वै । \newline
54. प्रा॒णा॒पा॒नौ वै वै प्रा॑णापा॒नौ प्रा॑णापा॒नौ वा ए॒ता वे॒तौ वै प्रा॑णापा॒नौ प्रा॑णापा॒नौ वा ए॒तौ । \newline
55. प्रा॒णा॒पा॒नाविति॑ प्राण - अ॒पा॒नौ । \newline
56. वा ए॒ता वे॒तौ वै वा ए॒तौ प॑शू॒नां प॑शू॒ना मे॒तौ वै वा ए॒तौ प॑शू॒नां । \newline
57. ए॒तौ प॑शू॒नां प॑शू॒ना मे॒ता वे॒तौ प॑शू॒नां ॅयद् यत् प॑शू॒ना मे॒ता वे॒तौ प॑शू॒नां ॅयत् । \newline
58. प॒शू॒नां ॅयद् यत् प॑शू॒नां प॑शू॒नां ॅयत् पृ॑षदा॒ज्यम् पृ॑षदा॒ज्यं ॅयत् प॑शू॒नां प॑शू॒नां ॅयत् पृ॑षदा॒ज्यम् । \newline
\pagebreak
\markright{ TS 6.3.10.2  \hfill https://www.vedavms.in \hfill}

\section{ TS 6.3.10.2 }

\textbf{TS 6.3.10.2 } \newline
\textbf{Samhita Paata} \newline

ॅयत् पृ॑षदा॒ज्यं प॒शोः खलु॒ वा आल॑ब्धस्य॒ हृद॑यमा॒त्माऽभि समे॑ति॒ यत् पृ॑षदा॒ज्येन॒ हृद॑य-मभिघा॒रय॑त्या॒त्मन्ने॒व प॑शू॒नां प्रा॑णापा॒नौ द॑धाति प॒शुना॒ वै दे॒वाः सु॑व॒र्गं ॅलो॒कमा॑य॒न् ते॑ऽमन्यन्त मनु॒ष्या॑ नो॒ऽन्वाभ॑विष्य॒न्तीति॒ तस्य॒ शिरः॑ छि॒त्त्वा मेधं॒ प्राक्षा॑रय॒न्थ्स प्र॒क्षो॑ऽभव॒त् तत् प्र॒क्षस्य॑ प्रक्ष॒त्वं ॅयत् प्ल॑क्षशा॒खो-त्त॑रब॒र्॒.हि-र्भव॑ति॒ समे॑धस्यै॒व- [  ] \newline

\textbf{Pada Paata} \newline

यत् । पृ॒ष॒दा॒ज्यमिति॑ पृषत् - आ॒ज्यम् । प॒शोः । खलु॑ । वै । आल॑ब्ध॒स्येत्या - ल॒ब्ध॒स्य॒ । हृद॑यम् । आ॒त्मा । अ॒भि । समिति॑ । ए॒ति॒ । यत् । पृ॒ष॒दा॒ज्येनेति॑ पृषत् - आ॒ज्येन॑ । हृद॑यम् । अ॒भि॒घा॒रय॒तीत्य॑भि - घा॒रय॑ति । आ॒त्मन्न् । ए॒व । प॒शू॒नाम् । प्रा॒णा॒पा॒नाविति॑ प्राण - अ॒पा॒नौ । द॒धा॒ति॒ । प॒शुना᳚ । वै । दे॒वाः । सु॒व॒र्गमिति॑ सुवः - गम् । लो॒कम् । आ॒य॒न्न् । ते । अ॒म॒न्य॒न्त॒ । म॒नु॒ष्याः᳚ । नः॒ । अ॒न्वाभ॑विष्य॒न्तीत्य॑नु-आभ॑विष्यन्ति । इति॑ । तस्य॑ । शिरः॑ । छि॒त्त्वा । मेध᳚म् । प्रेति॑ । अ॒क्षा॒र॒य॒न्न् । सः । प्र॒क्षः । अ॒भ॒व॒त् । तत् । प्र॒क्षस्य॑ । प्र॒क्ष॒त्वमिति॑ प्रक्ष - त्वम् । यत् । प्ल॒क्ष॒शा॒खेति॑ प्लक्ष - शा॒खा । उ॒त्त॒र॒ब॒र्॒.हिरित्युत्त॑र - ब॒र॒.हिः । भव॑ति । समे॑ध॒स्येति॒ स - मे॒ध॒स्य॒ । ए॒व ।  \newline


\textbf{Krama Paata} \newline

यत् पृ॑षदा॒ज्यम् । पृ॒ष॒दा॒ज्यम् प॒शोः । पृ॒ष॒दा॒ज्यमिति॑ पृषत् - आ॒ज्यम् । प॒शोः खलु॑ । खलु॒ वै । वा आल॑ब्धस्य । आल॑ब्धस्य॒ हृद॑यम् । आल॑ब्ध॒स्येत्या - ल॒ब्ध॒स्य॒ । हृद॑यमा॒त्मा । आ॒त्माऽभि । अ॒भि सम् । समे॑ति । ए॒ति॒ यत् । यत् पृ॑षदा॒ज्येन॑ । पृ॒ष॒दा॒ज्येन॒ हृद॑यम् । पृ॒ष॒दा॒ज्येनेति॑ पृषत् - आ॒ज्येन॑ । हृद॑यमभिघा॒रय॑ति । अ॒भि॒घा॒रय॑त्या॒त्मन्न् । अ॒भि॒घा॒रय॒तीत्य॑भि - घा॒रय॑ति । आ॒त्मन्ने॒व । ए॒व प॑शू॒नाम् । प॒शू॒नाम् प्रा॑णापा॒नौ । प्रा॒णा॒पा॒नौ द॑धाति । प्रा॒णा॒पा॒नाविति॑ प्राणा - अ॒पा॒नौ । द॒धा॒ति॒ प॒शुना᳚ । प॒शुना॒ वै । वै दे॒वाः । दे॒वाः सु॑व॒र्गम् । सु॒व॒र्गम् ॅलो॒कम् । सु॒व॒र्गमिति॑ सुवः - गम् । लो॒कमा॑यन्न् । आ॒य॒न् ते । ते॑ऽमन्यन्त । अ॒म॒न्य॒न्त॒ म॒नु॒ष्याः᳚ । म॒नु॒ष्या॑ नः । नो॒ऽन्वाभ॑विष्यन्ति । अ॒न्वाभ॑विष्य॒न्तीति॑ । अ॒न्वाभ॑विष्य॒न्तीत्य॑नु - आभ॑विष्यन्ति । इति॒ तस्य॑ । तस्य॒ शिरः॑ । शिर॑श्छि॒त्वा । छि॒त्वामेध᳚म् । मेध॒म् प्र । प्राक्षा॑रयन्न् । अ॒क्षा॒र॒य॒न्थ् सः । स प्र॒क्षः । प्र॒क्षो॑ऽभवत् । अ॒भ॒व॒त् तत् । तत् प्र॒क्षस्य॑ । प्र॒क्षस्य॑ प्रक्ष॒त्वम् । प्र॒क्ष॒त्वम् ॅयत् । प्र॒क्ष॒त्वमिति॑ प्रक्ष - त्वम् । यत् प्ल॑क्षशा॒खा । प्ल॒क्ष॒शा॒खोत्त॑रब॒र्.॒हिः । प्ल॒क्ष॒शा॒खेति॑ प्लक्ष - शा॒खा । उ॒त्त॒र॒ब॒र्.॒हिर् भव॑ति । उ॒त्त॒र॒ब॒र्.॒हिरित्यु॑त्तर - ब॒र्.॒हिः । भव॑ति॒ समे॑धस्य । समे॑धस्यै॒व । समे॑ध॒स्येति॒ स - मे॒ध॒स्य॒ । ए॒व प॒शोः \newline

\textbf{Jatai Paata} \newline

1. यत् पृ॑षदा॒ज्यम् पृ॑षदा॒ज्यं ॅयद् यत् पृ॑षदा॒ज्यम् । \newline
2. पृ॒ष॒दा॒ज्यम् प॒शोः प॒शोः पृ॑षदा॒ज्यम् पृ॑षदा॒ज्यम् प॒शोः । \newline
3. पृ॒ष॒दा॒ज्यमिति॑ पृषत् - आ॒ज्यम् । \newline
4. प॒शोः खलु॒ खलु॑ प॒शोः प॒शोः खलु॑ । \newline
5. खलु॒ वै वै खलु॒ खलु॒ वै । \newline
6. वा आल॑ब्ध॒स्या ल॑ब्धस्य॒ वै वा आल॑ब्धस्य । \newline
7. आल॑ब्धस्य॒ हृद॑यꣳ॒॒ हृद॑य॒ माल॑ब्ध॒स्या ल॑ब्धस्य॒ हृद॑यम् । \newline
8. आल॑ब्ध॒स्येत्या - ल॒ब्ध॒स्य॒ । \newline
9. हृद॑य मा॒त्मा ऽऽत्मा हृद॑यꣳ॒॒ हृद॑य मा॒त्मा । \newline
10. आ॒त्मा ऽभ्या᳚(1॒)भ्या᳚त्मा ऽऽत्मा ऽभि । \newline
11. अ॒भि सꣳ स म॒भ्य॑भि सम् । \newline
12. स मे᳚त्येति॒ सꣳ स मे॑ति । \newline
13. ए॒ति॒ यद् यदे᳚ त्येति॒ यत् । \newline
14. यत् पृ॑षदा॒ज्येन॑ पृषदा॒ज्येन॒ यद् यत् पृ॑षदा॒ज्येन॑ । \newline
15. पृ॒ष॒दा॒ज्येन॒ हृद॑यꣳ॒॒ हृद॑यम् पृषदा॒ज्येन॑ पृषदा॒ज्येन॒ हृद॑यम् । \newline
16. पृ॒ष॒दा॒ज्येनेति॑ पृषत् - आ॒ज्येन॑ । \newline
17. हृद॑य मभिघा॒रय॑ त्यभिघा॒रय॑ति॒ हृद॑यꣳ॒॒ हृद॑य मभिघा॒रय॑ति । \newline
18. अ॒भि॒घा॒रय॑ त्या॒त्मन् ना॒त्मन् न॑भिघा॒रय॑ त्यभिघा॒रय॑ त्या॒त्मन्न् । \newline
19. अ॒भि॒घा॒रय॒तीत्य॑भि - घा॒रय॑ति । \newline
20. आ॒त्मन् ने॒वै वात्मन् ना॒त्मन् ने॒व । \newline
21. ए॒व प॑शू॒नाम् प॑शू॒ना मे॒वैव प॑शू॒नाम् । \newline
22. प॒शू॒नाम् प्रा॑णापा॒नौ प्रा॑णापा॒नौ प॑शू॒नाम् प॑शू॒नाम् प्रा॑णापा॒नौ । \newline
23. प्रा॒णा॒पा॒नौ द॑धाति दधाति प्राणापा॒नौ प्रा॑णापा॒नौ द॑धाति । \newline
24. प्रा॒णा॒पा॒नाविति॑ प्राण - अ॒पा॒नौ । \newline
25. द॒धा॒ति॒ प॒शुना॑ प॒शुना॑ दधाति दधाति प॒शुना᳚ । \newline
26. प॒शुना॒ वै वै प॒शुना॑ प॒शुना॒ वै । \newline
27. वै दे॒वा दे॒वा वै वै दे॒वाः । \newline
28. दे॒वाः सु॑व॒र्गꣳ सु॑व॒र्गम् दे॒वा दे॒वाः सु॑व॒र्गम् । \newline
29. सु॒व॒र्गम् ॅलो॒कम् ॅलो॒कꣳ सु॑व॒र्गꣳ सु॑व॒र्गम् ॅलो॒कम् । \newline
30. सु॒व॒र्गमिति॑ सुवः - गम् । \newline
31. लो॒क मा॑यन् नायन् ॅलो॒कम् ॅलो॒क मा॑यन्न् । \newline
32. आ॒य॒न् ते त आ॑यन् नाय॒न् ते । \newline
33. ते॑ ऽमन्यन्ता मन्यन्त॒ ते ते॑ ऽमन्यन्त । \newline
34. अ॒म॒न्य॒न्त॒ म॒नु॒ष्या॑ मनु॒ष्या॑ अमन्यन्ता मन्यन्त मनु॒ष्याः᳚ । \newline
35. म॒नु॒ष्या॑ नो नो मनु॒ष्या॑ मनु॒ष्या॑ नः । \newline
36. नो॒ ऽन्वाभ॑विष्य न्त्य॒न्वाभ॑विष्यन्ति नो नो॒ ऽन्वाभ॑विष्यन्ति । \newline
37. अ॒न्वाभ॑विष्य॒न्तीती त्य॒न्वाभ॑विष्य न्त्य॒न्वाभ॑विष्य॒न्तीति॑ । \newline
38. अ॒न्वाभ॑विष्य॒न्तीत्य॑नु - आभ॑विष्यन्ति । \newline
39. इति॒ तस्य॒ तस्ये तीति॒ तस्य॑ । \newline
40. तस्य॒ शिरः॒ शिर॒ स्तस्य॒ तस्य॒ शिरः॑ । \newline
41. शिर॑ श्छि॒त्त्वा छि॒त्त्वा शिरः॒ शिर॑ श्छि॒त्त्वा । \newline
42. छि॒त्त्वा मेध॒म् मेध॑म् छि॒त्त्वा छि॒त्त्वा मेध᳚म् । \newline
43. मेध॒म् प्र प्र मेध॒म् मेध॒म् प्र । \newline
44. प्राक्षा॑रयन् नक्षारय॒न् प्र प्राक्षा॑रयन्न् । \newline
45. अ॒क्षा॒र॒य॒न् थ्स सो᳚ ऽक्षारयन् नक्षारय॒न् थ्सः । \newline
46. स प्र॒क्षः प्र॒क्षः स स प्र॒क्षः । \newline
47. प्र॒क्षो॑ ऽभव दभवत् प्र॒क्षः प्र॒क्षो॑ ऽभवत् । \newline
48. अ॒भ॒व॒त् तत् तद॑भव दभव॒त् तत् । \newline
49. तत् प्र॒क्षस्य॑ प्र॒क्षस्य॒ तत् तत् प्र॒क्षस्य॑ । \newline
50. प्र॒क्षस्य॑ प्रक्ष॒त्वम् प्र॑क्ष॒त्वम् प्र॒क्षस्य॑ प्र॒क्षस्य॑ प्रक्ष॒त्वम् । \newline
51. प्र॒क्ष॒त्वं ॅयद् यत् प्र॑क्ष॒त्वम् प्र॑क्ष॒त्वं ॅयत् । \newline
52. प्र॒क्ष॒त्वमिति॑ प्रक्ष - त्वम् । \newline
53. यत् प्ल॑क्षशा॒खा प्ल॑क्षशा॒खा यद् यत् प्ल॑क्षशा॒खा । \newline
54. प्ल॒क्ष॒शा॒ खोत्त॑रब॒र्॒.हि रु॑त्तरब॒र्॒.हिः प्ल॑क्षशा॒खा प्ल॑क्षशा॒ खोत्त॑रब॒र्॒.हिः । \newline
55. प्ल॒क्ष॒शा॒खेति॑ प्लक्ष - शा॒खा । \newline
56. उ॒त्त॒र॒ब॒र्॒.हिर् भव॑ति॒ भव॑ त्युत्तरब॒र्॒.हि रु॑त्तरब॒र्॒.हिर् भव॑ति । \newline
57. उ॒त्त॒र॒ब॒र्॒.हिरित्युत्त॑र - ब॒र्॒.हिः । \newline
58. भव॑ति॒ समे॑धस्य॒ समे॑धस्य॒ भव॑ति॒ भव॑ति॒ समे॑धस्य । \newline
59. समे॑धस्यै॒ वैव समे॑धस्य॒ समे॑धस्यै॒व । \newline
60. समे॑ध॒स्येति॒ स - मे॒ध॒स्य॒ । \newline
61. ए॒व प॒शोः प॒शो रे॒वैव प॒शोः । \newline

\textbf{Ghana Paata } \newline

1. यत् पृ॑षदा॒ज्यम् पृ॑षदा॒ज्यं ॅयद् यत् पृ॑षदा॒ज्यम् प॒शोः प॒शोः पृ॑षदा॒ज्यं ॅयद् यत् पृ॑षदा॒ज्यम् प॒शोः । \newline
2. पृ॒ष॒दा॒ज्यम् प॒शोः प॒शोः पृ॑षदा॒ज्यम् पृ॑षदा॒ज्यम् प॒शोः खलु॒ खलु॑ प॒शोः पृ॑षदा॒ज्यम् पृ॑षदा॒ज्यम् प॒शोः खलु॑ । \newline
3. पृ॒ष॒दा॒ज्यमिति॑ पृषत् - आ॒ज्यम् । \newline
4. प॒शोः खलु॒ खलु॑ प॒शोः प॒शोः खलु॒ वै वै खलु॑ प॒शोः प॒शोः खलु॒ वै । \newline
5. खलु॒ वै वै खलु॒ खलु॒ वा आल॑ब्ध॒स्या ल॑ब्धस्य॒ वै खलु॒ खलु॒ वा आल॑ब्धस्य । \newline
6. वा आल॑ब्ध॒स्या ल॑ब्धस्य॒ वै वा आल॑ब्धस्य॒ हृद॑यꣳ॒॒ हृद॑य॒ माल॑ब्धस्य॒ वै वा आल॑ब्धस्य॒ हृद॑यम् । \newline
7. आल॑ब्धस्य॒ हृद॑यꣳ॒॒ हृद॑य॒ माल॑ब्ध॒स्या ल॑ब्धस्य॒ हृद॑य मा॒त्मा ऽऽत्मा हृद॑य॒ माल॑ब्ध॒स्या ल॑ब्धस्य॒ हृद॑य मा॒त्मा । \newline
8. आल॑ब्ध॒स्येत्या - ल॒ब्ध॒स्य॒ । \newline
9. हृद॑य मा॒त्मा ऽऽत्मा हृद॑यꣳ॒॒ हृद॑य मा॒त्मा ऽभ्या᳚(1॒)भ्या᳚त्मा हृद॑यꣳ॒॒ हृद॑य मा॒त्मा ऽभि । \newline
10. आ॒त्मा ऽभ्या᳚(1॒)भ्या᳚त्मा ऽऽत्मा ऽभि सꣳ स म॒भ्या᳚त्मा ऽऽत्मा ऽभि सम् । \newline
11. अ॒भि सꣳ स म॒भ्य॑भि स मे᳚त्येति॒ स म॒भ्य॑भि स मे॑ति । \newline
12. स मे᳚त्येति॒ सꣳ स मे॑ति॒ यद् यदे॑ति॒ सꣳ स मे॑ति॒ यत् । \newline
13. ए॒ति॒ यद् यदे᳚ त्येति॒ यत् पृ॑षदा॒ज्येन॑ पृषदा॒ज्येन॒ यदे᳚ त्येति॒ यत् पृ॑षदा॒ज्येन॑ । \newline
14. यत् पृ॑षदा॒ज्येन॑ पृषदा॒ज्येन॒ यद् यत् पृ॑षदा॒ज्येन॒ हृद॑यꣳ॒॒ हृद॑यम् पृषदा॒ज्येन॒ यद् यत् पृ॑षदा॒ज्येन॒ हृद॑यम् । \newline
15. पृ॒ष॒दा॒ज्येन॒ हृद॑यꣳ॒॒ हृद॑यम् पृषदा॒ज्येन॑ पृषदा॒ज्येन॒ हृद॑य मभिघा॒रय॑ त्यभिघा॒रय॑ति॒ हृद॑यम् पृषदा॒ज्येन॑ पृषदा॒ज्येन॒ हृद॑य मभिघा॒रय॑ति । \newline
16. पृ॒ष॒दा॒ज्येनेति॑ पृषत् - आ॒ज्येन॑ । \newline
17. हृद॑य मभिघा॒रय॑ त्यभिघा॒रय॑ति॒ हृद॑यꣳ॒॒ हृद॑य मभिघा॒रय॑ त्या॒त्मन् ना॒त्मन् न॑भिघा॒रय॑ति॒ हृद॑यꣳ॒॒ हृद॑य मभिघा॒रय॑ त्या॒त्मन्न् । \newline
18. अ॒भि॒घा॒रय॑ त्या॒त्मन् ना॒त्मन् न॑भिघा॒रय॑ त्यभिघा॒रय॑ त्या॒त्मन् ने॒वै वात्मन् न॑भिघा॒रय॑ त्यभिघा॒रय॑ त्या॒त्मन् ने॒व । \newline
19. अ॒भि॒घा॒रय॒तीत्य॑भि - घा॒रय॑ति । \newline
20. आ॒त्मन् ने॒वै वात्मन् ना॒त्मन् ने॒व प॑शू॒नाम् प॑शू॒ना मे॒वात्मन् ना॒त्मन् ने॒व प॑शू॒नाम् । \newline
21. ए॒व प॑शू॒नाम् प॑शू॒ना मे॒वैव प॑शू॒नाम् प्रा॑णापा॒नौ प्रा॑णापा॒नौ प॑शू॒ना मे॒वैव प॑शू॒नाम् प्रा॑णापा॒नौ । \newline
22. प॒शू॒नाम् प्रा॑णापा॒नौ प्रा॑णापा॒नौ प॑शू॒नाम् प॑शू॒नाम् प्रा॑णापा॒नौ द॑धाति दधाति प्राणापा॒नौ प॑शू॒नाम् प॑शू॒नाम् प्रा॑णापा॒नौ द॑धाति । \newline
23. प्रा॒णा॒पा॒नौ द॑धाति दधाति प्राणापा॒नौ प्रा॑णापा॒नौ द॑धाति प॒शुना॑ प॒शुना॑ दधाति प्राणापा॒नौ प्रा॑णापा॒नौ द॑धाति प॒शुना᳚ । \newline
24. प्रा॒णा॒पा॒नाविति॑ प्राण - अ॒पा॒नौ । \newline
25. द॒धा॒ति॒ प॒शुना॑ प॒शुना॑ दधाति दधाति प॒शुना॒ वै वै प॒शुना॑ दधाति दधाति प॒शुना॒ वै । \newline
26. प॒शुना॒ वै वै प॒शुना॑ प॒शुना॒ वै दे॒वा दे॒वा वै प॒शुना॑ प॒शुना॒ वै दे॒वाः । \newline
27. वै दे॒वा दे॒वा वै वै दे॒वाः सु॑व॒र्गꣳ सु॑व॒र्गम् दे॒वा वै वै दे॒वाः सु॑व॒र्गम् । \newline
28. दे॒वाः सु॑व॒र्गꣳ सु॑व॒र्गम् दे॒वा दे॒वाः सु॑व॒र्गम् ॅलो॒कम् ॅलो॒कꣳ सु॑व॒र्गम् दे॒वा दे॒वाः सु॑व॒र्गम् ॅलो॒कम् । \newline
29. सु॒व॒र्गम् ॅलो॒कम् ॅलो॒कꣳ सु॑व॒र्गꣳ सु॑व॒र्गम् ॅलो॒क मा॑यन् नायन् ॅलो॒कꣳ सु॑व॒र्गꣳ सु॑व॒र्गम् ॅलो॒क मा॑यन्न् । \newline
30. सु॒व॒र्गमिति॑ सुवः - गम् । \newline
31. लो॒क मा॑यन् नायन् ॅलो॒कम् ॅलो॒क मा॑य॒न् ते त आ॑यन् ॅलो॒कम् ॅलो॒क मा॑य॒न् ते । \newline
32. आ॒य॒न् ते त आ॑यन् नाय॒न् ते॑ ऽमन्यन्ता मन्यन्त॒ त आ॑यन् नाय॒न् ते॑ ऽमन्यन्त । \newline
33. ते॑ ऽमन्यन्ता मन्यन्त॒ ते ते॑ ऽमन्यन्त मनु॒ष्या॑ मनु॒ष्या॑ अमन्यन्त॒ ते ते॑ ऽमन्यन्त मनु॒ष्याः᳚ । \newline
34. अ॒म॒न्य॒न्त॒ म॒नु॒ष्या॑ मनु॒ष्या॑ अमन्यन्ता मन्यन्त मनु॒ष्या॑ नो नो मनु॒ष्या॑ अमन्यन्ता मन्यन्त मनु॒ष्या॑ नः । \newline
35. म॒नु॒ष्या॑ नो नो मनु॒ष्या॑ मनु॒ष्या॑ नो॒ ऽन्वाभ॑विष्य न्त्य॒न्वाभ॑विष्यन्ति नो मनु॒ष्या॑ मनु॒ष्या॑ नो॒ ऽन्वाभ॑विष्यन्ति । \newline
36. नो॒ ऽन्वाभ॑विष्य न्त्य॒न्वाभ॑विष्यन्ति नो नो॒ ऽन्वाभ॑विष्य॒न्तीती त्य॒न्वाभ॑विष्यन्ति नो नो॒ ऽन्वाभ॑विष्य॒न्तीति॑ । \newline
37. अ॒न्वाभ॑विष्य॒न्तीती त्य॒न्वाभ॑विष्य न्त्य॒न्वाभ॑विष्य॒न्तीति॒ तस्य॒ तस्ये त्य॒न्वाभ॑विष्य न्त्य॒न्वाभ॑विष्य॒
न्तीति॒ तस्य॑ । \newline
38. अ॒न्वाभ॑विष्य॒न्तीत्य॑नु - आभ॑विष्यन्ति । \newline
39. इति॒ तस्य॒ तस्ये तीति॒ तस्य॒ शिरः॒ शिर॒ स्तस्येतीति॒ तस्य॒ शिरः॑ । \newline
40. तस्य॒ शिरः॒ शिर॒ स्तस्य॒ तस्य॒ शिर॑ श्छि॒त्त्वा छि॒त्त्वा शिर॒ स्तस्य॒ तस्य॒ शिर॑ श्छि॒त्त्वा । \newline
41. शिर॑ श्छि॒त्त्वा छि॒त्त्वा शिरः॒ शिर॑ श्छि॒त्त्वा मेध॒म् मेध॑म् छि॒त्त्वा शिरः॒ शिर॑ श्छि॒त्त्वा मेध᳚म् । \newline
42. छि॒त्त्वा मेध॒म् मेध॑म् छि॒त्त्वा छि॒त्त्वा मेध॒म् प्र प्र मेध॑म् छि॒त्त्वा छि॒त्त्वा मेध॒म् प्र । \newline
43. मेध॒म् प्र प्र मेध॒म् मेध॒म् प्राक्षा॑रयन् नक्षारय॒न् प्र मेध॒म् मेध॒म् प्राक्षा॑रयन्न् । \newline
44. प्राक्षा॑रयन् नक्षारय॒न् प्र प्राक्षा॑रय॒न् थ्स सो᳚ ऽक्षारय॒न् प्र प्राक्षा॑रय॒न् थ्सः । \newline
45. अ॒क्षा॒र॒य॒न् थ्स सो᳚ ऽक्षारयन् नक्षारय॒न् थ्स प्र॒क्षः प्र॒क्षः सो᳚ ऽक्षारयन् नक्षारय॒न् थ्स प्र॒क्षः । \newline
46. स प्र॒क्षः प्र॒क्षः स स प्र॒क्षो॑ ऽभव दभवत् प्र॒क्षः स स प्र॒क्षो॑ ऽभवत् । \newline
47. प्र॒क्षो॑ ऽभव दभवत् प्र॒क्षः प्र॒क्षो॑ ऽभव॒त् तत् तद॑भवत् प्र॒क्षः प्र॒क्षो॑ ऽभव॒त् तत् । \newline
48. अ॒भ॒व॒त् तत् तद॑भव दभव॒त् तत् प्र॒क्षस्य॑ प्र॒क्षस्य॒ तद॑भव दभव॒त् तत् प्र॒क्षस्य॑ । \newline
49. तत् प्र॒क्षस्य॑ प्र॒क्षस्य॒ तत् तत् प्र॒क्षस्य॑ प्रक्ष॒त्वम् प्र॑क्ष॒त्वम् प्र॒क्षस्य॒ तत् तत् प्र॒क्षस्य॑ प्रक्ष॒त्वम् । \newline
50. प्र॒क्षस्य॑ प्रक्ष॒त्वम् प्र॑क्ष॒त्वम् प्र॒क्षस्य॑ प्र॒क्षस्य॑ प्रक्ष॒त्वं ॅयद् यत् प्र॑क्ष॒त्वम् प्र॒क्षस्य॑ प्र॒क्षस्य॑ प्रक्ष॒त्वं ॅयत् । \newline
51. प्र॒क्ष॒त्वं ॅयद् यत् प्र॑क्ष॒त्वम् प्र॑क्ष॒त्वं ॅयत् प्ल॑क्षशा॒खा प्ल॑क्षशा॒खा यत् प्र॑क्ष॒त्वम् प्र॑क्ष॒त्वं ॅयत् प्ल॑क्षशा॒खा । \newline
52. प्र॒क्ष॒त्वमिति॑ प्रक्ष - त्वम् । \newline
53. यत् प्ल॑क्षशा॒खा प्ल॑क्षशा॒खा यद् यत् प्ल॑क्षशा॒ खोत्त॑रब॒र्॒.हि रु॑त्तरब॒र्॒.हिः प्ल॑क्षशा॒खा यद् यत् प्ल॑क्षशा॒ खोत्त॑रब॒र्॒.हिः । \newline
54. प्ल॒क्ष॒शा॒ खोत्त॑रब॒र्॒.हि रु॑त्तरब॒र्॒.हिः प्ल॑क्षशा॒खा प्ल॑क्षशा॒ खोत्त॑रब॒र्॒.हिर् भव॑ति॒ भव॑ त्युत्तरब॒र्॒.हिः प्ल॑क्षशा॒खा प्ल॑क्षशा॒ खोत्त॑रब॒र्॒.हिर् भव॑ति । \newline
55. प्ल॒क्ष॒शा॒खेति॑ प्लक्ष - शा॒खा । \newline
56. उ॒त्त॒र॒ब॒र्॒.हिर् भव॑ति॒ भव॑ त्युत्तरब॒र्॒.हि रु॑त्तरब॒र्॒.हिर् भव॑ति॒ समे॑धस्य॒ समे॑धस्य॒ भव॑ त्युत्तरब॒र्॒.हि रु॑त्तरब॒र्॒.हिर् भव॑ति॒ समे॑धस्य । \newline
57. उ॒त्त॒र॒ब॒र्॒.हिरित्युत्त॑र - ब॒र्॒.हिः । \newline
58. भव॑ति॒ समे॑धस्य॒ समे॑धस्य॒ भव॑ति॒ भव॑ति॒ समे॑धस्यै॒ वैव समे॑धस्य॒ भव॑ति॒ भव॑ति॒ समे॑धस्यै॒व । \newline
59. समे॑धस्यै॒वैव समे॑धस्य॒ समे॑धस्यै॒व प॒शोः प॒शोरे॒व समे॑धस्य॒ समे॑धस्यै॒व प॒शोः । \newline
60. समे॑ध॒स्येति॒ स - मे॒ध॒स्य॒ । \newline
61. ए॒व प॒शोः प॒शो रे॒वैव प॒शो रवाव॑ प॒शो रे॒वैव प॒शो रव॑ । \newline
\pagebreak
\markright{ TS 6.3.10.3  \hfill https://www.vedavms.in \hfill}

\section{ TS 6.3.10.3 }

\textbf{TS 6.3.10.3 } \newline
\textbf{Samhita Paata} \newline

प॒शोरव॑ द्यति प॒शुं ॅवै ह्रि॒यमा॑णꣳ॒॒ रक्षाꣳ॒॒स्यनु॑ सचन्तेऽन्त॒रा यूपं॑ चाऽऽ*हव॒नीयं॑ च हरति॒ रक्ष॑सा॒मप॑हत्यै प॒शोर्वा आल॑ब्धस्य॒ मनोऽप॑ क्रामति म॒नोता॑यै ह॒विषो॑ऽवदी॒यमा॑न॒स्यानु॑ ब्रू॒हीत्या॑ह॒ मन॑ ए॒वास्याव॑ रुन्ध॒ एका॑दशाव॒दाना॒न्यव॑ द्यति॒ दश॒ वै प॒शोः प्रा॒णा आ॒त्मैका॑द॒शो यावा॑ने॒व प॒शुस्तस्याव॑- [  ] \newline

\textbf{Pada Paata} \newline

प॒शोः । अवेति॑ । द्य॒ति॒ । प॒शुम् । वै । ह्रि॒यमा॑णम् । रक्षाꣳ॑सि । अन्विति॑ । स॒च॒न्ते॒ । अ॒न्त॒रा । यूप᳚म् । च॒ । आ॒ह॒व॒नीय॒मित्या᳚ - ह॒व॒नीय᳚म् । च॒ । ह॒र॒ति॒ । रक्ष॑साम् । अप॑हत्या॒ इत्यप॑ - ह॒त्यै॒ । प॒शोः । वै । आल॑ब्ध॒स्येत्या - ल॒ब्ध॒स्य॒ । मनः॑ । अपेति॑ । क्रा॒म॒ति॒ । म॒नोता॑यै । ह॒विषः॑ । अ॒व॒दी॒यमा॑न॒स्येत्य॑व - दी॒यमा॑नस्य । अन्विति॑ । ब्रू॒हि॒ । इति॑ । आ॒ह॒ । मनः॑ । ए॒व । अ॒स्य॒ । अवेति॑ । रु॒न्धे॒ । एका॑दश । अ॒व॒दाना॒नीत्य॑व - दाना॑नि । अवेति॑ । द्य॒ति॒ । दश॑ । वै । प॒शोः । प्रा॒णा इति॑ प्र - अ॒नाः । आ॒त्मा । ए॒का॒द॒शः । यावान्॑ । ए॒व । प॒शुः । तस्य॑ । अवेति॑ ।  \newline


\textbf{Krama Paata} \newline

प॒शोरव॑ । अव॑ द्यति । द्य॒ति॒ प॒शुम् । प॒शुम् ॅवै । वै ह्रि॒यमा॑णम् । ह्रि॒यमा॑णꣳ॒॒ रक्षाꣳ॑सि । रक्षाꣳ॒॒स्यनु॑ । अनु॑ सचन्ते । स॒च॒न्ते॒ऽन्त॒रा । अ॒न्त॒रा यूप᳚म् । यूप॑म् च । चा॒ह॒व॒नीय᳚म् । आ॒ह॒व॒नीय॑म् च । आ॒ह॒व॒नीय॒मित्या᳚ - ह॒व॒नीय᳚म् । च॒ ह॒र॒ति॒ । ह॒र॒ति॒ रक्ष॑साम् । रक्ष॑सा॒मप॑हत्यै । अप॑हत्यै प॒शोः । अप॑हत्या॒ इत्यप॑ - ह॒त्यै॒ । प॒शोर् वै । वा आल॑ब्धस्य । आल॑ब्धस्य॒ मनः॑ । आल॑ब्ध॒स्येत्या - ल॒ब्ध॒स्य॒ । मनोऽप॑ । अप॑ क्रामति । क्रा॒म॒ति॒ म॒नोता॑यै । म॒नोता॑यै ह॒विषः॑ । ह॒विषो॑ऽवदी॒यमा॑नस्य । अ॒व॒दी॒यमा॑न॒स्यानु॑ । अ॒व॒दी॒यमा॑न॒स्येत्य॑व - दी॒यमा॑नस्य । अनु॑ ब्रूहि । ब्रू॒हीति॑ । इत्या॑ह । आ॒ह॒ मनः॑ । मन॑ ए॒व । ए॒वास्य॑ । अ॒स्याव॑ । अव॑ रुन्धे । रु॒न्ध॒ एका॑दश । एका॑दशाव॒दाना॑नि । अ॒व॒दाना॒न्यव॑ । अ॒व॒दाना॒नीत्य॑व - दाना॑नि । अव॑ द्यति । द्य॒ति॒ दश॑ । दश॒ वै । वै प॒शोः । प॒शोः प्रा॒णाः । प्रा॒णा आ॒त्मा । प्रा॒णा इति॑ प्र - अ॒नाः । आ॒त्मैका॑द॒शः । ए॒का॒द॒शो यावान्॑ । यावा॑ने॒व । ए॒व प॒शुः । प॒शुस्तस्य॑ । तस्याव॑ । अव॑ द्यति \newline

\textbf{Jatai Paata} \newline

1. प॒शोर वाव॑ प॒शोः प॒शो रव॑ । \newline
2. अव॑ द्यति द्य॒त्यवाव॑ द्यति । \newline
3. द्य॒ति॒ प॒शुम् प॒शुम् द्य॑ति द्यति प॒शुम् । \newline
4. प॒शुं ॅवै वै प॒शुम् प॒शुं ॅवै । \newline
5. वै ह्रि॒यमा॑णꣳ ह्रि॒यमा॑णं॒ ॅवै वै ह्रि॒यमा॑णम् । \newline
6. ह्रि॒यमा॑णꣳ॒॒ रक्षाꣳ॑सि॒ रक्षाꣳ॑सि ह्रि॒यमा॑णꣳ ह्रि॒यमा॑णꣳ॒॒ रक्षाꣳ॑सि । \newline
7. रक्षाꣳ॒॒ स्यन्वनु॒ रक्षाꣳ॑सि॒ रक्षाꣳ॒॒ स्यनु॑ । \newline
8. अनु॑ सचन्ते सच॒न्ते ऽन्वनु॑ सचन्ते । \newline
9. स॒च॒न्ते॒ ऽन्त॒रा ऽन्त॒रा स॑चन्ते सचन्ते ऽन्त॒रा । \newline
10. अ॒न्त॒रा यूपं॒ ॅयूप॑ मन्त॒रा ऽन्त॒रा यूप᳚म् । \newline
11. यूप॑म् च च॒ यूपं॒ ॅयूप॑म् च । \newline
12. चा॒ह॒व॒नीय॑ माहव॒नीय॑म् च चाहव॒नीय᳚म् । \newline
13. आ॒ह॒व॒नीय॑म् च चाहव॒नीय॑ माहव॒नीय॑म् च । \newline
14. आ॒ह॒व॒नीय॒मित्या᳚ - ह॒व॒नीय᳚म् । \newline
15. च॒ ह॒र॒ति॒ ह॒र॒ति॒ च॒ च॒ ह॒र॒ति॒ । \newline
16. ह॒र॒ति॒ रक्ष॑साꣳ॒॒ रक्ष॑साꣳ हरति हरति॒ रक्ष॑साम् । \newline
17. रक्ष॑सा॒ मप॑हत्या॒ अप॑हत्यै॒ रक्ष॑साꣳ॒॒ रक्ष॑सा॒ मप॑हत्यै । \newline
18. अप॑हत्यै प॒शोः प॒शो रप॑हत्या॒ अप॑हत्यै प॒शोः । \newline
19. अप॑हत्या॒ इत्यप॑ - ह॒त्यै॒ । \newline
20. प॒शोर् वै वै प॒शोः प॒शोर् वै । \newline
21. वा आल॑ब्ध॒स्या ल॑ब्धस्य॒ वै वा आल॑ब्धस्य । \newline
22. आल॑ब्धस्य॒ मनो॒ मन॒ आल॑ब्ध॒स्या ल॑ब्धस्य॒ मनः॑ । \newline
23. आल॑ब्ध॒स्येत्या - ल॒ब्ध॒स्य॒ । \newline
24. मनो ऽपाप॒ मनो॒ मनो ऽप॑ । \newline
25. अप॑ क्रामति क्राम॒ त्यपाप॑ क्रामति । \newline
26. क्रा॒म॒ति॒ म॒नोता॑यै म॒नोता॑यै क्रामति क्रामति म॒नोता॑यै । \newline
27. म॒नोता॑यै ह॒विषो॑ ह॒विषो॑ म॒नोता॑यै म॒नोता॑यै ह॒विषः॑ । \newline
28. ह॒विषो॑ ऽवदी॒यमा॑नस्या वदी॒यमा॑नस्य ह॒विषो॑ ह॒विषो॑ ऽवदी॒यमा॑नस्य । \newline
29. अ॒व॒दी॒यमा॑न॒स्यान् वन् व॑ वदी॒यमा॑नस्यावदी॒यमा॑न॒ स्यानु॑ । \newline
30. अ॒व॒दी॒यमा॑न॒स्येत्य॑व - दी॒यमा॑नस्य । \newline
31. अनु॑ ब्रूहि ब्रू॒ह्यन् वनु॑ ब्रूहि । \newline
32. ब्रू॒ही तीति॑ ब्रूहि ब्रू॒हीति॑ । \newline
33. इत्या॑हा॒हे तीत्या॑ह । \newline
34. आ॒ह॒ मनो॒ मन॑ आहाह॒ मनः॑ । \newline
35. मन॑ ए॒वैव मनो॒ मन॑ ए॒व । \newline
36. ए॒वास्या᳚ स्यै॒वै वास्य॑ । \newline
37. अ॒स्या वावा᳚ स्या॒स्याव॑ । \newline
38. अव॑ रुन्धे रु॒न्धे ऽवाव॑ रुन्धे । \newline
39. रु॒न्ध॒ एका॑द॒ शैका॑दश रुन्धे रुन्ध॒ एका॑दश । \newline
40. एका॑दशा व॒दाना᳚ न्यव॒दाना॒ न्येका॑द॒ शैका॑दशा व॒दाना॑नि । \newline
41. अ॒व॒दाना॒ न्यवावा॑ व॒दाना᳚ न्यव॒दाना॒ न्यव॑ । \newline
42. अ॒व॒दाना॒नीत्य॑व - दाना॑नि । \newline
43. अव॑ द्यति द्य॒त्यवाव॑ द्यति । \newline
44. द्य॒ति॒ दश॒ दश॑ द्यति द्यति॒ दश॑ । \newline
45. दश॒ वै वै दश॒ दश॒ वै । \newline
46. वै प॒शोः प॒शोर् वै वै प॒शोः । \newline
47. प॒शोः प्रा॒णाः प्रा॒णाः प॒शोः प॒शोः प्रा॒णाः । \newline
48. प्रा॒णा आ॒त्मा ऽऽत्मा प्रा॒णाः प्रा॒णा आ॒त्मा । \newline
49. प्रा॒णा इति॑ प्र - अ॒नाः । \newline
50. आ॒त्मैका॑द॒श ए॑काद॒श आ॒त्मा ऽऽत्मैका॑द॒शः । \newline
51. ए॒का॒द॒शो यावा॒न्॒. यावा॑ नेकाद॒श ए॑काद॒शो यावान्॑ । \newline
52. यावा॑ने॒वैव यावा॒न्॒. यावा॑ने॒व । \newline
53. ए॒व प॒शुः प॒शु रे॒वैव प॒शुः । \newline
54. प॒शु स्तस्य॒ तस्य॑ प॒शुः प॒शु स्तस्य॑ । \newline
55. तस्या वाव॒ तस्य॒ तस्याव॑ । \newline
56. अव॑ द्यति द्य॒त्यवाव॑ द्यति । \newline

\textbf{Ghana Paata } \newline

1. प॒शो रवाव॑ प॒शोः प॒शो रव॑ द्यति द्य॒त्यव॑ प॒शोः प॒शो रव॑ द्यति । \newline
2. अव॑ द्यति द्य॒त्य वाव॑ द्यति प॒शुम् प॒शुम् द्य॒त्य वाव॑ द्यति प॒शुम् । \newline
3. द्य॒ति॒ प॒शुम् प॒शुम् द्य॑ति द्यति प॒शुं ॅवै वै प॒शुम् द्य॑ति द्यति प॒शुं ॅवै । \newline
4. प॒शुं ॅवै वै प॒शुम् प॒शुं ॅवै ह्रि॒यमा॑णꣳ ह्रि॒यमा॑णं॒ ॅवै प॒शुम् प॒शुं ॅवै ह्रि॒यमा॑णम् । \newline
5. वै ह्रि॒यमा॑णꣳ ह्रि॒यमा॑णं॒ ॅवै वै ह्रि॒यमा॑णꣳ॒॒ रक्षाꣳ॑सि॒ रक्षाꣳ॑सि ह्रि॒यमा॑णं॒ ॅवै वै ह्रि॒यमा॑णꣳ॒॒ रक्षाꣳ॑सि । \newline
6. ह्रि॒यमा॑णꣳ॒॒ रक्षाꣳ॑सि॒ रक्षाꣳ॑सि ह्रि॒यमा॑णꣳ ह्रि॒यमा॑णꣳ॒॒ रक्षाꣳ॒॒ स्यन्वनु॒ रक्षाꣳ॑सि ह्रि॒यमा॑णꣳ ह्रि॒यमा॑णꣳ॒॒ रक्षाꣳ॒॒ स्यनु॑ । \newline
7. रक्षाꣳ॒॒ स्यन्वनु॒ रक्षाꣳ॑सि॒ रक्षाꣳ॒॒ स्यनु॑ सचन्ते सच॒न्ते ऽनु॒ रक्षाꣳ॑सि॒ रक्षाꣳ॒॒ स्यनु॑ सचन्ते । \newline
8. अनु॑ सचन्ते सच॒न्ते ऽन्वनु॑ सचन्ते ऽन्त॒रा ऽन्त॒रा स॑च॒न्ते ऽन्वनु॑ सचन्ते ऽन्त॒रा । \newline
9. स॒च॒न्ते॒ ऽन्त॒रा ऽन्त॒रा स॑चन्ते सचन्ते ऽन्त॒रा यूपं॒ ॅयूप॑ मन्त॒रा स॑चन्ते सचन्ते ऽन्त॒रा यूप᳚म् । \newline
10. अ॒न्त॒रा यूपं॒ ॅयूप॑ मन्त॒रा ऽन्त॒रा यूप॑म् च च॒ यूप॑ मन्त॒रा ऽन्त॒रा यूप॑म् च । \newline
11. यूप॑म् च च॒ यूपं॒ ॅयूप॑म् चाहव॒नीय॑ माहव॒नीय॑म् च॒ यूपं॒ ॅयूप॑म् चाहव॒नीय᳚म् । \newline
12. चा॒ह॒व॒नीय॑ माहव॒नीय॑म् च चाहव॒नीय॑म् च चाहव॒नीय॑म् च चाहव॒नीय॑म् च । \newline
13. आ॒ह॒व॒नीय॑म् च चाहव॒नीय॑ माहव॒नीय॑म् च हरति हरति चाहव॒नीय॑ माहव॒नीय॑म् च हरति । \newline
14. आ॒ह॒व॒नीय॒मित्या᳚ - ह॒व॒नीय᳚म् । \newline
15. च॒ ह॒र॒ति॒ ह॒र॒ति॒ च॒ च॒ ह॒र॒ति॒ रक्ष॑साꣳ॒॒ रक्ष॑साꣳ हरति च च हरति॒ रक्ष॑साम् । \newline
16. ह॒र॒ति॒ रक्ष॑साꣳ॒॒ रक्ष॑साꣳ हरति हरति॒ रक्ष॑सा॒ मप॑हत्या॒ अप॑हत्यै॒ रक्ष॑साꣳ हरति हरति॒ रक्ष॑सा॒ मप॑हत्यै । \newline
17. रक्ष॑सा॒ मप॑हत्या॒ अप॑हत्यै॒ रक्ष॑साꣳ॒॒ रक्ष॑सा॒ मप॑हत्यै प॒शोः प॒शो रप॑हत्यै॒ रक्ष॑साꣳ॒॒ रक्ष॑सा॒ मप॑हत्यै प॒शोः । \newline
18. अप॑हत्यै प॒शोः प॒शो रप॑हत्या॒ अप॑हत्यै प॒शोर् वै वै प॒शो रप॑हत्या॒ अप॑हत्यै प॒शोर् वै । \newline
19. अप॑हत्या॒ इत्यप॑ - ह॒त्यै॒ । \newline
20. प॒शोर् वै वै प॒शोः प॒शोर् वा आल॑ब्ध॒स्या ल॑ब्धस्य॒ वै प॒शोः प॒शोर् वा आल॑ब्धस्य । \newline
21. वा आल॑ब्ध॒स्या ल॑ब्धस्य॒ वै वा आल॑ब्धस्य॒ मनो॒ मन॒ आल॑ब्धस्य॒ वै वा आल॑ब्धस्य॒ मनः॑ । \newline
22. आल॑ब्धस्य॒ मनो॒ मन॒ आल॑ब्ध॒स्या ल॑ब्धस्य॒ मनो ऽपाप॒ मन॒ आल॑ब्ध॒स्या ल॑ब्धस्य॒ मनो ऽप॑ । \newline
23. आल॑ब्ध॒स्येत्या - ल॒ब्ध॒स्य॒ । \newline
24. मनो ऽपाप॒ मनो॒ मनो ऽप॑ क्रामति क्राम॒ त्यप॒ मनो॒ मनो ऽप॑ क्रामति । \newline
25. अप॑ क्रामति क्राम॒ त्यपाप॑ क्रामति म॒नोता॑यै म॒नोता॑यै क्राम॒ त्यपाप॑ क्रामति म॒नोता॑यै । \newline
26. क्रा॒म॒ति॒ म॒नोता॑यै म॒नोता॑यै क्रामति क्रामति म॒नोता॑यै ह॒विषो॑ ह॒विषो॑ म॒नोता॑यै क्रामति क्रामति म॒नोता॑यै ह॒विषः॑ । \newline
27. म॒नोता॑यै ह॒विषो॑ ह॒विषो॑ म॒नोता॑यै म॒नोता॑यै ह॒विषो॑ ऽवदी॒यमा॑नस्या वदी॒यमा॑नस्य ह॒विषो॑ म॒नोता॑यै म॒नोता॑यै ह॒विषो॑ ऽवदी॒यमा॑नस्य । \newline
28. ह॒विषो॑ ऽवदी॒यमा॑नस्या वदी॒यमा॑नस्य ह॒विषो॑ ह॒विषो॑ ऽवदी॒यमा॑न॒स्या न्व न्‍व॑ वदी॒यमा॑नस्य ह॒विषो॑ ह॒विषो॑ ऽवदी॒यमा॑न॒ स्यानु॑ । \newline
29. अ॒व॒दी॒यमा॑न॒स्यान् वन् व॑ वदी॒यमा॑नस्या वदी॒यमा॑न॒ स्यानु॑ ब्रूहि ब्रू॒ह्यन्व॑ वदी॒यमा॑नस्या वदी॒यमा॑न॒ स्यानु॑ ब्रूहि । \newline
30. अ॒व॒दी॒यमा॑न॒स्येत्य॑व - दी॒यमा॑नस्य । \newline
31. अनु॑ ब्रूहि ब्रू॒ह्यन् वनु॑ ब्रू॒ही तीति॑ ब्रू॒ह्यन् वनु॑ ब्रू॒हीति॑ । \newline
32. ब्रू॒हीतीति॑ ब्रूहि ब्रू॒हीत्या॑ हा॒हेति॑ ब्रूहि ब्रू॒ही त्या॑ह । \newline
33. इत्या॑हा॒हे तीत्या॑ह॒ मनो॒ मन॑ आ॒हे तीत्या॑ह॒ मनः॑ । \newline
34. आ॒ह॒ मनो॒ मन॑ आहाह॒ मन॑ ए॒वैव मन॑ आहाह॒ मन॑ ए॒व । \newline
35. मन॑ ए॒वैव मनो॒ मन॑ ए॒वास्या᳚ स्यै॒व मनो॒ मन॑ ए॒वास्य॑ । \newline
36. ए॒वास्या᳚ स्यै॒वै वास्या वावा᳚ स्यै॒वै वास्याव॑ । \newline
37. अ॒स्या वावा᳚स्या ॒स्याव॑ रुन्धे रु॒न्धे ऽवा᳚स्या॒ स्याव॑ रुन्धे । \newline
38. अव॑ रुन्धे रु॒न्धे ऽवाव॑ रुन्ध॒ एका॑द॒ शैका॑दश रु॒न्धे ऽवाव॑ रुन्ध॒ एका॑दश । \newline
39. रु॒न्ध॒ एका॑द॒ शैका॑दश रुन्धे रुन्ध॒ एका॑दशा व॒दाना᳚ न्यव॒दाना॒ न्येका॑दश रुन्धे रुन्ध॒ एका॑दशा व॒दाना॑नि । \newline
40. एका॑दशा व॒दाना᳚ न्यव॒दाना॒ न्येका॑द॒ शैका॑दशा व॒दाना॒ न्यवावा॑ व॒दाना॒ न्येका॑द॒ शैका॑दशा व॒दाना॒ न्यव॑ । \newline
41. अ॒व॒दाना॒ न्यवावा॑ व॒दाना᳚ न्यव॒दाना॒ न्यव॑ द्यति द्य॒त् यवा॑ व॒दाना᳚ न्यव॒दाना॒ न्यव॑ द्यति । \newline
42. अ॒व॒दाना॒नीत्य॑व - दाना॑नि । \newline
43. अव॑ द्यति द्य॒त्य वाव॑ द्यति॒ दश॒ दश॑ द्य॒त्य वाव॑ द्यति॒ दश॑ । \newline
44. द्य॒ति॒ दश॒ दश॑ द्यति द्यति॒ दश॒ वै वै दश॑ द्यति द्यति॒ दश॒ वै । \newline
45. दश॒ वै वै दश॒ दश॒ वै प॒शोः प॒शोर् वै दश॒ दश॒ वै प॒शोः । \newline
46. वै प॒शोः प॒शोर् वै वै प॒शोः प्रा॒णाः प्रा॒णाः प॒शोर् वै वै प॒शोः प्रा॒णाः । \newline
47. प॒शोः प्रा॒णाः प्रा॒णाः प॒शोः प॒शोः प्रा॒णा आ॒त्मा ऽऽत्मा प्रा॒णाः प॒शोः प॒शोः प्रा॒णा आ॒त्मा । \newline
48. प्रा॒णा आ॒त्मा ऽऽत्मा प्रा॒णाः प्रा॒णा आ॒त्मैका॑द॒श ए॑काद॒श आ॒त्मा प्रा॒णाः प्रा॒णा आ॒त्मैका॑द॒शः । \newline
49. प्रा॒णा इति॑ प्र - अ॒नाः । \newline
50. आ॒त्मैका॑द॒श ए॑काद॒श आ॒त्मा ऽऽत्मैका॑द॒शो यावा॒न्॒. यावा॑ नेकाद॒श आ॒त्मा ऽऽत्मैका॑द॒शो यावान्॑ । \newline
51. ए॒का॒द॒शो यावा॒न्॒. यावा॑ नेकाद॒श ए॑काद॒शो यावा॑ने॒ वैव यावा॑ नेकाद॒श ए॑काद॒शो यावा॑ने॒व । \newline
52. यावा॑ ने॒वैव यावा॒न्॒. यावा॑ने॒व प॒शुः प॒शु रे॒व यावा॒न्॒. यावा॑ने॒व प॒शुः । \newline
53. ए॒व प॒शुः प॒शु रे॒वैव प॒शु स्तस्य॒ तस्य॑ प॒शु रे॒वैव प॒शु स्तस्य॑ । \newline
54. प॒शु स्तस्य॒ तस्य॑ प॒शुः प॒शु स्तस्या वाव॒ तस्य॑ प॒शुः प॒शु स्तस्याव॑ । \newline
55. तस्या वाव॒ तस्य॒ तस्याव॑ द्यति द्य॒त्यव॒ तस्य॒ तस्याव॑ द्यति । \newline
56. अव॑ द्यति द्य॒त्य वाव॑ द्यति॒ हृद॑यस्य॒ हृद॑यस्य द्य॒त्य वाव॑ द्यति॒ हृद॑यस्य । \newline
\pagebreak
\markright{ TS 6.3.10.4  \hfill https://www.vedavms.in \hfill}

\section{ TS 6.3.10.4 }

\textbf{TS 6.3.10.4 } \newline
\textbf{Samhita Paata} \newline

द्यति॒ हृद॑य॒स्याग्रेऽव॑ द्य॒त्यथ॑ जि॒ह्वाया॒ अथ॒ वक्ष॑सो॒ यद्वै हृद॑येनाभि॒गच्छ॑ति॒ तज्जि॒ह्वया॑ वदति॒ यज्जि॒ह्वया॒ वद॑ति॒ तदुर॒सोऽधि॒ निर्व॑दत्ये॒तद्वै प॒शोर्य॑थापू॒र्वं ॅयस्यै॒वम॑व॒दाय॑ यथा॒काम॒-मुत्त॑रेषामव॒द्यति॑ यथा पू॒र्वमे॒वास्य॑ प॒शोरव॑त्तं भवति मद्ध्य॒तो गु॒दस्याव॑ द्यति मद्ध्य॒तो हि प्रा॒ण उ॑त्त॒मस्याव॑ द्यत्यु- [  ] \newline

\textbf{Pada Paata} \newline

द्य॒ति॒ । हृद॑यस्य । अग्रे᳚ । अवेति॑ । द्य॒ति॒ । अथ॑ । जि॒ह्वायाः᳚ । अथ॑ । वक्ष॑सः । यत् । वै । हृद॑येन । अ॒भि॒गच्छ॒तीत्य॑भि - गच्छ॑ति । तत् । जि॒ह्वया᳚ । व॒द॒ति॒ । यत् । जि॒ह्वया᳚ । वद॑ति । तत् । उर॑सः । अधि॑ । निरिति॑ । व॒द॒ति॒ । ए॒तत् । वै । प॒शोः । य॒था॒पू॒र्वमिति॑ यथा - पू॒र्वम् । यस्य॑ । ए॒वम् । अ॒व॒दायेत्य॑व - दाय॑ । य॒था॒काम॒मिति॑ यथा - काम᳚म् । उत्त॑रेषा॒मित्युत् - त॒रे॒षा॒म् । अ॒व॒द्यतीत्य॑व - द्यति॑ । य॒था॒पू॒र्वमिति॑ यथा - पू॒र्वम् । ए॒व । अ॒स्य॒ । प॒शोः । अव॑त्तम् । भ॒व॒ति॒ । म॒द्ध्य॒तः । गु॒दस्य॑ । अवेति॑ । द्य॒ति॒ । म॒द्ध्य॒तः । हि । प्रा॒ण इति॑ प्र - अ॒नः । उ॒त्त॒मस्येत्यु॑त् - त॒मस्य॑ । अवेति॑ । द्य॒ति॒ ।  \newline


\textbf{Krama Paata} \newline

द्य॒ति॒ हृद॑यस्य । हृद॑य॒स्याग्रे᳚ । अग्रेऽव॑ । अव॑ द्यति । द्य॒त्यथ॑ । अथ॑ जि॒ह्वायाः᳚ । जि॒ह्वाया॒ अथ॑ । अथ॒ वक्ष॑सः । वक्ष॑सो॒ यत् । यद् वै । वै हृद॑येन । हृद॑येनाभि॒गच्छ॑ति । अ॒भि॒गच्छ॑ति॒ तत् । अ॒भि॒गच्छ॒तीत्य॑भि - गच्छ॑ति । तज् जि॒ह्वया᳚ । जि॒ह्वया॑ वदति । व॒द॒ति॒ यत् । यज् जि॒ह्वया᳚ । जि॒ह्वया॒ वद॑ति । वद॑ति॒ तत् । तदुर॑सः । उर॒सोऽधि॑ । अधि॒ निः । निर् व॑दति । व॒द॒त्ये॒तत् । ए॒तद् वै । वै प॒शोः । प॒शोर्,य॑थापू॒र्वम् । य॒था॒पू॒र्वम् ॅयस्य॑ । य॒था॒पू॒र्वमिति॑ यथा - पू॒र्वम् । यस्यै॒वम् । ए॒वम॑व॒दाय॑ । अ॒व॒दाय॑ यथा॒काम᳚म् । अ॒व॒दायेत्य॑व - दाय॑ । य॒था॒काम॒मुत्त॑रेषाम् । य॒था॒काम॒मिति॑ यथा - काम᳚म् । उत्त॑रेषामव॒द्यति॑ । उत्त॑रेषा॒मित्युत् - त॒रे॒षा॒म् । अ॒व॒द्यति॑ यथापू॒र्वम् । अ॒व॒द्यतीत्य॑व - द्यति॑ । य॒था॒पू॒र्वमे॒व । य॒था॒पू॒र्वमिति॑ यथा - पू॒र्वम् । ए॒वास्य॑ । अ॒स्य॒ प॒शोः । प॒शोरव॑त्तम् । अव॑त्तम् भवति । भ॒व॒ति॒ म॒द्ध्य॒तः । म॒द्ध्य॒तो गु॒दस्य॑ । गु॒दस्याव॑ । अव॑ द्यति । द्य॒ति॒ म॒द्ध्य॒तः । म॒द्ध्य॒तो हि । हि प्रा॒णः । प्रा॒ण उ॑त्त॒मस्य॑ । प्रा॒ण इति॑ प्र - अ॒नः । उ॒त्त॒मस्याव॑ । उ॒त्त॒मस्येत्यु॑त् - त॒मस्य॑ । अव॑ द्यति । द्य॒त्यु॒त्त॒मः \newline

\textbf{Jatai Paata} \newline

1. द्य॒ति॒ हृद॑यस्य॒ हृद॑यस्य द्यति द्यति॒ हृद॑यस्य । \newline
2. हृद॑य॒ स्याग्रे ऽग्रे॒ हृद॑यस्य॒ हृद॑य॒ स्याग्रे᳚ । \newline
3. अग्रे ऽवावा ग्रे ऽग्रे ऽव॑ । \newline
4. अव॑ द्यति द्य॒त्यवाव॑ द्यति । \newline
5. द्य॒त्यथाथ॑ द्यति द्य॒त्यथ॑ । \newline
6. अथ॑ जि॒ह्वाया॑ जि॒ह्वाया॒ अथाथ॑ जि॒ह्वायाः᳚ । \newline
7. जि॒ह्वाया॒ अथाथ॑ जि॒ह्वाया॑ जि॒ह्वाया॒ अथ॑ । \newline
8. अथ॒ वक्ष॑सो॒ वक्ष॒सो ऽथाथ॒ वक्ष॑सः । \newline
9. वक्ष॑सो॒ यद् यद् वक्ष॑सो॒ वक्ष॑सो॒ यत् । \newline
10. यद् वै वै यद् यद् वै । \newline
11. वै हृद॑येन॒ हृद॑येन॒ वै वै हृद॑येन । \newline
12. हृद॑येना भि॒गच्छ॑ त्यभि॒गच्छ॑ति॒ हृद॑येन॒ हृद॑येना भि॒गच्छ॑ति । \newline
13. अ॒भि॒गच्छ॑ति॒ तत् तद॑भि॒गच्छ॑ त्यभि॒गच्छ॑ति॒ तत् । \newline
14. अ॒भि॒गच्छ॒तीत्य॑भि - गच्छ॑ति । \newline
15. तज् जि॒ह्वया॑ जि॒ह्वया॒ तत् तज् जि॒ह्वया᳚ । \newline
16. जि॒ह्वया॑ वदति वदति जि॒ह्वया॑ जि॒ह्वया॑ वदति । \newline
17. व॒द॒ति॒ यद् यद् व॑दति वदति॒ यत् । \newline
18. यज् जि॒ह्वया॑ जि॒ह्वया॒ यद् यज् जि॒ह्वया᳚ । \newline
19. जि॒ह्वया॒ वद॑ति॒ वद॑ति जि॒ह्वया॑ जि॒ह्वया॒ वद॑ति । \newline
20. वद॑ति॒ तत् तद् वद॑ति॒ वद॑ति॒ तत् । \newline
21. तदुर॑स॒ उर॑स॒ स्तत् तदुर॑सः । \newline
22. उर॒सो ऽध्य ध्युर॑स॒ उर॒सो ऽधि॑ । \newline
23. अधि॒ निर् णिरध्यधि॒ निः । \newline
24. निर् व॑दति वदति॒ निर् णिर् व॑दति । \newline
25. व॒द॒ त्ये॒त दे॒तद् व॑दति वद त्ये॒तत् । \newline
26. ए॒तद् वै वा ए॒त दे॒तद् वै । \newline
27. वै प॒शोः प॒शोर् वै वै प॒शोः । \newline
28. प॒शोर् य॑थापू॒र्वं ॅय॑थापू॒र्वम् प॒शोः प॒शोर् य॑थापू॒र्वम् । \newline
29. य॒था॒पू॒र्वं ॅयस्य॒ यस्य॑ यथापू॒र्वं ॅय॑थापू॒र्वं ॅयस्य॑ । \newline
30. य॒था॒पू॒र्वमिति॑ यथा - पू॒र्वम् । \newline
31. यस्यै॒व मे॒वं ॅयस्य॒ यस्यै॒वम् । \newline
32. ए॒व म॑व॒दाया॑ व॒दायै॒व मे॒व म॑व॒दाय॑ । \newline
33. अ॒व॒दाय॑ यथा॒कामं॑ ॅयथा॒काम॑ मव॒दाया॑ व॒दाय॑ यथा॒काम᳚म् । \newline
34. अ॒व॒दायेत्य॑व - दाय॑ । \newline
35. य॒था॒काम॒ मुत्त॑रेषा॒ मुत्त॑रेषां ॅयथा॒कामं॑ ॅयथा॒काम॒ मुत्त॑रेषाम् । \newline
36. य॒था॒काम॒मिति॑ यथा - काम᳚म् । \newline
37. उत्त॑रेषा मव॒द्य त्य॑व॒द्य त्युत्त॑रेषा॒ मुत्त॑रेषा मव॒द्यति॑ । \newline
38. उत्त॑रेषा॒मित्युत् - त॒रे॒षा॒म् । \newline
39. अ॒व॒द्यति॑ यथापू॒र्वं ॅय॑थापू॒र्व म॑व॒द्य त्य॑व॒द्यति॑ यथापू॒र्वम् । \newline
40. अ॒व॒द्यतीत्य॑व - द्यति॑ । \newline
41. य॒था॒पू॒र्व मे॒वैव य॑थापू॒र्वं ॅय॑थापू॒र्व मे॒व । \newline
42. य॒था॒पू॒र्वमिति॑ यथा - पू॒र्वम् । \newline
43. ए॒वास्या᳚ स्यै॒वै वास्य॑ । \newline
44. अ॒स्य॒ प॒शोः प॒शोर॑ स्यास्य प॒शोः । \newline
45. प॒शो रव॑त्त॒ मव॑त्तम् प॒शोः प॒शो रव॑त्तम् । \newline
46. अव॑त्तम् भवति भव॒ त्यव॑त्त॒ मव॑त्तम् भवति । \newline
47. भ॒व॒ति॒ म॒द्ध्य॒तो म॑द्ध्य॒तो भ॑वति भवति मद्ध्य॒तः । \newline
48. म॒द्ध्य॒तो गु॒दस्य॑ गु॒दस्य॑ मद्ध्य॒तो म॑द्ध्य॒तो गु॒दस्य॑ । \newline
49. गु॒दस्या वाव॑ गु॒दस्य॑ गु॒दस्याव॑ । \newline
50. अव॑ द्यति द्य॒त्यवाव॑ द्यति । \newline
51. द्य॒ति॒ म॒द्ध्य॒तो म॑द्ध्य॒तो द्य॑ति द्यति मद्ध्य॒तः । \newline
52. म॒द्ध्य॒तो हि हि म॑द्ध्य॒तो म॑द्ध्य॒तो हि । \newline
53. हि प्रा॒णः प्रा॒णो हि हि प्रा॒णः । \newline
54. प्रा॒ण उ॑त्त॒म स्यो᳚त्त॒मस्य॑ प्रा॒णः प्रा॒ण उ॑त्त॒मस्य॑ । \newline
55. प्रा॒ण इति॑ प्र - अ॒नः । \newline
56. उ॒त्त॒मस्या वावो᳚ त्त॒मस्यो᳚त्त॒मस्याव॑ । \newline
57. उ॒त्त॒मस्येत्यु॑त् - त॒मस्य॑ । \newline
58. अव॑ द्यति द्य॒त्यवाव॑ द्यति । \newline
59. द्य॒त्यु॒त्त॒म उ॑त्त॒मो द्य॑ति द्यत्युत्त॒मः । \newline

\textbf{Ghana Paata } \newline

1. द्य॒ति॒ हृद॑यस्य॒ हृद॑यस्य द्यति द्यति॒ हृद॑य॒ स्याग्रे ऽग्रे॒ हृद॑यस्य द्यति द्यति॒ हृद॑य॒ स्याग्रे᳚ । \newline
2. हृद॑य॒ स्याग्रे ऽग्रे॒ हृद॑यस्य॒ हृद॑य॒ स्याग्रे ऽवावाग्रे॒ हृद॑यस्य॒ हृद॑य॒ स्याग्रे ऽव॑ । \newline
3. अग्रे ऽवावाग्रे ऽग्रे ऽव॑ द्यति द्य॒त्य वाग्रे ऽग्रे ऽव॑ द्यति । \newline
4. अव॑ द्यति द्य॒त्य वाव॑ द्य॒त्य थाथ॑ द्य॒त्य वाव॑ द्य॒त्यथ॑ । \newline
5. द्य॒त्य थाथ॑ द्यति द्य॒त्यथ॑ जि॒ह्वाया॑ जि॒ह्वाया॒ अथ॑ द्यति द्य॒त्यथ॑ जि॒ह्वायाः᳚ । \newline
6. अथ॑ जि॒ह्वाया॑ जि॒ह्वाया॒ अथाथ॑ जि॒ह्वाया॒ अथाथ॑ जि॒ह्वाया॒ अथाथ॑ जि॒ह्वाया॒ अथ॑ । \newline
7. जि॒ह्वाया॒ अथाथ॑ जि॒ह्वाया॑ जि॒ह्वाया॒ अथ॒ वक्ष॑सो॒ वक्ष॒सो ऽथ॑ जि॒ह्वाया॑ जि॒ह्वाया॒ अथ॒ वक्ष॑सः । \newline
8. अथ॒ वक्ष॑सो॒ वक्ष॒सो ऽथाथ॒ वक्ष॑सो॒ यद् यद् वक्ष॒सो ऽथाथ॒ वक्ष॑सो॒ यत् । \newline
9. वक्ष॑सो॒ यद् यद् वक्ष॑सो॒ वक्ष॑सो॒ यद् वै वै यद् वक्ष॑सो॒ वक्ष॑सो॒ यद् वै । \newline
10. यद् वै वै यद् यद् वै हृद॑येन॒ हृद॑येन॒ वै यद् यद् वै हृद॑येन । \newline
11. वै हृद॑येन॒ हृद॑येन॒ वै वै हृद॑येना भि॒गच्छ॑ त्यभि॒गच्छ॑ति॒ हृद॑येन॒ वै वै हृद॑येना भि॒गच्छ॑ति । \newline
12. हृद॑येना भि॒गच्छ॑ त्यभि॒गच्छ॑ति॒ हृद॑येन॒ हृद॑येना भि॒गच्छ॑ति॒ तत् तद॑भि॒गच्छ॑ति॒ हृद॑येन॒ हृद॑येना भि॒गच्छ॑ति॒ तत् । \newline
13. अ॒भि॒गच्छ॑ति॒ तत् तद॑भि॒गच्छ॑ त्यभि॒गच्छ॑ति॒ तज् जि॒ह्वया॑ जि॒ह्वया॒ तद॑भि॒गच्छ॑ त्यभि॒गच्छ॑ति॒ तज् जि॒ह्वया᳚ । \newline
14. अ॒भि॒गच्छ॒तीत्य॑भि - गच्छ॑ति । \newline
15. तज् जि॒ह्वया॑ जि॒ह्वया॒ तत् तज् जि॒ह्वया॑ वदति वदति जि॒ह्वया॒ तत् तज् जि॒ह्वया॑ वदति । \newline
16. जि॒ह्वया॑ वदति वदति जि॒ह्वया॑ जि॒ह्वया॑ वदति॒ यद् यद् व॑दति जि॒ह्वया॑ जि॒ह्वया॑ वदति॒ यत् । \newline
17. व॒द॒ति॒ यद् यद् व॑दति वदति॒ यज् जि॒ह्वया॑ जि॒ह्वया॒ यद् व॑दति वदति॒ यज् जि॒ह्वया᳚ । \newline
18. यज् जि॒ह्वया॑ जि॒ह्वया॒ यद् यज् जि॒ह्वया॒ वद॑ति॒ वद॑ति जि॒ह्वया॒ यद् यज् जि॒ह्वया॒ वद॑ति । \newline
19. जि॒ह्वया॒ वद॑ति॒ वद॑ति जि॒ह्वया॑ जि॒ह्वया॒ वद॑ति॒ तत् तद् वद॑ति जि॒ह्वया॑ जि॒ह्वया॒ वद॑ति॒ तत् । \newline
20. वद॑ति॒ तत् तद् वद॑ति॒ वद॑ति॒ तदुर॑स॒ उर॑स॒ स्तद् वद॑ति॒ वद॑ति॒ तदुर॑सः । \newline
21. तदुर॑स॒ उर॑स॒ स्तत् तदुर॒सो ऽध्य ध्युर॑स॒ स्तत् तदुर॒सो ऽधि॑ । \newline
22. उर॒सो ऽध्य ध्युर॑स॒ उर॒सो ऽधि॒ निर् णिर ध्युर॑स॒ उर॒सो ऽधि॒ निः । \newline
23. अधि॒ निर् णिर ध्यधि॒ निर् व॑दति वदति॒ निर ध्यधि॒ निर् व॑दति । \newline
24. निर् व॑दति वदति॒ निर् णिर् व॑द त्ये॒त दे॒तद् व॑दति॒ निर् णिर् व॑द त्ये॒तत् । \newline
25. व॒द॒ त्ये॒त दे॒तद् व॑दति वद त्ये॒तद् वै वा ए॒तद् व॑दति वद त्ये॒तद् वै । \newline
26. ए॒तद् वै वा ए॒त दे॒तद् वै प॒शोः प॒शोर् वा ए॒त दे॒तद् वै प॒शोः । \newline
27. वै प॒शोः प॒शोर् वै वै प॒शोर् य॑थापू॒र्वं ॅय॑थापू॒र्वम् प॒शोर् वै वै प॒शोर् य॑थापू॒र्वम् । \newline
28. प॒शोर् य॑थापू॒र्वं ॅय॑थापू॒र्वम् प॒शोः प॒शोर् य॑थापू॒र्वं ॅयस्य॒ यस्य॑ यथापू॒र्वम् प॒शोः प॒शोर् य॑थापू॒र्वं ॅयस्य॑ । \newline
29. य॒था॒पू॒र्वं ॅयस्य॒ यस्य॑ यथापू॒र्वं ॅय॑थापू॒र्वं ॅयस्यै॒व मे॒वं ॅयस्य॑ यथापू॒र्वं ॅय॑थापू॒र्वं ॅयस्यै॒वम् । \newline
30. य॒था॒पू॒र्वमिति॑ यथा - पू॒र्वम् । \newline
31. यस्यै॒व मे॒वं ॅयस्य॒ यस्यै॒व म॑व॒दाया॑ व॒दायै॒वं ॅयस्य॒ यस्यै॒व म॑व॒दाय॑ । \newline
32. ए॒व म॑व॒दाया॑ व॒दायै॒व मे॒व म॑व॒दाय॑ यथा॒कामं॑ ॅयथा॒काम॑ मव॒दायै॒व मे॒व म॑व॒दाय॑ यथा॒काम᳚म् । \newline
33. अ॒व॒दाय॑ यथा॒कामं॑ ॅयथा॒काम॑ मव॒दाया॑ व॒दाय॑ यथा॒काम॒ मुत्त॑रेषा॒ मुत्त॑रेषां ॅयथा॒काम॑ मव॒दाया॑ व॒दाय॑ यथा॒काम॒ मुत्त॑रेषाम् । \newline
34. अ॒व॒दायेत्य॑व - दाय॑ । \newline
35. य॒था॒काम॒ मुत्त॑रेषा॒ मुत्त॑रेषां ॅयथा॒कामं॑ ॅयथा॒काम॒ मुत्त॑रेषा मव॒द्य त्य॑व॒द्य त्युत्त॑रेषां ॅयथा॒कामं॑ ॅयथा॒काम॒ मुत्त॑रेषा मव॒द्यति॑ । \newline
36. य॒था॒काम॒मिति॑ यथा - काम᳚म् । \newline
37. उत्त॑रेषा मव॒द्य त्य॑व॒द्य त्युत्त॑रेषा॒ मुत्त॑रेषा मव॒द्यति॑ यथापू॒र्वं ॅय॑थापू॒र्व म॑व॒द्य त्युत्त॑रेषा॒ मुत्त॑रेषा मव॒द्यति॑ यथापू॒र्वम् । \newline
38. उत्त॑रेषा॒मित्युत् - त॒रे॒षा॒म् । \newline
39. अ॒व॒द्यति॑ यथापू॒र्वं ॅय॑थापू॒र्व म॑व॒द्य त्य॑व॒द्यति॑ यथापू॒र्व मे॒वैव य॑थापू॒र्व म॑व॒द्य त्य॑व॒द्यति॑ यथापू॒र्व मे॒व । \newline
40. अ॒व॒द्यतीत्य॑व - द्यति॑ । \newline
41. य॒था॒पू॒र्व मे॒वैव य॑थापू॒र्वं ॅय॑थापू॒र्व मे॒वास्या᳚ स्यै॒व य॑थापू॒र्वं ॅय॑थापू॒र्व मे॒वास्य॑ । \newline
42. य॒था॒पू॒र्वमिति॑ यथा - पू॒र्वम् । \newline
43. ए॒वास्या᳚ स्यै॒वै वास्य॑ प॒शोः प॒शो र॑स्यै॒वै वास्य॑ प॒शोः । \newline
44. अ॒स्य॒ प॒शोः प॒शो र॑स्यास्य प॒शो रव॑त्त॒ मव॑त्तम् प॒शो र॑स्यास्य प॒शो रव॑त्तम् । \newline
45. प॒शो रव॑त्त॒ मव॑त्तम् प॒शोः प॒शो रव॑त्तम् भवति भव॒ त्यव॑त्तम् प॒शोः प॒शो रव॑त्तम् भवति । \newline
46. अव॑त्तम् भवति भव॒ त्यव॑त्त॒ मव॑त्तम् भवति मद्ध्य॒तो म॑द्ध्य॒तो भ॑व॒ त्यव॑त्त॒ मव॑त्तम् भवति मद्ध्य॒तः । \newline
47. भ॒व॒ति॒ म॒द्ध्य॒तो म॑द्ध्य॒तो भ॑वति भवति मद्ध्य॒तो गु॒दस्य॑ गु॒दस्य॑ मद्ध्य॒तो भ॑वति भवति मद्ध्य॒तो गु॒दस्य॑ । \newline
48. म॒द्ध्य॒तो गु॒दस्य॑ गु॒दस्य॑ मद्ध्य॒तो म॑द्ध्य॒तो गु॒दस्या वाव॑ गु॒दस्य॑ मद्ध्य॒तो म॑द्ध्य॒तो गु॒दस्याव॑ । \newline
49. गु॒दस्या वाव॑ गु॒दस्य॑ गु॒दस्याव॑ द्यति द्य॒त्यव॑ गु॒दस्य॑ गु॒दस्याव॑ द्यति । \newline
50. अव॑ द्यति द्य॒त्य वाव॑ द्यति मद्ध्य॒तो म॑द्ध्य॒तो द्य॒त्य वाव॑ द्यति मद्ध्य॒तः । \newline
51. द्य॒ति॒ म॒द्ध्य॒तो म॑द्ध्य॒तो द्य॑ति द्यति मद्ध्य॒तो हि हि म॑द्ध्य॒तो द्य॑ति द्यति मद्ध्य॒तो हि । \newline
52. म॒द्ध्य॒तो हि हि म॑द्ध्य॒तो म॑द्ध्य॒तो हि प्रा॒णः प्रा॒णो हि म॑द्ध्य॒तो म॑द्ध्य॒तो हि प्रा॒णः । \newline
53. हि प्रा॒णः प्रा॒णो हि हि प्रा॒ण उ॑त्त॒म स्यो᳚त्त॒मस्य॑ प्रा॒णो हि हि प्रा॒ण उ॑त्त॒मस्य॑ । \newline
54. प्रा॒ण उ॑त्त॒म स्यो᳚त्त॒मस्य॑ प्रा॒णः प्रा॒ण उ॑त्त॒मस्या वावो᳚त्त॒मस्य॑ प्रा॒णः प्रा॒ण उ॑त्त॒मस्याव॑ । \newline
55. प्रा॒ण इति॑ प्र - अ॒नः । \newline
56. उ॒त्त॒मस्या वावो᳚ त्त॒म स्यो᳚त्त॒मस्याव॑ द्यति द्य॒त्यवो᳚त्त॒म स्यो᳚त्त॒मस्याव॑ द्यति । \newline
57. उ॒त्त॒मस्येत्यु॑त् - त॒मस्य॑ । \newline
58. अव॑ द्यति द्य॒त्यवाव॑ द्यत्युत्त॒म उ॑त्त॒मो द्य॒त्य वाव॑ द्यत्युत्त॒मः । \newline
59. द्य॒त्यु॒त्त॒म उ॑त्त॒मो द्य॑ति द्यत्युत्त॒मो हि ह्यु॑त्त॒मो द्य॑ति द्यत्युत्त॒मो हि । \newline
\pagebreak
\markright{ TS 6.3.10.5  \hfill https://www.vedavms.in \hfill}

\section{ TS 6.3.10.5 }

\textbf{TS 6.3.10.5 } \newline
\textbf{Samhita Paata} \newline

-त्त॒मो हि प्रा॒णो यदीत॑रं॒ ॅयदीत॑र-मु॒भय॑मे॒वाजा॑मि॒ जाय॑मानो॒ वै ब्रा᳚ह्म॒णस्त्रि॒भिर्.ऋ॑ण॒वा जा॑यते ब्रह्म॒चर्ये॒णर्षि॑भ्यो य॒ज्ञेन॑ दे॒वेभ्यः॑ प्र॒जया॑ पि॒तृभ्य॑ ए॒ष वा अ॑नृ॒णो यः पु॒त्री यज्वा᳚ ब्रह्मचारिवा॒सी तद॑व॒दानै॑रे॒वाव॑ दयत॒ तद॑व॒दाना॑ना-मवदान॒त्वं दे॑वासु॒राः संॅय॑त्ता आस॒न् ते दे॒वा अ॒ग्निम॑ब्रुव॒न् त्वया॑ वी॒रेणासु॑रान॒भि भ॑वा॒मेति॒- [  ] \newline

\textbf{Pada Paata} \newline

उ॒त्त॒म इत्यु॑त् - त॒मः । हि । प्रा॒ण इति॑ प्र - अ॒नः । यदि॑ । इत॑रम् । यदि॑ । इत॑रम् । उ॒भय᳚म् । ए॒व । अजा॑मि । जाय॑मानः । वै । ब्रा॒ह्म॒णः । त्रि॒भिरिति॑ त्रि - भिः । ऋ॒ण॒वेत्यृ॑ण - वा । जा॒य॒ते॒ । ब्र॒ह्म॒चर्ये॒णेति॑ ब्रह्म - चर्ये॑ण । ऋषि॑भ्य॒ इत्यृषि॑ - भ्यः॒ । य॒ज्ञेन॑ । दे॒वेभ्यः॑ । प्र॒जयेति॑ प्र - जया᳚ । पि॒तृभ्य॒ इति॑ पि॒तृ - भ्यः॒ । ए॒षः । वै । अ॒नृ॒णः । यः । पु॒त्री । यज्वा᳚ । ब्र॒ह्म॒चा॒रि॒वा॒सीति॑ ब्रह्मचारि - वा॒सी । तत् । अ॒व॒दानै॒रित्य॑व - दानैः᳚ । ए॒व । अवेति॑ । द॒य॒ते॒ । तत् । अ॒व॒दाना॑ना॒मित्य॑व - दाना॑नाम् । अ॒व॒दा॒न॒त्वमित्य॑वदान - त्वम् । दे॒वा॒सु॒रा इति॑ देव - अ॒सु॒राः । संॅय॑त्ता॒ इति॒ सं - य॒त्ताः॒ । आ॒स॒न्न् । ते । दे॒वाः । अ॒ग्निम् । अ॒ब्रु॒व॒न्न् । त्वया᳚ । वी॒रेण॑ । असु॑रान् । अ॒भीति॑ । भ॒वा॒म॒ । इति॑ ।  \newline


\textbf{Krama Paata} \newline

उ॒त्त॒मो हि । उ॒त्त॒म इत्यु॑त् - त॒मः । हि प्रा॒णः । प्रा॒णो यदि॑ । प्रा॒ण इति॑ प्र - अ॒नः । यदीत॑रम् । इत॑र॒म् ॅयदि॑ । यदीत॑रम् । इत॑रमु॒भय᳚म् । उ॒भय॑मे॒व । ए॒वाजा॑मि । अजा॑मि॒ जाय॑मानः । जाय॑मनो॒ वै । वै ब्रा᳚ह्म॒णः । ब्रा॒ह्म॒णस्त्रि॒भिः । त्रि॒भिर्. ऋ॑ण॒वा । त्रि॒भिरिति॑ त्रि - भिः । ऋ॒ण॒वा जा॑यते । ऋ॒ण॒वेत्यृ॑ण - वा । जा॒य॒ते॒ ब्र॒ह्म॒चर्ये॑ण । ब्र॒ह्म॒चर्ये॒णर्‌षि॑भ्यः । ब्र॒ह्म॒चर्ये॒णेति॑ ब्रह्म - चर्ये॑ण । ऋषि॑भ्यो य॒ज्ञेन॑ । ऋषि॑भ्य॒ इत्यृषि॑ - भ्यः॒ । य॒ज्ञेन॑ दे॒वेभ्यः॑ । दे॒वेभ्यः॑ प्र॒जया᳚ । प्र॒जया॑ पि॒तृभ्यः॑ । प्र॒जयेति॑ प्र - जया᳚ । पि॒तृभ्य॑ ए॒षः । पि॒तृभ्य॒ इति॑ पि॒तृ - भ्यः॒ । ए॒ष वै । वा अ॑नृ॒णः । अ॒नृ॒णो यः । यः पु॒त्री । पु॒त्री यज्वा᳚ । यज्वा᳚ ब्रह्मचारिवा॒सी । ब्र॒ह्म॒चा॒रि॒वा॒सी तत् । ब्र॒ह्म॒चा॒रि॒वा॒सीति॑ ब्रह्मचारि - वा॒सी । तद॑व॒दानैः᳚ । अ॒व॒दानै॑रे॒व । अ॒व॒दानै॒रित्य॑व - दानैः᳚ । ए॒वाव॑ । अव॑ दयते । द॒य॒ते॒ तत् । तद॑व॒दाना॑नाम् । अ॒व॒दाना॑ना,मवदान॒त्वम् । अ॒व॒दाना॑ना॒मित्य॑व - दाना॑नाम् । अ॒व॒दा॒न॒त्वम् दे॑वासु॒राः । अ॒व॒दा॒न॒त्वमित्य॑वदान - त्वम् । दे॒वा॒सु॒राः सम्ॅय॑त्ताः । दे॒वा॒सु॒रा इति॑ देव - अ॒सु॒राः । सम्ॅय॑त्ता आसन्न् । सम्ॅय॑त्ता॒ इति॒ सम् - य॒त्ताः॒ । आ॒स॒न् ते । ते दे॒वाः । दे॒वा अ॒ग्निम् । अ॒ग्निम॑ब्रुवन्न् । अ॒ब्रु॒व॒न् त्वया᳚ । त्वया॑ वी॒रेण॑ । वी॒रेणासु॑रान् । असु॑रान॒भि । अ॒भि भ॑वाम । 
भ॒वा॒मेति॑ ( ) । इति॒ सः \newline

\textbf{Jatai Paata} \newline

1. उ॒त्त॒मो हि ह्यु॑त्त॒म उ॑त्त॒मो हि । \newline
2. उ॒त्त॒म इत्यु॑त् - त॒मः । \newline
3. हि प्रा॒णः प्रा॒णो हि हि प्रा॒णः । \newline
4. प्रा॒णो यदि॒ यदि॑ प्रा॒णः प्रा॒णो यदि॑ । \newline
5. प्रा॒ण इति॑ प्र - अ॒नः । \newline
6. यदीत॑र॒ मित॑रं॒ ॅयदि॒ यदीत॑रम् । \newline
7. इत॑रं॒ ॅयदि॒ यदीत॑र॒ मित॑रं॒ ॅयदि॑ । \newline
8. यदीत॑र॒ मित॑रं॒ ॅयदि॒ यदीत॑रम् । \newline
9. इत॑र मु॒भय॑ मु॒भय॒ मित॑र॒ मित॑र मु॒भय᳚म् । \newline
10. उ॒भय॑ मे॒वै वोभय॑ मु॒भय॑ मे॒व । \newline
11. ए॒वाजा॒ म्यजा᳚ म्ये॒वै वाजा॑मि । \newline
12. अजा॑मि॒ जाय॑मानो॒ जाय॑मा॒नो ऽजा॒म्यजा॑मि॒ जाय॑मानः । \newline
13. जाय॑मानो॒ वै वै जाय॑मानो॒ जाय॑मानो॒ वै । \newline
14. वै ब्रा᳚ह्म॒णो ब्रा᳚ह्म॒णो वै वै ब्रा᳚ह्म॒णः । \newline
15. ब्रा॒ह्म॒ण स्त्रि॒भि स्त्रि॒भिर् ब्रा᳚ह्म॒णो ब्रा᳚ह्म॒ण स्त्रि॒भिः । \newline
16. त्रि॒भिर्. ऋ॑ण॒व र्‌ण॒वा त्रि॒भि स्त्रि॒भिर्. ऋ॑ण॒वा । \newline
17. त्रि॒भिरिति॑ त्रि - भिः । \newline
18. ऋ॒ण॒वा जा॑यते जायत ऋण॒व र्‌ण॒वा जा॑यते । \newline
19. ऋ॒ण॒वेत्यृ॑ण - वा । \newline
20. जा॒य॒ते॒ ब्र॒ह्म॒चर्ये॑ण ब्रह्म॒चर्ये॑ण जायते जायते ब्रह्म॒चर्ये॑ण । \newline
21. ब्र॒ह्म॒चर्ये॒ण र्.षि॑भ्य॒ ऋषि॑भ्यो ब्रह्म॒चर्ये॑ण ब्रह्म॒चर्ये॒ण र्.षि॑भ्यः । \newline
22. ब्र॒ह्म॒चर्ये॒णेति॑ ब्रह्म - चर्ये॑ण । \newline
23. ऋषि॑भ्यो य॒ज्ञेन॑ य॒ज्ञेन र्.षि॑भ्य॒ ऋषि॑भ्यो य॒ज्ञेन॑ । \newline
24. ऋषि॑भ्य॒ इत्यृषि॑ - भ्यः॒ । \newline
25. य॒ज्ञेन॑ दे॒वेभ्यो॑ दे॒वेभ्यो॑ य॒ज्ञेन॑ य॒ज्ञेन॑ दे॒वेभ्यः॑ । \newline
26. दे॒वेभ्यः॑ प्र॒जया᳚ प्र॒जया॑ दे॒वेभ्यो॑ दे॒वेभ्यः॑ प्र॒जया᳚ । \newline
27. प्र॒जया॑ पि॒तृभ्यः॑ पि॒तृभ्यः॑ प्र॒जया᳚ प्र॒जया॑ पि॒तृभ्यः॑ । \newline
28. प्र॒जयेति॑ प्र - जया᳚ । \newline
29. पि॒तृभ्य॑ ए॒ष ए॒ष पि॒तृभ्यः॑ पि॒तृभ्य॑ ए॒षः । \newline
30. पि॒तृभ्य॒ इति॑ पि॒तृ - भ्यः॒ । \newline
31. ए॒ष वै वा ए॒ष ए॒ष वै । \newline
32. वा अ॑नृ॒णो अ॑नृ॒णो वै वा अ॑नृ॒णः । \newline
33. अ॒नृ॒णो यो यो अ॑नृ॒णो अ॑नृ॒णो यः । \newline
34. यः पु॒त्री पु॒त्री यो यः पु॒त्री । \newline
35. पु॒त्री यज्वा॒ यज्वा॑ पु॒त्री पु॒त्री यज्वा᳚ । \newline
36. यज्वा᳚ ब्रह्मचारिवा॒सी ब्र॑ह्मचारिवा॒सी यज्वा॒ यज्वा᳚ ब्रह्मचारिवा॒सी । \newline
37. ब्र॒ह्म॒चा॒रि॒वा॒सी तत् तद् ब्र॑ह्मचारिवा॒सी ब्र॑ह्मचारिवा॒सी तत् । \newline
38. ब्र॒ह्म॒चा॒रि॒वा॒सीति॑ ब्रह्मचारि - वा॒सी । \newline
39. तद॑व॒दानै॑ रव॒दानै॒ स्तत् तद॑व॒दानैः᳚ । \newline
40. अ॒व॒दानै॑ रे॒वैवा व॒दानै॑ रव॒दानै॑ रे॒व । \newline
41. अ॒व॒दानै॒रित्य॑व - दानैः᳚ । \newline
42. ए॒वावा वै॒वै वाव॑ । \newline
43. अव॑ दयते दय॒ते ऽवाव॑ दयते । \newline
44. द॒य॒ते॒ तत् तद् द॑यते दयते॒ तत् । \newline
45. तद॑व॒दाना॑ना मव॒दाना॑ना॒म् तत् तद॑व॒दाना॑नाम् । \newline
46. अ॒व॒दाना॑ना मवदान॒त्व म॑वदान॒त्व म॑व॒दाना॑ना मव॒दाना॑ना मवदान॒त्वम् । \newline
47. अ॒व॒दाना॑ना॒मित्य॑व - दाना॑नाम् । \newline
48. अ॒व॒दा॒न॒त्वम् दे॑वासु॒रा दे॑वासु॒रा अ॑वदान॒त्व म॑वदान॒त्वम् दे॑वासु॒राः । \newline
49. अ॒व॒दा॒न॒त्वमित्य॑वदान - त्वम् । \newline
50. दे॒वा॒सु॒राः संॅय॑त्ताः॒ संॅय॑त्ता देवासु॒रा दे॑वासु॒राः संॅय॑त्ताः । \newline
51. दे॒वा॒सु॒रा इति॑ देव - अ॒सु॒राः । \newline
52. संॅय॑त्ता आसन् नास॒न् थ्संॅय॑त्ताः॒ संॅय॑त्ता आसन्न् । \newline
53. संॅय॑त्ता॒ इति॒ सं - य॒त्ताः॒ । \newline
54. आ॒स॒न् ते त आ॑सन् नास॒न् ते । \newline
55. ते दे॒वा दे॒वा स्ते ते दे॒वाः । \newline
56. दे॒वा अ॒ग्नि म॒ग्निम् दे॒वा दे॒वा अ॒ग्निम् । \newline
57. अ॒ग्नि म॑ब्रुवन् नब्रुवन् न॒ग्नि म॒ग्नि म॑ब्रुवन्न् । \newline
58. अ॒ब्रु॒व॒न् त्वया॒ त्वया᳚ ऽब्रुवन् नब्रुव॒न् त्वया᳚ । \newline
59. त्वया॑ वी॒रेण॑ वी॒रेण॒ त्वया॒ त्वया॑ वी॒रेण॑ । \newline
60. वी॒रेणा सु॑रा॒ नसु॑रान्. वी॒रेण॑ वी॒रेणा सु॑रान् । \newline
61. असु॑रान॒ भ्य॑भ्य सु॑रा॒न सु॑रा न॒भि । \newline
62. अ॒भि भ॑वाम भवामा॒ भ्य॑भि भ॑वाम । \newline
63. भ॒वा॒मे तीति॑ भवाम भवा॒मेति॑ । \newline
64. इति॒ स स इतीति॒ सः । \newline

\textbf{Ghana Paata } \newline

1. उ॒त्त॒मो हि ह्यु॑त्त॒म उ॑त्त॒मो हि प्रा॒णः प्रा॒णो ह्यु॑त्त॒म उ॑त्त॒मो हि प्रा॒णः । \newline
2. उ॒त्त॒म इत्यु॑त् - त॒मः । \newline
3. हि प्रा॒णः प्रा॒णो हि हि प्रा॒णो यदि॒ यदि॑ प्रा॒णो हि हि प्रा॒णो यदि॑ । \newline
4. प्रा॒णो यदि॒ यदि॑ प्रा॒णः प्रा॒णो यदीत॑र॒ मित॑रं॒ ॅयदि॑ प्रा॒णः प्रा॒णो यदीत॑रम् । \newline
5. प्रा॒ण इति॑ प्र - अ॒नः । \newline
6. यदीत॑र॒ मित॑रं॒ ॅयदि॒ यदीत॑रं॒ ॅयदि॒ यदीत॑रं॒ ॅयदि॒ यदीत॑रं॒ ॅयदि॑ । \newline
7. इत॑रं॒ ॅयदि॒ यदीत॑र॒ मित॑रं॒ ॅयदीत॑र॒ मित॑रं॒ ॅयदीत॑र॒ मित॑रं॒ ॅयदीत॑रम् । \newline
8. यदीत॑र॒ मित॑रं॒ ॅयदि॒ यदीत॑र मु॒भय॑ मु॒भय॒ मित॑रं॒ ॅयदि॒ यदीत॑र मु॒भय᳚म् । \newline
9. इत॑र मु॒भय॑ मु॒भय॒ मित॑र॒ मित॑र मु॒भय॑ मे॒वै वोभय॒ मित॑र॒ मित॑र मु॒भय॑ मे॒व । \newline
10. उ॒भय॑ मे॒वै वोभय॑ मु॒भय॑ मे॒वाजा॒ म्यजा᳚ म्ये॒वो भय॑ मु॒भय॑ मे॒वाजा॑मि । \newline
11. ए॒वाजा॒ म्यजा᳚ म्ये॒वैवा जा॑मि॒ जाय॑मानो॒ जाय॑मा॒नो ऽजा᳚म्ये॒वैवा जा॑मि॒ जाय॑मानः । \newline
12. अजा॑मि॒ जाय॑मानो॒ जाय॑मा॒नो ऽजा॒ म्यजा॑मि॒ जाय॑मानो॒ वै वै जाय॑मा॒नो ऽजा॒ म्यजा॑मि॒ जाय॑मानो॒ वै । \newline
13. जाय॑मानो॒ वै वै जाय॑मानो॒ जाय॑मानो॒ वै ब्रा᳚ह्म॒णो ब्रा᳚ह्म॒णो वै जाय॑मानो॒ जाय॑मानो॒ वै ब्रा᳚ह्म॒णः । \newline
14. वै ब्रा᳚ह्म॒णो ब्रा᳚ह्म॒णो वै वै ब्रा᳚ह्म॒ण स्त्रि॒भि स्त्रि॒भिर् ब्रा᳚ह्म॒णो वै वै ब्रा᳚ह्म॒ण स्त्रि॒भिः । \newline
15. ब्रा॒ह्म॒ण स्त्रि॒भि स्त्रि॒भिर् ब्रा᳚ह्म॒णो ब्रा᳚ह्म॒ण स्त्रि॒भिर्. ऋ॑ण॒व र्‌ण॒वा त्रि॒भिर् ब्रा᳚ह्म॒णो ब्रा᳚ह्म॒ण स्त्रि॒भिर्. ऋ॑ण॒वा । \newline
16. त्रि॒भिर्. ऋ॑ण॒व र्‌ण॒वा त्रि॒भि स्त्रि॒भिर्. ऋ॑ण॒वा जा॑यते जायत ऋण॒वा त्रि॒भि स्त्रि॒भिर्. ऋ॑ण॒वा जा॑यते । \newline
17. त्रि॒भिरिति॑ त्रि - भिः । \newline
18. ऋ॒ण॒वा जा॑यते जायत ऋण॒व र्‌ण॒वा जा॑यते ब्रह्म॒चर्ये॑ण ब्रह्म॒चर्ये॑ण जायत ऋण॒व र्‌ण॒वा जा॑यते ब्रह्म॒चर्ये॑ण । \newline
19. ऋ॒ण॒वेत्यृ॑ण - वा । \newline
20. जा॒य॒ते॒ ब्र॒ह्म॒चर्ये॑ण ब्रह्म॒चर्ये॑ण जायते जायते ब्रह्म॒चर्ये॒ण र्.षि॑भ्य॒ ऋषि॑भ्यो ब्रह्म॒चर्ये॑ण जायते जायते ब्रह्म॒चर्ये॒ण र्.षि॑भ्यः । \newline
21. ब्र॒ह्म॒चर्ये॒ण र्.षि॑भ्य॒ ऋषि॑भ्यो ब्रह्म॒चर्ये॑ण ब्रह्म॒चर्ये॒ण र्.षि॑भ्यो य॒ज्ञेन॑ य॒ज्ञेन र्.षि॑भ्यो ब्रह्म॒चर्ये॑ण ब्रह्म॒चर्ये॒ण र्.षि॑भ्यो य॒ज्ञेन॑ । \newline
22. ब्र॒ह्म॒चर्ये॒णेति॑ ब्रह्म - चर्ये॑ण । \newline
23. ऋषि॑भ्यो य॒ज्ञेन॑ य॒ज्ञेन र्.षि॑भ्य॒ ऋषि॑भ्यो य॒ज्ञेन॑ दे॒वेभ्यो॑ दे॒वेभ्यो॑ य॒ज्ञेन र्.षि॑भ्य॒ ऋषि॑भ्यो य॒ज्ञेन॑ दे॒वेभ्यः॑ । \newline
24. ऋषि॑भ्य॒ इत्यृषि॑ - भ्यः॒ । \newline
25. य॒ज्ञेन॑ दे॒वेभ्यो॑ दे॒वेभ्यो॑ य॒ज्ञेन॑ य॒ज्ञेन॑ दे॒वेभ्यः॑ प्र॒जया᳚ प्र॒जया॑ दे॒वेभ्यो॑ य॒ज्ञेन॑ य॒ज्ञेन॑ दे॒वेभ्यः॑ प्र॒जया᳚ । \newline
26. दे॒वेभ्यः॑ प्र॒जया᳚ प्र॒जया॑ दे॒वेभ्यो॑ दे॒वेभ्यः॑ प्र॒जया॑ पि॒तृभ्यः॑ पि॒तृभ्यः॑ प्र॒जया॑ दे॒वेभ्यो॑ दे॒वेभ्यः॑ प्र॒जया॑ पि॒तृभ्यः॑ । \newline
27. प्र॒जया॑ पि॒तृभ्यः॑ पि॒तृभ्यः॑ प्र॒जया᳚ प्र॒जया॑ पि॒तृभ्य॑ ए॒ष ए॒ष पि॒तृभ्यः॑ प्र॒जया᳚ प्र॒जया॑ पि॒तृभ्य॑ ए॒षः । \newline
28. प्र॒जयेति॑ प्र - जया᳚ । \newline
29. पि॒तृभ्य॑ ए॒ष ए॒ष पि॒तृभ्यः॑ पि॒तृभ्य॑ ए॒ष वै वा ए॒ष पि॒तृभ्यः॑ पि॒तृभ्य॑ ए॒ष वै । \newline
30. पि॒तृभ्य॒ इति॑ पि॒तृ - भ्यः॒ । \newline
31. ए॒ष वै वा ए॒ष ए॒ष वा अ॑नृ॒णो अ॑नृ॒णो वा ए॒ष ए॒ष वा अ॑नृ॒णः । \newline
32. वा अ॑नृ॒णो अ॑नृ॒णो वै वा अ॑नृ॒णो यो यो अ॑नृ॒णो वै वा अ॑नृ॒णो यः । \newline
33. अ॒नृ॒णो यो यो अ॑नृ॒णो अ॑नृ॒णो यः पु॒त्री पु॒त्री यो अ॑नृ॒णो अ॑नृ॒णो यः पु॒त्री । \newline
34. यः पु॒त्री पु॒त्री यो यः पु॒त्री यज्वा॒ यज्वा॑ पु॒त्री यो यः पु॒त्री यज्वा᳚ । \newline
35. पु॒त्री यज्वा॒ यज्वा॑ पु॒त्री पु॒त्री यज्वा᳚ ब्रह्मचारिवा॒सी ब्र॑ह्मचारिवा॒सी यज्वा॑ पु॒त्री पु॒त्री यज्वा᳚ ब्रह्मचारिवा॒सी । \newline
36. यज्वा᳚ ब्रह्मचारिवा॒सी ब्र॑ह्मचारिवा॒सी यज्वा॒ यज्वा᳚ ब्रह्मचारिवा॒सी तत् तद् ब्र॑ह्मचारिवा॒सी यज्वा॒ यज्वा᳚ ब्रह्मचारिवा॒सी तत् । \newline
37. ब्र॒ह्म॒चा॒रि॒वा॒सी तत् तद् ब्र॑ह्मचारिवा॒सी ब्र॑ह्मचारिवा॒सी तद॑व॒दानै॑ रव॒दानै॒ स्तद् ब्र॑ह्मचारिवा॒सी ब्र॑ह्मचारिवा॒सी तद॑व॒दानैः᳚ । \newline
38. ब्र॒ह्म॒चा॒रि॒वा॒सीति॑ ब्रह्मचारि - वा॒सी । \newline
39. तद॑व॒दानै॑ रव॒दानै॒ स्तत् तद॑व॒दानै॑ रे॒वैवा व॒दानै॒ स्तत् तद॑व॒दानै॑ रे॒व । \newline
40. अ॒व॒दानै॑ रे॒वैवा व॒दानै॑ रव॒दानै॑ रे॒वावा वै॒वा व॒दानै॑ रव॒दानै॑ रे॒वाव॑ । \newline
41. अ॒व॒दानै॒रित्य॑व - दानैः᳚ । \newline
42. ए॒वावा वै॒वै वाव॑ दयते दय॒ते ऽवै॒वै वाव॑ दयते । \newline
43. अव॑ दयते दय॒ते ऽवाव॑ दयते॒ तत् तद् द॑य॒ते ऽवाव॑ दयते॒ तत् । \newline
44. द॒य॒ते॒ तत् तद् द॑यते दयते॒ तद॑व॒दाना॑ना मव॒दाना॑ना॒म् तद् द॑यते दयते॒ तद॑व॒दाना॑नाम् । \newline
45. तद॑व॒दाना॑ना मव॒दाना॑ना॒म् तत् तद॑व॒दाना॑ना मवदान॒त्व म॑वदान॒त्व म॑व॒दाना॑ना॒म् तत् तद॑व॒दाना॑ना मवदान॒त्वम् । \newline
46. अ॒व॒दाना॑ना मवदान॒त्व म॑वदान॒त्व म॑व॒दाना॑ना मव॒दाना॑ना मवदान॒त्वम् दे॑वासु॒रा दे॑वासु॒रा अ॑वदान॒त्व म॑व॒दाना॑ना मव॒दाना॑ना मवदान॒त्वम् दे॑वासु॒राः । \newline
47. अ॒व॒दाना॑ना॒मित्य॑व - दाना॑नाम् । \newline
48. अ॒व॒दा॒न॒त्वम् दे॑वासु॒रा दे॑वासु॒रा अ॑वदान॒त्व म॑वदान॒त्वम् दे॑वासु॒राः संॅय॑त्ताः॒ संॅय॑त्ता देवासु॒रा अ॑वदान॒त्व म॑वदान॒त्वम् दे॑वासु॒राः संॅय॑त्ताः । \newline
49. अ॒व॒दा॒न॒त्वमित्य॑वदान - त्वम् । \newline
50. दे॒वा॒सु॒राः संॅय॑त्ताः॒ संॅय॑त्ता देवासु॒रा दे॑वासु॒राः संॅय॑त्ता आसन् नास॒न् थ्संॅय॑त्ता देवासु॒रा दे॑वासु॒राः संॅय॑त्ता आसन्न् । \newline
51. दे॒वा॒सु॒रा इति॑ देव - अ॒सु॒राः । \newline
52. संॅय॑त्ता आसन् नास॒न् थ्संॅय॑त्ताः॒ संॅय॑त्ता आस॒न् ते त आ॑स॒न् थ्संॅय॑त्ताः॒ संॅय॑त्ता आस॒न् ते । \newline
53. संॅय॑त्ता॒ इति॒ सं - य॒त्ताः॒ । \newline
54. आ॒स॒न् ते त आ॑सन् नास॒न् ते दे॒वा दे॒वा स्त आ॑सन् नास॒न् ते दे॒वाः । \newline
55. ते दे॒वा दे॒वा स्ते ते दे॒वा अ॒ग्नि म॒ग्निम् दे॒वा स्ते ते दे॒वा अ॒ग्निम् । \newline
56. दे॒वा अ॒ग्नि म॒ग्निम् दे॒वा दे॒वा अ॒ग्नि म॑ब्रुवन् नब्रुवन् न॒ग्निम् दे॒वा दे॒वा अ॒ग्नि म॑ब्रुवन्न् । \newline
57. अ॒ग्नि म॑ब्रुवन् नब्रुवन् न॒ग्नि म॒ग्नि म॑ब्रुव॒न् त्वया॒ त्वया᳚ ऽब्रुवन् न॒ग्नि म॒ग्नि म॑ब्रुव॒न् त्वया᳚ । \newline
58. अ॒ब्रु॒व॒न् त्वया॒ त्वया᳚ ऽब्रुवन् नब्रुव॒न् त्वया॑ वी॒रेण॑ वी॒रेण॒ त्वया᳚ ऽब्रुवन् नब्रुव॒न् त्वया॑ वी॒रेण॑ । \newline
59. त्वया॑ वी॒रेण॑ वी॒रेण॒ त्वया॒ त्वया॑ वी॒रेणा सु॑रा॒ नसु॑रान्. वी॒रेण॒ त्वया॒ त्वया॑ वी॒रेणा सु॑रान् । \newline
60. वी॒रेणा सु॑रा॒ नसु॑रान्. वी॒रेण॑ वी॒रेणा सु॑रा न॒भ्य॑भ्य सु॑रान्. वी॒रेण॑ वी॒रेणा सु॑रान॒भि । \newline
61. असु॑रा न॒भ्य॑ भ्यसु॑रा॒ नसु॑रा न॒भि भ॑वाम भवामा॒ भ्यसु॑रा॒ नसु॑रा न॒भि भ॑वाम । \newline
62. अ॒भि भ॑वाम भवामा॒ भ्य॑भि भ॑वा॒मे तीति॑ भवामा॒ भ्य॑भि भ॑वा॒मेति॑ । \newline
63. भ॒वा॒मे तीति॑ भवाम भवा॒मेति॒ स स इति॑ भवाम भवा॒मेति॒ सः । \newline
64. इति॒ स स इतीति॒ सो᳚ ऽब्रवी दब्रवी॒थ् स इतीति॒ सो᳚ ऽब्रवीत् । \newline
\pagebreak
\markright{ TS 6.3.10.6  \hfill https://www.vedavms.in \hfill}

\section{ TS 6.3.10.6 }

\textbf{TS 6.3.10.6 } \newline
\textbf{Samhita Paata} \newline

सो᳚ऽब्रवी॒द् वरं॑ ॅवृणै प॒शोरु॑द्धा॒रमुद्ध॑रा॒ इति॒ स ए॒तमु॑द्धा॒रमुद॑हरत॒ दोः पू᳚र्वा॒र्द्धस्य॑ गु॒दं म॑द्ध्य॒तः श्रोणिं॑ जघना॒र्द्धस्य॒ ततो॑ दे॒वा अभ॑व॒न् पराऽसु॑रा॒ यत् त्र्य॒ङ्गाणाꣳ॑ समव॒द्यति॒ भ्रातृ॑व्याभिभूत्यै॒ भव॑त्या॒त्मना॒ परा᳚ऽस्य॒ भ्रातृ॑व्यो भवत्यक्ष्ण॒याऽव॑ द्यति॒ तस्मा॑दक्ष्ण॒या प॒शवोऽङ्गा॑नि॒ प्र ह॑रन्ति॒ प्रति॑ष्ठित्यै ॥ \newline

\textbf{Pada Paata} \newline

सः । अ॒ब्र॒वी॒त् । वर᳚म् । वृ॒णै॒ । प॒शोः । उ॒द्धा॒रमित्यु॑त् - हा॒रम् । उदिति॑ । ह॒रै॒ । इति॑ । सः । ए॒तम् । उ॒द्धा॒रमित्यु॑त् - हा॒रम् । उदिति॑ । अ॒ह॒र॒त॒ । दोः । पू॒र्वा॒द्‌र्धस्येति॑ पूर्व - अ॒द्‌र्धस्य॑ । गु॒दम् । म॒द्ध्य॒तः । श्रोणि᳚म् । ज॒घ॒ना॒द्‌र्धस्येति॑ जघन-अ॒द्‌र्धस्य॑ । ततः॑ । दे॒वाः । अभ॑वन्न् । परेति॑ । असु॑राः । यत् । त्र्य॒ङ्गाणा॒मिति॑ त्रि - अ॒ङ्गाना᳚म् । स॒म॒व॒द्यतीति॑ सं - अ॒व॒द्यति॑ । भ्रातृ॑व्याभिभूत्या॒ इति॒ भ्रातृ॑व्य-अ॒भि॒भू॒त्यै॒ । भव॑ति । आ॒त्मना᳚ । परेति॑ । अ॒स्य॒ । भ्रातृ॑व्यः । भ॒व॒ति॒ । अ॒क्ष्ण॒या । अवेति॑ । द्य॒ति॒ । तस्मा᳚त् । अ॒क्ष्ण॒या । प॒शवः॑ । अङ्गा॑नि । प्रेति॑ । ह॒र॒न्ति॒ । प्रति॑ष्ठित्या॒ इति॒ प्रति॑ - स्थि॒त्यै॒ ॥  \newline


\textbf{Krama Paata} \newline

सो᳚ऽब्रवीत् । अ॒ब्र॒वी॒द् वर᳚म् । वर॑म् ॅवृणै । वृ॒णै॒ प॒शोः । प॒शोरु॑द्धा॒रम् । उ॒द्धा॒रमुत् । उ॒द्धा॒रमित्यु॑त् - हा॒रम् । उद्‍ध॑रै । ह॒रा॒ इति॑ । इति॒ सः । स ए॒तम् । ए॒तमु॑द्‍धा॒रम् । उ॒द्‍धा॒रमुत् । उ॒द्‍धा॒रमित्यु॑त् - हा॒रम् । उद॑हरत । अ॒ह॒र॒त॒ दोः । दोः पू᳚र्वा॒र्द्धस्य॑ । पू॒र्वा॒र्द्धस्य॑ गु॒दम् । पू॒र्वा॒र्द्धस्येति॑ पूर्व - अ॒र्द्धस्य॑ । गु॒दम् म॑द्ध्य॒तः । म॒द्ध्य॒तः श्रोणि᳚म् । श्रोणि॑म् जघना॒र्द्धस्य॑ । ज॒घ॒ना॒र्द्धस्य॒ ततः॑ । ज॒घ॒ना॒र्द्धस्येति॑ जघन - अ॒र्द्धस्य॑ । ततो॑ दे॒वाः । दे॒वा अभ॑वन्न् । अभ॑व॒न् परा᳚ । पराऽसु॑राः । असु॑रा॒ यत् । यत् त्र्य॒ङ्‍गाणा᳚म् । त्र्य॒ङ्‍गाणाꣳ॑ समव॒द्यति॑ । त्र्य॒ङ्‍गाणा॒मिति॑ त्रि - अ॒ङ्‍गाना᳚म् । स॒म॒व॒द्यति॒ भ्रातृ॑व्याभिभूत्यै । स॒म॒व॒द्यतीति॑ सम् - अ॒व॒द्यति॑ । भ्रातृ॑व्याभिभूत्यै॒ भव॑ति । भ्रातृ॑व्याभिभूत्या॒ इति॒ भ्रातृ॑व्य - अ॒भि॒भू॒त्यै॒ । भव॑त्या॒त्मना᳚ । आ॒त्मना॒ परा᳚ । परा᳚ऽस्य । अ॒स्य॒ भ्रातृ॑व्यः । भ्रातृ॑व्यो भवति । भ॒व॒त्य॒क्ष्ण॒या । अ॒क्ष्ण॒याऽव॑ । अव॑ द्यति । द्य॒ति॒ तस्मा᳚त् । तस्मा॑दक्ष्ण॒या । अ॒क्ष्ण॒या प॒शवः॑ । प॒शवोऽङ्‍गा॑नि । अङ्‍गा॑नि॒ प्र । प्र ह॑रन्ति । ह॒र॒न्ति॒ प्रति॑ष्ठित्यै । प्रति॑ष्ठित्या॒ इति॒ प्रति॑ - स्थि॒त्यै॒ । \newline

\textbf{Jatai Paata} \newline

1. सो᳚ ऽब्रवी दब्रवी॒थ् स सो᳚ ऽब्रवीत् । \newline
2. अ॒ब्र॒वी॒द् वरं॒ ॅवर॑ मब्रवी दब्रवी॒द् वर᳚म् । \newline
3. वरं॑ ॅवृणै वृणै॒ वरं॒ ॅवरं॑ ॅवृणै । \newline
4. वृ॒णै॒ प॒शोः प॒शोर् वृ॑णै वृणै प॒शोः । \newline
5. प॒शो रु॑द्धा॒र मु॑द्धा॒रम् प॒शोः प॒शो रु॑द्धा॒रम् । \newline
6. उ॒द्धा॒र मुदु दु॑द्धा॒र मु॑द्धा॒र मुत् । \newline
7. उ॒द्धा॒रमित्यु॑त् - हा॒रम् । \newline
8. उद्ध॑रै हरा॒ उदुद् ध॑रै । \newline
9. ह॒रा॒ इतीति॑ हरै हरा॒ इति॑ । \newline
10. इति॒ स स इतीति॒ सः । \newline
11. स ए॒त मे॒तꣳ स स ए॒तम् । \newline
12. ए॒त मु॑द्धा॒र मु॑द्धा॒र मे॒त मे॒त मु॑द्धा॒रम् । \newline
13. उ॒द्धा॒र मुदु दु॑द्धा॒र मु॑द्धा॒र मुत् । \newline
14. उ॒द्धा॒रमित्यु॑त् - हा॒रम् । \newline
15. उद॑हरता हर॒ तोदुद॑ हरत । \newline
16. अ॒ह॒र॒त॒ दोर् दोर॑हरता हरत॒ दोः । \newline
17. दोः पू᳚र्वा॒र्द्धस्य॑ पूर्वा॒र्द्धस्य॒ दोर् दोः पू᳚र्वा॒र्द्धस्य॑ । \newline
18. पू॒र्वा॒र्द्धस्य॑ गु॒दम् गु॒दम् पू᳚र्वा॒र्द्धस्य॑ पूर्वा॒र्द्धस्य॑ गु॒दम् । \newline
19. पू॒र्वा॒र्द्धस्येति॑ पूर्व - अ॒र्द्धस्य॑ । \newline
20. गु॒दम् म॑द्ध्य॒तो म॑द्ध्य॒तो गु॒दम् गु॒दम् म॑द्ध्य॒तः । \newline
21. म॒द्ध्य॒तः श्रोणिꣳ॒॒ श्रोणि॑म् मद्ध्य॒तो म॑द्ध्य॒तः श्रोणि᳚म् । \newline
22. श्रोणि॑म् जघना॒र्द्धस्य॑ जघना॒र्द्धस्य॒ श्रोणिꣳ॒॒ श्रोणि॑म् जघना॒र्द्धस्य॑ । \newline
23. ज॒घ॒ना॒र्द्धस्य॒ तत॒ स्ततो॑ जघना॒र्द्धस्य॑ जघना॒र्द्धस्य॒ ततः॑ । \newline
24. ज॒घ॒ना॒र्द्धस्येति॑ जघन - अ॒र्द्धस्य॑ । \newline
25. ततो॑ दे॒वा दे॒वा स्तत॒ स्ततो॑ दे॒वाः । \newline
26. दे॒वा अभ॑व॒न् नभ॑वन् दे॒वा दे॒वा अभ॑वन्न् । \newline
27. अभ॑व॒न् परा॒ परा ऽभ॑व॒न् नभ॑व॒न् परा᳚ । \newline
28. परा ऽसु॑रा॒ असु॑राः॒ परा॒ परा ऽसु॑राः । \newline
29. असु॑रा॒ यद् यदसु॑रा॒ असु॑रा॒ यत् । \newline
30. यत् त्र्य॒ङ्गाणा᳚म् त्र्य॒ङ्गाणां॒ ॅयद् यत् त्र्य॒ङ्गाणा᳚म् । \newline
31. त्र्य॒ङ्गाणाꣳ॑ समव॒द्यति॑ समव॒द्यति॑ त्र्य॒ङ्गाणा᳚म् त्र्य॒ङ्गाणाꣳ॑ समव॒द्यति॑ । \newline
32. त्र्य॒ङ्गाणा॒मिति॑ त्रि - अ॒ङ्गाना᳚म् । \newline
33. स॒म॒व॒द्यति॒ भ्रातृ॑व्याभिभूत्यै॒ भ्रातृ॑व्याभिभूत्यै समव॒द्यति॑ समव॒द्यति॒ भ्रातृ॑व्याभिभूत्यै । \newline
34. स॒म॒व॒द्यतीति॑ सं - अ॒व॒द्यति॑ । \newline
35. भ्रातृ॑व्याभिभूत्यै॒ भव॑ति॒ भव॑ति॒ भ्रातृ॑व्याभिभूत्यै॒ भ्रातृ॑व्याभिभूत्यै॒ भव॑ति । \newline
36. भ्रातृ॑व्याभिभूत्या॒ इति॒ भ्रातृ॑व्य - अ॒भि॒भू॒त्यै॒ । \newline
37. भव॑ त्या॒त्मना॒ ऽऽत्मना॒ भव॑ति॒ भव॑ त्या॒त्मना᳚ । \newline
38. आ॒त्मना॒ परा॒ परा॒ ऽऽत्मना॒ ऽऽत्मना॒ परा᳚ । \newline
39. परा᳚ ऽस्यास्य॒ परा॒ परा᳚ ऽस्य । \newline
40. अ॒स्य॒ भ्रातृ॑व्यो॒ भ्रातृ॑व्यो ऽस्यास्य॒ भ्रातृ॑व्यः । \newline
41. भ्रातृ॑व्यो भवति भवति॒ भ्रातृ॑व्यो॒ भ्रातृ॑व्यो भवति । \newline
42. भ॒व॒ त्य॒क्ष्ण॒या ऽक्ष्ण॒या भ॑वति भव त्यक्ष्ण॒या । \newline
43. अ॒क्ष्ण॒या ऽवावा᳚ क्ष्ण॒या ऽक्ष्ण॒या ऽव॑ । \newline
44. अव॑ द्यति द्य॒त्यवाव॑ द्यति । \newline
45. द्य॒ति॒ तस्मा॒त् तस्मा᳚द् द्यति द्यति॒ तस्मा᳚त् । \newline
46. तस्मा॑ दक्ष्ण॒या ऽक्ष्ण॒या तस्मा॒त् तस्मा॑ दक्ष्ण॒या । \newline
47. अ॒क्ष्ण॒या प॒शवः॑ प॒शवो᳚ ऽक्ष्ण॒या ऽक्ष्ण॒या प॒शवः॑ । \newline
48. प॒शवो ऽङ्गा॒ न्यङ्गा॑नि प॒शवः॑ प॒शवो ऽङ्गा॑नि । \newline
49. अङ्गा॑नि॒ प्र प्राङ्गा॒ न्यङ्गा॑नि॒ प्र । \newline
50. प्र ह॑रन्ति हरन्ति॒ प्र प्र ह॑रन्ति । \newline
51. ह॒र॒न्ति॒ प्रति॑ष्ठित्यै॒ प्रति॑ष्ठित्यै हरन्ति हरन्ति॒ प्रति॑ष्ठित्यै । \newline
52. प्रति॑ष्ठित्या॒ इति॒ प्रति॑ - स्थि॒त्यै॒ । \newline

\textbf{Ghana Paata } \newline

1. सो᳚ ऽब्रवी दब्रवी॒थ् स सो᳚ ऽब्रवी॒द् वरं॒ ॅवर॑ मब्रवी॒थ् स सो᳚ ऽब्रवी॒द् वर᳚म् । \newline
2. अ॒ब्र॒वी॒द् वरं॒ ॅवर॑ मब्रवी दब्रवी॒द् वरं॑ ॅवृणै वृणै॒ वर॑ मब्रवी दब्रवी॒द् वरं॑ ॅवृणै । \newline
3. वरं॑ ॅवृणै वृणै॒ वरं॒ ॅवरं॑ ॅवृणै प॒शोः प॒शोर् वृ॑णै॒ वरं॒ ॅवरं॑ ॅवृणै प॒शोः । \newline
4. वृ॒णै॒ प॒शोः प॒शोर् वृ॑णै वृणै प॒शो रु॑द्धा॒र मु॑द्धा॒रम् प॒शोर् वृ॑णै वृणै प॒शो रु॑द्धा॒रम् । \newline
5. प॒शो रु॑द्धा॒र मु॑द्धा॒रम् प॒शोः प॒शो रु॑द्धा॒र मुदु दु॑द्धा॒रम् प॒शोः प॒शो रु॑द्धा॒र मुत् । \newline
6. उ॒द्धा॒र मुदु दु॑द्धा॒र मु॑द्धा॒र मुद्ध॑रै हरा॒ उदु॑द्धा॒र मु॑द्धा॒र मुद्ध॑रै । \newline
7. उ॒द्धा॒रमित्यु॑त् - हा॒रम् । \newline
8. उद्ध॑रै हरा॒ उदुद्ध॑रा॒ इतीति॑ हरा॒ उदुद्ध॑रा॒ इति॑ । \newline
9. ह॒रा॒ इतीति॑ हरै हरा॒ इति॒ स स इति॑ हरै हरा॒ इति॒ सः । \newline
10. इति॒ स स इतीति॒ स ए॒त मे॒तꣳ स इतीति॒ स ए॒तम् । \newline
11. स ए॒त मे॒तꣳ स स ए॒त मु॑द्धा॒र मु॑द्धा॒र मे॒तꣳ स स ए॒त मु॑द्धा॒रम् । \newline
12. ए॒त मु॑द्धा॒र मु॑द्धा॒र मे॒त मे॒त मु॑द्धा॒र मुदु दु॑द्धा॒र मे॒त मे॒त मु॑द्धा॒र मुत् । \newline
13. उ॒द्धा॒र मुदु दु॑द्धा॒र मु॑द्धा॒र मुद॑हरता हर॒तो दु॑द्धा॒र मु॑द्धा॒र मुद॑हरत । \newline
14. उ॒द्धा॒रमित्यु॑त् - हा॒रम् । \newline
15. उद॑हरता हर॒तो दुद॑हरत॒ दोर् दो र॑हर॒ तोदु द॑हरत॒ दोः । \newline
16. अ॒ह॒र॒त॒ दोर् दो र॑हरता हरत॒ दोः पू᳚र्वा॒र्द्धस्य॑ पूर्वा॒र्द्धस्य॒ दो र॑हरता हरत॒ दोः पू᳚र्वा॒र्द्धस्य॑ । \newline
17. दोः पू᳚र्वा॒र्द्धस्य॑ पूर्वा॒र्द्धस्य॒ दोर् दोः पू᳚र्वा॒र्द्धस्य॑ गु॒दम् गु॒दम् पू᳚र्वा॒र्द्धस्य॒ दोर् दोः पू᳚र्वा॒र्द्धस्य॑ गु॒दम् । \newline
18. पू॒र्वा॒र्द्धस्य॑ गु॒दम् गु॒दम् पू᳚र्वा॒र्द्धस्य॑ पूर्वा॒र्द्धस्य॑ गु॒दम् म॑द्ध्य॒तो म॑द्ध्य॒तो गु॒दम् पू᳚र्वा॒र्द्धस्य॑ पूर्वा॒र्द्धस्य॑ गु॒दम् म॑द्ध्य॒तः । \newline
19. पू॒र्वा॒र्द्धस्येति॑ पूर्व - अ॒र्द्धस्य॑ । \newline
20. गु॒दम् म॑द्ध्य॒तो म॑द्ध्य॒तो गु॒दम् गु॒दम् म॑द्ध्य॒तः श्रोणिꣳ॒॒ श्रोणि॑म् मद्ध्य॒तो गु॒दम् गु॒दम् म॑द्ध्य॒तः श्रोणि᳚म् । \newline
21. म॒द्ध्य॒तः श्रोणिꣳ॒॒ श्रोणि॑म् मद्ध्य॒तो म॑द्ध्य॒तः श्रोणि॑म् जघना॒र्द्धस्य॑ जघना॒र्द्धस्य॒ श्रोणि॑म् मद्ध्य॒तो म॑द्ध्य॒तः श्रोणि॑म् जघना॒र्द्धस्य॑ । \newline
22. श्रोणि॑म् जघना॒र्द्धस्य॑ जघना॒र्द्धस्य॒ श्रोणिꣳ॒॒ श्रोणि॑म् जघना॒र्द्धस्य॒ तत॒ स्ततो॑ जघना॒र्द्धस्य॒ श्रोणिꣳ॒॒ श्रोणि॑म् जघना॒र्द्धस्य॒ ततः॑ । \newline
23. ज॒घ॒ना॒र्द्धस्य॒ तत॒ स्ततो॑ जघना॒र्द्धस्य॑ जघना॒र्द्धस्य॒ ततो॑ दे॒वा दे॒वा स्ततो॑ जघना॒र्द्धस्य॑ जघना॒र्द्धस्य॒ ततो॑ दे॒वाः । \newline
24. ज॒घ॒ना॒र्द्धस्येति॑ जघन - अ॒र्द्धस्य॑ । \newline
25. ततो॑ दे॒वा दे॒वा स्तत॒ स्ततो॑ दे॒वा अभ॑व॒न् नभ॑वन् दे॒वा स्तत॒ स्ततो॑ दे॒वा अभ॑वन्न् । \newline
26. दे॒वा अभ॑व॒न् नभ॑वन् दे॒वा दे॒वा अभ॑व॒न् परा॒ परा ऽभ॑वन् दे॒वा दे॒वा अभ॑व॒न् परा᳚ । \newline
27. अभ॑व॒न् परा॒ परा ऽभ॑व॒न् नभ॑व॒न् परा ऽसु॑रा॒ असु॑राः॒ परा ऽभ॑व॒न् नभ॑व॒न् परा ऽसु॑राः । \newline
28. परा ऽसु॑रा॒ असु॑राः॒ परा॒ परा ऽसु॑रा॒ यद् यदसु॑राः॒ परा॒ परा ऽसु॑रा॒ यत् । \newline
29. असु॑रा॒ यद् यदसु॑रा॒ असु॑रा॒ यत् त्र्य॒ङ्गाणा᳚म् त्र्य॒ङ्गाणां॒ ॅयदसु॑रा॒ असु॑रा॒ यत् त्र्य॒ङ्गाणा᳚म् । \newline
30. यत् त्र्य॒ङ्गाणा᳚म् त्र्य॒ङ्गाणां॒ ॅयद् यत् त्र्य॒ङ्गाणाꣳ॑ समव॒द्यति॑ समव॒द्यति॑ त्र्य॒ङ्गाणां॒ ॅयद् यत् त्र्य॒ङ्गाणाꣳ॑ समव॒द्यति॑ । \newline
31. त्र्य॒ङ्गाणाꣳ॑ समव॒द्यति॑ समव॒द्यति॑ त्र्य॒ङ्गाणा᳚म् त्र्य॒ङ्गाणाꣳ॑ समव॒द्यति॒ भ्रातृ॑व्याभिभूत्यै॒ भ्रातृ॑व्याभिभूत्यै समव॒द्यति॑ त्र्य॒ङ्गाणा᳚म् त्र्य॒ङ्गाणाꣳ॑ समव॒द्यति॒ भ्रातृ॑व्याभिभूत्यै । \newline
32. त्र्य॒ङ्गाणा॒मिति॑ त्रि - अ॒ङ्गाना᳚म् । \newline
33. स॒म॒व॒द्यति॒ भ्रातृ॑व्याभिभूत्यै॒ भ्रातृ॑व्याभिभूत्यै समव॒द्यति॑ समव॒द्यति॒ भ्रातृ॑व्याभिभूत्यै॒ भव॑ति॒ भव॑ति॒ भ्रातृ॑व्याभिभूत्यै समव॒द्यति॑ समव॒द्यति॒ भ्रातृ॑व्याभिभूत्यै॒ भव॑ति । \newline
34. स॒म॒व॒द्यतीति॑ सं - अ॒व॒द्यति॑ । \newline
35. भ्रातृ॑व्याभिभूत्यै॒ भव॑ति॒ भव॑ति॒ भ्रातृ॑व्याभिभूत्यै॒ भ्रातृ॑व्याभिभूत्यै॒ भव॑ त्या॒त्मना॒ ऽऽत्मना॒ भव॑ति॒ भ्रातृ॑व्याभिभूत्यै॒ भ्रातृ॑व्याभिभूत्यै॒ भव॑ त्या॒त्मना᳚ । \newline
36. भ्रातृ॑व्याभिभूत्या॒ इति॒ भ्रातृ॑व्य - अ॒भि॒भू॒त्यै॒ । \newline
37. भव॑ त्या॒त्मना॒ ऽऽत्मना॒ भव॑ति॒ भव॑ त्या॒त्मना॒ परा॒ परा॒ ऽऽत्मना॒ भव॑ति॒ भव॑ त्या॒त्मना॒ परा᳚ । \newline
38. आ॒त्मना॒ परा॒ परा॒ ऽऽत्मना॒ ऽऽत्मना॒ परा᳚ ऽस्यास्य॒ परा॒ ऽऽत्मना॒ ऽऽत्मना॒ परा᳚ ऽस्य । \newline
39. परा᳚ ऽस्यास्य॒ परा॒ परा᳚ ऽस्य॒ भ्रातृ॑व्यो॒ भ्रातृ॑व्यो ऽस्य॒ परा॒ परा᳚ ऽस्य॒ भ्रातृ॑व्यः । \newline
40. अ॒स्य॒ भ्रातृ॑व्यो॒ भ्रातृ॑व्यो ऽस्यास्य॒ भ्रातृ॑व्यो भवति भवति॒ भ्रातृ॑व्यो ऽस्यास्य॒ भ्रातृ॑व्यो भवति । \newline
41. भ्रातृ॑व्यो भवति भवति॒ भ्रातृ॑व्यो॒ भ्रातृ॑व्यो भव त्यक्ष्ण॒या ऽक्ष्ण॒या भ॑वति॒ भ्रातृ॑व्यो॒ भ्रातृ॑व्यो भव त्यक्ष्ण॒या । \newline
42. भ॒व॒ त्य॒क्ष्ण॒या ऽक्ष्ण॒या भ॑वति भव त्यक्ष्ण॒या ऽवावा᳚ क्ष्ण॒या भ॑वति भव त्यक्ष्ण॒या ऽव॑ । \newline
43. अ॒क्ष्ण॒या ऽवावा᳚ क्ष्ण॒या ऽक्ष्ण॒या ऽव॑ द्यति द्य॒त्यवा᳚ क्ष्ण॒या ऽक्ष्ण॒या ऽव॑ द्यति । \newline
44. अव॑ द्यति द्य॒त्य वाव॑ द्यति॒ तस्मा॒त् तस्मा᳚द् द्य॒त्य वाव॑ द्यति॒ तस्मा᳚त् । \newline
45. द्य॒ति॒ तस्मा॒त् तस्मा᳚द् द्यति द्यति॒ तस्मा॑ दक्ष्ण॒या ऽक्ष्ण॒या तस्मा᳚द् द्यति द्यति॒ तस्मा॑ दक्ष्ण॒या । \newline
46. तस्मा॑ दक्ष्ण॒या ऽक्ष्ण॒या तस्मा॒त् तस्मा॑ दक्ष्ण॒या प॒शवः॑ प॒शवो᳚ ऽक्ष्ण॒या तस्मा॒त् तस्मा॑ दक्ष्ण॒या प॒शवः॑ । \newline
47. अ॒क्ष्ण॒या प॒शवः॑ प॒शवो᳚ ऽक्ष्ण॒या ऽक्ष्ण॒या प॒शवो ऽङ्गा॒ न्यङ्गा॑नि प॒शवो᳚ ऽक्ष्ण॒या ऽक्ष्ण॒या प॒शवो ऽङ्गा॑नि । \newline
48. प॒शवो ऽङ्गा॒ न्यङ्गा॑नि प॒शवः॑ प॒शवो ऽङ्गा॑नि॒ प्र प्राङ्गा॑नि प॒शवः॑ प॒शवो ऽङ्गा॑नि॒ प्र । \newline
49. अङ्गा॑नि॒ प्र प्राङ्गा॒ न्यङ्गा॑नि॒ प्र ह॑रन्ति हरन्ति॒ प्राङ्गा॒ न्यङ्गा॑नि॒ प्र ह॑रन्ति । \newline
50. प्र ह॑रन्ति हरन्ति॒ प्र प्र ह॑रन्ति॒ प्रति॑ष्ठित्यै॒ प्रति॑ष्ठित्यै हरन्ति॒ प्र प्र ह॑रन्ति॒ प्रति॑ष्ठित्यै । \newline
51. ह॒र॒न्ति॒ प्रति॑ष्ठित्यै॒ प्रति॑ष्ठित्यै हरन्ति हरन्ति॒ प्रति॑ष्ठित्यै । \newline
52. प्रति॑ष्ठित्या॒ इति॒ प्रति॑ - स्थि॒त्यै॒ । \newline
\pagebreak
\markright{ TS 6.3.11.1  \hfill https://www.vedavms.in \hfill}

\section{ TS 6.3.11.1 }

\textbf{TS 6.3.11.1 } \newline
\textbf{Samhita Paata} \newline

मेद॑सा॒ स्रुचौ॒ प्रोर्णो॑ति॒ मेदो॑रूपा॒ वै प॒शवो॑ रू॒पमे॒व प॒शुषु॑ दधाति यू॒षन्न॑व॒धाय॒ प्रोर्णो॑ति॒ रसो॒ वा ए॒ष प॑शू॒नां ॅयद्यू रस॑मे॒व प॒शुषु॑ दधाति पा॒र्श्वेन॑ वसाहो॒मं प्रयौ॑ति॒ मद्ध्यं॒ ॅवा ए॒तत् प॑शू॒नां ॅयत् पा॒र्श्वꣳ रस॑ ए॒ष प॑शू॒नां ॅयद्वसा॒ यत् पा॒र्श्वेन॑ वसाहो॒मं प्र॒यौति॑ मद्ध्य॒त ए॒व प॑शू॒नाꣳ रसं॑ दधाति॒ घ्नन्ति॒- [  ] \newline

\textbf{Pada Paata} \newline

मेद॑सा । स्रुचौ᳚ । प्रेति॑ । ऊ॒र्णो॒ति॒ । मेदो॑रूपा॒ इति॒ मेदः॑ - रू॒पाः॒ । वै । प॒शवः॑ । रू॒पम् । ए॒व । प॒शुषु॑ । द॒धा॒ति॒ । यू॒षन्न् । अ॒व॒धायेत्य॑व-धाय॑ । प्रेति॑ । ऊ॒र्णो॒ति॒ । रसः॑ । वै । ए॒षः । प॒शू॒नाम् । यत् । यूः । रस᳚म् । ए॒व । प॒शुषु॑ । द॒धा॒ति॒ । पा॒र्श्वेन॑ । व॒सा॒हो॒ममिति॑ वसा - हो॒मम् । प्रेति॑ । यौ॒ति॒ । मद्ध्य᳚म् । वै । ए॒तत् । प॒शू॒नाम् । यत् । पा॒र्श्वम् । रसः॑ । ए॒षः । प॒शू॒नाम् । यत् । वसा᳚ । यत् । पा॒र्श्वेन॑ । व॒सा॒हो॒ममिति॑ वसा - हो॒मम् । प्र॒यौतीति॑ प्र - यौति॑ । म॒द्ध्य॒तः । ए॒व । प॒शू॒नाम् । रस᳚म् । द॒धा॒ति॒ । घ्नन्ति॑ ।  \newline


\textbf{Krama Paata} \newline

मेद॑सा॒ स्रुचौ᳚ । स्रुचौ॒ प्र । प्रोर्णो॑ति । ऊ॒र्णो॒ति॒ मेदो॑रूपाः । मेदो॑रूपा॒ वै । मेदो॑रूपा॒ इति॒ मेदः॑ - रू॒पाः॒ । वै प॒शवः॑ । प॒शवो॑ रू॒पम् । रू॒पमे॒व । ए॒व प॒शुषु॑ । प॒शुषु॑ दधाति । द॒धा॒ति॒ यू॒षन्न् । यू॒षन्न॑व॒धाय॑ । अ॒व॒धाय॒ प्र । अ॒व॒धायेत्य॑व - धाय॑ । प्रोर्णो॑ति । ऊ॒र्णो॒ति॒ रसः॑ । रसो॒ वै । वा ए॒षः । ए॒ष प॑शू॒नाम् । प॒शू॒नाम् ॅयत् । यद् यूः । यू रस᳚म् । रस॑मे॒व । ए॒व प॒शुषु॑ । प॒शुषु॑ दधाति । द॒धा॒ति॒ पा॒र्श्वेन॑ । पा॒र्श्वेन॑ वसाहो॒मम् । व॒सा॒हो॒मम् प्र । व॒सा॒हो॒ममिति॑ वसा - हो॒मम् । प्र यौ॑ति । यौ॒ति॒ मद्ध्य᳚म् । मद्ध्य॒म् ॅवै । वा ए॒तत् । ए॒तत् प॑शू॒नाम् । प॒शू॒नाम् ॅयत् । यत् पा॒र्श्वम् । पा॒र्श्वꣳ रसः॑ । रस॑ ए॒षः । ए॒ष प॑शू॒नाम् । प॒शू॒नाम् ॅयत् । यद् वसा᳚ । वसा॒ यत् । यत् पा॒र्श्वेन॑ । पा॒र्श्वेन॑ वसाहो॒मम् । व॒सा॒हो॒मम् प्र॒यौति॑ । व॒सा॒हो॒ममिति॑ वसा - हो॒मम् । प्र॒यौति॑ मद्ध्य॒तः । प्र॒यौतीति॑ प्र - यौति॑ । म॒द्ध्य॒त ए॒व । ए॒व प॑शू॒नाम् । प॒शू॒नाꣳ रस᳚म् । रस॑म् दधाति । द॒धा॒ति॒ घ्नन्ति॑ । घ्नन्ति॒ वै \newline

\textbf{Jatai Paata} \newline

1. मेद॑सा॒ स्रुचौ॒ स्रुचौ॒ मेद॑सा॒ मेद॑सा॒ स्रुचौ᳚ । \newline
2. स्रुचौ॒ प्र प्र स्रुचौ॒ स्रुचौ॒ प्र । \newline
3. प्रोर्णो᳚ त्यूर्णोति॒ प्र प्रोर्णो॑ति । \newline
4. ऊ॒र्णो॒ति॒ मेदो॑रूपा॒ मेदो॑रूपा ऊर्णो त्यूर्णोति॒ मेदो॑रूपाः । \newline
5. मेदो॑रूपा॒ वै वै मेदो॑रूपा॒ मेदो॑रूपा॒ वै । \newline
6. मेदो॑रूपा॒ इति॒ मेदः॑ - रू॒पाः॒ । \newline
7. वै प॒शवः॑ प॒शवो॒ वै वै प॒शवः॑ । \newline
8. प॒शवो॑ रू॒पꣳ रू॒पम् प॒शवः॑ प॒शवो॑ रू॒पम् । \newline
9. रू॒प मे॒वैव रू॒पꣳ रू॒प मे॒व । \newline
10. ए॒व प॒शुषु॑ प॒शु ष्वे॒वैव प॒शुषु॑ । \newline
11. प॒शुषु॑ दधाति दधाति प॒शुषु॑ प॒शुषु॑ दधाति । \newline
12. द॒धा॒ति॒ यू॒षन्. यू॒षन् द॑धाति दधाति यू॒षन्न् । \newline
13. यू॒षन् न॑व॒धाया॑ व॒धाय॑ यू॒षन्. यू॒षन् न॑व॒धाय॑ । \newline
14. अ॒व॒धाय॒ प्र प्राव॒धाया॑ व॒धाय॒ प्र । \newline
15. अ॒व॒धायेत्य॑व - धाय॑ । \newline
16. प्रोर्णो᳚ त्यूर्णोति॒ प्र प्रोर्णो॑ति । \newline
17. ऊ॒र्णो॒ति॒ रसो॒ रस॑ ऊर्णो त्यूर्णोति॒ रसः॑ । \newline
18. रसो॒ वै वै रसो॒ रसो॒ वै । \newline
19. वा ए॒ष ए॒ष वै वा ए॒षः । \newline
20. ए॒ष प॑शू॒नाम् प॑शू॒ना मे॒ष ए॒ष प॑शू॒नाम् । \newline
21. प॒शू॒नां ॅयद् यत् प॑शू॒नाम् प॑शू॒नां ॅयत् । \newline
22. यद् यूर् यूर् यद् यद् यूः । \newline
23. यू रसꣳ॒॒ रसं॒ ॅयूर् यू रस᳚म् । \newline
24. रस॑ मे॒वैव रसꣳ॒॒ रस॑ मे॒व । \newline
25. ए॒व प॒शुषु॑ प॒शु ष्वे॒वैव प॒शुषु॑ । \newline
26. प॒शुषु॑ दधाति दधाति प॒शुषु॑ प॒शुषु॑ दधाति । \newline
27. द॒धा॒ति॒ पा॒र्श्वेन॑ पा॒र्श्वेन॑ दधाति दधाति पा॒र्श्वेन॑ । \newline
28. पा॒र्श्वेन॑ वसाहो॒मं ॅव॑साहो॒मम् पा॒र्श्वेन॑ पा॒र्श्वेन॑ वसाहो॒मम् । \newline
29. व॒सा॒हो॒मम् प्र प्र व॑साहो॒मं ॅव॑साहो॒मम् प्र । \newline
30. व॒सा॒हो॒ममिति॑ वसा - हो॒मम् । \newline
31. प्र यौ॑ति यौति॒ प्र प्र यौ॑ति । \newline
32. यौ॒ति॒ मद्ध्य॒म् मद्ध्यं॑ ॅयौति यौति॒ मद्ध्य᳚म् । \newline
33. मद्ध्यं॒ ॅवै वै मद्ध्य॒म् मद्ध्यं॒ ॅवै । \newline
34. वा ए॒त दे॒तद् वै वा ए॒तत् । \newline
35. ए॒तत् प॑शू॒नाम् प॑शू॒ना मे॒त दे॒तत् प॑शू॒नाम् । \newline
36. प॒शू॒नां ॅयद् यत् प॑शू॒नाम् प॑शू॒नां ॅयत् । \newline
37. यत् पा॒र्श्वम् पा॒र्श्वं ॅयद् यत् पा॒र्श्वम् । \newline
38. पा॒र्श्वꣳ रसो॒ रसः॑ पा॒र्श्वम् पा॒र्श्वꣳ रसः॑ । \newline
39. रस॑ ए॒ष ए॒ष रसो॒ रस॑ ए॒षः । \newline
40. ए॒ष प॑शू॒नाम् प॑शू॒ना मे॒ष ए॒ष प॑शू॒नाम् । \newline
41. प॒शू॒नां ॅयद् यत् प॑शू॒नाम् प॑शू॒नां ॅयत् । \newline
42. यद् वसा॒ वसा॒ यद् यद् वसा᳚ । \newline
43. वसा॒ यद् यद् वसा॒ वसा॒ यत् । \newline
44. यत् पा॒र्श्वेन॑ पा॒र्श्वेन॒ यद् यत् पा॒र्श्वेन॑ । \newline
45. पा॒र्श्वेन॑ वसाहो॒मं ॅव॑साहो॒मम् पा॒र्श्वेन॑ पा॒र्श्वेन॑ वसाहो॒मम् । \newline
46. व॒सा॒हो॒मम् प्र॒यौति॑ प्र॒यौति॑ वसाहो॒मं ॅव॑साहो॒मम् प्र॒यौति॑ । \newline
47. व॒सा॒हो॒ममिति॑ वसा - हो॒मम् । \newline
48. प्र॒यौति॑ मद्ध्य॒तो म॑द्ध्य॒तः प्र॒यौति॑ प्र॒यौति॑ मद्ध्य॒तः । \newline
49. प्र॒यौतीति॑ प्र - यौति॑ । \newline
50. म॒द्ध्य॒त ए॒वैव म॑द्ध्य॒तो म॑द्ध्य॒त ए॒व । \newline
51. ए॒व प॑शू॒नाम् प॑शू॒ना मे॒वैव प॑शू॒नाम् । \newline
52. प॒शू॒नाꣳ रसꣳ॒॒ रस॑म् पशू॒नाम् प॑शू॒नाꣳ रस᳚म् । \newline
53. रस॑म् दधाति दधाति॒ रसꣳ॒॒ रस॑म् दधाति । \newline
54. द॒धा॒ति॒ घ्नन्ति॒ घ्नन्ति॑ दधाति दधाति॒ घ्नन्ति॑ । \newline
55. घ्नन्ति॒ वै वै घ्नन्ति॒ घ्नन्ति॒ वै । \newline

\textbf{Ghana Paata } \newline

1. मेद॑सा॒ स्रुचौ॒ स्रुचौ॒ मेद॑सा॒ मेद॑सा॒ स्रुचौ॒ प्र प्र स्रुचौ॒ मेद॑सा॒ मेद॑सा॒ स्रुचौ॒ प्र । \newline
2. स्रुचौ॒ प्र प्र स्रुचौ॒ स्रुचौ॒ प्रोर्णो᳚ त्यूर्णोति॒ प्र स्रुचौ॒ स्रुचौ॒ प्रोर्णो॑ति । \newline
3. प्रोर्णो᳚ त्यूर्णोति॒ प्र प्रोर्णो॑ति॒ मेदो॑रूपा॒ मेदो॑रूपा ऊर्णोति॒ प्र प्रोर्णो॑ति॒ मेदो॑रूपाः । \newline
4. ऊ॒र्णो॒ति॒ मेदो॑रूपा॒ मेदो॑रूपा ऊर्णो त्यूर्णोति॒ मेदो॑रूपा॒ वै वै मेदो॑रूपा ऊर्णो त्यूर्णोति॒ मेदो॑रूपा॒ वै । \newline
5. मेदो॑रूपा॒ वै वै मेदो॑रूपा॒ मेदो॑रूपा॒ वै प॒शवः॑ प॒शवो॒ वै मेदो॑रूपा॒ मेदो॑रूपा॒ वै प॒शवः॑ । \newline
6. मेदो॑रूपा॒ इति॒ मेदः॑ - रू॒पाः॒ । \newline
7. वै प॒शवः॑ प॒शवो॒ वै वै प॒शवो॑ रू॒पꣳ रू॒पम् प॒शवो॒ वै वै प॒शवो॑ रू॒पम् । \newline
8. प॒शवो॑ रू॒पꣳ रू॒पम् प॒शवः॑ प॒शवो॑ रू॒प मे॒वैव रू॒पम् प॒शवः॑ प॒शवो॑ रू॒प मे॒व । \newline
9. रू॒प मे॒वैव रू॒पꣳ रू॒प मे॒व प॒शुषु॑ प॒शुष्वे॒व रू॒पꣳ रू॒प मे॒व प॒शुषु॑ । \newline
10. ए॒व प॒शुषु॑ प॒शुष्वे॒ वैव प॒शुषु॑ दधाति दधाति प॒शुष्वे॒ वैव प॒शुषु॑ दधाति । \newline
11. प॒शुषु॑ दधाति दधाति प॒शुषु॑ प॒शुषु॑ दधाति यू॒षन्. यू॒षन् द॑धाति प॒शुषु॑ प॒शुषु॑ दधाति यू॒षन्न् । \newline
12. द॒धा॒ति॒ यू॒षन्. यू॒षन् द॑धाति दधाति यू॒षन् न॑व॒धाया॑ व॒धाय॑ यू॒षन् द॑धाति दधाति यू॒षन् न॑व॒धाय॑ । \newline
13. यू॒षन् न॑व॒धाया॑ व॒धाय॑ यू॒षन्. यू॒षन् न॑व॒धाय॒ प्र प्राव॒धाय॑ यू॒षन्. यू॒षन् न॑व॒धाय॒ प्र । \newline
14. अ॒व॒धाय॒ प्र प्राव॒धाया॑ व॒धाय॒ प्रोर्णो᳚ त्यूर्णोति॒ प्राव॒धाया॑ व॒धाय॒ प्रोर्णो॑ति । \newline
15. अ॒व॒धायेत्य॑व - धाय॑ । \newline
16. प्रोर्णो᳚ त्यूर्णोति॒ प्र प्रोर्णो॑ति॒ रसो॒ रस॑ ऊर्णोति॒ प्र प्रोर्णो॑ति॒ रसः॑ । \newline
17. ऊ॒र्णो॒ति॒ रसो॒ रस॑ ऊर्णो त्यूर्णोति॒ रसो॒ वै वै रस॑ ऊर्णो त्यूर्णोति॒ रसो॒ वै । \newline
18. रसो॒ वै वै रसो॒ रसो॒ वा ए॒ष ए॒ष वै रसो॒ रसो॒ वा ए॒षः । \newline
19. वा ए॒ष ए॒ष वै वा ए॒ष प॑शू॒नाम् प॑शू॒ना मे॒ष वै वा ए॒ष प॑शू॒नाम् । \newline
20. ए॒ष प॑शू॒नाम् प॑शू॒ना मे॒ष ए॒ष प॑शू॒नां ॅयद् यत् प॑शू॒ना मे॒ष ए॒ष प॑शू॒नां ॅयत् । \newline
21. प॒शू॒नां ॅयद् यत् प॑शू॒नाम् प॑शू॒नां ॅयद् यूर् यूर् यत् प॑शू॒नाम् प॑शू॒नां ॅयद् यूः । \newline
22. यद् यूर् यूर् यद् यद् यू रसꣳ॒॒ रसं॒ ॅयूर् यद् यद् यू रस᳚म् । \newline
23. यू रसꣳ॒॒ रसं॒ ॅयूर् यू रस॑ मे॒वैव रसं॒ ॅयूर् यू रस॑ मे॒व । \newline
24. रस॑ मे॒वैव रसꣳ॒॒ रस॑ मे॒व प॒शुषु॑ प॒शुष्वे॒व रसꣳ॒॒ रस॑ मे॒व प॒शुषु॑ । \newline
25. ए॒व प॒शुषु॑ प॒शुष्वे॒ वैव प॒शुषु॑ दधाति दधाति प॒शुष्वे॒ वैव प॒शुषु॑ दधाति । \newline
26. प॒शुषु॑ दधाति दधाति प॒शुषु॑ प॒शुषु॑ दधाति पा॒र्श्वेन॑ पा॒र्श्वेन॑ दधाति प॒शुषु॑ प॒शुषु॑ दधाति पा॒र्श्वेन॑ । \newline
27. द॒धा॒ति॒ पा॒र्श्वेन॑ पा॒र्श्वेन॑ दधाति दधाति पा॒र्श्वेन॑ वसाहो॒मं ॅव॑साहो॒मम् पा॒र्श्वेन॑ दधाति दधाति पा॒र्श्वेन॑ वसाहो॒मम् । \newline
28. पा॒र्श्वेन॑ वसाहो॒मं ॅव॑साहो॒मम् पा॒र्श्वेन॑ पा॒र्श्वेन॑ वसाहो॒मम् प्र प्र व॑साहो॒मम् पा॒र्श्वेन॑ पा॒र्श्वेन॑ वसाहो॒मम् प्र । \newline
29. व॒सा॒हो॒मम् प्र प्र व॑साहो॒मं ॅव॑साहो॒मम् प्र यौ॑ति यौति॒ प्र व॑साहो॒मं ॅव॑साहो॒मम् प्र यौ॑ति । \newline
30. व॒सा॒हो॒ममिति॑ वसा - हो॒मम् । \newline
31. प्र यौ॑ति यौति॒ प्र प्र यौ॑ति॒ मद्ध्य॒म् मद्ध्यं॑ ॅयौति॒ प्र प्र यौ॑ति॒ मद्ध्य᳚म् । \newline
32. यौ॒ति॒ मद्ध्य॒म् मद्ध्यं॑ ॅयौति यौति॒ मद्ध्यं॒ ॅवै वै मद्ध्यं॑ ॅयौति यौति॒ मद्ध्यं॒ ॅवै । \newline
33. मद्ध्यं॒ ॅवै वै मद्ध्य॒म् मद्ध्यं॒ ॅवा ए॒त दे॒तद् वै मद्ध्य॒म् मद्ध्यं॒ ॅवा ए॒तत् । \newline
34. वा ए॒त दे॒तद् वै वा ए॒तत् प॑शू॒नाम् प॑शू॒ना मे॒तद् वै वा ए॒तत् प॑शू॒नाम् । \newline
35. ए॒तत् प॑शू॒नाम् प॑शू॒ना मे॒त दे॒तत् प॑शू॒नां ॅयद् यत् प॑शू॒ना मे॒त दे॒तत् प॑शू॒नां ॅयत् । \newline
36. प॒शू॒नां ॅयद् यत् प॑शू॒नाम् प॑शू॒नां ॅयत् पा॒र्श्वम् पा॒र्श्वं ॅयत् प॑शू॒नाम् प॑शू॒नां ॅयत् पा॒र्श्वम् । \newline
37. यत् पा॒र्श्वम् पा॒र्श्वं ॅयद् यत् पा॒र्श्वꣳ रसो॒ रसः॑ पा॒र्श्वं ॅयद् यत् पा॒र्श्वꣳ रसः॑ । \newline
38. पा॒र्श्वꣳ रसो॒ रसः॑ पा॒र्श्वम् पा॒र्श्वꣳ रस॑ ए॒ष ए॒ष रसः॑ पा॒र्श्वम् पा॒र्श्वꣳ रस॑ ए॒षः । \newline
39. रस॑ ए॒ष ए॒ष रसो॒ रस॑ ए॒ष प॑शू॒नाम् प॑शू॒ना मे॒ष रसो॒ रस॑ ए॒ष प॑शू॒नाम् । \newline
40. ए॒ष प॑शू॒नाम् प॑शू॒ना मे॒ष ए॒ष प॑शू॒नां ॅयद् यत् प॑शू॒ना मे॒ष ए॒ष प॑शू॒नां ॅयत् । \newline
41. प॒शू॒नां ॅयद् यत् प॑शू॒नाम् प॑शू॒नां ॅयद् वसा॒ वसा॒ यत् प॑शू॒नाम् प॑शू॒नां ॅयद् वसा᳚ । \newline
42. यद् वसा॒ वसा॒ यद् यद् वसा॒ यद् यद् वसा॒ यद् यद् वसा॒ यत् । \newline
43. वसा॒ यद् यद् वसा॒ वसा॒ यत् पा॒र्श्वेन॑ पा॒र्श्वेन॒ यद् वसा॒ वसा॒ यत् पा॒र्श्वेन॑ । \newline
44. यत् पा॒र्श्वेन॑ पा॒र्श्वेन॒ यद् यत् पा॒र्श्वेन॑ वसाहो॒मं ॅव॑साहो॒मम् पा॒र्श्वेन॒ यद् यत् पा॒र्श्वेन॑ वसाहो॒मम् । \newline
45. पा॒र्श्वेन॑ वसाहो॒मं ॅव॑साहो॒मम् पा॒र्श्वेन॑ पा॒र्श्वेन॑ वसाहो॒मम् प्र॒यौति॑ प्र॒यौति॑ वसाहो॒मम् पा॒र्श्वेन॑ पा॒र्श्वेन॑ वसाहो॒मम् प्र॒यौति॑ । \newline
46. व॒सा॒हो॒मम् प्र॒यौति॑ प्र॒यौति॑ वसाहो॒मं ॅव॑साहो॒मम् प्र॒यौति॑ मद्ध्य॒तो म॑द्ध्य॒तः प्र॒यौति॑ वसाहो॒मं ॅव॑साहो॒मम् प्र॒यौति॑ मद्ध्य॒तः । \newline
47. व॒सा॒हो॒ममिति॑ वसा - हो॒मम् । \newline
48. प्र॒यौति॑ मद्ध्य॒तो म॑द्ध्य॒तः प्र॒यौति॑ प्र॒यौति॑ मद्ध्य॒त ए॒वैव म॑द्ध्य॒तः प्र॒यौति॑ प्र॒यौति॑ मद्ध्य॒त ए॒व । \newline
49. प्र॒यौतीति॑ प्र - यौति॑ । \newline
50. म॒द्ध्य॒त ए॒वैव म॑द्ध्य॒तो म॑द्ध्य॒त ए॒व प॑शू॒नाम् प॑शू॒ना मे॒व म॑द्ध्य॒तो म॑द्ध्य॒त ए॒व प॑शू॒नाम् । \newline
51. ए॒व प॑शू॒नाम् प॑शू॒ना मे॒वैव प॑शू॒नाꣳ रसꣳ॒॒ रस॑म् पशू॒ना मे॒वैव प॑शू॒नाꣳ रस᳚म् । \newline
52. प॒शू॒नाꣳ रसꣳ॒॒ रस॑म् पशू॒नाम् प॑शू॒नाꣳ रस॑म् दधाति दधाति॒ रस॑म् पशू॒नाम् प॑शू॒नाꣳ रस॑म् दधाति । \newline
53. रस॑म् दधाति दधाति॒ रसꣳ॒॒ रस॑म् दधाति॒ घ्नन्ति॒ घ्नन्ति॑ दधाति॒ रसꣳ॒॒ रस॑म् दधाति॒ घ्नन्ति॑ । \newline
54. द॒धा॒ति॒ घ्नन्ति॒ घ्नन्ति॑ दधाति दधाति॒ घ्नन्ति॒ वै वै घ्नन्ति॑ दधाति दधाति॒ घ्नन्ति॒ वै । \newline
55. घ्नन्ति॒ वै वै घ्नन्ति॒ घ्नन्ति॒ वा ए॒त दे॒तद् वै घ्नन्ति॒ घ्नन्ति॒ वा ए॒तत् । \newline
\pagebreak
\markright{ TS 6.3.11.2  \hfill https://www.vedavms.in \hfill}

\section{ TS 6.3.11.2 }

\textbf{TS 6.3.11.2 } \newline
\textbf{Samhita Paata} \newline

वा ए॒तत् प॒शुं ॅयथ् स᳚ज्ञ्ं॒पय॑न्त्यै॒न्द्रः खलु॒ वै दे॒वत॑या प्रा॒ण ऐ॒न्द्रो॑ऽपा॒न ऐ॒न्द्रः प्रा॒णो अङ्गे॑अङ्गे॒ नि दे᳚द्ध्य॒दित्या॑ह प्राणापा॒नावे॒व प॒शुषु॑ दधाति॒ देव॑ त्वष्ट॒र्भूरि॑ ते॒ सꣳ स॑मे॒त्वित्या॑ह त्वा॒ष्ट्रा हि दे॒वत॑या प॒शवो॒ विषु॑रूपा॒ यथ् सल॑क्ष्माणो॒ भव॒थेत्या॑ह॒ विषु॑रूपा॒ ह्ये॑ते सन्तः॒ सल॑क्ष्माण ए॒तर्.हि॒ भव॑न्ति देव॒त्रा यन्त॒- [  ] \newline

\textbf{Pada Paata} \newline

वै । ए॒तत् । प॒शुम् । यत् । स॒ज्ञ्ं॒पय॒न्तीति॑ सं - ज्ञ्॒पय॑न्ति । ऐ॒न्द्रः । खलु॑ । वै । दे॒वत॑या । प्रा॒ण इति॑ प्र - अ॒नः । ऐ॒न्द्रः । अ॒पा॒न इत्य॑प -अ॒नः । ऐ॒न्द्रः । प्रा॒ण इति॑ प्र - अ॒नः । अङ्गे॑अङ्ग॒ इत्यङ्गे᳚ - अ॒ङ्गे॒ । नीति॑ । दे॒द्ध्य॒त् । इति॑ । आ॒ह॒ । प्रा॒णा॒पा॒नाविति॑ प्राण - अ॒पा॒नौ । ए॒व । प॒शुषु॑ । द॒धा॒ति॒ । देव॑ । त्व॒ष्टः॒ । भूरि॑ । ते॒ । सꣳस॒मिति॒ सं-स॒म् । ए॒तु॒ । इति॑ । आ॒ह॒ । त्वा॒ष्ट्राः । हि । दे॒वत॑या । प॒शवः॑ । विषु॑रूपा॒ इति॒ विषु॑ - रू॒पाः॒ । यत् । सल॑क्ष्माण॒ इति॒ स-ल॒क्ष्मा॒णः॒ । भव॑थ । इति॑ । आ॒ह॒ । विषु॑रूपा॒ इति॒ विषु॑-रू॒पाः॒ । हि । ए॒ते । सन्तः॑ । सल॑क्ष्माण॒ इति॒ स - ल॒क्ष्मा॒णः॒ । ए॒तर्.हि॑ । भव॑न्ति । दे॒व॒त्रेति॑ देव - त्रा । यन्त᳚म् ।  \newline


\textbf{Krama Paata} \newline

वा ए॒तत् । ए॒तत् प॒शुम् । प॒शुम् ॅयत् । यथ् स᳚म्(2)ज्ञ्॒पय॑न्ति । स॒म्(2)ज्ञ्॒पय॑न्त्यै॒न्द्रः । स॒म्(2)ज्ञ्॒पय॒न्तीति॑ सम् - ज्ञ्॒पय॑न्ति । ऐ॒न्द्रः खलु॑ । खलु॒ वै । वै दे॒वत॑या । दे॒वत॑या प्रा॒णः । प्रा॒ण ऐ॒न्द्रः । प्रा॒ण इति॑ प्र - अ॒नः । ऐ॒न्द्रो॑ऽपा॒नः । अ॒पा॒न ऐ॒न्द्रः । अ॒पा॒न इत्य॑प - अ॒नः । ऐ॒न्द्रः प्रा॒णः । प्रा॒णो अङ्‍गे॑अङ्‍गे । प्रा॒ण इति॑ प्र - अ॒नः । अङ्‍गे॑अङ्‍गे॒ नि । अङ्‍गे॑अङ्‍ग॒ इत्यङ्‍गे᳚ - अ॒ङ्‍गे॒ । नि दे᳚द्ध्यत् । दे॒द्ध्य॒दिति॑ । इत्या॑ह । आ॒ह॒ प्रा॒णा॒पा॒नौ । प्रा॒णा॒पा॒नावे॒व । प्रा॒णा॒पा॒नाविति॑ प्राण - अ॒पा॒नौ । ए॒व प॒शुषु॑ । प॒शुषु॑ दधाति । द॒धा॒ति॒ देव॑ । देव॑ त्वष्टः । त्व॒ष्ट॒र् भूरि॑ । भूरि॑ ते । ते॒ सꣳस᳚म् । सꣳस॑मेतु । सꣳस॒मिति॒ सम् - स॒म् । ए॒त्विति॑ । इत्या॑ह । आ॒ह॒ त्वा॒ष्ट्राः । त्वा॒ष्ट्रा हि । हि दे॒वत॑या । दे॒वत॑या प॒शवः॑ । प॒शवो॒ विषु॑रूपाः । विषु॑रूपा॒ यत् । विषु॑रूपा॒ इति॒ विषु॑ - रू॒पाः॒ । यथ् सल॑क्ष्माणः । सल॑क्ष्माणो॒ भव॑थ । सल॑क्ष्माण॒ इति॒ स - ल॒क्ष्मा॒णः॒ । भव॒थेति॑ । इत्या॑ह । आ॒ह॒ विषु॑रूपाः । विषु॑रूपा॒ हि । विषु॑रूपा॒ इति॒ विषु॑ - रू॒पाः॒ । ह्ये॑ते । ए॒ते सन्तः॑ । सन्तः॒ सल॑क्ष्माणः । सल॑क्ष्माण ए॒तर्.हि॑ । सल॑क्ष्माण॒ इति॒ स - ल॒क्ष्मा॒णः॒ । ए॒तर्.हि॒ भव॑न्ति । भव॑न्ति देव॒त्रा । दे॒व॒त्रा यन्त᳚म् । दे॒व॒त्रेति॑ देव - त्रा । यन्त॒मव॑से \newline

\textbf{Jatai Paata} \newline

1. वा ए॒त दे॒तद् वै वा ए॒तत् । \newline
2. ए॒तत् प॒शुम् प॒शु मे॒त दे॒तत् प॒शुम् । \newline
3. प॒शुं ॅयद् यत् प॒शुम् प॒शुं ॅयत् । \newline
4. यथ् सं᳚.ज्ञ्॒पय॑न्ति सं.ज्ञ्॒पय॑न्ति॒ यद् यथ् सं᳚.ज्ञ्॒पय॑न्ति । \newline
5. सं॒.ज्ञ्॒पय॑ न्त्यै॒न्द्र ऐ॒न्द्रः सं᳚.ज्ञ्॒पय॑न्ति सं.ज्ञ्॒पय॑ न्त्यै॒न्द्रः । \newline
6. सं॒.ज्ञ्॒पय॒न्तीति॑ सं. - ज्ञ्॒पय॑न्ति । \newline
7. ऐ॒न्द्रः खलु॒ खल्वै॒न्द्र ऐ॒न्द्रः खलु॑ । \newline
8. खलु॒ वै वै खलु॒ खलु॒ वै । \newline
9. वै दे॒वत॑या दे॒वत॑या॒ वै वै दे॒वत॑या । \newline
10. दे॒वत॑या प्रा॒णः प्रा॒णो दे॒वत॑या दे॒वत॑या प्रा॒णः । \newline
11. प्रा॒ण ऐ॒न्द्र ऐ॒न्द्रः प्रा॒णः प्रा॒ण ऐ॒न्द्रः । \newline
12. प्रा॒ण इति॑ प्र - अ॒नः । \newline
13. ऐ॒न्द्रो॑ ऽपा॒नो॑ ऽपा॒न ऐ॒न्द्र ऐ॒न्द्रो॑ ऽपा॒नः । \newline
14. अ॒पा॒न ऐ॒न्द्र ऐ॒न्द्रो॑ ऽपा॒नो॑ ऽपा॒न ऐ॒न्द्रः । \newline
15. अ॒पा॒न इत्य॑प - अ॒नः । \newline
16. ऐ॒न्द्रः प्रा॒णः प्रा॒ण ऐ॒न्द्र ऐ॒न्द्रः प्रा॒णः । \newline
17. प्रा॒णो अङ्गे॑अङ्गे॒ अङ्गे॑अङ्गे प्रा॒णः प्रा॒णो अङ्गे॑अङ्गे । \newline
18. प्रा॒ण इति॑ प्र - अ॒नः । \newline
19. अङ्गे॑अङ्गे॒ नि न्यङ्गे॑अङ्गे॒ अङ्गे॑अङ्गे॒ नि । \newline
20. अङ्गे॑अङ्ग॒ इत्यङ्गे᳚ - अ॒ङ्गे॒ । \newline
21. नि दे᳚द्ध्यद् देद्ध्य॒न् नि नि दे᳚द्ध्यत् । \newline
22. दे॒द्ध्य॒ दितीति॑ देद्ध्यद् देद्ध्य॒ दिति॑ । \newline
23. इत्या॑हा॒हे तीत्या॑ह । \newline
24. आ॒ह॒ प्रा॒णा॒पा॒नौ प्रा॑णापा॒ना वा॑हाह प्राणापा॒नौ । \newline
25. प्रा॒णा॒पा॒ना वे॒वैव प्रा॑णापा॒नौ प्रा॑णापा॒ना वे॒व । \newline
26. प्रा॒णा॒पा॒नाविति॑ प्राण - अ॒पा॒नौ । \newline
27. ए॒व प॒शुषु॑ प॒शु ष्वे॒वैव प॒शुषु॑ । \newline
28. प॒शुषु॑ दधाति दधाति प॒शुषु॑ प॒शुषु॑ दधाति । \newline
29. द॒धा॒ति॒ देव॒ देव॑ दधाति दधाति॒ देव॑ । \newline
30. देव॑ त्वष्ट स्त्वष्ट॒र् देव॒ देव॑ त्वष्टः । \newline
31. त्व॒ष्ट॒र् भूरि॒ भूरि॑ त्वष्ट स्त्वष्ट॒र् भूरि॑ । \newline
32. भूरि॑ ते ते॒ भूरि॒ भूरि॑ ते । \newline
33. ते॒ सꣳसꣳ॒॒ सꣳस॑म् ते ते॒ सꣳस᳚म् । \newline
34. सꣳस॑ मेत्वेतु॒ सꣳसꣳ॒॒ सꣳस॑ मेतु । \newline
35. सꣳस॒मिति॒ सं - स॒म् । \newline
36. ए॒त्विती त्ये᳚त्वे॒ त्विति॑ । \newline
37. इत्या॑हा॒हे तीत्या॑ह । \newline
38. आ॒ह॒ त्वा॒ष्ट्रा स्त्वा॒ष्ट्रा आ॑हाह त्वा॒ष्ट्राः । \newline
39. त्वा॒ष्ट्रा हि हि त्वा॒ष्ट्रा स्त्वा॒ष्ट्रा हि । \newline
40. हि दे॒वत॑या दे॒वत॑या॒ हि हि दे॒वत॑या । \newline
41. दे॒वत॑या प॒शवः॑ प॒शवो॑ दे॒वत॑या दे॒वत॑या प॒शवः॑ । \newline
42. प॒शवो॒ विषु॑रूपा॒ विषु॑रूपाः प॒शवः॑ प॒शवो॒ विषु॑रूपाः । \newline
43. विषु॑रूपा॒ यद् यद् विषु॑रूपा॒ विषु॑रूपा॒ यत् । \newline
44. विषु॑रूपा॒ इति॒ विषु॑ - रू॒पाः॒ । \newline
45. यथ् सल॑क्ष्माणः॒ सल॑क्ष्माणो॒ यद् यथ् सल॑क्ष्माणः । \newline
46. सल॑क्ष्माणो॒ भव॑थ॒ भव॑थ॒ सल॑क्ष्माणः॒ सल॑क्ष्माणो॒ भव॑थ । \newline
47. सल॑क्ष्माण॒ इति॒ स - ल॒क्ष्मा॒णः॒ । \newline
48. भव॒थे तीति॒ भव॑थ॒ भव॒थेति॑ । \newline
49. इत्या॑हा॒हे तीत्या॑ह । \newline
50. आ॒ह॒ विषु॑रूपा॒ विषु॑रूपा आहाह॒ विषु॑रूपाः । \newline
51. विषु॑रूपा॒ हि हि विषु॑रूपा॒ विषु॑रूपा॒ हि । \newline
52. विषु॑रूपा॒ इति॒ विषु॑ - रू॒पाः॒ । \newline
53. ह्ये॑त ए॒ते हि ह्ये॑ते । \newline
54. ए॒ते सन्तः॒ सन्त॑ ए॒त ए॒ते सन्तः॑ । \newline
55. सन्तः॒ सल॑क्ष्माणः॒ सल॑क्ष्माणः॒ सन्तः॒ सन्तः॒ सल॑क्ष्माणः । \newline
56. सल॑क्ष्माण ए॒तर् ह्ये॒तर्.हि॒ सल॑क्ष्माणः॒ सल॑क्ष्माण ए॒तर्.हि॑ । \newline
57. सल॑क्ष्माण॒ इति॒ स - ल॒क्ष्मा॒णः॒ । \newline
58. ए॒तर्.हि॒ भव॑न्ति॒ भव॑ न्त्ये॒तर् ह्ये॒तर्.हि॒ भव॑न्ति । \newline
59. भव॑न्ति देव॒त्रा दे॑व॒त्रा भव॑न्ति॒ भव॑न्ति देव॒त्रा । \newline
60. दे॒व॒त्रा यन्तं॒ ॅयन्त॑म् देव॒त्रा दे॑व॒त्रा यन्त᳚म् । \newline
61. दे॒व॒त्रेति॑ देव - त्रा । \newline
62. यन्त॒ मव॒से ऽव॑से॒ यन्तं॒ ॅयन्त॒ मव॑से । \newline

\textbf{Ghana Paata } \newline

1. वा ए॒त दे॒तद् वै वा ए॒तत् प॒शुम् प॒शु मे॒तद् वै वा ए॒तत् प॒शुम् । \newline
2. ए॒तत् प॒शुम् प॒शु मे॒त दे॒तत् प॒शुं ॅयद् यत् प॒शु मे॒त दे॒तत् प॒शुं ॅयत् । \newline
3. प॒शुं ॅयद् यत् प॒शुम् प॒शुं ॅयथ् सं᳚.ज्ञ्॒पय॑न्ति सं.ज्ञ्॒पय॑न्ति॒ यत् प॒शुम् प॒शुं ॅयथ् 
सं᳚.ज्ञ्॒पय॑न्ति । \newline
4. यथ् सं᳚.ज्ञ्॒पय॑न्ति सं.ज्ञ्॒पय॑न्ति॒ यद् यथ् सं᳚.ज्ञ्॒पय॑ न्त्यै॒न्द्र ऐ॒न्द्रः सं᳚.ज्ञ्॒पय॑न्ति॒ यद् यथ् 
सं᳚.ज्ञ्॒पय॑ न्त्यै॒न्द्रः । \newline
5. सं॒.ज्ञ्॒पय॑ न्त्यै॒न्द्र ऐ॒न्द्रः सं᳚.ज्ञ्॒पय॑न्ति सं.ज्ञ्॒पय॑ न्त्यै॒न्द्रः खलु॒ खल्वै॒न्द्रः 
सं᳚.ज्ञ्॒पय॑न्ति सं.ज्ञ्॒पय॑ न्त्यै॒न्द्रः खलु॑ । \newline
6. सं॒.ज्ञ्॒पय॒न्तीति॑ सं. - ज्ञ्॒पय॑न्ति । \newline
7. ऐ॒न्द्रः खलु॒ खल्वै॒न्द्र ऐ॒न्द्रः खलु॒ वै वै खल्वै॒न्द्र ऐ॒न्द्रः खलु॒ वै । \newline
8. खलु॒ वै वै खलु॒ खलु॒ वै दे॒वत॑या दे॒वत॑या॒ वै खलु॒ खलु॒ वै दे॒वत॑या । \newline
9. वै दे॒वत॑या दे॒वत॑या॒ वै वै दे॒वत॑या प्रा॒णः प्रा॒णो दे॒वत॑या॒ वै वै दे॒वत॑या प्रा॒णः । \newline
10. दे॒वत॑या प्रा॒णः प्रा॒णो दे॒वत॑या दे॒वत॑या प्रा॒ण ऐ॒न्द्र ऐ॒न्द्रः प्रा॒णो दे॒वत॑या दे॒वत॑या प्रा॒ण ऐ॒न्द्रः । \newline
11. प्रा॒ण ऐ॒न्द्र ऐ॒न्द्रः प्रा॒णः प्रा॒ण ऐ॒न्द्रो॑ ऽपा॒नो॑ ऽपा॒न ऐ॒न्द्रः प्रा॒णः प्रा॒ण ऐ॒न्द्रो॑ ऽपा॒नः । \newline
12. प्रा॒ण इति॑ प्र - अ॒नः । \newline
13. ऐ॒न्द्रो॑ ऽपा॒नो॑ ऽपा॒न ऐ॒न्द्र ऐ॒न्द्रो॑ ऽपा॒न ऐ॒न्द्र ऐ॒न्द्रो॑ ऽपा॒न ऐ॒न्द्र ऐ॒न्द्रो॑ ऽपा॒न ऐ॒न्द्रः । \newline
14. अ॒पा॒न ऐ॒न्द्र ऐ॒न्द्रो॑ ऽपा॒नो॑ ऽपा॒न ऐ॒न्द्रः प्रा॒णः प्रा॒ण ऐ॒न्द्रो॑ ऽपा॒नो॑ ऽपा॒न ऐ॒न्द्रः प्रा॒णः । \newline
15. अ॒पा॒न इत्य॑प - अ॒नः । \newline
16. ऐ॒न्द्रः प्रा॒णः प्रा॒ण ऐ॒न्द्र ऐ॒न्द्रः प्रा॒णो अङ्गे॑अङ्गे॒ अङ्गे॑अङ्गे प्रा॒ण ऐ॒न्द्र ऐ॒न्द्रः प्रा॒णो अङ्गे॑अङ्गे । \newline
17. प्रा॒णो अङ्गे॑अङ्गे॒ अङ्गे॑अङ्गे प्रा॒णः प्रा॒णो अङ्गे॑अङ्गे॒ नि न्यङ्गे॑अङ्गे प्रा॒णः प्रा॒णो अङ्गे॑अङ्गे॒ नि । \newline
18. प्रा॒ण इति॑ प्र - अ॒नः । \newline
19. अङ्गे॑अङ्गे॒ नि न्यङ्गे॑अङ्गे॒ अङ्गे॑अङ्गे॒ नि दे᳚द्ध्यद् देद्ध्य॒न् न्यङ्गे॑अङ्गे॒ अङ्गे॑अङ्गे॒ नि दे᳚द्ध्यत् । \newline
20. अङ्गे॑अङ्ग॒ इत्यङ्गे᳚ - अ॒ङ्गे॒ । \newline
21. नि दे᳚द्ध्यद् देद्ध्य॒न् नि नि दे᳚द्ध्य॒ दितीति॑ देद्ध्य॒न् नि नि दे᳚द्ध्य॒ दिति॑ । \newline
22. दे॒द्ध्य॒ दितीति॑ देद्ध्यद् देद्ध्य॒ दित्या॑हा॒ हेति॑ देद्ध्यद् देद्ध्य॒ दित्या॑ह । \newline
23. इत्या॑हा॒हे तीत्या॑ह प्राणापा॒नौ प्रा॑णापा॒ना वा॒हे तीत्या॑ह प्राणापा॒नौ । \newline
24. आ॒ह॒ प्रा॒णा॒पा॒नौ प्रा॑णापा॒ना वा॑हाह प्राणापा॒ना वे॒वैव प्रा॑णापा॒ना वा॑हाह प्राणापा॒ना वे॒व । \newline
25. प्रा॒णा॒पा॒ना वे॒वैव प्रा॑णापा॒नौ प्रा॑णापा॒ना वे॒व प॒शुषु॑ प॒शुष्वे॒व प्रा॑णापा॒नौ प्रा॑णापा॒ना वे॒व प॒शुषु॑ । \newline
26. प्रा॒णा॒पा॒नाविति॑ प्राण - अ॒पा॒नौ । \newline
27. ए॒व प॒शुषु॑ प॒शुष्वे॒ वैव प॒शुषु॑ दधाति दधाति प॒शुष्वे॒ वैव प॒शुषु॑ दधाति । \newline
28. प॒शुषु॑ दधाति दधाति प॒शुषु॑ प॒शुषु॑ दधाति॒ देव॒ देव॑ दधाति प॒शुषु॑ प॒शुषु॑ दधाति॒ देव॑ । \newline
29. द॒धा॒ति॒ देव॒ देव॑ दधाति दधाति॒ देव॑ त्वष्ट स्त्वष्ट॒र् देव॑ दधाति दधाति॒ देव॑ त्वष्टः । \newline
30. देव॑ त्वष्ट स्त्वष्ट॒र् देव॒ देव॑ त्वष्ट॒र् भूरि॒ भूरि॑ त्वष्ट॒र् देव॒ देव॑ त्वष्ट॒र् भूरि॑ । \newline
31. त्व॒ष्ट॒र् भूरि॒ भूरि॑ त्वष्ट स्त्वष्ट॒र् भूरि॑ ते ते॒ भूरि॑ त्वष्ट स्त्वष्ट॒र् भूरि॑ ते । \newline
32. भूरि॑ ते ते॒ भूरि॒ भूरि॑ ते॒ सꣳसꣳ॒॒ सꣳस॑म् ते॒ भूरि॒ भूरि॑ ते॒ सꣳस᳚म् । \newline
33. ते॒ सꣳसꣳ॒॒ सꣳस॑म् ते ते॒ सꣳस॑ मेत्वेतु॒ सꣳस॑म् ते ते॒ सꣳस॑ मेतु । \newline
34. सꣳस॑ मेत्वेतु॒ सꣳसꣳ॒॒ सꣳस॑ मे॒त्विती त्ये॑तु॒ सꣳसꣳ॒॒ सꣳस॑ मे॒त्विति॑ । \newline
35. सꣳस॒मिति॒ सं - स॒म् । \newline
36. ए॒त्वि तीत्ये᳚ त्वे॒त्वि त्या॑हा॒हे त्ये᳚ त्वे॒त्वि त्या॑ह । \newline
37. इत्या॑हा॒हे तीत्या॑ह त्वा॒ष्ट्रा स्त्वा॒ष्ट्रा आ॒हे तीत्या॑ह त्वा॒ष्ट्राः । \newline
38. आ॒ह॒ त्वा॒ष्ट्रा स्त्वा॒ष्ट्रा आ॑हाह त्वा॒ष्ट्रा हि हि त्वा॒ष्ट्रा आ॑हाह त्वा॒ष्ट्रा हि । \newline
39. त्वा॒ष्ट्रा हि हि त्वा॒ष्ट्रा स्त्वा॒ष्ट्रा हि दे॒वत॑या दे॒वत॑या॒ हि त्वा॒ष्ट्रा स्त्वा॒ष्ट्रा हि दे॒वत॑या । \newline
40. हि दे॒वत॑या दे॒वत॑या॒ हि हि दे॒वत॑या प॒शवः॑ प॒शवो॑ दे॒वत॑या॒ हि हि दे॒वत॑या प॒शवः॑ । \newline
41. दे॒वत॑या प॒शवः॑ प॒शवो॑ दे॒वत॑या दे॒वत॑या प॒शवो॒ विषु॑रूपा॒ विषु॑रूपाः प॒शवो॑ दे॒वत॑या दे॒वत॑या प॒शवो॒ विषु॑रूपाः । \newline
42. प॒शवो॒ विषु॑रूपा॒ विषु॑रूपाः प॒शवः॑ प॒शवो॒ विषु॑रूपा॒ यद् यद् विषु॑रूपाः प॒शवः॑ प॒शवो॒ विषु॑रूपा॒ यत् । \newline
43. विषु॑रूपा॒ यद् यद् विषु॑रूपा॒ विषु॑रूपा॒ यथ् सल॑क्ष्माणः॒ सल॑क्ष्माणो॒ यद् विषु॑रूपा॒ विषु॑रूपा॒ यथ् सल॑क्ष्माणः । \newline
44. विषु॑रूपा॒ इति॒ विषु॑ - रू॒पाः॒ । \newline
45. यथ् सल॑क्ष्माणः॒ सल॑क्ष्माणो॒ यद् यथ् सल॑क्ष्माणो॒ भव॑थ॒ भव॑थ॒ सल॑क्ष्माणो॒ यद् यथ् सल॑क्ष्माणो॒ भव॑थ । \newline
46. सल॑क्ष्माणो॒ भव॑थ॒ भव॑थ॒ सल॑क्ष्माणः॒ सल॑क्ष्माणो॒ भव॒थे तीति॒ भव॑थ॒ सल॑क्ष्माणः॒ सल॑क्ष्माणो॒ भव॒थेति॑ । \newline
47. सल॑क्ष्माण॒ इति॒ स - ल॒क्ष्मा॒णः॒ । \newline
48. भव॒थे तीति॒ भव॑थ॒ भव॒थे त्या॑हा॒ हेति॒ भव॑थ॒ भव॒थे त्या॑ह । \newline
49. इत्या॑हा॒हे तीत्या॑ह॒ विषु॑रूपा॒ विषु॑रूपा आ॒हे तीत्या॑ह॒ विषु॑रूपाः । \newline
50. आ॒ह॒ विषु॑रूपा॒ विषु॑रूपा आहाह॒ विषु॑रूपा॒ हि हि विषु॑रूपा आहाह॒ विषु॑रूपा॒ हि । \newline
51. विषु॑रूपा॒ हि हि विषु॑रूपा॒ विषु॑रूपा॒ ह्ये॑त ए॒ते हि विषु॑रूपा॒ विषु॑रूपा॒ ह्ये॑ते । \newline
52. विषु॑रूपा॒ इति॒ विषु॑ - रू॒पाः॒ । \newline
53. ह्ये॑त ए॒ते हि ह्ये॑ते सन्तः॒ सन्त॑ ए॒ते हि ह्ये॑ते सन्तः॑ । \newline
54. ए॒ते सन्तः॒ सन्त॑ ए॒त ए॒ते सन्तः॒ सल॑क्ष्माणः॒ सल॑क्ष्माणः॒ सन्त॑ ए॒त ए॒ते सन्तः॒ सल॑क्ष्माणः । \newline
55. सन्तः॒ सल॑क्ष्माणः॒ सल॑क्ष्माणः॒ सन्तः॒ सन्तः॒ सल॑क्ष्माण ए॒तर् ह्ये॒तर्.हि॒ सल॑क्ष्माणः॒ सन्तः॒ सन्तः॒ सल॑क्ष्माण ए॒तर्.हि॑ । \newline
56. सल॑क्ष्माण ए॒तर् ह्ये॒तर्.हि॒ सल॑क्ष्माणः॒ सल॑क्ष्माण ए॒तर्.हि॒ भव॑न्ति॒ भव॑ न्त्ये॒तर्.हि॒ सल॑क्ष्माणः॒ सल॑क्ष्माण ए॒तर्.हि॒ भव॑न्ति । \newline
57. सल॑क्ष्माण॒ इति॒ स - ल॒क्ष्मा॒णः॒ । \newline
58. ए॒तर्.हि॒ भव॑न्ति॒ भव॑ न्त्ये॒तर् ह्ये॒तर्.हि॒ भव॑न्ति देव॒त्रा दे॑व॒त्रा भव॑ न्त्ये॒तर् ह्ये॒तर्.हि॒ भव॑न्ति देव॒त्रा । \newline
59. भव॑न्ति देव॒त्रा दे॑व॒त्रा भव॑न्ति॒ भव॑न्ति देव॒त्रा यन्तं॒ ॅयन्त॑म् देव॒त्रा भव॑न्ति॒ भव॑न्ति देव॒त्रा यन्त᳚म् । \newline
60. दे॒व॒त्रा यन्तं॒ ॅयन्त॑म् देव॒त्रा दे॑व॒त्रा यन्त॒ मव॒से ऽव॑से॒ यन्त॑म् देव॒त्रा दे॑व॒त्रा यन्त॒ मव॑से । \newline
61. दे॒व॒त्रेति॑ देव - त्रा । \newline
62. यन्त॒ मव॒से ऽव॑से॒ यन्तं॒ ॅयन्त॒ मव॑से॒ सखा॑यः॒ सखा॒यो ऽव॑से॒ यन्तं॒ ॅयन्त॒ मव॑से॒ सखा॑यः । \newline
\pagebreak
\markright{ TS 6.3.11.3  \hfill https://www.vedavms.in \hfill}

\section{ TS 6.3.11.3 }

\textbf{TS 6.3.11.3 } \newline
\textbf{Samhita Paata} \newline

-मव॑से॒ सखा॒योऽनु॑ त्वा मा॒ता पि॒तरो॑ मद॒न्त्वित्या॒हा-नु॑मतमे॒वैनं॑ मा॒त्रा पि॒त्रा सु॑व॒र्गं ॅलो॒कं ग॑मयत्यर्द्ध॒र्चे व॑साहो॒मं जु॑होत्य॒सौ वा अ॑र्द्ध॒र्च इ॒यम॑र्द्ध॒च इ॒मे ए॒व रसे॑नानक्ति॒ दिशो॑ जुहोति॒ दिश॑ ए॒व रसे॑नान॒क्त्यथो॑ दि॒ग्भ्य ए॒वोर्जꣳ॒॒ रस॒मव॑ रुन्धे प्राणापा॒नौ वा ए॒तौ प॑शू॒नां ॅयत् पृ॑षदा॒ज्यं ॅवा॑नस्प॒त्याः खलु॒- [  ] \newline

\textbf{Pada Paata} \newline

अव॑से । सखा॑यः । अन्विति॑ । त्वा॒ । मा॒ता । पि॒तरः॑ । म॒द॒न्तु॒ । इति॑ । आ॒ह॒ । अनु॑मत॒मित्यनु॑ - म॒त॒म् । ए॒व । ए॒न॒म् । मा॒त्रा । पि॒त्रा । सु॒व॒र्गमिति॑ सुवः - गम् । लो॒कम् । ग॒म॒य॒ति॒ । अ॒द्‌र्ध॒र्च इत्य॑द्‌र्ध-ऋ॒चे । व॒सा॒हो॒ममिति॑ वसा - हो॒मम् । जु॒हो॒ति॒ । अ॒सौ । वै । अ॒द्‌र्ध॒र्च इत्य॑द्‌र्ध-ऋ॒चः । इ॒यम् । अ॒द्‌र्ध॒र्च इत्य॑द्‌र्ध - ऋ॒चः । इ॒मे इति॑ । ए॒व । रसे॑न । अ॒न॒क्ति॒ । दिशः॑ । जु॒हो॒ति॒ । दिशः॑ । ए॒व । रसे॑न । अ॒न॒क्ति॒ । अथो॒ इति॑ । दि॒ग्भ्य इति॑ दिक् - भ्यः । ए॒व । ऊर्ज᳚म् । रस᳚म् । अवेति॑ । रु॒न्धे॒ । प्रा॒णा॒पा॒नाविति॑ प्राण - अ॒पा॒नौ । वै । ए॒तौ । प॒शू॒नाम् । यत् । पृ॒ष॒दा॒ज्यमिति॑ पृषत् - आ॒ज्यम् । वा॒न॒स्प॒त्याः । खलु॑ ।  \newline


\textbf{Krama Paata} \newline

अव॑से॒ सखा॑यः । सखा॒योऽनु॑ । अनु॑ त्वा । त्वा॒ मा॒ता । मा॒ता पि॒तरः॑ । पि॒तरो॑ मदन्तु । म॒द॒न्त्विति॑ । इत्या॑ह । आ॒हानु॑मतम् । अनु॑मतमे॒व । अनु॑मत॒मित्यनु॑ - म॒त॒म् । ए॒वैन᳚म् । ए॒न॒म् मा॒त्रा । मा॒त्रा पि॒त्रा । पि॒त्रा सु॑व॒र्गम् । सु॒व॒र्गम् ॅलो॒कम् । सु॒व॒र्गमिति॑ सुवः - गम् । लो॒कम् ग॑मयति । ग॒म॒य॒त्य॒र्द्ध॒र्चे । अ॒र्द्ध॒र्चे व॑साहो॒मम् । अ॒र्द्ध॒र्च इत्य॑र्द्ध - ऋ॒चे । व॒सा॒हो॒मम् जु॑होति । व॒सा॒हो॒ममिति॑ वसा - हो॒मम् । जु॒हो॒त्य॒सौ । अ॒सौ वै । वा अ॑र्द्ध॒र्चः । अ॒र्द्ध॒र्च इ॒यम् । अ॒र्द्ध॒र्च इत्य॑र्द्ध - ऋ॒चः । इ॒यम॑र्द्ध॒र्चः । अ॒र्द्ध॒र्च इ॒मे । अ॒र्द्ध॒र्च इत्य॑र्द्ध - ऋ॒चः । इ॒मे ए॒व । इ॒मे इती॒मे । ए॒व रसे॑न । रसे॑नानक्ति । अ॒न॒क्ति॒ दिशः॑ । दिशो॑ जुहोति । जु॒हो॒ति॒ दिशः॑ । दिश॑ ए॒व । ए॒व रसे॑न । रसे॑नानक्ति । अ॒न॒क्त्यथो᳚ । अथो॑ दि॒ग्भ्यः । अथो॒ इत्यथो᳚ । दि॒ग्भ्य ए॒व । दि॒ग्भ्य इति॑ दिक् - भ्यः । ए॒वोर्ज᳚म् । ऊर्जꣳ॒॒ रस᳚म् । रस॒मव॑ । अव॑ रुन्धे । रु॒न्धे॒ प्रा॒णा॒पा॒नौ । प्रा॒णा॒पा॒नौ वै । प्रा॒णा॒पा॒नाविति॑ प्राण - अ॒पा॒नौ । वा ए॒तौ । ए॒तौ प॑शू॒नाम् । प॒शू॒नाम् ॅयत् । यत् पृ॑षदा॒ज्यम् । पृ॒ष॒दा॒ज्यम् ॅवा॑नस्प॒त्याः । पृ॒ष॒दा॒ज्यमिति॑ पृषत् - आ॒ज्यम् । वा॒न॒स्प॒त्याः खलु॑ । खलु॒ वै \newline

\textbf{Jatai Paata} \newline

1. अव॑से॒ सखा॑यः॒ सखा॒यो ऽव॒से ऽव॑से॒ सखा॑यः । \newline
2. सखा॒यो ऽन्वनु॒ सखा॑यः॒ सखा॒यो ऽनु॑ । \newline
3. अनु॑ त्वा॒ त्वा ऽन्वनु॑ त्वा । \newline
4. त्वा॒ मा॒ता मा॒ता त्वा᳚ त्वा मा॒ता । \newline
5. मा॒ता पि॒तरः॑ पि॒तरो॑ मा॒ता मा॒ता पि॒तरः॑ । \newline
6. पि॒तरो॑ मदन्तु मदन्तु पि॒तरः॑ पि॒तरो॑ मदन्तु । \newline
7. म॒द॒न् त्वितीति॑ मदन्तु मद॒न् त्विति॑ । \newline
8. इत्या॑हा॒हे तीत्या॑ह । \newline
9. आ॒हा नु॑मत॒ मनु॑मत माहा॒ हानु॑मतम् । \newline
10. अनु॑मत मे॒वै वानु॑मत॒ मनु॑मत मे॒व । \newline
11. अनु॑मत॒मित्यनु॑ - म॒त॒म् । \newline
12. ए॒वैन॑ मेन मे॒वै वैन᳚म् । \newline
13. ए॒न॒म् मा॒त्रा मा॒त्रैन॑ मेनम् मा॒त्रा । \newline
14. मा॒त्रा पि॒त्रा पि॒त्रा मा॒त्रा मा॒त्रा पि॒त्रा । \newline
15. पि॒त्रा सु॑व॒र्गꣳ सु॑व॒र्गम् पि॒त्रा पि॒त्रा सु॑व॒र्गम् । \newline
16. सु॒व॒र्गम् ॅलो॒कम् ॅलो॒कꣳ सु॑व॒र्गꣳ सु॑व॒र्गम् ॅलो॒कम् । \newline
17. सु॒व॒र्गमिति॑ सुवः - गम् । \newline
18. लो॒कम् ग॑मयति गमयति लो॒कम् ॅलो॒कम् ग॑मयति । \newline
19. ग॒म॒य॒ त्य॒र्द्ध॒र्चे᳚ ऽर्द्ध॒र्चे ग॑मयति गमय त्यर्द्ध॒र्चे । \newline
20. अ॒र्द्ध॒र्चे व॑साहो॒मं ॅव॑साहो॒म म॑र्द्ध॒र्चे᳚ ऽर्द्ध॒र्चे व॑साहो॒मम् । \newline
21. अ॒र्द्ध॒र्च इत्य॑र्द्ध - ऋ॒चे । \newline
22. व॒सा॒हो॒मम् जु॑होति जुहोति वसाहो॒मं ॅव॑साहो॒मम् जु॑होति । \newline
23. व॒सा॒हो॒ममिति॑ वसा - हो॒मम् । \newline
24. जु॒हो॒ त्य॒सा व॒सौ जु॑होति जुहो त्य॒सौ । \newline
25. अ॒सौ वै वा अ॒सा व॒सौ वै । \newline
26. वा अ॑र्द्ध॒र्चो᳚ ऽर्द्ध॒र्चो वै वा अ॑र्द्ध॒र्चः । \newline
27. अ॒र्द्ध॒र्च इ॒य मि॒य म॑र्द्ध॒र्चो᳚ ऽर्द्ध॒र्च इ॒यम् । \newline
28. अ॒र्द्ध॒र्च इत्य॑र्द्ध - ऋ॒चः । \newline
29. इ॒य म॑र्द्ध॒र्चो᳚ ऽर्द्ध॒र्च इ॒य मि॒य म॑र्द्ध॒र्चः । \newline
30. अ॒र्द्ध॒र्च इ॒मे इ॒मे अ॑र्द्ध॒र्चो᳚ ऽर्द्ध॒र्च इ॒मे । \newline
31. अ॒र्द्ध॒र्च इत्य॑र्द्ध - ऋ॒चः । \newline
32. इ॒मे ए॒वैवेमे इ॒मे ए॒व । \newline
33. इ॒मे इती॒मे । \newline
34. ए॒व रसे॑न॒ रसे॑नै॒ वैव रसे॑न । \newline
35. रसे॑ना नक्त्य नक्ति॒ रसे॑न॒ रसे॑ना नक्ति । \newline
36. अ॒न॒क्ति॒ दिशो॒ दिशो॑ ऽनक् त्यनक्ति॒ दिशः॑ । \newline
37. दिशो॑ जुहोति जुहोति॒ दिशो॒ दिशो॑ जुहोति । \newline
38. जु॒हो॒ति॒ दिशो॒ दिशो॑ जुहोति जुहोति॒ दिशः॑ । \newline
39. दिश॑ ए॒वैव दिशो॒ दिश॑ ए॒व । \newline
40. ए॒व रसे॑न॒ रसे॑नै॒ वैव रसे॑न । \newline
41. रसे॑ना नक्त्य नक्ति॒ रसे॑न॒ रसे॑ना नक्ति । \newline
42. अ॒न॒क् त्यथो॒ अथो॑ अनक् त्यन॒क् त्यथो᳚ । \newline
43. अथो॑ दि॒ग्भ्यो दि॒ग्भ्यो ऽथो॒ अथो॑ दि॒ग्भ्यः । \newline
44. अथो॒ इत्यथो᳚ । \newline
45. दि॒ग्भ्य ए॒वैव दि॒ग्भ्यो दि॒ग्भ्य ए॒व । \newline
46. दि॒ग्भ्य इति॑ दिक् - भ्यः । \newline
47. ए॒वोर्ज॒ मूर्ज॑ मे॒वै वोर्ज᳚म् । \newline
48. ऊर्जꣳ॒॒ रसꣳ॒॒ रस॒ मूर्ज॒ मूर्जꣳ॒॒ रस᳚म् । \newline
49. रस॒ मवाव॒ रसꣳ॒॒ रस॒ मव॑ । \newline
50. अव॑ रुन्धे रु॒न्धे ऽवाव॑ रुन्धे । \newline
51. रु॒न्धे॒ प्रा॒णा॒पा॒नौ प्रा॑णापा॒नौ रु॑न्धे रुन्धे प्राणापा॒नौ । \newline
52. प्रा॒णा॒पा॒नौ वै वै प्रा॑णापा॒नौ प्रा॑णापा॒नौ वै । \newline
53. प्रा॒णा॒पा॒नाविति॑ प्राण - अ॒पा॒नौ । \newline
54. वा ए॒ता वे॒तौ वै वा ए॒तौ । \newline
55. ए॒तौ प॑शू॒नाम् प॑शू॒ना मे॒ता वे॒तौ प॑शू॒नाम् । \newline
56. प॒शू॒नां ॅयद् यत् प॑शू॒नाम् प॑शू॒नां ॅयत् । \newline
57. यत् पृ॑षदा॒ज्यम् पृ॑षदा॒ज्यं ॅयद् यत् पृ॑षदा॒ज्यम् । \newline
58. पृ॒ष॒दा॒ज्यं ॅवा॑नस्प॒त्या वा॑नस्प॒त्याः पृ॑षदा॒ज्यम् पृ॑षदा॒ज्यं ॅवा॑नस्प॒त्याः । \newline
59. पृ॒ष॒दा॒ज्यमिति॑ पृषत् - आ॒ज्यम् । \newline
60. वा॒न॒स्प॒त्याः खलु॒ खलु॑ वानस्प॒त्या वा॑नस्प॒त्याः खलु॑ । \newline
61. खलु॒ वै वै खलु॒ खलु॒ वै । \newline

\textbf{Ghana Paata } \newline

1. अव॑से॒ सखा॑यः॒ सखा॒यो ऽव॒से ऽव॑से॒ सखा॒यो ऽन्वनु॒ सखा॒यो ऽव॒से ऽव॑से॒ सखा॒यो ऽनु॑ । \newline
2. सखा॒यो ऽन्वनु॒ सखा॑यः॒ सखा॒यो ऽनु॑ त्वा॒ त्वा ऽनु॒ सखा॑यः॒ सखा॒यो ऽनु॑ त्वा । \newline
3. अनु॑ त्वा॒ त्वा ऽन्वनु॑ त्वा मा॒ता मा॒ता त्वा ऽन्वनु॑ त्वा मा॒ता । \newline
4. त्वा॒ मा॒ता मा॒ता त्वा᳚ त्वा मा॒ता पि॒तरः॑ पि॒तरो॑ मा॒ता त्वा᳚ त्वा मा॒ता पि॒तरः॑ । \newline
5. मा॒ता पि॒तरः॑ पि॒तरो॑ मा॒ता मा॒ता पि॒तरो॑ मदन्तु मदन्तु पि॒तरो॑ मा॒ता मा॒ता पि॒तरो॑ मदन्तु । \newline
6. पि॒तरो॑ मदन्तु मदन्तु पि॒तरः॑ पि॒तरो॑ मद॒ न्त्वितीति॑ मदन्तु पि॒तरः॑ पि॒तरो॑ मद॒ न्त्विति॑ । \newline
7. म॒द॒ न्त्वितीति॑ मदन्तु मद॒ न्त्वित्या॑हा॒ हेति॑ मदन्तु मद॒ न्त्वित्या॑ह । \newline
8. इत्या॑हा॒हे तीत्या॒हा नु॑मत॒ मनु॑मत मा॒हे तीत्या॒हा नु॑मतम् । \newline
9. आ॒हा नु॑मत॒ मनु॑मत माहा॒हा नु॑मत मे॒वै वानु॑मत माहा॒हा नु॑मत मे॒व । \newline
10. अनु॑मत मे॒वैवा नु॑मत॒ मनु॑मत मे॒वैन॑ मेन मे॒वा नु॑मत॒ मनु॑मत मे॒वैन᳚म् । \newline
11. अनु॑मत॒मित्यनु॑ - म॒त॒म् । \newline
12. ए॒वैन॑ मेन मे॒वै वैन॑म् मा॒त्रा मा॒त्रैन॑ मे॒वै वैन॑म् मा॒त्रा । \newline
13. ए॒न॒म् मा॒त्रा मा॒त्रैन॑ मेनम् मा॒त्रा पि॒त्रा पि॒त्रा मा॒त्रैन॑ मेनम् मा॒त्रा पि॒त्रा । \newline
14. मा॒त्रा पि॒त्रा पि॒त्रा मा॒त्रा मा॒त्रा पि॒त्रा सु॑व॒र्गꣳ सु॑व॒र्गम् पि॒त्रा मा॒त्रा मा॒त्रा पि॒त्रा सु॑व॒र्गम् । \newline
15. पि॒त्रा सु॑व॒र्गꣳ सु॑व॒र्गम् पि॒त्रा पि॒त्रा सु॑व॒र्गम् ॅलो॒कम् ॅलो॒कꣳ सु॑व॒र्गम् पि॒त्रा पि॒त्रा सु॑व॒र्गम् ॅलो॒कम् । \newline
16. सु॒व॒र्गम् ॅलो॒कम् ॅलो॒कꣳ सु॑व॒र्गꣳ सु॑व॒र्गम् ॅलो॒कम् ग॑मयति गमयति लो॒कꣳ सु॑व॒र्गꣳ सु॑व॒र्गम् ॅलो॒कम् ग॑मयति । \newline
17. सु॒व॒र्गमिति॑ सुवः - गम् । \newline
18. लो॒कम् ग॑मयति गमयति लो॒कम् ॅलो॒कम् ग॑मय त्यर्द्ध॒र्चे᳚ ऽर्द्ध॒र्चे ग॑मयति लो॒कम् ॅलो॒कम् ग॑मय त्यर्द्ध॒र्चे । \newline
19. ग॒म॒य॒ त्य॒र्द्ध॒र्चे᳚ ऽर्द्ध॒र्चे ग॑मयति गमय त्यर्द्ध॒र्चे व॑साहो॒मं ॅव॑साहो॒म म॑र्द्ध॒र्चे ग॑मयति गमय त्यर्द्ध॒र्चे व॑साहो॒मम् । \newline
20. अ॒र्द्ध॒र्चे व॑साहो॒मं ॅव॑साहो॒म म॑र्द्ध॒र्चे᳚ ऽर्द्ध॒र्चे व॑साहो॒मम् जु॑होति जुहोति वसाहो॒म म॑र्द्ध॒र्चे᳚ ऽर्द्ध॒र्चे व॑साहो॒मम् जु॑होति । \newline
21. अ॒र्द्ध॒र्च इत्य॑र्द्ध - ऋ॒चे । \newline
22. व॒सा॒हो॒मम् जु॑होति जुहोति वसाहो॒मं ॅव॑साहो॒मम् जु॑हो त्य॒सा व॒सौ जु॑होति वसाहो॒मं ॅव॑साहो॒मम् जु॑हो त्य॒सौ । \newline
23. व॒सा॒हो॒ममिति॑ वसा - हो॒मम् । \newline
24. जु॒हो॒ त्य॒सा व॒सौ जु॑होति जुहो त्य॒सौ वै वा अ॒सौ जु॑होति जुहो त्य॒सौ वै । \newline
25. अ॒सौ वै वा अ॒सा व॒सौ वा अ॑र्द्ध॒र्चो᳚ ऽर्द्ध॒र्चो वा अ॒सा व॒सौ वा अ॑र्द्ध॒र्चः । \newline
26. वा अ॑र्द्ध॒र्चो᳚ ऽर्द्ध॒र्चो वै वा अ॑र्द्ध॒र्च इ॒य मि॒य म॑र्द्ध॒र्चो वै वा अ॑र्द्ध॒र्च इ॒यम् । \newline
27. अ॒र्द्ध॒र्च इ॒य मि॒य म॑र्द्ध॒र्चो᳚ ऽर्द्ध॒र्च इ॒य म॑र्द्ध॒र्चो᳚ ऽर्द्ध॒र्च इ॒य म॑र्द्ध॒र्चो᳚ ऽर्द्ध॒र्च इ॒य म॑र्द्ध॒र्चः । \newline
28. अ॒र्द्ध॒र्च इत्य॑र्द्ध - ऋ॒चः । \newline
29. इ॒य म॑र्द्ध॒र्चो᳚ ऽर्द्ध॒र्च इ॒य मि॒य म॑र्द्ध॒र्च इ॒मे इ॒मे अ॑र्द्ध॒र्च इ॒य मि॒य म॑र्द्ध॒र्च इ॒मे । \newline
30. अ॒र्द्ध॒र्च इ॒मे इ॒मे अ॑र्द्ध॒र्चो᳚ ऽर्द्ध॒र्च इ॒मे ए॒वैवेमे अ॑र्द्ध॒र्चो᳚ ऽर्द्ध॒र्च इ॒मे ए॒व । \newline
31. अ॒र्द्ध॒र्च इत्य॑र्द्ध - ऋ॒चः । \newline
32. इ॒मे ए॒वैवेमे इ॒मे ए॒व रसे॑न॒ रसे॑नै॒वेमे इ॒मे ए॒व रसे॑न । \newline
33. इ॒मे इती॒मे । \newline
34. ए॒व रसे॑न॒ रसे॑नै॒ वैव रसे॑नानक् त्यनक्ति॒ रसे॑नै॒ वैव रसे॑नानक्ति । \newline
35. रसे॑नानक् त्यनक्ति॒ रसे॑न॒ रसे॑नानक्ति॒ दिशो॒ दिशो॑ ऽनक्ति॒ रसे॑न॒ रसे॑नानक्ति॒ दिशः॑ । \newline
36. अ॒न॒क्ति॒ दिशो॒ दिशो॑ ऽनक्त्यनक्ति॒ दिशो॑ जुहोति जुहोति॒ दिशो॑ ऽनक्त्यनक्ति॒ दिशो॑ जुहोति । \newline
37. दिशो॑ जुहोति जुहोति॒ दिशो॒ दिशो॑ जुहोति॒ दिशो॒ दिशो॑ जुहोति॒ दिशो॒ दिशो॑ जुहोति॒ दिशः॑ । \newline
38. जु॒हो॒ति॒ दिशो॒ दिशो॑ जुहोति जुहोति॒ दिश॑ ए॒वैव दिशो॑ जुहोति जुहोति॒ दिश॑ ए॒व । \newline
39. दिश॑ ए॒वैव दिशो॒ दिश॑ ए॒व रसे॑न॒ रसे॑नै॒व दिशो॒ दिश॑ ए॒व रसे॑न । \newline
40. ए॒व रसे॑न॒ रसे॑ नै॒वैव रसे॑ना नक् त्यनक्ति॒ रसे॑नै॒ वैव रसे॑नानक्ति । \newline
41. रसे॑ना नक् त्यनक्ति॒ रसे॑न॒ रसे॑ना न॒क्त्यथो॒ अथो॑ अनक्ति॒ रसे॑न॒ रसे॑ना न॒क्त्यथो᳚ । \newline
42. अ॒न॒क् त्यथो॒ अथो॑ अनक् त्यन॒क् त्यथो॑ दि॒ग्भ्यो दि॒ग्भ्यो ऽथो॑ अनक् त्यन॒ क्त्यथो॑ दि॒ग्भ्यः । \newline
43. अथो॑ दि॒ग्भ्यो दि॒ग्भ्यो ऽथो॒ अथो॑ दि॒ग्भ्य ए॒वैव दि॒ग्भ्यो ऽथो॒ अथो॑ दि॒ग्भ्य ए॒व । \newline
44. अथो॒ इत्यथो᳚ । \newline
45. दि॒ग्भ्य ए॒वैव दि॒ग्भ्यो दि॒ग्भ्य ए॒वोर्ज॒ मूर्ज॑ मे॒व दि॒ग्भ्यो दि॒ग्भ्य ए॒वोर्ज᳚म् । \newline
46. दि॒ग्भ्य इति॑ दिक् - भ्यः । \newline
47. ए॒वोर्ज॒ मूर्ज॑ मे॒वै वोर्जꣳ॒॒ रसꣳ॒॒ रस॒ मूर्ज॑ मे॒वै वोर्जꣳ॒॒ रस᳚म् । \newline
48. ऊर्जꣳ॒॒ रसꣳ॒॒ रस॒ मूर्ज॒ मूर्जꣳ॒॒ रस॒ मवाव॒ रस॒ मूर्ज॒ मूर्जꣳ॒॒ रस॒ मव॑ । \newline
49. रस॒ मवाव॒ रसꣳ॒॒ रस॒ मव॑ रुन्धे रु॒न्धे ऽव॒ रसꣳ॒॒ रस॒ मव॑ रुन्धे । \newline
50. अव॑ रुन्धे रु॒न्धे ऽवाव॑ रुन्धे प्राणापा॒नौ प्रा॑णापा॒नौ रु॒न्धे ऽवाव॑ रुन्धे प्राणापा॒नौ । \newline
51. रु॒न्धे॒ प्रा॒णा॒पा॒नौ प्रा॑णापा॒नौ रु॑न्धे रुन्धे प्राणापा॒नौ वै वै प्रा॑णापा॒नौ रु॑न्धे रुन्धे प्राणापा॒नौ वै । \newline
52. प्रा॒णा॒पा॒नौ वै वै प्रा॑णापा॒नौ प्रा॑णापा॒नौ वा ए॒ता वे॒तौ वै प्रा॑णापा॒नौ प्रा॑णापा॒नौ वा ए॒तौ । \newline
53. प्रा॒णा॒पा॒नाविति॑ प्राण - अ॒पा॒नौ । \newline
54. वा ए॒ता वे॒तौ वै वा ए॒तौ प॑शू॒नाम् प॑शू॒ना मे॒तौ वै वा ए॒तौ प॑शू॒नाम् । \newline
55. ए॒तौ प॑शू॒नाम् प॑शू॒ना मे॒ता वे॒तौ प॑शू॒नां ॅयद् यत् प॑शू॒ना मे॒ता वे॒तौ प॑शू॒नां ॅयत् । \newline
56. प॒शू॒नां ॅयद् यत् प॑शू॒नाम् प॑शू॒नां ॅयत् पृ॑षदा॒ज्यम् पृ॑षदा॒ज्यं ॅयत् प॑शू॒नाम् प॑शू॒नां ॅयत् पृ॑षदा॒ज्यम् । \newline
57. यत् पृ॑षदा॒ज्यम् पृ॑षदा॒ज्यं ॅयद् यत् पृ॑षदा॒ज्यं ॅवा॑नस्प॒त्या वा॑नस्प॒त्याः पृ॑षदा॒ज्यं ॅयद् यत् पृ॑षदा॒ज्यं ॅवा॑नस्प॒त्याः । \newline
58. पृ॒ष॒दा॒ज्यं ॅवा॑नस्प॒त्या वा॑नस्प॒त्याः पृ॑षदा॒ज्यम् पृ॑षदा॒ज्यं ॅवा॑नस्प॒त्याः खलु॒ खलु॑ वानस्प॒त्याः पृ॑षदा॒ज्यम् पृ॑षदा॒ज्यं ॅवा॑नस्प॒त्याः खलु॑ । \newline
59. पृ॒ष॒दा॒ज्यमिति॑ पृषत् - आ॒ज्यम् । \newline
60. वा॒न॒स्प॒त्याः खलु॒ खलु॑ वानस्प॒त्या वा॑नस्प॒त्याः खलु॒ वै वै खलु॑ वानस्प॒त्या वा॑नस्प॒त्याः खलु॒ वै । \newline
61. खलु॒ वै वै खलु॒ खलु॒ वै दे॒वत॑या दे॒वत॑या॒ वै खलु॒ खलु॒ वै दे॒वत॑या । \newline
\pagebreak
\markright{ TS 6.3.11.4  \hfill https://www.vedavms.in \hfill}

\section{ TS 6.3.11.4 }

\textbf{TS 6.3.11.4 } \newline
\textbf{Samhita Paata} \newline

वै दे॒वत॑या प॒शवो॒ यत् पृ॑षदा॒ज्यस्यो॑-प॒हत्याऽऽ*ह॒ वन॒स्पत॒येऽनु॑ ब्रूहि॒ वन॒स्पत॑ये॒ प्रेष्येति॑ प्राणापा॒नावे॒व प॒शुषु॑ दधात्य॒न्यस्या᳚न्यस्य समव॒त्तꣳ स॒मव॑द्यति॒ तस्मा॒न्नाना॑रूपाः प॒शवो॑ यू॒ष्णोप॑ सिञ्चति॒ रसो॒ वा ए॒ष प॑शू॒नां ॅयद्यू रस॑मे॒व प॒शुषु॑ दधा॒तीडा॒मुप॑ ह्वयते प॒शवो॒ वा इडा॑ प॒शूने॒वोप॑ ह्वयते च॒तुरुप॑ ह्वयते॒- [  ] \newline

\textbf{Pada Paata} \newline

वै । दे॒वत॑या । प॒शवः॑ । यत् । पृ॒ष॒दा॒ज्यस्येति॑ पृषत् - आ॒ज्यस्य॑ । उ॒प॒हत्येत्यु॑प - हत्य॑ । आह॑ । वन॒स्पत॑ये । अन्विति॑ । ब्रू॒हि॒ । वन॒स्पत॑ये । प्रेति॑ । इ॒ष्य॒ । इति॑ । प्रा॒णा॒पा॒नाविति॑ प्राण - अ॒पा॒नौ । ए॒व । प॒शुषु॑ । द॒धा॒ति॒ । अ॒न्यस्या᳚न्य॒स्येत्य॒न्यस्य॑ - अ॒न्य॒स्य॒ । स॒म॒व॒त्तमिति॑ सं - अ॒व॒त्तम् । स॒मव॑द्य॒तीति॑ सं-अव॑द्यति । तस्मा᳚त् । नाना॑रूपा॒ इति॒ नाना᳚-रू॒पाः । प॒शवः॑ । यू॒ष्णा । उपेति॑ । सि॒ञ्च॒ति॒ । रसः॑ । वै । ए॒षः । प॒शू॒नाम् । यत् । यूः । रस᳚म् । ए॒व । प॒शुषु॑ । द॒धा॒ति॒ । इडा᳚म् । उपेति॑ । ह्व॒य॒ते॒ । प॒शवः॑ । वै । इडा᳚ । प॒शून् । ए॒व । उपेति॑ । ह्व॒य॒ते॒ । च॒तुः । उपेति॑ । ह्व॒य॒ते॒ ।  \newline


\textbf{Krama Paata} \newline

वै दे॒वत॑या । दे॒वत॑या प॒शवः॑ । प॒शवो॒ यत् । यत् पृ॑षदा॒ज्यस्य॑ । पृ॒ष॒दा॒ज्यस्यो॑प॒हत्य॑ । पृ॒ष॒दा॒जस्येति॑ पृषत् - आ॒ज्यस्य॑ । उ॒प॒हत्याह॑ । उ॒प॒हत्येत्यु॑प - हत्य॑ । आह॒ वन॒स्पत॑ये । वन॒स्पत॒येऽनु॑ । अनु॑ ब्रूहि । ब्रू॒हि॒ वन॒स्पत॑ये । वन॒स्पत॑ये॒ प्र । प्रेष्य॑ । इ॒ष्येति॑ । इति॑ प्राणापा॒नौ । प्रा॒णा॒पा॒नावे॒व । प्रा॒णा॒पा॒नाविति॑ प्राण - अ॒पा॒नौ । ए॒व प॒शुषु॑ । प॒शुषु॑ दधाति । द॒धा॒त्य॒न्यस्या᳚न्यस्य । अ॒न्यस्या᳚न्यस्य समव॒त्तम् । अ॒न्यस्या᳚न्य॒स्येत्य॒न्यस्य॑ - अ॒न्य॒स्य॒ । स॒म॒व॒त्तꣳ स॒मव॑द्यति । स॒म॒व॒त्तमिति॑ सम् - अ॒व॒त्तम् । स॒मव॑द्यति॒ तस्मा᳚त् । स॒मव॑द्य॒तीति॑ सम् - अव॑द्यति । तस्मा॒न् नाना॑रूपाः । नाना॑रूपाः प॒शवः॑ । नाना॑रूपा॒ इति॒ नाना᳚ - रू॒पाः॒ । प॒शवो॑ यू॒ष्णा । यू॒ष्णोप॑ । उप॑ सिञ्चति । सि॒ञ्च॒ति॒ रसः॑ । रसो॒ वै । वा ए॒षः । ए॒ष प॑शू॒नाम् । प॒शू॒नाम् ॅयत् । यद् यूः । यू रस᳚म् । रस॑मे॒व । ए॒व प॒शुषु॑ । प॒शुषु॑ दधाति । द॒धा॒तीडा᳚म् । इडा॒मुप॑ । उप॑ ह्वयते । ह्व॒य॒ते॒ प॒शवः॑ । प॒शवो॒ वै । वा इडा᳚ । इडा॑ प॒शून् । प॒शूने॒व । ए॒वोप॑ । उप॑ ह्वयते । ह्व॒य॒ते॒ च॒तुः । च॒तुरुप॑ । उप॑ ह्वयते । ह्व॒य॒ते॒ चतु॑ष्पादः \newline

\textbf{Jatai Paata} \newline

1. वै दे॒वत॑या दे॒वत॑या॒ वै वै दे॒वत॑या । \newline
2. दे॒वत॑या प॒शवः॑ प॒शवो॑ दे॒वत॑या दे॒वत॑या प॒शवः॑ । \newline
3. प॒शवो॒ यद् यत् प॒शवः॑ प॒शवो॒ यत् । \newline
4. यत् पृ॑षदा॒ज्यस्य॑ पृषदा॒ज्यस्य॒ यद् यत् पृ॑षदा॒ज्यस्य॑ । \newline
5. पृ॒ष॒दा॒ज्य स्यो॑प॒हत्यो॑ प॒हत्य॑ पृषदा॒ज्यस्य॑ पृषदा॒ज्य स्यो॑प॒हत्य॑ । \newline
6. पृ॒ष॒दा॒ज्यस्येति॑ पृषत् - आ॒ज्यस्य॑ । \newline
7. उ॒प॒हत्या हाहो॑प॒ह त्यो॑प॒हत्याह॑ । \newline
8. उ॒प॒हत्येत्यु॑प - हत्य॑ । \newline
9. आह॒ वन॒स्पत॑ये॒ वन॒स्पत॑य॒ आहाह॒ वन॒स्पत॑ये । \newline
10. वन॒स्पत॒ये ऽन्वनु॒ वन॒स्पत॑ये॒ वन॒स्पत॒ये ऽनु॑ । \newline
11. अनु॑ ब्रूहि ब्रू॒ह्यन् वनु॑ ब्रूहि । \newline
12. ब्रू॒हि॒ वन॒स्पत॑ये॒ वन॒स्पत॑ये ब्रूहि ब्रूहि॒ वन॒स्पत॑ये । \newline
13. वन॒स्पत॑ये॒ प्र प्र वण॒स्पत॑ये॒ वन॒स्पत॑ये॒ प्र । \newline
14. प्रेष्ये᳚ ष्य॒ प्र प्रेष्य॑ । \newline
15. इ॒ष्ये तीती᳚ष्ये॒ ष्येति॑ । \newline
16. इति॑ प्राणापा॒नौ प्रा॑णापा॒ना वितीति॑ प्राणापा॒नौ । \newline
17. प्रा॒णा॒पा॒ना वे॒वैव प्रा॑णापा॒नौ प्रा॑णापा॒ना वे॒व । \newline
18. प्रा॒णा॒पा॒नाविति॑ प्राण - अ॒पा॒नौ । \newline
19. ए॒व प॒शुषु॑ प॒शु ष्वे॒वैव प॒शुषु॑ । \newline
20. प॒शुषु॑ दधाति दधाति प॒शुषु॑ प॒शुषु॑ दधाति । \newline
21. द॒धा॒ त्य॒न्यस्या᳚न्यस्या॒ न्यस्या᳚न्यस्य दधाति दधा त्य॒न्यस्या᳚न्यस्य । \newline
22. अ॒न्यस्या᳚न्यस्य समव॒त्तꣳ स॑मव॒त्त म॒न्यस्या᳚न्यस्या॒ न्यस्या᳚न्यस्य समव॒त्तम् । \newline
23. अ॒न्यस्या᳚न्य॒स्येत्य॒न्यस्य॑ - अ॒न्य॒स्य॒ । \newline
24. स॒म॒व॒त्तꣳ स॒मव॑द्यति स॒मव॑द्यति समव॒त्तꣳ स॑मव॒त्तꣳ स॒मव॑द्यति । \newline
25. स॒म॒व॒त्तमिति॑ सं - अ॒व॒त्तम् । \newline
26. स॒मव॑द्यति॒ तस्मा॒त् तस्मा᳚थ् स॒मव॑द्यति स॒मव॑द्यति॒ तस्मा᳚त् । \newline
27. स॒मव॑द्य॒तीति॑ सं - अव॑द्यति । \newline
28. तस्मा॒न् नाना॑रूपा॒ नाना॑रूपा॒ स्तस्मा॒त् तस्मा॒न् नाना॑रूपाः । \newline
29. नाना॑रूपाः प॒शवः॑ प॒शवो॒ नाना॑रूपा॒ नाना॑रूपाः प॒शवः॑ । \newline
30. नाना॑रूपा॒ इति॒ नाना᳚ - रू॒पाः॒ । \newline
31. प॒शवो॑ यू॒ष्णा यू॒ष्णा प॒शवः॑ प॒शवो॑ यू॒ष्णा । \newline
32. यू॒ष्णोपोप॑ यू॒ष्णा यू॒ष्णोप॑ । \newline
33. उप॑ सिञ्चति सिञ्च॒ त्युपोप॑ सिञ्चति । \newline
34. सि॒ञ्च॒ति॒ रसो॒ रसः॑ सिञ्चति सिञ्चति॒ रसः॑ । \newline
35. रसो॒ वै वै रसो॒ रसो॒ वै । \newline
36. वा ए॒ष ए॒ष वै वा ए॒षः । \newline
37. ए॒ष प॑शू॒नाम् प॑शू॒ना मे॒ष ए॒ष प॑शू॒नाम् । \newline
38. प॒शू॒नां ॅयद् यत् प॑शू॒नाम् प॑शू॒नां ॅयत् । \newline
39. यद् यूर् यूर् यद् यद् यूः । \newline
40. यू रसꣳ॒॒ रसं॒ ॅयूर् यू रस᳚म् । \newline
41. रस॑ मे॒वैव रसꣳ॒॒ रस॑ मे॒व । \newline
42. ए॒व प॒शुषु॑ प॒शु ष्वे॒वैव प॒शुषु॑ । \newline
43. प॒शुषु॑ दधाति दधाति प॒शुषु॑ प॒शुषु॑ दधाति । \newline
44. द॒धा॒तीडा॒ मिडा᳚म् दधाति दधा॒तीडा᳚म् । \newline
45. इडा॒ मुपोपेडा॒ मिडा॒ मुप॑ । \newline
46. उप॑ ह्वयते ह्वयत॒ उपोप॑ ह्वयते । \newline
47. ह्व॒य॒ते॒ प॒शवः॑ प॒शवो᳚ ह्वयते ह्वयते प॒शवः॑ । \newline
48. प॒शवो॒ वै वै प॒शवः॑ प॒शवो॒ वै । \newline
49. वा इडेडा॒ वै वा इडा᳚ । \newline
50. इडा॑ प॒शून् प॒शूनि डेडा॑ प॒शून् । \newline
51. प॒शूने॒ वैव प॒शून् प॒शूने॒व । \newline
52. ए॒वोपो पै॒वै वोप॑ । \newline
53. उप॑ ह्वयते ह्वयत॒ उपोप॑ ह्वयते । \newline
54. ह्व॒य॒ते॒ च॒तु श्च॒तुर् ह्व॑यते ह्वयते च॒तुः । \newline
55. च॒तु रुपोप॑ च॒तु श्च॒तु रुप॑ । \newline
56. उप॑ ह्वयते ह्वयत॒ उपोप॑ ह्वयते । \newline
57. ह्व॒य॒ते॒ चतु॑ष्पा द॒श्चतु॑ष्पादो ह्वयते ह्वयते॒ चतु॑ष्पादः । \newline

\textbf{Ghana Paata } \newline

1. वै दे॒वत॑या दे॒वत॑या॒ वै वै दे॒वत॑या प॒शवः॑ प॒शवो॑ दे॒वत॑या॒ वै वै दे॒वत॑या प॒शवः॑ । \newline
2. दे॒वत॑या प॒शवः॑ प॒शवो॑ दे॒वत॑या दे॒वत॑या प॒शवो॒ यद् यत् प॒शवो॑ दे॒वत॑या दे॒वत॑या प॒शवो॒ यत् । \newline
3. प॒शवो॒ यद् यत् प॒शवः॑ प॒शवो॒ यत् पृ॑षदा॒ज्यस्य॑ पृषदा॒ज्यस्य॒ यत् प॒शवः॑ प॒शवो॒ यत् पृ॑षदा॒ज्यस्य॑ । \newline
4. यत् पृ॑षदा॒ज्यस्य॑ पृषदा॒ज्यस्य॒ यद् यत् पृ॑षदा॒ज्य स्यो॑प॒ह त्यो॑प॒हत्य॑ पृषदा॒ज्यस्य॒ यद् यत् पृ॑षदा॒ज्य स्यो॑प॒हत्य॑ । \newline
5. पृ॒ष॒दा॒ज्य स्यो॑प॒ह त्यो॑प॒हत्य॑ पृषदा॒ज्यस्य॑ पृषदा॒ज्य स्यो॑प॒हत्या हाहो॑ प॒हत्य॑ पृषदा॒ज्यस्य॑ पृषदा॒ज्य स्यो॑प॒हत्याह॑ । \newline
6. पृ॒ष॒दा॒ज्यस्येति॑ पृषत् - आ॒ज्यस्य॑ । \newline
7. उ॒प॒हत्या हाहो॑ प॒हत्यो॑ प॒हत्याह॒ वन॒स्पत॑ये॒ वन॒स्पत॑य॒ आहो॑ प॒हत्यो॑ प॒हत्याह॒ वन॒स्पत॑ये । \newline
8. उ॒प॒हत्येत्यु॑प - हत्य॑ । \newline
9. आह॒ वन॒स्पत॑ये॒ वन॒स्पत॑य॒ आहाह॒ वन॒स्पत॒ये ऽन्वनु॒ वन॒स्पत॑य॒ आहाह॒ वन॒स्पत॒ये ऽनु॑ । \newline
10. वन॒स्पत॒ये ऽन्वनु॒ वन॒स्पत॑ये॒ वन॒स्पत॒ये ऽनु॑ ब्रूहि ब्रू॒ह्यनु॒ वन॒स्पत॑ये॒ वन॒स्पत॒ये ऽनु॑ ब्रूहि । \newline
11. अनु॑ ब्रूहि ब्रू॒ह्यन् वनु॑ ब्रूहि॒ वन॒स्पत॑ये॒ वन॒स्पत॑ये ब्रू॒ह्यन् वनु॑ ब्रूहि॒ वन॒स्पत॑ये । \newline
12. ब्रू॒हि॒ वन॒स्पत॑ये॒ वन॒स्पत॑ये ब्रूहि ब्रूहि॒ वन॒स्पत॑ये॒ प्र प्र वण॒स्पत॑ये ब्रूहि ब्रूहि॒ वन॒स्पत॑ये॒ प्र । \newline
13. वन॒स्पत॑ये॒ प्र प्र वण॒स्पत॑ये॒ वन॒स्पत॑ये॒ प्रेष्ये᳚ ष्य॒ प्र वण॒स्पत॑ये॒ वन॒स्पत॑ये॒ प्रेष्य॑ । \newline
14. प्रेष्ये᳚ ष्य॒ प्र प्रेष्ये तीती᳚ष्य॒ प्र प्रेष्येति॑ । \newline
15. इ॒ष्ये तीती᳚ष्ये॒ ष्येति॑ प्राणापा॒नौ प्रा॑णापा॒ना विती᳚ष्ये॒ ष्येति॑ प्राणापा॒नौ । \newline
16. इति॑ प्राणापा॒नौ प्रा॑णापा॒ना वितीति॑ प्राणापा॒ना वे॒वैव प्रा॑णापा॒ना वितीति॑ प्राणापा॒ना वे॒व । \newline
17. प्रा॒णा॒पा॒ना वे॒वैव प्रा॑णापा॒नौ प्रा॑णापा॒ना वे॒व प॒शुषु॑ प॒शुष्वे॒व प्रा॑णापा॒नौ प्रा॑णापा॒ना वे॒व प॒शुषु॑ । \newline
18. प्रा॒णा॒पा॒नाविति॑ प्राण - अ॒पा॒नौ । \newline
19. ए॒व प॒शुषु॑ प॒शुष्वे॒ वैव प॒शुषु॑ दधाति दधाति प॒शुष्वे॒ वैव प॒शुषु॑ दधाति । \newline
20. प॒शुषु॑ दधाति दधाति प॒शुषु॑ प॒शुषु॑ दधा त्य॒न्यस्या᳚न्यस्या॒ न्यस्या᳚न्यस्य दधाति प॒शुषु॑ प॒शुषु॑ दधा त्य॒न्यस्या᳚न्यस्य । \newline
21. द॒धा॒ त्य॒न्यस्या᳚न्यस्या॒ न्यस्या᳚न्यस्य दधाति दधा त्य॒न्यस्या᳚न्यस्य समव॒त्तꣳ स॑मव॒त्त म॒न्यस्या᳚न्यस्य दधाति दधा त्य॒न्यस्या᳚न्यस्य समव॒त्तम् । \newline
22. अ॒न्यस्या᳚न्यस्य समव॒त्तꣳ स॑मव॒त्त म॒न्यस्या᳚न्यस्या॒ न्यस्या᳚न्यस्य समव॒त्तꣳ स॒मव॑द्यति स॒मव॑द्यति समव॒त्त म॒न्यस्या᳚न्यस्या॒ न्यस्या᳚न्यस्य समव॒त्तꣳ स॒मव॑द्यति । \newline
23. अ॒न्यस्या᳚न्य॒स्येत्य॒न्यस्य॑ - अ॒न्य॒स्य॒ । \newline
24. स॒म॒व॒त्तꣳ स॒मव॑द्यति स॒मव॑द्यति समव॒त्तꣳ स॑मव॒त्तꣳ स॒मव॑द्यति॒ तस्मा॒त् तस्मा᳚थ् स॒मव॑द्यति समव॒त्तꣳ स॑मव॒त्तꣳ स॒मव॑द्यति॒ तस्मा᳚त् । \newline
25. स॒म॒व॒त्तमिति॑ सं - अ॒व॒त्तम् । \newline
26. स॒मव॑द्यति॒ तस्मा॒त् तस्मा᳚थ् स॒मव॑द्यति स॒मव॑द्यति॒ तस्मा॒न् नाना॑रूपा॒ नाना॑रूपा॒ स्तस्मा᳚थ् स॒मव॑द्यति स॒मव॑द्यति॒ तस्मा॒न् नाना॑रूपाः । \newline
27. स॒मव॑द्य॒तीति॑ सं - अव॑द्यति । \newline
28. तस्मा॒न् नाना॑रूपा॒ नाना॑रूपा॒ स्तस्मा॒त् तस्मा॒न् नाना॑रूपाः प॒शवः॑ प॒शवो॒ नाना॑रूपा॒ स्तस्मा॒त् तस्मा॒न् नाना॑रूपाः प॒शवः॑ । \newline
29. नाना॑रूपाः प॒शवः॑ प॒शवो॒ नाना॑रूपा॒ नाना॑रूपाः प॒शवो॑ यू॒ष्णा यू॒ष्णा प॒शवो॒ नाना॑रूपा॒ नाना॑रूपाः प॒शवो॑ यू॒ष्णा । \newline
30. नाना॑रूपा॒ इति॒ नाना᳚ - रू॒पाः॒ । \newline
31. प॒शवो॑ यू॒ष्णा यू॒ष्णा प॒शवः॑ प॒शवो॑ यू॒ष्णोपोप॑ यू॒ष्णा प॒शवः॑ प॒शवो॑ यू॒ष्णोप॑ । \newline
32. यू॒ष्णोपोप॑ यू॒ष्णा यू॒ष्णोप॑ सिञ्चति सिञ्च॒ त्युप॑ यू॒ष्णा यू॒ष्णोप॑ सिञ्चति । \newline
33. उप॑ सिञ्चति सिञ्च॒ त्युपोप॑ सिञ्चति॒ रसो॒ रसः॑ सिञ्च॒ त्युपोप॑ सिञ्चति॒ रसः॑ । \newline
34. सि॒ञ्च॒ति॒ रसो॒ रसः॑ सिञ्चति सिञ्चति॒ रसो॒ वै वै रसः॑ सिञ्चति सिञ्चति॒ रसो॒ वै । \newline
35. रसो॒ वै वै रसो॒ रसो॒ वा ए॒ष ए॒ष वै रसो॒ रसो॒ वा ए॒षः । \newline
36. वा ए॒ष ए॒ष वै वा ए॒ष प॑शू॒नाम् प॑शू॒ना मे॒ष वै वा ए॒ष प॑शू॒नाम् । \newline
37. ए॒ष प॑शू॒नाम् प॑शू॒ना मे॒ष ए॒ष प॑शू॒नां ॅयद् यत् प॑शू॒ना मे॒ष ए॒ष प॑शू॒नां ॅयत् । \newline
38. प॒शू॒नां ॅयद् यत् प॑शू॒नाम् प॑शू॒नां ॅयद् यूर् यूर् यत् प॑शू॒नाम् प॑शू॒नां ॅयद् यूः । \newline
39. यद् यूर् यूर् यद् यद् यू रसꣳ॒॒ रसं॒ ॅयूर् यद् यद् यू रस᳚म् । \newline
40. यू रसꣳ॒॒ रसं॒ ॅयूर् यू रस॑ मे॒वैव रसं॒ ॅयूर् यू रस॑ मे॒व । \newline
41. रस॑ मे॒वैव रसꣳ॒॒ रस॑ मे॒व प॒शुषु॑ प॒शुष्वे॒व रसꣳ॒॒ रस॑ मे॒व प॒शुषु॑ । \newline
42. ए॒व प॒शुषु॑ प॒शुष्वे॒ वैव प॒शुषु॑ दधाति दधाति प॒शुष्वे॒ वैव प॒शुषु॑ दधाति । \newline
43. प॒शुषु॑ दधाति दधाति प॒शुषु॑ प॒शुषु॑ दधा॒ तीडा॒ मिडा᳚म् दधाति प॒शुषु॑ प॒शुषु॑ दधा॒ तीडा᳚म् । \newline
44. द॒धा॒ तीडा॒ मिडा᳚म् दधाति दधा॒ तीडा॒ मुपोपे डा᳚म् दधाति दधा॒ तीडा॒ मुप॑ । \newline
45. इडा॒ मुपोपे डा॒ मिडा॒ मुप॑ ह्वयते ह्वयत॒ उपेडा॒ मिडा॒ मुप॑ ह्वयते । \newline
46. उप॑ ह्वयते ह्वयत॒ उपोप॑ ह्वयते प॒शवः॑ प॒शवो᳚ ह्वयत॒ उपोप॑ ह्वयते प॒शवः॑ । \newline
47. ह्व॒य॒ते॒ प॒शवः॑ प॒शवो᳚ ह्वयते ह्वयते प॒शवो॒ वै वै प॒शवो᳚ ह्वयते ह्वयते प॒शवो॒ वै । \newline
48. प॒शवो॒ वै वै प॒शवः॑ प॒शवो॒ वा इडेडा॒ वै प॒शवः॑ प॒शवो॒ वा इडा᳚ । \newline
49. वा इडेडा॒ वै वा इडा॑ प॒शून् प॒शू निडा॒ वै वा इडा॑ प॒शून् । \newline
50. इडा॑ प॒शून् प॒शू निडेडा॑ प॒शूने॒ वैव प॒शू निडेडा॑ प॒शूने॒व । \newline
51. प॒शू ने॒वैव प॒शून् प॒शू ने॒वोपोपै॒व प॒शून् प॒शूने॒वोप॑ । \newline
52. ए॒वोपो पै॒वै वोप॑ ह्वयते ह्वयत॒ उपै॒वै वोप॑ ह्वयते । \newline
53. उप॑ ह्वयते ह्वयत॒ उपोप॑ ह्वयते च॒तु श्च॒तुर् ह्व॑यत॒ उपोप॑ ह्वयते च॒तुः । \newline
54. ह्व॒य॒ते॒ च॒तु श्च॒तुर् ह्व॑यते ह्वयते च॒तु रुपोप॑ च॒तुर् ह्व॑यते ह्वयते च॒तु रुप॑ । \newline
55. च॒तु रुपोप॑ च॒तु श्च॒तु रुप॑ ह्वयते ह्वयत॒ उप॑ च॒तु श्च॒तु रुप॑ ह्वयते । \newline
56. उप॑ ह्वयते ह्वयत॒ उपोप॑ ह्वयते॒ चतु॑ष्पाद॒ श्चतु॑ष्पादो ह्वयत॒ उपोप॑ ह्वयते॒ चतु॑ष्पादः । \newline
57. ह्व॒य॒ते॒ चतु॑ष्पाद॒ श्चतु॑ष्पादो ह्वयते ह्वयते॒ चतु॑ष्पादो॒ हि हि चतु॑ष्पादो ह्वयते ह्वयते॒ चतु॑ष्पादो॒ हि । \newline
\pagebreak
\markright{ TS 6.3.11.5  \hfill https://www.vedavms.in \hfill}

\section{ TS 6.3.11.5 }

\textbf{TS 6.3.11.5 } \newline
\textbf{Samhita Paata} \newline

चतु॑ष्पादो॒ हि प॒शवो॒ यं का॒मये॑ता प॒शुः स्या॒दित्य॑मे॒दस्कं॒ तस्मा॒ आ द॑द्ध्या॒न्मेदो॑रूपा॒ वै प॒शवो॑ रू॒पेणै॒वैनं॑ प॒शुभ्यो॒ निर्भ॑जत्यप॒शुरे॒व भ॑वति॒ यं का॒मये॑त पशु॒मान्थ् स्या॒दिति॒ मेद॑स्व॒त् तस्मा॒ आ द॑द्ध्या॒न्मेदो॑रूपा॒ वै प॒शवो॑ रू॒पेणै॒वास्मै॑ प॒शूनव॑ रुन्धे पशु॒माने॒व भ॑वति प्र॒जाप॑तिर्य॒ज्ञ्म॑सृजत॒ स आज्यं॑- [  ] \newline

\textbf{Pada Paata} \newline

चतु॑ष्पाद॒ इति॒ चतुः॑ - पा॒दः॒ । हि । प॒शवः॑ । यम् । का॒मये॑त । अ॒प॒शुः । स्या॒त् । इति॑ । अ॒मे॒दस्क॒मित्य॑मे॒दः-क॒म् । तस्मै᳚ । एति॑ । द॒द्ध्या॒त् । मेदो॑रूपा॒ इति॒ मेदः॑-रू॒पाः॒ । वै । प॒शवः॑ । रू॒पेण॑ । ए॒व । ए॒न॒म् । प॒शुभ्य॒ इति॑ प॒शु-भ्यः॒ । निरिति॑ । भ॒ज॒ति॒ । अ॒प॒शुः । ए॒व । भ॒व॒ति॒ । यम् । का॒मये॑त । प॒शु॒मानिति॑ पशु - मान् । स्या॒त् । इति॑ । मेद॑स्वत् । तस्मै᳚ । एति॑ । द॒द्ध्या॒त् । मेदो॑रूपा॒ इति॒ मेदः॑ - रू॒पाः॒ । वै । प॒शवः॑ । रू॒पेण॑ । ए॒व । अ॒स्मै॒ । प॒शून् । अवेति॑ । रु॒न्धे॒ । प॒शु॒मानिति॑ पशु - मान् । ए॒व । भ॒व॒ति॒ । प्र॒जाप॑ति॒रिति॑ प्र॒जा-प॒तिः॒ । य॒ज्ञ्म् । अ॒सृ॒ज॒त॒ । सः । आज्यं᳚ ।  \newline


\textbf{Krama Paata} \newline

चतु॑ष्पादो॒ हि । चतु॑ष्पाद॒ इति॒ चतुः॑ - पा॒दः॒ । हि प॒शवः॑ । प॒शवो॒ यम् । यम् का॒मये॑त । 
का॒मये॑ताप॒शुः । अ॒प॒शुः स्या᳚त् । स्या॒दिति॑ । इत्य॑मे॒दस्क᳚म् । अ॒मे॒दस्क॒म् तस्मै᳚ । अ॒मे॒दस्क॒मित्य॑मे॒दः - क॒म् । तस्मा॒ आ । आ द॑द्ध्यात् । द॒द्ध्या॒न् मेदो॑रूपाः । मेदो॑रूपा॒ वै । मेदो॑रूपा॒ इति॒ मेदः॑ - रू॒पाः॒ । वै प॒शवः॑ । प॒शवो॑ रू॒पेण॑ । रू॒पेणै॒व । ए॒वैन᳚म् । ए॒न॒म् प॒शुभ्यः॑ । प॒शुभ्यो॒ निः । प॒शुभ्य॒ इति॑ प॒शु - भ्यः॒ । निर् भ॑जति । भ॒ज॒त्य॒प॒शुः । अ॒प॒शुरे॒व । ए॒व भ॑वति । भ॒व॒ति॒ यम् । यम् का॒मये॑त । का॒मये॑त पशु॒मान् । प॒शु॒मान्थ् स्या᳚त् । प॒शु॒मानिति॑ पशु - मान् । स्या॒दिति॑ । इति॒ मेद॑स्वत् । मेद॑स्व॒त् तस्मै᳚ । तस्मा॒ आ । आ द॑द्ध्यात् । द॒द्ध्या॒न् मेदो॑रूपाः । मेदो॑रूपा॒ वै । मेदो॑रूपा॒ इति॒ मेदः॑ - रू॒पाः॒ । वै प॒शवः॑ । प॒शवो॑ रू॒पेण॑ । रू॒पेणै॒व । ए॒वास्मै᳚ । अ॒स्मै॒ प॒शून् । प॒शूनव॑ । अव॑ रुन्धे । रु॒न्धे॒ प॒शु॒मान् । प॒शु॒माने॒व । प॒शु॒मानिति॑ पशु - मान् । ए॒व भ॑वति । भ॒व॒ति॒ प्र॒जाप॑तिः । प्र॒जाप॑तिर् य॒ज्ञ्म् । प्र॒जाप॑ति॒रिति॑ प्र॒जा - प॒तिः॒ । य॒ज्ञ्म॑सृजत । अ॒सृ॒ज॒त॒ सः । स आज्य᳚म् । आज्य॑म् पु॒रस्ता᳚त् \newline

\textbf{Jatai Paata} \newline

1. चतु॑ष्पादो॒ हि हि चतु॑ष्पाद॒ श्चतु॑ष्पादो॒ हि । \newline
2. चतु॑ष्पाद॒ इति॒ चतुः॑ - पा॒दः॒ । \newline
3. हि प॒शवः॑ प॒शवो॒ हि हि प॒शवः॑ । \newline
4. प॒शवो॒ यं ॅयम् प॒शवः॑ प॒शवो॒ यम् । \newline
5. यम् का॒मये॑त का॒मये॑त॒ यं ॅयम् का॒मये॑त । \newline
6. का॒मये॑ता प॒शु र॑प॒शुः का॒मये॑त का॒मये॑ता प॒शुः । \newline
7. अ॒प॒शुः स्या᳚थ् स्या दप॒शु र॑प॒शुः स्या᳚त् । \newline
8. स्या॒दि तीति॑ स्याथ् स्या॒ दिति॑ । \newline
9. इत्य॑ मे॒दस्क॑ ममे॒दस्क॒ मिती त्य॑मे॒दस्क᳚म् । \newline
10. अ॒मे॒दस्क॒म् तस्मै॒ तस्मा॑ अमे॒दस्क॑ ममे॒दस्क॒म् तस्मै᳚ । \newline
11. अ॒मे॒दस्क॒मित्य॑मे॒दः - क॒म् । \newline
12. तस्मा॒ आ तस्मै॒ तस्मा॒ आ । \newline
13. आ द॑द्ध्याद् दद्ध्या॒दा द॑द्ध्यात् । \newline
14. द॒द्ध्या॒न् मेदो॑रूपा॒ मेदो॑रूपा दद्ध्याद् दद्ध्या॒न् मेदो॑रूपाः । \newline
15. मेदो॑रूपा॒ वै वै मेदो॑रूपा॒ मेदो॑रूपा॒ वै । \newline
16. मेदो॑रूपा॒ इति॒ मेदः॑ - रू॒पाः॒ । \newline
17. वै प॒शवः॑ प॒शवो॒ वै वै प॒शवः॑ । \newline
18. प॒शवो॑ रू॒पेण॑ रू॒पेण॑ प॒शवः॑ प॒शवो॑ रू॒पेण॑ । \newline
19. रू॒पेणै॒ वैव रू॒पेण॑ रू॒पेणै॒व । \newline
20. ए॒वैन॑ मेन मे॒वै वैन᳚म् । \newline
21. ए॒न॒म् प॒शुभ्यः॑ प॒शुभ्य॑ एन मेनम् प॒शुभ्यः॑ । \newline
22. प॒शुभ्यो॒ निर् णिष् प॒शुभ्यः॑ प॒शुभ्यो॒ निः । \newline
23. प॒शुभ्य॒ इति॑ प॒शु - भ्यः॒ । \newline
24. निर् भ॑जति भजति॒ निर् णिर् भ॑जति । \newline
25. भ॒ज॒ त्य॒प॒शु र॑प॒शुर् भ॑जति भज त्यप॒शुः । \newline
26. अ॒प॒शु रे॒वै वाप॒शु र॑प॒शु रे॒व । \newline
27. ए॒व भ॑वति भव त्ये॒वैव भ॑वति । \newline
28. भ॒व॒ति॒ यं ॅयम् भ॑वति भवति॒ यम् । \newline
29. यम् का॒मये॑त का॒मये॑त॒ यं ॅयम् का॒मये॑त । \newline
30. का॒मये॑त पशु॒मान् प॑शु॒मान् का॒मये॑त का॒मये॑त पशु॒मान् । \newline
31. प॒शु॒मान् थ्स्या᳚थ् स्यात् पशु॒मान् प॑शु॒मान् थ्स्या᳚त् । \newline
32. प॒शु॒मानिति॑ पशु - मान् । \newline
33. स्या॒दि तीति॑ स्याथ् स्या॒ दिति॑ । \newline
34. इति॒ मेद॑स्व॒न् मेद॑स्व॒ दितीति॒ मेद॑स्वत् । \newline
35. मेद॑स्व॒त् तस्मै॒ तस्मै॒ मेद॑स्व॒न् मेद॑स्व॒त् तस्मै᳚ । \newline
36. तस्मा॒ आ तस्मै॒ तस्मा॒ आ । \newline
37. आ द॑द्ध्याद् दद्ध्या॒दा द॑द्ध्यात् । \newline
38. द॒द्ध्या॒न् मेदो॑रूपा॒ मेदो॑रूपा दद्ध्याद् दद्ध्या॒न् मेदो॑रूपाः । \newline
39. मेदो॑रूपा॒ वै वै मेदो॑रूपा॒ मेदो॑रूपा॒ वै । \newline
40. मेदो॑रूपा॒ इति॒ मेदः॑ - रू॒पाः॒ । \newline
41. वै प॒शवः॑ प॒शवो॒ वै वै प॒शवः॑ । \newline
42. प॒शवो॑ रू॒पेण॑ रू॒पेण॑ प॒शवः॑ प॒शवो॑ रू॒पेण॑ । \newline
43. रू॒पेणै॒ वैव रू॒पेण॑ रू॒पे णै॒व । \newline
44. ए॒वास्मा॑ अस्मा ए॒वै वास्मै᳚ । \newline
45. अ॒स्मै॒ प॒शून् प॒शू न॑स्मा अस्मै प॒शून् । \newline
46. प॒शून वाव॑ प॒शून् प॒शूनव॑ । \newline
47. अव॑ रुन्धे रु॒न्धे ऽवाव॑ रुन्धे । \newline
48. रु॒न्धे॒ प॒शु॒मान् प॑शु॒मान् रु॑न्धे रुन्धे पशु॒मान् । \newline
49. प॒शु॒मा ने॒वैव प॑शु॒मान् प॑शु॒माने॒व । \newline
50. प॒शु॒मानिति॑ पशु - मान् । \newline
51. ए॒व भ॑वति भव त्ये॒वैव भ॑वति । \newline
52. भ॒व॒ति॒ प्र॒जाप॑तिः प्र॒जाप॑तिर् भवति भवति प्र॒जाप॑तिः । \newline
53. प्र॒जाप॑तिर् य॒ज्ञ्ं ॅय॒ज्ञ्म् प्र॒जाप॑तिः प्र॒जाप॑तिर् य॒ज्ञ्म् । \newline
54. प्र॒जाप॑ति॒रिति॑ प्र॒जा - प॒तिः॒ । \newline
55. य॒ज्ञ् म॑सृजता सृजत य॒ज्ञ्ं ॅय॒ज्ञ् म॑सृजत । \newline
56. अ॒सृ॒ज॒त॒ स सो॑ ऽसृजता सृजत॒ सः । \newline
57. स आज्य॒ माज्यꣳ॒॒ स स आज्यं᳚ । \newline
58. आज्यं॑ पु॒रस्ता᳚त् पु॒रस्ता॒ दाज्य॒ माज्यं॑ पु॒रस्ता᳚त् । \newline

\textbf{Ghana Paata } \newline

1. चतु॑ष्पादो॒ हि हि चतु॑ष्पाद॒ श्चतु॑ष्पादो॒ हि प॒शवः॑ प॒शवो॒ हि चतु॑ष्पाद॒ श्चतु॑ष्पादो॒ हि प॒शवः॑ । \newline
2. चतु॑ष्पाद॒ इति॒ चतुः॑ - पा॒दः॒ । \newline
3. हि प॒शवः॑ प॒शवो॒ हि हि प॒शवो॒ यं ॅयम् प॒शवो॒ हि हि प॒शवो॒ यम् । \newline
4. प॒शवो॒ यं ॅयम् प॒शवः॑ प॒शवो॒ यम् का॒मये॑त का॒मये॑त॒ यम् प॒शवः॑ प॒शवो॒ यम् का॒मये॑त । \newline
5. यम् का॒मये॑त का॒मये॑त॒ यं ॅयम् का॒मये॑ता प॒शु र॑प॒शुः का॒मये॑त॒ यं ॅयम् का॒मये॑ता प॒शुः । \newline
6. का॒मये॑ता प॒शु र॑प॒शुः का॒मये॑त का॒मये॑ता प॒शुः स्या᳚थ् स्या दप॒शुः का॒मये॑त का॒मये॑ता प॒शुः स्या᳚त् । \newline
7. अ॒प॒शुः स्या᳚थ् स्या दप॒शु र॑प॒शुः स्या॒दि तीति॑ स्या दप॒शु र॑प॒शुः स्या॒ दिति॑ । \newline
8. स्या॒दि तीति॑ स्याथ् स्या॒ दित्य॑मे॒दस्क॑ ममे॒दस्क॒ मिति॑ स्याथ् स्या॒ दित्य॑मे॒दस्क᳚म् । \newline
9. इत्य॑मे॒दस्क॑ ममे॒दस्क॒ मिती त्य॑मे॒दस्क॒म् तस्मै॒ तस्मा॑ अमे॒दस्क॒ मिती त्य॑मे॒दस्क॒म् तस्मै᳚ । \newline
10. अ॒मे॒दस्क॒म् तस्मै॒ तस्मा॑ अमे॒दस्क॑ ममे॒दस्क॒म् तस्मा॒ आ तस्मा॑ अमे॒दस्क॑ ममे॒दस्क॒म् तस्मा॒ आ । \newline
11. अ॒मे॒दस्क॒मित्य॑मे॒दः - क॒म् । \newline
12. तस्मा॒ आ तस्मै॒ तस्मा॒ आ द॑द्ध्याद् दद्ध्या॒दा तस्मै॒ तस्मा॒ आ द॑द्ध्यात् । \newline
13. आ द॑द्ध्याद् दद्ध्या॒दा द॑द्ध्या॒न् मेदो॑रूपा॒ मेदो॑रूपा दद्ध्या॒दा द॑द्ध्या॒न् मेदो॑रूपाः । \newline
14. द॒द्ध्या॒न् मेदो॑रूपा॒ मेदो॑रूपा दद्ध्याद् दद्ध्या॒न् मेदो॑रूपा॒ वै वै मेदो॑रूपा दद्ध्याद् दद्ध्या॒न् मेदो॑रूपा॒ वै । \newline
15. मेदो॑रूपा॒ वै वै मेदो॑रूपा॒ मेदो॑रूपा॒ वै प॒शवः॑ प॒शवो॒ वै मेदो॑रूपा॒ मेदो॑रूपा॒ वै प॒शवः॑ । \newline
16. मेदो॑रूपा॒ इति॒ मेदः॑ - रू॒पाः॒ । \newline
17. वै प॒शवः॑ प॒शवो॒ वै वै प॒शवो॑ रू॒पेण॑ रू॒पेण॑ प॒शवो॒ वै वै प॒शवो॑ रू॒पेण॑ । \newline
18. प॒शवो॑ रू॒पेण॑ रू॒पेण॑ प॒शवः॑ प॒शवो॑ रू॒पे णै॒वैव रू॒पेण॑ प॒शवः॑ प॒शवो॑ रू॒पेणै॒व । \newline
19. रू॒पे णै॒वैव रू॒पेण॑ रू॒पेणै॒ वैन॑ मेन मे॒व रू॒पेण॑ रू॒पे णै॒वैन᳚म् । \newline
20. ए॒वैन॑ मेन मे॒वै वैन॑म् प॒शुभ्यः॑ प॒शुभ्य॑ एन मे॒वै वैन॑म् प॒शुभ्यः॑ । \newline
21. ए॒न॒म् प॒शुभ्यः॑ प॒शुभ्य॑ एन मेनम् प॒शुभ्यो॒ निर् णिष् प॒शुभ्य॑ एन मेनम् प॒शुभ्यो॒ निः । \newline
22. प॒शुभ्यो॒ निर् णिष् प॒शुभ्यः॑ प॒शुभ्यो॒ निर् भ॑जति भजति॒ निष् प॒शुभ्यः॑ प॒शुभ्यो॒ निर् भ॑जति । \newline
23. प॒शुभ्य॒ इति॑ प॒शु - भ्यः॒ । \newline
24. निर् भ॑जति भजति॒ निर् णिर् भ॑ज त्यप॒शु र॑प॒शुर् भ॑जति॒ निर् णिर् भ॑ज त्यप॒शुः । \newline
25. भ॒ज॒ त्य॒प॒शु र॑प॒शुर् भ॑जति भज त्यप॒शु रे॒वै वाप॒शुर् भ॑जति भज त्यप॒शु रे॒व । \newline
26. अ॒प॒शु रे॒वै वाप॒शु र॑प॒शु रे॒व भ॑वति भव त्ये॒वाप॒शु र॑प॒शु रे॒व भ॑वति । \newline
27. ए॒व भ॑वति भव त्ये॒वैव भ॑वति॒ यं ॅयम् भ॑व त्ये॒वैव भ॑वति॒ यम् । \newline
28. भ॒व॒ति॒ यं ॅयम् भ॑वति भवति॒ यम् का॒मये॑त का॒मये॑त॒ यम् भ॑वति भवति॒ यम् का॒मये॑त । \newline
29. यम् का॒मये॑त का॒मये॑त॒ यं ॅयम् का॒मये॑त पशु॒मान् प॑शु॒मान् का॒मये॑त॒ यं ॅयम् का॒मये॑त पशु॒मान् । \newline
30. का॒मये॑त पशु॒मान् प॑शु॒मान् का॒मये॑त का॒मये॑त पशु॒मान् थ्स्या᳚थ् स्यात् पशु॒मान् का॒मये॑त का॒मये॑त पशु॒मान् थ्स्या᳚त् । \newline
31. प॒शु॒मान् थ्स्या᳚थ् स्यात् पशु॒मान् प॑शु॒मान् थ्स्या॒ दितीति॑ स्यात् पशु॒मान् प॑शु॒मान् थ्स्या॒ दिति॑ । \newline
32. प॒शु॒मानिति॑ पशु - मान् । \newline
33. स्या॒दि तीति॑ स्याथ् स्या॒ दिति॒ मेद॑स्व॒न् मेद॑स्व॒ दिति॑ स्याथ् स्या॒ दिति॒ मेद॑स्वत् । \newline
34. इति॒ मेद॑स्व॒न् मेद॑स्व॒ दितीति॒ मेद॑स्व॒त् तस्मै॒ तस्मै॒ मेद॑स्व॒ दितीति॒ मेद॑स्व॒त् तस्मै᳚ । \newline
35. मेद॑स्व॒त् तस्मै॒ तस्मै॒ मेद॑स्व॒न् मेद॑स्व॒त् तस्मा॒ आ तस्मै॒ मेद॑स्व॒न् मेद॑स्व॒त् तस्मा॒ आ । \newline
36. तस्मा॒ आ तस्मै॒ तस्मा॒ आ द॑द्ध्याद् दद्ध्या॒दा तस्मै॒ तस्मा॒ आ द॑द्ध्यात् । \newline
37. आ द॑द्ध्याद् दद्ध्या॒दा द॑द्ध्या॒न् मेदो॑रूपा॒ मेदो॑रूपा दद्ध्या॒दा द॑द्ध्या॒न् मेदो॑रूपाः । \newline
38. द॒द्ध्या॒न् मेदो॑रूपा॒ मेदो॑रूपा दद्ध्याद् दद्ध्या॒न् मेदो॑रूपा॒ वै वै मेदो॑रूपा दद्ध्याद् दद्ध्या॒न् मेदो॑रूपा॒ वै । \newline
39. मेदो॑रूपा॒ वै वै मेदो॑रूपा॒ मेदो॑रूपा॒ वै प॒शवः॑ प॒शवो॒ वै मेदो॑रूपा॒ मेदो॑रूपा॒ वै प॒शवः॑ । \newline
40. मेदो॑रूपा॒ इति॒ मेदः॑ - रू॒पाः॒ । \newline
41. वै प॒शवः॑ प॒शवो॒ वै वै प॒शवो॑ रू॒पेण॑ रू॒पेण॑ प॒शवो॒ वै वै प॒शवो॑ रू॒पेण॑ । \newline
42. प॒शवो॑ रू॒पेण॑ रू॒पेण॑ प॒शवः॑ प॒शवो॑ रू॒पे णै॒वैव रू॒पेण॑ प॒शवः॑ प॒शवो॑ रू॒पेणै॒व । \newline
43. रू॒पे णै॒वैव रू॒पेण॑ रू॒पे णै॒वास्मा॑ अस्मा ए॒व रू॒पेण॑ रू॒पे णै॒वास्मै᳚ । \newline
44. ए॒वास्मा॑ अस्मा ए॒वै वास्मै॑ प॒शून् प॒शू न॑स्मा ए॒वै वास्मै॑ प॒शून् । \newline
45. अ॒स्मै॒ प॒शून् प॒शू न॑स्मा अस्मै प॒शून वाव॑ प॒शू न॑स्मा अस्मै प॒शूनव॑ । \newline
46. प॒शून वाव॑ प॒शून् प॒शू नव॑ रुन्धे रु॒न्धे ऽव॑ प॒शून् प॒शू नव॑ रुन्धे । \newline
47. अव॑ रुन्धे रु॒न्धे ऽवाव॑ रुन्धे पशु॒मान् प॑शु॒मान् रु॒न्धे ऽवाव॑ रुन्धे पशु॒मान् । \newline
48. रु॒न्धे॒ प॒शु॒मान् प॑शु॒मान् रु॑न्धे रुन्धे पशु॒मा ने॒वैव प॑शु॒मान् रु॑न्धे रुन्धे पशु॒माने॒व । \newline
49. प॒शु॒मा ने॒वैव प॑शु॒मान् प॑शु॒माने॒व भ॑वति भव त्ये॒व प॑शु॒मान् प॑शु॒माने॒व भ॑वति । \newline
50. प॒शु॒मानिति॑ पशु - मान् । \newline
51. ए॒व भ॑वति भव त्ये॒वैव भ॑वति प्र॒जाप॑तिः प्र॒जाप॑तिर् भव त्ये॒वैव भ॑वति प्र॒जाप॑तिः । \newline
52. भ॒व॒ति॒ प्र॒जाप॑तिः प्र॒जाप॑तिर् भवति भवति प्र॒जाप॑तिर् य॒ज्ञ्ं ॅय॒ज्ञ्म् प्र॒जाप॑तिर् भवति भवति प्र॒जाप॑तिर् य॒ज्ञ्म् । \newline
53. प्र॒जाप॑तिर् य॒ज्ञ्ं ॅय॒ज्ञ्म् प्र॒जाप॑तिः प्र॒जाप॑तिर् य॒ज्ञ् म॑सृजता सृजत य॒ज्ञ्म् प्र॒जाप॑तिः प्र॒जाप॑तिर् य॒ज्ञ् म॑सृजत । \newline
54. प्र॒जाप॑ति॒रिति॑ प्र॒जा - प॒तिः॒ । \newline
55. य॒ज्ञ् म॑सृजता सृजत य॒ज्ञ्ं ॅय॒ज्ञ् म॑सृजत॒ स सो॑ ऽसृजत य॒ज्ञ्ं ॅय॒ज्ञ् म॑सृजत॒ सः । \newline
56. अ॒सृ॒ज॒त॒ स सो॑ ऽसृजता सृजत॒ स आज्य॒ माज्यꣳ॒॒ सो॑ ऽसृजता सृजत॒ स आज्यं᳚ । \newline
57. स आज्य॒ माज्यꣳ॒॒ स स आज्यं॑ पु॒रस्ता᳚त् पु॒रस्ता॒ दाज्यꣳ॒॒ स स आज्यं॑ पु॒रस्ता᳚त् । \newline
58. आज्यं॑ पु॒रस्ता᳚त् पु॒रस्ता॒ दाज्य॒ माज्यं॑ पु॒रस्ता॑ दसृजता सृजत पु॒रस्ता॒ दाज्य॒ माज्यं॑ पु॒रस्ता॑ दसृजत । \newline
\pagebreak
\markright{ TS 6.3.11.6  \hfill https://www.vedavms.in \hfill}

\section{ TS 6.3.11.6 }

\textbf{TS 6.3.11.6 } \newline
\textbf{Samhita Paata} \newline

पु॒रस्ता॑दसृजत प॒शुं म॑द्ध्य॒तः पृ॑षदा॒ज्यं प॒श्चात् तस्मा॒दाज्ये॑न प्रया॒जा इ॑ज्यन्ते प॒शुना॑ मद्ध्य॒तः पृ॑षदा॒ज्येना॑-नूया॒जा-स्तस्मा॑दे॒तन्मि॒श्रमि॑व पश्चाथ् सृ॒ष्टꣳ ह्येका॑दशानूया॒जान्. य॑जति॒ दश॒ वै प॒शोः प्रा॒णा आ॒त्मैका॑द॒शो यावा॑ने॒व प॒शुस्तमनु॑ यजति॒ घ्नन्ति॒ वा ए॒तत् प॒शुं ॅयथ् स᳚ज्ञ्ं॒पय॑न्ति प्रा॒णापा॒नौ खलु॒ वा ए॒तौ प॑शू॒नां ॅयत् पृ॑षदा॒ज्यं ॅयत् पृ॑षदा॒ज्येना॑ ( ) नूया॒जान्. यज॑ति प्राणापा॒नावे॒व प॒शुषु॑ दधाति ॥ \newline

\textbf{Pada Paata} \newline

पु॒रस्ता᳚त् । अ॒सृ॒ज॒त॒ । प॒शुम् । म॒द्ध्य॒तः । पृ॒ष॒दा॒ज्यमिति॑ पृषत् - आ॒ज्यम् । प॒श्चात् । तस्मा᳚त् । आज्ये॑न । प्र॒या॒जा इति॑ प्र - या॒जाः । इ॒ज्य॒न्ते॒ । प॒शुना᳚ । म॒द्ध्य॒तः । पृ॒ष॒दा॒ज्येनेति॑ पृषत् - आ॒ज्येन॑ । अ॒नू॒या॒जा इत्य॑नु - या॒जाः । तस्मा᳚त् । ए॒तत् । मि॒श्रम् । इ॒व॒ । प॒श्चा॒थ्सृ॒ष्टमिति॑ पश्चात् - सृ॒ष्टम् । हि । एका॑दश । अ॒नू॒या॒जानित्य॑नु - या॒जान् । य॒ज॒ति॒ । दश॑ । वै । प॒शोः । प्रा॒णा इति॑ प्र - अ॒नाः । आ॒त्मा । ए॒का॒द॒शः । यावान्॑ । ए॒व । प॒शुः । तम् । अन्विति॑ । य॒ज॒ति॒ । घ्नन्ति॑ । वै । ए॒तत् । प॒शुम् । यत् । स॒ज्ञ्ं॒पय॒न्तीति॑ सं - ज्ञ्॒पय॑न्ति । प्रा॒णा॒पा॒नाविति॑ प्राण - अ॒पा॒नौ । खलु॑ । वै । ए॒तौ । प॒शू॒नाम् । यत् । पृ॒ष॒दा॒ज्यमिति॑ पृषत् - आ॒ज्यम् । यत् । पृ॒ष॒दा॒ज्येनेति॑ पृषत् - आ॒ज्येन॑ ( ) । अ॒नू॒या॒जानित्य॑नु - या॒जान् । यज॑ति । प्रा॒णा॒पा॒नाविति॑ प्राण - अ॒पा॒नौ । ए॒व । प॒शुषु॑ । द॒धा॒ति॒ ॥  \newline


\textbf{Krama Paata} \newline

पु॒रस्ता॑दसृजत । अ॒सृ॒ज॒त॒ प॒शुम् । प॒शुम् म॑द्ध्य॒तः । म॒द्ध्य॒तः पृ॑षदा॒ज्यम् । पृ॒ष॒दा॒ज्यम् प॒श्चात् । पृ॒ष॒दा॒ज्यमिति॑ पृषत् - आ॒ज्यम् । प॒श्चात् तस्मा᳚त् । तस्मा॒दाज्ये॑न । आज्ये॑न प्रया॒जाः । प्र॒या॒जा इ॑ज्यन्ते । प्र॒या॒जा इति॑ प्र - या॒जाः । इ॒ज्य॒न्ते॒ प॒शुना᳚ । प॒शुना॑ मद्ध्य॒तः । म॒द्ध्य॒तः पृ॑षदा॒ज्येन॑ । पृ॒ष॒दा॒ज्ये,ना॑नूया॒जाः । पृ॒ष॒दा॒ज्येनेति॑ पृषत् - आ॒ज्येन॑ । अ॒नू॒या॒जास्तस्मा᳚त् । अ॒नू॒या॒जा इत्य॑नु - या॒जाः । तस्मा॑दे॒तत् । ए॒तन् मि॒श्रम् । मि॒श्रमि॑व । इ॒व॒ प॒श्चा॒थ्सृ॒ष्टम् । प॒श्चा॒थ्सृ॒ष्टꣳ हि । प॒श्चा॒थ्सृ॒ष्टमिति॑ पश्चात् - सृ॒ष्टम् । ह्येका॑दश । एका॑दशानूया॒जान् । अ॒नू॒या॒जान्. य॑जति । अ॒नू॒या॒जानित्य॑नु - या॒जान् । य॒ज॒ति॒ दश॑ । दश॒ वै । वै प॒शोः । प॒शोः प्रा॒णाः । प्रा॒णा आ॒त्मा । प्रा॒णा इति॑ प्र - अ॒नाः । आ॒त्मैका॑द॒शः । ए॒का॒द॒शो यावान्॑ । यावा॑ने॒व । ए॒व प॒शुः । प॒शुस्तम् । तमनु॑ । अनु॑ यजति । य॒ज॒ति॒ घ्नन्ति॑ । घ्नन्ति॒ वै । वा ए॒तत् । ए॒तत् प॒शुम् । प॒शुम् ॅयत् । यथ् स᳚म्(2)ज्ञ्॒पय॑न्ति । स॒म्(2)ज्ञ्॒पय॑न्ति प्राणापा॒नौ । स॒म्(2)ज्ञ्॒पय॒न्तीति॑ सम् - ज्ञ्॒पय॑न्ति । प्रा॒णा॒पा॒नौ खलु॑ । प्रा॒णा॒पा॒नाविति॑ प्राण - अ॒पा॒नौ । खलु॒ वै । वा ए॒तौ । ए॒तौ प॑शू॒नाम् । प॒शू॒नाम् ॅयत् । यत् पृ॑षदा॒ज्यम् । पृ॒ष॒दा॒ज्यम् ॅयत् । पृ॒ष॒दा॒ज्यमिति॑ पृषत् - आ॒ज्यम् । यत् पृ॑षदा॒ज्येन॑ ( ) । पृ॒ष॒दा॒ज्येना॑नूया॒जान् । पृ॒ष॒दा॒ज्येनेति॑ पृषत् - आ॒ज्येन॑ । अ॒नू॒या॒जान्. यज॑ति । अ॒नू॒या॒जानित्य॑नु - या॒जान् । यज॑ति प्राणापा॒नौ । प्रा॒णा॒पा॒नावे॒व । प्रा॒णा॒पा॒नाविति॑ प्राण - अ॒पा॒नौ । ए॒व प॒शुषु॑ । प॒शुषु॑ दधाति । द॒धा॒तीति॑ दधाति । \newline

\textbf{Jatai Paata} \newline

1. पु॒रस्ता॑ दसृजता सृजत पु॒रस्ता᳚त् पु॒रस्ता॑ दसृजत । \newline
2. अ॒सृ॒ज॒त॒ प॒शुम् प॒शु म॑सृजता सृजत प॒शुम् । \newline
3. प॒शुम् म॑द्ध्य॒तो म॑द्ध्य॒तः प॒शुम् प॒शुम् म॑द्ध्य॒तः । \newline
4. म॒द्ध्य॒तः पृ॑षदा॒ज्यम् पृ॑षदा॒ज्यम् म॑द्ध्य॒तो म॑द्ध्य॒तः पृ॑षदा॒ज्यम् । \newline
5. पृ॒ष॒दा॒ज्यम् प॒श्चात् प॒श्चात् पृ॑षदा॒ज्यम् पृ॑षदा॒ज्यम् प॒श्चात् । \newline
6. पृ॒ष॒दा॒ज्यमिति॑ पृषत् - आ॒ज्यम् । \newline
7. प॒श्चात् तस्मा॒त् तस्मा᳚त् प॒श्चात् प॒श्चात् तस्मा᳚त् । \newline
8. तस्मा॒ दाज्ये॒ना ज्ये॑न॒ तस्मा॒त् तस्मा॒ दाज्ये॑न । \newline
9. आज्ये॑न प्रया॒जाः प्र॑या॒जा आज्ये॒ना ज्ये॑न प्रया॒जाः । \newline
10. प्र॒या॒जा इ॑ज्यन्त इज्यन्ते प्रया॒जाः प्र॑या॒जा इ॑ज्यन्ते । \newline
11. प्र॒या॒जा इति॑ प्र - या॒जाः । \newline
12. इ॒ज्य॒न्ते॒ प॒शुना॑ प॒शुने᳚ ज्यन्त इज्यन्ते प॒शुना᳚ । \newline
13. प॒शुना॑ मद्ध्य॒तो म॑द्ध्य॒तः प॒शुना॑ प॒शुना॑ मद्ध्य॒तः । \newline
14. म॒द्ध्य॒तः पृ॑षदा॒ज्येन॑ पृषदा॒ज्येन॑ मद्ध्य॒तो म॑द्ध्य॒तः पृ॑षदा॒ज्येन॑ । \newline
15. पृ॒ष॒दा॒ज्येना॑ नूया॒जा अ॑नूया॒जाः पृ॑षदा॒ज्येन॑ पृषदा॒ज्येना॑ नूया॒जाः । \newline
16. पृ॒ष॒दा॒ज्येनेति॑ पृषत् - आ॒ज्येन॑ । \newline
17. अ॒नू॒या॒जा स्तस्मा॒त् तस्मा॑ दनूया॒जा अ॑नूया॒जा स्तस्मा᳚त् । \newline
18. अ॒नू॒या॒जा इत्य॑नु - या॒जाः । \newline
19. तस्मा॑ दे॒त दे॒तत् तस्मा॒त् तस्मा॑ दे॒तत् । \newline
20. ए॒तन् मि॒श्रम् मि॒श्र मे॒त दे॒तन् मि॒श्रम् । \newline
21. मि॒श्र मि॑वेव मि॒श्रम् मि॒श्र मि॑व । \newline
22. इ॒व॒ प॒श्चा॒थ्सृ॒ष्टम् प॑श्चाथ्सृ॒ष्ट मि॑वेव पश्चाथ्सृ॒ष्टम् । \newline
23. प॒श्चा॒थ्सृ॒ष्टꣳ हि हि प॑श्चाथ्सृ॒ष्टम् प॑श्चाथ्सृ॒ष्टꣳ हि । \newline
24. प॒श्चा॒थ्सृ॒ष्टमिति॑ पश्चात् - सृ॒ष्टम् । \newline
25. ह्येका॑द॒ शैका॑दश॒ हि ह्येका॑दश । \newline
26. एका॑दशा नूया॒जान॑ नूया॒जा नेका॑द॒ शैका॑दशा नूया॒जान् । \newline
27. अ॒नू॒या॒जान्. य॑जति यज त्यनूया॒जा न॑नूया॒जान्. य॑जति । \newline
28. अ॒नू॒या॒जानित्य॑नु - या॒जान् । \newline
29. य॒ज॒ति॒ दश॒ दश॑ यजति यजति॒ दश॑ । \newline
30. दश॒ वै वै दश॒ दश॒ वै । \newline
31. वै प॒शोः प॒शोर् वै वै प॒शोः । \newline
32. प॒शोः प्रा॒णाः प्रा॒णाः प॒शोः प॒शोः प्रा॒णाः । \newline
33. प्रा॒णा आ॒त्मा ऽऽत्मा प्रा॒णाः प्रा॒णा आ॒त्मा । \newline
34. प्रा॒णा इति॑ प्र - अ॒नाः । \newline
35. आ॒त्मैका॑द॒श ए॑काद॒श आ॒त्मा ऽऽत्मैका॑द॒शः । \newline
36. ए॒का॒द॒शो यावा॒न्॒. यावा॑ने काद॒श ए॑काद॒शो यावान्॑ । \newline
37. यावा॑ने॒ वैव यावा॒न्॒. यावा॑ने॒व । \newline
38. ए॒व प॒शुः प॒शु रे॒वैव प॒शुः । \newline
39. प॒शु स्तम् तम् प॒शुः प॒शु स्तम् । \newline
40. त मन् वनु॒ तम् त मनु॑ । \newline
41. अनु॑ यजति यज॒ त्यन् वनु॑ यजति । \newline
42. य॒ज॒ति॒ घ्नन्ति॒ घ्नन्ति॑ यजति यजति॒ घ्नन्ति॑ । \newline
43. घ्नन्ति॒ वै वै घ्नन्ति॒ घ्नन्ति॒ वै । \newline
44. वा ए॒त दे॒तद् वै वा ए॒तत् । \newline
45. ए॒तत् प॒शुम् प॒शु मे॒त दे॒तत् प॒शुम् । \newline
46. प॒शुं ॅयद् यत् प॒शुम् प॒शुं ॅयत् । \newline
47. यथ् सं᳚.ज्ञ्॒पय॑न्ति सं.ज्ञ्॒पय॑न्ति॒ यद् यथ् सं᳚.ज्ञ्॒पय॑न्ति । \newline
48. सं॒.ज्ञ्॒पय॑न्ति प्राणापा॒नौ प्रा॑णापा॒नौ सं᳚.ज्ञ्॒पय॑न्ति सं.ज्ञ्॒पय॑न्ति प्राणापा॒नौ । \newline
49. सं॒.ज्ञ्॒पय॒न्तीति॑ सं. - ज्ञ्॒पय॑न्ति । \newline
50. प्रा॒णा॒पा॒नौ खलु॒ खलु॑ प्राणापा॒नौ प्रा॑णापा॒नौ खलु॑ । \newline
51. प्रा॒णा॒पा॒नाविति॑ प्राण - अ॒पा॒नौ । \newline
52. खलु॒ वै वै खलु॒ खलु॒ वै । \newline
53. वा ए॒ता वे॒तौ वै वा ए॒तौ । \newline
54. ए॒तौ प॑शू॒नाम् प॑शू॒ना मे॒ता वे॒तौ प॑शू॒नाम् । \newline
55. प॒शू॒नां ॅयद् यत् प॑शू॒नाम् प॑शू॒नां ॅयत् । \newline
56. यत् पृ॑षदा॒ज्यम् पृ॑षदा॒ज्यं ॅयद् यत् पृ॑षदा॒ज्यम् । \newline
57. पृ॒ष॒दा॒ज्यं ॅयद् यत् पृ॑षदा॒ज्यम् पृ॑षदा॒ज्यं ॅयत् । \newline
58. पृ॒ष॒दा॒ज्यमिति॑ पृषत् - आ॒ज्यम् । \newline
59. यत् पृ॑षदा॒ज्येन॑ पृषदा॒ज्येन॒ यद् यत् पृ॑षदा॒ज्येन॑ । \newline
60. पृ॒ष॒दा॒ज्येना॑ नूया॒जा न॑नूया॒जान् पृ॑षदा॒ज्येन॑ पृषदा॒ज्येना॑ नूया॒जान् । \newline
61. पृ॒ष॒दा॒ज्येनेति॑ पृषत् - आ॒ज्येन॑ । \newline
62. अ॒नू॒या॒जान्. यज॑ति॒ यज॑ त्यनूया॒जान॑ नूया॒जान्. यज॑ति । \newline
63. अ॒नू॒या॒जानित्य॑नु - या॒जान् । \newline
64. यज॑ति प्राणापा॒नौ प्रा॑णापा॒नौ यज॑ति॒ यज॑ति प्राणापा॒नौ । \newline
65. प्रा॒णा॒पा॒ना वे॒वैव प्रा॑णापा॒नौ प्रा॑णापा॒ना वे॒व । \newline
66. प्रा॒णा॒पा॒नाविति॑ प्राण - अ॒पा॒नौ । \newline
67. ए॒व प॒शुषु॑ प॒शु ष्वे॒वैव प॒शुषु॑ । \newline
68. प॒शुषु॑ दधाति दधाति प॒शुषु॑ प॒शुषु॑ दधाति । \newline
69. द॒धा॒तीति॑ दधाति । \newline

\textbf{Ghana Paata } \newline

1. पु॒रस्ता॑ दसृजता सृजत पु॒रस्ता᳚त् पु॒रस्ता॑ दसृजत प॒शुम् प॒शु म॑सृजत पु॒रस्ता᳚त् पु॒रस्ता॑ दसृजत प॒शुम् । \newline
2. अ॒सृ॒ज॒त॒ प॒शुम् प॒शु म॑सृजता सृजत प॒शुम् म॑द्ध्य॒तो म॑द्ध्य॒तः प॒शु म॑सृजता सृजत प॒शुम् म॑द्ध्य॒तः । \newline
3. प॒शुम् म॑द्ध्य॒तो म॑द्ध्य॒तः प॒शुम् प॒शुम् म॑द्ध्य॒तः पृ॑षदा॒ज्यम् पृ॑षदा॒ज्यम् म॑द्ध्य॒तः प॒शुम् प॒शुम् म॑द्ध्य॒तः पृ॑षदा॒ज्यम् । \newline
4. म॒द्ध्य॒तः पृ॑षदा॒ज्यम् पृ॑षदा॒ज्यम् म॑द्ध्य॒तो म॑द्ध्य॒तः पृ॑षदा॒ज्यम् प॒श्चात् प॒श्चात् पृ॑षदा॒ज्यम् म॑द्ध्य॒तो म॑द्ध्य॒तः पृ॑षदा॒ज्यम् प॒श्चात् । \newline
5. पृ॒ष॒दा॒ज्यम् प॒श्चात् प॒श्चात् पृ॑षदा॒ज्यम् पृ॑षदा॒ज्यम् प॒श्चात् तस्मा॒त् तस्मा᳚त् प॒श्चात् पृ॑षदा॒ज्यम् पृ॑षदा॒ज्यम् प॒श्चात् तस्मा᳚त् । \newline
6. पृ॒ष॒दा॒ज्यमिति॑ पृषत् - आ॒ज्यम् । \newline
7. प॒श्चात् तस्मा॒त् तस्मा᳚त् प॒श्चात् प॒श्चात् तस्मा॒ दाज्ये॒ना ज्ये॑न॒ तस्मा᳚त् प॒श्चात् प॒श्चात् तस्मा॒ दाज्ये॑न । \newline
8. तस्मा॒ दाज्ये॒ना ज्ये॑न॒ तस्मा॒त् तस्मा॒ दाज्ये॑न प्रया॒जाः प्र॑या॒जा आज्ये॑न॒ तस्मा॒त् तस्मा॒ दाज्ये॑न प्रया॒जाः । \newline
9. आज्ये॑न प्रया॒जाः प्र॑या॒जा आज्ये॒ना ज्ये॑न प्रया॒जा इ॑ज्यन्त इज्यन्ते प्रया॒जा आज्ये॒ना ज्ये॑न प्रया॒जा इ॑ज्यन्ते । \newline
10. प्र॒या॒जा इ॑ज्यन्त इज्यन्ते प्रया॒जाः प्र॑या॒जा इ॑ज्यन्ते प॒शुना॑ प॒शुने᳚ज्यन्ते प्रया॒जाः प्र॑या॒जा इ॑ज्यन्ते प॒शुना᳚ । \newline
11. प्र॒या॒जा इति॑ प्र - या॒जाः । \newline
12. इ॒ज्य॒न्ते॒ प॒शुना॑ प॒शुने᳚ज्यन्त इज्यन्ते प॒शुना॑ मद्ध्य॒तो म॑द्ध्य॒तः प॒शुने᳚ज्यन्त इज्यन्ते प॒शुना॑ मद्ध्य॒तः । \newline
13. प॒शुना॑ मद्ध्य॒तो म॑द्ध्य॒तः प॒शुना॑ प॒शुना॑ मद्ध्य॒तः पृ॑षदा॒ज्येन॑ पृषदा॒ज्येन॑ मद्ध्य॒तः प॒शुना॑ प॒शुना॑ मद्ध्य॒तः पृ॑षदा॒ज्येन॑ । \newline
14. म॒द्ध्य॒तः पृ॑षदा॒ज्येन॑ पृषदा॒ज्येन॑ मद्ध्य॒तो म॑द्ध्य॒तः पृ॑षदा॒ज्येना॑ नूया॒जा अ॑नूया॒जाः पृ॑षदा॒ज्येन॑ मद्ध्य॒तो म॑द्ध्य॒तः पृ॑षदा॒ज्येना॑ नूया॒जाः । \newline
15. पृ॒ष॒दा॒ज्येना॑ नूया॒जा अ॑नूया॒जाः पृ॑षदा॒ज्येन॑ पृषदा॒ज्येना॑ नूया॒जा स्तस्मा॒त् तस्मा॑ दनूया॒जाः पृ॑षदा॒ज्येन॑ पृषदा॒ज्येना॑ नूया॒जा स्तस्मा᳚त् । \newline
16. पृ॒ष॒दा॒ज्येनेति॑ पृषत् - आ॒ज्येन॑ । \newline
17. अ॒नू॒या॒जा स्तस्मा॒त् तस्मा॑ दनूया॒जा अ॑नूया॒जा स्तस्मा॑ दे॒त दे॒तत् तस्मा॑ दनूया॒जा अ॑नूया॒जा स्तस्मा॑ दे॒तत् । \newline
18. अ॒नू॒या॒जा इत्य॑नु - या॒जाः । \newline
19. तस्मा॑ दे॒त दे॒तत् तस्मा॒त् तस्मा॑ दे॒तन् मि॒श्रम् मि॒श्र मे॒तत् तस्मा॒त् तस्मा॑ दे॒तन् मि॒श्रम् । \newline
20. ए॒तन् मि॒श्रम् मि॒श्र मे॒त दे॒तन् मि॒श्र मि॑वेव मि॒श्र मे॒त दे॒तन् मि॒श्र मि॑व । \newline
21. मि॒श्र मि॑वेव मि॒श्रम् मि॒श्र मि॑व पश्चाथ्सृ॒ष्टम् प॑श्चाथ्सृ॒ष्ट मि॑व मि॒श्रम् मि॒श्र मि॑व पश्चाथ्सृ॒ष्टम् । \newline
22. इ॒व॒ प॒श्चा॒थ्सृ॒ष्टम् प॑श्चाथ्सृ॒ष्ट मि॑वेव पश्चाथ्सृ॒ष्टꣳ हि हि प॑श्चाथ्सृ॒ष्ट मि॑वेव पश्चाथ्सृ॒ष्टꣳ हि । \newline
23. प॒श्चा॒थ्सृ॒ष्टꣳ हि हि प॑श्चाथ्सृ॒ष्टम् प॑श्चाथ्सृ॒ष्टꣳ ह्येका॑द॒ शैका॑दश॒ हि प॑श्चाथ्सृ॒ष्टम् प॑श्चाथ्सृ॒ष्टꣳ ह्येका॑दश । \newline
24. प॒श्चा॒थ्सृ॒ष्टमिति॑ पश्चात् - सृ॒ष्टम् । \newline
25. ह्येका॑द॒ शैका॑दश॒ हि ह्येका॑दशा नूया॒जा न॑नूया॒जा नेका॑दश॒ हि ह्येका॑दशा नूया॒जान् । \newline
26. एका॑दशा नूया॒जा न॑नूया॒जा नेका॑द॒ शैका॑दशा नूया॒जान्. य॑जति यज त्यनूया॒जा नेका॑द॒ शैका॑दशा नूया॒जान्. य॑जति । \newline
27. अ॒नू॒या॒जान्. य॑जति यज त्यनूया॒जा न॑नूया॒जान्. य॑जति॒ दश॒ दश॑ यज त्यनूया॒जा न॑नूया॒जान्. य॑जति॒ दश॑ । \newline
28. अ॒नू॒या॒जानित्य॑नु - या॒जान् । \newline
29. य॒ज॒ति॒ दश॒ दश॑ यजति यजति॒ दश॒ वै वै दश॑ यजति यजति॒ दश॒ वै । \newline
30. दश॒ वै वै दश॒ दश॒ वै प॒शोः प॒शोर् वै दश॒ दश॒ वै प॒शोः । \newline
31. वै प॒शोः प॒शोर् वै वै प॒शोः प्रा॒णाः प्रा॒णाः प॒शोर् वै वै प॒शोः प्रा॒णाः । \newline
32. प॒शोः प्रा॒णाः प्रा॒णाः प॒शोः प॒शोः प्रा॒णा आ॒त्मा ऽऽत्मा प्रा॒णाः प॒शोः प॒शोः प्रा॒णा आ॒त्मा । \newline
33. प्रा॒णा आ॒त्मा ऽऽत्मा प्रा॒णाः प्रा॒णा आ॒त्मैका॑द॒श ए॑काद॒श आ॒त्मा प्रा॒णाः प्रा॒णा आ॒त्मैका॑द॒शः । \newline
34. प्रा॒णा इति॑ प्र - अ॒नाः । \newline
35. आ॒त्मैका॑द॒श ए॑काद॒श आ॒त्मा ऽऽत्मैका॑द॒शो यावा॒न्॒. यावा॑ नेकाद॒श आ॒त्मा 
ऽऽत्मैका॑द॒शो यावान्॑ । \newline
36. ए॒का॒द॒शो यावा॒न्॒. यावा॑ नेकाद॒श ए॑काद॒शो यावा॑ ने॒वैव यावा॑ नेकाद॒श ए॑काद॒शो यावा॑ने॒व । \newline
37. यावा॑ ने॒वैव यावा॒न्॒. यावा॑ने॒व प॒शुः प॒शु रे॒व यावा॒न्॒. यावा॑ने॒व प॒शुः । \newline
38. ए॒व प॒शुः प॒शु रे॒वैव प॒शु स्तम् तम् प॒शु रे॒वैव प॒शु स्तम् । \newline
39. प॒शु स्तम् तम् प॒शुः प॒शु स्त मन्वनु॒ तम् प॒शुः प॒शु स्त मनु॑ । \newline
40. त मन्वनु॒ तम् त मनु॑ यजति यज॒ त्यनु॒ तम् त मनु॑ यजति । \newline
41. अनु॑ यजति यज॒ त्यन् वनु॑ यजति॒ घ्नन्ति॒ घ्नन्ति॑ यज॒ त्यन् वनु॑ यजति॒ घ्नन्ति॑ । \newline
42. य॒ज॒ति॒ घ्नन्ति॒ घ्नन्ति॑ यजति यजति॒ घ्नन्ति॒ वै वै घ्नन्ति॑ यजति यजति॒ घ्नन्ति॒ वै । \newline
43. घ्नन्ति॒ वै वै घ्नन्ति॒ घ्नन्ति॒ वा ए॒त दे॒तद् वै घ्नन्ति॒ घ्नन्ति॒ वा ए॒तत् । \newline
44. वा ए॒त दे॒तद् वै वा ए॒तत् प॒शुम् प॒शु मे॒तद् वै वा ए॒तत् प॒शुम् । \newline
45. ए॒तत् प॒शुम् प॒शु मे॒त दे॒तत् प॒शुं ॅयद् यत् प॒शु मे॒त दे॒तत् प॒शुं ॅयत् । \newline
46. प॒शुं ॅयद् यत् प॒शुम् प॒शुं ॅयथ् सं᳚.ज्ञ्॒पय॑न्ति सं.ज्ञ्॒पय॑न्ति॒ यत् प॒शुम् प॒शुं ॅयथ् 
सं᳚.ज्ञ्॒पय॑न्ति । \newline
47. यथ् सं᳚.ज्ञ्॒पय॑न्ति सं.ज्ञ्॒पय॑न्ति॒ यद् यथ् सं᳚.ज्ञ्॒पय॑न्ति प्राणापा॒नौ प्रा॑णापा॒नौ सं᳚.ज्ञ्॒पय॑न्ति॒ यद् यथ् सं᳚.ज्ञ्॒पय॑न्ति प्राणापा॒नौ । \newline
48. सं॒.ज्ञ्॒पय॑न्ति प्राणापा॒नौ प्रा॑णापा॒नौ सं᳚.ज्ञ्॒पय॑न्ति सं.ज्ञ्॒पय॑न्ति प्राणापा॒नौ खलु॒ खलु॑ प्राणापा॒नौ सं᳚.ज्ञ्॒पय॑न्ति सं.ज्ञ्॒पय॑न्ति प्राणापा॒नौ खलु॑ । \newline
49. सं॒.ज्ञ्॒पय॒न्तीति॑ सं. - ज्ञ्॒पय॑न्ति । \newline
50. प्रा॒णा॒पा॒नौ खलु॒ खलु॑ प्राणापा॒नौ प्रा॑णापा॒नौ खलु॒ वै वै खलु॑ प्राणापा॒नौ प्रा॑णापा॒नौ खलु॒ वै । \newline
51. प्रा॒णा॒पा॒नाविति॑ प्राण - अ॒पा॒नौ । \newline
52. खलु॒ वै वै खलु॒ खलु॒ वा ए॒ता वे॒तौ वै खलु॒ खलु॒ वा ए॒तौ । \newline
53. वा ए॒ता वे॒तौ वै वा ए॒तौ प॑शू॒नाम् प॑शू॒ना मे॒तौ वै वा ए॒तौ प॑शू॒नाम् । \newline
54. ए॒तौ प॑शू॒नाम् प॑शू॒ना मे॒ता वे॒तौ प॑शू॒नां ॅयद् यत् प॑शू॒ना मे॒ता वे॒तौ प॑शू॒नां ॅयत् । \newline
55. प॒शू॒नां ॅयद् यत् प॑शू॒नाम् प॑शू॒नां ॅयत् पृ॑षदा॒ज्यम् पृ॑षदा॒ज्यं ॅयत् प॑शू॒नाम् प॑शू॒नां ॅयत् पृ॑षदा॒ज्यम् । \newline
56. यत् पृ॑षदा॒ज्यम् पृ॑षदा॒ज्यं ॅयद् यत् पृ॑षदा॒ज्यं ॅयद् यत् पृ॑षदा॒ज्यं ॅयद् यत् पृ॑षदा॒ज्यं ॅयत् । \newline
57. पृ॒ष॒दा॒ज्यं ॅयद् यत् पृ॑षदा॒ज्यम् पृ॑षदा॒ज्यं ॅयत् पृ॑षदा॒ज्येन॑ पृषदा॒ज्येन॒ यत् पृ॑षदा॒ज्यम् पृ॑षदा॒ज्यं ॅयत् पृ॑षदा॒ज्येन॑ । \newline
58. पृ॒ष॒दा॒ज्यमिति॑ पृषत् - आ॒ज्यम् । \newline
59. यत् पृ॑षदा॒ज्येन॑ पृषदा॒ज्येन॒ यद् यत् पृ॑षदा॒ज्येना॑ नूया॒जा न॑नूया॒जान् पृ॑षदा॒ज्येन॒ यद् यत् पृ॑षदा॒ज्येना॑ नूया॒जान् । \newline
60. पृ॒ष॒दा॒ज्येना॑ नूया॒जा न॑नूया॒जान् पृ॑षदा॒ज्येन॑ पृषदा॒ज्येना॑ नूया॒जान्. यज॑ति॒ यज॑ त्यनूया॒जान् पृ॑षदा॒ज्येन॑ पृषदा॒ज्येना॑ नूया॒जान्. यज॑ति । \newline
61. पृ॒ष॒दा॒ज्येनेति॑ पृषत् - आ॒ज्येन॑ । \newline
62. अ॒नू॒या॒जान्. यज॑ति॒ यज॑ त्यनूया॒जा न॑नूया॒जान्. यज॑ति प्राणापा॒नौ प्रा॑णापा॒नौ यज॑ त्यनूया॒जा न॑नूया॒जान्. यज॑ति प्राणापा॒नौ । \newline
63. अ॒नू॒या॒जानित्य॑नु - या॒जान् । \newline
64. यज॑ति प्राणापा॒नौ प्रा॑णापा॒नौ यज॑ति॒ यज॑ति प्राणापा॒ना वे॒वैव प्रा॑णापा॒नौ यज॑ति॒ यज॑ति प्राणापा॒ना वे॒व । \newline
65. प्रा॒णा॒पा॒ना वे॒वैव प्रा॑णापा॒नौ प्रा॑णापा॒ना वे॒व प॒शुषु॑ प॒शुष्वे॒व प्रा॑णापा॒नौ प्रा॑णापा॒ना वे॒व प॒शुषु॑ । \newline
66. प्रा॒णा॒पा॒नाविति॑ प्राण - अ॒पा॒नौ । \newline
67. ए॒व प॒शुषु॑ प॒शु ष्वे॒वैव प॒शुषु॑ दधाति दधाति प॒शु ष्वे॒वैव प॒शुषु॑ दधाति । \newline
68. प॒शुषु॑ दधाति दधाति प॒शुषु॑ प॒शुषु॑ दधाति । \newline
69. द॒धा॒तीति॑ दधाति । \newline
\pagebreak


\end{document}