\documentclass[17pt]{extarticle}
\usepackage{babel}
\usepackage{fontspec}
\usepackage{polyglossia}
\usepackage{extsizes}

\usepackage{color}   %May be necessary if you want to color links
\usepackage{hyperref}
\hypersetup{
    colorlinks=true, %set true if you want colored links
    linktoc=all,     %set to all if you want both sections and subsections linked
    linkcolor=black,  %choose some color if you want links to stand out
}

\setmainlanguage{sanskrit}
\setotherlanguages{english} %% or other languages
\setlength{\parindent}{0pt}
\pagestyle{myheadings}
\newfontfamily\devanagarifont[Script=Devanagari]{AdishilaVedic}
\renewcommand{\theHsection}{\thepart.section.\thesection}

\newcommand{\VAR}[1]{}
\newcommand{\BLOCK}[1]{}




\begin{document}
\begin{titlepage}
    \begin{center}
 
\begin{sanskrit}
    { \Large
    कृष्ण यजुर्वेदीय तैत्तिरीय संहिता,पद,जटा,घन पाठः 
    }
    \\
    \vspace{2.5cm}
    \mbox{ \Large
    6.3      षष्ठकाण्डे तृतीयः प्रश्नः - सोममन्त्रब्राह्मणनिरूपणं   }
\end{sanskrit}
\end{center}

\end{titlepage}
\tableofcontents
\phantomsection
\pagebreak

\markright{ TS 6.3.1.1  \hfill https://www.vedavms.in \hfill}

\section{ TS 6.3.1.1 }

\textbf{TS 6.3.1.1 } \newline
\textbf{Samhita Paata} \newline

चात्वा॑ला॒द्-धिष्णि॑या॒नुप॑ वपति॒ योनि॒र्वै य॒ज्ञ्स्य॒ चात्वा॑लं ॅय॒ज्ञ्स्य॑ सयोनि॒त्वाय॑ दे॒वा वै य॒ज्ञ्ं परा॑ऽजयन्त॒ तमाग्नी᳚द्ध्रा॒त् पुन॒रपा॑जयन्ने॒तद्वै य॒ज्ञ्स्या-प॑राजितं॒ ॅयदाग्नी᳚द्ध्रं॒ ॅयदाग्नी᳚द्ध्रा॒द्धिष्णि॑यान्. वि॒हर॑ति॒ यदे॒व  य॒ज्ञ्स्या-प॑राजितं॒ तत॑ ए॒वैनं॒ पुन॑स्तनुते परा॒जित्ये॑व॒ खलु॒ वा ए॒ते य॑न्ति॒ ये ब॑हिष्पवमा॒नꣳ सर्प॑न्ति बहिष्पवमा॒ने स्तु॒त- [  ] \newline

\textbf{Pada Paata} \newline

चात्वा॑लात् । धिष्णि॑यान् । उपेति॑ । व॒प॒ति॒ । योनिः॑ । वै । य॒ज्ञ्स्य॑ । चात्वा॑लम् । य॒ज्ञ्स्य॑ । स॒यो॒नि॒त्वायेति॑ सयोनि - त्वाय॑ । दे॒वाः । वै । य॒ज्ञ्म् । परेति॑ । अ॒ज॒य॒न्त॒ । तम् । आग्नी᳚᳚द्ध्रा॒दित्याग्नि॑ - इ॒द्ध्रा॒त् । पुनः॑ । अपेति॑ । अ॒ज॒य॒न्न् । ए॒तत् । वै । य॒ज्ञ्स्य॑ । अप॑राजित॒मित्यप॑रा - जि॒त॒म् । यत् । आग्नी᳚द्ध्र॒मित्याग्नि॑ - इ॒द्ध्र॒म् । यत् । आग्नी᳚द्ध्रा॒दित्याग्नि॑ - इ॒द्ध्रा॒त् । धिष्णि॑यान् । वि॒हर॒तीति॑ वि-हर॑ति । यत् । ए॒व । य॒ज्ञ्स्य॑ । अप॑राजित॒मित्यप॑रा - जि॒त॒म् । ततः॑ । ए॒व । ए॒न॒म् । पुनः॑ । त॒नु॒ते॒ । प॒रा॒जित्येति॑ परा-जित्य॑ । इ॒व॒ । खलु॑ । वै । ए॒ते । य॒न्ति॒ । ये । ब॒हि॒ष्प॒व॒मा॒नमिति॑ बहिः - प॒व॒मा॒नम् । सर्प॑न्ति । ब॒हि॒ष्प॒व॒मा॒न इति॑ बहिः - प॒व॒मा॒ने । स्तु॒ते ।  \newline




\markright{ TS 6.3.1.2  \hfill https://www.vedavms.in \hfill}

\section{ TS 6.3.1.2 }

\textbf{TS 6.3.1.2 } \newline
\textbf{Samhita Paata} \newline

आ॒हाग्नी॑द॒ग्नीन्. वि ह॑र ब॒र्॒.हिः स्तृ॑णाहि पुरो॒डाशाꣳ॒॒ अलं॑ कु॒र्विति॑ य॒ज्ञ्मे॒वाप॒जित्य॒ पुन॑स्तन्वा॒ना य॒न्त्यङ्गा॑रै॒र्द्वे सव॑ने॒ वि ह॑रति श॒लाका॑भि-स्तृ॒तीयꣳ॑ सशुक्र॒त्वायाथो॒ सं भ॑रत्ये॒वैन॒द्धिष्णि॑या॒ वा अ॒मुष्मि॑न् ॅलो॒के सोम॑मरक्ष॒न् तेभ्योऽधि॒ सोम॒माऽह॑र॒न् त म॑न्व॒वाय॒न्तं पर्य॑विश॒न्॒. य ए॒वं ॅवेद॑ वि॒न्दते॑- [  ] \newline

\textbf{Pada Paata} \newline

आ॒ह॒ । अग्नी॒दित्यग्नि॑ - इ॒त् । अ॒ग्नीन् । वीति॑ । ह॒र॒ । ब॒र॒.हिः । स्तृ॒णा॒हि॒ । पु॒रो॒डाशान्॑ । अल᳚म् । कु॒रु॒ । इति॑ । य॒ज्ञ्म् । ए॒व । अ॒प॒जित्येत्य॑प-जित्य॑ । पुनः॑ । त॒न्वा॒नाः । य॒न्ति॒ । अङ्गा॑रैः । द्वे इति॑ । सव॑ने॒ इति॑ । वीति॑ । ह॒र॒ति॒ । श॒लाका॑भिः । तृ॒तीय᳚म् । स॒शु॒क्र॒त्वायेति॑ सशुक्र - त्वाय॑ । अथो॒ इति॑ । समिति॑ । भ॒र॒ति॒ । ए॒व । ए॒न॒त् । धिष्णि॑याः । वै । अ॒मुष्मिन्न्॑ । लो॒के । सोम᳚म् । अ॒र॒क्ष॒न्न् । तेभ्यः॑ । अधीति॑ । सोम᳚म् । एति॑ । अ॒ह॒र॒न्न् । तम् । अ॒न्व॒वाय॒न्नित्य॑नु-अ॒वायन्न्॑ । तम् । परीति॑ । अ॒वि॒श॒न्न् । य । ए॒वम् । वेद॑ । वि॒न्दते᳚ ।  \newline




\markright{ TS 6.3.1.3  \hfill https://www.vedavms.in \hfill}

\section{ TS 6.3.1.3 }

\textbf{TS 6.3.1.3 } \newline
\textbf{Samhita Paata} \newline

परिवे॒ष्टारं॒ ते सो॑मपी॒थेन॒ व्या᳚र्द्ध्यन्त॒ ते दे॒वेषु॑ सोमपी॒थमै᳚च्छन्त॒ तान् दे॒वा अ॑ब्रुव॒न् द्वेद्वे॒ नाम॑नी कुरुद्ध्व॒मथ॒ प्र वा॒ऽऽफ्स्यथ॒ न वेत्य॒ग्नयो॒ वा अथ॒ धिष्णि॑या॒स्तस्मा᳚द् द्वि॒नामा᳚ ब्राह्म॒णोऽर्द्धु॑क॒स्तेषां॒ ॅये नेदि॑ष्ठं प॒र्यवि॑श॒न् ते सो॑मपी॒थं प्राऽ*प्नु॑वन्नाहव॒नीय॑ आग्नी॒द्ध्रीयो॑ हो॒त्रीयो॑ मार्जा॒लीय॒स्तस्मा॒त् तेषु॑ जुह्वत्यति॒हाय॒ वष॑ट् करोति॒ वि ह्ये॑- [  ] \newline

\textbf{Pada Paata} \newline

प॒रि॒वे॒ष्टार॒मिति॑ परि - वे॒ष्टार᳚म् । ते । सो॒म॒पी॒थेनेति॑ सोम - पी॒थेन॑ । वीति॑ । आ॒द्‌र्ध्य॒न्त॒ । ते । दे॒वेषु॑ । सो॒म॒पी॒थमिति॑ सोम - पी॒थम् । ऐ॒च्छ॒न्त॒ । तान् । दे॒वाः । अ॒ब्रु॒व॒न्न् । द्वे द्वे॒ इति॒ द्वे-द्वे॒ । नाम॑नी॒ इति॑ । कु॒रु॒द्ध्व॒म् । अथ॑ । प्रेति॑ । वा॒ । आ॒फ्स्यथ॑ । न । वा॒ । इति॑ । अ॒ग्नयः॑ । वै । अथ॑ । धिष्णि॑याः । तस्मा᳚त् । द्वि॒नामेति॑ द्वि - नामा᳚ । ब्रा॒ह्म॒णः । अद्‌र्धु॑कः । तेषा᳚म् । ये । नेदि॑ष्ठम् । प॒र्यवि॑श॒निति॑ परि - अवि॑शन्न् । ते । सो॒म॒पी॒थमिति॑ सोम - पी॒थम् । प्रेति॑ । आ॒प्नु॒व॒न्न् । आ॒ह॒व॒नीय॒ इत्या᳚ - ह॒व॒नीयः॑ । आ॒ग्नी॒द्ध्रीय॒ इत्या᳚ग्नि - इ॒ध्रीयः॑ । हो॒त्रीयः॑ । मा॒र्जा॒लीयः॑ । तस्मा᳚त् । तेषु॑ । जु॒ह्व॒ति॒ । अ॒ति॒हायेत्य॑ति-हाय॑ । वष॑ट् । क॒रो॒ति॒ । वीति॑ । हि ।  \newline




\markright{ TS 6.3.1.4  \hfill https://www.vedavms.in \hfill}

\section{ TS 6.3.1.4 }

\textbf{TS 6.3.1.4 } \newline
\textbf{Samhita Paata} \newline

-ते सो॑मपी॒थेनाऽऽ*र्द्ध्य॑न्त दे॒वा वै याः प्राची॒-राहु॑ती॒-रजु॑हवु॒र्ये पु॒रस्ता॒दसु॑रा॒ आस॒न् ताꣳ स्ताभिः॒ प्राणु॑दन्त॒ याः प्र॒तीची॒र्ये प॒श्चादसु॑रा॒ आस॒न् ताꣳस्ताभि॒-रपा॑नुदन्त॒ प्राची॑र॒न्या आहु॑तयो हू॒यन्ते᳚ प्र॒त्यङ्ङासी॑नो॒ धिष्णि॑या॒न्. व्याघा॑रयति प॒श्चाच्चै॒व पुरस्ता᳚च्च॒ यज॑मानो॒ भ्रातृ॑व्या॒न् प्र णु॑दते॒ तस्मा॒त् परा॑चीः प्र॒जाः प्र वी॑यन्ते प्र॒तीची᳚- [  ] \newline

\textbf{Pada Paata} \newline

ए॒ते । सो॒म॒पी॒थेनेति॑ सोम - पी॒थेन॑ । आद्‌र्ध्य॑न्त । दे॒वाः । वै । याः । प्राचीः᳚ । आहु॑ती॒रित्या - हु॒तीः॒ । अजु॑हवुः । ये । पु॒रस्ता᳚त् । असु॑राः । आसन्न्॑ । तान् । ताभिः॑ । प्रेति॑ । अ॒नु॒द॒न्त॒ । याः । प्र॒तीचीः᳚ । ये । प॒श्चात् । असु॑राः । आसन्न्॑ । तान् । ताभिः॑ । अपेति॑ । अ॒नु॒द॒न्त॒ । प्राचीः᳚ । अ॒न्याः । आहु॑तय॒ इत्या - हु॒त॒यः॒ । हू॒यन्ते᳚ । प्र॒त्यङ् । आसी॑नः । धिष्णि॑यान् । व्याघा॑रय॒तीति॑ वि - आघा॑रयति । प॒श्चात् । च॒ । ए॒व । पु॒रस्ता᳚त् । च॒ । यज॑मानः । भ्रातृ॑व्यान् । प्रेति॑ । नु॒द॒ते॒ । तस्मा᳚त् । परा॑चीः । प्र॒जा इति॑ प्र-जाः । प्रेति॑ । वी॒य॒न्ते॒ । प्र॒तीचीः᳚ ।  \newline




\markright{ TS 6.3.1.5  \hfill https://www.vedavms.in \hfill}

\section{ TS 6.3.1.5 }

\textbf{TS 6.3.1.5 } \newline
\textbf{Samhita Paata} \newline

र्जायन्ते प्रा॒णा वा ए॒ते यद्धिष्णि॑या॒ यद॑द्ध्व॒र्युः प्र॒त्यङ् धिष्णि॑या-नति॒सर्पे᳚त् प्रा॒णान्थ् संक॑र्.षेत् प्र॒मायु॑कः स्या॒न्नाभि॒र्वा ए॒षा य॒ज्ञ्स्य॒ यद्धोतो॒र्द्ध्वः खलु॒ वै नाभ्यै᳚ प्रा॒णोऽवा॑ङपा॒नो यद॑ध्व॒र्युः प्र॒त्यङ् होता॑रमति॒सर्पे॑दपा॒ने प्रा॒णं द॑ध्यात् प्र॒मायु॑कः स्या॒न्नाद्ध्व॒र्युरुप॑ गाये॒द् वाग्वी᳚र्यो॒ वा अ॑द्ध्व॒र्यु-र्यद॑द्ध्व॒र्युरु॑प॒-गाये॑दुद्-गा॒त्रे- [  ] \newline

\textbf{Pada Paata} \newline

जा॒य॒न्ते॒ । प्रा॒णा इति॑ प्र-अ॒नाः । वै । ए॒ते । यत् । धिष्णि॑याः । यत् । अ॒द्ध्व॒र्युः । प्र॒त्यङ् । धिष्णि॑यान् । अ॒ति॒सर्पे॒दित्य॑ति - सर्पे᳚त् । प्रा॒णानिति॑ प्र - अ॒नान् । समिति॑ । क॒र्॒.षे॒त् । प्र॒मायु॑क॒ इति॑ प्र - मायु॑कः । स्या॒त् । नाभिः॑ । वै । ए॒षा । य॒ज्ञ्स्य॑ । यत् । होता᳚ । ऊ॒द्‌र्ध्वः । खलु॑ । वै । नाभ्यै᳚ । प्रा॒ण इति॑ प्र - अ॒नः । अवाङ्॑ । अ॒पा॒न इत्य॑प - अ॒नः । यत् । अ॒द्ध्व॒र्युः । प्र॒त्यङ् । होता॑रम् । अ॒ति॒सर्पे॒दित्य॑ति - सर्पे᳚त् । अ॒पा॒न इत्य॑प - अ॒ने । प्रा॒णमिति॑ प्र - अ॒नम् । द॒द्ध्या॒त् । प्र॒मायु॑क॒ इति॑ प्र - मायु॑कः । स्या॒त् । न । अ॒द्ध्व॒र्युः । उपेति॑ । गा॒ये॒त् । वाग्वी᳚र्य॒ इति॒ वाक् - वी॒र्यः॒ । वै । अ॒द्ध्व॒र्युः । यत् । अ॒द्ध्व॒र्युः । उ॒प॒गाये॒दित्यु॑प - गाये᳚त् । उ॒द्गा॒त्र इत्यु॑त् - गा॒त्रे ।  \newline




\markright{ TS 6.3.1.6  \hfill https://www.vedavms.in \hfill}

\section{ TS 6.3.1.6 }

\textbf{TS 6.3.1.6 } \newline
\textbf{Samhita Paata} \newline

वाचꣳ॒॒ सं प्र य॑च्छे-दुप॒दासु॑काऽस्य॒ वाख् स्या᳚द्ब्रह्मवा॒दिनो॑ वदन्ति॒ नासꣳ॑स्थिते॒ सोमे᳚ऽद्ध्व॒र्युः प्र॒त्यङ्ख् सदोऽती॑या॒दथ॑ क॒था दा᳚क्षि॒णानि॒ होतु॑मेति॒ यामो॒ हि स तेषां॒ कस्मा॒ अह॑ दे॒वा यामं॒ ॅवाऽया॑मं॒ ॅवाऽनु॑ ज्ञास्य॒न्तीत्यु-त्त॑रे॒णाऽऽ*ग्नी᳚द्ध्रं प॒रीत्य॑ जुहोति दाक्षि॒णानि॒ न प्रा॒णान्थ्सं क॑र्.षति॒ न्य॑न्ये धिष्णि॑या उ॒प्यन्ते॒ ( ) नान्ये यान् नि॒वप॑ति॒ तेन॒ तान् प्री॑णाति॒ यान् ननि॒वप॑ति॒ यद॑नुदि॒शति॒ तेन॒ तान् ॥ \newline

\textbf{Pada Paata} \newline

वाच᳚म् । सम् । प्रेति॑ । य॒च्छे॒त् । उ॒प॒दासु॒केत्यु॑प-दासु॑का । अ॒स्य॒ । वाक् । स्या॒त् । ब्र॒ह्म॒वा॒दिन॒ इति॑ ब्रह्म - वा॒दिनः॑ । व॒द॒न्ति॒ । न । असꣳ॑स्थित॒ इत्यसं᳚ - स्थि॒ते॒ । सोमे᳚ । अ॒द्ध्व॒र्युः । प्र॒त्यङ् । सदः॑ । अतीति॑ । इ॒या॒त् । अथ॑ । क॒था । दा॒क्षि॒णानि॑ । होतु᳚म् । ए॒ति॒ । यामः॑ । हि । सः । तेषा᳚म् । कस्मै᳚ । अह॑ । दे॒वाः । याम᳚म् । वा॒ । अया॑मम् । वा॒ । अन्विति॑ । ज्ञा॒स्य॒न्ति॒ । इति॑ । उत्त॑रे॒णेत्युत्- त॒रे॒ण॒ । आग्नी᳚द्ध्र॒मित्याग्नि॑ - इ॒द्ध्र॒म् । प॒रीत्येति॑ परि - इत्य॑ । जु॒हो॒ति॒ । दा॒क्षि॒णानि॑ । न । प्रा॒णानिति॑ प्र - अ॒नान् । समिति॑ । क॒र्॒.ष॒ति॒ । नीति॑ । अ॒न्ये । धिष्णि॑याः । उ॒प्यन्ते᳚ ( ) । न । अ॒न्ये । यान् । नि॒वप॒तीति॑ नि - वप॑ति । तेन॑ । तान् । प्री॒णा॒ति॒ । यान् । न । नि॒वप॒तीति॑ नि - वप॑ति । यत् । अ॒नु॒दि॒शतीत्य॑नु - दि॒शति॑ । तेन॑ । तान् ॥  \newline




\markright{ TS 6.3.2.1  \hfill https://www.vedavms.in \hfill}

\section{ TS 6.3.2.1 }

\textbf{TS 6.3.2.1 } \newline
\textbf{Samhita Paata} \newline

सु॒व॒र्गाय॒ वा ए॒तानि॑ लो॒काय॑ हूयन्ते॒ यद् वै॑सर्ज॒नानि॒ द्वाभ्यां॒ गार्.ह॑पत्ये जुहोति द्वि॒पाद् यज॑मानः॒ प्रति॑ष्ठित्या॒ आग्नी᳚द्ध्रे जुहोत्य॒न्तरि॑क्ष ए॒वाऽऽ*क्र॑मत आहव॒नीये॑ जुहोति सुव॒र्गमे॒वैनं॑ ॅलो॒कं ग॑मयति दे॒वान्. वै सु॑व॒र्गं ॅलो॒कं ॅय॒तो रक्षाꣳ॑स्य जिघाꣳस॒न्ते सोमे॑न॒ राज्ञा॒ रक्षाꣳ॑-स्यप॒हत्या॒प्तु-मा॒त्मानं॑ कृ॒त्वा सु॑व॒र्गं ॅलो॒कमा॑य॒न् रक्ष॑सा॒-मनु॑पलाभा॒या ऽऽ*त्तः॒ सोमो॑ भव॒त्यथ॑- [  ] \newline

\textbf{Pada Paata} \newline

सु॒व॒र्गायेति॑ सुवः - गाय॑ । वै । ए॒तानि॑ । लो॒काय॑ । हू॒य॒न्ते॒ । यत् । वै॒स॒र्ज॒नानि॑ । द्वाभ्या᳚म् । गार्.ह॑पत्य॒ इति॒ गार्.ह॑ - प॒त्ये॒ । जु॒हो॒ति॒ । द्वि॒पादिति॑ द्वि - पात् । यज॑मानः । प्रति॑ष्ठित्या॒ इति॒ प्रति॑ - स्थि॒त्यै॒ । आग्नी᳚द्ध्र॒ इत्याग्नि॑-इ॒द्ध्रे॒ । जु॒हो॒ति॒ । अ॒न्तरि॑क्षे । ए॒व । एति॑ । क्र॒म॒ते॒ । आ॒ह॒व॒नीय॒ इत्या᳚ - ह॒व॒नीये᳚ । जु॒हो॒ति॒ । सु॒व॒र्गमिति॑ सुवः - गम् । ए॒व । ए॒न॒म् । लो॒कम् । ग॒म॒य॒ति॒ । दे॒वान् । वै । सु॒व॒र्गमिति॑ सुवः - गम् । लो॒कम् । य॒तः । रक्षाꣳ॑सि । अ॒जि॒घाꣳ॒॒स॒न्न् । ते । सोमे॑न । राज्ञा᳚ । रक्षाꣳ॑सि । अ॒प॒हत्येत्य॑प - हत्य॑ । अ॒प्तुम् । आ॒त्मान᳚म् । कृ॒त्वा । सु॒व॒र्गमिति॑ सुवः - गम् । लो॒कम् । आ॒य॒न्न् । रक्ष॑साम् । अनु॑पलाभा॒येत्यनु॑प - ला॒भा॒य॒ । आत्तः॑ । सोमः॑ । भ॒व॒ति॒ । अथ॑ ।  \newline




\markright{ TS 6.3.2.2  \hfill https://www.vedavms.in \hfill}

\section{ TS 6.3.2.2 }

\textbf{TS 6.3.2.2 } \newline
\textbf{Samhita Paata} \newline

वैसर्ज॒नानि॑ जुहोति॒ रक्ष॑सा॒मप॑हत्यै॒ त्वꣳ सो॑म तनू॒कृद्भ्य॒ इत्या॑ह तनू॒कृद्ध्य॑ष द्वेषो᳚भ्यो॒ऽन्यकृ॑तेभ्य॒ इत्या॑हा॒न्यकृ॑तानि॒ हि रक्षाꣳ॑स्यु॒रु य॒न्ताऽसि॒ वरू॑थ॒मित्या॑हो॒रु ण॑स्कृ॒धीति॒ वावैतदा॑ह जुषा॒णो अ॒प्तुराज्य॑स्य वे॒त्वित्या॑हा॒प्तुमे॒व यज॑मानं कृ॒त्वा सु॑व॒र्गं ॅलो॒कं ग॑मयति॒ रक्ष॑सा॒-मनु॑पलाभा॒याऽऽ* सोमं॑ ददत॒- [  ] \newline

\textbf{Pada Paata} \newline

वै॒स॒र्ज॒नानि॑ । जु॒हो॒ति॒ । रक्ष॑साम् । अप॑हत्या॒ इत्यप॑ - ह॒त्यै॒ । त्वम् । सो॒म॒ । त॒नू॒कृद्भ्य॒ इति॑ तनू॒कृत् - भ्यः॒ । इति॑ । आ॒ह॒ । त॒नू॒कृदिति॑ तनू - कृत् । हि । ए॒षः । द्वेषो᳚भ्य॒ इति॒ द्वेषः॑ - भ्यः॒ । अ॒न्यकृ॑तेभ्य॒ इत्य॒न्य - कृ॒ते॒भ्यः॒ । इति॑ । आ॒ह॒ । अ॒न्यकृ॑ता॒नीत्य॒न्य - कृ॒ता॒नि॒ । हि । रक्षाꣳ॑सि । उ॒रु । य॒न्ता । अ॒सि॒ । वरू॑थम् । इति॑ । आ॒ह॒ । उ॒रु । नः॒ । कृ॒धि॒ । इति॑ । वाव । ए॒तत् । आ॒ह॒ । जु॒षा॒णः । अ॒प्तुः । आज्य॑स्य । वे॒तु॒ । इति॑ । आ॒ह॒ । अ॒प्तुम् । ए॒व । यज॑मानम् । कृ॒त्वा । सु॒व॒र्गमिति॑ सुवः - गम् । लो॒कम् । ग॒म॒य॒ति॒ । रक्ष॑साम् । अनु॑पलाभा॒येत्यनु॑प - ला॒भा॒य॒ । एति॑ । सोम᳚म् । द॒द॒ते॒ ।  \newline




\markright{ TS 6.3.2.3  \hfill https://www.vedavms.in \hfill}

\section{ TS 6.3.2.3 }

\textbf{TS 6.3.2.3 } \newline
\textbf{Samhita Paata} \newline

आ ग्राव्ण्ण॒ आ वा॑य॒व्या᳚न्या द्रो॑णकल॒शमुत् पत्नी॒मा न॑य॒न्त्यन्वनाꣳ॑सि॒ प्र व॑र्तयन्ति॒ याव॑दे॒वास्यास्ति॒ तेन॑ स॒ह सु॑व॒र्गं ॅलो॒कमे॑ति॒ नय॑वत्य॒र्चाऽऽग्नी᳚द्ध्रे जुहोति सुव॒र्गस्य॑ लो॒कस्या॒भिनी᳚त्यै॒ ग्राव्ण्णो॑ वाय॒व्या॑नि द्रोणकल॒शमाग्नी᳚द्ध्र॒ उप॑ वासयति॒ वि ह्ये॑नं॒ तैर्गृ॒ह्णते॒ यथ् स॒होप॑वा॒सये॑-दपुवा॒येत॑ सौ॒म्यर्चा प्र पा॑दयति॒ स्वयै॒- [  ] \newline

\textbf{Pada Paata} \newline

एति॑ । ग्राव्‌ण्णः॑ । एति॑ । वा॒य॒व्या॑नि । एति॑ । द्रो॒ण॒क॒ल॒शमिति॑ द्रोण - क॒ल॒शम् । उदिति॑ । पत्नी᳚म् । एति॑ । न॒य॒न्ति॒ । अन्विति॑ । अनाꣳ॑सि । प्रेति॑ । व॒र्त॒य॒न्ति॒ । याव॑त् । ए॒व । अ॒स्य॒ । अस्ति॑ । तेन॑ । स॒ह । सु॒व॒र्गमिति॑ सुवः - गम् । लो॒कम् । ए॒ति॒ । नय॑व॒त्येति॒ नय॑ - व॒त्या॒ । ऋ॒चा । आग्नी᳚द्ध्र॒ इत्याग्नि॑ - इ॒द्ध्रे॒ । जु॒हो॒ति॒ । सु॒व॒र्गस्येति॑ सुवः - गस्य॑ । लो॒कस्य॑ । अ॒भिनी᳚त्या॒ इत्य॒भि-नी॒त्यै॒ । ग्राव्‌ण्णः॑ । वा॒य॒व्या॑नि । द्रो॒ण॒क॒ल॒शमिति॑ द्रोण-क॒ल॒शम् । आग्नी᳚द्ध्र॒ इत्याग्नि॑ - इ॒द्ध्रे॒ । उपेति॑ । वा॒स॒य॒ति॒ । वीति॑ । हि । ए॒न॒म् । तैः । गृ॒ह्णते᳚ । यत् । स॒ह । उ॒प॒वा॒सये॒दित्यु॑प - वा॒सये᳚त् । अ॒पु॒वा॒येत॑ । सौ॒म्य । ऋ॒चा । प्रेति॑ । पा॒द॒य॒ति॒ । स्वया᳚ ।  \newline




\markright{ TS 6.3.2.4  \hfill https://www.vedavms.in \hfill}

\section{ TS 6.3.2.4 }

\textbf{TS 6.3.2.4 } \newline
\textbf{Samhita Paata} \newline

-वैनं॑ दे॒वत॑या॒ प्र पा॑दय॒त्यदि॑त्याः॒ सदो॒ऽस्यदि॑त्याः॒ सद॒ आ सी॒देत्या॑ह यथाय॒जुरे॒वैतद्-यज॑मानो॒ वा ए॒तस्य॑ पु॒रा गो॒प्ता भ॑वत्ये॒ष वो॑ देव सवितः॒ सोम॒ इत्या॑ह सवि॒तृप्र॑सूत ए॒वैनं॑ दे॒वता᳚भ्यः॒ सं प्रय॑च्छत्ये॒तत् त्वꣳ सो॑म दे॒वो दे॒वानुपा॑गा॒ इत्या॑ह दे॒वो ह्ये॑ष सन्- [  ] \newline

\textbf{Pada Paata} \newline

ए॒व । ए॒न॒म् । दे॒वत॑या । प्रेति॑ । पा॒द॒य॒ति॒ । अदि॑त्याः । सदः॑ । अ॒सि॒ । अदि॑त्याः । सदः॑ । एति॑ । सी॒द॒ । इति॑ । आ॒ह॒ । य॒था॒य॒जुरिति॑ यथा-य॒जुः । ए॒व । ए॒तत् । यज॑मानः । वै । ए॒तस्य॑ । पु॒रा । गो॒प्ता । भ॒व॒ति॒ । ए॒षः । वः॒ । दे॒व॒ । स॒वि॒तः॒ । सोमः॑ । इति॑ । आ॒ह॒ । स॒वि॒तृप्र॑सूत॒ इति॑ सवि॒तृ-प्र॒सू॒तः॒ । ए॒व । ए॒न॒म् । दे॒वता᳚भ्यः । सम् । प्रेति॑ । य॒च्छ॒ति॒ । ए॒तत् । त्वम् । सो॒म॒ । दे॒वः । दे॒वान् । उपेति॑ । अ॒गाः॒ । इति॑ । आ॒ह॒ । दे॒वः । हि । ए॒षः । सन्न् ।  \newline




\markright{ TS 6.3.2.5  \hfill https://www.vedavms.in \hfill}

\section{ TS 6.3.2.5 }

\textbf{TS 6.3.2.5 } \newline
\textbf{Samhita Paata} \newline

दे॒वानु॒पैती॒दम॒हं म॑नु॒ष्यो॑ मनु॒ष्या॑नित्या॑ह मनु॒ष्यो᳚(1॒) ह्ये॑ष सन् म॑नु॒ष्या॑नु॒पैति॒ यदे॒तद् यजु॒र्न ब्रू॒यादप्र॑जा अप॒शुर्यज॑मानः स्याथ् स॒ह प्र॒जया॑ सह रा॒यस्पोषे॒णेत्या॑ह प्र॒जयै॒व प॒शुभिः॑ स॒हेमं ॅलो॒कमु॒पाव॑र्तते॒ नमो॑ दे॒वेभ्य॒ इत्या॑ह नमस्का॒रो हि दे॒वानाꣳ॑ स्व॒धा पि॒तृभ्य॒ इत्या॑ह स्वधाका॒रो हि- [  ] \newline

\textbf{Pada Paata} \newline

दे॒वान् । उ॒पैतीत्यु॑प - एति॑ । इ॒दम् । अ॒हम् । म॒नु॒ष्यः॑ । म॒नु॒ष्यान्॑ । इति॑ । आ॒ह॒ । म॒नु॒ष्यः॑ । हि । ए॒षः । सन्न् । म॒नु॒ष्यान्॑ । उ॒पैतीत्यु॑प - एति॑ । यत् । ए॒तत् । यजुः॑ । न । ब्रू॒यात् । अप्र॑जा॒ इत्यप्र॑ - जाः॒ । अ॒प॒शुः । यज॑मानः । स्या॒त् । स॒ह । प्र॒जयेति॑ प्र - जया᳚ । स॒ह । रा॒यः । पोषे॑ण । इति॑ । आ॒ह॒ । प्र॒जयेति॑ प्र - जया᳚ । ए॒व । प॒शुभि॒रिति॑ प॒शु-भिः॒ । स॒ह । इ॒मम् । लो॒कम् । उ॒पाव॑र्तत॒ इत्यु॑प - आव॑र्तते । नमः॑ । दे॒वेभ्यः॑ । इति॑ । आ॒ह॒ । न॒म॒स्का॒र इति॑ नमः - का॒रः । हि । दे॒वाना᳚म् । स्व॒धेति॑ स्व - धा । पि॒तृभ्य॒ इति॑ पि॒तृ - भ्यः॒ । इति॑ । आ॒ह॒ । स्व॒धा॒का॒र इति॑ स्वधा - का॒रः । हि ।  \newline




\markright{ TS 6.3.2.6  \hfill https://www.vedavms.in \hfill}

\section{ TS 6.3.2.6 }

\textbf{TS 6.3.2.6 } \newline
\textbf{Samhita Paata} \newline

पि॑तृ॒णामि॒दम॒हं निर्वरु॑णस्य॒ पाशा॒दित्या॑ह वरुणपा॒शादे॒व निर्मु॑च्य॒ते ऽग्ने᳚ व्रतपत आ॒त्मनः॒ पूर्वा॑ त॒नूरा॒देयेत्या॑हुः॒ को हि तद्वेद॒ यद् वसी॑या॒न्थ् स्वे वशे॑ भू॒ते पुन॑र्वा॒ ददा॑ति॒ न वेति॒ ग्रावा॑णो॒ वै सोम॑स्य॒ राज्ञो॑ मलिम्लुसे॒ना य ए॒वं ॅवि॒द्वान् ग्राव्ण्ण॒॒ आग्नी᳚द्ध्र उपवा॒सय॑ति॒ नैनं॑ मलिम्लुसे॒ना वि॑न्दति ( ) ॥ \newline

\textbf{Pada Paata} \newline

पि॒तृ॒णाम् । इ॒दम् । अ॒हम् । निरिति॑ । वरु॑णस्य । पाशा᳚त् । इति॑ । आ॒ह॒ । व॒रु॒ण॒पा॒शादिति॑ वरुण - पा॒शात् । ए॒व । निरिति॑ । मु॒च्य॒ते॒ । अग्ने᳚ । व्र॒त॒प॒त॒ इति॑ व्रत - प॒ते॒ । आ॒त्मनः॑ । पूर्वा᳚ । त॒नूः । आ॒देयेत्या᳚ - देया᳚ । इति॑ । आ॒हुः॒ । कः । हि । तत् । वेद॑ । यत् । वसी॑यान् । स्वे । वशे᳚ । भू॒ते । पुनः॑ । वा॒ । ददा॑ति । न । वा॒ । इति॑ । ग्रावा॑णः । वै । सोम॑स्य । राज्ञ्ः॑ । म॒लि॒म्लु॒से॒नेति॑ मलिम्लु - से॒ना । यः । ए॒वम् । वि॒द्वान् । ग्राव्‌ण्णः॑  । आग्नी᳚द्ध्र॒ इत्याग्नि॑ - इ॒द्ध्रे॒ । उ॒प॒वा॒सय॒तीत्यु॑प - वा॒सय॑ति । न । ए॒न॒म् । म॒लि॒म्लु॒से॒नेति॑ मलिम्लु - से॒ना । वि॒न्द॒ति॒ ( ) ॥  \newline




\markright{ TS 6.3.3.1  \hfill https://www.vedavms.in \hfill}

\section{ TS 6.3.3.1 }

\textbf{TS 6.3.3.1 } \newline
\textbf{Samhita Paata} \newline

वै॒ष्ण॒व्यर्चा हु॒त्वा यूप॒मच्छै॑ति वैष्ण॒वो वै दे॒वत॑या॒ यूपः॒ स्वयै॒वैनं॑ दे॒वत॒या ऽच्छै॒त्यत्य॒न्यानगां॒ नान्या-नुपा॑गा॒मित्या॒हाति॒ ह्य॑न्यानेति॒ नान्या-नु॒पैत्य॒र्वाक्त्वा॒ परै॑रविदं प॒रोऽव॑रै॒रित्या॑हा॒र्वाग्घ्ये॑नं॒ परै᳚र्वि॒न्दति॑ प॒रोऽव॑रै॒स्तं त्वा॑ जुषे- [  ] \newline

\textbf{Pada Paata} \newline

वै॒ष्ण॒व्या । ऋ॒चा । हु॒त्वा । यूप᳚म् । अच्छ॑ । ए॒ति॒ । वै॒ष्ण॒वः । वै । दे॒वत॑या । यूपः॑ । स्वया᳚ । ए॒व । ए॒न॒म् । दे॒वत॑या । अच्छ॑ । ए॒ति॒ । अतीति॑ । अ॒न्यान् । अगा᳚म् । न । अ॒न्यान् । उपेति॑ । अ॒गा॒म् । इति॑ । आ॒ह॒ । अतीति॑ । हि । अ॒न्यान् । एति॑ । न । अ॒न्यान् । उ॒पैतीत्यु॑प-एति॑ । अ॒र्वाक् । त्वा॒ । परैः᳚ । अ॒वि॒द॒म् । प॒रः । अव॑रैः । इति॑ । आ॒ह॒ । अ॒र्वाक् । हि । ए॒न॒म् । परैः᳚ । वि॒न्दति॑ । प॒रः । अव॑रैः । तम् । त्वा॒ । जु॒षे॒ ।  \newline




\markright{ TS 6.3.3.2  \hfill https://www.vedavms.in \hfill}

\section{ TS 6.3.3.2 }

\textbf{TS 6.3.3.2 } \newline
\textbf{Samhita Paata} \newline

वैष्ण॒वं दे॑वय॒ज्याया॒ इत्या॑ह देवय॒ज्यायै॒ ह्ये॑नं जु॒षते॑ दे॒वस्त्वा॑ सवि॒ता मद्ध्वा॑न॒त्तिवत्या॑ह॒ तेज॑सै॒वैन॑-मन॒क्त्योष॑धे॒ त्राय॑स्वैनꣳ॒॒ स्वधि॑ते॒ मैनꣳ॑ हिꣳसी॒रित्या॑ह॒ वज्रो॒ वै स्वधि॑तिः॒ शान्त्यै॒ स्वधि॑तेर्वृ॒क्षस्य॒ बिभ्य॑तः प्रथ॒मेन॒ शक॑लेन स॒ह तेजः॒ परा॑ पतति॒ यः प्र॑थ॒मः शक॑लः परा॒पते॒त् तमप्या ह॑रे॒थ् सते॑जस- [  ] \newline

\textbf{Pada Paata} \newline

वै॒ष्ण॒वम् । दे॒व॒य॒ज्याया॒ इति॑ देव - य॒ज्यायै᳚ । इति॑ । आ॒ह॒ । दे॒व॒य॒ज्याया॒ इति॑ देव-य॒ज्यायै᳚ । हि । ए॒न॒म् । जु॒षते᳚ । दे॒वः । त्वा॒ । स॒वि॒ता । मद्ध्वा᳚ । अ॒न॒क्तु॒ । इति॑ । आ॒ह॒ । तेज॑सा । ए॒व । ए॒न॒म् । अ॒न॒क्ति॒ । ओष॑धे । त्राय॑स्व । ए॒न॒म् । स्वधि॑त॒ इति॒ स्व-धि॒ते॒ । मा । ए॒न॒म् । हिꣳ॒॒सीः॒ । इति॑ । आ॒ह॒ । वज्रः॑ । वै । स्वधि॑ति॒रिति॒ स्व - धि॒तिः॒ । शान्त्यै᳚ । स्वधि॑ते॒रिति॒ स्व - धि॒तेः॒ । वृ॒क्षस्य॑ । बिभ्य॑तः । प्र॒थ॒मेन॑ । शक॑लेन । स॒ह । तेजः॑ । परेति॑ । प॒त॒ति॒ । यः । प्र॒थ॒मः । शक॑लः । प॒रा॒पते॒दिति॑ परा - पते᳚त् । तम् । अपि॑ । एति॑ । ह॒रे॒त् । सते॑जस॒मिति॒ स - ते॒ज॒स॒म् ।  \newline




\markright{ TS 6.3.3.3  \hfill https://www.vedavms.in \hfill}

\section{ TS 6.3.3.3 }

\textbf{TS 6.3.3.3 } \newline
\textbf{Samhita Paata} \newline

-मे॒वैन॒मा ह॑रती॒मे वै लो॒का यूपा᳚त् प्रय॒तो बि॑भ्यति॒ दिव॒मग्रे॑ण॒ मा ले॑खीर॒न्तरि॑क्षं॒ मद्ध्ये॑न॒ मा हिꣳ॑सी॒रित्या॑है॒भ्य ए॒वैनं॑ ॅलो॒केभ्यः॑ शमयति॒ वन॑स्पते श॒तव॑ल्.शो॒ वि रो॒हेत्या॒व्रश्च॑ने जुहोति॒ तस्मा॑-दा॒व्रश्च॑नाद्-वृ॒क्षाणां॒ भूयाꣳ॑स॒ उत्ति॑ष्ठन्ति स॒हस्र॑वल्.शा॒ वि व॒यꣳ रु॑हे॒मेत्या॑हा॒- ऽऽ*शिष॑मे॒वैतामा शा॒स्ते ऽन॑क्षसङ्गं- [  ] \newline

\textbf{Pada Paata} \newline

ए॒व । ए॒न॒म् । एति॑ । ह॒र॒ति॒ । इ॒मे । वै । लो॒काः । यूपा᳚त् । प्र॒य॒त इति॑ प्र - य॒तः । बि॒भ्य॒ति॒ । दिव᳚म् । अग्रे॑ण । मा । ले॒खीः॒ । अ॒न्तरि॑क्षम् । मद्ध्ये॑न । मा । हिꣳ॒॒सीः॒ । इति॑ । आ॒ह॒ । ए॒भ्यः । ए॒व । ए॒न॒म् । लो॒केभ्यः॑ । श॒म॒य॒ति॒ । वन॑स्पते । श॒तव॑ल्.श॒ इति॑ श॒त - व॒ल॒.शः॒ । वीति॑ । रो॒ह॒ । इति॑ । आ॒व्रश्च॑न॒ इत्या᳚ - व्रश्च॑ने । जु॒हो॒ति॒ । तस्मा᳚त् । आ॒व्रश्च॑ना॒दित्या᳚ - व्रश्च॑नात् । वृ॒क्षाणा᳚म् । भूयाꣳ॑सः । उदिति॑ । ति॒ष्ठ॒न्ति॒ । स॒हस्र॑वल्.शा॒ इति॑ स॒हस्र॑ - व॒ल॒.शाः॒ । वीति॑ । व॒यम् । रु॒हे॒म॒ । इति॑ । आ॒ह॒ । आ॒शिष॒मित्या᳚-शिष᳚म् । ए॒व । ए॒ताम् । एति॑ । शा॒स्ते॒ । अन॑क्षसङ्ग॒मित्यन॑क्ष - स॒ङ्गम् ।  \newline




\markright{ TS 6.3.3.4  \hfill https://www.vedavms.in \hfill}

\section{ TS 6.3.3.4 }

\textbf{TS 6.3.3.4 } \newline
\textbf{Samhita Paata} \newline

ॅवृश्चे॒द्-यद॑क्षस॒ङ्गं ॅवृ॒श्चेद॑धई॒षं ॅयज॑मानस्य प्र॒मायु॑कꣳ स्या॒द्यं का॒मये॒ताप्र॑तिष्ठितः स्या॒दित्या॑रो॒हं तस्म॑ वृश्चेदे॒ष वै वन॒स्पती॑ना॒-मप्र॑तिष्ठि॒तोऽप्र॑तिष्ठित ए॒व भ॑वति॒ यं का॒मये॑ताप॒शुः स्या॒दित्य॑प॒र्णं तस्मै॒ शुष्का᳚ग्रं ॅवृश्चेदे॒ष वै वन॒स्पती॑ना-मपश॒व्यो॑ऽप॒शुरे॒व भ॑वति॒ यं का॒मये॑त पशु॒मान्थ् स्या॒दिति॑ बहुप॒र्णं तस्मै॑ बहुशा॒खं ॅवृ॑श्चेदे॒ष वै- [  ] \newline

\textbf{Pada Paata} \newline

वृ॒श्चे॒त् । यत् । अ॒क्ष॒स॒ङ्गमित्य॑क्ष - स॒ङ्गम् । वृ॒श्चेत् । अ॒ध॒ई॒षमित्य॑धः - ई॒षम् । यज॑मानस्य । प्र॒मायु॑क॒मिति॑ प्र-मायु॑कम् । स्या॒त् । यम् । का॒मये॑त । अप्र॑तिष्ठित॒ इत्यप्र॑ति - स्थि॒तः॒ । स्या॒त् । इति॑ । आ॒रो॒हमित्या᳚ - रो॒हम् । तस्मै᳚ । वृ॒श्चे॒त् । ए॒षः । वै । वन॒स्पती॑नाम् । अप्र॑तिष्ठित॒ इत्यप्र॑ति - स्थि॒तः॒ । अप्र॑तिष्ठित॒ इत्यप्र॑ति - स्थि॒तः॒ । ए॒व । भ॒व॒ति॒ । यम् । का॒मये॑त । अ॒प॒शुः । स्या॒त् । इति॑ । अ॒प॒र्णम् । तस्मै᳚ । शुष्का᳚ग्र॒मिति॒ शुष्क॑ - अ॒ग्र॒म् । वृ॒श्चे॒त् । ए॒षः । वै । वन॒स्पती॑नाम् । अ॒प॒श॒व्यः । अ॒प॒शुः । ए॒व । भ॒व॒ति॒ । यम् । का॒मये॑त । प॒शु॒मानिति॑ पशु - मान् । स्या॒त् । इति॑ । ब॒हु॒प॒र्णमिति॑ बहु - प॒र्णम् । तस्मै᳚ । ब॒हु॒शा॒खमिति॑ बहु - शा॒खम् । वृ॒श्चे॒त् । ए॒षः । वै ।  \newline




\markright{ TS 6.3.3.5  \hfill https://www.vedavms.in \hfill}

\section{ TS 6.3.3.5 }

\textbf{TS 6.3.3.5 } \newline
\textbf{Samhita Paata} \newline

वन॒स्पती॑नां पश॒व्यः॑ पशु॒माने॒व भ॑वति॒ प्रति॑ष्ठितं ॅवृश्चेत् प्रति॒ष्ठाका॑मस्यै॒ष वै वन॒स्पती॑नां॒ प्रति॑ष्ठितो॒ यः स॒मे भूम्यै॒ स्वाद्योने॑ रू॒ढः प्रत्ये॒व ति॑ष्ठति॒ यः प्र॒त्यङ्ङुप॑नत॒स्तं ॅवृ॑श्चे॒थ् स हि मेध॑म॒भ्युप॑नतः॒ पञ्चा॑रत्निं॒ तस्मै॑ वृश्चे॒द्यं का॒मये॒तोपै॑न॒मुत्त॑रो य॒ज्ञो न॑मे॒दिति॒ पञ्चा᳚क्षरा प॒ङ्क्तिः पाङ्क्तो॑ य॒ज्ञ् उपै॑न॒मुत्त॑रो य॒ज्ञो- [  ] \newline

\textbf{Pada Paata} \newline

वन॒स्पती॑नाम् । प॒श॒व्यः॑ । प॒शु॒मानिति॑ पशु - मान् । ए॒व । भ॒व॒ति॒ । प्रति॑ष्ठित॒मिति॒ प्रति॑ - स्थि॒त॒म् । वृ॒श्चे॒त् । प्र॒ति॒ष्ठाका॑म॒स्येति॑ प्रति॒ष्ठा- का॒म॒स्य॒ । ए॒षः । वै । वन॒स्पती॑नाम् । प्रति॑ष्ठित॒ इति॒ प्रति॑ - स्थि॒तः॒ । यः । स॒मे । भूम्यै᳚ । स्वात् । योनेः᳚ । रू॒ढः । प्रतीति॑ । ए॒व । ति॒ष्ठ॒ति॒ । यः । प्र॒त्यङ् । उप॑नत॒ इत्युप॑-न॒तः॒ । तम् । वृ॒श्चे॒त् । सः । हि । मेध᳚म् । अ॒भीति॑ । उप॑नत॒ इत्युप॑ - न॒तः॒ । पञ्चा॑रत्नि॒मिति॒ पञ्च॑- अ॒र॒त्नि॒म् । तस्मै᳚ । वृ॒श्चे॒त् । यम् । का॒मये॑त । उपेति॑ । ए॒न॒म् । उत्त॑र॒ इत्युत् - त॒रः॒ । य॒ज्ञ्ः । न॒मे॒त् । इति॑ । पञ्चा᳚क्ष॒रेति॒ पञ्च॑-आ॒क्ष॒रा॒ । प॒ङ्क्तिः । पाङ्क्तः॑ । य॒ज्ञ्ः । उपेति॑ । ए॒न॒म् । उत्त॑र॒ इत्युत् - त॒रः॒ । य॒ज्ञ्ः ।  \newline




\markright{ TS 6.3.3.6  \hfill https://www.vedavms.in \hfill}

\section{ TS 6.3.3.6 }

\textbf{TS 6.3.3.6 } \newline
\textbf{Samhita Paata} \newline

न॑मति॒ षड॑रत्निं प्रति॒ष्ठाका॑मस्य॒ षड्वा ऋ॒तव॑ ऋ॒तुष्वे॒व प्रति॑ तिष्ठति स॒प्तार॑त्निं प॒शुका॑मस्य स॒प्तप॑दा॒ शक्व॑री प॒शवः॒ शक्व॑री प॒शूने॒वाव॑ रुन्धे॒ नवा॑रत्निं॒ तेज॑स्कामस्य त्रि॒वृता॒ स्तोमे॑न॒ सम्मि॑तं॒ तेज॑स्त्रि॒वृत् ते॑ज॒स्व्ये॑व भ॑व॒-त्येका॑दशारत्नि-मिन्द्रि॒यका॑म॒-स्यैका॑दशाक्षरा त्रि॒ष्टुगि॑न्द्रि॒यं त्रि॒ष्टुगि॑न्द्रिया॒व्ये॑व भ॑वति॒ पञ्च॑दशारत्निं॒ भ्रातृ॑व्यवतः पञ्चद॒शो वज्रो॒ भ्रातृ॑व्याभिभूत्यै स॒प्तद॑शारत्निं प्र॒जाका॑मस्य सप्तद॒शः प्र॒जाप॑तिः प्र॒जाप॑ते॒राप्त्या॒ ( ) एक॑विꣳशत्यरत्निं प्रति॒ष्ठाका॑म-स्यैकविꣳ॒॒शः स्तोमा॑नां प्रति॒ष्ठा प्रति॑ष्ठित्या अ॒ष्टाश्रि॑र्भव-त्य॒ष्टाक्ष॑रा गाय॒त्री तेजो॑ गाय॒त्री गा॑य॒त्री य॑ज्ञ्मु॒खं तेज॑सै॒व गा॑यत्रि॒या य॑ज्ञ्मु॒खेन॒ संमि॑तः ॥ \newline

\textbf{Pada Paata} \newline

न॒म॒ति॒ । षड॑रत्नि॒मिति॒ षट् - अ॒र॒त्नि॒म् । प्र॒ति॒ष्ठाका॑म॒स्येति॑ प्रति॒ष्ठा- का॒म॒स्य॒ । षट् । वै । ऋ॒तवः॑ । ऋ॒तुषु॑ । ए॒व । प्रतीति॑ । ति॒ष्ठ॒ति॒ । स॒प्तार॑त्नि॒मिति॑ स॒प्त - अ॒र॒त्नि॒म् । प॒शुका॑म॒स्येति॑ प॒शु - का॒म॒स्य॒ । स॒प्तप॒देति॑ स॒प्त-प॒दा॒ । शक्व॑री । प॒शवः॑ । शक्व॑री । प॒शून् । ए॒व । अवेति॑ । रु॒न्धे॒ । नवा॑रत्नि॒मिति॒ नव॑ - अ॒र॒त्नि॒म् । तेज॑स्काम॒स्येति॒ तेजः॑ - का॒म॒स्य॒ । त्रि॒वृतेति॑ त्रि - वृता᳚ । स्तोमे॑न । सम्मि॑त॒मिति॒ सं - मि॒त॒म् । तेजः॑ । त्रि॒वृदिति॑ त्रि - वृत् । ते॒ज॒स्वी । ए॒व । भ॒व॒ति॒ । एका॑दशारत्नि॒मित्येका॑दश-अ॒र॒त्नि॒म् । इ॒न्द्रि॒यका॑म॒स्येती᳚न्द्रि॒य-का॒म॒स्य॒ । एका॑दशाक्ष॒रेत्येका॑दश - अ॒क्ष॒रा॒ । त्रि॒ष्टुक् । इ॒न्द्रि॒यम् । त्रि॒ष्टुक् । इ॒न्द्रि॒या॒वी । ए॒व । भ॒व॒ति॒ । पञ्च॑दशारत्नि॒मिति॒ पञ्च॑दश - अ॒र॒त्नि॒म् । भ्रातृ॑व्यवत॒ इति॒ भ्रातृ॑व्य - व॒तः॒ । प॒ञ्च॒द॒श इति॑ पञ्च - द॒शः । वज्रः॑ । भ्रातृ॑व्याभिभूत्या॒ इति॒ भ्रातृ॑व्य - अ॒भि॒भू॒त्यै॒ । स॒प्तद॑शारत्नि॒मिति॑ स॒प्तद॑श - अ॒र॒त्नि॒म् । प्र॒जाका॑म॒स्येति॑ प्र॒जा - का॒म॒स्य॒ । स॒प्त॒द॒श इति॑ सप्त - द॒शः । प्र॒जाप॑ति॒रिति॑ प्र॒जा-प॒तिः॒ । प्र॒जाप॑ते॒रिति॑ प्र॒जा - प॒तेः॒ । आप्त्यै᳚ ( ) । एक॑विꣳशत्यरत्नि॒मित्येक॑विꣳशति - अ॒र॒त्नि॒म् । प्र॒ति॒ष्ठाका॑म॒स्येति॑ प्रति॒ष्ठा - का॒म॒स्य॒ । ए॒क॒विꣳ॒॒श इत्ये॑क - विꣳ॒॒शः । स्तोमा॑नाम् । प्र॒ति॒ष्ठेति॑ प्रति - स्था । प्रति॑ष्ठित्या॒ इति॒ प्रति॑ - स्थि॒त्यै॒ । अ॒ष्टाश्रि॒रित्य॒ष्टा - अ॒श्रिः॒ । भ॒व॒ति॒ । अ॒ष्टाक्ष॒रेत्य॒ष्टा - अ॒क्ष॒रा॒ । गा॒य॒त्री । तेजः॑ । गा॒य॒त्री । गा॒य॒त्री । य॒ज्ञ्॒मु॒खमिति॑ यज्ञ् - मु॒खम् । तेज॑सा । ए॒व । गा॒य॒त्रि॒या । य॒ज्ञ्॒मु॒खेनेति॑ यज्ञ् - मु॒खेन॑ । संमि॑त॒ इति॒ सं - मि॒तः॒ ॥  \newline




\markright{ TS 6.3.4.1  \hfill https://www.vedavms.in \hfill}

\section{ TS 6.3.4.1 }

\textbf{TS 6.3.4.1 } \newline
\textbf{Samhita Paata} \newline

पृ॒थि॒व्यै त्वा॒ऽन्तरि॑क्षाय त्वा दि॒वे त्वेत्या॑है॒भ्य ए॒वैनं॑ ॅलो॒केभ्यः॒ प्रोक्ष॑ति॒ परा᳚ञ्चं॒ प्रोक्ष॑ति॒ परा॑ङिव॒ हि सु॑व॒र्गो लो॒कः क्रू॒रमि॑व॒ वा ए॒तत् क॑रोति॒ यत् खन॑त्य॒पोऽव॑ नयति॒ शान्त्यै॒ यव॑मती॒रव॑ नय॒त्यूर्ग्वै यवो॒ यज॑मानेन॒ यूपः॒ संमि॑तो॒ यावा॑ने॒व यज॑मान॒-स्ताव॑ती-मे॒वास्मि॒-न्नूर्जं॑ दधाति- [  ] \newline

\textbf{Pada Paata} \newline

पृ॒थि॒व्यै । त्वा॒ । अ॒न्तरि॑क्षाय । त्वा॒ । दि॒वे । त्वा॒ । इति॑ । आ॒ह॒ । ए॒भ्यः । ए॒व । ए॒न॒म् । लो॒केभ्यः॑ । प्रेति॑ । उ॒क्ष॒ति॒ । परा᳚ञ्चम् । प्रेति॑ । उ॒क्ष॒ति॒ । पराङ्॑ । इ॒व॒ । हि । सु॒व॒र्ग इति॑ सुवः-गः । लो॒कः । क्रू॒रम् । इ॒व॒ । वै । ए॒तत् । क॒रो॒ति॒ । यत् । खन॑ति । अ॒पः । अवेति॑ । न॒य॒ति॒ । शान्त्यै᳚ । यव॑मती॒रिति॒ यव॑-म॒तीः॒ । अवेति॑ । न॒य॒ति॒ । ऊर्क् । वै । यवः॑ । यज॑मानेन । यूपः॑ । संमि॑त॒ इति॒ सं - मि॒तः॒ । यावान्॑ । ए॒व । यज॑मानः । ताव॑तीम् । ए॒व । अ॒स्मि॒न्न् । ऊर्ज᳚म् । द॒धा॒ति॒ ।  \newline




\markright{ TS 6.3.4.2  \hfill https://www.vedavms.in \hfill}

\section{ TS 6.3.4.2 }

\textbf{TS 6.3.4.2 } \newline
\textbf{Samhita Paata} \newline

पितृ॒णाꣳ सद॑नम॒सीति॑ ब॒र्॒.हिरव॑ स्तृणाति पितृदेव॒त्या᳚(1॒)ꣳ॒ ह्ये॑तद्-यन्निखा॑तं॒ ॅयद्-ब॒र्॒.हिरन॑वस्तीर्य मिनु॒यात् पि॑तृदेव॒त्यो॑ निखा॑तः स्याद् ब॒र्॒.हिर॑व॒स्तीर्य॑ मिनोत्य॒स्यामे॒वैनं॑ मिनोति यूपशक॒लमवा᳚स्यति॒ सते॑जसमे॒वैनं॑ मिनोति दे॒वस्त्वा॑ सवि॒ता मद्ध्वा॑ऽन॒क्त्वित्या॑ह॒ तेज॑सै॒वैन॑मनक्ति सुपिप्प॒लाभ्य॒-स्त्वौष॑धीभ्य॒ इति॑ च॒षालं॒ प्रति॑- [  ] \newline

\textbf{Pada Paata} \newline

पि॒तृ॒णाम् । सद॑नम् । अ॒सि॒ । इति॑ । ब॒र्॒.हिः । अवेति॑ । स्तृ॒णा॒ति॒ । पि॒तृ॒दे॒व॒त्य॑मिति॑ पितृ - दे॒व॒त्य᳚म् । हि । ए॒तत् । यत् । निखा॑त॒मिति॒ नि - खा॒त॒म् । यत् । ब॒र॒.हिः । अन॑वस्ती॒र्येत्यन॑व - स्ती॒र्य॒ । मि॒नु॒यात् । पि॒तृ॒दे॒वत्य॑ इति॑ पितृ-दे॒व॒त्यः॑ । निखा॑त॒ इति॒ नि-खा॒तः॒ । स्या॒त् । ब॒र॒.हिः । अ॒व॒स्तीर्येत्य॑व - स्तीर्य॑ । मि॒नो॒ति॒ । अ॒स्याम् । ए॒व । ए॒न॒म् । मि॒नो॒ति॒ । यू॒प॒श॒क॒लमिति॑ यूप - श॒क॒लम् । अवेति॑ । अ॒स्य॒ति॒ । सते॑जस॒मिति॒ स - ते॒ज॒स॒म् । ए॒व । ए॒न॒म् । मि॒नो॒ति॒ । दे॒वः । त्वा॒ । स॒वि॒ता । मद्ध्वा᳚ । अ॒न॒क्तु॒ । इति॑ । आ॒ह॒ । तेज॑सा । ए॒व । ए॒न॒म् । अ॒न॒क्ति॒ । सु॒पि॒प्प॒लाभ्य॒ इति॑ सु-पि॒प्प॒लाभ्यः॑ । त्वा॒ । ओष॑धीभ्य॒ इत्योष॑धि - भ्यः॒ । इति॑ । च॒षाल᳚म् । प्रतीति॑ ।  \newline




\markright{ TS 6.3.4.3  \hfill https://www.vedavms.in \hfill}

\section{ TS 6.3.4.3 }

\textbf{TS 6.3.4.3 } \newline
\textbf{Samhita Paata} \newline

मुञ्चति॒ तस्मा᳚च्छीर्.ष॒त ओष॑धयः॒ फलं॑ गृह्णन्त्य॒नक्ति॒ तेजो॒ वा आज्यं॒ ॅयज॑मानेनाग्नि॒ष्ठाऽश्रिः॒ संमि॑ता॒ यद॑ग्नि॒ष्ठा-मश्रि॑म॒नक्ति॒ यज॑मानमे॒व तेज॑सा ऽनक्त्या॒न्त-म॑नक्त्या॒न्तमे॒व यज॑मानं॒ तेज॑सानक्ति स॒र्वतः॒ परि॑ मृश॒त्यप॑रिवर्ग-मे॒वास्मि॒न् तेजो॑ दधा॒त्युद् दिवꣳ॑ स्तभा॒नाऽन्तरि॑क्षं पृ॒णेत्या॑है॒षां ॅलो॒कानां॒ ॅविधृ॑त्यै वैष्ण॒व्यर्चा- [  ] \newline

\textbf{Pada Paata} \newline

मु॒ञ्च॒ति॒ । तस्मा᳚त् । शी॒र्॒.ष॒तः । ओष॑धयः । फल᳚म् । गृ॒ह्ण॒न्ति॒ । अ॒नक्ति॑ । तेजः॑ । वै । आज्य᳚म् । यज॑मानेन । अ॒ग्नि॒ष्ठेत्य॑ग्नि-स्था । अश्रिः॑ । संमि॒तेति॒ सं - मि॒ता॒ । यत् । अ॒ग्नि॒ष्ठामित्य॑ग्नि- स्थाम् । अश्रि᳚म् । अ॒न॒क्ति॒ । यज॑मानम् । ए॒व । तेज॑सा । अ॒न॒क्ति॒ । आ॒न्तमित्या᳚ - अ॒न्तम् । अ॒न॒क्ति॒ । आ॒न्तमित्या᳚ - अ॒न्तम् । ए॒व । यज॑मानम् । तेज॑सा । अ॒न॒क्ति॒ । स॒र्वतः॑ । परीति॑ । मृ॒श॒ति॒ । अप॑रिवर्ग॒मित्यप॑रि - व॒र्ग॒म् । ए॒व । अ॒स्मि॒न्न् । तेजः॑ । द॒धा॒ति॒ । उदिति॑ । दिव᳚म् । स्त॒भा॒न॒ । एति॑ । अ॒न्तरि॑क्षम् । पृ॒ण॒ । इति॑ । आ॒ह॒ । ए॒षाम् । लो॒काना᳚म् । विधृ॑त्या॒ इति॒ वि - धृ॒त्यै॒ । वै॒ष्ण॒व्या । ऋ॒चा ।  \newline




\markright{ TS 6.3.4.4  \hfill https://www.vedavms.in \hfill}

\section{ TS 6.3.4.4 }

\textbf{TS 6.3.4.4 } \newline
\textbf{Samhita Paata} \newline

क॑ल्पयति वैष्ण॒वो वै दे॒वत॑या॒ यूपः॒ स्वयै॒वैनं॑ दे॒वत॑या कल्पयति॒ द्वाभ्यां᳚ कल्पयति द्वि॒पाद् यज॑मानः॒ प्रति॑ष्ठित्यै॒ यं का॒मये॑त॒ तेज॑सैनं दे॒वता॑भिरिन्द्रि॒येण॒ व्य॑र्द्धयेय॒-मित्य॑ग्नि॒ष्ठां तस्याश्रि॑-माहव॒नीया॑दि॒त्थं ॅवे॒त्थं ॅवाऽति॑ नावये॒त् तेज॑सै॒वैनं॑ दे॒वता॑भिरिन्द्रि॒येण॒ व्य॑र्द्धयति॒ यं का॒मये॑त॒ तेज॑सैनं दे॒वता॑भिरिन्द्रि॒येण॒ सम॑र्द्धयेय॒मित्य॑- [  ] \newline

\textbf{Pada Paata} \newline

क॒ल्प॒य॒ति॒ । वै॒ष्ण॒वः । वै । दे॒वत॑या । यूपः॑ । स्वया᳚ । ए॒व । ए॒न॒म् । दे॒वत॑या । क॒ल्प॒य॒ति॒ । द्वाभ्या᳚म् । क॒ल्प॒य॒ति॒ । द्वि॒पादिति॑ द्वि - पात् । यज॑मानः । प्रति॑ष्ठित्या॒ इति॒ प्रति॑ - स्थि॒त्यै॒ । यम् । का॒मये॑त । तेज॑सा । ए॒न॒म् । दे॒वता॑भिः । इ॒न्द्रि॒येण॑ । वीति॑ । अ॒द्‌र्ध॒ये॒य॒म् । इति॑ । अ॒ग्नि॒ष्ठामित्य॑ग्नि - स्थाम् । तस्य॑ । अश्रि᳚म् । आ॒ह॒व॒नीया॒दित्या᳚-ह॒व॒नीया᳚त् । इ॒त्थम् । वा॒ । इ॒त्थम् । वा॒ । अतीति॑ । ना॒व॒ये॒त् । तेज॑सा । ए॒व । ए॒न॒म् । दे॒वता॑भिः । इ॒न्द्रि॒येण॑ । वीति॑ । अ॒द्‌र्ध॒य॒ति॒ । यम् । का॒मये॑त । तेज॑सा । ए॒न॒म् । दे॒वता॑भिः । इ॒न्द्रि॒येण॑ । समिति॑ । अ॒द्‌र्ध॒ये॒य॒म् । इति॑ ।  \newline




\markright{ TS 6.3.4.5  \hfill https://www.vedavms.in \hfill}

\section{ TS 6.3.4.5 }

\textbf{TS 6.3.4.5 } \newline
\textbf{Samhita Paata} \newline

-ग्नि॒ष्ठां तस्याश्रि॑माहव॒नीये॑न॒ सं मि॑नुया॒त् तेज॑सै॒वैनं॑ दे॒वता॑भिरिन्द्रि॒येण॒ सम॑र्द्धयति ब्रह्म॒वनिं॑ त्वा क्षत्र॒वनि॒मित्या॑ह यथाय॒जुरे॒वैतत् परि॑ व्यय॒त्यूर्ग्वै र॑श॒ना यज॑मानेन॒ यूपः॒ संमि॑तो॒ यज॑मानमे॒वोर्जा सम॑र्द्धयति नाभिद॒घ्ने परि॑ व्ययति नाभिद॒घ्न ए॒वास्मा॒ ऊर्जं॑ दधाति॒ तस्मा᳚न्नाभिद॒घ्न ऊ॒र्जा भु॑ञ्जते॒ यं का॒मये॑तो॒र्जैनं॒- [  ] \newline

\textbf{Pada Paata} \newline

अ॒ग्नि॒ष्ठामित्य॑ग्नि - स्थाम् । तस्य॑ । अश्रि᳚म् । आ॒ह॒व॒नीये॒नेत्या᳚ - ह॒व॒नीये॑न । समिति॑ । मि॒नु॒या॒त् । तेज॑सा । ए॒व । ए॒न॒म् । दे॒वता॑भिः । इ॒न्द्रि॒येण॑ । समिति॑ । अ॒द्‌र्ध॒य॒ति॒ । ब्र॒ह्म॒वनि॒मिति॑ ब्रह्म - वनि᳚म् । त्वा॒ । क्ष॒त्र॒वनि॒मिति॑ क्षत्र-वनि᳚म् । इति॑ । आ॒ह॒ । य॒था॒य॒जुरिति॑ यथा-य॒जुः । ए॒व । ए॒तत् । परीति॑ । व्य॒य॒ति॒ । ऊर्क् । वै । र॒श॒ना । यज॑मानेन । यूपः॑ । संमि॑त॒ इति॒ सं-मि॒तः॒ । यज॑मानम् । ए॒व । ऊ॒र्जा । समिति॑ । अ॒द्‌र्ध॒य॒ति॒ । ना॒भि॒द॒घ्न इति॑ नाभि - द॒घ्ने । परीति॑ । व्य॒य॒ति॒ । ना॒भि॒द॒घ्न इति॑ नाभि - द॒घ्ने । ए॒व । अ॒स्मै॒ । ऊर्ज᳚म् । द॒धा॒ति॒ । तस्मा᳚त् । ना॒भि॒द॒घ्न इति॑ नाभि - द॒घ्ने । ऊ॒र्जा । भु॒ञ्ज॒ते॒ । यम् । का॒मये॑त । ऊ॒र्जा । ए॒न॒म् ।  \newline




\markright{ TS 6.3.4.6  \hfill https://www.vedavms.in \hfill}

\section{ TS 6.3.4.6 }

\textbf{TS 6.3.4.6 } \newline
\textbf{Samhita Paata} \newline

ॅव्य॑र्द्धयेय॒-मित्यू॒र्द्ध्वां ॅवा॒ तस्यावा॑चीं॒ ॅवाऽवो॑हेदू॒र्जैवैनं॒ ॅव्य॑र्द्धयति॒ यदि॑ का॒मये॑त॒ वर्.षु॑कः प॒र्जन्यः॑ स्या॒दित्य-वा॑ची॒मवो॑हे॒द् वृष्टि॑मे॒व नि य॑च्छति॒ यदि॑ का॒मये॒ताव॑र्.षुकः स्या॒दित्यू॒र्द्ध्वामुदू॑हे॒द् वृष्टि॑मे॒वोद् य॑च्छति पितृ॒णां निखा॑तं मनु॒ष्या॑णामू॒र्द्ध्वं निखा॑ता॒दा र॑श॒नाया॒ ओष॑धीनाꣳ रश॒ना विश्वे॑षां- [  ] \newline

\textbf{Pada Paata} \newline

वीति॑ । अ॒द्‌र्ध॒ये॒य॒म् । इति॑ । ऊ॒द्‌र्ध्वाम् । वा॒ । तस्य॑ । अवा॑चीम् । वा॒ । अवेति॑ । ऊ॒हे॒त् । ऊ॒र्जा । ए॒व । ए॒न॒म् । वीति॑ । अ॒द्‌र्ध॒य॒ति॒ । यदि॑ । का॒मये॑त । वर्.षु॑कः । प॒र्जन्यः॑ । स्या॒त् । इति॑ । अवा॑चीम् । अवेति॑ । ऊ॒हे॒त् । वृष्टि᳚म् । ए॒व । नीति॑ । य॒च्छ॒ति॒ । यदि॑ । का॒मये॑त । अव॑र्.षुकः । स्या॒त् । इति॑ । ऊ॒द्‌र्ध्वाम् । उदिति॑ । ऊ॒हे॒त् । वृष्टि᳚म् । ए॒व । उदिति॑ । य॒च्छ॒ति॒ । पि॒तृ॒णाम् । निखा॑त॒मिति॒ नि - खा॒त॒म् । म॒नु॒ष्या॑णाम् । ऊ॒द्‌र्ध्वम् । निखा॑ता॒दिति॒ नि - खा॒ता॒त् । एति॑ । र॒श॒नायाः᳚ । ओष॑धीनाम् । र॒श॒ना । विश्वे॑षां ।  \newline




\markright{ TS 6.3.4.7  \hfill https://www.vedavms.in \hfill}

\section{ TS 6.3.4.7 }

\textbf{TS 6.3.4.7 } \newline
\textbf{Samhita Paata} \newline

दे॒वाना॑मू॒र्द्ध्वꣳ र॑श॒नाया॒ आ च॒षाला॒दिन्द्र॑स्य च॒षालꣳ॑ सा॒द्ध्याना॒मति॑रिक्तꣳ॒॒ स वा ए॒ष स॑र्वदेव॒त्यो॑ यद्यूपो॒ यद्यूपं॑ मि॒नोति॒ सर्वा॑ ए॒व दे॒वताः᳚ प्रीणाति य॒ज्ञेन॒ वै दे॒वाः सु॑व॒र्गं ॅलो॒कमा॑य॒न् ते॑ऽमन्यन्त मनु॒ष्या॑ नो॒ऽन्वाभ॑विष्य॒न्तीति॒ ते यूपे॑न योपयि॒त्वा सु॑व॒र्गं ॅलो॒कमा॑य॒न् तमृष॑यो॒ यूपे॑नै॒वानु॒ प्राजा॑न॒न् तद् यूप॑स्य यूप॒त्वं- [  ] \newline

\textbf{Pada Paata} \newline

दे॒वाना᳚म् । ऊ॒द्‌र्ध्वम् । र॒श॒नायाः᳚ । एति॑ । च॒षाला᳚त् । इन्द्र॑स्य । च॒षाल᳚म् । सा॒द्ध्याना᳚म् । अति॑रिक्त॒मित्यति॑ - रि॒क्त॒म् । सः । वै । ए॒षः । स॒र्व॒दे॒व॒त्य॑ इति॑ सर्व-दे॒व॒त्यः॑ । यत् । यूपः॑ । यत् । यूप᳚म् । मि॒नोति॑ । सर्वाः᳚ । ए॒व । दे॒वताः᳚ । प्री॒णा॒ति॒ । य॒ज्ञेन॑ । वै । दे॒वाः । सु॒व॒र्गमिति॑ सुवः-गम् । लो॒कम् । आ॒य॒न्न् । ते । अ॒म॒न्य॒न्त॒ । म॒नु॒ष्याः᳚ । नः॒ । अ॒न्वाभ॑विष्य॒न्तीत्य॑नु-आभ॑विष्यन्ति । इति॑ । ते । यूपे॑न । यो॒प॒यि॒त्वा । सु॒व॒र्गमिति॑ सुवः-गम् । लो॒कम् । आ॒य॒न्न् । तम् । ऋष॑यः । यूपे॑न । ए॒व । अनु॑ । प्रेति॑ । अ॒जा॒न॒न्न् । तत् । यूप॑स्य । यू॒प॒त्वमिति॑ यूप- त्वम् ।  \newline




\markright{ TS 6.3.4.8  \hfill https://www.vedavms.in \hfill}

\section{ TS 6.3.4.8 }

\textbf{TS 6.3.4.8 } \newline
\textbf{Samhita Paata} \newline

ॅयद् यूपं॑ मि॒नोति॑ सुव॒र्गस्य॑ लो॒कस्य॒ प्रज्ञा᳚त्यै पु॒रस्ता᳚न् मिनोति पु॒रस्ता॒द्धि य॒ज्ञ्स्य॑ प्रज्ञा॒यते प्र॑ज्ञातꣳ॒॒ हि तद् यदति॑पन्न आ॒हुरि॒दं का॒र्य॑मासी॒दिति॑ सा॒द्ध्या वै दे॒वा य॒ज्ञ्मत्य॑मन्यन्त॒ तान्. य॒ज्ञो नास्पृ॑श॒त् तान्. यद् य॒ज्ञ्स्याति॑रिक्त॒मासी॒त् तद॑स्पृश॒दति॑रिक्तं॒ ॅवा ए॒तद् य॒ज्ञ्स्य॒ यद॒ग्नाव॒ग्निं म॑थि॒त्वा प्र॒हर॒त्यति॑रिक्तमे॒त- [  ] \newline

\textbf{Pada Paata} \newline

यत् । यूप᳚म् । मि॒नोति॑ । सु॒व॒र्गस्येति॑ सुवः - गस्य॑ । लो॒कस्य॑ । प्रज्ञा᳚त्या॒ इति॒ प्र - ज्ञा॒त्यै॒ । पु॒रस्ता᳚त् । मि॒नो॒ति॒ । पु॒रस्ता᳚त् । हि । य॒ज्ञ्स्य॑ । प्र॒ज्ञा॒यत॒ इति॑ प्र - ज्ञा॒यते᳚ । अप्र॑ज्ञात॒मित्यप्र॑ - ज्ञा॒त॒म् । हि । तत् । यत् । अति॑पन्न॒ इत्यति॑ - प॒न्ने॒ । आ॒हुः । इ॒दम् । का॒र्य᳚म् । आ॒सी॒त् । इति॑ । सा॒द्ध्याः । वै । दे॒वाः । य॒ज्ञ्म् । अतीति॑ । अ॒म॒न्य॒न्त॒ । तान् । य॒ज्ञ्ः । न । अ॒स्पृ॒श॒त् । तान् । यत् । य॒ज्ञ्स्य॑ । अति॑रिक्त॒मित्यति॑ - रि॒क्त॒म् । आसी᳚त् । तत् । अ॒स्पृ॒श॒त् । अति॑रिक्त॒मित्यति॑ - रि॒क्त॒म् । वै । ए॒तत् । य॒ज्ञ्स्य॑ । यत् । अ॒ग्नौ । अ॒ग्निम् । म॒थि॒त्वा । प्र॒हर॒तीति॑ प्र - हर॑ति । अति॑रिक्त॒मित्यति॑ - रि॒क्त॒म् । ए॒तत् ।  \newline




\markright{ TS 6.3.4.9  \hfill https://www.vedavms.in \hfill}

\section{ TS 6.3.4.9 }

\textbf{TS 6.3.4.9 } \newline
\textbf{Samhita Paata} \newline

-द्यूप॑स्य॒ यदू॒र्द्ध्वं च॒षाला॒त् तेषां॒ तद् भा॑ग॒धेयं॒ ताने॒व तेन॑ प्रीणाति दे॒वा वै सꣳस्थि॑ते॒ सोमे॒ प्र स्रुचोऽह॑र॒न् प्र यूपं॒ ते॑ऽमन्यन्त यज्ञ्वेश॒सं ॅवा इ॒दं कु॑र्म॒ इति॒ ते प्र॑स्त॒रꣳ स्रु॒चां नि॒ष्क्रय॑ण-मपश्य॒न्थ् स्वरुं॒ ॅयूप॑स्य॒ सꣳस्थि॑ते॒ सोमे॒ प्र प्र॑स्त॒रꣳ हर॑ति जु॒होति॒ स्वरु॒मय॑ज्ञ्वेशसाय ॥ \newline

\textbf{Pada Paata} \newline

यूप॑स्य । यत् । ऊ॒द्‌र्ध्वम् । च॒षाला᳚त् । तेषा᳚म् । तत् । भा॒ग॒धेय॒मिति॑ भाग - धेय᳚म् । तान् । ए॒व । तेन॑ । प्री॒णा॒ति॒ । दे॒वाः । वै । सꣳस्थि॑त॒ इति॒ सं-स्थि॒ते॒ । सोमे᳚ । प्रेति॑ । स्रुचः॑ । अह॑रन्न् । प्रेति॑ । यूप᳚म् । ते । अ॒म॒न्य॒न्त॒ । य॒ज्ञ्॒वे॒श॒समिति॑ यज्ञ् - वे॒श॒स॒म् । वै । इ॒दम् । कु॒र्मः॒ । इति॑ । ते । प्र॒स्त॒रमिति॑ प्र - स्त॒रम् । स्रु॒चाम् । नि॒ष्क्रय॑ण॒मिति॑ निः - क्रय॑णम् । अ॒प॒श्य॒न्न् । स्वरु᳚म् । यूप॑स्य । सꣳस्थि॑त॒ इति॒ सं - स्थि॒ते॒ । सोमे᳚ । प्रेति॑ । प्र॒स्त॒रमिति॑ प्र-स्त॒रम् । हर॑ति । जु॒होति॑ । स्वरु᳚म् । अय॑ज्ञ्वेशसा॒येत्यय॑ज्ञ्-वे॒श॒सा॒य॒ ॥  \newline




\markright{ TS 6.3.5.1  \hfill https://www.vedavms.in \hfill}

\section{ TS 6.3.5.1 }

\textbf{TS 6.3.5.1 } \newline
\textbf{Samhita Paata} \newline

सा॒द्ध्या वै दे॒वा अ॒स्मिॅल्लो॒क आ॑स॒न् नान्यत् किं॑ च॒न मि॒षत् ते᳚ऽग्निमे॒वाग्नये॒ मेधा॒या ऽल॑भन्त॒ न ह्य॑न्यदा॑ल॒भ्यं॑-मवि॑न्द॒न् ततो॒ वा इ॒माः प्र॒जाः प्राजा॑यन्त॒ यद॒ग्नाव॒ग्निं म॑थि॒त्वा प्र॒हर॑ति प्र॒जानां᳚ प्र॒जन॑नाय रु॒द्रो वा ए॒ष यद॒ग्निर्यज॑मानः प॒शुर्यत् प॒शुमा॒लभ्या॒ग्निं मन्थे᳚द् रु॒द्राय॒ यज॑मान॒- [  ] \newline

\textbf{Pada Paata} \newline

सा॒द्ध्याः । वै । दे॒वाः । अ॒स्मिन्न् । लो॒के । आ॒स॒न्न् । न । अ॒न्यत् । किम् । च॒न । मि॒षत् । ते । अ॒ग्निम् । ए॒व । अ॒ग्नये᳚ । मेधा॑य । एति॑ । अ॒ल॒भ॒न्त॒ । न । हि । अ॒न्यत् । आ॒ल॒भ्यं॑मित्या᳚-ल॒भ्यं᳚म् । अवि॑न्दन्न् । ततः॑ । वै । इ॒माः । प्र॒जा इति॑ प्र - जाः । प्रेति॑ । अ॒जा॒य॒न्त॒ । यत् । अ॒ग्नौ । अ॒ग्निम् । म॒थि॒त्वा । प्र॒हर॒तीति॑ प्र - हर॑ति । प्र॒जाना॒मिति॑ प्र - जाना᳚म् । प्र॒जन॑ना॒येति॑ प्र-जन॑नाय । रु॒द्रः । वै । ए॒षः । यत् । अ॒ग्निः । यज॑मानः । प॒शुः । यत् । प॒शुम् । आ॒लभ्येत्या᳚ - लभ्य॑ । अ॒ग्निम् । मन्थे᳚त् । रु॒द्राय॑ । यज॑मानम् ।  \newline




\markright{ TS 6.3.5.2  \hfill https://www.vedavms.in \hfill}

\section{ TS 6.3.5.2 }

\textbf{TS 6.3.5.2 } \newline
\textbf{Samhita Paata} \newline

-मपि॑ दद्ध्यात् प्र॒मायु॑कः स्या॒दथो॒ खल्वा॑हुर॒ग्निः सर्वा॑ दे॒वता॑ ह॒विरे॒तद्यत् प॒शुरिति॒ यत् प॒शुमा॒लभ्या॒ग्निं मन्थ॑ति ह॒व्यायै॒वाऽऽ*स॑न्नाय॒ सर्वा॑ दे॒वता॑ जनय-त्युपा॒कृत्यै॒व मन्थ्य॒-स्तन्नेवाऽऽ*ल॑ब्धं॒ नेवाना॑लब्ध-म॒ग्ने-र्ज॒नित्र॑-म॒सीत्या॑हा॒ग्नेर्ह्ये॑त-ज्ज॒नित्रं॒ ॅवृष॑णौ स्थ॒ इत्या॑ह॒ वृष॑णौ॒- [  ] \newline

\textbf{Pada Paata} \newline

अपीति॑ । द॒द्ध्या॒त् । प्र॒मायु॑क॒ इति॑ प्र-मायु॑कः । स्या॒त् । अथो॒ इति॑ । खलु॑ । आ॒हुः॒ । अ॒ग्निः । सर्वाः᳚ । दे॒वताः᳚ । ह॒विः । ए॒तत् । यत् । प॒शुः । इति॑ । यत् । प॒शुम् । आ॒लभ्येत्या᳚-लभ्य॑ । अ॒ग्निम् । मन्थ॑ति । ह॒व्याय॑ । ए॒व । आस॑न्ना॒येत्या-स॒न्ना॒य॒ । सर्वाः᳚ । दे॒वताः᳚ । ज॒न॒य॒ति॒ । उ॒पा॒कृत्येत्यु॑प - आ॒कृत्य॑ । ए॒व । मन्थ्यः॑ । तत् । न । इ॒व॒ । आल॑ब्ध॒मित्या - ल॒ब्ध॒म् । न । इ॒व॒ । आल॑ब्ध॒मित्यना᳚ - ल॒ब्ध॒म् । अ॒ग्नेः । ज॒नित्र᳚म् । अ॒सि॒ । इति॑ । आ॒ह॒ । अ॒ग्नेः । हि । ए॒तत् । ज॒नित्र᳚म् । वृष॑णौ । स्थः॒ । इति॑ । आ॒ह॒ । वृष॑णौ ।  \newline




\markright{ TS 6.3.5.3  \hfill https://www.vedavms.in \hfill}

\section{ TS 6.3.5.3 }

\textbf{TS 6.3.5.3 } \newline
\textbf{Samhita Paata} \newline

ह्ये॑ता-वु॒र्वश्य॑स्या॒यु-र॒सीत्या॑ह मिथुन॒त्वाय॑ घृ॒तेना॒क्ते वृष॑णं दधाथा॒मित्या॑ह॒ वृष॑णꣳ॒॒ ह्ये॑ते दधा॑ते॒ ये अ॒ग्निं गा॑य॒त्रं छन्दोऽनु॒ प्र जा॑य॒स्वेत्या॑ह॒ छन्दो॑भिरे॒वैनं॒ प्र ज॑नयत्य॒ग्नये॑ म॒थ्यमा॑ना॒यानु॑ ब्रू॒हीत्या॑ह सावि॒त्रीमृच॒मन्वा॑ह सवि॒तृप्र॑सूत ए॒वैनं॑ मन्थति जा॒तायानु॑ ब्रूहि- [  ] \newline

\textbf{Pada Paata} \newline

हि । ए॒तौ । उ॒र्वशी᳚ । अ॒सि॒ । आ॒युः । अ॒सि॒ । इति॑ । आ॒ह॒ । मि॒थु॒न॒त्वायेति॑ मिथुन - त्वाय॑ । घृ॒तेन॑ । अ॒क्ते इति॑ । वृष॑णम् । द॒धा॒था॒म् । इति॑ । आ॒ह॒ । वृष॑णम् । हि । ए॒ते इति॑ । दधा॑ते॒ इति॑ । ये इति॑ । अ॒ग्निम् । गा॒य॒त्रम् । छन्दः॑ । अनु॑ । प्रेति॑ । जा॒य॒स्व॒ । इति॑ । आ॒ह॒ । छन्दो॑भि॒रिति॒ छन्दः॑ - भिः॒ । ए॒व । ए॒न॒म् । प्रेति॑ । ज॒न॒य॒ति॒ । अ॒ग्नये᳚ । म॒थ्यमा॑नाय । अन्विति॑ । ब्रू॒हि॒ । इति॑ । आ॒ह॒ । सा॒वि॒त्रीम् । ऋच᳚म् । अन्विति॑ । आ॒ह॒ । स॒वि॒तृप्र॑सूत॒ इति॑ सवि॒तृ-प्र॒सू॒तः॒ । ए॒व । ए॒न॒म् । म॒न्थ॒ति॒ । जा॒ताय॑ । अन्विति॑ । ब्रू॒हि॒ ।  \newline




\markright{ TS 6.3.5.4  \hfill https://www.vedavms.in \hfill}

\section{ TS 6.3.5.4 }

\textbf{TS 6.3.5.4 } \newline
\textbf{Samhita Paata} \newline

प्रह्रि॒यमा॑णा॒यानु॑ ब्रू॒हीत्या॑ह॒ काण्डे॑काण्ड ए॒वैनं॑ क्रि॒यमा॑णे॒ सम॑र्द्धयति गाय॒त्रीः सर्वा॒ अन्वा॑ह गाय॒त्रछ॑न्दा॒ वा अ॒ग्निः स्वेनै॒वैनं॒ छन्द॑सा॒ सम॑र्द्धयत्य॒ग्निः पु॒रा भव॑त्य॒ग्निं म॑थि॒त्वा प्र ह॑रति॒ तौ स॒भंव॑न्तौ॒ यज॑मानम॒भि सं भ॑वतो॒ भव॑तं नः॒ सम॑नसा॒वित्या॑ह॒ शान्त्यै᳚ प्र॒हृत्य॑ जुहोति जा॒तायै॒वास्मा॒ अन्न॒मपि॑ ( ) दधा॒त्याज्ये॑न जुहोत्ये॒तद्वा अ॒ग्नेः प्रि॒यं धाम॒ यदाज्यं॑ प्रि॒येणै॒वैनं॒ धाम्ना॒ सम॑र्द्धय॒त्यथो॒ तेज॑सा ॥ \newline

\textbf{Pada Paata} \newline

प्र॒ह्रि॒यमा॑णा॒येति॑ प्र - ह्रि॒यमा॑णाय । अन्विति॑ । ब्रू॒हि॒ । इति॑ । आ॒ह॒ । काण्डे॑काण्ड॒ इति॒ काण्डे᳚ - का॒ण्डे॒ । ए॒व । ए॒न॒म् । क्रि॒यमा॑णे । समिति॑ । अ॒द्‌र्ध॒य॒ति॒ । गा॒य॒त्रीः । सर्वाः᳚ । अन्विति॑ । आ॒ह॒ । गा॒य॒त्रछ॑न्दा॒ इति॑ गाय॒त्र-छ॒न्दाः॒ । वै । अ॒ग्निः । स्वेन॑ । ए॒व । ए॒न॒म् । छन्द॑सा । समिति॑ । अ॒द्‌र्ध॒य॒ति॒ । अ॒ग्निः । पु॒रा । भव॑ति । अ॒ग्निम् । म॒थि॒त्वा । प्रेति॑ । ह॒र॒ति॒ । तौ । स॒भंव॑न्ता॒विति॑ सं - भव॑न्तौ । यज॑मानम् । अ॒भि । समिति॑ । भ॒व॒तः॒ । भव॑तम् । नः॒ । सम॑नसा॒विति॒ स - म॒न॒सौ॒ । इति॑ । आ॒ह॒ । शान्त्यै᳚ । प्र॒हृत्येति॑ प्र-हृत्य॑ । जु॒हो॒ति॒ । जा॒ताय॑ । ए॒व । अ॒स्मै॒ । अन्न᳚म् । अपीति॑ ( ) । द॒धा॒ति॒ । आज्ये॑न । जु॒हो॒ति॒ । ए॒तत् । वै । अ॒ग्नेः । प्रि॒यम् । धाम॑ । यत् । आज्य᳚म् । प्रि॒येण॑ । ए॒व । ए॒न॒म् । धाम्ना᳚ । समिति॑ । अ॒द्‌र्ध॒य॒ति॒ । अथो॒ इति॑ । तेज॑सा ॥  \newline




\markright{ TS 6.3.6.1  \hfill https://www.vedavms.in \hfill}

\section{ TS 6.3.6.1 }

\textbf{TS 6.3.6.1 } \newline
\textbf{Samhita Paata} \newline

इ॒षे त्वेति॑ ब॒र्॒.हिरा द॑त्त इ॒च्छत॑ इव॒ ह्ये॑ष यो यज॑त उप॒वीर॒सीत्या॒होप॒ ह्ये॑नानाक॒रोत्युपो॑ दे॒वान् दैवी॒र्विशः॒ प्रागु॒रित्या॑ह॒ दैवी॒र्ह्ये॑ता विशः॑ स॒तीर्दे॒वानु॑प॒यन्ति॒ वह्नी॑रु॒शिज॒ इत्या॑ह॒र्त्विजो॒ वै वह्न॑य उ॒शिज॒-स्तस्मा॑दे॒वमा॑ह॒ बृह॑स्पते धा॒रया॒ वसू॒नीत्या॑- [  ] \newline

\textbf{Pada Paata} \newline

इ॒षे । त्वा॒ । इति॑ । ब॒र्॒.हिः । एति॑ । द॒त्ते॒ । इ॒च्छते᳚ । इ॒व॒ । हि । ए॒षः । यः । यज॑ते । उ॒प॒वीरित्यु॑प - वीः । अ॒सि॒ । इति॑ । आ॒ह॒ । उपेति॑ । हि । ए॒ना॒न् । आ॒क॒रोतीत्या᳚ - क॒रोति॑ । उपो॒ इति॑ । दे॒वान् । दैवीः᳚ । विशः॑ । प्रेति॑ । अ॒गुः॒ । इति॑ । आ॒ह॒ । दैवीः᳚ । हि । ए॒ताः । विशः॑ । स॒तीः । दे॒वान् । उ॒प॒यन्तीत्यु॑प - यन्ति॑ । वह्नीः᳚ । उ॒शिजः॑ । इति॑ । आ॒ह॒ । ऋ॒त्विजः॑ । वै । वह्न॑यः । उ॒शिजः॑ । तस्मा᳚त् । ए॒वम् । आ॒ह॒ । बृह॑स्पते । धा॒रय॑ । वसू॑नि । इति॑ ।  \newline




\markright{ TS 6.3.6.2  \hfill https://www.vedavms.in \hfill}

\section{ TS 6.3.6.2 }

\textbf{TS 6.3.6.2 } \newline
\textbf{Samhita Paata} \newline

-ह॒ ब्रह्म॒ वै दे॒वानां॒ बृह॒स्पति॒ र्ब्रह्म॑णै॒वास्मै॑ प॒शूनव॑ रुन्धे ह॒व्या ते᳚ स्वदन्ता॒मित्या॑ह स्व॒दय॑त्ये॒वैना॒न् देव॑ त्वष्ट॒र्वसु॑ र॒ण्वेत्या॑ह॒ त्वष्टा॒ वै प॑शू॒नां मि॑थु॒नानाꣳ॑ रूप॒कृद् रू॒पमे॒व प॒शुषु॑ दधाति॒ रेव॑ती॒ रम॑द्ध्व॒मित्या॑ह प॒शवो॒ वै रे॒वतीः᳚ प॒शूने॒वास्मै॑ रमयति दे॒वस्य॑ त्वा सवि॒तुः प्र॑स॒व इति॑- [  ] \newline

\textbf{Pada Paata} \newline

आ॒ह॒ । ब्रह्म॑ । वै । दे॒वाना᳚म् । बृह॒स्पतिः॑ । ब्रह्म॑णा । ए॒व । अ॒स्मै॒ । प॒शून् । अवेति॑ । रु॒न्धे॒ । ह॒व्या । ते॒ । स्व॒द॒न्ता॒म् । इति॑ । आ॒ह॒ । स्व॒दय॑ति । ए॒व । ए॒ना॒न् । देव॑ । त्व॒ष्टः॒ । वसु॑ । र॒ण्व॒ । इति॑ । आ॒ह॒ । त्वष्टा᳚ । वै । प॒शू॒नाम् । मि॒थु॒नाना᳚म् । रू॒प॒कृदिति॑ रूप - कृत् । रू॒पम् । ए॒व । प॒शुषु॑ । द॒धा॒ति॒ । रेव॑तीः । रम॑द्ध्वम् । इति॑ । आ॒ह॒ । प॒शवः॑ । वै । रे॒वतीः᳚ । प॒शून् । ए॒व । अ॒स्मै॒ । र॒म॒य॒ति॒ । दे॒वस्य॑ । त्वा॒ । स॒वि॒तुः । प॒स॒व इति॑ प्र - स॒वे । इति॑ ।  \newline




\markright{ TS 6.3.6.3  \hfill https://www.vedavms.in \hfill}

\section{ TS 6.3.6.3 }

\textbf{TS 6.3.6.3 } \newline
\textbf{Samhita Paata} \newline

रश॒नामा द॑त्ते॒ प्रसू᳚त्या अ॒श्विनो᳚र्बा॒हुभ्या॒-मित्या॑हा॒श्विनौ॒ हि दे॒वाना॑मद्ध्व॒र्यू आस्तां᳚ पू॒ष्णो हस्ता᳚भ्या॒मित्या॑ह॒ यत्या॑ ऋ॒तस्य॑ त्वा देवहविः॒ पाशे॒नाऽऽ* र॑भ॒ इत्या॑ह स॒त्यं ॅवा ऋ॒तꣳ स॒त्येनै॒वैन॑मृ॒तेना ऽऽ*र॑भते ऽक्ष्ण॒या परि॑ हरति॒ वद्ध्यꣳ॒॒ हि प्र॒त्यञ्चं॑ प्रति मु॒ञ्चन्ति॒ व्यावृ॑त्त्यै॒ धर्.षा॒ मानु॑षा॒निति॒ नि यु॑नक्ति॒ धृत्या॑ अ॒द्भ्य- [  ] \newline

\textbf{Pada Paata} \newline

र॒श॒नाम् । एति॑ । द॒त्ते॒ । प्रसू᳚त्या॒ इति॒ प्र - सू॒त्यै॒ । अ॒श्विनोः᳚ । बा॒हुभ्या॒मिति॑ बा॒हु-भ्या॒म् । इति॑ । आ॒ह॒ । अ॒श्विनौ᳚ । हि । दे॒वाना᳚म् । अ॒द्ध्व॒र्यू इति॑ । आस्ता᳚म् । पू॒ष्णः । हस्ता᳚भ्याम् । इति॑ । आ॒ह॒ । यत्यै᳚ । ऋ॒तस्य॑ । त्वा॒ । दे॒व॒ह॒वि॒रिति॑ देव - ह॒विः॒ । पाशे॑न । एति॑ । र॒भे॒ । इति॑ । आ॒ह॒ । स॒त्यम् । वै । ऋ॒तम् । स॒त्येन॑ । ए॒व । ए॒न॒म् । ऋ॒तेन॑ । एति॑ । र॒भ॒ते॒ । अ॒क्ष्ण॒या । परीति॑ । ह॒र॒ति॒ । वद्ध्य᳚म् । हि । प्र॒त्यञ्च᳚म् । प्र॒ति॒मु॒ञ्चन्तीति॑ प्रति - मु॒ञ्चन्ति॑ । व्यावृ॑त्त्या॒ इति॑ वि - आवृ॑त्त्यै । धर्.ष॑ । मानु॑षान् । इति॑ । नीति॑ । यु॒न॒क्ति॒ । धृत्यै᳚ । अ॒द्भ्य इत्य॑त् - भ्यः ।  \newline




\markright{ TS 6.3.6.4  \hfill https://www.vedavms.in \hfill}

\section{ TS 6.3.6.4 }

\textbf{TS 6.3.6.4 } \newline
\textbf{Samhita Paata} \newline

स्त्वौष॑धीभ्यः॒ प्रोक्षा॒मीत्या॑हा॒द्भ्यो ह्ये॑ष ओष॑धीभ्यः स॒भंव॑ति॒ यत् प॒शुर॒पां पे॒रुर॒सीत्या॑है॒ष ह्य॑पां पा॒ता यो मेधा॑याऽऽ* र॒भ्यते᳚ स्वा॒त्तं चि॒थ् सदे॑वꣳ ह॒व्यमापो॑ देवीः॒ स्वद॑तैन॒मित्या॑ह स्व॒दय॑त्ये॒वैन॑-मु॒परि॑ष्टा॒त् प्रोक्ष॑त्यु॒परि॑ष्टादे॒वैनं॒ मेद्ध्यं॑ करोति पा॒यय॑त्यन्तर॒त ए॒वैनं॒ ( ) मेद्ध्यं॑ करोत्य॒धस्ता॒दुपो᳚क्षति स॒र्वत॑ ए॒वैनं॒ मेद्ध्यं॑ करोति ॥ \newline

\textbf{Pada Paata} \newline

त्वा॒ । ओष॑धीभ्य॒ इत्योष॑धि - भ्यः॒ । प्रेति॑ । उ॒क्षा॒मि॒ । इति॑ । आ॒ह॒ । अ॒द्भ्य इत्य॑त् - भ्यः । हि । ए॒षः । ओष॑धीभ्य॒ इत्योष॑धि - भ्यः॒ । स॒भंव॒तीति॑ सं - भव॑ति । यत् । प॒शुः । अ॒पाम् । पे॒रुः । अ॒सि॒ । इति॑ । आ॒ह॒ । ए॒षः । हि । अ॒पाम् । पा॒ता । यः । मेधा॑य । आ॒र॒भ्यत॒ इत्या᳚-र॒भ्यते᳚ । स्वा॒त्तम् । चि॒त् । सदे॑व॒मिति॒ स - दे॒व॒म् । ह॒व्यम् । आपः॑ । दे॒वीः॒ । स्वद॑त । ए॒न॒म् । इति॑ । आ॒ह॒ । स्व॒दय॑ति । ए॒व । ए॒न॒म् । उ॒परि॑ष्टात् । प्रेति॑ । उ॒क्ष॒ति॒ । उ॒परि॑ष्टात् । ए॒व । ए॒न॒म् । मेद्ध्य᳚म् । क॒रो॒ति॒ । पा॒यय॑ति । अ॒न्त॒र॒तः । ए॒व । ए॒न॒म् ( ) । मेद्ध्य᳚म् । क॒रो॒ति॒ । अ॒धस्ता᳚त् । उपेति॑ । उ॒क्ष॒ति॒ । स॒र्वतः॑ । ए॒व । ए॒न॒म् । मेद्ध्य᳚म् । क॒रो॒ति॒ ॥  \newline




\markright{ TS 6.3.7.1  \hfill https://www.vedavms.in \hfill}

\section{ TS 6.3.7.1 }

\textbf{TS 6.3.7.1 } \newline
\textbf{Samhita Paata} \newline

अ॒ग्निना॒ वै होत्रा॑ दे॒वा असु॑रा-न॒भ्य॑भव-न्न॒ग्नये॑ समि॒द्ध्यमा॑ना॒यानु॑ ब्रू॒हीत्या॑ह॒ भ्रातृ॑व्याभिभूत्यै स॒प्तद॑श सामिधे॒नीरन्वा॑ह सप्तद॒शः प्र॒जाप॑तिः प्र॒जाप॑ते॒राप्त्यै॑ स॒प्तद॒शान्वा॑ह॒ द्वाद॑श॒ मासाः॒ पञ्च॒र्तवः॒ स सं॑ॅवथ्स॒रः सं॑ॅवथ्स॒रं प्र॒जा अनु॒ प्रजा॑यन्ते प्र॒जानां᳚ प्र॒जन॑नाय दे॒वा वै सा॑मिधे॒नीर॒नूच्य॑ य॒ज्ञ्ं नान्व॑पश्य॒न्थ् स प्र॒जाप॑ति-स्तू॒ष्णीमा॑घा॒र- [  ] \newline

\textbf{Pada Paata} \newline

अ॒ग्निना᳚ । वै । होत्रा᳚ । दे॒वाः । असु॑रान् । अ॒भीति॑ । अ॒भ॒व॒न्न् । अ॒ग्नये᳚ । स॒मि॒द्ध्यमा॑ना॒येति॑ सं - इ॒द्ध्यमा॑नाय । अन्विति॑ । ब्रू॒हि॒ । इति॑ । आ॒ह॒ । भ्रातृ॑व्याभिभूत्या॒ इति॒ भ्रातृ॑व्य-अ॒भि॒भू॒त्यै॒ । स॒प्तद॒शेति॑ स॒प्त - द॒श॒ । सा॒मि॒धे॒नीरिति॑ सां - इ॒धे॒नीः । अन्विति॑ । आ॒ह॒ । स॒प्त॒द॒श इति॑ सप्त - द॒शः । प्र॒जाप॑ति॒रिति॑ प्र॒जा - प॒तिः॒ । प्र॒जाप॑ते॒रिति॑ प्र॒जा - प॒तेः॒ । आप्त्यै᳚ । स॒प्तद॒शेति॑ स॒प्त - द॒श॒ । अन्विति॑ । आ॒ह॒ । द्वाद॑श । मासाः᳚ । पञ्च॑ । ऋ॒तवः॑ । सः । सं॒ॅव॒थ्स॒र इति॑ सं - व॒थ्स॒रः । सं॒ॅव॒थ्स॒रमिति॑ सं - व॒थ्स॒रम् । प्र॒जा इति॑ प्र - जाः । अनु॑ । प्रेति॑ । जा॒य॒न्ते॒ । प्र॒जाना॒मिति॑ प्र-जाना᳚म् । प्र॒जन॑ना॒येति॑ प्र-जन॑नाय । दे॒वाः । वै । सा॒मि॒धे॒नीरिति॑ सां-इ॒धे॒नीः । अ॒नूच्येत्य॑नु - उच्य॑ । य॒ज्ञ्म् । न । अन्विति॑ । अ॒प॒श्य॒न्न् । सः । प्र॒जाप॑ति॒रिति॑ प्र॒जा-प॒तिः॒ । तू॒ष्णीम् । आ॒घा॒रमित्या᳚-घा॒रम् ।  \newline




\markright{ TS 6.3.7.2  \hfill https://www.vedavms.in \hfill}

\section{ TS 6.3.7.2 }

\textbf{TS 6.3.7.2 } \newline
\textbf{Samhita Paata} \newline

-मा ऽघा॑रय॒त् ततो॒ वै दे॒वा य॒ज्ञ्मन्व॑पश्य॒न्॒. यत् तू॒ष्णी-मा॑घा॒र-मा॑घा॒रय॑ति य॒ज्ञ्स्यानु॑ख्यात्या॒ असु॑रेषु॒ वै य॒ज्ञ् आ॑सी॒त् तं दे॒वास्तू᳚ष्णीꣳ हो॒मेना॑वृञ्जत॒ यत् तू॒ष्णी-मा॑घा॒र-मा॑घा॒रय॑ति॒ भ्रातृ॑व्यस्यै॒ व तद् य॒ज्ञ्ं ॅवृ॑ङ्क्ते परि॒धी॑न्थ् सं मा᳚र्ष्टि पु॒नात्ये॒वैना॒न् त्रिस्त्रिः॒ सं मा᳚र्ष्टि॒ त्र्या॑वृ॒द्धि य॒ज्ञोऽथो॒ रक्ष॑सा॒मप॑हत्यै॒ द्वाद॑श॒ सं प॑द्यन्ते॒ द्वाद॑श॒- [  ] \newline

\textbf{Pada Paata} \newline

एति॑ । अ॒घा॒र॒य॒त् । ततः॑ । वै । दे॒वाः । य॒ज्ञ्म् । अन्विति॑ । अ॒प॒श्य॒न्न् । यत् । तू॒ष्णीम् । आ॒घा॒रमित्या᳚ - घा॒रम् । आ॒घा॒रय॒तीत्या᳚ - घा॒रय॑ति । य॒ज्ञ्स्य॑ । अनु॑ख्यात्या॒ इत्यनु॑ - ख्या॒त्यै॒ । असु॑रेषु । वै । य॒ज्ञ्ः । आ॒सी॒त् । तम् । दे॒वाः । तू॒ष्णीꣳ॒॒हो॒मेनेति॑ तूष्णीम्-हो॒मेन॑ । अ॒वृ॒ञ्ज॒त॒ । यत् । तू॒ष्णीम् । आ॒घा॒रमित्या᳚ - घा॒रम् । आ॒घा॒रय॒तीत्या᳚ - घा॒रय॑ति । भ्रातृ॑व्यस्य । ए॒व । तत् । य॒ज्ञ्म् । वृ॒ङ्क्ते॒ । प॒रि॒धीनिति॑ परि - धीन् । समिति॑ । मा॒र्ष्टि॒ । पु॒नाति॑ । ए॒व । ए॒ना॒न् । त्रिस्त्रि॒रिति॒ त्रिः - त्रिः॒ । समिति॑ । मा॒र्ष्टि॒ । त्र्या॑वृ॒दिति॒ त्रि - आ॒वृ॒त् । हि । य॒ज्ञ्ः । अथो॒ इति॑ । रक्ष॑साम् । अप॑हत्या॒ इत्यप॑ - ह॒त्यै॒ । द्वाद॑श । समिति॑ । प॒द्य॒न्ते॒ । द्वाद॑श ।  \newline




\markright{ TS 6.3.7.3  \hfill https://www.vedavms.in \hfill}

\section{ TS 6.3.7.3 }

\textbf{TS 6.3.7.3 } \newline
\textbf{Samhita Paata} \newline

मासाः᳚ संॅवथ्स॒रः सं॑ॅवथ्स॒रमे॒व प्री॑णा॒त्यथो॑ संॅवथ्स॒रमे॒वास्मा॒ उप॑ दधाति सुव॒र्गस्य॑ लो॒कस्य॒ सम॑ष्ट्यै॒ शिरो॒ वा ए॒तद् य॒ज्ञ्स्य॒ यदा॑घा॒रो᳚ऽग्निः सर्वा॑ दे॒वता॒ यदा॑घा॒र-मा॑घा॒रय॑ति शीर्.ष॒त ए॒व य॒ज्ञ्स्य॒ यज॑मानः॒ सर्वा॑ दे॒वता॒ अव॑ रुन्धे॒ शिरो॒ वा ए॒तद् य॒ज्ञ्स्य॒ यदा॑घा॒र आ॒त्मा प॒शुरा॑घा॒रमा॒घार्य॑ प॒शुꣳ सम॑नक्त्या॒त्मन्ने॒व य॒ज्ञ्स्य॒- [  ] \newline

\textbf{Pada Paata} \newline

मासाः᳚ । सं॒ॅव॒थ्स॒र इति॑ सं-व॒थ्स॒रः । सं॒ॅव॒थ्स॒रमिति॑ सं-व॒थ्स॒रम् । ए॒व । प्री॒णा॒ति॒ । अथो॒ इति॑ । सं॒ॅव॒थ्स॒रमिति॑ सं - व॒थ्स॒रम् । ए॒व । अ॒स्मै॒ । उपेति॑ । द॒धा॒ति॒ । सु॒व॒र्गस्येति॑ सुवः - गस्य॑ । लो॒कस्य॑ । सम॑ष्ट्या॒ इति॒ सं - अ॒ष्ट्यै॒ । शिरः॑ । वै । ए॒तत् । य॒ज्ञ्स्य॑ । यत् । आ॒घा॒र इत्या᳚ - घा॒रः । अ॒ग्निः । सर्वाः᳚ । दे॒वताः᳚ । यत् । आ॒घा॒रमित्या᳚-घा॒रम् । आ॒घा॒रय॒तीत्या᳚-घा॒रय॑ति । शी॒र्॒.ष॒तः । ए॒व । य॒ज्ञ्स्य॑ । यज॑मानः । सर्वाः᳚ । दे॒वताः᳚ । अवेति॑ । रु॒न्धे॒ । शिरः॑ । वै । ए॒तत् । य॒ज्ञ्स्य॑ । यत् । आ॒घा॒र इत्या᳚ - घा॒रः । आ॒त्मा । प॒शुः । आ॒घा॒रमित्या᳚ - घा॒रम् । आ॒घार्येत्या᳚ - घार्य॑ । प॒शुम् । समिति॑ । अ॒न॒क्ति॒ । आ॒त्मन्न् । ए॒व । य॒ज्ञ्स्य॑ ।  \newline




\markright{ TS 6.3.7.4  \hfill https://www.vedavms.in \hfill}

\section{ TS 6.3.7.4 }

\textbf{TS 6.3.7.4 } \newline
\textbf{Samhita Paata} \newline

शिरः॒ प्रति॑ दधाति॒ सं ते᳚ प्रा॒णो वा॒युना॑ गच्छता॒मित्या॑ह वायुदेव॒त्यो॑ वै प्रा॒णो वा॒यावे॒वास्य॑ प्रा॒णं जु॑होति॒ सं ॅयज॑त्रै॒रङ्गा॑नि॒ सं ॅय॒ज्ञ्प॑तिरा॒शिषेत्या॑ह य॒ज्ञ्प॑तिमे॒वास्या॒ऽऽ*शिषं॑ गमयति वि॒श्वरू॑पो॒ वै त्वा॒ष्ट्र उ॒परि॑ष्टात् प॒शुम॒भ्य॑वमी॒त् तस्मा॑-दु॒परि॑ष्टात् प॒शोर्नाव॑ द्यन्ति॒ यदु॒परि॑ष्टात् प॒शुꣳ स॑म॒नक्ति॒ मेद्ध्य॑मे॒वै- [  ] \newline

\textbf{Pada Paata} \newline

शिरः॑ । प्रतीति॑ । द॒धा॒ति॒ । समिति॑ । ते॒ । प्रा॒ण इति॑ प्र - अ॒नः । वा॒युना᳚ । ग॒च्छ॒ता॒म् । इति॑ । आ॒ह॒ । वा॒यु॒दे॒व॒त्य॑ इति॑ वायु-दे॒व॒त्यः॑ । वै । प्रा॒ण इति॑ प्र - अ॒नः । वा॒यौ । ए॒व । अ॒स्य॒ । प्रा॒णमिति॑ प्र - अ॒नम् । जु॒हो॒ति॒ । समिति॑ । यज॑त्रैः । अङ्गा॑नि । समिति॑ । य॒ज्ञ्प॑ति॒रिति॑ य॒ज्ञ् - प॒तिः॒ । आ॒शिषेत्या᳚ - शिषा᳚ । इति॑ । आ॒ह॒ । य॒ज्ञ्प॑ति॒मिति॑ य॒ज्ञ् - प॒ति॒म् । ए॒व । अ॒स्य॒ । आ॒शिष॒मित्या᳚-शिष᳚म् । ग॒म॒य॒ति॒ । वि॒श्वरू॑प॒ इति॑ वि॒श्व - रू॒पः॒ । वै । त्वा॒ष्ट्रः । उ॒परि॑ष्टात् । प॒शुम् । अ॒भीति॑ । अ॒व॒मी॒त् । तस्मा᳚त् । उ॒परि॑ष्टात् । प॒शोः । न । अवेति॑ । द्य॒न्ति॒ । यत् । उ॒परि॑ष्टात् । प॒शुम् । स॒म॒नक्तीति॑ सं - अ॒नक्ति॑ । मेद्ध्य᳚म् । ए॒व ।  \newline




\markright{ TS 6.3.7.5  \hfill https://www.vedavms.in \hfill}

\section{ TS 6.3.7.5 }

\textbf{TS 6.3.7.5 } \newline
\textbf{Samhita Paata} \newline

-नं॑ करोत्यृ॒त्विजो॑ वृणीते॒ छन्दाꣳ॑स्ये॒व वृ॑णीते स॒प्त वृ॑णीते स॒प्त ग्रा॒म्याः प॒शवः॑ स॒प्ताऽऽ*र॒ण्याः स॒प्त छन्दाꣳ॑स्यु॒भय॒स्या व॑रुद्ध्या॒ एका॑दश प्रया॒जान्. य॑जति॒ दश॒ वै प॒शोः प्रा॒णा आ॒त्मैका॑द॒शो यावा॑ने॒व प॒शुस्तं प्र य॑जति व॒पामेकः॒ परि॑ शय आ॒त्मैवाऽऽ*त्मानं॒ परि॑ शये॒ वज्रो॒ वै स्वधि॑ति॒र्वज्रो॑ यूपशक॒लो घृ॒तं खलु॒ वै ( )  दे॒वा वज्रं॑ कृ॒त्वा सोम॑मघ्नन् घृ॒तेना॒क्तौ प॒शुं त्रा॑येथा॒मित्या॑ह॒ वज्रे॑णै॒वैनं॒ ॅवशे॑ कृ॒त्वाऽऽल॑भते ॥ \newline

\textbf{Pada Paata} \newline

ए॒न॒म् । क॒रो॒ति॒ । ऋ॒त्विजः॑ । वृ॒णी॒ते॒ । छन्दाꣳ॑सि । ए॒व । वृ॒णी॒ते॒ । स॒प्त । वृ॒णी॒ते॒ । स॒प्त । ग्रा॒म्याः । प॒शवः॑ । स॒प्त । आ॒र॒ण्याः । स॒प्त । छन्दाꣳ॑सि । उ॒भय॑स्य । अव॑रुद्ध्या॒ इत्यव॑ - रु॒द्ध्यै॒ । एका॑दश । प्र॒या॒जानिति॑ प्र - या॒जान् । य॒ज॒ति॒ । दश॑ । वै । प॒शोः । प्रा॒णा इति॑ प्र-अ॒नाः । आ॒त्मा । ए॒का॒द॒शः । यावान्॑ । ए॒व । प॒शुः । तम् । प्रेति॑ । य॒ज॒ति॒ । व॒पाम् । एकः॑ । परीति॑ । श॒ये॒ । आ॒त्मा । ए॒व । आ॒त्मान᳚म् । परीति॑ । श॒ये॒ । वज्रः॑ । वै । स्वधि॑ति॒रिति॒ स्व - धि॒तिः॒ । वज्रः॑ । यू॒प॒श॒क॒ल इति॑ यूप - श॒क॒लः । घृ॒तम् । खलु॑ । वै ( ) । दे॒वाः । वज्र᳚म् । कृ॒त्वा । सोम᳚म् । अ॒घ्न॒न्न् । घृ॒तेन॑ । अ॒क्तौ । प॒शुम् । त्रा॒ये॒था॒म् । इति॑ । आ॒ह॒ । वज्रे॑ण । ए॒व । ए॒न॒म् । वशे᳚ । कृ॒त्वा । एति॑ । ल॒भ॒ते॒ ॥  \newline




\markright{ TS 6.3.8.1  \hfill https://www.vedavms.in \hfill}

\section{ TS 6.3.8.1 }

\textbf{TS 6.3.8.1 } \newline
\textbf{Samhita Paata} \newline

पर्य॑ग्नि करोति सर्व॒हुत॑मे॒वैनं॑ करो॒त्य-स्क॑न्दा॒या-स्क॑न्नꣳ॒॒ हि तद् यद् धु॒तस्य॒ स्कन्द॑ति॒ त्रिः पर्य॑ग्नि करोति॒ त्र्या॑वृ॒द्धि य॒ज्ञोऽथो॒ रक्ष॑सा॒मप॑हत्यै ब्रह्मवा॒दिनो॑ वदन्त्यन्वा॒रभ्यः॑ प॒शू(3)र्नान्वा॒रभ्या(3)इति॑ मृ॒त्यवे॒ वा ए॒ष नी॑यते॒ यत् प॒शुस्तं ॅयद॑न्वा॒रभे॑त प्र॒मायु॑को॒ यज॑मानः स्या॒दथो॒ खल्वा॑हुः सुव॒र्गाय॒ वा ए॒ष लो॒काय॑ नीयते॒ यत्- [  ] \newline

\textbf{Pada Paata} \newline

पर्य॒ग्नीति॒ परि॑ - अ॒ग्नि॒ । क॒रो॒ति॒ । स॒र्व॒हुत॒मिति॑ सर्व-हुत᳚म् । ए॒व । ए॒न॒म् । क॒रो॒ति॒ । अस्क॑न्दाय । अस्क॑न्नम् । हि । तत् । यत् । हु॒तस्य॑ । स्कन्द॑ति । त्रिः । पर्य॒ग्नीति॒ परि॑ - अ॒ग्नि॒ । क॒रो॒ति॒ । त्र्या॑वृ॒दिति॒ त्रि - आ॒वृ॒त् । हि । य॒ज्ञ्ः । अथो॒ इति॑ । रक्ष॑साम् । अप॑हत्या॒ इत्यप॑ - ह॒त्यै॒ । ब्र॒ह्म॒वा॒दिन॒ इति॑ ब्रह्म-वा॒दिनः॑ । व॒द॒न्ति॒ । अ॒न्वा॒रभ्य॒ इत्य॑नु - आ॒रभ्यः॑ । प॒शू(3)ः । न । अ॒न्वा॒रभ्या(3) इत्य॑नु -आ॒रभ्या(3)ः । इति॑ । मृ॒त्यवे᳚ । वै । ए॒षः । नी॒य॒ते॒ । यत् ।  प॒शुः । तम् । यत् । अ॒न्वा॒रभे॒तेत्य॑नु-आ॒रभे॑त । प्र॒मायु॑क॒ इति॑ प्र-मायु॑कः । यज॑मानः । स्या॒त् । अथो॒ इति॑ । खलु॑ । आ॒हुः॒ । सु॒व॒र्गायेति॑ सुवः - गाय॑ । वै । ए॒षः । लो॒काय॑ । नी॒य॒ते॒ । यत् ।  \newline




\markright{ TS 6.3.8.2  \hfill https://www.vedavms.in \hfill}

\section{ TS 6.3.8.2 }

\textbf{TS 6.3.8.2 } \newline
\textbf{Samhita Paata} \newline

प॒शुरिति॒ यन्नान्वा॒रभे॑त सुव॒र्गाल्लो॒काद् यज॑मानो हीयेत वपा॒श्रप॑णीभ्या-म॒न्वार॑भते॒ तन्नेवा॒न्वार॑ब्धं॒ नेवान॑न्वारब्ध॒मुप॒ प्रेष्य॑ होतर्.ह॒व्या दे॒वेभ्य॒ इत्या॑हेषि॒तꣳ हि कर्म॑ क्रि॒यते॒ रेव॑तीर्य॒ज्ञ्प॑तिं प्रिय॒धा ऽऽवि॑श॒तेत्या॑ह यथाय॒जुरे॒वैतद॒ग्निना॑ पु॒रस्ता॑देति॒ रक्ष॑सा॒मप॑हत्यै पृथि॒व्याः स॒पृंचः॑ पा॒हीति॑ ब॒र्॒.हि- [  ] \newline

\textbf{Pada Paata} \newline

प॒शुः । इति॑ । यत् । न । अ॒न्वा॒रभे॒तेत्य॑नु - आ॒रभे॑त । सु॒व॒र्गादिति॑ सुवः - गात् । लो॒कात् । यज॑मानः । ही॒ये॒त॒ । व॒पा॒श्रप॑णीभ्या॒मिति॑ वपा - श्रप॑णीभ्याम् । अन्वार॑भत॒ इत्य॑नु - आर॑भते । तत् । न । इ॒व॒ । अ॒न्वार॑ब्ध॒मित्य॑नु-आर॑ब्धम् । न । इ॒व॒ । अन॑न्वारब्ध॒मित्यन॑नु-आ॒र॒ब्ध॒म् । उप॑ । प्रेति॑ । इ॒ष्य॒ । हो॒तः॒ । ह॒व्या । दे॒वेभ्यः॑ । इति॑ । आ॒ह॒ । इ॒षि॒तम् । हि । कर्म॑ । क्रि॒यते᳚ । रेव॑तीः । य॒ज्ञ्प॑ति॒मिति॑ य॒ज्ञ् - प॒ति॒म् । प्रि॒य॒धेति॑ प्रिय - धा । एति॑ । वि॒श॒त॒ । इति॑ । आ॒ह॒ । य॒था॒य॒जुरिति॑ यथा - य॒जुः । ए॒व । ए॒तत् । अ॒ग्निना᳚ । पु॒रस्ता᳚त् । ए॒ति॒ । रक्ष॑साम् । अप॑हत्या॒ इत्यप॑ - ह॒त्यै॒ । पृ॒थि॒व्याः । स॒पृंच॒ इति॑ सं - पृचः॑ । पा॒हि॒ । इति॑ । ब॒र्॒.हिः ।  \newline




\markright{ TS 6.3.8.3  \hfill https://www.vedavms.in \hfill}

\section{ TS 6.3.8.3 }

\textbf{TS 6.3.8.3 } \newline
\textbf{Samhita Paata} \newline

-रुपा᳚स्य॒त्य-स्क॑न्दा॒या-स्क॑न्नꣳ॒॒ हि तद् यद् ब॒र्॒.हिषि॒ स्कन्द॒त्यथो॑ बर्.हि॒षद॑मे॒वैनं॑ करोति॒ परा॒ङा व॑र्ततेऽद्ध्व॒र्युः प॒शोः स᳚ज्ञ्ं॒प्यमा॑नात् प॒शुभ्य॑ ए॒व तन्नि ह्नु॑त आ॒त्मनोऽना᳚व्रस्काय॒ गच्छ॑ति॒ श्रियं॒ प्र प॒शूना᳚प्नोति॒ य ए॒वं ॅवेद॑ प॒श्चाल्लो॑का॒ वा ए॒षा प्राच्यु॒दानी॑यते॒ यत् पत्नी॒ नम॑स्त आता॒नेत्या॑हाऽऽ*दि॒त्यस्य॒ वै र॒श्मय॑- [  ] \newline

\textbf{Pada Paata} \newline

उपेति॑ । अ॒स्य॒ति॒ । अस्क॑न्दाय । अस्क॑न्नम् । हि । तत् । यत् । ब॒र्॒.हिषि॑ । स्कन्द॑ति । अथो॒ इति॑ । ब॒र्॒.हि॒षद॒मिति॑ बर्.हि - सद᳚म् । ए॒व । ए॒न॒म् । क॒रो॒ति॒ । पराङ्॑ । एति॑ । व॒र्त॒ते॒ । अ॒द्ध्व॒र्युः । प॒शोः । स॒ज्ञ्ं॒प्यमा॑ना॒दिति॑ सं - ज्ञ्॒प्यमा॑नात् । प॒शुभ्य॒ इति॑ प॒शु - भ्यः॒ । ए॒व । तत् । नीति॑ । ह्नु॒ते॒ । आ॒त्मनः॑ । अना᳚व्रस्का॒येत्यना᳚-व्र॒स्का॒य॒ । गच्छ॑ति । श्रिय᳚म् । प्रेति॑ । प॒शून् । आ॒प्नो॒ति॒ । यः । ए॒वम् । वेद॑ । प॒श्चाल्लो॒केति॑ प॒श्चात् - लो॒का॒ । वै । ए॒षा । प्राची᳚ । उ॒दानी॑यत॒ इत्यु॑त् - आनी॑यते । यत् । पत्नी᳚ । नमः॑ । ते॒ । आ॒ता॒नेत्या᳚ - ता॒न॒ । इति॑ । आ॒ह॒ । आ॒दि॒त्यस्य॑ । वै । र॒श्मयः॑ ।  \newline




\markright{ TS 6.3.8.4  \hfill https://www.vedavms.in \hfill}

\section{ TS 6.3.8.4 }

\textbf{TS 6.3.8.4 } \newline
\textbf{Samhita Paata} \newline

आता॒नास्तेभ्य॑ ए॒व नम॑स्करोत्यन॒र्वा प्रेहीत्या॑ह॒ भ्रातृ॑व्यो॒ वा अर्वा॒ भ्रातृ॑व्यापनुत्त्यै घृ॒तस्य॑ कु॒ल्यामनु॑ स॒ह प्र॒जया॑ स॒ह रा॒यस्पोषे॒णे-त्या॑हा॒ ऽऽ*शिष॑मे॒वैतामा शा᳚स्त॒ आपो॑ देवीः शुद्धायुव॒ इत्या॑ह यथाय॒जुरे॒वैतत् ॥ \newline

\textbf{Pada Paata} \newline

आ॒ता॒ना इत्या᳚ - ता॒नाः । तेभ्यः॑ । ए॒व । नमः॑ । क॒रो॒ति॒ । अ॒न॒र्वा । प्रेति॑ । इ॒हि॒ । इति॑ । आ॒ह॒ । भ्रातृ॑व्यः । वै । अर्वा᳚ । भ्रातृ॑व्यापनुत्त्या॒ इति॒ भ्रातृ॑व्य - अ॒प॒नु॒त्त्यै॒ । घृ॒तस्य॑ । कु॒ल्याम् । अन्विति॑ । स॒ह । प्र॒जयेति॑ प्र - जया᳚ । स॒ह । रा॒यः । पोषे॑ण । इति॑ । आ॒ह॒ । आ॒शिष॒मित्या᳚ - शिष᳚म् । ए॒व । ए॒ताम् । एति॑ । शा॒स्ते॒ । आपः॑ ।दे॒वीः॒ । शु॒द्धा॒यु॒व॒ इति॑ शुद्ध - यु॒वः॒ । इति॑ । आ॒ह॒ । य॒था॒य॒जुरिति॑ यथा - य॒जुः । ए॒व । ए॒तत् ॥  \newline




\markright{ TS 6.3.9.1  \hfill https://www.vedavms.in \hfill}

\section{ TS 6.3.9.1 }

\textbf{TS 6.3.9.1 } \newline
\textbf{Samhita Paata} \newline

प॒शोर्वा आल॑ब्धस्य प्रा॒णाञ्छुगृ॑च्छति॒ वाक्त॒ आ प्या॑यतां प्रा॒णस्त॒ आ प्या॑यता॒मित्या॑ह प्रा॒णेभ्य॑ ए॒वास्य॒ शुचꣳ॑ शमयति॒ सा प्रा॒णेभ्योऽधि॑ पृथि॒वीꣳ शुक् प्र वि॑शति॒ शमहो᳚भ्या॒मिति॒ नि न॑यत्यहोरा॒त्राभ्या॑मे॒व पृ॑थि॒व्यै शुचꣳ॑ शमय॒त्योष॑धे॒ त्रा॑यस्वैनꣳ॒॒ स्वधि॑ते॒ मैनꣳ॑ हिꣳसी॒रित्या॑ह॒ वज्रो॒॒ वै स्वधि॑तिः॒- [  ] \newline

\textbf{Pada Paata} \newline

प॒शोः । वै । आल॑ब्ध॒स्येत्या - ल॒ब्ध॒स्य॒ । प्रा॒णानिति॑ प्र - अ॒नान् । शुक् । ऋ॒च्छ॒ति॒ । वाक् । ते॒ । एति॑ । प्या॒य॒ता॒म् । प्रा॒ण इति॑ प्र - अ॒नः । ते॒ । एति॑ । प्या॒य॒ता॒म् । इति॑ । आ॒ह॒ । प्रा॒णेभ्य॒ इति॑ प्र - अ॒नेभ्यः॑ । ए॒व । अ॒स्य॒ । शुच᳚म् । श॒म॒य॒ति॒ । सा । प्रा॒णेभ्य॒ इति॑ प्र - अ॒नेभ्यः॑ । अधीति॑ । पृ॒थि॒वीम् । शुक् । प्रेति॑ । वि॒श॒ति॒ । शम् । अहो᳚भ्या॒मित्यहः॑ - भ्या॒म् । इति॑ । नीति॑ । न॒य॒ति॒ । अ॒हो॒रा॒त्राभ्या॒मित्य॑हः-रा॒त्राभ्या᳚म् । ए॒व । पृ॒थि॒व्यै । शुच᳚म् । श॒म॒य॒ति॒ । ओष॑धे । त्राय॑स्व । ए॒न॒म् । स्वधि॑त॒ इति॒ स्व - धि॒ते॒ । मा । ए॒न॒म् । हिꣳ॒॒सीः॒ । इति॑ । आ॒ह॒ । वज्रः॑ । वै । स्वधि॑ति॒रिति॒ स्व - धि॒तिः॒ ।  \newline




\markright{ TS 6.3.9.2  \hfill https://www.vedavms.in \hfill}

\section{ TS 6.3.9.2 }

\textbf{TS 6.3.9.2 } \newline
\textbf{Samhita Paata} \newline

शान्त्यै॑ पार्श्व॒त आ च्छ्य॑ति मद्ध्य॒तो हि म॑नु॒ष्या॑ आ॒ च्छ्यन्ति॑ तिर॒श्चीन॒मा च्छ्य॑त्यनू॒चीनꣳ॒॒ हि म॑नु॒ष्या॑ आ॒च्छ्यन्ति॒ व्यावृ॑त्त्यै॒ रक्ष॑सां भा॒गो॑ऽसीति॑ स्थविम॒तो ब॒र्॒.हिर॒क्त्वाऽपा᳚स्यत्य॒स्नैव रक्षाꣳ॑सि नि॒रव॑दयत इ॒दम॒हꣳ रक्षो॑ऽध॒मं तमो॑ नयामि॒ यो᳚ऽस्मान् द्वेष्टि॒ यं च॑ व॒यं द्वि॒ष्म इत्या॑ह॒ द्वौ वाव पुरु॑षौ॒ यं चै॒व- [  ] \newline

\textbf{Pada Paata} \newline

शान्त्यै᳚ । पा॒र्श्व॒तः । एति॑ । छ्‌य॒ति॒ । म॒द्ध्य॒तः । हि । म॒नु॒ष्याः᳚ । आ॒च्छ्यन्तीत्या᳚-छ्यन्ति॑ । ति॒र॒श्चीन᳚म् । एति॑ । छ्य॒ति॒ । अ॒नू॒चीन᳚म् । हि । म॒नु॒ष्याः᳚ । आ॒च्छ्यन्तीत्या᳚ - छ्यन्ति॑ । व्यावृ॑त्त्या॒ इति॑ वि - आवृ॑त्त्यै । रक्ष॑साम् । भा॒गः । अ॒सि॒ । इति॑ । स्थ॒वि॒म॒तः । ब॒र्॒.हिः । अ॒क्त्वा । अपेति॑ । अ॒स्य॒ति॒ । अ॒स्ना । ए॒व । रक्षाꣳ॑सि । नि॒रव॑दयत॒ इति॑ निः - अव॑दयते । इ॒दम् । अ॒हम् । रक्षः॑ । अ॒ध॒मम् । तमः॑ । न॒या॒मि॒ । यः । अ॒स्मान् । द्वेष्टि॑ । यम् । च॒ । व॒यम् । द्वि॒ष्मः । इति॑ । आ॒ह॒ । द्वौ । वाव । पुरु॑षौ । यम् । च॒ । ए॒व ।  \newline




\markright{ TS 6.3.9.3  \hfill https://www.vedavms.in \hfill}

\section{ TS 6.3.9.3 }

\textbf{TS 6.3.9.3 } \newline
\textbf{Samhita Paata} \newline

द्वेष्टि॒ यश्चै॑नं॒ द्वेष्टि॒ तावु॒भाव॑ध॒मं तमो॑ नयती॒षे त्वेति॑ व॒पामुत्खि॑दती॒च्छत॑ इव॒ ह्ये॑ष यो यज॑ते॒ यदु॑पतृ॒न्द्याद्-रु॒द्रो᳚ऽस्य प॒शून् घातु॑कः स्या॒द् यन्नोप॑तृ॒न्द्या-दय॑ता स्या-द॒न्ययो॑पतृ॒णत्त्य॒न्यया॒ न धृत्यै॑ घृ॒तेन॑ द्यावापृथिवी॒ प्रोर्ण्वा॑था॒मित्या॑ह॒ द्यावा॑पृथि॒वी ए॒व रसे॑नान॒क्त्यच्छि॑न्नो॒- [  ] \newline

\textbf{Pada Paata} \newline

द्वेष्टि॑ । यः । च॒ । ए॒न॒म् । द्वेष्टि॑ । तौ । उ॒भौ । अ॒ध॒मम् । तमः॑ । न॒य॒ति॒ । इ॒षे । त्वा॒ । इति॑ । व॒पाम् । उदिति॑ । खि॒द॒ति॒ । इ॒च्छते᳚ । इ॒व॒ । हि । ए॒षः । यः । यज॑ते । यत् । उ॒प॒तृ॒न्द्यादित्यु॑प - तृ॒न्द्यात् । रु॒द्रः । अ॒स्य॒ । प॒शून् । घातु॑कः । स्या॒त् । यत् । न । उ॒प॒तृ॒न्द्यादित्यु॑प - तृ॒न्द्यात् । अय॑ता । स्या॒त् । अ॒न्यया᳚ । उ॒प॒तृ॒णत्तीत्यु॑प - तृ॒णत्ति॑ । अ॒न्यया᳚ । न । धृत्यै᳚ । घृ॒तेन॑ । द्या॒वा॒पृ॒थि॒वी॒ इति॑ द्यावा-पृ॒थि॒वी॒ । प्रेति॑ । ऊ॒र्ण्वा॒था॒म् । इति॑ । आ॒ह॒ । द्यावा॑पृथि॒वी इति॒ द्यावा᳚ - पृ॒थि॒वी । ए॒व । रसे॑न । अ॒न॒क्ति॒ । अच्छि॑न्नः ।  \newline




\markright{ TS 6.3.9.4  \hfill https://www.vedavms.in \hfill}

\section{ TS 6.3.9.4 }

\textbf{TS 6.3.9.4 } \newline
\textbf{Samhita Paata} \newline

रायः॑ सु॒वीर॒ इत्या॑ह यथाय॒जुरे॒वैतत् क्रू॒रमि॑व॒ वा ए॒तत् क॑रोति॒ यद् व॒पा-मु॑त्खि॒द-त्यु॒र्व॑न्तरि॑क्ष॒-मन्वि॒हीत्या॑ह॒ शान्त्यै॒ प्र वा ए॒षो᳚ऽस्माल्लो॒काच्च्य॑वते॒ यः प॒शुं मृ॒त्यवे॑ नी॒यमा॑नमन्वा॒रभ॑ते वपा॒श्रप॑णी॒ पुन॑र॒न्वार॑भते॒ऽस्मिन्ने॒व लो॒के प्रति॑ तिष्ठत्य॒ग्निना॑ पु॒रस्ता॑देति॒ रक्ष॑सा॒मप॑हत्या॒ अथो॑ दे॒वता॑ ए॒व ह॒व्येना- [  ] \newline

\textbf{Pada Paata} \newline

रायः॑ । सु॒वीर॒ इति॑ सु - वीरः॑ । इति॑ । आ॒ह॒ । य॒था॒य॒जुरिति॑ यथा - य॒जुः । ए॒व । ए॒तत् । क्रू॒रम् । इ॒व॒ । वै । ए॒तत् । क॒रो॒ति॒ । यत् । व॒पाम् । उ॒त्खि॒दतीत्यु॑त् - खि॒दति॑ । उ॒रु । अ॒न्तरि॑क्षम् । अन्विति॑ । इ॒हि॒ । इति॑ । आ॒ह॒ । शान्त्यै᳚ । प्रेति॑ । वै । ए॒षः । अ॒स्मात् । लो॒कात् । च्य॒व॒ते॒ । यः । प॒शुम् । मृ॒त्यवे᳚ । नी॒यमा॑नम् । अ॒न्वा॒रभ॑त॒ इत्य॑नु - आ॒रभ॑ते । व॒पा॒श्रप॑णी॒ इति॑ वपा - श्रप॑णी । पुनः॑ । अ॒न्वार॑भत॒ इत्य॑नु - आर॑भते । अ॒स्मिन्न् । ए॒व । लो॒के । प्रतीति॑ । ति॒ष्ठ॒ति॒ । अ॒ग्निना᳚ । पु॒रस्ता᳚त् । ए॒ति॒ । रक्ष॑साम् । अप॑हत्या॒ इत्यप॑-ह॒त्यै॒ । अथो॒ इति॑ । दे॒वताः᳚ । ए॒व । ह॒व्येन॑ ।  \newline




\markright{ TS 6.3.9.5  \hfill https://www.vedavms.in \hfill}

\section{ TS 6.3.9.5 }

\textbf{TS 6.3.9.5 } \newline
\textbf{Samhita Paata} \newline

-न्वे॑ति॒ नान्त॒ममङ्गा॑र॒मति॑ हरे॒द् यद॑न्त॒ममङ्गा॑रमति॒ हरे᳚द्-दे॒वता॒ अति॑ मन्येत॒ वायो॒ वीहि॑ स्तो॒काना॒मित्या॑ह॒ तस्मा॒द् विभ॑क्ताः स्तो॒का अव॑ पद्य॒न्तेऽग्रं॒ ॅवा ए॒तत् प॑शू॒नां ॅयद् व॒पाऽग्र॒मोष॑धीनां ब॒र्॒.हिरग्रे॑णै॒वाग्रꣳ॒॒ सम॑र्द्धय॒त्यथो॒ ओष॑धीष्वे॒व प॒शून् प्रति॑ष्ठापयति॒ स्वाहा॑कृतीभ्यः॒ प्रेष्येत्या॑ह- [  ] \newline

\textbf{Pada Paata} \newline

अन्विति॑ । ए॒ति॒ । न । अ॒न्त॒मम् । अङ्गा॑रम् । अतीति॑ । ह॒रे॒त् । यत् । अ॒न्त॒मम् । अङ्गा॑रम् । अ॒ति॒हरे॒दित्य॑ति - हरे᳚त् । दे॒वताः᳚ । अतीति॑ । म॒न्ये॒त॒ । वायो॒ इति॑ । वीति॑ । इ॒हि॒ । स्तो॒काना᳚म् । इति॑ । आ॒ह॒ । तस्मा᳚त् । विभ॑क्ता॒ इति॒ वि - भ॒क्ताः॒ । स्तो॒काः । अवेति॑ । प॒द्य॒न्ते॒ । अग्र᳚म् । वै । ए॒तत् । प॒शू॒नाम् । यत् । व॒पा । अग्र᳚म् । ओष॑धीनाम् । ब॒र्॒.हिः । अग्रे॑ण । ए॒व । अग्र᳚म् । समिति॑ । अ॒द्‌र्ध॒य॒ति॒ । अथो॒ इति॑ । ओष॑धीषु । ए॒व । प॒शून् । प्रतीति॑ । स्था॒प॒य॒ति॒ । स्वाहा॑कृतीभ्य॒ इति॒ स्वाहा॑कृति - भ्यः॒ । प्रेति॑ । इ॒ष्य॒ । इति॑ । आ॒ह॒ ।  \newline




\markright{ TS 6.3.9.6  \hfill https://www.vedavms.in \hfill}

\section{ TS 6.3.9.6 }

\textbf{TS 6.3.9.6 } \newline
\textbf{Samhita Paata} \newline

य॒ज्ञ्स्य॒ समि॑ष्ट्यै प्राणापा॒नौ वा ए॒तौ प॑शू॒नां ॅयत् पृ॑षदा॒ज्यमा॒त्मा व॒पा पृ॑षदा॒ज्यम॑भि॒घार्य॑ व॒पाम॒भि घा॑रयत्या॒त्मन्ने॒व प॑शू॒नां प्रा॑णापा॒नौ द॑धाति॒ स्वाहो॒र्द्ध्वन॑भसं मारु॒तं ग॑च्छत॒मित्या॑हो॒र्द्ध्वन॑भा ह स्म॒ वै मा॑रु॒तो दे॒वानां᳚ ॅवपा॒श्रप॑णी॒ प्र ह॑रति॒ तेनै॒वैने॒ प्र ह॑रति॒ विषू॑ची॒ प्र ह॑रति॒ तस्मा॒द् विष्व॑ञ्चौ प्राणापा॒नौ ॥ \newline

\textbf{Pada Paata} \newline

य॒ज्ञ्स्य॑ । समि॑ष्ट्या॒ इति॒ सं-इ॒ष्ट्यै॒ । प्रा॒णा॒पा॒नाविति॑ प्राण-अ॒पा॒नौ । वै । ए॒तौ । प॒शू॒नाम् । यत् । पृ॒ष॒दा॒ज्यमिति॑ पृषत् - आ॒ज्यम् । आ॒त्मा । व॒पा । पृ॒ष॒दा॒ज्यमिति॑ पृषत् - आ॒ज्यम् । अ॒भि॒घार्येत्य॑भि - घार्य॑ । व॒पाम् । अ॒भीति॑ । घा॒र॒य॒ति॒ । आ॒त्मन् । ए॒व । प॒शू॒नाम् । प्रा॒णा॒पा॒नाविति॑ प्राण - अ॒पा॒नौ । द॒धा॒ति॒ । स्वाहा᳚ । ऊ॒द्‌र्ध्वन॑भस॒मित्यू॒द्‌र्ध्व - न॒भ॒स॒म् । मा॒रु॒तम् । ग॒च्छ॒त॒म् । इति॑ । आ॒ह॒ । ऊ॒द्‌र्ध्वन॑भा॒ इत्यू॒द्‌र्ध्व - न॒भाः॒ । ह॒ । स्म॒ । वै । मा॒रु॒तः । दे॒वाना᳚म् । व॒पा॒श्रप॑णी॒ इति॑ वपा - श्रप॑णी । प्रेति॑ । ह॒र॒ति॒ । तेन॑ । ए॒व । ए॒ने॒ इति॑ । प्रेति॑ । ह॒र॒ति॒ । विषू॑ची॒ इति॑ । प्रेति॑ । ह॒र॒ति॒ । तस्मा᳚त् । विष्व॑ञ्चौ । प्रा॒णा॒पा॒नाविति॑ प्राण - अ॒पा॒नौ ॥  \newline




\markright{ TS 6.3.10.1  \hfill https://www.vedavms.in \hfill}

\section{ TS 6.3.10.1 }

\textbf{TS 6.3.10.1 } \newline
\textbf{Samhita Paata} \newline

प॒शुमा॒लभ्य॑ पुरो॒डाशं॒ निर्व॑पति॒ समे॑धमे॒वैन॒मा ल॑भते व॒पया᳚ प्र॒चर्य॑ पुरो॒डाशे॑न॒ प्र च॑र॒त्यूर्ग्वै पु॑रो॒डाश॒ ऊर्ज॑मे॒व प॑शू॒नां म॑द्ध्य॒तो द॑धा॒त्यथो॑ प॒शोरे॒व छि॒द्रमपि॑ दधाति पृषदा॒ज्यस्यो॑प॒हत्य॒ त्रिः पृ॑च्छति शृ॒तꣳ ह॒वी(3)ः श॑मित॒रिति॒ त्रिष॑त्या॒ हि दे॒वा योऽशृ॑तꣳ शृ॒तमाह॒ स एन॑सा प्राणापा॒नौ वा ए॒तौ प॑शू॒नां- [  ] \newline

\textbf{Pada Paata} \newline

प॒शुम् । आ॒लभ्येत्या᳚ - लभ्य॑ । पु॒रो॒डाश᳚म् । निरिति॑ । व॒प॒ति॒ । समे॑ध॒मिति॒ स - मे॒ध॒म् । ए॒व । ए॒न॒म् । एति॑ । ल॒भ॒ते॒ । व॒पया᳚ । प्र॒चर्येति॑ प्र-चर्य॑ । पु॒रो॒डाशे॑न । प्रेति॑ । च॒र॒ति॒ । ऊर्क् । वै । पु॒रो॒डाशः॑ । ऊर्ज᳚म् । ए॒व । प॒शू॒नाम् । म॒द्ध्य॒तः । द॒धा॒ति॒ । अथो॒ इति॑ । प॒शोः । ए॒व । छि॒द्रम् । अपीति॑ । द॒धा॒ति॒ । पृ॒ष॒दा॒ज्यस्येति॑ पृषत् - आ॒ज्यस्य॑ । उ॒प॒हत्येत्यु॑प - हत्य॑ । त्रिः । पृ॒च्छ॒ति॒ । शृ॒तम् । ह॒वी(3)ः । श॒मि॒तः॒ । इति॑ । त्रिष॑त्या॒ इति॒ त्रि - स॒त्याः॒ । हि । दे॒वाः । यः । अशृ॑तम् । शृ॒तम् । आह॑ । सः । एन॑सा । प्रा॒णा॒पा॒नाविति॑ प्राण - अ॒पा॒नौ । वै । ए॒तौ । प॒शू॒नां ।  \newline




\markright{ TS 6.3.10.2  \hfill https://www.vedavms.in \hfill}

\section{ TS 6.3.10.2 }

\textbf{TS 6.3.10.2 } \newline
\textbf{Samhita Paata} \newline

ॅयत् पृ॑षदा॒ज्यं प॒शोः खलु॒ वा आल॑ब्धस्य॒ हृद॑यमा॒त्माऽभि समे॑ति॒ यत् पृ॑षदा॒ज्येन॒ हृद॑य-मभिघा॒रय॑त्या॒त्मन्ने॒व प॑शू॒नां प्रा॑णापा॒नौ द॑धाति प॒शुना॒ वै दे॒वाः सु॑व॒र्गं ॅलो॒कमा॑य॒न् ते॑ऽमन्यन्त मनु॒ष्या॑ नो॒ऽन्वाभ॑विष्य॒न्तीति॒ तस्य॒ शिरः॑ छि॒त्त्वा मेधं॒ प्राक्षा॑रय॒न्थ्स प्र॒क्षो॑ऽभव॒त् तत् प्र॒क्षस्य॑ प्रक्ष॒त्वं ॅयत् प्ल॑क्षशा॒खो-त्त॑रब॒र्॒.हि-र्भव॑ति॒ समे॑धस्यै॒व- [  ] \newline

\textbf{Pada Paata} \newline

यत् । पृ॒ष॒दा॒ज्यमिति॑ पृषत् - आ॒ज्यम् । प॒शोः । खलु॑ । वै । आल॑ब्ध॒स्येत्या - ल॒ब्ध॒स्य॒ । हृद॑यम् । आ॒त्मा । अ॒भि । समिति॑ । ए॒ति॒ । यत् । पृ॒ष॒दा॒ज्येनेति॑ पृषत् - आ॒ज्येन॑ । हृद॑यम् । अ॒भि॒घा॒रय॒तीत्य॑भि - घा॒रय॑ति । आ॒त्मन्न् । ए॒व । प॒शू॒नाम् । प्रा॒णा॒पा॒नाविति॑ प्राण - अ॒पा॒नौ । द॒धा॒ति॒ । प॒शुना᳚ । वै । दे॒वाः । सु॒व॒र्गमिति॑ सुवः - गम् । लो॒कम् । आ॒य॒न्न् । ते । अ॒म॒न्य॒न्त॒ । म॒नु॒ष्याः᳚ । नः॒ । अ॒न्वाभ॑विष्य॒न्तीत्य॑नु-आभ॑विष्यन्ति । इति॑ । तस्य॑ । शिरः॑ । छि॒त्त्वा । मेध᳚म् । प्रेति॑ । अ॒क्षा॒र॒य॒न्न् । सः । प्र॒क्षः । अ॒भ॒व॒त् । तत् । प्र॒क्षस्य॑ । प्र॒क्ष॒त्वमिति॑ प्रक्ष - त्वम् । यत् । प्ल॒क्ष॒शा॒खेति॑ प्लक्ष - शा॒खा । उ॒त्त॒र॒ब॒र्॒.हिरित्युत्त॑र - ब॒र॒.हिः । भव॑ति । समे॑ध॒स्येति॒ स - मे॒ध॒स्य॒ । ए॒व ।  \newline




\markright{ TS 6.3.10.3  \hfill https://www.vedavms.in \hfill}

\section{ TS 6.3.10.3 }

\textbf{TS 6.3.10.3 } \newline
\textbf{Samhita Paata} \newline

प॒शोरव॑ द्यति प॒शुं ॅवै ह्रि॒यमा॑णꣳ॒॒ रक्षाꣳ॒॒स्यनु॑ सचन्तेऽन्त॒रा यूपं॑ चाऽऽ*हव॒नीयं॑ च हरति॒ रक्ष॑सा॒मप॑हत्यै प॒शोर्वा आल॑ब्धस्य॒ मनोऽप॑ क्रामति म॒नोता॑यै ह॒विषो॑ऽवदी॒यमा॑न॒स्यानु॑ ब्रू॒हीत्या॑ह॒ मन॑ ए॒वास्याव॑ रुन्ध॒ एका॑दशाव॒दाना॒न्यव॑ द्यति॒ दश॒ वै प॒शोः प्रा॒णा आ॒त्मैका॑द॒शो यावा॑ने॒व प॒शुस्तस्याव॑- [  ] \newline

\textbf{Pada Paata} \newline

प॒शोः । अवेति॑ । द्य॒ति॒ । प॒शुम् । वै । ह्रि॒यमा॑णम् । रक्षाꣳ॑सि । अन्विति॑ । स॒च॒न्ते॒ । अ॒न्त॒रा । यूप᳚म् । च॒ । आ॒ह॒व॒नीय॒मित्या᳚ - ह॒व॒नीय᳚म् । च॒ । ह॒र॒ति॒ । रक्ष॑साम् । अप॑हत्या॒ इत्यप॑ - ह॒त्यै॒ । प॒शोः । वै । आल॑ब्ध॒स्येत्या - ल॒ब्ध॒स्य॒ । मनः॑ । अपेति॑ । क्रा॒म॒ति॒ । म॒नोता॑यै । ह॒विषः॑ । अ॒व॒दी॒यमा॑न॒स्येत्य॑व - दी॒यमा॑नस्य । अन्विति॑ । ब्रू॒हि॒ । इति॑ । आ॒ह॒ । मनः॑ । ए॒व । अ॒स्य॒ । अवेति॑ । रु॒न्धे॒ । एका॑दश । अ॒व॒दाना॒नीत्य॑व - दाना॑नि । अवेति॑ । द्य॒ति॒ । दश॑ । वै । प॒शोः । प्रा॒णा इति॑ प्र - अ॒नाः । आ॒त्मा । ए॒का॒द॒शः । यावान्॑ । ए॒व । प॒शुः । तस्य॑ । अवेति॑ ।  \newline




\markright{ TS 6.3.10.4  \hfill https://www.vedavms.in \hfill}

\section{ TS 6.3.10.4 }

\textbf{TS 6.3.10.4 } \newline
\textbf{Samhita Paata} \newline

द्यति॒ हृद॑य॒स्याग्रेऽव॑ द्य॒त्यथ॑ जि॒ह्वाया॒ अथ॒ वक्ष॑सो॒ यद्वै हृद॑येनाभि॒गच्छ॑ति॒ तज्जि॒ह्वया॑ वदति॒ यज्जि॒ह्वया॒ वद॑ति॒ तदुर॒सोऽधि॒ निर्व॑दत्ये॒तद्वै प॒शोर्य॑थापू॒र्वं ॅयस्यै॒वम॑व॒दाय॑ यथा॒काम॒-मुत्त॑रेषामव॒द्यति॑ यथा पू॒र्वमे॒वास्य॑ प॒शोरव॑त्तं भवति मद्ध्य॒तो गु॒दस्याव॑ द्यति मद्ध्य॒तो हि प्रा॒ण उ॑त्त॒मस्याव॑ द्यत्यु- [  ] \newline

\textbf{Pada Paata} \newline

द्य॒ति॒ । हृद॑यस्य । अग्रे᳚ । अवेति॑ । द्य॒ति॒ । अथ॑ । जि॒ह्वायाः᳚ । अथ॑ । वक्ष॑सः । यत् । वै । हृद॑येन । अ॒भि॒गच्छ॒तीत्य॑भि - गच्छ॑ति । तत् । जि॒ह्वया᳚ । व॒द॒ति॒ । यत् । जि॒ह्वया᳚ । वद॑ति । तत् । उर॑सः । अधि॑ । निरिति॑ । व॒द॒ति॒ । ए॒तत् । वै । प॒शोः । य॒था॒पू॒र्वमिति॑ यथा - पू॒र्वम् । यस्य॑ । ए॒वम् । अ॒व॒दायेत्य॑व - दाय॑ । य॒था॒काम॒मिति॑ यथा - काम᳚म् । उत्त॑रेषा॒मित्युत् - त॒रे॒षा॒म् । अ॒व॒द्यतीत्य॑व - द्यति॑ । य॒था॒पू॒र्वमिति॑ यथा - पू॒र्वम् । ए॒व । अ॒स्य॒ । प॒शोः । अव॑त्तम् । भ॒व॒ति॒ । म॒द्ध्य॒तः । गु॒दस्य॑ । अवेति॑ । द्य॒ति॒ । म॒द्ध्य॒तः । हि । प्रा॒ण इति॑ प्र - अ॒नः । उ॒त्त॒मस्येत्यु॑त् - त॒मस्य॑ । अवेति॑ । द्य॒ति॒ ।  \newline




\markright{ TS 6.3.10.5  \hfill https://www.vedavms.in \hfill}

\section{ TS 6.3.10.5 }

\textbf{TS 6.3.10.5 } \newline
\textbf{Samhita Paata} \newline

-त्त॒मो हि प्रा॒णो यदीत॑रं॒ ॅयदीत॑र-मु॒भय॑मे॒वाजा॑मि॒ जाय॑मानो॒ वै ब्रा᳚ह्म॒णस्त्रि॒भिर्.ऋ॑ण॒वा जा॑यते ब्रह्म॒चर्ये॒णर्षि॑भ्यो य॒ज्ञेन॑ दे॒वेभ्यः॑ प्र॒जया॑ पि॒तृभ्य॑ ए॒ष वा अ॑नृ॒णो यः पु॒त्री यज्वा᳚ ब्रह्मचारिवा॒सी तद॑व॒दानै॑रे॒वाव॑ दयत॒ तद॑व॒दाना॑ना-मवदान॒त्वं दे॑वासु॒राः संॅय॑त्ता आस॒न् ते दे॒वा अ॒ग्निम॑ब्रुव॒न् त्वया॑ वी॒रेणासु॑रान॒भि भ॑वा॒मेति॒- [  ] \newline

\textbf{Pada Paata} \newline

उ॒त्त॒म इत्यु॑त् - त॒मः । हि । प्रा॒ण इति॑ प्र - अ॒नः । यदि॑ । इत॑रम् । यदि॑ । इत॑रम् । उ॒भय᳚म् । ए॒व । अजा॑मि । जाय॑मानः । वै । ब्रा॒ह्म॒णः । त्रि॒भिरिति॑ त्रि - भिः । ऋ॒ण॒वेत्यृ॑ण - वा । जा॒य॒ते॒ । ब्र॒ह्म॒चर्ये॒णेति॑ ब्रह्म - चर्ये॑ण । ऋषि॑भ्य॒ इत्यृषि॑ - भ्यः॒ । य॒ज्ञेन॑ । दे॒वेभ्यः॑ । प्र॒जयेति॑ प्र - जया᳚ । पि॒तृभ्य॒ इति॑ पि॒तृ - भ्यः॒ । ए॒षः । वै । अ॒नृ॒णः । यः । पु॒त्री । यज्वा᳚ । ब्र॒ह्म॒चा॒रि॒वा॒सीति॑ ब्रह्मचारि - वा॒सी । तत् । अ॒व॒दानै॒रित्य॑व - दानैः᳚ । ए॒व । अवेति॑ । द॒य॒ते॒ । तत् । अ॒व॒दाना॑ना॒मित्य॑व - दाना॑नाम् । अ॒व॒दा॒न॒त्वमित्य॑वदान - त्वम् । दे॒वा॒सु॒रा इति॑ देव - अ॒सु॒राः । संॅय॑त्ता॒ इति॒ सं - य॒त्ताः॒ । आ॒स॒न्न् । ते । दे॒वाः । अ॒ग्निम् । अ॒ब्रु॒व॒न्न् । त्वया᳚ । वी॒रेण॑ । असु॑रान् । अ॒भीति॑ । भ॒वा॒म॒ । इति॑ ।  \newline




\markright{ TS 6.3.10.6  \hfill https://www.vedavms.in \hfill}

\section{ TS 6.3.10.6 }

\textbf{TS 6.3.10.6 } \newline
\textbf{Samhita Paata} \newline

सो᳚ऽब्रवी॒द् वरं॑ ॅवृणै प॒शोरु॑द्धा॒रमुद्ध॑रा॒ इति॒ स ए॒तमु॑द्धा॒रमुद॑हरत॒ दोः पू᳚र्वा॒र्द्धस्य॑ गु॒दं म॑द्ध्य॒तः श्रोणिं॑ जघना॒र्द्धस्य॒ ततो॑ दे॒वा अभ॑व॒न् पराऽसु॑रा॒ यत् त्र्य॒ङ्गाणाꣳ॑ समव॒द्यति॒ भ्रातृ॑व्याभिभूत्यै॒ भव॑त्या॒त्मना॒ परा᳚ऽस्य॒ भ्रातृ॑व्यो भवत्यक्ष्ण॒याऽव॑ द्यति॒ तस्मा॑दक्ष्ण॒या प॒शवोऽङ्गा॑नि॒ प्र ह॑रन्ति॒ प्रति॑ष्ठित्यै ॥ \newline

\textbf{Pada Paata} \newline

सः । अ॒ब्र॒वी॒त् । वर᳚म् । वृ॒णै॒ । प॒शोः । उ॒द्धा॒रमित्यु॑त् - हा॒रम् । उदिति॑ । ह॒रै॒ । इति॑ । सः । ए॒तम् । उ॒द्धा॒रमित्यु॑त् - हा॒रम् । उदिति॑ । अ॒ह॒र॒त॒ । दोः । पू॒र्वा॒द्‌र्धस्येति॑ पूर्व - अ॒द्‌र्धस्य॑ । गु॒दम् । म॒द्ध्य॒तः । श्रोणि᳚म् । ज॒घ॒ना॒द्‌र्धस्येति॑ जघन-अ॒द्‌र्धस्य॑ । ततः॑ । दे॒वाः । अभ॑वन्न् । परेति॑ । असु॑राः । यत् । त्र्य॒ङ्गाणा॒मिति॑ त्रि - अ॒ङ्गाना᳚म् । स॒म॒व॒द्यतीति॑ सं - अ॒व॒द्यति॑ । भ्रातृ॑व्याभिभूत्या॒ इति॒ भ्रातृ॑व्य-अ॒भि॒भू॒त्यै॒ । भव॑ति । आ॒त्मना᳚ । परेति॑ । अ॒स्य॒ । भ्रातृ॑व्यः । भ॒व॒ति॒ । अ॒क्ष्ण॒या । अवेति॑ । द्य॒ति॒ । तस्मा᳚त् । अ॒क्ष्ण॒या । प॒शवः॑ । अङ्गा॑नि । प्रेति॑ । ह॒र॒न्ति॒ । प्रति॑ष्ठित्या॒ इति॒ प्रति॑ - स्थि॒त्यै॒ ॥  \newline




\markright{ TS 6.3.11.1  \hfill https://www.vedavms.in \hfill}

\section{ TS 6.3.11.1 }

\textbf{TS 6.3.11.1 } \newline
\textbf{Samhita Paata} \newline

मेद॑सा॒ स्रुचौ॒ प्रोर्णो॑ति॒ मेदो॑रूपा॒ वै प॒शवो॑ रू॒पमे॒व प॒शुषु॑ दधाति यू॒षन्न॑व॒धाय॒ प्रोर्णो॑ति॒ रसो॒ वा ए॒ष प॑शू॒नां ॅयद्यू रस॑मे॒व प॒शुषु॑ दधाति पा॒र्श्वेन॑ वसाहो॒मं प्रयौ॑ति॒ मद्ध्यं॒ ॅवा ए॒तत् प॑शू॒नां ॅयत् पा॒र्श्वꣳ रस॑ ए॒ष प॑शू॒नां ॅयद्वसा॒ यत् पा॒र्श्वेन॑ वसाहो॒मं प्र॒यौति॑ मद्ध्य॒त ए॒व प॑शू॒नाꣳ रसं॑ दधाति॒ घ्नन्ति॒- [  ] \newline

\textbf{Pada Paata} \newline

मेद॑सा । स्रुचौ᳚ । प्रेति॑ । ऊ॒र्णो॒ति॒ । मेदो॑रूपा॒ इति॒ मेदः॑ - रू॒पाः॒ । वै । प॒शवः॑ । रू॒पम् । ए॒व । प॒शुषु॑ । द॒धा॒ति॒ । यू॒षन्न् । अ॒व॒धायेत्य॑व-धाय॑ । प्रेति॑ । ऊ॒र्णो॒ति॒ । रसः॑ । वै । ए॒षः । प॒शू॒नाम् । यत् । यूः । रस᳚म् । ए॒व । प॒शुषु॑ । द॒धा॒ति॒ । पा॒र्श्वेन॑ । व॒सा॒हो॒ममिति॑ वसा - हो॒मम् । प्रेति॑ । यौ॒ति॒ । मद्ध्य᳚म् । वै । ए॒तत् । प॒शू॒नाम् । यत् । पा॒र्श्वम् । रसः॑ । ए॒षः । प॒शू॒नाम् । यत् । वसा᳚ । यत् । पा॒र्श्वेन॑ । व॒सा॒हो॒ममिति॑ वसा - हो॒मम् । प्र॒यौतीति॑ प्र - यौति॑ । म॒द्ध्य॒तः । ए॒व । प॒शू॒नाम् । रस᳚म् । द॒धा॒ति॒ । घ्नन्ति॑ ।  \newline




\markright{ TS 6.3.11.2  \hfill https://www.vedavms.in \hfill}

\section{ TS 6.3.11.2 }

\textbf{TS 6.3.11.2 } \newline
\textbf{Samhita Paata} \newline

वा ए॒तत् प॒शुं ॅयथ् स᳚ज्ञ्ं॒पय॑न्त्यै॒न्द्रः खलु॒ वै दे॒वत॑या प्रा॒ण ऐ॒न्द्रो॑ऽपा॒न ऐ॒न्द्रः प्रा॒णो अङ्गे॑अङ्गे॒ नि दे᳚द्ध्य॒दित्या॑ह प्राणापा॒नावे॒व प॒शुषु॑ दधाति॒ देव॑ त्वष्ट॒र्भूरि॑ ते॒ सꣳ स॑मे॒त्वित्या॑ह त्वा॒ष्ट्रा हि दे॒वत॑या प॒शवो॒ विषु॑रूपा॒ यथ् सल॑क्ष्माणो॒ भव॒थेत्या॑ह॒ विषु॑रूपा॒ ह्ये॑ते सन्तः॒ सल॑क्ष्माण ए॒तर्.हि॒ भव॑न्ति देव॒त्रा यन्त॒- [  ] \newline

\textbf{Pada Paata} \newline

वै । ए॒तत् । प॒शुम् । यत् । स॒ज्ञ्ं॒पय॒न्तीति॑ सं - ज्ञ्॒पय॑न्ति । ऐ॒न्द्रः । खलु॑ । वै । दे॒वत॑या । प्रा॒ण इति॑ प्र - अ॒नः । ऐ॒न्द्रः । अ॒पा॒न इत्य॑प -अ॒नः । ऐ॒न्द्रः । प्रा॒ण इति॑ प्र - अ॒नः । अङ्गे॑अङ्ग॒ इत्यङ्गे᳚ - अ॒ङ्गे॒ । नीति॑ । दे॒द्ध्य॒त् । इति॑ । आ॒ह॒ । प्रा॒णा॒पा॒नाविति॑ प्राण - अ॒पा॒नौ । ए॒व । प॒शुषु॑ । द॒धा॒ति॒ । देव॑ । त्व॒ष्टः॒ । भूरि॑ । ते॒ । सꣳस॒मिति॒ सं-स॒म् । ए॒तु॒ । इति॑ । आ॒ह॒ । त्वा॒ष्ट्राः । हि । दे॒वत॑या । प॒शवः॑ । विषु॑रूपा॒ इति॒ विषु॑ - रू॒पाः॒ । यत् । सल॑क्ष्माण॒ इति॒ स-ल॒क्ष्मा॒णः॒ । भव॑थ । इति॑ । आ॒ह॒ । विषु॑रूपा॒ इति॒ विषु॑-रू॒पाः॒ । हि । ए॒ते । सन्तः॑ । सल॑क्ष्माण॒ इति॒ स - ल॒क्ष्मा॒णः॒ । ए॒तर्.हि॑ । भव॑न्ति । दे॒व॒त्रेति॑ देव - त्रा । यन्त᳚म् ।  \newline




\markright{ TS 6.3.11.3  \hfill https://www.vedavms.in \hfill}

\section{ TS 6.3.11.3 }

\textbf{TS 6.3.11.3 } \newline
\textbf{Samhita Paata} \newline

-मव॑से॒ सखा॒योऽनु॑ त्वा मा॒ता पि॒तरो॑ मद॒न्त्वित्या॒हा-नु॑मतमे॒वैनं॑ मा॒त्रा पि॒त्रा सु॑व॒र्गं ॅलो॒कं ग॑मयत्यर्द्ध॒र्चे व॑साहो॒मं जु॑होत्य॒सौ वा अ॑र्द्ध॒र्च इ॒यम॑र्द्ध॒च इ॒मे ए॒व रसे॑नानक्ति॒ दिशो॑ जुहोति॒ दिश॑ ए॒व रसे॑नान॒क्त्यथो॑ दि॒ग्भ्य ए॒वोर्जꣳ॒॒ रस॒मव॑ रुन्धे प्राणापा॒नौ वा ए॒तौ प॑शू॒नां ॅयत् पृ॑षदा॒ज्यं ॅवा॑नस्प॒त्याः खलु॒- [  ] \newline

\textbf{Pada Paata} \newline

अव॑से । सखा॑यः । अन्विति॑ । त्वा॒ । मा॒ता । पि॒तरः॑ । म॒द॒न्तु॒ । इति॑ । आ॒ह॒ । अनु॑मत॒मित्यनु॑ - म॒त॒म् । ए॒व । ए॒न॒म् । मा॒त्रा । पि॒त्रा । सु॒व॒र्गमिति॑ सुवः - गम् । लो॒कम् । ग॒म॒य॒ति॒ । अ॒द्‌र्ध॒र्च इत्य॑द्‌र्ध-ऋ॒चे । व॒सा॒हो॒ममिति॑ वसा - हो॒मम् । जु॒हो॒ति॒ । अ॒सौ । वै । अ॒द्‌र्ध॒र्च इत्य॑द्‌र्ध-ऋ॒चः । इ॒यम् । अ॒द्‌र्ध॒र्च इत्य॑द्‌र्ध - ऋ॒चः । इ॒मे इति॑ । ए॒व । रसे॑न । अ॒न॒क्ति॒ । दिशः॑ । जु॒हो॒ति॒ । दिशः॑ । ए॒व । रसे॑न । अ॒न॒क्ति॒ । अथो॒ इति॑ । दि॒ग्भ्य इति॑ दिक् - भ्यः । ए॒व । ऊर्ज᳚म् । रस᳚म् । अवेति॑ । रु॒न्धे॒ । प्रा॒णा॒पा॒नाविति॑ प्राण - अ॒पा॒नौ । वै । ए॒तौ । प॒शू॒नाम् । यत् । पृ॒ष॒दा॒ज्यमिति॑ पृषत् - आ॒ज्यम् । वा॒न॒स्प॒त्याः । खलु॑ ।  \newline




\markright{ TS 6.3.11.4  \hfill https://www.vedavms.in \hfill}

\section{ TS 6.3.11.4 }

\textbf{TS 6.3.11.4 } \newline
\textbf{Samhita Paata} \newline

वै दे॒वत॑या प॒शवो॒ यत् पृ॑षदा॒ज्यस्यो॑-प॒हत्याऽऽ*ह॒ वन॒स्पत॒येऽनु॑ ब्रूहि॒ वन॒स्पत॑ये॒ प्रेष्येति॑ प्राणापा॒नावे॒व प॒शुषु॑ दधात्य॒न्यस्या᳚न्यस्य समव॒त्तꣳ स॒मव॑द्यति॒ तस्मा॒न्नाना॑रूपाः प॒शवो॑ यू॒ष्णोप॑ सिञ्चति॒ रसो॒ वा ए॒ष प॑शू॒नां ॅयद्यू रस॑मे॒व प॒शुषु॑ दधा॒तीडा॒मुप॑ ह्वयते प॒शवो॒ वा इडा॑ प॒शूने॒वोप॑ ह्वयते च॒तुरुप॑ ह्वयते॒- [  ] \newline

\textbf{Pada Paata} \newline

वै । दे॒वत॑या । प॒शवः॑ । यत् । पृ॒ष॒दा॒ज्यस्येति॑ पृषत् - आ॒ज्यस्य॑ । उ॒प॒हत्येत्यु॑प - हत्य॑ । आह॑ । वन॒स्पत॑ये । अन्विति॑ । ब्रू॒हि॒ । वन॒स्पत॑ये । प्रेति॑ । इ॒ष्य॒ । इति॑ । प्रा॒णा॒पा॒नाविति॑ प्राण - अ॒पा॒नौ । ए॒व । प॒शुषु॑ । द॒धा॒ति॒ । अ॒न्यस्या᳚न्य॒स्येत्य॒न्यस्य॑ - अ॒न्य॒स्य॒ । स॒म॒व॒त्तमिति॑ सं - अ॒व॒त्तम् । स॒मव॑द्य॒तीति॑ सं-अव॑द्यति । तस्मा᳚त् । नाना॑रूपा॒ इति॒ नाना᳚-रू॒पाः । प॒शवः॑ । यू॒ष्णा । उपेति॑ । सि॒ञ्च॒ति॒ । रसः॑ । वै । ए॒षः । प॒शू॒नाम् । यत् । यूः । रस᳚म् । ए॒व । प॒शुषु॑ । द॒धा॒ति॒ । इडा᳚म् । उपेति॑ । ह्व॒य॒ते॒ । प॒शवः॑ । वै । इडा᳚ । प॒शून् । ए॒व । उपेति॑ । ह्व॒य॒ते॒ । च॒तुः । उपेति॑ । ह्व॒य॒ते॒ ।  \newline




\markright{ TS 6.3.11.5  \hfill https://www.vedavms.in \hfill}

\section{ TS 6.3.11.5 }

\textbf{TS 6.3.11.5 } \newline
\textbf{Samhita Paata} \newline

चतु॑ष्पादो॒ हि प॒शवो॒ यं का॒मये॑ता प॒शुः स्या॒दित्य॑मे॒दस्कं॒ तस्मा॒ आ द॑द्ध्या॒न्मेदो॑रूपा॒ वै प॒शवो॑ रू॒पेणै॒वैनं॑ प॒शुभ्यो॒ निर्भ॑जत्यप॒शुरे॒व भ॑वति॒ यं का॒मये॑त पशु॒मान्थ् स्या॒दिति॒ मेद॑स्व॒त् तस्मा॒ आ द॑द्ध्या॒न्मेदो॑रूपा॒ वै प॒शवो॑ रू॒पेणै॒वास्मै॑ प॒शूनव॑ रुन्धे पशु॒माने॒व भ॑वति प्र॒जाप॑तिर्य॒ज्ञ्म॑सृजत॒ स आज्यं॑- [  ] \newline

\textbf{Pada Paata} \newline

चतु॑ष्पाद॒ इति॒ चतुः॑ - पा॒दः॒ । हि । प॒शवः॑ । यम् । का॒मये॑त । अ॒प॒शुः । स्या॒त् । इति॑ । अ॒मे॒दस्क॒मित्य॑मे॒दः-क॒म् । तस्मै᳚ । एति॑ । द॒द्ध्या॒त् । मेदो॑रूपा॒ इति॒ मेदः॑-रू॒पाः॒ । वै । प॒शवः॑ । रू॒पेण॑ । ए॒व । ए॒न॒म् । प॒शुभ्य॒ इति॑ प॒शु-भ्यः॒ । निरिति॑ । भ॒ज॒ति॒ । अ॒प॒शुः । ए॒व । भ॒व॒ति॒ । यम् । का॒मये॑त । प॒शु॒मानिति॑ पशु - मान् । स्या॒त् । इति॑ । मेद॑स्वत् । तस्मै᳚ । एति॑ । द॒द्ध्या॒त् । मेदो॑रूपा॒ इति॒ मेदः॑ - रू॒पाः॒ । वै । प॒शवः॑ । रू॒पेण॑ । ए॒व । अ॒स्मै॒ । प॒शून् । अवेति॑ । रु॒न्धे॒ । प॒शु॒मानिति॑ पशु - मान् । ए॒व । भ॒व॒ति॒ । प्र॒जाप॑ति॒रिति॑ प्र॒जा-प॒तिः॒ । य॒ज्ञ्म् । अ॒सृ॒ज॒त॒ । सः । आज्यं᳚ ।  \newline




\markright{ TS 6.3.11.6  \hfill https://www.vedavms.in \hfill}

\section{ TS 6.3.11.6 }

\textbf{TS 6.3.11.6 } \newline
\textbf{Samhita Paata} \newline

पु॒रस्ता॑दसृजत प॒शुं म॑द्ध्य॒तः पृ॑षदा॒ज्यं प॒श्चात् तस्मा॒दाज्ये॑न प्रया॒जा इ॑ज्यन्ते प॒शुना॑ मद्ध्य॒तः पृ॑षदा॒ज्येना॑-नूया॒जा-स्तस्मा॑दे॒तन्मि॒श्रमि॑व पश्चाथ् सृ॒ष्टꣳ ह्येका॑दशानूया॒जान्. य॑जति॒ दश॒ वै प॒शोः प्रा॒णा आ॒त्मैका॑द॒शो यावा॑ने॒व प॒शुस्तमनु॑ यजति॒ घ्नन्ति॒ वा ए॒तत् प॒शुं ॅयथ् स᳚ज्ञ्ं॒पय॑न्ति प्रा॒णापा॒नौ खलु॒ वा ए॒तौ प॑शू॒नां ॅयत् पृ॑षदा॒ज्यं ॅयत् पृ॑षदा॒ज्येना॑ ( ) नूया॒जान्. यज॑ति प्राणापा॒नावे॒व प॒शुषु॑ दधाति ॥ \newline

\textbf{Pada Paata} \newline

पु॒रस्ता᳚त् । अ॒सृ॒ज॒त॒ । प॒शुम् । म॒द्ध्य॒तः । पृ॒ष॒दा॒ज्यमिति॑ पृषत् - आ॒ज्यम् । प॒श्चात् । तस्मा᳚त् । आज्ये॑न । प्र॒या॒जा इति॑ प्र - या॒जाः । इ॒ज्य॒न्ते॒ । प॒शुना᳚ । म॒द्ध्य॒तः । पृ॒ष॒दा॒ज्येनेति॑ पृषत् - आ॒ज्येन॑ । अ॒नू॒या॒जा इत्य॑नु - या॒जाः । तस्मा᳚त् । ए॒तत् । मि॒श्रम् । इ॒व॒ । प॒श्चा॒थ्सृ॒ष्टमिति॑ पश्चात् - सृ॒ष्टम् । हि । एका॑दश । अ॒नू॒या॒जानित्य॑नु - या॒जान् । य॒ज॒ति॒ । दश॑ । वै । प॒शोः । प्रा॒णा इति॑ प्र - अ॒नाः । आ॒त्मा । ए॒का॒द॒शः । यावान्॑ । ए॒व । प॒शुः । तम् । अन्विति॑ । य॒ज॒ति॒ । घ्नन्ति॑ । वै । ए॒तत् । प॒शुम् । यत् । स॒ज्ञ्ं॒पय॒न्तीति॑ सं - ज्ञ्॒पय॑न्ति । प्रा॒णा॒पा॒नाविति॑ प्राण - अ॒पा॒नौ । खलु॑ । वै । ए॒तौ । प॒शू॒नाम् । यत् । पृ॒ष॒दा॒ज्यमिति॑ पृषत् - आ॒ज्यम् । यत् । पृ॒ष॒दा॒ज्येनेति॑ पृषत् - आ॒ज्येन॑ ( ) । अ॒नू॒या॒जानित्य॑नु - या॒जान् । यज॑ति । प्रा॒णा॒पा॒नाविति॑ प्राण - अ॒पा॒नौ । ए॒व । प॒शुषु॑ । द॒धा॒ति॒ ॥  \newline






\end{document}