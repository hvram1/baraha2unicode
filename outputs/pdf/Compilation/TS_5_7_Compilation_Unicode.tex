\documentclass[17pt]{extarticle}
\usepackage{babel}
\usepackage{fontspec}
\usepackage{polyglossia}
\usepackage{extsizes}

\usepackage{color}   %May be necessary if you want to color links
\usepackage{hyperref}
\hypersetup{
    colorlinks=true, %set true if you want colored links
    linktoc=all,     %set to all if you want both sections and subsections linked
    linkcolor=black,  %choose some color if you want links to stand out
}

\setmainlanguage{sanskrit}
\setotherlanguages{english} %% or other languages
\setlength{\parindent}{0pt}
\pagestyle{myheadings}
\newfontfamily\devanagarifont[Script=Devanagari]{AdishilaVedic}
\renewcommand{\theHsection}{\thepart.section.\thesection}

\newcommand{\VAR}[1]{}
\newcommand{\BLOCK}[1]{}




\begin{document}
\begin{titlepage}
    \begin{center}
 
\begin{sanskrit}
    { \Large
    कृष्ण यजुर्वेदीय तैत्तिरीय संहिता,पद,जटा,घन पाठः 
    }
    \\
    \vspace{2.5cm}
    \mbox{ \Large
    5.7      पञ्चमकाण्डे सप्तमः प्रश्नः-उपानुवाक्यावशिष्टकर्मनिरूपणं   }
\end{sanskrit}
\end{center}

\end{titlepage}
\tableofcontents
\phantomsection
\pagebreak

\markright{ TS 5.7.1.1  \hfill https://www.vedavms.in \hfill}

\section{ TS 5.7.1.1 }

\textbf{TS 5.7.1.1 } \newline
\textbf{Samhita Paata} \newline

यो वा अय॑थादेवतम॒ग्निं चि॑नु॒त आ दे॒वता᳚भ्यो वृश्च्यते॒ पापी॑यान् भवति॒ यो य॑थादेव॒तं न दे॒वता᳚भ्य॒ आ वृ॑श्च्यते॒ वसी॑यान् भवत्याग्ने॒य्या गा॑यत्रि॒या प्र॑थ॒मां चिति॑म॒भि मृ॑शेत् त्रि॒ष्टुभा᳚ द्वि॒तीयां॒ जग॑त्या तृ॒तीया॑मनु॒ष्टुभा॑ चतु॒र्थीं प॒ङ्क्त्या प॑ञ्च॒मीं ॅय॑थादेव॒तमे॒वाग्निं चि॑नुते॒ न दे॒वता᳚भ्य॒ आ वृ॑श्च्यते॒ वसी॑यान् भव॒तीडा॑यै॒ वा ए॒षा विभ॑क्तिः प॒शव॒ इडा॑ प॒शुभि॑रेनं - [  ] \newline

\textbf{Pada Paata} \newline

यः । वै । अय॑थादेवत॒मित्यय॑था-दे॒व॒त॒म् । अ॒ग्निम् । चि॒नु॒ते । एति॑ । दे॒वता᳚भ्यः । वृ॒श्च्य॒ते॒ । पापी॑यान् । भ॒व॒ति॒ । यः । य॒था॒दे॒व॒तमिति॑ यथा - दे॒व॒तम् । न । दे॒वता᳚भ्यः । एति॑ । वृ॒श्च्य॒ते॒ । वसी॑यान् । भ॒व॒ति॒ । आ॒ग्ने॒य्या । गा॒य॒त्रि॒या । प्र॒थ॒माम् । चिति᳚म् । अ॒भीति॑ । मृ॒शे॒त् । त्रि॒ष्टुभाः᳚ । द्वि॒तीया᳚म् । जग॑त्या । तृ॒तीया᳚म् । अ॒नु॒ष्टुभेत्य॑नु - स्तुभा᳚ । च॒तु॒र्थीम् । प॒ङ्क्त्या । प॒ञ्च॒मीम् । य॒था॒दे॒व॒तमिति॑ यथा - दे॒व॒तम् । ए॒व । अ॒ग्निम् । चि॒नु॒ते॒ । न । दे॒वता᳚भ्यः । एति॑ । वृ॒श्च्य॒ते॒ । वसी॑यान् । भ॒व॒ति॒ । इडा॑यै । वै । ए॒षा । विभ॑क्ति॒रिति॒ वि - भ॒क्तिः॒ । प॒शवः॑ । इडा᳚ । प॒शुभि॒रिति॑ प॒शु-भिः॒ । ए॒न॒म् ।  \newline


\textbf{Krama Paata} \newline

यो वै । वा अय॑थादेवतम् । अय॑थादेवतम॒ग्निम् । अय॑थादेवत॒मित्यय॑था - दे॒व॒त॒म् । अ॒ग्निम् चि॑नु॒ते । चि॒नु॒त आ । आ दे॒वता᳚भ्यः । दे॒वता᳚भ्यो वृश्च्यते । वृ॒श्च्य॒ते॒ पापी॑यान् । पापी॑यान् भवति । भ॒व॒ति॒ यः । यो य॑थादेव॒तम् । य॒था॒दे॒व॒तम् न । य॒था॒दे॒व॒तमिति॑ यथा - दे॒व॒तम् । न दे॒वता᳚भ्यः । दे॒वता᳚भ्य॒ आ । आ वृ॑श्च्यते । वृ॒श्च्य॒ते॒ वसी॑यान् । वसी॑यान् भवति । भ॒व॒त्या॒ग्ने॒य्या । आ॒ग्ने॒य्या गा॑यत्रि॒या । गा॒य॒त्रि॒या प्र॑थ॒माम् । प्र॒थ॒माम् चिति᳚म् । चिति॑म॒भि । अ॒भि मृ॑शेत् । मृ॒शे॒त् त्रि॒ष्टुभा᳚ । त्रि॒ष्टुभा᳚ द्वि॒तीया᳚म् । द्वि॒तीया॒म् जग॑त्या । जग॑त्या तृ॒तीया᳚म् । तृ॒तीया॑मनु॒ष्टुभा᳚ । अ॒नु॒ष्टुभा॑ चतु॒र्त्थीम् । अ॒नु॒ष्टुभेत्य॑नु - स्तुभा᳚ । च॒तु॒र्त्थीम् प॒ङ्क्त्या । प॒ङ्क्त्या प॑ञ्च॒मीम् । प॒ञ्च॒मीम् ॅय॑थादेव॒तम् । य॒था॒दे॒व॒तमे॒व । य॒था॒दे॒व॒तमिति॑ यथा - दे॒व॒तम् । ए॒वाग्निम् । अ॒ग्निम् चि॑नुते । चि॒नु॒ते॒ न । न दे॒वता᳚भ्यः । दे॒वता᳚भ्य॒ आ । आ वृ॑श्च्यते । वृ॒श्च्य॒ते॒ वसी॑यान् । वसी॑यान् भवति । भ॒व॒तीडा॑यै । इडा॑यै॒ वै । वा ए॒षा । ए॒षा विभ॑क्तिः । विभ॑क्तिः प॒शवः॑ । विभ॑क्ति॒रिति॒ वि - भ॒क्तिः॒ । प॒शव॒ इडा᳚ । इडा॑ प॒शुभिः॑ । प॒शुभि॑रेनम् । प॒शुभि॒रिति॑ प॒शु - भिः॒ । ए॒न॒म् चि॒नु॒ते॒ \newline

\textbf{Jatai Paata} \newline

1. यो वै वै यो यो वै । \newline
2. वा अय॑थादेवत॒ मय॑थादेवतं॒ ॅवै वा अय॑थादेवतम् । \newline
3. अय॑थादेवत म॒ग्नि म॒ग्नि मय॑थादेवत॒ मय॑थादेवत म॒ग्निम् । \newline
4. अय॑थादेवत॒मित्यय॑था - दे॒व॒त॒म् । \newline
5. अ॒ग्निम् चि॑नु॒ते चि॑नु॒ते᳚ ऽग्नि म॒ग्निम् चि॑नु॒ते । \newline
6. चि॒नु॒त आ चि॑नु॒ते चि॑नु॒त आ । \newline
7. आ दे॒वता᳚भ्यो दे॒वता᳚भ्य॒ आ दे॒वता᳚भ्यः । \newline
8. दे॒वता᳚भ्यो वृश्च्यते वृश्च्यते दे॒वता᳚भ्यो दे॒वता᳚भ्यो वृश्च्यते । \newline
9. वृ॒श्च्य॒ते॒ पापी॑या॒न् पापी॑यान् वृश्च्यते वृश्च्यते॒ पापी॑यान् । \newline
10. पापी॑यान् भवति भवति॒ पापी॑या॒न् पापी॑यान् भवति । \newline
11. भ॒व॒ति॒ यो यो भ॑वति भवति॒ यः । \newline
12. यो य॑थादेव॒तं ॅय॑थादेव॒तं ॅयो यो य॑थादेव॒तम् । \newline
13. य॒था॒दे॒व॒तम् न न य॑थादेव॒तं ॅय॑थादेव॒तम् न । \newline
14. य॒था॒दे॒व॒तमिति॑ यथा - दे॒व॒तम् । \newline
15. न दे॒वता᳚भ्यो दे॒वता᳚भ्यो॒ न न दे॒वता᳚भ्यः । \newline
16. दे॒वता᳚भ्य॒ आ दे॒वता᳚भ्यो दे॒वता᳚भ्य॒ आ । \newline
17. आ वृ॑श्च्यते वृश्च्यत॒ आ वृ॑श्च्यते । \newline
18. वृ॒श्च्य॒ते॒ वसी॑या॒न्॒. वसी॑यान् वृश्च्यते वृश्च्यते॒ वसी॑यान् । \newline
19. वसी॑यान् भवति भवति॒ वसी॑या॒न्॒. वसी॑यान् भवति । \newline
20. भ॒व॒ त्या॒ग्ने॒य्या ऽऽग्ने॒य्या भ॑वति भव त्याग्ने॒य्या । \newline
21. आ॒ग्ने॒य्या गा॑यत्रि॒या गा॑यत्रि॒या ऽऽग्ने॒य्या ऽऽग्ने॒य्या गा॑यत्रि॒या । \newline
22. गा॒य॒त्रि॒या प्र॑थ॒माम् प्र॑थ॒माम् गा॑यत्रि॒या गा॑यत्रि॒या प्र॑थ॒माम् । \newline
23. प्र॒थ॒माम् चिति॒म् चिति॑म् प्रथ॒माम् प्र॑थ॒माम् चिति᳚म् । \newline
24. चिति॑ म॒भ्य॑भि चिति॒म् चिति॑ म॒भि । \newline
25. अ॒भि मृ॑शेन् मृशे द॒भ्य॑भि मृ॑शेत् । \newline
26. मृ॒शे॒त् त्रि॒ष्टुभा᳚ स्त्रि॒ष्टुभा॑ मृशेन् मृशेत् त्रि॒ष्टुभाः᳚ । \newline
27. त्रि॒ष्टुभा᳚ द्वि॒तीया᳚म् द्वि॒तीया᳚म् त्रि॒ष्टुभा᳚ स्त्रि॒ष्टुभा᳚ द्वि॒तीया᳚म् । \newline
28. द्वि॒तीया॒म् जग॑त्या॒ जग॑त्या द्वि॒तीया᳚म् द्वि॒तीया॒म् जग॑त्या । \newline
29. जग॑त्या तृ॒तीया᳚म् तृ॒तीया॒म् जग॑त्या॒ जग॑त्या तृ॒तीया᳚म् । \newline
30. तृ॒तीया॑ मनु॒ष्टुभा॑ ऽनु॒ष्टुभा॑ तृ॒तीया᳚म् तृ॒तीया॑ मनु॒ष्टुभा᳚ । \newline
31. अ॒नु॒ष्टुभा॑ चतु॒र्थीम् च॑तु॒र्थी म॑नु॒ष्टुभा॑ ऽनु॒ष्टुभा॑ चतु॒र्थीम् । \newline
32. अ॒नु॒ष्टुभेत्य॑नु - स्तुभा᳚ । \newline
33. च॒तु॒र्थीम् प॒ङ्क्त्या प॒ङ्क्त्या च॑तु॒र्थीम् च॑तु॒र्थीम् प॒ङ्क्त्या । \newline
34. प॒ङ्क्त्या प॑ञ्च॒मीम् प॑ञ्च॒मीम् प॒ङ्क्त्या प॒ङ्क्त्या प॑ञ्च॒मीम् । \newline
35. प॒ञ्च॒मीं ॅय॑थादेव॒तं ॅय॑थादेव॒तम् प॑ञ्च॒मीम् प॑ञ्च॒मीं ॅय॑थादेव॒तम् । \newline
36. य॒था॒दे॒व॒त मे॒वैव य॑थादेव॒तं ॅय॑थादेव॒त मे॒व । \newline
37. य॒था॒दे॒व॒तमिति॑ यथा - दे॒व॒तम् । \newline
38. ए॒वाग्नि म॒ग्नि मे॒वै वाग्निम् । \newline
39. अ॒ग्निम् चि॑नुते चिनुते॒ ऽग्नि म॒ग्निम् चि॑नुते । \newline
40. चि॒नु॒ते॒ न न चि॑नुते चिनुते॒ न । \newline
41. न दे॒वता᳚भ्यो दे॒वता᳚भ्यो॒ न न दे॒वता᳚भ्यः । \newline
42. दे॒वता᳚भ्य॒ आ दे॒वता᳚भ्यो दे॒वता᳚भ्य॒ आ । \newline
43. आ वृ॑श्च्यते वृश्च्यत॒ आ वृ॑श्च्यते । \newline
44. वृ॒श्च्य॒ते॒ वसी॑या॒न्॒. वसी॑यान् वृश्च्यते वृश्च्यते॒ वसी॑यान् । \newline
45. वसी॑यान् भवति भवति॒ वसी॑या॒न्॒. वसी॑यान् भवति । \newline
46. भ॒व॒ तीडा॑या॒ इडा॑यै भवति भव॒ तीडा॑यै । \newline
47. इडा॑यै॒ वै वा इडा॑या॒ इडा॑यै॒ वै । \newline
48. वा ए॒षैषा वै वा ए॒षा । \newline
49. ए॒षा विभ॑क्ति॒र् विभ॑क्ति रे॒षैषा विभ॑क्तिः । \newline
50. विभ॑क्तिः प॒शवः॑ प॒शवो॒ विभ॑क्ति॒र् विभ॑क्तिः प॒शवः॑ । \newline
51. विभ॑क्ति॒रिति॒ वि - भ॒क्तिः॒ । \newline
52. प॒शव॒ इडेडा॑ प॒शवः॑ प॒शव॒ इडा᳚ । \newline
53. इडा॑ प॒शुभिः॑ प॒शुभि॒ रिडेडा॑ प॒शुभिः॑ । \newline
54. प॒शुभि॑ रेन मेनम् प॒शुभिः॑ प॒शुभि॑ रेनम् । \newline
55. प॒शुभि॒रिति॑ प॒शु - भिः॒ । \newline
56. ए॒न॒म् चि॒नु॒ते॒ चि॒नु॒त॒ ए॒न॒ मे॒न॒म् चि॒नु॒ते॒ । \newline

\textbf{Ghana Paata } \newline

1. यो वै वै यो यो वा अय॑थादेवत॒ मय॑थादेवतं॒ ॅवै यो यो वा अय॑थादेवतम् । \newline
2. वा अय॑थादेवत॒ मय॑थादेवतं॒ ॅवै वा अय॑थादेवत म॒ग्नि म॒ग्नि मय॑थादेवतं॒ ॅवै वा अय॑थादेवत म॒ग्निम् । \newline
3. अय॑थादेवत म॒ग्नि म॒ग्नि मय॑थादेवत॒ मय॑थादेवत म॒ग्निम् चि॑नु॒ते चि॑नु॒ते᳚ ऽग्नि मय॑थादेवत॒ मय॑थादेवत म॒ग्निम् चि॑नु॒ते । \newline
4. अय॑थादेवत॒मित्यय॑था - दे॒व॒त॒म् । \newline
5. अ॒ग्निम् चि॑नु॒ते चि॑नु॒ते᳚ ऽग्नि म॒ग्निम् चि॑नु॒त आ चि॑नु॒ते᳚ ऽग्नि म॒ग्निम् चि॑नु॒त आ । \newline
6. चि॒नु॒त आ चि॑नु॒ते चि॑नु॒त आ दे॒वता᳚भ्यो दे॒वता᳚भ्य॒ आ चि॑नु॒ते चि॑नु॒त आ दे॒वता᳚भ्यः । \newline
7. आ दे॒वता᳚भ्यो दे॒वता᳚भ्य॒ आ दे॒वता᳚भ्यो वृश्च्यते वृश्च्यते दे॒वता᳚भ्य॒ आ दे॒वता᳚भ्यो वृश्च्यते । \newline
8. दे॒वता᳚भ्यो वृश्च्यते वृश्च्यते दे॒वता᳚भ्यो दे॒वता᳚भ्यो वृश्च्यते॒ पापी॑या॒न् पापी॑यान् वृश्च्यते दे॒वता᳚भ्यो दे॒वता᳚भ्यो वृश्च्यते॒ पापी॑यान् । \newline
9. वृ॒श्च्य॒ते॒ पापी॑या॒न् पापी॑यान् वृश्च्यते वृश्च्यते॒ पापी॑यान् भवति भवति॒ पापी॑यान् वृश्च्यते वृश्च्यते॒ पापी॑यान् भवति । \newline
10. पापी॑यान् भवति भवति॒ पापी॑या॒न् पापी॑यान् भवति॒ यो यो भ॑वति॒ पापी॑या॒न् पापी॑यान् भवति॒ यः । \newline
11. भ॒व॒ति॒ यो यो भ॑वति भवति॒ यो य॑थादेव॒तं ॅय॑थादेव॒तं ॅयो भ॑वति भवति॒ यो य॑थादेव॒तम् । \newline
12. यो य॑थादेव॒तं ॅय॑थादेव॒तं ॅयो यो य॑थादेव॒तम् न न य॑थादेव॒तं ॅयो यो य॑थादेव॒तम् न । \newline
13. य॒था॒दे॒व॒तम् न न य॑थादेव॒तं ॅय॑थादेव॒तम् न दे॒वता᳚भ्यो दे॒वता᳚भ्यो॒ न य॑थादेव॒तं ॅय॑थादेव॒तम् न दे॒वता᳚भ्यः । \newline
14. य॒था॒दे॒व॒तमिति॑ यथा - दे॒व॒तम् । \newline
15. न दे॒वता᳚भ्यो दे॒वता᳚भ्यो॒ न न दे॒वता᳚भ्य॒ आ दे॒वता᳚भ्यो॒ न न दे॒वता᳚भ्य॒ आ । \newline
16. दे॒वता᳚भ्य॒ आ दे॒वता᳚भ्यो दे॒वता᳚भ्य॒ आ वृ॑श्च्यते वृश्च्यत॒ आ दे॒वता᳚भ्यो दे॒वता᳚भ्य॒ आ वृ॑श्च्यते । \newline
17. आ वृ॑श्च्यते वृश्च्यत॒ आ वृ॑श्च्यते॒ वसी॑या॒न्॒. वसी॑यान् वृश्च्यत॒ आ वृ॑श्च्यते॒ वसी॑यान् । \newline
18. वृ॒श्च्य॒ते॒ वसी॑या॒न्॒. वसी॑यान् वृश्च्यते वृश्च्यते॒ वसी॑यान् भवति भवति॒ वसी॑यान् वृश्च्यते वृश्च्यते॒ वसी॑यान् भवति । \newline
19. वसी॑यान् भवति भवति॒ वसी॑या॒न्॒. वसी॑यान् भव त्याग्ने॒य्या ऽऽग्ने॒य्या भ॑वति॒ वसी॑या॒न्॒. वसी॑यान् भव त्याग्ने॒य्या । \newline
20. भ॒व॒ त्या॒ग्ने॒य्या ऽऽग्ने॒य्या भ॑वति भव त्याग्ने॒य्या गा॑यत्रि॒या गा॑यत्रि॒या ऽऽग्ने॒य्या भ॑वति भव त्याग्ने॒य्या गा॑यत्रि॒या । \newline
21. आ॒ग्ने॒य्या गा॑यत्रि॒या गा॑यत्रि॒या ऽऽग्ने॒य्या ऽऽग्ने॒य्या गा॑यत्रि॒या प्र॑थ॒माम् प्र॑थ॒माम् गा॑यत्रि॒या ऽऽग्ने॒य्या ऽऽग्ने॒य्या गा॑यत्रि॒या प्र॑थ॒माम् । \newline
22. गा॒य॒त्रि॒या प्र॑थ॒माम् प्र॑थ॒माम् गा॑यत्रि॒या गा॑यत्रि॒या प्र॑थ॒माम् चिति॒म् चिति॑म् प्रथ॒माम् गा॑यत्रि॒या गा॑यत्रि॒या प्र॑थ॒माम् चिति᳚म् । \newline
23. प्र॒थ॒माम् चिति॒म् चिति॑म् प्रथ॒माम् प्र॑थ॒माम् चिति॑ म॒भ्य॑भि चिति॑म् प्रथ॒माम् प्र॑थ॒माम् चिति॑ म॒भि । \newline
24. चिति॑ म॒भ्य॑भि चिति॒म् चिति॑ म॒भि मृ॑शेन् मृशे द॒भि चिति॒म् चिति॑ म॒भि मृ॑शेत् । \newline
25. अ॒भि मृ॑शेन् मृशे द॒भ्य॑भि मृ॑शेत् त्रि॒ष्टुभा᳚ स्त्रि॒ष्टुभा॑ मृशे द॒भ्य॑भि मृ॑शेत् त्रि॒ष्टुभाः᳚ । \newline
26. मृ॒शे॒त् त्रि॒ष्टुभा᳚ स्त्रि॒ष्टुभा॑ मृशेन् मृशेत् त्रि॒ष्टुभा᳚ द्वि॒तीया᳚म् द्वि॒तीया᳚म् त्रि॒ष्टुभा॑ मृशेन् मृशेत् त्रि॒ष्टुभा᳚ द्वि॒तीया᳚म् । \newline
27. त्रि॒ष्टुभा᳚ द्वि॒तीया᳚म् द्वि॒तीया᳚म् त्रि॒ष्टुभा᳚ स्त्रि॒ष्टुभा᳚ द्वि॒तीया॒म् जग॑त्या॒ जग॑त्या द्वि॒तीया᳚म् त्रि॒ष्टुभा᳚ स्त्रि॒ष्टुभा᳚ द्वि॒तीया॒म् जग॑त्या । \newline
28. द्वि॒तीया॒म् जग॑त्या॒ जग॑त्या द्वि॒तीया᳚म् द्वि॒तीया॒म् जग॑त्या तृ॒तीया᳚म् तृ॒तीया॒म् जग॑त्या द्वि॒तीया᳚म् द्वि॒तीया॒म् जग॑त्या तृ॒तीया᳚म् । \newline
29. जग॑त्या तृ॒तीया᳚म् तृ॒तीया॒म् जग॑त्या॒ जग॑त्या तृ॒तीया॑ मनु॒ष्टुभा॑ ऽनु॒ष्टुभा॑ तृ॒तीया॒म् जग॑त्या॒ जग॑त्या तृ॒तीया॑ मनु॒ष्टुभा᳚ । \newline
30. तृ॒तीया॑ मनु॒ष्टुभा॑ ऽनु॒ष्टुभा॑ तृ॒तीया᳚म् तृ॒तीया॑ मनु॒ष्टुभा॑ चतु॒र्थीम् च॑तु॒र्थी म॑नु॒ष्टुभा॑ तृ॒तीया᳚म् तृ॒तीया॑ मनु॒ष्टुभा॑ चतु॒र्थीम् । \newline
31. अ॒नु॒ष्टुभा॑ चतु॒र्थीम् च॑तु॒र्थी म॑नु॒ष्टुभा॑ ऽनु॒ष्टुभा॑ चतु॒र्थीम् प॒ङ्क्त्या प॒ङ्क्त्या च॑तु॒र्थी म॑नु॒ष्टुभा॑ ऽनु॒ष्टुभा॑ चतु॒र्थीम् प॒ङ्क्त्या । \newline
32. अ॒नु॒ष्टुभेत्य॑नु - स्तुभा᳚ । \newline
33. च॒तु॒र्थीम् प॒ङ्क्त्या प॒ङ्क्त्या च॑तु॒र्थीम् च॑तु॒र्थीम् प॒ङ्क्त्या प॑ञ्च॒मीम् प॑ञ्च॒मीम् प॒ङ्क्त्या च॑तु॒र्थीम् च॑तु॒र्थीम् प॒ङ्क्त्या प॑ञ्च॒मीम् । \newline
34. प॒ङ्क्त्या प॑ञ्च॒मीम् प॑ञ्च॒मीम् प॒ङ्क्त्या प॒ङ्क्त्या प॑ञ्च॒मीं ॅय॑थादेव॒तं ॅय॑थादेव॒तम् प॑ञ्च॒मीम् प॒ङ्क्त्या प॒ङ्क्त्या प॑ञ्च॒मीं ॅय॑थादेव॒तम् । \newline
35. प॒ञ्च॒मीं ॅय॑थादेव॒तं ॅय॑थादेव॒तम् प॑ञ्च॒मीम् प॑ञ्च॒मीं ॅय॑थादेव॒त मे॒वैव य॑थादेव॒तम् प॑ञ्च॒मीम् प॑ञ्च॒मीं ॅय॑थादेव॒त मे॒व । \newline
36. य॒था॒दे॒व॒त मे॒वैव य॑थादेव॒तं ॅय॑थादेव॒त मे॒वाग्नि म॒ग्नि मे॒व य॑थादेव॒तं ॅय॑थादेव॒त मे॒वाग्निम् । \newline
37. य॒था॒दे॒व॒तमिति॑ यथा - दे॒व॒तम् । \newline
38. ए॒वाग्नि म॒ग्नि मे॒वै वाग्निम् चि॑नुते चिनुते॒ ऽग्नि मे॒वै वाग्निम् चि॑नुते । \newline
39. अ॒ग्निम् चि॑नुते चिनुते॒ ऽग्नि म॒ग्निम् चि॑नुते॒ न न चि॑नुते॒ ऽग्नि म॒ग्निम् चि॑नुते॒ न । \newline
40. चि॒नु॒ते॒ न न चि॑नुते चिनुते॒ न दे॒वता᳚भ्यो दे॒वता᳚भ्यो॒ न चि॑नुते चिनुते॒ न दे॒वता᳚भ्यः । \newline
41. न दे॒वता᳚भ्यो दे॒वता᳚भ्यो॒ न न दे॒वता᳚भ्य॒ आ दे॒वता᳚भ्यो॒ न न दे॒वता᳚भ्य॒ आ । \newline
42. दे॒वता᳚भ्य॒ आ दे॒वता᳚भ्यो दे॒वता᳚भ्य॒ आ वृ॑श्च्यते वृश्च्यत॒ आ दे॒वता᳚भ्यो दे॒वता᳚भ्य॒ आ वृ॑श्च्यते । \newline
43. आ वृ॑श्च्यते वृश्च्यत॒ आ वृ॑श्च्यते॒ वसी॑या॒न्॒. वसी॑यान् वृश्च्यत॒ आ वृ॑श्च्यते॒ वसी॑यान् । \newline
44. वृ॒श्च्य॒ते॒ वसी॑या॒न्॒. वसी॑यान् वृश्च्यते वृश्च्यते॒ वसी॑यान् भवति भवति॒ वसी॑यान् वृश्च्यते वृश्च्यते॒ वसी॑यान् भवति । \newline
45. वसी॑यान् भवति भवति॒ वसी॑या॒न्॒. वसी॑यान् भव॒ती डा॑या॒ इडा॑यै भवति॒ वसी॑या॒न्॒. वसी॑यान् भव॒तीडा॑यै । \newline
46. भ॒व॒ती डा॑या॒ इडा॑यै भवति भव॒तीडा॑यै॒ वै वा इडा॑यै भवति भव॒तीडा॑यै॒ वै । \newline
47. इडा॑यै॒ वै वा इडा॑या॒ इडा॑यै॒ वा ए॒षैषा वा इडा॑या॒ इडा॑यै॒ वा ए॒षा । \newline
48. वा ए॒षैषा वै वा ए॒षा विभ॑क्ति॒र् विभ॑क्ति रे॒षा वै वा ए॒षा विभ॑क्तिः । \newline
49. ए॒षा विभ॑क्ति॒र् विभ॑क्ति रे॒षैषा विभ॑क्तिः प॒शवः॑ प॒शवो॒ विभ॑क्ति रे॒षैषा विभ॑क्तिः प॒शवः॑ । \newline
50. विभ॑क्तिः प॒शवः॑ प॒शवो॒ विभ॑क्ति॒र् विभ॑क्तिः प॒शव॒ इडेडा॑ प॒शवो॒ विभ॑क्ति॒र् विभ॑क्तिः प॒शव॒ इडा᳚ । \newline
51. विभ॑क्ति॒रिति॒ वि - भ॒क्तिः॒ । \newline
52. प॒शव॒ इडेडा॑ प॒शवः॑ प॒शव॒ इडा॑ प॒शुभिः॑ प॒शुभि॒ रिडा॑ प॒शवः॑ प॒शव॒ इडा॑ प॒शुभिः॑ । \newline
53. इडा॑ प॒शुभिः॑ प॒शुभि॒ रिडेडा॑ प॒शुभि॑ रेन मेनम् प॒शुभि॒ रिडेडा॑ प॒शुभि॑ रेनम् । \newline
54. प॒शुभि॑ रेन मेनम् प॒शुभिः॑ प॒शुभि॑ रेनम् चिनुते चिनुत एनम् प॒शुभिः॑ प॒शुभि॑ रेनम् चिनुते । \newline
55. प॒शुभि॒रिति॑ प॒शु - भिः॒ । \newline
56. ए॒न॒म् चि॒नु॒ते॒ चि॒नु॒त॒ ए॒न॒ मे॒न॒म् चि॒नु॒ते॒ यो यश्चि॑नुत एन मेनम् चिनुते॒ यः । \newline
\pagebreak
\markright{ TS 5.7.1.2  \hfill https://www.vedavms.in \hfill}

\section{ TS 5.7.1.2 }

\textbf{TS 5.7.1.2 } \newline
\textbf{Samhita Paata} \newline

-चिनुते॒ यो वै प्र॒जाप॑तये प्रति॒ प्रोच्या॒ग्निं चि॒नोति॒ नाऽऽ*र्ति॒मार्च्छ॒-त्यश्वा॑व॒भित॑स्तिष्ठेतां कृ॒ष्ण उ॑त्तर॒तः श्वे॒तो दक्षि॑ण॒स्तावा॒लभ्येष्ट॑का॒ उप॑ दद्ध्यादे॒तद्वै प्र॒जाप॑ते रू॒पं प्रा॑जाप॒त्योऽश्वः॑ सा॒क्षादे॒व प्र॒जाप॑तये प्रति॒प्रोच्या॒ग्निं चि॑नोति॒ नाऽऽ*र्ति॒मार्च्छ॑त्ये॒तद्वा अह्नो॑ रू॒पं ॅयच्छ्वे॒तोऽश्वो॒ रात्रि॑यै कृ॒ष्ण ए॒तदह्नो॑ - [  ] \newline

\textbf{Pada Paata} \newline

चि॒नु॒ते॒ । यः । वै । प्र॒जाप॑तय॒ इति॑ प्र॒जा -  प॒त॒ये॒ । प्र॒ति॒प्रोच्येति॑ प्रति - प्रोच्य॑ । अ॒ग्निम् । चि॒नोति॑ । न । आर्ति᳚म् । एति॑ । ऋ॒च्छ॒ति॒ । अश्वौ᳚ । अ॒भितः॑ । ति॒ष्ठे॒ता॒म् । कृ॒ष्णः । उ॒त्त॒र॒त इत्यु॑त् -   त॒र॒तः । श्वे॒तः । दक्षि॑णः । तौ । आ॒लभ्येत्या᳚ - लभ्य॑ । इष्ट॑काः । उपेति॑ । द॒द्ध्या॒त् । ए॒तत् । वै । प्र॒जाप॑ते॒रिति॑ प्र॒जा - प॒तेः॒ । रू॒पम् । प्रा॒जा॒प॒त्य इति॑ प्राजा - प॒त्यः । अश्वः॑ । सा॒क्षादिति॑ स - अ॒क्षात् । ए॒व । प्र॒जाप॑तय॒ इति॑ प्र॒जा - प॒त॒ये॒ । प्र॒ति॒प्रोच्येति॑ प्रति - प्रोच्य॑ । अ॒ग्निम् । चि॒नो॒ति॒ । न । आर्ति᳚म् । एति॑ । ऋ॒च्छ॒ति॒ । ए॒तत् । वै । अह्नः॑ । रू॒पम् । यत् । श्वे॒तः । अश्वः॑ । रात्रि॑यै । कृ॒ष्णः । ए॒तत् । अह्नः॑ ।  \newline


\textbf{Krama Paata} \newline

चि॒नु॒ते॒ यः । यो वै । वै प्र॒जाप॑तये । प्र॒जाप॑तये प्रति॒प्रोच्य॑ । प्र॒जाप॑तय॒ इति॑ प्र॒जा - प॒त॒ये॒ । प्र॒ति॒प्रोच्या॒ग्निम् । प्र॒ति॒प्रोच्येति॑ प्रति - प्रोच्य॑ । अ॒ग्निम् चि॒नोति॑ । चि॒नोति॒ न । नार्ति᳚म् । आर्ति॒मा । आर्च्छ॑ति । ऋ॒च्छ॒त्यश्वौ᳚ । अश्वा॑व॒भितः॑ । अ॒भित॑स्तिष्ठेताम् । ति॒ष्ठे॒ता॒म् कृ॒ष्णः । कृ॒ष्ण उ॑त्तर॒तः । उ॒त्त॒र॒तः श्वे॒तः । उ॒त्त॒र॒त इत्यु॑त् - त॒र॒तः । श्वे॒तो दक्षि॑णः । दक्षि॑ण॒स्तौ । तावा॒लभ्य॑ । आ॒लभ्येष्ट॑काः । आ॒लभ्येत्या᳚ - लभ्य॑ । इष्ट॑का॒ उप॑ । उप॑ दद्ध्यात् । द॒द्ध्या॒दे॒तत् । ए॒तद् वै । वै प्र॒जाप॑तेः । प्र॒जाप॑ते रू॒पम् । प्र॒जाप॑ते॒रिति॑ प्र॒जा - प॒तेः॒ । रू॒पम् प्रा॑जाप॒त्यः । प्रा॒जा॒प॒त्योऽश्वः॑ । प्रा॒जा॒प॒त्य इति॑ प्राजा - प॒त्यः । अश्वः॑ सा॒क्षात् । सा॒क्षादे॒व । सा॒क्षादिति॑ स - अ॒क्षात् । ए॒व प्र॒जाप॑तये । प्र॒जाप॑तये प्रति॒प्रोच्य॑ । प्र॒जाप॑तय॒ इति॑ प्र॒जा - प॒त॒ये॒ । प्र॒ति॒प्रोच्या॒ग्निम् । प्र॒ति॒प्रोच्येति॑ प्रति - प्रोच्य॑ । अ॒ग्निम् चि॑नोति । चि॒नो॒ति॒ न । नार्ति᳚म् । आर्ति॒मा । आर्च्छ॑ति । ऋ॒च्छ॒त्ये॒तत् । ए॒तद् वै । वा अह्नः॑ । अह्नो॑ रू॒पम् । रू॒पम् ॅयत् । यच्छ्वे॒तः । श्वे॒तोऽश्वः॑ । अश्वो॒ रात्रि॑यै । रात्रि॑यै कृ॒ष्णः । कृ॒ष्ण ए॒तत् । ए॒तदह्नः॑ ( ) । अह्नो॑ रू॒पम् \newline

\textbf{Jatai Paata} \newline

1. चि॒नु॒ते॒ यो यश्चि॑नुते चिनुते॒ यः । \newline
2. यो वै वै यो यो वै । \newline
3. वै प्र॒जाप॑तये प्र॒जाप॑तये॒ वै वै प्र॒जाप॑तये । \newline
4. प्र॒जाप॑तये प्रति॒प्रोच्य॑ प्रति॒प्रोच्य॑ प्र॒जाप॑तये प्र॒जाप॑तये प्रति॒प्रोच्य॑ । \newline
5. प्र॒जाप॑तय॒ इति॑ प्र॒जा - प॒त॒ये॒ । \newline
6. प्र॒ति॒प्रो च्या॒ग्नि म॒ग्निम् प्र॑ति॒प्रोच्य॑ प्रति॒प्रो च्या॒ग्निम् । \newline
7. प्र॒ति॒प्रोच्येति॑ प्रति - प्रोच्य॑ । \newline
8. अ॒ग्निम् चि॒नोति॑ चि॒नो त्य॒ग्नि म॒ग्निम् चि॒नोति॑ । \newline
9. चि॒नोति॒ न न चि॒नोति॑ चि॒नोति॒ न । \newline
10. नार्ति॒ मार्ति॒म् न नार्ति᳚म् । \newline
11. आर्ति॒ मा ऽऽर्ति॒ मार्ति॒ मा । \newline
12. आर्च्छ॑ त्यृच्छ॒ त्यार्च्छ॑ति । \newline
13. ऋ॒च्छ॒ त्यश्वा॒ वश्वा॑ वृच्छ त्यृच्छ॒ त्यश्वौ᳚ । \newline
14. अश्वा॑ व॒भितो॒ ऽभितो ऽश्वा॒ वश्वा॑ व॒भितः॑ । \newline
15. अ॒भित॑ स्तिष्ठेताम् तिष्ठेता म॒भितो॒ ऽभित॑ स्तिष्ठेताम् । \newline
16. ति॒ष्ठे॒ता॒म् कृ॒ष्णः कृ॒ष्ण स्ति॑ष्ठेताम् तिष्ठेताम् कृ॒ष्णः । \newline
17. कृ॒ष्ण उ॑त्तर॒त उ॑त्तर॒तः कृ॒ष्णः कृ॒ष्ण उ॑त्तर॒तः । \newline
18. उ॒त्त॒र॒तः श्वे॒तः श्वे॒त उ॑त्तर॒त उ॑त्तर॒तः श्वे॒तः । \newline
19. उ॒त्त॒र॒त इत्यु॑त् - त॒र॒तः । \newline
20. श्वे॒तो दक्षि॑णो॒ दक्षि॑णः श्वे॒तः श्वे॒तो दक्षि॑णः । \newline
21. दक्षि॑ण॒ स्तौ तौ दक्षि॑णो॒ दक्षि॑ण॒ स्तौ । \newline
22. ता वा॒लभ्या॒ लभ्य॒ तौ ता वा॒लभ्य॑ । \newline
23. आ॒ल भ्येष्ट॑का॒ इष्ट॑का आ॒लभ्या॒ लभ्येष्ट॑काः । \newline
24. आ॒लभ्येत्या᳚ - लभ्य॑ । \newline
25. इष्ट॑का॒ उपो पेष्ट॑का॒ इष्ट॑का॒ उप॑ । \newline
26. उप॑ दद्ध्याद् दद्ध्या॒ दुपोप॑ दद्ध्यात् । \newline
27. द॒द्ध्या॒ दे॒त दे॒तद् द॑द्ध्याद् दद्ध्या दे॒तत् । \newline
28. ए॒तद् वै वा ए॒त दे॒तद् वै । \newline
29. वै प्र॒जाप॑तेः प्र॒जाप॑ते॒र् वै वै प्र॒जाप॑तेः । \newline
30. प्र॒जाप॑ते रू॒पꣳ रू॒पम् प्र॒जाप॑तेः प्र॒जाप॑ते रू॒पम् । \newline
31. प्र॒जाप॑ते॒रिति॑ प्र॒जा - प॒तेः॒ । \newline
32. रू॒पम् प्रा॑जाप॒त्यः प्रा॑जाप॒त्यो रू॒पꣳ रू॒पम् प्रा॑जाप॒त्यः । \newline
33. प्रा॒जा॒प॒त्यो ऽश्वो ऽश्वः॑ प्राजाप॒त्यः प्रा॑जाप॒त्यो ऽश्वः॑ । \newline
34. प्रा॒जा॒प॒त्य इति॑ प्राजा - प॒त्यः । \newline
35. अश्वः॑ सा॒क्षाथ् सा॒क्षा दश्वो ऽश्वः॑ सा॒क्षात् । \newline
36. सा॒क्षा दे॒वैव सा॒क्षाथ् सा॒क्षा दे॒व । \newline
37. सा॒क्षादिति॑ स - अ॒क्षात् । \newline
38. ए॒व प्र॒जाप॑तये प्र॒जाप॑तय ए॒वैव प्र॒जाप॑तये । \newline
39. प्र॒जाप॑तये प्रति॒प्रोच्य॑ प्रति॒प्रोच्य॑ प्र॒जाप॑तये प्र॒जाप॑तये प्रति॒प्रोच्य॑ । \newline
40. प्र॒जाप॑तय॒ इति॑ प्र॒जा - प॒त॒ये॒ । \newline
41. प्र॒ति॒प्रोच्या॒ग्नि म॒ग्निम् प्र॑ति॒प्रोच्य॑ प्रति॒प्रोच्या॒ग्निम् । \newline
42. प्र॒ति॒प्रोच्येति॑ प्रति - प्रोच्य॑ । \newline
43. अ॒ग्निम् चि॑नोति चिनो त्य॒ग्नि म॒ग्निम् चि॑नोति । \newline
44. चि॒नो॒ति॒ न न चि॑नोति चिनोति॒ न । \newline
45. नार्ति॒ मार्ति॒म् न नार्ति᳚म् । \newline
46. आर्ति॒ मा ऽऽर्ति॒ मार्ति॒ मा । \newline
47. आर्च्छ॑ त्यृच्छ॒ त्यार्च्छ॑ति । \newline
48. ऋ॒च्छ॒ त्ये॒त दे॒त दृ॑च्छ त्यृच्छ त्ये॒तत् । \newline
49. ए॒तद् वै वा ए॒त दे॒तद् वै । \newline
50. वा अह्नो ऽह्नो॒ वै वा अह्नः॑ । \newline
51. अह्नो॑ रू॒पꣳ रू॒प मह्नो ऽह्नो॑ रू॒पम् । \newline
52. रू॒पं ॅयद् यद् रू॒पꣳ रू॒पं ॅयत् । \newline
53. यच्छ्वे॒तः श्वे॒तो यद् यच्छ्वे॒तः । \newline
54. श्वे॒तो ऽश्वो ऽश्वः॑ श्वे॒तः श्वे॒तो ऽश्वः॑ । \newline
55. अश्वो॒ रात्रि॑यै॒ रात्रि॑या॒ अश्वो ऽश्वो॒ रात्रि॑यै । \newline
56. रात्रि॑यै कृ॒ष्णः कृ॒ष्णो रात्रि॑यै॒ रात्रि॑यै कृ॒ष्णः । \newline
57. कृ॒ष्ण ए॒त दे॒तत् कृ॒ष्णः कृ॒ष्ण ए॒तत् । \newline
58. ए॒त दह्नो ऽह्न॑ ए॒त दे॒त दह्नः॑ । \newline
59. अह्नो॑ रू॒पꣳ रू॒प मह्नो ऽह्नो॑ रू॒पम् । \newline

\textbf{Ghana Paata } \newline

1. चि॒नु॒ते॒ यो यश्चि॑नुते चिनुते॒ यो वै वै यश्चि॑नुते चिनुते॒ यो वै । \newline
2. यो वै वै यो यो वै प्र॒जाप॑तये प्र॒जाप॑तये॒ वै यो यो वै प्र॒जाप॑तये । \newline
3. वै प्र॒जाप॑तये प्र॒जाप॑तये॒ वै वै प्र॒जाप॑तये प्रति॒प्रोच्य॑ प्रति॒प्रोच्य॑ प्र॒जाप॑तये॒ वै वै प्र॒जाप॑तये प्रति॒प्रोच्य॑ । \newline
4. प्र॒जाप॑तये प्रति॒प्रोच्य॑ प्रति॒प्रोच्य॑ प्र॒जाप॑तये प्र॒जाप॑तये प्रति॒प्रो च्या॒ग्नि म॒ग्निम् प्र॑ति॒प्रोच्य॑ प्र॒जाप॑तये प्र॒जाप॑तये प्रति॒प्रो च्या॒ग्निम् । \newline
5. प्र॒जाप॑तय॒ इति॑ प्र॒जा - प॒त॒ये॒ । \newline
6. प्र॒ति॒प्रो च्या॒ग्नि म॒ग्निम् प्र॑ति॒प्रोच्य॑ प्रति॒प्रो च्या॒ग्निम् चि॒नोति॑ चि॒नो त्य॒ग्निम् प्र॑ति॒प्रोच्य॑ प्रति॒प्रो च्या॒ग्निम् चि॒नोति॑ । \newline
7. प्र॒ति॒प्रोच्येति॑ प्रति - प्रोच्य॑ । \newline
8. अ॒ग्निम् चि॒नोति॑ चि॒नो त्य॒ग्नि म॒ग्निम् चि॒नोति॒ न न चि॒नो त्य॒ग्नि म॒ग्निम् चि॒नोति॒ न । \newline
9. चि॒नोति॒ न न चि॒नोति॑ चि॒नोति॒ नार्ति॒ मार्ति॒म् न चि॒नोति॑ चि॒नोति॒ नार्ति᳚म् । \newline
10. नार्ति॒ मार्ति॒म् न नार्ति॒ मा ऽऽर्ति॒म् न नार्ति॒ मा । \newline
11. आर्ति॒ मा ऽऽर्ति॒ मार्ति॒ मार्च्छ॑ त्यृच्छ॒ त्याऽऽर्ति॒ मार्ति॒ मार्च्छ॑ति । \newline
12. आर्च्छ॑ त्यृच्छ॒ त्यार्च्छ॒ त्यश्वा॒ वश्वा॑ वृच्छ॒ त्यार्च्छ॒ त्यश्वौ᳚ । \newline
13. ऋ॒च्छ॒ त्यश्वा॒ वश्वा॑ वृच्छ त्यृच्छ॒ त्यश्वा॑ व॒भितो॒ ऽभितो ऽश्वा॑ वृच्छ त्यृच्छ॒ त्यश्वा॑ व॒भितः॑ । \newline
14. अश्वा॑ व॒भितो॒ ऽभितो ऽश्वा॒ वश्वा॑ व॒भित॑ स्तिष्ठेताम् तिष्ठेता म॒भितो ऽश्वा॒ वश्वा॑ व॒भित॑ स्तिष्ठेताम् । \newline
15. अ॒भित॑ स्तिष्ठेताम् तिष्ठेता म॒भितो॒ ऽभित॑ स्तिष्ठेताम् कृ॒ष्णः कृ॒ष्ण स्ति॑ष्ठेता म॒भितो॒ ऽभित॑ स्तिष्ठेताम् कृ॒ष्णः । \newline
16. ति॒ष्ठे॒ता॒म् कृ॒ष्णः कृ॒ष्ण स्ति॑ष्ठेताम् तिष्ठेताम् कृ॒ष्ण उ॑त्तर॒त उ॑त्तर॒तः कृ॒ष्ण स्ति॑ष्ठेताम् तिष्ठेताम् कृ॒ष्ण उ॑त्तर॒तः । \newline
17. कृ॒ष्ण उ॑त्तर॒त उ॑त्तर॒तः कृ॒ष्णः कृ॒ष्ण उ॑त्तर॒तः श्वे॒तः श्वे॒त उ॑त्तर॒तः कृ॒ष्णः कृ॒ष्ण उ॑त्तर॒तः श्वे॒तः । \newline
18. उ॒त्त॒र॒तः श्वे॒तः श्वे॒त उ॑त्तर॒त उ॑त्तर॒तः श्वे॒तो दक्षि॑णो॒ दक्षि॑णः श्वे॒त उ॑त्तर॒त उ॑त्तर॒तः श्वे॒तो दक्षि॑णः । \newline
19. उ॒त्त॒र॒त इत्यु॑त् - त॒र॒तः । \newline
20. श्वे॒तो दक्षि॑णो॒ दक्षि॑णः श्वे॒तः श्वे॒तो दक्षि॑ण॒ स्तौ तौ दक्षि॑णः श्वे॒तः श्वे॒तो दक्षि॑ण॒ स्तौ । \newline
21. दक्षि॑ण॒ स्तौ तौ दक्षि॑णो॒ दक्षि॑ण॒ स्ता वा॒लभ्या॒ लभ्य॒ तौ दक्षि॑णो॒ दक्षि॑ण॒ स्ता वा॒लभ्य॑ । \newline
22. ता वा॒लभ्या॒ लभ्य॒ तौ ता वा॒लभ्येष्ट॑का॒ इष्ट॑का आ॒लभ्य॒ तौ ता वा॒लभ्येष्ट॑काः । \newline
23. आ॒लभ्येष्ट॑का॒ इष्ट॑का आ॒लभ्या॒ लभ्येष्ट॑का॒ उपोपेष्ट॑का आ॒लभ्या॒ लभ्येष्ट॑का॒ उप॑ । \newline
24. आ॒लभ्येत्या᳚ - लभ्य॑ । \newline
25. इष्ट॑का॒ उपोपेष्ट॑का॒ इष्ट॑का॒ उप॑ दद्ध्याद् दद्ध्या॒ दुपेष्ट॑का॒ इष्ट॑का॒ उप॑ दद्ध्यात् । \newline
26. उप॑ दद्ध्याद् दद्ध्या॒ दुपोप॑ दद्ध्या दे॒त दे॒तद् द॑द्ध्या॒ दुपोप॑ दद्ध्या दे॒तत् । \newline
27. द॒द्ध्या॒ दे॒त दे॒तद् द॑द्ध्याद् दद्ध्या दे॒तद् वै वा ए॒तद् द॑द्ध्याद् दद्ध्या दे॒तद् वै । \newline
28. ए॒तद् वै वा ए॒त दे॒तद् वै प्र॒जाप॑तेः प्र॒जाप॑ते॒र् वा ए॒त दे॒तद् वै प्र॒जाप॑तेः । \newline
29. वै प्र॒जाप॑तेः प्र॒जाप॑ते॒र् वै वै प्र॒जाप॑ते रू॒पꣳ रू॒पम् प्र॒जाप॑ते॒र् वै वै प्र॒जाप॑ते रू॒पम् । \newline
30. प्र॒जाप॑ते रू॒पꣳ रू॒पम् प्र॒जाप॑तेः प्र॒जाप॑ते रू॒पम् प्रा॑जाप॒त्यः प्रा॑जाप॒त्यो रू॒पम् प्र॒जाप॑तेः प्र॒जाप॑ते रू॒पम् प्रा॑जाप॒त्यः । \newline
31. प्र॒जाप॑ते॒रिति॑ प्र॒जा - प॒तेः॒ । \newline
32. रू॒पम् प्रा॑जाप॒त्यः प्रा॑जाप॒त्यो रू॒पꣳ रू॒पम् प्रा॑जाप॒त्यो ऽश्वो ऽश्वः॑ प्राजाप॒त्यो रू॒पꣳ रू॒पम् प्रा॑जाप॒त्यो ऽश्वः॑ । \newline
33. प्रा॒जा॒प॒त्यो ऽश्वो ऽश्वः॑ प्राजाप॒त्यः प्रा॑जाप॒त्यो ऽश्वः॑ सा॒क्षाथ् सा॒क्षा दश्वः॑ प्राजाप॒त्यः प्रा॑जाप॒त्यो ऽश्वः॑ सा॒क्षात् । \newline
34. प्रा॒जा॒प॒त्य इति॑ प्राजा - प॒त्यः । \newline
35. अश्वः॑ सा॒क्षाथ् सा॒क्षा दश्वो ऽश्वः॑ सा॒क्षा दे॒वैव सा॒क्षा दश्वो ऽश्वः॑ सा॒क्षा दे॒व । \newline
36. सा॒क्षा दे॒वैव सा॒क्षाथ् सा॒क्षा दे॒व प्र॒जाप॑तये प्र॒जाप॑तय ए॒व सा॒क्षाथ् सा॒क्षा दे॒व प्र॒जाप॑तये । \newline
37. सा॒क्षादिति॑ स - अ॒क्षात् । \newline
38. ए॒व प्र॒जाप॑तये प्र॒जाप॑तय ए॒वैव प्र॒जाप॑तये प्रति॒प्रोच्य॑ प्रति॒प्रोच्य॑ प्र॒जाप॑तय ए॒वैव प्र॒जाप॑तये प्रति॒प्रोच्य॑ । \newline
39. प्र॒जाप॑तये प्रति॒प्रोच्य॑ प्रति॒प्रोच्य॑ प्र॒जाप॑तये प्र॒जाप॑तये प्रति॒प्रो च्या॒ग्नि म॒ग्निम् प्र॑ति॒प्रोच्य॑ प्र॒जाप॑तये प्र॒जाप॑तये प्रति॒प्रो च्या॒ग्निम् । \newline
40. प्र॒जाप॑तय॒ इति॑ प्र॒जा - प॒त॒ये॒ । \newline
41. प्र॒ति॒प्रो च्या॒ग्नि म॒ग्निम् प्र॑ति॒प्रोच्य॑ प्रति॒प्रो च्या॒ग्निम् चि॑नोति चिनो त्य॒ग्निम् प्र॑ति॒प्रोच्य॑ प्रति॒प्रो च्या॒ग्निम् चि॑नोति । \newline
42. प्र॒ति॒प्रोच्येति॑ प्रति - प्रोच्य॑ । \newline
43. अ॒ग्निम् चि॑नोति चिनो त्य॒ग्नि म॒ग्निम् चि॑नोति॒ न न चि॑नो त्य॒ग्नि म॒ग्निम् चि॑नोति॒ न । \newline
44. चि॒नो॒ति॒ न न चि॑नोति चिनोति॒ नार्ति॒ मार्ति॒म् न चि॑नोति चिनोति॒ नार्ति᳚म् । \newline
45. नार्ति॒ मार्ति॒म् न नार्ति॒ मा ऽऽर्ति॒म् न नार्ति॒ मा । \newline
46. आर्ति॒ मा ऽऽर्ति॒ मार्ति॒ मार्च्छ॑ त्यृच्छ॒ त्याऽऽर्ति॒ मार्ति॒ मार्च्छ॑ति । \newline
47. आर्च्छ॑ त्यृच्छ॒ त्यार्च्छ॑ त्ये॒त दे॒त दृ॑च्छ॒ त्यार्च्छ॑ त्ये॒तत् । \newline
48. ऋ॒च्छ॒ त्ये॒त दे॒त दृ॑च्छ त्यृच्छ त्ये॒तद् वै वा ए॒तदृ॑च्छ त्यृच्छ त्ये॒तद् वै । \newline
49. ए॒तद् वै वा ए॒त दे॒तद् वा अह्नो ऽह्नो॒ वा ए॒त दे॒तद् वा अह्नः॑ । \newline
50. वा अह्नो ऽह्नो॒ वै वा अह्नो॑ रू॒पꣳ रू॒प मह्नो॒ वै वा अह्नो॑ रू॒पम् । \newline
51. अह्नो॑ रू॒पꣳ रू॒प मह्नो ऽह्नो॑ रू॒पं ॅयद् यद् रू॒प मह्नो ऽह्नो॑ रू॒पं ॅयत् । \newline
52. रू॒पं ॅयद् यद् रू॒पꣳ रू॒पं ॅयच् छ्वे॒तः श्वे॒तो यद् रू॒पꣳ रू॒पं ॅयच् छ्वे॒तः । \newline
53. यच् छ्वे॒तः श्वे॒तो यद् यच् छ्वे॒तो ऽश्वो ऽश्वः॑ श्वे॒तो यद् यच् छ्वे॒तो ऽश्वः॑ । \newline
54. श्वे॒तो ऽश्वो ऽश्वः॑ श्वे॒तः श्वे॒तो ऽश्वो॒ रात्रि॑यै॒ रात्रि॑या॒ अश्वः॑ श्वे॒तः श्वे॒तो ऽश्वो॒ रात्रि॑यै । \newline
55. अश्वो॒ रात्रि॑यै॒ रात्रि॑या॒ अश्वो ऽश्वो॒ रात्रि॑यै कृ॒ष्णः कृ॒ष्णो रात्रि॑या॒ अश्वो ऽश्वो॒ रात्रि॑यै कृ॒ष्णः । \newline
56. रात्रि॑यै कृ॒ष्णः कृ॒ष्णो रात्रि॑यै॒ रात्रि॑यै कृ॒ष्ण ए॒त दे॒तत् कृ॒ष्णो रात्रि॑यै॒ रात्रि॑यै कृ॒ष्ण ए॒तत् । \newline
57. कृ॒ष्ण ए॒त दे॒तत् कृ॒ष्णः कृ॒ष्ण ए॒त दह्नो ऽह्न॑ ए॒तत् कृ॒ष्णः कृ॒ष्ण ए॒त दह्नः॑ । \newline
58. ए॒त दह्नो ऽह्न॑ ए॒त दे॒त दह्नो॑ रू॒पꣳ रू॒प मह्न॑ ए॒त दे॒त दह्नो॑ रू॒पम् । \newline
59. अह्नो॑ रू॒पꣳ रू॒प मह्नो ऽह्नो॑ रू॒पं ॅयद् यद् रू॒प मह्नो ऽह्नो॑ रू॒पं ॅयत् । \newline
\pagebreak
\markright{ TS 5.7.1.3  \hfill https://www.vedavms.in \hfill}

\section{ TS 5.7.1.3 }

\textbf{TS 5.7.1.3 } \newline
\textbf{Samhita Paata} \newline

रू॒पं ॅयदिष्ट॑का॒ रात्रि॑यै॒ पुरी॑ष॒मिष्ट॑का उपधा॒स्यञ्छ्वे॒त-मश्व॑म॒भि मृ॑शे॒त् पुरी॑षमुपधा॒स्यन् कृ॒ष्णम॑होरा॒त्राभ्या॑मे॒वैनं॑ चिनुते हिरण्य पा॒त्रं मधोः᳚ पू॒र्णं द॑दाति मध॒व्यो॑ ऽसा॒नीति॑ सौ॒र्या चि॒त्रव॒त्याऽवे᳚क्षते चि॒त्रमे॒व भ॑वति म॒द्ध्यन्दि॒नेऽश्व॒मव॑ घ्रापयत्य॒सौ वा आ॑दि॒त्य इन्द्र॑ ए॒ष प्र॒जाप॑तिः प्राजाप॒त्योऽश्व॒स्तमे॒व सा॒क्षादृ॑द्ध्नोति ॥ \newline

\textbf{Pada Paata} \newline

रू॒पम् । यत् । इष्ट॑काः । रात्रि॑यै । पुरी॑षम् । इष्ट॑काः । उ॒प॒धा॒स्यन्नित्यु॑प - धा॒स्यन्न् । श्वे॒तम् । अश्व᳚म् । अ॒भीति॑ । मृ॒शे॒त् । पुरी॑षम् । उ॒प॒धा॒स्यन्नित्यु॑प - धा॒स्यन्न् । कृ॒ष्णम् । अ॒हो॒रा॒त्राभ्या॒मित्य॑हः - रा॒त्राभ्या᳚म् । ए॒व । ए॒न॒म् । चि॒नु॒ते॒ । हि॒र॒ण्य॒पा॒त्रमिति॑ हिरण्य-पा॒त्रम् । मधोः᳚ । पू॒र्णम् । द॒दा॒ति॒ । म॒ध॒व्यः॑ । अ॒सा॒नि॒ । इति॑ । सौ॒र्या । चि॒त्रव॒त्येति॑ चि॒त्र-व॒त्या॒ । अवेति॑ । ई॒क्ष॒ते॒ । चि॒त्रम् । ए॒व । भ॒व॒ति॒ । म॒द्ध्यन्दि॑ने । अश्व᳚म् । अवेति॑ । घ्रा॒प॒य॒ति॒ । अ॒सौ । वै । आ॒दि॒त्यः । इन्द्रः॑ । ए॒षः । प्र॒जाप॑ति॒रिति॑ प्र॒जा-प॒तिः॒ । प्रा॒जा॒प॒त्य इति॑ प्राजा - प॒त्यः । अश्वः॑ । तम् । ए॒व । सा॒क्षादिति॑ स - अ॒क्षात् । ऋ॒द्ध्नो॒ति॒ ॥  \newline


\textbf{Krama Paata} \newline

रू॒पम् ॅयत् । यदिष्ट॑काः । इष्ट॑का॒ रात्रि॑यै । रात्रि॑यै॒ पुरी॑षम् । पुरी॑ष॒मिष्ट॑काः । इष्ट॑का उपधा॒स्यन्न् । उ॒प॒धा॒स्यञ्छ्वे॒तम् । उ॒प॒धा॒स्यन्नित्यु॑प - धा॒स्यन्न् । श्वे॒तमश्व᳚म् । अश्व॑म॒भि । अ॒भि मृ॑शेत् । मृ॒शे॒त् पुरी॑षम् । पुरी॑षमुपधा॒स्यन्न् । उ॒प॒धा॒स्यन् कृ॒ष्णम् । उ॒प॒धा॒स्यन्नित्यु॑प - धा॒स्यन्न् । कृ॒ष्णम॑होरा॒त्राभ्या᳚म् । अ॒हो॒रा॒त्राभ्या॑मे॒व । अ॒हो॒रा॒त्राभ्या॒मित्य॑हः - रा॒त्राभ्या᳚म् । ए॒वैन᳚म् । ए॒न॒म् चि॒नु॒ते॒ । चि॒नु॒ते॒ हि॒र॒ण्य॒पा॒त्रम् । हि॒र॒ण्य॒पा॒त्रम् मधोः᳚ । हि॒र॒ण्य॒पा॒त्रमिति॑ हिरण्य - पा॒त्रम् । मधोः᳚ पू॒र्णम् । पू॒र्णम् द॑दाति । द॒दा॒ति॒ म॒ध॒व्यः॑ । म॒ध॒व्यो॑ऽसानि । अ॒सा॒नीति॑ । इति॑ सौ॒र्या । सौ॒र्या चि॒त्रव॑त्या । चि॒त्रव॒त्याऽव॑ । चि॒त्रव॒त्येति॑ चि॒त्र - व॒त्या॒ । अवे᳚क्षते । ई॒क्ष॒ते॒ चि॒त्रम् । चि॒त्रमे॒व । ए॒व भ॑वति । भ॒व॒ति॒ म॒द्ध्यन्दि॑ने । म॒द्ध्यन्दि॒नेऽश्व᳚म् । अश्व॒मव॑ । अव॑ घ्रापयति । घ्रा॒प॒य॒त्य॒सौ । अ॒सौ वै । वा आ॑दि॒त्यः । आ॒दि॒त्य इन्द्रः॑ । इन्द्र॑ ए॒षः । ए॒ष प्र॒जाप॑तिः । प्र॒जाप॑तिः प्राजाप॒त्यः । प्र॒जाप॑ति॒रिति॑ प्र॒जा - प॒तिः॒ । प्रा॒जा॒प॒त्योऽश्वः॑ । प्रा॒जा॒प॒त्य इति॑ प्राजा - प॒त्यः । अश्व॒स्तम् । तमे॒व । ए॒व सा॒क्षात् । सा॒क्षादृ॑द्ध्नोति । सा॒क्षादिति॑ स - अ॒क्षात् । ऋ॒द्ध्नो॒तीत्यृ॑द्ध्नोति । \newline

\textbf{Jatai Paata} \newline

1. रू॒पं ॅयद् यद् रू॒पꣳ रू॒पं ॅयत् । \newline
2. यदिष्ट॑का॒ इष्ट॑का॒ यद् यदिष्ट॑काः । \newline
3. इष्ट॑का॒ रात्रि॑यै॒ रात्रि॑या॒ इष्ट॑का॒ इष्ट॑का॒ रात्रि॑यै । \newline
4. रात्रि॑यै॒ पुरी॑ष॒म् पुरी॑षꣳ॒॒ रात्रि॑यै॒ रात्रि॑यै॒ पुरी॑षम् । \newline
5. पुरी॑ष॒ मिष्ट॑का॒ इष्ट॑काः॒ पुरी॑ष॒म् पुरी॑ष॒ मिष्ट॑काः । \newline
6. इष्ट॑का उपधा॒स्यन् नु॑पधा॒स्यन् निष्ट॑का॒ इष्ट॑का उपधा॒स्यन्न् । \newline
7. उ॒प॒धा॒स्यञ् छ्वे॒तꣳ श्वे॒त मु॑पधा॒स्यन् नु॑पधा॒स्यञ् छ्वे॒तम् । \newline
8. उ॒प॒धा॒स्यन्नित्यु॑प - धा॒स्यन्न् । \newline
9. श्वे॒त मश्व॒ मश्वꣳ॑ श्वे॒तꣳ श्वे॒त मश्व᳚म् । \newline
10. अश्व॑ म॒भ्य॑ भ्यश्व॒ मश्व॑ म॒भि । \newline
11. अ॒भि मृ॑शेन् मृशे द॒भ्य॑भि मृ॑शेत् । \newline
12. मृ॒शे॒त् पुरी॑ष॒म् पुरी॑षम् मृशेन् मृशे॒त् पुरी॑षम् । \newline
13. पुरी॑ष मुपधा॒स्यन् नु॑पधा॒स्यन् पुरी॑ष॒म् पुरी॑ष मुपधा॒स्यन्न् । \newline
14. उ॒प॒धा॒स्यन् कृ॒ष्णम् कृ॒ष्ण मु॑पधा॒स्यन् नु॑पधा॒स्यन् कृ॒ष्णम् । \newline
15. उ॒प॒धा॒स्यन्नित्यु॑प - धा॒स्यन्न् । \newline
16. कृ॒ष्ण म॑होरा॒त्राभ्या॑ महोरा॒त्राभ्या᳚म् कृ॒ष्णम् कृ॒ष्ण म॑होरा॒त्राभ्या᳚म् । \newline
17. अ॒हो॒रा॒त्राभ्या॑ मे॒वै वाहो॑रा॒त्राभ्या॑ महोरा॒त्राभ्या॑ मे॒व । \newline
18. अ॒हो॒रा॒त्राभ्या॒मित्य॑हः - रा॒त्राभ्या᳚म् । \newline
19. ए॒वैन॑ मेन मे॒वै वैन᳚म् । \newline
20. ए॒न॒म् चि॒नु॒ते॒ चि॒नु॒त॒ ए॒न॒ मे॒न॒म् चि॒नु॒ते॒ । \newline
21. चि॒नु॒ते॒ हि॒र॒ण्य॒पा॒त्रꣳ हि॑रण्यपा॒त्रम् चि॑नुते चिनुते हिरण्यपा॒त्रम् । \newline
22. हि॒र॒ण्य॒पा॒त्रम् मधो॒र् मधोर्॑. हिरण्यपा॒त्रꣳ हि॑रण्यपा॒त्रम् मधोः᳚ । \newline
23. हि॒र॒ण्य॒पा॒त्रमिति॑ हिरण्य - पा॒त्रम् । \newline
24. मधोः᳚ पू॒र्णम् पू॒र्णम् मधो॒र् मधोः᳚ पू॒र्णम् । \newline
25. पू॒र्णम् द॑दाति ददाति पू॒र्णम् पू॒र्णम् द॑दाति । \newline
26. द॒दा॒ति॒ म॒ध॒व्यो॑ मध॒व्यो॑ ददाति ददाति मध॒व्यः॑ । \newline
27. म॒ध॒व्यो॑ ऽसा न्यसानि मध॒व्यो॑ मध॒व्यो॑ ऽसानि । \newline
28. अ॒सा॒नी तीत्य॑ सा न्य सा॒नीति॑ । \newline
29. इति॑ सौ॒र्या सौ॒र्ये तीति॑ सौ॒र्या । \newline
30. सौ॒र्या चि॒त्रव॑त्या चि॒त्रव॑त्या सौ॒र्या सौ॒र्या चि॒त्रव॑त्या । \newline
31. चि॒त्रव॒त्या ऽवाव॑ चि॒त्रव॑त्या चि॒त्रव॒त्या ऽव॑ । \newline
32. चि॒त्रव॒त्येति॑ चि॒त्र - व॒त्या॒ । \newline
33. अवे᳚क्षत ईक्ष॒ते ऽवावे᳚ क्षते । \newline
34. ई॒क्ष॒ते॒ चि॒त्रम् चि॒त्र मी᳚क्षत ईक्षते चि॒त्रम् । \newline
35. चि॒त्र मे॒वैव चि॒त्रम् चि॒त्र मे॒व । \newline
36. ए॒व भ॑वति भव त्ये॒वैव भ॑वति । \newline
37. भ॒व॒ति॒ म॒द्ध्यन्दि॑ने म॒द्ध्यन्दि॑ने भवति भवति म॒द्ध्यन्दि॑ने । \newline
38. म॒द्ध्यन्दि॒ने ऽश्व॒ मश्व॑म् म॒द्ध्यन्दि॑ने म॒द्ध्यन्दि॒ने ऽश्व᳚म् । \newline
39. अश्व॒ मवा वाश्व॒ मश्व॒ मव॑ । \newline
40. अव॑ घ्रापयति घ्रापय॒ त्यवाव॑ घ्रापयति । \newline
41. घ्रा॒प॒य॒ त्य॒सा व॒सौ घ्रा॑पयति घ्रापय त्य॒सौ । \newline
42. अ॒सौ वै वा अ॒सा व॒सौ वै । \newline
43. वा आ॑दि॒त्य आ॑दि॒त्यो वै वा आ॑दि॒त्यः । \newline
44. आ॒दि॒त्य इन्द्र॒ इन्द्र॑ आदि॒त्य आ॑दि॒त्य इन्द्रः॑ । \newline
45. इन्द्र॑ ए॒ष ए॒ष इन्द्र॒ इन्द्र॑ ए॒षः । \newline
46. ए॒ष प्र॒जाप॑तिः प्र॒जाप॑ति रे॒ष ए॒ष प्र॒जाप॑तिः । \newline
47. प्र॒जाप॑तिः प्राजाप॒त्यः प्रा॑जाप॒त्यः प्र॒जाप॑तिः प्र॒जाप॑तिः प्राजाप॒त्यः । \newline
48. प्र॒जाप॑ति॒रिति॑ प्र॒जा - प॒तिः॒ । \newline
49. प्रा॒जा॒प॒त्यो ऽश्वो ऽश्वः॑ प्राजाप॒त्यः प्रा॑जाप॒त्यो ऽश्वः॑ । \newline
50. प्रा॒जा॒प॒त्य इति॑ प्राजा - प॒त्यः । \newline
51. अश्व॒ स्तम् त मश्वो ऽश्व॒ स्तम् । \newline
52. त मे॒वैव तम् त मे॒व । \newline
53. ए॒व सा॒क्षाथ् सा॒क्षा दे॒वैव सा॒क्षात् । \newline
54. सा॒क्षा दृ॑द्ध्नो त्यृद्ध्नोति सा॒क्षाथ् सा॒क्षा दृ॑द्ध्नोति । \newline
55. सा॒क्षादिति॑ स - अ॒क्षात् । \newline
56. ऋ॒द्ध्नो॒तीत्यृ॑द्ध्नोति । \newline

\textbf{Ghana Paata } \newline

1. रू॒पं ॅयद् यद् रू॒पꣳ रू॒पं ॅयदिष्ट॑का॒ इष्ट॑का॒ यद् रू॒पꣳ रू॒पं ॅयदिष्ट॑काः । \newline
2. यदिष्ट॑का॒ इष्ट॑का॒ यद् यदिष्ट॑का॒ रात्रि॑यै॒ रात्रि॑या॒ इष्ट॑का॒ यद् यदिष्ट॑का॒ रात्रि॑यै । \newline
3. इष्ट॑का॒ रात्रि॑यै॒ रात्रि॑या॒ इष्ट॑का॒ इष्ट॑का॒ रात्रि॑यै॒ पुरी॑ष॒म् पुरी॑षꣳ॒॒ रात्रि॑या॒ इष्ट॑का॒ इष्ट॑का॒ रात्रि॑यै॒ पुरी॑षम् । \newline
4. रात्रि॑यै॒ पुरी॑ष॒म् पुरी॑षꣳ॒॒ रात्रि॑यै॒ रात्रि॑यै॒ पुरी॑ष॒ मिष्ट॑का॒ इष्ट॑काः॒ पुरी॑षꣳ॒॒ रात्रि॑यै॒ रात्रि॑यै॒ पुरी॑ष॒ मिष्ट॑काः । \newline
5. पुरी॑ष॒ मिष्ट॑का॒ इष्ट॑काः॒ पुरी॑ष॒म् पुरी॑ष॒ मिष्ट॑का उपधा॒स्यन् नु॑पधा॒स्यन् निष्ट॑काः॒ पुरी॑ष॒म् पुरी॑ष॒ मिष्ट॑का उपधा॒स्यन्न् । \newline
6. इष्ट॑का उपधा॒स्यन् नु॑पधा॒स्यन् निष्ट॑का॒ इष्ट॑का उपधा॒स्यञ् छ्वे॒तꣳ श्वे॒त मु॑पधा॒स्यन् निष्ट॑का॒ इष्ट॑का उपधा॒स्यञ् छ्वे॒तम् । \newline
7. उ॒प॒धा॒स्यञ् छ्वे॒तꣳ श्वे॒त मु॑पधा॒स्यन् नु॑पधा॒स्यञ् छ्वे॒त मश्व॒ मश्वꣳ॑ श्वे॒त मु॑पधा॒स्यन् नु॑पधा॒स्यञ् छ्वे॒त मश्व᳚म् । \newline
8. उ॒प॒धा॒स्यन्नित्यु॑प - धा॒स्यन्न् । \newline
9. श्वे॒त मश्व॒ मश्वꣳ॑ श्वे॒तꣳ श्वे॒त मश्व॑ म॒भ्य॑ भ्यश्वꣳ॑ श्वे॒तꣳ श्वे॒त मश्व॑ म॒भि । \newline
10. अश्व॑ म॒भ्य॑ भ्यश्व॒ मश्व॑ म॒भि मृ॑शेन् मृशे द॒भ्य श्व॒ मश्व॑ म॒भि मृ॑शेत् । \newline
11. अ॒भि मृ॑शेन् मृशे द॒भ्य॑भि मृ॑शे॒त् पुरी॑ष॒म् पुरी॑षम् मृशे द॒भ्य॑भि मृ॑शे॒त् पुरी॑षम् । \newline
12. मृ॒शे॒त् पुरी॑ष॒म् पुरी॑षम् मृशेन् मृशे॒त् पुरी॑ष मुपधा॒स्यन् नु॑पधा॒स्यन् पुरी॑षम् मृशेन् मृशे॒त् पुरी॑ष मुपधा॒स्यन्न् । \newline
13. पुरी॑ष मुपधा॒स्यन् नु॑पधा॒स्यन् पुरी॑ष॒म् पुरी॑ष मुपधा॒स्यन् कृ॒ष्णम् कृ॒ष्ण मु॑पधा॒स्यन् पुरी॑ष॒म् पुरी॑ष मुपधा॒स्यन् कृ॒ष्णम् । \newline
14. उ॒प॒धा॒स्यन् कृ॒ष्णम् कृ॒ष्ण मु॑पधा॒स्यन् नु॑पधा॒स्यन् कृ॒ष्ण म॑होरा॒त्राभ्या॑ महोरा॒त्राभ्या᳚म् कृ॒ष्ण मु॑पधा॒स्यन् नु॑पधा॒स्यन् कृ॒ष्ण म॑होरा॒त्राभ्या᳚म् । \newline
15. उ॒प॒धा॒स्यन्नित्यु॑प - धा॒स्यन्न् । \newline
16. कृ॒ष्ण म॑होरा॒त्राभ्या॑ महोरा॒त्राभ्या᳚म् कृ॒ष्णम् कृ॒ष्ण म॑होरा॒त्राभ्या॑ मे॒वै वाहो॑रा॒त्राभ्या᳚म् कृ॒ष्णम् कृ॒ष्ण म॑होरा॒त्राभ्या॑ मे॒व । \newline
17. अ॒हो॒रा॒त्राभ्या॑ मे॒वै वाहो॑रा॒त्राभ्या॑ महोरा॒त्राभ्या॑ मे॒वैन॑ मेन मे॒वाहो॑रा॒त्राभ्या॑ महोरा॒त्राभ्या॑ मे॒वैन᳚म् । \newline
18. अ॒हो॒रा॒त्राभ्या॒मित्य॑हः - रा॒त्राभ्या᳚म् । \newline
19. ए॒वैन॑ मेन मे॒वै वैन॑म् चिनुते चिनुत एन मे॒वै वैन॑म् चिनुते । \newline
20. ए॒न॒म् चि॒नु॒ते॒ चि॒नु॒त॒ ए॒न॒ मे॒न॒म् चि॒नु॒ते॒ हि॒र॒ण्य॒पा॒त्रꣳ हि॑रण्यपा॒त्रम् चि॑नुत एन मेनम् चिनुते हिरण्यपा॒त्रम् । \newline
21. चि॒नु॒ते॒ हि॒र॒ण्य॒पा॒त्रꣳ हि॑रण्यपा॒त्रम् चि॑नुते चिनुते हिरण्यपा॒त्रम् मधो॒र् मधोर्॑. हिरण्यपा॒त्रम् चि॑नुते चिनुते हिरण्यपा॒त्रम् मधोः᳚ । \newline
22. हि॒र॒ण्य॒पा॒त्रम् मधो॒र् मधोर्॑. हिरण्यपा॒त्रꣳ हि॑रण्यपा॒त्रम् मधोः᳚ पू॒र्णम् पू॒र्णम् मधोर्॑. हिरण्यपा॒त्रꣳ हि॑रण्यपा॒त्रम् मधोः᳚ पू॒र्णम् । \newline
23. हि॒र॒ण्य॒पा॒त्रमिति॑ हिरण्य - पा॒त्रम् । \newline
24. मधोः᳚ पू॒र्णम् पू॒र्णम् मधो॒र् मधोः᳚ पू॒र्णम् द॑दाति ददाति पू॒र्णम् मधो॒र् मधोः᳚ पू॒र्णम् द॑दाति । \newline
25. पू॒र्णम् द॑दाति ददाति पू॒र्णम् पू॒र्णम् द॑दाति मध॒व्यो॑ मध॒व्यो॑ ददाति पू॒र्णम् पू॒र्णम् द॑दाति मध॒व्यः॑ । \newline
26. द॒दा॒ति॒ म॒ध॒व्यो॑ मध॒व्यो॑ ददाति ददाति मध॒व्यो॑ ऽसा न्यसानि मध॒व्यो॑ ददाति ददाति मध॒व्यो॑ ऽसानि । \newline
27. म॒ध॒व्यो॑ ऽसा न्यसानि मध॒व्यो॑ मध॒व्यो॑ ऽसा॒नी तीत्य॑सानि मध॒व्यो॑ मध॒व्यो॑ ऽसा॒नीति॑ । \newline
28. अ॒सा॒नी तीत्य॑सा न्यसा॒नीति॑ सौ॒र्या सौ॒र्ये त्य॑सा न्यसा॒नीति॑ सौ॒र्या । \newline
29. इति॑ सौ॒र्या सौ॒र्येतीति॑ सौ॒र्या चि॒त्रव॑त्या चि॒त्रव॑त्या सौ॒र्येतीति॑ सौ॒र्या चि॒त्रव॑त्या । \newline
30. सौ॒र्या चि॒त्रव॑त्या चि॒त्रव॑त्या सौ॒र्या सौ॒र्या चि॒त्रव॒त्या ऽवाव॑ चि॒त्रव॑त्या सौ॒र्या सौ॒र्या चि॒त्रव॒त्या ऽव॑ । \newline
31. चि॒त्रव॒त्या ऽवाव॑ चि॒त्रव॑त्या चि॒त्रव॒त्या ऽवे᳚क्षत ईक्ष॒ते ऽव॑ चि॒त्रव॑त्या चि॒त्रव॒त्या ऽवे᳚क्षते । \newline
32. चि॒त्रव॒त्येति॑ चि॒त्र - व॒त्या॒ । \newline
33. अवे᳚क्षत ईक्ष॒ते ऽवावे᳚क्षते चि॒त्रम् चि॒त्र मी᳚क्ष॒ते ऽवावे᳚क्षते चि॒त्रम् । \newline
34. ई॒क्ष॒ते॒ चि॒त्रम् चि॒त्र मी᳚क्षत ईक्षते चि॒त्र मे॒वैव चि॒त्र मी᳚क्षत ईक्षते चि॒त्र मे॒व । \newline
35. चि॒त्र मे॒वैव चि॒त्रम् चि॒त्र मे॒व भ॑वति भव त्ये॒व चि॒त्रम् चि॒त्र मे॒व भ॑वति । \newline
36. ए॒व भ॑वति भव त्ये॒वैव भ॑वति म॒द्ध्यन्दि॑ने म॒द्ध्यन्दि॑ने भव त्ये॒वैव भ॑वति म॒द्ध्यन्दि॑ने । \newline
37. भ॒व॒ति॒ म॒द्ध्यन्दि॑ने म॒द्ध्यन्दि॑ने भवति भवति म॒द्ध्यन्दि॒ने ऽश्व॒ मश्व॑म् म॒द्ध्यन्दि॑ने भवति भवति म॒द्ध्यन्दि॒ने ऽश्व᳚म् । \newline
38. म॒द्ध्यन्दि॒ने ऽश्व॒ मश्व॑म् म॒द्ध्यन्दि॑ने म॒द्ध्यन्दि॒ने ऽश्व॒ मवा वाश्व॑म् म॒द्ध्यन्दि॑ने म॒द्ध्यन्दि॒ने ऽश्व॒ मव॑ । \newline
39. अश्व॒ मवावाश्व॒ मश्व॒ मव॑ घ्रापयति घ्रापय॒ त्यवाश्व॒ मश्व॒ मव॑ घ्रापयति । \newline
40. अव॑ घ्रापयति घ्रापय॒ त्यवाव॑ घ्रापय त्य॒सा व॒सौ घ्रा॑पय॒ त्यवाव॑ घ्रापय त्य॒सौ । \newline
41. घ्रा॒प॒य॒ त्य॒सा व॒सौ घ्रा॑पयति घ्रापय त्य॒सौ वै वा अ॒सौ घ्रा॑पयति घ्रापय त्य॒सौ वै । \newline
42. अ॒सौ वै वा अ॒सा व॒सौ वा आ॑दि॒त्य आ॑दि॒त्यो वा अ॒सा व॒सौ वा आ॑दि॒त्यः । \newline
43. वा आ॑दि॒त्य आ॑दि॒त्यो वै वा आ॑दि॒त्य इन्द्र॒ इन्द्र॑ आदि॒त्यो वै वा आ॑दि॒त्य इन्द्रः॑ । \newline
44. आ॒दि॒त्य इन्द्र॒ इन्द्र॑ आदि॒त्य आ॑दि॒त्य इन्द्र॑ ए॒ष ए॒ष इन्द्र॑ आदि॒त्य आ॑दि॒त्य इन्द्र॑ ए॒षः । \newline
45. इन्द्र॑ ए॒ष ए॒ष इन्द्र॒ इन्द्र॑ ए॒ष प्र॒जाप॑तिः प्र॒जाप॑ति रे॒ष इन्द्र॒ इन्द्र॑ ए॒ष प्र॒जाप॑तिः । \newline
46. ए॒ष प्र॒जाप॑तिः प्र॒जाप॑ति रे॒ष ए॒ष प्र॒जाप॑तिः प्राजाप॒त्यः प्रा॑जाप॒त्यः प्र॒जाप॑ति रे॒ष ए॒ष प्र॒जाप॑तिः प्राजाप॒त्यः । \newline
47. प्र॒जाप॑तिः प्राजाप॒त्यः प्रा॑जाप॒त्यः प्र॒जाप॑तिः प्र॒जाप॑तिः प्राजाप॒त्यो ऽश्वो ऽश्वः॑ प्राजाप॒त्यः प्र॒जाप॑तिः प्र॒जाप॑तिः प्राजाप॒त्यो ऽश्वः॑ । \newline
48. प्र॒जाप॑ति॒रिति॑ प्र॒जा - प॒तिः॒ । \newline
49. प्रा॒जा॒प॒त्यो ऽश्वो ऽश्वः॑ प्राजाप॒त्यः प्रा॑जाप॒त्यो ऽश्व॒ स्तम् त मश्वः॑ प्राजाप॒त्यः प्रा॑जाप॒त्यो ऽश्व॒ स्तम् । \newline
50. प्रा॒जा॒प॒त्य इति॑ प्राजा - प॒त्यः । \newline
51. अश्व॒ स्तम् त मश्वो ऽश्व॒ स्त मे॒वैव त मश्वो ऽश्व॒ स्त मे॒व । \newline
52. त मे॒वैव तम् त मे॒व सा॒क्षाथ् सा॒क्षा दे॒व तम् त मे॒व सा॒क्षात् । \newline
53. ए॒व सा॒क्षाथ् सा॒क्षा दे॒वैव सा॒क्षा दृ॑द्ध्नो त्यृद्ध्नोति सा॒क्षा दे॒वैव सा॒क्षा दृ॑द्ध्नोति । \newline
54. सा॒क्षा दृ॑द्ध्नो त्यृद्ध्नोति सा॒क्षाथ् सा॒क्षा दृ॑द्ध्नोति । \newline
55. सा॒क्षादिति॑ स - अ॒क्षात् । \newline
56. ऋ॒द्ध्नो॒तीत्यृ॑द्ध्नोति । \newline
\pagebreak
\markright{ TS 5.7.2.1  \hfill https://www.vedavms.in \hfill}

\section{ TS 5.7.2.1 }

\textbf{TS 5.7.2.1 } \newline
\textbf{Samhita Paata} \newline

त्वाम॑ग्ने वृष॒भं चेकि॑तानं॒ पुन॒र्युवा॑नं ज॒नय॑न्नु॒पागां᳚ । अ॒स्थू॒रि णो॒ गार्.ह॑पत्यानि सन्तु ति॒ग्मेन॑ नो॒ ब्रह्म॑णा॒ सꣳ शि॑शाधि ॥प॒शवो॒ वा ए॒ते यदिष्ट॑का॒श्चित्यां᳚ चित्यामृष॒भमुप॑ दधाति मिथु॒नमे॒वास्य॒ तद्-य॒ज्ञे क॑रोति प्र॒जन॑नाय॒ तस्मा᳚द्-यू॒थेयू॑थ ऋष॒भः ॥ सं॒ॅव॒थ्स॒रस्य॑ प्रति॒मां ॅयां त्वा॑ रात्र्यु॒पास॑ते । प्र॒जाꣳ सु॒वीरां᳚ कृ॒त्वा विश्व॒मायु॒र्व्य॑श्नवत् ॥ प्रा॒जा॒प॒त्या - [  ] \newline

\textbf{Pada Paata} \newline

त्वाम् । अ॒ग्ने॒ । वृ॒ष॒भम् । चेकि॑तानम् । पुनः॑ । युवा॑नम् । ज॒नयन्न्॑ । उ॒पागा॒मित्यु॑प- आगा᳚म् ॥ अ॒स्थू॒रि । नः॒ । गार्.ह॑पत्या॒नीति॒ गार्.ह॑ - प॒त्या॒नि॒ । स॒न्तु॒ । ति॒ग्मेन॑ । नः॒ । ब्रह्म॑णा । समिति॑ । शि॒शा॒धि॒ ॥ प॒शवः॑ । वै । ए॒ते । यत् । इष्ट॑काः । चित्या᳚चिंत्या॒मिति॒ चित्यां᳚ - चि॒त्या॒म् । ऋ॒ष॒भम् । उपेति॑ । द॒धा॒ति॒ । मि॒थु॒नम् । ए॒व । अ॒स्य॒ । तत् । य॒ज्ञे । क॒रो॒ति॒ । प्र॒जन॑ना॒येति॑ प्र - जन॑नाय । तस्मा᳚त् । यू॒थेयू॑थ॒ इति॑ यू॒थे - यू॒थे॒ । ऋ॒ष॒भः ॥ सं॒ॅव॒थ्स॒रस्येति॑ सं - व॒थ्स॒रस्य॑ । प्र॒ति॒मामिति॑ प्रति - माम् । याम् । त्वा॒ । रा॒त्रि॒ । उ॒पास॑त॒ इत्यु॑प - आस॑ते ॥ प्र॒जामिति॑ प्र- जाम् । सु॒वीरा॒मिति॑ सु - वीरा᳚म् । कृ॒त्वा । विश्व᳚म् । आयुः॑ । वीति॑ । अ॒श्न॒व॒त् ॥ प्रा॒जा॒प॒त्यामिति॑ प्राजा - प॒त्याम् ।  \newline


\textbf{Krama Paata} \newline

त्वाम॑ग्ने । अ॒ग्ने॒ वृ॒ष॒भम् । वृ॒ष॒भम् चेकि॑तानम् । चेकि॑तान॒म् पुनः॑ । पुन॒र् युवा॑नम् । युवा॑नम् ज॒नयन्न्॑ । ज॒नय॑न्नु॒पागा᳚म् । उ॒पागा॒मित्यु॑प - आगा᳚म् ॥ अ॒स्थू॒रिणः॑ । नो॒ गार्.ह॑पत्यानि । गार्.ह॑पत्यानि सन्तु । गार्.ह॑पत्या॒नीति॒ गार्.ह॑ - प॒त्या॒नि॒ । स॒न्तु॒ ति॒ग्मेन॑ । ति॒ग्मेन॑ नः । नो॒ ब्रह्म॑णा । ब्रह्म॑णा॒ सम् । सꣳ शि॑शाधि । शि॒शा॒धीति॑ शिशाधि ॥ प॒शवो॒ वै । वा ए॒ते । ए॒ते यत् । यदिष्ट॑काः । इष्ट॑का॒श्चित्या᳚ञ्चित्याम् । चित्या᳚ञ्चित्यामृष॒भम् । चित्या᳚ञ्चित्या॒मिति॒ चित्या᳚म् - चि॒त्या॒म् । ऋ॒ष॒भमुप॑ । उप॑ दधाति । द॒धा॒ति॒ मि॒थु॒नम् । मि॒थु॒नमे॒व । ए॒वास्य॑ । अ॒स्य॒ तत् । तद् य॒ज्ञे । य॒ज्ञे क॑रोति । क॒रो॒ति॒ प्र॒जन॑नाय । प्र॒जन॑नाय॒ तस्मा᳚त् । प्र॒जन॑ना॒येति॑ प्र - जन॑नाय । तस्मा᳚द् यू॒थेयू॑थे । यू॒थेयू॑थ ऋष॒भः । यू॒थेयू॑थ॒ इति॑ यू॒थे - यू॒थे॒ । ऋ॒ष॒भ इत्यृ॑ष॒भः ॥ स॒म्ॅव॒थ्स॒रस्य॑ प्रति॒माम् । स॒म्ॅव॒थ्स॒रस्येति॑ सम् - व॒थ्स॒रस्य॑ । प्र॒ति॒माम् ॅयाम् । प्र॒ति॒मामिति॑ प्रति - माम् । याम् त्वा᳚ । त्वा॒ रा॒त्रि॒ । रा॒त्र्यु॒पास॑ते । उ॒पास॑त॒ इत्यु॑प - आस॑ते ॥ प्र॒जाꣳ सु॒वीरा᳚म् । प्र॒जामिति॑ प्र - जाम् । सु॒वीरा᳚म् कृ॒त्वा । सु॒वीरा॒मिति॑ सु - वीरा᳚म् । कृ॒त्वा विश्व᳚म् । विश्व॒मायुः॑ । आयु॒र् वि । व्य॑श्ञवत् । अ॒श्ञ॒व॒दित्य॑श्ञवत् ॥ प्रा॒जा॒प॒त्यामे॒ताम् । प्रा॒जा॒प॒त्यामिति॑ प्राजा - प॒त्याम् \newline

\textbf{Jatai Paata} \newline

1. त्वा म॑ग्ने ऽग्ने॒ त्वाम् त्वा म॑ग्ने । \newline
2. अ॒ग्ने॒ वृ॒ष॒भं ॅवृ॑ष॒भ म॑ग्ने ऽग्ने वृष॒भम् । \newline
3. वृ॒ष॒भम् चेकि॑तान॒म् चेकि॑तानं ॅवृष॒भं ॅवृ॑ष॒भम् चेकि॑तानम् । \newline
4. चेकि॑तान॒म् पुनः॒ पुन॒ श्चेकि॑तान॒म् चेकि॑तान॒म् पुनः॑ । \newline
5. पुन॒र् युवा॑नं॒ ॅयुवा॑न॒म् पुनः॒ पुन॒र् युवा॑नम् । \newline
6. युवा॑नम् ज॒नय॑न् ज॒नय॒न्॒. युवा॑नं॒ ॅयुवा॑नम् ज॒नयन्न्॑ । \newline
7. ज॒नय॑न् नु॒पागा॑ मु॒पागा᳚म् ज॒नय॑न् ज॒नय॑न् नु॒पागा᳚म् । \newline
8. उ॒पागा॒मित्यु॑प - आगा᳚म् । \newline
9. अ॒स्थू॒रि नो॑ नो ऽस्थू॒र्य॑स्थू॒रि नः॑ । \newline
10. नो॒ गार्.ह॑पत्यानि॒ गार्.ह॑पत्यानि नो नो॒ गार्.ह॑पत्यानि । \newline
11. गार्.ह॑पत्यानि सन्तु सन्तु॒ गार्.ह॑पत्यानि॒ गार्.ह॑पत्यानि सन्तु । \newline
12. गार्.ह॑पत्या॒नीति॒ गार्.ह॑ - प॒त्या॒नि॒ । \newline
13. स॒न्तु॒ ति॒ग्मेन॑ ति॒ग्मेन॑ सन्तु सन्तु ति॒ग्मेन॑ । \newline
14. ति॒ग्मेन॑ नो न स्ति॒ग्मेन॑ ति॒ग्मेन॑ नः । \newline
15. नो॒ ब्रह्म॑णा॒ ब्रह्म॑णा नो नो॒ ब्रह्म॑णा । \newline
16. ब्रह्म॑णा॒ सꣳ सम् ब्रह्म॑णा॒ ब्रह्म॑णा॒ सम् । \newline
17. सꣳ शि॑शाधि शिशाधि॒ सꣳ सꣳ शि॑शाधि । \newline
18. शि॒शा॒धीति॑ शिशाधि । \newline
19. प॒शवो॒ वै वै प॒शवः॑ प॒शवो॒ वै । \newline
20. वा ए॒त ए॒ते वै वा ए॒ते । \newline
21. ए॒ते यद् यदे॒त ए॒ते यत् । \newline
22. यदिष्ट॑का॒ इष्ट॑का॒ यद् यदिष्ट॑काः । \newline
23. इष्ट॑का॒ श्चित्यां᳚चित्या॒म् चित्यां᳚चित्या॒ मिष्ट॑का॒ इष्ट॑का॒ श्चित्यां᳚चित्याम् । \newline
24. चित्यां᳚चित्या मृष॒भ मृ॑ष॒भम् चित्यां᳚चित्या॒म् चित्यां᳚चित्या मृष॒भम् । \newline
25. चित्यां᳚चित्या॒मिति॒ चित्यां᳚ - चि॒त्या॒म् । \newline
26. ऋ॒ष॒भ मुपोपा॑. र्‌ष॒भ मृ॑ष॒भ मुप॑ । \newline
27. उप॑ दधाति दधा॒ त्युपोप॑ दधाति । \newline
28. द॒धा॒ति॒ मि॒थु॒नम् मि॑थु॒नम् द॑धाति दधाति मिथु॒नम् । \newline
29. मि॒थु॒न मे॒वैव मि॑थु॒नम् मि॑थु॒न मे॒व । \newline
30. ए॒वास्या᳚ स्यै॒वै वास्य॑ । \newline
31. अ॒स्य॒ तत् तद॑ स्यास्य॒ तत् । \newline
32. तद् य॒ज्ञे य॒ज्ञे तत् तद् य॒ज्ञे । \newline
33. य॒ज्ञे क॑रोति करोति य॒ज्ञे य॒ज्ञे क॑रोति । \newline
34. क॒रो॒ति॒ प्र॒जन॑नाय प्र॒जन॑नाय करोति करोति प्र॒जन॑नाय । \newline
35. प्र॒जन॑नाय॒ तस्मा॒त् तस्मा᳚त् प्र॒जन॑नाय प्र॒जन॑नाय॒ तस्मा᳚त् । \newline
36. प्र॒जन॑ना॒येति॑ प्र - जन॑नाय । \newline
37. तस्मा᳚द् यू॒थेयू॑थे यू॒थेयू॑थे॒ तस्मा॒त् तस्मा᳚द् यू॒थेयू॑थे । \newline
38. यू॒थेयू॑थ ऋष॒भ ऋ॑ष॒भो यू॒थेयू॑थे यू॒थेयू॑थ ऋष॒भः । \newline
39. यू॒थेयू॑थ॒ इति॑ यू॒थे - यू॒थे॒ । \newline
40. ऋ॒ष॒भ इत्यृ॑ष॒भः । \newline
41. सं॒ॅव॒थ्स॒रस्य॑ प्रति॒माम् प्र॑ति॒माꣳ सं॑ॅवथ्स॒रस्य॑ संॅवथ्स॒रस्य॑ प्रति॒माम् । \newline
42. सं॒ॅव॒थ्स॒रस्येति॑ सं - व॒थ्स॒रस्य॑ । \newline
43. प्र॒ति॒मां ॅयां ॅयाम् प्र॑ति॒माम् प्र॑ति॒मां ॅयाम् । \newline
44. प्र॒ति॒मामिति॑ प्रति - माम् । \newline
45. याम् त्वा᳚ त्वा॒ यां ॅयाम् त्वा᳚ । \newline
46. त्वा॒ रा॒त्रि॒ रा॒त्रि॒ त्वा॒ त्वा॒ रा॒त्रि॒ । \newline
47. रा॒त्र्यु॒पास॑त उ॒पास॑ते रात्रि रात्र्यु॒पास॑ते । \newline
48. उ॒पास॑त॒ इत्यु॑प - आस॑ते । \newline
49. प्र॒जाꣳ सु॒वीराꣳ॑ सु॒वीरा᳚म् प्र॒जाम् प्र॒जाꣳ सु॒वीरा᳚म् । \newline
50. प्र॒जामिति॑ प्र - जाम् । \newline
51. सु॒वीरा᳚म् कृ॒त्वा कृ॒त्वा सु॒वीराꣳ॑ सु॒वीरा᳚म् कृ॒त्वा । \newline
52. सु॒वीरा॒मिति॑ सु - वीरा᳚म् । \newline
53. कृ॒त्वा विश्वं॒ ॅविश्व॑म् कृ॒त्वा कृ॒त्वा विश्व᳚म् । \newline
54. विश्व॒ मायु॒ रायु॒र् विश्वं॒ ॅविश्व॒ मायुः॑ । \newline
55. आयु॒र् वि व्यायु॒ रायु॒र् वि । \newline
56. व्य॑श्ञव दश्ञव॒द् वि व्य॑श्ञवत् । \newline
57. अ॒श्न॒व॒दित्य॑श्नवत् । \newline
58. प्रा॒जा॒प॒त्या मे॒ता मे॒ताम् प्रा॑जाप॒त्याम् प्रा॑जाप॒त्या मे॒ताम् । \newline
59. प्रा॒जा॒प॒त्यामिति॑ प्राजा - प॒त्याम् । \newline

\textbf{Ghana Paata } \newline

1. त्वा म॑ग्ने ऽग्ने॒ त्वाम् त्वा म॑ग्ने वृष॒भं ॅवृ॑ष॒भ म॑ग्ने॒ त्वाम् त्वा म॑ग्ने वृष॒भम् । \newline
2. अ॒ग्ने॒ वृ॒ष॒भं ॅवृ॑ष॒भ म॑ग्ने ऽग्ने वृष॒भम् चेकि॑तान॒म् चेकि॑तानं ॅवृष॒भ म॑ग्ने ऽग्ने वृष॒भम् चेकि॑तानम् । \newline
3. वृ॒ष॒भम् चेकि॑तान॒म् चेकि॑तानं ॅवृष॒भं ॅवृ॑ष॒भम् चेकि॑तान॒म् पुनः॒ पुन॒ श्चेकि॑तानं ॅवृष॒भं ॅवृ॑ष॒भम् चेकि॑तान॒म् पुनः॑ । \newline
4. चेकि॑तान॒म् पुनः॒ पुन॒श्चेकि॑तान॒म् चेकि॑तान॒म् पुन॒र् युवा॑नं॒ ॅयुवा॑न॒म् पुन॒श्चेकि॑तान॒म् चेकि॑तान॒म् पुन॒र् युवा॑नम् । \newline
5. पुन॒र् युवा॑नं॒ ॅयुवा॑न॒म् पुनः॒ पुन॒र् युवा॑नम् ज॒नय॑न् ज॒नय॒न्॒. युवा॑न॒म् पुनः॒ पुन॒र् युवा॑नम् ज॒नयन्न्॑ । \newline
6. युवा॑नम् ज॒नय॑न् ज॒नय॒न्॒. युवा॑नं॒ ॅयुवा॑नम् ज॒नय॑न् नु॒पागा॑ मु॒पागा᳚म् ज॒नय॒न्॒. युवा॑नं॒ ॅयुवा॑नम् ज॒नय॑न् नु॒पागा᳚म् । \newline
7. ज॒नय॑न् नु॒पागा॑ मु॒पागा᳚म् ज॒नय॑न् ज॒नय॑न् नु॒पागा᳚म् । \newline
8. उ॒पागा॒मित्यु॑प - आगा᳚म् । \newline
9. अ॒स्थू॒रि नो॑ नो ऽस्थू॒र्य॑स्थू॒रि नो॒ गार्.ह॑पत्यानि॒ गार्.ह॑पत्यानि नो ऽस्थू॒र्य॑स्थू॒रि नो॒ गार्.ह॑पत्यानि । \newline
10. नो॒ गार्.ह॑पत्यानि॒ गार्.ह॑पत्यानि नो नो॒ गार्.ह॑पत्यानि सन्तु सन्तु॒ गार्.ह॑पत्यानि नो नो॒ गार्.ह॑पत्यानि सन्तु । \newline
11. गार्.ह॑पत्यानि सन्तु सन्तु॒ गार्.ह॑पत्यानि॒ गार्.ह॑पत्यानि सन्तु ति॒ग्मेन॑ ति॒ग्मेन॑ सन्तु॒ गार्.ह॑पत्यानि॒ गार्.ह॑पत्यानि सन्तु ति॒ग्मेन॑ । \newline
12. गार्.ह॑पत्या॒नीति॒ गार्.ह॑ - प॒त्या॒नि॒ । \newline
13. स॒न्तु॒ ति॒ग्मेन॑ ति॒ग्मेन॑ सन्तु सन्तु ति॒ग्मेन॑ नो न स्ति॒ग्मेन॑ सन्तु सन्तु ति॒ग्मेन॑ नः । \newline
14. ति॒ग्मेन॑ नो न स्ति॒ग्मेन॑ ति॒ग्मेन॑ नो॒ ब्रह्म॑णा॒ ब्रह्म॑णा न स्ति॒ग्मेन॑ ति॒ग्मेन॑ नो॒ ब्रह्म॑णा । \newline
15. नो॒ ब्रह्म॑णा॒ ब्रह्म॑णा नो नो॒ ब्रह्म॑णा॒ सꣳ सम् ब्रह्म॑णा नो नो॒ ब्रह्म॑णा॒ सम् । \newline
16. ब्रह्म॑णा॒ सꣳ सम् ब्रह्म॑णा॒ ब्रह्म॑णा॒ सꣳ शि॑शाधि शिशाधि॒ सम् ब्रह्म॑णा॒ ब्रह्म॑णा॒ सꣳ शि॑शाधि । \newline
17. सꣳ शि॑शाधि शिशाधि॒ सꣳ सꣳ शि॑शाधि । \newline
18. शि॒शा॒धीति॑ शिशाधि । \newline
19. प॒शवो॒ वै वै प॒शवः॑ प॒शवो॒ वा ए॒त ए॒ते वै प॒शवः॑ प॒शवो॒ वा ए॒ते । \newline
20. वा ए॒त ए॒ते वै वा ए॒ते यद् यदे॒ते वै वा ए॒ते यत् । \newline
21. ए॒ते यद् यदे॒त ए॒ते यदिष्ट॑का॒ इष्ट॑का॒ यदे॒त ए॒ते यदिष्ट॑काः । \newline
22. यदिष्ट॑का॒ इष्ट॑का॒ यद् यदिष्ट॑का॒ श्चित्यां᳚चित्या॒म् चित्यां᳚चित्या॒ मिष्ट॑का॒ यद् यदिष्ट॑का॒ श्चित्यां᳚चित्याम् । \newline
23. इष्ट॑का॒ श्चित्यां᳚चित्या॒म् चित्यां᳚चित्या॒ मिष्ट॑का॒ इष्ट॑का॒ श्चित्यां᳚चित्या मृष॒भ मृ॑ष॒भम् चित्यां᳚चित्या॒ मिष्ट॑का॒ इष्ट॑का॒ श्चित्यां᳚चित्या मृष॒भम् । \newline
24. चित्यां᳚चित्या मृष॒भ मृ॑ष॒भम् चित्यां᳚चित्या॒म् चित्यां᳚चित्या मृष॒भ मुपोपा॑ र्.ष॒भम् चित्यां᳚चित्या॒म् चित्यां᳚चित्या मृष॒भ मुप॑ । \newline
25. चित्यां᳚चित्या॒मिति॒ चित्यां᳚ - चि॒त्या॒म् । \newline
26. ऋ॒ष॒भ मुपोपा॑ र्.ष॒भ मृ॑ष॒भ मुप॑ दधाति दधा॒ त्युपा॑ र्.ष॒भ मृ॑ष॒भ मुप॑ दधाति । \newline
27. उप॑ दधाति दधा॒ त्युपोप॑ दधाति मिथु॒नम् मि॑थु॒नम् द॑धा॒ त्युपोप॑ दधाति मिथु॒नम् । \newline
28. द॒धा॒ति॒ मि॒थु॒नम् मि॑थु॒नम् द॑धाति दधाति मिथु॒न मे॒वैव मि॑थु॒नम् द॑धाति दधाति मिथु॒न मे॒व । \newline
29. मि॒थु॒न मे॒वैव मि॑थु॒नम् मि॑थु॒न मे॒वास्या᳚ स्यै॒व मि॑थु॒नम् मि॑थु॒न मे॒वास्य॑ । \newline
30. ए॒वास्या᳚ स्यै॒वै वास्य॒ तत् तद॑स्यै॒ वैवास्य॒ तत् । \newline
31. अ॒स्य॒ तत् तद॑स्यास्य॒ तद् य॒ज्ञे य॒ज्ञे तद॑स्यास्य॒ तद् य॒ज्ञे । \newline
32. तद् य॒ज्ञे य॒ज्ञे तत् तद् य॒ज्ञे क॑रोति करोति य॒ज्ञे तत् तद् य॒ज्ञे क॑रोति । \newline
33. य॒ज्ञे क॑रोति करोति य॒ज्ञे य॒ज्ञे क॑रोति प्र॒जन॑नाय प्र॒जन॑नाय करोति य॒ज्ञे य॒ज्ञे क॑रोति प्र॒जन॑नाय । \newline
34. क॒रो॒ति॒ प्र॒जन॑नाय प्र॒जन॑नाय करोति करोति प्र॒जन॑नाय॒ तस्मा॒त् तस्मा᳚त् प्र॒जन॑नाय करोति करोति प्र॒जन॑नाय॒ तस्मा᳚त् । \newline
35. प्र॒जन॑नाय॒ तस्मा॒त् तस्मा᳚त् प्र॒जन॑नाय प्र॒जन॑नाय॒ तस्मा᳚द् यू॒थेयू॑थे यू॒थेयू॑थे॒ तस्मा᳚त् प्र॒जन॑नाय प्र॒जन॑नाय॒ तस्मा᳚द् यू॒थेयू॑थे । \newline
36. प्र॒जन॑ना॒येति॑ प्र - जन॑नाय । \newline
37. तस्मा᳚द् यू॒थेयू॑थे यू॒थेयू॑थे॒ तस्मा॒त् तस्मा᳚द् यू॒थेयू॑थ ऋष॒भ ऋ॑ष॒भो यू॒थेयू॑थे॒ तस्मा॒त् तस्मा᳚द् यू॒थेयू॑थ ऋष॒भः । \newline
38. यू॒थेयू॑थ ऋष॒भ ऋ॑ष॒भो यू॒थेयू॑थे यू॒थेयू॑थ ऋष॒भः । \newline
39. यू॒थेयू॑थ॒ इति॑ यू॒थे - यू॒थे॒ । \newline
40. ऋ॒ष॒भ इत्यृ॑ष॒भः । \newline
41. सं॒ॅव॒थ्स॒रस्य॑ प्रति॒माम् प्र॑ति॒माꣳ सं॑ॅवथ्स॒रस्य॑ संॅवथ्स॒रस्य॑ प्रति॒मां ॅयां ॅयाम् प्र॑ति॒माꣳ सं॑ॅवथ्स॒रस्य॑ संॅवथ्स॒रस्य॑ प्रति॒मां ॅयाम् । \newline
42. सं॒ॅव॒थ्स॒रस्येति॑ सं - व॒थ्स॒रस्य॑ । \newline
43. प्र॒ति॒मां ॅयां ॅयाम् प्र॑ति॒माम् प्र॑ति॒मां ॅयाम् त्वा᳚ त्वा॒ याम् प्र॑ति॒माम् प्र॑ति॒मां ॅयाम् त्वा᳚ । \newline
44. प्र॒ति॒मामिति॑ प्रति - माम् । \newline
45. याम् त्वा᳚ त्वा॒ यां ॅयाम् त्वा॑ रात्रि रात्रि त्वा॒ यां ॅयाम् त्वा॑ रात्रि । \newline
46. त्वा॒ रा॒त्रि॒ रा॒त्रि॒ त्वा॒ त्वा॒ रा॒त्र्यु॒पास॑त उ॒पास॑ते रात्रि त्वा त्वा रात्र्यु॒पास॑ते । \newline
47. रा॒त्र्यु॒पास॑त उ॒पास॑ते रात्रि रात्र्यु॒पास॑ते । \newline
48. उ॒पास॑त॒ इत्यु॑प - आस॑ते । \newline
49. प्र॒जाꣳ सु॒वीराꣳ॑ सु॒वीरा᳚म् प्र॒जाम् प्र॒जाꣳ सु॒वीरा᳚म् कृ॒त्वा कृ॒त्वा सु॒वीरा᳚म् प्र॒जाम् प्र॒जाꣳ सु॒वीरा᳚म् कृ॒त्वा । \newline
50. प्र॒जामिति॑ प्र - जाम् । \newline
51. सु॒वीरा᳚म् कृ॒त्वा कृ॒त्वा सु॒वीराꣳ॑ सु॒वीरा᳚म् कृ॒त्वा विश्वं॒ ॅविश्व॑म् कृ॒त्वा सु॒वीराꣳ॑ सु॒वीरा᳚म् कृ॒त्वा विश्व᳚म् । \newline
52. सु॒वीरा॒मिति॑ सु - वीरा᳚म् । \newline
53. कृ॒त्वा विश्वं॒ ॅविश्व॑म् कृ॒त्वा कृ॒त्वा विश्व॒ मायु॒ रायु॒र् विश्व॑म् कृ॒त्वा कृ॒त्वा विश्व॒ मायुः॑ । \newline
54. विश्व॒ मायु॒ रायु॒र् विश्वं॒ ॅविश्व॒ मायु॒र् वि व्यायु॒र् विश्वं॒ ॅविश्व॒ मायु॒र् वि । \newline
55. आयु॒र् वि व्यायु॒ रायु॒र् व्य॑श्ञव दश्ञव॒द् व्यायु॒ रायु॒र् व्य॑श्ञवत् । \newline
56. व्य॑श्ञव दश्ञव॒द् वि व्य॑श्ञवत् । \newline
57. अ॒श्न॒व॒दित्य॑श्नवत् । \newline
58. प्रा॒जा॒प॒त्या मे॒ता मे॒ताम् प्रा॑जाप॒त्याम् प्रा॑जाप॒त्या मे॒ता मुपोपै॒ताम् प्रा॑जाप॒त्याम् प्रा॑जाप॒त्या मे॒ता मुप॑ । \newline
59. प्रा॒जा॒प॒त्यामिति॑ प्राजा - प॒त्याम् । \newline
\pagebreak
\markright{ TS 5.7.2.2  \hfill https://www.vedavms.in \hfill}

\section{ TS 5.7.2.2 }

\textbf{TS 5.7.2.2 } \newline
\textbf{Samhita Paata} \newline

-मे॒तामुप॑ दधाती॒यं ॅवा वैषैका᳚ष्ट॒का यदे॒वैका᳚ष्ट॒काया॒मन्नं॑ क्रि॒यते॒ तदे॒वैतयाव॑ रुन्ध ए॒षा वै प्र॒जाप॑तेः काम॒दुघा॒ तयै॒व यज॑मानो॒ऽमुष्मि॑न् ॅलो॒के᳚ऽग्निं दु॑हे॒ येन॑ दे॒वा ज्योति॑षो॒र्द्ध्वा उ॒दाय॒न्॒ येना॑ऽऽदि॒त्या वस॑वो॒ येन॑ रु॒द्राः । येनाङ्गि॑रसो महि॒मान॑-मान॒शुस्तेनै॑तु॒ यज॑मानः स्व॒स्ति ॥ सु॒व॒र्गाय॒ वा ए॒ष लो॒काय॑ - [  ] \newline

\textbf{Pada Paata} \newline

ए॒ताम् । उपेति॑ । द॒धा॒ति॒ । इ॒यम् । वाव । ए॒षा । ए॒का॒ष्ट॒केत्ये॑क - अ॒ष्ट॒का । यत् । ए॒व । ए॒का॒ष्ट॒काया॒मित्ये॑क - अ॒ष्ट॒काया᳚म् । अन्न᳚म् । क्रि॒यते᳚ । तत् । ए॒व । ए॒तया᳚ । अवेति॑ । रु॒न्धे॒ । ए॒षा । वै । प्र॒जाप॑ते॒रिति॑ प्र॒जा - प॒तेः॒ । का॒म॒दुघेति॑ काम - दुघा᳚ । तया᳚ । ए॒व । यज॑मानः । अ॒मुष्मिन्न्॑ । लो॒के । अ॒ग्निम् । दु॒हे॒ । येन॑ । दे॒वाः । ज्योति॑षा । ऊ॒द्‌र्ध्वाः । उ॒दाय॒न्नित्यु॑त् - आयन्न्॑ । येन॑ । आ॒दि॒त्याः । वस॑वः । येन॑ । रु॒द्राः ॥ येन॑ । अङ्गि॑रसः । म॒हि॒मान᳚म् । आ॒न॒शुः । तेन॑ । ए॒तु॒ । यज॑मानः । स्व॒स्ति ॥ सु॒व॒र्गायेति॑ सुवः - गाय॑ । वै । ए॒षः । लो॒काय॑ ।  \newline


\textbf{Krama Paata} \newline

ए॒तामुप॑ । उप॑ दधाति । द॒धा॒ती॒यम् । इ॒यम् ॅवाव । वावैषा । ए॒षैका᳚ष्ट॒का । ए॒का॒ष्ट॒का यत् । ए॒का॒ष्ट॒केत्ये॑क - अ॒ष्ट॒का । यदे॒व । ए॒वैका᳚ष्ट॒काया᳚म् । ए॒का॒ष्ट॒काया॒मन्न᳚म् । ए॒का॒ष्ट॒काया॒मित्ये॑क - अ॒ष्ट॒काया᳚म् । अन्न॑म् क्रि॒यते᳚ । क्रि॒यते॒ तत् । तदे॒व । ए॒वैतया᳚ । ए॒तयाऽव॑ । अव॑ रुन्धे । रु॒न्ध॒ ए॒षा । ए॒षा वै । वै प्र॒जाप॑तेः । प्र॒जाप॑तेः काम॒दुघा᳚ । प्र॒जाप॑ते॒रिति॑ प्र॒जा - प॒तेः॒ । का॒म॒दुघा॒ तया᳚ । का॒म॒दुघेति॑ काम - दुघा᳚ । तयै॒व । ए॒व यज॑मानः । यज॑मानो॒ऽमुष्मिन्न्॑ । अ॒मुष्मि॑न् ॅलो॒के । लो॒के᳚ऽग्निम् । अ॒ग्निम् दु॑हे । दु॒हे॒ येन॑ । येन॑ दे॒वाः । दे॒वा ज्योति॑षा । ज्योति॑षो॒र्द्ध्वाः । ऊ॒र्द्ध्वा उ॒दायन्न्॑ । उ॒दाय॒न्न्॒. येन॑ । उ॒दाय॒न्नित्यु॑त् - आयन्न्॑ । येना॑दि॒त्याः । आ॒दि॒त्या वस॑वः । वस॑वो॒ येन॑ । येन॑ रु॒द्राः । रु॒द्रा इति॑ रु॒द्राः ॥ येनाङ्गि॑रसः । अङ्गि॑रसो महि॒मान᳚म् । म॒हि॒मान॑मान॒शुः । आ॒न॒शुस्तेन॑ । तेनै॑तु । ए॒तु॒ यज॑मानः । यज॑मानः स्व॒स्ति । स्व॒स्तीति॑ स्व॒स्ति ॥ सु॒र्व॒गाय॒ वै । सु॒व॒र्गायेति॑ सुवः - गाय॑ । वा ए॒षः । ए॒ष लो॒काय॑ । लो॒काय॑ चीयते \newline

\textbf{Jatai Paata} \newline

1. ए॒ता मुपो पै॒ता मे॒ता मुप॑ । \newline
2. उप॑ दधाति दधा॒ त्युपोप॑ दधाति । \newline
3. द॒धा॒ ती॒य मि॒यम् द॑धाति दधा ती॒यम् । \newline
4. इ॒यं ॅवाव वावे य मि॒यं ॅवाव । \newline
5. वावैषैषा वाव वावैषा । \newline
6. ए॒षै का᳚ष्ट॒ कैका᳚ष्ट॒ कैषैषै का᳚ष्ट॒का । \newline
7. ए॒का॒ष्ट॒का यद् यदे॑काष्ट॒ कैका᳚ष्ट॒का यत् । \newline
8. ए॒का॒ष्ट॒केत्ये॑क - अ॒ष्ट॒का । \newline
9. यदे॒ वैव यद् यदे॒व । \newline
10. ए॒व इका᳚ष्ट॒काया॑ मेकाष्ट॒काया॑ मे॒वैवै का᳚ष्ट॒काया᳚म् । \newline
11. ए॒का॒ष्ट॒काया॒ मन्न॒ मन्न॑ मेकाष्ट॒काया॑ मेकाष्ट॒काया॒ मन्न᳚म् । \newline
12. ए॒का॒ष्ट॒काया॒मित्ये॑क - अ॒ष्ट॒काया᳚म् । \newline
13. अन्न॑म् क्रि॒यते᳚ क्रि॒यते ऽन्न॒ मन्न॑म् क्रि॒यते᳚ । \newline
14. क्रि॒यते॒ तत् तत् क्रि॒यते᳚ क्रि॒यते॒ तत् । \newline
15. तदे॒ वैव तत् तदे॒व । \newline
16. ए॒वै तयै॒ तयै॒ वैवै तया᳚ । \newline
17. ए॒तया ऽवावै॒ तयै॒ तया ऽव॑ । \newline
18. अव॑ रुन्धे रु॒न्धे ऽवाव॑ रुन्धे । \newline
19. रु॒न्ध॒ ए॒षैषा रु॑न्धे रुन्ध ए॒षा । \newline
20. ए॒षा वै वा ए॒षैषा वै । \newline
21. वै प्र॒जाप॑तेः प्र॒जाप॑ते॒र् वै वै प्र॒जाप॑तेः । \newline
22. प्र॒जाप॑तेः काम॒दुघा॑ काम॒दुघा᳚ प्र॒जाप॑तेः प्र॒जाप॑तेः काम॒दुघा᳚ । \newline
23. प्र॒जाप॑ते॒रिति॑ प्र॒जा - प॒तेः॒ । \newline
24. का॒म॒दुघा॒ तया॒ तया॑ काम॒दुघा॑ काम॒दुघा॒ तया᳚ । \newline
25. का॒म॒दुघेति॑ काम - दुघा᳚ । \newline
26. तयै॒ वैव तया॒ तयै॒व । \newline
27. ए॒व यज॑मानो॒ यज॑मान ए॒वैव यज॑मानः । \newline
28. यज॑मानो॒ ऽमुष्मि॑न् न॒मुष्मि॒न्॒. यज॑मानो॒ यज॑मानो॒ ऽमुष्मिन्न्॑ । \newline
29. अ॒मुष्मि॑न् ॅलो॒के लो॒के॑ ऽमुष्मि॑न् न॒मुष्मि॑न् ॅलो॒के । \newline
30. लो॒के᳚ ऽग्नि म॒ग्निम् ॅलो॒के लो॒के᳚ ऽग्निम् । \newline
31. अ॒ग्निम् दु॑हे दुहे॒ ऽग्नि म॒ग्निम् दु॑हे । \newline
32. दु॒हे॒ येन॒ येन॑ दुहे दुहे॒ येन॑ । \newline
33. येन॑ दे॒वा दे॒वा येन॒ येन॑ दे॒वाः । \newline
34. दे॒वा ज्योति॑षा॒ ज्योति॑षा दे॒वा दे॒वा ज्योति॑षा । \newline
35. ज्योति॑ षो॒र्द्ध्वा ऊ॒र्द्ध्वा ज्योति॑षा॒ ज्योति॑ षो॒र्द्ध्वाः । \newline
36. ऊ॒र्द्ध्वा उ॒दाय॑न् नु॒दाय॑न् नू॒र्द्ध्वा ऊ॒र्द्ध्वा उ॒दायन्न्॑ । \newline
37. उ॒दाय॒न्॒. येन॒ येनो॒दाय॑न् नु॒दाय॒न्॒. येन॑ । \newline
38. उ॒दाय॒न्नित्यु॑त् - आयन्न्॑ । \newline
39. येना॑दि॒त्या आ॑दि॒त्या येन॒ येना॑दि॒त्याः । \newline
40. आ॒दि॒त्या वस॑वो॒ वस॑व आदि॒त्या आ॑दि॒त्या वस॑वः । \newline
41. वस॑वो॒ येन॒ येन॒ वस॑वो॒ वस॑वो॒ येन॑ । \newline
42. येन॑ रु॒द्रा रु॒द्रा येन॒ येन॑ रु॒द्राः । \newline
43. रु॒द्रा इति॑ रु॒द्राः । \newline
44. येनाङ्गि॑र॒सो ऽङ्गि॑रसो॒ येन॒ येनाङ्गि॑रसः । \newline
45. अङ्गि॑रसो महि॒मान॑म् महि॒मान॒ मङ्गि॑र॒सो ऽङ्गि॑रसो महि॒मान᳚म् । \newline
46. म॒हि॒मान॑ मान॒शु रा॑न॒शुर् म॑हि॒मान॑म् महि॒मान॑ मान॒शुः । \newline
47. आ॒न॒शु स्तेन॒ तेना॑ न॒शु रा॑न॒शु स्तेन॑ । \newline
48. तेनै᳚ त्वेतु॒ तेन॒ तेनै॑तु । \newline
49. ए॒तु॒ यज॑मानो॒ यज॑मान एत्वेतु॒ यज॑मानः । \newline
50. यज॑मानः स्व॒स्ति स्व॒स्ति यज॑मानो॒ यज॑मानः स्व॒स्ति । \newline
51. स्व॒स्तीति॑ स्व॒स्ति । \newline
52. सु॒व॒र्गाय॒ वै वै सु॑व॒र्गाय॑ सुव॒र्गाय॒ वै । \newline
53. सु॒व॒र्गायेति॑ सुवः - गाय॑ । \newline
54. वा ए॒ष ए॒ष वै वा ए॒षः । \newline
55. ए॒ष लो॒काय॑ लो॒कायै॒ष ए॒ष लो॒काय॑ । \newline
56. लो॒काय॑ चीयते चीयते लो॒काय॑ लो॒काय॑ चीयते । \newline

\textbf{Ghana Paata } \newline

1. ए॒ता मुपोपै॒ता मे॒ता मुप॑ दधाति दधा॒ त्युपै॒ता मे॒ता मुप॑ दधाति । \newline
2. उप॑ दधाति दधा॒ त्युपोप॑ दधाती॒य मि॒यम् द॑धा॒ त्युपोप॑ दधाती॒यम् । \newline
3. द॒धा॒ती॒य मि॒यम् द॑धाति दधाती॒यं ॅवाव वावेयम् द॑धाति दधाती॒यं ॅवाव । \newline
4. इ॒यं ॅवाव वावेय मि॒यं ॅवावै षैषा वावेय मि॒यं ॅवावैषा । \newline
5. वावै षैषा वाव वावैषै का᳚ष्ट॒ कैका᳚ष्ट॒कैषा वाव वावैषै का᳚ष्ट॒का । \newline
6. ए॒षै का᳚ष्ट॒ कैका᳚ष्ट॒ कैषै षैका᳚ष्ट॒का यद् यदे॑काष्ट॒ कैषै षैका᳚ष्ट॒का यत् । \newline
7. ए॒का॒ष्ट॒का यद् यदे॑काष्ट॒ कैका᳚ष्ट॒का यदे॒वैव यदे॑काष्ट॒ कैका᳚ष्ट॒का यदे॒व । \newline
8. ए॒का॒ष्ट॒केत्ये॑क - अ॒ष्ट॒का । \newline
9. यदे॒वैव यद् यदे॒ वैका᳚ष्ट॒काया॑ मेकाष्ट॒काया॑ मे॒व यद् यदे॒ वैका᳚ष्ट॒काया᳚म् । \newline
10. ए॒वैका᳚ष्ट॒काया॑ मेकाष्ट॒काया॑ मे॒वैवै का᳚ष्ट॒काया॒ मन्न॒ मन्न॑ मेकाष्ट॒काया॑ मे॒वैवै का᳚ष्ट॒काया॒ मन्न᳚म् । \newline
11. ए॒का॒ष्ट॒काया॒ मन्न॒ मन्न॑ मेकाष्ट॒काया॑ मेकाष्ट॒काया॒ मन्न॑म् क्रि॒यते᳚ क्रि॒यते ऽन्न॑ मेकाष्ट॒काया॑ मेकाष्ट॒काया॒ मन्न॑म् क्रि॒यते᳚ । \newline
12. ए॒का॒ष्ट॒काया॒मित्ये॑क - अ॒ष्ट॒काया᳚म् । \newline
13. अन्न॑म् क्रि॒यते᳚ क्रि॒यते ऽन्न॒ मन्न॑म् क्रि॒यते॒ तत् तत् क्रि॒यते ऽन्न॒ मन्न॑म् क्रि॒यते॒ तत् । \newline
14. क्रि॒यते॒ तत् तत् क्रि॒यते᳚ क्रि॒यते॒ तदे॒वैव तत् क्रि॒यते᳚ क्रि॒यते॒ तदे॒व । \newline
15. तदे॒ वैव तत् तदे॒वैत यै॒त यै॒व तत् तदे॒वैतया᳚ । \newline
16. ए॒वैत यै॒त यै॒वै वैतया ऽवावै॒त यै॒वै वैतया ऽव॑ । \newline
17. ए॒तया ऽवावै॒त यै॒तया ऽव॑ रुन्धे रु॒न्धे ऽवै॒त यै॒तया ऽव॑ रुन्धे । \newline
18. अव॑ रुन्धे रु॒न्धे ऽवाव॑ रुन्ध ए॒षैषा रु॒न्धे ऽवाव॑ रुन्ध ए॒षा । \newline
19. रु॒न्ध॒ ए॒षैषा रु॑न्धे रुन्ध ए॒षा वै वा ए॒षा रु॑न्धे रुन्ध ए॒षा वै । \newline
20. ए॒षा वै वा ए॒षैषा वै प्र॒जाप॑तेः प्र॒जाप॑ते॒र् वा ए॒षैषा वै प्र॒जाप॑तेः । \newline
21. वै प्र॒जाप॑तेः प्र॒जाप॑ते॒र् वै वै प्र॒जाप॑तेः काम॒दुघा॑ काम॒दुघा᳚ प्र॒जाप॑ते॒र् वै वै प्र॒जाप॑तेः काम॒दुघा᳚ । \newline
22. प्र॒जाप॑तेः काम॒दुघा॑ काम॒दुघा᳚ प्र॒जाप॑तेः प्र॒जाप॑तेः काम॒दुघा॒ तया॒ तया॑ काम॒दुघा᳚ प्र॒जाप॑तेः प्र॒जाप॑तेः काम॒दुघा॒ तया᳚ । \newline
23. प्र॒जाप॑ते॒रिति॑ प्र॒जा - प॒तेः॒ । \newline
24. का॒म॒दुघा॒ तया॒ तया॑ काम॒दुघा॑ काम॒दुघा॒ तयै॒ वैव तया॑ काम॒दुघा॑ काम॒दुघा॒ तयै॒व । \newline
25. का॒म॒दुघेति॑ काम - दुघा᳚ । \newline
26. तयै॒वैव तया॒ तयै॒व यज॑मानो॒ यज॑मान ए॒व तया॒ तयै॒व यज॑मानः । \newline
27. ए॒व यज॑मानो॒ यज॑मान ए॒वैव यज॑मानो॒ ऽमुष्मि॑न् न॒मुष्मि॒न्॒. यज॑मान ए॒वैव यज॑मानो॒ ऽमुष्मिन्न्॑ । \newline
28. यज॑मानो॒ ऽमुष्मि॑न् न॒मुष्मि॒न्॒. यज॑मानो॒ यज॑मानो॒ ऽमुष्मि॑न् ॅलो॒के लो॒के॑ ऽमुष्मि॒न्॒. यज॑मानो॒ यज॑मानो॒ ऽमुष्मि॑न् ॅलो॒के । \newline
29. अ॒मुष्मि॑न् ॅलो॒के लो॒के॑ ऽमुष्मि॑न् न॒मुष्मि॑न् ॅलो॒के᳚ ऽग्नि म॒ग्निम् ॅलो॒के॑ ऽमुष्मि॑न् न॒मुष्मि॑न् ॅलो॒के᳚ ऽग्निम् । \newline
30. लो॒के᳚ ऽग्नि म॒ग्निम् ॅलो॒के लो॒के᳚ ऽग्निम् दु॑हे दुहे॒ ऽग्निम् ॅलो॒के लो॒के᳚ ऽग्निम् दु॑हे । \newline
31. अ॒ग्निम् दु॑हे दुहे॒ ऽग्नि म॒ग्निम् दु॑हे॒ येन॒ येन॑ दुहे॒ ऽग्नि म॒ग्निम् दु॑हे॒ येन॑ । \newline
32. दु॒हे॒ येन॒ येन॑ दुहे दुहे॒ येन॑ दे॒वा दे॒वा येन॑ दुहे दुहे॒ येन॑ दे॒वाः । \newline
33. येन॑ दे॒वा दे॒वा येन॒ येन॑ दे॒वा ज्योति॑षा॒ ज्योति॑षा दे॒वा येन॒ येन॑ दे॒वा ज्योति॑षा । \newline
34. दे॒वा ज्योति॑षा॒ ज्योति॑षा दे॒वा दे॒वा ज्योति॑षो॒र्द्ध्वा ऊ॒र्द्ध्वा ज्योति॑षा दे॒वा दे॒वा ज्योति॑षो॒र्द्ध्वाः । \newline
35. ज्योति॑षो॒र्द्ध्वा ऊ॒र्द्ध्वा ज्योति॑षा॒ ज्योति॑षो॒र्द्ध्वा उ॒दाय॑न् नु॒दाय॑न् नू॒र्द्ध्वा ज्योति॑षा॒ ज्योति॑षो॒र्द्ध्वा उ॒दायन्न्॑ । \newline
36. ऊ॒र्द्ध्वा उ॒दाय॑न् नु॒दाय॑न् नू॒र्द्ध्वा ऊ॒र्द्ध्वा उ॒दाय॒न्॒. येन॒ येनो॒दाय॑न् नू॒र्द्ध्वा ऊ॒र्द्ध्वा उ॒दाय॒न्॒. येन॑ । \newline
37. उ॒दाय॒न्॒. येन॒ येनो॒दाय॑न् नु॒दाय॒न्॒. येना॑दि॒त्या आ॑दि॒त्या येनो॒दाय॑न् नु॒दाय॒न्॒. येना॑दि॒त्याः । \newline
38. उ॒दाय॒न्नित्यु॑त् - आयन्न्॑ । \newline
39. येना॑दि॒त्या आ॑दि॒त्या येन॒ येना॑दि॒त्या वस॑वो॒ वस॑व आदि॒त्या येन॒ येना॑दि॒त्या वस॑वः । \newline
40. आ॒दि॒त्या वस॑वो॒ वस॑व आदि॒त्या आ॑दि॒त्या वस॑वो॒ येन॒ येन॒ वस॑व आदि॒त्या आ॑दि॒त्या वस॑वो॒ येन॑ । \newline
41. वस॑वो॒ येन॒ येन॒ वस॑वो॒ वस॑वो॒ येन॑ रु॒द्रा रु॒द्रा येन॒ वस॑वो॒ वस॑वो॒ येन॑ रु॒द्राः । \newline
42. येन॑ रु॒द्रा रु॒द्रा येन॒ येन॑ रु॒द्राः । \newline
43. रु॒द्रा इति॑ रु॒द्राः । \newline
44. येनाङ्गि॑र॒सो ऽङ्गि॑रसो॒ येन॒ येनाङ्गि॑रसो महि॒मान॑म् महि॒मान॒ मङ्गि॑रसो॒ येन॒ येनाङ्गि॑रसो महि॒मान᳚म् । \newline
45. अङ्गि॑रसो महि॒मान॑म् महि॒मान॒ मङ्गि॑र॒सो ऽङ्गि॑रसो महि॒मान॑ मान॒शु रा॑न॒शुर् म॑हि॒मान॒ मङ्गि॑र॒सो ऽङ्गि॑रसो महि॒मान॑ मान॒शुः । \newline
46. म॒हि॒मान॑ मान॒शु रा॑न॒शुर् म॑हि॒मान॑म् महि॒मान॑ मान॒शु स्तेन॒ तेना॑न॒शुर् म॑हि॒मान॑म् महि॒मान॑ मान॒शु स्तेन॑ । \newline
47. आ॒न॒शु स्तेन॒ तेना॑ न॒शु रा॑न॒शु स्तेनै᳚त्वेतु॒ तेना॑ न॒शु रा॑न॒शु स्तेनै॑तु । \newline
48. तेनै᳚त्वेतु॒ तेन॒ तेनै॑तु॒ यज॑मानो॒ यज॑मान एतु॒ तेन॒ तेनै॑तु॒ यज॑मानः । \newline
49. ए॒तु॒ यज॑मानो॒ यज॑मान एत्वेतु॒ यज॑मानः स्व॒स्ति स्व॒स्ति यज॑मान एत्वेतु॒ यज॑मानः स्व॒स्ति । \newline
50. यज॑मानः स्व॒स्ति स्व॒स्ति यज॑मानो॒ यज॑मानः स्व॒स्ति । \newline
51. स्व॒स्तीति॑ स्व॒स्ति । \newline
52. सु॒व॒र्गाय॒ वै वै सु॑व॒र्गाय॑ सुव॒र्गाय॒ वा ए॒ष ए॒ष वै सु॑व॒र्गाय॑ सुव॒र्गाय॒ वा ए॒षः । \newline
53. सु॒व॒र्गायेति॑ सुवः - गाय॑ । \newline
54. वा ए॒ष ए॒ष वै वा ए॒ष लो॒काय॑ लो॒कायै॒ष वै वा ए॒ष लो॒काय॑ । \newline
55. ए॒ष लो॒काय॑ लो॒कायै॒ष ए॒ष लो॒काय॑ चीयते चीयते लो॒कायै॒ष ए॒ष लो॒काय॑ चीयते । \newline
56. लो॒काय॑ चीयते चीयते लो॒काय॑ लो॒काय॑ चीयते॒ यद् यच् ची॑यते लो॒काय॑ लो॒काय॑ चीयते॒ यत् । \newline
\pagebreak
\markright{ TS 5.7.2.3  \hfill https://www.vedavms.in \hfill}

\section{ TS 5.7.2.3 }

\textbf{TS 5.7.2.3 } \newline
\textbf{Samhita Paata} \newline

चीयते॒ यद॒ग्निर्येन॑ दे॒वा ज्योति॑षो॒र्द्ध्वा उ॒दाय॒न्नित्युख्यꣳ॒॒ समि॑न्ध॒  इष्ट॑का ए॒वैता उप॑ धत्ते वानस्प॒त्याः सु॑व॒र्गस्य॑ लो॒कस्य॒ सम॑ष्ट्यै श॒तायु॑धाय श॒तवी᳚र्याय श॒तोत॑ये ऽभिमाति॒षाहे᳚ । श॒तं ॅयो नः॑ श॒रदो॒ अजी॑ता॒निन्द्रो॑ नेष॒दति॑ दुरि॒तानि॒ विश्वा᳚ ॥ये च॒त्वारः॑ प॒थयो॑ देव॒याना॑ अन्त॒रा द्यावा॑पृथि॒वी वि॒यन्ति॑ ।तेषां॒ ॅयो अज्या॑नि॒- मजी॑ति-मा॒वहा॒त् तस्मै॑ नो देवाः॒ - [  ] \newline

\textbf{Pada Paata} \newline

ची॒य॒ते॒ । यत् । अ॒ग्निः । येन॑ । दे॒वाः । ज्योति॑षा । उ॒द्‌र्ध्वाः । उ॒दाय॒न्नित्यु॑त्-आयन्न्॑ । इति॑ । उख्य᳚म् । समिति॑ । इ॒न्धे॒ । इष्ट॑काः । ए॒व । ए॒ताः । उपेति॑ । ध॒त्ते॒ । वा॒न॒स्प॒त्याः । सु॒व॒र्गस्येति॑ सुवः - गस्य॑ । लो॒कस्य॑ । सम॑ष्ट्या॒ इति॒ सं - अ॒ष्ट्यै॒ । श॒तायु॑धा॒येति॑ श॒त - आ॒यु॒धा॒य॒ । श॒तवी᳚र्या॒येति॑ श॒त - वी॒र्या॒य॒ । श॒तोत॑य॒ इति॑ श॒त - ऊ॒त॒ये॒ । अ॒भि॒मा॒ति॒षाह॒ इत्य॑भिमाति -   साहे᳚ ॥ श॒तम् । यः । नः॒ । श॒रदः॑ । अजी॑तान् । इन्द्रः॑ । ने॒ष॒त् । अतीति॑ । दु॒रि॒तानीति॑ दुः - इ॒तानि॑ । विश्वा᳚ ॥ ये । च॒त्वारः॑ । प॒थयः॑ । दे॒व॒याना॒ इति॑ देव - यानाः᳚ । अ॒न्त॒रा । द्यावा॑पृथि॒वी इति॒ द्यावा᳚ - पृ॒थि॒वी । वि॒यन्तीति॑ वि - यन्ति॑ ॥ तेषा᳚म् । यः । अज्या॑निम् । अजी॑तिम् । आ॒वहा॒दित्या᳚ - वहा᳚त् । तस्मै᳚ । नः॒ । दे॒वाः॒ ।  \newline


\textbf{Krama Paata} \newline

ची॒य॒ते॒ यत् । यद॒ग्निः । अ॒ग्निर् येन॑ । येन॑ दे॒वाः । दे॒वा ज्योति॑षा । ज्योति॑षो॒र्द्ध्वाः । ऊ॒र्द्ध्वा उ॒दायन्न्॑ । उ॒दाय॒न्निति॑ । उ॒दाय॒न्नित्यु॑त् - आयन्न्॑ । इत्युख्य᳚म् । उख्यꣳ॒॒ सम् । समि॑न्धे । इ॒न्ध॒ इष्ट॑काः । इष्ट॑का ए॒व । ए॒वैताः । ए॒ता उप॑ । उप॑ धत्ते । ध॒त्ते॒ वा॒न॒स्प॒त्याः । वा॒न॒स्प॒त्याः सु॑व॒र्गस्य॑ । सु॒व॒र्गस्य॑ लो॒कस्य॑ । सु॒व॒र्गस्येति॑ सुवः - गस्य॑ । लो॒कस्य॒ सम॑ष्ट्यै । सम॑ष्ट्यै श॒तायु॑धाय । सम॑ष्ट्या॒ इति॒ सम् - अ॒ष्ट्यै॒ । श॒तायु॑धाय श॒तवी᳚र्याय । श॒तायु॑धा॒येति॑ श॒त - आ॒यु॒धा॒य॒ । श॒तवी᳚र्याय श॒तोत॑ये । श॒तवी᳚र्या॒येति॑ श॒त - वी॒र्या॒य॒ । श॒तोत॑येऽभिमाति॒षाहे᳚ । श॒तोत॑य॒ इति॑ श॒त - ऊ॒त॒ये॒ । अ॒भि॒मा॒ति॒षाह॒ इत्य॑भिमाति - साहे᳚ ॥ श॒तम् ॅयः । यो नः॑ । नः॒ श॒रदः॑ । श॒रदो॒ अजी॑तान् । अजी॑ता॒निन्द्रः॑ । इन्द्रो॑ नेषत् । ने॒ष॒दति॑ । अति॑ दुरि॒तानि॑ । दु॒रि॒तानि॒ विश्वा᳚ । दु॒रि॒तानीति॑ दुः - इ॒तानि॑ । विश्वेति॒ विश्वा᳚ ॥ ये च॒त्वारः॑ । च॒त्वारः॑ प॒थयः॑ । प॒थयो॑ देव॒यानाः᳚ । दे॒व॒याना॑ अन्त॒रा । दे॒व॒याना॒ इति॑ देव - यानाः᳚ । अ॒न्त॒रा द्यावा॑पृथि॒वी । द्यावा॑पृथि॒वी वि॒यन्ति॑ । द्यावा॑पृथि॒वी इति॒ द्यावा᳚ - पृ॒थि॒वी । वि॒यन्तीति॑ वि - यन्ति॑ ॥ तेषा॒म् ॅयः । यो अज्या॑निम् । अज्या॑नि॒मजी॑तिम् । अजी॑तिमा॒वहा᳚त् । आ॒वहा॒त् तस्मै᳚ । आ॒वहा॒दित्या᳚ - वहा᳚त् । तस्मै॑ नः । नो॒ दे॒वाः॒ । दे॒वाः॒ परि॑ \newline

\textbf{Jatai Paata} \newline

1. ची॒य॒ते॒ यद् यच् ची॑यते चीयते॒ यत् । \newline
2. यद॒ग्नि र॒ग्निर् यद् यद॒ग्निः । \newline
3. अ॒ग्निर् येन॒ येना॒ग्नि र॒ग्निर् येन॑ । \newline
4. येन॑ दे॒वा दे॒वा येन॒ येन॑ दे॒वाः । \newline
5. दे॒वा ज्योति॑षा॒ ज्योति॑षा दे॒वा दे॒वा ज्योति॑षा । \newline
6. ज्योति॑ षो॒र्द्ध्वा उ॒र्द्ध्वा ज्योति॑षा॒ ज्योति॑ षो॒र्द्ध्वाः । \newline
7. उ॒र्द्ध्वा उ॒दाय॑न् नु॒दाय॑न् नु॒र्द्ध्वा उ॒र्द्ध्वा उ॒दायन्न्॑ । \newline
8. उ॒दाय॒न् निती त्यु॒दाय॑न् नु॒दाय॒न् निति॑ । \newline
9. उ॒दाय॒न्नित्यु॑त् - आयन्न्॑ । \newline
10. इत्युख्य॒ मुख्य॒ मिती त्युख्य᳚म् । \newline
11. उख्यꣳ॒॒ सꣳ स मुख्य॒ मुख्यꣳ॒॒ सम् । \newline
12. स मि॑न्ध इन्धे॒ सꣳ स मि॑न्धे । \newline
13. इ॒न्ध॒ इष्ट॑का॒ इष्ट॑का इन्ध इन्ध॒ इष्ट॑काः । \newline
14. इष्ट॑का ए॒वैवेष्ट॑का॒ इष्ट॑का ए॒व । \newline
15. ए॒वैता ए॒ता ए॒वै वैताः । \newline
16. ए॒ता उपोपै॒ता ए॒ता उप॑ । \newline
17. उप॑ धत्ते धत्त॒ उपोप॑ धत्ते । \newline
18. ध॒त्ते॒ वा॒न॒स्प॒त्या वा॑नस्प॒त्या ध॑त्ते धत्ते वानस्प॒त्याः । \newline
19. वा॒न॒स्प॒त्याः सु॑व॒र्गस्य॑ सुव॒र्गस्य॑ वानस्प॒त्या वा॑नस्प॒त्याः सु॑व॒र्गस्य॑ । \newline
20. सु॒व॒र्गस्य॑ लो॒कस्य॑ लो॒कस्य॑ सुव॒र्गस्य॑ सुव॒र्गस्य॑ लो॒कस्य॑ । \newline
21. सु॒व॒र्गस्येति॑ सुवः - गस्य॑ । \newline
22. लो॒कस्य॒ सम॑ष्ट्यै॒ सम॑ष्ट्यै लो॒कस्य॑ लो॒कस्य॒ सम॑ष्ट्यै । \newline
23. सम॑ष्ट्यै श॒तायु॑धाय श॒तायु॑धाय॒ सम॑ष्ट्यै॒ सम॑ष्ट्यै श॒तायु॑धाय । \newline
24. सम॑ष्ट्या॒ इति॒ सं - अ॒ष्ट्यै॒ । \newline
25. श॒तायु॑धाय श॒तवी᳚र्याय श॒तवी᳚र्याय श॒तायु॑धाय श॒तायु॑धाय श॒तवी᳚र्याय । \newline
26. श॒तायु॑धा॒येति॑ श॒त - आ॒यु॒धा॒य॒ । \newline
27. श॒तवी᳚र्याय श॒तोत॑ये श॒तोत॑ये श॒तवी᳚र्याय श॒तवी᳚र्याय श॒तोत॑ये । \newline
28. श॒तवी᳚र्या॒येति॑ श॒त - वी॒र्या॒य॒ । \newline
29. श॒तोत॑ये ऽभिमाति॒षाहे॑ ऽभिमाति॒षाहे॑ श॒तोत॑ये श॒तोत॑ये ऽभिमाति॒षाहे᳚ । \newline
30. श॒तोत॑य॒ इति॑ श॒त - ऊ॒त॒ये॒ । \newline
31. अ॒भि॒मा॒ति॒षाह॒ इत्य॑भिमाति - साहे᳚ । \newline
32. श॒तं ॅयो यः श॒तꣳ श॒तं ॅयः । \newline
33. यो नो॑ नो॒ यो यो नः॑ । \newline
34. नः॒ श॒रदः॑ श॒रदो॑ नो नः श॒रदः॑ । \newline
35. श॒रदो॒ अजी॑ता॒ नजी॑ताञ् छ॒रदः॑ श॒रदो॒ अजी॑तान् । \newline
36. अजी॑ता॒ निन्द्र॒ इन्द्रो॒ अजी॑ता॒ नजी॑ता॒ निन्द्रः॑ । \newline
37. इन्द्रो॑ नेषन् नेष॒ दिन्द्र॒ इन्द्रो॑ नेषत् । \newline
38. ने॒ष॒ दत्यति॑ नेषन् नेष॒ दति॑ । \newline
39. अति॑ दुरि॒तानि॑ दुरि॒ता न्यत्यति॑ दुरि॒तानि॑ । \newline
40. दु॒रि॒तानि॒ विश्वा॒ विश्वा॑ दुरि॒तानि॑ दुरि॒तानि॒ विश्वा᳚ । \newline
41. दु॒रि॒तानीति॑ दुः - इ॒तानि॑ । \newline
42. विश्वेति॒ विश्वा᳚ । \newline
43. ये च॒त्वार॑ श्च॒त्वारो॒ ये ये च॒त्वारः॑ । \newline
44. च॒त्वारः॑ प॒थयः॑ प॒थय॑ श्च॒त्वार॑ श्च॒त्वारः॑ प॒थयः॑ । \newline
45. प॒थयो॑ देव॒याना॑ देव॒यानाः᳚ प॒थयः॑ प॒थयो॑ देव॒यानाः᳚ । \newline
46. दे॒व॒याना॑ अन्त॒रा ऽन्त॒रा दे॑व॒याना॑ देव॒याना॑ अन्त॒रा । \newline
47. दे॒व॒याना॒ इति॑ देव - यानाः᳚ । \newline
48. अ॒न्त॒रा द्यावा॑पृथि॒वी द्यावा॑पृथि॒वी अ॑न्त॒रा ऽन्त॒रा द्यावा॑पृथि॒वी । \newline
49. द्यावा॑पृथि॒वी वि॒यन्ति॑ वि॒यन्ति॒ द्यावा॑पृथि॒वी द्यावा॑पृथि॒वी वि॒यन्ति॑ । \newline
50. द्यावा॑पृथि॒वी इति॒ द्यावा᳚ - पृ॒थि॒वी । \newline
51. वि॒यन्तीति॑ वि - यन्ति॑ । \newline
52. तेषां॒ ॅयो य स्तेषा॒म् तेषां॒ ॅयः । \newline
53. यो अज्या॑नि॒ मज्या॑निं॒ ॅयो यो अज्या॑निम् । \newline
54. अज्या॑नि॒ मजी॑ति॒ मजी॑ति॒ मज्या॑नि॒ मज्या॑नि॒ मजी॑तिम् । \newline
55. अजी॑ति मा॒वहा॑ दा॒वहा॒ दजी॑ति॒ मजी॑ति मा॒वहा᳚त् । \newline
56. आ॒वहा॒त् तस्मै॒ तस्मा॑ आ॒वहा॑ दा॒वहा॒त् तस्मै᳚ । \newline
57. आ॒वहा॒दित्या᳚ - वहा᳚त् । \newline
58. तस्मै॑ नो न॒ स्तस्मै॒ तस्मै॑ नः । \newline
59. नो॒ दे॒वा॒ दे॒वा॒ नो॒ नो॒ दे॒वाः॒ । \newline
60. दे॒वाः॒ परि॒ परि॑ देवा देवाः॒ परि॑ । \newline

\textbf{Ghana Paata } \newline

1. ची॒य॒ते॒ यद् यच् ची॑यते चीयते॒ यद॒ग्नि र॒ग्निर् यच् ची॑यते चीयते॒ यद॒ग्निः । \newline
2. यद॒ग्नि र॒ग्निर् यद् यद॒ग्निर् येन॒ येना॒ग्निर् यद् यद॒ग्निर् येन॑ । \newline
3. अ॒ग्निर् येन॒ येना॒ग्नि र॒ग्निर् येन॑ दे॒वा दे॒वा येना॒ग्नि र॒ग्निर् येन॑ दे॒वाः । \newline
4. येन॑ दे॒वा दे॒वा येन॒ येन॑ दे॒वा ज्योति॑षा॒ ज्योति॑षा दे॒वा येन॒ येन॑ दे॒वा ज्योति॑षा । \newline
5. दे॒वा ज्योति॑षा॒ ज्योति॑षा दे॒वा दे॒वा ज्योति॑षो॒र्द्ध्वा उ॒र्द्ध्वा ज्योति॑षा दे॒वा दे॒वा ज्योति॑षो॒र्द्ध्वाः । \newline
6. ज्योति॑षो॒र्द्ध्वा उ॒र्द्ध्वा ज्योति॑षा॒ ज्योति॑षो॒र्द्ध्वा उ॒दाय॑न् नु॒दाय॑न् नु॒र्द्ध्वा ज्योति॑षा॒ ज्योति॑षो॒र्द्ध्वा उ॒दायन्न्॑ । \newline
7. उ॒र्द्ध्वा उ॒दाय॑न् नु॒दाय॑न् नु॒र्द्ध्वा उ॒र्द्ध्वा उ॒दाय॒न् नितीत्यु॒दाय॑न् नु॒र्द्ध्वा उ॒र्द्ध्वा उ॒दाय॒न् निति॑ । \newline
8. उ॒दाय॒न् निती त्यु॒दाय॑न् नु॒दाय॒न् नित्युख्य॒ मुख्य॒ मित्यु॒दाय॑न् नु॒दाय॒न् नित्युख्य᳚म् । \newline
9. उ॒दाय॒न्नित्यु॑त् - आयन्न्॑ । \newline
10. इत्युख्य॒ मुख्य॒ मिती त्युख्यꣳ॒॒ सꣳ स मुख्य॒ मिती त्युख्यꣳ॒॒ सम् । \newline
11. उख्यꣳ॒॒ सꣳ स मुख्य॒ मुख्यꣳ॒॒ स मि॑न्ध इन्धे॒ स मुख्य॒ मुख्यꣳ॒॒ स मि॑न्धे । \newline
12. स मि॑न्ध इन्धे॒ सꣳ स मि॑न्ध॒ इष्ट॑का॒ इष्ट॑का इन्धे॒ सꣳ स मि॑न्ध॒ इष्ट॑काः । \newline
13. इ॒न्ध॒ इष्ट॑का॒ इष्ट॑का इन्ध इन्ध॒ इष्ट॑का ए॒वैवेष्ट॑का इन्ध इन्ध॒ इष्ट॑का ए॒व । \newline
14. इष्ट॑का ए॒वैवेष्ट॑का॒ इष्ट॑का ए॒वैता ए॒ता ए॒वेष्ट॑का॒ इष्ट॑का ए॒वैताः । \newline
15. ए॒वैता ए॒ता ए॒वै वैता उपोपै॒ता ए॒वै वैता उप॑ । \newline
16. ए॒ता उपोपै॒ता ए॒ता उप॑ धत्ते धत्त॒ उपै॒ता ए॒ता उप॑ धत्ते । \newline
17. उप॑ धत्ते धत्त॒ उपोप॑ धत्ते वानस्प॒त्या वा॑नस्प॒त्या ध॑त्त॒ उपोप॑ धत्ते वानस्प॒त्याः । \newline
18. ध॒त्ते॒ वा॒न॒स्प॒त्या वा॑नस्प॒त्या ध॑त्ते धत्ते वानस्प॒त्याः सु॑व॒र्गस्य॑ सुव॒र्गस्य॑ वानस्प॒त्या ध॑त्ते धत्ते वानस्प॒त्याः सु॑व॒र्गस्य॑ । \newline
19. वा॒न॒स्प॒त्याः सु॑व॒र्गस्य॑ सुव॒र्गस्य॑ वानस्प॒त्या वा॑नस्प॒त्याः सु॑व॒र्गस्य॑ लो॒कस्य॑ लो॒कस्य॑ सुव॒र्गस्य॑ वानस्प॒त्या वा॑नस्प॒त्याः सु॑व॒र्गस्य॑ लो॒कस्य॑ । \newline
20. सु॒व॒र्गस्य॑ लो॒कस्य॑ लो॒कस्य॑ सुव॒र्गस्य॑ सुव॒र्गस्य॑ लो॒कस्य॒ सम॑ष्ट्यै॒ सम॑ष्ट्यै लो॒कस्य॑ सुव॒र्गस्य॑ सुव॒र्गस्य॑ लो॒कस्य॒ सम॑ष्ट्यै । \newline
21. सु॒व॒र्गस्येति॑ सुवः - गस्य॑ । \newline
22. लो॒कस्य॒ सम॑ष्ट्यै॒ सम॑ष्ट्यै लो॒कस्य॑ लो॒कस्य॒ सम॑ष्ट्यै श॒तायु॑धाय श॒तायु॑धाय॒ सम॑ष्ट्यै लो॒कस्य॑ लो॒कस्य॒ सम॑ष्ट्यै श॒तायु॑धाय । \newline
23. सम॑ष्ट्यै श॒तायु॑धाय श॒तायु॑धाय॒ सम॑ष्ट्यै॒ सम॑ष्ट्यै श॒तायु॑धाय श॒तवी᳚र्याय श॒तवी᳚र्याय श॒तायु॑धाय॒ सम॑ष्ट्यै॒ सम॑ष्ट्यै श॒तायु॑धाय श॒तवी᳚र्याय । \newline
24. सम॑ष्ट्या॒ इति॒ सं - अ॒ष्ट्यै॒ । \newline
25. श॒तायु॑धाय श॒तवी᳚र्याय श॒तवी᳚र्याय श॒तायु॑धाय श॒तायु॑धाय श॒तवी᳚र्याय श॒तोत॑ये श॒तोत॑ये श॒तवी᳚र्याय श॒तायु॑धाय श॒तायु॑धाय श॒तवी᳚र्याय श॒तोत॑ये । \newline
26. श॒तायु॑धा॒येति॑ श॒त - आ॒यु॒धा॒य॒ । \newline
27. श॒तवी᳚र्याय श॒तोत॑ये श॒तोत॑ये श॒तवी᳚र्याय श॒तवी᳚र्याय श॒तोत॑ये ऽभिमाति॒षाहे॑ ऽभिमाति॒षाहे॑ श॒तोत॑ये श॒तवी᳚र्याय श॒तवी᳚र्याय श॒तोत॑ये ऽभिमाति॒षाहे᳚ । \newline
28. श॒तवी᳚र्या॒येति॑ श॒त - वी॒र्या॒य॒ । \newline
29. श॒तोत॑ये ऽभिमाति॒षाहे॑ ऽभिमाति॒षाहे॑ श॒तोत॑ये श॒तोत॑ये ऽभिमाति॒षाहे᳚ । \newline
30. श॒तोत॑य॒ इति॑ श॒त - ऊ॒त॒ये॒ । \newline
31. अ॒भि॒मा॒ति॒षाह॒ इत्य॑भिमाति - साहे᳚ । \newline
32. श॒तं ॅयो यः श॒तꣳ श॒तं ॅयो नो॑ नो॒ यः श॒तꣳ श॒तं ॅयो नः॑ । \newline
33. यो नो॑ नो॒ यो यो नः॑ श॒रदः॑ श॒रदो॑ नो॒ यो यो नः॑ श॒रदः॑ । \newline
34. नः॒ श॒रदः॑ श॒रदो॑ नो नः श॒रदो॒ अजी॑ता॒ नजी॑ताञ् छ॒रदो॑ नो नः श॒रदो॒ अजी॑तान् । \newline
35. श॒रदो॒ अजी॑ता॒ नजी॑ताञ् छ॒रदः॑ श॒रदो॒ अजी॑ता॒ निन्द्र॒ इन्द्रो॒ अजी॑ताञ् छ॒रदः॑ श॒रदो॒ अजी॑ता॒ निन्द्रः॑ । \newline
36. अजी॑ता॒ निन्द्र॒ इन्द्रो॒ अजी॑ता॒ नजी॑ता॒ निन्द्रो॑ नेषन् नेष॒ दिन्द्रो॒ अजी॑ता॒ नजी॑ता॒ निन्द्रो॑ नेषत् । \newline
37. इन्द्रो॑ नेषन् नेष॒ दिन्द्र॒ इन्द्रो॑ नेष॒ दत्यति॑ नेष॒ दिन्द्र॒ इन्द्रो॑ नेष॒दति॑ । \newline
38. ने॒ष॒ दत्यति॑ नेषन् नेष॒दति॑ दुरि॒तानि॑ दुरि॒ता न्यति॑ नेषन् नेष॒दति॑ दुरि॒तानि॑ । \newline
39. अति॑ दुरि॒तानि॑ दुरि॒ता न्यत्यति॑ दुरि॒तानि॒ विश्वा॒ विश्वा॑ दुरि॒ता न्यत्यति॑ दुरि॒तानि॒ विश्वा᳚ । \newline
40. दु॒रि॒तानि॒ विश्वा॒ विश्वा॑ दुरि॒तानि॑ दुरि॒तानि॒ विश्वा᳚ । \newline
41. दु॒रि॒तानीति॑ दुः - इ॒तानि॑ । \newline
42. विश्वेति॒ विश्वा᳚ । \newline
43. ये च॒त्वार॑ श्च॒त्वारो॒ ये ये च॒त्वारः॑ प॒थयः॑ प॒थय॑ श्च॒त्वारो॒ ये ये च॒त्वारः॑ प॒थयः॑ । \newline
44. च॒त्वारः॑ प॒थयः॑ प॒थय॑ श्च॒त्वार॑ श्च॒त्वारः॑ प॒थयो॑ देव॒याना॑ देव॒यानाः᳚ प॒थय॑ श्च॒त्वार॑ श्च॒त्वारः॑ प॒थयो॑ देव॒यानाः᳚ । \newline
45. प॒थयो॑ देव॒याना॑ देव॒यानाः᳚ प॒थयः॑ प॒थयो॑ देव॒याना॑ अन्त॒रा ऽन्त॒रा दे॑व॒यानाः᳚ प॒थयः॑ प॒थयो॑ देव॒याना॑ अन्त॒रा । \newline
46. दे॒व॒याना॑ अन्त॒रा ऽन्त॒रा दे॑व॒याना॑ देव॒याना॑ अन्त॒रा द्यावा॑पृथि॒वी द्यावा॑पृथि॒वी अ॑न्त॒रा दे॑व॒याना॑ देव॒याना॑ अन्त॒रा द्यावा॑पृथि॒वी । \newline
47. दे॒व॒याना॒ इति॑ देव - यानाः᳚ । \newline
48. अ॒न्त॒रा द्यावा॑पृथि॒वी द्यावा॑पृथि॒वी अ॑न्त॒रा ऽन्त॒रा द्यावा॑पृथि॒वी वि॒यन्ति॑ वि॒यन्ति॒ द्यावा॑पृथि॒वी अ॑न्त॒रा ऽन्त॒रा द्यावा॑पृथि॒वी वि॒यन्ति॑ । \newline
49. द्यावा॑पृथि॒वी वि॒यन्ति॑ वि॒यन्ति॒ द्यावा॑पृथि॒वी द्यावा॑पृथि॒वी वि॒यन्ति॑ । \newline
50. द्यावा॑पृथि॒वी इति॒ द्यावा᳚ - पृ॒थि॒वी । \newline
51. वि॒यन्तीति॑ वि - यन्ति॑ । \newline
52. तेषां॒ ॅयो यस्तेषा॒म् तेषां॒ ॅयो अज्या॑नि॒ मज्या॑निं॒ ॅयस्तेषा॒म् तेषां॒ ॅयो अज्या॑निम् । \newline
53. यो अज्या॑नि॒ मज्या॑निं॒ ॅयो यो अज्या॑नि॒ मजी॑ति॒ मजी॑ति॒ मज्या॑निं॒ ॅयो यो अज्या॑नि॒ मजी॑तिम् । \newline
54. अज्या॑नि॒ मजी॑ति॒ मजी॑ति॒ मज्या॑नि॒ मज्या॑नि॒ मजी॑ति मा॒वहा॑ दा॒वहा॒ दजी॑ति॒ मज्या॑नि॒ मज्या॑नि॒ मजी॑ति मा॒वहा᳚त् । \newline
55. अजी॑ति मा॒वहा॑ दा॒वहा॒ दजी॑ति॒ मजी॑ति मा॒वहा॒त् तस्मै॒ तस्मा॑ आ॒वहा॒ दजी॑ति॒ मजी॑ति मा॒वहा॒त् तस्मै᳚ । \newline
56. आ॒वहा॒त् तस्मै॒ तस्मा॑ आ॒वहा॑ दा॒वहा॒त् तस्मै॑ नो न॒ स्तस्मा॑ आ॒वहा॑ दा॒वहा॒त् तस्मै॑ नः । \newline
57. आ॒वहा॒दित्या᳚ - वहा᳚त् । \newline
58. तस्मै॑ नो न॒ स्तस्मै॒ तस्मै॑ नो देवा देवा न॒ स्तस्मै॒ तस्मै॑ नो देवाः । \newline
59. नो॒ दे॒वा॒ दे॒वा॒ नो॒ नो॒ दे॒वाः॒ परि॒ परि॑ देवा नो नो देवाः॒ परि॑ । \newline
60. दे॒वाः॒ परि॒ परि॑ देवा देवाः॒ परि॑ दत्त दत्त॒ परि॑ देवा देवाः॒ परि॑ दत्त । \newline
\pagebreak
\markright{ TS 5.7.2.4  \hfill https://www.vedavms.in \hfill}

\section{ TS 5.7.2.4 }

\textbf{TS 5.7.2.4 } \newline
\textbf{Samhita Paata} \newline

परि॑ दत्ते॒ह सर्वे᳚ ॥ग्री॒ष्मो हे॑म॒न्त उ॒त नो॑ वस॒न्तः श॒रद् व॒र्॒.षाः सु॑वि॒तन्नो॑ अस्तु । तेषा॑मृतू॒नाꣳ श॒त शा॑रदानां निवा॒त ए॑षा॒मभ॑ये स्याम ॥इ॒दु॒व॒थ्स॒राय॑ परिवथ्स॒राय॑ संवॅथ्स॒राय॑ कृणुता बृ॒हन्नमः॑ । तेषां᳚ ॅव॒यꣳ सु॑म॒तौ य॒ज्ञिया॑नां॒ ज्योगजी॑ता॒ अह॑ताः स्याम ॥ भ॒द्रान्नः॒ श्रेयः॒ सम॑नैष्ट देवा॒स्त्वया॑ऽव॒सेन॒ सम॑शीमहि त्वा ।स नो॑ मयो॒ भूः पि॑तो॒ - [  ] \newline

\textbf{Pada Paata} \newline

परीति॑ । द॒त्त॒ । इ॒ह । सर्वे᳚ ॥ ग्री॒ष्मः । हे॒म॒न्तः । उ॒त । नः॒ । व॒स॒न्तः । श॒रत् । व॒र्.॒षाः । सु॒वि॒तम् । नः॒ । अ॒स्तु॒ ॥ तेषा᳚म् । ऋ॒तू॒नाम् । श॒तशा॑रदाना॒मिति॑ श॒त - शा॒र॒दा॒ना॒म् । नि॒वा॒त इति॑ नि - वा॒ते । ए॒षा॒म् । अभ॑ये । स्या॒म॒ ॥ इ॒दु॒व॒थ्स॒रायेती॑दु - व॒थ्स॒राय॑ । प॒रि॒व॒थ्स॒रायेति॑ परि - व॒थ्स॒राय॑ । सं॒ॅव॒थ्स॒रायेति॑ सं - व॒थ्स॒राय॑ । कृ॒णु॒त॒ । बृ॒हत् । नमः॑ ॥ तेषा᳚म् । व॒यम् । सु॒म॒ताविति॑ सु - म॒तौ । य॒ज्ञिया॑नाम् । ज्योक् । अजी॑ताः । अह॑ताः । स्या॒म॒ ॥ भ॒द्रात् । नः॒ । श्रेयः॑ । समिति॑ । अ॒नै॒ष्ट॒ । दे॒वाः॒ । त्वया᳚ । अ॒व॒सेन॑ । समिति॑ । अ॒शी॒म॒हि॒ । त्वा॒ ॥ सः । नः॒ । म॒यो॒भूरिति॑ मयः-भूः । पि॒तो॒ इति॑ ।  \newline


\textbf{Krama Paata} \newline

परि॑ दत्त । द॒त्ते॒ह । इ॒ह सर्वे᳚ । सर्व॒ इति॒ सर्वे᳚ ॥ ग्री॒ष्मो हे॑म॒न्तः । हे॒म॒न्त उ॒त । उ॒त नः॑ । नो॒ व॒स॒न्तः । व॒स॒न्तः श॒रत् । श॒रद् व॒र्.॒षाः । व॒र्.॒षाः सु॑वि॒तम् । सु॒वि॒तम् नः॑ । नो॒ अ॒स्तु॒ । अ॒स्त्वित्य॑स्तु ॥ तेषा॑मृतू॒नाम् । ऋ॒तू॒नाꣳ श॒तशा॑रदानाम् । श॒तशा॑रदानाम् निवा॒ते । श॒तशा॑रदाना॒मिति॑ श॒त - शा॒र॒दा॒ना॒म् । नि॒वा॒त ए॑षाम् । नि॒वा॒त इति॑ नि - वा॒ते । ए॒षा॒मभ॑ये । अभ॑ये स्याम । स्या॒मेति॑ स्याम ॥ इ॒दु॒व॒थ्स॒राय॑ परिवथ्स॒राय॑ । इ॒दु॒व॒थ्स॒रायेती॑दु - व॒थ्स॒राय॑ । प॒रि॒व॒थ्स॒राय॑ सम्ॅवथ्स॒राय॑ । प॒रि॒व॒थ्स॒रायेति॑ परि - व॒थ्स॒राय॑ । स॒म्ॅव॒थ्स॒राय॑ कृणुत । स॒म्ॅव॒थ्स॒रायेति॑ सम् - व॒थ्स॒राय॑ । कृ॒णु॒ता॒ बृ॒हत् । बृ॒हन् नमः॑ । नम॒ इति॒ नमः॑ ॥ तेषा᳚म् ॅव॒यम् । व॒यꣳ सु॑म॒तौ । सु॒म॒तौ य॒ज्ञिया॑नाम् । सु॒म॒ताविति॑ सु - म॒तौ । य॒ज्ञिया॑ना॒म् ज्योक् । ज्योगजी॑ताः । अजी॑ता॒ अह॑ताः । अह॑ताः स्याम । स्या॒मेति॑ स्याम ॥ भ॒द्रान् नः॑ । नः॒ श्रेयः॑ । श्रेयः॒ सम् । सम॑नैष्ट । अ॒नै॒ष्ट॒ दे॒वाः॒ । दे॒वा॒स्त्वया᳚ । त्वया॑ऽव॒सेन॑ । अ॒व॒सेन॒ सम् । सम॑शीमहि । अ॒शी॒म॒हि॒ त्वा॒ । त्वेति॑ त्वा ॥ स नः॑ । नो॒ म॒यो॒भूः । म॒यो॒भूः पि॑तो । म॒यो॒भूरिति॑ मयः - भूः । पि॒तो॒ आ । पि॒तो॒ इति॑ पितो \newline

\textbf{Jatai Paata} \newline

1. परि॑ दत्त दत्त॒ परि॒ परि॑ दत्त । \newline
2. द॒त्ते॒ हेह द॑त्त दत्ते॒ह । \newline
3. इ॒ह सर्वे॒ सर्व॑ इ॒हे ह सर्वे᳚ । \newline
4. सर्व॒ इति॒ सर्वे᳚ । \newline
5. ग्री॒ष्मो हे॑म॒न्तो हे॑म॒न्तो ग्री॒ष्मो ग्री॒ष्मो हे॑म॒न्तः । \newline
6. हे॒म॒न्त उ॒तोत हे॑म॒न्तो हे॑म॒न्त उ॒त । \newline
7. उ॒त नो॑ न उ॒तोत नः॑ । \newline
8. नो॒ व॒स॒न्तो व॑स॒न्तो नो॑ नो वस॒न्तः । \newline
9. व॒स॒न्तः श॒रच् छ॒रद् व॑स॒न्तो व॑स॒न्तः श॒रत् । \newline
10. श॒रद् व॒र्॒.षा व॒र्॒.षाः श॒रच् छ॒रद् व॒र्॒.षाः । \newline
11. व॒र्॒.षाः सु॑वि॒तꣳ सु॑वि॒तं ॅव॒र्॒.षा व॒र्॒.षाः सु॑वि॒तम् । \newline
12. सु॒वि॒तम् नो॑ नः सुवि॒तꣳ सु॑वि॒तम् नः॑ । \newline
13. नो॒ अ॒स्त्व॒स्तु॒ नो॒ नो॒ अ॒स्तु॒ । \newline
14. अ॒स्त्वित्य॑स्तु । \newline
15. तेषा॑ मृतू॒ना मृ॑तू॒नाम् तेषा॒म् तेषा॑ मृतू॒नाम् । \newline
16. ऋ॒तू॒नाꣳ श॒तशा॑रदानाꣳ श॒तशा॑रदाना मृतू॒ना मृ॑तू॒नाꣳ श॒तशा॑रदानाम् । \newline
17. श॒तशा॑रदानान् निवा॒ते नि॑वा॒ते श॒तशा॑रदानाꣳ श॒तशा॑रदानान् निवा॒ते । \newline
18. श॒तशा॑रदाना॒मिति॑ श॒त - शा॒र॒दा॒ना॒म् । \newline
19. नि॒वा॒त ए॑षा मेषान् निवा॒ते नि॑वा॒त ए॑षाम् । \newline
20. नि॒वा॒त इति॑ नि - वा॒ते । \newline
21. ए॒षा॒ मभ॒ये ऽभ॑य एषा मेषा॒ मभ॑ये । \newline
22. अभ॑ये स्याम स्या॒मा भ॒ये ऽभ॑ये स्याम । \newline
23. स्या॒मेति॑ स्याम । \newline
24. इ॒दु॒व॒थ्स॒राय॑ परिवथ्स॒राय॑ परिवथ्स॒राये॑ दुवथ्स॒राये॑ दुवथ्स॒राय॑ परिवथ्स॒राय॑ । \newline
25. इ॒दु॒व॒थ्स॒रायेती॑दु - व॒थ्स॒राय॑ । \newline
26. प॒रि॒व॒थ्स॒राय॑ संॅवथ्स॒राय॑ संॅवथ्स॒राय॑ परिवथ्स॒राय॑ परिवथ्स॒राय॑ संॅवथ्स॒राय॑ । \newline
27. प॒रि॒व॒थ्स॒रायेति॑ परि - व॒थ्स॒राय॑ । \newline
28. सं॒ॅव॒थ्स॒राय॑ कृणुत कृणुत संॅवथ्स॒राय॑ संॅवथ्स॒राय॑ कृणुत । \newline
29. सं॒ॅव॒थ्स॒रायेति॑ सं - व॒थ्स॒राय॑ । \newline
30. कृ॒णु॒ता॒ बृ॒हद् बृ॒हत् कृ॑णुत कृणुता बृ॒हत् । \newline
31. बृ॒हन् नमो॒ नमो॑ बृ॒हद् बृ॒हन् नमः॑ । \newline
32. नम॒ इति॒ नमः॑ । \newline
33. तेषां᳚ ॅव॒यं ॅव॒यम् तेषा॒म् तेषां᳚ ॅव॒यम् । \newline
34. व॒यꣳ सु॑म॒तौ सु॑म॒तौ व॒यं ॅव॒यꣳ सु॑म॒तौ । \newline
35. सु॒म॒तौ य॒ज्ञिया॑नां ॅय॒ज्ञिया॑नाꣳ सुम॒तौ सु॑म॒तौ य॒ज्ञिया॑नाम् । \newline
36. सु॒म॒ताविति॑ सु - म॒तौ । \newline
37. य॒ज्ञिया॑ना॒म् ज्योग् ज्योग् य॒ज्ञिया॑नां ॅय॒ज्ञिया॑ना॒म् ज्योक् । \newline
38. ज्योगजी॑ता॒ अजी॑ता॒ ज्योग् ज्योगजी॑ताः । \newline
39. अजी॑ता॒ अह॑ता॒ अह॑ता॒ अजी॑ता॒ अजी॑ता॒ अह॑ताः । \newline
40. अह॑ताः स्याम स्या॒मा ह॑ता॒ अह॑ताः स्याम । \newline
41. स्या॒मेति॑ स्याम । \newline
42. भ॒द्रान् नो॑ नो भ॒द्राद् भ॒द्रान् नः॑ । \newline
43. नः॒ श्रेयः॒ श्रेयो॑ नो नः॒ श्रेयः॑ । \newline
44. श्रेयः॒ सꣳ सꣳ श्रेयः॒ श्रेयः॒ सम् । \newline
45. स म॑नैष्टा नैष्ट॒ सꣳ स म॑नैष्ट । \newline
46. अ॒नै॒ष्ट॒ दे॒वा॒ दे॒वा॒ अ॒नै॒ष्टा॒ नै॒ष्ट॒ दे॒वाः॒ । \newline
47. दे॒वा॒ स्त्वया॒ त्वया॑ देवा देवा॒ स्त्वया᳚ । \newline
48. त्वया॑ ऽव॒सेना॑ व॒सेन॒ त्वया॒ त्वया॑ ऽव॒सेन॑ । \newline
49. अ॒व॒सेन॒ सꣳ स म॑व॒सेना॑ व॒सेन॒ सम् । \newline
50. स म॑शीमह्य शीमहि॒ सꣳ स म॑शीमहि । \newline
51. अ॒शी॒म॒हि॒ त्वा॒ त्वा॒ ऽशी॒म॒ह्य॒ शी॒म॒हि॒ त्वा॒ । \newline
52. त्वेति॑ त्वा । \newline
53. स नो॑ नः॒ स स नः॑ । \newline
54. नो॒ म॒यो॒भूर् म॑यो॒भूर् नो॑ नो मयो॒भूः । \newline
55. म॒यो॒भूः पि॑तो पितो मयो॒भूर् म॑यो॒भूः पि॑तो । \newline
56. म॒यो॒भूरिति॑ मयः - भूः । \newline
57. पि॒तो॒ आ पि॑तो पितो॒ आ । \newline
58. पि॒तो॒ इति॑ पितो । \newline

\textbf{Ghana Paata } \newline

1. परि॑ दत्त दत्त॒ परि॒ परि॑ दत्ते॒हेह द॑त्त॒ परि॒ परि॑ दत्ते॒ह । \newline
2. द॒त्ते॒हेह द॑त्त दत्ते॒ह सर्वे॒ सर्व॑ इ॒ह द॑त्त दत्ते॒ह सर्वे᳚ । \newline
3. इ॒ह सर्वे॒ सर्व॑ इ॒हेह सर्वे᳚ । \newline
4. सर्व॒ इति॒ सर्वे᳚ । \newline
5. ग्री॒ष्मो हे॑म॒न्तो हे॑म॒न्तो ग्री॒ष्मो ग्री॒ष्मो हे॑म॒न्त उ॒तोत हे॑म॒न्तो ग्री॒ष्मो ग्री॒ष्मो हे॑म॒न्त उ॒त । \newline
6. हे॒म॒न्त उ॒तोत हे॑म॒न्तो हे॑म॒न्त उ॒त नो॑ न उ॒त हे॑म॒न्तो हे॑म॒न्त उ॒त नः॑ । \newline
7. उ॒त नो॑ न उ॒तोत नो॑ वस॒न्तो व॑स॒न्तो न॑ उ॒तोत नो॑ वस॒न्तः । \newline
8. नो॒ व॒स॒न्तो व॑स॒न्तो नो॑ नो वस॒न्तः श॒रच् छ॒रद् व॑स॒न्तो नो॑ नो वस॒न्तः श॒रत् । \newline
9. व॒स॒न्तः श॒रच् छ॒रद् व॑स॒न्तो व॑स॒न्तः श॒रद् व॒र्॒.षा व॒र्॒.षाः श॒रद् व॑स॒न्तो व॑स॒न्तः श॒रद् व॒र्॒.षाः । \newline
10. श॒रद् व॒र्॒.षा व॒र्॒.षाः श॒रच् छ॒रद् व॒र्॒.षाः सु॑वि॒तꣳ सु॑वि॒तं ॅव॒र्॒.षाः श॒रच् छ॒रद् व॒र्॒.षाः सु॑वि॒तम् । \newline
11. व॒र्॒.षाः सु॑वि॒तꣳ सु॑वि॒तं ॅव॒र्॒.षा व॒र्॒.षाः सु॑वि॒तम् नो॑ नः सुवि॒तं ॅव॒र्॒.षा व॒र्॒.षाः सु॑वि॒तम् नः॑ । \newline
12. सु॒वि॒तम् नो॑ नः सुवि॒तꣳ सु॑वि॒तम् नो॑ अस्त्वस्तु नः सुवि॒तꣳ सु॑वि॒तम् नो॑ अस्तु । \newline
13. नो॒ अ॒स्त्व॒स्तु॒ नो॒ नो॒ अ॒स्तु॒ । \newline
14. अ॒स्त्वित्य॑स्तु । \newline
15. तेषा॑ मृतू॒ना मृ॑तू॒नाम् तेषा॒म् तेषा॑ मृतू॒नाꣳ श॒तशा॑रदानाꣳ श॒तशा॑रदाना मृतू॒नाम् तेषा॒म् तेषा॑ मृतू॒नाꣳ श॒तशा॑रदानाम् । \newline
16. ऋ॒तू॒नाꣳ श॒तशा॑रदानाꣳ श॒तशा॑रदाना मृतू॒ना मृ॑तू॒नाꣳ श॒तशा॑रदानाम् निवा॒ते नि॑वा॒ते श॒तशा॑रदाना मृतू॒ना मृ॑तू॒नाꣳ श॒तशा॑रदानाम् निवा॒ते । \newline
17. श॒तशा॑रदानाम् निवा॒ते नि॑वा॒ते श॒तशा॑रदानाꣳ श॒तशा॑रदानाम् निवा॒त ए॑षा मेषाम् निवा॒ते श॒तशा॑रदानाꣳ श॒तशा॑रदानाम् निवा॒त ए॑षाम् । \newline
18. श॒तशा॑रदाना॒मिति॑ श॒त - शा॒र॒दा॒ना॒म् । \newline
19. नि॒वा॒त ए॑षा मेषाम् निवा॒ते नि॑वा॒त ए॑षा॒ मभ॒ये ऽभ॑य एषाम् निवा॒ते नि॑वा॒त ए॑षा॒ मभ॑ये । \newline
20. नि॒वा॒त इति॑ नि - वा॒ते । \newline
21. ए॒षा॒ मभ॒ये ऽभ॑य एषा मेषा॒ मभ॑ये स्याम स्या॒माभ॑य एषा मेषा॒ मभ॑ये स्याम । \newline
22. अभ॑ये स्याम स्या॒माभ॒ये ऽभ॑ये स्याम । \newline
23. स्या॒मेति॑ स्याम । \newline
24. इ॒दु॒व॒थ्स॒राय॑ परिवथ्स॒राय॑ परिवथ्स॒रा ये॑दुवथ्स॒राये॑ दुवथ्स॒राय॑ परिवथ्स॒राय॑ संॅवथ्स॒राय॑ संॅवथ्स॒राय॑ परिवथ्स॒रा ये॑दुवथ्स॒राये॑ दुवथ्स॒राय॑ परिवथ्स॒राय॑ संॅवथ्स॒राय॑ । \newline
25. इ॒दु॒व॒थ्स॒रायेती॑दु - व॒थ्स॒राय॑ । \newline
26. प॒रि॒व॒थ्स॒राय॑ संॅवथ्स॒राय॑ संॅवथ्स॒राय॑ परिवथ्स॒राय॑ परिवथ्स॒राय॑ संॅवथ्स॒राय॑ कृणुत कृणुत संॅवथ्स॒राय॑ परिवथ्स॒राय॑ परिवथ्स॒राय॑ संॅवथ्स॒राय॑ कृणुत । \newline
27. प॒रि॒व॒थ्स॒रायेति॑ परि - व॒थ्स॒राय॑ । \newline
28. सं॒ॅव॒थ्स॒राय॑ कृणुत कृणुत संॅवथ्स॒राय॑ संॅवथ्स॒राय॑ कृणुता बृ॒हद् बृ॒हत् कृ॑णुत संॅवथ्स॒राय॑ संॅवथ्स॒राय॑ कृणुता बृ॒हत् । \newline
29. सं॒ॅव॒थ्स॒रायेति॑ सं - व॒थ्स॒राय॑ । \newline
30. कृ॒णु॒ता॒ बृ॒हद् बृ॒हत् कृ॑णुत कृणुता बृ॒हन् नमो॒ नमो॑ बृ॒हत् कृ॑णुत कृणुता बृ॒हन् नमः॑ । \newline
31. बृ॒हन् नमो॒ नमो॑ बृ॒हद् बृ॒हन् नमः॑ । \newline
32. नम॒ इति॒ नमः॑ । \newline
33. तेषां᳚ ॅव॒यं ॅव॒यम् तेषा॒म् तेषां᳚ ॅव॒यꣳ सु॑म॒तौ सु॑म॒तौ व॒यम् तेषा॒म् तेषां᳚ ॅव॒यꣳ सु॑म॒तौ । \newline
34. व॒यꣳ सु॑म॒तौ सु॑म॒तौ व॒यं ॅव॒यꣳ सु॑म॒तौ य॒ज्ञिया॑नां ॅय॒ज्ञिया॑नाꣳ सुम॒तौ व॒यं ॅव॒यꣳ सु॑म॒तौ य॒ज्ञिया॑नाम् । \newline
35. सु॒म॒तौ य॒ज्ञिया॑नां ॅय॒ज्ञिया॑नाꣳ सुम॒तौ सु॑म॒तौ य॒ज्ञिया॑ना॒म् ज्योग् ज्योग् य॒ज्ञिया॑नाꣳ सुम॒तौ सु॑म॒तौ य॒ज्ञिया॑ना॒म् ज्योक् । \newline
36. सु॒म॒ताविति॑ सु - म॒तौ । \newline
37. य॒ज्ञिया॑ना॒म् ज्योग् ज्योग् य॒ज्ञिया॑नां ॅय॒ज्ञिया॑ना॒म् ज्योगजी॑ता॒ अजी॑ता॒ ज्योग् य॒ज्ञिया॑नां ॅय॒ज्ञिया॑ना॒म् ज्योगजी॑ताः । \newline
38. ज्योगजी॑ता॒ अजी॑ता॒ ज्योग् ज्योगजी॑ता॒ अह॑ता॒ अह॑ता॒ अजी॑ता॒ ज्योग् ज्योगजी॑ता॒ अह॑ताः । \newline
39. अजी॑ता॒ अह॑ता॒ अह॑ता॒ अजी॑ता॒ अजी॑ता॒ अह॑ताः स्याम स्या॒मा ह॑ता॒ अजी॑ता॒ अजी॑ता॒ अह॑ताः स्याम । \newline
40. अह॑ताः स्याम स्या॒मा ह॑ता॒ अह॑ताः स्याम । \newline
41. स्या॒मेति॑ स्याम । \newline
42. भ॒द्रान् नो॑ नो भ॒द्राद् भ॒द्रान् नः॒ श्रेयः॒ श्रेयो॑ नो भ॒द्राद् भ॒द्रान् नः॒ श्रेयः॑ । \newline
43. नः॒ श्रेयः॒ श्रेयो॑ नो नः॒ श्रेयः॒ सꣳ सꣳ श्रेयो॑ नो नः॒ श्रेयः॒ सम् । \newline
44. श्रेयः॒ सꣳ सꣳ श्रेयः॒ श्रेयः॒ स म॑नैष्टा नैष्ट॒ सꣳ श्रेयः॒ श्रेयः॒ स म॑नैष्ट । \newline
45. स म॑नैष्टा नैष्ट॒ सꣳ स म॑नैष्ट देवा देवा अनैष्ट॒ सꣳ स म॑नैष्ट देवाः । \newline
46. अ॒नै॒ष्ट॒ दे॒वा॒ दे॒वा॒ अ॒नै॒ष्टा॒ नै॒ष्ट॒ दे॒वा॒ स्त्वया॒ त्वया॑ देवा अनैष्टा नैष्ट देवा॒ स्त्वया᳚ । \newline
47. दे॒वा॒ स्त्वया॒ त्वया॑ देवा देवा॒ स्त्वया॑ ऽव॒सेना॑ व॒सेन॒ त्वया॑ देवा देवा॒ स्त्वया॑ ऽव॒सेन॑ । \newline
48. त्वया॑ ऽव॒सेना॑ व॒सेन॒ त्वया॒ त्वया॑ ऽव॒सेन॒ सꣳ स म॑व॒सेन॒ त्वया॒ त्वया॑ ऽव॒सेन॒ सम् । \newline
49. अ॒व॒सेन॒ सꣳ स म॑व॒सेना॑ व॒सेन॒ स म॑शीमह्य शीमहि॒ स म॑व॒सेना॑ व॒सेन॒ स म॑शीमहि । \newline
50. स म॑शीमह्य शीमहि॒ सꣳ स म॑शीमहि त्वा त्वा ऽशीमहि॒ सꣳ स म॑शीमहि त्वा । \newline
51. अ॒शी॒म॒हि॒ त्वा॒ त्वा॒ ऽशी॒म॒ह्य॒ शी॒म॒हि॒ त्वा॒ । \newline
52. त्वेति॑ त्वा । \newline
53. स नो॑ नः॒ स स नो॑ मयो॒भूर् म॑यो॒भूर् नः॒ स स नो॑ मयो॒भूः । \newline
54. नो॒ म॒यो॒भूर् म॑यो॒भूर् नो॑ नो मयो॒भूः पि॑तो पितो मयो॒भूर् नो॑ नो मयो॒भूः पि॑तो । \newline
55. म॒यो॒भूः पि॑तो पितो मयो॒भूर् म॑यो॒भूः पि॑तो॒ आ पि॑तो मयो॒भूर् म॑यो॒भूः पि॑तो॒ आ । \newline
56. म॒यो॒भूरिति॑ मयः - भूः । \newline
57. पि॒तो॒ आ पि॑तो पितो॒ आ वि॑शस्व विश॒स्वा पि॑तो पितो॒ आ वि॑शस्व । \newline
58. पि॒तो॒ इति॑ पितो । \newline
\pagebreak
\markright{ TS 5.7.2.5  \hfill https://www.vedavms.in \hfill}

\section{ TS 5.7.2.5 }

\textbf{TS 5.7.2.5 } \newline
\textbf{Samhita Paata} \newline

आ वि॑शस्व॒ शं तो॒काय॑ त॒नुवे᳚ स्यो॒नः ॥ अज्या॑नीरे॒ता उप॑ दधात्ये॒ता वै दे॒वता॒ अप॑राजिता॒स्ता ए॒व प्र वि॑शति॒ नैव जी॑यते ब्रह्मवा॒दिनो॑ वदन्ति॒ यद॑र्द्धमा॒सा मासा॑ ऋ॒तवः॑ संॅवथ्स॒र ओष॑धीः॒ पच॒न्त्यथ॒ कस्मा॑द॒न्याभ्यो॑ दे॒वता᳚भ्य आग्रय॒णं निरु॑प्यत॒ इत्ये॒ता हि तद्-दे॒वता॑ उ॒दज॑य॒न्॒ यदृ॒तुभ्यो॑ नि॒र्वपे᳚द् दे॒वता᳚भ्यः स॒मदं॑ दद्ध्यादाग्रय॒णं ( ) नि॒रुप्यै॒ता आहु॑ती र्जुहोत्यर्द्धमा॒साने॒व मासा॑नृ॒तून्थ् सं॑ॅवथ्स॒रं प्री॑णाति॒ न दे॒वता᳚भ्यः स॒मदं॑ दधाति भ॒द्रान्नः॒ श्रेयः॒ सम॑नैष्ट देवा॒ इत्या॑ह हु॒ताद्या॑य॒ यज॑मान॒स्याऽ*प॑राभावाय ॥ \newline

\textbf{Pada Paata} \newline

एति॑ । वि॒श॒स्व॒ । शम् । तो॒काय॑ । त॒नुवे᳚ । स्यो॒नः ॥ अज्या॑नीः । ए॒ताः । उपेति॑ । द॒धा॒ति॒ । ए॒ताः । वै । दे॒वताः᳚ । अप॑राजिता॒ इत्यप॑रा - जि॒ताः॒ । ताः । ए॒व । प्रेति॑ । वि॒श॒ति॒ । न । ए॒व । जी॒य॒ते॒ । ब्र॒ह्म॒वा॒दिन॒ इति॑ ब्रह्म - वा॒दिनः॑ । व॒द॒न्ति॒ । यत् । अ॒द्‌र्ध॒मा॒सा इत्य॑द्‌र्ध - मा॒साः । मासाः᳚ । ऋ॒तवः॑ । सं॒ॅव॒थ्स॒र इति॑ सं - व॒थ्स॒रः । ओष॑धीः । पच॑न्ति । अथ॑ । कस्मा᳚त् । अ॒न्याभ्यः॑ । दे॒वता᳚भ्यः । आ॒ग्र॒य॒णम् । निरिति॑ । उ॒प्य॒ते॒ । इति॑ । ए॒ताः । हि । तत् । दे॒वताः᳚ । उ॒दज॑य॒न्नित्यु॑त् - अज॑यन्न् । यत् । ऋ॒तुभ्य॒ इत्यृ॒तु-भ्यः॒ । नि॒र्वपे॒दिति॑ निः - वपे᳚त् । दे॒वता᳚भ्यः । स॒मद॒मिति॑ स - मद᳚म् । द॒द्ध्या॒त् । आ॒ग्र॒य॒णम् ( ) । नि॒रुप्येति॑ निः - उप्य॑ । ए॒ताः । आहु॑ती॒रित्या- हु॒तीः॒ । जु॒हो॒ति॒ । अ॒द्‌र्ध॒मा॒सानित्य॑र्ध - मा॒सान् । ए॒व । मासान्॑ । ऋ॒तून् । सं॒ॅव॒थ्स॒रमिति॑ सं - व॒थ्स॒रम् । प्री॒णा॒ति॒ । न । दे॒वता᳚भ्यः । स॒मद॒मिति॑ स - मद᳚म् । द॒धा॒ति॒ । भ॒द्रात् । नः॒ । श्रेयः॑ । समिति॑ । अ॒नै॒ष्ट॒ । दे॒वाः॒ । इति॑ । आ॒ह॒ । हु॒ताद्या॒येति॑ हुत - अद्या॑य । यज॑मानस्य । अप॑राभावा॒येत्यप॑रा - भा॒वा॒य॒ ॥  \newline


\textbf{Krama Paata} \newline

आ वि॑शस्व । वि॒श॒स्व॒ शम् । शम् तो॒काय॑ । तो॒काय॑ त॒नुवे᳚ । त॒नुवे᳚ स्यो॒नः । स्यो॒न इति॑ स्यो॒नः ॥ अज्या॑नीरे॒ताः । ए॒ता उप॑ । उप॑ दधाति । द॒धा॒त्ये॒ताः । ए॒ता वै । वै दे॒वताः᳚ । दे॒वता॒ अप॑राजिताः । अप॑राजिता॒स्ताः । अप॑राजिता॒ इत्यप॑रा - जि॒ताः॒ । ता ए॒व । ए॒व प्र । प्र वि॑शति । वि॒श॒ति॒ न । नैव । ए॒व जी॑यते । जी॒य॒ते॒ ब्र॒ह्म॒वा॒दिनः॑ । ब्र॒ह्म॒वा॒दिनो॑ वदन्ति । ब्र॒ह्म॒वा॒दिन॒ इति॑ ब्रह्म - वा॒दिनः॑ । व॒द॒न्ति॒ यत् । यद॑र्द्धमा॒साः । अ॒र्द्ध॒मा॒सा मासाः᳚ । अ॒र्द्ध॒मा॒सा इत्य॑र्द्ध - मा॒साः । मासा॑ ऋ॒तवः॑ । ऋ॒तवः॑ सम्ॅवथ्स॒रः । स॒म्ॅव॒थ्स॒र ओष॑धीः । स॒म्ॅव॒थ्स॒र इति॑ सम् - व॒थ्स॒रः । ओष॑धीः॒ पच॑न्ति । पच॒न्त्यथ॑ । अथ॒ कस्मा᳚त् । कस्मा॑द॒न्याभ्यः॑ । अ॒न्याभ्यो॑ दे॒वता᳚भ्यः । दे॒वता᳚भ्य आग्रय॒णम् । आ॒ग्र॒य॒णम् निः । निरु॑प्यते । उ॒प्य॒त॒ इति॑ । इत्ये॒ताः । ए॒ता हि । हि तत् । तद् दे॒वताः᳚ । दे॒वता॑ उ॒दज॑यन्न् । उ॒दज॑य॒न्न्.॒ यत् । उ॒दज॑य॒न्नित्यु॑त् - अज॑यन्न् । यदृ॒तुभ्यः॑ । ऋ॒तुभ्यो॑ नि॒र्वपे᳚त् । ऋ॒तुभ्य॒ इत्यृ॒तु - भ्यः॒ । नि॒र्वपे᳚द् दे॒वता᳚भ्यः । नि॒र्वपे॒दिति॑ निः - वपे᳚त् । दे॒वता᳚भ्यः स॒मद᳚म् । स॒मद॑म् दद्ध्यात् । स॒मद॒मिति॑ स - मद᳚म् । द॒द्ध्या॒दा॒ग्र॒य॒णम् ( ) । आ॒ग्र॒य॒णम् नि॒रुप्य॑ । नि॒रुप्यै॒ताः । नि॒रुप्येति॑ निः - उप्य॑ । ए॒ता आहु॑तीः । आहु॑तीर् जुहोति । आहु॑ती॒रित्या - हु॒तीः॒ । जु॒हो॒त्य॒र्द्ध॒मा॒सान् । अ॒र्द्ध॒मा॒साने॒व । अ॒र्द्ध॒मा॒सानित्य॑र्द्ध - मा॒सान् । ए॒व मासान्॑ । मासा॑नृ॒तून् । ऋ॒तून्थ् स॑म्ॅवथ्स॒रम् । स॒म्ॅव॒थ्स॒रम् प्री॑णाति । स॒म्ॅव॒थ्स॒रमिति॑ सम् - व॒थ्स॒रम् । प्री॒णा॒ति॒ न । न दे॒वता᳚भ्यः । दे॒वता᳚भ्यः स॒मद᳚म् । स॒मद॑म् दधाति । स॒मद॒मिति॑ स - मद᳚म् । द॒धा॒ति॒ भ॒द्रात् । भ॒द्रान् नः॑ । नः॒ श्रेयः॑ । श्रेयः॒ सम् । सम॑नैष्ट । अ॒नै॒ष्ट॒ दे॒वाः॒ । दे॒वा॒ इति॑ । इत्या॑ह । आ॒ह॒ हु॒ताद्या॑य । हु॒ताद्या॑य॒ यज॑मानस्य । हु॒ताद्या॒येति॑ हुत - अद्या॑य । यज॑मन॒स्याप॑राभावाय । अप॑राभावा॒येत्यप॑रा - भा॒वा॒य॒ । \newline

\textbf{Jatai Paata} \newline

1. आ वि॑शस्व विश॒स्वा वि॑शस्व । \newline
2. वि॒श॒स्व॒ शꣳ शं ॅवि॑शस्व विशस्व॒ शम् । \newline
3. शम् तो॒काय॑ तो॒काय॒ शꣳ शम् तो॒काय॑ । \newline
4. तो॒काय॑ त॒नुवे॑ त॒नुवे॑ तो॒काय॑ तो॒काय॑ त॒नुवे᳚ । \newline
5. त॒नुवे᳚ स्यो॒नः स्यो॒न स्त॒नुवे॑ त॒नुवे᳚ स्यो॒नः । \newline
6. स्यो॒न इति॑ स्यो॒नः । \newline
7. अज्या॑नी रे॒ता ए॒ता अज्या॑नी॒ रज्या॑नी रे॒ताः । \newline
8. ए॒ता उपोपै॒ता ए॒ता उप॑ । \newline
9. उप॑ दधाति दधा॒ त्युपोप॑ दधाति । \newline
10. द॒धा॒ त्ये॒ता ए॒ता द॑धाति दधा त्ये॒ताः । \newline
11. ए॒ता वै वा ए॒ता ए॒ता वै । \newline
12. वै दे॒वता॑ दे॒वता॒ वै वै दे॒वताः᳚ । \newline
13. दे॒वता॒ अप॑राजिता॒ अप॑राजिता दे॒वता॑ दे॒वता॒ अप॑राजिताः । \newline
14. अप॑राजिता॒ स्ता स्ता अप॑राजिता॒ अप॑राजिता॒ स्ताः । \newline
15. अप॑राजिता॒ इत्यप॑रा - जि॒ताः॒ । \newline
16. ता ए॒वैव ता स्ता ए॒व । \newline
17. ए॒व प्र प्रैवैव प्र । \newline
18. प्र वि॑शति विशति॒ प्र प्र वि॑शति । \newline
19. वि॒श॒ति॒ न न वि॑शति विशति॒ न । \newline
20. नैवैव न नैव । \newline
21. ए॒व जी॑यते जीयत ए॒वैव जी॑यते । \newline
22. जी॒य॒ते॒ ब्र॒ह्म॒वा॒दिनो᳚ ब्रह्मवा॒दिनो॑ जीयते जीयते ब्रह्मवा॒दिनः॑ । \newline
23. ब्र॒ह्म॒वा॒दिनो॑ वदन्ति वदन्ति ब्रह्मवा॒दिनो᳚ ब्रह्मवा॒दिनो॑ वदन्ति । \newline
24. ब्र॒ह्म॒वा॒दिन॒ इति॑ ब्रह्म - वा॒दिनः॑ । \newline
25. व॒द॒न्ति॒ यद् यद् व॑दन्ति वदन्ति॒ यत् । \newline
26. यद॑र्द्धमा॒सा अ॑र्द्धमा॒सा यद् यद॑र्द्धमा॒साः । \newline
27. अ॒र्द्ध॒मा॒सा मासा॒ मासा॑ अर्द्धमा॒सा अ॑र्द्धमा॒सा मासाः᳚ । \newline
28. अ॒र्द्ध॒मा॒सा इत्य॑र्द्ध - मा॒साः । \newline
29. मासा॑ ऋ॒तव॑ ऋ॒तवो॒ मासा॒ मासा॑ ऋ॒तवः॑ । \newline
30. ऋ॒तवः॑ संॅवथ्स॒रः सं॑ॅवथ्स॒र ऋ॒तव॑ ऋ॒तवः॑ संॅवथ्स॒रः । \newline
31. सं॒ॅव॒थ्स॒र ओष॑धी॒ रोष॑धीः संॅवथ्स॒रः सं॑ॅवथ्स॒र ओष॑धीः । \newline
32. सं॒ॅव॒थ्स॒र इति॑ सं - व॒थ्स॒रः । \newline
33. ओष॑धीः॒ पच॑न्ति॒ पच॒न् त्योष॑धी॒ रोष॑धीः॒ पच॑न्ति । \newline
34. पच॒न् त्यथाथ॒ पच॑न्ति॒ पच॒न् त्यथ॑ । \newline
35. अथ॒ कस्मा॒त् कस्मा॒ दथाथ॒ कस्मा᳚त् । \newline
36. कस्मा॑ द॒न्याभ्यो॒ ऽन्याभ्यः॒ कस्मा॒त् कस्मा॑ द॒न्याभ्यः॑ । \newline
37. अ॒न्याभ्यो॑ दे॒वता᳚भ्यो दे॒वता᳚भ्यो॒ ऽन्याभ्यो॒ ऽन्याभ्यो॑ दे॒वता᳚भ्यः । \newline
38. दे॒वता᳚भ्य आग्रय॒ण मा᳚ग्रय॒णम् दे॒वता᳚भ्यो दे॒वता᳚भ्य आग्रय॒णम् । \newline
39. आ॒ग्र॒य॒णम् निर् णिरा᳚ग्रय॒ण मा᳚ग्रय॒णम् निः । \newline
40. निरु॑प्यत उप्यते॒ निर् णिरु॑प्यते । \newline
41. उ॒प्य॒त॒ इती त्यु॑प्यत उप्यत॒ इति॑ । \newline
42. इत्ये॒ता ए॒ता इतीत्ये॒ताः । \newline
43. ए॒ता हि ह्ये॑ता ए॒ता हि । \newline
44. हि तत् तद्धि हि तत् । \newline
45. तद् दे॒वता॑ दे॒वता॒ स्तत् तद् दे॒वताः᳚ । \newline
46. दे॒वता॑ उ॒दज॑यन् नु॒दज॑यन् दे॒वता॑ दे॒वता॑ उ॒दज॑यन्न् । \newline
47. उ॒दज॑य॒न्॒. यद् यदु॒दज॑यन् नु॒दज॑य॒न्॒. यत् । \newline
48. उ॒दज॑य॒न्नित्यु॑त् - अज॑यन्न् । \newline
49. यदृ॒तुभ्य॑ ऋ॒तुभ्यो॒ यद् यदृ॒तुभ्यः॑ । \newline
50. ऋ॒तुभ्यो॑ नि॒र्वपे᳚न् नि॒र्वपे॑ दृ॒तुभ्य॑ ऋ॒तुभ्यो॑ नि॒र्वपे᳚त् । \newline
51. ऋ॒तुभ्य॒ इत्यृ॒तु - भ्यः॒ । \newline
52. नि॒र्वपे᳚द् दे॒वता᳚भ्यो दे॒वता᳚भ्यो नि॒र्वपे᳚न् नि॒र्वपे᳚द् दे॒वता᳚भ्यः । \newline
53. नि॒र्वपे॒दिति॑ निः - वपे᳚त् । \newline
54. दे॒वता᳚भ्यः स॒मदꣳ॑ स॒मद॑म् दे॒वता᳚भ्यो दे॒वता᳚भ्यः स॒मद᳚म् । \newline
55. स॒मद॑म् दद्ध्याद् दद्ध्याथ् स॒मदꣳ॑ स॒मद॑म् दद्ध्यात् । \newline
56. स॒मद॒मिति॑ स - मद᳚म् । \newline
57. द॒द्ध्या॒ दा॒ग्र॒य॒ण मा᳚ग्रय॒णम् द॑द्ध्याद् दद्ध्या दाग्रय॒णम् । \newline
58. आ॒ग्र॒य॒णम् नि॒रुप्य॑ नि॒रुप्या᳚ ग्रय॒ण मा᳚ग्रय॒णम् नि॒रुप्य॑ । \newline
59. नि॒रुप्यै॒ता ए॒ता नि॒रुप्य॑ नि॒रुप्यै॒ताः । \newline
60. नि॒रुप्येति॑ निः - उप्य॑ । \newline
61. ए॒ता आहु॑ती॒ राहु॑ती रे॒ता ए॒ता आहु॑तीः । \newline
62. आहु॑तीर् जुहोति जुहो॒ त्याहु॑ती॒ राहु॑तीर् जुहोति । \newline
63. आहु॑ती॒रित्या - हु॒तीः॒ । \newline
64. जु॒हो॒ त्य॒र्द्ध॒मा॒सा न॑र्द्धमा॒सानि जु॑होति जुहो त्यर्द्धमा॒सानि । \newline
65. अ॒र्द्ध॒मा॒सा ने॑वैवा र्द्ध॑मा॒सा न॑र्द्धमा॒सा ने॑व । \newline
66. अ॒र्द्ध॒मा॒सानित्य॑र्ध - मा॒सान् । \newline
67. ए॒व मासा॒न् मासा॑ ने॒वैव मासान्॑ । \newline
68. मासा॑ नृ॒तू नृ॒तून् मासा॒न् मासा॑ नृ॒तून् । \newline
69. ऋ॒तून् थ्सं॑ॅवथ्स॒रꣳ सं॑ॅवथ्स॒र मृ॒तू नृ॒तून् थ्सं॑ॅवथ्स॒रम् । \newline
70. सं॒ॅव॒थ्स॒रम् प्री॑णाति प्रीणाति संॅवथ्स॒रꣳ सं॑ॅवथ्स॒रम् प्री॑णाति । \newline
71. सं॒ॅव॒थ्स॒रमिति॑ सं - व॒थ्स॒रम् । \newline
72. प्री॒णा॒ति॒ न न प्री॑णाति प्रीणाति॒ न । \newline
73. न दे॒वता᳚भ्यो दे॒वता᳚भ्यो॒ न न दे॒वता᳚भ्यः । \newline
74. दे॒वता᳚भ्यः स॒मदꣳ॑ स॒मद॑म् दे॒वता᳚भ्यो दे॒वता᳚भ्यः स॒मद᳚म् । \newline
75. स॒मद॑म् दधाति दधाति स॒मदꣳ॑ स॒मद॑म् दधाति । \newline
76. स॒मद॒मिति॑ स - मद᳚म् । \newline
77. द॒धा॒ति॒ भ॒द्राद् भ॒द्राद् द॑धाति दधाति भ॒द्रात् । \newline
78. भ॒द्रान् नो॑ नो भ॒द्राद् भ॒द्रान् नः॑ । \newline
79. नः॒ श्रेयः॒ श्रेयो॑ नो नः॒ श्रेयः॑ । \newline
80. श्रेयः॒ सꣳ सꣳ श्रेयः॒ श्रेयः॒ सम् । \newline
81. स म॑नैष्टा नैष्ट॒ सꣳ स म॑नैष्ट । \newline
82. अ॒नै॒ष्ट॒ दे॒वा॒ दे॒वा॒ अ॒नै॒ष्टा॒ नै॒ष्ट॒ दे॒वाः॒ । \newline
83. दे॒वा॒ इतीति॑ देवा देवा॒ इति॑ । \newline
84. इत्या॑हा॒हे तीत्या॑ह । \newline
85. आ॒ह॒ हु॒ताद्या॑य हु॒ताद्या॑ याहाह हु॒ताद्या॑य । \newline
86. हु॒ताद्या॑य॒ यज॑मानस्य॒ यज॑मानस्य हु॒ताद्या॑य हु॒ताद्या॑य॒ यज॑मानस्य । \newline
87. हु॒ताद्या॒येति॑ हुत - अद्या॑य । \newline
88. यज॑मान॒स्या प॑राभावा॒या प॑राभावाय॒ यज॑मानस्य॒ यज॑मान॒स्या प॑राभावाय । \newline
89. अप॑राभावा॒येत्यप॑रा - भा॒वा॒य॒ । \newline

\textbf{Ghana Paata } \newline

1. आ वि॑शस्व विश॒स्वा वि॑शस्व॒ शꣳ शं ॅवि॑श॒स्वा वि॑शस्व॒ शम् । \newline
2. वि॒श॒स्व॒ शꣳ शं ॅवि॑शस्व विशस्व॒ शम् तो॒काय॑ तो॒काय॒ शं ॅवि॑शस्व विशस्व॒ शम् तो॒काय॑ । \newline
3. शम् तो॒काय॑ तो॒काय॒ शꣳ शम् तो॒काय॑ त॒नुवे॑ त॒नुवे॑ तो॒काय॒ शꣳ शम् तो॒काय॑ त॒नुवे᳚ । \newline
4. तो॒काय॑ त॒नुवे॑ त॒नुवे॑ तो॒काय॑ तो॒काय॑ त॒नुवे᳚ स्यो॒नः स्यो॒न स्त॒नुवे॑ तो॒काय॑ तो॒काय॑ त॒नुवे᳚ स्यो॒नः । \newline
5. त॒नुवे᳚ स्यो॒नः स्यो॒न स्त॒नुवे॑ त॒नुवे᳚ स्यो॒नः । \newline
6. स्यो॒न इति॑ स्यो॒नः । \newline
7. अज्या॑नी रे॒ता ए॒ता अज्या॑नी॒ रज्या॑नी रे॒ता उपोपै॒ता अज्या॑नी॒ रज्या॑नी रे॒ता उप॑ । \newline
8. ए॒ता उपोपै॒ता ए॒ता उप॑ दधाति दधा॒ त्युपै॒ता ए॒ता उप॑ दधाति । \newline
9. उप॑ दधाति दधा॒ त्युपोप॑ दधा त्ये॒ता ए॒ता द॑धा॒ त्युपोप॑ दधा त्ये॒ताः । \newline
10. द॒धा॒ त्ये॒ता ए॒ता द॑धाति दधा त्ये॒ता वै वा ए॒ता द॑धाति दधा त्ये॒ता वै । \newline
11. ए॒ता वै वा ए॒ता ए॒ता वै दे॒वता॑ दे॒वता॒ वा ए॒ता ए॒ता वै दे॒वताः᳚ । \newline
12. वै दे॒वता॑ दे॒वता॒ वै वै दे॒वता॒ अप॑राजिता॒ अप॑राजिता दे॒वता॒ वै वै दे॒वता॒ अप॑राजिताः । \newline
13. दे॒वता॒ अप॑राजिता॒ अप॑राजिता दे॒वता॑ दे॒वता॒ अप॑राजिता॒ स्ता स्ता अप॑राजिता दे॒वता॑ दे॒वता॒ अप॑राजिता॒ स्ताः । \newline
14. अप॑राजिता॒ स्ता स्ता अप॑राजिता॒ अप॑राजिता॒ स्ता ए॒वैव ता अप॑राजिता॒ अप॑राजिता॒ स्ता ए॒व । \newline
15. अप॑राजिता॒ इत्यप॑रा - जि॒ताः॒ । \newline
16. ता ए॒वैव ता स्ता ए॒व प्र प्रैव ता स्ता ए॒व प्र । \newline
17. ए॒व प्र प्रैवैव प्र वि॑शति विशति॒ प्रैवैव प्र वि॑शति । \newline
18. प्र वि॑शति विशति॒ प्र प्र वि॑शति॒ न न वि॑शति॒ प्र प्र वि॑शति॒ न । \newline
19. वि॒श॒ति॒ न न वि॑शति विशति॒ नैवैव न वि॑शति विशति॒ नैव । \newline
20. नैवैव न नैव जी॑यते जीयत ए॒व न नैव जी॑यते । \newline
21. ए॒व जी॑यते जीयत ए॒वैव जी॑यते ब्रह्मवा॒दिनो᳚ ब्रह्मवा॒दिनो॑ जीयत ए॒वैव जी॑यते ब्रह्मवा॒दिनः॑ । \newline
22. जी॒य॒ते॒ ब्र॒ह्म॒वा॒दिनो᳚ ब्रह्मवा॒दिनो॑ जीयते जीयते ब्रह्मवा॒दिनो॑ वदन्ति वदन्ति ब्रह्मवा॒दिनो॑ जीयते जीयते ब्रह्मवा॒दिनो॑ वदन्ति । \newline
23. ब्र॒ह्म॒वा॒दिनो॑ वदन्ति वदन्ति ब्रह्मवा॒दिनो᳚ ब्रह्मवा॒दिनो॑ वदन्ति॒ यद् यद् व॑दन्ति ब्रह्मवा॒दिनो᳚ ब्रह्मवा॒दिनो॑ वदन्ति॒ यत् । \newline
24. ब्र॒ह्म॒वा॒दिन॒ इति॑ ब्रह्म - वा॒दिनः॑ । \newline
25. व॒द॒न्ति॒ यद् यद् व॑दन्ति वदन्ति॒ यद॑र्द्धमा॒सा अ॑र्द्धमा॒सा यद् व॑दन्ति वदन्ति॒ यद॑र्द्धमा॒साः । \newline
26. यद॑र्द्धमा॒सा अ॑र्द्धमा॒सा यद् यद॑र्द्धमा॒सा मासा॒ मासा॑ अर्द्धमा॒सा यद् यद॑र्द्धमा॒सा मासाः᳚ । \newline
27. अ॒र्द्ध॒मा॒सा मासा॒ मासा॑ अर्द्धमा॒सा अ॑र्द्धमा॒सा मासा॑ ऋ॒तव॑ ऋ॒तवो॒ मासा॑ अर्द्धमा॒सा अ॑र्द्धमा॒सा मासा॑ ऋ॒तवः॑ । \newline
28. अ॒र्द्ध॒मा॒सा इत्य॑र्द्ध - मा॒साः । \newline
29. मासा॑ ऋ॒तव॑ ऋ॒तवो॒ मासा॒ मासा॑ ऋ॒तवः॑ संॅवथ्स॒रः सं॑ॅवथ्स॒र ऋ॒तवो॒ मासा॒ मासा॑ ऋ॒तवः॑ संॅवथ्स॒रः । \newline
30. ऋ॒तवः॑ संॅवथ्स॒रः सं॑ॅवथ्स॒र ऋ॒तव॑ ऋ॒तवः॑ संॅवथ्स॒र ओष॑धी॒ रोष॑धीः संॅवथ्स॒र ऋ॒तव॑ ऋ॒तवः॑ संॅवथ्स॒र ओष॑धीः । \newline
31. सं॒ॅव॒थ्स॒र ओष॑धी॒ रोष॑धीः संॅवथ्स॒रः सं॑ॅवथ्स॒र ओष॑धीः॒ पच॑न्ति॒ पच॒न्त्यो ष॑धीः संॅवथ्स॒रः सं॑ॅवथ्स॒र ओष॑धीः॒ पच॑न्ति । \newline
32. सं॒ॅव॒थ्स॒र इति॑ सं - व॒थ्स॒रः । \newline
33. ओष॑धीः॒ पच॑न्ति॒ पच॒न्त्यो ष॑धी॒ रोष॑धीः॒ पच॒न्त्यथाथ॒ पच॒न्त्यो ष॑धी॒ रोष॑धीः॒ पच॒न्त्यथ॑ । \newline
34. पच॒न्त्य थाथ॒ पच॑न्ति॒ पच॒न्त्यथ॒ कस्मा॒त् कस्मा॒ दथ॒ पच॑न्ति॒ पच॒न्त्यथ॒ कस्मा᳚त् । \newline
35. अथ॒ कस्मा॒त् कस्मा॒ दथाथ॒ कस्मा॑ द॒न्याभ्यो॒ ऽन्याभ्यः॒ कस्मा॒ दथाथ॒ कस्मा॑ द॒न्याभ्यः॑ । \newline
36. कस्मा॑ द॒न्याभ्यो॒ ऽन्याभ्यः॒ कस्मा॒त् कस्मा॑ द॒न्याभ्यो॑ दे॒वता᳚भ्यो दे॒वता᳚भ्यो॒ ऽन्याभ्यः॒ कस्मा॒त् कस्मा॑ द॒न्याभ्यो॑ दे॒वता᳚भ्यः । \newline
37. अ॒न्याभ्यो॑ दे॒वता᳚भ्यो दे॒वता᳚भ्यो॒ ऽन्याभ्यो॒ ऽन्याभ्यो॑ दे॒वता᳚भ्य आग्रय॒ण मा᳚ग्रय॒णम् दे॒वता᳚भ्यो॒ ऽन्याभ्यो॒ ऽन्याभ्यो॑ दे॒वता᳚भ्य आग्रय॒णम् । \newline
38. दे॒वता᳚भ्य आग्रय॒ण मा᳚ग्रय॒णम् दे॒वता᳚भ्यो दे॒वता᳚भ्य आग्रय॒णम् निर् णिरा᳚ग्रय॒णम् दे॒वता᳚भ्यो दे॒वता᳚भ्य आग्रय॒णम् निः । \newline
39. आ॒ग्र॒य॒णम् निर् णिरा᳚ग्रय॒ण मा᳚ग्रय॒णम् निरु॑प्यत उप्यते॒ निरा᳚ग्रय॒ण मा᳚ग्रय॒णम् निरु॑प्यते । \newline
40. निरु॑प्यत उप्यते॒ निर् णिरु॑प्यत॒ इतीत्यु॑प्यते॒ निर् णिरु॑प्यत॒ इति॑ । \newline
41. उ॒प्य॒त॒ इतीत्यु॑प्यत उप्यत॒ इत्ये॒ता ए॒ता इत्यु॑प्यत उप्यत॒ इत्ये॒ताः । \newline
42. इत्ये॒ता ए॒ता इतीत्ये॒ता हि ह्ये॑ता इतीत्ये॒ता हि । \newline
43. ए॒ता हि ह्ये॑ता ए॒ता हि तत् तद्ध्ये॑ता ए॒ता हि तत् । \newline
44. हि तत् तद्धि हि तद् दे॒वता॑ दे॒वता॒ स्तद्धि हि तद् दे॒वताः᳚ । \newline
45. तद् दे॒वता॑ दे॒वता॒ स्तत् तद् दे॒वता॑ उ॒दज॑यन् नु॒दज॑यन् दे॒वता॒ स्तत् तद् दे॒वता॑ उ॒दज॑यन्न् । \newline
46. दे॒वता॑ उ॒दज॑यन् नु॒दज॑यन् दे॒वता॑ दे॒वता॑ उ॒दज॑य॒न्॒. यद् यदु॒दज॑यन् दे॒वता॑ दे॒वता॑ उ॒दज॑य॒न्॒. यत् । \newline
47. उ॒दज॑य॒न्॒. यद् यदु॒दज॑यन् नु॒दज॑य॒न्॒. यदृ॒तुभ्य॑ ऋ॒तुभ्यो॒ यदु॒दज॑यन् नु॒दज॑य॒न्॒. यदृ॒तुभ्यः॑ । \newline
48. उ॒दज॑य॒न्नित्यु॑त् - अज॑यन्न् । \newline
49. यदृ॒तुभ्य॑ ऋ॒तुभ्यो॒ यद् यदृ॒तुभ्यो॑ नि॒र्वपे᳚न् नि॒र्वपे॑ दृ॒तुभ्यो॒ यद् यदृ॒तुभ्यो॑ नि॒र्वपे᳚त् । \newline
50. ऋ॒तुभ्यो॑ नि॒र्वपे᳚न् नि॒र्वपे॑ दृ॒तुभ्य॑ ऋ॒तुभ्यो॑ नि॒र्वपे᳚द् दे॒वता᳚भ्यो दे॒वता᳚भ्यो नि॒र्वपे॑ दृ॒तुभ्य॑ ऋ॒तुभ्यो॑ नि॒र्वपे᳚द् दे॒वता᳚भ्यः । \newline
51. ऋ॒तुभ्य॒ इत्यृ॒तु - भ्यः॒ । \newline
52. नि॒र्वपे᳚द् दे॒वता᳚भ्यो दे॒वता᳚भ्यो नि॒र्वपे᳚न् नि॒र्वपे᳚द् दे॒वता᳚भ्यः स॒मदꣳ॑ स॒मद॑म् दे॒वता᳚भ्यो नि॒र्वपे᳚न् नि॒र्वपे᳚द् दे॒वता᳚भ्यः स॒मद᳚म् । \newline
53. नि॒र्वपे॒दिति॑ निः - वपे᳚त् । \newline
54. दे॒वता᳚भ्यः स॒मदꣳ॑ स॒मद॑म् दे॒वता᳚भ्यो दे॒वता᳚भ्यः स॒मद॑म् दद्ध्याद् दद्ध्याथ् स॒मद॑म् दे॒वता᳚भ्यो दे॒वता᳚भ्यः स॒मद॑म् दद्ध्यात् । \newline
55. स॒मद॑म् दद्ध्याद् दद्ध्याथ् स॒मदꣳ॑ स॒मद॑म् दद्ध्या दाग्रय॒ण मा᳚ग्रय॒णम् द॑द्ध्याथ् स॒मदꣳ॑ स॒मद॑म् दद्ध्या दाग्रय॒णम् । \newline
56. स॒मद॒मिति॑ स - मद᳚म् । \newline
57. द॒द्ध्या॒ दा॒ग्र॒य॒ण मा᳚ग्रय॒णम् द॑द्ध्याद् दद्ध्या दाग्रय॒णम् नि॒रुप्य॑ नि॒रुप्या᳚ग्रय॒णम् द॑द्ध्याद् दद्ध्या दाग्रय॒णम् नि॒रुप्य॑ । \newline
58. आ॒ग्र॒य॒णम् नि॒रुप्य॑ नि॒रुप्या᳚ ग्रय॒ण मा᳚ग्रय॒णम् नि॒रुप्यै॒ता ए॒ता नि॒रुप्या᳚ ग्रय॒ण मा᳚ग्रय॒णम् नि॒रुप्यै॒ताः । \newline
59. नि॒रुप्यै॒ता ए॒ता नि॒रुप्य॑ नि॒रुप्यै॒ता आहु॑ती॒ राहु॑ती रे॒ता नि॒रुप्य॑ नि॒रुप्यै॒ता आहु॑तीः । \newline
60. नि॒रुप्येति॑ निः - उप्य॑ । \newline
61. ए॒ता आहु॑ती॒ राहु॑ती रे॒ता ए॒ता आहु॑तीर् जुहोति जुहो॒ त्याहु॑ती रे॒ता ए॒ता आहु॑तीर् जुहोति । \newline
62. आहु॑तीर् जुहोति जुहो॒ त्याहु॑ती॒ राहु॑तीर् जुहो त्यर्द्धमा॒सा न॑र्द्धमा॒सान् जु॑हो॒ त्याहु॑ती॒ राहु॑तीर् जुहो त्यर्द्धमा॒सान् । \newline
63. आहु॑ती॒रित्या - हु॒तीः॒ । \newline
64. जु॒हो॒ त्य॒र्द्ध॒मा॒सा न॑र्द्धमा॒सान् जु॑होति जुहो त्यर्द्धमा॒सा ने॑वैवा र्द्ध॑मा॒सानि जु॑होति जुहो
त्यर्द्धमा॒सा ने॑व । \newline
65. अ॒र्द्ध॒मा॒सा ने॑वैवा र्द्ध॑मा॒सा न॑र्द्धमा॒सा ने॑व मासा॒न् मासा॑ ने॒वा र्द्ध॑मा॒सा
न॑र्द्धमा॒सा ने॑व मासान्॑ । \newline
66. अ॒र्द्ध॒मा॒सानित्य॑र्ध - मा॒सान् । \newline
67. ए॒व मासा॒न् मासा॑ ने॒वैव मासा॑ नृ॒तू नृ॒तून् मासा॑ ने॒वैव मासा॑ नृ॒तून् । \newline
68. मासा॑ नृ॒तू नृ॒तून् मासा॒न् मासा॑ नृ॒तून् थ्सं॑ॅवथ्स॒रꣳ सं॑ॅवथ्स॒र मृ॒तून् मासा॒न् मासा॑ नृ॒तून् थ्सं॑ॅवथ्स॒रम् । \newline
69. ऋ॒तून् थ्सं॑ॅवथ्स॒रꣳ सं॑ॅवथ्स॒र मृ॒तू नृ॒तून् थ्सं॑ॅवथ्स॒रम् प्री॑णाति प्रीणाति संॅवथ्स॒र मृ॒तू नृ॒तून् थ्सं॑ॅवथ्स॒रम् प्री॑णाति । \newline
70. सं॒ॅव॒थ्स॒रम् प्री॑णाति प्रीणाति संॅवथ्स॒रꣳ सं॑ॅवथ्स॒रम् प्री॑णाति॒ न न प्री॑णाति संॅवथ्स॒रꣳ सं॑ॅवथ्स॒रम् प्री॑णाति॒ न । \newline
71. सं॒ॅव॒थ्स॒रमिति॑ सं - व॒थ्स॒रम् । \newline
72. प्री॒णा॒ति॒ न न प्री॑णाति प्रीणाति॒ न दे॒वता᳚भ्यो दे॒वता᳚भ्यो॒ न प्री॑णाति प्रीणाति॒ न दे॒वता᳚भ्यः । \newline
73. न दे॒वता᳚भ्यो दे॒वता᳚भ्यो॒ न न दे॒वता᳚भ्यः स॒मदꣳ॑ स॒मद॑म् दे॒वता᳚भ्यो॒ न न दे॒वता᳚भ्यः स॒मद᳚म् । \newline
74. दे॒वता᳚भ्यः स॒मदꣳ॑ स॒मद॑म् दे॒वता᳚भ्यो दे॒वता᳚भ्यः स॒मद॑म् दधाति दधाति स॒मद॑म् दे॒वता᳚भ्यो दे॒वता᳚भ्यः स॒मद॑म् दधाति । \newline
75. स॒मद॑म् दधाति दधाति स॒मदꣳ॑ स॒मद॑म् दधाति भ॒द्राद् भ॒द्राद् द॑धाति स॒मदꣳ॑ स॒मद॑म् दधाति भ॒द्रात् । \newline
76. स॒मद॒मिति॑ स - मद᳚म् । \newline
77. द॒धा॒ति॒ भ॒द्राद् भ॒द्राद् द॑धाति दधाति भ॒द्रान् नो॑ नो भ॒द्राद् द॑धाति दधाति भ॒द्रान् नः॑ । \newline
78. भ॒द्रान् नो॑ नो भ॒द्राद् भ॒द्रान् नः॒ श्रेयः॒ श्रेयो॑ नो भ॒द्राद् भ॒द्रान् नः॒ श्रेयः॑ । \newline
79. नः॒ श्रेयः॒ श्रेयो॑ नो नः॒ श्रेयः॒ सꣳ सꣳ श्रेयो॑ नो नः॒ श्रेयः॒ सम् । \newline
80. श्रेयः॒ सꣳ सꣳ श्रेयः॒ श्रेयः॒ स म॑नैष्टा नैष्ट॒ सꣳ श्रेयः॒ श्रेयः॒ स म॑नैष्ट । \newline
81. स म॑नैष्टा नैष्ट॒ सꣳ स म॑नैष्ट देवा देवा अनैष्ट॒ सꣳ स म॑नैष्ट देवाः । \newline
82. अ॒नै॒ष्ट॒ दे॒वा॒ दे॒वा॒ अ॒नै॒ष्टा॒ नै॒ष्ट॒ दे॒वा॒ इतीति॑ देवा अनैष्टा नैष्ट देवा॒ इति॑ । \newline
83. दे॒वा॒ इतीति॑ देवा देवा॒ इत्या॑हा॒ हेति॑ देवा देवा॒ इत्या॑ह । \newline
84. इत्या॑हा॒हे तीत्या॑ह हु॒ताद्या॑य हु॒ताद्या॑ या॒हे तीत्या॑ह हु॒ताद्या॑य । \newline
85. आ॒ह॒ हु॒ताद्या॑य हु॒ताद्या॑ याहाह हु॒ताद्या॑य॒ यज॑मानस्य॒ यज॑मानस्य हु॒ताद्या॑ याहाह हु॒ताद्या॑य॒ यज॑मानस्य । \newline
86. हु॒ताद्या॑य॒ यज॑मानस्य॒ यज॑मानस्य हु॒ताद्या॑य हु॒ताद्या॑य॒ यज॑मान॒स्या प॑राभावा॒या प॑राभावाय॒ यज॑मानस्य हु॒ताद्या॑य हु॒ताद्या॑य॒ यज॑मान॒स्या प॑राभावाय । \newline
87. हु॒ताद्या॒येति॑ हुत - अद्या॑य । \newline
88. यज॑मान॒स्या प॑राभावा॒या प॑राभावाय॒ यज॑मानस्य॒ यज॑मान॒स्या प॑राभावाय । \newline
89. अप॑राभावा॒येत्यप॑रा - भा॒वा॒य॒ । \newline
\pagebreak
\markright{ TS 5.7.3.1  \hfill https://www.vedavms.in \hfill}

\section{ TS 5.7.3.1 }

\textbf{TS 5.7.3.1 } \newline
\textbf{Samhita Paata} \newline

इन्द्र॑स्य॒ वज्रो॑ऽसि॒ वार्त्र॑घ्नस्तनू॒पा नः॑ प्रतिस्प॒शः । यो नः॑ पु॒रस्ता᳚द् दक्षिण॒तः प॒श्चा-दु॑त्तर॒तो॑-ऽघा॒युर॑भि॒दास॑त्ये॒तꣳ सोऽश्मा॑नमृच्छतु ॥ दे॒वा॒सु॒राः संॅय॑त्ता आस॒न् तेऽसु॑रा दि॒ग्भ्य आऽबा॑धन्त॒ तान् दे॒वा इष्वा॑ च॒ वज्रे॑ण॒ चापा॑नुदन्त॒ यद्-व॒ज्रिणी॑रुप॒दधा॒तीष्वा॑ चै॒व तद्-वज्रे॑ण च॒ यज॑मानो॒ भ्रातृ॑व्या॒नप॑ नुदते दि॒क्षूप॑ - [  ] \newline

\textbf{Pada Paata} \newline

इन्द्र॑स्य । वज्रः॑ । अ॒सि॒ । वार्त्र॑घ्न॒ इति॒ वार्त्र॑ - घ्नः॒ । त॒नू॒पा इति॑ तनू - पाः । नः॒ । प्र॒ति॒स्प॒श इति॑ प्रति - स्प॒शः ॥ यः । नः॒ । पु॒रस्ता᳚त् । द॒क्षि॒ण॒तः । प॒श्चात् । उ॒त्त॒र॒त इत्यु॑त् - त॒र॒तः । अ॒घा॒युरित्य॑घ - युः । अ॒भि॒दास॒तीत्य॑भि - दास॑ति । ए॒तम् । सः । अश्मा॑नम् । ऋ॒च्छ॒तु॒ ॥ दे॒वा॒सु॒रा इति॑ देव - अ॒सु॒राः । संॅय॑त्ता॒ इति॒ सं - य॒त्ताः॒ । आ॒स॒न्न् । ते । असु॑राः । दि॒ग्भ्य इति॑ दिक्-भ्यः । एति॑ । अ॒बा॒ध॒न्त॒ । तान् । दे॒वाः । इष्वा᳚ । च॒ । वज्रे॑ण । च॒ । अपेति॑ । अ॒नु॒द॒न्त॒ । यत् । व॒ज्रिणीः᳚ । उ॒प॒दधा॒तीत्यु॑प-दधा॑ति । इष्वा᳚ । च॒ । ए॒व । तत् । वज्रे॑ण । च॒ । यज॑मानः । भ्रातृ॑व्यान् । अपेति॑ । नु॒द॒ते॒ । दि॒क्षु । उपेति॑ ।  \newline


\textbf{Krama Paata} \newline

इन्द्र॑स्य॒ वज्रः॑ । वज्रो॑ऽसि । अ॒सि॒ वार्त्र॑घ्नः । वार्त्र॑घ्नस्तनू॒पाः । वार्त्र॑घ्न॒ इति॒ वार्त्र॑ - घ्नः॒ । त॒नू॒पा नः॑ । त॒नू॒पा इति॑ तनू - पाः । नः॒ प्र॒ति॒स्प॒शः । प्र॒ति॒स्प॒श इति॑ प्रति - स्प॒शः ॥ यो नः॑ । नः॒ पु॒रस्ता᳚त् । पु॒रस्ता᳚द् दक्षिण॒तः । द॒क्षि॒ण॒तः प॒श्चात् । प॒श्चादु॑त्तर॒तः । उ॒त्त॒र॒तो॑ऽघा॒युः । उ॒त्त॒र॒त इत्यु॑त् - त॒र॒तः । अ॒घा॒युर॑भि॒दास॑ति । अ॒घा॒युरित्य॑घ - युः । अ॒भि॒दास॑त्ये॒तम् । अ॒भि॒दास॒तीत्य॑भि - दास॑ति । ए॒तꣳ सः । सोऽश्मा॑नम् । अश्मा॑नमृच्छतु । ऋ॒च्छ॒त्वित्यृ॑च्छतु ॥ दे॒वा॒सु॒राः सम्ॅय॑त्ताः । दे॒वा॒सु॒रा इति॑ देव - अ॒सु॒राः । सम्ॅय॑त्ता आसन्न् । सम्ॅय॑त्ता॒ इति॒ सम् - य॒त्ताः॒ । आ॒स॒न् ते । तेऽसु॑राः । असु॑रा दि॒ग्भ्यः । दि॒ग्भ्य आ । दि॒ग्भ्य इति॑ दिक् - भ्यः । आऽबा॑धन्त । अ॒बा॒ध॒न्त॒ तान् । तान् दे॒वाः । दे॒वा इष्वा᳚ । इष्वा॑ च । च॒ वज्रे॑ण । वज्रे॑ण च । चाप॑ । अपा॑नुदन्त । अ॒नु॒द॒न्त॒ यत् । यद् व॒ज्रिणीः᳚ । व॒ज्रिणी॑रुप॒दधा॑ति । उ॒प॒दधा॒तीष्वा᳚ । उ॒प॒दधा॒तीत्यु॑प - दधा॑ति । इष्वा॑ च । चै॒व । ए॒व तत् । तद् वज्रे॑ण । वज्रे॑ण च । च॒ यज॑मानः । यज॑मानो॒ भ्रातृ॑व्यान् । भ्रात्य॑व्या॒नप॑ । अप॑ नुदते । नु॒द॒ते॒ दि॒क्षु । दि॒क्षूप॑ । उप॑ दधाति \newline

\textbf{Jatai Paata} \newline

1. इन्द्र॑स्य॒ वज्रो॒ वज्र॒ इन्द्र॒ स्येन्द्र॑स्य॒ वज्रः॑ । \newline
2. वज्रो᳚ ऽस्यसि॒ वज्रो॒ वज्रो॑ ऽसि । \newline
3. अ॒सि॒ वार्त्र॑घ्नो॒ वार्त्र॑घ्नो ऽस्यसि॒ वार्त्र॑घ्नः । \newline
4. वार्त्र॑घ्न स्तनू॒पा स्त॑नू॒पा वार्त्र॑घ्नो॒ वार्त्र॑घ्न स्तनू॒पाः । \newline
5. वार्त्र॑घ्न॒ इति॒ वार्त्र॑ - घ्नः॒ । \newline
6. त॒नू॒पा नो॑ न स्तनू॒पा स्त॑नू॒पा नः॑ । \newline
7. त॒नू॒पा इति॑ तनू - पाः । \newline
8. नः॒ प्र॒ति॒स्प॒शः प्र॑तिस्प॒शो नो॑ नः प्रतिस्प॒शः । \newline
9. प्र॒ति॒स्प॒श इति॑ प्रति - स्प॒शः । \newline
10. यो नो॑ नो॒ यो यो नः॑ । \newline
11. नः॒ पु॒रस्ता᳚त् पु॒रस्ता᳚न् नो नः पु॒रस्ता᳚त् । \newline
12. पु॒रस्ता᳚द् दक्षिण॒तो द॑क्षिण॒तः पु॒रस्ता᳚त् पु॒रस्ता᳚द् दक्षिण॒तः । \newline
13. द॒क्षि॒ण॒तः प॒श्चात् प॒श्चाद् द॑क्षिण॒तो द॑क्षिण॒तः प॒श्चात् । \newline
14. प॒श्चा दु॑त्तर॒त उ॑त्तर॒तः प॒श्चात् प॒श्चा दु॑त्तर॒तः । \newline
15. उ॒त्त॒र॒तो॑ ऽघा॒यु र॑घा॒यु रु॑त्तर॒त उ॑त्तर॒तो॑ ऽघा॒युः । \newline
16. उ॒त्त॒र॒त इत्यु॑त् - त॒र॒तः । \newline
17. अ॒घा॒यु र॑भि॒दास॑ त्यभि॒दास॑ त्यघा॒यु र॑घा॒यु र॑भि॒दास॑ति । \newline
18. अ॒घा॒युरित्य॑घ - युः । \newline
19. अ॒भि॒दास॑ त्ये॒त मे॒त म॑भि॒दास॑ त्यभि॒दास॑ त्ये॒तम् । \newline
20. अ॒भि॒दास॒तीत्य॑भि - दास॑ति । \newline
21. ए॒तꣳ स स ए॒त मे॒तꣳ सः । \newline
22. सो ऽश्मा॑न॒ मश्मा॑नꣳ॒॒ स सो ऽश्मा॑नम् । \newline
23. अश्मा॑न मृच्छ त्वृच्छ॒ त्वश्मा॑न॒ मश्मा॑न मृच्छतु । \newline
24. ऋ॒च्छ॒त्वित्यृ॑च्छतु । \newline
25. दे॒वा॒सु॒राः संॅय॑त्ताः॒ संॅय॑त्ता देवासु॒रा दे॑वासु॒राः संॅय॑त्ताः । \newline
26. दे॒वा॒सु॒रा इति॑ देव - अ॒सु॒राः । \newline
27. संॅय॑त्ता आसन् नास॒न् थ्संॅय॑त्ताः॒ संॅय॑त्ता आसन्न् । \newline
28. संॅय॑त्ता॒ इति॒ सं - य॒त्ताः॒ । \newline
29. आ॒स॒न् ते त आ॑सन् नास॒न् ते । \newline
30. ते ऽसु॑रा॒ असु॑रा॒ स्ते ते ऽसु॑राः । \newline
31. असु॑रा दि॒ग्भ्यो दि॒ग्भ्यो ऽसु॑रा॒ असु॑रा दि॒ग्भ्यः । \newline
32. दि॒ग्भ्य आ दि॒ग्भ्यो दि॒ग्भ्य आ । \newline
33. दि॒ग्भ्य इति॑ दिक् - भ्यः । \newline
34. आ ऽबा॑धन्ता बाध॒न्ता ऽबा॑धन्त । \newline
35. अ॒बा॒ध॒न्त॒ ताꣳ स्ता न॑बाधन्ता बाधन्त॒ तान् । \newline
36. तान् दे॒वा दे॒वा स्ताꣳ स्तान् दे॒वाः । \newline
37. दे॒वा इष्वेष्वा॑ दे॒वा दे॒वा इष्वा᳚ । \newline
38. इष्वा॑ च॒ चेष्वेष्वा॑ च । \newline
39. च॒ वज्रे॑ण॒ वज्रे॑ण च च॒ वज्रे॑ण । \newline
40. वज्रे॑ण च च॒ वज्रे॑ण॒ वज्रे॑ण च । \newline
41. चापाप॑ च॒ चाप॑ । \newline
42. अपा॑ नुदन्ता नुद॒न्ता पापा॑ नुदन्त । \newline
43. अ॒नु॒द॒न्त॒ यद् यद॑नुदन्ता नुदन्त॒ यत् । \newline
44. यद् व॒ज्रिणी᳚र् व॒ज्रिणी॒र् यद् यद् व॒ज्रिणीः᳚ । \newline
45. व॒ज्रिणी॑ रुप॒दधा᳚ त्युप॒दधा॑ति व॒ज्रिणी᳚र् व॒ज्रिणी॑ रुप॒दधा॑ति । \newline
46. उ॒प॒दधा॒ तीष्वे ष्वो॑प॒दधा᳚ त्युप॒दधा॒ तीष्वा᳚ । \newline
47. उ॒प॒दधा॒तीत्यु॑प - दधा॑ति । \newline
48. इष्वा॑ च॒ चेष्वेष्वा॑ च । \newline
49. चै॒वैव च॑ चै॒व । \newline
50. ए॒व तत् तदे॒ वैव तत् । \newline
51. तद् वज्रे॑ण॒ वज्रे॑ण॒ तत् तद् वज्रे॑ण । \newline
52. वज्रे॑ण च च॒ वज्रे॑ण॒ वज्रे॑ण च । \newline
53. च॒ यज॑मानो॒ यज॑मानश्च च॒ यज॑मानः । \newline
54. यज॑मानो॒ भ्रातृ॑व्या॒न् भ्रातृ॑व्या॒न्॒. यज॑मानो॒ यज॑मानो॒ भ्रातृ॑व्यान् । \newline
55. भ्रातृ॑व्या॒ नपाप॒ भ्रातृ॑व्या॒न् भ्रातृ॑व्या॒ नप॑ । \newline
56. अप॑ नुदते नुद॒ते ऽपाप॑ नुदते । \newline
57. नु॒द॒ते॒ दि॒क्षु दि॒क्षु नु॑दते नुदते दि॒क्षु । \newline
58. दि॒क्षू पोप॑ दि॒क्षु दि॒क्षूप॑ । \newline
59. उप॑ दधाति दधा॒ त्युपोप॑ दधाति । \newline

\textbf{Ghana Paata } \newline

1. इन्द्र॑स्य॒ वज्रो॒ वज्र॒ इन्द्र॒ स्येन्द्र॑स्य॒ वज्रो᳚ ऽस्यसि॒ वज्र॒ इन्द्र॒ स्येन्द्र॑स्य॒ वज्रो॑ ऽसि । \newline
2. वज्रो᳚ ऽस्यसि॒ वज्रो॒ वज्रो॑ ऽसि॒ वार्त्र॑घ्नो॒ वार्त्र॑घ्नो ऽसि॒ वज्रो॒ वज्रो॑ ऽसि॒ वार्त्र॑घ्नः । \newline
3. अ॒सि॒ वार्त्र॑घ्नो॒ वार्त्र॑घ्नो ऽस्यसि॒ वार्त्र॑घ्न स्तनू॒पा स्त॑नू॒पा वार्त्र॑घ्नो ऽस्यसि॒ वार्त्र॑घ्न स्तनू॒पाः । \newline
4. वार्त्र॑घ्न स्तनू॒पा स्त॑नू॒पा वार्त्र॑घ्नो॒ वार्त्र॑घ्न स्तनू॒पा नो॑ न स्तनू॒पा वार्त्र॑घ्नो॒ वार्त्र॑घ्न स्तनू॒पा नः॑ । \newline
5. वार्त्र॑घ्न॒ इति॒ वार्त्र॑ - घ्नः॒ । \newline
6. त॒नू॒पा नो॑ न स्तनू॒पा स्त॑नू॒पा नः॑ प्रतिस्प॒शः प्र॑तिस्प॒शो न॑ स्तनू॒पा स्त॑नू॒पा नः॑ प्रतिस्प॒शः । \newline
7. त॒नू॒पा इति॑ तनू - पाः । \newline
8. नः॒ प्र॒ति॒स्प॒शः प्र॑तिस्प॒शो नो॑ नः प्रतिस्प॒शः । \newline
9. प्र॒ति॒स्प॒श इति॑ प्रति - स्प॒शः । \newline
10. यो नो॑ नो॒ यो यो नः॑ पु॒रस्ता᳚त् पु॒रस्ता᳚न् नो॒ यो यो नः॑ पु॒रस्ता᳚त् । \newline
11. नः॒ पु॒रस्ता᳚त् पु॒रस्ता᳚न् नो नः पु॒रस्ता᳚द् दक्षिण॒तो द॑क्षिण॒तः पु॒रस्ता᳚न् नो नः पु॒रस्ता᳚द् दक्षिण॒तः । \newline
12. पु॒रस्ता᳚द् दक्षिण॒तो द॑क्षिण॒तः पु॒रस्ता᳚त् पु॒रस्ता᳚द् दक्षिण॒तः प॒श्चात् प॒श्चाद् द॑क्षिण॒तः पु॒रस्ता᳚त् पु॒रस्ता᳚द् दक्षिण॒तः प॒श्चात् । \newline
13. द॒क्षि॒ण॒तः प॒श्चात् प॒श्चाद् द॑क्षिण॒तो द॑क्षिण॒तः प॒श्चा दु॑त्तर॒त उ॑त्तर॒तः प॒श्चाद् द॑क्षिण॒तो द॑क्षिण॒तः प॒श्चा दु॑त्तर॒तः । \newline
14. प॒श्चा दु॑त्तर॒त उ॑त्तर॒तः प॒श्चात् प॒श्चा दु॑त्तर॒तो॑ ऽघा॒यु र॑घा॒यु रु॑त्तर॒तः प॒श्चात् प॒श्चा दु॑त्तर॒तो॑ ऽघा॒युः । \newline
15. उ॒त्त॒र॒तो॑ ऽघा॒यु र॑घा॒यु रु॑त्तर॒त उ॑त्तर॒तो॑ ऽघा॒यु र॑भि॒दास॑ त्यभि॒दास॑ त्यघा॒यु रु॑त्तर॒त उ॑त्तर॒तो॑ ऽघा॒यु र॑भि॒दास॑ति । \newline
16. उ॒त्त॒र॒त इत्यु॑त् - त॒र॒तः । \newline
17. अ॒घा॒यु र॑भि॒दास॑ त्यभि॒दास॑ त्यघा॒यु र॑घा॒यु र॑भि॒दास॑ त्ये॒त मे॒त म॑भि॒दास॑ त्यघा॒युर॑ घा॒यु र॑भि॒दास॑ त्ये॒तम् । \newline
18. अ॒घा॒युरित्य॑घ - युः । \newline
19. अ॒भि॒दास॑ त्ये॒त मे॒त म॑भि॒दास॑ त्यभि॒दास॑ त्ये॒तꣳ स स ए॒त म॑भि॒दास॑ त्यभि॒दास॑ त्ये॒तꣳ सः । \newline
20. अ॒भि॒दास॒तीत्य॑भि - दास॑ति । \newline
21. ए॒तꣳ स स ए॒त मे॒तꣳ सो ऽश्मा॑न॒ मश्मा॑नꣳ॒॒ स ए॒त मे॒तꣳ सो ऽश्मा॑नम् । \newline
22. सो ऽश्मा॑न॒ मश्मा॑नꣳ॒॒ स सो ऽश्मा॑न मृच्छ त्वृच्छ॒ त्वश्मा॑नꣳ॒॒ स सो ऽश्मा॑न मृच्छतु । \newline
23. अश्मा॑न मृच्छ त्वृच्छ॒ त्वश्मा॑न॒ मश्मा॑न मृच्छतु । \newline
24. ऋ॒च्छ॒त्वित्यृ॑च्छतु । \newline
25. दे॒वा॒सु॒राः संॅय॑त्ताः॒ संॅय॑त्ता देवासु॒रा दे॑वासु॒राः संॅय॑त्ता आसन् नास॒न् थ्संॅय॑त्ता देवासु॒रा दे॑वासु॒राः संॅय॑त्ता आसन्न् । \newline
26. दे॒वा॒सु॒रा इति॑ देव - अ॒सु॒राः । \newline
27. संॅय॑त्ता आसन् नास॒न् थ्संॅय॑त्ताः॒ संॅय॑त्ता आस॒न् ते त आ॑स॒न् थ्संॅय॑त्ताः॒ संॅय॑त्ता आस॒न् ते । \newline
28. संॅय॑त्ता॒ इति॒ सं - य॒त्ताः॒ । \newline
29. आ॒स॒न् ते त आ॑सन् नास॒न् ते ऽसु॑रा॒ असु॑रा॒ स्त आ॑सन् नास॒न् ते ऽसु॑राः । \newline
30. ते ऽसु॑रा॒ असु॑रा॒ स्ते ते ऽसु॑रा दि॒ग्भ्यो दि॒ग्भ्यो ऽसु॑रा॒ स्ते ते ऽसु॑रा दि॒ग्भ्यः । \newline
31. असु॑रा दि॒ग्भ्यो दि॒ग्भ्यो ऽसु॑रा॒ असु॑रा दि॒ग्भ्य आ दि॒ग्भ्यो ऽसु॑रा॒ असु॑रा दि॒ग्भ्य आ । \newline
32. दि॒ग्भ्य आ दि॒ग्भ्यो दि॒ग्भ्य आ ऽबा॑धन्ता बाध॒न्ता दि॒ग्भ्यो दि॒ग्भ्य आ ऽबा॑धन्त । \newline
33. दि॒ग्भ्य इति॑ दिक् - भ्यः । \newline
34. आ ऽबा॑धन्ता बाध॒न्ता ऽबा॑धन्त॒ ताꣳस्ता न॑बाध॒न्ता ऽबा॑धन्त॒ तान् । \newline
35. अ॒बा॒ध॒न्त॒ ताꣳ स्ता न॑बाधन्ता बाधन्त॒ तान् दे॒वा दे॒वा स्ता न॑बाधन्ता बाधन्त॒ तान् दे॒वाः । \newline
36. तान् दे॒वा दे॒वा स्ताꣳ स्तान् दे॒वा इष्वेष्वा॑ दे॒वा स्ताꣳ स्तान् दे॒वा इष्वा᳚ । \newline
37. दे॒वा इष्वेष्वा॑ दे॒वा दे॒वा इष्वा॑ च॒ चेष्वा॑ दे॒वा दे॒वा इष्वा॑ च । \newline
38. इष्वा॑ च॒ चेष्वेष्वा॑ च॒ वज्रे॑ण॒ वज्रे॑ण॒ चेष्वेष्वा॑ च॒ वज्रे॑ण । \newline
39. च॒ वज्रे॑ण॒ वज्रे॑ण च च॒ वज्रे॑ण च च॒ वज्रे॑ण च च॒ वज्रे॑ण च । \newline
40. वज्रे॑ण च च॒ वज्रे॑ण॒ वज्रे॑ण॒ चापाप॑ च॒ वज्रे॑ण॒ वज्रे॑ण॒ चाप॑ । \newline
41. चापाप॑ च॒ चापा॑ नुदन्ता नुद॒न्ताप॑ च॒ चापा॑ नुदन्त । \newline
42. अपा॑ नुदन्ता नुद॒न्ता पापा॑ नुदन्त॒ यद् यद॑नुद॒न्ता पापा॑ नुदन्त॒ यत् । \newline
43. अ॒नु॒द॒न्त॒ यद् यद॑नुदन्ता नुदन्त॒ यद् व॒ज्रिणी᳚र् व॒ज्रिणी॒र् यद॑नुदन्ता नुदन्त॒ यद् व॒ज्रिणीः᳚ । \newline
44. यद् व॒ज्रिणी᳚र् व॒ज्रिणी॒र् यद् यद् व॒ज्रिणी॑ रुप॒दधा᳚ त्युप॒दधा॑ति व॒ज्रिणी॒र् यद् यद् व॒ज्रिणी॑ रुप॒दधा॑ति । \newline
45. व॒ज्रिणी॑ रुप॒दधा᳚ त्युप॒दधा॑ति व॒ज्रिणी᳚र् व॒ज्रिणी॑ रुप॒दधा॒ तीष्वे ष्वो॑ प॒दधा॑ति व॒ज्रिणी᳚र् व॒ज्रिणी॑ रुप॒दधा॒ तीष्वा᳚ । \newline
46. उ॒प॒दधा॒ तीष्वे ष्वो॑प॒दधा᳚ त्युप॒दधा॒ तीष्वा॑ च॒ चेष्वो॑प॒दधा᳚ त्युप॒दधा॒ तीष्वा॑ च । \newline
47. उ॒प॒दधा॒तीत्यु॑प - दधा॑ति । \newline
48. इष्वा॑ च॒ चेष्वे ष्वा॑ चै॒वैव चेष्वेष्वा॑ चै॒व । \newline
49. चै॒वैव च॑ चै॒व तत् तदे॒व च॑ चै॒व तत् । \newline
50. ए॒व तत् तदे॒वैव तद् वज्रे॑ण॒ वज्रे॑ण॒ तदे॒वैव तद् वज्रे॑ण । \newline
51. तद् वज्रे॑ण॒ वज्रे॑ण॒ तत् तद् वज्रे॑ण च च॒ वज्रे॑ण॒ तत् तद् वज्रे॑ण च । \newline
52. वज्रे॑ण च च॒ वज्रे॑ण॒ वज्रे॑ण च॒ यज॑मानो॒ यज॑मानश्च॒ वज्रे॑ण॒ वज्रे॑ण च॒ यज॑मानः । \newline
53. च॒ यज॑मानो॒ यज॑मानश्च च॒ यज॑मानो॒ भ्रातृ॑व्या॒न् भ्रातृ॑व्या॒न्॒. यज॑मानश्च च॒ यज॑मानो॒ भ्रातृ॑व्यान् । \newline
54. यज॑मानो॒ भ्रातृ॑व्या॒न् भ्रातृ॑व्या॒न्॒. यज॑मानो॒ यज॑मानो॒ भ्रातृ॑व्या॒ नपाप॒ भ्रातृ॑व्या॒न्॒. यज॑मानो॒ यज॑मानो॒ भ्रातृ॑व्या॒ नप॑ । \newline
55. भ्रातृ॑व्या॒ नपाप॒ भ्रातृ॑व्या॒न् भ्रातृ॑व्या॒ नप॑ नुदते नुद॒ते ऽप॒ भ्रातृ॑व्या॒न् भ्रातृ॑व्या॒ नप॑ नुदते । \newline
56. अप॑ नुदते नुद॒ते ऽपाप॑ नुदते दि॒क्षु दि॒क्षु नु॑द॒ते ऽपाप॑ नुदते दि॒क्षु । \newline
57. नु॒द॒ते॒ दि॒क्षु दि॒क्षु नु॑दते नुदते दि॒क्षूपोप॑ दि॒क्षु नु॑दते नुदते दि॒क्षूप॑ । \newline
58. दि॒क्षूपोप॑ दि॒क्षु दि॒क्षूप॑ दधाति दधा॒ त्युप॑ दि॒क्षु दि॒क्षूप॑ दधाति । \newline
59. उप॑ दधाति दधा॒ त्युपोप॑ दधाति देवपु॒रा दे॑वपु॒रा द॑धा॒ त्युपोप॑ दधाति देवपु॒राः । \newline
\pagebreak
\markright{ TS 5.7.3.2  \hfill https://www.vedavms.in \hfill}

\section{ TS 5.7.3.2 }

\textbf{TS 5.7.3.2 } \newline
\textbf{Samhita Paata} \newline

दधाति देवपु॒रा ए॒वैतास्त॑नू॒पानीः॒ पर्यू॑ह॒ते ऽग्ना॑विष्णू स॒जोष॑से॒मा व॑र्द्धन्तु वा॒गिंरः॑ । द्यु॒म्नैर्वाजे॑भि॒रा ग॑तं ॥ ब्र॒ह्म॒वा॒दिनो॑ वदन्ति॒ यन्न दे॒वता॑यै॒ जुह्व॒त्यथ॑ किं देव॒त्या॑ वसो॒र्द्धारेत्य॒ग्नि-र्वसु॒स्तस्यै॒षा धारा॒ विष्णु॒-र्वसु॒स्तस्यै॒षा धारा᳚ ऽऽग्नावैष्ण॒व्यर्चा वसो॒र्द्धारां᳚ जुहोति भाग॒धेये॑नै॒वैनौ॒ सम॑र्द्धय॒त्यथो॑ ए॒ता- [  ] \newline

\textbf{Pada Paata} \newline

द॒धा॒ति॒ । दे॒व॒पु॒रा इति॑ देव - पु॒राः । ए॒व । ए॒ताः । त॒नू॒पानी॒रिति॑ तनू - पानीः᳚ । परीति॑ । ऊ॒ह॒ते॒ । अग्ना॑विष्णू॒ इत्यग्ना᳚ - वि॒ष्णू॒ । स॒जोष॒सेति॑ स - जोष॑सा । इ॒माः । व॒द्‌र्ध॒न्तु॒ । वा॒म् । गिरः॑ ॥ द्यु॒म्नैः । वाजे॑भिः । एति॑ । ग॒त॒म् ॥ ब्र॒ह्म॒वा॒दिन॒ इति॑ ब्रह्म - वा॒दिनः॑ । व॒द॒न्ति॒ । यत् । न । दे॒वता॑यै । जुह्व॑ति । अथ॑ । कि॒दें॒व॒त्येति॑ किं - दे॒व॒त्या᳚ । वसोः᳚ । धारा᳚ । इति॑ । अ॒ग्निः । वसुः॑ । तस्य॑ । ए॒षा । धारा᳚ । विष्णुः॑ । वसुः॑ । तस्य॑ । ए॒षा । धारा᳚ । आ॒ग्ना॒वै॒ष्ण॒व्येत्या᳚ग्ना - वै॒ष्ण॒व्या । ऋ॒चा । वसोः᳚ । धारा᳚म् । जु॒हो॒ति॒ । भा॒ग॒धेये॒नेति॑ भाग - धेये॑न । ए॒व । ए॒नौ॒ । समिति॑ । अ॒द्‌र्ध॒य॒ति॒ । अथो॒ इति॑ । ए॒ताम् ।  \newline


\textbf{Krama Paata} \newline

द॒धा॒ति॒ दे॒व॒पु॒राः । दे॒व॒पु॒रा ए॒व । दे॒व॒पु॒रा इति॑ देव - पु॒राः । ए॒वैताः । ए॒तास्त॑नू॒पानीः᳚ । त॒नू॒पानीः॒ परि॑ । त॒नू॒पानी॒रिति॑ तनू - पानीः᳚ । पर्यू॑हते । ऊ॒ह॒तेऽग्ना॑विष्णू । अग्ना॑विष्णू स॒जोष॑सा । अग्ना॑विष्णू॒ इत्यग्ना᳚ - वि॒ष्णू॒ । स॒जोष॑से॒माः । स॒जोष॒सेति॑ स - जोष॑सा । इ॒मा व॑र्द्धन्तु । व॒र्द्ध॒न्तु॒ वा॒म् । वा॒म् गिरः॑ । गिर॒ इति॒ गिरः॑ ॥ द्यु॒म्नैर् वाजे॑भिः । वाजे॑भि॒रा । आ ग॑तम् । ग॒त॒मिति॑ गतम् ॥ ब्र॒ह्म॒वा॒दिनो॑ वदन्ति । ब्र॒ह्म॒वा॒दिन॒ इति॑ ब्रह्म - वा॒दिनः॑ । व॒द॒न्ति॒ यत् । यन् न । न दे॒वता॑यै । दे॒वता॑यै॒ जुह्व॑ति । जुह्व॒त्यथ॑ । अथ॑ किन्देव॒त्या᳚ । कि॒न्दे॒व॒त्या॑ वसोः᳚ । कि॒न्दे॒व॒त्येति॑ किम् - दे॒व॒त्या᳚ । वसो॒र्धारा᳚ । धारेति॑ । इत्य॒ग्निः । अ॒ग्निर् वसुः॑ । वसु॒स्तस्य॑ । तस्यै॒षा । ए॒षा धारा᳚ । धारा॒ विष्णुः॑ । विष्णु॒र् वसुः॑ । वसु॒स्तस्य॑ । तस्यै॒षा । ए॒षा धारा᳚ । धारा᳚ऽऽग्नावैष्ण॒व्या । आ॒ग्ना॒वै॒ष्ण॒व्यर्चा । आ॒ग्ना॒वै॒ष्ण॒व्येत्या᳚ग्ना - वै॒ष्ण॒व्या । ऋ॒चा वसोः᳚ । वसो॒र्धारा᳚म् । धारा᳚म् जुहोति । जु॒हो॒ति॒ भा॒ग॒धेये॑न । भा॒ग॒धेये॑नै॒व । भा॒ग॒धेये॒नेति॑ भाग - धेये॑न । ए॒वैनौ᳚ । ए॒नौ॒ सम् । सम॑र्द्धयति । अ॒र्द्ध॒य॒त्यथो᳚ । अथो॑ ए॒ताम् । अथो॒ इत्यथो᳚ । ए॒तामे॒व \newline

\textbf{Jatai Paata} \newline

1. द॒धा॒ति॒ दे॒व॒पु॒रा दे॑वपु॒रा द॑धाति दधाति देवपु॒राः । \newline
2. दे॒व॒पु॒रा ए॒वैव दे॑वपु॒रा दे॑वपु॒रा ए॒व । \newline
3. दे॒व॒पु॒रा इति॑ देव - पु॒राः । \newline
4. ए॒वैता ए॒ता ए॒वै वैताः । \newline
5. ए॒ता स्त॑नू॒पानी᳚ स्तनू॒पानी॑ रे॒ता ए॒ता स्त॑नू॒पानीः᳚ । \newline
6. त॒नू॒पानीः॒ परि॒ परि॑ तनू॒पानी᳚ स्तनू॒पानीः॒ परि॑ । \newline
7. त॒नू॒पानी॒रिति॑ तनू - पानीः᳚ । \newline
8. पर्यू॑हत ऊहते॒ परि॒ पर्यू॑हते । \newline
9. ऊ॒ह॒ते ऽग्ना॑विष्णू॒ अग्ना॑विष्णू ऊहत ऊह॒ते ऽग्ना॑विष्णू । \newline
10. अग्ना॑विष्णू स॒जोष॑सा स॒जोष॒सा ऽग्ना॑विष्णू॒ अग्ना॑विष्णू स॒जोष॑सा । \newline
11. अग्ना॑विष्णू॒ इत्यग्ना᳚ - वि॒ष्णू॒ । \newline
12. स॒जोष॑ से॒मा इ॒माः स॒जोष॑सा स॒जोष॑ से॒माः । \newline
13. स॒जोष॒सेति॑ स - जोष॑सा । \newline
14. इ॒मा व॑र्द्धन्तु वर्द्धन् त्वि॒मा इ॒मा व॑र्द्धन्तु । \newline
15. व॒र्द्ध॒न्तु॒ वां॒ ॅवां॒ ॅव॒र्द्ध॒न्तु॒ व॒र्द्ध॒न्तु॒ वा॒म् । \newline
16. वा॒म् गिरो॒ गिरो॑ वां ॅवा॒म् गिरः॑ । \newline
17. गिर॒ इति॒ गिरः॑ । \newline
18. द्यु॒म्नैर् वाजे॑भि॒र् वाजे॑भिर् द्यु॒म्नैर् द्यु॒म्नैर् वाजे॑भिः । \newline
19. वाजे॑भि॒रा वाजे॑भि॒र् वाजे॑भि॒रा । \newline
20. आ ग॑तम् गत॒ मा ग॑तम् । \newline
21. ग॒त॒मिति॑ गतम् । \newline
22. ब्र॒ह्म॒वा॒दिनो॑ वदन्ति वदन्ति ब्रह्मवा॒दिनो᳚ ब्रह्मवा॒दिनो॑ वदन्ति । \newline
23. ब्र॒ह्म॒वा॒दिन॒ इति॑ ब्रह्म - वा॒दिनः॑ । \newline
24. व॒द॒न्ति॒ यद् यद् व॑दन्ति वदन्ति॒ यत् । \newline
25. यन् न न यद् यन् न । \newline
26. न दे॒वता॑यै दे॒वता॑यै॒ न न दे॒वता॑यै । \newline
27. दे॒वता॑यै॒ जुह्व॑ति॒ जुह्व॑ति दे॒वता॑यै दे॒वता॑यै॒ जुह्व॑ति । \newline
28. जुह्व॒ त्यथाथ॒ जुह्व॑ति॒ जुह्व॒ त्यथ॑ । \newline
29. अथ॑ किन्देव॒त्या॑ किन्देव॒त्या॑ ऽथाथ॑ किन्देव॒त्या᳚ । \newline
30. कि॒न्दे॒व॒त्या॑ वसो॒र् वसोः᳚ किन्देव॒त्या॑ किन्देव॒त्या॑ वसोः᳚ । \newline
31. कि॒न्दे॒व॒त्येति॑ किं - दे॒व॒त्या᳚ । \newline
32. वसो॒र् धारा॒ धारा॒ वसो॒र् वसो॒र् धारा᳚ । \newline
33. धारेतीति॒ धारा॒ धारेति॑ । \newline
34. इत्य॒ग्नि र॒ग्नि रिती त्य॒ग्निः । \newline
35. अ॒ग्निर् वसु॒र् वसु॑ र॒ग्नि र॒ग्निर् वसुः॑ । \newline
36. वसु॒ स्तस्य॒ तस्य॒ वसु॒र् वसु॒ स्तस्य॑ । \newline
37. तस्यै॒ षैषा तस्य॒ तस्यै॒षा । \newline
38. ए॒षा धारा॒ धारै॒ षैषा धारा᳚ । \newline
39. धारा॒ विष्णु॒र् विष्णु॒र् धारा॒ धारा॒ विष्णुः॑ । \newline
40. विष्णु॒र् वसु॒र् वसु॒र् विष्णु॒र् विष्णु॒र् वसुः॑ । \newline
41. वसु॒ स्तस्य॒ तस्य॒ वसु॒र् वसु॒ स्तस्य॑ । \newline
42. तस्यै॒ षैषा तस्य॒ तस्यै॒षा । \newline
43. ए॒षा धारा॒ धारै॒ षैषा धारा᳚ । \newline
44. धारा᳚ ऽऽग्नावैष्ण॒व्या ऽऽग्ना॑वैष्ण॒व्या धारा॒ धारा᳚ ऽऽग्नावैष्ण॒व्या । \newline
45. आ॒ग्ना॒वै॒ष्ण॒व्य र्चर्चा ऽऽग्ना॑वैष्ण॒व्या ऽऽग्ना॑वैष्ण॒व्य र्चा । \newline
46. आ॒ग्ना॒वै॒ष्ण॒व्येत्या᳚ग्ना - वै॒ष्ण॒व्या । \newline
47. ऋ॒चा वसो॒र् वसोर्॑. ऋ॒च र्‌चा वसोः᳚ । \newline
48. वसो॒र् धारा॒म् धारां॒ ॅवसो॒र् वसो॒र् धारा᳚म् । \newline
49. धारा᳚म् जुहोति जुहोति॒ धारा॒म् धारा᳚म् जुहोति । \newline
50. जु॒हो॒ति॒ भा॒ग॒धेये॑न भाग॒धेये॑न जुहोति जुहोति भाग॒धेये॑न । \newline
51. भा॒ग॒धेये॑ नै॒वैव भा॑ग॒धेये॑न भाग॒धेये॑नै॒व । \newline
52. भा॒ग॒धेये॒नेति॑ भाग - धेये॑न । \newline
53. ए॒वैना॑ वेना वे॒वै वैनौ᳚ । \newline
54. ए॒नौ॒ सꣳ स मे॑ना वेनौ॒ सम् । \newline
55. स म॑र्द्धय त्यर्द्धयति॒ सꣳ स म॑र्द्धयति । \newline
56. अ॒र्द्ध॒य॒ त्यथो॒ अथो॑ अर्द्धय त्यर्द्धय॒ त्यथो᳚ । \newline
57. अथो॑ ए॒ता मे॒ता मथो॒ अथो॑ ए॒ताम् । \newline
58. अथो॒ इत्यथो᳚ । \newline
59. ए॒ता मे॒वै वैता मे॒ता मे॒व । \newline

\textbf{Ghana Paata } \newline

1. द॒धा॒ति॒ दे॒व॒पु॒रा दे॑वपु॒रा द॑धाति दधाति देवपु॒रा ए॒वैव दे॑वपु॒रा द॑धाति दधाति देवपु॒रा ए॒व । \newline
2. दे॒व॒पु॒रा ए॒वैव दे॑वपु॒रा दे॑वपु॒रा ए॒वैता ए॒ता ए॒व दे॑वपु॒रा दे॑वपु॒रा ए॒वैताः । \newline
3. दे॒व॒पु॒रा इति॑ देव - पु॒राः । \newline
4. ए॒वैता ए॒ता ए॒वैवैता स्त॑नू॒पानी᳚ स्तनू॒पानी॑ रे॒ता ए॒वैवैता स्त॑नू॒पानीः᳚ । \newline
5. ए॒ता स्त॑नू॒पानी᳚ स्तनू॒पानी॑ रे॒ता ए॒ता स्त॑नू॒पानीः॒ परि॒ परि॑ तनू॒पानी॑ रे॒ता ए॒ता स्त॑नू॒पानीः॒ परि॑ । \newline
6. त॒नू॒पानीः॒ परि॒ परि॑ तनू॒पानी᳚ स्तनू॒पानीः॒ पर्यू॑हत ऊहते॒ परि॑ तनू॒पानी᳚ स्तनू॒पानीः॒ पर्यू॑हते । \newline
7. त॒नू॒पानी॒रिति॑ तनू - पानीः᳚ । \newline
8. पर्यू॑हत ऊहते॒ परि॒ पर्यू॑ह॒ते ऽग्ना॑विष्णू॒ अग्ना॑विष्णू ऊहते॒ परि॒ पर्यू॑ह॒ते ऽग्ना॑विष्णू । \newline
9. ऊ॒ह॒ते ऽग्ना॑विष्णू॒ अग्ना॑विष्णू ऊहत ऊह॒ते ऽग्ना॑विष्णू स॒जोष॑सा स॒जोष॒सा ऽग्ना॑विष्णू ऊहत ऊह॒ते ऽग्ना॑विष्णू स॒जोष॑सा । \newline
10. अग्ना॑विष्णू स॒जोष॑सा स॒जोष॒सा ऽग्ना॑विष्णू॒ अग्ना॑विष्णू स॒जोष॑से॒मा इ॒माः स॒जोष॒सा ऽग्ना॑विष्णू॒ अग्ना॑विष्णू स॒जोष॑से॒माः । \newline
11. अग्ना॑विष्णू॒ इत्यग्ना᳚ - वि॒ष्णू॒ । \newline
12. स॒जोष॑से॒मा इ॒माः स॒जोष॑सा स॒जोष॑से॒मा व॑र्द्धन्तु वर्द्धन् त्वि॒माः स॒जोष॑सा स॒जोष॑से॒मा व॑र्द्धन्तु । \newline
13. स॒जोष॒सेति॑ स - जोष॑सा । \newline
14. इ॒मा व॑र्द्धन्तु वर्द्धन् त्वि॒मा इ॒मा व॑र्द्धन्तु वां ॅवां ॅवर्द्धन् त्वि॒मा इ॒मा व॑र्द्धन्तु वाम् । \newline
15. व॒र्द्ध॒न्तु॒ वां॒ ॅवां॒ ॅव॒र्द्ध॒न्तु॒ व॒र्द्ध॒न्तु॒ वा॒म् गिरो॒ गिरो॑ वां ॅवर्द्धन्तु वर्द्धन्तु वा॒म् गिरः॑ । \newline
16. वा॒म् गिरो॒ गिरो॑ वां ॅवा॒म् गिरः॑ । \newline
17. गिर॒ इति॒ गिरः॑ । \newline
18. द्यु॒म्नैर् वाजे॑भि॒र् वाजे॑भिर् द्यु॒म्नैर् द्यु॒म्नैर् वाजे॑भि॒रा वाजे॑भिर् द्यु॒म्नैर् द्यु॒म्नैर् वाजे॑भि॒रा । \newline
19. वाजे॑भि॒रा वाजे॑भि॒र् वाजे॑भि॒रा ग॑तम् गत॒ मा वाजे॑भि॒र् वाजे॑भि॒रा ग॑तम् । \newline
20. आ ग॑तम् गत॒ मा ग॑तम् । \newline
21. ग॒त॒मिति॑ गतम् । \newline
22. ब्र॒ह्म॒वा॒दिनो॑ वदन्ति वदन्ति ब्रह्मवा॒दिनो᳚ ब्रह्मवा॒दिनो॑ वदन्ति॒ यद् यद् व॑दन्ति ब्रह्मवा॒दिनो᳚ ब्रह्मवा॒दिनो॑ वदन्ति॒ यत् । \newline
23. ब्र॒ह्म॒वा॒दिन॒ इति॑ ब्रह्म - वा॒दिनः॑ । \newline
24. व॒द॒न्ति॒ यद् यद् व॑दन्ति वदन्ति॒ यन् न न यद् व॑दन्ति वदन्ति॒ यन् न । \newline
25. यन् न न यद् यन् न दे॒वता॑यै दे॒वता॑यै॒ न यद् यन् न दे॒वता॑यै । \newline
26. न दे॒वता॑यै दे॒वता॑यै॒ न न दे॒वता॑यै॒ जुह्व॑ति॒ जुह्व॑ति दे॒वता॑यै॒ न न दे॒वता॑यै॒ जुह्व॑ति । \newline
27. दे॒वता॑यै॒ जुह्व॑ति॒ जुह्व॑ति दे॒वता॑यै दे॒वता॑यै॒ जुह्व॒ त्यथाथ॒ जुह्व॑ति दे॒वता॑यै दे॒वता॑यै॒ जुह्व॒ त्यथ॑ । \newline
28. जुह्व॒त्य थाथ॒ जुह्व॑ति॒ जुह्व॒ त्यथ॑ किन्देव॒त्या॑ किन्देव॒त्या॑ ऽथ॒ जुह्व॑ति॒ जुह्व॒ त्यथ॑ किन्देव॒त्या᳚ । \newline
29. अथ॑ किन्देव॒त्या॑ किन्देव॒त्या॑ ऽथाथ॑ किन्देव॒त्या॑ वसो॒र् वसोः᳚ किन्देव॒त्या॑ ऽथाथ॑ किन्देव॒त्या॑ वसोः᳚ । \newline
30. कि॒न्दे॒व॒त्या॑ वसो॒र् वसोः᳚ किन्देव॒त्या॑ किन्देव॒त्या॑ वसो॒र् धारा॒ धारा॒ वसोः᳚ किन्देव॒त्या॑ किन्देव॒त्या॑ वसो॒र् धारा᳚ । \newline
31. कि॒न्दे॒व॒त्येति॑ किं - दे॒व॒त्या᳚ । \newline
32. वसो॒र् धारा॒ धारा॒ वसो॒र् वसो॒र् धारेतीति॒ धारा॒ वसो॒र् वसो॒र् धारेति॑ । \newline
33. धारेतीति॒ धारा॒ धारे त्य॒ग्नि र॒ग्नि रिति॒ धारा॒ धारे त्य॒ग्निः । \newline
34. इत्य॒ग्नि र॒ग्नि रिती त्य॒ग्निर् वसु॒र् वसु॑ र॒ग्नि रिती त्य॒ग्निर् वसुः॑ । \newline
35. अ॒ग्निर् वसु॒र् वसु॑ र॒ग्नि र॒ग्निर् वसु॒ स्तस्य॒ तस्य॒ वसु॑ र॒ग्नि र॒ग्निर् वसु॒ स्तस्य॑ । \newline
36. वसु॒ स्तस्य॒ तस्य॒ वसु॒र् वसु॒ स्तस्यै॒षैषा तस्य॒ वसु॒र् वसु॒स्त स्यै॒षा । \newline
37. तस्यै॒षैषा तस्य॒ तस्यै॒षा धारा॒ धारै॒षा तस्य॒ तस्यै॒षा धारा᳚ । \newline
38. ए॒षा धारा॒ धारै॒ षैषा धारा॒ विष्णु॒र् विष्णु॒र् धारै॒ षैषा धारा॒ विष्णुः॑ । \newline
39. धारा॒ विष्णु॒र् विष्णु॒र् धारा॒ धारा॒ विष्णु॒र् वसु॒र् वसु॒र् विष्णु॒र् धारा॒ धारा॒ विष्णु॒र् वसुः॑ । \newline
40. विष्णु॒र् वसु॒र् वसु॒र् विष्णु॒र् विष्णु॒र् वसु॒ स्तस्य॒ तस्य॒ वसु॒र् विष्णु॒र् विष्णु॒र् वसु॒ स्तस्य॑ । \newline
41. वसु॒ स्तस्य॒ तस्य॒ वसु॒र् वसु॒ स्तस्यै॒ षैषा तस्य॒ वसु॒र् वसु॒ स्तस्यै॒षा । \newline
42. तस्यै॒ षैषा तस्य॒ तस्यै॒षा धारा॒ धारै॒षा तस्य॒ तस्यै॒षा धारा᳚ । \newline
43. ए॒षा धारा॒ धारै॒षैषा धारा᳚ ऽऽग्नावैष्ण॒व्या ऽऽग्ना॑वैष्ण॒व्या धारै॒षैषा धारा᳚ ऽऽग्नावैष्ण॒व्या । \newline
44. धारा᳚ ऽऽग्नावैष्ण॒व्या ऽऽग्ना॑वैष्ण॒व्या धारा॒ धारा᳚ ऽऽग्नावैष्ण॒व्य र्‌च र्‌चा ऽऽग्ना॑वैष्ण॒व्या धारा॒ धारा᳚ ऽऽग्नावैष्ण॒व्य र्‌चा । \newline
45. आ॒ग्ना॒वै॒ष्ण॒व्य र्‌च र्‌चा ऽऽग्ना॑वैष्ण॒व्या ऽऽग्ना॑वैष्ण॒व्य र्‌चा वसो॒र् वसोर्॑. ऋ॒चा ऽऽग्ना॑वैष्ण॒व्या ऽऽग्ना॑वैष्ण॒व्य र्‌चा वसोः᳚ । \newline
46. आ॒ग्ना॒वै॒ष्ण॒व्येत्या᳚ग्ना - वै॒ष्ण॒व्या । \newline
47. ऋ॒चा वसो॒र् वसोर्॑. ऋ॒च र्‌चा वसो॒र् धारा॒म् धारां॒ ॅवसोर्॑. ऋ॒च र्‌चा वसो॒र् धारा᳚म् । \newline
48. वसो॒र् धारा॒म् धारां॒ ॅवसो॒र् वसो॒र् धारा᳚म् जुहोति जुहोति॒ धारां॒ ॅवसो॒र् वसो॒र् धारा᳚म् जुहोति । \newline
49. धारा᳚म् जुहोति जुहोति॒ धारा॒म् धारा᳚म् जुहोति भाग॒धेये॑न भाग॒धेये॑न जुहोति॒ धारा॒म् धारा᳚म् जुहोति भाग॒धेये॑न । \newline
50. जु॒हो॒ति॒ भा॒ग॒धेये॑न भाग॒धेये॑न जुहोति जुहोति भाग॒धेये॑ नै॒वैव भा॑ग॒धेये॑न जुहोति जुहोति भाग॒धेये॑नै॒व । \newline
51. भा॒ग॒धेये॑ नै॒वैव भा॑ग॒धेये॑न भाग॒धेये॑ नै॒वैना॑ वेना वे॒व भा॑ग॒धेये॑न भाग॒धेये॑
नै॒वैनौ᳚ । \newline
52. भा॒ग॒धेये॒नेति॑ भाग - धेये॑न । \newline
53. ए॒वैना॑ वेना वे॒वै वैनौ॒ सꣳ स मे॑ना वे॒वै वैनौ॒ सम् । \newline
54. ए॒नौ॒ सꣳ स मे॑ना वेनौ॒ स म॑र्द्धय त्यर्द्धयति॒ स मे॑ना वेनौ॒ स म॑र्द्धयति । \newline
55. स म॑र्द्धय त्यर्द्धयति॒ सꣳ स म॑र्द्धय॒ त्यथो॒ अथो॑ अर्द्धयति॒ सꣳ स म॑र्द्धय॒ त्यथो᳚ । \newline
56. अ॒र्द्ध॒य॒ त्यथो॒ अथो॑ अर्द्धय त्यर्द्धय॒ त्यथो॑ ए॒ता मे॒ता मथो॑ अर्द्धय त्यर्द्धय॒ त्यथो॑ ए॒ताम् । \newline
57. अथो॑ ए॒ता मे॒ता मथो॒ अथो॑ ए॒ता मे॒वै वैता मथो॒ अथो॑ ए॒ता मे॒व । \newline
58. अथो॒ इत्यथो᳚ । \newline
59. ए॒ता मे॒वै वैता मे॒ता मे॒वाहु॑ति॒ माहु॑ति मे॒वैता मे॒ता मे॒वाहु॑तिम् । \newline
\pagebreak
\markright{ TS 5.7.3.3  \hfill https://www.vedavms.in \hfill}

\section{ TS 5.7.3.3 }

\textbf{TS 5.7.3.3 } \newline
\textbf{Samhita Paata} \newline

-मे॒वाहु॑तिमा॒यत॑नवतीं करोति॒ यत्का॑म एनां जु॒होति॒ तदे॒वाव॑ रुन्धे रु॒द्रो वा ए॒ष यद॒ग्निस्तस्यै॒ते त॒नुवौ॑ घो॒राऽन्या शि॒वाऽन्या यच्छ॑तरु॒द्रीयं॑ जु॒होति॒ यैवास्य॑ घो॒रा त॒नूस्तां तेन॑ शमयति॒ यद्-वसो॒र्द्धारां᳚ जु॒होति॒ यैवास्य॑ शि॒वा त॒नूस्तां तेन॑ प्रीणाति॒ यो वै वसो॒र्द्धारा॑यै - [  ] \newline

\textbf{Pada Paata} \newline

ए॒व । आहु॑ति॒मित्या - हु॒ति॒म् । आ॒यत॑नवती॒मित्या॒यत॑न - व॒ती॒म् । क॒रो॒ति॒ । यत्का॑म॒ इति॒ यत् - का॒मः॒ । ए॒ना॒म् । जु॒होति॑ । तत् । ए॒व । अवेति॑ । रु॒न्धे॒ । रु॒द्रः । वै । ए॒षः । यत् । अ॒ग्निः । तस्य॑ । ए॒ते इति॑ । त॒नुवौ᳚ । घो॒रा । अ॒न्या । शि॒वा । अ॒न्या । यत् । श॒त॒रु॒द्रीय॒मिति॑ शत-रु॒द्रीय᳚म् । जु॒होति॑ । या । ए॒व । अ॒स्य॒ । घो॒रा । त॒नूः । ताम् । तेन॑ । श॒म॒य॒ति॒ । यत् । वसोः᳚ । धारा᳚म् । जु॒होति॑ । या । ए॒व । अ॒स्य॒ । शि॒वा । त॒नूः । ताम् । तेन॑ । प्री॒णा॒ति॒ । यः । वै । वसोः᳚ । धारा॑यै ।  \newline


\textbf{Krama Paata} \newline

ए॒वाहु॑तिम् । आहु॑तिमा॒यत॑नवतीम् । आहु॑ति॒मित्या - हु॒ति॒म् । आ॒यत॑नवतीम् करोति । आ॒यत॑नवती॒मित्या॒यत॑न - व॒ती॒म् । क॒रो॒ति॒ यत्का॑मः । यत्का॑म एनाम् । यत्का॑म॒ इति॒ यत् - का॒मः॒ । ए॒ना॒म् जु॒होति॑ । जु॒होति॒ तत् । तदे॒व । ए॒वाव॑ । अव॑ रुन्धे । रु॒न्धे॒ रु॒द्रः । रु॒द्रो वै । वा ए॒षः । ए॒ष यत् । यद॒ग्निः । अ॒ग्निस्तस्य॑ । तस्यै॒ते । ए॒ते त॒नुवौ᳚ । ए॒ते इत्ये॒ते । त॒नुवौ॑ घो॒रा । घो॒राऽन्या । अ॒न्या शि॒वा । शि॒वाऽन्या । अ॒न्या यत् । यच्छ॑तरु॒द्रीय᳚म् । श॒त॒रु॒द्रीय॑म् जु॒होति॑ । श॒त॒रु॒द्रीय॒मिति॑ शत - रु॒द्रीय᳚म् । जु॒होति॒ या । यैव । ए॒वास्य॑ । अ॒स्य॒ घो॒रा । घो॒रा त॒नूः । त॒नूस्ताम् । ताम् तेन॑ । तेन॑ शमयति । श॒म॒य॒ति॒ यत् । यद् वसोः᳚ । वसो॒र् धारा᳚म् । धारा᳚म् जु॒होति॑ । जु॒होति॒ या । यैव । ए॒वास्य॑ । अ॒स्य॒ शि॒वा । शि॒वा त॒नूः । त॒नूस्ताम् । ताम् तेन॑ । तेन॑ प्रीणाति । प्री॒णा॒ति॒ यः । यो वै । वै वसोः᳚ । वसो॒र् धारा॑यै ( ) । धारा॑यै प्रति॒ष्ठाम् \newline

\textbf{Jatai Paata} \newline

1. ए॒वा हु॑ति॒ माहु॑ति मे॒वै वाहु॑तिम् । \newline
2. आहु॑ति मा॒यत॑नवती मा॒यत॑नवती॒ माहु॑ति॒ माहु॑ति मा॒यत॑नवतीम् । \newline
3. आहु॑ति॒मित्या - हु॒ति॒म् । \newline
4. आ॒यत॑नवतीम् करोति करो त्या॒यत॑नवती मा॒यत॑नवतीम् करोति । \newline
5. आ॒यत॑नवती॒मित्या॒यत॑न - व॒ती॒म् । \newline
6. क॒रो॒ति॒ यत्का॑मो॒ यत्का॑मः करोति करोति॒ यत्का॑मः । \newline
7. यत्का॑म एना मेनां॒ ॅयत्का॑मो॒ यत्का॑म एनाम् । \newline
8. यत्का॑म॒ इति॒ यत् - का॒मः॒ । \newline
9. ए॒ना॒म् जु॒होति॑ जु॒हो त्ये॑ना मेनाम् जु॒होति॑ । \newline
10. जु॒होति॒ तत् तज् जु॒होति॑ जु॒होति॒ तत् । \newline
11. तदे॒ वैव तत् तदे॒व । \newline
12. ए॒वावा वै॒वै वाव॑ । \newline
13. अव॑ रुन्धे रु॒न्धे ऽवाव॑ रुन्धे । \newline
14. रु॒न्धे॒ रु॒द्रो रु॒द्रो रु॑न्धे रुन्धे रु॒द्रः । \newline
15. रु॒द्रो वै वै रु॒द्रो रु॒द्रो वै । \newline
16. वा ए॒ष ए॒ष वै वा ए॒षः । \newline
17. ए॒ष यद् यदे॒ष ए॒ष यत् । \newline
18. यद॒ग्नि र॒ग्निर् यद् यद॒ग्निः । \newline
19. अ॒ग्नि स्तस्य॒ तस्या॒ग्नि र॒ग्नि स्तस्य॑ । \newline
20. तस्यै॒ते ए॒ते तस्य॒ तस्यै॒ते । \newline
21. ए॒ते त॒नुवौ॑ त॒नुवा॑ वे॒ते ए॒ते त॒नुवौ᳚ । \newline
22. ए॒ते इत्ये॒ते । \newline
23. त॒नुवौ॑ घो॒रा घो॒रा त॒नुवौ॑ त॒नुवौ॑ घो॒रा । \newline
24. घो॒रा ऽन्या ऽन्या घो॒रा घो॒रा ऽन्या । \newline
25. अ॒न्या शि॒वा शि॒वा ऽन्या ऽन्या शि॒वा । \newline
26. शि॒वा ऽन्या ऽन्या शि॒वा शि॒वा ऽन्या । \newline
27. अ॒न्या यद् यद॒न्या ऽन्या यत् । \newline
28. यच् छ॑तरु॒द्रीयꣳ॑ शतरु॒द्रीयं॒ ॅयद् यच् छ॑तरु॒द्रीय᳚म् । \newline
29. श॒त॒रु॒द्रीय॑म् जु॒होति॑ जु॒होति॑ शतरु॒द्रीयꣳ॑ शतरु॒द्रीय॑म् जु॒होति॑ । \newline
30. श॒त॒रु॒द्रीय॒मिति॑ शत - रु॒द्रीय᳚म् । \newline
31. जु॒होति॒ या या जु॒होति॑ जु॒होति॒ या । \newline
32. यैवैव या यैव । \newline
33. ए॒वास्या᳚ स्यै॒वै वास्य॑ । \newline
34. अ॒स्य॒ घो॒रा घो॒रा ऽस्या᳚स्य घो॒रा । \newline
35. घो॒रा त॒नू स्त॒नूर् घो॒रा घो॒रा त॒नूः । \newline
36. त॒नू स्ताम् ताम् त॒नू स्त॒नू स्ताम् । \newline
37. ताम् तेन॒ तेन॒ ताम् ताम् तेन॑ । \newline
38. तेन॑ शमयति शमयति॒ तेन॒ तेन॑ शमयति । \newline
39. श॒म॒य॒ति॒ यद् यच् छ॑मयति शमयति॒ यत् । \newline
40. यद् वसो॒र् वसो॒र् यद् यद् वसोः᳚ । \newline
41. वसो॒र् धारा॒म् धारां॒ ॅवसो॒र् वसो॒र् धारा᳚म् । \newline
42. धारा᳚म् जु॒होति॑ जु॒होति॒ धारा॒म् धारा᳚म् जु॒होति॑ । \newline
43. जु॒होति॒ या या जु॒होति॑ जु॒होति॒ या । \newline
44. यैवैव या यैव । \newline
45. ए॒वास्या᳚ स्यै॒वै वास्य॑ । \newline
46. अ॒स्य॒ शि॒वा शि॒वा ऽस्या᳚स्य शि॒वा । \newline
47. शि॒वा त॒नू स्त॒नूः शि॒वा शि॒वा त॒नूः । \newline
48. त॒नू स्ताम् ताम् त॒नू स्त॒नू स्ताम् । \newline
49. ताम् तेन॒ तेन॒ ताम् ताम् तेन॑ । \newline
50. तेन॑ प्रीणाति प्रीणाति॒ तेन॒ तेन॑ प्रीणाति । \newline
51. प्री॒णा॒ति॒ यो यः प्री॑णाति प्रीणाति॒ यः । \newline
52. यो वै वै यो यो वै । \newline
53. वै वसो॒र् वसो॒र् वै वै वसोः᳚ । \newline
54. वसो॒र् धारा॑यै॒ धारा॑यै॒ वसो॒र् वसो॒र् धारा॑यै । \newline
55. धारा॑यै प्रति॒ष्ठाम् प्र॑ति॒ष्ठाम् धारा॑यै॒ धारा॑यै प्रति॒ष्ठाम् । \newline

\textbf{Ghana Paata } \newline

1. ए॒वाहु॑ति॒ माहु॑ति मे॒वैवाहु॑ति मा॒यत॑नवती मा॒यत॑नवती॒ माहु॑ति मे॒वै वाहु॑ति मा॒यत॑नवतीम् । \newline
2. आहु॑ति मा॒यत॑नवती मा॒यत॑नवती॒ माहु॑ति॒ माहु॑ति मा॒यत॑नवतीम् करोति करो त्या॒यत॑नवती॒ माहु॑ति॒ माहु॑ति मा॒यत॑नवतीम् करोति । \newline
3. आहु॑ति॒मित्या - हु॒ति॒म् । \newline
4. आ॒यत॑नवतीम् करोति करो त्या॒यत॑नवती मा॒यत॑नवतीम् करोति॒ यत्का॑मो॒ यत्का॑मः करो त्या॒यत॑नवती मा॒यत॑नवतीम् करोति॒ यत्का॑मः । \newline
5. आ॒यत॑नवती॒मित्या॒यत॑न - व॒ती॒म् । \newline
6. क॒रो॒ति॒ यत्का॑मो॒ यत्का॑मः करोति करोति॒ यत्का॑म एना मेनां॒ ॅयत्का॑मः करोति करोति॒ यत्का॑म एनाम् । \newline
7. यत्का॑म एना मेनां॒ ॅयत्का॑मो॒ यत्का॑म एनाम् जु॒होति॑ जु॒हो त्ये॑नां॒ ॅयत्का॑मो॒ यत्का॑म एनाम् जु॒होति॑ । \newline
8. यत्का॑म॒ इति॒ यत् - का॒मः॒ । \newline
9. ए॒ना॒म् जु॒होति॑ जु॒हो त्ये॑ना मेनाम् जु॒होति॒ तत् तज् जु॒हो त्ये॑ना मेनाम् जु॒होति॒ तत् । \newline
10. जु॒होति॒ तत् तज् जु॒होति॑ जु॒होति॒ तदे॒ वैव तज् जु॒होति॑ जु॒होति॒ तदे॒व । \newline
11. तदे॒ वैव तत् तदे॒ वावा वै॒व तत् तदे॒ वाव॑ । \newline
12. ए॒वावा वै॒वै वाव॑ रुन्धे रु॒न्धे ऽवै॒वै वाव॑ रुन्धे । \newline
13. अव॑ रुन्धे रु॒न्धे ऽवाव॑ रुन्धे रु॒द्रो रु॒द्रो रु॒न्धे ऽवाव॑ रुन्धे रु॒द्रः । \newline
14. रु॒न्धे॒ रु॒द्रो रु॒द्रो रु॑न्धे रुन्धे रु॒द्रो वै वै रु॒द्रो रु॑न्धे रुन्धे रु॒द्रो वै । \newline
15. रु॒द्रो वै वै रु॒द्रो रु॒द्रो वा ए॒ष ए॒ष वै रु॒द्रो रु॒द्रो वा ए॒षः । \newline
16. वा ए॒ष ए॒ष वै वा ए॒ष यद् यदे॒ष वै वा ए॒ष यत् । \newline
17. ए॒ष यद् यदे॒ष ए॒ष यद॒ग्नि र॒ग्निर् यदे॒ष ए॒ष यद॒ग्निः । \newline
18. यद॒ग्नि र॒ग्निर् यद् यद॒ग्नि स्तस्य॒ तस्या॒ग्निर् यद् यद॒ग्नि स्तस्य॑ । \newline
19. अ॒ग्नि स्तस्य॒ तस्या॒ग्नि र॒ग्नि स्तस्यै॒ते ए॒ते तस्या॒ग्नि र॒ग्नि स्तस्यै॒ते । \newline
20. तस्यै॒ते ए॒ते तस्य॒ तस्यै॒ते त॒नुवौ॑ त॒नुवा॑ वे॒ते तस्य॒ तस्यै॒ते त॒नुवौ᳚ । \newline
21. ए॒ते त॒नुवौ॑ त॒नुवा॑ वे॒ते ए॒ते त॒नुवौ॑ घो॒रा घो॒रा त॒नुवा॑ वे॒ते ए॒ते त॒नुवौ॑ घो॒रा । \newline
22. ए॒ते इत्ये॒ते । \newline
23. त॒नुवौ॑ घो॒रा घो॒रा त॒नुवौ॑ त॒नुवौ॑ घो॒रा ऽन्या ऽन्या घो॒रा त॒नुवौ॑ त॒नुवौ॑ घो॒रा ऽन्या । \newline
24. घो॒रा ऽन्या ऽन्या घो॒रा घो॒रा ऽन्या शि॒वा शि॒वा ऽन्या घो॒रा घो॒रा ऽन्या शि॒वा । \newline
25. अ॒न्या शि॒वा शि॒वा ऽन्या ऽन्या शि॒वा ऽन्या ऽन्या शि॒वा ऽन्या ऽन्या शि॒वा ऽन्या । \newline
26. शि॒वा ऽन्या ऽन्या शि॒वा शि॒वा ऽन्या यद् यद॒न्या शि॒वा शि॒वा ऽन्या यत् । \newline
27. अ॒न्या यद् यद॒न्या ऽन्या यच्छ॑तरु॒द्रीयꣳ॑ शतरु॒द्रीयं॒ ॅयद॒न्या ऽन्या यच्छ॑तरु॒द्रीय᳚म् । \newline
28. यच्छ॑तरु॒द्रीयꣳ॑ शतरु॒द्रीयं॒ ॅयद् यच्छ॑तरु॒द्रीय॑म् जु॒होति॑ जु॒होति॑ शतरु॒द्रीयं॒ ॅयद् 
यच्छ॑तरु॒द्रीय॑म् जु॒होति॑ । \newline
29. श॒त॒रु॒द्रीय॑म् जु॒होति॑ जु॒होति॑ शतरु॒द्रीयꣳ॑ शतरु॒द्रीय॑म् जु॒होति॒ या या जु॒होति॑ शतरु॒द्रीयꣳ॑ शतरु॒द्रीय॑म् जु॒होति॒ या । \newline
30. श॒त॒रु॒द्रीय॒मिति॑ शत - रु॒द्रीय᳚म् । \newline
31. जु॒होति॒ या या जु॒होति॑ जु॒होति॒ यैवैव या जु॒होति॑ जु॒होति॒ यैव । \newline
32. यैवैव या यैवास्या᳚ स्यै॒व या यैवास्य॑ । \newline
33. ए॒वास्या᳚ स्यै॒वै वास्य॑ घो॒रा घो॒रा ऽस्यै॒वै वास्य॑ घो॒रा । \newline
34. अ॒स्य॒ घो॒रा घो॒रा ऽस्या᳚स्य घो॒रा त॒नू स्त॒नूर् घो॒रा ऽस्या᳚स्य घो॒रा त॒नूः । \newline
35. घो॒रा त॒नू स्त॒नूर् घो॒रा घो॒रा त॒नू स्ताम् ताम् त॒नूर् घो॒रा घो॒रा त॒नू स्ताम् । \newline
36. त॒नू स्ताम् ताम् त॒नू स्त॒नू स्ताम् तेन॒ तेन॒ ताम् त॒नू स्त॒नू स्ताम् तेन॑ । \newline
37. ताम् तेन॒ तेन॒ ताम् ताम् तेन॑ शमयति शमयति॒ तेन॒ ताम् ताम् तेन॑ शमयति । \newline
38. तेन॑ शमयति शमयति॒ तेन॒ तेन॑ शमयति॒ यद् यच्छ॑मयति॒ तेन॒ तेन॑ शमयति॒ यत् । \newline
39. श॒म॒य॒ति॒ यद् यच्छ॑मयति शमयति॒ यद् वसो॒र् वसो॒र् यच्छ॑मयति शमयति॒ यद् वसोः᳚ । \newline
40. यद् वसो॒र् वसो॒र् यद् यद् वसो॒र् धारा॒म् धारां॒ ॅवसो॒र् यद् यद् वसो॒र् धारा᳚म् । \newline
41. वसो॒र् धारा॒म् धारां॒ ॅवसो॒र् वसो॒र् धारा᳚म् जु॒होति॑ जु॒होति॒ धारां॒ ॅवसो॒र् वसो॒र् धारा᳚म् जु॒होति॑ । \newline
42. धारा᳚म् जु॒होति॑ जु॒होति॒ धारा॒म् धारा᳚म् जु॒होति॒ या या जु॒होति॒ धारा॒म् धारा᳚म् जु॒होति॒ या । \newline
43. जु॒होति॒ या या जु॒होति॑ जु॒होति॒ यैवैव या जु॒होति॑ जु॒होति॒ यैव । \newline
44. यैवैव या यैवास्या᳚ स्यै॒व या यैवास्य॑ । \newline
45. ए॒वास्या᳚स्यै॒वै वास्य॑ शि॒वा शि॒वा ऽस्यै॒वै वास्य॑ शि॒वा । \newline
46. अ॒स्य॒ शि॒वा शि॒वा ऽस्या᳚स्य शि॒वा त॒नू स्त॒नूः शि॒वा ऽस्या᳚स्य शि॒वा त॒नूः । \newline
47. शि॒वा त॒नू स्त॒नूः शि॒वा शि॒वा त॒नू स्ताम् ताम् त॒नूः शि॒वा शि॒वा त॒नू स्ताम् । \newline
48. त॒नू स्ताम् ताम् त॒नू स्त॒नू स्ताम् तेन॒ तेन॒ ताम् त॒नू स्त॒नू स्ताम् तेन॑ । \newline
49. ताम् तेन॒ तेन॒ ताम् ताम् तेन॑ प्रीणाति प्रीणाति॒ तेन॒ ताम् ताम् तेन॑ प्रीणाति । \newline
50. तेन॑ प्रीणाति प्रीणाति॒ तेन॒ तेन॑ प्रीणाति॒ यो यः प्री॑णाति॒ तेन॒ तेन॑ प्रीणाति॒ यः । \newline
51. प्री॒णा॒ति॒ यो यः प्री॑णाति प्रीणाति॒ यो वै वै यः प्री॑णाति प्रीणाति॒ यो वै । \newline
52. यो वै वै यो यो वै वसो॒र् वसो॒र् वै यो यो वै वसोः᳚ । \newline
53. वै वसो॒र् वसो॒र् वै वै वसो॒र् धारा॑यै॒ धारा॑यै॒ वसो॒र् वै वै वसो॒र् धारा॑यै । \newline
54. वसो॒र् धारा॑यै॒ धारा॑यै॒ वसो॒र् वसो॒र् धारा॑यै प्रति॒ष्ठाम् प्र॑ति॒ष्ठाम् धारा॑यै॒ वसो॒र् वसो॒र् धारा॑यै प्रति॒ष्ठाम् । \newline
55. धारा॑यै प्रति॒ष्ठाम् प्र॑ति॒ष्ठाम् धारा॑यै॒ धारा॑यै प्रति॒ष्ठां ॅवेद॒ वेद॑ प्रति॒ष्ठाम् धारा॑यै॒ धारा॑यै प्रति॒ष्ठां ॅवेद॑ । \newline
\pagebreak
\markright{ TS 5.7.3.4  \hfill https://www.vedavms.in \hfill}

\section{ TS 5.7.3.4 }

\textbf{TS 5.7.3.4 } \newline
\textbf{Samhita Paata} \newline

प्रति॒ष्ठां ॅवेद॒ प्रत्ये॒व ति॑ष्ठति॒ यदाज्य॑मु॒च्छिष्ये॑त॒ तस्मि॑न् ब्रह्मौद॒नं प॑चे॒त् तं ब्रा᳚ह्म॒णाश्च॒त्वारः॒ प्राश्नी॑युरे॒ष वा अ॒ग्निर्वै᳚श्वान॒रो यद्ब्रा᳚ह्म॒ण ए॒षा खलु॒ वा अ॒ग्नेः प्रि॒या त॒नूर्यद्-वै᳚श्वान॒रः प्रि॒याया॑मे॒वैनां᳚ त॒नुवां॒ प्रति॑ ष्ठापयति॒ चत॑स्रो धे॒नूर्द॑द्या॒त् ताभि॑रे॒व यज॑मानो॒ऽमुष्मि॑न् ॅलो॒के᳚ऽग्निं दु॑हे ॥ \newline

\textbf{Pada Paata} \newline

प्र॒ति॒ष्ठामिति॑ प्रति - स्थाम् । वेद॑ । प्रतीति॑ । ए॒व । ति॒ष्ठ॒ति॒ । यत् । आज्य᳚म् । उ॒च्छिष्ये॒तेत्यु॑त् - शिष्ये॑त । तस्मिन्न्॑ । ब्र॒ह्मौ॒द॒नमिति॑ ब्रह्म - ओ॒द॒नम् । प॒चे॒त् । तम् । ब्रा॒ह्म॒णाः । च॒त्वारः॑ । प्रेति॑ । अ॒श्नी॒युः॒ । ए॒षः । वै । अ॒ग्निः । वै॒श्वा॒न॒रः । यत् । ब्रा॒ह्म॒णः । ए॒षा । खलु॑ । वै । अ॒ग्नेः । प्रि॒या । त॒नूः । यत् । वै॒श्वा॒न॒रः । प्रि॒याया᳚म् । ए॒व । ए॒ना॒म् । त॒नुवा᳚म् । प्रतीति॑ । स्था॒प॒य॒ति॒ । चत॑स्रः । धे॒नूः । द॒द्या॒त् । ताभिः॑ । ए॒व । यज॑मानः । अ॒मुष्मिन्न्॑ । लो॒के । अ॒ग्निम् । दु॒हे॒ ॥  \newline


\textbf{Krama Paata} \newline

प्र॒ति॒ष्ठाम् ॅवेद॑ । प्र॒ति॒ष्ठामिति॑ प्रति - स्थाम् । वेद॒ प्रति॑ । प्रत्ये॒व । ए॒व ति॑ष्ठति । ति॒ष्ठ॒ति॒ यत् । यदाज्य᳚म् । आज्य॑मु॒च्छिष्ये॑त । उ॒च्छिष्ये॑त॒ तस्मिन्न्॑ । उ॒च्छिष्ये॒तेत्यु॑त् - शिष्ये॑त । तस्मि॑न् ब्रह्मौद॒नम् । ब्र॒ह्मौ॒द॒नम् प॑चेत् । ब्र॒ह्मौ॒द॒नमिति॑ ब्रह्म - ओ॒द॒नम् । प॒चे॒त् तम् । तम् ब्रा᳚ह्म॒णाः । ब्रा॒ह्म॒णाश्च॒त्वारः॑ । च॒त्वारः॒ प्र । प्राश्ञी॑युः । अ॒श्ञी॒यु॒रे॒षः । ए॒ष वै । वा अ॒ग्निः । अ॒ग्निर् वै᳚श्वान॒रः । वै॒श्वा॒न॒रो यत् । यद् ब्रा᳚ह्म॒णः । ब्रा॒ह्म॒ण ए॒षा । ए॒षा खलु॑ । खलु॒ वै । वा अ॒ग्नेः । अ॒ग्नेः प्रि॒या । प्रि॒या त॒नूः । त॒नूर् यत् । यद् वै᳚श्वान॒रः । वै॒श्वा॒न॒रः प्रि॒याया᳚म् । प्रि॒याया॑मे॒व । ए॒वैना᳚म् । ए॒ना॒म् त॒नुवा᳚म् । त॒नुवा॒म् प्रति॑ । प्रति॑ष्ठापयति । स्था॒प॒य॒ति॒ चत॑स्रः । चत॑स्रो धे॒नूः । धे॒नूर् द॑द्यात् । द॒द्या॒त् ताभिः॑ । ताभि॑रे॒व । ए॒व यज॑मानः । यज॑मानो॒ऽमुष्मिन्न्॑ । अ॒मुष्मि॑न् ॅलो॒के । लो॒के᳚ऽग्निम् । अ॒ग्निम् दु॑हे । दु॒ह॒ इति॑ दुहे । \newline

\textbf{Jatai Paata} \newline

1. प्र॒ति॒ष्ठां ॅवेद॒ वेद॑ प्रति॒ष्ठाम् प्र॑ति॒ष्ठां ॅवेद॑ । \newline
2. प्र॒ति॒ष्ठामिति॑ प्रति - स्थाम् । \newline
3. वेद॒ प्रति॒ प्रति॒ वेद॒ वेद॒ प्रति॑ । \newline
4. प्रत्ये॒ वैव प्रति॒ प्रत्ये॒व । \newline
5. ए॒व ति॑ष्ठति तिष्ठत्ये॒ वैव ति॑ष्ठति । \newline
6. ति॒ष्ठ॒ति॒ यद् यत् ति॑ष्ठति तिष्ठति॒ यत् । \newline
7. यदाज्य॒ माज्यं॒ ॅयद् यदाज्य᳚म् । \newline
8. आज्य॑ मु॒च्छिष्ये॑ तो॒च्छिष्ये॒ ताज्य॒ माज्य॑ मु॒च्छिष्ये॑त । \newline
9. उ॒च्छिष्ये॑त॒ तस्मिꣳ॒॒ स्तस्मि॑न् नु॒च्छिष्ये॑ तो॒च्छिष्ये॑त॒ तस्मिन्न्॑ । \newline
10. उ॒च्छिष्ये॒तेत्यु॑त् - शिष्ये॑त । \newline
11. तस्मि॑न् ब्रह्मौद॒नम् ब्र॑ह्मौद॒नम् तस्मिꣳ॒॒ स्तस्मि॑न् ब्रह्मौद॒नम् । \newline
12. ब्र॒ह्मौ॒द॒नम् प॑चेत् पचेद् ब्रह्मौद॒नम् ब्र॑ह्मौद॒नम् प॑चेत् । \newline
13. ब्र॒ह्मौ॒द॒नमिति॑ ब्रह्म - ओ॒द॒नम् । \newline
14. प॒चे॒त् तम् तम् प॑चेत् पचे॒त् तम् । \newline
15. तम् ब्रा᳚ह्म॒णा ब्रा᳚ह्म॒णा स्तम् तम् ब्रा᳚ह्म॒णाः । \newline
16. ब्रा॒ह्म॒णा श्च॒त्वार॑ श्च॒त्वारो᳚ ब्राह्म॒णा ब्रा᳚ह्म॒णा श्च॒त्वारः॑ । \newline
17. च॒त्वारः॒ प्र प्र च॒त्वार॑ श्च॒त्वारः॒ प्र । \newline
18. प्राश्ञी॑यु रश्ञीयुः॒ प्र प्राश्ञी॑युः । \newline
19. अ॒श्ञी॒यु॒ रे॒ष ए॒षो᳚ ऽश्ञीयु रश्ञीयु रे॒षः । \newline
20. ए॒ष वै वा ए॒ष ए॒ष वै । \newline
21. वा अ॒ग्नि र॒ग्निर् वै वा अ॒ग्निः । \newline
22. अ॒ग्निर् वै᳚श्वान॒रो वै᳚श्वान॒रो᳚ ऽग्नि र॒ग्निर् वै᳚श्वान॒रः । \newline
23. वै॒श्वा॒न॒रो यद् यद् वै᳚श्वान॒रो वै᳚श्वान॒रो यत् । \newline
24. यद् ब्रा᳚ह्म॒णो ब्रा᳚ह्म॒णो यद् यद् ब्रा᳚ह्म॒णः । \newline
25. ब्रा॒ह्म॒ण ए॒षैषा ब्रा᳚ह्म॒णो ब्रा᳚ह्म॒ण ए॒षा । \newline
26. ए॒षा खलु॒ खल्वे॒ षैषा खलु॑ । \newline
27. खलु॒ वै वै खलु॒ खलु॒ वै । \newline
28. वा अ॒ग्ने र॒ग्नेर् वै वा अ॒ग्नेः । \newline
29. अ॒ग्नेः प्रि॒या प्रि॒या ऽग्ने र॒ग्नेः प्रि॒या । \newline
30. प्रि॒या त॒नू स्त॒नूः प्रि॒या प्रि॒या त॒नूः । \newline
31. त॒नूर् यद् यत् त॒नू स्त॒नूर् यत् । \newline
32. यद् वै᳚श्वान॒रो वै᳚श्वान॒रो यद् यद् वै᳚श्वान॒रः । \newline
33. वै॒श्वा॒न॒रः प्रि॒याया᳚म् प्रि॒यायां᳚ ॅवैश्वान॒रो वै᳚श्वान॒रः प्रि॒याया᳚म् । \newline
34. प्रि॒याया॑ मे॒वैव प्रि॒याया᳚म् प्रि॒याया॑ मे॒व । \newline
35. ए॒वैना॑ मेना मे॒वै वैना᳚म् । \newline
36. ए॒ना॒म् त॒नुवा᳚म् त॒नुवा॑ मेना मेनाम् त॒नुवा᳚म् । \newline
37. त॒नुवा॒म् प्रति॒ प्रति॑ त॒नुवा᳚म् त॒नुवा॒म् प्रति॑ । \newline
38. प्रति॑ ष्ठापयति स्थापयति॒ प्रति॒ प्रति॑ ष्ठापयति । \newline
39. स्था॒प॒य॒ति॒ चत॑स्र॒ श्चत॑स्रः स्थापयति स्थापयति॒ चत॑स्रः । \newline
40. चत॑स्रो धे॒नूर् धे॒नू श्चत॑स्र॒ श्चत॑स्रो धे॒नूः । \newline
41. धे॒नूर् द॑द्याद् दद्याद् धे॒नूर् धे॒नूर् द॑द्यात् । \newline
42. द॒द्या॒त् ताभि॒ स्ताभि॑र् दद्याद् दद्या॒त् ताभिः॑ । \newline
43. ताभि॑ रे॒वैव ताभि॒ स्ताभि॑ रे॒व । \newline
44. ए॒व यज॑मानो॒ यज॑मान ए॒वैव यज॑मानः । \newline
45. यज॑मानो॒ ऽमुष्मि॑न् न॒मुष्मि॒न्॒. यज॑मानो॒ यज॑मानो॒ ऽमुष्मिन्न्॑ । \newline
46. अ॒मुष्मि॑न् ॅलो॒के लो॒के॑ ऽमुष्मि॑न् न॒मुष्मि॑न् ॅलो॒के । \newline
47. लो॒के᳚ ऽग्नि म॒ग्निम् ॅलो॒के लो॒के᳚ ऽग्निम् । \newline
48. अ॒ग्निम् दु॑हे दुहे॒ ऽग्नि म॒ग्निम् दु॑हे । \newline
49. दु॒ह॒ इति॑ दुहे । \newline

\textbf{Ghana Paata } \newline

1. प्र॒ति॒ष्ठां ॅवेद॒ वेद॑ प्रति॒ष्ठाम् प्र॑ति॒ष्ठां ॅवेद॒ प्रति॒ प्रति॒ वेद॑ प्रति॒ष्ठाम् प्र॑ति॒ष्ठां ॅवेद॒ प्रति॑ । \newline
2. प्र॒ति॒ष्ठामिति॑ प्रति - स्थाम् । \newline
3. वेद॒ प्रति॒ प्रति॒ वेद॒ वेद॒ प्रत्ये॒ वैव प्रति॒ वेद॒ वेद॒ प्रत्ये॒व । \newline
4. प्रत्ये॒ वैव प्रति॒ प्रत्ये॒व ति॑ष्ठति तिष्ठ त्ये॒व प्रति॒ प्रत्ये॒व ति॑ष्ठति । \newline
5. ए॒व ति॑ष्ठति तिष्ठ त्ये॒वैव ति॑ष्ठति॒ यद् यत् ति॑ष्ठ त्ये॒वैव ति॑ष्ठति॒ यत् । \newline
6. ति॒ष्ठ॒ति॒ यद् यत् ति॑ष्ठति तिष्ठति॒ यदाज्य॒ माज्यं॒ ॅयत् ति॑ष्ठति तिष्ठति॒ यदाज्य᳚म् । \newline
7. यदाज्य॒ माज्यं॒ ॅयद् यदाज्य॑ मु॒च्छिष्ये॑ तो॒च्छिष्ये॒ताज्यं॒ ॅयद् यदाज्य॑ मु॒च्छिष्ये॑त । \newline
8. आज्य॑ मु॒च्छिष्ये॑ तो॒च्छिष्ये॒ताज्य॒ माज्य॑ मु॒च्छिष्ये॑त॒ तस्मिꣳ॒॒ स्तस्मि॑न् नु॒च्छिष्ये॒ताज्य॒ माज्य॑ मु॒च्छिष्ये॑त॒ तस्मिन्न्॑ । \newline
9. उ॒च्छिष्ये॑त॒ तस्मिꣳ॒॒ स्तस्मि॑न् नु॒च्छिष्ये॑ तो॒च्छिष्ये॑त॒ तस्मि॑न् ब्रह्मौद॒नम् ब्र॑ह्मौद॒नम् तस्मि॑न् नु॒च्छिष्ये॑ तो॒च्छिष्ये॑त॒ तस्मि॑न् ब्रह्मौद॒नम् । \newline
10. उ॒च्छिष्ये॒तेत्यु॑त् - शिष्ये॑त । \newline
11. तस्मि॑न् ब्रह्मौद॒नम् ब्र॑ह्मौद॒नम् तस्मिꣳ॒॒ स्तस्मि॑न् ब्रह्मौद॒नम् प॑चेत् पचेद् ब्रह्मौद॒नम् तस्मिꣳ॒॒ स्तस्मि॑न् ब्रह्मौद॒नम् प॑चेत् । \newline
12. ब्र॒ह्मौ॒द॒नम् प॑चेत् पचेद् ब्रह्मौद॒नम् ब्र॑ह्मौद॒नम् प॑चे॒त् तम् तम् प॑चेद् ब्रह्मौद॒नम् ब्र॑ह्मौद॒नम् प॑चे॒त् तम् । \newline
13. ब्र॒ह्मौ॒द॒नमिति॑ ब्रह्म - ओ॒द॒नम् । \newline
14. प॒चे॒त् तम् तम् प॑चेत् पचे॒त् तम् ब्रा᳚ह्म॒णा ब्रा᳚ह्म॒णा स्तम् प॑चेत् पचे॒त् तम् ब्रा᳚ह्म॒णाः । \newline
15. तम् ब्रा᳚ह्म॒णा ब्रा᳚ह्म॒णा स्तम् तम् ब्रा᳚ह्म॒णा श्च॒त्वार॑ श्च॒त्वारो᳚ ब्राह्म॒णा स्तम् तम् ब्रा᳚ह्म॒णा श्च॒त्वारः॑ । \newline
16. ब्रा॒ह्म॒णा श्च॒त्वार॑ श्च॒त्वारो᳚ ब्राह्म॒णा ब्रा᳚ह्म॒णा श्च॒त्वारः॒ प्र प्र च॒त्वारो᳚ ब्राह्म॒णा ब्रा᳚ह्म॒णा श्च॒त्वारः॒ प्र । \newline
17. च॒त्वारः॒ प्र प्र च॒त्वार॑ श्च॒त्वारः॒ प्राश्ञी॑यु रश्ञीयुः॒ प्र च॒त्वार॑ श्च॒त्वारः॒ प्राश्ञी॑युः । \newline
18. प्राश्ञी॑यु रश्ञीयुः॒ प्र प्राश्ञी॑यु रे॒ष ए॒षो᳚ ऽश्ञीयुः॒ प्र प्राश्ञी॑यु रे॒षः । \newline
19. अ॒श्ञी॒यु॒ रे॒ष ए॒षो᳚ ऽश्ञीयु रश्ञीयु रे॒ष वै वा ए॒षो᳚ ऽश्ञीयु रश्ञीयु रे॒ष वै । \newline
20. ए॒ष वै वा ए॒ष ए॒ष वा अ॒ग्नि र॒ग्निर् वा ए॒ष ए॒ष वा अ॒ग्निः । \newline
21. वा अ॒ग्नि र॒ग्निर् वै वा अ॒ग्निर् वै᳚श्वान॒रो वै᳚श्वान॒रो᳚ ऽग्निर् वै वा अ॒ग्निर् वै᳚श्वान॒रः । \newline
22. अ॒ग्निर् वै᳚श्वान॒रो वै᳚श्वान॒रो᳚ ऽग्नि र॒ग्निर् वै᳚श्वान॒रो यद् यद् वै᳚श्वान॒रो᳚ ऽग्नि र॒ग्निर् वै᳚श्वान॒रो यत् । \newline
23. वै॒श्वा॒न॒रो यद् यद् वै᳚श्वान॒रो वै᳚श्वान॒रो यद् ब्रा᳚ह्म॒णो ब्रा᳚ह्म॒णो यद् वै᳚श्वान॒रो वै᳚श्वान॒रो यद् ब्रा᳚ह्म॒णः । \newline
24. यद् ब्रा᳚ह्म॒णो ब्रा᳚ह्म॒णो यद् यद् ब्रा᳚ह्म॒ण ए॒षैषा ब्रा᳚ह्म॒णो यद् यद् ब्रा᳚ह्म॒ण ए॒षा । \newline
25. ब्रा॒ह्म॒ण ए॒षैषा ब्रा᳚ह्म॒णो ब्रा᳚ह्म॒ण ए॒षा खलु॒ खल्वे॒षा ब्रा᳚ह्म॒णो ब्रा᳚ह्म॒ण ए॒षा खलु॑ । \newline
26. ए॒षा खलु॒ खल्वे॒ षैषा खलु॒ वै वै खल्वे॒ षैषा खलु॒ वै । \newline
27. खलु॒ वै वै खलु॒ खलु॒ वा अ॒ग्ने र॒ग्नेर् वै खलु॒ खलु॒ वा अ॒ग्नेः । \newline
28. वा अ॒ग्ने र॒ग्नेर् वै वा अ॒ग्नेः प्रि॒या प्रि॒या ऽग्नेर् वै वा अ॒ग्नेः प्रि॒या । \newline
29. अ॒ग्नेः प्रि॒या प्रि॒या ऽग्ने र॒ग्नेः प्रि॒या त॒नू स्त॒नूः प्रि॒या ऽग्ने र॒ग्नेः प्रि॒या त॒नूः । \newline
30. प्रि॒या त॒नू स्त॒नूः प्रि॒या प्रि॒या त॒नूर् यद् यत् त॒नूः प्रि॒या प्रि॒या त॒नूर् यत् । \newline
31. त॒नूर् यद् यत् त॒नू स्त॒नूर् यद् वै᳚श्वान॒रो वै᳚श्वान॒रो यत् त॒नू स्त॒नूर् यद् वै᳚श्वान॒रः । \newline
32. यद् वै᳚श्वान॒रो वै᳚श्वान॒रो यद् यद् वै᳚श्वान॒रः प्रि॒याया᳚म् प्रि॒यायां᳚ ॅवैश्वान॒रो यद् यद् वै᳚श्वान॒रः प्रि॒याया᳚म् । \newline
33. वै॒श्वा॒न॒रः प्रि॒याया᳚म् प्रि॒यायां᳚ ॅवैश्वान॒रो वै᳚श्वान॒रः प्रि॒याया॑ मे॒वैव प्रि॒यायां᳚ ॅवैश्वान॒रो वै᳚श्वान॒रः प्रि॒याया॑ मे॒व । \newline
34. प्रि॒याया॑ मे॒वैव प्रि॒याया᳚म् प्रि॒याया॑ मे॒वैना॑ मेना मे॒व प्रि॒याया᳚म् प्रि॒याया॑ मे॒वैना᳚म् । \newline
35. ए॒वैना॑ मेना मे॒वै वैना᳚म् त॒नुवा᳚म् त॒नुवा॑ मेना मे॒वै वैना᳚म् त॒नुवा᳚म् । \newline
36. ए॒ना॒म् त॒नुवा᳚म् त॒नुवा॑ मेना मेनाम् त॒नुवा॒म् प्रति॒ प्रति॑ त॒नुवा॑ मेना मेनाम् त॒नुवा॒म् प्रति॑ । \newline
37. त॒नुवा॒म् प्रति॒ प्रति॑ त॒नुवा᳚म् त॒नुवा॒म् प्रति॑ ष्ठापयति स्थापयति॒ प्रति॑ त॒नुवा᳚म् त॒नुवा॒म् प्रति॑ ष्ठापयति । \newline
38. प्रति॑ ष्ठापयति स्थापयति॒ प्रति॒ प्रति॑ ष्ठापयति॒ चत॑स्र॒ श्चत॑स्रः स्थापयति॒ प्रति॒ प्रति॑ ष्ठापयति॒ चत॑स्रः । \newline
39. स्था॒प॒य॒ति॒ चत॑स्र॒ श्चत॑स्रः स्थापयति स्थापयति॒ चत॑स्रो धे॒नूर् धे॒नू श्चत॑स्रः स्थापयति स्थापयति॒ चत॑स्रो धे॒नूः । \newline
40. चत॑स्रो धे॒नूर् धे॒नू श्चत॑स्र॒ श्चत॑स्रो धे॒नूर् द॑द्याद् दद्याद् धे॒नू श्चत॑स्र॒ श्चत॑स्रो धे॒नूर् द॑द्यात् । \newline
41. धे॒नूर् द॑द्याद् दद्याद् धे॒नूर् धे॒नूर् द॑द्या॒त् ताभि॒ स्ताभि॑र् दद्याद् धे॒नूर् धे॒नूर् द॑द्या॒त् ताभिः॑ । \newline
42. द॒द्या॒त् ताभि॒ स्ताभि॑र् दद्याद् दद्या॒त् ताभि॑ रे॒वैव ताभि॑र् दद्याद् दद्या॒त् ताभि॑ रे॒व । \newline
43. ताभि॑ रे॒वैव ताभि॒ स्ताभि॑ रे॒व यज॑मानो॒ यज॑मान ए॒व ताभि॒ स्ताभि॑ रे॒व यज॑मानः । \newline
44. ए॒व यज॑मानो॒ यज॑मान ए॒वैव यज॑मानो॒ ऽमुष्मि॑न् न॒मुष्मि॒न्॒. यज॑मान ए॒वैव यज॑मानो॒ ऽमुष्मिन्न्॑ । \newline
45. यज॑मानो॒ ऽमुष्मि॑न् न॒मुष्मि॒न्॒. यज॑मानो॒ यज॑मानो॒ ऽमुष्मि॑न् ॅलो॒के लो॒के॑ ऽमुष्मि॒न्॒. यज॑मानो॒ यज॑मानो॒ ऽमुष्मि॑न् ॅलो॒के । \newline
46. अ॒मुष्मि॑न् ॅलो॒के लो॒के॑ ऽमुष्मि॑न् न॒मुष्मि॑न् ॅलो॒के᳚ ऽग्नि म॒ग्निम् ॅलो॒के॑ ऽमुष्मि॑न् न॒मुष्मि॑न् ॅलो॒के᳚ ऽग्निम् । \newline
47. लो॒के᳚ ऽग्नि म॒ग्निम् ॅलो॒के लो॒के᳚ ऽग्निम् दु॑हे दुहे॒ ऽग्निम् ॅलो॒के लो॒के᳚ ऽग्निम् दु॑हे । \newline
48. अ॒ग्निम् दु॑हे दुहे॒ ऽग्नि म॒ग्निम् दु॑हे । \newline
49. दु॒ह॒ इति॑ दुहे । \newline
\pagebreak
\markright{ TS 5.7.4.1  \hfill https://www.vedavms.in \hfill}

\section{ TS 5.7.4.1 }

\textbf{TS 5.7.4.1 } \newline
\textbf{Samhita Paata} \newline

चित्तिं॑ जुहोमि॒ मन॑सा घृ॒तेनेत्या॒हादा᳚भ्या॒ वै नामै॒षाऽऽ*हु॑तिर्वैश्वकर्म॒णी नैनं॑ चिक्या॒नं भ्रातृ॑व्यो दभ्नो॒त्यथो॑ दे॒वता॑ ए॒वाव॑ रु॒न्धे ऽग्ने॒ तम॒द्येति॑ प॒ङ्क्त्या जु॑होति प॒ङ्क्त्याऽऽहु॑त्या यज्ञ्मु॒खमा र॑भते स॒प्त ते॑ अग्ने स॒मिधः॑ स॒प्तजि॒ह्वा इत्या॑ह॒ होत्रा॑ ए॒वाव॑ रुन्धे॒ ऽग्निर्दे॒वेभ्योऽपा᳚क्रामद्-भाग॒धेय॑ - [  ] \newline

\textbf{Pada Paata} \newline

चित्ति᳚म् । जु॒हो॒मि॒ । मन॑सा । घृ॒तेन॑ । इति॑ । आ॒ह॒ । अदा᳚भ्या । वै । नाम॑ । ए॒षा । आहु॑ति॒रित्या-हु॒तिः॒ । वै॒श्व॒कर्म॒णीति॑ वैश्व - क॒र्म॒णी । न । ए॒न॒म् । चि॒क्या॒नम् । भ्रातृ॑व्यः । द॒भ्नो॒ति॒ । अथो॒ इति॑ । दे॒वताः᳚ । ए॒व । अवेति॑ । रु॒न्धे॒ । अग्ने᳚ । तम् । अ॒द्य । इति॑ । प॒ङ्क्त्या । जु॒हो॒ति॒ । प॒ङ्क्त्या । आहु॒त्येत्या - हु॒त्या॒ । य॒ज्ञ्॒मु॒खमिति॑ यज्ञ् - मु॒खम् । एति॑ । र॒भ॒ते॒ । स॒प्त । ते॒ । अ॒ग्ने॒ । स॒मिध॒ इति॑ सं - इधः॑ । स॒प्त । जि॒ह्वाः । इति॑ । आ॒ह॒ । होत्राः᳚ । ए॒व । अवेति॑ । रु॒न्धे॒ । अ॒ग्निः । दे॒वेभ्यः॑ । अपेति॑ । अ॒क्रा॒म॒त् । भा॒ग॒धेय॒मिति॑ भाग - धेय᳚म् ।  \newline


\textbf{Krama Paata} \newline

चित्ति॑म् जुहोमि । जु॒हो॒मि॒ मन॑सा । मन॑सा घृ॒तेन॑ । घृ॒तेनेति॑ । इत्या॑ह । आ॒हादा᳚भ्या । अदा᳚भ्या॒ वै । वै नाम॑ । नामै॒षा । ए॒षाऽऽहु॑तिः । आहु॑तिर् वैश्वकर्म॒णी । आहु॑ति॒रित्या - हु॒तिः॒ । वै॒श्व॒क॒र्म॒णी न । वै॒श्व॒क॒र्म॒णीति॑ वैश्व - क॒र्म॒णी । नैन᳚म् । ए॒न॒म् चि॒क्या॒नम् । चि॒क्या॒नम् भ्रातृ॑व्यः । भ्रातृ॑व्यो दभ्नोति । द॒भ्नो॒त्यथो᳚ । अथो॑ दे॒वताः᳚ । अथो॒ इत्यथो᳚ । दे॒वता॑ ए॒व । ए॒वाव॑ । अव॑ रुन्धे । रु॒न्धेऽग्ने᳚ । अग्ने॒ तम् । तम॑द्य । अ॒द्येति॑ । इति॑ प॒ङ्क्त्या । प॒ङ्क्त्या जु॑होति । जु॒हो॒ति॒ प॒ङ्क्त्या । प॒ङ्क्त्याऽऽहु॑त्या । आहु॑त्या यज्ञ्मु॒खम् । आहु॒त्येत्या - हु॒त्या॒ । य॒ज्ञ्॒मु॒खमा । य॒ज्ञ्॒मु॒खमिति॑ यज्ञ् - मु॒खम् । आ र॑भते । र॒भ॒ते॒ स॒प्त । स॒प्त ते᳚ । ते॒ अ॒ग्ने॒ । अ॒ग्ने॒ स॒मिधः॑ । स॒मिधः॑ स॒प्त । स॒मिध॒ इति॑ सम् - इधः॑ । स॒प्त जि॒ह्वाः । जि॒ह्वा इति॑ । इत्या॑ह । आ॒ह॒ होत्राः᳚ । होत्रा॑ ए॒व । ए॒वाव॑ । अव॑ रुन्धे । रु॒न्धे॒ऽग्निः । अ॒ग्निर् दे॒वेभ्यः॑ । दे॒वेभ्योऽप॑ । अपा᳚क्रामत् । अ॒क्रा॒म॒द् भा॒ग॒धेय᳚म् । भा॒ग॒धेय॑मि॒च्छमा॑नः । भा॒ग॒धेय॒मिति॑ भाग - धेय᳚म् \newline

\textbf{Jatai Paata} \newline

1. चित्ति॑म् जुहोमि जुहोमि॒ चित्ति॒म् चित्ति॑म् जुहोमि । \newline
2. जु॒हो॒मि॒ मन॑सा॒ मन॑सा जुहोमि जुहोमि॒ मन॑सा । \newline
3. मन॑सा घृ॒तेन॑ घृ॒तेन॒ मन॑सा॒ मन॑सा घृ॒तेन॑ । \newline
4. घृ॒तेनेतीति॑ घृ॒तेन॑ घृ॒तेनेति॑ । \newline
5. इत्या॑ हा॒हे तीत्या॑ह । \newline
6. आ॒हा दा॒भ्या ऽदा᳚भ्या ऽऽहा॒हा दा᳚भ्या । \newline
7. अदा᳚भ्या॒ वै वा अदा॒भ्या ऽदा᳚भ्या॒ वै । \newline
8. वै नाम॒ नाम॒ वै वै नाम॑ । \newline
9. नामै॒ षैषा नाम॒ नामै॒षा । \newline
10. ए॒षा ऽऽहु॑ति॒ राहु॑ति रे॒षैषा ऽऽहु॑तिः । \newline
11. आहु॑तिर् वैश्वकर्म॒णी वै᳚श्वकर्म॒ ण्याहु॑ति॒ राहु॑तिर् वैश्वकर्म॒णी । \newline
12. आहु॑ति॒रित्या - हु॒तिः॒ । \newline
13. वै॒श्व॒क॒र्म॒णी न न वै᳚श्वकर्म॒णी वै᳚श्वकर्म॒णी न । \newline
14. वै॒श्व॒क॒र्म॒णीति॑ वैश्व - क॒र्म॒णी । \newline
15. नैन॑ मेन॒म् न नैन᳚म् । \newline
16. ए॒न॒म् चि॒क्या॒नम् चि॑क्या॒न मे॑न मेनम् चिक्या॒नम् । \newline
17. चि॒क्या॒नम् भ्रातृ॑व्यो॒ भ्रातृ॑व्यश्चिक्या॒नम् चि॑क्या॒नम् भ्रातृ॑व्यः । \newline
18. भ्रातृ॑व्यो दभ्नोति दभ्नोति॒ भ्रातृ॑व्यो॒ भ्रातृ॑व्यो दभ्नोति । \newline
19. द॒भ्नो॒ त्यथो॒ अथो॑ दभ्नोति दभ्नो॒ त्यथो᳚ । \newline
20. अथो॑ दे॒वता॑ दे॒वता॒ अथो॒ अथो॑ दे॒वताः᳚ । \newline
21. अथो॒ इत्यथो᳚ । \newline
22. दे॒वता॑ ए॒वैव दे॒वता॑ दे॒वता॑ ए॒व । \newline
23. ए॒वावा वै॒वै वाव॑ । \newline
24. अव॑ रुन्धे रु॒न्धे ऽवाव॑ रुन्धे । \newline
25. रु॒न्धे ऽग्ने ऽग्ने॑ रुन्धे रु॒न्धे ऽग्ने᳚ । \newline
26. अग्ने॒ तम् त मग्ने ऽग्ने॒ तम् । \newline
27. त म॒द्याद्य तम् त म॒द्य । \newline
28. अ॒द्येती त्य॒द्याद्येति॑ । \newline
29. इति॑ प॒ङ्क्त्या प॒ङ्क्त्ये तीति॑ प॒ङ्क्त्या । \newline
30. प॒ङ्क्त्या जु॑होति जुहोति प॒ङ्क्त्या प॒ङ्क्त्या जु॑होति । \newline
31. जु॒हो॒ति॒ प॒ङ्क्त्या प॒ङ्क्त्या जु॑होति जुहोति प॒ङ्क्त्या । \newline
32. प॒ङ्क्त्या ऽऽहु॒त्या ऽऽहु॑त्या प॒ङ्क्त्या प॒ङ्क्त्या ऽऽहु॑त्या । \newline
33. आहु॑त्या यज्ञ्मु॒खं ॅय॑ज्ञ्मु॒ख माहु॒त्या ऽऽहु॑त्या यज्ञ्मु॒खम् । \newline
34. आहु॒त्येत्या - हु॒त्या॒ । \newline
35. य॒ज्ञ्॒मु॒ख मा य॑ज्ञ्मु॒खं ॅय॑ज्ञ्मु॒ख मा । \newline
36. य॒ज्ञ्॒मु॒खमिति॑ यज्ञ् - मु॒खम् । \newline
37. आ र॑भते रभत॒ आ र॑भते । \newline
38. र॒भ॒ते॒ स॒प्त स॒प्त र॑भते रभते स॒प्त । \newline
39. स॒प्त ते॑ ते स॒प्त स॒प्त ते᳚ । \newline
40. ते॒ अ॒ग्ने॒ ऽग्ने॒ ते॒ ते॒ अ॒ग्ने॒ । \newline
41. अ॒ग्ने॒ स॒मिधः॑ स॒मिधो᳚ ऽग्ने ऽग्ने स॒मिधः॑ । \newline
42. स॒मिधः॑ स॒प्त स॒प्त स॒मिधः॑ स॒मिधः॑ स॒प्त । \newline
43. स॒मिध॒ इति॑ सं - इधः॑ । \newline
44. स॒प्त जि॒ह्वा जि॒ह्वाः स॒प्त स॒प्त जि॒ह्वाः । \newline
45. जि॒ह्वा इतीति॑ जि॒ह्वा जि॒ह्वा इति॑ । \newline
46. इत्या॑ हा॒हे तीत्या॑ह । \newline
47. आ॒ह॒ होत्रा॒ होत्रा॑ आहाह॒ होत्राः᳚ । \newline
48. होत्रा॑ ए॒वैव होत्रा॒ होत्रा॑ ए॒व । \newline
49. ए॒वावा वै॒वै वाव॑ । \newline
50. अव॑ रुन्धे रु॒न्धे ऽवाव॑ रुन्धे । \newline
51. रु॒न्धे॒ ऽग्नि र॒ग्नी रु॑न्धे रुन्धे॒ ऽग्निः । \newline
52. अ॒ग्निर् दे॒वेभ्यो॑ दे॒वेभ्यो॒ ऽग्नि र॒ग्निर् दे॒वेभ्यः॑ । \newline
53. दे॒वेभ्यो ऽपाप॑ दे॒वेभ्यो॑ दे॒वेभ्यो ऽप॑ । \newline
54. अपा᳚ क्राम दक्राम॒ दपापा᳚ क्रामत् । \newline
55. अ॒क्रा॒म॒द् भा॒ग॒धेय॑म् भाग॒धेय॑ मक्राम दक्रामद् भाग॒धेय᳚म् । \newline
56. भा॒ग॒धेय॑ मि॒च्छमा॑न इ॒च्छमा॑नो भाग॒धेय॑म् भाग॒धेय॑ मि॒च्छमा॑नः । \newline
57. भा॒ग॒धेय॒मिति॑ भाग - धेय᳚म् । \newline

\textbf{Ghana Paata } \newline

1. चित्ति॑म् जुहोमि जुहोमि॒ चित्ति॒म् चित्ति॑म् जुहोमि॒ मन॑सा॒ मन॑सा जुहोमि॒ चित्ति॒म् चित्ति॑म् जुहोमि॒ मन॑सा । \newline
2. जु॒हो॒मि॒ मन॑सा॒ मन॑सा जुहोमि जुहोमि॒ मन॑सा घृ॒तेन॑ घृ॒तेन॒ मन॑सा जुहोमि जुहोमि॒ मन॑सा घृ॒तेन॑ । \newline
3. मन॑सा घृ॒तेन॑ घृ॒तेन॒ मन॑सा॒ मन॑सा घृ॒तेने तीति॑ घृ॒तेन॒ मन॑सा॒ मन॑सा घृ॒तेनेति॑ । \newline
4. घृ॒तेने तीति॑ घृ॒तेन॑ घृ॒तेने त्या॑हा॒हेति॑ घृ॒तेन॑ घृ॒तेने त्या॑ह । \newline
5. इत्या॑हा॒हे तीत्या॒हा दा॒भ्या ऽदा᳚भ्या॒ ऽऽहे तीत्या॒हा दा᳚भ्या । \newline
6. आ॒हादा॒भ्या ऽदा᳚भ्या ऽऽहा॒हा दा᳚भ्या॒ वै वा अदा᳚भ्या ऽऽहा॒हा दा᳚भ्या॒ वै । \newline
7. अदा᳚भ्या॒ वै वा अदा॒भ्या ऽदा᳚भ्या॒ वै नाम॒ नाम॒ वा अदा॒भ्या ऽदा᳚भ्या॒ वै नाम॑ । \newline
8. वै नाम॒ नाम॒ वै वै नामै॒ षैषा नाम॒ वै वै नामै॒षा । \newline
9. नामै॒ षैषा नाम॒ नामै॒षा ऽऽहु॑ति॒ राहु॑ति रे॒षा नाम॒ नामै॒षा ऽऽहु॑तिः । \newline
10. ए॒षा ऽऽहु॑ति॒ राहु॑ति रे॒षैषा ऽऽहु॑तिर् वैश्वकर्म॒णी वै᳚श्वकर्म॒ ण्याहु॑ति रे॒षैषा ऽऽहु॑तिर् वैश्वकर्म॒णी । \newline
11. आहु॑तिर् वैश्वकर्म॒णी वै᳚श्वकर्म॒ ण्याहु॑ति॒ राहु॑तिर् वैश्वकर्म॒णी न न वै᳚श्वकर्म॒ ण्याहु॑ति॒ राहु॑तिर् वैश्वकर्म॒णी न । \newline
12. आहु॑ति॒रित्या - हु॒तिः॒ । \newline
13. वै॒श्व॒क॒र्म॒णी न न वै᳚श्वकर्म॒णी वै᳚श्वकर्म॒णी नैन॑ मेन॒म् न वै᳚श्वकर्म॒णी वै᳚श्वकर्म॒णी नैन᳚म् । \newline
14. वै॒श्व॒क॒र्म॒णीति॑ वैश्व - क॒र्म॒णी । \newline
15. नैन॑ मेन॒म् न नैन॑म् चिक्या॒नम् चि॑क्या॒न मे॑न॒म् न नैन॑म् चिक्या॒नम् । \newline
16. ए॒न॒म् चि॒क्या॒नम् चि॑क्या॒न मे॑न मेनम् चिक्या॒नम् भ्रातृ॑व्यो॒ भ्रातृ॑व्य श्चिक्या॒न मे॑न मेनम् चिक्या॒नम् भ्रातृ॑व्यः । \newline
17. चि॒क्या॒नम् भ्रातृ॑व्यो॒ भ्रातृ॑व्य श्चिक्या॒नम् चि॑क्या॒नम् भ्रातृ॑व्यो दभ्नोति दभ्नोति॒ भ्रातृ॑व्य श्चिक्या॒नम् चि॑क्या॒नम् भ्रातृ॑व्यो दभ्नोति । \newline
18. भ्रातृ॑व्यो दभ्नोति दभ्नोति॒ भ्रातृ॑व्यो॒ भ्रातृ॑व्यो दभ्नो॒ त्यथो॒ अथो॑ दभ्नोति॒ भ्रातृ॑व्यो॒ भ्रातृ॑व्यो दभ्नो॒ त्यथो᳚ । \newline
19. द॒भ्नो॒ त्यथो॒ अथो॑ दभ्नोति दभ्नो॒ त्यथो॑ दे॒वता॑ दे॒वता॒ अथो॑ दभ्नोति दभ्नो॒ त्यथो॑ दे॒वताः᳚ । \newline
20. अथो॑ दे॒वता॑ दे॒वता॒ अथो॒ अथो॑ दे॒वता॑ ए॒वैव दे॒वता॒ अथो॒ अथो॑ दे॒वता॑ ए॒व । \newline
21. अथो॒ इत्यथो᳚ । \newline
22. दे॒वता॑ ए॒वैव दे॒वता॑ दे॒वता॑ ए॒वावा वै॒व दे॒वता॑ दे॒वता॑ ए॒वाव॑ । \newline
23. ए॒वावा वै॒वै वाव॑ रुन्धे रु॒न्धे ऽवै॒वै वाव॑ रुन्धे । \newline
24. अव॑ रुन्धे रु॒न्धे ऽवाव॑ रु॒न्धे ऽग्ने ऽग्ने॑ रु॒न्धे ऽवाव॑ रु॒न्धे ऽग्ने᳚ । \newline
25. रु॒न्धे ऽग्ने ऽग्ने॑ रुन्धे रु॒न्धे ऽग्ने॒ तम् त मग्ने॑ रुन्धे रु॒न्धे ऽग्ने॒ तम् । \newline
26. अग्ने॒ तम् त मग्ने ऽग्ने॒ त म॒द्याद्य त मग्ने ऽग्ने॒ त म॒द्य । \newline
27. त म॒द्याद्य तम् त म॒द्ये तीत्य॒द्य तम् त म॒द्येति॑ । \newline
28. अ॒द्येती त्य॒द्याद्येति॑ प॒ङ्क्त्या प॒ङ्क्त्ये त्य॒द्याद्येति॑ प॒ङ्क्त्या । \newline
29. इति॑ प॒ङ्क्त्या प॒ङ्क्त्येतीति॑ प॒ङ्क्त्या जु॑होति जुहोति प॒ङ्क्त्येतीति॑ प॒ङ्क्त्या जु॑होति । \newline
30. प॒ङ्क्त्या जु॑होति जुहोति प॒ङ्क्त्या प॒ङ्क्त्या जु॑होति प॒ङ्क्त्या प॒ङ्क्त्या जु॑होति प॒ङ्क्त्या प॒ङ्क्त्या जु॑होति प॒ङ्क्त्या । \newline
31. जु॒हो॒ति॒ प॒ङ्क्त्या प॒ङ्क्त्या जु॑होति जुहोति प॒ङ्क्त्या ऽऽहु॒त्या ऽऽहु॑त्या प॒ङ्क्त्या जु॑होति जुहोति प॒ङ्क्त्या ऽऽहु॑त्या । \newline
32. प॒ङ्क्त्या ऽऽहु॒त्या ऽऽहु॑त्या प॒ङ्क्त्या प॒ङ्क्त्या ऽऽहु॑त्या यज्ञ्मु॒खं ॅय॑ज्ञ्मु॒ख माहु॑त्या प॒ङ्क्त्या प॒ङ्क्त्या ऽऽहु॑त्या यज्ञ्मु॒खम् । \newline
33. आहु॑त्या यज्ञ्मु॒खं ॅय॑ज्ञ्मु॒ख माहु॒त्या ऽऽहु॑त्या यज्ञ्मु॒ख मा य॑ज्ञ्मु॒ख माहु॒त्या ऽऽहु॑त्या यज्ञ्मु॒ख मा । \newline
34. आहु॒त्येत्या - हु॒त्या॒ । \newline
35. य॒ज्ञ्॒मु॒ख मा य॑ज्ञ्मु॒खं ॅय॑ज्ञ्मु॒ख मा र॑भते रभत॒ आ य॑ज्ञ्मु॒खं ॅय॑ज्ञ्मु॒ख मा र॑भते । \newline
36. य॒ज्ञ्॒मु॒खमिति॑ यज्ञ् - मु॒खम् । \newline
37. आ र॑भते रभत॒ आ र॑भते स॒प्त स॒प्त र॑भत॒ आ र॑भते स॒प्त । \newline
38. र॒भ॒ते॒ स॒प्त स॒प्त र॑भते रभते स॒प्त ते॑ ते स॒प्त र॑भते रभते स॒प्त ते᳚ । \newline
39. स॒प्त ते॑ ते स॒प्त स॒प्त ते॑ अग्ने ऽग्ने ते स॒प्त स॒प्त ते॑ अग्ने । \newline
40. ते॒ अ॒ग्ने॒ ऽग्ने॒ ते॒ ते॒ अ॒ग्ने॒ स॒मिधः॑ स॒मिधो᳚ ऽग्ने ते ते अग्ने स॒मिधः॑ । \newline
41. अ॒ग्ने॒ स॒मिधः॑ स॒मिधो᳚ ऽग्ने ऽग्ने स॒मिधः॑ स॒प्त स॒प्त स॒मिधो᳚ ऽग्ने ऽग्ने स॒मिधः॑ स॒प्त । \newline
42. स॒मिधः॑ स॒प्त स॒प्त स॒मिधः॑ स॒मिधः॑ स॒प्त जि॒ह्वा जि॒ह्वाः स॒प्त स॒मिधः॑ स॒मिधः॑ स॒प्त जि॒ह्वाः । \newline
43. स॒मिध॒ इति॑ सं - इधः॑ । \newline
44. स॒प्त जि॒ह्वा जि॒ह्वाः स॒प्त स॒प्त जि॒ह्वा इतीति॑ जि॒ह्वाः स॒प्त स॒प्त जि॒ह्वा इति॑ । \newline
45. जि॒ह्वा इतीति॑ जि॒ह्वा जि॒ह्वा इत्या॑हा॒ हेति॑ जि॒ह्वा जि॒ह्वा इत्या॑ह । \newline
46. इत्या॑हा॒हे तीत्या॑ह॒ होत्रा॒ होत्रा॑ आ॒हे तीत्या॑ह॒ होत्राः᳚ । \newline
47. आ॒ह॒ होत्रा॒ होत्रा॑ आहाह॒ होत्रा॑ ए॒वैव होत्रा॑ आहाह॒ होत्रा॑ ए॒व । \newline
48. होत्रा॑ ए॒वैव होत्रा॒ होत्रा॑ ए॒वावा वै॒व होत्रा॒ होत्रा॑ ए॒वाव॑ । \newline
49. ए॒वावा वै॒वै वाव॑ रुन्धे रु॒न्धे ऽवै॒वै वाव॑ रुन्धे । \newline
50. अव॑ रुन्धे रु॒न्धे ऽवाव॑ रुन्धे॒ ऽग्नि र॒ग्नी रु॒न्धे ऽवाव॑ रुन्धे॒ ऽग्निः । \newline
51. रु॒न्धे॒ ऽग्नि र॒ग्नी रु॑न्धे रुन्धे॒ ऽग्निर् दे॒वेभ्यो॑ दे॒वेभ्यो॒ ऽग्नी रु॑न्धे रुन्धे॒ ऽग्निर् दे॒वेभ्यः॑ । \newline
52. अ॒ग्निर् दे॒वेभ्यो॑ दे॒वेभ्यो॒ ऽग्नि र॒ग्निर् दे॒वेभ्यो ऽपाप॑ दे॒वेभ्यो॒ ऽग्नि र॒ग्निर् दे॒वेभ्यो ऽप॑ । \newline
53. दे॒वेभ्यो ऽपाप॑ दे॒वेभ्यो॑ दे॒वेभ्यो ऽपा᳚क्राम दक्राम॒ दप॑ दे॒वेभ्यो॑ दे॒वेभ्यो ऽपा᳚क्रामत् । \newline
54. अपा᳚क्राम दक्राम॒ दपापा᳚क्रामद् भाग॒धेय॑म् भाग॒धेय॑ मक्राम॒ दपापा᳚क्रामद् भाग॒धेय᳚म् । \newline
55. अ॒क्रा॒म॒द् भा॒ग॒धेय॑म् भाग॒धेय॑ मक्राम दक्रामद् भाग॒धेय॑ मि॒च्छमा॑न इ॒च्छमा॑नो भाग॒धेय॑ मक्राम दक्रामद् भाग॒धेय॑ मि॒च्छमा॑नः । \newline
56. भा॒ग॒धेय॑ मि॒च्छमा॑न इ॒च्छमा॑नो भाग॒धेय॑म् भाग॒धेय॑ मि॒च्छमा॑न॒ स्तस्मै॒ तस्मा॑ इ॒च्छमा॑नो भाग॒धेय॑म् भाग॒धेय॑ मि॒च्छमा॑न॒ स्तस्मै᳚ । \newline
57. भा॒ग॒धेय॒मिति॑ भाग - धेय᳚म् । \newline
\pagebreak
\markright{ TS 5.7.4.2  \hfill https://www.vedavms.in \hfill}

\section{ TS 5.7.4.2 }

\textbf{TS 5.7.4.2 } \newline
\textbf{Samhita Paata} \newline

-मि॒च्छमा॑न॒स्तस्मा॑ ए॒तद्-भा॑ग॒धेयं॒ प्राय॑च्छन्ने॒तद्वा अ॒ग्नेर॑ग्निहो॒त्रमे॒तर्.हि॒ खलु॒ वा ए॒ष जा॒तो यर्.हि॒ सर्व॑श्चि॒तो जा॒तायै॒वास्मा॒ अन्न॒मपि॑ दधाति॒ स ए॑नं प्री॒तः प्री॑णाति॒ वसी॑यान् भवति ब्रह्मवा॒दिनो॑ वदन्ति॒ यदे॒ष गार्.ह॑पत्यश्ची॒यतेऽथ॒ क्वा᳚स्याऽऽहव॒नीय॒ इत्य॒सावा॑दि॒त्य इति॑ ब्रूयादे॒तस्मि॒न्॒.हि सर्वा᳚भ्यो दे॒वता᳚भ्यो॒ जुह्व॑ति॒ - [  ] \newline

\textbf{Pada Paata} \newline

इ॒च्छमा॑नः । तस्मै᳚ । ए॒तत् । भा॒ग॒धेय॒मिति॑ भाग - धेय᳚म् । प्रेति॑ । अ॒य॒च्छ॒न्न् । ए॒तत् । वै । अ॒ग्नेः । अ॒ग्नि॒हो॒त्रमित्य॑ग्नि - हो॒त्रम् । ए॒तर्.हि॑ । खलु॑ । वै । ए॒षः । जा॒तः । यर्.हि॑ । सर्वः॑ । चि॒तः । जा॒ताय॑ । ए॒व । अ॒स्मै॒ । अन्न᳚म् । अपीति॑ । द॒धा॒ति॒ । सः । ए॒न॒म् । प्री॒तः । प्री॒णा॒ति॒ । वसी॑यान् । भ॒व॒ति॒ । ब्र॒ह्म॒वा॒दिन॒ इति॑ ब्रह्म-वा॒दिनः॑ । व॒द॒न्ति॒ । यत् । ए॒षः । गार्.ह॑पत्य॒ इति॒ गार्.ह॑-प॒त्यः॒ । ची॒यते᳚ । अथ॑ । क्व॑ । अ॒स्य॒ । आ॒ह॒व॒नीय॒ इत्या᳚ -   ह॒व॒नीयः॑ । इति॑ । अ॒सौ । आ॒दि॒त्यः । इति॑ । ब्रू॒या॒त् । ए॒तस्मिन्न्॑ । हि । सर्वा᳚भ्यः । दे॒वता᳚भ्यः । जुह्व॑ति ।  \newline


\textbf{Krama Paata} \newline

इ॒च्छमा॑न॒स्तस्मै᳚ । तस्मा॑ ए॒तत् । ए॒तद् भा॑ग॒धेय᳚म् । भा॒ग॒धेय॒म् प्र । भा॒ग॒धेय॒मिति॑ भाग - धेय᳚म् । प्राय॑च्छन्न् । अ॒य॒च्छ॒न्ने॒तत् । ए॒तद् वै । वा अ॒ग्ने । अ॒ग्नेर॑ग्निहो॒त्रम् । अ॒ग्नि॒हो॒त्रमे॒तर्.हि॑ । अ॒ग्नि॒हो॒त्रमित्य॑ग्नि - हो॒त्रम् । ए॒तर्.हि॒ खलु॑ । खलु॒ वै । वा ए॒षः । ए॒ष जा॒तः । जा॒तो यर्.हि॑ । यर्.हि॒ सर्वः॑ । सर्व॑श्चि॒तः । चि॒तो जा॒ताय॑ । जा॒तायै॒व । ए॒वास्मै᳚ । अ॒स्मा॒ अन्न᳚म् । अन्न॒मपि॑ । अपि॑ दधाति । द॒धा॒ति॒ सः । स ए॑नम् । ए॒न॒म् प्री॒तः । प्री॒तः प्री॑णाति । प्री॒णा॒ति॒ वसी॑यान् । वसी॑यान् भवति । भ॒व॒ति॒ ब्र॒ह्म॒वा॒दिनः॑ । ब्र॒ह्म॒वा॒दिनो॑ वदन्ति । ब्र॒ह्म॒वा॒दिन॒ इति॑ ब्रह्म - वा॒दिनः॑ । व॒द॒न्ति॒ यत् । यदे॒षः । ए॒ष गार्.ह॑पत्यः । गार्.ह॑पत्यश्ची॒यते᳚ । गार्.ह॑पत्य॒ इति॒ गार्.ह॑ - प॒त्यः॒ । ची॒यतेऽथ॑ । अथ॒ क्व॑ । क्वा᳚स्य । अ॒स्या॒ह॒व॒नीयः॑ । आ॒ह॒व॒नीय॒ इति॑ । आ॒ह॒व॒नीय॒ इत्या᳚ - ह॒व॒नीयः॑ । इत्य॒सौ । अ॒सावा॑दि॒त्यः । आ॒दि॒त्य इति॑ । इति॑ ब्रूयात् । ब्रू॒या॒दे॒तस्मिन्न्॑ । ए॒तस्मि॒न्.॒ हि । हि सर्वा᳚भ्यः । सर्वा᳚भ्यो दे॒वता᳚भ्यः । दे॒वता᳚भ्यो॒ जुह्व॑ति । जुह्व॑ति॒ यः \newline

\textbf{Jatai Paata} \newline

1. इ॒च्छमा॑न॒ स्तस्मै॒ तस्मा॑ इ॒च्छमा॑न इ॒च्छमा॑न॒ स्तस्मै᳚ । \newline
2. तस्मा॑ ए॒त दे॒तत् तस्मै॒ तस्मा॑ ए॒तत् । \newline
3. ए॒तद् भा॑ग॒धेय॑म् भाग॒धेय॑ मे॒त दे॒तद् भा॑ग॒धेय᳚म् । \newline
4. भा॒ग॒धेय॒म् प्र प्र भा॑ग॒धेय॑म् भाग॒धेय॒म् प्र । \newline
5. भा॒ग॒धेय॒मिति॑ भाग - धेय᳚म् । \newline
6. प्राय॑च्छन् नयच्छ॒न् प्र प्राय॑च्छन्न् । \newline
7. अ॒य॒च्छ॒न् ने॒त दे॒त द॑यच्छन् नयच्छन् ने॒तत् । \newline
8. ए॒तद् वै वा ए॒त दे॒तद् वै । \newline
9. वा अ॒ग्ने र॒ग्नेर् वै वा अ॒ग्नेः । \newline
10. अ॒ग्ने र॑ग्निहो॒त्र म॑ग्निहो॒त्र म॒ग्ने र॒ग्ने र॑ग्निहो॒त्रम् । \newline
11. अ॒ग्नि॒हो॒त्र मे॒तर् ह्ये॒तर् ह्य॑ग्निहो॒त्र म॑ग्निहो॒त्र मे॒तर्.हि॑ । \newline
12. अ॒ग्नि॒हो॒त्रमित्य॑ग्नि - हो॒त्रम् । \newline
13. ए॒तर्.हि॒ खलु॒ खल्वे॒तर् ह्ये॒तर्.हि॒ खलु॑ । \newline
14. खलु॒ वै वै खलु॒ खलु॒ वै । \newline
15. वा ए॒ष ए॒ष वै वा ए॒षः । \newline
16. ए॒ष जा॒तो जा॒त ए॒ष ए॒ष जा॒तः । \newline
17. जा॒तो यर्.हि॒ यर्.हि॑ जा॒तो जा॒तो यर्.हि॑ । \newline
18. यर्.हि॒ सर्वः॒ सर्वो॒ यर्.हि॒ यर्.हि॒ सर्वः॑ । \newline
19. सर्व॑ श्चि॒त श्चि॒तः सर्वः॒ सर्व॑ श्चि॒तः । \newline
20. चि॒तो जा॒ताय॑ जा॒ताय॑ चि॒त श्चि॒तो जा॒ताय॑ । \newline
21. जा॒तायै॒ वैव जा॒ताय॑ जा॒तायै॒व । \newline
22. ए॒वास्मा॑ अस्मा ए॒वै वास्मै᳚ । \newline
23. अ॒स्मा॒ अन्न॒ मन्न॑ मस्मा अस्मा॒ अन्न᳚म् । \newline
24. अन्न॒ मप्य प्यन्न॒ मन्न॒ मपि॑ । \newline
25. अपि॑ दधाति दधा॒ त्यप्यपि॑ दधाति । \newline
26. द॒धा॒ति॒ स स द॑धाति दधाति॒ सः । \newline
27. स ए॑न मेनꣳ॒॒ स स ए॑नम् । \newline
28. ए॒न॒म् प्री॒तः प्री॒त ए॑न मेनम् प्री॒तः । \newline
29. प्री॒तः प्री॑णाति प्रीणाति प्री॒तः प्री॒तः प्री॑णाति । \newline
30. प्री॒णा॒ति॒ वसी॑या॒न्॒. वसी॑यान् प्रीणाति प्रीणाति॒ वसी॑यान् । \newline
31. वसी॑यान् भवति भवति॒ वसी॑या॒न्॒. वसी॑यान् भवति । \newline
32. भ॒व॒ति॒ ब्र॒ह्म॒वा॒दिनो᳚ ब्रह्मवा॒दिनो॑ भवति भवति ब्रह्मवा॒दिनः॑ । \newline
33. ब्र॒ह्म॒वा॒दिनो॑ वदन्ति वदन्ति ब्रह्मवा॒दिनो᳚ ब्रह्मवा॒दिनो॑ वदन्ति । \newline
34. ब्र॒ह्म॒वा॒दिन॒ इति॑ ब्रह्म - वा॒दिनः॑ । \newline
35. व॒द॒न्ति॒ यद् यद् व॑दन्ति वदन्ति॒ यत् । \newline
36. यदे॒ष ए॒ष यद् यदे॒षः । \newline
37. ए॒ष गार्.ह॑पत्यो॒ गार्.ह॑पत्य ए॒ष ए॒ष गार्.ह॑पत्यः । \newline
38. गार्.ह॑पत्य श्ची॒यते॑ ची॒यते॒ गार्.ह॑पत्यो॒ गार्.ह॑पत्य श्ची॒यते᳚ । \newline
39. गार्.ह॑पत्य॒ इति॒ गार्.ह॑ - प॒त्यः॒ । \newline
40. ची॒यते ऽथाथ॑ ची॒यते॑ ची॒यते ऽथ॑ । \newline
41. अथ॒ क्व॑ क्वाथाथ॒ क्व॑ । \newline
42. क्वा᳚स्यास्य॒ क्वा᳚(1॒) क्वा᳚स्य । \newline
43. अ॒स्या॒ ह॒व॒नीय॑ आहव॒नीयो᳚ ऽस्यास्या हव॒नीयः॑ । \newline
44. आ॒ह॒व॒नीय॒ इतीत्या॑ हव॒नीय॑ आहव॒नीय॒ इति॑ । \newline
45. आ॒ह॒व॒नीय॒ इत्या᳚ - ह॒व॒नीयः॑ । \newline
46. इत्य॒सा व॒सा वितीत्य॒सौ । \newline
47. अ॒सा वा॑दि॒त्य आ॑दि॒त्यो॑ ऽसा व॒सा वा॑दि॒त्यः । \newline
48. आ॒दि॒त्य इती त्या॑दि॒त्य आ॑दि॒त्य इति॑ । \newline
49. इति॑ ब्रूयाद् ब्रूया॒ दितीति॑ ब्रूयात् । \newline
50. ब्रू॒या॒ दे॒तस्मि॑न् ने॒तस्मि॑न् ब्रूयाद् ब्रूया दे॒तस्मिन्न्॑ । \newline
51. ए॒तस्मि॒न्॒. हि ह्ये॑तस्मि॑न् ने॒तस्मि॒न्॒. हि । \newline
52. हि सर्वा᳚भ्यः॒ सर्वा᳚भ्यो॒ हि हि सर्वा᳚भ्यः । \newline
53. सर्वा᳚भ्यो दे॒वता᳚भ्यो दे॒वता᳚भ्यः॒ सर्वा᳚भ्यः॒ सर्वा᳚भ्यो दे॒वता᳚भ्यः । \newline
54. दे॒वता᳚भ्यो॒ जुह्व॑ति॒ जुह्व॑ति दे॒वता᳚भ्यो दे॒वता᳚भ्यो॒ जुह्व॑ति । \newline
55. जुह्व॑ति॒ यो यो जुह्व॑ति॒ जुह्व॑ति॒ यः । \newline

\textbf{Ghana Paata } \newline

1. इ॒च्छमा॑न॒ स्तस्मै॒ तस्मा॑ इ॒च्छमा॑न इ॒च्छमा॑न॒ स्तस्मा॑ ए॒त दे॒तत् तस्मा॑ इ॒च्छमा॑न इ॒च्छमा॑न॒ स्तस्मा॑ ए॒तत् । \newline
2. तस्मा॑ ए॒त दे॒तत् तस्मै॒ तस्मा॑ ए॒तद् भा॑ग॒धेय॑म् भाग॒धेय॑ मे॒तत् तस्मै॒ तस्मा॑ ए॒तद् भा॑ग॒धेय᳚म् । \newline
3. ए॒तद् भा॑ग॒धेय॑म् भाग॒धेय॑ मे॒त दे॒तद् भा॑ग॒धेय॒म् प्र प्र भा॑ग॒धेय॑ मे॒त दे॒तद् भा॑ग॒धेय॒म् प्र । \newline
4. भा॒ग॒धेय॒म् प्र प्र भा॑ग॒धेय॑म् भाग॒धेय॒म् प्राय॑च्छन् नयच्छ॒न् प्र भा॑ग॒धेय॑म् भाग॒धेय॒म् प्राय॑च्छन्न् । \newline
5. भा॒ग॒धेय॒मिति॑ भाग - धेय᳚म् । \newline
6. प्राय॑च्छन् नयच्छ॒न् प्र प्राय॑च्छन् ने॒त दे॒त द॑यच्छ॒न् प्र प्राय॑च्छन् ने॒तत् । \newline
7. अ॒य॒च्छ॒न् ने॒त दे॒त द॑यच्छन् नयच्छन् ने॒तद् वै वा ए॒त द॑यच्छन् नयच्छन् ने॒तद् वै । \newline
8. ए॒तद् वै वा ए॒त दे॒तद् वा अ॒ग्ने र॒ग्नेर् वा ए॒त दे॒तद् वा अ॒ग्नेः । \newline
9. वा अ॒ग्ने र॒ग्नेर् वै वा अ॒ग्ने र॑ग्निहो॒त्र म॑ग्निहो॒त्र म॒ग्नेर् वै वा अ॒ग्ने र॑ग्निहो॒त्रम् । \newline
10. अ॒ग्ने र॑ग्निहो॒त्र म॑ग्निहो॒त्र म॒ग्ने र॒ग्ने र॑ग्निहो॒त्र मे॒तर् ह्ये॒तर् ह्य॑ग्निहो॒त्र म॒ग्ने र॒ग्ने र॑ग्निहो॒त्र मे॒तर्.हि॑ । \newline
11. अ॒ग्नि॒हो॒त्र मे॒तर् ह्ये॒तर् ह्य॑ग्निहो॒त्र म॑ग्निहो॒त्र मे॒तर्.हि॒ खलु॒ खल्वे॒तर् ह्य॑ग्निहो॒त्र म॑ग्निहो॒त्र मे॒तर्.हि॒ खलु॑ । \newline
12. अ॒ग्नि॒हो॒त्रमित्य॑ग्नि - हो॒त्रम् । \newline
13. ए॒तर्.हि॒ खलु॒ खल्वे॒तर् ह्ये॒तर्.हि॒ खलु॒ वै वै खल्वे॒तर् ह्ये॒तर्.हि॒ खलु॒ वै । \newline
14. खलु॒ वै वै खलु॒ खलु॒ वा ए॒ष ए॒ष वै खलु॒ खलु॒ वा ए॒षः । \newline
15. वा ए॒ष ए॒ष वै वा ए॒ष जा॒तो जा॒त ए॒ष वै वा ए॒ष जा॒तः । \newline
16. ए॒ष जा॒तो जा॒त ए॒ष ए॒ष जा॒तो यर्.हि॒ यर्.हि॑ जा॒त ए॒ष ए॒ष जा॒तो यर्.हि॑ । \newline
17. जा॒तो यर्.हि॒ यर्.हि॑ जा॒तो जा॒तो यर्.हि॒ सर्वः॒ सर्वो॒ यर्.हि॑ जा॒तो जा॒तो यर्.हि॒ सर्वः॑ । \newline
18. यर्.हि॒ सर्वः॒ सर्वो॒ यर्.हि॒ यर्.हि॒ सर्व॑ श्चि॒त श्चि॒तः सर्वो॒ यर्.हि॒ यर्.हि॒ सर्व॑ श्चि॒तः । \newline
19. सर्व॑ श्चि॒त श्चि॒तः सर्वः॒ सर्व॑ श्चि॒तो जा॒ताय॑ जा॒ताय॑ चि॒तः सर्वः॒ सर्व॑ श्चि॒तो जा॒ताय॑ । \newline
20. चि॒तो जा॒ताय॑ जा॒ताय॑ चि॒त श्चि॒तो जा॒ता यै॒वैव जा॒ताय॑ चि॒त श्चि॒तो जा॒तायै॒व । \newline
21. जा॒ता यै॒वैव जा॒ताय॑ जा॒तायै॒ वास्मा॑ अस्मा ए॒व जा॒ताय॑ जा॒ता यै॒वास्मै᳚ । \newline
22. ए॒वास्मा॑ अस्मा ए॒वै वास्मा॒ अन्न॒ मन्न॑ मस्मा ए॒वै वास्मा॒ अन्न᳚म् । \newline
23. अ॒स्मा॒ अन्न॒ मन्न॑ मस्मा अस्मा॒ अन्न॒ मप्य प्यन्न॑ मस्मा अस्मा॒ अन्न॒ मपि॑ । \newline
24. अन्न॒ मप्य प्यन्न॒ मन्न॒ मपि॑ दधाति दधा॒ त्यप्यन्न॒ मन्न॒ मपि॑ दधाति । \newline
25. अपि॑ दधाति दधा॒ त्यप्यपि॑ दधाति॒ स स द॑धा॒ त्यप्यपि॑ दधाति॒ सः । \newline
26. द॒धा॒ति॒ स स द॑धाति दधाति॒ स ए॑न मेनꣳ॒॒ स द॑धाति दधाति॒ स ए॑नम् । \newline
27. स ए॑न मेनꣳ॒॒ स स ए॑नम् प्री॒तः प्री॒त ए॑नꣳ॒॒ स स ए॑नम् प्री॒तः । \newline
28. ए॒न॒म् प्री॒तः प्री॒त ए॑न मेनम् प्री॒तः प्री॑णाति प्रीणाति प्री॒त ए॑न मेनम् प्री॒तः प्री॑णाति । \newline
29. प्री॒तः प्री॑णाति प्रीणाति प्री॒तः प्री॒तः प्री॑णाति॒ वसी॑या॒न्॒. वसी॑यान् प्रीणाति प्री॒तः प्री॒तः प्री॑णाति॒ वसी॑यान् । \newline
30. प्री॒णा॒ति॒ वसी॑या॒न्॒. वसी॑यान् प्रीणाति प्रीणाति॒ वसी॑यान् भवति भवति॒ वसी॑यान् प्रीणाति प्रीणाति॒ वसी॑यान् भवति । \newline
31. वसी॑यान् भवति भवति॒ वसी॑या॒न्॒. वसी॑यान् भवति ब्रह्मवा॒दिनो᳚ ब्रह्मवा॒दिनो॑ भवति॒ वसी॑या॒न्॒. वसी॑यान् भवति ब्रह्मवा॒दिनः॑ । \newline
32. भ॒व॒ति॒ ब्र॒ह्म॒वा॒दिनो᳚ ब्रह्मवा॒दिनो॑ भवति भवति ब्रह्मवा॒दिनो॑ वदन्ति वदन्ति ब्रह्मवा॒दिनो॑ भवति भवति ब्रह्मवा॒दिनो॑ वदन्ति । \newline
33. ब्र॒ह्म॒वा॒दिनो॑ वदन्ति वदन्ति ब्रह्मवा॒दिनो᳚ ब्रह्मवा॒दिनो॑ वदन्ति॒ यद् यद् व॑दन्ति ब्रह्मवा॒दिनो᳚ ब्रह्मवा॒दिनो॑ वदन्ति॒ यत् । \newline
34. ब्र॒ह्म॒वा॒दिन॒ इति॑ ब्रह्म - वा॒दिनः॑ । \newline
35. व॒द॒न्ति॒ यद् यद् व॑दन्ति वदन्ति॒ यदे॒ष ए॒ष यद् व॑दन्ति वदन्ति॒ यदे॒षः । \newline
36. यदे॒ष ए॒ष यद् यदे॒ष गार्.ह॑पत्यो॒ गार्.ह॑पत्य ए॒ष यद् यदे॒ष गार्.ह॑पत्यः । \newline
37. ए॒ष गार्.ह॑पत्यो॒ गार्.ह॑पत्य ए॒ष ए॒ष गार्.ह॑पत्य श्ची॒यते॑ ची॒यते॒ गार्.ह॑पत्य ए॒ष ए॒ष गार्.ह॑पत्य श्ची॒यते᳚ । \newline
38. गार्.ह॑पत्य श्ची॒यते॑ ची॒यते॒ गार्.ह॑पत्यो॒ गार्.ह॑पत्य श्ची॒यते ऽथाथ॑ ची॒यते॒ गार्.ह॑पत्यो॒ गार्.ह॑पत्य श्ची॒यते ऽथ॑ । \newline
39. गार्.ह॑पत्य॒ इति॒ गार्.ह॑ - प॒त्यः॒ । \newline
40. ची॒यते ऽथाथ॑ ची॒यते॑ ची॒यते ऽथ॒ क्व॑ क्वाथ॑ ची॒यते॑ ची॒यते ऽथ॒ क्व॑ । \newline
41. अथ॒ क्व॑ क्वाथाथ॒ क्वा᳚स्यास्य॒ क्वाथाथ॒ क्वा᳚स्य । \newline
42. क्वा᳚स्यास्य॒ क्वा᳚(1॒) क्वा᳚स्या हव॒नीय॑ आहव॒नीयो᳚ ऽस्य॒ क्वा᳚(1॒) क्वा᳚स्या हव॒नीयः॑ । \newline
43. अ॒स्या॒ ह॒व॒नीय॑ आहव॒नीयो᳚ ऽस्यास्या हव॒नीय॒ इती त्या॑हव॒नीयो᳚ ऽस्यास्या हव॒नीय॒ इति॑ । \newline
44. आ॒ह॒व॒नीय॒ इतीत्या॑ हव॒नीय॑ आहव॒नीय॒ इत्य॒सा व॒सा वित्या॑हव॒नीय॑ आहव॒नीय॒ इत्य॒सौ । \newline
45. आ॒ह॒व॒नीय॒ इत्या᳚ - ह॒व॒नीयः॑ । \newline
46. इत्य॒सा व॒सा विती त्य॒सा वा॑दि॒त्य आ॑दि॒त्यो॑ ऽसा विती त्य॒सा वा॑दि॒त्यः । \newline
47. अ॒सा वा॑दि॒त्य आ॑दि॒त्यो॑ ऽसा व॒सा वा॑दि॒त्य इतीत्या॑ दि॒त्यो॑ ऽसा व॒सा वा॑दि॒त्य इति॑ । \newline
48. आ॒दि॒त्य इतीत्या॑ दि॒त्य आ॑दि॒त्य इति॑ ब्रूयाद् ब्रूया॒ दित्या॑ दि॒त्य आ॑दि॒त्य इति॑ ब्रूयात् । \newline
49. इति॑ ब्रूयाद् ब्रूया॒ दितीति॑ ब्रूया दे॒तस्मि॑न् ने॒तस्मि॑न् ब्रूया॒ दितीति॑ ब्रूया दे॒तस्मिन्न्॑ । \newline
50. ब्रू॒या॒ दे॒तस्मि॑न् ने॒तस्मि॑न् ब्रूयाद् ब्रूया दे॒तस्मि॒न्॒. हि ह्ये॑तस्मि॑न् ब्रूयाद् ब्रूया दे॒तस्मि॒न्॒. हि । \newline
51. ए॒तस्मि॒न्॒. हि ह्ये॑तस्मि॑न् ने॒तस्मि॒न्॒. हि सर्वा᳚भ्यः॒ सर्वा᳚भ्यो॒ ह्ये॑तस्मि॑न् ने॒तस्मि॒न्॒. हि सर्वा᳚भ्यः । \newline
52. हि सर्वा᳚भ्यः॒ सर्वा᳚भ्यो॒ हि हि सर्वा᳚भ्यो दे॒वता᳚भ्यो दे॒वता᳚भ्यः॒ सर्वा᳚भ्यो॒ हि हि सर्वा᳚भ्यो दे॒वता᳚भ्यः । \newline
53. सर्वा᳚भ्यो दे॒वता᳚भ्यो दे॒वता᳚भ्यः॒ सर्वा᳚भ्यः॒ सर्वा᳚भ्यो दे॒वता᳚भ्यो॒ जुह्व॑ति॒ जुह्व॑ति दे॒वता᳚भ्यः॒ सर्वा᳚भ्यः॒ सर्वा᳚भ्यो दे॒वता᳚भ्यो॒ जुह्व॑ति । \newline
54. दे॒वता᳚भ्यो॒ जुह्व॑ति॒ जुह्व॑ति दे॒वता᳚भ्यो दे॒वता᳚भ्यो॒ जुह्व॑ति॒ यो यो जुह्व॑ति दे॒वता᳚भ्यो दे॒वता᳚भ्यो॒ जुह्व॑ति॒ यः । \newline
55. जुह्व॑ति॒ यो यो जुह्व॑ति॒ जुह्व॑ति॒ य ए॒व मे॒वं ॅयो जुह्व॑ति॒ जुह्व॑ति॒ य ए॒वम् । \newline
\pagebreak
\markright{ TS 5.7.4.3  \hfill https://www.vedavms.in \hfill}

\section{ TS 5.7.4.3 }

\textbf{TS 5.7.4.3 } \newline
\textbf{Samhita Paata} \newline

य ए॒वं ॅवि॒द्वान॒ग्निं चि॑नु॒ते सा॒क्षादे॒व दे॒वता॑ ऋद्ध्नो॒त्यग्ने॑ यशस्वि॒न्॒ यश॑से॒ मम॑र्प॒येन्द्रा॑वती॒ मप॑चिती मि॒हाऽऽव॑ह । अ॒यं मू॒र्द्धा प॑रमे॒ष्ठी सु॒वर्चाः᳚ समा॒नाना॑मुत्त॒म श्लो॑को अस्तु ॥भ॒द्रं पश्य॑न्त॒ उप॑ सेदु॒रग्रे॒ तपो॑ दी॒क्षामृष॑यः सुव॒र्विदः॑ । ततः॑ क्ष॒त्रं बल॒मोज॑श्च जा॒तं तद॒स्मै दे॒वा अ॒भि सं न॑मन्तु ॥ धा॒ता वि॑धा॒ता प॑र॒मो - [  ] \newline

\textbf{Pada Paata} \newline

यः । ए॒वम् । वि॒द्वान् । अ॒ग्निम् । चि॒नु॒ते । सा॒क्षादिति॑ स - अ॒क्षात् । ए॒व । दे॒वताः᳚ । ऋ॒द्ध्नो॒ति॒ । अग्ने᳚ । य॒श॒स्वि॒न्न् । यश॑सा । इ॒मम् । अ॒र्प॒य॒ । इन्द्रा॑वती॒मितीन्द्र॑ - व॒ती॒म् । अप॑चिती॒मित्यप॑ - चि॒ती॒म् । इ॒ह । एति॑ । व॒ह॒ ॥ अ॒यम् । मू॒द्‌र्धा । प॒र॒मे॒ष्ठी । सु॒वर्चा॒ इति॑ सु - वर्चाः᳚ । स॒मा॒नाना᳚म् । उ॒त्त॒मश्लो॑क॒ इत्यु॑त्त॒म - श्लो॒कः॒ । अ॒स्तु॒ ॥ भ॒द्रम् । पश्य॑न्तः । उपेति॑ । से॒दुः॒ । अग्रे᳚ । तपः॑ । दी॒क्षाम् । ऋष॑यः । सु॒व॒र्विद॒ इति॑ सुवः - विदः॑ ॥ ततः॑ । क्ष॒त्रम् । बल᳚म् । ओजः॑ । च॒ । जा॒तम् । तत् । अ॒स्मै । दे॒वाः । अ॒भि । समिति॑ । न॒म॒न्तु॒ ॥ धा॒ता । वि॒धा॒तेति॑ वि - धा॒ता । प॒र॒मा ।  \newline


\textbf{Krama Paata} \newline

य ए॒वम् । ए॒वम् ॅवि॒द्वान् । वि॒द्वान॒ग्निम् । अ॒ग्निम् चि॑नु॒ते । चि॒नु॒ते सा॒क्षात् । सा॒क्षादे॒व । सा॒क्षादिति॑ स - अ॒क्षात् । ए॒व दे॒वताः᳚ । दे॒वता॑ ऋद्ध्नोति । ऋ॒द्ध्यो॒त्यग्ने᳚ । अग्ने॑ यशस्विन्न् । य॒श॒स्वि॒न्॒. यश॑सा । यश॑से॒मम् । इ॒मम॑र्पय । अ॒र्प॒येन्द्रा॑वतीम् । इन्द्रा॑वती॒मप॑चितीम् । इन्द्रा॑वती॒मितीन्द्र॑ - व॒ती॒म् । अप॑चितीमि॒ह । अप॑चिती॒मित्यप॑ - चि॒ती॒म् । इ॒हा । आ व॑ह । व॒हेति॑ वह ॥ अ॒यम् मू॒र्द्धा । मू॒र्द्धा प॑रमे॒ष्ठि । प॒र॒मे॒ष्ठी सु॒वर्चाः᳚ । सु॒वर्चाः᳚ समा॒नाना᳚म् । सु॒वर्चा॒ इति॑ सु - वर्चाः᳚ । स॒मा॒नाना॑मुत्त॒मश्लो॑कः । उ॒त्त॒मश्लो॑को अस्तु । उ॒त्त॒मश्लो॑क॒ इत्यु॑त्त॒म - श्लो॒कः॒ । अ॒स्त्वित्य॑स्तु ॥ भ॒द्रम् पश्य॑न्तः । पश्य॑न्त॒ उप॑ । उप॑ सेदुः । से॒दु॒रग्रे᳚ । अग्रे॒ तपः॑ । तपो॑ दी॒क्षाम् । दी॒क्षामृष॑यः । ऋष॑यः सुव॒र्विदः॑ । सु॒व॒र्विद॒ इति॑ सुवः - विदः॑ ॥ ततः॑ क्ष॒त्रम् । क्ष॒त्रम् बल᳚म् । बल॒मोजः॑ । ओज॑श्च । च॒ जा॒तम् । जा॒तम् तत् । तद॒स्मै । अ॒स्मै दे॒वाः । दे॒वा अ॒भि । अ॒भि सम् । सम् न॑मन्तु । न॒म॒न्त्विति॑ नमन्तु ॥ धा॒ता वि॑धा॒ता । वि॒धा॒ता प॑र॒मा । वि॒धा॒तेति॑ वि - धा॒ता । प॒र॒मोत \newline

\textbf{Jatai Paata} \newline

1. य ए॒व मे॒वं ॅयो य ए॒वम् । \newline
2. ए॒वं ॅवि॒द्वान्. वि॒द्वा ने॒व मे॒वं ॅवि॒द्वान् । \newline
3. वि॒द्वा न॒ग्नि म॒ग्निं ॅवि॒द्वान्. वि॒द्वा न॒ग्निम् । \newline
4. अ॒ग्निम् चि॑नु॒ते चि॑नु॒ते᳚ ऽग्नि म॒ग्निम् चि॑नु॒ते । \newline
5. चि॒नु॒ते सा॒क्षाथ् सा॒क्षाच् चि॑नु॒ते चि॑नु॒ते सा॒क्षात् । \newline
6. सा॒क्षा दे॒वैव सा॒क्षाथ् सा॒क्षा दे॒व । \newline
7. सा॒क्षादिति॑ स - अ॒क्षात् । \newline
8. ए॒व दे॒वता॑ दे॒वता॑ ए॒वैव दे॒वताः᳚ । \newline
9. दे॒वता॑ ऋद्ध्नो त्यृद्ध्नोति दे॒वता॑ दे॒वता॑ ऋद्ध्नोति । \newline
10. ऋ॒द्ध्नो॒ त्यग्ने ऽग्न॑ ऋद्ध्नो त्यृद्ध्नो॒ त्यग्ने᳚ । \newline
11. अग्ने॑ यशस्विन्. यशस्वि॒न् नग्ने ऽग्ने॑ यशस्विन्न् । \newline
12. य॒श॒स्वि॒न्॒. यश॑सा॒ यश॑सा यशस्विन्. यशस्वि॒न्॒. यश॑सा । \newline
13. यश॑से॒म मि॒मं ॅयश॑सा॒ यश॑से॒मम् । \newline
14. इ॒म म॑र्पयार् पये॒ म मि॒म म॑र्पय । \newline
15. अ॒र्प॒ येन्द्रा॑वती॒ मिन्द्रा॑वती मर्पयार् प॒येन्द्रा॑वतीम् । \newline
16. इन्द्रा॑वती॒ मप॑चिती॒ मप॑चिती॒ मिन्द्रा॑वती॒ मिन्द्रा॑वती॒ मप॑चितीम् । \newline
17. इन्द्रा॑वती॒मितीन्द्र॑ - व॒ती॒म् । \newline
18. अप॑चिती मि॒हे हाप॑चिती॒ मप॑चिती मि॒ह । \newline
19. अप॑चिती॒मित्यप॑ - चि॒ती॒म् । \newline
20. इ॒हेहे हा । \newline
21. आ व॑ह व॒हा व॑ह । \newline
22. व॒हेति॑ वह । \newline
23. अ॒यम् मू॒र्द्धा मू॒र्द्धा ऽय म॒यम् मू॒र्द्धा । \newline
24. मू॒र्द्धा प॑रमे॒ष्ठी प॑रमे॒ष्ठी मू॒र्द्धा मू॒र्द्धा प॑रमे॒ष्ठी । \newline
25. प॒र॒मे॒ष्ठी सु॒वर्चाः᳚ सु॒वर्चाः᳚ परमे॒ष्ठी प॑रमे॒ष्ठी सु॒वर्चाः᳚ । \newline
26. सु॒वर्चाः᳚ समा॒नानाꣳ॑ समा॒नानाꣳ॑ सु॒वर्चाः᳚ सु॒वर्चाः᳚ समा॒नाना᳚म् । \newline
27. सु॒वर्चा॒ इति॑ सु - वर्चाः᳚ । \newline
28. स॒मा॒नाना॑ मुत्त॒मश्लो॑क उत्त॒मश्लो॑कः समा॒नानाꣳ॑ समा॒नाना॑ मुत्त॒मश्लो॑कः । \newline
29. उ॒त्त॒मश्लो॑को अस्त्व स्तूत्त॒मश्लो॑क उत्त॒मश्लो॑को अस्तु । \newline
30. उ॒त्त॒मश्लो॑क॒ इत्यु॑त्त॒म - श्लो॒कः॒ । \newline
31. अ॒स्त्वित्य॑स्तु । \newline
32. भ॒द्रम् पश्य॑न्तः॒ पश्य॑न्तो भ॒द्रम् भ॒द्रम् पश्य॑न्तः । \newline
33. पश्य॑न्त॒ उपोप॒ पश्य॑न्तः॒ पश्य॑न्त॒ उप॑ । \newline
34. उप॑ सेदुः सेदु॒ रुपोप॑ सेदुः । \newline
35. से॒दु॒ रग्रे ऽग्रे॑ सेदुः सेदु॒ रग्रे᳚ । \newline
36. अग्रे॒ तप॒ स्तपो ऽग्रे ऽग्रे॒ तपः॑ । \newline
37. तपो॑ दी॒क्षाम् दी॒क्षाम् तप॒ स्तपो॑ दी॒क्षाम् । \newline
38. दी॒क्षा मृष॑य॒ ऋष॑यो दी॒क्षाम् दी॒क्षा मृष॑यः । \newline
39. ऋष॑यः सुव॒र्विदः॑ सुव॒र्विद॒ ऋष॑य॒ ऋष॑यः सुव॒र्विदः॑ । \newline
40. सु॒व॒र्विद॒ इति॑ सुवः - विदः॑ । \newline
41. ततः॑ क्ष॒त्रम् क्ष॒त्रम् तत॒ स्ततः॑ क्ष॒त्रम् । \newline
42. क्ष॒त्रम् बल॒म् बल॑म् क्ष॒त्रम् क्ष॒त्रम् बल᳚म् । \newline
43. बल॒ मोज॒ ओजो॒ बल॒म् बल॒ मोजः॑ । \newline
44. ओज॑श्च॒ चौज॒ ओज॑श्च । \newline
45. च॒ जा॒तम् जा॒तम् च॑ च जा॒तम् । \newline
46. जा॒तम् तत् तज् जा॒तम् जा॒तम् तत् । \newline
47. तद॒स्मा अ॒स्मै तत् तद॒स्मै । \newline
48. अ॒स्मै दे॒वा दे॒वा अ॒स्मा अ॒स्मै दे॒वाः । \newline
49. दे॒वा अ॒भ्य॑भि दे॒वा दे॒वा अ॒भि । \newline
50. अ॒भि सꣳ स म॒भ्य॑भि सम् । \newline
51. सन् न॑मन्तु नमन्तु॒ सꣳ सन् न॑मन्तु । \newline
52. न॒म॒न्त्विति॑ नमन्तु । \newline
53. धा॒ता वि॑धा॒ता वि॑धा॒ता धा॒ता धा॒ता वि॑धा॒ता । \newline
54. वि॒धा॒ता प॑र॒मा प॑र॒मा वि॑धा॒ता वि॑धा॒ता प॑र॒मा । \newline
55. वि॒धा॒तेति॑ वि - धा॒ता । \newline
56. प॒र॒मोतोत प॑र॒मा प॑र॒मोत । \newline

\textbf{Ghana Paata } \newline

1. य ए॒व मे॒वं ॅयो य ए॒वं ॅवि॒द्वान्. वि॒द्वा ने॒वं ॅयो य ए॒वं ॅवि॒द्वान् । \newline
2. ए॒वं ॅवि॒द्वान्. वि॒द्वा ने॒व मे॒वं ॅवि॒द्वा न॒ग्नि म॒ग्निं ॅवि॒द्वा ने॒व मे॒वं ॅवि॒द्वा न॒ग्निम् । \newline
3. वि॒द्वा न॒ग्नि म॒ग्निं ॅवि॒द्वान्. वि॒द्वा न॒ग्निम् चि॑नु॒ते चि॑नु॒ते᳚ ऽग्निं ॅवि॒द्वान्. वि॒द्वा न॒ग्निम् चि॑नु॒ते । \newline
4. अ॒ग्निम् चि॑नु॒ते चि॑नु॒ते᳚ ऽग्नि म॒ग्निम् चि॑नु॒ते सा॒क्षाथ् सा॒क्षाच् चि॑नु॒ते᳚ ऽग्नि म॒ग्निम् चि॑नु॒ते सा॒क्षात् । \newline
5. चि॒नु॒ते सा॒क्षाथ् सा॒क्षाच् चि॑नु॒ते चि॑नु॒ते सा॒क्षा दे॒वैव सा॒क्षाच् चि॑नु॒ते चि॑नु॒ते सा॒क्षा दे॒व । \newline
6. सा॒क्षा दे॒वैव सा॒क्षाथ् सा॒क्षा दे॒व दे॒वता॑ दे॒वता॑ ए॒व सा॒क्षाथ् सा॒क्षा दे॒व दे॒वताः᳚ । \newline
7. सा॒क्षादिति॑ स - अ॒क्षात् । \newline
8. ए॒व दे॒वता॑ दे॒वता॑ ए॒वैव दे॒वता॑ ऋद्ध्नो त्यृद्ध्नोति दे॒वता॑ ए॒वैव दे॒वता॑ ऋद्ध्नोति । \newline
9. दे॒वता॑ ऋद्ध्नो त्यृद्ध्नोति दे॒वता॑ दे॒वता॑ ऋद्ध्नो॒ त्यग्ने ऽग्न॑ ऋद्ध्नोति दे॒वता॑ दे॒वता॑ ऋद्ध्नो॒ त्यग्ने᳚ । \newline
10. ऋ॒द्ध्नो॒ त्यग्ने ऽग्न॑ ऋद्ध्नो त्यृद्ध्नो॒ त्यग्ने॑ यशस्विन्. यशस्वि॒न् नग्न॑ ऋद्ध्नो त्यृद्ध्नो॒ त्यग्ने॑ यशस्विन्न् । \newline
11. अग्ने॑ यशस्विन्. यशस्वि॒न् नग्ने ऽग्ने॑ यशस्वि॒न्॒. यश॑सा॒ यश॑सा यशस्वि॒न् नग्ने ऽग्ने॑ यशस्वि॒न्॒. यश॑सा । \newline
12. य॒श॒स्वि॒न्॒. यश॑सा॒ यश॑सा यशस्विन्. यशस्वि॒न्॒. यश॑से॒म मि॒मं ॅयश॑सा यशस्विन्. यशस्वि॒न्॒. यश॑से॒मम् । \newline
13. यश॑से॒म मि॒मं ॅयश॑सा॒ यश॑से॒म म॑र्पया र्पये॒मं ॅयश॑सा॒ यश॑से॒म म॑र्पय । \newline
14. इ॒म म॑र्पया र्पये॒म मि॒म म॑र्प॒ येन्द्रा॑वती॒ मिन्द्रा॑वती मर्पये॒म मि॒म म॑र्प॒ येन्द्रा॑वतीम् । \newline
15. अ॒र्प॒येन्द्रा॑वती॒ मिन्द्रा॑वती मर्पया र्प॒येन्द्रा॑वती॒ मप॑चिती॒ मप॑चिती॒ मिन्द्रा॑वती मर्पया र्प॒येन्द्रा॑वती॒ मप॑चितीम् । \newline
16. इन्द्रा॑वती॒ मप॑चिती॒ मप॑चिती॒ मिन्द्रा॑वती॒ मिन्द्रा॑वती॒ मप॑चिती मि॒हे हाप॑चिती॒ मिन्द्रा॑वती॒ मिन्द्रा॑वती॒ मप॑चिती मि॒ह । \newline
17. इन्द्रा॑वती॒मितीन्द्र॑ - व॒ती॒म् । \newline
18. अप॑चिती मि॒हे हाप॑चिती॒ मप॑चिती मि॒हेहा प॑चिती॒ मप॑चिती मि॒हा । \newline
19. अप॑चिती॒मित्यप॑ - चि॒ती॒म् । \newline
20. इ॒हे हेहा व॑ह व॒हे हेहा व॑ह । \newline
21. आ व॑ह व॒हा व॑ह । \newline
22. व॒हेति॑ वह । \newline
23. अ॒यम् मू॒र्द्धा मू॒र्द्धा ऽय म॒यम् मू॒र्द्धा प॑रमे॒ष्ठी प॑रमे॒ष्ठी मू॒र्द्धा ऽय म॒यम् मू॒र्द्धा प॑रमे॒ष्ठी । \newline
24. मू॒र्द्धा प॑रमे॒ष्ठी प॑रमे॒ष्ठी मू॒र्द्धा मू॒र्द्धा प॑रमे॒ष्ठी सु॒वर्चाः᳚ सु॒वर्चाः᳚ परमे॒ष्ठी मू॒र्द्धा मू॒र्द्धा प॑रमे॒ष्ठी सु॒वर्चाः᳚ । \newline
25. प॒र॒मे॒ष्ठी सु॒वर्चाः᳚ सु॒वर्चाः᳚ परमे॒ष्ठी प॑रमे॒ष्ठी सु॒वर्चाः᳚ समा॒नानाꣳ॑ समा॒नानाꣳ॑ सु॒वर्चाः᳚ परमे॒ष्ठी प॑रमे॒ष्ठी सु॒वर्चाः᳚ समा॒नाना᳚म् । \newline
26. सु॒वर्चाः᳚ समा॒नानाꣳ॑ समा॒नानाꣳ॑ सु॒वर्चाः᳚ सु॒वर्चाः᳚ समा॒नाना॑ मुत्त॒मश्लो॑क उत्त॒मश्लो॑कः समा॒नानाꣳ॑ सु॒वर्चाः᳚ सु॒वर्चाः᳚ समा॒नाना॑ मुत्त॒मश्लो॑कः । \newline
27. सु॒वर्चा॒ इति॑ सु - वर्चाः᳚ । \newline
28. स॒मा॒नाना॑ मुत्त॒मश्लो॑क उत्त॒मश्लो॑कः समा॒नानाꣳ॑ समा॒नाना॑ मुत्त॒मश्लो॑को अस्त्वस्तू त्त॒मश्लो॑कः समा॒नानाꣳ॑ समा॒नाना॑ मुत्त॒मश्लो॑को अस्तु । \newline
29. उ॒त्त॒मश्लो॑को अस्त्वस्तू त्त॒मश्लो॑क उत्त॒मश्लो॑को अस्तु । \newline
30. उ॒त्त॒मश्लो॑क॒ इत्यु॑त्त॒म - श्लो॒कः॒ । \newline
31. अ॒स्त्वित्य॑स्तु । \newline
32. भ॒द्रम् पश्य॑न्तः॒ पश्य॑न्तो भ॒द्रम् भ॒द्रम् पश्य॑न्त॒ उपोप॒ पश्य॑न्तो भ॒द्रम् भ॒द्रम् पश्य॑न्त॒ उप॑ । \newline
33. पश्य॑न्त॒ उपोप॒ पश्य॑न्तः॒ पश्य॑न्त॒ उप॑ सेदुः सेदु॒ रुप॒ पश्य॑न्तः॒ पश्य॑न्त॒ उप॑ सेदुः । \newline
34. उप॑ सेदुः सेदु॒ रुपोप॑ सेदु॒ रग्रे ऽग्रे॑ सेदु॒ रुपोप॑ सेदु॒ रग्रे᳚ । \newline
35. से॒दु॒ रग्रे ऽग्रे॑ सेदुः सेदु॒ रग्रे॒ तप॒ स्तपो ऽग्रे॑ सेदुः सेदु॒ रग्रे॒ तपः॑ । \newline
36. अग्रे॒ तप॒ स्तपो ऽग्रे ऽग्रे॒ तपो॑ दी॒क्षाम् दी॒क्षाम् तपो ऽग्रे ऽग्रे॒ तपो॑ दी॒क्षाम् । \newline
37. तपो॑ दी॒क्षाम् दी॒क्षाम् तप॒ स्तपो॑ दी॒क्षा मृष॑य॒ ऋष॑यो दी॒क्षाम् तप॒ स्तपो॑ दी॒क्षा मृष॑यः । \newline
38. दी॒क्षा मृष॑य॒ ऋष॑यो दी॒क्षाम् दी॒क्षा मृष॑यः सुव॒र्विदः॑ सुव॒र्विद॒ ऋष॑यो दी॒क्षाम् दी॒क्षा मृष॑यः सुव॒र्विदः॑ । \newline
39. ऋष॑यः सुव॒र्विदः॑ सुव॒र्विद॒ ऋष॑य॒ ऋष॑यः सुव॒र्विदः॑ । \newline
40. सु॒व॒र्विद॒ इति॑ सुवः - विदः॑ । \newline
41. ततः॑ क्ष॒त्रम् क्ष॒त्रम् तत॒ स्ततः॑ क्ष॒त्रम् बल॒म् बल॑म् क्ष॒त्रम् तत॒ स्ततः॑ क्ष॒त्रम् बल᳚म् । \newline
42. क्ष॒त्रम् बल॒म् बल॑म् क्ष॒त्रम् क्ष॒त्रम् बल॒ मोज॒ ओजो॒ बल॑म् क्ष॒त्रम् क्ष॒त्रम् बल॒ मोजः॑ । \newline
43. बल॒ मोज॒ ओजो॒ बल॒म् बल॒ मोज॑श्च॒ चौजो॒ बल॒म् बल॒ मोज॑श्च । \newline
44. ओज॑श्च॒ चौज॒ ओज॑श्च जा॒तम् जा॒तम् चौज॒ ओज॑श्च जा॒तम् । \newline
45. च॒ जा॒तम् जा॒तम् च॑ च जा॒तम् तत् तज् जा॒तम् च॑ च जा॒तम् तत् । \newline
46. जा॒तम् तत् तज् जा॒तम् जा॒तम् तद॒स्मा अ॒स्मै तज् जा॒तम् जा॒तम् तद॒स्मै । \newline
47. तद॒स्मा अ॒स्मै तत् तद॒स्मै दे॒वा दे॒वा अ॒स्मै तत् तद॒स्मै दे॒वाः । \newline
48. अ॒स्मै दे॒वा दे॒वा अ॒स्मा अ॒स्मै दे॒वा अ॒भ्य॑भि दे॒वा अ॒स्मा अ॒स्मै दे॒वा अ॒भि । \newline
49. दे॒वा अ॒भ्य॑भि दे॒वा दे॒वा अ॒भि सꣳ स म॒भि दे॒वा दे॒वा अ॒भि सम् । \newline
50. अ॒भि सꣳ स म॒भ्य॑भि सन् न॑मन्तु नमन्तु॒ स म॒भ्य॑भि सन् न॑मन्तु । \newline
51. सन् न॑मन्तु नमन्तु॒ सꣳ सन् न॑मन्तु । \newline
52. न॒म॒न्त्विति॑ नमन्तु । \newline
53. धा॒ता वि॑धा॒ता वि॑धा॒ता धा॒ता धा॒ता वि॑धा॒ता प॑र॒मा प॑र॒मा वि॑धा॒ता धा॒ता धा॒ता वि॑धा॒ता प॑र॒मा । \newline
54. वि॒धा॒ता प॑र॒मा प॑र॒मा वि॑धा॒ता वि॑धा॒ता प॑र॒मोतोत प॑र॒मा वि॑धा॒ता वि॑धा॒ता प॑र॒मोत । \newline
55. वि॒धा॒तेति॑ वि - धा॒ता । \newline
56. प॒र॒मोतोत प॑र॒मा प॑र॒मोत स॒न्दृख् स॒न्दृगु॒त प॑र॒मा प॑र॒मोत स॒न्दृक् । \newline
\pagebreak
\markright{ TS 5.7.4.4  \hfill https://www.vedavms.in \hfill}

\section{ TS 5.7.4.4 }

\textbf{TS 5.7.4.4 } \newline
\textbf{Samhita Paata} \newline

-त स॒न्दृक् प्र॒जाप॑तिः परमे॒ष्ठी वि॒राजा᳚ । स्तोमाः॒ छन्दाꣳ॑सि नि॒विदो॑ म आहुरे॒तस्मै॑ रा॒ष्ट्रम॒भि सं न॑माम ॥अ॒भ्याव॑र्तद्ध्व॒मुप॒ मेत॑ सा॒कम॒यꣳ शा॒स्ताऽधि॑पतिर्वो अस्तु । अ॒स्य वि॒ज्ञान॒मनु॒ सꣳ र॑भद्ध्वमि॒मं प॒श्चादनु॑ जीवाथ॒ सर्वे᳚ ॥रा॒ष्ट्र॒भृत॑ ए॒ता उप॑ दधात्ये॒षा वा अ॒ग्नेश्चिती॑ राष्ट्र॒भृत् तयै॒वास्मि॑न् रा॒ष्ट्रं द॑धाति ( ) रा॒ष्ट्रमे॒व भ॑वति॒ नास्मा᳚द् रा॒ष्ट्रं भ्रꣳ॑शते ॥ \newline

\textbf{Pada Paata} \newline

उ॒त । स॒दृंगिति॑ सं - दृक् । प्र॒जाप॑ति॒रिति॑ प्र॒जा - प॒तिः॒ । प॒र॒मे॒ष्ठी । वि॒राजेति॑ वि - राजा᳚ ॥ स्तोमाः᳚ । छन्दाꣳ॑सि । नि॒विद॒ इति॑ नि - विदः॑ । मे॒ । आ॒हुः॒ । ए॒तस्मै᳚ । रा॒ष्ट्रम् । अ॒भि । समिति॑ । न॒मा॒म॒ ॥ अ॒भ्याव॑र्तद्ध्व॒मित्य॑भि - आव॑र्तद्ध्वम् । उपेति॑ । मा॒ । एति॑ । इ॒त॒ । सा॒कम् । अ॒यम् । शा॒स्ता । अधि॑पति॒रित्यधि॑ - प॒तिः॒ । वः॒ । अ॒स्तु॒ ॥ अ॒स्य । वि॒ज्ञान॒मिति॑ वि - ज्ञान᳚म् । अनु॑ । समिति॑ । र॒भ॒द्ध्व॒म् । इ॒मम् । प॒श्चात् । अन्विति॑ । जी॒वा॒थ॒ । सर्वे᳚ ॥ रा॒ष्ट्र॒भृत॒ इति॑ राष्ट्र - भृतः॑ । ए॒ताः । उपेति॑ । द॒धा॒ति॒ । ए॒षा । वै । अ॒ग्नेः । चितिः॑ । रा॒ष्ट्र॒भृदिति॑ राष्ट्र - भृत् । तया᳚ । ए॒व । अ॒स्मि॒न्न् । रा॒ष्ट्रम् । द॒धा॒ति॒ ( ) । रा॒ष्ट्रम् । ए॒व । भ॒व॒ति॒ । न । अ॒स्मा॒त् । रा॒ष्ट्रम् । भ्रꣳ॒॒श॒ते॒ ॥  \newline


\textbf{Krama Paata} \newline

उ॒त स॒न्दृक् । स॒न्दृक् प्र॒जाप॑तिः । स॒न्दृगिति॑ सम् - दृक् । प्र॒जाप॑तिः परमे॒ष्ठी । प्र॒जाप॑ति॒रिति॑ प्र॒जा - प॒तिः॒ । प॒र॒मे॒ष्ठी वि॒राजा᳚ । वि॒राजेति॑ वि - राजा᳚ ॥ स्तोमा॒श्छन्दाꣳ॑सि । छन्दाꣳ॑सि नि॒विदः॑ । नि॒विदो॑ मे । नि॒विद॒ इति॑ नि - विदः॑ । म॒ आ॒हुः॒ । आ॒हु॒रे॒तस्मै᳚ । ए॒तस्मै॑ रा॒ष्ट्रम् । रा॒ष्टम॒भि । अ॒भि सम् । सम् न॑माम । न॒मा॒मेति॑ नमाम ॥ अ॒भ्याव॑र्तद्ध्व॒मुप॑ । अ॒भ्याव॑र्तद्ध्व॒मित्य॑भि - आव॑र्तद्ध्वम् । उप॒मा । मा । एत॑ । इ॒त॒ सा॒कम् । सा॒कम॒यम् । अ॒यꣳ शा॒स्ता । शा॒स्ताऽधि॑पतिः । अधि॑पतिर् वः । अधि॑पतिरि॒त्यधि॑ - प॒तिः॒ । वो॒ अ॒स्तु॒ । अ॒स्त्वित्य॑स्तु ॥ अ॒स्य वि॒ज्ञान᳚म् । वि॒ज्ञान॒मनु॑ । वि॒ज्ञान॒मिति॑ वि - ज्ञान᳚म् । अनु॒ सम् । सꣳ र॑भद्ध्वम् । र॒भ॒द्ध्व॒मि॒मम् । इ॒मम् प॒श्चात् । प॒श्चादनु॑ । अनु॑ जीवाथ । जी॒वा॒थ॒ सर्वे᳚ । सर्व॒ इति॒ सर्वे᳚ ॥ रा॒ष्ट्र॒भृत॑ ए॒ताः । रा॒ष्ट्र॒भृत॒ इति॑ राष्ट्र - भृतः॑ । ए॒ता उप॑ । उप॑ दधाति । द॒धा॒त्ये॒षा । ए॒षा वै । वा अ॒ग्नेः । अ॒ग्नेश्चितिः॑ । चिती॑ राष्ट्र॒भृत् । रा॒ष्ट्र॒भृत् तया᳚ । रा॒ष्ट्र॒भृदिति॑ राष्ट्र - भृत् । तयै॒व । ए॒वास्मिन्न्॑ । अ॒स्मि॒न् रा॒ष्ट्रम् । रा॒ष्ट्रम् द॑धाति ( ) । द॒धा॒ति॒ रा॒ष्ट्रम् । रा॒ष्ट्रमे॒व । ए॒व भ॑वति । भ॒व॒ति॒ न । नास्मा᳚त् । अ॒स्मा॒द् रा॒ष्ट्रम् । रा॒ष्ट्रम् भ्रꣳ॑शते । भ्रꣳ॒॒श॒त॒ इति॑ भ्रꣳशते । \newline

\textbf{Jatai Paata} \newline

1. उ॒त स॒न्दृख् स॒न्दृ गु॒तोत स॒न्दृक् । \newline
2. स॒न्दृक् प्र॒जाप॑तिः प्र॒जाप॑तिः स॒न्दृख् स॒न्दृक् प्र॒जाप॑तिः । \newline
3. स॒न्दृगिति॑ सं - दृक् । \newline
4. प्र॒जाप॑तिः परमे॒ष्ठी प॑रमे॒ष्ठी प्र॒जाप॑तिः प्र॒जाप॑तिः परमे॒ष्ठी । \newline
5. प्र॒जाप॑ति॒रिति॑ प्र॒जा - प॒तिः॒ । \newline
6. प॒र॒मे॒ष्ठी वि॒राजा॑ वि॒राजा॑ परमे॒ष्ठी प॑रमे॒ष्ठी वि॒राजा᳚ । \newline
7. वि॒राजेति॑ वि - राजा᳚ । \newline
8. स्तोमा॒ श्छन्दाꣳ॑सि॒ छन्दाꣳ॑सि॒ स्तोमाः॒ स्तोमा॒ श्छन्दाꣳ॑सि । \newline
9. छन्दाꣳ॑सि नि॒विदो॑ नि॒विद॒ श्छन्दाꣳ॑सि॒ छन्दाꣳ॑सि नि॒विदः॑ । \newline
10. नि॒विदो॑ मे मे नि॒विदो॑ नि॒विदो॑ मे । \newline
11. नि॒विद॒ इति॑ नि - विदः॑ । \newline
12. म॒ आ॒हु॒ रा॒हु॒र् मे॒ म॒ आ॒हुः॒ । \newline
13. आ॒हु॒ रे॒तस्मा॑ ए॒तस्मा॑ आहु राहु रे॒तस्मै᳚ । \newline
14. ए॒तस्मै॑ रा॒ष्ट्रꣳ रा॒ष्ट्र मे॒तस्मा॑ ए॒तस्मै॑ रा॒ष्ट्रम् । \newline
15. रा॒ष्ट्र म॒भ्य॑भि रा॒ष्ट्रꣳ रा॒ष्ट्र म॒भि । \newline
16. अ॒भि सꣳ स म॒भ्य॑भि सम् । \newline
17. सन् न॑माम नमाम॒ सꣳ सन् न॑माम । \newline
18. न॒मा॒मेति॑ नमाम । \newline
19. अ॒भ्याव॑र्तद्ध्व॒ मुपोपा॒ भ्याव॑र्तद्ध्व म॒भ्याव॑र्तद्ध्व॒ मुप॑ । \newline
20. अ॒भ्याव॑र्तद्ध्व॒मित्य॑भि - आव॑र्तद्ध्वम् । \newline
21. उप॑ मा॒ मोपोप॑ मा । \newline
22. मा ऽऽमा॒ मा । \newline
23. एते॒ तेत॑ । \newline
24. इ॒त॒ सा॒कꣳ सा॒क मि॑तेत सा॒कम् । \newline
25. सा॒क म॒य म॒यꣳ सा॒कꣳ सा॒क म॒यम् । \newline
26. अ॒यꣳ शा॒स्ता शा॒स्ता ऽय म॒यꣳ शा॒स्ता । \newline
27. शा॒स्ता ऽधि॑पति॒ रधि॑पतिः शा॒स्ता शा॒स्ता ऽधि॑पतिः । \newline
28. अधि॑पतिर् वो॒ वो ऽधि॑पति॒ रधि॑पतिर् वः । \newline
29. अधि॑पति॒रित्यधि॑ - प॒तिः॒ । \newline
30. वो॒ अ॒स्त्व॒स्तु॒ वो॒ वो॒ अ॒स्तु॒ । \newline
31. अ॒स्त्वित्य॑स्तु । \newline
32. अ॒स्य वि॒ज्ञानं॑ ॅवि॒ज्ञान॑ म॒स्यास्य वि॒ज्ञान᳚म् । \newline
33. वि॒ज्ञान॒ मन्वनु॑ वि॒ज्ञानं॑ ॅवि॒ज्ञान॒ मनु॑ । \newline
34. वि॒ज्ञान॒मिति॑ वि - ज्ञान᳚म् । \newline
35. अनु॒ सꣳ स मन्वनु॒ सम् । \newline
36. सꣳ र॑भद्ध्वꣳ रभद्ध्वꣳ॒॒ सꣳ सꣳ र॑भद्ध्वम् । \newline
37. र॒भ॒द्ध्व॒ मि॒म मि॒मꣳ र॑भद्ध्वꣳ रभद्ध्व मि॒मम् । \newline
38. इ॒मम् प॒श्चात् प॒श्चा दि॒म मि॒मम् प॒श्चात् । \newline
39. प॒श्चा दन्वनु॑ प॒श्चात् प॒श्चा दनु॑ । \newline
40. अनु॑ जीवाथ जीवा॒थान् वनु॑ जीवाथ । \newline
41. जी॒वा॒थ॒ सर्वे॒ सर्वे॑ जीवाथ जीवाथ॒ सर्वे᳚ । \newline
42. सर्व॒ इति॒ सर्वे᳚ । \newline
43. रा॒ष्ट्र॒भृत॑ ए॒ता ए॒ता रा᳚ष्ट्र॒भृतो॑ राष्ट्र॒भृत॑ ए॒ताः । \newline
44. रा॒ष्ट्र॒भृत॒ इति॑ राष्ट्र - भृतः॑ । \newline
45. ए॒ता उपोपै॒ता ए॒ता उप॑ । \newline
46. उप॑ दधाति दधा॒ त्युपोप॑ दधाति । \newline
47. द॒धा॒ त्ये॒षैषा द॑धाति दधा त्ये॒षा । \newline
48. ए॒षा वै वा ए॒षैषा वै । \newline
49. वा अ॒ग्ने र॒ग्नेर् वै वा अ॒ग्नेः । \newline
50. अ॒ग्ने श्चिति॒ श्चिति॑ र॒ग्ने र॒ग्ने श्चितिः॑ । \newline
51. चिती॑ राष्ट्र॒भृद् रा᳚ष्ट्र॒भृच् चिति॒ श्चिती॑ राष्ट्र॒भृत् । \newline
52. रा॒ष्ट्र॒भृत् तया॒ तया॑ राष्ट्र॒भृद् रा᳚ष्ट्र॒भृत् तया᳚ । \newline
53. रा॒ष्ट्र॒भृदिति॑ राष्ट्र - भृत् । \newline
54. तयै॒ वैव तया॒ तयै॒व । \newline
55. ए॒वास्मि॑न् नस्मिन् ने॒वै वास्मिन्न्॑ । \newline
56. अ॒स्मि॒न् रा॒ष्ट्रꣳ रा॒ष्ट्र म॑स्मिन् नस्मिन् रा॒ष्ट्रम् । \newline
57. रा॒ष्ट्रम् द॑धाति दधाति रा॒ष्ट्रꣳ रा॒ष्ट्रम् द॑धाति । \newline
58. द॒धा॒ति॒ रा॒ष्ट्रꣳ रा॒ष्ट्रम् द॑धाति दधाति रा॒ष्ट्रम् । \newline
59. रा॒ष्ट्र मे॒वैव रा॒ष्ट्रꣳ रा॒ष्ट्र मे॒व । \newline
60. ए॒व भ॑वति भव त्ये॒वैव भ॑वति । \newline
61. भ॒व॒ति॒ न न भ॑वति भवति॒ न । \newline
62. नास्मा॑ दस्मा॒न् न नास्मा᳚त् । \newline
63. अ॒स्मा॒द् रा॒ष्ट्रꣳ रा॒ष्ट्र म॑स्मा दस्माद् रा॒ष्ट्रम् । \newline
64. रा॒ष्ट्रम् भ्रꣳ॑शते भ्रꣳशते रा॒ष्ट्रꣳ रा॒ष्ट्रम् भ्रꣳ॑शते । \newline
65. भ्रꣳ॒॒श॒त॒ इति॑ भ्रꣳशते । \newline

\textbf{Ghana Paata } \newline

1. उ॒त स॒न्दृख् स॒न्दृगु॒तोत स॒न्दृक् प्र॒जाप॑तिः प्र॒जाप॑तिः स॒न्दृगु॒तोत स॒न्दृक् प्र॒जाप॑तिः । \newline
2. स॒न्दृक् प्र॒जाप॑तिः प्र॒जाप॑तिः स॒न्दृख् स॒न्दृक् प्र॒जाप॑तिः परमे॒ष्ठी प॑रमे॒ष्ठी प्र॒जाप॑तिः स॒न्दृख् स॒न्दृक् प्र॒जाप॑तिः परमे॒ष्ठी । \newline
3. स॒न्दृगिति॑ सं - दृक् । \newline
4. प्र॒जाप॑तिः परमे॒ष्ठी प॑रमे॒ष्ठी प्र॒जाप॑तिः प्र॒जाप॑तिः परमे॒ष्ठी वि॒राजा॑ वि॒राजा॑ परमे॒ष्ठी प्र॒जाप॑तिः प्र॒जाप॑तिः परमे॒ष्ठी वि॒राजा᳚ । \newline
5. प्र॒जाप॑ति॒रिति॑ प्र॒जा - प॒तिः॒ । \newline
6. प॒र॒मे॒ष्ठी वि॒राजा॑ वि॒राज॑ परमे॒ष्ठी प॑रमे॒ष्ठी वि॒राजा᳚ । \newline
7. वि॒राजेति॑ वि - राजा᳚ । \newline
8. स्तोमा॒ श्छन्दाꣳ॑सि॒ छन्दाꣳ॑सि॒ स्तोमाः॒ स्तोमा॒ श्छन्दाꣳ॑सि नि॒विदो॑ नि॒विद॒ श्छन्दाꣳ॑सि॒ स्तोमाः॒ स्तोमा॒ श्छन्दाꣳ॑सि नि॒विदः॑ । \newline
9. छन्दाꣳ॑सि नि॒विदो॑ नि॒विद॒ श्छन्दाꣳ॑सि॒ छन्दाꣳ॑सि नि॒विदो॑ मे मे नि॒विद॒ श्छन्दाꣳ॑सि॒ छन्दाꣳ॑सि नि॒विदो॑ मे । \newline
10. नि॒विदो॑ मे मे नि॒विदो॑ नि॒विदो॑ म आहु राहुर् मे नि॒विदो॑ नि॒विदो॑ म आहुः । \newline
11. नि॒विद॒ इति॑ नि - विदः॑ । \newline
12. म॒ आ॒हु॒ रा॒हु॒र् मे॒ म॒ आ॒हु॒ रे॒तस्मा॑ ए॒तस्मा॑ आहुर् मे म आहु रे॒तस्मै᳚ । \newline
13. आ॒हु॒ रे॒तस्मा॑ ए॒तस्मा॑ आहु राहु रे॒तस्मै॑ रा॒ष्ट्रꣳ रा॒ष्ट्र मे॒तस्मा॑ आहु राहु रे॒तस्मै॑ रा॒ष्ट्रम् । \newline
14. ए॒तस्मै॑ रा॒ष्ट्रꣳ रा॒ष्ट्र मे॒तस्मा॑ ए॒तस्मै॑ रा॒ष्ट्र म॒भ्य॑भि रा॒ष्ट्र मे॒तस्मा॑ ए॒तस्मै॑ रा॒ष्ट्र म॒भि । \newline
15. रा॒ष्ट्र म॒भ्य॑भि रा॒ष्ट्रꣳ रा॒ष्ट्र म॒भि सꣳ स म॒भि रा॒ष्ट्रꣳ रा॒ष्ट्र म॒भि सम् । \newline
16. अ॒भि सꣳ स म॒भ्य॑भि सन् न॑माम नमाम॒ स म॒भ्य॑भि सन् न॑माम । \newline
17. सन् न॑माम नमाम॒ सꣳ सन् न॑माम । \newline
18. न॒मा॒मेति॑ नमाम । \newline
19. अ॒भ्याव॑र्तद्ध्व॒ मुपोपा॒ भ्याव॑र्तद्ध्व म॒भ्याव॑र्तद्ध्व॒ मुप॑ मा॒ मोपा॒भ्याव॑र्तद्ध्व म॒भ्याव॑र्तद्ध्व॒ मुप॑ मा । \newline
20. अ॒भ्याव॑र्तद्ध्व॒मित्य॑भि - आव॑र्तद्ध्वम् । \newline
21. उप॑ मा॒ मोपोप॒ मा ऽऽमोपोप॒ मा । \newline
22. मा ऽऽमा॒ मेते॒ता मा॒ मेत॑ । \newline
23. एते॒ तेत॑ सा॒कꣳ सा॒क मि॒तेत॑ सा॒कम् । \newline
24. इ॒त॒ सा॒कꣳ सा॒क मि॑तेत सा॒क म॒य म॒यꣳ सा॒क मि॑तेत सा॒क म॒यम् । \newline
25. सा॒क म॒य म॒यꣳ सा॒कꣳ सा॒क म॒यꣳ शा॒स्ता शा॒स्ता ऽयꣳ सा॒कꣳ सा॒क म॒यꣳ शा॒स्ता । \newline
26. अ॒यꣳ शा॒स्ता शा॒स्ता ऽय म॒यꣳ शा॒स्ता ऽधि॑पति॒ रधि॑पतिः शा॒स्ता ऽय म॒यꣳ शा॒स्ता ऽधि॑पतिः । \newline
27. शा॒स्ता ऽधि॑पति॒ रधि॑पतिः शा॒स्ता शा॒स्ता ऽधि॑पतिर् वो॒ वो ऽधि॑पतिः शा॒स्ता शा॒स्ता ऽधि॑पतिर् वः । \newline
28. अधि॑पतिर् वो॒ वो ऽधि॑पति॒ रधि॑पतिर् वो अस्त्वस्तु॒ वो ऽधि॑पति॒ रधि॑पतिर् वो अस्तु । \newline
29. अधि॑पति॒रित्यधि॑ - प॒तिः॒ । \newline
30. वो॒ अ॒स्त्व॒स्तु॒ वो॒ वो॒ अ॒स्तु॒ । \newline
31. अ॒स्त्वित्य॑स्तु । \newline
32. अ॒स्य वि॒ज्ञानं॑ ॅवि॒ज्ञान॑ म॒स्यास्य वि॒ज्ञान॒ मन्वनु॑ वि॒ज्ञान॑ म॒स्यास्य वि॒ज्ञान॒ मनु॑ । \newline
33. वि॒ज्ञान॒ मन्वनु॑ वि॒ज्ञानं॑ ॅवि॒ज्ञान॒ मनु॒ सꣳ स मनु॑ वि॒ज्ञानं॑ ॅवि॒ज्ञान॒ मनु॒ सम् । \newline
34. वि॒ज्ञान॒मिति॑ वि - ज्ञान᳚म् । \newline
35. अनु॒ सꣳ स मन्वनु॒ सꣳ र॑भद्ध्वꣳ रभद्ध्वꣳ॒॒ स मन्वनु॒ सꣳ र॑भद्ध्वम् । \newline
36. सꣳ र॑भद्ध्वꣳ रभद्ध्वꣳ॒॒ सꣳ सꣳ र॑भद्ध्व मि॒म मि॒मꣳ र॑भद्ध्वꣳ॒॒ सꣳ सꣳ र॑भद्ध्व मि॒मम् । \newline
37. र॒भ॒द्ध्व॒ मि॒म मि॒मꣳ र॑भद्ध्वꣳ रभद्ध्व मि॒मम् प॒श्चात् प॒श्चा दि॒मꣳ र॑भद्ध्वꣳ रभद्ध्व मि॒मम् प॒श्चात् । \newline
38. इ॒मम् प॒श्चात् प॒श्चा दि॒म मि॒मम् प॒श्चा दन्वनु॑ प॒श्चा दि॒म मि॒मम् प॒श्चा दनु॑ । \newline
39. प॒श्चा दन्वनु॑ प॒श्चात् प॒श्चा दनु॑ जीवाथ जीवा॒थानु॑ प॒श्चात् प॒श्चा दनु॑ जीवाथ । \newline
40. अनु॑ जीवाथ जीवा॒थान्वनु॑ जीवाथ॒ सर्वे॒ सर्वे॑ जीवा॒थान्वनु॑ जीवाथ॒ सर्वे᳚ । \newline
41. जी॒वा॒थ॒ सर्वे॒ सर्वे॑ जीवाथ जीवाथ॒ सर्वे᳚ । \newline
42. सर्व॒ इति॒ सर्वे᳚ । \newline
43. रा॒ष्ट्र॒भृत॑ ए॒ता ए॒ता रा᳚ष्ट्र॒भृतो॑ राष्ट्र॒भृत॑ ए॒ता उपोपै॒ता रा᳚ष्ट्र॒भृतो॑ राष्ट्र॒भृत॑ ए॒ता उप॑ । \newline
44. रा॒ष्ट्र॒भृत॒ इति॑ राष्ट्र - भृतः॑ । \newline
45. ए॒ता उपोपै॒ता ए॒ता उप॑ दधाति दधा॒ त्युपै॒ता ए॒ता उप॑ दधाति । \newline
46. उप॑ दधाति दधा॒ त्युपोप॑ दधा त्ये॒षैषा द॑धा॒ त्युपोप॑ दधा त्ये॒षा । \newline
47. द॒धा॒ त्ये॒षैषा द॑धाति दधा त्ये॒षा वै वा ए॒षा द॑धाति दधा त्ये॒षा वै । \newline
48. ए॒षा वै वा ए॒षैषा वा अ॒ग्ने र॒ग्नेर् वा ए॒षैषा वा अ॒ग्नेः । \newline
49. वा अ॒ग्ने र॒ग्नेर् वै वा अ॒ग्ने श्चिति॒ श्चिति॑ र॒ग्नेर् वै वा अ॒ग्ने श्चितिः॑ । \newline
50. अ॒ग्ने श्चिति॒ श्चिति॑ र॒ग्ने र॒ग्ने श्चिती॑ राष्ट्र॒भृद् रा᳚ष्ट्र॒भृच् चिति॑ र॒ग्ने र॒ग्ने श्चिती॑ राष्ट्र॒भृत् । \newline
51. चिती॑ राष्ट्र॒भृद् रा᳚ष्ट्र॒भृच् चिति॒ श्चिती॑ राष्ट्र॒भृत् तया॒ तया॑ राष्ट्र॒भृच् चिति॒ श्चिती॑ राष्ट्र॒भृत् तया᳚ । \newline
52. रा॒ष्ट्र॒भृत् तया॒ तया॑ राष्ट्र॒भृद् रा᳚ष्ट्र॒भृत् तयै॒वैव तया॑ राष्ट्र॒भृद् रा᳚ष्ट्र॒भृत् तयै॒व । \newline
53. रा॒ष्ट्र॒भृदिति॑ राष्ट्र - भृत् । \newline
54. तयै॒ वैव तया॒ तयै॒ वास्मि॑न् नस्मिन् ने॒व तया॒ तयै॒ वास्मिन्न्॑ । \newline
55. ए॒वास्मि॑न् नस्मिन् ने॒वै वास्मि॑न् रा॒ष्ट्रꣳ रा॒ष्ट्र म॑स्मिन् ने॒वै वास्मि॑न् रा॒ष्ट्रम् । \newline
56. अ॒स्मि॒न् रा॒ष्ट्रꣳ रा॒ष्ट्र म॑स्मिन् नस्मिन् रा॒ष्ट्रम् द॑धाति दधाति रा॒ष्ट्र म॑स्मिन् नस्मिन् रा॒ष्ट्रम् द॑धाति । \newline
57. रा॒ष्ट्रम् द॑धाति दधाति रा॒ष्ट्रꣳ रा॒ष्ट्रम् द॑धाति रा॒ष्ट्रꣳ रा॒ष्ट्रम् द॑धाति रा॒ष्ट्रꣳ रा॒ष्ट्रम् द॑धाति रा॒ष्ट्रम् । \newline
58. द॒धा॒ति॒ रा॒ष्ट्रꣳ रा॒ष्ट्रम् द॑धाति दधाति रा॒ष्ट्र मे॒वैव रा॒ष्ट्रम् द॑धाति दधाति रा॒ष्ट्र मे॒व । \newline
59. रा॒ष्ट्र मे॒वैव रा॒ष्ट्रꣳ रा॒ष्ट्र मे॒व भ॑वति भव त्ये॒व रा॒ष्ट्रꣳ रा॒ष्ट्र मे॒व भ॑वति । \newline
60. ए॒व भ॑वति भव त्ये॒वैव भ॑वति॒ न न भ॑व त्ये॒वैव भ॑वति॒ न । \newline
61. भ॒व॒ति॒ न न भ॑वति भवति॒ नास्मा॑ दस्मा॒न् न भ॑वति भवति॒ नास्मा᳚त् । \newline
62. नास्मा॑ दस्मा॒न् न नास्मा᳚द् रा॒ष्ट्रꣳ रा॒ष्ट्र म॑स्मा॒न् न नास्मा᳚द् रा॒ष्ट्रम् । \newline
63. अ॒स्मा॒द् रा॒ष्ट्रꣳ रा॒ष्ट्र म॑स्मा दस्माद् रा॒ष्ट्रम् भ्रꣳ॑शते भ्रꣳशते रा॒ष्ट्र म॑स्मा दस्माद् रा॒ष्ट्रम् भ्रꣳ॑शते । \newline
64. रा॒ष्ट्रम् भ्रꣳ॑शते भ्रꣳशते रा॒ष्ट्रꣳ रा॒ष्ट्रम् भ्रꣳ॑शते । \newline
65. भ्रꣳ॒॒श॒त॒ इति॑ भ्रꣳशते । \newline
\pagebreak
\markright{ TS 5.7.5.1  \hfill https://www.vedavms.in \hfill}

\section{ TS 5.7.5.1 }

\textbf{TS 5.7.5.1 } \newline
\textbf{Samhita Paata} \newline

यथा॒ वै पु॒त्रो जा॒तो म्रि॒यत॑ ए॒वं ॅवा ए॒ष म्रि॑यते॒ यस्या॒ग्निरुख्य॑ उ॒द्वाय॑ति॒ यन्नि॑र्म॒न्थ्यं॑ कु॒र्याद्-विच्छि॑न्द्या॒द्-भ्रातृ॑व्यमस्मै जनये॒थ् स ए॒व पुनः॑ प॒रीद्ध्यः॒ स्वादे॒वैनं॒ ॅयोने᳚र्जनयति॒ नास्मै॒ भ्रातृ॑व्यं जनयति॒ तमो॒ वा ए॒तं गृ॑ह्णाति॒ यस्या॒ग्निरुख्य॑ उ॒द्वाय॑ति मृ॒त्युस्तमः॑ कृ॒ष्णं ॅवासः॑ कृ॒ष्णा धे॒नुर्दक्षि॑णा॒ तम॑सै॒ - [  ] \newline

\textbf{Pada Paata} \newline

यथा᳚ । वै । पु॒त्रः । जा॒तः । म्रि॒यते᳚ । ए॒वम् । वै । ए॒षः । म्रि॒य॒ते॒ । यस्य॑ । अ॒ग्निः । उख्यः॑ । उ॒द्वाय॒तीयु॑त् - वाय॑ति । यत् । नि॒र्म॒न्थ्य॑मिति॑ निः - म॒न्थ्य᳚म् । कु॒र्यात् । वीति॑ । छि॒न्द्या॒त् । भ्रातृ॑व्यम् । अ॒स्मै॒ । ज॒न॒ये॒त् । सः । ए॒व । पुनः॑ । प॒रीद्ध्य॒ इति॑ परि-इद्ध्यः॑ । स्वात् । ए॒व । ए॒न॒म् । योनेः᳚ । ज॒न॒य॒ति॒ । न । अ॒स्मै॒ । भ्रातृ॑व्यम् । ज॒न॒य॒ति॒ । तमः॑ । वै । ए॒तम् । गृ॒ह्णा॒ति॒ । यस्य॑ । अ॒ग्निः । उख्यः॑ । उ॒द्वाय॒तीत्यु॑त् - वाय॑ति । मृ॒त्युः । तमः॑ । कृ॒ष्णम् । वासः॑ । कृ॒ष्णा । धे॒नुः । दक्षि॑णा । तम॑सा ।  \newline


\textbf{Krama Paata} \newline

यथा॒ वै । वै पु॒त्रः । पु॒त्रो जा॒तः । जा॒तो म्रि॒यते᳚ । म्रि॒यत॑ ए॒वम् । ए॒वम् ॅवै । वा ए॒षः । ए॒ष म्रि॑यते । म्रि॒य॒ते॒ यस्य॑ । यस्या॒ग्निः । अ॒ग्निरुख्यः॑ । उख्य॑ उ॒द्वाय॑ति । उ॒द्वाय॑ति॒ यत् । उ॒द्वाय॒तीत्यु॑त् - वाय॑ति । यन् नि॑र्म॒न्थ्य᳚म् । नि॒र्म॒न्थ्य॑म् कु॒र्यात् । नि॒र्म॒न्थ्य॑मिति॑ निः - म॒न्थ्य᳚म् । कु॒र्याद् वि । विच्छि॑न्द्यात् । छि॒न्द्या॒द् भ्रातृ॑व्यम् । भ्रातृ॑व्यमस्मै । अ॒स्मै॒ ज॒न॒ये॒त्॒ । ज॒न॒ये॒थ् सः । स ए॒व । ए॒व पुनः॑ । पुनः॑ प॒रीद्ध्यः॑ । प॒रीद्ध्यः॒ स्वात् । प॒रीद्ध्य॒ इति॑ परि - इद्ध्यः॑ । स्वादे॒व । ए॒वैन᳚म् । ए॒न॒म् ॅयोनेः᳚ । योने᳚र् जनयति । ज॒न॒य॒ति॒ न । नास्मै᳚ । अ॒स्मै॒ भ्रातृ॑व्यम् । भ्रातृ॑व्यम् जनयति । ज॒न॒य॒ति॒ तमः॑ । तमो॒ वै । वा ए॒तम् । ए॒तम् गृ॑ह्णाति । गृ॒ह्णा॒ति॒ यस्य॑ । यस्या॒ग्निः । अ॒ग्निरुख्यः॑ । उख्य॑ उ॒द्वाय॑ति । उ॒द्वाय॑ति मृ॒त्युः । उ॒द्वाय॒तीत्यु॑त् - वाय॑ति । मृ॒त्युस्तमः॑ । तमः॑ कृ॒ष्णम् । कृ॒ष्णम् ॅवासः॑ । वासः॑ कृ॒ष्णा । कृ॒ष्णा धे॒नुः । धे॒नुर् दक्षि॑णा । दक्षि॑णा॒ तम॑सा । तम॑सै॒व \newline

\textbf{Jatai Paata} \newline

1. यथा॒ वै वै यथा॒ यथा॒ वै । \newline
2. वै पु॒त्रः पु॒त्रो वै वै पु॒त्रः । \newline
3. पु॒त्रो जा॒तो जा॒तः पु॒त्रः पु॒त्रो जा॒तः । \newline
4. जा॒तो म्रि॒यते᳚ म्रि॒यते॑ जा॒तो जा॒तो म्रि॒यते᳚ । \newline
5. म्रि॒यत॑ ए॒व मे॒वम् म्रि॒यते᳚ म्रि॒यत॑ ए॒वम् । \newline
6. ए॒वं ॅवै वा ए॒व मे॒वं ॅवै । \newline
7. वा ए॒ष ए॒ष वै वा ए॒षः । \newline
8. ए॒ष म्रि॑यते म्रियत ए॒ष ए॒ष म्रि॑यते । \newline
9. म्रि॒य॒ते॒ यस्य॒ यस्य॑ म्रियते म्रियते॒ यस्य॑ । \newline
10. यस्या॒ग्नि र॒ग्निर् यस्य॒ यस्या॒ग्निः । \newline
11. अ॒ग्नि रुख्य॒ उख्यो॒ ऽग्नि र॒ग्नि रुख्यः॑ । \newline
12. उख्य॑ उ॒द्वाय॑ त्यु॒द्वाय॒ त्युख्य॒ उख्य॑ उ॒द्वाय॑ति । \newline
13. उ॒द्वाय॑ति॒ यद् यदु॒द्वाय॑ त्यु॒द्वाय॑ति॒ यत् । \newline
14. उ॒द्वाय॒तीयु॑त् - वाय॑ति । \newline
15. यन् नि॑र्म॒न्थ्य॑म् निर्म॒न्थ्यं॑ ॅयद् यन् नि॑र्म॒न्थ्य᳚म् । \newline
16. नि॒र्म॒न्थ्य॑म् कु॒र्यात् कु॒र्यान् नि॑र्म॒न्थ्य॑म् निर्म॒न्थ्य॑म् कु॒र्यात् । \newline
17. नि॒र्म॒न्थ्य॑मिति॑ निः - म॒न्थ्य᳚म् । \newline
18. कु॒र्याद् वि वि कु॒र्यात् कु॒र्याद् वि । \newline
19. वि च्छि॑न्द्याच् छिन्द्या॒द् वि वि च्छि॑न्द्यात् । \newline
20. छि॒न्द्या॒द् भ्रातृ॑व्य॒म् भ्रातृ॑व्यम् छिन्द्याच् छिन्द्या॒द् भ्रातृ॑व्यम् । \newline
21. भ्रातृ॑व्य मस्मा अस्मै॒ भ्रातृ॑व्य॒म् भ्रातृ॑व्य मस्मै । \newline
22. अ॒स्मै॒ ज॒न॒ये॒ज् ज॒न॒ये॒ द॒स्मा॒ अ॒स्मै॒ ज॒न॒ये॒त् । \newline
23. ज॒न॒ये॒थ् स स ज॑नयेज् जनये॒थ् सः । \newline
24. स ए॒वैव स स ए॒व । \newline
25. ए॒व पुनः॒ पुन॑ रे॒वैव पुनः॑ । \newline
26. पुनः॑ प॒रीद्ध्यः॑ प॒रीद्ध्यः॒ पुनः॒ पुनः॑ प॒रीद्ध्यः॑ । \newline
27. प॒रीद्ध्यः॒ स्वाथ् स्वात् प॒रीद्ध्यः॑ प॒रीद्ध्यः॒ स्वात् । \newline
28. प॒रीद्ध्य॒ इति॑ परि - इद्ध्यः॑ । \newline
29. स्वादे॒ वैव स्वाथ् स्वादे॒व । \newline
30. ए॒वैन॑ मेन मे॒वै वैन᳚म् । \newline
31. ए॒नं॒ ॅयोने॒र् योने॑ रेन मेनं॒ ॅयोनेः᳚ । \newline
32. योने᳚र् जनयति जनयति॒ योने॒र् योने᳚र् जनयति । \newline
33. ज॒न॒य॒ति॒ न न ज॑नयति जनयति॒ न । \newline
34. नास्मा॑ अस्मै॒ न नास्मै᳚ । \newline
35. अ॒स्मै॒ भ्रातृ॑व्य॒म् भ्रातृ॑व्य मस्मा अस्मै॒ भ्रातृ॑व्यम् । \newline
36. भ्रातृ॑व्यम् जनयति जनयति॒ भ्रातृ॑व्य॒म् भ्रातृ॑व्यम् जनयति । \newline
37. ज॒न॒य॒ति॒ तम॒ स्तमो॑ जनयति जनयति॒ तमः॑ । \newline
38. तमो॒ वै वै तम॒ स्तमो॒ वै । \newline
39. वा ए॒त मे॒तं ॅवै वा ए॒तम् । \newline
40. ए॒तम् गृ॑ह्णाति गृह्णा त्ये॒त मे॒तम् गृ॑ह्णाति । \newline
41. गृ॒ह्णा॒ति॒ यस्य॒ यस्य॑ गृह्णाति गृह्णाति॒ यस्य॑ । \newline
42. यस्या॒ग्नि र॒ग्निर् यस्य॒ यस्या॒ग्निः । \newline
43. अ॒ग्नि रुख्य॒ उख्यो॒ ऽग्नि र॒ग्नि रुख्यः॑ । \newline
44. उख्य॑ उ॒द्वाय॑ त्यु॒द्वाय॒ त्युख्य॒ उख्य॑ उ॒द्वाय॑ति । \newline
45. उ॒द्वाय॑ति मृ॒त्युर् मृ॒त्यु रु॒द्वाय॑ त्यु॒द्वाय॑ति मृ॒त्युः । \newline
46. उ॒द्वाय॒तीत्यु॑त् - वाय॑ति । \newline
47. मृ॒त्यु स्तम॒ स्तमो॑ मृ॒त्युर् मृ॒त्यु स्तमः॑ । \newline
48. तमः॑ कृ॒ष्णम् कृ॒ष्णम् तम॒ स्तमः॑ कृ॒ष्णम् । \newline
49. कृ॒ष्णं ॅवासो॒ वासः॑ कृ॒ष्णम् कृ॒ष्णं ॅवासः॑ । \newline
50. वासः॑ कृ॒ष्णा कृ॒ष्णा वासो॒ वासः॑ कृ॒ष्णा । \newline
51. कृ॒ष्णा धे॒नुर् धे॒नुः कृ॒ष्णा कृ॒ष्णा धे॒नुः । \newline
52. धे॒नुर् दक्षि॑णा॒ दक्षि॑णा धे॒नुर् धे॒नुर् दक्षि॑णा । \newline
53. दक्षि॑णा॒ तम॑सा॒ तम॑सा॒ दक्षि॑णा॒ दक्षि॑णा॒ तम॑सा । \newline
54. तम॑सै॒ वैव तम॑सा॒ तम॑सै॒व । \newline

\textbf{Ghana Paata } \newline

1. यथा॒ वै वै यथा॒ यथा॒ वै पु॒त्रः पु॒त्रो वै यथा॒ यथा॒ वै पु॒त्रः । \newline
2. वै पु॒त्रः पु॒त्रो वै वै पु॒त्रो जा॒तो जा॒तः पु॒त्रो वै वै पु॒त्रो जा॒तः । \newline
3. पु॒त्रो जा॒तो जा॒तः पु॒त्रः पु॒त्रो जा॒तो म्रि॒यते᳚ म्रि॒यते॑ जा॒तः पु॒त्रः पु॒त्रो जा॒तो म्रि॒यते᳚ । \newline
4. जा॒तो म्रि॒यते᳚ म्रि॒यते॑ जा॒तो जा॒तो म्रि॒यत॑ ए॒व मे॒वम् म्रि॒यते॑ जा॒तो जा॒तो म्रि॒यत॑ ए॒वम् । \newline
5. म्रि॒यत॑ ए॒व मे॒वम् म्रि॒यते᳚ म्रि॒यत॑ ए॒वं ॅवै वा ए॒वम् म्रि॒यते᳚ म्रि॒यत॑ ए॒वं ॅवै । \newline
6. ए॒वं ॅवै वा ए॒व मे॒वं ॅवा ए॒ष ए॒ष वा ए॒व मे॒वं ॅवा ए॒षः । \newline
7. वा ए॒ष ए॒ष वै वा ए॒ष म्रि॑यते म्रियत ए॒ष वै वा ए॒ष म्रि॑यते । \newline
8. ए॒ष म्रि॑यते म्रियत ए॒ष ए॒ष म्रि॑यते॒ यस्य॒ यस्य॑ म्रियत ए॒ष ए॒ष म्रि॑यते॒ यस्य॑ । \newline
9. म्रि॒य॒ते॒ यस्य॒ यस्य॑ म्रियते म्रियते॒ यस्या॒ग्नि र॒ग्निर् यस्य॑ म्रियते म्रियते॒ यस्या॒ग्निः । \newline
10. यस्या॒ग्नि र॒ग्निर् यस्य॒ यस्या॒ग्नि रुख्य॒ उख्यो॒ ऽग्निर् यस्य॒ यस्या॒ग्नि रुख्यः॑ । \newline
11. अ॒ग्नि रुख्य॒ उख्यो॒ ऽग्नि र॒ग्नि रुख्य॑ उ॒द्वाय॑ त्यु॒द्वाय॒ त्युख्यो॒ ऽग्नि र॒ग्नि रुख्य॑ उ॒द्वाय॑ति । \newline
12. उख्य॑ उ॒द्वाय॑ त्यु॒द्वाय॒ त्युख्य॒ उख्य॑ उ॒द्वाय॑ति॒ यद् यदु॒द्वाय॒ त्युख्य॒ उख्य॑ उ॒द्वाय॑ति॒ यत् । \newline
13. उ॒द्वाय॑ति॒ यद् यदु॒द्वाय॑ त्यु॒द्वाय॑ति॒ यन् नि॑र्म॒न्थ्य॑म् निर्म॒न्थ्यं॑ ॅयदु॒द्वाय॑ त्यु॒द्वाय॑ति॒ यन् नि॑र्म॒न्थ्य᳚म् । \newline
14. उ॒द्वाय॒तीयु॑त् - वाय॑ति । \newline
15. यन् नि॑र्म॒न्थ्य॑म् निर्म॒न्थ्यं॑ ॅयद् यन् नि॑र्म॒न्थ्य॑म् कु॒र्यात् कु॒र्यान् नि॑र्म॒न्थ्यं॑ ॅयद् यन् नि॑र्म॒न्थ्य॑म् कु॒र्यात् । \newline
16. नि॒र्म॒न्थ्य॑म् कु॒र्यात् कु॒र्यान् नि॑र्म॒न्थ्य॑म् निर्म॒न्थ्य॑म् कु॒र्याद् वि वि कु॒र्यान् नि॑र्म॒न्थ्य॑म् निर्म॒न्थ्य॑म् कु॒र्याद् वि । \newline
17. नि॒र्म॒न्थ्य॑मिति॑ निः - म॒न्थ्य᳚म् । \newline
18. कु॒र्याद् वि वि कु॒र्यात् कु॒र्याद् वि च्छि॑न्द्याच् छिन्द्या॒द् वि कु॒र्यात् कु॒र्याद् वि च्छि॑न्द्यात् । \newline
19. वि च्छि॑न्द्याच् छिन्द्या॒द् वि वि च्छि॑न्द्या॒द् भ्रातृ॑व्य॒म् भ्रातृ॑व्यम् छिन्द्या॒द् वि वि च्छि॑न्द्या॒द् भ्रातृ॑व्यम् । \newline
20. छि॒न्द्या॒द् भ्रातृ॑व्य॒म् भ्रातृ॑व्यम् छिन्द्याच् छिन्द्या॒द् भ्रातृ॑व्य मस्मा अस्मै॒ भ्रातृ॑व्यम् छिन्द्याच् छिन्द्या॒द् भ्रातृ॑व्य मस्मै । \newline
21. भ्रातृ॑व्य मस्मा अस्मै॒ भ्रातृ॑व्य॒म् भ्रातृ॑व्य मस्मै जनयेज् जनये दस्मै॒ भ्रातृ॑व्य॒म् भ्रातृ॑व्य मस्मै जनयेत् । \newline
22. अ॒स्मै॒ ज॒न॒ये॒ज् ज॒न॒ये॒ द॒स्मा॒ अ॒स्मै॒ ज॒न॒ये॒थ् स स ज॑नये दस्मा अस्मै जनये॒थ् सः । \newline
23. ज॒न॒ये॒थ् स स ज॑नयेज् जनये॒थ् स ए॒वैव स ज॑नयेज् जनये॒थ् स ए॒व । \newline
24. स ए॒वैव स स ए॒व पुनः॒ पुन॑ रे॒व स स ए॒व पुनः॑ । \newline
25. ए॒व पुनः॒ पुन॑ रे॒वैव पुनः॑ प॒रीद्ध्यः॑ प॒रीद्ध्यः॒ पुन॑ रे॒वैव पुनः॑ प॒रीद्ध्यः॑ । \newline
26. पुनः॑ प॒रीद्ध्यः॑ प॒रीद्ध्यः॒ पुनः॒ पुनः॑ प॒रीद्ध्यः॒ स्वाथ् स्वात् प॒रीद्ध्यः॒ पुनः॒ पुनः॑ प॒रीद्ध्यः॒ स्वात् । \newline
27. प॒रीद्ध्यः॒ स्वाथ् स्वात् प॒रीद्ध्यः॑ प॒रीद्ध्यः॒ स्वादे॒वैव स्वात् प॒रीद्ध्यः॑ प॒रीद्ध्यः॒ स्वादे॒व । \newline
28. प॒रीद्ध्य॒ इति॑ परि - इद्ध्यः॑ । \newline
29. स्वा दे॒वैव स्वाथ् स्वा दे॒वैन॑ मेन मे॒व स्वाथ् स्वा दे॒वैन᳚म् । \newline
30. ए॒वैन॑ मेन मे॒वै वैनं॒ ॅयोने॒र् योने॑ रेन मे॒वै वैनं॒ ॅयोनेः᳚ । \newline
31. ए॒नं॒ ॅयोने॒र् योने॑ रेन मेनं॒ ॅयोने᳚र् जनयति जनयति॒ योने॑ रेन मेनं॒ ॅयोने᳚र् जनयति । \newline
32. योने᳚र् जनयति जनयति॒ योने॒र् योने᳚र् जनयति॒ न न ज॑नयति॒ योने॒र् योने᳚र् जनयति॒ न । \newline
33. ज॒न॒य॒ति॒ न न ज॑नयति जनयति॒ नास्मा॑ अस्मै॒ न ज॑नयति जनयति॒ नास्मै᳚ । \newline
34. नास्मा॑ अस्मै॒ न नास्मै॒ भ्रातृ॑व्य॒म् भ्रातृ॑व्य मस्मै॒ न नास्मै॒ भ्रातृ॑व्यम् । \newline
35. अ॒स्मै॒ भ्रातृ॑व्य॒म् भ्रातृ॑व्य मस्मा अस्मै॒ भ्रातृ॑व्यम् जनयति जनयति॒ भ्रातृ॑व्य मस्मा अस्मै॒ भ्रातृ॑व्यम् जनयति । \newline
36. भ्रातृ॑व्यम् जनयति जनयति॒ भ्रातृ॑व्य॒म् भ्रातृ॑व्यम् जनयति॒ तम॒ स्तमो॑ जनयति॒ भ्रातृ॑व्य॒म् भ्रातृ॑व्यम् जनयति॒ तमः॑ । \newline
37. ज॒न॒य॒ति॒ तम॒ स्तमो॑ जनयति जनयति॒ तमो॒ वै वै तमो॑ जनयति जनयति॒ तमो॒ वै । \newline
38. तमो॒ वै वै तम॒ स्तमो॒ वा ए॒त मे॒तं ॅवै तम॒ स्तमो॒ वा ए॒तम् । \newline
39. वा ए॒त मे॒तं ॅवै वा ए॒तम् गृ॑ह्णाति गृह्णा त्ये॒तं ॅवै वा ए॒तम् गृ॑ह्णाति । \newline
40. ए॒तम् गृ॑ह्णाति गृह्णा त्ये॒त मे॒तम् गृ॑ह्णाति॒ यस्य॒ यस्य॑ गृह्णा त्ये॒त मे॒तम् गृ॑ह्णाति॒ यस्य॑ । \newline
41. गृ॒ह्णा॒ति॒ यस्य॒ यस्य॑ गृह्णाति गृह्णाति॒ यस्या॒ग्नि र॒ग्निर् यस्य॑ गृह्णाति गृह्णाति॒ यस्या॒ग्निः । \newline
42. यस्या॒ग्नि र॒ग्निर् यस्य॒ यस्या॒ग्नि रुख्य॒ उख्यो॒ ऽग्निर् यस्य॒ यस्या॒ग्नि रुख्यः॑ । \newline
43. अ॒ग्नि रुख्य॒ उख्यो॒ ऽग्नि र॒ग्नि रुख्य॑ उ॒द्वाय॑ त्यु॒द्वाय॒ त्युख्यो॒ ऽग्नि र॒ग्नि रुख्य॑ उ॒द्वाय॑ति । \newline
44. उख्य॑ उ॒द्वाय॑ त्यु॒द्वाय॒ त्युख्य॒ उख्य॑ उ॒द्वाय॑ति मृ॒त्युर् मृ॒त्यु रु॒द्वाय॒ त्युख्य॒ उख्य॑ उ॒द्वाय॑ति मृ॒त्युः । \newline
45. उ॒द्वाय॑ति मृ॒त्युर् मृ॒त्यु रु॒द्वाय॑ त्यु॒द्वाय॑ति मृ॒त्यु स्तम॒ स्तमो॑ मृ॒त्यु रु॒द्वाय॑ त्यु॒द्वाय॑ति मृ॒त्यु स्तमः॑ । \newline
46. उ॒द्वाय॒तीत्यु॑त् - वाय॑ति । \newline
47. मृ॒त्यु स्तम॒ स्तमो॑ मृ॒त्युर् मृ॒त्यु स्तमः॑ कृ॒ष्णम् कृ॒ष्णम् तमो॑ मृ॒त्युर् मृ॒त्यु स्तमः॑ कृ॒ष्णम् । \newline
48. तमः॑ कृ॒ष्णम् कृ॒ष्णम् तम॒ स्तमः॑ कृ॒ष्णं ॅवासो॒ वासः॑ कृ॒ष्णम् तम॒ स्तमः॑ कृ॒ष्णं ॅवासः॑ । \newline
49. कृ॒ष्णं ॅवासो॒ वासः॑ कृ॒ष्णम् कृ॒ष्णं ॅवासः॑ कृ॒ष्णा कृ॒ष्णा वासः॑ कृ॒ष्णम् कृ॒ष्णं ॅवासः॑ कृ॒ष्णा । \newline
50. वासः॑ कृ॒ष्णा कृ॒ष्णा वासो॒ वासः॑ कृ॒ष्णा धे॒नुर् धे॒नुः कृ॒ष्णा वासो॒ वासः॑ कृ॒ष्णा धे॒नुः । \newline
51. कृ॒ष्णा धे॒नुर् धे॒नुः कृ॒ष्णा कृ॒ष्णा धे॒नुर् दक्षि॑णा॒ दक्षि॑णा धे॒नुः कृ॒ष्णा कृ॒ष्णा धे॒नुर् दक्षि॑णा । \newline
52. धे॒नुर् दक्षि॑णा॒ दक्षि॑णा धे॒नुर् धे॒नुर् दक्षि॑णा॒ तम॑सा॒ तम॑सा॒ दक्षि॑णा धे॒नुर् धे॒नुर् दक्षि॑णा॒ तम॑सा । \newline
53. दक्षि॑णा॒ तम॑सा॒ तम॑सा॒ दक्षि॑णा॒ दक्षि॑णा॒ तम॑सै॒वैव तम॑सा॒ दक्षि॑णा॒ दक्षि॑णा॒ तम॑सै॒व । \newline
54. तम॑सै॒वैव तम॑सा॒ तम॑सै॒व तम॒ स्तम॑ ए॒व तम॑सा॒ तम॑सै॒व तमः॑ । \newline
\pagebreak
\markright{ TS 5.7.5.2  \hfill https://www.vedavms.in \hfill}

\section{ TS 5.7.5.2 }

\textbf{TS 5.7.5.2 } \newline
\textbf{Samhita Paata} \newline

-व तमो॑ मृ॒त्युमप॑ हते॒ हिर॑ण्यं ददाति॒ ज्योति॒र्वै हिर॑ण्यं॒ ज्योति॑षै॒व तमोऽप॑ ह॒तेऽथो॒ तेजो॒ वै हिर॑ण्यं॒ तेज॑ ए॒वाऽऽ*त्मन् ध॑त्ते॒ सुव॒र्न घ॒र्मः स्वाहा॒ सुव॒र्नाऽर्कः स्वाहा॒ सुव॒र्न शु॒क्रः स्वाहा॒ सुव॒र्न ज्योतिः॒ स्वाहा॒ सुव॒र्न सूर्यः॒ स्वाहा॒ ऽर्को वा ए॒ष यद॒ग्निर॒सावा॑दि॒त्यो᳚ - [  ] \newline

\textbf{Pada Paata} \newline

ए॒व । तमः॑ । मृ॒त्युम् । अपेति॑ । ह॒ते॒ । हिर॑ण्यम् । द॒दा॒ति॒ । ज्योतिः॑ । वै । हिर॑ण्यम् । ज्योति॑षा । ए॒व । तमः॑ । अपेति॑ । ह॒ते॒ । अथो॒ इति॑ । तेजः॑ । वै । हिर॑ण्यम् । तेजः॑ । ए॒व । आ॒त्मन्न् । ध॒त्ते॒ । सुवः॑ । न । घ॒र्मः । स्वाहा᳚ । सुवः॑ । न । अ॒र्कः । स्वाहा᳚ । सुवः॑ । न । शु॒क्रः । स्वाहा᳚ । सुवः॑ । न । ज्योतिः॑ । स्वाहा᳚ । सुवः॑ । न । सूर्यः॑ । स्वाहा᳚ । अ॒र्कः । वै । ए॒षः । यत् । अ॒ग्निः । अ॒सौ । आ॒दि॒त्यः ।  \newline


\textbf{Krama Paata} \newline

ए॒व तमः॑ । तमो॑ मृ॒त्युम् । मृ॒त्युमप॑ । अप॑ हते । ह॒ते॒ हिर॑ण्यम् । हिर॑ण्यम् ददाति । द॒दा॒ति॒ ज्योतिः॑ । ज्योति॒र् वै । वै हिर॑ण्यम् । हिर॑ण्य॒म् ज्योति॑षा । ज्योति॑षै॒व । ए॒व तमः॑ । तमोऽप॑ । अप॑ हते । ह॒तेऽथो᳚ । अथो॒ तेजः॑ । अथो॒ इत्यथो᳚ । तेजो॒ वै । वै हिर॑ण्यम् । हिर॑ण्य॒म् तेजः॑ । तेज॑ ए॒व । ए॒वात्मन्न् । आ॒त्मन् ध॑त्ते । ध॒त्ते॒ सुवः॑ । सुव॒र् न । न घ॒र्मः । घ॒र्मः स्वाहा᳚ । स्वाहा॒ सुवः॑ । सुव॒र् न । नार्कः । अ॒र्कः स्वाहा᳚ । स्वाहा॒ सुवः॑ । सुव॒र्न । न शु॒क्रः । शु॒क्रः स्वाहा᳚ । स्वाहा॒ सुवः॑ । सुव॒र् न । न ज्योतिः॑ । ज्योतिः॒ स्वाहा᳚ । स्वाहा॒ सुवः॑ । सुव॒र् न । न सूर्यः॑ । सूर्यः॒ स्वाहा᳚ । स्वाहा॒ऽर्कः । अ॒र्को वै । वा ए॒षः । ए॒ष यत् । यद॒ग्निः । अ॒ग्निर॒सौ । अ॒सावा॑दि॒त्यः । आ॒दि॒त्यो᳚ऽश्वमे॒धः \newline

\textbf{Jatai Paata} \newline

1. ए॒व तम॒ स्तम॑ ए॒वैव तमः॑ । \newline
2. तमो॑ मृ॒त्युम् मृ॒त्युम् तम॒ स्तमो॑ मृ॒त्युम् । \newline
3. मृ॒त्यु मपाप॑ मृ॒त्युम् मृ॒त्यु मप॑ । \newline
4. अप॑ हते ह॒ते ऽपाप॑ हते । \newline
5. ह॒ते॒ हिर॑ण्यꣳ॒॒ हिर॑ण्यꣳ हते हते॒ हिर॑ण्यम् । \newline
6. हिर॑ण्यम् ददाति ददाति॒ हिर॑ण्यꣳ॒॒ हिर॑ण्यम् ददाति । \newline
7. द॒दा॒ति॒ ज्योति॒र् ज्योति॑र् ददाति ददाति॒ ज्योतिः॑ । \newline
8. ज्योति॒र् वै वै ज्योति॒र् ज्योति॒र् वै । \newline
9. वै हिर॑ण्यꣳ॒॒ हिर॑ण्यं॒ ॅवै वै हिर॑ण्यम् । \newline
10. हिर॑ण्य॒म् ज्योति॑षा॒ ज्योति॑षा॒ हिर॑ण्यꣳ॒॒ हिर॑ण्य॒म् ज्योति॑षा । \newline
11. ज्योति॑ षै॒वैव ज्योति॑षा॒ ज्योति॑ षै॒व । \newline
12. ए॒व तम॒ स्तम॑ ए॒वैव तमः॑ । \newline
13. तमो ऽपाप॒ तम॒ स्तमो ऽप॑ । \newline
14. अप॑ हते ह॒ते ऽपाप॑ हते । \newline
15. ह॒ते ऽथो॒ अथो॑ हते ह॒ते ऽथो᳚ । \newline
16. अथो॒ तेज॒ स्तेजो ऽथो॒ अथो॒ तेजः॑ । \newline
17. अथो॒ इत्यथो᳚ । \newline
18. तेजो॒ वै वै तेज॒ स्तेजो॒ वै । \newline
19. वै हिर॑ण्यꣳ॒॒ हिर॑ण्यं॒ ॅवै वै हिर॑ण्यम् । \newline
20. हिर॑ण्य॒म् तेज॒ स्तेजो॒ हिर॑ण्यꣳ॒॒ हिर॑ण्य॒म् तेजः॑ । \newline
21. तेज॑ ए॒वैव तेज॒ स्तेज॑ ए॒व । \newline
22. ए॒वात्मन् ना॒त्मन् ने॒वै वात्मन्न् । \newline
23. आ॒त्मन् ध॑त्ते धत्त आ॒त्मन् ना॒त्मन् ध॑त्ते । \newline
24. ध॒त्ते॒ सुवः॒ सुव॑र् धत्ते धत्ते॒ सुवः॑ । \newline
25. सुव॒र् न न सुवः॒ सुव॒र् न । \newline
26. न घ॒र्मो घ॒र्मो न न घ॒र्मः । \newline
27. घ॒र्मः स्वाहा॒ स्वाहा॑ घ॒र्मो घ॒र्मः स्वाहा᳚ । \newline
28. स्वाहा॒ सुवः॒ सुवः॒ स्वाहा॒ स्वाहा॒ सुवः॑ । \newline
29. सुव॒र् न न सुवः॒ सुव॒र् न । \newline
30. नार्को᳚ ऽर्को न नार्कः । \newline
31. अ॒र्कः स्वाहा॒ स्वाहा॒ ऽर्को᳚ ऽर्कः स्वाहा᳚ । \newline
32. स्वाहा॒ सुवः॒ सुवः॒ स्वाहा॒ स्वाहा॒ सुवः॑ । \newline
33. सुव॒र् न न सुवः॒ सुव॒र् न । \newline
34. न शु॒क्रः शु॒क्रो न न शु॒क्रः । \newline
35. शु॒क्रः स्वाहा॒ स्वाहा॑ शु॒क्रः शु॒क्रः स्वाहा᳚ । \newline
36. स्वाहा॒ सुवः॒ सुवः॒ स्वाहा॒ स्वाहा॒ सुवः॑ । \newline
37. सुव॒र् न न सुवः॒ सुव॒र् न । \newline
38. न ज्योति॒र् ज्योति॒र् न न ज्योतिः॑ । \newline
39. ज्योतिः॒ स्वाहा॒ स्वाहा॒ ज्योति॒र् ज्योतिः॒ स्वाहा᳚ । \newline
40. स्वाहा॒ सुवः॒ सुवः॒ स्वाहा॒ स्वाहा॒ सुवः॑ । \newline
41. सुव॒र् न न सुवः॒ सुव॒र् न । \newline
42. न सूर्यः॒ सूर्यो॒ न न सूर्यः॑ । \newline
43. सूर्यः॒ स्वाहा॒ स्वाहा॒ सूर्यः॒ सूर्यः॒ स्वाहा᳚ । \newline
44. स्वाहा॒ ऽर्को᳚ ऽर्कः स्वाहा॒ स्वाहा॒ ऽर्कः । \newline
45. अ॒र्को वै वा अ॒र्को᳚ ऽर्को वै । \newline
46. वा ए॒ष ए॒ष वै वा ए॒षः । \newline
47. ए॒ष यद् यदे॒ष ए॒ष यत् । \newline
48. यद॒ग्नि र॒ग्निर् यद् यद॒ग्निः । \newline
49. अ॒ग्नि र॒सा व॒सा व॒ग्नि र॒ग्नि र॒सौ । \newline
50. अ॒सा वा॑दि॒त्य आ॑दि॒त्यो॑ ऽसा व॒सा वा॑दि॒त्यः । \newline
51. आ॒दि॒त्यो᳚ ऽश्वमे॒धो᳚ ऽश्वमे॒ध आ॑दि॒त्य आ॑दि॒त्यो᳚ ऽश्वमे॒धः । \newline

\textbf{Ghana Paata } \newline

1. ए॒व तम॒ स्तम॑ ए॒वैव तमो॑ मृ॒त्युम् मृ॒त्युम् तम॑ ए॒वैव तमो॑ मृ॒त्युम् । \newline
2. तमो॑ मृ॒त्युम् मृ॒त्युम् तम॒ स्तमो॑ मृ॒त्यु मपाप॑ मृ॒त्युम् तम॒ स्तमो॑ मृ॒त्यु मप॑ । \newline
3. मृ॒त्यु मपाप॑ मृ॒त्युम् मृ॒त्यु मप॑ हते ह॒ते ऽप॑ मृ॒त्युम् मृ॒त्यु मप॑ हते । \newline
4. अप॑ हते ह॒ते ऽपाप॑ हते॒ हिर॑ण्यꣳ॒॒ हिर॑ण्यꣳ ह॒ते ऽपाप॑ हते॒ हिर॑ण्यम् । \newline
5. ह॒ते॒ हिर॑ण्यꣳ॒॒ हिर॑ण्यꣳ हते हते॒ हिर॑ण्यम् ददाति ददाति॒ हिर॑ण्यꣳ हते हते॒ हिर॑ण्यम् ददाति । \newline
6. हिर॑ण्यम् ददाति ददाति॒ हिर॑ण्यꣳ॒॒ हिर॑ण्यम् ददाति॒ ज्योति॒र् ज्योति॑र् ददाति॒ हिर॑ण्यꣳ॒॒ हिर॑ण्यम् ददाति॒ ज्योतिः॑ । \newline
7. द॒दा॒ति॒ ज्योति॒र् ज्योति॑र् ददाति ददाति॒ ज्योति॒र् वै वै ज्योति॑र् ददाति ददाति॒ ज्योति॒र् वै । \newline
8. ज्योति॒र् वै वै ज्योति॒र् ज्योति॒र् वै हिर॑ण्यꣳ॒॒ हिर॑ण्यं॒ ॅवै ज्योति॒र् ज्योति॒र् वै हिर॑ण्यम् । \newline
9. वै हिर॑ण्यꣳ॒॒ हिर॑ण्यं॒ ॅवै वै हिर॑ण्य॒म् ज्योति॑षा॒ ज्योति॑षा॒ हिर॑ण्यं॒ ॅवै वै हिर॑ण्य॒म् ज्योति॑षा । \newline
10. हिर॑ण्य॒म् ज्योति॑षा॒ ज्योति॑षा॒ हिर॑ण्यꣳ॒॒ हिर॑ण्य॒म् ज्योति॑षै॒ वैव ज्योति॑षा॒ हिर॑ण्यꣳ॒॒ हिर॑ण्य॒म् ज्योति॑षै॒व । \newline
11. ज्योति॑षै॒ वैव ज्योति॑षा॒ ज्योति॑षै॒व तम॒ स्तम॑ ए॒व ज्योति॑षा॒ ज्योति॑षै॒व तमः॑ । \newline
12. ए॒व तम॒ स्तम॑ ए॒वैव तमो ऽपाप॒ तम॑ ए॒वैव तमो ऽप॑ । \newline
13. तमो ऽपाप॒ तम॒ स्तमो ऽप॑ हते ह॒ते ऽप॒ तम॒ स्तमो ऽप॑ हते । \newline
14. अप॑ हते ह॒ते ऽपाप॑ ह॒ते ऽथो॒ अथो॑ ह॒ते ऽपाप॑ ह॒ते ऽथो᳚ । \newline
15. ह॒ते ऽथो॒ अथो॑ हते ह॒ते ऽथो॒ तेज॒ स्तेजो ऽथो॑ हते ह॒ते ऽथो॒ तेजः॑ । \newline
16. अथो॒ तेज॒ स्तेजो ऽथो॒ अथो॒ तेजो॒ वै वै तेजो ऽथो॒ अथो॒ तेजो॒ वै । \newline
17. अथो॒ इत्यथो᳚ । \newline
18. तेजो॒ वै वै तेज॒ स्तेजो॒ वै हिर॑ण्यꣳ॒॒ हिर॑ण्यं॒ ॅवै तेज॒ स्तेजो॒ वै हिर॑ण्यम् । \newline
19. वै हिर॑ण्यꣳ॒॒ हिर॑ण्यं॒ ॅवै वै हिर॑ण्य॒म् तेज॒ स्तेजो॒ हिर॑ण्यं॒ ॅवै वै हिर॑ण्य॒म् तेजः॑ । \newline
20. हिर॑ण्य॒म् तेज॒ स्तेजो॒ हिर॑ण्यꣳ॒॒ हिर॑ण्य॒म् तेज॑ ए॒वैव तेजो॒ हिर॑ण्यꣳ॒॒ हिर॑ण्य॒म् तेज॑ ए॒व । \newline
21. तेज॑ ए॒वैव तेज॒ स्तेज॑ ए॒वात्मन् ना॒त्मन् ने॒व तेज॒ स्तेज॑ ए॒वात्मन्न् । \newline
22. ए॒वात्मन् ना॒त्मन् ने॒वै वात्मन् ध॑त्ते धत्त आ॒त्मन् ने॒वै वात्मन् ध॑त्ते । \newline
23. आ॒त्मन् ध॑त्ते धत्त आ॒त्मन् ना॒त्मन् ध॑त्ते॒ सुवः॒ सुव॑र् धत्त आ॒त्मन् ना॒त्मन् ध॑त्ते॒ सुवः॑ । \newline
24. ध॒त्ते॒ सुवः॒ सुव॑र् धत्ते धत्ते॒ सुव॒र् न न सुव॑र् धत्ते धत्ते॒ सुव॒र् न । \newline
25. सुव॒र् न न सुवः॒ सुव॒र् न घ॒र्मो घ॒र्मो न सुवः॒ सुव॒र् न घ॒र्मः । \newline
26. न घ॒र्मो घ॒र्मो न न घ॒र्मः स्वाहा॒ स्वाहा॑ घ॒र्मो न न घ॒र्मः स्वाहा᳚ । \newline
27. घ॒र्मः स्वाहा॒ स्वाहा॑ घ॒र्मो घ॒र्मः स्वाहा॒ सुवः॒ सुवः॒ स्वाहा॑ घ॒र्मो घ॒र्मः स्वाहा॒ सुवः॑ । \newline
28. स्वाहा॒ सुवः॒ सुवः॒ स्वाहा॒ स्वाहा॒ सुव॒र् न न सुवः॒ स्वाहा॒ स्वाहा॒ सुव॒र् न । \newline
29. सुव॒र् न न सुवः॒ सुव॒र् नार्को᳚ ऽर्को न सुवः॒ सुव॒र् नार्कः । \newline
30. नार्को᳚ ऽर्को न नार्कः स्वाहा॒ स्वाहा॒ ऽर्को न नार्कः स्वाहा᳚ । \newline
31. अ॒र्कः स्वाहा॒ स्वाहा॒ ऽर्को᳚ ऽर्कः स्वाहा॒ सुवः॒ सुवः॒ स्वाहा॒ ऽर्को᳚ ऽर्कः स्वाहा॒ सुवः॑ । \newline
32. स्वाहा॒ सुवः॒ सुवः॒ स्वाहा॒ स्वाहा॒ सुव॒र् न न सुवः॒ स्वाहा॒ स्वाहा॒ सुव॒र् न । \newline
33. सुव॒र् न न सुवः॒ सुव॒र् न शु॒क्रः शु॒क्रो न सुवः॒ सुव॒र् न शु॒क्रः । \newline
34. न शु॒क्रः शु॒क्रो न न शु॒क्रः स्वाहा॒ स्वाहा॑ शु॒क्रो न न शु॒क्रः स्वाहा᳚ । \newline
35. शु॒क्रः स्वाहा॒ स्वाहा॑ शु॒क्रः शु॒क्रः स्वाहा॒ सुवः॒ सुवः॒ स्वाहा॑ शु॒क्रः शु॒क्रः स्वाहा॒ सुवः॑ । \newline
36. स्वाहा॒ सुवः॒ सुवः॒ स्वाहा॒ स्वाहा॒ सुव॒र् न न सुवः॒ स्वाहा॒ स्वाहा॒ सुव॒र् न । \newline
37. सुव॒र् न न सुवः॒ सुव॒र् न ज्योति॒र् ज्योति॒र् न सुवः॒ सुव॒र् न ज्योतिः॑ । \newline
38. न ज्योति॒र् ज्योति॒र् न न ज्योतिः॒ स्वाहा॒ स्वाहा॒ ज्योति॒र् न न ज्योतिः॒ स्वाहा᳚ । \newline
39. ज्योतिः॒ स्वाहा॒ स्वाहा॒ ज्योति॒र् ज्योतिः॒ स्वाहा॒ सुवः॒ सुवः॒ स्वाहा॒ ज्योति॒र् ज्योतिः॒ स्वाहा॒ सुवः॑ । \newline
40. स्वाहा॒ सुवः॒ सुवः॒ स्वाहा॒ स्वाहा॒ सुव॒र् न न सुवः॒ स्वाहा॒ स्वाहा॒ सुव॒र् न । \newline
41. सुव॒र् न न सुवः॒ सुव॒र् न सूर्यः॒ सूर्यो॒ न सुवः॒ सुव॒र् न सूर्यः॑ । \newline
42. न सूर्यः॒ सूर्यो॒ न न सूर्यः॒ स्वाहा॒ स्वाहा॒ सूर्यो॒ न न सूर्यः॒ स्वाहा᳚ । \newline
43. सूर्यः॒ स्वाहा॒ स्वाहा॒ सूर्यः॒ सूर्यः॒ स्वाहा॒ ऽर्को᳚ ऽर्कः स्वाहा॒ सूर्यः॒ सूर्यः॒ स्वाहा॒ ऽर्कः । \newline
44. स्वाहा॒ ऽर्को᳚ ऽर्कः स्वाहा॒ स्वाहा॒ ऽर्को वै वा अ॒र्कः स्वाहा॒ स्वाहा॒ ऽर्को वै । \newline
45. अ॒र्को वै वा अ॒र्को᳚ ऽर्को वा ए॒ष ए॒ष वा अ॒र्को᳚ ऽर्को वा ए॒षः । \newline
46. वा ए॒ष ए॒ष वै वा ए॒ष यद् यदे॒ष वै वा ए॒ष यत् । \newline
47. ए॒ष यद् यदे॒ष ए॒ष यद॒ग्नि र॒ग्निर् यदे॒ष ए॒ष यद॒ग्निः । \newline
48. यद॒ग्नि र॒ग्निर् यद् यद॒ग्नि र॒सा व॒सा व॒ग्निर् यद् यद॒ग्नि र॒सौ । \newline
49. अ॒ग्नि र॒सा व॒सा व॒ग्नि र॒ग्नि र॒सा वा॑दि॒त्य आ॑दि॒त्यो॑ ऽसा व॒ग्नि र॒ग्नि र॒सा वा॑दि॒त्यः । \newline
50. अ॒सा वा॑दि॒त्य आ॑दि॒त्यो॑ ऽसा व॒सा वा॑दि॒त्यो᳚ ऽश्वमे॒धो᳚ ऽश्वमे॒ध आ॑दि॒त्यो॑ ऽसा व॒सा वा॑दि॒त्यो᳚ ऽश्वमे॒धः । \newline
51. आ॒दि॒त्यो᳚ ऽश्वमे॒धो᳚ ऽश्वमे॒ध आ॑दि॒त्य आ॑दि॒त्यो᳚ ऽश्वमे॒धो यद् यद॑श्वमे॒ध आ॑दि॒त्य आ॑दि॒त्यो᳚ ऽश्वमे॒धो यत् । \newline
\pagebreak
\markright{ TS 5.7.5.3  \hfill https://www.vedavms.in \hfill}

\section{ TS 5.7.5.3 }

\textbf{TS 5.7.5.3 } \newline
\textbf{Samhita Paata} \newline

ऽश्वमे॒धो यदे॒ता आहु॑ती र्जुहोत्य॑र्का-श्वमे॒धयो॑रे॒व ज्योतीꣳ॑षि॒ सं द॑धात्ये॒ष ह॒ त्वा अ॑र्काश्वमे॒धी यस्यै॒तद॒ग्नौ क्रि॒यत॒ आपो॒ वा इ॒दमग्रे॑ सलि॒लमा॑सी॒थ् स ए॒तां प्र॒जाप॑तिः प्रथ॒मां चिति॑मपश्य॒त् तामुपा॑धत्त॒ तदि॒यम॑भव॒त् तं ॅवि॒श्वक॑र्माऽब्रवी॒दुप॒ त्वाऽऽया॒नीति॒ नेह लो॒को᳚ऽस्तीत्य॑ - [  ] \newline

\textbf{Pada Paata} \newline

अ॒श्व॒मे॒ध इत्य॑श्व - मे॒धः । यत् । ए॒ताः । आहु॑ती॒रित्या - हु॒तीः॒ । जु॒होति॑ । अ॒र्का॒श्व॒मे॒धयो॒रित्य॑र्क - अ॒श्व॒मे॒धयोः᳚ । ए॒व । ज्योतीꣳ॑षि । समिति॑ । द॒धा॒ति॒ । ए॒षः । ह॒ । तु । वै । अ॒र्का॒श्व॒मे॒धीत्य॑र्क-अ॒श्व॒मे॒धी । यस्य॑ । ए॒तत् । अ॒ग्नौ । क्रि॒यते᳚ । आपः॑ । वै । इ॒दम् । अग्रे᳚ । स॒लि॒लम् । आ॒सी॒त् । सः । ए॒ताम् । प्र॒जाप॑ति॒रिति॑ प्र॒जा - प॒तिः॒ । प्र॒थ॒माम् । चिति᳚म् । अ॒प॒श्य॒त् । ताम् । उपेति॑ । अ॒ध॒त्त॒ । तत् । इ॒यम् । अ॒भ॒व॒त् । तम् । वि॒श्वक॒र्मेति॑ वि॒श्व - क॒र्मा॒ । अ॒ब्र॒वी॒त् । उपेति॑ । त्वा॒ । एति॑ । अ॒या॒नि॒ । इति॑ । न । इ॒ह । लो॒कः । अ॒स्ति॒ । इति॑ ।  \newline


\textbf{Krama Paata} \newline

अ॒श्व॒मे॒धो यत् । अ॒श्व॒मे॒ध इत्य॑श्व - मे॒धः । यदे॒ताः । ए॒ता आहु॑तीः । आहु॑तीर् जु॒होति॑ । आहु॑ती॒रित्या - हु॒तीः॒ । जु॒होत्य॑र्काश्वमे॒धयोः᳚ । अ॒र्का॒श्व॒मे॒धयो॑रे॒व । अ॒र्का॒श्व॒मे॒धयो॒रित्य॑र्क - अ॒श्व॒मे॒धयोः᳚ । ए॒व ज्योतीꣳ॑षि । ज्योतीꣳ॑षि॒ सम् । सम् द॑धाति । द॒धा॒त्ये॒षः । ए॒ष ह॑ । ह॒ तु । त्वै । वा अ॑र्काश्वमे॒धी । अ॒र्का॒श्व॒मे॒धी यस्य॑ । अ॒र्का॒श्व॒मे॒धीत्य॑र्क - अ॒श्व॒मे॒धी । यस्यै॒तत् । ए॒तद॒ग्नौ । अ॒ग्नौ क्रि॒यते᳚ । क्रि॒यत॒ आपः॑ । आपो॒ वै । वा इ॒दम् । इ॒दमग्रे᳚ । अग्रे॑ सलि॒लम् । स॒लि॒लमा॑सीत् । आ॒सी॒थ् सः । स ए॒ताम् । ए॒ताम् प्र॒जाप॑तिः । प्र॒जाप॑तिः प्रथ॒माम् । प्र॒जाप॑ति॒रिति॑ प्र॒जा - प॒तिः॒ । प्र॒थ॒माम् चिति᳚म् । चिति॑मपश्यत् । अ॒प॒श्य॒त् ताम् । तामुप॑ । उपा॑धत्त । अ॒ध॒त्त॒ तत् । तदि॒यम् । इ॒यम॑भवत् । अ॒भ॒व॒त् तम् । तम् ॅवि॒श्वक॑र्मा । वि॒श्वक॑र्माऽब्रवीत् । वि॒श्वक॒र्मेति॑ वि॒श्व - क॒र्मा॒ । अ॒ब्र॒वी॒दुप॑ । उप॑ त्वा । त्वा । आऽया॑नि । अ॒या॒नीति॑ । इति॒ न । नेह । इ॒ह लो॒कः । लो॒को᳚ऽस्ति । अ॒स्तीति॑ । इत्य॑ब्रवीत् \newline

\textbf{Jatai Paata} \newline

1. अ॒श्व॒मे॒धो यद् यद॑श्वमे॒धो᳚ ऽश्वमे॒धो यत् । \newline
2. अ॒श्व॒मे॒ध इत्य॑श्व - मे॒धः । \newline
3. यदे॒ता ए॒ता यद् यदे॒ताः । \newline
4. ए॒ता आहु॑ती॒ राहु॑ती रे॒ता ए॒ता आहु॑तीः । \newline
5. आहु॑तीर् जु॒होति॑ जु॒हो त्याहु॑ती॒ राहु॑तीर् जु॒होति॑ । \newline
6. आहु॑ती॒रित्या - हु॒तीः॒ । \newline
7. जु॒हो त्य॑र्काश्वमे॒धयो॑ रर्काश्वमे॒धयो᳚र् जु॒होति॑ जु॒हो त्य॑र्काश्वमे॒धयोः᳚ । \newline
8. अ॒र्का॒श्व॒मे॒धयो॑ रे॒वै वार्का᳚श्वमे॒धयो॑ रर्काश्वमे॒धयो॑ रे॒व । \newline
9. अ॒र्का॒श्व॒मे॒धयो॒रित्य॑र्क - अ॒श्व॒मे॒धयोः᳚ । \newline
10. ए॒व ज्योतीꣳ॑षि॒ ज्योतीꣳ॑ष्ये॒ वैव ज्योतीꣳ॑षि । \newline
11. ज्योतीꣳ॑षि॒ सꣳ सम् ज्योतीꣳ॑षि॒ ज्योतीꣳ॑षि॒ सम् । \newline
12. सम् द॑धाति दधाति॒ सꣳ सम् द॑धाति । \newline
13. द॒धा॒ त्ये॒ष ए॒ष द॑धाति दधा त्ये॒षः । \newline
14. ए॒ष ह॑ है॒ष ए॒ष ह॑ । \newline
15. ह॒ तु तु ह॑ ह॒ तु । \newline
16. त्वै वै तु त्वै । \newline
17. वा अ॑र्काश्वमे॒ ध्य॑र्काश्वमे॒धी वै वा अ॑र्काश्वमे॒धी । \newline
18. अ॒र्का॒श्व॒मे॒धी यस्य॒ यस्या᳚र्काश्वमे॒ ध्य॑र्काश्वमे॒धी यस्य॑ । \newline
19. अ॒र्का॒श्व॒मे॒धीत्य॑र्क - अ॒श्व॒मे॒धी । \newline
20. यस्यै॒त दे॒तद् यस्य॒ यस्यै॒तत् । \newline
21. ए॒त द॒ग्ना व॒ग्ना वे॒त दे॒त द॒ग्नौ । \newline
22. अ॒ग्नौ क्रि॒यते᳚ क्रि॒यते॒ ऽग्ना व॒ग्नौ क्रि॒यते᳚ । \newline
23. क्रि॒यत॒ आप॒ आपः॑ क्रि॒यते᳚ क्रि॒यत॒ आपः॑ । \newline
24. आपो॒ वै वा आप॒ आपो॒ वै । \newline
25. वा इ॒द मि॒दं ॅवै वा इ॒दम् । \newline
26. इ॒द मग्रे ऽग्र॑ इ॒द मि॒द मग्रे᳚ । \newline
27. अग्रे॑ सलि॒लꣳ स॑लि॒ल मग्रे ऽग्रे॑ सलि॒लम् । \newline
28. स॒लि॒ल मा॑सी दासीथ् सलि॒लꣳ स॑लि॒ल मा॑सीत् । \newline
29. आ॒सी॒थ् स स आ॑सी दासी॒थ् सः । \newline
30. स ए॒ता मे॒ताꣳ स स ए॒ताम् । \newline
31. ए॒ताम् प्र॒जाप॑तिः प्र॒जाप॑ति रे॒ता मे॒ताम् प्र॒जाप॑तिः । \newline
32. प्र॒जाप॑तिः प्रथ॒माम् प्र॑थ॒माम् प्र॒जाप॑तिः प्र॒जाप॑तिः प्रथ॒माम् । \newline
33. प्र॒जाप॑ति॒रिति॑ प्र॒जा - प॒तिः॒ । \newline
34. प्र॒थ॒माम् चिति॒म् चिति॑म् प्रथ॒माम् प्र॑थ॒माम् चिति᳚म् । \newline
35. चिति॑ मपश्य दपश्य॒च् चिति॒म् चिति॑ मपश्यत् । \newline
36. अ॒प॒श्य॒त् ताम् ता म॑पश्य दपश्य॒त् ताम् । \newline
37. ता मुपोप॒ ताम् ता मुप॑ । \newline
38. उपा॑ धत्ता ध॒त्तो पोपा॑ धत्त । \newline
39. अ॒ध॒त्त॒ तत् तद॑धत्ता धत्त॒ तत् । \newline
40. तदि॒य मि॒यम् तत् तदि॒यम् । \newline
41. इ॒य म॑भव दभव दि॒य मि॒य म॑भवत् । \newline
42. अ॒भ॒व॒त् तम् त म॑भव दभव॒त् तम् । \newline
43. तं ॅवि॒श्वक॑मा वि॒श्वक॑मा॒ तम् तं ॅवि॒श्वक॑मा । \newline
44. वि॒श्वक॑मा ऽब्रवी दब्रवीद् वि॒श्वक॑मा वि॒श्वक॑मा ऽब्रवीत् । \newline
45. वि॒श्वक॒र्मेति॑ वि॒श्व - क॒र्मा॒ । \newline
46. अ॒ब्र॒वी॒ दुपोपा᳚ ब्रवी दब्रवी॒ दुप॑ । \newline
47. उप॑ त्वा॒ त्वोपोप॑ त्वा । \newline
48. त्वा ऽऽत्वा॒ त्वा । \newline
49. आ ऽया᳚ न्यया॒ न्या ऽया॑नि । \newline
50. अ॒या॒नीती त्य॑या न्यया॒नीति॑ । \newline
51. इति॒ न ने तीति॒ न । \newline
52. नेहे ह न नेह । \newline
53. इ॒ह लो॒को लो॒क इ॒हेह लो॒कः । \newline
54. लो॒को᳚ ऽस्त्यस्ति लो॒को लो॒को᳚ ऽस्ति । \newline
55. अ॒स्तीती त्य॑स्त्य॒स्तीति॑ । \newline
56. इत्य॑ब्रवी दब्रवी॒ दिती त्य॑ब्रवीत् । \newline

\textbf{Ghana Paata } \newline

1. अ॒श्व॒मे॒धो यद् यद॑श्वमे॒धो᳚ ऽश्वमे॒धो यदे॒ता ए॒ता यद॑श्वमे॒धो᳚ ऽश्वमे॒धो यदे॒ताः । \newline
2. अ॒श्व॒मे॒ध इत्य॑श्व - मे॒धः । \newline
3. यदे॒ता ए॒ता यद् यदे॒ता आहु॑ती॒ राहु॑ती रे॒ता यद् यदे॒ता आहु॑तीः । \newline
4. ए॒ता आहु॑ती॒ राहु॑ती रे॒ता ए॒ता आहु॑तीर् जु॒होति॑ जु॒हो त्याहु॑ती रे॒ता ए॒ता आहु॑तीर् जु॒होति॑ । \newline
5. आहु॑तीर् जु॒होति॑ जु॒हो त्याहु॑ती॒ राहु॑तीर् जु॒हो त्य॑र्काश्वमे॒धयो॑ रर्काश्वमे॒धयो᳚र् जु॒हो त्याहु॑ती॒ राहु॑तीर् जु॒हो
त्य॑र्काश्वमे॒धयोः᳚ । \newline
6. आहु॑ती॒रित्या - हु॒तीः॒ । \newline
7. जु॒हो त्य॑र्काश्वमे॒धयो॑ रर्काश्वमे॒धयो᳚र् जु॒होति॑ जु॒हो त्य॑र्काश्वमे॒धयो॑ रे॒वैवा र्का᳚श्वमे॒धयो᳚र् जु॒होति॑ जु॒हो त्य॑र्काश्वमे॒धयो॑ रे॒व । \newline
8. अ॒र्का॒श्व॒मे॒धयो॑ रे॒वैवा र्का᳚श्वमे॒धयो॑ रर्काश्वमे॒धयो॑ रे॒व ज्योतीꣳ॑षि॒ ज्योतीꣳ॑ ष्ये॒वा र्का᳚श्वमे॒धयो॑ रर्काश्वमे॒धयो॑ रे॒व ज्योतीꣳ॑षि । \newline
9. अ॒र्का॒श्व॒मे॒धयो॒रित्य॑र्क - अ॒श्व॒मे॒धयोः᳚ । \newline
10. ए॒व ज्योतीꣳ॑षि॒ ज्योतीꣳ॑ ष्ये॒वैव ज्योतीꣳ॑षि॒ सꣳ सम् ज्योतीꣳ॑ ष्ये॒वैव ज्योतीꣳ॑षि॒ सम् । \newline
11. ज्योतीꣳ॑षि॒ सꣳ सम् ज्योतीꣳ॑षि॒ ज्योतीꣳ॑षि॒ सम् द॑धाति दधाति॒ सम् ज्योतीꣳ॑षि॒ ज्योतीꣳ॑षि॒ सम् द॑धाति । \newline
12. सम् द॑धाति दधाति॒ सꣳ सम् द॑धा त्ये॒ष ए॒ष द॑धाति॒ सꣳ सम् द॑धा त्ये॒षः । \newline
13. द॒धा॒ त्ये॒ष ए॒ष द॑धाति दधा त्ये॒ष ह॑ है॒ष द॑धाति दधा त्ये॒ष ह॑ । \newline
14. ए॒ष ह॑ है॒ष ए॒ष ह॒ तु तु है॒ष ए॒ष ह॒ तु । \newline
15. ह॒ तु तु ह॑ ह॒ त्वै वै तु ह॑ ह॒ त्वै । \newline
16. त्वै वै तु त्वा अ॑र्काश्वमे॒ ध्य॑र्काश्वमे॒धी वै तु त्वा अ॑र्काश्वमे॒धी । \newline
17. वा अ॑र्काश्वमे॒ ध्य॑र्काश्वमे॒धी वै वा अ॑र्काश्वमे॒धी यस्य॒ यस्या᳚ र्काश्वमे॒धी वै वा अ॑र्काश्वमे॒धी यस्य॑ । \newline
18. अ॒र्का॒श्व॒मे॒धी यस्य॒ यस्या᳚र्काश्वमे॒ ध्य॑र्काश्वमे॒धी यस्यै॒त दे॒तद् यस्या᳚र्काश्वमे॒ 
ध्य॑र्काश्वमे॒धी यस्यै॒तत् । \newline
19. अ॒र्का॒श्व॒मे॒धीत्य॑र्क - अ॒श्व॒मे॒धी । \newline
20. यस्यै॒ तदे॒तद् यस्य॒ यस्यै॒ तद॒ग्ना व॒ग्ना वे॒तद् यस्य॒ यस्यै॒ तद॒ग्नौ । \newline
21. ए॒तद॒ग्ना व॒ग्ना वे॒तदे॒ तद॒ग्नौ क्रि॒यते᳚ क्रि॒यते॒ ऽग्ना वे॒तदे॒ तद॒ग्नौ क्रि॒यते᳚ । \newline
22. अ॒ग्नौ क्रि॒यते᳚ क्रि॒यते॒ ऽग्ना व॒ग्नौ क्रि॒यत॒ आप॒ आपः॑ क्रि॒यते॒ ऽग्ना व॒ग्नौ क्रि॒यत॒ आपः॑ । \newline
23. क्रि॒यत॒ आप॒ आपः॑ क्रि॒यते᳚ क्रि॒यत॒ आपो॒ वै वा आपः॑ क्रि॒यते᳚ क्रि॒यत॒ आपो॒ वै । \newline
24. आपो॒ वै वा आप॒ आपो॒ वा इ॒द मि॒दं ॅवा आप॒ आपो॒ वा इ॒दम् । \newline
25. वा इ॒द मि॒दं ॅवै वा इ॒द मग्रे ऽग्र॑ इ॒दं ॅवै वा इ॒द मग्रे᳚ । \newline
26. इ॒द मग्रे ऽग्र॑ इ॒द मि॒द मग्रे॑ सलि॒लꣳ स॑लि॒ल मग्र॑ इ॒द मि॒द मग्रे॑ सलि॒लम् । \newline
27. अग्रे॑ सलि॒लꣳ स॑लि॒ल मग्रे ऽग्रे॑ सलि॒ल मा॑सी दासीथ् सलि॒ल मग्रे ऽग्रे॑ सलि॒ल मा॑सीत् । \newline
28. स॒लि॒ल मा॑सी दासीथ् सलि॒लꣳ स॑लि॒ल मा॑सी॒थ् स स आ॑सीथ् सलि॒लꣳ स॑लि॒ल मा॑सी॒थ् सः । \newline
29. आ॒सी॒थ् स स आ॑सीदा सी॒थ् स ए॒ता मे॒ताꣳ स आ॑सी दासी॒थ् स ए॒ताम् । \newline
30. स ए॒ता मे॒ताꣳ स स ए॒ताम् प्र॒जाप॑तिः प्र॒जाप॑ति रे॒ताꣳ स स ए॒ताम् प्र॒जाप॑तिः । \newline
31. ए॒ताम् प्र॒जाप॑तिः प्र॒जाप॑ति रे॒ता मे॒ताम् प्र॒जाप॑तिः प्रथ॒माम् प्र॑थ॒माम् प्र॒जाप॑ति रे॒ता मे॒ताम् प्र॒जाप॑तिः प्रथ॒माम् । \newline
32. प्र॒जाप॑तिः प्रथ॒माम् प्र॑थ॒माम् प्र॒जाप॑तिः प्र॒जाप॑तिः प्रथ॒माम् चिति॒म् चिति॑म् प्रथ॒माम् प्र॒जाप॑तिः प्र॒जाप॑तिः प्रथ॒माम् चिति᳚म् । \newline
33. प्र॒जाप॑ति॒रिति॑ प्र॒जा - प॒तिः॒ । \newline
34. प्र॒थ॒माम् चिति॒म् चिति॑म् प्रथ॒माम् प्र॑थ॒माम् चिति॑ मपश्य दपश्य॒च् चिति॑म् प्रथ॒माम् प्र॑थ॒माम् चिति॑ मपश्यत् । \newline
35. चिति॑ मपश्य दपश्य॒च् चिति॒म् चिति॑ मपश्य॒त् ताम् ता म॑पश्य॒च् चिति॒म् चिति॑ मपश्य॒त् ताम् । \newline
36. अ॒प॒श्य॒त् ताम् ता म॑पश्य दपश्य॒त् ता मुपोप॒ ता म॑पश्य दपश्य॒त् ता मुप॑ । \newline
37. ता मुपोप॒ ताम् ता मुपा॑धत्ता ध॒त्तोप॒ ताम् ता मुपा॑धत्त । \newline
38. उपा॑धत्ता ध॒त्तोपोपा॑ धत्त॒ तत् तद॑ध॒त्तोपोपा॑ धत्त॒ तत् । \newline
39. अ॒ध॒त्त॒ तत् तद॑धत्ता धत्त॒ तदि॒य मि॒यम् तद॑धत्ता धत्त॒ तदि॒यम् । \newline
40. तदि॒य मि॒यम् तत् तदि॒य म॑भव दभव दि॒यम् तत् तदि॒य म॑भवत् । \newline
41. इ॒य म॑भव दभव दि॒य मि॒य म॑भव॒त् तम् त म॑भव दि॒य मि॒य म॑भव॒त् तम् । \newline
42. अ॒भ॒व॒त् तम् त म॑भव दभव॒त् तं ॅवि॒श्वक॑मा वि॒श्वक॑मा॒ त म॑भव दभव॒त् तं ॅवि॒श्वक॑मा । \newline
43. तं ॅवि॒श्वक॑मा वि॒श्वक॑मा॒ तम् तं ॅवि॒श्वक॑मा ऽब्रवी दब्रवीद् वि॒श्वक॑मा॒ तम् तं ॅवि॒श्वक॑मा ऽब्रवीत् । \newline
44. वि॒श्वक॑मा ऽब्रवी दब्रवीद् वि॒श्वक॑मा वि॒श्वक॑मा ऽब्रवी॒ दुपोपा᳚ ब्रवीद् वि॒श्वक॑मा वि॒श्वक॑मा ऽब्रवी॒ दुप॑ । \newline
45. वि॒श्वक॒र्मेति॑ वि॒श्व - क॒र्मा॒ । \newline
46. अ॒ब्र॒वी॒ दुपोपा᳚ ब्रवी दब्रवी॒ दुप॑ त्वा॒ त्वोपा᳚ब्रवी दब्रवी॒ दुप॑ त्वा । \newline
47. उप॑ त्वा॒ त्वोपोप॒ त्वा ऽऽत्वोपोप॒ त्वा । \newline
48. त्वा ऽऽत्वा॒ त्वा ऽया᳚ न्यया॒ न्या त्वा॒ त्वा ऽया॑नि । \newline
49. आ ऽया᳚ न्यया॒ न्या ऽया॒नीती त्य॑या॒ न्या ऽया॒नीति॑ । \newline
50. अ॒या॒नीती त्य॑या न्यया॒नीति॒ न नेत्य॑या न्यया॒नीति॒ न । \newline
51. इति॒ न नेतीति॒ नेहेह नेतीति॒ नेह । \newline
52. नेहेह न नेह लो॒को लो॒क इ॒ह न नेह लो॒कः । \newline
53. इ॒ह लो॒को लो॒क इ॒हेह लो॒को᳚ ऽस्त्यस्ति लो॒क इ॒हेह लो॒को᳚ ऽस्ति । \newline
54. लो॒को᳚ ऽस्त्यस्ति लो॒को लो॒को᳚ ऽस्तीती त्य॑स्ति लो॒को लो॒को᳚ ऽस्तीति॑ । \newline
55. अ॒स्तीती त्य॑स्त्य॒ स्तीत्य॑ब्रवी दब्रवी॒ दित्य॑ स्त्य॒ स्तीत्य॑ब्रवीत् । \newline
56. इत्य॑ब्रवी दब्रवी॒ दिती त्य॑ब्रवी॒थ् स सो᳚ ऽब्रवी॒ दिती त्य॑ब्रवी॒थ् सः । \newline
\pagebreak
\markright{ TS 5.7.5.4  \hfill https://www.vedavms.in \hfill}

\section{ TS 5.7.5.4 }

\textbf{TS 5.7.5.4 } \newline
\textbf{Samhita Paata} \newline

-ब्रवी॒थ् स ए॒तां द्वि॒तीयां॒ चिति॑मपश्य॒त् तामुपा॑धत्त॒ तद॒न्तरि॑क्षमभव॒थ् स य॒ज्ञ्ः प्र॒जाप॑तिमब्रवी॒दुप॒ त्वाऽऽया॒नीति॒ नेह लो॒को᳚ऽस्तीत्य॑ब्रवी॒थ् स वि॒श्वक॑र्माण-मब्रवी॒दुप॒ त्वाऽऽया॒नीति॒ केन॑ मो॒पैष्य॒सीति॒ दिश्या॑भि॒रित्य॑ब्रवी॒त् तं दिश्या॑भिरु॒पैत् ता उपा॑धत्त॒ ता दिशो॑ - [  ] \newline

\textbf{Pada Paata} \newline

अ॒ब्र॒वी॒त् । सः । ए॒ताम् । द्वि॒तीया᳚म् । चिति᳚म् । अ॒प॒श्य॒त् । ताम् । उपेति॑ । अ॒ध॒त्त॒ । तत् । अ॒न्तरि॑क्षम् । अ॒भ॒व॒त् । सः । य॒ज्ञ्ः । प्र॒जाप॑ति॒मिति॑ प्र॒जा - प॒ति॒म् । अ॒ब्र॒वी॒त् । उपेति॑ । त्वा॒ । एति॑ । अ॒या॒नि॒ । इति॑ । न । इ॒ह । लो॒कः । अ॒स्ति॒ । इति॑ । अ॒ब्र॒वी॒त् । सः । वि॒श्वक॑र्माण॒मिति॑ वि॒श्व-क॒र्मा॒ण॒म् । अ॒ब्र॒वी॒त् । उपेति॑ । त्वा॒ । एति॑ । अ॒या॒नि॒ । इति॑ । केन॑ । मा॒ । उ॒पैष्य॒सीत्यु॑प - ऐष्य॑सि । इति॑ । दिश्या॑भिः । इति॑ । अ॒ब्र॒वी॒त् । तम् । दिश्या॑भिः । उ॒पैदित्यु॑प - ऐत् । ताः । उपेति॑ । अ॒ध॒त्त॒ । ताः । दिशः॑ ।  \newline


\textbf{Krama Paata} \newline

अ॒ब्र॒वी॒थ् सः । स ए॒ताम् । ए॒ताम् द्वि॒तीया᳚म् । द्वि॒तीया॒म् चिति᳚म् । चिति॑मपश्यत् । अ॒प॒श्य॒त् ताम् । तामुप॑ । उपा॑धत्त । अ॒ध॒त्त॒ तत् । तद॒न्तरि॑क्षम् । अ॒न्तरि॑क्षमभवत् । अ॒भ॒व॒थ् सः । स य॒ज्ञ्ः । य॒ज्ञ्ः प्र॒जाप॑तिम् । प्र॒जाप॑तिमब्रवीत् । प्र॒जाप॑ति॒मिति॑ प्र॒जा - प॒ति॒म् । अ॒ब्र॒वी॒दुप॑ । उप॑ त्वा । त्वा । आऽया॑नि । अ॒या॒नीति॑ । इति॒ न । नेह । इ॒ह लो॒कः । लो॒को᳚ऽस्ति । अ॒स्तीति॑ । इत्य॑ब्रवीत् । अ॒ब्र॒वी॒थ् सः । स वि॒श्वक॑र्माणम् । वि॒श्वक॑र्माणमब्रवीत् । वि॒श्वक॑र्माण॒मिति॑ वि॒श्व - क॒र्मा॒ण॒म् । अ॒ब्र॒वी॒दुप॑ । उप॑ त्वा । त्वा । आऽया॑नि । अ॒या॒नीति॑ । इति॒ केन॑ । केन॑ मा । मो॒पैष्य॑सि । उ॒पैष्य॒सीति॑ । उ॒पैष्य॒सीत्यु॑प - ऐष्य॑सि । इति॒ दिश्या॑भिः । दिश्या॑भि॒रिति॑ । इत्य॑ब्रवीत् । अ॒ब्र॒वी॒त् तम् । तम् दिश्या॑भिः । दिश्या॑भिरु॒पैत् । उ॒पैत् ताः । उ॒पैदित्यु॑प - ऐत् । ता उप॑ । उपा॑धत्त । अ॒ध॒त्त॒ ताः । ता दिशः॑ । दिशो॑ऽभवन्न् \newline

\textbf{Jatai Paata} \newline

1. अ॒ब्र॒वी॒थ् स सो᳚ ऽब्रवी दब्रवी॒थ् सः । \newline
2. स ए॒ता मे॒ताꣳ स स ए॒ताम् । \newline
3. ए॒ताम् द्वि॒तीया᳚म् द्वि॒तीया॑ मे॒ता मे॒ताम् द्वि॒तीया᳚म् । \newline
4. द्वि॒तीया॒म् चिति॒म् चिति॑म् द्वि॒तीया᳚म् द्वि॒तीया॒म् चिति᳚म् । \newline
5. चिति॑ मपश्य दपश्य॒च् चिति॒म् चिति॑ मपश्यत् । \newline
6. अ॒प॒श्य॒त् ताम् ता म॑पश्य दपश्य॒त् ताम् । \newline
7. ता मुपोप॒ ताम् ता मुप॑ । \newline
8. उपा॑ धत्ता ध॒त्तो पोपा॑ धत्त । \newline
9. अ॒ध॒त्त॒ तत् तद॑धत्ता धत्त॒ तत् । \newline
10. तद॒न्तरि॑क्ष म॒न्तरि॑क्ष॒म् तत् तद॒न्तरि॑क्षम् । \newline
11. अ॒न्तरि॑क्ष मभव दभव द॒न्तरि॑क्ष म॒न्तरि॑क्ष मभवत् । \newline
12. अ॒भ॒व॒थ् स सो॑ ऽभव दभव॒थ् सः । \newline
13. स य॒ज्ञो य॒ज्ञ्ः स स य॒ज्ञ्ः । \newline
14. य॒ज्ञ्ः प्र॒जाप॑तिम् प्र॒जाप॑तिं ॅय॒ज्ञो य॒ज्ञ्ः प्र॒जाप॑तिम् । \newline
15. प्र॒जाप॑ति मब्रवी दब्रवीत् प्र॒जाप॑तिम् प्र॒जाप॑ति मब्रवीत् । \newline
16. प्र॒जाप॑ति॒मिति॑ प्र॒जा - प॒ति॒म् । \newline
17. अ॒ब्र॒वी॒ दुपोपा᳚ ब्रवी दब्रवी॒ दुप॑ । \newline
18. उप॑ त्वा॒ त्वोपोप॑ त्वा । \newline
19. त्वा ऽऽत्वा॒ त्वा । \newline
20. आ ऽया᳚ न्यया॒ न्याऽया॑नि । \newline
21. अ॒या॒नीती त्य॑या न्यया॒नीति॑ । \newline
22. इति॒ न ने तीति॒ न । \newline
23. नेहेह न नेह । \newline
24. इ॒ह लो॒को लो॒क इ॒हेह लो॒कः । \newline
25. लो॒को᳚ ऽस्त्यस्ति लो॒को लो॒को᳚ ऽस्ति । \newline
26. अ॒स्ती तीत्य॑ स्त्य॒ स्तीति॑ । \newline
27. इत्य॑ब्रवी दब्रवी॒ दिती त्य॑ब्रवीत् । \newline
28. अ॒ब्र॒वी॒थ् स सो᳚ ऽब्रवी दब्रवी॒थ् सः । \newline
29. स वि॒श्वक॑र्माणं ॅवि॒श्वक॑र्माणꣳ॒॒ स स वि॒श्वक॑र्माणम् । \newline
30. वि॒श्वक॑र्माण मब्रवी दब्रवीद् वि॒श्वक॑र्माणं ॅवि॒श्वक॑र्माण मब्रवीत् । \newline
31. वि॒श्वक॑र्माण॒मिति॑ वि॒श्व - क॒र्मा॒ण॒म् । \newline
32. अ॒ब्र॒वी॒ दुपोपा᳚ ब्रवी दब्रवी॒ दुप॑ । \newline
33. उप॑ त्वा॒ त्वोपोप॑ त्वा । \newline
34. त्वा ऽऽत्वा॒ त्वा । \newline
35. आ ऽया᳚ न्यया॒ न्याऽया॑नि । \newline
36. अ॒या॒नीती त्य॑या न्यया॒नीति॑ । \newline
37. इति॒ केन॒ केने तीति॒ केन॑ । \newline
38. केन॑ मा मा॒ केन॒ केन॑ मा । \newline
39. मो॒पैष्य॑ स्यु॒पैष्य॑सि मा मो॒पैष्य॑सि । \newline
40. उ॒पैष्य॒ सीती त्यु॒पैष्य॑ स्यु॒पैष्य॒ सीति॑ । \newline
41. उ॒पैष्य॒सीत्यु॑प - ऐष्य॑सि । \newline
42. इति॒ दिश्या॑भि॒र् दिश्या॑भि॒ रितीति॒ दिश्या॑भिः । \newline
43. दिश्या॑भि॒ रितीति॒ दिश्या॑भि॒र् दिश्या॑भि॒ रिति॑ । \newline
44. इत्य॑ब्रवी दब्रवी॒ दिती त्य॑ब्रवीत् । \newline
45. अ॒ब्र॒वी॒त् तम् त म॑ब्रवी दब्रवी॒त् तम् । \newline
46. तम् दिश्या॑भि॒र् दिश्या॑भि॒ स्तम् तम् दिश्या॑भिः । \newline
47. दिश्या॑भि रु॒पै दु॒पैद् दिश्या॑भि॒र् दिश्या॑भि रु॒पैत् । \newline
48. उ॒पैत् ता स्ता उ॒पै दु॒पैत् ताः । \newline
49. उ॒पैदित्यु॑प - ऐत् । \newline
50. ता उपोप॒ ता स्ता उप॑ । \newline
51. उपा॑धत्ता ध॒त्तो पोपा॑ धत्त । \newline
52. अ॒ध॒त्त॒ ता स्ता अ॑धत्ता धत्त॒ ताः । \newline
53. ता दिशो॒ दिश॒ स्ता स्ता दिशः॑ । \newline
54. दिशो॑ ऽभवन् नभव॒न् दिशो॒ दिशो॑ ऽभवन्न् । \newline

\textbf{Ghana Paata } \newline

1. अ॒ब्र॒वी॒थ् स सो᳚ ऽब्रवी दब्रवी॒थ् स ए॒ता मे॒ताꣳ सो᳚ ऽब्रवी दब्रवी॒थ् स ए॒ताम् । \newline
2. स ए॒ता मे॒ताꣳ स स ए॒ताम् द्वि॒तीया᳚म् द्वि॒तीया॑ मे॒ताꣳ स स ए॒ताम् द्वि॒तीया᳚म् । \newline
3. ए॒ताम् द्वि॒तीया᳚म् द्वि॒तीया॑ मे॒ता मे॒ताम् द्वि॒तीया॒म् चिति॒म् चिति॑म् द्वि॒तीया॑ मे॒ता मे॒ताम् द्वि॒तीया॒म् चिति᳚म् । \newline
4. द्वि॒तीया॒म् चिति॒म् चिति॑म् द्वि॒तीया᳚म् द्वि॒तीया॒म् चिति॑ मपश्य दपश्य॒च् चिति॑म् द्वि॒तीया᳚म् द्वि॒तीया॒म् चिति॑ मपश्यत् । \newline
5. चिति॑ मपश्य दपश्य॒च् चिति॒म् चिति॑ मपश्य॒त् ताम् ता म॑पश्य॒च् चिति॒म् चिति॑ मपश्य॒त् ताम् । \newline
6. अ॒प॒श्य॒त् ताम् ता म॑पश्य दपश्य॒त् ता मुपोप॒ ता म॑पश्य दपश्य॒त् ता मुप॑ । \newline
7. ता मुपोप॒ ताम् ता मुपा॑धत्ता ध॒त्तोप॒ ताम् ता मुपा॑धत्त । \newline
8. उपा॑धत्ता ध॒त्तोपोपा॑ धत्त॒ तत् तद॑ध॒त्तो पोपा॑ धत्त॒ तत् । \newline
9. अ॒ध॒त्त॒ तत् तद॑धत्ता धत्त॒ तद॒न्तरि॑क्ष म॒न्तरि॑क्ष॒म् तद॑धत्ता धत्त॒ तद॒न्तरि॑क्षम् । \newline
10. तद॒न्तरि॑क्ष म॒न्तरि॑क्ष॒म् तत् तद॒न्तरि॑क्ष मभव दभव द॒न्तरि॑क्ष॒म् तत् तद॒न्तरि॑क्ष मभवत् । \newline
11. अ॒न्तरि॑क्ष मभव दभव द॒न्तरि॑क्ष म॒न्तरि॑क्ष मभव॒थ् स सो॑ ऽभव द॒न्तरि॑क्ष म॒न्तरि॑क्ष मभव॒थ् सः । \newline
12. अ॒भ॒व॒थ् स सो॑ ऽभव दभव॒थ् स य॒ज्ञो य॒ज्ञ्ः सो॑ ऽभव दभव॒थ् स य॒ज्ञ्ः । \newline
13. स य॒ज्ञो य॒ज्ञ्ः स स य॒ज्ञ्ः प्र॒जाप॑तिम् प्र॒जाप॑तिं ॅय॒ज्ञ्ः स स य॒ज्ञ्ः प्र॒जाप॑तिम् । \newline
14. य॒ज्ञ्ः प्र॒जाप॑तिम् प्र॒जाप॑तिं ॅय॒ज्ञो य॒ज्ञ्ः प्र॒जाप॑ति मब्रवी दब्रवीत् प्र॒जाप॑तिं ॅय॒ज्ञो य॒ज्ञ्ः प्र॒जाप॑ति मब्रवीत् । \newline
15. प्र॒जाप॑ति मब्रवी दब्रवीत् प्र॒जाप॑तिम् प्र॒जाप॑ति मब्रवी॒ दुपोपा᳚ब्रवीत् प्र॒जाप॑तिम् प्र॒जाप॑ति मब्रवी॒ दुप॑ । \newline
16. प्र॒जाप॑ति॒मिति॑ प्र॒जा - प॒ति॒म् । \newline
17. अ॒ब्र॒वी॒ दुपोपा᳚ ब्रवी दब्रवी॒ दुप॑ त्वा॒ त्वोपा᳚ब्रवी दब्रवी॒ दुप॑ त्वा । \newline
18. उप॑ त्वा॒ त्वोपोप॒ त्वा ऽऽत्वोपोप॒ त्वा । \newline
19. त्वा ऽऽत्वा॒ त्वा ऽया᳚ न्यया॒ न्या त्वा॒ त्वा ऽया॑नि । \newline
20. आ ऽया᳚ न्यया॒ न्या ऽया॒नीती त्य॑या॒न्या ऽया॒नीति॑ । \newline
21. अ॒या॒नीती त्य॑या न्यया॒नीति॒ न नेत्य॑यान् यया॒नीति॒ न । \newline
22. इति॒ न नेतीति॒ नेहेह नेतीति॒ नेह । \newline
23. नेहेह न नेह लो॒को लो॒क इ॒ह न नेह लो॒कः । \newline
24. इ॒ह लो॒को लो॒क इ॒हेह लो॒को᳚ ऽस्त्यस्ति लो॒क इ॒हेह लो॒को᳚ ऽस्ति । \newline
25. लो॒को᳚ ऽस्त्यस्ति लो॒को लो॒को᳚ ऽस्तीती त्य॑स्ति लो॒को लो॒को᳚ ऽस्तीति॑ । \newline
26. अ॒स्तीती त्य॑स्त्य॒ स्तीत्य॑ब्रवी दब्रवी॒ दित्य॑ स्त्य॒ स्तीत्य॑ब्रवीत् । \newline
27. इत्य॑ब्रवी दब्रवी॒ दिती त्य॑ब्रवी॒थ् स सो᳚ ऽब्रवी॒ दिती त्य॑ब्रवी॒थ् सः । \newline
28. अ॒ब्र॒वी॒थ् स सो᳚ ऽब्रवी दब्रवी॒थ् स वि॒श्वक॑र्माणं ॅवि॒श्वक॑र्माणꣳ॒॒ सो᳚ ऽब्रवी दब्रवी॒थ् स वि॒श्वक॑र्माणम् । \newline
29. स वि॒श्वक॑र्माणं ॅवि॒श्वक॑र्माणꣳ॒॒ स स वि॒श्वक॑र्माण मब्रवी दब्रवीद् वि॒श्वक॑र्माणꣳ॒॒ स स वि॒श्वक॑र्माण मब्रवीत् । \newline
30. वि॒श्वक॑र्माण मब्रवी दब्रवीद् वि॒श्वक॑र्माणं ॅवि॒श्वक॑र्माण मब्रवी॒ दुपोपा᳚ ब्रवीद् वि॒श्वक॑र्माणं ॅवि॒श्वक॑र्माण मब्रवी॒ दुप॑ । \newline
31. वि॒श्वक॑र्माण॒मिति॑ वि॒श्व - क॒र्मा॒ण॒म् । \newline
32. अ॒ब्र॒वी॒ दुपोपा᳚ ब्रवी दब्रवी॒ दुप॑ त्वा॒ त्वोपा᳚ब्रवी दब्रवी॒ दुप॑ त्वा । \newline
33. उप॑ त्वा॒ त्वोपोप॒ त्वा ऽऽत्वोपोप॒ त्वा । \newline
34. त्वा ऽऽत्वा॒ त्वा ऽया᳚ न्यया॒ न्या त्वा॒ त्वा ऽया॑नि । \newline
35. आ ऽया᳚ न्यया॒ न्या ऽया॒नी तीत्य॑या॒ न्या ऽया॒नीति॑ । \newline
36. अ॒या॒नीती त्य॑या न्यया॒नीति॒ केन॒ केने त्य॑या न्यया॒नीति॒ केन॑ । \newline
37. इति॒ केन॒ केने तीति॒ केन॑ मा मा॒ केने तीति॒ केन॑ मा । \newline
38. केन॑ मा मा॒ केन॒ केन॑ मो॒पैष्य॑ स्यु॒पैष्य॑सि मा॒ केन॒ केन॑ मो॒पैष्य॑सि । \newline
39. मो॒पैष्य॑ स्यु॒पैष्य॑सि मा मो॒पैष्य॒ सीती त्यु॒पैष्य॑सि मा मो॒पैष्य॒सीति॑ । \newline
40. उ॒पैष्य॒सीती त्यु॒पैष्य॑ स्यु॒पैष्य॒सीति॒ दिश्या॑भि॒र् दिश्या॑भि॒ रित्यु॒पैष्य॑ स्यु॒पैष्य॒सीति॒ दिश्या॑भिः । \newline
41. उ॒पैष्य॒सीत्यु॑प - ऐष्य॑सि । \newline
42. इति॒ दिश्या॑भि॒र् दिश्या॑भि॒ रितीति॒ दिश्या॑भि॒ रितीति॒ दिश्या॑भि॒ रितीति॒ दिश्या॑भि॒ रिति॑ । \newline
43. दिश्या॑भि॒ रितीति॒ दिश्या॑भि॒र् दिश्या॑भि॒रि त्य॑ब्रवी दब्रवी॒ दिति॒ दिश्या॑भि॒र् दिश्या॑भि॒रि त्य॑ब्रवीत् । \newline
44. इत्य॑ब्रवी दब्रवी॒दिती त्य॑ब्रवी॒त् तम् त म॑ब्रवी॒दिती त्य॑ब्रवी॒त् तम् । \newline
45. अ॒ब्र॒वी॒त् तम् त म॑ब्रवी दब्रवी॒त् तम् दिश्या॑भि॒र् दिश्या॑भि॒ स्त म॑ब्रवी दब्रवी॒त् तम् दिश्या॑भिः । \newline
46. तम् दिश्या॑भि॒र् दिश्या॑भि॒ स्तम् तम् दिश्या॑भि रु॒पै दु॒पैद् दिश्या॑भि॒ स्तम् तम् दिश्या॑भि रु॒पैत् । \newline
47. दिश्या॑भि रु॒पै दु॒पैद् दिश्या॑भि॒र् दिश्या॑भि रु॒पैत् ता स्ता उ॒पैद् दिश्या॑भि॒र् दिश्या॑भि रु॒पैत् ताः । \newline
48. उ॒पैत् ता स्ता उ॒पै दु॒पैत् ता उपोप॒ ता उ॒पै दु॒पैत् ता उप॑ । \newline
49. उ॒पैदित्यु॑प - ऐत् । \newline
50. ता उपोप॒ ता स्ता उपा॑धत्ता ध॒त्तोप॒ ता स्ता उपा॑धत्त । \newline
51. उपा॑धत्ता ध॒त्तोपोपा॑ धत्त॒ ता स्ता अ॑ध॒त्तो पोपा॑ धत्त॒ ताः । \newline
52. अ॒ध॒त्त॒ ता स्ता अ॑धत्ता धत्त॒ ता दिशो॒ दिश॒ स्ता अ॑धत्ता धत्त॒ ता दिशः॑ । \newline
53. ता दिशो॒ दिश॒ स्ता स्ता दिशो॑ ऽभवन् नभव॒न् दिश॒ स्ता स्ता दिशो॑ ऽभवन्न् । \newline
54. दिशो॑ ऽभवन् नभव॒न् दिशो॒ दिशो॑ ऽभव॒न् थ्स सो॑ ऽभव॒न् दिशो॒ दिशो॑ ऽभव॒न् थ्सः । \newline
\pagebreak
\markright{ TS 5.7.5.5  \hfill https://www.vedavms.in \hfill}

\section{ TS 5.7.5.5 }

\textbf{TS 5.7.5.5 } \newline
\textbf{Samhita Paata} \newline

ऽभव॒न्थ् स प॑रमे॒ष्ठी प्र॒जाप॑तिमब्रवी॒दुप॒ त्वाऽऽ*या॒नीति॒ नेह लो॒को᳚ऽस्तीत्य॑ब्रवी॒थ् स वि॒श्वक॑र्माणं च य॒ज्ञ्ं चा᳚ब्रवी॒दुप॑ वा॒मा ऽया॒नीति॒ नेह लो॒को᳚ऽस्तीत्य॑ब्रूताꣳ॒॒ स ए॒तां तृ॒तीयां॒ चिति॑मपश्य॒त् तामुपा॑धत्त॒ तद॒साव॑भव॒थ् स आ॑दि॒त्यः प्र॒जाप॑ति-मब्रवी॒दुप॒ त्वा - [  ] \newline

\textbf{Pada Paata} \newline

अ॒भ॒व॒न्न् । सः । प॒र॒मे॒ष्ठी । प्र॒जाप॑ति॒मिति॑ प्र॒जा-प॒ति॒म् । अ॒ब्र॒वी॒त् । उपेति॑ । त्वा॒ । एति॑ । अ॒या॒नि॒ । इति॑ । न । इ॒ह । लो॒कः । अ॒स्ति॒ । इति॑ । अ॒ब्र॒वी॒त् । सः । वि॒श्वक॑र्माण॒मिति॑ वि॒श्व - क॒र्मा॒ण॒म् । च॒ । य॒ज्ञ्म् । च॒ । अ॒ब्र॒वी॒त् । उपेति॑ । वा॒म् । एति॑ । अ॒या॒नि॒ । इति॑ । न । इ॒ह । लो॒कः । अ॒स्ति॒ । इति॑ । अ॒ब्रू॒ता॒म् । सः । ए॒ताम् । तृ॒तीया᳚म् । चिति᳚म् । अ॒प॒श्य॒त् । ताम् । उपेति॑ । अ॒ध॒त्त॒ । तत् । अ॒सौ । अ॒भ॒व॒त् । सः । आ॒दि॒त्यः । प्र॒जाप॑ति॒मिति॑ प्र॒जा-प॒ति॒म् । अ॒ब्र॒वी॒त् । उपेति॑ । त्वा॒ ।  \newline


\textbf{Krama Paata} \newline

अ॒भ॒व॒न्थ् सः । स प॑रमे॒ष्ठी । प॒र॒मे॒ष्ठी प्र॒जाप॑तिम् । प्र॒जाप॑तिमब्रवीत् । प्र॒जाप॑ति॒मिति॑ प्र॒जा - प॒ति॒म् । अ॒ब्र॒वी॒दुप॑ । उप॑ त्वा । त्वा । आऽया॑नि । अ॒या॒नीति॑ । इति॒ न । नेह । इ॒ह लो॒कः । लो॒को᳚ऽस्ति । अ॒स्तीति॑ । इत्य॑ब्रवीत् । अ॒ब्र॒वी॒थ् सः । स वि॒श्वक॑र्माणम् । वि॒श्वक॑र्माणम् च । वि॒श्वक॑र्माण॒मिति॑ वि॒श्व - क॒र्मा॒ण॒म् । च॒ य॒ज्ञ्म् । य॒ज्ञ्म् च॑ । चा॒ब्र॒वी॒त्॒ । अ॒ब्र॒वी॒दुप॑ । उप॑ वाम् । वा॒मा । आऽया॑नि । अ॒या॒नीति॑ । इति॒ न । नेह । इ॒ह लो॒कः । लो॒को᳚ऽस्ति । अ॒स्तीति॑ । इत्य॑ब्रूताम् । अ॒ब्रू॒ताꣳ॒॒ सः । स ए॒ताम् । ए॒ताम् तृ॒तीया᳚म् । तृ॒तीया॒म् चिति᳚म् । चिति॑मपश्यत् । अ॒प॒श्य॒त् ताम् । तामुप॑ । उपा॑धत्त । अ॒ध॒त्त॒ तत् । तद॒सौ । अ॒साव॑भवत् । अ॒भ॒व॒थ् सः । स आ॑दि॒त्यः । आ॒दि॒त्यः प्र॒जाप॑तिम् । प्र॒जाप॑तिमब्रवित् । प्र॒जाप॑ति॒मिति॑ प्र॒जा - प॒ति॒म् । अ॒ब्र॒वी॒दुप॑ । उप॑ त्वा । त्वा \newline

\textbf{Jatai Paata} \newline

1. अ॒भ॒व॒न् थ्स सो॑ ऽभवन् नभव॒न् थ्सः । \newline
2. स प॑रमे॒ष्ठी प॑रमे॒ष्ठी स स प॑रमे॒ष्ठी । \newline
3. प॒र॒मे॒ष्ठी प्र॒जाप॑तिम् प्र॒जाप॑तिम् परमे॒ष्ठी प॑रमे॒ष्ठी प्र॒जाप॑तिम् । \newline
4. प्र॒जाप॑ति मब्रवी दब्रवीत् प्र॒जाप॑तिम् प्र॒जाप॑ति मब्रवीत् । \newline
5. प्र॒जाप॑ति॒मिति॑ प्र॒जा - प॒ति॒म् । \newline
6. अ॒ब्र॒वी॒ दुपोपा᳚ ब्रवी दब्रवी॒ दुप॑ । \newline
7. उप॑ त्वा॒ त्वोपोप॑ त्वा । \newline
8. त्वा ऽऽत्वा॒ त्वा । \newline
9. आ ऽया᳚ न्यया॒ न्याऽया॑नि । \newline
10. अ॒या॒नीती त्य॑या न्यया॒नीति॑ । \newline
11. इति॒ न ने तीति॒ न । \newline
12. नेहेह न नेह । \newline
13. इ॒ह लो॒को लो॒क इ॒हेह लो॒कः । \newline
14. लो॒को᳚ ऽस्त्यस्ति लो॒को लो॒को᳚ ऽस्ति । \newline
15. अ॒स्तीती त्य॑स्त्य॒ स्तीति॑ । \newline
16. इत्य॑ब्रवी दब्रवी॒ दिती त्य॑ब्रवीत् । \newline
17. अ॒ब्र॒वी॒थ् स सो᳚ ऽब्रवी दब्रवी॒थ् सः । \newline
18. स वि॒श्वक॑र्माणं ॅवि॒श्वक॑र्माणꣳ॒॒ स स वि॒श्वक॑र्माणम् । \newline
19. वि॒श्वक॑र्माणम् च च वि॒श्वक॑र्माणं ॅवि॒श्वक॑र्माणम् च । \newline
20. वि॒श्वक॑र्माण॒मिति॑ वि॒श्व - क॒र्मा॒ण॒म् । \newline
21. च॒ य॒ज्ञ्ं ॅय॒ज्ञ्म् च॑ च य॒ज्ञ्म् । \newline
22. य॒ज्ञ्म् च॑ च य॒ज्ञ्ं ॅय॒ज्ञ्म् च॑ । \newline
23. चा॒ब्र॒वी॒ द॒ब्र॒वी॒च् च॒ चा॒ब्र॒वी॒त् । \newline
24. अ॒ब्र॒वी॒ दुपोपा᳚ ब्रवी दब्रवी॒ दुप॑ । \newline
25. उप॑ वां ॅवा॒ मुपोप॑ वाम् । \newline
26. वा॒ मा वां᳚ ॅवा॒ मा । \newline
27. आ ऽया᳚ न्यया॒ न्याऽया॑नि । \newline
28. अ॒या॒नी तीत्य॑या न्यया॒नीति॑ । \newline
29. इति॒ न नेतीति॒ न । \newline
30. नेहेह न नेह । \newline
31. इ॒ह लो॒को लो॒क इ॒हेह लो॒कः । \newline
32. लो॒को᳚ ऽस्त्यस्ति लो॒को लो॒को᳚ ऽस्ति । \newline
33. अ॒स्ती तीत्य॑ स्त्य॒स्तीति॑ । \newline
34. इत्य॑ब्रूता मब्रूता॒ मिती त्य॑ब्रूताम् । \newline
35. अ॒ब्रू॒ताꣳ॒॒ स सो᳚ ऽब्रूता मब्रूताꣳ॒॒ सः । \newline
36. स ए॒ता मे॒ताꣳ स स ए॒ताम् । \newline
37. ए॒ताम् तृ॒तीया᳚म् तृ॒तीया॑ मे॒ता मे॒ताम् तृ॒तीया᳚म् । \newline
38. तृ॒तीया॒म् चिति॒म् चिति॑म् तृ॒तीया᳚म् तृ॒तीया॒म् चिति᳚म् । \newline
39. चिति॑ मपश्य दपश्य॒च् चिति॒म् चिति॑ मपश्यत् । \newline
40. अ॒प॒श्य॒त् ताम् ता म॑पश्य दपश्य॒त् ताम् । \newline
41. ता मुपोप॒ ताम् ता मुप॑ । \newline
42. उपा॑धत्ता ध॒त्तो पोपा॑ धत्त । \newline
43. अ॒ध॒त्त॒ तत् तद॑धत्ता धत्त॒ तत् । \newline
44. तद॒सा व॒सौ तत् तद॒सौ । \newline
45. अ॒सा व॑भव दभव द॒सा व॒सा व॑भवत् । \newline
46. अ॒भ॒व॒थ् स सो॑ ऽभव दभव॒थ् सः । \newline
47. स आ॑दि॒त्य आ॑दि॒त्यः स स आ॑दि॒त्यः । \newline
48. आ॒दि॒त्यः प्र॒जाप॑तिम् प्र॒जाप॑ति मादि॒त्य आ॑दि॒त्यः प्र॒जाप॑तिम् । \newline
49. प्र॒जाप॑ति मब्रवी दब्रवीत् प्र॒जाप॑तिम् प्र॒जाप॑ति मब्रवीत् । \newline
50. प्र॒जाप॑ति॒मिति॑ प्र॒जा - प॒ति॒म् । \newline
51. अ॒ब्र॒वी॒ दुपोपा᳚ ब्रवी दब्रवी॒ दुप॑ । \newline
52. उप॑ त्वा॒ त्वोपोप॑ त्वा । \newline
53. त्वा ऽऽत्वा॒ त्वा । \newline

\textbf{Ghana Paata } \newline

1. अ॒भ॒व॒न् थ्स सो॑ ऽभवन् नभव॒न् थ्स प॑रमे॒ष्ठी प॑रमे॒ष्ठी सो॑ ऽभवन् नभव॒न् थ्स प॑रमे॒ष्ठी । \newline
2. स प॑रमे॒ष्ठी प॑रमे॒ष्ठी स स प॑रमे॒ष्ठी प्र॒जाप॑तिम् प्र॒जाप॑तिम् परमे॒ष्ठी स स प॑रमे॒ष्ठी प्र॒जाप॑तिम् । \newline
3. प॒र॒मे॒ष्ठी प्र॒जाप॑तिम् प्र॒जाप॑तिम् परमे॒ष्ठी प॑रमे॒ष्ठी प्र॒जाप॑ति मब्रवी दब्रवीत् प्र॒जाप॑तिम् परमे॒ष्ठी प॑रमे॒ष्ठी प्र॒जाप॑ति मब्रवीत् । \newline
4. प्र॒जाप॑ति मब्रवी दब्रवीत् प्र॒जाप॑तिम् प्र॒जाप॑ति मब्रवी॒ दुपोपा᳚ ब्रवीत् प्र॒जाप॑तिम् प्र॒जाप॑ति मब्रवी॒ दुप॑ । \newline
5. प्र॒जाप॑ति॒मिति॑ प्र॒जा - प॒ति॒म् । \newline
6. अ॒ब्र॒वी॒ दुपोपा᳚ ब्रवी दब्रवी॒ दुप॑ त्वा॒ त्वोपा᳚ब्रवी दब्रवी॒ दुप॑ त्वा । \newline
7. उप॑ त्वा॒ त्वोपोप॒ त्वा ऽऽत्वो पोप॒ त्वा । \newline
8. त्वा ऽऽत्वा॒ त्वा ऽया᳚ न्यया॒ न्या त्वा॒ त्वा ऽया॑नि । \newline
9. आ ऽया᳚ न्यया॒ न्या ऽया॒नीती त्य॑या॒ न्या ऽया॒नीति॑ । \newline
10. अ॒या॒नी तीत्य॑या न्यया॒नीति॒ न नेत्य॑या न्यया॒नीति॒ न । \newline
11. इति॒ न नेतीति॒ नेहेह नेतीति॒ नेह । \newline
12. नेहेह न नेह लो॒को लो॒क इ॒ह न नेह लो॒कः । \newline
13. इ॒ह लो॒को लो॒क इ॒हेह लो॒को᳚ ऽस्त्यस्ति लो॒क इ॒हेह लो॒को᳚ ऽस्ति । \newline
14. लो॒को᳚ ऽस्त्यस्ति लो॒को लो॒को᳚ ऽस्तीती त्य॑स्ति लो॒को लो॒को᳚ ऽस्तीति॑ । \newline
15. अ॒स्तीती त्य॑स्त्य॒ स्तीत्य॑ब्रवी दब्रवी॒ दित्य॑ स्त्य॒ स्तीत्य॑ब्रवीत् । \newline
16. इत्य॑ब्रवी दब्रवी॒ दिती त्य॑ब्रवी॒थ् स सो᳚ ऽब्रवी॒ दिती त्य॑ब्रवी॒थ् सः । \newline
17. अ॒ब्र॒वी॒थ् स सो᳚ ऽब्रवी दब्रवी॒थ् स वि॒श्वक॑र्माणं ॅवि॒श्वक॑र्माणꣳ॒॒ सो᳚ ऽब्रवी दब्रवी॒थ् स वि॒श्वक॑र्माणम् । \newline
18. स वि॒श्वक॑र्माणं ॅवि॒श्वक॑र्माणꣳ॒॒ स स वि॒श्वक॑र्माणम् च च वि॒श्वक॑र्माणꣳ॒॒ स स वि॒श्वक॑र्माणम् च । \newline
19. वि॒श्वक॑र्माणम् च च वि॒श्वक॑र्माणं ॅवि॒श्वक॑र्माणम् च य॒ज्ञ्ं ॅय॒ज्ञ्म् च॑ वि॒श्वक॑र्माणं ॅवि॒श्वक॑र्माणम् च य॒ज्ञ्म् । \newline
20. वि॒श्वक॑र्माण॒मिति॑ वि॒श्व - क॒र्मा॒ण॒म् । \newline
21. च॒ य॒ज्ञ्ं ॅय॒ज्ञ्म् च॑ च य॒ज्ञ्म् च॑ च य॒ज्ञ्म् च॑ च य॒ज्ञ्म् च॑ । \newline
22. य॒ज्ञ्म् च॑ च य॒ज्ञ्ं ॅय॒ज्ञ्म् चा᳚ब्रवी दब्रवीच् च य॒ज्ञ्ं ॅय॒ज्ञ्म् चा᳚ब्रवीत् । \newline
23. चा॒ब्र॒वी॒ द॒ब्र॒वी॒च् च॒ चा॒ब्र॒वी॒ दुपोपा᳚ ब्रवीच् च चाब्रवी॒ दुप॑ । \newline
24. अ॒ब्र॒वी॒ दुपोपा᳚ ब्रवी दब्रवी॒ दुप॑ वां ॅवा॒ मुपा᳚ब्रवी दब्रवी॒ दुप॑ वाम् । \newline
25. उप॑ वां ॅवा॒ मुपोप॑ वा॒ मा वा॒ मुपोप॑ वा॒ मा । \newline
26. वा॒ मा वां᳚ ॅवा॒ मा ऽया᳚ न्यया॒ न्या वां᳚ ॅवा॒ मा ऽया॑नि । \newline
27. आ ऽया᳚ न्यया॒ न्याऽया॒नीती त्य॑या॒ न्याऽया॒नीति॑ । \newline
28. अ॒या॒नी तीत्य॑या न्यया॒नीति॒ न नेत्य॑या न्यया॒नीति॒ न । \newline
29. इति॒ न नेतीति॒ नेहेह नेतीति॒ नेह । \newline
30. नेहे ह न नेह लो॒को लो॒क इ॒ह न नेह लो॒कः । \newline
31. इ॒ह लो॒को लो॒क इ॒हेह लो॒को᳚ ऽस्त्यस्ति लो॒क इ॒हेह लो॒को᳚ ऽस्ति । \newline
32. लो॒को᳚ ऽस्त्यस्ति लो॒को लो॒को᳚ ऽस्तीती त्य॑स्ति लो॒को लो॒को᳚ ऽस्तीति॑ । \newline
33. अ॒स्तीती त्य॑स् त्य॒स्ती त्य॑ब्रूता मब्रूता॒ मित्य॑स् त्य॒स्ती त्य॑ब्रूताम् । \newline
34. इत्य॑ब्रूता मब्रूता॒ मिती त्य॑ब्रूताꣳ॒॒ स सो᳚ ऽब्रूता॒ मिती त्य॑ब्रूताꣳ॒॒ सः । \newline
35. अ॒ब्रू॒ताꣳ॒॒ स सो᳚ ऽब्रूता मब्रूताꣳ॒॒ स ए॒ता मे॒ताꣳ सो᳚ ऽब्रूता मब्रूताꣳ॒॒ स ए॒ताम् । \newline
36. स ए॒ता मे॒ताꣳ स स ए॒ताम् तृ॒तीया᳚म् तृ॒तीया॑ मे॒ताꣳ स स ए॒ताम् तृ॒तीया᳚म् । \newline
37. ए॒ताम् तृ॒तीया᳚म् तृ॒तीया॑ मे॒ता मे॒ताम् तृ॒तीया॒म् चिति॒म् चिति॑म् तृ॒तीया॑ मे॒ता मे॒ताम् तृ॒तीया॒म् चिति᳚म् । \newline
38. तृ॒तीया॒म् चिति॒म् चिति॑म् तृ॒तीया᳚म् तृ॒तीया॒म् चिति॑ मपश्य दपश्य॒च् चिति॑म् तृ॒तीया᳚म् तृ॒तीया॒म् चिति॑ मपश्यत् । \newline
39. चिति॑ मपश्य दपश्य॒च् चिति॒म् चिति॑ मपश्य॒त् ताम् ता म॑पश्य॒च् चिति॒म् चिति॑ मपश्य॒त् ताम् । \newline
40. अ॒प॒श्य॒त् ताम् ता म॑पश्य दपश्य॒त् ता मुपोप॒ ता म॑पश्य दपश्य॒त् ता मुप॑ । \newline
41. ता मुपोप॒ ताम् ता मुपा॑धत्ता ध॒त्तोप॒ ताम् ता मुपा॑धत्त । \newline
42. उपा॑धत्ता ध॒त्तो पोपा॑ धत्त॒ तत् तद॑ध॒त्तो पोपा॑ धत्त॒ तत् । \newline
43. अ॒ध॒त्त॒ तत् तद॑धत्ता धत्त॒ तद॒सा व॒सौ तद॑धत्ता धत्त॒ तद॒सौ । \newline
44. तद॒सा व॒सौ तत् तद॒सा व॑भव दभव द॒सौ तत् तद॒सा व॑भवत् । \newline
45. अ॒सा व॑भव दभव द॒सा व॒सा व॑भव॒थ् स सो॑ ऽभवद॒सा व॒सा व॑भव॒थ् सः । \newline
46. अ॒भ॒व॒थ् स सो॑ ऽभव दभव॒थ् स आ॑दि॒त्य आ॑दि॒त्यः सो॑ ऽभव दभव॒थ् स आ॑दि॒त्यः । \newline
47. स आ॑दि॒त्य आ॑दि॒त्यः स स आ॑दि॒त्यः प्र॒जाप॑तिम् प्र॒जाप॑ति मादि॒त्यः स स आ॑दि॒त्यः प्र॒जाप॑तिम् । \newline
48. आ॒दि॒त्यः प्र॒जाप॑तिम् प्र॒जाप॑ति मादि॒त्य आ॑दि॒त्यः प्र॒जाप॑ति मब्रवी दब्रवीत् प्र॒जाप॑ति मादि॒त्य आ॑दि॒त्यः प्र॒जाप॑ति मब्रवीत् । \newline
49. प्र॒जाप॑ति मब्रवी दब्रवीत् प्र॒जाप॑तिम् प्र॒जाप॑ति मब्रवी॒ दुपोपा᳚ ब्रवीत् प्र॒जाप॑तिम् प्र॒जाप॑ति मब्रवी॒ दुप॑ । \newline
50. प्र॒जाप॑ति॒मिति॑ प्र॒जा - प॒ति॒म् । \newline
51. अ॒ब्र॒वी॒ दुपोपा᳚ ब्रवी दब्रवी॒ दुप॑ त्वा॒ त्वोपा᳚ ब्रवी दब्रवी॒ दुप॑ त्वा । \newline
52. उप॑ त्वा॒ त्वोपोप॒ त्वा ऽऽत्वो पोप॒ त्वा । \newline
53. त्वा ऽऽत्वा॒ त्वा ऽया᳚ न्यया॒ न्यात्वा॒ त्वा ऽया॑नि । \newline
\pagebreak
\markright{ TS 5.7.5.6  \hfill https://www.vedavms.in \hfill}

\section{ TS 5.7.5.6 }

\textbf{TS 5.7.5.6 } \newline
\textbf{Samhita Paata} \newline

ऽऽया॒नीति॒ नेह लो॒को᳚ऽस्तीत्य॑ब्रवी॒थ् स वि॒श्वक॑र्माणं च य॒ज्ञ्ं चा᳚ब्रवी॒दुप॑ वा॒माऽया॒नीति॒ नेह लो॒को᳚ऽस्तीत्य॑ब्रूताꣳ॒॒ स प॑रमे॒ष्ठिन॑मब्रवी॒दुप॒ त्वाऽऽया॒नीति॒ केन॑ मो॒पैष्य॒सीति॑ लोकं पृ॒णयेत्य॑ब्रवी॒त् तं ॅलो॑कं पृ॒णयो॒पैत् तस्मा॒दया॑तयाम्नी लोकं पृ॒णाऽया॑तयामा॒ ह्य॑सा - [  ] \newline

\textbf{Pada Paata} \newline

एति॑ । अ॒या॒नि॒ । इति॑ । न । इ॒ह । लो॒कः । अ॒स्ति॒ । इति॑ । अ॒ब्र॒वी॒त् । सः । वि॒श्वक॑र्माण॒मिति॑ वि॒श्व - क॒र्मा॒ण॒म् । च॒ । य॒ज्ञ्म् । च॒ । अ॒ब्र॒वी॒त् । उपेति॑ । वा॒म् । एति॑ । अ॒या॒नि॒ । इति॑ । न । इ॒ह । लो॒कः । अ॒स्ति॒ । इति॑ । अ॒ब्रू॒ता॒म् । सः । प॒र॒मे॒ष्ठिन᳚म् । अ॒ब्र॒वी॒त् । उपेति॑ । त्वा॒ । एति॑ । अ॒या॒नि॒ । इति॑ । केन॑ । मा॒ । उ॒पैष्य॒सीत्यु॑प - ऐष्य॑सि । इति॑ । लो॒कं॒पृ॒णयेति॑ लोकं - पृ॒णया᳚ । इति॑ । अ॒ब्र॒वी॒त् । तम् । लो॒क॒पृं॒णयेति॑ लोकं - पृ॒णया᳚ । उ॒पैदित्यु॑प - ऐत् । तस्मा᳚त् । अया॑तया॒म्नीत्यया॑त - या॒म्नी॒ । लो॒क॒पृं॒णेति॑ लोकं - पृ॒णा । अया॑तया॒मेत्यया॑त - या॒मा॒ । हि । अ॒सौ ।  \newline


\textbf{Krama Paata} \newline

आऽया॑नि । अ॒या॒नीति॑ । इति॒ न । नेह । इ॒ह लो॒कः । लो॒को᳚ऽस्ति । अ॒स्तीति॑ । इत्य॑ब्रवीत् । अ॒ब्र॒वी॒थ् सः । स वि॒श्वक॑र्माणम् । वि॒श्वक॑र्माणम् च । वि॒श्वक॑र्माण॒मिति॑ वि॒श्व - क॒र्मा॒ण॒म् । च॒ य॒ज्ञ्म् । य॒ज्ञ्म् च॑ । चा॒ब्र॒वी॒त्॒ । अ॒ब्र॒वी॒दुप॑ । उप॑ वाम् । वा॒मा । आऽया॑नि । अ॒या॒नीति॑ । इति॒ न । नेह । इ॒ह लो॒कः । लो॒को᳚ऽस्ति । अ॒स्तीति॑ । इत्य॑ब्रूताम् । अ॒ब्रू॒ताꣳ॒॒ सः । स प॑रमे॒ष्ठिन᳚म् । प॒र॒मे॒ष्ठिन॑मब्रवीत् । 
अ॒ब्र॒वी॒दुप॑ । उप॑ त्वा । त्वा । आऽया॑नि । अ॒या॒नीति॑ । इति॒ केन॑ । केन॑ मा । मो॒पैष्य॑सि । उ॒पैष्य॒सीति॑ । उ॒पैष्य॒सीत्यु॑प - ऐष्य॑सि । इति॑ लोकम्पृ॒णया᳚ । लो॒क॒म्पृ॒णयेति॑ । लो॒क॒म्पृ॒णयेति॑ लोकम् - पृ॒णया᳚ । इत्य॑ब्रवीत् । अ॒ब्र॒वी॒त् तम् । तम् ॅलो॑कम्पृ॒णया᳚ । लो॒क॒म्पृ॒णयो॒पैत् । लो॒क॒म्पृ॒णयेति॑ लोकम् - पृ॒णया᳚ । उ॒पैत् तस्मा᳚त् । उ॒पैदित्यु॑प - ऐत् । तस्मा॒दया॑तयाम्नी । अया॑तयाम्नी लोकम्पृ॒णा । अया॑तया॒म्नीत्यया॑त - या॒म्नी॒ । लो॒क॒म्पृ॒णाऽया॑तयामा । लो॒क॒म्पृ॒णेति॑ लोकम् - पृ॒णा । अया॑तयामा॒ हि । अया॑तया॒मेत्यया॑त - या॒मा॒ । ह्य॑सौ ( ) । अ॒सावा॑दि॒त्यः \newline

\textbf{Jatai Paata} \newline

1. आ ऽया᳚ न्यया॒ न्याऽया॑नि । \newline
2. अ॒या॒नीती त्य॑या न्यया॒नीति॑ । \newline
3. इति॒ न नेतीति॒ न । \newline
4. नेहेह न नेह । \newline
5. इ॒ह लो॒को लो॒क इ॒हेह लो॒कः । \newline
6. लो॒को᳚ ऽस्त्यस्ति लो॒को लो॒को᳚ ऽस्ति । \newline
7. अ॒स्ती तीत्य॑ स्त्य॒ स्तीति॑ । \newline
8. इत्य॑ब्रवी दब्रवी॒ दिती त्य॑ब्रवीत् । \newline
9. अ॒ब्र॒वी॒थ् स सो᳚ ऽब्रवी दब्रवी॒थ् सः । \newline
10. स वि॒श्वक॑र्माणं ॅवि॒श्वक॑र्माणꣳ॒॒ स स वि॒श्वक॑र्माणम् । \newline
11. वि॒श्वक॑र्माणम् च च वि॒श्वक॑र्माणं ॅवि॒श्वक॑र्माणम् च । \newline
12. वि॒श्वक॑र्माण॒मिति॑ वि॒श्व - क॒र्मा॒ण॒म् । \newline
13. च॒ य॒ज्ञ्ं ॅय॒ज्ञ्म् च॑ च य॒ज्ञ्म् । \newline
14. य॒ज्ञ्म् च॑ च य॒ज्ञ्ं ॅय॒ज्ञ्म् च॑ । \newline
15. चा॒ब्र॒वी॒ द॒ब्र॒वी॒च् च॒ चा॒ब्र॒वी॒त् । \newline
16. अ॒ब्र॒वी॒ दुपोपा᳚ ब्रवी दब्रवी॒ दुप॑ । \newline
17. उप॑ वां ॅवा॒ मुपोप॑ वाम् । \newline
18. वा॒ मा वां᳚ ॅवा॒ मा । \newline
19. आ ऽया᳚ न्यया॒ न्याऽया॑नि । \newline
20. अ॒या॒नीती त्य॑या न्यया॒नीति॑ । \newline
21. इति॒ न नेतीति॒ न । \newline
22. नेहेह न नेह । \newline
23. इ॒ह लो॒को लो॒क इ॒हेह लो॒कः । \newline
24. लो॒को᳚ ऽस्त्यस्ति लो॒को लो॒को᳚ ऽस्ति । \newline
25. अ॒स्ती तीत्य॑ स्त्य॒स्तीति॑ । \newline
26. इत्य॑ब्रूता मब्रूता॒ मिती त्य॑ब्रूताम् । \newline
27. अ॒ब्रू॒ताꣳ॒॒ स सो᳚ ऽब्रूता मब्रूताꣳ॒॒ सः । \newline
28. स प॑रमे॒ष्ठिन॑म् परमे॒ष्ठिनꣳ॒॒ स स प॑रमे॒ष्ठिन᳚म् । \newline
29. प॒र॒मे॒ष्ठिन॑ मब्रवी दब्रवीत् परमे॒ष्ठिन॑म् परमे॒ष्ठिन॑ मब्रवीत् । \newline
30. अ॒ब्र॒वी॒ दुपोपा᳚ ब्रवी दब्रवी॒ दुप॑ । \newline
31. उप॑ त्वा॒ त्वोपोप॑ त्वा । \newline
32. त्वा ऽऽत्वा॒ त्वा । \newline
33. आ ऽया᳚ न्यया॒ न्याऽया॑नि । \newline
34. अ॒या॒नीती त्य॑या न्यया॒नीति॑ । \newline
35. इति॒ केन॒ केने तीति॒ केन॑ । \newline
36. केन॑ मा मा॒ केन॒ केन॑ मा । \newline
37. मो॒पैष्य॑ स्यु॒पैष्य॑सि मा मो॒पैष्य॑सि । \newline
38. उ॒पैष्य॒ सीती त्यु॒पैष्य॑ स्यु॒पैष्य॒ सीति॑ । \newline
39. उ॒पैष्य॒सीत्यु॑प - ऐष्य॑सि । \newline
40. इति॑ लोकंपृ॒णया॑ लोकंपृ॒ण येतीति॑ लोकंपृ॒णया᳚ । \newline
41. लो॒कं॒पृ॒ण येतीति॑ लोकंपृ॒णया॑ लोकंपृ॒ण येति॑ । \newline
42. लो॒कं॒पृ॒णयेति॑ लोकं - पृ॒णया᳚ । \newline
43. इत्य॑ब्रवी दब्रवी॒ दिती त्य॑ब्रवीत् । \newline
44. अ॒ब्र॒वी॒त् तम् त म॑ब्रवी दब्रवी॒त् तम् । \newline
45. तम् ॅलो॑कंपृ॒णया॑ लोकंपृ॒णया॒ तम् तम् ॅलो॑कंपृ॒णया᳚ । \newline
46. लो॒कं॒पृ॒ण यो॒पै दु॒पै ल्लो॑कंपृ॒णया॑ लोकंपृ॒ण यो॒पैत् । \newline
47. लो॒कं॒पृ॒णयेति॑ लोकं - पृ॒णया᳚ । \newline
48. उ॒पैत् तस्मा॒त् तस्मा॑ दु॒पै दु॒पैत् तस्मा᳚त् । \newline
49. उ॒पैदित्यु॑प - ऐत् । \newline
50. तस्मा॒ दया॑तया॒ म्न्यया॑तयाम्नी॒ तस्मा॒त् तस्मा॒ दया॑तयाम्नी । \newline
51. अया॑तयाम्नी लोकंपृ॒णा लो॑कंपृ॒णा ऽया॑तया॒ म्न्यया॑तयाम्नी लोकंपृ॒णा । \newline
52. अया॑तया॒म्नीत्यया॑त - या॒म्नी॒ । \newline
53. लो॒कं॒पृ॒णा ऽया॑तया॒मा ऽया॑तयामा लोकंपृ॒णा लो॑कंपृ॒णा ऽया॑तयामा । \newline
54. लो॒कं॒पृ॒णेति॑ लोकं - पृ॒णा । \newline
55. अया॑तयामा॒ हि ह्यया॑तया॒मा ऽया॑तयामा॒ हि । \newline
56. अया॑तया॒मेत्यया॑त - या॒मा॒ । \newline
57. ह्य॑सा व॒सौ हि ह्य॑सौ । \newline
58. अ॒सा वा॑दि॒त्य आ॑दि॒त्यो॑ ऽसा व॒सा वा॑दि॒त्यः । \newline

\textbf{Ghana Paata } \newline

1. आ ऽया᳚न्य या॒ न्याऽया॒नीती त्य॑या॒ न्याऽया॒नीति॑ । \newline
2. अ॒या॒नीती त्य॑या न्यया॒नीति॒ न नेत्य॑ या न्यया॒नीति॒ न । \newline
3. इति॒ न नेतीति॒ नेहेह नेतीति॒ नेह । \newline
4. नेहेह न नेह लो॒को लो॒क इ॒ह न नेह लो॒कः । \newline
5. इ॒ह लो॒को लो॒क इ॒हेह लो॒को᳚ ऽस्त्यस्ति लो॒क इ॒हेह लो॒को᳚ ऽस्ति । \newline
6. लो॒को᳚ ऽस्त्यस्ति लो॒को लो॒को᳚ ऽस्तीती त्य॑स्ति लो॒को लो॒को᳚ ऽस्तीति॑ । \newline
7. अ॒स्तीती त्य॑स्त्य॒ स्तीत्य॑ब्रवी दब्रवी॒ दित्य॑ स्त्य॒ स्तीत्य॑ब्रवीत् । \newline
8. इत्य॑ब्रवी दब्रवी॒दिती त्य॑ब्रवी॒थ् स सो᳚ ऽब्रवी॒दिती त्य॑ब्रवी॒थ् सः । \newline
9. अ॒ब्र॒वी॒थ् स सो᳚ ऽब्रवी दब्रवी॒थ् स वि॒श्वक॑र्माणं ॅवि॒श्वक॑र्माणꣳ॒॒ सो᳚ ऽब्रवी दब्रवी॒थ् स वि॒श्वक॑र्माणम् । \newline
10. स वि॒श्वक॑र्माणं ॅवि॒श्वक॑र्माणꣳ॒॒ स स वि॒श्वक॑र्माणम् च च वि॒श्वक॑र्माणꣳ॒॒ स स वि॒श्वक॑र्माणम् च । \newline
11. वि॒श्वक॑र्माणम् च च वि॒श्वक॑र्माणं ॅवि॒श्वक॑र्माणम् च य॒ज्ञ्ं ॅय॒ज्ञ्म् च॑ वि॒श्वक॑र्माणं ॅवि॒श्वक॑र्माणम् च य॒ज्ञ्म् । \newline
12. वि॒श्वक॑र्माण॒मिति॑ वि॒श्व - क॒र्मा॒ण॒म् । \newline
13. च॒ य॒ज्ञ्ं ॅय॒ज्ञ्म् च॑ च य॒ज्ञ्म् च॑ च य॒ज्ञ्म् च॑ च य॒ज्ञ्म् च॑ । \newline
14. य॒ज्ञ्म् च॑ च य॒ज्ञ्ं ॅय॒ज्ञ्म् चा᳚ब्रवी दब्रवीच् च य॒ज्ञ्ं ॅय॒ज्ञ्म् चा᳚ब्रवीत् । \newline
15. चा॒ब्र॒वी॒ द॒ब्र॒वी॒च् च॒ चा॒ब्र॒वी॒ दुपोपा᳚ब्रवीच् च चाब्रवी॒ दुप॑ । \newline
16. अ॒ब्र॒वी॒ दुपोपा᳚ ब्रवी दब्रवी॒ दुप॑ वां ॅवा॒ मुपा᳚ब्रवी दब्रवी॒ दुप॑ वाम् । \newline
17. उप॑ वां ॅवा॒ मुपोप॑ वा॒ मा वा॒ मुपोप॑ वा॒ मा । \newline
18. वा॒ मा वां᳚ ॅवा॒ मा ऽया᳚ न्यया॒ न्या वां᳚ ॅवा॒ मा ऽया॑नि । \newline
19. आ ऽया᳚न्य या॒न्या ऽया॒नी तीत्य॑या॒ न्याऽया॒नीति॑ । \newline
20. अ॒या॒नी तीत्य॑या न्यया॒नीति॒ न नेत्य॑या न्यया॒नीति॒ न । \newline
21. इति॒ न नेतीति॒ नेहेह नेतीति॒ नेह । \newline
22. नेहेह न नेह लो॒को लो॒क इ॒ह न नेह लो॒कः । \newline
23. इ॒ह लो॒को लो॒क इ॒हेह लो॒को᳚ ऽस्त्यस्ति लो॒क इ॒हेह लो॒को᳚ ऽस्ति । \newline
24. लो॒को᳚ ऽस्त्यस्ति लो॒को लो॒को᳚ ऽस्तीती त्य॑स्ति लो॒को लो॒को᳚ ऽस्तीति॑ । \newline
25. अ॒स्तीती त्य॑स्त्य॒स्ती त्य॑ब्रूता मब्रूता॒ मित्य॑ स्त्य॒स्ती त्य॑ब्रूताम् । \newline
26. इत्य॑ब्रूता मब्रूता॒ मिती त्य॑ब्रूताꣳ॒॒ स सो᳚ ऽब्रूता॒ मिती त्य॑ब्रूताꣳ॒॒ सः । \newline
27. अ॒ब्रू॒ताꣳ॒॒ स सो᳚ ऽब्रूता मब्रूताꣳ॒॒ स प॑रमे॒ष्ठिन॑म् परमे॒ष्ठिनꣳ॒॒ सो᳚ ऽब्रूता मब्रूताꣳ॒॒ स प॑रमे॒ष्ठिन᳚म् । \newline
28. स प॑रमे॒ष्ठिन॑म् परमे॒ष्ठिनꣳ॒॒ स स प॑रमे॒ष्ठिन॑ मब्रवी दब्रवीत् परमे॒ष्ठिनꣳ॒॒ स स प॑रमे॒ष्ठिन॑ मब्रवीत् । \newline
29. प॒र॒मे॒ष्ठिन॑ मब्रवी दब्रवीत् परमे॒ष्ठिन॑म् परमे॒ष्ठिन॑ मब्रवी॒ दुपोपा᳚ ब्रवीत् परमे॒ष्ठिन॑म् परमे॒ष्ठिन॑ मब्रवी॒ दुप॑ । \newline
30. अ॒ब्र॒वी॒ दुपोपा᳚ ब्रवी दब्रवी॒ दुप॑ त्वा॒ त्वोपा᳚ ब्रवी दब्रवी॒ दुप॑ त्वा । \newline
31. उप॑ त्वा॒ त्वोपोप॒ त्वा ऽऽत्वो पोप॒ त्वा । \newline
32. त्वा ऽऽत्वा॒ त्वा ऽया᳚ न्यया॒ न्या त्वा॒ त्वा ऽया॑नि । \newline
33. आ ऽया᳚ न्यया॒ न्याऽया॒नी तीत्य॑ या॒ न्याऽया॒नीति॑ । \newline
34. अ॒या॒नी तीत्य॑या न्यया॒नीति॒ केन॒ केने त्य॑या न्यया॒नीति॒ केन॑ । \newline
35. इति॒ केन॒ केने तीति॒ केन॑ मा मा॒ केने तीति॒ केन॑ मा । \newline
36. केन॑ मा मा॒ केन॒ केन॑ मो॒पैष्य॑ स्यु॒पैष्य॑सि मा॒ केन॒ केन॑ मो॒पैष्य॑सि । \newline
37. मो॒पैष्य॑ स्यु॒पैष्य॑सि मा मो॒पैष्य॒सीती त्यु॒पैष्य॑सि मा मो॒पैष्य॒सीति॑ । \newline
38. उ॒पैष्य॒सीती त्यु॒पैष्य॑ स्यु॒पैष्य॒सीति॑ लोकंपृ॒णया॑ लोकंपृ॒ण येत्यु॒पैष्य॑ स्यु॒पैष्य॒सीति॑ लोकंपृ॒णया᳚ । \newline
39. उ॒पैष्य॒सीत्यु॑प - ऐष्य॑सि । \newline
40. इति॑ लोकंपृ॒णया॑ लोकंपृ॒णयेतीति॑ लोकंपृ॒णयेतीति॑ लोकंपृ॒णयेतीति॑ लोकंपृ॒णयेति॑ । \newline
41. लो॒कं॒पृ॒णयेतीति॑ लोकंपृ॒णया॑ लोकंपृ॒णये त्य॑ब्रवी दब्रवी॒ दिति॑ लोकंपृ॒णया॑ लोकंपृ॒णये त्य॑ब्रवीत् । \newline
42. लो॒कं॒पृ॒णयेति॑ लोकं - पृ॒णया᳚ । \newline
43. इत्य॑ब्रवी दब्रवी॒ दिती त्य॑ब्रवी॒त् तम् त म॑ब्रवी॒ दिती त्य॑ब्रवी॒त् तम् । \newline
44. अ॒ब्र॒वी॒त् तम् त म॑ब्रवी दब्रवी॒त् तम् ॅलो॑कंपृ॒णया॑ लोकंपृ॒णया॒ त म॑ब्र वीदब्रवी॒त् तम् ॅलो॑कंपृ॒णया᳚ । \newline
45. तम् ॅलो॑कंपृ॒णया॑ लोकंपृ॒णया॒ तम् तम् ॅलो॑कंपृ॒णयो॒ पैदु॒पै ल्लो॑कंपृ॒णया॒ तम् तम् ॅलो॑कंपृ॒णयो॒पैत् । \newline
46. लो॒कं॒पृ॒णयो॒ पैदु॒पै ल्लो॑कंपृ॒णया॑ लोकंपृ॒णयो॒पैत् तस्मा॒त् तस्मा॑ दु॒पै ल्लो॑कंपृ॒णया॑ लोकंपृ॒णयो॒पैत् तस्मा᳚त् । \newline
47. लो॒कं॒पृ॒णयेति॑ लोकं - पृ॒णया᳚ । \newline
48. उ॒पैत् तस्मा॒त् तस्मा॑ दु॒पै दु॒पैत् तस्मा॒ दया॑तया॒ म्न्यया॑तयाम्नी॒ तस्मा॑ दु॒पै दु॒पैत् तस्मा॒ दया॑तयाम्नी । \newline
49. उ॒पैदित्यु॑प - ऐत् । \newline
50. तस्मा॒ दया॑तया॒ म्न्यया॑तयाम्नी॒ तस्मा॒त् तस्मा॒ दया॑तयाम्नी लोकंपृ॒णा लो॑कंपृ॒णा ऽया॑तयाम्नी॒ तस्मा॒त् तस्मा॒ दया॑तयाम्नी लोकंपृ॒णा । \newline
51. अया॑तयाम्नी लोकंपृ॒णा लो॑कंपृ॒णा ऽया॑तया॒ म्न्यया॑तयाम्नी लोकंपृ॒णा ऽया॑तया॒मा ऽया॑तयामा लोकंपृ॒णा ऽया॑तया॒म्न्य या॑तयाम्नी लोकंपृ॒णा ऽया॑तयामा । \newline
52. अया॑तया॒म्नीत्यया॑त - या॒म्नी॒ । \newline
53. लो॒कं॒पृ॒णा ऽया॑तया॒मा ऽया॑तयामा लोकंपृ॒णा लो॑कंपृ॒णा ऽया॑तयामा॒ हि ह्यया॑तयामा लोकंपृ॒णा लो॑कंपृ॒णा ऽया॑तयामा॒ हि । \newline
54. लो॒कं॒पृ॒णेति॑ लोकं - पृ॒णा । \newline
55. अया॑तयामा॒ हि ह्यया॑तया॒मा ऽया॑तयामा॒ ह्य॑सा व॒सौ ह्यया॑तया॒मा ऽया॑तयामा॒ ह्य॑सौ । \newline
56. अया॑तया॒मेत्यया॑त - या॒मा॒ । \newline
57. ह्य॑सा व॒सौ हि ह्य॑सा वा॑दि॒त्य आ॑दि॒त्यो॑ ऽसौ हि ह्य॑सा वा॑दि॒त्यः । \newline
58. अ॒सा वा॑दि॒त्य आ॑दि॒त्यो॑ ऽसा व॒सा वा॑दि॒त्य स्ताꣳ स्ताना॑दि॒त्यो॑ ऽसा व॒सा वा॑दि॒त्य स्तान् । \newline
\pagebreak
\markright{ TS 5.7.5.7  \hfill https://www.vedavms.in \hfill}

\section{ TS 5.7.5.7 }

\textbf{TS 5.7.5.7 } \newline
\textbf{Samhita Paata} \newline

-वा॑दि॒त्यस्तानृष॑यो ऽब्रुव॒न्नुप॑ व॒ आऽया॒मेति॒ केन॑ न उ॒पैष्य॒थेति॑ भू॒म्नेत्य॑ब्रुव॒न् तान् द्वाभ्यां॒ चिती᳚भ्यामु॒पाय॒न्थ् स पञ्च॑चितीकः॒ सम॑पद्यत॒ य ए॒वं ॅवि॒द्वान॒ग्निं चि॑नु॒ते भूया॑ने॒व भ॑वत्य॒भीमान् ॅलो॒काञ्ज॑यति वि॒दुरे॑नं दे॒वा अथो॑ ए॒तासा॑मे॒व दे॒वता॑नाꣳ॒॒ सायु॑ज्यं गच्छति ॥ \newline

\textbf{Pada Paata} \newline

आ॒दि॒त्यः । तान् । ऋष॑यः । अ॒ब्रु॒व॒न्न् । उपेति॑ । वः॒ । एति॑ । अ॒या॒म॒ । इति॑ । केन॑ । नः॒ । उ॒पैष्य॒थेत्यु॑प - ऐष्य॑थ । इति॑ । भू॒म्ना । इति॑ । अ॒ब्रु॒व॒न्न् । तान् । द्वाभ्या᳚म् । चिती᳚भ्या॒मिति॑ चिति॑ - भ्या॒म् । उ॒पाय॒न्नित्यु॑प - आयन्न्॑ । सः । पञ्च॑चितीक॒ इति॒ पञ्च॑-चि॒ती॒कः॒ । समिति॑ । अ॒प॒द्य॒त॒ । यः । ए॒वम् । वि॒द्वान् । अ॒ग्निम् । चि॒नु॒ते । भूयान्॑ । ए॒व । भ॒व॒ति॒ । अ॒भीति॑ । इ॒मान् । लो॒कान् । ज॒य॒ति॒ । वि॒दुः । ए॒न॒म् । दे॒वाः । अथो॒ इति॑ । ए॒तासा᳚म् । ए॒व । दे॒वता॑नाम् । सायु॑ज्यम् । ग॒च्छ॒ति॒ ॥  \newline


\textbf{Krama Paata} \newline

आ॒दि॒त्यस्तान् । तानृष॑यः । ऋष॑योऽब्रुवन्न् । अ॒ब्रु॒व॒न्नुप॑ । उप॑ वः । व॒ आ । आऽया॑म । अ॒या॒मेति॑ । इति॒ केन॑ । केन॑ नः । न॒ उ॒पैष्य॑थ । उ॒पैष्य॒थेति॑ । उ॒पैष्य॒थेत्यु॑प - ऐष्य॑थ । इति॑ भू॒म्ना । भू॒म्नेति॑ । इत्य॑ब्रुवन्न् । अ॒ब्रु॒व॒न् तान् । तान् द्वाभ्या᳚म् । द्वाभ्या॒म् चिती᳚भ्याम् । चिती᳚भ्यामु॒पायन्न्॑ । चिती᳚भ्या॒मिति॒ चिति॑ - भ्या॒म् । उ॒पाय॒न्थ् सः । उ॒पाय॒न्नित्यु॑प - आयन्न्॑ । स पञ्च॑चितीकः । पञ्च॑चितीकः॒ सम् । पञ्च॑चितीक॒ इति॒ पञ्च॑ - चि॒ती॒कः॒ । सम॑पद्यत । अ॒प॒द्य॒त॒ यः । य ए॒वम् । ए॒वम् ॅवि॒द्वान् । वि॒द्वान॒ग्निम् । अ॒ग्निम् चि॑नु॒ते । चि॒नु॒ते भूयान्॑ । भूया॑ने॒व । ए॒व भ॑वति । भ॒व॒त्य॒भि । अ॒भीमान् । इ॒मान् ॅलो॒कान् । लो॒कान् ज॑यति । ज॒य॒ति॒ वि॒दुः । वि॒दुरे॑नम् । ए॒न॒म् दे॒वाः । दे॒वा अथो᳚ । अथो॑ ए॒तासा᳚म् । अथो॒ इत्यथो᳚ । ए॒तासा॑मे॒व । ए॒व दे॒वता॑नाम् । दे॒वता॑नाꣳ॒॒ सायु॑ज्यम् । सायु॑ज्यम् गच्छति । ग॒च्छ॒तीति॑ गच्छति । \newline

\textbf{Jatai Paata} \newline

1. आ॒दि॒त्य स्ताꣳ स्ता ना॑दि॒त्य आ॑दि॒त्य स्तान् । \newline
2. तानृष॑य॒ ऋष॑य॒ स्ताꣳ स्ता नृष॑यः । \newline
3. ऋष॑यो ऽब्रुवन् नब्रुव॒न् नृष॑य॒ ऋष॑यो ऽब्रुवन्न् । \newline
4. अ॒ब्रु॒व॒न् नुपोपा᳚ ब्रुवन् नब्रुव॒न् नुप॑ । \newline
5. उप॑ वो व॒ उपोप॑ वः । \newline
6. व॒ आ वो॑ व॒ आ । \newline
7. आ ऽया॑मा या॒मा ऽया॑म । \newline
8. अ॒या॒मे तीत्य॑यामा या॒मेति॑ । \newline
9. इति॒ केन॒ केने तीति॒ केन॑ । \newline
10. केन॑ नो नः॒ केन॒ केन॑ नः । \newline
11. न॒ उ॒पैष्य॑ थो॒पैष्य॑थ नो न उ॒पैष्य॑थ । \newline
12. उ॒पैष्य॒थेती त्यु॒पैष्य॑ थो॒पैष्य॒थेति॑ । \newline
13. उ॒पैष्य॒थेत्यु॑प - ऐष्य॑थ । \newline
14. इति॑ भू॒म्ना भू॒म्नेतीति॑ भू॒म्ना । \newline
15. भू॒म्नेतीति॑ भू॒म्ना भू॒म्नेति॑ । \newline
16. इत्य॑ब्रुवन् नब्रुव॒न् निती त्य॑ब्रुवन्न् । \newline
17. अ॒ब्रु॒व॒न् ताꣳ स्ता न॑ब्रुवन् नब्रुव॒न् तान् । \newline
18. तान् द्वाभ्या॒म् द्वाभ्या॒म् ताꣳ स्तान् द्वाभ्या᳚म् । \newline
19. द्वाभ्या॒म् चिती᳚भ्या॒म् चिती᳚भ्या॒म् द्वाभ्या॒म् द्वाभ्या॒म् चिती᳚भ्याम् । \newline
20. चिती᳚भ्या मु॒पाय॑न् नु॒पाय॒ ञ्चिती᳚भ्या॒म् चिती᳚भ्या मु॒पायन्न्॑ । \newline
21. चिती᳚भ्या॒मिति॒ चिति॑ - भ्या॒म् । \newline
22. उ॒पाय॒न् थ्स स उ॒पाय॑न् नु॒पाय॒न् थ्सः । \newline
23. उ॒पाय॒न्नित्यु॑प - आयन्न्॑ । \newline
24. स पञ्च॑चितीकः॒ पञ्च॑चितीकः॒ स स पञ्च॑चितीकः । \newline
25. पञ्च॑चितीकः॒ सꣳ सम् पञ्च॑चितीकः॒ पञ्च॑चितीकः॒ सम् । \newline
26. पञ्च॑चितीक॒ इति॒ पञ्च॑ - चि॒ती॒कः॒ । \newline
27. स म॑पद्यता पद्यत॒ सꣳ स म॑पद्यत । \newline
28. अ॒प॒द्य॒त॒ यो यो॑ ऽपद्यता पद्यत॒ यः । \newline
29. य ए॒व मे॒वं ॅयो य ए॒वम् । \newline
30. ए॒वं ॅवि॒द्वान्. वि॒द्वा ने॒व मे॒वं ॅवि॒द्वान् । \newline
31. वि॒द्वा न॒ग्नि म॒ग्निं ॅवि॒द्वान्. वि॒द्वा न॒ग्निम् । \newline
32. अ॒ग्निम् चि॑नु॒ते चि॑नु॒ते᳚ ऽग्नि म॒ग्निम् चि॑नु॒ते । \newline
33. चि॒नु॒ते भूया॒न् भूयाꣳ॑ श्चिनु॒ते चि॑नु॒ते भूयान्॑ । \newline
34. भूया॑ ने॒वैव भूया॒न् भूया॑ ने॒व । \newline
35. ए॒व भ॑वति भव त्ये॒वैव भ॑वति । \newline
36. भ॒व॒ त्य॒भ्य॑भि भ॑वति भव त्य॒भि । \newline
37. अ॒भीमा नि॒मा न॒भ्य॑भीमान् । \newline
38. इ॒मान् ॅलो॒कान् ॅलो॒का नि॒मा नि॒मान् ॅलो॒कान् । \newline
39. लो॒कान् ज॑यति जयति लो॒कान् ॅलो॒कान् ज॑यति । \newline
40. ज॒य॒ति॒ वि॒दुर् वि॒दुर् ज॑यति जयति वि॒दुः । \newline
41. वि॒दु रे॑न मेनं ॅवि॒दुर् वि॒दु रे॑नम् । \newline
42. ए॒न॒म् दे॒वा दे॒वा ए॑न मेनम् दे॒वाः । \newline
43. दे॒वा अथो॒ अथो॑ दे॒वा दे॒वा अथो᳚ । \newline
44. अथो॑ ए॒तासा॑ मे॒तासा॒ मथो॒ अथो॑ ए॒तासा᳚म् । \newline
45. अथो॒ इत्यथो᳚ । \newline
46. ए॒तासा॑ मे॒वै वैतासा॑ मे॒तासा॑ मे॒व । \newline
47. ए॒व दे॒वता॑नाम् दे॒वता॑ना मे॒वैव दे॒वता॑नाम् । \newline
48. दे॒वता॑नाꣳ॒॒ सायु॑ज्यꣳ॒॒ सायु॑ज्यम् दे॒वता॑नाम् दे॒वता॑नाꣳ॒॒ सायु॑ज्यम् । \newline
49. सायु॑ज्यम् गच्छति गच्छति॒ सायु॑ज्यꣳ॒॒ सायु॑ज्यम् गच्छति । \newline
50. ग॒च्छ॒तीति॑ गच्छति । \newline

\textbf{Ghana Paata } \newline

1. आ॒दि॒त्य स्ताꣳ स्ता ना॑दि॒त्य आ॑दि॒त्य स्ता नृष॑य॒ ऋष॑य॒ स्ता ना॑दि॒त्य आ॑दि॒त्य स्ता नृष॑यः । \newline
2. तानृष॑य॒ ऋष॑य॒ स्ताꣳ स्ता नृष॑यो ऽब्रुवन् नब्रुव॒न् नृष॑य॒ स्ताꣳ स्ता नृष॑यो ऽब्रुवन्न् । \newline
3. ऋष॑यो ऽब्रुवन् नब्रुव॒न् नृष॑य॒ ऋष॑यो ऽब्रुव॒न् नुपोपा᳚ब्रुव॒न् नृष॑य॒ ऋष॑यो ऽब्रुव॒न् नुप॑ । \newline
4. अ॒ब्रु॒व॒न् नुपोपा᳚ब्रुवन् नब्रुव॒न् नुप॑ वो व॒ उपा᳚ब्रुवन् नब्रुव॒न् नुप॑ वः । \newline
5. उप॑ वो व॒ उपोप॑ व॒ आ व॒ उपोप॑ व॒ आ । \newline
6. व॒ आ वो॑ व॒ आ ऽया॑मा या॒मा वो॑ व॒ आ ऽया॑म । \newline
7. आ ऽया॑मा या॒मा ऽया॒मे तीत्य॑या॒मा ऽया॒मेति॑ । \newline
8. अ॒या॒मे तीत्य॑ यामा या॒मेति॒ केन॒ केने त्य॑यामा या॒मेति॒ केन॑ । \newline
9. इति॒ केन॒ केने तीति॒ केन॑ नो नः॒ केने तीति॒ केन॑ नः । \newline
10. केन॑ नो नः॒ केन॒ केन॑ न उ॒पैष्य॑ थो॒पैष्य॑थ नः॒ केन॒ केन॑ न उ॒पैष्य॑थ । \newline
11. न॒ उ॒पैष्य॑ थो॒पैष्य॑थ नो न उ॒पैष्य॒थे तीत्यु॒पैष्य॑थ नो न उ॒पैष्य॒थेति॑ । \newline
12. उ॒पैष्य॒थे तीत्यु॒पैष्य॑ थो॒पैष्य॒थेति॑ भू॒म्ना भू॒म्ने त्यु॒पैष्य॑ थो॒पैष्य॒थेति॑ भू॒म्ना । \newline
13. उ॒पैष्य॒थेत्यु॑प - ऐष्य॑थ । \newline
14. इति॑ भू॒म्ना भू॒म्नेतीति॑ भू॒म्नेतीति॑ भू॒म्नेतीति॑ भू॒म्नेति॑ । \newline
15. भू॒म्नेतीति॑ भू॒म्ना भू॒म्ने त्य॑ब्रुवन् नब्रुव॒न् निति॑ भू॒म्ना भू॒म्ने त्य॑ब्रुवन्न् । \newline
16. इत्य॑ब्रुवन् नब्रुव॒न् निती त्य॑ब्रुव॒न् ताꣳ स्ता न॑ब्रुव॒न् नितीत्य॑ ब्रुव॒न् तान् । \newline
17. अ॒ब्रु॒व॒न् ताꣳ स्तान॑ब्रुवन् नब्रुव॒न् तान् द्वाभ्या॒म् द्वाभ्या॒म् तान॑ब्रुवन् नब्रुव॒न् तान् द्वाभ्या᳚म् । \newline
18. तान् द्वाभ्या॒म् द्वाभ्या॒म् ताꣳ स्तान् द्वाभ्या॒म् चिती᳚भ्या॒म् चिती᳚भ्या॒म् द्वाभ्या॒म् ताꣳ स्तान् द्वाभ्या॒म् चिती᳚भ्याम् । \newline
19. द्वाभ्या॒म् चिती᳚भ्या॒म् चिती᳚भ्या॒म् द्वाभ्या॒म् द्वाभ्या॒म् चिती᳚भ्या मु॒पाय॑न् नु॒पाय॒ ञ्चिती᳚भ्या॒म् द्वाभ्या॒म् द्वाभ्या॒म् चिती᳚भ्या मु॒पायन्न्॑ । \newline
20. चिती᳚भ्या मु॒पाय॑न् नु॒पाय॒ ञ्चिती᳚भ्या॒म् चिती᳚भ्या मु॒पाय॒न् थ्स स उ॒पाय॒ ञ्चिती᳚भ्या॒म् चिती᳚भ्या मु॒पाय॒न् थ्सः । \newline
21. चिती᳚भ्या॒मिति॒ चिति॑ - भ्या॒म् । \newline
22. उ॒पाय॒न् थ्स स उ॒पाय॑न् नु॒पाय॒न् थ्स पञ्च॑चितीकः॒ पञ्च॑चितीकः॒ स उ॒पाय॑न् नु॒पाय॒न् थ्स पञ्च॑चितीकः । \newline
23. उ॒पाय॒न्नित्यु॑प - आयन्न्॑ । \newline
24. स पञ्च॑चितीकः॒ पञ्च॑चितीकः॒ स स पञ्च॑चितीकः॒ सꣳ सम् पञ्च॑चितीकः॒ स स पञ्च॑चितीकः॒ सम् । \newline
25. पञ्च॑चितीकः॒ सꣳ सम् पञ्च॑चितीकः॒ पञ्च॑चितीकः॒ स म॑पद्यता पद्यत॒ सम् पञ्च॑चितीकः॒ पञ्च॑चितीकः॒ स म॑पद्यत । \newline
26. पञ्च॑चितीक॒ इति॒ पञ्च॑ - चि॒ती॒कः॒ । \newline
27. स म॑पद्यता पद्यत॒ सꣳ स म॑पद्यत॒ यो यो॑ ऽपद्यत॒ सꣳ स म॑पद्यत॒ यः । \newline
28. अ॒प॒द्य॒त॒ यो यो॑ ऽपद्यता पद्यत॒ य ए॒व मे॒वं ॅयो॑ ऽपद्यता पद्यत॒ य ए॒वम् । \newline
29. य ए॒व मे॒वं ॅयो य ए॒वं ॅवि॒द्वान्. वि॒द्वा ने॒वं ॅयो य ए॒वं ॅवि॒द्वान् । \newline
30. ए॒वं ॅवि॒द्वान्. वि॒द्वा ने॒व मे॒वं ॅवि॒द्वा न॒ग्नि म॒ग्निं ॅवि॒द्वा ने॒व मे॒वं ॅवि॒द्वा न॒ग्निम् । \newline
31. वि॒द्वा न॒ग्नि म॒ग्निं ॅवि॒द्वान्. वि॒द्वा न॒ग्निम् चि॑नु॒ते चि॑नु॒ते᳚ ऽग्निं ॅवि॒द्वान्. वि॒द्वा न॒ग्निम् चि॑नु॒ते । \newline
32. अ॒ग्निम् चि॑नु॒ते चि॑नु॒ते᳚ ऽग्नि म॒ग्निम् चि॑नु॒ते भूया॒न् भूयाꣳ॑ श्चिनु॒ते᳚ ऽग्नि म॒ग्निम् चि॑नु॒ते भूयान्॑ । \newline
33. चि॒नु॒ते भूया॒न् भूयाꣳ॑ श्चिनु॒ते चि॑नु॒ते भूया॑ ने॒वैव भूयाꣳ॑ श्चिनु॒ते चि॑नु॒ते भूया॑ ने॒व । \newline
34. भूया॑ ने॒वैव भूया॒न् भूया॑ ने॒व भ॑वति भव त्ये॒व भूया॒न् भूया॑ ने॒व भ॑वति । \newline
35. ए॒व भ॑वति भव त्ये॒वैव भ॑व त्य॒भ्य॑भि भ॑व त्ये॒वैव भ॑व त्य॒भि । \newline
36. भ॒व॒ त्य॒भ्य॑भि भ॑वति भव त्य॒भीमा नि॒मा न॒भि भ॑वति भव त्य॒भीमान् । \newline
37. अ॒भीमा नि॒मा न॒भ्य॑ भीमान् ॅलो॒कान् ॅलो॒का नि॒मा न॒भ्य॑ भीमान् ॅलो॒कान् । \newline
38. इ॒मान् ॅलो॒कान् ॅलो॒का नि॒मा नि॒मान् ॅलो॒कान् ज॑यति जयति लो॒का नि॒मा नि॒मान् ॅलो॒कान् ज॑यति । \newline
39. लो॒कान् ज॑यति जयति लो॒कान् ॅलो॒कान् ज॑यति वि॒दुर् वि॒दुर् ज॑यति लो॒कान् ॅलो॒कान् ज॑यति वि॒दुः । \newline
40. ज॒य॒ति॒ वि॒दुर् वि॒दुर् ज॑यति जयति वि॒दु रे॑न मेनं ॅवि॒दुर् ज॑यति जयति वि॒दु रे॑नम् । \newline
41. वि॒दु रे॑न मेनं ॅवि॒दुर् वि॒दु रे॑नम् दे॒वा दे॒वा ए॑नं ॅवि॒दुर् वि॒दु रे॑नम् दे॒वाः । \newline
42. ए॒न॒म् दे॒वा दे॒वा ए॑न मेनम् दे॒वा अथो॒ अथो॑ दे॒वा ए॑न मेनम् दे॒वा अथो᳚ । \newline
43. दे॒वा अथो॒ अथो॑ दे॒वा दे॒वा अथो॑ ए॒तासा॑ मे॒तासा॒ मथो॑ दे॒वा दे॒वा अथो॑ ए॒तासा᳚म् । \newline
44. अथो॑ ए॒तासा॑ मे॒तासा॒ मथो॒ अथो॑ ए॒तासा॑ मे॒वैवै तासा॒ मथो॒ अथो॑ ए॒तासा॑ मे॒व । \newline
45. अथो॒ इत्यथो᳚ । \newline
46. ए॒तासा॑ मे॒वैवै तासा॑ मे॒तासा॑ मे॒व दे॒वता॑नाम् दे॒वता॑ना मे॒वै तासा॑ मे॒तासा॑ मे॒व दे॒वता॑नाम् । \newline
47. ए॒व दे॒वता॑नाम् दे॒वता॑ना मे॒वैव दे॒वता॑नाꣳ॒॒ सायु॑ज्यꣳ॒॒ सायु॑ज्यम् दे॒वता॑ना मे॒वैव दे॒वता॑नाꣳ॒॒ सायु॑ज्यम् । \newline
48. दे॒वता॑नाꣳ॒॒ सायु॑ज्यꣳ॒॒ सायु॑ज्यम् दे॒वता॑नाम् दे॒वता॑नाꣳ॒॒ सायु॑ज्यम् गच्छति गच्छति॒ सायु॑ज्यम् दे॒वता॑नाम् दे॒वता॑नाꣳ॒॒ सायु॑ज्यम् गच्छति । \newline
49. सायु॑ज्यम् गच्छति गच्छति॒ सायु॑ज्यꣳ॒॒ सायु॑ज्यम् गच्छति । \newline
50. ग॒च्छ॒तीति॑ गच्छति । \newline
\pagebreak
\markright{ TS 5.7.6.1  \hfill https://www.vedavms.in \hfill}

\section{ TS 5.7.6.1 }

\textbf{TS 5.7.6.1 } \newline
\textbf{Samhita Paata} \newline

वयो॒ वा अ॒ग्निर्यद॑ग्नि॒चित् प॒क्षिणो᳚ऽश्नी॒यात् तमे॒वाग्निम॑द्या॒दा-र्ति॒मार्च्छे᳚थ् संॅवथ्स॒रं ॅव्र॒तं च॑रेथ् संॅवथ्स॒रꣳ हि व्र॒तं नाति॑ प॒शुर्वा ए॒ष यद॒ग्निर्.हि॒नस्ति॒ खलु॒ वै तं प॒शुर्य ए॑नं पु॒रस्ता᳚त् प्र॒त्यञ्च॑मुप॒चर॑ति॒ तस्मा᳚त् प॒श्चात् प्राङु॑प॒चर्य॑ आ॒त्मनोऽहिꣳ॑सायै॒ तेजो॑ऽसि॒ तेजो॑ मे यच्छ पृथि॒वीं ॅय॑च्छ - [  ] \newline

\textbf{Pada Paata} \newline

वयः॑ । वै । अ॒ग्निः । यत् । अ॒ग्नि॒चिदित्य॑ग्नि - चित् । प॒क्षिणः॑ । अ॒श्नी॒यात् । तम् । ए॒व । अ॒ग्निम् । अ॒द्या॒त् । आर्ति᳚म् । एति॑ । ऋ॒च्छे॒त् । सं॒ॅव॒थ्स॒रमिति॑ सं - व॒थ्स॒रम् । व्र॒तम् । च॒रे॒त् । सं॒ॅव॒थ्स॒रमिति॑ सं - व॒थ्स॒रम् । हि । व्र॒तम् । न । अतीति॑ । प॒शुः । वै । ए॒षः । यत् । अ॒ग्निः । हि॒नस्ति॑ । खलु॑ । वै । तम् । प॒शुः । यः । ए॒न॒म् । पु॒रस्ता᳚त् । प्र॒त्यञ्च᳚म् । उ॒प॒चर॒तीत्यु॑प - चर॑ति । तस्मा᳚त् । प॒श्चात् । प्राङ् । उ॒प॒चर्य॒ इत्यु॑प - चर्यः॑ । आ॒त्मनः॑ । अहिꣳ॑सायै । तेजः॑ । अ॒सि॒ । तेजः॑ । मे॒ । य॒च्छ॒ । पृ॒थि॒वीम् । य॒च्छ॒ ।  \newline


\textbf{Krama Paata} \newline

वयो॒ वै । वा अ॒ग्निः । अ॒ग्निर् यत् । यद॑ग्नि॒चित् । अ॒ग्नि॒चित् प॒क्षिणः॑ । अ॒ग्नि॒चिदित्य॑ग्नि - चित् । प॒क्षिणो᳚ऽश्ञी॒यात् । अ॒श्ञी॒यात् तम् । तमे॒व । ए॒वाग्निम् । अ॒ग्निम॑द्यात् । अ॒द्या॒दार्ति᳚म् । आर्ति॒मा । आर्च्छे᳚त् । ऋ॒च्छे॒थ् स॒म्ॅव॒थ्स॒रम् । स॒म्ॅव॒थ्स॒रम् ॅव्र॒तम् । स॒म्ॅव॒थ्स॒रमिति॑ सम् - व॒थ्स॒रम् । व्र॒तम् च॑रेत् । च॒रे॒थ् स॒म्ॅव॒थ्स॒रम् । स॒म्ॅव॒थ्स॒रꣳ हि । स॒म्ॅव॒थ्स॒रमिति॑ सम् - व॒थ्स॒रम् । हि व्र॒तम् । व्र॒तम् न । नाति॑ । अति॑ प॒शुः । प॒शुर् वै । वा ए॒षः । ए॒ष यत् । यद॒ग्निः । अ॒ग्निर्. हि॒नस्ति॑ । हि॒नस्ति॒ खलु॑ । खलु॒ वै । वै तम् । तम् प॒शुः । प॒शुर् यः । य ए॑नम् । ए॒न॒म् पु॒रस्ता᳚त् । पु॒रस्ता᳚त् प्र॒त्यञ्च᳚म् । प्र॒त्यञ्च॑मुप॒चर॑ति । उ॒प॒चर॑ति॒ तस्मा᳚त् । उ॒प॒चर॒तीत्यु॑प - चर॑ति । तस्मा᳚त् प॒श्चात् । प॒श्चात् प्राङ्ङ् । प्राङ्ङु॑प॒चर्यः॑ । उ॒प॒चर्य॑ आ॒त्मनः॑ । उ॒प॒चर्य॒ इत्यु॑प - चर्यः॑ । आ॒त्मनोऽहिꣳ॑सायै । अहिꣳ॑सायै॒ तेजः॑ । तेजो॑ऽसि । अ॒सि॒ तेजः॑ । तेजो॑ मे । मे॒ य॒च्छ॒ । य॒च्छ॒ पृ॒थि॒वीम् । पृ॒थि॒वीम् ॅय॑च्छ । य॒च्छ॒ पृ॒थि॒व्यै \newline

\textbf{Jatai Paata} \newline

1. वयो॒ वै वै वयो॒ वयो॒ वै । \newline
2. वा अ॒ग्नि र॒ग्निर् वै वा अ॒ग्निः । \newline
3. अ॒ग्निर् यद् यद॒ग्नि र॒ग्निर् यत् । \newline
4. यद॑ग्नि॒चि द॑ग्नि॒चिद् यद् यद॑ग्नि॒चित् । \newline
5. अ॒ग्नि॒चित् प॒क्षिणः॑ प॒क्षिणो᳚ ऽग्नि॒चि द॑ग्नि॒चित् प॒क्षिणः॑ । \newline
6. अ॒ग्नि॒चिदित्य॑ग्नि - चित् । \newline
7. प॒क्षिणो᳚ ऽश्ञी॒या द॑श्ञी॒यात् प॒क्षिणः॑ प॒क्षिणो᳚ ऽश्ञी॒यात् । \newline
8. अ॒श्ञी॒यात् तम् त म॑श्ञी॒या द॑श्ञी॒यात् तम् । \newline
9. त मे॒वैव तम् त मे॒व । \newline
10. ए॒वाग्नि म॒ग्नि मे॒वै वाग्निम् । \newline
11. अ॒ग्नि म॑द्या दद्या द॒ग्नि म॒ग्नि म॑द्यात् । \newline
12. अ॒द्या॒ दार्ति॒ मार्ति॑ मद्या दद्या॒ दार्ति᳚म् । \newline
13. आर्ति॒ मा ऽऽर्ति॒ मार्ति॒ मा । \newline
14. आर्च्छे॑ दृच्छे॒ दार्च्छे᳚त् । \newline
15. ऋ॒च्छे॒थ् सं॒ॅव॒थ्स॒रꣳ सं॑ॅवथ्स॒र मृ॑च्छे दृच्छेथ् संॅवथ्स॒रम् । \newline
16. सं॒ॅव॒थ्स॒रं ॅव्र॒तं ॅव्र॒तꣳ सं॑ॅवथ्स॒रꣳ सं॑ॅवथ्स॒रं ॅव्र॒तम् । \newline
17. सं॒ॅव॒थ्स॒रमिति॑ सं - व॒थ्स॒रम् । \newline
18. व्र॒तम् च॑रेच् चरेद् व्र॒तं ॅव्र॒तम् च॑रेत् । \newline
19. च॒रे॒थ् सं॒ॅव॒थ्स॒रꣳ सं॑ॅवथ्स॒रम् च॑रेच् चरेथ् संॅवथ्स॒रम् । \newline
20. सं॒ॅव॒थ्स॒रꣳ हि हि सं॑ॅवथ्स॒रꣳ सं॑ॅवथ्स॒रꣳ हि । \newline
21. सं॒ॅव॒थ्स॒रमिति॑ सं - व॒थ्स॒रम् । \newline
22. हि व्र॒तं ॅव्र॒तꣳ हि हि व्र॒तम् । \newline
23. व्र॒तम् न न व्र॒तं ॅव्र॒तम् न । \newline
24. नात्यति॒ न नाति॑ । \newline
25. अति॑ प॒शुः प॒शु रत्यति॑ प॒शुः । \newline
26. प॒शुर् वै वै प॒शुः प॒शुर् वै । \newline
27. वा ए॒ष ए॒ष वै वा ए॒षः । \newline
28. ए॒ष यद् यदे॒ष ए॒ष यत् । \newline
29. यद॒ग्नि र॒ग्निर् यद् यद॒ग्निः । \newline
30. अ॒ग्निर्. हि॒नस्ति॑ हि॒नस् त्य॒ग्नि र॒ग्निर्. हि॒नस्ति॑ । \newline
31. हि॒नस्ति॒ खलु॒ खलु॑ हि॒नस्ति॑ हि॒नस्ति॒ खलु॑ । \newline
32. खलु॒ वै वै खलु॒ खलु॒ वै । \newline
33. वै तम् तं ॅवै वै तम् । \newline
34. तम् प॒शुः प॒शु स्तम् तम् प॒शुः । \newline
35. प॒शुर् यो यः प॒शुः प॒शुर् यः । \newline
36. य ए॑न मेनं॒ ॅयो य ए॑नम् । \newline
37. ए॒न॒म् पु॒रस्ता᳚त् पु॒रस्ता॑ देन मेनम् पु॒रस्ता᳚त् । \newline
38. पु॒रस्ता᳚त् प्र॒त्यञ्च॑म् प्र॒त्यञ्च॑म् पु॒रस्ता᳚त् पु॒रस्ता᳚त् प्र॒त्यञ्च᳚म् । \newline
39. प्र॒त्यञ्च॑ मुप॒चर॑ त्युप॒चर॑ति प्र॒त्यञ्च॑म् प्र॒त्यञ्च॑ मुप॒चर॑ति । \newline
40. उ॒प॒चर॑ति॒ तस्मा॒त् तस्मा॑ दुप॒चर॑ त्युप॒चर॑ति॒ तस्मा᳚त् । \newline
41. उ॒प॒चर॒तीत्यु॑प - चर॑ति । \newline
42. तस्मा᳚त् प॒श्चात् प॒श्चात् तस्मा॒त् तस्मा᳚त् प॒श्चात् । \newline
43. प॒श्चात् प्राङ् प्राङ् प॒श्चात् प॒श्चात् प्राङ् । \newline
44. प्राङ् ङु॑प॒चर्य॑ उप॒चर्यः॒ प्राङ् प्राङ् ङु॑प॒चर्यः॑ । \newline
45. उ॒प॒चर्य॑ आ॒त्मन॑ आ॒त्मन॑ उप॒चर्य॑ उप॒चर्य॑ आ॒त्मनः॑ । \newline
46. उ॒प॒चर्य॒ इत्यु॑प - चर्यः॑ । \newline
47. आ॒त्मनो ऽहिꣳ॑साया॒ अहिꣳ॑साया आ॒त्मन॑ आ॒त्मनो ऽहिꣳ॑सायै । \newline
48. अहिꣳ॑सायै॒ तेज॒ स्तेजो ऽहिꣳ॑साया॒ अहिꣳ॑सायै॒ तेजः॑ । \newline
49. तेजो᳚ ऽस्यसि॒ तेज॒ स्तेजो॑ ऽसि । \newline
50. अ॒सि॒ तेज॒ स्तेजो᳚ ऽस्यसि॒ तेजः॑ । \newline
51. तेजो॑ मे मे॒ तेज॒ स्तेजो॑ मे । \newline
52. मे॒ य॒च्छ॒ य॒च्छ॒ मे॒ मे॒ य॒च्छ॒ । \newline
53. य॒च्छ॒ पृ॒थि॒वीम् पृ॑थि॒वीं ॅय॑च्छ यच्छ पृथि॒वीम् । \newline
54. पृ॒थि॒वीं ॅय॑च्छ यच्छ पृथि॒वीम् पृ॑थि॒वीं ॅय॑च्छ । \newline
55. य॒च्छ॒ पृ॒थि॒व्यै पृ॑थि॒व्यै य॑च्छ यच्छ पृथि॒व्यै । \newline

\textbf{Ghana Paata } \newline

1. वयो॒ वै वै वयो॒ वयो॒ वा अ॒ग्नि र॒ग्निर् वै वयो॒ वयो॒ वा अ॒ग्निः । \newline
2. वा अ॒ग्नि र॒ग्निर् वै वा अ॒ग्निर् यद् यद॒ग्निर् वै वा अ॒ग्निर् यत् । \newline
3. अ॒ग्निर् यद् यद॒ग्नि र॒ग्निर् यद॑ग्नि॒चि द॑ग्नि॒चिद् यद॒ग्नि र॒ग्निर् यद॑ग्नि॒चित् । \newline
4. यद॑ग्नि॒चि द॑ग्नि॒चिद् यद् यद॑ग्नि॒चित् प॒क्षिणः॑ प॒क्षिणो᳚ ऽग्नि॒चिद् यद् यद॑ग्नि॒चित् प॒क्षिणः॑ । \newline
5. अ॒ग्नि॒चित् प॒क्षिणः॑ प॒क्षिणो᳚ ऽग्नि॒चि द॑ग्नि॒चित् प॒क्षिणो᳚ ऽश्ञी॒या द॑श्ञी॒यात् प॒क्षिणो᳚ ऽग्नि॒चि द॑ग्नि॒चित् प॒क्षिणो᳚ ऽश्ञी॒यात् । \newline
6. अ॒ग्नि॒चिदित्य॑ग्नि - चित् । \newline
7. प॒क्षिणो᳚ ऽश्ञी॒या द॑श्ञी॒यात् प॒क्षिणः॑ प॒क्षिणो᳚ ऽश्ञी॒यात् तम् त म॑श्ञी॒यात् प॒क्षिणः॑ प॒क्षिणो᳚ ऽश्ञी॒यात् तम् । \newline
8. अ॒श्ञी॒यात् तम् त म॑श्ञी॒या द॑श्ञी॒यात् त मे॒वैव त म॑श्ञी॒या द॑श्ञी॒यात् त मे॒व । \newline
9. त मे॒वैव तम् त मे॒वाग्नि म॒ग्नि मे॒व तम् त मे॒वाग्निम् । \newline
10. ए॒वाग्नि म॒ग्नि मे॒वै वाग्नि म॑द्या दद्या द॒ग्नि मे॒वै वाग्नि म॑द्यात् । \newline
11. अ॒ग्नि म॑द्या दद्या द॒ग्नि म॒ग्नि म॑द्या॒ दार्ति॒ मार्ति॑ मद्या द॒ग्नि म॒ग्नि म॑द्या॒ दार्ति᳚म् । \newline
12. अ॒द्या॒ दार्ति॒ मार्ति॑ मद्या दद्या॒ दार्ति॒ मा ऽऽर्ति॑ मद्या दद्या॒ दार्ति॒ मा । \newline
13. आर्ति॒ मा ऽऽर्ति॒ मार्ति॒ मार्च्छे॑ दृच्छे॒दा ऽऽर्ति॒ मार्ति॒ मार्च्छे᳚त् । \newline
14. आर्च्छे॑ दृच्छे॒ दार्च्छे᳚थ् संॅवथ्स॒रꣳ सं॑ॅवथ्स॒र मृ॑च्छे॒ दार्च्छे᳚थ् संॅवथ्स॒रम् । \newline
15. ऋ॒च्छे॒थ् सं॒ॅव॒थ्स॒रꣳ सं॑ॅवथ्स॒र मृ॑च्छे दृच्छेथ् संॅवथ्स॒रं ॅव्र॒तं ॅव्र॒तꣳ सं॑ॅवथ्स॒र मृ॑च्छे दृच्छेथ् संॅवथ्स॒रं ॅव्र॒तम् । \newline
16. सं॒ॅव॒थ्स॒रं ॅव्र॒तं ॅव्र॒तꣳ सं॑ॅवथ्स॒रꣳ सं॑ॅवथ्स॒रं ॅव्र॒तम् च॑रेच् चरेद् व्र॒तꣳ सं॑ॅवथ्स॒रꣳ सं॑ॅवथ्स॒रं ॅव्र॒तम् च॑रेत् । \newline
17. सं॒ॅव॒थ्स॒रमिति॑ सं - व॒थ्स॒रम् । \newline
18. व्र॒तम् च॑रेच् चरेद् व्र॒तं ॅव्र॒तम् च॑रेथ् संॅवथ्स॒रꣳ सं॑ॅवथ्स॒रम् च॑रेद् व्र॒तं ॅव्र॒तम् च॑रेथ् संॅवथ्स॒रम् । \newline
19. च॒रे॒थ् सं॒ॅव॒थ्स॒रꣳ सं॑ॅवथ्स॒रम् च॑रेच् चरेथ् संॅवथ्स॒रꣳ हि हि सं॑ॅवथ्स॒रम् च॑रेच् चरेथ् संॅवथ्स॒रꣳ हि । \newline
20. सं॒ॅव॒थ्स॒रꣳ हि हि सं॑ॅवथ्स॒रꣳ सं॑ॅवथ्स॒रꣳ हि व्र॒तं ॅव्र॒तꣳ हि सं॑ॅवथ्स॒रꣳ सं॑ॅवथ्स॒रꣳ हि व्र॒तम् । \newline
21. सं॒ॅव॒थ्स॒रमिति॑ सं - व॒थ्स॒रम् । \newline
22. हि व्र॒तं ॅव्र॒तꣳ हि हि व्र॒तम् न न व्र॒तꣳ हि हि व्र॒तम् न । \newline
23. व्र॒तम् न न व्र॒तं ॅव्र॒तम् नात्यति॒ न व्र॒तं ॅव्र॒तम् नाति॑ । \newline
24. नात्यति॒ न नाति॑ प॒शुः प॒शु रति॒ न नाति॑ प॒शुः । \newline
25. अति॑ प॒शुः प॒शु रत्यति॑ प॒शुर् वै वै प॒शु रत्यति॑ प॒शुर् वै । \newline
26. प॒शुर् वै वै प॒शुः प॒शुर् वा ए॒ष ए॒ष वै प॒शुः प॒शुर् वा ए॒षः । \newline
27. वा ए॒ष ए॒ष वै वा ए॒ष यद् यदे॒ष वै वा ए॒ष यत् । \newline
28. ए॒ष यद् यदे॒ष ए॒ष यद॒ग्नि र॒ग्निर् यदे॒ष ए॒ष यद॒ग्निः । \newline
29. यद॒ग्नि र॒ग्निर् यद् यद॒ग्निर्. हि॒नस्ति॑ हि॒नस्त्य॒ग्निर् यद् यद॒ग्निर्. हि॒नस्ति॑ । \newline
30. अ॒ग्निर्. हि॒नस्ति॑ हि॒नस्त्य॒ग्नि र॒ग्निर्. हि॒नस्ति॒ खलु॒ खलु॑ हि॒नस्त्य॒ग्नि र॒ग्निर्. हि॒नस्ति॒ खलु॑ । \newline
31. हि॒नस्ति॒ खलु॒ खलु॑ हि॒नस्ति॑ हि॒नस्ति॒ खलु॒ वै वै खलु॑ हि॒नस्ति॑ हि॒नस्ति॒ खलु॒ वै । \newline
32. खलु॒ वै वै खलु॒ खलु॒ वै तम् तं ॅवै खलु॒ खलु॒ वै तम् । \newline
33. वै तम् तं ॅवै वै तम् प॒शुः प॒शु स्तं ॅवै वै तम् प॒शुः । \newline
34. तम् प॒शुः प॒शु स्तम् तम् प॒शुर् यो यः प॒शु स्तम् तम् प॒शुर् यः । \newline
35. प॒शुर् यो यः प॒शुः प॒शुर् य ए॑न मेनं॒ ॅयः प॒शुः प॒शुर् य ए॑नम् । \newline
36. य ए॑न मेनं॒ ॅयो य ए॑नम् पु॒रस्ता᳚त् पु॒रस्ता॑ देनं॒ ॅयो य ए॑नम् पु॒रस्ता᳚त् । \newline
37. ए॒न॒म् पु॒रस्ता᳚त् पु॒रस्ता॑ देन मेनम् पु॒रस्ता᳚त् प्र॒त्यञ्च॑म् प्र॒त्यञ्च॑म् पु॒रस्ता॑ देन मेनम् पु॒रस्ता᳚त् प्र॒त्यञ्च᳚म् । \newline
38. पु॒रस्ता᳚त् प्र॒त्यञ्च॑म् प्र॒त्यञ्च॑म् पु॒रस्ता᳚त् पु॒रस्ता᳚त् प्र॒त्यञ्च॑ मुप॒चर॑ त्युप॒चर॑ति प्र॒त्यञ्च॑म् पु॒रस्ता᳚त् पु॒रस्ता᳚त् प्र॒त्यञ्च॑ मुप॒चर॑ति । \newline
39. प्र॒त्यञ्च॑ मुप॒चर॑ त्युप॒चर॑ति प्र॒त्यञ्च॑म् प्र॒त्यञ्च॑ मुप॒चर॑ति॒ तस्मा॒त् तस्मा॑ दुप॒चर॑ति प्र॒त्यञ्च॑म् प्र॒त्यञ्च॑ मुप॒चर॑ति॒ तस्मा᳚त् । \newline
40. उ॒प॒चर॑ति॒ तस्मा॒त् तस्मा॑ दुप॒चर॑ त्युप॒चर॑ति॒ तस्मा᳚त् प॒श्चात् प॒श्चात् तस्मा॑ दुप॒चर॑ त्युप॒चर॑ति॒ तस्मा᳚त् प॒श्चात् । \newline
41. उ॒प॒चर॒तीत्यु॑प - चर॑ति । \newline
42. तस्मा᳚त् प॒श्चात् प॒श्चात् तस्मा॒त् तस्मा᳚त् प॒श्चात् प्राङ् प्राङ् प॒श्चात् तस्मा॒त् तस्मा᳚त् प॒श्चात् प्राङ् । \newline
43. प॒श्चात् प्राङ् प्राङ् प॒श्चात् प॒श्चात् प्राङ् ङु॑प॒चर्य॑ उप॒चर्यः॒ प्राङ् प॒श्चात् प॒श्चात् प्राङ् ङु॑प॒चर्यः॑ । \newline
44. प्राङ् ङु॑प॒चर्य॑ उप॒चर्यः॒ प्राङ् प्राङ् ङु॑प॒चर्य॑ आ॒त्मन॑ आ॒त्मन॑ उप॒चर्यः॒ प्राङ् प्राङ्
ङु॑प॒चर्य॑ आ॒त्मनः॑ । \newline
45. उ॒प॒चर्य॑ आ॒त्मन॑ आ॒त्मन॑ उप॒चर्य॑ उप॒चर्य॑ आ॒त्मनो ऽहिꣳ॑साया॒ अहिꣳ॑साया आ॒त्मन॑ उप॒चर्य॑ उप॒चर्य॑ आ॒त्मनो ऽहिꣳ॑सायै । \newline
46. उ॒प॒चर्य॒ इत्यु॑प - चर्यः॑ । \newline
47. आ॒त्मनो ऽहिꣳ॑साया॒ अहिꣳ॑साया आ॒त्मन॑ आ॒त्मनो ऽहिꣳ॑सायै॒ तेज॒ स्तेजो ऽहिꣳ॑साया आ॒त्मन॑ आ॒त्मनो ऽहिꣳ॑सायै॒ तेजः॑ । \newline
48. अहिꣳ॑सायै॒ तेज॒ स्तेजो ऽहिꣳ॑साया॒ अहिꣳ॑सायै॒ तेजो᳚ ऽस्यसि॒ तेजो ऽहिꣳ॑साया॒ अहिꣳ॑सायै॒ तेजो॑ ऽसि । \newline
49. तेजो᳚ ऽस्यसि॒ तेज॒ स्तेजो॑ ऽसि॒ तेज॒ स्तेजो॑ ऽसि॒ तेज॒ स्तेजो॑ ऽसि॒ तेजः॑ । \newline
50. अ॒सि॒ तेज॒ स्तेजो᳚ ऽस्यसि॒ तेजो॑ मे मे॒ तेजो᳚ ऽस्यसि॒ तेजो॑ मे । \newline
51. तेजो॑ मे मे॒ तेज॒ स्तेजो॑ मे यच्छ यच्छ मे॒ तेज॒ स्तेजो॑ मे यच्छ । \newline
52. मे॒ य॒च्छ॒ य॒च्छ॒ मे॒ मे॒ य॒च्छ॒ पृ॒थि॒वीम् पृ॑थि॒वीं ॅय॑च्छ मे मे यच्छ पृथि॒वीम् । \newline
53. य॒च्छ॒ पृ॒थि॒वीम् पृ॑थि॒वीं ॅय॑च्छ यच्छ पृथि॒वीं ॅय॑च्छ यच्छ पृथि॒वीं ॅय॑च्छ यच्छ पृथि॒वीं ॅय॑च्छ । \newline
54. पृ॒थि॒वीं ॅय॑च्छ यच्छ पृथि॒वीम् पृ॑थि॒वीं ॅय॑च्छ पृथि॒व्यै पृ॑थि॒व्यै य॑च्छ पृथि॒वीम् पृ॑थि॒वीं ॅय॑च्छ पृथि॒व्यै । \newline
55. य॒च्छ॒ पृ॒थि॒व्यै पृ॑थि॒व्यै य॑च्छ यच्छ पृथि॒व्यै मा॑ मा पृथि॒व्यै य॑च्छ यच्छ पृथि॒व्यै मा᳚ । \newline
\pagebreak
\markright{ TS 5.7.6.2  \hfill https://www.vedavms.in \hfill}

\section{ TS 5.7.6.2 }

\textbf{TS 5.7.6.2 } \newline
\textbf{Samhita Paata} \newline

पृथि॒व्यै मा॑ पाहि॒ ज्योति॑रसि॒ ज्योति॑र्मे यच्छा॒न्तरि॑क्षं ॅयच्छा॒न्तरि॑क्षान्मा पाहि॒ सुव॑रसि॒ सुव॑र्मे यच्छ॒ दिवं॑ ॅयच्छ दि॒वो मा॑ पा॒हीत्या॑है॒ताभि॒र्वा इ॒मे लो॒का विधृ॑ता॒ यदे॒ता उ॑प॒दधा᳚त्ये॒षां ॅलो॒कानां॒ ॅविधृ॑त्यै स्वयमातृ॒ण्णा उ॑प॒धाय॑ हिरण्येष्ट॒का उप॑दधाती॒मे वै लो॒काः स्व॑यमातृ॒ण्णा ज्योति॒र्॒.हिर॑ण्यं॒ ॅयथ् स्व॑यमातृ॒ण्णा उ॑प॒धाय॑ - [  ] \newline

\textbf{Pada Paata} \newline

पृ॒थि॒व्यै । मा॒ । पा॒हि॒ । ज्योतिः॑ । अ॒सि॒ । ज्योतिः॑ । मे॒ । य॒च्छ॒ । अ॒न्तरि॑क्षम् । य॒च्छ॒ । अ॒न्तरि॑क्षात् । मा॒ । पा॒हि॒ । सुवः॑ । अ॒सि॒ । सुवः॑ । मे॒ । य॒च्छ॒ । दिव᳚म् । य॒च्छ॒ । दि॒वः । मा॒ । पा॒हि॒ । इति॑ । आ॒ह॒ । ए॒ताभिः॑ । वै । इ॒मे । लो॒काः । विधृ॑ता॒ इति॒ वि - धृ॒ताः॒ । यत् । ए॒ताः । उ॒प॒दधा॒तीत्यु॑प - दधा॑ति । ए॒षाम् । लो॒काना᳚म् । विधृ॑त्या॒ इति॒ वि - धृ॒त्यै॒ । स्व॒य॒मा॒तृ॒ण्णा इति॑ स्वयं - आ॒तृ॒ण्णाः । उ॒प॒धायेत्यु॑प - धाय॑ । हि॒र॒ण्ये॒ष्ट॒का इति॑ हिरण्य - इ॒ष्ट॒काः । उपेति॑ । द॒धा॒ति॒ । इ॒मे । वै । लो॒काः । स्व॒य॒मा॒तृ॒ण्णा इति॑ स्वयं-आ॒तृ॒ण्णाः । ज्योतिः॑ । हिर॑ण्यम् । यत् । स्व॒य॒मा॒तृ॒ण्णा इति॑ स्वयं - आ॒तृ॒ण्णाः । उ॒प॒धायेत्यु॑प - धाय॑ ।  \newline


\textbf{Krama Paata} \newline

पृ॒थि॒व्यै मा᳚ । मा॒ पा॒हि॒ । पा॒हि॒ ज्योतिः॑ । ज्योति॑रसि । अ॒सि॒ ज्योतिः॑ । ज्योति॑र् मे । मे॒ य॒च्छ॒ । य॒च्छा॒न्तरि॑क्षम् । अ॒न्तरि॑क्षम् ॅयच्छ । य॒च्छा॒न्तरि॑क्षात् । अ॒न्तरि॑क्षान् मा । मा॒ पा॒हि॒ । पा॒हि॒ सुवः॑ । सुव॑रसि । अ॒सि॒ सुवः॑ । सुव॑र् मे । मे॒ य॒च्छ॒ । य॒च्छ॒ दिव᳚म् । दिव॑म् ॅयच्छ । य॒च्छ॒ दि॒वः । दि॒वो मा᳚ । मा॒ पा॒हि॒ । पा॒हीति॑ । इत्या॑ह । आ॒है॒ताभिः॑ । ए॒ताभि॒र् वै । वा इ॒मे । इ॒मे लो॒काः । लो॒का विधृ॑ताः । विधृ॑ता॒ यत् । विधृ॑ता॒ इति॒ वि - धृ॒ताः॒ । यदे॒ताः । ए॒ता उ॑प॒दधा॑ति । उ॒प॒दधा᳚त्ये॒षाम् । उ॒प॒दधा॒तीत्यु॑प - दधा॑ति । ए॒षाम् ॅलो॒काना᳚म् । लो॒काना॒म् ॅविधृ॑त्यै । विधृ॑त्यै स्वयमातृ॒ण्णाः । विधृ॑त्या॒ इति॒ वि - धृ॒त्यै॒ । स्व॒य॒मा॒तृ॒ण्णा उ॑प॒धाय॑ । स्व॒य॒मा॒तृ॒ण्णा इति॑ स्वयम् - आ॒तृ॒ण्णाः । उ॒प॒धाय॑ हिरण्येष्ट॒काः । उ॒प॒धायेत्यु॑प - धाय॑ । हि॒र॒ण्ये॒ष्ट॒का उप॑ । हि॒र॒ण्ये॒ष्ट॒का इति॑ हिरण्य - इ॒ष्ट॒काः । उप॑ दधाति । द॒धा॒ती॒मे । इ॒मे वै । वै लो॒काः । लो॒काः स्व॑यमातृ॒ण्णाः । स्व॒य॒मा॒तृ॒ण्णा ज्योतिः॑ । स्व॒य॒मा॒तृ॒ण्णा इति॑ स्वयम् - आ॒तृ॒ण्णाः । ज्योति॒र्॒. हिर॑ण्यम् । हिर॑ण्य॒म् ॅयत् । यथ् स्व॑यमातृ॒ण्णाः । स्व॒य॒मा॒तृ॒ण्णा उ॑प॒धाय॑ । स्व॒य॒मा॒तृ॒ण्णा इति॑ स्वयम् - आ॒तृ॒ण्णाः । उ॒प॒धाय॑ हिरण्येष्ट॒काः । उ॒प॒धायेत्यु॑प - धाय॑ \newline

\textbf{Jatai Paata} \newline

1. पृ॒थि॒व्यै मा॑ मा पृथि॒व्यै पृ॑थि॒व्यै मा᳚ । \newline
2. मा॒ पा॒हि॒ पा॒हि॒ मा॒ मा॒ पा॒हि॒ । \newline
3. पा॒हि॒ ज्योति॒र् ज्योतिः॑ पाहि पाहि॒ ज्योतिः॑ । \newline
4. ज्योति॑ रस्यसि॒ ज्योति॒र् ज्योति॑ रसि । \newline
5. अ॒सि॒ ज्योति॒र् ज्योति॑ रस्यसि॒ ज्योतिः॑ । \newline
6. ज्योति॑र् मे मे॒ ज्योति॒र् ज्योति॑र् मे । \newline
7. मे॒ य॒च्छ॒ य॒च्छ॒ मे॒ मे॒ य॒च्छ॒ । \newline
8. य॒च्छा॒न्तरि॑क्ष म॒न्तरि॑क्षं ॅयच्छ यच्छा॒न्तरि॑क्षम् । \newline
9. अ॒न्तरि॑क्षं ॅयच्छ यच्छा॒न्तरि॑क्ष म॒न्तरि॑क्षं ॅयच्छ । \newline
10. य॒च्छा॒न्तरि॑क्षा द॒न्तरि॑क्षाद् यच्छ यच्छा॒न्तरि॑क्षात् । \newline
11. अ॒न्तरि॑क्षान् मा मा॒ ऽन्तरि॑क्षा द॒न्तरि॑क्षान् मा । \newline
12. मा॒ पा॒हि॒ पा॒हि॒ मा॒ मा॒ पा॒हि॒ । \newline
13. पा॒हि॒ सुवः॒ सुवः॑ पाहि पाहि॒ सुवः॑ । \newline
14. सुव॑ रस्यसि॒ सुवः॒ सुव॑ रसि । \newline
15. अ॒सि॒ सुवः॒ सुव॑ रस्यसि॒ सुवः॑ । \newline
16. सुव॑र् मे मे॒ सुवः॒ सुव॑र् मे । \newline
17. मे॒ य॒च्छ॒ य॒च्छ॒ मे॒ मे॒ य॒च्छ॒ । \newline
18. य॒च्छ॒ दिव॒म् दिवं॑ ॅयच्छ यच्छ॒ दिव᳚म् । \newline
19. दिवं॑ ॅयच्छ यच्छ॒ दिव॒म् दिवं॑ ॅयच्छ । \newline
20. य॒च्छ॒ दि॒वो दि॒वो य॑च्छ यच्छ दि॒वः । \newline
21. दि॒वो मा॑ मा दि॒वो दि॒वो मा᳚ । \newline
22. मा॒ पा॒हि॒ पा॒हि॒ मा॒ मा॒ पा॒हि॒ । \newline
23. पा॒हीतीति॑ पाहि पा॒हीति॑ । \newline
24. इत्या॑हा॒हे तीत्या॑ह । \newline
25. आ॒है॒ताभि॑ रे॒ताभि॑ राहा है॒ताभिः॑ । \newline
26. ए॒ताभि॒र् वै वा ए॒ताभि॑ रे॒ताभि॒र् वै । \newline
27. वा इ॒म इ॒मे वै वा इ॒मे । \newline
28. इ॒मे लो॒का लो॒का इ॒म इ॒मे लो॒काः । \newline
29. लो॒का विधृ॑ता॒ विधृ॑ता लो॒का लो॒का विधृ॑ताः । \newline
30. विधृ॑ता॒ यद् यद् विधृ॑ता॒ विधृ॑ता॒ यत् । \newline
31. विधृ॑ता॒ इति॒ वि - धृ॒ताः॒ । \newline
32. यदे॒ता ए॒ता यद् यदे॒ताः । \newline
33. ए॒ता उ॑प॒दधा᳚ त्युप॒दधा᳚ त्ये॒ता ए॒ता उ॑प॒दधा॑ति । \newline
34. उ॒प॒दधा᳚ त्ये॒षा मे॒षा मु॑प॒दधा᳚ त्युप॒दधा᳚ त्ये॒षाम् । \newline
35. उ॒प॒दधा॒तीत्यु॑प - दधा॑ति । \newline
36. ए॒षाम् ॅलो॒काना᳚म् ॅलो॒काना॑ मे॒षा मे॒षाम् ॅलो॒काना᳚म् । \newline
37. लो॒कानां॒ ॅविधृ॑त्यै॒ विधृ॑त्यै लो॒काना᳚म् ॅलो॒कानां॒ ॅविधृ॑त्यै । \newline
38. विधृ॑त्यै स्वयमातृ॒ण्णाः स्व॑यमातृ॒ण्णा विधृ॑त्यै॒ विधृ॑त्यै स्वयमातृ॒ण्णाः । \newline
39. विधृ॑त्या॒ इति॒ वि - धृ॒त्यै॒ । \newline
40. स्व॒य॒मा॒तृ॒ण्णा उ॑प॒धायो॑ प॒धाय॑ स्वयमातृ॒ण्णाः स्व॑यमातृ॒ण्णा उ॑प॒धाय॑ । \newline
41. स्व॒य॒मा॒तृ॒ण्णा इति॑ स्वयं - आ॒तृ॒ण्णाः । \newline
42. उ॒प॒धाय॑ हिरण्येष्ट॒का हि॑रण्येष्ट॒का उ॑प॒धायो॑ प॒धाय॑ हिरण्येष्ट॒काः । \newline
43. उ॒प॒धायेत्यु॑प - धाय॑ । \newline
44. हि॒र॒ण्ये॒ष्ट॒का उपोप॑ हिरण्येष्ट॒का हि॑रण्येष्ट॒का उप॑ । \newline
45. हि॒र॒ण्ये॒ष्ट॒का इति॑ हिरण्य - इ॒ष्ट॒काः । \newline
46. उप॑ दधाति दधा॒ त्युपोप॑ दधाति । \newline
47. द॒धा॒ ती॒म इ॒मे द॑धाति दधा ती॒मे । \newline
48. इ॒मे वै वा इ॒म इ॒मे वै । \newline
49. वै लो॒का लो॒का वै वै लो॒काः । \newline
50. लो॒काः स्व॑यमातृ॒ण्णाः स्व॑यमातृ॒ण्णा लो॒का लो॒काः स्व॑यमातृ॒ण्णाः । \newline
51. स्व॒य॒मा॒तृ॒ण्णा ज्योति॒र् ज्योतिः॑ स्वयमातृ॒ण्णाः स्व॑यमातृ॒ण्णा ज्योतिः॑ । \newline
52. स्व॒य॒मा॒तृ॒ण्णा इति॑ स्वयं - आ॒तृ॒ण्णाः । \newline
53. ज्योति॒र्॒. हिर॑ण्यꣳ॒॒ हिर॑ण्य॒म् ज्योति॒र् ज्योति॒र्॒. हिर॑ण्यम् । \newline
54. हिर॑ण्यं॒ ॅयद् यद्धिर॑ण्यꣳ॒॒ हिर॑ण्यं॒ ॅयत् । \newline
55. यथ् स्व॑यमातृ॒ण्णाः स्व॑यमातृ॒ण्णा यद् यथ् स्व॑यमातृ॒ण्णाः । \newline
56. स्व॒य॒मा॒तृ॒ण्णा उ॑प॒धायो॑ प॒धाय॑ स्वयमातृ॒ण्णाः स्व॑यमातृ॒ण्णा उ॑प॒धाय॑ । \newline
57. स्व॒य॒मा॒तृ॒ण्णा इति॑ स्वयं - आ॒तृ॒ण्णाः । \newline
58. उ॒प॒धाय॑ हिरण्येष्ट॒का हि॑रण्येष्ट॒का उ॑प॒धायो॑ प॒धाय॑ हिरण्येष्ट॒काः । \newline
59. उ॒प॒धायेत्यु॑प - धाय॑ । \newline

\textbf{Ghana Paata } \newline

1. पृ॒थि॒व्यै मा॑ मा पृथि॒व्यै पृ॑थि॒व्यै मा॑ पाहि पाहि मा पृथि॒व्यै पृ॑थि॒व्यै मा॑ पाहि । \newline
2. मा॒ पा॒हि॒ पा॒हि॒ मा॒ मा॒ पा॒हि॒ ज्योति॒र् ज्योतिः॑ पाहि मा मा पाहि॒ ज्योतिः॑ । \newline
3. पा॒हि॒ ज्योति॒र् ज्योतिः॑ पाहि पाहि॒ ज्योति॑ रस्यसि॒ ज्योतिः॑ पाहि पाहि॒ ज्योति॑ रसि । \newline
4. ज्योति॑ रस्यसि॒ ज्योति॒र् ज्योति॑ रसि॒ ज्योति॒र् ज्योति॑ रसि॒ ज्योति॒र् ज्योति॑ रसि॒ ज्योतिः॑ । \newline
5. अ॒सि॒ ज्योति॒र् ज्योति॑ रस्यसि॒ ज्योति॑र् मे मे॒ ज्योति॑ रस्यसि॒ ज्योति॑र् मे । \newline
6. ज्योति॑र् मे मे॒ ज्योति॒र् ज्योति॑र् मे यच्छ यच्छ मे॒ ज्योति॒र् ज्योति॑र् मे यच्छ । \newline
7. मे॒ य॒च्छ॒ य॒च्छ॒ मे॒ मे॒ य॒च्छा॒ न्तरि॑क्ष म॒न्तरि॑क्षं ॅयच्छ मे मे यच्छा॒ न्तरि॑क्षम् । \newline
8. य॒च्छा॒ न्तरि॑क्ष म॒न्तरि॑क्षं ॅयच्छ यच्छा॒ न्तरि॑क्षं ॅयच्छ यच्छा॒ न्तरि॑क्षं ॅयच्छ यच्छा॒ न्तरि॑क्षं ॅयच्छ । \newline
9. अ॒न्तरि॑क्षं ॅयच्छ यच्छा॒ न्तरि॑क्ष म॒न्तरि॑क्षं ॅयच्छा॒ न्तरि॑क्षा द॒न्तरि॑क्षाद् यच्छा॒ न्तरि॑क्ष म॒न्तरि॑क्षं ॅयच्छा॒ न्तरि॑क्षात् । \newline
10. य॒च्छा॒ न्तरि॑क्षा द॒न्तरि॑क्षाद् यच्छ यच्छा॒ न्तरि॑क्षान् मा मा॒ ऽन्तरि॑क्षाद् यच्छ यच्छा॒ न्तरि॑क्षान् मा । \newline
11. अ॒न्तरि॑क्षान् मा मा॒ ऽन्तरि॑क्षा द॒न्तरि॑क्षान् मा पाहि पाहि मा॒ ऽन्तरि॑क्षा द॒न्तरि॑क्षान् मा पाहि । \newline
12. मा॒ पा॒हि॒ पा॒हि॒ मा॒ मा॒ पा॒हि॒ सुवः॒ सुवः॑ पाहि मा मा पाहि॒ सुवः॑ । \newline
13. पा॒हि॒ सुवः॒ सुवः॑ पाहि पाहि॒ सुव॑ रस्यसि॒ सुवः॑ पाहि पाहि॒ सुव॑ रसि । \newline
14. सुव॑ रस्यसि॒ सुवः॒ सुव॑ रसि॒ सुवः॒ सुव॑ रसि॒ सुवः॒ सुव॑ रसि॒ सुवः॑ । \newline
15. अ॒सि॒ सुवः॒ सुव॑ रस्यसि॒ सुव॑र् मे मे॒ सुव॑ रस्यसि॒ सुव॑र् मे । \newline
16. सुव॑र् मे मे॒ सुवः॒ सुव॑र् मे यच्छ यच्छ मे॒ सुवः॒ सुव॑र् मे यच्छ । \newline
17. मे॒ य॒च्छ॒ य॒च्छ॒ मे॒ मे॒ य॒च्छ॒ दिव॒म् दिवं॑ ॅयच्छ मे मे यच्छ॒ दिव᳚म् । \newline
18. य॒च्छ॒ दिव॒म् दिवं॑ ॅयच्छ यच्छ॒ दिवं॑ ॅयच्छ यच्छ॒ दिवं॑ ॅयच्छ यच्छ॒ दिवं॑ ॅयच्छ । \newline
19. दिवं॑ ॅयच्छ यच्छ॒ दिव॒म् दिवं॑ ॅयच्छ दि॒वो दि॒वो य॑च्छ॒ दिव॒म् दिवं॑ ॅयच्छ दि॒वः । \newline
20. य॒च्छ॒ दि॒वो दि॒वो य॑च्छ यच्छ दि॒वो मा॑ मा दि॒वो य॑च्छ यच्छ दि॒वो मा᳚ । \newline
21. दि॒वो मा॑ मा दि॒वो दि॒वो मा॑ पाहि पाहि मा दि॒वो दि॒वो मा॑ पाहि । \newline
22. मा॒ पा॒हि॒ पा॒हि॒ मा॒ मा॒ पा॒हीतीति॑ पाहि मा मा पा॒हीति॑ । \newline
23. पा॒हीतीति॑ पाहि पा॒हीत्या॑ हा॒हेति॑ पाहि पा॒हीत्या॑ह । \newline
24. इत्या॑हा॒हे तीत्या॑है॒ ताभि॑रे॒ ताभि॑ रा॒हे तीत्या॑ है॒ताभिः॑ । \newline
25. आ॒है॒ ताभि॑ रे॒ताभि॑ राहा है॒ताभि॒र् वै वा ए॒ताभि॑ राहा है॒ताभि॒र् वै । \newline
26. ए॒ताभि॒र् वै वा ए॒ताभि॑ रे॒ताभि॒र् वा इ॒म इ॒मे वा ए॒ताभि॑ रे॒ताभि॒र् वा इ॒मे । \newline
27. वा इ॒म इ॒मे वै वा इ॒मे लो॒का लो॒का इ॒मे वै वा इ॒मे लो॒काः । \newline
28. इ॒मे लो॒का लो॒का इ॒म इ॒मे लो॒का विधृ॑ता॒ विधृ॑ता लो॒का इ॒म इ॒मे लो॒का विधृ॑ताः । \newline
29. लो॒का विधृ॑ता॒ विधृ॑ता लो॒का लो॒का विधृ॑ता॒ यद् यद् विधृ॑ता लो॒का लो॒का विधृ॑ता॒ यत् । \newline
30. विधृ॑ता॒ यद् यद् विधृ॑ता॒ विधृ॑ता॒ यदे॒ता ए॒ता यद् विधृ॑ता॒ विधृ॑ता॒ यदे॒ताः । \newline
31. विधृ॑ता॒ इति॒ वि - धृ॒ताः॒ । \newline
32. यदे॒ता ए॒ता यद् यदे॒ता उ॑प॒दधा᳚ त्युप॒दधा᳚ त्ये॒ता यद् यदे॒ता उ॑प॒दधा॑ति । \newline
33. ए॒ता उ॑प॒दधा᳚ त्युप॒दधा᳚ त्ये॒ता ए॒ता उ॑प॒दधा᳚ त्ये॒षा मे॒षा मु॑प॒दधा᳚ त्ये॒ता ए॒ता उ॑प॒दधा᳚ त्ये॒षाम् । \newline
34. उ॒प॒दधा᳚ त्ये॒षा मे॒षा मु॑प॒दधा᳚ त्युप॒दधा᳚ त्ये॒षाम् ॅलो॒काना᳚म् ॅलो॒काना॑ मे॒षा मु॑प॒दधा᳚ त्युप॒दधा᳚ त्ये॒षाम् ॅलो॒काना᳚म् । \newline
35. उ॒प॒दधा॒तीत्यु॑प - दधा॑ति । \newline
36. ए॒षाम् ॅलो॒काना᳚म् ॅलो॒काना॑ मे॒षा मे॒षाम् ॅलो॒कानां॒ ॅविधृ॑त्यै॒ विधृ॑त्यै लो॒काना॑ मे॒षा मे॒षाम् ॅलो॒कानां॒ ॅविधृ॑त्यै । \newline
37. लो॒कानां॒ ॅविधृ॑त्यै॒ विधृ॑त्यै लो॒काना᳚म् ॅलो॒कानां॒ ॅविधृ॑त्यै स्वयमातृ॒ण्णाः स्व॑यमातृ॒ण्णा विधृ॑त्यै लो॒काना᳚म् ॅलो॒कानां॒ ॅविधृ॑त्यै स्वयमातृ॒ण्णाः । \newline
38. विधृ॑त्यै स्वयमातृ॒ण्णाः स्व॑यमातृ॒ण्णा विधृ॑त्यै॒ विधृ॑त्यै स्वयमातृ॒ण्णा उ॑प॒धा यो॑प॒धाय॑ स्वयमातृ॒ण्णा विधृ॑त्यै॒ विधृ॑त्यै स्वयमातृ॒ण्णा उ॑प॒धाय॑ । \newline
39. विधृ॑त्या॒ इति॒ वि - धृ॒त्यै॒ । \newline
40. स्व॒य॒मा॒तृ॒ण्णा उ॑प॒धा यो॑प॒धाय॑ स्वयमातृ॒ण्णाः स्व॑यमातृ॒ण्णा उ॑प॒धाय॑ हिरण्येष्ट॒का हि॑रण्येष्ट॒का उ॑प॒धाय॑ स्वयमातृ॒ण्णाः स्व॑यमातृ॒ण्णा उ॑प॒धाय॑ हिरण्येष्ट॒काः । \newline
41. स्व॒य॒मा॒तृ॒ण्णा इति॑ स्वयं - आ॒तृ॒ण्णाः । \newline
42. उ॒प॒धाय॑ हिरण्येष्ट॒का हि॑रण्येष्ट॒का उ॑प॒धा यो॑प॒धाय॑ हिरण्येष्ट॒का उपोप॑ हिरण्येष्ट॒का उ॑प॒धा यो॑प॒धाय॑ हिरण्येष्ट॒का उप॑ । \newline
43. उ॒प॒धायेत्यु॑प - धाय॑ । \newline
44. हि॒र॒ण्ये॒ष्ट॒का उपोप॑ हिरण्येष्ट॒का हि॑रण्येष्ट॒का उप॑ दधाति दधा॒ त्युप॑ हिरण्येष्ट॒का हि॑रण्येष्ट॒का उप॑ दधाति । \newline
45. हि॒र॒ण्ये॒ष्ट॒का इति॑ हिरण्य - इ॒ष्ट॒काः । \newline
46. उप॑ दधाति दधा॒ त्युपोप॑ दधाती॒म इ॒मे द॑धा॒ त्युपोप॑ दधाती॒मे । \newline
47. द॒धा॒ ती॒म इ॒मे द॑धाति दधा ती॒मे वै वा इ॒मे द॑धाति दधा ती॒मे वै । \newline
48. इ॒मे वै वा इ॒म इ॒मे वै लो॒का लो॒का वा इ॒म इ॒मे वै लो॒काः । \newline
49. वै लो॒का लो॒का वै वै लो॒काः स्व॑यमातृ॒ण्णाः स्व॑यमातृ॒ण्णा लो॒का वै वै लो॒काः स्व॑यमातृ॒ण्णाः । \newline
50. लो॒काः स्व॑यमातृ॒ण्णाः स्व॑यमातृ॒ण्णा लो॒का लो॒काः स्व॑यमातृ॒ण्णा ज्योति॒र् ज्योतिः॑ स्वयमातृ॒ण्णा लो॒का लो॒काः स्व॑यमातृ॒ण्णा ज्योतिः॑ । \newline
51. स्व॒य॒मा॒तृ॒ण्णा ज्योति॒र् ज्योतिः॑ स्वयमातृ॒ण्णाः स्व॑यमातृ॒ण्णा ज्योति॒र्॒. हिर॑ण्यꣳ॒॒ हिर॑ण्य॒म् ज्योतिः॑ स्वयमातृ॒ण्णाः स्व॑यमातृ॒ण्णा ज्योति॒र्॒. हिर॑ण्यम् । \newline
52. स्व॒य॒मा॒तृ॒ण्णा इति॑ स्वयं - आ॒तृ॒ण्णाः । \newline
53. ज्योति॒र्॒. हिर॑ण्यꣳ॒॒ हिर॑ण्य॒म् ज्योति॒र् ज्योति॒र्॒. हिर॑ण्यं॒ ॅयद् यद्धिर॑ण्य॒म् ज्योति॒र् ज्योति॒र्॒. हिर॑ण्यं॒ ॅयत् । \newline
54. हिर॑ण्यं॒ ॅयद् यद्धिर॑ण्यꣳ॒॒ हिर॑ण्यं॒ ॅयथ् स्व॑यमातृ॒ण्णाः स्व॑यमातृ॒ण्णा यद्धिर॑ण्यꣳ॒॒ हिर॑ण्यं॒ ॅयथ् स्व॑यमातृ॒ण्णाः । \newline
55. यथ् स्व॑यमातृ॒ण्णाः स्व॑यमातृ॒ण्णा यद् यथ् स्व॑यमातृ॒ण्णा उ॑प॒धा यो॑प॒धाय॑ स्वयमातृ॒ण्णा यद् यथ् स्व॑यमातृ॒ण्णा उ॑प॒धाय॑ । \newline
56. स्व॒य॒मा॒तृ॒ण्णा उ॑प॒धा यो॑प॒धाय॑ स्वयमातृ॒ण्णाः स्व॑यमातृ॒ण्णा उ॑प॒धाय॑ हिरण्येष्ट॒का हि॑रण्येष्ट॒का उ॑प॒धाय॑ स्वयमातृ॒ण्णाः स्व॑यमातृ॒ण्णा उ॑प॒धाय॑ हिरण्येष्ट॒काः । \newline
57. स्व॒य॒मा॒तृ॒ण्णा इति॑ स्वयं - आ॒तृ॒ण्णाः । \newline
58. उ॒प॒धाय॑ हिरण्येष्ट॒का हि॑रण्येष्ट॒का उ॑प॒धा यो॑प॒धाय॑ हिरण्येष्ट॒का उ॑प॒दधा᳚ त्युप॒दधा॑ति हिरण्येष्ट॒का उ॑प॒धा यो॑प॒धाय॑ हिरण्येष्ट॒का उ॑प॒दधा॑ति । \newline
59. उ॒प॒धायेत्यु॑प - धाय॑ । \newline
\pagebreak
\markright{ TS 5.7.6.3  \hfill https://www.vedavms.in \hfill}

\section{ TS 5.7.6.3 }

\textbf{TS 5.7.6.3 } \newline
\textbf{Samhita Paata} \newline

हिरण्येष्ट॒का उ॑प॒दधा॑ती॒-माने॒वैताभि॑-र्लो॒का-ञ्ज्योति॑ष्मतः कुरु॒तेऽथो॑ ए॒ताभि॑रे॒वास्मा॑ इ॒मे लो॒काः प्र भा᳚न्ति॒ यास्ते॑ अग्ने॒ सूर्ये॒ रुच॑ उद्य॒तो दिव॑मात॒न्वन्ति॑ र॒श्मिभिः॑ । ताभिः॒ सर्वा॑भी रु॒चे जना॑य नस्कृधि ॥ या वो॑ देवाः॒ सूर्ये॒ रुचो॒ गोष्वश्वे॑षु॒ या रुचः॑ । इन्द्रा᳚ग्नी॒ ताभिः॒ सर्वा॑भी॒ रुचं॑ नो धत्त बृहस्पते ॥ रुच॑न्नो धेहि - [  ] \newline

\textbf{Pada Paata} \newline

हि॒र॒ण्ये॒ष्ट॒का इति॑ हिरण्य - इ॒ष्ट॒काः । उ॒प॒दधा॒तीत्यु॑प - दधा॑ति । इ॒मान् । ए॒व । ए॒ताभिः॑ । लो॒कान् । ज्योति॑ष्मतः । कु॒रु॒ते॒ । अथो॒ इति॑ । ए॒ताभिः॑ । ए॒व । अ॒स्मै॒ । इ॒मे । लो॒काः । प्रेति॑ । भा॒न्ति॒ । याः । ते॒ । अ॒ग्ने॒ । सूर्ये᳚ । रुचः॑ । उ॒द्य॒त इत्यु॑त् - य॒तः । दिव᳚म् । आ॒त॒न्वन्तीत्या᳚ - त॒न्वन्ति॑ । र॒श्मिभि॒रिति॑ र॒श्मि - भिः॒ ॥ ताभिः॑ । सर्वा॑भिः । रु॒चे । जना॑य । नः॒ । कृ॒धि॒ ॥ याः । वः॒ । दे॒वाः॒ । सूर्ये᳚ । रुचः॑ । गोषु॑ । अश्वे॑षु । याः । रुचः॑ ॥ इन्द्रा᳚ग्नी॒ इतीन्द्र॑-अ॒ग्नी॒ । ताभिः॑ । सर्वा॑भिः । रुच᳚म् । नः॒ । ध॒त्त॒ । बृ॒ह॒स्प॒ते॒ ॥ रुच᳚म् । नः॒ । धे॒हि॒ ।  \newline


\textbf{Krama Paata} \newline

हि॒र॒ण्ये॒ष्ट॒का उ॑प॒दधा॑ति । हि॒र॒ण्ये॒ष्ट॒का इति॑ हिरण्य - इ॒ष्ट॒काः । उ॒प॒दधा॑ती॒मान् । उ॒प॒दधा॒तीत्यु॑प - दधा॑ति । इ॒माने॒व । ए॒वैताभिः॑ । ए॒ताभि॑र् लो॒कान् । लो॒कान् ज्योति॑ष्मतः । ज्योति॑ष्मतः कुरुते । कु॒रु॒तेऽथो᳚ । अथो॑ ए॒ताभिः॑ । अथो॒ इत्यथो᳚ । ए॒ताभि॑रे॒व । ए॒वास्मै᳚ । अ॒स्मा॒ इ॒मे । इ॒मे लो॒काः । लो॒काः प्र । प्र भा᳚न्ति । भा॒न्ति॒ याः । यास्ते᳚ । ते॒ अ॒ग्ने॒ । अ॒ग्ने॒ सूर्ये᳚ । सूर्ये॒ रुचः॑ । रुच॑ उद्य॒तः । उ॒द्य॒तो दिव᳚म् । उ॒द्य॒त इत्यु॑त् - य॒तः । दिव॑मात॒न्वन्ति॑ । आ॒त॒न्वन्ति॑ र॒श्मिभिः॑ । आ॒त॒न्वन्तीत्या᳚ - त॒न्वन्ति॑ । र॒श्मिभि॒रिति॑ र॒श्मि - भिः॒ ॥ ताभिः॒ सर्वा॑भिः । सर्वा॑भी रु॒चे । रु॒चे जना॑य । जना॑य नः । न॒स्कृ॒धि॒ । कृ॒धीति॑ कृधि ॥ या वः॑ । वो॒ दे॒वाः॒ । दे॒वाः॒ सूर्ये᳚ । सूर्ये॒ रुचः॑ । रुचो॒ गोषु॑ । गोष्वश्वे॑षु । अश्वे॑षु॒ याः । या रुचः॑ । रुच॒ इति॒ रुचः॑ ॥ इन्द्रा᳚ग्नी॒ ताभिः॑ । इन्द्रा᳚ग्नी॒ इतीन्द्र॑ - अ॒ग्नी॒ । ताभिः॒ सर्वा॑भिः । सर्वा॑भी॒ रुच᳚म् । रुच॑म् नः । नो॒ ध॒त्त॒ । ध॒त्त॒ बृ॒ह॒स्प॒ते॒ । बृ॒ह॒स्प॒ते॒ इति॑ बृहस्पते ॥ रुच॑म् नः । नो॒ धे॒हि॒ । धे॒हि॒ ब्रा॒ह्म॒णेषु॑ \newline

\textbf{Jatai Paata} \newline

1. हि॒र॒ण्ये॒ष्ट॒का उ॑प॒दधा᳚ त्युप॒दधा॑ति हिरण्येष्ट॒का हि॑रण्येष्ट॒का उ॑प॒दधा॑ति । \newline
2. हि॒र॒ण्ये॒ष्ट॒का इति॑ हिरण्य - इ॒ष्ट॒काः । \newline
3. उ॒प॒दधा॑ ती॒मा नि॒मा नु॑प॒दधा᳚ त्युप॒दधा॑ ती॒मान् । \newline
4. उ॒प॒दधा॒तीत्यु॑प - दधा॑ति । \newline
5. इ॒मा ने॒वैवे मा नि॒मा ने॒व । \newline
6. ए॒वै ताभि॑ रे॒ताभि॑ रे॒वै वैताभिः॑ । \newline
7. ए॒ताभि॑र् लो॒कान् ॅलो॒का ने॒ताभि॑ रे॒ताभि॑र् लो॒कान् । \newline
8. लो॒कान् ज्योति॑ष्मतो॒ ज्योति॑ष्मतो लो॒कान् ॅलो॒कान् ज्योति॑ष्मतः । \newline
9. ज्योति॑ष्मतः कुरुते कुरुते॒ ज्योति॑ष्मतो॒ ज्योति॑ष्मतः कुरुते । \newline
10. कु॒रु॒ते ऽथो॒ अथो॑ कुरुते कुरु॒ते ऽथो᳚ । \newline
11. अथो॑ ए॒ताभि॑ रे॒ताभि॒ रथो॒ अथो॑ ए॒ताभिः॑ । \newline
12. अथो॒ इत्यथो᳚ । \newline
13. ए॒ताभि॑ रे॒वै वैताभि॑ रे॒ताभि॑ रे॒व । \newline
14. ए॒वास्मा॑ अस्मा ए॒वै वास्मै᳚ । \newline
15. अ॒स्मा॒ इ॒म इ॒मे᳚ ऽस्मा अस्मा इ॒मे । \newline
16. इ॒मे लो॒का लो॒का इ॒म इ॒मे लो॒काः । \newline
17. लो॒काः प्र प्र लो॒का लो॒काः प्र । \newline
18. प्र भा᳚न्ति भान्ति॒ प्र प्र भा᳚न्ति । \newline
19. भा॒न्ति॒ या या भा᳚न्ति भान्ति॒ याः । \newline
20. या स्ते॑ ते॒ या या स्ते᳚ । \newline
21. ते॒ अ॒ग्ने॒ अ॒ग्ने॒ ते॒ ते॒ अ॒ग्ने॒ । \newline
22. अ॒ग्ने॒ सूर्ये॒ सूर्ये॑ अग्ने अग्ने॒ सूर्ये᳚ । \newline
23. सूर्ये॒ रुचो॒ रुचः॒ सूर्ये॒ सूर्ये॒ रुचः॑ । \newline
24. रुच॑ उद्य॒त उ॑द्य॒तो रुचो॒ रुच॑ उद्य॒तः । \newline
25. उ॒द्य॒तो दिव॒म् दिव॑ मुद्य॒त उ॑द्य॒तो दिव᳚म् । \newline
26. उ॒द्य॒त इत्यु॑त् - य॒तः । \newline
27. दिव॑ मात॒न्वन् त्या॑त॒न्वन्ति॒ दिव॒म् दिव॑ मात॒न्वन्ति॑ । \newline
28. आ॒त॒न्वन्ति॑ र॒श्मिभी॑ र॒श्मिभि॑ रात॒न्वन् त्या॑त॒न्वन्ति॑ र॒श्मिभिः॑ । \newline
29. आ॒त॒न्वन्तीत्या᳚ - त॒न्वन्ति॑ । \newline
30. र॒श्मिभि॒रिति॑ र॒श्मि - भिः॒ । \newline
31. ताभिः॒ सर्वा॑भिः॒ सर्वा॑भि॒ स्ताभि॒ स्ताभिः॒ सर्वा॑भिः । \newline
32. सर्वा॑भी रु॒चे रु॒चे सर्वा॑भिः॒ सर्वा॑भी रु॒चे । \newline
33. रु॒चे जना॑य॒ जना॑य रु॒चे रु॒चे जना॑य । \newline
34. जना॑य नो नो॒ जना॑य॒ जना॑य नः । \newline
35. न॒ स्कृ॒धि॒ कृ॒धि॒ नो॒ न॒ स्कृ॒धि॒ । \newline
36. कृ॒धीति॑ कृधि । \newline
37. या वो॑ वो॒ या या वः॑ । \newline
38. वो॒ दे॒वा॒ दे॒वा॒ वो॒ वो॒ दे॒वाः॒ । \newline
39. दे॒वाः॒ सूर्ये॒ सूर्ये॑ देवा देवाः॒ सूर्ये᳚ । \newline
40. सूर्ये॒ रुचो॒ रुचः॒ सूर्ये॒ सूर्ये॒ रुचः॑ । \newline
41. रुचो॒ गोषु॒ गोषु॒ रुचो॒ रुचो॒ गोषु॑ । \newline
42. गोष्वश्वे॒ ष्वश्वे॑षु॒ गोषु॒ गोष्वश्वे॑षु । \newline
43. अश्वे॑षु॒ या या अश्वे॒ ष्वश्वे॑षु॒ याः । \newline
44. या रुचो॒ रुचो॒ या या रुचः॑ । \newline
45. रुच॒ इति॒ रुचः॑ । \newline
46. इन्द्रा᳚ग्नी॒ ताभि॒ स्ताभि॒ रिन्द्रा᳚ग्नी॒ इन्द्रा᳚ग्नी॒ ताभिः॑ । \newline
47. इन्द्रा᳚ग्नी॒ इतीन्द्र॑ - अ॒ग्नी॒ । \newline
48. ताभिः॒ सर्वा॑भिः॒ सर्वा॑भि॒ स्ताभि॒ स्ताभिः॒ सर्वा॑भिः । \newline
49. सर्वा॑भी॒ रुचꣳ॒॒ रुचꣳ॒॒ सर्वा॑भिः॒ सर्वा॑भी॒ रुच᳚म् । \newline
50. रुच॑म् नो नो॒ रुचꣳ॒॒ रुच॑म् नः । \newline
51. नो॒ ध॒त्त॒ ध॒त्त॒ नो॒ नो॒ ध॒त्त॒ । \newline
52. ध॒त्त॒ बृ॒ह॒स्प॒ते॒ बृ॒ह॒स्प॒ते॒ ध॒त्त॒ ध॒त्त॒ बृ॒ह॒स्प॒ते॒ । \newline
53. बृ॒ह॒स्प॒त॒ इति॑ बृहस्पते । \newline
54. रुच॑म् नो नो॒ रुचꣳ॒॒ रुच॑म् नः । \newline
55. नो॒ धे॒हि॒ धे॒हि॒ नो॒ नो॒ धे॒हि॒ । \newline
56. धे॒हि॒ ब्रा॒ह्म॒णेषु॑ ब्राह्म॒णेषु॑ धेहि धेहि ब्राह्म॒णेषु॑ । \newline

\textbf{Ghana Paata } \newline

1. हि॒र॒ण्ये॒ष्ट॒का उ॑प॒दधा᳚ त्युप॒दधा॑ति हिरण्येष्ट॒का हि॑रण्येष्ट॒का उ॑प॒दधा॑ ती॒मा नि॒मा नु॑प॒दधा॑ति हिरण्येष्ट॒का हि॑रण्येष्ट॒का उ॑प॒दधा॑ ती॒मान् । \newline
2. हि॒र॒ण्ये॒ष्ट॒का इति॑ हिरण्य - इ॒ष्ट॒काः । \newline
3. उ॒प॒दधा॑ ती॒मा नि॒मा नु॑प॒दधा᳚ त्युप॒दधा॑ ती॒मा ने॒वैवेमा नु॑प॒दधा᳚ त्युप॒दधा॑ ती॒मा ने॒व । \newline
4. उ॒प॒दधा॒तीत्यु॑प - दधा॑ति । \newline
5. इ॒मा ने॒वैवे मा नि॒मा ने॒वै ताभि॑ रे॒ताभि॑ रे॒वे मा नि॒मा ने॒वै ताभिः॑ । \newline
6. ए॒वैताभि॑ रे॒ताभि॑ रे॒वैवै ताभि॑र् लो॒कान् ॅलो॒का ने॒ताभि॑ रे॒वैवै ताभि॑र् लो॒कान् । \newline
7. ए॒ताभि॑र् लो॒कान् ॅलो॒का ने॒ताभि॑ रे॒ताभि॑र् लो॒कान् ज्योति॑ष्मतो॒ ज्योति॑ष्मतो लो॒का ने॒ताभि॑ रे॒ताभि॑र् लो॒कान् ज्योति॑ष्मतः । \newline
8. लो॒कान् ज्योति॑ष्मतो॒ ज्योति॑ष्मतो लो॒कान् ॅलो॒कान् ज्योति॑ष्मतः कुरुते कुरुते॒ ज्योति॑ष्मतो लो॒कान् ॅलो॒कान् ज्योति॑ष्मतः कुरुते । \newline
9. ज्योति॑ष्मतः कुरुते कुरुते॒ ज्योति॑ष्मतो॒ ज्योति॑ष्मतः कुरु॒ते ऽथो॒ अथो॑ कुरुते॒ ज्योति॑ष्मतो॒ ज्योति॑ष्मतः कुरु॒ते ऽथो᳚ । \newline
10. कु॒रु॒ते ऽथो॒ अथो॑ कुरुते कुरु॒ते ऽथो॑ ए॒ताभि॑ रे॒ताभि॒ रथो॑ कुरुते कुरु॒ते ऽथो॑ ए॒ताभिः॑ । \newline
11. अथो॑ ए॒ताभि॑ रे॒ताभि॒ रथो॒ अथो॑ ए॒ताभि॑ रे॒वैवै ताभि॒ रथो॒ अथो॑ ए॒ताभि॑ रे॒व । \newline
12. अथो॒ इत्यथो᳚ । \newline
13. ए॒ताभि॑ रे॒वैवै ताभि॑ रे॒ताभि॑रे॒ वास्मा॑ अस्मा ए॒वै ताभि॑ रे॒ताभि॑ रे॒वास्मै᳚ । \newline
14. ए॒वास्मा॑ अस्मा ए॒वै वास्मा॑ इ॒म इ॒मे᳚ ऽस्मा ए॒वै वास्मा॑ इ॒मे । \newline
15. अ॒स्मा॒ इ॒म इ॒मे᳚ ऽस्मा अस्मा इ॒मे लो॒का लो॒का इ॒मे᳚ ऽस्मा अस्मा इ॒मे लो॒काः । \newline
16. इ॒मे लो॒का लो॒का इ॒म इ॒मे लो॒काः प्र प्र लो॒का इ॒म इ॒मे लो॒काः प्र । \newline
17. लो॒काः प्र प्र लो॒का लो॒काः प्र भा᳚न्ति भान्ति॒ प्र लो॒का लो॒काः प्र भा᳚न्ति । \newline
18. प्र भा᳚न्ति भान्ति॒ प्र प्र भा᳚न्ति॒ या या भा᳚न्ति॒ प्र प्र भा᳚न्ति॒ याः । \newline
19. भा॒न्ति॒ या या भा᳚न्ति भान्ति॒ या स्ते॑ ते॒ या भा᳚न्ति भान्ति॒ या स्ते᳚ । \newline
20. या स्ते॑ ते॒ या या स्ते॑ अग्ने अग्ने ते॒ या या स्ते॑ अग्ने । \newline
21. ते॒ अ॒ग्ने॒ अ॒ग्ने॒ ते॒ ते॒ अ॒ग्ने॒ सूर्ये॒ सूर्ये॑ अग्ने ते ते अग्ने॒ सूर्ये᳚ । \newline
22. अ॒ग्ने॒ सूर्ये॒ सूर्ये॑ अग्ने अग्ने॒ सूर्ये॒ रुचो॒ रुचः॒ सूर्ये॑ अग्ने अग्ने॒ सूर्ये॒ रुचः॑ । \newline
23. सूर्ये॒ रुचो॒ रुचः॒ सूर्ये॒ सूर्ये॒ रुच॑ उद्य॒त उ॑द्य॒तो रुचः॒ सूर्ये॒ सूर्ये॒ रुच॑ उद्य॒तः । \newline
24. रुच॑ उद्य॒त उ॑द्य॒तो रुचो॒ रुच॑ उद्य॒तो दिव॒म् दिव॑ मुद्य॒तो रुचो॒ रुच॑ उद्य॒तो दिव᳚म् । \newline
25. उ॒द्य॒तो दिव॒म् दिव॑ मुद्य॒त उ॑द्य॒तो दिव॑ मात॒न्वन् त्या॑त॒न्वन्ति॒ दिव॑ मुद्य॒त उ॑द्य॒तो दिव॑ मात॒न्वन्ति॑ । \newline
26. उ॒द्य॒त इत्यु॑त् - य॒तः । \newline
27. दिव॑ मात॒न्वन् त्या॑त॒न्वन्ति॒ दिव॒म् दिव॑ मात॒न्वन्ति॑ र॒श्मिभी॑ र॒श्मिभि॑ रात॒न्वन्ति॒ दिव॒म् दिव॑ मात॒न्वन्ति॑ र॒श्मिभिः॑ । \newline
28. आ॒त॒न्वन्ति॑ र॒श्मिभी॑ र॒श्मिभि॑ रात॒न्वन् त्या॑त॒न्वन्ति॑ र॒श्मिभिः॑ । \newline
29. आ॒त॒न्वन्तीत्या᳚ - त॒न्वन्ति॑ । \newline
30. र॒श्मिभि॒रिति॑ र॒श्मि - भिः॒ । \newline
31. ताभिः॒ सर्वा॑भिः॒ सर्वा॑भि॒ स्ताभि॒ स्ताभिः॒ सर्वा॑भी रु॒चे रु॒चे सर्वा॑भि॒ स्ताभि॒ स्ताभिः॒ सर्वा॑भी रु॒चे । \newline
32. सर्वा॑भी रु॒चे रु॒चे सर्वा॑भिः॒ सर्वा॑भी रु॒चे जना॑य॒ जना॑य रु॒चे सर्वा॑भिः॒ सर्वा॑भी रु॒चे जना॑य । \newline
33. रु॒चे जना॑य॒ जना॑य रु॒चे रु॒चे जना॑य नो नो॒ जना॑य रु॒चे रु॒चे जना॑य नः । \newline
34. जना॑य नो नो॒ जना॑य॒ जना॑य न स्कृधि कृधि नो॒ जना॑य॒ जना॑य न स्कृधि । \newline
35. न॒ स्कृ॒धि॒ कृ॒धि॒ नो॒ न॒ स्कृ॒धि॒ । \newline
36. कृ॒धीति॑ कृधि । \newline
37. या वो॑ वो॒ या या वो॑ देवा देवा वो॒ या या वो॑ देवाः । \newline
38. वो॒ दे॒वा॒ दे॒वा॒ वो॒ वो॒ दे॒वाः॒ सूर्ये॒ सूर्ये॑ देवा वो वो देवाः॒ सूर्ये᳚ । \newline
39. दे॒वाः॒ सूर्ये॒ सूर्ये॑ देवा देवाः॒ सूर्ये॒ रुचो॒ रुचः॒ सूर्ये॑ देवा देवाः॒ सूर्ये॒ रुचः॑ । \newline
40. सूर्ये॒ रुचो॒ रुचः॒ सूर्ये॒ सूर्ये॒ रुचो॒ गोषु॒ गोषु॒ रुचः॒ सूर्ये॒ सूर्ये॒ रुचो॒ गोषु॑ । \newline
41. रुचो॒ गोषु॒ गोषु॒ रुचो॒ रुचो॒ गो ष्वश्वे॒ ष्वश्वे॑षु॒ गोषु॒ रुचो॒ रुचो॒ गो ष्वश्वे॑षु । \newline
42. गो ष्वश्वे॒ ष्वश्वे॑षु॒ गोषु॒ गो ष्वश्वे॑षु॒ या या अश्वे॑षु॒ गोषु॒ गो ष्वश्वे॑षु॒ याः । \newline
43. अश्वे॑षु॒ या या अश्वे॒ ष्वश्वे॑षु॒ या रुचो॒ रुचो॒ या अश्वे॒ ष्वश्वे॑षु॒ या रुचः॑ । \newline
44. या रुचो॒ रुचो॒ या या रुचः॑ । \newline
45. रुच॒ इति॒ रुचः॑ । \newline
46. इन्द्रा᳚ग्नी॒ ताभि॒ स्ताभि॒ रिन्द्रा᳚ग्नी॒ इन्द्रा᳚ग्नी॒ ताभिः॒ सर्वा॑भिः॒ सर्वा॑भि॒ स्ताभि॒ रिन्द्रा᳚ग्नी॒ इन्द्रा᳚ग्नी॒ ताभिः॒ सर्वा॑भिः । \newline
47. इन्द्रा᳚ग्नी॒ इतीन्द्र॑ - अ॒ग्नी॒ । \newline
48. ताभिः॒ सर्वा॑भिः॒ सर्वा॑भि॒ स्ताभि॒ स्ताभिः॒ सर्वा॑भी॒ रुचꣳ॒॒ रुचꣳ॒॒ सर्वा॑भि॒ स्ताभि॒ स्ताभिः॒ सर्वा॑भी॒ रुच᳚म् । \newline
49. सर्वा॑भी॒ रुचꣳ॒॒ रुचꣳ॒॒ सर्वा॑भिः॒ सर्वा॑भी॒ रुच॑म् नो नो॒ रुचꣳ॒॒ सर्वा॑भिः॒ सर्वा॑भी॒ रुच॑म् नः । \newline
50. रुच॑म् नो नो॒ रुचꣳ॒॒ रुच॑म् नो धत्त धत्त नो॒ रुचꣳ॒॒ रुच॑म् नो धत्त । \newline
51. नो॒ ध॒त्त॒ ध॒त्त॒ नो॒ नो॒ ध॒त्त॒ बृ॒ह॒स्प॒ते॒ बृ॒ह॒स्प॒ते॒ ध॒त्त॒ नो॒ नो॒ ध॒त्त॒ बृ॒ह॒स्प॒ते॒ । \newline
52. ध॒त्त॒ बृ॒ह॒स्प॒ते॒ बृ॒ह॒स्प॒ते॒ ध॒त्त॒ ध॒त्त॒ बृ॒ह॒स्प॒ते॒ । \newline
53. बृ॒ह॒स्प॒त॒ इति॑ बृहस्पते । \newline
54. रुच॑म् नो नो॒ रुचꣳ॒॒ रुच॑म् नो धेहि धेहि नो॒ रुचꣳ॒॒ रुच॑म् नो धेहि । \newline
55. नो॒ धे॒हि॒ धे॒हि॒ नो॒ नो॒ धे॒हि॒ ब्रा॒ह्म॒णेषु॑ ब्राह्म॒णेषु॑ धेहि नो नो धेहि ब्राह्म॒णेषु॑ । \newline
56. धे॒हि॒ ब्रा॒ह्म॒णेषु॑ ब्राह्म॒णेषु॑ धेहि धेहि ब्राह्म॒णेषु॒ रुचꣳ॒॒ रुच॑म् ब्राह्म॒णेषु॑ धेहि धेहि ब्राह्म॒णेषु॒ रुच᳚म् । \newline
\pagebreak
\markright{ TS 5.7.6.4  \hfill https://www.vedavms.in \hfill}

\section{ TS 5.7.6.4 }

\textbf{TS 5.7.6.4 } \newline
\textbf{Samhita Paata} \newline

ब्राह्म॒णेषु॒ रुचꣳ॒॒ राज॑सु नस्कृधि । रुचं॑ ॅवि॒श्ये॑षु शू॒द्रेषु॒ मयि॑ धेहि रु॒चा रुचं᳚ ॥ द्वे॒धा वा अ॒ग्निं चि॑क्या॒नस्य॒ यश॑ इन्द्रि॒यं ग॑च्छत्य॒ग्निं ॅवा॑ चि॒तमी॑जा॒नं ॅवा॒ यदे॒ता आहु॑तीर्जु॒होत्या॒त्मन्ने॒व यश॑ इन्द्रि॒यं ध॑त्त ईश्व॒रो वा ए॒ष आर्ति॒मार्तो॒र्यो᳚ऽग्निं चि॒न्वन्न॑धि॒ क्राम॑ति॒ तत्त्वा॑ यामि॒ ब्रह्म॑णा॒ वन्द॑मान॒ इति॑ वारु॒ण्यर्चा - [  ] \newline

\textbf{Pada Paata} \newline

ब्रा॒ह्म॒णेषु॑ । रुच᳚म् । राज॒स्विति॒ राज॑ - सु॒ । नः॒ । कृ॒धि॒ ॥ रुच᳚म् । वि॒श्ये॑षु । शू॒द्रेषु॑ । मयि॑ । धे॒हि॒ । रु॒चा । रुच᳚म् ॥ द्वे॒धा । वै । अ॒ग्निम् । चि॒क्या॒नस्य॑ । यशः॑ । इ॒न्द्रि॒यम् । ग॒च्छ॒ति॒ । अ॒ग्निम् । वा॒ । चि॒तम् । ई॒जा॒नम् । वा॒ । यत् । ए॒ताः । आहु॑ती॒रित्या - हु॒तीः॒ । जु॒होति॑ । आ॒त्मन्न् । ए॒व । यशः॑ । इ॒न्द्रि॒यम् । ध॒त्ते॒ । ई॒श्व॒रः । वै । ए॒षः । आर्ति᳚म् । आर्तो॒रित्या - अ॒र्तोः॒ । यः । अ॒ग्निम् । चि॒न्वन्न् । अ॒धि॒क्राम॒तीत्य॑धि - क्राम॑ति । तत् । त्वा॒ । या॒मि॒ । ब्रह्म॑णा । वन्द॑मानः । इति॑ । वा॒रु॒ण्या । ऋ॒चा ।  \newline


\textbf{Krama Paata} \newline

ब्रा॒ह्म॒णेषु॒ रुच᳚म् । रुचꣳ॒॒ राज॑सु । राज॑सु नः । राज॒स्विति॒ राज॑ - सु॒ । न॒स्कृ॒धि॒ । कृ॒धीति॑ कृधि ॥ रुच॑म् ॅवि॒श्ये॑षु । वि॒श्ये॑षु शू॒द्रेषु॑ । शू॒द्रेषु॒ मयि॑ । मयि॑ धेहि । धे॒हि॒ रु॒चा । रु॒चा रुच᳚म् । रुच॒मिति॒ रुच᳚म् ॥ द्वे॒धा वै । वा अ॒ग्निम् । अ॒ग्निम् चि॑क्या॒नस्य॑ । चि॒क्या॒नस्य॒ यशः॑ । यश॑ इन्द्रि॒यम् । इ॒न्द्रि॒यम् ग॑च्छति । ग॒च्छ॒त्य॒ग्निम् । अ॒ग्निम् ॅवा᳚ । वा॒ चि॒तम् । चि॒तमी॑जा॒नम् । ई॒जा॒नम् ॅवा᳚ । वा॒ यत् । यदे॒ताः । ए॒ता आहु॑तीः । आहु॑तीर् जु॒होति॑ । आहु॑ती॒रित्या - हु॒तीः॒ । जु॒होत्या॒त्मन्न् । आ॒त्मन्ने॒व । ए॒व यशः॑ । यश॑ इन्द्रि॒यम् । इ॒न्द्रि॒यम् ध॑त्ते । ध॒त्त॒ ई॒श्व॒रः । ई॒श्व॒रो वै । वा ए॒षः । ए॒ष आर्ति᳚म् । आर्ति॒मार्तोः᳚ । आर्तो॒र् यः । आर्तो॒रित्या - अ॒र्तोः॒ । यो᳚ऽग्निम् । अ॒ग्निम् चि॒न्वन्न् । चि॒न्वन्न॑धि॒क्राम॑ति । अ॒धि॒क्राम॑ति॒ तत् । अ॒धि॒क्राम॒तीत्य॑धि - क्राम॑ति । तत् त्वा᳚ । त्वा॒ या॒मि॒ । या॒मि॒ ब्रह्म॑णा । ब्रह्म॑णा॒ वन्द॑मानः । वन्द॑मान॒ इति॑ । इति॑ वारु॒ण्या । वा॒रु॒ण्यर्चा । ऋ॒चा जु॑हुयात् \newline

\textbf{Jatai Paata} \newline

1. ब्रा॒ह्म॒णेषु॒ रुचꣳ॒॒ रुच॑म् ब्राह्म॒णेषु॑ ब्राह्म॒णेषु॒ रुच᳚म् । \newline
2. रुचꣳ॒॒ राज॑सु॒ राज॑सु॒ रुचꣳ॒॒ रुचꣳ॒॒ राज॑सु । \newline
3. राज॑सु नो नो॒ राज॑सु॒ राज॑सु नः । \newline
4. राज॒स्विति॒ राज॑ - सु॒ । \newline
5. न॒ स्कृ॒धि॒ कृ॒धि॒ नो॒ न॒ स्कृ॒धि॒ । \newline
6. कृ॒धीति॑ कृधि । \newline
7. रुचं॑ ॅवि॒श्ये॑षु वि॒श्ये॑षु॒ रुचꣳ॒॒ रुचं॑ ॅवि॒श्ये॑षु । \newline
8. वि॒श्ये॑षु शू॒द्रेषु॑ शू॒द्रेषु॑ वि॒श्ये॑षु वि॒श्ये॑षु शू॒द्रेषु॑ । \newline
9. शू॒द्रेषु॒ मयि॒ मयि॑ शू॒द्रेषु॑ शू॒द्रेषु॒ मयि॑ । \newline
10. मयि॑ धेहि धेहि॒ मयि॒ मयि॑ धेहि । \newline
11. धे॒हि॒ रु॒चा रु॒चा धे॑हि धेहि रु॒चा । \newline
12. रु॒चा रुचꣳ॒॒ रुचꣳ॑ रु॒चा रु॒चा रुच᳚म् । \newline
13. रुच॒मिति॒ रुच᳚म् । \newline
14. द्वे॒धा वै वै द्वे॒धा द्वे॒धा वै । \newline
15. वा अ॒ग्नि म॒ग्निं ॅवै वा अ॒ग्निम् । \newline
16. अ॒ग्निम् चि॑क्या॒नस्य॑ चिक्या॒न स्या॒ग्नि म॒ग्निम् चि॑क्या॒नस्य॑ । \newline
17. चि॒क्या॒नस्य॒ यशो॒ यश॑ श्चिक्या॒नस्य॑ चिक्या॒नस्य॒ यशः॑ । \newline
18. यश॑ इन्द्रि॒य मि॑न्द्रि॒यं ॅयशो॒ यश॑ इन्द्रि॒यम् । \newline
19. इ॒न्द्रि॒यम् ग॑च्छति गच्छ तीन्द्रि॒य मि॑न्द्रि॒यम् ग॑च्छति । \newline
20. ग॒च्छ॒ त्य॒ग्नि म॒ग्निम् ग॑च्छति गच्छ त्य॒ग्निम् । \newline
21. अ॒ग्निं ॅवा॑ वा॒ ऽग्नि म॒ग्निं ॅवा᳚ । \newline
22. वा॒ चि॒तम् चि॒तं ॅवा॑ वा चि॒तम् । \newline
23. चि॒त मी॑जा॒न मी॑जा॒नम् चि॒तम् चि॒त मी॑जा॒नम् । \newline
24. ई॒जा॒नं ॅवा॑ वेजा॒न मी॑जा॒नं ॅवा᳚ । \newline
25. वा॒ यद् यद् वा॑ वा॒ यत् । \newline
26. यदे॒ता ए॒ता यद् यदे॒ताः । \newline
27. ए॒ता आहु॑ती॒ राहु॑ती रे॒ता ए॒ता आहु॑तीः । \newline
28. आहु॑तीर् जु॒होति॑ जु॒हो त्याहु॑ती॒ राहु॑तीर् जु॒होति॑ । \newline
29. आहु॑ती॒रित्या - हु॒तीः॒ । \newline
30. जु॒हो त्या॒त्मन् ना॒त्मन् जु॒होति॑ जु॒हो त्या॒त्मन्न् । \newline
31. आ॒त्मन् ने॒वै वात्मन् ना॒त्मन् ने॒व । \newline
32. ए॒व यशो॒ यश॑ ए॒वैव यशः॑ । \newline
33. यश॑ इन्द्रि॒य मि॑न्द्रि॒यं ॅयशो॒ यश॑ इन्द्रि॒यम् । \newline
34. इ॒न्द्रि॒यम् ध॑त्ते धत्त इन्द्रि॒य मि॑न्द्रि॒यम् ध॑त्ते । \newline
35. ध॒त्त॒ ई॒श्व॒र ई᳚श्व॒रो ध॑त्ते धत्त ईश्व॒रः । \newline
36. ई॒श्व॒रो वै वा ई᳚श्व॒र ई᳚श्व॒रो वै । \newline
37. वा ए॒ष ए॒ष वै वा ए॒षः । \newline
38. ए॒ष आर्ति॒ मार्ति॑ मे॒ष ए॒ष आर्ति᳚म् । \newline
39. आर्ति॒ मार्तो॒ रार्तो॒ रार्ति॒ मार्ति॒ मार्तोः᳚ । \newline
40. आर्तो॒र् यो य आर्तो॒ रार्तो॒र् यः । \newline
41. आर्तो॒रित्या - अ॒र्तोः॒ । \newline
42. यो᳚ ऽग्नि म॒ग्निं ॅयो यो᳚ ऽग्निम् । \newline
43. अ॒ग्निम् चि॒न्वꣳ श्चि॒न्वन् न॒ग्नि म॒ग्निम् चि॒न्वन्न् । \newline
44. चि॒न्वन् न॑धि॒क्राम॑ त्यधि॒क्राम॑ति चि॒न्वꣳ श्चि॒न्वन् न॑धि॒क्राम॑ति । \newline
45. अ॒धि॒क्राम॑ति॒ तत् तद॑धि॒क्राम॑ त्यधि॒क्राम॑ति॒ तत् । \newline
46. अ॒धि॒क्राम॒तीत्य॑धि - क्राम॑ति । \newline
47. तत् त्वा᳚ त्वा॒ तत् तत् त्वा᳚ । \newline
48. त्वा॒ या॒मि॒ या॒मि॒ त्वा॒ त्वा॒ या॒मि॒ । \newline
49. या॒मि॒ ब्रह्म॑णा॒ ब्रह्म॑णा यामि यामि॒ ब्रह्म॑णा । \newline
50. ब्रह्म॑णा॒ वन्द॑मानो॒ वन्द॑मानो॒ ब्रह्म॑णा॒ ब्रह्म॑णा॒ वन्द॑मानः । \newline
51. वन्द॑मान॒ इतीति॒ वन्द॑मानो॒ वन्द॑मान॒ इति॑ । \newline
52. इति॑ वारु॒ण्या वा॑रु॒ण्येतीति॑ वारु॒ण्या । \newline
53. वा॒रु॒ण्य र्‌च र्‌चा वा॑रु॒ण्या वा॑रु॒ण्य र्‌चा । \newline
54. ऋ॒चा जु॑हुयाज् जुहुया दृ॒च र्‌चा जु॑हुयात् । \newline

\textbf{Ghana Paata } \newline

1. ब्रा॒ह्म॒णेषु॒ रुचꣳ॒॒ रुच॑म् ब्राह्म॒णेषु॑ ब्राह्म॒णेषु॒ रुचꣳ॒॒ राज॑सु॒ राज॑सु॒ रुच॑म् ब्राह्म॒णेषु॑ ब्राह्म॒णेषु॒ रुचꣳ॒॒ राज॑सु । \newline
2. रुचꣳ॒॒ राज॑सु॒ राज॑सु॒ रुचꣳ॒॒ रुचꣳ॒॒ राज॑सु नो नो॒ राज॑सु॒ रुचꣳ॒॒ रुचꣳ॒॒ राज॑सु नः । \newline
3. राज॑सु नो नो॒ राज॑सु॒ राज॑सु न स्कृधि कृधि नो॒ राज॑सु॒ राज॑सु न स्कृधि । \newline
4. राज॒स्विति॒ राज॑ - सु॒ । \newline
5. न॒ स्कृ॒धि॒ कृ॒धि॒ नो॒ न॒ स्कृ॒धि॒ । \newline
6. कृ॒धीति॑ कृधि । \newline
7. रुचं॑ ॅवि॒श्ये॑षु वि॒श्ये॑षु॒ रुचꣳ॒॒ रुचं॑ ॅवि॒श्ये॑षु शू॒द्रेषु॑ शू॒द्रेषु॑ वि॒श्ये॑षु॒ रुचꣳ॒॒ रुचं॑ ॅवि॒श्ये॑षु शू॒द्रेषु॑ । \newline
8. वि॒श्ये॑षु शू॒द्रेषु॑ शू॒द्रेषु॑ वि॒श्ये॑षु वि॒श्ये॑षु शू॒द्रेषु॒ मयि॒ मयि॑ शू॒द्रेषु॑ वि॒श्ये॑षु वि॒श्ये॑षु शू॒द्रेषु॒ मयि॑ । \newline
9. शू॒द्रेषु॒ मयि॒ मयि॑ शू॒द्रेषु॑ शू॒द्रेषु॒ मयि॑ धेहि धेहि॒ मयि॑ शू॒द्रेषु॑ शू॒द्रेषु॒ मयि॑ धेहि । \newline
10. मयि॑ धेहि धेहि॒ मयि॒ मयि॑ धेहि रु॒चा रु॒चा धे॑हि॒ मयि॒ मयि॑ धेहि रु॒चा । \newline
11. धे॒हि॒ रु॒चा रु॒चा धे॑हि धेहि रु॒चा रुचꣳ॒॒ रुचꣳ॑ रु॒चा धे॑हि धेहि रु॒चा रुच᳚म् । \newline
12. रु॒चा रुचꣳ॒॒ रुचꣳ॑ रु॒चा रु॒चा रुच᳚म् । \newline
13. रुच॒मिति॒ रुच᳚म् । \newline
14. द्वे॒धा वै वै द्वे॒धा द्वे॒धा वा अ॒ग्नि म॒ग्निं ॅवै द्वे॒धा द्वे॒धा वा अ॒ग्निम् । \newline
15. वा अ॒ग्नि म॒ग्निं ॅवै वा अ॒ग्निम् चि॑क्या॒नस्य॑ चिक्या॒नस्या॒ग्निं ॅवै वा अ॒ग्निम् चि॑क्या॒नस्य॑ । \newline
16. अ॒ग्निम् चि॑क्या॒नस्य॑ चिक्या॒नस्या॒ग्नि म॒ग्निम् चि॑क्या॒नस्य॒ यशो॒ यश॑ श्चिक्या॒नस्या॒ग्नि म॒ग्निम् चि॑क्या॒नस्य॒ यशः॑ । \newline
17. चि॒क्या॒नस्य॒ यशो॒ यश॑ श्चिक्या॒नस्य॑ चिक्या॒नस्य॒ यश॑ इन्द्रि॒य मि॑न्द्रि॒यं ॅयश॑ श्चिक्या॒नस्य॑ चिक्या॒नस्य॒ यश॑ इन्द्रि॒यम् । \newline
18. यश॑ इन्द्रि॒य मि॑न्द्रि॒यं ॅयशो॒ यश॑ इन्द्रि॒यम् ग॑च्छति गच्छतीन्द्रि॒यं ॅयशो॒ यश॑ इन्द्रि॒यम् ग॑च्छति । \newline
19. इ॒न्द्रि॒यम् ग॑च्छति गच्छतीन्द्रि॒य मि॑न्द्रि॒यम् ग॑च्छ त्य॒ग्नि म॒ग्निम् ग॑च्छतीन्द्रि॒य मि॑न्द्रि॒यम् ग॑च्छ त्य॒ग्निम् । \newline
20. ग॒च्छ॒ त्य॒ग्नि म॒ग्निम् ग॑च्छति गच्छ त्य॒ग्निं ॅवा॑ वा॒ ऽग्निम् ग॑च्छति गच्छ त्य॒ग्निं ॅवा᳚ । \newline
21. अ॒ग्निं ॅवा॑ वा॒ ऽग्नि म॒ग्निं ॅवा॑ चि॒तम् चि॒तं ॅवा॒ ऽग्नि म॒ग्निं ॅवा॑ चि॒तम् । \newline
22. वा॒ चि॒तम् चि॒तं ॅवा॑ वा चि॒त मी॑जा॒न मी॑जा॒नम् चि॒तं ॅवा॑ वा चि॒त मी॑जा॒नम् । \newline
23. चि॒त मी॑जा॒न मी॑जा॒नम् चि॒तम् चि॒त मी॑जा॒नं ॅवा॑ वेजा॒नम् चि॒तम् चि॒त मी॑जा॒नं ॅवा᳚ । \newline
24. ई॒जा॒नं ॅवा॑ वेजा॒न मी॑जा॒नं ॅवा॒ यद् यद् वे॑जा॒न मी॑जा॒नं ॅवा॒ यत् । \newline
25. वा॒ यद् यद् वा॑ वा॒ यदे॒ता ए॒ता यद् वा॑ वा॒ यदे॒ताः । \newline
26. यदे॒ता ए॒ता यद् यदे॒ता आहु॑ती॒ राहु॑ती रे॒ता यद् यदे॒ता आहु॑तीः । \newline
27. ए॒ता आहु॑ती॒ राहु॑ती रे॒ता ए॒ता आहु॑तीर् जु॒होति॑ जु॒हो त्याहु॑ती रे॒ता ए॒ता आहु॑तीर् जु॒होति॑ । \newline
28. आहु॑तीर् जु॒होति॑ जु॒हो त्याहु॑ती॒ राहु॑तीर् जु॒हो त्या॒त्मन् ना॒त्मन् जु॒हो त्याहु॑ती॒ राहु॑तीर् जु॒हो त्या॒त्मन्न् । \newline
29. आहु॑ती॒रित्या - हु॒तीः॒ । \newline
30. जु॒हो त्या॒त्मन् ना॒त्मन् जु॒होति॑ जु॒हो त्या॒त्मन् ने॒वै वात्मन् जु॒होति॑ जु॒हो त्या॒त्मन् ने॒व । \newline
31. आ॒त्मन् ने॒वै वात्मन् ना॒त्मन् ने॒व यशो॒ यश॑ ए॒वात्मन् ना॒त्मन् ने॒व यशः॑ । \newline
32. ए॒व यशो॒ यश॑ ए॒वैव यश॑ इन्द्रि॒य मि॑न्द्रि॒यं ॅयश॑ ए॒वैव यश॑ इन्द्रि॒यम् । \newline
33. यश॑ इन्द्रि॒य मि॑न्द्रि॒यं ॅयशो॒ यश॑ इन्द्रि॒यम् ध॑त्ते धत्त इन्द्रि॒यं ॅयशो॒ यश॑ इन्द्रि॒यम् ध॑त्ते । \newline
34. इ॒न्द्रि॒यम् ध॑त्ते धत्त इन्द्रि॒य मि॑न्द्रि॒यम् ध॑त्त ईश्व॒र ई᳚श्व॒रो ध॑त्त इन्द्रि॒य मि॑न्द्रि॒यम् ध॑त्त ईश्व॒रः । \newline
35. ध॒त्त॒ ई॒श्व॒र ई᳚श्व॒रो ध॑त्ते धत्त ईश्व॒रो वै वा ई᳚श्व॒रो ध॑त्ते धत्त ईश्व॒रो वै । \newline
36. ई॒श्व॒रो वै वा ई᳚श्व॒र ई᳚श्व॒रो वा ए॒ष ए॒ष वा ई᳚श्व॒र ई᳚श्व॒रो वा ए॒षः । \newline
37. वा ए॒ष ए॒ष वै वा ए॒ष आर्ति॒ मार्ति॑ मे॒ष वै वा ए॒ष आर्ति᳚म् । \newline
38. ए॒ष आर्ति॒ मार्ति॑ मे॒ष ए॒ष आर्ति॒ मार्तो॒ रार्तो॒ रार्ति॑ मे॒ष ए॒ष आर्ति॒ मार्तोः᳚ । \newline
39. आर्ति॒ मार्तो॒ रार्तो॒ रार्ति॒ मार्ति॒ मार्तो॒र् यो य आर्तो॒ रार्ति॒ मार्ति॒ मार्तो॒र् यः । \newline
40. आर्तो॒र् यो य आर्तो॒ रार्तो॒र् यो᳚ ऽग्नि म॒ग्निं ॅय आर्तो॒ रार्तो॒र् यो᳚ ऽग्निम् । \newline
41. आर्तो॒रित्या - अ॒र्तोः॒ । \newline
42. यो᳚ ऽग्नि म॒ग्निं ॅयो यो᳚ ऽग्निम् चि॒न्वꣳ श्चि॒न्वन् न॒ग्निं ॅयो यो᳚ ऽग्निम् चि॒न्वन्न् । \newline
43. अ॒ग्निम् चि॒न्वꣳ श्चि॒न्वन् न॒ग्नि म॒ग्निम् चि॒न्वन् न॑धि॒क्राम॑ त्यधि॒क्राम॑ति चि॒न्वन् न॒ग्नि म॒ग्निम् चि॒न्वन् न॑धि॒क्राम॑ति । \newline
44. चि॒न्वन् न॑धि॒क्राम॑ त्यधि॒क्राम॑ति चि॒न्वꣳ श्चि॒न्वन् न॑धि॒क्राम॑ति॒ तत् तद॑धि॒क्राम॑ति चि॒न्वꣳ श्चि॒न्वन् न॑धि॒क्राम॑ति॒ तत् । \newline
45. अ॒धि॒क्राम॑ति॒ तत् तद॑धि॒क्राम॑ त्यधि॒क्राम॑ति॒ तत् त्वा᳚ त्वा॒ तद॑धि॒क्राम॑ त्यधि॒क्राम॑ति॒ तत् त्वा᳚ । \newline
46. अ॒धि॒क्राम॒तीत्य॑धि - क्राम॑ति । \newline
47. तत् त्वा᳚ त्वा॒ तत् तत् त्वा॑ यामि यामि त्वा॒ तत् तत् त्वा॑ यामि । \newline
48. त्वा॒ या॒मि॒ या॒मि॒ त्वा॒ त्वा॒ या॒मि॒ ब्रह्म॑णा॒ ब्रह्म॑णा यामि त्वा त्वा यामि॒ ब्रह्म॑णा । \newline
49. या॒मि॒ ब्रह्म॑णा॒ ब्रह्म॑णा यामि यामि॒ ब्रह्म॑णा॒ वन्द॑मानो॒ वन्द॑मानो॒ ब्रह्म॑णा यामि यामि॒ ब्रह्म॑णा॒ वन्द॑मानः । \newline
50. ब्रह्म॑णा॒ वन्द॑मानो॒ वन्द॑मानो॒ ब्रह्म॑णा॒ ब्रह्म॑णा॒ वन्द॑मान॒ इतीति॒ वन्द॑मानो॒ ब्रह्म॑णा॒ ब्रह्म॑णा॒ वन्द॑मान॒ इति॑ । \newline
51. वन्द॑मान॒ इतीति॒ वन्द॑मानो॒ वन्द॑मान॒ इति॑ वारु॒ण्या वा॑रु॒ण्येति॒ वन्द॑मानो॒ वन्द॑मान॒ इति॑ वारु॒ण्या । \newline
52. इति॑ वारु॒ण्या वा॑रु॒ण्येतीति॑ वारु॒ण्य र्‌च र्‌चा वा॑रु॒ण्येतीति॑ वारु॒ण्य र्‌चा । \newline
53. वा॒रु॒ण्य र्‌च र्‌चा वा॑रु॒ण्या वा॑रु॒ण्य र्‌चा जु॑हुयाज् जुहुया दृ॒चा वा॑रु॒ण्या वा॑रु॒ण्य र्‌चा जु॑हुयात् । \newline
54. ऋ॒चा जु॑हुयाज् जुहुया दृ॒च र्‌चा जु॑हुया॒च् छान्तिः॒ शान्ति॑र् जुहुया दृ॒च र्‌चा जु॑हुया॒च् छान्तिः॑ । \newline
\pagebreak
\markright{ TS 5.7.6.5  \hfill https://www.vedavms.in \hfill}

\section{ TS 5.7.6.5 }

\textbf{TS 5.7.6.5 } \newline
\textbf{Samhita Paata} \newline

जु॑हुया॒च्छान्ति॑रे॒वैषा ऽग्नेर्गुप्ति॑रा॒त्मनो॑ ह॒विष्कृ॑तो॒ वा ए॒ष यो᳚ऽग्निं चि॑नु॒ते यथा॒ वै ह॒विः स्कन्द॑त्ये॒वं ॅवा ए॒ष स्क॑न्दति॒ यो᳚ऽग्निं चि॒त्वा स्त्रिय॑मु॒पैति॑ मैत्रावरु॒ण्याऽऽ*मिक्ष॑या यजेत मैत्रावरु॒णता॑-मे॒वोपै᳚त्या॒त्मनो ऽस्क॑न्दाय॒ यो वा अ॒ग्निमृ॑तु॒स्थां ॅवेद॒र्तुर्.ऋ॑तुरस्मै॒ कल्प॑मान एति॒ प्रत्ये॒व ति॑ष्ठति संॅवथ्स॒रो वा अ॒ग्निर् - [  ] \newline

\textbf{Pada Paata} \newline

जु॒हु॒या॒त् । शान्तिः॑ । ए॒व । ए॒षा । अ॒ग्नेः । गुप्तिः॑ । आ॒त्मनः॑ । ह॒विष्कृ॑त॒ इति॑ ह॒विः - कृ॒तः॒ । वै । ए॒षः । यः । अ॒ग्निम् । चि॒नु॒ते । यथा᳚ । वै । ह॒विः । स्कन्द॑ति । ए॒वम् । वै । ए॒षः । स्क॒न्द॒ति॒ । यः । अ॒ग्निम् । चि॒त्वा । स्त्रिय᳚म् । उ॒पैतीत्यु॑प - एति॑ । मै॒त्रा॒व॒रु॒ण्येति॑ मैत्रा - व॒रु॒ण्या । आ॒मिक्ष॑या । य॒जे॒त॒ । मै॒त्रा॒व॒रु॒णता॒मिति॑ मैत्रा - व॒रु॒णता᳚म् । ए॒व । उपेति॑ । ए॒ति॒ । आ॒त्मनः॑ । अस्क॑न्दाय । यः । वै । अ॒ग्निम् । ऋ॒तु॒स्थामित्यृ॑तु - स्थाम् । वेद॑ । ऋ॒तुर्.ऋ॑तु॒रित्यृ॒तुः - ऋ॒तुः॒ । अ॒स्मै॒ । कल्प॑मानः । ए॒ति॒ । प्रतीति॑ । ए॒व । ति॒ष्ठ॒ति॒ । सं॒ॅव॒थ्स॒र इति॑ सं - व॒थ्स॒रः । वै । अ॒ग्निः ।  \newline


\textbf{Krama Paata} \newline

जु॒हु॒या॒च्छान्तिः॑ । शान्ति॑रे॒व । ए॒वैषा । ए॒षाऽग्ने । अ॒ग्नेर् गुप्तिः॑ । गुप्ति॑रा॒त्मनः॑ । आ॒त्मनो॑ ह॒विष्कृ॑तः । ह॒विष्कृ॑तो॒ वै । ह॒विष्कृ॑त॒ इति॑ ह॒विः - कृ॒तः॒ । वा ए॒षः । ए॒ष यः । यो᳚ऽग्निम् । अ॒ग्निम् चि॑नु॒ते । चि॒नु॒ते यथा᳚ । यथा॒ वै । वै ह॒विः । ह॒विः स्कन्द॑ति । स्कन्द॑त्ये॒वम् । ए॒वम् ॅवै । वा ए॒षः । ए॒ष स्क॑न्दति । स्क॒न्द॒ति॒ यः । यो᳚ऽग्निम् । अ॒ग्निम् चि॒त्वा । चि॒त्वा स्त्रिय᳚म् । स्त्रिय॑मु॒पैति॑ । उ॒पैति॑ मैत्रावरु॒ण्या । उ॒पैतीत्यु॑प - एति॑ । मै॒त्रा॒व॒रु॒ण्याऽऽमिक्ष॑या । मै॒त्रा॒व॒रु॒ण्येति॑ मैत्रा - व॒रु॒ण्या । आ॒मिक्ष॑या यजेत । य॒जे॒त॒ मै॒त्रा॒व॒रु॒णता᳚म् । मै॒त्रा॒व॒रु॒णता॑मे॒व । मै॒त्रा॒व॒रु॒णता॒मिति॑ मैत्रा - व॒रु॒णता᳚म् । ए॒वोप॑ । उपै॑ति । ए॒त्या॒त्मनः॑ । आ॒त्मनोऽस्क॑न्दाय । अस्क॑न्दाय॒ यः । यो वै । वा अ॒ग्निम् । अ॒ग्निमृ॑तु॒स्थाम् । ऋ॒तु॒स्थाम् ॅवेद॑ । ऋ॒तु॒स्थामित्यृ॑तु - स्थाम् । वेद॒र्.तुर्.ऋ॑तुः । ऋ॒तुर्.ऋ॑तुरस्मै । ऋ॒तुर्.ऋ॑तु॒रित्यृ॒तुः - ऋ॒तुः॒ । अ॒स्मै॒ कल्प॑मानः । कल्प॑मान एति । ए॒ति॒ प्रति॑ । प्रत्ये॒व । ए॒व ति॑ष्ठति । ति॒ष्ठ॒ति॒ स॒म्ॅव॒थ्स॒रः । स॒म्ॅव॒थ्स॒रो वै । स॒म्ॅव॒थ्स॒र इति॑ सम् - व॒थ्स॒रः । वा अ॒ग्निः । अ॒ग्निर्. ऋ॑तु॒स्थाः \newline

\textbf{Jatai Paata} \newline

1. जु॒हु॒या॒च् छान्तिः॒ शान्ति॑र् जुहुयाज् जुहुया॒च् छान्तिः॑ । \newline
2. शान्ति॑ रे॒वैव शान्तिः॒ शान्ति॑ रे॒व । \newline
3. ए॒वैषै षैवै वैषा । \newline
4. ए॒षा ऽग्ने र॒ग्ने रे॒षैषा ऽग्नेः । \newline
5. अ॒ग्नेर् गुप्ति॒र् गुप्ति॑ र॒ग्ने र॒ग्नेर् गुप्तिः॑ । \newline
6. गुप्ति॑ रा॒त्मन॑ आ॒त्मनो॒ गुप्ति॒र् गुप्ति॑ रा॒त्मनः॑ । \newline
7. आ॒त्मनो॑ ह॒विष्कृ॑तो ह॒विष्कृ॑त आ॒त्मन॑ आ॒त्मनो॑ ह॒विष्कृ॑तः । \newline
8. ह॒विष्कृ॑तो॒ वै वै ह॒विष्कृ॑तो ह॒विष्कृ॑तो॒ वै । \newline
9. ह॒विष्कृ॑त॒ इति॑ ह॒विः - कृ॒तः॒ । \newline
10. वा ए॒ष ए॒ष वै वा ए॒षः । \newline
11. ए॒ष यो य ए॒ष ए॒ष यः । \newline
12. यो᳚ ऽग्नि म॒ग्निं ॅयो यो᳚ ऽग्निम् । \newline
13. अ॒ग्निम् चि॑नु॒ते चि॑नु॒ते᳚ ऽग्नि म॒ग्निम् चि॑नु॒ते । \newline
14. चि॒नु॒ते यथा॒ यथा॑ चिनु॒ते चि॑नु॒ते यथा᳚ । \newline
15. यथा॒ वै वै यथा॒ यथा॒ वै । \newline
16. वै ह॒विर्. ह॒विर् वै वै ह॒विः । \newline
17. ह॒विः स्कन्द॑ति॒ स्कन्द॑ति ह॒विर्. ह॒विः स्कन्द॑ति । \newline
18. स्कन्द॑ त्ये॒व मे॒वꣳ स्कन्द॑ति॒ स्कन्द॑ त्ये॒वम् । \newline
19. ए॒वं ॅवै वा ए॒व मे॒वं ॅवै । \newline
20. वा ए॒ष ए॒ष वै वा ए॒षः । \newline
21. ए॒ष स्क॑न्दति स्कन्द त्ये॒ष ए॒ष स्क॑न्दति । \newline
22. स्क॒न्द॒ति॒ यो यः स्क॑न्दति स्कन्दति॒ यः । \newline
23. यो᳚ ऽग्नि म॒ग्निं ॅयो यो᳚ ऽग्निम् । \newline
24. अ॒ग्निम् चि॒त्वा चि॒त्वा ऽग्नि म॒ग्निम् चि॒त्वा । \newline
25. चि॒त्वा स्त्रियꣳ॒॒ स्त्रिय॑म् चि॒त्वा चि॒त्वा स्त्रिय᳚म् । \newline
26. स्त्रिय॑ मु॒पै त्यु॒पैति॒ स्त्रियꣳ॒॒ स्त्रिय॑ मु॒पैति॑ । \newline
27. उ॒पैति॑ मैत्रावरु॒ण्या मै᳚त्रावरु॒ ण्योपैत्यु॒पैति॑ मैत्रावरु॒ण्या । \newline
28. उ॒पैतीत्यु॑प - एति॑ । \newline
29. मै॒त्रा॒व॒रु॒ण्या ऽऽमिक्ष॑या॒ ऽऽमिक्ष॑या मैत्रावरु॒ण्या मै᳚त्रावरु॒ण्या ऽऽमिक्ष॑या । \newline
30. मै॒त्रा॒व॒रु॒ण्येति॑ मैत्रा - व॒रु॒ण्या । \newline
31. आ॒मिक्ष॑या यजेत यजेता॒ मिक्ष॑या॒ ऽऽमिक्ष॑या यजेत । \newline
32. य॒जे॒त॒ मै॒त्रा॒व॒रु॒णता᳚म् मैत्रावरु॒णतां᳚ ॅयजेत यजेत मैत्रावरु॒णता᳚म् । \newline
33. मै॒त्रा॒व॒रु॒णता॑ मे॒वैव मै᳚त्रावरु॒णता᳚म् मैत्रावरु॒णता॑ मे॒व । \newline
34. मै॒त्रा॒व॒रु॒णता॒मिति॑ मैत्रा - व॒रु॒णता᳚म् । \newline
35. ए॒वो पोपै॒ वैवोप॑ । \newline
36. उपै᳚त्ये॒ त्युपो पै॑ति । \newline
37. ए॒त्या॒त्मन॑ आ॒त्मन॑ एत्ये त्या॒त्मनः॑ । \newline
38. आ॒त्मनो ऽस्क॑न्दा॒या स्क॑न्दा या॒त्मन॑ आ॒त्मनो ऽस्क॑न्दाय । \newline
39. अस्क॑न्दाय॒ यो यो ऽस्क॑न्दा॒या स्क॑न्दाय॒ यः । \newline
40. यो वै वै यो यो वै । \newline
41. वा अ॒ग्नि म॒ग्निं ॅवै वा अ॒ग्निम् । \newline
42. अ॒ग्नि मृ॑तु॒स्था मृ॑तु॒स्था म॒ग्नि म॒ग्नि मृ॑तु॒स्थाम् । \newline
43. ऋ॒तु॒स्थां ॅवेद॒ वेद॑ र्‌तु॒स्था मृ॑तु॒स्थां ॅवेद॑ । \newline
44. ऋ॒तु॒स्थामित्यृ॑तु - स्थाम् । \newline
45. वेद॒ र्‌तुर्.ऋ॑तुर्. ऋ॒तुर्.ऋ॑तु॒र् वेद॒ वेद॒ र्‌तुर्.ऋ॑तुः । \newline
46. ऋ॒तुर्.ऋ॑तु रस्मा अस्मा ऋ॒तुर्.ऋ॑तुर्. ऋ॒तुर्.ऋ॑तु रस्मै । \newline
47. ऋ॒तुर्.ऋ॑तु॒रित्यृ॒तुः - ऋ॒तुः॒ । \newline
48. अ॒स्मै॒ कल्प॑मानः॒ कल्प॑मानो ऽस्मा अस्मै॒ कल्प॑मानः । \newline
49. कल्प॑मान एत्येति॒ कल्प॑मानः॒ कल्प॑मान एति । \newline
50. ए॒ति॒ प्रति॒ प्रत्ये᳚ त्येति॒ प्रति॑ । \newline
51. प्रत्ये॒ वैव प्रति॒ प्रत्ये॒व । \newline
52. ए॒व ति॑ष्ठति तिष्ठ त्ये॒वैव ति॑ष्ठति । \newline
53. ति॒ष्ठ॒ति॒ सं॒ॅव॒थ्स॒रः सं॑ॅवथ्स॒र स्ति॑ष्ठति तिष्ठति संॅवथ्स॒रः । \newline
54. सं॒ॅव॒थ्स॒रो वै वै सं॑ॅवथ्स॒रः सं॑ॅवथ्स॒रो वै । \newline
55. सं॒ॅव॒थ्स॒र इति॑ सं - व॒थ्स॒रः । \newline
56. वा अ॒ग्नि र॒ग्निर् वै वा अ॒ग्निः । \newline
57. अ॒ग्निर्. ऋ॑तु॒स्था ऋ॑तु॒स्था अ॒ग्नि र॒ग्निर्. ऋ॑तु॒स्थाः । \newline

\textbf{Ghana Paata } \newline

1. जु॒हु॒या॒च् छान्तिः॒ शान्ति॑र् जुहुयाज् जुहुया॒च् छान्ति॑ रे॒वैव शान्ति॑र् जुहुयाज् जुहुया॒च् छान्ति॑ रे॒व । \newline
2. शान्ति॑ रे॒वैव शान्तिः॒ शान्ति॑ रे॒वै षैषैव शान्तिः॒ शान्ति॑ रे॒वैषा । \newline
3. ए॒वै षैषै वैवैषा ऽग्ने र॒ग्ने रे॒षै वैवैषा ऽग्नेः । \newline
4. ए॒षा ऽग्ने र॒ग्ने रे॒षैषा ऽग्नेर् गुप्ति॒र् गुप्ति॑ र॒ग्ने रे॒षैषा ऽग्नेर् गुप्तिः॑ । \newline
5. अ॒ग्नेर् गुप्ति॒र् गुप्ति॑ र॒ग्ने र॒ग्नेर् गुप्ति॑ रा॒त्मन॑ आ॒त्मनो॒ गुप्ति॑ र॒ग्ने र॒ग्नेर् गुप्ति॑ रा॒त्मनः॑ । \newline
6. गुप्ति॑ रा॒त्मन॑ आ॒त्मनो॒ गुप्ति॒र् गुप्ति॑ रा॒त्मनो॑ ह॒विष्कृ॑तो ह॒विष्कृ॑त आ॒त्मनो॒ गुप्ति॒र् गुप्ति॑ रा॒त्मनो॑ ह॒विष्कृ॑तः । \newline
7. आ॒त्मनो॑ ह॒विष्कृ॑तो ह॒विष्कृ॑त आ॒त्मन॑ आ॒त्मनो॑ ह॒विष्कृ॑तो॒ वै वै ह॒विष्कृ॑त आ॒त्मन॑ आ॒त्मनो॑ ह॒विष्कृ॑तो॒ वै । \newline
8. ह॒विष्कृ॑तो॒ वै वै ह॒विष्कृ॑तो ह॒विष्कृ॑तो॒ वा ए॒ष ए॒ष वै ह॒विष्कृ॑तो ह॒विष्कृ॑तो॒ वा ए॒षः । \newline
9. ह॒विष्कृ॑त॒ इति॑ ह॒विः - कृ॒तः॒ । \newline
10. वा ए॒ष ए॒ष वै वा ए॒ष यो य ए॒ष वै वा ए॒ष यः । \newline
11. ए॒ष यो य ए॒ष ए॒ष यो᳚ ऽग्नि म॒ग्निं ॅय ए॒ष ए॒ष यो᳚ ऽग्निम् । \newline
12. यो᳚ ऽग्नि म॒ग्निं ॅयो यो᳚ ऽग्निम् चि॑नु॒ते चि॑नु॒ते᳚ ऽग्निं ॅयो यो᳚ ऽग्निम् चि॑नु॒ते । \newline
13. अ॒ग्निम् चि॑नु॒ते चि॑नु॒ते᳚ ऽग्नि म॒ग्निम् चि॑नु॒ते यथा॒ यथा॑ चिनु॒ते᳚ ऽग्नि म॒ग्निम् चि॑नु॒ते यथा᳚ । \newline
14. चि॒नु॒ते यथा॒ यथा॑ चिनु॒ते चि॑नु॒ते यथा॒ वै वै यथा॑ चिनु॒ते चि॑नु॒ते यथा॒ वै । \newline
15. यथा॒ वै वै यथा॒ यथा॒ वै ह॒विर्. ह॒विर् वै यथा॒ यथा॒ वै ह॒विः । \newline
16. वै ह॒विर्. ह॒विर् वै वै ह॒विः स्कन्द॑ति॒ स्कन्द॑ति ह॒विर् वै वै ह॒विः स्कन्द॑ति । \newline
17. ह॒विः स्कन्द॑ति॒ स्कन्द॑ति ह॒विर्. ह॒विः स्कन्द॑ त्ये॒व मे॒वꣳ स्कन्द॑ति ह॒विर्. ह॒विः स्कन्द॑ त्ये॒वम् । \newline
18. स्कन्द॑ त्ये॒व मे॒वꣳ स्कन्द॑ति॒ स्कन्द॑ त्ये॒वं ॅवै वा ए॒वꣳ स्कन्द॑ति॒ स्कन्द॑ त्ये॒वं ॅवै । \newline
19. ए॒वं ॅवै वा ए॒व मे॒वं ॅवा ए॒ष ए॒ष वा ए॒व मे॒वं ॅवा ए॒षः । \newline
20. वा ए॒ष ए॒ष वै वा ए॒ष स्क॑न्दति स्कन्द त्ये॒ष वै वा ए॒ष स्क॑न्दति । \newline
21. ए॒ष स्क॑न्दति स्कन्द त्ये॒ष ए॒ष स्क॑न्दति॒ यो यः स्क॑न्द त्ये॒ष ए॒ष स्क॑न्दति॒ यः । \newline
22. स्क॒न्द॒ति॒ यो यः स्क॑न्दति स्कन्दति॒ यो᳚ ऽग्नि म॒ग्निं ॅयः स्क॑न्दति स्कन्दति॒ यो᳚ ऽग्निम् । \newline
23. यो᳚ ऽग्नि म॒ग्निं ॅयो यो᳚ ऽग्निम् चि॒त्वा चि॒त्वा ऽग्निं ॅयो यो᳚ ऽग्निम् चि॒त्वा । \newline
24. अ॒ग्निम् चि॒त्वा चि॒त्वा ऽग्नि म॒ग्निम् चि॒त्वा स्त्रियꣳ॒॒ स्त्रिय॑म् चि॒त्वा ऽग्नि म॒ग्निम् चि॒त्वा स्त्रिय᳚म् । \newline
25. चि॒त्वा स्त्रियꣳ॒॒ स्त्रिय॑म् चि॒त्वा चि॒त्वा स्त्रिय॑ मु॒पै त्यु॒पैति॒ स्त्रिय॑म् चि॒त्वा चि॒त्वा स्त्रिय॑ मु॒पैति॑ । \newline
26. स्त्रिय॑ मु॒पै त्यु॒पैति॒ स्त्रियꣳ॒॒ स्त्रिय॑ मु॒पैति॑ मैत्रावरु॒ण्या मै᳚त्रावरु॒ण्योपैति॒ स्त्रियꣳ॒॒ स्त्रिय॑ मु॒पैति॑ मैत्रावरु॒ण्या । \newline
27. उ॒पैति॑ मैत्रावरु॒ण्या मै᳚त्रावरु॒ ण्योपै त्यु॒पैति॑ मैत्रावरु॒ण्या ऽऽमिक्ष॑या॒ ऽऽमिक्ष॑या मैत्रावरु॒
ण्योपै त्यु॒पैति॑ मैत्रावरु॒ण्या ऽऽमिक्ष॑या । \newline
28. उ॒पैतीत्यु॑प - एति॑ । \newline
29. मै॒त्रा॒व॒रु॒ण्या ऽऽमिक्ष॑या॒ ऽऽमिक्ष॑या मैत्रावरु॒ण्या मै᳚त्रावरु॒ण्या ऽऽमिक्ष॑या यजेत यजेता॒ मिक्ष॑या मैत्रावरु॒ण्या मै᳚त्रावरु॒ण्या ऽऽमिक्ष॑या यजेत । \newline
30. मै॒त्रा॒व॒रु॒ण्येति॑ मैत्रा - व॒रु॒ण्या । \newline
31. आ॒मिक्ष॑या यजेत यजेता॒ मिक्ष॑या॒ ऽऽमिक्ष॑या यजेत मैत्रावरु॒णता᳚म् मैत्रावरु॒णतां᳚ ॅयजेता॒ मिक्ष॑या॒ ऽऽमिक्ष॑या यजेत मैत्रावरु॒णता᳚म् । \newline
32. य॒जे॒त॒ मै॒त्रा॒व॒रु॒णता᳚म् मैत्रावरु॒णतां᳚ ॅयजेत यजेत मैत्रावरु॒णता॑ मे॒वैव मै᳚त्रावरु॒णतां᳚ ॅयजेत यजेत मैत्रावरु॒णता॑ मे॒व । \newline
33. मै॒त्रा॒व॒रु॒णता॑ मे॒वैव मै᳚त्रावरु॒णता᳚म् मैत्रावरु॒णता॑ मे॒वोपो पै॒व मै᳚त्रावरु॒णता᳚म् मैत्रावरु॒णता॑ मे॒वोप॑ । \newline
34. मै॒त्रा॒व॒रु॒णता॒मिति॑ मैत्रा - व॒रु॒णता᳚म् । \newline
35. ए॒वोपोपै॒ वैवोपै᳚ त्ये॒ त्युपै॒ वैवोपै॑ति । \newline
36. उपै᳚ त्ये॒ त्युपोपै᳚ त्या॒त्मन॑ आ॒त्मन॑ ए॒त्यु पोपै᳚ त्या॒त्मनः॑ । \newline
37. ए॒त्या॒त्मन॑ आ॒त्मन॑ एत्ये त्या॒त्मनो ऽस्क॑न्दा॒या स्क॑न्दा या॒त्मन॑ एत्ये त्या॒त्मनो ऽस्क॑न्दाय । \newline
38. आ॒त्मनो ऽस्क॑न्दा॒या स्क॑न्दा या॒त्मन॑ आ॒त्मनो ऽस्क॑न्दाय॒ यो यो ऽस्क॑न्दा या॒त्मन॑ आ॒त्मनो ऽस्क॑न्दाय॒ यः । \newline
39. अस्क॑न्दाय॒ यो यो ऽस्क॑न्दा॒या स्क॑न्दाय॒ यो वै वै यो ऽस्क॑न्दा॒या स्क॑न्दाय॒ यो वै । \newline
40. यो वै वै यो यो वा अ॒ग्नि म॒ग्निं ॅवै यो यो वा अ॒ग्निम् । \newline
41. वा अ॒ग्नि म॒ग्निं ॅवै वा अ॒ग्नि मृ॑तु॒स्था मृ॑तु॒स्था म॒ग्निं ॅवै वा अ॒ग्नि मृ॑तु॒स्थाम् । \newline
42. अ॒ग्नि मृ॑तु॒स्था मृ॑तु॒स्था म॒ग्नि म॒ग्नि मृ॑तु॒स्थां ॅवेद॒ वेद॑ र्‌तु॒स्था म॒ग्नि म॒ग्नि मृ॑तु॒स्थां ॅवेद॑ । \newline
43. ऋ॒तु॒स्थां ॅवेद॒ वेद॑ र्‌तु॒स्था मृ॑तु॒स्थां ॅवेद॒ र्‌तुर्.ऋ॑तुर्. ऋ॒तुर्.ऋ॑तु॒र् वेद॑  र्‌तु॒स्था मृ॑तु॒स्थां ॅवेद॒ र्‌तुर्.ऋ॑तुः । \newline
44. ऋ॒तु॒स्थामित्यृ॑तु - स्थाम् । \newline
45. वेद॒ र्‌तुर्.ऋ॑तुर्. ऋ॒तुर्.ऋ॑तु॒र् वेद॒ वेद॒ र्‌तुर्.ऋ॑तु रस्मा अस्मा ऋ॒तुर्.ऋ॑तु॒र् वेद॒ वेद॒ र्‌तुर्.ऋ॑तु रस्मै । \newline
46. ऋ॒तुर्.ऋ॑तु रस्मा अस्मा ऋ॒तुर्.ऋ॑तुर्. ऋ॒तुर्.ऋ॑तु रस्मै॒ कल्प॑मानः॒ कल्प॑मानो ऽस्मा ऋ॒तुर्.ऋ॑तुर्. ऋ॒तुर्.ऋ॑तु रस्मै॒ कल्प॑मानः । \newline
47. ऋ॒तुर्.ऋ॑तु॒रित्यृ॒तुः - ऋ॒तुः॒ । \newline
48. अ॒स्मै॒ कल्प॑मानः॒ कल्प॑मानो ऽस्मा अस्मै॒ कल्प॑मान एत्येति॒ कल्प॑मानो ऽस्मा अस्मै॒ कल्प॑मान एति । \newline
49. कल्प॑मान एत्येति॒ कल्प॑मानः॒ कल्प॑मान एति॒ प्रति॒ प्रत्ये॑ति॒ कल्प॑मानः॒ कल्प॑मान एति॒ प्रति॑ । \newline
50. ए॒ति॒ प्रति॒ प्रत्ये᳚ त्येति॒ प्रत्ये॒ वैव प्रत्ये᳚ त्येति॒ प्रत्ये॒व । \newline
51. प्रत्ये॒ वैव प्रति॒ प्रत्ये॒व ति॑ष्ठति तिष्ठ त्ये॒व प्रति॒ प्रत्ये॒व ति॑ष्ठति । \newline
52. ए॒व ति॑ष्ठति तिष्ठ त्ये॒वैव ति॑ष्ठति संॅवथ्स॒रः सं॑ॅवथ्स॒र स्ति॑ष्ठ त्ये॒वैव ति॑ष्ठति संॅवथ्स॒रः । \newline
53. ति॒ष्ठ॒ति॒ सं॒ॅव॒थ्स॒रः सं॑ॅवथ्स॒र स्ति॑ष्ठति तिष्ठति संॅवथ्स॒रो वै वै सं॑ॅवथ्स॒र स्ति॑ष्ठति तिष्ठति संॅवथ्स॒रो वै । \newline
54. सं॒ॅव॒थ्स॒रो वै वै सं॑ॅवथ्स॒रः सं॑ॅवथ्स॒रो वा अ॒ग्नि र॒ग्निर् वै सं॑ॅवथ्स॒रः सं॑ॅवथ्स॒रो वा अ॒ग्निः । \newline
55. सं॒ॅव॒थ्स॒र इति॑ सं - व॒थ्स॒रः । \newline
56. वा अ॒ग्नि र॒ग्निर् वै वा अ॒ग्निर्. ऋ॑तु॒स्था ऋ॑तु॒स्था अ॒ग्निर् वै वा अ॒ग्निर्. ऋ॑तु॒स्थाः । \newline
57. अ॒ग्निर्. ऋ॑तु॒स्था ऋ॑तु॒स्था अ॒ग्नि र॒ग्निर्. ऋ॑तु॒स्था स्तस्य॒ तस्य॑ र्‌तु॒स्था अ॒ग्नि र॒ग्निर्. ऋ॑तु॒स्था स्तस्य॑ । \newline
\pagebreak
\markright{ TS 5.7.6.6  \hfill https://www.vedavms.in \hfill}

\section{ TS 5.7.6.6 }

\textbf{TS 5.7.6.6 } \newline
\textbf{Samhita Paata} \newline

ऋ॑तु॒स्थास्तस्य॑ वस॒न्तः शिरो᳚ ग्री॒ष्मो दक्षि॑णः प॒क्षो व॒र्॒.षाः पुच्छꣳ॑ श॒रदुत्त॑रः प॒क्षो हे॑म॒न्तो मद्ध्यं॑ पूर्वप॒क्षा-श्चित॑योऽपरप॒क्षाः पुरी॑ष-महोरा॒त्राणीष्ट॑का ए॒ष वा अ॒ग्निर्.ऋ॑तु॒स्था य ए॒वं ॅवेद॒र्तुर्.ऋ॑तुरस्मै॒ कल्प॑मान एति॒ प्रत्ये॒व ति॑ष्ठति प्र॒जाप॑ति॒र्वा ए॒तं ज्यैष्ठ्य॑कामो॒ न्य॑धत्त॒ ततो॒ वै स ज्यैष्ठ्य॑मगच्छ॒द्य ए॒वं ॅवि॒द्वान॒ग्निं चि॑नु॒ते ( ) ज्यैष्ठ्य॑मे॒व ग॑च्छति ॥ \newline

\textbf{Pada Paata} \newline

ऋ॒तु॒स्था इत्यृ॑तु - स्थाः । तस्य॑ । व॒स॒न्तः । शिरः॑ । ग्री॒ष्मः । दक्षि॑णः । प॒क्षः । व॒र्॒.षाः । पुच्छ᳚म् । श॒रत् । उत्त॑र॒ इत्युत् - त॒रः॒ । प॒क्षः । हे॒म॒न्तः । मद्ध्य᳚म् । पू॒र्व॒प॒क्षा इति॑ पूर्व - प॒क्षाः । चित॑यः । अ॒प॒र॒प॒क्षा इत्य॑पर - प॒क्षाः । पुरी॑षम् । अ॒हो॒रा॒त्राणीत्य॑हः-रा॒त्राणि॑ । इष्ट॑काः । ए॒षः । वै । अ॒ग्निः । ऋ॒तु॒स्था इत्यृ॑तु - स्थाः । यः । ए॒वम् । वेद॑ । ऋ॒तुर्.ऋ॑तु॒रित्यृ॒तुः - ऋ॒तुः॒ । अ॒स्मै॒ । कल्प॑मानः । ए॒ति॒ । प्रतीति॑ । ए॒व । ति॒ष्ठ॒ति॒ । प्र॒जाप॑ति॒रिति॑ प्र॒जा - प॒तिः॒ । वै । ए॒तम् । ज्यैष्ठ्य॑काम॒ इति॒ ज्यैष्ठ्य॑ - का॒मः॒ । नीति॑ । अ॒ध॒त्त॒ । ततः॑ । वै । सः । ज्यैष्ठ्य᳚म् । अ॒ग॒च्छ॒त् । यः । ए॒वम् । वि॒द्वान् । अ॒ग्निम् । चि॒नु॒ते ( ) । ज्यैष्ठ्य᳚म् । ए॒व । ग॒च्छ॒ति॒ ॥  \newline


\textbf{Krama Paata} \newline

ऋ॒तु॒स्थास्तस्य॑ । ऋ॒तु॒स्था इत्यृ॑तु - स्थाः । तस्य॑ वस॒न्तः । व॒स॒न्तः शिरः॑ । शिरो᳚ ग्री॒ष्मः । ग्री॒ष्मो दक्षि॑णः । दक्षि॑णः प॒क्षः । प॒क्षो व॒र्॒.षाः । व॒र्॒.षाः पुच्छ᳚म् । पुच्छꣳ॑ श॒रत् । श॒रदुत्त॑रः । उत्त॑रः प॒क्षः । उत्त॑र॒ इत्युत् - त॒रः॒ । प॒क्षो हे॑म॒न्तः । हे॒म॒न्तो मद्ध्य᳚म् । मद्ध्य॑म् पूर्वप॒क्षाः । पू॒र्व॒प॒क्षा श्चित॑यः । पू॒र्व॒प॒क्षा इति॑ पूर्व - प॒क्षाः । चित॑योऽपरप॒क्षाः । अ॒प॒र॒प॒क्षाः पुरी॑षम् । अ॒प॒र॒प॒क्षा इत्य॑पर - प॒क्षाः । पुरी॑षमहोरा॒त्राणि॑ । अ॒हो॒रा॒त्राणीष्ट॑काः । अ॒हो॒रा॒त्राणीत्य॑हः - रा॒त्राणि॑ । इष्ट॑का ए॒षः । ए॒ष वै । वा अ॒ग्निः । अ॒ग्निर्. ऋ॑तु॒स्थाः । ऋ॒तु॒स्था यः । ऋ॒तु॒स्था इत्यृ॑तु - स्थाः । य ए॒वम् । ए॒वम् ॅवेद॑ । वेद॒र्तुर्.ऋ॑तुः । ऋ॒तुर्.ऋ॑तुरस्मै । ऋ॒तुर्.ऋ॑तु॒रित्यृ॒तुः - ऋ॒तुः॒ । अ॒स्मै॒ कल्प॑मानः । कल्प॑मान एति । ए॒ति॒ प्रति॑ । प्रत्ये॒व । ए॒व ति॑ष्ठति । ति॒ष्ठ॒ति॒ प्र॒जाप॑तिः । प्र॒जाप॑ति॒र् वै । प्र॒जाप॑ति॒रिति॑ प्र॒जा - प॒तिः॒ । वा ए॒तम् । ए॒तम् ज्यैष्ठ्य॑कामः । ज्यैष्ठ्य॑कामो॒ नि । ज्यैष्ठ्य॑काम॒ इति॒ ज्यैष्ठ्य॑ - का॒मः॒ । न्य॑धत्त । अ॒ध॒त्त॒ ततः॑ । ततो॒ वै । वै सः । स ज्यैष्ठ्य᳚म् । ज्यैष्ठ्य॑मगच्छत् । अ॒ग॒च्छ॒द् यः । य ए॒वम् । ए॒वम् ॅवि॒द्वान् । वि॒द्वान॒ग्निम् । अ॒ग्निम् चि॑नु॒ते ( ) । चि॒नु॒ते ज्यैष्ठ्य᳚म् । ज्यैष्ठ्य॑मे॒व । ए॒व ग॑च्छति । ग॒च्छ॒तीति॑ गच्छति । \newline

\textbf{Jatai Paata} \newline

1. ऋ॒तु॒स्था स्तस्य॒ तस्य॑ र्‌तु॒स्था ऋ॑तु॒स्था स्तस्य॑ । \newline
2. ऋ॒तु॒स्था इत्यृ॑तु - स्थाः । \newline
3. तस्य॑ वस॒न्तो व॑स॒न्त स्तस्य॒ तस्य॑ वस॒न्तः । \newline
4. व॒स॒न्तः शिरः॒ शिरो॑ वस॒न्तो व॑स॒न्तः शिरः॑ । \newline
5. शिरो᳚ ग्री॒ष्मो ग्री॒ष्मः शिरः॒ शिरो᳚ ग्री॒ष्मः । \newline
6. ग्री॒ष्मो दक्षि॑णो॒ दक्षि॑णो ग्री॒ष्मो ग्री॒ष्मो दक्षि॑णः । \newline
7. दक्षि॑णः प॒क्षः प॒क्षो दक्षि॑णो॒ दक्षि॑णः प॒क्षः । \newline
8. प॒क्षो व॒र्॒.षा व॒र्॒.षाः प॒क्षः प॒क्षो व॒र्॒.षाः । \newline
9. व॒र्॒.षाः पुच्छ॒म् पुच्छं॑ ॅव॒र्॒.षा व॒र्॒.षाः पुच्छ᳚म् । \newline
10. पुच्छꣳ॑ श॒रच् छ॒रत् पुच्छ॒म् पुच्छꣳ॑ श॒रत् । \newline
11. श॒र दुत्त॑र॒ उत्त॑रः श॒रच् छ॒र दुत्त॑रः । \newline
12. उत्त॑रः प॒क्षः प॒क्ष उत्त॑र॒ उत्त॑रः प॒क्षः । \newline
13. उत्त॑र॒ इत्युत् - त॒रः॒ । \newline
14. प॒क्षो हे॑म॒न्तो हे॑म॒न्तः प॒क्षः प॒क्षो हे॑म॒न्तः । \newline
15. हे॒म॒न्तो मद्ध्य॒म् मद्ध्यꣳ॑ हेम॒न्तो हे॑म॒न्तो मद्ध्य᳚म् । \newline
16. मद्ध्य॑म् पूर्वप॒क्षाः पू᳚र्वप॒क्षा मद्ध्य॒म् मद्ध्य॑म् पूर्वप॒क्षाः । \newline
17. पू॒र्व॒प॒क्षा श्चित॑य॒ श्चित॑यः पूर्वप॒क्षाः पू᳚र्वप॒क्षा श्चित॑यः । \newline
18. पू॒र्व॒प॒क्षा इति॑ पूर्व - प॒क्षाः । \newline
19. चित॑यो ऽपरप॒क्षा अ॑परप॒क्षा श्चित॑य॒ श्चित॑यो ऽपरप॒क्षाः । \newline
20. अ॒प॒र॒प॒क्षाः पुरी॑ष॒म् पुरी॑ष मपरप॒क्षा अ॑परप॒क्षाः पुरी॑षम् । \newline
21. अ॒प॒र॒प॒क्षा इत्य॑पर - प॒क्षाः । \newline
22. पुरी॑ष महोरा॒त्रा ण्य॑होरा॒त्राणि॒ पुरी॑ष॒म् पुरी॑ष महोरा॒त्राणि॑ । \newline
23. अ॒हो॒रा॒त्राणी ष्ट॑का॒ इष्ट॑का अहोरा॒त्रा ण्य॑होरा॒त्राणी ष्ट॑काः । \newline
24. अ॒हो॒रा॒त्राणीत्य॑हः - रा॒त्राणि॑ । \newline
25. इष्ट॑का ए॒ष ए॒ष इष्ट॑का॒ इष्ट॑का ए॒षः । \newline
26. ए॒ष वै वा ए॒ष ए॒ष वै । \newline
27. वा अ॒ग्नि र॒ग्निर् वै वा अ॒ग्निः । \newline
28. अ॒ग्निर्. ऋ॑तु॒स्था ऋ॑तु॒स्था अ॒ग्नि र॒ग्निर्. ऋ॑तु॒स्थाः । \newline
29. ऋ॒तु॒स्था यो य ऋ॑तु॒स्था ऋ॑तु॒स्था यः । \newline
30. ऋ॒तु॒स्था इत्यृ॑तु - स्थाः । \newline
31. य ए॒व मे॒वं ॅयो य ए॒वम् । \newline
32. ए॒वं ॅवेद॒ वेदै॒व मे॒वं ॅवेद॑ । \newline
33. वेद॒ र्‌तुर्.ऋ॑तुर्. ऋ॒तुर्.ऋ॑तु॒र् वेद॒ वेद॒ र्‌तुर्.ऋ॑तुः । \newline
34. ऋ॒तुर्.ऋ॑तु रस्मा अस्मा ऋ॒तुर्.ऋ॑तुर्. ऋ॒तुर्.ऋ॑तु रस्मै । \newline
35. ऋ॒तुर्.ऋ॑तु॒रित्यृ॒तुः - ऋ॒तुः॒ । \newline
36. अ॒स्मै॒ कल्प॑मानः॒ कल्प॑मानो ऽस्मा अस्मै॒ कल्प॑मानः । \newline
37. कल्प॑मान एत्येति॒ कल्प॑मानः॒ कल्प॑मान एति । \newline
38. ए॒ति॒ प्रति॒ प्रत्ये᳚ त्येति॒ प्रति॑ । \newline
39. प्रत्ये॒ वैव प्रति॒ प्रत्ये॒व । \newline
40. ए॒व ति॑ष्ठति तिष्ठ त्ये॒वैव ति॑ष्ठति । \newline
41. ति॒ष्ठ॒ति॒ प्र॒जाप॑तिः प्र॒जाप॑ति स्तिष्ठति तिष्ठति प्र॒जाप॑तिः । \newline
42. प्र॒जाप॑ति॒र् वै वै प्र॒जाप॑तिः प्र॒जाप॑ति॒र् वै । \newline
43. प्र॒जाप॑ति॒रिति॑ प्र॒जा - प॒तिः॒ । \newline
44. वा ए॒त मे॒तं ॅवै वा ए॒तम् । \newline
45. ए॒तम् ज्यैष्ठ्य॑कामो॒ ज्यैष्ठ्य॑काम ए॒त मे॒तम् ज्यैष्ठ्य॑कामः । \newline
46. ज्यैष्ठ्य॑कामो॒ नि नि ज्यैष्ठ्य॑कामो॒ ज्यैष्ठ्य॑कामो॒ नि । \newline
47. ज्यैष्ठ्य॑काम॒ इति॒ ज्यैष्ठ्य॑ - का॒मः॒ । \newline
48. न्य॑धत्ता धत्त॒ नि न्य॑धत्त । \newline
49. अ॒ध॒त्त॒ तत॒ स्ततो॑ ऽधत्ता धत्त॒ ततः॑ । \newline
50. ततो॒ वै वै तत॒ स्ततो॒ वै । \newline
51. वै स स वै वै सः । \newline
52. स ज्यैष्ठ्य॒म् ज्यैष्ठ्यꣳ॒॒ स स ज्यैष्ठ्य᳚म् । \newline
53. ज्यैष्ठ्य॑ मगच्छ दगच्छ॒ज् ज्यैष्ठ्य॒म् ज्यैष्ठ्य॑ मगच्छत् । \newline
54. अ॒ग॒च्छ॒द् यो यो॑ ऽगच्छ दगच्छ॒द् यः । \newline
55. य ए॒व मे॒वं ॅयो य ए॒वम् । \newline
56. ए॒वं ॅवि॒द्वान्. वि॒द्वा ने॒व मे॒वं ॅवि॒द्वान् । \newline
57. वि॒द्वा न॒ग्नि म॒ग्निं ॅवि॒द्वान्. वि॒द्वा न॒ग्निम् । \newline
58. अ॒ग्निम् चि॑नु॒ते चि॑नु॒ते᳚ ऽग्नि म॒ग्निम् चि॑नु॒ते । \newline
59. चि॒नु॒ते ज्यैष्ठ्य॒म् ज्यैष्ठ्य॑म् चिनु॒ते चि॑नु॒ते ज्यैष्ठ्य᳚म् । \newline
60. ज्यैष्ठ्य॑ मे॒वैव ज्यैष्ठ्य॒म् ज्यैष्ठ्य॑ मे॒व । \newline
61. ए॒व ग॑च्छति गच्छ त्ये॒वैव ग॑च्छति । \newline
62. ग॒च्छ॒तीति॑ गच्छति । \newline

\textbf{Ghana Paata } \newline

1. ऋ॒तु॒स्था स्तस्य॒ तस्य॑ र्‌तु॒स्था ऋ॑तु॒स्था स्तस्य॑ वस॒न्तो व॑स॒न्त स्तस्य॑ र्‌तु॒स्था ऋ॑तु॒स्था स्तस्य॑ वस॒न्तः । \newline
2. ऋ॒तु॒स्था इत्यृ॑तु - स्थाः । \newline
3. तस्य॑ वस॒न्तो व॑स॒न्त स्तस्य॒ तस्य॑ वस॒न्तः शिरः॒ शिरो॑ वस॒न्त स्तस्य॒ तस्य॑ वस॒न्तः शिरः॑ । \newline
4. व॒स॒न्तः शिरः॒ शिरो॑ वस॒न्तो व॑स॒न्तः शिरो᳚ ग्री॒ष्मो ग्री॒ष्मः शिरो॑ वस॒न्तो व॑स॒न्तः शिरो᳚ ग्री॒ष्मः । \newline
5. शिरो᳚ ग्री॒ष्मो ग्री॒ष्मः शिरः॒ शिरो᳚ ग्री॒ष्मो दक्षि॑णो॒ दक्षि॑णो ग्री॒ष्मः शिरः॒ शिरो᳚ ग्री॒ष्मो दक्षि॑णः । \newline
6. ग्री॒ष्मो दक्षि॑णो॒ दक्षि॑णो ग्री॒ष्मो ग्री॒ष्मो दक्षि॑णः प॒क्षः प॒क्षो दक्षि॑णो ग्री॒ष्मो ग्री॒ष्मो दक्षि॑णः प॒क्षः । \newline
7. दक्षि॑णः प॒क्षः प॒क्षो दक्षि॑णो॒ दक्षि॑णः प॒क्षो व॒र्॒.षा व॒र्॒.षाः प॒क्षो दक्षि॑णो॒ दक्षि॑णः प॒क्षो व॒र्॒.षाः । \newline
8. प॒क्षो व॒र्॒.षा व॒र्॒.षाः प॒क्षः प॒क्षो व॒र्॒.षाः पुच्छ॒म् पुच्छं॑ ॅव॒र्॒.षाः प॒क्षः प॒क्षो व॒र्॒.षाः पुच्छ᳚म् । \newline
9. व॒र्॒.षाः पुच्छ॒म् पुच्छं॑ ॅव॒र्॒.षा व॒र्॒.षाः पुच्छꣳ॑ श॒रच् छ॒रत् पुच्छं॑ ॅव॒र्॒.षा व॒र्॒.षाः पुच्छꣳ॑ श॒रत् । \newline
10. पुच्छꣳ॑ श॒रच् छ॒रत् पुच्छ॒म् पुच्छꣳ॑ श॒र दुत्त॑र॒ उत्त॑रः श॒रत् पुच्छ॒म् पुच्छꣳ॑ श॒र दुत्त॑रः । \newline
11. श॒र दुत्त॑र॒ उत्त॑रः श॒रच् छ॒र दुत्त॑रः प॒क्षः प॒क्ष उत्त॑रः श॒रच् छ॒र दुत्त॑रः प॒क्षः । \newline
12. उत्त॑रः प॒क्षः प॒क्ष उत्त॑र॒ उत्त॑रः प॒क्षो हे॑म॒न्तो हे॑म॒न्तः प॒क्ष उत्त॑र॒ उत्त॑रः प॒क्षो हे॑म॒न्तः । \newline
13. उत्त॑र॒ इत्युत् - त॒रः॒ । \newline
14. प॒क्षो हे॑म॒न्तो हे॑म॒न्तः प॒क्षः प॒क्षो हे॑म॒न्तो मद्ध्य॒म् मद्ध्यꣳ॑ हेम॒न्तः प॒क्षः प॒क्षो हे॑म॒न्तो मद्ध्य᳚म् । \newline
15. हे॒म॒न्तो मद्ध्य॒म् मद्ध्यꣳ॑ हेम॒न्तो हे॑म॒न्तो मद्ध्य॑म् पूर्वप॒क्षाः पू᳚र्वप॒क्षा मद्ध्यꣳ॑ हेम॒न्तो हे॑म॒न्तो मद्ध्य॑म् पूर्वप॒क्षाः । \newline
16. मद्ध्य॑म् पूर्वप॒क्षाः पू᳚र्वप॒क्षा मद्ध्य॒म् मद्ध्य॑म् पूर्वप॒क्षा श्चित॑य॒ श्चित॑यः पूर्वप॒क्षा मद्ध्य॒म् मद्ध्य॑म् पूर्वप॒क्षा श्चित॑यः । \newline
17. पू॒र्व॒प॒क्षा श्चित॑य॒ श्चित॑यः पूर्वप॒क्षाः पू᳚र्वप॒क्षा श्चित॑यो ऽपरप॒क्षा अ॑परप॒क्षा श्चित॑यः पूर्वप॒क्षाः पू᳚र्वप॒क्षा श्चित॑यो ऽपरप॒क्षाः । \newline
18. पू॒र्व॒प॒क्षा इति॑ पूर्व - प॒क्षाः । \newline
19. चित॑यो ऽपरप॒क्षा अ॑परप॒क्षा श्चित॑य॒ श्चित॑यो ऽपरप॒क्षाः पुरी॑ष॒म् पुरी॑ष मपरप॒क्षा श्चित॑य॒ श्चित॑यो ऽपरप॒क्षाः पुरी॑षम् । \newline
20. अ॒प॒र॒प॒क्षाः पुरी॑ष॒म् पुरी॑ष मपरप॒क्षा अ॑परप॒क्षाः पुरी॑ष महोरा॒त्रा ण्य॑होरा॒त्राणि॒ पुरी॑ष मपरप॒क्षा अ॑परप॒क्षाः पुरी॑ष महोरा॒त्राणि॑ । \newline
21. अ॒प॒र॒प॒क्षा इत्य॑पर - प॒क्षाः । \newline
22. पुरी॑ष महोरा॒त्रा ण्य॑होरा॒त्राणि॒ पुरी॑ष॒म् पुरी॑ष महोरा॒त्राणी ष्ट॑का॒ इष्ट॑का अहोरा॒त्राणि॒ पुरी॑ष॒म् पुरी॑ष महोरा॒त्राणी ष्ट॑काः । \newline
23. अ॒हो॒रा॒त्राणी ष्ट॑का॒ इष्ट॑का अहोरा॒त्रा ण्य॑होरा॒त्राणी ष्ट॑का ए॒ष ए॒ष इष्ट॑का अहोरा॒त्रा ण्य॑होरा॒त्राणी ष्ट॑का ए॒षः । \newline
24. अ॒हो॒रा॒त्राणीत्य॑हः - रा॒त्राणि॑ । \newline
25. इष्ट॑का ए॒ष ए॒ष इष्ट॑का॒ इष्ट॑का ए॒ष वै वा ए॒ष इष्ट॑का॒ इष्ट॑का ए॒ष वै । \newline
26. ए॒ष वै वा ए॒ष ए॒ष वा अ॒ग्नि र॒ग्निर् वा ए॒ष ए॒ष वा अ॒ग्निः । \newline
27. वा अ॒ग्नि र॒ग्निर् वै वा अ॒ग्निर्. ऋ॑तु॒स्था ऋ॑तु॒स्था अ॒ग्निर् वै वा अ॒ग्निर्. ऋ॑तु॒स्थाः । \newline
28. अ॒ग्निर्. ऋ॑तु॒स्था ऋ॑तु॒स्था अ॒ग्नि र॒ग्निर्. ऋ॑तु॒स्था यो य ऋ॑तु॒स्था अ॒ग्नि र॒ग्निर्. ऋ॑तु॒स्था यः । \newline
29. ऋ॒तु॒स्था यो य ऋ॑तु॒स्था ऋ॑तु॒स्था य ए॒व मे॒वं ॅय ऋ॑तु॒स्था ऋ॑तु॒स्था य ए॒वम् । \newline
30. ऋ॒तु॒स्था इत्यृ॑तु - स्थाः । \newline
31. य ए॒व मे॒वं ॅयो य ए॒वं ॅवेद॒ वेदै॒वं ॅयो य ए॒वं ॅवेद॑ । \newline
32. ए॒वं ॅवेद॒ वेदै॒व मे॒वं ॅवेद॒ र्‌तुर्.ऋ॑तुर्. ऋ॒तुर्.ऋ॑तु॒र् वेदै॒व मे॒वं ॅवेद॒ र्‌तुर्.ऋ॑तुः । \newline
33. वेद॒ र्तुर्.ऋ॑तुर्. ऋ॒तुर्.ऋ॑तु॒र् वेद॒ वेद॒ र्‌तुर्.ऋ॑तु रस्मा अस्मा ऋ॒तुर्.ऋ॑तु॒र् वेद॒ वेद॒ र्‌तुर्.ऋ॑तु रस्मै । \newline
34. ऋ॒तुर्.ऋ॑तु रस्मा अस्मा ऋ॒तुर्.ऋ॑तुर्. ऋ॒तुर्.ऋ॑तु रस्मै॒ कल्प॑मानः॒ कल्प॑मानो ऽस्मा ऋ॒तुर्.ऋ॑तुर्. ऋ॒तुर्.ऋ॑तु रस्मै॒ कल्प॑मानः । \newline
35. ऋ॒तुर्.ऋ॑तु॒रित्यृ॒तुः - ऋ॒तुः॒ । \newline
36. अ॒स्मै॒ कल्प॑मानः॒ कल्प॑मानो ऽस्मा अस्मै॒ कल्प॑मान एत्येति॒ कल्प॑मानो ऽस्मा अस्मै॒ कल्प॑मान एति । \newline
37. कल्प॑मान एत्येति॒ कल्प॑मानः॒ कल्प॑मान एति॒ प्रति॒ प्रत्ये॑ति॒ कल्प॑मानः॒ कल्प॑मान एति॒ प्रति॑ । \newline
38. ए॒ति॒ प्रति॒ प्रत्ये᳚ त्येति॒ प्रत्ये॒ वैव प्रत्ये᳚ त्येति॒ प्रत्ये॒व । \newline
39. प्रत्ये॒ वैव प्रति॒ प्रत्ये॒व ति॑ष्ठति तिष्ठ त्ये॒व प्रति॒ प्रत्ये॒व ति॑ष्ठति । \newline
40. ए॒व ति॑ष्ठति तिष्ठ त्ये॒वैव ति॑ष्ठति प्र॒जाप॑तिः प्र॒जाप॑ति स्तिष्ठ त्ये॒वैव ति॑ष्ठति प्र॒जाप॑तिः । \newline
41. ति॒ष्ठ॒ति॒ प्र॒जाप॑तिः प्र॒जाप॑ति स्तिष्ठति तिष्ठति प्र॒जाप॑ति॒र् वै वै प्र॒जाप॑ति स्तिष्ठति तिष्ठति प्र॒जाप॑ति॒र् वै । \newline
42. प्र॒जाप॑ति॒र् वै वै प्र॒जाप॑तिः प्र॒जाप॑ति॒र् वा ए॒त मे॒तं ॅवै प्र॒जाप॑तिः प्र॒जाप॑ति॒र् वा ए॒तम् । \newline
43. प्र॒जाप॑ति॒रिति॑ प्र॒जा - प॒तिः॒ । \newline
44. वा ए॒त मे॒तं ॅवै वा ए॒तम् ज्यैष्ठ्य॑कामो॒ ज्यैष्ठ्य॑काम ए॒तं ॅवै वा ए॒तम् ज्यैष्ठ्य॑कामः । \newline
45. ए॒तम् ज्यैष्ठ्य॑कामो॒ ज्यैष्ठ्य॑काम ए॒त मे॒तम् ज्यैष्ठ्य॑कामो॒ नि नि ज्यैष्ठ्य॑काम ए॒त मे॒तम् ज्यैष्ठ्य॑कामो॒ नि । \newline
46. ज्यैष्ठ्य॑कामो॒ नि नि ज्यैष्ठ्य॑कामो॒ ज्यैष्ठ्य॑कामो॒ न्य॑धत्ता धत्त॒ नि ज्यैष्ठ्य॑कामो॒ ज्यैष्ठ्य॑कामो॒ न्य॑धत्त । \newline
47. ज्यैष्ठ्य॑काम॒ इति॒ ज्यैष्ठ्य॑ - का॒मः॒ । \newline
48. न्य॑धत्ता धत्त॒ नि न्य॑धत्त॒ तत॒ स्ततो॑ ऽधत्त॒ नि न्य॑धत्त॒ ततः॑ । \newline
49. अ॒ध॒त्त॒ तत॒ स्ततो॑ ऽधत्ता धत्त॒ ततो॒ वै वै ततो॑ ऽधत्ता धत्त॒ ततो॒ वै । \newline
50. ततो॒ वै वै तत॒ स्ततो॒ वै स स वै तत॒ स्ततो॒ वै सः । \newline
51. वै स स वै वै स ज्यैष्ठ्य॒म् ज्यैष्ठ्यꣳ॒॒ स वै वै स ज्यैष्ठ्य᳚म् । \newline
52. स ज्यैष्ठ्य॒म् ज्यैष्ठ्यꣳ॒॒ स स ज्यैष्ठ्य॑ मगच्छ दगच्छ॒ज् ज्यैष्ठ्यꣳ॒॒ स स ज्यैष्ठ्य॑ मगच्छत् । \newline
53. ज्यैष्ठ्य॑ मगच्छ दगच्छ॒ज् ज्यैष्ठ्य॒म् ज्यैष्ठ्य॑ मगच्छ॒द् यो यो॑ ऽगच्छ॒ज् ज्यैष्ठ्य॒म् ज्यैष्ठ्य॑ मगच्छ॒द् यः । \newline
54. अ॒ग॒च्छ॒द् यो यो॑ ऽगच्छ दगच्छ॒द् य ए॒व मे॒वं ॅयो॑ ऽगच्छ दगच्छ॒द् य ए॒वम् । \newline
55. य ए॒व मे॒वं ॅयो य ए॒वं ॅवि॒द्वान्. वि॒द्वा ने॒वं ॅयो य ए॒वं ॅवि॒द्वान् । \newline
56. ए॒वं ॅवि॒द्वान्. वि॒द्वा ने॒व मे॒वं ॅवि॒द्वा न॒ग्नि म॒ग्निं ॅवि॒द्वा ने॒व मे॒वं ॅवि॒द्वा न॒ग्निम् । \newline
57. वि॒द्वा न॒ग्नि म॒ग्निं ॅवि॒द्वान्. वि॒द्वा न॒ग्निम् चि॑नु॒ते चि॑नु॒ते᳚ ऽग्निं ॅवि॒द्वान्. वि॒द्वा न॒ग्निम् चि॑नु॒ते । \newline
58. अ॒ग्निम् चि॑नु॒ते चि॑नु॒ते᳚ ऽग्नि म॒ग्निम् चि॑नु॒ते ज्यैष्ठ्य॒म् ज्यैष्ठ्य॑म् चिनु॒ते᳚ ऽग्नि म॒ग्निम् चि॑नु॒ते ज्यैष्ठ्य᳚म् । \newline
59. चि॒नु॒ते ज्यैष्ठ्य॒म् ज्यैष्ठ्य॑म् चिनु॒ते चि॑नु॒ते ज्यैष्ठ्य॑ मे॒वैव ज्यैष्ठ्य॑म् चिनु॒ते चि॑नु॒ते ज्यैष्ठ्य॑ मे॒व । \newline
60. ज्यैष्ठ्य॑ मे॒वैव ज्यैष्ठ्य॒म् ज्यैष्ठ्य॑ मे॒व ग॑च्छति गच्छ त्ये॒व ज्यैष्ठ्य॒म् ज्यैष्ठ्य॑ मे॒व ग॑च्छति । \newline
61. ए॒व ग॑च्छति गच्छ त्ये॒वैव ग॑च्छति । \newline
62. ग॒च्छ॒तीति॑ गच्छति । \newline
\pagebreak
\markright{ TS 5.7.7.1  \hfill https://www.vedavms.in \hfill}

\section{ TS 5.7.7.1 }

\textbf{TS 5.7.7.1 } \newline
\textbf{Samhita Paata} \newline

यदाकू॑ताथ् स॒मसु॑स्रोद्धृ॒दो वा॒ मन॑सो वा॒ संभृ॑तं॒ चक्षु॑षो वा । तमनु॒ प्रेहि॑ सुकृ॒तस्य॑ लो॒कं ॅयत्रर्.ष॑यः प्रथम॒जा ये पु॑रा॒णाः ॥ए॒तꣳ स॑धस्थ॒ परि॑ ते ददामि॒ यमा॒वहा᳚च्छेव॒धिं जा॒तवे॑दाः ।अ॒न्वा॒ग॒न्ता य॒ज्ञ्प॑तिर्वो॒ अत्र॒ तꣳ स्म॑ जानीत पर॒मे व्यो॑मन्न् ॥जा॒नी॒तादे॑नं पर॒मे व्यो॑म॒न् देवाः᳚ सधस्था वि॒द रू॒पम॑स्य । यदा॒गच्छा᳚त् - [  ] \newline

\textbf{Pada Paata} \newline

यत् । आकू॑ता॒दित्या - कू॒ता॒त् । स॒मसु॑स्रो॒दिति॑ सं - असु॑स्रोत् । हृ॒दः । वा॒ । मन॑सः । वा॒ । संभृ॑त॒मिति॒ सं - भृ॒त॒म् । चक्षु॑षः । वा॒ ॥ तम् । अनु॑ । प्रेति॑ । इ॒हि॒ । सु॒कृ॒तस्येति॑ सु - कृ॒तस्य॑ । लो॒कम् । यत्र॑ । ऋष॑यः । प्र॒थ॒म॒जा इति॑ प्रथम-जाः । ये । पु॒रा॒णाः ॥ ए॒तम् । स॒ध॒स्थेति॑ सध - स्थ॒ । परीति॑ । ते॒ । द॒दा॒मि॒ । यम् । आ॒वहा॒दित्या᳚ - वहा᳚त् । शे॒व॒धिमिति॑ शेव - धिम् । जा॒तवे॑दा॒ इति॑ जा॒त - वे॒दाः॒ ॥ अ॒न्वा॒ग॒न्तेत्य॑नु-आ॒ग॒न्ता । य॒ज्ञ्प॑ति॒रिति॑ य॒ज्ञ्-प॒तिः॒ । वः॒ । अत्र॑ । तम् । स्म॒ । जा॒नी॒त॒ । प॒र॒मे । व्यो॑म॒न्निति॒ वि-ओ॒म॒न्न् ॥ जा॒नी॒तात् । ए॒न॒म् । प॒र॒मे । व्यो॑म॒न्निति॒ वि - ओ॒म॒न्न् । देवाः᳚ । स॒ध॒स्था॒ इति॑ सध - स्थाः॒ । वि॒द । रू॒पम् । अ॒स्य॒ ॥ यत् । आ॒गच्छा॒दित्या᳚ - गच्छा᳚त् ।  \newline


\textbf{Krama Paata} \newline

यदाकू॑तात् । आकू॑ताथ् स॒मसु॑स्रोत् । आकू॑ता॒दित्या - कू॒ता॒त्॒ । स॒मसु॑स्रोद्धृ॒दः । स॒मसु॑स्रो॒दिति॑ सम् - असु॑स्रोत् । हृ॒दो वा᳚ । वा॒ मन॑सः । मन॑सो वा । वा॒ सम्भृ॑तम् । सम्भृ॑त॒म् चक्षु॑षः । सम्भृ॑त॒मिति॒ सम् - भृ॒त॒म् । चक्षु॑षो वा । वेति॑ वा ॥ तमनु॑ । अनु॒ प्र । प्रेहि॑ । इ॒हि॒ सु॒कृ॒तस्य॑ । सु॒कृ॒तस्य॑ लो॒कम् । सु॒कृ॒तस्येति॑ सु - कृ॒तस्य॑ । लो॒कम् ॅयत्र॑ । यत्रर्.ष॑यः । ऋष॑यः प्रथम॒जाः । प्र॒थ॒म॒जा ये । प्र॒थ॒म॒जा इति॑ प्रथम - जाः । ये पु॑रा॒णाः । पु॒रा॒णा इति॑ पुरा॒णाः ॥ ए॒तꣳ स॑धस्थ । स॒ध॒स्थ॒ परि॑ । स॒ध॒स्थेति॑ सध - स्थ॒ । परि॑ ते । ते॒ द॒दा॒मि॒ । द॒दा॒मि॒ यम् । यमा॒वहा᳚त् । आ॒वहा᳚च्छेवधिम् । आ॒वहा॒दित्या᳚ - वहा᳚त् । शे॒व॒धिम् जा॒तवे॑दाः । शे॒व॒धिमिति॑ शेव - धिम् । जा॒तवे॑दा॒ इति॑ जा॒त - वे॒दाः॒ ॥ अ॒न्वा॒ग॒न्ता ॅय॒ज्ञ्प॑तिः । अ॒न्वा॒ग॒न्तेत्य॑नु - आ॒ग॒न्ता । य॒ज्ञ्प॑तिर् वः । य॒ज्ञ्प॑ति॒रिति॑ य॒ज्ञ् - प॒तिः॒ । वो॒ अत्र॑ । अत्र॒ तम् । तꣳ स्म॑ । स्म॒ जा॒नी॒त॒ । जा॒नी॒त॒ प॒र॒मे । प॒र॒मे व्यो॑मन्न् । व्यो॑म॒न्निति॒ वि - ओ॒म॒न्न्॒ ॥ जा॒नी॒तादे॑नम् । ए॒न॒म् प॒र॒मे । प॒र॒मे व्यो॑मन्न् । व्यो॑म॒न् देवाः᳚ । व्यो॑म॒न्निति॒ वि - ओ॒म॒न्न्॒ । देवाः᳚ सधस्थाः । स॒ध॒स्था॒ वि॒द । स॒ध॒स्था॒ इति॑ सध - स्थाः॒ । वि॒द रू॒पम् । रू॒पम॑स्य । अ॒स्येत्य॑स्य ॥ यदा॒गच्छा᳚त् । आ॒गच्छा᳚त् प॒थिभिः॑ । आ॒गच्छा॒दित्या᳚ - गच्छा᳚त् \newline

\textbf{Jatai Paata} \newline

1. यदाकू॑ता॒ दाकू॑ता॒द् यद् यदाकू॑तात् । \newline
2. आकू॑ताथ् स॒मसु॑स्रोथ् स॒मसु॑स्रो॒ दाकू॑ता॒ दाकू॑ताथ् स॒मसु॑स्रोत् । \newline
3. आकू॑ता॒दित्या - कू॒ता॒त् । \newline
4. स॒मसु॑स्रो द्धृ॒दो हृ॒दः स॒मसु॑स्रोथ् स॒मसु॑स्रो द्धृ॒दः । \newline
5. स॒मसु॑स्रो॒दिति॑ सं - असु॑स्रोत् । \newline
6. हृ॒दो वा॑ वा हृ॒दो हृ॒दो वा᳚ । \newline
7. वा॒ मन॑सो॒ मन॑सो वा वा॒ मन॑सः । \newline
8. मन॑सो वा वा॒ मन॑सो॒ मन॑सो वा । \newline
9. वा॒ संभृ॑तꣳ॒॒ संभृ॑तं ॅवा वा॒ संभृ॑तम् । \newline
10. संभृ॑त॒म् चक्षु॑ष॒ श्चक्षु॑षः॒ संभृ॑तꣳ॒॒ संभृ॑त॒म् चक्षु॑षः । \newline
11. संभृ॑त॒मिति॒ सं - भृ॒त॒म् । \newline
12. चक्षु॑षो वा वा॒ चक्षु॑ष॒ श्चक्षु॑षो वा । \newline
13. वेति॑ वा । \newline
14. त मन्वनु॒ तम् त मनु॑ । \newline
15. अनु॒ प्र प्राण्वनु॒ प्र । \newline
16. प्रेही॑हि॒ प्र प्रेहि॑ । \newline
17. इ॒हि॒ सु॒कृ॒तस्य॑ सुकृ॒त स्ये॑हीहि सुकृ॒तस्य॑ । \newline
18. सु॒कृ॒तस्य॑ लो॒कम् ॅलो॒कꣳ सु॑कृ॒तस्य॑ सुकृ॒तस्य॑ लो॒कम् । \newline
19. सु॒कृ॒तस्येति॑ सु - कृ॒तस्य॑ । \newline
20. लो॒कं ॅयत्र॒ यत्र॑ लो॒कम् ॅलो॒कं ॅयत्र॑ । \newline
21. यत्र र्.ष॑य॒ ऋष॑यो॒ यत्र॒ यत्र र्.ष॑यः । \newline
22. ऋष॑यः प्रथम॒जाः प्र॑थम॒जा ऋष॑य॒ ऋष॑यः प्रथम॒जाः । \newline
23. प्र॒थ॒म॒जा ये ये प्र॑थम॒जाः प्र॑थम॒जा ये । \newline
24. प्र॒थ॒म॒जा इति॑ प्रथम - जाः । \newline
25. ये पु॑रा॒णाः पु॑रा॒णा ये ये पु॑रा॒णाः । \newline
26. पु॒रा॒णा इति॑ पुरा॒णाः । \newline
27. ए॒तꣳ स॑धस्थ सधस्थै॒त मे॒तꣳ स॑धस्थ । \newline
28. स॒ध॒स्थ॒ परि॒ परि॑ सधस्थ सधस्थ॒ परि॑ । \newline
29. स॒ध॒स्थेति॑ सध - स्थ॒ । \newline
30. परि॑ ते ते॒ परि॒ परि॑ ते । \newline
31. ते॒ द॒दा॒मि॒ द॒दा॒मि॒ ते॒ ते॒ द॒दा॒मि॒ । \newline
32. द॒दा॒मि॒ यं ॅयम् द॑दामि ददामि॒ यम् । \newline
33. य मा॒वहा॑ दा॒वहा॒द् यं ॅय मा॒वहा᳚त् । \newline
34. आ॒वहा᳚च् छेव॒धिꣳ शे॑व॒धि मा॒वहा॑ दा॒वहा᳚च् छेव॒धिम् । \newline
35. आ॒वहा॒दित्या᳚ - वहा᳚त् । \newline
36. शे॒व॒धिम् जा॒तवे॑दा जा॒तवे॑दाः शेव॒धिꣳ शे॑व॒धिम् जा॒तवे॑दाः । \newline
37. शे॒व॒धिमिति॑ शेव - धिम् । \newline
38. जा॒तवे॑दा॒ इति॑ जा॒त - वे॒दाः॒ । \newline
39. अ॒न्वा॒ग॒न्ता य॒ज्ञ्प॑तिर् य॒ज्ञ्प॑ति रन्वाग॒न्ता ऽन्वा॑ग॒न्ता य॒ज्ञ्प॑तिः । \newline
40. अ॒न्वा॒ग॒न्तेत्य॑नु - आ॒ग॒न्ता । \newline
41. य॒ज्ञ्प॑तिर् वो वो य॒ज्ञ्प॑तिर् य॒ज्ञ्प॑तिर् वः । \newline
42. य॒ज्ञ्प॑ति॒रिति॑ य॒ज्ञ् - प॒तिः॒ । \newline
43. वो॒ अत्रात्र॑ वो वो॒ अत्र॑ । \newline
44. अत्र॒ तम् त मत्रात्र॒ तम् । \newline
45. तꣳ स्म॑ स्म॒ तम् तꣳ स्म॑ । \newline
46. स्म॒ जा॒नी॒त॒ जा॒नी॒त॒ स्म॒ स्म॒ जा॒नी॒त॒ । \newline
47. जा॒नी॒त॒ प॒र॒मे प॑र॒मे जा॑नीत जानीत पर॒मे । \newline
48. प॒र॒मे व्यो॑म॒न् व्यो॑मन् पर॒मे प॑र॒मे व्यो॑मन्न् । \newline
49. व्यो॑म॒न्निति॒ वि - ओ॒म॒न्न् । \newline
50. जा॒नी॒ता दे॑न मेनम् जानी॒ताज् जा॑नी॒ता दे॑नम् । \newline
51. ए॒न॒म् प॒र॒मे प॑र॒म ए॑न मेनम् पर॒मे । \newline
52. प॒र॒मे व्यो॑म॒न् व्यो॑मन् पर॒मे प॑र॒मे व्यो॑मन्न् । \newline
53. व्यो॑म॒न् देवा॒ देवा॒ व्यो॑म॒न् व्यो॑म॒न् देवाः᳚ । \newline
54. व्यो॑म॒न्निति॒ वि - ओ॒म॒न्न् । \newline
55. देवाः᳚ सधस्थाः सधस्था॒ देवा॒ देवाः᳚ सधस्थाः । \newline
56. स॒ध॒स्था॒ वि॒द वि॒द स॑धस्थाः सधस्था वि॒द । \newline
57. स॒ध॒स्था॒ इति॑ सध - स्थाः॒ । \newline
58. वि॒द रू॒पꣳ रू॒पं ॅवि॒द वि॒द रू॒पम् । \newline
59. रू॒प म॑स्यास्य रू॒पꣳ रू॒प म॑स्य । \newline
60. अ॒स्येत्य॑स्य । \newline
61. यदा॒गच्छा॑ दा॒गच्छा॒द् यद् यदा॒गच्छा᳚त् । \newline
62. आ॒गच्छा᳚त् प॒थिभिः॑ प॒थिभि॑ रा॒गच्छा॑ दा॒गच्छा᳚त् प॒थिभिः॑ । \newline
63. आ॒गच्छा॒दित्या᳚ - गच्छा᳚त् । \newline

\textbf{Ghana Paata } \newline

1. यदाकू॑ता॒ दाकू॑ता॒द् यद् यदाकू॑ताथ् स॒मसु॑स्रोथ् स॒मसु॑स्रो॒ दाकू॑ता॒द् यद् यदाकू॑ताथ् स॒मसु॑स्रोत् । \newline
2. आकू॑ताथ् स॒मसु॑स्रोथ् स॒मसु॑स्रो॒ दाकू॑ता॒ दाकू॑ताथ् स॒मसु॑स्रो द्धृ॒दो हृ॒दः स॒मसु॑स्रो॒ दाकू॑ता॒ दाकू॑ताथ् स॒मसु॑स्रो द्धृ॒दः । \newline
3. आकू॑ता॒दित्या - कू॒ता॒त् । \newline
4. स॒मसु॑स्रो द्धृ॒दो हृ॒दः स॒मसु॑स्रोथ् स॒मसु॑स्रो द्धृ॒दो वा॑ वा हृ॒दः स॒मसु॑स्रोथ् स॒मसु॑स्रो
द्धृ॒दो वा᳚ । \newline
5. स॒मसु॑स्रो॒दिति॑ सं - असु॑स्रोत् । \newline
6. हृ॒दो वा॑ वा हृ॒दो हृ॒दो वा॒ मन॑सो॒ मन॑सो वा हृ॒दो हृ॒दो वा॒ मन॑सः । \newline
7. वा॒ मन॑सो॒ मन॑सो वा वा॒ मन॑सो वा वा॒ मन॑सो वा वा॒ मन॑सो वा । \newline
8. मन॑सो वा वा॒ मन॑सो॒ मन॑सो वा॒ संभृ॑तꣳ॒॒ संभृ॑तं ॅवा॒ मन॑सो॒ मन॑सो वा॒ संभृ॑तम् । \newline
9. वा॒ संभृ॑तꣳ॒॒ संभृ॑तं ॅवा वा॒ संभृ॑त॒म् चक्षु॑ष॒ श्चक्षु॑षः॒ संभृ॑तं ॅवा वा॒ संभृ॑त॒म् चक्षु॑षः । \newline
10. संभृ॑त॒म् चक्षु॑ष॒ श्चक्षु॑षः॒ संभृ॑तꣳ॒॒ संभृ॑त॒म् चक्षु॑षो वा वा॒ चक्षु॑षः॒ संभृ॑तꣳ॒॒ संभृ॑त॒म् चक्षु॑षो वा । \newline
11. संभृ॑त॒मिति॒ सं - भृ॒त॒म् । \newline
12. चक्षु॑षो वा वा॒ चक्षु॑ष॒ श्चक्षु॑षो वा । \newline
13. वेति॑ वा । \newline
14. त मन्वनु॒ तम् त मनु॒ प्र प्राणु॒ तम् त मनु॒ प्र । \newline
15. अनु॒ प्र प्राण्वनु॒ प्रेही॑हि॒ प्राण्वनु॒ प्रेहि॑ । \newline
16. प्रेही॑हि॒ प्र प्रेहि॑ सुकृ॒तस्य॑ सुकृ॒तस्ये॑हि॒ प्र प्रेहि॑ सुकृ॒तस्य॑ । \newline
17. इ॒हि॒ सु॒कृ॒तस्य॑ सुकृ॒तस्ये॑हीहि सुकृ॒तस्य॑ लो॒कम् ॅलो॒कꣳ सु॑कृ॒तस्ये॑हीहि सुकृ॒तस्य॑ लो॒कम् । \newline
18. सु॒कृ॒तस्य॑ लो॒कम् ॅलो॒कꣳ सु॑कृ॒तस्य॑ सुकृ॒तस्य॑ लो॒कं ॅयत्र॒ यत्र॑ लो॒कꣳ सु॑कृ॒तस्य॑ सुकृ॒तस्य॑ लो॒कं ॅयत्र॑ । \newline
19. सु॒कृ॒तस्येति॑ सु - कृ॒तस्य॑ । \newline
20. लो॒कं ॅयत्र॒ यत्र॑ लो॒कम् ॅलो॒कं ॅयत्र र्.ष॑य॒ ऋष॑यो॒ यत्र॑ लो॒कम् ॅलो॒कं ॅयत्र र्.ष॑यः । \newline
21. यत्र र्.ष॑य॒ ऋष॑यो॒ यत्र॒ यत्र र्.ष॑यः प्रथम॒जाः प्र॑थम॒जा ऋष॑यो॒ यत्र॒ यत्र र्.ष॑यः प्रथम॒जाः । \newline
22. ऋष॑यः प्रथम॒जाः प्र॑थम॒जा ऋष॑य॒ ऋष॑यः प्रथम॒जा ये ये प्र॑थम॒जा ऋष॑य॒ ऋष॑यः प्रथम॒जा ये । \newline
23. प्र॒थ॒म॒जा ये ये प्र॑थम॒जाः प्र॑थम॒जा ये पु॑रा॒णाः पु॑रा॒णा ये प्र॑थम॒जाः प्र॑थम॒जा ये पु॑रा॒णाः । \newline
24. प्र॒थ॒म॒जा इति॑ प्रथम - जाः । \newline
25. ये पु॑रा॒णाः पु॑रा॒णा ये ये पु॑रा॒णाः । \newline
26. पु॒रा॒णा इति॑ पुरा॒णाः । \newline
27. ए॒तꣳ स॑धस्थ सधस्थै॒त मे॒तꣳ स॑धस्थ॒ परि॒ परि॑ सधस्थै॒त मे॒तꣳ स॑धस्थ॒ परि॑ । \newline
28. स॒ध॒स्थ॒ परि॒ परि॑ सधस्थ सधस्थ॒ परि॑ ते ते॒ परि॑ सधस्थ सधस्थ॒ परि॑ ते । \newline
29. स॒ध॒स्थेति॑ सध - स्थ॒ । \newline
30. परि॑ ते ते॒ परि॒ परि॑ ते ददामि ददामि ते॒ परि॒ परि॑ ते ददामि । \newline
31. ते॒ द॒दा॒मि॒ द॒दा॒मि॒ ते॒ ते॒ द॒दा॒मि॒ यं ॅयम् द॑दामि ते ते ददामि॒ यम् । \newline
32. द॒दा॒मि॒ यं ॅयम् द॑दामि ददामि॒ य मा॒वहा॑ दा॒वहा॒द् यम् द॑दामि ददामि॒ य मा॒वहा᳚त् । \newline
33. य मा॒वहा॑ दा॒वहा॒द् यं ॅय मा॒वहा᳚च् छेव॒धिꣳ शे॑व॒धि मा॒वहा॒द् यं ॅय मा॒वहा᳚च् छेव॒धिम् । \newline
34. आ॒वहा᳚च् छेव॒धिꣳ शे॑व॒धि मा॒वहा॑ दा॒वहा᳚च् छेव॒धिम् जा॒तवे॑दा जा॒तवे॑दाः शेव॒धि मा॒वहा॑ दा॒वहा᳚च् छेव॒धिम् जा॒तवे॑दाः । \newline
35. आ॒वहा॒दित्या᳚ - वहा᳚त् । \newline
36. शे॒व॒धिम् जा॒तवे॑दा जा॒तवे॑दाः शेव॒धिꣳ शे॑व॒धिम् जा॒तवे॑दाः । \newline
37. शे॒व॒धिमिति॑ शेव - धिम् । \newline
38. जा॒तवे॑दा॒ इति॑ जा॒त - वे॒दाः॒ । \newline
39. अ॒न्वा॒ग॒न्ता य॒ज्ञ्प॑तिर् य॒ज्ञ्प॑ति रन्वाग॒न्ता ऽन्वा॑ग॒न्ता य॒ज्ञ्प॑तिर् वो वो य॒ज्ञ्प॑ति रन्वाग॒न्ता ऽन्वा॑ग॒न्ता य॒ज्ञ्प॑तिर् वः । \newline
40. अ॒न्वा॒ग॒न्तेत्य॑नु - आ॒ग॒न्ता । \newline
41. य॒ज्ञ्प॑तिर् वो वो य॒ज्ञ्प॑तिर् य॒ज्ञ्प॑तिर् वो॒ अत्रात्र॑ वो य॒ज्ञ्प॑तिर् य॒ज्ञ्प॑तिर् वो॒ अत्र॑ । \newline
42. य॒ज्ञ्प॑ति॒रिति॑ य॒ज्ञ् - प॒तिः॒ । \newline
43. वो॒ अत्रात्र॑ वो वो॒ अत्र॒ तम् त मत्र॑ वो वो॒ अत्र॒ तम् । \newline
44. अत्र॒ तम् त मत्रात्र॒ तꣳ स्म॑ स्म॒ त मत्रात्र॒ तꣳ स्म॑ । \newline
45. तꣳ स्म॑ स्म॒ तम् तꣳ स्म॑ जानीत जानीत स्म॒ तम् तꣳ स्म॑ जानीत । \newline
46. स्म॒ जा॒नी॒त॒ जा॒नी॒त॒ स्म॒ स्म॒ जा॒नी॒त॒ प॒र॒मे प॑र॒मे जा॑नीत स्म स्म जानीत पर॒मे । \newline
47. जा॒नी॒त॒ प॒र॒मे प॑र॒मे जा॑नीत जानीत पर॒मे व्यो॑म॒न् व्यो॑मन् पर॒मे जा॑नीत जानीत पर॒मे व्यो॑मन्न् । \newline
48. प॒र॒मे व्यो॑म॒न् व्यो॑मन् पर॒मे प॑र॒मे व्यो॑मन्न् । \newline
49. व्यो॑म॒न्निति॒ वि - ओ॒म॒न्न् । \newline
50. जा॒नी॒ता दे॑न मेनम् जानी॒ताज् जा॑नी॒ता दे॑नम् पर॒मे प॑र॒म ए॑नम् जानी॒ताज् जा॑नी॒ता दे॑नम् पर॒मे । \newline
51. ए॒न॒म् प॒र॒मे प॑र॒म ए॑न मेनम् पर॒मे व्यो॑म॒न् व्यो॑मन् पर॒म ए॑न मेनम् पर॒मे व्यो॑मन्न् । \newline
52. प॒र॒मे व्यो॑म॒न् व्यो॑मन् पर॒मे प॑र॒मे व्यो॑म॒न् देवा॒ देवा॒ व्यो॑मन् पर॒मे प॑र॒मे व्यो॑म॒न् देवाः᳚ । \newline
53. व्यो॑म॒न् देवा॒ देवा॒ व्यो॑म॒न् व्यो॑म॒न् देवाः᳚ सधस्थाः सधस्था॒ देवा॒ व्यो॑म॒न् व्यो॑म॒न् देवाः᳚ सधस्थाः । \newline
54. व्यो॑म॒न्निति॒ वि - ओ॒म॒न्न् । \newline
55. देवाः᳚ सधस्थाः सधस्था॒ देवा॒ देवाः᳚ सधस्था वि॒द वि॒द स॑धस्था॒ देवा॒ देवाः᳚ सधस्था वि॒द । \newline
56. स॒ध॒स्था॒ वि॒द वि॒द स॑धस्थाः सधस्था वि॒द रू॒पꣳ रू॒पं ॅवि॒द स॑धस्थाः सधस्था वि॒द रू॒पम् । \newline
57. स॒ध॒स्था॒ इति॑ सध - स्थाः॒ । \newline
58. वि॒द रू॒पꣳ रू॒पं ॅवि॒द वि॒द रू॒प म॑स्यास्य रू॒पं ॅवि॒द वि॒द रू॒प म॑स्य । \newline
59. रू॒प म॑स्यास्य रू॒पꣳ रू॒प म॑स्य । \newline
60. अ॒स्येत्य॑स्य । \newline
61. यदा॒गच्छा॑ दा॒गच्छा॒द् यद् यदा॒गच्छा᳚त् प॒थिभिः॑ प॒थिभि॑ रा॒गच्छा॒द् यद् यदा॒गच्छा᳚त् प॒थिभिः॑ । \newline
62. आ॒गच्छा᳚त् प॒थिभिः॑ प॒थिभि॑ रा॒गच्छा॑ दा॒गच्छा᳚त् प॒थिभि॑र् देव॒यानै᳚र् देव॒यानैः᳚ प॒थिभि॑ रा॒गच्छा॑ दा॒गच्छा᳚त् प॒थिभि॑र् देव॒यानैः᳚ । \newline
63. आ॒गच्छा॒दित्या᳚ - गच्छा᳚त् । \newline
\pagebreak
\markright{ TS 5.7.7.2  \hfill https://www.vedavms.in \hfill}

\section{ TS 5.7.7.2 }

\textbf{TS 5.7.7.2 } \newline
\textbf{Samhita Paata} \newline

प॒थिभि॑र्देव॒यानै॑रिष्टापू॒र्ते कृ॑णुतादा॒विर॑स्मै ॥ सं प्र च्य॑वद्ध्व॒मनु॒ सं प्र या॒ताग्ने॑ प॒थो दे॑व॒याना᳚न् कृणुद्ध्वं ।अ॒स्मिन्थ् स॒धस्थे॒ अद्ध्युत्त॑रस्मि॒न् विश्वे॑ देवा॒ यज॑मानश्च सीदत ॥प्र॒स्त॒रेण॑ परि॒धिना᳚ स्रु॒चा वेद्या॑ च ब॒र्॒.हिषा᳚ । ऋ॒चेमं ॅय॒ज्ञ्ं नो॑ वह॒ सुव॑र्दे॒वेषु॒ गन्त॑वे ॥यदि॒ष्टं ॅयत् प॑रा॒दानं॒ ॅयद्द॒त्तं ॅया च॒ दक्षि॑णा । त - [  ] \newline

\textbf{Pada Paata} \newline

प॒थिभि॒रिति॑ प॒थि - भिः॒ । दे॒व॒यानै॒रिति॑ देव - यानैः᳚ । इ॒ष्टा॒पू॒र्ते इती᳚ष्टा-पू॒र्ते । कृ॒णु॒ता॒त् । आ॒विः । अ॒स्मै॒ ॥ सम् । प्रेति॑ । च्य॒व॒द्ध्व॒म् । अन्विति॑ । सम् । प्रेति॑ । या॒त॒ । अग्ने᳚ । प॒थः । दे॒व॒याना॒निति॑ देव - यानान्॑ । कृ॒णु॒द्ध्व॒म् ॥ अ॒स्मिन्न् । स॒धस्थ॒ इति॑ स॒ध - स्थे॒ । अधीति॑ । उत्त॑रस्मि॒न्नित्युत् - त॒र॒स्मि॒न्न् । विश्वे᳚ । दे॒वाः॒ । यज॑मानः । च॒ । सी॒द॒त॒ ॥ प्र॒स्त॒रेणेति॑ प्र - स्त॒रेण॑ । प॒रि॒धिनेति॑ परि - धिना᳚ । स्रु॒चा । वेद्या᳚ । च॒ । ब॒र्॒.हिषा᳚ ॥ ऋ॒चा । इ॒मम् । य॒ज्ञ्म् । नः॒ । व॒ह॒ । सुवः॑ । दे॒वेषु॑ । गन्त॑वे ॥ यत् । इ॒ष्टम् । यत् । प॒रा॒दान॒मिति॑ परा - दान᳚म् । यत् । द॒त्तम् । या । च॒ । दक्षि॑णा ॥ तत् ।  \newline


\textbf{Krama Paata} \newline

प॒थिभि॑र् देव॒यानैः᳚ । प॒थिभि॒रिति॑ प॒थि - भिः॒ । दे॒व॒यानै॑रिष्टापू॒र्ते । दे॒व॒यानै॒रिति॑ देव - यानैः᳚ । इ॒ष्टा॒पू॒र्ते कृ॑णुतात् । इ॒ष्टा॒पू॒र्ते इती᳚ष्टा - पू॒र्ते । कृ॒णु॒ता॒दा॒विः । आ॒विर॑स्मै । अ॒स्मा॒ इत्य॑स्मै ॥ सम् प्र । प्र च्य॑वद्ध्वम् । च्य॒व॒द्ध्व॒मनु॑ । अनु॒ सम् । सम् प्र । प्र या॑त । या॒ताग्ने᳚ । अग्ने॑ प॒थः । प॒थो दे॑व॒यानान्॑ । दे॒व॒याना᳚न् कृणुद्ध्वम् । दे॒व॒याना॒निति॑ देव - यानान्॑ । कृ॒णु॒द्ध्व॒मिति॑ कृणुद्ध्वम् ॥ अ॒स्मिन्थ् स॒धस्थे᳚ । स॒धस्थे॒ अधि॑ । स॒धस्थ॒ इति॑ स॒ध - स्थे॒ । अद्ध्युत्त॑रस्मिन्न् । उत्त॑रस्मि॒न् विश्वे᳚ । उत्त॑रस्मि॒न्नित्युत् - त॒र॒स्मि॒न्न्॒ । विश्वे॑ देवाः । दे॒वा॒ यज॑मानः । यज॑मानश्च । च॒ सी॒द॒त॒ । सी॒द॒तेति॑ सीदत ॥ प्र॒स्त॒रेण॑ परि॒धिना᳚ । प्र॒स्त॒रेणेति॑ प्र - स्त॒रेण॑ । प॒रि॒धिना᳚ स्रु॒चा । प॒रि॒धिनेति॑ परि - धिना᳚ । स्रु॒चा वेद्या᳚ । वेद्या॑ च । च॒ ब॒र्॒.हिषा᳚ । ब॒र्॒.हिषेति॑ ब॒र्॒.हिषा᳚ ॥ ऋ॒चेमम् । इ॒मम् ॅय॒ज्ञ्म् । य॒ज्ञ्म् नः॑ । नो॒ व॒ह॒ । व॒ह॒ सुवः॑ । सुव॑र् दे॒वेषु॑ । दे॒वेषु॒ गन्त॑वे । गन्त॑व॒ इति॒ गन्त॑वे ॥ यदि॒ष्टम् । इ॒ष्टम् ॅयत् । यत् प॑रा॒दान᳚म् । प॒रा॒दान॒म् ॅयत् । प॒रा॒दान॒मिति॑ परा - दान᳚म् । यद् द॒त्तम् । द॒त्तम् ॅया । या च॑ । च॒ दक्षि॑णा । दक्षि॒णेति॒ दक्षि॑णा ॥ तद॒ग्निः \newline

\textbf{Jatai Paata} \newline

1. प॒थिभि॑र् देव॒यानै᳚र् देव॒यानैः᳚ प॒थिभिः॑ प॒थिभि॑र् देव॒यानैः᳚ । \newline
2. प॒थिभि॒रिति॑ प॒थि - भिः॒ । \newline
3. दे॒व॒यानै॑ रिष्टापू॒र्ते इ॑ष्टापू॒र्ते दे॑व॒यानै᳚र् देव॒यानै॑ रिष्टापू॒र्ते । \newline
4. दे॒व॒यानै॒रिति॑ देव - यानैः᳚ । \newline
5. इ॒ष्टा॒पू॒र्ते कृ॑णुतात् कृणुता दिष्टापू॒र्ते इ॑ष्टापू॒र्ते कृ॑णुतात् । \newline
6. इ॒ष्टा॒पू॒र्ते इती᳚ष्टा - पू॒र्ते । \newline
7. कृ॒णु॒ता॒ दा॒वि रा॒विष् कृ॑णुतात् कृणुता दा॒विः । \newline
8. आ॒वि र॑स्मा अस्मा आ॒विरा॒वि र॑स्मै । \newline
9. अ॒स्मा इत्य॑स्मै । \newline
10. सम् प्र प्र सꣳ सम् प्र । \newline
11. प्र च्य॑वद्ध्वम् च्यवद्ध्व॒म् प्र प्र च्य॑वद्ध्वम् । \newline
12. च्य॒व॒द्ध्व॒ मन्वनु॑ च्यवद्ध्वम् च्यवद्ध्व॒ मनु॑ । \newline
13. अनु॒ सꣳ स मन्वनु॒ सम् । \newline
14. सम् प्र प्र सꣳ सम् प्र । \newline
15. प्र या॑त यात॒ प्र प्र या॑त । \newline
16. या॒ताग्ने ऽग्ने॑ यात या॒ताग्ने᳚ । \newline
17. अग्ने॑ प॒थः प॒थो ऽग्ने ऽग्ने॑ प॒थः । \newline
18. प॒थो दे॑व॒याना᳚न् देव॒याना᳚न् प॒थः प॒थो दे॑व॒यानान्॑ । \newline
19. दे॒व॒याना᳚न् कृणुद्ध्वम् कृणुद्ध्वम् देव॒याना᳚न् देव॒याना᳚न् कृणुद्ध्वम् । \newline
20. दे॒व॒याना॒निति॑ देव - यानान्॑ । \newline
21. कृ॒णु॒द्ध्व॒मिति॑ कृणुद्ध्वम् । \newline
22. अ॒स्मिन् थ्स॒धस्थे॑ स॒धस्थे॑ अ॒स्मिन् न॒स्मिन् थ्स॒धस्थे᳚ । \newline
23. स॒धस्थे॒ अध्यधि॑ स॒धस्थे॑ स॒धस्थे॒ अधि॑ । \newline
24. स॒धस्थ॒ इति॑ स॒ध - स्थे॒ । \newline
25. अध्युत्त॑रस्मि॒न् नुत्त॑रस्मि॒न् नध्यध्युत् त॑रस्मिन्न् । \newline
26. उत्त॑रस्मि॒न्॒. विश्वे॒ विश्व॒ उत्त॑रस्मि॒न् नुत्त॑रस्मि॒न्॒. विश्वे᳚ । \newline
27. उत्त॑रस्मि॒न्नित्युत् - त॒र॒स्मि॒न्न् । \newline
28. विश्वे॑ देवा देवा॒ विश्वे॒ विश्वे॑ देवाः । \newline
29. दे॒वा॒ यज॑मानो॒ यज॑मानो देवा देवा॒ यज॑मानः । \newline
30. यज॑मानश्च च॒ यज॑मानो॒ यज॑मानश्च । \newline
31. च॒ सी॒द॒त॒ सी॒द॒त॒ च॒ च॒ सी॒द॒त॒ । \newline
32. सी॒द॒तेति॑ सीदत । \newline
33. प्र॒स्त॒रेण॑ परि॒धिना॑ परि॒धिना᳚ प्रस्त॒रेण॑ प्रस्त॒रेण॑ परि॒धिना᳚ । \newline
34. प्र॒स्त॒रेणेति॑ प्र - स्त॒रेण॑ । \newline
35. प॒रि॒धिना᳚ स्रु॒चा स्रु॒चा प॑रि॒धिना॑ परि॒धिना᳚ स्रु॒चा । \newline
36. प॒रि॒धिनेति॑ परि - धिना᳚ । \newline
37. स्रु॒चा वेद्या॒ वेद्या᳚ स्रु॒चा स्रु॒चा वेद्या᳚ । \newline
38. वेद्या॑ च च॒ वेद्या॒ वेद्या॑ च । \newline
39. च॒ ब॒र्॒.हिषा॑ ब॒र्॒.हिषा॑ च च ब॒र्॒.हिषा᳚ । \newline
40. ब॒र्॒.हिषेति॑ ब॒र्॒.हिषा᳚ । \newline
41. ऋ॒चेम मि॒म मृ॒चर् चेमम् । \newline
42. इ॒मं ॅय॒ज्ञ्ं ॅय॒ज्ञ् मि॒म मि॒मं ॅय॒ज्ञ्म् । \newline
43. य॒ज्ञ्म् नो॑ नो य॒ज्ञ्ं ॅय॒ज्ञ्म् नः॑ । \newline
44. नो॒ व॒ह॒ व॒ह॒ नो॒ नो॒ व॒ह॒ । \newline
45. व॒ह॒ सुवः॒ सुव॑र् वह वह॒ सुवः॑ । \newline
46. सुव॑र् दे॒वेषु॑ दे॒वेषु॒ सुवः॒ सुव॑र् दे॒वेषु॑ । \newline
47. दे॒वेषु॒ गन्त॑वे॒ गन्त॑वे दे॒वेषु॑ दे॒वेषु॒ गन्त॑वे । \newline
48. गन्त॑व॒ इति॒ गन्त॑वे । \newline
49. यदि॒ष्ट मि॒ष्टं ॅयद् यदि॒ष्टम् । \newline
50. इ॒ष्टं ॅयद् यदि॒ष्ट मि॒ष्टं ॅयत् । \newline
51. यत् प॑रा॒दान॑म् परा॒दानं॒ ॅयद् यत् प॑रा॒दान᳚म् । \newline
52. प॒रा॒दानं॒ ॅयद् यत् प॑रा॒दान॑म् परा॒दानं॒ ॅयत् । \newline
53. प॒रा॒दान॒मिति॑ परा - दान᳚म् । \newline
54. यद् द॒त्तम् द॒त्तं ॅयद् यद् द॒त्तम् । \newline
55. द॒त्तं ॅया या द॒त्तम् द॒त्तं ॅया । \newline
56. या च॑ च॒ या या च॑ । \newline
57. च॒ दक्षि॑णा॒ दक्षि॑णा च च॒ दक्षि॑णा । \newline
58. दक्षि॒णेति॒ दक्षि॑णा । \newline
59. तद॒ग्नि र॒ग्नि स्तत् तद॒ग्निः । \newline

\textbf{Ghana Paata } \newline

1. प॒थिभि॑र् देव॒यानै᳚र् देव॒यानैः᳚ प॒थिभिः॑ प॒थिभि॑र् देव॒यानै॑ रिष्टापू॒र्ते इ॑ष्टापू॒र्ते दे॑व॒यानैः᳚ प॒थिभिः॑ प॒थिभि॑र् देव॒यानै॑ रिष्टापू॒र्ते । \newline
2. प॒थिभि॒रिति॑ प॒थि - भिः॒ । \newline
3. दे॒व॒यानै॑ रिष्टापू॒र्ते इ॑ष्टापू॒र्ते दे॑व॒यानै᳚र् देव॒यानै॑ रिष्टापू॒र्ते कृ॑णुतात् कृणुता दिष्टापू॒र्ते दे॑व॒यानै᳚र् देव॒यानै॑ रिष्टापू॒र्ते कृ॑णुतात् । \newline
4. दे॒व॒यानै॒रिति॑ देव - यानैः᳚ । \newline
5. इ॒ष्टा॒पू॒र्ते कृ॑णुतात् कृणुता दिष्टापू॒र्ते इ॑ष्टापू॒र्ते कृ॑णुता दा॒वि रा॒विष् कृ॑णुता दिष्टापू॒र्ते इ॑ष्टापू॒र्ते कृ॑णुता दा॒विः । \newline
6. इ॒ष्टा॒पू॒र्ते इती᳚ष्टा - पू॒र्ते । \newline
7. कृ॒णु॒ता॒ दा॒वि रा॒विष् कृ॑णुतात् कृणुता दा॒वि र॑स्मा अस्मा आ॒विष् कृ॑णुतात् कृणुता दा॒वि र॑स्मै । \newline
8. आ॒वि र॑स्मा अस्मा आ॒वि रा॒वि र॑स्मै । \newline
9. अ॒स्मा इत्य॑स्मै । \newline
10. सम् प्र प्र सꣳ सम् प्र च्य॑वद्ध्वम् च्यवद्ध्व॒म् प्र सꣳ सम् प्र च्य॑वद्ध्वम् । \newline
11. प्र च्य॑वद्ध्वम् च्यवद्ध्व॒म् प्र प्र च्य॑वद्ध्व॒ मन्वनु॑ च्यवद्ध्व॒म् प्र प्र च्य॑वद्ध्व॒ मनु॑ । \newline
12. च्य॒व॒द्ध्व॒ मन्वनु॑ च्यवद्ध्वम् च्यवद्ध्व॒ मनु॒ सꣳ स मनु॑ च्यवद्ध्वम् च्यवद्ध्व॒ मनु॒ सम् । \newline
13. अनु॒ सꣳ स मन्वनु॒ सम् प्र प्र स मन्वनु॒ सम् प्र । \newline
14. सम् प्र प्र सꣳ सम् प्र या॑त यात॒ प्र सꣳ सम् प्र या॑त । \newline
15. प्र या॑त यात॒ प्र प्र या॒ताग्ने ऽग्ने॑ यात॒ प्र प्र या॒ताग्ने᳚ । \newline
16. या॒ताग्ने ऽग्ने॑ यात या॒ताग्ने॑ प॒थः प॒थो ऽग्ने॑ यात या॒ताग्ने॑ प॒थः । \newline
17. अग्ने॑ प॒थः प॒थो ऽग्ने ऽग्ने॑ प॒थो दे॑व॒याना᳚न् देव॒याना᳚न् प॒थो ऽग्ने ऽग्ने॑ प॒थो दे॑व॒यानान्॑ । \newline
18. प॒थो दे॑व॒याना᳚न् देव॒याना᳚न् प॒थः प॒थो दे॑व॒याना᳚न् कृणुद्ध्वम् कृणुद्ध्वम् देव॒याना᳚न् प॒थः प॒थो दे॑व॒याना᳚न् कृणुद्ध्वम् । \newline
19. दे॒व॒याना᳚न् कृणुद्ध्वम् कृणुद्ध्वम् देव॒याना᳚न् देव॒याना᳚न् कृणुद्ध्वम् । \newline
20. दे॒व॒याना॒निति॑ देव - यानान्॑ । \newline
21. कृ॒णु॒द्ध्व॒मिति॑ कृणुद्ध्वम् । \newline
22. अ॒स्मिन् थ्स॒धस्थे॑ स॒धस्थे॑ अ॒स्मिन् न॒स्मिन् थ्स॒धस्थे॒ अध्यधि॑ स॒धस्थे॑ अ॒स्मिन् न॒स्मिन् थ्स॒धस्थे॒ अधि॑ । \newline
23. स॒धस्थे॒ अध्यधि॑ स॒धस्थे॑ स॒धस्थे॒ अध्युत्त॑रस्मि॒न् नुत्त॑रस्मि॒न् नधि॑ स॒धस्थे॑ स॒धस्थे॒ अध्युत्त॑रस्मिन्न् । \newline
24. स॒धस्थ॒ इति॑ स॒ध - स्थे॒ । \newline
25. अध्युत्त॑रस्मि॒न् नुत्त॑रस्मि॒न् नध्य ध्युत्त॑रस्मि॒न्॒. विश्वे॒ विश्व॒ उत्त॑रस्मि॒न् नध्य ध्युत्त॑रस्मि॒न्॒. विश्वे᳚ । \newline
26. उत्त॑रस्मि॒न्॒. विश्वे॒ विश्व॒ उत्त॑रस्मि॒न् नुत्त॑रस्मि॒न्॒. विश्वे॑ देवा देवा॒ विश्व॒ उत्त॑रस्मि॒न् नुत्त॑रस्मि॒न्॒. विश्वे॑ देवाः । \newline
27. उत्त॑रस्मि॒न्नित्युत् - त॒र॒स्मि॒न्न् । \newline
28. विश्वे॑ देवा देवा॒ विश्वे॒ विश्वे॑ देवा॒ यज॑मानो॒ यज॑मानो देवा॒ विश्वे॒ विश्वे॑ देवा॒ यज॑मानः । \newline
29. दे॒वा॒ यज॑मानो॒ यज॑मानो देवा देवा॒ यज॑मानश्च च॒ यज॑मानो देवा देवा॒ यज॑मानश्च । \newline
30. यज॑मानश्च च॒ यज॑मानो॒ यज॑मानश्च सीदत सीदत च॒ यज॑मानो॒ यज॑मानश्च सीदत । \newline
31. च॒ सी॒द॒त॒ सी॒द॒त॒ च॒ च॒ सी॒द॒त॒ । \newline
32. सी॒द॒तेति॑ सीदत । \newline
33. प्र॒स्त॒रेण॑ परि॒धिना॑ परि॒धिना᳚ प्रस्त॒रेण॑ प्रस्त॒रेण॑ परि॒धिना᳚ स्रु॒चा स्रु॒चा प॑रि॒धिना᳚ प्रस्त॒रेण॑ प्रस्त॒रेण॑ परि॒धिना᳚ स्रु॒चा । \newline
34. प्र॒स्त॒रेणेति॑ प्र - स्त॒रेण॑ । \newline
35. प॒रि॒धिना᳚ स्रु॒चा स्रु॒चा प॑रि॒धिना॑ परि॒धिना᳚ स्रु॒चा वेद्या॒ वेद्या᳚ स्रु॒चा प॑रि॒धिना॑ परि॒धिना᳚ स्रु॒चा वेद्या᳚ । \newline
36. प॒रि॒धिनेति॑ परि - धिना᳚ । \newline
37. स्रु॒चा वेद्या॒ वेद्या᳚ स्रु॒चा स्रु॒चा वेद्या॑ च च॒ वेद्या᳚ स्रु॒चा स्रु॒चा वेद्या॑ च । \newline
38. वेद्या॑ च च॒ वेद्या॒ वेद्या॑ च ब॒र्॒.हिषा॑ ब॒र्॒.हिषा॑ च॒ वेद्या॒ वेद्या॑ च ब॒र्॒.हिषा᳚ । \newline
39. च॒ ब॒र्॒.हिषा॑ ब॒र्॒.हिषा॑ च च ब॒र्॒.हिषा᳚ । \newline
40. ब॒र्॒.हिषेति॑ ब॒र्॒.हिषा᳚ । \newline
41. ऋ॒चेम मि॒म मृ॒च र्‌चेमं ॅय॒ज्ञ्ं ॅय॒ज्ञ् मि॒म मृ॒च र्‌चेमं ॅय॒ज्ञ्म् । \newline
42. इ॒मं ॅय॒ज्ञ्ं ॅय॒ज्ञ् मि॒म मि॒मं ॅय॒ज्ञ्न्नो॑ नो य॒ज्ञ् मि॒म मि॒मं ॅय॒ज्ञ्न्नः॑ । \newline
43. य॒ज्ञ्न्नो॑ नो य॒ज्ञ्ं ॅय॒ज्ञ्न्नो॑ वह वह नो य॒ज्ञ्ं ॅय॒ज्ञ्न्नो॑ वह । \newline
44. नो॒ व॒ह॒ व॒ह॒ नो॒ नो॒ व॒ह॒ सुवः॒ सुव॑र् वह नो नो वह॒ सुवः॑ । \newline
45. व॒ह॒ सुवः॒ सुव॑र् वह वह॒ सुव॑र् दे॒वेषु॑ दे॒वेषु॒ सुव॑र् वह वह॒ सुव॑र् दे॒वेषु॑ । \newline
46. सुव॑र् दे॒वेषु॑ दे॒वेषु॒ सुवः॒ सुव॑र् दे॒वेषु॒ गन्त॑वे॒ गन्त॑वे दे॒वेषु॒ सुवः॒ सुव॑र् दे॒वेषु॒ गन्त॑वे । \newline
47. दे॒वेषु॒ गन्त॑वे॒ गन्त॑वे दे॒वेषु॑ दे॒वेषु॒ गन्त॑वे । \newline
48. गन्त॑व॒ इति॒ गन्त॑वे । \newline
49. यदि॒ष्ट मि॒ष्टं ॅयद् यदि॒ष्टं ॅयद् यदि॒ष्टं ॅयद् यदि॒ष्टं ॅयत् । \newline
50. इ॒ष्टं ॅयद् यदि॒ष्ट मि॒ष्टं ॅयत् प॑रा॒दान॑म् परा॒दानं॒ ॅयदि॒ष्ट मि॒ष्टं ॅयत् प॑रा॒दान᳚म् । \newline
51. यत् प॑रा॒दान॑म् परा॒दानं॒ ॅयद् यत् प॑रा॒दानं॒ ॅयद् यत् प॑रा॒दानं॒ ॅयद् यत् प॑रा॒दानं॒ ॅयत् । \newline
52. प॒रा॒दानं॒ ॅयद् यत् प॑रा॒दान॑म् परा॒दानं॒ ॅयद् द॒त्तम् द॒त्तं ॅयत् प॑रा॒दान॑म् परा॒दानं॒ ॅयद् द॒त्तम् । \newline
53. प॒रा॒दान॒मिति॑ परा - दान᳚म् । \newline
54. यद् द॒त्तम् द॒त्तं ॅयद् यद् द॒त्तं ॅया या द॒त्तं ॅयद् यद् द॒त्तं ॅया । \newline
55. द॒त्तं ॅया या द॒त्तम् द॒त्तं ॅया च॑ च॒ या द॒त्तम् द॒त्तं ॅया च॑ । \newline
56. या च॑ च॒ या या च॒ दक्षि॑णा॒ दक्षि॑णा च॒ या या च॒ दक्षि॑णा । \newline
57. च॒ दक्षि॑णा॒ दक्षि॑णा च च॒ दक्षि॑णा । \newline
58. दक्षि॒णेति॒ दक्षि॑णा । \newline
59. तद॒ग्नि र॒ग्नि स्तत् तद॒ग्निर् वै᳚श्वकर्म॒णो वै᳚श्वकर्म॒णो᳚ ऽग्नि स्तत् तद॒ग्निर् वै᳚श्वकर्म॒णः । \newline
\pagebreak
\markright{ TS 5.7.7.3  \hfill https://www.vedavms.in \hfill}

\section{ TS 5.7.7.3 }

\textbf{TS 5.7.7.3 } \newline
\textbf{Samhita Paata} \newline

-द॒ग्निर्वै᳚श्वकर्म॒णः सुव॑र्दे॒वेषु॑ नो दधत् ॥येना॑ स॒हस्रं॒ ॅवह॑सि॒ येना᳚ग्ने सर्ववेद॒सं । तेने॒मं ॅय॒ज्ञ्ं नो॑ वह॒ सुव॑र्दे॒वेषु॒ गन्त॑वे ॥येना᳚ग्ने॒ दक्षि॑णा यु॒क्ता य॒ज्ञ्ं ॅवह॑न्त्यृ॒त्विजः॑ । तेने॒मं ॅय॒ज्ञ्ं नो॑ वह॒ सुव॑र्दे॒वेषु॒ गन्त॑वे ॥येना᳚ग्ने॒ सु॒कृतः॑ प॒था मधो॒र्द्धारा᳚ व्यान॒शुः । तेने॒मं ॅय॒ज्ञ्ं नो॑ वह॒ सुव॑र्दे॒वेषु॒ गन्त॑वे ( ) ॥यत्र॒ धारा॒ अन॑पेता॒ मधो᳚र्घृ॒तस्य॑ च॒ याः । तद॒ग्निर्वै᳚श्वकर्म॒णः सुव॑र्दे॒वेषु॑ नो दधत् ॥ \newline

\textbf{Pada Paata} \newline

अ॒ग्निः । वै॒श्व॒क॒र्म॒ण इति॑ वैश्व - क॒र्म॒णः । सुवः॑ । दे॒वेषु॑ । नः॒ । द॒ध॒त् ॥ येन॑ । स॒हस्र᳚म् । वह॑सि । येन॑ । अ॒ग्ने॒ । स॒र्व॒वे॒द॒समिति॑ सर्व - वे॒द॒सम् ॥ तेन॑ । इ॒मम् । य॒ज्ञ्म् । नः॒ । व॒ह॒ । सुवः॑ । दे॒वेषु॑ । गन्त॑वे ॥ येन॑ । अ॒ग्ने॒ । दक्षि॑णाः । यु॒क्ताः । य॒ज्ञ्म् । वह॑न्ति । ऋ॒त्विजः॑ ॥ तेन॑ । इ॒मम् । य॒ज्ञ्म् । नः॒ । व॒ह॒ । सुवः॑ । दे॒वेषु॑ । गन्त॑वे ॥ येन॑ । अ॒ग्ने॒ । सु॒कृत॒ इति॑ सु - कृतः॑ । प॒था । मधोः᳚ । धाराः᳚ । व्या॒न॒शुरिति॑ वि - आ॒न॒शुः ॥ तेन॑ । इ॒मम् । य॒ज्ञ्म् । नः॒ । व॒ह॒ । सुवः॑ । दे॒वेषु॑ । गन्त॑वे ( ) ॥ यत्र॑ । धाराः᳚ । अन॑पेता॒ इत्यन॑प - इ॒ताः॒ । मधोः᳚ । घृ॒तस्य॑ । च॒ । याः ॥ तत् । अ॒ग्निः । वै॒श्व॒क॒र्म॒ण इति॑ वैश्व - क॒र्म॒णः । सुवः॑ । दे॒वेषु॑ । नः॒ । द॒ध॒त् ॥  \newline


\textbf{Krama Paata} \newline

अ॒ग्निर् वै᳚श्वकर्म॒णः । वै॒श्व॒क॒र्म॒णः सुवः॑ । वै॒श्व॒क॒र्म॒ण इति॑ वैश्व - क॒र्म॒णः । सुव॑र् दे॒वेषु॑ । दे॒वेषु॑ नः । नो॒ द॒ध॒त्॒ । द॒ध॒दिति॑ दधत् ॥ येना॑ स॒हस्र᳚म् । स॒हस्र॒म् ॅवह॑सि । वह॑सि॒ येन॑ । येना᳚ग्ने । अ॒ग्ने॒ स॒र्व॒वे॒द॒सम् । स॒र्व॒वे॒द॒समिति॑ सर्व - वे॒द॒सम् ॥ तेने॒मम् । इ॒मम् ॅय॒ज्ञ्म् । य॒ज्ञ्म् नः॑ । नो॒ व॒ह॒ । व॒ह॒ सुवः॑ । सुव॑र् दे॒वेषु॑ । दे॒वेषु॒ गन्त॑वे । गन्त॑व॒ इति॒ गन्त॑वे ॥ येना᳚ग्ने । अ॒ग्ने॒ दक्षि॑णाः । दक्षि॑णा यु॒क्ताः । यु॒क्ता य॒ज्ञ्म् । य॒ज्ञ्म् ॅवह॑न्ति । वह॑न्त्यृ॒त्विजः॑ । ऋ॒त्विज॒ इत्यृ॒त्विजः॑ ॥ तेने॒मम् । इ॒मम् ॅय॒ज्ञ्म् । य॒ज्ञ्म् नः॑ । नो॒ व॒ह॒ । व॒ह॒ सुवः॑ । सुव॑र् दे॒वेषु॑ । दे॒वेषु॒ गन्त॑वे । गन्त॑व॒ इति॒ गन्त॑वे ॥ येना᳚ग्ने । अ॒ग्ने॒ सु॒कृतः॑ । सु॒कृतः॑ प॒था । सु॒कृत॒ इति॑ सु - कृतः॑ । प॒था मधोः᳚ । मधो॒र् धाराः᳚ । धारा᳚ व्यान॒शुः । व्या॒न॒शुरिति॑ वि - आ॒न॒शुः ॥ तेने॒मम् । इ॒मम् ॅय॒ज्ञ्म् । य॒ज्ञ्म् नः॑ । नो॒ व॒ह॒ । व॒ह॒ सुवः॑ । सुव॑र् दे॒वेषु॑ । दे॒वेषु॒ गन्त॑वे ( ) । गन्त॑व॒ इति॒ गन्त॑वे ॥ यत्र॒ धाराः᳚ । धारा॒ अन॑पेताः । अन॑पेता॒ मधोः᳚ । अन॑पेता॒ इत्यन॑प - इ॒ताः॒ । मधो᳚र् घृ॒तस्य॑ । घृ॒तस्य॑ च । च॒ याः । या इति॒ याः ॥ तद॒ग्निः । अ॒ग्निर् वै᳚श्वकर्म॒णः । वै॒श्व॒क॒र्म॒णः सुवः॑ । वै॒श्व॒क॒र्म॒ण इति॑ वैश्व - क॒र्म॒णः । सुव॑र् दे॒वेषु॑ । दे॒वेषु॑ नः । नो॒ द॒ध॒त्॒ । द॒ध॒दिति॑ दधत् । \newline

\textbf{Jatai Paata} \newline

1. अ॒ग्निर् वै᳚श्वकर्म॒णो वै᳚श्वकर्म॒णो᳚ ऽग्नि र॒ग्निर् वै᳚श्वकर्म॒णः । \newline
2. वै॒श्व॒क॒र्म॒णः सुवः॒ सुव॑र् वैश्वकर्म॒णो वै᳚श्वकर्म॒णः सुवः॑ । \newline
3. वै॒श्व॒क॒र्म॒ण इति॑ वैश्व - क॒र्म॒णः । \newline
4. सुव॑र् दे॒वेषु॑ दे॒वेषु॒ सुवः॒ सुव॑र् दे॒वेषु॑ । \newline
5. दे॒वेषु॑ नो नो दे॒वेषु॑ दे॒वेषु॑ नः । \newline
6. नो॒ द॒ध॒द् द॒ध॒न् नो॒ नो॒ द॒ध॒त् । \newline
7. द॒ध॒दिति॑ दधत् । \newline
8. येना॑ स॒हस्रꣳ॑ स॒हस्रं॒ ॅयेन॒ येना॑ स॒हस्र᳚म् । \newline
9. स॒हस्रं॒ ॅवह॑सि॒ वह॑सि स॒हस्रꣳ॑ स॒हस्रं॒ ॅवह॑सि । \newline
10. वह॑सि॒ येन॒ येन॒ वह॑सि॒ वह॑सि॒ येन॑ । \newline
11. येना᳚ग्ने अग्ने॒ येन॒ येना᳚ग्ने । \newline
12. अ॒ग्ने॒ स॒र्व॒वे॒द॒सꣳ स॑र्ववेद॒स म॑ग्ने अग्ने सर्ववेद॒सम् । \newline
13. स॒र्व॒वे॒द॒समिति॑ सर्व - वे॒द॒सम् । \newline
14. तेने॒म मि॒मम् तेन॒ तेने॒मम् । \newline
15. इ॒मं ॅय॒ज्ञ्ं ॅय॒ज्ञ् मि॒म मि॒मं ॅय॒ज्ञ्म् । \newline
16. य॒ज्ञ्न्नो॑ नो य॒ज्ञ्ं ॅय॒ज्ञ्न्नः॑ । \newline
17. नो॒ व॒ह॒ व॒ह॒ नो॒ नो॒ व॒ह॒ । \newline
18. व॒ह॒ सुवः॒ सुव॑र् वह वह॒ सुवः॑ । \newline
19. सुव॑र् दे॒वेषु॑ दे॒वेषु॒ सुवः॒ सुव॑र् दे॒वेषु॑ । \newline
20. दे॒वेषु॒ गन्त॑वे॒ गन्त॑वे दे॒वेषु॑ दे॒वेषु॒ गन्त॑वे । \newline
21. गन्त॑व॒ इति॒ गन्त॑वे । \newline
22. येना᳚ग्ने ऽग्ने॒ येन॒ येना᳚ग्ने । \newline
23. अ॒ग्ने॒ दक्षि॑णा॒ दक्षि॑णा अग्ने ऽग्ने॒ दक्षि॑णाः । \newline
24. दक्षि॑णा यु॒क्ता यु॒क्ता दक्षि॑णा॒ दक्षि॑णा यु॒क्ताः । \newline
25. यु॒क्ता य॒ज्ञ्ं ॅय॒ज्ञ्ं ॅयु॒क्ता यु॒क्ता य॒ज्ञ्म् । \newline
26. य॒ज्ञ्ं ॅवह॑न्ति॒ वह॑न्ति य॒ज्ञ्ं ॅय॒ज्ञ्ं ॅवह॑न्ति । \newline
27. वह॑न् त्यृ॒त्विज॑ ऋ॒त्विजो॒ वह॑न्ति॒ वह॑न् त्यृ॒त्विजः॑ । \newline
28. ऋ॒त्विज॒ इत्यृ॒त्विजः॑ । \newline
29. तेने॒म मि॒मम् तेन॒ तेने॒मम् । \newline
30. इ॒मं ॅय॒ज्ञ्ं ॅय॒ज्ञ् मि॒म मि॒मं ॅय॒ज्ञ्म् । \newline
31. य॒ज्ञ्न्नो॑ नो य॒ज्ञ्ं ॅय॒ज्ञ्न्नः॑ । \newline
32. नो॒ व॒ह॒ व॒ह॒ नो॒ नो॒ व॒ह॒ । \newline
33. व॒ह॒ सुवः॒ सुव॑र् वह वह॒ सुवः॑ । \newline
34. सुव॑र् दे॒वेषु॑ दे॒वेषु॒ सुवः॒ सुव॑र् दे॒वेषु॑ । \newline
35. दे॒वेषु॒ गन्त॑वे॒ गन्त॑वे दे॒वेषु॑ दे॒वेषु॒ गन्त॑वे । \newline
36. गन्त॑व॒ इति॒ गन्त॑वे । \newline
37. येना᳚ग्ने ऽग्ने॒ येन॒ येना᳚ग्ने । \newline
38. अ॒ग्ने॒ सु॒कृतः॑ सु॒कृतो᳚ ऽग्ने ऽग्ने सु॒कृतः॑ । \newline
39. सु॒कृतः॑ प॒था प॒था सु॒कृतः॑ सु॒कृतः॑ प॒था । \newline
40. सु॒कृत॒ इति॑ सु - कृतः॑ । \newline
41. प॒था मधो॒र् मधोः᳚ प॒था प॒था मधोः᳚ । \newline
42. मधो॒र् धारा॒ धारा॒ मधो॒र् मधो॒र् धाराः᳚ । \newline
43. धारा᳚ व्यान॒शुर् व्या॑न॒शुर् धारा॒ धारा᳚ व्यान॒शुः । \newline
44. व्या॒न॒शुरिति॑ वि - आ॒न॒शुः । \newline
45. तेने॒म मि॒मम् तेन॒ तेने॒मम् । \newline
46. इ॒मं ॅय॒ज्ञ्ं ॅय॒ज्ञ् मि॒म मि॒मं ॅय॒ज्ञ्म् । \newline
47. य॒ज्ञ्न्नो॑ नो य॒ज्ञ्ं ॅय॒ज्ञ्न्नः॑ । \newline
48. नो॒ व॒ह॒ व॒ह॒ नो॒ नो॒ व॒ह॒ । \newline
49. व॒ह॒ सुवः॒ सुव॑र् वह वह॒ सुवः॑ । \newline
50. सुव॑र् दे॒वेषु॑ दे॒वेषु॒ सुवः॒ सुव॑र् दे॒वेषु॑ । \newline
51. दे॒वेषु॒ गन्त॑वे॒ गन्त॑वे दे॒वेषु॑ दे॒वेषु॒ गन्त॑वे । \newline
52. गन्त॑व॒ इति॒ गन्त॑वे । \newline
53. यत्र॒ धारा॒ धारा॒ यत्र॒ यत्र॒ धाराः᳚ । \newline
54. धारा॒ अन॑पेता॒ अन॑पेता॒ धारा॒ धारा॒ अन॑पेताः । \newline
55. अन॑पेता॒ मधो॒र् मधो॒ रन॑पेता॒ अन॑पेता॒ मधोः᳚ । \newline
56. अन॑पेता॒ इत्यन॑प - इ॒ताः॒ । \newline
57. मधो᳚र् घृ॒तस्य॑ घृ॒तस्य॒ मधो॒र् मधो᳚र् घृ॒तस्य॑ । \newline
58. घृ॒तस्य॑ च च घृ॒तस्य॑ घृ॒तस्य॑ च । \newline
59. च॒ या याश्च॑ च॒ याः । \newline
60. या इति॒ याः । \newline
61. तद॒ग्नि र॒ग्नि स्तत् तद॒ग्निः । \newline
62. अ॒ग्निर् वै᳚श्वकर्म॒णो वै᳚श्वकर्म॒णो᳚ ऽग्निर॒ग्निर् वै᳚श्वकर्म॒णः । \newline
63. वै॒श्व॒क॒र्म॒णः सुवः॒ सुव॑र् वैश्वकर्म॒णो वै᳚श्वकर्म॒णः सुवः॑ । \newline
64. वै॒श्व॒क॒र्म॒ण इति॑ वैश्व - क॒र्म॒णः । \newline
65. सुव॑र् दे॒वेषु॑ दे॒वेषु॒ सुवः॒ सुव॑र् दे॒वेषु॑ । \newline
66. दे॒वेषु॑ नो नो दे॒वेषु॑ दे॒वेषु॑ नः । \newline
67. नो॒ द॒ध॒द् द॒ध॒न् नो॒ नो॒ द॒ध॒त् । \newline
68. द॒ध॒दिति॑ दधत् । \newline

\textbf{Ghana Paata } \newline

1. अ॒ग्निर् वै᳚श्वकर्म॒णो वै᳚श्वकर्म॒णो᳚ ऽग्नि र॒ग्निर् वै᳚श्वकर्म॒णः सुवः॒ सुव॑र् वैश्वकर्म॒णो᳚ ऽग्नि र॒ग्निर् वै᳚श्वकर्म॒णः सुवः॑ । \newline
2. वै॒श्व॒क॒र्म॒णः सुवः॒ सुव॑र् वैश्वकर्म॒णो वै᳚श्वकर्म॒णः सुव॑र् दे॒वेषु॑ दे॒वेषु॒ सुव॑र् वैश्वकर्म॒णो वै᳚श्वकर्म॒णः सुव॑र् दे॒वेषु॑ । \newline
3. वै॒श्व॒क॒र्म॒ण इति॑ वैश्व - क॒र्म॒णः । \newline
4. सुव॑र् दे॒वेषु॑ दे॒वेषु॒ सुवः॒ सुव॑र् दे॒वेषु॑ नो नो दे॒वेषु॒ सुवः॒ सुव॑र् दे॒वेषु॑ नः । \newline
5. दे॒वेषु॑ नो नो दे॒वेषु॑ दे॒वेषु॑ नो दधद् दधन् नो दे॒वेषु॑ दे॒वेषु॑ नो दधत् । \newline
6. नो॒ द॒ध॒द् द॒ध॒न् नो॒ नो॒ द॒ध॒त् । \newline
7. द॒ध॒दिति॑ दधत् । \newline
8. येना॑ स॒हस्रꣳ॑ स॒हस्रं॒ ॅयेन॒ येना॑ स॒हस्रं॒ ॅवह॑सि॒ वह॑सि स॒हस्रं॒ ॅयेन॒ येना॑ स॒हस्रं॒ ॅवह॑सि । \newline
9. स॒हस्रं॒ ॅवह॑सि॒ वह॑सि स॒हस्रꣳ॑ स॒हस्रं॒ ॅवह॑सि॒ येन॒ येन॒ वह॑सि स॒हस्रꣳ॑ स॒हस्रं॒ ॅवह॑सि॒ येन॑ । \newline
10. वह॑सि॒ येन॒ येन॒ वह॑सि॒ वह॑सि॒ येना᳚ग्ने अग्ने॒ येन॒ वह॑सि॒ वह॑सि॒ येना᳚ग्ने । \newline
11. येना᳚ग्ने अग्ने॒ येन॒ येना᳚ग्ने सर्ववेद॒सꣳ स॑र्ववेद॒स म॑ग्ने॒ येन॒ येना᳚ग्ने सर्ववेद॒सम् । \newline
12. अ॒ग्ने॒ स॒र्व॒वे॒द॒सꣳ स॑र्ववेद॒स म॑ग्ने अग्ने सर्ववेद॒सम् । \newline
13. स॒र्व॒वे॒द॒समिति॑ सर्व - वे॒द॒सम् । \newline
14. तेने॒म मि॒मम् तेन॒ तेने॒मं ॅय॒ज्ञ्ं ॅय॒ज्ञ् मि॒मम् तेन॒ तेने॒मं ॅय॒ज्ञ्म् । \newline
15. इ॒मं ॅय॒ज्ञ्ं ॅय॒ज्ञ् मि॒म मि॒मं ॅय॒ज्ञ्न् नो॑ नो य॒ज्ञ् मि॒म मि॒मं ॅय॒ज्ञ्न् नः॑ । \newline
16. य॒ज्ञ्न् नो॑ नो य॒ज्ञ्ं ॅय॒ज्ञ्न् नो॑ वह वह नो य॒ज्ञ्ं ॅय॒ज्ञ्न् नो॑ वह । \newline
17. नो॒ व॒ह॒ व॒ह॒ नो॒ नो॒ व॒ह॒ सुवः॒ सुव॑र् वह नो नो वह॒ सुवः॑ । \newline
18. व॒ह॒ सुवः॒ सुव॑र् वह वह॒ सुव॑र् दे॒वेषु॑ दे॒वेषु॒ सुव॑र् वह वह॒ सुव॑र् दे॒वेषु॑ । \newline
19. सुव॑र् दे॒वेषु॑ दे॒वेषु॒ सुवः॒ सुव॑र् दे॒वेषु॒ गन्त॑वे॒ गन्त॑वे दे॒वेषु॒ सुवः॒ सुव॑र् दे॒वेषु॒ गन्त॑वे । \newline
20. दे॒वेषु॒ गन्त॑वे॒ गन्त॑वे दे॒वेषु॑ दे॒वेषु॒ गन्त॑वे । \newline
21. गन्त॑व॒ इति॒ गन्त॑वे । \newline
22. येना᳚ग्ने ऽग्ने॒ येन॒ येना᳚ग्ने॒ दक्षि॑णा॒ दक्षि॑णा अग्ने॒ येन॒ येना᳚ग्ने॒ दक्षि॑णाः । \newline
23. अ॒ग्ने॒ दक्षि॑णा॒ दक्षि॑णा अग्ने ऽग्ने॒ दक्षि॑णा यु॒क्ता यु॒क्ता दक्षि॑णा अग्ने ऽग्ने॒ दक्षि॑णा यु॒क्ताः । \newline
24. दक्षि॑णा यु॒क्ता यु॒क्ता दक्षि॑णा॒ दक्षि॑णा यु॒क्ता य॒ज्ञ्ं ॅय॒ज्ञ्ं ॅयु॒क्ता दक्षि॑णा॒ दक्षि॑णा यु॒क्ता य॒ज्ञ्म् । \newline
25. यु॒क्ता य॒ज्ञ्ं ॅय॒ज्ञ्ं ॅयु॒क्ता यु॒क्ता य॒ज्ञ्ं ॅवह॑न्ति॒ वह॑न्ति य॒ज्ञ्ं ॅयु॒क्ता यु॒क्ता य॒ज्ञ्ं ॅवह॑न्ति । \newline
26. य॒ज्ञ्ं ॅवह॑न्ति॒ वह॑न्ति य॒ज्ञ्ं ॅय॒ज्ञ्ं ॅवह॑न् त्यृ॒त्विज॑ ऋ॒त्विजो॒ वह॑न्ति य॒ज्ञ्ं ॅय॒ज्ञ्ं ॅवह॑न् त्यृ॒त्विजः॑ । \newline
27. वह॑न् त्यृ॒त्विज॑ ऋ॒त्विजो॒ वह॑न्ति॒ वह॑न् त्यृ॒त्विजः॑ । \newline
28. ऋ॒त्विज॒ इत्यृ॒त्विजः॑ । \newline
29. तेने॒म मि॒मम् तेन॒ तेने॒मं ॅय॒ज्ञ्ं ॅय॒ज्ञ् मि॒मम् तेन॒ तेने॒मं ॅय॒ज्ञ्म् । \newline
30. इ॒मं ॅय॒ज्ञ्ं ॅय॒ज्ञ् मि॒म मि॒मं ॅय॒ज्ञ्म् नो॑ नो य॒ज्ञ् मि॒म मि॒मं ॅय॒ज्ञ्म् नः॑ । \newline
31. य॒ज्ञ्म् नो॑ नो य॒ज्ञ्ं ॅय॒ज्ञ्म् नो॑ वह वह नो य॒ज्ञ्ं ॅय॒ज्ञ्म् नो॑ वह । \newline
32. नो॒ व॒ह॒ व॒ह॒ नो॒ नो॒ व॒ह॒ सुवः॒ सुव॑र् वह नो नो वह॒ सुवः॑ । \newline
33. व॒ह॒ सुवः॒ सुव॑र् वह वह॒ सुव॑र् दे॒वेषु॑ दे॒वेषु॒ सुव॑र् वह वह॒ सुव॑र् दे॒वेषु॑ । \newline
34. सुव॑र् दे॒वेषु॑ दे॒वेषु॒ सुवः॒ सुव॑र् दे॒वेषु॒ गन्त॑वे॒ गन्त॑वे दे॒वेषु॒ सुवः॒ सुव॑र् दे॒वेषु॒ गन्त॑वे । \newline
35. दे॒वेषु॒ गन्त॑वे॒ गन्त॑वे दे॒वेषु॑ दे॒वेषु॒ गन्त॑वे । \newline
36. गन्त॑व॒ इति॒ गन्त॑वे । \newline
37. येना᳚ग्ने ऽग्ने॒ येन॒ येना᳚ग्ने सु॒कृतः॑ सु॒कृतो᳚ ऽग्ने॒ येन॒ येना᳚ग्ने सु॒कृतः॑ । \newline
38. अ॒ग्ने॒ सु॒कृतः॑ सु॒कृतो᳚ ऽग्ने ऽग्ने सु॒कृतः॑ प॒था प॒था सु॒कृतो᳚ ऽग्ने ऽग्ने सु॒कृतः॑ प॒था । \newline
39. सु॒कृतः॑ प॒था प॒था सु॒कृतः॑ सु॒कृतः॑ प॒था मधो॒र् मधोः᳚ प॒था सु॒कृतः॑ सु॒कृतः॑ प॒था मधोः᳚ । \newline
40. सु॒कृत॒ इति॑ सु - कृतः॑ । \newline
41. प॒था मधो॒र् मधोः᳚ प॒था प॒था मधो॒र् धारा॒ धारा॒ मधोः᳚ प॒था प॒था मधो॒र् धाराः᳚ । \newline
42. मधो॒र् धारा॒ धारा॒ मधो॒र् मधो॒र् धारा᳚ व्यान॒शुर् व्या॑न॒शुर् धारा॒ मधो॒र् मधो॒र् धारा᳚ व्यान॒शुः । \newline
43. धारा᳚ व्यान॒शुर् व्या॑न॒शुर् धारा॒ धारा᳚ व्यान॒शुः । \newline
44. व्या॒न॒शुरिति॑ वि - आ॒न॒शुः । \newline
45. तेने॒म मि॒मम् तेन॒ तेने॒मं ॅय॒ज्ञ्ं ॅय॒ज्ञ् मि॒मम् तेन॒ तेने॒मं ॅय॒ज्ञ्म् । \newline
46. इ॒मं ॅय॒ज्ञ्ं ॅय॒ज्ञ् मि॒म मि॒मं ॅय॒ज्ञ्म् नो॑ नो य॒ज्ञ् मि॒म मि॒मं ॅय॒ज्ञ्म् नः॑ । \newline
47. य॒ज्ञ्म् नो॑ नो य॒ज्ञ्ं ॅय॒ज्ञ्म् नो॑ वह वह नो य॒ज्ञ्ं ॅय॒ज्ञ्म् नो॑ वह । \newline
48. नो॒ व॒ह॒ व॒ह॒ नो॒ नो॒ व॒ह॒ सुवः॒ सुव॑र् वह नो नो वह॒ सुवः॑ । \newline
49. व॒ह॒ सुवः॒ सुव॑र् वह वह॒ सुव॑र् दे॒वेषु॑ दे॒वेषु॒ सुव॑र् वह वह॒ सुव॑र् दे॒वेषु॑ । \newline
50. सुव॑र् दे॒वेषु॑ दे॒वेषु॒ सुवः॒ सुव॑र् दे॒वेषु॒ गन्त॑वे॒ गन्त॑वे दे॒वेषु॒ सुवः॒ सुव॑र् दे॒वेषु॒ गन्त॑वे । \newline
51. दे॒वेषु॒ गन्त॑वे॒ गन्त॑वे दे॒वेषु॑ दे॒वेषु॒ गन्त॑वे । \newline
52. गन्त॑व॒ इति॒ गन्त॑वे । \newline
53. यत्र॒ धारा॒ धारा॒ यत्र॒ यत्र॒ धारा॒ अन॑पेता॒ अन॑पेता॒ धारा॒ यत्र॒ यत्र॒ धारा॒ अन॑पेताः । \newline
54. धारा॒ अन॑पेता॒ अन॑पेता॒ धारा॒ धारा॒ अन॑पेता॒ मधो॒र् मधो॒ रन॑पेता॒ धारा॒ धारा॒ अन॑पेता॒ मधोः᳚ । \newline
55. अन॑पेता॒ मधो॒र् मधो॒ रन॑पेता॒ अन॑पेता॒ मधो᳚र् घृ॒तस्य॑ घृ॒तस्य॒ मधो॒ रन॑पेता॒ अन॑पेता॒ मधो᳚र् घृ॒तस्य॑ । \newline
56. अन॑पेता॒ इत्यन॑प - इ॒ताः॒ । \newline
57. मधो᳚र् घृ॒तस्य॑ घृ॒तस्य॒ मधो॒र् मधो᳚र् घृ॒तस्य॑ च च घृ॒तस्य॒ मधो॒र् मधो᳚र् घृ॒तस्य॑ च । \newline
58. घृ॒तस्य॑ च च घृ॒तस्य॑ घृ॒तस्य॑ च॒ या याश्च॑ घृ॒तस्य॑ घृ॒तस्य॑ च॒ याः । \newline
59. च॒ या याश्च॑ च॒ याः । \newline
60. या इति॒ याः । \newline
61. तद॒ग्नि र॒ग्नि स्तत् तद॒ग्निर् वै᳚श्वकर्म॒णो वै᳚श्वकर्म॒णो᳚ ऽग्नि स्तत् तद॒ग्निर् वै᳚श्वकर्म॒णः । \newline
62. अ॒ग्निर् वै᳚श्वकर्म॒णो वै᳚श्वकर्म॒णो᳚ ऽग्नि र॒ग्निर् वै᳚श्वकर्म॒णः सुवः॒ सुव॑र् वैश्वकर्म॒णो᳚ ऽग्नि र॒ग्निर् वै᳚श्वकर्म॒णः सुवः॑ । \newline
63. वै॒श्व॒क॒र्म॒णः सुवः॒ सुव॑र् वैश्वकर्म॒णो वै᳚श्वकर्म॒णः सुव॑र् दे॒वेषु॑ दे॒वेषु॒ सुव॑र् वैश्वकर्म॒णो वै᳚श्वकर्म॒णः सुव॑र् दे॒वेषु॑ । \newline
64. वै॒श्व॒क॒र्म॒ण इति॑ वैश्व - क॒र्म॒णः । \newline
65. सुव॑र् दे॒वेषु॑ दे॒वेषु॒ सुवः॒ सुव॑र् दे॒वेषु॑ नो नो दे॒वेषु॒ सुवः॒ सुव॑र् दे॒वेषु॑ नः । \newline
66. दे॒वेषु॑ नो नो दे॒वेषु॑ दे॒वेषु॑ नो दधद् दधन् नो दे॒वेषु॑ दे॒वेषु॑ नो दधत् । \newline
67. नो॒ द॒ध॒द् द॒ध॒न् नो॒ नो॒ द॒ध॒त् । \newline
68. द॒ध॒दिति॑ दधत् । \newline
\pagebreak
\markright{ TS 5.7.8.1  \hfill https://www.vedavms.in \hfill}

\section{ TS 5.7.8.1 }

\textbf{TS 5.7.8.1 } \newline
\textbf{Samhita Paata} \newline

यास्ते॑ अग्ने स॒मिधो॒ यानि॒ धाम॒ या जि॒ह्वा जा॑तवेदो॒ यो अ॒र्चिः । ये ते॑ अग्ने मे॒डयो॒ य इन्द॑व॒स्तेभि॑रा॒त्मानं॑ चिनुहि प्रजा॒नन्न् ॥उ॒थ्स॒न्न॒य॒ज्ञो वा ए॒ष यद॒ग्निः किं ॅवाऽहै॒तस्य॑ क्रि॒यते॒ किं ॅवा॒ न यद्वा अ॑द्ध्व॒र्यु-र॒ग्नेश्चि॒न्वन्न॑-न्त॒रेत्या॒त्मनो॒ वै तद॒न्तरे॑ति॒ यास्ते॑ अग्ने स॒मिधो॒ यानि॒ - [  ] \newline

\textbf{Pada Paata} \newline

याः । ते॒ । अ॒ग्ने॒ । स॒मिध॒ इति॑ सं - इधः॑ । यानि॑ । धाम॑ । या । जि॒ह्वा । जा॒त॒वे॒द॒ इति॑ जात-वे॒दः॒ । यः । अ॒र्चिः ॥ ये । ते॒ । अ॒ग्ने॒ । मे॒डयः॑ । ये । इन्द॑वः । तेभिः॑ । आ॒त्मान᳚म् । चि॒नु॒हि॒ । प्र॒जा॒नन्निति॑ प्र - जा॒नन्न् ॥ उ॒थ्स॒न्न॒य॒ज्ञ् इत्यु॑थ्सन्न - य॒ज्ञ्ः । वै । ए॒षः । यत् । अ॒ग्निः । किम् । वा॒ । अह॑ । ए॒तस्य॑ । क्रि॒यते᳚ । किम् । वा॒ । न । यत् । वै । अ॒द्ध्व॒र्युः । अ॒ग्नेः । चि॒न्वन्न् । अ॒न्त॒रेतीय॑न्तः - एति॑ । आ॒त्मनः॑ । वै । तत् । अ॒न्तः । ए॒ति॒ । याः । ते॒ । अ॒ग्ने॒ । स॒मिध॒ इति॑ सं - इधः॑ । यानि॑ ।  \newline


\textbf{Krama Paata} \newline

यास्ते᳚ । ते॒ अ॒ग्ने॒ । अ॒ग्ने॒ स॒मिधः॑ । स॒मिधो॒ यानि॑ । स॒मिध॒ इति॑ सम् - इधः॑ । यानि॒ धाम॑ । धाम॒ या । या जि॒ह्वा । जि॒ह्वा जा॑तवेदः । जा॒त॒वे॒दो॒ यः । जा॒त॒वे॒द॒ इति॑ जात - वे॒दः॒ । यो अ॒र्चिः । अ॒र्चिरित्य॒र्चिः ॥ ये ते᳚ । ते॒ अ॒ग्ने॒ । अ॒ग्ने॒ मे॒डयः॑ । मे॒डयो॒ ये । य इन्द॑वः । इन्द॑व॒स्तेभिः॑ । तेभि॑रा॒त्मान᳚म् । आ॒त्मान॑म् चिनुहि । चि॒नु॒हि॒ प्र॒जा॒नन्न् । प्र॒जा॒नन्निति॑ प्र - जा॒नन्न् ॥ उ॒थ्स॒न्न॒य॒ज्ञो वै । उ॒थ्स॒न्न॒य॒ज्ञ् इत्यु॑थ्सन्न - य॒ज्ञ्ः । वा ए॒षः । ए॒ष यत् । यद॒ग्निः । अ॒ग्निः किम् । किम् ॅवा᳚ । वाऽह॑ । अहै॒तस्य॑ । ए॒तस्य॑ क्रि॒यते᳚ । क्रि॒यते॒ किम् । किम् ॅवा᳚ । वा॒ न । न यत् । यद् वै । वा अ॑द्ध्व॒र्युः । अ॒द्ध्व॒र्युर॒ग्नेः । अ॒ग्नेश्चि॒न्वन्न् । चि॒न्वन्न॑न्त॒रेति॑ । अ॒न्त॒रेत्या॒त्मनः॑ । अ॒न्त॒रेतीत्य॑न्तः - एति॑ । आ॒त्मनो॒ वै । वै तत् । तद॒न्तः । अ॒न्तरे॑ति । ए॒ति॒ याः । यास्ते᳚ । ते॒ अ॒ग्ने॒ । अ॒ग्ने॒ स॒मिधः॑ । स॒मिधो॒ यानि॑ । स॒मिध॒ इति॑ सम् - इधः॑ । यानि॒ धाम॑ \newline

\textbf{Jatai Paata} \newline

1. या स्ते॑ ते॒ या या स्ते᳚ । \newline
2. ते॒ अ॒ग्ने॒ ऽग्ने॒ ते॒ ते॒ अ॒ग्ने॒ । \newline
3. अ॒ग्ने॒ स॒मिधः॑ स॒मिधो᳚ ऽग्ने ऽग्ने स॒मिधः॑ । \newline
4. स॒मिधो॒ यानि॒ यानि॑ स॒मिधः॑ स॒मिधो॒ यानि॑ । \newline
5. स॒मिध॒ इति॑ सं - इधः॑ । \newline
6. यानि॒ धाम॒ धाम॒ यानि॒ यानि॒ धाम॑ । \newline
7. धाम॒ या या धाम॒ धाम॒ या । \newline
8. या जि॒ह्वा जि॒ह्वा या या जि॒ह्वा । \newline
9. जि॒ह्वा जा॑तवेदो जातवेदो जि॒ह्वा जि॒ह्वा जा॑तवेदः । \newline
10. जा॒त॒वे॒दो॒ यो यो जा॑तवेदो जातवेदो॒ यः । \newline
11. जा॒त॒वे॒द॒ इति॑ जात - वे॒दः॒ । \newline
12. यो अ॒र्चि र॒र्चिर् यो यो अ॒र्चिः । \newline
13. अ॒र्चिरित्य॒र्चिः । \newline
14. ये ते॑ ते॒ ये ये ते᳚ । \newline
15. ते॒ अ॒ग्ने॒ ऽग्ने॒ ते॒ ते॒ अ॒ग्ने॒ । \newline
16. अ॒ग्ने॒ मे॒डयो॑ मे॒डयो᳚ ऽग्ने ऽग्ने मे॒डयः॑ । \newline
17. मे॒डयो॒ ये ये मे॒डयो॑ मे॒डयो॒ ये । \newline
18. य इन्द॑व॒ इन्द॑वो॒ ये य इन्द॑वः । \newline
19. इन्द॑व॒ स्तेभि॒ स्तेभि॒ रिन्द॑व॒ इन्द॑व॒ स्तेभिः॑ । \newline
20. तेभि॑ रा॒त्मान॑ मा॒त्मान॒म् तेभि॒ स्तेभि॑ रा॒त्मान᳚म् । \newline
21. आ॒त्मान॑म् चिनुहि चिनु ह्या॒त्मान॑ मा॒त्मान॑म् चिनुहि । \newline
22. चि॒नु॒हि॒ प्र॒जा॒नन् प्र॑जा॒नꣳ श्चि॑नुहि चिनुहि प्रजा॒नन्न् । \newline
23. प्र॒जा॒नन्निति॑ प्र - जा॒नन्न् । \newline
24. उ॒थ्स॒न्न॒य॒ज्ञो वै वा उ॑थ्सन्नय॒ज्ञ् उ॑थ्सन्नय॒ज्ञो वै । \newline
25. उ॒थ्स॒न्न॒य॒ज्ञ् इत्यु॑थ्सन्न - य॒ज्ञ्ः । \newline
26. वा ए॒ष ए॒ष वै वा ए॒षः । \newline
27. ए॒ष यद् यदे॒ष ए॒ष यत् । \newline
28. यद॒ग्नि र॒ग्निर् यद् यद॒ग्निः । \newline
29. अ॒ग्निः किम् कि म॒ग्नि र॒ग्निः किम् । \newline
30. किं ॅवा॑ वा॒ किम् किं ॅवा᳚ । \newline
31. वा ऽहाह॑ वा॒ वा ऽह॑ । \newline
32. अहै॒तस्यै॒ तस्याहा है॒तस्य॑ । \newline
33. ए॒तस्य॑ क्रि॒यते᳚ क्रि॒यत॑ ए॒त स्यै॒तस्य॑ क्रि॒यते᳚ । \newline
34. क्रि॒यते॒ किम् किम् क्रि॒यते᳚ क्रि॒यते॒ किम् । \newline
35. किं ॅवा॑ वा॒ किम् किं ॅवा᳚ । \newline
36. वा॒ न न वा॑ वा॒ न । \newline
37. न यद् यन् न न यत् । \newline
38. यद् वै वै यद् यद् वै । \newline
39. वा अ॑द्ध्व॒र्यु र॑द्ध्व॒र्युर् वै वा अ॑द्ध्व॒र्युः । \newline
40. अ॒द्ध्व॒र्यु र॒ग्ने र॒ग्ने र॑द्ध्व॒र्यु र॑द्ध्व॒र्यु र॒ग्नेः । \newline
41. अ॒ग्ने श्चि॒न्वꣳ श्चि॒न्वन् न॒ग्ने र॒ग्ने श्चि॒न्वन्न् । \newline
42. चि॒न्वन् न॑न्त॒रे त्य॑न्त॒रेति॑ चि॒न्वꣳ श्चि॒न्वन् न॑न्त॒रेति॑ । \newline
43. अ॒न्त॒रे त्या॒त्मन॑ आ॒त्मनो᳚ ऽन्त॒रे त्य॑न्त॒रे त्या॒त्मनः॑ । \newline
44. अ॒न्त॒रेतीत्य॑न्तः - एति॑ । \newline
45. आ॒त्मनो॒ वै वा आ॒त्मन॑ आ॒त्मनो॒ वै । \newline
46. वै तत् तद् वै वै तत् । \newline
47. तद॒न्त र॒न्त स्तत् तद॒न्तः । \newline
48. अ॒न्त रे᳚त्ये त्य॒न्त र॒न्त रे॑ति । \newline
49. ए॒ति॒ या या ए᳚त्येति॒ याः । \newline
50. या स्ते॑ ते॒ या या स्ते᳚ । \newline
51. ते॒ अ॒ग्ने॒ ऽग्ने॒ ते॒ ते॒ अ॒ग्ने॒ । \newline
52. अ॒ग्ने॒ स॒मिधः॑ स॒मिधो᳚ ऽग्ने ऽग्ने स॒मिधः॑ । \newline
53. स॒मिधो॒ यानि॒ यानि॑ स॒मिधः॑ स॒मिधो॒ यानि॑ । \newline
54. स॒मिध॒ इति॑ सं - इधः॑ । \newline
55. यानि॒ धाम॒ धाम॒ यानि॒ यानि॒ धाम॑ । \newline

\textbf{Ghana Paata } \newline

1. या स्ते॑ ते॒ या या स्ते॑ अग्ने ऽग्ने ते॒ या या स्ते॑ अग्ने । \newline
2. ते॒ अ॒ग्ने॒ ऽग्ने॒ ते॒ ते॒ अ॒ग्ने॒ स॒मिधः॑ स॒मिधो᳚ ऽग्ने ते ते अग्ने स॒मिधः॑ । \newline
3. अ॒ग्ने॒ स॒मिधः॑ स॒मिधो᳚ ऽग्ने ऽग्ने स॒मिधो॒ यानि॒ यानि॑ स॒मिधो᳚ ऽग्ने ऽग्ने स॒मिधो॒ यानि॑ । \newline
4. स॒मिधो॒ यानि॒ यानि॑ स॒मिधः॑ स॒मिधो॒ यानि॒ धाम॒ धाम॒ यानि॑ स॒मिधः॑ स॒मिधो॒ यानि॒ धाम॑ । \newline
5. स॒मिध॒ इति॑ सं - इधः॑ । \newline
6. यानि॒ धाम॒ धाम॒ यानि॒ यानि॒ धाम॒ या या धाम॒ यानि॒ यानि॒ धाम॒ या । \newline
7. धाम॒ या या धाम॒ धाम॒ या जि॒ह्वा जि॒ह्वा या धाम॒ धाम॒ या जि॒ह्वा । \newline
8. या जि॒ह्वा जि॒ह्वा या या जि॒ह्वा जा॑तवेदो जातवेदो जि॒ह्वा या या जि॒ह्वा जा॑तवेदः । \newline
9. जि॒ह्वा जा॑तवेदो जातवेदो जि॒ह्वा जि॒ह्वा जा॑तवेदो॒ यो यो जा॑तवेदो जि॒ह्वा जि॒ह्वा जा॑तवेदो॒ यः । \newline
10. जा॒त॒वे॒दो॒ यो यो जा॑तवेदो जातवेदो॒ यो अ॒र्चि र॒र्चिर् यो जा॑तवेदो जातवेदो॒ यो अ॒र्चिः । \newline
11. जा॒त॒वे॒द॒ इति॑ जात - वे॒दः॒ । \newline
12. यो अ॒र्चि र॒र्चिर् यो यो अ॒र्चिः । \newline
13. अ॒र्चिरित्य॒र्चिः । \newline
14. ये ते॑ ते॒ ये ये ते॑ अग्ने ऽग्ने ते॒ ये ये ते॑ अग्ने । \newline
15. ते॒ अ॒ग्ने॒ ऽग्ने॒ ते॒ ते॒ अ॒ग्ने॒ मे॒डयो॑ मे॒डयो᳚ ऽग्ने ते ते अग्ने मे॒डयः॑ । \newline
16. अ॒ग्ने॒ मे॒डयो॑ मे॒डयो᳚ ऽग्ने ऽग्ने मे॒डयो॒ ये ये मे॒डयो᳚ ऽग्ने ऽग्ने मे॒डयो॒ ये । \newline
17. मे॒डयो॒ ये ये मे॒डयो॑ मे॒डयो॒ य इन्द॑व॒ इन्द॑वो॒ ये मे॒डयो॑ मे॒डयो॒ य इन्द॑वः । \newline
18. य इन्द॑व॒ इन्द॑वो॒ ये य इन्द॑व॒ स्तेभि॒ स्तेभि॒ रिन्द॑वो॒ ये य इन्द॑व॒ स्तेभिः॑ । \newline
19. इन्द॑व॒ स्तेभि॒ स्तेभि॒ रिन्द॑व॒ इन्द॑व॒ स्तेभि॑ रा॒त्मान॑ मा॒त्मान॒म् तेभि॒ रिन्द॑व॒ इन्द॑व॒ स्तेभि॑ रा॒त्मान᳚म् । \newline
20. तेभि॑ रा॒त्मान॑ मा॒त्मान॒म् तेभि॒ स्तेभि॑ रा॒त्मान॑म् चिनुहि चिनु ह्या॒त्मान॒म् तेभि॒ स्तेभि॑ रा॒त्मान॑म् चिनुहि । \newline
21. आ॒त्मान॑म् चिनुहि चिनु ह्या॒त्मान॑ मा॒त्मान॑म् चिनुहि प्रजा॒नन् प्र॑जा॒नꣳ श्चि॑नु ह्या॒त्मान॑ मा॒त्मान॑म् चिनुहि प्रजा॒नन्न् । \newline
22. चि॒नु॒हि॒ प्र॒जा॒नन् प्र॑जा॒नꣳ श्चि॑नुहि चिनुहि प्रजा॒नन्न् । \newline
23. प्र॒जा॒नन्निति॑ प्र - जा॒नन्न् । \newline
24. उ॒थ्स॒न्न॒य॒ज्ञो वै वा उ॑थ्सन्नय॒ज्ञ् उ॑थ्सन्नय॒ज्ञो वा ए॒ष ए॒ष वा उ॑थ्सन्नय॒ज्ञ् उ॑थ्सन्नय॒ज्ञो वा ए॒षः । \newline
25. उ॒थ्स॒न्न॒य॒ज्ञ् इत्यु॑थ्सन्न - य॒ज्ञ्ः । \newline
26. वा ए॒ष ए॒ष वै वा ए॒ष यद् यदे॒ष वै वा ए॒ष यत् । \newline
27. ए॒ष यद् यदे॒ष ए॒ष यद॒ग्नि र॒ग्निर् यदे॒ष ए॒ष यद॒ग्निः । \newline
28. यद॒ग्नि र॒ग्निर् यद् यद॒ग्निः किम् कि म॒ग्निर् यद् यद॒ग्निः किम् । \newline
29. अ॒ग्निः किम् कि म॒ग्नि र॒ग्निः किं ॅवा॑ वा॒ कि म॒ग्नि र॒ग्निः किं ॅवा᳚ । \newline
30. किं ॅवा॑ वा॒ किम् किं ॅवा ऽहाह॑ वा॒ किम् किं ॅवा ऽह॑ । \newline
31. वा ऽहाह॑ वा॒ वा ऽहै॒त स्यै॒तस्याह॑ वा॒ वा ऽहै॒ तस्य॑ । \newline
32. अहै॒ तस्यै॒तस्या हा है॒तस्य॑ क्रि॒यते᳚ क्रि॒यत॑ ए॒तस्या हाहै॒तस्य॑ क्रि॒यते᳚ । \newline
33. ए॒तस्य॑ क्रि॒यते᳚ क्रि॒यत॑ ए॒तस्यै॒तस्य॑ क्रि॒यते॒ किम् किम् क्रि॒यत॑ ए॒तस्यै॒तस्य॑ क्रि॒यते॒ किम् । \newline
34. क्रि॒यते॒ किम् किम् क्रि॒यते᳚ क्रि॒यते॒ किं ॅवा॑ वा॒ किम् क्रि॒यते᳚ क्रि॒यते॒ किं ॅवा᳚ । \newline
35. किं ॅवा॑ वा॒ किम् किं ॅवा॒ न न वा॒ किम् किं ॅवा॒ न । \newline
36. वा॒ न न वा॑ वा॒ न यद् यन् न वा॑ वा॒ न यत् । \newline
37. न यद् यन् न न यद् वै वै यन् न न यद् वै । \newline
38. यद् वै वै यद् यद् वा अ॑द्ध्व॒र्यु र॑द्ध्व॒र्युर् वै यद् यद् वा अ॑द्ध्व॒र्युः । \newline
39. वा अ॑द्ध्व॒र्यु र॑द्ध्व॒र्युर् वै वा अ॑द्ध्व॒र्यु र॒ग्ने र॒ग्ने र॑द्ध्व॒र्युर् वै वा अ॑द्ध्व॒र्यु र॒ग्नेः । \newline
40. अ॒द्ध्व॒र्यु र॒ग्ने र॒ग्ने र॑द्ध्व॒र्यु र॑द्ध्व॒र्यु र॒ग्ने श्चि॒न्वꣳ श्चि॒न्वन् न॒ग्ने र॑द्ध्व॒र्यु र॑द्ध्व॒र्यु र॒ग्ने श्चि॒न्वन्न् । \newline
41. अ॒ग्ने श्चि॒न्वꣳ श्चि॒न्वन् न॒ग्ने र॒ग्ने श्चि॒न्वन् न॑न्त॒रे त्य॑न्त॒रेति॑ चि॒न्वन् न॒ग्ने र॒ग्ने श्चि॒न्वन् न॑न्त॒रेति॑ । \newline
42. चि॒न्वन् न॑न्त॒रे त्य॑न्त॒रेति॑ चि॒न्वꣳ श्चि॒न्वन् न॑न्त॒रे त्या॒त्मन॑ आ॒त्मनो᳚ ऽन्त॒रेति॑ चि॒न्वꣳ श्चि॒न्वन् न॑न्त॒रे त्या॒त्मनः॑ । \newline
43. अ॒न्त॒रे त्या॒त्मन॑ आ॒त्मनो᳚ ऽन्त॒रे त्य॑न्त॒रे त्या॒त्मनो॒ वै वा आ॒त्मनो᳚ ऽन्त॒रे त्य॑न्त॒रे त्या॒त्मनो॒ वै । \newline
44. अ॒न्त॒रेतीत्य॑न्तः - एति॑ । \newline
45. आ॒त्मनो॒ वै वा आ॒त्मन॑ आ॒त्मनो॒ वै तत् तद् वा आ॒त्मन॑ आ॒त्मनो॒ वै तत् । \newline
46. वै तत् तद् वै वै तद॒न्त र॒न्त स्तद् वै वै तद॒न्तः । \newline
47. तद॒न्त र॒न्त स्तत् तद॒न्त रे᳚त्ये त्य॒न्त स्तत् तद॒न्त रे॑ति । \newline
48. अ॒न्त रे᳚त्ये त्य॒न्त र॒न्त रे॑ति॒ या या ए᳚त्य॒न्त र॒न्त रे॑ति॒ याः । \newline
49. ए॒ति॒ या या ए᳚त्येति॒ या स्ते॑ ते॒ या ए᳚त्येति॒ या स्ते᳚ । \newline
50. या स्ते॑ ते॒ या या स्ते॑ अग्ने ऽग्ने ते॒ या या स्ते॑ अग्ने । \newline
51. ते॒ अ॒ग्ने॒ ऽग्ने॒ ते॒ ते॒ अ॒ग्ने॒ स॒मिधः॑ स॒मिधो᳚ ऽग्ने ते ते अग्ने स॒मिधः॑ । \newline
52. अ॒ग्ने॒ स॒मिधः॑ स॒मिधो᳚ ऽग्ने ऽग्ने स॒मिधो॒ यानि॒ यानि॑ स॒मिधो᳚ ऽग्ने ऽग्ने स॒मिधो॒ यानि॑ । \newline
53. स॒मिधो॒ यानि॒ यानि॑ स॒मिधः॑ स॒मिधो॒ यानि॒ धाम॒ धाम॒ यानि॑ स॒मिधः॑ स॒मिधो॒ यानि॒ धाम॑ । \newline
54. स॒मिध॒ इति॑ सं - इधः॑ । \newline
55. यानि॒ धाम॒ धाम॒ यानि॒ यानि॒ धामे तीति॒ धाम॒ यानि॒ यानि॒ धामेति॑ । \newline
\pagebreak
\markright{ TS 5.7.8.2  \hfill https://www.vedavms.in \hfill}

\section{ TS 5.7.8.2 }

\textbf{TS 5.7.8.2 } \newline
\textbf{Samhita Paata} \newline

धामेत्या॑है॒षा वा अ॒ग्नेः स्व॑यं चि॒तिर॒ग्निरे॒व तद॒ग्निं चि॑नोति॒ नाद्ध्व॒र्युरा॒त्मनो॒ऽन्तरे॑ति॒ चत॑स्र॒ आशाः॒ प्रच॑रन्त्व॒ग्नय॑ इ॒मं नो॑ य॒ज्ञ्ं न॑यतु प्रजा॒नन्न्  । घृ॒तं पिन्व॑न्न॒जरꣳ॑ सु॒वीरं॒ ब्रह्म॑ स॒मिद्-भ॑व॒त्याहु॑तीनां ॥सु॒व॒र्गाय॒ वा ए॒ष लो॒कायोप॑ धीयते॒ यत् कू॒र्मश्चत॑स्र॒ आशाः॒ प्र च॑रन्त्व॒ग्नय॒ इत्या॑ह॒ - [  ] \newline

\textbf{Pada Paata} \newline

धाम॑ । इति॑ । आ॒ह॒ । ए॒षा । वै । अ॒ग्नेः । स्व॒यं॒चि॒तिरिति॑ स्वयं - चि॒तिः । अ॒ग्निः । ए॒व । तत् । अ॒ग्निम् । चि॒नो॒ति । न । अ॒द्ध्व॒र्युः । आ॒त्मनः॑ । अ॒न्तः । ए॒ति॒ । चत॑स्रः । आशाः᳚ । प्रेति॑ । च॒र॒न्तु॒ । अ॒ग्नयः॑ । इ॒मम् । नः॒ । य॒ज्ञ्म् । न॒य॒तु॒ । प्र॒जा॒नन्निति॑ प्र - जा॒नन्न् ॥ घृ॒तम् । पिन्वन्न्॑ । अ॒जर᳚म् । सु॒वीर॒मिति॑ सु-वीर᳚म् । ब्रह्म॑ । स॒मिदिति॑ सं - इत् । भ॒व॒ति॒ । आहु॑तीना॒मित्या - हु॒ती॒ना॒म् ॥ सु॒व॒र्गायेति॑ सुवः - गाय॑ । वै । ए॒षः । लो॒काय॑ । उपेति॑ । धी॒य॒ते॒ । यत् । कू॒र्मः । चत॑स्रः । आशाः᳚ । प्रेति॑ । च॒र॒न्तु॒ । अ॒ग्नयः॑ । इति॑ । आ॒ह॒ ।  \newline


\textbf{Krama Paata} \newline

धामेति॑ । इत्या॑ह । आ॒है॒षा । ए॒षा वै । वा अ॒ग्नेः । अ॒ग्नेः स्व॑यञ्चि॒तिः । स्व॒य॒ञ्चि॒तिर॒ग्निः । स्व॒य॒ञ्चि॒तिरिति॑ स्वयम् - चि॒तिः । अ॒ग्निरे॒व । ए॒व तत् । तद॒ग्निम् । अ॒ग्निम् चि॑नोति । चि॒नो॒ति॒ न । नाद्ध्व॒र्युः । अ॒द्ध्व॒र्युरा॒त्मनः॑ । आ॒त्मनो॒ऽन्तः । अ॒न्तरे॑ति । ए॒ति॒ चत॑स्रः । चत॑स्र॒ आशाः᳚ । आशाः॒ प्र । प्र च॑रन्तु । च॒र॒न्त्व॒ग्नयः॑ । अ॒ग्नय॑ इ॒मम् । इ॒मम् नः॑ । नो॒ य॒ज्ञ्म् । य॒ज्ञ्म् न॑यतु । न॒य॒तु॒ प्र॒जा॒नन्न् । प्र॒जा॒नन्निति॑ प्र - जा॒नन्न् ॥ घृ॒तम् पिन्वन्न्॑ । पिन्व॑न्न॒जर᳚म् । अ॒जरꣳ॑ सु॒वीर᳚म् । सु॒वीर॒म् ब्रह्म॑ । सु॒वीर॒मिति॑ सु - वीर᳚म् । ब्रह्म॑ स॒मित् । स॒मिद् भ॑वति । स॒मिदिति॑ सम् - इत् । भ॒व॒त्याहु॑तीनाम् । आहु॑तीना॒मित्या - हु॒ती॒ना॒म् ॥ सु॒व॒र्गाय॒ वै । सु॒र्व॒र्गायेति॑ सुवः - गाय॑ । वा ए॒षः । ए॒ष लो॒काय॑ । लो॒कायोप॑ । उप॑ धीयते । धी॒य॒ते॒ यत् । यत् कू॒र्मः । कू॒र्मश्चत॑स्रः । चत॑स्र॒ आशाः᳚ । आशाः॒ प्र । प्र च॑रन्तु । च॒र॒न्त्व॒ग्नयः॑ । अ॒ग्नय॒ इति॑ । इत्या॑ह । आ॒ह॒ दिशः॑ \newline

\textbf{Jatai Paata} \newline

1. धामे तीति॒ धाम॒ धामेति॑ । \newline
2. इत्या॑हा॒हे तीत्या॑ह । \newline
3. आ॒है॒षैषा ऽऽहा॑ है॒षा । \newline
4. ए॒षा वै वा ए॒षैषा वै । \newline
5. वा अ॒ग्ने र॒ग्नेर् वै वा अ॒ग्नेः । \newline
6. अ॒ग्नेः स्व॑यञ्चि॒तिः स्व॑यञ्चि॒ति र॒ग्ने र॒ग्नेः स्व॑यञ्चि॒तिः । \newline
7. स्व॒य॒ञ्चि॒ति र॒ग्नि र॒ग्निः स्व॑यञ्चि॒तिः स्व॑यञ्चि॒ति र॒ग्निः । \newline
8. स्व॒य॒ञ्चि॒तिरिति॑ स्वयं - चि॒तिः । \newline
9. अ॒ग्नि रे॒वै वाग्नि र॒ग्नि रे॒व । \newline
10. ए॒व तत् तदे॒ वैव तत् । \newline
11. तद॒ग्नि म॒ग्निम् तत् तद॒ग्निम् । \newline
12. अ॒ग्निम् चि॑नोति चिनो त्य॒ग्नि म॒ग्निम् चि॑नोति । \newline
13. चि॒नो॒ति॒ न न चि॑नोति चिनोति॒ न । \newline
14. नाद्ध्व॒र्यु र॑द्ध्व॒र्युर् न नाद्ध्व॒र्युः । \newline
15. अ॒द्ध्व॒र्यु रा॒त्मन॑ आ॒त्मनो᳚ ऽद्ध्व॒र्यु र॑द्ध्व॒र्यु रा॒त्मनः॑ । \newline
16. आ॒त्मनो॒ ऽन्त र॒न्त रा॒त्मन॑ आ॒त्मनो॒ ऽन्तः । \newline
17. अ॒न्त रे᳚त्ये त्य॒न्त र॒न्त रे॑ति । \newline
18. ए॒ति॒ चत॑स्र॒ श्चत॑स्र एत्येति॒ चत॑स्रः । \newline
19. चत॑स्र॒ आशा॒ आशा॒ श्चत॑स्र॒ श्चत॑स्र॒ आशाः᳚ । \newline
20. आशाः॒ प्र प्राशा॒ आशाः॒ प्र । \newline
21. प्र च॑रन्तु चरन्तु॒ प्र प्र च॑रन्तु । \newline
22. च॒र॒न् त्व॒ग्नयो॒ ऽग्नय॑ श्चरन्तु चरन् त्व॒ग्नयः॑ । \newline
23. अ॒ग्नय॑ इ॒म मि॒म म॒ग्नयो॒ ऽग्नय॑ इ॒मम् । \newline
24. इ॒मन् नो॑ न इ॒म मि॒मन् नः॑ । \newline
25. नो॒ य॒ज्ञ्ं ॅय॒ज्ञ्म् नो॑ नो य॒ज्ञ्म् । \newline
26. य॒ज्ञ्म् न॑यतु नयतु य॒ज्ञ्ं ॅय॒ज्ञ्म् न॑यतु । \newline
27. न॒य॒तु॒ प्र॒जा॒नन् प्र॑जा॒नन् न॑यतु नयतु प्रजा॒नन्न् । \newline
28. प्र॒जा॒नन्निति॑ प्र - जा॒नन्न् । \newline
29. घृ॒तम् पिन्व॒न् पिन्व॑न् घृ॒तम् घृ॒तम् पिन्वन्न्॑ । \newline
30. पिन्व॑न् न॒जर॑ म॒जर॒म् पिन्व॒न् पिन्व॑न् न॒जर᳚म् । \newline
31. अ॒जरꣳ॑ सु॒वीरꣳ॑ सु॒वीर॑ म॒जर॑ म॒जरꣳ॑ सु॒वीर᳚म् । \newline
32. सु॒वीर॒म् ब्रह्म॒ ब्रह्म॑ सु॒वीरꣳ॑ सु॒वीर॒म् ब्रह्म॑ । \newline
33. सु॒वीर॒मिति॑ सु - वीर᳚म् । \newline
34. ब्रह्म॑ स॒मिथ् स॒मिद् ब्रह्म॒ ब्रह्म॑ स॒मित् । \newline
35. स॒मिद् भ॑वति भवति स॒मिथ् स॒मिद् भ॑वति । \newline
36. स॒मिदिति॑ सं - इत् । \newline
37. भ॒व॒ त्याहु॑तीना॒ माहु॑तीनाम् भवति भव॒ त्याहु॑तीनाम् । \newline
38. आहु॑तीना॒मित्या - हु॒ती॒ना॒म् । \newline
39. सु॒व॒र्गाय॒ वै वै सु॑व॒र्गाय॑ सुव॒र्गाय॒ वै । \newline
40. सु॒व॒र्गायेति॑ सुवः - गाय॑ । \newline
41. वा ए॒ष ए॒ष वै वा ए॒षः । \newline
42. ए॒ष लो॒काय॑ लो॒का यै॒ष ए॒ष लो॒काय॑ । \newline
43. लो॒का योपोप॑ लो॒काय॑ लो॒कायोप॑ । \newline
44. उप॑ धीयते धीयत॒ उपोप॑ धीयते । \newline
45. धी॒य॒ते॒ यद् यद् धी॑यते धीयते॒ यत् । \newline
46. यत् कू॒र्मः कू॒र्मो यद् यत् कू॒र्मः । \newline
47. कू॒र्म श्चत॑स्र॒ श्चत॑स्रः कू॒र्मः कू॒र्म श्चत॑स्रः । \newline
48. चत॑स्र॒ आशा॒ आशा॒ श्चत॑स्र॒ श्चत॑स्र॒ आशाः᳚ । \newline
49. आशाः॒ प्र प्राशा॒ आशाः॒ प्र । \newline
50. प्र च॑रन्तु चरन्तु॒ प्र प्र च॑रन्तु । \newline
51. च॒र॒न् त्व॒ग्नयो॒ ऽग्नय॑ श्चरन्तु चरन् त्व॒ग्नयः॑ । \newline
52. अ॒ग्नय॒ इती त्य॒ग्नयो॒ ऽग्नय॒ इति॑ । \newline
53. इत्या॑हा॒हे तीत्या॑ह । \newline
54. आ॒ह॒ दिशो॒ दिश॑ आहाह॒ दिशः॑ । \newline

\textbf{Ghana Paata } \newline

1. धामे तीति॒ धाम॒ धामे त्या॑हा॒ हेति॒ धाम॒ धामे त्या॑ह । \newline
2. इत्या॑हा॒हे तीत्या॑ है॒षैषा ऽऽहेतीत्या॑ है॒षा । \newline
3. आ॒है॒ षैषा ऽऽहा॑ है॒षा वै वा ए॒षा ऽऽहा॑ है॒षा वै । \newline
4. ए॒षा वै वा ए॒षैषा वा अ॒ग्ने र॒ग्नेर् वा ए॒षैषा वा अ॒ग्नेः । \newline
5. वा अ॒ग्ने र॒ग्नेर् वै वा अ॒ग्नेः स्व॑यञ्चि॒तिः स्व॑यञ्चि॒ति र॒ग्नेर् वै वा अ॒ग्नेः स्व॑यञ्चि॒तिः । \newline
6. अ॒ग्नेः स्व॑यञ्चि॒तिः स्व॑यञ्चि॒ति र॒ग्ने र॒ग्नेः स्व॑यञ्चि॒ति र॒ग्नि र॒ग्निः स्व॑यञ्चि॒ति र॒ग्ने र॒ग्नेः स्व॑यञ्चि॒ति र॒ग्निः । \newline
7. स्व॒य॒ञ्चि॒ति र॒ग्नि र॒ग्निः स्व॑यञ्चि॒तिः स्व॑यञ्चि॒ति र॒ग्नि रे॒वै वाग्निः स्व॑यञ्चि॒तिः स्व॑यञ्चि॒ति र॒ग्नि रे॒व । \newline
8. स्व॒य॒ञ्चि॒तिरिति॑ स्वयं - चि॒तिः । \newline
9. अ॒ग्नि रे॒वै वाग्नि र॒ग्नि रे॒व तत् तदे॒वाग्नि र॒ग्नि रे॒व तत् । \newline
10. ए॒व तत् तदे॒ वैव तद॒ग्नि म॒ग्निम् तदे॒ वैव तद॒ग्निम् । \newline
11. तद॒ग्नि म॒ग्निम् तत् तद॒ग्निम् चि॑नोति चिनो त्य॒ग्निम् तत् तद॒ग्निम् चि॑नोति । \newline
12. अ॒ग्निम् चि॑नोति चिनो त्य॒ग्नि म॒ग्निम् चि॑नोति॒ न न चि॑नो त्य॒ग्नि म॒ग्निम् चि॑नोति॒ न । \newline
13. चि॒नो॒ति॒ न न चि॑नोति चिनोति॒ नाद्ध्व॒र्यु र॑द्ध्व॒र्युर् न चि॑नोति चिनोति॒ नाद्ध्व॒र्युः । \newline
14. नाद्ध्व॒र्यु र॑द्ध्व॒र्युर् न नाद्ध्व॒र्यु रा॒त्मन॑ आ॒त्मनो᳚ ऽद्ध्व॒र्युर् न नाद्ध्व॒र्यु रा॒त्मनः॑ । \newline
15. अ॒द्ध्व॒र्यु रा॒त्मन॑ आ॒त्मनो᳚ ऽद्ध्व॒र्यु र॑द्ध्व॒र्यु रा॒त्मनो॒ ऽन्त र॒न्त रा॒त्मनो᳚ ऽद्ध्व॒र्यु र॑द्ध्व॒र्यु रा॒त्मनो॒ ऽन्तः । \newline
16. आ॒त्मनो॒ ऽन्त र॒न्त रा॒त्मन॑ आ॒त्मनो॒ ऽन्त रे᳚त्ये त्य॒न्त रा॒त्मन॑ आ॒त्मनो॒ ऽन्त रे॑ति । \newline
17. अ॒न्त रे᳚त्ये त्य॒न्त र॒न्त रे॑ति॒ चत॑स्र॒ श्चत॑स्र एत्य॒न्त र॒न्त रे॑ति॒ चत॑स्रः । \newline
18. ए॒ति॒ चत॑स्र॒ श्चत॑स्र एत्येति॒ चत॑स्र॒ आशा॒ आशा॒ श्चत॑स्र एत्येति॒ चत॑स्र॒ आशाः᳚ । \newline
19. चत॑स्र॒ आशा॒ आशा॒ श्चत॑स्र॒ श्चत॑स्र॒ आशाः॒ प्र प्राशा॒ श्चत॑स्र॒ श्चत॑स्र॒ आशाः॒ प्र । \newline
20. आशाः॒ प्र प्राशा॒ आशाः॒ प्र च॑रन्तु चरन्तु॒ प्राशा॒ आशाः॒ प्र च॑रन्तु । \newline
21. प्र च॑रन्तु चरन्तु॒ प्र प्र च॑रन् त्व॒ग्नयो॒ ऽग्नय॑ श्चरन्तु॒ प्र प्र च॑रन् त्व॒ग्नयः॑ । \newline
22. च॒र॒न् त्व॒ग्नयो॒ ऽग्नय॑ श्चरन्तु चरन् त्व॒ग्नय॑ इ॒म मि॒म म॒ग्नय॑ श्चरन्तु चरन् त्व॒ग्नय॑ इ॒मम् । \newline
23. अ॒ग्नय॑ इ॒म मि॒म म॒ग्नयो॒ ऽग्नय॑ इ॒मम् नो॑ न इ॒म म॒ग्नयो॒ ऽग्नय॑ इ॒मम् नः॑ । \newline
24. इ॒मम् नो॑ न इ॒म मि॒मम् नो॑ य॒ज्ञ्ं ॅय॒ज्ञ्म् न॑ इ॒म मि॒मम् नो॑ य॒ज्ञ्म् । \newline
25. नो॒ य॒ज्ञ्ं ॅय॒ज्ञ्न्नो॑ नो य॒ज्ञ्म् न॑यतु नयतु य॒ज्ञ्न्नो॑ नो य॒ज्ञ्म् न॑यतु । \newline
26. य॒ज्ञ्म् न॑यतु नयतु य॒ज्ञ्ं ॅय॒ज्ञ्म् न॑यतु प्रजा॒नन् प्र॑जा॒नन् न॑यतु य॒ज्ञ्ं ॅय॒ज्ञ्म् न॑यतु प्रजा॒नन्न् । \newline
27. न॒य॒तु॒ प्र॒जा॒नन् प्र॑जा॒नन् न॑यतु नयतु प्रजा॒नन्न् । \newline
28. प्र॒जा॒नन्निति॑ प्र - जा॒नन्न् । \newline
29. घृ॒तम् पिन्व॒न् पिन्व॑न् घृ॒तम् घृ॒तम् पिन्व॑न् न॒जर॑ म॒जर॒म् पिन्व॑न् घृ॒तम् घृ॒तम् पिन्व॑न् न॒जर᳚म् । \newline
30. पिन्व॑न् न॒जर॑ म॒जर॒म् पिन्व॒न् पिन्व॑न् न॒जरꣳ॑ सु॒वीरꣳ॑ सु॒वीर॑ म॒जर॒म् पिन्व॒न् पिन्व॑न् न॒जरꣳ॑ सु॒वीर᳚म् । \newline
31. अ॒जरꣳ॑ सु॒वीरꣳ॑ सु॒वीर॑ म॒जर॑ म॒जरꣳ॑ सु॒वीर॒म् ब्रह्म॒ ब्रह्म॑ सु॒वीर॑ म॒जर॑ म॒जरꣳ॑ सु॒वीर॒म् ब्रह्म॑ । \newline
32. सु॒वीर॒म् ब्रह्म॒ ब्रह्म॑ सु॒वीरꣳ॑ सु॒वीर॒म् ब्रह्म॑ स॒मिथ् स॒मिद् ब्रह्म॑ सु॒वीरꣳ॑ सु॒वीर॒म् ब्रह्म॑ स॒मित् । \newline
33. सु॒वीर॒मिति॑ सु - वीर᳚म् । \newline
34. ब्रह्म॑ स॒मिथ् स॒मिद् ब्रह्म॒ ब्रह्म॑ स॒मिद् भ॑वति भवति स॒मिद् ब्रह्म॒ ब्रह्म॑ स॒मिद् भ॑वति । \newline
35. स॒मिद् भ॑वति भवति स॒मिथ् स॒मिद् भ॑व॒ त्याहु॑तीना॒ माहु॑तीनाम् भवति स॒मिथ् स॒मिद् भ॑व॒ त्याहु॑तीनाम् । \newline
36. स॒मिदिति॑ सं - इत् । \newline
37. भ॒व॒ त्याहु॑तीना॒ माहु॑तीनाम् भवति भव॒ त्याहु॑तीनाम् । \newline
38. आहु॑तीना॒मित्या - हु॒ती॒ना॒म् । \newline
39. सु॒व॒र्गाय॒ वै वै सु॑व॒र्गाय॑ सुव॒र्गाय॒ वा ए॒ष ए॒ष वै सु॑व॒र्गाय॑ सुव॒र्गाय॒ वा ए॒षः । \newline
40. सु॒व॒र्गायेति॑ सुवः - गाय॑ । \newline
41. वा ए॒ष ए॒ष वै वा ए॒ष लो॒काय॑ लो॒कायै॒ष वै वा ए॒ष लो॒काय॑ । \newline
42. ए॒ष लो॒काय॑ लो॒कायै॒ष ए॒ष लो॒कायो पोप॑ लो॒कायै॒ष ए॒ष लो॒कायोप॑ । \newline
43. लो॒कायो पोप॑ लो॒काय॑ लो॒कायोप॑ धीयते धीयत॒ उप॑ लो॒काय॑ लो॒कायोप॑ धीयते । \newline
44. उप॑ धीयते धीयत॒ उपोप॑ धीयते॒ यद् यद् धी॑यत॒ उपोप॑ धीयते॒ यत् । \newline
45. धी॒य॒ते॒ यद् यद् धी॑यते धीयते॒ यत् कू॒र्मः कू॒र्मो यद् धी॑यते धीयते॒ यत् कू॒र्मः । \newline
46. यत् कू॒र्मः कू॒र्मो यद् यत् कू॒र्म श्चत॑स्र॒ श्चत॑स्रः कू॒र्मो यद् यत् कू॒र्म श्चत॑स्रः । \newline
47. कू॒र्म श्चत॑स्र॒ श्चत॑स्रः कू॒र्मः कू॒र्म श्चत॑स्र॒ आशा॒ आशा॒ श्चत॑स्रः कू॒र्मः कू॒र्म श्चत॑स्र॒ आशाः᳚ । \newline
48. चत॑स्र॒ आशा॒ आशा॒ श्चत॑स्र॒ श्चत॑स्र॒ आशाः॒ प्र प्राशा॒ श्चत॑स्र॒ श्चत॑स्र॒ आशाः॒ प्र । \newline
49. आशाः॒ प्र प्राशा॒ आशाः॒ प्र च॑रन्तु चरन्तु॒ प्राशा॒ आशाः॒ प्र च॑रन्तु । \newline
50. प्र च॑रन्तु चरन्तु॒ प्र प्र च॑रन् त्व॒ग्नयो॒ ऽग्नय॑ श्चरन्तु॒ प्र प्र च॑रन् त्व॒ग्नयः॑ । \newline
51. च॒र॒न् त्व॒ग्नयो॒ ऽग्नय॑ श्चरन्तु चरन् त्व॒ग्नय॒ इती त्य॒ग्नय॑ श्चरन्तु चरन् त्व॒ग्नय॒ इति॑ । \newline
52. अ॒ग्नय॒ इती त्य॒ग्नयो॒ ऽग्नय॒ इत्या॑हा॒हे त्य॒ग्नयो॒ ऽग्नय॒ इत्या॑ह । \newline
53. इत्या॑हा॒हे तीत्या॑ह॒ दिशो॒ दिश॑ आ॒हे तीत्या॑ह॒ दिशः॑ । \newline
54. आ॒ह॒ दिशो॒ दिश॑ आहाह॒ दिश॑ ए॒वैव दिश॑ आहाह॒ दिश॑ ए॒व । \newline
\pagebreak
\markright{ TS 5.7.8.3  \hfill https://www.vedavms.in \hfill}

\section{ TS 5.7.8.3 }

\textbf{TS 5.7.8.3 } \newline
\textbf{Samhita Paata} \newline

दिश॑ ए॒वैतेन॒ प्र जा॑नाती॒मं नो॑ य॒ज्ञ्ं न॑यतु प्रजा॒नन्नित्या॑ह सुव॒र्गस्य॑ लो॒कस्या॒भिनी᳚त्यै॒ ब्रह्म॑ स॒मिद्-भ॑व॒त्याहु॑तीना॒-मित्या॑ह॒ ब्रह्म॑णा॒ वै दे॒वाः सु॑व॒र्गं ॅलो॒कमा॑य॒न्॒ यद्-ब्रह्म॑ण्वत्योप॒दधा॑ति॒ ब्रह्म॑णै॒व तद्-यज॑मानः सुव॒र्गं ॅलो॒कमे॑ति प्र॒जाप॑ति॒र्वा ए॒ष यद॒ग्निस्तस्य॑ प्र॒जाः प॒शवः॒ छन्दाꣳ॑सि रू॒पꣳ सर्वा॒न्॒ वर्णा॒निष्ट॑कानां ( ) कुर्याद् रू॒पेणै॒व प्र॒जां प॒शून् छन्दाꣳ॒॒स्यव॑ रु॒न्धेऽथो᳚ प्र॒जाभ्य॑ ए॒वैनं॑ प॒शुभ्यः॒ छन्दो᳚भ्यो ऽव॒रुद्ध्य॑ चिनुते ॥ \newline

\textbf{Pada Paata} \newline

दिशः॑ । ए॒व । ए॒तेन॑ । प्रेति॑ । जा॒ना॒ति॒ । इ॒मम् । नः । य॒ज्ञ्म् । न॒य॒तु॒ । प्र॒जा॒नन्निति॑ प्र - जा॒नन्न् । इति॑ । आ॒ह॒ । सु॒व॒र्गस्येति॑ सुवः - गस्य॑ । लो॒कस्य॑ । अ॒भिनी᳚त्या॒ इत्य॒भि-नी॒त्यै॒ । ब्रह्म॑ । स॒मिदिति॑ सं - इत् । भ॒व॒ति॒ । आहु॑तीना॒मित्या - हु॒ती॒ना॒म् । इति॑ । आ॒ह॒ । ब्रह्म॑णा । वै । दे॒वाः । सु॒व॒र्गमिति॑ सुवः - गम् । लो॒कम् । आ॒य॒न्न् । यत् । ब्रह्म॑ण्व॒त्येति॒ ब्रह्मण्॑ - व॒त्या॒ । उ॒प॒दधा॒तीत्यु॑प - दधा॑ति । ब्रह्म॑णा । ए॒व । तत् । यज॑मानः । सु॒व॒र्गमिति॑ सुवः - गम् । लो॒कम् । ए॒ति॒ । प्र॒जाप॑ति॒रिति॑ प्र॒जा - प॒तिः॒ । वै । ए॒षः । यत् । अ॒ग्निः । तस्य॑ । प्र॒जा इति॑ प्र - जाः । प॒शवः॑ । छन्दाꣳ॑सि । रू॒पम् । सर्वान्॑ । वर्णान्॑ । इष्ट॑कानाम् ( ) । कु॒र्या॒त् । रू॒पेण॑ । ए॒व । प्र॒जामिति॑ प्र - जाम् । प॒शून् । छन्दाꣳ॑सि । अवेति॑ । रु॒न्धे॒ । अथो॒ इति॑ । प्र॒जाभ्य॒ इति॑ प्र - जाभ्यः॑ । ए॒व । ए॒न॒म् । प॒शुभ्य॒ इति॑ प॒शु-भ्यः॒ । छन्दो᳚भ्य॒ इति॒ छन्दः॑ - भ्यः॒ । अ॒व॒रुद्ध्येत्य॑व - रुद्ध्य॑ । चि॒नु॒ते॒ ॥  \newline


\textbf{Krama Paata} \newline

दिश॑ ए॒व । ए॒वैतेन॑ । ए॒तेन॒ प्र । प्र जा॑नाति । जा॒ना॒ती॒मम् । इ॒मम् नः॑ । नो॒ य॒ज्ञ्म् । य॒ज्ञ्म् न॑यतु । न॒य॒तु॒ प्र॒जा॒नन्न् । प्र॒जा॒नन्निति॑ । प्र॒जा॒नन्निति॑ प्र - जा॒नन्न् । इत्या॑ह । आ॒ह॒ सु॒व॒र्गस्य॑ । सु॒व॒र्गस्य॑ लो॒कस्य॑ । सु॒व॒र्गस्येति॑ सुवः - गस्य॑ । लो॒कस्या॒भिनी᳚त्यै । अ॒भिनी᳚त्यै॒ ब्रह्म॑ । अ॒भिनी᳚त्या॒ इत्य॒भि - नी॒त्यै॒ । ब्रह्म॑ स॒मित् । स॒मिद् भ॑वति । स॒मिदिति॑ सम् - इत् । भ॒व॒त्याहु॑तीनाम् । आहु॑तीना॒मिति॑ । आहु॑तीना॒मित्या - हु॒ती॒ना॒म् । इत्या॑ह । आ॒ह॒ ब्रह्म॑णा । ब्रह्म॑णा॒ वै । वै दे॒वाः । दे॒वाः सु॑व॒र्गम् । सु॒व॒र्गम् ॅलो॒कम् । सु॒व॒र्गमिति॑ सुवः - गम् । लो॒कमा॑यन्न् । आ॒य॒न्.॒ यत् । यद् ब्रह्म॑ण्वत्या । ब्रह्म॑ण्वत्योप॒दधा॑ति । ब्रह्म॑ण्व॒त्येति॒ ब्रह्मण्ण्॑ - व॒त्या॒ । उ॒प॒दधा॑ति॒ ब्रह्म॑णा । उ॒प॒दधा॒तीत्यु॑प - दधा॑ति । ब्रह्म॑णै॒व । ए॒व तत् । तद् यज॑मानः । यज॑मानः सुव॒र्गम् । सु॒व॒र्गम् ॅलो॒कम् । सु॒व॒र्गमिति॑ सुवः - गम् । लो॒कमे॑ति । ए॒ति॒ प्र॒जाप॑तिः । प्र॒जाप॑ति॒र् वै । प्र॒जाप॑ति॒रिति॑ प्र॒जा - प॒तिः॒ । वा ए॒षः । ए॒ष यत् । यद॒ग्निः । अ॒ग्निस्तस्य॑ । तस्य॑ प्र॒जाः । प्र॒जाः प॒शवः॑ । प्र॒जा इति॑ प्र - जाः । प॒शव॒श्छन्दाꣳ॑सि । छन्दाꣳ॑सि रू॒पम् । रू॒पꣳ सर्वान्॑ । सर्वा॒न्॒. वर्णान्॑ । वर्णा॒निष्ट॑कानाम् ( ) । इष्ट॑कानाम् कुर्यात् । कु॒र्या॒द् रू॒पेण॑ । रू॒पेणै॒व । ए॒व प्र॒जाम् । प्र॒जाम् प॒शून् । प्र॒जामिति॑ प्र - जाम् । प॒शूञ्छन्दाꣳ॑सि । छन्दाꣳ॒॒स्यव॑ । अव॑ रुन्धे । रु॒न्धेऽथो᳚ । अथो᳚ प्र॒जाभ्यः॑ । अथो॒ इत्यथो᳚ । प्र॒जाभ्य॑ ए॒व । प्र॒जाभ्य॒ इति॑ प्र - जाभ्यः॑ । ए॒वैन᳚म् । ए॒न॒म् प॒शुभ्यः॑ । प॒शुभ्य॒श्छन्दो᳚भ्यः । प॒शुभ्य॒ इति॑ प॒शु - भ्यः॒ । छन्दो᳚भ्योऽव॒रुद्ध्य॑ । छन्दो᳚भ्य॒ इति॒ छन्दः॑ - भ्यः॒ । अ॒व॒रुद्ध्य॑ चिनुते । अ॒व॒रुद्ध्येत्य॑व - रुद्ध्य॑ । चि॒नु॒त॒ इति॑ चिनुते । \newline

\textbf{Jatai Paata} \newline

1. दिश॑ ए॒वैव दिशो॒ दिश॑ ए॒व । \newline
2. ए॒वैते नै॒ते नै॒वै वैतेन॑ । \newline
3. ए॒तेन॒ प्र प्रैते नै॒तेन॒ प्र । \newline
4. प्र जा॑नाति जानाति॒ प्र प्र जा॑नाति । \newline
5. जा॒ना॒ ती॒म मि॒मम् जा॑नाति जाना ती॒मम् । \newline
6. इ॒मम् नो॑ न इ॒म मि॒मम् नः॑ । \newline
7. नो॒ य॒ज्ञ्ं ॅय॒ज्ञ्न्नो॑ नो य॒ज्ञ्म् । \newline
8. य॒ज्ञ्म् न॑यतु नयतु य॒ज्ञ्ं ॅय॒ज्ञ्म् न॑यतु । \newline
9. न॒य॒तु॒ प्र॒जा॒नन् प्र॑जा॒नन् न॑यतु नयतु प्रजा॒नन्न् । \newline
10. प्र॒जा॒नन् नितीति॑ प्रजा॒नन् प्र॑जा॒नन् निति॑ । \newline
11. प्र॒जा॒नन्निति॑ प्र - जा॒नन्न् । \newline
12. इत्या॑हा॒हे तीत्या॑ह । \newline
13. आ॒ह॒ सु॒व॒र्गस्य॑ सुव॒र्गस्या॑ हाह सुव॒र्गस्य॑ । \newline
14. सु॒व॒र्गस्य॑ लो॒कस्य॑ लो॒कस्य॑ सुव॒र्गस्य॑ सुव॒र्गस्य॑ लो॒कस्य॑ । \newline
15. सु॒व॒र्गस्येति॑ सुवः - गस्य॑ । \newline
16. लो॒कस्या॒ भिनी᳚त्या अ॒भिनी᳚त्यै लो॒कस्य॑ लो॒कस्या॒ भिनी᳚त्यै । \newline
17. अ॒भिनी᳚त्यै॒ ब्रह्म॒ ब्रह्मा॒भिनी᳚त्या अ॒भिनी᳚त्यै॒ ब्रह्म॑ । \newline
18. अ॒भिनी᳚त्या॒ इत्य॒भि - नी॒त्यै॒ । \newline
19. ब्रह्म॑ स॒मिथ् स॒मिद् ब्रह्म॒ ब्रह्म॑ स॒मित् । \newline
20. स॒मिद् भ॑वति भवति स॒मिथ् स॒मिद् भ॑वति । \newline
21. स॒मिदिति॑ सं - इत् । \newline
22. भ॒व॒ त्याहु॑तीना॒ माहु॑तीनाम् भवति भव॒ त्याहु॑तीनाम् । \newline
23. आहु॑तीना॒ मितीत्या हु॑तीना॒ माहु॑तीना॒ मिति॑ । \newline
24. आहु॑तीना॒मित्या - हु॒ती॒ना॒म् । \newline
25. इत्या॑हा॒हे तीत्या॑ह । \newline
26. आ॒ह॒ ब्रह्म॑णा॒ ब्रह्म॑णा ऽऽहाह॒ ब्रह्म॑णा । \newline
27. ब्रह्म॑णा॒ वै वै ब्रह्म॑णा॒ ब्रह्म॑णा॒ वै । \newline
28. वै दे॒वा दे॒वा वै वै दे॒वाः । \newline
29. दे॒वाः सु॑व॒र्गꣳ सु॑व॒र्गम् दे॒वा दे॒वाः सु॑व॒र्गम् । \newline
30. सु॒व॒र्गम् ॅलो॒कम् ॅलो॒कꣳ सु॑व॒र्गꣳ सु॑व॒र्गम् ॅलो॒कम् । \newline
31. सु॒व॒र्गमिति॑ सुवः - गम् । \newline
32. लो॒क मा॑यन् नायन् ॅलो॒कम् ॅलो॒क मा॑यन्न् । \newline
33. आ॒य॒न्॒. यद् यदा॑यन् नाय॒न्॒. यत् । \newline
34. यद् ब्रह्म॑ण्वत्या॒ ब्रह्म॑ण्वत्या॒ यद् यद् ब्रह्म॑ण्वत्या । \newline
35. ब्रह्म॑ण्व त्योप॒दधा᳚ त्युप॒दधा॑ति॒ ब्रह्म॑ण्वत्या॒ ब्रह्म॑ण्व त्योप॒दधा॑ति । \newline
36. ब्रह्म॑ण्व॒त्येति॒ ब्रह्मण्॑ - व॒त्या॒ । \newline
37. उ॒प॒दधा॑ति॒ ब्रह्म॑णा॒ ब्रह्म॑णो प॒दधा᳚ त्युप॒दधा॑ति॒ ब्रह्म॑णा । \newline
38. उ॒प॒दधा॒तीत्यु॑प - दधा॑ति । \newline
39. ब्रह्म॑ णै॒वैव ब्रह्म॑णा॒ ब्रह्म॑णै॒व । \newline
40. ए॒व तत् तदे॒ वैव तत् । \newline
41. तद् यज॑मानो॒ यज॑मान॒ स्तत् तद् यज॑मानः । \newline
42. यज॑मानः सुव॒र्गꣳ सु॑व॒र्गं ॅयज॑मानो॒ यज॑मानः सुव॒र्गम् । \newline
43. सु॒व॒र्गम् ॅलो॒कम् ॅलो॒कꣳ सु॑व॒र्गꣳ सु॑व॒र्गम् ॅलो॒कम् । \newline
44. सु॒व॒र्गमिति॑ सुवः - गम् । \newline
45. लो॒क मे᳚त्येति लो॒कम् ॅलो॒क मे॑ति । \newline
46. ए॒ति॒ प्र॒जाप॑तिः प्र॒जाप॑ति रेत्येति प्र॒जाप॑तिः । \newline
47. प्र॒जाप॑ति॒र् वै वै प्र॒जाप॑तिः प्र॒जाप॑ति॒र् वै । \newline
48. प्र॒जाप॑ति॒रिति॑ प्र॒जा - प॒तिः॒ । \newline
49. वा ए॒ष ए॒ष वै वा ए॒षः । \newline
50. ए॒ष यद् यदे॒ष ए॒ष यत् । \newline
51. यद॒ग्नि र॒ग्निर् यद् यद॒ग्निः । \newline
52. अ॒ग्नि स्तस्य॒ तस्या॒ग्नि र॒ग्नि स्तस्य॑ । \newline
53. तस्य॑ प्र॒जाः प्र॒जा स्तस्य॒ तस्य॑ प्र॒जाः । \newline
54. प्र॒जाः प॒शवः॑ प॒शवः॑ प्र॒जाः प्र॒जाः प॒शवः॑ । \newline
55. प्र॒जा इति॑ प्र - जाः । \newline
56. प॒शव॒ श्छन्दाꣳ॑सि॒ छन्दाꣳ॑सि प॒शवः॑ प॒शव॒ श्छन्दाꣳ॑सि । \newline
57. छन्दाꣳ॑सि रू॒पꣳ रू॒पम् छन्दाꣳ॑सि॒ छन्दाꣳ॑सि रू॒पम् । \newline
58. रू॒पꣳ सर्वा॒न् थ्सर्वा᳚न् रू॒पꣳ रू॒पꣳ सर्वान्॑ । \newline
59. सर्वा॒न्॒. वर्णा॒न्॒. वर्णा॒न् थ्सर्वा॒न् थ्सर्वा॒न्॒. वर्णान्॑ । \newline
60. वर्णा॒ निष्ट॑काना॒ मिष्ट॑कानां॒ ॅवर्णा॒न्॒. वर्णा॒ निष्ट॑कानाम् । \newline
61. इष्ट॑कानाम् कुर्यात् कुर्या॒ दिष्ट॑काना॒ मिष्ट॑कानाम् कुर्यात् । \newline
62. कु॒र्या॒द् रू॒पेण॑ रू॒पेण॑ कुर्यात् कुर्याद् रू॒पेण॑ । \newline
63. रू॒पेणै॒ वैव रू॒पेण॑ रू॒पेणै॒व । \newline
64. ए॒व प्र॒जाम् प्र॒जा मे॒वैव प्र॒जाम् । \newline
65. प्र॒जाम् प॒शून् प॒शून् प्र॒जाम् प्र॒जाम् प॒शून् । \newline
66. प्र॒जामिति॑ प्र - जाम् । \newline
67. प॒शून् छन्दाꣳ॑सि॒ छन्दाꣳ॑सि प॒शून् प॒शून् छन्दाꣳ॑सि । \newline
68. छन्दाꣳ॒॒ स्यवाव॒ च्छन्दाꣳ॑सि॒ छन्दाꣳ॒॒स्यव॑ । \newline
69. अव॑ रुन्धे रु॒न्धे ऽवाव॑ रुन्धे । \newline
70. रु॒न्धे ऽथो॒ अथो॑ रुन्धे रु॒न्धे ऽथो᳚ । \newline
71. अथो᳚ प्र॒जाभ्यः॑ प्र॒जाभ्यो ऽथो॒ अथो᳚ प्र॒जाभ्यः॑ । \newline
72. अथो॒ इत्यथो᳚ । \newline
73. प्र॒जाभ्य॑ ए॒वैव प्र॒जाभ्यः॑ प्र॒जाभ्य॑ ए॒व । \newline
74. प्र॒जाभ्य॒ इति॑ प्र - जाभ्यः॑ । \newline
75. ए॒वैन॑ मेन मे॒वै वैन᳚म् । \newline
76. ए॒न॒म् प॒शुभ्यः॑ प॒शुभ्य॑ एन मेनम् प॒शुभ्यः॑ । \newline
77. प॒शुभ्य॒ श्छन्दो᳚भ्य॒ श्छन्दो᳚भ्यः प॒शुभ्यः॑ प॒शुभ्य॒ श्छन्दो᳚भ्यः । \newline
78. प॒शुभ्य॒ इति॑ प॒शु - भ्यः॒ । \newline
79. छन्दो᳚भ्यो ऽव॒रुद्ध्या॑ व॒रुद्ध्य॒ छन्दो᳚भ्य॒ श्छन्दो᳚भ्यो ऽव॒रुद्ध्य॑ । \newline
80. छन्दो᳚भ्य॒ इति॒ छन्दः॑ - भ्यः॒ । \newline
81. अ॒व॒रुद्ध्य॑ चिनुते चिनुते ऽव॒रुद्ध्या॑ व॒रुद्ध्य॑ चिनुते । \newline
82. अ॒व॒रुद्ध्येत्य॑व - रुद्ध्य॑ । \newline
83. चि॒नु॒त॒ इति॑ चिनुते । \newline

\textbf{Ghana Paata } \newline

1. दिश॑ ए॒वैव दिशो॒ दिश॑ ए॒वैते नै॒ते नै॒व दिशो॒ दिश॑ ए॒वैतेन॑ । \newline
2. ए॒वैते नै॒ते नै॒वैवैतेन॒ प्र प्रैते नै॒वैवैतेन॒ प्र । \newline
3. ए॒तेन॒ प्र प्रैते नै॒तेन॒ प्र जा॑नाति जानाति॒ प्रैते नै॒तेन॒ प्र जा॑नाति । \newline
4. प्र जा॑नाति जानाति॒ प्र प्र जा॑नाती॒म मि॒मम् जा॑नाति॒ प्र प्र जा॑नाती॒मम् । \newline
5. जा॒ना॒ती॒म मि॒मम् जा॑नाति जानाती॒मम् नो॑ न इ॒मम् जा॑नाति जानाती॒मम् नः॑ । \newline
6. इ॒मम् नो॑ न इ॒म मि॒मम् नो॑ य॒ज्ञ्ं ॅय॒ज्ञ्न्न॑ इ॒म मि॒मम् नो॑ य॒ज्ञ्म् । \newline
7. नो॒ य॒ज्ञ्ं ॅय॒ज्ञ्न् नो॑ नो य॒ज्ञ्म् न॑यतु नयतु य॒ज्ञ्न् नो॑ नो य॒ज्ञ्म् न॑यतु । \newline
8. य॒ज्ञ्म् न॑यतु नयतु य॒ज्ञ्ं ॅय॒ज्ञ्म् न॑यतु प्रजा॒नन् प्र॑जा॒नन् न॑यतु य॒ज्ञ्ं ॅय॒ज्ञ्म् न॑यतु प्रजा॒नन्न् । \newline
9. न॒य॒तु॒ प्र॒जा॒नन् प्र॑जा॒नन् न॑यतु नयतु प्रजा॒नन् नितीति॑ प्रजा॒नन् न॑यतु नयतु प्रजा॒नन् निति॑ । \newline
10. प्र॒जा॒नन् नितीति॑ प्रजा॒नन् प्र॑जा॒नन् नित्या॑हा॒हेति॑ प्रजा॒नन् प्र॑जा॒नन् नित्या॑ह । \newline
11. प्र॒जा॒नन्निति॑ प्र - जा॒नन्न् । \newline
12. इत्या॑हा॒हे तीत्या॑ह सुव॒र्गस्य॑ सुव॒र्गस्या॒हे तीत्या॑ह सुव॒र्गस्य॑ । \newline
13. आ॒ह॒ सु॒व॒र्गस्य॑ सुव॒र्गस्या॑हाह सुव॒र्गस्य॑ लो॒कस्य॑ लो॒कस्य॑ सुव॒र्ग स्या॑हाह सुव॒र्गस्य॑ लो॒कस्य॑ । \newline
14. सु॒व॒र्गस्य॑ लो॒कस्य॑ लो॒कस्य॑ सुव॒र्गस्य॑ सुव॒र्गस्य॑ लो॒कस्या॒ भिनी᳚त्या अ॒भिनी᳚त्यै लो॒कस्य॑ सुव॒र्गस्य॑ सुव॒र्गस्य॑ लो॒कस्या॒ भिनी᳚त्यै । \newline
15. सु॒व॒र्गस्येति॑ सुवः - गस्य॑ । \newline
16. लो॒कस्या॒ भिनी᳚त्या अ॒भिनी᳚त्यै लो॒कस्य॑ लो॒कस्या॒ भिनी᳚त्यै॒ ब्रह्म॒ ब्रह्मा॒ भिनी᳚त्यै लो॒कस्य॑ लो॒कस्या॒ भिनी᳚त्यै॒ ब्रह्म॑ । \newline
17. अ॒भिनी᳚त्यै॒ ब्रह्म॒ ब्रह्मा॒ भिनी᳚त्या अ॒भिनी᳚त्यै॒ ब्रह्म॑ स॒मिथ् स॒मिद् ब्रह्मा॒ भिनी᳚त्या अ॒भिनी᳚त्यै॒ ब्रह्म॑ स॒मित् । \newline
18. अ॒भिनी᳚त्या॒ इत्य॒भि - नी॒त्यै॒ । \newline
19. ब्रह्म॑ स॒मिथ् स॒मिद् ब्रह्म॒ ब्रह्म॑ स॒मिद् भ॑वति भवति स॒मिद् ब्रह्म॒ ब्रह्म॑ स॒मिद् भ॑वति । \newline
20. स॒मिद् भ॑वति भवति स॒मिथ् स॒मिद् भ॑व॒ त्याहु॑तीना॒ माहु॑तीनाम् भवति स॒मिथ् स॒मिद् भ॑व॒ त्याहु॑तीनाम् । \newline
21. स॒मिदिति॑ सं - इत् । \newline
22. भ॒व॒ त्याहु॑तीना॒ माहु॑तीनाम् भवति भव॒ त्याहु॑तीना॒ मिती त्याहु॑तीनाम् भवति भव॒ त्याहु॑तीना॒ मिति॑ । \newline
23. आहु॑तीना॒ मिती त्याहु॑तीना॒ माहु॑तीना॒ मित्या॑हा॒हे त्याहु॑तीना॒ माहु॑तीना॒ मित्या॑ह । \newline
24. आहु॑तीना॒मित्या - हु॒ती॒ना॒म् । \newline
25. इत्या॑हा॒हे तीत्या॑ह॒ ब्रह्म॑णा॒ ब्रह्म॑णा॒ ऽऽहेतीत्या॑ह॒ ब्रह्म॑णा । \newline
26. आ॒ह॒ ब्रह्म॑णा॒ ब्रह्म॑णा ऽऽहाह॒ ब्रह्म॑णा॒ वै वै ब्रह्म॑णा ऽऽहाह॒ ब्रह्म॑णा॒ वै । \newline
27. ब्रह्म॑णा॒ वै वै ब्रह्म॑णा॒ ब्रह्म॑णा॒ वै दे॒वा दे॒वा वै ब्रह्म॑णा॒ ब्रह्म॑णा॒ वै दे॒वाः । \newline
28. वै दे॒वा दे॒वा वै वै दे॒वाः सु॑व॒र्गꣳ सु॑व॒र्गम् दे॒वा वै वै दे॒वाः सु॑व॒र्गम् । \newline
29. दे॒वाः सु॑व॒र्गꣳ सु॑व॒र्गम् दे॒वा दे॒वाः सु॑व॒र्गम् ॅलो॒कम् ॅलो॒कꣳ सु॑व॒र्गम् दे॒वा दे॒वाः सु॑व॒र्गम् ॅलो॒कम् । \newline
30. सु॒व॒र्गम् ॅलो॒कम् ॅलो॒कꣳ सु॑व॒र्गꣳ सु॑व॒र्गम् ॅलो॒क मा॑यन् नायन् ॅलो॒कꣳ सु॑व॒र्गꣳ सु॑व॒र्गम् ॅलो॒क मा॑यन्न् । \newline
31. सु॒व॒र्गमिति॑ सुवः - गम् । \newline
32. लो॒क मा॑यन् नायन् ॅलो॒कम् ॅलो॒क मा॑य॒न्॒. यद् यदा॑यन् ॅलो॒कम् ॅलो॒क मा॑य॒न्॒. यत् । \newline
33. आ॒य॒न्॒. यद् यदा॑यन् नाय॒न्॒. यद् ब्रह्म॑ण्वत्या॒ ब्रह्म॑ण्वत्या॒ यदा॑यन् नाय॒न्॒. यद् ब्रह्म॑ण्वत्या । \newline
34. यद् ब्रह्म॑ण्वत्या॒ ब्रह्म॑ण्वत्या॒ यद् यद् ब्रह्म॑ण्व त्योप॒दधा᳚ त्युप॒दधा॑ति॒ ब्रह्म॑ण्वत्या॒ यद् यद् ब्रह्म॑ण्व त्योप॒दधा॑ति । \newline
35. ब्रह्म॑ण्व त्योप॒दधा᳚ त्युप॒दधा॑ति॒ ब्रह्म॑ण्वत्या॒ ब्रह्म॑ण्व त्योप॒दधा॑ति॒ ब्रह्म॑णा॒ ब्रह्म॑णो प॒दधा॑ति॒ ब्रह्म॑ण्वत्या॒ ब्रह्म॑ण्व त्योप॒दधा॑ति॒ ब्रह्म॑णा । \newline
36. ब्रह्म॑ण्व॒त्येति॒ ब्रह्मण्॑ - व॒त्या॒ । \newline
37. उ॒प॒दधा॑ति॒ ब्रह्म॑णा॒ ब्रह्म॑ णोप॒दधा᳚ त्युप॒दधा॑ति॒ ब्रह्म॑णै॒वैव ब्रह्म॑ णोप॒दधा᳚ त्युप॒दधा॑ति॒ ब्रह्म॑णै॒व । \newline
38. उ॒प॒दधा॒तीत्यु॑प - दधा॑ति । \newline
39. ब्रह्म॑णै॒वैव ब्रह्म॑णा॒ ब्रह्म॑णै॒व तत् तदे॒व ब्रह्म॑णा॒ ब्रह्म॑णै॒व तत् । \newline
40. ए॒व तत् तदे॒वैव तद् यज॑मानो॒ यज॑मान॒ स्तदे॒ वैव तद् यज॑मानः । \newline
41. तद् यज॑मानो॒ यज॑मान॒ स्तत् तद् यज॑मानः सुव॒र्गꣳ सु॑व॒र्गं ॅयज॑मान॒ स्तत् तद् यज॑मानः सुव॒र्गम् । \newline
42. यज॑मानः सुव॒र्गꣳ सु॑व॒र्गं ॅयज॑मानो॒ यज॑मानः सुव॒र्गम् ॅलो॒कम् ॅलो॒कꣳ सु॑व॒र्गं ॅयज॑मानो॒ यज॑मानः सुव॒र्गम् ॅलो॒कम् । \newline
43. सु॒व॒र्गम् ॅलो॒कम् ॅलो॒कꣳ सु॑व॒र्गꣳ सु॑व॒र्गम् ॅलो॒क मे᳚त्येति लो॒कꣳ सु॑व॒र्गꣳ सु॑व॒र्गम् ॅलो॒क मे॑ति । \newline
44. सु॒व॒र्गमिति॑ सुवः - गम् । \newline
45. लो॒क मे᳚त्येति लो॒कम् ॅलो॒क मे॑ति प्र॒जाप॑तिः प्र॒जाप॑ति रेति लो॒कम् ॅलो॒क मे॑ति प्र॒जाप॑तिः । \newline
46. ए॒ति॒ प्र॒जाप॑तिः प्र॒जाप॑ति रेत्येति प्र॒जाप॑ति॒र् वै वै प्र॒जाप॑ति रेत्येति प्र॒जाप॑ति॒र् वै । \newline
47. प्र॒जाप॑ति॒र् वै वै प्र॒जाप॑तिः प्र॒जाप॑ति॒र् वा ए॒ष ए॒ष वै प्र॒जाप॑तिः प्र॒जाप॑ति॒र् वा ए॒षः । \newline
48. प्र॒जाप॑ति॒रिति॑ प्र॒जा - प॒तिः॒ । \newline
49. वा ए॒ष ए॒ष वै वा ए॒ष यद् यदे॒ष वै वा ए॒ष यत् । \newline
50. ए॒ष यद् यदे॒ष ए॒ष यद॒ग्नि र॒ग्निर् यदे॒ष ए॒ष यद॒ग्निः । \newline
51. यद॒ग्नि र॒ग्निर् यद् यद॒ग्नि स्तस्य॒ तस्या॒ग्निर् यद् यद॒ग्नि स्तस्य॑ । \newline
52. अ॒ग्नि स्तस्य॒ तस्या॒ग्नि र॒ग्नि स्तस्य॑ प्र॒जाः प्र॒जा स्तस्या॒ग्नि र॒ग्नि स्तस्य॑ प्र॒जाः । \newline
53. तस्य॑ प्र॒जाः प्र॒जा स्तस्य॒ तस्य॑ प्र॒जाः प॒शवः॑ प॒शवः॑ प्र॒जा स्तस्य॒ तस्य॑ प्र॒जाः प॒शवः॑ । \newline
54. प्र॒जाः प॒शवः॑ प॒शवः॑ प्र॒जाः प्र॒जाः प॒शव॒ श्छन्दाꣳ॑सि॒ छन्दाꣳ॑सि प॒शवः॑ प्र॒जाः प्र॒जाः प॒शव॒ श्छन्दाꣳ॑सि । \newline
55. प्र॒जा इति॑ प्र - जाः । \newline
56. प॒शव॒ श्छन्दाꣳ॑सि॒ छन्दाꣳ॑सि प॒शवः॑ प॒शव॒ श्छन्दाꣳ॑सि रू॒पꣳ रू॒पम् छन्दाꣳ॑सि प॒शवः॑ प॒शव॒ श्छन्दाꣳ॑सि रू॒पम् । \newline
57. छन्दाꣳ॑सि रू॒पꣳ रू॒पम् छन्दाꣳ॑सि॒ छन्दाꣳ॑सि रू॒पꣳ सर्वा॒न् थ्सर्वा᳚न् रू॒पम् छन्दाꣳ॑सि॒ छन्दाꣳ॑सि रू॒पꣳ सर्वान्॑ । \newline
58. रू॒पꣳ सर्वा॒न् थ्सर्वा᳚न् रू॒पꣳ रू॒पꣳ सर्वा॒न्॒. वर्णा॒न्॒. वर्णा॒न् थ्सर्वा᳚न् रू॒पꣳ रू॒पꣳ सर्वा॒न्॒. वर्णान्॑ । \newline
59. सर्वा॒न्॒. वर्णा॒न्॒. वर्णा॒न् थ्सर्वा॒न् थ्सर्वा॒न्॒. वर्णा॒ निष्ट॑काना॒ मिष्ट॑कानां॒ ॅवर्णा॒न् थ्सर्वा॒न् थ्सर्वा॒न्॒. वर्णा॒ निष्ट॑कानाम् । \newline
60. वर्णा॒ निष्ट॑काना॒ मिष्ट॑कानां॒ ॅवर्णा॒न्॒. वर्णा॒ निष्ट॑कानाम् कुर्यात् कुर्या॒ दिष्ट॑कानां॒ ॅवर्णा॒न्॒. वर्णा॒ निष्ट॑कानाम् कुर्यात् । \newline
61. इष्ट॑कानाम् कुर्यात् कुर्या॒ दिष्ट॑काना॒ मिष्ट॑कानाम् कुर्याद् रू॒पेण॑ रू॒पेण॑ कुर्या॒ दिष्ट॑काना॒ मिष्ट॑कानाम् कुर्याद् रू॒पेण॑ । \newline
62. कु॒र्या॒द् रू॒पेण॑ रू॒पेण॑ कुर्यात् कुर्याद् रू॒पेणै॒वैव रू॒पेण॑ कुर्यात् कुर्याद् रू॒पेणै॒व । \newline
63. रू॒पेणै॒वैव रू॒पेण॑ रू॒पेणै॒व प्र॒जाम् प्र॒जा मे॒व रू॒पेण॑ रू॒पेणै॒व प्र॒जाम् । \newline
64. ए॒व प्र॒जाम् प्र॒जा मे॒वैव प्र॒जाम् प॒शून् प॒शून् प्र॒जा मे॒वैव प्र॒जाम् प॒शून् । \newline
65. प्र॒जाम् प॒शून् प॒शून् प्र॒जाम् प्र॒जाम् प॒शून् छन्दाꣳ॑सि॒ छन्दाꣳ॑सि प॒शून् प्र॒जाम् प्र॒जाम् प॒शून् छन्दाꣳ॑सि । \newline
66. प्र॒जामिति॑ प्र - जाम् । \newline
67. प॒शून् छन्दाꣳ॑सि॒ छन्दाꣳ॑सि प॒शून् प॒शून् छन्दाꣳ॒॒ स्यवाव॒ च्छन्दाꣳ॑सि प॒शून् प॒शून् 
छन्दाꣳ॒॒ स्यव॑ । \newline
68. छन्दादाꣳ॒॒ स्यवाव॒ च्छन्दाꣳ॑सि॒ छन्दाꣳ॒॒ स्यव॑ रुन्धे रु॒न्धे ऽव॒ च्छन्दाꣳ॑सि॒ छन्दाꣳ॒॒स्यव॑ रुन्धे । \newline
69. अव॑ रुन्धे रु॒न्धे ऽवाव॑ रु॒न्धे ऽथो॒ अथो॑ रु॒न्धे ऽवाव॑ रु॒न्धे ऽथो᳚ । \newline
70. रु॒न्धे ऽथो॒ अथो॑ रुन्धे रु॒न्धे ऽथो᳚ प्र॒जाभ्यः॑ प्र॒जाभ्यो ऽथो॑ रुन्धे रु॒न्धे ऽथो᳚ प्र॒जाभ्यः॑ । \newline
71. अथो᳚ प्र॒जाभ्यः॑ प्र॒जाभ्यो ऽथो॒ अथो᳚ प्र॒जाभ्य॑ ए॒वैव प्र॒जाभ्यो ऽथो॒ अथो᳚ प्र॒जाभ्य॑ ए॒व । \newline
72. अथो॒ इत्यथो᳚ । \newline
73. प्र॒जाभ्य॑ ए॒वैव प्र॒जाभ्यः॑ प्र॒जाभ्य॑ ए॒वैन॑ मेन मे॒व प्र॒जाभ्यः॑ प्र॒जाभ्य॑ ए॒वैन᳚म् । \newline
74. प्र॒जाभ्य॒ इति॑ प्र - जाभ्यः॑ । \newline
75. ए॒वैन॑ मेन मे॒वै वैन॑म् प॒शुभ्यः॑ प॒शुभ्य॑ एन मे॒वै वैन॑म् प॒शुभ्यः॑ । \newline
76. ए॒न॒म् प॒शुभ्यः॑ प॒शुभ्य॑ एन मेनम् प॒शुभ्य॒ श्छन्दो᳚भ्य॒ श्छन्दो᳚भ्यः प॒शुभ्य॑ एन मेनम् प॒शुभ्य॒ श्छन्दो᳚भ्यः । \newline
77. प॒शुभ्य॒ श्छन्दो᳚भ्य॒ श्छन्दो᳚भ्यः प॒शुभ्यः॑ प॒शुभ्य॒ श्छन्दो᳚भ्यो ऽव॒रुद्ध्या॑ व॒रुद्ध्य॒ छन्दो᳚भ्यः प॒शुभ्यः॑ प॒शुभ्य॒ श्छन्दो᳚भ्यो ऽव॒रुद्ध्य॑ । \newline
78. प॒शुभ्य॒ इति॑ प॒शु - भ्यः॒ । \newline
79. छन्दो᳚भ्यो ऽव॒रुद्ध्या॑ व॒रुद्ध्य॒ छन्दो᳚भ्य॒ श्छन्दो᳚भ्यो ऽव॒रुद्ध्य॑ चिनुते चिनुते ऽव॒रुद्ध्य॒ छन्दो᳚भ्य॒ श्छन्दो᳚भ्यो ऽव॒रुद्ध्य॑ चिनुते । \newline
80. छन्दो᳚भ्य॒ इति॒ छन्दः॑ - भ्यः॒ । \newline
81. अ॒व॒रुद्ध्य॑ चिनुते चिनुते ऽव॒रुद्ध्या॑ व॒रुद्ध्य॑ चिनुते । \newline
82. अ॒व॒रुद्ध्येत्य॑व - रुद्ध्य॑ । \newline
83. चि॒नु॒त॒ इति॑ चिनुते । \newline
\pagebreak
\markright{ TS 5.7.9.1  \hfill https://www.vedavms.in \hfill}

\section{ TS 5.7.9.1 }

\textbf{TS 5.7.9.1 } \newline
\textbf{Samhita Paata} \newline

मयि॑ गृह्णा॒म्यग्रे॑ अ॒ग्निꣳ रा॒यस्पोषा॑य सुप्रजा॒स्त्वाय॑ सु॒वीर्या॑य । मयि॑ प्र॒जां मयि॒ वर्चो॑ दधा॒म्यरि॑ष्टाः स्याम त॒नुवा॑ सु॒वीराः᳚ ॥यो नो॑ अ॒ग्निः पि॑तरो हृ॒थ्स्व॑न्तरम॑र्त्यो॒ मर्त्याꣳ॑ आवि॒वेश॑ । तमा॒त्मन् परि॑ गृह्णीमहे व॒यं मा सो अ॒स्माꣳ अ॑व॒हाय॒ परा॑ गात् ॥ यद॑द्ध्व॒र्युरा॒त्मन्न॒ग्निम-गृ॑हीत्वा॒ऽग्निं चि॑नु॒याद्यो᳚ऽस्य॒ स्वो᳚ऽग्निस्तमपि॒ - [  ] \newline

\textbf{Pada Paata} \newline

मयि॑ । गृ॒ह्णा॒मि॒ । अग्रे᳚ । अ॒ग्निम् । रा॒यः । पोषा॑य । सु॒प्र॒जा॒स्त्वायेति॑ सुप्रजाः - त्वाय॑ । सु॒वीर्या॒येति॑ सु - वीर्या॑य ॥ मयि॑ । प्र॒जामिति॑ प्र - जाम् । मयि॑ । वर्चः॑ । द॒धा॒मि॒ । अरि॑ष्टाः । स्या॒म॒ । त॒नुवा᳚ । सु॒वीरा॒ इति॑ सु - वीराः᳚ ॥ यः । नः॒ । अ॒ग्निः । पि॒त॒रः॒ । हृ॒थ्स्विति॑ हृत् - सु । अ॒न्तः । अम॑र्त्यः । मर्त्यान्॑ । आ॒वि॒वेशेत्या᳚ - वि॒वेश॑ ॥ तम् । आ॒त्मन्न् । परीति॑ । गृ॒ह्णी॒म॒हे॒ । व॒यम् । मा । सः । अ॒स्मान् । अ॒व॒हायेत्य॑व - हाय॑ । परेति॑ । गा॒त् ॥ यत् । अ॒द्ध्व॒र्युः । आ॒त्मन्न् । अ॒ग्निम् । अगृ॑हीत्वा । अ॒ग्निम् । चि॒नु॒यात् । यः । अ॒स्य॒ । स्वः । अ॒ग्निः । तम् । अपीति॑ ।  \newline


\textbf{Krama Paata} \newline

मयि॑ गृह्णामि । गृ॒ह्णा॒म्यग्रे᳚ । अग्रे॑ अ॒ग्निम् । अ॒ग्निꣳ रा॒यः । रा॒यस्पोषा॑य । पोषा॑य सुप्रजा॒स्त्वाय॑ । सु॒प्र॒जा॒स्त्वाय॑ सु॒वीर्या॑य । सु॒प्र॒जा॒स्त्वायेति॑ सुप्रजाः - त्वाय॑ । सु॒वीर्या॒येति॑ सु - वीर्या॑य ॥ मयि॑ प्र॒जाम् । प्र॒जाम् मयि॑ । प्र॒जामिति॑ प्र - जाम् । मयि॒ वर्चः॑ । वर्चो॑ दधामि । द॒धा॒म्यरि॑ष्टाः । अरि॑ष्टाः स्याम । स्या॒म॒ त॒नुवा᳚ । त॒नुवा॑ सु॒वीराः᳚ । सु॒वीरा॒ इति॑ सु - वीराः᳚ ॥ यो नः॑ । नो॒ अ॒ग्निः । अ॒ग्निः पि॑तरः । पि॒त॒रो॒ हृ॒थ्सु । हृ॒थ्स्व॑न्तः । हृ॒थ्स्विति॑ हृत् - सु । अ॒न्तरम॑र्त्यः । अम॑र्त्यो॒ मर्त्यान्॑ । मर्त्याꣳ॑ आवि॒वेश॑ । आ॒वि॒वेशेत्या᳚ - वि॒वेश॑ ॥ तमा॒त्मन्न् । आ॒त्मन् परि॑ । परि॑ गृह्णीमहे । गृ॒ह्णी॒म॒हे॒ व॒यम् । व॒यम् मा । मा सः । सो अ॒स्मान् । अ॒स्माꣳ अ॑व॒हाय॑ । अ॒व॒हाय॒ परा᳚ । अ॒व॒हायेत्य॑व - हाय॑ । परा॑ गात् । गा॒दिति॑ गात् ॥ यद॑द्ध्व॒र्युः । अ॒र्द्ध॒र्युरा॒त्मन्न् । आ॒त्मन्न॒ग्निम् । अ॒ग्निमगृ॑हीत्वा । अगृ॑हीत्वा॒ऽग्निम् । अ॒ग्निम् चि॑नु॒यात् । चि॒नु॒याद् यः । यो᳚ऽस्य । अ॒स्य॒ स्वः । स्वो᳚ऽग्निः । अ॒ग्निस्तम् । तमपि॑ । अपि॒ यज॑मानाय \newline

\textbf{Jatai Paata} \newline

1. मयि॑ गृह्णामि गृह्णामि॒ मयि॒ मयि॑ गृह्णामि । \newline
2. गृ॒ह्णा॒ म्यग्रे॒ अग्रे॑ गृह्णामि गृह्णा॒ म्यग्रे᳚ । \newline
3. अग्रे॑ अ॒ग्नि म॒ग्नि मग्रे ऽग्रे॑ अ॒ग्निम् । \newline
4. अ॒ग्निꣳ रा॒यो रा॒यो᳚ ऽग्नि म॒ग्निꣳ रा॒यः । \newline
5. रा॒यस् पोषा॑य॒ पोषा॑य रा॒यो रा॒यस् पोषा॑य । \newline
6. पोषा॑य सुप्रजा॒स्त्वाय॑ सुप्रजा॒स्त्वाय॒ पोषा॑य॒ पोषा॑य सुप्रजा॒स्त्वाय॑ । \newline
7. सु॒प्र॒जा॒स्त्वाय॑ सु॒वीर्या॑य सु॒वीर्या॑य सुप्रजा॒स्त्वाय॑ सुप्रजा॒स्त्वाय॑ सु॒वीर्या॑य । \newline
8. सु॒प्र॒जा॒स्त्वायेति॑ सुप्रजाः - त्वाय॑ । \newline
9. सु॒वीर्या॒येति॑ सु - वीर्या॑य । \newline
10. मयि॑ प्र॒जाम् प्र॒जाम् मयि॒ मयि॑ प्र॒जाम् । \newline
11. प्र॒जाम् मयि॒ मयि॑ प्र॒जाम् प्र॒जाम् मयि॑ । \newline
12. प्र॒जामिति॑ प्र - जाम् । \newline
13. मयि॒ वर्चो॒ वर्चो॒ मयि॒ मयि॒ वर्चः॑ । \newline
14. वर्चो॑ दधामि दधामि॒ वर्चो॒ वर्चो॑ दधामि । \newline
15. द॒धा॒ म्यरि॑ष्टा॒ अरि॑ष्टा दधामि दधा॒ म्यरि॑ष्टाः । \newline
16. अरि॑ष्टाः स्याम स्या॒मा रि॑ष्टा॒ अरि॑ष्टाः स्याम । \newline
17. स्या॒म॒ त॒नुवा॑ त॒नुवा᳚ स्याम स्याम त॒नुवा᳚ । \newline
18. त॒नुवा॑ सु॒वीराः᳚ सु॒वीरा᳚ स्त॒नुवा॑ त॒नुवा॑ सु॒वीराः᳚ । \newline
19. सु॒वीरा॒ इति॑ सु - वीराः᳚ । \newline
20. यो नो॑ नो॒ यो यो नः॑ । \newline
21. नो॒ अ॒ग्नि र॒ग्निर् नो॑ नो अ॒ग्निः । \newline
22. अ॒ग्निः पि॑तरः पितरो॒ ऽग्नि र॒ग्निः पि॑तरः । \newline
23. पि॒त॒रो॒ हृ॒थ्सु हृ॒थ्सु पि॑तरः पितरो हृ॒थ्सु । \newline
24. हृ॒थ् स्व॑न्त र॒न्तर् हृ॒थ्सु हृ॒थ् स्व॑न्तः । \newline
25. हृ॒थ्स्विति॑ हृत् - सु । \newline
26. अ॒न्त रम॒र्त्यो ऽम॑र्त्यो॒ ऽन्त र॒न्त रम॑र्त्यः । \newline
27. अम॑र्त्यो॒ मर्त्या॒न् मर्त्याꣳ॒॒ अम॒र्त्यो ऽम॑र्त्यो॒ मर्त्यान्॑ । \newline
28. मर्त्याꣳ॑ आवि॒वेशा॑ वि॒वेश॒ मर्त्या॒न् मर्त्याꣳ॑ आवि॒वेश॑ । \newline
29. आ॒वि॒वेशेत्या᳚ - वि॒वेश॑ । \newline
30. त मा॒त्मन् ना॒त्मन् तम् त मा॒त्मन्न् । \newline
31. आ॒त्मन् परि॒ पर्या॒त्मन् ना॒त्मन् परि॑ । \newline
32. परि॑ गृह्णीमहे गृह्णीमहे॒ परि॒ परि॑ गृह्णीमहे । \newline
33. गृ॒ह्णी॒म॒हे॒ व॒यं ॅव॒यम् गृ॑ह्णीमहे गृह्णीमहे व॒यम् । \newline
34. व॒यम् मा मा व॒यं ॅव॒यम् मा । \newline
35. मा स स मा मा सः । \newline
36. सो अ॒स्माꣳ अ॒स्मान् थ्स सो अ॒स्मान् । \newline
37. अ॒स्माꣳ अ॑व॒हाया॑ व॒हाया॒स्माꣳ अ॒स्माꣳ अ॑व॒हाय॑ । \newline
38. अ॒व॒हाय॒ परा॒ परा॑ ऽव॒हाया॑ व॒हाय॒ परा᳚ । \newline
39. अ॒व॒हायेत्य॑व - हाय॑ । \newline
40. परा॑ गाद् गा॒त् परा॒ परा॑ गात् । \newline
41. गा॒दिति॑ गात् । \newline
42. यद॑द्ध्व॒र्यु र॑द्ध्व॒र्युर् यद् यद॑द्ध्व॒र्युः । \newline
43. अ॒द्ध्व॒र्यु रा॒त्मन् ना॒त्मन् न॑द्ध्व॒र्यु र॑द्ध्व॒र्यु रा॒त्मन्न् । \newline
44. आ॒त्मन् न॒ग्नि म॒ग्नि मा॒त्मन् ना॒त्मन् न॒ग्निम् । \newline
45. अ॒ग्नि मगृ॑ही॒त्वा ऽगृ॑हीत्वा॒ ऽग्नि म॒ग्नि मगृ॑हीत्वा । \newline
46. अगृ॑हीत्वा॒ ऽग्नि म॒ग्नि मगृ॑ही॒त्वा ऽगृ॑हीत्वा॒ ऽग्निम् । \newline
47. अ॒ग्निम् चि॑नु॒याच् चि॑नु॒या द॒ग्नि म॒ग्निम् चि॑नु॒यात् । \newline
48. चि॒नु॒याद् यो यश्चि॑नु॒याच् चि॑नु॒याद् यः । \newline
49. यो᳚ ऽस्यास्य॒ यो यो᳚ ऽस्य । \newline
50. अ॒स्य॒ स्वः स्वो᳚ ऽस्यास्य॒ स्वः । \newline
51. स्वो᳚ ऽग्नि र॒ग्निः स्वः स्वो᳚ ऽग्निः । \newline
52. अ॒ग्नि स्तम् त म॒ग्नि र॒ग्नि स्तम् । \newline
53. त मप्यपि॒ तम् त मपि॑ । \newline
54. अपि॒ यज॑मानाय॒ यज॑माना॒या प्यपि॒ यज॑मानाय । \newline

\textbf{Ghana Paata } \newline

1. मयि॑ गृह्णामि गृह्णामि॒ मयि॒ मयि॑ गृह्णा॒ म्यग्रे॒ अग्रे॑ गृह्णामि॒ मयि॒ मयि॑ गृह्णा॒ म्यग्रे᳚ । \newline
2. गृ॒ह्णा॒ म्यग्रे॒ अग्रे॑ गृह्णामि गृह्णा॒ म्यग्रे॑ अ॒ग्नि म॒ग्नि मग्रे॑ गृह्णामि गृह्णा॒ म्यग्रे॑ अ॒ग्निम् । \newline
3. अग्रे॑ अ॒ग्नि म॒ग्नि मग्रे ऽग्रे॑ अ॒ग्निꣳ रा॒यो रा॒यो᳚ ऽग्नि मग्रे ऽग्रे॑ अ॒ग्निꣳ रा॒यः । \newline
4. अ॒ग्निꣳ रा॒यो रा॒यो᳚ ऽग्नि म॒ग्निꣳ रा॒य स्पोषा॑य॒ पोषा॑य रा॒यो᳚ ऽग्नि म॒ग्निꣳ रा॒य स्पोषा॑य । \newline
5. रा॒य स्पोषा॑य॒ पोषा॑य रा॒यो रा॒य स्पोषा॑य सुप्रजा॒स्त्वाय॑ सुप्रजा॒स्त्वाय॒ पोषा॑य रा॒यो रा॒य स्पोषा॑य सुप्रजा॒स्त्वाय॑ । \newline
6. पोषा॑य सुप्रजा॒स्त्वाय॑ सुप्रजा॒स्त्वाय॒ पोषा॑य॒ पोषा॑य सुप्रजा॒स्त्वाय॑ सु॒वीर्या॑य सु॒वीर्या॑य सुप्रजा॒स्त्वाय॒ पोषा॑य॒ पोषा॑य सुप्रजा॒स्त्वाय॑ सु॒वीर्या॑य । \newline
7. सु॒प्र॒जा॒स्त्वाय॑ सु॒वीर्या॑य सु॒वीर्या॑य सुप्रजा॒स्त्वाय॑ सुप्रजा॒स्त्वाय॑ सु॒वीर्या॑य । \newline
8. सु॒प्र॒जा॒स्त्वायेति॑ सुप्रजाः - त्वाय॑ । \newline
9. सु॒वीर्या॒येति॑ सु - वीर्या॑य । \newline
10. मयि॑ प्र॒जाम् प्र॒जाम् मयि॒ मयि॑ प्र॒जाम् मयि॒ मयि॑ प्र॒जाम् मयि॒ मयि॑ प्र॒जाम् मयि॑ । \newline
11. प्र॒जाम् मयि॒ मयि॑ प्र॒जाम् प्र॒जाम् मयि॒ वर्चो॒ वर्चो॒ मयि॑ प्र॒जाम् प्र॒जाम् मयि॒ वर्चः॑ । \newline
12. प्र॒जामिति॑ प्र - जाम् । \newline
13. मयि॒ वर्चो॒ वर्चो॒ मयि॒ मयि॒ वर्चो॑ दधामि दधामि॒ वर्चो॒ मयि॒ मयि॒ वर्चो॑ दधामि । \newline
14. वर्चो॑ दधामि दधामि॒ वर्चो॒ वर्चो॑ दधा॒ म्यरि॑ष्टा॒ अरि॑ष्टा दधामि॒ वर्चो॒ वर्चो॑ दधा॒ म्यरि॑ष्टाः । \newline
15. द॒धा॒ म्यरि॑ष्टा॒ अरि॑ष्टा दधामि दधा॒ म्यरि॑ष्टाः स्याम स्या॒मारि॑ष्टा दधामि दधा॒ म्यरि॑ष्टाः स्याम । \newline
16. अरि॑ष्टाः स्याम स्या॒मारि॑ष्टा॒ अरि॑ष्टाः स्याम त॒नुवा॑ त॒नुवा᳚ स्या॒मारि॑ष्टा॒ अरि॑ष्टाः स्याम त॒नुवा᳚ । \newline
17. स्या॒म॒ त॒नुवा॑ त॒नुवा᳚ स्याम स्याम त॒नुवा॑ सु॒वीराः᳚ सु॒वीरा᳚ स्त॒नुवा᳚ स्याम स्याम त॒नुवा॑ सु॒वीराः᳚ । \newline
18. त॒नुवा॑ सु॒वीराः᳚ सु॒वीरा᳚ स्त॒नुवा॑ त॒नुवा॑ सु॒वीराः᳚ । \newline
19. सु॒वीरा॒ इति॑ सु - वीराः᳚ । \newline
20. यो नो॑ नो॒ यो यो नो॑ अ॒ग्नि र॒ग्निर् नो॒ यो यो नो॑ अ॒ग्निः । \newline
21. नो॒ अ॒ग्नि र॒ग्निर् नो॑ नो अ॒ग्निः पि॑तरः पितरो॒ ऽग्निर् नो॑ नो अ॒ग्निः पि॑तरः । \newline
22. अ॒ग्निः पि॑तरः पितरो॒ ऽग्नि र॒ग्निः पि॑तरो हृ॒थ्सु हृ॒थ्सु पि॑तरो॒ ऽग्नि र॒ग्निः पि॑तरो हृ॒थ्सु । \newline
23. पि॒त॒रो॒ हृ॒थ्सु हृ॒थ्सु पि॑तरः पितरो हृ॒थ्स्व॑न्त र॒न्तर् हृ॒थ्सु पि॑तरः पितरो हृ॒थ्स्व॑न्तः । \newline
24. हृ॒थ्स्व॑न्त र॒न्तर् हृ॒थ्सु हृ॒थ्स्व॑न्त रम॒र्त्यो ऽम॑र्त्यो॒ ऽन्तर् हृ॒थ्सु हृ॒थ्स्व॑न्त रम॑र्त्यः । \newline
25. हृ॒थ्स्विति॑ हृत् - सु । \newline
26. अ॒न्त रम॒र्त्यो ऽम॑र्त्यो॒ ऽन्त र॒न्त रम॑र्त्यो॒ मर्त्या॒न् मर्त्याꣳ॒॒ अम॑र्त्यो॒ ऽन्त र॒न्त रम॑र्त्यो॒ मर्त्यान्॑ । \newline
27. अम॑र्त्यो॒ मर्त्या॒न् मर्त्याꣳ॒॒ अम॒र्त्यो ऽम॑र्त्यो॒ मर्त्याꣳ॑ आवि॒वेशा॑ वि॒वेश॒ मर्त्याꣳ॒॒ अम॒र्त्यो ऽम॑र्त्यो॒ मर्त्याꣳ॑ आवि॒वेश॑ । \newline
28. मर्त्याꣳ॑ आवि॒वेशा॑ वि॒वेश॒ मर्त्या॒न् मर्त्याꣳ॑ आवि॒वेश॑ । \newline
29. आ॒वि॒वेशेत्या᳚ - वि॒वेश॑ । \newline
30. त मा॒त्मन् ना॒त्मन् तम् त मा॒त्मन् परि॒ पर्या॒त्मन् तम् त मा॒त्मन् परि॑ । \newline
31. आ॒त्मन् परि॒ पर्या॒त्मन् ना॒त्मन् परि॑ गृह्णीमहे गृह्णीमहे॒ पर्या॒त्मन् ना॒त्मन् परि॑ गृह्णीमहे । \newline
32. परि॑ गृह्णीमहे गृह्णीमहे॒ परि॒ परि॑ गृह्णीमहे व॒यं ॅव॒यम् गृ॑ह्णीमहे॒ परि॒ परि॑ गृह्णीमहे व॒यम् । \newline
33. गृ॒ह्णी॒म॒हे॒ व॒यं ॅव॒यम् गृ॑ह्णीमहे गृह्णीमहे व॒यम् मा मा व॒यम् गृ॑ह्णीमहे गृह्णीमहे व॒यम् मा । \newline
34. व॒यम् मा मा व॒यं ॅव॒यम् मा स स मा व॒यं ॅव॒यम् मा सः । \newline
35. मा स स मा मा सो अ॒स्माꣳ अ॒स्मान् थ्स मा मा सो अ॒स्मान् । \newline
36. सो अ॒स्माꣳ अ॒स्मान् थ्स सो अ॒स्माꣳ अ॑व॒हाया॑ व॒हाया॒स्मान् थ्स सो अ॒स्माꣳ अ॑व॒हाय॑ । \newline
37. अ॒स्माꣳ अ॑व॒हाया॑ व॒हाया॒स्माꣳ अ॒स्माꣳ अ॑व॒हाय॒ परा॒ परा॑ ऽव॒हाया॒स्माꣳ अ॒स्माꣳ अ॑व॒हाय॒ परा᳚ । \newline
38. अ॒व॒हाय॒ परा॒ परा॑ ऽव॒हाया॑ व॒हाय॒ परा॑ गाद् गा॒त् परा॑ ऽव॒हाया॑ व॒हाय॒ परा॑ गात् । \newline
39. अ॒व॒हायेत्य॑व - हाय॑ । \newline
40. परा॑ गाद् गा॒त् परा॒ परा॑ गात् । \newline
41. गा॒दिति॑ गात् । \newline
42. यद॑द्ध्व॒र्यु र॑द्ध्व॒र्युर् यद् यद॑द्ध्व॒र्यु रा॒त्मन् ना॒त्मन् न॑द्ध्व॒र्युर् यद् यद॑द्ध्व॒र्यु रा॒त्मन्न् । \newline
43. अ॒द्ध्व॒र्यु रा॒त्मन् ना॒त्मन् न॑द्ध्व॒र्यु र॑द्ध्व॒र्यु रा॒त्मन् न॒ग्नि म॒ग्नि मा॒त्मन् न॑द्ध्व॒र्यु र॑द्ध्व॒र्यु रा॒त्मन् न॒ग्निम् । \newline
44. आ॒त्मन् न॒ग्नि म॒ग्नि मा॒त्मन् ना॒त्मन् न॒ग्नि मगृ॑ही॒त्वा ऽगृ॑हीत्वा॒ ऽग्नि मा॒त्मन् ना॒त्मन् न॒ग्नि मगृ॑हीत्वा । \newline
45. अ॒ग्नि मगृ॑ही॒त्वा ऽगृ॑हीत्वा॒ ऽग्नि म॒ग्नि मगृ॑हीत्वा॒ ऽग्नि म॒ग्नि मगृ॑हीत्वा॒ ऽग्नि म॒ग्नि मगृ॑हीत्वा॒ ऽग्निम् । \newline
46. अगृ॑हीत्वा॒ ऽग्नि म॒ग्नि मगृ॑ही॒त्वा ऽगृ॑हीत्वा॒ ऽग्निम् चि॑नु॒याच् चि॑नु॒या द॒ग्नि मगृ॑ही॒त्वा ऽगृ॑हीत्वा॒ ऽग्निम् चि॑नु॒यात् । \newline
47. अ॒ग्निम् चि॑नु॒याच् चि॑नु॒या द॒ग्नि म॒ग्निम् चि॑नु॒याद् यो यश्चि॑नु॒या द॒ग्नि म॒ग्निम् चि॑नु॒याद् यः । \newline
48. चि॒नु॒याद् यो यश्चि॑नु॒याच् चि॑नु॒याद् यो᳚ ऽस्यास्य॒ यश्चि॑नु॒याच् चि॑नु॒याद् यो᳚ ऽस्य । \newline
49. यो᳚ ऽस्यास्य॒ यो यो᳚ ऽस्य॒ स्वः स्वो᳚ ऽस्य॒ यो यो᳚ ऽस्य॒ स्वः । \newline
50. अ॒स्य॒ स्वः स्वो᳚ ऽस्यास्य॒ स्वो᳚ ऽग्नि र॒ग्निः स्वो᳚ ऽस्यास्य॒ स्वो᳚ ऽग्निः । \newline
51. स्वो᳚ ऽग्नि र॒ग्निः स्वः स्वो᳚ ऽग्नि स्तम् त म॒ग्निः स्वः स्वो᳚ ऽग्नि स्तम् । \newline
52. अ॒ग्नि स्तम् त म॒ग्नि र॒ग्नि स्त मप्यपि॒ त म॒ग्नि र॒ग्नि स्त मपि॑ । \newline
53. त मप्यपि॒ तम् त मपि॒ यज॑मानाय॒ यज॑माना॒यापि॒ तम् त मपि॒ यज॑मानाय । \newline
54. अपि॒ यज॑मानाय॒ यज॑माना॒या प्यपि॒ यज॑मानाय चिनुयाच् चिनुया॒द् यज॑माना॒या प्यपि॒ यज॑मानाय चिनुयात् । \newline
\pagebreak
\markright{ TS 5.7.9.2  \hfill https://www.vedavms.in \hfill}

\section{ TS 5.7.9.2 }

\textbf{TS 5.7.9.2 } \newline
\textbf{Samhita Paata} \newline

यज॑मानाय चिनुयाद॒ग्निं खलु॒ वै प॒शवोऽनूप॑ तिष्ठन्तेऽप॒क्रामु॑का अस्मात् प॒शवः॑ स्यु॒र्मयि॑ गृह्णा॒म्यग्रे॑ अ॒ग्निमित्या॑हा॒ऽऽत्मन्ने॒व स्वम॒ग्निं दा॑धार॒ नास्मा᳚त् प॒शवोऽप॑ क्रामन्ति ब्रह्मवा॒दिनो॑ वदन्ति॒ यन्मृच्चाऽऽप॑श्चा॒ग्नेर॑ना॒द्यमथ॒ कस्मा᳚न्मृ॒दा चा॒द्भिश्चा॒ग्निश्ची॑यत॒ इति॒ यद॒द्भिः सं॒ ॅयौ - [  ] \newline

\textbf{Pada Paata} \newline

यज॑मानाय । चि॒नु॒या॒त् । अ॒ग्निम् । खलु॑ । वै । प॒शवः॑ । अनु॑ । उपेति॑ । ति॒ष्ठ॒न्ते॒ । अ॒प॒क्रामु॑का॒ इत्य॑प - क्रामु॑काः । अ॒स्मा॒त् । प॒शवः॑ । स्युः॒ । मयि॑ । गृ॒ह्णा॒मि॒ । अग्रे᳚ । अ॒ग्निम् । इति॑ । आ॒ह॒ । आ॒त्मन्न् । ए॒व । स्वम् । अ॒ग्निम् । दा॒धा॒र॒ । न । अ॒स्मा॒त् । प॒शवः॑ । अपेति॑ । क्रा॒म॒न्ति॒ । ब्र॒ह्म॒वा॒दिन॒ इति॑ ब्रह्म-वा॒दिनः॑ । व॒द॒न्ति॒ । यत् । मृत् । च॒ । आपः॑ । च॒ । अ॒ग्नेः । अ॒ना॒द्यम् । अथ॑ । कस्मा᳚त् । मृ॒दा । च॒ । अ॒द्भिरित्य॑त्-भिः । च॒ । अ॒ग्निः । ची॒य॒ते॒ । इति॑ । यत् । अ॒द्भिरित्य॑त् - भिः । सं॒ॅयौतीति॑ सं - यौति॑ ।  \newline


\textbf{Krama Paata} \newline

यज॑मानाय चिनुयात् । चि॒नु॒या॒द॒ग्निम् । अ॒ग्निम् खलु॑ । खलु॒ वै । वै प॒शवः॑ । प॒शवोऽनु॑ । अनूप॑ । उप॑ तिष्ठन्ते । ति॒ष्ठ॒न्ते॒ऽप॒क्रामु॑काः । अ॒प॒क्रामु॑का अस्मात् । अ॒प॒क्रामु॑का॒ इत्य॑प - क्रामु॑काः । अ॒स्मा॒त् प॒शवः॑ । प॒शवः॑ स्युः । स्यु॒र् मयि॑ । मयि॑ गृह्णामि । गृ॒ह्णा॒म्यग्रे᳚ । अग्रे॑ अ॒ग्निम् । अ॒ग्निमिति॑ । इत्या॑ह । आ॒हा॒त्मन्न् । आ॒त्मन्ने॒व । ए॒व स्वम् । स्वम॒ग्निम् । अ॒ग्निम् दा॑धार । दा॒धा॒र॒ न । नास्मा᳚त् । अ॒स्मा॒त् प॒शवः॑ । प॒शवोऽप॑ । अप॑ क्रामन्ति । क्रा॒म॒न्ति॒ ब्र॒ह्म॒वा॒दिनः॑ । ब्र॒ह्म॒वा॒दिनो॑ वदन्ति । ब्र॒ह्म॒वा॒दिन॒ इति॑ ब्रह्म - वा॒दिनः॑ । व॒द॒न्ति॒ यत् । यन् मृत् । मृच् च॑ । चापः॑ । आप॑श्च । चा॒ग्नेः । अ॒ग्नेर॑ना॒द्यम् । अ॒ना॒द्यमथ॑ । अथ॒ कस्मा᳚त् । कस्मा᳚न् मृ॒दा । मृ॒दा च॑ । चा॒द्भिः । अ॒द्भिश्च॑ । अ॒द्भिरित्य॑त् - भिः । चा॒ग्निः । अ॒ग्निश्ची॑यते । ची॒य॒त॒ इति॑ । इति॒ यत् । यद॒द्भिः । अ॒द्भिः स॒म्ॅयौति॑ । अ॒द्भिरित्य॑त् - भिः । स॒म्ॅयौत्यापः॑ । स॒म्ॅयौतीति॑ सम् - यौति॑ \newline

\textbf{Jatai Paata} \newline

1. यज॑मानाय चिनुयाच् चिनुया॒द् यज॑मानाय॒ यज॑मानाय चिनुयात् । \newline
2. चि॒नु॒या॒ द॒ग्नि म॒ग्निम् चि॑नुयाच् चिनुया द॒ग्निम् । \newline
3. अ॒ग्निम् खलु॒ खल्व॒ग्नि म॒ग्निम् खलु॑ । \newline
4. खलु॒ वै वै खलु॒ खलु॒ वै । \newline
5. वै प॒शवः॑ प॒शवो॒ वै वै प॒शवः॑ । \newline
6. प॒शवो ऽन्वनु॑ प॒शवः॑ प॒शवो ऽनु॑ । \newline
7. अनूपोपा न्वनूप॑ । \newline
8. उप॑ तिष्ठन्ते तिष्ठन्त॒ उपोप॑ तिष्ठन्ते । \newline
9. ति॒ष्ठ॒न्ते॒ ऽप॒क्रामु॑का अप॒क्रामु॑का स्तिष्ठन्ते तिष्ठन्ते ऽप॒क्रामु॑काः । \newline
10. अ॒प॒क्रामु॑का अस्मा दस्मा दप॒क्रामु॑का अप॒क्रामु॑का अस्मात् । \newline
11. अ॒प॒क्रामु॑का॒ इत्य॑प - क्रामु॑काः । \newline
12. अ॒स्मा॒त् प॒शवः॑ प॒शवो᳚ ऽस्मा दस्मात् प॒शवः॑ । \newline
13. प॒शवः॑ स्युः स्युः प॒शवः॑ प॒शवः॑ स्युः । \newline
14. स्यु॒र् मयि॒ मयि॑ स्युः स्यु॒र् मयि॑ । \newline
15. मयि॑ गृह्णामि गृह्णामि॒ मयि॒ मयि॑ गृह्णामि । \newline
16. गृ॒ह्णा॒ म्यग्रे॒ अग्रे॑ गृह्णामि गृह्णा॒ म्यग्रे᳚ । \newline
17. अग्रे᳚ ऽग्नि म॒ग्नि मग्रे ऽग्रे॑ अ॒ग्निम् । \newline
18. अ॒ग्नि मितीत्य॒ग्नि म॒ग्नि मिति॑ । \newline
19. इत्या॑हा॒हे तीत्या॑ह । \newline
20. आ॒हा॒त्मन् ना॒त्मन् ना॑हा हा॒त्मन्न् । \newline
21. आ॒त्मन् ने॒वै वात्मन् ना॒त्मन् ने॒व । \newline
22. ए॒व स्वꣳ स्व मे॒वैव स्वम् । \newline
23. स्व म॒ग्नि म॒ग्निꣳ स्वꣳ स्व म॒ग्निम् । \newline
24. अ॒ग्निम् दा॑धार दाधारा॒ग्नि म॒ग्निम् दा॑धार । \newline
25. दा॒धा॒र॒ न न दा॑धार दाधार॒ न । \newline
26. नास्मा॑ दस्मा॒न् न नास्मा᳚त् । \newline
27. अ॒स्मा॒त् प॒शवः॑ प॒शवो᳚ ऽस्मा दस्मात् प॒शवः॑ । \newline
28. प॒शवो ऽपाप॑ प॒शवः॑ प॒शवो ऽप॑ । \newline
29. अप॑ क्रामन्ति क्राम॒न् त्यपाप॑ क्रामन्ति । \newline
30. क्रा॒म॒न्ति॒ ब्र॒ह्म॒वा॒दिनो᳚ ब्रह्मवा॒दिनः॑ क्रामन्ति क्रामन्ति ब्रह्मवा॒दिनः॑ । \newline
31. ब्र॒ह्म॒वा॒दिनो॑ वदन्ति वदन्ति ब्रह्मवा॒दिनो᳚ ब्रह्मवा॒दिनो॑ वदन्ति । \newline
32. ब्र॒ह्म॒वा॒दिन॒ इति॑ ब्रह्म - वा॒दिनः॑ । \newline
33. व॒द॒न्ति॒ यद् यद् व॑दन्ति वदन्ति॒ यत् । \newline
34. यन् मृण् मृद् यद् यन् मृत् । \newline
35. मृच् च॑ च॒ मृण् मृच् च॑ । \newline
36. चाप॒ आप॑ श्च॒ चापः॑ । \newline
37. आप॑ श्च॒ चाप॒ आप॑ श्च । \newline
38. चा॒ग्ने र॒ग्ने श्च॑ चा॒ग्नेः । \newline
39. अ॒ग्ने र॑ना॒द्य म॑ना॒द्य म॒ग्ने र॒ग्ने र॑ना॒द्यम् । \newline
40. अ॒ना॒द्य मथाथा॑ ना॒द्य म॑ना॒द्य मथ॑ । \newline
41. अथ॒ कस्मा॒त् कस्मा॒ दथाथ॒ कस्मा᳚त् । \newline
42. कस्मा᳚न् मृ॒दा मृ॒दा कस्मा॒त् कस्मा᳚न् मृ॒दा । \newline
43. मृ॒दा च॑ च मृ॒दा मृ॒दा च॑ । \newline
44. चा॒द्भि र॒द्भि श्च॑ चा॒द्भिः । \newline
45. अ॒द्भि श्च॑ चा॒द्भि र॒द्भि श्च॑ । \newline
46. अ॒द्भिरित्य॑त् - भिः । \newline
47. चा॒ग्नि र॒ग्नि श्च॑ चा॒ग्निः । \newline
48. अ॒ग्नि श्ची॑यते चीयते॒ ऽग्नि र॒ग्नि श्ची॑यते । \newline
49. ची॒य॒त॒ इतीति॑ चीयते चीयत॒ इति॑ । \newline
50. इति॒ यद् यदितीति॒ यत् । \newline
51. यद॒द्भि र॒द्भिर् यद् यद॒द्भिः । \newline
52. अ॒द्भिः सं॒ॅयौति॑ सं॒ॅयौ त्य॒द्भि र॒द्भिः सं॒ॅयौति॑ । \newline
53. अ॒द्भिरित्य॑त् - भिः । \newline
54. सं॒ॅयौ त्याप॒ आपः॑ सं॒ॅयौति॑ सं॒ॅयौ त्यापः॑ । \newline
55. सं॒ॅयौतीति॑ सं - यौति॑ । \newline

\textbf{Ghana Paata } \newline

1. यज॑मानाय चिनुयाच् चिनुया॒द् यज॑मानाय॒ यज॑मानाय चिनुया द॒ग्नि म॒ग्निम् चि॑नुया॒द् यज॑मानाय॒ यज॑मानाय चिनुया द॒ग्निम् । \newline
2. चि॒नु॒या॒ द॒ग्नि म॒ग्निम् चि॑नुयाच् चिनुया द॒ग्निम् खलु॒ खल्व॒ग्निम् चि॑नुयाच् चिनुया द॒ग्निम् खलु॑ । \newline
3. अ॒ग्निम् खलु॒ खल्व॒ग्नि म॒ग्निम् खलु॒ वै वै खल्व॒ग्नि म॒ग्निम् खलु॒ वै । \newline
4. खलु॒ वै वै खलु॒ खलु॒ वै प॒शवः॑ प॒शवो॒ वै खलु॒ खलु॒ वै प॒शवः॑ । \newline
5. वै प॒शवः॑ प॒शवो॒ वै वै प॒शवो ऽन्वनु॑ प॒शवो॒ वै वै प॒शवो ऽनु॑ । \newline
6. प॒शवो ऽन्वनु॑ प॒शवः॑ प॒शवो ऽनूपो पानु॑ प॒शवः॑ प॒शवो ऽनूप॑ । \newline
7. अनूपोपान् वनूप॑ तिष्ठन्ते तिष्ठन्त॒ उपान् वनूप॑ तिष्ठन्ते । \newline
8. उप॑ तिष्ठन्ते तिष्ठन्त॒ उपोप॑ तिष्ठन्ते ऽप॒क्रामु॑का अप॒क्रामु॑का स्तिष्ठन्त॒ उपोप॑ तिष्ठन्ते ऽप॒क्रामु॑काः । \newline
9. ति॒ष्ठ॒न्ते॒ ऽप॒क्रामु॑का अप॒क्रामु॑का स्तिष्ठन्ते तिष्ठन्ते ऽप॒क्रामु॑का अस्मा दस्मा दप॒क्रामु॑का स्तिष्ठन्ते तिष्ठन्ते ऽप॒क्रामु॑का अस्मात् । \newline
10. अ॒प॒क्रामु॑का अस्मा दस्मा दप॒क्रामु॑का अप॒क्रामु॑का अस्मात् प॒शवः॑ प॒शवो᳚ ऽस्मा दप॒क्रामु॑का अप॒क्रामु॑का अस्मात् प॒शवः॑ । \newline
11. अ॒प॒क्रामु॑का॒ इत्य॑प - क्रामु॑काः । \newline
12. अ॒स्मा॒त् प॒शवः॑ प॒शवो᳚ ऽस्मा दस्मात् प॒शवः॑ स्युः स्युः प॒शवो᳚ ऽस्मा दस्मात् प॒शवः॑ स्युः । \newline
13. प॒शवः॑ स्युः स्युः प॒शवः॑ प॒शवः॑ स्यु॒र् मयि॒ मयि॑ स्युः प॒शवः॑ प॒शवः॑ स्यु॒र् मयि॑ । \newline
14. स्यु॒र् मयि॒ मयि॑ स्युः स्यु॒र् मयि॑ गृह्णामि गृह्णामि॒ मयि॑ स्युः स्यु॒र् मयि॑ गृह्णामि । \newline
15. मयि॑ गृह्णामि गृह्णामि॒ मयि॒ मयि॑ गृह्णा॒ म्यग्रे॒ अग्रे॑ गृह्णामि॒ मयि॒ मयि॑ गृह्णा॒ म्यग्रे᳚ । \newline
16. गृ॒ह्णा॒ म्यग्रे॒ अग्रे॑ गृह्णामि गृह्णा॒ म्यग्रे॑ अ॒ग्नि म॒ग्नि मग्रे॑ गृह्णामि गृह्णा॒ म्यग्रे॑ अ॒ग्निम् । \newline
17. अग्रे॑ अ॒ग्नि म॒ग्नि मग्रे ऽग्रे॑ अ॒ग्नि मिती त्य॒ग्नि मग्रे ऽग्रे॑ अ॒ग्नि मिति॑ । \newline
18. अ॒ग्नि मिती त्य॒ग्नि म॒ग्नि मित्या॑हा॒हे त्य॒ग्नि म॒ग्नि मित्या॑ह । \newline
19. इत्या॑हा॒हे तीत्या॑ हा॒त्मन् ना॒त्मन् ना॒हे तीत्या॑ हा॒त्मन्न् । \newline
20. आ॒हा॒त्मन् ना॒त्मन् ना॑हा हा॒त्मन् ने॒वै वात्मन् ना॑हा हा॒त्मन् ने॒व । \newline
21. आ॒त्मन् ने॒वै वात्मन् ना॒त्मन् ने॒व स्वꣳ स्व मे॒वात्मन् ना॒त्मन् ने॒व स्वम् । \newline
22. ए॒व स्वꣳ स्व मे॒वैव स्व म॒ग्नि म॒ग्निꣳ स्व मे॒वैव स्व म॒ग्निम् । \newline
23. स्व म॒ग्नि म॒ग्निꣳ स्वꣳ स्व म॒ग्निम् दा॑धार दाधा रा॒ग्निꣳ स्वꣳ स्व म॒ग्निम् दा॑धार । \newline
24. अ॒ग्निम् दा॑धार दाधा रा॒ग्नि म॒ग्निम् दा॑धार॒ न न दा॑धारा॒ग्नि म॒ग्निम् दा॑धार॒ न । \newline
25. दा॒धा॒र॒ न न दा॑धार दाधार॒ नास्मा॑ दस्मा॒न् न दा॑धार दाधार॒ नास्मा᳚त् । \newline
26. नास्मा॑ दस्मा॒न् न नास्मा᳚त् प॒शवः॑ प॒शवो᳚ ऽस्मा॒न् न नास्मा᳚त् प॒शवः॑ । \newline
27. अ॒स्मा॒त् प॒शवः॑ प॒शवो᳚ ऽस्मा दस्मात् प॒शवो ऽपाप॑ प॒शवो᳚ ऽस्मा दस्मात् प॒शवो ऽप॑ । \newline
28. प॒शवो ऽपाप॑ प॒शवः॑ प॒शवो ऽप॑ क्रामन्ति क्राम॒न् त्यप॑ प॒शवः॑ प॒शवो ऽप॑ क्रामन्ति । \newline
29. अप॑ क्रामन्ति क्राम॒न् त्यपाप॑ क्रामन्ति ब्रह्मवा॒दिनो᳚ ब्रह्मवा॒दिनः॑ क्राम॒न् त्यपाप॑ क्रामन्ति ब्रह्मवा॒दिनः॑ । \newline
30. क्रा॒म॒न्ति॒ ब्र॒ह्म॒वा॒दिनो᳚ ब्रह्मवा॒दिनः॑ क्रामन्ति क्रामन्ति ब्रह्मवा॒दिनो॑ वदन्ति वदन्ति ब्रह्मवा॒दिनः॑ क्रामन्ति क्रामन्ति ब्रह्मवा॒दिनो॑ वदन्ति । \newline
31. ब्र॒ह्म॒वा॒दिनो॑ वदन्ति वदन्ति ब्रह्मवा॒दिनो᳚ ब्रह्मवा॒दिनो॑ वदन्ति॒ यद् यद् व॑दन्ति ब्रह्मवा॒दिनो᳚ ब्रह्मवा॒दिनो॑ वदन्ति॒ यत् । \newline
32. ब्र॒ह्म॒वा॒दिन॒ इति॑ ब्रह्म - वा॒दिनः॑ । \newline
33. व॒द॒न्ति॒ यद् यद् व॑दन्ति वदन्ति॒ यन् मृण् मृद् यद् व॑दन्ति वदन्ति॒ यन् मृत् । \newline
34. यन् मृण् मृद् यद् यन् मृच् च॑ च॒ मृद् यद् यन् मृच् च॑ । \newline
35. मृच् च॑ च॒ मृण् मृच् चाप॒ आप॑श्च॒ मृण् मृच् चापः॑ । \newline
36. चाप॒ आप॑श्च॒ चाप॑श्च॒ चाप॑श्च॒ चाप॑श्च । \newline
37. आप॑श्च॒ चाप॒ आप॑ श्चा॒ग्ने र॒ग्ने श्चाप॒ आप॑ श्चा॒ग्नेः । \newline
38. चा॒ग्ने र॒ग्नेश्च॑ चा॒ग्ने र॑ना॒द्य म॑ना॒द्य म॒ग्नेश्च॑ चा॒ग्ने र॑ना॒द्यम् । \newline
39. अ॒ग्ने र॑ना॒द्य म॑ना॒द्य म॒ग्ने र॒ग्ने र॑ना॒द्य मथाथा॑ ना॒द्य म॒ग्ने र॒ग्ने र॑ना॒द्य मथ॑ । \newline
40. अ॒ना॒द्य मथाथा॑ ना॒द्य म॑ना॒द्य मथ॒ कस्मा॒त् कस्मा॒ दथा॑ ना॒द्य म॑ना॒द्य मथ॒ कस्मा᳚त् । \newline
41. अथ॒ कस्मा॒त् कस्मा॒ दथाथ॒ कस्मा᳚न् मृ॒दा मृ॒दा कस्मा॒ दथाथ॒ कस्मा᳚न् मृ॒दा । \newline
42. कस्मा᳚न् मृ॒दा मृ॒दा कस्मा॒त् कस्मा᳚न् मृ॒दा च॑ च मृ॒दा कस्मा॒त् कस्मा᳚न् मृ॒दा च॑ । \newline
43. मृ॒दा च॑ च मृ॒दा मृ॒दा चा॒द्भि र॒द्भिश्च॑ मृ॒दा मृ॒दा चा॒द्भिः । \newline
44. चा॒द्भि र॒द्भिश्च॑ चा॒द्भिश्च॑ चा॒द्भिश्च॑ चा॒द्भिश्च॑ । \newline
45. अ॒द्भिश्च॑ चा॒द्भि र॒द्भि श्चा॒ग्नि र॒ग्नि श्चा॒द्भि र॒द्भि श्चा॒ग्निः । \newline
46. अ॒द्भिरित्य॑त् - भिः । \newline
47. चा॒ग्नि र॒ग्नि श्च॑ चा॒ग्नि श्ची॑यते चीयते॒ ऽग्निश्च॑ चा॒ग्नि श्ची॑यते । \newline
48. अ॒ग्नि श्ची॑यते चीयते॒ ऽग्नि र॒ग्नि श्ची॑यत॒ इतीति॑ चीयते॒ ऽग्नि र॒ग्नि श्ची॑यत॒ इति॑ । \newline
49. ची॒य॒त॒ इतीति॑ चीयते चीयत॒ इति॒ यद् यदिति॑ चीयते चीयत॒ इति॒ यत् । \newline
50. इति॒ यद् यदितीति॒ यद॒द्भि र॒द्भिर् यदितीति॒ यद॒द्भिः । \newline
51. यद॒द्भि र॒द्भिर् यद् यद॒द्भिः सं॒ॅयौति॑ सं॒ॅयौ त्य॒द्भिर् यद् यद॒द्भिः सं॒ॅयौति॑ । \newline
52. अ॒द्भिः सं॒ॅयौति॑ सं॒ॅयौ त्य॒द्भि र॒द्भिः सं॒ॅयौ त्याप॒ आपः॑ सं॒ॅयौ त्य॒द्भि र॒द्भिः सं॒ॅयौ त्यापः॑ । \newline
53. अ॒द्भिरित्य॑त् - भिः । \newline
54. सं॒ॅयौ त्याप॒ आपः॑ सं॒ॅयौति॑ सं॒ॅयौ त्यापो॒ वै वा आपः॑ सं॒ॅयौति॑ सं॒ॅयौ त्यापो॒ वै । \newline
55. सं॒ॅयौतीति॑ सं - यौति॑ । \newline
\pagebreak
\markright{ TS 5.7.9.3  \hfill https://www.vedavms.in \hfill}

\section{ TS 5.7.9.3 }

\textbf{TS 5.7.9.3 } \newline
\textbf{Samhita Paata} \newline

-त्यापो॒ वै सर्वा॑ दे॒वता॑ दे॒वता॑भिरे॒वैनꣳ॒॒ सꣳ सृ॑जति॒ यन्मृ॒दा चि॒नोती॒यं ॅवा अ॒ग्निर्वै᳚श्वान॒रो᳚ऽग्निनै॒व तद॒ग्निं चि॑नोति ब्रह्मवा॒दिनो॑ वदन्ति॒ यन्मृ॒दा चा॒द्भिश्चा॒ग्निश्ची॒यतेऽथ॒ कस्मा॑द॒ग्निरु॑च्यत॒ इति॒ यच्छन्दो॑भि-श्चि॒नोत्य॒ग्नयो॒ वै छन्दाꣳ॑सि॒ तस्मा॑द॒ग्निरु॑च्य॒तेऽथो॑ इ॒यं ॅवा अ॒ग्निर्वै᳚श्वान॒रो य - [  ] \newline

\textbf{Pada Paata} \newline

आपः॑ । वै । सर्वाः᳚ । दे॒वताः᳚ । दे॒वता॑भिः । ए॒व । ए॒न॒म् । समिति॑ । सृ॒ज॒ति॒ । यत् । मृ॒दा । चि॒नोति॑ । इ॒यम् । वै । अ॒ग्निः । वै॒श्वा॒न॒रः । अ॒ग्निना᳚ । ए॒व । तत् । अ॒ग्निम् । चि॒नो॒ति॒ । ब्र॒ह्म॒वा॒दिन॒ इति॑ ब्रह्म - वा॒दिनः॑ । व॒द॒न्ति॒ । यत् । मृ॒दा । च॒ । अ॒द्भिरित्य॑त् - भिः । च॒ । अ॒ग्निः । ची॒यते᳚ । अथ॑ । कस्मा᳚त् । अ॒ग्निः । उ॒च्य॒ते॒ । इति॑ । यत् । छन्दो॑भि॒रिति॒ छन्दः॑ - भिः॒ । चि॒नोति॑ । अ॒ग्नयः॑ । वै । छन्दाꣳ॑सि । तस्मा᳚त् । अ॒ग्निः । उ॒च्य॒ते॒ । अथो॒ इति॑ । इ॒यम् । वै । अ॒ग्निः । वै॒श्वा॒न॒रः॒ । यत् ।  \newline


\textbf{Krama Paata} \newline

आपो॒ वै । वै सर्वाः᳚ । सर्वा॑ दे॒वताः᳚ । दे॒वता॑ दे॒वता॑भिः । दे॒वता॑भिरे॒व । ए॒वैन᳚म् । ए॒नꣳ॒॒ सम् । सꣳ सृ॑जति । सृ॒ज॒ति॒ यत् । यन् मृ॒दा । मृ॒दा चि॒नोति॑ । चि॒नोती॒यम् । इ॒यम् ॅवै । वा अ॒ग्निः । अ॒ग्निर् वै᳚श्वान॒रः । वै॒श्वा॒न॒रो᳚ऽग्निना᳚ । अ॒ग्निनै॒व । ए॒व तत् । तद॒ग्निम् । अ॒ग्निम् चि॑नोति । चि॒नो॒ति॒ ब्र॒ह्म॒वा॒दिनः॑ । ब्र॒ह्म॒वा॒दिनो॑ वदन्ति । ब्र॒ह्म॒वा॒दिन॒ इति॑ ब्रह्म - वा॒दिनः॑ । व॒द॒न्ति॒ यत् । यन् मृ॒दा । मृ॒दा च॑ । चा॒द्भिः । अ॒द्भिश्च॑ । अ॒द्भिरित्य॑त् - भिः । चा॒ग्निः । अ॒ग्निश्ची॒यते᳚ । ची॒यतेऽथ॑ । अथ॒ कस्मा᳚त् । कस्मा॑द॒ग्निः । अ॒ग्निरु॑च्यते । उ॒च्य॒त॒ इति॑ । इति॒ यत् । यच्छन्दो॑भिः । छन्दो॑भिश्चि॒नोति॑ । छन्दो॑भि॒रिति॒ छन्दः॑ - भिः॒ । चि॒नोत्य॒ग्नयः॑ । अ॒ग्नयो॒ वै । वै छन्दाꣳ॑सि । छन्दाꣳ॑सि॒ तस्मा᳚त् । तस्मा॑द॒ग्निः । अ॒ग्निरु॑च्यते । उ॒च्य॒तेऽथो᳚ । अथो॑ इ॒यम् । अथो॒ इत्यथो᳚ । इ॒यम् ॅवै । वा अ॒ग्निः । अ॒ग्निर् वै᳚श्वान॒रः । वै॒श्वा॒न॒रो यत् । यन् मृ॒दा \newline

\textbf{Jatai Paata} \newline

1. आपो॒ वै वा आप॒ आपो॒ वै । \newline
2. वै सर्वाः॒ सर्वा॒ वै वै सर्वाः᳚ । \newline
3. सर्वा॑ दे॒वता॑ दे॒वताः॒ सर्वाः॒ सर्वा॑ दे॒वताः᳚ । \newline
4. दे॒वता॑ दे॒वता॑भिर् दे॒वता॑भिर् दे॒वता॑ दे॒वता॑ दे॒वता॑भिः । \newline
5. दे॒वता॑भि रे॒वैव दे॒वता॑भिर् दे॒वता॑भि रे॒व । \newline
6. ए॒वैन॑ मेन मे॒वै वैन᳚म् । \newline
7. ए॒नꣳ॒॒ सꣳ स मे॑न मेनꣳ॒॒ सम् । \newline
8. सꣳ सृ॑जति सृजति॒ सꣳ सꣳ सृ॑जति । \newline
9. सृ॒ज॒ति॒ यद् यथ् सृ॑जति सृजति॒ यत् । \newline
10. यन् मृ॒दा मृ॒दा यद् यन् मृ॒दा । \newline
11. मृ॒दा चि॒नोति॑ चि॒नोति॑ मृ॒दा मृ॒दा चि॒नोति॑ । \newline
12. चि॒नोती॒य मि॒यम् चि॒नोति॑ चि॒नोती॒यम् । \newline
13. इ॒यं ॅवै वा इ॒य मि॒यं ॅवै । \newline
14. वा अ॒ग्नि र॒ग्निर् वै वा अ॒ग्निः । \newline
15. अ॒ग्निर् वै᳚श्वान॒रो वै᳚श्वान॒रो᳚ ऽग्नि र॒ग्निर् वै᳚श्वान॒रः । \newline
16. वै॒श्वा॒न॒रो᳚ ऽग्निना॒ ऽग्निना॑ वैश्वान॒रो वै᳚श्वान॒रो᳚ ऽग्निना᳚ । \newline
17. अ॒ग्नि नै॒वै वाग्निना॒ ऽग्निनै॒व । \newline
18. ए॒व तत् तदे॒ वैव तत् । \newline
19. तद॒ग्नि म॒ग्निम् तत् तद॒ग्निम् । \newline
20. अ॒ग्निम् चि॑नोति चिनो त्य॒ग्नि म॒ग्निम् चि॑नोति । \newline
21. चि॒नो॒ति॒ ब्र॒ह्म॒वा॒दिनो᳚ ब्रह्मवा॒दिन॑ श्चिनोति चिनोति ब्रह्मवा॒दिनः॑ । \newline
22. ब्र॒ह्म॒वा॒दिनो॑ वदन्ति वदन्ति ब्रह्मवा॒दिनो᳚ ब्रह्मवा॒दिनो॑ वदन्ति । \newline
23. ब्र॒ह्म॒वा॒दिन॒ इति॑ ब्रह्म - वा॒दिनः॑ । \newline
24. व॒द॒न्ति॒ यद् यद् व॑दन्ति वदन्ति॒ यत् । \newline
25. यन् मृ॒दा मृ॒दा यद् यन् मृ॒दा । \newline
26. मृ॒दा च॑ च मृ॒दा मृ॒दा च॑ । \newline
27. चा॒द्भि र॒द्भि श्च॑ चा॒द्भिः । \newline
28. अ॒द्भि श्च॑ चा॒द्भि र॒द्भि श्च॑ । \newline
29. अ॒द्भिरित्य॑त् - भिः । \newline
30. चा॒ग्नि र॒ग्नि श्च॑ चा॒ग्निः । \newline
31. अ॒ग्नि श्ची॒यते॑ ची॒यते॒ ऽग्नि र॒ग्नि श्ची॒यते᳚ । \newline
32. ची॒यते ऽथाथ॑ ची॒यते॑ ची॒यते ऽथ॑ । \newline
33. अथ॒ कस्मा॒त् कस्मा॒ दथाथ॒ कस्मा᳚त् । \newline
34. कस्मा॑ द॒ग्नि र॒ग्निः कस्मा॒त् कस्मा॑ द॒ग्निः । \newline
35. अ॒ग्नि रु॑च्यत उच्यते॒ ऽग्नि र॒ग्नि रु॑च्यते । \newline
36. उ॒च्य॒त॒ इती त्यु॑च्यत उच्यत॒ इति॑ । \newline
37. इति॒ यद् यदितीति॒ यत् । \newline
38. यच् छन्दो॑भि॒ श्छन्दो॑भि॒र् यद् यच् छन्दो॑भिः । \newline
39. छन्दो॑भि श्चि॒नोति॑ चि॒नोति॒ छन्दो॑भि॒ श्छन्दो॑भि श्चि॒नोति॑ । \newline
40. छन्दो॑भि॒रिति॒ छन्दः॑ - भिः॒ । \newline
41. चि॒नो त्य॒ग्नयो॒ ऽग्नय॑ श्चि॒नोति॑ चि॒नो त्य॒ग्नयः॑ । \newline
42. अ॒ग्नयो॒ वै वा अ॒ग्नयो॒ ऽग्नयो॒ वै । \newline
43. वै छन्दाꣳ॑सि॒ छन्दाꣳ॑सि॒ वै वै छन्दाꣳ॑सि । \newline
44. छन्दाꣳ॑सि॒ तस्मा॒त् तस्मा॒च् छन्दाꣳ॑सि॒ छन्दाꣳ॑सि॒ तस्मा᳚त् । \newline
45. तस्मा॑ द॒ग्नि र॒ग्नि स्तस्मा॒त् तस्मा॑ द॒ग्निः । \newline
46. अ॒ग्नि रु॑च्यत उच्यते॒ ऽग्नि र॒ग्नि रु॑च्यते । \newline
47. उ॒च्य॒ते ऽथो॒ अथो॑ उच्यत उच्य॒ते ऽथो᳚ । \newline
48. अथो॑ इ॒य मि॒य मथो॒ अथो॑ इ॒यम् । \newline
49. अथो॒ इत्यथो᳚ । \newline
50. इ॒यं ॅवै वा इ॒य मि॒यं ॅवै । \newline
51. वा अ॒ग्नि र॒ग्निर् वै वा अ॒ग्निः । \newline
52. अ॒ग्निर् वै᳚श्वान॒रो वै᳚श्वान॒रो᳚ ऽग्नि र॒ग्निर् वै᳚श्वान॒रः । \newline
53. वै॒श्वा॒न॒रो यद् यद् वै᳚श्वान॒रो वै᳚श्वान॒रो यत् । \newline
54. यन् मृ॒दा मृ॒दा यद् यन् मृ॒दा । \newline

\textbf{Ghana Paata } \newline

1. आपो॒ वै वा आप॒ आपो॒ वै सर्वाः॒ सर्वा॒ वा आप॒ आपो॒ वै सर्वाः᳚ । \newline
2. वै सर्वाः॒ सर्वा॒ वै वै सर्वा॑ दे॒वता॑ दे॒वताः॒ सर्वा॒ वै वै सर्वा॑ दे॒वताः᳚ । \newline
3. सर्वा॑ दे॒वता॑ दे॒वताः॒ सर्वाः॒ सर्वा॑ दे॒वता॑ दे॒वता॑भिर् दे॒वता॑भिर् दे॒वताः॒ सर्वाः॒ सर्वा॑ दे॒वता॑ दे॒वता॑भिः । \newline
4. दे॒वता॑ दे॒वता॑भिर् दे॒वता॑भिर् दे॒वता॑ दे॒वता॑ दे॒वता॑भि रे॒वैव दे॒वता॑भिर् दे॒वता॑ दे॒वता॑ दे॒वता॑भि रे॒व । \newline
5. दे॒वता॑भि रे॒वैव दे॒वता॑भिर् दे॒वता॑भि रे॒वैन॑ मेन मे॒व दे॒वता॑भिर् दे॒वता॑भि रे॒वैन᳚म् । \newline
6. ए॒वैन॑ मेन मे॒वै वैनꣳ॒॒ सꣳ स मे॑न मे॒वै वैनꣳ॒॒ सम् । \newline
7. ए॒नꣳ॒॒ सꣳ स मे॑न मेनꣳ॒॒ सꣳ सृ॑जति सृजति॒ स मे॑न मेनꣳ॒॒ सꣳ सृ॑जति । \newline
8. सꣳ सृ॑जति सृजति॒ सꣳ सꣳ सृ॑जति॒ यद् यथ् सृ॑जति॒ सꣳ सꣳ सृ॑जति॒ यत् । \newline
9. सृ॒ज॒ति॒ यद् यथ् सृ॑जति सृजति॒ यन् मृ॒दा मृ॒दा यथ् सृ॑जति सृजति॒ यन् मृ॒दा । \newline
10. यन् मृ॒दा मृ॒दा यद् यन् मृ॒दा चि॒नोति॑ चि॒नोति॑ मृ॒दा यद् यन् मृ॒दा चि॒नोति॑ । \newline
11. मृ॒दा चि॒नोति॑ चि॒नोति॑ मृ॒दा मृ॒दा चि॒नोती॒य मि॒यम् चि॒नोति॑ मृ॒दा मृ॒दा चि॒नोती॒यम् । \newline
12. चि॒नोती॒य मि॒यम् चि॒नोति॑ चि॒नोती॒यं ॅवै वा इ॒यम् चि॒नोति॑ चि॒नोती॒यं ॅवै । \newline
13. इ॒यं ॅवै वा इ॒य मि॒यं ॅवा अ॒ग्नि र॒ग्निर् वा इ॒य मि॒यं ॅवा अ॒ग्निः । \newline
14. वा अ॒ग्नि र॒ग्निर् वै वा अ॒ग्निर् वै᳚श्वान॒रो वै᳚श्वान॒रो᳚ ऽग्निर् वै वा अ॒ग्निर् वै᳚श्वान॒रः । \newline
15. अ॒ग्निर् वै᳚श्वान॒रो वै᳚श्वान॒रो᳚ ऽग्नि र॒ग्निर् वै᳚श्वान॒रो᳚ ऽग्निना॒ ऽग्निना॑ वैश्वान॒रो᳚ ऽग्नि र॒ग्निर् वै᳚श्वान॒रो᳚ ऽग्निना᳚ । \newline
16. वै॒श्वा॒न॒रो᳚ ऽग्निना॒ ऽग्निना॑ वैश्वान॒रो वै᳚श्वान॒रो᳚ ऽग्नि नै॒वैवाग्निना॑ वैश्वान॒रो वै᳚श्वान॒रो᳚ ऽग्निनै॒व । \newline
17. अ॒ग्नि नै॒वैवाग्निना॒ ऽग्निनै॒व तत् तदे॒वाग्निना॒ ऽग्निनै॒व तत् । \newline
18. ए॒व तत् तदे॒ वैव तद॒ग्नि म॒ग्निम् तदे॒ वैव तद॒ग्निम् । \newline
19. तद॒ग्नि म॒ग्निम् तत् तद॒ग्निम् चि॑नोति चिनो त्य॒ग्निम् तत् तद॒ग्निम् चि॑नोति । \newline
20. अ॒ग्निम् चि॑नोति चिनो त्य॒ग्नि म॒ग्निम् चि॑नोति ब्रह्मवा॒दिनो᳚ ब्रह्मवा॒दिन॑ श्चिनो त्य॒ग्नि म॒ग्निम् चि॑नोति ब्रह्मवा॒दिनः॑ । \newline
21. चि॒नो॒ति॒ ब्र॒ह्म॒वा॒दिनो᳚ ब्रह्मवा॒दिन॑ श्चिनोति चिनोति ब्रह्मवा॒दिनो॑ वदन्ति वदन्ति ब्रह्मवा॒दिन॑ श्चिनोति चिनोति ब्रह्मवा॒दिनो॑ वदन्ति । \newline
22. ब्र॒ह्म॒वा॒दिनो॑ वदन्ति वदन्ति ब्रह्मवा॒दिनो᳚ ब्रह्मवा॒दिनो॑ वदन्ति॒ यद् यद् व॑दन्ति ब्रह्मवा॒दिनो᳚ ब्रह्मवा॒दिनो॑ वदन्ति॒ यत् । \newline
23. ब्र॒ह्म॒वा॒दिन॒ इति॑ ब्रह्म - वा॒दिनः॑ । \newline
24. व॒द॒न्ति॒ यद् यद् व॑दन्ति वदन्ति॒ यन् मृ॒दा मृ॒दा यद् व॑दन्ति वदन्ति॒ यन् मृ॒दा । \newline
25. यन् मृ॒दा मृ॒दा यद् यन् मृ॒दा च॑ च मृ॒दा यद् यन् मृ॒दा च॑ । \newline
26. मृ॒दा च॑ च मृ॒दा मृ॒दा चा॒द्भि र॒द्भिश्च॑ मृ॒दा मृ॒दा चा॒द्भिः । \newline
27. चा॒द्भि र॒द्भिश्च॑ चा॒द्भिश्च॑ चा॒द्भिश्च॑ चा॒द्भिश्च॑ । \newline
28. अ॒द्भिश्च॑ चा॒द्भि र॒द्भि श्चा॒ग्नि र॒ग्नि श्चा॒द्भि र॒द्भि श्चा॒ग्निः । \newline
29. अ॒द्भिरित्य॑त् - भिः । \newline
30. चा॒ग्नि र॒ग्निश्च॑ चा॒ग्नि श्ची॒यते॑ ची॒यते॒ ऽग्निश्च॑ चा॒ग्नि श्ची॒यते᳚ । \newline
31. अ॒ग्नि श्ची॒यते॑ ची॒यते॒ ऽग्नि र॒ग्नि श्ची॒यते ऽथाथ॑ ची॒यते॒ ऽग्नि र॒ग्नि श्ची॒यते ऽथ॑ । \newline
32. ची॒यते ऽथाथ॑ ची॒यते॑ ची॒यते ऽथ॒ कस्मा॒त् कस्मा॒ दथ॑ ची॒यते॑ ची॒यते ऽथ॒ कस्मा᳚त् । \newline
33. अथ॒ कस्मा॒त् कस्मा॒ दथाथ॒ कस्मा॑ द॒ग्नि र॒ग्निः कस्मा॒ दथाथ॒ कस्मा॑ द॒ग्निः । \newline
34. कस्मा॑ द॒ग्नि र॒ग्निः कस्मा॒त् कस्मा॑ द॒ग्नि रु॑च्यत उच्यते॒ ऽग्निः कस्मा॒त् कस्मा॑ द॒ग्नि रु॑च्यते । \newline
35. अ॒ग्नि रु॑च्यत उच्यते॒ ऽग्नि र॒ग्नि रु॑च्यत॒ इती त्यु॑च्यते॒ ऽग्नि र॒ग्नि रु॑च्यत॒ इति॑ । \newline
36. उ॒च्य॒त॒ इती त्यु॑च्यत उच्यत॒ इति॒ यद् यदि त्यु॑च्यत उच्यत॒ इति॒ यत् । \newline
37. इति॒ यद् यदितीति॒ यच् छन्दो॑भि॒ श्छन्दो॑भि॒र् यदितीति॒ यच् छन्दो॑भिः । \newline
38. यच् छन्दो॑भि॒ श्छन्दो॑भि॒र् यद् यच् छन्दो॑भि श्चि॒नोति॑ चि॒नोति॒ छन्दो॑भि॒र् यद् यच् छन्दो॑भि श्चि॒नोति॑ । \newline
39. छन्दो॑भि श्चि॒नोति॑ चि॒नोति॒ छन्दो॑भि॒ श्छन्दो॑भि श्चि॒नो त्य॒ग्नयो॒ ऽग्नय॑ श्चि॒नोति॒ छन्दो॑भि॒ श्छन्दो॑भि श्चि॒नो त्य॒ग्नयः॑ । \newline
40. छन्दो॑भि॒रिति॒ छन्दः॑ - भिः॒ । \newline
41. चि॒नो त्य॒ग्नयो॒ ऽग्नय॑ श्चि॒नोति॑ चि॒नो त्य॒ग्नयो॒ वै वा अ॒ग्नय॑ श्चि॒नोति॑ चि॒नो त्य॒ग्नयो॒ वै । \newline
42. अ॒ग्नयो॒ वै वा अ॒ग्नयो॒ ऽग्नयो॒ वै छन्दाꣳ॑सि॒ छन्दाꣳ॑सि॒ वा अ॒ग्नयो॒ ऽग्नयो॒ वै छन्दाꣳ॑सि । \newline
43. वै छन्दाꣳ॑सि॒ छन्दाꣳ॑सि॒ वै वै छन्दाꣳ॑सि॒ तस्मा॒त् तस्मा॒च् छन्दाꣳ॑सि॒ वै वै छन्दाꣳ॑सि॒ तस्मा᳚त् । \newline
44. छन्दाꣳ॑सि॒ तस्मा॒त् तस्मा॒च् छन्दाꣳ॑सि॒ छन्दाꣳ॑सि॒ तस्मा॑ द॒ग्नि र॒ग्नि स्तस्मा॒च् छन्दाꣳ॑सि॒ छन्दाꣳ॑सि॒ तस्मा॑द॒ग्निः । \newline
45. तस्मा॑ द॒ग्नि र॒ग्नि स्तस्मा॒त् तस्मा॑ द॒ग्नि रु॑च्यत उच्यते॒ ऽग्नि स्तस्मा॒त् तस्मा॑ द॒ग्नि रु॑च्यते । \newline
46. अ॒ग्नि रु॑च्यत उच्यते॒ ऽग्नि र॒ग्नि रु॑च्य॒ते ऽथो॒ अथो॑ उच्यते॒ ऽग्नि र॒ग्नि रु॑च्य॒ते ऽथो᳚ । \newline
47. उ॒च्य॒ते ऽथो॒ अथो॑ उच्यत उच्य॒ते ऽथो॑ इ॒य मि॒य मथो॑ उच्यत उच्य॒ते ऽथो॑ इ॒यम् । \newline
48. अथो॑ इ॒य मि॒य मथो॒ अथो॑ इ॒यं ॅवै वा इ॒य मथो॒ अथो॑ इ॒यं ॅवै । \newline
49. अथो॒ इत्यथो᳚ । \newline
50. इ॒यं ॅवै वा इ॒य मि॒यं ॅवा अ॒ग्नि र॒ग्निर् वा इ॒य मि॒यं ॅवा अ॒ग्निः । \newline
51. वा अ॒ग्नि र॒ग्निर् वै वा अ॒ग्निर् वै᳚श्वान॒रो वै᳚श्वान॒रो᳚ ऽग्निर् वै वा अ॒ग्निर् वै᳚श्वान॒रः । \newline
52. अ॒ग्निर् वै᳚श्वान॒रो वै᳚श्वान॒रो᳚ ऽग्नि र॒ग्निर् वै᳚श्वान॒रो यद् यद् वै᳚श्वान॒रो᳚ ऽग्नि र॒ग्निर् वै᳚श्वान॒रो यत् । \newline
53. वै॒श्वा॒न॒रो यद् यद् वै᳚श्वान॒रो वै᳚श्वान॒रो यन् मृ॒दा मृ॒दा यद् वै᳚श्वान॒रो वै᳚श्वान॒रो यन् मृ॒दा । \newline
54. यन् मृ॒दा मृ॒दा यद् यन् मृ॒दा चि॒नोति॑ चि॒नोति॑ मृ॒दा यद् यन् मृ॒दा चि॒नोति॑ । \newline
\pagebreak
\markright{ TS 5.7.9.4  \hfill https://www.vedavms.in \hfill}

\section{ TS 5.7.9.4 }

\textbf{TS 5.7.9.4 } \newline
\textbf{Samhita Paata} \newline

-न्मृ॒दा चि॒नोति॒ तस्मा॑द॒ग्निरु॑च्यते हिरण्येष्ट॒का उप॑ दधाति॒ ज्योति॒र्वै हिर॑ण्यं॒ ज्योति॑रे॒वाऽस्मि॑न् दधा॒त्यथो॒ तेजो॒ वै हिर॑ण्यं॒ तेज॑ ए॒वाऽऽ*त्मन् ध॑त्ते॒ यो वा अ॒ग्निꣳ स॒र्वतो॑मुखं चिनु॒ते सर्वा॑सु प्र॒जास्वन्न॑मत्ति॒ सर्वा॒ दिशो॒ऽभि ज॑यति गाय॒त्रीं पु॒रस्ता॒दुप॑ दधाति त्रि॒ष्टुभं॑ दक्षिण॒तो जग॑तीं प॒श्चाद॑नु॒ष्टुभ॑मुत्तर॒तः प॒ङ्क्तिं मद्ध्य॑ ए॒ष वा ( ) अ॒ग्निः स॒र्वतो॑मुख॒स्तं ॅय ए॒वं ॅवि॒द्वाꣳश्चि॑नु॒ते सर्वा॑सु प्र॒जास्वन्न॑मत्ति॒ सर्वा॒ दिशो॒ऽभि ज॑य॒त्यथो॑ दि॒श्ये॑व दिशं॒ प्र व॑यति॒ तस्मा᳚द्-दि॒शि दिक् प्रोता᳚ ॥ \newline

\textbf{Pada Paata} \newline

मृ॒दा । चि॒नोति॑ । तस्मा᳚त् । अ॒ग्निः । उ॒च्य॒ते॒ । हि॒र॒ण्ये॒ष्ट॒का इति॑ हिरण्य - इ॒ष्ट॒काः । उपेति॑ । द॒धा॒ति॒ । ज्योतिः॑ । वै । हिर॑ण्यम् । ज्योतिः॑ । ए॒व । अ॒स्मि॒न्न् । द॒धा॒ति॒ । अथो॒ इति॑ । तेजः॑ । वै । हिर॑ण्यम् । तेजः॑ । ए॒व । आ॒त्मन्न् । ध॒त्ते॒ । यः । वै । अ॒ग्निम् । स॒र्वतो॑मुख॒मिति॑ स॒र्वतः॑ - मु॒ख॒म् । चि॒नु॒ते । सर्वा॑सु । प्र॒जास्विति॑ प्र - जासु॑ । अन्न᳚म् । अ॒त्ति॒ । सर्वाः᳚ । दिशः॑ । अ॒भीति॑ । ज॒य॒ति॒ । गा॒य॒त्रीम् । पु॒रस्ता᳚त् । उपेति॑ । द॒धा॒ति॒ । त्रि॒ष्टुभ᳚म् । द॒क्षि॒ण॒तः । जग॑तीम् । प॒श्चात् । अ॒नु॒ष्टुभ॒मित्य॑नु-स्तुभ᳚म् । उ॒त्त॒र॒त इत्यु॑त्-त॒र॒तः । प॒ङ्क्तिम् । मद्ध्य᳚ । ए॒षः । वै ( ) । अ॒ग्निः । स॒र्वतो॑मुख॒ इति॑ स॒र्वतः॑ - मु॒खः॒ । तम् । यः । ए॒वम् । वि॒द्वान् । चि॒नु॒ते । सर्वा॑सु । प्र॒जास्विति॑ प्र-जासु॑ । अन्न᳚म् । अ॒त्ति॒ । सर्वाः᳚ । दिशः॑ । अ॒भीति॑ । ज॒य॒ति॒ । अथो॒ इति॑ । दि॒शि । ए॒व । दिश᳚म् । प्रेति॑ । व॒य॒ति॒ । तस्मा᳚त् । दि॒शि । दिक् । प्रोतेति॒ प्र - उ॒ता॒ ॥  \newline


\textbf{Krama Paata} \newline

मृ॒दा चि॒नोति॑ । चि॒नोति॒ तस्मा᳚त् । तस्मा॑द॒ग्निः । अ॒ग्निरु॑च्यते । उ॒च्य॒ते॒ हि॒र॒ण्ये॒ष्ट॒काः । हि॒र॒ण्ये॒ष्ट॒का उप॑ । हि॒र॒ण्ये॒ष्ट॒का इति॑ हिरण्य - इ॒ष्ट॒काः । उप॑ दधाति । द॒धा॒ति॒ ज्योतिः॑ । ज्योति॒र् वै । वै हिर॑ण्यम् । हिर॑ण्य॒म् ज्योतिः॑ । ज्योति॑रे॒व । ए॒वास्मिन्न्॑ । अ॒स्मि॒न् द॒धा॒ति॒ । द॒धा॒त्यथो᳚ । अथो॒ तेजः॑ । अथो॒ इत्यथो᳚ । तेजो॒ वै । वै हिर॑ण्यम् । हिर॑ण्य॒म् तेजः॑ । तेज॑ ए॒व । ए॒वात्मन्न् । आ॒त्मन् ध॑त्ते । ध॒त्ते॒ यः । यो वै । वा अ॒ग्निम् । अ॒ग्निꣳ स॒र्वतो॑मुखम् । स॒र्वतो॑मुखम् चिनु॒ते । स॒र्वतो॑मुख॒मिति॑ स॒र्वतः॑ - मु॒ख॒म् । चि॒नु॒ते सर्वा॑सु । सर्वा॑सु प्र॒जासु॑ । प्र॒जास्वन्न᳚म् । प्र॒जास्विति॑ प्र - जासु॑ । अन्न॑मत्ति । अ॒त्ति॒ सर्वाः᳚ । सर्वा॒ दिशः॑ । दिशो॒ऽभि । अ॒भि ज॑यति । ज॒य॒ति॒ गा॒य॒त्रीम् । गा॒य॒त्रीम् पु॒रस्ता᳚त् । पु॒रस्ता॒दुप॑ । उप॑ दधाति । द॒धा॒ति॒ त्रि॒ष्टुभ᳚म् । त्रि॒ष्टुभ॑म् दक्षिण॒तः । द॒क्षि॒ण॒तो जग॑तीम् । जग॑तीम् प॒श्चात् । प॒श्चाद॑नु॒ष्टुभ᳚म् । अ॒नु॒ष्टुभ॑मुत्तर॒तः । अ॒नु॒ष्टुभ॒मित्य॑नु - स्तुभ᳚म् । उ॒त्त॒र॒तः प॒ङ्क्तिम् । उ॒त्त॒र॒त इत्यु॑त् - त॒र॒तः । प॒ङ्क्तिम् मद्ध्ये᳚ । मद्ध्य॑ ए॒षः । ए॒ष वै ( ) । वा अ॒ग्निः । अ॒ग्निः स॒र्वतो॑मुखः । स॒र्वतो॑मुख॒स्तम् । स॒र्वतो॑मुख॒ इति॑ स॒र्वतः॑ - मु॒खः॒ । तम् ॅयः । य ए॒वम् । ए॒वम् ॅवि॒द्वान् । वि॒द्वाꣳश्चि॑नु॒ते । चि॒नु॒ते सर्वा॑सु । सर्वा॑सु प्र॒जासु॑ । प्र॒जास्वन्न᳚म् । प्र॒जास्विति॑ प्र - जासु॑ । अन्न॑मत्ति । अ॒त्ति॒ सर्वाः᳚ । सर्वा॒ दिशः॑ । दिशो॒ऽभि । अ॒भि ज॑यति । ज॒य॒त्यथो᳚ । अथो॑ दि॒शि । अथो॒ इत्यथो᳚ । दि॒श्ये॑व । ए॒व दिश᳚म् । दिश॒म् प्र । प्र व॑यति । व॒य॒ति॒ तस्मा᳚त् । तस्मा᳚द् दि॒शि । दि॒शि दिक् । दिक् प्रोता᳚ । प्रोतेति॒ प्र - उ॒ता॒ । \newline

\textbf{Jatai Paata} \newline

1. मृ॒दा चि॒नोति॑ चि॒नोति॑ मृ॒दा मृ॒दा चि॒नोति॑ । \newline
2. चि॒नोति॒ तस्मा॒त् तस्मा᳚च् चि॒नोति॑ चि॒नोति॒ तस्मा᳚त् । \newline
3. तस्मा॑ द॒ग्नि र॒ग्नि स्तस्मा॒त् तस्मा॑ द॒ग्निः । \newline
4. अ॒ग्नि रु॑च्यत उच्यते॒ ऽग्नि र॒ग्नि रु॑च्यते । \newline
5. उ॒च्य॒ते॒ हि॒र॒ण्ये॒ष्ट॒का हि॑रण्येष्ट॒का उ॑च्यत उच्यते हिरण्येष्ट॒काः । \newline
6. हि॒र॒ण्ये॒ष्ट॒का उपोप॑ हिरण्येष्ट॒का हि॑रण्येष्ट॒का उप॑ । \newline
7. हि॒र॒ण्ये॒ष्ट॒का इति॑ हिरण्य - इ॒ष्ट॒काः । \newline
8. उप॑ दधाति दधा॒ त्युपोप॑ दधाति । \newline
9. द॒धा॒ति॒ ज्योति॒र् ज्योति॑र् दधाति दधाति॒ ज्योतिः॑ । \newline
10. ज्योति॒र् वै वै ज्योति॒र् ज्योति॒र् वै । \newline
11. वै हिर॑ण्यꣳ॒॒ हिर॑ण्यं॒ ॅवै वै हिर॑ण्यम् । \newline
12. हिर॑ण्य॒म् ज्योति॒र् ज्योति॒र्॒. हिर॑ण्यꣳ॒॒ हिर॑ण्य॒म् ज्योतिः॑ । \newline
13. ज्योति॑रे॒ वैव ज्योति॒र् ज्योति॑ रे॒व । \newline
14. ए॒वास्मि॑न् नस्मिन् ने॒वै वास्मिन्न्॑ । \newline
15. अ॒स्मि॒न् द॒धा॒ति॒ द॒धा॒ त्य॒स्मि॒न् न॒स्मि॒न् द॒धा॒ति॒ । \newline
16. द॒धा॒ त्यथो॒ अथो॑ दधाति दधा॒ त्यथो᳚ । \newline
17. अथो॒ तेज॒ स्तेजो ऽथो॒ अथो॒ तेजः॑ । \newline
18. अथो॒ इत्यथो᳚ । \newline
19. तेजो॒ वै वै तेज॒ स्तेजो॒ वै । \newline
20. वै हिर॑ण्यꣳ॒॒ हिर॑ण्यं॒ ॅवै वै हिर॑ण्यम् । \newline
21. हिर॑ण्य॒म् तेज॒ स्तेजो॒ हिर॑ण्यꣳ॒॒ हिर॑ण्य॒म् तेजः॑ । \newline
22. तेज॑ ए॒वैव तेज॒ स्तेज॑ ए॒व । \newline
23. ए॒वात्मन् ना॒त्मन् ने॒वै वात्मन्न् । \newline
24. आ॒त्मन् ध॑त्ते धत्त आ॒त्मन् ना॒त्मन् ध॑त्ते । \newline
25. ध॒त्ते॒ यो यो ध॑त्ते धत्ते॒ यः । \newline
26. यो वै वै यो यो वै । \newline
27. वा अ॒ग्नि म॒ग्निं ॅवै वा अ॒ग्निम् । \newline
28. अ॒ग्निꣳ स॒र्वतो॑मुखꣳ स॒र्वतो॑मुख म॒ग्नि म॒ग्निꣳ स॒र्वतो॑मुखम् । \newline
29. स॒र्वतो॑मुखम् चिनु॒ते चि॑नु॒ते स॒र्वतो॑मुखꣳ स॒र्वतो॑मुखम् चिनु॒ते । \newline
30. स॒र्वतो॑मुख॒मिति॑ स॒र्वतः॑ - मु॒ख॒म् । \newline
31. चि॒नु॒ते सर्वा॑सु॒ सर्वा॑सु चिनु॒ते चि॑नु॒ते सर्वा॑सु । \newline
32. सर्वा॑सु प्र॒जासु॑ प्र॒जासु॒ सर्वा॑सु॒ सर्वा॑सु प्र॒जासु॑ । \newline
33. प्र॒जा स्वन्न॒ मन्न॑म् प्र॒जासु॑ प्र॒जा स्वन्न᳚म् । \newline
34. प्र॒जास्विति॑ प्र - जासु॑ । \newline
35. अन्न॑ मत्त्य॒त् त्यन्न॒ मन्न॑ मत्ति । \newline
36. अ॒त्ति॒ सर्वाः॒ सर्वा॑ अत्त्यत्ति॒ सर्वाः᳚ । \newline
37. सर्वा॒ दिशो॒ दिशः॒ सर्वाः॒ सर्वा॒ दिशः॑ । \newline
38. दिशो॒ ऽभ्य॑भि दिशो॒ दिशो॒ ऽभि । \newline
39. अ॒भि ज॑यति जय त्य॒भ्य॑भि ज॑यति । \newline
40. ज॒य॒ति॒ गा॒य॒त्रीम् गा॑य॒त्रीम् ज॑यति जयति गाय॒त्रीम् । \newline
41. गा॒य॒त्रीम् पु॒रस्ता᳚त् पु॒रस्ता᳚द् गाय॒त्रीम् गा॑य॒त्रीम् पु॒रस्ता᳚त् । \newline
42. पु॒रस्ता॒ दुपोप॑ पु॒रस्ता᳚त् पु॒रस्ता॒ दुप॑ । \newline
43. उप॑ दधाति दधा॒ त्युपोप॑ दधाति । \newline
44. द॒धा॒ति॒ त्रि॒ष्टुभ॑म् त्रि॒ष्टुभ॑म् दधाति दधाति त्रि॒ष्टुभ᳚म् । \newline
45. त्रि॒ष्टुभ॑म् दक्षिण॒तो द॑क्षिण॒त स्त्रि॒ष्टुभ॑म् त्रि॒ष्टुभ॑म् दक्षिण॒तः । \newline
46. द॒क्षि॒ण॒तो जग॑ती॒म् जग॑तीम् दक्षिण॒तो द॑क्षिण॒तो जग॑तीम् । \newline
47. जग॑तीम् प॒श्चात् प॒श्चाज् जग॑ती॒म् जग॑तीम् प॒श्चात् । \newline
48. प॒श्चा द॑नु॒ष्टुभ॑ मनु॒ष्टुभ॑म् प॒श्चात् प॒श्चा द॑नु॒ष्टुभ᳚म् । \newline
49. अ॒नु॒ष्टुभ॑ मुत्तर॒त उ॑त्तर॒तो॑ ऽनु॒ष्टुभ॑ मनु॒ष्टुभ॑ मुत्तर॒तः । \newline
50. अ॒नु॒ष्टुभ॒मित्य॑नु - स्तुभ᳚म् । \newline
51. उ॒त्त॒र॒तः प॒ङ्क्तिम् प॒ङ्क्ति मु॑त्तर॒त उ॑त्तर॒तः प॒ङ्क्तिम् । \newline
52. उ॒त्त॒र॒त इत्यु॑त् - त॒र॒तः । \newline
53. प॒ङ्क्तिम् मद्ध्ये॒ मद्ध्ये॑ प॒ङ्क्तिम् प॒ङ्क्तिम् मद्ध्ये᳚ । \newline
54. मद्ध्य॑ ए॒ष ए॒ष मद्ध्ये॒ मद्ध्य॑ ए॒षः । \newline
55. ए॒ष वै वा ए॒ष ए॒ष वै । \newline
56. वा अ॒ग्नि र॒ग्निर् वै वा अ॒ग्निः । \newline
57. अ॒ग्निः स॒र्वतो॑मुखः स॒र्वतो॑मुखो॒ ऽग्नि र॒ग्निः स॒र्वतो॑मुखः । \newline
58. स॒र्वतो॑मुख॒स्तम् तꣳ स॒र्वतो॑मुखः स॒र्वतो॑मुख॒स्तम् । \newline
59. स॒र्वतो॑मुख॒ इति॑ स॒र्वतः॑ - मु॒खः॒ । \newline
60. तं ॅयो य स्तम् तं ॅयः । \newline
61. य ए॒व मे॒वं ॅयो य ए॒वम् । \newline
62. ए॒वं ॅवि॒द्वान्. वि॒द्वा ने॒व मे॒वं ॅवि॒द्वान् । \newline
63. वि॒द्वाꣳ श्चि॑नु॒ते चि॑नु॒ते वि॒द्वान्. वि॒द्वाꣳ श्चि॑नु॒ते । \newline
64. चि॒नु॒ते सर्वा॑सु॒ सर्वा॑सु चिनु॒ते चि॑नु॒ते सर्वा॑सु । \newline
65. सर्वा॑सु प्र॒जासु॑ प्र॒जासु॒ सर्वा॑सु॒ सर्वा॑सु प्र॒जासु॑ । \newline
66. प्र॒जा स्वन्न॒ मन्न॑म् प्र॒जासु॑ प्र॒जा स्वन्न᳚म् । \newline
67. प्र॒जास्विति॑ प्र - जासु॑ । \newline
68. अन्न॑ मत्त्य॒त् त्यन्न॒ मन्न॑ मत्ति । \newline
69. अ॒त्ति॒ सर्वाः॒ सर्वा॑ अत्त्यत्ति॒ सर्वाः᳚ । \newline
70. सर्वा॒ दिशो॒ दिशः॒ सर्वाः॒ सर्वा॒ दिशः॑ । \newline
71. दिशो॒ ऽभ्य॑भि दिशो॒ दिशो॒ ऽभि । \newline
72. अ॒भि ज॑यति जय त्य॒भ्य॑भि ज॑यति । \newline
73. ज॒य॒ त्यथो॒ अथो॑ जयति जय॒ त्यथो᳚ । \newline
74. अथो॑ दि॒शि दि॒श्यथो॒ अथो॑ दि॒शि । \newline
75. अथो॒ इत्यथो᳚ । \newline
76. दि॒श्ये॑ वैव दि॒शि दि॒श्ये॑व । \newline
77. ए॒व दिश॒म् दिश॑ मे॒वैव दिश᳚म् । \newline
78. दिश॒म् प्र प्र दिश॒म् दिश॒म् प्र । \newline
79. प्र व॑यति वयति॒ प्र प्र व॑यति । \newline
80. व॒य॒ति॒ तस्मा॒त् तस्मा᳚द् वयति वयति॒ तस्मा᳚त् । \newline
81. तस्मा᳚द् दि॒शि दि॒शि तस्मा॒त् तस्मा᳚द् दि॒शि । \newline
82. दि॒शि दिग् दिग् दि॒शि दि॒शि दिक् । \newline
83. दिक् प्रोता॒ प्रोता॒ दिग् दिक् प्रोता᳚ । \newline
84. प्रोतेति॒ प्र - उ॒ता॒ । \newline

\textbf{Ghana Paata } \newline

1. मृ॒दा चि॒नोति॑ चि॒नोति॑ मृ॒दा मृ॒दा चि॒नोति॒ तस्मा॒त् तस्मा᳚च् चि॒नोति॑ मृ॒दा मृ॒दा चि॒नोति॒ तस्मा᳚त् । \newline
2. चि॒नोति॒ तस्मा॒त् तस्मा᳚च् चि॒नोति॑ चि॒नोति॒ तस्मा॑ द॒ग्नि र॒ग्नि स्तस्मा᳚च् चि॒नोति॑ चि॒नोति॒ तस्मा॑ द॒ग्निः । \newline
3. तस्मा॑ द॒ग्नि र॒ग्नि स्तस्मा॒त् तस्मा॑ द॒ग्नि रु॑च्यत उच्यते॒ ऽग्नि स्तस्मा॒त् तस्मा॑ द॒ग्नि रु॑च्यते । \newline
4. अ॒ग्नि रु॑च्यत उच्यते॒ ऽग्नि र॒ग्नि रु॑च्यते हिरण्येष्ट॒का हि॑रण्येष्ट॒का उ॑च्यते॒ ऽग्नि र॒ग्नि रु॑च्यते हिरण्येष्ट॒काः । \newline
5. उ॒च्य॒ते॒ हि॒र॒ण्ये॒ष्ट॒का हि॑रण्येष्ट॒का उ॑च्यत उच्यते हिरण्येष्ट॒का उपोप॑ हिरण्येष्ट॒का उ॑च्यत उच्यते हिरण्येष्ट॒का उप॑ । \newline
6. हि॒र॒ण्ये॒ष्ट॒का उपोप॑ हिरण्येष्ट॒का हि॑रण्येष्ट॒का उप॑ दधाति दधा॒ त्युप॑ हिरण्येष्ट॒का हि॑रण्येष्ट॒का उप॑ दधाति । \newline
7. हि॒र॒ण्ये॒ष्ट॒का इति॑ हिरण्य - इ॒ष्ट॒काः । \newline
8. उप॑ दधाति दधा॒ त्युपोप॑ दधाति॒ ज्योति॒र् ज्योति॑र् दधा॒ त्युपोप॑ दधाति॒ ज्योतिः॑ । \newline
9. द॒धा॒ति॒ ज्योति॒र् ज्योति॑र् दधाति दधाति॒ ज्योति॒र् वै वै ज्योति॑र् दधाति दधाति॒ ज्योति॒र् वै । \newline
10. ज्योति॒र् वै वै ज्योति॒र् ज्योति॒र् वै हिर॑ण्यꣳ॒॒ हिर॑ण्यं॒ ॅवै ज्योति॒र् ज्योति॒र् वै हिर॑ण्यम् । \newline
11. वै हिर॑ण्यꣳ॒॒ हिर॑ण्यं॒ ॅवै वै हिर॑ण्य॒म् ज्योति॒र् ज्योति॒र्॒. हिर॑ण्यं॒ ॅवै वै हिर॑ण्य॒म् ज्योतिः॑ । \newline
12. हिर॑ण्य॒म् ज्योति॒र् ज्योति॒र्॒. हिर॑ण्यꣳ॒॒ हिर॑ण्य॒म् ज्योति॑रे॒ वैव ज्योति॒र्॒. हिर॑ण्यꣳ॒॒ हिर॑ण्य॒म् ज्योति॑ रे॒व । \newline
13. ज्योति॑ रे॒वैव ज्योति॒र् ज्योति॑ रे॒वास्मि॑न् नस्मिन् ने॒व ज्योति॒र् ज्योति॑ रे॒वास्मिन्न्॑ । \newline
14. ए॒वास्मि॑न् नस्मिन् ने॒वै वास्मि॑न् दधाति दधा त्यस्मिन् ने॒वै वास्मि॑न् दधाति । \newline
15. अ॒स्मि॒न् द॒धा॒ति॒ द॒धा॒ त्य॒स्मि॒न् न॒स्मि॒न् द॒धा॒ त्यथो॒ अथो॑ दधा त्यस्मिन् नस्मिन् दधा॒ त्यथो᳚ । \newline
16. द॒धा॒ त्यथो॒ अथो॑ दधाति दधा॒ त्यथो॒ तेज॒ स्तेजो ऽथो॑ दधाति दधा॒ त्यथो॒ तेजः॑ । \newline
17. अथो॒ तेज॒ स्तेजो ऽथो॒ अथो॒ तेजो॒ वै वै तेजो ऽथो॒ अथो॒ तेजो॒ वै । \newline
18. अथो॒ इत्यथो᳚ । \newline
19. तेजो॒ वै वै तेज॒ स्तेजो॒ वै हिर॑ण्यꣳ॒॒ हिर॑ण्यं॒ ॅवै तेज॒ स्तेजो॒ वै हिर॑ण्यम् । \newline
20. वै हिर॑ण्यꣳ॒॒ हिर॑ण्यं॒ ॅवै वै हिर॑ण्य॒म् तेज॒ स्तेजो॒ हिर॑ण्यं॒ ॅवै वै हिर॑ण्य॒म् तेजः॑ । \newline
21. हिर॑ण्य॒म् तेज॒ स्तेजो॒ हिर॑ण्यꣳ॒॒ हिर॑ण्य॒म् तेज॑ ए॒वैव तेजो॒ हिर॑ण्यꣳ॒॒ हिर॑ण्य॒म् तेज॑ ए॒व । \newline
22. तेज॑ ए॒वैव तेज॒ स्तेज॑ ए॒वात्मन् ना॒त्मन् ने॒व तेज॒ स्तेज॑ ए॒वात्मन्न् । \newline
23. ए॒वात्मन् ना॒त्मन् ने॒वै वात्मन् ध॑त्ते धत्त आ॒त्मन् ने॒वै वात्मन् ध॑त्ते । \newline
24. आ॒त्मन् ध॑त्ते धत्त आ॒त्मन् ना॒त्मन् ध॑त्ते॒ यो यो ध॑त्त आ॒त्मन् ना॒त्मन् ध॑त्ते॒ यः । \newline
25. ध॒त्ते॒ यो यो ध॑त्ते धत्ते॒ यो वै वै यो ध॑त्ते धत्ते॒ यो वै । \newline
26. यो वै वै यो यो वा अ॒ग्नि म॒ग्निं ॅवै यो यो वा अ॒ग्निम् । \newline
27. वा अ॒ग्नि म॒ग्निं ॅवै वा अ॒ग्निꣳ स॒र्वतो॑मुखꣳ स॒र्वतो॑मुख म॒ग्निं ॅवै वा अ॒ग्निꣳ स॒र्वतो॑मुखम् । \newline
28. अ॒ग्निꣳ स॒र्वतो॑मुखꣳ स॒र्वतो॑मुख म॒ग्नि म॒ग्निꣳ स॒र्वतो॑मुखम् चिनु॒ते चि॑नु॒ते स॒र्वतो॑मुख म॒ग्नि म॒ग्निꣳ स॒र्वतो॑मुखम् चिनु॒ते । \newline
29. स॒र्वतो॑मुखम् चिनु॒ते चि॑नु॒ते स॒र्वतो॑मुखꣳ स॒र्वतो॑मुखम् चिनु॒ते सर्वा॑सु॒ सर्वा॑सु चिनु॒ते स॒र्वतो॑मुखꣳ स॒र्वतो॑मुखम् चिनु॒ते सर्वा॑सु । \newline
30. स॒र्वतो॑मुख॒मिति॑ स॒र्वतः॑ - मु॒ख॒म् । \newline
31. चि॒नु॒ते सर्वा॑सु॒ सर्वा॑सु चिनु॒ते चि॑नु॒ते सर्वा॑सु प्र॒जासु॑ प्र॒जासु॒ सर्वा॑सु चिनु॒ते चि॑नु॒ते सर्वा॑सु प्र॒जासु॑ । \newline
32. सर्वा॑सु प्र॒जासु॑ प्र॒जासु॒ सर्वा॑सु॒ सर्वा॑सु प्र॒जा स्वन्न॒ मन्न॑म् प्र॒जासु॒ सर्वा॑सु॒ सर्वा॑सु प्र॒जा स्वन्न᳚म् । \newline
33. प्र॒जा स्वन्न॒ मन्न॑म् प्र॒जासु॑ प्र॒जा स्वन्न॑ मत्त्य॒त् त्यन्न॑म् प्र॒जासु॑ प्र॒जा स्वन्न॑ मत्ति । \newline
34. प्र॒जास्विति॑ प्र - जासु॑ । \newline
35. अन्न॑ मत्त्य॒त् त्यन्न॒ मन्न॑ मत्ति॒ सर्वाः॒ सर्वा॑ अ॒त्त्यन्न॒ मन्न॑ मत्ति॒ सर्वाः᳚ । \newline
36. अ॒त्ति॒ सर्वाः॒ सर्वा॑ अत्त्यत्ति॒ सर्वा॒ दिशो॒ दिशः॒ सर्वा॑ अत्त्यत्ति॒ सर्वा॒ दिशः॑ । \newline
37. सर्वा॒ दिशो॒ दिशः॒ सर्वाः॒ सर्वा॒ दिशो॒ ऽभ्य॑भि दिशः॒ सर्वाः॒ सर्वा॒ दिशो॒ ऽभि । \newline
38. दिशो॒ ऽभ्य॑भि दिशो॒ दिशो॒ ऽभि ज॑यति जय त्य॒भि दिशो॒ दिशो॒ ऽभि ज॑यति । \newline
39. अ॒भि ज॑यति जय त्य॒भ्य॑भि ज॑यति गाय॒त्रीम् गा॑य॒त्रीम् ज॑य त्य॒भ्य॑भि ज॑यति गाय॒त्रीम् । \newline
40. ज॒य॒ति॒ गा॒य॒त्रीम् गा॑य॒त्रीम् ज॑यति जयति गाय॒त्रीम् पु॒रस्ता᳚त् पु॒रस्ता᳚द् गाय॒त्रीम् ज॑यति जयति गाय॒त्रीम् पु॒रस्ता᳚त् । \newline
41. गा॒य॒त्रीम् पु॒रस्ता᳚त् पु॒रस्ता᳚द् गाय॒त्रीम् गा॑य॒त्रीम् पु॒रस्ता॒ दुपोप॑ पु॒रस्ता᳚द् गाय॒त्रीम् गा॑य॒त्रीम् पु॒रस्ता॒ दुप॑ । \newline
42. पु॒रस्ता॒ दुपोप॑ पु॒रस्ता᳚त् पु॒रस्ता॒ दुप॑ दधाति दधा॒ त्युप॑ पु॒रस्ता᳚त् पु॒रस्ता॒ दुप॑ दधाति । \newline
43. उप॑ दधाति दधा॒ त्युपोप॑ दधाति त्रि॒ष्टुभ॑म् त्रि॒ष्टुभ॑म् दधा॒ त्युपोप॑ दधाति त्रि॒ष्टुभ᳚म् । \newline
44. द॒धा॒ति॒ त्रि॒ष्टुभ॑म् त्रि॒ष्टुभ॑म् दधाति दधाति त्रि॒ष्टुभ॑म् दक्षिण॒तो द॑क्षिण॒त स्त्रि॒ष्टुभ॑म् दधाति दधाति त्रि॒ष्टुभ॑म् दक्षिण॒तः । \newline
45. त्रि॒ष्टुभ॑म् दक्षिण॒तो द॑क्षिण॒त स्त्रि॒ष्टुभ॑म् त्रि॒ष्टुभ॑म् दक्षिण॒तो जग॑ती॒म् जग॑तीम् दक्षिण॒त स्त्रि॒ष्टुभ॑म् त्रि॒ष्टुभ॑म् दक्षिण॒तो जग॑तीम् । \newline
46. द॒क्षि॒ण॒तो जग॑ती॒म् जग॑तीम् दक्षिण॒तो द॑क्षिण॒तो जग॑तीम् प॒श्चात् प॒श्चाज् जग॑तीम् दक्षिण॒तो द॑क्षिण॒तो जग॑तीम् प॒श्चात् । \newline
47. जग॑तीम् प॒श्चात् प॒श्चाज् जग॑ती॒म् जग॑तीम् प॒श्चा द॑नु॒ष्टुभ॑ मनु॒ष्टुभ॑म् प॒श्चाज् जग॑ती॒म् जग॑तीम् प॒श्चा द॑नु॒ष्टुभ᳚म् । \newline
48. प॒श्चा द॑नु॒ष्टुभ॑ मनु॒ष्टुभ॑म् प॒श्चात् प॒श्चा द॑नु॒ष्टुभ॑ मुत्तर॒त उ॑त्तर॒तो॑ ऽनु॒ष्टुभ॑म् प॒श्चात् प॒श्चा द॑नु॒ष्टुभ॑ मुत्तर॒तः । \newline
49. अ॒नु॒ष्टुभ॑ मुत्तर॒त उ॑त्तर॒तो॑ ऽनु॒ष्टुभ॑ मनु॒ष्टुभ॑ मुत्तर॒तः प॒ङ्क्तिम् प॒ङ्क्ति मु॑त्तर॒तो॑ ऽनु॒ष्टुभ॑ मनु॒ष्टुभ॑ मुत्तर॒तः प॒ङ्क्तिम् । \newline
50. अ॒नु॒ष्टुभ॒मित्य॑नु - स्तुभ᳚म् । \newline
51. उ॒त्त॒र॒तः प॒ङ्क्तिम् प॒ङ्क्ति मु॑त्तर॒त उ॑त्तर॒तः प॒ङ्क्तिम् मद्ध्ये॒ मद्ध्ये॑ प॒ङ्क्ति मु॑त्तर॒त उ॑त्तर॒तः प॒ङ्क्तिम् मद्ध्ये᳚ । \newline
52. उ॒त्त॒र॒त इत्यु॑त् - त॒र॒तः । \newline
53. प॒ङ्क्तिम् मद्ध्ये॒ मद्ध्ये॑ प॒ङ्क्तिम् प॒ङ्क्तिम् मद्ध्य॑ ए॒ष ए॒ष मद्ध्ये॑ प॒ङ्क्तिम् प॒ङ्क्तिम् मद्ध्य॑ ए॒षः । \newline
54. मद्ध्य॑ ए॒ष ए॒ष मद्ध्ये॒ मद्ध्य॑ ए॒ष वै वा ए॒ष मद्ध्ये॒ मद्ध्य॑ ए॒ष वै । \newline
55. ए॒ष वै वा ए॒ष ए॒ष वा अ॒ग्नि र॒ग्निर् वा ए॒ष ए॒ष वा अ॒ग्निः । \newline
56. वा अ॒ग्नि र॒ग्निर् वै वा अ॒ग्निः स॒र्वतो॑मुखः स॒र्वतो॑मुखो॒ ऽग्निर् वै वा अ॒ग्निः स॒र्वतो॑मुखः । \newline
57. अ॒ग्निः स॒र्वतो॑मुखः स॒र्वतो॑मुखो॒ ऽग्नि र॒ग्निः स॒र्वतो॑मुख॒ स्तम् तꣳ स॒र्वतो॑मुखो॒ ऽग्नि र॒ग्निः स॒र्वतो॑मुख॒ स्तम् । \newline
58. स॒र्वतो॑मुख॒ स्तम् तꣳ स॒र्वतो॑मुखः स॒र्वतो॑मुख॒ स्तं ॅयो य स्तꣳ स॒र्वतो॑मुखः स॒र्वतो॑मुख॒ स्तं ॅयः । \newline
59. स॒र्वतो॑मुख॒ इति॑ स॒र्वतः॑ - मु॒खः॒ । \newline
60. तं ॅयो य स्तम् तं ॅय ए॒व मे॒वं ॅय स्तम् तं ॅय ए॒वम् । \newline
61. य ए॒व मे॒वं ॅयो य ए॒वं ॅवि॒द्वान्. वि॒द्वा ने॒वं ॅयो य ए॒वं ॅवि॒द्वान् । \newline
62. ए॒वं ॅवि॒द्वान्. वि॒द्वा ने॒व मे॒वं ॅवि॒द्वाꣳ श्चि॑नु॒ते चि॑नु॒ते वि॒द्वा ने॒व मे॒वं ॅवि॒द्वाꣳ श्चि॑नु॒ते । \newline
63. वि॒द्वाꣳ श्चि॑नु॒ते चि॑नु॒ते वि॒द्वान्. वि॒द्वाꣳ श्चि॑नु॒ते सर्वा॑सु॒ सर्वा॑सु चिनु॒ते वि॒द्वान्. वि॒द्वाꣳ श्चि॑नु॒ते सर्वा॑सु । \newline
64. चि॒नु॒ते सर्वा॑सु॒ सर्वा॑सु चिनु॒ते चि॑नु॒ते सर्वा॑सु प्र॒जासु॑ प्र॒जासु॒ सर्वा॑सु चिनु॒ते चि॑नु॒ते सर्वा॑सु प्र॒जासु॑ । \newline
65. सर्वा॑सु प्र॒जासु॑ प्र॒जासु॒ सर्वा॑सु॒ सर्वा॑सु प्र॒जा स्वन्न॒ मन्न॑म् प्र॒जासु॒ सर्वा॑सु॒ सर्वा॑सु प्र॒जा स्वन्न᳚म् । \newline
66. प्र॒जा स्वन्न॒ मन्न॑म् प्र॒जासु॑ प्र॒जा स्वन्न॑ मत्त्य॒त् त्यन्न॑म् प्र॒जासु॑ प्र॒जा स्वन्न॑ मत्ति । \newline
67. प्र॒जास्विति॑ प्र - जासु॑ । \newline
68. अन्न॑ मत्त्य॒त् त्यन्न॒ मन्न॑ मत्ति॒ सर्वाः॒ सर्वा॑ अ॒त्त्यन्न॒ मन्न॑ मत्ति॒ सर्वाः᳚ । \newline
69. अ॒त्ति॒ सर्वाः॒ सर्वा॑ अत्त्यत्ति॒ सर्वा॒ दिशो॒ दिशः॒ सर्वा॑ अत्त्यत्ति॒ सर्वा॒ दिशः॑ । \newline
70. सर्वा॒ दिशो॒ दिशः॒ सर्वाः॒ सर्वा॒ दिशो॒ ऽभ्य॑भि दिशः॒ सर्वाः॒ सर्वा॒ दिशो॒ ऽभि । \newline
71. दिशो॒ ऽभ्य॑भि दिशो॒ दिशो॒ ऽभि ज॑यति जय त्य॒भि दिशो॒ दिशो॒ ऽभि ज॑यति । \newline
72. अ॒भि ज॑यति जय त्य॒भ्य॑भि ज॑य॒ त्यथो॒ अथो॑ जय त्य॒भ्य॑भि ज॑य॒ त्यथो᳚ । \newline
73. ज॒य॒ त्यथो॒ अथो॑ जयति जय॒ त्यथो॑ दि॒शि दि॒श्यथो॑ जयति जय॒ त्यथो॑ दि॒शि । \newline
74. अथो॑ दि॒शि दि॒श्यथो॒ अथो॑ दि॒श्ये॑वैव दि॒श्यथो॒ अथो॑ दि॒श्ये॑व । \newline
75. अथो॒ इत्यथो᳚ । \newline
76. दि॒श्ये॑वैव दि॒शि दि॒श्ये॑व दिश॒म् दिश॑ मे॒व दि॒शि दि॒श्ये॑व दिश᳚म् । \newline
77. ए॒व दिश॒म् दिश॑ मे॒वैव दिश॒म् प्र प्र दिश॑ मे॒वैव दिश॒म् प्र । \newline
78. दिश॒म् प्र प्र दिश॒म् दिश॒म् प्र व॑यति वयति॒ प्र दिश॒म् दिश॒म् प्र व॑यति । \newline
79. प्र व॑यति वयति॒ प्र प्र व॑यति॒ तस्मा॒त् तस्मा᳚द् वयति॒ प्र प्र व॑यति॒ तस्मा᳚त् । \newline
80. व॒य॒ति॒ तस्मा॒त् तस्मा᳚द् वयति वयति॒ तस्मा᳚द् दि॒शि दि॒शि तस्मा᳚द् वयति वयति॒ तस्मा᳚द् दि॒शि । \newline
81. तस्मा᳚द् दि॒शि दि॒शि तस्मा॒त् तस्मा᳚द् दि॒शि दिग् दिग् दि॒शि तस्मा॒त् तस्मा᳚द् दि॒शि दिक् । \newline
82. दि॒शि दिग् दिग् दि॒शि दि॒शि दिक् प्रोता॒ प्रोता॒ दिग् दि॒शि दि॒शि दिक् प्रोता᳚ । \newline
83. दिक् प्रोता॒ प्रोता॒ दिग् दिक् प्रोता᳚ । \newline
84. प्रोतेति॒ प्र - उ॒ता॒ । \newline
\pagebreak
\markright{ TS 5.7.10.1  \hfill https://www.vedavms.in \hfill}

\section{ TS 5.7.10.1 }

\textbf{TS 5.7.10.1 } \newline
\textbf{Samhita Paata} \newline

प्र॒जाप॑तिर॒ग्निम॑सृजत॒ सो᳚ऽस्माथ् सृ॒ष्टः प्राङ् प्राऽ*द्र॑व॒त् तस्मा॒ अश्वं॒ प्रत्या᳚स्य॒थ् स द॑क्षि॒णाऽऽव॑र्तत॒ तस्मै॑ वृ॒ष्णिं प्रत्या᳚स्य॒थ् स प्र॒त्यङ्ङाऽव॑र्तत॒ तस्मा॑ ऋष॒भं प्रत्या᳚स्य॒थ् स उद॒ङ्ङाऽव॑र्तत॒ तस्मै᳚ ब॒स्तं प्रत्या᳚स्य॒थ् स ऊ॒र्द्ध्वो᳚ऽद्रव॒त् तस्मै॒ पुरु॑षं॒ प्रत्या᳚स्य॒द्यत् प॑शुशी॒र्॒.षाण्यु॑प॒दधा॑ति स॒र्वत॑ ए॒वैन॑ - [  ] \newline

\textbf{Pada Paata} \newline

प्र॒जाप॑ति॒रिति॑ प्र॒जा - प॒तिः॒ । अ॒ग्निम् । अ॒सृ॒ज॒त॒ । सः । अ॒स्मा॒त् । सृ॒ष्टः । प्राङ् । प्रेति॑ । अ॒द्र॒व॒त् । तस्मै᳚ । अश्व᳚म् । प्रतीति॑ । आ॒स्य॒त् । सः । द॒क्षि॒णा । एति॑ । अ॒व॒र्त॒त॒ । तस्मै᳚ । वृ॒ष्णिम् । प्रतीति॑ । आ॒स्य॒त् । सः । प्र॒त्यङ् । एति॑ । अ॒व॒र्त॒त॒ । तस्मै᳚ । ऋ॒ष॒भम् । प्रतीति॑ । आ॒स्य॒त् । सः । उदङ्॑ । एति॑ । अ॒व॒र्त॒त॒ । तस्मै᳚ । ब॒स्तम् । प्रतीति॑ । आ॒स्य॒त् । सः । ऊ॒द्‌र्ध्वः । अ॒द्र॒व॒त् । तस्मै᳚ । पुरु॑षम् । प्रतीति॑ । आ॒स्य॒त् । यत् । प॒शु॒शी॒र्॒.षाणीति॑ पशु - शी॒र्॒.षाणि॑ । उ॒प॒दधा॒तीत्यु॑प - दधा॑ति । स॒र्वतः॑ । ए॒व । ए॒न॒म् ।  \newline


\textbf{Krama Paata} \newline

प्र॒जाप॑तिर॒ग्निम् । प्र॒जाप॑ति॒रिति॑ प्र॒जा - प॒तिः॒ । अ॒ग्निम॑सृजत । अ॒सृ॒ज॒त॒ सः । सो᳚ऽस्मात् । अ॒स्मा॒थ् सृ॒ष्टः । सृ॒ष्टः प्राङ्ङ् । प्राङ् प्र । प्राद्र॑वत् । अ॒द्र॒व॒त् तस्मै᳚ । तस्मा॒ अश्व᳚म् । अश्व॒म् प्रति॑ । प्रत्या᳚स्यत् । आ॒स्य॒थ् सः । स द॑क्षि॒णा । द॒क्षि॒णा ऽऽव॑र्तत । आ ऽव॑र्तत । अ॒व॒र्त॒त॒ तस्मै᳚ । तस्मै॑ वृ॒ष्णिम् । वृ॒ष्णिम् प्रति॑ । प्रत्या᳚स्यत् । आ॒स्य॒थ् सः । स प्र॒त्यङ्ङ् । प्र॒त्यङ्ङा । आऽव॑र्तत । अ॒व॒र्त॒त॒ तस्मै᳚ । तस्मा॑ ऋष॒भम् । ऋ॒ष॒भम् प्रति॑ । 
प्रत्या᳚स्यत् । आ॒स्य॒थ् सः । स उदङ्ङ्॑ । उद॒ङ्ङा । आऽव॑र्तत । अ॒व॒र्त॒त॒ तस्मै᳚ । तस्मै॑ ब॒स्तम् । ब॒स्तम् प्रति॑ । प्रत्या᳚स्यत् । आ॒स्य॒थ् सः । स ऊ॒र्द्ध्वः । ऊ॒र्द्ध्वो᳚ऽद्रवत् । अ॒द्र॒व॒त् तस्मै᳚ । तस्मै॒ पुरु॑षम् । पुरु॑ष॒म् प्रति॑ । प्रत्या᳚स्यत् । आ॒स्य॒द् यत् । यत् प॑शुशी॒र्॒.षाणि॑ । प॒शु॒शी॒र्॒.षाण्यु॑प॒दधा॑ति । प॒शु॒शी॒र्॒.षाणीति॑ पशु - शी॒र्॒.षाणि॑ । उ॒प॒दधा॑ति स॒र्वतः॑ । उ॒प॒दधा॒तीत्यु॑प - दधा॑ति । स॒र्वत॑ ए॒व । ए॒वैन᳚म् । ए॒न॒म॒व॒रुद्ध्य॑ \newline

\textbf{Jatai Paata} \newline

1. प्र॒जाप॑ति र॒ग्नि म॒ग्निम् प्र॒जाप॑तिः प्र॒जाप॑ति र॒ग्निम् । \newline
2. प्र॒जाप॑ति॒रिति॑ प्र॒जा - प॒तिः॒ । \newline
3. अ॒ग्नि म॑सृजता सृजता॒ग्नि म॒ग्नि म॑सृजत । \newline
4. अ॒सृ॒ज॒त॒ स सो॑ ऽसृजता सृजत॒ सः । \newline
5. सो᳚ ऽस्मा दस्मा॒थ् स सो᳚ ऽस्मात् । \newline
6. अ॒स्मा॒थ् सृ॒ष्टः सृ॒ष्टो᳚ ऽस्मा दस्माथ् सृ॒ष्टः । \newline
7. सृ॒ष्टः प्राङ् प्राङ् ख्सृ॒ष्टः सृ॒ष्टः प्राङ् । \newline
8. प्राङ् प्र प्र प्राङ् प्राङ् प्र । \newline
9. प्राद्र॑व दद्रव॒त् प्र प्राद्र॑वत् । \newline
10. अ॒द्र॒व॒त् तस्मै॒ तस्मा॑ अद्रव दद्रव॒त् तस्मै᳚ । \newline
11. तस्मा॒ अश्व॒ मश्व॒म् तस्मै॒ तस्मा॒ अश्व᳚म् । \newline
12. अश्व॒म् प्रति॒ प्रत्यश्व॒ मश्व॒म् प्रति॑ । \newline
13. प्रत्या᳚स्य दास्य॒त् प्रति॒ प्रत्या᳚स्यत् । \newline
14. आ॒स्य॒थ् स स आ᳚स्य दास्य॒थ् सः । \newline
15. स द॑क्षि॒णा द॑क्षि॒णा स स द॑क्षि॒णा । \newline
16. द॒क्षि॒णा ऽव॑र्तता वर्त॒ता द॑क्षि॒णा द॑क्षि॒णा ऽव॑र्तत । \newline
17. आ ऽव॑र्तता वर्त॒ता ऽव॑र्तत । \newline
18. अ॒व॒र्त॒त॒ तस्मै॒ तस्मा॑ अवर्तता वर्तत॒ तस्मै᳚ । \newline
19. तस्मै॑ वृ॒ष्णिं ॅवृ॒ष्णिम् तस्मै॒ तस्मै॑ वृ॒ष्णिम् । \newline
20. वृ॒ष्णिम् प्रति॒ प्रति॑ वृ॒ष्णिं ॅवृ॒ष्णिम् प्रति॑ । \newline
21. प्रत्या᳚स्य दास्य॒त् प्रति॒ प्रत्या᳚स्यत् । \newline
22. आ॒स्य॒थ् स स आ᳚स्य दास्य॒थ् सः । \newline
23. स प्र॒त्यङ् प्र॒त्यङ् ख्स स प्र॒त्यङ् । \newline
24. प्र॒त्यङ् आ प्र॒त्यङ् प्र॒त्यङ् आ । \newline
25. आ ऽव॑र्तता वर्त॒ता ऽव॑र्तत । \newline
26. अ॒व॒र्त॒त॒ तस्मै॒ तस्मा॑ अवर्तता वर्तत॒ तस्मै᳚ । \newline
27. तस्मा॑ ऋष॒भ मृ॑ष॒भम् तस्मै॒ तस्मा॑ ऋष॒भम् । \newline
28. ऋ॒ष॒भम् प्रति॒ प्रत्यृ॑ष॒भ मृ॑ष॒भम् प्रति॑ । \newline
29. प्रत्या᳚स्य दास्य॒त् प्रति॒ प्रत्या᳚स्यत् । \newline
30. आ॒स्य॒थ् स स आ᳚स्य दास्य॒थ् सः । \newline
31. स उद॒ङ् ङुद॒ङ् ख्स स उदङ्॑ । \newline
32. उद॒ङ् ओद॒ङ् ङुद॒ङ् आ । \newline
33. आ ऽव॑र्तता वर्त॒ता ऽव॑र्तत । \newline
34. अ॒व॒र्त॒त॒ तस्मै॒ तस्मा॑ अवर्तता वर्तत॒ तस्मै᳚ । \newline
35. तस्मै॑ ब॒स्तम् ब॒स्तम् तस्मै॒ तस्मै॑ ब॒स्तम् । \newline
36. ब॒स्तम् प्रति॒ प्रति॑ ब॒स्तम् ब॒स्तम् प्रति॑ । \newline
37. प्रत्या᳚स्य दास्य॒त् प्रति॒ प्रत्या᳚स्यत् । \newline
38. आ॒स्य॒थ् स स आ᳚स्य दास्य॒थ् सः । \newline
39. स ऊ॒र्द्ध्व ऊ॒र्द्ध्वः स स ऊ॒र्द्ध्वः । \newline
40. ऊ॒र्द्ध्वो᳚ ऽद्रव दद्रव दू॒र्द्ध्व ऊ॒र्द्ध्वो᳚ ऽद्रवत् । \newline
41. अ॒द्र॒व॒त् तस्मै॒ तस्मा॑ अद्रव दद्रव॒त् तस्मै᳚ । \newline
42. तस्मै॒ पुरु॑ष॒म् पुरु॑ष॒म् तस्मै॒ तस्मै॒ पुरु॑षम् । \newline
43. पुरु॑ष॒म् प्रति॒ प्रति॒ पुरु॑ष॒म् पुरु॑ष॒म् प्रति॑ । \newline
44. प्रत्या᳚स्य दास्य॒त् प्रति॒ प्रत्या᳚स्यत् । \newline
45. आ॒स्य॒द् यद् यदा᳚स्य दास्य॒द् यत् । \newline
46. यत् प॑शुशी॒र्॒.षाणि॑ पशुशी॒र्॒.षाणि॒ यद् यत् प॑शुशी॒र्॒.षाणि॑ । \newline
47. प॒शु॒शी॒र्॒.षा ण्यु॑प॒दधा᳚ त्युप॒दधा॑ति पशुशी॒र्॒.षाणि॑ पशुशी॒र्॒.षा ण्यु॑प॒दधा॑ति । \newline
48. प॒शु॒शी॒र्॒.षाणीति॑ पशु - शी॒र्॒.षाणि॑ । \newline
49. उ॒प॒दधा॑ति स॒र्वतः॑ स॒र्वत॑ उप॒दधा᳚ त्युप॒दधा॑ति स॒र्वतः॑ । \newline
50. उ॒प॒दधा॒तीत्यु॑प - दधा॑ति । \newline
51. स॒र्वत॑ ए॒वैव स॒र्वतः॑ स॒र्वत॑ ए॒व । \newline
52. ए॒वैन॑ मेन मे॒वै वैन᳚म् । \newline
53. ए॒न॒ म॒व॒रुद्ध्या॑ व॒रुद्ध्यै॑न मेन मव॒रुद्ध्य॑ । \newline

\textbf{Ghana Paata } \newline

1. प्र॒जाप॑ति र॒ग्नि म॒ग्निम् प्र॒जाप॑तिः प्र॒जाप॑ति र॒ग्नि म॑सृजता सृजता॒ग्निम् प्र॒जाप॑तिः प्र॒जाप॑ति र॒ग्नि म॑सृजत । \newline
2. प्र॒जाप॑ति॒रिति॑ प्र॒जा - प॒तिः॒ । \newline
3. अ॒ग्नि म॑सृजता सृजता॒ग्नि म॒ग्नि म॑सृजत॒ स सो॑ ऽसृजता॒ग्नि म॒ग्नि म॑सृजत॒ सः । \newline
4. अ॒सृ॒ज॒त॒ स सो॑ ऽसृजता सृजत॒ सो᳚ ऽस्मा दस्मा॒थ् सो॑ ऽसृजता सृजत॒ सो᳚ ऽस्मात् । \newline
5. सो᳚ ऽस्मा दस्मा॒थ् स सो᳚ ऽस्माथ् सृ॒ष्टः सृ॒ष्टो᳚ ऽस्मा॒थ् स सो᳚ ऽस्माथ् सृ॒ष्टः । \newline
6. अ॒स्मा॒थ् सृ॒ष्टः सृ॒ष्टो᳚ ऽस्मा दस्माथ् सृ॒ष्टः प्राङ् प्राङ् ख्सृ॒ष्टो᳚ ऽस्मा दस्माथ् सृ॒ष्टः प्राङ् । \newline
7. सृ॒ष्टः प्राङ् प्राङ् ख्सृ॒ष्टः सृ॒ष्टः प्राङ् प्र प्र प्राङ् ख्सृ॒ष्टः सृ॒ष्टः प्राङ् प्र । \newline
8. प्राङ् प्र प्र प्राङ् प्राङ् प्राद्र॑व दद्रव॒त् प्र प्राङ् प्राङ् प्राद्र॑वत् । \newline
9. प्राद्र॑व दद्रव॒त् प्र प्राद्र॑व॒त् तस्मै॒ तस्मा॑ अद्रव॒त् प्र प्राद्र॑व॒त् तस्मै᳚ । \newline
10. अ॒द्र॒व॒त् तस्मै॒ तस्मा॑ अद्रव दद्रव॒त् तस्मा॒ अश्व॒ मश्व॒म् तस्मा॑ अद्रव दद्रव॒त् तस्मा॒ अश्व᳚म् । \newline
11. तस्मा॒ अश्व॒ मश्व॒म् तस्मै॒ तस्मा॒ अश्व॒म् प्रति॒ प्रत्यश्व॒म् तस्मै॒ तस्मा॒ अश्व॒म् प्रति॑ । \newline
12. अश्व॒म् प्रति॒ प्रत्यश्व॒ मश्व॒म् प्रत्या᳚स्य दास्य॒त् प्रत्यश्व॒ मश्व॒म् प्रत्या᳚स्यत् । \newline
13. प्रत्या᳚स्य दास्य॒त् प्रति॒ प्रत्या᳚स्य॒थ् स स आ᳚स्य॒त् प्रति॒ प्रत्या᳚स्य॒थ् सः । \newline
14. आ॒स्य॒थ् स स आ᳚स्य दास्य॒थ् स द॑क्षि॒णा द॑क्षि॒णा स आ᳚स्य दास्य॒थ् स द॑क्षि॒णा । \newline
15. स द॑क्षि॒णा द॑क्षि॒णा स स द॑क्षि॒णा ऽव॑र्तता वर्त॒ता द॑क्षि॒णा स स द॑क्षि॒णा ऽव॑र्तत । \newline
16. द॒क्षि॒णा ऽव॑र्तता वर्त॒ता द॑क्षि॒णा द॑क्षि॒णा ऽव॑र्तत॒ तस्मै॒ तस्मा॑ अवर्त॒ता द॑क्षि॒णा द॑क्षि॒णा ऽव॑र्तत॒ तस्मै᳚ । \newline
17. आ ऽव॑र्तता वर्त॒ता ऽव॑र्तत॒ तस्मै॒ तस्मा॑ अवर्त॒ता ऽव॑र्तत॒ तस्मै᳚ । \newline
18. अ॒व॒र्त॒त॒ तस्मै॒ तस्मा॑ अवर्तता वर्तत॒ तस्मै॑ वृ॒ष्णिं ॅवृ॒ष्णिम् तस्मा॑ अवर्तता वर्तत॒ तस्मै॑ वृ॒ष्णिम् । \newline
19. तस्मै॑ वृ॒ष्णिं ॅवृ॒ष्णिम् तस्मै॒ तस्मै॑ वृ॒ष्णिम् प्रति॒ प्रति॑ वृ॒ष्णिम् तस्मै॒ तस्मै॑ वृ॒ष्णिम् प्रति॑ । \newline
20. वृ॒ष्णिम् प्रति॒ प्रति॑ वृ॒ष्णिं ॅवृ॒ष्णिम् प्रत्या᳚स्य दास्य॒त् प्रति॑ वृ॒ष्णिं ॅवृ॒ष्णिम् प्रत्या᳚स्यत् । \newline
21. प्रत्या᳚स्य दास्य॒त् प्रति॒ प्रत्या᳚स्य॒थ् स स आ᳚स्य॒त् प्रति॒ प्रत्या᳚स्य॒थ् सः । \newline
22. आ॒स्य॒थ् स स आ᳚स्य दास्य॒थ् स प्र॒त्यङ् प्र॒त्यङ् ख्स आ᳚स्य दास्य॒थ् स प्र॒त्यङ् । \newline
23. स प्र॒त्यङ् प्र॒त्यङ् ख्स स प्र॒त्यङा प्र॒त्यङ् ख्स स प्र॒त्यङा । \newline
24. प्र॒त्यङा प्र॒त्यङ् प्र॒त्यङा ऽव॑र्तता वर्त॒ता प्र॒त्यङ् प्र॒त्यङा ऽव॑र्तत । \newline
25. आ ऽव॑र्तता वर्त॒ता ऽव॑र्तत॒ तस्मै॒ तस्मा॑ अवर्त॒ता ऽव॑र्तत॒ तस्मै᳚ । \newline
26. अ॒व॒र्त॒त॒ तस्मै॒ तस्मा॑ अवर्तता वर्तत॒ तस्मा॑ ऋष॒भ मृ॑ष॒भम् तस्मा॑ अवर्तता वर्तत॒ तस्मा॑ ऋष॒भम् । \newline
27. तस्मा॑ ऋष॒भ मृ॑ष॒भम् तस्मै॒ तस्मा॑ ऋष॒भम् प्रति॒ प्रत्यृ॑ष॒भम् तस्मै॒ तस्मा॑ ऋष॒भम् प्रति॑ । \newline
28. ऋ॒ष॒भम् प्रति॒ प्रत्यृ॑ष॒भ मृ॑ष॒भम् प्रत्या᳚स्य दास्य॒त् प्रत्यृ॑ष॒भ मृ॑ष॒भम् प्रत्या᳚स्यत् । \newline
29. प्रत्या᳚स्य दास्य॒त् प्रति॒ प्रत्या᳚स्य॒थ् स स आ᳚स्य॒त् प्रति॒ प्रत्या᳚स्य॒थ् सः । \newline
30. आ॒स्य॒थ् स स आ᳚स्य दास्य॒थ् स उद॒ङ् ङुद॒ङ् ख्स आ᳚स्य दास्य॒थ् स उदङ्॑ । \newline
31. स उद॒ङ् ङुद॒ङ् ख्स स उद॒ ङोद॒ङ् ख्स स उद॒ङा । \newline
32. उद॒ ङोद॒ङ् ङुद॒ङा ऽव॑र्तता वर्त॒ तोद॒ङ् ङुद॒ङा ऽव॑र्तत । \newline
33. आ ऽव॑र्तता वर्त॒ता ऽव॑र्तत॒ तस्मै॒ तस्मा॑ अवर्त॒ता ऽव॑र्तत॒ तस्मै᳚ । \newline
34. अ॒व॒र्त॒त॒ तस्मै॒ तस्मा॑ अवर्तता वर्तत॒ तस्मै॑ ब॒स्तम् ब॒स्तम् तस्मा॑ अवर्तता वर्तत॒ तस्मै॑ ब॒स्तम् । \newline
35. तस्मै॑ ब॒स्तम् ब॒स्तम् तस्मै॒ तस्मै॑ ब॒स्तम् प्रति॒ प्रति॑ ब॒स्तम् तस्मै॒ तस्मै॑ ब॒स्तम् प्रति॑ । \newline
36. ब॒स्तम् प्रति॒ प्रति॑ ब॒स्तम् ब॒स्तम् प्रत्या᳚स्य दास्य॒त् प्रति॑ ब॒स्तम् ब॒स्तम् प्रत्या᳚स्यत् । \newline
37. प्रत्या᳚स्य दास्य॒त् प्रति॒ प्रत्या᳚स्य॒थ् स स आ᳚स्य॒त् प्रति॒ प्रत्या᳚स्य॒थ् सः । \newline
38. आ॒स्य॒थ् स स आ᳚स्य दास्य॒थ् स ऊ॒र्द्ध्व ऊ॒र्द्ध्वः स आ᳚स्य दास्य॒थ् स ऊ॒र्द्ध्वः । \newline
39. स ऊ॒र्द्ध्व ऊ॒र्द्ध्वः स स ऊ॒र्द्ध्वो᳚ ऽद्रव दद्रव दू॒र्द्ध्वः स स ऊ॒र्द्ध्वो᳚ ऽद्रवत् । \newline
40. ऊ॒र्द्ध्वो᳚ ऽद्रव दद्रव दू॒र्द्ध्व ऊ॒र्द्ध्वो᳚ ऽद्रव॒त् तस्मै॒ तस्मा॑ अद्रव दू॒र्द्ध्व ऊ॒र्द्ध्वो᳚ ऽद्रव॒त् तस्मै᳚ । \newline
41. अ॒द्र॒व॒त् तस्मै॒ तस्मा॑ अद्रव दद्रव॒त् तस्मै॒ पुरु॑ष॒म् पुरु॑ष॒म् तस्मा॑ अद्रव दद्रव॒त् तस्मै॒ पुरु॑षम् । \newline
42. तस्मै॒ पुरु॑ष॒म् पुरु॑ष॒म् तस्मै॒ तस्मै॒ पुरु॑ष॒म् प्रति॒ प्रति॒ पुरु॑ष॒म् तस्मै॒ तस्मै॒ पुरु॑ष॒म् प्रति॑ । \newline
43. पुरु॑ष॒म् प्रति॒ प्रति॒ पुरु॑ष॒म् पुरु॑ष॒म् प्रत्या᳚स्य दास्य॒त् प्रति॒ पुरु॑ष॒म् पुरु॑ष॒म् प्रत्या᳚स्यत् । \newline
44. प्रत्या᳚स्य दास्य॒त् प्रति॒ प्रत्या᳚स्य॒द् यद् यदा᳚स्य॒त् प्रति॒ प्रत्या᳚स्य॒द् यत् । \newline
45. आ॒स्य॒द् यद् यदा᳚स्य दास्य॒द् यत् प॑शुशी॒र्॒.षाणि॑ पशुशी॒र्॒.षाणि॒ यदा᳚स्य दास्य॒द् यत् प॑शुशी॒र्॒.षाणि॑ । \newline
46. यत् प॑शुशी॒र्॒.षाणि॑ पशुशी॒र्॒.षाणि॒ यद् यत् प॑शुशी॒र्॒.षा ण्यु॑प॒दधा᳚ त्युप॒दधा॑ति पशुशी॒र्॒.षाणि॒ यद् यत् प॑शुशी॒र्॒.षा ण्यु॑प॒दधा॑ति । \newline
47. प॒शु॒शी॒र्॒.षा ण्यु॑प॒दधा᳚ त्युप॒दधा॑ति पशुशी॒र्॒.षाणि॑ पशुशी॒र्॒.षा ण्यु॑प॒दधा॑ति स॒र्वतः॑ स॒र्वत॑ उप॒दधा॑ति पशुशी॒र्॒.षाणि॑ पशुशी॒र्॒.षा ण्यु॑प॒दधा॑ति स॒र्वतः॑ । \newline
48. प॒शु॒शी॒र्॒.षाणीति॑ पशु - शी॒र्॒.षाणि॑ । \newline
49. उ॒प॒दधा॑ति स॒र्वतः॑ स॒र्वत॑ उप॒दधा᳚ त्युप॒दधा॑ति स॒र्वत॑ ए॒वैव स॒र्वत॑ उप॒दधा᳚ त्युप॒दधा॑ति स॒र्वत॑ ए॒व । \newline
50. उ॒प॒दधा॒तीत्यु॑प - दधा॑ति । \newline
51. स॒र्वत॑ ए॒वैव स॒र्वतः॑ स॒र्वत॑ ए॒वैन॑ मेन मे॒व स॒र्वतः॑ स॒र्वत॑ ए॒वैन᳚म् । \newline
52. ए॒वैन॑ मेन मे॒वै वैन॑ मव॒रुद्ध्या॑ व॒रुद्ध्यै॑न मे॒वै वैन॑ मव॒रुद्ध्य॑ । \newline
53. ए॒न॒ म॒व॒रुद्ध्या॑ व॒रुद्ध्यै॑न मेन मव॒रुद्ध्य॑ चिनुते चिनुते ऽव॒रुद्ध्यै॑न मेन मव॒रुद्ध्य॑ चिनुते । \newline
\pagebreak
\markright{ TS 5.7.10.2  \hfill https://www.vedavms.in \hfill}

\section{ TS 5.7.10.2 }

\textbf{TS 5.7.10.2 } \newline
\textbf{Samhita Paata} \newline

मव॒रुद्ध्य॑ चिनुत ए॒ता वै प्रा॑ण॒भृत॒-श्चक्षु॑ष्मती॒रिष्ट॑का॒ यत् प॑शुशी॒र्.षाणि॒ यत् प॑शुशी॒र्.षाण्यु॑प॒दधा॑ति॒ ताभि॑रे॒व यज॑मानो॒ऽमुष्मि॑न् ॅलो॒के प्राणि॒त्यथो॒ ताभि॑रे॒वास्मा॑ इ॒मे लो॒काः प्र भा᳚न्ति मृ॒दाऽभि॒लिप्योप॑ दधाति मेद्ध्य॒त्वाय॑ प॒शुर्वा ए॒ष यद॒ग्निरन्नं॑ प॒शव॑ ए॒ष खलु॒ वा अ॒ग्निर्यत् प॑शुशी॒र्॒.षाणि॒ यं का॒मये॑त॒ कनी॑यो॒ऽस्यान्नꣳ॑ - [  ] \newline

\textbf{Pada Paata} \newline

अ॒व॒रुद्ध्येत्य॑व - रुद्ध्य॑ । चि॒नु॒ते॒ । ए॒ताः । वै । प्रा॒ण॒भृत॒ इति॑ प्राण - भृतः॑ । चक्षु॑ष्मतीः । इष्ट॑काः । यत् । प॒शु॒शी॒र्.॒षाणीति॑ पशु - शी॒र्.॒षाणि॑ । यत् । प॒शु॒शी॒र्.॒षाणीति॑ पशु - शी॒र्.॒षाणि॑ । उ॒प॒दधा॒तीत्यु॑प - दधा॑ति । ताभिः॑ । ए॒व । यज॑मानः । अ॒मुष्मिन्न्॑ । लो॒के । प्रेति॑ । अ॒नि॒ति॒ । अथो॒ इति॑ । ताभिः॑ । ए॒व । अ॒स्मै॒ । इ॒मे । लो॒काः । प्रेति॑ । भा॒न्ति॒ । मृ॒दा । अ॒भि॒लिप्येत्य॑भि - लिप्य॑ । उपेति॑ । द॒धा॒ति॒ । मे॒द्ध्य॒त्वायेति॑ मेद्ध्य-त्वाय॑ । प॒शुः । वै । ए॒षः । यत् । अ॒ग्निः । अन्न᳚म् । प॒शवः॑ । ए॒षः । खलु॑ । वै । अ॒ग्निः । यत् । प॒शु॒शी॒र्.॒षाणीति॑ पशु-शी॒र्.॒षाणि॑ । यम् । का॒मये॑त । कनी॑यः । अ॒स्य॒ । अन्न᳚म् ।  \newline


\textbf{Krama Paata} \newline

अ॒व॒रुद्ध्य॑ चिनुते । अ॒व॒रुद्ध्येत्य॑व - रुद्ध्य॑ । चि॒नु॒त॒ ए॒ताः । ए॒ता वै । वै प्रा॑ण॒भृतः॑ । पा॒ण॒भृत॒श्चक्षु॑ष्मतीः । प्रा॒ण॒भृत॒ इति॑ प्राण - भृतः॑ । चक्षु॑ष्मती॒रिष्ट॑काः । इष्ट॑का॒ यत् । यत् प॑शुशी॒र्॒.षाणि॑ । प॒शु॒शी॒र्॒.षाणि॒ यत् । प॒शु॒शी॒र्॒.षाणीति॑ पशु - शी॒र्॒.षाणि॑ । यत् प॑शुशी॒र्॒.षाणि॑ । प॒शु॒शी॒र्॒.षाण्यु॑प॒दधा॑ति । प॒शु॒शी॒र्॒.षाणीति॑ पशु - शी॒र्॒.षाणि॑ । उ॒प॒दधा॑ति॒ ताभिः॑ । उ॒प॒दधा॒तीत्यु॑प - दधा॑ति । ताभि॑रे॒व । ए॒व यज॑मानः । यज॑मानो॒ऽमुष्मिन्न्॑ । अ॒मुष्मि॑न् ॅलो॒के । लो॒के प्र । प्राणि॑ति । अ॒नि॒त्यथो᳚ । अथो॒ ताभिः॑ । अथो॒ इत्यथो᳚ । ताभि॑रे॒व । ए॒वास्मै᳚ । अ॒स्मा॒ इ॒मे । इ॒मे लो॒काः । लो॒काः प्र । प्र भा᳚न्ति । भा॒न्ति॒ मृ॒दा । मृ॒दाऽभि॒लिप्य॑ । अ॒भि॒लिप्योप॑ । अ॒भि॒लिप्येत्य॑भि - लिप्य॑ । उप॑ दधाति । द॒धा॒ति॒ मे॒द्ध्य॒त्वाय॑ । मे॒द्ध्य॒त्वाय॑ प॒शुः । मे॒द्ध्य॒त्वायेति॑ मेद्ध्य - त्वाय॑ । प॒शुर् वै । वा ए॒षः । ए॒ष यत् । यद॒ग्निः । अ॒ग्निरन्न᳚म् । अन्न॑म् प॒शवः॑ । प॒शव॑ ए॒षः । ए॒ष खलु॑ । खलु॒ वै । वा अ॒ग्निः । अ॒ग्निर् यत् । यत् प॑शुशी॒र्॒.षाणि॑ । प॒शु॒शी॒र्॒.षाणि॒ यम् । प॒शु॒शी॒र्॒.षाणीति॑ पशु - शी॒र्॒.षाणि॑ । यम् का॒मये॑त । का॒मये॑त॒ कनी॑यः । कनी॑योऽस्य । अ॒स्यान्न᳚म् । अन्नꣳ॑ स्यात् \newline

\textbf{Jatai Paata} \newline

1. अ॒व॒रुद्ध्य॑ चिनुते चिनुते ऽव॒रुद्ध्या॑ व॒रुद्ध्य॑ चिनुते । \newline
2. अ॒व॒रुद्ध्येत्य॑व - रुद्ध्य॑ । \newline
3. चि॒नु॒त॒ ए॒ता ए॒ता श्चि॑नुते चिनुत ए॒ताः । \newline
4. ए॒ता वै वा ए॒ता ए॒ता वै । \newline
5. वै प्रा॑ण॒भृतः॑ प्राण॒भृतो॒ वै वै प्रा॑ण॒भृतः॑ । \newline
6. प्रा॒ण॒भृत॒ श्चक्षु॑ष्मती॒ श्चक्षु॑ष्मतीः प्राण॒भृतः॑ प्राण॒भृत॒ श्चक्षु॑ष्मतीः । \newline
7. प्रा॒ण॒भृत॒ इति॑ प्राण - भृतः॑ । \newline
8. चक्षु॑ष्मती॒ रिष्ट॑का॒ इष्ट॑का॒ श्चक्षु॑ष्मती॒ श्चक्षु॑ष्मती॒ रिष्ट॑काः । \newline
9. इष्ट॑का॒ यद् यदिष्ट॑का॒ इष्ट॑का॒ यत् । \newline
10. यत् प॑शुशी॒र्॒.षाणि॑ पशुशी॒र्॒.षाणि॒ यद् यत् प॑शुशी॒र्॒.षाणि॑ । \newline
11. प॒शु॒शी॒र्॒.षाणि॒ यद् यत् प॑शुशी॒र्॒.षाणि॑ पशुशी॒र्॒.षाणि॒ यत् । \newline
12. प॒शु॒शी॒र्.॒षाणीति॑ पशु - शी॒र्.॒षाणि॑ । \newline
13. यत् प॑शुशी॒र्॒.षाणि॑ पशुशी॒र्॒.षाणि॒ यद् यत् प॑शुशी॒र्॒.षाणि॑ । \newline
14. प॒शु॒शी॒र्॒.षा ण्यु॑प॒दधा᳚ त्युप॒दधा॑ति पशुशी॒र्॒.षाणि॑ पशुशी॒र्॒.षा ण्यु॑प॒दधा॑ति । \newline
15. प॒शु॒शी॒र्.॒षाणीति॑ पशु - शी॒र्.॒षाणि॑ । \newline
16. उ॒प॒दधा॑ति॒ ताभि॒ स्ताभि॑ रुप॒दधा᳚ त्युप॒दधा॑ति॒ ताभिः॑ । \newline
17. उ॒प॒दधा॒तीत्यु॑प - दधा॑ति । \newline
18. ताभि॑ रे॒वैव ताभि॒ स्ताभि॑ रे॒व । \newline
19. ए॒व यज॑मानो॒ यज॑मान ए॒वैव यज॑मानः । \newline
20. यज॑मानो॒ ऽमुष्मि॑न् न॒मुष्मि॒न्॒. यज॑मानो॒ यज॑मानो॒ ऽमुष्मिन्न्॑ । \newline
21. अ॒मुष्मि॑न् ॅलो॒के लो॒के॑ ऽमुष्मि॑न् न॒मुष्मि॑न् ॅलो॒के । \newline
22. लो॒के प्र प्र लो॒के लो॒के प्र । \newline
23. प्राणि॑ त्यनिति॒ प्र प्राणि॑ति । \newline
24. अ॒नि॒ त्यथो॒ अथो॑ अनि त्यनि॒ त्यथो᳚ । \newline
25. अथो॒ ताभि॒ स्ताभि॒ रथो॒ अथो॒ ताभिः॑ । \newline
26. अथो॒ इत्यथो᳚ । \newline
27. ताभि॑ रे॒वैव ताभि॒ स्ताभि॑ रे॒व । \newline
28. ए॒वास्मा॑ अस्मा ए॒वै वास्मै᳚ । \newline
29. अ॒स्मा॒ इ॒म इ॒मे᳚ ऽस्मा अस्मा इ॒मे । \newline
30. इ॒मे लो॒का लो॒का इ॒म इ॒मे लो॒काः । \newline
31. लो॒काः प्र प्र लो॒का लो॒काः प्र । \newline
32. प्र भा᳚न्ति भान्ति॒ प्र प्र भा᳚न्ति । \newline
33. भा॒न्ति॒ मृ॒दा मृ॒दा भा᳚न्ति भान्ति मृ॒दा । \newline
34. मृ॒दा ऽभि॒लिप्या॑ भि॒लिप्य॑ मृ॒दा मृ॒दा ऽभि॒लिप्य॑ । \newline
35. अ॒भि॒लिप्यो पोपा॑भि॒लिप्या॑ भि॒लिप्योप॑ । \newline
36. अ॒भि॒लिप्येत्य॑भि - लिप्य॑ । \newline
37. उप॑ दधाति दधा॒ त्युपोप॑ दधाति । \newline
38. द॒धा॒ति॒ मे॒द्ध्य॒त्वाय॑ मेद्ध्य॒त्वाय॑ दधाति दधाति मेद्ध्य॒त्वाय॑ । \newline
39. मे॒द्ध्य॒त्वाय॑ प॒शुः प॒शुर् मे᳚द्ध्य॒त्वाय॑ मेद्ध्य॒त्वाय॑ प॒शुः । \newline
40. मे॒द्ध्य॒त्वायेति॑ मेद्ध्य - त्वाय॑ । \newline
41. प॒शुर् वै वै प॒शुः प॒शुर् वै । \newline
42. वा ए॒ष ए॒ष वै वा ए॒षः । \newline
43. ए॒ष यद् यदे॒ष ए॒ष यत् । \newline
44. यद॒ग्नि र॒ग्निर् यद् यद॒ग्निः । \newline
45. अ॒ग्नि रन्न॒ मन्न॑ म॒ग्नि र॒ग्नि रन्न᳚म् । \newline
46. अन्न॑म् प॒शवः॑ प॒शवो ऽन्न॒ मन्न॑म् प॒शवः॑ । \newline
47. प॒शव॑ ए॒ष ए॒ष प॒शवः॑ प॒शव॑ ए॒षः । \newline
48. ए॒ष खलु॒ खल्वे॒ष ए॒ष खलु॑ । \newline
49. खलु॒ वै वै खलु॒ खलु॒ वै । \newline
50. वा अ॒ग्नि र॒ग्निर् वै वा अ॒ग्निः । \newline
51. अ॒ग्निर् यद् यद॒ग्नि र॒ग्निर् यत् । \newline
52. यत् प॑शुशी॒र्॒.षाणि॑ पशुशी॒र्॒.षाणि॒ यद् यत् प॑शुशी॒र्॒.षाणि॑ । \newline
53. प॒शु॒शी॒र्॒.षाणि॒ यं ॅयम् प॑शुशी॒र्॒.षाणि॑ पशुशी॒र्॒.षाणि॒ यम् । \newline
54. प॒शु॒शी॒र्.॒षाणीति॑ पशु - शी॒र्.॒षाणि॑ । \newline
55. यम् का॒मये॑त का॒मये॑त॒ यं ॅयम् का॒मये॑त । \newline
56. का॒मये॑त॒ कनी॑यः॒ कनी॑यः का॒मये॑त का॒मये॑त॒ कनी॑यः । \newline
57. कनी॑यो ऽस्यास्य॒ कनी॑यः॒ कनी॑यो ऽस्य । \newline
58. अ॒स्यान्न॒ मन्न॑ मस्या॒ स्यान्न᳚म् । \newline
59. अन्नꣳ॑ स्याथ् स्या॒ दन्न॒ मन्नꣳ॑ स्यात् । \newline

\textbf{Ghana Paata } \newline

1. अ॒व॒रुद्ध्य॑ चिनुते चिनुते ऽव॒रुद्ध्या॑ व॒रुद्ध्य॑ चिनुत ए॒ता ए॒ता श्चि॑नुते ऽव॒रुद्ध्या॑ व॒रुद्ध्य॑ चिनुत ए॒ताः । \newline
2. अ॒व॒रुद्ध्येत्य॑व - रुद्ध्य॑ । \newline
3. चि॒नु॒त॒ ए॒ता ए॒ता श्चि॑नुते चिनुत ए॒ता वै वा ए॒ता श्चि॑नुते चिनुत ए॒ता वै । \newline
4. ए॒ता वै वा ए॒ता ए॒ता वै प्रा॑ण॒भृतः॑ प्राण॒भृतो॒ वा ए॒ता ए॒ता वै प्रा॑ण॒भृतः॑ । \newline
5. वै प्रा॑ण॒भृतः॑ प्राण॒भृतो॒ वै वै प्रा॑ण॒भृत॒ श्चक्षु॑ष्मती॒ श्चक्षु॑ष्मतीः प्राण॒भृतो॒ वै वै प्रा॑ण॒भृत॒ श्चक्षु॑ष्मतीः । \newline
6. प्रा॒ण॒भृत॒ श्चक्षु॑ष्मती॒ श्चक्षु॑ष्मतीः प्राण॒भृतः॑ प्राण॒भृत॒ श्चक्षु॑ष्मती॒ रिष्ट॑का॒ इष्ट॑का॒ श्चक्षु॑ष्मतीः प्राण॒भृतः॑ प्राण॒भृत॒ श्चक्षु॑ष्मती॒ रिष्ट॑काः । \newline
7. प्रा॒ण॒भृत॒ इति॑ प्राण - भृतः॑ । \newline
8. चक्षु॑ष्मती॒ रिष्ट॑का॒ इष्ट॑का॒ श्चक्षु॑ष्मती॒ श्चक्षु॑ष्मती॒ रिष्ट॑का॒ यद् यदिष्ट॑का॒ श्चक्षु॑ष्मती॒ श्चक्षु॑ष्मती॒ रिष्ट॑का॒ यत् । \newline
9. इष्ट॑का॒ यद् यदिष्ट॑का॒ इष्ट॑का॒ यत् प॑शुशी॒र्॒.षाणि॑ पशुशी॒र्॒.षाणि॒ यदिष्ट॑का॒ इष्ट॑का॒ यत् प॑शुशी॒र्॒.षाणि॑ । \newline
10. यत् प॑शुशी॒र्॒.षाणि॑ पशुशी॒र्॒.षाणि॒ यद् यत् प॑शुशी॒र्॒.षाणि॒ यद् यत् प॑शुशी॒र्॒.षाणि॒ यद् यत् प॑शुशी॒र्॒.षाणि॒ यत् । \newline
11. प॒शु॒शी॒र्॒.षाणि॒ यद् यत् प॑शुशी॒र्॒.षाणि॑ पशुशी॒र्॒.षाणि॒ यत् प॑शुशी॒र्॒.षाणि॑ पशुशी॒र्॒.षाणि॒ यत् प॑शुशी॒र्॒.षाणि॑ पशुशी॒र्॒.षाणि॒ यत् प॑शुशी॒र्॒.षाणि॑ । \newline
12. प॒शु॒शी॒र्.॒षाणीति॑ पशु - शी॒र्.॒षाणि॑ । \newline
13. यत् प॑शुशी॒र्॒.षाणि॑ पशुशी॒र्॒.षाणि॒ यद् यत् प॑शुशी॒र्॒.षा ण्यु॑प॒दधा᳚ त्युप॒दधा॑ति पशुशी॒र्॒.षाणि॒ यद् यत् प॑शुशी॒र्॒.षा ण्यु॑प॒दधा॑ति । \newline
14. प॒शु॒शी॒र्॒.षा ण्यु॑प॒दधा᳚ त्युप॒दधा॑ति पशुशी॒र्॒.षाणि॑ पशुशी॒र्॒.षा ण्यु॑प॒दधा॑ति॒ ताभि॒ स्ताभि॑ रुप॒दधा॑ति पशुशी॒र्॒.षाणि॑ पशुशी॒र्॒.षा ण्यु॑प॒दधा॑ति॒ ताभिः॑ । \newline
15. प॒शु॒शी॒र्.॒षाणीति॑ पशु - शी॒र्.॒षाणि॑ । \newline
16. उ॒प॒दधा॑ति॒ ताभि॒ स्ताभि॑ रुप॒दधा᳚ त्युप॒दधा॑ति॒ ताभि॑ रे॒वैव ताभि॑ रुप॒दधा᳚ त्युप॒दधा॑ति॒ ताभि॑ रे॒व । \newline
17. उ॒प॒दधा॒तीत्यु॑प - दधा॑ति । \newline
18. ताभि॑ रे॒वैव ताभि॒ स्ताभि॑ रे॒व यज॑मानो॒ यज॑मान ए॒व ताभि॒ स्ताभि॑ रे॒व यज॑मानः । \newline
19. ए॒व यज॑मानो॒ यज॑मान ए॒वैव यज॑मानो॒ ऽमुष्मि॑न् न॒मुष्मि॒न्॒. यज॑मान ए॒वैव यज॑मानो॒ ऽमुष्मिन्न्॑ । \newline
20. यज॑मानो॒ ऽमुष्मि॑न् न॒मुष्मि॒न्॒. यज॑मानो॒ यज॑मानो॒ ऽमुष्मि॑न् ॅलो॒के लो॒के॑ ऽमुष्मि॒न्॒. यज॑मानो॒ यज॑मानो॒ ऽमुष्मि॑न् ॅलो॒के । \newline
21. अ॒मुष्मि॑न् ॅलो॒के लो॒के॑ ऽमुष्मि॑न् न॒मुष्मि॑न् ॅलो॒के प्र प्र लो॒के॑ ऽमुष्मि॑न् न॒मुष्मि॑न् ॅलो॒के प्र । \newline
22. लो॒के प्र प्र लो॒के लो॒के प्राणि॑ त्यनिति॒ प्र लो॒के लो॒के प्राणि॑ति । \newline
23. प्राणि॑ त्यनिति॒ प्र प्राणि॒ त्यथो॒ अथो॑ अनिति॒ प्र प्राणि॒ त्यथो᳚ । \newline
24. अ॒नि॒ त्यथो॒ अथो॑ अनि त्यनि॒ त्यथो॒ ताभि॒ स्ताभि॒ रथो॑ अनि त्यनि॒ त्यथो॒ ताभिः॑ । \newline
25. अथो॒ ताभि॒ स्ताभि॒ रथो॒ अथो॒ ताभि॑ रे॒वैव ताभि॒ रथो॒ अथो॒ ताभि॑ रे॒व । \newline
26. अथो॒ इत्यथो᳚ । \newline
27. ताभि॑ रे॒वैव ताभि॒ स्ताभि॑ रे॒वास्मा॑ अस्मा ए॒व ताभि॒ स्ताभि॑ रे॒वास्मै᳚ । \newline
28. ए॒वास्मा॑ अस्मा ए॒वै वास्मा॑ इ॒म इ॒मे᳚ ऽस्मा ए॒वै वास्मा॑ इ॒मे । \newline
29. अ॒स्मा॒ इ॒म इ॒मे᳚ ऽस्मा अस्मा इ॒मे लो॒का लो॒का इ॒मे᳚ ऽस्मा अस्मा इ॒मे लो॒काः । \newline
30. इ॒मे लो॒का लो॒का इ॒म इ॒मे लो॒काः प्र प्र लो॒का इ॒म इ॒मे लो॒काः प्र । \newline
31. लो॒काः प्र प्र लो॒का लो॒काः प्र भा᳚न्ति भान्ति॒ प्र लो॒का लो॒काः प्र भा᳚न्ति । \newline
32. प्र भा᳚न्ति भान्ति॒ प्र प्र भा᳚न्ति मृ॒दा मृ॒दा भा᳚न्ति॒ प्र प्र भा᳚न्ति मृ॒दा । \newline
33. भा॒न्ति॒ मृ॒दा मृ॒दा भा᳚न्ति भान्ति मृ॒दा ऽभि॒लिप्या॑ भि॒लिप्य॑ मृ॒दा भा᳚न्ति भान्ति मृ॒दा ऽभि॒लिप्य॑ । \newline
34. मृ॒दा ऽभि॒लिप्या॑ भि॒लिप्य॑ मृ॒दा मृ॒दा ऽभि॒लिप् योपोपा॑ भि॒लिप्य॑ मृ॒दा मृ॒दा ऽभि॒लिप्योप॑ । \newline
35. अ॒भि॒लिप्योपोपा॑ भि॒लिप्या॑ भि॒लिप्योप॑ दधाति दधा॒ त्युपा॑ भि॒लिप्या॑ भि॒लिप्योप॑ दधाति । \newline
36. अ॒भि॒लिप्येत्य॑भि - लिप्य॑ । \newline
37. उप॑ दधाति दधा॒ त्युपोप॑ दधाति मेद्ध्य॒त्वाय॑ मेद्ध्य॒त्वाय॑ दधा॒ त्युपोप॑ दधाति मेद्ध्य॒त्वाय॑ । \newline
38. द॒धा॒ति॒ मे॒द्ध्य॒त्वाय॑ मेद्ध्य॒त्वाय॑ दधाति दधाति मेद्ध्य॒त्वाय॑ प॒शुः प॒शुर् मे᳚द्ध्य॒त्वाय॑ दधाति दधाति मेद्ध्य॒त्वाय॑ प॒शुः । \newline
39. मे॒द्ध्य॒त्वाय॑ प॒शुः प॒शुर् मे᳚द्ध्य॒त्वाय॑ मेद्ध्य॒त्वाय॑ प॒शुर् वै वै प॒शुर् मे᳚द्ध्य॒त्वाय॑ मेद्ध्य॒त्वाय॑ प॒शुर् वै । \newline
40. मे॒द्ध्य॒त्वायेति॑ मेद्ध्य - त्वाय॑ । \newline
41. प॒शुर् वै वै प॒शुः प॒शुर् वा ए॒ष ए॒ष वै प॒शुः प॒शुर् वा ए॒षः । \newline
42. वा ए॒ष ए॒ष वै वा ए॒ष यद् यदे॒ष वै वा ए॒ष यत् । \newline
43. ए॒ष यद् यदे॒ष ए॒ष यद॒ग्नि र॒ग्निर् यदे॒ष ए॒ष यद॒ग्निः । \newline
44. यद॒ग्नि र॒ग्निर् यद् यद॒ग्नि रन्न॒ मन्न॑ म॒ग्निर् यद् यद॒ग्नि रन्न᳚म् । \newline
45. अ॒ग्नि रन्न॒ मन्न॑ म॒ग्नि र॒ग्नि रन्न॑म् प॒शवः॑ प॒शवो ऽन्न॑ म॒ग्नि र॒ग्नि रन्न॑म् प॒शवः॑ । \newline
46. अन्न॑म् प॒शवः॑ प॒शवो ऽन्न॒ मन्न॑म् प॒शव॑ ए॒ष ए॒ष प॒शवो ऽन्न॒ मन्न॑म् प॒शव॑ ए॒षः । \newline
47. प॒शव॑ ए॒ष ए॒ष प॒शवः॑ प॒शव॑ ए॒ष खलु॒ खल्वे॒ष प॒शवः॑ प॒शव॑ ए॒ष खलु॑ । \newline
48. ए॒ष खलु॒ खल्वे॒ष ए॒ष खलु॒ वै वै खल्वे॒ष ए॒ष खलु॒ वै । \newline
49. खलु॒ वै वै खलु॒ खलु॒ वा अ॒ग्नि र॒ग्निर् वै खलु॒ खलु॒ वा अ॒ग्निः । \newline
50. वा अ॒ग्नि र॒ग्निर् वै वा अ॒ग्निर् यद् यद॒ग्निर् वै वा अ॒ग्निर् यत् । \newline
51. अ॒ग्निर् यद् यद॒ग्नि र॒ग्निर् यत् प॑शुशी॒र्॒.षाणि॑ पशुशी॒र्॒.षाणि॒ यद॒ग्नि र॒ग्निर् यत् प॑शुशी॒र्॒.षाणि॑ । \newline
52. यत् प॑शुशी॒र्॒.षाणि॑ पशुशी॒र्॒.षाणि॒ यद् यत् प॑शुशी॒र्॒.षाणि॒ यं ॅयम् प॑शुशी॒र्॒.षाणि॒ यद् यत् प॑शुशी॒र्॒.षाणि॒ यम् । \newline
53. प॒शु॒शी॒र्॒.षाणि॒ यं ॅयम् प॑शुशी॒र्॒.षाणि॑ पशुशी॒र्॒.षाणि॒ यम् का॒मये॑त का॒मये॑त॒ यम् प॑शुशी॒र्॒.षाणि॑ पशुशी॒र्॒.षाणि॒ यम् का॒मये॑त । \newline
54. प॒शु॒शी॒र्.॒षाणीति॑ पशु - शी॒र्.॒षाणि॑ । \newline
55. यम् का॒मये॑त का॒मये॑त॒ यं ॅयम् का॒मये॑त॒ कनी॑यः॒ कनी॑यः का॒मये॑त॒ यं ॅयम् का॒मये॑त॒ कनी॑यः । \newline
56. का॒मये॑त॒ कनी॑यः॒ कनी॑यः का॒मये॑त का॒मये॑त॒ कनी॑यो ऽस्यास्य॒ कनी॑यः का॒मये॑त का॒मये॑त॒ कनी॑यो ऽस्य । \newline
57. कनी॑यो ऽस्यास्य॒ कनी॑यः॒ कनी॑यो॒ ऽस्यान्न॒ मन्न॑ मस्य॒ कनी॑यः॒ कनी॑यो॒ ऽस्यान्न᳚म् । \newline
58. अ॒स्यान्न॒ मन्न॑ मस्या॒ स्यान्नꣳ॑ स्याथ् स्या॒ दन्न॑ मस्या॒ स्यान्नꣳ॑ स्यात् । \newline
59. अन्नꣳ॑ स्याथ् स्या॒ दन्न॒ मन्नꣳ॑ स्या॒ दितीति॑ स्या॒ दन्न॒ मन्नꣳ॑ स्या॒दिति॑ । \newline
\pagebreak
\markright{ TS 5.7.10.3  \hfill https://www.vedavms.in \hfill}

\section{ TS 5.7.10.3 }

\textbf{TS 5.7.10.3 } \newline
\textbf{Samhita Paata} \newline

स्या॒दिति॑ संत॒रां तस्य॑ पशुशी॒र्॒.षाण्युप॑ दद्ध्या॒त् कनी॑य ए॒वास्यान्नं॑ भवति॒ यं का॒मये॑त स॒माव॑द॒स्यान्नꣳ॑ स्या॒दिति॑ मद्ध्य॒तस्तस्योप॑ दद्ध्याथ् स॒माव॑-दे॒वास्यान्नं॑ भवति॒ यं का॒मये॑त॒ भूयो॒ऽस्याऽन्नꣳ॑ स्या॒दित्यन्ते॑षु॒ तस्य॑ व्यु॒दूह्योप॑ दद्ध्यादन्त॒त ए॒वास्मा॒ अन्न॒मव॑ रुन्धे॒ भूयो॒ऽस्यान्नं॑ भवति ( ) ॥ \newline

\textbf{Pada Paata} \newline

स्या॒त् । इति॑ । स॒तं॒रामिति॑ सं - त॒राम् । तस्य॑ । प॒शु॒शी॒र्.॒षाणीति॑ पशु - शी॒र्.॒षाणि॑ । उपेति॑ । द॒द्ध्या॒त् । कनी॑यः । ए॒व । अ॒स्य॒ । अन्न᳚म् । भ॒व॒ति॒ । यम् । का॒मये॑त । स॒माव॑त् । अ॒स्य॒ । अन्न᳚म् । स्या॒त् । इति॑ । म॒द्ध्य॒तः । तस्य॑ । उपेति॑ । द॒द्ध्या॒त् । स॒माव॑त् । ए॒व । अ॒स्य॒ । अन्न᳚म् । भ॒व॒ति॒ । यम् । का॒मये॑त । भूयः॑ । अ॒स्य॒ । अन्न᳚म् । स्या॒त् । इति॑ । अन्ते॑षु । तस्य॑ । व्यु॒दूह्येति॑ वि - उ॒दूह्य॑ । उपेति॑ । द॒द्ध्या॒त् । अ॒न्त॒तः । ए॒व । अ॒स्मै॒ । अन्न᳚म् । अवेति॑ । रु॒न्धे॒ । भूयः॑ । अ॒स्य॒ । अन्न᳚म् । भ॒व॒ति॒ ( ) ॥  \newline


\textbf{Krama Paata} \newline

स्या॒दिति॑ । इति॑ सन्त॒राम् । स॒न्त॒राम् तस्य॑ । स॒न्त॒रामिति॑ सम् - त॒राम् । तस्य॑ पशुशी॒र्॒.षाणि॑ । प॒शु॒शी॒र्॒.षाण्युप॑ । प॒शु॒शी॒र्॒.षाणीति॑ पशु - शी॒र्॒.षाणि॑ । उप॑ दद्ध्यात् । द॒द्ध्या॒त् कनी॑यः । कनी॑य ए॒व । ए॒वास्य॑ । अ॒स्यान्न᳚म् । अन्न॑म् भवति । भ॒व॒ति॒ यम् । यम् का॒मये॑त । का॒मये॑त स॒माव॑त् । स॒माव॑दस्य । अ॒स्यान᳚म् । अन्नꣳ᳚ स्यात् । स्या॒दिति॑ । इति॑ मद्ध्य॒तः । म॒द्ध्य॒तस्तस्य॑ । तस्योप॑ । उप॑ दद्ध्यात् । द॒द्ध्या॒थ् स॒माव॑त् । स॒माव॑दे॒व । ए॒वास्य॑ । अ॒स्यान्न᳚म् । अन्न॑म् भवति । भ॒व॒ति॒ यम् । यम् का॒मये॑त । का॒मये॑त॒ भूयः॑ । भूयो᳚ऽस्य । अ॒स्यान्न᳚म् । अन्नꣳ॑ स्यात् । स्या॒दिति॑ । इत्यन्ते॑षु । अन्ते॑षु॒ तस्य॑ । तस्य॑ व्यु॒दूह्य॑ । व्यु॒दूह्योप॑ । व्यु॒दूह्येति॑ वि - उ॒दूह्य॑ । उप॑ दद्ध्यात् । द॒द्ध्या॒द॒न्त॒तः । अ॒न्त॒त ए॒व । ए॒वास्मै᳚ । अ॒स्मा॒ अन्न᳚म् । अन्न॒मव॑ । अव॑ रुन्धे । रु॒न्धे॒ भूयः॑ । भूयो᳚ऽस्य । अ॒स्यान्न᳚म् । अन्न॑म् भवति ( ) । भ॒व॒तीति॑ भवति । \newline

\textbf{Jatai Paata} \newline

1. स्या॒ दितीति॑ स्याथ् स्या॒ दिति॑ । \newline
2. इति॑ सन्त॒राꣳ स॑न्त॒रा मितीति॑ सन्त॒राम् । \newline
3. स॒न्त॒राम् तस्य॒ तस्य॑ सन्त॒राꣳ स॑न्त॒राम् तस्य॑ । \newline
4. स॒न्त॒रामिति॑ सं - त॒राम् । \newline
5. तस्य॑ पशुशी॒र्॒.षाणि॑ पशुशी॒र्॒.षाणि॒ तस्य॒ तस्य॑ पशुशी॒र्॒.षाणि॑ । \newline
6. प॒शु॒शी॒र्॒.षा ण्युपोप॑ पशुशी॒र्॒.षाणि॑ पशुशी॒र्॒.षा ण्युप॑ । \newline
7. प॒शु॒शी॒र्.॒षाणीति॑ पशु - शी॒र्.॒षाणि॑ । \newline
8. उप॑ दद्ध्याद् दद्ध्या॒ दुपोप॑ दद्ध्यात् । \newline
9. द॒द्ध्या॒त् कनी॑यः॒ कनी॑यो दद्ध्याद् दद्ध्या॒त् कनी॑यः । \newline
10. कनी॑य ए॒वैव कनी॑यः॒ कनी॑य ए॒व । \newline
11. ए॒वास्या᳚ स्यै॒वै वास्य॑ । \newline
12. अ॒स्यान्न॒ मन्न॑ मस्या॒ स्यान्न᳚म् । \newline
13. अन्न॑म् भवति भव॒ त्यन्न॒ मन्न॑म् भवति । \newline
14. भ॒व॒ति॒ यं ॅयम् भ॑वति भवति॒ यम् । \newline
15. यम् का॒मये॑त का॒मये॑त॒ यं ॅयम् का॒मये॑त । \newline
16. का॒मये॑त स॒माव॑थ् स॒माव॑त् का॒मये॑त का॒मये॑त स॒माव॑त् । \newline
17. स॒माव॑ दस्यास्य स॒माव॑थ् स॒माव॑ दस्य । \newline
18. अ॒स्यान्न॒ मन्न॑ मस्या॒ स्यान्न᳚म् । \newline
19. अन्नꣳ॑ स्याथ् स्या॒ दन्न॒ मन्नꣳ॑ स्यात् । \newline
20. स्या॒ दितीति॑ स्याथ् स्या॒ दिति॑ । \newline
21. इति॑ मद्ध्य॒तो म॑द्ध्य॒त इतीति॑ मद्ध्य॒तः । \newline
22. म॒द्ध्य॒त स्तस्य॒ तस्य॑ मद्ध्य॒तो म॑द्ध्य॒त स्तस्य॑ । \newline
23. तस्यो पोप॒ तस्य॒ तस्योप॑ । \newline
24. उप॑ दद्ध्याद् दद्ध्या॒ दुपोप॑ दद्ध्यात् । \newline
25. द॒द्ध्या॒थ् स॒माव॑थ् स॒माव॑द् दद्ध्याद् दद्ध्याथ् स॒माव॑त् । \newline
26. स॒माव॑ दे॒वैव स॒माव॑थ् स॒माव॑ दे॒व । \newline
27. ए॒वास्या᳚ स्यै॒वै वास्य॑ । \newline
28. अ॒स्यान्न॒ मन्न॑ मस्या॒ स्यान्न᳚म् । \newline
29. अन्न॑म् भवति भव॒ त्यन्न॒ मन्न॑म् भवति । \newline
30. भ॒व॒ति॒ यं ॅयम् भ॑वति भवति॒ यम् । \newline
31. यम् का॒मये॑त का॒मये॑त॒ यं ॅयम् का॒मये॑त । \newline
32. का॒मये॑त॒ भूयो॒ भूयः॑ का॒मये॑त का॒मये॑त॒ भूयः॑ । \newline
33. भूयो᳚ ऽस्यास्य॒ भूयो॒ भूयो᳚ ऽस्य । \newline
34. अ॒स्यान्न॒ मन्न॑ मस्या॒ स्यान्न᳚म् । \newline
35. अन्नꣳ॑ स्याथ् स्या॒ दन्न॒ मन्नꣳ॑ स्यात् । \newline
36. स्या॒ दितीति॑ स्याथ् स्या॒ दिति॑ । \newline
37. इत्यन्ते॒ ष्वन्ते॒ ष्विती त्यन्ते॑षु । \newline
38. अन्ते॑षु॒ तस्य॒ तस्या न्ते॒ ष्वन्ते॑षु॒ तस्य॑ । \newline
39. तस्य॑ व्यु॒दूह्य॑ व्यु॒दूह्य॒ तस्य॒ तस्य॑ व्यु॒दूह्य॑ । \newline
40. व्यु॒दूह्यो पोप॑ व्यु॒दूह्य॑ व्यु॒दूह्योप॑ । \newline
41. व्यु॒दूह्येति॑ वि - उ॒दूह्य॑ । \newline
42. उप॑ दद्ध्याद् दद्ध्या॒ दुपोप॑ दद्ध्यात् । \newline
43. द॒द्ध्या॒ द॒न्त॒तो᳚ ऽन्त॒तो द॑द्ध्याद् दद्ध्या दन्त॒तः । \newline
44. अ॒न्त॒त ए॒वै वान्त॒तो᳚ ऽन्त॒त ए॒व । \newline
45. ए॒वास्मा॑ अस्मा ए॒वै वास्मै᳚ । \newline
46. अ॒स्मा॒ अन्न॒ मन्न॑ मस्मा अस्मा॒ अन्न᳚म् । \newline
47. अन्न॒ मवा वान्न॒ मन्न॒ मव॑ । \newline
48. अव॑ रुन्धे रु॒न्धे ऽवाव॑ रुन्धे । \newline
49. रु॒न्धे॒ भूयो॒ भूयो॑ रुन्धे रुन्धे॒ भूयः॑ । \newline
50. भूयो᳚ ऽस्यास्य॒ भूयो॒ भूयो᳚ ऽस्य । \newline
51. अ॒स्यान्न॒ मन्न॑ मस्या॒ स्यान्न᳚म् । \newline
52. अन्न॑म् भवति भव॒ त्यन्न॒ मन्न॑म् भवति । \newline
53. भ॒व॒तीति॑ भवति । \newline

\textbf{Ghana Paata } \newline

1. स्या॒ दितीति॑ स्याथ् स्या॒ दिति॑ सन्त॒राꣳ स॑न्त॒रा मिति॑ स्याथ् स्या॒ दिति॑ सन्त॒राम् । \newline
2. इति॑ सन्त॒राꣳ स॑न्त॒रा मितीति॑ सन्त॒राम् तस्य॒ तस्य॑ सन्त॒रा मितीति॑ सन्त॒राम् तस्य॑ । \newline
3. स॒न्त॒राम् तस्य॒ तस्य॑ सन्त॒राꣳ स॑न्त॒राम् तस्य॑ पशुशी॒र्॒.षाणि॑ पशुशी॒र्॒.षाणि॒ तस्य॑ सन्त॒राꣳ स॑न्त॒राम् तस्य॑ पशुशी॒र्॒.षाणि॑ । \newline
4. स॒न्त॒रामिति॑ सं - त॒राम् । \newline
5. तस्य॑ पशुशी॒र्॒.षाणि॑ पशुशी॒र्॒.षाणि॒ तस्य॒ तस्य॑ पशुशी॒र्॒.षा ण्युपोप॑ पशुशी॒र्॒.षाणि॒ तस्य॒ तस्य॑ पशुशी॒र्॒.षाण्युप॑ । \newline
6. प॒शु॒शी॒र्॒.षा ण्युपोप॑ पशुशी॒र्॒.षाणि॑ पशुशी॒र्॒.षा ण्युप॑ दद्ध्याद् दद्ध्या॒ दुप॑ पशुशी॒र्॒.षाणि॑ पशुशी॒र्॒.षा ण्युप॑ दद्ध्यात् । \newline
7. प॒शु॒शी॒र्.॒षाणीति॑ पशु - शी॒र्.॒षाणि॑ । \newline
8. उप॑ दद्ध्याद् दद्ध्या॒ दुपोप॑ दद्ध्या॒त् कनी॑यः॒ कनी॑यो दद्ध्या॒ दुपोप॑ दद्ध्या॒त् कनी॑यः । \newline
9. द॒द्ध्या॒त् कनी॑यः॒ कनी॑यो दद्ध्याद् दद्ध्या॒त् कनी॑य ए॒वैव कनी॑यो दद्ध्याद् दद्ध्या॒त् कनी॑य ए॒व । \newline
10. कनी॑य ए॒वैव कनी॑यः॒ कनी॑य ए॒वास्या᳚ स्यै॒व कनी॑यः॒ कनी॑य ए॒वास्य॑ । \newline
11. ए॒वास्या᳚ स्यै॒वैवा स्यान्न॒ मन्न॑ मस्यै॒वै वास्यान्न᳚म् । \newline
12. अ॒स्यान्न॒ मन्न॑ मस्या॒ स्यान्न॑म् भवति भव॒ त्यन्न॑ मस्या॒ स्यान्न॑म् भवति । \newline
13. अन्न॑म् भवति भव॒ त्यन्न॒ मन्न॑म् भवति॒ यं ॅयम् भ॑व॒ त्यन्न॒ मन्न॑म् भवति॒ यम् । \newline
14. भ॒व॒ति॒ यं ॅयम् भ॑वति भवति॒ यम् का॒मये॑त का॒मये॑त॒ यम् भ॑वति भवति॒ यम् का॒मये॑त । \newline
15. यम् का॒मये॑त का॒मये॑त॒ यं ॅयम् का॒मये॑त स॒माव॑थ् स॒माव॑त् का॒मये॑त॒ यं ॅयम् का॒मये॑त स॒माव॑त् । \newline
16. का॒मये॑त स॒माव॑थ् स॒माव॑त् का॒मये॑त का॒मये॑त स॒माव॑ दस्यास्य स॒माव॑त् का॒मये॑त का॒मये॑त स॒माव॑ दस्य । \newline
17. स॒माव॑ दस्यास्य स॒माव॑थ् स॒माव॑ द॒स्यान्न॒ मन्न॑ मस्य स॒माव॑थ् स॒माव॑ द॒स्यान्न᳚म् । \newline
18. अ॒स्यान्न॒ मन्न॑ मस्या॒ स्यान्नꣳ॑ स्याथ् स्या॒ दन्न॑ मस्या॒ स्यान्नꣳ॑ स्यात् । \newline
19. अन्नꣳ॑ स्याथ् स्या॒ दन्न॒ मन्नꣳ॑ स्या॒ दितीति॑ स्या॒ दन्न॒ मन्नꣳ॑ स्या॒दिति॑ । \newline
20. स्या॒दितीति॑ स्याथ् स्या॒ दिति॑ मद्ध्य॒तो म॑द्ध्य॒त इति॑ स्याथ् स्या॒ दिति॑ मद्ध्य॒तः । \newline
21. इति॑ मद्ध्य॒तो म॑द्ध्य॒त इतीति॑ मद्ध्य॒त स्तस्य॒ तस्य॑ मद्ध्य॒त इतीति॑ मद्ध्य॒त स्तस्य॑ । \newline
22. म॒द्ध्य॒त स्तस्य॒ तस्य॑ मद्ध्य॒तो म॑द्ध्य॒त स्तस्योपोप॒ तस्य॑ मद्ध्य॒तो म॑द्ध्य॒त स्तस्योप॑ । \newline
23. तस्योपोप॒ तस्य॒ तस्योप॑ दद्ध्याद् दद्ध्या॒ दुप॒ तस्य॒ तस्योप॑ दद्ध्यात् । \newline
24. उप॑ दद्ध्याद् दद्ध्या॒ दुपोप॑ दद्ध्याथ् स॒माव॑थ् स॒माव॑द् दद्ध्या॒ दुपोप॑ दद्ध्याथ् स॒माव॑त् । \newline
25. द॒द्ध्या॒थ् स॒माव॑थ् स॒माव॑द् दद्ध्याद् दद्ध्याथ् स॒माव॑ दे॒वैव स॒माव॑द् दद्ध्याद् दद्ध्याथ् स॒माव॑ दे॒व । \newline
26. स॒माव॑ दे॒वैव स॒माव॑थ् स॒माव॑ दे॒वास्या᳚ स्यै॒व स॒माव॑थ् स॒माव॑ दे॒वास्य॑ । \newline
27. ए॒वास्या᳚ स्यै॒वै वास्यान्न॒ मन्न॑ मस्यै॒ वैवास्यान्न᳚म् । \newline
28. अ॒स्यान्न॒ मन्न॑ मस्या॒ स्यान्न॑म् भवति भव॒ त्यन्न॑ मस्या॒ स्यान्न॑म् भवति । \newline
29. अन्न॑म् भवति भव॒ त्यन्न॒ मन्न॑म् भवति॒ यं ॅयम् भ॑व॒ त्यन्न॒ मन्न॑म् भवति॒ यम् । \newline
30. भ॒व॒ति॒ यं ॅयम् भ॑वति भवति॒ यम् का॒मये॑त का॒मये॑त॒ यम् भ॑वति भवति॒ यम् का॒मये॑त । \newline
31. यम् का॒मये॑त का॒मये॑त॒ यं ॅयम् का॒मये॑त॒ भूयो॒ भूयः॑ का॒मये॑त॒ यं ॅयम् का॒मये॑त॒ भूयः॑ । \newline
32. का॒मये॑त॒ भूयो॒ भूयः॑ का॒मये॑त का॒मये॑त॒ भूयो᳚ ऽस्यास्य॒ भूयः॑ का॒मये॑त का॒मये॑त॒ भूयो᳚ ऽस्य । \newline
33. भूयो᳚ ऽस्यास्य॒ भूयो॒ भूयो॒ ऽस्यान्न॒ मन्न॑ मस्य॒ भूयो॒ भूयो॒ ऽस्यान्न᳚म् । \newline
34. अ॒स्यान्न॒ मन्न॑ मस्या॒ स्यान्नꣳ॑ स्याथ् स्या॒ दन्न॑ मस्या॒ स्यान्नꣳ॑ स्यात् । \newline
35. अन्नꣳ॑ स्याथ् स्या॒ दन्न॒ मन्नꣳ॑ स्या॒ दितीति॑ स्या॒ दन्न॒ मन्नꣳ॑ स्या॒ दिति॑ । \newline
36. स्या॒ दितीति॑ स्याथ् स्या॒ दित्यन्ते॒ष्वन् ते॒ष्विति॑ स्याथ् स्या॒दि त्यन्ते॑षु । \newline
37. इत्यन्ते॒ ष्वन्ते॒ ष्विती त्यन्ते॑षु॒ तस्य॒ तस्यान्ते॒ ष्विती त्यन्ते॑षु॒ तस्य॑ । \newline
38. अन्ते॑षु॒ तस्य॒ तस्यान्ते॒ ष्वन्ते॑षु॒ तस्य॑ व्यु॒दूह्य॑ व्यु॒दूह्य॒ तस्यान्ते॒ ष्वन्ते॑षु॒ तस्य॑ व्यु॒दूह्य॑ । \newline
39. तस्य॑ व्यु॒दूह्य॑ व्यु॒दूह्य॒ तस्य॒ तस्य॑ व्यु॒दूह्योपोप॑ व्यु॒दूह्य॒ तस्य॒ तस्य॑ व्यु॒दूह्योप॑ । \newline
40. व्यु॒दूह्यो पोप॑ व्यु॒दूह्य॑ व्यु॒दूह्योप॑ दद्ध्याद् दद्ध्या॒ दुप॑ व्यु॒दूह्य॑ व्यु॒दूह्योप॑ दद्ध्यात् । \newline
41. व्यु॒दूह्येति॑ वि - उ॒दूह्य॑ । \newline
42. उप॑ दद्ध्याद् दद्ध्या॒ दुपोप॑ दद्ध्या दन्त॒तो᳚ ऽन्त॒तो द॑द्ध्या॒ दुपोप॑ दद्ध्या दन्त॒तः । \newline
43. द॒द्ध्या॒ द॒न्त॒तो᳚ ऽन्त॒तो द॑द्ध्याद् दद्ध्या दन्त॒त ए॒वैवा न्त॒तो द॑द्ध्याद् दद्ध्या दन्त॒त ए॒व । \newline
44. अ॒न्त॒त ए॒वैवा न्त॒तो᳚ ऽन्त॒त ए॒वास्मा॑ अस्मा ए॒वान्त॒तो᳚ ऽन्त॒त ए॒वास्मै᳚ । \newline
45. ए॒वास्मा॑ अस्मा ए॒वै वास्मा॒ अन्न॒ मन्न॑ मस्मा ए॒वै वास्मा॒ अन्न᳚म् । \newline
46. अ॒स्मा॒ अन्न॒ मन्न॑ मस्मा अस्मा॒ अन्न॒ मवा वान्न॑ मस्मा अस्मा॒ अन्न॒ मव॑ । \newline
47. अन्न॒ मवा वान्न॒ मन्न॒ मव॑ रुन्धे रु॒न्धे ऽवान्न॒ मन्न॒ मव॑ रुन्धे । \newline
48. अव॑ रुन्धे रु॒न्धे ऽवाव॑ रुन्धे॒ भूयो॒ भूयो॑ रु॒न्धे ऽवाव॑ रुन्धे॒ भूयः॑ । \newline
49. रु॒न्धे॒ भूयो॒ भूयो॑ रुन्धे रुन्धे॒ भूयो᳚ ऽस्यास्य॒ भूयो॑ रुन्धे रुन्धे॒ भूयो᳚ ऽस्य । \newline
50. भूयो᳚ ऽस्यास्य॒ भूयो॒ भूयो॒ ऽस्यान्न॒ मन्न॑ मस्य॒ भूयो॒ भूयो॒ ऽस्यान्न᳚म् । \newline
51. अ॒स्यान्न॒ मन्न॑ मस्या॒ स्यान्न॑म् भवति भव॒ त्यन्न॑ मस्या॒ स्यान्न॑म् भवति । \newline
52. अन्न॑म् भवति भव॒ त्यन्न॒ मन्न॑म् भवति । \newline
53. भ॒व॒तीति॑ भवति । \newline
\pagebreak
\markright{ TS 5.7.11.1  \hfill https://www.vedavms.in \hfill}

\section{ TS 5.7.11.1 }

\textbf{TS 5.7.11.1 } \newline
\textbf{Samhita Paata} \newline

स्ते॒गान् दꣳष्ट्रा᳚भ्यां म॒ण्डूका॒न् जंभ्ये॑भि॒राद॑कां-खा॒देनोर्जꣳ॑ सꣳ सू॒देना*ऽर॑ण्यं॒ जांबी॑लेन॒ मृदं॑ ब॒र्.स्वे॑भिः॒ शर्क॑राभि॒रव॑का॒मव॑काभिः॒ शर्क॑रामुथ्सा॒देन॑ जि॒ह्वाम॑वक्र॒न्देन॒ तालुꣳ॒॒ सर॑स्वतीं जिह्वा॒ग्रेण॑ ॥ \newline

\textbf{Pada Paata} \newline

स्ते॒गान् । दꣳष्ट्रा᳚भ्याम् । म॒ण्डूकान्॑ । जंभ्ये॑भिः । आद॑काम् । खा॒देन॑ । ऊर्ज᳚म् । सꣳ॒॒सू॒देनेति॑ सं-सू॒देन॑ । अर॑ण्यम् । जांबी॑लेन । मृद᳚म् । ब॒र्स्वे॑भिः । शर्क॑राभिः । अव॑काम् । अव॑काभिः । शर्क॑राम् । उ॒थ्सा॒देनेत्यु॑त् - सा॒देन॑ । जि॒ह्वाम् । अ॒व॒क्र॒न्देनेत्य॑व - क्र॒न्देन॑ । तालु᳚म् । सर॑स्वतीम् । जि॒ह्वा॒ग्रेणेति॑ जिह्वा - अ॒ग्रेण॑ ॥  \newline


\textbf{Krama Paata} \newline

स्ते॒गान् दꣳष्ट्रा᳚भ्याम् । दꣳष्ट्रा᳚भ्याम् म॒ण्डूकान्॑ । म॒ण्डूका॒न् जम्भ्ये॑भिः । जम्भ्ये॑भि॒राद॑काम् । आद॑काम् खा॒देन॑ । खा॒देनोर्ज᳚म् । ऊर्जꣳ॑ सꣳसू॒देन॑ । सꣳ॒॒सू॒देनार॑ण्यम् । सꣳ॒॒सू॒देनेति॑ सम् - सू॒देन॑ । अर॑ण्य॒म् जाम्बी॑लेन । जाम्बी॑लेन॒ मृद᳚म् । मृद॑म् ब॒र्स्वे॑भिः । ब॒र्स्वे॑भिः॒ शर्क॑राभिः । शर्क॑राभि॒रव॑काम् । अव॑का॒मव॑काभिः । अव॑काभिः॒ शर्क॑राम् । शर्क॑रामुथ्सा॒देन॑ । उ॒थ्सा॒देन॑ जि॒ह्वाम् । उ॒थ्सा॒देनेत्यु॑त् - सा॒देन॑ । जि॒ह्वाम॑वक्र॒न्देन॑ । अ॒व॒क्र॒न्देन॒ तालु᳚म् । अ॒व॒क्र॒न्देनेत्य॑व - क्र॒न्देन॑ । तालुꣳ॒॒ सर॑स्वतीम् । सर॑स्वतीम् जिह्वा॒ग्रेण॑ । जि॒ह्वा॒ग्राणेति॑ जिह्वा - अ॒ग्रेण॑ । \newline

\textbf{Jatai Paata} \newline

1. स्ते॒गान् दꣳष्ट्रा᳚भ्या॒म् दꣳष्ट्रा᳚भ्याꣳ स्ते॒गान् थ्स्ते॒गान् दꣳष्ट्रा᳚भ्याम् । \newline
2. दꣳष्ट्रा᳚भ्याम् म॒ण्डूका᳚न् म॒ण्डूका॒न् दꣳष्ट्रा᳚भ्या॒म् दꣳष्ट्रा᳚भ्याम् म॒ण्डूकान्॑ । \newline
3. म॒ण्डूका॒न् जंभ्ये॑भि॒र् जंभ्ये॑भिर् म॒ण्डूका᳚न् म॒ण्डूका॒न् जंभ्ये॑भिः । \newline
4. जंभ्ये॑भि॒ राद॑का॒ माद॑का॒म् जंभ्ये॑भि॒र् जंभ्ये॑भि॒ राद॑काम् । \newline
5. आद॑काम् खा॒देन॑ खा॒देना द॑का॒ माद॑काम् खा॒देन॑ । \newline
6. खा॒देनोर्ज॒ मूर्ज॑म् खा॒देन॑ खा॒देनोर्ज᳚म् । \newline
7. ऊर्जꣳ॑ सꣳसू॒देन॑ सꣳसू॒देनोर्ज॒ मूर्जꣳ॑ सꣳसू॒देन॑ । \newline
8. सꣳ॒॒सू॒देना र॑ण्य॒ मर॑ण्यꣳ सꣳसू॒देन॑ सꣳसू॒देना र॑ण्यम् । \newline
9. सꣳ॒॒सू॒देनेति॑ सं - सू॒देन॑ । \newline
10. अर॑ण्य॒म् जांबी॑लेन॒ जांबी॑ले॒ना र॑ण्य॒ मर॑ण्य॒म् जांबी॑लेन । \newline
11. जांबी॑लेन॒ मृद॒म् मृद॒म् जांबी॑लेन॒ जांबी॑लेन॒ मृद᳚म् । \newline
12. मृद॑म् ब॒र्स्वे॑भिर् ब॒र्स्वे॑भि॒र् मृद॒म् मृद॑म् ब॒र्स्वे॑भिः । \newline
13. ब॒र्स्वे॑भिः॒ शर्क॑राभिः॒ शर्क॑राभिर् ब॒र्स्वे॑भिर् ब॒र्स्वे॑भिः॒ शर्क॑राभिः । \newline
14. शर्क॑राभि॒ रव॑का॒ मव॑काꣳ॒॒ शर्क॑राभिः॒ शर्क॑राभि॒ रव॑काम् । \newline
15. अव॑का॒ मव॑काभि॒ रव॑काभि॒ रव॑का॒ मव॑का॒ मव॑काभिः । \newline
16. अव॑काभिः॒ शर्क॑राꣳ॒॒ शर्क॑रा॒ मव॑काभि॒ रव॑काभिः॒ शर्क॑राम् । \newline
17. शर्क॑रा मुथ्सा॒देनो᳚ थ्सा॒देन॒ शर्क॑राꣳ॒॒ शर्क॑रा मुथ्सा॒देन॑ । \newline
18. उ॒थ्सा॒देन॑ जि॒ह्वाम् जि॒ह्वा मु॑थ्सा॒देनो᳚ थ्सा॒देन॑ जि॒ह्वाम् । \newline
19. उ॒थ्सा॒देनेत्यु॑त् - सा॒देन॑ । \newline
20. जि॒ह्वा म॑वक्र॒न्देना॑ वक्र॒न्देन॑ जि॒ह्वाम् जि॒ह्वा म॑वक्र॒न्देन॑ । \newline
21. अ॒व॒क्र॒न्देन॒ तालु॒म् तालु॑ मवक्र॒न्देना॑ वक्र॒न्देन॒ तालु᳚म् । \newline
22. अ॒व॒क्र॒न्देनेत्य॑व - क्र॒न्देन॑ । \newline
23. तालुꣳ॒॒ सर॑स्वतीꣳ॒॒ सर॑स्वती॒म् तालु॒म् तालुꣳ॒॒ सर॑स्वतीम् । \newline
24. सर॑स्वतीम् जिह्वा॒ग्रेण॑ जिह्वा॒ग्रेण॒ सर॑स्वतीꣳ॒॒ सर॑स्वतीम् जिह्वा॒ग्रेण॑ । \newline
25. जि॒ह्वा॒ग्रेणेति॑ जिह्वा - अ॒ग्रेण॑ । \newline

\textbf{Ghana Paata } \newline

1. स्ते॒गान् दꣳष्ट्रा᳚भ्या॒म् दꣳष्ट्रा᳚भ्याꣳ स्ते॒गान् थ्स्ते॒गान् दꣳष्ट्रा᳚भ्याम् म॒ण्डूका᳚न् म॒ण्डूका॒न् दꣳष्ट्रा᳚भ्याꣳ स्ते॒गान् थ्स्ते॒गान् दꣳष्ट्रा᳚भ्याम् म॒ण्डूकान्॑ । \newline
2. दꣳष्ट्रा᳚भ्याम् म॒ण्डूका᳚न् म॒ण्डूका॒न् दꣳष्ट्रा᳚भ्या॒म् दꣳष्ट्रा᳚भ्याम् म॒ण्डूका॒न् जंभ्ये॑भि॒र् जंभ्ये॑भिर् म॒ण्डूका॒न् दꣳष्ट्रा᳚भ्या॒म् दꣳष्ट्रा᳚भ्याम् म॒ण्डूका॒न् जंभ्ये॑भिः । \newline
3. म॒ण्डूका॒न् जंभ्ये॑भि॒र् जंभ्ये॑भिर् म॒ण्डूका᳚न् म॒ण्डूका॒न् जंभ्ये॑भि॒ राद॑का॒ माद॑का॒म् जंभ्ये॑भिर् म॒ण्डूका᳚न् म॒ण्डूका॒न् जंभ्ये॑भि॒ राद॑काम् । \newline
4. जंभ्ये॑भि॒ राद॑का॒ माद॑का॒म् जंभ्ये॑भि॒र् जंभ्ये॑भि॒ राद॑काम् खा॒देन॑ खा॒देना द॑का॒म् जंभ्ये॑भि॒र् जंभ्ये॑भि॒रा द॑काम् खा॒देन॑ । \newline
5. आद॑काम् खा॒देन॑ खा॒देना द॑का॒ माद॑काम् खा॒देनोर्ज॒ मूर्ज॑म् खा॒देना द॑का॒ माद॑काम् खा॒देनोर्ज᳚म् । \newline
6. खा॒दे नोर्ज॒ मूर्ज॑म् खा॒देन॑ खा॒दे नोर्जꣳ॑ सꣳसू॒देन॑ सꣳसू॒दे नोर्ज॑म् खा॒देन॑ खा॒दे नोर्जꣳ॑ सꣳसू॒देन॑ । \newline
7. ऊर्जꣳ॑ सꣳसू॒देन॑ सꣳसू॒दे नोर्ज॒ मूर्जꣳ॑ सꣳसू॒दे नार॑ण्य॒ मर॑ण्यꣳ सꣳसू॒दे नोर्ज॒ मूर्जꣳ॑ सꣳसू॒दे नार॑ण्यम् । \newline
8. सꣳ॒॒सू॒दे नार॑ण्य॒ मर॑ण्यꣳ सꣳसू॒देन॑ सꣳसू॒दे नार॑ण्य॒म् जांबी॑लेन॒ जांबी॑ले॒ नार॑ण्यꣳ सꣳसू॒देन॑ सꣳसू॒दे नार॑ण्य॒म् जांबी॑लेन । \newline
9. सꣳ॒॒सू॒देनेति॑ सं - सू॒देन॑ । \newline
10. अर॑ण्य॒म् जांबी॑लेन॒ जांबी॑ले॒ नार॑ण्य॒ मर॑ण्य॒म् जांबी॑लेन॒ मृद॒म् मृद॒म् जांबी॑ले॒ नार॑ण्य॒ मर॑ण्य॒म् जांबी॑लेन॒ मृद᳚म् । \newline
11. जांबी॑लेन॒ मृद॒म् मृद॒म् जांबी॑लेन॒ जांबी॑लेन॒ मृद॑म् ब॒र्स्वे॑भिर् ब॒र्स्वे॑भि॒र् मृद॒म् जांबी॑लेन॒ जांबी॑लेन॒ मृद॑म् ब॒र्स्वे॑भिः । \newline
12. मृद॑म् ब॒र्स्वे॑भिर् ब॒र्स्वे॑भि॒र् मृद॒म् मृद॑म् ब॒र्स्वे॑भिः॒ शर्क॑राभिः॒ शर्क॑राभिर् ब॒र्स्वे॑भि॒र् मृद॒म् मृद॑म् ब॒र्स्वे॑भिः॒ शर्क॑राभिः । \newline
13. ब॒र्स्वे॑भिः॒ शर्क॑राभिः॒ शर्क॑राभिर् ब॒र्स्वे॑भिर् ब॒र्स्वे॑भिः॒ शर्क॑राभि॒ रव॑का॒ मव॑काꣳ॒॒ शर्क॑राभिर् ब॒र्स्वे॑भिर् ब॒र्स्वे॑भिः॒ शर्क॑राभि॒ रव॑काम् । \newline
14. शर्क॑राभि॒ रव॑का॒ मव॑काꣳ॒॒ शर्क॑राभिः॒ शर्क॑राभि॒ रव॑का॒ मव॑काभि॒ रव॑काभि॒ रव॑काꣳ॒॒ शर्क॑राभिः॒ शर्क॑राभि॒ रव॑का॒ मव॑काभिः । \newline
15. अव॑का॒ मव॑काभि॒ रव॑काभि॒ रव॑का॒ मव॑का॒ मव॑काभिः॒ शर्क॑राꣳ॒॒ शर्क॑रा॒ मव॑काभि॒ रव॑का॒ मव॑का॒ मव॑काभिः॒ शर्क॑राम् । \newline
16. अव॑काभिः॒ शर्क॑राꣳ॒॒ शर्क॑रा॒ मव॑काभि॒ रव॑काभिः॒ शर्क॑रा मुथ्सा॒देनो᳚ थ्सा॒देन॒ शर्क॑रा॒ मव॑काभि॒ रव॑काभिः॒ शर्क॑रा मुथ्सा॒देन॑ । \newline
17. शर्क॑रा मुथ्सा॒देनो᳚ थ्सा॒देन॒ शर्क॑राꣳ॒॒ शर्क॑रा मुथ्सा॒देन॑ जि॒ह्वाम् जि॒ह्वा मु॑थ्सा॒देन॒ शर्क॑राꣳ॒॒ शर्क॑रा मुथ्सा॒देन॑ जि॒ह्वाम् । \newline
18. उ॒थ्सा॒देन॑ जि॒ह्वाम् जि॒ह्वा मु॑थ्सा॒देनो᳚ थ्सा॒देन॑ जि॒ह्वा म॑वक्र॒न्देना॑ वक्र॒न्देन॑ जि॒ह्वा मु॑थ्सा॒देनो᳚ थ्सा॒देन॑ जि॒ह्वा म॑वक्र॒न्देन॑ । \newline
19. उ॒थ्सा॒देनेत्यु॑त् - सा॒देन॑ । \newline
20. जि॒ह्वा म॑वक्र॒न्देना॑ वक्र॒न्देन॑ जि॒ह्वाम् जि॒ह्वा म॑वक्र॒न्देन॒ तालु॒म् तालु॑ मवक्र॒न्देन॑ जि॒ह्वाम् जि॒ह्वा म॑वक्र॒न्देन॒ तालु᳚म् । \newline
21. अ॒व॒क्र॒न्देन॒ तालु॒म् तालु॑ मवक्र॒न्देना॑ वक्र॒न्देन॒ तालुꣳ॒॒ सर॑स्वतीꣳ॒॒ सर॑स्वती॒म् तालु॑ मवक्र॒न्देना॑ वक्र॒न्देन॒ तालुꣳ॒॒ सर॑स्वतीम् । \newline
22. अ॒व॒क्र॒न्देनेत्य॑व - क्र॒न्देन॑ । \newline
23. तालुꣳ॒॒ सर॑स्वतीꣳ॒॒ सर॑स्वती॒म् तालु॒म् तालुꣳ॒॒ सर॑स्वतीम् जिह्वा॒ग्रेण॑ जिह्वा॒ग्रेण॒ सर॑स्वती॒म् तालु॒म् तालुꣳ॒॒ सर॑स्वतीम् जिह्वा॒ग्रेण॑ । \newline
24. सर॑स्वतीम् जिह्वा॒ग्रेण॑ जिह्वा॒ग्रेण॒ सर॑स्वतीꣳ॒॒ सर॑स्वतीम् जिह्वा॒ग्रेण॑ । \newline
25. जि॒ह्वा॒ग्रेणेति॑ जिह्वा - अ॒ग्रेण॑ । \newline
\pagebreak
\markright{ TS 5.7.12.1  \hfill https://www.vedavms.in \hfill}

\section{ TS 5.7.12.1 }

\textbf{TS 5.7.12.1 } \newline
\textbf{Samhita Paata} \newline

वाजꣳ॒॒ हनू᳚भ्याम॒प आ॒स्ये॑नाऽऽ*दि॒त्या-ञ्छ्मश्रु॑भि-रुपया॒म-मध॑रे॒णोष्ठे॑न॒ सदुत्त॑रे॒णान्त॑रेणा-नूका॒शं प्र॑का॒शेन॒ बाह्यꣳ॑ स्तनयि॒त्नुं नि॑र्बा॒धेन॑ सूर्या॒ग्नी चक्षु॑र्भ्यां ॅवि॒द्युतौ॑ क॒नान॑काभ्याम॒शनिं॑ म॒स्तिष्के॑ण॒ बलं॑ म॒ज्जभिः॑ ॥ \newline

\textbf{Pada Paata} \newline

वाज᳚म् । हनू᳚भ्या॒मिति॒ हनु॑ - भ्या॒म् । अ॒पः । आ॒स्ये॑न । आ॒दि॒त्यान् । श्मश्रु॑भि॒रिति॒ श्मश्रु॑ - भिः॒ । उ॒प॒या॒ममित्यु॑प - या॒मम् । अध॑रेण । ओष्ठे॑न । सत् । उत्त॑रे॒णेत्युत् - त॒रे॒ण॒ । अन्त॑रेण । अ॒नू॒का॒शमित्य॑नु - का॒शम् । प्र॒का॒शेनेति॑ प्र - का॒शेन॑ । बाह्य᳚म् । स्त॒न॒यि॒त्नुम् । नि॒र्बा॒धेनेति॑ निः-बा॒धेन॑ । सू॒र्या॒ग्नी इति॑ सूर्य-अ॒ग्नी । चक्षु॑र्भ्या॒मिति॒ चक्षुः॑ - भ्या॒म् । वि॒द्युता॒विति॑ वि - द्युतौ᳚ । क॒नान॑काभ्याम् । अ॒शनि᳚म् । म॒स्तिष्के॑ण । बल᳚म् । म॒ज्जभि॒रिति॑ म॒ज्ज - भिः॒ ॥  \newline


\textbf{Krama Paata} \newline

वाजꣳ॒॒ हनू᳚भ्याम् । हनू᳚भ्याम॒पः । हनू᳚भ्या॒मिति॒ हनु॑ - भ्या॒म् । अ॒प आ॒स्ये॑न । आ॒स्ये॑नादि॒त्यान् । 
आ॒दि॒त्यान् श्मश्रु॑भिः । श्मश्रु॑भिरुपया॒मम् । श्मश्रु॑भि॒रिति॒ श्मश्रु॑ - भिः॒ । उ॒प॒या॒ममध॑रेण । उ॒प॒या॒ममित्यु॑प - या॒मम् । अध॑रे॒णोष्ठे॑न । ओष्ठे॑न॒ सत् । सदुत्त॑रेण । उत्त॑रे॒णान्त॑रेण । उत्त॑रे॒णेत्युत् - त॒रे॒ण॒ । अन्त॑रेणानूका॒शम् । अ॒नू॒का॒शम् प्र॑का॒शेन॑ । अ॒नू॒का॒शमित्य॑नु - का॒शम् । प्र॒का॒शेन॒ बाह्य᳚म् । प्र॒का॒शेनेति॑ प्र - का॒शेन॑ । बाह्यꣳ॑ स्तनयि॒त्नुम् । स्त॒न॒यि॒त्नुम् नि॑र्बा॒धेन॑ । नि॒र्बा॒धेन॑ सूर्या॒ग्नी । नि॒र्बा॒धेनेति॑ निः - बा॒धेन॑ । सू॒र्या॒ग्नी चक्षु॑र्भ्याम् । सू॒र्या॒ग्नी इति॑ सूर्य - अ॒ग्नी । चक्षु॑र्भ्याम् ॅवि॒द्युतौ᳚ । चक्षु॑र्भ्या॒मिति॒ चक्षुः॑ - भ्या॒म् । वि॒द्युतौ॑ क॒नान॑काभ्याम् । वि॒द्युता॒विति॑ वि - द्युतौ᳚ । क॒नान॑काभ्याम॒शनि᳚म् । अ॒शनि॑म् म॒स्तिष्के॑ण । म॒स्तिष्के॑ण॒ बल᳚म् । बल॑म् म॒ज्जभिः॑ । म॒ज्जभि॒रिति॑ म॒ज्ज - भिः॒ । \newline

\textbf{Jatai Paata} \newline

1. वाजꣳ॒॒ हनू᳚भ्याꣳ॒॒ हनू᳚भ्यां॒ ॅवाजं॒ ॅवाजꣳ॒॒ हनू᳚भ्याम् । \newline
2. हनू᳚भ्या म॒पो॑ ऽपो हनू᳚भ्याꣳ॒॒ हनू᳚भ्या म॒पः । \newline
3. हनू᳚भ्या॒मिति॒ हनु॑ - भ्या॒म् । \newline
4. अ॒प आ॒स्ये॑ना॒ स्ये॑ना॒पो॑ ऽप आ॒स्ये॑न । \newline
5. आ॒स्ये॑ नादि॒त्या ना॑दि॒त्या ना॒स्ये॑ ना॒स्ये॑ नादि॒त्यान् । \newline
6. आ॒दि॒त्याञ् छ्मश्रु॑भिः॒ श्मश्रु॑भि रादि॒त्या ना॑दि॒त्याञ् छ्मश्रु॑भिः । \newline
7. श्मश्रु॑भि रुपया॒म मु॑पया॒मꣳ श्मश्रु॑भिः॒ श्मश्रु॑भि रुपया॒मम् । \newline
8. श्मश्रु॑भि॒रिति॒ श्मश्रु॑ - भिः॒ । \newline
9. उ॒प॒या॒म मध॑रे॒णा ध॑रेणो पया॒म मु॑पया॒म मध॑रेण । \newline
10. उ॒प॒या॒ममित्यु॑प - या॒मम् । \newline
11. अध॑रे॒ णौष्ठे॒ नौष्ठे॒ना ध॑रे॒णा ध॑रे॒णौ ष्ठे॑न । \newline
12. ओष्ठे॑न॒ सथ् सदोष्ठे॒ नौष्ठे॑न॒ सत् । \newline
13. सदुत्त॑रे॒णो त्त॑रेण॒ सथ् सदुत्त॑रेण । \newline
14. उत्त॑रे॒णा न्त॑रे॒णा न्त॑रे॒णो त्त॑रे॒णो त्त॑रे॒णा न्त॑रेण । \newline
15. उत्त॑रे॒णेत्युत् - त॒रे॒ण॒ । \newline
16. अन्त॑रेणा नूका॒श म॑नूका॒श मन्त॑रे॒णा न्त॑रेणा नूका॒शम् । \newline
17. अ॒नू॒का॒शम् प्र॑का॒शेन॑ प्रका॒शेना॑ नूका॒श म॑नूका॒शम् प्र॑का॒शेन॑ । \newline
18. अ॒नू॒का॒शमित्य॑नु - का॒शम् । \newline
19. प्र॒का॒शेन॒ बाह्य॒म् बाह्य॑म् प्रका॒शेन॑ प्रका॒शेन॒ बाह्य᳚म् । \newline
20. प्र॒का॒शेनेति॑ प्र - का॒शेन॑ । \newline
21. बाह्यꣳ॑ स्तनयि॒त्नुꣳ स्त॑नयि॒त्नुम् बाह्य॒म् बाह्यꣳ॑ स्तनयि॒त्नुम् । \newline
22. स्त॒न॒यि॒त्नुन् नि॑र्बा॒धेन॑ निर्बा॒धेन॑ स्तनयि॒त्नुꣳ स्त॑नयि॒त्नुन् नि॑र्बा॒धेन॑ । \newline
23. नि॒र्बा॒धेन॑ सूर्या॒ग्नी सू᳚र्या॒ग्नी नि॑र्बा॒धेन॑ निर्बा॒धेन॑ सूर्या॒ग्नी । \newline
24. नि॒र्बा॒धेनेति॑ निः - बा॒धेन॑ । \newline
25. सू॒र्या॒ग्नी चक्षु॑र्भ्या॒म् चक्षु॑र्भ्याꣳ सूर्या॒ग्नी सू᳚र्या॒ग्नी चक्षु॑र्भ्याम् । \newline
26. सू॒र्या॒ग्नी इति॑ सूर्य - अ॒ग्नी । \newline
27. चक्षु॑र्भ्यां ॅवि॒द्युतौ॑ वि॒द्युतौ॒ चक्षु॑र्भ्या॒म् चक्षु॑र्भ्यां ॅवि॒द्युतौ᳚ । \newline
28. चक्षु॑र्भ्या॒मिति॒ चक्षुः॑ - भ्या॒म् । \newline
29. वि॒द्युतौ॑ क॒नान॑काभ्याम् क॒नान॑काभ्यां ॅवि॒द्युतौ॑ वि॒द्युतौ॑ क॒नान॑काभ्याम् । \newline
30. वि॒द्युता॒विति॑ वि - द्युतौ᳚ । \newline
31. क॒नान॑काभ्या म॒शनि॑ म॒शनि॑म् क॒नान॑काभ्याम् क॒नान॑काभ्या म॒शनि᳚म् । \newline
32. अ॒शनि॑म् म॒स्तिष्के॑ण म॒स्तिष्के॑णा॒ शनि॑ म॒शनि॑म् म॒स्तिष्के॑ण । \newline
33. म॒स्तिष्के॑ण॒ बल॒म् बल॑म् म॒स्तिष्के॑ण म॒स्तिष्के॑ण॒ बल᳚म् । \newline
34. बल॑म् म॒ज्जभि॑र् म॒ज्जभि॒र् बल॒म् बल॑म् म॒ज्जभिः॑ । \newline
35. म॒ज्जभि॒रिति॑ म॒ज्ज - भिः॒ । \newline

\textbf{Ghana Paata } \newline

1. वाजꣳ॒॒ हनू᳚भ्याꣳ॒॒ हनू᳚भ्यां॒ ॅवाजं॒ ॅवाजꣳ॒॒ हनू᳚भ्या म॒पो॑ ऽपो हनू᳚भ्यां॒ ॅवाजं॒ ॅवाजꣳ॒॒ हनू᳚भ्या म॒पः । \newline
2. हनू᳚भ्या म॒पो॑ ऽपो हनू᳚भ्याꣳ॒॒ हनू᳚भ्या म॒प आ॒स्ये॑ना॒ स्ये॑ना॒पो हनू᳚भ्याꣳ॒॒ हनू᳚भ्या म॒प आ॒स्ये॑न । \newline
3. हनू᳚भ्या॒मिति॒ हनु॑ - भ्या॒म् । \newline
4. अ॒प आ॒स्ये॑ना॒ स्ये॑ना॒पो॑ ऽप आ॒स्ये॑ नादि॒त्या ना॑दि॒त्या ना॒स्ये॑ना॒पो॑ ऽप आ॒स्ये॑ नादि॒त्यान् । \newline
5. आ॒स्ये॑ नादि॒त्या ना॑दि॒त्या ना॒स्ये॑ना॒ स्ये॑नादि॒त्याञ् छ्मश्रु॑भिः॒ श्मश्रु॑भि रादि॒त्या ना॒स्ये॑ना॒ स्ये॑नादि॒त्याञ् छ्मश्रु॑भिः । \newline
6. आ॒दि॒त्याञ् छ्मश्रु॑भिः॒ श्मश्रु॑भि रादि॒त्या ना॑दि॒त्याञ् छ्मश्रु॑भि रुपया॒म मु॑पया॒मꣳ श्मश्रु॑भि रादि॒त्या ना॑दि॒त्याञ् छ्मश्रु॑भि रुपया॒मम् । \newline
7. श्मश्रु॑भि रुपया॒म मु॑पया॒मꣳ श्मश्रु॑भिः॒ श्मश्रु॑भि रुपया॒म मध॑रे॒णा ध॑रेणोपया॒मꣳ श्मश्रु॑भिः॒ श्मश्रु॑भि रुपया॒म मध॑रेण । \newline
8. श्मश्रु॑भि॒रिति॒ श्मश्रु॑ - भिः॒ । \newline
9. उ॒प॒या॒म मध॑रे॒णा ध॑रे णोपया॒म मु॑पया॒म मध॑रे॒णौष्ठे॒ नौष्ठे॒ना ध॑रे णोपया॒म मु॑पया॒म मध॑रे॒णौ ष्ठे॑न । \newline
10. उ॒प॒या॒ममित्यु॑प - या॒मम् । \newline
11. अध॑रे॒ णौष्ठे॒ नौष्ठे॒ना ध॑रे॒णा ध॑रे॒ णौष्ठे॑न॒ सथ् सदोष्ठे॒ना ध॑रे॒णा ध॑रे॒
णौष्ठे॑न॒ सत् । \newline
12. ओष्ठे॑न॒ सथ् सदोष्ठे॒ नौष्ठे॑न॒ सदुत्त॑रे॒ णोत्त॑रेण॒ सदोष्ठे॒ नौष्ठे॑न॒ सदुत्त॑रेण । \newline
13. सदुत्त॑रे॒ णोत्त॑रेण॒ सथ् सदुत्त॑रे॒णा न्त॑रे॒णा न्त॑रे॒ णोत्त॑रेण॒ सथ् सदुत्त॑रे॒णा न्त॑रेण । \newline
14. उत्त॑रे॒णा न्त॑रे॒णा न्त॑रे॒णो त्त॑रे॒ णोत्त॑रे॒णा न्त॑रेणा नूका॒श म॑नूका॒श मन्त॑रे॒ णोत्त॑रे॒णो त्त॑रे॒णा न्त॑रेणा नूका॒शम् । \newline
15. उत्त॑रे॒णेत्युत् - त॒रे॒ण॒ । \newline
16. अन्त॑रेणा नूका॒श म॑नूका॒श मन्त॑रे॒णा न्त॑रेणा नूका॒शम् प्र॑का॒शेन॑ प्रका॒शेना॑ नूका॒श मन्त॑रे॒णा न्त॑रेणा नूका॒शम् प्र॑का॒शेन॑ । \newline
17. अ॒नू॒का॒शम् प्र॑का॒शेन॑ प्रका॒शेना॑ नूका॒श म॑नूका॒शम् प्र॑का॒शेन॒ बाह्य॒म् बाह्य॑म् प्रका॒शेना॑ नूका॒श म॑नूका॒शम् प्र॑का॒शेन॒ बाह्य᳚म् । \newline
18. अ॒नू॒का॒शमित्य॑नु - का॒शम् । \newline
19. प्र॒का॒शेन॒ बाह्य॒म् बाह्य॑म् प्रका॒शेन॑ प्रका॒शेन॒ बाह्यꣳ॑ स्तनयि॒त्नुꣳ स्त॑नयि॒त्नुम् बाह्य॑म् प्रका॒शेन॑ प्रका॒शेन॒ बाह्यꣳ॑ स्तनयि॒त्नुम् । \newline
20. प्र॒का॒शेनेति॑ प्र - का॒शेन॑ । \newline
21. बाह्यꣳ॑ स्तनयि॒त्नुꣳ स्त॑नयि॒त्नुम् बाह्य॒म् बाह्यꣳ॑ स्तनयि॒त्नुन् नि॑र्बा॒धेन॑ निर्बा॒धेन॑ स्तनयि॒त्नुम् बाह्य॒म् बाह्यꣳ॑ स्तनयि॒त्नुम् नि॑र्बा॒धेन॑ । \newline
22. स्त॒न॒यि॒त्नुन् नि॑र्बा॒धेन॑ निर्बा॒धेन॑ स्तनयि॒त्नुꣳ स्त॑नयि॒त्नुन् नि॑र्बा॒धेन॑ सूर्या॒ग्नी सू᳚र्या॒ग्नी नि॑र्बा॒धेन॑ स्तनयि॒त्नुꣳ स्त॑नयि॒त्नुम् नि॑र्बा॒धेन॑ सूर्या॒ग्नी । \newline
23. नि॒र्बा॒धेन॑ सूर्या॒ग्नी सू᳚र्या॒ग्नी नि॑र्बा॒धेन॑ निर्बा॒धेन॑ सूर्या॒ग्नी चक्षु॑र्भ्या॒म् चक्षु॑र्भ्याꣳ सूर्या॒ग्नी नि॑र्बा॒धेन॑ निर्बा॒धेन॑ सूर्या॒ग्नी चक्षु॑र्भ्याम् । \newline
24. नि॒र्बा॒धेनेति॑ निः - बा॒धेन॑ । \newline
25. सू॒र्या॒ग्नी चक्षु॑र्भ्या॒म् चक्षु॑र्भ्याꣳ सूर्या॒ग्नी सू᳚र्या॒ग्नी चक्षु॑र्भ्यां ॅवि॒द्युतौ॑ वि॒द्युतौ॒ चक्षु॑र्भ्याꣳ सूर्या॒ग्नी सू᳚र्या॒ग्नी चक्षु॑र्भ्यां ॅवि॒द्युतौ᳚ । \newline
26. सू॒र्या॒ग्नी इति॑ सूर्य - अ॒ग्नी । \newline
27. चक्षु॑र्भ्यां ॅवि॒द्युतौ॑ वि॒द्युतौ॒ चक्षु॑र्भ्या॒म् चक्षु॑र्भ्यां ॅवि॒द्युतौ॑ क॒नान॑काभ्याम् क॒नान॑काभ्यां ॅवि॒द्युतौ॒ चक्षु॑र्भ्या॒म् चक्षु॑र्भ्यां ॅवि॒द्युतौ॑ क॒नान॑काभ्याम् । \newline
28. चक्षु॑र्भ्या॒मिति॒ चक्षुः॑ - भ्या॒म् । \newline
29. वि॒द्युतौ॑ क॒नान॑काभ्याम् क॒नान॑काभ्यां ॅवि॒द्युतौ॑ वि॒द्युतौ॑ क॒नान॑काभ्या म॒शनि॑ म॒शनि॑म् क॒नान॑काभ्यां ॅवि॒द्युतौ॑ वि॒द्युतौ॑ क॒नान॑काभ्या म॒शनि᳚म् । \newline
30. वि॒द्युता॒विति॑ वि - द्युतौ᳚ । \newline
31. क॒नान॑काभ्या म॒शनि॑ म॒शनि॑म् क॒नान॑काभ्याम् क॒नान॑काभ्या म॒शनि॑म् म॒स्तिष्के॑ण म॒स्तिष्के॑णा॒ शनि॑म् क॒नान॑काभ्याम् क॒नान॑काभ्या म॒शनि॑म् म॒स्तिष्के॑ण । \newline
32. अ॒शनि॑म् म॒स्तिष्के॑ण म॒स्तिष्के॑णा॒ शनि॑ म॒शनि॑म् म॒स्तिष्के॑ण॒ बल॒म् बल॑म् म॒स्तिष्के॑णा॒ शनि॑ म॒शनि॑म् म॒स्तिष्के॑ण॒ बल᳚म् । \newline
33. म॒स्तिष्के॑ण॒ बल॒म् बल॑म् म॒स्तिष्के॑ण म॒स्तिष्के॑ण॒ बल॑म् म॒ज्जभि॑र् म॒ज्जभि॒र् बल॑म् म॒स्तिष्के॑ण म॒स्तिष्के॑ण॒ बल॑म् म॒ज्जभिः॑ । \newline
34. बल॑म् म॒ज्जभि॑र् म॒ज्जभि॒र् बल॒म् बल॑म् म॒ज्जभिः॑ । \newline
35. म॒ज्जभि॒रिति॑ म॒ज्ज - भिः॒ । \newline
\pagebreak
\markright{ TS 5.7.13.1  \hfill https://www.vedavms.in \hfill}

\section{ TS 5.7.13.1 }

\textbf{TS 5.7.13.1 } \newline
\textbf{Samhita Paata} \newline

कू॒र्मा-ञ्छ॒फैर॒च्छला॑भिः क॒पिञ्ज॑ला॒न्थ्साम॒ कुष्ठि॑काभिर्ज॒वं जङ्घा॑भिरग॒दं जानु॑भ्यां ॅवी॒र्यं॑ कु॒हाभ्यां᳚ भ॒यं प्र॑चा॒लाभ्यां॒ गुहो॑पप॒क्षाभ्या॑-म॒श्विना॒वꣳ सा᳚भ्या॒मदि॑तिꣳ शी॒र्ष्णा निर्.ऋ॑तिं॒ निर्जा᳚ल्मकेन शी॒र्ष्णा ॥ \newline

\textbf{Pada Paata} \newline

कू॒र्मान् । श॒फैः । अ॒च्छला॑भिः । क॒पिञ्ज॑लान् । साम॑ । कुष्ठि॑काभिः । ज॒वम् । जङ्घा॑भिः । अ॒ग॒दम् । जानु॑भ्या॒मिति॒ जानु॑-भ्या॒म् । वी॒र्य᳚म् । कु॒हाभ्या᳚म् । भ॒यम् । प्र॒चा॒लाभ्या॒मिति॑ प्र - चा॒लाभ्या᳚म् । गुहा᳚ । उ॒प॒प॒क्षाभ्या॒मित्यु॑प - प॒क्षाभ्या᳚म् । अ॒श्विनौ᳚ । अꣳसा᳚भ्याम् । अदि॑तिम् । शी॒र्ष्णा । निर्.ऋ॑ति॒मिति॒ निः - ऋ॒ति॒म् । निर्जा᳚ल्मके॒नेति॒ निः-जा॒ल्म॒के॒न॒ । शी॒र्ष्णा ॥  \newline


\textbf{Krama Paata} \newline

कू॒र्माञ्छ॒फैः । श॒फैर॒च्छला॑भिः । अ॒च्छला॑भिः क॒पिञ्ज॑लान् । क॒पिञ्ज॑ला॒न्थ् साम॑ । साम॒ कुष्ठि॑काभिः । कुष्ठि॑काभिर् ज॒वम् । ज॒वम् जङ्घा॑भिः । जङ्घा॑भिरग॒दम् । अ॒ग॒दम् जानु॑भ्याम् । जानु॑भ्याम् ॅवी॒र्य᳚म् । जानु॑भ्या॒मिति॒ जानु॑ - भ्या॒म् । वी॒र्य॑म् कु॒हाभ्या᳚म् । कु॒हाभ्या᳚म् भ॒यम् । भ॒यम् प्र॑चा॒लाभ्या᳚म् । प्र॒चा॒लाभ्या॒म् गुहा᳚ । प्र॒चा॒लाभ्या॒मिति॑ प्र - चा॒लाभ्या᳚म् । गुहो॑पप॒क्षाभ्या᳚म् । उ॒प॒प॒क्षाभ्या॑म॒श्विनौ᳚ । उ॒प॒प॒क्षाभ्या॒मित्यु॑प - प॒क्षाभ्या᳚म् । अ॒श्विना॒वꣳसा᳚भ्याम् । अꣳसा᳚भ्या॒मदि॑तिम् । अदि॑तिꣳ शी॒र्ष्णा । शी॒र्ष्णा निर्.ऋ॑तिम् । निर्.ऋ॑ति॒म् निर्जा᳚ल्मकेन । निर्.ऋ॑ति॒मिति॒ निः - ऋ॒ति॒म् । निर्जा᳚ल्मकेन शी॒र्ष्णा । निर्जा᳚ल्मके॒नेति॒ निः - जा॒ल्म॒के॒न॒ । शी॒र्ष्णेति॑ शी॒र्ष्णा । \newline

\textbf{Jatai Paata} \newline

1. कू॒र्माञ् छ॒फैः श॒फैः कू॒र्मान् कू॒र्माञ् छ॒फैः । \newline
2. श॒फै र॒च्छला॑भि र॒च्छला॑भिः श॒फैः श॒फै र॒च्छला॑भिः । \newline
3. अ॒च्छला॑भिः क॒पिञ्ज॑लान् क॒पिञ्ज॑ला न॒च्छला॑भि र॒च्छला॑भिः क॒पिञ्ज॑लान् । \newline
4. क॒पिञ्ज॑ला॒न् थ्साम॒ साम॑ क॒पिञ्ज॑लान् क॒पिञ्ज॑ला॒न् थ्साम॑ । \newline
5. साम॒ कुष्ठि॑काभिः॒ कुष्ठि॑काभिः॒ साम॒ साम॒ कुष्ठि॑काभिः । \newline
6. कुष्ठि॑काभिर् ज॒वम् ज॒वम् कुष्ठि॑काभिः॒ कुष्ठि॑काभिर् ज॒वम् । \newline
7. ज॒वम् जङ्घा॑भि॒र् जङ्घा॑भिर् ज॒वम् ज॒वम् जङ्घा॑भिः । \newline
8. जङ्घा॑भि रग॒द म॑ग॒दम् जङ्घा॑भि॒र् जङ्घा॑भि रग॒दम् । \newline
9. अ॒ग॒दम् जानु॑भ्या॒म् जानु॑भ्या मग॒द म॑ग॒दम् जानु॑भ्याम् । \newline
10. जानु॑भ्यां ॅवी॒र्यं॑ ॅवी॒र्य॑म् जानु॑भ्या॒म् जानु॑भ्यां ॅवी॒र्य᳚म् । \newline
11. जानु॑भ्या॒मिति॒ जानु॑ - भ्या॒म् । \newline
12. वी॒र्य॑म् कु॒हाभ्या᳚म् कु॒हाभ्यां᳚ ॅवी॒र्यं॑ ॅवी॒र्य॑म् कु॒हाभ्या᳚म् । \newline
13. कु॒हाभ्या᳚म् भ॒यम् भ॒यम् कु॒हाभ्या᳚म् कु॒हाभ्या᳚म् भ॒यम् । \newline
14. भ॒यम् प्र॑चा॒लाभ्या᳚म् प्रचा॒लाभ्या᳚म् भ॒यम् भ॒यम् प्र॑चा॒लाभ्या᳚म् । \newline
15. प्र॒चा॒लाभ्या॒म् गुहा॒ गुहा᳚ प्रचा॒लाभ्या᳚म् प्रचा॒लाभ्या॒म् गुहा᳚ । \newline
16. प्र॒चा॒लाभ्या॒मिति॑ प्र - चा॒लाभ्या᳚म् । \newline
17. गुहो॑ पप॒क्षाभ्या॑ मुपप॒क्षाभ्या॒म् गुहा॒ गुहो॑ पप॒क्षाभ्या᳚म् । \newline
18. उ॒प॒प॒क्षाभ्या॑ म॒श्विना॑ व॒श्विना॑ वुपप॒क्षाभ्या॑ मुपप॒क्षाभ्या॑ म॒श्विनौ᳚ । \newline
19. उ॒प॒प॒क्षाभ्या॒मित्यु॑प - प॒क्षाभ्या᳚म् । \newline
20. अ॒श्विना॒ वꣳसा᳚भ्या॒ मꣳसा᳚भ्या म॒श्विना॑ व॒श्विना॒ वꣳसा᳚भ्याम् । \newline
21. अꣳसा᳚भ्या॒ मदि॑ति॒ मदि॑ति॒ मꣳसा᳚भ्या॒ मꣳसा᳚भ्या॒ मदि॑तिम् । \newline
22. अदि॑तिꣳ शी॒र्ष्णा शी॒र्ष्णा ऽदि॑ति॒ मदि॑तिꣳ शी॒र्ष्णा । \newline
23. शी॒र्ष्णा निर्.ऋ॑ति॒न् निर्.ऋ॑तिꣳ शी॒र्ष्णा शी॒र्ष्णा निर्.ऋ॑तिम् । \newline
24. निर्.ऋ॑ति॒म् निर्जा᳚ल्मकेन॒ निर्जा᳚ल्मकेन॒ निर्.ऋ॑ति॒म् निर्.ऋ॑ति॒म् निर्जा᳚ल्मकेन । \newline
25. निर्.ऋ॑ति॒मिति॒ निः - ऋ॒ति॒म् । \newline
26. निर्जा᳚ल्मकेन शी॒र्ष्णा शी॒र्ष्णा निर्जा᳚ल्मकेन॒ निर्जा᳚ल्मकेन शी॒र्ष्णा । \newline
27. निर्जा᳚ल्मके॒नेति॒ निः - जा॒ल्म॒के॒न॒ । \newline
28. शी॒र्ष्णेति॑ शी॒र्ष्णा । \newline

\textbf{Ghana Paata } \newline

1. कू॒र्माञ् छ॒फैः श॒फैः कू॒र्मान् कू॒र्माञ् छ॒फै र॒च्छला॑भि र॒च्छला॑भिः श॒फैः कू॒र्मान् कू॒र्माञ् छ॒फै र॒च्छला॑भिः । \newline
2. श॒फै र॒च्छला॑भि र॒च्छला॑भिः श॒फैः श॒फै र॒च्छला॑भिः क॒पिञ्ज॑लान् क॒पिञ्ज॑ला न॒च्छला॑भिः श॒फैः श॒फै र॒च्छला॑भिः क॒पिञ्ज॑लान् । \newline
3. अ॒च्छला॑भिः क॒पिञ्ज॑लान् क॒पिञ्ज॑ला न॒च्छला॑भि र॒च्छला॑भिः क॒पिञ्ज॑ला॒न् थ्साम॒ साम॑ क॒पिञ्ज॑ला न॒च्छला॑भि र॒च्छला॑भिः क॒पिञ्ज॑ला॒न् थ्साम॑ । \newline
4. क॒पिञ्ज॑ला॒न् थ्साम॒ साम॑ क॒पिञ्ज॑लान् क॒पिञ्ज॑ला॒न् थ्साम॒ कुष्ठि॑काभिः॒ कुष्ठि॑काभिः॒ साम॑ क॒पिञ्ज॑लान् क॒पिञ्ज॑ला॒न् थ्साम॒ कुष्ठि॑काभिः । \newline
5. साम॒ कुष्ठि॑काभिः॒ कुष्ठि॑काभिः॒ साम॒ साम॒ कुष्ठि॑काभिर् ज॒वम् ज॒वम् कुष्ठि॑काभिः॒ साम॒ साम॒ कुष्ठि॑काभिर् ज॒वम् । \newline
6. कुष्ठि॑काभिर् ज॒वम् ज॒वम् कुष्ठि॑काभिः॒ कुष्ठि॑काभिर् ज॒वम् जङ्घा॑भि॒र् जङ्घा॑भिर् ज॒वम् कुष्ठि॑काभिः॒ कुष्ठि॑काभिर् ज॒वम् जङ्घा॑भिः । \newline
7. ज॒वम् जङ्घा॑भि॒र् जङ्घा॑भिर् ज॒वम् ज॒वम् जङ्घा॑भि रग॒द म॑ग॒दम् जङ्घा॑भिर् ज॒वम् ज॒वम् जङ्घा॑भि रग॒दम् । \newline
8. जङ्घा॑भि रग॒द म॑ग॒दम् जङ्घा॑भि॒र् जङ्घा॑भि रग॒दम् जानु॑भ्या॒म् जानु॑भ्या मग॒दम् जङ्घा॑भि॒र् जङ्घा॑भि रग॒दम् जानु॑भ्याम् । \newline
9. अ॒ग॒दम् जानु॑भ्या॒म् जानु॑भ्या मग॒द म॑ग॒दम् जानु॑भ्यां ॅवी॒र्यं॑ ॅवी॒र्य॑म् जानु॑भ्या मग॒द म॑ग॒दम् जानु॑भ्यां ॅवी॒र्य᳚म् । \newline
10. जानु॑भ्यां ॅवी॒र्यं॑ ॅवी॒र्य॑म् जानु॑भ्या॒म् जानु॑भ्यां ॅवी॒र्य॑म् कु॒हाभ्या᳚म् कु॒हाभ्यां᳚ ॅवी॒र्य॑म् जानु॑भ्या॒म् जानु॑भ्यां ॅवी॒र्य॑म् कु॒हाभ्या᳚म् । \newline
11. जानु॑भ्या॒मिति॒ जानु॑ - भ्या॒म् । \newline
12. वी॒र्य॑म् कु॒हाभ्या᳚म् कु॒हाभ्यां᳚ ॅवी॒र्यं॑ ॅवी॒र्य॑म् कु॒हाभ्या᳚म् भ॒यम् भ॒यम् कु॒हाभ्यां᳚ ॅवी॒र्यं॑ ॅवी॒र्य॑म् कु॒हाभ्या᳚म् भ॒यम् । \newline
13. कु॒हाभ्या᳚म् भ॒यम् भ॒यम् कु॒हाभ्या᳚म् कु॒हाभ्या᳚म् भ॒यम् प्र॑चा॒लाभ्या᳚म् प्रचा॒लाभ्या᳚म् भ॒यम् कु॒हाभ्या᳚म् कु॒हाभ्या᳚म् भ॒यम् प्र॑चा॒लाभ्या᳚म् । \newline
14. भ॒यम् प्र॑चा॒लाभ्या᳚म् प्रचा॒लाभ्या᳚म् भ॒यम् भ॒यम् प्र॑चा॒लाभ्या॒म् गुहा॒ गुहा᳚ प्रचा॒लाभ्या᳚म् भ॒यम् भ॒यम् प्र॑चा॒लाभ्या॒म् गुहा᳚ । \newline
15. प्र॒चा॒लाभ्या॒म् गुहा॒ गुहा᳚ प्रचा॒लाभ्या᳚म् प्रचा॒लाभ्या॒म् गुहो॑ पप॒क्षाभ्या॑ मुपप॒क्षाभ्या॒म् गुहा᳚ प्रचा॒लाभ्या᳚म् प्रचा॒लाभ्या॒म् गुहो॑ पप॒क्षाभ्या᳚म् । \newline
16. प्र॒चा॒लाभ्या॒मिति॑ प्र - चा॒लाभ्या᳚म् । \newline
17. गुहो॑ पप॒क्षाभ्या॑ मुपप॒क्षाभ्या॒म् गुहा॒ गुहो॑ पप॒क्षाभ्या॑ म॒श्विना॑ व॒श्विना॑ वुपप॒क्षाभ्या॒म् गुहा॒ गुहो॑ पप॒क्षाभ्या॑ म॒श्विनौ᳚ । \newline
18. उ॒प॒प॒क्षाभ्या॑ म॒श्विना॑ व॒श्विना॑ वुपप॒क्षाभ्या॑ मुपप॒क्षाभ्या॑ म॒श्विना॒ वꣳसा᳚भ्या॒ मꣳसा᳚भ्या म॒श्विना॑ वुपप॒क्षाभ्या॑ मुपप॒क्षाभ्या॑ म॒श्विना॒ वꣳसा᳚भ्याम् । \newline
19. उ॒प॒प॒क्षाभ्या॒मित्यु॑प - प॒क्षाभ्या᳚म् । \newline
20. अ॒श्विना॒ वꣳसा᳚भ्या॒ मꣳसा᳚भ्या म॒श्विना॑ व॒श्विना॒ वꣳसा᳚भ्या॒ मदि॑ति॒ मदि॑ति॒ मꣳसा᳚भ्या म॒श्विना॑ व॒श्विना॒ वꣳसा᳚भ्या॒ मदि॑तिम् । \newline
21. अꣳसा᳚भ्या॒ मदि॑ति॒ मदि॑ति॒ मꣳसा᳚भ्या॒ मꣳसा᳚भ्या॒ मदि॑तिꣳ शी॒र्ष्णा शी॒र्ष्णा ऽदि॑ति॒ मꣳसा᳚भ्या॒ मꣳसा᳚भ्या॒ मदि॑तिꣳ शी॒र्ष्णा । \newline
22. अदि॑तिꣳ शी॒र्ष्णा शी॒र्ष्णा ऽदि॑ति॒ मदि॑तिꣳ शी॒र्ष्णा निर्.ऋ॑ति॒म् निर्.ऋ॑तिꣳ शी॒र्ष्णा ऽदि॑ति॒ मदि॑तिꣳ शी॒र्ष्णा निर्.ऋ॑तिम् । \newline
23. शी॒र्ष्णा निर्.ऋ॑ति॒म् निर्.ऋ॑तिꣳ शी॒र्ष्णा शी॒र्ष्णा निर्.ऋ॑ति॒म् निर्जा᳚ल्मकेन॒ निर्जा᳚ल्मकेन॒ निर्.ऋ॑तिꣳ शी॒र्ष्णा शी॒र्ष्णा निर्.ऋ॑ति॒म् निर्जा᳚ल्मकेन । \newline
24. निर्.ऋ॑ति॒म् निर्जा᳚ल्मकेन॒ निर्जा᳚ल्मकेन॒ निर्.ऋ॑ति॒म् निर्.ऋ॑ति॒म् निर्जा᳚ल्मकेन शी॒र्ष्णा शी॒र्ष्णा निर्जा᳚ल्मकेन॒ निर्.ऋ॑ति॒म् निर्.ऋ॑ति॒म् निर्जा᳚ल्मकेन शी॒र्ष्णा । \newline
25. निर्.ऋ॑ति॒मिति॒ निः - ऋ॒ति॒म् । \newline
26. निर्जा᳚ल्मकेन शी॒र्ष्णा शी॒र्ष्णा निर्जा᳚ल्मकेन॒ निर्जा᳚ल्मकेन शी॒र्ष्णा । \newline
27. निर्जा᳚ल्मके॒नेति॒ निः - जा॒ल्म॒के॒न॒ । \newline
28. शी॒र्ष्णेति॑ शी॒र्ष्णा । \newline
\pagebreak
\markright{ TS 5.7.14.1  \hfill https://www.vedavms.in \hfill}

\section{ TS 5.7.14.1 }

\textbf{TS 5.7.14.1 } \newline
\textbf{Samhita Paata} \newline

योक्त्रं॒ गृद्ध्रा॑भिर्यु॒गमान॑तेन चि॒त्तं मन्या॑भिः संक्रो॒शान् प्रा॒णैः प्र॑का॒शेन॒ त्वचं॑ पराका॒शेनान्त॑रां म॒शका॒न् केशै॒रिन्द्रꣳ॒॒ स्वप॑सा॒ वहे॑न॒ बृह॒स्पतिꣳ॑ शकुनिसा॒देन॒ रथ॑मु॒ष्णिहा॑भिः ॥ \newline

\textbf{Pada Paata} \newline

योक्त्र᳚म् । गृद्ध्रा॑भिः । यु॒गम् । आन॑ते॒नेत्या - न॒ते॒न॒ । चि॒त्तम् । मन्या॑भिः । स॒क्र्ॐ॒शानिति॑ सं - क्रो॒शान् । प्रा॒णैरिति॑ प्र - अ॒नैः । प्र॒का॒शेनेति॑ प्र - का॒शेन॑ । त्वच᳚म् । प॒रा॒का॒शेनेति॑ परा - का॒शेन॑ । अन्त॑राम् । म॒शकान्॑ । केशैः᳚ । इन्द्र᳚म् । स्वप॒सेति॑ सु - अप॑सा । वहे॑न । बृह॒स्पति᳚म् । श॒कु॒नि॒सा॒देनेति॑ शकुनि - सा॒देन॑ । रथ᳚म् । उ॒ष्णिहा॑भिः ॥  \newline


\textbf{Krama Paata} \newline

योक्त्र॒म् गृध्रा॑भिः । गृध्रा॑भिर् यु॒गम् । यु॒गमान॑तेन । आन॑तेन चि॒त्तम् । आन॑ते॒नेत्या - न॒ते॒न॒ । चि॒त्तम् मन्या॑भिः । मन्या॑भिः सङ्क्रो॒शान् । स॒ङ्क्रो॒शान् प्रा॒णैः । स॒ङ्क्रो॒शानिति॑ सम् - क्रो॒शान् । प्रा॒णैः प्र॑का॒शेन॑ । प्रा॒णैरिति॑ प्र - अ॒नैः । प्र॒का॒शेन॒ त्वच᳚म् । प्र॒का॒शेनेति॑ प्र - का॒शेन॑ । त्वच॑म् पराका॒शेन॑ । प॒रा॒का॒शेनान्त॑राम् । प॒रा॒का॒शेनेति॑ परा -  का॒शेन॑ । अन्त॑राम् म॒शकान्॑ । म॒शका॒न् केशैः᳚ । केशै॒रिन्द्र᳚म् । इन्द्रꣳ॒॒ स्वप॑सा । स्वप॑सा॒ वहे॑न । स्वप॒सेति॑ सु - अप॑सा । वहे॑न॒ बृह॒स्पति᳚म् । बृह॒स्पतिꣳ॑ शकुनिसा॒देन॑ । श॒कु॒नि॒सा॒देन॒ रथ᳚म् । श॒कु॒नि॒सा॒देनेति॑ शकुनि - सा॒देन॑ । रथ॑मु॒ष्णिहा॑भिः । उ॒ष्णिहा॑भि॒रित्यु॒ष्णिहा॑भिः । \newline

\textbf{Jatai Paata} \newline

1. योक्त्र॒म् गृद्ध्रा॑भि॒र् गृद्ध्रा॑भि॒र् योक्त्रं॒ ॅयोक्त्र॒म् गृद्ध्रा॑भिः । \newline
2. गृद्ध्रा॑भिर् यु॒गं ॅयु॒गम् गृद्ध्रा॑भि॒र् गृद्ध्रा॑भिर् यु॒गम् । \newline
3. यु॒ग मान॑ते॒ना न॑तेन यु॒गं ॅयु॒ग मान॑तेन । \newline
4. आन॑तेन चि॒त्तम् चि॒त्त मान॑ते॒ना न॑तेन चि॒त्तम् । \newline
5. आन॑ते॒नेत्या - न॒ते॒न॒ । \newline
6. चि॒त्तम् मन्या॑भि॒र् मन्या॑भि श्चि॒त्तम् चि॒त्तम् मन्या॑भिः । \newline
7. मन्या॑भिः सङ्क्रो॒शान् थ्स॑ङ्क्रो॒शान् मन्या॑भि॒र् मन्या॑भिः सङ्क्रो॒शान् । \newline
8. स॒ङ्क्रो॒शान् प्रा॒णैः प्रा॒णैः स॑ङ्क्रो॒शान् थ्स॑ङ्क्रो॒शान् प्रा॒णैः । \newline
9. स॒ङ्क्रो॒शानिति॑ सं - क्रो॒शान् । \newline
10. प्रा॒णैः प्र॑का॒शेन॑ प्रका॒शेन॑ प्रा॒णैः प्रा॒णैः प्र॑का॒शेन॑ । \newline
11. प्रा॒णैरिति॑ प्र - अ॒नैः । \newline
12. प्र॒का॒शेन॒ त्वच॒म् त्वच॑म् प्रका॒शेन॑ प्रका॒शेन॒ त्वच᳚म् । \newline
13. प्र॒का॒शेनेति॑ प्र - का॒शेन॑ । \newline
14. त्वच॑म् पराका॒शेन॑ पराका॒शेन॒ त्वच॒म् त्वच॑म् पराका॒शेन॑ । \newline
15. प॒रा॒का॒शेना न्त॑रा॒ मन्त॑राम् पराका॒शेन॑ पराका॒शेना न्त॑राम् । \newline
16. प॒रा॒का॒शेनेति॑ परा - का॒शेन॑ । \newline
17. अन्त॑राम् म॒शका᳚न् म॒शका॒ नन्त॑रा॒ मन्त॑राम् म॒शकान्॑ । \newline
18. म॒शका॒न् केशैः॒ केशै᳚र् म॒शका᳚न् म॒शका॒न् केशैः᳚ । \newline
19. केशै॒ रिन्द्र॒ मिन्द्र॒म् केशैः॒ केशै॒ रिन्द्र᳚म् । \newline
20. इन्द्रꣳ॒॒ स्वप॑सा॒ स्वप॒ सेन्द्र॒ मिन्द्रꣳ॒॒ स्वप॑सा । \newline
21. स्वप॑सा॒ वहे॑न॒ वहे॑न॒ स्वप॑सा॒ स्वप॑सा॒ वहे॑न । \newline
22. स्वप॒सेति॑ सु - अप॑सा । \newline
23. वहे॑न॒ बृह॒स्पति॒म् बृह॒स्पतिं॒ ॅवहे॑न॒ वहे॑न॒ बृह॒स्पति᳚म् । \newline
24. बृह॒स्पतिꣳ॑ शकुनिसा॒देन॑ शकुनिसा॒देन॒ बृह॒स्पति॒म् बृह॒स्पतिꣳ॑ शकुनिसा॒देन॑ । \newline
25. श॒कु॒नि॒सा॒देन॒ रथꣳ॒॒ रथꣳ॑ शकुनिसा॒देन॑ शकुनिसा॒देन॒ रथ᳚म् । \newline
26. श॒कु॒नि॒सा॒देनेति॑ शकुनि - सा॒देन॑ । \newline
27. रथ॑ मु॒ष्णिहा॑भि रु॒ष्णिहा॑भी॒ रथꣳ॒॒ रथ॑ मु॒ष्णिहा॑भिः । \newline
28. उ॒ष्णिहा॑भि॒रित्यु॒ष्णिहा॑भिः । \newline

\textbf{Ghana Paata } \newline

1. योक्त्र॒म् गृद्ध्रा॑भि॒र् गृद्ध्रा॑भि॒र् योक्त्रं॒ ॅयोक्त्र॒म् गृद्ध्रा॑भिर् यु॒गं ॅयु॒गम् गृद्ध्रा॑भि॒र् योक्त्रं॒ ॅयोक्त्र॒म् गृद्ध्रा॑भिर् यु॒गम् । \newline
2. गृद्ध्रा॑भिर् यु॒गं ॅयु॒गम् गृद्ध्रा॑भि॒र् गृद्ध्रा॑भिर् यु॒ग मान॑ते॒ना न॑तेन यु॒गम् गृद्ध्रा॑भि॒र् गृद्ध्रा॑भिर् यु॒ग मान॑तेन । \newline
3. यु॒ग मान॑ते॒ना न॑तेन यु॒गं ॅयु॒ग मान॑तेन चि॒त्तम् चि॒त्त मान॑तेन यु॒गं ॅयु॒ग मान॑तेन चि॒त्तम् । \newline
4. आन॑तेन चि॒त्तम् चि॒त्त मान॑ते॒ना न॑तेन चि॒त्तम् मन्या॑भि॒र् मन्या॑भि श्चि॒त्त मान॑ते॒ना न॑तेन चि॒त्तम् मन्या॑भिः । \newline
5. आन॑ते॒नेत्या - न॒ते॒न॒ । \newline
6. चि॒त्तम् मन्या॑भि॒र् मन्या॑भि श्चि॒त्तम् चि॒त्तम् मन्या॑भिः सङ्क्रो॒शान् थ्स॑ङ्क्रो॒शान् मन्या॑भि श्चि॒त्तम् चि॒त्तम् मन्या॑भिः सङ्क्रो॒शान् । \newline
7. मन्या॑भिः सङ्क्रो॒शान् थ्स॑ङ्क्रो॒शान् मन्या॑भि॒र् मन्या॑भिः सङ्क्रो॒शान् प्रा॒णैः प्रा॒णैः स॑ङ्क्रो॒शान् मन्या॑भि॒र् मन्या॑भिः सङ्क्रो॒शान् प्रा॒णैः । \newline
8. स॒ङ्क्रो॒शान् प्रा॒णैः प्रा॒णैः स॑ङ्क्रो॒शान् थ्स॑ङ्क्रो॒शान् प्रा॒णैः प्र॑का॒शेन॑ प्रका॒शेन॑ प्रा॒णैः स॑ङ्क्रो॒शान् थ्स॑ङ्क्रो॒शान् प्रा॒णैः प्र॑का॒शेन॑ । \newline
9. स॒ङ्क्रो॒शानिति॑ सं - क्रो॒शान् । \newline
10. प्रा॒णैः प्र॑का॒शेन॑ प्रका॒शेन॑ प्रा॒णैः प्रा॒णैः प्र॑का॒शेन॒ त्वच॒म् त्वच॑म् प्रका॒शेन॑ प्रा॒णैः प्रा॒णैः प्र॑का॒शेन॒ त्वच᳚म् । \newline
11. प्रा॒णैरिति॑ प्र - अ॒नैः । \newline
12. प्र॒का॒शेन॒ त्वच॒म् त्वच॑म् प्रका॒शेन॑ प्रका॒शेन॒ त्वच॑म् पराका॒शेन॑ पराका॒शेन॒ त्वच॑म् प्रका॒शेन॑ प्रका॒शेन॒ त्वच॑म् पराका॒शेन॑ । \newline
13. प्र॒का॒शेनेति॑ प्र - का॒शेन॑ । \newline
14. त्वच॑म् पराका॒शेन॑ पराका॒शेन॒ त्वच॒म् त्वच॑म् पराका॒शे नान्त॑रा॒ मन्त॑राम् पराका॒शेन॒ त्वच॒म् त्वच॑म् पराका॒शे नान्त॑राम् । \newline
15. प॒रा॒का॒शे नान्त॑रा॒ मन्त॑राम् पराका॒शेन॑ पराका॒शे नान्त॑राम् म॒शका᳚न् म॒शका॒ नन्त॑राम् पराका॒शेन॑ पराका॒शे नान्त॑राम् म॒शकान्॑ । \newline
16. प॒रा॒का॒शेनेति॑ परा - का॒शेन॑ । \newline
17. अन्त॑राम् म॒शका᳚न् म॒शका॒ नन्त॑रा॒ मन्त॑राम् म॒शका॒न् केशैः॒ केशै᳚र् म॒शका॒ नन्त॑रा॒ मन्त॑राम् म॒शका॒न् केशैः᳚ । \newline
18. म॒शका॒न् केशैः॒ केशै᳚र् म॒शका᳚न् म॒शका॒न् केशै॒ रिन्द्र॒ मिन्द्र॒म् केशै᳚र् म॒शका᳚न् म॒शका॒न् केशै॒ रिन्द्र᳚म् । \newline
19. केशै॒ रिन्द्र॒ मिन्द्र॒म् केशैः॒ केशै॒ रिन्द्रꣳ॒॒ स्वप॑सा॒ स्वप॒सेन्द्र॒म् केशैः॒ केशै॒ रिन्द्रꣳ॒॒ स्वप॑सा । \newline
20. इन्द्रꣳ॒॒ स्वप॑सा॒ स्वप॒सेन्द्र॒ मिन्द्रꣳ॒॒ स्वप॑सा॒ वहे॑न॒ वहे॑न॒ स्वप॒सेन्द्र॒ मिन्द्रꣳ॒॒ स्वप॑सा॒ वहे॑न । \newline
21. स्वप॑सा॒ वहे॑न॒ वहे॑न॒ स्वप॑सा॒ स्वप॑सा॒ वहे॑न॒ बृह॒स्पति॒म् बृह॒स्पतिं॒ ॅवहे॑न॒ स्वप॑सा॒ स्वप॑सा॒ वहे॑न॒ बृह॒स्पति᳚म् । \newline
22. स्वप॒सेति॑ सु - अप॑सा । \newline
23. वहे॑न॒ बृह॒स्पति॒म् बृह॒स्पतिं॒ ॅवहे॑न॒ वहे॑न॒ बृह॒स्पतिꣳ॑ शकुनिसा॒देन॑ शकुनिसा॒देन॒ बृह॒स्पतिं॒ ॅवहे॑न॒ वहे॑न॒ बृह॒स्पतिꣳ॑ शकुनिसा॒देन॑ । \newline
24. बृह॒स्पतिꣳ॑ शकुनिसा॒देन॑ शकुनिसा॒देन॒ बृह॒स्पति॒म् बृह॒स्पतिꣳ॑ शकुनिसा॒देन॒ रथꣳ॒॒ रथꣳ॑ शकुनिसा॒देन॒ बृह॒स्पति॒म् बृह॒स्पतिꣳ॑ शकुनिसा॒देन॒ रथ᳚म् । \newline
25. श॒कु॒नि॒सा॒देन॒ रथꣳ॒॒ रथꣳ॑ शकुनिसा॒देन॑ शकुनिसा॒देन॒ रथ॑ मु॒ष्णिहा॑भि रु॒ष्णिहा॑भी॒ रथꣳ॑ शकुनिसा॒देन॑ शकुनिसा॒देन॒ रथ॑ मु॒ष्णिहा॑भिः । \newline
26. श॒कु॒नि॒सा॒देनेति॑ शकुनि - सा॒देन॑ । \newline
27. रथ॑ मु॒ष्णिहा॑भि रु॒ष्णिहा॑भी॒ रथꣳ॒॒ रथ॑ मु॒ष्णिहा॑भिः । \newline
28. उ॒ष्णिहा॑भि॒रित्यु॒ष्णिहा॑भिः । \newline
\pagebreak
\markright{ TS 5.7.15.1  \hfill https://www.vedavms.in \hfill}

\section{ TS 5.7.15.1 }

\textbf{TS 5.7.15.1 } \newline
\textbf{Samhita Paata} \newline

मि॒त्रावरु॑णौ॒ श्रोणी᳚भ्यामिन्द्रा॒ग्नी शि॑ख॒ण्डाभ्या॒-मिन्द्रा॒बृह॒स्पती॑ ऊ॒रुभ्या॒मिन्द्रा॒विष्णू॑ अष्ठी॒वद्भ्याꣳ॑ सवि॒तारं॒ पुच्छे॑न गन्ध॒र्वाञ्छेपे॑ना-फ्स॒रसो॑ मु॒ष्काभ्यां॒ पव॑मानं पा॒युना॑ प॒वित्रं॒ पोत्रा᳚भ्यामा॒क्रम॑णꣳ स्थू॒राभ्यां᳚ प्रति॒क्रम॑णं॒ कुष्ठा᳚भ्यां ॥ \newline

\textbf{Pada Paata} \newline

मि॒त्रावरु॑णा॒विति॑ मि॒त्रा - वरु॑णौ । श्रोणी᳚भ्या॒मिति॒ श्रोणि॑ - भ्या॒म् । इ॒न्द्रा॒ग्नी इती᳚न्द्र - अ॒ग्नी । शि॒ख॒ण्डाभ्या᳚म् । इन्द्रा॒बृह॒स्पती॒ इतीन्द्रा᳚ - बृह॒स्पती᳚ । ऊ॒रुभ्या॒मित्यू॒रु - भ्या॒म् । इन्द्रा॒विष्णू॒ इतीन्द्रा᳚ - विष्णू᳚ । अ॒ष्ठी॒वद्भ्या॒मित्य॑ष्ठी॒वत् - भ्या॒म् । स॒वि॒तार᳚म् । पुच्छे॑न । ग॒न्ध॒र्वान् । शेपे॑न । अ॒फ्स॒रसः॑ । मु॒ष्काभ्या᳚म् । पव॑मानम् । पा॒युना᳚ । प॒वित्र᳚म् । पोत्रा᳚भ्याम् । आ॒क्रम॑ण॒मित्या᳚ - क्रम॑णम् । स्थू॒राभ्या᳚म् । प्र॒ति॒क्रम॑ण॒मिति॑ प्रति - क्रम॑णम् । कुष्ठा᳚भ्याम् ॥  \newline


\textbf{Krama Paata} \newline

मि॒त्रावरु॑णौ॒ श्रोणी᳚भ्याम् । मि॒त्रावरु॑णा॒विति॑ मि॒त्रा - वरु॑णौ । श्रोणी᳚भ्यामिन्द्रा॒ग्नी । श्रोणी᳚भ्या॒मिति॒ श्रोणि॑ - भ्या॒म् । इ॒न्द्रा॒ग्नी शि॑ख॒ण्डाभ्या᳚म् । इ॒न्द्रा॒ग्नी इती᳚न्द्र - अ॒ग्नी । शि॒ख॒ण्डाभ्या॒मिन्द्रा॒बृह॒स्पती᳚ । इन्द्रा॒बृह॒स्पती॑ ऊ॒रुभ्या᳚म् । इन्द्रा॒बृह॒स्पती॒ इतीन्द्रा᳚ - बृह॒स्पती᳚ । ऊ॒रुभ्या॒मिन्द्रा॒विष्णू᳚ । ऊ॒रुभ्या॒मित्यू॒रु - भ्या॒म् । इन्द्रा॒विष्णू॑ अष्ठी॒वद्भ्या᳚म् । इन्द्रा॒विष्णू॒ इतीन्द्रा᳚ - विष्णू᳚ । अ॒ष्ठी॒वद्भ्याꣳ॑ सवि॒तार᳚म् । अ॒ष्ठी॒वद्भ्या॒मित्य॑ष्ठी॒वत् - भ्या॒म् । स॒वि॒तार॒म् पुच्छे॑न । पुच्छे॑न गन्ध॒र्वान् । ग॒न्ध॒र्वाञ्छेपे॑न । शेपे॑नाफ्स॒रसः॑ । अ॒फ्स॒रसो॑ मु॒ष्काभ्या᳚म् । मु॒ष्काभ्या॒म् पव॑मानम् । पव॑मानम् पा॒युना᳚ । पा॒युना॑ प॒वित्र᳚म् । प॒वित्र॒म् पोत्रा᳚भ्याम् । पोत्रा᳚भ्यामा॒क्रम॑णम् । आ॒क्रम॑णꣳ स्थू॒राभ्या᳚म् । आ॒क्रम॑ण॒मित्या᳚ - क्रम॑णम् । स्थू॒राभ्या᳚म् प्रति॒क्रम॑णम् । प्र॒ति॒क्रम॑ण॒म् कुष्ठा᳚भ्याम् । प्र॒ति॒क्रम॑ण॒मिति॑ प्रति - क्रम॑णम् । कुष्ठा᳚भ्या॒मिति॒ कुष्ठा᳚भ्याम् । \newline

\textbf{Jatai Paata} \newline

1. मि॒त्रावरु॑णौ॒ श्रोणी᳚भ्याꣳ॒॒ श्रोणी᳚भ्याम् मि॒त्रावरु॑णौ मि॒त्रावरु॑णौ॒ श्रोणी᳚भ्याम् । \newline
2. मि॒त्रावरु॑णा॒विति॑ मि॒त्रा - वरु॑णौ । \newline
3. श्रोणी᳚भ्या मिन्द्रा॒ग्नी इ॑न्द्रा॒ग्नी श्रोणी᳚भ्याꣳ॒॒ श्रोणी᳚भ्या मिन्द्रा॒ग्नी । \newline
4. श्रोणी᳚भ्या॒मिति॒ श्रोणि॑ - भ्या॒म् । \newline
5. इ॒न्द्रा॒ग्नी शि॑ख॒ण्डाभ्याꣳ॑ शिख॒ण्डाभ्या॑ मिन्द्रा॒ग्नी इ॑न्द्रा॒ग्नी शि॑ख॒ण्डाभ्या᳚म् । \newline
6. इ॒न्द्रा॒ग्नी इती᳚न्द्र - अ॒ग्नी । \newline
7. शि॒ख॒ण्डाभ्या॒ मिन्द्रा॒बृह॒स्पती॒ इन्द्रा॒बृह॒स्पती॑ शिख॒ण्डाभ्याꣳ॑ शिख॒ण्डाभ्या॒ मिन्द्रा॒बृह॒स्पती᳚ । \newline
8. इन्द्रा॒बृह॒स्पती॑ ऊ॒रुभ्या॑ मू॒रुभ्या॒ मिन्द्रा॒बृह॒स्पती॒ इन्द्रा॒बृह॒स्पती॑ ऊ॒रुभ्या᳚म् । \newline
9. इन्द्रा॒बृह॒स्पती॒ इतीन्द्रा᳚ - बृह॒स्पती᳚ । \newline
10. ऊ॒रुभ्या॒ मिन्द्रा॒विष्णू॒ इन्द्रा॒विष्णू॑ ऊ॒रुभ्या॑ मू॒रुभ्या॒ मिन्द्रा॒विष्णू᳚ । \newline
11. ऊ॒रुभ्या॒मित्यू॒रु - भ्या॒म् । \newline
12. इन्द्रा॒विष्णू॑ अष्ठी॒वद्भ्या॑ मष्ठी॒वद्भ्या॒ मिन्द्रा॒विष्णू॒ इन्द्रा॒विष्णू॑ अष्ठी॒वद्भ्या᳚म् । \newline
13. इन्द्रा॒विष्णू॒ इतीन्द्रा᳚ - विष्णू᳚ । \newline
14. अ॒ष्ठी॒वद्भ्याꣳ॑ सवि॒तारꣳ॑ सवि॒तार॑ मष्ठी॒वद्भ्या॑ मष्ठी॒वद्भ्याꣳ॑ सवि॒तार᳚म् । \newline
15. अ॒ष्ठी॒वद्भ्या॒मित्य॑ष्ठी॒वत् - भ्या॒म् । \newline
16. स॒वि॒तार॒म् पुच्छे॑न॒ पुच्छे॑न सवि॒तारꣳ॑ सवि॒तार॒म् पुच्छे॑न । \newline
17. पुच्छे॑न गन्ध॒र्वान् ग॑न्ध॒र्वान् पुच्छे॑न॒ पुच्छे॑न गन्ध॒र्वान् । \newline
18. ग॒न्ध॒र्वाञ् छेपे॑न॒ शेपे॑न गन्ध॒र्वान् ग॑न्ध॒र्वाञ् छेपे॑न । \newline
19. शेपे॑ना फ्स॒रसो᳚ ऽफ्स॒रसः॒ शेपे॑न॒ शेपे॑ना फ्स॒रसः॑ । \newline
20. अ॒फ्स॒रसो॑ मु॒ष्काभ्या᳚म् मु॒ष्काभ्या॑ मफ्स॒रसो᳚ ऽफ्स॒रसो॑ मु॒ष्काभ्या᳚म् । \newline
21. मु॒ष्काभ्या॒म् पव॑मान॒म् पव॑मानम् मु॒ष्काभ्या᳚म् मु॒ष्काभ्या॒म् पव॑मानम् । \newline
22. पव॑मानम् पा॒युना॑ पा॒युना॒ पव॑मान॒म् पव॑मानम् पा॒युना᳚ । \newline
23. पा॒युना॑ प॒वित्र॑म् प॒वित्र॑म् पा॒युना॑ पा॒युना॑ प॒वित्र᳚म् । \newline
24. प॒वित्र॒म् पोत्रा᳚भ्या॒म् पोत्रा᳚भ्याम् प॒वित्र॑म् प॒वित्र॒म् पोत्रा᳚भ्याम् । \newline
25. पोत्रा᳚भ्या मा॒क्रम॑ण मा॒क्रम॑ण॒म् पोत्रा᳚भ्या॒म् पोत्रा᳚भ्या मा॒क्रम॑णम् । \newline
26. आ॒क्रम॑णꣳ स्थू॒राभ्याꣳ॑ स्थू॒राभ्या॑ मा॒क्रम॑ण मा॒क्रम॑णꣳ स्थू॒राभ्या᳚म् । \newline
27. आ॒क्रम॑ण॒मित्या᳚ - क्रम॑णम् । \newline
28. स्थू॒राभ्या᳚म् प्रति॒क्रम॑णम् प्रति॒क्रम॑णꣳ स्थू॒राभ्याꣳ॑ स्थू॒राभ्या᳚म् प्रति॒क्रम॑णम् । \newline
29. प्र॒ति॒क्रम॑ण॒म् कुष्ठा᳚भ्या॒म् कुष्ठा᳚भ्याम् प्रति॒क्रम॑णम् प्रति॒क्रम॑ण॒म् कुष्ठा᳚भ्याम् । \newline
30. प्र॒ति॒क्रम॑ण॒मिति॑ प्रति - क्रम॑णम् । \newline
31. कुष्ठा᳚भ्या॒मिति॒ कुष्ठा᳚भ्याम् । \newline

\textbf{Ghana Paata } \newline

1. मि॒त्रावरु॑णौ॒ श्रोणी᳚भ्याꣳ॒॒ श्रोणी᳚भ्याम् मि॒त्रावरु॑णौ मि॒त्रावरु॑णौ॒ श्रोणी᳚भ्या मिन्द्रा॒ग्नी इ॑न्द्रा॒ग्नी श्रोणी᳚भ्याम् मि॒त्रावरु॑णौ मि॒त्रावरु॑णौ॒ श्रोणी᳚भ्या मिन्द्रा॒ग्नी । \newline
2. मि॒त्रावरु॑णा॒विति॑ मि॒त्रा - वरु॑णौ । \newline
3. श्रोणी᳚भ्या मिन्द्रा॒ग्नी इ॑न्द्रा॒ग्नी श्रोणी᳚भ्याꣳ॒॒ श्रोणी᳚भ्या मिन्द्रा॒ग्नी शि॑ख॒ण्डाभ्याꣳ॑ शिख॒ण्डाभ्या॑ मिन्द्रा॒ग्नी श्रोणी᳚भ्याꣳ॒॒ श्रोणी᳚भ्या मिन्द्रा॒ग्नी शि॑ख॒ण्डाभ्या᳚म् । \newline
4. श्रोणी᳚भ्या॒मिति॒ श्रोणि॑ - भ्या॒म् । \newline
5. इ॒न्द्रा॒ग्नी शि॑ख॒ण्डाभ्याꣳ॑ शिख॒ण्डाभ्या॑ मिन्द्रा॒ग्नी इ॑न्द्रा॒ग्नी शि॑ख॒ण्डाभ्या॒ मिन्द्रा॒बृह॒स्पती॒ इन्द्रा॒बृह॒स्पती॑ शिख॒ण्डाभ्या॑ मिन्द्रा॒ग्नी इ॑न्द्रा॒ग्नी शि॑ख॒ण्डाभ्या॒ मिन्द्रा॒बृह॒स्पती᳚ । \newline
6. इ॒न्द्रा॒ग्नी इती᳚न्द्र - अ॒ग्नी । \newline
7. शि॒ख॒ण्डाभ्या॒ मिन्द्रा॒बृह॒स्पती॒ इन्द्रा॒बृह॒स्पती॑ शिख॒ण्डाभ्याꣳ॑ शिख॒ण्डाभ्या॒ मिन्द्रा॒बृह॒स्पती॑ ऊ॒रुभ्या॑ मू॒रुभ्या॒ मिन्द्रा॒बृह॒स्पती॑ शिख॒ण्डाभ्याꣳ॑ शिख॒ण्डाभ्या॒ मिन्द्रा॒बृह॒स्पती॑ ऊ॒रुभ्या᳚म् । \newline
8. इन्द्रा॒बृह॒स्पती॑ ऊ॒रुभ्या॑ मू॒रुभ्या॒ मिन्द्रा॒बृह॒स्पती॒ इन्द्रा॒बृह॒स्पती॑ ऊ॒रुभ्या॒ मिन्द्रा॒विष्णू॒ इन्द्रा॒विष्णू॑ ऊ॒रुभ्या॒ मिन्द्रा॒बृह॒स्पती॒ इन्द्रा॒बृह॒स्पती॑ ऊ॒रुभ्या॒ मिन्द्रा॒विष्णू᳚ । \newline
9. इन्द्रा॒बृह॒स्पती॒ इतीन्द्रा᳚ - बृह॒स्पती᳚ । \newline
10. ऊ॒रुभ्या॒ मिन्द्रा॒विष्णू॒ इन्द्रा॒विष्णू॑ ऊ॒रुभ्या॑ मू॒रुभ्या॒ मिन्द्रा॒विष्णू॑ अष्ठी॒वद्भ्या॑ मष्ठी॒वद्भ्या॒ मिन्द्रा॒विष्णू॑ ऊ॒रुभ्या॑ मू॒रुभ्या॒ मिन्द्रा॒विष्णू॑ अष्ठी॒वद्भ्या᳚म् । \newline
11. ऊ॒रुभ्या॒मित्यू॒रु - भ्या॒म् । \newline
12. इन्द्रा॒विष्णू॑ अष्ठी॒वद्भ्या॑ मष्ठी॒वद्भ्या॒ मिन्द्रा॒विष्णू॒ इन्द्रा॒विष्णू॑ अष्ठी॒वद्भ्याꣳ॑ सवि॒तारꣳ॑ सवि॒तार॑ मष्ठी॒वद्भ्या॒ मिन्द्रा॒विष्णू॒ इन्द्रा॒विष्णू॑ अष्ठी॒वद्भ्याꣳ॑ सवि॒तार᳚म् । \newline
13. इन्द्रा॒विष्णू॒ इतीन्द्रा᳚ - विष्णू᳚ । \newline
14. अ॒ष्ठी॒वद्भ्याꣳ॑ सवि॒तारꣳ॑ सवि॒तार॑ मष्ठी॒वद्भ्या॑ मष्ठी॒वद्भ्याꣳ॑ सवि॒तार॒म् पुच्छे॑न॒ पुच्छे॑न सवि॒तार॑ मष्ठी॒वद्भ्या॑ मष्ठी॒वद्भ्याꣳ॑ सवि॒तार॒म् पुच्छे॑न । \newline
15. अ॒ष्ठी॒वद्भ्या॒मित्य॑ष्ठी॒वत् - भ्या॒म् । \newline
16. स॒वि॒तार॒म् पुच्छे॑न॒ पुच्छे॑न सवि॒तारꣳ॑ सवि॒तार॒म् पुच्छे॑न गन्ध॒र्वान् ग॑न्ध॒र्वान् पुच्छे॑न सवि॒तारꣳ॑ सवि॒तार॒म् पुच्छे॑न गन्ध॒र्वान् । \newline
17. पुच्छे॑न गन्ध॒र्वान् ग॑न्ध॒र्वान् पुच्छे॑न॒ पुच्छे॑न गन्ध॒र्वाञ् छेपे॑न॒ शेपे॑न गन्ध॒र्वान् पुच्छे॑न॒ पुच्छे॑न गन्ध॒र्वाञ् छेपे॑न । \newline
18. ग॒न्ध॒र्वाञ् छेपे॑न॒ शेपे॑न गन्ध॒र्वान् ग॑न्ध॒र्वाञ् छेपे॑ना फ्स॒रसो᳚ ऽफ्स॒रसः॒ शेपे॑न गन्ध॒र्वान् ग॑न्ध॒र्वाञ् छेपे॑ना फ्स॒रसः॑ । \newline
19. शेपे॑ना फ्स॒रसो᳚ ऽफ्स॒रसः॒ शेपे॑न॒ शेपे॑ना फ्स॒रसो॑ मु॒ष्काभ्या᳚म् मु॒ष्काभ्या॑ मफ्स॒रसः॒ शेपे॑न॒ शेपे॑ना फ्स॒रसो॑ मु॒ष्काभ्या᳚म् । \newline
20. अ॒फ्स॒रसो॑ मु॒ष्काभ्या᳚म् मु॒ष्काभ्या॑ मफ्स॒रसो᳚ ऽफ्स॒रसो॑ मु॒ष्काभ्या॒म् पव॑मान॒म् पव॑मानम् मु॒ष्काभ्या॑ मफ्स॒रसो᳚ ऽफ्स॒रसो॑ मु॒ष्काभ्या॒म् पव॑मानम् । \newline
21. मु॒ष्काभ्या॒म् पव॑मान॒म् पव॑मानम् मु॒ष्काभ्या᳚म् मु॒ष्काभ्या॒म् पव॑मानम् पा॒युना॑ पा॒युना॒ पव॑मानम् मु॒ष्काभ्या᳚म् मु॒ष्काभ्या॒म् पव॑मानम् पा॒युना᳚ । \newline
22. पव॑मानम् पा॒युना॑ पा॒युना॒ पव॑मान॒म् पव॑मानम् पा॒युना॑ प॒वित्र॑म् प॒वित्र॑म् पा॒युना॒ पव॑मान॒म् पव॑मानम् पा॒युना॑ प॒वित्र᳚म् । \newline
23. पा॒युना॑ प॒वित्र॑म् प॒वित्र॑म् पा॒युना॑ पा॒युना॑ प॒वित्र॒म् पोत्रा᳚भ्या॒म् पोत्रा᳚भ्याम् प॒वित्र॑म् पा॒युना॑ पा॒युना॑ प॒वित्र॒म् पोत्रा᳚भ्याम् । \newline
24. प॒वित्र॒म् पोत्रा᳚भ्या॒म् पोत्रा᳚भ्याम् प॒वित्र॑म् प॒वित्र॒म् पोत्रा᳚भ्या मा॒क्रम॑ण मा॒क्रम॑ण॒म् पोत्रा᳚भ्याम् प॒वित्र॑म् प॒वित्र॒म् पोत्रा᳚भ्या मा॒क्रम॑णम् । \newline
25. पोत्रा᳚भ्या मा॒क्रम॑ण मा॒क्रम॑ण॒म् पोत्रा᳚भ्या॒म् पोत्रा᳚भ्या मा॒क्रम॑णꣳ स्थू॒राभ्याꣳ॑ स्थू॒राभ्या॑ मा॒क्रम॑ण॒म् पोत्रा᳚भ्या॒म् पोत्रा᳚भ्या मा॒क्रम॑णꣳ स्थू॒राभ्या᳚म् । \newline
26. आ॒क्रम॑णꣳ स्थू॒राभ्याꣳ॑ स्थू॒राभ्या॑ मा॒क्रम॑ण मा॒क्रम॑णꣳ स्थू॒राभ्या᳚म् प्रति॒क्रम॑णम् प्रति॒क्रम॑णꣳ स्थू॒राभ्या॑ मा॒क्रम॑ण मा॒क्रम॑णꣳ स्थू॒राभ्या᳚म् प्रति॒क्रम॑णम् । \newline
27. आ॒क्रम॑ण॒मित्या᳚ - क्रम॑णम् । \newline
28. स्थू॒राभ्या᳚म् प्रति॒क्रम॑णम् प्रति॒क्रम॑णꣳ स्थू॒राभ्याꣳ॑ स्थू॒राभ्या᳚म् प्रति॒क्रम॑ण॒म् कुष्ठा᳚भ्या॒म् कुष्ठा᳚भ्याम् प्रति॒क्रम॑णꣳ स्थू॒राभ्याꣳ॑ स्थू॒राभ्या᳚म् प्रति॒क्रम॑ण॒म् कुष्ठा᳚भ्याम् । \newline
29. प्र॒ति॒क्रम॑ण॒म् कुष्ठा᳚भ्या॒म् कुष्ठा᳚भ्याम् प्रति॒क्रम॑णम् प्रति॒क्रम॑ण॒म् कुष्ठा᳚भ्याम् । \newline
30. प्र॒ति॒क्रम॑ण॒मिति॑ प्रति - क्रम॑णम् । \newline
31. कुष्ठा᳚भ्या॒मिति॒ कुष्ठा᳚भ्याम् । \newline
\pagebreak
\markright{ TS 5.7.16.1  \hfill https://www.vedavms.in \hfill}

\section{ TS 5.7.16.1 }

\textbf{TS 5.7.16.1 } \newline
\textbf{Samhita Paata} \newline

इन्द्र॑स्य क्रो॒डो ऽदि॑त्यै पाज॒स्यं॑ दि॒शां ज॒त्रवो॑ जी॒मूता᳚न् हृदयौप॒शाभ्या॑-म॒न्तरि॑क्षं पुरि॒तता॒ नभ॑ उद॒र्ये॑णेन्द्रा॒णीं प्ली॒ह्ना व॒ल्मीका᳚न् क्लो॒म्ना गि॒रीन् प्ला॒शिभिः॑ समु॒द्रमु॒दरे॑ण वैश्वान॒रं भस्म॑ना ॥ \newline

\textbf{Pada Paata} \newline

इन्द्र॑स्य । क्रो॒डः । अदि॑त्यै । पा॒ज॒स्य᳚म् । दि॒शाम् । ज॒त्रवः॑ । जी॒मूतान्॑ । हृ॒द॒यौ॒प॒शाभ्या॒मिति॑ हृदय - औ॒प॒शाभ्या᳚म् । अ॒न्तरि॑क्षम् । पु॒रि॒तता᳚ । नभः॑ । उ॒द॒र्ये॑ण । इ॒न्द्रा॒णीम् । प्ली॒ह्ना । व॒ल्मीकान्॑ । क्लो॒म्ना । गि॒रीन् । प्ला॒शिभि॒रिति॑ प्ला॒शि-भिः॒ । स॒मु॒द्रम् । उ॒दरे॑ण । वै॒श्वा॒न॒रम् । भस्म॑ना ॥  \newline


\textbf{Krama Paata} \newline

इन्द्र॑स्य क्रो॒डः । क्रो॒डोऽदि॑त्यै । अदि॑त्यै पाज॒स्य᳚म् । पा॒ज॒स्य॑म् दि॒शाम् । दि॒शाम् ज॒त्रवः॑ । ज॒त्रवो॑ जी॒मूतान्॑ । जी॒मूता᳚न् हृदयौप॒शाभ्या᳚म् । हृ॒द॒यौ॒प॒शाभ्या॑म॒न्तरि॑क्षम् । हृ॒द॒यौ॒प॒शाभ्या॒मिति॑ हृदय - औ॒प॒शाभ्या᳚म् । अ॒न्तरि॑क्षम् पुरि॒तता᳚ । पु॒रि॒तता॒ नभः॑ । नभ॑ उद॒र्ये॑ण । उ॒द॒र्ये॑णेन्द्रा॒णीम् । इ॒न्द्रा॒णीम् प्ली॒ह्ना । प्ली॒ह्ना व॒ल्मीकान्॑ । व॒ल्मीका᳚न् क्लो॒म्ना । क्लो॒म्ना गि॒रीन् । गि॒रीन् प्ला॒शिभिः॑ । प्ला॒शिभिः॑ समु॒द्रम् । प्ला॒शिभि॒रिति॑ प्ला॒शि - भिः॒ । स॒मु॒द्रमु॒दरे॑ण । उ॒दरे॑ण वैश्वान॒रम् । वै॒श्वा॒न॒रम् भस्म॑ना । भस्म॒नेति॒ भस्म॑ना । \newline

\textbf{Jatai Paata} \newline

1. इन्द्र॑स्य क्रो॒डः क्रो॒ड इन्द्र॒ स्येन्द्र॑स्य क्रो॒डः । \newline
2. क्रो॒डो ऽदि॑त्या॒ अदि॑त्यै क्रो॒डः क्रो॒डो ऽदि॑त्यै । \newline
3. अदि॑त्यै पाज॒स्य॑म् पाज॒स्य॑ मदि॑त्या॒ अदि॑त्यै पाज॒स्य᳚म् । \newline
4. पा॒ज॒स्य॑म् दि॒शाम् दि॒शाम् पा॑ज॒स्य॑म् पाज॒स्य॑म् दि॒शाम् । \newline
5. दि॒शाम् ज॒त्रवो॑ ज॒त्रवो॑ दि॒शाम् दि॒शाम् ज॒त्रवः॑ । \newline
6. ज॒त्रवो॑ जी॒मूता᳚न् जी॒मूता᳚न् ज॒त्रवो॑ ज॒त्रवो॑ जी॒मूतान्॑ । \newline
7. जी॒मूता᳚न् हृदयौप॒शाभ्याꣳ॑ हृदयौप॒शाभ्या᳚म् जी॒मूता᳚न् जी॒मूता᳚न् हृदयौप॒शाभ्या᳚म् । \newline
8. हृ॒द॒यौ॒प॒शाभ्या॑ म॒न्तरि॑क्ष म॒न्तरि॑क्षꣳ हृदयौप॒शाभ्याꣳ॑ हृदयौप॒शाभ्या॑ म॒न्तरि॑क्षम् । \newline
9. हृ॒द॒यौ॒प॒शाभ्या॒मिति॑ हृदय - औ॒प॒शाभ्या᳚म् । \newline
10. अ॒न्तरि॑क्षम् पुरि॒तता॑ पुरि॒तता॒ ऽन्तरि॑क्ष म॒न्तरि॑क्षम् पुरि॒तता᳚ । \newline
11. पु॒रि॒तता॒ नभो॒ नभः॑ पुरि॒तता॑ पुरि॒तता॒ नभः॑ । \newline
12. नभ॑ उद॒र्ये॑ णोद॒र्ये॑ण॒ नभो॒ नभ॑ उद॒र्ये॑ण । \newline
13. उ॒द॒र्ये॑ णेन्द्रा॒णी मि॑न्द्रा॒णी मु॑द॒र्ये॑ णोद॒र्ये॑ णेन्द्रा॒णीम् । \newline
14. इ॒न्द्रा॒णीम् प्ली॒ह्ना प्ली॒ह्नेन्द्रा॒णी मि॑न्द्रा॒णीम् प्ली॒ह्ना । \newline
15. प्ली॒ह्ना व॒ल्मीकान्॑. व॒ल्मीका᳚न् प्ली॒ह्ना प्ली॒ह्ना व॒ल्मीकान्॑ । \newline
16. व॒ल्मीका᳚न् क्लो॒म्ना क्लो॒म्ना व॒ल्मीकान्॑. व॒ल्मीका᳚न् क्लो॒म्ना । \newline
17. क्लो॒म्ना गि॒रीन् गि॒रीन् क्लो॒म्ना क्लो॒म्ना गि॒रीन् । \newline
18. गि॒रीन् प्ला॒शिभिः॑ प्ला॒शिभि॑र् गि॒रीन् गि॒रीन् प्ला॒शिभिः॑ । \newline
19. प्ला॒शिभिः॑ समु॒द्रꣳ स॑मु॒द्रम् प्ला॒शिभिः॑ प्ला॒शिभिः॑ समु॒द्रम् । \newline
20. प्ला॒शिभि॒रिति॑ प्ला॒शि - भिः॒ । \newline
21. स॒मु॒द्र मु॒दरे॑ णो॒दरे॑ण समु॒द्रꣳ स॑मु॒द्र मु॒दरे॑ण । \newline
22. उ॒दरे॑ण वैश्वान॒रं ॅवै᳚श्वान॒र मु॒दरे॑ णो॒दरे॑ण वैश्वान॒रम् । \newline
23. वै॒श्वा॒न॒रम् भस्म॑ना॒ भस्म॑ना वैश्वान॒रं ॅवै᳚श्वान॒रम् भस्म॑ना । \newline
24. भस्म॒नेति॒ भस्म॑ना । \newline

\textbf{Ghana Paata } \newline

1. इन्द्र॑स्य क्रो॒डः क्रो॒ड इन्द्र॒स्येन्द्र॑स्य क्रो॒डो ऽदि॑त्या॒ अदि॑त्यै क्रो॒ड इन्द्र॒स्येन्द्र॑स्य क्रो॒डो ऽदि॑त्यै । \newline
2. क्रो॒डो ऽदि॑त्या॒ अदि॑त्यै क्रो॒डः क्रो॒डो ऽदि॑त्यै पाज॒स्य॑म् पाज॒स्य॑ मदि॑त्यै क्रो॒डः क्रो॒डो ऽदि॑त्यै पाज॒स्य᳚म् । \newline
3. अदि॑त्यै पाज॒स्य॑म् पाज॒स्य॑ मदि॑त्या॒ अदि॑त्यै पाज॒स्य॑म् दि॒शाम् दि॒शाम् पा॑ज॒स्य॑ मदि॑त्या॒ अदि॑त्यै पाज॒स्य॑म् दि॒शाम् । \newline
4. पा॒ज॒स्य॑म् दि॒शाम् दि॒शाम् पा॑ज॒स्य॑म् पाज॒स्य॑म् दि॒शाम् ज॒त्रवो॑ ज॒त्रवो॑ दि॒शाम् पा॑ज॒स्य॑म् पाज॒स्य॑म् दि॒शाम् ज॒त्रवः॑ । \newline
5. दि॒शाम् ज॒त्रवो॑ ज॒त्रवो॑ दि॒शाम् दि॒शाम् ज॒त्रवो॑ जी॒मूता᳚न् जी॒मूता᳚न् ज॒त्रवो॑ दि॒शाम् दि॒शाम् ज॒त्रवो॑ जी॒मूतान्॑ । \newline
6. ज॒त्रवो॑ जी॒मूता᳚न् जी॒मूता᳚न् ज॒त्रवो॑ ज॒त्रवो॑ जी॒मूता᳚न् हृदयौप॒शाभ्याꣳ॑ हृदयौप॒शाभ्या᳚म् जी॒मूता᳚न् ज॒त्रवो॑ ज॒त्रवो॑ जी॒मूता᳚न् हृदयौप॒शाभ्या᳚म् । \newline
7. जी॒मूता᳚न् हृदयौप॒शाभ्याꣳ॑ हृदयौप॒शाभ्या᳚म् जी॒मूता᳚न् जी॒मूता᳚न् हृदयौप॒शाभ्या॑ म॒न्तरि॑क्ष म॒न्तरि॑क्षꣳ हृदयौप॒शाभ्या᳚म् जी॒मूता᳚न् जी॒मूता᳚न् हृदयौप॒शाभ्या॑ म॒न्तरि॑क्षम् । \newline
8. हृ॒द॒यौ॒प॒शाभ्या॑ म॒न्तरि॑क्ष म॒न्तरि॑क्षꣳ हृदयौप॒शाभ्याꣳ॑ हृदयौप॒शाभ्या॑ म॒न्तरि॑क्षम् पुरि॒तता॑ पुरि॒तता॒ ऽन्तरि॑क्षꣳ हृदयौप॒शाभ्याꣳ॑ हृदयौप॒शाभ्या॑ म॒न्तरि॑क्षम् पुरि॒तता᳚ । \newline
9. हृ॒द॒यौ॒प॒शाभ्या॒मिति॑ हृदय - औ॒प॒शाभ्या᳚म् । \newline
10. अ॒न्तरि॑क्षम् पुरि॒तता॑ पुरि॒तता॒ ऽन्तरि॑क्ष म॒न्तरि॑क्षम् पुरि॒तता॒ नभो॒ नभः॑ पुरि॒तता॒ ऽन्तरि॑क्ष म॒न्तरि॑क्षम् पुरि॒तता॒ नभः॑ । \newline
11. पु॒रि॒तता॒ नभो॒ नभः॑ पुरि॒तता॑ पुरि॒तता॒ नभ॑ उद॒र्ये॑ णोद॒र्ये॑ण॒ नभः॑ पुरि॒तता॑ पुरि॒तता॒ नभ॑ उद॒र्ये॑ण । \newline
12. नभ॑ उद॒र्ये॑ णोद॒र्ये॑ण॒ नभो॒ नभ॑ उद॒र्ये॑ णेन्द्रा॒णी मि॑न्द्रा॒णी मु॑द॒र्ये॑ण॒ नभो॒ नभ॑ उद॒र्ये॑
णेन्द्रा॒णीम् । \newline
13. उ॒द॒र्ये॑ णेन्द्रा॒णी मि॑न्द्रा॒णी मु॑द॒र्ये॑ णोद॒र्ये॑ णेन्द्रा॒णीम् प्ली॒ह्ना प्ली॒ह्नेन्द्रा॒णी मु॑द॒र्ये॑ णोद॒र्ये॑
णेन्द्रा॒णीम् प्ली॒ह्ना । \newline
14. इ॒न्द्रा॒णीम् प्ली॒ह्ना प्ली॒ह्नेन्द्रा॒णी मि॑न्द्रा॒णीम् प्ली॒ह्ना व॒ल्मीकान्॑. व॒ल्मीका᳚न् प्ली॒ह्नेन्द्रा॒णी मि॑न्द्रा॒णीम् प्ली॒ह्ना व॒ल्मीकान्॑ । \newline
15. प्ली॒ह्ना व॒ल्मीकान्॑. व॒ल्मीका᳚न् प्ली॒ह्ना प्ली॒ह्ना व॒ल्मीका᳚न् क्लो॒म्ना क्लो॒म्ना व॒ल्मीका᳚न् प्ली॒ह्ना प्ली॒ह्ना व॒ल्मीका᳚न् क्लो॒म्ना । \newline
16. व॒ल्मीका᳚न् क्लो॒म्ना क्लो॒म्ना व॒ल्मीकान्॑. व॒ल्मीका᳚न् क्लो॒म्ना गि॒रीन् गि॒रीन् क्लो॒म्ना व॒ल्मीकान्॑. व॒ल्मीका᳚न् क्लो॒म्ना गि॒रीन् । \newline
17. क्लो॒म्ना गि॒रीन् गि॒रीन् क्लो॒म्ना क्लो॒म्ना गि॒रीन् प्ला॒शिभिः॑ प्ला॒शिभि॑र् गि॒रीन् क्लो॒म्ना क्लो॒म्ना गि॒रीन् प्ला॒शिभिः॑ । \newline
18. गि॒रीन् प्ला॒शिभिः॑ प्ला॒शिभि॑र् गि॒रीन् गि॒रीन् प्ला॒शिभिः॑ समु॒द्रꣳ स॑मु॒द्रम् प्ला॒शिभि॑र् गि॒रीन् गि॒रीन् प्ला॒शिभिः॑ समु॒द्रम् । \newline
19. प्ला॒शिभिः॑ समु॒द्रꣳ स॑मु॒द्रम् प्ला॒शिभिः॑ प्ला॒शिभिः॑ समु॒द्र मु॒दरे॑ णो॒दरे॑ण समु॒द्रम् प्ला॒शिभिः॑ प्ला॒शिभिः॑ समु॒द्र मु॒दरे॑ण । \newline
20. प्ला॒शिभि॒रिति॑ प्ला॒शि - भिः॒ । \newline
21. स॒मु॒द्र मु॒दरे॑ णो॒दरे॑ण समु॒द्रꣳ स॑मु॒द्र मु॒दरे॑ण वैश्वान॒रं ॅवै᳚श्वान॒र मु॒दरे॑ण समु॒द्रꣳ स॑मु॒द्र मु॒दरे॑ण वैश्वान॒रम् । \newline
22. उ॒दरे॑ण वैश्वान॒रं ॅवै᳚श्वान॒र मु॒दरे॑ णो॒दरे॑ण वैश्वान॒रम् भस्म॑ना॒ भस्म॑ना वैश्वान॒र मु॒दरे॑ णो॒दरे॑ण वैश्वान॒रम् भस्म॑ना । \newline
23. वै॒श्वा॒न॒रम् भस्म॑ना॒ भस्म॑ना वैश्वान॒रं ॅवै᳚श्वान॒रम् भस्म॑ना । \newline
24. भस्म॒नेति॒ भस्म॑ना । \newline
\pagebreak
\markright{ TS 5.7.17.1  \hfill https://www.vedavms.in \hfill}

\section{ TS 5.7.17.1 }

\textbf{TS 5.7.17.1 } \newline
\textbf{Samhita Paata} \newline

पू॒ष्णो व॑नि॒ष्ठुर॑न्धा॒हेः स्थू॑रगु॒दा स॒र्पान् गुदा॑भिर्. ऋ॒तून् पृ॒ष्टीभि॒र्दिवं॑ पृ॒ष्ठेन॒ वसू॑नां प्रथ॒मा कीक॑सा रु॒द्राणां᳚ द्वि॒तीया॑ ऽऽदि॒त्यानां᳚ तृ॒तीया ऽङ्गि॑रसां चतु॒र्थी सा॒द्ध्यानां᳚ पञ्च॒मी विश्वे॑षां दे॒वानाꣳ॑ ष॒ष्ठी ॥ \newline

\textbf{Pada Paata} \newline

पू॒ष्णः । व॒नि॒ष्ठुः । अ॒न्धा॒हेरित्य॑न्ध-अ॒हेः । स्थू॒र॒गु॒देति॑ स्थूर - गु॒दा । स॒र्पान् । गुदा॑भिः । ऋ॒तून् । पृ॒ष्टीभि॒रिति॑ पृ॒ष्टि - भिः॒ । दिव᳚म् । पृ॒ष्ठेन॑ । वसू॑नाम् । प्र॒थ॒मा । कीक॑सा । रु॒द्राणा᳚म् । द्वि॒तीया᳚ । आ॒दि॒त्याना᳚म् । तृ॒तीया᳚ । अङ्गि॑रसाम् । च॒तु॒र्थी । सा॒द्ध्याना᳚म् । प॒ञ्च॒मी । विश्वे॑षाम् । दे॒वाना᳚म् । ष॒ष्ठी ॥  \newline


\textbf{Krama Paata} \newline

पू॒ष्णो व॑नि॒ष्ठुः । व॒नि॒ष्ठुर॑न्धा॒हेः । अ॒न्धा॒हेः स्थू॑रगु॒दा । अ॒न्धा॒हेरित्य॑न्ध - अ॒हेः । स्थू॒र॒गु॒दा स॒र्पान् । स्थू॒र॒गु॒देति॑ स्थूर - गु॒दा । स॒र्पान् गुदा॑भिः । गुदा॑भिर्. ऋ॒तून् । ऋ॒तून् पृ॒ष्टीभिः॑ । पृ॒ष्ठीभि॒र् दिव᳚म् । पृ॒ष्टीभि॒रिति॑ पृ॒ष्टि - भिः॒ । दिव॑म् पृ॒ष्ठेन॑ । पृ॒ष्ठेन॒ वसू॑नाम् । वसू॑नाम् प्रथ॒मा । प्र॒थ॒मा कीक॑सा । कीक॑सा रु॒द्राणा᳚म् । रु॒द्राणा᳚म् द्वि॒तीया᳚ । द्वि॒तीया॑ऽऽदि॒त्याना᳚म् । आ॒दि॒त्याना᳚म् तृ॒तीया᳚ । तृ॒तीयाऽङ्गि॑रसाम् । अङ्गि॑रसाम् चतु॒र्त्थी । च॒तु॒र्त्थी सा॒द्ध्याना᳚म् । सा॒द्ध्याना᳚म् पञ्च॒मी । प॒ञ्च॒मी विश्वे॑षाम् । विश्वे॑षाम् दे॒वाना᳚म् । दे॒वानाꣳ॑ ष॒ष्ठी । ष॒ष्ठीति॑ ष॒ष्ठी । \newline

\textbf{Jatai Paata} \newline

1. पू॒ष्णो व॑नि॒ष्ठुर् व॑नि॒ष्ठुः पू॒ष्णः पू॒ष्णो व॑नि॒ष्ठुः । \newline
2. व॒नि॒ष्ठु र॑न्धा॒हे र॑न्धा॒हेर् व॑नि॒ष्ठुर् व॑नि॒ष्ठु र॑न्धा॒हेः । \newline
3. अ॒न्धा॒हेः स्थू॑रगु॒दा स्थू॑रगु॒दा ऽन्धा॒हे र॑न्धा॒हेः स्थू॑रगु॒दा । \newline
4. अ॒न्धा॒हेरित्य॑न्ध - अ॒हेः । \newline
5. स्थू॒र॒गु॒दा स॒र्पान् थ्स॒र्पान् थ्स्थू॑रगु॒दा स्थू॑रगु॒दा स॒र्पान् । \newline
6. स्थू॒र॒गु॒देति॑ स्थूर - गु॒दा । \newline
7. स॒र्पान् गुदा॑भि॒र् गुदा॑भिः स॒र्पान् थ्स॒र्पान् गुदा॑भिः । \newline
8. गुदा॑भिर्. ऋ॒तू नृ॒तून् गुदा॑भि॒र् गुदा॑भिर्. ऋ॒तून् । \newline
9. ऋ॒तून् पृ॒ष्टीभिः॑ पृ॒ष्टीभिर्॑. ऋ॒तू नृ॒तून् पृ॒ष्टीभिः॑ । \newline
10. पृ॒ष्टीभि॒र् दिव॒म् दिव॑म् पृ॒ष्टीभिः॑ पृ॒ष्टीभि॒र् दिव᳚म् । \newline
11. पृ॒ष्टीभि॒रिति॑ पृ॒ष्टि - भिः॒ । \newline
12. दिव॑म् पृ॒ष्ठेन॑ पृ॒ष्ठेन॒ दिव॒म् दिव॑म् पृ॒ष्ठेन॑ । \newline
13. पृ॒ष्ठेन॒ वसू॑नां॒ ॅवसू॑नाम् पृ॒ष्ठेन॑ पृ॒ष्ठेन॒ वसू॑नाम् । \newline
14. वसू॑नाम् प्रथ॒मा प्र॑थ॒मा वसू॑नां॒ ॅवसू॑नाम् प्रथ॒मा । \newline
15. प्र॒थ॒मा कीक॑सा॒ कीक॑सा प्रथ॒मा प्र॑थ॒मा कीक॑सा । \newline
16. कीक॑सा रु॒द्राणाꣳ॑ रु॒द्राणा॒म् कीक॑सा॒ कीक॑सा रु॒द्राणा᳚म् । \newline
17. रु॒द्राणा᳚म् द्वि॒तीया᳚ द्वि॒तीया॑ रु॒द्राणाꣳ॑ रु॒द्राणा᳚म् द्वि॒तीया᳚ । \newline
18. द्वि॒तीया॑ ऽऽदि॒त्याना॑ मादि॒त्याना᳚म् द्वि॒तीया᳚ द्वि॒तीया॑ ऽऽदि॒त्याना᳚म् । \newline
19. आ॒दि॒त्याना᳚म् तृ॒तीया॑ तृ॒तीया॑ ऽऽदि॒त्याना॑ मादि॒त्याना᳚म् तृ॒तीया᳚ । \newline
20. तृ॒तीया ऽङ्गि॑रसा॒ मङ्गि॑रसाम् तृ॒तीया॑ तृ॒तीया ऽङ्गि॑रसाम् । \newline
21. अङ्गि॑रसाम् चतु॒र्थी च॑तु॒र् थ्यङ्गि॑रसा॒ मङ्गि॑रसाम् चतु॒र्थी । \newline
22. च॒तु॒र्थी सा॒द्ध्यानाꣳ॑ सा॒द्ध्याना᳚म् चतु॒र्थी च॑तु॒र्थी सा॒द्ध्याना᳚म् । \newline
23. सा॒द्ध्याना᳚म् पञ्च॒मी प॑ञ्च॒मी सा॒द्ध्यानाꣳ॑ सा॒द्ध्याना᳚म् पञ्च॒मी । \newline
24. प॒ञ्च॒मी विश्वे॑षां॒ ॅविश्वे॑षाम् पञ्च॒मी प॑ञ्च॒मी विश्वे॑षाम् । \newline
25. विश्वे॑षाम् दे॒वाना᳚म् दे॒वानां॒ ॅविश्वे॑षां॒ ॅविश्वे॑षाम् दे॒वाना᳚म् । \newline
26. दे॒वानाꣳ॑ ष॒ष्ठी ष॒ष्ठी दे॒वाना᳚म् दे॒वानाꣳ॑ ष॒ष्ठी । \newline
27. ष॒ष्ठीति॑ ष॒ष्ठी । \newline

\textbf{Ghana Paata } \newline

1. पू॒ष्णो व॑नि॒ष्ठुर् व॑नि॒ष्ठुः पू॒ष्णः पू॒ष्णो व॑नि॒ष्ठु र॑न्धा॒हे र॑न्धा॒हेर् व॑नि॒ष्ठुः पू॒ष्णः पू॒ष्णो व॑नि॒ष्ठु र॑न्धा॒हेः । \newline
2. व॒नि॒ष्ठु र॑न्धा॒हे र॑न्धा॒हेर् व॑नि॒ष्ठुर् व॑नि॒ष्ठु र॑न्धा॒हेः स्थू॑रगु॒दा स्थू॑रगु॒दा ऽन्धा॒हेर् व॑नि॒ष्ठुर् व॑नि॒ष्ठु र॑न्धा॒हेः स्थू॑रगु॒दा । \newline
3. अ॒न्धा॒हेः स्थू॑रगु॒दा स्थू॑रगु॒दा ऽन्धा॒हे र॑न्धा॒हेः स्थू॑रगु॒दा स॒र्पान् थ्स॒र्पान् थ्स्थू॑रगु॒दा ऽन्धा॒हे र॑न्धा॒हेः स्थू॑रगु॒दा स॒र्पान् । \newline
4. अ॒न्धा॒हेरित्य॑न्ध - अ॒हेः । \newline
5. स्थू॒र॒गु॒दा स॒र्पान् थ्स॒र्पान् थ्स्थू॑रगु॒दा स्थू॑रगु॒दा स॒र्पान् गुदा॑भि॒र् गुदा॑भिः स॒र्पान् थ्स्थू॑रगु॒दा स्थू॑रगु॒दा स॒र्पान् गुदा॑भिः । \newline
6. स्थू॒र॒गु॒देति॑ स्थूर - गु॒दा । \newline
7. स॒र्पान् गुदा॑भि॒र् गुदा॑भिः स॒र्पान् थ्स॒र्पान् गुदा॑भिर्. ऋ॒तू नृ॒तून् गुदा॑भिः स॒र्पान् थ्स॒र्पान् गुदा॑भिर्. ऋ॒तून् । \newline
8. गुदा॑भिर्. ऋ॒तू नृ॒तून् गुदा॑भि॒र् गुदा॑भिर्. ऋ॒तून् पृ॒ष्टीभिः॑ पृ॒ष्टीभिर्॑. ऋ॒तून् गुदा॑भि॒र् गुदा॑भिर्. ऋ॒तून् पृ॒ष्टीभिः॑ । \newline
9. ऋ॒तून् पृ॒ष्टीभिः॑ पृ॒ष्टीभिर्॑. ऋ॒तू नृ॒तून् पृ॒ष्टीभि॒र् दिव॒म् दिव॑म् पृ॒ष्टीभिर्॑. ऋ॒तू नृ॒तून् पृ॒ष्टीभि॒र् दिव᳚म् । \newline
10. पृ॒ष्टीभि॒र् दिव॒म् दिव॑म् पृ॒ष्टीभिः॑ पृ॒ष्टीभि॒र् दिव॑म् पृ॒ष्ठेन॑ पृ॒ष्ठेन॒ दिव॑म् पृ॒ष्टीभिः॑ पृ॒ष्टीभि॒र् दिव॑म् पृ॒ष्ठेन॑ । \newline
11. पृ॒ष्टीभि॒रिति॑ पृ॒ष्टि - भिः॒ । \newline
12. दिव॑म् पृ॒ष्ठेन॑ पृ॒ष्ठेन॒ दिव॒म् दिव॑म् पृ॒ष्ठेन॒ वसू॑नां॒ ॅवसू॑नाम् पृ॒ष्ठेन॒ दिव॒म् दिव॑म् पृ॒ष्ठेन॒ वसू॑नाम् । \newline
13. पृ॒ष्ठेन॒ वसू॑नां॒ ॅवसू॑नाम् पृ॒ष्ठेन॑ पृ॒ष्ठेन॒ वसू॑नाम् प्रथ॒मा प्र॑थ॒मा वसू॑नाम् पृ॒ष्ठेन॑ पृ॒ष्ठेन॒ वसू॑नाम् प्रथ॒मा । \newline
14. वसू॑नाम् प्रथ॒मा प्र॑थ॒मा वसू॑नां॒ ॅवसू॑नाम् प्रथ॒मा कीक॑सा॒ कीक॑सा प्रथ॒मा वसू॑नां॒ ॅवसू॑नाम् प्रथ॒मा कीक॑सा । \newline
15. प्र॒थ॒मा कीक॑सा॒ कीक॑सा प्रथ॒मा प्र॑थ॒मा कीक॑सा रु॒द्राणाꣳ॑ रु॒द्राणा॒म् कीक॑सा प्रथ॒मा प्र॑थ॒मा कीक॑सा रु॒द्राणा᳚म् । \newline
16. कीक॑सा रु॒द्राणाꣳ॑ रु॒द्राणा॒म् कीक॑सा॒ कीक॑सा रु॒द्राणा᳚म् द्वि॒तीया᳚ द्वि॒तीया॑ रु॒द्राणा॒म् कीक॑सा॒ कीक॑सा रु॒द्राणा᳚म् द्वि॒तीया᳚ । \newline
17. रु॒द्राणा᳚म् द्वि॒तीया᳚ द्वि॒तीया॑ रु॒द्राणाꣳ॑ रु॒द्राणा᳚म् द्वि॒तीया॑ ऽऽदि॒त्याना॑ मादि॒त्याना᳚म् द्वि॒तीया॑ रु॒द्राणाꣳ॑ रु॒द्राणा᳚म् द्वि॒तीया॑ ऽऽदि॒त्याना᳚म् । \newline
18. द्वि॒तीया॑ ऽऽदि॒त्याना॑ मादि॒त्याना᳚म् द्वि॒तीया᳚ द्वि॒तीया॑ ऽऽदि॒त्याना᳚म् तृ॒तीया॑ तृ॒तीया॑ ऽऽदि॒त्याना᳚म् द्वि॒तीया᳚ द्वि॒तीया॑ ऽऽदि॒त्याना᳚म् तृ॒तीया᳚ । \newline
19. आ॒दि॒त्याना᳚म् तृ॒तीया॑ तृ॒तीया॑ ऽऽदि॒त्याना॑ मादि॒त्याना᳚म् तृ॒तीया ऽङ्गि॑रसा॒ मङ्गि॑रसाम् तृ॒तीया॑ ऽऽदि॒त्याना॑ मादि॒त्याना᳚म् तृ॒तीया ऽङ्गि॑रसाम् । \newline
20. तृ॒तीया ऽङ्गि॑रसा॒ मङ्गि॑रसाम् तृ॒तीया॑ तृ॒तीया ऽङ्गि॑रसाम् चतु॒र्थी च॑तु॒र् थ्यङ्गि॑रसाम्  तृ॒तीया॑ तृ॒तीया ऽङ्गि॑रसाम् चतु॒र्थी । \newline
21. अङ्गि॑रसाम् चतु॒र्थी च॑तु॒र् थ्यङ्गि॑रसा॒ मङ्गि॑रसाम् चतु॒र्थी सा॒द्ध्यानाꣳ॑ सा॒द्ध्याना᳚म् चतु॒र्थ्यङ्गि॑रसा॒ मङ्गि॑रसाम् चतु॒र्थी सा॒द्ध्याना᳚म् । \newline
22. च॒तु॒र्थी सा॒द्ध्यानाꣳ॑ सा॒द्ध्याना᳚म् चतु॒र्थी च॑तु॒र्थी सा॒द्ध्याना᳚म् पञ्च॒मी प॑ञ्च॒मी सा॒द्ध्याना᳚म् चतु॒र्थी च॑तु॒र्थी सा॒द्ध्याना᳚म् पञ्च॒मी । \newline
23. सा॒द्ध्याना᳚म् पञ्च॒मी प॑ञ्च॒मी सा॒द्ध्यानाꣳ॑ सा॒द्ध्याना᳚म् पञ्च॒मी विश्वे॑षां॒ ॅविश्वे॑षाम् पञ्च॒मी सा॒द्ध्यानाꣳ॑ सा॒द्ध्याना᳚म् पञ्च॒मी विश्वे॑षाम् । \newline
24. प॒ञ्च॒मी विश्वे॑षां॒ ॅविश्वे॑षाम् पञ्च॒मी प॑ञ्च॒मी विश्वे॑षाम् दे॒वाना᳚म् दे॒वानां॒ ॅविश्वे॑षाम् पञ्च॒मी प॑ञ्च॒मी विश्वे॑षाम् दे॒वाना᳚म् । \newline
25. विश्वे॑षाम् दे॒वाना᳚म् दे॒वानां॒ ॅविश्वे॑षां॒ ॅविश्वे॑षाम् दे॒वानाꣳ॑ ष॒ष्ठी ष॒ष्ठी दे॒वानां॒ ॅविश्वे॑षां॒ ॅविश्वे॑षाम् दे॒वानाꣳ॑ ष॒ष्ठी । \newline
26. दे॒वानाꣳ॑ ष॒ष्ठी ष॒ष्ठी दे॒वाना᳚म् दे॒वानाꣳ॑ ष॒ष्ठी । \newline
27. ष॒ष्ठीति॑ ष॒ष्ठी । \newline
\pagebreak
\markright{ TS 5.7.18.1  \hfill https://www.vedavms.in \hfill}

\section{ TS 5.7.18.1 }

\textbf{TS 5.7.18.1 } \newline
\textbf{Samhita Paata} \newline

ओजो᳚ ग्री॒वाभि॒र्-निर्.ऋ॑तिम॒स्थभि॒रिन्द्रꣳ॒॒ स्वप॑सा॒ वहे॑न रु॒द्रस्य॑ विच॒लः स्क॒न्धो॑ ऽहोरा॒त्रयो᳚र्द्वि॒तीयो᳚ ऽर्द्धमा॒सानां᳚ तृ॒तीयो॑ मा॒सां च॑तु॒र्थ ऋ॑तू॒नां प॑ञ्च॒मः सं॑ॅवथ्स॒रस्य॑ ष॒ष्ठः ॥ \newline

\textbf{Pada Paata} \newline

ओजः॑ । ग्री॒वाभिः॑ । निर्.ऋ॑ति॒मिति॒ निः-ऋ॒ति॒म् । अ॒स्थभि॒रित्य॒स्थ-भिः॒ । इन्द्र᳚म् । स्वप॒सेति॑ सु - अप॑सा । वहे॑न । रु॒द्रस्य॑ । वि॒च॒ल इति॑ वि - च॒लः । स्क॒न्धः । अ॒हो॒रा॒त्रयो॒रित्य॑हः - रा॒त्रयोः᳚ । द्वि॒तीयः॑ । अ॒द्‌र्ध॒मा॒साना॒मित्य॑द्‌र्ध - मा॒साना᳚म् । तृ॒तीयः॑ । मा॒साम् । च॒तु॒र्थः । ऋ॒तू॒नाम् । प॒ञ्च॒मः । सं॒ॅव॒थ्स॒रस्येति॑ सं - व॒थ्स॒रस्य॑ । ष॒ष्ठः ॥  \newline


\textbf{Krama Paata} \newline

ओजो᳚ ग्री॒वाभिः॑ । ग्री॒वाभि॒र् निर्.ऋ॑तिम् । निर्.ऋ॑तिम॒स्थभिः॑ । निर्.ऋ॑ति॒मिति॒ निः - ऋ॒ति॒म् । अ॒स्थभि॒रिन्द्र᳚म् । अ॒स्थभि॒रित्य॒स्थ - भिः॒ । इन्द्रꣳ॒॒ स्वप॑सा । स्वप॑सा॒ वहे॑न । स्वप॒सेति॑ सु - अप॑सा । वहे॑न रु॒द्रस्य॑ । रु॒द्रस्य॑ विच॒लः । वि॒च॒लः स्क॒न्धः । वि॒च॒ल इति॑ वि - च॒लः । स्क॒न्धो॑ऽहोरा॒त्रयोः᳚ । अ॒हो॒रा॒त्रयो᳚र् द्वि॒तीयः॑ । अ॒हो॒रा॒त्रयो॒रित्य॑हः - रा॒त्रयोः᳚ । द्वि॒तीयो᳚ऽर्द्धमा॒साना᳚म् । अ॒र्द्ध॒मा॒साना᳚म् तृ॒तीयः॑ । अ॒र्द्ध॒मा॒साना॒मित्य॑र्द्ध - मा॒साना᳚म् । तृ॒तीयो॑ मा॒साम् । मा॒साम् च॑तु॒र्त्थः । च॒तु॒र्त्थ ऋ॑तू॒नाम् । ऋ॒तू॒नाम् प॑ञ्च॒मः । प॒ञ्च॒मः स॑म्ॅवथ्स॒रस्य॑ । स॒म्ॅव॒थ्स॒रस्य॑ ष॒ष्ठः । स॒म्व॒थ्स॒रस्येति॑ सम् - व॒थ्स॒रस्य॑ । ष॒ष्ठ इति॑ ष॒ष्ठः । \newline

\textbf{Jatai Paata} \newline

1. ओजो᳚ ग्री॒वाभि॑र् ग्री॒वाभि॒ रोज॒ ओजो᳚ ग्री॒वाभिः॑ । \newline
2. ग्री॒वाभि॒र् निर्.ऋ॑ति॒म् निर्.ऋ॑तिम् ग्री॒वाभि॑र् ग्री॒वाभि॒र् निर्.ऋ॑तिम् । \newline
3. निर्.ऋ॑ति म॒स्थभि॑ र॒स्थभि॒र् निर्.ऋ॑ति॒म् निर्.ऋ॑ति म॒स्थभिः॑ । \newline
4. निर्.ऋ॑ति॒मिति॒ निः - ऋ॒ति॒म् । \newline
5. अ॒स्थभि॒ रिन्द्र॒ मिन्द्र॑ म॒स्थभि॑ र॒स्थभि॒ रिन्द्र᳚म् । \newline
6. अ॒स्थभि॒रित्य॒स्थ - भिः॒ । \newline
7. इन्द्रꣳ॒॒ स्वप॑सा॒ स्वप॒ सेन्द्र॒ मिन्द्रꣳ॒॒ स्वप॑सा । \newline
8. स्वप॑सा॒ वहे॑न॒ वहे॑न॒ स्वप॑सा॒ स्वप॑सा॒ वहे॑न । \newline
9. स्वप॒सेति॑ सु - अप॑सा । \newline
10. वहे॑न रु॒द्रस्य॑ रु॒द्रस्य॒ वहे॑न॒ वहे॑न रु॒द्रस्य॑ । \newline
11. रु॒द्रस्य॑ विच॒लो वि॑च॒लो रु॒द्रस्य॑ रु॒द्रस्य॑ विच॒लः । \newline
12. वि॒च॒लः स्क॒न्धः स्क॒न्धो वि॑च॒लो वि॑च॒लः स्क॒न्धः । \newline
13. वि॒च॒ल इति॑ वि - च॒लः । \newline
14. स्क॒न्धो॑ ऽहोरा॒त्रयो॑ रहोरा॒त्रयोः᳚ स्क॒न्धः स्क॒न्धो॑ ऽहोरा॒त्रयोः᳚ । \newline
15. अ॒हो॒रा॒त्रयो᳚र् द्वि॒तीयो᳚ द्वि॒तीयो॑ ऽहोरा॒त्रयो॑ रहोरा॒त्रयो᳚र् द्वि॒तीयः॑ । \newline
16. अ॒हो॒रा॒त्रयो॒रित्य॑हः - रा॒त्रयोः᳚ । \newline
17. द्वि॒तीयो᳚ ऽर्द्धमा॒साना॑ मर्द्धमा॒साना᳚म् द्वि॒तीयो᳚ द्वि॒तीयो᳚ ऽर्द्धमा॒साना᳚म् । \newline
18. अ॒र्द्ध॒मा॒साना᳚म् तृ॒तीय॑ स्तृ॒तीयो᳚ ऽर्द्धमा॒साना॑ मर्द्धमा॒साना᳚म् तृ॒तीयः॑ । \newline
19. अ॒र्द्ध॒मा॒साना॒मित्य॑र्द्ध - मा॒साना᳚म् । \newline
20. तृ॒तीयो॑ मा॒साम् मा॒साम् तृ॒तीय॑ स्तृ॒तीयो॑ मा॒साम् । \newline
21. मा॒साम् च॑तु॒र्थ श्च॑तु॒र्थो मा॒साम् मा॒साम् च॑तु॒र्थः । \newline
22. च॒तु॒र्थ ऋ॑तू॒ना मृ॑तू॒नाम् च॑तु॒र्थ श्च॑तु॒र्थ ऋ॑तू॒नाम् । \newline
23. ऋ॒तू॒नाम् प॑ञ्च॒मः प॑ञ्च॒म ऋ॑तू॒ना मृ॑तू॒नाम् प॑ञ्च॒मः । \newline
24. प॒ञ्च॒मः सं॑ॅवथ्स॒रस्य॑ संॅवथ्स॒रस्य॑ पञ्च॒मः प॑ञ्च॒मः सं॑ॅवथ्स॒रस्य॑ । \newline
25. सं॒ॅव॒थ्स॒रस्य॑ ष॒ष्ठ ष्ष॒ष्ठः सं॑ॅवथ्स॒रस्य॑ संॅवथ्स॒रस्य॑ ष॒ष्ठः । \newline
26. सं॒ॅव॒थ्स॒रस्येति॑ सं - व॒थ्स॒रस्य॑ । \newline
27. ष॒ष्ठ इति॑ ष॒ष्ठः । \newline

\textbf{Ghana Paata } \newline

1. ओजो᳚ ग्री॒वाभि॑र् ग्री॒वाभि॒ रोज॒ ओजो᳚ ग्री॒वाभि॒र् निर्.ऋ॑ति॒म् निर्.ऋ॑तिम् ग्री॒वाभि॒ रोज॒ ओजो᳚ ग्री॒वाभि॒र् निर्.ऋ॑तिम् । \newline
2. ग्री॒वाभि॒र् निर्.ऋ॑ति॒म् निर्.ऋ॑तिम् ग्री॒वाभि॑र् ग्री॒वाभि॒र् निर्.ऋ॑ति म॒स्थभि॑ र॒स्थभि॒र् निर्.ऋ॑तिम् ग्री॒वाभि॑र् ग्री॒वाभि॒र् निर्.ऋ॑ति म॒स्थभिः॑ । \newline
3. निर्.ऋ॑ति म॒स्थभि॑ र॒स्थभि॒र् निर्.ऋ॑ति॒म् निर्.ऋ॑ति म॒स्थभि॒ रिन्द्र॒ मिन्द्र॑ म॒स्थभि॒र् निर्.ऋ॑ति॒म् निर्.ऋ॑ति म॒स्थभि॒ रिन्द्र᳚म् । \newline
4. निर्.ऋ॑ति॒मिति॒ निः - ऋ॒ति॒म् । \newline
5. अ॒स्थभि॒ रिन्द्र॒ मिन्द्र॑ म॒स्थभि॑ र॒स्थभि॒ रिन्द्रꣳ॒॒ स्वप॑सा॒ स्वप॒सेन्द्र॑ म॒स्थभि॑ र॒स्थभि॒ रिन्द्रꣳ॒॒ स्वप॑सा । \newline
6. अ॒स्थभि॒रित्य॒स्थ - भिः॒ । \newline
7. इन्द्रꣳ॒॒ स्वप॑सा॒ स्वप॒सेन्द्र॒ मिन्द्रꣳ॒॒ स्वप॑सा॒ वहे॑न॒ वहे॑न॒ स्वप॒सेन्द्र॒ मिन्द्रꣳ॒॒ स्वप॑सा॒ वहे॑न । \newline
8. स्वप॑सा॒ वहे॑न॒ वहे॑न॒ स्वप॑सा॒ स्वप॑सा॒ वहे॑न रु॒द्रस्य॑ रु॒द्रस्य॒ वहे॑न॒ स्वप॑सा॒ स्वप॑सा॒ वहे॑न रु॒द्रस्य॑ । \newline
9. स्वप॒सेति॑ सु - अप॑सा । \newline
10. वहे॑न रु॒द्रस्य॑ रु॒द्रस्य॒ वहे॑न॒ वहे॑न रु॒द्रस्य॑ विच॒लो वि॑च॒लो रु॒द्रस्य॒ वहे॑न॒ वहे॑न रु॒द्रस्य॑ विच॒लः । \newline
11. रु॒द्रस्य॑ विच॒लो वि॑च॒लो रु॒द्रस्य॑ रु॒द्रस्य॑ विच॒लः स्क॒न्धः स्क॒न्धो वि॑च॒लो रु॒द्रस्य॑ रु॒द्रस्य॑ विच॒लः स्क॒न्धः । \newline
12. वि॒च॒लः स्क॒न्धः स्क॒न्धो वि॑च॒लो वि॑च॒लः स्क॒न्धो॑ ऽहोरा॒त्रयो॑ रहोरा॒त्रयोः᳚ स्क॒न्धो वि॑च॒लो वि॑च॒लः स्क॒न्धो॑ ऽहोरा॒त्रयोः᳚ । \newline
13. वि॒च॒ल इति॑ वि - च॒लः । \newline
14. स्क॒न्धो॑ ऽहोरा॒त्रयो॑ रहोरा॒त्रयोः᳚ स्क॒न्धः स्क॒न्धो॑ ऽहोरा॒त्रयो᳚र् द्वि॒तीयो᳚ द्वि॒तीयो॑ ऽहोरा॒त्रयोः᳚ स्क॒न्धः स्क॒न्धो॑ ऽहोरा॒त्रयो᳚र् द्वि॒तीयः॑ । \newline
15. अ॒हो॒रा॒त्रयो᳚र् द्वि॒तीयो᳚ द्वि॒तीयो॑ ऽहोरा॒त्रयो॑ रहोरा॒त्रयो᳚र् द्वि॒तीयो᳚ ऽर्द्धमा॒साना॑ मर्द्धमा॒साना᳚म् द्वि॒तीयो॑ ऽहोरा॒त्रयो॑ रहोरा॒त्रयो᳚र् द्वि॒तीयो᳚ ऽर्द्धमा॒साना᳚म् । \newline
16. अ॒हो॒रा॒त्रयो॒रित्य॑हः - रा॒त्रयोः᳚ । \newline
17. द्वि॒तीयो᳚ ऽर्द्धमा॒साना॑ मर्द्धमा॒साना᳚म् द्वि॒तीयो᳚ द्वि॒तीयो᳚ ऽर्द्धमा॒साना᳚म् तृ॒तीय॑ स्तृ॒तीयो᳚ ऽर्द्धमा॒साना᳚म् द्वि॒तीयो᳚ द्वि॒तीयो᳚ ऽर्द्धमा॒साना᳚म् तृ॒तीयः॑ । \newline
18. अ॒र्द्ध॒मा॒साना᳚म् तृ॒तीय॑ स्तृ॒तीयो᳚ ऽर्द्धमा॒साना॑ मर्द्धमा॒साना᳚म् तृ॒तीयो॑ मा॒साम् मा॒साम् तृ॒तीयो᳚ ऽर्द्धमा॒साना॑ मर्द्धमा॒साना᳚म् तृ॒तीयो॑ मा॒साम् । \newline
19. अ॒र्द्ध॒मा॒साना॒मित्य॑र्द्ध - मा॒साना᳚म् । \newline
20. तृ॒तीयो॑ मा॒साम् मा॒साम् तृ॒तीय॑ स्तृ॒तीयो॑ मा॒साम् च॑तु॒र्थ श्च॑तु॒र्थो मा॒साम् तृ॒तीय॑ स्तृ॒तीयो॑ मा॒साम् च॑तु॒र्थः । \newline
21. मा॒साम् च॑तु॒र्थ श्च॑तु॒र्थो मा॒साम् मा॒साम् च॑तु॒र्थ ऋ॑तू॒ना मृ॑तू॒नाम् च॑तु॒र्थो मा॒साम् मा॒साम् च॑तु॒र्थ ऋ॑तू॒नाम् । \newline
22. च॒तु॒र्थ ऋ॑तू॒ना मृ॑तू॒नाम् च॑तु॒र्थ श्च॑तु॒र्थ ऋ॑तू॒नाम् प॑ञ्च॒मः प॑ञ्च॒म ऋ॑तू॒नाम् च॑तु॒र्थ श्च॑तु॒र्थ ऋ॑तू॒नाम् प॑ञ्च॒मः । \newline
23. ऋ॒तू॒नाम् प॑ञ्च॒मः प॑ञ्च॒म ऋ॑तू॒ना मृ॑तू॒नाम् प॑ञ्च॒मः सं॑ॅवथ्स॒रस्य॑ संॅवथ्स॒रस्य॑ पञ्च॒म ऋ॑तू॒ना मृ॑तू॒नाम् प॑ञ्च॒मः सं॑ॅवथ्स॒रस्य॑ । \newline
24. प॒ञ्च॒मः सं॑ॅवथ्स॒रस्य॑ संॅवथ्स॒रस्य॑ पञ्च॒मः प॑ञ्च॒मः सं॑ॅवथ्स॒रस्य॑ ष॒ष्ठ ष्ष॒ष्ठः सं॑ॅवथ्स॒रस्य॑ पञ्च॒मः प॑ञ्च॒मः सं॑ॅवथ्स॒रस्य॑ ष॒ष्ठः । \newline
25. सं॒ॅव॒थ्स॒रस्य॑ ष॒ष्ठ ष्ष॒ष्ठः सं॑ॅवथ्स॒रस्य॑ संॅवथ्स॒रस्य॑ ष॒ष्ठः । \newline
26. सं॒ॅव॒थ्स॒रस्येति॑ सं - व॒थ्स॒रस्य॑ । \newline
27. ष॒ष्ठ इति॑ ष॒ष्ठः । \newline
\pagebreak
\markright{ TS 5.7.19.1  \hfill https://www.vedavms.in \hfill}

\section{ TS 5.7.19.1 }

\textbf{TS 5.7.19.1 } \newline
\textbf{Samhita Paata} \newline

आ॒न॒न्दं न॒न्दथु॑ना॒ कामं॑ प्रत्या॒साभ्यां᳚ भ॒यꣳ शि॑ती॒मभ्यां᳚ प्र॒शिषं॑ प्रशा॒साभ्याꣳ॑ सूर्याचन्द्र॒मसौ॒ वृक्या᳚भ्याꣳ श्यामशब॒लौ मत॑स्नाभ्यां॒ ॅव्यु॑ष्टिꣳ रू॒पेण॒ निम्रु॑क्ति॒मरू॑पेण ॥ \newline

\textbf{Pada Paata} \newline

आ॒न॒न्दमित्या᳚ - न॒न्दम् । न॒न्दथु॑ना । काम᳚म् । प्र॒त्या॒साभ्या॒मिति॑ प्रति - आ॒साभ्या᳚म् । भ॒यम् । शि॒ती॒मभ्या॒मिति॑ शिती॒म - भ्या॒म् । प्र॒शिष॒मिति॑ प्र - शिष᳚म् । प्र॒शा॒साभ्या॒मिति॑ प्र - शा॒साभ्या᳚म् । सू॒र्या॒च॒न्द्र॒मसा॒विति॑ सूर्या - च॒न्द्र॒मसौ᳚ । वृक्या᳚भ्याम् । श्या॒म॒श॒ब॒लाविति॑ श्याम - श॒ब॒लौ । मत॑स्नाभ्याम् । व्यु॑ष्टि॒मिति॒ वि-उ॒ष्टि॒म् । रू॒पेण॑ । निम्रु॑क्ति॒मिति॒ नि-म्रु॒क्ति॒म् । अरू॑पेण ॥  \newline


\textbf{Krama Paata} \newline

आ॒न॒न्दम् न॒न्दथु॑ना । आ॒न॒न्दमित्या᳚ - न॒न्दम् । न॒न्दथु॑ना॒ काम᳚म् । काम॑म् प्रत्या॒साभ्या᳚म् । प्र॒त्या॒साभ्या᳚म् भ॒यम् । प्र॒त्या॒साभ्या॒मिति॑ प्रति - आ॒साभ्या᳚म् । भ॒यꣳ शि॑ती॒मभ्या᳚म् । शि॒ती॒मभ्या᳚म् प्र॒शिष᳚म् । शि॒ती॒मभ्या॒मिति॑ शिती॒म - भ्या॒म् । प्र॒शिष॑म् प्रशा॒साभ्या᳚म् । प्र॒शिष॒मिति॑ प्र - शिष᳚म् । प्र॒शा॒साभ्याꣳ॑ सूर्याचन्द्र॒मसौ᳚ । प्र॒शा॒साभ्या॒मिति॑ प्र - शा॒साभ्या᳚म् । सू॒र्या॒च॒न्द्र॒मसौ॒ वृक्या᳚भ्याम् । सू॒र्या॒च॒न्द्र॒मसा॒विति॑ सूर्या - च॒न्द्र॒मसौ᳚ । वृक्या᳚भ्याꣳ श्यामशब॒लौ । श्या॒म॒श॒ब॒लौ मत॑स्नाभ्याम् । श्या॒म॒श॒ब॒लाविति॑ श्याम - श॒ब॒लौ । मत॑स्नाभ्या॒म् ॅव्यु॑ष्टिम् । व्यु॑ष्टिꣳ रू॒पेण॑ । व्यु॑ष्टि॒मिति॒ वि - उ॒ष्टि॒म् । रू॒पेण॒ निम्रु॑क्तिम् । निम्रु॑क्ति॒मरू॑पेण । निम्रु॑क्ति॒मिति॒ नि - म्रु॒क्ति॒म् । अरू॑पे॒णेत्यरू॑पेण । \newline

\textbf{Jatai Paata} \newline

1. आ॒न॒न्दम् न॒न्दथु॑ना न॒न्दथु॑ना ऽऽन॒न्द मा॑न॒न्दम् न॒न्दथु॑ना । \newline
2. आ॒न॒न्दमित्या᳚ - न॒न्दम् । \newline
3. न॒न्दथु॑ना॒ काम॒म् काम॑न् न॒न्दथु॑ना न॒न्दथु॑ना॒ काम᳚म् । \newline
4. काम॑म् प्रत्या॒साभ्या᳚म् प्रत्या॒साभ्या॒म् काम॒म् काम॑म् प्रत्या॒साभ्या᳚म् । \newline
5. प्र॒त्या॒साभ्या᳚म् भ॒यम् भ॒यम् प्र॑त्या॒साभ्या᳚म् प्रत्या॒साभ्या᳚म् भ॒यम् । \newline
6. प्र॒त्या॒साभ्या॒मिति॑ प्रति - आ॒साभ्या᳚म् । \newline
7. भ॒यꣳ शि॑ती॒मभ्याꣳ॑ शिती॒मभ्या᳚म् भ॒यम् भ॒यꣳ शि॑ती॒मभ्या᳚म् । \newline
8. शि॒ती॒मभ्या᳚म् प्र॒शिष॑म् प्र॒शिषꣳ॑ शिती॒मभ्याꣳ॑ शिती॒मभ्या᳚म् प्र॒शिष᳚म् । \newline
9. शि॒ती॒मभ्या॒मिति॑ शिती॒म - भ्या॒म् । \newline
10. प्र॒शिष॑म् प्रशा॒साभ्या᳚म् प्रशा॒साभ्या᳚म् प्र॒शिष॑म् प्र॒शिष॑म् प्रशा॒साभ्या᳚म् । \newline
11. प्र॒शिष॒मिति॑ प्र - शिष᳚म् । \newline
12. प्र॒शा॒साभ्याꣳ॑ सूर्याचन्द्र॒मसौ॑ सूर्याचन्द्र॒मसौ᳚ प्रशा॒साभ्या᳚म् प्रशा॒साभ्याꣳ॑ सूर्याचन्द्र॒मसौ᳚ । \newline
13. प्र॒शा॒साभ्या॒मिति॑ प्र - शा॒साभ्या᳚म् । \newline
14. सू॒र्या॒च॒न्द्र॒मसौ॒ वृक्या᳚भ्यां॒ ॅवृक्या᳚भ्याꣳ सूर्याचन्द्र॒मसौ॑ सूर्याचन्द्र॒मसौ॒ वृक्या᳚भ्याम् । \newline
15. सू॒र्या॒च॒न्द्र॒मसा॒विति॑ सूर्या - च॒न्द्र॒मसौ᳚ । \newline
16. वृक्या᳚भ्याꣳ श्यामशब॒लौ श्या॑मशब॒लौ वृक्या᳚भ्यां॒ ॅवृक्या᳚भ्याꣳ श्यामशब॒लौ । \newline
17. श्या॒म॒श॒ब॒लौ मत॑स्नाभ्या॒म् मत॑स्नाभ्याꣳ श्यामशब॒लौ श्या॑मशब॒लौ मत॑स्नाभ्याम् । \newline
18. श्या॒म॒श॒ब॒लाविति॑ श्याम - श॒ब॒लौ । \newline
19. मत॑स्नाभ्यां॒ ॅव्यु॑ष्टिं॒ ॅव्यु॑ष्टि॒म् मत॑स्नाभ्या॒म् मत॑स्नाभ्यां॒ ॅव्यु॑ष्टिम् । \newline
20. व्यु॑ष्टिꣳ रू॒पेण॑ रू॒पेण॒ व्यु॑ष्टिं॒ ॅव्यु॑ष्टिꣳ रू॒पेण॑ । \newline
21. व्यु॑ष्टि॒मिति॒ वि - उ॒ष्टि॒म् । \newline
22. रू॒पेण॒ निम्रु॑क्ति॒म् निम्रु॑क्तिꣳ रू॒पेण॑ रू॒पेण॒ निम्रु॑क्तिम् । \newline
23. निम्रु॑क्ति॒ मरू॑पे॒णारू॑पेण॒ निम्रु॑क्ति॒म् निम्रु॑क्ति॒ मरू॑पेण । \newline
24. निम्रु॑क्ति॒मिति॒ नि - म्रु॒क्ति॒म् । \newline
25. अरू॑पे॒णेत्यरू॑पेण । \newline

\textbf{Ghana Paata } \newline

1. आ॒न॒न्दम् न॒न्दथु॑ना न॒न्दथु॑ना ऽऽन॒न्द मा॑न॒न्दम् न॒न्दथु॑ना॒ काम॒म् काम॑म् न॒न्दथु॑ना ऽऽन॒न्द मा॑न॒न्दम् न॒न्दथु॑ना॒ काम᳚म् । \newline
2. आ॒न॒न्दमित्या᳚ - न॒न्दम् । \newline
3. न॒न्दथु॑ना॒ काम॒म् काम॑म् न॒न्दथु॑ना न॒न्दथु॑ना॒ काम॑म् प्रत्या॒साभ्या᳚म् प्रत्या॒साभ्या॒म् काम॑म् न॒न्दथु॑ना न॒न्दथु॑ना॒ काम॑म् प्रत्या॒साभ्या᳚म् । \newline
4. काम॑म् प्रत्या॒साभ्या᳚म् प्रत्या॒साभ्या॒म् काम॒म् काम॑म् प्रत्या॒साभ्या᳚म् भ॒यम् भ॒यम् प्र॑त्या॒साभ्या॒म् काम॒म् काम॑म् प्रत्या॒साभ्या᳚म् भ॒यम् । \newline
5. प्र॒त्या॒साभ्या᳚म् भ॒यम् भ॒यम् प्र॑त्या॒साभ्या᳚म् प्रत्या॒साभ्या᳚म् भ॒यꣳ शि॑ती॒मभ्याꣳ॑ शिती॒मभ्या᳚म् भ॒यम् प्र॑त्या॒साभ्या᳚म् प्रत्या॒साभ्या᳚म् भ॒यꣳ शि॑ती॒मभ्या᳚म् । \newline
6. प्र॒त्या॒साभ्या॒मिति॑ प्रति - आ॒साभ्या᳚म् । \newline
7. भ॒यꣳ शि॑ती॒मभ्याꣳ॑ शिती॒मभ्या᳚म् भ॒यम् भ॒यꣳ शि॑ती॒मभ्या᳚म् प्र॒शिष॑म् प्र॒शिषꣳ॑ शिती॒मभ्या᳚म् भ॒यम् भ॒यꣳ शि॑ती॒मभ्या᳚म् प्र॒शिष᳚म् । \newline
8. शि॒ती॒मभ्या᳚म् प्र॒शिष॑म् प्र॒शिषꣳ॑ शिती॒मभ्याꣳ॑ शिती॒मभ्या᳚म् प्र॒शिष॑म् प्रशा॒साभ्या᳚म् प्रशा॒साभ्या᳚म् प्र॒शिषꣳ॑ शिती॒मभ्याꣳ॑ शिती॒मभ्या᳚म् प्र॒शिष॑म् प्रशा॒साभ्या᳚म् । \newline
9. शि॒ती॒मभ्या॒मिति॑ शिती॒म - भ्या॒म् । \newline
10. प्र॒शिष॑म् प्रशा॒साभ्या᳚म् प्रशा॒साभ्या᳚म् प्र॒शिष॑म् प्र॒शिष॑म् प्रशा॒साभ्याꣳ॑ सूर्याचन्द्र॒मसौ॑ सूर्याचन्द्र॒मसौ᳚ प्रशा॒साभ्या᳚म् प्र॒शिष॑म् प्र॒शिष॑म् प्रशा॒साभ्याꣳ॑ सूर्याचन्द्र॒मसौ᳚ । \newline
11. प्र॒शिष॒मिति॑ प्र - शिष᳚म् । \newline
12. प्र॒शा॒साभ्याꣳ॑ सूर्याचन्द्र॒मसौ॑ सूर्याचन्द्र॒मसौ᳚ प्रशा॒साभ्या᳚म् प्रशा॒साभ्याꣳ॑ सूर्याचन्द्र॒मसौ॒ वृक्या᳚भ्यां॒ ॅवृक्या᳚भ्याꣳ सूर्याचन्द्र॒मसौ᳚ प्रशा॒साभ्या᳚म् प्रशा॒साभ्याꣳ॑ सूर्याचन्द्र॒मसौ॒ वृक्या᳚भ्याम् । \newline
13. प्र॒शा॒साभ्या॒मिति॑ प्र - शा॒साभ्या᳚म् । \newline
14. सू॒र्या॒च॒न्द्र॒मसौ॒ वृक्या᳚भ्यां॒ ॅवृक्या᳚भ्याꣳ सूर्याचन्द्र॒मसौ॑ सूर्याचन्द्र॒मसौ॒ वृक्या᳚भ्याꣳ श्यामशब॒लौ श्या॑मशब॒लौ वृक्या᳚भ्याꣳ सूर्याचन्द्र॒मसौ॑ सूर्याचन्द्र॒मसौ॒ वृक्या᳚भ्याꣳ श्यामशब॒लौ । \newline
15. सू॒र्या॒च॒न्द्र॒मसा॒विति॑ सूर्या - च॒न्द्र॒मसौ᳚ । \newline
16. वृक्या᳚भ्याꣳ श्यामशब॒लौ श्या॑मशब॒लौ वृक्या᳚भ्यां॒ ॅवृक्या᳚भ्याꣳ श्यामशब॒लौ मत॑स्नाभ्या॒म् मत॑स्नाभ्याꣳ श्यामशब॒लौ वृक्या᳚भ्यां॒ ॅवृक्या᳚भ्याꣳ श्यामशब॒लौ मत॑स्नाभ्याम् । \newline
17. श्या॒म॒श॒ब॒लौ मत॑स्नाभ्या॒म् मत॑स्नाभ्याꣳ श्यामशब॒लौ श्या॑मशब॒लौ मत॑स्नाभ्यां॒ ॅव्यु॑ष्टिं॒ ॅव्यु॑ष्टि॒म् मत॑स्नाभ्याꣳ श्यामशब॒लौ श्या॑मशब॒लौ मत॑स्नाभ्यां॒ ॅव्यु॑ष्टिम् । \newline
18. श्या॒म॒श॒ब॒लाविति॑ श्याम - श॒ब॒लौ । \newline
19. मत॑स्नाभ्यां॒ ॅव्यु॑ष्टिं॒ ॅव्यु॑ष्टि॒म् मत॑स्नाभ्या॒म् मत॑स्नाभ्यां॒ ॅव्यु॑ष्टिꣳ रू॒पेण॑ रू॒पेण॒ व्यु॑ष्टि॒म् मत॑स्नाभ्या॒म् मत॑स्नाभ्यां॒ ॅव्यु॑ष्टिꣳ रू॒पेण॑ । \newline
20. व्यु॑ष्टिꣳ रू॒पेण॑ रू॒पेण॒ व्यु॑ष्टिं॒ ॅव्यु॑ष्टिꣳ रू॒पेण॒ निम्रु॑क्ति॒म् निम्रु॑क्तिꣳ रू॒पेण॒ व्यु॑ष्टिं॒ ॅव्यु॑ष्टिꣳ रू॒पेण॒ निम्रु॑क्तिम् । \newline
21. व्यु॑ष्टि॒मिति॒ वि - उ॒ष्टि॒म् । \newline
22. रू॒पेण॒ निम्रु॑क्ति॒म् निम्रु॑क्तिꣳ रू॒पेण॑ रू॒पेण॒ निम्रु॑क्ति॒ मरू॑पे॒णा रू॑पेण॒ निम्रु॑क्तिꣳ रू॒पेण॑ रू॒पेण॒ निम्रु॑क्ति॒ मरू॑पेण । \newline
23. निम्रु॑क्ति॒ मरू॑पे॒णा रू॑पेण॒ निम्रु॑क्ति॒म् निम्रु॑क्ति॒ मरू॑पेण । \newline
24. निम्रु॑क्ति॒मिति॒ नि - म्रु॒क्ति॒म् । \newline
25. अरू॑पे॒णेत्यरू॑पेण । \newline
\pagebreak
\markright{ TS 5.7.20.1  \hfill https://www.vedavms.in \hfill}

\section{ TS 5.7.20.1 }

\textbf{TS 5.7.20.1 } \newline
\textbf{Samhita Paata} \newline

अह॑र्माꣳ॒॒सेन॒ रात्रिं॒ पीव॑सा॒ऽपो यू॒षेण॑ घृ॒तꣳ रसे॑न॒ श्यां ॅवस॑या दू॒षीका॑भि-र्ह्रा॒दुनि॒-मश्रु॑भिः॒ पृष्वां॒ दिवꣳ॑ रू॒पेण॒ नक्ष॑त्राणि॒ प्रति॑रूपेण पृथि॒वीं चर्म॑णा छ॒वीं छ॒व्यो॑ पाकृ॑ताय॒ स्वाहा ऽऽल॑ब्धाय॒ स्वाहा॑ हु॒ताय॒ स्वाहा᳚ ॥ \newline

\textbf{Pada Paata} \newline

अहः॑ । माꣳ॒॒सेन॑ । रात्रि᳚म् । पीव॑सा । अ॒पः । यू॒षेण॑ । घृ॒तम् । रसे॑न । श्याम् । वस॑या । दू॒षीका॑भिः । ह्रा॒दुनि᳚म् । अश्रु॑भि॒रित्यश्रु॑ - भिः॒ । पृष्वा᳚म् । दिव᳚म् । रू॒पेण॑ । नक्ष॑त्राणि । प्रति॑रूपे॒णेति॒ प्रति॑ - रू॒पे॒ण॒ । पृ॒थि॒वीम् । चर्म॑णा । छ॒वीम् । छ॒व्या᳚ । उ॒पाकृ॑ता॒येत्यु॑प - आकृ॑ताय । स्वाहा᳚ । आल॑ब्धा॒येत्या - ल॒ब्धा॒य॒ । स्वाहा᳚ । हु॒ताय॑ । स्वाहा᳚ ॥  \newline


\textbf{Krama Paata} \newline

अह॑र्माꣳ॒॒सेन॑ । माꣳ॒॒सेन॒ रात्रि᳚म् । रात्रि॒म् पीव॑सा । पीव॑सा॒ऽपः । अ॒पो यू॒षेण॑ । यू॒षेण॑ घृ॒तम् । घृ॒तꣳ रसे॑न । रसे॑न॒ श्याम् । श्याम् ॅवस॑या । वस॑या दू॒षीका॑भिः । दू॒षीका॑भिर् ह्रा॒दुनि᳚म् । ह्रा॒दुनि॒मश्रु॑भिः । अश्रु॑भिः॒ पृष्वा᳚म् । अश्रु॑भि॒रित्यश्रु॑ - भिः॒ । पृष्वा॒म् दिव᳚म् । दिवꣳ॑ रू॒पेण॑ । रू॒पेण॒ नक्ष॑त्राणि । नक्ष॑त्राणि॒ प्रति॑रूपेण । प्रति॑रूपेण पृथि॒वीम् । प्रति॑रूपे॒णेति॒ प्रति॑ - रू॒पे॒ण॒ । पृ॒थि॒वीम् चर्म॑णा । चर्म॑णा छ॒वीम् । छ॒वीम् छ॒व्या᳚ । छ॒व्यो॑पाकृ॑ताय । उ॒पाकृ॑ताय॒ स्वाहा᳚ । उ॒पाकृ॑ता॒येत्यु॑प - आकृ॑ताय । स्वाहाऽऽल॑ब्धाय । आल॑ब्धाय॒ स्वाहा᳚ । आल॑ब्धा॒येत्या - ल॒ब्धा॒य॒ । स्वाहा॑ हु॒ताय॑ । हु॒ताय॒ स्वाहा᳚ । स्वाहेति॒ स्वाहा᳚ । \newline

\textbf{Jatai Paata} \newline

1. अह॑र् माꣳ॒॒सेन॑ माꣳ॒॒सेनाह॒ रह॑र् माꣳ॒॒सेन॑ । \newline
2. माꣳ॒॒सेन॒ रात्रिꣳ॒॒ रात्रि॑म् माꣳ॒॒सेन॑ माꣳ॒॒सेन॒ रात्रि᳚म् । \newline
3. रात्रि॒म् पीव॑सा॒ पीव॑सा॒ रात्रिꣳ॒॒ रात्रि॒म् पीव॑सा । \newline
4. पीव॑सा॒ ऽपो॑ ऽपः पीव॑सा॒ पीव॑सा॒ ऽपः । \newline
5. अ॒पो यू॒षेण॑ यू॒षेणा॒पो॑ ऽपो यू॒षेण॑ । \newline
6. यू॒षेण॑ घृ॒तम् घृ॒तं ॅयू॒षेण॑ यू॒षेण॑ घृ॒तम् । \newline
7. घृ॒तꣳ रसे॑न॒ रसे॑न घृ॒तम् घृ॒तꣳ रसे॑न । \newline
8. रसे॑न॒ श्याꣳ श्याꣳ रसे॑न॒ रसे॑न॒ श्याम् । \newline
9. श्यां ॅवस॑या॒ वस॑या॒ श्याꣳ श्यां ॅवस॑या । \newline
10. वस॑या दू॒षीका॑भिर् दू॒षीका॑भि॒र् वस॑या॒ वस॑या दू॒षीका॑भिः । \newline
11. दू॒षीका॑भिर् ह्रा॒दुनिꣳ॑ ह्रा॒दुनि॑म् दू॒षीका॑भिर् दू॒षीका॑भिर् ह्रा॒दुनि᳚म् । \newline
12. ह्रा॒दुनि॒ मश्रु॑भि॒ रश्रु॑भिर् ह्रा॒दुनिꣳ॑ ह्रा॒दुनि॒ मश्रु॑भिः । \newline
13. अश्रु॑भिः॒ पृष्वा॒म् पृष्वा॒ मश्रु॑भि॒ रश्रु॑भिः॒ पृष्वा᳚म् । \newline
14. अश्रु॑भि॒रित्यश्रु॑ - भिः॒ । \newline
15. पृष्वा॒म् दिव॒म् दिव॒म् पृष्वा॒म् पृष्वा॒म् दिव᳚म् । \newline
16. दिवꣳ॑ रू॒पेण॑ रू॒पेण॒ दिव॒म् दिवꣳ॑ रू॒पेण॑ । \newline
17. रू॒पेण॒ नक्ष॑त्राणि॒ नक्ष॑त्राणि रू॒पेण॑ रू॒पेण॒ नक्ष॑त्राणि । \newline
18. नक्ष॑त्राणि॒ प्रति॑रूपेण॒ प्रति॑रूपेण॒ नक्ष॑त्राणि॒ नक्ष॑त्राणि॒ प्रति॑रूपेण । \newline
19. प्रति॑रूपेण पृथि॒वीम् पृ॑थि॒वीम् प्रति॑रूपेण॒ प्रति॑रूपेण पृथि॒वीम् । \newline
20. प्रति॑रूपे॒णेति॒ प्रति॑ - रू॒पे॒ण॒ । \newline
21. पृ॒थि॒वीम् चर्म॑णा॒ चर्म॑णा पृथि॒वीम् पृ॑थि॒वीम् चर्म॑णा । \newline
22. चर्म॑णा छ॒वीम् छ॒वीम् चर्म॑णा॒ चर्म॑णा छ॒वीम् । \newline
23. छ॒वीम् छ॒व्या॑ छ॒व्या॑ छ॒वीम् छ॒वीम् छ॒व्या᳚ । \newline
24. छ॒व्यो॑ पाकृ॑तायो॒ पाकृ॑ताय छ॒व्या॑ छ॒व्यो॑ पाकृ॑ताय । \newline
25. उ॒पाकृ॑ताय॒ स्वाहा॒ स्वाहो॒ पाकृ॑तायो॒ पाकृ॑ताय॒ स्वाहा᳚ । \newline
26. उ॒पाकृ॑ता॒येत्यु॑प - आकृ॑ताय । \newline
27. स्वाहा ऽऽल॑ब्धा॒या ल॑ब्धाय॒ स्वाहा॒ स्वाहा ऽऽल॑ब्धाय । \newline
28. आल॑ब्धाय॒ स्वाहा॒ स्वाहा ऽऽल॑ब्धा॒या ल॑ब्धाय॒ स्वाहा᳚ । \newline
29. आल॑ब्धा॒येत्या - ल॒ब्धा॒य॒ । \newline
30. स्वाहा॑ हु॒ताय॑ हु॒ताय॒ स्वाहा॒ स्वाहा॑ हु॒ताय॑ । \newline
31. हु॒ताय॒ स्वाहा॒ स्वाहा॑ हु॒ताय॑ हु॒ताय॒ स्वाहा᳚ । \newline
32. स्वाहेति॒ स्वाहा᳚ । \newline

\textbf{Ghana Paata } \newline

1. अह॑र् माꣳ॒॒सेन॑ माꣳ॒॒सेना ह॒ रह॑र् माꣳ॒॒सेन॒ रात्रिꣳ॒॒ रात्रि॑म् माꣳ॒॒सेना ह॒ रह॑र् माꣳ॒॒सेन॒ रात्रि᳚म् । \newline
2. माꣳ॒॒सेन॒ रात्रिꣳ॒॒ रात्रि॑म् माꣳ॒॒सेन॑ माꣳ॒॒सेन॒ रात्रि॒म् पीव॑सा॒ पीव॑सा॒ रात्रि॑म् माꣳ॒॒सेन॑ माꣳ॒॒सेन॒ रात्रि॒म् पीव॑सा । \newline
3. रात्रि॒म् पीव॑सा॒ पीव॑सा॒ रात्रिꣳ॒॒ रात्रि॒म् पीव॑सा॒ ऽपो॑ ऽपः पीव॑सा॒ रात्रिꣳ॒॒ रात्रि॒म् पीव॑सा॒ ऽपः । \newline
4. पीव॑सा॒ ऽपो॑ ऽपः पीव॑सा॒ पीव॑सा॒ ऽपो यू॒षेण॑ यू॒षेणा॒पः पीव॑सा॒ पीव॑सा॒ ऽपो यू॒षेण॑ । \newline
5. अ॒पो यू॒षेण॑ यू॒षेणा॒पो॑ ऽपो यू॒षेण॑ घृ॒तम् घृ॒तं ॅयू॒षेणा॒पो॑ ऽपो यू॒षेण॑ घृ॒तम् । \newline
6. यू॒षेण॑ घृ॒तम् घृ॒तं ॅयू॒षेण॑ यू॒षेण॑ घृ॒तꣳ रसे॑न॒ रसे॑न घृ॒तं ॅयू॒षेण॑ यू॒षेण॑ घृ॒तꣳ रसे॑न । \newline
7. घृ॒तꣳ रसे॑न॒ रसे॑न घृ॒तम् घृ॒तꣳ रसे॑न॒ श्याꣳ श्याꣳ रसे॑न घृ॒तम् घृ॒तꣳ रसे॑न॒ श्याम् । \newline
8. रसे॑न॒ श्याꣳ श्याꣳ रसे॑न॒ रसे॑न॒ श्यां ॅवस॑या॒ वस॑या॒ श्याꣳ रसे॑न॒ रसे॑न॒ श्यां ॅवस॑या । \newline
9. श्यां ॅवस॑या॒ वस॑या॒ श्याꣳ श्यां ॅवस॑या दू॒षीका॑भिर् दू॒षीका॑भि॒र् वस॑या॒ श्याꣳ श्यां ॅवस॑या दू॒षीका॑भिः । \newline
10. वस॑या दू॒षीका॑भिर् दू॒षीका॑भि॒र् वस॑या॒ वस॑या दू॒षीका॑भिर् ह्रा॒दुनिꣳ॑ ह्रा॒दुनि॑म् दू॒षीका॑भि॒र् वस॑या॒ वस॑या दू॒षीका॑भिर् ह्रा॒दुनि᳚म् । \newline
11. दू॒षीका॑भिर् ह्रा॒दुनिꣳ॑ ह्रा॒दुनि॑म् दू॒षीका॑भिर् दू॒षीका॑भिर् ह्रा॒दुनि॒ मश्रु॑भि॒ रश्रु॑भिर् ह्रा॒दुनि॑म् दू॒षीका॑भिर् दू॒षीका॑भिर् ह्रा॒दुनि॒ मश्रु॑भिः । \newline
12. ह्रा॒दुनि॒ मश्रु॑भि॒ रश्रु॑भिर् ह्रा॒दुनिꣳ॑ ह्रा॒दुनि॒ मश्रु॑भिः॒ पृष्वा॒म् पृष्वा॒ मश्रु॑भिर् ह्रा॒दुनिꣳ॑ ह्रा॒दुनि॒ मश्रु॑भिः॒ पृष्वा᳚म् । \newline
13. अश्रु॑भिः॒ पृष्वा॒म् पृष्वा॒ मश्रु॑भि॒ रश्रु॑भिः॒ पृष्वा॒म् दिव॒म् दिव॒म् पृष्वा॒ मश्रु॑भि॒ रश्रु॑भिः॒ पृष्वा॒म् दिव᳚म् । \newline
14. अश्रु॑भि॒रित्यश्रु॑ - भिः॒ । \newline
15. पृष्वा॒म् दिव॒म् दिव॒म् पृष्वा॒म् पृष्वा॒म् दिवꣳ॑ रू॒पेण॑ रू॒पेण॒ दिव॒म् पृष्वा॒म् पृष्वा॒म् दिवꣳ॑ रू॒पेण॑ । \newline
16. दिवꣳ॑ रू॒पेण॑ रू॒पेण॒ दिव॒म् दिवꣳ॑ रू॒पेण॒ नक्ष॑त्राणि॒ नक्ष॑त्राणि रू॒पेण॒ दिव॒म् दिवꣳ॑ रू॒पेण॒ नक्ष॑त्राणि । \newline
17. रू॒पेण॒ नक्ष॑त्राणि॒ नक्ष॑त्राणि रू॒पेण॑ रू॒पेण॒ नक्ष॑त्राणि॒ प्रति॑रूपेण॒ प्रति॑रूपेण॒ नक्ष॑त्राणि रू॒पेण॑ रू॒पेण॒ नक्ष॑त्राणि॒ प्रति॑रूपेण । \newline
18. नक्ष॑त्राणि॒ प्रति॑रूपेण॒ प्रति॑रूपेण॒ नक्ष॑त्राणि॒ नक्ष॑त्राणि॒ प्रति॑रूपेण पृथि॒वीम् पृ॑थि॒वीम् प्रति॑रूपेण॒ नक्ष॑त्राणि॒ नक्ष॑त्राणि॒ प्रति॑रूपेण पृथि॒वीम् । \newline
19. प्रति॑रूपेण पृथि॒वीम् पृ॑थि॒वीम् प्रति॑रूपेण॒ प्रति॑रूपेण पृथि॒वीम् चर्म॑णा॒ चर्म॑णा पृथि॒वीम् प्रति॑रूपेण॒ प्रति॑रूपेण पृथि॒वीम् चर्म॑णा । \newline
20. प्रति॑रूपे॒णेति॒ प्रति॑ - रू॒पे॒ण॒ । \newline
21. पृ॒थि॒वीम् चर्म॑णा॒ चर्म॑णा पृथि॒वीम् पृ॑थि॒वीम् चर्म॑णा छ॒वीम् छ॒वीम् चर्म॑णा पृथि॒वीम् पृ॑थि॒वीम् चर्म॑णा छ॒वीम् । \newline
22. चर्म॑णा छ॒वीम् छ॒वीम् चर्म॑णा॒ चर्म॑णा छ॒वीम् छ॒व्या॑ छ॒व्या॑ छ॒वीम् चर्म॑णा॒ चर्म॑णा छ॒वीम् छ॒व्या᳚ । \newline
23. छ॒वीम् छ॒व्या॑ छ॒व्या॑ छ॒वीम् छ॒वीम् छ॒व्यो॑ पाकृ॑तायो॒ पाकृ॑ताय छ॒व्या॑ छ॒वीम् छ॒वीम् छ॒व्यो॑पाकृ॑ताय । \newline
24. छ॒व्यो॑पाकृ॑ता यो॒पाकृ॑ताय छ॒व्या॑ छ॒व्यो॑ पाकृ॑ताय॒ स्वाहा॒ स्वाहो॒पाकृ॑ताय छ॒व्या॑ 
छ॒व्यो॑पाकृ॑ताय॒ स्वाहा᳚ । \newline
25. उ॒पाकृ॑ताय॒ स्वाहा॒ स्वाहो॒पाकृ॑ता यो॒पाकृ॑ताय॒ स्वाहा ऽऽल॑ब्धा॒या ल॑ब्धाय॒ स्वाहो॒पाकृ॑ता
यो॒पाकृ॑ताय॒ स्वाहा ऽऽल॑ब्धाय । \newline
26. उ॒पाकृ॑ता॒येत्यु॑प - आकृ॑ताय । \newline
27. स्वाहा ऽऽल॑ब्धा॒या ल॑ब्धाय॒ स्वाहा॒ स्वाहा ऽऽल॑ब्धाय॒ स्वाहा॒ स्वाहा ऽऽल॑ब्धाय॒ स्वाहा॒ स्वाहा ऽऽल॑ब्धाय॒ स्वाहा᳚ । \newline
28. आल॑ब्धाय॒ स्वाहा॒ स्वाहा ऽऽल॑ब्धा॒या ल॑ब्धाय॒ स्वाहा॑ हु॒ताय॑ हु॒ताय॒ स्वाहा ऽऽल॑ब्धा॒या ल॑ब्धाय॒ स्वाहा॑ हु॒ताय॑ । \newline
29. आल॑ब्धा॒येत्या - ल॒ब्धा॒य॒ । \newline
30. स्वाहा॑ हु॒ताय॑ हु॒ताय॒ स्वाहा॒ स्वाहा॑ हु॒ताय॒ स्वाहा॒ स्वाहा॑ हु॒ताय॒ स्वाहा॒ स्वाहा॑ हु॒ताय॒ स्वाहा᳚ । \newline
31. हु॒ताय॒ स्वाहा॒ स्वाहा॑ हु॒ताय॑ हु॒ताय॒ स्वाहा᳚ । \newline
32. स्वाहेति॒ स्वाहा᳚ । \newline
\pagebreak
\markright{ TS 5.7.21.1  \hfill https://www.vedavms.in \hfill}

\section{ TS 5.7.21.1 }

\textbf{TS 5.7.21.1 } \newline
\textbf{Samhita Paata} \newline

अ॒ग्नेः प॑क्ष॒तिः सर॑स्वत्यै॒ निप॑क्षतिः॒ सोम॑स्य तृ॒तीया॒ऽपां च॑तु॒र्थ्योष॑धीनां पञ्च॒मी सं॑ॅवथ्स॒रस्य॑ ष॒ष्ठी म॒रुताꣳ॑ सप्त॒मी बृह॒स्पते॑रष्ट॒मी मि॒त्रस्य॑ नव॒मी वरु॑णस्य दश॒मीन्द्र॑स्यैकाद॒शी विश्वे॑षां दे॒वानां᳚ द्वाद॒शी द्यावा॑पृथि॒व्योः पा॒र्श्वं ॅय॒मस्य॑ पाटू॒रः ॥ \newline

\textbf{Pada Paata} \newline

अ॒ग्नेः । प॒क्ष॒तिः । सर॑स्वत्यै । निप॑क्षति॒रिति॒ नि-प॒क्ष॒तिः॒ । सोम॑स्य । तृ॒तीया᳚ । अ॒पाम् । च॒तु॒र्थी । ओष॑धीनाम् । प॒ञ्च॒मी । सं॒ॅव॒थ्स॒रस्येति॑ सं - व॒थ्स॒रस्य॑ । ष॒ष्ठी । म॒रुता᳚म् । स॒प्त॒मी । बृह॒स्पतेः᳚ । अ॒ष्ट॒मी । मि॒त्रस्य॑ । न॒व॒मी । वरु॑णस्य । द॒श॒मी । इन्द्र॑स्य । ए॒का॒द॒शी । विश्वे॑षाम् । दे॒वाना᳚म् । द्वा॒द॒शी । द्यावा॑पृथि॒व्योरिति॒ द्यावा᳚-पृ॒थि॒व्योः । पा॒र्श्वम् । य॒मस्य॑ । पा॒टू॒रः ॥  \newline


\textbf{Krama Paata} \newline

अ॒ग्नेः प॑क्ष॒तिः । प॒क्ष॒तिः सर॑स्वत्यै । सर॑स्वत्यै॒ निप॑क्षतिः । निप॑क्षतिः॒ सोम॑स्य । निप॑क्षति॒रिति॒ नि - प॒क्ष॒तिः॒ । सोम॑स्य तृ॒तीया᳚ । तृ॒तीया॒ऽपाम् । अ॒पाम् च॑तु॒र्त्थी । च॒तु॒र्त्थ्योष॑धीनाम् । ओष॑धीनाम् पञ्च॒मी । प॒ञ्च॒मी स॑म्ॅवथ्स॒रस्य॑ । स॒म्ॅव॒थ्स॒रस्य॑ ष॒ष्ठी । स॒म्ॅव॒थ्स॒रस्येति॑ सम् - व॒थ्स॒रस्य॑ । ष॒ष्ठी म॒रुता᳚म् । म॒रुताꣳ॑ सप्त॒मी । स॒प्त॒मी बृह॒स्पतेः᳚ । बृह॒स्पते॑रष्ट॒मी । अ॒ष्ट॒मी मि॒त्रस्य॑ । मि॒त्रस्य॑ नव॒मी । न॒व॒मी वरु॑णस्य । वरु॑णस्य दश॒मी । द॒श॒मीन्द्र॑स्य । इन्द्र॑स्यैकाद॒शी । ए॒का॒द॒शी विश्वे॑षाम् । विश्वे॑षाम् दे॒वाना᳚म् । दे॒वाना᳚म् द्वाद॒शी । द्वा॒द॒शी द्यावा॑पृथि॒व्योः । द्यावा॑पृथि॒व्योः पा॒र्श्वम् । द्यावा॑पृथि॒व्योरिति॒ द्यावा᳚ - पृ॒थि॒व्योः । पा॒र्श्वम् ॅय॒मस्य॑ । य॒मस्य॑ पाटू॒रः । पा॒टू॒र इति॑ पाटू॒रः । \newline

\textbf{Jatai Paata} \newline

1. अ॒ग्नेः प॑क्ष॒तिः प॑क्ष॒ति र॒ग्ने र॒ग्नेः प॑क्ष॒तिः । \newline
2. प॒क्ष॒तिः सर॑स्वत्यै॒ सर॑स्वत्यै पक्ष॒तिः प॑क्ष॒तिः सर॑स्वत्यै । \newline
3. सर॑स्वत्यै॒ निप॑क्षति॒र् निप॑क्षतिः॒ सर॑स्वत्यै॒ सर॑स्वत्यै॒ निप॑क्षतिः । \newline
4. निप॑क्षतिः॒ सोम॑स्य॒ सोम॑स्य॒ निप॑क्षति॒र् निप॑क्षतिः॒ सोम॑स्य । \newline
5. निप॑क्षति॒रिति॒ नि - प॒क्ष॒तिः॒ । \newline
6. सोम॑स्य तृ॒तीया॑ तृ॒तीया॒ सोम॑स्य॒ सोम॑स्य तृ॒तीया᳚ । \newline
7. तृ॒तीया॒ ऽपा म॒पाम् तृ॒तीया॑ तृ॒तीया॒ ऽपाम् । \newline
8. अ॒पाम् च॑तु॒र्थी च॑तु॒र्थ्य॑पा म॒पाम् च॑तु॒र्थी । \newline
9. च॒तु॒र् थ्योष॑धीना॒ मोष॑धीनाम् चतु॒र्थी च॑तु॒र् थ्योष॑धीनाम् । \newline
10. ओष॑धीनाम् पञ्च॒मी प॑ञ्च॒म्यो ष॑धीना॒ मोष॑धीनाम् पञ्च॒मी । \newline
11. प॒ञ्च॒मी सं॑ॅवथ्स॒रस्य॑ संॅवथ्स॒रस्य॑ पञ्च॒मी प॑ञ्च॒मी सं॑ॅवथ्स॒रस्य॑ । \newline
12. सं॒ॅव॒थ्स॒रस्य॑ ष॒ष्ठी ष॒ष्ठी सं॑ॅवथ्स॒रस्य॑ संॅवथ्स॒रस्य॑ ष॒ष्ठी । \newline
13. सं॒ॅव॒थ्स॒रस्येति॑ सं - व॒थ्स॒रस्य॑ । \newline
14. ष॒ष्ठी म॒रुता᳚म् म॒रुताꣳ॑ ष॒ष्ठी ष॒ष्ठी म॒रुता᳚म् । \newline
15. म॒रुताꣳ॑ सप्त॒मी स॑प्त॒मी म॒रुता᳚म् म॒रुताꣳ॑ सप्त॒मी । \newline
16. स॒प्त॒मी बृह॒स्पते॒र् बृह॒स्पतेः᳚ सप्त॒मी स॑प्त॒मी बृह॒स्पतेः᳚ । \newline
17. बृह॒स्पते॑ रष्ट॒ म्य॑ष्ट॒मी बृह॒स्पते॒र् बृह॒स्पते॑ रष्ट॒मी । \newline
18. अ॒ष्ट॒मी मि॒त्रस्य॑ मि॒त्रस्या᳚ ष्ट॒म्य॑ ष्ट॒मी मि॒त्रस्य॑ । \newline
19. मि॒त्रस्य॑ नव॒मी न॑व॒मी मि॒त्रस्य॑ मि॒त्रस्य॑ नव॒मी । \newline
20. न॒व॒मी वरु॑णस्य॒ वरु॑णस्य नव॒मी न॑व॒मी वरु॑णस्य । \newline
21. वरु॑णस्य दश॒मी द॑श॒मी वरु॑णस्य॒ वरु॑णस्य दश॒मी । \newline
22. द॒श॒ मीन्द्र॒ स्येन्द्र॑स्य दश॒मी द॑श॒ मीन्द्र॑स्य । \newline
23. इन्द्र॑ स्यैकाद॒ श्ये॑काद॒ शीन्द्र॒ स्येन्द्र॑ स्यैकाद॒शी । \newline
24. ए॒का॒द॒शी विश्वे॑षां॒ ॅविश्वे॑षा मेकाद॒ श्ये॑काद॒शी विश्वे॑षाम् । \newline
25. विश्वे॑षाम् दे॒वाना᳚म् दे॒वानां॒ ॅविश्वे॑षां॒ ॅविश्वे॑षाम् दे॒वाना᳚म् । \newline
26. दे॒वाना᳚म् द्वाद॒शी द्वा॑द॒शी दे॒वाना᳚म् दे॒वाना᳚म् द्वाद॒शी । \newline
27. द्वा॒द॒शी द्यावा॑पृथि॒व्योर् द्यावा॑पृथि॒व्योर् द्वा॑द॒शी द्वा॑द॒शी द्यावा॑पृथि॒व्योः । \newline
28. द्यावा॑पृथि॒व्योः पा॒र्श्वम् पा॒र्श्वम् द्यावा॑पृथि॒व्योर् द्यावा॑पृथि॒व्योः पा॒र्श्वम् । \newline
29. द्यावा॑पृथि॒व्योरिति॒ द्यावा᳚ - पृ॒थि॒व्योः । \newline
30. पा॒र्श्वं ॅय॒मस्य॑ य॒मस्य॑ पा॒र्श्वम् पा॒र्श्वं ॅय॒मस्य॑ । \newline
31. य॒मस्य॑ पाटू॒रः पा॑टू॒रो य॒मस्य॑ य॒मस्य॑ पाटू॒रः । \newline
32. पा॒टू॒र इति॑ पाटू॒रः । \newline

\textbf{Ghana Paata } \newline

1. अ॒ग्नेः प॑क्ष॒तिः प॑क्ष॒ति र॒ग्ने र॒ग्नेः प॑क्ष॒तिः सर॑स्वत्यै॒ सर॑स्वत्यै पक्ष॒ति र॒ग्ने र॒ग्नेः प॑क्ष॒तिः सर॑स्वत्यै । \newline
2. प॒क्ष॒तिः सर॑स्वत्यै॒ सर॑स्वत्यै पक्ष॒तिः प॑क्ष॒तिः सर॑स्वत्यै॒ निप॑क्षति॒र् निप॑क्षतिः॒ सर॑स्वत्यै पक्ष॒तिः प॑क्ष॒तिः सर॑स्वत्यै॒ निप॑क्षतिः । \newline
3. सर॑स्वत्यै॒ निप॑क्षति॒र् निप॑क्षतिः॒ सर॑स्वत्यै॒ सर॑स्वत्यै॒ निप॑क्षतिः॒ सोम॑स्य॒ सोम॑स्य॒ निप॑क्षतिः॒ सर॑स्वत्यै॒ सर॑स्वत्यै॒ निप॑क्षतिः॒ सोम॑स्य । \newline
4. निप॑क्षतिः॒ सोम॑स्य॒ सोम॑स्य॒ निप॑क्षति॒र् निप॑क्षतिः॒ सोम॑स्य तृ॒तीया॑ तृ॒तीया॒ सोम॑स्य॒ निप॑क्षति॒र् निप॑क्षतिः॒ सोम॑स्य तृ॒तीया᳚ । \newline
5. निप॑क्षति॒रिति॒ नि - प॒क्ष॒तिः॒ । \newline
6. सोम॑स्य तृ॒तीया॑ तृ॒तीया॒ सोम॑स्य॒ सोम॑स्य तृ॒तीया॒ ऽपा म॒पाम् तृ॒तीया॒ सोम॑स्य॒ सोम॑स्य तृ॒तीया॒ ऽपाम् । \newline
7. तृ॒तीया॒ ऽपा म॒पाम् तृ॒तीया॑ तृ॒तीया॒ ऽपाम् च॑तु॒र्थी च॑तु॒र्थ्य॑पाम् तृ॒तीया॑ तृ॒तीया॒ ऽपाम् च॑तु॒र्थी । \newline
8. अ॒पाम् च॑तु॒र्थी च॑तु॒र्थ्य॑पा म॒पाम् च॑तु॒र्थ्योष॑धीना॒ मोष॑धीनाम् चतु॒र्थ्य॑पा म॒पाम् च॑तु॒र्थ्यो ष॑धीनाम् । \newline
9. च॒तु॒र्थ्योष॑धीना॒ मोष॑धीनाम् चतु॒र्थी च॑तु॒र्थ्योष॑धीनाम् पञ्च॒मी प॑ञ्च॒ म्योष॑धीनाम् चतु॒र्थी च॑तु॒र्थ्योष॑धीनाम् पञ्च॒मी । \newline
10. ओष॑धीनाम् पञ्च॒मी प॑ञ्च॒ म्योष॑धीना॒ मोष॑धीनाम् पञ्च॒मी सं॑ॅवथ्स॒रस्य॑ संॅवथ्स॒रस्य॑ पञ्च॒ म्योष॑धीना॒ मोष॑धीनाम् पञ्च॒मी सं॑ॅवथ्स॒रस्य॑ । \newline
11. प॒ञ्च॒मी सं॑ॅवथ्स॒रस्य॑ संॅवथ्स॒रस्य॑ पञ्च॒मी प॑ञ्च॒मी सं॑ॅवथ्स॒रस्य॑ ष॒ष्ठी ष॒ष्ठी सं॑ॅवथ्स॒रस्य॑ पञ्च॒मी प॑ञ्च॒मी सं॑ॅवथ्स॒रस्य॑ ष॒ष्ठी । \newline
12. सं॒ॅव॒थ्स॒रस्य॑ ष॒ष्ठी ष॒ष्ठी सं॑ॅवथ्स॒रस्य॑ संॅवथ्स॒रस्य॑ ष॒ष्ठी म॒रुता᳚म् म॒रुताꣳ॑ ष॒ष्ठी सं॑ॅवथ्स॒रस्य॑ संॅवथ्स॒रस्य॑ ष॒ष्ठी म॒रुता᳚म् । \newline
13. सं॒ॅव॒थ्स॒रस्येति॑ सं - व॒थ्स॒रस्य॑ । \newline
14. ष॒ष्ठी म॒रुता᳚म् म॒रुताꣳ॑ ष॒ष्ठी ष॒ष्ठी म॒रुताꣳ॑ सप्त॒मी स॑प्त॒मी म॒रुताꣳ॑ ष॒ष्ठी ष॒ष्ठी म॒रुताꣳ॑ सप्त॒मी । \newline
15. म॒रुताꣳ॑ सप्त॒मी स॑प्त॒मी म॒रुता᳚म् म॒रुताꣳ॑ सप्त॒मी बृह॒स्पते॒र् बृह॒स्पतेः᳚ सप्त॒मी म॒रुता᳚म् म॒रुताꣳ॑ सप्त॒मी बृह॒स्पतेः᳚ । \newline
16. स॒प्त॒मी बृह॒स्पते॒र् बृह॒स्पतेः᳚ सप्त॒मी स॑प्त॒मी बृह॒स्पते॑ रष्ट॒म्य॑ष्ट॒मी बृह॒स्पतेः᳚ सप्त॒मी स॑प्त॒मी बृह॒स्पते॑ रष्ट॒मी । \newline
17. बृह॒स्पते॑ रष्ट॒ म्य॑ष्ट॒मी बृह॒स्पते॒र् बृह॒स्पते॑ रष्ट॒मी मि॒त्रस्य॑ मि॒त्रस्या᳚ष्ट॒मी बृह॒स्पते॒र् बृह॒स्पते॑ रष्ट॒मी मि॒त्रस्य॑ । \newline
18. अ॒ष्ट॒मी मि॒त्रस्य॑ मि॒त्रस्या᳚ष्ट॒ म्य॑ष्ट॒मी मि॒त्रस्य॑ नव॒मी न॑व॒मी मि॒त्रस्या᳚ष्ट॒ म्य॑ष्ट॒मी मि॒त्रस्य॑ नव॒मी । \newline
19. मि॒त्रस्य॑ नव॒मी न॑व॒मी मि॒त्रस्य॑ मि॒त्रस्य॑ नव॒मी वरु॑णस्य॒ वरु॑णस्य नव॒मी मि॒त्रस्य॑ मि॒त्रस्य॑ नव॒मी वरु॑णस्य । \newline
20. न॒व॒मी वरु॑णस्य॒ वरु॑णस्य नव॒मी न॑व॒मी वरु॑णस्य दश॒मी द॑श॒मी वरु॑णस्य नव॒मी न॑व॒मी वरु॑णस्य दश॒मी । \newline
21. वरु॑णस्य दश॒मी द॑श॒मी वरु॑णस्य॒ वरु॑णस्य दश॒मीन्द्र॒ स्येन्द्र॑स्य दश॒मी वरु॑णस्य॒ वरु॑णस्य दश॒मीन्द्र॑स्य । \newline
22. द॒श॒मीन्द्र॒ स्येन्द्र॑स्य दश॒मी द॑श॒मीन्द्र॑ स्यैकाद॒ श्ये॑काद॒ शीन्द्र॑स्य दश॒मी द॑श॒मीन्द्र॑स्यै काद॒शी । \newline
23. इन्द्र॑स्यैकाद॒ श्ये॑काद॒ शीन्द्र॒ स्येन्द्र॑ स्यैकाद॒शी विश्वे॑षां॒ ॅविश्वे॑षा मेकाद॒ शीन्द्र॒ स्येन्द्र॑
स्यैकाद॒शी विश्वे॑षाम् । \newline
24. ए॒का॒द॒शी विश्वे॑षां॒ ॅविश्वे॑षा मेकाद॒ श्ये॑काद॒शी विश्वे॑षाम् दे॒वाना᳚म् दे॒वानां॒ ॅविश्वे॑षा मेकाद॒ श्ये॑काद॒शी विश्वे॑षाम् दे॒वाना᳚म् । \newline
25. विश्वे॑षाम् दे॒वाना᳚म् दे॒वानां॒ ॅविश्वे॑षां॒ ॅविश्वे॑षाम् दे॒वाना᳚म् द्वाद॒शी द्वा॑द॒शी दे॒वानां॒ ॅविश्वे॑षां॒ ॅविश्वे॑षाम् दे॒वाना᳚म् द्वाद॒शी । \newline
26. दे॒वाना᳚म् द्वाद॒शी द्वा॑द॒शी दे॒वाना᳚म् दे॒वाना᳚म् द्वाद॒शी द्यावा॑पृथि॒व्योर् द्यावा॑पृथि॒व्योर् द्वा॑द॒शी दे॒वाना᳚म् दे॒वाना᳚म् द्वाद॒शी द्यावा॑पृथि॒व्योः । \newline
27. द्वा॒द॒शी द्यावा॑पृथि॒व्योर् द्यावा॑पृथि॒व्योर् द्वा॑द॒शी द्वा॑द॒शी द्यावा॑पृथि॒व्योः पा॒र्श्वम् पा॒र्श्वम् द्यावा॑पृथि॒व्योर् द्वा॑द॒शी द्वा॑द॒शी द्यावा॑पृथि॒व्योः पा॒र्श्वम् । \newline
28. द्यावा॑पृथि॒व्योः पा॒र्श्वम् पा॒र्श्वम् द्यावा॑पृथि॒व्योर् द्यावा॑पृथि॒व्योः पा॒र्श्वं ॅय॒मस्य॑ य॒मस्य॑ पा॒र्श्वम् द्यावा॑पृथि॒व्योर् द्यावा॑पृथि॒व्योः पा॒र्श्वं ॅय॒मस्य॑ । \newline
29. द्यावा॑पृथि॒व्योरिति॒ द्यावा᳚ - पृ॒थि॒व्योः । \newline
30. पा॒र्श्वं ॅय॒मस्य॑ य॒मस्य॑ पा॒र्श्वम् पा॒र्श्वं ॅय॒मस्य॑ पाटू॒रः पा॑टू॒रो य॒मस्य॑ पा॒र्श्वम् पा॒र्श्वं ॅय॒मस्य॑ पाटू॒रः । \newline
31. य॒मस्य॑ पाटू॒रः पा॑टू॒रो य॒मस्य॑ य॒मस्य॑ पाटू॒रः । \newline
32. पा॒टू॒र इति॑ पाटू॒रः । \newline
\pagebreak
\markright{ TS 5.7.22.1  \hfill https://www.vedavms.in \hfill}

\section{ TS 5.7.22.1 }

\textbf{TS 5.7.22.1 } \newline
\textbf{Samhita Paata} \newline

वा॒योः प॑क्ष॒तिः सर॑स्वतो॒ निप॑क्षति-श्च॒न्द्रम॑स-स्तृ॒तीया॒ नक्ष॑त्राणां चतु॒र्थी स॑वि॒तुः प॑ञ्च॒मी रु॒द्रस्य॑ ष॒ष्ठी स॒र्पाणाꣳ॑ सप्त॒म्य॑र्य॒म्णो᳚ऽष्ट॒मी त्वष्टु॑र्नव॒मी धा॒तुर्द॑श॒मीन्द्रा॒ण्या ए॑काद॒श्यदि॑त्यै द्वाद॒शी द्यावा॑पृथि॒व्योः पा॒र्श्वं ॅय॒म्यै॑ पाटू॒रः ॥ \newline

\textbf{Pada Paata} \newline

वा॒योः । प॒क्ष॒तिः । सर॑स्वतः । निप॑क्षति॒रिति॒ नि - प॒क्ष॒तिः॒ । च॒न्द्रम॑सः । तृ॒तीया᳚ । नक्ष॑त्राणाम् । च॒तु॒र्थी । स॒वि॒तुः । प॒ञ्च॒मी । रु॒द्रस्य॑ । ष॒ष्ठी । स॒र्पाणा᳚म् । स॒प्त॒मी । अ॒र्य॒म्णः । अ॒ष्ट॒मी । त्वष्टुः॑ । न॒व॒मी । धा॒तुः । द॒श॒मी । इ॒न्द्रा॒ण्याः । ए॒का॒द॒शी । अदि॑त्यै । द्वा॒द॒शी । द्यावा॑पृथि॒व्योरिति॒ द्यावा᳚ - पृ॒थि॒व्योः । पा॒र्श्वम् । य॒म्यै᳚ । पा॒टू॒रः ॥  \newline


\textbf{Krama Paata} \newline

वा॒योः प॑क्ष॒तिः । प॒क्ष॒तिः सर॑स्वतः । सर॑स्वतो॒ निप॑क्षतिः । निप॑क्षतिश्च॒न्द्रम॑सः । निप॑क्षति॒रिति॒ नि - प॒क्ष॒तिः॒ । च॒न्द्रम॑सस्तृ॒तीया᳚ । तृ॒तीया॒ नक्ष॑त्राणाम् । नक्ष॑त्राणाम् चतु॒र्त्थी । च॒तु॒र्त्थी स॑वि॒तुः । स॒वि॒तुः प॑ञ्च॒मी । प॒ञ्च॒मी रु॒द्रस्य॑ । रु॒द्रस्य॑ ष॒ष्ठी । ष॒ष्ठी स॒र्पाणा᳚म् । स॒र्पाणाꣳ॑ सप्त॒मी । स॒प्त॒म्य॑र्य॒म्णः । अ॒र्य॒म्णो᳚ऽष्ट॒मी । अ॒ष्ट॒मी त्वष्टुः॑ । त्वष्टु॑र् नव॒मी । न॒व॒मी धा॒तुः । धा॒तुर् द॑श॒मी । द॒श॒मीन्द्रा॒ण्याः । इ॒न्द्रा॒ण्या ए॑काद॒शी । ए॒का॒द॒श्यदि॑त्यै । अदि॑त्यै द्वाद॒शी । द्वा॒द॒शी द्यावा॑पृथि॒व्योः । द्यावा॑पृथि॒व्योः पा॒र्श्वम् । द्यावा॑पृथी॒व्योरिति॒ द्यावा᳚ - पृ॒थि॒व्योः । पा॒र्श्वम् ॅय॒म्यै᳚ । य॒म्यै॑ पाटू॒रः । पा॒टू॒र इति॑ पाटू॒रः । \newline

\textbf{Jatai Paata} \newline

1. वा॒योः प॑क्ष॒तिः प॑क्ष॒तिर् वा॒योर् वा॒योः प॑क्ष॒तिः । \newline
2. प॒क्ष॒तिः सर॑स्वतः॒ सर॑स्वतः पक्ष॒तिः प॑क्ष॒तिः सर॑स्वतः । \newline
3. सर॑स्वतो॒ निप॑क्षति॒र् निप॑क्षतिः॒ सर॑स्वतः॒ सर॑स्वतो॒ निप॑क्षतिः । \newline
4. निप॑क्षति श्च॒न्द्रम॑स श्च॒न्द्रम॑सो॒ निप॑क्षति॒र् निप॑क्षति श्च॒न्द्रम॑सः । \newline
5. निप॑क्षति॒रिति॒ नि - प॒क्ष॒तिः॒ । \newline
6. च॒न्द्रम॑स स्तृ॒तीया॑ तृ॒तीया॑ च॒न्द्रम॑स श्च॒न्द्रम॑स स्तृ॒तीया᳚ । \newline
7. तृ॒तीया॒ नक्ष॑त्राणा॒म् नक्ष॑त्राणाम् तृ॒तीया॑ तृ॒तीया॒ नक्ष॑त्राणाम् । \newline
8. नक्ष॑त्राणाम् चतु॒र्थी च॑तु॒र्थी नक्ष॑त्राणा॒म् नक्ष॑त्राणाम् चतु॒र्थी । \newline
9. च॒तु॒र्थी स॑वि॒तुः स॑वि॒तु श्च॑तु॒र्थी च॑तु॒र्थी स॑वि॒तुः । \newline
10. स॒वि॒तुः प॑ञ्च॒मी प॑ञ्च॒मी स॑वि॒तुः स॑वि॒तुः प॑ञ्च॒मी । \newline
11. प॒ञ्च॒मी रु॒द्रस्य॑ रु॒द्रस्य॑ पञ्च॒मी प॑ञ्च॒मी रु॒द्रस्य॑ । \newline
12. रु॒द्रस्य॑ ष॒ष्ठी ष॒ष्ठी रु॒द्रस्य॑ रु॒द्रस्य॑ ष॒ष्ठी । \newline
13. ष॒ष्ठी स॒र्पाणाꣳ॑ स॒र्पाणाꣳ॑ ष॒ष्ठी ष॒ष्ठी स॒र्पाणा᳚म् । \newline
14. स॒र्पाणाꣳ॑ सप्त॒मी स॑प्त॒मी स॒र्पाणाꣳ॑ स॒र्पाणाꣳ॑ सप्त॒मी । \newline
15. स॒प्त॒ म्य॑र्य॒म्णो᳚ ऽर्य॒म्णः स॑प्त॒मी स॑प्त॒ म्य॑र्य॒म्णः । \newline
16. अ॒र्य॒म्णो᳚ ऽष्ट॒ म्य॑ष्ट॒ म्य॑र्य॒म्णो᳚ ऽर्य॒म्णो᳚ ऽष्ट॒मी । \newline
17. अ॒ष्ट॒मी त्वष्टु॒ स्त्वष्टु॑ रष्ट॒ म्य॑ष्ट॒मी त्वष्टुः॑ । \newline
18. त्वष्टु॑र् नव॒मी न॑व॒मी त्वष्टु॒ स्त्वष्टु॑र् नव॒मी । \newline
19. न॒व॒मी धा॒तुर् धा॒तुर् न॑व॒मी न॑व॒मी धा॒तुः । \newline
20. धा॒तुर् द॑श॒मी द॑श॒मी धा॒तुर् धा॒तुर् द॑श॒मी । \newline
21. द॒श॒ मीन्द्रा॒ण्या इ॑न्द्रा॒ण्या द॑श॒मी द॑श॒ मीन्द्रा॒ण्याः । \newline
22. इ॒न्द्रा॒ण्या ए॑काद॒ श्ये॑काद॒ शीन्द्रा॒ण्या इ॑न्द्रा॒ण्या ए॑काद॒शी । \newline
23. ए॒का॒द॒ श्यदि॑त्या॒ अदि॑त्या एकाद॒ श्ये॑काद॒ श्यदि॑त्यै । \newline
24. अदि॑त्यै द्वाद॒शी द्वा॑द॒ श्यदि॑त्या॒ अदि॑त्यै द्वाद॒शी । \newline
25. द्वा॒द॒शी द्यावा॑पृथि॒व्योर् द्यावा॑पृथि॒व्योर् द्वा॑द॒शी द्वा॑द॒शी द्यावा॑पृथि॒व्योः । \newline
26. द्यावा॑पृथि॒व्योः पा॒र्श्वम् पा॒र्श्वम् द्यावा॑पृथि॒व्योर् द्यावा॑पृथि॒व्योः पा॒र्श्वम् । \newline
27. द्यावा॑पृथि॒व्योरिति॒ द्यावा᳚ - पृ॒थि॒व्योः । \newline
28. पा॒र्श्वं ॅय॒म्यै॑ य॒म्यै॑ पा॒र्श्वम् पा॒र्श्वं ॅय॒म्यै᳚ । \newline
29. य॒म्यै॑ पाटू॒रः पा॑टू॒रो य॒म्यै॑ य॒म्यै॑ पाटू॒रः । \newline
30. पा॒टू॒र इति॑ पाटू॒रः । \newline

\textbf{Ghana Paata } \newline

1. वा॒योः प॑क्ष॒तिः प॑क्ष॒तिर् वा॒योर् वा॒योः प॑क्ष॒तिः सर॑स्वतः॒ सर॑स्वतः पक्ष॒तिर् वा॒योर् वा॒योः प॑क्ष॒तिः सर॑स्वतः । \newline
2. प॒क्ष॒तिः सर॑स्वतः॒ सर॑स्वतः पक्ष॒तिः प॑क्ष॒तिः सर॑स्वतो॒ निप॑क्षति॒र् निप॑क्षतिः॒ सर॑स्वतः पक्ष॒तिः प॑क्ष॒तिः सर॑स्वतो॒ निप॑क्षतिः । \newline
3. सर॑स्वतो॒ निप॑क्षति॒र् निप॑क्षतिः॒ सर॑स्वतः॒ सर॑स्वतो॒ निप॑क्षति श्च॒न्द्रम॑स श्च॒न्द्रम॑सो॒ निप॑क्षतिः॒ सर॑स्वतः॒ सर॑स्वतो॒ निप॑क्षति श्च॒न्द्रम॑सः । \newline
4. निप॑क्षति श्च॒न्द्रम॑स श्च॒न्द्रम॑सो॒ निप॑क्षति॒र् निप॑क्षति श्च॒न्द्रम॑स स्तृ॒तीया॑ तृ॒तीया॑ च॒न्द्रम॑सो॒ निप॑क्षति॒र् निप॑क्षति श्च॒न्द्रम॑स स्तृ॒तीया᳚ । \newline
5. निप॑क्षति॒रिति॒ नि - प॒क्ष॒तिः॒ । \newline
6. च॒न्द्रम॑स स्तृ॒तीया॑ तृ॒तीया॑ च॒न्द्रम॑स श्च॒न्द्रम॑स स्तृ॒तीया॒ नक्ष॑त्राणा॒म् नक्ष॑त्राणाम् तृ॒तीया॑ च॒न्द्रम॑स श्च॒न्द्रम॑स स्तृ॒तीया॒ नक्ष॑त्राणाम् । \newline
7. तृ॒तीया॒ नक्ष॑त्राणा॒म् नक्ष॑त्राणाम् तृ॒तीया॑ तृ॒तीया॒ नक्ष॑त्राणाम् चतु॒र्थी च॑तु॒र्थी नक्ष॑त्राणाम् तृ॒तीया॑ तृ॒तीया॒ नक्ष॑त्राणाम् चतु॒र्थी । \newline
8. नक्ष॑त्राणाम् चतु॒र्थी च॑तु॒र्थी नक्ष॑त्राणा॒म् नक्ष॑त्राणाम् चतु॒र्थी स॑वि॒तुः स॑वि॒तु श्च॑तु॒र्थी नक्ष॑त्राणा॒म् नक्ष॑त्राणाम् चतु॒र्थी स॑वि॒तुः । \newline
9. च॒तु॒र्थी स॑वि॒तुः स॑वि॒तु श्च॑तु॒र्थी च॑तु॒र्थी स॑वि॒तुः प॑ञ्च॒मी प॑ञ्च॒मी स॑वि॒तु श्च॑तु॒र्थी च॑तु॒र्थी स॑वि॒तुः प॑ञ्च॒मी । \newline
10. स॒वि॒तुः प॑ञ्च॒मी प॑ञ्च॒मी स॑वि॒तुः स॑वि॒तुः प॑ञ्च॒मी रु॒द्रस्य॑ रु॒द्रस्य॑ पञ्च॒मी स॑वि॒तुः स॑वि॒तुः प॑ञ्च॒मी रु॒द्रस्य॑ । \newline
11. प॒ञ्च॒मी रु॒द्रस्य॑ रु॒द्रस्य॑ पञ्च॒मी प॑ञ्च॒मी रु॒द्रस्य॑ ष॒ष्ठी ष॒ष्ठी रु॒द्रस्य॑ पञ्च॒मी प॑ञ्च॒मी रु॒द्रस्य॑ ष॒ष्ठी । \newline
12. रु॒द्रस्य॑ ष॒ष्ठी ष॒ष्ठी रु॒द्रस्य॑ रु॒द्रस्य॑ ष॒ष्ठी स॒र्पाणाꣳ॑ स॒र्पाणाꣳ॑ ष॒ष्ठी रु॒द्रस्य॑ रु॒द्रस्य॑ ष॒ष्ठी स॒र्पाणा᳚म् । \newline
13. ष॒ष्ठी स॒र्पाणाꣳ॑ स॒र्पाणाꣳ॑ ष॒ष्ठी ष॒ष्ठी स॒र्पाणाꣳ॑ सप्त॒मी स॑प्त॒मी स॒र्पाणाꣳ॑ ष॒ष्ठी ष॒ष्ठी स॒र्पाणाꣳ॑ सप्त॒मी । \newline
14. स॒र्पाणाꣳ॑ सप्त॒मी स॑प्त॒मी स॒र्पाणाꣳ॑ स॒र्पाणाꣳ॑ सप्त॒म्य॑ र्य॒म्णो᳚ ऽर्य॒म्णः स॑प्त॒मी स॒र्पाणाꣳ॑ स॒र्पाणाꣳ॑ सप्त॒म्य॑ र्य॒म्णः । \newline
15. स॒प्त॒ म्य॑र्य॒म्णो᳚ ऽर्य॒म्णः स॑प्त॒मी स॑प्त॒म्य॑ र्य॒म्णो᳚ ऽष्ट॒ म्य॑ष्ट॒ म्य॑र्य॒म्णः स॑प्त॒मी स॑प्त॒म्य॑ र्य॒म्णो᳚ ऽष्ट॒मी । \newline
16. अ॒र्य॒म्णो᳚ ऽष्ट॒ म्य॑ष्ट॒ म्य॑र्य॒म्णो᳚ ऽर्य॒म्णो᳚ ऽष्ट॒मी त्वष्टु॒ स्त्वष्टु॑ रष्ट॒ म्य॑र्य॒म्णो᳚ ऽर्य॒म्णो᳚ ऽष्ट॒मी त्वष्टुः॑ । \newline
17. अ॒ष्ट॒मी त्वष्टु॒ स्त्वष्टु॑ रष्ट॒ म्य॑ष्ट॒मी त्वष्टु॑र् नव॒मी न॑व॒मी त्वष्टु॑ रष्ट॒ म्य॑ष्ट॒मी त्वष्टु॑र् नव॒मी । \newline
18. त्वष्टु॑र् नव॒मी न॑व॒मी त्वष्टु॒ स्त्वष्टु॑र् नव॒मी धा॒तुर् धा॒तुर् न॑व॒मी त्वष्टु॒ स्त्वष्टु॑र् नव॒मी धा॒तुः । \newline
19. न॒व॒मी धा॒तुर् धा॒तुर् न॑व॒मी न॑व॒मी धा॒तुर् द॑श॒मी द॑श॒मी धा॒तुर् न॑व॒मी न॑व॒मी धा॒तुर् द॑श॒मी । \newline
20. धा॒तुर् द॑श॒मी द॑श॒मी धा॒तुर् धा॒तुर् द॑श॒ मीन्द्रा॒ण्या इ॑न्द्रा॒ण्या द॑श॒मी धा॒तुर् धा॒तुर् द॑श॒ मीन्द्रा॒ण्याः । \newline
21. द॒श॒ मीन्द्रा॒ण्या इ॑न्द्रा॒ण्या द॑श॒मी द॑श॒ मीन्द्रा॒ण्या ए॑काद॒ श्ये॑काद॒शीन्द्रा॒ण्या द॑श॒मी द॑श॒ मीन्द्रा॒ण्या ए॑काद॒शी । \newline
22. इ॒न्द्रा॒ण्या ए॑काद॒ श्ये॑काद॒ शीन्द्रा॒ण्या इ॑न्द्रा॒ण्या ए॑काद॒श्य दि॑त्या॒ अदि॑त्या एकाद॒ शीन्द्रा॒ण्या इ॑न्द्रा॒ण्या ए॑काद॒ श्यदि॑त्यै । \newline
23. ए॒का॒द॒ श्यदि॑त्या॒ अदि॑त्या एकाद॒ श्ये॑काद॒ श्यदि॑त्यै द्वाद॒शी द्वा॑द॒ श्यदि॑त्या एकाद॒ श्ये॑काद॒ श्यदि॑त्यै द्वाद॒शी । \newline
24. अदि॑त्यै द्वाद॒शी द्वा॑द॒ श्यदि॑त्या॒ अदि॑त्यै द्वाद॒शी द्यावा॑पृथि॒व्योर् द्यावा॑पृथि॒व्योर् द्वा॑द॒ श्यदि॑त्या॒ अदि॑त्यै द्वाद॒शी द्यावा॑पृथि॒व्योः । \newline
25. द्वा॒द॒शी द्यावा॑पृथि॒व्योर् द्यावा॑पृथि॒व्योर् द्वा॑द॒शी द्वा॑द॒शी द्यावा॑पृथि॒व्योः पा॒र्श्वम् पा॒र्श्वम् द्यावा॑पृथि॒व्योर् द्वा॑द॒शी द्वा॑द॒शी द्यावा॑पृथि॒व्योः पा॒र्श्वम् । \newline
26. द्यावा॑पृथि॒व्योः पा॒र्श्वम् पा॒र्श्वम् द्यावा॑पृथि॒व्योर् द्यावा॑पृथि॒व्योः पा॒र्श्वं ॅय॒म्यै॑ य॒म्यै॑ पा॒र्श्वम् द्यावा॑पृथि॒व्योर् द्यावा॑पृथि॒व्योः पा॒र्श्वं ॅय॒म्यै᳚ । \newline
27. द्यावा॑पृथि॒व्योरिति॒ द्यावा᳚ - पृ॒थि॒व्योः । \newline
28. पा॒र्श्वं ॅय॒म्यै॑ य॒म्यै॑ पा॒र्श्वम् पा॒र्श्वं ॅय॒म्यै॑ पाटू॒रः पा॑टू॒रो य॒म्यै॑ पा॒र्श्वम् पा॒र्श्वं ॅय॒म्यै॑ पाटू॒रः । \newline
29. य॒म्यै॑ पाटू॒रः पा॑टू॒रो य॒म्यै॑ य॒म्यै॑ पाटू॒रः । \newline
30. पा॒टू॒र इति॑ पाटू॒रः । \newline
\pagebreak
\markright{ TS 5.7.23.1  \hfill https://www.vedavms.in \hfill}

\section{ TS 5.7.23.1 }

\textbf{TS 5.7.23.1 } \newline
\textbf{Samhita Paata} \newline

पन्था॑मनू॒वृग्भ्याꣳ॒॒ संत॑तिꣳ स्नाव॒न्या᳚भ्याꣳ॒॒ शुका᳚न् पि॒त्तेन॑ हरि॒माणं॑ ॅय॒क्ना हली᳚क्ष्णान् पापवा॒तेन॑ कू॒श्माञ्छक॑भिः शव॒र्तानूव॑द्ध्येन॒ शुनो॑ वि॒शस॑नेन स॒र्पान् ॅलो॑हितग॒न्धेन॒ वयाꣳ॑सि पक्वग॒न्धेन॑ पि॒पीलि॑काः प्रशा॒देन॑ ॥ \newline

\textbf{Pada Paata} \newline

पन्था᳚म् । अ॒नू॒वृग्भ्या॒मित्य॑नू॒वृक् - भ्या॒म् । संत॑ति॒मिति॒ सं-त॒ति॒म् । स्ना॒व॒न्या᳚भ्याम् । शुकान्॑ । पि॒त्तेन॑ । ह॒रि॒माण᳚म् । य॒क्ना । हली᳚क्ष्णान् । पा॒प॒वा॒तेनेति॑ पाप - वा॒तेन॑ । कू॒श्मान् । शक॑भि॒रिति॒ शक॑ - भिः॒ । श॒व॒र्तान् । ऊव॑द्ध्येन । शुनः॑ । वि॒शस॑ने॒नेति॑ वि - शस॑नेन । स॒र्पान् । लो॒हि॒त॒ग॒न्धेनेति॑ लोहित - ग॒न्धेन॑ । वयाꣳ॑सि । प॒क्व॒ग॒न्धेनेति॑ पक्व - ग॒न्धेन॑ । पि॒पीलि॑काः । प्र॒शा॒देनेति॑ प्र - शा॒देन॑ ॥  \newline


\textbf{Krama Paata} \newline

पन्था॑मनू॒वृग्भ्या᳚म् । अ॒नू॒वृग्भ्याꣳ॒॒ सन्त॑तिम् । अ॒नू॒वृग्भ्या॒मित्य॑नू॒वृक् - भ्या॒म् । सन्त॑तिꣳ स्नाव॒न्या᳚भ्याम् । सन्त॑ति॒मिति॒ सम् - त॒ति॒म् । स्ना॒व॒न्या᳚भ्याꣳ॒॒ शुकान्॑ । शुका᳚न् पि॒त्तेन॑ । पि॒त्तेन॑ हरि॒माण᳚म् । ह॒रि॒माण॑म् ॅय॒क्ना । य॒क्ना हली᳚क्ष्णान् । हली᳚क्ष्णान् पापवा॒तेन॑ । पा॒प॒वा॒तेन॑ कू॒श्मान् । पा॒प॒वा॒तेनेति॑ पाप - वा॒तेन॑ । कू॒श्माञ्छक॑भिः । शक॑भिः शव॒र्तान् । शक॑भि॒रिति॒ शक॑ - भिः॒ । श॒व॒र्तानूव॑द्ध्येन । ऊव॑द्ध्येन॒ शुनः॑ । शुनो॑ वि॒शस॑नेन । वि॒शस॑नेन स॒र्पान् । वि॒शस॑ने॒नेति॑ वि - शस॑नेन । स॒र्पान् ॅलो॑हितग॒न्धेन॑ । लो॒हि॒त॒ग॒न्धेन॒ वयाꣳ॑सि । लो॒हि॒त॒ग॒न्धेनेति॑ लोहित - ग॒न्धेन॑ । वयाꣳ॑सि पक्वग॒न्धेन॑ । प॒क्व॒ग॒न्धेन॑ पि॒पीलि॑काः । प॒क्व॒ग॒न्धेनेति॑ पक्व - ग॒न्धेन॑ । पि॒पीलि॑काः प्रशा॒देन॑ । प्र॒शा॒देनेति॑ प्र - शा॒देन॑ । \newline

\textbf{Jatai Paata} \newline

1. पन्था॑ मनू॒वृग्भ्या॑ मनू॒वृग्भ्या॒म् पन्था॒म् पन्था॑ मनू॒वृग्भ्या᳚म् । \newline
2. अ॒नू॒वृग्भ्याꣳ॒॒ सन्त॑तिꣳ॒॒ सन्त॑ति मनू॒वृग्भ्या॑ मनू॒वृग्भ्याꣳ॒॒ सन्त॑तिम् । \newline
3. अ॒नू॒वृग्भ्या॒मित्य॑नू॒वृक् - भ्या॒म् । \newline
4. सन्त॑तिꣳ स्नाव॒न्या᳚भ्याꣳ स्नाव॒न्या᳚भ्याꣳ॒॒ सन्त॑तिꣳ॒॒ सन्त॑तिꣳ स्नाव॒न्या᳚भ्याम् । \newline
5. सन्त॑ति॒मिति॒ सं - त॒ति॒म् । \newline
6. स्ना॒व॒न्या᳚भ्याꣳ॒॒ शुका॒ञ् छुका᳚न् थ्स्नाव॒न्या᳚भ्याꣳ स्नाव॒न्या᳚भ्याꣳ॒॒ शुकान्॑ । \newline
7. शुका᳚न् पि॒त्तेन॑ पि॒त्तेन॒ शुका॒ञ् छुका᳚न् पि॒त्तेन॑ । \newline
8. पि॒त्तेन॑ हरि॒माणꣳ॑ हरि॒माण॑म् पि॒त्तेन॑ पि॒त्तेन॑ हरि॒माण᳚म् । \newline
9. ह॒रि॒माणं॑ ॅय॒क्ना य॒क्ना ह॑रि॒माणꣳ॑ हरि॒माणं॑ ॅय॒क्ना । \newline
10. य॒क्ना हली᳚क्ष्णा॒न्॒. हली᳚क्ष्णान्. य॒क्ना य॒क्ना हली᳚क्ष्णान् । \newline
11. हली᳚क्ष्णान् पापवा॒तेन॑ पापवा॒तेन॒ हली᳚क्ष्णा॒न्॒. हली᳚क्ष्णान् पापवा॒तेन॑ । \newline
12. पा॒प॒वा॒तेन॑ कू॒श्मान् कू॒श्मान् पा॑पवा॒तेन॑ पापवा॒तेन॑ कू॒श्मान् । \newline
13. पा॒प॒वा॒तेनेति॑ पाप - वा॒तेन॑ । \newline
14. कू॒श्माञ् छक॑भिः॒ शक॑भिः कू॒श्मान् कू॒श्माञ् छक॑भिः । \newline
15. शक॑भिः शव॒र्ताञ् छ॑व॒र्ताञ् छक॑भिः॒ शक॑भिः शव॒र्तान् । \newline
16. शक॑भि॒रिति॒ शक॑ - भिः॒ । \newline
17. श॒व॒र्ता नूव॑द्ध्ये॒नो व॑द्ध्येन शव॒र्ताञ् छ॑व॒र्ता नूव॑द्ध्येन । \newline
18. ऊव॑द्ध्येन॒ शुनः॒ शुन॒ ऊव॑द्ध्ये॒नो व॑द्ध्येन॒ शुनः॑ । \newline
19. शुनो॑ वि॒शस॑नेन वि॒शस॑नेन॒ शुनः॒ शुनो॑ वि॒शस॑नेन । \newline
20. वि॒शस॑नेन स॒र्पान् थ्स॒र्पान्. वि॒शस॑नेन वि॒शस॑नेन स॒र्पान् । \newline
21. वि॒शस॑ने॒नेति॑ वि - शस॑नेन । \newline
22. स॒र्पान् ॅलो॑हितग॒न्धेन॑ लोहितग॒न्धेन॑ स॒र्पान् थ्स॒र्पान् ॅलो॑हितग॒न्धेन॑ । \newline
23. लो॒हि॒त॒ग॒न्धेन॒ वयाꣳ॑सि॒ वयाꣳ॑सि लोहितग॒न्धेन॑ लोहितग॒न्धेन॒ वयाꣳ॑सि । \newline
24. लो॒हि॒त॒ग॒न्धेनेति॑ लोहित - ग॒न्धेन॑ । \newline
25. वयाꣳ॑सि पक्वग॒न्धेन॑ पक्वग॒न्धेन॒ वयाꣳ॑सि॒ वयाꣳ॑सि पक्वग॒न्धेन॑ । \newline
26. प॒क्व॒ग॒न्धेन॑ पि॒पीलि॑काः पि॒पीलि॑काः पक्वग॒न्धेन॑ पक्वग॒न्धेन॑ पि॒पीलि॑काः । \newline
27. प॒क्व॒ग॒न्धेनेति॑ पक्व - ग॒न्धेन॑ । \newline
28. पि॒पीलि॑काः प्रशा॒देन॑ प्रशा॒देन॑ पि॒पीलि॑काः पि॒पीलि॑काः प्रशा॒देन॑ । \newline
29. प्र॒शा॒देनेति॑ प्र - शा॒देन॑ । \newline

\textbf{Ghana Paata } \newline

1. पन्था॑ मनू॒वृग्भ्या॑ मनू॒वृग्भ्या॒म् पन्था॒म् पन्था॑ मनू॒वृग्भ्याꣳ॒॒ सन्त॑तिꣳ॒॒ सन्त॑ति मनू॒वृग्भ्या॒म् पन्था॒म् पन्था॑ मनू॒वृग्भ्याꣳ॒॒ सन्त॑तिम् । \newline
2. अ॒नू॒वृग्भ्याꣳ॒॒ सन्त॑तिꣳ॒॒ सन्त॑ति मनू॒वृग्भ्या॑ मनू॒वृग्भ्याꣳ॒॒ सन्त॑तिꣳ स्नाव॒न्या᳚भ्याꣳ स्नाव॒न्या᳚भ्याꣳ॒॒ सन्त॑ति मनू॒वृग्भ्या॑ मनू॒वृग्भ्याꣳ॒॒ सन्त॑तिꣳ स्नाव॒न्या᳚भ्याम् । \newline
3. अ॒नू॒वृग्भ्या॒मित्य॑नू॒वृक् - भ्या॒म् । \newline
4. सन्त॑तिꣳ स्नाव॒न्या᳚भ्याꣳ स्नाव॒न्या᳚भ्याꣳ॒॒ सन्त॑तिꣳ॒॒ सन्त॑तिꣳ स्नाव॒न्या᳚भ्याꣳ॒॒ शुका॒ञ् 
छुका᳚न् थ्स्नाव॒न्या᳚भ्याꣳ॒॒ सन्त॑तिꣳ॒॒ सन्त॑तिꣳ स्नाव॒न्या᳚भ्याꣳ॒॒ शुकान्॑ । \newline
5. सन्त॑ति॒मिति॒ सं - त॒ति॒म् । \newline
6. स्ना॒व॒न्या᳚भ्याꣳ॒॒ शुका॒ञ् छुका᳚न् थ्स्नाव॒न्या᳚भ्याꣳ स्नाव॒न्या᳚भ्याꣳ॒॒ शुका᳚न् पि॒त्तेन॑ पि॒त्तेन॒ शुका᳚न् थ्स्नाव॒न्या᳚भ्याꣳ स्नाव॒न्या᳚भ्याꣳ॒॒ शुका᳚न् पि॒त्तेन॑ । \newline
7. शुका᳚न् पि॒त्तेन॑ पि॒त्तेन॒ शुका॒ञ् छुका᳚न् पि॒त्तेन॑ हरि॒माणꣳ॑ हरि॒माण॑म् पि॒त्तेन॒ शुका॒ञ् छुका᳚न् पि॒त्तेन॑ हरि॒माण᳚म् । \newline
8. पि॒त्तेन॑ हरि॒माणꣳ॑ हरि॒माण॑म् पि॒त्तेन॑ पि॒त्तेन॑ हरि॒माणं॑ ॅय॒क्ना य॒क्ना ह॑रि॒माण॑म् पि॒त्तेन॑ पि॒त्तेन॑ हरि॒माणं॑ ॅय॒क्ना । \newline
9. ह॒रि॒माणं॑ ॅय॒क्ना य॒क्ना ह॑रि॒माणꣳ॑ हरि॒माणं॑ ॅय॒क्ना हली᳚क्ष्णा॒न्॒. हली᳚क्ष्णान्. य॒क्ना ह॑रि॒माणꣳ॑ हरि॒माणं॑ ॅय॒क्ना हली᳚क्ष्णान् । \newline
10. य॒क्ना हली᳚क्ष्णा॒न्॒. हली᳚क्ष्णान्. य॒क्ना य॒क्ना हली᳚क्ष्णान् पापवा॒तेन॑ पापवा॒तेन॒ हली᳚क्ष्णान्. य॒क्ना य॒क्ना हली᳚क्ष्णान् पापवा॒तेन॑ । \newline
11. हली᳚क्ष्णान् पापवा॒तेन॑ पापवा॒तेन॒ हली᳚क्ष्णा॒न्॒. हली᳚क्ष्णान् पापवा॒तेन॑ कू॒श्मान् कू॒श्मान् पा॑पवा॒तेन॒ हली᳚क्ष्णा॒न्॒. हली᳚क्ष्णान् पापवा॒तेन॑ कू॒श्मान् । \newline
12. पा॒प॒वा॒तेन॑ कू॒श्मान् कू॒श्मान् पा॑पवा॒तेन॑ पापवा॒तेन॑ कू॒श्माञ् छक॑भिः॒ शक॑भिः कू॒श्मान् पा॑पवा॒तेन॑ पापवा॒तेन॑ कू॒श्माञ् छक॑भिः । \newline
13. पा॒प॒वा॒तेनेति॑ पाप - वा॒तेन॑ । \newline
14. कू॒श्माञ् छक॑भिः॒ शक॑भिः कू॒श्मान् कू॒श्माञ् छक॑भिः शव॒र्ताञ् छ॑व॒र्ताञ् छक॑भिः कू॒श्मान् कू॒श्माञ् छक॑भिः शव॒र्तान् । \newline
15. शक॑भिः शव॒र्ताञ् छ॑व॒र्ताञ् छक॑भिः॒ शक॑भिः शव॒र्ता नूव॑द्ध्ये॒ नोव॑द्ध्येन शव॒र्ताञ् छक॑भिः॒ शक॑भिः शव॒र्ता नूव॑द्ध्येन । \newline
16. शक॑भि॒रिति॒ शक॑ - भिः॒ । \newline
17. श॒व॒र्ता नूव॑द्ध्ये॒ नोव॑द्ध्येन शव॒र्ताञ् छ॑व॒र्ता नूव॑द्ध्येन॒ शुनः॒ शुन॒ ऊव॑द्ध्येन शव॒र्ताञ् छ॑व॒र्ता नूव॑द्ध्येन॒ शुनः॑ । \newline
18. ऊव॑द्ध्येन॒ शुनः॒ शुन॒ ऊव॑द्ध्ये॒नो व॑द्ध्येन॒ शुनो॑ वि॒शस॑नेन वि॒शस॑नेन॒ शुन॒ ऊव॑द्ध्ये॒नो व॑द्ध्येन॒ शुनो॑ वि॒शस॑नेन । \newline
19. शुनो॑ वि॒शस॑नेन वि॒शस॑नेन॒ शुनः॒ शुनो॑ वि॒शस॑नेन स॒र्पान् थ्स॒र्पान्. वि॒शस॑नेन॒ शुनः॒ शुनो॑ वि॒शस॑नेन स॒र्पान् । \newline
20. वि॒शस॑नेन स॒र्पान् थ्स॒र्पान्. वि॒शस॑नेन वि॒शस॑नेन स॒र्पान् ॅलो॑हितग॒न्धेन॑ लोहितग॒न्धेन॑ स॒र्पान्. वि॒शस॑नेन वि॒शस॑नेन स॒र्पान् ॅलो॑हितग॒न्धेन॑ । \newline
21. वि॒शस॑ने॒नेति॑ वि - शस॑नेन । \newline
22. स॒र्पान् ॅलो॑हितग॒न्धेन॑ लोहितग॒न्धेन॑ स॒र्पान् थ्स॒र्पान् ॅलो॑हितग॒न्धेन॒ वयाꣳ॑सि॒ वयाꣳ॑सि लोहितग॒न्धेन॑ स॒र्पान् थ्स॒र्पान् ॅलो॑हितग॒न्धेन॒ वयाꣳ॑सि । \newline
23. लो॒हि॒त॒ग॒न्धेन॒ वयाꣳ॑सि॒ वयाꣳ॑सि लोहितग॒न्धेन॑ लोहितग॒न्धेन॒ वयाꣳ॑सि पक्वग॒न्धेन॑ पक्वग॒न्धेन॒ वयाꣳ॑सि लोहितग॒न्धेन॑ लोहितग॒न्धेन॒ वयाꣳ॑सि पक्वग॒न्धेन॑ । \newline
24. लो॒हि॒त॒ग॒न्धेनेति॑ लोहित - ग॒न्धेन॑ । \newline
25. वयाꣳ॑सि पक्वग॒न्धेन॑ पक्वग॒न्धेन॒ वयाꣳ॑सि॒ वयाꣳ॑सि पक्वग॒न्धेन॑ पि॒पीलि॑काः पि॒पीलि॑काः पक्वग॒न्धेन॒ वयाꣳ॑सि॒ वयाꣳ॑सि पक्वग॒न्धेन॑ पि॒पीलि॑काः । \newline
26. प॒क्व॒ग॒न्धेन॑ पि॒पीलि॑काः पि॒पीलि॑काः पक्वग॒न्धेन॑ पक्वग॒न्धेन॑ पि॒पीलि॑काः प्रशा॒देन॑ प्रशा॒देन॑ पि॒पीलि॑काः पक्वग॒न्धेन॑ पक्वग॒न्धेन॑ पि॒पीलि॑काः प्रशा॒देन॑ । \newline
27. प॒क्व॒ग॒न्धेनेति॑ पक्व - ग॒न्धेन॑ । \newline
28. पि॒पीलि॑काः प्रशा॒देन॑ प्रशा॒देन॑ पि॒पीलि॑काः पि॒पीलि॑काः प्रशा॒देन॑ । \newline
29. प्र॒शा॒देनेति॑ प्र - शा॒देन॑ । \newline
\pagebreak
\markright{ TS 5.7.24.1  \hfill https://www.vedavms.in \hfill}

\section{ TS 5.7.24.1 }

\textbf{TS 5.7.24.1 } \newline
\textbf{Samhita Paata} \newline

क्रमै॒रत्य॑क्रमीद्-वा॒जी विश्वै᳚र्दे॒वैर्य॒ज्ञियैः᳚ संॅविदा॒नः ।स नो॑ नय सुकृ॒तस्य॑ लो॒कं तस्य॑ ते व॒यꣳ स्व॒धया॑ मदेम ॥ \newline

\textbf{Pada Paata} \newline

क्रमैः᳚ । अतीति॑ । अ॒क्र॒मी॒त् । वा॒जी । विश्वैः᳚ । दे॒वैः । य॒ज्ञियैः᳚ । सं॒ॅवि॒दा॒न इति॑ सं - वि॒दा॒नः ॥ सः । नः॒ । न॒य॒ । सु॒कृ॒तस्येति॑ सु - कृ॒तस्य॑ । लो॒कम् । तस्य॑ । ते॒ । व॒यम् । स्व॒धयेति॑ स्व-धया᳚ । म॒दे॒म॒ ॥  \newline


\textbf{Krama Paata} \newline

क्रमै॒रति॑ । अत्य॑क्रमीत् । अ॒क्र॒मी॒द् वा॒जी । वा॒जी विश्वैः᳚ । विश्वै᳚र् दे॒वैः । दे॒वैर् य॒ज्ञियैः᳚ । य॒ज्ञियैः᳚ सम्ॅविदा॒नः । स॒म्ॅवि॒दा॒न इति॑ सम् - वि॒दा॒नः ॥ स नः॑ । नो॒ न॒य॒ । न॒य॒ सु॒कृ॒तस्य॑ । सु॒कृ॒तस्य॑ लो॒कम् । सु॒कृ॒तस्येति॑ सु - कृ॒तस्य॑ । लो॒कम् तस्य॑ । तस्य॑ ते । ते॒ व॒यम् । व॒यꣳ स्व॒धया᳚ । स्व॒धया॑ मदेम । स्व॒धयेति॑ स्व - धया᳚ । म॒दे॒मेति॑ मदेम । \newline

\textbf{Jatai Paata} \newline

1. क्रमै॒ रत्यति॒ क्रमैः॒ क्रमै॒ रति॑ । \newline
2. अत्य॑ क्रमी दक्रमी॒ दत्य त्य॑क्रमीत् । \newline
3. अ॒क्र॒मी॒द् वा॒जी वा॒ज्य॑ क्रमी दक्रमीद् वा॒जी । \newline
4. वा॒जी विश्वै॒र् विश्वै᳚र् वा॒जी वा॒जी विश्वैः᳚ । \newline
5. विश्वै᳚र् दे॒वैर् दे॒वैर् विश्वै॒र् विश्वै᳚र् दे॒वैः । \newline
6. दे॒वैर् य॒ज्ञियै᳚र् य॒ज्ञियै᳚र् दे॒वैर् दे॒वैर् य॒ज्ञियैः᳚ । \newline
7. य॒ज्ञियैः᳚ संॅविदा॒नः सं॑ॅविदा॒नो य॒ज्ञियै᳚र् य॒ज्ञियैः᳚ संॅविदा॒नः । \newline
8. सं॒ॅवि॒दा॒न इति॑ सं - वि॒दा॒नः । \newline
9. स नो॑ नः॒ स स नः॑ । \newline
10. नो॒ न॒य॒ न॒य॒ नो॒ नो॒ न॒य॒ । \newline
11. न॒य॒ सु॒कृ॒तस्य॑ सुकृ॒तस्य॑ नय नय सुकृ॒तस्य॑ । \newline
12. सु॒कृ॒तस्य॑ लो॒कम् ॅलो॒कꣳ सु॑कृ॒तस्य॑ सुकृ॒तस्य॑ लो॒कम् । \newline
13. सु॒कृ॒तस्येति॑ सु - कृ॒तस्य॑ । \newline
14. लो॒कम् तस्य॒ तस्य॑ लो॒कम् ॅलो॒कम् तस्य॑ । \newline
15. तस्य॑ ते ते॒ तस्य॒ तस्य॑ ते । \newline
16. ते॒ व॒यं ॅव॒यम् ते॑ ते व॒यम् । \newline
17. व॒यꣳ स्व॒धया᳚ स्व॒धया॑ व॒यं ॅव॒यꣳ स्व॒धया᳚ । \newline
18. स्व॒धया॑ मदेम मदेम स्व॒धया᳚ स्व॒धया॑ मदेम । \newline
19. स्व॒धयेति॑ स्व - धया᳚ । \newline
20. म॒दे॒मेति॑ मदेम । \newline

\textbf{Ghana Paata } \newline

1. क्रमै॒ रत्यति॒ क्रमैः॒ क्रमै॒ रत्य॑ क्रमी दक्रमी॒ दति॒ क्रमैः॒ क्रमै॒ रत्य॑क्रमीत् । \newline
2. अत्य॑क्रमी दक्रमी॒ दत्य त्य॑क्रमीद् वा॒जी वा॒ज्य॑ क्रमी॒ दत्य त्य॑क्रमीद् वा॒जी । \newline
3. अ॒क्र॒मी॒द् वा॒जी वा॒ज्य॑क्रमी दक्रमीद् वा॒जी विश्वै॒र् विश्वै᳚र् वा॒ज्य॑क्रमी दक्रमीद् वा॒जी विश्वैः᳚ । \newline
4. वा॒जी विश्वै॒र् विश्वै᳚र् वा॒जी वा॒जी विश्वै᳚र् दे॒वैर् दे॒वैर् विश्वै᳚र् वा॒जी वा॒जी विश्वै᳚र् दे॒वैः । \newline
5. विश्वै᳚र् दे॒वैर् दे॒वैर् विश्वै॒र् विश्वै᳚र् दे॒वैर् य॒ज्ञियै᳚र् य॒ज्ञियै᳚र् दे॒वैर् विश्वै॒र् विश्वै᳚र् दे॒वैर् य॒ज्ञियैः᳚ । \newline
6. दे॒वैर् य॒ज्ञियै᳚र् य॒ज्ञियै᳚र् दे॒वैर् दे॒वैर् य॒ज्ञियैः᳚ संॅविदा॒नः सं॑ॅविदा॒नो य॒ज्ञियै᳚र् दे॒वैर् दे॒वैर् य॒ज्ञियैः᳚ संॅविदा॒नः । \newline
7. य॒ज्ञियैः᳚ संॅविदा॒नः सं॑ॅविदा॒नो य॒ज्ञियै᳚र् य॒ज्ञियैः᳚ संॅविदा॒नः । \newline
8. सं॒ॅवि॒दा॒न इति॑ सं - वि॒दा॒नः । \newline
9. स नो॑ नः॒ स स नो॑ नय नय नः॒ स स नो॑ नय । \newline
10. नो॒ न॒य॒ न॒य॒ नो॒ नो॒ न॒य॒ सु॒कृ॒तस्य॑ सुकृ॒तस्य॑ नय नो नो नय सुकृ॒तस्य॑ । \newline
11. न॒य॒ सु॒कृ॒तस्य॑ सुकृ॒तस्य॑ नय नय सुकृ॒तस्य॑ लो॒कम् ॅलो॒कꣳ सु॑कृ॒तस्य॑ नय नय सुकृ॒तस्य॑ लो॒कम् । \newline
12. सु॒कृ॒तस्य॑ लो॒कम् ॅलो॒कꣳ सु॑कृ॒तस्य॑ सुकृ॒तस्य॑ लो॒कम् तस्य॒ तस्य॑ लो॒कꣳ सु॑कृ॒तस्य॑ सुकृ॒तस्य॑ लो॒कम् तस्य॑ । \newline
13. सु॒कृ॒तस्येति॑ सु - कृ॒तस्य॑ । \newline
14. लो॒कम् तस्य॒ तस्य॑ लो॒कम् ॅलो॒कम् तस्य॑ ते ते॒ तस्य॑ लो॒कम् ॅलो॒कम् तस्य॑ ते । \newline
15. तस्य॑ ते ते॒ तस्य॒ तस्य॑ ते व॒यं ॅव॒यम् ते॒ तस्य॒ तस्य॑ ते व॒यम् । \newline
16. ते॒ व॒यं ॅव॒यम् ते॑ ते व॒यꣳ स्व॒धया᳚ स्व॒धया॑ व॒यम् ते॑ ते व॒यꣳ स्व॒धया᳚ । \newline
17. व॒यꣳ स्व॒धया᳚ स्व॒धया॑ व॒यं ॅव॒यꣳ स्व॒धया॑ मदेम मदेम स्व॒धया॑ व॒यं ॅव॒यꣳ स्व॒धया॑ मदेम । \newline
18. स्व॒धया॑ मदेम मदेम स्व॒धया᳚ स्व॒धया॑ मदेम । \newline
19. स्व॒धयेति॑ स्व - धया᳚ । \newline
20. म॒दे॒मेति॑ मदेम । \newline
\pagebreak
\markright{ TS 5.7.25.1  \hfill https://www.vedavms.in \hfill}

\section{ TS 5.7.25.1 }

\textbf{TS 5.7.25.1 } \newline
\textbf{Samhita Paata} \newline

द्यौस्ते॑ पृ॒ष्ठं पृ॑थि॒वी स॒धस्थ॑मा॒त्मान्तरि॑क्षꣳ समु॒द्रो योनिः॒ सूर्य॑स्ते॒ चक्षु॒र्वातः॑ प्रा॒णश्च॒न्द्रमाः॒ श्रोत्रं॒ मासा᳚श्चार्द्धमा॒साश्च॒ पर्वा᳚ण्यृ॒तवोङ्गा॑नि संॅवथ्स॒रो म॑हि॒मा ॥ \newline

\textbf{Pada Paata} \newline

द्यौः । ते॒ । पृ॒ष्ठम् । पृ॒थि॒वी । स॒धस्थ॒मिति॑ स॒ध - स्थ॒म् । आ॒त्मा । अ॒न्तरि॑क्षम् । स॒मु॒द्रः । योनिः॑ । सूर्यः॑ । ते॒ । चक्षुः॑ । वातः॑ । प्रा॒ण इति॑ प्र - अ॒नः । च॒न्द्रमाः᳚ । श्रोत्र᳚म् । मासाः᳚ । च॒ । अ॒द्‌र्ध॒मा॒सा इत्य॑द्‌र्ध - मा॒साः । च॒ । पर्वा॑णि । ऋ॒तवः॑ । अङ्गा॑नि । सं॒ॅव॒थ्स॒र इति॑ सं - व॒थ्स॒रः । म॒हि॒मा ॥  \newline


\textbf{Krama Paata} \newline

द्यौस्ते᳚ । ते॒ पृ॒ष्ठम् । पृ॒ष्ठम् पृ॑थि॒वी । पृ॒थि॒वी स॒धस्थ᳚म् । स॒धस्थ॑मा॒त्मा । स॒धस्थ॒मिति॑ स॒ध - स्थ॒म् । आ॒त्माऽन्तरि॑क्षम् । अ॒न्तरि॑क्षꣳ समु॒द्रः । स॒मु॒द्रो योनिः॑ । योनिः॒ सूर्यः॑ । सूर्य॑स्ते । ते॒ चक्षुः॑ । चक्षु॒र् वातः॑ । वातः॑ प्रा॒णः । प्रा॒णश्च॒न्द्रमाः᳚ । प्रा॒ण इति॑ प्र - अ॒नः । च॒न्द्रमाः॒ श्रोत्र᳚म् । श्रोत्र॒म् मासाः᳚ । मासा᳚श्च । चा॒र्द्ध॒मा॒साः । अ॒र्द्ध॒मा॒साश्च॑ । अ॒र्द्ध॒मा॒सा इत्य॑र्द्ध - मा॒साः । च॒ पर्वा॑णि । पर्वा᳚ण्यृ॒तवः॑ । ऋ॒तवोऽङ्गा॑नि । अङ्गा॑नि सम्ॅवथ्स॒रः । स॒म्ॅव॒थ्स॒रो म॑हि॒मा । स॒म्ॅव॒थ्स॒र इति॑ सम् - व॒थ्स॒रः । म॒हि॒मेति॑ महि॒मा । \newline

\textbf{Jatai Paata} \newline

1. द्यौ स्ते॑ ते॒ द्यौर् द्यौ स्ते᳚ । \newline
2. ते॒ पृ॒ष्ठम् पृ॒ष्ठम् ते॑ ते पृ॒ष्ठम् । \newline
3. पृ॒ष्ठम् पृ॑थि॒वी पृ॑थि॒वी पृ॒ष्ठम् पृ॒ष्ठम् पृ॑थि॒वी । \newline
4. पृ॒थि॒वी स॒धस्थꣳ॑ स॒धस्थ॑म् पृथि॒वी पृ॑थि॒वी स॒धस्थ᳚म् । \newline
5. स॒धस्थ॑ मा॒त्मा ऽऽत्मा स॒धस्थꣳ॑ स॒धस्थ॑ मा॒त्मा । \newline
6. स॒धस्थ॒मिति॑ स॒ध - स्थ॒म् । \newline
7. आ॒त्मा ऽन्तरि॑क्ष म॒न्तरि॑क्ष मा॒त्मा ऽऽत्मा ऽन्तरि॑क्षम् । \newline
8. अ॒न्तरि॑क्षꣳ समु॒द्रः स॑मु॒द्रो अ॒न्तरि॑क्ष म॒न्तरि॑क्षꣳ समु॒द्रः । \newline
9. स॒मु॒द्रो योनि॒र् योनिः॑ समु॒द्रः स॑मु॒द्रो योनिः॑ । \newline
10. योनिः॒ सूर्यः॒ सूर्यो॒ योनि॒र् योनिः॒ सूर्यः॑ । \newline
11. सूर्य॑ स्ते ते॒ सूर्यः॒ सूर्य॑ स्ते । \newline
12. ते॒ चक्षु॒ श्चक्षु॑ स्ते ते॒ चक्षुः॑ । \newline
13. चक्षु॒र् वातो॒ वात॒ श्चक्षु॒ श्चक्षु॒र् वातः॑ । \newline
14. वातः॑ प्रा॒णः प्रा॒णो वातो॒ वातः॑ प्रा॒णः । \newline
15. प्रा॒ण श्च॒न्द्रमा᳚ श्च॒न्द्रमाः᳚ प्रा॒णः प्रा॒ण श्च॒न्द्रमाः᳚ । \newline
16. प्रा॒ण इति॑ प्र - अ॒नः । \newline
17. च॒न्द्रमाः॒ श्रोत्रꣳ॒॒ श्रोत्र॑म् च॒न्द्रमा᳚ श्च॒न्द्रमाः॒ श्रोत्र᳚म् । \newline
18. श्रोत्र॒म् मासा॒ मासाः॒ श्रोत्रꣳ॒॒ श्रोत्र॒म् मासाः᳚ । \newline
19. मासा᳚श्च च॒ मासा॒ मासा᳚श्च । \newline
20. चा॒र्द्ध॒मा॒सा अ॑र्द्धमा॒सा श्च॑ चार्द्धमा॒साः । \newline
21. अ॒र्द्ध॒मा॒सा श्च॑ चार्द्धमा॒सा अ॑र्द्धमा॒सा श्च॑ । \newline
22. अ॒र्द्ध॒मा॒सा इत्य॑र्द्ध - मा॒साः । \newline
23. च॒ पर्वा॑णि॒ पर्वा॑णि च च॒ पर्वा॑णि । \newline
24. पर्वा᳚ण्यृ॒तव॑ ऋ॒तवः॒ पर्वा॑णि॒ पर्वा᳚ण्यृ॒तवः॑ । \newline
25. ऋ॒तवो ऽङ्गा॒ न्यङ्गा᳚ न्यृ॒तव॑ ऋ॒तवो ऽङ्गा॑नि । \newline
26. अङ्गा॑नि संॅवथ्स॒रः सं॑ॅवथ्स॒रो ऽङ्गा॒ न्यङ्गा॑नि संॅवथ्स॒रः । \newline
27. सं॒ॅव॒थ्स॒रो म॑हि॒मा म॑हि॒मा सं॑ॅवथ्स॒रः सं॑ॅवथ्स॒रो म॑हि॒मा । \newline
28. सं॒ॅव॒थ्स॒र इति॑ सं - व॒थ्स॒रः । \newline
29. म॒हि॒मेति॑ महि॒मा । \newline

\textbf{Ghana Paata } \newline

1. द्यौ स्ते॑ ते॒ द्यौर् द्यौ स्ते॑ पृ॒ष्ठम् पृ॒ष्ठम् ते॒ द्यौर् द्यौ स्ते॑ पृ॒ष्ठम् । \newline
2. ते॒ पृ॒ष्ठम् पृ॒ष्ठम् ते॑ ते पृ॒ष्ठम् पृ॑थि॒वी पृ॑थि॒वी पृ॒ष्ठम् ते॑ ते पृ॒ष्ठम् पृ॑थि॒वी । \newline
3. पृ॒ष्ठम् पृ॑थि॒वी पृ॑थि॒वी पृ॒ष्ठम् पृ॒ष्ठम् पृ॑थि॒वी स॒धस्थꣳ॑ स॒धस्थ॑म् पृथि॒वी पृ॒ष्ठम् पृ॒ष्ठम् पृ॑थि॒वी स॒धस्थ᳚म् । \newline
4. पृ॒थि॒वी स॒धस्थꣳ॑ स॒धस्थ॑म् पृथि॒वी पृ॑थि॒वी स॒धस्थ॑ मा॒त्मा ऽऽत्मा स॒धस्थ॑म् पृथि॒वी पृ॑थि॒वी स॒धस्थ॑ मा॒त्मा । \newline
5. स॒धस्थ॑ मा॒त्मा ऽऽत्मा स॒धस्थꣳ॑ स॒धस्थ॑ मा॒त्मा ऽन्तरि॑क्ष म॒न्तरि॑क्ष मा॒त्मा स॒धस्थꣳ॑ स॒धस्थ॑ मा॒त्मा ऽन्तरि॑क्षम् । \newline
6. स॒धस्थ॒मिति॑ स॒ध - स्थ॒म् । \newline
7. आ॒त्मा ऽन्तरि॑क्ष म॒न्तरि॑क्ष मा॒त्मा ऽऽत्मा ऽन्तरि॑क्षꣳ समु॒द्रः स॑मु॒द्रो अ॒न्तरि॑क्ष मा॒त्मा ऽऽत्मा ऽन्तरि॑क्षꣳ समु॒द्रः । \newline
8. अ॒न्तरि॑क्षꣳ समु॒द्रः स॑मु॒द्रो अ॒न्तरि॑क्ष म॒न्तरि॑क्षꣳ समु॒द्रो योनि॒र् योनिः॑ समु॒द्रो अ॒न्तरि॑क्ष म॒न्तरि॑क्षꣳ समु॒द्रो योनिः॑ । \newline
9. स॒मु॒द्रो योनि॒र् योनिः॑ समु॒द्रः स॑मु॒द्रो योनिः॒ सूर्यः॒ सूर्यो॒ योनिः॑ समु॒द्रः स॑मु॒द्रो योनिः॒ सूर्यः॑ । \newline
10. योनिः॒ सूर्यः॒ सूर्यो॒ योनि॒र् योनिः॒ सूर्य॑ स्ते ते॒ सूर्यो॒ योनि॒र् योनिः॒ सूर्य॑ स्ते । \newline
11. सूर्य॑ स्ते ते॒ सूर्यः॒ सूर्य॑ स्ते॒ चक्षु॒ श्चक्षु॑ स्ते॒ सूर्यः॒ सूर्य॑ स्ते॒ चक्षुः॑ । \newline
12. ते॒ चक्षु॒ श्चक्षु॑ स्ते ते॒ चक्षु॒र् वातो॒ वात॒ श्चक्षु॑ स्ते ते॒ चक्षु॒र् वातः॑ । \newline
13. चक्षु॒र् वातो॒ वात॒ श्चक्षु॒ श्चक्षु॒र् वातः॑ प्रा॒णः प्रा॒णो वात॒ श्चक्षु॒ श्चक्षु॒र् वातः॑ प्रा॒णः । \newline
14. वातः॑ प्रा॒णः प्रा॒णो वातो॒ वातः॑ प्रा॒ण श्च॒न्द्रमा᳚ श्च॒न्द्रमाः᳚ प्रा॒णो वातो॒ वातः॑ प्रा॒ण श्च॒न्द्रमाः᳚ । \newline
15. प्रा॒ण श्च॒न्द्रमा᳚ श्च॒न्द्रमाः᳚ प्रा॒णः प्रा॒ण श्च॒न्द्रमाः॒ श्रोत्रꣳ॒॒ श्रोत्र॑म् च॒न्द्रमाः᳚ प्रा॒णः प्रा॒ण श्च॒न्द्रमाः॒ श्रोत्र᳚म् । \newline
16. प्रा॒ण इति॑ प्र - अ॒नः । \newline
17. च॒न्द्रमाः॒ श्रोत्रꣳ॒॒ श्रोत्र॑म् च॒न्द्रमा᳚ श्च॒न्द्रमाः॒ श्रोत्र॒म् मासा॒ मासाः॒ श्रोत्र॑म् च॒न्द्रमा᳚ श्च॒न्द्रमाः॒ श्रोत्र॒म् मासाः᳚ । \newline
18. श्रोत्र॒म् मासा॒ मासाः॒ श्रोत्रꣳ॒॒ श्रोत्र॒म् मासा᳚श्च च॒ मासाः॒ श्रोत्रꣳ॒॒ श्रोत्र॒म् मासा᳚श्च । \newline
19. मासा᳚ श्च च॒ मासा॒ मासा᳚ श्चार्द्धमा॒सा अ॑र्द्धमा॒सा श्च॒ मासा॒ मासा᳚ श्चार्द्धमा॒साः । \newline
20. चा॒र्द्ध॒मा॒सा अ॑र्द्धमा॒सा श्च॑ चार्द्धमा॒सा श्च॑ चार्द्धमा॒सा श्च॑ चार्द्धमा॒सा श्च॑ । \newline
21. अ॒र्द्ध॒मा॒सा श्च॑ चार्द्धमा॒सा अ॑र्द्धमा॒सा श्च॒ पर्वा॑णि॒ पर्वा॑णि चार्द्धमा॒सा अ॑र्द्धमा॒सा श्च॒ पर्वा॑णि । \newline
22. अ॒र्द्ध॒मा॒सा इत्य॑र्द्ध - मा॒साः । \newline
23. च॒ पर्वा॑णि॒ पर्वा॑णि च च॒ पर्वा᳚ण्यृ॒तव॑ ऋ॒तवः॒ पर्वा॑णि च च॒ पर्वा᳚ण्यृ॒तवः॑ । \newline
24. पर्वा᳚ण्यृ॒तव॑ ऋ॒तवः॒ पर्वा॑णि॒ पर्वा᳚ण्यृ॒तवो ऽङ्गा॒ न्यङ्गा᳚ न्यृ॒तवः॒ पर्वा॑णि॒ पर्वा᳚ण्यृ॒तवो ऽङ्गा॑नि । \newline
25. ऋ॒तवो ऽङ्गा॒न्य ङ्गा᳚ न्यृ॒तव॑ ऋ॒तवो ऽङ्गा॑नि संॅवथ्स॒रः सं॑ॅवथ्स॒रो ऽङ्गा᳚ न्यृ॒तव॑ ऋ॒तवो ऽङ्गा॑नि संॅवथ्स॒रः । \newline
26. अङ्गा॑नि संॅवथ्स॒रः सं॑ॅवथ्स॒रो ऽङ्गा॒ न्यङ्गा॑नि संॅवथ्स॒रो म॑हि॒मा म॑हि॒मा सं॑ॅवथ्स॒रो ऽङ्गा॒ न्यङ्गा॑नि संॅवथ्स॒रो म॑हि॒मा । \newline
27. सं॒ॅव॒थ्स॒रो म॑हि॒मा म॑हि॒मा सं॑ॅवथ्स॒रः सं॑ॅवथ्स॒रो म॑हि॒मा । \newline
28. सं॒ॅव॒थ्स॒र इति॑ सं - व॒थ्स॒रः । \newline
29. म॒हि॒मेति॑ महि॒मा । \newline
\pagebreak
\markright{ TS 5.7.26.1  \hfill https://www.vedavms.in \hfill}

\section{ TS 5.7.26.1 }

\textbf{TS 5.7.26.1 } \newline
\textbf{Samhita Paata} \newline

अ॒ग्निः प॒शुरा॑सी॒त् तेना॑यजन्त॒ स ए॒तं ॅलो॒कम॑जय॒द्-यस्मि॑न्न॒ग्निः स ते॑ लो॒कस्तं जे᳚ष्य॒स्यथाव॑ जिघ्र वा॒युः प॒शुरा॑सी॒त् तेना॑यजन्त॒ स ए॒तं ॅलो॒कम॑जय॒द्-यस्मि॑न् वा॒युः स ते॑ लो॒कस्तस्मा᳚त् त्वा॒ऽन्तरे᳚ष्यामि॒ यदि॒ नाव॒जिघ्र॑स्यादि॒त्यः प॒शुरा॑सी॒त् तेना॑यजन्त॒ स ए॒तं ॅलो॒कम॑जय॒द् यस्मि॑ -( )-न्नादि॒त्यः स ते॑ लो॒कस्तं जे᳚ष्यसि॒ यद्य॑व॒जिघ्र॑सि ॥ \newline

\textbf{Pada Paata} \newline

अ॒ग्निः । प॒शुः । आ॒सी॒त् । तेन॑ । अ॒य॒ज॒न्त॒ । सः । ए॒तम् । लो॒कम् । अ॒ज॒य॒त् । यस्मिन्न्॑ । अ॒ग्निः । सः । ते॒ । लो॒कः । तम् । जे॒ष्य॒सि॒ । अथ॑ । अवेति॑ । जि॒घ्र॒ । वा॒युः । प॒शुः । आ॒सी॒त् । तेन॑ । अ॒य॒ज॒न्त॒ । सः । ए॒तम् । लो॒कम् । अ॒ज॒य॒त् । यस्मिन्न्॑ । वा॒युः । सः । ते॒ । लो॒कः । तस्मा᳚त् । त्वा॒ । अ॒न्तः । ए॒ष्या॒मि॒ । यदि॑ । न । अ॒व॒जिघ्र॒सीत्य॑व-जिघ्र॑सि । आ॒दि॒त्यः । प॒शुः । आ॒सी॒त् । तेन॑ । अ॒य॒ज॒न्त॒ । सः । ए॒तम् । लो॒कम् । अ॒ज॒य॒त् । यस्मिन्न्॑ ( ) । आ॒दि॒त्यः । सः । ते॒ । लो॒कः । तम् । जे॒ष्य॒सि॒ । यदि॑ । अ॒व॒जिघ्र॒सीत्य॑व - जिघ्र॑सि ॥  \newline


\textbf{Krama Paata} \newline

अ॒ग्निः प॒शुः । प॒शुरा॑सीत् । आ॒सी॒त् तेन॑ । तेना॑यजन्त । अ॒य॒ज॒न्त॒ सः । स ए॒तम् । ए॒तम् ॅलो॒कम् । लो॒कम॑जयत् । अ॒ज॒य॒द् यस्मिन्न्॑ । यस्मि॑न्न॒ग्निः । अ॒ग्निः सः । स ते᳚ । ते॒ लो॒कः । लो॒क स्तम् । तम् जे᳚ष्यसि । जे॒ष्य॒स्यथ॑ । अथाव॑ । अव॑ जिघ्र । जि॒घ्र॒ वा॒युः । वा॒युः प॒शुः । प॒शुरा॑सीत् । आ॒सी॒त् तेन॑ । तेना॑यजन्त । अ॒ज॒य॒न्त॒ सः । स ए॒तम् । ए॒तम् ॅलो॒कम् । लो॒कम॑जयत् । अ॒ज॒य॒द् यस्मिन्न्॑ । यस्मि॑न् वा॒युः । वा॒युः सः । स ते᳚ । ते॒ लो॒कः । लो॒कस्तस्मा᳚त् । तस्मा᳚त् त्वा । त्वा॒ऽन्तः । अ॒न्तरे᳚ष्यामि । ए॒ष्या॒मि॒ यदि॑ । यदि॒ न । नाव॒जिघ्र॑सि । अ॒व॒जिघ्र॑स्यादि॒त्यः । अ॒व॒जिघ्र॒सीत्य॑व - जिघ्र॑सि । आ॒दि॒त्यः प॒शुः । प॒शुरा॑सीत् । आ॒सी॒त् तेन॑ । तेना॑यजन्त । अ॒य॒ज॒न्त॒ सः । स ए॒तम् । ए॒तम् ॅलो॒कम् । लो॒कम॑जयत् । अ॒ज॒य॒द् यस्मिन्न्॑ । यस्मि॑न्नादि॒त्यः । आ॒दि॒त्यः सः । स ते᳚ । ते॒ लो॒कः । लो॒कस्तम् । तम् जे᳚ष्यसि । जे॒ष्य॒सि॒ यदि॑ । यद्य॑व॒जिघ्र॑सि । अ॒व॒जिघ्र॒सीत्य॑व - जिघ्र॑सि । \newline

\textbf{Jatai Paata} \newline

1. अ॒ग्निः प॒शुः प॒शु र॒ग्नि र॒ग्निः प॒शुः । \newline
2. प॒शु रा॑सी दासीत् प॒शुः प॒शु रा॑सीत् । \newline
3. आ॒सी॒त् तेन॒ तेना॑सी दासी॒त् तेन॑ । \newline
4. तेना॑ यजन्ता यजन्त॒ तेन॒ तेना॑ यजन्त । \newline
5. अ॒य॒ज॒न्त॒ स सो॑ ऽयजन्ता यजन्त॒ सः । \newline
6. स ए॒त मे॒तꣳ स स ए॒तम् । \newline
7. ए॒तम् ॅलो॒कम् ॅलो॒क मे॒त मे॒तम् ॅलो॒कम् । \newline
8. लो॒क म॑जय दजयल् लो॒कम् ॅलो॒क म॑जयत् । \newline
9. अ॒ज॒य॒द् यस्मि॒न्॒. यस्मि॑न् नजय दजय॒द् यस्मिन्न्॑ । \newline
10. यस्मि॑न् न॒ग्नि र॒ग्निर् यस्मि॒न्॒. यस्मि॑न् न॒ग्निः । \newline
11. अ॒ग्निः स सो᳚ ऽग्नि र॒ग्निः सः । \newline
12. स ते॑ ते॒ स स ते᳚ । \newline
13. ते॒ लो॒को लो॒क स्ते॑ ते लो॒कः । \newline
14. लो॒क स्तम् तम् ॅलो॒को लो॒क स्तम् । \newline
15. तम् जे᳚ष्यसि जेष्यसि॒ तम् तम् जे᳚ष्यसि । \newline
16. जे॒ष्य॒ स्यथाथ॑ जेष्यसि जेष्य॒स्यथ॑ । \newline
17. अथावा वाथा थाव॑ । \newline
18. अव॑ जिघ्र जि॒घ्रा वाव॑ जिघ्र । \newline
19. जि॒घ्र॒ वा॒युर् वा॒युर् जि॑घ्र जिघ्र वा॒युः । \newline
20. वा॒युः प॒शुः प॒शुर् वा॒युर् वा॒युः प॒शुः । \newline
21. प॒शु रा॑सी दासीत् प॒शुः प॒शु रा॑सीत् । \newline
22. आ॒सी॒त् तेन॒ तेना॑सी दासी॒त् तेन॑ । \newline
23. तेना॑ यजन्ता यजन्त॒ तेन॒ तेना॑ यजन्त । \newline
24. अ॒य॒ज॒न्त॒ स सो॑ ऽयजन्ता यजन्त॒ सः । \newline
25. स ए॒त मे॒तꣳ स स ए॒तम् । \newline
26. ए॒तम् ॅलो॒कम् ॅलो॒क मे॒त मे॒तम् ॅलो॒कम् । \newline
27. लो॒क म॑जय दजयल् लो॒कम् ॅलो॒क म॑जयत् । \newline
28. अ॒ज॒य॒द् यस्मि॒न्॒. यस्मि॑न् नजय दजय॒द् यस्मिन्न्॑ । \newline
29. यस्मि॑न् वा॒युर् वा॒युर् यस्मि॒न्॒. यस्मि॑न् वा॒युः । \newline
30. वा॒युः स स वा॒युर् वा॒युः सः । \newline
31. स ते॑ ते॒ स स ते᳚ । \newline
32. ते॒ लो॒को लो॒क स्ते॑ ते लो॒कः । \newline
33. लो॒क स्तस्मा॒त् तस्मा᳚ल् लो॒को लो॒क स्तस्मा᳚त् । \newline
34. तस्मा᳚त् त्वा त्वा॒ तस्मा॒त् तस्मा᳚त् त्वा । \newline
35. त्वा॒ ऽन्त र॒न्त स्त्वा᳚ त्वा॒ ऽन्तः । \newline
36. अ॒न्त रे᳚ष्या म्येष्या म्य॒न्त र॒न्त रे᳚ष्यामि । \newline
37. ए॒ष्या॒मि॒ यदि॒ यद्ये᳚ष्या म्येष्यामि॒ यदि॑ । \newline
38. यदि॒ न न यदि॒ यदि॒ न । \newline
39. नाव॒जिघ्र॑ स्यव॒जिघ्र॑सि॒ न नाव॒जिघ्र॑सि । \newline
40. अ॒व॒जिघ्र॑ स्यादि॒त्य आ॑दि॒त्यो॑ ऽव॒जिघ्र॑ स्यव॒जिघ्र॑ स्यादि॒त्यः । \newline
41. अ॒व॒जिघ्र॒सीत्य॑व - जिघ्र॑सि । \newline
42. आ॒दि॒त्यः प॒शुः प॒शु रा॑दि॒त्य आ॑दि॒त्यः प॒शुः । \newline
43. प॒शु रा॑सी दासीत् प॒शुः प॒शु रा॑सीत् । \newline
44. आ॒सी॒त् तेन॒ तेना॑सी दासी॒त् तेन॑ । \newline
45. तेना॑ यजन्ता यजन्त॒ तेन॒ तेना॑ यजन्त । \newline
46. अ॒य॒ज॒न्त॒ स सो॑ ऽयजन्ता यजन्त॒ सः । \newline
47. स ए॒त मे॒तꣳ स स ए॒तम् । \newline
48. ए॒तम् ॅलो॒कम् ॅलो॒क मे॒त मे॒तम् ॅलो॒कम् । \newline
49. लो॒क म॑जय दजयल् लो॒कम् ॅलो॒क म॑जयत् । \newline
50. अ॒ज॒य॒द् यस्मि॒न्॒. यस्मि॑न् नजय दजय॒द् यस्मिन्न्॑ । \newline
51. यस्मि॑न् नादि॒त्य आ॑दि॒त्यो यस्मि॒न्॒. यस्मि॑न् नादि॒त्यः । \newline
52. आ॒दि॒त्यः स स आ॑दि॒त्य आ॑दि॒त्यः सः । \newline
53. स ते॑ ते॒ स स ते᳚ । \newline
54. ते॒ लो॒को लो॒क स्ते॑ ते लो॒कः । \newline
55. लो॒क स्तम् तम् ॅलो॒को लो॒क स्तम् । \newline
56. तम् जे᳚ष्यसि जेष्यसि॒ तम् तम् जे᳚ष्यसि । \newline
57. जे॒ष्य॒सि॒ यदि॒ यदि॑ जेष्यसि जेष्यसि॒ यदि॑ । \newline
58. यद्य॑व॒जिघ्र॑ स्यव॒जिघ्र॑सि॒ यदि॒ यद्य॑व॒जिघ्र॑सि । \newline
59. अ॒व॒जिघ्र॒सीत्य॑व - जिघ्र॑सि । \newline

\textbf{Ghana Paata } \newline

1. अ॒ग्निः प॒शुः प॒शु र॒ग्नि र॒ग्निः प॒शु रा॑सी दासीत् प॒शु र॒ग्नि र॒ग्निः प॒शु रा॑सीत् । \newline
2. प॒शु रा॑सी दासीत् प॒शुः प॒शु रा॑सी॒त् तेन॒ तेना॑सीत् प॒शुः प॒शु रा॑सी॒त् तेन॑ । \newline
3. आ॒सी॒त् तेन॒ तेना॑सी दासी॒त् तेना॑ यजन्ता यजन्त॒ तेना॑सी दासी॒त् तेना॑ यजन्त । \newline
4. तेना॑ यजन्ता यजन्त॒ तेन॒ तेना॑ यजन्त॒ स सो॑ ऽयजन्त॒ तेन॒ तेना॑ यजन्त॒ सः । \newline
5. अ॒य॒ज॒न्त॒ स सो॑ ऽयजन्ता यजन्त॒ स ए॒त मे॒तꣳ सो॑ ऽयजन्ता यजन्त॒ स ए॒तम् । \newline
6. स ए॒त मे॒तꣳ स स ए॒तम् ॅलो॒कम् ॅलो॒क मे॒तꣳ स स ए॒तम् ॅलो॒कम् । \newline
7. ए॒तम् ॅलो॒कम् ॅलो॒क मे॒त मे॒तम् ॅलो॒क म॑जय दजय ल्लो॒क मे॒त मे॒तम् ॅलो॒क म॑जयत् । \newline
8. लो॒क म॑जय दजय ल्लो॒कम् ॅलो॒क म॑जय॒द् यस्मि॒न्॒. यस्मि॑न् नजय ल्लो॒कम् ॅलो॒क म॑जय॒द् यस्मिन्न्॑ । \newline
9. अ॒ज॒य॒द् यस्मि॒न्॒. यस्मि॑न् नजय दजय॒द् यस्मि॑न् न॒ग्नि र॒ग्निर् यस्मि॑न् नजय दजय॒द् यस्मि॑न् न॒ग्निः । \newline
10. यस्मि॑न् न॒ग्नि र॒ग्निर् यस्मि॒न्॒. यस्मि॑न् न॒ग्निः स सो᳚ ऽग्निर् यस्मि॒न्॒. यस्मि॑न् न॒ग्निः सः । \newline
11. अ॒ग्निः स सो᳚ ऽग्नि र॒ग्निः स ते॑ ते॒ सो᳚ ऽग्नि र॒ग्निः स ते᳚ । \newline
12. स ते॑ ते॒ स स ते॑ लो॒को लो॒क स्ते॒ स स ते॑ लो॒कः । \newline
13. ते॒ लो॒को लो॒क स्ते॑ ते लो॒क स्तम् तम् ॅलो॒क स्ते॑ ते लो॒क स्तम् । \newline
14. लो॒क स्तम् तम् ॅलो॒को लो॒क स्तम् जे᳚ष्यसि जेष्यसि॒ तम् ॅलो॒को लो॒क स्तम् जे᳚ष्यसि । \newline
15. तम् जे᳚ष्यसि जेष्यसि॒ तम् तम् जे᳚ष्य॒ स्यथाथ॑ जेष्यसि॒ तम् तम् जे᳚ष्य॒स्यथ॑ । \newline
16. जे॒ष्य॒ स्यथाथ॑ जेष्यसि जेष्य॒ स्यथावा वाथ॑ जेष्यसि जेष्य॒ स्यथाव॑ । \newline
17. अथावावा थाथाव॑ जिघ्र जि॒घ्रावा थाथाव॑ जिघ्र । \newline
18. अव॑ जिघ्र जि॒घ्रा वाव॑ जिघ्र वा॒युर् वा॒युर् जि॒घ्रा वाव॑ जिघ्र वा॒युः । \newline
19. जि॒घ्र॒ वा॒युर् वा॒युर् जि॑घ्र जिघ्र वा॒युः प॒शुः प॒शुर् वा॒युर् जि॑घ्र जिघ्र वा॒युः प॒शुः । \newline
20. वा॒युः प॒शुः प॒शुर् वा॒युर् वा॒युः प॒शु रा॑सी दासीत् प॒शुर् वा॒युर् वा॒युः प॒शु रा॑सीत् । \newline
21. प॒शु रा॑सी दासीत् प॒शुः प॒शु रा॑सी॒त् तेन॒ तेना॑सीत् प॒शुः प॒शु रा॑सी॒त् तेन॑ । \newline
22. आ॒सी॒त् तेन॒ तेना॑सी दासी॒त् तेना॑यजन्ता यजन्त॒ तेना॑सी दासी॒त् तेना॑ यजन्त । \newline
23. तेना॑ यजन्ता यजन्त॒ तेन॒ तेना॑ यजन्त॒ स सो॑ ऽयजन्त॒ तेन॒ तेना॑ यजन्त॒ सः । \newline
24. अ॒य॒ज॒न्त॒ स सो॑ ऽयजन्ता यजन्त॒ स ए॒त मे॒तꣳ सो॑ ऽयजन्ता यजन्त॒ स ए॒तम् । \newline
25. स ए॒त मे॒तꣳ स स ए॒तम् ॅलो॒कम् ॅलो॒क मे॒तꣳ स स ए॒तम् ॅलो॒कम् । \newline
26. ए॒तम् ॅलो॒कम् ॅलो॒क मे॒त मे॒तम् ॅलो॒क म॑जय दजय ल्लो॒क मे॒त मे॒तम् ॅलो॒क म॑जयत् । \newline
27. लो॒क म॑जय दजय ल्लो॒कम् ॅलो॒क म॑जय॒द् यस्मि॒न्॒. यस्मि॑न् नजय ल्लो॒कम् ॅलो॒क म॑जय॒द् यस्मिन्न्॑ । \newline
28. अ॒ज॒य॒द् यस्मि॒न्॒. यस्मि॑न् नजय दजय॒द् यस्मि॑न् वा॒युर् वा॒युर् यस्मि॑न् नजय दजय॒द् यस्मि॑न् वा॒युः । \newline
29. यस्मि॑न् वा॒युर् वा॒युर् यस्मि॒न्॒. यस्मि॑न् वा॒युः स स वा॒युर् यस्मि॒न्॒. यस्मि॑न् वा॒युः सः । \newline
30. वा॒युः स स वा॒युर् वा॒युः स ते॑ ते॒ स वा॒युर् वा॒युः स ते᳚ । \newline
31. स ते॑ ते॒ स स ते॑ लो॒को लो॒क स्ते॒ स स ते॑ लो॒कः । \newline
32. ते॒ लो॒को लो॒क स्ते॑ ते लो॒क स्तस्मा॒त् तस्मा᳚ ल्लो॒क स्ते॑ ते लो॒क स्तस्मा᳚त् । \newline
33. लो॒क स्तस्मा॒त् तस्मा᳚ ल्लो॒को लो॒क स्तस्मा᳚त् त्वा त्वा॒ तस्मा᳚ ल्लो॒को लो॒क स्तस्मा᳚त् त्वा । \newline
34. तस्मा᳚त् त्वा त्वा॒ तस्मा॒त् तस्मा᳚त् त्वा॒ ऽन्त र॒न्त स्त्वा॒ तस्मा॒त् तस्मा᳚त् त्वा॒ ऽन्तः । \newline
35. त्वा॒ ऽन्त र॒न्त स्त्वा᳚ त्वा॒ ऽन्त रे᳚ष्या म्येष्या म्य॒न्त स्त्वा᳚ त्वा॒ ऽन्त रे᳚ष्यामि । \newline
36. अ॒न्त रे᳚ष्या म्येष्या म्य॒न्त र॒न्त रे᳚ष्यामि॒ यदि॒ यद्ये᳚ष्या म्य॒न्त र॒न्त रे᳚ष्यामि॒ यदि॑ । \newline
37. ए॒ष्या॒मि॒ यदि॒ यद्ये᳚ष्या म्येष्यामि॒ यदि॒ न न यद्ये᳚ष्या म्येष्यामि॒ यदि॒ न । \newline
38. यदि॒ न न यदि॒ यदि॒ नाव॒जिघ्र॑ स्यव॒जिघ्र॑सि॒ न यदि॒ यदि॒ नाव॒जिघ्र॑सि । \newline
39. नाव॒जिघ्र॑ स्यव॒जिघ्र॑सि॒ न नाव॒जिघ्र॑ स्यादि॒त्य आ॑दि॒त्यो॑ ऽव॒जिघ्र॑सि॒ न नाव॒जिघ्र॑ स्यादि॒त्यः । \newline
40. अ॒व॒जिघ्र॑ स्यादि॒त्य आ॑दि॒त्यो॑ ऽव॒जिघ्र॑ स्यव॒जिघ्र॑ स्यादि॒त्यः प॒शुः प॒शु रा॑दि॒त्यो॑ ऽव॒जिघ्र॑ स्यव॒जिघ्र॑ स्यादि॒त्यः प॒शुः । \newline
41. अ॒व॒जिघ्र॒सीत्य॑व - जिघ्र॑सि । \newline
42. आ॒दि॒त्यः प॒शुः प॒शु रा॑दि॒त्य आ॑दि॒त्यः प॒शु रा॑सी दासीत् प॒शु रा॑दि॒त्य आ॑दि॒त्यः प॒शु रा॑सीत् । \newline
43. प॒शु रा॑सी दासीत् प॒शुः प॒शु रा॑सी॒त् तेन॒ तेना॑सीत् प॒शुः प॒शु रा॑सी॒त् तेन॑ । \newline
44. आ॒सी॒त् तेन॒ तेना॑सी दासी॒त् तेना॑यजन्ता यजन्त॒ तेना॑ सीदासी॒त् तेना॑ यजन्त । \newline
45. तेना॑ यजन्ता यजन्त॒ तेन॒ तेना॑ यजन्त॒ स सो॑ ऽयजन्त॒ तेन॒ तेना॑ यजन्त॒ सः । \newline
46. अ॒य॒ज॒न्त॒ स सो॑ ऽयजन्ता यजन्त॒ स ए॒त मे॒तꣳ सो॑ ऽयजन्ता यजन्त॒ स ए॒तम् । \newline
47. स ए॒त मे॒तꣳ स स ए॒तम् ॅलो॒कम् ॅलो॒क मे॒तꣳ स स ए॒तम् ॅलो॒कम् । \newline
48. ए॒तम् ॅलो॒कम् ॅलो॒क मे॒त मे॒तम् ॅलो॒क म॑जय दजय ल्लो॒क मे॒त मे॒तम् ॅलो॒क म॑जयत् । \newline
49. लो॒क म॑जय दजय ल्लो॒कम् ॅलो॒क म॑जय॒द् यस्मि॒न्॒. यस्मि॑न् नजय ल्लो॒कम् ॅलो॒क म॑जय॒द् यस्मिन्न्॑ । \newline
50. अ॒ज॒य॒द् यस्मि॒न्॒. यस्मि॑न् नजय दजय॒द् यस्मि॑न् नादि॒त्य आ॑दि॒त्यो यस्मि॑न् नजय दजय॒द् यस्मि॑न् नादि॒त्यः । \newline
51. यस्मि॑न् नादि॒त्य आ॑दि॒त्यो यस्मि॒न्॒. यस्मि॑न् नादि॒त्यः स स आ॑दि॒त्यो यस्मि॒न्॒. यस्मि॑न् नादि॒त्यः सः । \newline
52. आ॒दि॒त्यः स स आ॑दि॒त्य आ॑दि॒त्यः स ते॑ ते॒ स आ॑दि॒त्य आ॑दि॒त्यः स ते᳚ । \newline
53. स ते॑ ते॒ स स ते॑ लो॒को लो॒क स्ते॒ स स ते॑ लो॒कः । \newline
54. ते॒ लो॒को लो॒क स्ते॑ ते लो॒क स्तम् तम् ॅलो॒क स्ते॑ ते लो॒क स्तम् । \newline
55. लो॒क स्तम् तम् ॅलो॒को लो॒क स्तम् जे᳚ष्यसि जेष्यसि॒ तम् ॅलो॒को लो॒क स्तम् जे᳚ष्यसि । \newline
56. तम् जे᳚ष्यसि जेष्यसि॒ तम् तम् जे᳚ष्यसि॒ यदि॒ यदि॑ जेष्यसि॒ तम् तम् जे᳚ष्यसि॒ यदि॑ । \newline
57. जे॒ष्य॒सि॒ यदि॒ यदि॑ जेष्यसि जेष्यसि॒ यद्य॑व॒जिघ्र॑ स्यव॒जिघ्र॑सि॒ यदि॑ जेष्यसि जेष्यसि॒ यद्य॑व॒जिघ्र॑सि । \newline
58. यद्य॑व॒जिघ्र॑ स्यव॒जिघ्र॑सि॒ यदि॒ यद्य॑व॒जिघ्र॑सि । \newline
59. अ॒व॒जिघ्र॒सीत्य॑व - जिघ्र॑सि । \newline
\pagebreak


\end{document}