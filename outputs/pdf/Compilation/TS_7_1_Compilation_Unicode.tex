\documentclass[17pt]{extarticle}
\usepackage{babel}
\usepackage{fontspec}
\usepackage{polyglossia}
\usepackage{extsizes}

\usepackage{color}   %May be necessary if you want to color links
\usepackage{hyperref}
\hypersetup{
    colorlinks=true, %set true if you want colored links
    linktoc=all,     %set to all if you want both sections and subsections linked
    linkcolor=black,  %choose some color if you want links to stand out
}

\setmainlanguage{sanskrit}
\setotherlanguages{english} %% or other languages
\setlength{\parindent}{0pt}
\pagestyle{myheadings}
\newfontfamily\devanagarifont[Script=Devanagari]{AdishilaVedic}
\renewcommand{\theHsection}{\thepart.section.\thesection}

\newcommand{\VAR}[1]{}
\newcommand{\BLOCK}[1]{}




\begin{document}
\begin{titlepage}
    \begin{center}
 
\begin{sanskrit}
    { \Large
    कृष्ण यजुर्वेदीय तैत्तिरीय संहिता,पद,जटा,घन पाठः 
    }
    \\
    \vspace{2.5cm}
    \mbox{ \Large
    7.1      सप्तमकाण्डे प्रथमः प्रश्नः- अश्वमेधगतमन्त्राणामभिधानं   }
\end{sanskrit}
\end{center}

\end{titlepage}
\tableofcontents
\phantomsection
\pagebreak

\markright{ TS 7.1.1.1  \hfill https://www.vedavms.in \hfill}

\section{ TS 7.1.1.1 }

\textbf{TS 7.1.1.1 } \newline
\textbf{Samhita Paata} \newline

प्र॒जन॑नं॒ ज्योति॑र॒ग्नि-र्दे॒वता॑नां॒ ज्योति॑र्वि॒राट् छन्द॑सां॒ ज्योति॑र्वि॒राड् वा॒चो᳚ऽग्नौ सं ति॑ष्ठते वि॒राज॑म॒भि संप॑द्यते॒ तस्मा॒त् तज्ज्योति॑रुच्यते॒ द्वौ स्तोमौ᳚ प्रातस्सव॒नं ॅव॑हतो॒ यथा᳚ प्रा॒णश्चा॑ऽपा॒नश्च॒ द्वौ माद्ध्य॑दिंनꣳ॒॒ सव॑नं॒ ॅयथा॒ चक्षु॑श्च॒ श्रोत्रं॑ च॒ द्वौ तृ॑तीयसव॒नं ॅयथा॒ वाक् च॑ प्रति॒ष्ठा च॒ पुरु॑षसंमितो॒ वा ए॒ष य॒ज्ञोऽस्थू॑रि॒ - [  ] \newline

\textbf{Pada Paata} \newline

प्र॒जन॑न॒मिति॑ प्र-जन॑नम् । ज्योतिः॑ । अ॒ग्निः । दे॒वता॑नाम् । ज्योतिः॑ । वि॒राडिति॑ वि - राट् । छन्द॑साम् । ज्योतिः॑ । वि॒राडिति॑ वि - राट् । वा॒चः । अ॒ग्नौ । समिति॑ । ति॒ष्ठ॒ते॒ । वि॒राज॒मिति॑ वि - राज᳚म् । अ॒भि । समिति॑ । प॒द्य॒ते॒ । तस्मा᳚त् । तत् । ज्योतिः॑ । उ॒च्य॒ते॒ । द्वौ । स्तोमौ᳚ । प्रा॒त॒स्स॒व॒नमिति॑ प्रातः - स॒व॒नम् । व॒ह॒तः॒ । यथा᳚ । प्रा॒ण इति॑ प्र - अ॒नः । च॒ । अ॒पा॒न इत्य॑प - अ॒नः । च॒ । द्वौ । माद्ध्य॑न्दिनम् । सव॑नम् । यथा᳚ । चक्षुः॑ । च॒ । श्रोत्र᳚म् । च॒ । द्वौ । तृ॒ती॒य॒स॒व॒नमिति॑ तृतीय - स॒व॒नम् । यथा᳚ । वाक् । च॒ । प्र॒ति॒ष्ठेति॑ प्रति - स्था । च॒ । पुरु॑षसम्मित॒ इति॒ पुरु॑ष - स॒म्मि॒तः॒ । वै । ए॒षः । य॒ज्ञ्ः । अस्थू॑रिः ।  \newline




\markright{ TS 7.1.1.2  \hfill https://www.vedavms.in \hfill}

\section{ TS 7.1.1.2 }

\textbf{TS 7.1.1.2 } \newline
\textbf{Samhita Paata} \newline

र्यं कामं॑ का॒मय॑ते॒ तमे॒तेना॒भ्य॑श्नुते॒ सर्वꣳ॒॒ ह्यस्थु॑रिणाऽभ्यश्नु॒ते᳚ ऽग्निष्टो॒मेन॒ वै प्र॒जाप॑तिः प्र॒जा अ॑सृजत॒ ता अ॑ग्निष्टो॒मेनै॒व पर्य॑गृह्णा॒त् तासां॒ परि॑गृहीताना-मश्वत॒रोऽत्य॑प्रवत॒ तस्या॑नु॒हाय॒रेत॒ आऽद॑त्त॒ तद्-ग॑र्द॒भे न्य॑मा॒र्ट् तस्मा᳚द्-गर्द॒भो द्वि॒रेता॒ अथो॑ आहु॒र्वड॑बायां॒ न्य॑मा॒र्डिति॒ तस्मा॒द्-वड॑बा द्वि॒रेता॒ अथो॑ आहु॒रोष॑धीषु॒ - [  ] \newline

\textbf{Pada Paata} \newline

यम् । काम᳚म् । का॒मय॑ते । तम् । ए॒तेन॑ । अ॒भीति॑ । अ॒श्नु॒ते॒ । सर्व᳚म् । हि । अस्थू॑रिणा । अ॒भ्य॒श्नु॒त इत्य॑भि - अ॒श्नु॒ते । अ॒ग्नि॒ष्टो॒मेनेत्य॑ग्नि - स्तो॒मेन॑ । वै । प्र॒जाप॑ति॒रिति॑ प्र॒जा - प॒तिः॒ । प्र॒जा इति॑ प्र-जाः । अ॒सृ॒ज॒त॒ । ताः । अ॒ग्नि॒ष्टो॒मेनेत्य॑ग्नि-स्तो॒मेन॑ । ए॒व । परीति॑ । अ॒गृ॒ह्णा॒त् । तासा᳚म् । परि॑गृहीताना॒मिति॒ परि॑ - गृ॒ही॒ता॒ना॒म् । अ॒श्व॒त॒रः । अतीति॑ । अ॒प्र॒व॒त॒ । तस्य॑ । अ॒नु॒हायेत्य॑नु - हाय॑ । रेतः॑ । एति॑ । अ॒द॒त्त॒ । तत् । ग॒र्द॒भे । नीति॑ । अ॒मा॒र्ट् । तस्मा᳚त् । ग॒र्द॒भः । द्वि॒रेता॒ इति॑ द्वि - रेताः᳚ । अथो॒ इति॑ । आ॒हुः॒ । वड॑बायाम् । नीति॑ । अ॒मा॒र्ट्॒ । इति॑ । तस्मा᳚त् । वड॑बा । द्वि॒रेता॒ इति॑ द्वि - रेताः᳚ । अथो॒ इति॑ । आ॒हुः॒ । ओष॑धीषु ।  \newline




\markright{ TS 7.1.1.3  \hfill https://www.vedavms.in \hfill}

\section{ TS 7.1.1.3 }

\textbf{TS 7.1.1.3 } \newline
\textbf{Samhita Paata} \newline

न्य॑मा॒र्डिति॒ तस्मा॒दोष॑ध॒यो ऽन॑भ्यक्ता रेभ॒न्त्यथो॑ आहुः प्र॒जासु॒ न्य॑मा॒र्डिति॒ तस्मा᳚द्-य॒मौ जा॑येते॒ तस्मा॑दश्वत॒रो न प्र जा॑यत॒ आत्त॑रेता॒ हि तस्मा᳚द् ब॒र्॒.हिष्यन॑वक्लृप्तः सर्ववेद॒से वा॑ स॒हस्रे॒ वाऽव॑ क्लृ॒प्तोऽति॒ ह्यप्र॑वत॒ य ए॒वं ॅवि॒द्वान॑ग्निष्टो॒मेन॒ यज॑ते॒ प्राजा॑ताः प्र॒जा ज॒नय॑ति॒ परि॒ प्रजा॑ता गृह्णाति॒ तस्मा॑दाहुर्ज्येष्ठय॒ज्ञ् इति॑ - [  ] \newline

\textbf{Pada Paata} \newline

नीति॑ । अ॒मा॒र्ट॒ । इति॑ । तस्मा᳚त् । ओष॑धयः । अन॑भ्यक्ता॒ इत्यन॑भि - अ॒क्ताः॒ । रे॒भ॒न्ति॒ । अथो॒ इति॑ । आ॒हुः॒ । प्र॒जास्विति॑ प्र - जासु॑ । नीति॑ । अ॒मा॒र्ट्॒ । इति॑ । तस्मा᳚त् । य॒मौ । जा॒ये॒ते॒ इति॑ । तस्मा᳚त् । अ॒श्व॒त॒रः । न । प्रेति॑ । जा॒य॒ते॒ । आत्त॑रेता॒ इत्यात्त॑ - रे॒ताः॒ । हि । तस्मा᳚त् । ब॒र्॒.हिषि॑ । अन॑वक्लृप्त॒ इत्यन॑व - क्लृ॒प्तः॒ । स॒र्व॒वे॒द॒स इति॑ सर्व-वे॒द॒से । वा॒ । स॒हस्रे᳚ । वा॒ । अव॑क्लृप्त॒ इत्यव॑-क्लृ॒प्तः॒ । अतीति॑ । हि । अप्र॑वत । यः । ए॒वम् । वि॒द्वान् । अ॒ग्नि॒ष्टो॒मेनेत्य॑ग्नि- स्तो॒मेन॑ । यज॑ते । प्रेति॑ । अजा॑ताः । प्र॒जा इति॑ प्र-जाः । ज॒नय॑ति । परीति॑ । प्रजा॑ता॒ इति॒ प्र - जा॒ताः॒ । गृ॒ह्णा॒ति॒ । तस्मा᳚त् । आ॒हुः॒ । ज्ये॒ष्ठ॒य॒ज्ञ् इति॑ ज्येष्ठ - य॒ज्ञ्ः । इति॑ ।  \newline




\markright{ TS 7.1.1.4  \hfill https://www.vedavms.in \hfill}

\section{ TS 7.1.1.4 }

\textbf{TS 7.1.1.4 } \newline
\textbf{Samhita Paata} \newline

प्र॒जाप॑ति॒र्वाव ज्येष्ठः॒ स ह्ये॑तेनाग्रेऽय॑जत प्र॒जाप॑तिरकामयत॒ प्र जा॑ये॒येति॒ स मु॑ख॒तस्त्रि॒वृतं॒ निर॑मिमीत॒ तम॒ग्निर्दे॒वता ऽन्व॑सृज्यत गाय॒त्री छन्दो॑ रथन्त॒रꣳ साम॑ ब्राह्म॒णो म॑नु॒ष्या॑णाम॒जः प॑शू॒नां तस्मा॒त् ते मुख्या॑ मुख॒तो ह्यसृ॑ज्य॒न्तोर॑सो बा॒हुभ्यां᳚ पञ्चद॒शं निर॑मिमीत॒ तमिन्द्रो॑ दे॒वता ऽन्व॑सृज्यत त्रि॒ष्टुप् छन्दो॑ बृ॒हथ् - [  ] \newline

\textbf{Pada Paata} \newline

प्र॒जाप॑ति॒रिति॑ प्र॒जा - प॒तिः॒ । वाव । ज्येष्ठः॑ । सः । हि । ए॒तेन॑ । अग्रे᳚ । अय॑जत । प्र॒जाप॑ति॒रिति॑ प्र॒जा - प॒तिः॒ । अ॒का॒म॒य॒त॒ । प्रेति॑ । जा॒ये॒य॒ । इति॑ । सः । मु॒ख॒तः । त्रि॒वृत॒मिति॑ त्रि - वृत᳚म् । निरिति॑ । अ॒मि॒मी॒त॒ । तम् । अ॒ग्निः । दे॒वता᳚ । अन्विति॑ । अ॒सृ॒ज्य॒त॒ । गा॒य॒त्री । छन्दः॑ । र॒थ॒न्त॒रमिति॑ रथं - त॒रम् । साम॑ । ब्रा॒ह्म॒णः । म॒नु॒ष्या॑णाम् । अ॒जः । प॒शू॒नाम् । तस्मा᳚त् । ते । मुख्याः᳚ । मु॒ख॒तः । हि । असृ॑ज्यन्त । उर॑सः । बा॒हुभ्या॒मिति॑ बा॒हु - भ्या॒म् । प॒ञ्च॒द॒शमिति॑ पञ्च - द॒शम् । निरिति॑ । अ॒मि॒मी॒त॒ । तम् । इन्द्रः॑ । दे॒वता᳚ । अन्विति॑ । अ॒सृ॒ज्य॒त॒ । त्रि॒ष्टुप् । छन्दः॑ । बृ॒हत् ।  \newline




\markright{ TS 7.1.1.5  \hfill https://www.vedavms.in \hfill}

\section{ TS 7.1.1.5 }

\textbf{TS 7.1.1.5 } \newline
\textbf{Samhita Paata} \newline

साम॑ राज॒न्यो॑ मनु॒ष्या॑णा॒मविः॑ पशू॒नां तस्मा॒त् ते वी॒र्या॑वन्तो वी॒र्या᳚द्ध्यसृ॑ज्यन्त मद्ध्य॒तः स॑प्तद॒शं निर॑मिमीत॒ तं ॅविश्वे॑ दे॒वा दे॒वता॒ अन्व॑सृज्यन्त॒ जग॑ती॒ छन्दो॑ वै रू॒पꣳ साम॒ वैश्यो॑ मनु॒ष्या॑णां॒ गावः॑ पशू॒नां तस्मा॒त् त आ॒द्या॑ अन्न॒धाना॒द्ध्य सृ॑ज्यन्त॒ तस्मा॒द्-भूयाꣳ॑सो॒ ऽन्येभ्यो॒ भूयि॑ष्ठा॒ हि दे॒वता॒ अन्वसृ॑ज्यन्त प॒त्त ए॑कविꣳ॒॒ शं निर॑मिमीत॒ तम॑नु॒ष्टुप् छन्दो - [  ] \newline

\textbf{Pada Paata} \newline

साम॑ । रा॒ज॒न्यः॑ । म॒नु॒ष्या॑णाम् । अविः॑ । प॒शू॒नाम् । तस्मा᳚त् । ते । वी॒र्या॑वन्त॒ इति॑ वी॒र्य॑ - व॒न्तः॒ । वी॒र्या᳚त् । हि । असृ॑ज्यन्त । म॒द्ध्य॒तः । स॒प्त॒द॒शमिति॑ सप्त - द॒शम् । निरिति॑ । अ॒मि॒मी॒त॒ । तम् । विश्वे᳚ । दे॒वाः । दे॒वताः᳚ । अन्विति॑ । अ॒सृ॒ज्य॒न्त॒ । जग॑ती । छन्दः॑ । वै॒रू॒पम् । साम॑ । वैश्यः॑ । म॒नु॒ष्या॑णाम् । गावः॑ । प॒शू॒नाम् । तस्मा᳚त् । ते । आ॒द्याः᳚ । अ॒न्न॒धाना॒दित्य॑न्न - धाना᳚त् । हि । असृ॑ज्यन्त । तस्मा᳚त् । भूयाꣳ॑सः । अ॒न्येभ्यः॑ । भूयि॑ष्ठाः । हि । दे॒वताः᳚ । अन्विति॑ । असृ॑ज्यन्त । प॒त्तः । ए॒क॒विꣳ॒॒शमित्ये॑क - विꣳ॒॒शम् । निरिति॑ । अ॒मि॒मी॒त॒ । तम् । अ॒नु॒ष्टुबित्य॑नु - स्तुप् । छन्दः॑ ।  \newline




\markright{ TS 7.1.1.6  \hfill https://www.vedavms.in \hfill}

\section{ TS 7.1.1.6 }

\textbf{TS 7.1.1.6 } \newline
\textbf{Samhita Paata} \newline

ऽन्व॑सृज्यत वैरा॒जꣳ साम॑ शू॒द्रो म॑नु॒ष्या॑णा॒मश्वः॑ पशू॒नां तस्मा॒त् तौ भू॑तसं क्रा॒मिणा॒वश्व॑श्च शू॒द्रश्च॒ तस्मा᳚च्छू॒द्रो य॒ज्ञेऽन॑वक्लृप्तो॒ न हि दे॒वता॒ अन्वसृ॑ज्यत॒ तस्मा॒त् पादा॒वुप॑ जीवतः प॒त्तो ह्यसृ॑ज्येतां प्रा॒णा वै त्रि॒वृद॑र्द्धमा॒साः प॑ञ्चद॒शः प्र॒जाप॑तिः सप्तद॒शस्त्रय॑ इ॒मे लो॒का अ॒सावा॑दि॒त्य ए॑कविꣳ॒॒श ए॒तस्मि॒न् वा ए॒ते श्रि॒ता ए॒तस्मि॒न् प्रति॑ष्ठिता॒ ( ) य ए॒वं ॅवेदै॒तस्मि॑न्ने॒व श्र॑यत ए॒तस्मि॒न् प्रति॑ तिष्ठति ॥ \newline

\textbf{Pada Paata} \newline

अन्विति॑ । अ॒सृ॒ज्य॒त॒ । वै॒रा॒जम् । साम॑ । शू॒द्रः । म॒नु॒ष्या॑णाम् । अश्वः॑ । प॒शू॒नाम् । तस्मा᳚त् । तौ । भू॒त॒स॒ङ्क्रा॒मिणा॒विति॑ भूत - स॒ङ्क्रा॒मिणौ᳚ । अश्वः॑ । च॒ । शू॒द्रः । च॒ । तस्मा᳚त् । शू॒द्रः । य॒ज्ञे । अन॑वक्लृप्त॒ इत्यन॑व - क्लृ॒प्तः॒ । न । हि । दे॒वताः᳚ । अन्विति॑ । असृ॑ज्यत । तस्मा᳚त् । पादौ᳚ । उपेति॑ । जी॒व॒तः॒ । प॒त्तः । हि । असृ॑ज्येताम् । प्रा॒णा इति॑ प्र - अ॒नाः । वै । त्रि॒वृदिति॑ त्रि - वृत् । अ॒द्‌र्ध॒मा॒सा इत्य॑द्‌र्ध - मा॒साः । प॒ञ्च॒द॒श इति॑ पञ्च - द॒शः । प्र॒जाप॑ति॒रिति॑ प्र॒जा - प॒तिः॒ । स॒प्त॒द॒श इति॑ सप्त - द॒शः । त्रयः॑ । इ॒मे । लो॒काः । अ॒सौ । आ॒दि॒त्यः । ए॒क॒विꣳ॒॒श इत्ये॑क - विꣳ॒॒शः । ए॒तस्मिन्न्॑ । वै । ए॒ते । श्रि॒ताः । ए॒तस्मिन्न्॑ । प्रति॑ष्ठिता॒ इति॒ प्रति॑ - स्थि॒ताः॒ ( ) । यः । ए॒वम् । वेद॑ । ए॒तस्मिन्न्॑ । ए॒व । श्र॒य॒ते॒ । ए॒तस्मिन्न्॑ । प्रतीति॑ । ति॒ष्ठ॒ति॒ ॥  \newline




\markright{ TS 7.1.2.1  \hfill https://www.vedavms.in \hfill}

\section{ TS 7.1.2.1 }

\textbf{TS 7.1.2.1 } \newline
\textbf{Samhita Paata} \newline

प्रा॒त॒स्स॒व॒ने वै गा॑य॒त्रेण॒ छन्द॑सा त्रि॒वृते॒ स्तोमा॑य॒ ज्योति॒र्दध॑देति त्रि॒वृता᳚ ब्रह्मवर्च॒सेन॑ पञ्चद॒शाय॒ ज्योति॒र्दध॑देति पञ्चद॒शेनौज॑सा वी॒र्ये॑ण सप्तद॒शाय॒ ज्योति॒र्दध॑देति सप्तद॒शेन॑ प्राजाप॒त्येन॑ प्र॒जन॑नेनैकविꣳ॒॒शाय॒ ज्योति॒र्दध॑देति॒ स्तोम॑ ए॒व तथ् स्तोमा॑य॒ ज्योति॒र्दध॑दे॒त्यथो॒ स्तोम॑ ए॒व स्तोम॑म॒भि प्र ण॑यति॒ याव॑न्तो॒ वै स्तोमा॒स्ताव॑न्तः॒ कामा॒स्ताव॑न्तो लो॒का ( ) -स्ताव॑न्ति॒ ज्योतीꣳ॑ष्ये॒ताव॑त ए॒व स्तोमा॑ने॒ताव॑तः॒ कामा॑ने॒ताव॑तो लो॒काने॒ताव॑न्ति॒ ज्योतीꣳ॒॒ष्यव॑ रुन्धे ॥ \newline

\textbf{Pada Paata} \newline

प्रा॒त॒स्स॒व॒न इति॑ प्रातः - स॒व॒ने । वै । गा॒य॒त्रेण॑ । छन्द॑सा । त्रि॒वृत॒ इति॑ त्रि - वृते᳚ । स्तोमा॑य । ज्योतिः॑ । दध॑त् । ए॒ति॒ । त्रि॒वृतेति॑ त्रि - वृता᳚ । ब्र॒ह्म॒व॒र्च॒सेनेति॑ ब्रह्म - व॒र्च॒सेन॑ । प॒ञ्च॒द॒शायेति॑ पञ्च - द॒शाय॑ । ज्योतिः॑ । दध॑त् । ए॒ति॒ । प॒ञ्च॒द॒शेनेति॑ पञ्च-द॒शेन॑ । ओज॑सा । वी॒र्ये॑ण । स॒प्त॒द॒शायेति॑ सप्त - द॒शाय॑ । ज्योतिः॑ । दध॑त् । ए॒ति॒ । स॒प्त॒द॒शेनेति॑ सप्त - द॒शेन॑ । प्रा॒जा॒प॒त्येनेति॑ प्राजा - प॒त्येन॑ । प्र॒जन॑ने॒नेति॑ प्र - जन॑नेन । ए॒क॒विꣳ॒॒शायेत्ये॑क - विꣳ॒॒शाय॑ । ज्योतिः॑ । दध॑त् । ए॒ति॒ । स्तोमः॑ । ए॒व । तत् । स्तोमा॑य । ज्योतिः॑ । दध॑त् । ए॒ति॒ । अथो॒ इति॑ । स्तोमे᳚ । ए॒व । स्तोम᳚म् । अ॒भि । प्रेति॑ । न॒य॒ति॒ । याव॑न्तः । वै । स्तोमाः᳚ । ताव॑न्तः । कामाः᳚ । ताव॑न्तः । लो॒काः ( ) । ताव॑न्ति । ज्योतीꣳ॑षि । ए॒ताव॑तः । ए॒व । स्तोमान्॑ । ए॒ताव॑तः । कामान्॑ । ए॒ताव॑तः । लो॒कान् । ए॒ताव॑न्ति । ज्योतीꣳ॑षि । अवेति॑ । रु॒न्धे॒ ॥  \newline




\markright{ TS 7.1.3.1  \hfill https://www.vedavms.in \hfill}

\section{ TS 7.1.3.1 }

\textbf{TS 7.1.3.1 } \newline
\textbf{Samhita Paata} \newline

ब्र॒ह्म॒वा॒दिनो॑ वदन्ति॒ स त्वै य॑जेत॒ यो᳚ऽग्निष्टो॒मेन॒ यज॑मा॒नोऽथ॒ सर्व॑स्तोमेन॒ यजे॒तेति॒ यस्य॑ त्रि॒वृत॑मन्त॒र्यन्ति॑ प्रा॒णाꣳ-स्तस्या॒न्तर्य॑न्ति प्रा॒णेषु॒ मेऽप्य॑स॒दिति॒ खलु॒ वै य॒ज्ञेन॒ यज॑मानो यजते॒ यस्य॑ पञ्चद॒शम॑न्त॒र्यन्ति॑ वी॒र्यं॑ तस्या॒न्तर्य॑न्ति वी॒र्ये॑ मेऽप्य॑स॒दिति॒ खलु॒ वै य॒ज्ञेन॒ यज॑मानो यजते॒ यस्य॑ सप्तद॒श-म॑न्त॒र्यन्ति॑ - [  ] \newline

\textbf{Pada Paata} \newline

ब्र॒ह्म॒वा॒दिन॒ इति॑ ब्रह्म - वा॒दिनः॑ । व॒द॒न्ति॒ । सः । तु । वै । य॒जे॒त॒ । यः । अ॒ग्नि॒ष्टो॒मेनेत्य॑ग्नि - स्तो॒मेन॑ । यज॑मानः । अथ॑ । सर्व॑स्तोमे॒नेति॒ सर्व॑ - स्तो॒मे॒न॒ । यजे॑त । इति॑ । यस्य॑ । त्रि॒वृत॒मिति॑ त्रि - वृत᳚म् । अ॒न्त॒र्यन्तीत्य॑न्तः-यन्ति॑ । प्रा॒णानिति॑ प्र - अ॒नान् । तस्य॑ । अ॒न्तः । य॒न्ति॒ । प्रा॒णेष्विति॑ प्र - अ॒नेषु॑ । मे॒ । अपीति॑ । अ॒स॒त् । इति॑ । खलु॑ । वै । य॒ज्ञेन॑ । यज॑मानः । य॒ज॒ते॒ । यस्य॑ । प॒ञ्च॒द॒शमिति॑ पञ्च - द॒शम् । अ॒न्त॒र्यन्तीत्य॑न्तः - यन्ति॑ । वी॒र्य᳚म् । तस्य॑ । अ॒न्तः । य॒न्ति॒ । वी॒र्ये᳚ । मे॒ । अपीति॑ । अ॒स॒त् । इति॑ । खलु॑ । वै । य॒ज्ञेन॑ । यज॑मानः । य॒ज॒ते॒ । यस्य॑ । स॒प्त॒द॒शमिति॑ सप्त - द॒शम् । अ॒न्त॒र्यन्तीत्य॑न्तः - यन्ति॑ ।  \newline




\markright{ TS 7.1.3.2  \hfill https://www.vedavms.in \hfill}

\section{ TS 7.1.3.2 }

\textbf{TS 7.1.3.2 } \newline
\textbf{Samhita Paata} \newline

प्र॒जां तस्या॒न्तर्य॑न्ति प्र॒जायां॒ मेऽप्य॑स॒दिति॒ खलु॒ वै य॒ज्ञेन॒ यज॑मानो यजते॒ यस्यै॑कविꣳ॒॒शम॑न्त॒र्यन्ति॑ प्रति॒ष्ठां तस्या॒न्तर्य॑न्ति प्रति॒ष्ठायां॒ मेऽप्य॑स॒दिति॒ खलु॒ वै य॒ज्ञेन॒ यज॑मानो यजते॒ यस्य॑ त्रिण॒वम॑न्त॒र्यन्त्यृ॒तूꣳश्च॒ तस्य॑ नक्ष॒त्रियां᳚ च वि॒राज॑म॒न्तर्य॑न्त्यृ॒तुषु॒ मेऽप्य॑सन्नक्ष॒त्रिया॑यां च वि॒राजीति॒ - [  ] \newline

\textbf{Pada Paata} \newline

प्र॒जामिति॑ प्र - जाम् । तस्य॑ । अ॒न्तः । य॒न्ति॒ । प्र॒जाया॒मिति॑ प्र - जाया᳚म् । मे॒ । अपीति॑ । अ॒स॒त् । इति॑ । खलु॑ । वै । य॒ज्ञेन॑ । यज॑मानः । य॒ज॒ते॒ । यस्य॑ । ए॒क॒विꣳ॒॒शमित्ये॑क - विꣳ॒॒शम् । अ॒न्त॒र्यन्तित्य॑न्तः - यन्ति॑ । प्र॒ति॒ष्ठामिति॑ प्रति - स्थाम् । तस्य॑ । अ॒न्तः । य॒न्ति॒ । प्र॒ति॒ष्ठाया॒मिति॑ प्रति - स्थाया᳚म् । मे॒ । अपीति॑ । अ॒स॒त् । इति॑ । खलु॑ । वै । य॒ज्ञेन॑ । यज॑मानः । य॒ज॒ते॒ । यस्य॑ । त्रि॒ण॒वमिति॑ त्रि - न॒वम् । अ॒न्त॒र्यन्तीत्य॑न्तः - यन्ति॑ । ऋ॒तून् । च॒ । तस्य॑ । न॒क्ष॒त्रिया᳚म् । च॒ । वि॒राज॒मिति॑ वि-राज᳚म् । अ॒न्तः । य॒न्ति॒ । ऋ॒तुषु॑ । मे॒ । अपीति॑ । अ॒स॒त् । न॒क्ष॒त्रिया॑याम् । च॒ । वि॒राजीति॑ वि - राजि॑ । इति॑ ।  \newline




\markright{ TS 7.1.3.3  \hfill https://www.vedavms.in \hfill}

\section{ TS 7.1.3.3 }

\textbf{TS 7.1.3.3 } \newline
\textbf{Samhita Paata} \newline

खलु॒ वै य॒ज्ञेन॒ यज॑मानो यजते॒ यस्य॑ त्रयस्त्रिꣳ॒॒शम॑न्त॒र्यन्ति॑ दे॒वता॒स्तस्या॒न्तर्य॑न्ति दे॒वता॑सु॒ मेऽप्य॑स॒दिति॒ खलु॒ वै य॒ज्ञेन॒ यज॑मानो यजते॒ यो वै स्तोमा॑नामव॒मं प॑र॒मतां॒ गच्छ॑न्तं॒ ॅवेद॑ पर॒मता॑मे॒व ग॑च्छति त्रि॒वृद्वै स्तोमा॑नामव॒मस्त्रि॒वृत् प॑र॒मो य ए॒वं ॅवेद॑ पर॒मता॑मे॒व ग॑च्छति ॥ \newline

\textbf{Pada Paata} \newline

खलु॑ । वै । य॒ज्ञेन॑ । यज॑मानः । य॒ज॒ते॒ । यस्य॑ । त्र॒य॒स्त्रिꣳ॒॒शमिति॑ त्रयः - त्रिꣳ॒॒शम् । अ॒न्त॒र्यन्तित्य॑न्तः - यन्ति॑ । दे॒वताः᳚ । तस्य॑ । अ॒न्तः । य॒न्ति॒ । दे॒वता॑सु । मे॒ । अपीति॑ । अ॒स॒त् । इति॑ । खलु॑ । वै । य॒ज्ञेन॑ । यज॑मानः । य॒ज॒ते॒ । यः । वै । स्तोमा॑नाम् । अ॒व॒मम् । प॒र॒मता᳚म् । गच्छ॑न्तम् । वेद॑ । प॒र॒मता᳚म् । ए॒व । ग॒च्छ॒ति॒ । त्रि॒वृदिति॑ त्रि - वृत् । वै । स्तोमा॑नाम् । अ॒व॒मः । त्रि॒वृदिति॑ त्रि - वृत् । प॒र॒मः । यः । ए॒वम् । वेद॑ । प॒र॒मता᳚म् । ए॒व । ग॒च्छ॒ति॒ ॥  \newline




\markright{ TS 7.1.4.1  \hfill https://www.vedavms.in \hfill}

\section{ TS 7.1.4.1 }

\textbf{TS 7.1.4.1 } \newline
\textbf{Samhita Paata} \newline

अङ्गि॑रसो॒ वै स॒त्रमा॑सत॒ ते सु॑व॒र्गं ॅलो॒कमा॑य॒न् तेषाꣳ॑ ह॒विष्माꣳ॑श्च हवि॒ष्कृच्चा॑ऽहीयेतां॒ ताव॑कामयेताꣳ सुव॒र्गं ॅलो॒कमि॑या॒वेति॒ तावे॒तं द्वि॑रा॒त्रम॑पश्यतां॒ तमाऽह॑रतां॒ तेना॑यजेतां॒ ततो॒ वै तौ सु॑व॒र्गं ॅलो॒कमै॑तां॒ ॅय ए॒वं ॅवि॒द्वान् द्वि॑रा॒त्रेण॒ यज॑ते सुव॒र्गमे॒व लो॒कमे॑ति॒ तावैतां॒ पूर्वे॒णाह्ना ऽग॑च्छता॒मुत्त॑रेणा - [  ] \newline

\textbf{Pada Paata} \newline

अङ्गि॑रसः । वै । स॒त्रम् । आ॒स॒त॒ । ते । सु॒व॒र्गमिति॑ सुवः - गम् । लो॒कम् । आ॒य॒न्न् । तेषा᳚म् । ह॒विष्मान्॑ । च॒ । ह॒वि॒ष्कृदिति॑ हविः - कृत् । च॒ । अ॒ही॒ये॒ता॒म् । तौ । अ॒का॒म॒ये॒ता॒म् । सु॒व॒र्गमिति॑ सुवः - गम् । लो॒कम् । इ॒या॒व॒ । इति॑ । तौ । ए॒तम् । द्वि॒रा॒त्रमिति॑ द्वि - रा॒त्रम् । अ॒प॒श्य॒ता॒म् । तम् । एति॑ । अ॒ह॒र॒ता॒म् । तेन॑ । अ॒य॒जे॒ता॒म् । ततः॑ । वै । तौ । सु॒व॒र्गमिति॑ सुवः - गम् । लो॒कम् । ऐ॒ता॒म् । यः । ए॒वम् । वि॒द्वान् । द्वि॒रा॒त्रेणेति॑ द्वि - रा॒त्रेण॑ । यज॑ते । सु॒व॒र्गमिति॑ सुवः - गम् । ए॒व । लो॒कम् । ए॒ति॒ । तौ । ऐता᳚म् । पूर्वे॑ण । अह्ना᳚ । अग॑च्छताम् । उत्त॑रे॒णेत्युत् - त॒रे॒ण॒ ।  \newline




\markright{ TS 7.1.4.2  \hfill https://www.vedavms.in \hfill}

\section{ TS 7.1.4.2 }

\textbf{TS 7.1.4.2 } \newline
\textbf{Samhita Paata} \newline

-भिप्ल॒वः पूर्व॒मह॑र्भवति॒ गति॒रुत्त॑रं॒ ज्योति॑ष्टोमोऽग्निष्टो॒मः पूर्व॒मह॑र्भवति॒ तेज॒स्तेनाव॑ रुन्धे॒ सर्व॑स्तोमोऽतिरा॒त्र उत्त॑रꣳ॒॒ सर्व॒स्याऽऽ*प्त्यै॒ सर्व॒स्याऽव॑रुद्ध्यै गाय॒त्रं पूर्वेऽह॒न्थ्साम॑ भवति॒ तेजो॒ वै गा॑य॒त्री गा॑य॒त्री ब्र॑ह्मवर्च॒सं तेज॑ ए॒व ब्र॑ह्मवर्च॒समा॒त्मन् ध॑त्ते॒ त्रैष्टु॑भ॒मुत्त॑र॒ ओजो॒ वै वी॒र्यं॑ त्रि॒ष्टुगोज॑ ए॒व वी॒र्य॑मा॒त्मन् ध॑त्ते रथंत॒रं पूर्वे॑ - [  ] \newline

\textbf{Pada Paata} \newline

अ॒भि॒प्ल॒व इत्य॑भि - प्ल॒वः । पूर्व᳚म् । अहः॑ । भ॒व॒ति॒ । गतिः॑ । उत्त॑र॒मित्युत् - त॒र॒म् । ज्योति॑ष्टोम॒ इति॒ ज्योति॑ - स्तो॒मः॒ । अ॒ग्नि॒ष्टो॒म इत्य॑ग्नि-स्तो॒मः । पूर्व᳚म् । अहः॑ । भ॒व॒ति॒ । तेजः॑ । तेन॑ । अवेति॑ । रु॒न्धे॒ । सर्व॑स्तोम॒ इति॒ सर्व॑-स्तो॒मः॒ । अ॒ति॒रा॒त्र इत्य॑ति-रा॒त्रः । उत्त॑र॒मित्युत् - त॒र॒म् । सर्व॑स्य । आप्त्यै᳚ । सर्व॑स्य । अव॑रुद्ध्या॒ इत्यव॑ - रु॒द्ध्यै॒ । गा॒य॒त्रम् । पूर्वे᳚ । अहन्न्॑ । साम॑ । भ॒व॒ति॒ । तेजः॑ । वै । गा॒य॒त्री । गा॒य॒त्री । ब्र॒ह्म॒व॒र्च॒समिति॑ ब्रह्म - व॒र्च॒सम् । तेजः॑ । ए॒व । ब्र॒ह्म॒व॒र्च॒समिति॑ ब्रह्म - व॒र्च॒सम् । आ॒त्मन्न् । ध॒त्ते॒ । त्रैष्टु॑भम् । उत्त॑र॒ इत्युत् - त॒रे॒ । ओजः॑ । वै । वी॒र्य᳚म् । त्रि॒ष्टुक् । ओजः॑ । ए॒व । वी॒र्य᳚म् । आ॒त्मन्न् । ध॒त्ते॒ । र॒थ॒न्त॒रमिति॑ रथं - त॒रम् । पूर्वे᳚ ।  \newline




\markright{ TS 7.1.4.3  \hfill https://www.vedavms.in \hfill}

\section{ TS 7.1.4.3 }

\textbf{TS 7.1.4.3 } \newline
\textbf{Samhita Paata} \newline

ऽह॒न्थ् साम॑ भवती॒यं ॅवै र॑थन्त॒रम॒स्यामे॒व प्रति॑ तिष्ठति बृ॒हदुत्त॑रे॒ऽसौ वै बृ॒हद॒मुष्या॑मे॒व प्रति॑ तिष्ठति॒ तदा॑हुः॒ क्व॑ जग॑ती चाऽनु॒ष्टुप् चेति॑ वैखान॒सं पूर्वेऽह॒न्थ् साम॑ भवति॒ तेन॒ जग॑त्यै॒ नैति॑ षोड॒श्युत्त॑रे॒ तेना॑नु॒ष्टुभोऽथा॑ ऽऽ*हु॒र्यथ् स॑मा॒ने᳚ऽर्द्धमा॒से स्याता॑-मन्यत॒रस्याह्नो॑ वी॒र्य॑मनु॑ ( ) पद्ये॒तेत्य॑-मावा॒स्या॑यां॒ पूर्व॒मह॑-र्भव॒त्युत्त॑रस्मि॒-न्नुत्त॑रं॒ नानै॒वा ऽर्द्ध॑मा॒सयो᳚र्भवतो॒ नाना॑वीर्ये भवतो ह॒विष्म॑न्निधनं॒ पूर्व॒मह॑र्भवति हवि॒ष्कृन्नि॑धन॒-मुत्त॑रं॒ प्रति॑ष्ठित्यै ॥ \newline

\textbf{Pada Paata} \newline

अहन्न्॑ । साम॑ । भ॒व॒ति॒ । इ॒यम् । वै । र॒थ॒न्त॒रमिति॑ रथं - त॒रम् । अ॒स्याम् । ए॒व । प्रतीति॑ । ति॒ष्ठ॒ति॒ । बृ॒हत् । उत्त॑र॒ इत्युत् - त॒रे॒ । अ॒सौ । वै । बृ॒हत् । अ॒मुष्या᳚म् । ए॒व । प्रतीति॑ । ति॒ष्ठ॒ति॒ । तत् । आ॒हुः॒ । क्व॑ । जग॑ती । च॒ । अ॒नु॒ष्टुबित्य॑नु - स्तुप् । च॒ । इति॑ । वै॒खा॒न॒सम् । पूर्वे᳚ । अहन्न्॑ । साम॑ । भ॒व॒ति॒ । तेन॑ । जग॑त्यै । न । ए॒ति॒ । षो॒ड॒शि । उत्त॑र॒ इत्युत् - त॒रे॒ । तेन॑ । अ॒नु॒ष्टुभ॒ इत्य॑नु - स्तुभः॑ । अथ॑ । आ॒हुः॒ । यत् । स॒मा॒ने । अ॒द्‌र्ध॒मा॒स इत्य॑द्‌र्ध - मा॒से । स्याता᳚म् । अ॒न्य॒त॒रस्य॑ । अह्नः॑ । वी॒र्य᳚म् । अन्विति॑ ( ) । प॒द्ये॒त॒ । इति॑ । अ॒मा॒वा॒स्या॑या॒मित्य॑मा - वा॒स्या॑याम् । पूर्व᳚म् । अहः॑ । भ॒व॒ति॒ । उत्त॑रस्मि॒न्नित्युत् - त॒र॒स्मि॒न्न् । उत्त॑र॒मित्युत् - त॒र॒म् । नाना᳚ । ए॒व । अ॒द्‌र्ध॒मा॒सयो॒रित्य॑द्‌र्ध - मा॒सयोः᳚ । भ॒व॒तः॒ । नाना॑वीर्ये॒ इति॒ नाना᳚ - वी॒र्ये॒ । भ॒व॒तः॒ । ह॒विष्म॑न्निधन॒मिति॑ ह॒विष्म॑त् - नि॒ध॒न॒म् । पूर्व᳚म् । अहः॑ । भ॒व॒ति॒ । ह॒वि॒ष्कृन्नि॑धन॒मिति॑ हवि॒ष्कृत् - नि॒ध॒न॒म् । उत्त॑र॒मित्युत् - त॒र॒म् । प्रति॑ष्ठित्या॒ इति॒ प्रति॑ - स्थि॒त्यै॒ ॥  \newline




\markright{ TS 7.1.5.1  \hfill https://www.vedavms.in \hfill}

\section{ TS 7.1.5.1 }

\textbf{TS 7.1.5.1 } \newline
\textbf{Samhita Paata} \newline

आपो॒ वा इ॒दमग्रे॑ सलि॒लमा॑सी॒त् तस्मि॑न् प्र॒जाप॑ति-र्वा॒युर्भू॒त्वा ऽच॑र॒थ् स इ॒माम॑पश्य॒त् तां ॅव॑रा॒हो भू॒त्वाऽह॑र॒त् तां ॅवि॒श्वक॑र्मा भू॒त्वा व्य॑मा॒ट्र्थ् साऽप्र॑थत॒ सा पृ॑थि॒व्य॑भव॒त् तत् पृ॑थि॒व्यै पृ॑थिवि॒त्वं तस्या॑मश्राम्यत् प्र॒जाप॑तिः॒ स दे॒वान॑सृजत॒ वसू᳚न् रु॒द्राना॑दि॒त्यान् ते दे॒वाः प्र॒जाप॑तिमब्रुव॒न् प्रजा॑यामहा॒ इति॒ सो᳚ऽब्रवी॒द् - [  ] \newline

\textbf{Pada Paata} \newline

आपः॑ । वै । इ॒दम् । अग्रे᳚ । स॒लि॒लम् । आ॒सी॒त् । तस्मिन्न्॑ । प्र॒जाप॑ति॒रिति॑ प्र॒जा - प॒तिः॒ । वा॒युः । भू॒त्वा । अ॒च॒र॒त् । सः । इ॒माम् । अ॒प॒श्य॒त् । ताम् । व॒रा॒हः । भू॒त्वा । एति॑ । अ॒ह॒र॒त् । ताम् । वि॒श्वक॒र्मेति॑ वि॒श्व-क॒र्मा॒ । भू॒त्वा । वीति॑ । अ॒मा॒र्ट्॒ । सा । अ॒प्र॒थ॒त॒ । सा । पृ॒थि॒वी । अ॒भ॒व॒त् । तत् । पृ॒थि॒व्यै । पृ॒थि॒वि॒त्वमिति॑ पृ॒थिवि - त्वम् । तस्या᳚म् । अ॒श्रा॒म्य॒त् । प्र॒जाप॑ति॒रिति॑ प्र॒जा-प॒तिः॒ । सः । दे॒वान् । अ॒सृ॒ज॒त॒ । वसून्॑ । रु॒द्रान् । आ॒दि॒त्यान् । ते । दे॒वाः । प्र॒जाप॑ति॒मिति॑ प्र॒जा - प॒ति॒म् । अ॒ब्रु॒व॒न्न् । प्रेति॑ । जा॒या॒म॒है॒ । इति॑ । सः । अ॒ब्र॒वी॒त् ।  \newline




\markright{ TS 7.1.5.2  \hfill https://www.vedavms.in \hfill}

\section{ TS 7.1.5.2 }

\textbf{TS 7.1.5.2 } \newline
\textbf{Samhita Paata} \newline

-यथा॒ऽहं ॅयु॒ष्माꣳस्तप॒सा ऽसृ॑क्ष्ये॒वं तप॑सि प्र॒जन॑न-मिच्छद्ध्व॒मिति॒ तेभ्यो॒ऽग्निमा॒यत॑नं॒ प्राऽय॑च्छदे॒तेना॒ऽऽ*यत॑नेन श्राम्य॒तेति॒ ते᳚ऽग्निना॒ऽऽ*यत॑नेनाऽ-श्राम्य॒न् ते सं॑ॅवथ्स॒र एकां॒ गाम॑सृजन्त॒ तां ॅवसु॑भ्यो रु॒द्रेभ्य॑ आदि॒त्येभ्यः॒ प्राऽय॑च्छन्ने॒ताꣳ र॑क्षद्ध्व॒मिति॒ तां ॅवस॑वो रु॒द्रा आ॑दि॒त्या अ॑रक्षन्त॒ सा वसु॑भ्यो रु॒द्रेभ्य॑ आदि॒त्येभ्यः॒ प्राजा॑यत॒त्रीणि॑ च - [  ] \newline

\textbf{Pada Paata} \newline

यथा᳚ । अ॒हम् । यु॒ष्मान् । तप॑सा । असृ॑क्षि । ए॒वम् । तप॑सि । प्र॒जन॑न॒मिति॑ प्र - जन॑नम् । इ॒च्छ॒द्ध्व॒म् । इति॑ । तेभ्यः॑ । अ॒ग्निम् । आ॒यत॑न॒मित्या᳚ - यत॑नम् । प्रेति॑ । अ॒य॒च्छ॒त् । ए॒तेन॑ । आ॒यत॑ने॒नेत्या᳚ - यत॑नेन । श्रा॒म्य॒त॒ । इति॑ । ते । अ॒ग्निना᳚ । आ॒यत॑ने॒नेत्या᳚ - यत॑नेन । अ॒श्रा॒म्य॒न्न् । ते । सं॒ॅव॒थ्स॒र इति॑ सं - व॒थ्स॒रे । एका᳚म् । गाम् । अ॒सृ॒ज॒न्त॒ । ताम् । वसु॑भ्य॒ इति॒ वसु॑ - भ्यः॒ । रु॒द्रेभ्यः॑ । आ॒दि॒त्येभ्यः॑ । प्रेति॑ । अ॒य॒च्छ॒न्न् । ए॒ताम् । र॒क्ष॒द्ध्व॒म् । इति॑ । ताम् । वस॑वः । रु॒द्राः । आ॒दि॒त्याः । अ॒र॒क्ष॒न्त॒ । सा । वसु॑भ्य॒ इति॒ वसु॑ - भ्यः॒ । रु॒द्रेभ्यः॑ । आ॒दि॒त्येभ्यः॑ । प्रेति॑ । अ॒जा॒य॒त॒ । त्रीणि॑ । च॒ ।  \newline




\markright{ TS 7.1.5.3  \hfill https://www.vedavms.in \hfill}

\section{ TS 7.1.5.3 }

\textbf{TS 7.1.5.3 } \newline
\textbf{Samhita Paata} \newline

श॒तानि॒ त्रय॑स्त्रिꣳशतं॒ चाथ॒ सैव स॑हस्रत॒म्य॑भव॒त् ते दे॒वाः प्र॒जाप॑तिमब्रुवन्थ् स॒हस्रे॑ण नो याज॒येति॒ सो᳚ऽग्निष्टो॒मेन॒ वसू॑नयाजय॒त् त इ॒मं ॅलो॒कम॑जय॒न् तच्चा॑ददुः॒ स उ॒क्थ्ये॑न रु॒द्रान॑याजय॒त् ते᳚ऽन्तरि॑क्षमजय॒न् तच्चा॑ददुः॒ सो॑ऽतिरा॒त्रेणा॑ऽऽ*दि॒त्यान॑याजय॒त् ते॑ऽमुं ॅलो॒कम॑जय॒न् तच्चा॑ददु॒ -स्तद॒न्तरि॑क्षं॒ - [  ] \newline

\textbf{Pada Paata} \newline

श॒तानि॑ । त्रय॑स्त्रिꣳशत॒मिति॒ त्रयः॑ - त्रिꣳ॒॒श॒त॒म् । च॒ । अथ॑ । सा । ए॒व । स॒ह॒स्र॒त॒मीति॑ सहस्र - त॒मी । अ॒भ॒व॒त् । ते । दे॒वाः । प्र॒जाप॑ति॒मिति॑ प्र॒जा - प॒ति॒म् । अ॒ब्रु॒व॒न्न् । स॒हस्रे॑ण । नः॒ । या॒ज॒य॒ । इति॑ । सः । अ॒ग्नि॒ष्टो॒मेनेत्य॑ग्नि-स्तो॒मेन॑ । वसून्॑ । अ॒या॒ज॒य॒त् । ते । इ॒मम् । लो॒कम् । अ॒ज॒य॒न्न् । तत् । च॒ । अ॒द॒दुः॒ । सः । उ॒क्थ्ये॑न । रु॒द्रान् । अ॒या॒ज॒य॒त् । ते । अ॒न्तरि॑क्षम् । अ॒ज॒य॒न्न् । तत् । च॒ । अ॒द॒दुः॒ । सः । अ॒ति॒रा॒त्रेणेत्य॑ति - रा॒त्रेण॑ । आ॒दि॒त्यान् । अ॒या॒ज॒य॒त् । ते । अ॒मुम् । लो॒कम् । अ॒ज॒य॒न्न् । तत् । च॒ । अ॒द॒दुः॒ । तत् । अ॒न्तरि॑क्षम् ।  \newline




\markright{ TS 7.1.5.4  \hfill https://www.vedavms.in \hfill}

\section{ TS 7.1.5.4 }

\textbf{TS 7.1.5.4 } \newline
\textbf{Samhita Paata} \newline

ॅव्यवै᳚र्यत॒ तस्मा᳚द्-रु॒द्रा घातु॑का अनायत॒ना हि तस्मा॑दाहुः शिथि॒लं ॅवै म॑द्ध्य॒म-मह॑स्त्रिरा॒त्रस्य॒ वि हि तद॒वैर्य॒तेति॒ त्रैष्टु॑भं मद्ध्य॒मस्याह्न॒ आज्यं॑ भवति सं॒ॅयाना॑नि सू॒क्तानि॑ शꣳसति षोड॒शिनꣳ॑ शꣳस॒त्यह्नो॒ धृत्या॒ अशि॑थिलंभावाय॒ तस्मा᳚त् त्रिरा॒त्रस्या᳚ग्निष्टो॒म ए॒व प्र॑थ॒ममहः॑ स्या॒दथो॒क्थ्यो ऽथा॑ऽतिरा॒त्र ए॒षां ॅलो॒कानां॒ ॅविधृ॑त्यै॒ त्रीणि॑त्रीणि श॒ता-न्य॑नूचीना॒ह-मव्य॑वच्छिन्नानि ददा - [  ] \newline

\textbf{Pada Paata} \newline

व्यवै᳚र्य॒तेति॑ वि - अवै᳚र्यत । तस्मा᳚त् । रु॒द्राः । घातु॑काः । अ॒ना॒य॒त॒ना इत्य॑ना - य॒त॒नाः । हि । तस्मा᳚त् । आ॒हुः॒ । शि॒थि॒लम् । वै । म॒द्ध्य॒मम् । अहः॑ । त्रि॒रा॒त्रस्येति॑ त्रि - रा॒त्रस्य॑ । वीति॑ । हि । तत् । अ॒वैर्य॒तेत्य॑व - ऐर्य॑त । इति॑ । त्रैष्टु॑भम् । म॒द्ध्य॒मस्य॑ । अह्नः॑ । आज्य᳚म् । भ॒व॒ति॒ । सं॒ॅयाना॒निति॑ सम् - याना॑नि । सू॒क्तानीति॑ सु - उ॒क्तानि॑ । शꣳ॒॒स॒ति॒ । षो॒ड॒शिन᳚म् । शꣳ॒॒स॒ति॒ । अह्नः॑ । धृत्यै᳚ । अशि॑थिलम्भावा॒येत्यशि॑थिलं - भा॒वा॒य॒ । तस्मा᳚त् । त्रि॒रा॒त्रस्येति॑ त्रि - रा॒त्रस्य॑ । अ॒ग्नि॒ष्टो॒म इत्य॑ग्नि - स्तो॒मः । ए॒व । प्र॒थ॒मम् । अहः॑ । स्या॒त् । अथ॑ । उ॒क्थ्यः॑ । अथ॑ । अ॒ति॒रा॒त्र इत्य॑ति - रा॒त्रः । ए॒षाम् । लो॒काना᳚म् । विधृ॑त्या॒ इति॒ वि - धृ॒त्यै॒ । त्रीणि॑त्री॒णीति॒ त्रीणि॑ - त्री॒णि॒ । श॒तानि॑ । अ॒नू॒ची॒ना॒हमित्य॑नूचीन - अ॒हम् । अव्य॑वच्छिन्ना॒नीत्यवि॑ - अ॒व॒च्छि॒न्ना॒नि॒ । द॒दा॒ति॒ ।  \newline




\markright{ TS 7.1.5.5  \hfill https://www.vedavms.in \hfill}

\section{ TS 7.1.5.5 }

\textbf{TS 7.1.5.5 } \newline
\textbf{Samhita Paata} \newline

-त्ये॒षां ॅलो॒काना॒मनु॒ सन्त॑त्यै द॒शतं॒ न विच्छि॑न्द्याद्-वि॒राजं॒ नेद्वि॑च्छि॒नदा॒नीत्यथ॒ या स॑हस्रत॒म्यासी॒त् तस्या॒मिन्द्र॑श्च॒ विष्णु॑श्च॒ व्याय॑च्छेताꣳ॒॒ स इन्द्रो॑ऽमन्यता॒नया॒ वा इ॒दं ॅविष्णुः॑ स॒हस्रं॑ ॅवर्क्ष्यत॒ इति॒ तस्या॑मकल्पेतां॒ द्विभा॑ग॒ इन्द्र॒स्तृती॑ये॒ विष्णु॒स्तद्वा ए॒षाऽभ्यनू᳚च्यत उ॒भा जि॑ग्यथु॒रिति॒ तां ॅवा ए॒ताम॑च्छावा॒क - [  ] \newline

\textbf{Pada Paata} \newline

ए॒षाम् । लो॒काना᳚म् । अन्विति॑ । सन्त॑त्या॒ इति॒ सं - त॒त्यै॒ । द॒शत᳚म् । न । वीति॑ । छि॒न्द्या॒त् । वि॒राज॒मिति॑ वि - राज᳚म् । न । इत् । वि॒च्छि॒नदा॒नीति॑ वि - छि॒नदा॑नि । इति॑ । अथ॑ । या । स॒ह॒स्र॒त॒मीति॑ सहस्र - त॒मी । आसी᳚त् । तस्या᳚म् । इन्द्रः॑ । च॒ । विष्णुः॑ । च॒ । व्याय॑च्छेता॒मिति॑ वि - आय॑च्छेताम् । सः । इन्द्रः॑ । अ॒म॒न्य॒त॒ । अ॒नया᳚ । वै । इ॒दम् । विष्णुः॑ । स॒हस्र᳚म् । व॒र्क्ष्य॒ते॒ । इति॑ । तस्या᳚म् । अ॒क॒ल्पे॒ता॒म् । द्विभा॑ग॒ इति॒ द्वि - भा॒गे॒ । इन्द्रः॑ । तृती॑ये । विष्णुः॑ । तत् । वै । ए॒षा । अ॒भ्यनू᳚च्यत॒ इत्य॑भि-अनू᳚च्यते । उ॒भा । जि॒ग्य॒थुः॒ । इति॑ । ताम् । वै । ए॒ताम् । अ॒च्छा॒वा॒कः ।  \newline




\markright{ TS 7.1.5.6  \hfill https://www.vedavms.in \hfill}

\section{ TS 7.1.5.6 }

\textbf{TS 7.1.5.6 } \newline
\textbf{Samhita Paata} \newline

ए॒व शꣳ॑स॒त्यथ॒ या स॑हस्रत॒मी सा होत्रे॒ देयेति॒ होता॑रं॒ ॅवा अ॒भ्यति॑रिच्यते॒ यद॑ति॒रिच्य॑ते॒ होता ऽना᳚प्तस्याऽऽ*पयि॒ता ऽथा॑ऽ*हुरुन्ने॒त्रे देयेत्यति॑रिक्ता॒ वा ए॒षा स॒हस्र॒स्याति॑रिक्त उन्ने॒तर्त्विजा॒मथा॑ऽऽ*हुः॒ सर्वे᳚भ्यः सद॒स्ये᳚भ्यो॒ देयेत्यथा॑ऽऽहुरुदा॒ कृत्या॒ सा वशं॑ चरे॒दित्यथा॑ऽऽ*हुर्ब्र॒ह्मणे॑ चा॒ग्नीधे॑ च॒ देयेति॒ - [  ] \newline

\textbf{Pada Paata} \newline

ए॒व । शꣳ॒॒स॒ति॒ । अथ॑ । या । स॒ह॒स्र॒त॒मीति॑ सहस्र - त॒मी । सा । होत्रे᳚ । देया᳚ । इति॑ । होता॑रम् । वै । अ॒भ्यति॑रिच्यत॒ इत्य॑भि-अति॑रिच्यते । यत् । अ॒ति॒रिच्य॑त॒ इत्य॑ति - रिच्य॑ते । होता᳚ । अना᳚प्तस्य । आ॒प॒यि॒ता । अथ॑ । आ॒हुः॒ । उ॒न्ने॒त्र इत्यु॑त् - ने॒त्रे । देया᳚ । इति॑ । अति॑रि॒क्तेत्यति॑ - रि॒क्ता॒ । वै । ए॒षा । स॒हस्र॑स्य । अति॑रिक्त॒ इत्यति॑ - रि॒क्तः॒ । उ॒न्ने॒तेत्यु॑त् - ने॒ता । ऋ॒त्विजा᳚म् । अथ॑ । आ॒हुः॒ । सर्वे᳚भ्यः । स॒द॒स्ये᳚भ्यः । देया᳚ । इति॑ । अथ॑ । आ॒हुः॒ । उ॒दा॒कृत्येत्यु॑त् - आ॒कृत्या᳚ । सा । वश᳚म् । च॒रे॒त् । इति॑ । अथ॑ । आ॒हुः॒ । ब्र॒ह्मणे᳚ । च॒ । अ॒ग्नीध॒ इत्य॑ग्नि - इधे᳚ । च॒ । देया᳚ । इति॑ ।  \newline




\markright{ TS 7.1.5.7  \hfill https://www.vedavms.in \hfill}

\section{ TS 7.1.5.7 }

\textbf{TS 7.1.5.7 } \newline
\textbf{Samhita Paata} \newline

द्विभा॑गं ब्र॒ह्मणे॒ तृती॑यम॒ग्नीध॑ ऐ॒न्द्रो वै ब्र॒ह्मा वै᳚ष्ण॒वो᳚ऽग्नीद्यथै॒व तावक॑ल्पेता॒मित्यथा॑ ऽऽ*हु॒र्या क॑ल्या॒णी ब॑हुरू॒पा सा देयेत्यथा॑ ऽऽ*हु॒र्या द्वि॑रू॒पोभ॒यत॑एनी॒ सा देयेति॑ स॒हस्र॑स्य॒ परि॑गृहीत्यै॒ तद्वा ए॒तथ् स॒हस्र॒स्याऽय॑नꣳ स॒हस्रꣳ॑ स्तो॒त्रीयाः᳚ स॒हस्रं॒ दक्षि॑णाः स॒हस्र॑सम्मितः सुव॒र्गो लो॒कः सु॑व॒र्गस्य॑ लो॒कस्या॒भिजि॑त्यै ॥ \newline

\textbf{Pada Paata} \newline

द्विभा॑ग॒मिति॒ द्वि - भा॒ग॒म् । ब्र॒ह्मणे᳚ । तृती॑यम् । अ॒ग्नीध॒ इत्य॑ग्नि- इधे᳚ । ऐ॒न्द्रः । वै । ब्र॒ह्मा । वै॒ष्ण॒वः । अ॒ग्नीदित्य॑ग्नि-इत् । यथा᳚ । ए॒व । तौ । अक॑ल्पेताम् । इति॑ । अथ॑ । आ॒हुः॒ । या । क॒ल्या॒णी । ब॒हु॒रू॒पेति॑ बहु - रू॒पा । सा । देया᳚ । इति॑ । अथ॑ । आ॒हुः॒ । या । द्वि॒रू॒पेति॑ द्वि - रू॒पा । उ॒भ॒यत॑ए॒नीत्य॑भ॒यतः॑ - ए॒नी॒ । सा । देया᳚ । इति॑ । स॒हस्र॑स्य । परि॑गृहीत्या॒ इति॒ परि॑-गृ॒ही॒त्यै॒ । तत् । वै । ए॒तत् । स॒हस्र॑स्य । अय॑नम् । स॒हस्र᳚म् । स्तो॒त्रीयाः᳚ । स॒हस्र᳚म् । दक्षि॑णाः । स॒हस्र॑संमित॒ इति॑ स॒हस्र॑ - स॒मिं॒तः॒ । सु॒व॒र्ग इति॑ सुवः - गः । लो॒कः । सु॒व॒र्गस्येति॑ सुवः - गस्य॑ । लो॒कस्य॑ । अ॒भिजि॑त्या॒ इत्य॒भि - जि॒त्यै॒ ॥  \newline




\markright{ TS 7.1.6.1  \hfill https://www.vedavms.in \hfill}

\section{ TS 7.1.6.1 }

\textbf{TS 7.1.6.1 } \newline
\textbf{Samhita Paata} \newline

सोमो॒ वै स॒हस्र॑मविन्द॒त् तमिन्द्रो ऽन्व॑विन्द॒त् तौ य॒मो न्याग॑च्छ॒त् ताव॑ब्रवी॒दस्तु॒ मेऽत्राऽपीत्यस्तु॒ ही(3) इत्य॑ब्रूताꣳ॒॒ स य॒म एक॑स्यां ॅवी॒र्यं॑ पर्य॑पश्यदि॒यं ॅवा अ॒स्य स॒हस्र॑स्य वी॒र्यं॑ बिभ॒र्तीति॒ ताव॑ब्रवीदि॒यं ममास्त्वे॒तद्-यु॒वयो॒रिति॒ ताव॑ब्रूताꣳ॒॒ सर्वे॒ वा ए॒तदे॒तस्यां᳚ ॅवी॒र्यं॑ - [  ] \newline

\textbf{Pada Paata} \newline

सोमः॑ । वै । स॒हस्र᳚म् । अ॒वि॒न्द॒त् । तम् । इन्द्रः॑ । अन्विति॑ । अ॒न्वि॒न्द॒त् । तौ । य॒मः । न्याग॑च्छ॒दिति॑ नि - आग॑च्छत् । तौ । अ॒ब्र॒वी॒त् । अस्तु॑ । मे॒ । अत्र॑ । अपीति॑ । इति॑ । अस्तु॑ । ही(3) । इति॑ । अ॒ब्रू॒ता॒म् । सः । य॒मः । एक॑स्याम् । वी॒र्य᳚म् । परीति॑ । अ॒प॒श्य॒त् । इ॒यम् । वै । अ॒स्य । स॒हस्र॑स्य । वी॒र्य᳚म् । बि॒भ॒र्ति॒ । इति॑ । तौ । अ॒ब्र॒वी॒त् । इ॒यम् । मम॑ । अस्तु॑ । ए॒तत् । यु॒वयोः᳚ । इति॑ । तौ । अ॒ब्रू॒ता॒म् । सर्वे᳚ । वै । ए॒तत् । ए॒तस्या᳚म् । वी॒र्य᳚म् ।  \newline




\markright{ TS 7.1.6.2  \hfill https://www.vedavms.in \hfill}

\section{ TS 7.1.6.2 }

\textbf{TS 7.1.6.2 } \newline
\textbf{Samhita Paata} \newline

परि॑ पश्या॒मोऽꣳश॒मा ह॑रामहा॒ इति॒ तस्या॒मꣳश॒माऽह॑रन्त॒ ताम॒फ्सु प्राऽवे॑शय॒न्थ् सोमा॑यो॒देहीति॒ सा रोहि॑णी पिङ्ग॒लैक॑हायनी रू॒पं कृ॒त्वा त्रय॑स्त्रिꣳशता च त्रि॒भिश्च॑ श॒तैः स॒होदैत् तस्मा॒द्-रोहि॑ण्या पिङ्ग॒लयैक॑हायन्या॒ सोमं॑ क्रीणीया॒द्य ए॒वं ॅवि॒द्वान् रोहि॑ण्या पिङ्ग॒लयैक॑हायन्या॒ सोमं॑ क्री॒णाति॒ त्रय॑स्त्रिꣳशता चै॒वास्य॑ त्रि॒भिश्च॑ - [  ] \newline

\textbf{Pada Paata} \newline

परीति॑ । प॒श्या॒मः॒ । अꣳश᳚म् । एति॑ । ह॒रा॒म॒है॒ । इति॑ । तस्या᳚म् । अꣳश᳚म् ।  एति॑ । अ॒ह॒र॒न्त॒ । ताम् । अ॒फ्स्वित्य॑प् - सु । प्रेति॑ । अ॒वे॒श॒य॒न्न् । सोमा॑य । उ॒देहीत्यु॑त् - एहि॑ । इति॑ । सा । रोहि॑णी । पि॒ङ्ग॒ला । एक॑हाय॒नीत्येक॑ - हा॒य॒नी॒ । रू॒पम् । कृ॒त्वा । त्रय॑स्त्रिꣳश॒तेति॒ त्रयः॑ - त्रिꣳ॒॒श॒ता॒ । च॒ । त्रि॒भिरिति॑ त्रि - भिः । च॒ । श॒तैः । स॒ह । उ॒दैदित्यु॑त् - ऐत् । तस्मा᳚त् । रोहि॑ण्या । पि॒ङ्ग॒लया᳚ । एक॑हाय॒न्येत्येक॑ - हा॒य॒न्या॒ । सोम᳚म् । क्री॒णी॒या॒त् । यः । ए॒वम् । वि॒द्वान् । रोहि॑ण्या । पि॒ङ्ग॒लया᳚ । एक॑हाय॒न्येत्येक॑ - हा॒य॒न्या॒ । सोम᳚म् । क्री॒णाति॑ । त्रय॑स्त्रिꣳश॒तेति॒ त्रयः॑ - त्रिꣳ॒॒श॒ता॒ । च॒ । ए॒व । अ॒स्य॒ । त्रि॒भिरिति॑ त्रि - भिः । च॒ ।  \newline




\markright{ TS 7.1.6.3  \hfill https://www.vedavms.in \hfill}

\section{ TS 7.1.6.3 }

\textbf{TS 7.1.6.3 } \newline
\textbf{Samhita Paata} \newline

श॒तैः सोमः॑ क्री॒तो भ॑वति॒ सुक्री॑तेन यजते॒ ताम॒फ्सु प्रावे॑शय॒-न्निन्द्रा॑यो॒देहीति॒ सा रोहि॑णी लक्ष्म॒णा प॑ष्ठौ॒ही वार्त्र॑घ्नी रू॒पं कृ॒त्वा त्रय॑स्त्रिꣳशता च त्रि॒भिश्च॑ श॒तैः स॒होदैत् तस्मा॒द् रोहि॑णीं ॅलक्ष्म॒णां प॑ष्ठौ॒हीं ॅवार्त्र॑घ्नीं दद्या॒द्य ए॒वं ॅवि॒द्वान् रोहि॑णीं ॅलक्ष्म॒णां प॑ष्ठौ॒हीं ॅवार्त्र॑घ्नीं॒ ददा॑ति॒ त्रय॑स्त्रिꣳशच्चै॒वास्य॒ त्रीणि॑ च श॒तानि॒ सा द॒त्ता - [  ] \newline

\textbf{Pada Paata} \newline

श॒तैः । सोमः॑ । क्री॒तः । भ॒व॒ति॒ । सुक्री॑ते॒नेति॒ सु-क्री॒ते॒न॒ । य॒ज॒ते॒ । ताम् । अ॒फ्स्वित्य॑प् - सु । प्रेति॑ । अ॒वे॒श॒य॒न्न् । इन्द्रा॑य । उ॒देहीत्यु॑त् - एहि॑ । इति॑ । सा । रोहि॑णी । ल॒क्ष्म॒णा । प॒ष्ठौ॒ही । वार्त्र॒घ्नीति॒ वार्त्र॑ - घ्नी॒ । रू॒पम् । कृ॒त्वा । त्रय॑स्त्रिꣳश॒तेति॒ त्रयः॑ - त्रिꣳ॒॒श॒ता॒ । च॒ । त्रि॒भिरिति॑ त्रि - भिः । च॒ । श॒तैः । स॒ह । उ॒दैदित्यु॑त् - ऐत् । तस्मा᳚त् । रोहि॑णीम् । ल॒क्ष्म॒णाम् । प॒ष्ठौ॒हीम् । वार्त्र॑घ्नी॒मिति॒ वार्त्र॑ - घ्नी॒म् । द॒द्या॒त् । यः । ए॒वम् । वि॒द्वान् । रोहि॑णीम् । ल॒क्ष्म॒णाम् । प॒ष्ठौ॒हीम् । वार्त्र॑घ्नी॒मिति॒ वार्त्र॑ - घ्नी॒म् । ददा॑ति । त्रय॑स्त्रिꣳश॒दिति॒ त्रयः॑ - त्रिꣳ॒॒श॒त् । च॒ । ए॒व॒ । अ॒स्य॒ । त्रीणि॑ । च॒ । श॒तानि॑ । सा । द॒त्ता ।  \newline




\markright{ TS 7.1.6.4  \hfill https://www.vedavms.in \hfill}

\section{ TS 7.1.6.4 }

\textbf{TS 7.1.6.4 } \newline
\textbf{Samhita Paata} \newline

भ॑वति॒ ताम॒फ्सु प्रावे॑शयन् य॒मायो॒देहीति॒ सा जर॑ती मू॒र्खा त॑ज्जघ॒न्या रू॒पं कृ॒त्वा त्रय॑स्त्रिꣳशता च त्रि॒भिश्च॑ श॒तैः स॒होदैत् तस्मा॒ज्जर॑तीं मू॒र्खां त॑ज्जघ॒न्या-म॑नु॒स्तर॑णीं कुर्वीत॒ य ए॒वं ॅवि॒द्वाञ्जर॑तीं मू॒र्खां त॑ज्जघ॒न्या-म॑नु॒स्तर॑णीं कुरु॒ते त्रय॑स्त्रिꣳशच्चै॒वास्य॒ त्रीणि॑ च श॒तानि॒ साऽमुष्मि॑ॅल्लो॒के भ॑वति॒ वागे॒व स॑हस्रत॒मी तस्मा॒ - [  ] \newline

\textbf{Pada Paata} \newline

भ॒व॒ति॒ । ताम् । अ॒फ्स्वित्य॑प् - सु । प्रेति॑ । अ॒वे॒श॒य॒न्न् । य॒माय॑ । उ॒देहीत्यु॑त् - एहि॑ । इति॑ । सा । जर॑ती । मू॒र्खा । त॒ज्ज॒घ॒न्येति॑ तत् - ज॒घ॒न्या । रू॒पम् । कृ॒त्वा । त्रय॑स्त्रिꣳश॒तेति॒ त्रयः॑-त्रिꣳ॒॒श॒ता॒ । च॒ । त्रि॒भिरिति॑ त्रि - भिः । च॒ । श॒तैः । स॒ह । उ॒दैदित्यु॑त् - ऐत् । तस्मा᳚त् । जर॑तीम् । मू॒र्खाम् । त॒ज्ज॒घ॒न्यामिति॑ तत् - ज॒घ॒न्याम् । अ॒नु॒स्तर॑णी॒मित्य॑नु - स्तर॑णीम् । कु॒र्वी॒त॒ । यः । ए॒वम् । वि॒द्वान् । जर॑तीम् । मू॒र्खाम् । त॒ज्ज॒घ॒न्यामिति॑ तत् - ज॒घ॒न्याम् । अ॒नु॒स्तर॑णी॒मित्यु॑नु - स्तर॑णीम् । कु॒रु॒ते । त्रय॑स्त्रिꣳश॒दिति॒ त्रयः॑ - त्रिꣳ॒॒श॒त् । च॒ । ए॒व । अ॒स्य॒ । त्रीणि॑ । च॒ । श॒तानि॑ । सा । अ॒मुष्मिन्न्॑ । लो॒के । भ॒व॒ति॒ । वाक् । ए॒व । स॒ह॒स्र॒त॒मीति॑ सहस्र - त॒मी । तस्मा᳚त् ।  \newline




\markright{ TS 7.1.6.5  \hfill https://www.vedavms.in \hfill}

\section{ TS 7.1.6.5 }

\textbf{TS 7.1.6.5 } \newline
\textbf{Samhita Paata} \newline

द्वरो॒ देयः॒ सा हि वरः॑ स॒हस्र॑मस्य॒ सा द॒त्ता भ॑वति॒ तस्मा॒द् वरो॒ न प्र॑ति॒गृह्यः॒ सा हि वरः॑ स॒हस्र॑मस्य॒ प्रति॑गृहीतं भवती॒यं ॅवर॒ इति॑ ब्रूया॒दथा॒न्यां ब्रू॑यादि॒यं ममेति॒ तथा᳚ऽस्य॒ तथ् स॒हस्र॒-मप्र॑तिगृहीतं भवत्युभयतए॒नी स्या॒त् तदा॑हुरन्यत ए॒नी स्या᳚थ् स॒हस्रं॑ प॒रस्ता॒देत॒मिति॒ यैव वरः॑ - [  ] \newline

\textbf{Pada Paata} \newline

वरः॑ । देयः॑ । सा । हि । वरः॑ । स॒हस्र᳚म् । अ॒स्य॒ । सा । द॒त्ता । भ॒व॒ति॒ । तस्मा᳚त् । वरः॑ । न । प्र॒ति॒गृह्य॒ इति॑ प्रति - गृह्यः॑ । सा । हि । वरः॑ । स॒हस्र᳚म् । अ॒स्य॒ । प्रति॑गृहीत॒मिति॒ प्रति॑ - गृ॒ही॒त॒म् । भ॒व॒ति॒ । इ॒यम् । वरः॑ । इति॑ । ब्रू॒या॒त् । अथ॑ । अ॒न्याम् । ब्रू॒या॒त् । इ॒यम् । मम॑ । इति॑ । तथा᳚ । अ॒स्य॒ । तत् । स॒हस्र᳚म् । अप्र॑तिगृहीत॒मित्यप्र॑ति - गृ॒ही॒त॒म् । भ॒व॒ति॒ । उ॒भ॒य॒त॒ए॒नीत्यु॑भयतः - ए॒नी । स्या॒त् । तत् । आ॒हुः॒ । अ॒न्य॒त॒ए॒नीत्य॑न्यतः - ए॒नी । स्या॒त् । स॒हस्र᳚म् । प॒रस्ता᳚त् । एत᳚म् । इति॑ । या । ए॒व । वरः॑ ।  \newline




\markright{ TS 7.1.6.6  \hfill https://www.vedavms.in \hfill}

\section{ TS 7.1.6.6 }

\textbf{TS 7.1.6.6 } \newline
\textbf{Samhita Paata} \newline

कल्या॒णी रू॒पस॑मृद्धा॒ सा स्या॒थ् सा हि वरः॒ समृ॑द्ध्यै॒ तामुत्त॑रे॒णाऽऽ*ग्नी᳚द्ध्रं पर्या॒णीया॑ऽऽ*हव॒नीय॒स्यान्ते᳚ द्रोणकल॒शमव॑ घ्रापये॒दा जि॑घ्र क॒लशं॑ मह्यु॒रुधा॑रा॒ पय॑स्व॒त्या त्वा॑ विश॒न्त्विन्द॑वः समु॒द्रमि॑व॒ सिन्ध॑वः॒ सा मा॑ स॒हस्र॒ आ भ॑ज प्र॒जया॑ प॒शुभिः॑ स॒ह पुन॒र्मा ऽऽ*वि॑शताद्-र॒यिरिति॑ प्र॒जयै॒वैनं॑ प॒शुभी॑ र॒य्या स - [  ] \newline

\textbf{Pada Paata} \newline

क॒ल्या॒णी । रू॒पस॑मृ॒द्धेति॑ रू॒प - स॒मृ॒द्धा॒ । सा । स्या॒त् । सा । हि । वरः॑ । समृ॑द्ध्या॒ इति॒ सं - ऋ॒द्ध्यै॒ । ताम् । उत्त॑रे॒णेत्युत् - त॒रे॒ण॒ । आग्नी᳚द्ध्र॒मित्याग्नि॑ - इ॒द्ध्र॒म् । प॒र्या॒णीयेति॑ परि - आ॒नीय॑ । आ॒ह॒व॒नीय॒स्येत्या᳚ - ह॒व॒नीय॑स्य । अन्ते᳚ । द्रो॒ण॒क॒ल॒शमिति॑ द्रोण - क॒ल॒शम् । अवेति॑ । घ्रा॒प॒ये॒त् । एति॑ । जि॒घ्र॒ । क॒लश᳚म् । म॒हि॒ । उ॒रुधा॒रेत्यु॒रु - धा॒रा॒ । पय॑स्वती । एति॑ । त्वा॒ । वि॒श॒न्तु॒ । इन्द॑वः । स॒मु॒द्रम् । इ॒व॒ । सिन्ध॑वः । सा । मा॒ । स॒हस्रे᳚ । एति॑ । भ॒ज॒ । प्र॒जयेति॑ प्र - जया᳚ । प॒शुभि॒रिति॑ प॒शु-भिः॒ । स॒ह । पुनः॑ । मा॒ । एति॑ । वि॒श॒ता॒त् । र॒यिः । इति॑ । प्र॒जयेति॑ प्र - जया᳚ । ए॒व । ए॒न॒म् । प॒शुभि॒रिति॑ प॒शु - भिः॒ । र॒य्या । समिति॑ ।  \newline




\markright{ TS 7.1.6.7  \hfill https://www.vedavms.in \hfill}

\section{ TS 7.1.6.7 }

\textbf{TS 7.1.6.7 } \newline
\textbf{Samhita Paata} \newline

-म॑र्द्धयति प्र॒जावा᳚न् पशु॒मान् र॑यि॒मान् भ॑वति॒ य ए॒वं ॅवेद॒ तया॑ स॒हाऽऽ*ग्नी᳚द्ध्रं प॒रेत्य॑ पु॒रस्ता᳚त् प्र॒तीच्यां॒ तिष्ठ॑न्त्यां जुहुयादु॒भा जि॑ग्यथु॒र्न परा॑ जयेथे॒ न परा॑ जिग्ये कत॒रश्च॒नैनोः᳚ । इन्द्र॑श्च विष्णो॒ यदप॑स्पृधेथां त्रे॒धा स॒हस्रं॒ ॅवि तदै॑रयेथा॒मिति॑, त्रेधाविभ॒क्तं ॅवै त्रि॑रा॒त्रे स॒हस्रꣳ॑ साह॒स्रीमे॒वैनां᳚ करोति स॒हस्र॑स्यै॒वैनां॒ मात्रां᳚ - [  ] \newline

\textbf{Pada Paata} \newline

अ॒द्‌र्ध॒य॒ति॒ । प्र॒जावा॒निति॑ प्रजा - वा॒न् । प॒शु॒मानिति॑ पशु - मान् । र॒यि॒मानिति॑ रयि - मान् । भ॒व॒ति॒ । यः । ए॒वम् । वेद॑ । तया᳚ । स॒ह । आग्नी᳚द्ध्र॒मित्याग्नि॑ - इ॒द्ध्र॒म् । प॒रेत्येति॑ परा - इत्य॑ । पु॒रस्ता᳚त् । प्र॒तीच्या᳚म् । तिष्ठ॑न्त्याम् । जु॒हु॒या॒त् । उ॒भा । जि॒ग्य॒थुः॒ । न । परेति॑ । ज॒ये॒थे॒ इति॑ । न । परेति॑ । जि॒ग्ये॒ । क॒त॒रः । च॒न । ए॒नोः॒ ॥ इन्द्रः॑ । च॒ । वि॒ष्णो॒ इति॑ । यत् । अप॑स्पृधेथाम् । त्रे॒धा । स॒हस्र᳚म् । वीति॑ । तत् । ऐ॒र॒ये॒था॒म् । इति॑ । त्रे॒धा॒वि॒भ॒क्तमिति॑ त्रेधा - वि॒भ॒क्तम् । वै । त्रि॒रा॒त्र इति॑ त्रि - रा॒त्रे । स॒हस्र᳚म् । सा॒ह॒स्रीम् । ए॒व । ए॒ना॒म् । क॒रो॒ति॒ । स॒हस्र॑स्य । ए॒व । ए॒ना॒म् । मात्रा᳚म् ।  \newline




\markright{ TS 7.1.6.8  \hfill https://www.vedavms.in \hfill}

\section{ TS 7.1.6.8 }

\textbf{TS 7.1.6.8 } \newline
\textbf{Samhita Paata} \newline

करोति रू॒पाणि॑ जुहोति रू॒पैरे॒वैनाꣳ॒॒ सम॑र्द्धयति॒ तस्या॑ उपो॒त्थाय॒ कर्ण॒मा ज॑पे॒दिडे॒ रन्तेऽदि॑ते॒ सर॑स्वति॒ प्रिये॒ प्रेय॑सि॒ महि॒ विश्रु॑त्ये॒तानि॑ ते अघ्निये॒ नामा॑नि सु॒कृतं॑ मा दे॒वेषु॑ ब्रूता॒दिति॑ दे॒वेभ्य॑ ए॒वैन॒मा वे॑दय॒त्यन्वे॑नं दे॒वा बु॑द्ध्यन्ते ॥ \newline

\textbf{Pada Paata} \newline

क॒रो॒ति॒ । रू॒पाणि॑ । जु॒हो॒ति॒ । रू॒पैः । ए॒व । ए॒ना॒म् । समिति॑ । अ॒द्‌र्ध॒य॒ति॒ । तस्याः᳚ ।      उ॒पो॒त्थायेत्यु॑प - उ॒त्थाय॑ । कर्ण᳚म् । एति॑ । ज॒पे॒त् । इडे᳚ । रन्ते᳚ । अदि॑ते । सर॑स्वति । प्रिये᳚ । प्रेय॑सि । महि॑ । विश्रु॒तीति॒ वि - श्रु॒ति॒ । ए॒तानि॑ । ते॒ । अ॒घ्नि॒ये॒ । नामा॑नि । सु॒कृत॒मिति॑ सु - कृत᳚म् । मा॒ । दे॒वेषु॑ । ब्रू॒ता॒त् । इति॑ । दे॒वेभ्यः॑ । ए॒व । ए॒न॒म् । एति॑ । वे॒द॒य॒ति॒ । अन्विति॑ । ए॒न॒म् । दे॒वाः । बु॒द्ध्य॒न्ते॒ ॥  \newline




\markright{ TS 7.1.7.1  \hfill https://www.vedavms.in \hfill}

\section{ TS 7.1.7.1 }

\textbf{TS 7.1.7.1 } \newline
\textbf{Samhita Paata} \newline

स॒ह॒स्र॒त॒म्या॑ वै यज॑मानः सुव॒र्गं ॅलो॒कमे॑ति॒ सैनꣳ॑ सुव॒र्गं ॅलो॒कं ग॑मयति॒ सा मा॑ सुव॒र्गं ॅलो॒कं ग॑म॒येत्या॑ह सुव॒र्गमे॒वैनं॑ ॅलो॒कं ग॑मयति॒ सा मा॒ ज्योति॑ष्मन्तं ॅलो॒कं ग॑म॒येत्या॑ह॒ ज्योति॑ष्मन्तमे॒वैनं॑ ॅलो॒कं ग॑मयति॒ सा मा॒ सर्वा॒न् पुण्यां᳚ ॅलो॒कान् ग॑म॒येत्या॑ह॒ सर्वा॑ने॒वैनं॒ पुण्यां᳚ ॅलो॒कान् ग॑मयति॒ सा - [  ] \newline

\textbf{Pada Paata} \newline

स॒ह॒स्र॒त॒म्येति॑ सहस्र - त॒म्या᳚ । वै । यज॑मानः । सु॒व॒र्गमिति॑ सुवः - गम् । लो॒कम् । ए॒ति॒ । सा । ए॒न॒म् । सु॒व॒र्गमिति॑ सुवः - गम् । लो॒कम् । ग॒म॒य॒ति॒ । सा । मा॒ । सु॒व॒र्गमिति॑ सुवः - गम् । लो॒कम् । ग॒म॒य॒ । इति॑ । आ॒ह॒ । सु॒व॒र्गमिति॑ सुवः - गम् । ए॒व । ए॒न॒म् । लो॒कम् । ग॒म॒य॒ति॒ । सा । मा॒ । ज्योति॑ष्मन्तम् । लो॒कम् । ग॒म॒य॒ । इति॑ । आ॒ह॒ । ज्योति॑ष्मन्तम् । ए॒व । ए॒न॒म् । लो॒कम् । ग॒म॒य॒ति॒ । सा । मा॒ । सर्वान्॑ । पुण्यान्॑ । लो॒कान् । ग॒म॒य॒ । इति॑ । आ॒ह॒ । सर्वान्॑ । ए॒व । ए॒न॒म् । पुण्यान्॑ । लो॒कान् । ग॒म॒य॒ति॒ । सा ।  \newline




\markright{ TS 7.1.7.2  \hfill https://www.vedavms.in \hfill}

\section{ TS 7.1.7.2 }

\textbf{TS 7.1.7.2 } \newline
\textbf{Samhita Paata} \newline

मा᳚ प्रति॒ष्ठां ग॑मय प्र॒जया॑ प॒शुभिः॑ स॒ह पुन॒र्माऽऽ वि॑शताद्-र॒यिरिति॑ प्र॒जयै॒वैनं॑ प॒शुभी॑ र॒य्यां प्रति॑ ष्ठापयति प्र॒जावा᳚न् पशु॒मान् र॑यि॒मान् भ॑वति॒ य ए॒वं ॅवेद॒ ताम॒ग्नीधे॑ वा ब्र॒ह्मणे॑ वा॒ होत्रे॑ वोद्गा॒त्रे वा᳚ऽद्ध्व॒र्यवे॑ वा दद्याथ् स॒हस्र॑मस्य॒ सा द॒त्ता भ॑वति स॒हस्र॑मस्य॒ प्रति॑गृहीतं भवति॒ यस्तामवि॑द्वान् - [  ] \newline

\textbf{Pada Paata} \newline

मा॒ । प्र॒ति॒ष्ठामिति॑ प्रति - स्थाम् । ग॒म॒य॒ । प्र॒जयेति॑ प्र - जया᳚ । प॒शुभि॒रिति॑ प॒शु - भिः॒ । स॒ह । पुनः॑ । मा॒ । एति॑ । वि॒श॒ता॒त् । र॒यिः । इति॑ । प्र॒जयेति॑ प्र - जया᳚ । ए॒व । ए॒न॒म् । प॒शुभि॒रिति॑ प॒शु - भिः॒ । र॒य्याम् । प्रतीति॑ । स्था॒प॒य॒ति॒ । प्र॒जावा॒निति॑ प्र॒जा - वा॒न् । प॒शु॒मानिति॑ पशु - मान् । र॒यि॒मानिति॑ रयि - मान् । भ॒व॒ति॒ । यः । ए॒वम् । वेद॑ । ताम् । अ॒ग्नीध॒ इत्य॑ग्नि - इधे᳚ । वा॒ । ब्र॒ह्मणे᳚ । वा॒ । होत्रे᳚ । वा॒ । उ॒द्गा॒त्र इत्यु॑त् - गा॒त्रे । वा॒ । अ॒द्ध्व॒र्यवे᳚ । वा॒ । द॒द्या॒त् । स॒हस्र᳚म् । अ॒स्य॒ । सा । द॒त्ता । भ॒व॒ति॒ । स॒हस्र᳚म् । अ॒स्य॒ । प्रति॑गृहीत॒मिति॒ प्रति॑ - गृ॒ही॒त॒म् । भ॒व॒ति॒ । यः । ताम् । अवि॑द्वान् ।  \newline




\markright{ TS 7.1.7.3  \hfill https://www.vedavms.in \hfill}

\section{ TS 7.1.7.3 }

\textbf{TS 7.1.7.3 } \newline
\textbf{Samhita Paata} \newline

प्रतिगृ॒ह्णाति॒ तां प्रति॑गृह्णीया॒देका॑ऽसि॒ न स॒हस्र॒मेकां᳚ त्वा भू॒तां प्रति॑ गृह्णामि॒ न स॒हस्र॒मेका॑ मा भू॒ताऽऽ वि॑श॒ मा स॒हस्र॒मित्येका॑मे॒वैनां᳚ भू॒तां प्रति॑गृह्णाति॒ न स॒हस्रं॒ ॅय ए॒वं ॅवेद॑ स्यो॒नाऽसि॑ सु॒षदा॑ सु॒शेवा᳚ स्यो॒ना मा ऽऽवि॑श सु॒षदा॒ मा ऽऽवि॑श सु॒शेवा॒ मा ऽऽवि॒शे -[  ] \newline

\textbf{Pada Paata} \newline

प्र॒ति॒गृ॒ह्णातीति॑ प्रति - गृ॒ह्णाति॑ । ताम् । प्रतीति॑ । गृ॒ह्णी॒या॒त् । एका᳚ । अ॒सि॒ । न । स॒हस्र᳚म् । एका᳚म् । त्वा॒ । भू॒ताम् । प्रतीति॑ । गृ॒ह्णा॒मि॒ । न । स॒हस्र᳚म् । एका᳚ । मा॒ । भू॒ता । एति॑ । वि॒श॒ । मा । स॒हस्र᳚म् । इति॑ । एका᳚म् । ए॒व । ए॒ना॒म् । भू॒ताम् । प्रतीति॑ । गृ॒ह्णा॒ति॒ । न । स॒हस्र᳚म् । यः । ए॒वम् । वेद॑ । स्यो॒ना । अ॒सि॒ । सु॒षदेति॑ सु-सदा᳚ । सु॒शेवेति॑ सु - शेवा᳚ । स्यो॒ना । मा॒ । एति॑ । वि॒श॒ । सु॒षदेति॑ सु - सदा᳚ । मा॒ । एति॑ । वि॒श॒ । सु॒शेवेति॑ सु - शेवा᳚ । मा॒ । एति॑ । वि॒श॒ ।  \newline




\markright{ TS 7.1.7.4  \hfill https://www.vedavms.in \hfill}

\section{ TS 7.1.7.4 }

\textbf{TS 7.1.7.4 } \newline
\textbf{Samhita Paata} \newline

-त्या॑ह स्यो॒नैवैनꣳ॑ सु॒षदा॑ सु॒शेवा॑ भू॒ताऽऽ वि॑शति॒ नैनꣳ॑ हिनस्ति ब्रह्मवा॒दिनो॑ वदन्ति स॒हस्रꣳ॑ सहस्रत॒म्यन्वे॒ती(3) स॑हस्रत॒मीꣳ स॒हस्रा(3)मिति॒ यत् प्राची॑मुथ् सृ॒जेथ् स॒हस्रꣳ॑ सहस्रत॒म्यन्वि॑या॒त् तथ् स॒हस्र॑मप्रज्ञा॒त्रꣳ सु॑व॒र्गं ॅलो॒कं न प्र जा॑नीयात् प्र॒तीची॒मुथ्- सृ॑जति॒ ताꣳ स॒हस्र॒मनु॑ प॒र्याव॑र्तते॒ सा प्र॑जान॒ती सु॑व॒र्गं ॅलो॒कमे॑ति॒ यज॑मान ( ) -म॒भ्युथ् सृ॑जति क्षि॒प्रे स॒हस्रं॒ प्र जा॑यत उत्त॒मा नी॒यते᳚ प्रथ॒मा दे॒वान् ग॑च्छति ॥ \newline

\textbf{Pada Paata} \newline

इति॑ । आ॒ह॒ । स्यो॒ना । ए॒व । ए॒न॒म् । सु॒षदेति॑ सु - सदा᳚ । सु॒शेवेति॑ सु - शेवा᳚ । भू॒ता । एति॑ । वि॒श॒ति॒ । न । ए॒न॒म् । हि॒न॒स्ति॒ । ब्र॒ह्म॒वा॒दिन॒ इति॑ ब्रह्म - वा॒दिनः॑ । व॒द॒न्ति॒ । स॒हस्र᳚म् । स॒ह॒स्र॒त॒मीति॑ सहस्र - त॒मी । अन्विति॑ । ए॒ती(3) । स॒ह॒स्र॒त॒मीमिति॑ सहस्र - त॒मीम् । स॒हस्रा(3)म् । इति॑ । यत् । प्राची᳚म् । उ॒थ्सृ॒जेदित्यु॑त् - सृ॒जेत् । स॒हस्र᳚म् । स॒ह॒स्र॒त॒मीति॑ सहस्र - त॒मी । अन्विति॑ । इ॒या॒त् । तत् । स॒हस्र᳚म् । अ॒प्र॒ज्ञा॒त्रमित्य॑प्र - ज्ञा॒त्रम् । सु॒व॒र्गमिति॑ सुवः - गम् । लो॒कम् । न । प्रेति॑ । जा॒नी॒या॒त् । प्र॒तीची᳚म् । उदिति॑ । सृ॒ज॒ति॒ । ताम् । स॒हस्र᳚म् । अन्विति॑ । प॒र्याव॑र्तत॒ इति॑ परि - आव॑र्तते । सा । प्र॒जा॒न॒तीति॑ प्र - जा॒न॒ती । सु॒व॒र्गमिति॑ सुवः - गम् । लो॒कम् । ए॒ति॒ । यज॑मानम् ( ) । अ॒भि । उदिति॑ । सृ॒ज॒ति॒ । क्षि॒प्रे । स॒हस्र᳚म् । प्रेति॑ । जा॒य॒ते॒ । उ॒त्त॒मेत्यु॑त् - त॒मा । नी॒यते᳚ । प्र॒थ॒मा । दे॒वान् । ग॒च्छ॒ति॒ ॥  \newline




\markright{ TS 7.1.8.1  \hfill https://www.vedavms.in \hfill}

\section{ TS 7.1.8.1 }

\textbf{TS 7.1.8.1 } \newline
\textbf{Samhita Paata} \newline

अत्रि॑रददा॒दौर्वा॑य प्र॒जां पु॒त्रका॑माय॒ स रि॑रिचा॒नो॑ऽमन्यत॒ निर्वी᳚र्यः शिथि॒लो या॒तया॑मा॒ स ए॒तं च॑तूरा॒त्र-म॑पश्य॒त् तमाऽह॑र॒त् तेना॑यजत॒ ततो॒ वै तस्य॑ च॒त्वारो॑ वी॒रा आऽजा॑यन्त॒ सुहो॑ता॒ सू᳚द्गाता॒ स्व॑द्ध्वर्युः॒ सुस॑भेयो॒ य ए॒वं ॅवि॒द्वाꣳश्च॑तूरा॒त्रेण॒ यज॑त॒ आऽस्य॑ च॒त्वारो॑ वी॒रा जा॑यन्ते॒ सुहो॑ता॒ सू᳚द्गाता॒ स्व॑द्ध्वर्युः॒ सुस॑भेयो॒ ये च॑तुर्विꣳ॒॒शाः पव॑माना ब्रह्मवर्च॒सं त - [  ] \newline

\textbf{Pada Paata} \newline

अत्रिः॑ । अ॒द॒दा॒त् । और्वा॑य । प्र॒जामिति॑ प्र - जाम् । पु॒त्रका॑मा॒येति॑ पु॒त्र - का॒मा॒य॒ । सः । रि॒रि॒चा॒नः । अ॒म॒न्य॒त॒ । निर्वी᳚र्य॒ इति॒ निः - वी॒र्यः॒ । शि॒थि॒लः । या॒तया॒मेति॑ या॒त - या॒मा॒ । सः । ए॒तम् । च॒तू॒रा॒त्रमिति॑ चतुः - रा॒त्रम् । अ॒प॒श्य॒त् । तम् । एति॑ । अ॒ह॒र॒त् । तेन॑ । अ॒य॒ज॒त॒ । ततः॑ । वै । तस्य॑ । च॒त्वारः॑ । वी॒राः । एति॑ । अ॒जा॒य॒न्त॒ । सुहो॒तेति॒ सु - हो॒ता॒ । सू᳚द्गा॒तेति॒ सु - उ॒द्गा॒ता॒ । स्व॑द्ध्वर्यु॒रिति॒ सु - अ॒द्ध्व॒र्युः॒ । सुस॑भेय॒ इति॒ सु - स॒भे॒यः॒ । यः । ए॒वम् । वि॒द्वान् । च॒तू॒रा॒त्रेणेति॑ चतुः - रा॒त्रेण॑ । यज॑ते । एति॑ । अ॒स्य॒ । च॒त्वारः॑ । वी॒राः । जा॒य॒न्ते॒ । सुहो॒तेति॒ सु - हो॒ता॒ । सू᳚द्गा॒तेति॒ सु - उ॒द्गा॒ता॒ । स्व॑द्ध्वर्यु॒रिति॒ सु - अ॒द्ध्व॒र्युः॒ । सुस॑भेय॒ इति॒ सु - स॒भे॒यः॒ । ये । च॒तु॒र्विꣳ॒॒शा इति॑ चतुः - विꣳ॒॒शाः । पव॑मानाः । ब्र॒ह्म॒व॒र्च॒समिति॑ ब्रह्म - व॒र्च॒सम् । तत् ।  \newline




\markright{ TS 7.1.8.2  \hfill https://www.vedavms.in \hfill}

\section{ TS 7.1.8.2 }

\textbf{TS 7.1.8.2 } \newline
\textbf{Samhita Paata} \newline

- द्य उ॒द्यन्तः॒ स्तोमाः॒ श्रीः सा ऽत्रिꣳ॑ श्र॒द्धादे॑वं॒ ॅयज॑मानं च॒त्वारि॑ वी॒र्या॑णि॒ नोपा॑ऽनम॒न् तेज॑ इन्द्रि॒यं ब्र॑ह्मवर्च॒स-म॒न्नाद्यꣳ॒॒ स ए॒ताꣳश्च॒तुर॒श्चतु॑ष्टोमा॒न्थ् सोमा॑न-पश्य॒त् तानाऽह॑र॒त् तैर॑यजत॒ तेज॑ ए॒व प्र॑थ॒मेना ऽवा॑रुन्धेन्द्रि॒यं द्वि॒तीये॑न ब्रह्मवर्च॒सं तृ॒तीये॑ना॒न्नाद्यं॑ चतु॒र्थेन॒ य ए॒वं ॅवि॒द्वाꣳश्च॒तुर॒श्चतु॑ष्टोमा॒न्थ् सोमा॑ना॒हर॑ति॒ तैर्यज॑ते॒ तेज॑ ए॒व ( ) प्र॑थ॒मेनाव॑ रुन्ध इन्द्रि॒यं द्वि॒तीये॑न ब्रह्मवर्च॒सं तृ॒तीये॑ना॒ऽन्नाद्यं॑ चतु॒र्थेन॒ यामे॒वात्रि॒र्॒. ऋद्धि॒मार्द्ध्नो॒त् तामे॒व यज॑मान ऋद्ध्नोति ॥ \newline

\textbf{Pada Paata} \newline

ये । उ॒द्यन्त॒ इत्यु॑त् - यन्तः॑ । स्तोमाः᳚ । श्रीः । सा । अत्रि᳚म् । श्र॒द्धादे॑व॒मिति॑ श्र॒द्धा - दे॒व॒म् । यज॑मानम् । च॒त्वारि॑ । वी॒र्या॑णि । न । उपेति॑ । अ॒न॒म॒न्न् । तेजः॑ । इ॒न्द्रि॒यम् । ब्र॒ह्म॒व॒र्च॒समिति॑ ब्रह्म-व॒र्च॒सम् । अ॒न्नाद्य॒मित्य॑न्न - अद्य᳚म् । सः । ए॒तान् । च॒तुरः॑ । चतु॑ष्टोमा॒निति॒ चतुः॑ - स्तो॒मा॒न्न् । सोमान्॑ । अ॒प॒श्य॒त् । तान् । एति॑ । अ॒ह॒र॒त् । तैः । अ॒य॒ज॒त॒ । तेजः॑ । ए॒व । प्र॒थ॒मेन॑ । अवेति॑ । अ॒रु॒न्ध॒ । इ॒न्द्रि॒यम् । द्वि॒तीये॑न । ब्र॒ह्म॒व॒र्च॒समिति॑ ब्रह्म - व॒र्च॒सम् । तृ॒तीये॑न । अ॒न्नाद्य॒मित्य॑न्न -अद्य᳚म् । च॒तु॒र्थेन॑ । यः । ए॒वम् । वि॒द्वान् । च॒तुरः॑ । चतु॑ष्टोमा॒निति॒ चतुः॑ - स्तो॒मा॒न् । सोमान्॑ । आ॒हर॒तीत्या᳚ - हर॑ति । तैः । यज॑ते । तेजः॑ । ए॒व ( ) । प्र॒थ॒मेन॑ । अवेति॑ । रु॒न्धे॒ । इ॒न्द्रि॒यम् । द्वि॒तीये॑न । ब्र॒ह्म॒व॒र्च॒समिति॑ ब्रह्म - व॒र्च॒सम् । तृ॒तीये॑न । अ॒न्नाद्य॒मित्य॑न्न - अद्य᳚म् । च॒तु॒र्थेन॑ । याम् । ए॒व । अत्रिः॑ । ऋद्धि᳚म् । आद्‌र्ध्नो᳚त् । ताम् । ए॒व । यज॑मानः । ऋ॒द्ध्नो॒ति॒ ॥  \newline




\markright{ TS 7.1.9.1  \hfill https://www.vedavms.in \hfill}

\section{ TS 7.1.9.1 }

\textbf{TS 7.1.9.1 } \newline
\textbf{Samhita Paata} \newline

ज॒मद॑ग्निः॒ पुष्टि॑काम-श्चतूरा॒त्रेणा॑-यजत॒ स ए॒तान् पोषाꣳ॑ अपुष्य॒त् तस्मा᳚त् पलि॒तौ जाम॑दग्नियौ॒ न सं जा॑नाते ए॒ताने॒व पोषा᳚न् पुष्यति॒ य ए॒वं ॅवि॒द्वाꣳश्च॑तूरा॒त्रेण॒ यज॑ते पुरोडा॒शिन्य॑ उप॒सदो॑ भवन्ति प॒शवो॒ वै पु॑रो॒डाशः॑ प॒शूने॒वाव॑ रु॒न्धे ऽन्नं॒ ॅवै पु॑रो॒डाशोऽन्न॑मे॒वाव॑ रुन्धे ऽन्ना॒दः प॑शु॒मान् भ॑वति॒ य ए॒वं ॅवि॒द्वाꣳश्च॑तूरा॒त्रेण॒ यज॑ते ॥ \newline

\textbf{Pada Paata} \newline

ज॒मद॑ग्निः । पुष्टि॑काम॒ इति॒ पुष्टि॑ - का॒मः॒ । च॒तू॒रा॒त्रेणेति॑ चतुः - रा॒त्रेण॑ । अ॒य॒ज॒त॒ । सः । ए॒तान् । पोषान्॑ । अ॒पु॒ष्य॒त् । तस्मा᳚त् । प॒लि॒तौ । जाम॑दग्नियौ । न । समिति॑ । जा॒ना॒ते॒ इति॑ । ए॒तान् । ए॒व । पोषान्॑ । पु॒ष्य॒ति॒ । यः । ए॒वम् । वि॒द्वान् । च॒तू॒रा॒त्रेणेति॑ चतुः - रा॒त्रेण॑ । यज॑ते । पु॒रो॒डा॒शिन्यः॑ । उ॒प॒सद॒ इत्यु॑प - सदः॑ । भ॒व॒न्ति॒ । प॒शवः॑ । वै । पु॒रो॒डाशः॑ । प॒शून् । ए॒व । अवेति॑ । रु॒न्धे॒ । अन्न᳚म् । वै । पु॒रो॒डाशः॑ । अन्न᳚म् । ए॒व । अवेति॑ । रु॒न्धे॒ । अ॒न्ना॒द इत्य॑न्न - अ॒दः । प॒शु॒मानिति॑ पशु - मान् । भ॒व॒ति॒ । यः । ए॒वम् । वि॒द्वान् । च॒तू॒रा॒त्रेणेति॑ चतुः-रा॒त्रेण॑ । यज॑ते ॥  \newline




\markright{ TS 7.1.10.1  \hfill https://www.vedavms.in \hfill}

\section{ TS 7.1.10.1 }

\textbf{TS 7.1.10.1 } \newline
\textbf{Samhita Paata} \newline

सं॒ॅव॒थ्स॒रो वा इ॒दमेक॑ आसी॒थ् सो॑ऽकामयत॒र्तून्थ् सृ॑जे॒येति॒ स ए॒तं प॑ञ्चरा॒त्रम॑पश्य॒त् तमाऽह॑र॒त् तेना॑यजत॒ ततो॒ वै स ऋ॒तून॑सृजत॒ य ए॒वं ॅवि॒द्वान् प॑ञ्चरा॒त्रेण॒ यज॑ते॒ प्रैव जा॑यते॒ त ऋ॒तवः॑ सृ॒ष्टा न व्याव॑र्तन्त॒ त ए॒तं प॑ञ्चरा॒त्रम॑पश्य॒न् तमाऽह॑र॒न् तेना॑यजन्त॒ ततो॒ वै ते व्याव॑र्तन्त॒-[  ] \newline

\textbf{Pada Paata} \newline

सं॒ॅव॒थ्स॒र इति॑ सं - व॒थ्स॒रः । वै । इ॒दम् । एकः॑ । आ॒सी॒त् । सः । अ॒का॒म॒य॒त॒ । ऋ॒तून् । सृ॒जे॒य॒ । इति॑ । सः । ए॒तम् । प॒ञ्च॒रा॒त्रमिति॑ पञ्च - रा॒त्रम् । अ॒प॒श्य॒त् । तम् । एति॑ । अ॒ह॒र॒त् । तेन॑ । अ॒य॒ज॒त॒ । ततः॑ । वै । सः । ऋ॒तून् । अ॒सृ॒ज॒त॒ । यः । ए॒वम् । वि॒द्वान् । प॒ञ्च॒रा॒त्रेणेति॑ पञ्च - रा॒त्रेण॑ । यज॑ते । प्रेति॑ । ए॒व । जा॒य॒ते॒ । ते । ऋ॒तवः॑ । सृ॒ष्टाः । न । व्याव॑र्त॒न्तेति॑ वि - आव॑र्तन्त । ते । ए॒तम् । प॒ञ्च॒रा॒त्रमिति॑ पञ्च - रा॒त्रम् । अ॒प॒श्य॒न्न् । तम् । एति॑ । अ॒ह॒र॒न्न् । तेन॑ । अ॒य॒ज॒न्त॒ । ततः॑ । वै । ते । व्याव॑र्त॒न्तेति॑ वि - आव॑र्तन्त ।  \newline




\markright{ TS 7.1.10.2  \hfill https://www.vedavms.in \hfill}

\section{ TS 7.1.10.2 }

\textbf{TS 7.1.10.2 } \newline
\textbf{Samhita Paata} \newline

य ए॒वं ॅवि॒द्वान् प॑ञ्चरा॒त्रेण॒ यज॑ते॒ वि पा॒प्मना॒ भ्रातृ॑व्ये॒णाऽऽ* व॑र्तते॒ सार्व॑सेनिः शौचे॒यो॑ऽकामयत पशु॒मान्थ् स्या॒मिति॒ स ए॒तं प॑ञ्चरा॒त्रमाऽह॑र॒त् तेना॑ऽयजत॒ ततो॒ वै स स॒हस्रं॑ प॒शून् प्राऽऽ*प्नो॒द्य ए॒वं ॅवि॒द्वान् प॑ञ्चरा॒त्रेण॒ यज॑ते॒ प्र स॒हस्रं॑ प॒शूना᳚प्नोति बब॒रः प्रावा॑हणि-रकामयत वा॒चः प्र॑वदि॒ता स्या॒मिति॒ स ए॒तं प॑ञ्चरा॒त्रमा - [  ] \newline

\textbf{Pada Paata} \newline

यः । ए॒वम् । वि॒द्वान् । प॒ञ्च॒रा॒त्रेणेति॑ पञ्च-रा॒त्रेण॑ । यज॑ते । वीति॑ । पा॒प्मना᳚ । भ्रातृ॑व्येण । एति॑ । व॒र्त॒ते॒ । सार्व॑सेनि॒रिति॒ सार्व॑ - से॒निः॒ । शौ॒चे॒यः । अ॒का॒म॒य॒त॒ । प॒शु॒मानिति॑ पशु - मान् । स्या॒म् । इति॑ । सः । ए॒तम् । प॒ञ्च॒रा॒त्रमिति॑ पञ्च - रा॒त्रम् । एति॑ । अ॒ह॒र॒त् । तेन॑ । अ॒य॒ज॒त॒ । ततः॑ । वै । सः । स॒हस्र᳚म् । प॒शून् । प्रेति॑ । आ॒प्नो॒त् । यः । ए॒वम् । वि॒द्वान् । प॒ञ्च॒रा॒त्रेणेति॑ पञ्च-रा॒त्रेण॑ । यज॑ते । प्रेति॑ । स॒हस्र᳚म् । प॒शून् । आ॒प्नो॒ति॒ । ब॒ब॒रः । प्रावा॑हणिः । अ॒का॒म॒य॒त॒ । वा॒चः । प्र॒व॒दि॒तेति॑ प्र - व॒दि॒ता । स्या॒म् । इति॑ । सः । ए॒तम् । प॒ञ्च॒रा॒त्रमिति॑ पञ्च - रा॒त्रम् । एति॑ ।  \newline




\markright{ TS 7.1.10.3  \hfill https://www.vedavms.in \hfill}

\section{ TS 7.1.10.3 }

\textbf{TS 7.1.10.3 } \newline
\textbf{Samhita Paata} \newline

ऽह॑र॒त् तेना॑यजत॒ ततो॒ वै स वा॒चः प्र॑वदि॒ताऽभ॑व॒द्य ए॒वं ॅवि॒द्वान् प॑ञ्चरा॒त्रेण॒ यज॑ते प्रवदि॒तैव वा॒चो भ॑व॒त्यथो॑ एनं ॅवा॒चस्पति॒-रित्या॑हु॒रना᳚प्त-श्चतूरा॒त्रोऽति॑रिक्तः षड्-रा॒त्रोऽथ॒ वा ए॒ष स॑प्रं॒ति य॒ज्ञो यत् प॑ञ्चरा॒त्रो य ए॒वं ॅवि॒द्वान् प॑ञ्चरा॒त्रेण॒ यज॑ते संप्र॒त्ये॑व य॒ज्ञेन॑ यजते पञ्चरा॒त्रो भ॑वति॒ पञ्च॒ वा ऋ॒तवः॑ संॅवथ्स॒र - [  ] \newline

\textbf{Pada Paata} \newline

अ॒ह॒र॒त् । तेन॑ । अ॒य॒ज॒त॒ । ततः॑ । वै । सः । वा॒चः । प्र॒व॒दि॒तेति॑ प्र - व॒दि॒ता । अ॒भ॒व॒त् । यः । ए॒वम् । वि॒द्वान् । प॒ञ्च॒रा॒त्रेणेति॑ पञ्च - रा॒त्रेण॑ । यज॑ते । प्र॒व॒दि॒तेति॑ प्र - व॒दि॒ता । ए॒व । वा॒चः । भ॒व॒ति॒ । अथो॒ इति॑ । ए॒न॒म् । वा॒चः । पतिः॑ । इति॑ । आ॒हुः॒ । अना᳚प्तः । च॒तू॒रा॒त्र इति॑ चतुः - रा॒त्रः । अति॑रिक्त॒ इत्यति॑ - रि॒क्तः॒ । ष॒ड्रा॒त्र इति॑ षट् - रा॒त्रः । अथ॑ । वै । ए॒षः । स॒प्रं॒तीति॑ सं - प्र॒ति । य॒ज्ञ्ः । यत् । प॒ञ्च॒रा॒त्र इति॑ पञ्च - रा॒त्रः । यः । ए॒वम् । वि॒द्वान् । प॒ञ्च॒रा॒त्रेणेति॑ पञ्च - रा॒त्रेण॑ । यज॑ते । स॒म्प्र॒तीति॑ सं - प्र॒ति । ए॒व । य॒ज्ञेन॑ । य॒ज॒ते॒ । प॒ञ्च॒रा॒त्र इति॑ पञ्च - रा॒त्रः । भ॒व॒ति॒ । पञ्च॑ । वै । ऋ॒तवः॑ । सं॒ॅव॒थ्स॒र इति॑ सं - व॒थ्स॒रः ।  \newline




\markright{ TS 7.1.10.4  \hfill https://www.vedavms.in \hfill}

\section{ TS 7.1.10.4 }

\textbf{TS 7.1.10.4 } \newline
\textbf{Samhita Paata} \newline

ऋ॒तुष्वे॒व सं॑ॅवथ्स॒रे प्रति॑ तिष्ठ॒त्यथो॒ पञ्चा᳚क्षरा प॒ङ्क्तिः पाङ्क्तो॑ य॒ज्ञो य॒ज्ञ्मे॒वाव॑ रुन्धे त्रि॒वृद॑ग्निष्टो॒मो भ॑वति॒ तेज॑ ए॒वाव॑ रुन्धे पञ्चद॒शो भ॑वतीन्द्रि॒यमे॒वाव॑ रुन्धे सप्तद॒शो भ॑वत्य॒न्नाद्य॒स्या-व॑रुद्ध्या॒ अथो॒ प्रैव तेन॑ जायते पञ्चविꣳ॒॒शो᳚ ऽग्निष्टो॒मो भ॑वति प्र॒जाप॑ते॒राप्त्यै॑ महाव्र॒तवा॑-न॒न्नाद्य॒स्या-व॑रुद्ध्यै विश्व॒जिथ् सर्व॑पृष्ठो-ऽतिरा॒त्रो भ॑वति॒ सर्व॑स्या॒भिजि॑त्यै ( ) ॥ \newline

\textbf{Pada Paata} \newline

ऋ॒तुषु॑ । ए॒व । सं॒ॅव॒थ्स॒र इति॑ सं - व॒थ्स॒रे । प्रतीति॑ । ति॒ष्ठ॒न्ति॒ । अथो॒ इति॑ । पञ्चा᳚क्ष॒रेति॒ पञ्च॑ - अ॒क्ष॒रा॒ । प॒ङ्क्तिः । पाङ्क्तः॑ । य॒ज्ञ्ः । य॒ज्ञ्म् । ए॒व । अवेति॑ । रु॒न्धे॒ । त्रि॒वृदिति॑ त्रि - वृत् । अ॒ग्नि॒ष्टो॒म इत्य॑ग्नि - स्तो॒मः । भ॒व॒ति॒ । तेजः॑ । ए॒व । अवेति॑ । रु॒न्धे॒ । प॒ञ्च॒द॒श इति॑ पञ्च - द॒शः । भ॒व॒ति॒ । इ॒न्द्रि॒यम् । ए॒व । अवेति॑ । रु॒न्धे॒ । स॒प्त॒द॒श इति॑ सप्त - द॒शः । भ॒व॒ति॒ । अ॒न्नाद्य॒स्येत्य॑न्न - अद्य॑स्य । अव॑रुद्ध्या॒ इत्यव॑ - रु॒द्ध्यै॒ । अथो॒ इति॑ । प्रेति॑ । ए॒व । तेन॑ । जा॒य॒ते॒ । प॒ञ्च॒विꣳ॒॒श इति॑ पञ्च - विꣳ॒॒शः । अ॒ग्नि॒ष्टो॒म इत्य॑ग्नि - स्तो॒मः । भ॒व॒ति॒ । प्र॒जाप॑ते॒रिति॑ प्र॒जा-प॒तेः॒ । आप्त्यै᳚ । म॒हा॒व्र॒तवा॒निति॑ महाव्र॒त-वा॒न् । अ॒न्नाद्य॒स्येत्य॑न्न - अद्य॑स्य । अव॑रुद्ध्या॒ इत्यव॑ - रु॒द्ध्यै॒ । वि॒श्व॒जिदिति॑ विश्व - जित् । सर्व॑पृष्ठ॒ इति॒ सर्व॑ - पृ॒ष्ठः॒ । अ॒ति॒रा॒त्र इत्य॑ति - रा॒त्रः । भ॒व॒ति॒ । सर्व॑स्य । अ॒भिजि॑त्या॒ इत्य॒भि-जि॒त्यै॒ ( ) ॥  \newline




\markright{ TS 7.1.11.1  \hfill https://www.vedavms.in \hfill}

\section{ TS 7.1.11.1 }

\textbf{TS 7.1.11.1 } \newline
\textbf{Samhita Paata} \newline

दे॒वस्य॑त्वा सवि॒तुः प्र॑स॒वे᳚ऽश्विनो᳚र्बा॒हुभ्यां᳚ पू॒ष्णो हस्ता᳚भ्या॒मा द॑द इ॒माम॑गृभ्णन् रश॒नामृ॒तस्य॒ पूर्व॒ आयु॑षि वि॒दथे॑षु क॒व्या । तया॑ दे॒वाः सु॒तमा ब॑भूवुर् ऋ॒तस्य॒ सामन्थ᳚-स॒रमा॒रप॑न्ती ॥अ॒भि॒धा अ॑सि॒ भुव॑नमसि य॒न्ताऽसि॑ ध॒र्ताऽसि॒ सो᳚ऽग्निं ॅवै᳚श्वान॒रꣳ सप्र॑थसं गच्छ॒ स्वाहा॑कृतः पृथि॒व्यां ॅय॒न्ता राड् य॒न्ताऽसि॒ यम॑नो ध॒र्ताऽसि॑ ध॒रुणः॑ ( ) कृ॒ष्यै त्वा॒ क्षेमा॑य त्वा र॒य्यै त्वा॒ पोषा॑य त्वा पृथि॒व्यै त्वा॒ ऽन्तरि॑क्षाय त्वा दि॒वे त्वा॑ स॒ते त्वाऽस॑ते त्वा॒द्भ्यस्त्वौ-ष॑धीभ्यस्त्वा॒ विश्वे᳚भ्यस्त्वा भू॒तेभ्यः॑ ॥ \newline

\textbf{Pada Paata} \newline

दे॒वस्य॑ । त्वा॒ । स॒वि॒तुः । प्र॒स॒व इति॑ प्र - स॒वे । अ॒श्विनोः᳚ । बा॒हुभ्या॒मिति॑ बा॒हु - भ्या॒म् । पू॒ष्णः । हस्ता᳚भ्याम् । एति॑ । द॒दे॒ । इ॒माम् । अ॒गृ॒भ्ण॒न्न् । र॒श॒नाम् । ऋ॒तस्य॑ । पूर्वे᳚ । आयु॑षि । वि॒दथे॑षु । क॒व्या ॥ तया᳚ । दे॒वाः । सु॒तम् । एति॑ । ब॒भू॒वुः॒ । ऋ॒तस्य॑ । सामन्न्॑ । स॒रम् । आ॒रप॒न्तीत्या᳚ - रप॑न्ती ॥ अ॒भि॒धा इत्य॑भि - धाः । अ॒सि॒ । भुव॑नम् । अ॒सि॒ । य॒न्ता । अ॒सि॒ । ध॒र्ता । अ॒सि॒ । सः । अ॒ग्निम् । वै॒श्वा॒न॒रम् । सप्र॑थस॒मिति॒ स - प्र॒थ॒स॒म् । ग॒च्छ॒ । स्वाहा॑कृत॒ इति॒ स्वाहा᳚ - कृ॒तः॒ । पृ॒थि॒व्याम् । य॒न्ता । राट् । य॒न्ता । अ॒सि॒ । यम॑नः । ध॒र्ता । अ॒सि॒ । ध॒रुणः॑ ( ) । कृ॒ष्यै । त्वा॒ । क्षेमा॑य । त्वा॒ । र॒य्यै । त्वा॒ । पोषा॑य । त्वा॒ । पृ॒थि॒व्यै । त्वा॒ । अ॒न्तरि॑क्षाय । त्वा॒ । दि॒वे । त्वा॒ । स॒ते । त्वा॒ । अस॑ते । त्वा॒ । अ॒द्भ्य इत्य॑त् - भ्यः । त्वा॒ । ओष॑धीभ्य॒ इत्योष॑धि - भ्यः॒ । त्वा॒ । विश्वे᳚भ्यः । त्वा॒ । भू॒तेभ्यः॑ ॥  \newline




\markright{ TS 7.1.12.1  \hfill https://www.vedavms.in \hfill}

\section{ TS 7.1.12.1 }

\textbf{TS 7.1.12.1 } \newline
\textbf{Samhita Paata} \newline

वि॒भूर्मा॒त्रा प्र॒भूः पि॒त्राश्वो॑ऽसि॒ हयो॒ऽस्यत्यो॑ऽसि॒ नरो॒ऽस्यर्वा॑ऽसि॒ सप्ति॑रसि वा॒ज्य॑सि॒ वृषा॑ऽसि नृ॒मणा॑ असि॒ ययु॒र्नामा᳚स्यादि॒त्यानां॒ पत्वान्वि॑ह्य॒ग्नये॒ स्वाहा॒ स्वाहे᳚न्द्रा॒ग्निभ्याꣳ॒॒ स्वाहा᳚ प्र॒जाप॑तये॒ स्वाहा॒ विश्वे᳚भ्यो दे॒वेभ्यः॒ स्वाहा॒ सर्वा᳚भ्यो दे॒वेता᳚भ्य इ॒ह धृतिः॒ स्वाहे॒ह विधृ॑तिः॒ स्वाहे॒ह रन्तिः॒ स्वाहे॒ ( ) -ह रम॑तिः॒ स्वाहा॒ भूर॑सि भु॒वे त्वा॒ भव्या॑य त्वा भविष्य॒ते त्वा॒ विश्वे᳚भ्यस्त्वा भू॒तेभ्यो॒ देवा॑ आशापाला ए॒तं दे॒वेभ्योऽश्वं॒ मेधा॑य॒ प्रोक्षि॑तं गोपायत ॥ \newline

\textbf{Pada Paata} \newline

वि॒भूरिति॑ वि - भूः । मा॒त्रा । प्र॒भूरिति॑ प्र - भूः । पि॒त्रा । अश्वः॑ । अ॒सि॒ । हयः॑ । अ॒सि॒ । अत्यः॑ । अ॒सि॒ । नरः॑ । अ॒सि॒ । अर्वा᳚ । अ॒सि॒ । सप्तिः॑ । अ॒सि॒ । वा॒जी । अ॒सि॒ । वृषा᳚ । अ॒सि॒ । नृ॒मणा॒ इति॑ नृ - मनाः᳚ । अ॒सि॒ । ययुः॑ । नाम॑ । अ॒सि॒ । आ॒दि॒त्याना᳚म् । पत्व॑ । अन्विति॑ । इ॒हि॒ । अ॒ग्नये᳚ । स्वाहा᳚ । स्वाहा᳚ । इ॒न्द्रा॒ग्निभ्या॒मिती᳚न्द्रा॒ग्नि - भ्या॒म् । स्वाहा᳚ । प्र॒जाप॑तय॒ इति॑ प्र॒जा - प॒त॒ये॒ । स्वाहा᳚ । विश्वे᳚भ्यः । दे॒वेभ्यः॑ । स्वाहा᳚ । सर्वा᳚भ्यः । दे॒वेता᳚भ्यः । इ॒ह । धृतिः॑ । स्वाहा᳚ । इ॒ह । विधृ॑ति॒रिति॒ वि - धृ॒तिः॒ । स्वाहा᳚ । इ॒ह । रन्तिः॑ । स्वाहा᳚ ( ) । इ॒ह । रम॑तिः । स्वाहा᳚ । भूः । अ॒सि॒ । भु॒वे । त्वा॒ । भव्या॑य । त्वा॒ । भ॒वि॒ष्य॒ते । त्वा॒ । विश्वे᳚भ्यः । त्वा॒ । भू॒तेभ्यः॑ । देवाः᳚ । आ॒शा॒पा॒ला॒ इत्या॑शा - पा॒लाः॒ । ए॒तम् । दे॒वेभ्यः॑ । अश्व᳚म् । मेधा॑य । प्रोक्षि॑त॒मिति॒ प्र - उ॒क्षि॒त॒म् । गो॒पा॒य॒त॒ ॥  \newline




\markright{ TS 7.1.13.1  \hfill https://www.vedavms.in \hfill}

\section{ TS 7.1.13.1 }

\textbf{TS 7.1.13.1 } \newline
\textbf{Samhita Paata} \newline

आय॑नाय॒ स्वाहा॒ प्राय॑णाय॒ स्वाहो᳚द्द्रा॒वाय॒ स्वाहोद्द्रु॑ताय॒ स्वाहा॑ शूका॒राय॒ स्वाहा॒ शूकृ॑ताय॒ स्वाहा॒ पला॑यिताय॒ स्वाहा॒ ऽऽपला॑यिताय॒ स्वाहा॒ ऽऽवल्ग॑ते॒ स्वाहा॑ परा॒वल्ग॑ते॒ स्वाहा॑ ऽऽय॒ते स्वाहा᳚ प्रय॒ते स्वाहा॒ सर्व॑स्मै॒ स्वाहा᳚ ॥ \newline

\textbf{Pada Paata} \newline

आय॑ना॒येत्या᳚ - अय॑नाय । स्वाहा᳚ । प्राय॑णा॒येति॑ प्र - अय॑नाय । स्वाहा᳚ । उ॒द्द्रा॒वायेत्यु॑त् - द्रा॒वाय॑ । स्वाहा᳚ । उद्द्रु॑ता॒येत्युत्-द्रु॒ता॒य॒ । स्वाहा᳚ । शू॒का॒रायेति॑ शू-का॒राय॑ । स्वाहा᳚ । शूकृ॑ता॒येति॒ शू-कृ॒ता॒य॒ । स्वाहा᳚ । पला॑यिताय । स्वाहा᳚ । आ॒पला॑यिता॒येत्या᳚ - पला॑यिताय । स्वाहा᳚ । आ॒वल्ग॑त॒ इत्या᳚ - वल्ग॑ते । स्वाहा᳚ । प॒रा॒वल्ग॑त॒ इति॑ परा - वल्ग॑ते । स्वाहा᳚ । आ॒य॒त इत्या᳚ - य॒ते । स्वाहा᳚ । प्र॒य॒त इति॑ प्र - य॒ते । स्वाहा᳚ । सर्व॑स्मै । स्वाहा᳚ ॥  \newline




\markright{ TS 7.1.14.1  \hfill https://www.vedavms.in \hfill}

\section{ TS 7.1.14.1 }

\textbf{TS 7.1.14.1 } \newline
\textbf{Samhita Paata} \newline

अ॒ग्नये॒ स्वाहा॒ सोमा॑य॒ स्वाहा॑ वा॒यवे॒ स्वाहा॒ ऽपां मोदा॑य॒ स्वाहा॑ सवि॒त्रे स्वाहा॒ सर॑स्वत्यै॒ स्वाहे-न्द्रा॑य॒ स्वाहा॒ बृह॒स्पत॑ये॒ स्वाहा॑ मि॒त्राय॒ स्वाहा॒ वरु॑णाय॒ स्वाहा॒ सर्व॑स्मै॒ स्वाहा᳚ ॥ \newline

\textbf{Pada Paata} \newline

अ॒ग्नये᳚ । स्वाहा᳚ । सोमा॑य । स्वाहा᳚ । वा॒यवे᳚ । स्वाहा᳚ । अ॒पाम् । मोदा॑य । स्वाहा᳚ । स॒वि॒त्रे । स्वाहा᳚ । सर॑स्वत्यै । स्वाहा᳚ । इन्द्रा॑य । स्वाहा᳚ । बृह॒स्पत॑ये । स्वाहा᳚ । मि॒त्राय॑ । स्वाहा᳚ । वरु॑णाय । स्वाहा᳚ । सर्व॑स्मै । स्वाहा᳚ ॥  \newline




\markright{ TS 7.1.15.1  \hfill https://www.vedavms.in \hfill}

\section{ TS 7.1.15.1 }

\textbf{TS 7.1.15.1 } \newline
\textbf{Samhita Paata} \newline

पृ॒थि॒व्यै स्वाहा॒ ऽन्तरि॑क्षाय॒ स्वाहा॑ दि॒वे स्वाहा॒ सूर्या॑य॒ स्वाहा॑ च॒न्द्रम॑से॒ स्वाहा॒ नक्ष॑त्रेभ्यः॒ स्वाहा॒ प्राच्यै॑ दि॒शे स्वाहा॒ दक्षि॑णायै दि॒शे स्वाहा᳚ प्र॒तीच्यै॑ दि॒शे स्वाहो-दी᳚च्यै दि॒शे स्वाहो॒र्द्ध्वायै॑ दि॒शे स्वाहा॑ दि॒ग्भ्यः स्वाहा॑ ऽवान्तरदि॒शाभ्यः॒ स्वाहा॒ समा᳚भ्यः॒ स्वाहा॑ श॒रद्भ्यः॒ स्वाहा॑ ऽहोरा॒त्रेभ्यः॒ स्वाहा᳚ ऽर्द्धमा॒सेभ्यः॒ स्वाहा॒ मासे᳚भ्यः॒ स्वाहा॒र्तुभ्यः॒ स्वाहा॑ संॅवथ्स॒राय॒ स्वाहा॒ सर्व॑स्मै॒ स्वाहा᳚ ॥ \newline

\textbf{Pada Paata} \newline

पृ॒थि॒व्यै । स्वाहा᳚ । अ॒न्तरि॑क्षाय । स्वाहा᳚ । दि॒वे । स्वाहा᳚ । सूर्या॑य । स्वाहा᳚ । च॒न्द्रम॑से । स्वाहा᳚ । नक्ष॑त्रेभ्यः । स्वाहा᳚ । प्राच्यै᳚ । दि॒शे । स्वाहा᳚ । दक्षि॑णायै । दि॒शे । स्वाहा᳚ । प्र॒तीच्यै᳚ । दि॒शे । स्वाहा᳚ । उदी᳚च्यै । दि॒शे । स्वाहा᳚ । ऊ॒र्ध्वायै᳚ । दि॒शे । स्वाहा᳚ । दि॒ग्भ्य इति॑ दिक् - भ्यः । स्वाहा᳚ । अ॒वा॒न्त॒र॒दि॒शाभ्य॒ इत्य॑वान्तर - दि॒शाभ्यः॑ । स्वाहा᳚ । समा᳚भ्यः । स्वाहा᳚ । श॒रद्भ्य॒ इति॑ श॒रत् - भ्यः॒ । स्वाहा᳚ । अ॒हो॒रा॒त्रेभ्य॒ इत्य॑हः - रा॒त्रेभ्यः॑ । स्वाहा᳚ । अ॒द्‌र्ध॒मा॒सेभ्य॒ इत्य॑द्‌र्ध-मा॒सेभ्यः॑ । स्वाहा᳚ । मासे᳚भ्यः । स्वाहा᳚ । ऋ॒तुभ्य॒ इत्यृ॒तु - भ्यः॒ । स्वाहा᳚ । सं॒ॅव॒थ्स॒रायेति॑ सं - व॒थ्स॒राय॑ । स्वाहा᳚ । सर्व॑स्मै । स्वाहा᳚ ॥  \newline




\markright{ TS 7.1.16.1  \hfill https://www.vedavms.in \hfill}

\section{ TS 7.1.16.1 }

\textbf{TS 7.1.16.1 } \newline
\textbf{Samhita Paata} \newline

अ॒ग्नये॒ स्वाहा॒ सोमा॑य॒ स्वाहा॑ सवि॒त्रे स्वाहा॒ सर॑स्वत्यै॒ स्वाहा॑ पू॒ष्णे स्वाहा॒ बृह॒स्पत॑ये॒ स्वाहा॒ ऽपां मोदा॑य॒ स्वाहा॑ वा॒यवे॒ स्वाहा॑ मि॒त्राय॒ स्वाहा॒ वरु॑णाय॒ स्वाहा॒ सर्व॑स्मै॒ स्वाहा᳚ ॥ \newline

\textbf{Pada Paata} \newline

अ॒ग्नये᳚ । स्वाहा᳚ । सोमा॑य । स्वाहा᳚ । स॒वि॒त्रे । स्वाहा᳚ । सर॑स्वत्यै । स्वाहा᳚ । पू॒ष्णे । स्वाहा᳚ । बृह॒स्पत॑ये । स्वाहा᳚ । अ॒पाम् । मोदा॑य । स्वाहा᳚ । वा॒यवे᳚ । स्वाहा᳚ । मि॒त्राय॑ । स्वाहा᳚ । वरु॑णाय । स्वाहा᳚ । सर्व॑स्मै । स्वाहा᳚ ॥  \newline




\markright{ TS 7.1.17.1  \hfill https://www.vedavms.in \hfill}

\section{ TS 7.1.17.1 }

\textbf{TS 7.1.17.1 } \newline
\textbf{Samhita Paata} \newline

पृ॒थि॒व्यै स्वाहा॒ ऽन्तरि॑क्षाय॒ स्वाहा॑ दि॒वे स्वाहा॒ ऽग्नये॒ स्वाहा॒ सोमा॑य॒ स्वाहा॒ सूर्या॑य॒ स्वाहा॑ च॒न्द्रम॑से॒ स्वाहा ऽह्ने॒ स्वाहा॒ रात्रि॑यै॒ स्वाह॒र्जवे॒ स्वाहा॑ सा॒धवे॒ स्वाहा॑ सुक्षि॒त्यै स्वाहा᳚ क्षु॒धे स्वाहा॑ ऽऽशिति॒म्ने स्वाहा॒ रोगा॑य॒ स्वाहा॑ हि॒माय॒ स्वाहा॑ शी॒ताय॒ स्वाहा॑ ऽऽत॒पाय॒ स्वाहा ऽर॑ण्याय॒ स्वाहा॑ सुव॒र्गाय॒ स्वाहा॑ लो॒काय॒ स्वाहा॒ सर्व॑स्मै॒ स्वाहा᳚ ॥ \newline

\textbf{Pada Paata} \newline

पृ॒थि॒व्यै । स्वाहा᳚ । अ॒न्तरि॑क्षाय । स्वाहा᳚ । दि॒वे । स्वाहा᳚ । अ॒ग्नये᳚ । स्वाहा᳚ । सोमा॑य । स्वाहा᳚ । सूर्या॑य । स्वाहा᳚ । च॒न्द्रम॑से । स्वाहा᳚ । अह्ने᳚ । स्वाहा᳚ । रात्रि॑यै । स्वाहा᳚ । ऋ॒जवे᳚ । स्वाहा᳚ । सा॒धवे᳚ । स्वाहा᳚ । सु॒क्षि॒त्या इति॑ सु - क्षि॒त्यै । स्वाहा᳚ । क्षु॒धे । स्वाहा᳚ । आ॒शि॒ति॒म्ने । स्वाहा᳚ । रोगा॑य । स्वाहा᳚ । हि॒माय॑ । स्वाहा᳚ । शी॒ताय॑ । स्वाहा᳚ । आ॒त॒पायेत्या᳚ - त॒पाय॑ । स्वाहा᳚ । अर॑ण्याय । स्वाहा᳚ । सु॒व॒र्गायेति॑ सुवः - गाय॑ । स्वाहा᳚ । लो॒काय॑ । स्वाहा᳚ । सर्व॑स्मै । स्वाहा᳚ ॥  \newline




\markright{ TS 7.1.18.1  \hfill https://www.vedavms.in \hfill}

\section{ TS 7.1.18.1 }

\textbf{TS 7.1.18.1 } \newline
\textbf{Samhita Paata} \newline

भुवो॑ दे॒वानां॒ कर्म॑णा॒ऽपस॒र्तस्य॑ प॒थ्या॑ऽसि॒ वसु॑भि-र्दे॒वेभि॑-र्दे॒वत॑या गाय॒त्रेण॑ त्वा॒ छन्द॑सा युनज्मि वस॒न्तेन॑ त्व॒र्तुना॑ ह॒विषा॑ दीक्षयामि रु॒द्रेभि॑-र्दे॒वेभि॑-र्दे॒वत॑या॒ त्रैष्टु॑भेन त्वा॒ छन्द॑सा युनज्मि ग्री॒ष्मेण॑ त्व॒र्तुना॑ ह॒विषा॑ दीक्षया-म्यादि॒त्येभि॑-र्दे॒वेभि॑-र्दे॒वत॑या॒ जाग॑तेन त्वा॒ छन्द॑सा युनज्मि व॒र्॒.षाभि॑स्त्व॒र्तुना॑ ह॒विषा॑ दीक्षयामि॒ विश्वे॑भि-र्दे॒वेभि॑-र्दे॒वत॒या ऽऽनु॑ष्टुभेन त्वा॒ छन्द॑सा युनज्मि - [  ] \newline

\textbf{Pada Paata} \newline

भुवः॑ । दे॒वाना᳚म् । कर्म॑णा । अ॒पसा᳚ । ऋ॒तस्य॑ । प॒थ्या᳚ । अ॒सि॒ । वसु॑भि॒रिति॒ वसु॑ - भिः॒ । दे॒वेभिः॑ । दे॒वत॑या । गा॒य॒त्रेण॑ । त्वा॒ । छन्द॑सा । यु॒न॒ज्मि॒ । व॒स॒न्तेन॑ । त्वा॒ । ऋ॒तुना᳚ । ह॒विषा᳚ । दी॒क्ष॒या॒मि॒ । रु॒द्रेभिः॑ । दे॒वेभिः॑ । दे॒वत॑या । त्रैष्टु॑भेन । त्वा॒ । छन्द॑सा । यु॒न॒ज्मि॒ । ग्री॒ष्मेण॑ । त्वा॒ । ऋ॒तुना᳚ । ह॒विषा᳚ । दी॒क्ष॒या॒मि॒ । आ॒दि॒त्येभिः॑ । दे॒वेभिः॑ । दे॒वत॑या । जाग॑तेन । त्वा॒ । छन्द॑सा । यु॒न॒ज्मि॒ । व॒र्॒.षाभिः॑ । त्वा॒ । ऋ॒तुना᳚ । ह॒विषा᳚ । दी॒क्ष॒या॒मि॒ । विश्वे॑भिः । दे॒वेभिः॑ । दे॒वत॑या । आनु॑ष्टुभे॒नेत्यानु॑ - स्तु॒भे॒न॒ । त्वा॒ । छन्द॑सा । यु॒न॒ज्मि॒ ।  \newline




\markright{ TS 7.1.18.2  \hfill https://www.vedavms.in \hfill}

\section{ TS 7.1.18.2 }

\textbf{TS 7.1.18.2 } \newline
\textbf{Samhita Paata} \newline

श॒रदा᳚ त्व॒र्तुना॑ ह॒विषा॑ दीक्षया॒म्यङ्गि॑रोभि-र्दे॒वेभि॑-र्दे॒वत॑या॒ पाङ्क्ते॑न त्वा॒ छन्द॑सा युनज्मि हेमन्तशिशि॒राभ्यां᳚ त्व॒र्तुना॑ ह॒विषा॑ दीक्षया॒म्याऽहं दी॒क्षाम॑रुहमृ॒तस्य॒ पत्नीं᳚ गाय॒त्रेण॒ छन्द॑सा॒ ब्रह्म॑णा च॒र्तꣳ स॒त्ये॑ऽधाꣳ स॒त्यमृ॒ते॑ऽधां ॥म॒ही मू ॒षु >1सु॒त्रामा॑ण >2-मि॒ह धृतिः॒ स्वाहे॒ह विधृ॑तिः॒ स्वाहे॒ह रन्तिः॒ स्वाहे॒ह रम॑तिः॒ स्वाहा᳚ ॥ \newline

\textbf{Pada Paata} \newline

श॒रदा᳚ । त्वा॒ । ऋ॒तुना᳚ । ह॒विषा᳚ । दी॒क्ष॒या॒मि॒ । अङ्गि॑रोभि॒रित्यङ्गि॑रः- भिः॒ । दे॒वेभिः॑ । दे॒वत॑या । पाङ्क्ते॑न । त्वा॒ । छन्द॑सा । यु॒न॒ज्मि॒ । हे॒म॒न्त॒शि॒शि॒राभ्या॒मिति॑ हेमन्त - शि॒शि॒राभ्या᳚म् । त्वा॒ । ऋ॒तुना᳚ । ह॒विषा᳚ । दी॒क्ष॒या॒मि॒ । एति॑ । अ॒हम् । दी॒क्षाम् । अ॒रु॒ह॒म् । ऋ॒तस्य॑ । पत्नी᳚म् । गा॒य॒त्रेण॑ । छन्द॑सा । ब्रह्म॑णा । च॒ । ऋ॒तम् । स॒त्ये । अ॒धा॒म् । स॒त्यम् । ऋ॒ते । अ॒धा॒म् ॥ म॒हीम् । उ॒ । स्विति॑ । सु॒त्रामा॑ण॒मिति॑ सु - त्रामा॑णम् । इ॒ह । धृतिः॑ । स्वाहा᳚ । इ॒ह । विधृ॑ति॒रिति॒ वि - धृ॒तिः॒ । स्वाहा᳚ । इ॒ह । रन्तिः॑ । स्वाहा᳚ । इ॒ह । रम॑तिः । स्वाहा᳚ ॥  \newline




\markright{ TS 7.1.19.1  \hfill https://www.vedavms.in \hfill}

\section{ TS 7.1.19.1 }

\textbf{TS 7.1.19.1 } \newline
\textbf{Samhita Paata} \newline

ई॒कां॒राय॒ स्वाहें कृ॑ताय॒ स्वाहा॒ क्रन्द॑ते॒ स्वाहा॑ ऽव॒क्रन्द॑ते॒ स्वाहा॒ प्रोथ॑ते॒ स्वाहा᳚ प्र॒प्रोथ॑ते॒ स्वाहा॑ ग॒न्धाय॒ स्वाहा᳚ घ्रा॒ताय॒ स्वाहा᳚ प्रा॒णाय॒ स्वाहा᳚ व्या॒नाय॒ स्वाहा॑ ऽपा॒नाय॒ स्वाहा॑ सन्दी॒यमा॑नाय॒ स्वाहा॒ सन्दि॑ताय॒ स्वाहा॑ विचृ॒त्यमा॑नाय॒ स्वाहा॒ विचृ॑त्ताय॒ स्वाहा॑ पलायि॒ष्यमा॑णाय॒ स्वाहा॒ पला॑यिताय॒ स्वाहो॑परꣳस्य॒ते स्वाहोप॑रताय॒ स्वाहा॑ निवेक्ष्य॒ते स्वाहा॑ निवि॒शमा॑नाय॒ स्वाहा॒ निवि॑ष्टाय॒ स्वाहा॑ निषथ्स्य॒ते स्वाहा॑ नि॒षीद॑ते॒ स्वाहा॒ निष॑ण्णाय॒ स्वाहा॑ - [  ] \newline

\textbf{Pada Paata} \newline

ई॒कां॒रायेती᳚म् - का॒राय॑ । स्वाहा᳚ । ईकृं॑ता॒येती᳚म् - कृ॒ता॒य॒ । स्वाहा᳚ । क्रन्द॑ते । स्वाहा᳚ । अ॒व॒क्रन्द॑त॒ इत्य॑व - क्रन्द॑ते । स्वाहा᳚ । प्रोथ॑ते । स्वाहा᳚ । प्र॒प्रोथ॑त॒ इति॑ प्र - प्रोथ॑ते । स्वाहा᳚ । ग॒न्धाय॑ । स्वाहा᳚ । घ्रा॒ताय॑ । स्वाहा᳚ । प्रा॒णायेति॑ प्र - अ॒नाय॑ । स्वाहा᳚ । व्या॒नायेति॑ वि - अ॒नाय॑ । स्वाहा᳚ । अ॒पा॒नायेत्य॑प - अ॒नाय॑ । स्वाहा᳚ । स॒दीं॒यमा॑ना॒येति॑ सं-दी॒यमा॑नाय । स्वाहा᳚ । संदि॑ता॒येति॒ सं-दि॒ता॒य॒ । स्वाहा᳚ । वि॒चृ॒त्यमा॑ना॒येति॑ वि - चृ॒त्यमा॑नाय । स्वाहा᳚ । विचृ॑त्ता॒येति॒ वि - चृ॒त्ता॒य॒ । स्वाहा᳚ । प॒ला॒यि॒ष्यमा॑णाय । स्वाहा᳚ । पला॑यिताय । स्वाहा᳚ । उ॒प॒रꣳ॒॒स्य॒त इत्यु॑प - रꣳ॒॒स्य॒ते । स्वाहा᳚ । उप॑रता॒येत्युप॑ - र॒ता॒य॒ । स्वाहा᳚ । नि॒वे॒क्ष्य॒त इति॑ नि - वे॒क्ष्य॒ते । स्वाहा᳚ । नि॒वि॒शमा॑ना॒येति॑ नि - वि॒शमा॑नाय । स्वाहा᳚ । निवि॑ष्टा॒येति॒ नि - वि॒ष्टा॒य॒ । स्वाहा᳚ । नि॒ष॒थ्स्य॒त इति॑ नि - स॒थ्स्य॒ते । स्वाहा᳚ । नि॒षीद॑त॒ इति॑ नि - सीद॑ते । स्वाहा᳚ । निष॑ण्णा॒येति॒ नि - स॒न्ना॒य॒ । स्वाहा᳚ ।  \newline




\markright{ TS 7.1.19.2  \hfill https://www.vedavms.in \hfill}

\section{ TS 7.1.19.2 }

\textbf{TS 7.1.19.2 } \newline
\textbf{Samhita Paata} \newline

ऽऽसिष्य॒ते स्वाहा ऽऽसी॑नाय॒ स्वाहा॑ ऽऽसि॒ताय॒ स्वाहा॑ निपथ्स्य॒ते स्वाहा॑ नि॒पद्य॑मानाय॒ स्वाहा॒ निप॑न्नाय॒ स्वाहा॑ शयिष्य॒ते स्वाहा॒ शया॑नाय॒ स्वाहा॑ शयि॒ताय॒ स्वाहा॑ संमीलिष्य॒ते स्वाहा॑ स॒मींल॑ते॒ स्वाहा॒ संमी॑लिताय॒ स्वाहा᳚ स्वफ्स्य॒ते स्वाहा᳚ स्वप॒ते स्वाहा॑ सु॒प्ताय॒ स्वाहा᳚ प्रभोथ्स्य॒ते स्वाहा᳚ प्र॒बुद्ध्य॑मानाय॒ स्वाहा॒ प्रबु॑द्धाय॒ स्वाहा॑ जागरिष्य॒ते स्वाहा॒ जाग्र॑ते॒ स्वाहा॑ जागरि॒ताय॒ स्वाहा॒ शुश्रू॑षमाणाय॒ स्वाहा॑ शृण्व॒ते स्वाहा᳚ श्रु॒ताय॒ स्वाहा॑ वीक्षिष्य॒ते स्वाहा॒ - [  ] \newline

\textbf{Pada Paata} \newline

आ॒सि॒ष्य॒ते । स्वाहा᳚ । आसी॑नाय । स्वाहा᳚ । आ॒सि॒ताय॑ । स्वाहा᳚ । नि॒प॒थ्स्य॒त इति॑ नि - प॒थ्स्य॒ते । स्वाहा᳚ । नि॒पद्य॑माना॒येति॑ नि - पद्य॑मानाय । स्वाहा᳚ । निप॑न्ना॒येति॒ नि - प॒न्ना॒य॒ । स्वाहा᳚ । श॒यि॒ष्य॒ते । स्वाहा᳚ । शया॑नाय । स्वाहा᳚ । श॒यि॒ताय॑ । स्वाहा᳚ । स॒म्मी॒लि॒ष्य॒त इति॑ सं - मी॒लि॒ष्य॒ते । स्वाहा᳚ । स॒म्मील॑त॒ इति॑ सं - मील॑ते । स्वाहा᳚ । सम्मी॑लिता॒येति॒ सं - मी॒लि॒ता॒य॒ । स्वाहा᳚ । स्व॒फ्स्य॒ते । स्वाहा᳚ । स्व॒प॒ते । स्वाहा᳚ । सु॒प्ताय॑ । स्वाहा᳚ । प्र॒भो॒थ्स्य॒त इति॑ प्र - भो॒थ्स्य॒ते । स्वाहा᳚ । प्र॒बुद्ध्य॑माना॒येति॑ प्र - बुद्ध्य॑मानाय । स्वाहा᳚ । प्रबु॑द्धा॒येति॒ प्र - बु॒द्धा॒य॒ । स्वाहा᳚ । जा॒ग॒रि॒ष्य॒ते । स्वाहा᳚ । जाग्र॑ते । स्वाहा᳚ । जा॒ग॒रि॒ताय॑ । स्वाहा᳚ । शुश्रू॑षमाणाय । स्वाहा᳚ । शृ॒ण्व॒ते । स्वाहा᳚ । श्रु॒ताय॑ । स्वाहा᳚ । वी॒क्षि॒ष्य॒त इति॑ वि - ई॒क्षि॒ष्य॒ते । स्वाहा᳚ ।  \newline




\markright{ TS 7.1.19.3  \hfill https://www.vedavms.in \hfill}

\section{ TS 7.1.19.3 }

\textbf{TS 7.1.19.3 } \newline
\textbf{Samhita Paata} \newline

वीक्ष॑माणाय॒ स्वाहा॒ वीक्षि॑ताय॒ स्वाहा॑ सꣳहास्य॒ते स्वाहा॑ स॒जिंहा॑नाय॒ स्वाहो॒-ज्जिहा॑नाय॒ स्वाहा॑ विवर्थ्स्य॒ते स्वाहा॑ वि॒वर्त॑मानाय॒ स्वाहा॒ विवृ॑त्ताय॒ स्वाहो᳚-त्थास्य॒ते स्वाहो॒त्तिष्ठ॑ते॒ स्वाहोत्थि॑ताय॒ स्वाहा॑ विधविष्य॒ते स्वाहा॑ विधून्वा॒नाय॒ स्वाहा॒ विधू॑ताय॒ स्वाहो᳚-त्क्रꣳस्य॒ते स्वाहो॒त्क्राम॑ते॒ स्वाहोत्क्रा᳚न्ताय॒ स्वाहा॑ चङ्क्रमिष्य॒ते स्वाहा॑ चङ्क्र॒म्यमा॑णाय॒ स्वाहा॑ चङ्क्रमि॒ताय॒ स्वाहा॑ कण्डूयिष्य॒ते स्वाहा॑ कण्डू॒यमा॑नाय॒ स्वाहा॑ कण्डूयि॒ताय॒ स्वाहा॑ निकषिष्य॒ते स्वाहा॑ नि॒कष॑माणाय॒ स्वाहा॒ ( ) निक॑षिताय॒ स्वाहा॒ यदत्ति॒ तस्मै॒ स्वाहा॒ यत् पिब॑ति॒ तस्मै॒ स्वाहा॒ यन्मेह॑ति॒ तस्मै॒ स्वाहा॒ यच्छकृ॑त् क॒रोति॒ तस्मै॒ स्वाहा॒ रेत॑से॒ स्वाहा᳚ प्र॒जाभ्यः॒ स्वाहा᳚ प्र॒जन॑नाय॒ स्वाहा॒ सर्व॑स्मै॒ स्वाहा᳚ ॥ \newline

\textbf{Pada Paata} \newline

वीक्ष॑माणा॒येति॑ वि-ईक्ष॑माणाय । स्वाहा᳚ । वीक्षि॑ता॒येति॒ वि-ई॒क्षि॒ता॒य॒ । स्वाहा᳚ । सꣳ॒॒हा॒स्य॒त इति॑ सं - हा॒स्य॒ते । स्वाहा᳚ । स॒जिंहा॑ना॒येति॑ सं - जिहा॑नाय । स्वाहा᳚ । उ॒ज्जिहा॑ना॒येत्यु॑त्- जिहा॑नाय । स्वाहा᳚ । वि॒व॒र्थ्स्य॒त इति॑ वि - व॒र्थ्स्य॒ते । स्वाहा᳚ । वि॒वर्त॑माना॒येति॑ वि - वर्त॑मानाय । स्वाहा᳚ । विवृ॑त्ता॒येति॒ वि-वृ॒त्ता॒य॒ । स्वाहा᳚ । उ॒त्था॒स्य॒त इत्यु॑त् - स्था॒स्य॒ते । स्वाहा᳚ । उ॒त्तिष्ठ॑त॒ इत्यु॑त्- तिष्ठ॑ते । स्वाहा᳚ । उत्थि॑ता॒येत्युत् - स्थि॒ता॒य॒ । स्वाहा᳚ । वि॒ध॒वि॒ष्य॒त इति॑ वि - ध॒वि॒ष्य॒ते । स्वाहा᳚ । वि॒धू॒न्वा॒नायेति॑ वि - धू॒न्वा॒नाय॑ । स्वाहा᳚ । विधू॑ता॒येति॒ वि - धू॒ता॒य॒ । स्वाहा᳚ । उ॒त्क्रꣳ॒॒स्य॒त इत्यु॑त्- क्रꣳ॒॒स्य॒ते । स्वाहा᳚ । उ॒त्क्राम॑त॒ इत्यु॑त् - क्राम॑ते । स्वाहा᳚ । उत्क्रा᳚न्ता॒येत्युत् - क्रा॒न्ता॒य॒ । स्वाहा᳚ । च॒ङ्क्र॒मि॒ष्य॒ते । स्वाहा᳚ । च॒ङ्क्र॒म्यमा॑णाय । स्वाहा᳚ । च॒ङ्क्र॒मि॒ताय॑ । स्वाहा᳚ । क॒ण्डू॒यि॒ष्य॒ते । स्वाहा᳚ । क॒ण्डू॒यमा॑नाय । स्वाहा᳚ । क॒ण्डू॒यि॒ताय॑ । स्वाहा᳚ । नि॒क॒षि॒ष्य॒त इति॑ नि - क॒षि॒ष्य॒ते । स्वाहा᳚ । नि॒कष॑माणा॒येति॑ नि - कष॑माणाय । स्वाहा᳚ ( ) । निक॑षिता॒येति॒ नि - क॒षि॒ता॒य॒ । स्वाहा᳚ । यत् । अत्ति॑ । तस्मै᳚ । स्वाहा᳚ । यत् । पिब॑ति । तस्मै᳚ । स्वाहा᳚ । यत् । मेह॑ति । तस्मै᳚ । स्वाहा᳚ । यत् । शकृ॑त् । क॒रोति॑ । तस्मै᳚ । स्वाहा᳚ । रेत॑से । स्वाहा᳚ । प्र॒जाभ्य॒ इति॑ प्र - जाभ्यः॑ । स्वाहा᳚ । प्र॒जन॑ना॒येति॑ प्र - जन॑नाय । स्वाहा᳚ । सर्व॑स्मै । स्वाहा᳚ ॥  \newline




\markright{ TS 7.1.20.1  \hfill https://www.vedavms.in \hfill}

\section{ TS 7.1.20.1 }

\textbf{TS 7.1.20.1 } \newline
\textbf{Samhita Paata} \newline

अ॒ग्नये॒ स्वाहा॑ वा॒यवे॒ स्वाहा॒ सूर्या॑य॒ स्वाहा॒-र्तम॑स्यृ॒तस्य॒र्तम॑सि स॒त्यम॑सि स॒त्यस्य॑ स॒त्यम॑स्यृ॒तस्य॒ पन्था॑ असि दे॒वानां᳚ छा॒याऽमृत॑स्य॒ नाम॒ तथ् स॒त्यं ॅयत् त्वं प्र॒जाप॑ति॒रस्यधि॒ यद॑स्मिन् वा॒जिनी॑व॒ शुभः॒ स्पर्द्ध॑न्ते॒ दिवः॒ सूर्ये॑ण॒ विशो॒ऽपो वृ॑णा॒नः प॑वते क॒व्यन् प॒शुं न गो॒पा इर्यः॒ परि॑ज्मा ॥ \newline

\textbf{Pada Paata} \newline

अ॒ग्नये᳚ । स्वाहा᳚ । वा॒यवे᳚ । स्वाहा᳚ । सूर्या॑य । स्वाहा᳚ । ऋ॒तम् । अ॒सि॒ । ऋ॒तस्य॑ । ऋ॒तम् । अ॒सि॒ । स॒त्यम् । अ॒सि॒ । स॒त्यस्य॑ । स॒त्यम् । अ॒सि॒ । ऋ॒तस्य॑ । पन्थाः᳚ । अ॒सि॒ । दे॒वाना᳚म् । छा॒या । अ॒मृत॑स्य । नाम॑ । तत् । स॒त्यम् । यत् । त्वम् । प्र॒जाप॑ति॒रिति॑ प्र॒जा - प॒तिः॒ । असि॑ । अधीति॑ । यत् । अ॒स्मि॒न्न् । वा॒जिनि॑ । इ॒व॒ । शुभः॑ । स्पद्‌र्ध॑न्ते । दिवः॑ । सूर्ये॑ण । विशः॑ । अ॒पः । वृ॒णा॒नः । प॒व॒ते॒ । क॒व्यन्न् । प॒शुम् । न । गो॒पा इति॑ गो - पाः । इर्यः॑ । परि॒ज्मेति॒ परि॑ - ज्मा॒ ॥  \newline






\end{document}