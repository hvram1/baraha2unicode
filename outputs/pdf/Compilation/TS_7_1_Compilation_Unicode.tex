\documentclass[17pt]{extarticle}
\usepackage{babel}
\usepackage{fontspec}
\usepackage{polyglossia}
\usepackage{extsizes}

\usepackage{color}   %May be necessary if you want to color links
\usepackage{hyperref}
\hypersetup{
    colorlinks=true, %set true if you want colored links
    linktoc=all,     %set to all if you want both sections and subsections linked
    linkcolor=black,  %choose some color if you want links to stand out
}

\setmainlanguage{sanskrit}
\setotherlanguages{english} %% or other languages
\setlength{\parindent}{0pt}
\pagestyle{myheadings}
\newfontfamily\devanagarifont[Script=Devanagari]{AdishilaVedic}
\renewcommand{\theHsection}{\thepart.section.\thesection}

\newcommand{\VAR}[1]{}
\newcommand{\BLOCK}[1]{}




\begin{document}
\begin{titlepage}
    \begin{center}
 
\begin{sanskrit}
    { \Large
    कृष्ण यजुर्वेदीय तैत्तिरीय संहिता,पद,जटा,घन पाठः 
    }
    \\
    \vspace{2.5cm}
    \mbox{ \Large
    7.1      सप्तमकाण्डे प्रथमः प्रश्नः- अश्वमेधगतमन्त्राणामभिधानं   }
\end{sanskrit}
\end{center}

\end{titlepage}
\tableofcontents
\phantomsection
\pagebreak

\markright{ TS 7.1.1.1  \hfill https://www.vedavms.in \hfill}

\section{ TS 7.1.1.1 }

\textbf{TS 7.1.1.1 } \newline
\textbf{Samhita Paata} \newline

प्र॒जन॑नं॒ ज्योति॑र॒ग्नि-र्दे॒वता॑नां॒ ज्योति॑र्वि॒राट् छन्द॑सां॒ ज्योति॑र्वि॒राड् वा॒चो᳚ऽग्नौ सं ति॑ष्ठते वि॒राज॑म॒भि संप॑द्यते॒ तस्मा॒त् तज्ज्योति॑रुच्यते॒ द्वौ स्तोमौ᳚ प्रातस्सव॒नं ॅव॑हतो॒ यथा᳚ प्रा॒णश्चा॑ऽपा॒नश्च॒ द्वौ माद्ध्य॑दिंनꣳ॒॒ सव॑नं॒ ॅयथा॒ चक्षु॑श्च॒ श्रोत्रं॑ च॒ द्वौ तृ॑तीयसव॒नं ॅयथा॒ वाक् च॑ प्रति॒ष्ठा च॒ पुरु॑षसंमितो॒ वा ए॒ष य॒ज्ञोऽस्थू॑रि॒ - [  ] \newline

\textbf{Pada Paata} \newline

प्र॒जन॑न॒मिति॑ प्र-जन॑नम् । ज्योतिः॑ । अ॒ग्निः । दे॒वता॑नाम् । ज्योतिः॑ । वि॒राडिति॑ वि - राट् । छन्द॑साम् । ज्योतिः॑ । वि॒राडिति॑ वि - राट् । वा॒चः । अ॒ग्नौ । समिति॑ । ति॒ष्ठ॒ते॒ । वि॒राज॒मिति॑ वि - राज᳚म् । अ॒भि । समिति॑ । प॒द्य॒ते॒ । तस्मा᳚त् । तत् । ज्योतिः॑ । उ॒च्य॒ते॒ । द्वौ । स्तोमौ᳚ । प्रा॒त॒स्स॒व॒नमिति॑ प्रातः - स॒व॒नम् । व॒ह॒तः॒ । यथा᳚ । प्रा॒ण इति॑ प्र - अ॒नः । च॒ । अ॒पा॒न इत्य॑प - अ॒नः । च॒ । द्वौ । माद्ध्य॑न्दिनम् । सव॑नम् । यथा᳚ । चक्षुः॑ । च॒ । श्रोत्र᳚म् । च॒ । द्वौ । तृ॒ती॒य॒स॒व॒नमिति॑ तृतीय - स॒व॒नम् । यथा᳚ । वाक् । च॒ । प्र॒ति॒ष्ठेति॑ प्रति - स्था । च॒ । पुरु॑षसम्मित॒ इति॒ पुरु॑ष - स॒म्मि॒तः॒ । वै । ए॒षः । य॒ज्ञ्ः । अस्थू॑रिः ।  \newline


\textbf{Krama Paata} \newline

प्र॒जन॑न॒म् ज्योतिः॑ । प्र॒जन॑न॒मिति॑ प्र - जन॑नम् । ज्योति॑र॒ग्निः । अ॒ग्निर् दे॒वता॑नाम् । दे॒वता॑ना॒म् ज्योतिः॑ । ज्योति॑र् वि॒राट् । वि॒राट् छन्द॑साम् । वि॒राडिति॑ वि - राट् । छन्द॑सा॒म् ज्योतिः॑ । ज्योति॑र् वि॒राट् । वि॒राड् वा॒चः । वि॒राडिति॑ वि - राट् । 
वा॒चो᳚ऽग्नौ । अ॒ग्नौ सम् । सम् ति॑ष्ठते । ति॒ष्ठ॒ते॒ वि॒राज᳚म् । वि॒राज॑म॒भि । वि॒राज॒मिति॑ वि - राज᳚म् । अ॒भि सम् । सम् प॑द्यते । प॒द्य॒ते॒ तस्मा᳚त् । तस्मा॒त् तत् । तज् ज्योतिः॑ । ज्योति॑रुच्यते । उ॒च्य॒ते॒ द्वौ । द्वौ स्तोमौ᳚ । स्तोमौ᳚ प्रातस्सव॒नम् । प्रा॒त॒स्स॒व॒नम् ॅव॑हतः । प्रा॒त॒स्स॒व॒नमिति॑ प्रातः - स॒व॒नम् । व॒ह॒तो॒ यथा᳚ । यथा᳚ प्रा॒णः । प्रा॒णश्च॑ । प्रा॒ण इति॑ प्र - अ॒नः । चा॒पा॒नः । अ॒पा॒नश्च॑ । अ॒पा॒न इत्य॑प - अ॒नः । च॒ द्वौ । द्वौ माद्ध्य॑न्दिनम् । माद्ध्य॑न्दिनꣳ॒॒ सव॑नम् । सव॑न॒म् ॅयथा᳚ । यथा॒ चक्षुः॑ । चक्षु॑श्च । च॒ श्रोत्र᳚म् । श्रोत्र॑म् च । च॒ द्वौ । द्वौ तृ॑तीयसव॒नम् । तृ॒ती॒य॒स॒व॒नम् ॅयथा᳚ । तृ॒ती॒य॒स॒व॒नमिति॑ तृतीय - स॒व॒नम् । यथा॒ वाक् । वाक् च॑ । च॒ प्र॒ति॒ष्ठा । प्र॒ति॒ष्ठा च॑ । प्र॒ति॒ष्ठेति॑ प्रति - स्था । च॒ पुरु॑षसम्मितः । पुरु॑षसम्मितो॒ वै । पुरु॑षसम्मित॒ इति॒ पुरु॑ष - स॒म्मि॒तः॒ । वा ए॒षः । ए॒ष य॒ज्ञ्ः । य॒ज्ञोऽस्थू॑रिः । अस्थू॑रि॒र् यम् \newline

\textbf{Jatai Paata} \newline

1. प्र॒जन॑न॒म् ज्योति॒र् ज्योतिः॑ प्र॒जन॑नम् प्र॒जन॑न॒म् ज्योतिः॑ । \newline
2. प्र॒जन॑न॒मिति॑ प्र - जन॑नम् । \newline
3. ज्योति॑ र॒ग्नि र॒ग्निर् ज्योति॒र् ज्योति॑ र॒ग्निः । \newline
4. अ॒ग्निर् दे॒वता॑नाम् दे॒वता॑ना म॒ग्नि र॒ग्निर् दे॒वता॑नाम् । \newline
5. दे॒वता॑ना॒म् ज्योति॒र् ज्योति॑र् दे॒वता॑नाम् दे॒वता॑ना॒म् ज्योतिः॑ । \newline
6. ज्योति॑र् वि॒राड् वि॒राड् ज्योति॒र् ज्योति॑र् वि॒राट् । \newline
7. वि॒राट् छन्द॑सा॒म् छन्द॑सां ॅवि॒राड् वि॒राट् छन्द॑साम् । \newline
8. वि॒राडिति॑ वि - राट् । \newline
9. छन्द॑सा॒म् ज्योति॒र् ज्योति॒ श्छन्द॑सा॒म् छन्द॑सा॒म् ज्योतिः॑ । \newline
10. ज्योति॑र् वि॒राड् वि॒राड् ज्योति॒र् ज्योति॑र् वि॒राट् । \newline
11. वि॒राड् वा॒चो वा॒चो वि॒राड् वि॒राड् वा॒चः । \newline
12. वि॒राडिति॑ वि - राट् । \newline
13. वा॒चो᳚ ऽग्ना व॒ग्नौ वा॒चो वा॒चो᳚ ऽग्नौ । \newline
14. अ॒ग्नौ सꣳ स म॒ग्ना व॒ग्नौ सम् । \newline
15. सम् ति॑ष्ठते तिष्ठते॒ सꣳ सम् ति॑ष्ठते । \newline
16. ति॒ष्ठ॒ते॒ वि॒राजं॑ ॅवि॒राज॑म् तिष्ठते तिष्ठते वि॒राज᳚म् । \newline
17. वि॒राज॑ म॒भ्य॑भि वि॒राजं॑ ॅवि॒राज॑ म॒भि । \newline
18. वि॒राज॒मिति॑ वि - राज᳚म् । \newline
19. अ॒भि सꣳ स म॒भ्य॑भि सम् । \newline
20. सम् प॑द्यते पद्यते॒ सꣳ सम् प॑द्यते । \newline
21. प॒द्य॒ते॒ तस्मा॒त् तस्मा᳚त् पद्यते पद्यते॒ तस्मा᳚त् । \newline
22. तस्मा॒त् तत् तत् तस्मा॒त् तस्मा॒त् तत् । \newline
23. तज् ज्योति॒र् ज्योति॒ स्तत् तज् ज्योतिः॑ । \newline
24. ज्योति॑ रुच्यत उच्यते॒ ज्योति॒र् ज्योति॑ रुच्यते । \newline
25. उ॒च्य॒ते॒ द्वौ द्वा वु॑च्यत उच्यते॒ द्वौ । \newline
26. द्वौ स्तोमौ॒ स्तोमौ॒ द्वौ द्वौ स्तोमौ᳚ । \newline
27. स्तोमौ᳚ प्रातस्सव॒नम् प्रा॑तस्सव॒नꣳ स्तोमौ॒ स्तोमौ᳚ प्रातस्सव॒नम् । \newline
28. प्रा॒त॒स्स॒व॒नं ॅव॑हतो वहतः प्रातस्सव॒नम् प्रा॑तस्सव॒नं ॅव॑हतः । \newline
29. प्रा॒त॒स्स॒व॒नमिति॑ प्रातः - स॒व॒नम् । \newline
30. व॒ह॒तो॒ यथा॒ यथा॑ वहतो वहतो॒ यथा᳚ । \newline
31. यथा᳚ प्रा॒णः प्रा॒णो यथा॒ यथा᳚ प्रा॒णः । \newline
32. प्रा॒ण श्च॑ च प्रा॒णः प्रा॒ण श्च॑ । \newline
33. प्रा॒ण इति॑ प्र - अ॒नः । \newline
34. चा॒पा॒नो॑ ऽपा॒न श्च॑ चापा॒नः । \newline
35. अ॒पा॒न श्च॑ चा पा॒नो॑ ऽपा॒न श्च॑ । \newline
36. अ॒पा॒न इत्य॑प - अ॒नः । \newline
37. च॒ द्वौ द्वौ च॑ च॒ द्वौ । \newline
38. द्वौ माद्ध्य॑न्दिन॒म् माद्ध्य॑न्दिन॒म् द्वौ द्वौ माद्ध्य॑न्दिनम् । \newline
39. माद्ध्य॑न्दिनꣳ॒॒ सव॑नꣳ॒॒ सव॑न॒म् माद्ध्य॑न्दिन॒म् माद्ध्य॑न्दिनꣳ॒॒ सव॑नम् । \newline
40. सव॑नं॒ ॅयथा॒ यथा॒ सव॑नꣳ॒॒ सव॑नं॒ ॅयथा᳚ । \newline
41. यथा॒ चक्षु॒ श्चक्षु॒र् यथा॒ यथा॒ चक्षुः॑ । \newline
42. चक्षु॑ श्च च॒ चक्षु॒ श्चक्षु॑ श्च । \newline
43. च॒ श्रोत्रꣳ॒॒ श्रोत्र॑म् च च॒ श्रोत्र᳚म् । \newline
44. श्रोत्र॑म् च च॒ श्रोत्रꣳ॒॒ श्रोत्र॑म् च । \newline
45. च॒ द्वौ द्वौ च॑ च॒ द्वौ । \newline
46. द्वौ तृ॑तीयसव॒नम् तृ॑तीयसव॒नम् द्वौ द्वौ तृ॑तीयसव॒नम् । \newline
47. तृ॒ती॒य॒स॒व॒नं ॅयथा॒ यथा॑ तृतीयसव॒नम् तृ॑तीयसव॒नं ॅयथा᳚ । \newline
48. तृ॒ती॒य॒स॒व॒नमिति॑ तृतीय - स॒व॒नम् । \newline
49. यथा॒ वाग् वाग् यथा॒ यथा॒ वाक् । \newline
50. वाक् च॑ च॒ वाग् वाक् च॑ । \newline
51. च॒ प्र॒ति॒ष्ठा प्र॑ति॒ष्ठा च॑ च प्रति॒ष्ठा । \newline
52. प्र॒ति॒ष्ठा च॑ च प्रति॒ष्ठा प्र॑ति॒ष्ठा च॑ । \newline
53. प्र॒ति॒ष्ठेति॑ प्रति - स्था । \newline
54. च॒ पुरु॑षसम्मितः॒ पुरु॑षसम्मितश्च च॒ पुरु॑षसम्मितः । \newline
55. पुरु॑षसम्मितो॒ वै वै पुरु॑षसम्मितः॒ पुरु॑षसम्मितो॒ वै । \newline
56. पुरु॑षसम्मित॒ इति॒ पुरु॑ष - स॒म्मि॒तः॒ । \newline
57. वा ए॒ष ए॒ष वै वा ए॒षः । \newline
58. ए॒ष य॒ज्ञो य॒ज्ञ् ए॒ष ए॒ष य॒ज्ञ्ः । \newline
59. य॒ज्ञो ऽस्थू॑रि॒ रस्थू॑रिर् य॒ज्ञो य॒ज्ञो ऽस्थू॑रिः । \newline
60. अस्थू॑रि॒र् यं ॅय मस्थू॑रि॒ रस्थू॑रि॒र् यम् । \newline

\textbf{Ghana Paata } \newline

1. प्र॒जन॑न॒म् ज्योति॒र् ज्योतिः॑ प्र॒जन॑नम् प्र॒जन॑न॒म् ज्योति॑ र॒ग्नि र॒ग्निर् ज्योतिः॑ प्र॒जन॑नम् प्र॒जन॑न॒म् ज्योति॑ र॒ग्निः । \newline
2. प्र॒जन॑न॒मिति॑ प्र - जन॑नम् । \newline
3. ज्योति॑ र॒ग्नि र॒ग्निर् ज्योति॒र् ज्योति॑ र॒ग्निर् दे॒वता॑नाम् दे॒वता॑ना म॒ग्निर् ज्योति॒र् ज्योति॑ र॒ग्निर् दे॒वता॑नाम् । \newline
4. अ॒ग्निर् दे॒वता॑नाम् दे॒वता॑ना म॒ग्नि र॒ग्निर् दे॒वता॑ना॒म् ज्योति॒र् ज्योति॑र् दे॒वता॑ना म॒ग्नि र॒ग्निर् दे॒वता॑ना॒म् ज्योतिः॑ । \newline
5. दे॒वता॑ना॒म् ज्योति॒र् ज्योति॑र् दे॒वता॑नाम् दे॒वता॑ना॒म् ज्योति॑र् वि॒राड् वि॒राड् ज्योति॑र् दे॒वता॑नाम् दे॒वता॑ना॒म् ज्योति॑र् वि॒राट् । \newline
6. ज्योति॑र् वि॒राड् वि॒राड् ज्योति॒र् ज्योति॑र् वि॒राट् छन्द॑सा॒म् छन्द॑सां ॅवि॒राड् ज्योति॒र् ज्योति॑र् वि॒राट् छन्द॑साम् । \newline
7. वि॒राट् छन्द॑सा॒म् छन्द॑सां ॅवि॒राड् वि॒राट् छन्द॑सा॒म् ज्योति॒र् ज्योति॒ श्छन्द॑सां ॅवि॒राड् वि॒राट् छन्द॑सा॒म् ज्योतिः॑ । \newline
8. वि॒राडिति॑ वि - राट् । \newline
9. छन्द॑सा॒म् ज्योति॒र् ज्योति॒ श्छन्द॑सा॒म् छन्द॑सा॒म् ज्योति॑र् वि॒राड् वि॒राड् ज्योति॒ श्छन्द॑सा॒म् छन्द॑सा॒म् ज्योति॑र् वि॒राट् । \newline
10. ज्योति॑र् वि॒राड् वि॒राड् ज्योति॒र् ज्योति॑र् वि॒राड् वा॒चो वा॒चो वि॒राड् ज्योति॒र् ज्योति॑र् वि॒राड् वा॒चः । \newline
11. वि॒राड् वा॒चो वा॒चो वि॒राड् वि॒राड् वा॒चो᳚ ऽग्ना व॒ग्नौ वा॒चो वि॒राड् वि॒राड् वा॒चो᳚ ऽग्नौ । \newline
12. वि॒राडिति॑ वि - राट् । \newline
13. वा॒चो᳚ ऽग्ना व॒ग्नौ वा॒चो वा॒चो᳚ ऽग्नौ सꣳ स म॒ग्नौ वा॒चो वा॒चो᳚ ऽग्नौ सम् । \newline
14. अ॒ग्नौ सꣳ स म॒ग्ना व॒ग्नौ सम् ति॑ष्ठते तिष्ठते॒ स म॒ग्ना व॒ग्नौ सम् ति॑ष्ठते । \newline
15. सम् ति॑ष्ठते तिष्ठते॒ सꣳ सम् ति॑ष्ठते वि॒राजं॑ ॅवि॒राज॑म् तिष्ठते॒ सꣳ सम् ति॑ष्ठते वि॒राज᳚म् । \newline
16. ति॒ष्ठ॒ते॒ वि॒राजं॑ ॅवि॒राज॑म् तिष्ठते तिष्ठते वि॒राज॑ म॒भ्य॑भि वि॒राज॑म् तिष्ठते तिष्ठते वि॒राज॑ म॒भि । \newline
17. वि॒राज॑ म॒भ्य॑भि वि॒राजं॑ ॅवि॒राज॑ म॒भि सꣳ स म॒भि वि॒राजं॑ ॅवि॒राज॑ म॒भि सम् । \newline
18. वि॒राज॒मिति॑ वि - राज᳚म् । \newline
19. अ॒भि सꣳ स म॒भ्य॑भि सम् प॑द्यते पद्यते॒ स म॒भ्य॑भि सम् प॑द्यते । \newline
20. सम् प॑द्यते पद्यते॒ सꣳ सम् प॑द्यते॒ तस्मा॒त् तस्मा᳚त् पद्यते॒ सꣳ सम् प॑द्यते॒ तस्मा᳚त् । \newline
21. प॒द्य॒ते॒ तस्मा॒त् तस्मा᳚त् पद्यते पद्यते॒ तस्मा॒त् तत् तत् तस्मा᳚त् पद्यते पद्यते॒ तस्मा॒त् तत् । \newline
22. तस्मा॒त् तत् तत् तस्मा॒त् तस्मा॒त् तज् ज्योति॒र् ज्योति॒ स्तत् तस्मा॒त् तस्मा॒त् तज् ज्योतिः॑ । \newline
23. तज् ज्योति॒र् ज्योति॒ स्तत् तज् ज्योति॑ रुच्यत उच्यते॒ ज्योति॒ स्तत् तज् ज्योति॑ रुच्यते । \newline
24. ज्योति॑ रुच्यत उच्यते॒ ज्योति॒र् ज्योति॑ रुच्यते॒ द्वौ द्वा वु॑च्यते॒ ज्योति॒र् ज्योति॑ रुच्यते॒ द्वौ । \newline
25. उ॒च्य॒ते॒ द्वौ द्वा वु॑च्यत उच्यते॒ द्वौ स्तोमौ॒ स्तोमौ॒ द्वा वु॑च्यत उच्यते॒ द्वौ स्तोमौ᳚ । \newline
26. द्वौ स्तोमौ॒ स्तोमौ॒ द्वौ द्वौ स्तोमौ᳚ प्रातस्सव॒नम् प्रा॑तस्सव॒नꣳ स्तोमौ॒ द्वौ द्वौ स्तोमौ᳚ प्रातस्सव॒नम् । \newline
27. स्तोमौ᳚ प्रातस्सव॒नम् प्रा॑तस्सव॒नꣳ स्तोमौ॒ स्तोमौ᳚ प्रातस्सव॒नं ॅव॑हतो वहतः प्रातस्सव॒नꣳ स्तोमौ॒ स्तोमौ᳚ प्रातस्सव॒नं ॅव॑हतः । \newline
28. प्रा॒त॒स्स॒व॒नं ॅव॑हतो वहतः प्रातस्सव॒नम् प्रा॑तस्सव॒नं ॅव॑हतो॒ यथा॒ यथा॑ वहतः प्रातस्सव॒नम् प्रा॑तस्सव॒नं ॅव॑हतो॒ यथा᳚ । \newline
29. प्रा॒त॒स्स॒व॒नमिति॑ प्रातः - स॒व॒नम् । \newline
30. व॒ह॒तो॒ यथा॒ यथा॑ वहतो वहतो॒ यथा᳚ प्रा॒णः प्रा॒णो यथा॑ वहतो वहतो॒ यथा᳚ प्रा॒णः । \newline
31. यथा᳚ प्रा॒णः प्रा॒णो यथा॒ यथा᳚ प्रा॒णश्च॑ च प्रा॒णो यथा॒ यथा᳚ प्रा॒णश्च॑ । \newline
32. प्रा॒ण श्च॑ च प्रा॒णः प्रा॒ण श्चा॑पा॒नो॑ ऽपा॒न श्च॑ प्रा॒णः प्रा॒ण श्चा॑पा॒नः । \newline
33. प्रा॒ण इति॑ प्र - अ॒नः । \newline
34. चा॒पा॒नो॑ ऽपा॒न श्च॑ चापा॒न श्च॑ चापा॒न श्च॑ चापा॒न श्च॑ । \newline
35. अ॒पा॒न श्च॑ चापा॒नो॑ ऽपा॒न श्च॒ द्वौ द्वौ चा॑पा॒नो॑ ऽपा॒न श्च॒ द्वौ । \newline
36. अ॒पा॒न इत्य॑प - अ॒नः । \newline
37. च॒ द्वौ द्वौ च॑ च॒ द्वौ माद्ध्य॑न्दिन॒म् माद्ध्य॑न्दिन॒म् द्वौ च॑ च॒ द्वौ माद्ध्य॑न्दिनम् । \newline
38. द्वौ माद्ध्य॑न्दिन॒म् माद्ध्य॑न्दिन॒म् द्वौ द्वौ माद्ध्य॑न्दिनꣳ॒॒ सव॑नꣳ॒॒ सव॑न॒म् माद्ध्य॑न्दिन॒म् द्वौ द्वौ माद्ध्य॑न्दिनꣳ॒॒ सव॑नम् । \newline
39. माद्ध्य॑न्दिनꣳ॒॒ सव॑नꣳ॒॒ सव॑न॒म् माद्ध्य॑न्दिन॒म् माद्ध्य॑न्दिनꣳ॒॒ सव॑नं॒ ॅयथा॒ यथा॒ सव॑न॒म् माद्ध्य॑न्दिन॒म् माद्ध्य॑न्दिनꣳ॒॒ सव॑नं॒ ॅयथा᳚ । \newline
40. सव॑नं॒ ॅयथा॒ यथा॒ सव॑नꣳ॒॒ सव॑नं॒ ॅयथा॒ चक्षु॒ श्चक्षु॒र् यथा॒ सव॑नꣳ॒॒ सव॑नं॒ ॅयथा॒ चक्षुः॑ । \newline
41. यथा॒ चक्षु॒ श्चक्षु॒र् यथा॒ यथा॒ चक्षु॑ श्च च॒ चक्षु॒र् यथा॒ यथा॒ चक्षु॑ श्च । \newline
42. चक्षु॑ श्च च॒ चक्षु॒ श्चक्षु॑ श्च॒ श्रोत्रꣳ॒॒ श्रोत्र॑म् च॒ चक्षु॒ श्चक्षु॑ श्च॒ श्रोत्र᳚म् । \newline
43. च॒ श्रोत्रꣳ॒॒ श्रोत्र॑म् च च॒ श्रोत्र॑म् च च॒ श्रोत्र॑म् च च॒ श्रोत्र॑म् च । \newline
44. श्रोत्र॑म् च च॒ श्रोत्रꣳ॒॒ श्रोत्र॑म् च॒ द्वौ द्वौ च॒ श्रोत्रꣳ॒॒ श्रोत्र॑म् च॒ द्वौ । \newline
45. च॒ द्वौ द्वौ च॑ च॒ द्वौ तृ॑तीयसव॒नम् तृ॑तीयसव॒नम् द्वौ च॑ च॒ द्वौ तृ॑तीयसव॒नम् । \newline
46. द्वौ तृ॑तीयसव॒नम् तृ॑तीयसव॒नम् द्वौ द्वौ तृ॑तीयसव॒नं ॅयथा॒ यथा॑ तृतीयसव॒नम् द्वौ द्वौ तृ॑तीयसव॒नं ॅयथा᳚ । \newline
47. तृ॒ती॒य॒स॒व॒नं ॅयथा॒ यथा॑ तृतीयसव॒नम् तृ॑तीयसव॒नं ॅयथा॒ वाग् वाग् यथा॑ तृतीयसव॒नम् तृ॑तीयसव॒नं ॅयथा॒ वाक् । \newline
48. तृ॒ती॒य॒स॒व॒नमिति॑ तृतीय - स॒व॒नम् । \newline
49. यथा॒ वाग् वाग् यथा॒ यथा॒ वाक् च॑ च॒ वाग् यथा॒ यथा॒ वाक् च॑ । \newline
50. वाक् च॑ च॒ वाग् वाक् च॑ प्रति॒ष्ठा प्र॑ति॒ष्ठा च॒ वाग् वाक् च॑ प्रति॒ष्ठा । \newline
51. च॒ प्र॒ति॒ष्ठा प्र॑ति॒ष्ठा च॑ च प्रति॒ष्ठा च॑ च प्रति॒ष्ठा च॑ च प्रति॒ष्ठा च॑ । \newline
52. प्र॒ति॒ष्ठा च॑ च प्रति॒ष्ठा प्र॑ति॒ष्ठा च॒ पुरु॑षसम्मितः॒ पुरु॑षसम्मित श्च प्रति॒ष्ठा प्र॑ति॒ष्ठा च॒ पुरु॑षसम्मितः । \newline
53. प्र॒ति॒ष्ठेति॑ प्रति - स्था । \newline
54. च॒ पुरु॑षसम्मितः॒ पुरु॑षसम्मित श्च च॒ पुरु॑षसम्मितो॒ वै वै पुरु॑षसम्मित श्च च॒ पुरु॑षसम्मितो॒ वै । \newline
55. पुरु॑षसम्मितो॒ वै वै पुरु॑षसम्मितः॒ पुरु॑षसम्मितो॒ वा ए॒ष ए॒ष वै पुरु॑षसम्मितः॒ पुरु॑षसम्मितो॒ वा ए॒षः । \newline
56. पुरु॑षसम्मित॒ इति॒ पुरु॑ष - स॒म्मि॒तः॒ । \newline
57. वा ए॒ष ए॒ष वै वा ए॒ष य॒ज्ञो य॒ज्ञ् ए॒ष वै वा ए॒ष य॒ज्ञ्ः । \newline
58. ए॒ष य॒ज्ञो य॒ज्ञ् ए॒ष ए॒ष य॒ज्ञो ऽस्थू॑रि॒ रस्थू॑रिर् य॒ज्ञ् ए॒ष ए॒ष य॒ज्ञो ऽस्थू॑रिः । \newline
59. य॒ज्ञो ऽस्थू॑रि॒ रस्थू॑रिर् य॒ज्ञो य॒ज्ञो ऽस्थू॑रि॒र् यं ॅय मस्थू॑रिर् य॒ज्ञो य॒ज्ञो ऽस्थू॑रि॒र् यम् । \newline
60. अस्थू॑रि॒र् यं ॅय मस्थू॑रि॒ रस्थू॑रि॒र् यम् काम॒म् कामं॒ ॅय मस्थू॑रि॒ रस्थू॑रि॒र् यम् काम᳚म् । \newline
\pagebreak
\markright{ TS 7.1.1.2  \hfill https://www.vedavms.in \hfill}

\section{ TS 7.1.1.2 }

\textbf{TS 7.1.1.2 } \newline
\textbf{Samhita Paata} \newline

र्यं कामं॑ का॒मय॑ते॒ तमे॒तेना॒भ्य॑श्नुते॒ सर्वꣳ॒॒ ह्यस्थु॑रिणाऽभ्यश्नु॒ते᳚ ऽग्निष्टो॒मेन॒ वै प्र॒जाप॑तिः प्र॒जा अ॑सृजत॒ ता अ॑ग्निष्टो॒मेनै॒व पर्य॑गृह्णा॒त् तासां॒ परि॑गृहीताना-मश्वत॒रोऽत्य॑प्रवत॒ तस्या॑नु॒हाय॒रेत॒ आऽद॑त्त॒ तद्-ग॑र्द॒भे न्य॑मा॒र्ट् तस्मा᳚द्-गर्द॒भो द्वि॒रेता॒ अथो॑ आहु॒र्वड॑बायां॒ न्य॑मा॒र्डिति॒ तस्मा॒द्-वड॑बा द्वि॒रेता॒ अथो॑ आहु॒रोष॑धीषु॒ - [  ] \newline

\textbf{Pada Paata} \newline

यम् । काम᳚म् । का॒मय॑ते । तम् । ए॒तेन॑ । अ॒भीति॑ । अ॒श्नु॒ते॒ । सर्व᳚म् । हि । अस्थू॑रिणा । अ॒भ्य॒श्नु॒त इत्य॑भि - अ॒श्नु॒ते । अ॒ग्नि॒ष्टो॒मेनेत्य॑ग्नि - स्तो॒मेन॑ । वै । प्र॒जाप॑ति॒रिति॑ प्र॒जा - प॒तिः॒ । प्र॒जा इति॑ प्र-जाः । अ॒सृ॒ज॒त॒ । ताः । अ॒ग्नि॒ष्टो॒मेनेत्य॑ग्नि-स्तो॒मेन॑ । ए॒व । परीति॑ । अ॒गृ॒ह्णा॒त् । तासा᳚म् । परि॑गृहीताना॒मिति॒ परि॑ - गृ॒ही॒ता॒ना॒म् । अ॒श्व॒त॒रः । अतीति॑ । अ॒प्र॒व॒त॒ । तस्य॑ । अ॒नु॒हायेत्य॑नु - हाय॑ । रेतः॑ । एति॑ । अ॒द॒त्त॒ । तत् । ग॒र्द॒भे । नीति॑ । अ॒मा॒र्ट् । तस्मा᳚त् । ग॒र्द॒भः । द्वि॒रेता॒ इति॑ द्वि - रेताः᳚ । अथो॒ इति॑ । आ॒हुः॒ । वड॑बायाम् । नीति॑ । अ॒मा॒र्ट्॒ । इति॑ । तस्मा᳚त् । वड॑बा । द्वि॒रेता॒ इति॑ द्वि - रेताः᳚ । अथो॒ इति॑ । आ॒हुः॒ । ओष॑धीषु ।  \newline


\textbf{Krama Paata} \newline

यम् काम᳚म् । काम॑म् का॒मय॑ते । का॒मय॑ते॒ तम् । तमे॒तेन॑ । ए॒तेना॒भि । अ॒भ्य॑श्ञुते । अ॒श्ञु॒ते॒ सर्व᳚म् । सर्वꣳ॒॒ हि । ह्यस्थू॑रिणा । अस्थू॑रिणाऽभ्यश्ञु॒ते । अ॒भ्य॒श्ञु॒ते᳚ऽग्निष्टो॒मेन॑ । अ॒भ्य॒श्ञु॒त इत्य॑भि - अ॒श्ञु॒ते । अ॒ग्नि॒ष्टो॒मेन॒ वै । अ॒ग्नि॒ष्टो॒मेनेत्य॑ग्नि - स्तो॒मेन॑ । वै प्र॒जाप॑तिः । प्र॒जाप॑तिः प्र॒जाः । प्र॒जाप॑ति॒रिति॑ प्र॒जा - प॒तिः॒ । प्र॒जा अ॑सृजत । प्र॒जा इति॑ प्र - जाः । अ॒सृ॒ज॒त॒ ताः । ता अ॑ग्निष्टो॒मेन॑ । अ॒ग्नि॒ष्टो॒मेनै॒व । अ॒ग्नि॒ष्टो॒मेनेत्य॑ग्नि - स्तो॒मेन॑ । ए॒व परि॑ । पर्य॑गृह्णात् । अ॒गृ॒ह्णा॒त् तासा᳚म् । तासा॒म् परि॑गृहीतानाम् । परि॑गृहीतानामश्वत॒रः । परि॑गृहीताना॒मिति॒ परि॑ - गृ॒ही॒ता॒ना॒म् । अ॒श्व॒त॒रोऽति॑ । अत्य॑प्रवत । अ॒प्र॒व॒त॒ तस्य॑ । तस्या॑नु॒हाय॑ । अ॒नु॒हाय॒ रेतः॑ । अ॒नु॒हायेत्य॑नु - हाय॑ । रेत॒ आ । आऽद॑त्त । अ॒द॒त्त॒ तत् । तद् ग॑र्द॒भे । ग॒र्द॒भे नि । न्य॑मार्ट् । अ॒मा॒र्ट् तस्मा᳚त् । तस्मा᳚द् गर्द॒भः । ग॒र्द॒भो द्वि॒रेताः᳚ । द्वि॒रेता॒ अथो᳚ । द्वि॒रेता॒ इति॑ द्वि - रेताः᳚ । अथो॑ आहुः । अथो॒ इत्यथो᳚ । आ॒हु॒र् वड॑बायाम् । वड॑बाया॒म् नि । न्य॑मार्ट् । अ॒मा॒र्डिति॑ । इति॒ तस्मा᳚त् । तस्मा॒द् वड॑बा । वड॑बा द्वि॒रेताः᳚ । द्वि॒रेता॒ अथो᳚ । द्वि॒रेता॒ इति॑ द्वि - रेताः᳚ । अथो॑ आहुः । अथो॒ इत्यथो᳚ । आ॒हु॒रोष॑धीषु । ओष॑धीषु॒ नि \newline

\textbf{Jatai Paata} \newline

1. यम् काम॒म् कामं॒ ॅयं ॅयम् काम᳚म् । \newline
2. काम॑म् का॒मय॑ते का॒मय॑ते॒ काम॒म् काम॑म् का॒मय॑ते । \newline
3. का॒मय॑ते॒ तम् तम् का॒मय॑ते का॒मय॑ते॒ तम् । \newline
4. त मे॒ते नै॒तेन॒ तम् त मे॒तेन॑ । \newline
5. ए॒ते ना॒भ्या᳚(1॒)भ्ये॑ते नै॒तेना॒भि । \newline
6. अ॒भ्य॑श्ञुते ऽश्ञुते॒ ऽभ्या᳚(1॒)भ्य॑श्ञुते । \newline
7. अ॒श्ञु॒ते॒ सर्वꣳ॒॒ सर्व॑ मश्ञुते ऽश्ञुते॒ सर्व᳚म् । \newline
8. सर्वꣳ॒॒ हि हि सर्वꣳ॒॒ सर्वꣳ॒॒ हि । \newline
9. ह्यस्थू॑रि॒णा ऽस्थू॑रिणा॒ हि ह्यस्थू॑रिणा । \newline
10. अस्थू॑रिणा ऽभ्यश्ञु॒ते᳚ ऽभ्यश्ञु॒ते ऽस्थू॑रि॒णा ऽस्थू॑रिणा ऽभ्यश्ञु॒ते । \newline
11. अ॒भ्य॒श्ञु॒ते᳚ ऽग्निष्टो॒मेना᳚ ग्निष्टो॒मेना᳚ भ्यश्ञु॒ते᳚ ऽभ्यश्ञु॒ते᳚ ऽग्निष्टो॒मेन॑ । \newline
12. अ॒भ्य॒श्ञु॒त इत्य॑भि - अ॒श्ञु॒ते । \newline
13. अ॒ग्नि॒ष्टो॒मेन॒ वै वा अ॑ग्निष्टो॒मेना᳚ ग्निष्टो॒मेन॒ वै । \newline
14. अ॒ग्नि॒ष्टो॒मेनेत्य॑ग्नि - स्तो॒मेन॑ । \newline
15. वै प्र॒जाप॑तिः प्र॒जाप॑ति॒र् वै वै प्र॒जाप॑तिः । \newline
16. प्र॒जाप॑तिः प्र॒जाः प्र॒जाः प्र॒जाप॑तिः प्र॒जाप॑तिः प्र॒जाः । \newline
17. प्र॒जाप॑ति॒रिति॑ प्र॒जा - प॒तिः॒ । \newline
18. प्र॒जा अ॑सृजता सृजत प्र॒जाः प्र॒जा अ॑सृजत । \newline
19. प्र॒जा इति॑ प्र - जाः । \newline
20. अ॒सृ॒ज॒त॒ ता स्ता अ॑सृजता सृजत॒ ताः । \newline
21. ता अ॑ग्निष्टो॒मेना᳚ ग्निष्टो॒मेन॒ ता स्ता अ॑ग्निष्टो॒मेन॑ । \newline
22. अ॒ग्नि॒ष्टो॒मे नै॒वै वाग्नि॑ष्टो॒मेना᳚ ग्निष्टो॒मेनै॒व । \newline
23. अ॒ग्नि॒ष्टो॒मेनेत्य॑ग्नि - स्तो॒मेन॑ । \newline
24. ए॒व परि॒ पर्ये॒ वैव परि॑ । \newline
25. पर्य॑ गृह्णा दगृह्णा॒त् परि॒ पर्य॑ गृह्णात् । \newline
26. अ॒गृ॒ह्णा॒त् तासा॒म् तासा॑ मगृह्णा दगृह्णा॒त् तासा᳚म् । \newline
27. तासा॒म् परि॑गृहीताना॒म् परि॑गृहीताना॒म् तासा॒म् तासा॒म् परि॑गृहीतानाम् । \newline
28. परि॑गृहीताना मश्वत॒रो᳚ ऽश्वत॒रः परि॑गृहीताना॒म् परि॑गृहीताना मश्वत॒रः । \newline
29. परि॑गृहीताना॒मिति॒ परि॑ - गृ॒ही॒ता॒ना॒म् । \newline
30. अ॒श्व॒त॒रो ऽत्य त्य॑श्वत॒रो᳚ ऽश्वत॒रो ऽति॑ । \newline
31. अत्य॑प्रवता प्रव॒ता त्य त्य॑प्रवत । \newline
32. अ॒प्र॒व॒त॒ तस्य॒ तस्या᳚ प्रवता प्रवत॒ तस्य॑ । \newline
33. तस्या॑ नु॒हाया॑ नु॒हाय॒ तस्य॒ तस्या॑ नु॒हाय॑ । \newline
34. अ॒नु॒हाय॒ रेतो॒ रेतो॑ ऽनु॒हाया॑ नु॒हाय॒ रेतः॑ । \newline
35. अ॒नु॒हायेत्य॑नु - हाय॑ । \newline
36. रेत॒ आ रेतो॒ रेत॒ आ । \newline
37. आ ऽद॑त्ता द॒त्ता ऽद॑त्त । \newline
38. अ॒द॒त्त॒ तत् तद॑दत्ता दत्त॒ तत् । \newline
39. तद् ग॑र्द॒भे ग॑र्द॒भे तत् तद् ग॑र्द॒भे । \newline
40. ग॒र्द॒भे नि नि ग॑र्द॒भे ग॑र्द॒भे नि । \newline
41. न्य॑मार् डमा॒र्ण् णि न्य॑मार्ट् । \newline
42. अ॒मा॒र्ट् तस्मा॒त् तस्मा॑ दमार् डमा॒र्ट् तस्मा᳚त् । \newline
43. तस्मा᳚द् गर्द॒भो ग॑र्द॒भ स्तस्मा॒त् तस्मा᳚द् गर्द॒भः । \newline
44. ग॒र्द॒भो द्वि॒रेता᳚ द्वि॒रेता॑ गर्द॒भो ग॑र्द॒भो द्वि॒रेताः᳚ । \newline
45. द्वि॒रेता॒ अथो॒ अथो᳚ द्वि॒रेता᳚ द्वि॒रेता॒ अथो᳚ । \newline
46. द्वि॒रेता॒ इति॑ द्वि - रेताः᳚ । \newline
47. अथो॑ आहु राहु॒ रथो॒ अथो॑ आहुः । \newline
48. अथो॒ इत्यथो᳚ । \newline
49. आ॒हु॒र् वड॑बायां॒ ॅवड॑बाया माहु राहु॒र् वड॑बायाम् । \newline
50. वड॑बाया॒न् नि नि वड॑बायां॒ ॅवड॑बाया॒न् नि । \newline
51. न्य॑मार् डमा॒र्ण् णि न्य॑मार्ट् । \newline
52. अ॒मा॒र् डिती त्य॑मार् डमा॒र् डिति॑ । \newline
53. इति॒ तस्मा॒त् तस्मा॒ दितीति॒ तस्मा᳚त् । \newline
54. तस्मा॒द् वड॑बा॒ वड॑बा॒ तस्मा॒त् तस्मा॒द् वड॑बा । \newline
55. वड॑बा द्वि॒रेता᳚ द्वि॒रेता॒ वड॑बा॒ वड॑बा द्वि॒रेताः᳚ । \newline
56. द्वि॒रेता॒ अथो॒ अथो᳚ द्वि॒रेता᳚ द्वि॒रेता॒ अथो᳚ । \newline
57. द्वि॒रेता॒ इति॑ द्वि - रेताः᳚ । \newline
58. अथो॑ आहु राहु॒ रथो॒ अथो॑ आहुः । \newline
59. अथो॒ इत्यथो᳚ । \newline
60. आ॒हु॒ रोष॑धी॒ ष्वोष॑धी ष्वाहु राहु॒ रोष॑धीषु । \newline
61. ओष॑धीषु॒ नि न्योष॑धी॒ ष्वोष॑धीषु॒ नि । \newline

\textbf{Ghana Paata } \newline

1. यम् काम॒म् कामं॒ ॅयं ॅयम् काम॑म् का॒मय॑ते का॒मय॑ते॒ कामं॒ ॅयं ॅयम् काम॑म् का॒मय॑ते । \newline
2. काम॑म् का॒मय॑ते का॒मय॑ते॒ काम॒म् काम॑म् का॒मय॑ते॒ तम् तम् का॒मय॑ते॒ काम॒म् काम॑म् का॒मय॑ते॒ तम् । \newline
3. का॒मय॑ते॒ तम् तम् का॒मय॑ते का॒मय॑ते॒ त मे॒ते नै॒तेन॒ तम् का॒मय॑ते का॒मय॑ते॒ त मे॒तेन॑ । \newline
4. त मे॒ते नै॒तेन॒ तम् त मे॒ते ना॒भ्या᳚(1॒)भ्ये॑तेन॒ तम् त मे॒तेना॒भि । \newline
5. ए॒तेना॒भ्या᳚(1॒)भ्ये॑ते नै॒ते ना॒भ्य॑श्ञुते ऽश्ञुते॒ ऽभ्ये॑ते नै॒ते ना॒भ्य॑श्ञुते । \newline
6. अ॒भ्य॑श्ञुते ऽश्ञुते॒ ऽभ्या᳚(1॒)भ्य॑श्ञुते॒ सर्वꣳ॒॒ सर्व॑ मश्ञुते॒ ऽभ्या᳚(1॒)भ्य॑श्ञुते॒ सर्व᳚म् । \newline
7. अ॒श्ञु॒ते॒ सर्वꣳ॒॒ सर्व॑ मश्ञुते ऽश्ञुते॒ सर्वꣳ॒॒ हि हि सर्व॑ मश्ञुते ऽश्ञुते॒ सर्वꣳ॒॒ हि । \newline
8. सर्वꣳ॒॒ हि हि सर्वꣳ॒॒ सर्वꣳ॒॒ ह्यस्थू॑रि॒णा ऽस्थू॑रिणा॒ हि सर्वꣳ॒॒ सर्वꣳ॒॒ ह्यस्थू॑रिणा । \newline
9. ह्यस्थू॑रि॒णा ऽस्थू॑रिणा॒ हि ह्यस्थू॑रिणा ऽभ्यश्ञु॒ते᳚ ऽभ्यश्ञु॒ते ऽस्थू॑रिणा॒ हि ह्यस्थू॑रिणा ऽभ्यश्ञु॒ते । \newline
10. अस्थू॑रिणा ऽभ्यश्ञु॒ते᳚ ऽभ्यश्ञु॒ते ऽस्थू॑रि॒णा ऽस्थू॑रिणा ऽभ्यश्ञु॒ते᳚ ऽग्निष्टो॒मेना᳚ ग्निष्टो॒मेना᳚ भ्यश्ञु॒ते ऽस्थू॑रि॒णा ऽस्थू॑रिणा ऽभ्यश्ञु॒ते᳚ ऽग्निष्टो॒मेन॑ । \newline
11. अ॒भ्य॒श्ञु॒ते᳚ ऽग्निष्टो॒मेना᳚ ग्निष्टो॒मेना᳚ भ्यश्ञु॒ते᳚ ऽभ्यश्ञु॒ते᳚ ऽग्निष्टो॒मेन॒ वै वा अ॑ग्निष्टो॒मेना᳚ भ्यश्ञु॒ते᳚ ऽभ्यश्ञु॒ते᳚ ऽग्निष्टो॒मेन॒ वै । \newline
12. अ॒भ्य॒श्ञु॒त इत्य॑भि - अ॒श्ञु॒ते । \newline
13. अ॒ग्नि॒ष्टो॒मेन॒ वै वा अ॑ग्निष्टो॒मेना᳚ ग्निष्टो॒मेन॒ वै प्र॒जाप॑तिः प्र॒जाप॑ति॒र् वा अ॑ग्निष्टो॒मेना᳚ ग्निष्टो॒मेन॒ वै प्र॒जाप॑तिः । \newline
14. अ॒ग्नि॒ष्टो॒मेनेत्य॑ग्नि - स्तो॒मेन॑ । \newline
15. वै प्र॒जाप॑तिः प्र॒जाप॑ति॒र् वै वै प्र॒जाप॑तिः प्र॒जाः प्र॒जाः प्र॒जाप॑ति॒र् वै वै प्र॒जाप॑तिः प्र॒जाः । \newline
16. प्र॒जाप॑तिः प्र॒जाः प्र॒जाः प्र॒जाप॑तिः प्र॒जाप॑तिः प्र॒जा अ॑सृजता सृजत प्र॒जाः प्र॒जाप॑तिः प्र॒जाप॑तिः प्र॒जा अ॑सृजत । \newline
17. प्र॒जाप॑ति॒रिति॑ प्र॒जा - प॒तिः॒ । \newline
18. प्र॒जा अ॑सृजता सृजत प्र॒जाः प्र॒जा अ॑सृजत॒ ता स्ता अ॑सृजत प्र॒जाः प्र॒जा अ॑सृजत॒ ताः । \newline
19. प्र॒जा इति॑ प्र - जाः । \newline
20. अ॒सृ॒ज॒त॒ ता स्ता अ॑सृजता सृजत॒ ता अ॑ग्निष्टो॒मेना᳚ ग्निष्टो॒मेन॒ ता अ॑सृजता सृजत॒ ता अ॑ग्निष्टो॒मेन॑ । \newline
21. ता अ॑ग्निष्टो॒मेना᳚ ग्निष्टो॒मेन॒ ता स्ता अ॑ग्निष्टो॒मे नै॒वैवा ग्नि॑ष्टो॒मेन॒ ता स्ता अ॑ग्निष्टो॒मे नै॒व । \newline
22. अ॒ग्नि॒ष्टो॒मे नै॒वैवा ग्नि॑ष्टो॒मेना᳚ ग्निष्टो॒मे नै॒व परि॒ पर्ये॒वा ग्नि॑ष्टो॒मेना᳚ ग्निष्टो॒मे नै॒व परि॑ । \newline
23. अ॒ग्नि॒ष्टो॒मेनेत्य॑ग्नि - स्तो॒मेन॑ । \newline
24. ए॒व परि॒ पर्ये॒ वैव पर्य॑गृह्णा दगृह्णा॒त् पर्ये॒वैव पर्य॑गृह्णात् । \newline
25. पर्य॑गृह्णा दगृह्णा॒त् परि॒ पर्य॑गृह्णा॒त् तासा॒म् तासा॑ मगृह्णा॒त् परि॒ पर्य॑गृह्णा॒त् तासा᳚म् । \newline
26. अ॒गृ॒ह्णा॒त् तासा॒म् तासा॑ मगृह्णा दगृह्णा॒त् तासा॒म् परि॑गृहीताना॒म् परि॑गृहीताना॒म् तासा॑ मगृह्णा दगृह्णा॒त् तासा॒म् परि॑गृहीतानाम् । \newline
27. तासा॒म् परि॑गृहीताना॒म् परि॑गृहीताना॒म् तासा॒म् तासा॒म् परि॑गृहीताना मश्वत॒रो᳚ ऽश्वत॒रः परि॑गृहीताना॒म् तासा॒म् तासा॒म् परि॑गृहीताना मश्वत॒रः । \newline
28. परि॑गृहीताना मश्वत॒रो᳚ ऽश्वत॒रः परि॑गृहीताना॒म् परि॑गृहीताना मश्वत॒रो ऽत्य त्य॑श्वत॒रः परि॑गृहीताना॒म् परि॑गृहीताना मश्वत॒रो ऽति॑ । \newline
29. परि॑गृहीताना॒मिति॒ परि॑ - गृ॒ही॒ता॒ना॒म् । \newline
30. अ॒श्व॒त॒रो ऽत्य त्य॑श्वत॒रो᳚ ऽश्वत॒रो ऽत्य॑प्रवता प्रव॒ता त्य॑श्वत॒रो᳚ ऽश्वत॒रो ऽत्य॑प्रवत । \newline
31. अत्य॑प्रवता प्रव॒ता त्य त्य॑प्रवत॒ तस्य॒ तस्या᳚ प्रव॒ता त्य त्य॑प्रवत॒ तस्य॑ । \newline
32. अ॒प्र॒व॒त॒ तस्य॒ तस्या᳚ प्रवता प्रवत॒ तस्या॑ नु॒हाया॑ नु॒हाय॒ तस्या᳚ प्रवता प्रवत॒ तस्या॑ नु॒हाय॑ । \newline
33. तस्या॑ नु॒हाया॑ नु॒हाय॒ तस्य॒ तस्या॑ नु॒हाय॒ रेतो॒ रेतो॑ ऽनु॒हाय॒ तस्य॒ तस्या॑ नु॒हाय॒ रेतः॑ । \newline
34. अ॒नु॒हाय॒ रेतो॒ रेतो॑ ऽनु॒हाया॑ नु॒हाय॒ रेत॒ आ रेतो॑ ऽनु॒हाया॑ नु॒हाय॒ रेत॒ आ । \newline
35. अ॒नु॒हायेत्य॑नु - हाय॑ । \newline
36. रेत॒ आ रेतो॒ रेत॒ आ ऽद॑त्ता द॒त्ता रेतो॒ रेत॒ आ ऽद॑त्त । \newline
37. आ ऽद॑त्ता द॒त्ता ऽद॑त्त॒ तत् तद॑द॒त्ता ऽद॑त्त॒ तत् । \newline
38. अ॒द॒त्त॒ तत् तद॑दत्ता दत्त॒ तद् ग॑र्द॒भे ग॑र्द॒भे तद॑दत्ता दत्त॒ तद् ग॑र्द॒भे । \newline
39. तद् ग॑र्द॒भे ग॑र्द॒भे तत् तद् ग॑र्द॒भे नि नि ग॑र्द॒भे तत् तद् ग॑र्द॒भे नि । \newline
40. ग॒र्द॒भे नि नि ग॑र्द॒भे ग॑र्द॒भे न्य॑मार् डमा॒र्ण् णिग॑र्द॒भे ग॑र्द॒भे न्य॑मार्ट् । \newline
41. न्य॑मार् डमा॒र्ण् णि न्य॑मा॒र्ट् तस्मा॒त् तस्मा॑ दमा॒र्ण् णि न्य॑मा॒र्ट् तस्मा᳚त् । \newline
42. अ॒मा॒र्ट् तस्मा॒त् तस्मा॑ दमार् डमा॒र्ट् तस्मा᳚द् गर्द॒भो ग॑र्द॒भ स्तस्मा॑ दमार् डमा॒र्ट् तस्मा᳚द् गर्द॒भः । \newline
43. तस्मा᳚द् गर्द॒भो ग॑र्द॒भ स्तस्मा॒त् तस्मा᳚द् गर्द॒भो द्वि॒रेता᳚ द्वि॒रेता॑ गर्द॒भ स्तस्मा॒त् तस्मा᳚द् गर्द॒भो द्वि॒रेताः᳚ । \newline
44. ग॒र्द॒भो द्वि॒रेता᳚ द्वि॒रेता॑ गर्द॒भो ग॑र्द॒भो द्वि॒रेता॒ अथो॒ अथो᳚ द्वि॒रेता॑ गर्द॒भो ग॑र्द॒भो द्वि॒रेता॒ अथो᳚ । \newline
45. द्वि॒रेता॒ अथो॒ अथो᳚ द्वि॒रेता᳚ द्वि॒रेता॒ अथो॑ आहु राहु॒ रथो᳚ द्वि॒रेता᳚ द्वि॒रेता॒ अथो॑ आहुः । \newline
46. द्वि॒रेता॒ इति॑ द्वि - रेताः᳚ । \newline
47. अथो॑ आहु राहु॒ रथो॒ अथो॑ आहु॒र् वड॑बायां॒ ॅवड॑बाया माहु॒ रथो॒ अथो॑ आहु॒र् वड॑बायाम् । \newline
48. अथो॒ इत्यथो᳚ । \newline
49. आ॒हु॒र् वड॑बायां॒ ॅवड॑बाया माहु राहु॒र् वड॑बाया॒न् नि नि वड॑बाया माहु राहु॒र् वड॑बाया॒न् नि । \newline
50. वड॑बाया॒न् नि नि वड॑बायां॒ ॅवड॑बाया॒न् न्य॑मार् डमा॒र्ण् णि वड॑बायां॒ ॅवड॑बाया॒न् न्य॑मार्ट् । \newline
51. न्य॑मार् डमा॒र्ण् णि न्य॑मा॒र् डितीत्य॑मा॒र्ण् णि न्य॑मा॒र्डिति॑ । \newline
52. अ॒मा॒र् डितीत्य॑मार् डमा॒र्डिति॒ तस्मा॒त् तस्मा॒ दित्य॑मार् डमा॒र्डिति॒ तस्मा᳚त् । \newline
53. इति॒ तस्मा॒त् तस्मा॒ दितीति॒ तस्मा॒द् वड॑बा॒ वड॑बा॒ तस्मा॒ दितीति॒ तस्मा॒द् वड॑बा । \newline
54. तस्मा॒द् वड॑बा॒ वड॑बा॒ तस्मा॒त् तस्मा॒द् वड॑बा द्वि॒रेता᳚ द्वि॒रेता॒ वड॑बा॒ तस्मा॒त् तस्मा॒द् वड॑बा द्वि॒रेताः᳚ । \newline
55. वड॑बा द्वि॒रेता᳚ द्वि॒रेता॒ वड॑बा॒ वड॑बा द्वि॒रेता॒ अथो॒ अथो᳚ द्वि॒रेता॒ वड॑बा॒ वड॑बा द्वि॒रेता॒ अथो᳚ । \newline
56. द्वि॒रेता॒ अथो॒ अथो᳚ द्वि॒रेता᳚ द्वि॒रेता॒ अथो॑ आहु राहु॒ रथो᳚ द्वि॒रेता᳚ द्वि॒रेता॒ अथो॑ आहुः । \newline
57. द्वि॒रेता॒ इति॑ द्वि - रेताः᳚ । \newline
58. अथो॑ आहु राहु॒ रथो॒ अथो॑ आहु॒ रोष॑धी॒ ष्वोष॑धी ष्वाहु॒ रथो॒ अथो॑ आहु॒ रोष॑धीषु । \newline
59. अथो॒ इत्यथो᳚ । \newline
60. आ॒हु॒ रोष॑धी॒ ष्वोष॑धी ष्वाहु राहु॒ रोष॑धीषु॒ नि न्योष॑धी ष्वाहु राहु॒ रोष॑धीषु॒ नि । \newline
61. ओष॑धीषु॒ नि न्योष॑धी॒ ष्वोष॑धीषु॒ न्य॑मार् डमा॒र्ण् ण्योष॑धी॒ ष्वोष॑धीषु॒ न्य॑मार्ट् । \newline
\pagebreak
\markright{ TS 7.1.1.3  \hfill https://www.vedavms.in \hfill}

\section{ TS 7.1.1.3 }

\textbf{TS 7.1.1.3 } \newline
\textbf{Samhita Paata} \newline

न्य॑मा॒र्डिति॒ तस्मा॒दोष॑ध॒यो ऽन॑भ्यक्ता रेभ॒न्त्यथो॑ आहुः प्र॒जासु॒ न्य॑मा॒र्डिति॒ तस्मा᳚द्-य॒मौ जा॑येते॒ तस्मा॑दश्वत॒रो न प्र जा॑यत॒ आत्त॑रेता॒ हि तस्मा᳚द् ब॒र्॒.हिष्यन॑वक्लृप्तः सर्ववेद॒से वा॑ स॒हस्रे॒ वाऽव॑ क्लृ॒प्तोऽति॒ ह्यप्र॑वत॒ य ए॒वं ॅवि॒द्वान॑ग्निष्टो॒मेन॒ यज॑ते॒ प्राजा॑ताः प्र॒जा ज॒नय॑ति॒ परि॒ प्रजा॑ता गृह्णाति॒ तस्मा॑दाहुर्ज्येष्ठय॒ज्ञ् इति॑ - [  ] \newline

\textbf{Pada Paata} \newline

नीति॑ । अ॒मा॒र्ट॒ । इति॑ । तस्मा᳚त् । ओष॑धयः । अन॑भ्यक्ता॒ इत्यन॑भि - अ॒क्ताः॒ । रे॒भ॒न्ति॒ । अथो॒ इति॑ । आ॒हुः॒ । प्र॒जास्विति॑ प्र - जासु॑ । नीति॑ । अ॒मा॒र्ट्॒ । इति॑ । तस्मा᳚त् । य॒मौ । जा॒ये॒ते॒ इति॑ । तस्मा᳚त् । अ॒श्व॒त॒रः । न । प्रेति॑ । जा॒य॒ते॒ । आत्त॑रेता॒ इत्यात्त॑ - रे॒ताः॒ । हि । तस्मा᳚त् । ब॒र्॒.हिषि॑ । अन॑वक्लृप्त॒ इत्यन॑व - क्लृ॒प्तः॒ । स॒र्व॒वे॒द॒स इति॑ सर्व-वे॒द॒से । वा॒ । स॒हस्रे᳚ । वा॒ । अव॑क्लृप्त॒ इत्यव॑-क्लृ॒प्तः॒ । अतीति॑ । हि । अप्र॑वत । यः । ए॒वम् । वि॒द्वान् । अ॒ग्नि॒ष्टो॒मेनेत्य॑ग्नि- स्तो॒मेन॑ । यज॑ते । प्रेति॑ । अजा॑ताः । प्र॒जा इति॑ प्र-जाः । ज॒नय॑ति । परीति॑ । प्रजा॑ता॒ इति॒ प्र - जा॒ताः॒ । गृ॒ह्णा॒ति॒ । तस्मा᳚त् । आ॒हुः॒ । ज्ये॒ष्ठ॒य॒ज्ञ् इति॑ ज्येष्ठ - य॒ज्ञ्ः । इति॑ ।  \newline


\textbf{Krama Paata} \newline

न्य॑मार्ट् । अ॒मा॒र्डिति॑ । इति॒ तस्मा᳚त् । तस्मा॒दोष॑धयः । ओष॑ध॒योऽन॑भ्यक्ताः । अन॑भ्यक्ता रेभन्ति । अन॑भ्यक्ता॒ इत्यन॑भि - अ॒क्ताः॒ । रे॒भ॒न्त्यथो᳚ । अथो॑ आहुः । अथो॒ इत्यथो᳚ । आ॒हुः॒ प्र॒जासु॑ । प्र॒जासु॒ नि । प्र॒जास्विति॑ प्र - जासु॑ । न्य॑मार्ट् । अ॒मा॒र्डिति॑ । इति॒ तस्मा᳚त् । तस्मा᳚द् य॒मौ । य॒मौ जा॑येते । जा॒ये॒ते॒ तस्मा᳚त् । जा॒ये॒ते॒ इति॑ जायेते । तस्मा॑दश्वत॒रः । अ॒श्व॒त॒रो न । न प्र । प्र जा॑यते । जा॒य॒त॒ आत्त॑रेताः । आत्त॑रेता॒ हि । आत्त॑रेता॒ इत्यात्त॑ - रे॒ताः॒ । हि तस्मा᳚त् । तस्मा᳚द् ब॒र्.॒हिषि॑ । ब॒र्.॒हिष्यन॑वक्लृप्तः । अन॑वक्लृप्तः सर्ववेद॒से । अन॑वक्लृप्त॒ इत्यन॑व - क्लृ॒प्तः॒ । स॒र्व॒वे॒द॒से वा᳚ । स॒र्व॒वे॒द॒स इति॑ सर्व - वे॒द॒से । वा॒ स॒हस्रे᳚ । स॒हस्रे॑ वा । वाऽव॑क्लृप्तः । अव॑क्लृ॒प्तोऽति॑ । अव॑क्लृप्त॒ इत्यव॑ - क्लृ॒प्तः॒ । अति॒ हि । ह्यप्र॑वत । अप्र॑वत॒ यः । य ए॒वम् । ए॒वम् ॅवि॒द्वान् । वि॒द्वान॑ग्निष्टो॒मेन॑ । अ॒ग्नि॒ष्टो॒मेन॒ यज॑ते । अ॒ग्नि॒ष्टो॒मेनेत्य॑ग्नि - स्तो॒मेन॑ । यज॑ते॒ प्र । प्राजा॑ताः । अजा॑ताः प्र॒जाः । प्र॒जा ज॒नय॑ति । प्र॒जा इति॑ प्र - जाः । ज॒नय॑ति॒ परि॑ । परि॒ प्रजा॑ताः । प्रजा॑ता गृह्णाति । प्रजा॑ता॒ इति॒ प्र - जा॒ताः॒ । गृ॒ह्णा॒ति॒ तस्मा᳚त् । तस्मा॑दाहुः । आ॒हु॒र् ज्ये॒ष्ठ॒य॒ज्ञ्ः । ज्ये॒ष्ठ॒य॒ज्ञ् इति॑ । ज्ये॒ष्ठ॒य॒ज्ञ् इति॑ ज्येष्ठ - य॒ज्ञ्ः । इति॑ प्र॒जाप॑तिः \newline

\textbf{Jatai Paata} \newline

1. न्य॑मार् डमा॒र्ण् णि न्य॑मार्ट् । \newline
2. अ॒मा॒र् डिती त्य॑मार् डमा॒र् डिति॑ । \newline
3. इति॒ तस्मा॒त् तस्मा॒ दितीति॒ तस्मा᳚त् । \newline
4. तस्मा॒ दोष॑धय॒ ओष॑धय॒ स्तस्मा॒त् तस्मा॒ दोष॑धयः । \newline
5. ओष॑ध॒यो ऽन॑भ्यक्ता॒ अन॑भ्यक्ता॒ ओष॑धय॒ ओष॑ध॒यो ऽन॑भ्यक्ताः । \newline
6. अन॑भ्यक्ता रेभन्ति रेभ॒ न्त्यन॑भ्यक्ता॒ अन॑भ्यक्ता रेभन्ति । \newline
7. अन॑भ्यक्ता॒ इत्यन॑भि - अ॒क्ताः॒ । \newline
8. रे॒भ॒ न्त्यथो॒ अथो॑ रेभन्ति रेभ॒ न्त्यथो᳚ । \newline
9. अथो॑ आहु राहु॒ रथो॒ अथो॑ आहुः । \newline
10. अथो॒ इत्यथो᳚ । \newline
11. आ॒हुः॒ प्र॒जासु॑ प्र॒जा स्वा॑हु राहुः प्र॒जासु॑ । \newline
12. प्र॒जासु॒ नि नि प्र॒जासु॑ प्र॒जासु॒ नि । \newline
13. प्र॒जास्विति॑ प्र - जासु॑ । \newline
14. न्य॑मार् डमा॒र्ण् णि न्य॑मार्ट् । \newline
15. अ॒मा॒र् डिती त्य॑मार् डमा॒र् डिति॑ । \newline
16. इति॒ तस्मा॒त् तस्मा॒ दितीति॒ तस्मा᳚त् । \newline
17. तस्मा᳚द् य॒मौ य॒मौ तस्मा॒त् तस्मा᳚द् य॒मौ । \newline
18. य॒मौ जा॑येते जायेते य॒मौ य॒मौ जा॑येते । \newline
19. जा॒ये॒ते॒ तस्मा॒त् तस्मा᳚ज् जायेते जायेते॒ तस्मा᳚त् । \newline
20. जा॒ये॒ते॒ इति॑ जायेते । \newline
21. तस्मा॑ दश्वत॒रो᳚ ऽश्वत॒र स्तस्मा॒त् तस्मा॑ दश्वत॒रः । \newline
22. अ॒श्व॒त॒रो न नाश्व॑त॒रो᳚ ऽश्वत॒रो न । \newline
23. न प्र प्र ण न प्र । \newline
24. प्र जा॑यते जायते॒ प्र प्र जा॑यते । \newline
25. जा॒य॒त॒ आत्त॑रेता॒ आत्त॑रेता जायते जायत॒ आत्त॑रेताः । \newline
26. आत्त॑रेता॒ हि ह्यात्त॑रेता॒ आत्त॑रेता॒ हि । \newline
27. आत्त॑रेता॒ इत्यात्त॑ - रे॒ताः॒ । \newline
28. हि तस्मा॒त् तस्मा॒द्धि हि तस्मा᳚त् । \newline
29. तस्मा᳚द् ब॒र्॒.हिषि॑ ब॒र्॒.हिषि॒ तस्मा॒त् तस्मा᳚द् ब॒र्॒.हिषि॑ । \newline
30. ब॒र्॒.हि ष्यन॑वक्लृ॒प्तो ऽन॑वक्लृप्तो ब॒र्॒.हिषि॑ ब॒र्॒.हि ष्यन॑वक्लृप्तः । \newline
31. अन॑वक्लृप्तः सर्ववेद॒से स॑र्ववेद॒से ऽन॑वक्लृ॒प्तो ऽन॑वक्लृप्तः सर्ववेद॒से । \newline
32. अन॑वक्लृप्त॒ इत्यन॑व - क्लृ॒प्तः॒ । \newline
33. स॒र्व॒वे॒द॒से वा॑ वा सर्ववेद॒से स॑र्ववेद॒से वा᳚ । \newline
34. स॒र्व॒वे॒द॒स इति॑ सर्व - वे॒द॒से । \newline
35. वा॒ स॒हस्रे॑ स॒हस्रे॑ वा वा स॒हस्रे᳚ । \newline
36. स॒हस्रे॑ वा वा स॒हस्रे॑ स॒हस्रे॑ वा । \newline
37. वा ऽव॑क्लृ॒प्तो ऽव॑क्लृप्तो वा॒ वा ऽव॑क्लृप्तः । \newline
38. अव॑क्लृ॒प्तो ऽत्य त्यव॑क्लृ॒प्तो ऽव॑क्लृ॒प्तो ऽति॑ । \newline
39. अव॑क्लृप्त॒ इत्यव॑ - क्लृ॒प्तः॒ । \newline
40. अति॒ हि ह्यत्यति॒ हि । \newline
41. ह्यप्र॑व॒ता प्र॑वत॒ हि ह्यप्र॑वत । \newline
42. अप्र॑वत॒ यो यो ऽप्र॑व॒ता प्र॑वत॒ यः । \newline
43. य ए॒व मे॒वं ॅयो य ए॒वम् । \newline
44. ए॒वं ॅवि॒द्वान्. वि॒द्वा ने॒व मे॒वं ॅवि॒द्वान् । \newline
45. वि॒द्वा न॑ग्निष्टो॒मेना᳚ ग्निष्टो॒मेन॑ वि॒द्वान्. वि॒द्वा न॑ग्निष्टो॒मेन॑ । \newline
46. अ॒ग्नि॒ष्टो॒मेन॒ यज॑ते॒ यज॑ते ऽग्निष्टो॒मेना᳚ ग्निष्टो॒मेन॒ यज॑ते । \newline
47. अ॒ग्नि॒ष्टो॒मेनेत्य॑ग्नि - स्तो॒मेन॑ । \newline
48. यज॑ते॒ प्र प्र यज॑ते॒ यज॑ते॒ प्र । \newline
49. प्राजा॑ता॒ अजा॑ताः॒ प्र प्राजा॑ताः । \newline
50. अजा॑ताः प्र॒जाः प्र॒जा अजा॑ता॒ अजा॑ताः प्र॒जाः । \newline
51. प्र॒जा ज॒नय॑ति ज॒नय॑ति प्र॒जाः प्र॒जा ज॒नय॑ति । \newline
52. प्र॒जा इति॑ प्र - जाः । \newline
53. ज॒नय॑ति॒ परि॒ परि॑ ज॒नय॑ति ज॒नय॑ति॒ परि॑ । \newline
54. परि॒ प्रजा॑ताः॒ प्रजा॑ताः॒ परि॒ परि॒ प्रजा॑ताः । \newline
55. प्रजा॑ता गृह्णाति गृह्णाति॒ प्रजा॑ताः॒ प्रजा॑ता गृह्णाति । \newline
56. प्रजा॑ता॒ इति॒ प्र - जा॒ताः॒ । \newline
57. गृ॒ह्णा॒ति॒ तस्मा॒त् तस्मा᳚द् गृह्णाति गृह्णाति॒ तस्मा᳚त् । \newline
58. तस्मा॑ दाहु राहु॒ स्तस्मा॒त् तस्मा॑ दाहुः । \newline
59. आ॒हु॒र् ज्ये॒ष्ठ॒य॒ज्ञो ज्ये᳚ष्ठय॒ज्ञ् आ॑हु राहुर् ज्येष्ठय॒ज्ञ्ः । \newline
60. ज्ये॒ष्ठ॒य॒ज्ञ् इतीति॑ ज्येष्ठय॒ज्ञो ज्ये᳚ष्ठय॒ज्ञ् इति॑ । \newline
61. ज्ये॒ष्ठ॒य॒ज्ञ् इति॑ ज्येष्ठ - य॒ज्ञ्ः । \newline
62. इति॑ प्र॒जाप॑तिः प्र॒जाप॑ति॒ रितीति॑ प्र॒जाप॑तिः । \newline

\textbf{Ghana Paata } \newline

1. न्य॑मार् डमा॒र्ण् णि न्य॑मा॒र् डिती त्य॑मा॒र्ण् णि न्य॑मा॒र्डिति॑ । \newline
2. अ॒मा॒र् डिती त्य॑मार् डमा॒र्डिति॒ तस्मा॒त् तस्मा॒ दित्य॑मार् डमा॒र्डिति॒ तस्मा᳚त् । \newline
3. इति॒ तस्मा॒त् तस्मा॒ दितीति॒ तस्मा॒ दोष॑धय॒ ओष॑धय॒ स्तस्मा॒ दितीति॒ तस्मा॒ दोष॑धयः । \newline
4. तस्मा॒ दोष॑धय॒ ओष॑धय॒ स्तस्मा॒त् तस्मा॒ दोष॑ध॒यो ऽन॑भ्यक्ता॒ अन॑भ्यक्ता॒ ओष॑धय॒ स्तस्मा॒त् तस्मा॒ दोष॑ध॒यो ऽन॑भ्यक्ताः । \newline
5. ओष॑ध॒यो ऽन॑भ्यक्ता॒ अन॑भ्यक्ता॒ ओष॑धय॒ ओष॑ध॒यो ऽन॑भ्यक्ता रेभन्ति रेभ॒ न्त्यन॑भ्यक्ता॒ ओष॑धय॒ ओष॑ध॒यो ऽन॑भ्यक्ता रेभन्ति । \newline
6. अन॑भ्यक्ता रेभन्ति रेभ॒ न्त्यन॑भ्यक्ता॒ अन॑भ्यक्ता रेभ॒ न्त्यथो॒ अथो॑ रेभ॒ न्त्यन॑भ्यक्ता॒ अन॑भ्यक्ता रेभ॒
न्त्यथो᳚ । \newline
7. अन॑भ्यक्ता॒ इत्यन॑भि - अ॒क्ताः॒ । \newline
8. रे॒भ॒ न्त्यथो॒ अथो॑ रेभन्ति रेभ॒ न्त्यथो॑ आहु राहु॒ रथो॑ रेभन्ति रेभ॒ न्त्यथो॑ आहुः । \newline
9. अथो॑ आहु राहु॒ रथो॒ अथो॑ आहुः प्र॒जासु॑ प्र॒जा स्वा॑हु॒ रथो॒ अथो॑ आहुः प्र॒जासु॑ । \newline
10. अथो॒ इत्यथो᳚ । \newline
11. आ॒हुः॒ प्र॒जासु॑ प्र॒जा स्वा॑हु राहुः प्र॒जासु॒ नि नि प्र॒जा स्वा॑हु राहुः प्र॒जासु॒ नि । \newline
12. प्र॒जासु॒ नि नि प्र॒जासु॑ प्र॒जासु॒ न्य॑मार् डमा॒र्ण् णि प्र॒जासु॑ प्र॒जासु॒ न्य॑मार्ट् । \newline
13. प्र॒जास्विति॑ प्र - जासु॑ । \newline
14. न्य॑मार् डमा॒र्ण् णि न्य॑मा॒र् डिती त्य॑मा॒र्ण् णि न्य॑मा॒र्डिति॑ । \newline
15. अ॒मा॒र् डितीत्य॑मार् डमा॒र्डिति॒ तस्मा॒त् तस्मा॒ दित्य॑मार् डमा॒र्डिति॒ तस्मा᳚त् । \newline
16. इति॒ तस्मा॒त् तस्मा॒ दितीति॒ तस्मा᳚द् य॒मौ य॒मौ तस्मा॒ दितीति॒ तस्मा᳚द् य॒मौ । \newline
17. तस्मा᳚द् य॒मौ य॒मौ तस्मा॒त् तस्मा᳚द् य॒मौ जा॑येते जायेते य॒मौ तस्मा॒त् तस्मा᳚द् य॒मौ जा॑येते । \newline
18. य॒मौ जा॑येते जायेते य॒मौ य॒मौ जा॑येते॒ तस्मा॒त् तस्मा᳚ज् जायेते य॒मौ य॒मौ जा॑येते॒ तस्मा᳚त् । \newline
19. जा॒ये॒ते॒ तस्मा॒त् तस्मा᳚ज् जायेते जायेते॒ तस्मा॑ दश्वत॒रो᳚ ऽश्वत॒र स्तस्मा᳚ज् जायेते जायेते॒ तस्मा॑ दश्वत॒रः । \newline
20. जा॒ये॒ते॒ इति॑ जायेते । \newline
21. तस्मा॑ दश्वत॒रो᳚ ऽश्वत॒र स्तस्मा॒त् तस्मा॑ दश्वत॒रो न नाश्व॑त॒र स्तस्मा॒त् तस्मा॑ दश्वत॒रो न । \newline
22. अ॒श्व॒त॒रो न नाश्व॑त॒रो᳚ ऽश्वत॒रो न प्र प्र णाश्व॑त॒रो᳚ ऽश्वत॒रो न प्र । \newline
23. न प्र प्र ण न प्र जा॑यते जायते॒ प्र ण न प्र जा॑यते । \newline
24. प्र जा॑यते जायते॒ प्र प्र जा॑यत॒ आत्त॑रेता॒ आत्त॑रेता जायते॒ प्र प्र जा॑यत॒ आत्त॑रेताः । \newline
25. जा॒य॒त॒ आत्त॑रेता॒ आत्त॑रेता जायते जायत॒ आत्त॑रेता॒ हि ह्यात्त॑रेता जायते जायत॒ आत्त॑रेता॒ हि । \newline
26. आत्त॑रेता॒ हि ह्यात्त॑रेता॒ आत्त॑रेता॒ हि तस्मा॒त् तस्मा॒ द्ध्यात्त॑रेता॒ आत्त॑रेता॒ हि तस्मा᳚त् । \newline
27. आत्त॑रेता॒ इत्यात्त॑ - रे॒ताः॒ । \newline
28. हि तस्मा॒त् तस्मा॒द्धि हि तस्मा᳚द् ब॒र्॒.हिषि॑ ब॒र्॒.हिषि॒ तस्मा॒द्धि हि तस्मा᳚द् ब॒र्॒.हिषि॑ । \newline
29. तस्मा᳚द् ब॒र्॒.हिषि॑ ब॒र्॒.हिषि॒ तस्मा॒त् तस्मा᳚द् ब॒र्॒.हि ष्यन॑वक्लृ॒प्तो ऽन॑वक्लृप्तो ब॒र्॒.हिषि॒ तस्मा॒त् तस्मा᳚द् ब॒र्॒.हि ष्यन॑वक्लृप्तः । \newline
30. ब॒र्॒.हि ष्यन॑वक्लृ॒प्तो ऽन॑वक्लृप्तो ब॒र्॒.हिषि॑ ब॒र्॒.हि ष्यन॑वक्लृप्तः सर्ववेद॒से स॑र्ववेद॒से ऽन॑वक्लृप्तो ब॒र्॒.हिषि॑ ब॒र्॒.हि ष्यन॑वक्लृप्तः सर्ववेद॒से । \newline
31. अन॑वक्लृप्तः सर्ववेद॒से स॑र्ववेद॒से ऽन॑वक्लृ॒प्तो ऽन॑वक्लृप्तः सर्ववेद॒से वा॑ वा सर्ववेद॒से ऽन॑वक्लृ॒प्तो ऽन॑वक्लृप्तः सर्ववेद॒से वा᳚ । \newline
32. अन॑वक्लृप्त॒ इत्यन॑व - क्लृ॒प्तः॒ । \newline
33. स॒र्व॒वे॒द॒से वा॑ वा सर्ववेद॒से स॑र्ववेद॒से वा॑ स॒हस्रे॑ स॒हस्रे॑ वा सर्ववेद॒से स॑र्ववेद॒से वा॑ स॒हस्रे᳚ । \newline
34. स॒र्व॒वे॒द॒स इति॑ सर्व - वे॒द॒से । \newline
35. वा॒ स॒हस्रे॑ स॒हस्रे॑ वा वा स॒हस्रे॑ वा वा स॒हस्रे॑ वा वा स॒हस्रे॑ वा । \newline
36. स॒हस्रे॑ वा वा स॒हस्रे॑ स॒हस्रे॒ वा ऽव॑क्लृ॒प्तो ऽव॑क्लृप्तो वा स॒हस्रे॑ स॒हस्रे॒ वा ऽव॑क्लृप्तः । \newline
37. वा ऽव॑क्लृ॒प्तो ऽव॑क्लृप्तो वा॒ वा ऽव॑क्लृ॒प्तो ऽत्य त्यव॑क्लृप्तो वा॒ वा ऽव॑क्लृ॒प्तो ऽति॑ । \newline
38. अव॑क्लृ॒प्तो ऽत्य त्यव॑क्लृ॒प्तो ऽव॑क्लृ॒प्तो ऽति॒ हि ह्यत्यव॑क्लृ॒प्तो ऽव॑क्लृ॒प्तो ऽति॒ हि । \newline
39. अव॑क्लृप्त॒ इत्यव॑ - क्लृ॒प्तः॒ । \newline
40. अति॒ हि ह्यत्यति॒ ह्यप्र॑व॒ता प्र॑वत॒ ह्यत्यति॒ ह्यप्र॑वत । \newline
41. ह्यप्र॑व॒ता प्र॑वत॒ हि ह्यप्र॑वत॒ यो यो ऽप्र॑वत॒ हि ह्यप्र॑वत॒ यः । \newline
42. अप्र॑वत॒ यो यो ऽप्र॑व॒ता प्र॑वत॒ य ए॒व मे॒वं ॅयो ऽप्र॑व॒ता प्र॑वत॒ य ए॒वम् । \newline
43. य ए॒व मे॒वं ॅयो य ए॒वं ॅवि॒द्वान्. वि॒द्वा ने॒वं ॅयो य ए॒वं ॅवि॒द्वान् । \newline
44. ए॒वं ॅवि॒द्वान्. वि॒द्वा ने॒व मे॒वं ॅवि॒द्वा न॑ग्निष्टो॒मेना᳚ ग्निष्टो॒मेन॑ वि॒द्वा ने॒व मे॒वं ॅवि॒द्वा न॑ग्निष्टो॒मेन॑ । \newline
45. वि॒द्वा न॑ग्निष्टो॒मेना᳚ ग्निष्टो॒मेन॑ वि॒द्वान्. वि॒द्वा न॑ग्निष्टो॒मेन॒ यज॑ते॒ यज॑ते ऽग्निष्टो॒मेन॑ वि॒द्वान्. वि॒द्वा न॑ग्निष्टो॒मेन॒ यज॑ते । \newline
46. अ॒ग्नि॒ष्टो॒मेन॒ यज॑ते॒ यज॑ते ऽग्निष्टो॒मेना᳚ ग्निष्टो॒मेन॒ यज॑ते॒ प्र प्र यज॑ते ऽग्निष्टो॒मेना᳚ ग्निष्टो॒मेन॒ यज॑ते॒ प्र । \newline
47. अ॒ग्नि॒ष्टो॒मेनेत्य॑ग्नि - स्तो॒मेन॑ । \newline
48. यज॑ते॒ प्र प्र यज॑ते॒ यज॑ते॒ प्राजा॑ता॒ अजा॑ताः॒ प्र यज॑ते॒ यज॑ते॒ प्राजा॑ताः । \newline
49. प्राजा॑ता॒ अजा॑ताः॒ प्र प्राजा॑ताः प्र॒जाः प्र॒जा अजा॑ताः॒ प्र प्राजा॑ताः प्र॒जाः । \newline
50. अजा॑ताः प्र॒जाः प्र॒जा अजा॑ता॒ अजा॑ताः प्र॒जा ज॒नय॑ति ज॒नय॑ति प्र॒जा अजा॑ता॒ अजा॑ताः प्र॒जा ज॒नय॑ति । \newline
51. प्र॒जा ज॒नय॑ति ज॒नय॑ति प्र॒जाः प्र॒जा ज॒नय॑ति॒ परि॒ परि॑ ज॒नय॑ति प्र॒जाः प्र॒जा ज॒नय॑ति॒ परि॑ । \newline
52. प्र॒जा इति॑ प्र - जाः । \newline
53. ज॒नय॑ति॒ परि॒ परि॑ ज॒नय॑ति ज॒नय॑ति॒ परि॒ प्रजा॑ताः॒ प्रजा॑ताः॒ परि॑ ज॒नय॑ति ज॒नय॑ति॒ परि॒ प्रजा॑ताः । \newline
54. परि॒ प्रजा॑ताः॒ प्रजा॑ताः॒ परि॒ परि॒ प्रजा॑ता गृह्णाति गृह्णाति॒ प्रजा॑ताः॒ परि॒ परि॒ प्रजा॑ता गृह्णाति । \newline
55. प्रजा॑ता गृह्णाति गृह्णाति॒ प्रजा॑ताः॒ प्रजा॑ता गृह्णाति॒ तस्मा॒त् तस्मा᳚द् गृह्णाति॒ प्रजा॑ताः॒ प्रजा॑ता गृह्णाति॒ तस्मा᳚त् । \newline
56. प्रजा॑ता॒ इति॒ प्र - जा॒ताः॒ । \newline
57. गृ॒ह्णा॒ति॒ तस्मा॒त् तस्मा᳚द् गृह्णाति गृह्णाति॒ तस्मा॑ दाहु राहु॒ स्तस्मा᳚द् गृह्णाति गृह्णाति॒ तस्मा॑ दाहुः । \newline
58. तस्मा॑ दाहु राहु॒ स्तस्मा॒त् तस्मा॑ दाहुर् ज्येष्ठय॒ज्ञो ज्ये᳚ष्ठय॒ज्ञ् आ॑हु॒ स्तस्मा॒त् तस्मा॑ दाहुर् ज्येष्ठय॒ज्ञ्ः । \newline
59. आ॒हु॒र् ज्ये॒ष्ठ॒य॒ज्ञो ज्ये᳚ष्ठय॒ज्ञ् आ॑हु राहुर् ज्येष्ठय॒ज्ञ् इतीति॑ ज्येष्ठय॒ज्ञ् आ॑हु राहुर् ज्येष्ठय॒ज्ञ् इति॑ । \newline
60. ज्ये॒ष्ठ॒य॒ज्ञ् इतीति॑ ज्येष्ठय॒ज्ञो ज्ये᳚ष्ठय॒ज्ञ् इति॑ प्र॒जाप॑तिः प्र॒जाप॑ति॒ रिति॑ ज्येष्ठय॒ज्ञो ज्ये᳚ष्ठय॒ज्ञ् इति॑ प्र॒जाप॑तिः । \newline
61. ज्ये॒ष्ठ॒य॒ज्ञ् इति॑ ज्येष्ठ - य॒ज्ञ्ः । \newline
62. इति॑ प्र॒जाप॑तिः प्र॒जाप॑ति॒ रितीति॑ प्र॒जाप॑ति॒र् वाव वाव प्र॒जाप॑ति॒ रितीति॑ प्र॒जाप॑ति॒र् वाव । \newline
\pagebreak
\markright{ TS 7.1.1.4  \hfill https://www.vedavms.in \hfill}

\section{ TS 7.1.1.4 }

\textbf{TS 7.1.1.4 } \newline
\textbf{Samhita Paata} \newline

प्र॒जाप॑ति॒र्वाव ज्येष्ठः॒ स ह्ये॑तेनाग्रेऽय॑जत प्र॒जाप॑तिरकामयत॒ प्र जा॑ये॒येति॒ स मु॑ख॒तस्त्रि॒वृतं॒ निर॑मिमीत॒ तम॒ग्निर्दे॒वता ऽन्व॑सृज्यत गाय॒त्री छन्दो॑ रथन्त॒रꣳ साम॑ ब्राह्म॒णो म॑नु॒ष्या॑णाम॒जः प॑शू॒नां तस्मा॒त् ते मुख्या॑ मुख॒तो ह्यसृ॑ज्य॒न्तोर॑सो बा॒हुभ्यां᳚ पञ्चद॒शं निर॑मिमीत॒ तमिन्द्रो॑ दे॒वता ऽन्व॑सृज्यत त्रि॒ष्टुप् छन्दो॑ बृ॒हथ् - [  ] \newline

\textbf{Pada Paata} \newline

प्र॒जाप॑ति॒रिति॑ प्र॒जा - प॒तिः॒ । वाव । ज्येष्ठः॑ । सः । हि । ए॒तेन॑ । अग्रे᳚ । अय॑जत । प्र॒जाप॑ति॒रिति॑ प्र॒जा - प॒तिः॒ । अ॒का॒म॒य॒त॒ । प्रेति॑ । जा॒ये॒य॒ । इति॑ । सः । मु॒ख॒तः । त्रि॒वृत॒मिति॑ त्रि - वृत᳚म् । निरिति॑ । अ॒मि॒मी॒त॒ । तम् । अ॒ग्निः । दे॒वता᳚ । अन्विति॑ । अ॒सृ॒ज्य॒त॒ । गा॒य॒त्री । छन्दः॑ । र॒थ॒न्त॒रमिति॑ रथं - त॒रम् । साम॑ । ब्रा॒ह्म॒णः । म॒नु॒ष्या॑णाम् । अ॒जः । प॒शू॒नाम् । तस्मा᳚त् । ते । मुख्याः᳚ । मु॒ख॒तः । हि । असृ॑ज्यन्त । उर॑सः । बा॒हुभ्या॒मिति॑ बा॒हु - भ्या॒म् । प॒ञ्च॒द॒शमिति॑ पञ्च - द॒शम् । निरिति॑ । अ॒मि॒मी॒त॒ । तम् । इन्द्रः॑ । दे॒वता᳚ । अन्विति॑ । अ॒सृ॒ज्य॒त॒ । त्रि॒ष्टुप् । छन्दः॑ । बृ॒हत् ।  \newline


\textbf{Krama Paata} \newline

प्र॒जाप॑ति॒र् वाव । प्र॒जाप॑ति॒रिति॑ प्र॒जा - प॒तिः॒ । वाव ज्येष्ठः॑ । ज्येष्ठः॒ सः । स हि । ह्ये॑तेन॑ । ए॒तेनाग्रे᳚ । अग्रेऽय॑जत । अय॑जत प्र॒जाप॑तिः । प्र॒जाप॑तिरकामयत । प्र॒जाप॑ति॒रिति॑ प्र॒जा - प॒तिः॒ । अ॒का॒म॒य॒त॒ प्र । प्र जा॑येय । जा॒ये॒येति॑ । इति॒ सः । स मु॑ख॒तः । मु॒ख॒त स्त्रि॒वृत᳚म् । त्रि॒वृत॒म् निः । त्रि॒वृत॒मिति॑ त्रि - वृत᳚म् । निर॑मिमीत । अ॒मि॒मी॒त॒ तम् । तम॒ग्निः । अ॒ग्निर् दे॒वता᳚ । दे॒वताऽनु॑ । अन्व॑सृज्यत । अ॒सृ॒ज्य॒त॒ गा॒य॒त्री । गा॒य॒त्री छन्दः॑ । छन्दो॑ रथन्त॒रम् । र॒थ॒न्त॒रꣳ साम॑ । र॒थ॒न्त॒रमिति॑ रथम् - त॒रम् । साम॑ ब्राह्म॒णः । ब्रा॒ह्म॒णो म॑नु॒ष्या॑णाम् । म॒नु॒ष्या॑णाम॒जः । अ॒जः प॑शू॒नाम् । प॒शू॒नाम् तस्मा᳚त् । तस्मा॒त् ते । ते मुख्याः᳚ । मुख्या॑ मुख॒तः । मु॒ख॒तो हि । ह्यसृ॑ज्यन्त । असृ॑ज्य॒न्तोर॑सः । उर॑सो बा॒हुभ्या᳚म् । बा॒हुभ्या᳚म् पञ्चद॒शम् । बा॒हुभ्या॒मिति॑ बा॒हु - भ्या॒म् । प॒ञ्च॒द॒शम् निः । प॒ञ्च॒द॒शमिति॑ पञ्च - द॒शम् । निर॑मिमीत । अ॒मि॒मी॒त॒ तम् । तमिन्द्रः॑ । इन्द्रो॑ दे॒वता᳚ । दे॒वताऽनु॑ । अन्व॑सृज्यत । अ॒सृ॒ज्य॒त॒ त्रि॒ष्टुप् । त्रि॒ष्टुप् छन्दः॑ । छन्दो॑ बृ॒हत् । बृ॒हथ् साम॑ \newline

\textbf{Jatai Paata} \newline

1. प्र॒जाप॑ति॒र् वाव वाव प्र॒जाप॑तिः प्र॒जाप॑ति॒र् वाव । \newline
2. प्र॒जाप॑ति॒रिति॑ प्र॒जा - प॒तिः॒ । \newline
3. वाव ज्येष्ठो॒ ज्येष्ठो॒ वाव वाव ज्येष्ठः॑ । \newline
4. ज्येष्ठः॒ स स ज्येष्ठो॒ ज्येष्ठः॒ सः । \newline
5. स हि हि स स हि । \newline
6. ह्ये॑ते नै॒तेन॒ हि ह्ये॑तेन॑ । \newline
7. ए॒तेनाग्रे ऽग्र॑ ए॒ते नै॒तेनाग्रे᳚ । \newline
8. अग्रे ऽय॑ज॒ता य॑ज॒ ताग्रे ऽग्रे ऽय॑जत । \newline
9. अय॑जत प्र॒जाप॑तिः प्र॒जाप॑ति॒ रय॑ज॒ता य॑जत प्र॒जाप॑तिः । \newline
10. प्र॒जाप॑ति रकामयता कामयत प्र॒जाप॑तिः प्र॒जाप॑ति रकामयत । \newline
11. प्र॒जाप॑ति॒रिति॑ प्र॒जा - प॒तिः॒ । \newline
12. अ॒का॒म॒य॒त॒ प्र प्राका॑मयता कामयत॒ प्र । \newline
13. प्र जा॑येय जायेय॒ प्र प्र जा॑येय । \newline
14. जा॒ये॒येतीति॑ जायेय जाये॒येति॑ । \newline
15. इति॒ स स इतीति॒ सः । \newline
16. स मु॑ख॒तो मु॑ख॒तः स स मु॑ख॒तः । \newline
17. मु॒ख॒त स्त्रि॒वृत॑म् त्रि॒वृत॑म् मुख॒तो मु॑ख॒त स्त्रि॒वृत᳚म् । \newline
18. त्रि॒वृत॒न् निर् णिष् ट्रि॒वृत॑म् त्रि॒वृत॒न् निः । \newline
19. त्रि॒वृत॒मिति॑ त्रि - वृत᳚म् । \newline
20. निर॑मिमीता मिमीत॒ निर् णिर॑मिमीत । \newline
21. अ॒मि॒मी॒त॒ तम् त म॑मिमीता मिमीत॒ तम् । \newline
22. त म॒ग्नि र॒ग्नि स्तम् त म॒ग्निः । \newline
23. अ॒ग्निर् दे॒वता॑ दे॒वता॒ ऽग्नि र॒ग्निर् दे॒वता᳚ । \newline
24. दे॒वता ऽन्वनु॑ दे॒वता॑ दे॒वता ऽनु॑ । \newline
25. अन्व॑सृज्यता सृज्य॒ता न्वन् व॑सृज्यत । \newline
26. अ॒सृ॒ज्य॒त॒ गा॒य॒त्री गा॑य॒ त्र्य॑सृज्यता सृज्यत गाय॒त्री । \newline
27. गा॒य॒त्री छन्द॒ श्छन्दो॑ गाय॒त्री गा॑य॒त्री छन्दः॑ । \newline
28. छन्दो॑ रथन्त॒रꣳ र॑थन्त॒रम् छन्द॒ श्छन्दो॑ रथन्त॒रम् । \newline
29. र॒थ॒न्त॒रꣳ साम॒ साम॑ रथन्त॒रꣳ र॑थन्त॒रꣳ साम॑ । \newline
30. र॒थ॒न्त॒रमिति॑ रथं - त॒रम् । \newline
31. साम॑ ब्राह्म॒णो ब्रा᳚ह्म॒णः साम॒ साम॑ ब्राह्म॒णः । \newline
32. ब्रा॒ह्म॒णो म॑नु॒ष्या॑णाम् मनु॒ष्या॑णाम् ब्राह्म॒णो ब्रा᳚ह्म॒णो म॑नु॒ष्या॑णाम् । \newline
33. म॒नु॒ष्या॑णा म॒जो॑ ऽजो म॑नु॒ष्या॑णाम् मनु॒ष्या॑णा म॒जः । \newline
34. अ॒जः प॑शू॒नाम् प॑शू॒ना म॒जो॑ ऽजः प॑शू॒नाम् । \newline
35. प॒शू॒नाम् तस्मा॒त् तस्मा᳚त् पशू॒नाम् प॑शू॒नाम् तस्मा᳚त् । \newline
36. तस्मा॒त् ते ते तस्मा॒त् तस्मा॒त् ते । \newline
37. ते मुख्या॒ मुख्या॒ स्ते ते मुख्याः᳚ । \newline
38. मुख्या॑ मुख॒तो मु॑ख॒तो मुख्या॒ मुख्या॑ मुख॒तः । \newline
39. मु॒ख॒तो हि हि मु॑ख॒तो मु॑ख॒तो हि । \newline
40. ह्यसृ॑ज्य॒न्ता सृ॑ज्यन्त॒ हि ह्यसृ॑ज्यन्त । \newline
41. असृ॑ज्य॒ न्तोर॑स॒ उर॒सो ऽसृ॑ज्य॒न्ता सृ॑ज्य॒ न्तोर॑सः । \newline
42. उर॑सो बा॒हुभ्या᳚म् बा॒हुभ्या॒ मुर॑स॒ उर॑सो बा॒हुभ्या᳚म् । \newline
43. बा॒हुभ्या᳚म् पञ्चद॒शम् प॑ञ्चद॒शम् बा॒हुभ्या᳚म् बा॒हुभ्या᳚म् पञ्चद॒शम् । \newline
44. बा॒हुभ्या॒मिति॑ बा॒हु - भ्या॒म् । \newline
45. प॒ञ्च॒द॒शन् निर् णिष् प॑ञ्चद॒शम् प॑ञ्चद॒शन् निः । \newline
46. प॒ञ्च॒द॒शमिति॑ पञ्च - द॒शम् । \newline
47. निर॑मिमीता मिमीत॒ निर् णिर॑मिमीत । \newline
48. अ॒मि॒मी॒त॒ तम् त म॑मिमीता मिमीत॒ तम् । \newline
49. त मिन्द्र॒ इन्द्र॒ स्तम् त मिन्द्रः॑ । \newline
50. इन्द्रो॑ दे॒वता॑ दे॒वतेन्द्र॒ इन्द्रो॑ दे॒वता᳚ । \newline
51. दे॒वता ऽन्वनु॑ दे॒वता॑ दे॒वता ऽनु॑ । \newline
52. अन्व॑ सृज्यता सृज्य॒तान् वन् व॑सृज्यत । \newline
53. अ॒सृ॒ज्य॒त॒ त्रि॒ष्टुप् त्रि॒ष्टुब॑ सृज्यता सृज्यत त्रि॒ष्टुप् । \newline
54. त्रि॒ष्टुप् छन्द॒ श्छन्द॑ स्त्रि॒ष्टुप् त्रि॒ष्टुप् छन्दः॑ । \newline
55. छन्दो॑ बृ॒हद् बृ॒हच् छन्द॒ श्छन्दो॑ बृ॒हत् । \newline
56. बृ॒हथ् साम॒ साम॑ बृ॒हद् बृ॒हथ् साम॑ । \newline

\textbf{Ghana Paata } \newline

1. प्र॒जाप॑ति॒र् वाव वाव प्र॒जाप॑तिः प्र॒जाप॑ति॒र् वाव ज्येष्ठो॒ ज्येष्ठो॒ वाव प्र॒जाप॑तिः प्र॒जाप॑ति॒र् वाव ज्येष्ठः॑ । \newline
2. प्र॒जाप॑ति॒रिति॑ प्र॒जा - प॒तिः॒ । \newline
3. वाव ज्येष्ठो॒ ज्येष्ठो॒ वाव वाव ज्येष्ठः॒ स स ज्येष्ठो॒ वाव वाव ज्येष्ठः॒ सः । \newline
4. ज्येष्ठः॒ स स ज्येष्ठो॒ ज्येष्ठः॒ स हि हि स ज्येष्ठो॒ ज्येष्ठः॒ स हि । \newline
5. स हि हि स स ह्ये॑ते नै॒तेन॒ हि स स ह्ये॑तेन॑ । \newline
6. ह्ये॑ते नै॒तेन॒ हि ह्ये॑ते नाग्रे ऽग्र॑ ए॒तेन॒ हि ह्ये॑ते नाग्रे᳚ । \newline
7. ए॒तेनाग्रे ऽग्र॑ ए॒ते नै॒तेनाग्रे ऽय॑ज॒ता य॑ज॒ ताग्र॑ ए॒ते नै॒तेनाग्रे ऽय॑जत । \newline
8. अग्रे ऽय॑ज॒ता य॑ज॒ताग्रे ऽग्रे ऽय॑जत प्र॒जाप॑तिः प्र॒जाप॑ति॒ रय॑ज॒ताग्रे ऽग्रे ऽय॑जत प्र॒जाप॑तिः । \newline
9. अय॑जत प्र॒जाप॑तिः प्र॒जाप॑ति॒ रय॑ज॒ता य॑जत प्र॒जाप॑ति रकामयता कामयत प्र॒जाप॑ति॒ रय॑ज॒ता य॑जत प्र॒जाप॑ति रकामयत । \newline
10. प्र॒जाप॑ति रकामयता कामयत प्र॒जाप॑तिः प्र॒जाप॑ति रकामयत॒ प्र प्राका॑मयत प्र॒जाप॑तिः प्र॒जाप॑ति रकामयत॒ प्र । \newline
11. प्र॒जाप॑ति॒रिति॑ प्र॒जा - प॒तिः॒ । \newline
12. अ॒का॒म॒य॒त॒ प्र प्राका॑मयता कामयत॒ प्र जा॑येय जायेय॒ प्राका॑मयता कामयत॒ प्र जा॑येय । \newline
13. प्र जा॑येय जायेय॒ प्र प्र जा॑ये॒येतीति॑ जायेय॒ प्र प्र जा॑ये॒येति॑ । \newline
14. जा॒ये॒येतीति॑ जायेय जाये॒येति॒ स स इति॑ जायेय जाये॒येति॒ सः । \newline
15. इति॒ स स इतीति॒ स मु॑ख॒तो मु॑ख॒तः स इतीति॒ स मु॑ख॒तः । \newline
16. स मु॑ख॒तो मु॑ख॒तः स स मु॑ख॒त स्त्रि॒वृत॑म् त्रि॒वृत॑म् मुख॒तः स स मु॑ख॒त स्त्रि॒वृत᳚म् । \newline
17. मु॒ख॒त स्त्रि॒वृत॑म् त्रि॒वृत॑म् मुख॒तो मु॑ख॒त स्त्रि॒वृत॒न् निर् णिष् ट्रि॒वृत॑म् मुख॒तो मु॑ख॒त स्त्रि॒वृत॒न् निः । \newline
18. त्रि॒वृत॒न् निर् णिष् ट्रि॒वृत॑म् त्रि॒वृत॒न् निर॑मिमीता मिमीत॒ निष् ट्रि॒वृत॑म् त्रि॒वृत॒न् निर॑मिमीत । \newline
19. त्रि॒वृत॒मिति॑ त्रि - वृत᳚म् । \newline
20. निर॑मिमीता मिमीत॒ निर् णिर॑मिमीत॒ तम् त म॑मिमीत॒ निर् णिर॑मिमीत॒ तम् । \newline
21. अ॒मि॒मी॒त॒ तम् त म॑मिमीता मिमीत॒ त म॒ग्नि र॒ग्नि स्त म॑मिमीता मिमीत॒ त म॒ग्निः । \newline
22. त म॒ग्नि र॒ग्नि स्तम् त म॒ग्निर् दे॒वता॑ दे॒वता॒ ऽग्नि स्तम् त म॒ग्निर् दे॒वता᳚ । \newline
23. अ॒ग्निर् दे॒वता॑ दे॒वता॒ ऽग्नि र॒ग्निर् दे॒वता ऽन्वनु॑ दे॒वता॒ ऽग्नि र॒ग्निर् दे॒वता ऽनु॑ । \newline
24. दे॒वता ऽन्वनु॑ दे॒वता॑ दे॒वता ऽन्व॑सृज्यता सृज्य॒तानु॑ दे॒वता॑ दे॒वता ऽन्व॑सृज्यत । \newline
25. अन्व॑सृज्यता सृज्य॒ता न्वन्व॑ सृज्यत गाय॒त्री गा॑य॒ त्र्य॑सृज्य॒ता न्वन्व॑ सृज्यत गाय॒त्री । \newline
26. अ॒सृ॒ज्य॒त॒ गा॒य॒त्री गा॑य॒ त्र्य॑सृज्यता सृज्यत गाय॒त्री छन्द॒ श्छन्दो॑ गाय॒ त्र्य॑सृज्यता सृज्यत गाय॒त्री छन्दः॑ । \newline
27. गा॒य॒त्री छन्द॒ श्छन्दो॑ गाय॒त्री गा॑य॒त्री छन्दो॑ रथन्त॒रꣳ र॑थन्त॒रम् छन्दो॑ गाय॒त्री गा॑य॒त्री छन्दो॑ रथन्त॒रम् । \newline
28. छन्दो॑ रथन्त॒रꣳ र॑थन्त॒रम् छन्द॒ श्छन्दो॑ रथन्त॒रꣳ साम॒ साम॑ रथन्त॒रम् छन्द॒ श्छन्दो॑ रथन्त॒रꣳ साम॑ । \newline
29. र॒थ॒न्त॒रꣳ साम॒ साम॑ रथन्त॒रꣳ र॑थन्त॒रꣳ साम॑ ब्राह्म॒णो ब्रा᳚ह्म॒णः साम॑ रथन्त॒रꣳ र॑थन्त॒रꣳ साम॑ ब्राह्म॒णः । \newline
30. र॒थ॒न्त॒रमिति॑ रथं - त॒रम् । \newline
31. साम॑ ब्राह्म॒णो ब्रा᳚ह्म॒णः साम॒ साम॑ ब्राह्म॒णो म॑नु॒ष्या॑णाम् मनु॒ष्या॑णाम् ब्राह्म॒णः साम॒ साम॑ ब्राह्म॒णो म॑नु॒ष्या॑णाम् । \newline
32. ब्रा॒ह्म॒णो म॑नु॒ष्या॑णाम् मनु॒ष्या॑णाम् ब्राह्म॒णो ब्रा᳚ह्म॒णो म॑नु॒ष्या॑णा म॒जो॑ ऽजो म॑नु॒ष्या॑णाम् ब्राह्म॒णो ब्रा᳚ह्म॒णो म॑नु॒ष्या॑णा म॒जः । \newline
33. म॒नु॒ष्या॑णा म॒जो॑ ऽजो म॑नु॒ष्या॑णाम् मनु॒ष्या॑णा म॒जः प॑शू॒नाम् प॑शू॒ना म॒जो म॑नु॒ष्या॑णाम् मनु॒ष्या॑णा म॒जः प॑शू॒नाम् । \newline
34. अ॒जः प॑शू॒नाम् प॑शू॒ना म॒जो॑ ऽजः प॑शू॒नाम् तस्मा॒त् तस्मा᳚त् पशू॒ना म॒जो॑ ऽजः प॑शू॒नाम् तस्मा᳚त् । \newline
35. प॒शू॒नाम् तस्मा॒त् तस्मा᳚त् पशू॒नाम् प॑शू॒नाम् तस्मा॒त् ते ते तस्मा᳚त् पशू॒नाम् प॑शू॒नाम् तस्मा॒त् ते । \newline
36. तस्मा॒त् ते ते तस्मा॒त् तस्मा॒त् ते मुख्या॒ मुख्या॒ स्ते तस्मा॒त् तस्मा॒त् ते मुख्याः᳚ । \newline
37. ते मुख्या॒ मुख्या॒ स्ते ते मुख्या॑ मुख॒तो मु॑ख॒तो मुख्या॒ स्ते ते मुख्या॑ मुख॒तः । \newline
38. मुख्या॑ मुख॒तो मु॑ख॒तो मुख्या॒ मुख्या॑ मुख॒तो हि हि मु॑ख॒तो मुख्या॒ मुख्या॑ मुख॒तो हि । \newline
39. मु॒ख॒तो हि हि मु॑ख॒तो मु॑ख॒तो ह्यसृ॑ज्य॒न्ता सृ॑ज्यन्त॒ हि मु॑ख॒तो मु॑ख॒तो ह्यसृ॑ज्यन्त । \newline
40. ह्यसृ॑ज्य॒न्ता सृ॑ज्यन्त॒ हि ह्यसृ॑ज्य॒न्तोर॑स॒ उर॒सो ऽसृ॑ज्यन्त॒ हि ह्यसृ॑ज्य॒ न्तोर॑सः । \newline
41. असृ॑ज्य॒ न्तोर॑स॒ उर॒सो ऽसृ॑ज्य॒न्ता सृ॑ज्य॒ न्तोर॑सो बा॒हुभ्या᳚म् बा॒हुभ्या॒ मुर॒सो ऽसृ॑ज्य॒न्ता सृ॑ज्य॒
न्तोर॑सो बा॒हुभ्या᳚म् । \newline
42. उर॑सो बा॒हुभ्या᳚म् बा॒हुभ्या॒ मुर॑स॒ उर॑सो बा॒हुभ्या᳚म् पञ्चद॒शम् प॑ञ्चद॒शम् बा॒हुभ्या॒ मुर॑स॒ उर॑सो बा॒हुभ्या᳚म् पञ्चद॒शम् । \newline
43. बा॒हुभ्या᳚म् पञ्चद॒शम् प॑ञ्चद॒शम् बा॒हुभ्या᳚म् बा॒हुभ्या᳚म् पञ्चद॒शन् निर् णिष् प॑ञ्चद॒शम् बा॒हुभ्या᳚म् बा॒हुभ्या᳚म् पञ्चद॒शन् निः । \newline
44. बा॒हुभ्या॒मिति॑ बा॒हु - भ्या॒म् । \newline
45. प॒ञ्च॒द॒शन् निर् णिष् प॑ञ्चद॒शम् प॑ञ्चद॒शम् निर॑मिमीता मिमीत॒ निष् प॑ञ्चद॒शम् प॑ञ्चद॒शम् निर॑मिमीत । \newline
46. प॒ञ्च॒द॒शमिति॑ पञ्च - द॒शम् । \newline
47. निर॑मिमीता मिमीत॒ निर् णिर॑मिमीत॒ तम् त म॑मिमीत॒ निर् णिर॑मिमीत॒ तम् । \newline
48. अ॒मि॒मी॒त॒ तम् त म॑मिमीता मिमीत॒ त मिन्द्र॒ इन्द्र॒ स्त म॑मिमीता मिमीत॒ त मिन्द्रः॑ । \newline
49. त मिन्द्र॒ इन्द्र॒ स्तम् त मिन्द्रो॑ दे॒वता॑ दे॒वतेन्द्र॒ स्तम् त मिन्द्रो॑ दे॒वता᳚ । \newline
50. इन्द्रो॑ दे॒वता॑ दे॒वतेन्द्र॒ इन्द्रो॑ दे॒वता ऽन्वनु॑ दे॒वतेन्द्र॒ इन्द्रो॑ दे॒वता ऽनु॑ । \newline
51. दे॒वता ऽन्वनु॑ दे॒वता॑ दे॒वता ऽन्व॑सृज्यता सृज्य॒तानु॑ दे॒वता॑ दे॒वता ऽन्व॑सृज्यत । \newline
52. अन्व॑ सृज्यता सृज्य॒ता न्वन् व॑सृज्यत त्रि॒ष्टुप् त्रि॒ष्टु ब॑सृज्य॒ता न्वन् व॑सृज्यत त्रि॒ष्टुप् । \newline
53. अ॒सृ॒ज्य॒त॒ त्रि॒ष्टुप् त्रि॒ष्टु ब॑सृज्यता सृज्यत त्रि॒ष्टुप् छन्द॒ श्छन्द॑ स्त्रि॒ष्टुब॑ सृज्यता सृज्यत त्रि॒ष्टुप् छन्दः॑ । \newline
54. त्रि॒ष्टुप् छन्द॒ श्छन्द॑ स्त्रि॒ष्टुप् त्रि॒ष्टुप् छन्दो॑ बृ॒हद् बृ॒हच् छन्द॑ स्त्रि॒ष्टुप् त्रि॒ष्टुप् छन्दो॑ बृ॒हत् । \newline
55. छन्दो॑ बृ॒हद् बृ॒हच् छन्द॒ श्छन्दो॑ बृ॒हथ् साम॒ साम॑ बृ॒हच् छन्द॒ श्छन्दो॑ बृ॒हथ् साम॑ । \newline
56. बृ॒हथ् साम॒ साम॑ बृ॒हद् बृ॒हथ् साम॑ राज॒न्यो॑ राज॒न्यः॑ साम॑ बृ॒हद् बृ॒हथ् साम॑ राज॒न्यः॑ । \newline
\pagebreak
\markright{ TS 7.1.1.5  \hfill https://www.vedavms.in \hfill}

\section{ TS 7.1.1.5 }

\textbf{TS 7.1.1.5 } \newline
\textbf{Samhita Paata} \newline

साम॑ राज॒न्यो॑ मनु॒ष्या॑णा॒मविः॑ पशू॒नां तस्मा॒त् ते वी॒र्या॑वन्तो वी॒र्या᳚द्ध्यसृ॑ज्यन्त मद्ध्य॒तः स॑प्तद॒शं निर॑मिमीत॒ तं ॅविश्वे॑ दे॒वा दे॒वता॒ अन्व॑सृज्यन्त॒ जग॑ती॒ छन्दो॑ वै रू॒पꣳ साम॒ वैश्यो॑ मनु॒ष्या॑णां॒ गावः॑ पशू॒नां तस्मा॒त् त आ॒द्या॑ अन्न॒धाना॒द्ध्य सृ॑ज्यन्त॒ तस्मा॒द्-भूयाꣳ॑सो॒ ऽन्येभ्यो॒ भूयि॑ष्ठा॒ हि दे॒वता॒ अन्वसृ॑ज्यन्त प॒त्त ए॑कविꣳ॒॒ शं निर॑मिमीत॒ तम॑नु॒ष्टुप् छन्दो - [  ] \newline

\textbf{Pada Paata} \newline

साम॑ । रा॒ज॒न्यः॑ । म॒नु॒ष्या॑णाम् । अविः॑ । प॒शू॒नाम् । तस्मा᳚त् । ते । वी॒र्या॑वन्त॒ इति॑ वी॒र्य॑ - व॒न्तः॒ । वी॒र्या᳚त् । हि । असृ॑ज्यन्त । म॒द्ध्य॒तः । स॒प्त॒द॒शमिति॑ सप्त - द॒शम् । निरिति॑ । अ॒मि॒मी॒त॒ । तम् । विश्वे᳚ । दे॒वाः । दे॒वताः᳚ । अन्विति॑ । अ॒सृ॒ज्य॒न्त॒ । जग॑ती । छन्दः॑ । वै॒रू॒पम् । साम॑ । वैश्यः॑ । म॒नु॒ष्या॑णाम् । गावः॑ । प॒शू॒नाम् । तस्मा᳚त् । ते । आ॒द्याः᳚ । अ॒न्न॒धाना॒दित्य॑न्न - धाना᳚त् । हि । असृ॑ज्यन्त । तस्मा᳚त् । भूयाꣳ॑सः । अ॒न्येभ्यः॑ । भूयि॑ष्ठाः । हि । दे॒वताः᳚ । अन्विति॑ । असृ॑ज्यन्त । प॒त्तः । ए॒क॒विꣳ॒॒शमित्ये॑क - विꣳ॒॒शम् । निरिति॑ । अ॒मि॒मी॒त॒ । तम् । अ॒नु॒ष्टुबित्य॑नु - स्तुप् । छन्दः॑ ।  \newline


\textbf{Krama Paata} \newline

साम॑ राज॒न्यः॑ । रा॒ज॒न्यो॑ मनु॒ष्या॑णाम् । म॒नु॒ष्या॑णा॒मविः॑ । अविः॑ पशू॒नाम् । प॒शू॒नाम् तस्मा᳚त् । तस्मा॒त् ते । ते वी॒र्या॑वन्तः । वी॒र्या॑वन्तो वी॒र्या᳚त् । वी॒र्या॑वन्त॒ इति॑ वी॒र्य॑ - व॒न्तः॒ । वी॒र्या᳚द्‌धि । ह्यसृ॑ज्यन्त । असृ॑ज्यन्त मद्ध्य॒तः । म॒द्ध्य॒तः स॑प्तद॒शम् । स॒प्त॒द॒शम् निः । स॒प्त॒द॒शमिति॑ सप्त - द॒शम् । निर॑मिमीत । अ॒मि॒मी॒त॒ तम् । तम् ॅविश्वे᳚ । विश्वे॑ दे॒वाः । दे॒वा दे॒वताः᳚ । दे॒वता॒ अनु॑ । अन्व॑सृज्यन्त । अ॒सृ॒ज्य॒न्त॒ जग॑ती । जग॑ती॒ छन्दः॑ । छन्दो॑ वैरू॒पम् । वै॒रू॒पꣳ साम॑ । साम॒ वैश्यः॑ । वैश्यो॑ मनु॒ष्या॑णाम् । म॒नु॒ष्या॑णा॒म् गावः॑ । गावः॑ पशू॒नाम् । प॒शू॒नाम् तस्मा᳚त् । तस्मा॒त् ते । त आ॒द्याः᳚ । आ॒द्या॑ अन्न॒धाना᳚त् । अ॒न्न॒धाना॒द्‌धि । अ॒न्न॒धाना॒दित्य॑न्न - धाना᳚त् । ह्यसृ॑ज्यन्त । असृ॑ज्यन्त॒ तस्मा᳚त् । तस्मा॒द् भूयाꣳ॑सः । भूयाꣳ॑सो॒ऽन्येभ्यः॑ । अ॒न्येभ्यो॒ भूयि॑ष्ठाः । भूयि॑ष्ठा॒ हि । हि दे॒वताः᳚ । दे॒वता॒ अनु॑ । अन्वसृ॑ज्यन्त । असृ॑ज्यन्त प॒त्तः । प॒त्त ए॑कविꣳ॒॒शम् । ए॒क॒विꣳ॒॒शम् निः । ए॒क॒विꣳ॒॒शमित्ये॑क - विꣳ॒॒शम् । निर॑मिमीत । अ॒मि॒मी॒त॒ तम् । तम॑नु॒ष्टुप् । अ॒नु॒ष्टुप् छन्दः॑ । अ॒नु॒ष्टुबित्य॑नु - स्तुप् । छन्दोऽनु॑ \newline

\textbf{Jatai Paata} \newline

1. साम॑ राज॒न्यो॑ राज॒न्यः॑ साम॒ साम॑ राज॒न्यः॑ । \newline
2. रा॒ज॒न्यो॑ मनु॒ष्या॑णाम् मनु॒ष्या॑णाꣳ राज॒न्यो॑ राज॒न्यो॑ मनु॒ष्या॑णाम् । \newline
3. म॒नु॒ष्या॑णा॒ मवि॒ रवि॑र् मनु॒ष्या॑णाम् मनु॒ष्या॑णा॒ मविः॑ । \newline
4. अविः॑ पशू॒नाम् प॑शू॒ना मवि॒ रविः॑ पशू॒नाम् । \newline
5. प॒शू॒नाम् तस्मा॒त् तस्मा᳚त् पशू॒नाम् प॑शू॒नाम् तस्मा᳚त् । \newline
6. तस्मा॒त् ते ते तस्मा॒त् तस्मा॒त् ते । \newline
7. ते वी॒र्या॑वन्तो वी॒र्या॑वन्त॒ स्ते ते वी॒र्या॑वन्तः । \newline
8. वी॒र्या॑वन्तो वी॒र्या᳚द् वी॒र्या᳚द् वी॒र्या॑वन्तो वी॒र्या॑वन्तो वी॒र्या᳚त् । \newline
9. वी॒र्या॑वन्त॒ इति॑ वी॒र्य॑ - व॒न्तः॒ । \newline
10. वी॒र्या᳚द्धि हि वी॒र्या᳚द् वी॒र्या᳚द्धि । \newline
11. ह्यसृ॑ज्य॒न्ता सृ॑ज्यन्त॒ हि ह्यसृ॑ज्यन्त । \newline
12. असृ॑ज्यन्त मद्ध्य॒तो म॑द्ध्य॒तो ऽसृ॑ज्य॒न्ता सृ॑ज्यन्त मद्ध्य॒तः । \newline
13. म॒द्ध्य॒तः स॑प्तद॒शꣳ स॑प्तद॒शम् म॑द्ध्य॒तो म॑द्ध्य॒तः स॑प्तद॒शम् । \newline
14. स॒प्त॒द॒शन् निर् णिः स॑प्तद॒शꣳ स॑प्तद॒शन् निः । \newline
15. स॒प्त॒द॒शमिति॑ सप्त - द॒शम् । \newline
16. निर॑मिमीता मिमीत॒ निर् णिर॑मिमीत । \newline
17. अ॒मि॒मी॒त॒ तम् त म॑मिमीता मिमीत॒ तम् । \newline
18. तं ॅविश्वे॒ विश्वे॒ तम् तं ॅविश्वे᳚ । \newline
19. विश्वे॑ दे॒वा दे॒वा विश्वे॒ विश्वे॑ दे॒वाः । \newline
20. दे॒वा दे॒वता॑ दे॒वता॑ दे॒वा दे॒वा दे॒वताः᳚ । \newline
21. दे॒वता॒ अन्वनु॑ दे॒वता॑ दे॒वता॒ अनु॑ । \newline
22. अन्व॑ सृज्यन्ता सृज्य॒न्तान् वन् व॑सृज्यन्त । \newline
23. अ॒सृ॒ज्य॒न्त॒ जग॑ती॒ जग॑ त्यसृज्यन्ता सृज्यन्त॒ जग॑ती । \newline
24. जग॑ती॒ छन्द॒ श्छन्दो॒ जग॑ती॒ जग॑ती॒ छन्दः॑ । \newline
25. छन्दो॑ वैरू॒पं ॅवै॑रू॒पम् छन्द॒ श्छन्दो॑ वैरू॒पम् । \newline
26. वै॒रू॒पꣳ साम॒ साम॑ वैरू॒पं ॅवै॑रू॒पꣳ साम॑ । \newline
27. साम॒ वैश्यो॒ वैश्यः॒ साम॒ साम॒ वैश्यः॑ । \newline
28. वैश्यो॑ मनु॒ष्या॑णाम् मनु॒ष्या॑णां॒ ॅवैश्यो॒ वैश्यो॑ मनु॒ष्या॑णाम् । \newline
29. म॒नु॒ष्या॑णा॒म् गावो॒ गावो॑ मनु॒ष्या॑णाम् मनु॒ष्या॑णा॒म् गावः॑ । \newline
30. गावः॑ पशू॒नाम् प॑शू॒नाम् गावो॒ गावः॑ पशू॒नाम् । \newline
31. प॒शू॒नाम् तस्मा॒त् तस्मा᳚त् पशू॒नाम् प॑शू॒नाम् तस्मा᳚त् । \newline
32. तस्मा॒त् ते ते तस्मा॒त् तस्मा॒त् ते । \newline
33. त आ॒द्या॑ आ॒द्या᳚ स्ते त आ॒द्याः᳚ । \newline
34. आ॒द्या॑ अन्न॒धाना॑ दन्न॒धाना॑ दा॒द्या॑ आ॒द्या॑ अन्न॒धाना᳚त् । \newline
35. अ॒न्न॒धा ना॒द्धि ह्य॑न्न॒धाना॑ दन्न॒धा ना॒द्धि । \newline
36. अ॒न्न॒धाना॒दित्य॑न्न - धाना᳚त् । \newline
37. ह्यसृ॑ज्य॒न्ता सृ॑ज्यन्त॒ हि ह्यसृ॑ज्यन्त । \newline
38. असृ॑ज्यन्त॒ तस्मा॒त् तस्मा॒ दसृ॑ज्य॒न्ता सृ॑ज्यन्त॒ तस्मा᳚त् । \newline
39. तस्मा॒द् भूयाꣳ॑सो॒ भूयाꣳ॑स॒ स्तस्मा॒त् तस्मा॒द् भूयाꣳ॑सः । \newline
40. भूयाꣳ॑सो॒ ऽन्येभ्यो॒ ऽन्येभ्यो॒ भूयाꣳ॑सो॒ भूयाꣳ॑सो॒ ऽन्येभ्यः॑ । \newline
41. अ॒न्येभ्यो॒ भूयि॑ष्ठा॒ भूयि॑ष्ठा अ॒न्येभ्यो॒ ऽन्येभ्यो॒ भूयि॑ष्ठाः । \newline
42. भूयि॑ष्ठा॒ हि हि भूयि॑ष्ठा॒ भूयि॑ष्ठा॒ हि । \newline
43. हि दे॒वता॑ दे॒वता॒ हि हि दे॒वताः᳚ । \newline
44. दे॒वता॒ अन्वनु॑ दे॒वता॑ दे॒वता॒ अनु॑ । \newline
45. अन्व सृ॑ज्य॒न्ता सृ॑ज्य॒न्तान् वन् वसृ॑ज्यन्त । \newline
46. असृ॑ज्यन्त प॒त्तः प॒त्तो ऽसृ॑ज्य॒न्ता सृ॑ज्यन्त प॒त्तः । \newline
47. प॒त्त ए॑कविꣳ॒॒श मे॑कविꣳ॒॒शम् प॒त्तः प॒त्त ए॑कविꣳ॒॒शम् । \newline
48. ए॒क॒विꣳ॒॒शन् निर् णिरे॑कविꣳ॒॒श मे॑कविꣳ॒॒शन् निः । \newline
49. ए॒क॒विꣳ॒॒शमित्ये॑क - विꣳ॒॒शम् । \newline
50. निर॑मिमीता मिमीत॒ निर् णिर॑मिमीत । \newline
51. अ॒मि॒मी॒त॒ तम् त म॑मिमीता मिमीत॒ तम् । \newline
52. त म॑नु॒ष्टु ब॑नु॒ष्टुप् तम् त म॑नु॒ष्टुप् । \newline
53. अ॒नु॒ष्टुप् छन्द॒ श्छन्दो॑ ऽनु॒ष्टु ब॑नु॒ष्टुप् छन्दः॑ । \newline
54. अ॒नु॒ष्टुबित्य॑नु - स्तुप् । \newline
55. छन्दो ऽन्वनु॒ च्छन्द॒ श्छन्दो ऽनु॑ । \newline

\textbf{Ghana Paata } \newline

1. साम॑ राज॒न्यो॑ राज॒न्यः॑ साम॒ साम॑ राज॒न्यो॑ मनु॒ष्या॑णाम् मनु॒ष्या॑णाꣳ राज॒न्यः॑ साम॒ साम॑ राज॒न्यो॑ मनु॒ष्या॑णाम् । \newline
2. रा॒ज॒न्यो॑ मनु॒ष्या॑णाम् मनु॒ष्या॑णाꣳ राज॒न्यो॑ राज॒न्यो॑ मनु॒ष्या॑णा॒ मवि॒ रवि॑र् मनु॒ष्या॑णाꣳ राज॒न्यो॑ राज॒न्यो॑ मनु॒ष्या॑णा॒ मविः॑ । \newline
3. म॒नु॒ष्या॑णा॒ मवि॒ रवि॑र् मनु॒ष्या॑णाम् मनु॒ष्या॑णा॒ मविः॑ पशू॒नाम् प॑शू॒ना मवि॑र् मनु॒ष्या॑णाम् मनु॒ष्या॑णा॒ मविः॑ पशू॒नाम् । \newline
4. अविः॑ पशू॒नाम् प॑शू॒ना मवि॒ रविः॑ पशू॒नाम् तस्मा॒त् तस्मा᳚त् पशू॒ना मवि॒ रविः॑ पशू॒नाम् तस्मा᳚त् । \newline
5. प॒शू॒नाम् तस्मा॒त् तस्मा᳚त् पशू॒नाम् प॑शू॒नाम् तस्मा॒त् ते ते तस्मा᳚त् पशू॒नाम् प॑शू॒नाम् तस्मा॒त् ते । \newline
6. तस्मा॒त् ते ते तस्मा॒त् तस्मा॒त् ते वी॒र्या॑वन्तो वी॒र्या॑वन्त॒ स्ते तस्मा॒त् तस्मा॒त् ते वी॒र्या॑वन्तः । \newline
7. ते वी॒र्या॑वन्तो वी॒र्या॑वन्त॒ स्ते ते वी॒र्या॑वन्तो वी॒र्या᳚द् वी॒र्या᳚द् वी॒र्या॑वन्त॒ स्ते ते वी॒र्या॑वन्तो वी॒र्या᳚त् । \newline
8. वी॒र्या॑वन्तो वी॒र्या᳚द् वी॒र्या᳚द् वी॒र्या॑वन्तो वी॒र्या॑वन्तो वी॒र्या᳚द्धि हि वी॒र्या᳚द् वी॒र्या॑वन्तो वी॒र्या॑वन्तो वी॒र्या᳚द्धि । \newline
9. वी॒र्या॑वन्त॒ इति॑ वी॒र्य॑ - व॒न्तः॒ । \newline
10. वी॒र्या᳚द्धि हि वी॒र्या᳚द् वी॒र्या᳚ द्ध्यसृ॑ज्य॒न्ता सृ॑ज्यन्त॒ हि वी॒र्या᳚द् वी॒र्या᳚ द्ध्यसृ॑ज्यन्त । \newline
11. ह्यसृ॑ज्य॒न्ता सृ॑ज्यन्त॒ हि ह्यसृ॑ज्यन्त मद्ध्य॒तो म॑द्ध्य॒तो ऽसृ॑ज्यन्त॒ हि ह्यसृ॑ज्यन्त मद्ध्य॒तः । \newline
12. असृ॑ज्यन्त मद्ध्य॒तो म॑द्ध्य॒तो ऽसृ॑ज्य॒न्ता सृ॑ज्यन्त मद्ध्य॒तः स॑प्तद॒शꣳ स॑प्तद॒शम् म॑द्ध्य॒तो ऽसृ॑ज्य॒न्ता सृ॑ज्यन्त मद्ध्य॒तः स॑प्तद॒शम् । \newline
13. म॒द्ध्य॒तः स॑प्तद॒शꣳ स॑प्तद॒शम् म॑द्ध्य॒तो म॑द्ध्य॒तः स॑प्तद॒शन् निर् णिः स॑प्तद॒शम् म॑द्ध्य॒तो म॑द्ध्य॒तः स॑प्तद॒शन् निः । \newline
14. स॒प्त॒द॒शन् निर् णिः स॑प्तद॒शꣳ स॑प्तद॒शम् निर॑मिमीता मिमीत॒ निः स॑प्तद॒शꣳ स॑प्तद॒शम् निर॑मिमीत । \newline
15. स॒प्त॒द॒शमिति॑ सप्त - द॒शम् । \newline
16. निर॑मिमीता मिमीत॒ निर् णिर॑मिमीत॒ तम् त म॑मिमीत॒ निर् णिर॑मिमीत॒ तम् । \newline
17. अ॒मि॒मी॒त॒ तम् त म॑मिमीता मिमीत॒ तं ॅविश्वे॒ विश्वे॒ त म॑मिमीता मिमीत॒ तं ॅविश्वे᳚ । \newline
18. तं ॅविश्वे॒ विश्वे॒ तम् तं ॅविश्वे॑ दे॒वा दे॒वा विश्वे॒ तम् तं ॅविश्वे॑ दे॒वाः । \newline
19. विश्वे॑ दे॒वा दे॒वा विश्वे॒ विश्वे॑ दे॒वा दे॒वता॑ दे॒वता॑ दे॒वा विश्वे॒ विश्वे॑ दे॒वा दे॒वताः᳚ । \newline
20. दे॒वा दे॒वता॑ दे॒वता॑ दे॒वा दे॒वा दे॒वता॒ अन्वनु॑ दे॒वता॑ दे॒वा दे॒वा दे॒वता॒ अनु॑ । \newline
21. दे॒वता॒ अन्वनु॑ दे॒वता॑ दे॒वता॒ अन्व॑ सृज्यन्ता सृज्य॒न्तानु॑ दे॒वता॑ दे॒वता॒ अन्व॑ सृज्यन्त । \newline
22. अन्व॑ सृज्यन्ता सृज्य॒न्तान् वन् व॑सृज्यन्त॒ जग॑ती॒ जग॑ त्यसृज्य॒न्तान् वन् व॑सृज्यन्त॒ जग॑ती । \newline
23. अ॒सृ॒ज्य॒न्त॒ जग॑ती॒ जग॑ त्यसृज्यन्ता सृज्यन्त॒ जग॑ती॒ छन्द॒ श्छन्दो॒ जग॑ त्यसृज्यन्ता सृज्यन्त॒ जग॑ती॒ छन्दः॑ । \newline
24. जग॑ती॒ छन्द॒ श्छन्दो॒ जग॑ती॒ जग॑ती॒ छन्दो॑ वैरू॒पं ॅवै॑रू॒पम् छन्दो॒ जग॑ती॒ जग॑ती॒ छन्दो॑ वैरू॒पम् । \newline
25. छन्दो॑ वैरू॒पं ॅवै॑रू॒पम् छन्द॒ श्छन्दो॑ वैरू॒पꣳ साम॒ साम॑ वैरू॒पम् छन्द॒ श्छन्दो॑ वैरू॒पꣳ साम॑ । \newline
26. वै॒रू॒पꣳ साम॒ साम॑ वैरू॒पं ॅवै॑रू॒पꣳ साम॒ वैश्यो॒ वैश्यः॒ साम॑ वैरू॒पं ॅवै॑रू॒पꣳ साम॒ वैश्यः॑ । \newline
27. साम॒ वैश्यो॒ वैश्यः॒ साम॒ साम॒ वैश्यो॑ मनु॒ष्या॑णाम् मनु॒ष्या॑णां॒ ॅवैश्यः॒ साम॒ साम॒ वैश्यो॑ मनु॒ष्या॑णाम् । \newline
28. वैश्यो॑ मनु॒ष्या॑णाम् मनु॒ष्या॑णां॒ ॅवैश्यो॒ वैश्यो॑ मनु॒ष्या॑णा॒म् गावो॒ गावो॑ मनु॒ष्या॑णां॒ ॅवैश्यो॒ वैश्यो॑ मनु॒ष्या॑णा॒म् गावः॑ । \newline
29. म॒नु॒ष्या॑णा॒म् गावो॒ गावो॑ मनु॒ष्या॑णाम् मनु॒ष्या॑णा॒म् गावः॑ पशू॒नाम् प॑शू॒नाम् गावो॑ मनु॒ष्या॑णाम् मनु॒ष्या॑णा॒म् गावः॑ पशू॒नाम् । \newline
30. गावः॑ पशू॒नाम् प॑शू॒नाम् गावो॒ गावः॑ पशू॒नाम् तस्मा॒त् तस्मा᳚त् पशू॒नाम् गावो॒ गावः॑ पशू॒नाम् तस्मा᳚त् । \newline
31. प॒शू॒नाम् तस्मा॒त् तस्मा᳚त् पशू॒नाम् प॑शू॒नाम् तस्मा॒त् ते ते तस्मा᳚त् पशू॒नाम् प॑शू॒नाम् तस्मा॒त् ते । \newline
32. तस्मा॒त् ते ते तस्मा॒त् तस्मा॒त् त आ॒द्या॑ आ॒द्या᳚ स्ते तस्मा॒त् तस्मा॒त् त आ॒द्याः᳚ । \newline
33. त आ॒द्या॑ आ॒द्या᳚ स्ते त आ॒द्या॑ अन्न॒धाना॑ दन्न॒धाना॑ दा॒द्या᳚ स्ते त आ॒द्या॑ अन्न॒धाना᳚त् । \newline
34. आ॒द्या॑ अन्न॒धाना॑ दन्न॒धाना॑ दा॒द्या॑ आ॒द्या॑ अन्न॒धा ना॒द्धि ह्य॑न्न॒धाना॑ दा॒द्या॑ आ॒द्या॑ अन्न॒धा ना॒द्धि । \newline
35. अ॒न्न॒धाना॒द्धि ह्य॑न्न॒धाना॑ दन्न॒धाना॒ द्ध्यसृ॑ज्य॒न्ता सृ॑ज्यन्त॒ ह्य॑न्न॒धाना॑ दन्न॒धाना॒
द्ध्यसृ॑ज्यन्त । \newline
36. अ॒न्न॒धाना॒दित्य॑न्न - धाना᳚त् । \newline
37. ह्यसृ॑ज्य॒न्ता सृ॑ज्यन्त॒ हि ह्यसृ॑ज्यन्त॒ तस्मा॒त् तस्मा॒ दसृ॑ज्यन्त॒ हि ह्यसृ॑ज्यन्त॒ तस्मा᳚त् । \newline
38. असृ॑ज्यन्त॒ तस्मा॒त् तस्मा॒ दसृ॑ज्य॒न्ता सृ॑ज्यन्त॒ तस्मा॒द् भूयाꣳ॑सो॒ भूयाꣳ॑स॒ स्तस्मा॒ दसृ॑ज्य॒न्ता सृ॑ज्यन्त॒ तस्मा॒द् भूयाꣳ॑सः । \newline
39. तस्मा॒द् भूयाꣳ॑सो॒ भूयाꣳ॑स॒ स्तस्मा॒त् तस्मा॒द् भूयाꣳ॑सो॒ ऽन्येभ्यो॒ ऽन्येभ्यो॒ भूयाꣳ॑स॒ स्तस्मा॒त् तस्मा॒द् भूयाꣳ॑सो॒ ऽन्येभ्यः॑ । \newline
40. भूयाꣳ॑सो॒ ऽन्येभ्यो॒ ऽन्येभ्यो॒ भूयाꣳ॑सो॒ भूयाꣳ॑सो॒ ऽन्येभ्यो॒ भूयि॑ष्ठा॒ भूयि॑ष्ठा अ॒न्येभ्यो॒ भूयाꣳ॑सो॒ भूयाꣳ॑सो॒ ऽन्येभ्यो॒ भूयि॑ष्ठाः । \newline
41. अ॒न्येभ्यो॒ भूयि॑ष्ठा॒ भूयि॑ष्ठा अ॒न्येभ्यो॒ ऽन्येभ्यो॒ भूयि॑ष्ठा॒ हि हि भूयि॑ष्ठा अ॒न्येभ्यो॒ ऽन्येभ्यो॒ भूयि॑ष्ठा॒ हि । \newline
42. भूयि॑ष्ठा॒ हि हि भूयि॑ष्ठा॒ भूयि॑ष्ठा॒ हि दे॒वता॑ दे॒वता॒ हि भूयि॑ष्ठा॒ भूयि॑ष्ठा॒ हि दे॒वताः᳚ । \newline
43. हि दे॒वता॑ दे॒वता॒ हि हि दे॒वता॒ अन्वनु॑ दे॒वता॒ हि हि दे॒वता॒ अनु॑ । \newline
44. दे॒वता॒ अन्वनु॑ दे॒वता॑ दे॒वता॒ अन्व सृ॑ज्य॒न्ता सृ॑ज्य॒न्तानु॑ दे॒वता॑ दे॒वता॒ अन्व सृ॑ज्यन्त । \newline
45. अन्व सृ॑ज्य॒न्ता सृ॑ज्य॒न्तान् वन् वसृ॑ज्यन्त प॒त्तः प॒त्तो ऽसृ॑ज्य॒न्तान् वन् वसृ॑ज्यन्त प॒त्तः । \newline
46. असृ॑ज्यन्त प॒त्तः प॒त्तो ऽसृ॑ज्य॒न्ता सृ॑ज्यन्त प॒त्त ए॑कविꣳ॒॒श मे॑कविꣳ॒॒शम् प॒त्तो ऽसृ॑ज्य॒न्ता सृ॑ज्यन्त प॒त्त ए॑कविꣳ॒॒शम् । \newline
47. प॒त्त ए॑कविꣳ॒॒श मे॑कविꣳ॒॒शम् प॒त्तः प॒त्त ए॑कविꣳ॒॒शन् निर् णिरे॑कविꣳ॒॒शम् प॒त्तः प॒त्त ए॑कविꣳ॒॒शन् निः । \newline
48. ए॒क॒विꣳ॒॒शन् निर् णिरे॑कविꣳ॒॒श मे॑कविꣳ॒॒शन् निर॑मिमीता मिमीत॒ निरे॑कविꣳ॒॒श मे॑कविꣳ॒॒शन् निर॑मिमीत । \newline
49. ए॒क॒विꣳ॒॒शमित्ये॑क - विꣳ॒॒शम् । \newline
50. निर॑मिमीता मिमीत॒ निर् णिर॑मिमीत॒ तम् त म॑मिमीत॒ निर् णिर॑मिमीत॒ तम् । \newline
51. अ॒मि॒मी॒त॒ तम् त म॑मिमीता मिमीत॒ त म॑नु॒ष्टु ब॑नु॒ष्टुप् त म॑मिमीता मिमीत॒ त म॑नु॒ष्टुप् । \newline
52. त म॑नु॒ष्टु ब॑नु॒ष्टुप् तम् त म॑नु॒ष्टुप् छन्द॒ श्छन्दो॑ ऽनु॒ष्टुप् तम् त म॑नु॒ष्टुप् छन्दः॑ । \newline
53. अ॒नु॒ष्टुप् छन्द॒ श्छन्दो॑ ऽनु॒ष्टु ब॑नु॒ष्टुप् छन्दो ऽन्वनु॒ च्छन्दो॑ ऽनु॒ष्टु ब॑नु॒ष्टुप् छन्दो ऽनु॑ । \newline
54. अ॒नु॒ष्टुबित्य॑नु - स्तुप् । \newline
55. छन्दो ऽन्वनु॒ च्छन्द॒ श्छन्दो ऽन्व॑सृज्यता सृज्य॒तानु॒ च्छन्द॒ श्छन्दो ऽन्व॑सृज्यत । \newline
\pagebreak
\markright{ TS 7.1.1.6  \hfill https://www.vedavms.in \hfill}

\section{ TS 7.1.1.6 }

\textbf{TS 7.1.1.6 } \newline
\textbf{Samhita Paata} \newline

ऽन्व॑सृज्यत वैरा॒जꣳ साम॑ शू॒द्रो म॑नु॒ष्या॑णा॒मश्वः॑ पशू॒नां तस्मा॒त् तौ भू॑तसं क्रा॒मिणा॒वश्व॑श्च शू॒द्रश्च॒ तस्मा᳚च्छू॒द्रो य॒ज्ञेऽन॑वक्लृप्तो॒ न हि दे॒वता॒ अन्वसृ॑ज्यत॒ तस्मा॒त् पादा॒वुप॑ जीवतः प॒त्तो ह्यसृ॑ज्येतां प्रा॒णा वै त्रि॒वृद॑र्द्धमा॒साः प॑ञ्चद॒शः प्र॒जाप॑तिः सप्तद॒शस्त्रय॑ इ॒मे लो॒का अ॒सावा॑दि॒त्य ए॑कविꣳ॒॒श ए॒तस्मि॒न् वा ए॒ते श्रि॒ता ए॒तस्मि॒न् प्रति॑ष्ठिता॒ ( ) य ए॒वं ॅवेदै॒तस्मि॑न्ने॒व श्र॑यत ए॒तस्मि॒न् प्रति॑ तिष्ठति ॥ \newline

\textbf{Pada Paata} \newline

अन्विति॑ । अ॒सृ॒ज्य॒त॒ । वै॒रा॒जम् । साम॑ । शू॒द्रः । म॒नु॒ष्या॑णाम् । अश्वः॑ । प॒शू॒नाम् । तस्मा᳚त् । तौ । भू॒त॒स॒ङ्क्रा॒मिणा॒विति॑ भूत - स॒ङ्क्रा॒मिणौ᳚ । अश्वः॑ । च॒ । शू॒द्रः । च॒ । तस्मा᳚त् । शू॒द्रः । य॒ज्ञे । अन॑वक्लृप्त॒ इत्यन॑व - क्लृ॒प्तः॒ । न । हि । दे॒वताः᳚ । अन्विति॑ । असृ॑ज्यत । तस्मा᳚त् । पादौ᳚ । उपेति॑ । जी॒व॒तः॒ । प॒त्तः । हि । असृ॑ज्येताम् । प्रा॒णा इति॑ प्र - अ॒नाः । वै । त्रि॒वृदिति॑ त्रि - वृत् । अ॒द्‌र्ध॒मा॒सा इत्य॑द्‌र्ध - मा॒साः । प॒ञ्च॒द॒श इति॑ पञ्च - द॒शः । प्र॒जाप॑ति॒रिति॑ प्र॒जा - प॒तिः॒ । स॒प्त॒द॒श इति॑ सप्त - द॒शः । त्रयः॑ । इ॒मे । लो॒काः । अ॒सौ । आ॒दि॒त्यः । ए॒क॒विꣳ॒॒श इत्ये॑क - विꣳ॒॒शः । ए॒तस्मिन्न्॑ । वै । ए॒ते । श्रि॒ताः । ए॒तस्मिन्न्॑ । प्रति॑ष्ठिता॒ इति॒ प्रति॑ - स्थि॒ताः॒ ( ) । यः । ए॒वम् । वेद॑ । ए॒तस्मिन्न्॑ । ए॒व । श्र॒य॒ते॒ । ए॒तस्मिन्न्॑ । प्रतीति॑ । ति॒ष्ठ॒ति॒ ॥  \newline


\textbf{Krama Paata} \newline

अन्व॑सृज्यत । अ॒सृ॒ज्य॒त॒ वै॒रा॒जम् । वै॒रा॒जꣳ साम॑ । साम॑ शू॒द्रः । शू॒द्रो म॑नु॒ष्या॑णाम् । म॒नु॒ष्या॑णा॒मश्वः॑ । अश्वः॑ पशू॒नाम् । प॒शू॒नाम् तस्मा᳚त् । तस्मा॒त् तौ । तौ भू॑तसङ्‍क्रा॒मिणौ᳚ । भू॒त॒स॒ङ्‍क्रा॒मिणा॒वश्वः॑ । भू॒त॒स॒ङ्‍क्रा॒मिणा॒विति॑ भूत - स॒ङ्‍क्रा॒मिणौ᳚ । अश्व॑श्च । च॒ शू॒द्रः । शू॒द्रश्च॑ । च॒ तस्मा᳚त् । तस्मा᳚च्छू॒द्रः । शू॒द्रो य॒ज्ञे । य॒ज्ञेऽन॑वक्लृप्तः । अन॑वक्लृप्तो॒ न । अन॑वक्लृप्त॒ इत्यन॑व - क्लृ॒प्तः॒ । न हि । हि दे॒वताः᳚ । दे॒वता॒ अनु॑ । अन्वसृ॑ज्यत । असृ॑ज्यत॒ तस्मा᳚त् । तस्मा॒त् पादौ᳚ । पादा॒वुप॑ । उप॑ जीवतः । जी॒व॒तः॒ प॒त्तः । प॒त्तो हि । ह्यसृ॑ज्येताम् । असृ॑ज्येताम् प्रा॒णाः । प्रा॒णा वै । प्रा॒णा इति॑ प्र - अ॒नाः । वै त्रि॒वृत् । त्रि॒वृद॑र्द्धमा॒साः । त्रि॒वृदिति॑ त्रि - वृत् । अ॒र्द्ध॒मा॒साः प॑ञ्चद॒शः । अ॒र्द्ध॒मा॒सा इत्य॑र्द्ध - मा॒साः । प॒ञ्च॒द॒शः प्र॒जाप॑तिः । प॒ञ्च॒द॒श इति॑ पञ्च - द॒शः । प्र॒जाप॑तिः सप्तद॒शः । प्र॒जाप॑ति॒रिति॑ प्र॒जा - प॒तिः॒ । स॒प्त॒द॒शस्त्रयः॑ । स॒प्त॒द॒श इति॑ सप्त - द॒शः । त्रय॑ इ॒मे । इ॒मे लो॒काः । लो॒का अ॒सौ । अ॒सावा॑दि॒त्यः । आ॒दि॒त्य ए॑कविꣳ॒॒शः । ए॒क॒विꣳ॒॒श ए॒तस्मिन्न्॑ । ए॒क॒विꣳ॒॒श इत्ये॑क - विꣳ॒॒शः । ए॒तस्मि॒न् वै । वा ए॒ते । ए॒ते श्रि॒ताः । श्रि॒ता ए॒तस्मिन्न्॑ । ए॒तस्मि॒न् प्रति॑ष्ठिताः ( ) । प्रति॑ष्ठिता॒ यः । प्रति॑ष्ठिता॒ इति॒ प्रति॑ - स्थि॒ताः॒ । य ए॒वम् । ए॒वम् ॅवेद॑ । वेदै॒तस्मिन्न्॑ । ए॒तस्मि॑न्ने॒व । ए॒व श्र॑यते । श्र॒य॒त॒ ए॒तस्मिन्न्॑ । ए॒तस्मि॒न् प्रति॑ । प्रति॑ तिष्ठति । ति॒ष्ठ॒तीति॑ तिष्ठति । \newline

\textbf{Jatai Paata} \newline

1. अन्व॑ सृज्यता सृज्य॒तान् वन् व॑सृज्यत । \newline
2. अ॒सृ॒ज्य॒त॒ वै॒रा॒जं ॅवै॑रा॒ज म॑सृज्यता सृज्यत वैरा॒जम् । \newline
3. वै॒रा॒जꣳ साम॒ साम॑ वैरा॒जं ॅवै॑रा॒जꣳ साम॑ । \newline
4. साम॑ शू॒द्रः शू॒द्रः साम॒ साम॑ शू॒द्रः । \newline
5. शू॒द्रो म॑नु॒ष्या॑णाम् मनु॒ष्या॑णाꣳ शू॒द्रः शू॒द्रो म॑नु॒ष्या॑णाम् । \newline
6. म॒नु॒ष्या॑णा॒ मश्वो ऽश्वो॑ मनु॒ष्या॑णाम् मनु॒ष्या॑णा॒ मश्वः॑ । \newline
7. अश्वः॑ पशू॒नाम् प॑शू॒ना मश्वो ऽश्वः॑ पशू॒नाम् । \newline
8. प॒शू॒नाम् तस्मा॒त् तस्मा᳚त् पशू॒नाम् प॑शू॒नाम् तस्मा᳚त् । \newline
9. तस्मा॒त् तौ तौ तस्मा॒त् तस्मा॒त् तौ । \newline
10. तौ भू॑तसङ्क्रा॒मिणौ॑ भूतसङ्क्रा॒मिणौ॒ तौ तौ भू॑तसङ्क्रा॒मिणौ᳚ । \newline
11. भू॒त॒स॒ङ्क्रा॒मिणा॒ वश्वो ऽश्वो॑ भूतसङ्क्रा॒मिणौ॑ भूतसङ्क्रा॒मिणा॒ वश्वः॑ । \newline
12. भू॒त॒स॒ङ्क्रा॒मिणा॒विति॑ भूत - स॒ङ्क्रा॒मिणौ᳚ । \newline
13. अश्व॑ श्च॒ चाश्वो ऽश्व॑ श्च । \newline
14. च॒ शू॒द्रः शू॒द्र श्च॑ च शू॒द्रः । \newline
15. शू॒द्र श्च॑ च शू॒द्रः शू॒द्र श्च॑ । \newline
16. च॒ तस्मा॒त् तस्मा᳚च् च च॒ तस्मा᳚त् । \newline
17. तस्मा᳚च् छू॒द्रः शू॒द्र स्तस्मा॒त् तस्मा᳚च् छू॒द्रः । \newline
18. शू॒द्रो य॒ज्ञे य॒ज्ञे शू॒द्रः शू॒द्रो य॒ज्ञे । \newline
19. य॒ज्ञे ऽन॑वक्लृ॒प्तो ऽन॑वक्लृप्तो य॒ज्ञे य॒ज्ञे ऽन॑वक्लृप्तः । \newline
20. अन॑वक्लृप्तो॒ न नान॑वक्लृ॒प्तो ऽन॑वक्लृप्तो॒ न । \newline
21. अन॑वक्लृप्त॒ इत्यन॑व - क्लृ॒प्तः॒ । \newline
22. न हि हि न न हि । \newline
23. हि दे॒वता॑ दे॒वता॒ हि हि दे॒वताः᳚ । \newline
24. दे॒वता॒ अन्वनु॑ दे॒वता॑ दे॒वता॒ अनु॑ । \newline
25. अन्व सृ॑ज्य॒ता सृ॑ज्य॒तान् वन् वसृ॑ज्यत । \newline
26. असृ॑ज्यत॒ तस्मा॒त् तस्मा॒ दसृ॑ज्य॒ता सृ॑ज्यत॒ तस्मा᳚त् । \newline
27. तस्मा॒त् पादौ॒ पादौ॒ तस्मा॒त् तस्मा॒त् पादौ᳚ । \newline
28. पादा॒ वुपोप॒ पादौ॒ पादा॒ वुप॑ । \newline
29. उप॑ जीवतो जीवत॒ उपोप॑ जीवतः । \newline
30. जी॒व॒तः॒ प॒त्तः प॒त्तो जी॑वतो जीवतः प॒त्तः । \newline
31. प॒त्तो हि हि प॒त्तः प॒त्तो हि । \newline
32. ह्यसृ॑ज्येता॒ मसृ॑ज्येताꣳ॒॒ हि ह्यसृ॑ज्येताम् । \newline
33. असृ॑ज्येताम् प्रा॒णाः प्रा॒णा असृ॑ज्येता॒ मसृ॑ज्येताम् प्रा॒णाः । \newline
34. प्रा॒णा वै वै प्रा॒णाः प्रा॒णा वै । \newline
35. प्रा॒णा इति॑ प्र - अ॒नाः । \newline
36. वै त्रि॒वृत् त्रि॒वृद् वै वै त्रि॒वृत् । \newline
37. त्रि॒वृ द॑र्द्धमा॒सा अ॑र्द्धमा॒सा स्त्रि॒वृत् त्रि॒वृ द॑र्द्धमा॒साः । \newline
38. त्रि॒वृदिति॑ त्रि - वृत् । \newline
39. अ॒र्द्ध॒मा॒साः प॑ञ्चद॒शः प॑ञ्चद॒शो᳚ ऽर्द्धमा॒सा अ॑र्द्धमा॒साः प॑ञ्चद॒शः । \newline
40. अ॒र्द्ध॒मा॒सा इत्य॑र्द्ध - मा॒साः । \newline
41. प॒ञ्च॒द॒शः प्र॒जाप॑तिः प्र॒जाप॑तिः पञ्चद॒शः प॑ञ्चद॒शः प्र॒जाप॑तिः । \newline
42. प॒ञ्च॒द॒श इति॑ पञ्च - द॒शः । \newline
43. प्र॒जाप॑तिः सप्तद॒शः स॑प्तद॒शः प्र॒जाप॑तिः प्र॒जाप॑तिः सप्तद॒शः । \newline
44. प्र॒जाप॑ति॒रिति॑ प्र॒जा - प॒तिः॒ । \newline
45. स॒प्त॒द॒श स्त्रय॒ स्त्रयः॑ सप्तद॒शः स॑प्तद॒श स्त्रयः॑ । \newline
46. स॒प्त॒द॒श इति॑ सप्त - द॒शः । \newline
47. त्रय॑ इ॒म इ॒मे त्रय॒ स्त्रय॑ इ॒मे । \newline
48. इ॒मे लो॒का लो॒का इ॒म इ॒मे लो॒काः । \newline
49. लो॒का अ॒सा व॒सौ लो॒का लो॒का अ॒सौ । \newline
50. अ॒सा वा॑दि॒त्य आ॑दि॒त्यो॑ ऽसा व॒सा वा॑दि॒त्यः । \newline
51. आ॒दि॒त्य ए॑कविꣳ॒॒श ए॑कविꣳ॒॒श आ॑दि॒त्य आ॑दि॒त्य ए॑कविꣳ॒॒शः । \newline
52. ए॒क॒विꣳ॒॒श ए॒तस्मि॑न् ने॒तस्मि॑न् नेकविꣳ॒॒श ए॑कविꣳ॒॒श ए॒तस्मिन्न्॑ । \newline
53. ए॒क॒विꣳ॒॒श इत्ये॑क - विꣳ॒॒शः । \newline
54. ए॒तस्मि॒न्॒. वै वा ए॒तस्मि॑न् ने॒तस्मि॒न्॒. वै । \newline
55. वा ए॒त ए॒ते वै वा ए॒ते । \newline
56. ए॒ते श्रि॒ताः श्रि॒ता ए॒त ए॒ते श्रि॒ताः । \newline
57. श्रि॒ता ए॒तस्मि॑न् ने॒तस्मि॑ञ् छ्रि॒ताः श्रि॒ता ए॒तस्मिन्न्॑ । \newline
58. ए॒तस्मि॒न् प्रति॑ष्ठिताः॒ प्रति॑ष्ठिता ए॒तस्मि॑न् ने॒तस्मि॒न् प्रति॑ष्ठिताः । \newline
59. प्रति॑ष्ठिता॒ यो यः प्रति॑ष्ठिताः॒ प्रति॑ष्ठिता॒ यः । \newline
60. प्रति॑ष्ठिता॒ इति॒ प्रति॑ - स्थि॒ताः॒ । \newline
61. य ए॒व मे॒वं ॅयो य ए॒वम् । \newline
62. ए॒वं ॅवेद॒ वेदै॒व मे॒वं ॅवेद॑ । \newline
63. वेदै॒ तस्मि॑न् ने॒तस्मि॒न्॒. वेद॒ वेदै॒ तस्मिन्न्॑ । \newline
64. ए॒तस्मि॑न् ने॒वै वैतस्मि॑न् ने॒तस्मि॑न् ने॒व । \newline
65. ए॒व श्र॑यते श्रयत ए॒वैव श्र॑यते । \newline
66. श्र॒य॒त॒ ए॒तस्मि॑न् ने॒तस्मि॑ञ् छ्रयते श्रयत ए॒तस्मिन्न्॑ । \newline
67. ए॒तस्मि॒न् प्रति॒ प्रत्ये॒तस्मि॑न् ने॒तस्मि॒न् प्रति॑ । \newline
68. प्रति॑ तिष्ठति तिष्ठति॒ प्रति॒ प्रति॑ तिष्ठति । \newline
69. ति॒ष्ठ॒तीति॑ तिष्ठति । \newline

\textbf{Ghana Paata } \newline

1. अन्व॑ सृज्यता सृज्य॒तान् वन् व॑सृज्यत वैरा॒जं ॅवै॑रा॒ज म॑सृज्य॒तान् वन् व॑सृज्यत वैरा॒जम् । \newline
2. अ॒सृ॒ज्य॒त॒ वै॒रा॒जं ॅवै॑रा॒ज म॑सृज्यता सृज्यत वैरा॒जꣳ साम॒ साम॑ वैरा॒ज म॑सृज्यता सृज्यत वैरा॒जꣳ साम॑ । \newline
3. वै॒रा॒जꣳ साम॒ साम॑ वैरा॒जं ॅवै॑रा॒जꣳ साम॑ शू॒द्रः शू॒द्रः साम॑ वैरा॒जं ॅवै॑रा॒जꣳ साम॑ शू॒द्रः । \newline
4. साम॑ शू॒द्रः शू॒द्रः साम॒ साम॑ शू॒द्रो म॑नु॒ष्या॑णाम् मनु॒ष्या॑णाꣳ शू॒द्रः साम॒ साम॑ शू॒द्रो म॑नु॒ष्या॑णाम् । \newline
5. शू॒द्रो म॑नु॒ष्या॑णाम् मनु॒ष्या॑णाꣳ शू॒द्रः शू॒द्रो म॑नु॒ष्या॑णा॒ मश्वो ऽश्वो॑ मनु॒ष्या॑णाꣳ शू॒द्रः शू॒द्रो म॑नु॒ष्या॑णा॒ मश्वः॑ । \newline
6. म॒नु॒ष्या॑णा॒ मश्वो ऽश्वो॑ मनु॒ष्या॑णाम् मनु॒ष्या॑णा॒ मश्वः॑ पशू॒नाम् प॑शू॒ना मश्वो॑ मनु॒ष्या॑णाम् मनु॒ष्या॑णा॒ मश्वः॑ पशू॒नाम् । \newline
7. अश्वः॑ पशू॒नाम् प॑शू॒ना मश्वो ऽश्वः॑ पशू॒नाम् तस्मा॒त् तस्मा᳚त् पशू॒ना मश्वो ऽश्वः॑ पशू॒नाम् तस्मा᳚त् । \newline
8. प॒शू॒नाम् तस्मा॒त् तस्मा᳚त् पशू॒नाम् प॑शू॒नाम् तस्मा॒त् तौ तौ तस्मा᳚त् पशू॒नाम् प॑शू॒नाम् तस्मा॒त् तौ । \newline
9. तस्मा॒त् तौ तौ तस्मा॒त् तस्मा॒त् तौ भू॑तसङ्क्रा॒मिणौ॑ भूतसङ्क्रा॒मिणौ॒ तौ तस्मा॒त् तस्मा॒त् तौ भू॑तसङ्क्रा॒मिणौ᳚ । \newline
10. तौ भू॑तसङ्क्रा॒मिणौ॑ भूतसङ्क्रा॒मिणौ॒ तौ तौ भू॑तसङ्क्रा॒मिणा॒ वश्वो ऽश्वो॑ भूतसङ्क्रा॒मिणौ॒ तौ तौ भू॑तसङ्क्रा॒मिणा॒ वश्वः॑ । \newline
11. भू॒त॒स॒ङ्क्रा॒मिणा॒ वश्वो ऽश्वो॑ भूतसङ्क्रा॒मिणौ॑ भूतसङ्क्रा॒मिणा॒ वश्व॑ श्च॒ चाश्वो॑ भूतसङ्क्रा॒मिणौ॑ भूतसङ्क्रा॒मिणा॒ वश्व॑ श्च । \newline
12. भू॒त॒स॒ङ्क्रा॒मिणा॒विति॑ भूत - स॒ङ्क्रा॒मिणौ᳚ । \newline
13. अश्व॑ श्च॒ चाश्वो ऽश्व॑ श्च शू॒द्रः शू॒द्र श्चा श्वो ऽश्व॑ श्च शू॒द्रः । \newline
14. च॒ शू॒द्रः शू॒द्र श्च॑ च शू॒द्र श्च॑ च शू॒द्र श्च॑ च शू॒द्र श्च॑ । \newline
15. शू॒द्र श्च॑ च शू॒द्रः शू॒द्र श्च॒ तस्मा॒त् तस्मा᳚च् च शू॒द्रः शू॒द्र श्च॒ तस्मा᳚त् । \newline
16. च॒ तस्मा॒त् तस्मा᳚च् च च॒ तस्मा᳚च् छू॒द्रः शू॒द्र स्तस्मा᳚च् च च॒ तस्मा᳚च् छू॒द्रः । \newline
17. तस्मा᳚च् छू॒द्रः शू॒द्र स्तस्मा॒त् तस्मा᳚च् छू॒द्रो य॒ज्ञे य॒ज्ञे शू॒द्र स्तस्मा॒त् तस्मा᳚च् छू॒द्रो य॒ज्ञे । \newline
18. शू॒द्रो य॒ज्ञे य॒ज्ञे शू॒द्रः शू॒द्रो य॒ज्ञे ऽन॑वक्लृ॒प्तो ऽन॑वक्लृप्तो य॒ज्ञे शू॒द्रः शू॒द्रो य॒ज्ञे ऽन॑वक्लृप्तः । \newline
19. य॒ज्ञे ऽन॑वक्लृ॒प्तो ऽन॑वक्लृप्तो य॒ज्ञे य॒ज्ञे ऽन॑वक्लृप्तो॒ न नान॑वक्लृप्तो य॒ज्ञे य॒ज्ञे ऽन॑वक्लृप्तो॒ न । \newline
20. अन॑वक्लृप्तो॒ न नान॑वक्लृ॒प्तो ऽन॑वक्लृप्तो॒ न हि हि नान॑वक्लृ॒प्तो ऽन॑वक्लृप्तो॒ न हि । \newline
21. अन॑वक्लृप्त॒ इत्यन॑व - क्लृ॒प्तः॒ । \newline
22. न हि हि न न हि दे॒वता॑ दे॒वता॒ हि न न हि दे॒वताः᳚ । \newline
23. हि दे॒वता॑ दे॒वता॒ हि हि दे॒वता॒ अन्वनु॑ दे॒वता॒ हि हि दे॒वता॒ अनु॑ । \newline
24. दे॒वता॒ अन्वनु॑ दे॒वता॑ दे॒वता॒ अन्वसृ॑ज्य॒ता सृ॑ज्य॒तानु॑ दे॒वता॑ दे॒वता॒ अन्वसृ॑ज्यत । \newline
25. अन्वसृ॑ज्य॒ता सृ॑ज्य॒तान् वन् वसृ॑ज्यत॒ तस्मा॒त् तस्मा॒ दसृ॑ज्य॒तान् वन् वसृ॑ज्यत॒ तस्मा᳚त् । \newline
26. असृ॑ज्यत॒ तस्मा॒त् तस्मा॒ दसृ॑ज्य॒ता सृ॑ज्यत॒ तस्मा॒त् पादौ॒ पादौ॒ तस्मा॒ दसृ॑ज्य॒ता सृ॑ज्यत॒ तस्मा॒त् पादौ᳚ । \newline
27. तस्मा॒त् पादौ॒ पादौ॒ तस्मा॒त् तस्मा॒त् पादा॒ वुपोप॒ पादौ॒ तस्मा॒त् तस्मा॒त् पादा॒ वुप॑ । \newline
28. पादा॒ वुपोप॒ पादौ॒ पादा॒ वुप॑ जीवतो जीवत॒ उप॒ पादौ॒ पादा॒ वुप॑ जीवतः । \newline
29. उप॑ जीवतो जीवत॒ उपोप॑ जीवतः प॒त्तः प॒त्तो जी॑वत॒ उपोप॑ जीवतः प॒त्तः । \newline
30. जी॒व॒तः॒ प॒त्तः प॒त्तो जी॑वतो जीवतः प॒त्तो हि हि प॒त्तो जी॑वतो जीवतः प॒त्तो हि । \newline
31. प॒त्तो हि हि प॒त्तः प॒त्तो ह्यसृ॑ज्येता॒ मसृ॑ज्येताꣳ॒॒ हि प॒त्तः प॒त्तो ह्यसृ॑ज्येताम् । \newline
32. ह्यसृ॑ज्येता॒ मसृ॑ज्येताꣳ॒॒ हि ह्यसृ॑ज्येताम् प्रा॒णाः प्रा॒णा असृ॑ज्येताꣳ॒॒ हि ह्यसृ॑ज्येताम् प्रा॒णाः । \newline
33. असृ॑ज्येताम् प्रा॒णाः प्रा॒णा असृ॑ज्येता॒ मसृ॑ज्येताम् प्रा॒णा वै वै प्रा॒णा असृ॑ज्येता॒ मसृ॑ज्येताम् प्रा॒णा वै । \newline
34. प्रा॒णा वै वै प्रा॒णाः प्रा॒णा वै त्रि॒वृत् त्रि॒वृद् वै प्रा॒णाः प्रा॒णा वै त्रि॒वृत् । \newline
35. प्रा॒णा इति॑ प्र - अ॒नाः । \newline
36. वै त्रि॒वृत् त्रि॒वृद् वै वै त्रि॒वृ द॑र्द्धमा॒सा अ॑र्द्धमा॒सा स्त्रि॒वृद् वै वै त्रि॒वृ द॑र्द्धमा॒साः । \newline
37. त्रि॒वृ द॑र्द्धमा॒सा अ॑र्द्धमा॒सा स्त्रि॒वृत् त्रि॒वृ द॑र्द्धमा॒साः प॑ञ्चद॒शः प॑ञ्चद॒शो᳚ ऽर्द्धमा॒सा स्त्रि॒वृत् त्रि॒वृ द॑र्द्धमा॒साः प॑ञ्चद॒शः । \newline
38. त्रि॒वृदिति॑ त्रि - वृत् । \newline
39. अ॒र्द्ध॒मा॒साः प॑ञ्चद॒शः प॑ञ्चद॒शो᳚ ऽर्द्धमा॒सा अ॑र्द्धमा॒साः प॑ञ्चद॒शः प्र॒जाप॑तिः प्र॒जाप॑तिः पञ्चद॒शो᳚ ऽर्द्धमा॒सा अ॑र्द्धमा॒साः प॑ञ्चद॒शः प्र॒जाप॑तिः । \newline
40. अ॒र्द्ध॒मा॒सा इत्य॑र्द्ध - मा॒साः । \newline
41. प॒ञ्च॒द॒शः प्र॒जाप॑तिः प्र॒जाप॑तिः पञ्चद॒शः प॑ञ्चद॒शः प्र॒जाप॑तिः सप्तद॒शः स॑प्तद॒शः प्र॒जाप॑तिः पञ्चद॒शः प॑ञ्चद॒शः प्र॒जाप॑तिः सप्तद॒शः । \newline
42. प॒ञ्च॒द॒श इति॑ पञ्च - द॒शः । \newline
43. प्र॒जाप॑तिः सप्तद॒शः स॑प्तद॒शः प्र॒जाप॑तिः प्र॒जाप॑तिः सप्तद॒श स्त्रय॒ स्त्रयः॑ सप्तद॒शः प्र॒जाप॑तिः प्र॒जाप॑तिः सप्तद॒श स्त्रयः॑ । \newline
44. प्र॒जाप॑ति॒रिति॑ प्र॒जा - प॒तिः॒ । \newline
45. स॒प्त॒द॒श स्त्रय॒ स्त्रयः॑ सप्तद॒शः स॑प्तद॒श स्त्रय॑ इ॒म इ॒मे त्रयः॑ सप्तद॒शः स॑प्तद॒श स्त्रय॑ इ॒मे । \newline
46. स॒प्त॒द॒श इति॑ सप्त - द॒शः । \newline
47. त्रय॑ इ॒म इ॒मे त्रय॒ स्त्रय॑ इ॒मे लो॒का लो॒का इ॒मे त्रय॒ स्त्रय॑ इ॒मे लो॒काः । \newline
48. इ॒मे लो॒का लो॒का इ॒म इ॒मे लो॒का अ॒सा व॒सौ लो॒का इ॒म इ॒मे लो॒का अ॒सौ । \newline
49. लो॒का अ॒सा व॒सौ लो॒का लो॒का अ॒सा वा॑दि॒त्य आ॑दि॒त्यो॑ ऽसौ लो॒का लो॒का अ॒सा वा॑दि॒त्यः । \newline
50. अ॒सा वा॑दि॒त्य आ॑दि॒त्यो॑ ऽसा व॒सा वा॑दि॒त्य ए॑कविꣳ॒॒श ए॑कविꣳ॒॒श आ॑दि॒त्यो॑ ऽसा व॒सा वा॑दि॒त्य ए॑कविꣳ॒॒शः । \newline
51. आ॒दि॒त्य ए॑कविꣳ॒॒श ए॑कविꣳ॒॒श आ॑दि॒त्य आ॑दि॒त्य ए॑कविꣳ॒॒श ए॒तस्मि॑न् ने॒तस्मि॑न् नेकविꣳ॒॒श आ॑दि॒त्य आ॑दि॒त्य ए॑कविꣳ॒॒श ए॒तस्मिन्न्॑ । \newline
52. ए॒क॒विꣳ॒॒श ए॒तस्मि॑न् ने॒तस्मि॑न् नेकविꣳ॒॒श ए॑कविꣳ॒॒श ए॒तस्मि॒न्॒. वै वा ए॒तस्मि॑न् नेकविꣳ॒॒श ए॑कविꣳ॒॒श ए॒तस्मि॒न्॒. वै । \newline
53. ए॒क॒विꣳ॒॒श इत्ये॑क - विꣳ॒॒शः । \newline
54. ए॒तस्मि॒न्॒. वै वा ए॒तस्मि॑न् ने॒तस्मि॒न्॒. वा ए॒त ए॒ते वा ए॒तस्मि॑न् ने॒तस्मि॒न्॒. वा ए॒ते । \newline
55. वा ए॒त ए॒ते वै वा ए॒ते श्रि॒ताः श्रि॒ता ए॒ते वै वा ए॒ते श्रि॒ताः । \newline
56. ए॒ते श्रि॒ताः श्रि॒ता ए॒त ए॒ते श्रि॒ता ए॒तस्मि॑न् ने॒तस्मि॑ञ् छ्रि॒ता ए॒त ए॒ते श्रि॒ता ए॒तस्मिन्न्॑ । \newline
57. श्रि॒ता ए॒तस्मि॑न् ने॒तस्मि॑ञ् छ्रि॒ताः श्रि॒ता ए॒तस्मि॒न् प्रति॑ष्ठिताः॒ प्रति॑ष्ठिता ए॒तस्मि॑ञ् छ्रि॒ताः श्रि॒ता ए॒तस्मि॒न् प्रति॑ष्ठिताः । \newline
58. ए॒तस्मि॒न् प्रति॑ष्ठिताः॒ प्रति॑ष्ठिता ए॒तस्मि॑न् ने॒तस्मि॒न् प्रति॑ष्ठिता॒ यो यः प्रति॑ष्ठिता ए॒तस्मि॑न् ने॒तस्मि॒न् प्रति॑ष्ठिता॒ यः । \newline
59. प्रति॑ष्ठिता॒ यो यः प्रति॑ष्ठिताः॒ प्रति॑ष्ठिता॒ य ए॒व मे॒वं ॅयः प्रति॑ष्ठिताः॒ प्रति॑ष्ठिता॒ य ए॒वम् । \newline
60. प्रति॑ष्ठिता॒ इति॒ प्रति॑ - स्थि॒ताः॒ । \newline
61. य ए॒व मे॒वं ॅयो य ए॒वं ॅवेद॒ वेदै॒वं ॅयो य ए॒वं ॅवेद॑ । \newline
62. ए॒वं ॅवेद॒ वेदै॒व मे॒वं ॅवेदै॒तस्मि॑न् ने॒तस्मि॒न्॒. वेदै॒व मे॒वं ॅवेदै॒तस्मिन्न्॑ । \newline
63. वेदै॒तस्मि॑न् ने॒तस्मि॒न्॒. वेद॒ वेदै॒तस्मि॑न् ने॒वैवै तस्मि॒न्॒. वेद॒ वेदै॒तस्मि॑न् ने॒व । \newline
64. ए॒तस्मि॑न् ने॒वैवै तस्मि॑न् ने॒तस्मि॑न् ने॒व श्र॑यते श्रयत ए॒वैतस्मि॑न् ने॒तस्मि॑न् ने॒व श्र॑यते । \newline
65. ए॒व श्र॑यते श्रयत ए॒वैव श्र॑यत ए॒तस्मि॑न् ने॒तस्मि॑ञ् छ्रयत ए॒वैव श्र॑यत ए॒तस्मिन्न्॑ । \newline
66. श्र॒य॒त॒ ए॒तस्मि॑न् ने॒तस्मि॑ञ् छ्रयते श्रयत ए॒तस्मि॒न् प्रति॒ प्रत्ये॒तस्मि॑ञ् छ्रयते श्रयत ए॒तस्मि॒न् प्रति॑ । \newline
67. ए॒तस्मि॒न् प्रति॒ प्रत्ये॒तस्मि॑न् ने॒तस्मि॒न् प्रति॑ तिष्ठति तिष्ठति॒ प्रत्ये॒तस्मि॑न् ने॒तस्मि॒न् प्रति॑ तिष्ठति । \newline
68. प्रति॑ तिष्ठति तिष्ठति॒ प्रति॒ प्रति॑ तिष्ठति । \newline
69. ति॒ष्ठ॒तीति॑ तिष्ठति । \newline
\pagebreak
\markright{ TS 7.1.2.1  \hfill https://www.vedavms.in \hfill}

\section{ TS 7.1.2.1 }

\textbf{TS 7.1.2.1 } \newline
\textbf{Samhita Paata} \newline

प्रा॒त॒स्स॒व॒ने वै गा॑य॒त्रेण॒ छन्द॑सा त्रि॒वृते॒ स्तोमा॑य॒ ज्योति॒र्दध॑देति त्रि॒वृता᳚ ब्रह्मवर्च॒सेन॑ पञ्चद॒शाय॒ ज्योति॒र्दध॑देति पञ्चद॒शेनौज॑सा वी॒र्ये॑ण सप्तद॒शाय॒ ज्योति॒र्दध॑देति सप्तद॒शेन॑ प्राजाप॒त्येन॑ प्र॒जन॑नेनैकविꣳ॒॒शाय॒ ज्योति॒र्दध॑देति॒ स्तोम॑ ए॒व तथ् स्तोमा॑य॒ ज्योति॒र्दध॑दे॒त्यथो॒ स्तोम॑ ए॒व स्तोम॑म॒भि प्र ण॑यति॒ याव॑न्तो॒ वै स्तोमा॒स्ताव॑न्तः॒ कामा॒स्ताव॑न्तो लो॒का ( ) -स्ताव॑न्ति॒ ज्योतीꣳ॑ष्ये॒ताव॑त ए॒व स्तोमा॑ने॒ताव॑तः॒ कामा॑ने॒ताव॑तो लो॒काने॒ताव॑न्ति॒ ज्योतीꣳ॒॒ष्यव॑ रुन्धे ॥ \newline

\textbf{Pada Paata} \newline

प्रा॒त॒स्स॒व॒न इति॑ प्रातः - स॒व॒ने । वै । गा॒य॒त्रेण॑ । छन्द॑सा । त्रि॒वृत॒ इति॑ त्रि - वृते᳚ । स्तोमा॑य । ज्योतिः॑ । दध॑त् । ए॒ति॒ । त्रि॒वृतेति॑ त्रि - वृता᳚ । ब्र॒ह्म॒व॒र्च॒सेनेति॑ ब्रह्म - व॒र्च॒सेन॑ । प॒ञ्च॒द॒शायेति॑ पञ्च - द॒शाय॑ । ज्योतिः॑ । दध॑त् । ए॒ति॒ । प॒ञ्च॒द॒शेनेति॑ पञ्च-द॒शेन॑ । ओज॑सा । वी॒र्ये॑ण । स॒प्त॒द॒शायेति॑ सप्त - द॒शाय॑ । ज्योतिः॑ । दध॑त् । ए॒ति॒ । स॒प्त॒द॒शेनेति॑ सप्त - द॒शेन॑ । प्रा॒जा॒प॒त्येनेति॑ प्राजा - प॒त्येन॑ । प्र॒जन॑ने॒नेति॑ प्र - जन॑नेन । ए॒क॒विꣳ॒॒शायेत्ये॑क - विꣳ॒॒शाय॑ । ज्योतिः॑ । दध॑त् । ए॒ति॒ । स्तोमः॑ । ए॒व । तत् । स्तोमा॑य । ज्योतिः॑ । दध॑त् । ए॒ति॒ । अथो॒ इति॑ । स्तोमे᳚ । ए॒व । स्तोम᳚म् । अ॒भि । प्रेति॑ । न॒य॒ति॒ । याव॑न्तः । वै । स्तोमाः᳚ । ताव॑न्तः । कामाः᳚ । ताव॑न्तः । लो॒काः ( ) । ताव॑न्ति । ज्योतीꣳ॑षि । ए॒ताव॑तः । ए॒व । स्तोमान्॑ । ए॒ताव॑तः । कामान्॑ । ए॒ताव॑तः । लो॒कान् । ए॒ताव॑न्ति । ज्योतीꣳ॑षि । अवेति॑ । रु॒न्धे॒ ॥  \newline


\textbf{Krama Paata} \newline

प्रा॒त॒स्स॒व॒ने वै । प्रा॒त॒स्स॒व॒न इति॑ प्रातः - स॒व॒ने । वै गा॑य॒त्रेण॑ । गा॒य॒त्रेण॒ छन्द॑सा । छन्द॑सा त्रि॒वृते᳚ । त्रि॒वृते॒ स्तोमा॑य । त्रि॒वृत॒ इति॑ त्रि - वृते᳚ । स्तोमा॑य॒ ज्योतिः॑ । ज्योति॒र् दध॑त् । दध॑देति । ए॒ति॒ त्रि॒वृता᳚ । त्रि॒वृता᳚ ब्रह्मवर्च॒सेन॑ । त्रि॒वृतेति॑ त्रि - वृता᳚ । ब्र॒ह्म॒व॒र्च॒सेन॑ पञ्चद॒शाय॑ । ब्र॒ह्म॒व॒र्च॒सेनेति॑ ब्रह्म - व॒र्च॒सेन॑ । प॒ञ्च॒द॒शाय॒ ज्योतिः॑ । प॒ञ्च॒द॒शायेति॑ पञ्च - द॒शाय॑ । ज्योति॒र् दध॑त् । दध॑देति । ए॒ति॒ प॒ञ्च॒द॒शेन॑ । प॒ञ्च॒द॒शेनौज॑सा । प॒ञ्च॒द॒शेनेति॑ पञ्च - द॒शेन॑ । ओज॑सा वी॒र्ये॑ण । वी॒र्ये॑ण सप्तद॒शाय॑ । स॒प्त॒द॒शाय॒ ज्योतिः॑ । स॒प्त॒द॒शायेति॑ सप्त - द॒शाय॑ । ज्योति॒र् दध॑त् । दध॑देति । ए॒ति॒ स॒प्त॒द॒शेन॑ । स॒प्त॒द॒शेन॑ प्राजाप॒त्येन॑ । स॒प्त॒द॒शेनेति॑ सप्त - द॒शेन॑ । प्रा॒जा॒प॒त्येन॑ प्र॒जन॑नेन । प्रा॒जा॒प॒त्येनेति॑ प्राजा - प॒त्येन॑ । प्र॒जन॑नेनैकविꣳ॒॒शाय॑ । प्र॒जन॑ने॒नेति॑ प्र - जन॑नेन । ए॒क॒विꣳ॒॒शाय॒ ज्योतिः॑ । ए॒क॒विꣳ॒॒शायेत्ये॑क - विꣳ॒॒शाय॑ । ज्योति॒र् दध॑त् । दध॑देति । ए॒ति॒ स्तोमः॑ । स्तोम॑ ए॒व । ए॒व तत् । तथ् स्तोमा॑य । स्तोमा॑य॒ ज्योतिः॑ । ज्योति॒र् दध॑त् । दध॑देति । ए॒त्यथो᳚ । अथो॒ स्तोमे᳚ । अथो॒ इत्यथो᳚ । स्तोम॑ ए॒व । ए॒व स्तोम᳚म् । स्तोम॑म॒भि । अ॒भि प्र । 
प्र ण॑यति । न॒य॒ति॒ याव॑न्तः । याव॑न्तो॒ वै । वै स्तोमाः᳚ । स्तोमा॒स्ताव॑न्तः । ताव॑न्तः॒ कामाः᳚ । कामा॒स्ताव॑न्तः । ताव॑न्तो लो॒काः ( ) । लो॒कास्ताव॑न्ति । ताव॑न्ति॒ ज्योतीꣳ॑षि । ज्योतीꣳ॑ष्ये॒ताव॑तः । ए॒ताव॑त ए॒व । ए॒व स्तोमान्॑ । स्तोमा॑ने॒ताव॑तः । ए॒ताव॑तः॒ कामान्॑ । कामा॑ने॒ताव॑तः । ए॒ताव॑तो लो॒कान् । लो॒काने॒ताव॑न्ति । ए॒ताव॑न्ति॒ ज्योतीꣳ॑षि । ज्योतीꣳ॒॒ष्यव॑ । अव॑ रुन्धे । रु॒न्ध॒ इति॑ रुन्धे । \newline

\textbf{Jatai Paata} \newline

1. प्रा॒त॒स्स॒व॒ने वै वै प्रा॑तस्सव॒ने प्रा॑तस्सव॒ने वै । \newline
2. प्रा॒त॒स्स॒व॒न इति॑ प्रातः - स॒व॒ने । \newline
3. वै गा॑य॒त्रेण॑ गाय॒त्रेण॒ वै वै गा॑य॒त्रेण॑ । \newline
4. गा॒य॒त्रेण॒ छन्द॑सा॒ छन्द॑सा गाय॒त्रेण॑ गाय॒त्रेण॒ छन्द॑सा । \newline
5. छन्द॑सा त्रि॒वृते᳚ त्रि॒वृते॒ छन्द॑सा॒ छन्द॑सा त्रि॒वृते᳚ । \newline
6. त्रि॒वृते॒ स्तोमा॑य॒ स्तोमा॑य त्रि॒वृते᳚ त्रि॒वृते॒ स्तोमा॑य । \newline
7. त्रि॒वृत॒ इति॑ त्रि - वृते᳚ । \newline
8. स्तोमा॑य॒ ज्योति॒र् ज्योतिः॒ स्तोमा॑य॒ स्तोमा॑य॒ ज्योतिः॑ । \newline
9. ज्योति॒र् दध॒द् दध॒ज् ज्योति॒र् ज्योति॒र् दध॑त् । \newline
10. दध॑ देत्येति॒ दध॒द् दध॑ देति । \newline
11. ए॒ति॒ त्रि॒वृता᳚ त्रि॒वृतै᳚ त्येति त्रि॒वृता᳚ । \newline
12. त्रि॒वृता᳚ ब्रह्मवर्च॒सेन॑ ब्रह्मवर्च॒सेन॑ त्रि॒वृता᳚ त्रि॒वृता᳚ ब्रह्मवर्च॒सेन॑ । \newline
13. त्रि॒वृतेति॑ त्रि - वृता᳚ । \newline
14. ब्र॒ह्म॒व॒र्च॒सेन॑ पञ्चद॒शाय॑ पञ्चद॒शाय॑ ब्रह्मवर्च॒सेन॑ ब्रह्मवर्च॒सेन॑ पञ्चद॒शाय॑ । \newline
15. ब्र॒ह्म॒व॒र्च॒सेनेति॑ ब्रह्म - व॒र्च॒सेन॑ । \newline
16. प॒ञ्च॒द॒शाय॒ ज्योति॒र् ज्योतिः॑ पञ्चद॒शाय॑ पञ्चद॒शाय॒ ज्योतिः॑ । \newline
17. प॒ञ्च॒द॒शायेति॑ पञ्च - द॒शाय॑ । \newline
18. ज्योति॒र् दध॒द् दध॒ज् ज्योति॒र् ज्योति॒र् दध॑त् । \newline
19. दध॑ देत्येति॒ दध॒द् दध॑ देति । \newline
20. ए॒ति॒ प॒ञ्च॒द॒शेन॑ पञ्चद॒शेनै᳚ त्येति पञ्चद॒शेन॑ । \newline
21. प॒ञ्च॒द॒शे नौज॒ सौज॑सा पञ्चद॒शेन॑ पञ्चद॒शे नौज॑सा । \newline
22. प॒ञ्च॒द॒शेनेति॑ पञ्च - द॒शेन॑ । \newline
23. ओज॑सा वी॒र्ये॑ण वी॒र्ये॑ णौज॒ सौज॑सा वी॒र्ये॑ण । \newline
24. वी॒र्ये॑ण सप्तद॒शाय॑ सप्तद॒शाय॑ वी॒र्ये॑ण वी॒र्ये॑ण सप्तद॒शाय॑ । \newline
25. स॒प्त॒द॒शाय॒ ज्योति॒र् ज्योतिः॑ सप्तद॒शाय॑ सप्तद॒शाय॒ ज्योतिः॑ । \newline
26. स॒प्त॒द॒शायेति॑ सप्त - द॒शाय॑ । \newline
27. ज्योति॒र् दध॒द् दध॒ज् ज्योति॒र् ज्योति॒र् दध॑त् । \newline
28. दध॑ देत्येति॒ दध॒द् दध॑ देति । \newline
29. ए॒ति॒ स॒प्त॒द॒शेन॑ सप्तद॒शे नै᳚त्येति सप्तद॒शेन॑ । \newline
30. स॒प्त॒द॒शेन॑ प्राजाप॒त्येन॑ प्राजाप॒त्येन॑ सप्तद॒शेन॑ सप्तद॒शेन॑ प्राजाप॒त्येन॑ । \newline
31. स॒प्त॒द॒शेनेति॑ सप्त - द॒शेन॑ । \newline
32. प्रा॒जा॒प॒त्येन॑ प्र॒जन॑नेन प्र॒जन॑नेन प्राजाप॒त्येन॑ प्राजाप॒त्येन॑ प्र॒जन॑नेन । \newline
33. प्रा॒जा॒प॒त्येनेति॑ प्राजा - प॒त्येन॑ । \newline
34. प्र॒जन॑ने नैक॑विꣳ॒॒शा यै॑कविꣳ॒॒शाय॑ प्र॒जन॑नेन प्र॒जन॑ने नैक॑विꣳ॒॒शाय॑ । \newline
35. प्र॒जन॑ने॒नेति॑ प्र - जन॑नेन । \newline
36. ए॒क॒विꣳ॒॒शाय॒ ज्योति॒र् ज्योति॑रेकविꣳ॒॒शा यै॑कविꣳ॒॒शाय॒ ज्योतिः॑ । \newline
37. ए॒क॒विꣳ॒॒शायेत्ये॑क - विꣳ॒॒शाय॑ । \newline
38. ज्योति॒र् दध॒द् दध॒ज् ज्योति॒र् ज्योति॒र् दध॑त् । \newline
39. दध॑ देत्येति॒ दध॒द् दध॑ देति । \newline
40. ए॒ति॒ स्तोमः॒ स्तोम॑ एत्येति॒ स्तोमः॑ । \newline
41. स्तोम॑ ए॒वैव स्तोमः॒ स्तोम॑ ए॒व । \newline
42. ए॒व तत् तदे॒ वैव तत् । \newline
43. तथ् स्तोमा॑य॒ स्तोमा॑य॒ तत् तथ् स्तोमा॑य । \newline
44. स्तोमा॑य॒ ज्योति॒र् ज्योतिः॒ स्तोमा॑य॒ स्तोमा॑य॒ ज्योतिः॑ । \newline
45. ज्योति॒र् दध॒द् दध॒ज् ज्योति॒र् ज्योति॒र् दध॑त् । \newline
46. दध॑ देत्येति॒ दध॒द् दध॑ देति । \newline
47. ए॒त्यथो॒ अथो॑ एत्ये॒ त्यथो᳚ । \newline
48. अथो॒ स्तोमे॒ स्तोमे ऽथो॒ अथो॒ स्तोमे᳚ । \newline
49. अथो॒ इत्यथो᳚ । \newline
50. स्तोम॑ ए॒वैव स्तोमे॒ स्तोम॑ ए॒व । \newline
51. ए॒व स्तोमꣳ॒॒ स्तोम॑ मे॒वैव स्तोम᳚म् । \newline
52. स्तोम॑ म॒भ्य॑भि स्तोमꣳ॒॒ स्तोम॑ म॒भि । \newline
53. अ॒भि प्र प्राभ्य॑भि प्र । \newline
54. प्र ण॑यति नयति॒ प्र प्र ण॑यति । \newline
55. न॒य॒ति॒ याव॑न्तो॒ याव॑न्तो नयति नयति॒ याव॑न्तः । \newline
56. याव॑न्तो॒ वै वै याव॑न्तो॒ याव॑न्तो॒ वै । \newline
57. वै स्तोमाः॒ स्तोमा॒ वै वै स्तोमाः᳚ । \newline
58. स्तोमा॒ स्ताव॑न्त॒ स्ताव॑न्तः॒ स्तोमाः॒ स्तोमा॒ स्ताव॑न्तः । \newline
59. ताव॑न्तः॒ कामाः॒ कामा॒ स्ताव॑न्त॒ स्ताव॑न्तः॒ कामाः᳚ । \newline
60. कामा॒ स्ताव॑न्त॒ स्ताव॑न्तः॒ कामाः॒ कामा॒ स्ताव॑न्तः । \newline
61. ताव॑न्तो लो॒का लो॒का स्ताव॑न्त॒ स्ताव॑न्तो लो॒काः । \newline
62. लो॒का स्ताव॑न्ति॒ ताव॑न्ति लो॒का लो॒का स्ताव॑न्ति । \newline
63. ताव॑न्ति॒ ज्योतीꣳ॑षि॒ ज्योतीꣳ॑षि॒ ताव॑न्ति॒ ताव॑न्ति॒ ज्योतीꣳ॑षि । \newline
64. ज्योतीꣳ॑ ष्ये॒ताव॑त ए॒ताव॑तो॒ ज्योतीꣳ॑षि॒ ज्योतीꣳ॑ ष्ये॒ताव॑तः । \newline
65. ए॒ताव॑त ए॒वै वैताव॑त ए॒ताव॑त ए॒व । \newline
66. ए॒व स्तोमा॒न् थ्स्तोमा॑ ने॒वैव स्तोमान्॑ । \newline
67. स्तोमा॑ ने॒ताव॑त ए॒ताव॑तः॒ स्तोमा॒न् थ्स्तोमा॑ ने॒ताव॑तः । \newline
68. ए॒ताव॑तः॒ कामा॒न् कामा॑ ने॒ताव॑त ए॒ताव॑तः॒ कामान्॑ । \newline
69. कामा॑ ने॒ताव॑त ए॒ताव॑तः॒ कामा॒न् कामा॑ ने॒ताव॑तः । \newline
70. ए॒ताव॑तो लो॒कान् ॅलो॒का ने॒ताव॑त ए॒ताव॑तो लो॒कान् । \newline
71. लो॒का ने॒ताव॑ न्त्ये॒ताव॑न्ति लो॒कान् ॅलो॒का ने॒ताव॑न्ति । \newline
72. ए॒ताव॑न्ति॒ ज्योतीꣳ॑षि॒ ज्योतीꣳ॑ ष्ये॒ताव॑ न्त्ये॒ताव॑न्ति॒ ज्योतीꣳ॑षि । \newline
73. ज्योतीꣳ॒॒ ष्यवाव॒ ज्योतीꣳ॑षि॒ ज्योतीꣳ॒॒ष्यव॑ । \newline
74. अव॑ रुन्धे रु॒न्धे ऽवाव॑ रुन्धे । \newline
75. रु॒न्ध॒ इति॑ रुन्धे । \newline

\textbf{Ghana Paata } \newline

1. प्रा॒त॒स्स॒व॒ने वै वै प्रा॑तस्सव॒ने प्रा॑तस्सव॒ने वै गा॑य॒त्रेण॑ गाय॒त्रेण॒ वै प्रा॑तस्सव॒ने प्रा॑तस्सव॒ने वै गा॑य॒त्रेण॑ । \newline
2. प्रा॒त॒स्स॒व॒न इति॑ प्रातः - स॒व॒ने । \newline
3. वै गा॑य॒त्रेण॑ गाय॒त्रेण॒ वै वै गा॑य॒त्रेण॒ छन्द॑सा॒ छन्द॑सा गाय॒त्रेण॒ वै वै गा॑य॒त्रेण॒ छन्द॑सा । \newline
4. गा॒य॒त्रेण॒ छन्द॑सा॒ छन्द॑सा गाय॒त्रेण॑ गाय॒त्रेण॒ छन्द॑सा त्रि॒वृते᳚ त्रि॒वृते॒ छन्द॑सा गाय॒त्रेण॑ गाय॒त्रेण॒ छन्द॑सा त्रि॒वृते᳚ । \newline
5. छन्द॑सा त्रि॒वृते᳚ त्रि॒वृते॒ छन्द॑सा॒ छन्द॑सा त्रि॒वृते॒ स्तोमा॑य॒ स्तोमा॑य त्रि॒वृते॒ छन्द॑सा॒ छन्द॑सा त्रि॒वृते॒ स्तोमा॑य । \newline
6. त्रि॒वृते॒ स्तोमा॑य॒ स्तोमा॑य त्रि॒वृते᳚ त्रि॒वृते॒ स्तोमा॑य॒ ज्योति॒र् ज्योतिः॒ स्तोमा॑य त्रि॒वृते᳚ त्रि॒वृते॒ स्तोमा॑य॒ ज्योतिः॑ । \newline
7. त्रि॒वृत॒ इति॑ त्रि - वृते᳚ । \newline
8. स्तोमा॑य॒ ज्योति॒र् ज्योतिः॒ स्तोमा॑य॒ स्तोमा॑य॒ ज्योति॒र् दध॒द् दध॒ज् ज्योतिः॒ स्तोमा॑य॒ स्तोमा॑य॒ ज्योति॒र् दध॑त् । \newline
9. ज्योति॒र् दध॒द् दध॒ज् ज्योति॒र् ज्योति॒र् दध॑ देत्येति॒ दध॒ज् ज्योति॒र् ज्योति॒र् दध॑ देति । \newline
10. दध॑ देत्येति॒ दध॒द् दध॑ देति त्रि॒वृता᳚ त्रि॒वृ तै॑ति॒ दध॒द् दध॑ देति त्रि॒वृता᳚ । \newline
11. ए॒ति॒ त्रि॒वृता᳚ त्रि॒वृ तै᳚त्येति त्रि॒वृता᳚ ब्रह्मवर्च॒सेन॑ ब्रह्मवर्च॒सेन॑ त्रि॒वृ तै᳚त्येति त्रि॒वृता᳚ ब्रह्मवर्च॒सेन॑ । \newline
12. त्रि॒वृता᳚ ब्रह्मवर्च॒सेन॑ ब्रह्मवर्च॒सेन॑ त्रि॒वृता᳚ त्रि॒वृता᳚ ब्रह्मवर्च॒सेन॑ पञ्चद॒शाय॑ पञ्चद॒शाय॑ ब्रह्मवर्च॒सेन॑ त्रि॒वृता᳚ त्रि॒वृता᳚ ब्रह्मवर्च॒सेन॑ पञ्चद॒शाय॑ । \newline
13. त्रि॒वृतेति॑ त्रि - वृता᳚ । \newline
14. ब्र॒ह्म॒व॒र्च॒सेन॑ पञ्चद॒शाय॑ पञ्चद॒शाय॑ ब्रह्मवर्च॒सेन॑ ब्रह्मवर्च॒सेन॑ पञ्चद॒शाय॒ ज्योति॒र् ज्योतिः॑ पञ्चद॒शाय॑ ब्रह्मवर्च॒सेन॑ ब्रह्मवर्च॒सेन॑ पञ्चद॒शाय॒ ज्योतिः॑ । \newline
15. ब्र॒ह्म॒व॒र्च॒सेनेति॑ ब्रह्म - व॒र्च॒सेन॑ । \newline
16. प॒ञ्च॒द॒शाय॒ ज्योति॒र् ज्योतिः॑ पञ्चद॒शाय॑ पञ्चद॒शाय॒ ज्योति॒र् दध॒द् दध॒ज् ज्योतिः॑ पञ्चद॒शाय॑ पञ्चद॒शाय॒ ज्योति॒र् दध॑त् । \newline
17. प॒ञ्च॒द॒शायेति॑ पञ्च - द॒शाय॑ । \newline
18. ज्योति॒र् दध॒द् दध॒ज् ज्योति॒र् ज्योति॒र् दध॑ देत्येति॒ दध॒ज् ज्योति॒र् ज्योति॒र् दध॑ देति । \newline
19. दध॑ देत्येति॒ दध॒द् दध॑ देति पञ्चद॒शेन॑ पञ्चद॒शे नै॑ति॒ दध॒द् दध॑ देति पञ्चद॒शेन॑ । \newline
20. ए॒ति॒ प॒ञ्च॒द॒शेन॑ पञ्चद॒शे नै᳚त्येति पञ्चद॒शे नौज॒ सौज॑सा पञ्चद॒शे नै᳚त्येति पञ्चद॒शे नौज॑सा । \newline
21. प॒ञ्च॒द॒शे नौज॒ सौज॑सा पञ्चद॒शेन॑ पञ्चद॒शे नौज॑सा वी॒र्ये॑ण वी॒र्ये॑ णौज॑सा पञ्चद॒शेन॑ पञ्चद॒शे नौज॑सा वी॒र्ये॑ण । \newline
22. प॒ञ्च॒द॒शेनेति॑ पञ्च - द॒शेन॑ । \newline
23. ओज॑सा वी॒र्ये॑ण वी॒र्ये॑ णौज॒ सौज॑सा वी॒र्ये॑ण सप्तद॒शाय॑ सप्तद॒शाय॑ वी॒र्ये॑ णौज॒ सौज॑सा वी॒र्ये॑ण सप्तद॒शाय॑ । \newline
24. वी॒र्ये॑ण सप्तद॒शाय॑ सप्तद॒शाय॑ वी॒र्ये॑ण वी॒र्ये॑ण सप्तद॒शाय॒ ज्योति॒र् ज्योतिः॑ सप्तद॒शाय॑ वी॒र्ये॑ण वी॒र्ये॑ण सप्तद॒शाय॒ ज्योतिः॑ । \newline
25. स॒प्त॒द॒शाय॒ ज्योति॒र् ज्योतिः॑ सप्तद॒शाय॑ सप्तद॒शाय॒ ज्योति॒र् दध॒द् दध॒ज् ज्योतिः॑ सप्तद॒शाय॑ सप्तद॒शाय॒ ज्योति॒र् दध॑त् । \newline
26. स॒प्त॒द॒शायेति॑ सप्त - द॒शाय॑ । \newline
27. ज्योति॒र् दध॒द् दध॒ज् ज्योति॒र् ज्योति॒र् दध॑ देत्येति॒ दध॒ज् ज्योति॒र् ज्योति॒र् दध॑ देति । \newline
28. दध॑ देत्येति॒ दध॒द् दध॑ देति सप्तद॒शेन॑ सप्तद॒शे नै॑ति॒ दध॒द् दध॑ देति सप्तद॒शेन॑ । \newline
29. ए॒ति॒ स॒प्त॒द॒शेन॑ सप्तद॒शे नै᳚त्येति सप्तद॒शेन॑ प्राजाप॒त्येन॑ प्राजाप॒त्येन॑ सप्तद॒शे नै᳚त्येति सप्तद॒शेन॑ प्राजाप॒त्येन॑ । \newline
30. स॒प्त॒द॒शेन॑ प्राजाप॒त्येन॑ प्राजाप॒त्येन॑ सप्तद॒शेन॑ सप्तद॒शेन॑ प्राजाप॒त्येन॑ प्र॒जन॑नेन प्र॒जन॑नेन प्राजाप॒त्येन॑ सप्तद॒शेन॑ सप्तद॒शेन॑ प्राजाप॒त्येन॑ प्र॒जन॑नेन । \newline
31. स॒प्त॒द॒शेनेति॑ सप्त - द॒शेन॑ । \newline
32. प्रा॒जा॒प॒त्येन॑ प्र॒जन॑नेन प्र॒जन॑नेन प्राजाप॒त्येन॑ प्राजाप॒त्येन॑ प्र॒जन॑ने नैकविꣳ॒॒शा यै॑कविꣳ॒॒शाय॑ प्र॒जन॑नेन प्राजाप॒त्येन॑ प्राजाप॒त्येन॑ प्र॒जन॑ने नैकविꣳ॒॒शाय॑ । \newline
33. प्रा॒जा॒प॒त्येनेति॑ प्राजा - प॒त्येन॑ । \newline
34. प्र॒जन॑ने नैकविꣳ॒॒शा यै॑कविꣳ॒॒शाय॑ प्र॒जन॑नेन प्र॒जन॑ने नैक॑विꣳ॒॒शाय॒ ज्योति॒र् ज्योति॑ रेकविꣳ॒॒शाय॑ प्र॒जन॑नेन प्र॒जन॑ने नैक॑विꣳ॒॒शाय॒ ज्योतिः॑ । \newline
35. प्र॒जन॑ने॒नेति॑ प्र - जन॑नेन । \newline
36. ए॒क॒विꣳ॒॒शाय॒ ज्योति॒र् ज्योति॑ रेकविꣳ॒॒शा यै॑कविꣳ॒॒शाय॒ ज्योति॒र् दध॒द् दध॒ज् ज्योति॑ रेकविꣳ॒॒शा
यै॑कविꣳ॒॒शाय॒ ज्योति॒र् दध॑त् । \newline
37. ए॒क॒विꣳ॒॒शायेत्ये॑क - विꣳ॒॒शाय॑ । \newline
38. ज्योति॒र् दध॒द् दध॒ज् ज्योति॒र् ज्योति॒र् दध॑ देत्येति॒ दध॒ज् ज्योति॒र् ज्योति॒र् दध॑ देति । \newline
39. दध॑ देत्येति॒ दध॒द् दध॑ देति॒ स्तोमः॒ स्तोम॑ एति॒ दध॒द् दध॑ देति॒ स्तोमः॑ । \newline
40. ए॒ति॒ स्तोमः॒ स्तोम॑ एत्येति॒ स्तोम॑ ए॒वैव स्तोम॑ एत्येति॒ स्तोम॑ ए॒व । \newline
41. स्तोम॑ ए॒वैव स्तोमः॒ स्तोम॑ ए॒व तत् तदे॒व स्तोमः॒ स्तोम॑ ए॒व तत् । \newline
42. ए॒व तत् तदे॒ वैव तथ् स्तोमा॑य॒ स्तोमा॑य॒ तदे॒ वैव तथ् स्तोमा॑य । \newline
43. तथ् स्तोमा॑य॒ स्तोमा॑य॒ तत् तथ् स्तोमा॑य॒ ज्योति॒र् ज्योतिः॒ स्तोमा॑य॒ तत् तथ् स्तोमा॑य॒ ज्योतिः॑ । \newline
44. स्तोमा॑य॒ ज्योति॒र् ज्योतिः॒ स्तोमा॑य॒ स्तोमा॑य॒ ज्योति॒र् दध॒द् दध॒ज् ज्योतिः॒ स्तोमा॑य॒ स्तोमा॑य॒ ज्योति॒र् दध॑त् । \newline
45. ज्योति॒र् दध॒द् दध॒ज् ज्योति॒र् ज्योति॒र् दध॑ देत्येति॒ दध॒ज् ज्योति॒र् ज्योति॒र् दध॑ देति । \newline
46. दध॑ देत्येति॒ दध॒द् दध॑ दे॒त्यथो॒ अथो॑ एति॒ दध॒द् दध॑ दे॒त्यथो᳚ । \newline
47. ए॒त्यथो॒ अथो॑ एत्ये॒ त्यथो॒ स्तोमे॒ स्तोमे ऽथो॑ एत्ये॒ त्यथो॒ स्तोमे᳚ । \newline
48. अथो॒ स्तोमे॒ स्तोमे ऽथो॒ अथो॒ स्तोम॑ ए॒वैव स्तोमे ऽथो॒ अथो॒ स्तोम॑ ए॒व । \newline
49. अथो॒ इत्यथो᳚ । \newline
50. स्तोम॑ ए॒वैव स्तोमे॒ स्तोम॑ ए॒व स्तोमꣳ॒॒ स्तोम॑ मे॒व स्तोमे॒ स्तोम॑ ए॒व स्तोम᳚म् । \newline
51. ए॒व स्तोमꣳ॒॒ स्तोम॑ मे॒वैव स्तोम॑ म॒भ्य॑भि स्तोम॑ मे॒वैव स्तोम॑ म॒भि । \newline
52. स्तोम॑ म॒भ्य॑भि स्तोमꣳ॒॒ स्तोम॑ म॒भि प्र प्राभि स्तोमꣳ॒॒ स्तोम॑ म॒भि प्र । \newline
53. अ॒भि प्र प्राभ्य॑भि प्र ण॑यति नयति॒ प्राभ्य॑भि प्र ण॑यति । \newline
54. प्र ण॑यति नयति॒ प्र प्र ण॑यति॒ याव॑न्तो॒ याव॑न्तो नयति॒ प्र प्र ण॑यति॒ याव॑न्तः । \newline
55. न॒य॒ति॒ याव॑न्तो॒ याव॑न्तो नयति नयति॒ याव॑न्तो॒ वै वै याव॑न्तो नयति नयति॒ याव॑न्तो॒ वै । \newline
56. याव॑न्तो॒ वै वै याव॑न्तो॒ याव॑न्तो॒ वै स्तोमाः॒ स्तोमा॒ वै याव॑न्तो॒ याव॑न्तो॒ वै स्तोमाः᳚ । \newline
57. वै स्तोमाः॒ स्तोमा॒ वै वै स्तोमा॒ स्ताव॑न्त॒ स्ताव॑न्तः॒ स्तोमा॒ वै वै स्तोमा॒ स्ताव॑न्तः । \newline
58. स्तोमा॒ स्ताव॑न्त॒ स्ताव॑न्तः॒ स्तोमाः॒ स्तोमा॒ स्ताव॑न्तः॒ कामाः॒ कामा॒ स्ताव॑न्तः॒ स्तोमाः॒ स्तोमा॒ स्ताव॑न्तः॒ कामाः᳚ । \newline
59. ताव॑न्तः॒ कामाः॒ कामा॒ स्ताव॑न्त॒ स्ताव॑न्तः॒ कामा॒ स्ताव॑न्त॒ स्ताव॑न्तः॒ कामा॒ स्ताव॑न्त॒ स्ताव॑न्तः॒ कामा॒ स्ताव॑न्तः । \newline
60. कामा॒ स्ताव॑न्त॒ स्ताव॑न्तः॒ कामाः॒ कामा॒ स्ताव॑न्तो लो॒का लो॒का स्ताव॑न्तः॒ कामाः॒ कामा॒ स्ताव॑न्तो लो॒काः । \newline
61. ताव॑न्तो लो॒का लो॒का स्ताव॑न्त॒ स्ताव॑न्तो लो॒का स्ताव॑न्ति॒ ताव॑न्ति लो॒का स्ताव॑न्त॒ स्ताव॑न्तो लो॒का स्ताव॑न्ति । \newline
62. लो॒का स्ताव॑न्ति॒ ताव॑न्ति लो॒का लो॒का स्ताव॑न्ति॒ ज्योतीꣳ॑षि॒ ज्योतीꣳ॑षि॒ ताव॑न्ति लो॒का लो॒का स्ताव॑न्ति॒ ज्योतीꣳ॑षि । \newline
63. ताव॑न्ति॒ ज्योतीꣳ॑षि॒ ज्योतीꣳ॑षि॒ ताव॑न्ति॒ ताव॑न्ति॒ ज्योतीꣳ॑ ष्ये॒ताव॑त ए॒ताव॑तो॒ ज्योतीꣳ॑षि॒ ताव॑न्ति॒ ताव॑न्ति॒ ज्योतीꣳ॑ ष्ये॒ताव॑तः । \newline
64. ज्योतीꣳ॑ ष्ये॒ताव॑त ए॒ताव॑तो॒ ज्योतीꣳ॑षि॒ ज्योतीꣳ॑ ष्ये॒ताव॑त ए॒वैवैताव॑तो॒ ज्योतीꣳ॑षि॒ ज्योतीꣳ॑
ष्ये॒ताव॑त ए॒व । \newline
65. ए॒ताव॑त ए॒वै वैताव॑त ए॒ताव॑त ए॒व स्तोमा॒न् थ्स्तोमा॑ ने॒वैताव॑त ए॒ताव॑त ए॒व स्तोमान्॑ । \newline
66. ए॒व स्तोमा॒न् थ्स्तोमा॑ ने॒वैव स्तोमा॑ ने॒ताव॑त ए॒ताव॑तः॒ स्तोमा॑ ने॒वैव स्तोमा॑ ने॒ताव॑तः । \newline
67. स्तोमा॑ ने॒ताव॑त ए॒ताव॑तः॒ स्तोमा॒न् थ्स्तोमा॑ ने॒ताव॑तः॒ कामा॒न् कामा॑ ने॒ताव॑तः॒ स्तोमा॒न् थ्स्तोमा॑ ने॒ताव॑तः॒ कामान्॑ । \newline
68. ए॒ताव॑तः॒ कामा॒न् कामा॑ ने॒ताव॑त ए॒ताव॑तः॒ कामा॑ ने॒ताव॑त ए॒ताव॑तः॒ कामा॑ ने॒ताव॑त ए॒ताव॑तः॒ कामा॑ ने॒ताव॑तः । \newline
69. कामा॑ ने॒ताव॑त ए॒ताव॑तः॒ कामा॒न् कामा॑ ने॒ताव॑तो लो॒कान् ॅलो॒का ने॒ताव॑तः॒ कामा॒न् कामा॑ ने॒ताव॑तो लो॒कान् । \newline
70. ए॒ताव॑तो लो॒कान् ॅलो॒का ने॒ताव॑त ए॒ताव॑तो लो॒का ने॒ताव॑ न्त्ये॒ताव॑न्ति लो॒का ने॒ताव॑त ए॒ताव॑तो लो॒का ने॒ताव॑न्ति । \newline
71. लो॒का ने॒ताव॑ न्त्ये॒ताव॑न्ति लो॒कान् ॅलो॒का ने॒ताव॑न्ति॒ ज्योतीꣳ॑षि॒ ज्योतीꣳ॑ ष्ये॒ताव॑न्ति लो॒कान् ॅलो॒का ने॒ताव॑न्ति॒ ज्योतीꣳ॑षि । \newline
72. ए॒ताव॑न्ति॒ ज्योतीꣳ॑षि॒ ज्योतीꣳ॑ ष्ये॒ताव॑ न्त्ये॒ताव॑न्ति॒ ज्योतीꣳ॒॒ ष्यवाव॒ ज्योतीꣳ॑ ष्ये॒ताव॑ न्त्ये॒ताव॑न्ति॒ ज्योतीꣳ॒॒ ष्यव॑ । \newline
73. ज्योतीꣳ॒॒ ष्यवाव॒ ज्योतीꣳ॑षि॒ ज्योतीꣳ॒॒ ष्यव॑ रुन्धे रु॒न्धे ऽव॒ ज्योतीꣳ॑षि॒ ज्योतीꣳ॒॒ष्यव॑ रुन्धे । \newline
74. अव॑ रुन्धे रु॒न्धे ऽवाव॑ रुन्धे । \newline
75. रु॒न्ध॒ इति॑ रुन्धे । \newline
\pagebreak
\markright{ TS 7.1.3.1  \hfill https://www.vedavms.in \hfill}

\section{ TS 7.1.3.1 }

\textbf{TS 7.1.3.1 } \newline
\textbf{Samhita Paata} \newline

ब्र॒ह्म॒वा॒दिनो॑ वदन्ति॒ स त्वै य॑जेत॒ यो᳚ऽग्निष्टो॒मेन॒ यज॑मा॒नोऽथ॒ सर्व॑स्तोमेन॒ यजे॒तेति॒ यस्य॑ त्रि॒वृत॑मन्त॒र्यन्ति॑ प्रा॒णाꣳ-स्तस्या॒न्तर्य॑न्ति प्रा॒णेषु॒ मेऽप्य॑स॒दिति॒ खलु॒ वै य॒ज्ञेन॒ यज॑मानो यजते॒ यस्य॑ पञ्चद॒शम॑न्त॒र्यन्ति॑ वी॒र्यं॑ तस्या॒न्तर्य॑न्ति वी॒र्ये॑ मेऽप्य॑स॒दिति॒ खलु॒ वै य॒ज्ञेन॒ यज॑मानो यजते॒ यस्य॑ सप्तद॒श-म॑न्त॒र्यन्ति॑ - [  ] \newline

\textbf{Pada Paata} \newline

ब्र॒ह्म॒वा॒दिन॒ इति॑ ब्रह्म - वा॒दिनः॑ । व॒द॒न्ति॒ । सः । तु । वै । य॒जे॒त॒ । यः । अ॒ग्नि॒ष्टो॒मेनेत्य॑ग्नि - स्तो॒मेन॑ । यज॑मानः । अथ॑ । सर्व॑स्तोमे॒नेति॒ सर्व॑ - स्तो॒मे॒न॒ । यजे॑त । इति॑ । यस्य॑ । त्रि॒वृत॒मिति॑ त्रि - वृत᳚म् । अ॒न्त॒र्यन्तीत्य॑न्तः-यन्ति॑ । प्रा॒णानिति॑ प्र - अ॒नान् । तस्य॑ । अ॒न्तः । य॒न्ति॒ । प्रा॒णेष्विति॑ प्र - अ॒नेषु॑ । मे॒ । अपीति॑ । अ॒स॒त् । इति॑ । खलु॑ । वै । य॒ज्ञेन॑ । यज॑मानः । य॒ज॒ते॒ । यस्य॑ । प॒ञ्च॒द॒शमिति॑ पञ्च - द॒शम् । अ॒न्त॒र्यन्तीत्य॑न्तः - यन्ति॑ । वी॒र्य᳚म् । तस्य॑ । अ॒न्तः । य॒न्ति॒ । वी॒र्ये᳚ । मे॒ । अपीति॑ । अ॒स॒त् । इति॑ । खलु॑ । वै । य॒ज्ञेन॑ । यज॑मानः । य॒ज॒ते॒ । यस्य॑ । स॒प्त॒द॒शमिति॑ सप्त - द॒शम् । अ॒न्त॒र्यन्तीत्य॑न्तः - यन्ति॑ ।  \newline


\textbf{Krama Paata} \newline

ब्र॒ह्म॒वा॒दिनो॑ वदन्ति । ब्र॒ह्म॒वा॒दिन॒ इति॑ ब्रह्म - वा॒दिनः॑ । व॒द॒न्ति॒ सः । स तु । त्वै । वै य॑जेत । य॒जे॒त॒ यः । यो᳚ऽग्निष्टो॒मेन॑ । अ॒ग्नि॒ष्टो॒मेन॒ यज॑मानः । अ॒ग्नि॒ष्टो॒मेनेत्य॑ग्नि - स्तो॒मेन॑ । यज॑मा॒नोऽथ॑ । अथ॒ सर्व॑स्तोमेन । सर्व॑स्तोमेन॒ यजे॑त । सर्व॑स्तोमे॒नेति॒ सर्व॑ - स्तो॒मे॒न॒ । यजे॒तेति॑ । इति॒ यस्य॑ । यस्य॑ त्रि॒वृत᳚म् । त्रि॒वृत॑मन्त॒र्यन्ति॑ । त्रि॒वृत॒मिति॑ त्रि - वृतम्᳚ । अ॒न्त॒र्यन्ति॑ प्रा॒णान् । अ॒न्त॒र्य॒न्तीत्य॑न्तः - यन्ति॑ । प्रा॒णाꣳस्तस्य॑ । प्रा॒णानिति॑ प्र - अ॒नान् । तस्या॒न्तः । अ॒न्तर्य॑न्ति । य॒न्ति॒ प्रा॒णेषु॑ । प्रा॒णेषु॑ मे । प्रा॒णेष्विति॑ प्र - अ॒नेषु॑ । मेऽपि॑ । अप्य॑सत् । अ॒स॒दिति॑ । इति॒ खलु॑ । खलु॒ वै । वै य॒ज्ञेन॑ । य॒ज्ञेन॒ यज॑मानः । यज॑मानो यजते । य॒ज॒ते॒ यस्य॑ । यस्य॑ पञ्चद॒शम् । प॒ञ्च॒द॒शम॑न्त॒र्यन्ति॑ । प॒ञ्च॒द॒शमिति॑ पञ्च - द॒शम् । अ॒न्त॒र्यन्ति॑ वी॒र्य᳚म् । अ॒न्त॒र्यन्तीत्य॑न्तः - यन्ति॑ । वी॒र्य॑म् तस्य॑ । तस्या॒न्तः । अ॒न्तर्य॑न्ति । य॒न्ति॒ वी॒र्ये᳚ । वी॒र्ये॑ मे । मेऽपि॑ । अप्य॑सत् । अ॒स॒दिति॑ । इति॒ खलु॑ । खलु॒ वै । वै य॒ज्ञेन॑ । य॒ज्ञेन॒ यज॑मानः । यज॑मानो यजते । य॒ज॒ते॒ यस्य॑ । यस्य॑ सप्तद॒शम् । स॒प्त॒द॒शम॑न्त॒र्यन्ति॑ । स॒प्त॒द॒शमिति॑ सप्त - द॒शम् । अ॒न्त॒र्यन्ति॑ प्र॒जाम् । अ॒न्त॒र्यन्तीत्य॑न्तः - यन्ति॑ \newline

\textbf{Jatai Paata} \newline

1. ब्र॒ह्म॒वा॒दिनो॑ वदन्ति वदन्ति ब्रह्मवा॒दिनो᳚ ब्रह्मवा॒दिनो॑ वदन्ति । \newline
2. ब्र॒ह्म॒वा॒दिन॒ इति॑ ब्रह्म - वा॒दिनः॑ । \newline
3. व॒द॒न्ति॒ स स व॑दन्ति वदन्ति॒ सः । \newline
4. स तु तु स स तु । \newline
5. त्वै वै तु त्वै । \newline
6. वै य॑जेत यजेत॒ वै वै य॑जेत । \newline
7. य॒जे॒त॒ यो यो य॑जेत यजेत॒ यः । \newline
8. यो᳚ ऽग्निष्टो॒मेना᳚ ग्निष्टो॒मेन॒ यो यो᳚ ऽग्निष्टो॒मेन॑ । \newline
9. अ॒ग्नि॒ष्टो॒मेन॒ यज॑मानो॒ यज॑मानो ऽग्निष्टो॒मेना᳚ ग्निष्टो॒मेन॒ यज॑मानः । \newline
10. अ॒ग्नि॒ष्टो॒मेनेत्य॑ग्नि - स्तो॒मेन॑ । \newline
11. यज॑मा॒नो ऽथाथ॒ यज॑मानो॒ यज॑मा॒नो ऽथ॑ । \newline
12. अथ॒ सर्व॑स्तोमेन॒ सर्व॑स्तोमे॒ना थाथ॒ सर्व॑स्तोमेन । \newline
13. सर्व॑स्तोमेन॒ यजे॑त॒ यजे॑त॒ सर्व॑स्तोमेन॒ सर्व॑स्तोमेन॒ यजे॑त । \newline
14. सर्व॑स्तोमे॒नेति॒ सर्व॑ - स्तो॒मे॒न॒ । \newline
15. यजे॒तेतीति॒ यजे॑त॒ यजे॒तेति॑ । \newline
16. इति॒ यस्य॒ यस्येतीति॒ यस्य॑ । \newline
17. यस्य॑ त्रि॒वृत॑म् त्रि॒वृतं॒ ॅयस्य॒ यस्य॑ त्रि॒वृत᳚म् । \newline
18. त्रि॒वृत॑ मन्त॒र्य न्त्य॑न्त॒र्यन्ति॑ त्रि॒वृत॑म् त्रि॒वृत॑ मन्त॒र्यन्ति॑ । \newline
19. त्रि॒वृत॒मिति॑ त्रि - वृत᳚म् । \newline
20. अ॒न्त॒र्यन्ति॑ प्रा॒णान् प्रा॒णा न॑न्त॒र्य न्त्य॑न्त॒र्यन्ति॑ प्रा॒णान् । \newline
21. अ॒न्त॒र्यन्तीत्य॑न्तः - यन्ति॑ । \newline
22. प्रा॒णाꣳ स्तस्य॒ तस्य॑ प्रा॒णान् प्रा॒णाꣳ स्तस्य॑ । \newline
23. प्रा॒णानिति॑ प्र - अ॒नान् । \newline
24. तस्या॒न्त र॒न्त स्तस्य॒ तस्या॒न्तः । \newline
25. अ॒न्तर् य॑न्ति यन्त्य॒न्त र॒न्तर् य॑न्ति । \newline
26. य॒न्ति॒ प्रा॒णेषु॑ प्रा॒णेषु॑ यन्ति यन्ति प्रा॒णेषु॑ । \newline
27. प्रा॒णेषु॑ मे मे प्रा॒णेषु॑ प्रा॒णेषु॑ मे । \newline
28. प्रा॒णेष्विति॑ प्र - अ॒नेषु॑ । \newline
29. मे ऽप्यपि॑ मे॒ मे ऽपि॑ । \newline
30. अप्य॑स दस॒ दप्य प्य॑सत् । \newline
31. अ॒स॒ दिती त्य॑स दस॒ दिति॑ । \newline
32. इति॒ खलु॒ खल्वितीति॒ खलु॑ । \newline
33. खलु॒ वै वै खलु॒ खलु॒ वै । \newline
34. वै य॒ज्ञेन॑ य॒ज्ञेन॒ वै वै य॒ज्ञेन॑ । \newline
35. य॒ज्ञेन॒ यज॑मानो॒ यज॑मानो य॒ज्ञेन॑ य॒ज्ञेन॒ यज॑मानः । \newline
36. यज॑मानो यजते यजते॒ यज॑मानो॒ यज॑मानो यजते । \newline
37. य॒ज॒ते॒ यस्य॒ यस्य॑ यजते यजते॒ यस्य॑ । \newline
38. यस्य॑ पञ्चद॒शम् प॑ञ्चद॒शं ॅयस्य॒ यस्य॑ पञ्चद॒शम् । \newline
39. प॒ञ्च॒द॒श म॑न्त॒र्य न्त्य॑न्त॒र्यन्ति॑ पञ्चद॒शम् प॑ञ्चद॒श म॑न्त॒र्यन्ति॑ । \newline
40. प॒ञ्च॒द॒शमिति॑ पञ्च - द॒शम् । \newline
41. अ॒न्त॒र्यन्ति॑ वी॒र्यं॑ ॅवी॒र्य॑ मन्त॒र्य न्त्य॑न्त॒र्यन्ति॑ वी॒र्य᳚म् । \newline
42. अ॒न्त॒र्यन्तीत्य॑न्तः - यन्ति॑ । \newline
43. वी॒र्य॑म् तस्य॒ तस्य॑ वी॒र्यं॑ ॅवी॒र्य॑म् तस्य॑ । \newline
44. तस्या॒न्त र॒न्त स्तस्य॒ तस्या॒न्तः । \newline
45. अ॒न्तर् य॑न्ति यन्त्य॒न्त र॒न्तर् य॑न्ति । \newline
46. य॒न्ति॒ वी॒र्ये॑ वी॒र्ये॑ यन्ति यन्ति वी॒र्ये᳚ । \newline
47. वी॒र्ये॑ मे मे वी॒र्ये॑ वी॒र्ये॑ मे । \newline
48. मे ऽप्यपि॑ मे॒ मे ऽपि॑ । \newline
49. अप्य॑स दस॒ दप्य प्य॑सत् । \newline
50. अ॒स॒ दिती त्य॑स दस॒ दिति॑ । \newline
51. इति॒ खलु॒ खल्वितीति॒ खलु॑ । \newline
52. खलु॒ वै वै खलु॒ खलु॒ वै । \newline
53. वै य॒ज्ञेन॑ य॒ज्ञेन॒ वै वै य॒ज्ञेन॑ । \newline
54. य॒ज्ञेन॒ यज॑मानो॒ यज॑मानो य॒ज्ञेन॑ य॒ज्ञेन॒ यज॑मानः । \newline
55. यज॑मानो यजते यजते॒ यज॑मानो॒ यज॑मानो यजते । \newline
56. य॒ज॒ते॒ यस्य॒ यस्य॑ यजते यजते॒ यस्य॑ । \newline
57. यस्य॑ सप्तद॒शꣳ स॑प्तद॒शं ॅयस्य॒ यस्य॑ सप्तद॒शम् । \newline
58. स॒प्त॒द॒श म॑न्त॒र्य न्त्य॑न्त॒र्यन्ति॑ सप्तद॒शꣳ स॑प्तद॒श म॑न्त॒र्यन्ति॑ । \newline
59. स॒प्त॒द॒शमिति॑ सप्त - द॒शम् । \newline
60. अ॒न्त॒र्यन्ति॑ प्र॒जाम् प्र॒जा म॑न्त॒र्य न्त्य॑न्त॒र्यन्ति॑ प्र॒जाम् । \newline
61. अ॒न्त॒र्यन्तीत्य॑न्तः - यन्ति॑ । \newline

\textbf{Ghana Paata } \newline

1. ब्र॒ह्म॒वा॒दिनो॑ वदन्ति वदन्ति ब्रह्मवा॒दिनो᳚ ब्रह्मवा॒दिनो॑ वदन्ति॒ स स व॑दन्ति ब्रह्मवा॒दिनो᳚ ब्रह्मवा॒दिनो॑ वदन्ति॒ सः । \newline
2. ब्र॒ह्म॒वा॒दिन॒ इति॑ ब्रह्म - वा॒दिनः॑ । \newline
3. व॒द॒न्ति॒ स स व॑दन्ति वदन्ति॒ स तु तु स व॑दन्ति वदन्ति॒ स तु । \newline
4. स तु तु स स त्वै वै तु स स त्वै । \newline
5. त्वै वै तु त्वै य॑जेत यजेत॒ वै तु त्वै य॑जेत । \newline
6. वै य॑जेत यजेत॒ वै वै य॑जेत॒ यो यो य॑जेत॒ वै वै य॑जेत॒ यः । \newline
7. य॒जे॒त॒ यो यो य॑जेत यजेत॒ यो᳚ ऽग्निष्टो॒मेना᳚ ग्निष्टो॒मेन॒ यो य॑जेत यजेत॒ यो᳚ ऽग्निष्टो॒मेन॑ । \newline
8. यो᳚ ऽग्निष्टो॒मेना᳚ ग्निष्टो॒मेन॒ यो यो᳚ ऽग्निष्टो॒मेन॒ यज॑मानो॒ यज॑मानो ऽग्निष्टो॒मेन॒ यो यो᳚ ऽग्निष्टो॒मेन॒ यज॑मानः । \newline
9. अ॒ग्नि॒ष्टो॒मेन॒ यज॑मानो॒ यज॑मानो ऽग्निष्टो॒मेना᳚ ग्निष्टो॒मेन॒ यज॑मा॒नो ऽथाथ॒ यज॑मानो ऽग्निष्टो॒मेना᳚ ग्निष्टो॒मेन॒ यज॑मा॒नो ऽथ॑ । \newline
10. अ॒ग्नि॒ष्टो॒मेनेत्य॑ग्नि - स्तो॒मेन॑ । \newline
11. यज॑मा॒नो ऽथाथ॒ यज॑मानो॒ यज॑मा॒नो ऽथ॒ सर्व॑स्तोमेन॒ सर्व॑स्तोमे॒ नाथ॒ यज॑मानो॒ यज॑मा॒नो ऽथ॒ सर्व॑स्तोमेन । \newline
12. अथ॒ सर्व॑स्तोमेन॒ सर्व॑स्तोमे॒ना थाथ॒ सर्व॑स्तोमेन॒ यजे॑त॒ यजे॑त॒ सर्व॑स्तोमे॒ना थाथ॒ सर्व॑स्तोमेन॒ यजे॑त । \newline
13. सर्व॑स्तोमेन॒ यजे॑त॒ यजे॑त॒ सर्व॑स्तोमेन॒ सर्व॑स्तोमेन॒ यजे॒तेतीति॒ यजे॑त॒ सर्व॑स्तोमेन॒ सर्व॑स्तोमेन॒ यजे॒तेति॑ । \newline
14. सर्व॑स्तोमे॒नेति॒ सर्व॑ - स्तो॒मे॒न॒ । \newline
15. यजे॒तेतीति॒ यजे॑त॒ यजे॒तेति॒ यस्य॒ यस्येति॒ यजे॑त॒ यजे॒तेति॒ यस्य॑ । \newline
16. इति॒ यस्य॒ यस्येतीति॒ यस्य॑ त्रि॒वृत॑म् त्रि॒वृतं॒ ॅयस्येतीति॒ यस्य॑ त्रि॒वृत᳚म् । \newline
17. यस्य॑ त्रि॒वृत॑म् त्रि॒वृतं॒ ॅयस्य॒ यस्य॑ त्रि॒वृत॑ मन्त॒र्य न्त्य॑न्त॒र्यन्ति॑ त्रि॒वृतं॒ ॅयस्य॒ यस्य॑ त्रि॒वृत॑ मन्त॒र्यन्ति॑ । \newline
18. त्रि॒वृत॑ मन्त॒र्य न्त्य॑न्त॒र्यन्ति॑ त्रि॒वृत॑म् त्रि॒वृत॑ मन्त॒र्यन्ति॑ प्रा॒णान् प्रा॒णा न॑न्त॒र्यन्ति॑ त्रि॒वृत॑म् त्रि॒वृत॑ मन्त॒र्यन्ति॑ प्रा॒णान् । \newline
19. त्रि॒वृत॒मिति॑ त्रि - वृत᳚म् । \newline
20. अ॒न्त॒र्यन्ति॑ प्रा॒णान् प्रा॒णा न॑न्त॒र्य न्त्य॑न्त॒र्यन्ति॑ प्रा॒णाꣳ स्तस्य॒ तस्य॑ प्रा॒णा न॑न्त॒र्य न्त्य॑न्त॒र्यन्ति॑ प्रा॒णाꣳ स्तस्य॑ । \newline
21. अ॒न्त॒र्यन्तीत्य॑न्तः - यन्ति॑ । \newline
22. प्रा॒णाꣳ स्तस्य॒ तस्य॑ प्रा॒णान् प्रा॒णाꣳ स्तस्या॒न्त र॒न्त स्तस्य॑ प्रा॒णान् प्रा॒णाꣳ स्तस्या॒न्तः । \newline
23. प्रा॒णानिति॑ प्र - अ॒नान् । \newline
24. तस्या॒न्त र॒न्त स्तस्य॒ तस्या॒न्तर् य॑न्ति यन्त्य॒न्त स्तस्य॒ तस्या॒न्तर् य॑न्ति । \newline
25. अ॒न्तर् य॑न्ति यन्त्य॒न्त र॒न्तर् य॑न्ति प्रा॒णेषु॑ प्रा॒णेषु॑ यन्त्य॒न्त र॒न्तर् य॑न्ति प्रा॒णेषु॑ । \newline
26. य॒न्ति॒ प्रा॒णेषु॑ प्रा॒णेषु॑ यन्ति यन्ति प्रा॒णेषु॑ मे मे प्रा॒णेषु॑ यन्ति यन्ति प्रा॒णेषु॑ मे । \newline
27. प्रा॒णेषु॑ मे मे प्रा॒णेषु॑ प्रा॒णेषु॒ मे ऽप्यपि॑ मे प्रा॒णेषु॑ प्रा॒णेषु॒ मे ऽपि॑ । \newline
28. प्रा॒णेष्विति॑ प्र - अ॒नेषु॑ । \newline
29. मे ऽप्यपि॑ मे॒ मे ऽप्य॑स दस॒ दपि॑ मे॒ मे ऽप्य॑सत् । \newline
30. अप्य॑स दस॒ दप्यप्य॑ स॒दिती त्य॑स॒ दप्यप्य॑ स॒दिति॑ । \newline
31. अ॒स॒ दिती त्य॑सद स॒दिति॒ खलु॒ खल्वि त्य॑स दस॒ दिति॒ खलु॑ । \newline
32. इति॒ खलु॒ खल्वितीति॒ खलु॒ वै वै खल्वितीति॒ खलु॒ वै । \newline
33. खलु॒ वै वै खलु॒ खलु॒ वै य॒ज्ञेन॑ य॒ज्ञेन॒ वै खलु॒ खलु॒ वै य॒ज्ञेन॑ । \newline
34. वै य॒ज्ञेन॑ य॒ज्ञेन॒ वै वै य॒ज्ञेन॒ यज॑मानो॒ यज॑मानो य॒ज्ञेन॒ वै वै य॒ज्ञेन॒ यज॑मानः । \newline
35. य॒ज्ञेन॒ यज॑मानो॒ यज॑मानो य॒ज्ञेन॑ य॒ज्ञेन॒ यज॑मानो यजते यजते॒ यज॑मानो य॒ज्ञेन॑ य॒ज्ञेन॒ यज॑मानो यजते । \newline
36. यज॑मानो यजते यजते॒ यज॑मानो॒ यज॑मानो यजते॒ यस्य॒ यस्य॑ यजते॒ यज॑मानो॒ यज॑मानो यजते॒ यस्य॑ । \newline
37. य॒ज॒ते॒ यस्य॒ यस्य॑ यजते यजते॒ यस्य॑ पञ्चद॒शम् प॑ञ्चद॒शं ॅयस्य॑ यजते यजते॒ यस्य॑ पञ्चद॒शम् । \newline
38. यस्य॑ पञ्चद॒शम् प॑ञ्चद॒शं ॅयस्य॒ यस्य॑ पञ्चद॒श म॑न्त॒र्य न्त्य॑न्त॒र्यन्ति॑ पञ्चद॒शं ॅयस्य॒ यस्य॑ पञ्चद॒श म॑न्त॒र्यन्ति॑ । \newline
39. प॒ञ्च॒द॒श म॑न्त॒र्य न्त्य॑न्त॒र्यन्ति॑ पञ्चद॒शम् प॑ञ्चद॒श म॑न्त॒र्यन्ति॑ वी॒र्यं॑ ॅवी॒र्य॑ मन्त॒र्यन्ति॑ पञ्चद॒शम् प॑ञ्चद॒श म॑न्त॒र्यन्ति॑ वी॒र्य᳚म् । \newline
40. प॒ञ्च॒द॒शमिति॑ पञ्च - द॒शम् । \newline
41. अ॒न्त॒र्यन्ति॑ वी॒र्यं॑ ॅवी॒र्य॑ मन्त॒र्य न्त्य॑न्त॒र्यन्ति॑ वी॒र्य॑म् तस्य॒ तस्य॑ वी॒र्य॑ मन्त॒र्य न्त्य॑न्त॒र्यन्ति॑ वी॒र्य॑म् तस्य॑ । \newline
42. अ॒न्त॒र्यन्तीत्य॑न्तः - यन्ति॑ । \newline
43. वी॒र्य॑म् तस्य॒ तस्य॑ वी॒र्यं॑ ॅवी॒र्य॑म् तस्या॒न्त र॒न्त स्तस्य॑ वी॒र्यं॑ ॅवी॒र्य॑म् तस्या॒न्तः । \newline
44. तस्या॒न्त र॒न्त स्तस्य॒ तस्या॒न्तर् य॑न्ति यन्त्य॒न्त स्तस्य॒ तस्या॒न्तर् य॑न्ति । \newline
45. अ॒न्तर् य॑न्ति यन्त्य॒न्त र॒न्तर् य॑न्ति वी॒र्ये॑ वी॒र्ये॑ यन्त्य॒न्त र॒न्तर् य॑न्ति वी॒र्ये᳚ । \newline
46. य॒न्ति॒ वी॒र्ये॑ वी॒र्ये॑ यन्ति यन्ति वी॒र्ये॑ मे मे वी॒र्ये॑ यन्ति यन्ति वी॒र्ये॑ मे । \newline
47. वी॒र्ये॑ मे मे वी॒र्ये॑ वी॒र्ये॑ मे ऽप्यपि॑ मे वी॒र्ये॑ वी॒र्ये॑ मे ऽपि॑ । \newline
48. मे ऽप्यपि॑ मे॒ मे ऽप्य॑स दस॒ दपि॑ मे॒ मे ऽप्य॑सत् । \newline
49. अप्य॑स दस॒ दप्यप्य॑ स॒दिती त्य॑स॒ दप्यप्य॑ स॒दिति॑ । \newline
50. अ॒स॒ दिती त्य॑सद स॒दिति॒ खलु॒ खल्वि त्य॑स दस॒ दिति॒ खलु॑ । \newline
51. इति॒ खलु॒ खल्वितीति॒ खलु॒ वै वै खल्वितीति॒ खलु॒ वै । \newline
52. खलु॒ वै वै खलु॒ खलु॒ वै य॒ज्ञेन॑ य॒ज्ञेन॒ वै खलु॒ खलु॒ वै य॒ज्ञेन॑ । \newline
53. वै य॒ज्ञेन॑ य॒ज्ञेन॒ वै वै य॒ज्ञेन॒ यज॑मानो॒ यज॑मानो य॒ज्ञेन॒ वै वै य॒ज्ञेन॒ यज॑मानः । \newline
54. य॒ज्ञेन॒ यज॑मानो॒ यज॑मानो य॒ज्ञेन॑ य॒ज्ञेन॒ यज॑मानो यजते यजते॒ यज॑मानो य॒ज्ञेन॑ य॒ज्ञेन॒ यज॑मानो यजते । \newline
55. यज॑मानो यजते यजते॒ यज॑मानो॒ यज॑मानो यजते॒ यस्य॒ यस्य॑ यजते॒ यज॑मानो॒ यज॑मानो यजते॒ यस्य॑ । \newline
56. य॒ज॒ते॒ यस्य॒ यस्य॑ यजते यजते॒ यस्य॑ सप्तद॒शꣳ स॑प्तद॒शं ॅयस्य॑ यजते यजते॒ यस्य॑ सप्तद॒शम् । \newline
57. यस्य॑ सप्तद॒शꣳ स॑प्तद॒शं ॅयस्य॒ यस्य॑ सप्तद॒श म॑न्त॒र्य न्त्य॑न्त॒र्यन्ति॑ सप्तद॒शं ॅयस्य॒ यस्य॑ सप्तद॒श म॑न्त॒र्यन्ति॑ । \newline
58. स॒प्त॒द॒श म॑न्त॒र्य न्त्य॑न्त॒र्यन्ति॑ सप्तद॒शꣳ स॑प्तद॒श म॑न्त॒र्यन्ति॑ प्र॒जाम् प्र॒जा म॑न्त॒र्यन्ति॑ सप्तद॒शꣳ स॑प्तद॒श म॑न्त॒र्यन्ति॑ प्र॒जाम् । \newline
59. स॒प्त॒द॒शमिति॑ सप्त - द॒शम् । \newline
60. अ॒न्त॒र्यन्ति॑ प्र॒जाम् प्र॒जा म॑न्त॒र्य न्त्य॑न्त॒र्यन्ति॑ प्र॒जाम् तस्य॒ तस्य॑ प्र॒जा म॑न्त॒र्य न्त्य॑न्त॒र्यन्ति॑ प्र॒जाम् तस्य॑ । \newline
61. अ॒न्त॒र्यन्तीत्य॑न्तः - यन्ति॑ । \newline
\pagebreak
\markright{ TS 7.1.3.2  \hfill https://www.vedavms.in \hfill}

\section{ TS 7.1.3.2 }

\textbf{TS 7.1.3.2 } \newline
\textbf{Samhita Paata} \newline

प्र॒जां तस्या॒न्तर्य॑न्ति प्र॒जायां॒ मेऽप्य॑स॒दिति॒ खलु॒ वै य॒ज्ञेन॒ यज॑मानो यजते॒ यस्यै॑कविꣳ॒॒शम॑न्त॒र्यन्ति॑ प्रति॒ष्ठां तस्या॒न्तर्य॑न्ति प्रति॒ष्ठायां॒ मेऽप्य॑स॒दिति॒ खलु॒ वै य॒ज्ञेन॒ यज॑मानो यजते॒ यस्य॑ त्रिण॒वम॑न्त॒र्यन्त्यृ॒तूꣳश्च॒ तस्य॑ नक्ष॒त्रियां᳚ च वि॒राज॑म॒न्तर्य॑न्त्यृ॒तुषु॒ मेऽप्य॑सन्नक्ष॒त्रिया॑यां च वि॒राजीति॒ - [  ] \newline

\textbf{Pada Paata} \newline

प्र॒जामिति॑ प्र - जाम् । तस्य॑ । अ॒न्तः । य॒न्ति॒ । प्र॒जाया॒मिति॑ प्र - जाया᳚म् । मे॒ । अपीति॑ । अ॒स॒त् । इति॑ । खलु॑ । वै । य॒ज्ञेन॑ । यज॑मानः । य॒ज॒ते॒ । यस्य॑ । ए॒क॒विꣳ॒॒शमित्ये॑क - विꣳ॒॒शम् । अ॒न्त॒र्यन्तित्य॑न्तः - यन्ति॑ । प्र॒ति॒ष्ठामिति॑ प्रति - स्थाम् । तस्य॑ । अ॒न्तः । य॒न्ति॒ । प्र॒ति॒ष्ठाया॒मिति॑ प्रति - स्थाया᳚म् । मे॒ । अपीति॑ । अ॒स॒त् । इति॑ । खलु॑ । वै । य॒ज्ञेन॑ । यज॑मानः । य॒ज॒ते॒ । यस्य॑ । त्रि॒ण॒वमिति॑ त्रि - न॒वम् । अ॒न्त॒र्यन्तीत्य॑न्तः - यन्ति॑ । ऋ॒तून् । च॒ । तस्य॑ । न॒क्ष॒त्रिया᳚म् । च॒ । वि॒राज॒मिति॑ वि-राज᳚म् । अ॒न्तः । य॒न्ति॒ । ऋ॒तुषु॑ । मे॒ । अपीति॑ । अ॒स॒त् । न॒क्ष॒त्रिया॑याम् । च॒ । वि॒राजीति॑ वि - राजि॑ । इति॑ ।  \newline


\textbf{Krama Paata} \newline

प्र॒जाम् तस्य॑ । प्र॒जामिति॑ प्र - जाम् । तस्या॒न्तः । अ॒न्तर् य॑न्ति । य॒न्ति॒ प्र॒जाया᳚म् । प्र॒जाया᳚म् मे । प्र॒जाया॒मिति॑ प्र - जाया᳚म् । मेऽपि॑ । अप्य॑सत् । अ॒स॒दिति॑ । इति॒ खलु॑ । खलु॒ वै । वै य॒ज्ञेन॑ । य॒ज्ञेन॒ यज॑मानः । यज॑मानो यजते । य॒ज॒ते॒ यस्य॑ । यस्यै॑कविꣳ॒॒शम् । ए॒क॒विꣳ॒॒शम॑न्त॒र्यन्ति॑ । ए॒क॒विꣳ॒॒शमित्ये॑क - विꣳ॒॒शम् । अ॒न्त॒र्यन्ति॑ प्रति॒ष्ठाम् । अ॒न्त॒र्यन्तीत्य॑न्तः - यन्ति॑ । प्र॒ति॒ष्ठाम् तस्य॑ । प्र॒ति॒ष्ठामिति॑ प्रति - स्थाम् । तस्या॒न्तः । अ॒न्तर् य॑न्ति । य॒न्ति॒ प्र॒ति॒ष्ठाया᳚म् । प्र॒ति॒ष्ठाया᳚म् मे । प्र॒ति॒ष्ठाया॒मिति॑ प्रति - स्थाया᳚म् । मेऽपि॑ । अप्य॑सत् । अ॒स॒दिति॑ । इति॒ खलु॑ । खलु॒ वै । वै य॒ज्ञेन॑ । य॒ज्ञेन॒ यज॑मानः । यज॑मानो यजते । य॒ज॒ते॒ यस्य॑ । यस्य॑ त्रिण॒वम् । त्रि॒ण॒वम॑न्त॒र्यन्ति॑ । त्रि॒ण॒वमिति॑ त्रि - न॒वम् । अ॒न्त॒र्यन्त्यृ॒तून् । अ॒न्त॒र्यन्तीत्य॑न्तः - यन्ति॑ । ऋ॒तूꣳश्च॑ । च॒ तस्य॑ । तस्य॑ नक्ष॒त्रिया᳚म् । न॒क्ष॒त्रिया᳚म् च । च॒ वि॒राज᳚म् । वि॒राज॑म॒न्तः । वि॒राज॒मिति॑ वि - राज᳚म् । अ॒न्तर् य॑न्ति । य॒न्त्यृ॒तुषु॑ । ऋ॒तुषु॑ मे । मेऽपि॑ । अप्य॑सत् । अ॒स॒न् न॒क्ष॒त्रिया॑याम् । न॒क्ष॒त्रिया॑याम् च । च॒ वि॒राजि॑ । वि॒राजीति॑ ( ) । वि॒राजीति॑ वि - राजि॑ । इति॒ खलु॑ \newline

\textbf{Jatai Paata} \newline

1. प्र॒जाम् तस्य॒ तस्य॑ प्र॒जाम् प्र॒जाम् तस्य॑ । \newline
2. प्र॒जामिति॑ प्र - जाम् । \newline
3. तस्या॒न्त र॒न्त स्तस्य॒ तस्या॒न्तः । \newline
4. अ॒न्तर् य॑न्ति यन्त्य॒न्त र॒न्तर् य॑न्ति । \newline
5. य॒न्ति॒ प्र॒जाया᳚म् प्र॒जायां᳚ ॅयन्ति यन्ति प्र॒जाया᳚म् । \newline
6. प्र॒जाया᳚म् मे मे प्र॒जाया᳚म् प्र॒जाया᳚म् मे । \newline
7. प्र॒जाया॒मिति॑ प्र - जाया᳚म् । \newline
8. मे ऽप्यपि॑ मे॒ मे ऽपि॑ । \newline
9. अप्य॑स दस॒ दप्य प्य॑सत् । \newline
10. अ॒स॒ दिती त्य॑स दस॒ दिति॑ । \newline
11. इति॒ खलु॒ खल्वितीति॒ खलु॑ । \newline
12. खलु॒ वै वै खलु॒ खलु॒ वै । \newline
13. वै य॒ज्ञेन॑ य॒ज्ञेन॒ वै वै य॒ज्ञेन॑ । \newline
14. य॒ज्ञेन॒ यज॑मानो॒ यज॑मानो य॒ज्ञेन॑ य॒ज्ञेन॒ यज॑मानः । \newline
15. यज॑मानो यजते यजते॒ यज॑मानो॒ यज॑मानो यजते । \newline
16. य॒ज॒ते॒ यस्य॒ यस्य॑ यजते यजते॒ यस्य॑ । \newline
17. यस्यै॑कविꣳ॒॒श मे॑कविꣳ॒॒शं ॅयस्य॒ यस्यै॑कविꣳ॒॒शम् । \newline
18. ए॒क॒विꣳ॒॒श म॑न्त॒र्य न्त्य॑न्त॒र्य न्त्ये॑कविꣳ॒॒श मे॑कविꣳ॒॒श म॑न्त॒र्यन्ति॑ । \newline
19. ए॒क॒विꣳ॒॒शमित्ये॑क - विꣳ॒॒शम् । \newline
20. अ॒न्त॒र्यन्ति॑ प्रति॒ष्ठाम् प्र॑ति॒ष्ठा म॑न्त॒र्य न्त्य॑न्त॒र्यन्ति॑ प्रति॒ष्ठाम् । \newline
21. अ॒न्त॒र्यन्तीत्य॑न्तः - यन्ति॑ । \newline
22. प्र॒ति॒ष्ठाम् तस्य॒ तस्य॑ प्रति॒ष्ठाम् प्र॑ति॒ष्ठाम् तस्य॑ । \newline
23. प्र॒ति॒ष्ठामिति॑ प्रति - स्थाम् । \newline
24. तस्या॒न्त र॒न्त स्तस्य॒ तस्या॒न्तः । \newline
25. अ॒न्तर् य॑न्ति यन्त्य॒न्त र॒न्तर् य॑न्ति । \newline
26. य॒न्ति॒ प्र॒ति॒ष्ठाया᳚म् प्रति॒ष्ठायां᳚ ॅयन्ति यन्ति प्रति॒ष्ठाया᳚म् । \newline
27. प्र॒ति॒ष्ठाया᳚म् मे मे प्रति॒ष्ठाया᳚म् प्रति॒ष्ठाया᳚म् मे । \newline
28. प्र॒ति॒ष्ठाया॒मिति॑ प्रति - स्थाया᳚म् । \newline
29. मे ऽप्यपि॑ मे॒ मे ऽपि॑ । \newline
30. अप्य॑स दस॒ दप्य प्य॑सत् । \newline
31. अ॒स॒ दिती त्य॑स दस॒ दिति॑ । \newline
32. इति॒ खलु॒ खल्वितीति॒ खलु॑ । \newline
33. खलु॒ वै वै खलु॒ खलु॒ वै । \newline
34. वै य॒ज्ञेन॑ य॒ज्ञेन॒ वै वै य॒ज्ञेन॑ । \newline
35. य॒ज्ञेन॒ यज॑मानो॒ यज॑मानो य॒ज्ञेन॑ य॒ज्ञेन॒ यज॑मानः । \newline
36. यज॑मानो यजते यजते॒ यज॑मानो॒ यज॑मानो यजते । \newline
37. य॒ज॒ते॒ यस्य॒ यस्य॑ यजते यजते॒ यस्य॑ । \newline
38. यस्य॑ त्रिण॒वम् त्रि॑ण॒वं ॅयस्य॒ यस्य॑ त्रिण॒वम् । \newline
39. त्रि॒ण॒व म॑न्त॒र्य न्त्य॑न्त॒र्यन्ति॑ त्रिण॒वम् त्रि॑ण॒व म॑न्त॒र्यन्ति॑ । \newline
40. त्रि॒ण॒वमिति॑ त्रि - न॒वम् । \newline
41. अ॒न्त॒र्य न्त्यृ॒तू नृ॒तू न॑न्त॒र्य न्त्य॑न्त॒र्य न्त्यृ॒तून् । \newline
42. अ॒न्त॒र्यन्तीत्य॑न्तः - यन्ति॑ । \newline
43. ऋ॒तूꣳ श्च॑ च॒ र्‌तू नृ॒तूꣳ श्च॑ । \newline
44. च॒ तस्य॒ तस्य॑ च च॒ तस्य॑ । \newline
45. तस्य॑ नक्ष॒त्रिया᳚म् नक्ष॒त्रिया॒म् तस्य॒ तस्य॑ नक्ष॒त्रिया᳚म् । \newline
46. न॒क्ष॒त्रिया᳚म् च च नक्ष॒त्रिया᳚म् नक्ष॒त्रिया᳚म् च । \newline
47. च॒ वि॒राजं॑ ॅवि॒राज॑म् च च वि॒राज᳚म् । \newline
48. वि॒राज॑ म॒न्त र॒न्तर् वि॒राजं॑ ॅवि॒राज॑ म॒न्तः । \newline
49. वि॒राज॒मिति॑ वि - राज᳚म् । \newline
50. अ॒न्तर् य॑न्ति यन्त्य॒न्त र॒न्तर् य॑न्ति । \newline
51. य॒न्त्यृ॒तुष् वृ॒तुषु॑ यन्ति यन्त्यृ॒तुषु॑ । \newline
52. ऋ॒तुषु॑ मे म ऋ॒तुष् वृ॒तुषु॑ मे । \newline
53. मे ऽप्यपि॑ मे॒ मे ऽपि॑ । \newline
54. अप्य॑स दस॒ दप्य प्य॑सत् । \newline
55. अ॒स॒न् न॒क्ष॒त्रिया॑याम् नक्ष॒त्रिया॑या मसदसम् नक्ष॒त्रिया॑याम् । \newline
56. न॒क्ष॒त्रिया॑याम् च च नक्ष॒त्रिया॑याम् नक्ष॒त्रिया॑याम् च । \newline
57. च॒ वि॒राजि॑ वि॒राजि॑ च च वि॒राजि॑ । \newline
58. वि॒राजीतीति॑ वि॒राजि॑ वि॒राजीति॑ । \newline
59. वि॒राजीति॑ वि - राजि॑ । \newline
60. इति॒ खलु॒ खल्वितीति॒ खलु॑ । \newline

\textbf{Ghana Paata } \newline

1. प्र॒जाम् तस्य॒ तस्य॑ प्र॒जाम् प्र॒जाम् तस्या॒न्त र॒न्त स्तस्य॑ प्र॒जाम् प्र॒जाम् तस्या॒न्तः । \newline
2. प्र॒जामिति॑ प्र - जाम् । \newline
3. तस्या॒न्त र॒न्त स्तस्य॒ तस्या॒न्तर् य॑न्ति यन्त्य॒न्त स्तस्य॒ तस्या॒न्तर् य॑न्ति । \newline
4. अ॒न्तर् य॑न्ति यन्त्य॒न्त र॒न्तर् य॑न्ति प्र॒जाया᳚म् प्र॒जायां᳚ ॅयन्त्य॒न्त र॒न्तर् य॑न्ति प्र॒जाया᳚म् । \newline
5. य॒न्ति॒ प्र॒जाया᳚म् प्र॒जायां᳚ ॅयन्ति यन्ति प्र॒जाया᳚म् मे मे प्र॒जायां᳚ ॅयन्ति यन्ति प्र॒जाया᳚म् मे । \newline
6. प्र॒जाया᳚म् मे मे प्र॒जाया᳚म् प्र॒जाया॒म् मे ऽप्यपि॑ मे प्र॒जाया᳚म् प्र॒जाया॒म् मे ऽपि॑ । \newline
7. प्र॒जाया॒मिति॑ प्र - जाया᳚म् । \newline
8. मे ऽप्यपि॑ मे॒ मे ऽप्य॑स दस॒ दपि॑ मे॒ मे ऽप्य॑सत् । \newline
9. अप्य॑स दस॒ दप्य प्य॑स॒ दिती त्य॑स॒ दप्य प्य॑स॒दिति॑ । \newline
10. अ॒स॒ दिती त्य॑स दस॒ दिति॒ खलु॒ खल्वि त्य॑स दस॒ दिति॒ खलु॑ । \newline
11. इति॒ खलु॒ खल्वितीति॒ खलु॒ वै वै खल्वितीति॒ खलु॒ वै । \newline
12. खलु॒ वै वै खलु॒ खलु॒ वै य॒ज्ञेन॑ य॒ज्ञेन॒ वै खलु॒ खलु॒ वै य॒ज्ञेन॑ । \newline
13. वै य॒ज्ञेन॑ य॒ज्ञेन॒ वै वै य॒ज्ञेन॒ यज॑मानो॒ यज॑मानो य॒ज्ञेन॒ वै वै य॒ज्ञेन॒ यज॑मानः । \newline
14. य॒ज्ञेन॒ यज॑मानो॒ यज॑मानो य॒ज्ञेन॑ य॒ज्ञेन॒ यज॑मानो यजते यजते॒ यज॑मानो य॒ज्ञेन॑ य॒ज्ञेन॒ यज॑मानो यजते । \newline
15. यज॑मानो यजते यजते॒ यज॑मानो॒ यज॑मानो यजते॒ यस्य॒ यस्य॑ यजते॒ यज॑मानो॒ यज॑मानो यजते॒ यस्य॑ । \newline
16. य॒ज॒ते॒ यस्य॒ यस्य॑ यजते यजते॒ यस्यै॑कविꣳ॒॒श मे॑कविꣳ॒॒शं ॅयस्य॑ यजते यजते॒ यस्यै॑कविꣳ॒॒शम् । \newline
17. यस्यै॑कविꣳ॒॒श मे॑कविꣳ॒॒शं ॅयस्य॒ यस्यै॑कविꣳ॒॒श म॑न्त॒र्य न्त्य॑न्त॒र्य न्त्ये॑कविꣳ॒॒शं ॅयस्य॒ यस्यै॑कविꣳ॒॒श म॑न्त॒र्यन्ति॑ । \newline
18. ए॒क॒विꣳ॒॒श म॑न्त॒र्य न्त्य॑न्त॒र्य न्त्ये॑कविꣳ॒॒श मे॑कविꣳ॒॒श म॑न्त॒र्यन्ति॑ प्रति॒ष्ठाम् प्र॑ति॒ष्ठा म॑न्त॒र्य न्त्ये॑कविꣳ॒॒श मे॑कविꣳ॒॒श म॑न्त॒र्यन्ति॑ प्रति॒ष्ठाम् । \newline
19. ए॒क॒विꣳ॒॒शमित्ये॑क - विꣳ॒॒शम् । \newline
20. अ॒न्त॒र्यन्ति॑ प्रति॒ष्ठाम् प्र॑ति॒ष्ठा म॑न्त॒र्य न्त्य॑न्त॒र्यन्ति॑ प्रति॒ष्ठाम् तस्य॒ तस्य॑ प्रति॒ष्ठा म॑न्त॒र्य
न्त्य॑न्त॒र्यन्ति॑ प्रति॒ष्ठाम् तस्य॑ । \newline
21. अ॒न्त॒र्यन्तीत्य॑न्तः - यन्ति॑ । \newline
22. प्र॒ति॒ष्ठाम् तस्य॒ तस्य॑ प्रति॒ष्ठाम् प्र॑ति॒ष्ठाम् तस्या॒न्त र॒न्त स्तस्य॑ प्रति॒ष्ठाम् प्र॑ति॒ष्ठाम् तस्या॒न्तः । \newline
23. प्र॒ति॒ष्ठामिति॑ प्रति - स्थाम् । \newline
24. तस्या॒न्त र॒न्त स्तस्य॒ तस्या॒न्तर् य॑न्ति यन्त्य॒न्त स्तस्य॒ तस्या॒न्तर् य॑न्ति । \newline
25. अ॒न्तर् य॑न्ति यन्त्य॒न्त र॒न्तर् य॑न्ति प्रति॒ष्ठाया᳚म् प्रति॒ष्ठायां᳚ ॅयन्त्य॒न्त र॒न्तर् य॑न्ति प्रति॒ष्ठाया᳚म् । \newline
26. य॒न्ति॒ प्र॒ति॒ष्ठाया᳚म् प्रति॒ष्ठायां᳚ ॅयन्ति यन्ति प्रति॒ष्ठाया᳚म् मे मे प्रति॒ष्ठायां᳚ ॅयन्ति यन्ति प्रति॒ष्ठाया᳚म् मे । \newline
27. प्र॒ति॒ष्ठाया᳚म् मे मे प्रति॒ष्ठाया᳚म् प्रति॒ष्ठाया॒म् मे ऽप्यपि॑ मे प्रति॒ष्ठाया᳚म् प्रति॒ष्ठाया॒म् मे ऽपि॑ । \newline
28. प्र॒ति॒ष्ठाया॒मिति॑ प्रति - स्थाया᳚म् । \newline
29. मे ऽप्यपि॑ मे॒ मे ऽप्य॑स दस॒ दपि॑ मे॒ मे ऽप्य॑सत् । \newline
30. अप्य॑स दस॒ दप्य प्य॑स॒दि तीत्य॑स॒ दप्य प्य॑स॒ दिति॑ । \newline
31. अ॒स॒ दिती त्य॑स दस॒ दिति॒ खलु॒ खल्वि त्य॑स दस॒ दिति॒ खलु॑ । \newline
32. इति॒ खलु॒ खल्वितीति॒ खलु॒ वै वै खल्वितीति॒ खलु॒ वै । \newline
33. खलु॒ वै वै खलु॒ खलु॒ वै य॒ज्ञेन॑ य॒ज्ञेन॒ वै खलु॒ खलु॒ वै य॒ज्ञेन॑ । \newline
34. वै य॒ज्ञेन॑ य॒ज्ञेन॒ वै वै य॒ज्ञेन॒ यज॑मानो॒ यज॑मानो य॒ज्ञेन॒ वै वै य॒ज्ञेन॒ यज॑मानः । \newline
35. य॒ज्ञेन॒ यज॑मानो॒ यज॑मानो य॒ज्ञेन॑ य॒ज्ञेन॒ यज॑मानो यजते यजते॒ यज॑मानो य॒ज्ञेन॑ य॒ज्ञेन॒ यज॑मानो यजते । \newline
36. यज॑मानो यजते यजते॒ यज॑मानो॒ यज॑मानो यजते॒ यस्य॒ यस्य॑ यजते॒ यज॑मानो॒ यज॑मानो यजते॒ यस्य॑ । \newline
37. य॒ज॒ते॒ यस्य॒ यस्य॑ यजते यजते॒ यस्य॑ त्रिण॒वम् त्रि॑ण॒वं ॅयस्य॑ यजते यजते॒ यस्य॑ त्रिण॒वम् । \newline
38. यस्य॑ त्रिण॒वम् त्रि॑ण॒वं ॅयस्य॒ यस्य॑ त्रिण॒व म॑न्त॒र्य न्त्य॑न्त॒र्यन्ति॑ त्रिण॒वं ॅयस्य॒ यस्य॑ त्रिण॒व म॑न्त॒र्यन्ति॑ । \newline
39. त्रि॒ण॒व म॑न्त॒र्य न्त्य॑न्त॒र्यन्ति॑ त्रिण॒वम् त्रि॑ण॒व म॑न्त॒र्य न्त्यृ॒तू नृ॒तू न॑न्त॒र्यन्ति॑ त्रिण॒वम् त्रि॑ण॒व म॑न्त॒र्य न्त्यृ॒तून् । \newline
40. त्रि॒ण॒वमिति॑ त्रि - न॒वम् । \newline
41. अ॒न्त॒र्य न्त्यृ॒तू नृ॒तू न॑न्त॒र्य न्त्य॑न्त॒र्य न्त्यृ॒तूꣳ श्च॑ च॒ र्‌तू न॑न्त॒र्य न्त्य॑न्त॒र्य न्त्यृ॒तूꣳ श्च॑ । \newline
42. अ॒न्त॒र्यन्तीत्य॑न्तः - यन्ति॑ । \newline
43. ऋ॒तूꣳ श्च॑ च॒ र्‌तू नृ॒तूꣳ श्च॒ तस्य॒ तस्य॑ च॒ र्‌तू नृ॒तूꣳ श्च॒ तस्य॑ । \newline
44. च॒ तस्य॒ तस्य॑ च च॒ तस्य॑ नक्ष॒त्रिया᳚म् नक्ष॒त्रिया॒म् तस्य॑ च च॒ तस्य॑ नक्ष॒त्रिया᳚म् । \newline
45. तस्य॑ नक्ष॒त्रिया᳚म् नक्ष॒त्रिया॒म् तस्य॒ तस्य॑ नक्ष॒त्रिया᳚म् च च नक्ष॒त्रिया॒म् तस्य॒ तस्य॑ नक्ष॒त्रिया᳚म् च । \newline
46. न॒क्ष॒त्रिया᳚म् च च नक्ष॒त्रिया᳚म् नक्ष॒त्रिया᳚म् च वि॒राजं॑ ॅवि॒राज॑म् च नक्ष॒त्रिया᳚म् नक्ष॒त्रिया᳚म् च वि॒राज᳚म् । \newline
47. च॒ वि॒राजं॑ ॅवि॒राज॑म् च च वि॒राज॑ म॒न्त र॒न्तर् वि॒राज॑म् च च वि॒राज॑ म॒न्तः । \newline
48. वि॒राज॑ म॒न्त र॒न्तर् वि॒राजं॑ ॅवि॒राज॑ म॒न्तर् य॑न्ति यन्त्य॒न्तर् वि॒राजं॑ ॅवि॒राज॑ म॒न्तर् य॑न्ति । \newline
49. वि॒राज॒मिति॑ वि - राज᳚म् । \newline
50. अ॒न्तर् य॑न्ति यन्त्य॒न्त र॒न्तर् य॑न्त्यृ॒तु ष्वृ॒तुषु॑ यन्त्य॒न्त र॒न्तर् य॑न्त्यृ॒तुषु॑ । \newline
51. य॒न्त्यृ॒तु ष्वृ॒तुषु॑ यन्ति यन्त्यृ॒तुषु॑ मे म ऋ॒तुषु॑ यन्ति यन्त्यृ॒तुषु॑ मे । \newline
52. ऋ॒तुषु॑ मे म ऋ॒तु ष्वृ॒तुषु॒ मे ऽप्यपि॑ म ऋ॒तु ष्वृ॒तुषु॒ मे ऽपि॑ । \newline
53. मे ऽप्यपि॑ मे॒ मे ऽप्य॑स दस॒ दपि॑ मे॒ मे ऽप्य॑सत् । \newline
54. अप्य॑स दस॒ दप्य प्य॑सन् नक्ष॒त्रिया॑याम् नक्ष॒त्रिया॑या मस॒ दप्य प्य॑सन् नक्ष॒त्रिया॑याम् । \newline
55. अ॒स॒न् न॒क्ष॒त्रिया॑याम् नक्ष॒त्रिया॑या मस दसन् नक्ष॒त्रिया॑याम् च च नक्ष॒त्रिया॑या मस दसन् नक्ष॒त्रिया॑याम् च । \newline
56. न॒क्ष॒त्रिया॑याम् च च नक्ष॒त्रिया॑याम् नक्ष॒त्रिया॑याम् च वि॒राजि॑ वि॒राजि॑ च नक्ष॒त्रिया॑याम् नक्ष॒त्रिया॑याम् च वि॒राजि॑ । \newline
57. च॒ वि॒राजि॑ वि॒राजि॑ च च वि॒राजीतीति॑ वि॒राजि॑ च च वि॒राजीति॑ । \newline
58. वि॒राजीतीति॑ वि॒राजि॑ वि॒राजीति॒ खलु॒ खल्विति॑ वि॒राजि॑ वि॒राजीति॒ खलु॑ । \newline
59. वि॒राजीति॑ वि - राजि॑ । \newline
60. इति॒ खलु॒ खल्वितीति॒ खलु॒ वै वै खल्वितीति॒ खलु॒ वै । \newline
\pagebreak
\markright{ TS 7.1.3.3  \hfill https://www.vedavms.in \hfill}

\section{ TS 7.1.3.3 }

\textbf{TS 7.1.3.3 } \newline
\textbf{Samhita Paata} \newline

खलु॒ वै य॒ज्ञेन॒ यज॑मानो यजते॒ यस्य॑ त्रयस्त्रिꣳ॒॒शम॑न्त॒र्यन्ति॑ दे॒वता॒स्तस्या॒न्तर्य॑न्ति दे॒वता॑सु॒ मेऽप्य॑स॒दिति॒ खलु॒ वै य॒ज्ञेन॒ यज॑मानो यजते॒ यो वै स्तोमा॑नामव॒मं प॑र॒मतां॒ गच्छ॑न्तं॒ ॅवेद॑ पर॒मता॑मे॒व ग॑च्छति त्रि॒वृद्वै स्तोमा॑नामव॒मस्त्रि॒वृत् प॑र॒मो य ए॒वं ॅवेद॑ पर॒मता॑मे॒व ग॑च्छति ॥ \newline

\textbf{Pada Paata} \newline

खलु॑ । वै । य॒ज्ञेन॑ । यज॑मानः । य॒ज॒ते॒ । यस्य॑ । त्र॒य॒स्त्रिꣳ॒॒शमिति॑ त्रयः - त्रिꣳ॒॒शम् । अ॒न्त॒र्यन्तित्य॑न्तः - यन्ति॑ । दे॒वताः᳚ । तस्य॑ । अ॒न्तः । य॒न्ति॒ । दे॒वता॑सु । मे॒ । अपीति॑ । अ॒स॒त् । इति॑ । खलु॑ । वै । य॒ज्ञेन॑ । यज॑मानः । य॒ज॒ते॒ । यः । वै । स्तोमा॑नाम् । अ॒व॒मम् । प॒र॒मता᳚म् । गच्छ॑न्तम् । वेद॑ । प॒र॒मता᳚म् । ए॒व । ग॒च्छ॒ति॒ । त्रि॒वृदिति॑ त्रि - वृत् । वै । स्तोमा॑नाम् । अ॒व॒मः । त्रि॒वृदिति॑ त्रि - वृत् । प॒र॒मः । यः । ए॒वम् । वेद॑ । प॒र॒मता᳚म् । ए॒व । ग॒च्छ॒ति॒ ॥  \newline


\textbf{Krama Paata} \newline

खलु॒ वै । वै य॒ज्ञेन॑ । य॒ज्ञेन॒ यज॑मानः । यज॑मानो यजते । य॒ज॒ते॒ यस्य॑ । यस्य॑ त्रयस्त्रिꣳ॒॒शम् । त्र॒य॒स्त्रिꣳ॒॒शम॑न्त॒र्यन्ति॑ । त्र॒य॒स्त्रिꣳ॒॒शमिति॑ त्रयः - त्रिꣳ॒॒शम् । अ॒न्त॒र्यन्ति॑ दे॒वताः᳚ । अ॒न्त॒र्यन्तीत्य॑न्तः - यन्ति॑ । दे॒वता॒स्तस्य॑ । तस्या॒न्तः । अ॒न्तर् य॑न्ति । य॒न्ति॒ दे॒वता॑सु । दे॒वता॑सु मे । मेऽपि॑ । अप्य॑सत् । अ॒स॒दिति॑ । इति॒ खलु॑ । खलु॒ वै । वै य॒ज्ञेन॑ । य॒ज्ञेन॒ यज॑मानः । यज॑मानो यजते । य॒ज॒ते॒ यः । यो वै । वै स्तोमा॑नाम् । स्तोमा॑नामव॒मम् । अ॒व॒मम् प॑र॒मता᳚म् । प॒र॒मता॒म् गच्छ॑न्तम् । गच्छ॑न्त॒म् ॅवेद॑ । वेद॑ पर॒मता᳚म् । प॒र॒मता॑मे॒व । ए॒व ग॑च्छति । ग॒च्छ॒ति॒ त्रि॒वृत् । त्रि॒वृद् वै । त्रि॒वृदिति॑ त्रि - वृत् । वै स्तोमा॑नाम् । स्तोमा॑नामव॒मः । अ॒व॒मस्त्रि॒वृत् । त्रि॒वृत् प॑र॒मः । त्रि॒वृदिति॑ त्रि - वृत् । प॒र॒मो यः । य ए॒वम् । ए॒वम् ॅवेद॑ । वेद॑ पर॒मता᳚म् । प॒र॒मता॑मे॒व । ए॒व ग॑च्छति । ग॒च्छ॒तीति॑ गच्छति । \newline

\textbf{Jatai Paata} \newline

1. खलु॒ वै वै खलु॒ खलु॒ वै । \newline
2. वै य॒ज्ञेन॑ य॒ज्ञेन॒ वै वै य॒ज्ञेन॑ । \newline
3. य॒ज्ञेन॒ यज॑मानो॒ यज॑मानो य॒ज्ञेन॑ य॒ज्ञेन॒ यज॑मानः । \newline
4. यज॑मानो यजते यजते॒ यज॑मानो॒ यज॑मानो यजते । \newline
5. य॒ज॒ते॒ यस्य॒ यस्य॑ यजते यजते॒ यस्य॑ । \newline
6. यस्य॑ त्रयस्त्रिꣳ॒॒शम् त्र॑यस्त्रिꣳ॒॒शं ॅयस्य॒ यस्य॑ त्रयस्त्रिꣳ॒॒शम् । \newline
7. त्र॒य॒स्त्रिꣳ॒॒श म॑न्त॒र्य न्त्य॑न्त॒र्यन्ति॑ त्रयस्त्रिꣳ॒॒शम् त्र॑यस्त्रिꣳ॒॒श म॑न्त॒र्यन्ति॑ । \newline
8. त्र॒य॒स्त्रिꣳ॒॒शमिति॑ त्रयः - त्रिꣳ॒॒शम् । \newline
9. अ॒न्त॒र्यन्ति॑ दे॒वता॑ दे॒वता॑ अन्त॒र्य न्त्य॑न्त॒र्यन्ति॑ दे॒वताः᳚ । \newline
10. अ॒न्त॒र्यन्तीत्य॑न्तः - यन्ति॑ । \newline
11. दे॒वता॒ स्तस्य॒ तस्य॑ दे॒वता॑ दे॒वता॒ स्तस्य॑ । \newline
12. तस्या॒न्त र॒न्त स्तस्य॒ तस्या॒न्तः । \newline
13. अ॒न्तर् य॑न्ति यन्त्य॒न्त र॒न्तर् य॑न्ति । \newline
14. य॒न्ति॒ दे॒वता॑सु दे॒वता॑सु यन्ति यन्ति दे॒वता॑सु । \newline
15. दे॒वता॑सु मे मे दे॒वता॑सु दे॒वता॑सु मे । \newline
16. मे ऽप्यपि॑ मे॒ मे ऽपि॑ । \newline
17. अप्य॑स दस॒ दप्य प्य॑सत् । \newline
18. अ॒स॒ दिती त्य॑स दस॒ दिति॑ । \newline
19. इति॒ खलु॒ खल्वितीति॒ खलु॑ । \newline
20. खलु॒ वै वै खलु॒ खलु॒ वै । \newline
21. वै य॒ज्ञेन॑ य॒ज्ञेन॒ वै वै य॒ज्ञेन॑ । \newline
22. य॒ज्ञेन॒ यज॑मानो॒ यज॑मानो य॒ज्ञेन॑ य॒ज्ञेन॒ यज॑मानः । \newline
23. यज॑मानो यजते यजते॒ यज॑मानो॒ यज॑मानो यजते । \newline
24. य॒ज॒ते॒ यो यो य॑जते यजते॒ यः । \newline
25. यो वै वै यो यो वै । \newline
26. वै स्तोमा॑नाꣳ॒॒ स्तोमा॑नां॒ ॅवै वै स्तोमा॑नाम् । \newline
27. स्तोमा॑ना मव॒म म॑व॒मꣳ स्तोमा॑नाꣳ॒॒ स्तोमा॑ना मव॒मम् । \newline
28. अ॒व॒मम् प॑र॒मता᳚म् पर॒मता॑ मव॒म म॑व॒मम् प॑र॒मता᳚म् । \newline
29. प॒र॒मता॒म् गच्छ॑न्त॒म् गच्छ॑न्तम् पर॒मता᳚म् पर॒मता॒म् गच्छ॑न्तम् । \newline
30. गच्छ॑न्तं॒ ॅवेद॒ वेद॒ गच्छ॑न्त॒म् गच्छ॑न्तं॒ ॅवेद॑ । \newline
31. वेद॑ पर॒मता᳚म् पर॒मतां॒ ॅवेद॒ वेद॑ पर॒मता᳚म् । \newline
32. प॒र॒मता॑ मे॒वैव प॑र॒मता᳚म् पर॒मता॑ मे॒व । \newline
33. ए॒व ग॑च्छति गच्छ त्ये॒वैव ग॑च्छति । \newline
34. ग॒च्छ॒ति॒ त्रि॒वृत् त्रि॒वृद् ग॑च्छति गच्छति त्रि॒वृत् । \newline
35. त्रि॒वृद् वै वै त्रि॒वृत् त्रि॒वृद् वै । \newline
36. त्रि॒वृदिति॑ त्रि - वृत् । \newline
37. वै स्तोमा॑नाꣳ॒॒ स्तोमा॑नां॒ ॅवै वै स्तोमा॑नाम् । \newline
38. स्तोमा॑ना मव॒मो॑ ऽव॒मः स्तोमा॑नाꣳ॒॒ स्तोमा॑ना मव॒मः । \newline
39. अ॒व॒म स्त्रि॒वृत् त्रि॒वृ द॑व॒मो॑ ऽव॒म स्त्रि॒वृत् । \newline
40. त्रि॒वृत् प॑र॒मः प॑र॒म स्त्रि॒वृत् त्रि॒वृत् प॑र॒मः । \newline
41. त्रि॒वृदिति॑ त्रि - वृत् । \newline
42. प॒र॒मो यो यः प॑र॒मः प॑र॒मो यः । \newline
43. य ए॒व मे॒वं ॅयो य ए॒वम् । \newline
44. ए॒वं ॅवेद॒ वेदै॒व मे॒वं ॅवेद॑ । \newline
45. वेद॑ पर॒मता᳚म् पर॒मतां॒ ॅवेद॒ वेद॑ पर॒मता᳚म् । \newline
46. प॒र॒मता॑ मे॒वैव प॑र॒मता᳚म् पर॒मता॑ मे॒व । \newline
47. ए॒व ग॑च्छति गच्छ त्ये॒वैव ग॑च्छति । \newline
48. ग॒च्छ॒तीति॑ गच्छति । \newline

\textbf{Ghana Paata } \newline

1. खलु॒ वै वै खलु॒ खलु॒ वै य॒ज्ञेन॑ य॒ज्ञेन॒ वै खलु॒ खलु॒ वै य॒ज्ञेन॑ । \newline
2. वै य॒ज्ञेन॑ य॒ज्ञेन॒ वै वै य॒ज्ञेन॒ यज॑मानो॒ यज॑मानो य॒ज्ञेन॒ वै वै य॒ज्ञेन॒ यज॑मानः । \newline
3. य॒ज्ञेन॒ यज॑मानो॒ यज॑मानो य॒ज्ञेन॑ य॒ज्ञेन॒ यज॑मानो यजते यजते॒ यज॑मानो य॒ज्ञेन॑ य॒ज्ञेन॒ यज॑मानो यजते । \newline
4. यज॑मानो यजते यजते॒ यज॑मानो॒ यज॑मानो यजते॒ यस्य॒ यस्य॑ यजते॒ यज॑मानो॒ यज॑मानो यजते॒ यस्य॑ । \newline
5. य॒ज॒ते॒ यस्य॒ यस्य॑ यजते यजते॒ यस्य॑ त्रयस्त्रिꣳ॒॒शम् त्र॑यस्त्रिꣳ॒॒शं ॅयस्य॑ यजते यजते॒ यस्य॑ त्रयस्त्रिꣳ॒॒शम् । \newline
6. यस्य॑ त्रयस्त्रिꣳ॒॒शम् त्र॑यस्त्रिꣳ॒॒शं ॅयस्य॒ यस्य॑ त्रयस्त्रिꣳ॒॒श म॑न्त॒र्य न्त्य॑न्त॒र्यन्ति॑ त्रयस्त्रिꣳ॒॒शं ॅयस्य॒ यस्य॑ त्रयस्त्रिꣳ॒॒श म॑न्त॒र्यन्ति॑ । \newline
7. त्र॒य॒स्त्रिꣳ॒॒श म॑न्त॒र्य न्त्य॑न्त॒र्यन्ति॑ त्रयस्त्रिꣳ॒॒शम् त्र॑यस्त्रिꣳ॒॒श म॑न्त॒र्यन्ति॑ दे॒वता॑ दे॒वता॑ अन्त॒र्यन्ति॑ त्रयस्त्रिꣳ॒॒शम् त्र॑यस्त्रिꣳ॒॒श म॑न्त॒र्यन्ति॑ दे॒वताः᳚ । \newline
8. त्र॒य॒स्त्रिꣳ॒॒शमिति॑ त्रयः - त्रिꣳ॒॒शम् । \newline
9. अ॒न्त॒र्यन्ति॑ दे॒वता॑ दे॒वता॑ अन्त॒र्य न्त्य॑न्त॒र्यन्ति॑ दे॒वता॒ स्तस्य॒ तस्य॑ दे॒वता॑ अन्त॒र्य न्त्य॑न्त॒र्यन्ति॑ दे॒वता॒ स्तस्य॑ । \newline
10. अ॒न्त॒र्यन्तीत्य॑न्तः - यन्ति॑ । \newline
11. दे॒वता॒ स्तस्य॒ तस्य॑ दे॒वता॑ दे॒वता॒ स्तस्या॒न्त र॒न्त स्तस्य॑ दे॒वता॑ दे॒वता॒ स्तस्या॒न्तः । \newline
12. तस्या॒न्त र॒न्त स्तस्य॒ तस्या॒न्तर् य॑न्ति यन्त्य॒न्त स्तस्य॒ तस्या॒न्तर् य॑न्ति । \newline
13. अ॒न्तर् य॑न्ति यन्त्य॒न्त र॒न्तर् य॑न्ति दे॒वता॑सु दे॒वता॑सु यन्त्य॒न्त र॒न्तर् य॑न्ति दे॒वता॑सु । \newline
14. य॒न्ति॒ दे॒वता॑सु दे॒वता॑सु यन्ति यन्ति दे॒वता॑सु मे मे दे॒वता॑सु यन्ति यन्ति दे॒वता॑सु मे । \newline
15. दे॒वता॑सु मे मे दे॒वता॑सु दे॒वता॑सु॒ मे ऽप्यपि॑ मे दे॒वता॑सु दे॒वता॑सु॒ मे ऽपि॑ । \newline
16. मे ऽप्यपि॑ मे॒ मे ऽप्य॑स दस॒ दपि॑ मे॒ मे ऽप्य॑सत् । \newline
17. अप्य॑स दस॒ दप्य प्य॑स॒दिती त्य॑स॒ दप्य प्य॑स॒ दिति॑ । \newline
18. अ॒स॒ दिती त्य॑स दस॒ दिति॒ खलु॒ खल्वि त्य॑स दस॒ दिति॒ खलु॑ । \newline
19. इति॒ खलु॒ खल्वितीति॒ खलु॒ वै वै खल्वितीति॒ खलु॒ वै । \newline
20. खलु॒ वै वै खलु॒ खलु॒ वै य॒ज्ञेन॑ य॒ज्ञेन॒ वै खलु॒ खलु॒ वै य॒ज्ञेन॑ । \newline
21. वै य॒ज्ञेन॑ य॒ज्ञेन॒ वै वै य॒ज्ञेन॒ यज॑मानो॒ यज॑मानो य॒ज्ञेन॒ वै वै य॒ज्ञेन॒ यज॑मानः । \newline
22. य॒ज्ञेन॒ यज॑मानो॒ यज॑मानो य॒ज्ञेन॑ य॒ज्ञेन॒ यज॑मानो यजते यजते॒ यज॑मानो य॒ज्ञेन॑ य॒ज्ञेन॒ यज॑मानो यजते । \newline
23. यज॑मानो यजते यजते॒ यज॑मानो॒ यज॑मानो यजते॒ यो यो य॑जते॒ यज॑मानो॒ यज॑मानो यजते॒ यः । \newline
24. य॒ज॒ते॒ यो यो य॑जते यजते॒ यो वै वै यो य॑जते यजते॒ यो वै । \newline
25. यो वै वै यो यो वै स्तोमा॑नाꣳ॒॒ स्तोमा॑नां॒ ॅवै यो यो वै स्तोमा॑नाम् । \newline
26. वै स्तोमा॑नाꣳ॒॒ स्तोमा॑नां॒ ॅवै वै स्तोमा॑ना मव॒म म॑व॒मꣳ स्तोमा॑नां॒ ॅवै वै स्तोमा॑ना मव॒मम् । \newline
27. स्तोमा॑ना मव॒म म॑व॒मꣳ स्तोमा॑नाꣳ॒॒ स्तोमा॑ना मव॒मम् प॑र॒मता᳚म् पर॒मता॑ मव॒मꣳ स्तोमा॑नाꣳ॒॒ स्तोमा॑ना मव॒मम् प॑र॒मता᳚म् । \newline
28. अ॒व॒मम् प॑र॒मता᳚म् पर॒मता॑ मव॒म म॑व॒मम् प॑र॒मता॒म् गच्छ॑न्त॒म् गच्छ॑न्तम् पर॒मता॑ मव॒म म॑व॒मम् प॑र॒मता॒म् गच्छ॑न्तम् । \newline
29. प॒र॒मता॒म् गच्छ॑न्त॒म् गच्छ॑न्तम् पर॒मता᳚म् पर॒मता॒म् गच्छ॑न्तं॒ ॅवेद॒ वेद॒ गच्छ॑न्तम् पर॒मता᳚म् पर॒मता॒म् गच्छ॑न्तं॒ ॅवेद॑ । \newline
30. गच्छ॑न्तं॒ ॅवेद॒ वेद॒ गच्छ॑न्त॒म् गच्छ॑न्तं॒ ॅवेद॑ पर॒मता᳚म् पर॒मतां॒ ॅवेद॒ गच्छ॑न्त॒म् गच्छ॑न्तं॒ ॅवेद॑ पर॒मता᳚म् । \newline
31. वेद॑ पर॒मता᳚म् पर॒मतां॒ ॅवेद॒ वेद॑ पर॒मता॑ मे॒वैव प॑र॒मतां॒ ॅवेद॒ वेद॑ पर॒मता॑ मे॒व । \newline
32. प॒र॒मता॑ मे॒वैव प॑र॒मता᳚म् पर॒मता॑ मे॒व ग॑च्छति गच्छ त्ये॒व प॑र॒मता᳚म् पर॒मता॑ मे॒व ग॑च्छति । \newline
33. ए॒व ग॑च्छति गच्छ त्ये॒वैव ग॑च्छति त्रि॒वृत् त्रि॒वृद् ग॑च्छ त्ये॒वैव ग॑च्छति त्रि॒वृत् । \newline
34. ग॒च्छ॒ति॒ त्रि॒वृत् त्रि॒वृद् ग॑च्छति गच्छति त्रि॒वृद् वै वै त्रि॒वृद् ग॑च्छति गच्छति त्रि॒वृद् वै । \newline
35. त्रि॒वृद् वै वै त्रि॒वृत् त्रि॒वृद् वै स्तोमा॑नाꣳ॒॒ स्तोमा॑नां॒ ॅवै त्रि॒वृत् त्रि॒वृद् वै स्तोमा॑नाम् । \newline
36. त्रि॒वृदिति॑ त्रि - वृत् । \newline
37. वै स्तोमा॑नाꣳ॒॒ स्तोमा॑नां॒ ॅवै वै स्तोमा॑ना मव॒मो॑ ऽव॒मः स्तोमा॑नां॒ ॅवै वै स्तोमा॑ना मव॒मः । \newline
38. स्तोमा॑ना मव॒मो॑ ऽव॒मः स्तोमा॑नाꣳ॒॒ स्तोमा॑ना मव॒म स्त्रि॒वृत् त्रि॒वृ द॑व॒मः स्तोमा॑नाꣳ॒॒ स्तोमा॑ना मव॒म स्त्रि॒वृत् । \newline
39. अ॒व॒म स्त्रि॒वृत् त्रि॒वृ द॑व॒मो॑ ऽव॒म स्त्रि॒वृत् प॑र॒मः प॑र॒म स्त्रि॒वृ द॑व॒मो॑ ऽव॒म स्त्रि॒वृत् प॑र॒मः । \newline
40. त्रि॒वृत् प॑र॒मः प॑र॒म स्त्रि॒वृत् त्रि॒वृत् प॑र॒मो यो यः प॑र॒म स्त्रि॒वृत् त्रि॒वृत् प॑र॒मो यः । \newline
41. त्रि॒वृदिति॑ त्रि - वृत् । \newline
42. प॒र॒मो यो यः प॑र॒मः प॑र॒मो य ए॒व मे॒वं ॅयः प॑र॒मः प॑र॒मो य ए॒वम् । \newline
43. य ए॒व मे॒वं ॅयो य ए॒वं ॅवेद॒ वेदै॒वं ॅयो य ए॒वं ॅवेद॑ । \newline
44. ए॒वं ॅवेद॒ वेदै॒व मे॒वं ॅवेद॑ पर॒मता᳚म् पर॒मतां॒ ॅवेदै॒व मे॒वं ॅवेद॑ पर॒मता᳚म् । \newline
45. वेद॑ पर॒मता᳚म् पर॒मतां॒ ॅवेद॒ वेद॑ पर॒मता॑ मे॒वैव प॑र॒मतां॒ ॅवेद॒ वेद॑ पर॒मता॑ मे॒व । \newline
46. प॒र॒मता॑ मे॒वैव प॑र॒मता᳚म् पर॒मता॑ मे॒व ग॑च्छति गच्छ त्ये॒व प॑र॒मता᳚म् पर॒मता॑ मे॒व ग॑च्छति । \newline
47. ए॒व ग॑च्छति गच्छ त्ये॒वैव ग॑च्छति । \newline
48. ग॒च्छ॒तीति॑ गच्छति । \newline
\pagebreak
\markright{ TS 7.1.4.1  \hfill https://www.vedavms.in \hfill}

\section{ TS 7.1.4.1 }

\textbf{TS 7.1.4.1 } \newline
\textbf{Samhita Paata} \newline

अङ्गि॑रसो॒ वै स॒त्रमा॑सत॒ ते सु॑व॒र्गं ॅलो॒कमा॑य॒न् तेषाꣳ॑ ह॒विष्माꣳ॑श्च हवि॒ष्कृच्चा॑ऽहीयेतां॒ ताव॑कामयेताꣳ सुव॒र्गं ॅलो॒कमि॑या॒वेति॒ तावे॒तं द्वि॑रा॒त्रम॑पश्यतां॒ तमाऽह॑रतां॒ तेना॑यजेतां॒ ततो॒ वै तौ सु॑व॒र्गं ॅलो॒कमै॑तां॒ ॅय ए॒वं ॅवि॒द्वान् द्वि॑रा॒त्रेण॒ यज॑ते सुव॒र्गमे॒व लो॒कमे॑ति॒ तावैतां॒ पूर्वे॒णाह्ना ऽग॑च्छता॒मुत्त॑रेणा - [  ] \newline

\textbf{Pada Paata} \newline

अङ्गि॑रसः । वै । स॒त्रम् । आ॒स॒त॒ । ते । सु॒व॒र्गमिति॑ सुवः - गम् । लो॒कम् । आ॒य॒न्न् । तेषा᳚म् । ह॒विष्मान्॑ । च॒ । ह॒वि॒ष्कृदिति॑ हविः - कृत् । च॒ । अ॒ही॒ये॒ता॒म् । तौ । अ॒का॒म॒ये॒ता॒म् । सु॒व॒र्गमिति॑ सुवः - गम् । लो॒कम् । इ॒या॒व॒ । इति॑ । तौ । ए॒तम् । द्वि॒रा॒त्रमिति॑ द्वि - रा॒त्रम् । अ॒प॒श्य॒ता॒म् । तम् । एति॑ । अ॒ह॒र॒ता॒म् । तेन॑ । अ॒य॒जे॒ता॒म् । ततः॑ । वै । तौ । सु॒व॒र्गमिति॑ सुवः - गम् । लो॒कम् । ऐ॒ता॒म् । यः । ए॒वम् । वि॒द्वान् । द्वि॒रा॒त्रेणेति॑ द्वि - रा॒त्रेण॑ । यज॑ते । सु॒व॒र्गमिति॑ सुवः - गम् । ए॒व । लो॒कम् । ए॒ति॒ । तौ । ऐता᳚म् । पूर्वे॑ण । अह्ना᳚ । अग॑च्छताम् । उत्त॑रे॒णेत्युत् - त॒रे॒ण॒ ।  \newline


\textbf{Krama Paata} \newline

अङ्‍गि॑रसो॒ वै । वै स॒त्रम् । स॒त्रमा॑सत । आ॒स॒त॒ ते । ते सु॑व॒र्गम् । सु॒व॒र्गम् ॅलो॒कम् । सु॒व॒र्गमिति॑ सुवः - गम् । लो॒कमा॑यन्न् । आ॒य॒न् तेषा᳚म् । तेषाꣳ॑ ह॒विष्मान्॑ । ह॒विष्माꣳ॑श्च । च॒ ह॒वि॒ष्कृत् । 
ह॒वि॒ष्कृच् च॑ । ह॒वि॒ष्कृदिति॑ हविः - कृत् । चा॒ही॒ये॒ता॒म् । अ॒ही॒ये॒ता॒म् तौ । ताव॑कामयेताम् । अ॒का॒म॒ये॒ताꣳ॒॒ सु॒व॒र्गम् । सु॒व॒र्गम् ॅलो॒कम् । सु॒व॒र्गमिति॑ सुवः - गम् । लो॒कमि॑याव । इ॒या॒वेति॑ । इति॒ तौ । तावे॒तम् । ए॒तम् द्वि॑रा॒त्रम् । द्वि॒रा॒त्रम॑पश्यताम् । द्वि॒रा॒त्रमिति॑ द्वि - रा॒त्रम् । अ॒प॒श्य॒ता॒म् तम् । तमा । आऽह॑रताम् । अ॒ह॒र॒ता॒म् तेन॑ । तेना॑यजेताम् । अ॒य॒जे॒ता॒म् ततः॑ । ततो॒ वै । वै तौ । तौ सु॑व॒र्गम् । सु॒व॒र्गम् ॅलो॒कम् । सु॒व॒र्गमिति॑ सुवः - गम् । लो॒कमै॑ताम् । ऐ॒ता॒म् ॅयः । य ए॒वम् । ए॒वम् ॅवि॒द्वान् । वि॒द्वान् द्वि॑रा॒त्रेण॑ । द्वि॒रा॒त्रेण॒ यज॑ते । द्वि॒रा॒त्रेणेति॑ द्वि - रा॒त्रेण॑ । यज॑ते सुव॒र्गम् । सु॒व॒र्गमे॒व । सु॒व॒र्गमिति॑ सुवः - गम् । ए॒व लो॒कम् । लो॒कमे॑ति । ए॒ति॒ तौ । तावैता᳚म् । ऐता॒म् पूर्वे॑ण । पूर्वे॒णाह्ना᳚ । अह्नाऽग॑च्छताम् । अग॑च्छता॒मुत्त॑रेण । उत्त॑रेणाभिप्ल॒वः । उत्त॑रे॒णेत्युत् - त॒रे॒ण॒ \newline

\textbf{Jatai Paata} \newline

1. अङ्गि॑रसो॒ वै वा अङ्गि॑र॒सो ऽङ्गि॑रसो॒ वै । \newline
2. वै स॒त्रꣳ स॒त्रं ॅवै वै स॒त्रम् । \newline
3. स॒त्र मा॑सता सत स॒त्रꣳ स॒त्र मा॑सत । \newline
4. आ॒स॒त॒ ते त आ॑सता सत॒ ते । \newline
5. ते सु॑व॒र्गꣳ सु॑व॒र्गम् ते ते सु॑व॒र्गम् । \newline
6. सु॒व॒र्गम् ॅलो॒कम् ॅलो॒कꣳ सु॑व॒र्गꣳ सु॑व॒र्गम् ॅलो॒कम् । \newline
7. सु॒व॒र्गमिति॑ सुवः - गम् । \newline
8. लो॒क मा॑यन् नायन् ॅलो॒कम् ॅलो॒क मा॑यन्न् । \newline
9. आ॒य॒न् तेषा॒म् तेषा॑ मायन् नाय॒न् तेषा᳚म् । \newline
10. तेषाꣳ॑ ह॒विष्मान्॑. ह॒विष्मा॒न् तेषा॒म् तेषाꣳ॑ ह॒विष्मान्॑ । \newline
11. ह॒विष्माꣳ॑श्च च ह॒विष्मान्॑. ह॒विष्माꣳ॑श्च । \newline
12. च॒ ह॒वि॒ष्कृ द्ध॑वि॒ष्कृच् च॑ च हवि॒ष्कृत् । \newline
13. ह॒वि॒ष्कृच् च॑ च हवि॒ष्कृ द्ध॑वि॒ष्कृच् च॑ । \newline
14. ह॒वि॒ष्कृदिति॑ हविः - कृत् । \newline
15. चा॒ही॒ये॒ता॒ म॒ही॒ये॒ता॒म् च॒ चा॒ही॒ये॒ता॒म् । \newline
16. अ॒ही॒ये॒ता॒म् तौ ता व॑हीयेता महीयेता॒म् तौ । \newline
17. ता व॑कामयेता मकामयेता॒म् तौ ता व॑कामयेताम् । \newline
18. अ॒का॒म॒ये॒ताꣳ॒॒ सु॒व॒र्गꣳ सु॑व॒र्ग म॑कामयेता मकामयेताꣳ सुव॒र्गम् । \newline
19. सु॒व॒र्गम् ॅलो॒कम् ॅलो॒कꣳ सु॑व॒र्गꣳ सु॑व॒र्गम् ॅलो॒कम् । \newline
20. सु॒व॒र्गमिति॑ सुवः - गम् । \newline
21. लो॒क मि॑या वेयाव लो॒कम् ॅलो॒क मि॑याव । \newline
22. इ॒या॒वे तीती॑या वेया॒वेति॑ । \newline
23. इति॒ तौ ता वितीति॒ तौ । \newline
24. ता वे॒त मे॒तम् तौ ता वे॒तम् । \newline
25. ए॒तम् द्वि॑रा॒त्रम् द्वि॑रा॒त्र मे॒त मे॒तम् द्वि॑रा॒त्रम् । \newline
26. द्वि॒रा॒त्र म॑पश्यता मपश्यताम् द्विरा॒त्रम् द्वि॑रा॒त्र म॑पश्यताम् । \newline
27. द्वि॒रा॒त्रमिति॑ द्वि - रा॒त्रम् । \newline
28. अ॒प॒श्य॒ता॒म् तम् त म॑पश्यता मपश्यता॒म् तम् । \newline
29. त मा तम् त मा । \newline
30. आ ऽह॑रता महरता॒ मा ऽह॑रताम् । \newline
31. अ॒ह॒र॒ता॒म् तेन॒ तेना॑ हरता महरता॒म् तेन॑ । \newline
32. तेना॑ यजेता मयजेता॒म् तेन॒ तेना॑ यजेताम् । \newline
33. अ॒य॒जे॒ता॒म् तत॒ स्ततो॑ ऽयजेता मयजेता॒म् ततः॑ । \newline
34. ततो॒ वै वै तत॒ स्ततो॒ वै । \newline
35. वै तौ तौ वै वै तौ । \newline
36. तौ सु॑व॒र्गꣳ सु॑व॒र्गम् तौ तौ सु॑व॒र्गम् । \newline
37. सु॒व॒र्गम् ॅलो॒कम् ॅलो॒कꣳ सु॑व॒र्गꣳ सु॑व॒र्गम् ॅलो॒कम् । \newline
38. सु॒व॒र्गमिति॑ सुवः - गम् । \newline
39. लो॒क मै॑ता मैताम् ॅलो॒कम् ॅलो॒क मै॑ताम् । \newline
40. ऐ॒तां॒ ॅयो य ऐ॑ता मैतां॒ ॅयः । \newline
41. य ए॒व मे॒वं ॅयो य ए॒वम् । \newline
42. ए॒वं ॅवि॒द्वान्. वि॒द्वा ने॒व मे॒वं ॅवि॒द्वान् । \newline
43. वि॒द्वान् द्वि॑रा॒त्रेण॑ द्विरा॒त्रेण॑ वि॒द्वान्. वि॒द्वान् द्वि॑रा॒त्रेण॑ । \newline
44. द्वि॒रा॒त्रेण॒ यज॑ते॒ यज॑ते द्विरा॒त्रेण॑ द्विरा॒त्रेण॒ यज॑ते । \newline
45. द्वि॒रा॒त्रेणेति॑ द्वि - रा॒त्रेण॑ । \newline
46. यज॑ते सुव॒र्गꣳ सु॑व॒र्गं ॅयज॑ते॒ यज॑ते सुव॒र्गम् । \newline
47. सु॒व॒र्ग मे॒वैव सु॑व॒र्गꣳ सु॑व॒र्ग मे॒व । \newline
48. सु॒व॒र्गमिति॑ सुवः - गम् । \newline
49. ए॒व लो॒कम् ॅलो॒क मे॒वैव लो॒कम् । \newline
50. लो॒क मे᳚त्येति लो॒कम् ॅलो॒क मे॑ति । \newline
51. ए॒ति॒ तौ ता वे᳚त्येति॒ तौ । \newline
52. ता वैता॒ मैता॒म् तौ ता वैता᳚म् । \newline
53. ऐता॒म् पूर्वे॑ण॒ पूर्वे॒ णैता॒ मैता॒म् पूर्वे॑ण । \newline
54. पूर्वे॒ णाह्ना ऽह्ना॒ पूर्वे॑ण॒ पूर्वे॒ णाह्ना᳚ । \newline
55. अह्ना ऽग॑च्छता॒ मग॑च्छता॒ मह्ना ऽह्ना ऽग॑च्छताम् । \newline
56. अग॑च्छता॒ मुत्त॑रे॒णो त्त॑रे॒णा ग॑च्छता॒ मग॑च्छता॒ मुत्त॑रेण । \newline
57. उत्त॑रेणा भिप्ल॒वो॑ ऽभिप्ल॒व उत्त॑रे॒ णोत्त॑रेणा भिप्ल॒वः । \newline
58. उत्त॑रे॒णेत्युत् - त॒रे॒ण॒ । \newline

\textbf{Ghana Paata } \newline

1. अङ्गि॑रसो॒ वै वा अङ्गि॑र॒सो ऽङ्गि॑रसो॒ वै स॒त्रꣳ स॒त्रं ॅवा अङ्गि॑र॒सो ऽङ्गि॑रसो॒ वै स॒त्रम् । \newline
2. वै स॒त्रꣳ स॒त्रं ॅवै वै स॒त्र मा॑सता सत स॒त्रं ॅवै वै स॒त्र मा॑सत । \newline
3. स॒त्र मा॑सता सत स॒त्रꣳ स॒त्र मा॑सत॒ ते त आ॑सत स॒त्रꣳ स॒त्र मा॑सत॒ ते । \newline
4. आ॒स॒त॒ ते त आ॑सता सत॒ ते सु॑व॒र्गꣳ सु॑व॒र्गम् त आ॑सता सत॒ ते सु॑व॒र्गम् । \newline
5. ते सु॑व॒र्गꣳ सु॑व॒र्गम् ते ते सु॑व॒र्गम् ॅलो॒कम् ॅलो॒कꣳ सु॑व॒र्गम् ते ते सु॑व॒र्गम् ॅलो॒कम् । \newline
6. सु॒व॒र्गम् ॅलो॒कम् ॅलो॒कꣳ सु॑व॒र्गꣳ सु॑व॒र्गम् ॅलो॒क मा॑यन् नायन् ॅलो॒कꣳ सु॑व॒र्गꣳ सु॑व॒र्गम् ॅलो॒क मा॑यन्न् । \newline
7. सु॒व॒र्गमिति॑ सुवः - गम् । \newline
8. लो॒क मा॑यन् नायन् ॅलो॒कम् ॅलो॒क मा॑य॒न् तेषा॒म् तेषा॑ मायन् ॅलो॒कम् ॅलो॒क मा॑य॒न् तेषा᳚म् । \newline
9. आ॒य॒न् तेषा॒म् तेषा॑ मायन् नाय॒न् तेषाꣳ॑ ह॒विष्मान्॑. ह॒विष्मा॒न् तेषा॑ मायन् नाय॒न् तेषाꣳ॑ ह॒विष्मान्॑ । \newline
10. तेषाꣳ॑ ह॒विष्मान्॑. ह॒विष्मा॒न् तेषा॒म् तेषाꣳ॑ ह॒विष्माꣳ॑ श्च च ह॒विष्मा॒न् तेषा॒म् तेषाꣳ॑ ह॒विष्माꣳ॑ श्च । \newline
11. ह॒विष्माꣳ॑ श्च च ह॒विष्मान्॑. ह॒विष्माꣳ॑ श्च हवि॒ष्कृ द्ध॑वि॒ष्कृच् च॑ ह॒विष्मान्॑. ह॒विष्माꣳ॑
श्च हवि॒ष्कृत् । \newline
12. च॒ ह॒वि॒ष्कृ द्ध॑वि॒ष्कृच् च॑ च हवि॒ष्कृच् च॑ च हवि॒ष्कृच् च॑ च हवि॒ष्कृच् च॑ । \newline
13. ह॒वि॒ष्कृच् च॑ च हवि॒ष्कृ द्ध॑वि॒ष्कृच् चा॑हीयेता महीयेताम् च हवि॒ष्कृ द्ध॑वि॒ष्कृच् चा॑हीयेताम् । \newline
14. ह॒वि॒ष्कृदिति॑ हविः - कृत् । \newline
15. चा॒ही॒ये॒ता॒ म॒ही॒ये॒ता॒म् च॒ चा॒ही॒ये॒ता॒म् तौ ता व॑हीयेताम् च चाहीयेता॒म् तौ । \newline
16. अ॒ही॒ये॒ता॒म् तौ ता व॑हीयेता महीयेता॒म् ता व॑कामयेता मकामयेता॒म् ता व॑हीयेता महीयेता॒म् ता व॑कामयेताम् । \newline
17. ता व॑कामयेता मकामयेता॒म् तौ ता व॑कामयेताꣳ सुव॒र्गꣳ सु॑व॒र्ग म॑कामयेता॒म् तौ ता व॑कामयेताꣳ सुव॒र्गम् । \newline
18. अ॒का॒म॒ये॒ताꣳ॒॒ सु॒व॒र्गꣳ सु॑व॒र्ग म॑कामयेता मकामयेताꣳ सुव॒र्गम् ॅलो॒कम् ॅलो॒कꣳ सु॑व॒र्ग म॑कामयेता मकामयेताꣳ सुव॒र्गम् ॅलो॒कम् । \newline
19. सु॒व॒र्गम् ॅलो॒कम् ॅलो॒कꣳ सु॑व॒र्गꣳ सु॑व॒र्गम् ॅलो॒क मि॑यावे याव लो॒कꣳ सु॑व॒र्गꣳ सु॑व॒र्गम् ॅलो॒क मि॑याव । \newline
20. सु॒व॒र्गमिति॑ सुवः - गम् । \newline
21. लो॒क मि॑यावे याव लो॒कम् ॅलो॒क मि॑या॒वे तीती॑याव लो॒कम् ॅलो॒क मि॑या॒ वेति॑ । \newline
22. इ॒या॒वे तीती॑यावे या॒वेति॒ तौ ता विती॑यावे या॒वेति॒ तौ । \newline
23. इति॒ तौ ता वितीति॒ ता वे॒त मे॒तम् ता वितीति॒ ता वे॒तम् । \newline
24. ता वे॒त मे॒तम् तौ ता वे॒तम् द्वि॑रा॒त्रम् द्वि॑रा॒त्र मे॒तम् तौ ता वे॒तम् द्वि॑रा॒त्रम् । \newline
25. ए॒तम् द्वि॑रा॒त्रम् द्वि॑रा॒त्र मे॒त मे॒तम् द्वि॑रा॒त्र म॑पश्यता मपश्यताम् द्विरा॒त्र मे॒त मे॒तम् द्वि॑रा॒त्र म॑पश्यताम् । \newline
26. द्वि॒रा॒त्र म॑पश्यता मपश्यताम् द्विरा॒त्रम् द्वि॑रा॒त्र म॑पश्यता॒म् तम् त म॑पश्यताम् द्विरा॒त्रम् द्वि॑रा॒त्र म॑पश्यता॒म् तम् । \newline
27. द्वि॒रा॒त्रमिति॑ द्वि - रा॒त्रम् । \newline
28. अ॒प॒श्य॒ता॒म् तम् त म॑पश्यता मपश्यता॒म् त मा त म॑पश्यता मपश्यता॒म् त मा । \newline
29. त मा तम् त मा ऽह॑रता महरता॒ मा तम् त मा ऽह॑रताम् । \newline
30. आ ऽह॑रता महरता॒ मा ऽह॑रता॒म् तेन॒ तेना॑ हरता॒ मा ऽह॑रता॒म् तेन॑ । \newline
31. अ॒ह॒र॒ता॒म् तेन॒ तेना॑ हरता महरता॒म् तेना॑ यजेता मयजेता॒म् तेना॑ हरता महरता॒म् तेना॑ यजेताम् । \newline
32. तेना॑ यजेता मयजेता॒म् तेन॒ तेना॑ यजेता॒म् तत॒ स्ततो॑ ऽयजेता॒म् तेन॒ तेना॑ यजेता॒म् ततः॑ । \newline
33. अ॒य॒जे॒ता॒म् तत॒ स्ततो॑ ऽयजेता मयजेता॒म् ततो॒ वै वै ततो॑ ऽयजेता मयजेता॒म् ततो॒ वै । \newline
34. ततो॒ वै वै तत॒ स्ततो॒ वै तौ तौ वै तत॒ स्ततो॒ वै तौ । \newline
35. वै तौ तौ वै वै तौ सु॑व॒र्गꣳ सु॑व॒र्गम् तौ वै वै तौ सु॑व॒र्गम् । \newline
36. तौ सु॑व॒र्गꣳ सु॑व॒र्गम् तौ तौ सु॑व॒र्गम् ॅलो॒कम् ॅलो॒कꣳ सु॑व॒र्गम् तौ तौ सु॑व॒र्गम् ॅलो॒कम् । \newline
37. सु॒व॒र्गम् ॅलो॒कम् ॅलो॒कꣳ सु॑व॒र्गꣳ सु॑व॒र्गम् ॅलो॒क मै॑ता मैताम् ॅलो॒कꣳ सु॑व॒र्गꣳ सु॑व॒र्गम् ॅलो॒क मै॑ताम् । \newline
38. सु॒व॒र्गमिति॑ सुवः - गम् । \newline
39. लो॒क मै॑ता मैताम् ॅलो॒कम् ॅलो॒क मै॑तां॒ ॅयो य ऐ॑ताम् ॅलो॒कम् ॅलो॒क मै॑तां॒ ॅयः । \newline
40. ऐ॒तां॒ ॅयो य ऐ॑ता मैतां॒ ॅय ए॒व मे॒वं ॅय ऐ॑ता मैतां॒ ॅय ए॒वम् । \newline
41. य ए॒व मे॒वं ॅयो य ए॒वं ॅवि॒द्वान्. वि॒द्वा ने॒वं ॅयो य ए॒वं ॅवि॒द्वान् । \newline
42. ए॒वं ॅवि॒द्वान्. वि॒द्वा ने॒व मे॒वं ॅवि॒द्वान् द्वि॑रा॒त्रेण॑ द्विरा॒त्रेण॑ वि॒द्वा ने॒व मे॒वं ॅवि॒द्वान् द्वि॑रा॒त्रेण॑ । \newline
43. वि॒द्वान् द्वि॑रा॒त्रेण॑ द्विरा॒त्रेण॑ वि॒द्वान्. वि॒द्वान् द्वि॑रा॒त्रेण॒ यज॑ते॒ यज॑ते द्विरा॒त्रेण॑ वि॒द्वान्. वि॒द्वान् द्वि॑रा॒त्रेण॒ यज॑ते । \newline
44. द्वि॒रा॒त्रेण॒ यज॑ते॒ यज॑ते द्विरा॒त्रेण॑ द्विरा॒त्रेण॒ यज॑ते सुव॒र्गꣳ सु॑व॒र्गं ॅयज॑ते द्विरा॒त्रेण॑ द्विरा॒त्रेण॒ यज॑ते सुव॒र्गम् । \newline
45. द्वि॒रा॒त्रेणेति॑ द्वि - रा॒त्रेण॑ । \newline
46. यज॑ते सुव॒र्गꣳ सु॑व॒र्गं ॅयज॑ते॒ यज॑ते सुव॒र्ग मे॒वैव सु॑व॒र्गं ॅयज॑ते॒ यज॑ते सुव॒र्ग मे॒व । \newline
47. सु॒व॒र्ग मे॒वैव सु॑व॒र्गꣳ सु॑व॒र्ग मे॒व लो॒कम् ॅलो॒क मे॒व सु॑व॒र्गꣳ सु॑व॒र्ग मे॒व लो॒कम् । \newline
48. सु॒व॒र्गमिति॑ सुवः - गम् । \newline
49. ए॒व लो॒कम् ॅलो॒क मे॒वैव लो॒क मे᳚त्येति लो॒क मे॒वैव लो॒क मे॑ति । \newline
50. लो॒क मे᳚त्येति लो॒कम् ॅलो॒क मे॑ति॒ तौ ता वे॑ति लो॒कम् ॅलो॒क मे॑ति॒ तौ । \newline
51. ए॒ति॒ तौ ता वे᳚त्येति॒ ता वैता॒ मैता॒म् ता वे᳚त्येति॒ ता वैता᳚म् । \newline
52. ता वैता॒ मैता॒म् तौ ता वैता॒म् पूर्वे॑ण॒ पूर्वे॒ णैता॒म् तौ ता वैता॒म् पूर्वे॑ण । \newline
53. ऐता॒म् पूर्वे॑ण॒ पूर्वे॒ णैता॒ मैता॒म् पूर्वे॒ णाह्ना ऽह्ना॒ पूर्वे॒ णैता॒ मैता॒म् पूर्वे॒ णाह्ना᳚ । \newline
54. पूर्वे॒ णाह्ना ऽह्ना॒ पूर्वे॑ण॒ पूर्वे॒ णाह्ना ऽग॑च्छता॒ मग॑च्छता॒ मह्ना॒ पूर्वे॑ण॒ पूर्वे॒ णाह्ना ऽग॑च्छताम् । \newline
55. अह्ना ऽग॑च्छता॒ मग॑च्छता॒ मह्ना ऽह्ना ऽग॑च्छता॒ मुत्त॑रे॒ णोत्त॑रे॒णा ग॑च्छता॒ मह्ना ऽह्ना ऽग॑च्छता॒ मुत्त॑रेण । \newline
56. अग॑च्छता॒ मुत्त॑रे॒ णोत्त॑रे॒णा ग॑च्छता॒ मग॑च्छता॒ मुत्त॑रेणा भिप्ल॒वो॑ ऽभिप्ल॒व उत्त॑रे॒णा ग॑च्छता॒ मग॑च्छता॒ मुत्त॑रेणा भिप्ल॒वः । \newline
57. उत्त॑रेणा भिप्ल॒वो॑ ऽभिप्ल॒व उत्त॑रे॒ णोत्त॑रेणा भिप्ल॒वः पूर्व॒म् पूर्व॑ मभिप्ल॒व उत्त॑रे॒ णोत्त॑रेणा भिप्ल॒वः पूर्व᳚म् । \newline
58. उत्त॑रे॒णेत्युत् - त॒रे॒ण॒ । \newline
\pagebreak
\markright{ TS 7.1.4.2  \hfill https://www.vedavms.in \hfill}

\section{ TS 7.1.4.2 }

\textbf{TS 7.1.4.2 } \newline
\textbf{Samhita Paata} \newline

-भिप्ल॒वः पूर्व॒मह॑र्भवति॒ गति॒रुत्त॑रं॒ ज्योति॑ष्टोमोऽग्निष्टो॒मः पूर्व॒मह॑र्भवति॒ तेज॒स्तेनाव॑ रुन्धे॒ सर्व॑स्तोमोऽतिरा॒त्र उत्त॑रꣳ॒॒ सर्व॒स्याऽऽ*प्त्यै॒ सर्व॒स्याऽव॑रुद्ध्यै गाय॒त्रं पूर्वेऽह॒न्थ्साम॑ भवति॒ तेजो॒ वै गा॑य॒त्री गा॑य॒त्री ब्र॑ह्मवर्च॒सं तेज॑ ए॒व ब्र॑ह्मवर्च॒समा॒त्मन् ध॑त्ते॒ त्रैष्टु॑भ॒मुत्त॑र॒ ओजो॒ वै वी॒र्यं॑ त्रि॒ष्टुगोज॑ ए॒व वी॒र्य॑मा॒त्मन् ध॑त्ते रथंत॒रं पूर्वे॑ - [  ] \newline

\textbf{Pada Paata} \newline

अ॒भि॒प्ल॒व इत्य॑भि - प्ल॒वः । पूर्व᳚म् । अहः॑ । भ॒व॒ति॒ । गतिः॑ । उत्त॑र॒मित्युत् - त॒र॒म् । ज्योति॑ष्टोम॒ इति॒ ज्योति॑ - स्तो॒मः॒ । अ॒ग्नि॒ष्टो॒म इत्य॑ग्नि-स्तो॒मः । पूर्व᳚म् । अहः॑ । भ॒व॒ति॒ । तेजः॑ । तेन॑ । अवेति॑ । रु॒न्धे॒ । सर्व॑स्तोम॒ इति॒ सर्व॑-स्तो॒मः॒ । अ॒ति॒रा॒त्र इत्य॑ति-रा॒त्रः । उत्त॑र॒मित्युत् - त॒र॒म् । सर्व॑स्य । आप्त्यै᳚ । सर्व॑स्य । अव॑रुद्ध्या॒ इत्यव॑ - रु॒द्ध्यै॒ । गा॒य॒त्रम् । पूर्वे᳚ । अहन्न्॑ । साम॑ । भ॒व॒ति॒ । तेजः॑ । वै । गा॒य॒त्री । गा॒य॒त्री । ब्र॒ह्म॒व॒र्च॒समिति॑ ब्रह्म - व॒र्च॒सम् । तेजः॑ । ए॒व । ब्र॒ह्म॒व॒र्च॒समिति॑ ब्रह्म - व॒र्च॒सम् । आ॒त्मन्न् । ध॒त्ते॒ । त्रैष्टु॑भम् । उत्त॑र॒ इत्युत् - त॒रे॒ । ओजः॑ । वै । वी॒र्य᳚म् । त्रि॒ष्टुक् । ओजः॑ । ए॒व । वी॒र्य᳚म् । आ॒त्मन्न् । ध॒त्ते॒ । र॒थ॒न्त॒रमिति॑ रथं - त॒रम् । पूर्वे᳚ ।  \newline


\textbf{Krama Paata} \newline

अ॒भि॒प्ल॒वः पूर्व᳚म् । अ॒भि॒प्ल॒व इत्य॑भि - प्ल॒वः । पूर्व॒महः॑ । अह॑र् भवति । भ॒व॒ति॒ गतिः॑ । गति॒रुत्त॑रम् । उत्त॑र॒म् ज्योति॑ष्टोमः । उत्त॑र॒मित्युत् - त॒र॒म् । ज्योति॑ष्टोमोऽग्निष्टो॒मः । ज्योति॑ष्टोम॒ इति॒ ज्योतिः॑ - स्तो॒मः॒ । अ॒ग्नि॒ष्टो॒मः पूर्व᳚म् । अ॒ग्नि॒ष्टो॒म इत्य॑ग्नि - स्तो॒मः । पूर्व॒महः॑ । अह॑र् भवति । भ॒व॒ति॒ तेजः॑ । तेज॒स्तेन॑ । तेनाव॑ । अव॑ रुन्धे । रु॒न्धे॒ सर्व॑स्तोमः । सर्व॑स्तोमोऽतिरा॒त्रः । सर्व॑स्तोम॒ इति॒ सर्व॑ - स्तो॒मः॒ । अ॒ति॒रा॒त्र उत्त॑रम् । अ॒ति॒रा॒त्र इत्य॑ति - रा॒त्रः । उत्त॑रꣳ॒॒ सर्व॑स्य । उत्त॑र॒मित्युत् - त॒र॒म् । सर्व॒स्याप्त्यै᳚ । आप्त्यै॒ सर्व॑स्य । सर्व॒स्याव॑रुद्ध्यै । अव॑रुद्ध्यै गाय॒त्रम् । अव॑रुद्ध्या॒ इत्यव॑ - रु॒द्ध्यै॒ । गा॒य॒त्रम् पूर्वे᳚ । पूर्वेऽहन्न्॑ । अह॒न्थ् साम॑ । साम॑ भवति । भ॒व॒ति॒ तेजः॑ । तेजो॒ वै । वै गा॑य॒त्री । गा॒य॒त्री गा॑य॒त्री । गा॒य॒त्री ब्र॑ह्मवर्च॒सम् । ब्र॒ह्म॒व॒र्च॒सम् तेजः॑ । ब्र॒ह्म॒व॒र्च॒समिति॑ ब्रह्म - व॒र्च॒सम् । तेज॑ ए॒व । ए॒व ब्र॑ह्मवर्च॒सम् । ब्र॒ह्म॒व॒र्च॒समा॒त्मन्न् । ब्र॒ह्म॒व॒र्च॒समिति॑ ब्रह्म - व॒र्च॒सम् । आ॒त्मन् ध॑त्ते । ध॒त्ते॒ त्रैष्टु॑भम् । त्रैष्टु॑भ॒मुत्त॑रे । उत्त॑र॒ ओजः॑ । उत्त॑र॒ इत्युत् - त॒रे॒ । ओजो॒ वै । वै वी॒र्य᳚म् । वी॒र्य॑म् त्रि॒ष्टुक् । त्रि॒ष्टुगोजः॑ । ओज॑ ए॒व । ए॒व वी॒र्य᳚म् । वी॒र्य॑मा॒त्मन्न् । आ॒त्मन् ध॑त्ते । ध॒त्ते॒ र॒थ॒न्त॒रम् । र॒थ॒न्त॒रम् पूर्वे᳚ । र॒थ॒न्त॒रमिति॑ रथम् - त॒रम् । पूर्वेऽहन्न्॑ \newline

\textbf{Jatai Paata} \newline

1. अ॒भि॒प्ल॒वः पूर्व॒म् पूर्व॑ मभिप्ल॒वो॑ ऽभिप्ल॒वः पूर्व᳚म् । \newline
2. अ॒भि॒प्ल॒व इत्य॑भि - प्ल॒वः । \newline
3. पूर्व॒ मह॒ रहः॒ पूर्व॒म् पूर्व॒ महः॑ । \newline
4. अह॑र् भवति भव॒ त्यह॒ रह॑र् भवति । \newline
5. भ॒व॒ति॒ गति॒र् गति॑र् भवति भवति॒ गतिः॑ । \newline
6. गति॒ रुत्त॑र॒ मुत्त॑र॒म् गति॒र् गति॒ रुत्त॑रम् । \newline
7. उत्त॑र॒म् ज्योति॑ष्टोमो॒ ज्योति॑ष्टोम॒ उत्त॑र॒ मुत्त॑र॒म् ज्योति॑ष्टोमः । \newline
8. उत्त॑र॒मित्युत् - त॒र॒म् । \newline
9. ज्योति॑ष्टोमो ऽग्निष्टो॒मो᳚ ऽग्निष्टो॒मो ज्योति॑ष्टोमो॒ ज्योति॑ष्टोमो ऽग्निष्टो॒मः । \newline
10. ज्योति॑ष्टोम॒ इति॒ ज्योति॑ - स्तो॒मः॒ । \newline
11. अ॒ग्नि॒ष्टो॒मः पूर्व॒म् पूर्व॑ मग्निष्टो॒मो᳚ ऽग्निष्टो॒मः पूर्व᳚म् । \newline
12. अ॒ग्नि॒ष्टो॒म इत्य॑ग्नि - स्तो॒मः । \newline
13. पूर्व॒ मह॒ रहः॒ पूर्व॒म् पूर्व॒ महः॑ । \newline
14. अह॑र् भवति भव॒ त्यह॒ रह॑र् भवति । \newline
15. भ॒व॒ति॒ तेज॒ स्तेजो॑ भवति भवति॒ तेजः॑ । \newline
16. तेज॒ स्तेन॒ तेन॒ तेज॒ स्तेज॒ स्तेन॑ । \newline
17. तेना वाव॒ तेन॒ तेनाव॑ । \newline
18. अव॑ रुन्धे रु॒न्धे ऽवाव॑ रुन्धे । \newline
19. रु॒न्धे॒ सर्व॑स्तोमः॒ सर्व॑स्तोमो रुन्धे रुन्धे॒ सर्व॑स्तोमः । \newline
20. सर्व॑स्तोमो ऽतिरा॒त्रो॑ ऽतिरा॒त्रः सर्व॑स्तोमः॒ सर्व॑स्तोमो ऽतिरा॒त्रः । \newline
21. सर्व॑स्तोम॒ इति॒ सर्व॑ - स्तो॒मः॒ । \newline
22. अ॒ति॒रा॒त्र उत्त॑र॒ मुत्त॑र मतिरा॒त्रो॑ ऽतिरा॒त्र उत्त॑रम् । \newline
23. अ॒ति॒रा॒त्र इत्य॑ति - रा॒त्रः । \newline
24. उत्त॑रꣳ॒॒ सर्व॑स्य॒ सर्व॒स्योत्त॑र॒ मुत्त॑रꣳ॒॒ सर्व॑स्य । \newline
25. उत्त॑र॒मित्युत् - त॒र॒म् । \newline
26. सर्व॒ स्याप्त्या॒ आप्त्यै॒ सर्व॑स्य॒ सर्व॒ स्याप्त्यै᳚ । \newline
27. आप्त्यै॒ सर्व॑स्य॒ सर्व॒ स्याप्त्या॒ आप्त्यै॒ सर्व॑स्य । \newline
28. सर्व॒स्या व॑रुद्ध्या॒ अव॑रुद्ध्यै॒ सर्व॑स्य॒ सर्व॒ स्याव॑रुद्ध्यै । \newline
29. अव॑रुद्ध्यै गाय॒त्रम् गा॑य॒त्र मव॑रुद्ध्या॒ अव॑रुद्ध्यै गाय॒त्रम् । \newline
30. अव॑रुद्ध्या॒ इत्यव॑ - रु॒द्ध्यै॒ । \newline
31. गा॒य॒त्रम् पूर्वे॒ पूर्वे॑ गाय॒त्रम् गा॑य॒त्रम् पूर्वे᳚ । \newline
32. पूर्वे ऽह॒न् नह॒न् पूर्वे॒ पूर्वे ऽहन्न्॑ । \newline
33. अह॒न् थ्साम॒ सामाह॒न् नह॒न् थ्साम॑ । \newline
34. साम॑ भवति भवति॒ साम॒ साम॑ भवति । \newline
35. भ॒व॒ति॒ तेज॒ स्तेजो॑ भवति भवति॒ तेजः॑ । \newline
36. तेजो॒ वै वै तेज॒ स्तेजो॒ वै । \newline
37. वै गा॑य॒त्री गा॑य॒त्री वै वै गा॑य॒त्री । \newline
38. गा॒य॒त्री गा॑य॒त्री । \newline
39. गा॒य॒त्री ब्र॑ह्मवर्च॒सम् ब्र॑ह्मवर्च॒सम् गा॑य॒त्री गा॑य॒त्री ब्र॑ह्मवर्च॒सम् । \newline
40. ब्र॒ह्म॒व॒र्च॒सम् तेज॒ स्तेजो᳚ ब्रह्मवर्च॒सम् ब्र॑ह्मवर्च॒सम् तेजः॑ । \newline
41. ब्र॒ह्म॒व॒र्च॒समिति॑ ब्रह्म - व॒र्च॒सम् । \newline
42. तेज॑ ए॒वैव तेज॒ स्तेज॑ ए॒व । \newline
43. ए॒व ब्र॑ह्मवर्च॒सम् ब्र॑ह्मवर्च॒स मे॒वैव ब्र॑ह्मवर्च॒सम् । \newline
44. ब्र॒ह्म॒व॒र्च॒स मा॒त्मन् ना॒त्मन् ब्र॑ह्मवर्च॒सम् ब्र॑ह्मवर्च॒स मा॒त्मन्न् । \newline
45. ब्र॒ह्म॒व॒र्च॒समिति॑ ब्रह्म - व॒र्च॒सम् । \newline
46. आ॒त्मन् ध॑त्ते धत्त आ॒त्मन् ना॒त्मन् ध॑त्ते । \newline
47. ध॒त्ते॒ त्रैष्टु॑भ॒म् त्रैष्टु॑भम् धत्ते धत्ते॒ त्रैष्टु॑भम् । \newline
48. त्रैष्टु॑भ॒ मुत्त॑र॒ उत्त॑रे॒ त्रैष्टु॑भ॒म् त्रैष्टु॑भ॒ मुत्त॑रे । \newline
49. उत्त॑र॒ ओज॒ ओज॒ उत्त॑र॒ उत्त॑र॒ ओजः॑ । \newline
50. उत्त॑र॒ इत्युत् - त॒रे॒ । \newline
51. ओजो॒ वै वा ओज॒ ओजो॒ वै । \newline
52. वै वी॒र्यं॑ ॅवी॒र्यं॑ ॅवै वै वी॒र्य᳚म् । \newline
53. वी॒र्य॑म् त्रि॒ष्टुक् त्रि॒ष्टुग् वी॒र्यं॑ ॅवी॒र्य॑म् त्रि॒ष्टुक् । \newline
54. त्रि॒ष्टु गोज॒ ओज॑ स्त्रि॒ष्टुक् त्रि॒ष्टु गोजः॑ । \newline
55. ओज॑ ए॒वै वौज॒ ओज॑ ए॒व । \newline
56. ए॒व वी॒र्यं॑ ॅवी॒र्य॑ मे॒वैव वी॒र्य᳚म् । \newline
57. वी॒र्य॑ मा॒त्मन् ना॒त्मन्. वी॒र्यं॑ ॅवी॒र्य॑ मा॒त्मन्न् । \newline
58. आ॒त्मन् ध॑त्ते धत्त आ॒त्मन् ना॒त्मन् ध॑त्ते । \newline
59. ध॒त्ते॒ र॒थ॒न्त॒रꣳ र॑थन्त॒रम् ध॑त्ते धत्ते रथन्त॒रम् । \newline
60. र॒थ॒न्त॒रम् पूर्वे॒ पूर्वे॑ रथन्त॒रꣳ र॑थन्त॒रम् पूर्वे᳚ । \newline
61. र॒थ॒न्त॒रमिति॑ रथं - त॒रम् । \newline
62. पूर्वे ऽह॒न् नह॒न् पूर्वे॒ पूर्वे ऽहन्न्॑ । \newline

\textbf{Ghana Paata } \newline

1. अ॒भि॒प्ल॒वः पूर्व॒म् पूर्व॑ मभिप्ल॒वो॑ ऽभिप्ल॒वः पूर्व॒ मह॒ रहः॒ पूर्व॑ मभिप्ल॒वो॑ ऽभिप्ल॒वः पूर्व॒ महः॑ । \newline
2. अ॒भि॒प्ल॒व इत्य॑भि - प्ल॒वः । \newline
3. पूर्व॒ मह॒ रहः॒ पूर्व॒म् पूर्व॒ मह॑र् भवति भव॒ त्यहः॒ पूर्व॒म् पूर्व॒ मह॑र् भवति । \newline
4. अह॑र् भवति भव॒ त्यह॒ रह॑र् भवति॒ गति॒र् गति॑र् भव॒ त्यह॒ रह॑र् भवति॒ गतिः॑ । \newline
5. भ॒व॒ति॒ गति॒र् गति॑र् भवति भवति॒ गति॒ रुत्त॑र॒ मुत्त॑र॒म् गति॑र् भवति भवति॒ गति॒ रुत्त॑रम् । \newline
6. गति॒ रुत्त॑र॒ मुत्त॑र॒म् गति॒र् गति॒ रुत्त॑र॒म् ज्योति॑ष्टोमो॒ ज्योति॑ष्टोम॒ उत्त॑र॒म् गति॒र् गति॒ रुत्त॑र॒म् ज्योति॑ष्टोमः । \newline
7. उत्त॑र॒म् ज्योति॑ष्टोमो॒ ज्योति॑ष्टोम॒ उत्त॑र॒ मुत्त॑र॒म् ज्योति॑ष्टोमो ऽग्निष्टो॒मो᳚ ऽग्निष्टो॒मो ज्योति॑ष्टोम॒ उत्त॑र॒ मुत्त॑र॒म् ज्योति॑ष्टोमो ऽग्निष्टो॒मः । \newline
8. उत्त॑र॒मित्युत् - त॒र॒म् । \newline
9. ज्योति॑ष्टोमो ऽग्निष्टो॒मो᳚ ऽग्निष्टो॒मो ज्योति॑ष्टोमो॒ ज्योति॑ष्टोमो ऽग्निष्टो॒मः पूर्व॒म् पूर्व॑ मग्निष्टो॒मो ज्योति॑ष्टोमो॒ ज्योति॑ष्टोमो ऽग्निष्टो॒मः पूर्व᳚म् । \newline
10. ज्योति॑ष्टोम॒ इति॒ ज्योति॑ - स्तो॒मः॒ । \newline
11. अ॒ग्नि॒ष्टो॒मः पूर्व॒म् पूर्व॑ मग्निष्टो॒मो᳚ ऽग्निष्टो॒मः पूर्व॒ मह॒ रहः॒ पूर्व॑ मग्निष्टो॒मो᳚ ऽग्निष्टो॒मः पूर्व॒ महः॑ । \newline
12. अ॒ग्नि॒ष्टो॒म इत्य॑ग्नि - स्तो॒मः । \newline
13. पूर्व॒ मह॒ रहः॒ पूर्व॒म् पूर्व॒ मह॑र् भवति भव॒ त्यहः॒ पूर्व॒म् पूर्व॒ मह॑र् भवति । \newline
14. अह॑र् भवति भव॒ त्यह॒ रह॑र् भवति॒ तेज॒ स्तेजो॑ भव॒ त्यह॒ रह॑र् भवति॒ तेजः॑ । \newline
15. भ॒व॒ति॒ तेज॒ स्तेजो॑ भवति भवति॒ तेज॒ स्तेन॒ तेन॒ तेजो॑ भवति भवति॒ तेज॒ स्तेन॑ । \newline
16. तेज॒ स्तेन॒ तेन॒ तेज॒ स्तेज॒ स्तेना वाव॒ तेन॒ तेज॒ स्तेज॒ स्तेनाव॑ । \newline
17. तेना वाव॒ तेन॒ तेनाव॑ रुन्धे रु॒न्धे ऽव॒ तेन॒ तेनाव॑ रुन्धे । \newline
18. अव॑ रुन्धे रु॒न्धे ऽवाव॑ रुन्धे॒ सर्व॑स्तोमः॒ सर्व॑स्तोमो रु॒न्धे ऽवाव॑ रुन्धे॒ सर्व॑स्तोमः । \newline
19. रु॒न्धे॒ सर्व॑स्तोमः॒ सर्व॑स्तोमो रुन्धे रुन्धे॒ सर्व॑स्तोमो ऽतिरा॒त्रो॑ ऽतिरा॒त्रः सर्व॑स्तोमो रुन्धे रुन्धे॒ सर्व॑स्तोमो ऽतिरा॒त्रः । \newline
20. सर्व॑स्तोमो ऽतिरा॒त्रो॑ ऽतिरा॒त्रः सर्व॑स्तोमः॒ सर्व॑स्तोमो ऽतिरा॒त्र उत्त॑र॒ मुत्त॑र मतिरा॒त्रः सर्व॑स्तोमः॒ सर्व॑स्तोमो ऽतिरा॒त्र उत्त॑रम् । \newline
21. सर्व॑स्तोम॒ इति॒ सर्व॑ - स्तो॒मः॒ । \newline
22. अ॒ति॒रा॒त्र उत्त॑र॒ मुत्त॑र मतिरा॒त्रो॑ ऽतिरा॒त्र उत्त॑रꣳ॒॒ सर्व॑स्य॒ सर्व॒स्योत्त॑र मतिरा॒त्रो॑ ऽतिरा॒त्र उत्त॑रꣳ॒॒ सर्व॑स्य । \newline
23. अ॒ति॒रा॒त्र इत्य॑ति - रा॒त्रः । \newline
24. उत्त॑रꣳ॒॒ सर्व॑स्य॒ सर्व॒स्योत्त॑र॒ मुत्त॑रꣳ॒॒ सर्व॒ स्याप्त्या॒ आप्त्यै॒ सर्व॒स्योत्त॑र॒ मुत्त॑रꣳ॒॒ सर्व॒ स्याप्त्यै᳚ । \newline
25. उत्त॑र॒मित्युत् - त॒र॒म् । \newline
26. सर्व॒ स्याप्त्या॒ आप्त्यै॒ सर्व॑स्य॒ सर्व॒ स्याप्त्यै॒ सर्व॑स्य॒ सर्व॒ स्याप्त्यै॒ सर्व॑स्य॒ सर्व॒स्याप्त्यै॒ सर्व॑स्य । \newline
27. आप्त्यै॒ सर्व॑स्य॒ सर्व॒स्याप्त्या॒ आप्त्यै॒ सर्व॒स्या व॑रुद्ध्या॒ अव॑रुद्ध्यै॒ सर्व॒ स्याप्त्या॒ आप्त्यै॒ सर्व॒ स्याव॑रुद्ध्यै । \newline
28. सर्व॒ स्याव॑रुद्ध्या॒ अव॑रुद्ध्यै॒ सर्व॑स्य॒ सर्व॒ स्याव॑रुद्ध्यै गाय॒त्रम् गा॑य॒त्र मव॑रुद्ध्यै॒ सर्व॑स्य॒ सर्व॒ स्याव॑रुद्ध्यै गाय॒त्रम् । \newline
29. अव॑रुद्ध्यै गाय॒त्रम् गा॑य॒त्र मव॑रुद्ध्या॒ अव॑रुद्ध्यै गाय॒त्रम् पूर्वे॒ पूर्वे॑ गाय॒त्र मव॑रुद्ध्या॒ अव॑रुद्ध्यै गाय॒त्रम् पूर्वे᳚ । \newline
30. अव॑रुद्ध्या॒ इत्यव॑ - रु॒द्ध्यै॒ । \newline
31. गा॒य॒त्रम् पूर्वे॒ पूर्वे॑ गाय॒त्रम् गा॑य॒त्रम् पूर्वे ऽह॒न् नह॒न् पूर्वे॑ गाय॒त्रम् गा॑य॒त्रम् पूर्वे ऽहन्न्॑ । \newline
32. पूर्वे ऽह॒न् नह॒न् पूर्वे॒ पूर्वे ऽह॒न् थ्साम॒ सामाह॒न् पूर्वे॒ पूर्वे ऽह॒न् थ्साम॑ । \newline
33. अह॒न् थ्साम॒ सामाह॒न् नह॒न् थ्साम॑ भवति भवति॒ सामाह॒न् नह॒न् थ्साम॑ भवति । \newline
34. साम॑ भवति भवति॒ साम॒ साम॑ भवति॒ तेज॒ स्तेजो॑ भवति॒ साम॒ साम॑ भवति॒ तेजः॑ । \newline
35. भ॒व॒ति॒ तेज॒ स्तेजो॑ भवति भवति॒ तेजो॒ वै वै तेजो॑ भवति भवति॒ तेजो॒ वै । \newline
36. तेजो॒ वै वै तेज॒ स्तेजो॒ वै गा॑य॒त्री गा॑य॒त्री वै तेज॒ स्तेजो॒ वै गा॑य॒त्री । \newline
37. वै गा॑य॒त्री गा॑य॒त्री वै वै गा॑य॒त्री । \newline
38. गा॒य॒त्री गा॑य॒त्री । \newline
39. गा॒य॒त्री ब्र॑ह्मवर्च॒सम् ब्र॑ह्मवर्च॒सम् गा॑य॒त्री गा॑य॒त्री ब्र॑ह्मवर्च॒सम् तेज॒ स्तेजो᳚ ब्रह्मवर्च॒सम् गा॑य॒त्री गा॑य॒त्री ब्र॑ह्मवर्च॒सम् तेजः॑ । \newline
40. ब्र॒ह्म॒व॒र्च॒सम् तेज॒ स्तेजो᳚ ब्रह्मवर्च॒सम् ब्र॑ह्मवर्च॒सम् तेज॑ ए॒वैव तेजो᳚ ब्रह्मवर्च॒सम् ब्र॑ह्मवर्च॒सम् तेज॑ ए॒व । \newline
41. ब्र॒ह्म॒व॒र्च॒समिति॑ ब्रह्म - व॒र्च॒सम् । \newline
42. तेज॑ ए॒वैव तेज॒ स्तेज॑ ए॒व ब्र॑ह्मवर्च॒सम् ब्र॑ह्मवर्च॒स मे॒व तेज॒ स्तेज॑ ए॒व ब्र॑ह्मवर्च॒सम् । \newline
43. ए॒व ब्र॑ह्मवर्च॒सम् ब्र॑ह्मवर्च॒स मे॒वैव ब्र॑ह्मवर्च॒स मा॒त्मन् ना॒त्मन् ब्र॑ह्मवर्च॒स मे॒वैव ब्र॑ह्मवर्च॒स मा॒त्मन्न् । \newline
44. ब्र॒ह्म॒व॒र्च॒स मा॒त्मन् ना॒त्मन् ब्र॑ह्मवर्च॒सम् ब्र॑ह्मवर्च॒स मा॒त्मन् ध॑त्ते धत्त आ॒त्मन् ब्र॑ह्मवर्च॒सम् ब्र॑ह्मवर्च॒स मा॒त्मन् ध॑त्ते । \newline
45. ब्र॒ह्म॒व॒र्च॒समिति॑ ब्रह्म - व॒र्च॒सम् । \newline
46. आ॒त्मन् ध॑त्ते धत्त आ॒त्मन् ना॒त्मन् ध॑त्ते॒ त्रैष्टु॑भ॒म् त्रैष्टु॑भम् धत्त आ॒त्मन् ना॒त्मन् ध॑त्ते॒ त्रैष्टु॑भम् । \newline
47. ध॒त्ते॒ त्रैष्टु॑भ॒म् त्रैष्टु॑भम् धत्ते धत्ते॒ त्रैष्टु॑भ॒ मुत्त॑र॒ उत्त॑रे॒ त्रैष्टु॑भम् धत्ते धत्ते॒ त्रैष्टु॑भ॒ मुत्त॑रे । \newline
48. त्रैष्टु॑भ॒ मुत्त॑र॒ उत्त॑रे॒ त्रैष्टु॑भ॒म् त्रैष्टु॑भ॒ मुत्त॑र॒ ओज॒ ओज॒ उत्त॑रे॒ त्रैष्टु॑भ॒म् त्रैष्टु॑भ॒ मुत्त॑र॒ ओजः॑ । \newline
49. उत्त॑र॒ ओज॒ ओज॒ उत्त॑र॒ उत्त॑र॒ ओजो॒ वै वा ओज॒ उत्त॑र॒ उत्त॑र॒ ओजो॒ वै । \newline
50. उत्त॑र॒ इत्युत् - त॒रे॒ । \newline
51. ओजो॒ वै वा ओज॒ ओजो॒ वै वी॒र्यं॑ ॅवी॒र्यं॑ ॅवा ओज॒ ओजो॒ वै वी॒र्य᳚म् । \newline
52. वै वी॒र्यं॑ ॅवी॒र्यं॑ ॅवै वै वी॒र्य॑म् त्रि॒ष्टुक् त्रि॒ष्टुग् वी॒र्यं॑ ॅवै वै वी॒र्य॑म् त्रि॒ष्टुक् । \newline
53. वी॒र्य॑म् त्रि॒ष्टुक् त्रि॒ष्टुग् वी॒र्यं॑ ॅवी॒र्य॑म् त्रि॒ष्टु गोज॒ ओज॑ स्त्रि॒ष्टुग् वी॒र्यं॑ ॅवी॒र्य॑म् त्रि॒ष्टु गोजः॑ । \newline
54. त्रि॒ष्टु गोज॒ ओज॑ स्त्रि॒ष्टुक् त्रि॒ष्टु गोज॑ ए॒वै वौज॑ स्त्रि॒ष्टुक् त्रि॒ष्टु गोज॑ ए॒व । \newline
55. ओज॑ ए॒वै वौज॒ ओज॑ ए॒व वी॒र्यं॑ ॅवी॒र्य॑ मे॒वौज॒ ओज॑ ए॒व वी॒र्य᳚म् । \newline
56. ए॒व वी॒र्यं॑ ॅवी॒र्य॑ मे॒वैव वी॒र्य॑ मा॒त्मन् ना॒त्मन्. वी॒र्य॑ मे॒वैव वी॒र्य॑ मा॒त्मन्न् । \newline
57. वी॒र्य॑ मा॒त्मन् ना॒त्मन्. वी॒र्यं॑ ॅवी॒र्य॑ मा॒त्मन् ध॑त्ते धत्त आ॒त्मन्. वी॒र्यं॑ ॅवी॒र्य॑ मा॒त्मन् ध॑त्ते । \newline
58. आ॒त्मन् ध॑त्ते धत्त आ॒त्मन् ना॒त्मन् ध॑त्ते रथन्त॒रꣳ र॑थन्त॒रम् ध॑त्त आ॒त्मन् ना॒त्मन् ध॑त्ते रथन्त॒रम् । \newline
59. ध॒त्ते॒ र॒थ॒न्त॒रꣳ र॑थन्त॒रम् ध॑त्ते धत्ते रथन्त॒रम् पूर्वे॒ पूर्वे॑ रथन्त॒रम् ध॑त्ते धत्ते रथन्त॒रम् पूर्वे᳚ । \newline
60. र॒थ॒न्त॒रम् पूर्वे॒ पूर्वे॑ रथन्त॒रꣳ र॑थन्त॒रम् पूर्वे ऽह॒न् नह॒न् पूर्वे॑ रथन्त॒रꣳ र॑थन्त॒रम् पूर्वे ऽहन्न्॑ । \newline
61. र॒थ॒न्त॒रमिति॑ रथं - त॒रम् । \newline
62. पूर्वे ऽह॒न् नह॒न् पूर्वे॒ पूर्वे ऽह॒न् थ्साम॒ सामाह॒न् पूर्वे॒ पूर्वे ऽह॒न् थ्साम॑ । \newline
\pagebreak
\markright{ TS 7.1.4.3  \hfill https://www.vedavms.in \hfill}

\section{ TS 7.1.4.3 }

\textbf{TS 7.1.4.3 } \newline
\textbf{Samhita Paata} \newline

ऽह॒न्थ् साम॑ भवती॒यं ॅवै र॑थन्त॒रम॒स्यामे॒व प्रति॑ तिष्ठति बृ॒हदुत्त॑रे॒ऽसौ वै बृ॒हद॒मुष्या॑मे॒व प्रति॑ तिष्ठति॒ तदा॑हुः॒ क्व॑ जग॑ती चाऽनु॒ष्टुप् चेति॑ वैखान॒सं पूर्वेऽह॒न्थ् साम॑ भवति॒ तेन॒ जग॑त्यै॒ नैति॑ षोड॒श्युत्त॑रे॒ तेना॑नु॒ष्टुभोऽथा॑ ऽऽ*हु॒र्यथ् स॑मा॒ने᳚ऽर्द्धमा॒से स्याता॑-मन्यत॒रस्याह्नो॑ वी॒र्य॑मनु॑ ( ) पद्ये॒तेत्य॑-मावा॒स्या॑यां॒ पूर्व॒मह॑-र्भव॒त्युत्त॑रस्मि॒-न्नुत्त॑रं॒ नानै॒वा ऽर्द्ध॑मा॒सयो᳚र्भवतो॒ नाना॑वीर्ये भवतो ह॒विष्म॑न्निधनं॒ पूर्व॒मह॑र्भवति हवि॒ष्कृन्नि॑धन॒-मुत्त॑रं॒ प्रति॑ष्ठित्यै ॥ \newline

\textbf{Pada Paata} \newline

अहन्न्॑ । साम॑ । भ॒व॒ति॒ । इ॒यम् । वै । र॒थ॒न्त॒रमिति॑ रथं - त॒रम् । अ॒स्याम् । ए॒व । प्रतीति॑ । ति॒ष्ठ॒ति॒ । बृ॒हत् । उत्त॑र॒ इत्युत् - त॒रे॒ । अ॒सौ । वै । बृ॒हत् । अ॒मुष्या᳚म् । ए॒व । प्रतीति॑ । ति॒ष्ठ॒ति॒ । तत् । आ॒हुः॒ । क्व॑ । जग॑ती । च॒ । अ॒नु॒ष्टुबित्य॑नु - स्तुप् । च॒ । इति॑ । वै॒खा॒न॒सम् । पूर्वे᳚ । अहन्न्॑ । साम॑ । भ॒व॒ति॒ । तेन॑ । जग॑त्यै । न । ए॒ति॒ । षो॒ड॒शि । उत्त॑र॒ इत्युत् - त॒रे॒ । तेन॑ । अ॒नु॒ष्टुभ॒ इत्य॑नु - स्तुभः॑ । अथ॑ । आ॒हुः॒ । यत् । स॒मा॒ने । अ॒द्‌र्ध॒मा॒स इत्य॑द्‌र्ध - मा॒से । स्याता᳚म् । अ॒न्य॒त॒रस्य॑ । अह्नः॑ । वी॒र्य᳚म् । अन्विति॑ ( ) । प॒द्ये॒त॒ । इति॑ । अ॒मा॒वा॒स्या॑या॒मित्य॑मा - वा॒स्या॑याम् । पूर्व᳚म् । अहः॑ । भ॒व॒ति॒ । उत्त॑रस्मि॒न्नित्युत् - त॒र॒स्मि॒न्न् । उत्त॑र॒मित्युत् - त॒र॒म् । नाना᳚ । ए॒व । अ॒द्‌र्ध॒मा॒सयो॒रित्य॑द्‌र्ध - मा॒सयोः᳚ । भ॒व॒तः॒ । नाना॑वीर्ये॒ इति॒ नाना᳚ - वी॒र्ये॒ । भ॒व॒तः॒ । ह॒विष्म॑न्निधन॒मिति॑ ह॒विष्म॑त् - नि॒ध॒न॒म् । पूर्व᳚म् । अहः॑ । भ॒व॒ति॒ । ह॒वि॒ष्कृन्नि॑धन॒मिति॑ हवि॒ष्कृत् - नि॒ध॒न॒म् । उत्त॑र॒मित्युत् - त॒र॒म् । प्रति॑ष्ठित्या॒ इति॒ प्रति॑ - स्थि॒त्यै॒ ॥  \newline


\textbf{Krama Paata} \newline

अह॒न्थ् साम॑ । साम॑ भवति । भ॒व॒ती॒यम् । इ॒यम् ॅवै । वै र॑थन्त॒रम् । र॒थ॒न्त॒रम॒स्याम् । र॒थ॒न्त॒रमिति॑ रथम् - त॒रम् । अ॒स्यामे॒व । ए॒व प्रति॑ । प्रति॑ तिष्ठति । ति॒ष्ठ॒ति॒ बृ॒हत् । बृ॒हदुत्त॑रे । उत्त॑रे॒ऽसौ । उत्त॑र॒ इत्युत् - त॒रे॒ । अ॒सौ वै । वै बृ॒हत् । बृ॒हद॒मुष्या᳚म् । अ॒मुष्या॑मे॒व । ए॒व प्रति॑ । प्रति॑ तिष्ठति । ति॒ष्ठ॒ति॒ तत् । तदा॑हुः । आ॒हुः॒ क्व॑ । क्व॑ जग॑ती । जग॑ती च । चा॒नु॒ष्टुप् । अ॒नु॒ष्टुप् च॑ । अ॒नु॒ष्टुबित्य॑नु - स्तुप् । चेति॑ । इति॑ वैखान॒सम् । वै॒खा॒न॒सम् पूर्वे᳚ । पूर्वेऽहन्न्॑ । अह॒न्थ् साम॑ । साम॑ भवति । भ॒व॒ति॒ तेन॑ । तेन॒ जग॑त्यै । जग॑त्यै॒ न । नैति॑ । ए॒ति॒ षो॒ड॒शि । षो॒ड॒श्युत्त॑रे । उत्त॑रे॒ तेन॑ । उत्त॑र॒ इत्युत् - त॒रे॒ । तेना॑नु॒ष्टुभः॑ । अ॒नु॒ष्टुभोऽथ॑ । अ॒नु॒ष्टुभ॒ इत्य॑नु - स्तुभः॑ । अथा॑हुः । आ॒हु॒र् यत् । यथ् स॑मा॒ने । स॒मा॒ने᳚ऽर्द्धमा॒से । अ॒र्द्ध॒मा॒से स्याता᳚म् । अ॒र्द्ध॒मा॒स इत्य॑र्द्ध - मा॒से । स्याता॑मन्यत॒रस्य॑ । अ॒न्य॒त॒रस्याह्नः॑ । अह्नो॑ वी॒र्य᳚म् । वी॒र्य॑मनु॑ ( ) । अनु॑ पद्येत । प॒द्ये॒तेति॑ । इत्य॑मावा॒स्या॑याम् । अ॒मा॒वा॒स्या॑या॒म् पूर्व᳚म् । अ॒मा॒वा॒स्या॑या॒मित्य॑मा - वा॒स्या॑याम् । पूर्व॒महः॑ । अह॑र् भवति । भ॒व॒त्युत्त॑रस्मिन्न् । उत्त॑रस्मि॒न्नुत्त॑रम् । उत्त॑रस्मि॒न्नित्युत् - त॒र॒स्मि॒न्न्॒ । उत्त॑र॒म् नाना᳚ । उत्त॑र॒मित्युत् - त॒र॒म् । नानै॒व । ए॒वार्द्ध॑मा॒सयोः᳚ । अ॒र्द्ध॒मा॒सयो᳚र् भवतः । अ॒र्द्ध॒मा॒सयो॒रित्य॑र्द्ध - मा॒सयोः᳚ । भ॒व॒तो॒ नाना॑वीर्ये । नाना॑वीर्ये भवतः । नाना॑वीर्ये॒ इति॒ नाना᳚ - वी॒र्ये॒ । भ॒व॒तो॒ ह॒विष्म॑न्निधनम् । ह॒विष्म॑न्निधन॒म् पूर्व᳚म् । ह॒विष्म॑न्निधन॒मिति॑ ह॒विष्म॑त् - नि॒ध॒न॒म् । पूर्व॒महः॑ । अह॑र् भवति । भ॒व॒ति॒ ह॒वि॒ष्कृन्नि॑धनम् । ह॒वि॒ष्कृन्नि॑धन॒मुत्त॑रम् । ह॒वि॒ष्कृन्नि॑धन॒मिति॑ हवि॒ष्कृत् - नि॒ध॒न॒म् । उत्त॑र॒म् प्रति॑ष्ठित्यै । उत्त॑र॒मित्युत् - त॒र॒म् । प्रति॑ष्ठित्या॒ इति॒ प्रति॑ - स्थि॒त्यै॒ । \newline

\textbf{Jatai Paata} \newline

1. अह॒न् थ्साम॒ सामाह॒न् नह॒न् थ्साम॑ । \newline
2. साम॑ भवति भवति॒ साम॒ साम॑ भवति । \newline
3. भ॒व॒ती॒य मि॒यम् भ॑वति भवती॒यम् । \newline
4. इ॒यं ॅवै वा इ॒य मि॒यं ॅवै । \newline
5. वै र॑थन्त॒रꣳ र॑थन्त॒रं ॅवै वै र॑थन्त॒रम् । \newline
6. र॒थ॒न्त॒र म॒स्या म॒स्याꣳ र॑थन्त॒रꣳ र॑थन्त॒र म॒स्याम् । \newline
7. र॒थ॒न्त॒रमिति॑ रथं - त॒रम् । \newline
8. अ॒स्या मे॒वै वास्या म॒स्या मे॒व । \newline
9. ए॒व प्रति॒ प्रत्ये॒ वैव प्रति॑ । \newline
10. प्रति॑ तिष्ठति तिष्ठति॒ प्रति॒ प्रति॑ तिष्ठति । \newline
11. ति॒ष्ठ॒ति॒ बृ॒हद् बृ॒हत् ति॑ष्ठति तिष्ठति बृ॒हत् । \newline
12. बृ॒ह दुत्त॑र॒ उत्त॑रे बृ॒हद् बृ॒ह दुत्त॑रे । \newline
13. उत्त॑रे॒ ऽसा व॒सा वुत्त॑र॒ उत्त॑रे॒ ऽसौ । \newline
14. उत्त॑र॒ इत्युत् - त॒रे॒ । \newline
15. अ॒सौ वै वा अ॒सा व॒सौ वै । \newline
16. वै बृ॒हद् बृ॒हद् वै वै बृ॒हत् । \newline
17. बृ॒ह द॒मुष्या॑ म॒मुष्या᳚म् बृ॒हद् बृ॒ह द॒मुष्या᳚म् । \newline
18. अ॒मुष्या॑ मे॒वैवामुष्या॑ म॒मुष्या॑ मे॒व । \newline
19. ए॒व प्रति॒ प्रत्ये॒ वैव प्रति॑ । \newline
20. प्रति॑ तिष्ठति तिष्ठति॒ प्रति॒ प्रति॑ तिष्ठति । \newline
21. ति॒ष्ठ॒ति॒ तत् तत् ति॑ष्ठति तिष्ठति॒ तत् । \newline
22. तदा॑हु राहु॒ स्तत् तदा॑हुः । \newline
23. आ॒हुः॒ क्वा᳚(1॒) क्वा॑हु राहुः॒ क्व॑ । \newline
24. क्व॑ जग॑ती॒ जग॑ती॒ क्वा᳚(1॒) क्व॑ जग॑ती । \newline
25. जग॑ती च च॒ जग॑ती॒ जग॑ती च । \newline
26. चा॒नु॒ष्टु ब॑नु॒ष्टुप् च॑ चानु॒ष्टुप् । \newline
27. अ॒नु॒ष्टुप् च॑ चानु॒ष्टु ब॑नु॒ष्टुप् च॑ । \newline
28. अ॒नु॒ष्टुबित्य॑नु - स्तुप् । \newline
29. चेतीति॑ च॒ चेति॑ । \newline
30. इति॑ वैखान॒सं ॅवै॑खान॒स मितीति॑ वैखान॒सम् । \newline
31. वै॒खा॒न॒सम् पूर्वे॒ पूर्वे॑ वैखान॒सं ॅवै॑खान॒सम् पूर्वे᳚ । \newline
32. पूर्वे ऽह॒न् नह॒न् पूर्वे॒ पूर्वे ऽहन्न्॑ । \newline
33. अह॒न् थ्साम॒ सामाह॒न् नह॒न् थ्साम॑ । \newline
34. साम॑ भवति भवति॒ साम॒ साम॑ भवति । \newline
35. भ॒व॒ति॒ तेन॒ तेन॑ भवति भवति॒ तेन॑ । \newline
36. तेन॒ जग॑त्यै॒ जग॑त्यै॒ तेन॒ तेन॒ जग॑त्यै । \newline
37. जग॑त्यै॒ न न जग॑त्यै॒ जग॑त्यै॒ न । \newline
38. नैत्ये॑ति॒ न नैति॑ । \newline
39. ए॒ति॒ षो॒ड॒शि षो॑ड॒ श्ये᳚त्येति षोड॒शि । \newline
40. षो॒ड॒ श्युत्त॑र॒ उत्त॑रे षोड॒शि षो॑ड॒ श्युत्त॑रे । \newline
41. उत्त॑रे॒ तेन॒ तेनो त्त॑र॒ उत्त॑रे॒ तेन॑ । \newline
42. उत्त॑र॒ इत्युत् - त॒रे॒ । \newline
43. तेना॑ नु॒ष्टुभो॑ ऽनु॒ष्टुभ॒ स्तेन॒ तेना॑ नु॒ष्टुभः॑ । \newline
44. अ॒नु॒ष्टुभो ऽथाथा॑ नु॒ष्टुभो॑ ऽनु॒ष्टुभो ऽथ॑ । \newline
45. अ॒नु॒ष्टुभ॒ इत्य॑नु - स्तुभः॑ । \newline
46. अथा॑हु राहु॒ रथा था॑हुः । \newline
47. आ॒हु॒र् यद् यदा॑हु राहु॒र् यत् । \newline
48. यथ् स॑मा॒ने स॑मा॒ने यद् यथ् स॑मा॒ने । \newline
49. स॒मा॒ने᳚ ऽर्द्धमा॒से᳚ ऽर्द्धमा॒से स॑मा॒ने स॑मा॒ने᳚ ऽर्द्धमा॒से । \newline
50. अ॒र्द्ध॒मा॒से स्याताꣳ॒॒ स्याता॑ मर्द्धमा॒से᳚ ऽर्द्धमा॒से स्याता᳚म् । \newline
51. अ॒र्द्ध॒मा॒स इत्य॑र्द्ध - मा॒से । \newline
52. स्याता॑ मन्यत॒रस्या᳚ न्यत॒रस्य॒ स्याताꣳ॒॒ स्याता॑ मन्यत॒रस्य॑ । \newline
53. अ॒न्य॒त॒र स्याह्नो ऽह्नो᳚ ऽन्यत॒रस्या᳚ न्यत॒र स्याह्नः॑ । \newline
54. अह्नो॑ वी॒र्यं॑ ॅवी॒र्य॑ मह्नो ऽह्नो॑ वी॒र्य᳚म् । \newline
55. वी॒र्य॑ मन्वनु॑ वी॒र्यं॑ ॅवी॒र्य॑ मनु॑ । \newline
56. अनु॑ पद्येत पद्ये॒ तान्वनु॑ पद्येत । \newline
57. प॒द्ये॒तेतीति॑ पद्येत पद्ये॒तेति॑ । \newline
58. इत्य॑ मावा॒स्या॑या ममावा॒स्या॑या॒ मिती त्य॑मावा॒स्या॑याम् । \newline
59. अ॒मा॒वा॒स्या॑या॒म् पूर्व॒म् पूर्व॑ ममावा॒स्या॑या ममावा॒स्या॑या॒म् पूर्व᳚म् । \newline
60. अ॒मा॒वा॒स्या॑या॒मित्य॑मा - वा॒स्या॑याम् । \newline
61. पूर्व॒ मह॒ रहः॒ पूर्व॒म् पूर्व॒ महः॑ । \newline
62. अह॑र् भवति भव॒ त्यह॒ रह॑र् भवति । \newline
63. भ॒व॒ त्युत्त॑रस्मि॒न् नुत्त॑रस्मिन् भवति भव॒ त्युत्त॑रस्मिन्न् । \newline
64. उत्त॑रस्मि॒न् नुत्त॑र॒ मुत्त॑र॒ मुत्त॑रस्मि॒न् नुत्त॑रस्मि॒न् नुत्त॑रम् । \newline
65. उत्त॑रस्मि॒न्नित्युत् - त॒र॒स्मि॒न्न् । \newline
66. उत्त॑र॒म् नाना॒ नानो त्त॑र॒ मुत्त॑र॒म् नाना᳚ । \newline
67. उत्त॑र॒मित्युत् - त॒र॒म् । \newline
68. नानै॒वैव नाना॒ नानै॒व । \newline
69. ए॒वार्द्ध॑मा॒सयो॑ रर्द्धमा॒सयो॑ रे॒वैवा र्द्ध॑मा॒सयोः᳚ । \newline
70. अ॒र्द्ध॒मा॒सयो᳚र् भवतो भवतो ऽर्द्धमा॒सयो॑ रर्द्धमा॒सयो᳚र् भवतः । \newline
71. अ॒र्द्ध॒मा॒सयो॒रित्य॑र्द्ध - मा॒सयोः᳚ । \newline
72. भ॒व॒तो॒ नाना॑वीर्ये॒ नाना॑वीर्ये भवतो भवतो॒ नाना॑वीर्ये । \newline
73. नाना॑वीर्ये भवतो भवतो॒ नाना॑वीर्ये॒ नाना॑वीर्ये भवतः । \newline
74. नाना॑वीर्ये॒ इति॒ नाना᳚ - वी॒र्ये॒ । \newline
75. भ॒व॒तो॒ ह॒विष्म॑न्निधनꣳ ह॒विष्म॑न्निधनम् भवतो भवतो ह॒विष्म॑न्निधनम् । \newline
76. ह॒विष्म॑न्निधन॒म् पूर्व॒म् पूर्वꣳ॑ ह॒विष्म॑न्निधनꣳ ह॒विष्म॑न्निधन॒म् पूर्व᳚म् । \newline
77. ह॒विष्म॑न्निधन॒मिति॑ ह॒विष्म॑त् - नि॒ध॒न॒म् । \newline
78. पूर्व॒ मह॒ रहः॒ पूर्व॒म् पूर्व॒ महः॑ । \newline
79. अह॑र् भवति भव॒ त्यह॒ रह॑र् भवति । \newline
80. भ॒व॒ति॒ ह॒वि॒ष्कृन्नि॑धनꣳ हवि॒ष्कृन्नि॑धनम् भवति भवति हवि॒ष्कृन्नि॑धनम् । \newline
81. ह॒वि॒ष्कृन्नि॑धन॒ मुत्त॑र॒ मुत्त॑रꣳ हवि॒ष्कृन्नि॑धनꣳ हवि॒ष्कृन्नि॑धन॒ मुत्त॑रम् । \newline
82. ह॒वि॒ष्कृन्नि॑धन॒मिति॑ हवि॒ष्कृत् - नि॒ध॒न॒म् । \newline
83. उत्त॑र॒म् प्रति॑ष्ठित्यै॒ प्रति॑ष्ठित्या॒ उत्त॑र॒ मुत्त॑र॒म् प्रति॑ष्ठित्यै । \newline
84. उत्त॑र॒मित्युत् - त॒र॒म् । \newline
85. प्रति॑ष्ठित्या॒ इति॒ प्रति॑ - स्थि॒त्यै॒ । \newline

\textbf{Ghana Paata } \newline

1. अह॒न् थ्साम॒ सामाह॒न् नह॒न् थ्साम॑ भवति भवति॒ सामाह॒न् नह॒न् थ्साम॑ भवति । \newline
2. साम॑ भवति भवति॒ साम॒ साम॑ भव ती॒य मि॒यम् भ॑वति॒ साम॒ साम॑ भव ती॒यम् । \newline
3. भ॒व॒ ती॒य मि॒यम् भ॑वति भव ती॒यं ॅवै वा इ॒यम् भ॑वति भव ती॒यं ॅवै । \newline
4. इ॒यं ॅवै वा इ॒य मि॒यं ॅवै र॑थन्त॒रꣳ र॑थन्त॒रं ॅवा इ॒य मि॒यं ॅवै र॑थन्त॒रम् । \newline
5. वै र॑थन्त॒रꣳ र॑थन्त॒रं ॅवै वै र॑थन्त॒र म॒स्या म॒स्याꣳ र॑थन्त॒रं ॅवै वै र॑थन्त॒र म॒स्याम् । \newline
6. र॒थ॒न्त॒र म॒स्या म॒स्याꣳ र॑थन्त॒रꣳ र॑थन्त॒र म॒स्या मे॒वै वास्याꣳ र॑थन्त॒रꣳ र॑थन्त॒र म॒स्या मे॒व । \newline
7. र॒थ॒न्त॒रमिति॑ रथं - त॒रम् । \newline
8. अ॒स्या मे॒वै वास्या म॒स्या मे॒व प्रति॒ प्रत्ये॒ वास्या म॒स्या मे॒व प्रति॑ । \newline
9. ए॒व प्रति॒ प्रत्ये॒ वैव प्रति॑ तिष्ठति तिष्ठति॒ प्रत्ये॒ वैव प्रति॑ तिष्ठति । \newline
10. प्रति॑ तिष्ठति तिष्ठति॒ प्रति॒ प्रति॑ तिष्ठति बृ॒हद् बृ॒हत् ति॑ष्ठति॒ प्रति॒ प्रति॑ तिष्ठति बृ॒हत् । \newline
11. ति॒ष्ठ॒ति॒ बृ॒हद् बृ॒हत् ति॑ष्ठति तिष्ठति बृ॒ह दुत्त॑र॒ उत्त॑रे बृ॒हत् ति॑ष्ठति तिष्ठति बृ॒ह दुत्त॑रे । \newline
12. बृ॒ह दुत्त॑र॒ उत्त॑रे बृ॒हद् बृ॒ह दुत्त॑रे॒ ऽसा व॒सा वुत्त॑रे बृ॒हद् बृ॒ह दुत्त॑रे॒ ऽसौ । \newline
13. उत्त॑रे॒ ऽसा व॒सा वुत्त॑र॒ उत्त॑रे॒ ऽसौ वै वा अ॒सा वुत्त॑र॒ उत्त॑रे॒ ऽसौ वै । \newline
14. उत्त॑र॒ इत्युत् - त॒रे॒ । \newline
15. अ॒सौ वै वा अ॒सा व॒सौ वै बृ॒हद् बृ॒हद् वा अ॒सा व॒सौ वै बृ॒हत् । \newline
16. वै बृ॒हद् बृ॒हद् वै वै बृ॒ह द॒मुष्या॑ म॒मुष्या᳚म् बृ॒हद् वै वै बृ॒ह द॒मुष्या᳚म् । \newline
17. बृ॒ह द॒मुष्या॑ म॒मुष्या᳚म् बृ॒हद् बृ॒ह द॒मुष्या॑ मे॒वै वामुष्या᳚म् बृ॒हद् बृ॒ह द॒मुष्या॑ मे॒व । \newline
18. अ॒मुष्या॑ मे॒वै वामुष्या॑ म॒मुष्या॑ मे॒व प्रति॒ प्रत्ये॒ वामुष्या॑ म॒मुष्या॑ मे॒व प्रति॑ । \newline
19. ए॒व प्रति॒ प्रत्ये॒ वैव प्रति॑ तिष्ठति तिष्ठति॒ प्रत्ये॒ वैव प्रति॑ तिष्ठति । \newline
20. प्रति॑ तिष्ठति तिष्ठति॒ प्रति॒ प्रति॑ तिष्ठति॒ तत् तत् ति॑ष्ठति॒ प्रति॒ प्रति॑ तिष्ठति॒ तत् । \newline
21. ति॒ष्ठ॒ति॒ तत् तत् ति॑ष्ठति तिष्ठति॒ तदा॑हु राहु॒ स्तत् ति॑ष्ठति तिष्ठति॒ तदा॑हुः । \newline
22. तदा॑हु राहु॒ स्तत् तदा॑हुः॒ क्वा᳚(1॒) क्वा॑हु॒ स्तत् तदा॑हुः॒ क्व॑ । \newline
23. आ॒हुः॒ क्वा᳚(1॒) क्वा॑हु राहुः॒ क्व॑ जग॑ती॒ जग॑ती॒ क्वा॑हु राहुः॒ क्व॑ जग॑ती । \newline
24. क्व॑ जग॑ती॒ जग॑ती॒ क्वा᳚(1॒) क्व॑ जग॑ती च च॒ जग॑ती॒ क्वा᳚(1॒) क्व॑ जग॑ती च । \newline
25. जग॑ती च च॒ जग॑ती॒ जग॑ती चानु॒ष्टु ब॑नु॒ष्टुप् च॒ जग॑ती॒ जग॑ती चानु॒ष्टुप् । \newline
26. चा॒नु॒ष्टु ब॑नु॒ष्टुप् च॑ चानु॒ष्टुप् च॑ चानु॒ष्टुप् च॑ चानु॒ष्टुप् च॑ । \newline
27. अ॒नु॒ष्टुप् च॑ चानु॒ष्टु ब॑नु॒ष्टुप् चेतीति॑ चानु॒ष्टु ब॑नु॒ष्टुप् चेति॑ । \newline
28. अ॒नु॒ष्टुबित्य॑नु - स्तुप् । \newline
29. चेतीति॑ च॒ चेति॑ वैखान॒सं ॅवै॑खान॒स मिति॑ च॒ चेति॑ वैखान॒सम् । \newline
30. इति॑ वैखान॒सं ॅवै॑खान॒स मितीति॑ वैखान॒सम् पूर्वे॒ पूर्वे॑ वैखान॒स मितीति॑ वैखान॒सम् पूर्वे᳚ । \newline
31. वै॒खा॒न॒सम् पूर्वे॒ पूर्वे॑ वैखान॒सं ॅवै॑खान॒सम् पूर्वे ऽह॒न् नह॒न् पूर्वे॑ वैखान॒सं ॅवै॑खान॒सम् पूर्वे ऽहन्न्॑ । \newline
32. पूर्वे ऽह॒न् नह॒न् पूर्वे॒ पूर्वे ऽह॒न् थ्साम॒ सामाह॒न् पूर्वे॒ पूर्वे ऽह॒न् थ्साम॑ । \newline
33. अह॒न् थ्साम॒ सामाह॒न् नह॒न् थ्साम॑ भवति भवति॒ सामाह॒न् नह॒न् थ्साम॑ भवति । \newline
34. साम॑ भवति भवति॒ साम॒ साम॑ भवति॒ तेन॒ तेन॑ भवति॒ साम॒ साम॑ भवति॒ तेन॑ । \newline
35. भ॒व॒ति॒ तेन॒ तेन॑ भवति भवति॒ तेन॒ जग॑त्यै॒ जग॑त्यै॒ तेन॑ भवति भवति॒ तेन॒ जग॑त्यै । \newline
36. तेन॒ जग॑त्यै॒ जग॑त्यै॒ तेन॒ तेन॒ जग॑त्यै॒ न न जग॑त्यै॒ तेन॒ तेन॒ जग॑त्यै॒ न । \newline
37. जग॑त्यै॒ न न जग॑त्यै॒ जग॑त्यै॒ नैत्ये॑ति॒ न जग॑त्यै॒ जग॑त्यै॒ नैति॑ । \newline
38. नैत्ये॑ति॒ न नैति॑ षोड॒शि षो॑ड॒ श्ये॑ति॒ न नैति॑ षोड॒शि । \newline
39. ए॒ति॒ षो॒ड॒शि षो॑ड॒ श्ये᳚त्येति षोड॒ श्युत्त॑र॒ उत्त॑रे षोड॒ श्ये᳚त्येति षोड॒ श्युत्त॑रे । \newline
40. षो॒ड॒ श्युत्त॑र॒ उत्त॑रे षोड॒शि षो॑ड॒ श्युत्त॑रे॒ तेन॒ तेनोत्त॑रे षोड॒शि षो॑ड॒ श्युत्त॑रे॒ तेन॑ । \newline
41. उत्त॑रे॒ तेन॒ तेनोत्त॑र॒ उत्त॑रे॒ तेना॑ नु॒ष्टुभो॑ ऽनु॒ष्टुभ॒ स्तेनोत्त॑र॒ उत्त॑रे॒ तेना॑ नु॒ष्टुभः॑ । \newline
42. उत्त॑र॒ इत्युत् - त॒रे॒ । \newline
43. तेना॑ नु॒ष्टुभो॑ ऽनु॒ष्टुभ॒ स्तेन॒ तेना॑ नु॒ष्टुभो ऽथाथा॑ नु॒ष्टुभ॒ स्तेन॒ तेना॑ नु॒ष्टुभो ऽथ॑ । \newline
44. अ॒नु॒ष्टुभो ऽथाथा॑ नु॒ष्टुभो॑ ऽनु॒ष्टुभो ऽथा॑हु राहु॒ रथा॑ नु॒ष्टुभो॑ ऽनु॒ष्टुभो ऽथा॑हुः । \newline
45. अ॒नु॒ष्टुभ॒ इत्य॑नु - स्तुभः॑ । \newline
46. अथा॑हु राहु॒ रथा था॑हु॒र् यद् यदा॑हु॒ रथा था॑हु॒र् यत् । \newline
47. आ॒हु॒र् यद् यदा॑हु राहु॒र् यथ् स॑मा॒ने स॑मा॒ने यदा॑हु राहु॒र् यथ् स॑मा॒ने । \newline
48. यथ् स॑मा॒ने स॑मा॒ने यद् यथ् स॑मा॒ने᳚ ऽर्द्धमा॒से᳚ ऽर्द्धमा॒से स॑मा॒ने यद् यथ् स॑मा॒ने᳚ ऽर्द्धमा॒से । \newline
49. स॒मा॒ने᳚ ऽर्द्धमा॒से᳚ ऽर्द्धमा॒से स॑मा॒ने स॑मा॒ने᳚ ऽर्द्धमा॒से स्याताꣳ॒॒ स्याता॑ मर्द्धमा॒से स॑मा॒ने स॑मा॒ने᳚ ऽर्द्धमा॒से स्याता᳚म् । \newline
50. अ॒र्द्ध॒मा॒से स्याताꣳ॒॒ स्याता॑ मर्द्धमा॒से᳚ ऽर्द्धमा॒से स्याता॑ मन्यत॒रस्या᳚ न्यत॒रस्य॒ स्याता॑ मर्द्धमा॒से᳚ ऽर्द्धमा॒से स्याता॑ मन्यत॒रस्य॑ । \newline
51. अ॒र्द्ध॒मा॒स इत्य॑र्द्ध - मा॒से । \newline
52. स्याता॑ मन्यत॒रस्या᳚ न्यत॒रस्य॒ स्याताꣳ॒॒ स्याता॑ मन्यत॒रस्या ह्नो ऽह्नो᳚ ऽन्यत॒रस्य॒ स्याताꣳ॒॒ स्याता॑ मन्यत॒रस्याह्नः॑ । \newline
53. अ॒न्य॒त॒रस्या ह्नो ऽह्नो᳚ ऽन्यत॒रस्या᳚ न्यत॒रस्या ह्नो॑ वी॒र्यं॑ ॅवी॒र्य॑ मह्नो᳚ ऽन्यत॒रस्या᳚ न्यत॒रस्याह्नो॑ वी॒र्य᳚म् । \newline
54. अह्नो॑ वी॒र्यं॑ ॅवी॒र्य॑ मह्नो ऽह्नो॑ वी॒र्य॑ मन्वनु॑ वी॒र्य॑ मह्नो ऽह्नो॑ वी॒र्य॑ मनु॑ । \newline
55. वी॒र्य॑ मन्वनु॑ वी॒र्यं॑ ॅवी॒र्य॑ मनु॑ पद्येत पद्ये॒ तानु॑ वी॒र्यं॑ ॅवी॒र्य॑ मनु॑ पद्येत । \newline
56. अनु॑ पद्येत पद्ये॒ तान्वनु॑ पद्ये॒तेतीति॑ पद्ये॒ तान्वनु॑ पद्ये॒तेति॑ । \newline
57. प॒द्ये॒तेतीति॑ पद्येत पद्ये॒ते त्य॑मावा॒स्या॑या ममावा॒स्या॑या॒ मिति॑ पद्येत पद्ये॒ते त्य॑मावा॒स्या॑याम् । \newline
58. इत्य॑मावा॒स्या॑या ममावा॒स्या॑या॒ मिती त्य॑मावा॒स्या॑या॒म् पूर्व॒म् पूर्व॑ ममावा॒स्या॑या॒ मिती त्य॑मावा॒स्या॑या॒म् पूर्व᳚म् । \newline
59. अ॒मा॒वा॒स्या॑या॒म् पूर्व॒म् पूर्व॑ ममावा॒स्या॑या ममावा॒स्या॑या॒म् पूर्व॒ मह॒ रहः॒ पूर्व॑ ममावा॒स्या॑या ममावा॒स्या॑या॒म् पूर्व॒ महः॑ । \newline
60. अ॒मा॒वा॒स्या॑या॒मित्य॑मा - वा॒स्या॑याम् । \newline
61. पूर्व॒ मह॒ रहः॒ पूर्व॒म् पूर्व॒ मह॑र् भवति भव॒ त्यहः॒ पूर्व॒म् पूर्व॒ मह॑र् भवति । \newline
62. अह॑र् भवति भव॒ त्यह॒ रह॑र् भव॒ त्युत्त॑रस्मि॒न् नुत्त॑रस्मिन् भव॒ त्यह॒ रह॑र् भव॒ त्युत्त॑रस्मिन्न् । \newline
63. भ॒व॒ त्युत्त॑रस्मि॒न् नुत्त॑रस्मिन् भवति भव॒ त्युत्त॑रस्मि॒न् नुत्त॑र॒ मुत्त॑र॒ मुत्त॑रस्मिन् भवति भव॒ त्युत्त॑रस्मि॒न् नुत्त॑रम् । \newline
64. उत्त॑रस्मि॒न् नुत्त॑र॒ मुत्त॑र॒ मुत्त॑रस्मि॒न् नुत्त॑रस्मि॒न् नुत्त॑र॒म् नाना॒ नानोत्त॑र॒ मुत्त॑रस्मि॒न् नुत्त॑रस्मि॒न् नुत्त॑र॒म् नाना᳚ । \newline
65. उत्त॑रस्मि॒न्नित्युत् - त॒र॒स्मि॒न्न् । \newline
66. उत्त॑र॒म् नाना॒ नानोत्त॑र॒ मुत्त॑र॒म् नानै॒ वैव नानोत्त॑र॒ मुत्त॑र॒म् नानै॒व । \newline
67. उत्त॑र॒मित्युत् - त॒र॒म् । \newline
68. नानै॒ वैव नाना॒ नानै॒ वार्द्ध॑मा॒सयो॑ रर्द्धमा॒सयो॑ रे॒व नाना॒ नानै॒ वार्द्ध॑मा॒सयोः᳚ । \newline
69. ए॒वार्द्ध॑मा॒सयो॑ रर्द्धमा॒सयो॑ रे॒वैवा र्द्ध॑मा॒सयो᳚र् भवतो भवतो ऽर्द्धमा॒सयो॑ रे॒वैवा र्द्ध॑मा॒सयो᳚र् भवतः । \newline
70. अ॒र्द्ध॒मा॒सयो᳚र् भवतो भवतो ऽर्द्धमा॒सयो॑ रर्द्धमा॒सयो᳚र् भवतो॒ नाना॑वीर्ये॒ नाना॑वीर्ये भवतो ऽर्द्धमा॒सयो॑ रर्द्धमा॒सयो᳚र् भवतो॒ नाना॑वीर्ये । \newline
71. अ॒र्द्ध॒मा॒सयो॒रित्य॑र्द्ध - मा॒सयोः᳚ । \newline
72. भ॒व॒तो॒ नाना॑वीर्ये॒ नाना॑वीर्ये भवतो भवतो॒ नाना॑वीर्ये भवतो भवतो॒ नाना॑वीर्ये भवतो भवतो॒ नाना॑वीर्ये भवतः । \newline
73. नाना॑वीर्ये भवतो भवतो॒ नाना॑वीर्ये॒ नाना॑वीर्ये भवतो ह॒विष्म॑न्निधनꣳ ह॒विष्म॑न्निधनम् भवतो॒ नाना॑वीर्ये॒ नाना॑वीर्ये भवतो ह॒विष्म॑न्निधनम् । \newline
74. नाना॑वीर्ये॒ इति॒ नाना᳚ - वी॒र्ये॒ । \newline
75. भ॒व॒तो॒ ह॒विष्म॑न्निधनꣳ ह॒विष्म॑न्निधनम् भवतो भवतो ह॒विष्म॑न्निधन॒म् पूर्व॒म् पूर्वꣳ॑ ह॒विष्म॑न्निधनम् भवतो भवतो ह॒विष्म॑न्निधन॒म् पूर्व᳚म् । \newline
76. ह॒विष्म॑न्निधन॒म् पूर्व॒म् पूर्वꣳ॑ ह॒विष्म॑न्निधनꣳ ह॒विष्म॑न्निधन॒म् पूर्व॒ मह॒ रहः॒ पूर्वꣳ॑ ह॒विष्म॑न्निधनꣳ ह॒विष्म॑न्निधन॒म् पूर्व॒ महः॑ । \newline
77. ह॒विष्म॑न्निधन॒मिति॑ ह॒विष्म॑त् - नि॒ध॒न॒म् । \newline
78. पूर्व॒ मह॒ रहः॒ पूर्व॒म् पूर्व॒ मह॑र् भवति भव॒ त्यहः॒ पूर्व॒म् पूर्व॒ मह॑र् भवति । \newline
79. अह॑र् भवति भव॒ त्यह॒ रह॑र् भवति हवि॒ष्कृन्नि॑धनꣳ हवि॒ष्कृन्नि॑धनम् भव॒ त्यह॒ रह॑र् भवति हवि॒ष्कृन्नि॑धनम् । \newline
80. भ॒व॒ति॒ ह॒वि॒ष्कृन्नि॑धनꣳ हवि॒ष्कृन्नि॑धनम् भवति भवति हवि॒ष्कृन्नि॑धन॒ मुत्त॑र॒ मुत्त॑रꣳ हवि॒ष्कृन्नि॑धनम् भवति भवति हवि॒ष्कृन्नि॑धन॒ मुत्त॑रम् । \newline
81. ह॒वि॒ष्कृन्नि॑धन॒ मुत्त॑र॒ मुत्त॑रꣳ हवि॒ष्कृन्नि॑धनꣳ हवि॒ष्कृन्नि॑धन॒ मुत्त॑र॒म् प्रति॑ष्ठित्यै॒ प्रति॑ष्ठित्या॒ उत्त॑रꣳ हवि॒ष्कृन्नि॑धनꣳ हवि॒ष्कृन्नि॑धन॒ मुत्त॑र॒म् प्रति॑ष्ठित्यै । \newline
82. ह॒वि॒ष्कृन्नि॑धन॒मिति॑ हवि॒ष्कृत् - नि॒ध॒न॒म् । \newline
83. उत्त॑र॒म् प्रति॑ष्ठित्यै॒ प्रति॑ष्ठित्या॒ उत्त॑र॒ मुत्त॑र॒म् प्रति॑ष्ठित्यै । \newline
84. उत्त॑र॒मित्युत् - त॒र॒म् । \newline
85. प्रति॑ष्ठित्या॒ इति॒ प्रति॑ - स्थि॒त्यै॒ । \newline
\pagebreak
\markright{ TS 7.1.5.1  \hfill https://www.vedavms.in \hfill}

\section{ TS 7.1.5.1 }

\textbf{TS 7.1.5.1 } \newline
\textbf{Samhita Paata} \newline

आपो॒ वा इ॒दमग्रे॑ सलि॒लमा॑सी॒त् तस्मि॑न् प्र॒जाप॑ति-र्वा॒युर्भू॒त्वा ऽच॑र॒थ् स इ॒माम॑पश्य॒त् तां ॅव॑रा॒हो भू॒त्वाऽह॑र॒त् तां ॅवि॒श्वक॑र्मा भू॒त्वा व्य॑मा॒ट्र्थ् साऽप्र॑थत॒ सा पृ॑थि॒व्य॑भव॒त् तत् पृ॑थि॒व्यै पृ॑थिवि॒त्वं तस्या॑मश्राम्यत् प्र॒जाप॑तिः॒ स दे॒वान॑सृजत॒ वसू᳚न् रु॒द्राना॑दि॒त्यान् ते दे॒वाः प्र॒जाप॑तिमब्रुव॒न् प्रजा॑यामहा॒ इति॒ सो᳚ऽब्रवी॒द् - [  ] \newline

\textbf{Pada Paata} \newline

आपः॑ । वै । इ॒दम् । अग्रे᳚ । स॒लि॒लम् । आ॒सी॒त् । तस्मिन्न्॑ । प्र॒जाप॑ति॒रिति॑ प्र॒जा - प॒तिः॒ । वा॒युः । भू॒त्वा । अ॒च॒र॒त् । सः । इ॒माम् । अ॒प॒श्य॒त् । ताम् । व॒रा॒हः । भू॒त्वा । एति॑ । अ॒ह॒र॒त् । ताम् । वि॒श्वक॒र्मेति॑ वि॒श्व-क॒र्मा॒ । भू॒त्वा । वीति॑ । अ॒मा॒र्ट्॒ । सा । अ॒प्र॒थ॒त॒ । सा । पृ॒थि॒वी । अ॒भ॒व॒त् । तत् । पृ॒थि॒व्यै । पृ॒थि॒वि॒त्वमिति॑ पृ॒थिवि - त्वम् । तस्या᳚म् । अ॒श्रा॒म्य॒त् । प्र॒जाप॑ति॒रिति॑ प्र॒जा-प॒तिः॒ । सः । दे॒वान् । अ॒सृ॒ज॒त॒ । वसून्॑ । रु॒द्रान् । आ॒दि॒त्यान् । ते । दे॒वाः । प्र॒जाप॑ति॒मिति॑ प्र॒जा - प॒ति॒म् । अ॒ब्रु॒व॒न्न् । प्रेति॑ । जा॒या॒म॒है॒ । इति॑ । सः । अ॒ब्र॒वी॒त् ।  \newline


\textbf{Krama Paata} \newline

आपो॒ वै । वा इ॒दम् । इ॒दमग्रे᳚ । अग्रे॑ सलि॒लम् । स॒लि॒लमा॑सीत् । आ॒सी॒त् तस्मिन्न्॑ । तस्मि॑न् प्र॒जाप॑तिः । प्र॒जाप॑तिर् वा॒युः । प्र॒जाप॑ति॒रिति॑ प्र॒जा - प॒तिः॒ । वा॒युर् भू॒त्वा । भू॒त्वाऽच॑रत् । अ॒च॒र॒थ् सः । स इ॒माम् । इ॒माम॑पश्यत् । अ॒प॒श्य॒त् ताम् । ताम् ॅव॑रा॒हः । व॒रा॒हो भू॒त्वा । भू॒त्वा ऽऽ ह॑रत् । आ ऽह॑रत् । अ॒ह॒र॒त् ताम् । ताम् ॅवि॒श्वक॑र्मा । वि॒श्वक॑र्मा भू॒त्वा । वि॒श्वक॒र्मेति॑ वि॒श्व - क॒र्मा॒ । भू॒त्वा वि । व्य॑मार्ट् । अ॒मा॒र्‌ट्थ् सा । साऽप्र॑थत । अ॒प्र॒थ॒त॒ सा । सा पृ॑थि॒वी । पृ॒थि॒व्य॑भवत् । अ॒भ॒व॒त् तत् । तत् पृ॑थि॒व्यै । पृ॒थि॒व्यै पृ॑थिवि॒त्वम् । पृ॒थि॒वि॒त्वम् तस्या᳚म् । पृ॒थि॒वि॒त्वमिति॑ पृथिवि - त्वम् । तस्या॑मश्राम्यत् । अ॒श्रा॒म्य॒त् प्र॒जाप॑तिः । प्र॒जाप॑तिः॒ सः । प्र॒जाप॑ति॒रिति॑ प्र॒जा - प॒तिः॒ । स दे॒वान् । दे॒वान॑सृजत । अ॒सृ॒ज॒त॒ वसून्॑ । वसू᳚न् रु॒द्रान् । रु॒द्राना॑दि॒त्यान् । आ॒दि॒त्यान् ते । ते दे॒वाः । दे॒वाः प्र॒जाप॑तिम् । प्र॒जाप॑तिमब्रुवन्न् । प्र॒जाप॑ति॒मिति॑ प्र॒जा - प॒ति॒म् । अ॒ब्रु॒व॒न् प्र । प्र जा॑यामहै । जा॒या॒म॒हा॒ इति॑ । इति॒ सः । सो᳚ऽब्रवीत् । अ॒ब्र॒वी॒द् यथा᳚ \newline

\textbf{Jatai Paata} \newline

1. आपो॒ वै वा आप॒ आपो॒ वै । \newline
2. वा इ॒द मि॒दं ॅवै वा इ॒दम् । \newline
3. इ॒द मग्रे ऽग्र॑ इ॒द मि॒द मग्रे᳚ । \newline
4. अग्रे॑ सलि॒लꣳ स॑लि॒ल मग्रे ऽग्रे॑ सलि॒लम् । \newline
5. स॒लि॒ल मा॑सी दासीथ् सलि॒लꣳ स॑लि॒ल मा॑सीत् । \newline
6. आ॒सी॒त् तस्मिꣳ॒॒ स्तस्मि॑न् नासी दासी॒त् तस्मिन्न्॑ । \newline
7. तस्मि॑न् प्र॒जाप॑तिः प्र॒जाप॑ति॒ स्तस्मिꣳ॒॒ स्तस्मि॑न् प्र॒जाप॑तिः । \newline
8. प्र॒जाप॑तिर् वा॒युर् वा॒युः प्र॒जाप॑तिः प्र॒जाप॑तिर् वा॒युः । \newline
9. प्र॒जाप॑ति॒रिति॑ प्र॒जा - प॒तिः॒ । \newline
10. वा॒युर् भू॒त्वा भू॒त्वा वा॒युर् वा॒युर् भू॒त्वा । \newline
11. भू॒त्वा ऽच॑र दचरद् भू॒त्वा भू॒त्वा ऽच॑रत् । \newline
12. अ॒च॒र॒थ् स सो॑ ऽचर दचर॒थ् सः । \newline
13. स इ॒मा मि॒माꣳ स स इ॒माम् । \newline
14. इ॒मा म॑पश्य दपश्य दि॒मा मि॒मा म॑पश्यत् । \newline
15. अ॒प॒श्य॒त् ताम् ता म॑पश्य दपश्य॒त् ताम् । \newline
16. तां ॅव॑रा॒हो व॑रा॒ह स्ताम् तां ॅव॑रा॒हः । \newline
17. व॒रा॒हो भू॒त्वा भू॒त्वा व॑रा॒हो व॑रा॒हो भू॒त्वा । \newline
18. भू॒त्वा ऽह॑र दहर॒दा भू॒त्वा भू॒त्वा ऽह॑रत् । \newline
19. आ ऽह॑र दहर॒दा ऽह॑रत् । \newline
20. अ॒ह॒र॒त् ताम् ता म॑हर दहर॒त् ताम् । \newline
21. तां ॅवि॒श्वक॑र्मा वि॒श्वक॑र्मा॒ ताम् तां ॅवि॒श्वक॑र्मा । \newline
22. वि॒श्वक॑र्मा भू॒त्वा भू॒त्वा वि॒श्वक॑र्मा वि॒श्वक॑र्मा भू॒त्वा । \newline
23. वि॒श्वक॒र्मेति॑ वि॒श्व - क॒र्मा॒ । \newline
24. भू॒त्वा वि वि भू॒त्वा भू॒त्वा वि । \newline
25. व्य॑मार् डमा॒र्ड् वि व्य॑मार्ट् । \newline
26. अ॒मा॒र्ट् थ्सा सा ऽमा᳚र् डमार्ट् थ्सा । \newline
27. सा ऽप्र॑थता प्रथत॒ सा सा ऽप्र॑थत । \newline
28. अ॒प्र॒थ॒त॒ सा सा ऽप्र॑थता प्रथत॒ सा । \newline
29. सा पृ॑थि॒वी पृ॑थि॒वी सा सा पृ॑थि॒वी । \newline
30. पृ॒थि॒ व्य॑भव दभवत् पृथि॒वी पृ॑थि॒ व्य॑भवत् । \newline
31. अ॒भ॒व॒त् तत् तद॑भव दभव॒त् तत् । \newline
32. तत् पृ॑थि॒व्यै पृ॑थि॒व्यै तत् तत् पृ॑थि॒व्यै । \newline
33. पृ॒थि॒व्यै पृ॑थिवि॒त्वम् पृ॑थिवि॒त्वम् पृ॑थि॒व्यै पृ॑थि॒व्यै पृ॑थिवि॒त्वम् । \newline
34. पृ॒थि॒वि॒त्वम् तस्या॒म् तस्या᳚म् पृथिवि॒त्वम् पृ॑थिवि॒त्वम् तस्या᳚म् । \newline
35. पृ॒थि॒वि॒त्वमिति॑ पृथिवि - त्वम् । \newline
36. तस्या॑ मश्राम्य दश्राम्य॒त् तस्या॒म् तस्या॑ मश्राम्यत् । \newline
37. अ॒श्रा॒म्य॒त् प्र॒जाप॑तिः प्र॒जाप॑ति रश्राम्य दश्राम्यत् प्र॒जाप॑तिः । \newline
38. प्र॒जाप॑तिः॒ स स प्र॒जाप॑तिः प्र॒जाप॑तिः॒ सः । \newline
39. प्र॒जाप॑ति॒रिति॑ प्र॒जा - प॒तिः॒ । \newline
40. स दे॒वान् दे॒वान् थ्स स दे॒वान् । \newline
41. दे॒वा न॑सृजता सृजत दे॒वान् दे॒वा न॑सृजत । \newline
42. अ॒सृ॒ज॒त॒ वसू॒न्॒. वसू॑ नसृजता सृजत॒ वसून्॑ । \newline
43. वसू᳚न् रु॒द्रान् रु॒द्रान्. वसू॒न्॒. वसू᳚न् रु॒द्रान् । \newline
44. रु॒द्रा ना॑दि॒त्या ना॑दि॒त्यान् रु॒द्रान् रु॒द्रा ना॑दि॒त्यान् । \newline
45. आ॒दि॒त्यान् ते त आ॑दि॒त्या ना॑दि॒त्यान् ते । \newline
46. ते दे॒वा दे॒वा स्ते ते दे॒वाः । \newline
47. दे॒वाः प्र॒जाप॑तिम् प्र॒जाप॑तिम् दे॒वा दे॒वाः प्र॒जाप॑तिम् । \newline
48. प्र॒जाप॑ति मब्रुवन् नब्रुवन् प्र॒जाप॑तिम् प्र॒जाप॑ति मब्रुवन्न् । \newline
49. प्र॒जाप॑ति॒मिति॑ प्र॒जा - प॒ति॒म् । \newline
50. अ॒ब्रु॒व॒न् प्र प्राब्रु॑वन् नब्रुव॒न् प्र । \newline
51. प्र जा॑यामहै जायामहै॒ प्र प्र जा॑यामहै । \newline
52. जा॒या॒म॒हा॒ इतीति॑ जायामहै जायामहा॒ इति॑ । \newline
53. इति॒ स स इतीति॒ सः । \newline
54. सो᳚ ऽब्रवी दब्रवी॒थ् स सो᳚ ऽब्रवीत् । \newline
55. अ॒ब्र॒वी॒द् यथा॒ यथा᳚ ऽब्रवी दब्रवी॒द् यथा᳚ । \newline

\textbf{Ghana Paata } \newline

1. आपो॒ वै वा आप॒ आपो॒ वा इ॒द मि॒दं ॅवा आप॒ आपो॒ वा इ॒दम् । \newline
2. वा इ॒द मि॒दं ॅवै वा इ॒द मग्रे ऽग्र॑ इ॒दं ॅवै वा इ॒द मग्रे᳚ । \newline
3. इ॒द मग्रे ऽग्र॑ इ॒द मि॒द मग्रे॑ सलि॒लꣳ स॑लि॒ल मग्र॑ इ॒द मि॒द मग्रे॑ सलि॒लम् । \newline
4. अग्रे॑ सलि॒लꣳ स॑लि॒ल मग्रे ऽग्रे॑ सलि॒ल मा॑सी दासीथ् सलि॒ल मग्रे ऽग्रे॑ सलि॒ल मा॑सीत् । \newline
5. स॒लि॒ल मा॑सी दासीथ् सलि॒लꣳ स॑लि॒ल मा॑सी॒त् तस्मिꣳ॒॒ स्तस्मि॑न् नासीथ् सलि॒लꣳ स॑लि॒ल मा॑सी॒त् तस्मिन्न्॑ । \newline
6. आ॒सी॒त् तस्मिꣳ॒॒ स्तस्मि॑न् नासी दासी॒त् तस्मि॑न् प्र॒जाप॑तिः प्र॒जाप॑ति॒ स्तस्मि॑न् नासी दासी॒त् तस्मि॑न् प्र॒जाप॑तिः । \newline
7. तस्मि॑न् प्र॒जाप॑तिः प्र॒जाप॑ति॒ स्तस्मिꣳ॒॒ स्तस्मि॑न् प्र॒जाप॑तिर् वा॒युर् वा॒युः प्र॒जाप॑ति॒ स्तस्मिꣳ॒॒ स्तस्मि॑न् प्र॒जाप॑तिर् वा॒युः । \newline
8. प्र॒जाप॑तिर् वा॒युर् वा॒युः प्र॒जाप॑तिः प्र॒जाप॑तिर् वा॒युर् भू॒त्वा भू॒त्वा वा॒युः प्र॒जाप॑तिः प्र॒जाप॑तिर् वा॒युर् भू॒त्वा । \newline
9. प्र॒जाप॑ति॒रिति॑ प्र॒जा - प॒तिः॒ । \newline
10. वा॒युर् भू॒त्वा भू॒त्वा वा॒युर् वा॒युर् भू॒त्वा ऽच॑र दचरद् भू॒त्वा वा॒युर् वा॒युर् भू॒त्वा ऽच॑रत् । \newline
11. भू॒त्वा ऽच॑र दचरद् भू॒त्वा भू॒त्वा ऽच॑र॒थ् स सो॑ ऽचरद् भू॒त्वा भू॒त्वा ऽच॑र॒थ् सः । \newline
12. अ॒च॒र॒थ् स सो॑ ऽचर दचर॒थ् स इ॒मा मि॒माꣳ सो॑ ऽचर दचर॒थ् स इ॒माम् । \newline
13. स इ॒मा मि॒माꣳ स स इ॒मा म॑पश्य दपश्य दि॒माꣳ स स इ॒मा म॑पश्यत् । \newline
14. इ॒मा म॑पश्य दपश्य दि॒मा मि॒मा म॑पश्य॒त् ताम् ता म॑पश्य दि॒मा मि॒मा म॑पश्य॒त् ताम् । \newline
15. अ॒प॒श्य॒त् ताम् ता म॑पश्य दपश्य॒त् तां ॅव॑रा॒हो व॑रा॒ह स्ता म॑पश्य दपश्य॒त् तां ॅव॑रा॒हः । \newline
16. तां ॅव॑रा॒हो व॑रा॒ह स्ताम् तां ॅव॑रा॒हो भू॒त्वा भू॒त्वा व॑रा॒ह स्ताम् तां ॅव॑रा॒हो भू॒त्वा । \newline
17. व॒रा॒हो भू॒त्वा भू॒त्वा व॑रा॒हो व॑रा॒हो भू॒त्वा ऽह॑र दहर॒दा भू॒त्वा व॑रा॒हो व॑रा॒हो भू॒त्वा ऽह॑रत् । \newline
18. भू॒त्वा ऽह॑र दहर॒दा भू॒त्वा भू॒त्वा ऽह॑र॒त् ताम् ता म॑हर॒दा भू॒त्वा भू॒त्वा ऽह॑र॒त् ताम् । \newline
19. आ ऽह॑र दहर॒दा ऽह॑र॒त् ताम् ता म॑हर॒दा ऽह॑र॒त् ताम् । \newline
20. अ॒ह॒र॒त् ताम् ता म॑हर दहर॒त् तां ॅवि॒श्वक॑र्मा वि॒श्वक॑र्मा॒ ता म॑हर दहर॒त् तां ॅवि॒श्वक॑र्मा । \newline
21. तां ॅवि॒श्वक॑र्मा वि॒श्वक॑र्मा॒ ताम् तां ॅवि॒श्वक॑र्मा भू॒त्वा भू॒त्वा वि॒श्वक॑र्मा॒ ताम् तां ॅवि॒श्वक॑र्मा भू॒त्वा । \newline
22. वि॒श्वक॑र्मा भू॒त्वा भू॒त्वा वि॒श्वक॑र्मा वि॒श्वक॑र्मा भू॒त्वा वि वि भू॒त्वा वि॒श्वक॑र्मा वि॒श्वक॑र्मा भू॒त्वा वि । \newline
23. वि॒श्वक॒र्मेति॑ वि॒श्व - क॒र्मा॒ । \newline
24. भू॒त्वा वि वि भू॒त्वा भू॒त्वा व्य॑मार् डमा॒र्ड् वि भू॒त्वा भू॒त्वा व्य॑मार्ट् । \newline
25. व्य॑मार् डमा॒र्ड् वि व्य॑मार्ट् थ्सा सा ऽमा॒र्ड् वि व्य॑मार्ट् थ्सा । \newline
26. अ॒मा॒र्ट् थ्सा सा ऽमा᳚र्डमार्ट् थ्सा ऽप्र॑थता प्रथत॒ सा ऽमा᳚र्डमार्ट् थ्सा ऽप्र॑थत । \newline
27. सा ऽप्र॑थता प्रथत॒ सा सा ऽप्र॑थत॒ सा सा ऽप्र॑थत॒ सा सा ऽप्र॑थत॒ सा । \newline
28. अ॒प्र॒थ॒त॒ सा सा ऽप्र॑थता प्रथत॒ सा पृ॑थि॒वी पृ॑थि॒वी सा ऽप्र॑थता प्रथत॒ सा पृ॑थि॒वी । \newline
29. सा पृ॑थि॒वी पृ॑थि॒वी सा सा पृ॑थि॒ व्य॑भव दभवत् पृथि॒वी सा सा पृ॑थि॒ व्य॑भवत् । \newline
30. पृ॒थि॒ व्य॑भव दभवत् पृथि॒वी पृ॑थि॒ व्य॑भव॒त् तत् तद॑भवत् पृथि॒वी पृ॑थि॒ व्य॑भव॒त् तत् । \newline
31. अ॒भ॒व॒त् तत् तद॑भव दभव॒त् तत् पृ॑थि॒व्यै पृ॑थि॒व्यै तद॑भव दभव॒त् तत् पृ॑थि॒व्यै । \newline
32. तत् पृ॑थि॒व्यै पृ॑थि॒व्यै तत् तत् पृ॑थि॒व्यै पृ॑थिवि॒त्वम् पृ॑थिवि॒त्वम् पृ॑थि॒व्यै तत् तत् पृ॑थि॒व्यै पृ॑थिवि॒त्वम् । \newline
33. पृ॒थि॒व्यै पृ॑थिवि॒त्वम् पृ॑थिवि॒त्वम् पृ॑थि॒व्यै पृ॑थि॒व्यै पृ॑थिवि॒त्वम् तस्या॒म् तस्या᳚म् पृथिवि॒त्वम् पृ॑थि॒व्यै पृ॑थि॒व्यै पृ॑थिवि॒त्वम् तस्या᳚म् । \newline
34. पृ॒थि॒वि॒त्वम् तस्या॒म् तस्या᳚म् पृथिवि॒त्वम् पृ॑थिवि॒त्वम् तस्या॑ मश्राम्य दश्राम्य॒त् तस्या᳚म् पृथिवि॒त्वम् पृ॑थिवि॒त्वम् तस्या॑ मश्राम्यत् । \newline
35. पृ॒थि॒वि॒त्वमिति॑ पृथिवि - त्वम् । \newline
36. तस्या॑ मश्राम्य दश्राम्य॒त् तस्या॒म् तस्या॑ मश्राम्यत् प्र॒जाप॑तिः प्र॒जाप॑ति रश्राम्य॒त् तस्या॒म् तस्या॑ मश्राम्यत् प्र॒जाप॑तिः । \newline
37. अ॒श्रा॒म्य॒त् प्र॒जाप॑तिः प्र॒जाप॑ति रश्राम्य दश्राम्यत् प्र॒जाप॑तिः॒ स स प्र॒जाप॑ति रश्राम्य दश्राम्यत् प्र॒जाप॑तिः॒ सः । \newline
38. प्र॒जाप॑तिः॒ स स प्र॒जाप॑तिः प्र॒जाप॑तिः॒ स दे॒वान् दे॒वान् थ्स प्र॒जाप॑तिः प्र॒जाप॑तिः॒ स दे॒वान् । \newline
39. प्र॒जाप॑ति॒रिति॑ प्र॒जा - प॒तिः॒ । \newline
40. स दे॒वान् दे॒वान् थ्स स दे॒वा न॑सृजता सृजत दे॒वान् थ्स स दे॒वा न॑सृजत । \newline
41. दे॒वा न॑सृजता सृजत दे॒वान् दे॒वा न॑सृजत॒ वसू॒न्॒. वसू॑ नसृजत दे॒वान् दे॒वा न॑सृजत॒ वसून्॑ । \newline
42. अ॒सृ॒ज॒त॒ वसू॒न्॒. वसू॑ नसृजता सृजत॒ वसू᳚न् रु॒द्रान् रु॒द्रान्. वसू॑ नसृजता सृजत॒ वसू᳚न् रु॒द्रान् । \newline
43. वसू᳚न् रु॒द्रान् रु॒द्रान्. वसू॒न्॒. वसू᳚न् रु॒द्रा ना॑दि॒त्या ना॑दि॒त्यान् रु॒द्रान्. वसू॒न्॒. वसू᳚न् रु॒द्रा ना॑दि॒त्यान् । \newline
44. रु॒द्रा ना॑दि॒त्या ना॑दि॒त्यान् रु॒द्रान् रु॒द्रा ना॑दि॒त्यान् ते त आ॑दि॒त्यान् रु॒द्रान् रु॒द्रा ना॑दि॒त्यान् ते । \newline
45. आ॒दि॒त्यान् ते त आ॑दि॒त्या ना॑दि॒त्यान् ते दे॒वा दे॒वा स्त आ॑दि॒त्या ना॑दि॒त्यान् ते दे॒वाः । \newline
46. ते दे॒वा दे॒वा स्ते ते दे॒वाः प्र॒जाप॑तिम् प्र॒जाप॑तिम् दे॒वा स्ते ते दे॒वाः प्र॒जाप॑तिम् । \newline
47. दे॒वाः प्र॒जाप॑तिम् प्र॒जाप॑तिम् दे॒वा दे॒वाः प्र॒जाप॑ति मब्रुवन् नब्रुवन् प्र॒जाप॑तिम् दे॒वा दे॒वाः प्र॒जाप॑ति मब्रुवन्न् । \newline
48. प्र॒जाप॑ति मब्रुवन् नब्रुवन् प्र॒जाप॑तिम् प्र॒जाप॑ति मब्रुव॒न् प्र प्राब्रु॑वन् प्र॒जाप॑तिम् प्र॒जाप॑ति मब्रुव॒न् प्र । \newline
49. प्र॒जाप॑ति॒मिति॑ प्र॒जा - प॒ति॒म् । \newline
50. अ॒ब्रु॒व॒न् प्र प्राब्रु॑वन् नब्रुव॒न् प्र जा॑यामहै जायामहै॒ प्राब्रु॑वन् नब्रुव॒न् प्र जा॑यामहै । \newline
51. प्र जा॑यामहै जायामहै॒ प्र प्र जा॑यामहा॒ इतीति॑ जायामहै॒ प्र प्र जा॑यामहा॒ इति॑ । \newline
52. जा॒या॒म॒हा॒ इतीति॑ जायामहै जायामहा॒ इति॒ स स इति॑ जायामहै जायामहा॒ इति॒ सः । \newline
53. इति॒ स स इतीति॒ सो᳚ ऽब्रवी दब्रवी॒थ् स इतीति॒ सो᳚ ऽब्रवीत् । \newline
54. सो᳚ ऽब्रवी दब्रवी॒थ् स सो᳚ ऽब्रवी॒द् यथा॒ यथा᳚ ऽब्रवी॒थ् स सो᳚ ऽब्रवी॒द् यथा᳚ । \newline
55. अ॒ब्र॒वी॒द् यथा॒ यथा᳚ ऽब्रवी दब्रवी॒द् यथा॒ ऽह म॒हं ॅयथा᳚ ऽब्रवी दब्रवी॒द् यथा॒ ऽहम् । \newline
\pagebreak
\markright{ TS 7.1.5.2  \hfill https://www.vedavms.in \hfill}

\section{ TS 7.1.5.2 }

\textbf{TS 7.1.5.2 } \newline
\textbf{Samhita Paata} \newline

-यथा॒ऽहं ॅयु॒ष्माꣳस्तप॒सा ऽसृ॑क्ष्ये॒वं तप॑सि प्र॒जन॑न-मिच्छद्ध्व॒मिति॒ तेभ्यो॒ऽग्निमा॒यत॑नं॒ प्राऽय॑च्छदे॒तेना॒ऽऽ*यत॑नेन श्राम्य॒तेति॒ ते᳚ऽग्निना॒ऽऽ*यत॑नेनाऽ-श्राम्य॒न् ते सं॑ॅवथ्स॒र एकां॒ गाम॑सृजन्त॒ तां ॅवसु॑भ्यो रु॒द्रेभ्य॑ आदि॒त्येभ्यः॒ प्राऽय॑च्छन्ने॒ताꣳ र॑क्षद्ध्व॒मिति॒ तां ॅवस॑वो रु॒द्रा आ॑दि॒त्या अ॑रक्षन्त॒ सा वसु॑भ्यो रु॒द्रेभ्य॑ आदि॒त्येभ्यः॒ प्राजा॑यत॒त्रीणि॑ च - [  ] \newline

\textbf{Pada Paata} \newline

यथा᳚ । अ॒हम् । यु॒ष्मान् । तप॑सा । असृ॑क्षि । ए॒वम् । तप॑सि । प्र॒जन॑न॒मिति॑ प्र - जन॑नम् । इ॒च्छ॒द्ध्व॒म् । इति॑ । तेभ्यः॑ । अ॒ग्निम् । आ॒यत॑न॒मित्या᳚ - यत॑नम् । प्रेति॑ । अ॒य॒च्छ॒त् । ए॒तेन॑ । आ॒यत॑ने॒नेत्या᳚ - यत॑नेन । श्रा॒म्य॒त॒ । इति॑ । ते । अ॒ग्निना᳚ । आ॒यत॑ने॒नेत्या᳚ - यत॑नेन । अ॒श्रा॒म्य॒न्न् । ते । सं॒ॅव॒थ्स॒र इति॑ सं - व॒थ्स॒रे । एका᳚म् । गाम् । अ॒सृ॒ज॒न्त॒ । ताम् । वसु॑भ्य॒ इति॒ वसु॑ - भ्यः॒ । रु॒द्रेभ्यः॑ । आ॒दि॒त्येभ्यः॑ । प्रेति॑ । अ॒य॒च्छ॒न्न् । ए॒ताम् । र॒क्ष॒द्ध्व॒म् । इति॑ । ताम् । वस॑वः । रु॒द्राः । आ॒दि॒त्याः । अ॒र॒क्ष॒न्त॒ । सा । वसु॑भ्य॒ इति॒ वसु॑ - भ्यः॒ । रु॒द्रेभ्यः॑ । आ॒दि॒त्येभ्यः॑ । प्रेति॑ । अ॒जा॒य॒त॒ । त्रीणि॑ । च॒ ।  \newline


\textbf{Krama Paata} \newline

यथा॒हम् । अ॒हम् ॅयु॒ष्मान् । यु॒ष्माꣳस्तप॑सा । तप॒साऽसृ॑क्षि । असृ॑क्ष्ये॒वम् । ए॒वम् तप॑सि । तप॑सि प्र॒जन॑नम् । प्र॒जन॑नमिच्छद्ध्वम् । प्र॒जन॑न॒मिति॑ प्र - जन॑नम् । इ॒च्छ॒द्ध्व॒मिति॑ । इति॒ तेभ्यः॑ । तेभ्यो॒ऽग्निम् । अ॒ग्निमा॒यत॑नम् । आ॒यत॑न॒म् प्र । आ॒यत॑न॒मित्या᳚ - यत॑नम् । प्राय॑च्छत् । अ॒य॒च्छ॒दे॒तेन॑ । ए॒तेना॒यत॑नेन । आ॒यत॑नेन श्राम्यत । आ॒यत॑ने॒नेत्या᳚ - यत॑नेन । श्रा॒म्य॒तेति॑ । इति॒ ते । ते᳚ऽग्निना᳚ । अ॒ग्निना॒ऽऽयत॑नेन । आ॒यत॑नेनाश्राम्यन्न् । आ॒यत॑ने॒नेत्या᳚ - यत॑नेन । अ॒श्रा॒म्य॒न् ते । ते स॑म्ॅवथ्स॒रे । स॒म्ॅव॒थ्स॒र एका᳚म् । स॒म्ॅव॒थ्स॒र इति॑ सम् - व॒थ्स॒रे । एका॒म् गाम् । गाम॑सृजन्त । अ॒सृ॒ज॒न्त॒ ताम् । ताम् ॅवसु॑भ्यः । वसु॑भ्यो रु॒द्रेभ्यः॑ । वसु॑भ्य॒ इति॒ वसु॑ - भ्यः॒ । रु॒द्रेभ्य॑ आदि॒त्येभ्यः॑ । आ॒दि॒त्येभ्यः॒ प्र । प्राय॑च्छन्न् । अ॒य॒च्छ॒न्ने॒ताम् । ए॒ताꣳ र॑क्षद्ध्वम् । र॒क्ष॒द्ध्व॒मिति॑ । इति॒ ताम् । ताम् ॅवस॑वः । वस॑वो रु॒द्राः । रु॒द्रा आ॑दि॒त्याः । आ॒दि॒त्या अ॑रक्षन्त । अ॒र॒क्ष॒न्त॒ सा । सा वसु॑भ्यः । वसु॑भ्यो रु॒द्रेभ्यः॑ । वसु॑भ्य॒ इति॒ वसु॑ - भ्यः॒ । रु॒द्रेभ्य॑ आदि॒त्येभ्यः॑ । आ॒दि॒त्येभ्यः॒ प्र । प्राजा॑यत । अ॒जा॒य॒त॒ त्रीणि॑ । त्रीणि॑ च । च॒ श॒तानि॑ \newline

\textbf{Jatai Paata} \newline

1. यथा॒ ऽह म॒हं ॅयथा॒ यथा॒ ऽहम् । \newline
2. अ॒हं ॅयु॒ष्मान्. यु॒ष्मा न॒ह म॒हं ॅयु॒ष्मान् । \newline
3. यु॒ष्माꣳ स्तप॑सा॒ तप॑सा यु॒ष्मान्. यु॒ष्माꣳ स्तप॑सा । \newline
4. तप॒सा ऽसृ॒क्ष्य सृ॑क्षि॒ तप॑सा॒ तप॒सा ऽसृ॑क्षि । \newline
5. असृ॑क्ष्ये॒व मे॒व मसृ॒क्ष्य सृ॑क्ष्ये॒वम् । \newline
6. ए॒वम् तप॑सि॒ तप॑ स्ये॒व मे॒वम् तप॑सि । \newline
7. तप॑सि प्र॒जन॑नम् प्र॒जन॑न॒म् तप॑सि॒ तप॑सि प्र॒जन॑नम् । \newline
8. प्र॒जन॑न मिच्छद्ध्व मिच्छद्ध्वम् प्र॒जन॑नम् प्र॒जन॑न मिच्छद्ध्वम् । \newline
9. प्र॒जन॑न॒मिति॑ प्र - जन॑नम् । \newline
10. इ॒च्छ॒द्ध्व॒ मितीती᳚च् छद्ध्व मिच्छद्ध्व॒ मिति॑ । \newline
11. इति॒ तेभ्य॒ स्तेभ्य॒ इतीति॒ तेभ्यः॑ । \newline
12. तेभ्यो॒ ऽग्नि म॒ग्निम् तेभ्य॒ स्तेभ्यो॒ ऽग्निम् । \newline
13. अ॒ग्नि मा॒यत॑न मा॒यत॑न म॒ग्नि म॒ग्नि मा॒यत॑नम् । \newline
14. आ॒यत॑न॒म् प्र प्रायत॑न मा॒यत॑न॒म् प्र । \newline
15. आ॒यत॑न॒मित्या᳚ - यत॑नम् । \newline
16. प्राय॑च्छ दयच्छ॒त् प्र प्राय॑च्छत् । \newline
17. अ॒य॒च्छ॒ दे॒ते नै॒तेना॑ यच्छ दयच्छ दे॒तेन॑ । \newline
18. ए॒तेना॒ यत॑नेना॒ यत॑ने नै॒ते नै॒तेना॒ यत॑नेन । \newline
19. आ॒यत॑नेन श्राम्यत श्राम्यता॒ यत॑नेना॒ यत॑नेन श्राम्यत । \newline
20. आ॒यत॑ने॒नेत्या᳚ - यत॑नेन । \newline
21. श्रा॒म्य॒तेतीति॑ श्राम्यत श्राम्य॒तेति॑ । \newline
22. इति॒ ते त इतीति॒ ते । \newline
23. ते᳚ ऽग्निना॒ ऽग्निना॒ ते ते᳚ ऽग्निना᳚ । \newline
24. अ॒ग्निना॒ ऽऽयत॑नेना॒ यत॑नेना॒ ग्निना॒ ऽग्निना॒ ऽऽयत॑नेन । \newline
25. आ॒यत॑नेना श्राम्यन् नश्राम्यन् ना॒यत॑नेना॒ यत॑नेना श्राम्यन्न् । \newline
26. आ॒यत॑ने॒नेत्या᳚ - यत॑नेन । \newline
27. अ॒श्रा॒म्य॒न् ते ते᳚ ऽश्राम्यन् नश्राम्य॒न् ते । \newline
28. ते सं॑ॅवथ्स॒रे सं॑ॅवथ्स॒रे ते ते सं॑ॅवथ्स॒रे । \newline
29. सं॒ॅव॒थ्स॒र एका॒ मेकाꣳ॑ संॅवथ्स॒रे सं॑ॅवथ्स॒र एका᳚म् । \newline
30. सं॒ॅव॒थ्स॒र इति॑ सं - व॒थ्स॒रे । \newline
31. एका॒म् गाम् गा मेका॒ मेका॒म् गाम् । \newline
32. गा म॑सृजन्ता सृजन्त॒ गाम् गा म॑सृजन्त । \newline
33. अ॒सृ॒ज॒न्त॒ ताम् ता म॑सृजन्ता सृजन्त॒ ताम् । \newline
34. तां ॅवसु॑भ्यो॒ वसु॑भ्य॒ स्ताम् तां ॅवसु॑भ्यः । \newline
35. वसु॑भ्यो रु॒द्रेभ्यो॑ रु॒द्रेभ्यो॒ वसु॑भ्यो॒ वसु॑भ्यो रु॒द्रेभ्यः॑ । \newline
36. वसु॑भ्य॒ इति॒ वसु॑ - भ्यः॒ । \newline
37. रु॒द्रेभ्य॑ आदि॒त्येभ्य॑ आदि॒त्येभ्यो॑ रु॒द्रेभ्यो॑ रु॒द्रेभ्य॑ आदि॒त्येभ्यः॑ । \newline
38. आ॒दि॒त्येभ्यः॒ प्र प्रादि॒त्येभ्य॑ आदि॒त्येभ्यः॒ प्र । \newline
39. प्राय॑च्छन् नयच्छ॒न् प्र प्राय॑च्छन्न् । \newline
40. अ॒य॒च्छ॒न् ने॒ता मे॒ता म॑यच्छन् नयच्छन् ने॒ताम् । \newline
41. ए॒ताꣳ र॑क्षद्ध्वꣳ रक्षद्ध्व मे॒ता मे॒ताꣳ र॑क्षद्ध्वम् । \newline
42. र॒क्ष॒द्ध्व॒ मितीति॑ रक्षद्ध्वꣳ रक्षद्ध्व॒ मिति॑ । \newline
43. इति॒ ताम् ता मितीति॒ ताम् । \newline
44. तां ॅवस॑वो॒ वस॑व॒ स्ताम् तां ॅवस॑वः । \newline
45. वस॑वो रु॒द्रा रु॒द्रा वस॑वो॒ वस॑वो रु॒द्राः । \newline
46. रु॒द्रा आ॑दि॒त्या आ॑दि॒त्या रु॒द्रा रु॒द्रा आ॑दि॒त्याः । \newline
47. आ॒दि॒त्या अ॑रक्षन्ता रक्षन्ता दि॒त्या आ॑दि॒त्या अ॑रक्षन्त । \newline
48. अ॒र॒क्ष॒न्त॒ सा सा ऽर॑क्षन्ता रक्षन्त॒ सा । \newline
49. सा वसु॑भ्यो॒ वसु॑भ्यः॒ सा सा वसु॑भ्यः । \newline
50. वसु॑भ्यो रु॒द्रेभ्यो॑ रु॒द्रेभ्यो॒ वसु॑भ्यो॒ वसु॑भ्यो रु॒द्रेभ्यः॑ । \newline
51. वसु॑भ्य॒ इति॒ वसु॑ - भ्यः॒ । \newline
52. रु॒द्रेभ्य॑ आदि॒त्येभ्य॑ आदि॒त्येभ्यो॑ रु॒द्रेभ्यो॑ रु॒द्रेभ्य॑ आदि॒त्येभ्यः॑ । \newline
53. आ॒दि॒त्येभ्यः॒ प्र प्रादि॒त्येभ्य॑ आदि॒त्येभ्यः॒ प्र । \newline
54. प्रा जा॑यता जायत॒ प्र प्रा जा॑यत । \newline
55. अ॒जा॒य॒त॒ त्रीणि॒ त्रीण्य॑ जायता जायत॒ त्रीणि॑ । \newline
56. त्रीणि॑ च च॒ त्रीणि॒ त्रीणि॑ च । \newline
57. च॒ श॒तानि॑ श॒तानि॑ च च श॒तानि॑ । \newline

\textbf{Ghana Paata } \newline

1. यथा॒ ऽह म॒हं ॅयथा॒ यथा॒ ऽहं ॅयु॒ष्मान्. यु॒ष्मा न॒हं ॅयथा॒ यथा॒ ऽहं ॅयु॒ष्मान् । \newline
2. अ॒हं ॅयु॒ष्मान्. यु॒ष्मा न॒ह म॒हं ॅयु॒ष्माꣳ स्तप॑सा॒ तप॑सा यु॒ष्मा न॒ह म॒हं ॅयु॒ष्माꣳ स्तप॑सा । \newline
3. यु॒ष्माꣳ स्तप॑सा॒ तप॑सा यु॒ष्मान्. यु॒ष्माꣳ स्तप॒सा ऽसृ॒क्ष्य सृ॑क्षि॒ तप॑सा यु॒ष्मान्. यु॒ष्माꣳ स्तप॒सा ऽसृ॑क्षि । \newline
4. तप॒सा ऽसृ॒क्ष्य सृ॑क्षि॒ तप॑सा॒ तप॒सा ऽसृ॑क्ष्ये॒व मे॒व मसृ॑क्षि॒ तप॑सा॒ तप॒सा ऽसृ॑क्ष्ये॒वम् । \newline
5. असृ॑क्ष्ये॒व मे॒व मसृ॒क्ष्य सृ॑क्ष्ये॒वम् तप॑सि॒ तप॑ स्ये॒व मसृ॒क्ष्य सृ॑क्ष्ये॒वम् तप॑सि । \newline
6. ए॒वम् तप॑सि॒ तप॑ स्ये॒व मे॒वम् तप॑सि प्र॒जन॑नम् प्र॒जन॑न॒म् तप॑ स्ये॒व मे॒वम् तप॑सि प्र॒जन॑नम् । \newline
7. तप॑सि प्र॒जन॑नम् प्र॒जन॑न॒म् तप॑सि॒ तप॑सि प्र॒जन॑न मिच्छद्ध्व मिच्छद्ध्वम् प्र॒जन॑न॒म् तप॑सि॒ तप॑सि प्र॒जन॑न मिच्छद्ध्वम् । \newline
8. प्र॒जन॑न मिच्छद्ध्व मिच्छद्ध्वम् प्र॒जन॑नम् प्र॒जन॑न मिच्छद्ध्व॒ मितीती᳚च्छद्ध्वम् प्र॒जन॑नम् प्र॒जन॑न मिच्छद्ध्व॒ मिति॑ । \newline
9. प्र॒जन॑न॒मिति॑ प्र - जन॑नम् । \newline
10. इ॒च्छ॒द्ध्व॒ मितीती᳚च्छद्ध्व मिच्छद्ध्व॒ मिति॒ तेभ्य॒ स्तेभ्य॒ इती᳚च्छद्ध्व मिच्छद्ध्व॒ मिति॒ तेभ्यः॑ । \newline
11. इति॒ तेभ्य॒ स्तेभ्य॒ इतीति॒ तेभ्यो॒ ऽग्नि म॒ग्निम् तेभ्य॒ इतीति॒ तेभ्यो॒ ऽग्निम् । \newline
12. तेभ्यो॒ ऽग्नि म॒ग्निम् तेभ्य॒ स्तेभ्यो॒ ऽग्नि मा॒यत॑न मा॒यत॑न म॒ग्निम् तेभ्य॒ स्तेभ्यो॒ ऽग्नि मा॒यत॑नम् । \newline
13. अ॒ग्नि मा॒यत॑न मा॒यत॑न म॒ग्नि म॒ग्नि मा॒यत॑न॒म् प्र प्रायत॑न म॒ग्नि म॒ग्नि मा॒यत॑न॒म् प्र । \newline
14. आ॒यत॑न॒म् प्र प्रायत॑न मा॒यत॑न॒म् प्राय॑च्छ दयच्छ॒त् प्रायत॑न मा॒यत॑न॒म् प्राय॑च्छत् । \newline
15. आ॒यत॑न॒मित्या᳚ - यत॑नम् । \newline
16. प्राय॑च्छ दयच्छ॒त् प्र प्राय॑च्छ दे॒ते नै॒तेना॑ यच्छ॒त् प्र प्राय॑च्छ दे॒तेन॑ । \newline
17. अ॒य॒च्छ॒ दे॒ते नै॒तेना॑यच्छ दयच्छ दे॒ते ना॒यत॑ने ना॒यत॑ नेनै॒तेना॑ यच्छ दयच्छ दे॒ते ना॒यत॑नेन । \newline
18. ए॒ते ना॒यत॑ने ना॒यत॑ने नै॒ते नै॒ते ना॒यत॑नेन श्राम्यत श्राम्यता॒ यत॑ने नै॒ते नै॒ते ना॒यत॑नेन श्राम्यत । \newline
19. आ॒यत॑नेन श्राम्यत श्राम्यता॒ यत॑ने ना॒यत॑नेन श्राम्य॒तेतीति॑ श्राम्यता॒ यत॑नेना॒ यत॑नेन श्राम्य॒तेति॑ । \newline
20. आ॒यत॑ने॒नेत्या᳚ - यत॑नेन । \newline
21. श्रा॒म्य॒तेतीति॑ श्राम्यत श्राम्य॒तेति॒ ते त इति॑ श्राम्यत श्राम्य॒तेति॒ ते । \newline
22. इति॒ ते त इतीति॒ ते᳚ ऽग्निना॒ ऽग्निना॒ त इतीति॒ ते᳚ ऽग्निना᳚ । \newline
23. ते᳚ ऽग्निना॒ ऽग्निना॒ ते ते᳚ ऽग्निना॒ ऽऽयत॑नेना॒ यत॑नेना॒ग्निना॒ ते ते᳚ ऽग्निना॒ ऽऽयत॑नेन । \newline
24. अ॒ग्निना॒ ऽऽयत॑नेना॒ यत॑नेना॒ ग्निना॒ ऽग्निना॒ ऽऽयत॑नेना श्राम्यन् नश्राम्यन् ना॒यत॑नेना॒ ग्निना॒ ऽग्निना॒ ऽऽयत॑नेना श्राम्यन्न् । \newline
25. आ॒यत॑नेना श्राम्यन् नश्राम्यन् ना॒यत॑नेना॒ यत॑नेना श्राम्य॒न् ते ते᳚ ऽश्राम्यन् ना॒यत॑नेना॒ यत॑नेना श्राम्य॒न् ते । \newline
26. आ॒यत॑ने॒नेत्या᳚ - यत॑नेन । \newline
27. अ॒श्रा॒म्य॒न् ते ते᳚ ऽश्राम्यन् नश्राम्य॒न् ते सं॑ॅवथ्स॒रे सं॑ॅवथ्स॒रे ते᳚ ऽश्राम्यन् नश्राम्य॒न् ते सं॑ॅवथ्स॒रे । \newline
28. ते सं॑ॅवथ्स॒रे सं॑ॅवथ्स॒रे ते ते सं॑ॅवथ्स॒र एका॒ मेकाꣳ॑ संॅवथ्स॒रे ते ते सं॑ॅवथ्स॒र एका᳚म् । \newline
29. सं॒ॅव॒थ्स॒र एका॒ मेकाꣳ॑ संॅवथ्स॒रे सं॑ॅवथ्स॒र एका॒म् गाम् गा मेकाꣳ॑ संॅवथ्स॒रे सं॑ॅवथ्स॒र एका॒म् गाम् । \newline
30. सं॒ॅव॒थ्स॒र इति॑ सं - व॒थ्स॒रे । \newline
31. एका॒म् गाम् गा मेका॒ मेका॒म् गा म॑सृजन्ता सृजन्त॒ गा मेका॒ मेका॒म् गा म॑सृजन्त । \newline
32. गा म॑सृजन्ता सृजन्त॒ गाम् गा म॑सृजन्त॒ ताम् ता म॑सृजन्त॒ गाम् गा म॑सृजन्त॒ ताम् । \newline
33. अ॒सृ॒ज॒न्त॒ ताम् ता म॑सृजन्ता सृजन्त॒ तां ॅवसु॑भ्यो॒ वसु॑भ्य॒ स्ता म॑सृजन्ता सृजन्त॒ तां ॅवसु॑भ्यः । \newline
34. तां ॅवसु॑भ्यो॒ वसु॑भ्य॒ स्ताम् तां ॅवसु॑भ्यो रु॒द्रेभ्यो॑ रु॒द्रेभ्यो॒ वसु॑भ्य॒ स्ताम् तां ॅवसु॑भ्यो रु॒द्रेभ्यः॑ । \newline
35. वसु॑भ्यो रु॒द्रेभ्यो॑ रु॒द्रेभ्यो॒ वसु॑भ्यो॒ वसु॑भ्यो रु॒द्रेभ्य॑ आदि॒त्येभ्य॑ आदि॒त्येभ्यो॑ रु॒द्रेभ्यो॒ वसु॑भ्यो॒ वसु॑भ्यो रु॒द्रेभ्य॑ आदि॒त्येभ्यः॑ । \newline
36. वसु॑भ्य॒ इति॒ वसु॑ - भ्यः॒ । \newline
37. रु॒द्रेभ्य॑ आदि॒त्येभ्य॑ आदि॒त्येभ्यो॑ रु॒द्रेभ्यो॑ रु॒द्रेभ्य॑ आदि॒त्येभ्यः॒ प्र प्रादि॒त्येभ्यो॑ रु॒द्रेभ्यो॑ रु॒द्रेभ्य॑ आदि॒त्येभ्यः॒ प्र । \newline
38. आ॒दि॒त्येभ्यः॒ प्र प्रादि॒त्येभ्य॑ आदि॒त्येभ्यः॒ प्राय॑च्छन् नयच्छ॒न् प्रादि॒त्येभ्य॑ आदि॒त्येभ्यः॒ प्राय॑च्छन्न् । \newline
39. प्राय॑च्छन् नयच्छ॒न् प्र प्राय॑च्छन् ने॒ता मे॒ता म॑यच्छ॒न् प्र प्राय॑च्छन् ने॒ताम् । \newline
40. अ॒य॒च्छ॒न् ने॒ता मे॒ता म॑यच्छन् नयच्छन् ने॒ताꣳ र॑क्षद्ध्वꣳ रक्षद्ध्व मे॒ता म॑यच्छन् नयच्छन् ने॒ताꣳ र॑क्षद्ध्वम् । \newline
41. ए॒ताꣳ र॑क्षद्ध्वꣳ रक्षद्ध्व मे॒ता मे॒ताꣳ र॑क्षद्ध्व॒ मितीति॑ रक्षद्ध्व मे॒ता मे॒ताꣳ र॑क्षद्ध्व॒ मिति॑ । \newline
42. र॒क्ष॒द्ध्व॒ मितीति॑ रक्षद्ध्वꣳ रक्षद्ध्व॒ मिति॒ ताम् ता मिति॑ रक्षद्ध्वꣳ रक्षद्ध्व॒ मिति॒ ताम् । \newline
43. इति॒ ताम् ता मितीति॒ तां ॅवस॑वो॒ वस॑व॒ स्ता मितीति॒ तां ॅवस॑वः । \newline
44. तां ॅवस॑वो॒ वस॑व॒ स्ताम् तां ॅवस॑वो रु॒द्रा रु॒द्रा वस॑व॒ स्ताम् तां ॅवस॑वो रु॒द्राः । \newline
45. वस॑वो रु॒द्रा रु॒द्रा वस॑वो॒ वस॑वो रु॒द्रा आ॑दि॒त्या आ॑दि॒त्या रु॒द्रा वस॑वो॒ वस॑वो रु॒द्रा आ॑दि॒त्याः । \newline
46. रु॒द्रा आ॑दि॒त्या आ॑दि॒त्या रु॒द्रा रु॒द्रा आ॑दि॒त्या अ॑रक्षन्ता रक्षन्तादि॒त्या रु॒द्रा रु॒द्रा आ॑दि॒त्या अ॑रक्षन्त । \newline
47. आ॒दि॒त्या अ॑रक्षन्ता रक्षन्तादि॒त्या आ॑दि॒त्या अ॑रक्षन्त॒ सा सा ऽर॑क्षन्तादि॒त्या आ॑दि॒त्या अ॑रक्षन्त॒ सा । \newline
48. अ॒र॒क्ष॒न्त॒ सा सा ऽर॑क्षन्ता रक्षन्त॒ सा वसु॑भ्यो॒ वसु॑भ्यः॒ सा ऽर॑क्षन्ता रक्षन्त॒ सा वसु॑भ्यः । \newline
49. सा वसु॑भ्यो॒ वसु॑भ्यः॒ सा सा वसु॑भ्यो रु॒द्रेभ्यो॑ रु॒द्रेभ्यो॒ वसु॑भ्यः॒ सा सा वसु॑भ्यो रु॒द्रेभ्यः॑ । \newline
50. वसु॑भ्यो रु॒द्रेभ्यो॑ रु॒द्रेभ्यो॒ वसु॑भ्यो॒ वसु॑भ्यो रु॒द्रेभ्य॑ आदि॒त्येभ्य॑ आदि॒त्येभ्यो॑ रु॒द्रेभ्यो॒ वसु॑भ्यो॒ वसु॑भ्यो रु॒द्रेभ्य॑ आदि॒त्येभ्यः॑ । \newline
51. वसु॑भ्य॒ इति॒ वसु॑ - भ्यः॒ । \newline
52. रु॒द्रेभ्य॑ आदि॒त्येभ्य॑ आदि॒त्येभ्यो॑ रु॒द्रेभ्यो॑ रु॒द्रेभ्य॑ आदि॒त्येभ्यः॒ प्र प्रादि॒त्येभ्यो॑ रु॒द्रेभ्यो॑ रु॒द्रेभ्य॑ आदि॒त्येभ्यः॒ प्र । \newline
53. आ॒दि॒त्येभ्यः॒ प्र प्रादि॒त्येभ्य॑ आदि॒त्येभ्यः॒ प्रा जा॑यता जायत॒ प्रादि॒त्येभ्य॑ आदि॒त्येभ्यः॒ प्रा जा॑यत । \newline
54. प्रा जा॑यता जायत॒ प्र प्रा जा॑यत॒ त्रीणि॒ त्रीण्य॑जायत॒ प्र प्रा जा॑यत॒ त्रीणि॑ । \newline
55. अ॒जा॒य॒त॒ त्रीणि॒त्री ण्य॑जायता जायत॒ त्रीणि॑ च च॒ त्रीण्य॑जायता जायत॒ त्रीणि॑ च । \newline
56. त्रीणि॑ च च॒ त्रीणि॒ त्रीणि॑ च श॒तानि॑ श॒तानि॑ च॒ त्रीणि॒ त्रीणि॑ च श॒तानि॑ । \newline
57. च॒ श॒तानि॑ श॒तानि॑ च च श॒तानि॒ त्रय॑स्त्रिꣳशत॒म् त्रय॑स्त्रिꣳशतꣳ श॒तानि॑ च च श॒तानि॒ त्रय॑स्त्रिꣳशतम् । \newline
\pagebreak
\markright{ TS 7.1.5.3  \hfill https://www.vedavms.in \hfill}

\section{ TS 7.1.5.3 }

\textbf{TS 7.1.5.3 } \newline
\textbf{Samhita Paata} \newline

श॒तानि॒ त्रय॑स्त्रिꣳशतं॒ चाथ॒ सैव स॑हस्रत॒म्य॑भव॒त् ते दे॒वाः प्र॒जाप॑तिमब्रुवन्थ् स॒हस्रे॑ण नो याज॒येति॒ सो᳚ऽग्निष्टो॒मेन॒ वसू॑नयाजय॒त् त इ॒मं ॅलो॒कम॑जय॒न् तच्चा॑ददुः॒ स उ॒क्थ्ये॑न रु॒द्रान॑याजय॒त् ते᳚ऽन्तरि॑क्षमजय॒न् तच्चा॑ददुः॒ सो॑ऽतिरा॒त्रेणा॑ऽऽ*दि॒त्यान॑याजय॒त् ते॑ऽमुं ॅलो॒कम॑जय॒न् तच्चा॑ददु॒ -स्तद॒न्तरि॑क्षं॒ - [  ] \newline

\textbf{Pada Paata} \newline

श॒तानि॑ । त्रय॑स्त्रिꣳशत॒मिति॒ त्रयः॑ - त्रिꣳ॒॒श॒त॒म् । च॒ । अथ॑ । सा । ए॒व । स॒ह॒स्र॒त॒मीति॑ सहस्र - त॒मी । अ॒भ॒व॒त् । ते । दे॒वाः । प्र॒जाप॑ति॒मिति॑ प्र॒जा - प॒ति॒म् । अ॒ब्रु॒व॒न्न् । स॒हस्रे॑ण । नः॒ । या॒ज॒य॒ । इति॑ । सः । अ॒ग्नि॒ष्टो॒मेनेत्य॑ग्नि-स्तो॒मेन॑ । वसून्॑ । अ॒या॒ज॒य॒त् । ते । इ॒मम् । लो॒कम् । अ॒ज॒य॒न्न् । तत् । च॒ । अ॒द॒दुः॒ । सः । उ॒क्थ्ये॑न । रु॒द्रान् । अ॒या॒ज॒य॒त् । ते । अ॒न्तरि॑क्षम् । अ॒ज॒य॒न्न् । तत् । च॒ । अ॒द॒दुः॒ । सः । अ॒ति॒रा॒त्रेणेत्य॑ति - रा॒त्रेण॑ । आ॒दि॒त्यान् । अ॒या॒ज॒य॒त् । ते । अ॒मुम् । लो॒कम् । अ॒ज॒य॒न्न् । तत् । च॒ । अ॒द॒दुः॒ । तत् । अ॒न्तरि॑क्षम् ।  \newline


\textbf{Krama Paata} \newline

श॒तानि॒ त्रय॑स्त्रिꣳशतम् । त्रय॑स्त्रिꣳशतम् च । त्रय॑स्त्रिꣳशत॒मिति॒ त्रयः॑ - त्रिꣳ॒॒श॒त॒म् । चाथ॑ । अथ॒ सा । सैव । ए॒व स॑हस्रत॒मी । स॒ह॒स्र॒त॒म्य॑भवत् । स॒ह॒स्र॒त॒मीति॑ सहस्र - त॒मी । अ॒भ॒व॒त् ते । ते दे॒वाः । दे॒वाः प्र॒जाप॑तिम् । प्र॒जाप॑तिमब्रुवन्न् । प्र॒जाप॑ति॒मिति॑ प्र॒जा - प॒ति॒म् । अ॒ब्रु॒व॒न्थ् स॒हस्रे॑ण । स॒हस्रे॑ण नः । नो॒ या॒ज॒य॒ । या॒ज॒येति॑ । इति॒ सः । सो᳚ऽग्निष्टो॒मेन॑ । अ॒ग्नि॒ष्टो॒मेन॒ वसून्॑ । अ॒ग्नि॒ष्टो॒मेनेत्य॑ग्नि - स्तो॒मेन॑ । वसू॑नयाजयत् । अ॒या॒ज॒य॒त् ते । त इ॒मम् । इ॒मम् ॅलो॒कम् । लो॒कम॑जयन्न् । अ॒ज॒य॒न् तत् । तच् च॑ । चा॒द॒दुः॒ । अ॒द॒दुः॒ सः । स उ॒क्थ्ये॑न । उ॒क्थ्ये॑न रु॒द्रान् । रु॒द्रान॑याजयत् । अ॒या॒ज॒य॒त् ते । ते᳚ऽन्तरि॑क्षम् । अ॒न्तरि॑क्षमजयन्न् । अ॒ज॒य॒न् तत् । तच् च॑ । चा॒द॒दुः॒ । अ॒द॒दुः॒ सः । सो॑ऽतिरा॒त्रेण॑ । अ॒ति॒रा॒त्रेणा॑दि॒त्यान् । अ॒ति॒रा॒त्रेणेत्य॑ति - रा॒त्रेण॑ । आ॒दि॒त्यान॑याजयत् । अ॒या॒ज॒य॒त् ते । ते॑ऽमुम् । अ॒मुम् ॅलो॒कम् । लो॒कम॑जयन्न् । अ॒ज॒य॒न् तत् । तच् च॑ । चा॒द॒दुः॒ । अ॒द॒दु॒स्तत् । तद॒न्तरि॑क्षम् । अ॒न्तरि॑क्ष॒म् ॅव्यवै᳚र्यत \newline

\textbf{Jatai Paata} \newline

1. श॒तानि॒ त्रय॑स्त्रिꣳशत॒म् त्रय॑स्त्रिꣳशतꣳ श॒तानि॑ श॒तानि॒ त्रय॑स्त्रिꣳशतम् । \newline
2. त्रय॑स्त्रिꣳशतम् च च॒ त्रय॑स्त्रिꣳशत॒म् त्रय॑स्त्रिꣳशतम् च । \newline
3. त्रय॑स्त्रिꣳशत॒मिति॒ त्रयः॑ - त्रिꣳ॒॒श॒त॒म् । \newline
4. चाथाथ॑ च॒ चाथ॑ । \newline
5. अथ॒ सा सा ऽथाथ॒ सा । \newline
6. सैवैव सा सैव । \newline
7. ए॒व स॑हस्रत॒मी स॑हस्रत॒ म्ये॑वैव स॑हस्रत॒मी । \newline
8. स॒ह॒स्र॒त॒ म्य॑भव दभवथ् सहस्रत॒मी स॑हस्रत॒ म्य॑भवत् । \newline
9. स॒ह॒स्र॒त॒मीति॑ सहस्र - त॒मी । \newline
10. अ॒भ॒व॒त् ते ते॑ ऽभव दभव॒त् ते । \newline
11. ते दे॒वा दे॒वा स्ते ते दे॒वाः । \newline
12. दे॒वाः प्र॒जाप॑तिम् प्र॒जाप॑तिम् दे॒वा दे॒वाः प्र॒जाप॑तिम् । \newline
13. प्र॒जाप॑ति मब्रुवन् नब्रुवन् प्र॒जाप॑तिम् प्र॒जाप॑ति मब्रुवन्न् । \newline
14. प्र॒जाप॑ति॒मिति॑ प्र॒जा - प॒ति॒म् । \newline
15. अ॒ब्रु॒व॒न् थ्स॒हस्रे॑ण स॒हस्रे॑णा ब्रुवन् नब्रुवन् थ्स॒हस्रे॑ण । \newline
16. स॒हस्रे॑ण नो नः स॒हस्रे॑ण स॒हस्रे॑ण नः । \newline
17. नो॒ या॒ज॒य॒ या॒ज॒य॒ नो॒ नो॒ या॒ज॒य॒ । \newline
18. या॒ज॒येतीति॑ याजय याज॒येति॑ । \newline
19. इति॒ स स इतीति॒ सः । \newline
20. सो᳚ ऽग्निष्टो॒मेना᳚ ग्निष्टो॒मेन॒ स सो᳚ ऽग्निष्टो॒मेन॑ । \newline
21. अ॒ग्नि॒ष्टो॒मेन॒ वसू॒न्॒. वसू॑ नग्निष्टो॒मेना᳚ ग्निष्टो॒मेन॒ वसून्॑ । \newline
22. अ॒ग्नि॒ष्टो॒मेनेत्य॑ग्नि - स्तो॒मेन॑ । \newline
23. वसू॑ नयाजय दयाजय॒द् वसू॒न्॒. वसू॑ नयाजयत् । \newline
24. अ॒या॒ज॒य॒त् ते ते॑ ऽयाजय दयाजय॒त् ते । \newline
25. त इ॒म मि॒मम् ते त इ॒मम् । \newline
26. इ॒मम् ॅलो॒कम् ॅलो॒क मि॒म मि॒मम् ॅलो॒कम् । \newline
27. लो॒क म॑जयन् नजयन् ॅलो॒कम् ॅलो॒क म॑जयन्न् । \newline
28. अ॒ज॒य॒न् तत् तद॑जयन् नजय॒न् तत् । \newline
29. तच् च॑ च॒ तत् तच् च॑ । \newline
30. चा॒ द॒दु॒ र॒द॒दु॒ श्च॒ चा॒ द॒दुः॒ । \newline
31. अ॒द॒दुः॒ स सो॑ ऽददु रददुः॒ सः । \newline
32. स उ॒क्थ्ये॑ नो॒क्थ्ये॑न॒ स स उ॒क्थ्ये॑न । \newline
33. उ॒क्थ्ये॑न रु॒द्रान् रु॒द्रा नु॒क्थ्ये॑ नो॒क्थ्ये॑न रु॒द्रान् । \newline
34. रु॒द्रा न॑याजय दयाजयद् रु॒द्रान् रु॒द्रा न॑याजयत् । \newline
35. अ॒या॒ज॒य॒त् ते ते॑ ऽयाजय दयाजय॒त् ते । \newline
36. ते᳚ ऽन्तरि॑क्ष म॒न्तरि॑क्ष॒म् ते ते᳚ ऽन्तरि॑क्षम् । \newline
37. अ॒न्तरि॑क्ष मजयन् नजयन् न॒न्तरि॑क्ष म॒न्तरि॑क्ष मजयन्न् । \newline
38. अ॒ज॒य॒न् तत् तद॑जयन् नजय॒न् तत् । \newline
39. तच् च॑ च॒ तत् तच् च॑ । \newline
40. चा॒ द॒दु॒ र॒द॒ दु॒श्च॒ चा॒ द॒दुः॒ । \newline
41. अ॒द॒दुः॒ स सो॑ ऽददु रददुः॒ सः । \newline
42. सो॑ ऽतिरा॒त्रेणा॑ तिरा॒त्रेण॒ स सो॑ ऽतिरा॒त्रेण॑ । \newline
43. अ॒ति॒रा॒त्रेणा॑ दि॒त्या ना॑दि॒त्या न॑तिरा॒त्रेणा॑ तिरा॒त्रेणा॑ दि॒त्यान् । \newline
44. अ॒ति॒रा॒त्रेणेत्य॑ति - रा॒त्रेण॑ । \newline
45. आ॒दि॒त्या न॑याजय दयाजय दादि॒त्या ना॑दि॒त्या न॑याजयत् । \newline
46. अ॒या॒ज॒य॒त् ते ते॑ ऽयाजय दयाजय॒त् ते । \newline
47. ते॑ ऽमु म॒मुम् ते ते॑ ऽमुम् । \newline
48. अ॒मुम् ॅलो॒कम् ॅलो॒क म॒मु म॒मुम् ॅलो॒कम् । \newline
49. लो॒क म॑जयन् नजयन् ॅलो॒कम् ॅलो॒क म॑जयन्न् । \newline
50. अ॒ज॒य॒न् तत् तद॑जयन् नजय॒न् तत् । \newline
51. तच् च॑ च॒ तत् तच् च॑ । \newline
52. चा॒ द॒दु॒ र॒द॒ दु॒श्च॒ चा॒ द॒दुः॒ । \newline
53. अ॒द॒दु॒ स्तत् तद॑ददु रददु॒ स्तत् । \newline
54. तद॒न्तरि॑क्ष म॒न्तरि॑क्ष॒म् तत् तद॒न्तरि॑क्षम् । \newline
55. अ॒न्तरि॑क्षं॒ ॅव्यवै᳚र्यत॒ व्यवै᳚र्यता॒ न्तरि॑क्ष म॒न्तरि॑क्षं॒ ॅव्यवै᳚र्यत । \newline

\textbf{Ghana Paata } \newline

1. श॒तानि॒ त्रय॑स्त्रिꣳशत॒म् त्रय॑स्त्रिꣳशतꣳ श॒तानि॑ श॒तानि॒ त्रय॑स्त्रिꣳशतम् च च॒ त्रय॑स्त्रिꣳशतꣳ श॒तानि॑ श॒तानि॒ त्रय॑स्त्रिꣳशतम् च । \newline
2. त्रय॑स्त्रिꣳशतम् च च॒ त्रय॑स्त्रिꣳशत॒म् त्रय॑स्त्रिꣳशत॒म् चाथाथ॑ च॒ त्रय॑स्त्रिꣳशत॒म् त्रय॑स्त्रिꣳशत॒म् चाथ॑ । \newline
3. त्रय॑स्त्रिꣳशत॒मिति॒ त्रयः॑ - त्रिꣳ॒॒श॒त॒म् । \newline
4. चाथाथ॑ च॒ चाथ॒ सा सा ऽथ॑ च॒ चाथ॒ सा । \newline
5. अथ॒ सा सा ऽथाथ॒ सैवैव सा ऽथाथ॒ सैव । \newline
6. सैवैव सा सैव स॑हस्रत॒मी स॑हस्रत॒ म्ये॑व सा सैव स॑हस्रत॒मी । \newline
7. ए॒व स॑हस्रत॒मी स॑हस्रत॒ म्ये॑वैव स॑हस्रत॒ म्य॑भव दभवथ् सहस्रत॒ म्ये॑वैव स॑हस्रत॒ म्य॑भवत् । \newline
8. स॒ह॒स्र॒त॒ म्य॑भव दभवथ् सहस्रत॒मी स॑हस्रत॒ म्य॑भव॒त् ते ते॑ ऽभवथ् सहस्रत॒मी स॑हस्रत॒ म्य॑भव॒त् ते । \newline
9. स॒ह॒स्र॒त॒मीति॑ सहस्र - त॒मी । \newline
10. अ॒भ॒व॒त् ते ते॑ ऽभव दभव॒त् ते दे॒वा दे॒वा स्ते॑ ऽभव दभव॒त् ते दे॒वाः । \newline
11. ते दे॒वा दे॒वा स्ते ते दे॒वाः प्र॒जाप॑तिम् प्र॒जाप॑तिम् दे॒वा स्ते ते दे॒वाः प्र॒जाप॑तिम् । \newline
12. दे॒वाः प्र॒जाप॑तिम् प्र॒जाप॑तिम् दे॒वा दे॒वाः प्र॒जाप॑ति मब्रुवन् नब्रुवन् प्र॒जाप॑तिम् दे॒वा दे॒वाः प्र॒जाप॑ति मब्रुवन्न् । \newline
13. प्र॒जाप॑ति मब्रुवन् नब्रुवन् प्र॒जाप॑तिम् प्र॒जाप॑ति मब्रुवन् थ्स॒हस्रे॑ण स॒हस्रे॑णा ब्रुवन् प्र॒जाप॑तिम् प्र॒जाप॑ति मब्रुवन् थ्स॒हस्रे॑ण । \newline
14. प्र॒जाप॑ति॒मिति॑ प्र॒जा - प॒ति॒म् । \newline
15. अ॒ब्रु॒व॒न् थ्स॒हस्रे॑ण स॒हस्रे॑णा ब्रुवन् नब्रुवन् थ्स॒हस्रे॑ण नो नः स॒हस्रे॑णा ब्रुवन् नब्रुवन् थ्स॒हस्रे॑ण नः । \newline
16. स॒हस्रे॑ण नो नः स॒हस्रे॑ण स॒हस्रे॑ण नो याजय याजय नः स॒हस्रे॑ण स॒हस्रे॑ण नो याजय । \newline
17. नो॒ या॒ज॒य॒ या॒ज॒य॒ नो॒ नो॒ या॒ज॒येतीति॑ याजय नो नो याज॒येति॑ । \newline
18. या॒ज॒येतीति॑ याजय याज॒येति॒ स स इति॑ याजय याज॒येति॒ सः । \newline
19. इति॒ स स इतीति॒ सो᳚ ऽग्निष्टो॒मेना᳚ ग्निष्टो॒मेन॒ स इतीति॒ सो᳚ ऽग्निष्टो॒मेन॑ । \newline
20. सो᳚ ऽग्निष्टो॒मेना᳚ ग्निष्टो॒मेन॒ स सो᳚ ऽग्निष्टो॒मेन॒ वसू॒न्॒. वसू॑ नग्निष्टो॒मेन॒ स सो᳚ ऽग्निष्टो॒मेन॒ वसून्॑ । \newline
21. अ॒ग्नि॒ष्टो॒मेन॒ वसू॒न्॒. वसू॑ नग्निष्टो॒मेना᳚ ग्निष्टो॒मेन॒ वसू॑ नयाजय दयाजय॒द् वसू॑ नग्निष्टो॒मेना᳚ग्निष्टो॒मेन॒ वसू॑ नयाजयत् । \newline
22. अ॒ग्नि॒ष्टो॒मेनेत्य॑ग्नि - स्तो॒मेन॑ । \newline
23. वसू॑ नयाजय दयाजय॒द् वसू॒न्॒. वसू॑ नयाजय॒त् ते ते॑ ऽयाजय॒द् वसू॒न्॒. वसू॑ नयाजय॒त् ते । \newline
24. अ॒या॒ज॒य॒त् ते ते॑ ऽयाजय दयाजय॒त् त इ॒म मि॒मम् ते॑ ऽयाजय दयाजय॒त् त इ॒मम् । \newline
25. त इ॒म मि॒मम् ते त इ॒मम् ॅलो॒कम् ॅलो॒क मि॒मम् ते त इ॒मम् ॅलो॒कम् । \newline
26. इ॒मम् ॅलो॒कम् ॅलो॒क मि॒म मि॒मम् ॅलो॒क म॑जयन् नजयन् ॅलो॒क मि॒म मि॒मम् ॅलो॒क म॑जयन्न् । \newline
27. लो॒क म॑जयन् नजयन् ॅलो॒कम् ॅलो॒क म॑जय॒न् तत् तद॑जयन् ॅलो॒कम् ॅलो॒क म॑जय॒न् तत् । \newline
28. अ॒ज॒य॒न् तत् तद॑जयन् नजय॒न् तच् च॑ च॒ तद॑जयन् नजय॒न् तच् च॑ । \newline
29. तच् च॑ च॒ तत् तच् चा॑ददु रददु श्च॒ तत् तच् चा॑ददुः । \newline
30. चा॒द॒दु॒ र॒द॒दु॒ श्च॒ चा॒द॒दुः॒ स सो॑ ऽददुश्च चाददुः॒ सः । \newline
31. अ॒द॒दुः॒ स सो॑ ऽददु रददुः॒ स उ॒क्थ्ये॑ नो॒क्थ्ये॑न॒ सो॑ ऽददु रददुः॒ स उ॒क्थ्ये॑न । \newline
32. स उ॒क्थ्ये॑ नो॒क्थ्ये॑न॒ स स उ॒क्थ्ये॑न रु॒द्रान् रु॒द्रा नु॒क्थ्ये॑न॒ स स उ॒क्थ्ये॑न रु॒द्रान् । \newline
33. उ॒क्थ्ये॑न रु॒द्रान् रु॒द्रा नु॒क्थ्ये॑ नो॒क्थ्ये॑न रु॒द्रा न॑याजय दयाजयद् रु॒द्रा नु॒क्थ्ये॑ नो॒क्थ्ये॑न रु॒द्रा न॑याजयत् । \newline
34. रु॒द्रा न॑याजय दयाजयद् रु॒द्रान् रु॒द्रा न॑याजय॒त् ते ते॑ ऽयाजयद् रु॒द्रान् रु॒द्रा न॑याजय॒त् ते । \newline
35. अ॒या॒ज॒य॒त् ते ते॑ ऽयाजय दयाजय॒त् ते᳚ ऽन्तरि॑क्ष म॒न्तरि॑क्ष॒म् ते॑ ऽयाजय दयाजय॒त् ते᳚ ऽन्तरि॑क्षम् । \newline
36. ते᳚ ऽन्तरि॑क्ष म॒न्तरि॑क्ष॒म् ते ते᳚ ऽन्तरि॑क्ष मजयन् नजयन् न॒न्तरि॑क्ष॒म् ते ते᳚ ऽन्तरि॑क्ष मजयन्न् । \newline
37. अ॒न्तरि॑क्ष मजयन् नजयन् न॒न्तरि॑क्ष म॒न्तरि॑क्ष मजय॒न् तत् तद॑जयन् न॒न्तरि॑क्ष म॒न्तरि॑क्ष मजय॒न् तत् । \newline
38. अ॒ज॒य॒न् तत् तद॑जयन् नजय॒न् तच् च॑ च॒ तद॑जयन् नजय॒न् तच् च॑ । \newline
39. तच् च॑ च॒ तत् तच् चा॑ददु रददुश्च॒ तत् तच् चा॑ददुः । \newline
40. चा॒द॒दु॒ र॒द॒दु॒श्च॒ चा॒द॒दुः॒ स सो॑ ऽददुश्च चाददुः॒ सः । \newline
41. अ॒द॒दुः॒ स सो॑ ऽददु रददुः॒ सो॑ ऽतिरा॒त्रेणा॑ तिरा॒त्रेण॒ सो॑ ऽददु रददुः॒ सो॑ ऽतिरा॒त्रेण॑ । \newline
42. सो॑ ऽतिरा॒त्रेणा॑ तिरा॒त्रेण॒ स सो॑ ऽतिरा॒त्रेणा॑ दि॒त्या ना॑दि॒त्या न॑तिरा॒त्रेण॒ स सो॑ ऽतिरा॒त्रेणा॑ दि॒त्यान् । \newline
43. अ॒ति॒रा॒त्रेणा॑ दि॒त्या ना॑दि॒त्या न॑तिरा॒त्रेणा॑ तिरा॒त्रेणा॑ दि॒त्या न॑याजय दयाजय दादि॒त्या न॑तिरा॒त्रेणा॑ तिरा॒त्रेणा॑ दि॒त्या न॑याजयत् । \newline
44. अ॒ति॒रा॒त्रेणेत्य॑ति - रा॒त्रेण॑ । \newline
45. आ॒दि॒त्या न॑याजय दयाजयदा दि॒त्या ना॑दि॒त्या न॑याजय॒त् ते ते॑ ऽयाजयदा दि॒त्या ना॑दि॒त्या न॑याजय॒त् ते । \newline
46. अ॒या॒ज॒य॒त् ते ते॑ ऽयाजय दयाजय॒त् ते॑ ऽमु म॒मुम् ते॑ ऽयाजय दयाजय॒त् ते॑ ऽमुम् । \newline
47. ते॑ ऽमु म॒मुम् ते ते॑ ऽमुम् ॅलो॒कम् ॅलो॒क म॒मुम् ते ते॑ ऽमुम् ॅलो॒कम् । \newline
48. अ॒मुम् ॅलो॒कम् ॅलो॒क म॒मु म॒मुम् ॅलो॒क म॑जयन् नजयन् ॅलो॒क म॒मु म॒मुम् ॅलो॒क म॑जयन्न् । \newline
49. लो॒क म॑जयन् नजयन् ॅलो॒कम् ॅलो॒क म॑जय॒न् तत् तद॑जयन् ॅलो॒कम् ॅलो॒क म॑जय॒न् तत् । \newline
50. अ॒ज॒य॒न् तत् तद॑जयन् नजय॒न् तच् च॑ च॒ तद॑जयन् नजय॒न् तच् च॑ । \newline
51. तच् च॑ च॒ तत् तच् चा॑ददु रददु श्च॒ तत् तच् चा॑ददुः । \newline
52. चा॒द॒दु॒ र॒द॒दु॒ श्च॒ चा॒द॒दु॒ स्तत् तद॑ददु श्च चाददु॒ स्तत् । \newline
53. अ॒द॒दु॒ स्तत् तद॑ददु रददु॒ स्त द॒न्तरि॑क्ष म॒न्तरि॑क्ष॒म् तद॑ददु रददु॒ स्त द॒न्तरि॑क्षम् । \newline
54. तद॒न्तरि॑क्ष म॒न्तरि॑क्ष॒म् तत् तद॒न्तरि॑क्षं॒ ॅव्यवै᳚र्यत॒ व्यवै᳚र्यता॒ न्तरि॑क्ष॒म् तत् तद॒न्तरि॑क्षं॒ ॅव्यवै᳚र्यत । \newline
55. अ॒न्तरि॑क्षं॒ ॅव्यवै᳚र्यत॒ व्यवै᳚र्यता॒ न्तरि॑क्ष म॒न्तरि॑क्षं॒ ॅव्यवै᳚र्यत॒ तस्मा॒त् तस्मा॒द् व्यवै᳚र्यता॒ न्तरि॑क्ष म॒न्तरि॑क्षं॒ ॅव्यवै᳚र्यत॒ तस्मा᳚त् । \newline
\pagebreak
\markright{ TS 7.1.5.4  \hfill https://www.vedavms.in \hfill}

\section{ TS 7.1.5.4 }

\textbf{TS 7.1.5.4 } \newline
\textbf{Samhita Paata} \newline

ॅव्यवै᳚र्यत॒ तस्मा᳚द्-रु॒द्रा घातु॑का अनायत॒ना हि तस्मा॑दाहुः शिथि॒लं ॅवै म॑द्ध्य॒म-मह॑स्त्रिरा॒त्रस्य॒ वि हि तद॒वैर्य॒तेति॒ त्रैष्टु॑भं मद्ध्य॒मस्याह्न॒ आज्यं॑ भवति सं॒ॅयाना॑नि सू॒क्तानि॑ शꣳसति षोड॒शिनꣳ॑ शꣳस॒त्यह्नो॒ धृत्या॒ अशि॑थिलंभावाय॒ तस्मा᳚त् त्रिरा॒त्रस्या᳚ग्निष्टो॒म ए॒व प्र॑थ॒ममहः॑ स्या॒दथो॒क्थ्यो ऽथा॑ऽतिरा॒त्र ए॒षां ॅलो॒कानां॒ ॅविधृ॑त्यै॒ त्रीणि॑त्रीणि श॒ता-न्य॑नूचीना॒ह-मव्य॑वच्छिन्नानि ददा - [  ] \newline

\textbf{Pada Paata} \newline

व्यवै᳚र्य॒तेति॑ वि - अवै᳚र्यत । तस्मा᳚त् । रु॒द्राः । घातु॑काः । अ॒ना॒य॒त॒ना इत्य॑ना - य॒त॒नाः । हि । तस्मा᳚त् । आ॒हुः॒ । शि॒थि॒लम् । वै । म॒द्ध्य॒मम् । अहः॑ । त्रि॒रा॒त्रस्येति॑ त्रि - रा॒त्रस्य॑ । वीति॑ । हि । तत् । अ॒वैर्य॒तेत्य॑व - ऐर्य॑त । इति॑ । त्रैष्टु॑भम् । म॒द्ध्य॒मस्य॑ । अह्नः॑ । आज्य᳚म् । भ॒व॒ति॒ । सं॒ॅयाना॒निति॑ सम् - याना॑नि । सू॒क्तानीति॑ सु - उ॒क्तानि॑ । शꣳ॒॒स॒ति॒ । षो॒ड॒शिन᳚म् । शꣳ॒॒स॒ति॒ । अह्नः॑ । धृत्यै᳚ । अशि॑थिलम्भावा॒येत्यशि॑थिलं - भा॒वा॒य॒ । तस्मा᳚त् । त्रि॒रा॒त्रस्येति॑ त्रि - रा॒त्रस्य॑ । अ॒ग्नि॒ष्टो॒म इत्य॑ग्नि - स्तो॒मः । ए॒व । प्र॒थ॒मम् । अहः॑ । स्या॒त् । अथ॑ । उ॒क्थ्यः॑ । अथ॑ । अ॒ति॒रा॒त्र इत्य॑ति - रा॒त्रः । ए॒षाम् । लो॒काना᳚म् । विधृ॑त्या॒ इति॒ वि - धृ॒त्यै॒ । त्रीणि॑त्री॒णीति॒ त्रीणि॑ - त्री॒णि॒ । श॒तानि॑ । अ॒नू॒ची॒ना॒हमित्य॑नूचीन - अ॒हम् । अव्य॑वच्छिन्ना॒नीत्यवि॑ - अ॒व॒च्छि॒न्ना॒नि॒ । द॒दा॒ति॒ ।  \newline


\textbf{Krama Paata} \newline

व्यवै᳚र्यत॒ तस्मा᳚त् । व्यवै᳚र्य॒तेति॑ वि - अवै᳚र्यत । तस्मा᳚द् रु॒द्राः । रु॒द्रा घातु॑काः । घातु॑का अनायत॒नाः । अ॒ना॒य॒त॒ना हि । अ॒ना॒य॒त॒ना इत्य॑ना - य॒त॒नाः । हि तस्मा᳚त् । तस्मा॑दाहुः । आ॒हुः॒ शि॒थि॒लम् । शि॒थि॒लम् ॅवै । वै म॑द्ध्य॒मम् । म॒द्ध्य॒ममहः॑ । अह॑स्त्रिरा॒त्रस्य॑ । त्रि॒रा॒त्रस्य॒ वि । त्रि॒रा॒त्रस्येति॑ त्रि - रा॒त्रस्य॑ । वि हि । हि तत् । तद॒वैर्य॑त । अ॒वैर्य॒तेति॑ । अ॒वैर्य॒तेत्य॑व - ऐर्य॑त । इति॒ त्रैष्टु॑भम् । त्रैष्टु॑भम् मद्ध्य॒मस्य॑ । म॒द्ध्य॒मस्याह्नः॑ । अह्न॒ आज्य᳚म् । आज्य॑म् भवति । भ॒व॒ति॒ स॒म्ॅयाना॑नि । स॒म्ॅयाना॑नि सू॒क्तानि॑ । स॒म्ॅयाना॒नीति॑ सम् - याना॑नि । सू॒क्तानि॑ शꣳसति । सू॒क्तानीति॑ सु - उ॒क्तानि॑ । शꣳ॒॒स॒ति॒ षो॒ड॒शिन᳚म् । षो॒ड॒शिनꣳ॑ शꣳसति । शꣳ॒॒स॒त्यह्नः॑ । अह्नो॒ धृत्यै᳚ । धृत्या॒ अशि॑थिलम्भावाय । अशि॑थिलम्भावाय॒ तस्मा᳚त् । अशि॑थिलम्भावा॒येत्यशि॑थिलम् - भा॒वा॒य॒ । तस्मा᳚त् त्रिरा॒त्रस्य॑ । त्रि॒रा॒त्रस्या᳚ग्निष्टो॒मः । त्रि॒रा॒त्रस्येति॑ त्रि - रा॒त्रस्य॑ । अ॒ग्नि॒ष्टो॒म ए॒व । अ॒ग्नि॒ष्टो॒म इत्य॑ग्नि - स्तो॒मः । ए॒व प्र॑थ॒मम् । प्र॒थ॒ममहः॑ । अहः॑ स्यात् । स्या॒दथ॑ । अथो॒क्थ्यः॑ । उ॒क्थ्योऽथ॑ । अथा॑तिरा॒त्रः । अ॒ति॒रा॒त्र ए॒षाम् । अ॒ति॒रा॒त्र इत्य॑ति - रा॒त्रः । ए॒षाम् ॅलो॒काना᳚म् । लो॒काना॒म् ॅविधृ॑त्यै । विधृ॑त्यै॒ त्रीणि॑त्रीणि । विधृ॑त्या॒ इति॒ वि - धृ॒त्यै॒ । त्रीणि॑त्रीणि श॒तानि॑ । त्रीणि॑त्री॒णीति॒ त्रीणि॑ - त्री॒णि॒ । श॒तान्य॑नूचीना॒हम् । अ॒नू॒ची॒ना॒हमव्य॑वच्छिन्नानि । अ॒नू॒ची॒ना॒हमित्य॑नूचीन - अ॒हम् । अव्य॑वच्छिन्नानि ददाति । अव्य॑वच्छिन्ना॒नीत्यवि॑ - अ॒व॒च्छ॒न्ना॒नि॒ । द॒दा॒त्ये॒षाम् \newline

\textbf{Jatai Paata} \newline

1. व्यवै᳚र्यत॒ तस्मा॒त् तस्मा॒द् व्यवै᳚र्यत॒ व्यवै᳚र्यत॒ तस्मा᳚त् । \newline
2. व्यवै᳚र्य॒तेति॑ वि - अवै᳚र्यत । \newline
3. तस्मा᳚द् रु॒द्रा रु॒द्रा स्तस्मा॒त् तस्मा᳚द् रु॒द्राः । \newline
4. रु॒द्रा घातु॑का॒ घातु॑का रु॒द्रा रु॒द्रा घातु॑काः । \newline
5. घातु॑का अनायत॒ना अ॑नायत॒ना घातु॑का॒ घातु॑का अनायत॒नाः । \newline
6. अ॒ना॒य॒त॒ना हि ह्य॑नायत॒ना अ॑नायत॒ना हि । \newline
7. अ॒ना॒य॒त॒ना इत्य॑ना - य॒त॒नाः । \newline
8. हि तस्मा॒त् तस्मा॒द्धि हि तस्मा᳚त् । \newline
9. तस्मा॑ दाहु राहु॒ स्तस्मा॒त् तस्मा॑ दाहुः । \newline
10. आ॒हुः॒ शि॒थि॒लꣳ शि॑थि॒ल मा॑हु राहुः शिथि॒लम् । \newline
11. शि॒थि॒लं ॅवै वै शि॑थि॒लꣳ शि॑थि॒लं ॅवै । \newline
12. वै म॑द्ध्य॒मम् म॑द्ध्य॒मं ॅवै वै म॑द्ध्य॒मम् । \newline
13. म॒द्ध्य॒म मह॒ रह॑र् मद्ध्य॒मम् म॑द्ध्य॒म महः॑ । \newline
14. अह॑ स्त्रिरा॒त्रस्य॑ त्रिरा॒त्र स्याह॒ रह॑ स्त्रिरा॒त्रस्य॑ । \newline
15. त्रि॒रा॒त्रस्य॒ वि वि त्रि॑रा॒त्रस्य॑ त्रिरा॒त्रस्य॒ वि । \newline
16. त्रि॒रा॒त्रस्येति॑ त्रि - रा॒त्रस्य॑ । \newline
17. वि हि हि वि वि हि । \newline
18. हि तत् तद्धि हि तत् । \newline
19. तद॒वैर्य॑ता॒ वैर्य॑त॒ तत् तद॒वैर्य॑त । \newline
20. अ॒वैर्य॒ते तीत्य॒ वैर्य॑ता॒ वैर्य॒ते ति॑ । \newline
21. अ॒वैर्य॒तेत्य॑व - ऐर्य॑त । \newline
22. इति॒ त्रैष्टु॑भ॒म् त्रैष्टु॑भ॒ मितीति॒ त्रैष्टु॑भम् । \newline
23. त्रैष्टु॑भम् मद्ध्य॒मस्य॑ मद्ध्य॒मस्य॒ त्रैष्टु॑भ॒म् त्रैष्टु॑भम् मद्ध्य॒मस्य॑ । \newline
24. म॒द्ध्य॒म स्याह्नो ऽह्नो॑ मद्ध्य॒मस्य॑ मद्ध्य॒म स्याह्नः॑ । \newline
25. अह्न॒ आज्य॒ माज्य॒ मह्नो ऽह्न॒ आज्य᳚म् । \newline
26. आज्य॑म् भवति भव॒ त्याज्य॒ माज्य॑म् भवति । \newline
27. भ॒व॒ति॒ सं॒ॅयाना॑नि सं॒ॅयाना॑नि भवति भवति सं॒ॅयाना॑नि । \newline
28. सं॒ॅयाना॑नि सू॒क्तानि॑ सू॒क्तानि॑ सं॒ॅयाना॑नि सं॒ॅयाना॑नि सू॒क्तानि॑ । \newline
29. सं॒ॅयाना॒नीति॑ सम् - याना॑नि । \newline
30. सू॒क्तानि॑ शꣳसति शꣳसति सू॒क्तानि॑ सू॒क्तानि॑ शꣳसति । \newline
31. सू॒क्तानीति॑ सु - उ॒क्तानि॑ । \newline
32. शꣳ॒॒स॒ति॒ षो॒ड॒शिनꣳ॑ षोड॒शिनꣳ॑ शꣳसति शꣳसति षोड॒शिन᳚म् । \newline
33. षो॒ड॒शिनꣳ॑ शꣳसति शꣳसति षोड॒शिनꣳ॑ षोड॒शिनꣳ॑ शꣳसति । \newline
34. शꣳ॒॒स॒ त्यह्नो ऽह्नः॑ शꣳसति शꣳस॒ त्यह्नः॑ । \newline
35. अह्नो॒ धृत्यै॒ धृत्या॒ अह्नो ऽह्नो॒ धृत्यै᳚ । \newline
36. धृत्या॒ अशि॑थिलम्भावा॒या शि॑थिलम्भावाय॒ धृत्यै॒ धृत्या॒ अशि॑थिलम्भावाय । \newline
37. अशि॑थिलम्भावाय॒ तस्मा॒त् तस्मा॒ दशि॑थिलम्भावा॒या शि॑थिलम्भावाय॒ तस्मा᳚त् । \newline
38. अशि॑थिलम्भावा॒येत्यशि॑थिलं - भा॒वा॒य॒ । \newline
39. तस्मा᳚त् त्रिरा॒त्रस्य॑ त्रिरा॒त्रस्य॒ तस्मा॒त् तस्मा᳚त् त्रिरा॒त्रस्य॑ । \newline
40. त्रि॒रा॒त्रस्या᳚ ग्निष्टो॒मो᳚ ऽग्निष्टो॒म स्त्रि॑रा॒त्रस्य॑ त्रिरा॒त्रस्या᳚ ग्निष्टो॒मः । \newline
41. त्रि॒रा॒त्रस्येति॑ त्रि - रा॒त्रस्य॑ । \newline
42. अ॒ग्नि॒ष्टो॒म ए॒वैवा ग्नि॑ष्टो॒मो᳚ ऽग्निष्टो॒म ए॒व । \newline
43. अ॒ग्नि॒ष्टो॒म इत्य॑ग्नि - स्तो॒मः । \newline
44. ए॒व प्र॑थ॒मम् प्र॑थ॒म मे॒वैव प्र॑थ॒मम् । \newline
45. प्र॒थ॒म मह॒ रहः॑ प्रथ॒मम् प्र॑थ॒म महः॑ । \newline
46. अहः॑ स्याथ् स्या॒ दह॒ रहः॑ स्यात् । \newline
47. स्या॒ दथाथ॑ स्याथ् स्या॒ दथ॑ । \newline
48. अथो॒ क्थ्य॑ उ॒क्थ्यो ऽथाथो॒ क्थ्यः॑ । \newline
49. उ॒क्थ्यो ऽथाथो॒ क्थ्य॑ उ॒क्थ्यो ऽथ॑ । \newline
50. अथा॑ति रा॒त्रो॑ ऽतिरा॒त्रो ऽथाथा॑ तिरा॒त्रः । \newline
51. अ॒ति॒रा॒त्र ए॒षा मे॒षा म॑तिरा॒त्रो॑ ऽतिरा॒त्र ए॒षाम् । \newline
52. अ॒ति॒रा॒त्र इत्य॑ति - रा॒त्रः । \newline
53. ए॒षाम् ॅलो॒काना᳚म् ॅलो॒काना॑ मे॒षा मे॒षाम् ॅलो॒काना᳚म् । \newline
54. लो॒कानां॒ ॅविधृ॑त्यै॒ विधृ॑त्यै लो॒काना᳚म् ॅलो॒कानां॒ ॅविधृ॑त्यै । \newline
55. विधृ॑त्यै॒ त्रीणि॑त्रीणि॒ त्रीणि॑त्रीणि॒ विधृ॑त्यै॒ विधृ॑त्यै॒ त्रीणि॑त्रीणि । \newline
56. विधृ॑त्या॒ इति॒ वि - धृ॒त्यै॒ । \newline
57. त्रीणि॑त्रीणि श॒तानि॑ श॒तानि॒ त्रीणि॑त्रीणि॒ त्रीणि॑त्रीणि श॒तानि॑ । \newline
58. त्रीणि॑त्री॒णीति॒ त्रीणि॑ - त्री॒णि॒ । \newline
59. श॒ता न्य॑नूचीना॒ह म॑नूचीना॒हꣳ श॒तानि॑ श॒ता न्य॑नूचीना॒हम् । \newline
60. अ॒नू॒ची॒ना॒ह मव्य॑वच्छिन्ना॒ न्यव्य॑वच्छिन्ना न्यनूचीना॒ह म॑नूचीना॒ह मव्य॑वच्छिन्नानि । \newline
61. अ॒नू॒ची॒ना॒हमित्य॑नूचीन - अ॒हम् । \newline
62. अव्य॑वच्छिन्नानि ददाति ददा॒ त्यव्य॑वच्छिन्ना॒ न्यव्य॑वच्छिन्नानि ददाति । \newline
63. अव्य॑वच्छिन्ना॒नीत्यवि॑ - अ॒व॒च्छि॒न्ना॒नि॒ । \newline
64. द॒दा॒ त्ये॒षा मे॒षाम् द॑दाति ददा त्ये॒षाम् । \newline

\textbf{Ghana Paata } \newline

1. व्यवै᳚र्यत॒ तस्मा॒त् तस्मा॒द् व्यवै᳚र्यत॒ व्यवै᳚र्यत॒ तस्मा᳚द् रु॒द्रा रु॒द्रा स्तस्मा॒द् व्यवै᳚र्यत॒ व्यवै᳚र्यत॒ तस्मा᳚द् रु॒द्राः । \newline
2. व्यवै᳚र्य॒तेति॑ वि - अवै᳚र्यत । \newline
3. तस्मा᳚द् रु॒द्रा रु॒द्रा स्तस्मा॒त् तस्मा᳚द् रु॒द्रा घातु॑का॒ घातु॑का रु॒द्रा स्तस्मा॒त् तस्मा᳚द् रु॒द्रा घातु॑काः । \newline
4. रु॒द्रा घातु॑का॒ घातु॑का रु॒द्रा रु॒द्रा घातु॑का अनायत॒ना अ॑नायत॒ना घातु॑का रु॒द्रा रु॒द्रा घातु॑का अनायत॒नाः । \newline
5. घातु॑का अनायत॒ना अ॑नायत॒ना घातु॑का॒ घातु॑का अनायत॒ना हि ह्य॑नायत॒ना घातु॑का॒ घातु॑का अनायत॒ना हि । \newline
6. अ॒ना॒य॒त॒ना हि ह्य॑नायत॒ना अ॑नायत॒ना हि तस्मा॒त् तस्मा॒ द्ध्य॑नायत॒ना अ॑नायत॒ना हि तस्मा᳚त् । \newline
7. अ॒ना॒य॒त॒ना इत्य॑ना - य॒त॒नाः । \newline
8. हि तस्मा॒त् तस्मा॒द्धि हि तस्मा॑ दाहु राहु॒ स्तस्मा॒द्धि हि तस्मा॑ दाहुः । \newline
9. तस्मा॑ दाहु राहु॒ स्तस्मा॒त् तस्मा॑ दाहुः शिथि॒लꣳ शि॑थि॒ल मा॑हु॒ स्तस्मा॒त् तस्मा॑ दाहुः शिथि॒लम् । \newline
10. आ॒हुः॒ शि॒थि॒लꣳ शि॑थि॒ल मा॑हु राहुः शिथि॒लं ॅवै वै शि॑थि॒ल मा॑हु राहुः शिथि॒लं ॅवै । \newline
11. शि॒थि॒लं ॅवै वै शि॑थि॒लꣳ शि॑थि॒लं ॅवै म॑द्ध्य॒मम् म॑द्ध्य॒मं ॅवै शि॑थि॒लꣳ शि॑थि॒लं ॅवै म॑द्ध्य॒मम् । \newline
12. वै म॑द्ध्य॒मम् म॑द्ध्य॒मं ॅवै वै म॑द्ध्य॒म मह॒ रह॑र् मद्ध्य॒मं ॅवै वै म॑द्ध्य॒म महः॑ । \newline
13. म॒द्ध्य॒म मह॒ रह॑र् मद्ध्य॒मम् म॑द्ध्य॒म मह॑ स्त्रिरा॒त्रस्य॑ त्रिरा॒त्रस्या ह॑र् मद्ध्य॒मम् म॑द्ध्य॒म मह॑ स्त्रिरा॒त्रस्य॑ । \newline
14. अह॑ स्त्रिरा॒त्रस्य॑ त्रिरा॒त्रस्या ह॒ रह॑ स्त्रिरा॒त्रस्य॒ वि वि त्रि॑रा॒त्रस्या ह॒ रह॑ स्त्रिरा॒त्रस्य॒ वि । \newline
15. त्रि॒रा॒त्रस्य॒ वि वि त्रि॑रा॒त्रस्य॑ त्रिरा॒त्रस्य॒ वि हि हि वि त्रि॑रा॒त्रस्य॑ त्रिरा॒त्रस्य॒ वि हि । \newline
16. त्रि॒रा॒त्रस्येति॑ त्रि - रा॒त्रस्य॑ । \newline
17. वि हि हि वि वि हि तत् तद्धि वि वि हि तत् । \newline
18. हि तत् तद्धि हि तद॒वैर्य॑ता॒ वैर्य॑त॒ तद्धि हि तद॒वैर्य॑त । \newline
19. तद॒वैर्य॑ता॒ वैर्य॑त॒ तत् तद॒वैर्य॒तेती त्य॒वैर्य॑त॒ तत् तद॒वैर्य॒तेति॑ । \newline
20. अ॒वैर्य॒तेती त्य॒वैर्य॑ता॒ वैर्य॒तेति॒ त्रैष्टु॑भ॒म् त्रैष्टु॑भ॒ मित्य॒वैर्य॑ता॒ वैर्य॒तेति॒ त्रैष्टु॑भम् । \newline
21. अ॒वैर्य॒तेत्य॑व - ऐर्य॑त । \newline
22. इति॒ त्रैष्टु॑भ॒म् त्रैष्टु॑भ॒ मितीति॒ त्रैष्टु॑भम् मद्ध्य॒मस्य॑ मद्ध्य॒मस्य॒ त्रैष्टु॑भ॒ मितीति॒ त्रैष्टु॑भम् मद्ध्य॒मस्य॑ । \newline
23. त्रैष्टु॑भम् मद्ध्य॒मस्य॑ मद्ध्य॒मस्य॒ त्रैष्टु॑भ॒म् त्रैष्टु॑भम् मद्ध्य॒म स्याह्नो ऽह्नो॑ मद्ध्य॒मस्य॒ त्रैष्टु॑भ॒म् त्रैष्टु॑भम् मद्ध्य॒म स्याह्नः॑ । \newline
24. म॒द्ध्य॒म स्याह्नो ऽह्नो॑ मद्ध्य॒मस्य॑ मद्ध्य॒म स्याह्न॒ आज्य॒ माज्य॒ मह्नो॑ मद्ध्य॒मस्य॑ मद्ध्य॒म स्याह्न॒ आज्य᳚म् । \newline
25. अह्न॒ आज्य॒ माज्य॒ मह्नो ऽह्न॒ आज्य॑म् भवति भव॒ त्याज्य॒ मह्नो ऽह्न॒ आज्य॑म् भवति । \newline
26. आज्य॑म् भवति भव॒ त्याज्य॒ माज्य॑म् भवति सं॒ॅयाना॑नि सं॒ॅयाना॑नि भव॒ त्याज्य॒ माज्य॑म् भवति सं॒ॅयाना॑नि । \newline
27. भ॒व॒ति॒ सं॒ॅयाना॑नि सं॒ॅयाना॑नि भवति भवति सं॒ॅयाना॑नि सू॒क्तानि॑ सू॒क्तानि॑ सं॒ॅयाना॑नि भवति भवति सं॒ॅयाना॑नि सू॒क्तानि॑ । \newline
28. सं॒ॅयाना॑नि सू॒क्तानि॑ सू॒क्तानि॑ सं॒ॅयाना॑नि सं॒ॅयाना॑नि सू॒क्तानि॑ शꣳसति शꣳसति सू॒क्तानि॑ सं॒ॅयाना॑नि सं॒ॅयाना॑नि सू॒क्तानि॑ शꣳसति । \newline
29. सं॒ॅयाना॒नीति॑ सम् - याना॑नि । \newline
30. सू॒क्तानि॑ शꣳसति शꣳसति सू॒क्तानि॑ सू॒क्तानि॑ शꣳसति षोड॒शिनꣳ॑ षोड॒शिनꣳ॑ शꣳसति सू॒क्तानि॑ सू॒क्तानि॑ शꣳसति षोड॒शिन᳚म् । \newline
31. सू॒क्तानीति॑ सु - उ॒क्तानि॑ । \newline
32. शꣳ॒॒स॒ति॒ षो॒ड॒शिनꣳ॑ षोड॒शिनꣳ॑ शꣳसति शꣳसति षोड॒शिनꣳ॑ शꣳसति शꣳसति षोड॒शिनꣳ॑ शꣳसति शꣳसति षोड॒शिनꣳ॑ शꣳसति । \newline
33. षो॒ड॒शिनꣳ॑ शꣳसति शꣳसति षोड॒शिनꣳ॑ षोड॒शिनꣳ॑ शꣳस॒ त्यह्नो ऽह्नः॑ शꣳसति षोड॒शिनꣳ॑ षोड॒शिनꣳ॑ शꣳस॒ त्यह्नः॑ । \newline
34. शꣳ॒॒स॒ त्यह्नो ऽह्नः॑ शꣳसति शꣳस॒ त्यह्नो॒ धृत्यै॒ धृत्या॒ अह्नः॑ शꣳसति शꣳस॒ त्यह्नो॒ धृत्यै᳚ । \newline
35. अह्नो॒ धृत्यै॒ धृत्या॒ अह्नो ऽह्नो॒ धृत्या॒ अशि॑थिलम्भावा॒या शि॑थिलम्भावाय॒ धृत्या॒ अह्नो ऽह्नो॒ धृत्या॒ अशि॑थिलम्भावाय । \newline
36. धृत्या॒ अशि॑थिलम्भावा॒या शि॑थिलम्भावाय॒ धृत्यै॒ धृत्या॒ अशि॑थिलम्भावाय॒ तस्मा॒त् तस्मा॒ दशि॑थिलम्भावाय॒ धृत्यै॒ धृत्या॒ अशि॑थिलम्भावाय॒ तस्मा᳚त् । \newline
37. अशि॑थिलम्भावाय॒ तस्मा॒त् तस्मा॒ दशि॑थिलम्भावा॒या शि॑थिलम्भावाय॒ तस्मा᳚त् त्रिरा॒त्रस्य॑ त्रिरा॒त्रस्य॒ तस्मा॒ दशि॑थिलम्भावा॒या शि॑थिलम्भावाय॒ तस्मा᳚त् त्रिरा॒त्रस्य॑ । \newline
38. अशि॑थिलम्भावा॒येत्यशि॑थिलं - भा॒वा॒य॒ । \newline
39. तस्मा᳚त् त्रिरा॒त्रस्य॑ त्रिरा॒त्रस्य॒ तस्मा॒त् तस्मा᳚त् त्रिरा॒त्रस्या᳚ ग्निष्टो॒मो᳚ ऽग्निष्टो॒म स्त्रि॑रा॒त्रस्य॒ तस्मा॒त् तस्मा᳚त् त्रिरा॒त्रस्या᳚ ग्निष्टो॒मः । \newline
40. त्रि॒रा॒त्रस्या᳚ ग्निष्टो॒मो᳚ ऽग्निष्टो॒म स्त्रि॑रा॒त्रस्य॑ त्रिरा॒त्रस्या᳚ ग्निष्टो॒म ए॒वैवा ग्नि॑ष्टो॒म स्त्रि॑रा॒त्रस्य॑ त्रिरा॒त्रस्या᳚ ग्निष्टो॒म ए॒व । \newline
41. त्रि॒रा॒त्रस्येति॑ त्रि - रा॒त्रस्य॑ । \newline
42. अ॒ग्नि॒ष्टो॒म ए॒वैवा ग्नि॑ष्टो॒मो᳚ ऽग्निष्टो॒म ए॒व प्र॑थ॒मम् प्र॑थ॒म मे॒वा ग्नि॑ष्टो॒मो᳚ ऽग्निष्टो॒म ए॒व प्र॑थ॒मम् । \newline
43. अ॒ग्नि॒ष्टो॒म इत्य॑ग्नि - स्तो॒मः । \newline
44. ए॒व प्र॑थ॒मम् प्र॑थ॒म मे॒वैव प्र॑थ॒म मह॒ रहः॑ प्रथ॒म मे॒वैव प्र॑थ॒म महः॑ । \newline
45. प्र॒थ॒म मह॒ रहः॑ प्रथ॒मम् प्र॑थ॒म महः॑ स्याथ् स्या॒ दहः॑ प्रथ॒मम् प्र॑थ॒म महः॑ स्यात् । \newline
46. अहः॑ स्याथ् स्या॒ दह॒ रहः॑ स्या॒ दथाथ॑ स्या॒ दह॒ रहः॑ स्या॒ दथ॑ । \newline
47. स्या॒ दथाथ॑ स्याथ् स्या॒ दथो॒क्थ्य॑ उ॒क्थ्यो ऽथ॑ स्याथ् स्या॒ दथो॒क्थ्यः॑ । \newline
48. अथो॒क्थ्य॑ उ॒क्थ्यो ऽथाथो॒ क्थ्यो ऽथाथो॒ क्थ्यो ऽथाथो॒ क्थ्यो ऽथ॑ । \newline
49. उ॒क्थ्यो ऽथाथो॒ क्थ्य॑ उ॒क्थ्यो ऽथा॑ तिरा॒त्रो॑ ऽतिरा॒त्रो ऽथो॒क्थ्य॑ उ॒क्थ्यो ऽथा॑ तिरा॒त्रः । \newline
50. अथा॑तिरा॒त्रो॑ ऽतिरा॒त्रो ऽथाथा॑ तिरा॒त्र ए॒षा मे॒षा म॑तिरा॒त्रो ऽथाथा॑ तिरा॒त्र ए॒षाम् । \newline
51. अ॒ति॒रा॒त्र ए॒षा मे॒षा म॑तिरा॒त्रो॑ ऽतिरा॒त्र ए॒षाम् ॅलो॒काना᳚म् ॅलो॒काना॑ मे॒षा म॑तिरा॒त्रो॑ ऽतिरा॒त्र ए॒षाम् ॅलो॒काना᳚म् । \newline
52. अ॒ति॒रा॒त्र इत्य॑ति - रा॒त्रः । \newline
53. ए॒षाम् ॅलो॒काना᳚म् ॅलो॒काना॑ मे॒षा मे॒षाम् ॅलो॒कानां॒ ॅविधृ॑त्यै॒ विधृ॑त्यै लो॒काना॑ मे॒षा मे॒षाम् ॅलो॒कानां॒ ॅविधृ॑त्यै । \newline
54. लो॒कानां॒ ॅविधृ॑त्यै॒ विधृ॑त्यै लो॒काना᳚म् ॅलो॒कानां॒ ॅविधृ॑त्यै॒ त्रीणि॑त्रीणि॒ त्रीणि॑त्रीणि॒ विधृ॑त्यै लो॒काना᳚म् ॅलो॒कानां॒ ॅविधृ॑त्यै॒ त्रीणि॑त्रीणि । \newline
55. विधृ॑त्यै॒ त्रीणि॑त्रीणि॒ त्रीणि॑त्रीणि॒ विधृ॑त्यै॒ विधृ॑त्यै॒ त्रीणि॑त्रीणि श॒तानि॑ श॒तानि॒ त्रीणि॑त्रीणि॒ विधृ॑त्यै॒ विधृ॑त्यै॒ त्रीणि॑त्रीणि श॒तानि॑ । \newline
56. विधृ॑त्या॒ इति॒ वि - धृ॒त्यै॒ । \newline
57. त्रीणि॑त्रीणि श॒तानि॑ श॒तानि॒ त्रीणि॑त्रीणि॒ त्रीणि॑त्रीणि श॒ता न्य॑नूचीना॒ह म॑नूचीना॒हꣳ श॒तानि॒ त्रीणि॑त्रीणि॒ त्रीणि॑त्रीणि श॒ता न्य॑नूचीना॒हम् । \newline
58. त्रीणि॑त्री॒णीति॒ त्रीणि॑ - त्री॒णि॒ । \newline
59. श॒ता न्य॑नूचीना॒ह म॑नूचीना॒हꣳ श॒तानि॑ श॒ता न्य॑नूचीना॒ह मव्य॑वच्छिन्ना॒ न्यव्य॑वच्छिन्ना
न्यनूचीना॒हꣳ श॒तानि॑ श॒ता न्य॑नूचीना॒ह मव्य॑वच्छिन्नानि । \newline
60. अ॒नू॒ची॒ना॒ह मव्य॑वच्छिन्ना॒ न्यव्य॑वच्छिन्ना न्यनूचीना॒ह म॑नूचीना॒ह मव्य॑वच्छिन्नानि ददाति ददा॒ त्यव्य॑वच्छिन्ना न्यनूचीना॒ह म॑नूचीना॒ह मव्य॑वच्छिन्नानि ददाति । \newline
61. अ॒नू॒ची॒ना॒हमित्य॑नूचीन - अ॒हम् । \newline
62. अव्य॑वच्छिन्नानि ददाति ददा॒ त्यव्य॑वच्छिन्ना॒ न्यव्य॑वच्छिन्नानि ददा त्ये॒षा मे॒षाम् द॑दा॒ त्यव्य॑वच्छिन्ना॒
न्यव्य॑वच्छिन्नानि ददात्ये॒षाम् । \newline
63. अव्य॑वच्छिन्ना॒नीत्यवि॑ - अ॒व॒च्छि॒न्ना॒नि॒ । \newline
64. द॒दा॒ त्ये॒षा मे॒षाम् द॑दाति ददा त्ये॒षाम् ॅलो॒काना᳚म् ॅलो॒काना॑ मे॒षाम् द॑दाति ददा त्ये॒षाम् ॅलो॒काना᳚म् । \newline
\pagebreak
\markright{ TS 7.1.5.5  \hfill https://www.vedavms.in \hfill}

\section{ TS 7.1.5.5 }

\textbf{TS 7.1.5.5 } \newline
\textbf{Samhita Paata} \newline

-त्ये॒षां ॅलो॒काना॒मनु॒ सन्त॑त्यै द॒शतं॒ न विच्छि॑न्द्याद्-वि॒राजं॒ नेद्वि॑च्छि॒नदा॒नीत्यथ॒ या स॑हस्रत॒म्यासी॒त् तस्या॒मिन्द्र॑श्च॒ विष्णु॑श्च॒ व्याय॑च्छेताꣳ॒॒ स इन्द्रो॑ऽमन्यता॒नया॒ वा इ॒दं ॅविष्णुः॑ स॒हस्रं॑ ॅवर्क्ष्यत॒ इति॒ तस्या॑मकल्पेतां॒ द्विभा॑ग॒ इन्द्र॒स्तृती॑ये॒ विष्णु॒स्तद्वा ए॒षाऽभ्यनू᳚च्यत उ॒भा जि॑ग्यथु॒रिति॒ तां ॅवा ए॒ताम॑च्छावा॒क - [  ] \newline

\textbf{Pada Paata} \newline

ए॒षाम् । लो॒काना᳚म् । अन्विति॑ । सन्त॑त्या॒ इति॒ सं - त॒त्यै॒ । द॒शत᳚म् । न । वीति॑ । छि॒न्द्या॒त् । वि॒राज॒मिति॑ वि - राज᳚म् । न । इत् । वि॒च्छि॒नदा॒नीति॑ वि - छि॒नदा॑नि । इति॑ । अथ॑ । या । स॒ह॒स्र॒त॒मीति॑ सहस्र - त॒मी । आसी᳚त् । तस्या᳚म् । इन्द्रः॑ । च॒ । विष्णुः॑ । च॒ । व्याय॑च्छेता॒मिति॑ वि - आय॑च्छेताम् । सः । इन्द्रः॑ । अ॒म॒न्य॒त॒ । अ॒नया᳚ । वै । इ॒दम् । विष्णुः॑ । स॒हस्र᳚म् । व॒र्क्ष्य॒ते॒ । इति॑ । तस्या᳚म् । अ॒क॒ल्पे॒ता॒म् । द्विभा॑ग॒ इति॒ द्वि - भा॒गे॒ । इन्द्रः॑ । तृती॑ये । विष्णुः॑ । तत् । वै । ए॒षा । अ॒भ्यनू᳚च्यत॒ इत्य॑भि-अनू᳚च्यते । उ॒भा । जि॒ग्य॒थुः॒ । इति॑ । ताम् । वै । ए॒ताम् । अ॒च्छा॒वा॒कः ।  \newline


\textbf{Krama Paata} \newline

ए॒षाम् ॅलो॒काना᳚म् । लो॒काना॒मनु॑ । अनु॒ सन्त॑त्यै । सन्त॑त्यै द॒शत᳚म् । सन्त॑त्या॒ इति॒ सम् - त॒त्यै॒ । द॒शत॒म् न । न वि । विच्छि॑न्द्यात् । छि॒न्द्या॒द् वि॒राज᳚म् । वि॒राज॒म् न । वि॒राज॒मिति॑ वि - राज᳚म् । नेत् । इद् वि॑च्छि॒नदा॑नि । वि॒च्छि॒नदा॒नीति॑ । वि॒च्छि॒नदा॒नीति॑ वि - छि॒नदा॑नि । इत्यथ॑ । अथ॒ या । या स॑हस्रत॒मी । स॒ह॒स्र॒त॒म्यासी᳚त् । स॒ह॒स्र॒त॒मीति॑ सहस्र - त॒मी । आसी॒त् तस्या᳚म् । तस्या॒मिन्द्रः॑ । इन्द्र॑श्च । च॒ विष्णुः॑ । विष्णु॑श्च । च॒ व्याय॑च्छेताम् । व्याय॑च्छेताꣳ॒॒ सः । व्याय॑च्छेता॒मिति॑ वि - आय॑च्छेताम् । स इन्द्रः॑ । इन्द्रो॑ऽमन्यत । अ॒म॒न्य॒ता॒नया᳚ । अ॒नया॒ वै । वा इ॒दम् । इ॒दम् ॅविष्णुः॑ । विष्णुः॑ स॒हस्र᳚म् । स॒हस्र॑म् ॅवर्क्ष्यते । व॒र्क्ष्य॒त॒ इति॑ । इति॒ तस्या᳚म् । तस्या॑मकल्पेताम् । अ॒क॒ल्पे॒ता॒म् द्विभा॑गे । द्विभा॑ग॒ इन्द्रः॑ । द्विभा॑ग॒ इति॒ द्वि - भा॒गे॒ । इन्द्र॒स्तृती॑ये । तृती॑ये॒ विष्णुः॑ । विष्णु॒स्तत् । तद् वै । वा ए॒षा । ए॒षाऽभ्यनू᳚च्यते । अ॒भ्यनू᳚च्यत उ॒भा । अ॒भ्यनू᳚च्यत॒ इत्य॑भि - अनू᳚च्यते । उ॒भा जि॑ग्यथुः । जि॒ग्य॒थु॒रिति॑ । इति॒ ताम् । ताम् ॅवै । वा ए॒ताम् । ए॒ताम॑च्छावा॒कः । अ॒च्छा॒वा॒क ए॒व \newline

\textbf{Jatai Paata} \newline

1. ए॒षाम् ॅलो॒काना᳚म् ॅलो॒काना॑ मे॒षा मे॒षाम् ॅलो॒काना᳚म् । \newline
2. लो॒काना॒ मन्वनु॑ लो॒काना᳚म् ॅलो॒काना॒ मनु॑ । \newline
3. अनु॒ सन्त॑त्यै॒ सन्त॑त्या॒ अन्वनु॒ सन्त॑त्यै । \newline
4. सन्त॑त्यै द॒शत॑म् द॒शतꣳ॒॒ सन्त॑त्यै॒ सन्त॑त्यै द॒शत᳚म् । \newline
5. सन्त॑त्या॒ इति॒ सं - त॒त्यै॒ । \newline
6. द॒शत॒न् न न द॒शत॑म् द॒शत॒न् न । \newline
7. न वि वि न न वि । \newline
8. वि च्छि॑न्द्याद् छिन्द्या॒द् वि वि च्छि॑न्द्यात् । \newline
9. छि॒न्द्या॒द् वि॒राजं॑ ॅवि॒राज॑म् छिन्द्याच् छिन्द्याद् वि॒राज᳚म् । \newline
10. वि॒राज॒न् न न वि॒राजं॑ ॅवि॒राज॒न् न । \newline
11. वि॒राज॒मिति॑ वि - राज᳚म् । \newline
12. नेदिन् न नेत् । \newline
13. इद् वि॑च्छि॒नदा॑नि विच्छि॒नदा॒ नीदिद् वि॑च्छि॒नदा॑नि । \newline
14. वि॒च्छि॒नदा॒ नीतीति॑ विच्छि॒नदा॑नि विच्छि॒नदा॒ नीति॑ । \newline
15. वि॒च्छि॒नदा॒नीति॑ वि - छि॒नदा॑नि । \newline
16. इत्यथाथे तीत्यथ॑ । \newline
17. अथ॒ या या ऽथाथ॒ या । \newline
18. या स॑हस्रत॒मी स॑हस्रत॒मी या या स॑हस्रत॒मी । \newline
19. स॒ह॒स्र॒त॒ म्यासी॒ दासी᳚थ् सहस्रत॒मी स॑हस्रत॒ म्यासी᳚त् । \newline
20. स॒ह॒स्र॒त॒मीति॑ सहस्र - त॒मी । \newline
21. आसी॒त् तस्या॒म् तस्या॒ मासी॒ दासी॒त् तस्या᳚म् । \newline
22. तस्या॒ मिन्द्र॒ इन्द्र॒ स्तस्या॒म् तस्या॒ मिन्द्रः॑ । \newline
23. इन्द्र॑ श्च॒ चेन्द्र॒ इन्द्र॑ श्च । \newline
24. च॒ विष्णु॒र् विष्णु॑ श्च च॒ विष्णुः॑ । \newline
25. विष्णु॑ श्च च॒ विष्णु॒र् विष्णु॑ श्च । \newline
26. च॒ व्याय॑च्छेतां॒ ॅव्याय॑च्छेताम् च च॒ व्याय॑च्छेताम् । \newline
27. व्याय॑च्छेताꣳ॒॒ स स व्याय॑च्छेतां॒ ॅव्याय॑च्छेताꣳ॒॒ सः । \newline
28. व्याय॑च्छेता॒मिति॑ वि - आय॑च्छेताम् । \newline
29. स इन्द्र॒ इन्द्रः॒ स स इन्द्रः॑ । \newline
30. इन्द्रो॑ ऽमन्यता मन्य॒तेन्द्र॒ इन्द्रो॑ ऽमन्यत । \newline
31. अ॒म॒न्य॒ता॒ नया॒ ऽनया॑ ऽमन्यता मन्यता॒ नया᳚ । \newline
32. अ॒नया॒ वै वा अ॒नया॒ ऽनया॒ वै । \newline
33. वा इ॒द मि॒दं ॅवै वा इ॒दम् । \newline
34. इ॒दं ॅविष्णु॒र् विष्णु॑ रि॒द मि॒दं ॅविष्णुः॑ । \newline
35. विष्णुः॑ स॒हस्रꣳ॑ स॒हस्रं॒ ॅविष्णु॒र् विष्णुः॑ स॒हस्र᳚म् । \newline
36. स॒हस्रं॑ ॅवर्क्ष्यते वर्क्ष्यते स॒हस्रꣳ॑ स॒हस्रं॑ ॅवर्क्ष्यते । \newline
37. व॒र्क्ष्य॒त॒ इतीति॑ वर्क्ष्यते वर्क्ष्यत॒ इति॑ । \newline
38. इति॒ तस्या॒म् तस्या॒ मितीति॒ तस्या᳚म् । \newline
39. तस्या॑ मकल्पेता मकल्पेता॒म् तस्या॒म् तस्या॑ मकल्पेताम् । \newline
40. अ॒क॒ल्पे॒ता॒म् द्विभा॑गे॒ द्विभा॑गे ऽकल्पेता मकल्पेता॒म् द्विभा॑गे । \newline
41. द्विभा॑ग॒ इन्द्र॒ इन्द्रो॒ द्विभा॑गे॒ द्विभा॑ग॒ इन्द्रः॑ । \newline
42. द्विभा॑ग॒ इति॒ द्वि - भा॒गे॒ । \newline
43. इन्द्र॒ स्तृती॑ये॒ तृती॑य॒ इन्द्र॒ इन्द्र॒ स्तृती॑ये । \newline
44. तृती॑ये॒ विष्णु॒र् विष्णु॒ स्तृती॑ये॒ तृती॑ये॒ विष्णुः॑ । \newline
45. विष्णु॒ स्तत् तद् विष्णु॒र् विष्णु॒ स्तत् । \newline
46. तद् वै वै तत् तद् वै । \newline
47. वा ए॒षैषा वै वा ए॒षा । \newline
48. ए॒षा ऽभ्यनू᳚च्यते॒ ऽभ्यनू᳚च्यत ए॒षैषा ऽभ्यनू᳚च्यते । \newline
49. अ॒भ्यनू᳚च्यत उ॒भोभा ऽभ्यनू᳚च्यते॒ ऽभ्यनू᳚च्यत उ॒भा । \newline
50. अ॒भ्यनू᳚च्यत॒ इत्य॑भि - अनू᳚च्यते । \newline
51. उ॒भा जि॑ग्यथुर् जिग्यथु रु॒भोभा जि॑ग्यथुः । \newline
52. जि॒ग्य॒थु॒ रितीति॑ जिग्यथुर् जिग्यथु॒ रिति॑ । \newline
53. इति॒ ताम् ता मितीति॒ ताम् । \newline
54. तां ॅवै वै ताम् तां ॅवै । \newline
55. वा ए॒ता मे॒तां ॅवै वा ए॒ताम् । \newline
56. ए॒ता म॑च्छावा॒को᳚ ऽच्छावा॒क ए॒ता मे॒ता म॑च्छावा॒कः । \newline
57. अ॒च्छा॒वा॒क ए॒वै वाच्छा॑वा॒को᳚ ऽच्छावा॒क ए॒व । \newline

\textbf{Ghana Paata } \newline

1. ए॒षाम् ॅलो॒काना᳚म् ॅलो॒काना॑ मे॒षा मे॒षाम् ॅलो॒काना॒ मन्वनु॑ लो॒काना॑ मे॒षा मे॒षाम् ॅलो॒काना॒ मनु॑ । \newline
2. लो॒काना॒ मन्वनु॑ लो॒काना᳚म् ॅलो॒काना॒ मनु॒ सन्त॑त्यै॒ सन्त॑त्या॒ अनु॑ लो॒काना᳚म् ॅलो॒काना॒ मनु॒ सन्त॑त्यै । \newline
3. अनु॒ सन्त॑त्यै॒ सन्त॑त्या॒ अन्वनु॒ सन्त॑त्यै द॒शत॑म् द॒शतꣳ॒॒ सन्त॑त्या॒ अन्वनु॒ सन्त॑त्यै द॒शत᳚म् । \newline
4. सन्त॑त्यै द॒शत॑म् द॒शतꣳ॒॒ सन्त॑त्यै॒ सन्त॑त्यै द॒शत॒न् न न द॒शतꣳ॒॒ सन्त॑त्यै॒ सन्त॑त्यै द॒शत॒न् न । \newline
5. सन्त॑त्या॒ इति॒ सं - त॒त्यै॒ । \newline
6. द॒शत॒न् न न द॒शत॑म् द॒शत॒न् न वि वि न द॒शत॑म् द॒शत॒न् न वि । \newline
7. न वि वि न न वि च्छि॑न्द्याच् छिन्द्या॒द् वि न न वि च्छि॑न्द्यात् । \newline
8. वि च्छि॑न्द्याच् छिन्द्या॒द् वि वि च्छि॑न्द्याद् वि॒राजं॑ ॅवि॒राज॑म् छिन्द्या॒द् वि वि च्छि॑न्द्याद् वि॒राज᳚म् । \newline
9. छि॒न्द्या॒द् वि॒राजं॑ ॅवि॒राज॑म् छिन्द्याच् छिन्द्याद् वि॒राज॒न् न न वि॒राज॑म् छिन्द्याच् छिन्द्याद् वि॒राज॒न् न । \newline
10. वि॒राज॒न् न न वि॒राजं॑ ॅवि॒राज॒न् नेदिन् न वि॒राजं॑ ॅवि॒राज॒न् नेत् । \newline
11. वि॒राज॒मिति॑ वि - राज᳚म् । \newline
12. ने दिन् न नेद् वि॑च्छि॒नदा॑नि विच्छि॒नदा॒नीन् न नेद् वि॑च्छि॒नदा॑नि । \newline
13. इद् वि॑च्छि॒नदा॑नि विच्छि॒नदा॒ नीदिद् वि॑च्छि॒नदा॒नीतीति॑ विच्छि॒नदा॒ नीदिद् वि॑च्छि॒नदा॒ नीति॑ । \newline
14. वि॒च्छि॒नदा॒नीतीति॑ विच्छि॒नदा॑नि विच्छि॒नदा॒नी त्यथा थेति॑ विच्छि॒नदा॑नि विच्छि॒नदा॒नी त्यथ॑ । \newline
15. वि॒च्छि॒नदा॒नीति॑ वि - छि॒नदा॑नि । \newline
16. इत्यथा थेतीत्यथ॒ या या ऽथे तीत्यथ॒ या । \newline
17. अथ॒ या या ऽथाथ॒ या स॑हस्रत॒मी स॑हस्रत॒मी या ऽथाथ॒ या स॑हस्रत॒मी । \newline
18. या स॑हस्रत॒मी स॑हस्रत॒मी या या स॑हस्रत॒ म्यासी॒ दासी᳚थ् सहस्रत॒मी या या स॑हस्रत॒ म्यासी᳚त् । \newline
19. स॒ह॒स्र॒त॒ म्यासी॒ दासी᳚थ् सहस्रत॒मी स॑हस्रत॒ म्यासी॒त् तस्या॒म् तस्या॒ मासी᳚थ् सहस्रत॒मी स॑हस्रत॒ म्यासी॒त् तस्या᳚म् । \newline
20. स॒ह॒स्र॒त॒मीति॑ सहस्र - त॒मी । \newline
21. आसी॒त् तस्या॒म् तस्या॒ मासी॒ दासी॒त् तस्या॒ मिन्द्र॒ इन्द्र॒ स्तस्या॒ मासी॒ दासी॒त् तस्या॒ मिन्द्रः॑ । \newline
22. तस्या॒ मिन्द्र॒ इन्द्र॒ स्तस्या॒म् तस्या॒ मिन्द्र॑श्च॒ चेन्द्र॒ स्तस्या॒म् तस्या॒ मिन्द्र॑श्च । \newline
23. इन्द्र॑श्च॒ चेन्द्र॒ इन्द्र॑श्च॒ विष्णु॒र् विष्णु॒ श्चेन्द्र॒ इन्द्र॑श्च॒ विष्णुः॑ । \newline
24. च॒ विष्णु॒र् विष्णु॑श्च च॒ विष्णु॑श्च च॒ विष्णु॑श्च च॒ विष्णु॑श्च । \newline
25. विष्णु॑श्च च॒ विष्णु॒र् विष्णु॑श्च॒ व्याय॑च्छेतां॒ ॅव्याय॑च्छेताम् च॒ विष्णु॒र् विष्णु॑श्च॒ व्याय॑च्छेताम् । \newline
26. च॒ व्याय॑च्छेतां॒ ॅव्याय॑च्छेताम् च च॒ व्याय॑च्छेताꣳ॒॒ स स व्याय॑च्छेताम् च च॒ व्याय॑च्छेताꣳ॒॒ सः । \newline
27. व्याय॑च्छेताꣳ॒॒ स स व्याय॑च्छेतां॒ ॅव्याय॑च्छेताꣳ॒॒ स इन्द्र॒ इन्द्रः॒ स व्याय॑च्छेतां॒ ॅव्याय॑च्छेताꣳ॒॒ स इन्द्रः॑ । \newline
28. व्याय॑च्छेता॒मिति॑ वि - आय॑च्छेताम् । \newline
29. स इन्द्र॒ इन्द्रः॒ स स इन्द्रो॑ ऽमन्यता मन्य॒तेन्द्रः॒ स स इन्द्रो॑ ऽमन्यत । \newline
30. इन्द्रो॑ ऽमन्यता मन्य॒तेन्द्र॒ इन्द्रो॑ ऽमन्यता॒ नया॒ ऽनया॑ ऽमन्य॒तेन्द्र॒ इन्द्रो॑ ऽमन्यता॒ नया᳚ । \newline
31. अ॒म॒न्य॒ता॒ नया॒ ऽनया॑ ऽमन्यता मन्यता॒ नया॒ वै वा अ॒नया॑ ऽमन्यता मन्यता॒ नया॒ वै । \newline
32. अ॒नया॒ वै वा अ॒नया॒ ऽनया॒ वा इ॒द मि॒दं ॅवा अ॒नया॒ ऽनया॒ वा इ॒दम् । \newline
33. वा इ॒द मि॒दं ॅवै वा इ॒दं ॅविष्णु॒र् विष्णु॑ रि॒दं ॅवै वा इ॒दं ॅविष्णुः॑ । \newline
34. इ॒दं ॅविष्णु॒र् विष्णु॑ रि॒द मि॒दं ॅविष्णुः॑ स॒हस्रꣳ॑ स॒हस्रं॒ ॅविष्णु॑ रि॒द मि॒दं ॅविष्णुः॑ स॒हस्र᳚म् । \newline
35. विष्णुः॑ स॒हस्रꣳ॑ स॒हस्रं॒ ॅविष्णु॒र् विष्णुः॑ स॒हस्रं॑ ॅवर्क्ष्यते वर्क्ष्यते स॒हस्रं॒ ॅविष्णु॒र् विष्णुः॑ स॒हस्रं॑ ॅवर्क्ष्यते । \newline
36. स॒हस्रं॑ ॅवर्क्ष्यते वर्क्ष्यते स॒हस्रꣳ॑ स॒हस्रं॑ ॅवर्क्ष्यत॒ इतीति॑ वर्क्ष्यते स॒हस्रꣳ॑ स॒हस्रं॑ ॅवर्क्ष्यत॒ इति॑ । \newline
37. व॒र्क्ष्य॒त॒ इतीति॑ वर्क्ष्यते वर्क्ष्यत॒ इति॒ तस्या॒म् तस्या॒ मिति॑ वर्क्ष्यते वर्क्ष्यत॒ इति॒ तस्या᳚म् । \newline
38. इति॒ तस्या॒म् तस्या॒ मितीति॒ तस्या॑ मकल्पेता मकल्पेता॒म् तस्या॒ मितीति॒ तस्या॑ मकल्पेताम् । \newline
39. तस्या॑ मकल्पेता मकल्पेता॒म् तस्या॒म् तस्या॑ मकल्पेता॒म् द्विभा॑गे॒ द्विभा॑गे ऽकल्पेता॒म् तस्या॒म् तस्या॑ मकल्पेता॒म् द्विभा॑गे । \newline
40. अ॒क॒ल्पे॒ता॒म् द्विभा॑गे॒ द्विभा॑गे ऽकल्पेता मकल्पेता॒म् द्विभा॑ग॒ इन्द्र॒ इन्द्रो॒ द्विभा॑गे ऽकल्पेता मकल्पेता॒म् द्विभा॑ग॒ इन्द्रः॑ । \newline
41. द्विभा॑ग॒ इन्द्र॒ इन्द्रो॒ द्विभा॑गे॒ द्विभा॑ग॒ इन्द्र॒ स्तृती॑ये॒ तृती॑य॒ इन्द्रो॒ द्विभा॑गे॒ द्विभा॑ग॒ इन्द्र॒ स्तृती॑ये । \newline
42. द्विभा॑ग॒ इति॒ द्वि - भा॒गे॒ । \newline
43. इन्द्र॒ स्तृती॑ये॒ तृती॑य॒ इन्द्र॒ इन्द्र॒ स्तृती॑ये॒ विष्णु॒र् विष्णु॒ स्तृती॑य॒ इन्द्र॒ इन्द्र॒ स्तृती॑ये॒ विष्णुः॑ । \newline
44. तृती॑ये॒ विष्णु॒र् विष्णु॒ स्तृती॑ये॒ तृती॑ये॒ विष्णु॒ स्तत् तद् विष्णु॒ स्तृती॑ये॒ तृती॑ये॒ विष्णु॒ स्तत् । \newline
45. विष्णु॒ स्तत् तद् विष्णु॒र् विष्णु॒ स्तद् वै वै तद् विष्णु॒र् विष्णु॒ स्तद् वै । \newline
46. तद् वै वै तत् तद् वा ए॒षैषा वै तत् तद् वा ए॒षा । \newline
47. वा ए॒षैषा वै वा ए॒षा ऽभ्यनू᳚च्यते॒ ऽभ्यनू᳚च्यत ए॒षा वै वा ए॒षा ऽभ्यनू᳚च्यते । \newline
48. ए॒षा ऽभ्यनू᳚च्यते॒ ऽभ्यनू᳚च्यत ए॒षैषा ऽभ्यनू᳚च्यत उ॒भोभा ऽभ्यनू᳚च्यत ए॒षैषा ऽभ्यनू᳚च्यत उ॒भा । \newline
49. अ॒भ्यनू᳚च्यत उ॒भोभा ऽभ्यनू᳚च्यते॒ ऽभ्यनू᳚च्यत उ॒भा जि॑ग्यथुर् जिग्यथु रु॒भा ऽभ्यनू᳚च्यते॒ ऽभ्यनू᳚च्यत उ॒भा जि॑ग्यथुः । \newline
50. अ॒भ्यनू᳚च्यत॒ इत्य॑भि - अनू᳚च्यते । \newline
51. उ॒भा जि॑ग्यथुर् जिग्यथु रु॒भोभा जि॑ग्यथु॒ रितीति॑ जिग्यथु रु॒भोभा जि॑ग्यथु॒ रिति॑ । \newline
52. जि॒ग्य॒थु॒ रितीति॑ जिग्यथुर् जिग्यथु॒ रिति॒ ताम् ता मिति॑ जिग्यथुर् जिग्यथु॒ रिति॒ ताम् । \newline
53. इति॒ ताम् ता मितीति॒ तां ॅवै वै ता मितीति॒ तां ॅवै । \newline
54. तां ॅवै वै ताम् तां ॅवा ए॒ता मे॒तां ॅवै ताम् तां ॅवा ए॒ताम् । \newline
55. वा ए॒ता मे॒तां ॅवै वा ए॒ता म॑च्छावा॒को᳚ ऽच्छावा॒क ए॒तां ॅवै वा ए॒ता म॑च्छावा॒कः । \newline
56. ए॒ता म॑च्छावा॒को᳚ ऽच्छावा॒क ए॒ता मे॒ता म॑च्छावा॒क ए॒वै वाच्छा॑वा॒क ए॒ता मे॒ता म॑च्छावा॒क ए॒व । \newline
57. अ॒च्छा॒वा॒क ए॒वै वाच्छा॑वा॒को᳚ ऽच्छावा॒क ए॒व शꣳ॑सति शꣳस त्ये॒वाच्छा॑वा॒को᳚ ऽच्छावा॒क ए॒व शꣳ॑सति । \newline
\pagebreak
\markright{ TS 7.1.5.6  \hfill https://www.vedavms.in \hfill}

\section{ TS 7.1.5.6 }

\textbf{TS 7.1.5.6 } \newline
\textbf{Samhita Paata} \newline

ए॒व शꣳ॑स॒त्यथ॒ या स॑हस्रत॒मी सा होत्रे॒ देयेति॒ होता॑रं॒ ॅवा अ॒भ्यति॑रिच्यते॒ यद॑ति॒रिच्य॑ते॒ होता ऽना᳚प्तस्याऽऽ*पयि॒ता ऽथा॑ऽ*हुरुन्ने॒त्रे देयेत्यति॑रिक्ता॒ वा ए॒षा स॒हस्र॒स्याति॑रिक्त उन्ने॒तर्त्विजा॒मथा॑ऽऽ*हुः॒ सर्वे᳚भ्यः सद॒स्ये᳚भ्यो॒ देयेत्यथा॑ऽऽहुरुदा॒ कृत्या॒ सा वशं॑ चरे॒दित्यथा॑ऽऽ*हुर्ब्र॒ह्मणे॑ चा॒ग्नीधे॑ च॒ देयेति॒ - [  ] \newline

\textbf{Pada Paata} \newline

ए॒व । शꣳ॒॒स॒ति॒ । अथ॑ । या । स॒ह॒स्र॒त॒मीति॑ सहस्र - त॒मी । सा । होत्रे᳚ । देया᳚ । इति॑ । होता॑रम् । वै । अ॒भ्यति॑रिच्यत॒ इत्य॑भि-अति॑रिच्यते । यत् । अ॒ति॒रिच्य॑त॒ इत्य॑ति - रिच्य॑ते । होता᳚ । अना᳚प्तस्य । आ॒प॒यि॒ता । अथ॑ । आ॒हुः॒ । उ॒न्ने॒त्र इत्यु॑त् - ने॒त्रे । देया᳚ । इति॑ । अति॑रि॒क्तेत्यति॑ - रि॒क्ता॒ । वै । ए॒षा । स॒हस्र॑स्य । अति॑रिक्त॒ इत्यति॑ - रि॒क्तः॒ । उ॒न्ने॒तेत्यु॑त् - ने॒ता । ऋ॒त्विजा᳚म् । अथ॑ । आ॒हुः॒ । सर्वे᳚भ्यः । स॒द॒स्ये᳚भ्यः । देया᳚ । इति॑ । अथ॑ । आ॒हुः॒ । उ॒दा॒कृत्येत्यु॑त् - आ॒कृत्या᳚ । सा । वश᳚म् । च॒रे॒त् । इति॑ । अथ॑ । आ॒हुः॒ । ब्र॒ह्मणे᳚ । च॒ । अ॒ग्नीध॒ इत्य॑ग्नि - इधे᳚ । च॒ । देया᳚ । इति॑ ।  \newline


\textbf{Krama Paata} \newline

ए॒व शꣳ॑सति । शꣳ॒॒स॒त्यथ॑ । अथ॒ या । या स॑हस्रत॒मी । स॒ह॒स्र॒त॒मी सा । स॒ह॒स्र॒त॒मीति॑ सहस्र - त॒मी । सा होत्रे᳚ । होत्रे॒ देया᳚ । देयेति॑ । इति॒ होता॑रम् । होता॑र॒म् ॅवै । वा अ॒भ्यति॑रिच्यते । अ॒भ्यति॑रिच्यते॒ यत् । अ॒भ्यति॑रिच्यत॒ इत्य॑भि - अति॑रिच्यते । यद॑ति॒रिच्य॑ते । अ॒ति॒रिच्य॑ते॒ होता᳚ । अ॒ति॒रिच्य॑त॒ इत्य॑ति - रिच्य॑ते । होताऽना᳚प्तस्य । अना᳚प्तस्यापयि॒ता । आ॒प॒यि॒ताऽथ॑ । अथा॑हुः । आ॒हु॒रु॒न्ने॒त्रे । उ॒न्ने॒त्रे देया᳚ । उ॒न्ने॒त्र इत्यु॑त् - ने॒त्रे । देयेति॑ । इत्यति॑रिक्ता । अति॑रिक्ता॒ वै । अति॑रि॒क्तेत्यति॑ - रि॒क्ता॒ । वा ए॒षा । ए॒षा स॒हस्र॑स्य । स॒हस्र॒स्याति॑रिक्तः । अति॑रिक्त उन्ने॒ता । अति॑रिक्त॒ इत्यति॑ - रि॒क्तः॒ । उ॒न्ने॒तर्त्विजा᳚म् । उ॒न्ने॒तेत्यु॑त् - ने॒ता । ऋ॒त्विजा॒मथ॑ । अथा॑हुः । आ॒हुः॒ सर्वे᳚भ्यः । सर्वे᳚भ्यः सद॒स्ये᳚भ्यः । स॒द॒स्ये᳚भ्यो॒ देया᳚ । देयेति॑ । इत्यथ॑ । अथा॑हुः । आ॒हु॒रु॒दा॒कृत्या᳚ । उ॒दा॒कृत्या॒ सा । उ॒दा॒कृत्येत्यु॑त् - आ॒कृत्या᳚ । सा वश᳚म् । वश॑म् चरेत् । च॒रे॒दिति॑ । इत्यथ॑ । अथा॑हुः । आ॒हु॒र् ब्र॒ह्मणे᳚ । ब्र॒ह्मणे॑ च । चा॒ग्नीधे᳚ । अ॒ग्नीधे॑ च । अ॒ग्नीध॒ इत्य॑ग्नि - इधे᳚ । च॒ देया᳚ । देयेति॑ ( ) । 
इति॒ द्विभा॑गम् \newline

\textbf{Jatai Paata} \newline

1. ए॒व शꣳ॑सति शꣳस त्ये॒वैव शꣳ॑सति । \newline
2. शꣳ॒॒स॒ त्यथाथ॑ शꣳसति शꣳस॒ त्यथ॑ । \newline
3. अथ॒ या या ऽथाथ॒ या । \newline
4. या स॑हस्रत॒मी स॑हस्रत॒मी या या स॑हस्रत॒मी । \newline
5. स॒ह॒स्र॒त॒मी सा सा स॑हस्रत॒मी स॑हस्रत॒मी सा । \newline
6. स॒ह॒स्र॒त॒मीति॑ सहस्र - त॒मी । \newline
7. सा होत्रे॒ होत्रे॒ सा सा होत्रे᳚ । \newline
8. होत्रे॒ देया॒ देया॒ होत्रे॒ होत्रे॒ देया᳚ । \newline
9. देयेतीति॒ देया॒ देयेति॑ । \newline
10. इति॒ होता॑रꣳ॒॒ होता॑र॒ मितीति॒ होता॑रम् । \newline
11. होता॑रं॒ ॅवै वै होता॑रꣳ॒॒ होता॑रं॒ ॅवै । \newline
12. वा अ॒भ्यति॑रिच्यते॒ ऽभ्यति॑रिच्यते॒ वै वा अ॒भ्यति॑रिच्यते । \newline
13. अ॒भ्यति॑रिच्यते॒ यद् यद॒भ्यति॑रिच्यते॒ ऽभ्यति॑रिच्यते॒ यत् । \newline
14. अ॒भ्यति॑रिच्यत॒ इत्य॑भि - अति॑रिच्यते । \newline
15. यद॑ति॒रिच्य॑ते ऽति॒रिच्य॑ते॒ यद् यद॑ति॒रिच्य॑ते । \newline
16. अ॒ति॒रिच्य॑ते॒ होता॒ होता॑ ऽति॒रिच्य॑ते ऽति॒रिच्य॑ते॒ होता᳚ । \newline
17. अ॒ति॒रिच्य॑त॒ इत्य॑ति - रिच्य॑ते । \newline
18. होता ऽना᳚प्त॒स्या ना᳚प्तस्य॒ होता॒ होता ऽना᳚प्तस्य । \newline
19. अना᳚प्तस्या पयि॒ता ऽऽप॑यि॒ता ऽना᳚प्त॒स्या ना᳚प्तस्या पयि॒ता । \newline
20. आ॒प॒यि॒ता ऽथाथा॑ पयि॒ता ऽऽप॑यि॒ता ऽथ॑ । \newline
21. अथा॑हु राहु॒ रथा था॑हुः । \newline
22. आ॒हु॒ रु॒न्ने॒त्र उ॑न्ने॒त्र आ॑हु राहु रुन्ने॒त्रे । \newline
23. उ॒न्ने॒त्रे देया॒ देयो᳚न्ने॒त्र उ॑न्ने॒त्रे देया᳚ । \newline
24. उ॒न्ने॒त्र इत्यु॑त् - ने॒त्रे । \newline
25. देयेतीति॒ देया॒ देयेति॑ । \newline
26. इत्यति॑रि॒क्ता ऽति॑रि॒क्तेती त्यति॑रिक्ता । \newline
27. अति॑रिक्ता॒ वै वा अति॑रि॒क्ता ऽति॑रिक्ता॒ वै । \newline
28. अति॑रि॒क्तेत्यति॑ - रि॒क्ता॒ । \newline
29. वा ए॒षैषा वै वा ए॒षा । \newline
30. ए॒षा स॒हस्र॑स्य स॒हस्र॑ स्यै॒षैषा स॒हस्र॑स्य । \newline
31. स॒हस्र॒ स्याति॑रि॒क्तो ऽति॑रिक्तः स॒हस्र॑स्य स॒हस्र॒ स्याति॑रिक्तः । \newline
32. अति॑रिक्त उन्ने॒ तोन्ने॒ता ऽति॑रि॒क्तो ऽति॑रिक्त उन्ने॒ता । \newline
33. अति॑रिक्त॒ इत्यति॑ - रि॒क्तः॒ । \newline
34. उ॒न्ने॒त र्‌त्विजा॑ मृ॒त्विजा॑ मुन्ने॒ तोन्ने॒त र्‌त्विजा᳚म् । \newline
35. उ॒न्ने॒तेत्यु॑त् - ने॒ता । \newline
36. ऋ॒त्विजा॒ मथाथ॒ र्‌त्विजा॑ मृ॒त्विजा॒ मथ॑ । \newline
37. अथा॑हु राहु॒ रथा था॑हुः । \newline
38. आ॒हुः॒ सर्वे᳚भ्यः॒ सर्वे᳚भ्य आहु राहुः॒ सर्वे᳚भ्यः । \newline
39. सर्वे᳚भ्यः सद॒स्ये᳚भ्यः सद॒स्ये᳚भ्यः॒ सर्वे᳚भ्यः॒ सर्वे᳚भ्यः सद॒स्ये᳚भ्यः । \newline
40. स॒द॒स्ये᳚भ्यो॒ देया॒ देया॑ सद॒स्ये᳚भ्यः सद॒स्ये᳚भ्यो॒ देया᳚ । \newline
41. देयेतीति॒ देया॒ देयेति॑ । \newline
42. इत्यथाथे तीत्यथ॑ । \newline
43. अथा॑हु राहु॒ रथा था॑हुः । \newline
44. आ॒हु॒ रु॒दा॒कृ त्यो॑दा॒कृत्या॑ ऽऽहु राहु रुदा॒कृत्या᳚ । \newline
45. उ॒दा॒कृत्या॒ सा सोदा॒कृ त्यो॑दा॒कृत्या॒ सा । \newline
46. उ॒दा॒कृत्येत्यु॑त् - आ॒कृत्या᳚ । \newline
47. सा वशं॒ ॅवशꣳ॒॒ सा सा वश᳚म् । \newline
48. वश॑म् चरेच् चरे॒द् वशं॒ ॅवश॑म् चरेत् । \newline
49. च॒रे॒ दितीति॑ चरेच् चरे॒ दिति॑ । \newline
50. इत्यथाथे तीत्यथ॑ । \newline
51. अथा॑हु राहु॒ रथा था॑हुः । \newline
52. आ॒हु॒र् ब्र॒ह्मणे᳚ ब्र॒ह्मण॑ आहु राहुर् ब्र॒ह्मणे᳚ । \newline
53. ब्र॒ह्मणे॑ च च ब्र॒ह्मणे᳚ ब्र॒ह्मणे॑ च । \newline
54. चा॒ग्नीधे॒ ऽग्नीधे॑ च चा॒ग्नीधे᳚ । \newline
55. अ॒ग्नीधे॑ च चा॒ग्नीधे॒ ऽग्नीधे॑ च । \newline
56. अ॒ग्नीध॒ इत्य॑ग्नि - इधे᳚ । \newline
57. च॒ देया॒ देया॑ च च॒ देया᳚ । \newline
58. देयेतीति॒ देया॒ देयेति॑ । \newline
59. इति॒ द्विभा॑ग॒म् द्विभा॑ग॒ मितीति॒ द्विभा॑गम् । \newline

\textbf{Ghana Paata } \newline

1. ए॒व शꣳ॑सति शꣳस त्ये॒वैव शꣳ॑स॒ त्यथाथ॑ शꣳस त्ये॒वैव शꣳ॑स॒ त्यथ॑ । \newline
2. शꣳ॒॒स॒ त्यथाथ॑ शꣳसति शꣳस॒ त्यथ॒ या या ऽथ॑ शꣳसति शꣳस॒ त्यथ॒ या । \newline
3. अथ॒ या या ऽथाथ॒ या स॑हस्रत॒मी स॑हस्रत॒मी या ऽथाथ॒ या स॑हस्रत॒मी । \newline
4. या स॑हस्रत॒मी स॑हस्रत॒मी या या स॑हस्रत॒मी सा सा स॑हस्रत॒मी या या स॑हस्रत॒मी सा । \newline
5. स॒ह॒स्र॒त॒मी सा सा स॑हस्रत॒मी स॑हस्रत॒मी सा होत्रे॒ होत्रे॒ सा स॑हस्रत॒मी स॑हस्रत॒मी सा होत्रे᳚ । \newline
6. स॒ह॒स्र॒त॒मीति॑ सहस्र - त॒मी । \newline
7. सा होत्रे॒ होत्रे॒ सा सा होत्रे॒ देया॒ देया॒ होत्रे॒ सा सा होत्रे॒ देया᳚ । \newline
8. होत्रे॒ देया॒ देया॒ होत्रे॒ होत्रे॒ देयेतीति॒ देया॒ होत्रे॒ होत्रे॒ देयेति॑ । \newline
9. देयेतीति॒ देया॒ देयेति॒ होता॑रꣳ॒॒ होता॑र॒ मिति॒ देया॒ देयेति॒ होता॑रम् । \newline
10. इति॒ होता॑रꣳ॒॒ होता॑र॒ मितीति॒ होता॑रं॒ ॅवै वै होता॑र॒ मितीति॒ होता॑रं॒ ॅवै । \newline
11. होता॑रं॒ ॅवै वै होता॑रꣳ॒॒ होता॑रं॒ ॅवा अ॒भ्यति॑रिच्यते॒ ऽभ्यति॑रिच्यते॒ वै होता॑रꣳ॒॒ होता॑रं॒ ॅवा अ॒भ्यति॑रिच्यते । \newline
12. वा अ॒भ्यति॑रिच्यते॒ ऽभ्यति॑रिच्यते॒ वै वा अ॒भ्यति॑रिच्यते॒ यद् यद॒भ्यति॑रिच्यते॒ वै वा अ॒भ्यति॑रिच्यते॒ यत् । \newline
13. अ॒भ्यति॑रिच्यते॒ यद् यद॒भ्यति॑रिच्यते॒ ऽभ्यति॑रिच्यते॒ यद॑ति॒रिच्य॑ते ऽति॒रिच्य॑ते॒ यद॒भ्यति॑रिच्यते॒ ऽभ्यति॑रिच्यते॒ यद॑ति॒रिच्य॑ते । \newline
14. अ॒भ्यति॑रिच्यत॒ इत्य॑भि - अति॑रिच्यते । \newline
15. यद॑ति॒रिच्य॑ते ऽति॒रिच्य॑ते॒ यद् यद॑ति॒रिच्य॑ते॒ होता॒ होता॑ ऽति॒रिच्य॑ते॒ यद् यद॑ति॒रिच्य॑ते॒ होता᳚ । \newline
16. अ॒ति॒रिच्य॑ते॒ होता॒ होता॑ ऽति॒रिच्य॑ते ऽति॒रिच्य॑ते॒ होता ऽना᳚प्त॒स्या ना᳚प्तस्य॒ होता॑ ऽति॒रिच्य॑ते ऽति॒रिच्य॑ते॒ होता ऽना᳚प्तस्य । \newline
17. अ॒ति॒रिच्य॑त॒ इत्य॑ति - रिच्य॑ते । \newline
18. होता ऽना᳚प्त॒स्या ना᳚प्तस्य॒ होता॒ होता ऽना᳚प्तस्या पयि॒ता ऽऽप॑यि॒ता ऽना᳚प्तस्य॒ होता॒ होता ऽना᳚प्तस्या पयि॒ता । \newline
19. अना᳚प्तस्या पयि॒ता ऽऽप॑यि॒ता ऽना᳚प्त॒स्या ना᳚प्तस्या पयि॒ता ऽथाथा॑ पयि॒ता ऽना᳚प्त॒स्या ना᳚प्तस्या पयि॒ता ऽथ॑ । \newline
20. आ॒प॒यि॒ता ऽथाथा॑ पयि॒ता ऽऽप॑यि॒ता ऽथा॑हु राहु॒ रथा॑ पयि॒ता ऽऽप॑यि॒ता ऽथा॑हुः । \newline
21. अथा॑हु राहु॒ रथा था॑हु रुन्ने॒त्र उ॑न्ने॒त्र आ॑हु॒र् अथाथा॑हु रुन्ने॒त्रे । \newline
22. आ॒हु॒ रु॒न्ने॒त्र उ॑न्ने॒त्र आ॑हु राहु रुन्ने॒त्रे देया॒ देयो᳚न्ने॒त्र आ॑हु राहु रुन्ने॒त्रे देया᳚ । \newline
23. उ॒न्ने॒त्रे देया॒ देयो᳚न्ने॒त्र उ॑न्ने॒त्रे देयेतीति॒ देयो᳚न्ने॒त्र उ॑न्ने॒त्रे देयेति॑ । \newline
24. उ॒न्ने॒त्र इत्यु॑त् - ने॒त्रे । \newline
25. देयेतीति॒ देया॒ देये त्यति॑रि॒क्ता ऽति॑रि॒क्तेति॒ देया॒ देये त्यति॑रिक्ता । \newline
26. इत्यति॑रि॒क्ता ऽति॑रि॒क्तेती त्यति॑रिक्ता॒ वै वा अति॑रि॒क्तेती त्यति॑रिक्ता॒ वै । \newline
27. अति॑रिक्ता॒ वै वा अति॑रि॒क्ता ऽति॑रिक्ता॒ वा ए॒षैषा वा अति॑रि॒क्ता ऽति॑रिक्ता॒ वा ए॒षा । \newline
28. अति॑रि॒क्तेत्यति॑ - रि॒क्ता॒ । \newline
29. वा ए॒षैषा वै वा ए॒षा स॒हस्र॑स्य स॒हस्र॑ स्यै॒षा वै वा ए॒षा स॒हस्र॑स्य । \newline
30. ए॒षा स॒हस्र॑स्य स॒हस्र॑ स्यै॒षैषा स॒हस्र॒ स्याति॑रि॒क्तो ऽति॑रिक्तः स॒हस्र॑ स्यै॒षैषा स॒हस्र॒ स्याति॑रिक्तः । \newline
31. स॒हस्र॒ स्याति॑रि॒क्तो ऽति॑रिक्तः स॒हस्र॑स्य स॒हस्र॒ स्याति॑रिक्त उन्ने॒ तोन्ने॒ता ऽति॑रिक्तः स॒हस्र॑स्य स॒हस्र॒ स्याति॑रिक्त उन्ने॒ता । \newline
32. अति॑रिक्त उन्ने॒ तोन्ने॒ता ऽति॑रि॒क्तो ऽति॑रिक्त उन्ने॒त र्‌त्विजा॑ मृ॒त्विजा॑ मुन्ने॒ता ऽति॑रि॒क्तो ऽति॑रिक्त उन्ने॒त र्‌त्विजा᳚म् । \newline
33. अति॑रिक्त॒ इत्यति॑ - रि॒क्तः॒ । \newline
34. उ॒न्ने॒त र्‌त्विजा॑ मृ॒त्विजा॑ मुन्ने॒ तोन्ने॒त र्‌त्विजा॒ मथाथ॒ र्त्विजा॑ मुन्ने॒ तोन्ने॒त र्‌त्विजा॒ मथ॑ । \newline
35. उ॒न्ने॒तेत्यु॑त् - ने॒ता । \newline
36. ऋ॒त्विजा॒ मथाथ॒ र्‌त्विजा॑ मृ॒त्विजा॒ मथा॑हु राहु॒ रथ॒ र्‌त्विजा॑ मृ॒त्विजा॒ मथा॑हुः । \newline
37. अथा॑हु राहु॒ रथा था॑हुः॒ सर्वे᳚भ्यः॒ सर्वे᳚भ्य आहु॒ रथा था॑हुः॒ सर्वे᳚भ्यः । \newline
38. आ॒हुः॒ सर्वे᳚भ्यः॒ सर्वे᳚भ्य आहु राहुः॒ सर्वे᳚भ्यः सद॒स्ये᳚भ्यः सद॒स्ये᳚भ्यः॒ सर्वे᳚भ्य आहु राहुः॒ सर्वे᳚भ्यः सद॒स्ये᳚भ्यः । \newline
39. सर्वे᳚भ्यः सद॒स्ये᳚भ्यः सद॒स्ये᳚भ्यः॒ सर्वे᳚भ्यः॒ सर्वे᳚भ्यः सद॒स्ये᳚भ्यो॒ देया॒ देया॑ सद॒स्ये᳚भ्यः॒ सर्वे᳚भ्यः॒ सर्वे᳚भ्यः सद॒स्ये᳚भ्यो॒ देया᳚ । \newline
40. स॒द॒स्ये᳚भ्यो॒ देया॒ देया॑ सद॒स्ये᳚भ्यः सद॒स्ये᳚भ्यो॒ देयेतीति॒ देया॑ सद॒स्ये᳚भ्यः सद॒स्ये᳚भ्यो॒ देयेति॑ । \newline
41. देये तीति॒ देया॒ देये त्यथा थेति॒ देया॒ देये त्यथ॑ । \newline
42. इत्य थाथे तीत्यथा॑हु राहु॒ रथेती त्यथा॑हुः । \newline
43. अथा॑हु राहु॒ रथा था॑हु रुदा॒कृत्यो॑ दा॒कृत्या॑ ऽऽहु॒ रथा था॑हु रुदा॒कृत्या᳚ । \newline
44. आ॒हु॒ रु॒दा॒कृत्यो॑ दा॒कृत्या॑ ऽऽहु राहु रुदा॒कृत्या॒ सा सोदा॒ कृत्या॑ ऽऽहु राहु रुदा॒कृत्या॒ सा । \newline
45. उ॒दा॒कृत्या॒ सा सोदा॒कृत्यो॑ दा॒कृत्या॒ सा वशं॒ ॅवशꣳ॒॒ सोदा॒कृत्यो॑ दा॒कृत्या॒ सा वश᳚म् । \newline
46. उ॒दा॒कृत्येत्यु॑त् - आ॒कृत्या᳚ । \newline
47. सा वशं॒ ॅवशꣳ॒॒ सा सा वश॑म् चरेच् चरे॒द् वशꣳ॒॒ सा सा वश॑म् चरेत् । \newline
48. वश॑म् चरेच् चरे॒द् वशं॒ ॅवश॑म् चरे॒ दितीति॑ चरे॒द् वशं॒ ॅवश॑म् चरे॒ दिति॑ । \newline
49. च॒रे॒ दितीति॑ चरेच् चरे॒ दित्यथा थेति॑ चरेच् चरे॒ दित्यथ॑ । \newline
50. इत्य था थेती त्यथा॑हु राहु॒ रथेती त्यथा॑हुः । \newline
51. अथा॑हु राहु॒ रथा था॑हुर् ब्र॒ह्मणे᳚ ब्र॒ह्मण॑ आहु॒ रथा था॑हुर् ब्र॒ह्मणे᳚ । \newline
52. आ॒हु॒र् ब्र॒ह्मणे᳚ ब्र॒ह्मण॑ आहु राहुर् ब्र॒ह्मणे॑ च च ब्र॒ह्मण॑ आहु राहुर् ब्र॒ह्मणे॑ च । \newline
53. ब्र॒ह्मणे॑ च च ब्र॒ह्मणे᳚ ब्र॒ह्मणे॑ चा॒ग्नीधे॒ ऽग्नीधे॑ च ब्र॒ह्मणे᳚ ब्र॒ह्मणे॑ चा॒ग्नीधे᳚ । \newline
54. चा॒ग्नीधे॒ ऽग्नीधे॑ च चा॒ग्नीधे॑ च चा॒ग्नीधे॑ च चा॒ग्नीधे॑ च । \newline
55. अ॒ग्नीधे॑ च चा॒ग्नीधे॒ ऽग्नीधे॑ च॒ देया॒ देया॑ चा॒ग्नीधे॒ ऽग्नीधे॑ च॒ देया᳚ । \newline
56. अ॒ग्नीध॒ इत्य॑ग्नि - इधे᳚ । \newline
57. च॒ देया॒ देया॑ च च॒ देयेतीति॒ देया॑ च च॒ देयेति॑ । \newline
58. देयेतीति॒ देया॒ देयेति॒ द्विभा॑ग॒म् द्विभा॑ग॒ मिति॒ देया॒ देयेति॒ द्विभा॑गम् । \newline
59. इति॒ द्विभा॑ग॒म् द्विभा॑ग॒ मितीति॒ द्विभा॑गम् ब्र॒ह्मणे᳚ ब्र॒ह्मणे॒ द्विभा॑ग॒ मितीति॒ द्विभा॑गम् ब्र॒ह्मणे᳚ । \newline
\pagebreak
\markright{ TS 7.1.5.7  \hfill https://www.vedavms.in \hfill}

\section{ TS 7.1.5.7 }

\textbf{TS 7.1.5.7 } \newline
\textbf{Samhita Paata} \newline

द्विभा॑गं ब्र॒ह्मणे॒ तृती॑यम॒ग्नीध॑ ऐ॒न्द्रो वै ब्र॒ह्मा वै᳚ष्ण॒वो᳚ऽग्नीद्यथै॒व तावक॑ल्पेता॒मित्यथा॑ ऽऽ*हु॒र्या क॑ल्या॒णी ब॑हुरू॒पा सा देयेत्यथा॑ ऽऽ*हु॒र्या द्वि॑रू॒पोभ॒यत॑एनी॒ सा देयेति॑ स॒हस्र॑स्य॒ परि॑गृहीत्यै॒ तद्वा ए॒तथ् स॒हस्र॒स्याऽय॑नꣳ स॒हस्रꣳ॑ स्तो॒त्रीयाः᳚ स॒हस्रं॒ दक्षि॑णाः स॒हस्र॑सम्मितः सुव॒र्गो लो॒कः सु॑व॒र्गस्य॑ लो॒कस्या॒भिजि॑त्यै ॥ \newline

\textbf{Pada Paata} \newline

द्विभा॑ग॒मिति॒ द्वि - भा॒ग॒म् । ब्र॒ह्मणे᳚ । तृती॑यम् । अ॒ग्नीध॒ इत्य॑ग्नि- इधे᳚ । ऐ॒न्द्रः । वै । ब्र॒ह्मा । वै॒ष्ण॒वः । अ॒ग्नीदित्य॑ग्नि-इत् । यथा᳚ । ए॒व । तौ । अक॑ल्पेताम् । इति॑ । अथ॑ । आ॒हुः॒ । या । क॒ल्या॒णी । ब॒हु॒रू॒पेति॑ बहु - रू॒पा । सा । देया᳚ । इति॑ । अथ॑ । आ॒हुः॒ । या । द्वि॒रू॒पेति॑ द्वि - रू॒पा । उ॒भ॒यत॑ए॒नीत्य॑भ॒यतः॑ - ए॒नी॒ । सा । देया᳚ । इति॑ । स॒हस्र॑स्य । परि॑गृहीत्या॒ इति॒ परि॑-गृ॒ही॒त्यै॒ । तत् । वै । ए॒तत् । स॒हस्र॑स्य । अय॑नम् । स॒हस्र᳚म् । स्तो॒त्रीयाः᳚ । स॒हस्र᳚म् । दक्षि॑णाः । स॒हस्र॑संमित॒ इति॑ स॒हस्र॑ - स॒मिं॒तः॒ । सु॒व॒र्ग इति॑ सुवः - गः । लो॒कः । सु॒व॒र्गस्येति॑ सुवः - गस्य॑ । लो॒कस्य॑ । अ॒भिजि॑त्या॒ इत्य॒भि - जि॒त्यै॒ ॥  \newline


\textbf{Krama Paata} \newline

द्विभा॑गम् ब्र॒ह्मणे᳚ । द्विभा॑ग॒मिति॒ द्वि - भा॒ग॒म् । ब्र॒ह्मणे॒ तृती॑यम् । तृती॑यम॒ग्नीधे᳚ । अ॒ग्नीध॑ ऐ॒न्द्रः । अ॒ग्नीध॒ इत्य॑ग्नि - इधे᳚ । ऐ॒न्द्रो वै । वै ब्र॒ह्मा । ब्र॒ह्मा वै᳚ष्ण॒वः । वै॒ष्ण॒वो᳚ऽग्नीत् । अ॒ग्नीद् यथा᳚ । अ॒ग्नीदित्य॑ग्नि - इत् । यथै॒व । ए॒व तौ । तावक॑ल्पेताम् । अक॑ल्पेता॒मिति॑ । इत्यथ॑ । अथा॑हुः । आ॒हु॒र् या । या क॑ल्या॒णी । क॒ल्या॒णी ब॑हुरू॒पा । ब॒हु॒रू॒पा सा । ब॒हु॒रू॒पेति॑ बहु - रू॒पा । सा देया᳚ । देयेति॑ । इत्यथ॑ । अथा॑हुः । आ॒हु॒र् या । या द्वि॑रू॒पा । द्वि॒रू॒पोभ॒यत॑एनी । द्वि॒रू॒पेति॑ द्वि - रू॒पा । उ॒भ॒यत॑एनी॒ सा । उ॒भ॒यत॑,ए॒नीत्यु॑भ॒यतः॑ - ए॒नी॒ । सा देया᳚ । देयेति॑ । इति॑ स॒हस्र॑स्य । स॒हस्र॑स्य॒ परि॑गृहीत्यै । परि॑गृहीत्यै॒ तत् । परि॑गृहीत्या॒ इति॒ परि॑ - गृ॒ही॒त्यै॒ । तद् वै । वा ए॒तत् । ए॒तथ् स॒हस्र॑स्य । स॒हस्र॒स्याय॑नम् । अय॑नꣳ स॒हस्र᳚म् । स॒हस्रꣳ॑ स्तो॒त्रीयाः᳚ । स्तो॒त्रीयाः᳚ स॒हस्र᳚म् । स॒हस्र॒म् दक्षि॑णाः । दक्षि॑णाः स॒हस्र॑सम्मितः । स॒हस्र॑सम्मितः सुव॒र्गः । स॒हस्र॑सम्मित॒ इति॑ स॒हस्र॑ - स॒म्मि॒तः॒ । सु॒व॒र्गा लो॒कः । सु॒व॒र्ग इति॑ सुवः - गः । लो॒कः सु॑व॒र्गस्य॑ । सु॒व॒र्गस्य॑ लो॒कस्य॑ । सु॒व॒र्गस्येति॑ सुवः - गस्य॑ । लो॒कस्या॒भिजि॑त्यै । अ॒भिजि॑त्या॒ इत्य॒भि - जि॒त्यै॒ । \newline

\textbf{Jatai Paata} \newline

1. द्विभा॑गम् ब्र॒ह्मणे᳚ ब्र॒ह्मणे॒ द्विभा॑ग॒म् द्विभा॑गम् ब्र॒ह्मणे᳚ । \newline
2. द्विभा॑ग॒मिति॒ द्वि - भा॒ग॒म् । \newline
3. ब्र॒ह्मणे॒ तृती॑य॒म् तृती॑यम् ब्र॒ह्मणे᳚ ब्र॒ह्मणे॒ तृती॑यम् । \newline
4. तृती॑य म॒ग्नीधे॒ ऽग्नीधे॒ तृती॑य॒म् तृती॑य म॒ग्नीधे᳚ । \newline
5. अ॒ग्नीध॑ ऐ॒न्द्र ऐ॒न्द्रो᳚ ऽग्नीधे॒ ऽग्नीध॑ ऐ॒न्द्रः । \newline
6. अ॒ग्नीध॒ इत्य॑ग्नि - इधे᳚ । \newline
7. ऐ॒न्द्रो वै वा ऐ॒न्द्र ऐ॒न्द्रो वै । \newline
8. वै ब्र॒ह्मा ब्र॒ह्मा वै वै ब्र॒ह्मा । \newline
9. ब्र॒ह्मा वै᳚ष्ण॒वो वै᳚ष्ण॒वो ब्र॒ह्मा ब्र॒ह्मा वै᳚ष्ण॒वः । \newline
10. वै॒ष्ण॒वो᳚ ऽग्नी द॒ग्नीद् वै᳚ष्ण॒वो वै᳚ष्ण॒वो᳚ ऽग्नीत् । \newline
11. अ॒ग्नीद् यथा॒ यथा॒ ऽग्नी द॒ग्नीद् यथा᳚ । \newline
12. अ॒ग्नीदित्य॑ग्नि - इत् । \newline
13. यथै॒ वैव यथा॒ यथै॒व । \newline
14. ए॒व तौ ता वे॒वैव तौ । \newline
15. ता वक॑ल्पेता॒ मक॑ल्पेता॒म् तौ ता वक॑ल्पेताम् । \newline
16. अक॑ल्पेता॒ मितीत्यक॑ल्पेता॒ मक॑ल्पेता॒ मिति॑ । \newline
17. इत्यथाथे तीत्यथ॑ । \newline
18. अथा॑हु राहु॒ रथा था॑हुः । \newline
19. आ॒हु॒र् या या ऽऽहु॑ राहु॒र् या । \newline
20. या क॑ल्या॒णी क॑ल्या॒णी या या क॑ल्या॒णी । \newline
21. क॒ल्या॒णी ब॑हुरू॒पा ब॑हुरू॒पा क॑ल्या॒णी क॑ल्या॒णी ब॑हुरू॒पा । \newline
22. ब॒हु॒रू॒पा सा सा ब॑हुरू॒पा ब॑हुरू॒पा सा । \newline
23. ब॒हु॒रू॒पेति॑ बहु - रू॒पा । \newline
24. सा देया॒ देया॒ सा सा देया᳚ । \newline
25. देयेतीति॒ देया॒ देयेति॑ । \newline
26. इत्यथाथे तीत्यथ॑ । \newline
27. अथा॑हु राहु॒ रथा था॑हुः । \newline
28. आ॒हु॒र् या या ऽऽहु॑ राहु॒र् या । \newline
29. या द्वि॑रू॒पा द्वि॑रू॒पा या या द्वि॑रू॒पा । \newline
30. द्वि॒रू॒ पोभ॒यत॑एन्यु भ॒यत॑एनी द्विरू॒पा द्वि॑रू॒ पोभ॒यत॑एनी । \newline
31. द्वि॒रू॒पेति॑ द्वि - रू॒पा । \newline
32. उ॒भ॒यत॑एनी॒ सा सोभ॒यत॑ए न्युभ॒यत॑एनी॒ सा । \newline
33. उ॒भ॒यत॑ए॒नीत्यु॑भ॒यतः॑ - ए॒नी॒ । \newline
34. सा देया॒ देया॒ सा सा देया᳚ । \newline
35. देयेतीति॒ देया॒ देयेति॑ । \newline
36. इति॑ स॒हस्र॑स्य स॒हस्र॒ स्येतीति॑ स॒हस्र॑स्य । \newline
37. स॒हस्र॑स्य॒ परि॑गृहीत्यै॒ परि॑गृहीत्यै स॒हस्र॑स्य स॒हस्र॑स्य॒ परि॑गृहीत्यै । \newline
38. परि॑गृहीत्यै॒ तत् तत् परि॑गृहीत्यै॒ परि॑गृहीत्यै॒ तत् । \newline
39. परि॑गृहीत्या॒ इति॒ परि॑ - गृ॒ही॒त्यै॒ । \newline
40. तद् वै वै तत् तद् वै । \newline
41. वा ए॒त दे॒तद् वै वा ए॒तत् । \newline
42. ए॒तथ् स॒हस्र॑स्य स॒हस्र॑ स्यै॒त दे॒तथ् स॒हस्र॑स्य । \newline
43. स॒हस्र॒ स्याय॑न॒ मय॑नꣳ स॒हस्र॑स्य स॒हस्र॒ स्याय॑नम् । \newline
44. अय॑नꣳ स॒हस्रꣳ॑ स॒हस्र॒ मय॑न॒ मय॑नꣳ स॒हस्र᳚म् । \newline
45. स॒हस्रꣳ॑ स्तो॒त्रीयाः᳚ स्तो॒त्रीयाः᳚ स॒हस्रꣳ॑ स॒हस्रꣳ॑ स्तो॒त्रीयाः᳚ । \newline
46. स्तो॒त्रीयाः᳚ स॒हस्रꣳ॑ स॒हस्रꣳ॑ स्तो॒त्रीयाः᳚ स्तो॒त्रीयाः᳚ स॒हस्र᳚म् । \newline
47. स॒हस्र॒म् दक्षि॑णा॒ दक्षि॑णाः स॒हस्रꣳ॑ स॒हस्र॒म् दक्षि॑णाः । \newline
48. दक्षि॑णाः स॒हस्र॑सम्मितः स॒हस्र॑सम्मितो॒ दक्षि॑णा॒ दक्षि॑णाः स॒हस्र॑सम्मितः । \newline
49. स॒हस्र॑सम्मितः सुव॒र्गः सु॑व॒र्गः स॒हस्र॑सम्मितः स॒हस्र॑सम्मितः सुव॒र्गः । \newline
50. स॒हस्र॑सम्मित॒ इति॑ स॒हस्र॑ - स॒म्मि॒तः॒ । \newline
51. सु॒व॒र्गो लो॒को लो॒कः सु॑व॒र्गः सु॑व॒र्गो लो॒कः । \newline
52. सु॒व॒र्ग इति॑ सुवः - गः । \newline
53. लो॒कः सु॑व॒र्गस्य॑ सुव॒र्गस्य॑ लो॒को लो॒कः सु॑व॒र्गस्य॑ । \newline
54. सु॒व॒र्गस्य॑ लो॒कस्य॑ लो॒कस्य॑ सुव॒र्गस्य॑ सुव॒र्गस्य॑ लो॒कस्य॑ । \newline
55. सु॒व॒र्गस्येति॑ सुवः - गस्य॑ । \newline
56. लो॒कस्या॒ भिजि॑त्या अ॒भिजि॑त्यै लो॒कस्य॑ लो॒कस्या॒ भिजि॑त्यै । \newline
57. अ॒भिजि॑त्या॒ इत्य॒भि - जि॒त्यै॒ । \newline

\textbf{Ghana Paata } \newline

1. द्विभा॑गम् ब्र॒ह्मणे᳚ ब्र॒ह्मणे॒ द्विभा॑ग॒म् द्विभा॑गम् ब्र॒ह्मणे॒ तृती॑य॒म् तृती॑यम् ब्र॒ह्मणे॒ द्विभा॑ग॒म् द्विभा॑गम् ब्र॒ह्मणे॒ तृती॑यम् । \newline
2. द्विभा॑ग॒मिति॒ द्वि - भा॒ग॒म् । \newline
3. ब्र॒ह्मणे॒ तृती॑य॒म् तृती॑यम् ब्र॒ह्मणे᳚ ब्र॒ह्मणे॒ तृती॑य म॒ग्नीधे॒ ऽग्नीधे॒ तृती॑यम् ब्र॒ह्मणे᳚ ब्र॒ह्मणे॒ तृती॑य म॒ग्नीधे᳚ । \newline
4. तृती॑य म॒ग्नीधे॒ ऽग्नीधे॒ तृती॑य॒म् तृती॑य म॒ग्नीध॑ ऐ॒न्द्र ऐ॒न्द्रो᳚ ऽग्नीधे॒ तृती॑य॒म् तृती॑य म॒ग्नीध॑ ऐ॒न्द्रः । \newline
5. अ॒ग्नीध॑ ऐ॒न्द्र ऐ॒न्द्रो᳚ ऽग्नीधे॒ ऽग्नीध॑ ऐ॒न्द्रो वै वा ऐ॒न्द्रो᳚ ऽग्नीधे॒ ऽग्नीध॑ ऐ॒न्द्रो वै । \newline
6. अ॒ग्नीध॒ इत्य॑ग्नि - इधे᳚ । \newline
7. ऐ॒न्द्रो वै वा ऐ॒न्द्र ऐ॒न्द्रो वै ब्र॒ह्मा ब्र॒ह्मा वा ऐ॒न्द्र ऐ॒न्द्रो वै ब्र॒ह्मा । \newline
8. वै ब्र॒ह्मा ब्र॒ह्मा वै वै ब्र॒ह्मा वै᳚ष्ण॒वो वै᳚ष्ण॒वो ब्र॒ह्मा वै वै ब्र॒ह्मा वै᳚ष्ण॒वः । \newline
9. ब्र॒ह्मा वै᳚ष्ण॒वो वै᳚ष्ण॒वो ब्र॒ह्मा ब्र॒ह्मा वै᳚ष्ण॒वो᳚ ऽग्नी द॒ग्नीद् वै᳚ष्ण॒वो ब्र॒ह्मा ब्र॒ह्मा वै᳚ष्ण॒वो᳚ ऽग्नीत् । \newline
10. वै॒ष्ण॒वो᳚ ऽग्नी द॒ग्नीद् वै᳚ष्ण॒वो वै᳚ष्ण॒वो᳚ ऽग्नीद् यथा॒ यथा॒ ऽग्नीद् वै᳚ष्ण॒वो वै᳚ष्ण॒वो᳚ ऽग्नीद् यथा᳚ । \newline
11. अ॒ग्नीद् यथा॒ यथा॒ ऽग्नी द॒ग्नीद् यथै॒ वैव यथा॒ ऽग्नी द॒ग्नीद् यथै॒व । \newline
12. अ॒ग्नीदित्य॑ग्नि - इत् । \newline
13. यथै॒ वैव यथा॒ यथै॒व तौ ता वे॒व यथा॒ यथै॒व तौ । \newline
14. ए॒व तौ ता वे॒वैव ता वक॑ल्पेता॒ मक॑ल्पेता॒म् ता वे॒वैव ता वक॑ल्पेताम् । \newline
15. ता वक॑ल्पेता॒ मक॑ल्पेता॒म् तौ ता वक॑ल्पेता॒ मिती त्यक॑ल्पेता॒म् तौ ता वक॑ल्पेता॒ मिति॑ । \newline
16. अक॑ल्पेता॒ मिती त्यक॑ल्पेता॒ मक॑ल्पेता॒ मित्य थाथे त्यक॑ल्पेता॒ मक॑ल्पेता॒ मित्यथ॑ । \newline
17. इत्यथा थेती त्यथा॑हु राहु॒ रथेती त्यथा॑हुः । \newline
18. अथा॑हु राहु॒ रथा था॑हु॒र् या या ऽऽहु॒ रथा था॑हु॒र् या । \newline
19. आ॒हु॒र् या या ऽऽहु॑ राहु॒र् या क॑ल्या॒णी क॑ल्या॒णी या ऽऽहु॑ राहु॒र् या क॑ल्या॒णी । \newline
20. या क॑ल्या॒णी क॑ल्या॒णी या या क॑ल्या॒णी ब॑हुरू॒पा ब॑हुरू॒पा क॑ल्या॒णी या या क॑ल्या॒णी ब॑हुरू॒पा । \newline
21. क॒ल्या॒णी ब॑हुरू॒पा ब॑हुरू॒पा क॑ल्या॒णी क॑ल्या॒णी ब॑हुरू॒पा सा सा ब॑हुरू॒पा क॑ल्या॒णी क॑ल्या॒णी ब॑हुरू॒पा सा । \newline
22. ब॒हु॒रू॒पा सा सा ब॑हुरू॒पा ब॑हुरू॒पा सा देया॒ देया॒ सा ब॑हुरू॒पा ब॑हुरू॒पा सा देया᳚ । \newline
23. ब॒हु॒रू॒पेति॑ बहु - रू॒पा । \newline
24. सा देया॒ देया॒ सा सा देयेतीति॒ देया॒ सा सा देयेति॑ । \newline
25. देयेतीति॒ देया॒ देये त्यथा थेति॒ देया॒ देये त्यथ॑ । \newline
26. इत्यथा थेती त्यथा॑हु राहु॒ रथेती त्यथा॑हुः । \newline
27. अथा॑हु राहु॒ रथा था॑हु॒र् या या ऽऽहु॒ रथा था॑हु॒र् या । \newline
28. आ॒हु॒र् या या ऽऽहु॑ राहु॒र् या द्वि॑रू॒पा द्वि॑रू॒पा या ऽऽहु॑ राहु॒र् या द्वि॑रू॒पा । \newline
29. या द्वि॑रू॒पा द्वि॑रू॒पा या या द्वि॑रू॒पो भ॒यत॑ एन्युभ॒यत॑एनी द्विरू॒पा या या द्वि॑रू॒पो भ॒यत॑एनी । \newline
30. द्वि॒रू॒पो भ॒यत॑ एन्युभ॒यत॑एनी द्विरू॒पा द्वि॑रू॒पो भ॒यत॑एनी॒ सा सोभ॒यत॑एनी द्विरू॒पा द्वि॑रू॒पो भ॒यत॑एनी॒ सा । \newline
31. द्वि॒रू॒पेति॑ द्वि - रू॒पा । \newline
32. उ॒भ॒यत॑एनी॒ सा सोभ॒यत॑ एन्युभ॒यत॑एनी॒ सा देया॒ देया॒ सोभ॒यत॑ एन्युभ॒यत॑एनी॒ सा देया᳚ । \newline
33. उ॒भ॒यत॑ए॒नीत्यु॑भ॒यतः॑ - ए॒नी॒ । \newline
34. सा देया॒ देया॒ सा सा देयेतीति॒ देया॒ सा सा देयेति॑ । \newline
35. देयेतीति॒ देया॒ देयेति॑ स॒हस्र॑स्य स॒हस्र॒ स्येति॒ देया॒ देयेति॑ स॒हस्र॑स्य । \newline
36. इति॑ स॒हस्र॑स्य स॒हस्र॒ स्येतीति॑ स॒हस्र॑स्य॒ परि॑गृहीत्यै॒ परि॑गृहीत्यै स॒हस्र॒ स्येतीति॑ स॒हस्र॑स्य॒ परि॑गृहीत्यै । \newline
37. स॒हस्र॑स्य॒ परि॑गृहीत्यै॒ परि॑गृहीत्यै स॒हस्र॑स्य स॒हस्र॑स्य॒ परि॑गृहीत्यै॒ तत् तत् परि॑गृहीत्यै स॒हस्र॑स्य स॒हस्र॑स्य॒ परि॑गृहीत्यै॒ तत् । \newline
38. परि॑गृहीत्यै॒ तत् तत् परि॑गृहीत्यै॒ परि॑गृहीत्यै॒ तद् वै वै तत् परि॑गृहीत्यै॒ परि॑गृहीत्यै॒ तद् वै । \newline
39. परि॑गृहीत्या॒ इति॒ परि॑ - गृ॒ही॒त्यै॒ । \newline
40. तद् वै वै तत् तद् वा ए॒त दे॒तद् वै तत् तद् वा ए॒तत् । \newline
41. वा ए॒त दे॒तद् वै वा ए॒तथ् स॒हस्र॑स्य स॒हस्र॑ स्यै॒तद् वै वा ए॒तथ् स॒हस्र॑स्य । \newline
42. ए॒तथ् स॒हस्र॑स्य स॒हस्र॑ स्यै॒त दे॒तथ् स॒हस्र॒स्याय॑न॒ मय॑नꣳ स॒हस्र॑ स्यै॒त दे॒तथ् स॒हस्र॒स्याय॑नम् । \newline
43. स॒हस्र॒स्या य॑न॒ मय॑नꣳ स॒हस्र॑स्य स॒हस्र॒ स्याय॑नꣳ स॒हस्रꣳ॑ स॒हस्र॒ मय॑नꣳ स॒हस्र॑स्य स॒हस्र॒ स्याय॑नꣳ स॒हस्र᳚म् । \newline
44. अय॑नꣳ स॒हस्रꣳ॑ स॒हस्र॒ मय॑न॒ मय॑नꣳ स॒हस्रꣳ॑ स्तो॒त्रीयाः᳚ स्तो॒त्रीयाः᳚ स॒हस्र॒ मय॑न॒ मय॑नꣳ स॒हस्रꣳ॑ स्तो॒त्रीयाः᳚ । \newline
45. स॒हस्रꣳ॑ स्तो॒त्रीयाः᳚ स्तो॒त्रीयाः᳚ स॒हस्रꣳ॑ स॒हस्रꣳ॑ स्तो॒त्रीयाः᳚ स॒हस्रꣳ॑ स॒हस्रꣳ॑ स्तो॒त्रीयाः᳚ स॒हस्रꣳ॑ स॒हस्रꣳ॑ स्तो॒त्रीयाः᳚ स॒हस्र᳚म् । \newline
46. स्तो॒त्रीयाः᳚ स॒हस्रꣳ॑ स॒हस्रꣳ॑ स्तो॒त्रीयाः᳚ स्तो॒त्रीयाः᳚ स॒हस्र॒म् दक्षि॑णा॒ दक्षि॑णाः स॒हस्रꣳ॑ स्तो॒त्रीयाः᳚ स्तो॒त्रीयाः᳚ स॒हस्र॒म् दक्षि॑णाः । \newline
47. स॒हस्र॒म् दक्षि॑णा॒ दक्षि॑णाः स॒हस्रꣳ॑ स॒हस्र॒म् दक्षि॑णाः स॒हस्र॑सम्मितः स॒हस्र॑सम्मितो॒ दक्षि॑णाः स॒हस्रꣳ॑ स॒हस्र॒म् दक्षि॑णाः स॒हस्र॑सम्मितः । \newline
48. दक्षि॑णाः स॒हस्र॑सम्मितः स॒हस्र॑सम्मितो॒ दक्षि॑णा॒ दक्षि॑णाः स॒हस्र॑सम्मितः सुव॒र्गः सु॑व॒र्गः स॒हस्र॑सम्मितो॒ दक्षि॑णा॒ दक्षि॑णाः स॒हस्र॑सम्मितः सुव॒र्गः । \newline
49. स॒हस्र॑सम्मितः सुव॒र्गः सु॑व॒र्गः स॒हस्र॑सम्मितः स॒हस्र॑सम्मितः सुव॒र्गो लो॒को लो॒कः सु॑व॒र्गः स॒हस्र॑सम्मितः स॒हस्र॑सम्मितः सुव॒र्गो लो॒कः । \newline
50. स॒हस्र॑सम्मित॒ इति॑ स॒हस्र॑ - स॒म्मि॒तः॒ । \newline
51. सु॒व॒र्गो लो॒को लो॒कः सु॑व॒र्गः सु॑व॒र्गो लो॒कः सु॑व॒र्गस्य॑ सुव॒र्गस्य॑ लो॒कः सु॑व॒र्गः सु॑व॒र्गो लो॒कः सु॑व॒र्गस्य॑ । \newline
52. सु॒व॒र्ग इति॑ सुवः - गः । \newline
53. लो॒कः सु॑व॒र्गस्य॑ सुव॒र्गस्य॑ लो॒को लो॒कः सु॑व॒र्गस्य॑ लो॒कस्य॑ लो॒कस्य॑ सुव॒र्गस्य॑ लो॒को लो॒कः सु॑व॒र्गस्य॑ लो॒कस्य॑ । \newline
54. सु॒व॒र्गस्य॑ लो॒कस्य॑ लो॒कस्य॑ सुव॒र्गस्य॑ सुव॒र्गस्य॑ लो॒कस्या॒ भिजि॑त्या अ॒भिजि॑त्यै लो॒कस्य॑ सुव॒र्गस्य॑ सुव॒र्गस्य॑ लो॒कस्या॒ भिजि॑त्यै । \newline
55. सु॒व॒र्गस्येति॑ सुवः - गस्य॑ । \newline
56. लो॒कस्या॒ भिजि॑त्या अ॒भिजि॑त्यै लो॒कस्य॑ लो॒कस्या॒ भिजि॑त्यै । \newline
57. अ॒भिजि॑त्या॒ इत्य॒भि - जि॒त्यै॒ । \newline
\pagebreak
\markright{ TS 7.1.6.1  \hfill https://www.vedavms.in \hfill}

\section{ TS 7.1.6.1 }

\textbf{TS 7.1.6.1 } \newline
\textbf{Samhita Paata} \newline

सोमो॒ वै स॒हस्र॑मविन्द॒त् तमिन्द्रो ऽन्व॑विन्द॒त् तौ य॒मो न्याग॑च्छ॒त् ताव॑ब्रवी॒दस्तु॒ मेऽत्राऽपीत्यस्तु॒ ही(3) इत्य॑ब्रूताꣳ॒॒ स य॒म एक॑स्यां ॅवी॒र्यं॑ पर्य॑पश्यदि॒यं ॅवा अ॒स्य स॒हस्र॑स्य वी॒र्यं॑ बिभ॒र्तीति॒ ताव॑ब्रवीदि॒यं ममास्त्वे॒तद्-यु॒वयो॒रिति॒ ताव॑ब्रूताꣳ॒॒ सर्वे॒ वा ए॒तदे॒तस्यां᳚ ॅवी॒र्यं॑ - [  ] \newline

\textbf{Pada Paata} \newline

सोमः॑ । वै । स॒हस्र᳚म् । अ॒वि॒न्द॒त् । तम् । इन्द्रः॑ । अन्विति॑ । अ॒न्वि॒न्द॒त् । तौ । य॒मः । न्याग॑च्छ॒दिति॑ नि - आग॑च्छत् । तौ । अ॒ब्र॒वी॒त् । अस्तु॑ । मे॒ । अत्र॑ । अपीति॑ । इति॑ । अस्तु॑ । ही(3) । इति॑ । अ॒ब्रू॒ता॒म् । सः । य॒मः । एक॑स्याम् । वी॒र्य᳚म् । परीति॑ । अ॒प॒श्य॒त् । इ॒यम् । वै । अ॒स्य । स॒हस्र॑स्य । वी॒र्य᳚म् । बि॒भ॒र्ति॒ । इति॑ । तौ । अ॒ब्र॒वी॒त् । इ॒यम् । मम॑ । अस्तु॑ । ए॒तत् । यु॒वयोः᳚ । इति॑ । तौ । अ॒ब्रू॒ता॒म् । सर्वे᳚ । वै । ए॒तत् । ए॒तस्या᳚म् । वी॒र्य᳚म् ।  \newline


\textbf{Krama Paata} \newline

सोमो॒ वै । वै स॒हस्र᳚म् । स॒हस्र॑मविन्दत् । अ॒वि॒न्द॒त् तम् । तमिन्द्रः॑ । इन्द्रोऽनु॑ । अन्व॑विन्दत् । अ॒वि॒न्द॒त् तौ । तौ य॒मः । य॒मो न्याग॑च्छत् । न्याग॑च्छ॒त् तौ । न्याग॑च्छ॒दिति॑ नि - आग॑च्छत् । ताव॑ब्रवीत् । अ॒ब्र॒वी॒दस्तु॑ । अस्तु॑ मे । मेऽत्र॑ । अत्रापि॑ । अपीति॑ । इत्यस्तु॑ । अस्तु॒ ही(3) । ही(3) इति॑ । इत्य॑ब्रूताम् । अ॒ब्रू॒ताꣳ॒॒ सः । स य॒मः । य॒म एक॑स्याम् । एक॑स्याम् ॅवी॒र्य᳚म् । वी॒र्य॑म् परि॑ । पर्य॑पश्यत् । अ॒प॒श्य॒दि॒यम् । इ॒यम् ॅवै । वा अ॒स्य । अ॒स्य स॒हस्र॑स्य । स॒हस्र॑स्य वी॒र्य᳚म् । वी॒र्य॑म् बिभर्ति । बि॒भ॒र्तीति॑ । इति॒ तौ । ताव॑ब्रवीत् । अ॒ब्र॒वी॒दि॒यम् । इ॒यम् मम॑ । ममास्तु॑ । अस्त्वे॒तत् । ए॒तद् यु॒वयोः᳚ । यु॒वयो॒रिति॑ । इति॒ तौ । ताव॑ब्रूताम् । अ॒ब्रू॒ताꣳ॒॒ सर्वे᳚ । सर्वे॒ वै । वा ए॒तत् । ए॒तदे॒तस्या᳚म् । ए॒तस्या᳚म् ॅवी॒र्य᳚म् । वी॒र्य॑म् परि॑ \newline

\textbf{Jatai Paata} \newline

1. सोमो॒ वै वै सोमः॒ सोमो॒ वै । \newline
2. वै स॒हस्रꣳ॑ स॒हस्रं॒ ॅवै वै स॒हस्र᳚म् । \newline
3. स॒हस्र॑ मविन्द दविन्दथ् स॒हस्रꣳ॑ स॒हस्र॑ मविन्दत् । \newline
4. अ॒वि॒न्द॒त् तम् त म॑विन्द दविन्द॒त् तम् । \newline
5. त मिन्द्र॒ इन्द्र॒ स्तम् त मिन्द्रः॑ । \newline
6. इन्द्रो ऽन्वन् विन्द्र॒ इन्द्रो ऽनु॑ । \newline
7. अन्व॑न्विन् ददन्विन् द॒दन्वन् व॑न्विन्दत् । \newline
8. अ॒न्वि॒न्द॒त् तौ ता व॑न्विन्द दन्विन्द॒त् तौ । \newline
9. तौ य॒मो य॒म स्तौ तौ य॒मः । \newline
10. य॒मो न्याग॑च्छ॒न् न्याग॑च्छद् य॒मो य॒मो न्याग॑च्छत् । \newline
11. न्याग॑च्छ॒त् तौ तौ न्याग॑च्छ॒न् न्याग॑च्छ॒त् तौ । \newline
12. न्याग॑च्छ॒दिति॑ नि - आग॑च्छत् । \newline
13. ता व॑ब्रवी दब्रवी॒त् तौ ता व॑ब्रवीत् । \newline
14. अ॒ब्र॒वी॒ दस्त्व स्त्व॑ब्रवी दब्रवी॒ दस्तु॑ । \newline
15. अस्तु॑ मे मे॒ अस्त्वस्तु॑ मे । \newline
16. मे ऽत्रात्र॑ मे॒ मे ऽत्र॑ । \newline
17. अत्राप्य प्यत्रा त्रापि॑ । \newline
18. अपीती त्य प्यपीति॑ । \newline
19. इत्य स्त्व स्त्विती त्यस्तु॑ । \newline
20. अस्तु॒ ही(3) ही(3) अस्त्वस्तु॒ ही(3) । \newline
21. ही(3) इतीति॒ ही(3) ही(3) इति॑ । \newline
22. इत्य॑ब्रूता मब्रूता॒ मिती त्य॑ब्रूताम् । \newline
23. अ॒ब्रू॒ताꣳ॒॒ स सो᳚ ऽब्रूता मब्रूताꣳ॒॒ सः । \newline
24. स य॒मो य॒मः स स य॒मः । \newline
25. य॒म एक॑स्या॒ मेक॑स्यां ॅय॒मो य॒म एक॑स्याम् । \newline
26. एक॑स्यां ॅवी॒र्यं॑ ॅवी॒र्य॑ मेक॑स्या॒ मेक॑स्यां ॅवी॒र्य᳚म् । \newline
27. वी॒र्य॑म् परि॒ परि॑ वी॒र्यं॑ ॅवी॒र्य॑म् परि॑ । \newline
28. पर्य॑पश्य दपश्य॒त् परि॒ पर्य॑पश्यत् । \newline
29. अ॒प॒श्य॒ दि॒य मि॒य म॑पश्य दपश्य दि॒यम् । \newline
30. इ॒यं ॅवै वा इ॒य मि॒यं ॅवै । \newline
31. वा अ॒स्यास्य वै वा अ॒स्य । \newline
32. अ॒स्य स॒हस्र॑स्य स॒हस्र॑स्या॒ स्यास्य स॒हस्र॑स्य । \newline
33. स॒हस्र॑स्य वी॒र्यं॑ ॅवी॒र्यꣳ॑ स॒हस्र॑स्य स॒हस्र॑स्य वी॒र्य᳚म् । \newline
34. वी॒र्य॑म् बिभर्ति बिभर्ति वी॒र्यं॑ ॅवी॒र्य॑म् बिभर्ति । \newline
35. बि॒भ॒र्तीतीति॑ बिभर्ति बिभ॒र्तीति॑ । \newline
36. इति॒ तौ ता वितीति॒ तौ । \newline
37. ता व॑ब्रवी दब्रवी॒त् तौ ता व॑ब्रवीत् । \newline
38. अ॒ब्र॒वी॒ दि॒य मि॒य म॑ब्रवी दब्रवी दि॒यम् । \newline
39. इ॒यम् मम॒ ममे॒य मि॒यम् मम॑ । \newline
40. ममा स्त्वस्तु॒ मम॒ ममास्तु॑ । \newline
41. अस्त्वे॒त दे॒त दस्त्व स्त्वे॒तत् । \newline
42. ए॒तद् यु॒वयो᳚र् यु॒वयो॑ रे॒त दे॒तद् यु॒वयोः᳚ । \newline
43. यु॒वयो॒ रितीति॑ यु॒वयो᳚र् यु॒वयो॒ रिति॑ । \newline
44. इति॒ तौ ता वितीति॒ तौ । \newline
45. ता व॑ब्रूता मब्रूता॒म् तौ ता व॑ब्रूताम् । \newline
46. अ॒ब्रू॒ताꣳ॒॒ सर्वे॒ सर्वे᳚ ऽब्रूता मब्रूताꣳ॒॒ सर्वे᳚ । \newline
47. सर्वे॒ वै वै सर्वे॒ सर्वे॒ वै । \newline
48. वा ए॒त दे॒तद् वै वा ए॒तत् । \newline
49. ए॒त दे॒तस्या॑ मे॒तस्या॑ मे॒त दे॒त दे॒तस्या᳚म् । \newline
50. ए॒तस्यां᳚ ॅवी॒र्यं॑ ॅवी॒र्य॑ मे॒तस्या॑ मे॒तस्यां᳚ ॅवी॒र्य᳚म् । \newline
51. वी॒र्य॑म् परि॒ परि॑ वी॒र्यं॑ ॅवी॒र्य॑म् परि॑ । \newline

\textbf{Ghana Paata } \newline

1. सोमो॒ वै वै सोमः॒ सोमो॒ वै स॒हस्रꣳ॑ स॒हस्रं॒ ॅवै सोमः॒ सोमो॒ वै स॒हस्र᳚म् । \newline
2. वै स॒हस्रꣳ॑ स॒हस्रं॒ ॅवै वै स॒हस्र॑ मविन्द दविन्दथ् स॒हस्रं॒ ॅवै वै स॒हस्र॑ मविन्दत् । \newline
3. स॒हस्र॑ मविन्द दविन्दथ् स॒हस्रꣳ॑ स॒हस्र॑ मविन्द॒त् तम् त म॑विन्दथ् स॒हस्रꣳ॑ स॒हस्र॑ मविन्द॒त् तम् । \newline
4. अ॒वि॒न्द॒त् तम् त म॑विन्द दविन्द॒त् त मिन्द्र॒ इन्द्र॒ स्त म॑विन्द दविन्द॒त् त मिन्द्रः॑ । \newline
5. त मिन्द्र॒ इन्द्र॒ स्तम् त मिन्द्रो ऽन्वन् विन्द्र॒ स्तम् त मिन्द्रो ऽनु॑ । \newline
6. इन्द्रो ऽन्वन् विन्द्र॒ इन्द्रो ऽन्व॑न्विन्द दन्विन्द॒ दन्विन्द्र॒ इन्द्रो ऽन्व॑न्विन्दत् । \newline
7. अन्व॑न्विन्द दन्विन्द॒ दन्वन् व॑न्विन्द॒त् तौ ता व॑न्विन्द॒ दन्वन् व॑न्विन्द॒त् तौ । \newline
8. अ॒न्वि॒न्द॒त् तौ ता व॑न्विन्द दन्विन्द॒त् तौ य॒मो य॒म स्ता व॑न्विन्द दन्विन्द॒त् तौ य॒मः । \newline
9. तौ य॒मो य॒म स्तौ तौ य॒मो न्याग॑च्छ॒न् न्याग॑च्छद् य॒म स्तौ तौ य॒मो न्याग॑च्छत् । \newline
10. य॒मो न्याग॑च्छ॒न् न्याग॑च्छद् य॒मो य॒मो न्याग॑च्छ॒त् तौ तौ न्याग॑च्छद् य॒मो य॒मो न्याग॑च्छ॒त् तौ । \newline
11. न्याग॑च्छ॒त् तौ तौ न्याग॑च्छ॒न् न्याग॑च्छ॒त् ता व॑ब्रवी दब्रवी॒त् तौ न्याग॑च्छ॒न् न्याग॑च्छ॒त् ता व॑ब्रवीत् । \newline
12. न्याग॑च्छ॒दिति॑ नि - आग॑च्छत् । \newline
13. ता व॑ब्रवी दब्रवी॒त् तौ ता व॑ब्रवी॒ दस्त्व स्त्व॑ब्रवी॒त् तौ ता व॑ब्रवी॒ दस्तु॑ । \newline
14. अ॒ब्र॒वी॒ दस्त्व स्त्व॑ब्रवी दब्रवी॒ दस्तु॑ मे मे॒ अस्त्व॑ब्रवी दब्रवी॒ दस्तु॑ मे । \newline
15. अस्तु॑ मे मे॒ अस्त्वस्तु॒ मे ऽत्रात्र॑ मे॒ अस्त्वस्तु॒ मे ऽत्र॑ । \newline
16. मे ऽत्रात्र॑ मे॒ मे ऽत्राप्य प्यत्र॑ मे॒ मे ऽत्रापि॑ । \newline
17. अत्राप्य प्यत्रा त्रापीती त्यप्य त्रात्रा पीति॑ । \newline
18. अपीती त्यप्य पीत्य स्त्व स्त्वित्य प्यपी त्यस्तु॑ । \newline
19. इत्यस्त्व स्त्विती त्यस्तु॒ ही(3) ही(3) अस्त्विती त्यस्तु॒ ही(3) । \newline
20. अस्तु॒ ही(3) हि(3) अस्त्वस्तु॒ ही(3) इतीति॒ ही(3) अस्त्वस्तु॒ ही(3) इति॑ । \newline
21. ही(3) इतीति॒ ही(3) ही(3) इत्य॑ब्रूता मब्रूता॒ मिति॒ ही(3) ही(3) इत्य॑ब्रूताम् । \newline
22. इत्य॑ब्रूता मब्रूता॒ मिती त्य॑ब्रूताꣳ॒॒ स सो᳚ ऽब्रूता॒ मिती त्य॑ब्रूताꣳ॒॒ सः । \newline
23. अ॒ब्रू॒ताꣳ॒॒ स सो᳚ ऽब्रूता मब्रूताꣳ॒॒ स य॒मो य॒मः सो᳚ ऽब्रूता मब्रूताꣳ॒॒ स य॒मः । \newline
24. स य॒मो य॒मः स स य॒म एक॑स्या॒ मेक॑स्यां ॅय॒मः स स य॒म एक॑स्याम् । \newline
25. य॒म एक॑स्या॒ मेक॑स्यां ॅय॒मो य॒म एक॑स्यां ॅवी॒र्यं॑ ॅवी॒र्य॑ मेक॑स्यां ॅय॒मो य॒म एक॑स्यां ॅवी॒र्य᳚म् । \newline
26. एक॑स्यां ॅवी॒र्यं॑ ॅवी॒र्य॑ मेक॑स्या॒ मेक॑स्यां ॅवी॒र्य॑म् परि॒ परि॑ वी॒र्य॑ मेक॑स्या॒ मेक॑स्यां ॅवी॒र्य॑म् परि॑ । \newline
27. वी॒र्य॑म् परि॒ परि॑ वी॒र्यं॑ ॅवी॒र्य॑म् पर्य॑पश्य दपश्य॒त् परि॑ वी॒र्यं॑ ॅवी॒र्य॑म् पर्य॑पश्यत् । \newline
28. पर्य॑पश्य दपश्य॒त् परि॒ पर्य॑पश्य दि॒य मि॒य म॑पश्य॒त् परि॒ पर्य॑पश्य दि॒यम् । \newline
29. अ॒प॒श्य॒ दि॒य मि॒य म॑पश्य दपश्य दि॒यं ॅवै वा इ॒य म॑पश्य दपश्य दि॒यं ॅवै । \newline
30. इ॒यं ॅवै वा इ॒य मि॒यं ॅवा अ॒स्यास्य वा इ॒य मि॒यं ॅवा अ॒स्य । \newline
31. वा अ॒स्यास्य वै वा अ॒स्य स॒हस्र॑स्य स॒हस्र॑ स्या॒स्य वै वा अ॒स्य स॒हस्र॑स्य । \newline
32. अ॒स्य स॒हस्र॑स्य स॒हस्र॑स्या॒ स्यास्य स॒हस्र॑स्य वी॒र्यं॑ ॅवी॒र्यꣳ॑ स॒हस्र॑स्या॒ स्यास्य स॒हस्र॑स्य वी॒र्य᳚म् । \newline
33. स॒हस्र॑स्य वी॒र्यं॑ ॅवी॒र्यꣳ॑ स॒हस्र॑स्य स॒हस्र॑स्य वी॒र्य॑म् बिभर्ति बिभर्ति वी॒र्यꣳ॑ स॒हस्र॑स्य स॒हस्र॑स्य वी॒र्य॑म् बिभर्ति । \newline
34. वी॒र्य॑म् बिभर्ति बिभर्ति वी॒र्यं॑ ॅवी॒र्य॑म् बिभ॒र्ती तीति॑ बिभर्ति वी॒र्यं॑ ॅवी॒र्य॑म् बिभ॒र्तीति॑ । \newline
35. बि॒भ॒र्ती तीति॑ बिभर्ति बिभ॒र्तीति॒ तौ ता विति॑ बिभर्ति बिभ॒र्तीति॒ तौ । \newline
36. इति॒ तौ ता वितीति॒ ता व॑ब्रवी दब्रवी॒त् ता वितीति॒ ता व॑ब्रवीत् । \newline
37. ता व॑ब्रवी दब्रवी॒त् तौ ता व॑ब्रवी दि॒य मि॒य म॑ब्रवी॒त् तौ ता व॑ब्रवी दि॒यम् । \newline
38. अ॒ब्र॒वी॒ दि॒य मि॒य म॑ब्रवी दब्रवी दि॒यम् मम॒ ममे॒य म॑ब्रवी दब्रवी दि॒यम् मम॑ । \newline
39. इ॒यम् मम॒ ममे॒य मि॒यम् ममा स्त्वस्तु॒ ममे॒य मि॒यम् ममास्तु॑ । \newline
40. ममा स्त्वस्तु॒ मम॒ ममा स्त्वे॒त दे॒त दस्तु॒ मम॒ ममा स्त्वे॒तत् । \newline
41. अस्त्वे॒त दे॒त दस्त्व स्त्वे॒तद् यु॒वयो᳚र् यु॒वयो॑ रे॒त दस्त्व स्त्वे॒तद् यु॒वयोः᳚ । \newline
42. ए॒तद् यु॒वयो᳚र् यु॒वयो॑ रे॒त दे॒तद् यु॒वयो॒ रितीति॑ यु॒वयो॑ रे॒त दे॒तद् यु॒वयो॒ रिति॑ । \newline
43. यु॒वयो॒ रितीति॑ यु॒वयो᳚र् यु॒वयो॒ रिति॒ तौ ता विति॑ यु॒वयो᳚र् यु॒वयो॒ रिति॒ तौ । \newline
44. इति॒ तौ ता वितीति॒ ता व॑ब्रूता मब्रूता॒म् ता वितीति॒ ता व॑ब्रूताम् । \newline
45. ता व॑ब्रूता मब्रूता॒म् तौ ता व॑ब्रूताꣳ॒॒ सर्वे॒ सर्वे᳚ ऽब्रूता॒म् तौ ता व॑ब्रूताꣳ॒॒ सर्वे᳚ । \newline
46. अ॒ब्रू॒ताꣳ॒॒ सर्वे॒ सर्वे᳚ ऽब्रूता मब्रूताꣳ॒॒ सर्वे॒ वै वै सर्वे᳚ ऽब्रूता मब्रूताꣳ॒॒ सर्वे॒ वै । \newline
47. सर्वे॒ वै वै सर्वे॒ सर्वे॒ वा ए॒त दे॒तद् वै सर्वे॒ सर्वे॒ वा ए॒तत् । \newline
48. वा ए॒त दे॒तद् वै वा ए॒त दे॒तस्या॑ मे॒तस्या॑ मे॒तद् वै वा ए॒त दे॒तस्या᳚म् । \newline
49. ए॒त दे॒तस्या॑ मे॒तस्या॑ मे॒त दे॒त दे॒तस्यां᳚ ॅवी॒र्यं॑ ॅवी॒र्य॑ मे॒तस्या॑ मे॒त दे॒त दे॒तस्यां᳚ ॅवी॒र्य᳚म् । \newline
50. ए॒तस्यां᳚ ॅवी॒र्यं॑ ॅवी॒र्य॑ मे॒तस्या॑ मे॒तस्यां᳚ ॅवी॒र्य॑म् परि॒ परि॑ वी॒र्य॑ मे॒तस्या॑ मे॒तस्यां᳚ ॅवी॒र्य॑म् परि॑ । \newline
51. वी॒र्य॑म् परि॒ परि॑ वी॒र्यं॑ ॅवी॒र्य॑म् परि॑ पश्यामः पश्यामः॒ परि॑ वी॒र्यं॑ ॅवी॒र्य॑म् परि॑ पश्यामः । \newline
\pagebreak
\markright{ TS 7.1.6.2  \hfill https://www.vedavms.in \hfill}

\section{ TS 7.1.6.2 }

\textbf{TS 7.1.6.2 } \newline
\textbf{Samhita Paata} \newline

परि॑ पश्या॒मोऽꣳश॒मा ह॑रामहा॒ इति॒ तस्या॒मꣳश॒माऽह॑रन्त॒ ताम॒फ्सु प्राऽवे॑शय॒न्थ् सोमा॑यो॒देहीति॒ सा रोहि॑णी पिङ्ग॒लैक॑हायनी रू॒पं कृ॒त्वा त्रय॑स्त्रिꣳशता च त्रि॒भिश्च॑ श॒तैः स॒होदैत् तस्मा॒द्-रोहि॑ण्या पिङ्ग॒लयैक॑हायन्या॒ सोमं॑ क्रीणीया॒द्य ए॒वं ॅवि॒द्वान् रोहि॑ण्या पिङ्ग॒लयैक॑हायन्या॒ सोमं॑ क्री॒णाति॒ त्रय॑स्त्रिꣳशता चै॒वास्य॑ त्रि॒भिश्च॑ - [  ] \newline

\textbf{Pada Paata} \newline

परीति॑ । प॒श्या॒मः॒ । अꣳश᳚म् । एति॑ । ह॒रा॒म॒है॒ । इति॑ । तस्या᳚म् । अꣳश᳚म् ।  एति॑ । अ॒ह॒र॒न्त॒ । ताम् । अ॒फ्स्वित्य॑प् - सु । प्रेति॑ । अ॒वे॒श॒य॒न्न् । सोमा॑य । उ॒देहीत्यु॑त् - एहि॑ । इति॑ । सा । रोहि॑णी । पि॒ङ्ग॒ला । एक॑हाय॒नीत्येक॑ - हा॒य॒नी॒ । रू॒पम् । कृ॒त्वा । त्रय॑स्त्रिꣳश॒तेति॒ त्रयः॑ - त्रिꣳ॒॒श॒ता॒ । च॒ । त्रि॒भिरिति॑ त्रि - भिः । च॒ । श॒तैः । स॒ह । उ॒दैदित्यु॑त् - ऐत् । तस्मा᳚त् । रोहि॑ण्या । पि॒ङ्ग॒लया᳚ । एक॑हाय॒न्येत्येक॑ - हा॒य॒न्या॒ । सोम᳚म् । क्री॒णी॒या॒त् । यः । ए॒वम् । वि॒द्वान् । रोहि॑ण्या । पि॒ङ्ग॒लया᳚ । एक॑हाय॒न्येत्येक॑ - हा॒य॒न्या॒ । सोम᳚म् । क्री॒णाति॑ । त्रय॑स्त्रिꣳश॒तेति॒ त्रयः॑ - त्रिꣳ॒॒श॒ता॒ । च॒ । ए॒व । अ॒स्य॒ । त्रि॒भिरिति॑ त्रि - भिः । च॒ ।  \newline


\textbf{Krama Paata} \newline

परि॑ पश्यामः । प॒श्या॒मोऽꣳश᳚म् । अꣳश॒मा । आ ह॑रामहै । ह॒रा॒म॒हा॒ इति॑ । इति॒ तस्या᳚म् । तस्या॒मꣳश᳚म् । अꣳश॒मा । आऽह॑रन्त । अ॒ह॒र॒न्त॒ ताम् । ताम॒फ्सु । अ॒फ्सु प्र । अ॒फ्स्वित्य॑प् - सु । प्रावे॑शयन्न् । अ॒वे॒श॒य॒न्थ् सोमा॑य । सोमा॑यो॒देहि॑ । उ॒देहीति॑ । उ॒देहीत्यु॑त् - एहि॑ । इति॒ सा । सा रोहि॑णी । रोहि॑णी पिङ्‍ग॒ला । पि॒ङ्‍ग॒लैक॑हायनी । एक॑हायनी रू॒पम् । एक॑हाय॒नीत्येक॑ - हा॒य॒नी॒ । रू॒पम् कृ॒त्वा । कृ॒त्वा त्रय॑स्त्रिꣳशता । त्रय॑स्त्रिꣳशता च । त्रय॑स्त्रिꣳश॒तेति॒ त्रयः॑ - त्रिꣳ॒॒श॒ता॒ । च॒ त्रि॒भिः । त्रि॒भिश्च॑ । त्रि॒भिरिति॑ त्रि - भिः । च॒ श॒तैः । श॒तैः स॒ह । स॒होदैत् । उ॒दैत् तस्मा᳚त् । उ॒दैदित्यु॑त् - ऐत् । तस्मा॒द् रोहि॑ण्या । रोहि॑ण्या पिङ्‍ग॒लया᳚ । पि॒ङ्‍ग॒लयैक॑हायन्या । एक॑हायन्या॒ सोमम्᳚ । एक॑हाय॒न्येत्येक॑ - हा॒य॒न्या॒ । सोम॑म् क्रीणीयात् । क्री॒णी॒या॒द् यः । य ए॒वम् । ए॒वम् ॅवि॒द्वान् । वि॒द्वान् रोहि॑ण्या । रोहि॑ण्या पिङ्‍ग॒लया᳚ । पि॒ङ्‍ग॒लयैक॑हायन्या । एक॑हायन्या॒ सोम᳚म् । एक॑हाय॒न्येत्येक॑ - हा॒य॒न्या॒ । सोम॑म् क्री॒णाति॑ । क्री॒णाति॒ त्रय॑स्त्रिꣳशता । त्रय॑स्त्रिꣳशता च । त्रय॑स्त्रिꣳश॒तेति॒ त्रयः॑ - त्रिꣳ॒॒श॒ता॒ । चै॒व । ए॒वास्य॑ । अ॒स्य॒ त्रि॒भिः । त्रि॒भिश्च॑ । त्रि॒भिरिति॑ त्रि - भिः । च॒ श॒तैः \newline

\textbf{Jatai Paata} \newline

1. परि॑ पश्यामः पश्यामः॒ परि॒ परि॑ पश्यामः । \newline
2. प॒श्या॒मो॒ ऽꣳश॒ मꣳश॑म् पश्यामः पश्या॒मो ऽꣳश᳚म् । \newline
3. अꣳश॒ मा ऽꣳश॒ मꣳश॒ मा । \newline
4. आ ह॑रामहै हरामहा॒ आ ह॑रामहै । \newline
5. ह॒रा॒म॒हा॒ इतीति॑ हरामहै हरामहा॒ इति॑ । \newline
6. इति॒ तस्या॒म् तस्या॒ मितीति॒ तस्या᳚म् । \newline
7. तस्या॒ मꣳश॒ मꣳश॒म् तस्या॒म् तस्या॒ मꣳश᳚म् । \newline
8. अꣳश॒ मा ऽꣳश॒ मꣳश॒ मा । \newline
9. आ ऽह॑रन्ता हर॒न्ता ऽह॑रन्त । \newline
10. अ॒ह॒र॒न्त॒ ताम् ता म॑हरन्ता हरन्त॒ ताम् । \newline
11. ता म॒फ्स्व॑फ्सु ताम् ता म॒फ्सु । \newline
12. अ॒फ्सु प्र प्राफ्स्व॑फ्सु प्र । \newline
13. अ॒फ्स्वित्य॑प् - सु । \newline
14. प्रा वे॑शयन् नवेशय॒न् प्र प्रा वे॑शयन्न् । \newline
15. अ॒वे॒श॒य॒न् थ्सोमा॑य॒ सोमा॑या वेशयन् नवेशय॒न् थ्सोमा॑य । \newline
16. सोमा॑ यो॒देह्यु॒ देहि॒ सोमा॑य॒ सोमा॑ यो॒देहि॑ । \newline
17. उ॒देहीती त्यु॒दे ह्यु॒देहीति॑ । \newline
18. उ॒देहीत्यु॑त् - एहि॑ । \newline
19. इति॒ सा सेतीति॒ सा । \newline
20. सा रोहि॑णी॒ रोहि॑णी॒ सा सा रोहि॑णी । \newline
21. रोहि॑णी पिङ्ग॒ला पि॑ङ्ग॒ला रोहि॑णी॒ रोहि॑णी पिङ्ग॒ला । \newline
22. पि॒ङ्ग॒ लैक॑हाय॒ न्येक॑हायनी पिङ्ग॒ला पि॑ङ्ग॒ लैक॑हायनी । \newline
23. एक॑हायनी रू॒पꣳ रू॒प मेक॑हाय॒ न्येक॑हायनी रू॒पम् । \newline
24. एक॑हाय॒नीत्येक॑ - हा॒य॒नी॒ । \newline
25. रू॒पम् कृ॒त्वा कृ॒त्वा रू॒पꣳ रू॒पम् कृ॒त्वा । \newline
26. कृ॒त्वा त्रय॑स्त्रिꣳशता॒ त्रय॑स्त्रिꣳशता कृ॒त्वा कृ॒त्वा त्रय॑स्त्रिꣳशता । \newline
27. त्रय॑स्त्रिꣳशता च च॒ त्रय॑स्त्रिꣳशता॒ त्रय॑स्त्रिꣳशता च । \newline
28. त्रय॑स्त्रिꣳश॒तेति॒ त्रयः॑ - त्रिꣳ॒॒श॒ता॒ । \newline
29. च॒ त्रि॒भि स्त्रि॒भि श्च॑ च त्रि॒भिः । \newline
30. त्रि॒भि श्च॑ च त्रि॒भि स्त्रि॒भि श्च॑ । \newline
31. त्रि॒भिरिति॑ त्रि - भिः । \newline
32. च॒ श॒तैः श॒तै श्च॑ च श॒तैः । \newline
33. श॒तैः स॒ह स॒ह श॒तैः श॒तैः स॒ह । \newline
34. स॒होदै दु॒दैथ् स॒ह स॒होदैत् । \newline
35. उ॒दैत् तस्मा॒त् तस्मा॑ दु॒दै दु॒दैत् तस्मा᳚त् । \newline
36. उ॒दैदित्यु॑त् - ऐत् । \newline
37. तस्मा॒द् रोहि॑ण्या॒ रोहि॑ण्या॒ तस्मा॒त् तस्मा॒द् रोहि॑ण्या । \newline
38. रोहि॑ण्या पिङ्ग॒लया॑ पिङ्ग॒लया॒ रोहि॑ण्या॒ रोहि॑ण्या पिङ्ग॒लया᳚ । \newline
39. पि॒ङ्ग॒ल यैक॑हाय॒ न्यैक॑हायन्या पिङ्ग॒लया॑ पिङ्ग॒ल यैक॑हायन्या । \newline
40. एक॑हायन्या॒ सोमꣳ॒॒ सोम॒ मेक॑हाय॒ न्यैक॑हायन्या॒ सोम᳚म् । \newline
41. एक॑हाय॒न्येत्येक॑ - हा॒य॒न्या॒ । \newline
42. सोम॑म् क्रीणीयात् क्रीणीया॒थ् सोमꣳ॒॒ सोम॑म् क्रीणीयात् । \newline
43. क्री॒णी॒या॒द् यो यः क्री॑णीयात् क्रीणीया॒द् यः । \newline
44. य ए॒व मे॒वं ॅयो य ए॒वम् । \newline
45. ए॒वं ॅवि॒द्वान्. वि॒द्वा ने॒व मे॒वं ॅवि॒द्वान् । \newline
46. वि॒द्वान् रोहि॑ण्या॒ रोहि॑ण्या वि॒द्वान्. वि॒द्वान् रोहि॑ण्या । \newline
47. रोहि॑ण्या पिङ्ग॒लया॑ पिङ्ग॒लया॒ रोहि॑ण्या॒ रोहि॑ण्या पिङ्ग॒लया᳚ । \newline
48. पि॒ङ्ग॒ल यैक॑हाय॒ न्यैक॑हायन्या पिङ्ग॒लया॑ पिङ्ग॒ल यैक॑हायन्या । \newline
49. एक॑हायन्या॒ सोमꣳ॒॒ सोम॒ मेक॑हाय॒ न्यैक॑हायन्या॒ सोम᳚म् । \newline
50. एक॑हाय॒न्येत्येक॑ - हा॒य॒न्या॒ । \newline
51. सोम॑म् क्री॒णाति॑ क्री॒णाति॒ सोमꣳ॒॒ सोम॑म् क्री॒णाति॑ । \newline
52. क्री॒णाति॒ त्रय॑स्त्रिꣳशता॒ त्रय॑स्त्रिꣳशता क्री॒णाति॑ क्री॒णाति॒ त्रय॑स्त्रिꣳशता । \newline
53. त्रय॑स्त्रिꣳशता च च॒ त्रय॑स्त्रिꣳशता॒ त्रय॑स्त्रिꣳशता च । \newline
54. त्रय॑स्त्रिꣳश॒तेति॒ त्रयः॑ - त्रिꣳ॒॒श॒ता॒ । \newline
55. चै॒वैव च॑ चै॒व । \newline
56. ए॒वास्या᳚ स्यै॒वै वास्य॑ । \newline
57. अ॒स्य॒ त्रि॒भि स्त्रि॒भि र॑स्यास्य त्रि॒भिः । \newline
58. त्रि॒भि श्च॑ च त्रि॒भि स्त्रि॒भि श्च॑ । \newline
59. त्रि॒भिरिति॑ त्रि - भिः । \newline
60. च॒ श॒तैः श॒तै श्च॑ च श॒तैः । \newline

\textbf{Ghana Paata } \newline

1. परि॑ पश्यामः पश्यामः॒ परि॒ परि॑ पश्या॒मो ऽꣳश॒ मꣳश॑म् पश्यामः॒ परि॒ परि॑ पश्या॒मो ऽꣳश᳚म् । \newline
2. प॒श्या॒मो॒ ऽꣳश॒ मꣳश॑म् पश्यामः पश्या॒मो ऽꣳश॒ मा ऽꣳश॑म् पश्यामः पश्या॒मो ऽꣳश॒ मा । \newline
3. अꣳश॒ मा ऽꣳश॒ मꣳश॒ मा ह॑रामहै हरामहा॒ आ ऽꣳश॒ मꣳश॒ मा ह॑रामहै । \newline
4. आ ह॑रामहै हरामहा॒ आ ह॑रामहा॒ इतीति॑ हरामहा॒ आ ह॑रामहा॒ इति॑ । \newline
5. ह॒रा॒म॒हा॒ इतीति॑ हरामहै हरामहा॒ इति॒ तस्या॒म् तस्या॒ मिति॑ हरामहै हरामहा॒ इति॒ तस्या᳚म् । \newline
6. इति॒ तस्या॒म् तस्या॒ मितीति॒ तस्या॒ मꣳश॒ मꣳश॒म् तस्या॒ मितीति॒ तस्या॒ मꣳश᳚म् । \newline
7. तस्या॒ मꣳश॒ मꣳश॒म् तस्या॒म् तस्या॒ मꣳश॒ मा ऽꣳश॒म् तस्या॒म् तस्या॒ मꣳश॒ मा । \newline
8. अꣳश॒ मा ऽꣳश॒ मꣳश॒ मा ऽह॑रन्ता हर॒न्ता ऽꣳश॒ मꣳश॒ मा ऽह॑रन्त । \newline
9. आ ऽह॑रन्ता हर॒न्ता ऽह॑रन्त॒ ताम् ता म॑हर॒न्ता ऽह॑रन्त॒ ताम् । \newline
10. अ॒ह॒र॒न्त॒ ताम् ता म॑हरन्ता हरन्त॒ ता म॒फ्स्व॑फ्सु ता म॑हरन्ता हरन्त॒ ता म॒फ्सु । \newline
11. ता म॒फ्स्व॑फ्सु ताम् ता म॒फ्सु प्र प्राफ्सु ताम् ता म॒फ्सु प्र । \newline
12. अ॒फ्सु प्र प्राफ्स्व॑फ्सु प्रावे॑शयन् नवेशय॒न् प्राफ्स्व॑फ्सु प्रावे॑शयन्न् । \newline
13. अ॒फ्स्वित्य॑प् - सु । \newline
14. प्रावे॑शयन् नवेशय॒न् प्र प्रावे॑शय॒न् थ्सोमा॑य॒ सोमा॑या वेशय॒न् प्र प्रावे॑शय॒न् थ्सोमा॑य । \newline
15. अ॒वे॒श॒य॒न् थ्सोमा॑य॒ सोमा॑या वेशयन् नवेशय॒न् थ्सोमा॑यो॒ देह्यु॒ देहि॒ सोमा॑या वेशयन् नवेशय॒न् थ्सोमा॑यो॒देहि॑ । \newline
16. सोमा॑यो॒ देह्यु॒ देहि॒ सोमा॑य॒ सोमा॑यो॒ देहीती त्यु॒देहि॒ सोमा॑य॒ सोमा॑यो॒ देहीति॑ । \newline
17. उ॒देहीती त्यु॒दे ह्यु॒देहीति॒ सा सेत्यु॒दे ह्यु॒देहीति॒ सा । \newline
18. उ॒देहीत्यु॑त् - एहि॑ । \newline
19. इति॒ सा सेतीति॒ सा रोहि॑णी॒ रोहि॑णी॒ सेतीति॒ सा रोहि॑णी । \newline
20. सा रोहि॑णी॒ रोहि॑णी॒ सा सा रोहि॑णी पिङ्ग॒ला पि॑ङ्ग॒ला रोहि॑णी॒ सा सा रोहि॑णी पिङ्ग॒ला । \newline
21. रोहि॑णी पिङ्ग॒ला पि॑ङ्ग॒ला रोहि॑णी॒ रोहि॑णी पिङ्ग॒ लैक॑हाय॒ न्येक॑हायनी पिङ्ग॒ला रोहि॑णी॒ रोहि॑णी पिङ्ग॒
लैक॑हायनी । \newline
22. पि॒ङ्ग॒ लैक॑हाय॒ न्येक॑हायनी पिङ्ग॒ला पि॑ङ्ग॒ लैक॑हायनी रू॒पꣳ रू॒प मेक॑हायनी पिङ्ग॒ला पि॑ङ्ग॒
लैक॑हायनी रू॒पम् । \newline
23. एक॑हायनी रू॒पꣳ रू॒प मेक॑हाय॒ न्येक॑हायनी रू॒पम् कृ॒त्वा कृ॒त्वा रू॒प मेक॑हाय॒ न्येक॑हायनी रू॒पम् कृ॒त्वा । \newline
24. एक॑हाय॒नीत्येक॑ - हा॒य॒नी॒ । \newline
25. रू॒पम् कृ॒त्वा कृ॒त्वा रू॒पꣳ रू॒पम् कृ॒त्वा त्रय॑स्त्रिꣳशता॒ त्रय॑स्त्रिꣳशता कृ॒त्वा रू॒पꣳ रू॒पम् कृ॒त्वा त्रय॑स्त्रिꣳशता । \newline
26. कृ॒त्वा त्रय॑स्त्रिꣳशता॒ त्रय॑स्त्रिꣳशता कृ॒त्वा कृ॒त्वा त्रय॑स्त्रिꣳशता च च॒ त्रय॑स्त्रिꣳशता कृ॒त्वा कृ॒त्वा त्रय॑स्त्रिꣳशता च । \newline
27. त्रय॑स्त्रिꣳशता च च॒ त्रय॑स्त्रिꣳशता॒ त्रय॑स्त्रिꣳशता च त्रि॒भि स्त्रि॒भि श्च॒ त्रय॑स्त्रिꣳशता॒ त्रय॑स्त्रिꣳशता च त्रि॒भिः । \newline
28. त्रय॑स्त्रिꣳश॒तेति॒ त्रयः॑ - त्रिꣳ॒॒श॒ता॒ । \newline
29. च॒ त्रि॒भि स्त्रि॒भि श्च॑ च त्रि॒भि श्च॑ च त्रि॒भिश्च॑ च त्रि॒भि श्च॑ । \newline
30. त्रि॒भि श्च॑ च त्रि॒भि स्त्रि॒भि श्च॑ श॒तैः श॒तै श्च॑ त्रि॒भि स्त्रि॒भि श्च॑ श॒तैः । \newline
31. त्रि॒भिरिति॑ त्रि - भिः । \newline
32. च॒ श॒तैः श॒तै श्च॑ च श॒तैः स॒ह स॒ह श॒तै श्च॑ च श॒तैः स॒ह । \newline
33. श॒तैः स॒ह स॒ह श॒तैः श॒तैः स॒होदै दु॒दैथ् स॒ह श॒तैः श॒तैः स॒होदैत् । \newline
34. स॒होदै दु॒दैथ् स॒ह स॒होदैत् तस्मा॒त् तस्मा॑ दु॒दैथ् स॒ह स॒होदैत् तस्मा᳚त् । \newline
35. उ॒दैत् तस्मा॒त् तस्मा॑ दु॒दै दु॒दैत् तस्मा॒द् रोहि॑ण्या॒ रोहि॑ण्या॒ तस्मा॑ दु॒दै दु॒दैत् तस्मा॒द् रोहि॑ण्या । \newline
36. उ॒दैदित्यु॑त् - ऐत् । \newline
37. तस्मा॒द् रोहि॑ण्या॒ रोहि॑ण्या॒ तस्मा॒त् तस्मा॒द् रोहि॑ण्या पिङ्ग॒लया॑ पिङ्ग॒लया॒ रोहि॑ण्या॒ तस्मा॒त् तस्मा॒द् रोहि॑ण्या पिङ्ग॒लया᳚ । \newline
38. रोहि॑ण्या पिङ्ग॒लया॑ पिङ्ग॒लया॒ रोहि॑ण्या॒ रोहि॑ण्या पिङ्ग॒ल यैक॑हाय॒ न्यैक॑हायन्या पिङ्ग॒लया॒ रोहि॑ण्या॒ रोहि॑ण्या पिङ्ग॒ल यैक॑हायन्या । \newline
39. पि॒ङ्ग॒ल यैक॑हाय॒ न्यैक॑हायन्या पिङ्ग॒लया॑ पिङ्ग॒ल यैक॑हायन्या॒ सोमꣳ॒॒ सोम॒ मेक॑हायन्या पिङ्ग॒लया॑ पिङ्ग॒ल यैक॑हायन्या॒ सोम᳚म् । \newline
40. एक॑हायन्या॒ सोमꣳ॒॒ सोम॒ मेक॑हाय॒ न्यैक॑हायन्या॒ सोम॑म् क्रीणीयात् क्रीणीया॒थ् सोम॒ मेक॑हाय॒
न्यैक॑हायन्या॒ सोम॑म् क्रीणीयात् । \newline
41. एक॑हाय॒न्येत्येक॑ - हा॒य॒न्या॒ । \newline
42. सोम॑म् क्रीणीयात् क्रीणीया॒थ् सोमꣳ॒॒ सोम॑म् क्रीणीया॒द् यो यः क्री॑णीया॒थ् सोमꣳ॒॒ सोम॑म् क्रीणीया॒द् यः । \newline
43. क्री॒णी॒या॒द् यो यः क्री॑णीयात् क्रीणीया॒द् य ए॒व मे॒वं ॅयः क्री॑णीयात् क्रीणीया॒द् य ए॒वम् । \newline
44. य ए॒व मे॒वं ॅयो य ए॒वं ॅवि॒द्वान्. वि॒द्वा ने॒वं ॅयो य ए॒वं ॅवि॒द्वान् । \newline
45. ए॒वं ॅवि॒द्वान्. वि॒द्वा ने॒व मे॒वं ॅवि॒द्वान् रोहि॑ण्या॒ रोहि॑ण्या वि॒द्वा ने॒व मे॒वं ॅवि॒द्वान् रोहि॑ण्या । \newline
46. वि॒द्वान् रोहि॑ण्या॒ रोहि॑ण्या वि॒द्वान्. वि॒द्वान् रोहि॑ण्या पिङ्ग॒लया॑ पिङ्ग॒लया॒ रोहि॑ण्या वि॒द्वान्. वि॒द्वान् रोहि॑ण्या पिङ्ग॒लया᳚ । \newline
47. रोहि॑ण्या पिङ्ग॒लया॑ पिङ्ग॒लया॒ रोहि॑ण्या॒ रोहि॑ण्या पिङ्ग॒ल यैक॑हाय॒ न्यैक॑हायन्या पिङ्ग॒लया॒ रोहि॑ण्या॒ रोहि॑ण्या पिङ्ग॒ल यैक॑हायन्या । \newline
48. पि॒ङ्ग॒ल यैक॑हाय॒ न्यैक॑हायन्या पिङ्ग॒लया॑ पिङ्ग॒ल यैक॑हायन्या॒ सोमꣳ॒॒ सोम॒ मेक॑हायन्या पिङ्ग॒लया॑ पिङ्ग॒ल यैक॑हायन्या॒ सोम᳚म् । \newline
49. एक॑हायन्या॒ सोमꣳ॒॒ सोम॒ मेक॑हाय॒ न्यैक॑हायन्या॒ सोम॑म् क्री॒णाति॑ क्री॒णाति॒ सोम॒ मेक॑हाय॒
न्यैक॑हायन्या॒ सोम॑म् क्री॒णाति॑ । \newline
50. एक॑हाय॒न्येत्येक॑ - हा॒य॒न्या॒ । \newline
51. सोम॑म् क्री॒णाति॑ क्री॒णाति॒ सोमꣳ॒॒ सोम॑म् क्री॒णाति॒ त्रय॑स्त्रिꣳशता॒ त्रय॑स्त्रिꣳशता क्री॒णाति॒ सोमꣳ॒॒ सोम॑म् क्री॒णाति॒ त्रय॑स्त्रिꣳशता । \newline
52. क्री॒णाति॒ त्रय॑स्त्रिꣳशता॒ त्रय॑स्त्रिꣳशता क्री॒णाति॑ क्री॒णाति॒ त्रय॑स्त्रिꣳशता च च॒ त्रय॑स्त्रिꣳशता क्री॒णाति॑ क्री॒णाति॒ त्रय॑स्त्रिꣳशता च । \newline
53. त्रय॑स्त्रिꣳशता च च॒ त्रय॑स्त्रिꣳशता॒ त्रय॑स्त्रिꣳशता चै॒वैव च॒ त्रय॑स्त्रिꣳशता॒ त्रय॑स्त्रिꣳशता चै॒व । \newline
54. त्रय॑स्त्रिꣳश॒तेति॒ त्रयः॑ - त्रिꣳ॒॒श॒ता॒ । \newline
55. चै॒वैव च॑ चै॒वास्या᳚ स्यै॒व च॑ चै॒वास्य॑ । \newline
56. ए॒वास्या᳚ स्यै॒वै वास्य॑ त्रि॒भि स्त्रि॒भि र॑स्यै॒वै वास्य॑ त्रि॒भिः । \newline
57. अ॒स्य॒ त्रि॒भि स्त्रि॒भि र॑स्यास्य त्रि॒भि श्च॑ च त्रि॒भि र॑स्यास्य त्रि॒भि श्च॑ । \newline
58. त्रि॒भि श्च॑ च त्रि॒भि स्त्रि॒भि श्च॑ श॒तैः श॒तै श्च॑ त्रि॒भि स्त्रि॒भि श्च॑ श॒तैः । \newline
59. त्रि॒भिरिति॑ त्रि - भिः । \newline
60. च॒ श॒तैः श॒तै श्च॑ च श॒तैः सोमः॒ सोमः॑ श॒तै श्च॑ च श॒तैः सोमः॑ । \newline
\pagebreak
\markright{ TS 7.1.6.3  \hfill https://www.vedavms.in \hfill}

\section{ TS 7.1.6.3 }

\textbf{TS 7.1.6.3 } \newline
\textbf{Samhita Paata} \newline

श॒तैः सोमः॑ क्री॒तो भ॑वति॒ सुक्री॑तेन यजते॒ ताम॒फ्सु प्रावे॑शय॒-न्निन्द्रा॑यो॒देहीति॒ सा रोहि॑णी लक्ष्म॒णा प॑ष्ठौ॒ही वार्त्र॑घ्नी रू॒पं कृ॒त्वा त्रय॑स्त्रिꣳशता च त्रि॒भिश्च॑ श॒तैः स॒होदैत् तस्मा॒द् रोहि॑णीं ॅलक्ष्म॒णां प॑ष्ठौ॒हीं ॅवार्त्र॑घ्नीं दद्या॒द्य ए॒वं ॅवि॒द्वान् रोहि॑णीं ॅलक्ष्म॒णां प॑ष्ठौ॒हीं ॅवार्त्र॑घ्नीं॒ ददा॑ति॒ त्रय॑स्त्रिꣳशच्चै॒वास्य॒ त्रीणि॑ च श॒तानि॒ सा द॒त्ता - [  ] \newline

\textbf{Pada Paata} \newline

श॒तैः । सोमः॑ । क्री॒तः । भ॒व॒ति॒ । सुक्री॑ते॒नेति॒ सु-क्री॒ते॒न॒ । य॒ज॒ते॒ । ताम् । अ॒फ्स्वित्य॑प् - सु । प्रेति॑ । अ॒वे॒श॒य॒न्न् । इन्द्रा॑य । उ॒देहीत्यु॑त् - एहि॑ । इति॑ । सा । रोहि॑णी । ल॒क्ष्म॒णा । प॒ष्ठौ॒ही । वार्त्र॒घ्नीति॒ वार्त्र॑ - घ्नी॒ । रू॒पम् । कृ॒त्वा । त्रय॑स्त्रिꣳश॒तेति॒ त्रयः॑ - त्रिꣳ॒॒श॒ता॒ । च॒ । त्रि॒भिरिति॑ त्रि - भिः । च॒ । श॒तैः । स॒ह । उ॒दैदित्यु॑त् - ऐत् । तस्मा᳚त् । रोहि॑णीम् । ल॒क्ष्म॒णाम् । प॒ष्ठौ॒हीम् । वार्त्र॑घ्नी॒मिति॒ वार्त्र॑ - घ्नी॒म् । द॒द्या॒त् । यः । ए॒वम् । वि॒द्वान् । रोहि॑णीम् । ल॒क्ष्म॒णाम् । प॒ष्ठौ॒हीम् । वार्त्र॑घ्नी॒मिति॒ वार्त्र॑ - घ्नी॒म् । ददा॑ति । त्रय॑स्त्रिꣳश॒दिति॒ त्रयः॑ - त्रिꣳ॒॒श॒त् । च॒ । ए॒व॒ । अ॒स्य॒ । त्रीणि॑ । च॒ । श॒तानि॑ । सा । द॒त्ता ।  \newline


\textbf{Krama Paata} \newline

श॒तैः सोमः॑ । सोमः॑ क्री॒तः । क्री॒तो भ॑वति । भ॒व॒ति॒ सुक्री॑तेन । सुक्री॑तेन यजते । सुक्री॑ते॒नेति॒ सु - क्री॒ते॒न॒ । य॒ज॒ते॒ ताम् । ताम॒फ्सु । अ॒फ्सु प्र । अ॒फ्स्वित्य॑प् - सु । प्रावे॑शयन्न् । अ॒वे॒श॒य॒न्निन्द्रा॑य । इन्द्रा॑यो॒देहि॑ । उ॒देहीति॑ । उ॒देहीत्यु॑त् - एहि॑ । इति॒ सा । सा रोहि॑णी । रोहि॑णी लक्ष्म॒णा । ल॒क्ष्म॒णा प॑ष्ठौ॒ही । प॒ष्ठौ॒ही वार्त्र॑घ्नी । वार्त्र॑घ्नी रू॒पम् । वार्त्र॒घ्नीति॒ वार्त्र॑ - घ्नी॒ । रू॒पम् कृ॒त्वा । कृ॒त्वा त्रय॑स्त्रिꣳशता । त्रय॑स्त्रिꣳशता च । त्रय॑स्त्रिꣳश॒तेति॒ त्रयः॑ - त्रिꣳ॒॒श॒ता॒ । च॒ त्रि॒भिः । त्रि॒भिश्च॑ । त्रि॒भिरिति॑ त्रि - भिः । च॒ श॒तैः । श॒तैः स॒ह । स॒होदैत् । उ॒दैत् तस्मा᳚त् । उ॒दैदित्यु॑त् - ऐत् । तस्मा॒द् रोहि॑णीम् । रोहि॑णीम् ॅलक्ष्म॒णाम् । ल॒क्ष्म॒णाम् प॑ष्ठौ॒हीम् । प॒ष्ठौ॒हीम् ॅवार्त्र॑घ्नीम् । वार्त्र॑घ्नीम् दद्यात् । वार्त्र॑घ्नी॒मिति॒ वार्त्र॑ - घ्नी॒म् । द॒द्या॒द् यः । य ए॒वम् । ए॒वम् ॅवि॒द्वान् । वि॒द्वान् रोहि॑णीम् । रोहि॑णीम् ॅलक्ष्म॒णाम् । ल॒क्ष्म॒णाम् प॑ष्ठौ॒हीम् । प॒ष्ठौ॒हीम् ॅवार्त्र॑घ्नीम् । वार्त्र॑घ्नी॒म् ददा॑ति । वार्त्र॑घ्नी॒मिति॒ वार्त्र॑ - घ्नी॒म् । ददा॑ति॒ त्रय॑स्त्रिꣳशत् । त्रय॑स्त्रिꣳशच्च । त्रय॑स्त्रिꣳश॒दिति॒ त्रयः॑ - त्रिꣳ॒॒श॒त्॒ । चै॒व । ए॒वास्य॑ । अ॒स्य॒ त्रीणि॑ । त्रीणि॑ च । च॒ श॒तानि॑ । श॒तानि॒ सा । सा द॒त्ता । द॒त्ता भ॑वति \newline

\textbf{Jatai Paata} \newline

1. श॒तैः सोमः॒ सोमः॑ श॒तैः श॒तैः सोमः॑ । \newline
2. सोमः॑ क्री॒तः क्री॒तः सोमः॒ सोमः॑ क्री॒तः । \newline
3. क्री॒तो भ॑वति भवति क्री॒तः क्री॒तो भ॑वति । \newline
4. भ॒व॒ति॒ सुक्री॑तेन॒ सुक्री॑तेन भवति भवति॒ सुक्री॑तेन । \newline
5. सुक्री॑तेन यजते यजते॒ सुक्री॑तेन॒ सुक्री॑तेन यजते । \newline
6. सुक्री॑ते॒नेति॒ सु - क्री॒ते॒न॒ । \newline
7. य॒ज॒ते॒ ताम् तां ॅय॑जते यजते॒ ताम् । \newline
8. ता म॒फ्स्व॑फ्सु ताम् ता म॒फ्सु । \newline
9. अ॒फ्सु प्र प्राफ्स्व॑फ्सु प्र । \newline
10. अ॒फ्स्वित्य॑प् - सु । \newline
11. प्रावे॑शयन् नवेशय॒न् प्र प्रा वे॑शयन्न् । \newline
12. अ॒वे॒श॒य॒न् निन्द्रा॒ येन्द्रा॑या वेशयन् नवेशय॒न् निन्द्रा॑य । \newline
13. इन्द्रा॑ यो॒देह्यु॒ देहीन्द्रा॒ येन्द्रा॑ यो॒देहि॑ । \newline
14. उ॒देही तीत्यु॒दे ह्यु॒देहीति॑ । \newline
15. उ॒देहीत्यु॑त् - एहि॑ । \newline
16. इति॒ सा सेतीति॒ सा । \newline
17. सा रोहि॑णी॒ रोहि॑णी॒ सा सा रोहि॑णी । \newline
18. रोहि॑णी लक्ष्म॒णा ल॑क्ष्म॒णा रोहि॑णी॒ रोहि॑णी लक्ष्म॒णा । \newline
19. ल॒क्ष्म॒णा प॑ष्ठौ॒ही प॑ष्ठौ॒ही ल॑क्ष्म॒णा ल॑क्ष्म॒णा प॑ष्ठौ॒ही । \newline
20. प॒ष्ठौ॒ही वार्त्र॑घ्नी॒ वार्त्र॑घ्नी पष्ठौ॒ही प॑ष्ठौ॒ही वार्त्र॑घ्नी । \newline
21. वार्त्र॑घ्नी रू॒पꣳ रू॒पं ॅवार्त्र॑घ्नी॒ वार्त्र॑घ्नी रू॒पम् । \newline
22. वार्त्र॒घ्नीति॒ वार्त्र॑ - घ्नी॒ । \newline
23. रू॒पम् कृ॒त्वा कृ॒त्वा रू॒पꣳ रू॒पम् कृ॒त्वा । \newline
24. कृ॒त्वा त्रय॑स्त्रिꣳशता॒ त्रय॑स्त्रिꣳशता कृ॒त्वा कृ॒त्वा त्रय॑स्त्रिꣳशता । \newline
25. त्रय॑स्त्रिꣳशता च च॒ त्रय॑स्त्रिꣳशता॒ त्रय॑स्त्रिꣳशता च । \newline
26. त्रय॑स्त्रिꣳश॒तेति॒ त्रयः॑ - त्रिꣳ॒॒श॒ता॒ । \newline
27. च॒ त्रि॒भि स्त्रि॒भिश्च॑ च त्रि॒भिः । \newline
28. त्रि॒भि श्च॑ च त्रि॒भि स्त्रि॒भि श्च॑ । \newline
29. त्रि॒भिरिति॑ त्रि - भिः । \newline
30. च॒ श॒तैः श॒तै श्च॑ च श॒तैः । \newline
31. श॒तैः स॒ह स॒ह श॒तैः श॒तैः स॒ह । \newline
32. स॒होदै दु॒दैथ् स॒ह स॒होदैत् । \newline
33. उ॒दैत् तस्मा॒त् तस्मा॑ दु॒दै दु॒दैत् तस्मा᳚त् । \newline
34. उ॒दैदित्यु॑त् - ऐत् । \newline
35. तस्मा॒द् रोहि॑णीꣳ॒॒ रोहि॑णी॒म् तस्मा॒त् तस्मा॒द् रोहि॑णीम् । \newline
36. रोहि॑णीम् ॅलक्ष्म॒णाम् ॅल॑क्ष्म॒णाꣳ रोहि॑णीꣳ॒॒ रोहि॑णीम् ॅलक्ष्म॒णाम् । \newline
37. ल॒क्ष्म॒णाम् प॑ष्ठौ॒हीम् प॑ष्ठौ॒हीम् ॅल॑क्ष्म॒णाम् ॅल॑क्ष्म॒णाम् प॑ष्ठौ॒हीम् । \newline
38. प॒ष्ठौ॒हीं ॅवार्त्र॑घ्नीं॒ ॅवार्त्र॑घ्नीम् पष्ठौ॒हीम् प॑ष्ठौ॒हीं ॅवार्त्र॑घ्नीम् । \newline
39. वार्त्र॑घ्नीम् दद्याद् दद्या॒द् वार्त्र॑घ्नीं॒ ॅवार्त्र॑घ्नीम् दद्यात् । \newline
40. वार्त्र॑घ्नी॒मिति॒ वार्त्र॑ - घ्नी॒म् । \newline
41. द॒द्या॒द् यो यो द॑द्याद् दद्या॒द् यः । \newline
42. य ए॒व मे॒वं ॅयो य ए॒वम् । \newline
43. ए॒वं ॅवि॒द्वान्. वि॒द्वा ने॒व मे॒वं ॅवि॒द्वान् । \newline
44. वि॒द्वान् रोहि॑णीꣳ॒॒ रोहि॑णीं ॅवि॒द्वान्. वि॒द्वान् रोहि॑णीम् । \newline
45. रोहि॑णीम् ॅलक्ष्म॒णाम् ॅल॑क्ष्म॒णाꣳ रोहि॑णीꣳ॒॒ रोहि॑णीम् ॅलक्ष्म॒णाम् । \newline
46. ल॒क्ष्म॒णाम् प॑ष्ठौ॒हीम् प॑ष्ठौ॒हीम् ॅल॑क्ष्म॒णाम् ॅल॑क्ष्म॒णाम् प॑ष्ठौ॒हीम् । \newline
47. प॒ष्ठौ॒हीं ॅवार्त्र॑घ्नीं॒ ॅवार्त्र॑घ्नीम् पष्ठौ॒हीम् प॑ष्ठौ॒हीं ॅवार्त्र॑घ्नीम् । \newline
48. वार्त्र॑घ्नी॒म् ददा॑ति॒ ददा॑ति॒ वार्त्र॑घ्नीं॒ ॅवार्त्र॑घ्नी॒म् ददा॑ति । \newline
49. वार्त्र॑घ्नी॒मिति॒ वार्त्र॑ - घ्नी॒म् । \newline
50. ददा॑ति॒ त्रय॑स्त्रिꣳश॒त् त्रय॑स्त्रिꣳश॒द् ददा॑ति॒ ददा॑ति॒ त्रय॑स्त्रिꣳशत् । \newline
51. त्रय॑स्त्रिꣳशच् च च॒ त्रय॑स्त्रिꣳश॒त् त्रय॑स्त्रिꣳशच् च । \newline
52. त्रय॑स्त्रिꣳश॒दिति॒ त्रयः॑ - त्रिꣳ॒॒श॒त् । \newline
53. चै॒वैव च॑ चै॒व । \newline
54. ए॒वास्या᳚ स्यै॒वैवास्य॑ । \newline
55. अ॒स्य॒ त्रीणि॒ त्रीण्य॑स्यास्य॒ त्रीणि॑ । \newline
56. त्रीणि॑ च च॒ त्रीणि॒ त्रीणि॑ च । \newline
57. च॒ श॒तानि॑ श॒तानि॑ च च श॒तानि॑ । \newline
58. श॒तानि॒ सा सा श॒तानि॑ श॒तानि॒ सा । \newline
59. सा द॒त्ता द॒त्ता सा सा द॒त्ता । \newline
60. द॒त्ता भ॑वति भवति द॒त्ता द॒त्ता भ॑वति । \newline

\textbf{Ghana Paata } \newline

1. श॒तैः सोमः॒ सोमः॑ श॒तैः श॒तैः सोमः॑ क्री॒तः क्री॒तः सोमः॑ श॒तैः श॒तैः सोमः॑ क्री॒तः । \newline
2. सोमः॑ क्री॒तः क्री॒तः सोमः॒ सोमः॑ क्री॒तो भ॑वति भवति क्री॒तः सोमः॒ सोमः॑ क्री॒तो भ॑वति । \newline
3. क्री॒तो भ॑वति भवति क्री॒तः क्री॒तो भ॑वति॒ सुक्री॑तेन॒ सुक्री॑तेन भवति क्री॒तः क्री॒तो भ॑वति॒ सुक्री॑तेन । \newline
4. भ॒व॒ति॒ सुक्री॑तेन॒ सुक्री॑तेन भवति भवति॒ सुक्री॑तेन यजते यजते॒ सुक्री॑तेन भवति भवति॒ सुक्री॑तेन यजते । \newline
5. सुक्री॑तेन यजते यजते॒ सुक्री॑तेन॒ सुक्री॑तेन यजते॒ ताम् तां ॅय॑जते॒ सुक्री॑तेन॒ सुक्री॑तेन यजते॒ ताम् । \newline
6. सुक्री॑ते॒नेति॒ सु - क्री॒ते॒न॒ । \newline
7. य॒ज॒ते॒ ताम् तां ॅय॑जते यजते॒ ता म॒फ्स्व॑फ्सु तां ॅय॑जते यजते॒ ता म॒फ्सु । \newline
8. ता म॒फ्स्व॑फ्सु ताम् ता म॒फ्सु प्र प्राफ्सु ताम् ता म॒फ्सु प्र । \newline
9. अ॒फ्सु प्र प्रा फ्स्व॑फ्सु प्रा वे॑शयन् नवेशय॒न् प्रा फ्स्व॑फ्सु प्रा वे॑शयन्न् । \newline
10. अ॒फ्स्वित्य॑प् - सु । \newline
11. प्रा वे॑शयन् नवेशय॒न् प्र प्रा वे॑शय॒न् निन्द्रा॒ येन्द्रा॑या वेशय॒न् प्र प्रा वे॑शय॒न् निन्द्रा॑य । \newline
12. अ॒वे॒श॒य॒न् निन्द्रा॒ येन्द्रा॑या वेशयन् नवेशय॒न् निन्द्रा॑यो॒ देह्यु॒ देहीन्द्रा॑या वेशयन् नवेशय॒न् निन्द्रा॑यो॒ देहि॑ । \newline
13. इन्द्रा॑यो॒ देह्यु॒ देहीन्द्रा॒ येन्द्रा॑यो॒ देहीती त्यु॒देहीन्द्रा॒ येन्द्रा॑यो॒ देहीति॑ । \newline
14. उ॒देहीती त्यु॒देह्यु॒ देहीति॒ सा सेत्यु॒दे ह्यु॒देहीति॒ सा । \newline
15. उ॒देहीत्यु॑त् - एहि॑ । \newline
16. इति॒ सा सेतीति॒ सा रोहि॑णी॒ रोहि॑णी॒ सेतीति॒ सा रोहि॑णी । \newline
17. सा रोहि॑णी॒ रोहि॑णी॒ सा सा रोहि॑णी लक्ष्म॒णा ल॑क्ष्म॒णा रोहि॑णी॒ सा सा रोहि॑णी लक्ष्म॒णा । \newline
18. रोहि॑णी लक्ष्म॒णा ल॑क्ष्म॒णा रोहि॑णी॒ रोहि॑णी लक्ष्म॒णा प॑ष्ठौ॒ही प॑ष्ठौ॒ही ल॑क्ष्म॒णा रोहि॑णी॒ रोहि॑णी लक्ष्म॒णा प॑ष्ठौ॒ही । \newline
19. ल॒क्ष्म॒णा प॑ष्ठौ॒ही प॑ष्ठौ॒ही ल॑क्ष्म॒णा ल॑क्ष्म॒णा प॑ष्ठौ॒ही वार्त्र॑घ्नी॒ वार्त्र॑घ्नी पष्ठौ॒ही ल॑क्ष्म॒णा ल॑क्ष्म॒णा प॑ष्ठौ॒ही वार्त्र॑घ्नी । \newline
20. प॒ष्ठौ॒ही वार्त्र॑घ्नी॒ वार्त्र॑घ्नी पष्ठौ॒ही प॑ष्ठौ॒ही वार्त्र॑घ्नी रू॒पꣳ रू॒पं ॅवार्त्र॑घ्नी पष्ठौ॒ही प॑ष्ठौ॒ही वार्त्र॑घ्नी रू॒पम् । \newline
21. वार्त्र॑घ्नी रू॒पꣳ रू॒पं ॅवार्त्र॑घ्नी॒ वार्त्र॑घ्नी रू॒पम् कृ॒त्वा कृ॒त्वा रू॒पं ॅवार्त्र॑घ्नी॒ वार्त्र॑घ्नी रू॒पम् कृ॒त्वा । \newline
22. वार्त्र॒घ्नीति॒ वार्त्र॑ - घ्नी॒ । \newline
23. रू॒पम् कृ॒त्वा कृ॒त्वा रू॒पꣳ रू॒पम् कृ॒त्वा त्रय॑स्त्रिꣳशता॒ त्रय॑स्त्रिꣳशता कृ॒त्वा रू॒पꣳ रू॒पम् कृ॒त्वा त्रय॑स्त्रिꣳशता । \newline
24. कृ॒त्वा त्रय॑स्त्रिꣳशता॒ त्रय॑स्त्रिꣳशता कृ॒त्वा कृ॒त्वा त्रय॑स्त्रिꣳशता च च॒ त्रय॑स्त्रिꣳशता कृ॒त्वा कृ॒त्वा त्रय॑स्त्रिꣳशता च । \newline
25. त्रय॑स्त्रिꣳशता च च॒ त्रय॑स्त्रिꣳशता॒ त्रय॑स्त्रिꣳशता च त्रि॒भि स्त्रि॒भि श्च॒ त्रय॑स्त्रिꣳशता॒ त्रय॑स्त्रिꣳशता च त्रि॒भिः । \newline
26. त्रय॑स्त्रिꣳश॒तेति॒ त्रयः॑ - त्रिꣳ॒॒श॒ता॒ । \newline
27. च॒ त्रि॒भि स्त्रि॒भि श्च॑ च त्रि॒भि श्च॑ च त्रि॒भि श्च॑ च त्रि॒भि श्च॑ । \newline
28. त्रि॒भि श्च॑ च त्रि॒भि स्त्रि॒भि श्च॑ श॒तैः श॒तै श्च॑ त्रि॒भि स्त्रि॒भि श्च॑ श॒तैः । \newline
29. त्रि॒भिरिति॑ त्रि - भिः । \newline
30. च॒ श॒तैः श॒तै श्च॑ च श॒तैः स॒ह स॒ह श॒तै श्च॑ च श॒तैः स॒ह । \newline
31. श॒तैः स॒ह स॒ह श॒तैः श॒तैः स॒होदै दु॒दैथ् स॒ह श॒तैः श॒तैः स॒होदैत् । \newline
32. स॒होदै दु॒दैथ् स॒ह स॒होदैत् तस्मा॒त् तस्मा॑ दु॒दैथ् स॒ह स॒होदैत् तस्मा᳚त् । \newline
33. उ॒दैत् तस्मा॒त् तस्मा॑ दु॒दै दु॒दैत् तस्मा॒द् रोहि॑णीꣳ॒॒ रोहि॑णी॒म् तस्मा॑ दु॒दै दु॒दैत् तस्मा॒द् रोहि॑णीम् । \newline
34. उ॒दैदित्यु॑त् - ऐत् । \newline
35. तस्मा॒द् रोहि॑णीꣳ॒॒ रोहि॑णी॒म् तस्मा॒त् तस्मा॒द् रोहि॑णीम् ॅलक्ष्म॒णाम् ॅल॑क्ष्म॒णाꣳ रोहि॑णी॒म् तस्मा॒त् तस्मा॒द् रोहि॑णीम् ॅलक्ष्म॒णाम् । \newline
36. रोहि॑णीम् ॅलक्ष्म॒णाम् ॅल॑क्ष्म॒णाꣳ रोहि॑णीꣳ॒॒ रोहि॑णीम् ॅलक्ष्म॒णाम् प॑ष्ठौ॒हीम् प॑ष्ठौ॒हीम् ॅल॑क्ष्म॒णाꣳ रोहि॑णीꣳ॒॒ रोहि॑णीम् ॅलक्ष्म॒णाम् प॑ष्ठौ॒हीम् । \newline
37. ल॒क्ष्म॒णाम् प॑ष्ठौ॒हीम् प॑ष्ठौ॒हीम् ॅल॑क्ष्म॒णाम् ॅल॑क्ष्म॒णाम् प॑ष्ठौ॒हीं ॅवार्त्र॑घ्नीं॒ ॅवार्त्र॑घ्नीम् पष्ठौ॒हीम् ॅल॑क्ष्म॒णाम् ॅल॑क्ष्म॒णाम् प॑ष्ठौ॒हीं ॅवार्त्र॑घ्नीम् । \newline
38. प॒ष्ठौ॒हीं ॅवार्त्र॑घ्नीं॒ ॅवार्त्र॑घ्नीम् पष्ठौ॒हीम् प॑ष्ठौ॒हीं ॅवार्त्र॑घ्नीम् दद्याद् दद्या॒द् वार्त्र॑घ्नीम् पष्ठौ॒हीम् प॑ष्ठौ॒हीं ॅवार्त्र॑घ्नीम् दद्यात् । \newline
39. वार्त्र॑घ्नीम् दद्याद् दद्या॒द् वार्त्र॑घ्नीं॒ ॅवार्त्र॑घ्नीम् दद्या॒द् यो यो द॑द्या॒द् वार्त्र॑घ्नीं॒ ॅवार्त्र॑घ्नीम् दद्या॒द् यः । \newline
40. वार्त्र॑घ्नी॒मिति॒ वार्त्र॑ - घ्नी॒म् । \newline
41. द॒द्या॒द् यो यो द॑द्याद् दद्या॒द् य ए॒व मे॒वं ॅयो द॑द्याद् दद्या॒द् य ए॒वम् । \newline
42. य ए॒व मे॒वं ॅयो य ए॒वं ॅवि॒द्वान्. वि॒द्वा ने॒वं ॅयो य ए॒वं ॅवि॒द्वान् । \newline
43. ए॒वं ॅवि॒द्वान्. वि॒द्वा ने॒व मे॒वं ॅवि॒द्वान् रोहि॑णीꣳ॒॒ रोहि॑णीं ॅवि॒द्वा ने॒व मे॒वं ॅवि॒द्वान् रोहि॑णीम् । \newline
44. वि॒द्वान् रोहि॑णीꣳ॒॒ रोहि॑णीं ॅवि॒द्वान्. वि॒द्वान् रोहि॑णीम् ॅलक्ष्म॒णाम् ॅल॑क्ष्म॒णाꣳ रोहि॑णीं ॅवि॒द्वान्. वि॒द्वान् रोहि॑णीम् ॅलक्ष्म॒णाम् । \newline
45. रोहि॑णीम् ॅलक्ष्म॒णाम् ॅल॑क्ष्म॒णाꣳ रोहि॑णीꣳ॒॒ रोहि॑णीम् ॅलक्ष्म॒णाम् प॑ष्ठौ॒हीम् प॑ष्ठौ॒हीम् ॅल॑क्ष्म॒णाꣳ रोहि॑णीꣳ॒॒ रोहि॑णीम् ॅलक्ष्म॒णाम् प॑ष्ठौ॒हीम् । \newline
46. ल॒क्ष्म॒णाम् प॑ष्ठौ॒हीम् प॑ष्ठौ॒हीम् ॅल॑क्ष्म॒णाम् ॅल॑क्ष्म॒णाम् प॑ष्ठौ॒हीं ॅवार्त्र॑घ्नीं॒ ॅवार्त्र॑घ्नीम् पष्ठौ॒हीम् ॅल॑क्ष्म॒णाम् ॅल॑क्ष्म॒णाम् प॑ष्ठौ॒हीं ॅवार्त्र॑घ्नीम् । \newline
47. प॒ष्ठौ॒हीं ॅवार्त्र॑घ्नीं॒ ॅवार्त्र॑घ्नीम् पष्ठौ॒हीम् प॑ष्ठौ॒हीं ॅवार्त्र॑घ्नी॒म् ददा॑ति॒ ददा॑ति॒ वार्त्र॑घ्नीम् पष्ठौ॒हीम् प॑ष्ठौ॒हीं ॅवार्त्र॑घ्नी॒म् ददा॑ति । \newline
48. वार्त्र॑घ्नी॒म् ददा॑ति॒ ददा॑ति॒ वार्त्र॑घ्नीं॒ ॅवार्त्र॑घ्नी॒म् ददा॑ति॒ त्रय॑स्त्रिꣳश॒त् त्रय॑स्त्रिꣳश॒द् ददा॑ति॒ वार्त्र॑घ्नीं॒ ॅवार्त्र॑घ्नी॒म् ददा॑ति॒ त्रय॑स्त्रिꣳशत् । \newline
49. वार्त्र॑घ्नी॒मिति॒ वार्त्र॑ - घ्नी॒म् । \newline
50. ददा॑ति॒ त्रय॑स्त्रिꣳश॒त् त्रय॑स्त्रिꣳश॒द् ददा॑ति॒ ददा॑ति॒ त्रय॑स्त्रिꣳशच् च च॒ त्रय॑स्त्रिꣳश॒द् ददा॑ति॒ ददा॑ति॒ त्रय॑स्त्रिꣳशच् च । \newline
51. त्रय॑स्त्रिꣳशच् च च॒ त्रय॑स्त्रिꣳश॒त् त्रय॑स्त्रिꣳशच् चै॒वैव च॒ त्रय॑स्त्रिꣳश॒त् त्रय॑स्त्रिꣳशच् चै॒व । \newline
52. त्रय॑स्त्रिꣳश॒दिति॒ त्रयः॑ - त्रिꣳ॒॒श॒त् । \newline
53. चै॒वैव च॑ चै॒वास्या᳚ स्यै॒व च॑ चै॒वास्य॑ । \newline
54. ए॒वास्या᳚ स्यै॒वै वास्य॒ त्रीणि॒ त्रीण्य॑ स्यै॒वैवास्य॒ त्रीणि॑ । \newline
55. अ॒स्य॒ त्रीणि॒ त्रीण्य॑ स्यास्य॒ त्रीणि॑ च च॒ त्रीण्य॑स्यास्य॒ त्रीणि॑ च । \newline
56. त्रीणि॑ च च॒ त्रीणि॒ त्रीणि॑ च श॒तानि॑ श॒तानि॑ च॒ त्रीणि॒ त्रीणि॑ च श॒तानि॑ । \newline
57. च॒ श॒तानि॑ श॒तानि॑ च च श॒तानि॒ सा सा श॒तानि॑ च च श॒तानि॒ सा । \newline
58. श॒तानि॒ सा सा श॒तानि॑ श॒तानि॒ सा द॒त्ता द॒त्ता सा श॒तानि॑ श॒तानि॒ सा द॒त्ता । \newline
59. सा द॒त्ता द॒त्ता सा सा द॒त्ता भ॑वति भवति द॒त्ता सा सा द॒त्ता भ॑वति । \newline
60. द॒त्ता भ॑वति भवति द॒त्ता द॒त्ता भ॑वति॒ ताम् ताम् भ॑वति द॒त्ता द॒त्ता भ॑वति॒ ताम् । \newline
\pagebreak
\markright{ TS 7.1.6.4  \hfill https://www.vedavms.in \hfill}

\section{ TS 7.1.6.4 }

\textbf{TS 7.1.6.4 } \newline
\textbf{Samhita Paata} \newline

भ॑वति॒ ताम॒फ्सु प्रावे॑शयन् य॒मायो॒देहीति॒ सा जर॑ती मू॒र्खा त॑ज्जघ॒न्या रू॒पं कृ॒त्वा त्रय॑स्त्रिꣳशता च त्रि॒भिश्च॑ श॒तैः स॒होदैत् तस्मा॒ज्जर॑तीं मू॒र्खां त॑ज्जघ॒न्या-म॑नु॒स्तर॑णीं कुर्वीत॒ य ए॒वं ॅवि॒द्वाञ्जर॑तीं मू॒र्खां त॑ज्जघ॒न्या-म॑नु॒स्तर॑णीं कुरु॒ते त्रय॑स्त्रिꣳशच्चै॒वास्य॒ त्रीणि॑ च श॒तानि॒ साऽमुष्मि॑ॅल्लो॒के भ॑वति॒ वागे॒व स॑हस्रत॒मी तस्मा॒ - [  ] \newline

\textbf{Pada Paata} \newline

भ॒व॒ति॒ । ताम् । अ॒फ्स्वित्य॑प् - सु । प्रेति॑ । अ॒वे॒श॒य॒न्न् । य॒माय॑ । उ॒देहीत्यु॑त् - एहि॑ । इति॑ । सा । जर॑ती । मू॒र्खा । त॒ज्ज॒घ॒न्येति॑ तत् - ज॒घ॒न्या । रू॒पम् । कृ॒त्वा । त्रय॑स्त्रिꣳश॒तेति॒ त्रयः॑-त्रिꣳ॒॒श॒ता॒ । च॒ । त्रि॒भिरिति॑ त्रि - भिः । च॒ । श॒तैः । स॒ह । उ॒दैदित्यु॑त् - ऐत् । तस्मा᳚त् । जर॑तीम् । मू॒र्खाम् । त॒ज्ज॒घ॒न्यामिति॑ तत् - ज॒घ॒न्याम् । अ॒नु॒स्तर॑णी॒मित्य॑नु - स्तर॑णीम् । कु॒र्वी॒त॒ । यः । ए॒वम् । वि॒द्वान् । जर॑तीम् । मू॒र्खाम् । त॒ज्ज॒घ॒न्यामिति॑ तत् - ज॒घ॒न्याम् । अ॒नु॒स्तर॑णी॒मित्यु॑नु - स्तर॑णीम् । कु॒रु॒ते । त्रय॑स्त्रिꣳश॒दिति॒ त्रयः॑ - त्रिꣳ॒॒श॒त् । च॒ । ए॒व । अ॒स्य॒ । त्रीणि॑ । च॒ । श॒तानि॑ । सा । अ॒मुष्मिन्न्॑ । लो॒के । भ॒व॒ति॒ । वाक् । ए॒व । स॒ह॒स्र॒त॒मीति॑ सहस्र - त॒मी । तस्मा᳚त् ।  \newline


\textbf{Krama Paata} \newline

भ॒व॒ति॒ ताम् । ताम॒फ्सु । अ॒फ्सु प्र । अ॒फ्स्वित्य॑प् - सु । प्रावे॑शयन्न् । अ॒वे॒श॒य॒न्.॒ य॒माय॑ । य॒मायो॒देहि॑ । उ॒देहीति॑ । उ॒देहीत्यु॑त् - एहि॑ । इति॒ सा । सा जर॑ती । जर॑ती मू॒र्खा । मू॒र्खा त॑ज्जघ॒न्या । त॒ज्ज॒घ॒न्या रू॒पम् । त॒ज्ज॒घ॒न्येति॑ तत् - ज॒घ॒न्या । रू॒पम् कृ॒त्वा । कृ॒त्वा त्रय॑स्त्रिꣳशता । त्रय॑स्त्रिꣳशता च । त्रय॑स्त्रिꣳश॒तेति॒ त्रयः॑ - त्रिꣳ॒॒श॒ता॒ । च॒ त्रि॒भिः । त्रि॒भिश्च॑ । त्रि॒भिरिति॑ त्रि - भिः । च॒ श॒तैः । श॒तैः स॒ह । स॒होदैत् । उ॒दैत् तस्मा᳚त् । उ॒दैदित्यु॑त् - ऐत् । तस्मा॒ज् जर॑तीम् । जर॑तीम् मू॒र्खाम् । मू॒र्खाम् त॑ज्जघ॒न्याम् । त॒ज्ज॒घ॒न्याम॑नु॒स्तर॑णीम् । त॒ज्ज॒घ॒न्यामिति॑ तत् - ज॒घ॒न्याम् । अ॒नु॒स्तर॑णीम् कुर्वीत । अ॒नु॒स्तर॑णी॒मित्य॑नु - स्तर॑णीम् । कु॒र्वी॒त॒ यः । य ए॒वम् । ए॒वम् ॅवि॒द्वान् । वि॒द्वान् जर॑तीम् । जर॑तीम् मू॒र्खाम् । मू॒र्खाम् त॑ज्जघ॒न्याम् । त॒ज्ज॒घ॒न्याम॑नु॒स्तर॑णीम् । त॒ज्ज॒घ॒न्यामिति॑ तत् - ज॒घ॒न्याम् । अ॒नु॒स्तर॑णीम् कुरु॒ते । अ॒नु॒स्तर॑णी॒मित्य॑नु - स्तर॑णीम् । कु॒रु॒ते त्रय॑स्त्रिꣳशत् । त्रय॑स्त्रिꣳशच्च । त्रय॑स्त्रिꣳश॒दिति॒ त्रयः॑ - त्रिꣳ॒॒श॒त्॒ । चै॒व । ए॒वास्य॑ । अ॒स्य॒ त्रीणि॑ । त्रीणि॑ च । च॒ श॒तानि॑ । श॒तानि॒ सा । साऽमुष्मिन्न्॑ । अ॒मुष्मि॑न् ॅलो॒के । लो॒के भ॑वति । भ॒व॒ति॒ वाक् । वागे॒व । ए॒व स॑हस्रत॒मी । स॒ह॒स्र॒त॒मी तस्मा᳚त् । स॒ह॒स्र॒त॒मीति॑ सहस्र - त॒मी । तस्मा॒द् वरः॑ \newline

\textbf{Jatai Paata} \newline

1. भ॒व॒ति॒ ताम् ताम् भ॑वति भवति॒ ताम् । \newline
2. ता म॒फ्स्व॑फ्सु ताम् ता म॒फ्सु । \newline
3. अ॒फ्सु प्र प्रा फ्स्व॑फ्सु प्र । \newline
4. अ॒फ्स्वित्य॑प् - सु । \newline
5. प्रा वे॑शयन् नवेशय॒न् प्र प्रा वे॑शयन्न् । \newline
6. अ॒वे॒श॒य॒न्॒. य॒माय॑ य॒माया॑ वेशयन् नवेशयन्. य॒माय॑ । \newline
7. य॒मायो॒दे ह्यु॒देहि॑ य॒माय॑ य॒मायो॒देहि॑ । \newline
8. उ॒देही तीत्यु॒दे ह्यु॒देहीति॑ । \newline
9. उ॒देहीत्यु॑त् - एहि॑ । \newline
10. इति॒ सा सेतीति॒ सा । \newline
11. सा जर॑ती॒ जर॑ती॒ सा सा जर॑ती । \newline
12. जर॑ती मू॒र्खा मू॒र्खा जर॑ती॒ जर॑ती मू॒र्खा । \newline
13. मू॒र्खा त॑ज्जघ॒न्या त॑ज्जघ॒न्या मू॒र्खा मू॒र्खा त॑ज्जघ॒न्या । \newline
14. त॒ज्ज॒घ॒न्या रू॒पꣳ रू॒पम् त॑ज्जघ॒न्या त॑ज्जघ॒न्या रू॒पम् । \newline
15. त॒ज्ज॒घ॒न्येति॑ तत् - ज॒घ॒न्या । \newline
16. रू॒पम् कृ॒त्वा कृ॒त्वा रू॒पꣳ रू॒पम् कृ॒त्वा । \newline
17. कृ॒त्वा त्रय॑स्त्रिꣳशता॒ त्रय॑स्त्रिꣳशता कृ॒त्वा कृ॒त्वा त्रय॑स्त्रिꣳशता । \newline
18. त्रय॑स्त्रिꣳशता च च॒ त्रय॑स्त्रिꣳशता॒ त्रय॑स्त्रिꣳशता च । \newline
19. त्रय॑स्त्रिꣳश॒तेति॒ त्रयः॑ - त्रिꣳ॒॒श॒ता॒ । \newline
20. च॒ त्रि॒भि स्त्रि॒भि श्च॑ च त्रि॒भिः । \newline
21. त्रि॒भि श्च॑ च त्रि॒भि स्त्रि॒भि श्च॑ । \newline
22. त्रि॒भिरिति॑ त्रि - भिः । \newline
23. च॒ श॒तैः श॒तै श्च॑ च श॒तैः । \newline
24. श॒तैः स॒ह स॒ह श॒तैः श॒तैः स॒ह । \newline
25. स॒होदै दु॒दैथ् स॒ह स॒होदैत् । \newline
26. उ॒दैत् तस्मा॒त् तस्मा॑ दु॒दै दु॒दैत् तस्मा᳚त् । \newline
27. उ॒दैदित्यु॑त् - ऐत् । \newline
28. तस्मा॒ज् जर॑ती॒म् जर॑ती॒म् तस्मा॒त् तस्मा॒ज् जर॑तीम् । \newline
29. जर॑तीम् मू॒र्खाम् मू॒र्खाम् जर॑ती॒म् जर॑तीम् मू॒र्खाम् । \newline
30. मू॒र्खाम् त॑ज्जघ॒न्याम् त॑ज्जघ॒न्याम् मू॒र्खाम् मू॒र्खाम् त॑ज्जघ॒न्याम् । \newline
31. त॒ज्ज॒घ॒न्या म॑नु॒स्तर॑णी मनु॒स्तर॑णीम् तज्जघ॒न्याम् त॑ज्जघ॒न्या म॑नु॒स्तर॑णीम् । \newline
32. त॒ज्ज॒घ॒न्यामिति॑ तत् - ज॒घ॒न्याम् । \newline
33. अ॒नु॒स्तर॑णीम् कुर्वीत कुर्वीता नु॒स्तर॑णी मनु॒स्तर॑णीम् कुर्वीत । \newline
34. अ॒नु॒स्तर॑णी॒मित्य॑नु - स्तर॑णीम् । \newline
35. कु॒र्वी॒त॒ यो यः कु॑र्वीत कुर्वीत॒ यः । \newline
36. य ए॒व मे॒वं ॅयो य ए॒वम् । \newline
37. ए॒वं ॅवि॒द्वान्. वि॒द्वा ने॒व मे॒वं ॅवि॒द्वान् । \newline
38. वि॒द्वान् जर॑ती॒म् जर॑तीं ॅवि॒द्वान्. वि॒द्वान् जर॑तीम् । \newline
39. जर॑तीम् मू॒र्खाम् मू॒र्खाम् जर॑ती॒म् जर॑तीम् मू॒र्खाम् । \newline
40. मू॒र्खाम् त॑ज्जघ॒न्याम् त॑ज्जघ॒न्याम् मू॒र्खाम् मू॒र्खाम् त॑ज्जघ॒न्याम् । \newline
41. त॒ज्ज॒घ॒न्या म॑नु॒स्तर॑णी मनु॒स्तर॑णीम् तज्जघ॒न्याम् त॑ज्जघ॒न्या म॑नु॒स्तर॑णीम् । \newline
42. त॒ज्ज॒घ॒न्यामिति॑ तत् - ज॒घ॒न्याम् । \newline
43. अ॒नु॒स्तर॑णीम् कुरु॒ते कु॑रु॒ते॑ ऽनु॒स्तर॑णी मनु॒स्तर॑णीम् कुरु॒ते । \newline
44. अ॒नु॒स्तर॑णी॒मित्य॑नु - स्तर॑णीम् । \newline
45. कु॒रु॒ते त्रय॑स्त्रिꣳश॒त् त्रय॑स्त्रिꣳशत् कुरु॒ते कु॑रु॒ते त्रय॑स्त्रिꣳशत् । \newline
46. त्रय॑स्त्रिꣳशच् च च॒ त्रय॑स्त्रिꣳश॒त् त्रय॑स्त्रिꣳशच् च । \newline
47. त्रय॑स्त्रिꣳश॒दिति॒ त्रयः॑ - त्रिꣳ॒॒श॒त् । \newline
48. चै॒वैव च॑ चै॒व । \newline
49. ए॒वास्या᳚ स्यै॒वै वास्य॑ । \newline
50. अ॒स्य॒ त्रीणि॒ त्रीण्य॑ स्यास्य॒ त्रीणि॑ । \newline
51. त्रीणि॑ च च॒ त्रीणि॒ त्रीणि॑ च । \newline
52. च॒ श॒तानि॑ श॒तानि॑ च च श॒तानि॑ । \newline
53. श॒तानि॒ सा सा श॒तानि॑ श॒तानि॒ सा । \newline
54. सा ऽमुष्मि॑न् न॒मुष्मि॒न् थ्सा सा ऽमुष्मिन्न्॑ । \newline
55. अ॒मुष्मि॑न् ॅलो॒के लो॒के॑ ऽमुष्मि॑न् न॒मुष्मि॑न् ॅलो॒के । \newline
56. लो॒के भ॑वति भवति लो॒के लो॒के भ॑वति । \newline
57. भ॒व॒ति॒ वाग् वाग् भ॑वति भवति॒ वाक् । \newline
58. वागे॒वैव वाग् वागे॒व । \newline
59. ए॒व स॑हस्रत॒मी स॑हस्रत॒ म्ये॑वैव स॑हस्रत॒मी । \newline
60. स॒ह॒स्र॒त॒मी तस्मा॒त् तस्मा᳚थ् सहस्रत॒मी स॑हस्रत॒मी तस्मा᳚त् । \newline
61. स॒ह॒स्र॒त॒मीति॑ सहस्र - त॒मी । \newline
62. तस्मा॒द् वरो॒ वर॒ स्तस्मा॒त् तस्मा॒द् वरः॑ । \newline

\textbf{Ghana Paata } \newline

1. भ॒व॒ति॒ ताम् ताम् भ॑वति भवति॒ ता म॒फ्स्व॑फ्सु ताम् भ॑वति भवति॒ ता म॒फ्सु । \newline
2. ता म॒फ्स्व॑फ्सु ताम् ता म॒फ्सु प्र प्राफ्सु ताम् ता म॒फ्सु प्र । \newline
3. अ॒फ्सु प्र प्रा फ्स्व॑फ्सु प्रा वे॑शयन् नवेशय॒न् प्रा फ्स्व॑फ्सु प्रा वे॑शयन्न् । \newline
4. अ॒फ्स्वित्य॑प् - सु । \newline
5. प्रा वे॑शयन् नवेशय॒न् प्र प्रा वे॑शयन्. य॒माय॑ य॒माया॑ वेशय॒न् प्र प्रा वे॑शयन्. य॒माय॑ । \newline
6. अ॒वे॒श॒य॒न्॒. य॒माय॑ य॒माया॑ वेशयन् नवेशयन्. य॒मा यो॒दे ह्यु॒देहि॑ य॒माया॑ वेशयन् नवेशयन्. य॒मा यो॒देहि॑ । \newline
7. य॒मा यो॒दे ह्यु॒देहि॑ य॒माय॑ य॒मा यो॒देहीती त्यु॒देहि॑ य॒माय॑ य॒मा यो॒देहीति॑ । \newline
8. उ॒देहीती त्यु॒दे ह्यु॒देहीति॒ सा सेत्यु॒दे ह्यु॒देहीति॒ सा । \newline
9. उ॒देहीत्यु॑त् - एहि॑ । \newline
10. इति॒ सा सेतीति॒ सा जर॑ती॒ जर॑ती॒ सेतीति॒ सा जर॑ती । \newline
11. सा जर॑ती॒ जर॑ती॒ सा सा जर॑ती मू॒र्खा मू॒र्खा जर॑ती॒ सा सा जर॑ती मू॒र्खा । \newline
12. जर॑ती मू॒र्खा मू॒र्खा जर॑ती॒ जर॑ती मू॒र्खा त॑ज्जघ॒न्या त॑ज्जघ॒न्या मू॒र्खा जर॑ती॒ जर॑ती मू॒र्खा त॑ज्जघ॒न्या । \newline
13. मू॒र्खा त॑ज्जघ॒न्या त॑ज्जघ॒न्या मू॒र्खा मू॒र्खा त॑ज्जघ॒न्या रू॒पꣳ रू॒पम् त॑ज्जघ॒न्या मू॒र्खा मू॒र्खा त॑ज्जघ॒न्या रू॒पम् । \newline
14. त॒ज्ज॒घ॒न्या रू॒पꣳ रू॒पम् त॑ज्जघ॒न्या त॑ज्जघ॒न्या रू॒पम् कृ॒त्वा कृ॒त्वा रू॒पम् त॑ज्जघ॒न्या त॑ज्जघ॒न्या रू॒पम् कृ॒त्वा । \newline
15. त॒ज्ज॒घ॒न्येति॑ तत् - ज॒घ॒न्या । \newline
16. रू॒पम् कृ॒त्वा कृ॒त्वा रू॒पꣳ रू॒पम् कृ॒त्वा त्रय॑स्त्रिꣳशता॒ त्रय॑स्त्रिꣳशता कृ॒त्वा रू॒पꣳ रू॒पम् कृ॒त्वा त्रय॑स्त्रिꣳशता । \newline
17. कृ॒त्वा त्रय॑स्त्रिꣳशता॒ त्रय॑स्त्रिꣳशता कृ॒त्वा कृ॒त्वा त्रय॑स्त्रिꣳशता च च॒ त्रय॑स्त्रिꣳशता कृ॒त्वा कृ॒त्वा त्रय॑स्त्रिꣳशता च । \newline
18. त्रय॑स्त्रिꣳशता च च॒ त्रय॑स्त्रिꣳशता॒ त्रय॑स्त्रिꣳशता च त्रि॒भि स्त्रि॒भि श्च॒ त्रय॑स्त्रिꣳशता॒ त्रय॑स्त्रिꣳशता च त्रि॒भिः । \newline
19. त्रय॑स्त्रिꣳश॒तेति॒ त्रयः॑ - त्रिꣳ॒॒श॒ता॒ । \newline
20. च॒ त्रि॒भि स्त्रि॒भि श्च॑ च त्रि॒भि श्च॑ च त्रि॒भि श्च॑ च त्रि॒भि श्च॑ । \newline
21. त्रि॒भि श्च॑ च त्रि॒भि स्त्रि॒भि श्च॑ श॒तैः श॒तै श्च॑ त्रि॒भि स्त्रि॒भि श्च॑ श॒तैः । \newline
22. त्रि॒भिरिति॑ त्रि - भिः । \newline
23. च॒ श॒तैः श॒तै श्च॑ च श॒तैः स॒ह स॒ह श॒तै श्च॑ च श॒तैः स॒ह । \newline
24. श॒तैः स॒ह स॒ह श॒तैः श॒तैः स॒होदै दु॒दैथ् स॒ह श॒तैः श॒तैः स॒होदैत् । \newline
25. स॒होदै दु॒दैथ् स॒ह स॒होदैत् तस्मा॒त् तस्मा॑ दु॒दैथ् स॒ह स॒होदैत् तस्मा᳚त् । \newline
26. उ॒दैत् तस्मा॒त् तस्मा॑ दु॒दै दु॒दैत् तस्मा॒ज् जर॑ती॒म् जर॑ती॒म् तस्मा॑ दु॒दै दु॒दैत् तस्मा॒ज् जर॑तीम् । \newline
27. उ॒दैदित्यु॑त् - ऐत् । \newline
28. तस्मा॒ज् जर॑ती॒म् जर॑ती॒म् तस्मा॒त् तस्मा॒ज् जर॑तीम् मू॒र्खाम् मू॒र्खाम् जर॑ती॒म् तस्मा॒त् तस्मा॒ज् जर॑तीम् मू॒र्खाम् । \newline
29. जर॑तीम् मू॒र्खाम् मू॒र्खाम् जर॑ती॒म् जर॑तीम् मू॒र्खाम् त॑ज्जघ॒न्याम् त॑ज्जघ॒न्याम् मू॒र्खाम् जर॑ती॒म् जर॑तीम् मू॒र्खाम् त॑ज्जघ॒न्याम् । \newline
30. मू॒र्खाम् त॑ज्जघ॒न्याम् त॑ज्जघ॒न्याम् मू॒र्खाम् मू॒र्खाम् त॑ज्जघ॒न्या म॑नु॒स्तर॑णी मनु॒स्तर॑णीम् तज्जघ॒न्याम् मू॒र्खाम् मू॒र्खाम् त॑ज्जघ॒न्या म॑नु॒स्तर॑णीम् । \newline
31. त॒ज्ज॒घ॒न्या म॑नु॒स्तर॑णी मनु॒स्तर॑णीम् तज्जघ॒न्याम् त॑ज्जघ॒न्या म॑नु॒स्तर॑णीम् कुर्वीत कुर्वीता नु॒स्तर॑णीम् तज्जघ॒न्याम् त॑ज्जघ॒न्या म॑नु॒स्तर॑णीम् कुर्वीत । \newline
32. त॒ज्ज॒घ॒न्यामिति॑ तत् - ज॒घ॒न्याम् । \newline
33. अ॒नु॒स्तर॑णीम् कुर्वीत कुर्वीता नु॒स्तर॑णी मनु॒स्तर॑णीम् कुर्वीत॒ यो यः कु॑र्वीता नु॒स्तर॑णी मनु॒स्तर॑णीम् कुर्वीत॒ यः । \newline
34. अ॒नु॒स्तर॑णी॒मित्य॑नु - स्तर॑णीम् । \newline
35. कु॒र्वी॒त॒ यो यः कु॑र्वीत कुर्वीत॒ य ए॒व मे॒वं ॅयः कु॑र्वीत कुर्वीत॒ य ए॒वम् । \newline
36. य ए॒व मे॒वं ॅयो य ए॒वं ॅवि॒द्वान्. वि॒द्वा ने॒वं ॅयो य ए॒वं ॅवि॒द्वान् । \newline
37. ए॒वं ॅवि॒द्वान्. वि॒द्वा ने॒व मे॒वं ॅवि॒द्वान् जर॑ती॒म् जर॑तीं ॅवि॒द्वा ने॒व मे॒वं ॅवि॒द्वान् जर॑तीम् । \newline
38. वि॒द्वान् जर॑ती॒म् जर॑तीं ॅवि॒द्वान्. वि॒द्वान् जर॑तीम् मू॒र्खाम् मू॒र्खाम् जर॑तीं ॅवि॒द्वान्. वि॒द्वान् जर॑तीम् मू॒र्खाम् । \newline
39. जर॑तीम् मू॒र्खाम् मू॒र्खाम् जर॑ती॒म् जर॑तीम् मू॒र्खाम् त॑ज्जघ॒न्याम् त॑ज्जघ॒न्याम् मू॒र्खाम् जर॑ती॒म् जर॑तीम् मू॒र्खाम् त॑ज्जघ॒न्याम् । \newline
40. मू॒र्खाम् त॑ज्जघ॒न्याम् त॑ज्जघ॒न्याम् मू॒र्खाम् मू॒र्खाम् त॑ज्जघ॒न्या म॑नु॒स्तर॑णी मनु॒स्तर॑णीम् तज्जघ॒न्याम् मू॒र्खाम् मू॒र्खाम् त॑ज्जघ॒न्या म॑नु॒स्तर॑णीम् । \newline
41. त॒ज्ज॒घ॒न्या म॑नु॒स्तर॑णी मनु॒स्तर॑णीम् तज्जघ॒न्याम् त॑ज्जघ॒न्या म॑नु॒स्तर॑णीम् कुरु॒ते कु॑रु॒ते॑ ऽनु॒स्तर॑णीम् तज्जघ॒न्याम् त॑ज्जघ॒न्या म॑नु॒स्तर॑णीम् कुरु॒ते । \newline
42. त॒ज्ज॒घ॒न्यामिति॑ तत् - ज॒घ॒न्याम् । \newline
43. अ॒नु॒स्तर॑णीम् कुरु॒ते कु॑रु॒ते॑ ऽनु॒स्तर॑णी मनु॒स्तर॑णीम् कुरु॒ते त्रय॑स्त्रिꣳश॒त् त्रय॑स्त्रिꣳशत् कुरु॒ते॑ ऽनु॒स्तर॑णी मनु॒स्तर॑णीम् कुरु॒ते त्रय॑स्त्रिꣳशत् । \newline
44. अ॒नु॒स्तर॑णी॒मित्य॑नु - स्तर॑णीम् । \newline
45. कु॒रु॒ते त्रय॑स्त्रिꣳश॒त् त्रय॑स्त्रिꣳशत् कुरु॒ते कु॑रु॒ते त्रय॑स्त्रिꣳशच् च च॒ त्रय॑स्त्रिꣳशत् कुरु॒ते कु॑रु॒ते त्रय॑स्त्रिꣳशच् च । \newline
46. त्रय॑स्त्रिꣳशच् च च॒ त्रय॑स्त्रिꣳश॒त् त्रय॑स्त्रिꣳशच् चै॒वैव च॒ त्रय॑स्त्रिꣳश॒त् त्रय॑स्त्रिꣳशच् चै॒व । \newline
47. त्रय॑स्त्रिꣳश॒दिति॒ त्रयः॑ - त्रिꣳ॒॒श॒त् । \newline
48. चै॒वैव च॑ चै॒वास्या᳚ स्यै॒व च॑ चै॒वास्य॑ । \newline
49. ए॒वास्या᳚ स्यै॒वै वास्य॒ त्रीणि॒ त्रीण्य॑ स्यै॒वै वास्य॒ त्रीणि॑ । \newline
50. अ॒स्य॒ त्रीणि॒ त्रीण्य॑ स्यास्य॒ त्रीणि॑ च च॒ त्रीण्य॑ स्यास्य॒ त्रीणि॑ च । \newline
51. त्रीणि॑ च च॒ त्रीणि॒ त्रीणि॑ च श॒तानि॑ श॒तानि॑ च॒ त्रीणि॒ त्रीणि॑ च श॒तानि॑ । \newline
52. च॒ श॒तानि॑ श॒तानि॑ च च श॒तानि॒ सा सा श॒तानि॑ च च श॒तानि॒ सा । \newline
53. श॒तानि॒ सा सा श॒तानि॑ श॒तानि॒ सा ऽमुष्मि॑न् न॒मुष्मि॒न् थ्सा श॒तानि॑ श॒तानि॒ सा ऽमुष्मिन्न्॑ । \newline
54. सा ऽमुष्मि॑न् न॒मुष्मि॒न् थ्सा सा ऽमुष्मि॑न् ॅलो॒के लो॒के॑ ऽमुष्मि॒न् थ्सा सा ऽमुष्मि॑न् ॅलो॒के । \newline
55. अ॒मुष्मि॑न् ॅलो॒के लो॒के॑ ऽमुष्मि॑न् न॒मुष्मि॑न् ॅलो॒के भ॑वति भवति लो॒के॑ ऽमुष्मि॑न् न॒मुष्मि॑न् ॅलो॒के भ॑वति । \newline
56. लो॒के भ॑वति भवति लो॒के लो॒के भ॑वति॒ वाग् वाग् भ॑वति लो॒के लो॒के भ॑वति॒ वाक् । \newline
57. भ॒व॒ति॒ वाग् वाग् भ॑वति भवति॒ वा गे॒वैव वाग् भ॑वति भवति॒ वा गे॒व । \newline
58. वागे॒ वैव वाग् वा गे॒व स॑हस्रत॒मी स॑हस्रत॒ म्ये॑व वाग् वा गे॒व स॑हस्रत॒मी । \newline
59. ए॒व स॑हस्रत॒मी स॑हस्रत॒ म्ये॑वैव स॑हस्रत॒मी तस्मा॒त् तस्मा᳚थ् सहस्रत॒ म्ये॑वैव स॑हस्रत॒मी तस्मा᳚त् । \newline
60. स॒ह॒स्र॒त॒मी तस्मा॒त् तस्मा᳚थ् सहस्रत॒मी स॑हस्रत॒मी तस्मा॒द् वरो॒ वर॒ स्तस्मा᳚थ् सहस्रत॒मी स॑हस्रत॒मी तस्मा॒द् वरः॑ । \newline
61. स॒ह॒स्र॒त॒मीति॑ सहस्र - त॒मी । \newline
62. तस्मा॒द् वरो॒ वर॒ स्तस्मा॒त् तस्मा॒द् वरो॒ देयो॒ देयो॒ वर॒ स्तस्मा॒त् तस्मा॒द् वरो॒ देयः॑ । \newline
\pagebreak
\markright{ TS 7.1.6.5  \hfill https://www.vedavms.in \hfill}

\section{ TS 7.1.6.5 }

\textbf{TS 7.1.6.5 } \newline
\textbf{Samhita Paata} \newline

द्वरो॒ देयः॒ सा हि वरः॑ स॒हस्र॑मस्य॒ सा द॒त्ता भ॑वति॒ तस्मा॒द् वरो॒ न प्र॑ति॒गृह्यः॒ सा हि वरः॑ स॒हस्र॑मस्य॒ प्रति॑गृहीतं भवती॒यं ॅवर॒ इति॑ ब्रूया॒दथा॒न्यां ब्रू॑यादि॒यं ममेति॒ तथा᳚ऽस्य॒ तथ् स॒हस्र॒-मप्र॑तिगृहीतं भवत्युभयतए॒नी स्या॒त् तदा॑हुरन्यत ए॒नी स्या᳚थ् स॒हस्रं॑ प॒रस्ता॒देत॒मिति॒ यैव वरः॑ - [  ] \newline

\textbf{Pada Paata} \newline

वरः॑ । देयः॑ । सा । हि । वरः॑ । स॒हस्र᳚म् । अ॒स्य॒ । सा । द॒त्ता । भ॒व॒ति॒ । तस्मा᳚त् । वरः॑ । न । प्र॒ति॒गृह्य॒ इति॑ प्रति - गृह्यः॑ । सा । हि । वरः॑ । स॒हस्र᳚म् । अ॒स्य॒ । प्रति॑गृहीत॒मिति॒ प्रति॑ - गृ॒ही॒त॒म् । भ॒व॒ति॒ । इ॒यम् । वरः॑ । इति॑ । ब्रू॒या॒त् । अथ॑ । अ॒न्याम् । ब्रू॒या॒त् । इ॒यम् । मम॑ । इति॑ । तथा᳚ । अ॒स्य॒ । तत् । स॒हस्र᳚म् । अप्र॑तिगृहीत॒मित्यप्र॑ति - गृ॒ही॒त॒म् । भ॒व॒ति॒ । उ॒भ॒य॒त॒ए॒नीत्यु॑भयतः - ए॒नी । स्या॒त् । तत् । आ॒हुः॒ । अ॒न्य॒त॒ए॒नीत्य॑न्यतः - ए॒नी । स्या॒त् । स॒हस्र᳚म् । प॒रस्ता᳚त् । एत᳚म् । इति॑ । या । ए॒व । वरः॑ ।  \newline


\textbf{Krama Paata} \newline

वरो॒ देयः॑ । देयः॒ सा । सा हि । हि वरः॑ । वरः॑ स॒हस्र᳚म् । स॒हस्र॑मस्य । अ॒स्य॒ सा । सा द॒त्ता । द॒त्ता भ॑वति । भ॒व॒ति॒ तस्मा᳚त् । तस्मा॒द् वरः॑ । वरो॒ न । न प्र॑ति॒गृह्यः॑ । प्र॒ति॒गृह्यः॒ सा । प्र॒ति॒गृह्य॒ इति॑ प्रति - गृह्यः॑ । सा हि । हि वरः॑ । वरः॑ स॒हस्र᳚म् । स॒हस्र॑मस्य । अ॒स्य॒ प्रति॑गृहीतम् । प्रति॑गृहीतम् भवति । प्रति॑गृहीत॒मिति॒ प्रति॑ - गृ॒ही॒त॒म् । भ॒व॒ती॒यम् । इ॒यम् ॅवरः॑ । वर॒ इति॑ । इति॑ ब्रूयात् । ब्रू॒या॒दथ॑ । अथा॒न्याम् । अ॒न्याम् ब्रू॑यात् । ब्रू॒या॒दि॒यम् । इ॒यम् मम॑ । ममेति॑ । इति॒ तथा᳚ । तथा᳚ऽस्य । अ॒स्य॒ तत् । तथ् स॒हस्र᳚म् । स॒हस्र॒मप्र॑तिगृहीतम् । अप्र॑तिगृहीतम् भवति । अप्र॑तिगृहीत॒मित्यप्र॑ति - गृ॒ही॒त॒म् । भ॒व॒त्यु॒भ॒य॒त॒,ए॒नी । उ॒भ॒य॒त॒,ए॒नी स्या᳚त् । उ॒भ॒य॒त॒,ए॒नीत्यु॑भयतः - ए॒नी । स्या॒त् तत् । तदा॑हुः । आ॒हु॒र॒न्य॒त॒,ए॒नी । अ॒न्य॒त॒,ए॒नी स्या᳚त् । अ॒न्य॒त॒,ए॒नीत्य॑न्यतः - ए॒नी । स्या॒थ् स॒हस्र᳚म् । स॒हस्र॑म् प॒रस्ता᳚त् । प॒रस्ता॒देत᳚म् । एत॒मिति॑ । इति॒ या । यैव । ए॒व वरः॑ । वरः॑ कल्या॒णी \newline

\textbf{Jatai Paata} \newline

1. वरो॒ देयो॒ देयो॒ वरो॒ वरो॒ देयः॑ । \newline
2. देयः॒ सा सा देयो॒ देयः॒ सा । \newline
3. सा हि हि सा सा हि । \newline
4. हि वरो॒ वरो॒ हि हि वरः॑ । \newline
5. वरः॑ स॒हस्रꣳ॑ स॒हस्रं॒ ॅवरो॒ वरः॑ स॒हस्र᳚म् । \newline
6. स॒हस्र॑ मस्यास्य स॒हस्रꣳ॑ स॒हस्र॑ मस्य । \newline
7. अ॒स्य॒ सा सा ऽस्या᳚स्य॒ सा । \newline
8. सा द॒त्ता द॒त्ता सा सा द॒त्ता । \newline
9. द॒त्ता भ॑वति भवति द॒त्ता द॒त्ता भ॑वति । \newline
10. भ॒व॒ति॒ तस्मा॒त् तस्मा᳚द् भवति भवति॒ तस्मा᳚त् । \newline
11. तस्मा॒द् वरो॒ वर॒ स्तस्मा॒त् तस्मा॒द् वरः॑ । \newline
12. वरो॒ न न वरो॒ वरो॒ न । \newline
13. न प्र॑ति॒गृह्यः॑ प्रति॒गृह्यो॒ न न प्र॑ति॒गृह्यः॑ । \newline
14. प्र॒ति॒गृह्यः॒ सा सा प्र॑ति॒गृह्यः॑ प्रति॒गृह्यः॒ सा । \newline
15. प्र॒ति॒गृह्य॒ इति॑ प्रति - गृह्यः॑ । \newline
16. सा हि हि सा सा हि । \newline
17. हि वरो॒ वरो॒ हि हि वरः॑ । \newline
18. वरः॑ स॒हस्रꣳ॑ स॒हस्रं॒ ॅवरो॒ वरः॑ स॒हस्र᳚म् । \newline
19. स॒हस्र॑ मस्यास्य स॒हस्रꣳ॑ स॒हस्र॑ मस्य । \newline
20. अ॒स्य॒ प्रति॑गृहीत॒म् प्रति॑गृहीत मस्यास्य॒ प्रति॑गृहीतम् । \newline
21. प्रति॑गृहीतम् भवति भवति॒ प्रति॑गृहीत॒म् प्रति॑गृहीतम् भवति । \newline
22. प्रति॑गृहीत॒मिति॒ प्रति॑ - गृ॒ही॒त॒म् । \newline
23. भ॒व॒ती॒य मि॒यम् भ॑वति भवती॒यम् । \newline
24. इ॒यं ॅवरो॒ वर॑ इ॒य मि॒यं ॅवरः॑ । \newline
25. वर॒ इतीति॒ वरो॒ वर॒ इति॑ । \newline
26. इति॑ ब्रूयाद् ब्रूया॒ दितीति॑ ब्रूयात् । \newline
27. ब्रू॒या॒ दथाथ॑ ब्रूयाद् ब्रूया॒ दथ॑ । \newline
28. अथा॒न्या म॒न्या मथा था॒न्याम् । \newline
29. अ॒न्याम् ब्रू॑याद् ब्रूया द॒न्या म॒न्याम् ब्रू॑यात् । \newline
30. ब्रू॒या॒ दि॒य मि॒यम् ब्रू॑याद् ब्रूया दि॒यम् । \newline
31. इ॒यम् मम॒ ममे॒य मि॒यम् मम॑ । \newline
32. ममेतीति॒ मम॒ ममेति॑ । \newline
33. इति॒ तथा॒ तथेतीति॒ तथा᳚ । \newline
34. तथा᳚ ऽस्यास्य॒ तथा॒ तथा᳚ ऽस्य । \newline
35. अ॒स्य॒ तत् तद॑स्यास्य॒ तत् । \newline
36. तथ् स॒हस्रꣳ॑ स॒हस्र॒म् तत् तथ् स॒हस्र᳚म् । \newline
37. स॒हस्र॒ मप्र॑तिगृहीत॒ मप्र॑तिगृहीतꣳ स॒हस्रꣳ॑ स॒हस्र॒ मप्र॑तिगृहीतम् । \newline
38. अप्र॑तिगृहीतम् भवति भव॒ त्यप्र॑तिगृहीत॒ मप्र॑तिगृहीतम् भवति । \newline
39. अप्र॑तिगृहीत॒मित्यप्र॑ति - गृ॒ही॒त॒म् । \newline
40. भ॒व॒ त्यु॒भ॒य॒त॒ए॒ न्यु॑भयतए॒नी भ॑वति भव त्युभयतए॒नी । \newline
41. उ॒भ॒य॒त॒ए॒नी स्या᳚थ् स्या दुभयतए॒ न्यु॑भयतए॒नी स्या᳚त् । \newline
42. उ॒भ॒य॒त॒ए॒नीत्यु॑भयतः - ए॒नी । \newline
43. स्या॒त् तत् तथ् स्या᳚थ् स्या॒त् तत् । \newline
44. तदा॑हु राहु॒ स्तत् तदा॑हुः । \newline
45. आ॒हु॒ र॒न्य॒त॒ए॒ न्य॑न्यतए॒ न्या॑हु राहु रन्यतए॒नी । \newline
46. अ॒न्य॒त॒ए॒नी स्या᳚थ् स्या दन्यतए॒ न्य॑न्यतए॒नी स्या᳚त् । \newline
47. अ॒न्य॒त॒ए॒नीत्य॑न्यतः - ए॒नी । \newline
48. स्या॒थ् स॒हस्रꣳ॑ स॒हस्रꣳ॑ स्याथ् स्याथ् स॒हस्र᳚म् । \newline
49. स॒हस्र॑म् प॒रस्ता᳚त् प॒रस्ता᳚थ् स॒हस्रꣳ॑ स॒हस्र॑म् प॒रस्ता᳚त् । \newline
50. प॒रस्ता॒ देत॒ मेत॑म् प॒रस्ता᳚त् प॒रस्ता॒ देत᳚म् । \newline
51. एत॒ मिती त्येत॒ मेत॒ मिति॑ । \newline
52. इति॒ या येतीति॒ या । \newline
53. यैवैव या यैव । \newline
54. ए॒व वरो॒ वर॑ ए॒वैव वरः॑ । \newline
55. वरः॑ कल्या॒णी क॑ल्या॒णी वरो॒ वरः॑ कल्या॒णी । \newline

\textbf{Ghana Paata } \newline

1. वरो॒ देयो॒ देयो॒ वरो॒ वरो॒ देयः॒ सा सा देयो॒ वरो॒ वरो॒ देयः॒ सा । \newline
2. देयः॒ सा सा देयो॒ देयः॒ सा हि हि सा देयो॒ देयः॒ सा हि । \newline
3. सा हि हि सा सा हि वरो॒ वरो॒ हि सा सा हि वरः॑ । \newline
4. हि वरो॒ वरो॒ हि हि वरः॑ स॒हस्रꣳ॑ स॒हस्रं॒ ॅवरो॒ हि हि वरः॑ स॒हस्र᳚म् । \newline
5. वरः॑ स॒हस्रꣳ॑ स॒हस्रं॒ ॅवरो॒ वरः॑ स॒हस्र॑ मस्यास्य स॒हस्रं॒ ॅवरो॒ वरः॑ स॒हस्र॑ मस्य । \newline
6. स॒हस्र॑ मस्यास्य स॒हस्रꣳ॑ स॒हस्र॑ मस्य॒ सा सा ऽस्य॑ स॒हस्रꣳ॑ स॒हस्र॑ मस्य॒ सा । \newline
7. अ॒स्य॒ सा सा ऽस्या᳚स्य॒ सा द॒त्ता द॒त्ता सा ऽस्या᳚स्य॒ सा द॒त्ता । \newline
8. सा द॒त्ता द॒त्ता सा सा द॒त्ता भ॑वति भवति द॒त्ता सा सा द॒त्ता भ॑वति । \newline
9. द॒त्ता भ॑वति भवति द॒त्ता द॒त्ता भ॑वति॒ तस्मा॒त् तस्मा᳚द् भवति द॒त्ता द॒त्ता भ॑वति॒ तस्मा᳚त् । \newline
10. भ॒व॒ति॒ तस्मा॒त् तस्मा᳚द् भवति भवति॒ तस्मा॒द् वरो॒ वर॒ स्तस्मा᳚द् भवति भवति॒ तस्मा॒द् वरः॑ । \newline
11. तस्मा॒द् वरो॒ वर॒ स्तस्मा॒त् तस्मा॒द् वरो॒ न न वर॒ स्तस्मा॒त् तस्मा॒द् वरो॒ न । \newline
12. वरो॒ न न वरो॒ वरो॒ न प्र॑ति॒गृह्यः॑ प्रति॒गृह्यो॒ न वरो॒ वरो॒ न प्र॑ति॒गृह्यः॑ । \newline
13. न प्र॑ति॒गृह्यः॑ प्रति॒गृह्यो॒ न न प्र॑ति॒गृह्यः॒ सा सा प्र॑ति॒गृह्यो॒ न न प्र॑ति॒गृह्यः॒ सा । \newline
14. प्र॒ति॒गृह्यः॒ सा सा प्र॑ति॒गृह्यः॑ प्रति॒गृह्यः॒ सा हि हि सा प्र॑ति॒गृह्यः॑ प्रति॒गृह्यः॒ सा हि । \newline
15. प्र॒ति॒गृह्य॒ इति॑ प्रति - गृह्यः॑ । \newline
16. सा हि हि सा सा हि वरो॒ वरो॒ हि सा सा हि वरः॑ । \newline
17. हि वरो॒ वरो॒ हि हि वरः॑ स॒हस्रꣳ॑ स॒हस्रं॒ ॅवरो॒ हि हि वरः॑ स॒हस्र᳚म् । \newline
18. वरः॑ स॒हस्रꣳ॑ स॒हस्रं॒ ॅवरो॒ वरः॑ स॒हस्र॑ मस्यास्य स॒हस्रं॒ ॅवरो॒ वरः॑ स॒हस्र॑ मस्य । \newline
19. स॒हस्र॑ मस्यास्य स॒हस्रꣳ॑ स॒हस्र॑ मस्य॒ प्रति॑गृहीत॒म् प्रति॑गृहीत मस्य स॒हस्रꣳ॑ स॒हस्र॑ मस्य॒ प्रति॑गृहीतम् । \newline
20. अ॒स्य॒ प्रति॑गृहीत॒म् प्रति॑गृहीत मस्यास्य॒ प्रति॑गृहीतम् भवति भवति॒ प्रति॑गृहीत मस्यास्य॒ प्रति॑गृहीतम् भवति । \newline
21. प्रति॑गृहीतम् भवति भवति॒ प्रति॑गृहीत॒म् प्रति॑गृहीतम् भव ती॒य मि॒यम् भ॑वति॒ प्रति॑गृहीत॒म् प्रति॑गृहीतम् भव ती॒यम् । \newline
22. प्रति॑गृहीत॒मिति॒ प्रति॑ - गृ॒ही॒त॒म् । \newline
23. भ॒व॒ ती॒य मि॒यम् भ॑वति भव ती॒यं ॅवरो॒ वर॑ इ॒यम् भ॑वति भव ती॒यं ॅवरः॑ । \newline
24. इ॒यं ॅवरो॒ वर॑ इ॒य मि॒यं ॅवर॒ इतीति॒ वर॑ इ॒य मि॒यं ॅवर॒ इति॑ । \newline
25. वर॒ इतीति॒ वरो॒ वर॒ इति॑ ब्रूयाद् ब्रूया॒ दिति॒ वरो॒ वर॒ इति॑ ब्रूयात् । \newline
26. इति॑ ब्रूयाद् ब्रूया॒ दितीति॑ ब्रूया॒ दथाथ॑ ब्रूया॒ दितीति॑ ब्रूया॒ दथ॑ । \newline
27. ब्रू॒या॒ दथाथ॑ ब्रूयाद् ब्रूया॒ दथा॒न्या म॒न्या मथ॑ ब्रूयाद् ब्रूया॒ दथा॒न्याम् । \newline
28. अथा॒न्या म॒न्या मथा था॒न्याम् ब्रू॑याद् ब्रूया द॒न्या मथा था॒न्याम् ब्रू॑यात् । \newline
29. अ॒न्याम् ब्रू॑याद् ब्रूया द॒न्या म॒न्याम् ब्रू॑या दि॒य मि॒यम् ब्रू॑या द॒न्या म॒न्याम् ब्रू॑या दि॒यम् । \newline
30. ब्रू॒या॒ दि॒य मि॒यम् ब्रू॑याद् ब्रूया दि॒यम् मम॒ ममे॒ यम् ब्रू॑याद् ब्रूया दि॒यम् मम॑ । \newline
31. इ॒यम् मम॒ ममे॒य मि॒यम् ममे तीति॒ ममे॒य मि॒यम् ममेति॑ । \newline
32. ममे तीति॒ मम॒ ममेति॒ तथा॒ तथेति॒ मम॒ ममेति॒ तथा᳚ । \newline
33. इति॒ तथा॒ तथेतीति॒ तथा᳚ ऽस्यास्य॒ तथेतीति॒ तथा᳚ ऽस्य । \newline
34. तथा᳚ ऽस्यास्य॒ तथा॒ तथा᳚ ऽस्य॒ तत् तद॑स्य॒ तथा॒ तथा᳚ ऽस्य॒ तत् । \newline
35. अ॒स्य॒ तत् तद॑ स्यास्य॒ तथ् स॒हस्रꣳ॑ स॒हस्र॒म् तद॑ स्यास्य॒ तथ् स॒हस्र᳚म् । \newline
36. तथ् स॒हस्रꣳ॑ स॒हस्र॒म् तत् तथ् स॒हस्र॒ मप्र॑तिगृहीत॒ मप्र॑तिगृहीतꣳ स॒हस्र॒म् तत् तथ् स॒हस्र॒ मप्र॑तिगृहीतम् । \newline
37. स॒हस्र॒ मप्र॑तिगृहीत॒ मप्र॑तिगृहीतꣳ स॒हस्रꣳ॑ स॒हस्र॒ मप्र॑तिगृहीतम् भवति भव॒ त्यप्र॑तिगृहीतꣳ स॒हस्रꣳ॑ स॒हस्र॒ मप्र॑तिगृहीतम् भवति । \newline
38. अप्र॑तिगृहीतम् भवति भव॒ त्यप्र॑तिगृहीत॒ मप्र॑तिगृहीतम् भव त्युभयतए॒ न्यु॑भयतए॒नी भ॑व॒ त्यप्र॑तिगृहीत॒ मप्र॑तिगृहीतम् भव त्युभयतए॒नी । \newline
39. अप्र॑तिगृहीत॒मित्यप्र॑ति - गृ॒ही॒त॒म् । \newline
40. भ॒व॒ त्यु॒भ॒य॒त॒ए॒ न्यु॑भयतए॒नी भ॑वति भव त्युभयतए॒नी स्या᳚थ् स्या दुभयतए॒नी भ॑वति भव त्युभयतए॒नी स्या᳚त् । \newline
41. उ॒भ॒य॒त॒ए॒नी स्या᳚थ् स्या दुभयतए॒ न्यु॑भयतए॒नी स्या॒त् तत् तथ् स्या॑ दुभयतए॒ न्यु॑भयतए॒नी स्या॒त् तत् । \newline
42. उ॒भ॒य॒त॒ए॒नीत्यु॑भयतः - ए॒नी । \newline
43. स्या॒त् तत् तथ् स्या᳚थ् स्या॒त् तदा॑हु राहु॒ स्तथ् स्या᳚थ् स्या॒त् तदा॑हुः । \newline
44. तदा॑हु राहु॒ स्तत् तदा॑हु रन्यतए॒ न्य॑न्यतए॒ न्या॑हु॒ स्तत् तदा॑हु रन्यतए॒नी । \newline
45. आ॒हु॒ र॒न्य॒त॒ए॒ न्य॑न्यतए॒ न्या॑हु राहुर न्यतए॒नी स्या᳚थ् स्या दन्यतए॒ न्या॑हु राहु रन्यतए॒नी स्या᳚त् । \newline
46. अ॒न्य॒त॒ए॒नी स्या᳚थ् स्या दन्यतए॒ न्य॑न्यतए॒नी स्या᳚थ् स॒हस्रꣳ॑ स॒हस्रꣳ॑ स्या दन्यतए॒ न्य॑न्यतए॒नी स्या᳚थ् स॒हस्र᳚म् । \newline
47. अ॒न्य॒त॒ए॒नीत्य॑न्यतः - ए॒नी । \newline
48. स्या॒थ् स॒हस्रꣳ॑ स॒हस्रꣳ॑ स्याथ् स्याथ् स॒हस्र॑म् प॒रस्ता᳚त् प॒रस्ता᳚थ् स॒हस्रꣳ॑ स्याथ् स्याथ् स॒हस्र॑म् प॒रस्ता᳚त् । \newline
49. स॒हस्र॑म् प॒रस्ता᳚त् प॒रस्ता᳚थ् स॒हस्रꣳ॑ स॒हस्र॑म् प॒रस्ता॒ देत॒ मेत॑म् प॒रस्ता᳚थ् स॒हस्रꣳ॑ स॒हस्र॑म् प॒रस्ता॒ देत᳚म् । \newline
50. प॒रस्ता॒ देत॒ मेत॑म् प॒रस्ता᳚त् प॒रस्ता॒ देत॒ मितीत्येत॑म् प॒रस्ता᳚त् प॒रस्ता॒ देत॒ मिति॑ । \newline
51. एत॒ मिती त्येत॒ मेत॒ मिति॒ या येत्येत॒ मेत॒ मिति॒ या । \newline
52. इति॒ या येतीति॒ यैवैव येतीति॒ यैव । \newline
53. यैवैव या यैव वरो॒ वर॑ ए॒व या यैव वरः॑ । \newline
54. ए॒व वरो॒ वर॑ ए॒वैव वरः॑ कल्या॒णी क॑ल्या॒णी वर॑ ए॒वैव वरः॑ कल्या॒णी । \newline
55. वरः॑ कल्या॒णी क॑ल्या॒णी वरो॒ वरः॑ कल्या॒णी रू॒पस॑मृद्धा रू॒पस॑मृद्धा कल्या॒णी वरो॒ वरः॑ कल्या॒णी रू॒पस॑मृद्धा । \newline
\pagebreak
\markright{ TS 7.1.6.6  \hfill https://www.vedavms.in \hfill}

\section{ TS 7.1.6.6 }

\textbf{TS 7.1.6.6 } \newline
\textbf{Samhita Paata} \newline

कल्या॒णी रू॒पस॑मृद्धा॒ सा स्या॒थ् सा हि वरः॒ समृ॑द्ध्यै॒ तामुत्त॑रे॒णाऽऽ*ग्नी᳚द्ध्रं पर्या॒णीया॑ऽऽ*हव॒नीय॒स्यान्ते᳚ द्रोणकल॒शमव॑ घ्रापये॒दा जि॑घ्र क॒लशं॑ मह्यु॒रुधा॑रा॒ पय॑स्व॒त्या त्वा॑ विश॒न्त्विन्द॑वः समु॒द्रमि॑व॒ सिन्ध॑वः॒ सा मा॑ स॒हस्र॒ आ भ॑ज प्र॒जया॑ प॒शुभिः॑ स॒ह पुन॒र्मा ऽऽ*वि॑शताद्-र॒यिरिति॑ प्र॒जयै॒वैनं॑ प॒शुभी॑ र॒य्या स - [  ] \newline

\textbf{Pada Paata} \newline

क॒ल्या॒णी । रू॒पस॑मृ॒द्धेति॑ रू॒प - स॒मृ॒द्धा॒ । सा । स्या॒त् । सा । हि । वरः॑ । समृ॑द्ध्या॒ इति॒ सं - ऋ॒द्ध्यै॒ । ताम् । उत्त॑रे॒णेत्युत् - त॒रे॒ण॒ । आग्नी᳚द्ध्र॒मित्याग्नि॑ - इ॒द्ध्र॒म् । प॒र्या॒णीयेति॑ परि - आ॒नीय॑ । आ॒ह॒व॒नीय॒स्येत्या᳚ - ह॒व॒नीय॑स्य । अन्ते᳚ । द्रो॒ण॒क॒ल॒शमिति॑ द्रोण - क॒ल॒शम् । अवेति॑ । घ्रा॒प॒ये॒त् । एति॑ । जि॒घ्र॒ । क॒लश᳚म् । म॒हि॒ । उ॒रुधा॒रेत्यु॒रु - धा॒रा॒ । पय॑स्वती । एति॑ । त्वा॒ । वि॒श॒न्तु॒ । इन्द॑वः । स॒मु॒द्रम् । इ॒व॒ । सिन्ध॑वः । सा । मा॒ । स॒हस्रे᳚ । एति॑ । भ॒ज॒ । प्र॒जयेति॑ प्र - जया᳚ । प॒शुभि॒रिति॑ प॒शु-भिः॒ । स॒ह । पुनः॑ । मा॒ । एति॑ । वि॒श॒ता॒त् । र॒यिः । इति॑ । प्र॒जयेति॑ प्र - जया᳚ । ए॒व । ए॒न॒म् । प॒शुभि॒रिति॑ प॒शु - भिः॒ । र॒य्या । समिति॑ ।  \newline


\textbf{Krama Paata} \newline

क॒ल्या॒णी रू॒पस॑मृद्धा । रू॒पस॑मृद्धा॒ सा । रू॒पस॑मृ॒द्धेति॑ रू॒प - स॒मृ॒द्धा॒ । सा स्या᳚त् । स्या॒थ् सा । सा हि । हि वरः॑ । वरः॒ समृ॑द्ध्यै । समृ॑द्ध्यै॒ ताम् । समृ॑द्ध्या॒ इति॒ सम् - ऋ॒द्ध्यै॒ । तामुत्त॑रेण । उत्त॑रे॒णाग्नी᳚द्ध्रम् । उत्त॑रे॒णेत्युत् - त॒रे॒ण॒ । आग्नी᳚द्ध्रम् पर्या॒णीय॑ । आग्नी᳚द्ध्र॒मित्याग्नि॑ - इ॒द्ध्र॒म् । प॒र्या॒णीया॑हव॒नीय॑स्य । प॒र्या॒णीयेति॑ परि - आ॒नीय॑ । आ॒ह॒व॒नीय॒स्यान्ते᳚ । आ॒ह॒व॒नीय॒स्येत्या᳚ - ह॒व॒नीय॑स्य । अन्ते᳚ द्रोणकल॒शम् । द्रो॒ण॒क॒ल॒शमव॑ । द्रो॒ण॒क॒ल॒शमिति॑ द्रोण - क॒ल॒शम् । अव॑ घ्रापयेत् । घ्रा॒प॒ये॒दा । आ जि॑घ्र । जि॒घ्र॒ क॒लश᳚म् । क॒लश॑म् महि । म॒ह्यु॒रुधा॑रा । उ॒रुधा॑रा॒ पय॑स्वती । उ॒रुधा॒रेत्यु॒रु - धा॒रा॒ । पय॑स्व॒त्या । आ त्वा᳚ । त्वा॒ वि॒श॒न्तु॒ । वि॒श॒न्त्विन्द॑वः । इन्द॑वः समु॒द्रम् । स॒मु॒द्रमि॑व । इ॒व॒ सिन्ध॑वः । सिन्ध॑वः॒ सा । सा मा᳚ । मा॒ स॒हस्रे᳚ । स॒हस्र॒ आ । आ भ॑ज । भ॒ज॒ प्र॒जया᳚ । प्र॒जया॑ प॒शुभिः॑ । प्र॒जयेति॑ प्र - जया᳚ । प॒शुभिः॑ स॒ह । प॒शुभि॒रिति॑ प॒शु - भिः॒ । स॒ह पुनः॑ । पुन॑र् मा । मा । आ वि॑शतात् । वि॒श॒ता॒द् र॒यिः । र॒यिरिति॑ । इति॑ प्र॒जया᳚ । प्र॒जयै॒व । प्र॒जयेति॑ प्र - जया᳚ । ए॒वैन᳚म् । ए॒न॒म् प॒शुभिः॑ । प॒शुभी॑ र॒य्या । प॒शभि॒रिति॑ प॒शु - भिः॒ । र॒य्या सम् । सम॑र्द्धयति \newline

\textbf{Jatai Paata} \newline

1. क॒ल्या॒णी रू॒पस॑मृद्धा रू॒पस॑मृद्धा कल्या॒णी क॑ल्या॒णी रू॒पस॑मृद्धा । \newline
2. रू॒पस॑मृद्धा॒ सा सा रू॒पस॑मृद्धा रू॒पस॑मृद्धा॒ सा । \newline
3. रू॒पस॑मृ॒द्धेति॑ रू॒प - स॒मृ॒द्धा॒ । \newline
4. सा स्या᳚थ् स्या॒थ् सा सा स्या᳚त् । \newline
5. स्या॒थ् सा सा स्या᳚थ् स्या॒थ् सा । \newline
6. सा हि हि सा सा हि । \newline
7. हि वरो॒ वरो॒ हि हि वरः॑ । \newline
8. वरः॒ समृ॑द्ध्यै॒ समृ॑द्ध्यै॒ वरो॒ वरः॒ समृ॑द्ध्यै । \newline
9. समृ॑द्ध्यै॒ ताम् ताꣳ समृ॑द्ध्यै॒ समृ॑द्ध्यै॒ ताम् । \newline
10. समृ॑द्ध्या॒ इति॒ सं - ऋ॒द्ध्यै॒ । \newline
11. ता मुत्त॑रे॒ णोत्त॑रेण॒ ताम् ता मुत्त॑रेण । \newline
12. उत्त॑रे॒णा ग्नी᳚द्ध्र॒ माग्नी᳚द्ध्र॒ मुत्त॑रे॒ णोत्त॑रे॒णा ग्नी᳚द्ध्रम् । \newline
13. उत्त॑रे॒णेत्युत् - त॒रे॒ण॒ । \newline
14. आग्नी᳚द्ध्रम् पर्या॒णीय॑ पर्या॒णीया ग्नी᳚द्ध्र॒ माग्नी᳚द्ध्रम् पर्या॒णीय॑ । \newline
15. आग्नी᳚द्ध्र॒मित्याग्नि॑ - इ॒द्ध्र॒म् । \newline
16. प॒र्या॒णीया॑ हव॒नीय॑स्या हव॒नीय॑स्य पर्या॒णीय॑ पर्या॒णीया॑ हव॒नीय॑स्य । \newline
17. प॒र्या॒णीयेति॑ परि - आ॒नीय॑ । \newline
18. आ॒ह॒व॒नीय॒ स्यान्ते ऽन्त॑ आहव॒नीय॑ स्याहव॒नीय॒ स्यान्ते᳚ । \newline
19. आ॒ह॒व॒नीय॒स्येत्या᳚ - ह॒व॒नीय॑स्य । \newline
20. अन्ते᳚ द्रोणकल॒शम् द्रो॑णकल॒श मन्ते ऽन्ते᳚ द्रोणकल॒शम् । \newline
21. द्रो॒ण॒क॒ल॒श मवाव॑ द्रोणकल॒शम् द्रो॑णकल॒श मव॑ । \newline
22. द्रो॒ण॒क॒ल॒शमिति॑ द्रोण - क॒ल॒शम् । \newline
23. अव॑ घ्रापयेद् घ्रापये॒ दवाव॑ घ्रापयेत् । \newline
24. घ्रा॒प॒ये॒दा घ्रा॑पयेद् घ्रापये॒दा । \newline
25. आ जि॑घ्र जि॒घ्रा जि॑घ्र । \newline
26. जि॒घ्र॒ क॒लश॑म् क॒लश॑म् जिघ्र जिघ्र क॒लश᳚म् । \newline
27. क॒लश॑म् महि महि क॒लश॑म् क॒लश॑म् महि । \newline
28. म॒ह्यु॒रुधा॑रो॒ रुधा॑रा महि मह्यु॒रुधा॑रा । \newline
29. उ॒रुधा॑रा॒ पय॑स्वती॒ पय॑स्व त्यु॒रुधा॑रो॒ रुधा॑रा॒ पय॑स्वती । \newline
30. उ॒रुधा॒रेत्यु॒रु - धा॒रा॒ । \newline
31. पय॑स्व॒त्या पय॑स्वती॒ पय॑स्व॒त्या । \newline
32. आ त्वा॒ त्वा ऽऽत्वा᳚ । \newline
33. त्वा॒ वि॒श॒न्तु॒ वि॒श॒न्तु॒ त्वा॒ त्वा॒ वि॒श॒न्तु॒ । \newline
34. वि॒श॒ न्त्विन्द॑व॒ इन्द॑वो विशन्तु विश॒ न्त्विन्द॑वः । \newline
35. इन्द॑वः समु॒द्रꣳ स॑मु॒द्र मिन्द॑व॒ इन्द॑वः समु॒द्रम् । \newline
36. स॒मु॒द्र मि॑वेव समु॒द्रꣳ स॑मु॒द्र मि॑व । \newline
37. इ॒व॒ सिन्ध॑वः॒ सिन्ध॑व इवेव॒ सिन्ध॑वः । \newline
38. सिन्ध॑वः॒ सा सा सिन्ध॑वः॒ सिन्ध॑वः॒ सा । \newline
39. सा मा॑ मा॒ सा सा मा᳚ । \newline
40. मा॒ स॒हस्रे॑ स॒हस्रे॑ मा मा स॒हस्रे᳚ । \newline
41. स॒हस्र॒ आ स॒हस्रे॑ स॒हस्र॒ आ । \newline
42. आ भ॑ज भ॒जा भ॑ज । \newline
43. भ॒ज॒ प्र॒जया᳚ प्र॒जया॑ भज भज प्र॒जया᳚ । \newline
44. प्र॒जया॑ प॒शुभिः॑ प॒शुभिः॑ प्र॒जया᳚ प्र॒जया॑ प॒शुभिः॑ । \newline
45. प्र॒जयेति॑ प्र - जया᳚ । \newline
46. प॒शुभिः॑ स॒ह स॒ह प॒शुभिः॑ प॒शुभिः॑ स॒ह । \newline
47. प॒शुभि॒रिति॑ प॒शु - भिः॒ । \newline
48. स॒ह पुनः॒ पुनः॑ स॒ह स॒ह पुनः॑ । \newline
49. पुन॑र् मा मा॒ पुनः॒ पुन॑र् मा । \newline
50. मा ऽऽमा॒ मा । \newline
51. आ वि॑शताद् विशता॒दा वि॑शतात् । \newline
52. वि॒श॒ता॒द् र॒यी र॒यिर् वि॑शताद् विशताद् र॒यिः । \newline
53. र॒यि रितीति॑ र॒यी र॒यि रिति॑ । \newline
54. इति॑ प्र॒जया᳚ प्र॒जये तीति॑ प्र॒जया᳚ । \newline
55. प्र॒जयै॒वैव प्र॒जया᳚ प्र॒ज यै॒व । \newline
56. प्र॒जयेति॑ प्र - जया᳚ । \newline
57. ए॒वैन॑ मेन मे॒वैवैन᳚म् । \newline
58. ए॒न॒म् प॒शुभिः॑ प॒शुभि॑ रेन मेनम् प॒शुभिः॑ । \newline
59. प॒शुभी॑ र॒य्या र॒य्या प॒शुभिः॑ प॒शुभी॑ र॒य्या । \newline
60. प॒शुभि॒रिति॑ प॒शु - भिः॒ । \newline
61. र॒य्या सꣳ सꣳ र॒य्या र॒य्या सम् । \newline
62. स म॑र्द्धय त्यर्द्धयति॒ सꣳ स म॑र्द्धयति । \newline

\textbf{Ghana Paata } \newline

1. क॒ल्या॒णी रू॒पस॑मृद्धा रू॒पस॑मृद्धा कल्या॒णी क॑ल्या॒णी रू॒पस॑मृद्धा॒ सा सा रू॒पस॑मृद्धा कल्या॒णी क॑ल्या॒णी रू॒पस॑मृद्धा॒ सा । \newline
2. रू॒पस॑मृद्धा॒ सा सा रू॒पस॑मृद्धा रू॒पस॑मृद्धा॒ सा स्या᳚थ् स्या॒थ् सा रू॒पस॑मृद्धा रू॒पस॑मृद्धा॒ सा स्या᳚त् । \newline
3. रू॒पस॑मृ॒द्धेति॑ रू॒प - स॒मृ॒द्धा॒ । \newline
4. सा स्या᳚थ् स्या॒थ् सा सा स्या॒थ् सा सा स्या॒थ् सा सा स्या॒थ् सा । \newline
5. स्या॒थ् सा सा स्या᳚थ् स्या॒थ् सा हि हि सा स्या᳚थ् स्या॒थ् सा हि । \newline
6. सा हि हि सा सा हि वरो॒ वरो॒ हि सा सा हि वरः॑ । \newline
7. हि वरो॒ वरो॒ हि हि वरः॒ समृ॑द्ध्यै॒ समृ॑द्ध्यै॒ वरो॒ हि हि वरः॒ समृ॑द्ध्यै । \newline
8. वरः॒ समृ॑द्ध्यै॒ समृ॑द्ध्यै॒ वरो॒ वरः॒ समृ॑द्ध्यै॒ ताम् ताꣳ समृ॑द्ध्यै॒ वरो॒ वरः॒ समृ॑द्ध्यै॒ ताम् । \newline
9. समृ॑द्ध्यै॒ ताम् ताꣳ समृ॑द्ध्यै॒ समृ॑द्ध्यै॒ ता मुत्त॑रे॒ णोत्त॑रेण॒ ताꣳ समृ॑द्ध्यै॒ समृ॑द्ध्यै॒ ता मुत्त॑रेण । \newline
10. समृ॑द्ध्या॒ इति॒ सं - ऋ॒द्ध्यै॒ । \newline
11. ता मुत्त॑रे॒ णोत्त॑रेण॒ ताम् ता मुत्त॑रे॒णा ग्नी᳚द्ध्र॒ माग्नी᳚द्ध्र॒ मुत्त॑रेण॒ ताम् ता मुत्त॑रे॒णा ग्नी᳚द्ध्रम् । \newline
12. उत्त॑रे॒णा ग्नी᳚द्ध्र॒ माग्नी᳚द्ध्र॒ मुत्त॑रे॒ णोत्त॑रे॒णा ग्नी᳚द्ध्रम् पर्या॒णीय॑ पर्या॒णीया ग्नी᳚द्ध्र॒ मुत्त॑रे॒
णोत्त॑रे॒णा ग्नी᳚द्ध्रम् पर्या॒णीय॑ । \newline
13. उत्त॑रे॒णेत्युत् - त॒रे॒ण॒ । \newline
14. आग्नी᳚द्ध्रम् पर्या॒णीय॑ पर्या॒णीया ग्नी᳚द्ध्र॒ माग्नी᳚द्ध्रम् पर्या॒णीया॑ हव॒नीय॑स्या हव॒नीय॑स्य पर्या॒णीया ग्नी᳚द्ध्र॒ माग्नी᳚द्ध्रम् पर्या॒णीया॑ हव॒नीय॑स्य । \newline
15. आग्नी᳚द्ध्र॒मित्याग्नि॑ - इ॒द्ध्र॒म् । \newline
16. प॒र्या॒णीया॑ हव॒नीय॑स्या हव॒नीय॑स्य पर्या॒णीय॑ पर्या॒णीया॑ हव॒नीय॒स्या न्ते ऽन्त॑ आहव॒नीय॑स्य पर्या॒णीय॑ पर्या॒णीया॑ हव॒नीय॒स्या न्ते᳚ । \newline
17. प॒र्या॒णीयेति॑ परि - आ॒नीय॑ । \newline
18. आ॒ह॒व॒नीय॒स्या न्ते ऽन्त॑ आहव॒नीय॑स्या हव॒नीय॒स्या न्ते᳚ द्रोणकल॒शम् द्रो॑णकल॒श मन्त॑ आहव॒नीय॑स्या हव॒नीय॒स्या न्ते᳚ द्रोणकल॒शम् । \newline
19. आ॒ह॒व॒नीय॒स्येत्या᳚ - ह॒व॒नीय॑स्य । \newline
20. अन्ते᳚ द्रोणकल॒शम् द्रो॑णकल॒श मन्ते ऽन्ते᳚ द्रोणकल॒श मवाव॑ द्रोणकल॒श मन्ते ऽन्ते᳚ द्रोणकल॒श मव॑ । \newline
21. द्रो॒ण॒क॒ल॒श मवाव॑ द्रोणकल॒शम् द्रो॑णकल॒श मव॑ घ्रापयेद् घ्रापये॒ दव॑ द्रोणकल॒शम् द्रो॑णकल॒श मव॑ घ्रापयेत् । \newline
22. द्रो॒ण॒क॒ल॒शमिति॑ द्रोण - क॒ल॒शम् । \newline
23. अव॑ घ्रापयेद् घ्रापये॒ दवाव॑ घ्रापये॒दा घ्रा॑पये॒ दवाव॑ घ्रापये॒दा । \newline
24. घ्रा॒प॒ये॒दा घ्रा॑पयेद् घ्रापये॒दा जि॑घ्र जि॒घ्रा घ्रा॑पयेद् घ्रापये॒दा जि॑घ्र । \newline
25. आ जि॑घ्र जि॒घ्रा जि॑घ्र क॒लश॑म् क॒लश॑म् जि॒घ्रा जि॑घ्र क॒लश᳚म् । \newline
26. जि॒घ्र॒ क॒लश॑म् क॒लश॑म् जिघ्र जिघ्र क॒लश॑म् महि महि क॒लश॑म् जिघ्र जिघ्र क॒लश॑म् महि । \newline
27. क॒लश॑म् महि महि क॒लश॑म् क॒लश॑म् मह्यु॒रुधा॑ रो॒रुधा॑रा महि क॒लश॑म् क॒लश॑म् मह्यु॒रुधा॑रा । \newline
28. म॒ह्यु॒रुधा॑ रो॒रुधा॑रा महिमह्यु॒ रुधा॑रा॒ पय॑स्वती॒ पय॑स्व त्यु॒रुधा॑रा महि मह्यु॒रुधा॑रा॒ पय॑स्वती । \newline
29. उ॒रुधा॑रा॒ पय॑स्वती॒ पय॑स्व त्यु॒रुधा॑ रो॒रुधा॑रा॒ पय॑स्व॒त्या पय॑स्व त्यु॒रुधा॑ रो॒रुधा॑रा॒ पय॑स्व॒त्या । \newline
30. उ॒रुधा॒रेत्यु॒रु - धा॒रा॒ । \newline
31. पय॑स्व॒त्या पय॑स्वती॒ पय॑स्व॒त्या त्वा॒ त्वा ऽऽपय॑स्वती॒ पय॑स्व॒त्या त्वा᳚ । \newline
32. आ त्वा॒ त्वा ऽऽत्वा॑ विशन्तु विशन्तु॒ त्वा ऽऽत्वा॑ विशन्तु । \newline
33. त्वा॒ वि॒श॒न्तु॒ वि॒श॒न्तु॒ त्वा॒ त्वा॒ वि॒श॒ न्त्विन्द॑व॒ इन्द॑वो विशन्तु त्वा त्वा विश॒ न्त्विन्द॑वः । \newline
34. वि॒श॒ न्त्विन्द॑व॒ इन्द॑वो विशन्तु विश॒ न्त्विन्द॑वः समु॒द्रꣳ स॑मु॒द्र मिन्द॑वो विशन्तु विश॒ न्त्विन्द॑वः समु॒द्रम् । \newline
35. इन्द॑वः समु॒द्रꣳ स॑मु॒द्र मिन्द॑व॒ इन्द॑वः समु॒द्र मि॑वेव समु॒द्र मिन्द॑व॒ इन्द॑वः समु॒द्र मि॑व । \newline
36. स॒मु॒द्र मि॑वेव समु॒द्रꣳ स॑मु॒द्र मि॑व॒ सिन्ध॑वः॒ सिन्ध॑व इव समु॒द्रꣳ स॑मु॒द्र मि॑व॒ सिन्ध॑वः । \newline
37. इ॒व॒ सिन्ध॑वः॒ सिन्ध॑व इवेव॒ सिन्ध॑वः॒ सा सा सिन्ध॑व इवेव॒ सिन्ध॑वः॒ सा । \newline
38. सिन्ध॑वः॒ सा सा सिन्ध॑वः॒ सिन्ध॑वः॒ सा मा॑ मा॒ सा सिन्ध॑वः॒ सिन्ध॑वः॒ सा मा᳚ । \newline
39. सा मा॑ मा॒ सा सा मा॑ स॒हस्रे॑ स॒हस्रे॑ मा॒ सा सा मा॑ स॒हस्रे᳚ । \newline
40. मा॒ स॒हस्रे॑ स॒हस्रे॑ मा मा स॒हस्र॒ आ स॒हस्रे॑ मा मा स॒हस्र॒ आ । \newline
41. स॒हस्र॒ आ स॒हस्रे॑ स॒हस्र॒ आ भ॑ज भ॒जा स॒हस्रे॑ स॒हस्र॒ आ भ॑ज । \newline
42. आ भ॑ज भ॒जा भ॑ज प्र॒जया᳚ प्र॒जया॑ भ॒जा भ॑ज प्र॒जया᳚ । \newline
43. भ॒ज॒ प्र॒जया᳚ प्र॒जया॑ भज भज प्र॒जया॑ प॒शुभिः॑ प॒शुभिः॑ प्र॒जया॑ भज भज प्र॒जया॑ प॒शुभिः॑ । \newline
44. प्र॒जया॑ प॒शुभिः॑ प॒शुभिः॑ प्र॒जया᳚ प्र॒जया॑ प॒शुभिः॑ स॒ह स॒ह प॒शुभिः॑ प्र॒जया᳚ प्र॒जया॑ प॒शुभिः॑ स॒ह । \newline
45. प्र॒जयेति॑ प्र - जया᳚ । \newline
46. प॒शुभिः॑ स॒ह स॒ह प॒शुभिः॑ प॒शुभिः॑ स॒ह पुनः॒ पुनः॑ स॒ह प॒शुभिः॑ प॒शुभिः॑ स॒ह पुनः॑ । \newline
47. प॒शुभि॒रिति॑ प॒शु - भिः॒ । \newline
48. स॒ह पुनः॒ पुनः॑ स॒ह स॒ह पुन॑र् मा मा॒ पुनः॑ स॒ह स॒ह पुन॑र् मा । \newline
49. पुन॑र् मा मा॒ पुनः॒ पुन॒र् मा ऽऽमा॒ पुनः॒ पुन॒र् मा । \newline
50. मा ऽऽमा॒ मा ऽऽवि॑शताद् विशता॒दा मा॒ मा ऽऽवि॑शतात् । \newline
51. आ वि॑शताद् विशता॒दा वि॑शताद् र॒यी र॒यिर् वि॑शता॒दा वि॑शताद् र॒यिः । \newline
52. वि॒श॒ता॒द् र॒यी र॒यिर् वि॑शताद् विशताद् र॒यि रितीति॑ र॒यिर् वि॑शताद् विशताद् र॒यि रिति॑ । \newline
53. र॒यि रितीति॑ र॒यी र॒यि रिति॑ प्र॒जया᳚ प्र॒जयेति॑ र॒यी र॒यि रिति॑ प्र॒जया᳚ । \newline
54. इति॑ प्र॒जया᳚ प्र॒जयेतीति॑ प्र॒जयै॒ वैव प्र॒जयेतीति॑ प्र॒जयै॒व । \newline
55. प्र॒जयै॒ वैव प्र॒जया᳚ प्र॒जयै॒ वैन॑ मेन मे॒व प्र॒जया᳚ प्र॒जयै॒ वैन᳚म् । \newline
56. प्र॒जयेति॑ प्र - जया᳚ । \newline
57. ए॒वैन॑ मेन मे॒वै वैन॑म् प॒शुभिः॑ प॒शुभि॑ रेन मे॒वै वैन॑म् प॒शुभिः॑ । \newline
58. ए॒न॒म् प॒शुभिः॑ प॒शुभि॑ रेन मेनम् प॒शुभी॑ र॒य्या र॒य्या प॒शुभि॑ रेन मेनम् प॒शुभी॑ र॒य्या । \newline
59. प॒शुभी॑ र॒य्या र॒य्या प॒शुभिः॑ प॒शुभी॑ र॒य्या सꣳ सꣳ र॒य्या प॒शुभिः॑ प॒शुभी॑ र॒य्या सम् । \newline
60. प॒शुभि॒रिति॑ प॒शु - भिः॒ । \newline
61. र॒य्या सꣳ सꣳ र॒य्या र॒य्या स म॑र्द्धय त्यर्द्धयति॒ सꣳ र॒य्या र॒य्या स म॑र्द्धयति । \newline
62. स म॑र्द्धय त्यर्द्धयति॒ सꣳ स म॑र्द्धयति प्र॒जावा᳚न् प्र॒जावा॑ नर्द्धयति॒ सꣳ स म॑र्द्धयति प्र॒जावान्॑ । \newline
\pagebreak
\markright{ TS 7.1.6.7  \hfill https://www.vedavms.in \hfill}

\section{ TS 7.1.6.7 }

\textbf{TS 7.1.6.7 } \newline
\textbf{Samhita Paata} \newline

-म॑र्द्धयति प्र॒जावा᳚न् पशु॒मान् र॑यि॒मान् भ॑वति॒ य ए॒वं ॅवेद॒ तया॑ स॒हाऽऽ*ग्नी᳚द्ध्रं प॒रेत्य॑ पु॒रस्ता᳚त् प्र॒तीच्यां॒ तिष्ठ॑न्त्यां जुहुयादु॒भा जि॑ग्यथु॒र्न परा॑ जयेथे॒ न परा॑ जिग्ये कत॒रश्च॒नैनोः᳚ । इन्द्र॑श्च विष्णो॒ यदप॑स्पृधेथां त्रे॒धा स॒हस्रं॒ ॅवि तदै॑रयेथा॒मिति॑, त्रेधाविभ॒क्तं ॅवै त्रि॑रा॒त्रे स॒हस्रꣳ॑ साह॒स्रीमे॒वैनां᳚ करोति स॒हस्र॑स्यै॒वैनां॒ मात्रां᳚ - [  ] \newline

\textbf{Pada Paata} \newline

अ॒द्‌र्ध॒य॒ति॒ । प्र॒जावा॒निति॑ प्रजा - वा॒न् । प॒शु॒मानिति॑ पशु - मान् । र॒यि॒मानिति॑ रयि - मान् । भ॒व॒ति॒ । यः । ए॒वम् । वेद॑ । तया᳚ । स॒ह । आग्नी᳚द्ध्र॒मित्याग्नि॑ - इ॒द्ध्र॒म् । प॒रेत्येति॑ परा - इत्य॑ । पु॒रस्ता᳚त् । प्र॒तीच्या᳚म् । तिष्ठ॑न्त्याम् । जु॒हु॒या॒त् । उ॒भा । जि॒ग्य॒थुः॒ । न । परेति॑ । ज॒ये॒थे॒ इति॑ । न । परेति॑ । जि॒ग्ये॒ । क॒त॒रः । च॒न । ए॒नोः॒ ॥ इन्द्रः॑ । च॒ । वि॒ष्णो॒ इति॑ । यत् । अप॑स्पृधेथाम् । त्रे॒धा । स॒हस्र᳚म् । वीति॑ । तत् । ऐ॒र॒ये॒था॒म् । इति॑ । त्रे॒धा॒वि॒भ॒क्तमिति॑ त्रेधा - वि॒भ॒क्तम् । वै । त्रि॒रा॒त्र इति॑ त्रि - रा॒त्रे । स॒हस्र᳚म् । सा॒ह॒स्रीम् । ए॒व । ए॒ना॒म् । क॒रो॒ति॒ । स॒हस्र॑स्य । ए॒व । ए॒ना॒म् । मात्रा᳚म् ।  \newline


\textbf{Krama Paata} \newline

अ॒र्द्ध॒य॒ति॒ प्र॒जावान्॑ । प्र॒जावा᳚न् पशु॒मान् । प्र॒जावा॒निति॑ प्र॒जा - वा॒न्॒ । प॒शु॒मान् र॑यि॒मान् । प॒शु॒मानिति॑ पशु - मान् । र॒यि॒मान् भ॑वति । र॒यि॒मानिति॑ रयि - मान् । भ॒व॒ति॒ यः । य ए॒वम् । ए॒वम् ॅवेद॑ । वेद॒ तया᳚ । तया॑ स॒ह । स॒हाग्नी᳚द्ध्रम् । आग्नी᳚द्ध्रम् प॒रेत्य॑ । आग्नी᳚द्ध्र॒मित्याग्नि॑ - इ॒द्ध्र॒म् । प॒रेत्य॑ पु॒रस्ता᳚त् । प॒रेत्येति॑ परा - इत्य॑ । पु॒रस्ता᳚त् प्र॒तीच्या᳚म् । प्र॒तीच्या॒म् तिष्ठ॑न्त्याम् । तिष्ठ॑न्त्याम् जहुयात् । जु॒हु॒या॒दु॒भा । उ॒भा जि॑ग्यथुः । जि॒ग्य॒थु॒र् न । न परा᳚ । परा॑ जयेथे । ज॒ये॒थे॒ न । ज॒ये॒थे॒ इति॑ जयेथे । न परा᳚ । परा॑ जिग्ये । जि॒ग्ये॒ क॒त॒रः । क॒त॒रश्च॒न । च॒नैनोः᳚ । ए॒नो॒रित्ये॑नोः ॥ इन्द्र॑श्च । च॒ वि॒ष्णो॒ । वि॒ष्णो॒ यत् । वि॒ष्णो॒ इति॑ विष्णो । यदप॑स्पृधेथाम् । अप॑स्पृधेथाम् त्रे॒धा । त्रे॒धा स॒हस्र᳚म् । स॒हस्र॒म् ॅवि । वि तत् । तदै॑रयेथाम् । ऐ॒र॒ये॒था॒मिति॑ । इति॑ त्रेधाविभ॒क्तम् । त्रे॒धा॒वि॒भ॒क्तम् ॅवै । त्रे॒धा॒वि॒भ॒क्तमिति॑ त्रेधा - वि॒भ॒क्तम् । वै त्रि॑रा॒त्रे । त्रि॒रा॒त्रे स॒हस्र᳚म् । त्रि॒रा॒त्र इति॑ त्रि - रा॒त्रे । स॒हसꣳ॑ साह॒स्रीम् । सा॒ह॒स्रीमे॒व । ए॒वैना᳚म् । ए॒ना॒म् क॒रो॒ति॒ । क॒रो॒ति॒ स॒हस्र॑स्य । स॒हस्र॑स्यै॒व । ए॒वैना᳚म् । ए॒ना॒म् मात्रा᳚म् ( ) । मात्रा᳚म् करोति \newline

\textbf{Jatai Paata} \newline

1. अ॒र्द्ध॒य॒ति॒ प्र॒जावा᳚न् प्र॒जावा॑ नर्द्धय त्यर्द्धयति प्र॒जावान्॑ । \newline
2. प्र॒जावा᳚न् पशु॒मान् प॑शु॒मान् प्र॒जावा᳚न् प्र॒जावा᳚न् पशु॒मान् । \newline
3. प्र॒जावा॒निति॑ प्र॒जा - वा॒न् । \newline
4. प॒शु॒मान् र॑यि॒मान् र॑यि॒मान् प॑शु॒मान् प॑शु॒मान् र॑यि॒मान् । \newline
5. प॒शु॒मानिति॑ पशु - मान् । \newline
6. र॒यि॒मान् भ॑वति भवति रयि॒मान् र॑यि॒मान् भ॑वति । \newline
7. र॒यि॒मानिति॑ रयि - मान् । \newline
8. भ॒व॒ति॒ यो यो भ॑वति भवति॒ यः । \newline
9. य ए॒व मे॒वं ॅयो य ए॒वम् । \newline
10. ए॒वं ॅवेद॒ वेदै॒व मे॒वं ॅवेद॑ । \newline
11. वेद॒ तया॒ तया॒ वेद॒ वेद॒ तया᳚ । \newline
12. तया॑ स॒ह स॒ह तया॒ तया॑ स॒ह । \newline
13. स॒हा ग्नी᳚द्ध्र॒ माग्नी᳚द्ध्रꣳ स॒ह स॒हा ग्नी᳚द्ध्रम् । \newline
14. आग्नी᳚द्ध्रम् प॒रेत्य॑ प॒रेत्या ग्नी᳚द्ध्र॒ माग्नी᳚द्ध्रम् प॒रेत्य॑ । \newline
15. आग्नी᳚द्ध्र॒मित्याग्नि॑ - इ॒द्ध्र॒म् । \newline
16. प॒रेत्य॑ पु॒रस्ता᳚त् पु॒रस्ता᳚त् प॒रेत्य॑ प॒रेत्य॑ पु॒रस्ता᳚त् । \newline
17. प॒रेत्येति॑ परा - इत्य॑ । \newline
18. पु॒रस्ता᳚त् प्र॒तीच्या᳚म् प्र॒तीच्या᳚म् पु॒रस्ता᳚त् पु॒रस्ता᳚त् प्र॒तीच्या᳚म् । \newline
19. प्र॒तीच्या॒म् तिष्ठ॑न्त्या॒म् तिष्ठ॑न्त्याम् प्र॒तीच्या᳚म् प्र॒तीच्या॒म् तिष्ठ॑न्त्याम् । \newline
20. तिष्ठ॑न्त्याम् जुहुयाज् जुहुया॒त् तिष्ठ॑न्त्या॒म् तिष्ठ॑न्त्याम् जुहुयात् । \newline
21. जु॒हु॒या॒ दु॒भोभा जु॑हुयाज् जुहुया दु॒भा । \newline
22. उ॒भा जि॑ग्यथुर् जिग्यथु रु॒भोभा जि॑ग्यथुः । \newline
23. जि॒ग्य॒थु॒र् न न जि॑ग्यथुर् जिग्यथु॒र् न । \newline
24. न परा॒ परा॒ न न परा᳚ । \newline
25. परा॑ जयेथे जयेथे॒ परा॒ परा॑ जयेथे । \newline
26. ज॒ये॒थे॒ न न ज॑येथे जयेथे॒ न । \newline
27. ज॒ये॒थे॒ इति॑ जयेथे । \newline
28. न परा॒ परा॒ न न परा᳚ । \newline
29. परा॑ जिग्ये जिग्ये॒ परा॒ परा॑ जिग्ये । \newline
30. जि॒ग्ये॒ क॒त॒रः क॑त॒रो जि॑ग्ये जिग्ये कत॒रः । \newline
31. क॒त॒र श्च॒न च॒न क॑त॒रः क॑त॒र श्च॒न । \newline
32. च॒नैनो॑ रेनो श्च॒न च॒नैनोः᳚ । \newline
33. ए॒नो॒रित्ये॑नोः । \newline
34. इन्द्र॑ श्च॒ चे न्द्र॒ इन्द्र॑ श्च । \newline
35. च॒ वि॒ष्णो॒ वि॒ष्णो॒ च॒ च॒ वि॒ष्णो॒ । \newline
36. वि॒ष्णो॒ यद् यद् वि॑ष्णो विष्णो॒ यत् । \newline
37. वि॒ष्णो॒ इति॑ विष्णो । \newline
38. यदप॑स्पृधेथा॒ मप॑स्पृधेथां॒ ॅयद् यदप॑स्पृधेथाम् । \newline
39. अप॑स्पृधेथाम् त्रे॒धा त्रे॒धा ऽप॑स्पृधेथा॒ मप॑स्पृधेथाम् त्रे॒धा । \newline
40. त्रे॒धा स॒हस्रꣳ॑ स॒हस्र॑म् त्रे॒धा त्रे॒धा स॒हस्र᳚म् । \newline
41. स॒हस्रं॒ ॅवि वि स॒हस्रꣳ॑ स॒हस्रं॒ ॅवि । \newline
42. वि तत् तद् वि वि तत् । \newline
43. तदै॑रयेथा मैरयेथा॒म् तत् तदै॑रयेथाम् । \newline
44. ऐ॒र॒ये॒था॒ मिती त्यै॑रयेथा मैरयेथा॒ मिति॑ । \newline
45. इति॑ त्रेधाविभ॒क्तम् त्रे॑धाविभ॒क्त मितीति॑ त्रेधाविभ॒क्तम् । \newline
46. त्रे॒धा॒वि॒भ॒क्तं ॅवै वै त्रे॑धाविभ॒क्तम् त्रे॑धाविभ॒क्तं ॅवै । \newline
47. त्रे॒धा॒वि॒भ॒क्तमिति॑ त्रेधा - वि॒भ॒क्तम् । \newline
48. वै त्रि॑रा॒त्रे त्रि॑रा॒त्रे वै वै त्रि॑रा॒त्रे । \newline
49. त्रि॒रा॒त्रे स॒हस्रꣳ॑ स॒हस्र॑म् त्रिरा॒त्रे त्रि॑रा॒त्रे स॒हस्र᳚म् । \newline
50. त्रि॒रा॒त्र इति॑ त्रि - रा॒त्रे । \newline
51. स॒हस्रꣳ॑ साह॒स्रीꣳ सा॑ह॒स्रीꣳ स॒हस्रꣳ॑ स॒हस्रꣳ॑ साह॒स्रीम् । \newline
52. सा॒ह॒स्री मे॒वैव सा॑ह॒स्रीꣳ सा॑ह॒स्री मे॒व । \newline
53. ए॒वैना॑ मेना मे॒वै वैना᳚म् । \newline
54. ए॒ना॒म् क॒रो॒ति॒ क॒रो॒ त्ये॒ना॒ मे॒ना॒म् क॒रो॒ति॒ । \newline
55. क॒रो॒ति॒ स॒हस्र॑स्य स॒हस्र॑स्य करोति करोति स॒हस्र॑स्य । \newline
56. स॒हस्र॑ स्यै॒वैव स॒हस्र॑स्य स॒हस्र॑ स्यै॒व । \newline
57. ए॒वैना॑ मेना मे॒वै वैना᳚म् । \newline
58. ए॒ना॒म् मात्रा॒म् मात्रा॑ मेना मेना॒म् मात्रा᳚म् । \newline
59. मात्रा᳚म् करोति करोति॒ मात्रा॒म् मात्रा᳚म् करोति । \newline

\textbf{Ghana Paata } \newline

1. अ॒र्द्ध॒य॒ति॒ प्र॒जावा᳚न् प्र॒जावा॑ नर्द्धय त्यर्द्धयति प्र॒जावा᳚न् पशु॒मान् प॑शु॒मान् प्र॒जावा॑ नर्द्धय त्यर्द्धयति प्र॒जावा᳚न् पशु॒मान् । \newline
2. प्र॒जावा᳚न् पशु॒मान् प॑शु॒मान् प्र॒जावा᳚न् प्र॒जावा᳚न् पशु॒मान् र॑यि॒मान् र॑यि॒मान् प॑शु॒मान् प्र॒जावा᳚न् प्र॒जावा᳚न् पशु॒मान् र॑यि॒मान् । \newline
3. प्र॒जावा॒निति॑ प्र॒जा - वा॒न् । \newline
4. प॒शु॒मान् र॑यि॒मान् र॑यि॒मान् प॑शु॒मान् प॑शु॒मान् र॑यि॒मान् भ॑वति भवति रयि॒मान् प॑शु॒मान् प॑शु॒मान् र॑यि॒मान् भ॑वति । \newline
5. प॒शु॒मानिति॑ पशु - मान् । \newline
6. र॒यि॒मान् भ॑वति भवति रयि॒मान् र॑यि॒मान् भ॑वति॒ यो यो भ॑वति रयि॒मान् र॑यि॒मान् भ॑वति॒ यः । \newline
7. र॒यि॒मानिति॑ रयि - मान् । \newline
8. भ॒व॒ति॒ यो यो भ॑वति भवति॒ य ए॒व मे॒वं ॅयो भ॑वति भवति॒ य ए॒वम् । \newline
9. य ए॒व मे॒वं ॅयो य ए॒वं ॅवेद॒ वेदै॒वं ॅयो य ए॒वं ॅवेद॑ । \newline
10. ए॒वं ॅवेद॒ वेदै॒व मे॒वं ॅवेद॒ तया॒ तया॒ वेदै॒व मे॒वं ॅवेद॒ तया᳚ । \newline
11. वेद॒ तया॒ तया॒ वेद॒ वेद॒ तया॑ स॒ह स॒ह तया॒ वेद॒ वेद॒ तया॑ स॒ह । \newline
12. तया॑ स॒ह स॒ह तया॒ तया॑ स॒हा ग्नी᳚द्ध्र॒ माग्नी᳚द्ध्रꣳ स॒ह तया॒ तया॑ स॒हा ग्नी᳚द्ध्रम् । \newline
13. स॒हा ग्नी᳚द्ध्र॒ माग्नी᳚द्ध्रꣳ स॒ह स॒हा ग्नी᳚द्ध्रम् प॒रेत्य॑ प॒रेत्या ग्नी᳚द्ध्रꣳ स॒ह स॒हा ग्नी᳚द्ध्रम् प॒रेत्य॑ । \newline
14. आग्नी᳚द्ध्रम् प॒रेत्य॑ प॒रेत्या ग्नी᳚द्ध्र॒ माग्नी᳚द्ध्रम् प॒रेत्य॑ पु॒रस्ता᳚त् पु॒रस्ता᳚त् प॒रेत्या ग्नी᳚द्ध्र॒ माग्नी᳚द्ध्रम् प॒रेत्य॑ पु॒रस्ता᳚त् । \newline
15. आग्नी᳚द्ध्र॒मित्याग्नि॑ - इ॒द्ध्र॒म् । \newline
16. प॒रेत्य॑ पु॒रस्ता᳚त् पु॒रस्ता᳚त् प॒रेत्य॑ प॒रेत्य॑ पु॒रस्ता᳚त् प्र॒तीच्या᳚म् प्र॒तीच्या᳚म् पु॒रस्ता᳚त् प॒रेत्य॑ प॒रेत्य॑ पु॒रस्ता᳚त् प्र॒तीच्या᳚म् । \newline
17. प॒रेत्येति॑ परा - इत्य॑ । \newline
18. पु॒रस्ता᳚त् प्र॒तीच्या᳚म् प्र॒तीच्या᳚म् पु॒रस्ता᳚त् पु॒रस्ता᳚त् प्र॒तीच्या॒म् तिष्ठ॑न्त्या॒म् तिष्ठ॑न्त्याम् प्र॒तीच्या᳚म् पु॒रस्ता᳚त् पु॒रस्ता᳚त् प्र॒तीच्या॒म् तिष्ठ॑न्त्याम् । \newline
19. प्र॒तीच्या॒म् तिष्ठ॑न्त्या॒म् तिष्ठ॑न्त्याम् प्र॒तीच्या᳚म् प्र॒तीच्या॒म् तिष्ठ॑न्त्याम् जुहुयाज् जुहुया॒त् तिष्ठ॑न्त्याम् प्र॒तीच्या᳚म् प्र॒तीच्या॒म् तिष्ठ॑न्त्याम् जुहुयात् । \newline
20. तिष्ठ॑न्त्याम् जुहुयाज् जुहुया॒त् तिष्ठ॑न्त्या॒म् तिष्ठ॑न्त्याम् जुहुया दु॒भोभा जु॑हुया॒त् तिष्ठ॑न्त्या॒म् तिष्ठ॑न्त्याम् जुहुया दु॒भा । \newline
21. जु॒हु॒या॒ दु॒भोभा जु॑हुयाज् जुहुया दु॒भा जि॑ग्यथुर् जिग्यथु रु॒भा जु॑हुयाज् जुहुया दु॒भा जि॑ग्यथुः । \newline
22. उ॒भा जि॑ग्यथुर् जिग्यथु रु॒भोभा जि॑ग्यथु॒र् न न जि॑ग्यथु रु॒भोभा जि॑ग्यथु॒र् न । \newline
23. जि॒ग्य॒थु॒र् न न जि॑ग्यथुर् जिग्यथु॒र् न परा॒ परा॒ न जि॑ग्यथुर् जिग्यथु॒र् न परा᳚ । \newline
24. न परा॒ परा॒ न न परा॑ जयेथे जयेथे॒ परा॒ न न परा॑ जयेथे । \newline
25. परा॑ जयेथे जयेथे॒ परा॒ परा॑ जयेथे॒ न न ज॑येथे॒ परा॒ परा॑ जयेथे॒ न । \newline
26. ज॒ये॒थे॒ न न ज॑येथे जयेथे॒ न परा॒ परा॒ न ज॑येथे जयेथे॒ न परा᳚ । \newline
27. ज॒ये॒थे॒ इति॑ जयेथे । \newline
28. न परा॒ परा॒ न न परा॑ जिग्ये जिग्ये॒ परा॒ न न परा॑ जिग्ये । \newline
29. परा॑ जिग्ये जिग्ये॒ परा॒ परा॑ जिग्ये कत॒रः क॑त॒रो जि॑ग्ये॒ परा॒ परा॑ जिग्ये कत॒रः । \newline
30. जि॒ग्ये॒ क॒त॒रः क॑त॒रो जि॑ग्ये जिग्ये कत॒र श्च॒न च॒न क॑त॒रो जि॑ग्ये जिग्ये कत॒र श्च॒न । \newline
31. क॒त॒र श्च॒न च॒न क॑त॒रः क॑त॒र श्च॒नैनो॑ रेनो श्च॒न क॑त॒रः क॑त॒र श्च॒नैनोः᳚ । \newline
32. च॒नैनो॑ रेनो श्च॒न च॒नैनोः᳚ । \newline
33. ए॒नो॒ रित्ये॑नोः । \newline
34. इन्द्र॑ श्च॒ चेन्द्र॒ इन्द्र॑ श्च विष्णो विष्णो॒ चेन्द्र॒ इन्द्र॑ श्च विष्णो । \newline
35. च॒ वि॒ष्णो॒ वि॒ष्णो॒ च॒ च॒ वि॒ष्णो॒ यद् यद् वि॑ष्णो च च विष्णो॒ यत् । \newline
36. वि॒ष्णो॒ यद् यद् वि॑ष्णो विष्णो॒ यदप॑स्पृधेथा॒ मप॑स्पृधेथां॒ ॅयद् वि॑ष्णो विष्णो॒ यदप॑स्पृधेथाम् । \newline
37. वि॒ष्णो॒ इति॑ विष्णो । \newline
38. यदप॑स्पृधेथा॒ मप॑स्पृधेथां॒ ॅयद् यदप॑स्पृधेथाम् त्रे॒धा त्रे॒धा ऽप॑स्पृधेथां॒ ॅयद् यदप॑स्पृधेथाम् त्रे॒धा । \newline
39. अप॑स्पृधेथाम् त्रे॒धा त्रे॒धा ऽप॑स्पृधेथा॒ मप॑स्पृधेथाम् त्रे॒धा स॒हस्रꣳ॑ स॒हस्र॑म् त्रे॒धा ऽप॑स्पृधेथा॒ मप॑स्पृधेथाम् त्रे॒धा स॒हस्र᳚म् । \newline
40. त्रे॒धा स॒हस्रꣳ॑ स॒हस्र॑म् त्रे॒धा त्रे॒धा स॒हस्रं॒ ॅवि वि स॒हस्र॑म् त्रे॒धा त्रे॒धा स॒हस्रं॒ ॅवि । \newline
41. स॒हस्रं॒ ॅवि वि स॒हस्रꣳ॑ स॒हस्रं॒ ॅवि तत् तद् वि स॒हस्रꣳ॑ स॒हस्रं॒ ॅवि तत् । \newline
42. वि तत् तद् वि वि तदै॑रयेथा मैरयेथा॒म् तद् वि वि तदै॑रयेथाम् । \newline
43. तदै॑रयेथा मैरयेथा॒म् तत् तदै॑रयेथा॒ मितीत्यै॑रयेथा॒म् तत् तदै॑रयेथा॒ मिति॑ । \newline
44. ऐ॒र॒ये॒था॒ मिती त्यै॑रयेथा मैरयेथा॒ मिति॑ त्रेधाविभ॒क्तम् त्रे॑धाविभ॒क्त मित्यै॑रयेथा मैरयेथा॒ मिति॑ त्रेधाविभ॒क्तम् । \newline
45. इति॑ त्रेधाविभ॒क्तम् त्रे॑धाविभ॒क्त मितीति॑ त्रेधाविभ॒क्तं ॅवै वै त्रे॑धाविभ॒क्त मितीति॑ त्रेधाविभ॒क्तं ॅवै । \newline
46. त्रे॒धा॒वि॒भ॒क्तं ॅवै वै त्रे॑धाविभ॒क्तम् त्रे॑धाविभ॒क्तं ॅवै त्रि॑रा॒त्रे त्रि॑रा॒त्रे वै त्रे॑धाविभ॒क्तम् त्रे॑धाविभ॒क्तं ॅवै त्रि॑रा॒त्रे । \newline
47. त्रे॒धा॒वि॒भ॒क्तमिति॑ त्रेधा - वि॒भ॒क्तम् । \newline
48. वै त्रि॑रा॒त्रे त्रि॑रा॒त्रे वै वै त्रि॑रा॒त्रे स॒हस्रꣳ॑ स॒हस्र॑म् त्रिरा॒त्रे वै वै त्रि॑रा॒त्रे स॒हस्र᳚म् । \newline
49. त्रि॒रा॒त्रे स॒हस्रꣳ॑ स॒हस्र॑म् त्रिरा॒त्रे त्रि॑रा॒त्रे स॒हस्रꣳ॑ साह॒स्रीꣳ सा॑ह॒स्रीꣳ स॒हस्र॑म् त्रिरा॒त्रे त्रि॑रा॒त्रे स॒हस्रꣳ॑ साह॒स्रीम् । \newline
50. त्रि॒रा॒त्र इति॑ त्रि - रा॒त्रे । \newline
51. स॒हस्रꣳ॑ साह॒स्रीꣳ सा॑ह॒स्रीꣳ स॒हस्रꣳ॑ स॒हस्रꣳ॑ साह॒स्री मे॒वैव सा॑ह॒स्रीꣳ स॒हस्रꣳ॑ स॒हस्रꣳ॑ साह॒स्री मे॒व । \newline
52. सा॒ह॒स्री मे॒वैव सा॑ह॒स्रीꣳ सा॑ह॒स्री मे॒वैना॑ मेना मे॒व सा॑ह॒स्रीꣳ सा॑ह॒स्री मे॒वैना᳚म् । \newline
53. ए॒वैना॑ मेना मे॒वै वैना᳚म् करोति करो त्येना मे॒वै वैना᳚म् करोति । \newline
54. ए॒ना॒म् क॒रो॒ति॒ क॒रो॒ त्ये॒ना॒ मे॒ना॒म् क॒रो॒ति॒ स॒हस्र॑स्य स॒हस्र॑स्य करो त्येना मेनाम् करोति स॒हस्र॑स्य । \newline
55. क॒रो॒ति॒ स॒हस्र॑स्य स॒हस्र॑स्य करोति करोति स॒हस्र॑ स्यै॒वैव स॒हस्र॑स्य करोति करोति स॒हस्र॑स्यै॒व । \newline
56. स॒हस्र॑ स्यै॒वैव स॒हस्र॑स्य स॒हस्र॑ स्यै॒वैना॑ मेना मे॒व स॒हस्र॑स्य स॒हस्र॑ स्यै॒वैना᳚म् । \newline
57. ए॒वैना॑ मेना मे॒वै वैना॒म् मात्रा॒म् मात्रा॑ मेना मे॒वै वैना॒म् मात्रा᳚म् । \newline
58. ए॒ना॒म् मात्रा॒म् मात्रा॑ मेना मेना॒म् मात्रा᳚म् करोति करोति॒ मात्रा॑ मेना मेना॒म् मात्रा᳚म् करोति । \newline
59. मात्रा᳚म् करोति करोति॒ मात्रा॒म् मात्रा᳚म् करोति रू॒पाणि॑ रू॒पाणि॑ करोति॒ मात्रा॒म् मात्रा᳚म् करोति रू॒पाणि॑ । \newline
\pagebreak
\markright{ TS 7.1.6.8  \hfill https://www.vedavms.in \hfill}

\section{ TS 7.1.6.8 }

\textbf{TS 7.1.6.8 } \newline
\textbf{Samhita Paata} \newline

करोति रू॒पाणि॑ जुहोति रू॒पैरे॒वैनाꣳ॒॒ सम॑र्द्धयति॒ तस्या॑ उपो॒त्थाय॒ कर्ण॒मा ज॑पे॒दिडे॒ रन्तेऽदि॑ते॒ सर॑स्वति॒ प्रिये॒ प्रेय॑सि॒ महि॒ विश्रु॑त्ये॒तानि॑ ते अघ्निये॒ नामा॑नि सु॒कृतं॑ मा दे॒वेषु॑ ब्रूता॒दिति॑ दे॒वेभ्य॑ ए॒वैन॒मा वे॑दय॒त्यन्वे॑नं दे॒वा बु॑द्ध्यन्ते ॥ \newline

\textbf{Pada Paata} \newline

क॒रो॒ति॒ । रू॒पाणि॑ । जु॒हो॒ति॒ । रू॒पैः । ए॒व । ए॒ना॒म् । समिति॑ । अ॒द्‌र्ध॒य॒ति॒ । तस्याः᳚ ।      उ॒पो॒त्थायेत्यु॑प - उ॒त्थाय॑ । कर्ण᳚म् । एति॑ । ज॒पे॒त् । इडे᳚ । रन्ते᳚ । अदि॑ते । सर॑स्वति । प्रिये᳚ । प्रेय॑सि । महि॑ । विश्रु॒तीति॒ वि - श्रु॒ति॒ । ए॒तानि॑ । ते॒ । अ॒घ्नि॒ये॒ । नामा॑नि । सु॒कृत॒मिति॑ सु - कृत᳚म् । मा॒ । दे॒वेषु॑ । ब्रू॒ता॒त् । इति॑ । दे॒वेभ्यः॑ । ए॒व । ए॒न॒म् । एति॑ । वे॒द॒य॒ति॒ । अन्विति॑ । ए॒न॒म् । दे॒वाः । बु॒द्ध्य॒न्ते॒ ॥  \newline


\textbf{Krama Paata} \newline

क॒रो॒ति॒ रू॒पाणि॑ । रू॒पाणि॑ जुहोति । जु॒हो॒ति॒ रू॒पैः । रू॒पैरे॒व । ए॒वैना᳚म् । ए॒नाꣳ॒॒ सम् । सम॑र्द्धयति । अ॒र्द्ध॒य॒ति॒ तस्याः᳚ । तस्या॑ उपो॒त्थाय॑ । उ॒पो॒त्थाय॒ कर्ण᳚म् । उ॒पो॒त्थायेत्यु॑प - उ॒त्थाय॑ । कर्ण॒मा । आ ज॑पेत् । ज॒पे॒दिडे᳚ । इडे॒ रन्ते᳚ । रन्तेऽदि॑ते । अदि॑ते॒ सर॑स्वति । सर॑स्वति॒ प्रिये᳚ । प्रिये॒ प्रेय॑सि । प्रेय॑सि॒ महि॑ । महि॒ विश्रु॑ति । विश्रु॑त्ये॒तानि॑ । विश्रु॒तीति॒ वि - श्रु॒ति॒ । ए॒तानि॑ ते । ते॒ अ॒घ्नि॒ये॒ । अ॒घ्नि॒ये॒ नामा॑नि । नामा॑नि सु॒कृत᳚म् । सु॒कृत॑म् मा । सु॒कृत॒मिति॑ सु - कृत᳚म् । मा॒ दे॒वेषु॑ । दे॒वेषु॑ ब्रूतात् । ब्रू॒ता॒दिति॑ । इति॑ दे॒वेभ्यः॑ । दे॒वेभ्य॑ ए॒व । ए॒वैन᳚म् । ए॒न॒मा । आ वे॑दयति । वे॒द॒य॒त्यनु॑ । अन्वे॑नम् । ए॒न॒म् दे॒वाः । दे॒वा बु॑द्ध्यन्ते । बु॒द्ध्य॒न्त॒ इति॑ बुद्ध्यन्ते । \newline

\textbf{Jatai Paata} \newline

1. क॒रो॒ति॒ रू॒पाणि॑ रू॒पाणि॑ करोति करोति रू॒पाणि॑ । \newline
2. रू॒पाणि॑ जुहोति जुहोति रू॒पाणि॑ रू॒पाणि॑ जुहोति । \newline
3. जु॒हो॒ति॒ रू॒पै रू॒पैर् जु॑होति जुहोति रू॒पैः । \newline
4. रू॒पै रे॒वैव रू॒पै रू॒पैरे॒व । \newline
5. ए॒वैना॑ मेना मे॒वै वैना᳚म् । \newline
6. ए॒नाꣳ॒॒ सꣳ स मे॑ना मेनाꣳ॒॒ सम् । \newline
7. स म॑र्द्धय त्यर्द्धयति॒ सꣳ स म॑र्द्धयति । \newline
8. अ॒र्द्ध॒य॒ति॒ तस्या॒ स्तस्या॑ अर्द्धय त्यर्द्धयति॒ तस्याः᳚ । \newline
9. तस्या॑ उपो॒त्था यो॑पो॒त्थाय॒ तस्या॒ स्तस्या॑ उपो॒त्थाय॑ । \newline
10. उ॒पो॒त्थाय॒ कर्ण॒म् कर्ण॑ मुपो॒त्था यो॑पो॒त्थाय॒ कर्ण᳚म् । \newline
11. उ॒पो॒त्थायेत्यु॑प - उ॒त्थाय॑ । \newline
12. कर्ण॒ मा कर्ण॒म् कर्ण॒ मा । \newline
13. आ ज॑पेज् जपे॒दा ज॑पेत् । \newline
14. ज॒पे॒ दिड॒ इडे॑ जपेज् जपे॒ दिडे᳚ । \newline
15. इडे॒ रन्ते॒ रन्त॒ इड॒ इडे॒ रन्ते᳚ । \newline
16. रन्ते ऽदि॒ते ऽदि॑ते॒ रन्ते॒ रन्ते ऽदि॑ते । \newline
17. अदि॑ते॒ सर॑स्वति॒ सर॑स्व॒ त्यदि॒ते ऽदि॑ते॒ सर॑स्वति । \newline
18. सर॑स्वति॒ प्रिये॒ प्रिये॒ सर॑स्वति॒ सर॑स्वति॒ प्रिये᳚ । \newline
19. प्रिये॒ प्रेय॑सि॒ प्रेय॑सि॒ प्रिये॒ प्रिये॒ प्रेय॑सि । \newline
20. प्रेय॑सि॒ महि॒ महि॒ प्रेय॑सि॒ प्रेय॑सि॒ महि॑ । \newline
21. महि॒ विश्रु॑ति॒ विश्रु॑ति॒ महि॒ महि॒ विश्रु॑ति । \newline
22. विश्रु॑ त्ये॒ता न्ये॒तानि॒ विश्रु॑ति॒ विश्रु॑ त्ये॒तानि॑ । \newline
23. विश्रु॒तीति॒ वि - श्रु॒ति॒ । \newline
24. ए॒तानि॑ ते त ए॒ता न्ये॒तानि॑ ते । \newline
25. ते॒ अ॒घ्नि॒ये॒ अ॒घ्नि॒ये॒ ते॒ ते॒ अ॒घ्नि॒ये॒ । \newline
26. अ॒घ्नि॒ये॒ नामा॑नि॒ नामा᳚ न्यघ्निये अघ्निये॒ नामा॑नि । \newline
27. नामा॑नि सु॒कृतꣳ॑ सु॒कृत॒म् नामा॑नि॒ नामा॑नि सु॒कृत᳚म् । \newline
28. सु॒कृत॑म् मा मा सु॒कृतꣳ॑ सु॒कृत॑म् मा । \newline
29. सु॒कृत॒मिति॑ सु - कृत᳚म् । \newline
30. मा॒ दे॒वेषु॑ दे॒वेषु॑ मा मा दे॒वेषु॑ । \newline
31. दे॒वेषु॑ ब्रूताद् ब्रूताद् दे॒वेषु॑ दे॒वेषु॑ ब्रूतात् । \newline
32. ब्रू॒ता॒ दितीति॑ ब्रूताद् ब्रूता॒ दिति॑ । \newline
33. इति॑ दे॒वेभ्यो॑ दे॒वेभ्य॒ इतीति॑ दे॒वेभ्यः॑ । \newline
34. दे॒वेभ्य॑ ए॒वैव दे॒वेभ्यो॑ दे॒वेभ्य॑ ए॒व । \newline
35. ए॒वैन॑ मेन मे॒वै वैन᳚म् । \newline
36. ए॒न॒ मैन॑ मेन॒ मा । \newline
37. आ वे॑दयति वेदय॒त्या वे॑दयति । \newline
38. वे॒द॒य॒ त्यन्वनु॑ वेदयति वेदय॒ त्यनु॑ । \newline
39. अन्वे॑न मेन॒ मन्वन् वे॑नम् । \newline
40. ए॒न॒म् दे॒वा दे॒वा ए॑न मेनम् दे॒वाः । \newline
41. दे॒वा बु॑द्ध्यन्ते बुद्ध्यन्ते दे॒वा दे॒वा बु॑द्ध्यन्ते । \newline
42. बु॒द्ध्य॒न्त॒ इति॑ बुद्ध्यन्ते । \newline

\textbf{Ghana Paata } \newline

1. क॒रो॒ति॒ रू॒पाणि॑ रू॒पाणि॑ करोति करोति रू॒पाणि॑ जुहोति जुहोति रू॒पाणि॑ करोति करोति रू॒पाणि॑ जुहोति । \newline
2. रू॒पाणि॑ जुहोति जुहोति रू॒पाणि॑ रू॒पाणि॑ जुहोति रू॒पै रू॒पैर् जु॑होति रू॒पाणि॑ रू॒पाणि॑ जुहोति रू॒पैः । \newline
3. जु॒हो॒ति॒ रू॒पै रू॒पैर् जु॑होति जुहोति रू॒पै रे॒वैव रू॒पैर् जु॑होति जुहोति रू॒पै रे॒व । \newline
4. रू॒पै रे॒वैव रू॒पै रू॒पै रे॒वैना॑ मेना मे॒व रू॒पै रू॒पै रे॒वैना᳚म् । \newline
5. ए॒वैना॑ मेना मे॒वै वैनाꣳ॒॒ सꣳ स मे॑ना मे॒वै वैनाꣳ॒॒ सम् । \newline
6. ए॒नाꣳ॒॒ सꣳ स मे॑ना मेनाꣳ॒॒ स म॑र्द्धय त्यर्द्धयति॒ स मे॑ना मेनाꣳ॒॒ स म॑र्द्धयति । \newline
7. स म॑र्द्धय त्यर्द्धयति॒ सꣳ स म॑र्द्धयति॒ तस्या॒ स्तस्या॑ अर्द्धयति॒ सꣳ स म॑र्द्धयति॒ तस्याः᳚ । \newline
8. अ॒र्द्ध॒य॒ति॒ तस्या॒ स्तस्या॑ अर्द्धय त्यर्द्धयति॒ तस्या॑ उपो॒त्था यो॑पो॒त्थाय॒ तस्या॑ अर्द्धय त्यर्द्धयति॒ तस्या॑ उपो॒त्थाय॑ । \newline
9. तस्या॑ उपो॒त्था यो॑पो॒त्थाय॒ तस्या॒ स्तस्या॑ उपो॒त्थाय॒ कर्ण॒म् कर्ण॑ मुपो॒त्थाय॒ तस्या॒ स्तस्या॑ उपो॒त्थाय॒ कर्ण᳚म् । \newline
10. उ॒पो॒त्थाय॒ कर्ण॒म् कर्ण॑ मुपो॒त्था यो॑पो॒त्थाय॒ कर्ण॒ मा कर्ण॑ मुपो॒त्था यो॑पो॒त्थाय॒ कर्ण॒ मा । \newline
11. उ॒पो॒त्थायेत्यु॑प - उ॒त्थाय॑ । \newline
12. कर्ण॒ मा कर्ण॒म् कर्ण॒ मा ज॑पेज् जपे॒दा कर्ण॒म् कर्ण॒ मा ज॑पेत् । \newline
13. आ ज॑पेज् जपे॒दा ज॑पे॒ दिड॒ इडे॑ जपे॒दा ज॑पे॒ दिडे᳚ । \newline
14. ज॒पे॒ दिड॒ इडे॑ जपेज् जपे॒ दिडे॒ रन्ते॒ रन्त॒ इडे॑ जपेज् जपे॒ दिडे॒ रन्ते᳚ । \newline
15. इडे॒ रन्ते॒ रन्त॒ इड॒ इडे॒ रन्ते ऽदि॒ते ऽदि॑ते॒ रन्त॒ इड॒ इडे॒ रन्ते ऽदि॑ते । \newline
16. रन्ते ऽदि॒ते ऽदि॑ते॒ रन्ते॒ रन्ते ऽदि॑ते॒ सर॑स्वति॒ सर॑स्व॒ त्यदि॑ते॒ रन्ते॒ रन्ते ऽदि॑ते॒ सर॑स्वति । \newline
17. अदि॑ते॒ सर॑स्वति॒ सर॑स्व॒ त्यदि॒ते ऽदि॑ते॒ सर॑स्वति॒ प्रिये॒ प्रिये॒ सर॑स्व॒ त्यदि॒ते ऽदि॑ते॒ सर॑स्वति॒ प्रिये᳚ । \newline
18. सर॑स्वति॒ प्रिये॒ प्रिये॒ सर॑स्वति॒ सर॑स्वति॒ प्रिये॒ प्रेय॑सि॒ प्रेय॑सि॒ प्रिये॒ सर॑स्वति॒ सर॑स्वति॒ प्रिये॒ प्रेय॑सि । \newline
19. प्रिये॒ प्रेय॑सि॒ प्रेय॑सि॒ प्रिये॒ प्रिये॒ प्रेय॑सि॒ महि॒ महि॒ प्रेय॑सि॒ प्रिये॒ प्रिये॒ प्रेय॑सि॒ महि॑ । \newline
20. प्रेय॑सि॒ महि॒ महि॒ प्रेय॑सि॒ प्रेय॑सि॒ महि॒ विश्रु॑ति॒ विश्रु॑ति॒ महि॒ प्रेय॑सि॒ प्रेय॑सि॒ महि॒ विश्रु॑ति । \newline
21. महि॒ विश्रु॑ति॒ विश्रु॑ति॒ महि॒ महि॒ विश्रु॑ त्ये॒ता न्ये॒तानि॒ विश्रु॑ति॒ महि॒ महि॒ विश्रु॑ त्ये॒तानि॑ । \newline
22. विश्रु॑ त्ये॒ता न्ये॒तानि॒ विश्रु॑ति॒ विश्रु॑ त्ये॒तानि॑ ते त ए॒तानि॒ विश्रु॑ति॒ विश्रु॑ त्ये॒तानि॑ ते । \newline
23. विश्रु॒तीति॒ वि - श्रु॒ति॒ । \newline
24. ए॒तानि॑ ते त ए॒ता न्ये॒तानि॑ ते अघ्निये अघ्निये त ए॒ता न्ये॒तानि॑ ते अघ्निये । \newline
25. ते॒ अ॒घ्नि॒ये॒ अ॒घ्नि॒ये॒ ते॒ ते॒ अ॒घ्नि॒ये॒ नामा॑नि॒ नामा᳚ न्यघ्निये ते ते अघ्निये॒ नामा॑नि । \newline
26. अ॒घ्नि॒ये॒ नामा॑नि॒ नामा᳚ न्यघ्निये अघ्निये॒ नामा॑नि सु॒कृतꣳ॑ सु॒कृत॒म् नामा᳚ न्यघ्निये अघ्निये॒ नामा॑नि सु॒कृत᳚म् । \newline
27. नामा॑नि सु॒कृतꣳ॑ सु॒कृत॒म् नामा॑नि॒ नामा॑नि सु॒कृत॑म् मा मा सु॒कृत॒म् नामा॑नि॒ नामा॑नि सु॒कृत॑म् मा । \newline
28. सु॒कृत॑म् मा मा सु॒कृतꣳ॑ सु॒कृत॑म् मा दे॒वेषु॑ दे॒वेषु॑ मा सु॒कृतꣳ॑ सु॒कृत॑म् मा दे॒वेषु॑ । \newline
29. सु॒कृत॒मिति॑ सु - कृत᳚म् । \newline
30. मा॒ दे॒वेषु॑ दे॒वेषु॑ मा मा दे॒वेषु॑ ब्रूताद् ब्रूताद् दे॒वेषु॑ मा मा दे॒वेषु॑ ब्रूतात् । \newline
31. दे॒वेषु॑ ब्रूताद् ब्रूताद् दे॒वेषु॑ दे॒वेषु॑ ब्रूता॒ दितीति॑ ब्रूताद् दे॒वेषु॑ दे॒वेषु॑ ब्रूता॒ दिति॑ । \newline
32. ब्रू॒ता॒ दितीति॑ ब्रूताद् ब्रूता॒ दिति॑ दे॒वेभ्यो॑ दे॒वेभ्य॒ इति॑ ब्रूताद् ब्रूता॒ दिति॑ दे॒वेभ्यः॑ । \newline
33. इति॑ दे॒वेभ्यो॑ दे॒वेभ्य॒ इतीति॑ दे॒वेभ्य॑ ए॒वैव दे॒वेभ्य॒ इतीति॑ दे॒वेभ्य॑ ए॒व । \newline
34. दे॒वेभ्य॑ ए॒वैव दे॒वेभ्यो॑ दे॒वेभ्य॑ ए॒वैन॑ मेन मे॒व दे॒वेभ्यो॑ दे॒वेभ्य॑ ए॒वैन᳚म् । \newline
35. ए॒वैन॑ मेन मे॒वै वैन॒ मैन॑ मे॒वै वैन॒ मा । \newline
36. ए॒न॒ मैन॑ मेन॒ मा वे॑दयति वेदय॒ त्यैन॑ मेन॒ मा वे॑दयति । \newline
37. आ वे॑दयति वेदय॒त्या वे॑दय॒ त्यन्वनु॑ वेदय॒त्या वे॑दय॒ त्यनु॑ । \newline
38. वे॒द॒य॒ त्यन्वनु॑ वेदयति वेदय॒ त्यन्वे॑न मेन॒ मनु॑ वेदयति वेदय॒ त्यन्वे॑नम् । \newline
39. अन्वे॑न मेन॒ मन् वन् वे॑नम् दे॒वा दे॒वा ए॑न॒ मन् वन् वे॑नम् दे॒वाः । \newline
40. ए॒न॒म् दे॒वा दे॒वा ए॑न मेनम् दे॒वा बु॑द्ध्यन्ते बुद्ध्यन्ते दे॒वा ए॑न मेनम् दे॒वा बु॑द्ध्यन्ते । \newline
41. दे॒वा बु॑द्ध्यन्ते बुद्ध्यन्ते दे॒वा दे॒वा बु॑द्ध्यन्ते । \newline
42. बु॒द्ध्य॒न्त॒ इति॑ बुद्ध्यन्ते । \newline
\pagebreak
\markright{ TS 7.1.7.1  \hfill https://www.vedavms.in \hfill}

\section{ TS 7.1.7.1 }

\textbf{TS 7.1.7.1 } \newline
\textbf{Samhita Paata} \newline

स॒ह॒स्र॒त॒म्या॑ वै यज॑मानः सुव॒र्गं ॅलो॒कमे॑ति॒ सैनꣳ॑ सुव॒र्गं ॅलो॒कं ग॑मयति॒ सा मा॑ सुव॒र्गं ॅलो॒कं ग॑म॒येत्या॑ह सुव॒र्गमे॒वैनं॑ ॅलो॒कं ग॑मयति॒ सा मा॒ ज्योति॑ष्मन्तं ॅलो॒कं ग॑म॒येत्या॑ह॒ ज्योति॑ष्मन्तमे॒वैनं॑ ॅलो॒कं ग॑मयति॒ सा मा॒ सर्वा॒न् पुण्यां᳚ ॅलो॒कान् ग॑म॒येत्या॑ह॒ सर्वा॑ने॒वैनं॒ पुण्यां᳚ ॅलो॒कान् ग॑मयति॒ सा - [  ] \newline

\textbf{Pada Paata} \newline

स॒ह॒स्र॒त॒म्येति॑ सहस्र - त॒म्या᳚ । वै । यज॑मानः । सु॒व॒र्गमिति॑ सुवः - गम् । लो॒कम् । ए॒ति॒ । सा । ए॒न॒म् । सु॒व॒र्गमिति॑ सुवः - गम् । लो॒कम् । ग॒म॒य॒ति॒ । सा । मा॒ । सु॒व॒र्गमिति॑ सुवः - गम् । लो॒कम् । ग॒म॒य॒ । इति॑ । आ॒ह॒ । सु॒व॒र्गमिति॑ सुवः - गम् । ए॒व । ए॒न॒म् । लो॒कम् । ग॒म॒य॒ति॒ । सा । मा॒ । ज्योति॑ष्मन्तम् । लो॒कम् । ग॒म॒य॒ । इति॑ । आ॒ह॒ । ज्योति॑ष्मन्तम् । ए॒व । ए॒न॒म् । लो॒कम् । ग॒म॒य॒ति॒ । सा । मा॒ । सर्वान्॑ । पुण्यान्॑ । लो॒कान् । ग॒म॒य॒ । इति॑ । आ॒ह॒ । सर्वान्॑ । ए॒व । ए॒न॒म् । पुण्यान्॑ । लो॒कान् । ग॒म॒य॒ति॒ । सा ।  \newline


\textbf{Krama Paata} \newline

स॒ह॒स्र॒त॒म्या॑ वै । स॒ह॒स्र॒त॒म्येति॑ सहस्र - त॒म्या᳚ । वै यज॑मानः । यज॑मानः सुव॒र्गम् । सु॒व॒र्गम् ॅलो॒कम् । सु॒व॒र्गमिति॑ सुवः - गम् । लो॒कमे॑ति । ए॒ति॒ सा । सैन᳚म् । ए॒नꣳ॒॒ सु॒व॒र्गम् । सु॒व॒र्गम् ॅलो॒कम् । सु॒व॒र्गमिति॑ सुवः - गम् । लो॒कम् ग॑मयति । ग॒म॒य॒ति॒ सा । सा मा᳚ । मा॒ सु॒व॒र्गम् । सु॒व॒र्गम् ॅलो॒कम् । सु॒व॒र्गमिति॑ सुवः - गम् । लो॒कम् ग॑मय । ग॒म॒येति॑ । इत्या॑ह । आ॒ह॒ सु॒व॒र्गम् । सु॒व॒र्गमे॒व । सु॒व॒र्गमिति॑ सुवः - गम् । ए॒वैन᳚म् । ए॒न॒म् ॅलो॒कम् । लो॒कम् ग॑मयति । ग॒म॒य॒ति॒ सा । सा मा᳚ । मा॒ ज्योति॑ष्मन्तम् । ज्योति॑ष्मन्तम् ॅलो॒कम् । लो॒कम् ग॑मय । ग॒म॒येति॑ । इत्या॑ह । आ॒ह॒ ज्योति॑ष्मन्तम् । ज्योति॑ष्मन्तमे॒व । ए॒वैन᳚म् । ए॒न॒म् ॅलो॒कम् । लो॒कम् ग॑मयति । ग॒म॒य॒ति॒ सा । सा मा᳚ । मा॒ सर्वान्॑ । सर्वा॒न् पुण्यान्॑ । पुण्या᳚न् ॅलो॒कान् । लो॒काङ्‍ग॑मय । ग॒म॒येति॑ । इत्या॑ह । आ॒ह॒ सर्वान्॑ । सर्वा॑ने॒व । ए॒वैन᳚म् । ए॒न॒म् पुण्यान्॑ । पुण्या᳚न् ॅलो॒कान् । लो॒काङ्‍ग॑मयति । ग॒म॒य॒ति॒ सा । सा मा᳚ \newline

\textbf{Jatai Paata} \newline

1. स॒ह॒स्र॒त॒म्या॑ वै वै स॑हस्रत॒म्या॑ सहस्रत॒म्या॑ वै । \newline
2. स॒ह॒स्र॒त॒म्येति॑ सहस्र - त॒म्या᳚ । \newline
3. वै यज॑मानो॒ यज॑मानो॒ वै वै यज॑मानः । \newline
4. यज॑मानः सुव॒र्गꣳ सु॑व॒र्गं ॅयज॑मानो॒ यज॑मानः सुव॒र्गम् । \newline
5. सु॒व॒र्गम् ॅलो॒कम् ॅलो॒कꣳ सु॑व॒र्गꣳ सु॑व॒र्गम् ॅलो॒कम् । \newline
6. सु॒व॒र्गमिति॑ सुवः - गम् । \newline
7. लो॒क मे᳚त्येति लो॒कम् ॅलो॒क मे॑ति । \newline
8. ए॒ति॒ सा सैत्ये॑ति॒ सा । \newline
9. सैन॑ मेनꣳ॒॒ सा सैन᳚म् । \newline
10. ए॒नꣳ॒॒ सु॒व॒र्गꣳ सु॑व॒र्ग मे॑न मेनꣳ सुव॒र्गम् । \newline
11. सु॒व॒र्गम् ॅलो॒कम् ॅलो॒कꣳ सु॑व॒र्गꣳ सु॑व॒र्गम् ॅलो॒कम् । \newline
12. सु॒व॒र्गमिति॑ सुवः - गम् । \newline
13. लो॒कम् ग॑मयति गमयति लो॒कम् ॅलो॒कम् ग॑मयति । \newline
14. ग॒म॒य॒ति॒ सा सा ग॑मयति गमयति॒ सा । \newline
15. सा मा॑ मा॒ सा सा मा᳚ । \newline
16. मा॒ सु॒व॒र्गꣳ सु॑व॒र्गम् मा॑ मा सुव॒र्गम् । \newline
17. सु॒व॒र्गम् ॅलो॒कम् ॅलो॒कꣳ सु॑व॒र्गꣳ सु॑व॒र्गम् ॅलो॒कम् । \newline
18. सु॒व॒र्गमिति॑ सुवः - गम् । \newline
19. लो॒कम् ग॑मय गमय लो॒कम् ॅलो॒कम् ग॑मय । \newline
20. ग॒म॒येतीति॑ गमय गम॒येति॑ । \newline
21. इत्या॑हा॒हे तीत्या॑ह । \newline
22. आ॒ह॒ सु॒व॒र्गꣳ सु॑व॒र्ग मा॑हाह सुव॒र्गम् । \newline
23. सु॒व॒र्ग मे॒वैव सु॑व॒र्गꣳ सु॑व॒र्ग मे॒व । \newline
24. सु॒व॒र्गमिति॑ सुवः - गम् । \newline
25. ए॒वैन॑ मेन मे॒वै वैन᳚म् । \newline
26. ए॒न॒म् ॅलो॒कम् ॅलो॒क मे॑न मेनम् ॅलो॒कम् । \newline
27. लो॒कम् ग॑मयति गमयति लो॒कम् ॅलो॒कम् ग॑मयति । \newline
28. ग॒म॒य॒ति॒ सा सा ग॑मयति गमयति॒ सा । \newline
29. सा मा॑ मा॒ सा सा मा᳚ । \newline
30. मा॒ ज्योति॑ष्मन्त॒म् ज्योति॑ष्मन्तम् मा मा॒ ज्योति॑ष्मन्तम् । \newline
31. ज्योति॑ष्मन्तम् ॅलो॒कम् ॅलो॒कम् ज्योति॑ष्मन्त॒म् ज्योति॑ष्मन्तम् ॅलो॒कम् । \newline
32. लो॒कम् ग॑मय गमय लो॒कम् ॅलो॒कम् ग॑मय । \newline
33. ग॒म॒येतीति॑ गमय गम॒येति॑ । \newline
34. इत्या॑हा॒हे तीत्या॑ह । \newline
35. आ॒ह॒ ज्योति॑ष्मन्त॒म् ज्योति॑ष्मन्त माहाह॒ ज्योति॑ष्मन्तम् । \newline
36. ज्योति॑ष्मन्त मे॒वैव ज्योति॑ष्मन्त॒म् ज्योति॑ष्मन्त मे॒व । \newline
37. ए॒वैन॑ मेन मे॒वै वैन᳚म् । \newline
38. ए॒न॒म् ॅलो॒कम् ॅलो॒क मे॑न मेनम् ॅलो॒कम् । \newline
39. लो॒कम् ग॑मयति गमयति लो॒कम् ॅलो॒कम् ग॑मयति । \newline
40. ग॒म॒य॒ति॒ सा सा ग॑मयति गमयति॒ सा । \newline
41. सा मा॑ मा॒ सा सा मा᳚ । \newline
42. मा॒ सर्वा॒न् थ्सर्वा᳚न् मा मा॒ सर्वान्॑ । \newline
43. सर्वा॒न् पुण्या॒न् पुण्या॒न् थ्सर्वा॒न् थ्सर्वा॒न् पुण्यान्॑ । \newline
44. पुण्या᳚न् ॅलो॒कान् ॅलो॒कान् पुण्या॒न् पुण्या᳚न् ॅलो॒कान् । \newline
45. लो॒कान् ग॑मय गमय लो॒कान् ॅलो॒कान् ग॑मय । \newline
46. ग॒म॒येतीति॑ गमय गम॒येति॑ । \newline
47. इत्या॑हा॒हे तीत्या॑ह । \newline
48. आ॒ह॒ सर्वा॒न् थ्सर्वा॑ नाहाह॒ सर्वान्॑ । \newline
49. सर्वा॑ ने॒वैव सर्वा॒न् थ्सर्वा॑ ने॒व । \newline
50. ए॒वैन॑ मेन मे॒वैवैन᳚म् । \newline
51. ए॒न॒म् पुण्या॒न् पुण्या॑नेन मेन॒म् पुण्यान्॑ । \newline
52. पुण्या᳚न् ॅलो॒कान् ॅलो॒कान् पुण्या॒न् पुण्या᳚न् ॅलो॒कान् । \newline
53. लो॒कान् ग॑मयति गमयति लो॒कान् ॅलो॒कान् ग॑मयति । \newline
54. ग॒म॒य॒ति॒ सा सा ग॑मयति गमयति॒ सा । \newline
55. सा मा॑ मा॒ सा सा मा᳚ । \newline

\textbf{Ghana Paata } \newline

1. स॒ह॒स्र॒त॒म्या॑ वै वै स॑हस्रत॒म्या॑ सहस्रत॒म्या॑ वै यज॑मानो॒ यज॑मानो॒ वै स॑हस्रत॒म्या॑ सहस्रत॒म्या॑ वै यज॑मानः । \newline
2. स॒ह॒स्र॒त॒म्येति॑ सहस्र - त॒म्या᳚ । \newline
3. वै यज॑मानो॒ यज॑मानो॒ वै वै यज॑मानः सुव॒र्गꣳ सु॑व॒र्गं ॅयज॑मानो॒ वै वै यज॑मानः सुव॒र्गम् । \newline
4. यज॑मानः सुव॒र्गꣳ सु॑व॒र्गं ॅयज॑मानो॒ यज॑मानः सुव॒र्गम् ॅलो॒कम् ॅलो॒कꣳ सु॑व॒र्गं ॅयज॑मानो॒ यज॑मानः सुव॒र्गम् ॅलो॒कम् । \newline
5. सु॒व॒र्गम् ॅलो॒कम् ॅलो॒कꣳ सु॑व॒र्गꣳ सु॑व॒र्गम् ॅलो॒क मे᳚त्येति लो॒कꣳ सु॑व॒र्गꣳ सु॑व॒र्गम् ॅलो॒क मे॑ति । \newline
6. सु॒व॒र्गमिति॑ सुवः - गम् । \newline
7. लो॒क मे᳚त्येति लो॒कम् ॅलो॒क मे॑ति॒ सा सैति॑ लो॒कम् ॅलो॒क मे॑ति॒ सा । \newline
8. ए॒ति॒ सा सैत्ये॑ति॒ सैन॑ मेनꣳ॒॒ सैत्ये॑ति॒ सैन᳚म् । \newline
9. सैन॑ मेनꣳ॒॒ सा सैनꣳ॑ सुव॒र्गꣳ सु॑व॒र्ग मे॑नꣳ॒॒ सा सैनꣳ॑ सुव॒र्गम् । \newline
10. ए॒नꣳ॒॒ सु॒व॒र्गꣳ सु॑व॒र्ग मे॑न मेनꣳ सुव॒र्गम् ॅलो॒कम् ॅलो॒कꣳ सु॑व॒र्ग मे॑न मेनꣳ सुव॒र्गम् ॅलो॒कम् । \newline
11. सु॒व॒र्गम् ॅलो॒कम् ॅलो॒कꣳ सु॑व॒र्गꣳ सु॑व॒र्गम् ॅलो॒कम् ग॑मयति गमयति लो॒कꣳ सु॑व॒र्गꣳ सु॑व॒र्गम् ॅलो॒कम् ग॑मयति । \newline
12. सु॒व॒र्गमिति॑ सुवः - गम् । \newline
13. लो॒कम् ग॑मयति गमयति लो॒कम् ॅलो॒कम् ग॑मयति॒ सा सा ग॑मयति लो॒कम् ॅलो॒कम् ग॑मयति॒ सा । \newline
14. ग॒म॒य॒ति॒ सा सा ग॑मयति गमयति॒ सा मा॑ मा॒ सा ग॑मयति गमयति॒ सा मा᳚ । \newline
15. सा मा॑ मा॒ सा सा मा॑ सुव॒र्गꣳ सु॑व॒र्गम् मा॒ सा सा मा॑ सुव॒र्गम् । \newline
16. मा॒ सु॒व॒र्गꣳ सु॑व॒र्गम् मा॑ मा सुव॒र्गम् ॅलो॒कम् ॅलो॒कꣳ सु॑व॒र्गम् मा॑ मा सुव॒र्गम् ॅलो॒कम् । \newline
17. सु॒व॒र्गम् ॅलो॒कम् ॅलो॒कꣳ सु॑व॒र्गꣳ सु॑व॒र्गम् ॅलो॒कम् ग॑मय गमय लो॒कꣳ सु॑व॒र्गꣳ सु॑व॒र्गम् ॅलो॒कम् ग॑मय । \newline
18. सु॒व॒र्गमिति॑ सुवः - गम् । \newline
19. लो॒कम् ग॑मय गमय लो॒कम् ॅलो॒कम् ग॑म॒येतीति॑ गमय लो॒कम् ॅलो॒कम् ग॑म॒येति॑ । \newline
20. ग॒म॒येतीति॑ गमय गम॒ येत्या॑हा॒हेति॑ गमय गम॒ येत्या॑ह । \newline
21. इत्या॑हा॒हे तीत्या॑ह सुव॒र्गꣳ सु॑व॒र्ग मा॒हे तीत्या॑ह सुव॒र्गम् । \newline
22. आ॒ह॒ सु॒व॒र्गꣳ सु॑व॒र्ग मा॑हाह सुव॒र्ग मे॒वैव सु॑व॒र्ग मा॑हाह सुव॒र्ग मे॒व । \newline
23. सु॒व॒र्ग मे॒वैव सु॑व॒र्गꣳ सु॑व॒र्ग मे॒वैन॑ मेन मे॒व सु॑व॒र्गꣳ सु॑व॒र्ग मे॒वैन᳚म् । \newline
24. सु॒व॒र्गमिति॑ सुवः - गम् । \newline
25. ए॒वैन॑ मेन मे॒वै वैन॑म् ॅलो॒कम् ॅलो॒क मे॑न मे॒वै वैन॑म् ॅलो॒कम् । \newline
26. ए॒न॒म् ॅलो॒कम् ॅलो॒क मे॑न मेनम् ॅलो॒कम् ग॑मयति गमयति लो॒क मे॑न मेनम् ॅलो॒कम् ग॑मयति । \newline
27. लो॒कम् ग॑मयति गमयति लो॒कम् ॅलो॒कम् ग॑मयति॒ सा सा ग॑मयति लो॒कम् ॅलो॒कम् ग॑मयति॒ सा । \newline
28. ग॒म॒य॒ति॒ सा सा ग॑मयति गमयति॒ सा मा॑ मा॒ सा ग॑मयति गमयति॒ सा मा᳚ । \newline
29. सा मा॑ मा॒ सा सा मा॒ ज्योति॑ष्मन्त॒म् ज्योति॑ष्मन्तम् मा॒ सा सा मा॒ ज्योति॑ष्मन्तम् । \newline
30. मा॒ ज्योति॑ष्मन्त॒म् ज्योति॑ष्मन्तम् मा मा॒ ज्योति॑ष्मन्तम् ॅलो॒कम् ॅलो॒कम् ज्योति॑ष्मन्तम् मा मा॒ ज्योति॑ष्मन्तम् ॅलो॒कम् । \newline
31. ज्योति॑ष्मन्तम् ॅलो॒कम् ॅलो॒कम् ज्योति॑ष्मन्त॒म् ज्योति॑ष्मन्तम् ॅलो॒कम् ग॑मय गमय लो॒कम् ज्योति॑ष्मन्त॒म् ज्योति॑ष्मन्तम् ॅलो॒कम् ग॑मय । \newline
32. लो॒कम् ग॑मय गमय लो॒कम् ॅलो॒कम् ग॑म॒येतीति॑ गमय लो॒कम् ॅलो॒कम् ग॑म॒येति॑ । \newline
33. ग॒म॒येतीति॑ गमय गम॒ येत्या॑हा॒ हेति॑ गमय गम॒ येत्या॑ह । \newline
34. इत्या॑हा॒हे तीत्या॑ह॒ ज्योति॑ष्मन्त॒म् ज्योति॑ष्मन्त मा॒हे तीत्या॑ह॒ ज्योति॑ष्मन्तम् । \newline
35. आ॒ह॒ ज्योति॑ष्मन्त॒म् ज्योति॑ष्मन्त माहाह॒ ज्योति॑ष्मन्त मे॒वैव ज्योति॑ष्मन्त माहाह॒ ज्योति॑ष्मन्त मे॒व । \newline
36. ज्योति॑ष्मन्त मे॒वैव ज्योति॑ष्मन्त॒म् ज्योति॑ष्मन्त मे॒वैन॑ मेन मे॒व ज्योति॑ष्मन्त॒म् ज्योति॑ष्मन्त मे॒वैन᳚म् । \newline
37. ए॒वैन॑ मेन मे॒वै वैन॑म् ॅलो॒कम् ॅलो॒क मे॑न मे॒वै वैन॑म् ॅलो॒कम् । \newline
38. ए॒न॒म् ॅलो॒कम् ॅलो॒क मे॑न मेनम् ॅलो॒कम् ग॑मयति गमयति लो॒क मे॑न मेनम् ॅलो॒कम् ग॑मयति । \newline
39. लो॒कम् ग॑मयति गमयति लो॒कम् ॅलो॒कम् ग॑मयति॒ सा सा ग॑मयति लो॒कम् ॅलो॒कम् ग॑मयति॒ सा । \newline
40. ग॒म॒य॒ति॒ सा सा ग॑मयति गमयति॒ सा मा॑ मा॒ सा ग॑मयति गमयति॒ सा मा᳚ । \newline
41. सा मा॑ मा॒ सा सा मा॒ सर्वा॒न् थ्सर्वा᳚न् मा॒ सा सा मा॒ सर्वान्॑ । \newline
42. मा॒ सर्वा॒न् थ्सर्वा᳚न् मा मा॒ सर्वा॒न् पुण्या॒न् पुण्या॒न् थ्सर्वा᳚न् मा मा॒ सर्वा॒न् पुण्यान्॑ । \newline
43. सर्वा॒न् पुण्या॒न् पुण्या॒न् थ्सर्वा॒न् थ्सर्वा॒न् पुण्या᳚न् ॅलो॒कान् ॅलो॒कान् पुण्या॒न् थ्सर्वा॒न् थ्सर्वा॒न् पुण्या᳚न् ॅलो॒कान् । \newline
44. पुण्या᳚न् ॅलो॒कान् ॅलो॒कान् पुण्या॒न् पुण्या᳚न् ॅलो॒कान् ग॑मय गमय लो॒कान् पुण्या॒न् पुण्या᳚न् ॅलो॒कान् ग॑मय । \newline
45. लो॒कान् ग॑मय गमय लो॒कान् ॅलो॒कान् ग॑म॒येतीति॑ गमय लो॒कान् ॅलो॒कान् ग॑म॒येति॑ । \newline
46. ग॒म॒येतीति॑ गमय गम॒ येत्या॑हा॒ हेति॑ गमय गम॒ येत्या॑ह । \newline
47. इत्या॑हा॒हे तीत्या॑ह॒ सर्वा॒न् थ्सर्वा॑ ना॒हे तीत्या॑ह॒ सर्वान्॑ । \newline
48. आ॒ह॒ सर्वा॒न् थ्सर्वा॑ नाहाह॒ सर्वा॑ ने॒वैव सर्वा॑ नाहाह॒ सर्वा॑ ने॒व । \newline
49. सर्वा॑ ने॒वैव सर्वा॒न् थ्सर्वा॑ ने॒वैन॑ मेन मे॒व सर्वा॒न् थ्सर्वा॑ ने॒वैन᳚म् । \newline
50. ए॒वैन॑ मेन मे॒वै वैन॒म् पुण्या॒न् पुण्या॑ नेन मे॒वै वैन॒म् पुण्यान्॑ । \newline
51. ए॒न॒म् पुण्या॒न् पुण्या॑ नेन मेन॒म् पुण्या᳚न् ॅलो॒कान् ॅलो॒कान् पुण्या॑ नेन मेन॒म् पुण्या᳚न् ॅलो॒कान् । \newline
52. पुण्या᳚न् ॅलो॒कान् ॅलो॒कान् पुण्या॒न् पुण्या᳚न् ॅलो॒कान् ग॑मयति गमयति लो॒कान् पुण्या॒न् पुण्या᳚न् ॅलो॒कान् ग॑मयति । \newline
53. लो॒कान् ग॑मयति गमयति लो॒कान् ॅलो॒कान् ग॑मयति॒ सा सा ग॑मयति लो॒कान् ॅलो॒कान् ग॑मयति॒ सा । \newline
54. ग॒म॒य॒ति॒ सा सा ग॑मयति गमयति॒ सा मा॑ मा॒ सा ग॑मयति गमयति॒ सा मा᳚ । \newline
55. सा मा॑ मा॒ सा सा मा᳚ प्रति॒ष्ठाम् प्र॑ति॒ष्ठाम् मा॒ सा सा मा᳚ प्रति॒ष्ठाम् । \newline
\pagebreak
\markright{ TS 7.1.7.2  \hfill https://www.vedavms.in \hfill}

\section{ TS 7.1.7.2 }

\textbf{TS 7.1.7.2 } \newline
\textbf{Samhita Paata} \newline

मा᳚ प्रति॒ष्ठां ग॑मय प्र॒जया॑ प॒शुभिः॑ स॒ह पुन॒र्माऽऽ वि॑शताद्-र॒यिरिति॑ प्र॒जयै॒वैनं॑ प॒शुभी॑ र॒य्यां प्रति॑ ष्ठापयति प्र॒जावा᳚न् पशु॒मान् र॑यि॒मान् भ॑वति॒ य ए॒वं ॅवेद॒ ताम॒ग्नीधे॑ वा ब्र॒ह्मणे॑ वा॒ होत्रे॑ वोद्गा॒त्रे वा᳚ऽद्ध्व॒र्यवे॑ वा दद्याथ् स॒हस्र॑मस्य॒ सा द॒त्ता भ॑वति स॒हस्र॑मस्य॒ प्रति॑गृहीतं भवति॒ यस्तामवि॑द्वान् - [  ] \newline

\textbf{Pada Paata} \newline

मा॒ । प्र॒ति॒ष्ठामिति॑ प्रति - स्थाम् । ग॒म॒य॒ । प्र॒जयेति॑ प्र - जया᳚ । प॒शुभि॒रिति॑ प॒शु - भिः॒ । स॒ह । पुनः॑ । मा॒ । एति॑ । वि॒श॒ता॒त् । र॒यिः । इति॑ । प्र॒जयेति॑ प्र - जया᳚ । ए॒व । ए॒न॒म् । प॒शुभि॒रिति॑ प॒शु - भिः॒ । र॒य्याम् । प्रतीति॑ । स्था॒प॒य॒ति॒ । प्र॒जावा॒निति॑ प्र॒जा - वा॒न् । प॒शु॒मानिति॑ पशु - मान् । र॒यि॒मानिति॑ रयि - मान् । भ॒व॒ति॒ । यः । ए॒वम् । वेद॑ । ताम् । अ॒ग्नीध॒ इत्य॑ग्नि - इधे᳚ । वा॒ । ब्र॒ह्मणे᳚ । वा॒ । होत्रे᳚ । वा॒ । उ॒द्गा॒त्र इत्यु॑त् - गा॒त्रे । वा॒ । अ॒द्ध्व॒र्यवे᳚ । वा॒ । द॒द्या॒त् । स॒हस्र᳚म् । अ॒स्य॒ । सा । द॒त्ता । भ॒व॒ति॒ । स॒हस्र᳚म् । अ॒स्य॒ । प्रति॑गृहीत॒मिति॒ प्रति॑ - गृ॒ही॒त॒म् । भ॒व॒ति॒ । यः । ताम् । अवि॑द्वान् ।  \newline


\textbf{Krama Paata} \newline

मा॒ प्र॒ति॒ष्ठाम् । प्र॒ति॒ष्ठाम् ग॑मय । प्र॒ति॒ष्ठामिति॑ प्रति - स्थाम् । ग॒म॒य॒ प्र॒जया᳚ । प्र॒जया॑ प॒शुभिः॑ । प्र॒जयेति॑ प्र - जया᳚ । प॒शुभिः॑ स॒ह । प॒शुभि॒रिति॑ प॒शु - भिः॒ । स॒ह पुनः॑ । पुन॑र् मा । मा । आ वि॑शतात् । वि॒श॒ता॒द् र॒यिः । र॒यिरिति॑ । इति॑ प्र॒जया᳚ । प्र॒जयै॒व । प्र॒जयेति॑ प्र - जया᳚ । ए॒वैन᳚म् । ए॒न॒म् प॒शुभिः॑ । प॒शुभी॑ र॒य्याम् । प॒शुभि॒रिति॑ प॒शु - भिः॒ । र॒य्याम् प्रति॑ । प्रति॑ ष्ठापयति । स्था॒प॒य॒ति॒ प्र॒जावान्॑ । प्र॒जावा᳚न् पशु॒मान् । प्र॒जावा॒निति॑ प्र॒जा - वा॒न्॒ । प॒शु॒मान् र॑यि॒मान् । प॒शु॒मानिति॑ पशु - मान् । र॒यि॒मान् भ॑वति । र॒यि॒मानिति॑ रयि - मान् । भ॒व॒ति॒ यः । य ए॒वम् । ए॒वम् ॅवेद॑ । वेद॒ ताम् । ताम॒ग्नीधे᳚ । अ॒ग्नीधे॑ वा । अ॒ग्नीध॒ इत्य॑ग्नि - इधे᳚ । वा॒ ब्र॒ह्मणे᳚ । ब्र॒ह्मणे॑ वा । वा॒ होत्रे᳚ । होत्रे॑ वा । वो॒द्‍गा॒त्रे । उ॒द्‍गा॒त्रे वा᳚ । उ॒द्‍गा॒त्र इत्यु॑त् - गा॒त्रे । वा॒ऽद्ध्व॒र्यवे᳚ । अ॒र्द्ध॒र्यवे॑ वा । वा॒ द॒द्या॒त्॒ । द॒द्या॒थ् स॒हस्र᳚म् । स॒हस्र॑मस्य । अ॒स्य॒ सा । सा द॒त्ता । द॒त्ता भ॑वति । भ॒व॒ति॒ स॒हस्र᳚म् । स॒हस्र॑मस्य । अ॒स्य॒ प्रति॑गृहीतम् । प्रति॑गृहीतम् भवति । प्रति॑गृहीत॒मिति॒ प्रति॑ - गृ॒ही॒त॒म् । भ॒व॒ति॒ यः । यस्ताम् । तामवि॑द्वान् । अवि॑द्वान् प्रतिगृ॒ह्णाति॑ \newline

\textbf{Jatai Paata} \newline

1. मा॒ प्र॒ति॒ष्ठाम् प्र॑ति॒ष्ठाम् मा॑ मा प्रति॒ष्ठाम् । \newline
2. प्र॒ति॒ष्ठाम् ग॑मय गमय प्रति॒ष्ठाम् प्र॑ति॒ष्ठाम् ग॑मय । \newline
3. प्र॒ति॒ष्ठामिति॑ प्रति - स्थाम् । \newline
4. ग॒म॒य॒ प्र॒जया᳚ प्र॒जया॑ गमय गमय प्र॒जया᳚ । \newline
5. प्र॒जया॑ प॒शुभिः॑ प॒शुभिः॑ प्र॒जया᳚ प्र॒जया॑ प॒शुभिः॑ । \newline
6. प्र॒जयेति॑ प्र - जया᳚ । \newline
7. प॒शुभिः॑ स॒ह स॒ह प॒शुभिः॑ प॒शुभिः॑ स॒ह । \newline
8. प॒शुभि॒रिति॑ प॒शु - भिः॒ । \newline
9. स॒ह पुनः॒ पुनः॑ स॒ह स॒ह पुनः॑ । \newline
10. पुन॑र् मा मा॒ पुनः॒ पुन॑र् मा । \newline
11. मा ऽऽमा॒ मा । \newline
12. आ वि॑शताद् विशता॒दा वि॑शतात् । \newline
13. वि॒श॒ता॒द् र॒यी र॒यिर् वि॑शताद् विशताद् र॒यिः । \newline
14. र॒यिरितीति॑ र॒यी र॒यिरिति॑ । \newline
15. इति॑ प्र॒जया᳚ प्र॒जये तीति॑ प्र॒जया᳚ । \newline
16. प्र॒ज यै॒वैव प्र॒जया᳚ प्र॒ज यै॒व । \newline
17. प्र॒जयेति॑ प्र - जया᳚ । \newline
18. ए॒वैन॑ मेन मे॒वै वैन᳚म् । \newline
19. ए॒न॒म् प॒शुभिः॑ प॒शुभि॑ रेन मेनम् प॒शुभिः॑ । \newline
20. प॒शुभी॑ र॒य्याꣳ र॒य्याम् प॒शुभिः॑ प॒शुभी॑ र॒य्याम् । \newline
21. प॒शुभि॒रिति॑ प॒शु - भिः॒ । \newline
22. र॒य्याम् प्रति॒ प्रति॑ र॒य्याꣳ र॒य्याम् प्रति॑ । \newline
23. प्रति॑ ष्ठापयति स्थापयति॒ प्रति॒ प्रति॑ ष्ठापयति । \newline
24. स्था॒प॒य॒ति॒ प्र॒जावा᳚न् प्र॒जावा᳚न् थ्स्थापयति स्थापयति प्र॒जावान्॑ । \newline
25. प्र॒जावा᳚न् पशु॒मान् प॑शु॒मान् प्र॒जावा᳚न् प्र॒जावा᳚न् पशु॒मान् । \newline
26. प्र॒जावा॒निति॑ प्र॒जा - वा॒न् । \newline
27. प॒शु॒मान् र॑यि॒मान् र॑यि॒मान् प॑शु॒मान् प॑शु॒मान् र॑यि॒मान् । \newline
28. प॒शु॒मानिति॑ पशु - मान् । \newline
29. र॒यि॒मान् भ॑वति भवति रयि॒मान् र॑यि॒मान् भ॑वति । \newline
30. र॒यि॒मानिति॑ रयि - मान् । \newline
31. भ॒व॒ति॒ यो यो भ॑वति भवति॒ यः । \newline
32. य ए॒व मे॒वं ॅयो य ए॒वम् । \newline
33. ए॒वं ॅवेद॒ वेदै॒व मे॒वं ॅवेद॑ । \newline
34. वेद॒ ताम् तां ॅवेद॒ वेद॒ ताम् । \newline
35. ता म॒ग्नीधे॒ ऽग्नीधे॒ ताम् ता म॒ग्नीधे᳚ । \newline
36. अ॒ग्नीधे॑ वा वा॒ ऽग्नीधे॒ ऽग्नीधे॑ वा । \newline
37. अ॒ग्नीध॒ इत्य॑ग्नि - इधे᳚ । \newline
38. वा॒ ब्र॒ह्मणे᳚ ब्र॒ह्मणे॑ वा वा ब्र॒ह्मणे᳚ । \newline
39. ब्र॒ह्मणे॑ वा वा ब्र॒ह्मणे᳚ ब्र॒ह्मणे॑ वा । \newline
40. वा॒ होत्रे॒ होत्रे॑ वा वा॒ होत्रे᳚ । \newline
41. होत्रे॑ वा वा॒ होत्रे॒ होत्रे॑ वा । \newline
42. वो॒द्‍गा॒त्र उ॑द्‍गा॒त्रे वा॑ वोद्‍गा॒त्रे । \newline
43. उ॒द्‍गा॒त्रे वा॑ वोद्‍गा॒त्र उ॑द्‍गा॒त्रे वा᳚ । \newline
44. उ॒द्‍गा॒त्र इत्यु॑त् - गा॒त्रे । \newline
45. वा॒ ऽद्ध्व॒र्यवे᳚ ऽद्ध्व॒र्यवे॑ वा वा ऽद्ध्व॒र्यवे᳚ । \newline
46. अ॒द्ध्व॒र्यवे॑ वा वा ऽद्ध्व॒र्यवे᳚ ऽद्ध्व॒र्यवे॑ वा । \newline
47. वा॒ द॒द्या॒द् द॒द्या॒द् वा॒ वा॒ द॒द्या॒त् । \newline
48. द॒द्या॒थ् स॒हस्रꣳ॑ स॒हस्र॑म् दद्याद् दद्याथ् स॒हस्र᳚म् । \newline
49. स॒हस्र॑ मस्यास्य स॒हस्रꣳ॑ स॒हस्र॑ मस्य । \newline
50. अ॒स्य॒ सा सा ऽस्या᳚स्य॒ सा । \newline
51. सा द॒त्ता द॒त्ता सा सा द॒त्ता । \newline
52. द॒त्ता भ॑वति भवति द॒त्ता द॒त्ता भ॑वति । \newline
53. भ॒व॒ति॒ स॒हस्रꣳ॑ स॒हस्र॑म् भवति भवति स॒हस्र᳚म् । \newline
54. स॒हस्र॑ मस्यास्य स॒हस्रꣳ॑ स॒हस्र॑ मस्य । \newline
55. अ॒स्य॒ प्रति॑गृहीत॒म् प्रति॑गृहीत मस्यास्य॒ प्रति॑गृहीतम् । \newline
56. प्रति॑गृहीतम् भवति भवति॒ प्रति॑गृहीत॒म् प्रति॑गृहीतम् भवति । \newline
57. प्रति॑गृहीत॒मिति॒ प्रति॑ - गृ॒ही॒त॒म् । \newline
58. भ॒व॒ति॒ यो यो भ॑वति भवति॒ यः । \newline
59. यस्ताम् तां ॅयो यस्ताम् । \newline
60. ता मवि॑द्वा॒ नवि॑द्वा॒न् ताम् ता मवि॑द्वान् । \newline
61. अवि॑द्वान् प्रतिगृ॒ह्णाति॑ प्रतिगृ॒ह्णा त्यवि॑द्वा॒ नवि॑द्वान् प्रतिगृ॒ह्णाति॑ । \newline

\textbf{Ghana Paata } \newline

1. मा॒ प्र॒ति॒ष्ठाम् प्र॑ति॒ष्ठाम् मा॑ मा प्रति॒ष्ठाम् ग॑मय गमय प्रति॒ष्ठाम् मा॑ मा प्रति॒ष्ठाम् ग॑मय । \newline
2. प्र॒ति॒ष्ठाम् ग॑मय गमय प्रति॒ष्ठाम् प्र॑ति॒ष्ठाम् ग॑मय प्र॒जया᳚ प्र॒जया॑ गमय प्रति॒ष्ठाम् प्र॑ति॒ष्ठाम् ग॑मय प्र॒जया᳚ । \newline
3. प्र॒ति॒ष्ठामिति॑ प्रति - स्थाम् । \newline
4. ग॒म॒य॒ प्र॒जया᳚ प्र॒जया॑ गमय गमय प्र॒जया॑ प॒शुभिः॑ प॒शुभिः॑ प्र॒जया॑ गमय गमय प्र॒जया॑ प॒शुभिः॑ । \newline
5. प्र॒जया॑ प॒शुभिः॑ प॒शुभिः॑ प्र॒जया᳚ प्र॒जया॑ प॒शुभिः॑ स॒ह स॒ह प॒शुभिः॑ प्र॒जया᳚ प्र॒जया॑ प॒शुभिः॑ स॒ह । \newline
6. प्र॒जयेति॑ प्र - जया᳚ । \newline
7. प॒शुभिः॑ स॒ह स॒ह प॒शुभिः॑ प॒शुभिः॑ स॒ह पुनः॒ पुनः॑ स॒ह प॒शुभिः॑ प॒शुभिः॑ स॒ह पुनः॑ । \newline
8. प॒शुभि॒रिति॑ प॒शु - भिः॒ । \newline
9. स॒ह पुनः॒ पुनः॑ स॒ह स॒ह पुन॑र् मा मा॒ पुनः॑ स॒ह स॒ह पुन॑र् मा । \newline
10. पुन॑र् मा मा॒ पुनः॒ पुन॒र् मा ऽऽमा॒ पुनः॒ पुन॒र् मा । \newline
11. मा ऽऽमा॒ मा ऽऽवि॑शताद् विशता॒दा मा॒ मा ऽऽवि॑शतात् । \newline
12. आ वि॑शताद् विशता॒दा वि॑शताद् र॒यी र॒यिर् वि॑शता॒दा वि॑शताद् र॒यिः । \newline
13. वि॒श॒ता॒द् र॒यी र॒यिर् वि॑शताद् विशताद् र॒यि रितीति॑ र॒यिर् वि॑शताद् विशताद् र॒यि रिति॑ । \newline
14. र॒यि रितीति॑ र॒यी र॒यि रिति॑ प्र॒जया᳚ प्र॒जयेति॑ र॒यी र॒यि रिति॑ प्र॒जया᳚ । \newline
15. इति॑ प्र॒जया᳚ प्र॒जयेतीति॑ प्र॒जयै॒ वैव प्र॒जयेतीति॑ प्र॒जयै॒व । \newline
16. प्र॒जयै॒वैव प्र॒जया᳚ प्र॒जयै॒वैन॑ मेन मे॒व प्र॒जया᳚ प्र॒जयै॒वैन᳚म् । \newline
17. प्र॒जयेति॑ प्र - जया᳚ । \newline
18. ए॒वैन॑ मेन मे॒वैवैन॑म् प॒शुभिः॑ प॒शुभि॑ रेन मे॒वै वैन॑म् प॒शुभिः॑ । \newline
19. ए॒न॒म् प॒शुभिः॑ प॒शुभि॑ रेन मेनम् प॒शुभी॑ र॒य्याꣳ र॒य्याम् प॒शुभि॑ रेन मेनम् प॒शुभी॑ र॒य्याम् । \newline
20. प॒शुभी॑ र॒य्याꣳ र॒य्याम् प॒शुभिः॑ प॒शुभी॑ र॒य्याम् प्रति॒ प्रति॑ र॒य्याम् प॒शुभिः॑ प॒शुभी॑ र॒य्याम् प्रति॑ । \newline
21. प॒शुभि॒रिति॑ प॒शु - भिः॒ । \newline
22. र॒य्याम् प्रति॒ प्रति॑ र॒य्याꣳ र॒य्याम् प्रति॑ ष्ठापयति स्थापयति॒ प्रति॑ र॒य्याꣳ र॒य्याम् प्रति॑ ष्ठापयति । \newline
23. प्रति॑ ष्ठापयति स्थापयति॒ प्रति॒ प्रति॑ ष्ठापयति प्र॒जावा᳚न् प्र॒जावा᳚न् थ्स्थापयति॒ प्रति॒ प्रति॑ ष्ठापयति प्र॒जावान्॑ । \newline
24. स्था॒प॒य॒ति॒ प्र॒जावा᳚न् प्र॒जावा᳚न् थ्स्थापयति स्थापयति प्र॒जावा᳚न् पशु॒मान् प॑शु॒मान् प्र॒जावा᳚न् थ्स्थापयति स्थापयति प्र॒जावा᳚न् पशु॒मान् । \newline
25. प्र॒जावा᳚न् पशु॒मान् प॑शु॒मान् प्र॒जावा᳚न् प्र॒जावा᳚न् पशु॒मान् र॑यि॒मान् र॑यि॒मान् प॑शु॒मान् प्र॒जावा᳚न् प्र॒जावा᳚न् पशु॒मान् र॑यि॒मान् । \newline
26. प्र॒जावा॒निति॑ प्र॒जा - वा॒न् । \newline
27. प॒शु॒मान् र॑यि॒मान् र॑यि॒मान् प॑शु॒मान् प॑शु॒मान् र॑यि॒मान् भ॑वति भवति रयि॒मान् प॑शु॒मान् प॑शु॒मान् र॑यि॒मान् भ॑वति । \newline
28. प॒शु॒मानिति॑ पशु - मान् । \newline
29. र॒यि॒मान् भ॑वति भवति रयि॒मान् र॑यि॒मान् भ॑वति॒ यो यो भ॑वति रयि॒मान् र॑यि॒मान् भ॑वति॒ यः । \newline
30. र॒यि॒मानिति॑ रयि - मान् । \newline
31. भ॒व॒ति॒ यो यो भ॑वति भवति॒ य ए॒व मे॒वं ॅयो भ॑वति भवति॒ य ए॒वम् । \newline
32. य ए॒व मे॒वं ॅयो य ए॒वं ॅवेद॒ वेदै॒वं ॅयो य ए॒वं ॅवेद॑ । \newline
33. ए॒वं ॅवेद॒ वेदै॒व मे॒वं ॅवेद॒ ताम् तां ॅवेदै॒व मे॒वं ॅवेद॒ ताम् । \newline
34. वेद॒ ताम् तां ॅवेद॒ वेद॒ ता म॒ग्नीधे॒ ऽग्नीधे॒ तां ॅवेद॒ वेद॒ ता म॒ग्नीधे᳚ । \newline
35. ता म॒ग्नीधे॒ ऽग्नीधे॒ ताम् ता म॒ग्नीधे॑ वा वा॒ ऽग्नीधे॒ ताम् ता म॒ग्नीधे॑ वा । \newline
36. अ॒ग्नीधे॑ वा वा॒ ऽग्नीधे॒ ऽग्नीधे॑ वा ब्र॒ह्मणे᳚ ब्र॒ह्मणे॑ वा॒ ऽग्नीधे॒ ऽग्नीधे॑ वा ब्र॒ह्मणे᳚ । \newline
37. अ॒ग्नीध॒ इत्य॑ग्नि - इधे᳚ । \newline
38. वा॒ ब्र॒ह्मणे᳚ ब्र॒ह्मणे॑ वा वा ब्र॒ह्मणे॑ वा वा ब्र॒ह्मणे॑ वा वा ब्र॒ह्मणे॑ वा । \newline
39. ब्र॒ह्मणे॑ वा वा ब्र॒ह्मणे᳚ ब्र॒ह्मणे॑ वा॒ होत्रे॒ होत्रे॑ वा ब्र॒ह्मणे᳚ ब्र॒ह्मणे॑ वा॒ होत्रे᳚ । \newline
40. वा॒ होत्रे॒ होत्रे॑ वा वा॒ होत्रे॑ वा वा॒ होत्रे॑ वा वा॒ होत्रे॑ वा । \newline
41. होत्रे॑ वा वा॒ होत्रे॒ होत्रे॑ वोद्‍गा॒त्र उ॑द्‍गा॒त्रे वा॒ होत्रे॒ होत्रे॑ वोद्‍गा॒त्रे । \newline
42. वो॒द्‍गा॒त्र उ॑द्‍गा॒त्रे वा॑ वोद्‍गा॒त्रे वा॑ वोद्‍गा॒त्रे वा॑ वोद्‍गा॒त्रे वा᳚ । \newline
43. उ॒द्‍गा॒त्रे वा॑ वोद्‍गा॒त्र उ॑द्‍गा॒त्रे वा᳚ ऽद्ध्व॒र्यवे᳚ ऽद्ध्व॒र्यवे॑ वोद्‍गा॒त्र उ॑द्‍गा॒त्रे वा᳚ ऽद्ध्व॒र्यवे᳚ । \newline
44. उ॒द्‍गा॒त्र इत्यु॑त् - गा॒त्रे । \newline
45. वा॒ ऽद्ध्व॒र्यवे᳚ ऽद्ध्व॒र्यवे॑ वा वा ऽद्ध्व॒र्यवे॑ वा वा ऽद्ध्व॒र्यवे॑ वा वा ऽद्ध्व॒र्यवे॑ वा । \newline
46. अ॒द्ध्व॒र्यवे॑ वा वा ऽद्ध्व॒र्यवे᳚ ऽद्ध्व॒र्यवे॑ वा दद्याद् दद्याद् वा ऽद्ध्व॒र्यवे᳚ ऽद्ध्व॒र्यवे॑ वा दद्यात् । \newline
47. वा॒ द॒द्या॒द् द॒द्या॒द् वा॒ वा॒ द॒द्या॒थ् स॒हस्रꣳ॑ स॒हस्र॑म् दद्याद् वा वा दद्याथ् स॒हस्र᳚म् । \newline
48. द॒द्या॒थ् स॒हस्रꣳ॑ स॒हस्र॑म् दद्याद् दद्याथ् स॒हस्र॑ मस्यास्य स॒हस्र॑म् दद्याद् दद्याथ् स॒हस्र॑ मस्य । \newline
49. स॒हस्र॑ मस्यास्य स॒हस्रꣳ॑ स॒हस्र॑ मस्य॒ सा सा ऽस्य॑ स॒हस्रꣳ॑ स॒हस्र॑ मस्य॒ सा । \newline
50. अ॒स्य॒ सा सा ऽस्या᳚स्य॒ सा द॒त्ता द॒त्ता सा ऽस्या᳚स्य॒ सा द॒त्ता । \newline
51. सा द॒त्ता द॒त्ता सा सा द॒त्ता भ॑वति भवति द॒त्ता सा सा द॒त्ता भ॑वति । \newline
52. द॒त्ता भ॑वति भवति द॒त्ता द॒त्ता भ॑वति स॒हस्रꣳ॑ स॒हस्र॑म् भवति द॒त्ता द॒त्ता भ॑वति स॒हस्र᳚म् । \newline
53. भ॒व॒ति॒ स॒हस्रꣳ॑ स॒हस्र॑म् भवति भवति स॒हस्र॑ मस्यास्य स॒हस्र॑म् भवति भवति स॒हस्र॑ मस्य । \newline
54. स॒हस्र॑ मस्यास्य स॒हस्रꣳ॑ स॒हस्र॑ मस्य॒ प्रति॑गृहीत॒म् प्रति॑गृहीत मस्य स॒हस्रꣳ॑ स॒हस्र॑ मस्य॒ प्रति॑गृहीतम् । \newline
55. अ॒स्य॒ प्रति॑गृहीत॒म् प्रति॑गृहीत मस्यास्य॒ प्रति॑गृहीतम् भवति भवति॒ प्रति॑गृहीत मस्यास्य॒ प्रति॑गृहीतम् भवति । \newline
56. प्रति॑गृहीतम् भवति भवति॒ प्रति॑गृहीत॒म् प्रति॑गृहीतम् भवति॒ यो यो भ॑वति॒ प्रति॑गृहीत॒म् प्रति॑गृहीतम् भवति॒ यः । \newline
57. प्रति॑गृहीत॒मिति॒ प्रति॑ - गृ॒ही॒त॒म् । \newline
58. भ॒व॒ति॒ यो यो भ॑वति भवति॒ य स्ताम् तां ॅयो भ॑वति भवति॒ य स्ताम् । \newline
59. यस्ताम् तां ॅयो यस्ता मवि॑द्वा॒ नवि॑द्वा॒न् तां ॅयो यस्ता मवि॑द्वान् । \newline
60. ता मवि॑द्वा॒ नवि॑द्वा॒न् ताम् ता मवि॑द्वान् प्रतिगृ॒ह्णाति॑ प्रतिगृ॒ह्णा त्यवि॑द्वा॒न् ताम् ता मवि॑द्वान् प्रतिगृ॒ह्णाति॑ । \newline
61. अवि॑द्वान् प्रतिगृ॒ह्णाति॑ प्रतिगृ॒ह्णा त्यवि॑द्वा॒ नवि॑द्वान् प्रतिगृ॒ह्णाति॒ ताम् ताम् प्र॑तिगृ॒ह्णा त्यवि॑द्वा॒ नवि॑द्वान् प्रतिगृ॒ह्णाति॒ ताम् । \newline
\pagebreak
\markright{ TS 7.1.7.3  \hfill https://www.vedavms.in \hfill}

\section{ TS 7.1.7.3 }

\textbf{TS 7.1.7.3 } \newline
\textbf{Samhita Paata} \newline

प्रतिगृ॒ह्णाति॒ तां प्रति॑गृह्णीया॒देका॑ऽसि॒ न स॒हस्र॒मेकां᳚ त्वा भू॒तां प्रति॑ गृह्णामि॒ न स॒हस्र॒मेका॑ मा भू॒ताऽऽ वि॑श॒ मा स॒हस्र॒मित्येका॑मे॒वैनां᳚ भू॒तां प्रति॑गृह्णाति॒ न स॒हस्रं॒ ॅय ए॒वं ॅवेद॑ स्यो॒नाऽसि॑ सु॒षदा॑ सु॒शेवा᳚ स्यो॒ना मा ऽऽवि॑श सु॒षदा॒ मा ऽऽवि॑श सु॒शेवा॒ मा ऽऽवि॒शे -[  ] \newline

\textbf{Pada Paata} \newline

प्र॒ति॒गृ॒ह्णातीति॑ प्रति - गृ॒ह्णाति॑ । ताम् । प्रतीति॑ । गृ॒ह्णी॒या॒त् । एका᳚ । अ॒सि॒ । न । स॒हस्र᳚म् । एका᳚म् । त्वा॒ । भू॒ताम् । प्रतीति॑ । गृ॒ह्णा॒मि॒ । न । स॒हस्र᳚म् । एका᳚ । मा॒ । भू॒ता । एति॑ । वि॒श॒ । मा । स॒हस्र᳚म् । इति॑ । एका᳚म् । ए॒व । ए॒ना॒म् । भू॒ताम् । प्रतीति॑ । गृ॒ह्णा॒ति॒ । न । स॒हस्र᳚म् । यः । ए॒वम् । वेद॑ । स्यो॒ना । अ॒सि॒ । सु॒षदेति॑ सु-सदा᳚ । सु॒शेवेति॑ सु - शेवा᳚ । स्यो॒ना । मा॒ । एति॑ । वि॒श॒ । सु॒षदेति॑ सु - सदा᳚ । मा॒ । एति॑ । वि॒श॒ । सु॒शेवेति॑ सु - शेवा᳚ । मा॒ । एति॑ । वि॒श॒ ।  \newline


\textbf{Krama Paata} \newline

प्र॒ति॒गृ॒ह्णाति॒ ताम् । प्र॒ति॒गृ॒ह्णातीति॑ प्रति - गृ॒ह्णाति॑ । ताम् प्रति॑ । प्रति॑ गृह्णीयात् । गृ॒ह्णी॒या॒देका᳚ । एका॑ऽसि । अ॒सि॒ न । न स॒हस्र᳚म् । स॒हस्र॒मेका᳚म् । एका᳚म् त्वा । त्वा॒ भू॒ताम् । भू॒ताम् प्रति॑ । प्रति॑ गृह्णामि । गृ॒ह्णा॒मि॒ न । न स॒हस्र᳚म् । स॒हस्र॒मेका᳚ । एका॑ मा । मा॒ भू॒ता । भू॒ता ऽऽ वि॑श । आ वि॑श । वि॒श॒ मा । मा स॒हस्र᳚म् । स॒हस्र॒मिति॑ । इत्येका᳚म् । एका॑मे॒व । ए॒वैना᳚म् । ए॒ना॒म् भू॒ताम् । भू॒ताम् प्रति॑ । प्रति॑ गृह्णाति । गृ॒ह्णा॒ति॒ न । न स॒हस्र᳚म् । स॒हस्र॒म् ॅयः । य ए॒वम् । ए॒वम् ॅवेद॑ । वेद॑ स्यो॒ना । स्यो॒नाऽसि॑ । अ॒सि॒ सु॒षदा᳚ । सु॒षदा॑ सु॒शेवा᳚ । सु॒षदेति॑ सु - सदा᳚ । सु॒शेवा᳚ स्यो॒ना । सु॒शेवेति॑ सु - शेवा᳚ । स्यो॒ना मा᳚ । मा । आ वि॑श । वि॒श॒ सु॒षदा᳚ । सु॒षदा॑ मा । सु॒षदेति॑ सु - सदा᳚ । 
मा । आ वि॑श । वि॒श॒ सु॒शेवा᳚ । सु॒शेवा॑ मा । सु॒शेवेति॑ सु - शेवा᳚ । मा । आ वि॑श । वि॒शेति॑ \newline

\textbf{Jatai Paata} \newline

1. प्र॒ति॒गृ॒ह्णाति॒ ताम् ताम् प्र॑तिगृ॒ह्णाति॑ प्रतिगृ॒ह्णाति॒ ताम् । \newline
2. प्र॒ति॒गृ॒ह्णातीति॑ प्रति - गृ॒ह्णाति॑ । \newline
3. ताम् प्रति॒ प्रति॒ ताम् ताम् प्रति॑ । \newline
4. प्रति॑ गृह्णीयाद् गृह्णीया॒त् प्रति॒ प्रति॑ गृह्णीयात् । \newline
5. गृ॒ह्णी॒या॒ देकैका॑ गृह्णीयाद् गृह्णीया॒ देका᳚ । \newline
6. एका᳚ ऽस्य॒ स्येकैका॑ ऽसि । \newline
7. अ॒सि॒ न नास्य॑सि॒ न । \newline
8. न स॒हस्रꣳ॑ स॒हस्र॒न् न न स॒हस्र᳚म् । \newline
9. स॒हस्र॒ मेका॒ मेकाꣳ॑ स॒हस्रꣳ॑ स॒हस्र॒ मेका᳚म् । \newline
10. एका᳚म् त्वा॒ त्वैका॒ मेका᳚म् त्वा । \newline
11. त्वा॒ भू॒ताम् भू॒ताम् त्वा᳚ त्वा भू॒ताम् । \newline
12. भू॒ताम् प्रति॒ प्रति॑ भू॒ताम् भू॒ताम् प्रति॑ । \newline
13. प्रति॑ गृह्णामि गृह्णामि॒ प्रति॒ प्रति॑ गृह्णामि । \newline
14. गृ॒ह्णा॒मि॒ न न गृ॑ह्णामि गृह्णामि॒ न । \newline
15. न स॒हस्रꣳ॑ स॒हस्र॒न्न न स॒हस्र᳚म् । \newline
16. स॒हस्र॒ मेकैका॑ स॒हस्रꣳ॑ स॒हस्र॒ मेका᳚ । \newline
17. एका॑ मा॒ मैकैका॑ मा । \newline
18. मा॒ भू॒ता भू॒ता मा॑ मा भू॒ता । \newline
19. भू॒ता ऽऽवि॑श वि॒शा भू॒ता भू॒ता ऽऽवि॑श । \newline
20. आ वि॑श वि॒शा वि॑श । \newline
21. वि॒श॒ मा मा वि॑श विश॒ मा । \newline
22. मा स॒हस्रꣳ॑ स॒हस्र॒म् मा मा स॒हस्र᳚म् । \newline
23. स॒हस्र॒ मितीति॑ स॒हस्रꣳ॑ स॒हस्र॒ मिति॑ । \newline
24. इत्येका॒ मेका॒ मिती त्येका᳚म् । \newline
25. एका॑ मे॒वै वैका॒ मेका॑ मे॒व । \newline
26. ए॒वैना॑ मेना मे॒वै वैना᳚म् । \newline
27. ए॒ना॒म् भू॒ताम् भू॒ता मे॑ना मेनाम् भू॒ताम् । \newline
28. भू॒ताम् प्रति॒ प्रति॑ भू॒ताम् भू॒ताम् प्रति॑ । \newline
29. प्रति॑ गृह्णाति गृह्णाति॒ प्रति॒ प्रति॑ गृह्णाति । \newline
30. गृ॒ह्णा॒ति॒ न न गृ॑ह्णाति गृह्णाति॒ न । \newline
31. न स॒हस्रꣳ॑ स॒हस्र॒न्न न स॒हस्र᳚म् । \newline
32. स॒हस्रं॒ ॅयो यः स॒हस्रꣳ॑ स॒हस्रं॒ ॅयः । \newline
33. य ए॒व मे॒वं ॅयो य ए॒वम् । \newline
34. ए॒वं ॅवेद॒ वेदै॒व मे॒वं ॅवेद॑ । \newline
35. वेद॑ स्यो॒ना स्यो॒ना वेद॒ वेद॑ स्यो॒ना । \newline
36. स्यो॒ना ऽस्य॑सि स्यो॒ना स्यो॒ना ऽसि॑ । \newline
37. अ॒सि॒ सु॒षदा॑ सु॒षदा᳚ ऽस्यसि सु॒षदा᳚ । \newline
38. सु॒षदा॑ सु॒शेवा॑ सु॒शेवा॑ सु॒षदा॑ सु॒षदा॑ सु॒शेवा᳚ । \newline
39. सु॒षदेति॑ सु - सदा᳚ । \newline
40. सु॒शेवा᳚ स्यो॒ना स्यो॒ना सु॒शेवा॑ सु॒शेवा᳚ स्यो॒ना । \newline
41. सु॒शेवेति॑ सु - शेवा᳚ । \newline
42. स्यो॒ना मा॑ मा स्यो॒ना स्यो॒ना मा᳚ । \newline
43. माऽऽ मा॒ मा । \newline
44. आ वि॑श वि॒शा वि॑श । \newline
45. वि॒श॒ सु॒षदा॑ सु॒षदा॑ विश विश सु॒षदा᳚ । \newline
46. सु॒षदा॑ मा मा सु॒षदा॑ सु॒षदा॑ मा । \newline
47. सु॒षदेति॑ सु - सदा᳚ । \newline
48. माऽऽ मा॒ मा । \newline
49. आ वि॑श वि॒शा वि॑श । \newline
50. वि॒श॒ सु॒शेवा॑ सु॒शेवा॑ विश विश सु॒शेवा᳚ । \newline
51. सु॒शेवा॑ मा मा सु॒शेवा॑ सु॒शेवा॑ मा । \newline
52. सु॒शेवेति॑ सु - शेवा᳚ । \newline
53. माऽऽ मा॒ मा । \newline
54. आ वि॑श वि॒शा वि॑श । \newline
55. वि॒शेतीति॑ विश वि॒शेति॑ । \newline

\textbf{Ghana Paata } \newline

1. प्र॒ति॒गृ॒ह्णाति॒ ताम् ताम् प्र॑तिगृ॒ह्णाति॑ प्रतिगृ॒ह्णाति॒ ताम् प्रति॒ प्रति॒ ताम् प्र॑तिगृ॒ह्णाति॑ प्रतिगृ॒ह्णाति॒ ताम् प्रति॑ । \newline
2. प्र॒ति॒गृ॒ह्णातीति॑ प्रति - गृ॒ह्णाति॑ । \newline
3. ताम् प्रति॒ प्रति॒ ताम् ताम् प्रति॑ गृह्णीयाद् गृह्णीया॒त् प्रति॒ ताम् ताम् प्रति॑ गृह्णीयात् । \newline
4. प्रति॑ गृह्णीयाद् गृह्णीया॒त् प्रति॒ प्रति॑ गृह्णीया॒ देकैका॑ गृह्णीया॒त् प्रति॒ प्रति॑ गृह्णीया॒ देका᳚ । \newline
5. गृ॒ह्णी॒या॒ देकैका॑ गृह्णीयाद् गृह्णीया॒ देका᳚ ऽस्य॒ स्येका॑ गृह्णीयाद् गृह्णीया॒ देका॑ ऽसि । \newline
6. एका᳚ ऽस्य॒ स्येकैका॑ ऽसि॒ न ना स्येकैका॑ ऽसि॒ न । \newline
7. अ॒सि॒ न नास्य॑सि॒ न स॒हस्रꣳ॑ स॒हस्र॒म् नास्य॑सि॒ न स॒हस्र᳚म् । \newline
8. न स॒हस्रꣳ॑ स॒हस्र॒न् न न स॒हस्र॒ मेका॒ मेकाꣳ॑ स॒हस्र॒न् न न स॒हस्र॒ मेका᳚म् । \newline
9. स॒हस्र॒ मेका॒ मेकाꣳ॑ स॒हस्रꣳ॑ स॒हस्र॒ मेका᳚म् त्वा॒ त्वैकाꣳ॑ स॒हस्रꣳ॑ स॒हस्र॒ मेका᳚म् त्वा । \newline
10. एका᳚म् त्वा॒ त्वैका॒ मेका᳚म् त्वा भू॒ताम् भू॒ताम् त्वैका॒ मेका᳚म् त्वा भू॒ताम् । \newline
11. त्वा॒ भू॒ताम् भू॒ताम् त्वा᳚ त्वा भू॒ताम् प्रति॒ प्रति॑ भू॒ताम् त्वा᳚ त्वा भू॒ताम् प्रति॑ । \newline
12. भू॒ताम् प्रति॒ प्रति॑ भू॒ताम् भू॒ताम् प्रति॑ गृह्णामि गृह्णामि॒ प्रति॑ भू॒ताम् भू॒ताम् प्रति॑ गृह्णामि । \newline
13. प्रति॑ गृह्णामि गृह्णामि॒ प्रति॒ प्रति॑ गृह्णामि॒ न न गृ॑ह्णामि॒ प्रति॒ प्रति॑ गृह्णामि॒ न । \newline
14. गृ॒ह्णा॒मि॒ न न गृ॑ह्णामि गृह्णामि॒ न स॒हस्रꣳ॑ स॒हस्र॒न् न गृ॑ह्णामि गृह्णामि॒ न स॒हस्र᳚म् । \newline
15. न स॒हस्रꣳ॑ स॒हस्र॒न् न न स॒हस्र॒ मेकैका॑ स॒हस्र॒न् न न स॒हस्र॒ मेका᳚ । \newline
16. स॒हस्र॒ मेकैका॑ स॒हस्रꣳ॑ स॒हस्र॒ मेका॑ मा॒ मैका॑ स॒हस्रꣳ॑ स॒हस्र॒ मेका॑ मा । \newline
17. एका॑ मा॒ मैकैका॑ मा भू॒ता भू॒ता मैकैका॑ मा भू॒ता । \newline
18. मा॒ भू॒ता भू॒ता मा॑ मा भू॒ता ऽऽवि॑श वि॒शा भू॒ता मा॑ मा भू॒ता ऽऽवि॑श । \newline
19. भू॒ता ऽऽवि॑श वि॒शा भू॒ता भू॒ता ऽऽवि॑श॒ मा मा वि॒शा भू॒ता भू॒ता ऽऽवि॑श॒ मा । \newline
20. आ वि॑श वि॒शा वि॑श॒ मा मा वि॒शा वि॑श॒ मा । \newline
21. वि॒श॒ मा मा वि॑श विश॒ मा स॒हस्रꣳ॑ स॒हस्र॒म् मा वि॑श विश॒ मा स॒हस्र᳚म् । \newline
22. मा स॒हस्रꣳ॑ स॒हस्र॒म् मा मा स॒हस्र॒ मितीति॑ स॒हस्र॒म् मा मा स॒हस्र॒ मिति॑ । \newline
23. स॒हस्र॒ मितीति॑ स॒हस्रꣳ॑ स॒हस्र॒ मित्येका॒ मेका॒ मिति॑ स॒हस्रꣳ॑ स॒हस्र॒ मित्येका᳚म् । \newline
24. इत्येका॒ मेका॒ मिती त्येका॑ मे॒वै वैका॒ मिती त्येका॑ मे॒व । \newline
25. एका॑ मे॒वै वैका॒ मेका॑ मे॒वैना॑ मेना मे॒वैका॒ मेका॑ मे॒वैना᳚म् । \newline
26. ए॒वैना॑ मेना मे॒वै वैना᳚म् भू॒ताम् भू॒ता मे॑ना मे॒वै वैना᳚म् भू॒ताम् । \newline
27. ए॒ना॒म् भू॒ताम् भू॒ता मे॑ना मेनाम् भू॒ताम् प्रति॒ प्रति॑ भू॒ता मे॑ना मेनाम् भू॒ताम् प्रति॑ । \newline
28. भू॒ताम् प्रति॒ प्रति॑ भू॒ताम् भू॒ताम् प्रति॑ गृह्णाति गृह्णाति॒ प्रति॑ भू॒ताम् भू॒ताम् प्रति॑ गृह्णाति । \newline
29. प्रति॑ गृह्णाति गृह्णाति॒ प्रति॒ प्रति॑ गृह्णाति॒ न न गृ॑ह्णाति॒ प्रति॒ प्रति॑ गृह्णाति॒ न । \newline
30. गृ॒ह्णा॒ति॒ न न गृ॑ह्णाति गृह्णाति॒ न स॒हस्रꣳ॑ स॒हस्र॒न् न गृ॑ह्णाति गृह्णाति॒ न स॒हस्र᳚म् । \newline
31. न स॒हस्रꣳ॑ स॒हस्र॒न् न न स॒हस्रं॒ ॅयो यः स॒हस्र॒न् न न स॒हस्रं॒ ॅयः । \newline
32. स॒हस्रं॒ ॅयो यः स॒हस्रꣳ॑ स॒हस्रं॒ ॅय ए॒व मे॒वं ॅयः स॒हस्रꣳ॑ स॒हस्रं॒ ॅय ए॒वम् । \newline
33. य ए॒व मे॒वं ॅयो य ए॒वं ॅवेद॒ वेदै॒वं ॅयो य ए॒वं ॅवेद॑ । \newline
34. ए॒वं ॅवेद॒ वेदै॒व मे॒वं ॅवेद॑ स्यो॒ना स्यो॒ना वेदै॒व मे॒वं ॅवेद॑ स्यो॒ना । \newline
35. वेद॑ स्यो॒ना स्यो॒ना वेद॒ वेद॑ स्यो॒ना ऽस्य॑सि स्यो॒ना वेद॒ वेद॑ स्यो॒ना ऽसि॑ । \newline
36. स्यो॒ना ऽस्य॑सि स्यो॒ना स्यो॒ना ऽसि॑ सु॒षदा॑ सु॒षदा॑ ऽसि स्यो॒ना स्यो॒ना ऽसि॑ सु॒षदा᳚ । \newline
37. अ॒सि॒ सु॒षदा॑ सु॒षदा᳚ ऽस्यसि सु॒षदा॑ सु॒शेवा॑ सु॒शेवा॑ सु॒षदा᳚ ऽस्यसि सु॒षदा॑ सु॒शेवा᳚ । \newline
38. सु॒षदा॑ सु॒शेवा॑ सु॒शेवा॑ सु॒षदा॑ सु॒षदा॑ सु॒शेवा᳚ स्यो॒ना स्यो॒ना सु॒शेवा॑ सु॒षदा॑ सु॒षदा॑ सु॒शेवा᳚ स्यो॒ना । \newline
39. सु॒षदेति॑ सु - सदा᳚ । \newline
40. सु॒शेवा᳚ स्यो॒ना स्यो॒ना सु॒शेवा॑ सु॒शेवा᳚ स्यो॒ना मा॑ मा स्यो॒ना सु॒शेवा॑ सु॒शेवा᳚ स्यो॒ना मा᳚ । \newline
41. सु॒शेवेति॑ सु - शेवा᳚ । \newline
42. स्यो॒ना मा॒ मा॒ स्यो॒ना स्यो॒ना माऽऽ मा᳚ स्यो॒ना स्यो॒ना मा । \newline
43. माऽऽ मा॒ माऽऽ वि॑श वि॒शा मा॒ माऽऽ वि॑श । \newline
44. आ वि॑श वि॒शा वि॑श सु॒षदा॑ सु॒षदा॑ वि॒शा वि॑श सु॒षदा᳚ । \newline
45. वि॒श॒ सु॒षदा॑ सु॒षदा॑ विश विश सु॒षदा॑ मा मा सु॒षदा॑ विश विश सु॒षदा॑ मा । \newline
46. सु॒षदा॑ मा मा सु॒षदा॑ सु॒षदा॒ माऽऽ मा॑ सु॒षदा॑ सु॒षदा॒ मा । \newline
47. सु॒षदेति॑ सु - सदा᳚ । \newline
48. माऽऽ मा॒ माऽऽ वि॑श वि॒शा मा॒ माऽऽ वि॑श । \newline
49. आ वि॑श वि॒शा वि॑श सु॒शेवा॑ सु॒शेवा॑ वि॒शा वि॑श सु॒शेवा᳚ । \newline
50. वि॒श॒ सु॒शेवा॑ सु॒शेवा॑ विश विश सु॒शेवा॑ मा मा सु॒शेवा॑ विश विश सु॒शेवा॑ मा । \newline
51. सु॒शेवा॑ मा मा सु॒शेवा॑ सु॒शेवा॒ माऽऽ मा॑ सु॒शेवा॑ सु॒शेवा॒ मा । \newline
52. सु॒शेवेति॑ सु - शेवा᳚ । \newline
53. माऽऽ मा॒ माऽऽ वि॑श वि॒शा मा॒ माऽऽ वि॑श । \newline
54. आ वि॑श वि॒शा वि॒शेतीति॑ वि॒शा वि॒शेति॑ । \newline
55. वि॒शेतीति॑ विश वि॒शे त्या॑हा॒ हेति॑ विश वि॒शे त्या॑ह । \newline
\pagebreak
\markright{ TS 7.1.7.4  \hfill https://www.vedavms.in \hfill}

\section{ TS 7.1.7.4 }

\textbf{TS 7.1.7.4 } \newline
\textbf{Samhita Paata} \newline

-त्या॑ह स्यो॒नैवैनꣳ॑ सु॒षदा॑ सु॒शेवा॑ भू॒ताऽऽ वि॑शति॒ नैनꣳ॑ हिनस्ति ब्रह्मवा॒दिनो॑ वदन्ति स॒हस्रꣳ॑ सहस्रत॒म्यन्वे॒ती(3) स॑हस्रत॒मीꣳ स॒हस्रा(3)मिति॒ यत् प्राची॑मुथ् सृ॒जेथ् स॒हस्रꣳ॑ सहस्रत॒म्यन्वि॑या॒त् तथ् स॒हस्र॑मप्रज्ञा॒त्रꣳ सु॑व॒र्गं ॅलो॒कं न प्र जा॑नीयात् प्र॒तीची॒मुथ्- सृ॑जति॒ ताꣳ स॒हस्र॒मनु॑ प॒र्याव॑र्तते॒ सा प्र॑जान॒ती सु॑व॒र्गं ॅलो॒कमे॑ति॒ यज॑मान ( ) -म॒भ्युथ् सृ॑जति क्षि॒प्रे स॒हस्रं॒ प्र जा॑यत उत्त॒मा नी॒यते᳚ प्रथ॒मा दे॒वान् ग॑च्छति ॥ \newline

\textbf{Pada Paata} \newline

इति॑ । आ॒ह॒ । स्यो॒ना । ए॒व । ए॒न॒म् । सु॒षदेति॑ सु - सदा᳚ । सु॒शेवेति॑ सु - शेवा᳚ । भू॒ता । एति॑ । वि॒श॒ति॒ । न । ए॒न॒म् । हि॒न॒स्ति॒ । ब्र॒ह्म॒वा॒दिन॒ इति॑ ब्रह्म - वा॒दिनः॑ । व॒द॒न्ति॒ । स॒हस्र᳚म् । स॒ह॒स्र॒त॒मीति॑ सहस्र - त॒मी । अन्विति॑ । ए॒ती(3) । स॒ह॒स्र॒त॒मीमिति॑ सहस्र - त॒मीम् । स॒हस्रा(3)म् । इति॑ । यत् । प्राची᳚म् । उ॒थ्सृ॒जेदित्यु॑त् - सृ॒जेत् । स॒हस्र᳚म् । स॒ह॒स्र॒त॒मीति॑ सहस्र - त॒मी । अन्विति॑ । इ॒या॒त् । तत् । स॒हस्र᳚म् । अ॒प्र॒ज्ञा॒त्रमित्य॑प्र - ज्ञा॒त्रम् । सु॒व॒र्गमिति॑ सुवः - गम् । लो॒कम् । न । प्रेति॑ । जा॒नी॒या॒त् । प्र॒तीची᳚म् । उदिति॑ । सृ॒ज॒ति॒ । ताम् । स॒हस्र᳚म् । अन्विति॑ । प॒र्याव॑र्तत॒ इति॑ परि - आव॑र्तते । सा । प्र॒जा॒न॒तीति॑ प्र - जा॒न॒ती । सु॒व॒र्गमिति॑ सुवः - गम् । लो॒कम् । ए॒ति॒ । यज॑मानम् ( ) । अ॒भि । उदिति॑ । सृ॒ज॒ति॒ । क्षि॒प्रे । स॒हस्र᳚म् । प्रेति॑ । जा॒य॒ते॒ । उ॒त्त॒मेत्यु॑त् - त॒मा । नी॒यते᳚ । प्र॒थ॒मा । दे॒वान् । ग॒च्छ॒ति॒ ॥  \newline


\textbf{Krama Paata} \newline

इत्या॑ह । आ॒ह॒ स्यो॒ना । स्यो॒नैव । ए॒वैन᳚म् । ए॒नꣳ॒॒ सु॒षदा᳚ । सु॒षदा॑ सु॒शेवा᳚ । सु॒षदेति॑ सु - सदा᳚ । सु॒शेवा॑ भू॒ता । सु॒शेवेति॑ सु - शेवा᳚ । भू॒ता ऽऽ वि॑शति । आ वि॑शति । वि॒श॒ति॒ न । नैन᳚म् । ए॒नꣳ॒॒ हि॒न॒स्ति॒ । हि॒न॒स्ति॒ ब्र॒ह्म॒वा॒दिनः॑ । ब्र॒ह्म॒वा॒दिनो॑ वदन्ति । ब्र॒ह्म॒वा॒दिन॒ इति॑ ब्रह्म - वा॒दिनः॑ । व॒द॒न्ति॒ स॒हस्र᳚म् । स॒हस्रꣳ॑ सहस्रत॒मी । स॒ह॒स्र॒त॒म्यनु॑ । स॒ह॒स्र॒त॒मीति॑ सहस्र - त॒मी । अन्वे॒ती(3) । ए॒ती(3) स॑हस्रत॒मीम् । स॒ह॒स्र॒त॒मीꣳ स॒हस्रा(3)म् । स॒ह॒स्र॒त॒मीमिति॑ सहस्र - त॒मीम् । स॒हस्रा(3)मिति॑ । इति॒ यत् । यत् प्राची᳚म् । प्राची॑मुथ्सृ॒जेत् । उ॒थ्सृ॒जेथ् स॒हस्र᳚म् । उ॒थ्सृ॒जेदित्यु॑त् - सृ॒जेत् । स॒हस्रꣳ॑ सहस्रत॒मी । स॒ह॒स्र॒त॒म्यनु॑ । स॒ह॒स्र॒त॒मीति॑ सहस्र - त॒मी । अन्वि॑यात् । इ॒या॒त् तत् । तथ् स॒हस्र᳚म् । स॒हस्र॑मप्रज्ञा॒त्रम् । अ॒प्र॒ज्ञा॒त्रꣳ सु॑व॒र्गम् । अ॒प्र॒ज्ञा॒त्रमित्य॑प्र - ज्ञा॒त्रम् । सु॒व॒र्गम् ॅलो॒कम् । सु॒व॒र्गमिति॑ सुवः - गम् । लो॒कम् न । न प्र । प्र जा॑नीयात् । जा॒नी॒या॒त् प्र॒तीची᳚म् । प्र॒तीची॒मुत् । 
उथ् सृ॑जति । सृ॒ज॒ति॒ ताम् । ताꣳ स॒हस्र᳚म् । स॒हस्र॒मनु॑ । अनु॑ प॒र्याव॑र्तते । प॒र्याव॑र्तते॒ सा । प॒र्याव॑र्तत॒ इति॑ परि - आव॑र्तते । सा प्र॑जान॒ती । प्र॒जा॒न॒ती सु॑व॒र्गम् । प्र॒जा॒न॒तीति॑ प्र - जा॒न॒ती । सु॒व॒र्गम् ॅलो॒कम् । सु॒व॒र्गमिति॑ सुवः - गम् । लो॒कमे॑ति । ए॒ति॒ यज॑मानम् ( ) । यज॑मानम॒भि । अ॒भ्युत् । उथ् सृ॑जति । सृ॒ज॒ति॒ क्षि॒प्रे । क्षि॒प्रे स॒हस्र᳚म् । स॒हस्र॒म् प्र । प्र जा॑यते । जा॒य॒त॒ उ॒त्त॒मा । उ॒त्त॒मा नी॒यते᳚ । उ॒त्त॒मेत्यु॑त् - त॒मा । नी॒यते᳚ प्रथ॒मा । प्र॒थ॒मा दे॒वान् । दे॒वान् ग॑च्छति । ग॒च्छ॒तीति॑ गच्छति । \newline

\textbf{Jatai Paata} \newline

1. इत्या॑हा॒हे तीत्या॑ह । \newline
2. आ॒ह॒ स्यो॒ना स्यो॒ना ऽऽहा॑ह स्यो॒ना । \newline
3. स्यो॒नैवैव स्यो॒ना स्यो॒नैव । \newline
4. ए॒वैन॑ मेन मे॒वै वैन᳚म् । \newline
5. ए॒नꣳ॒॒ सु॒षदा॑ सु॒षदै॑न मेनꣳ सु॒षदा᳚ । \newline
6. सु॒षदा॑ सु॒शेवा॑ सु॒शेवा॑ सु॒षदा॑ सु॒षदा॑ सु॒शेवा᳚ । \newline
7. सु॒षदेति॑ सु - सदा᳚ । \newline
8. सु॒शेवा॑ भू॒ता भू॒ता सु॒शेवा॑ सु॒शेवा॑ भू॒ता । \newline
9. सु॒शेवेति॑ सु - शेवा᳚ । \newline
10. भू॒ता ऽऽवि॑शति विश॒त्या भू॒ता भू॒ता ऽऽवि॑शति । \newline
11. आ वि॑शति विश॒त्या वि॑शति । \newline
12. वि॒श॒ति॒ न न वि॑शति विशति॒ न । \newline
13. नैन॑ मेन॒न् न नैन᳚म् । \newline
14. ए॒नꣳ॒॒ हि॒न॒स्ति॒ हि॒न॒ स्त्ये॒न॒ मे॒नꣳ॒॒ हि॒न॒स्ति॒ । \newline
15. हि॒न॒स्ति॒ ब्र॒ह्म॒वा॒दिनो᳚ ब्रह्मवा॒दिनो॑ हिनस्ति हिनस्ति ब्रह्मवा॒दिनः॑ । \newline
16. ब्र॒ह्म॒वा॒दिनो॑ वदन्ति वदन्ति ब्रह्मवा॒दिनो᳚ ब्रह्मवा॒दिनो॑ वदन्ति । \newline
17. ब्र॒ह्म॒वा॒दिन॒ इति॑ ब्रह्म - वा॒दिनः॑ । \newline
18. व॒द॒न्ति॒ स॒हस्रꣳ॑ स॒हस्रं॑ ॅवदन्ति वदन्ति स॒हस्र᳚म् । \newline
19. स॒हस्रꣳ॑ सहस्रत॒मी स॑हस्रत॒मी स॒हस्रꣳ॑ स॒हस्रꣳ॑ सहस्रत॒मी । \newline
20. स॒ह॒स्र॒त॒ म्यन् वनु॑ सहस्रत॒मी स॑हस्रत॒ म्यनु॑ । \newline
21. स॒ह॒स्र॒त॒मीति॑ सहस्र - त॒मी । \newline
22. अन्वे॒ती(3) ए॒ती(3) अन्वन् वे॒ती(3) । \newline
23. ए॒ती(3) स॑हस्रत॒मीꣳ स॑हस्रत॒मी मे॒ती(3) ए॑ती(3) स॑हस्रत॒मीम् । \newline
24. स॒ह॒स्र॒त॒मीꣳ स॒हस्रा(3)ꣳ स॒हस्रा(3)ꣳ स॑हस्रत॒मीꣳ स॑हस्रत॒मीꣳ स॒हस्रा(3)म् । \newline
25. स॒ह॒स्र॒त॒मीमिति॑ सहस्र - त॒मीम् । \newline
26. स॒हस्रा(3) मितीति॑ स॒हस्रा(3)ꣳ स॒हस्रा(3) मिति॑ । \newline
27. इति॒ यद् यदितीति॒ यत् । \newline
28. यत् प्राची॒म् प्राचीं॒ ॅयद् यत् प्राची᳚म् । \newline
29. प्राची॑ मुथ्सृ॒जे दु॑थ्सृ॒जेत् प्राची॒म् प्राची॑ मुथ्सृ॒जेत् । \newline
30. उ॒थ्सृ॒जेथ् स॒हस्रꣳ॑ स॒हस्र॑ मुथ्सृ॒जे दु॑थ्सृ॒जेथ् स॒हस्र᳚म् । \newline
31. उ॒थ्सृ॒जेदित्यु॑त् - सृ॒जेत् । \newline
32. स॒हस्रꣳ॑ सहस्रत॒मी स॑हस्रत॒मी स॒हस्रꣳ॑ स॒हस्रꣳ॑ सहस्रत॒मी । \newline
33. स॒ह॒स्र॒त॒ म्यन् वनु॑ सहस्रत॒मी स॑हस्रत॒ म्यनु॑ । \newline
34. स॒ह॒स्र॒त॒मीति॑ सहस्र - त॒मी । \newline
35. अन्वि॑या दिया॒ दन् वन् वि॑यात् । \newline
36. इ॒या॒त् तत् तदि॑या दिया॒त् तत् । \newline
37. तथ् स॒हस्रꣳ॑ स॒हस्र॒म् तत् तथ् स॒हस्र᳚म् । \newline
38. स॒हस्र॑ मप्रज्ञा॒त्र म॑प्रज्ञा॒त्रꣳ स॒हस्रꣳ॑ स॒हस्र॑ मप्रज्ञा॒त्रम् । \newline
39. अ॒प्र॒ज्ञा॒त्रꣳ सु॑व॒र्गꣳ सु॑व॒र्ग म॑प्रज्ञा॒त्र म॑प्रज्ञा॒त्रꣳ सु॑व॒र्गम् । \newline
40. अ॒प्र॒ज्ञा॒त्रमित्य॑प्र - ज्ञा॒त्रम् । \newline
41. सु॒व॒र्गम् ॅलो॒कम् ॅलो॒कꣳ सु॑व॒र्गꣳ सु॑व॒र्गम् ॅलो॒कम् । \newline
42. सु॒व॒र्गमिति॑ सुवः - गम् । \newline
43. लो॒कन् न न लो॒कम् ॅलो॒कन् न । \newline
44. न प्र प्र ण न प्र । \newline
45. प्र जा॑नीयाज् जानीया॒त् प्र प्र जा॑नीयात् । \newline
46. जा॒नी॒या॒त् प्र॒तीची᳚म् प्र॒तीची᳚म् जानीयाज् जानीयात् प्र॒तीची᳚म् । \newline
47. प्र॒तीची॒ मुदुत् प्र॒तीची᳚म् प्र॒तीची॒ मुत् । \newline
48. उथ् सृ॑जति सृज॒ त्युदुथ् सृ॑जति । \newline
49. सृ॒ज॒ति॒ ताम् ताꣳ सृ॑जति सृजति॒ ताम् । \newline
50. ताꣳ स॒हस्रꣳ॑ स॒हस्र॒म् ताम् ताꣳ स॒हस्र᳚म् । \newline
51. स॒हस्र॒ मन्वनु॑ स॒हस्रꣳ॑ स॒हस्र॒ मनु॑ । \newline
52. अनु॑ प॒र्याव॑र्तते प॒र्याव॑र्त॒ते ऽन्वनु॑ प॒र्याव॑र्तते । \newline
53. प॒र्याव॑र्तते॒ सा सा प॒र्याव॑र्तते प॒र्याव॑र्तते॒ सा । \newline
54. प॒र्याव॑र्तत॒ इति॑ परि - आव॑र्तते । \newline
55. सा प्र॑जान॒ती प्र॑जान॒ती सा सा प्र॑जान॒ती । \newline
56. प्र॒जा॒न॒ती सु॑व॒र्गꣳ सु॑व॒र्गम् प्र॑जान॒ती प्र॑जान॒ती सु॑व॒र्गम् । \newline
57. प्र॒जा॒न॒तीति॑ प्र - जा॒न॒ती । \newline
58. सु॒व॒र्गम् ॅलो॒कम् ॅलो॒कꣳ सु॑व॒र्गꣳ सु॑व॒र्गम् ॅलो॒कम् । \newline
59. सु॒व॒र्गमिति॑ सुवः - गम् । \newline
60. लो॒क मे᳚त्येति लो॒कम् ॅलो॒क मे॑ति । \newline
61. ए॒ति॒ यज॑मानं॒ ॅयज॑मान मेत्येति॒ यज॑मानम् । \newline
62. यज॑मान म॒भ्य॑भि यज॑मानं॒ ॅयज॑मान म॒भि । \newline
63. अ॒भ्यु दु द॒भ्य॑ भ्युत् । \newline
64. उथ् सृ॑जति सृज॒ त्युदुथ् सृ॑जति । \newline
65. सृ॒ज॒ति॒ क्षि॒प्रे क्षि॒प्रे सृ॑जति सृजति क्षि॒प्रे । \newline
66. क्षि॒प्रे स॒हस्रꣳ॑ स॒हस्र॑म् क्षि॒प्रे क्षि॒प्रे स॒हस्र᳚म् । \newline
67. स॒हस्र॒म् प्र प्र स॒हस्रꣳ॑ स॒हस्र॒म् प्र । \newline
68. प्र जा॑यते जायते॒ प्र प्र जा॑यते । \newline
69. जा॒य॒त॒ उ॒त्त॒ मोत्त॒मा जा॑यते जायत उत्त॒मा । \newline
70. उ॒त्त॒मा नी॒यते॑ नी॒यत॑ उत्त॒ मोत्त॒मा नी॒यते᳚ । \newline
71. उ॒त्त॒मेत्यु॑त् - त॒मा । \newline
72. नी॒यते᳚ प्रथ॒मा प्र॑थ॒मा नी॒यते॑ नी॒यते᳚ प्रथ॒मा । \newline
73. प्र॒थ॒मा दे॒वान् दे॒वान् प्र॑थ॒मा प्र॑थ॒मा दे॒वान् । \newline
74. दे॒वान् ग॑च्छति गच्छति दे॒वान् दे॒वान् ग॑च्छति । \newline
75. ग॒च्छ॒तीति॑ गच्छति । \newline

\textbf{Ghana Paata } \newline

1. इत्या॑हा॒हे तीत्या॑ह स्यो॒ना स्यो॒ना ऽऽहे तीत्या॑ह स्यो॒ना । \newline
2. आ॒ह॒ स्यो॒ना स्यो॒ना ऽऽहा॑ह स्यो॒नै वैव स्यो॒ना ऽऽहा॑ह स्यो॒नैव । \newline
3. स्यो॒नै वैव स्यो॒ना स्यो॒नै वैन॑ मेन मे॒व स्यो॒ना स्यो॒नै वैन᳚म् । \newline
4. ए॒वैन॑ मेन मे॒वै वैनꣳ॑ सु॒षदा॑ सु॒षदै॑न मे॒वै वैनꣳ॑ सु॒षदा᳚ । \newline
5. ए॒नꣳ॒॒ सु॒षदा॑ सु॒षदै॑न मेनꣳ सु॒षदा॑ सु॒शेवा॑ सु॒शेवा॑ सु॒षदै॑न मेनꣳ सु॒षदा॑ सु॒शेवा᳚ । \newline
6. सु॒षदा॑ सु॒शेवा॑ सु॒शेवा॑ सु॒षदा॑ सु॒षदा॑ सु॒शेवा॑ भू॒ता भू॒ता सु॒शेवा॑ सु॒षदा॑ सु॒षदा॑ सु॒शेवा॑ भू॒ता । \newline
7. सु॒षदेति॑ सु - सदा᳚ । \newline
8. सु॒शेवा॑ भू॒ता भू॒ता सु॒शेवा॑ सु॒शेवा॑ भू॒ता ऽऽवि॑शति विश॒त्या भू॒ता सु॒शेवा॑ सु॒शेवा॑ भू॒ता ऽऽवि॑शति । \newline
9. सु॒शेवेति॑ सु - शेवा᳚ । \newline
10. भू॒ता ऽऽवि॑शति विश॒त्या भू॒ता भू॒ता ऽऽवि॑शति॒ न न वि॑श॒त्या भू॒ता भू॒ता ऽऽवि॑शति॒ न । \newline
11. आ वि॑शति विश॒त्या वि॑शति॒ न न वि॑श॒त्या वि॑शति॒ न । \newline
12. वि॒श॒ति॒ न न वि॑शति विशति॒ नैन॑ मेन॒न् न वि॑शति विशति॒ नैन᳚म् । \newline
13. नैन॑ मेन॒न् न नैनꣳ॑ हिनस्ति हिन स्त्येन॒न् न नैनꣳ॑ हिनस्ति । \newline
14. ए॒नꣳ॒॒ हि॒न॒स्ति॒ हि॒न॒ स्त्ये॒न॒ मे॒नꣳ॒॒ हि॒न॒स्ति॒ ब्र॒ह्म॒वा॒दिनो᳚ ब्रह्मवा॒दिनो॑ हिनस्त्येन मेनꣳ हिनस्ति ब्रह्मवा॒दिनः॑ । \newline
15. हि॒न॒स्ति॒ ब्र॒ह्म॒वा॒दिनो᳚ ब्रह्मवा॒दिनो॑ हिनस्ति हिनस्ति ब्रह्मवा॒दिनो॑ वदन्ति वदन्ति ब्रह्मवा॒दिनो॑ हिनस्ति हिनस्ति ब्रह्मवा॒दिनो॑ वदन्ति । \newline
16. ब्र॒ह्म॒वा॒दिनो॑ वदन्ति वदन्ति ब्रह्मवा॒दिनो᳚ ब्रह्मवा॒दिनो॑ वदन्ति स॒हस्रꣳ॑ स॒हस्रं॑ ॅवदन्ति ब्रह्मवा॒दिनो᳚ ब्रह्मवा॒दिनो॑ वदन्ति स॒हस्र᳚म् । \newline
17. ब्र॒ह्म॒वा॒दिन॒ इति॑ ब्रह्म - वा॒दिनः॑ । \newline
18. व॒द॒न्ति॒ स॒हस्रꣳ॑ स॒हस्रं॑ ॅवदन्ति वदन्ति स॒हस्रꣳ॑ सहस्रत॒मी स॑हस्रत॒मी स॒हस्रं॑ ॅवदन्ति वदन्ति स॒हस्रꣳ॑ सहस्रत॒मी । \newline
19. स॒हस्रꣳ॑ सहस्रत॒मी स॑हस्रत॒मी स॒हस्रꣳ॑ स॒हस्रꣳ॑ सहस्रत॒ म्यन्वनु॑ सहस्रत॒मी स॒हस्रꣳ॑ स॒हस्रꣳ॑ सहस्रत॒ म्यनु॑ । \newline
20. स॒ह॒स्र॒त॒ म्यन्वनु॑ सहस्रत॒मी स॑हस्रत॒म्यन् वे॒ती(3) एती(3) अनु॑ सहस्रत॒मी स॑हस्रत॒म्यन् वे॒ती(3) । \newline
21. स॒ह॒स्र॒त॒मीति॑ सहस्र - त॒मी । \newline
22. अन्वे॒ती(3) एती(3) अन्वन् वे॒ती(3) स॑हस्रत॒मीꣳ स॑हस्रत॒मी मे॒ती(3) अन्वन् वे॒ती(3) स॑हस्रत॒मीम् । \newline
23. ए॒ती(3) स॑हस्रत॒मीꣳ स॑हस्रत॒मी मे॒ती(3) ए॒ती(3) स॑हस्रत॒मीꣳ स॒हस्रा(3)ꣳ स॒हस्रा(3)ꣳ स॑हस्रत॒मी मे॒ती(3) ए॒ती(3) स॑हस्रत॒मीꣳ स॒हस्रा(3)म् । \newline
24. स॒ह॒स्र॒त॒मीꣳ स॒हस्रा(3)ꣳ स॒हस्रा(3)ꣳ स॑हस्रत॒मीꣳ स॑हस्रत॒मीꣳ स॒हस्रा(3) मितीति॑ स॒हस्रा(3)ꣳ स॑हस्रत॒मीꣳ स॑हस्रत॒मीꣳ स॒हस्रा(3) मिति॑ । \newline
25. स॒ह॒स्र॒त॒मीमिति॑ सहस्र - त॒मीम् । \newline
26. स॒हस्रा(3) मितीति॑ स॒हस्रा(3)ꣳ स॒हस्रा(3) मिति॒ यद् यदिति॑ स॒हस्रा(3)ꣳ स॒हस्रा(3) मिति॒ यत् । \newline
27. इति॒ यद् यदितीति॒ यत् प्राची॒म् प्राचीं॒ ॅयदितीति॒ यत् प्राची᳚म् । \newline
28. यत् प्राची॒म् प्राचीं॒ ॅयद् यत् प्राची॑ मुथ्सृ॒जे दु॑थ्सृ॒जेत् प्राचीं॒ ॅयद् यत् प्राची॑ मुथ्सृ॒जेत् । \newline
29. प्राची॑ मुथ्सृ॒जे दु॑थ्सृ॒जेत् प्राची॒म् प्राची॑ मुथ्सृ॒जेथ् स॒हस्रꣳ॑ स॒हस्र॑ मुथ्सृ॒जेत् प्राची॒म् प्राची॑ मुथ्सृ॒जेथ् स॒हस्र᳚म् । \newline
30. उ॒थ्सृ॒जेथ् स॒हस्रꣳ॑ स॒हस्र॑ मुथ्सृ॒जे दु॑थ्सृ॒जेथ् स॒हस्रꣳ॑ सहस्रत॒मी स॑हस्रत॒मी स॒हस्र॑ मुथ्सृ॒जे दु॑थ्सृ॒जेथ् स॒हस्रꣳ॑ सहस्रत॒मी । \newline
31. उ॒थ्सृ॒जेदित्यु॑त् - सृ॒जेत् । \newline
32. स॒हस्रꣳ॑ सहस्रत॒मी स॑हस्रत॒मी स॒हस्रꣳ॑ स॒हस्रꣳ॑ सहस्रत॒ म्यन्वनु॑ सहस्रत॒मी स॒हस्रꣳ॑ स॒हस्रꣳ॑ सहस्रत॒ म्यनु॑ । \newline
33. स॒ह॒स्र॒त॒ म्यन्वनु॑ सहस्रत॒मी स॑हस्रत॒ म्यन्वि॑या दिया॒ दनु॑ सहस्रत॒मी स॑हस्रत॒ म्यन्वि॑यात् । \newline
34. स॒ह॒स्र॒त॒मीति॑ सहस्र - त॒मी । \newline
35. अन्वि॑या दिया॒ दन् वन् वि॑या॒त् तत् तदि॑या॒ दन् वन् वि॑या॒त् तत् । \newline
36. इ॒या॒त् तत् तदि॑या दिया॒त् तथ् स॒हस्रꣳ॑ स॒हस्र॒म् तदि॑या दिया॒त् तथ् स॒हस्र᳚म् । \newline
37. तथ् स॒हस्रꣳ॑ स॒हस्र॒म् तत् तथ् स॒हस्र॑ मप्रज्ञा॒त्र म॑प्रज्ञा॒त्रꣳ स॒हस्र॒म् तत् तथ् स॒हस्र॑ मप्रज्ञा॒त्रम् । \newline
38. स॒हस्र॑ मप्रज्ञा॒त्र म॑प्रज्ञा॒त्रꣳ स॒हस्रꣳ॑ स॒हस्र॑ मप्रज्ञा॒त्रꣳ सु॑व॒र्गꣳ सु॑व॒र्ग म॑प्रज्ञा॒त्रꣳ स॒हस्रꣳ॑ स॒हस्र॑ मप्रज्ञा॒त्रꣳ सु॑व॒र्गम् । \newline
39. अ॒प्र॒ज्ञा॒त्रꣳ सु॑व॒र्गꣳ सु॑व॒र्ग म॑प्रज्ञा॒त्र म॑प्रज्ञा॒त्रꣳ सु॑व॒र्गम् ॅलो॒कम् ॅलो॒कꣳ सु॑व॒र्ग म॑प्रज्ञा॒त्र म॑प्रज्ञा॒त्रꣳ सु॑व॒र्गम् ॅलो॒कम् । \newline
40. अ॒प्र॒ज्ञा॒त्रमित्य॑प्र - ज्ञा॒त्रम् । \newline
41. सु॒व॒र्गम् ॅलो॒कम् ॅलो॒कꣳ सु॑व॒र्गꣳ सु॑व॒र्गम् ॅलो॒कन् न न लो॒कꣳ सु॑व॒र्गꣳ सु॑व॒र्गम् ॅलो॒कन् न । \newline
42. सु॒व॒र्गमिति॑ सुवः - गम् । \newline
43. लो॒कन् न न लो॒कम् ॅलो॒कन् न प्र प्र ण लो॒कम् ॅलो॒कन् न प्र । \newline
44. न प्र प्र ण न प्र जा॑नीयाज् जानीया॒त् प्र ण न प्र जा॑नीयात् । \newline
45. प्र जा॑नीयाज् जानीया॒त् प्र प्र जा॑नीयात् प्र॒तीची᳚म् प्र॒तीची᳚म् जानीया॒त् प्र प्र जा॑नीयात् प्र॒तीची᳚म् । \newline
46. जा॒नी॒या॒त् प्र॒तीची᳚म् प्र॒तीची᳚म् जानीयाज् जानीयात् प्र॒तीची॒ मुदुत् प्र॒तीची᳚म् जानीयाज् जानीयात् प्र॒तीची॒ मुत् । \newline
47. प्र॒तीची॒ मुदुत् प्र॒तीची᳚म् प्र॒तीची॒ मुथ् सृ॑जति सृज॒त्युत् प्र॒तीची᳚म् प्र॒तीची॒ मुथ् सृ॑जति । \newline
48. उथ् सृ॑जति सृज॒ त्युदुथ् सृ॑जति॒ ताम् ताꣳ सृ॑ज॒ त्युदुथ् सृ॑जति॒ ताम् । \newline
49. सृ॒ज॒ति॒ ताम् ताꣳ सृ॑जति सृजति॒ ताꣳ स॒हस्रꣳ॑ स॒हस्र॒म् ताꣳ सृ॑जति सृजति॒ ताꣳ स॒हस्र᳚म् । \newline
50. ताꣳ स॒हस्रꣳ॑ स॒हस्र॒म् ताम् ताꣳ स॒हस्र॒ मन्वनु॑ स॒हस्र॒म् ताम् ताꣳ स॒हस्र॒ मनु॑ । \newline
51. स॒हस्र॒ मन्वनु॑ स॒हस्रꣳ॑ स॒हस्र॒ मनु॑ प॒र्याव॑र्तते प॒र्याव॑र्त॒ते ऽनु॑ स॒हस्रꣳ॑ स॒हस्र॒ मनु॑ प॒र्याव॑र्तते । \newline
52. अनु॑ प॒र्याव॑र्तते प॒र्याव॑र्त॒ते ऽन्वनु॑ प॒र्याव॑र्तते॒ सा सा प॒र्याव॑र्त॒ते ऽन्वनु॑ प॒र्याव॑र्तते॒ सा । \newline
53. प॒र्याव॑र्तते॒ सा सा प॒र्याव॑र्तते प॒र्याव॑र्तते॒ सा प्र॑जान॒ती प्र॑जान॒ती सा प॒र्याव॑र्तते प॒र्याव॑र्तते॒ सा प्र॑जान॒ती । \newline
54. प॒र्याव॑र्तत॒ इति॑ परि - आव॑र्तते । \newline
55. सा प्र॑जान॒ती प्र॑जान॒ती सा सा प्र॑जान॒ती सु॑व॒र्गꣳ सु॑व॒र्गम् प्र॑जान॒ती सा सा प्र॑जान॒ती सु॑व॒र्गम् । \newline
56. प्र॒जा॒न॒ती सु॑व॒र्गꣳ सु॑व॒र्गम् प्र॑जान॒ती प्र॑जान॒ती सु॑व॒र्गम् ॅलो॒कम् ॅलो॒कꣳ सु॑व॒र्गम् प्र॑जान॒ती प्र॑जान॒ती सु॑व॒र्गम् ॅलो॒कम् । \newline
57. प्र॒जा॒न॒तीति॑ प्र - जा॒न॒ती । \newline
58. सु॒व॒र्गम् ॅलो॒कम् ॅलो॒कꣳ सु॑व॒र्गꣳ सु॑व॒र्गम् ॅलो॒क मे᳚त्येति लो॒कꣳ सु॑व॒र्गꣳ सु॑व॒र्गम् ॅलो॒क मे॑ति । \newline
59. सु॒व॒र्गमिति॑ सुवः - गम् । \newline
60. लो॒क मे᳚त्येति लो॒कम् ॅलो॒क मे॑ति॒ यज॑मानं॒ ॅयज॑मान मेति लो॒कम् ॅलो॒क मे॑ति॒ यज॑मानम् । \newline
61. ए॒ति॒ यज॑मानं॒ ॅयज॑मान मेत्येति॒ यज॑मान म॒भ्य॑भि यज॑मान मेत्येति॒ यज॑मान म॒भि । \newline
62. यज॑मान म॒भ्य॑भि यज॑मानं॒ ॅयज॑मान म॒भ्यु दुद॒भि यज॑मानं॒ ॅयज॑मान म॒भ्युत् । \newline
63. अ॒भ्युदु द॒भ्य॑भ्युथ् सृ॑जति सृज॒त्यु द॒भ्य॑भ्युथ् सृ॑जति । \newline
64. उथ् सृ॑जति सृज॒ त्युदुथ् सृ॑जति क्षि॒प्रे क्षि॒प्रे सृ॑ज॒ त्युदुथ् सृ॑जति क्षि॒प्रे । \newline
65. सृ॒ज॒ति॒ क्षि॒प्रे क्षि॒प्रे सृ॑जति सृजति क्षि॒प्रे स॒हस्रꣳ॑ स॒हस्र॑म् क्षि॒प्रे सृ॑जति सृजति क्षि॒प्रे स॒हस्र᳚म् । \newline
66. क्षि॒प्रे स॒हस्रꣳ॑ स॒हस्र॑म् क्षि॒प्रे क्षि॒प्रे स॒हस्र॒म् प्र प्र स॒हस्र॑म् क्षि॒प्रे क्षि॒प्रे स॒हस्र॒म् प्र । \newline
67. स॒हस्र॒म् प्र प्र स॒हस्रꣳ॑ स॒हस्र॒म् प्र जा॑यते जायते॒ प्र स॒हस्रꣳ॑ स॒हस्र॒म् प्र जा॑यते । \newline
68. प्र जा॑यते जायते॒ प्र प्र जा॑यत उत्त॒मो त्त॒मा जा॑यते॒ प्र प्र जा॑यत उत्त॒मा । \newline
69. जा॒य॒त॒ उ॒त्त॒मो त्त॒मा जा॑यते जायत उत्त॒मा नी॒यते॑ नी॒यत॑ उत्त॒मा जा॑यते जायत उत्त॒मा नी॒यते᳚ । \newline
70. उ॒त्त॒मा नी॒यते॑ नी॒यत॑ उत्त॒मो त्त॒मा नी॒यते᳚ प्रथ॒मा प्र॑थ॒मा नी॒यत॑ उत्त॒मो त्त॒मा नी॒यते᳚ प्रथ॒मा । \newline
71. उ॒त्त॒मेत्यु॑त् - त॒मा । \newline
72. नी॒यते᳚ प्रथ॒मा प्र॑थ॒मा नी॒यते॑ नी॒यते᳚ प्रथ॒मा दे॒वान् दे॒वान् प्र॑थ॒मा नी॒यते॑ नी॒यते᳚ प्रथ॒मा दे॒वान् । \newline
73. प्र॒थ॒मा दे॒वान् दे॒वान् प्र॑थ॒मा प्र॑थ॒मा दे॒वान् ग॑च्छति गच्छति दे॒वान् प्र॑थ॒मा प्र॑थ॒मा दे॒वान् ग॑च्छति । \newline
74. दे॒वान् ग॑च्छति गच्छति दे॒वान् दे॒वान् ग॑च्छति । \newline
75. ग॒च्छ॒तीति॑ गच्छति । \newline
\pagebreak
\markright{ TS 7.1.8.1  \hfill https://www.vedavms.in \hfill}

\section{ TS 7.1.8.1 }

\textbf{TS 7.1.8.1 } \newline
\textbf{Samhita Paata} \newline

अत्रि॑रददा॒दौर्वा॑य प्र॒जां पु॒त्रका॑माय॒ स रि॑रिचा॒नो॑ऽमन्यत॒ निर्वी᳚र्यः शिथि॒लो या॒तया॑मा॒ स ए॒तं च॑तूरा॒त्र-म॑पश्य॒त् तमाऽह॑र॒त् तेना॑यजत॒ ततो॒ वै तस्य॑ च॒त्वारो॑ वी॒रा आऽजा॑यन्त॒ सुहो॑ता॒ सू᳚द्गाता॒ स्व॑द्ध्वर्युः॒ सुस॑भेयो॒ य ए॒वं ॅवि॒द्वाꣳश्च॑तूरा॒त्रेण॒ यज॑त॒ आऽस्य॑ च॒त्वारो॑ वी॒रा जा॑यन्ते॒ सुहो॑ता॒ सू᳚द्गाता॒ स्व॑द्ध्वर्युः॒ सुस॑भेयो॒ ये च॑तुर्विꣳ॒॒शाः पव॑माना ब्रह्मवर्च॒सं त - [  ] \newline

\textbf{Pada Paata} \newline

अत्रिः॑ । अ॒द॒दा॒त् । और्वा॑य । प्र॒जामिति॑ प्र - जाम् । पु॒त्रका॑मा॒येति॑ पु॒त्र - का॒मा॒य॒ । सः । रि॒रि॒चा॒नः । अ॒म॒न्य॒त॒ । निर्वी᳚र्य॒ इति॒ निः - वी॒र्यः॒ । शि॒थि॒लः । या॒तया॒मेति॑ या॒त - या॒मा॒ । सः । ए॒तम् । च॒तू॒रा॒त्रमिति॑ चतुः - रा॒त्रम् । अ॒प॒श्य॒त् । तम् । एति॑ । अ॒ह॒र॒त् । तेन॑ । अ॒य॒ज॒त॒ । ततः॑ । वै । तस्य॑ । च॒त्वारः॑ । वी॒राः । एति॑ । अ॒जा॒य॒न्त॒ । सुहो॒तेति॒ सु - हो॒ता॒ । सू᳚द्गा॒तेति॒ सु - उ॒द्गा॒ता॒ । स्व॑द्ध्वर्यु॒रिति॒ सु - अ॒द्ध्व॒र्युः॒ । सुस॑भेय॒ इति॒ सु - स॒भे॒यः॒ । यः । ए॒वम् । वि॒द्वान् । च॒तू॒रा॒त्रेणेति॑ चतुः - रा॒त्रेण॑ । यज॑ते । एति॑ । अ॒स्य॒ । च॒त्वारः॑ । वी॒राः । जा॒य॒न्ते॒ । सुहो॒तेति॒ सु - हो॒ता॒ । सू᳚द्गा॒तेति॒ सु - उ॒द्गा॒ता॒ । स्व॑द्ध्वर्यु॒रिति॒ सु - अ॒द्ध्व॒र्युः॒ । सुस॑भेय॒ इति॒ सु - स॒भे॒यः॒ । ये । च॒तु॒र्विꣳ॒॒शा इति॑ चतुः - विꣳ॒॒शाः । पव॑मानाः । ब्र॒ह्म॒व॒र्च॒समिति॑ ब्रह्म - व॒र्च॒सम् । तत् ।  \newline


\textbf{Krama Paata} \newline

अत्रि॑रददात् । अ॒द॒दा॒दौर्वा॑य । और्वा॑य प्र॒जाम् । प्र॒जाम् पु॒त्रका॑माय । प्र॒जामिति॑ प्र - जाम् । पु॒त्रका॑माय॒ सः । पु॒त्रका॑मा॒येति॑ पु॒त्र - का॒मा॒य॒ । स रि॑रिचा॒नः । रि॒रि॒चा॒नो॑ऽमन्यत । अ॒म॒न्य॒त॒ निर्वी᳚र्यः । निर्वी᳚र्यः शिथि॒लः । निर्वी᳚र्य॒ इति॒ निः - वी॒र्यः॒ । शि॒थि॒लो या॒तया॑मा । या॒तया॑मा॒ सः । या॒तया॒मेति॑ या॒त - या॒मा॒ । स ए॒तम् । ए॒तम् च॑तूरा॒त्रम् । च॒तू॒रा॒त्रम॑पश्यत् । च॒तू॒रा॒त्रमिति॑ चतुः - रा॒त्रम् । अ॒प॒श्य॒त् तम् । तमा । आऽह॑रत् । अ॒ह॒र॒त् तेन॑ । तेना॑यजत । अ॒य॒ज॒त॒ ततः॑ । ततो॒ वै । वै तस्य॑ । तस्य॑ च॒त्वारः॑ । च॒त्वारो॑ वी॒राः । वी॒रा आ । आऽजा॑यन्त । अ॒जा॒य॒न्त॒ सुहो॑ता । सुहो॑ता॒ सू᳚द्‍गाता । सुहो॒तेति॒ सु - हो॒ता॒ । सू᳚द्‍गाता॒ स्व॑द्ध्वर्युः । सू᳚द्‍गा॒तेति॒ सु - उ॒द्‍गा॒ता॒ । स्व॑द्ध्वर्युः॒ सुस॑भेयः । स्व॑द्ध्वर्यु॒रिति॒ सु - अ॒द्ध्व॒र्युः॒ । सुस॑भेयो॒ यः । सुस॑भेय॒ इति॒ सु - स॒भे॒यः॒ । य ए॒वम् । ए॒वम् ॅवि॒द्वान् । वि॒द्वाꣳश्च॑तूरा॒त्रेण॑ । च॒तू॒रा॒त्रेण॒ यज॑ते । च॒तू॒रा॒त्रेणेति॑ चतुः - रा॒त्रेण॑ । यज॑त॒ आ । आऽस्य॑ । अ॒स्य॒ च॒त्वारः॑ । च॒त्वारो॑ वी॒राः । वी॒रा जा॑यन्ते । जा॒य॒न्ते॒ सुहो॑ता । सुहो॑ता॒ सू᳚द्‍गाता । सुहो॒तेति॒ सु - हो॒ता॒ । सू᳚द्‍गाता॒ स्व॑द्ध्वर्युः । सू᳚द्‍गा॒तेति॒ सु - उ॒द्‍गा॒ता॒ । स्व॑द्ध्वर्युः॒ सुस॑भेयः । स्व॑द्ध्वर्यु॒रिति॒ सु - अ॒द्ध्व॒र्युः॒ । सुस॑भेयो॒ ये । सुस॑भेय॒ इति॒ सु - स॒भे॒यः॒ । ये च॑तुर्विꣳ॒॒शाः । च॒तु॒र्विꣳ॒॒शाः पव॑मानाः । च॒तु॒र्विꣳ॒॒शा इति॑ चतुः - विꣳ॒॒शाः । पव॑माना ब्रह्मवर्च॒सम् । ब्र॒ह्म॒व॒र्च॒सम् तत् । ब्र॒ह्म॒व॒र्च॒समिति॑ ब्रह्म - व॒र्च॒सम् । तद् ये \newline

\textbf{Jatai Paata} \newline

1. अत्रि॑ रददा दददा॒ दत्रि॒ रत्रि॑ रददात् । \newline
2. अ॒द॒दा॒ दौर्वा॒ यौर्वा॑या ददाद ददा॒ दौर्वा॑य । \newline
3. और्वा॑य प्र॒जाम् प्र॒जा मौर्वा॒ यौर्वा॑य प्र॒जाम् । \newline
4. प्र॒जाम् पु॒त्रका॑माय पु॒त्रका॑माय प्र॒जाम् प्र॒जाम् पु॒त्रका॑माय । \newline
5. प्र॒जामिति॑ प्र - जाम् । \newline
6. पु॒त्रका॑माय॒ स स पु॒त्रका॑माय पु॒त्रका॑माय॒ सः । \newline
7. पु॒त्रका॑मा॒येति॑ पु॒त्र - का॒मा॒य॒ । \newline
8. स रि॑रिचा॒नो रि॑रिचा॒नः स स रि॑रिचा॒नः । \newline
9. रि॒रि॒चा॒नो॑ ऽमन्यता मन्यत रिरिचा॒नो रि॑रिचा॒नो॑ ऽमन्यत । \newline
10. अ॒म॒न्य॒त॒ निर्वी᳚र्यो॒ निर्वी᳚र्यो ऽमन्यता मन्यत॒ निर्वी᳚र्यः । \newline
11. निर्वी᳚र्यः शिथि॒लः शि॑थि॒लो निर्वी᳚र्यो॒ निर्वी᳚र्यः शिथि॒लः । \newline
12. निर्वी᳚र्य॒ इति॒ निः - वी॒र्यः॒ । \newline
13. शि॒थि॒लो या॒तया॑मा या॒तया॑मा शिथि॒लः शि॑थि॒लो या॒तया॑मा । \newline
14. या॒तया॑मा॒ स स या॒तया॑मा या॒तया॑मा॒ सः । \newline
15. या॒तया॒मेति॑ या॒त - या॒मा॒ । \newline
16. स ए॒त मे॒तꣳ स स ए॒तम् । \newline
17. ए॒तम् च॑तूरा॒त्रम् च॑तूरा॒त्र मे॒त मे॒तम् च॑तूरा॒त्रम् । \newline
18. च॒तू॒रा॒त्र म॑पश्य दपश्यच् चतूरा॒त्रम् च॑तूरा॒त्र म॑पश्यत् । \newline
19. च॒तू॒रा॒त्रमिति॑ चतुः - रा॒त्रम् । \newline
20. अ॒प॒श्य॒त् तम् त म॑पश्य दपश्य॒त् तम् । \newline
21. त मा तम् त मा । \newline
22. आ ऽह॑र दहर॒दा ऽह॑रत् । \newline
23. अ॒ह॒र॒त् तेन॒ तेना॑ हर दहर॒त् तेन॑ । \newline
24. तेना॑ यजता यजत॒ तेन॒ तेना॑ यजत । \newline
25. अ॒य॒ज॒त॒ तत॒ स्ततो॑ ऽयजता यजत॒ ततः॑ । \newline
26. ततो॒ वै वै तत॒ स्ततो॒ वै । \newline
27. वै तस्य॒ तस्य॒ वै वै तस्य॑ । \newline
28. तस्य॑ च॒त्वार॑ श्च॒त्वार॒ स्तस्य॒ तस्य॑ च॒त्वारः॑ । \newline
29. च॒त्वारो॑ वी॒रा वी॒रा श्च॒त्वार॑ श्च॒त्वारो॑ वी॒राः । \newline
30. वी॒रा आ वी॒रा वी॒रा आ । \newline
31. आ ऽजा॑यन्ता जाय॒न्ता ऽजा॑यन्त । \newline
32. अ॒जा॒य॒न्त॒ सुहो॑ता॒ सुहो॑ता ऽजायन्ता जायन्त॒ सुहो॑ता । \newline
33. सुहो॑ता॒ सू᳚द्‍गाता॒ सू᳚द्‍गाता॒ सुहो॑ता॒ सुहो॑ता॒ सू᳚द्‍गाता । \newline
34. सुहो॒तेति॒ सु - हो॒ता॒ । \newline
35. सू᳚द्‍गाता॒ स्व॑द्ध्वर्युः॒ स्व॑द्ध्वर्युः॒ सू᳚द्‍गाता॒ सू᳚द्‍गाता॒ स्व॑द्ध्वर्युः । \newline
36. सू᳚द्‍गा॒तेति॒ सु - उ॒द्‍गा॒ता॒ । \newline
37. स्व॑द्ध्वर्युः॒ सुस॑भेयः॒ सुस॑भेयः॒ स्व॑द्ध्वर्युः॒ स्व॑द्ध्वर्युः॒ सुस॑भेयः । \newline
38. स्व॑द्ध्वर्यु॒रिति॒ सु - अ॒द्ध्व॒र्युः॒ । \newline
39. सुस॑भेयो॒ यो यः सुस॑भेयः॒ सुस॑भेयो॒ यः । \newline
40. सुस॑भेय॒ इति॒ सु - स॒भे॒यः॒ । \newline
41. य ए॒व मे॒वं ॅयो य ए॒वम् । \newline
42. ए॒वं ॅवि॒द्वान्. वि॒द्वा ने॒व मे॒वं ॅवि॒द्वान् । \newline
43. वि॒द्वाꣳ श्च॑तूरा॒त्रेण॑ चतूरा॒त्रेण॑ वि॒द्वान्. वि॒द्वाꣳ श्च॑तूरा॒त्रेण॑ । \newline
44. च॒तू॒रा॒त्रेण॒ यज॑ते॒ यज॑ते चतूरा॒त्रेण॑ चतूरा॒त्रेण॒ यज॑ते । \newline
45. च॒तू॒रा॒त्रेणेति॑ चतुः - रा॒त्रेण॑ । \newline
46. यज॑त॒ आ यज॑ते॒ यज॑त॒ आ । \newline
47. आ ऽस्या॒स्या ऽस्य॑ । \newline
48. अ॒स्य॒ च॒त्वार॑ श्च॒त्वारो᳚ ऽस्यास्य च॒त्वारः॑ । \newline
49. च॒त्वारो॑ वी॒रा वी॒रा श्च॒त्वार॑ श्च॒त्वारो॑ वी॒राः । \newline
50. वी॒रा जा॑यन्ते जायन्ते वी॒रा वी॒रा जा॑यन्ते । \newline
51. जा॒य॒न्ते॒ सुहो॑ता॒ सुहो॑ता जायन्ते जायन्ते॒ सुहो॑ता । \newline
52. सुहो॑ता॒ सू᳚द्‍गाता॒ सू᳚द्‍गाता॒ सुहो॑ता॒ सुहो॑ता॒ सू᳚द्‍गाता । \newline
53. सुहो॒तेति॒ सु - हो॒ता॒ । \newline
54. सू᳚द्‍गाता॒ स्व॑द्ध्वर्युः॒ स्व॑द्ध्वर्युः॒ सू᳚द्‍गाता॒ सू᳚द्‍गाता॒ स्व॑द्ध्वर्युः । \newline
55. सू᳚द्‍गा॒तेति॒ सु - उ॒द्‍गा॒ता॒ । \newline
56. स्व॑द्ध्वर्युः॒ सुस॑भेयः॒ सुस॑भेयः॒ स्व॑द्ध्वर्युः॒ स्व॑द्ध्वर्युः॒ सुस॑भेयः । \newline
57. स्व॑द्ध्वर्यु॒रिति॒ सु - अ॒द्ध्व॒र्युः॒ । \newline
58. सुस॑भेयो॒ ये ये सुस॑भेयः॒ सुस॑भेयो॒ ये । \newline
59. सुस॑भेय॒ इति॒ सु - स॒भे॒यः॒ । \newline
60. ये च॑तुर्विꣳ॒॒शा श्च॑तुर्विꣳ॒॒शा ये ये च॑तुर्विꣳ॒॒शाः । \newline
61. च॒तु॒र्विꣳ॒॒शाः पव॑मानाः॒ पव॑माना श्चतुर्विꣳ॒॒शा श्च॑तुर्विꣳ॒॒शाः पव॑मानाः । \newline
62. च॒तु॒र्विꣳ॒॒शा इति॑ चतुः - विꣳ॒॒शाः । \newline
63. पव॑माना ब्रह्मवर्च॒सम् ब्र॑ह्मवर्च॒सम् पव॑मानाः॒ पव॑माना ब्रह्मवर्च॒सम् । \newline
64. ब्र॒ह्म॒व॒र्च॒सम् तत् तद् ब्र॑ह्मवर्च॒सम् ब्र॑ह्मवर्च॒सम् तत् । \newline
65. ब्र॒ह्म॒व॒र्च॒समिति॑ ब्रह्म - व॒र्च॒सम् । \newline
66. तद् ये ये तत् तद् ये । \newline

\textbf{Ghana Paata } \newline

1. अत्रि॑ रददा दददा॒ दत्रि॒ रत्रि॑ रददा॒ दौर्वा॒ यौर्वा॑या ददा॒ दत्रि॒ रत्रि॑ रददा॒ दौर्वा॑य । \newline
2. अ॒द॒दा॒ दौर्वा॒ यौर्वा॑या ददा दददा॒ दौर्वा॑य प्र॒जाम् प्र॒जा मौर्वा॑या ददा दददा॒ दौर्वा॑य प्र॒जाम् । \newline
3. और्वा॑य प्र॒जाम् प्र॒जा मौर्वा॒ यौर्वा॑य प्र॒जाम् पु॒त्रका॑माय पु॒त्रका॑माय प्र॒जा मौर्वा॒ यौर्वा॑य प्र॒जाम् पु॒त्रका॑माय । \newline
4. प्र॒जाम् पु॒त्रका॑माय पु॒त्रका॑माय प्र॒जाम् प्र॒जाम् पु॒त्रका॑माय॒ स स पु॒त्रका॑माय प्र॒जाम् प्र॒जाम् पु॒त्रका॑माय॒ सः । \newline
5. प्र॒जामिति॑ प्र - जाम् । \newline
6. पु॒त्रका॑माय॒ स स पु॒त्रका॑माय पु॒त्रका॑माय॒ स रि॑रिचा॒नो रि॑रिचा॒नः स पु॒त्रका॑माय पु॒त्रका॑माय॒ स रि॑रिचा॒नः । \newline
7. पु॒त्रका॑मा॒येति॑ पु॒त्र - का॒मा॒य॒ । \newline
8. स रि॑रिचा॒नो रि॑रिचा॒नः स स रि॑रिचा॒नो॑ ऽमन्यता मन्यत रिरिचा॒नः स स रि॑रिचा॒नो॑ ऽमन्यत । \newline
9. रि॒रि॒चा॒नो॑ ऽमन्यता मन्यत रिरिचा॒नो रि॑रिचा॒नो॑ ऽमन्यत॒ निर्वी᳚र्यो॒ निर्वी᳚र्यो ऽमन्यत रिरिचा॒नो रि॑रिचा॒नो॑ ऽमन्यत॒ निर्वी᳚र्यः । \newline
10. अ॒म॒न्य॒त॒ निर्वी᳚र्यो॒ निर्वी᳚र्यो ऽमन्यता मन्यत॒ निर्वी᳚र्यः शिथि॒लः शि॑थि॒लो निर्वी᳚र्यो ऽमन्यता मन्यत॒ निर्वी᳚र्यः शिथि॒लः । \newline
11. निर्वी᳚र्यः शिथि॒लः शि॑थि॒लो निर्वी᳚र्यो॒ निर्वी᳚र्यः शिथि॒लो या॒तया॑मा या॒तया॑मा शिथि॒लो निर्वी᳚र्यो॒ निर्वी᳚र्यः शिथि॒लो या॒तया॑मा । \newline
12. निर्वी᳚र्य॒ इति॒ निः - वी॒र्यः॒ । \newline
13. शि॒थि॒लो या॒तया॑मा या॒तया॑मा शिथि॒लः शि॑थि॒लो या॒तया॑मा॒ स स या॒तया॑मा शिथि॒लः शि॑थि॒लो या॒तया॑मा॒ सः । \newline
14. या॒तया॑मा॒ स स या॒तया॑मा या॒तया॑मा॒ स ए॒त मे॒तꣳ स या॒तया॑मा या॒तया॑मा॒ स ए॒तम् । \newline
15. या॒तया॒मेति॑ या॒त - या॒मा॒ । \newline
16. स ए॒त मे॒तꣳ स स ए॒तम् च॑तूरा॒त्रम् च॑तूरा॒त्र मे॒तꣳ स स ए॒तम् च॑तूरा॒त्रम् । \newline
17. ए॒तम् च॑तूरा॒त्रम् च॑तूरा॒त्र मे॒त मे॒तम् च॑तूरा॒त्र म॑पश्य दपश्यच् चतूरा॒त्र मे॒त मे॒तम् च॑तूरा॒त्र म॑पश्यत् । \newline
18. च॒तू॒रा॒त्र म॑पश्य दपश्यच् चतूरा॒त्रम् च॑तूरा॒त्र म॑पश्य॒त् तम् त म॑पश्यच् चतूरा॒त्रम् च॑तूरा॒त्र म॑पश्य॒त् तम् । \newline
19. च॒तू॒रा॒त्रमिति॑ चतुः - रा॒त्रम् । \newline
20. अ॒प॒श्य॒त् तम् त म॑पश्य दपश्य॒त् त मा त म॑पश्य दपश्य॒त् त मा । \newline
21. त मा तम् त मा ऽह॑र दहर॒दा तम् त मा ऽह॑रत् । \newline
22. आ ऽह॑र दहर॒दा ऽह॑र॒त् तेन॒ तेना॑ हर॒दा ऽह॑र॒त् तेन॑ । \newline
23. अ॒ह॒र॒त् तेन॒ तेना॑ हर दहर॒त् तेना॑ यजता यजत॒ तेना॑ हर दहर॒त् तेना॑ यजत । \newline
24. तेना॑ यजता यजत॒ तेन॒ तेना॑ यजत॒ तत॒ स्ततो॑ ऽयजत॒ तेन॒ तेना॑ यजत॒ ततः॑ । \newline
25. अ॒य॒ज॒त॒ तत॒ स्ततो॑ ऽयजता यजत॒ ततो॒ वै वै ततो॑ ऽयजता यजत॒ ततो॒ वै । \newline
26. ततो॒ वै वै तत॒ स्ततो॒ वै तस्य॒ तस्य॒ वै तत॒ स्ततो॒ वै तस्य॑ । \newline
27. वै तस्य॒ तस्य॒ वै वै तस्य॑ च॒त्वार॑ श्च॒त्वार॒ स्तस्य॒ वै वै तस्य॑ च॒त्वारः॑ । \newline
28. तस्य॑ च॒त्वार॑ श्च॒त्वार॒ स्तस्य॒ तस्य॑ च॒त्वारो॑ वी॒रा वी॒रा श्च॒त्वार॒ स्तस्य॒ तस्य॑ च॒त्वारो॑ वी॒राः । \newline
29. च॒त्वारो॑ वी॒रा वी॒रा श्च॒त्वार॑ श्च॒त्वारो॑ वी॒रा आ वी॒रा श्च॒त्वार॑ श्च॒त्वारो॑ वी॒रा आ । \newline
30. वी॒रा आ वी॒रा वी॒रा आ ऽजा॑यन्ता जाय॒न्ता वी॒रा वी॒रा आ ऽजा॑यन्त । \newline
31. आ ऽजा॑यन्ता जाय॒न्ता ऽजा॑यन्त॒ सुहो॑ता॒ सुहो॑ता ऽजाय॒न्ता ऽजा॑यन्त॒ सुहो॑ता । \newline
32. अ॒जा॒य॒न्त॒ सुहो॑ता॒ सुहो॑ता ऽजायन्ता जायन्त॒ सुहो॑ता॒ सू᳚द्‍गाता॒ सू᳚द्‍गाता॒ सुहो॑ता ऽजायन्ता जायन्त॒ सुहोता॒ 
सू᳚द्‍गाता । \newline
33. सुहो॑ता॒ सू᳚द्‍गाता॒ सू᳚द्‍गाता॒ सुहो॑ता॒ सुहो॑ता॒ सू᳚द्‍गाता॒ स्व॑द्ध्वर्युः॒ स्व॑द्ध्वर्युः॒ सू᳚द्‍गाता॒ सुहो॑ता॒ सुहो॑ता॒ सू᳚द्‍गाता॒ स्व॑द्ध्वर्युः । \newline
34. सुहो॒तेति॒ सु - हो॒ता॒ । \newline
35. सू᳚द्‍गाता॒ स्व॑द्ध्वर्युः॒ स्व॑द्ध्वर्युः॒ सू᳚द्‍गाता॒ सू᳚द्‍गाता॒ स्व॑द्ध्वर्युः॒ सुस॑भेयः॒ सुस॑भेयः॒ स्व॑द्ध्वर्युः॒ सू᳚द्‍गाता॒ सू᳚द्‍गाता॒ स्व॑द्ध्वर्युः॒ सुस॑भेयः । \newline
36. सू᳚द्‍गा॒तेति॒ सु - उ॒द्‍गा॒ता॒ । \newline
37. स्व॑द्ध्वर्युः॒ सुस॑भेयः॒ सुस॑भेयः॒ स्व॑द्ध्वर्युः॒ स्व॑द्ध्वर्युः॒ सुस॑भेयो॒ यो यः सुस॑भेयः॒ स्व॑द्ध्वर्युः॒ स्व॑द्ध्वर्युः॒ सुस॑भेयो॒ यः । \newline
38. स्व॑द्ध्वर्यु॒रिति॒ सु - अ॒द्ध्व॒र्युः॒ । \newline
39. सुस॑भेयो॒ यो यः सुस॑भेयः॒ सुस॑भेयो॒ य ए॒व मे॒वं ॅयः सुस॑भेयः॒ सुस॑भेयो॒ य ए॒वम् । \newline
40. सुस॑भेय॒ इति॒ सु - स॒भे॒यः॒ । \newline
41. य ए॒व मे॒वं ॅयो य ए॒वं ॅवि॒द्वान्. वि॒द्वा ने॒वं ॅयो य ए॒वं ॅवि॒द्वान् । \newline
42. ए॒वं ॅवि॒द्वान्. वि॒द्वा ने॒व मे॒वं ॅवि॒द्वाꣳ श्च॑तूरा॒त्रेण॑ चतूरा॒त्रेण॑ वि॒द्वा ने॒व मे॒वं ॅवि॒द्वाꣳश्च॑तूरा॒त्रेण॑ । \newline
43. वि॒द्वाꣳ श्च॑तूरा॒त्रेण॑ चतूरा॒त्रेण॑ वि॒द्वान्. वि॒द्वाꣳ श्च॑तूरा॒त्रेण॒ यज॑ते॒ यज॑ते चतूरा॒त्रेण॑ वि॒द्वान्. वि॒द्वाꣳ श्च॑तूरा॒त्रेण॒ यज॑ते । \newline
44. च॒तू॒रा॒त्रेण॒ यज॑ते॒ यज॑ते चतूरा॒त्रेण॑ चतूरा॒त्रेण॒ यज॑त॒ आ यज॑ते चतूरा॒त्रेण॑ चतूरा॒त्रेण॒ यज॑त॒ आ । \newline
45. च॒तू॒रा॒त्रेणेति॑ चतुः - रा॒त्रेण॑ । \newline
46. यज॑त॒ आ यज॑ते॒ यज॑त॒ आ ऽस्या॒स्या यज॑ते॒ यज॑त॒ आ ऽस्य॑ । \newline
47. आ ऽस्या॒स्या ऽस्य॑ च॒त्वार॑ श्च॒त्वारो॒ ऽस्या ऽस्य॑ च॒त्वारः॑ । \newline
48. अ॒स्य॒ च॒त्वार॑ श्च॒त्वारो᳚ ऽस्यास्य च॒त्वारो॑ वी॒रा वी॒रा श्च॒त्वारो᳚ ऽस्यास्य च॒त्वारो॑ वी॒राः । \newline
49. च॒त्वारो॑ वी॒रा वी॒रा श्च॒त्वार॑ श्च॒त्वारो॑ वी॒रा जा॑यन्ते जायन्ते वी॒रा श्च॒त्वार॑ श्च॒त्वारो॑ वी॒रा जा॑यन्ते । \newline
50. वी॒रा जा॑यन्ते जायन्ते वी॒रा वी॒रा जा॑यन्ते॒ सुहो॑ता॒ सुहो॑ता जायन्ते वी॒रा वी॒रा जा॑यन्ते॒ सुहो॑ता । \newline
51. जा॒य॒न्ते॒ सुहो॑ता॒ सुहो॑ता जायन्ते जायन्ते॒ सुहोता॒ सू᳚द्‍गाता॒ सू᳚द्‍गाता॒ सुहो॑ता जायन्ते जायन्ते॒ सुहोता॒ सू᳚द्‍गाता । \newline
52. सुहो॑ता॒ सू᳚द्‍गाता॒ सू᳚द्‍गाता॒ सुहो॑ता॒ सुहो॑ता॒ सू᳚द्‍गाता॒ स्व॑द्ध्वर्युः॒ स्व॑द्ध्वर्युः॒ सू᳚द्‍गाता॒ सुहो॑ता॒ सुहो॑ता॒ सू᳚द्‍गाता॒ स्व॑द्ध्वर्युः । \newline
53. सुहो॒तेति॒ सु - हो॒ता॒ । \newline
54. सू᳚द्‍गाता॒ स्व॑द्ध्वर्युः॒ स्व॑द्ध्वर्युः॒ सू᳚द्‍गाता॒ सू᳚द्‍गाता॒ स्व॑द्ध्वर्युः॒ सुस॑भेयः॒ सुस॑भेयः॒ स्व॑द्ध्वर्युः॒ सू᳚द्‍गाता॒ सू᳚द्‍गाता॒ स्व॑द्ध्वर्युः॒ सुस॑भेयः । \newline
55. सू᳚द्‍गा॒तेति॒ सु - उ॒द्‍गा॒ता॒ । \newline
56. स्व॑द्ध्वर्युः॒ सुस॑भेयः॒ सुस॑भेयः॒ स्व॑द्ध्वर्युः॒ स्व॑द्ध्वर्युः॒ सुस॑भेयो॒ ये ये सुस॑भेयः॒ स्व॑द्ध्वर्युः॒ स्व॑द्ध्वर्युः॒ सुस॑भेयो॒ ये । \newline
57. स्व॑द्ध्वर्यु॒रिति॒ सु - अ॒द्ध्व॒र्युः॒ । \newline
58. सुस॑भेयो॒ ये ये सुस॑भेयः॒ सुस॑भेयो॒ ये च॑तुर्विꣳ॒॒शा श्च॑तुर्विꣳ॒॒शा ये सुस॑भेयः॒ सुस॑भेयो॒ ये च॑तुर्विꣳ॒॒शाः । \newline
59. सुस॑भेय॒ इति॒ सु - स॒भे॒यः॒ । \newline
60. ये च॑तुर्विꣳ॒॒शा श्च॑तुर्विꣳ॒॒शा ये ये च॑तुर्विꣳ॒॒शाः पव॑मानाः॒ पव॑माना श्चतुर्विꣳ॒॒शा ये ये च॑तुर्विꣳ॒॒शाः पव॑मानाः । \newline
61. च॒तु॒र्विꣳ॒॒शाः पव॑मानाः॒ पव॑माना श्चतुर्विꣳ॒॒शा श्च॑तुर्विꣳ॒॒शाः पव॑माना ब्रह्मवर्च॒सम् ब्र॑ह्मवर्च॒सम् पव॑माना श्चतुर्विꣳ॒॒शा श्च॑तुर्विꣳ॒॒शाः पव॑माना ब्रह्मवर्च॒सम् । \newline
62. च॒तु॒र्विꣳ॒॒शा इति॑ चतुः - विꣳ॒॒शाः । \newline
63. पव॑माना ब्रह्मवर्च॒सम् ब्र॑ह्मवर्च॒सम् पव॑मानाः॒ पव॑माना ब्रह्मवर्च॒सम् तत् तद् ब्र॑ह्मवर्च॒सम् पव॑मानाः॒ पव॑माना ब्रह्मवर्च॒सम् तत् । \newline
64. ब्र॒ह्म॒व॒र्च॒सम् तत् तद् ब्र॑ह्मवर्च॒सम् ब्र॑ह्मवर्च॒सम् तद् ये ये तद् ब्र॑ह्मवर्च॒सम् ब्र॑ह्मवर्च॒सम् तद् ये । \newline
65. ब्र॒ह्म॒व॒र्च॒समिति॑ ब्रह्म - व॒र्च॒सम् । \newline
66. तद् ये ये तत् तद् य उ॒द्यन्त॑ उ॒द्यन्तो॒ ये तत् तद् य उ॒द्यन्तः॑ । \newline
\pagebreak
\markright{ TS 7.1.8.2  \hfill https://www.vedavms.in \hfill}

\section{ TS 7.1.8.2 }

\textbf{TS 7.1.8.2 } \newline
\textbf{Samhita Paata} \newline

- द्य उ॒द्यन्तः॒ स्तोमाः॒ श्रीः सा ऽत्रिꣳ॑ श्र॒द्धादे॑वं॒ ॅयज॑मानं च॒त्वारि॑ वी॒र्या॑णि॒ नोपा॑ऽनम॒न् तेज॑ इन्द्रि॒यं ब्र॑ह्मवर्च॒स-म॒न्नाद्यꣳ॒॒ स ए॒ताꣳश्च॒तुर॒श्चतु॑ष्टोमा॒न्थ् सोमा॑न-पश्य॒त् तानाऽह॑र॒त् तैर॑यजत॒ तेज॑ ए॒व प्र॑थ॒मेना ऽवा॑रुन्धेन्द्रि॒यं द्वि॒तीये॑न ब्रह्मवर्च॒सं तृ॒तीये॑ना॒न्नाद्यं॑ चतु॒र्थेन॒ य ए॒वं ॅवि॒द्वाꣳश्च॒तुर॒श्चतु॑ष्टोमा॒न्थ् सोमा॑ना॒हर॑ति॒ तैर्यज॑ते॒ तेज॑ ए॒व ( ) प्र॑थ॒मेनाव॑ रुन्ध इन्द्रि॒यं द्वि॒तीये॑न ब्रह्मवर्च॒सं तृ॒तीये॑ना॒ऽन्नाद्यं॑ चतु॒र्थेन॒ यामे॒वात्रि॒र्॒. ऋद्धि॒मार्द्ध्नो॒त् तामे॒व यज॑मान ऋद्ध्नोति ॥ \newline

\textbf{Pada Paata} \newline

ये । उ॒द्यन्त॒ इत्यु॑त् - यन्तः॑ । स्तोमाः᳚ । श्रीः । सा । अत्रि᳚म् । श्र॒द्धादे॑व॒मिति॑ श्र॒द्धा - दे॒व॒म् । यज॑मानम् । च॒त्वारि॑ । वी॒र्या॑णि । न । उपेति॑ । अ॒न॒म॒न्न् । तेजः॑ । इ॒न्द्रि॒यम् । ब्र॒ह्म॒व॒र्च॒समिति॑ ब्रह्म-व॒र्च॒सम् । अ॒न्नाद्य॒मित्य॑न्न - अद्य᳚म् । सः । ए॒तान् । च॒तुरः॑ । चतु॑ष्टोमा॒निति॒ चतुः॑ - स्तो॒मा॒न्न् । सोमान्॑ । अ॒प॒श्य॒त् । तान् । एति॑ । अ॒ह॒र॒त् । तैः । अ॒य॒ज॒त॒ । तेजः॑ । ए॒व । प्र॒थ॒मेन॑ । अवेति॑ । अ॒रु॒न्ध॒ । इ॒न्द्रि॒यम् । द्वि॒तीये॑न । ब्र॒ह्म॒व॒र्च॒समिति॑ ब्रह्म - व॒र्च॒सम् । तृ॒तीये॑न । अ॒न्नाद्य॒मित्य॑न्न -अद्य᳚म् । च॒तु॒र्थेन॑ । यः । ए॒वम् । वि॒द्वान् । च॒तुरः॑ । चतु॑ष्टोमा॒निति॒ चतुः॑ - स्तो॒मा॒न् । सोमान्॑ । आ॒हर॒तीत्या᳚ - हर॑ति । तैः । यज॑ते । तेजः॑ । ए॒व ( ) । प्र॒थ॒मेन॑ । अवेति॑ । रु॒न्धे॒ । इ॒न्द्रि॒यम् । द्वि॒तीये॑न । ब्र॒ह्म॒व॒र्च॒समिति॑ ब्रह्म - व॒र्च॒सम् । तृ॒तीये॑न । अ॒न्नाद्य॒मित्य॑न्न - अद्य᳚म् । च॒तु॒र्थेन॑ । याम् । ए॒व । अत्रिः॑ । ऋद्धि᳚म् । आद्‌र्ध्नो᳚त् । ताम् । ए॒व । यज॑मानः । ऋ॒द्ध्नो॒ति॒ ॥  \newline


\textbf{Krama Paata} \newline

य उ॒द्यन्तः॑ । उ॒द्यन्तः॒ स्तोमाः᳚ । उ॒द्यन्त॒ इत्यु॑त् - यन्तः॑ । स्तोमाः॒ श्रीः । श्रीः सा । साऽत्रि᳚म् । अत्रिꣳ॑ श्र॒द्धादे॑वम् । श्र॒द्धादे॑व॒म् ॅयज॑मानम् । श्र॒द्धादे॑व॒मिति॑ श्र॒द्धा - दे॒व॒म् । यज॑मानम् च॒त्वारि॑ । च॒त्वारि॑ वी॒र्या॑णि । वी॒र्या॑णि॒ न । नोप॑ । उपा॑नमन्न् । अ॒न॒म॒न् तेजः॑ । तेज॑ इन्द्रि॒यम् । इ॒न्द्रि॒यम् ब्र॑ह्मवर्च॒सम् । ब्र॒ह्म॒व॒र्च॒सम॒न्नाद्य᳚म् । ब्र॒ह्म॒व॒र्च॒समिति॑ ब्रह्म - व॒र्च॒सम् । अ॒न्नाद्यꣳ॒॒ सः । अ॒न्नाद्य॒मित्य॑न्न - अद्य᳚म् । स ए॒तान् । ए॒ताꣳश्च॒तुरः॑ । च॒तुर॒श्चतु॑ष्टोमान् । चतु॑ष्टोमा॒न्थ् सोमान्॑ । चतु॑ष्टोमा॒निति॒ चतुः॑ - स्तो॒मा॒न्॒ । सोमा॑नपश्यत् । अ॒प॒श्य॒त् तान् । ताना । आऽह॑रत् । अ॒ह॒र॒त् तैः । तैर॑यजत । अ॒य॒ज॒त॒ तेजः॑ । तेज॑ ए॒व । ए॒व प्र॑थ॒मेन॑ । प्र॒थ॒मेनाव॑ । अवा॑रुन्ध । अ॒रु॒न्धे॒न्द्रि॒यम् । इ॒न्द्रि॒यम् द्वि॒तीये॑न । द्वि॒तीये॑न ब्रह्मवर्च॒सम् । ब्र॒ह्म॒व॒र्च॒सम् तृ॒तीये॑न । ब्र॒ह्म॒व॒र्च॒समिति॑ ब्रह्म - व॒र्च॒सम् । तृ॒तीये॑ना॒न्नाद्य᳚म् । अ॒न्नाद्य॑म् चतु॒र्थेन॑ । अ॒न्नाद्य॒मित्य॑न्न - अद्य᳚म् । च॒तु॒र्थेन॒ यः । य ए॒वम् । ए॒वम् ॅवि॒द्वान् । वि॒द्वाꣳश्च॒तुरः॑ । च॒तुर॒श्चतु॑ष्टोमान् । चतु॑ष्टोमा॒न् थ्सोमान्॑ । चतु॑ष्टोमा॒निति॒ चतुः॑ - स्तो॒मा॒न्॒ । सोमा॑ना॒हर॑ति । आ॒हर॑ति॒ तैः । आ॒हर॒तीत्या᳚ - हर॑ति । तैर् यज॑ते । यज॑ते॒ तेजः॑ । तेज॑ ए॒व ( ) । ए॒व प्र॑थ॒मेन॑ । प्र॒थ॒मेनाव॑ । अव॑ रुन्धे । रु॒न्ध॒ इ॒न्द्रि॒यम् । इ॒न्द्रि॒यम् द्वि॒तीये॑न । द्वि॒तीये॑न ब्रह्मवर्च॒सम् । ब्र॒ह्म॒व॒र्च॒सम् तृ॒तीये॑न । ब्र॒ह्म॒व॒र्च॒समिति॑ ब्रह्म - व॒र्च॒सम् । तृ॒तीये॑ना॒न्नाद्य᳚म् । अ॒न्नाद्य॑म् चतु॒र्थेन॑ । अ॒न्नाद्य॒मित्य॑न्न - अद्य᳚म् । च॒तु॒र्थेन॒ याम् । यामे॒व । ए॒वात्रिः॑ । अत्रि॒र्॒. ऋद्धि᳚म् । ऋद्धि॒मार्द्ध्नो᳚त् । आर्द्ध्नो॒त् ताम् । तामे॒व । ए॒व यज॑मानः । यज॑मान ऋद्ध्नोति । ऋ॒द्ध्नो॒तीत्यृ॑द्ध्नोति । \newline

\textbf{Jatai Paata} \newline

1. य उ॒द्यन्त॑ उ॒द्यन्तो॒ ये य उ॒द्यन्तः॑ । \newline
2. उ॒द्यन्तः॒ स्तोमाः॒ स्तोमा॑ उ॒द्यन्त॑ उ॒द्यन्तः॒ स्तोमाः᳚ । \newline
3. उ॒द्यन्त॒ इत्यु॑त् - यन्तः॑ । \newline
4. स्तोमाः॒ श्रीः श्रीः स्तोमाः॒ स्तोमाः॒ श्रीः । \newline
5. श्रीः सा सा श्रीः श्रीः सा । \newline
6. सा ऽत्रि॒ मत्रिꣳ॒॒ सा सा ऽत्रि᳚म् । \newline
7. अत्रिꣳ॑ श्र॒द्धादे॑वꣳ श्र॒द्धादे॑व॒ मत्रि॒ मत्रिꣳ॑ श्र॒द्धादे॑वम् । \newline
8. श्र॒द्धादे॑वं॒ ॅयज॑मानं॒ ॅयज॑मानꣳ श्र॒द्धादे॑वꣳ श्र॒द्धादे॑वं॒ ॅयज॑मानम् । \newline
9. श्र॒द्धादे॑व॒मिति॑ श्र॒द्धा - दे॒व॒म् । \newline
10. यज॑मानम् च॒त्वारि॑ च॒त्वारि॒ यज॑मानं॒ ॅयज॑मानम् च॒त्वारि॑ । \newline
11. च॒त्वारि॑ वी॒र्या॑णि वी॒र्या॑णि च॒त्वारि॑ च॒त्वारि॑ वी॒र्या॑णि । \newline
12. वी॒र्या॑णि॒ न न वी॒र्या॑णि वी॒र्या॑णि॒ न । \newline
13. नोपोप॒ न नोप॑ । \newline
14. उपा॑ नमन् ननम॒न् नुपोपा॑ नमन्न् । \newline
15. अ॒न॒म॒न् तेज॒ स्तेजो॑ ऽनमन् ननम॒न् तेजः॑ । \newline
16. तेज॑ इन्द्रि॒य मि॑न्द्रि॒यम् तेज॒ स्तेज॑ इन्द्रि॒यम् । \newline
17. इ॒न्द्रि॒यम् ब्र॑ह्मवर्च॒सम् ब्र॑ह्मवर्च॒स मि॑न्द्रि॒य मि॑न्द्रि॒यम् ब्र॑ह्मवर्च॒सम् । \newline
18. ब्र॒ह्म॒व॒र्च॒स म॒न्नाद्य॑ म॒न्नाद्य॑म् ब्रह्मवर्च॒सम् ब्र॑ह्मवर्च॒स म॒न्नाद्य᳚म् । \newline
19. ब्र॒ह्म॒व॒र्च॒समिति॑ ब्रह्म - व॒र्च॒सम् । \newline
20. अ॒न्नाद्यꣳ॒॒ स सो᳚ ऽन्नाद्य॑ म॒न्नाद्यꣳ॒॒ सः । \newline
21. अ॒न्नाद्य॒मित्य॑न्न - अद्य᳚म् । \newline
22. स ए॒ता ने॒तान् थ्स स ए॒तान् । \newline
23. ए॒ताꣳ श्च॒तुर॑ श्च॒तुर॑ ए॒ता ने॒ताꣳ श्च॒तुरः॑ । \newline
24. च॒तुर॒ श्चतु॑ष्टोमाꣳ॒॒ श्चतु॑ष्टोमाꣳ श्च॒तुर॑ श्च॒तुर॒ श्चतु॑ष्टोमान् । \newline
25. चतु॑ष्टोमा॒न् थ्सोमा॒न् थ्सोमाꣳ॒॒ श्चतु॑ष्टोमाꣳ॒॒ श्चतु॑ष्टोमा॒न् थ्सोमान्॑ । \newline
26. चतु॑ष्टोमा॒निति॒ चतुः॑ - स्तो॒मा॒न् । \newline
27. सोमा॑ नपश्य दपश्य॒थ् सोमा॒न् थ्सोमा॑ नपश्यत् । \newline
28. अ॒प॒श्य॒त् ताꣳ स्ता न॑पश्य अपश्य॒त् तान् । \newline
29. ताना ताꣳ स्ताना । \newline
30. आ ऽह॑र दहर॒दा ऽह॑रत् । \newline
31. अ॒ह॒र॒त् तै स्तै र॑हर दहर॒त् तैः । \newline
32. तै र॑यजता यजत॒ तै स्तै र॑यजत । \newline
33. अ॒य॒ज॒त॒ तेज॒ स्तेजो॑ ऽयजता यजत॒ तेजः॑ । \newline
34. तेज॑ ए॒वैव तेज॒ स्तेज॑ ए॒व । \newline
35. ए॒व प्र॑थ॒मेन॑ प्रथ॒मे नै॒वैव प्र॑थ॒मेन॑ । \newline
36. प्र॒थ॒मेना वाव॑ प्रथ॒मेन॑ प्रथ॒मे नाव॑ । \newline
37. अवा॑ रुन्धा रु॒न्धा वावा॑ रुन्ध । \newline
38. अ॒रु॒न्धे॒न्द्रि॒य मि॑न्द्रि॒य म॑रुन्धा रुन्धेन्द्रि॒यम् । \newline
39. इ॒न्द्रि॒यम् द्वि॒तीये॑न द्वि॒तीये॑ नेन्द्रि॒य मि॑न्द्रि॒यम् द्वि॒तीये॑न । \newline
40. द्वि॒तीये॑न ब्रह्मवर्च॒सम् ब्र॑ह्मवर्च॒सम् द्वि॒तीये॑न द्वि॒तीये॑न ब्रह्मवर्च॒सम् । \newline
41. ब्र॒ह्म॒व॒र्च॒सम् तृ॒तीये॑न तृ॒तीये॑न ब्रह्मवर्च॒सम् ब्र॑ह्मवर्च॒सम् तृ॒तीये॑न । \newline
42. ब्र॒ह्म॒व॒र्च॒समिति॑ ब्रह्म - व॒र्च॒सम् । \newline
43. तृ॒तीये॑ना॒ न्नाद्य॑ म॒न्नाद्य॑म् तृ॒तीये॑न तृ॒तीये॑ना॒ न्नाद्य᳚म् । \newline
44. अ॒न्नाद्य॑म् चतु॒र्थेन॑ चतु॒र्थेना॒ न्नाद्य॑ म॒न्नाद्य॑म् चतु॒र्थेन॑ । \newline
45. अ॒न्नाद्य॒मित्य॑न्न - अद्य᳚म् । \newline
46. च॒तु॒र्थेन॒ यो यश्च॑तु॒र्थेन॑ चतु॒र्थेन॒ यः । \newline
47. य ए॒व मे॒वं ॅयो य ए॒वम् । \newline
48. ए॒वं ॅवि॒द्वान्. वि॒द्वा ने॒व मे॒वं ॅवि॒द्वान् । \newline
49. वि॒द्वाꣳ श्च॒तुर॑ श्च॒तुरो॑ वि॒द्वान्. वि॒द्वाꣳ श्च॒तुरः॑ । \newline
50. च॒तुर॒ श्चतु॑ष्टोमाꣳ॒॒ श्चतु॑ष्टोमाꣳ श्च॒तुर॑ श्च॒तुर॒ श्चतु॑ष्टोमान् । \newline
51. चतु॑ष्टोमा॒न् थ्सोमा॒न् थ्सोमाꣳ॒॒ श्चतु॑ष्टोमाꣳ॒॒ श्चतु॑ष्टोमा॒न् थ्सोमान्॑ । \newline
52. चतु॑ष्टोमा॒निति॒ चतुः॑ - स्तो॒मा॒न् । \newline
53. सोमा॑ ना॒हर॑ त्या॒हर॑ति॒ सोमा॒न् थ्सोमा॑ ना॒हर॑ति । \newline
54. आ॒हर॑ति॒ तै स्तै रा॒हर॑ त्या॒हर॑ति॒ तैः । \newline
55. आ॒हर॒तीत्या᳚ - हर॑ति । \newline
56. तैर् यज॑ते॒ यज॑ते॒ तै स्तैर् यज॑ते । \newline
57. यज॑ते॒ तेज॒ स्तेजो॒ यज॑ते॒ यज॑ते॒ तेजः॑ । \newline
58. तेज॑ ए॒वैव तेज॒ स्तेज॑ ए॒व । \newline
59. ए॒व प्र॑थ॒मेन॑ प्रथ॒मे नै॒वैव प्र॑थ॒मेन॑ । \newline
60. प्र॒थ॒मेना वाव॑ प्रथ॒मेन॑ प्रथ॒मे नाव॑ । \newline
61. अव॑ रुन्धे रु॒न्धे ऽवाव॑ रुन्धे । \newline
62. रु॒न्ध॒ इ॒न्द्रि॒य मि॑न्द्रि॒यꣳ रु॑न्धे रुन्ध इन्द्रि॒यम् । \newline
63. इ॒न्द्रि॒यम् द्वि॒तीये॑न द्वि॒तीये॑ नेन्द्रि॒य मि॑न्द्रि॒यम् द्वि॒तीये॑न । \newline
64. द्वि॒तीये॑न ब्रह्मवर्च॒सम् ब्र॑ह्मवर्च॒सम् द्वि॒तीये॑न द्वि॒तीये॑न ब्रह्मवर्च॒सम् । \newline
65. ब्र॒ह्म॒व॒र्च॒सम् तृ॒तीये॑न तृ॒तीये॑न ब्रह्मवर्च॒सम् ब्र॑ह्मवर्च॒सम् तृ॒तीये॑न । \newline
66. ब्र॒ह्म॒व॒र्च॒समिति॑ ब्रह्म - व॒र्च॒सम् । \newline
67. तृ॒तीये॑ना॒ न्नाद्य॑ म॒न्नाद्य॑म् तृ॒तीये॑न तृ॒तीये॑ना॒ न्नाद्य᳚म् । \newline
68. अ॒न्नाद्य॑म् चतु॒र्थेन॑ चतु॒र्थेना॒ न्नाद्य॑ म॒न्नाद्य॑म् चतु॒र्थेन॑ । \newline
69. अ॒न्नाद्य॒मित्य॑न्न - अद्य᳚म् । \newline
70. च॒तु॒र्थेन॒ यां ॅयाम् च॑तु॒र्थेन॑ चतु॒र्थेन॒ याम् । \newline
71. या मे॒वैव यां ॅया मे॒व । \newline
72. ए॒वा त्रि॒ रत्रि॑ रे॒वैवा त्रिः॑ । \newline
73. अत्रि॒र्॒. ऋद्धि॒ मृद्धि॒ मत्रि॒ रत्रि॒र्॒. ऋद्धि᳚म् । \newline
74. ऋद्धि॒ मार्द्ध्नो॒ दार्द्ध्नो॒ दृद्धि॒ मृद्धि॒ मार्द्ध्नो᳚त् । \newline
75. आर्द्ध्नो॒त् ताम् ता मार्द्ध्नो॒ दार्द्ध्नो॒त् ताम् । \newline
76. ता मे॒वैव ताम् ता मे॒व । \newline
77. ए॒व यज॑मानो॒ यज॑मान ए॒वैव यज॑मानः । \newline
78. यज॑मान ऋद्ध्नो त्यृद्ध्नोति॒ यज॑मानो॒ यज॑मान ऋद्ध्नोति । \newline
79. ऋ॒द्ध्नो॒तीत्यृ॑द्ध्नोति । \newline

\textbf{Ghana Paata } \newline

1. य उ॒द्यन्त॑ उ॒द्यन्तो॒ ये य उ॒द्यन्तः॒ स्तोमाः॒ स्तोमा॑ उ॒द्यन्तो॒ ये य उ॒द्यन्तः॒ स्तोमाः᳚ । \newline
2. उ॒द्यन्तः॒ स्तोमाः॒ स्तोमा॑ उ॒द्यन्त॑ उ॒द्यन्तः॒ स्तोमाः॒ श्रीः श्रीः स्तोमा॑ उ॒द्यन्त॑ उ॒द्यन्तः॒ स्तोमाः॒ श्रीः । \newline
3. उ॒द्यन्त॒ इत्यु॑त् - यन्तः॑ । \newline
4. स्तोमाः॒ श्रीः श्रीः स्तोमाः॒ स्तोमाः॒ श्रीः सा सा श्रीः स्तोमाः॒ स्तोमाः॒ श्रीः सा । \newline
5. श्रीः सा सा श्रीः श्रीः सा ऽत्रि॒ मत्रिꣳ॒॒ सा श्रीः श्रीः सा ऽत्रि᳚म् । \newline
6. सा ऽत्रि॒ मत्रिꣳ॒॒ सा सा ऽत्रिꣳ॑ श्र॒द्धादे॑वꣳ श्र॒द्धादे॑व॒ मत्रिꣳ॒॒ सा सा ऽत्रिꣳ॑ श्र॒द्धादे॑वम् । \newline
7. अत्रिꣳ॑ श्र॒द्धादे॑वꣳ श्र॒द्धादे॑व॒ मत्रि॒ मत्रिꣳ॑ श्र॒द्धादे॑वं॒ ॅयज॑मानं॒ ॅयज॑मानꣳ श्र॒द्धादे॑व॒ मत्रि॒ मत्रिꣳ॑ श्र॒द्धादे॑वं॒ ॅयज॑मानम् । \newline
8. श्र॒द्धादे॑वं॒ ॅयज॑मानं॒ ॅयज॑मानꣳ श्र॒द्धादे॑वꣳ श्र॒द्धादे॑वं॒ ॅयज॑मानम् च॒त्वारि॑ च॒त्वारि॒ यज॑मानꣳ श्र॒द्धादे॑वꣳ श्र॒द्धादे॑वं॒ ॅयज॑मानम् च॒त्वारि॑ । \newline
9. श्र॒द्धादे॑व॒मिति॑ श्र॒द्धा - दे॒व॒म् । \newline
10. यज॑मानम् च॒त्वारि॑ च॒त्वारि॒ यज॑मानं॒ ॅयज॑मानम् च॒त्वारि॑ वी॒र्या॑णि वी॒र्या॑णि च॒त्वारि॒ यज॑मानं॒ ॅयज॑मानम् च॒त्वारि॑ वी॒र्या॑णि । \newline
11. च॒त्वारि॑ वी॒र्या॑णि वी॒र्या॑णि च॒त्वारि॑ च॒त्वारि॑ वी॒र्या॑णि॒ न न वी॒र्या॑णि च॒त्वारि॑ च॒त्वारि॑ वी॒र्या॑णि॒ न । \newline
12. वी॒र्या॑णि॒ न न वी॒र्या॑णि वी॒र्या॑णि॒ नोपोप॒ न वी॒र्या॑णि वी॒र्या॑णि॒ नोप॑ । \newline
13. नोपोप॒ न नोपा॑ नमन् ननम॒न् नुप॒ न नोपा॑ नमन्न् । \newline
14. उपा॑ नमन् ननम॒न् नुपोपा॑ नम॒न् तेज॒ स्तेजो॑ ऽनम॒न् नुपोपा॑ नम॒न् तेजः॑ । \newline
15. अ॒न॒म॒न् तेज॒ स्तेजो॑ ऽनमन् ननम॒न् तेज॑ इन्द्रि॒य मि॑न्द्रि॒यम् तेजो॑ ऽनमन् ननम॒न् तेज॑ इन्द्रि॒यम् । \newline
16. तेज॑ इन्द्रि॒य मि॑न्द्रि॒यम् तेज॒ स्तेज॑ इन्द्रि॒यम् ब्र॑ह्मवर्च॒सम् ब्र॑ह्मवर्च॒स मि॑न्द्रि॒यम् तेज॒स्तेज॑ इन्द्रि॒यम् ब्र॑ह्मवर्च॒सम् । \newline
17. इ॒न्द्रि॒यम् ब्र॑ह्मवर्च॒सम् ब्र॑ह्मवर्च॒स मि॑न्द्रि॒य मि॑न्द्रि॒यम् ब्र॑ह्मवर्च॒स म॒न्नाद्य॑ म॒न्नाद्य॑म् ब्रह्मवर्च॒स मि॑न्द्रि॒य मि॑न्द्रि॒यम् ब्र॑ह्मवर्च॒स म॒न्नाद्य᳚म् । \newline
18. ब्र॒ह्म॒व॒र्च॒स म॒न्नाद्य॑ म॒न्नाद्य॑म् ब्रह्मवर्च॒सम् ब्र॑ह्मवर्च॒स म॒न्नाद्यꣳ॒॒ स सो᳚ ऽन्नाद्य॑म् ब्रह्मवर्च॒सम् ब्र॑ह्मवर्च॒स म॒न्नाद्यꣳ॒॒ सः । \newline
19. ब्र॒ह्म॒व॒र्च॒समिति॑ ब्रह्म - व॒र्च॒सम् । \newline
20. अ॒न्नाद्यꣳ॒॒ स सो᳚ ऽन्नाद्य॑ म॒न्नाद्यꣳ॒॒ स ए॒ता ने॒तान् थ्सो᳚ ऽन्नाद्य॑ म॒न्नाद्यꣳ॒॒ स ए॒तान् । \newline
21. अ॒न्नाद्य॒मित्य॑न्न - अद्य᳚म् । \newline
22. स ए॒ता ने॒तान् थ्स स ए॒ताꣳ श्च॒तुर॑ श्च॒तुर॑ ए॒तान् थ्स स ए॒ताꣳ श्च॒तुरः॑ । \newline
23. ए॒ताꣳ श्च॒तुर॑ श्च॒तुर॑ ए॒ता ने॒ताꣳ श्च॒तुर॒ श्चतु॑ष्टोमाꣳ॒॒ श्चतु॑ष्टोमाꣳ श्च॒तुर॑ ए॒ता ने॒ताꣳश्च॒तुर॒ श्चतु॑ष्टोमान् । \newline
24. च॒तुर॒ श्चतु॑ष्टोमाꣳ॒॒ श्चतु॑ष्टोमाꣳ श्च॒तुर॑ श्च॒तुर॒ श्चतु॑ष्टोमा॒न् थ्सोमा॒न् थ्सोमाꣳ॒॒ श्चतु॑ष्टोमाꣳ श्च॒तुर॑ श्च॒तुर॒ श्चतु॑ष्टोमा॒न् थ्सोमान्॑ । \newline
25. चतु॑ष्टोमा॒न् थ्सोमा॒न् थ्सोमाꣳ॒॒ श्चतु॑ष्टोमाꣳ॒॒ श्चतु॑ष्टोमा॒न् थ्सोमा॑ नपश्य दपश्य॒थ् 
सोमाꣳ॒॒ श्चतु॑ष्टोमाꣳ॒॒ श्चतु॑ष्टोमा॒न् थ्सोमा॑ नपश्यत् । \newline
26. चतु॑ष्टोमा॒निति॒ चतुः॑ - स्तो॒मा॒न् । \newline
27. सोमा॑ नपश्य दपश्य॒थ् सोमा॒न् थ्सोमा॑न पश्य॒त् ताꣳ स्तान॑ पश्य॒थ् सोमा॒न् थ्सोमा॑ नपश्य॒त् तान् । \newline
28. अ॒प॒श्य॒त् ताꣳ स्तान॑ पश्य दपश्य॒त् ताना तान॑ पश्य दपश्य॒त् ताना । \newline
29. ताना ताꣳ स्ताना ऽह॑र दहर॒दा ताꣳ स्ताना ऽह॑रत् । \newline
30. आ ऽह॑र दहर॒दा ऽह॑र॒त् तै स्तै र॑हर॒दा ऽह॑र॒त् तैः । \newline
31. अ॒ह॒र॒त् तै स्तै र॑हर दहर॒त् तै र॑यजता यजत॒ तै र॑हर दहर॒त् तै र॑यजत । \newline
32. तै र॑यजता यजत॒ तै स्तै र॑यजत॒ तेज॒ स्तेजो॑ ऽयजत॒ तै स्तै र॑यजत॒ तेजः॑ । \newline
33. अ॒य॒ज॒त॒ तेज॒ स्तेजो॑ ऽयजता यजत॒ तेज॑ ए॒वैव तेजो॑ ऽयजता यजत॒ तेज॑ ए॒व । \newline
34. तेज॑ ए॒वैव तेज॒ स्तेज॑ ए॒व प्र॑थ॒मेन॑ प्रथ॒मे नै॒व तेज॒ स्तेज॑ ए॒व प्र॑थ॒मेन॑ । \newline
35. ए॒व प्र॑थ॒मेन॑ प्रथ॒मे नै॒वैव प्र॑थ॒मेना वाव॑ प्रथ॒मे नै॒वैव प्र॑थ॒मे नाव॑ । \newline
36. प्र॒थ॒मेना वाव॑ प्रथ॒मेन॑ प्रथ॒मे नावा॑ रुन्धा रु॒न्धाव॑ प्रथ॒मेन॑ प्रथ॒मे नावा॑ रुन्ध । \newline
37. अवा॑ रुन्धा रु॒न्धा वावा॑ रुन्धेन्द्रि॒य मि॑न्द्रि॒य म॑रु॒न्धा वावा॑ रुन्धेन्द्रि॒यम् । \newline
38. अ॒रु॒न्धे॒न्द्रि॒य मि॑न्द्रि॒य म॑रुन्धा रुन्धेन्द्रि॒यम् द्वि॒तीये॑न द्वि॒तीये॑ नेन्द्रि॒य म॑रुन्धा रुन्धेन्द्रि॒यम् द्वि॒तीये॑न । \newline
39. इ॒न्द्रि॒यम् द्वि॒तीये॑न द्वि॒तीये॑ नेन्द्रि॒य मि॑न्द्रि॒यम् द्वि॒तीये॑न ब्रह्मवर्च॒सम् ब्र॑ह्मवर्च॒सम् द्वि॒तीये॑
नेन्द्रि॒य मि॑न्द्रि॒यम् द्वि॒तीये॑न ब्रह्मवर्च॒सम् । \newline
40. द्वि॒तीये॑न ब्रह्मवर्च॒सम् ब्र॑ह्मवर्च॒सम् द्वि॒तीये॑न द्वि॒तीये॑न ब्रह्मवर्च॒सम् तृ॒तीये॑न तृ॒तीये॑न ब्रह्मवर्च॒सम् द्वि॒तीये॑न द्वि॒तीये॑न ब्रह्मवर्च॒सम् तृ॒तीये॑न । \newline
41. ब्र॒ह्म॒व॒र्च॒सम् तृ॒तीये॑न तृ॒तीये॑न ब्रह्मवर्च॒सम् ब्र॑ह्मवर्च॒सम् तृ॒तीये॑ना॒ न्नाद्य॑ म॒न्नाद्य॑म् तृ॒तीये॑न ब्रह्मवर्च॒सम् ब्र॑ह्मवर्च॒सम् तृ॒तीये॑ना॒ न्नाद्य᳚म् । \newline
42. ब्र॒ह्म॒व॒र्च॒समिति॑ ब्रह्म - व॒र्च॒सम् । \newline
43. तृ॒तीये॑ना॒ न्नाद्य॑ म॒न्नाद्य॑म् तृ॒तीये॑न तृ॒तीये॑ना॒ न्नाद्य॑म् चतु॒र्थेन॑ चतु॒र्थेना॒ न्नाद्य॑म् तृ॒तीये॑न तृ॒तीये॑ना॒ न्नाद्य॑म् चतु॒र्थेन॑ । \newline
44. अ॒न्नाद्य॑म् चतु॒र्थेन॑ चतु॒र्थेना॒ न्नाद्य॑ म॒न्नाद्य॑म् चतु॒र्थेन॒ यो यश्च॑तु॒र्थेना॒ न्नाद्य॑ म॒न्नाद्य॑म् चतु॒र्थेन॒ यः । \newline
45. अ॒न्नाद्य॒मित्य॑न्न - अद्य᳚म् । \newline
46. च॒तु॒र्थेन॒ यो य श्च॑तु॒र्थेन॑ चतु॒र्थेन॒ य ए॒व मे॒वं ॅयश्च॑तु॒र्थेन॑ चतु॒र्थेन॒ य ए॒वम् । \newline
47. य ए॒व मे॒वं ॅयो य ए॒वं ॅवि॒द्वान्. वि॒द्वा ने॒वं ॅयो य ए॒वं ॅवि॒द्वान् । \newline
48. ए॒वं ॅवि॒द्वान्. वि॒द्वा ने॒व मे॒वं ॅवि॒द्वाꣳ श्च॒तुर॑ श्च॒तुरो॑ वि॒द्वा ने॒व मे॒वं ॅवि॒द्वाꣳ श्च॒तुरः॑ । \newline
49. वि॒द्वाꣳ श्च॒तुर॑ श्च॒तुरो॑ वि॒द्वान्. वि॒द्वाꣳ श्च॒तुर॒ श्चतु॑ष्टोमाꣳ॒॒ श्चतु॑ष्टोमाꣳ श्च॒तुरो॑ वि॒द्वान्. वि॒द्वाꣳ श्च॒तुर॒ श्चतु॑ष्टोमान् । \newline
50. च॒तुर॒ श्चतु॑ष्टोमाꣳ॒॒ श्चतु॑ष्टोमाꣳ श्च॒तुर॑ श्च॒तुर॒ श्चतु॑ष्टोमा॒न् थ्सोमा॒न् थ्सोमाꣳ॒॒ श्चतु॑ष्टोमाꣳ श्च॒तुर॑ श्च॒तुर॒ श्चतु॑ष्टोमा॒न् थ्सोमान्॑ । \newline
51. चतु॑ष्टोमा॒न् थ्सोमा॒न् थ्सोमाꣳ॒॒ श्चतु॑ष्टोमाꣳ॒॒ श्चतु॑ष्टोमा॒न् थ्सोमा॑ ना॒हर॑ त्या॒हर॑ति॒ सोमाꣳ॒॒ श्चतु॑ष्टोमाꣳ॒॒ श्चतु॑ष्टोमा॒न् थ्सोमा॑ ना॒हर॑ति । \newline
52. चतु॑ष्टोमा॒निति॒ चतुः॑ - स्तो॒मा॒न् । \newline
53. सोमा॑ ना॒हर॑ त्या॒हर॑ति॒ सोमा॒न् थ्सोमा॑ ना॒हर॑ति॒ तै स्तै रा॒हर॑ति॒ सोमा॒न् थ्सोमा॑ ना॒हर॑ति॒ तैः । \newline
54. आ॒हर॑ति॒ तै स्तै रा॒हर॑ त्या॒हर॑ति॒ तैर् यज॑ते॒ यज॑ते॒ तै रा॒हर॑ त्या॒हर॑ति॒ तैर् यज॑ते । \newline
55. आ॒हर॒तीत्या᳚ - हर॑ति । \newline
56. तैर् यज॑ते॒ यज॑ते॒ तै स्तैर् यज॑ते॒ तेज॒ स्तेजो॒ यज॑ते॒ तै स्तैर् यज॑ते॒ तेजः॑ । \newline
57. यज॑ते॒ तेज॒ स्तेजो॒ यज॑ते॒ यज॑ते॒ तेज॑ ए॒वैव तेजो॒ यज॑ते॒ यज॑ते॒ तेज॑ ए॒व । \newline
58. तेज॑ ए॒वैव तेज॒ स्तेज॑ ए॒व प्र॑थ॒मेन॑ प्रथ॒मे नै॒व तेज॒ स्तेज॑ ए॒व प्र॑थ॒मेन॑ । \newline
59. ए॒व प्र॑थ॒मेन॑ प्रथ॒मे नै॒वैव प्र॑थ॒मेना वाव॑ प्रथ॒मे नै॒वैव प्र॑थ॒मेनाव॑ । \newline
60. प्र॒थ॒मेना वाव॑ प्रथ॒मेन॑ प्रथ॒मे नाव॑ रुन्धे रु॒न्धे ऽव॑ प्रथ॒मेन॑ प्रथ॒मे नाव॑ रुन्धे । \newline
61. अव॑ रुन्धे रु॒न्धे ऽवाव॑ रुन्ध इन्द्रि॒य मि॑न्द्रि॒यꣳ रु॒न्धे ऽवाव॑ रुन्ध इन्द्रि॒यम् । \newline
62. रु॒न्ध॒ इ॒न्द्रि॒य मि॑न्द्रि॒यꣳ रु॑न्धे रुन्ध इन्द्रि॒यम् द्वि॒तीये॑न द्वि॒तीये॑ नेन्द्रि॒यꣳ रु॑न्धे रुन्ध इन्द्रि॒यम् द्वि॒तीये॑न । \newline
63. इ॒न्द्रि॒यम् द्वि॒तीये॑न द्वि॒तीये॑ नेन्द्रि॒य मि॑न्द्रि॒यम् द्वि॒तीये॑न ब्रह्मवर्च॒सम् ब्र॑ह्मवर्च॒सम् द्वि॒तीये॑ 
नेन्द्रि॒य मि॑न्द्रि॒यम् द्वि॒तीये॑न ब्रह्मवर्च॒सम् । \newline
64. द्वि॒तीये॑न ब्रह्मवर्च॒सम् ब्र॑ह्मवर्च॒सम् द्वि॒तीये॑न द्वि॒तीये॑न ब्रह्मवर्च॒सम् तृ॒तीये॑न तृ॒तीये॑न ब्रह्मवर्च॒सम् द्वि॒तीये॑न द्वि॒तीये॑न ब्रह्मवर्च॒सम् तृ॒तीये॑न । \newline
65. ब्र॒ह्म॒व॒र्च॒सम् तृ॒तीये॑न तृ॒तीये॑न ब्रह्मवर्च॒सम् ब्र॑ह्मवर्च॒सम् तृ॒तीये॑ना॒ न्नाद्य॑ म॒न्नाद्य॑म् तृ॒तीये॑न ब्रह्मवर्च॒सम् ब्र॑ह्मवर्च॒सम् तृ॒तीये॑ना॒ न्नाद्य᳚म् । \newline
66. ब्र॒ह्म॒व॒र्च॒समिति॑ ब्रह्म - व॒र्च॒सम् । \newline
67. तृ॒तीये॑ना॒ न्नाद्य॑ म॒न्नाद्य॑म् तृ॒तीये॑न तृ॒तीये॑ना॒ न्नाद्य॑म् चतु॒र्थेन॑ चतु॒र्थेना॒ न्नाद्य॑म् तृ॒तीये॑न तृ॒तीये॑ना॒ न्नाद्य॑म् चतु॒र्थेन॑ । \newline
68. अ॒न्नाद्य॑म् चतु॒र्थेन॑ चतु॒र्थेना॒ न्नाद्य॑ म॒न्नाद्य॑म् चतु॒र्थेन॒ यां ॅयाम् च॑तु॒र्थेना॒ न्नाद्य॑ म॒न्नाद्य॑म् चतु॒र्थेन॒ याम् । \newline
69. अ॒न्नाद्य॒मित्य॑न्न - अद्य᳚म् । \newline
70. च॒तु॒र्थेन॒ यां ॅयाम् च॑तु॒र्थेन॑ चतु॒र्थेन॒ या मे॒वैव याम् च॑तु॒र्थेन॑ चतु॒र्थेन॒ या मे॒व । \newline
71. या मे॒वैव यां ॅया मे॒वात्रि॒ रत्रि॑ रे॒व यां ॅया मे॒वात्रिः॑ । \newline
72. ए॒वात्रि॒ रत्रि॑ रे॒वै वात्रि॒र्॒. ऋद्धि॒ मृद्धि॒ मत्रि॑ रे॒वै वात्रि॒र्॒. ऋद्धि᳚म् । \newline
73. अत्रि॒र्॒. ऋद्धि॒ मृद्धि॒ मत्रि॒ रत्रि॒र्॒. ऋद्धि॒ मार्द्ध्नो॒ दार्द्ध्नो॒ दृद्धि॒ मत्रि॒ रत्रि॒र्॒. ऋद्धि॒ मार्द्ध्नो᳚त् । \newline
74. ऋद्धि॒ मार्द्ध्नो॒ दार्द्ध्नो॒ दृद्धि॒ मृद्धि॒ मार्द्ध्नो॒त् ताम् ता मार्द्ध्नो॒ दृद्धि॒ मृद्धि॒ मार्द्ध्नो॒त् ताम् । \newline
75. आर्द्ध्नो॒त् ताम् ता मार्द्ध्नो॒ दार्द्ध्नो॒त् ता मे॒वैव ता मार्द्ध्नो॒ दार्द्ध्नो॒त् ता मे॒व । \newline
76. ता मे॒वैव ताम् ता मे॒व यज॑मानो॒ यज॑मान ए॒व ताम् ता मे॒व यज॑मानः । \newline
77. ए॒व यज॑मानो॒ यज॑मान ए॒वैव यज॑मान ऋद्ध्नो त्यृद्ध्नोति॒ यज॑मान ए॒वैव यज॑मान ऋद्ध्नोति । \newline
78. यज॑मान ऋद्ध्नो त्यृद्ध्नोति॒ यज॑मानो॒ यज॑मान ऋद्ध्नोति । \newline
79. ऋ॒द्ध्नो॒तीत्यृ॑द्ध्नोति । \newline
\pagebreak
\markright{ TS 7.1.9.1  \hfill https://www.vedavms.in \hfill}

\section{ TS 7.1.9.1 }

\textbf{TS 7.1.9.1 } \newline
\textbf{Samhita Paata} \newline

ज॒मद॑ग्निः॒ पुष्टि॑काम-श्चतूरा॒त्रेणा॑-यजत॒ स ए॒तान् पोषाꣳ॑ अपुष्य॒त् तस्मा᳚त् पलि॒तौ जाम॑दग्नियौ॒ न सं जा॑नाते ए॒ताने॒व पोषा᳚न् पुष्यति॒ य ए॒वं ॅवि॒द्वाꣳश्च॑तूरा॒त्रेण॒ यज॑ते पुरोडा॒शिन्य॑ उप॒सदो॑ भवन्ति प॒शवो॒ वै पु॑रो॒डाशः॑ प॒शूने॒वाव॑ रु॒न्धे ऽन्नं॒ ॅवै पु॑रो॒डाशोऽन्न॑मे॒वाव॑ रुन्धे ऽन्ना॒दः प॑शु॒मान् भ॑वति॒ य ए॒वं ॅवि॒द्वाꣳश्च॑तूरा॒त्रेण॒ यज॑ते ॥ \newline

\textbf{Pada Paata} \newline

ज॒मद॑ग्निः । पुष्टि॑काम॒ इति॒ पुष्टि॑ - का॒मः॒ । च॒तू॒रा॒त्रेणेति॑ चतुः - रा॒त्रेण॑ । अ॒य॒ज॒त॒ । सः । ए॒तान् । पोषान्॑ । अ॒पु॒ष्य॒त् । तस्मा᳚त् । प॒लि॒तौ । जाम॑दग्नियौ । न । समिति॑ । जा॒ना॒ते॒ इति॑ । ए॒तान् । ए॒व । पोषान्॑ । पु॒ष्य॒ति॒ । यः । ए॒वम् । वि॒द्वान् । च॒तू॒रा॒त्रेणेति॑ चतुः - रा॒त्रेण॑ । यज॑ते । पु॒रो॒डा॒शिन्यः॑ । उ॒प॒सद॒ इत्यु॑प - सदः॑ । भ॒व॒न्ति॒ । प॒शवः॑ । वै । पु॒रो॒डाशः॑ । प॒शून् । ए॒व । अवेति॑ । रु॒न्धे॒ । अन्न᳚म् । वै । पु॒रो॒डाशः॑ । अन्न᳚म् । ए॒व । अवेति॑ । रु॒न्धे॒ । अ॒न्ना॒द इत्य॑न्न - अ॒दः । प॒शु॒मानिति॑ पशु - मान् । भ॒व॒ति॒ । यः । ए॒वम् । वि॒द्वान् । च॒तू॒रा॒त्रेणेति॑ चतुः-रा॒त्रेण॑ । यज॑ते ॥  \newline


\textbf{Krama Paata} \newline

ज॒मद॑ग्निः॒ पुष्टि॑कामः । पुष्टि॑कामश्चतूरा॒त्रेण॑ । पुष्टि॑काम॒ इति॒ पुष्टि॑ - का॒मः॒ । च॒तू॒रा॒त्रेणा॑यजत । च॒तू॒रा॒त्रेणेति॑ चतुः - रा॒त्रेण॑ । अ॒य॒ज॒त॒ सः । स ए॒तान् । ए॒तान् पोषान्॑ । पोषाꣳ॑ अपुष्यत् । अ॒पु॒ष्य॒त् तस्मा᳚त् । तस्मा᳚त् पलि॒तौ । प॒लि॒तौ जाम॑दग्नियौ । जाम॑दग्नियौ॒ न । न सम् । सम् जा॑नाते । जा॒ना॒ते॒ ए॒तान् । जा॒ना॒ते॒ इति॑ जानाते । ए॒ताने॒व । ए॒व पोषान्॑ । पोषा᳚न् पुष्यति । पु॒ष्य॒ति॒ यः । य ए॒वम् । ए॒वम् ॅवि॒द्वान् । वि॒द्वाꣳश्च॑तूरा॒त्रेण॑ । च॒तू॒रा॒त्रेण॒ यज॑ते । च॒तू॒रा॒त्रेणेति॑ चतुः - रा॒त्रेण॑ । यज॑ते पुरोडा॒शिन्यः॑ । पु॒रो॒डा॒शिन्य॑ उप॒सदः॑ । उ॒प॒सदो॑ भवन्ति । उ॒प॒सद॒ इत्यु॑प - सदः॑ । भ॒व॒न्ति॒ प॒शवः॑ । प॒शवो॒ वै । वै पु॑रो॒डाशः॑ । पु॒रो॒डाशः॑ प॒शून् । प॒शूने॒व । ए॒वाव॑ । अव॑ रुन्धे । रु॒न्धेऽन्न᳚म् । अन्न॒म् ॅवै । वै पु॑रो॒डाशः॑ । पु॒रो॒डाशोऽन्न᳚म् । अन्न॑मे॒व । ए॒वाव॑ । अव॑ रुन्धे । रु॒न्धे॒ऽन्ना॒दः । अ॒न्ना॒दः प॑शु॒मान् । अ॒न्ना॒द इत्य॑न्न - अ॒दः । प॒शु॒मान् भ॑वति । प॒शु॒मानिति॑ पशु - मान् । भ॒व॒ति॒ यः । य ए॒वम् । ए॒वम् ॅवि॒द्वान् । वि॒द्वाꣳश्च॑तूरा॒त्रेण॑ । च॒तू॒रा॒त्रेण॒ यज॑ते । च॒तू॒रा॒त्रेणेति॑ चतुः - रा॒त्रेण॑ । यज॑त॒ इति॒ यज॑ते । \newline

\textbf{Jatai Paata} \newline

1. ज॒मद॑ग्निः॒ पुष्टि॑कामः॒ पुष्टि॑कामो ज॒मद॑ग्निर् ज॒मद॑ग्निः॒ पुष्टि॑कामः । \newline
2. पुष्टि॑काम श्चतूरा॒त्रेण॑ चतूरा॒त्रेण॒ पुष्टि॑कामः॒ पुष्टि॑काम श्चतूरा॒त्रेण॑ । \newline
3. पुष्टि॑काम॒ इति॒ पुष्टि॑ - का॒मः॒ । \newline
4. च॒तू॒रा॒त्रेणा॑ यजता यजत चतूरा॒त्रेण॑ चतूरा॒त्रेणा॑ यजत । \newline
5. च॒तू॒रा॒त्रेणेति॑ चतुः - रा॒त्रेण॑ । \newline
6. अ॒य॒ज॒त॒ स सो॑ ऽयजता यजत॒ सः । \newline
7. स ए॒ता ने॒तान् थ्स स ए॒तान् । \newline
8. ए॒तान् पोषा॒न् पोषाꣳ॑ ए॒ता ने॒तान् पोषान्॑ । \newline
9. पोषाꣳ॑ अपुष्य दपुष्य॒त् पोषा॒न् पोषाꣳ॑ अपुष्यत् । \newline
10. अ॒पु॒ष्य॒त् तस्मा॒त् तस्मा॑ दपुष्य दपुष्य॒त् तस्मा᳚त् । \newline
11. तस्मा᳚त् पलि॒तौ प॑लि॒तौ तस्मा॒त् तस्मा᳚त् पलि॒तौ । \newline
12. प॒लि॒तौ जाम॑दग्नियौ॒ जाम॑दग्नियौ पलि॒तौ प॑लि॒तौ जाम॑दग्नियौ । \newline
13. जाम॑दग्नियौ॒ न न जाम॑दग्नियौ॒ जाम॑दग्नियौ॒ न । \newline
14. न सꣳ सन् न न सम् । \newline
15. सम् जा॑नाते जानाते॒ सꣳ सम् जा॑नाते । \newline
16. जा॒ना॒ते॒ ए॒ता ने॒तान् जा॑नाते जानाते ए॒तान् । \newline
17. जा॒ना॒ते॒ इति॑ जानाते । \newline
18. ए॒ता ने॒वै वैता ने॒ता ने॒व । \newline
19. ए॒व पोषा॒न् पोषा॑ ने॒वैव पोषान्॑ । \newline
20. पोषा᳚न् पुष्यति पुष्यति॒ पोषा॒न् पोषा᳚न् पुष्यति । \newline
21. पु॒ष्य॒ति॒ यो यः पु॑ष्यति पुष्यति॒ यः । \newline
22. य ए॒व मे॒वं ॅयो य ए॒वम् । \newline
23. ए॒वं ॅवि॒द्वान्. वि॒द्वा ने॒व मे॒वं ॅवि॒द्वान् । \newline
24. वि॒द्वाꣳ श्च॑तूरा॒त्रेण॑ चतूरा॒त्रेण॑ वि॒द्वान्. वि॒द्वाꣳ श्च॑तूरा॒त्रेण॑ । \newline
25. च॒तू॒रा॒त्रेण॒ यज॑ते॒ यज॑ते चतूरा॒त्रेण॑ चतूरा॒त्रेण॒ यज॑ते । \newline
26. च॒तू॒रा॒त्रेणेति॑ चतुः - रा॒त्रेण॑ । \newline
27. यज॑ते पुरोडा॒शिन्यः॑ पुरोडा॒शिन्यो॒ यज॑ते॒ यज॑ते पुरोडा॒शिन्यः॑ । \newline
28. पु॒रो॒डा॒शिन्य॑ उप॒सद॑ उप॒सदः॑ पुरोडा॒शिन्यः॑ पुरोडा॒शिन्य॑ उप॒सदः॑ । \newline
29. उ॒प॒सदो॑ भवन्ति भव न्त्युप॒सद॑ उप॒सदो॑ भवन्ति । \newline
30. उ॒प॒सद॒ इत्यु॑प - सदः॑ । \newline
31. भ॒व॒न्ति॒ प॒शवः॑ प॒शवो॑ भवन्ति भवन्ति प॒शवः॑ । \newline
32. प॒शवो॒ वै वै प॒शवः॑ प॒शवो॒ वै । \newline
33. वै पु॑रो॒डाशः॑ पुरो॒डाशो॒ वै वै पु॑रो॒डाशः॑ । \newline
34. पु॒रो॒डाशः॑ प॒शून् प॒शून् पु॑रो॒डाशः॑ पुरो॒डाशः॑ प॒शून् । \newline
35. प॒शू ने॒वैव प॒शून् प॒शूने॒व । \newline
36. ए॒वावा वै॒वै वाव॑ । \newline
37. अव॑ रुन्धे रु॒न्धे ऽवाव॑ रुन्धे । \newline
38. रु॒न्धे ऽन्न॒ मन्नꣳ॑ रुन्धे रु॒न्धे ऽन्न᳚म् । \newline
39. अन्नं॒ ॅवै वा अन्न॒ मन्नं॒ ॅवै । \newline
40. वै पु॑रो॒डाशः॑ पुरो॒डाशो॒ वै वै पु॑रो॒डाशः॑ । \newline
41. पु॒रो॒डाशो ऽन्न॒ मन्न॑म् पुरो॒डाशः॑ पुरो॒डाशो ऽन्न᳚म् । \newline
42. अन्न॑ मे॒वै वान्न॒ मन्न॑ मे॒व । \newline
43. ए॒वावा वै॒वै वाव॑ । \newline
44. अव॑ रुन्धे रु॒न्धे ऽवाव॑ रुन्धे । \newline
45. रु॒न्धे॒ ऽन्ना॒दो᳚ ऽन्ना॒दो रु॑न्धे रुन्धे ऽन्ना॒दः । \newline
46. अ॒न्ना॒दः प॑शु॒मान् प॑शु॒मा न॑न्ना॒दो᳚ ऽन्ना॒दः प॑शु॒मान् । \newline
47. अ॒न्ना॒द इत्य॑न्न - अ॒दः । \newline
48. प॒शु॒मान् भ॑वति भवति पशु॒मान् प॑शु॒मान् भ॑वति । \newline
49. प॒शु॒मानिति॑ पशु - मान् । \newline
50. भ॒व॒ति॒ यो यो भ॑वति भवति॒ यः । \newline
51. य ए॒व मे॒वं ॅयो य ए॒वम् । \newline
52. ए॒वं ॅवि॒द्वान्. वि॒द्वा ने॒व मे॒वं ॅवि॒द्वान् । \newline
53. वि॒द्वाꣳ श्च॑तूरा॒त्रेण॑ चतूरा॒त्रेण॑ वि॒द्वान्. वि॒द्वाꣳ श्च॑तूरा॒त्रेण॑ । \newline
54. च॒तू॒रा॒त्रेण॒ यज॑ते॒ यज॑ते चतूरा॒त्रेण॑ चतूरा॒त्रेण॒ यज॑ते । \newline
55. च॒तू॒रा॒त्रेणेति॑ चतुः - रा॒त्रेण॑ । \newline
56. यज॑त॒ इति॒ यज॑ते । \newline

\textbf{Ghana Paata } \newline

1. ज॒मद॑ग्निः॒ पुष्टि॑कामः॒ पुष्टि॑कामो ज॒मद॑ग्निर् ज॒मद॑ग्निः॒ पुष्टि॑काम श्चतूरा॒त्रेण॑ चतूरा॒त्रेण॒ पुष्टि॑कामो ज॒मद॑ग्निर् ज॒मद॑ग्निः॒ पुष्टि॑काम श्चतूरा॒त्रेण॑ । \newline
2. पुष्टि॑काम श्चतूरा॒त्रेण॑ चतूरा॒त्रेण॒ पुष्टि॑कामः॒ पुष्टि॑काम श्चतूरा॒त्रेणा॑ यजता यजत चतूरा॒त्रेण॒ पुष्टि॑कामः॒ पुष्टि॑काम श्चतूरा॒त्रेणा॑ यजत । \newline
3. पुष्टि॑काम॒ इति॒ पुष्टि॑ - का॒मः॒ । \newline
4. च॒तू॒रा॒त्रेणा॑ यजता यजत चतूरा॒त्रेण॑ चतूरा॒त्रेणा॑ यजत॒ स सो॑ ऽयजत चतूरा॒त्रेण॑ चतूरा॒त्रेणा॑ यजत॒ सः । \newline
5. च॒तू॒रा॒त्रेणेति॑ चतुः - रा॒त्रेण॑ । \newline
6. अ॒य॒ज॒त॒ स सो॑ ऽयजता यजत॒ स ए॒ता ने॒तान् थ्सो॑ ऽयजता यजत॒ स ए॒तान् । \newline
7. स ए॒ता ने॒तान् थ्स स ए॒तान् पोषा॒न् पोषाꣳ॑ ए॒तान् थ्स स ए॒तान् पोषान्॑ । \newline
8. ए॒तान् पोषा॒न् पोषाꣳ॑ ए॒ता ने॒तान् पोषाꣳ॑ अपुष्य दपुष्य॒त् पोषाꣳ॑ ए॒ता ने॒तान् पोषाꣳ॑ अपुष्यत् । \newline
9. पोषाꣳ॑ अपुष्य दपुष्य॒त् पोषा॒न् पोषाꣳ॑ अपुष्य॒त् तस्मा॒त् तस्मा॑ दपुष्य॒त् पोषा॒न् पोषाꣳ॑ अपुष्य॒त् तस्मा᳚त् । \newline
10. अ॒पु॒ष्य॒त् तस्मा॒त् तस्मा॑ दपुष्य दपुष्य॒त् तस्मा᳚त् पलि॒तौ प॑लि॒तौ तस्मा॑ दपुष्य दपुष्य॒त् तस्मा᳚त् पलि॒तौ । \newline
11. तस्मा᳚त् पलि॒तौ प॑लि॒तौ तस्मा॒त् तस्मा᳚त् पलि॒तौ जाम॑दग्नियौ॒ जाम॑दग्नियौ पलि॒तौ तस्मा॒त् तस्मा᳚त् पलि॒तौ जाम॑दग्नियौ । \newline
12. प॒लि॒तौ जाम॑दग्नियौ॒ जाम॑दग्नियौ पलि॒तौ प॑लि॒तौ जाम॑दग्नियौ॒ न न जाम॑दग्नियौ पलि॒तौ प॑लि॒तौ जाम॑दग्नियौ॒ न । \newline
13. जाम॑दग्नियौ॒ न न जाम॑दग्नियौ॒ जाम॑दग्नियौ॒ न सꣳ सन् न जाम॑दग्नियौ॒ जाम॑दग्नियौ॒ न सम् । \newline
14. न सꣳ सन् न न सम् जा॑नाते जानाते॒ सन् न न सम् जा॑नाते । \newline
15. सम् जा॑नाते जानाते॒ सꣳ सम् जा॑नाते ए॒ता ने॒तान् जा॑नाते॒ सꣳ सम् जा॑नाते ए॒तान् । \newline
16. जा॒ना॒ते॒ ए॒ता ने॒तान् जा॑नाते जानाते ए॒ता ने॒वै वैतान् जा॑नाते जानाते ए॒ता ने॒व । \newline
17. जा॒ना॒ते॒ इति॑ जानाते । \newline
18. ए॒ता ने॒वै वैता ने॒ता ने॒व पोषा॒न् पोषा॑ ने॒वैता ने॒ता ने॒व पोषान्॑ । \newline
19. ए॒व पोषा॒न् पोषा॑ ने॒वैव पोषा᳚न् पुष्यति पुष्यति॒ पोषा॑ ने॒वैव पोषा᳚न् पुष्यति । \newline
20. पोषा᳚न् पुष्यति पुष्यति॒ पोषा॒न् पोषा᳚न् पुष्यति॒ यो यः पु॑ष्यति॒ पोषा॒न् पोषा᳚न् पुष्यति॒ यः । \newline
21. पु॒ष्य॒ति॒ यो यः पु॑ष्यति पुष्यति॒ य ए॒व मे॒वं ॅयः पु॑ष्यति पुष्यति॒ य ए॒वम् । \newline
22. य ए॒व मे॒वं ॅयो य ए॒वं ॅवि॒द्वान्. वि॒द्वा ने॒वं ॅयो य ए॒वं ॅवि॒द्वान् । \newline
23. ए॒वं ॅवि॒द्वान्. वि॒द्वा ने॒व मे॒वं ॅवि॒द्वाꣳ श्च॑तूरा॒त्रेण॑ चतूरा॒त्रेण॑ वि॒द्वा ने॒व मे॒वं ॅवि॒द्वाꣳ 
श्च॑तूरा॒त्रेण॑ । \newline
24. वि॒द्वाꣳ श्च॑तूरा॒त्रेण॑ चतूरा॒त्रेण॑ वि॒द्वान्. वि॒द्वाꣳ श्च॑तूरा॒त्रेण॒ यज॑ते॒ यज॑ते चतूरा॒त्रेण॑ वि॒द्वान्. वि॒द्वाꣳ श्च॑तूरा॒त्रेण॒ यज॑ते । \newline
25. च॒तू॒रा॒त्रेण॒ यज॑ते॒ यज॑ते चतूरा॒त्रेण॑ चतूरा॒त्रेण॒ यज॑ते पुरोडा॒शिन्यः॑ पुरोडा॒शिन्यो॒ यज॑ते चतूरा॒त्रेण॑ चतूरा॒त्रेण॒ यज॑ते पुरोडा॒शिन्यः॑ । \newline
26. च॒तू॒रा॒त्रेणेति॑ चतुः - रा॒त्रेण॑ । \newline
27. यज॑ते पुरोडा॒शिन्यः॑ पुरोडा॒शिन्यो॒ यज॑ते॒ यज॑ते पुरोडा॒शिन्य॑ उप॒सद॑ उप॒सदः॑ पुरोडा॒शिन्यो॒ यज॑ते॒ यज॑ते पुरोडा॒शिन्य॑ उप॒सदः॑ । \newline
28. पु॒रो॒डा॒शिन्य॑ उप॒सद॑ उप॒सदः॑ पुरोडा॒शिन्यः॑ पुरोडा॒शिन्य॑ उप॒सदो॑ भवन्ति भव न्त्युप॒सदः॑ पुरोडा॒शिन्यः॑ पुरोडा॒शिन्य॑ उप॒सदो॑ भवन्ति । \newline
29. उ॒प॒सदो॑ भवन्ति भव न्त्युप॒सद॑ उप॒सदो॑ भवन्ति प॒शवः॑ प॒शवो॑ भव न्त्युप॒सद॑ उप॒सदो॑ भवन्ति प॒शवः॑ । \newline
30. उ॒प॒सद॒ इत्यु॑प - सदः॑ । \newline
31. भ॒व॒न्ति॒ प॒शवः॑ प॒शवो॑ भवन्ति भवन्ति प॒शवो॒ वै वै प॒शवो॑ भवन्ति भवन्ति प॒शवो॒ वै । \newline
32. प॒शवो॒ वै वै प॒शवः॑ प॒शवो॒ वै पु॑रो॒डाशः॑ पुरो॒डाशो॒ वै प॒शवः॑ प॒शवो॒ वै पु॑रो॒डाशः॑ । \newline
33. वै पु॑रो॒डाशः॑ पुरो॒डाशो॒ वै वै पु॑रो॒डाशः॑ प॒शून् प॒शून् पु॑रो॒डाशो॒ वै वै पु॑रो॒डाशः॑ प॒शून् । \newline
34. पु॒रो॒डाशः॑ प॒शून् प॒शून् पु॑रो॒डाशः॑ पुरो॒डाशः॑ प॒शू ने॒वैव प॒शून् पु॑रो॒डाशः॑ पुरो॒डाशः॑ प॒शू ने॒व । \newline
35. प॒शूने॒ वैव प॒शून् प॒शूने॒ वावा वै॒व प॒शून् प॒शूने॒ वाव॑ । \newline
36. ए॒वावा वै॒वै वाव॑ रुन्धे रु॒न्धे ऽवै॒वै वाव॑ रुन्धे । \newline
37. अव॑ रुन्धे रु॒न्धे ऽवाव॑ रु॒न्धे ऽन्न॒ मन्नꣳ॑ रु॒न्धे ऽवाव॑ रु॒न्धे ऽन्न᳚म् । \newline
38. रु॒न्धे ऽन्न॒ मन्नꣳ॑ रुन्धे रु॒न्धे ऽन्नं॒ ॅवै वा अन्नꣳ॑ रुन्धे रु॒न्धे ऽन्नं॒ ॅवै । \newline
39. अन्नं॒ ॅवै वा अन्न॒ मन्नं॒ ॅवै पु॑रो॒डाशः॑ पुरो॒डाशो॒ वा अन्न॒ मन्नं॒ ॅवै पु॑रो॒डाशः॑ । \newline
40. वै पु॑रो॒डाशः॑ पुरो॒डाशो॒ वै वै पु॑रो॒डाशो ऽन्न॒ मन्न॑म् पुरो॒डाशो॒ वै वै पु॑रो॒डाशो ऽन्न᳚म् । \newline
41. पु॒रो॒डाशो ऽन्न॒ मन्न॑म् पुरो॒डाशः॑ पुरो॒डाशो ऽन्न॑ मे॒वै वान्न॑म् पुरो॒डाशः॑ पुरो॒डाशो ऽन्न॑ मे॒व । \newline
42. अन्न॑ मे॒वै वान्न॒ मन्न॑ मे॒वावा वै॒वान्न॒ मन्न॑ मे॒वाव॑ । \newline
43. ए॒वावा वै॒वै वाव॑ रुन्धे रु॒न्धे ऽवै॒वै वाव॑ रुन्धे । \newline
44. अव॑ रुन्धे रु॒न्धे ऽवाव॑ रुन्धे ऽन्ना॒दो᳚ ऽन्ना॒दो रु॒न्धे ऽवाव॑ रुन्धे ऽन्ना॒दः । \newline
45. रु॒न्धे॒ ऽन्ना॒दो᳚ ऽन्ना॒दो रु॑न्धे रुन्धे ऽन्ना॒दः प॑शु॒मान् प॑शु॒मा न॑न्ना॒दो रु॑न्धे रुन्धे ऽन्ना॒दः प॑शु॒मान् । \newline
46. अ॒न्ना॒दः प॑शु॒मान् प॑शु॒मा न॑न्ना॒दो᳚ ऽन्ना॒दः प॑शु॒मान् भ॑वति भवति पशु॒मा न॑न्ना॒दो᳚ ऽन्ना॒दः प॑शु॒मान् भ॑वति । \newline
47. अ॒न्ना॒द इत्य॑न्न - अ॒दः । \newline
48. प॒शु॒मान् भ॑वति भवति पशु॒मान् प॑शु॒मान् भ॑वति॒ यो यो भ॑वति पशु॒मान् प॑शु॒मान् भ॑वति॒ यः । \newline
49. प॒शु॒मानिति॑ पशु - मान् । \newline
50. भ॒व॒ति॒ यो यो भ॑वति भवति॒ य ए॒व मे॒वं ॅयो भ॑वति भवति॒ य ए॒वम् । \newline
51. य ए॒व मे॒वं ॅयो य ए॒वं ॅवि॒द्वान्. वि॒द्वा ने॒वं ॅयो य ए॒वं ॅवि॒द्वान् । \newline
52. ए॒वं ॅवि॒द्वान्. वि॒द्वा ने॒व मे॒वं ॅवि॒द्वाꣳ श्च॑तूरा॒त्रेण॑ चतूरा॒त्रेण॑ वि॒द्वा ने॒व मे॒वं ॅवि॒द्वाꣳ श्च॑तूरा॒त्रेण॑ । \newline
53. वि॒द्वाꣳ श्च॑तूरा॒त्रेण॑ चतूरा॒त्रेण॑ वि॒द्वान्. वि॒द्वाꣳ श्च॑तूरा॒त्रेण॒ यज॑ते॒ यज॑ते चतूरा॒त्रेण॑ वि॒द्वान्. वि॒द्वाꣳ श्च॑तूरा॒त्रेण॒ यज॑ते । \newline
54. च॒तू॒रा॒त्रेण॒ यज॑ते॒ यज॑ते चतूरा॒त्रेण॑ चतूरा॒त्रेण॒ यज॑ते । \newline
55. च॒तू॒रा॒त्रेणेति॑ चतुः - रा॒त्रेण॑ । \newline
56. यज॑त॒ इति॒ यज॑ते । \newline
\pagebreak
\markright{ TS 7.1.10.1  \hfill https://www.vedavms.in \hfill}

\section{ TS 7.1.10.1 }

\textbf{TS 7.1.10.1 } \newline
\textbf{Samhita Paata} \newline

सं॒ॅव॒थ्स॒रो वा इ॒दमेक॑ आसी॒थ् सो॑ऽकामयत॒र्तून्थ् सृ॑जे॒येति॒ स ए॒तं प॑ञ्चरा॒त्रम॑पश्य॒त् तमाऽह॑र॒त् तेना॑यजत॒ ततो॒ वै स ऋ॒तून॑सृजत॒ य ए॒वं ॅवि॒द्वान् प॑ञ्चरा॒त्रेण॒ यज॑ते॒ प्रैव जा॑यते॒ त ऋ॒तवः॑ सृ॒ष्टा न व्याव॑र्तन्त॒ त ए॒तं प॑ञ्चरा॒त्रम॑पश्य॒न् तमाऽह॑र॒न् तेना॑यजन्त॒ ततो॒ वै ते व्याव॑र्तन्त॒-[  ] \newline

\textbf{Pada Paata} \newline

सं॒ॅव॒थ्स॒र इति॑ सं - व॒थ्स॒रः । वै । इ॒दम् । एकः॑ । आ॒सी॒त् । सः । अ॒का॒म॒य॒त॒ । ऋ॒तून् । सृ॒जे॒य॒ । इति॑ । सः । ए॒तम् । प॒ञ्च॒रा॒त्रमिति॑ पञ्च - रा॒त्रम् । अ॒प॒श्य॒त् । तम् । एति॑ । अ॒ह॒र॒त् । तेन॑ । अ॒य॒ज॒त॒ । ततः॑ । वै । सः । ऋ॒तून् । अ॒सृ॒ज॒त॒ । यः । ए॒वम् । वि॒द्वान् । प॒ञ्च॒रा॒त्रेणेति॑ पञ्च - रा॒त्रेण॑ । यज॑ते । प्रेति॑ । ए॒व । जा॒य॒ते॒ । ते । ऋ॒तवः॑ । सृ॒ष्टाः । न । व्याव॑र्त॒न्तेति॑ वि - आव॑र्तन्त । ते । ए॒तम् । प॒ञ्च॒रा॒त्रमिति॑ पञ्च - रा॒त्रम् । अ॒प॒श्य॒न्न् । तम् । एति॑ । अ॒ह॒र॒न्न् । तेन॑ । अ॒य॒ज॒न्त॒ । ततः॑ । वै । ते । व्याव॑र्त॒न्तेति॑ वि - आव॑र्तन्त ।  \newline


\textbf{Krama Paata} \newline

स॒म्ॅव॒थ्स॒रो वै । स॒म्ॅव॒थ्स॒र इति॑ सम् - व॒थ्स॒रः । वा इ॒दम् । इ॒दमेकः॑ । एक॑ आसीत् । आ॒सी॒थ् सः । सो॑ऽकामयत । अ॒का॒म॒य॒त॒र्तून् । ऋ॒तून्थ् सृ॑जेय । सृ॒जे॒येति॑ । इति॒ सः । स ए॒तम् । ए॒तम् प॑ञ्चरा॒त्रम् । प॒ञ्च॒रा॒त्रम॑पश्यत् । प॒ञ्च॒रा॒त्रमिति॑ पञ्च - रा॒त्रम् । अ॒प॒श्य॒त् तम् । तमा । आऽह॑रत् । अ॒ह॒र॒त् तेन॑ । तेना॑यजत । अ॒य॒ज॒त॒ ततः॑ । ततो॒ वै । वै सः । स ऋ॒तून् । ऋ॒तून॑सृजत । अ॒सृ॒ज॒त॒ यः । य ए॒वम् । ए॒वम् ॅवि॒द्वान् । वि॒द्वान् प॑ञ्चरा॒त्रेण॑ । प॒ञ्च॒रा॒त्रेण॒ यज॑ते । प॒ञ्च॒रा॒त्रेणेति॑ पञ्च - रा॒त्रेण॑ । यज॑ते॒ प्र । प्रैव । ए॒व जा॑यते । जा॒य॒ते॒ ते । त ऋ॒तवः॑ । ऋ॒तवः॑ सृ॒ष्टाः । सृ॒ष्टा न । न व्याव॑र्तन्त । व्याव॑र्तन्त॒ ते । व्याव॑र्त॒न्तेति॑ वि - आव॑र्तन्त । त ए॒तम् । ए॒तम् प॑ञ्चरा॒त्रम् । प॒ञ्च॒रा॒त्रम॑पश्यन्न् । प॒ञ्च॒रा॒त्रमिति॑ पञ्च - रा॒त्रम् । अ॒प॒श्य॒न् तम् । तमा । आऽह॑रन्न् । अ॒ह॒र॒न् तेन॑ । तेना॑ऽयजन्त । अ॒य॒ज॒न्त॒ ततः॑ । ततो॒ वै । वै ते । ते व्याव॑र्तन्त । व्याव॑र्तन्त॒ यः । व्याव॑र्त॒न्तेति॑ वि - आव॑र्तन्त \newline

\textbf{Jatai Paata} \newline

1. सं॒ॅव॒थ्स॒रो वै वै सं॑ॅवथ्स॒रः सं॑ॅवथ्स॒रो वै । \newline
2. सं॒ॅव॒थ्स॒र इति॑ सं - व॒थ्स॒रः । \newline
3. वा इ॒द मि॒दं ॅवै वा इ॒दम् । \newline
4. इ॒द मेक॒ एक॑ इ॒द मि॒द मेकः॑ । \newline
5. एक॑ आसी दासी॒ देक॒ एक॑ आसीत् । \newline
6. आ॒सी॒थ् स स आ॑सी दासी॒थ् सः । \newline
7. सो॑ ऽकामयता कामयत॒ स सो॑ ऽकामयत । \newline
8. अ॒का॒म॒य॒त॒ र्‌तू नृ॒तू न॑कामयता कामयत॒ र्‌तून् । \newline
9. ऋ॒तून् थ्सृ॑जेय सृजेय॒ र्‌तू नृ॒तून् थ्सृ॑जेय । \newline
10. सृ॒जे॒येतीति॑ सृजेय सृजे॒येति॑ । \newline
11. इति॒ स स इतीति॒ सः । \newline
12. स ए॒त मे॒तꣳ स स ए॒तम् । \newline
13. ए॒तम् प॑ञ्चरा॒त्रम् प॑ञ्चरा॒त्र मे॒त मे॒तम् प॑ञ्चरा॒त्रम् । \newline
14. प॒ञ्च॒रा॒त्र म॑पश्य दपश्यत् पञ्चरा॒त्रम् प॑ञ्चरा॒त्र म॑पश्यत् । \newline
15. प॒ञ्च॒रा॒त्रमिति॑ पञ्च - रा॒त्रम् । \newline
16. अ॒प॒श्य॒त् तम् त म॑पश्य दपश्य॒त् तम् । \newline
17. त मा तम् त मा । \newline
18. आ ऽह॑र दहर॒दा ऽह॑रत् । \newline
19. अ॒ह॒र॒त् तेन॒ तेना॑ हर दहर॒त् तेन॑ । \newline
20. तेना॑ यजता यजत॒ तेन॒ तेना॑ यजत । \newline
21. अ॒य॒ज॒त॒ तत॒ स्ततो॑ ऽयजता यजत॒ ततः॑ । \newline
22. ततो॒ वै वै तत॒ स्ततो॒ वै । \newline
23. वै स स वै वै सः । \newline
24. स ऋ॒तू नृ॒तून् थ्स स ऋ॒तून् । \newline
25. ऋ॒तू न॑सृजता सृजत॒ र्‌तू नृ॒तू न॑सृजत । \newline
26. अ॒सृ॒ज॒त॒ यो यो॑ ऽसृजता सृजत॒ यः । \newline
27. य ए॒व मे॒वं ॅयो य ए॒वम् । \newline
28. ए॒वं ॅवि॒द्वान्. वि॒द्वा ने॒व मे॒वं ॅवि॒द्वान् । \newline
29. वि॒द्वान् प॑ञ्चरा॒त्रेण॑ पञ्चरा॒त्रेण॑ वि॒द्वान्. वि॒द्वान् प॑ञ्चरा॒त्रेण॑ । \newline
30. प॒ञ्च॒रा॒त्रेण॒ यज॑ते॒ यज॑ते पञ्चरा॒त्रेण॑ पञ्चरा॒त्रेण॒ यज॑ते । \newline
31. प॒ञ्च॒रा॒त्रेणेति॑ पञ्च - रा॒त्रेण॑ । \newline
32. यज॑ते॒ प्र प्र यज॑ते॒ यज॑ते॒ प्र । \newline
33. प्रैवैव प्र प्रैव । \newline
34. ए॒व जा॑यते जायत ए॒वैव जा॑यते । \newline
35. जा॒य॒ते॒ ते ते जा॑यते जायते॒ ते । \newline
36. त ऋ॒तव॑ ऋ॒तव॒ स्ते त ऋ॒तवः॑ । \newline
37. ऋ॒तवः॑ सृ॒ष्टाः सृ॒ष्टा ऋ॒तव॑ ऋ॒तवः॑ सृ॒ष्टाः । \newline
38. सृ॒ष्टा न न सृ॒ष्टाः सृ॒ष्टा न । \newline
39. न व्याव॑र्तन्त॒ व्याव॑र्तन्त॒ न न व्याव॑र्तन्त । \newline
40. व्याव॑र्तन्त॒ ते ते व्याव॑र्तन्त॒ व्याव॑र्तन्त॒ ते । \newline
41. व्याव॑र्त॒न्तेति॑ वि - आव॑र्तन्त । \newline
42. त ए॒त मे॒तम् ते त ए॒तम् । \newline
43. ए॒तम् प॑ञ्चरा॒त्रम् प॑ञ्चरा॒त्र मे॒त मे॒तम् प॑ञ्चरा॒त्रम् । \newline
44. प॒ञ्च॒रा॒त्र म॑पश्यन् नपश्यन् पञ्चरा॒त्रम् प॑ञ्चरा॒त्र म॑पश्यन्न् । \newline
45. प॒ञ्च॒रा॒त्रमिति॑ पञ्च - रा॒त्रम् । \newline
46. अ॒प॒श्य॒न् तम् त म॑पश्यन् नपश्य॒न् तम् । \newline
47. त मा तम् त मा । \newline
48. आ ऽह॑रन् नहर॒न् ना ऽह॑रन्न् । \newline
49. अ॒ह॒र॒न् तेन॒ तेना॑ हरन् नहर॒न् तेन॑ । \newline
50. तेना॑ यजन्ता यजन्त॒ तेन॒ तेना॑ यजन्त । \newline
51. अ॒य॒ज॒न्त॒ तत॒ स्ततो॑ ऽयजन्ता यजन्त॒ ततः॑ । \newline
52. ततो॒ वै वै तत॒ स्ततो॒ वै । \newline
53. वै ते ते वै वै ते । \newline
54. ते व्याव॑र्तन्त॒ व्याव॑र्तन्त॒ ते ते व्याव॑र्तन्त । \newline
55. व्याव॑र्तन्त॒ यो यो व्याव॑र्तन्त॒ व्याव॑र्तन्त॒ यः । \newline
56. व्याव॑र्त॒न्तेति॑ वि - आव॑र्तन्त । \newline

\textbf{Ghana Paata } \newline

1. सं॒ॅव॒थ्स॒रो वै वै सं॑ॅवथ्स॒रः सं॑ॅवथ्स॒रो वा इ॒द मि॒दं ॅवै सं॑ॅवथ्स॒रः सं॑ॅवथ्स॒रो वा इ॒दम् । \newline
2. सं॒ॅव॒थ्स॒र इति॑ सं - व॒थ्स॒रः । \newline
3. वा इ॒द मि॒दं ॅवै वा इ॒द मेक॒ एक॑ इ॒दं ॅवै वा इ॒द मेकः॑ । \newline
4. इ॒द मेक॒ एक॑ इ॒द मि॒द मेक॑ आसी दासी॒ देक॑ इ॒द मि॒द मेक॑ आसीत् । \newline
5. एक॑ आसी दासी॒ देक॒ एक॑ आसी॒थ् स स आ॑सी॒ देक॒ एक॑ आसी॒थ् सः । \newline
6. आ॒सी॒थ् स स आ॑सी दासी॒थ् सो॑ ऽकामयता कामयत॒ स आ॑सी दासी॒थ् सो॑ ऽकामयत । \newline
7. सो॑ ऽकामयता कामयत॒ स सो॑ ऽकामयत॒ र्‌तू नृ॒तू न॑कामयत॒ स सो॑ ऽकामयत॒ र्‌तून् । \newline
8. अ॒का॒म॒य॒त॒ र्‌तू नृ॒तू न॑कामयता कामयत॒ र्‌तून् थ्सृ॑जेय सृजेय॒ र्‌तू न॑कामयता कामयत॒ र्‌तून् थ्सृ॑जेय । \newline
9. ऋ॒तून् थ्सृ॑जेय सृजेय॒ र्‌तू नृ॒तून् थ्सृ॑जे॒येतीति॑ सृजेय॒ र्‌तू नृ॒तून् थ्सृ॑जे॒येति॑ । \newline
10. सृ॒जे॒येतीति॑ सृजेय सृजे॒येति॒ स स इति॑ सृजेय सृजे॒येति॒ सः । \newline
11. इति॒ स स इतीति॒ स ए॒त मे॒तꣳ स इतीति॒ स ए॒तम् । \newline
12. स ए॒त मे॒तꣳ स स ए॒तम् प॑ञ्चरा॒त्रम् प॑ञ्चरा॒त्र मे॒तꣳ स स ए॒तम् प॑ञ्चरा॒त्रम् । \newline
13. ए॒तम् प॑ञ्चरा॒त्रम् प॑ञ्चरा॒त्र मे॒त मे॒तम् प॑ञ्चरा॒त्र म॑पश्य दपश्यत् पञ्चरा॒त्र मे॒त मे॒तम् प॑ञ्चरा॒त्र म॑पश्यत् । \newline
14. प॒ञ्च॒रा॒त्र म॑पश्य दपश्यत् पञ्चरा॒त्रम् प॑ञ्चरा॒त्र म॑पश्य॒त् तम् त म॑पश्यत् पञ्चरा॒त्रम् प॑ञ्चरा॒त्र म॑पश्य॒त् तम् । \newline
15. प॒ञ्च॒रा॒त्रमिति॑ पञ्च - रा॒त्रम् । \newline
16. अ॒प॒श्य॒त् तम् त म॑पश्य दपश्य॒त् त मा त म॑पश्य दपश्य॒त् त मा । \newline
17. त मा तम् त मा ऽह॑र दहर॒दा तम् त मा ऽह॑रत् । \newline
18. आ ऽह॑र दहर॒दा ऽह॑र॒त् तेन॒ तेना॑ हर॒दा ऽह॑र॒त् तेन॑ । \newline
19. अ॒ह॒र॒त् तेन॒ तेना॑ हर दहर॒त् तेना॑ यजता यजत॒ तेना॑ हर दहर॒त् तेना॑ यजत । \newline
20. तेना॑ यजता यजत॒ तेन॒ तेना॑ यजत॒ तत॒ स्ततो॑ ऽयजत॒ तेन॒ तेना॑ यजत॒ ततः॑ । \newline
21. अ॒य॒ज॒त॒ तत॒ स्ततो॑ ऽयजता यजत॒ ततो॒ वै वै ततो॑ ऽयजता यजत॒ ततो॒ वै । \newline
22. ततो॒ वै वै तत॒ स्ततो॒ वै स स वै तत॒ स्ततो॒ वै सः । \newline
23. वै स स वै वै स ऋ॒तू नृ॒तून् थ्स वै वै स ऋ॒तून् । \newline
24. स ऋ॒तू नृ॒तून् थ्स स ऋ॒तू न॑सृजता सृजत॒ र्‌तून् थ्स स ऋ॒तू न॑सृजत । \newline
25. ऋ॒तू न॑सृजता सृजत॒ र्‌तू नृ॒तू न॑सृजत॒ यो यो॑ ऽसृजत॒ र्‌तू नृ॒तू न॑सृजत॒ यः । \newline
26. अ॒सृ॒ज॒त॒ यो यो॑ ऽसृजता सृजत॒ य ए॒व मे॒वं ॅयो॑ ऽसृजता सृजत॒ य ए॒वम् । \newline
27. य ए॒व मे॒वं ॅयो य ए॒वं ॅवि॒द्वान्. वि॒द्वा ने॒वं ॅयो य ए॒वं ॅवि॒द्वान् । \newline
28. ए॒वं ॅवि॒द्वान्. वि॒द्वा ने॒व मे॒वं ॅवि॒द्वान् प॑ञ्चरा॒त्रेण॑ पञ्चरा॒त्रेण॑ वि॒द्वा ने॒व मे॒वं ॅवि॒द्वान् प॑ञ्चरा॒त्रेण॑ । \newline
29. वि॒द्वान् प॑ञ्चरा॒त्रेण॑ पञ्चरा॒त्रेण॑ वि॒द्वान्. वि॒द्वान् प॑ञ्चरा॒त्रेण॒ यज॑ते॒ यज॑ते पञ्चरा॒त्रेण॑ वि॒द्वान्. वि॒द्वान् प॑ञ्चरा॒त्रेण॒ यज॑ते । \newline
30. प॒ञ्च॒रा॒त्रेण॒ यज॑ते॒ यज॑ते पञ्चरा॒त्रेण॑ पञ्चरा॒त्रेण॒ यज॑ते॒ प्र प्र यज॑ते पञ्चरा॒त्रेण॑ पञ्चरा॒त्रेण॒ यज॑ते॒ प्र । \newline
31. प॒ञ्च॒रा॒त्रेणेति॑ पञ्च - रा॒त्रेण॑ । \newline
32. यज॑ते॒ प्र प्र यज॑ते॒ यज॑ते॒ प्रैवैव प्र यज॑ते॒ यज॑ते॒ प्रैव । \newline
33. प्रैवैव प्र प्रैव जा॑यते जायत ए॒व प्र प्रैव जा॑यते । \newline
34. ए॒व जा॑यते जायत ए॒वैव जा॑यते॒ ते ते जा॑यत ए॒वैव जा॑यते॒ ते । \newline
35. जा॒य॒ते॒ ते ते जा॑यते जायते॒ त ऋ॒तव॑ ऋ॒तव॒ स्ते जा॑यते जायते॒ त ऋ॒तवः॑ । \newline
36. त ऋ॒तव॑ ऋ॒तव॒ स्ते त ऋ॒तवः॑ सृ॒ष्टाः सृ॒ष्टा ऋ॒तव॒ स्ते त ऋ॒तवः॑ सृ॒ष्टाः । \newline
37. ऋ॒तवः॑ सृ॒ष्टाः सृ॒ष्टा ऋ॒तव॑ ऋ॒तवः॑ सृ॒ष्टा न न सृ॒ष्टा ऋ॒तव॑ ऋ॒तवः॑ सृ॒ष्टा न । \newline
38. सृ॒ष्टा न न सृ॒ष्टाः सृ॒ष्टा न व्याव॑र्तन्त॒ व्याव॑र्तन्त॒ न सृ॒ष्टाः सृ॒ष्टा न व्याव॑र्तन्त । \newline
39. न व्याव॑र्तन्त॒ व्याव॑र्तन्त॒ न न व्याव॑र्तन्त॒ ते ते व्याव॑र्तन्त॒ न न व्याव॑र्तन्त॒ ते । \newline
40. व्याव॑र्तन्त॒ ते ते व्याव॑र्तन्त॒ व्याव॑र्तन्त॒ त ए॒त मे॒तम् ते व्याव॑र्तन्त॒ व्याव॑र्तन्त॒ त ए॒तम् । \newline
41. व्याव॑र्त॒न्तेति॑ वि - आव॑र्तन्त । \newline
42. त ए॒त मे॒तम् ते त ए॒तम् प॑ञ्चरा॒त्रम् प॑ञ्चरा॒त्र मे॒तम् ते त ए॒तम् प॑ञ्चरा॒त्रम् । \newline
43. ए॒तम् प॑ञ्चरा॒त्रम् प॑ञ्चरा॒त्र मे॒त मे॒तम् प॑ञ्चरा॒त्र म॑पश्यन् नपश्यन् पञ्चरा॒त्र मे॒त मे॒तम् प॑ञ्चरा॒त्र म॑पश्यन्न् । \newline
44. प॒ञ्च॒रा॒त्र म॑पश्यन् नपश्यन् पञ्चरा॒त्रम् प॑ञ्चरा॒त्र म॑पश्य॒न् तम् त म॑पश्यन् पञ्चरा॒त्रम् प॑ञ्चरा॒त्र म॑पश्य॒न् तम् । \newline
45. प॒ञ्च॒रा॒त्रमिति॑ पञ्च - रा॒त्रम् । \newline
46. अ॒प॒श्य॒न् तम् त म॑पश्यन् नपश्य॒न् त मा त म॑पश्यन् नपश्य॒न् त मा । \newline
47. त मा तम् त मा ऽह॑रन् नहर॒न् ना तम् त मा ऽह॑रन्न् । \newline
48. आ ऽह॑रन् नहर॒न् ना ऽह॑र॒न् तेन॒ तेना॑ हर॒न् ना ऽह॑र॒न् तेन॑ । \newline
49. अ॒ह॒र॒न् तेन॒ तेना॑ हरन् नहर॒न् तेना॑ यजन्ता यजन्त॒ तेना॑ हरन् नहर॒न् तेना॑ यजन्त । \newline
50. तेना॑ यजन्ता यजन्त॒ तेन॒ तेना॑ यजन्त॒ तत॒ स्ततो॑ ऽयजन्त॒ तेन॒ तेना॑ यजन्त॒ ततः॑ । \newline
51. अ॒य॒ज॒न्त॒ तत॒ स्ततो॑ ऽयजन्ता यजन्त॒ ततो॒ वै वै ततो॑ ऽयजन्ता यजन्त॒ ततो॒ वै । \newline
52. ततो॒ वै वै तत॒ स्ततो॒ वै ते ते वै तत॒ स्ततो॒ वै ते । \newline
53. वै ते ते वै वै ते व्याव॑र्तन्त॒ व्याव॑र्तन्त॒ ते वै वै ते व्याव॑र्तन्त । \newline
54. ते व्याव॑र्तन्त॒ व्याव॑र्तन्त॒ ते ते व्याव॑र्तन्त॒ यो यो व्याव॑र्तन्त॒ ते ते व्याव॑र्तन्त॒ यः । \newline
55. व्याव॑र्तन्त॒ यो यो व्याव॑र्तन्त॒ व्याव॑र्तन्त॒ य ए॒व मे॒वं ॅयो व्याव॑र्तन्त॒ व्याव॑र्तन्त॒ य ए॒वम् । \newline
56. व्याव॑र्त॒न्तेति॑ वि - आव॑र्तन्त । \newline
\pagebreak
\markright{ TS 7.1.10.2  \hfill https://www.vedavms.in \hfill}

\section{ TS 7.1.10.2 }

\textbf{TS 7.1.10.2 } \newline
\textbf{Samhita Paata} \newline

य ए॒वं ॅवि॒द्वान् प॑ञ्चरा॒त्रेण॒ यज॑ते॒ वि पा॒प्मना॒ भ्रातृ॑व्ये॒णाऽऽ* व॑र्तते॒ सार्व॑सेनिः शौचे॒यो॑ऽकामयत पशु॒मान्थ् स्या॒मिति॒ स ए॒तं प॑ञ्चरा॒त्रमाऽह॑र॒त् तेना॑ऽयजत॒ ततो॒ वै स स॒हस्रं॑ प॒शून् प्राऽऽ*प्नो॒द्य ए॒वं ॅवि॒द्वान् प॑ञ्चरा॒त्रेण॒ यज॑ते॒ प्र स॒हस्रं॑ प॒शूना᳚प्नोति बब॒रः प्रावा॑हणि-रकामयत वा॒चः प्र॑वदि॒ता स्या॒मिति॒ स ए॒तं प॑ञ्चरा॒त्रमा - [  ] \newline

\textbf{Pada Paata} \newline

यः । ए॒वम् । वि॒द्वान् । प॒ञ्च॒रा॒त्रेणेति॑ पञ्च-रा॒त्रेण॑ । यज॑ते । वीति॑ । पा॒प्मना᳚ । भ्रातृ॑व्येण । एति॑ । व॒र्त॒ते॒ । सार्व॑सेनि॒रिति॒ सार्व॑ - से॒निः॒ । शौ॒चे॒यः । अ॒का॒म॒य॒त॒ । प॒शु॒मानिति॑ पशु - मान् । स्या॒म् । इति॑ । सः । ए॒तम् । प॒ञ्च॒रा॒त्रमिति॑ पञ्च - रा॒त्रम् । एति॑ । अ॒ह॒र॒त् । तेन॑ । अ॒य॒ज॒त॒ । ततः॑ । वै । सः । स॒हस्र᳚म् । प॒शून् । प्रेति॑ । आ॒प्नो॒त् । यः । ए॒वम् । वि॒द्वान् । प॒ञ्च॒रा॒त्रेणेति॑ पञ्च-रा॒त्रेण॑ । यज॑ते । प्रेति॑ । स॒हस्र᳚म् । प॒शून् । आ॒प्नो॒ति॒ । ब॒ब॒रः । प्रावा॑हणिः । अ॒का॒म॒य॒त॒ । वा॒चः । प्र॒व॒दि॒तेति॑ प्र - व॒दि॒ता । स्या॒म् । इति॑ । सः । ए॒तम् । प॒ञ्च॒रा॒त्रमिति॑ पञ्च - रा॒त्रम् । एति॑ ।  \newline


\textbf{Krama Paata} \newline

य ए॒वम् । ए॒वम् ॅवि॒द्वान् । वि॒द्वान् प॑ञ्चरा॒त्रेण॑ । प॒ञ्च॒रा॒त्रेण॒ यज॑ते । प॒ञ्च॒रा॒त्रेणेति॑ पञ्च - रा॒त्रेण॑ । यज॑ते॒ वि । वि पा॒प्मना᳚ । पा॒प्मना॒ भ्रातृ॑व्येण । भ्रातृ॑व्ये॒णा । आ व॑र्तते । व॒र्त॒ते॒ सार्व॑सेनिः । सार्व॑सेनिः शौचे॒यः । सार्व॑सेनि॒रिति॒ सार्व॑ - से॒निः॒ । शौ॒चे॒यो॑ऽकामयत । अ॒का॒म॒य॒त॒ प॒शु॒मान् । प॒शु॒मान्थ् स्या᳚म् । प॒शु॒मानिति॑ पशु - मान् । स्या॒मिति॑ । इति॒ सः । स ए॒तम् । ए॒तम् प॑ञ्चरा॒त्रम् । प॒ञ्च॒रा॒त्रमा । प॒ञ्च॒रा॒त्रमिति॑ पञ्च - रा॒त्रम् । आऽह॑रत् । अ॒ह॒र॒त् तेन॑ । तेना॑यजत । अ॒य॒ज॒त॒ ततः॑ । ततो॒ वै । वै सः । स स॒हस्र᳚म् । स॒हस्र॑म् प॒शून् । प॒शून् प्र । प्राप्नो᳚त् । आ॒प्नो॒द् यः । य ए॒वम् । ए॒वम् ॅवि॒द्वान् । वि॒द्वान् प॑ञ्चरा॒त्रेण॑ । प॒ञ्च॒रा॒त्रेण॒ यज॑ते । प॒ञ्च॒रा॒त्रेणेति॑ पञ्च - रा॒त्रेण॑ । यज॑ते॒ प्र । प्र स॒हस्र᳚म् । स॒हस्र॑म् प॒शून् । प॒शूना᳚प्नोति । आ॒प्नो॒ति॒ ब॒ब॒रः । ब॒ब॒रः प्रावा॑हणिः । प्रावा॑हणिरकामयत । अ॒का॒म॒य॒त॒ वा॒चः । वा॒चः प्र॑वदि॒ता । प्र॒व॒दि॒ता स्या᳚म् । प्र॒व॒दि॒तेति॑ प्र - व॒दि॒ता । स्या॒मिति॑ । इति॒ सः । स ए॒तम् । ए॒तम् प॑ञ्चरा॒त्रम् । प॒ञ्च॒रा॒त्रमा । प॒ञ्च॒रा॒त्रमिति॑ पञ्च - रा॒त्रम् । आऽह॑रत् \newline

\textbf{Jatai Paata} \newline

1. य ए॒व मे॒वं ॅयो य ए॒वम् । \newline
2. ए॒वं ॅवि॒द्वान्. वि॒द्वा ने॒व मे॒वं ॅवि॒द्वान् । \newline
3. वि॒द्वान् प॑ञ्चरा॒त्रेण॑ पञ्चरा॒त्रेण॑ वि॒द्वान्. वि॒द्वान् प॑ञ्चरा॒त्रेण॑ । \newline
4. प॒ञ्च॒रा॒त्रेण॒ यज॑ते॒ यज॑ते पञ्चरा॒त्रेण॑ पञ्चरा॒त्रेण॒ यज॑ते । \newline
5. प॒ञ्च॒रा॒त्रेणेति॑ पञ्च - रा॒त्रेण॑ । \newline
6. यज॑ते॒ वि वि यज॑ते॒ यज॑ते॒ वि । \newline
7. वि पा॒प्मना॑ पा॒प्मना॒ वि वि पा॒प्मना᳚ । \newline
8. पा॒प्मना॒ भ्रातृ॑व्येण॒ भ्रातृ॑व्येण पा॒प्मना॑ पा॒प्मना॒ भ्रातृ॑व्येण । \newline
9. भ्रातृ॑व्ये॒णा भ्रातृ॑व्येण॒ भ्रातृ॑व्ये॒णा । \newline
10. आ व॑र्तते वर्तत॒ आ व॑र्तते । \newline
11. व॒र्त॒ते॒ सार्व॑सेनिः॒ सार्व॑सेनिर् वर्तते वर्तते॒ सार्व॑सेनिः । \newline
12. सार्व॑सेनिः शौचे॒यः शौ॑चे॒यः सार्व॑सेनिः॒ सार्व॑सेनिः शौचे॒यः । \newline
13. सार्व॑सेनि॒रिति॒ सार्व॑ - से॒निः॒ । \newline
14. शौ॒चे॒यो॑ ऽकामयता कामयत शौचे॒यः शौ॑चे॒यो॑ ऽकामयत । \newline
15. अ॒का॒म॒य॒त॒ प॒शु॒मान् प॑शु॒मा न॑कामयता कामयत पशु॒मान् । \newline
16. प॒शु॒मान् थ्स्याꣳ॑ स्याम् पशु॒मान् प॑शु॒मान् थ्स्या᳚म् । \newline
17. प॒शु॒मानिति॑ पशु - मान् । \newline
18. स्या॒ मितीति॑ स्याꣳ स्या॒ मिति॑ । \newline
19. इति॒ स स इतीति॒ सः । \newline
20. स ए॒त मे॒तꣳ स स ए॒तम् । \newline
21. ए॒तम् प॑ञ्चरा॒त्रम् प॑ञ्चरा॒त्र मे॒त मे॒तम् प॑ञ्चरा॒त्रम् । \newline
22. प॒ञ्च॒रा॒त्र मा प॑ञ्चरा॒त्रम् प॑ञ्चरा॒त्र मा । \newline
23. प॒ञ्च॒रा॒त्रमिति॑ पञ्च - रा॒त्रम् । \newline
24. आ ऽह॑र दहर॒दा ऽह॑रत् । \newline
25. अ॒ह॒र॒त् तेन॒ तेना॑ हर दहर॒त् तेन॑ । \newline
26. तेना॑ यजता यजत॒ तेन॒ तेना॑ यजत । \newline
27. अ॒य॒ज॒त॒ तत॒ स्ततो॑ ऽयजता यजत॒ ततः॑ । \newline
28. ततो॒ वै वै तत॒ स्ततो॒ वै । \newline
29. वै स स वै वै सः । \newline
30. स स॒हस्रꣳ॑ स॒हस्रꣳ॒॒ स स स॒हस्र᳚म् । \newline
31. स॒हस्र॑म् प॒शून् प॒शून् थ्स॒हस्रꣳ॑ स॒हस्र॑म् प॒शून् । \newline
32. प॒शून् प्र प्र प॒शून् प॒शून् प्र । \newline
33. प्राप्नो॑ दाप्नो॒त् प्र प्राप्नो᳚त् । \newline
34. आ॒प्नो॒द् यो य आ᳚प्नो दाप्नो॒द् यः । \newline
35. य ए॒व मे॒वं ॅयो य ए॒वम् । \newline
36. ए॒वं ॅवि॒द्वान्. वि॒द्वा ने॒व मे॒वं ॅवि॒द्वान् । \newline
37. वि॒द्वान् प॑ञ्चरा॒त्रेण॑ पञ्चरा॒त्रेण॑ वि॒द्वान्. वि॒द्वान् प॑ञ्चरा॒त्रेण॑ । \newline
38. प॒ञ्च॒रा॒त्रेण॒ यज॑ते॒ यज॑ते पञ्चरा॒त्रेण॑ पञ्चरा॒त्रेण॒ यज॑ते । \newline
39. प॒ञ्च॒रा॒त्रेणेति॑ पञ्च - रा॒त्रेण॑ । \newline
40. यज॑ते॒ प्र प्र यज॑ते॒ यज॑ते॒ प्र । \newline
41. प्र स॒हस्रꣳ॑ स॒हस्र॒म् प्र प्र स॒हस्र᳚म् । \newline
42. स॒हस्र॑म् प॒शून् प॒शून् थ्स॒हस्रꣳ॑ स॒हस्र॑म् प॒शून् । \newline
43. प॒शू ना᳚प्नो त्याप्नोति प॒शून् प॒शू ना᳚प्नोति । \newline
44. आ॒प्नो॒ति॒ ब॒ब॒रो ब॑ब॒र आ᳚प्नो त्याप्नोति बब॒रः । \newline
45. ब॒ब॒रः प्रावा॑हणिः॒ प्रावा॑हणिर् बब॒रो ब॑ब॒रः प्रावा॑हणिः । \newline
46. प्रावा॑हणि रकामयता कामयत॒ प्रावा॑हणिः॒ प्रावा॑हणि रकामयत । \newline
47. अ॒का॒म॒य॒त॒ वा॒चो वा॒चो॑ ऽकामयता कामयत वा॒चः । \newline
48. वा॒चः प्र॑वदि॒ता प्र॑वदि॒ता वा॒चो वा॒चः प्र॑वदि॒ता । \newline
49. प्र॒व॒दि॒ता स्याꣳ॑ स्याम् प्रवदि॒ता प्र॑वदि॒ता स्या᳚म् । \newline
50. प्र॒व॒दि॒तेति॑ प्र - व॒दि॒ता । \newline
51. स्या॒ मितीति॑ स्याꣳ स्या॒ मिति॑ । \newline
52. इति॒ स स इतीति॒ सः । \newline
53. स ए॒त मे॒तꣳ स स ए॒तम् । \newline
54. ए॒तम् प॑ञ्चरा॒त्रम् प॑ञ्चरा॒त्र मे॒त मे॒तम् प॑ञ्चरा॒त्रम् । \newline
55. प॒ञ्च॒रा॒त्र मा प॑ञ्चरा॒त्रम् प॑ञ्चरा॒त्र मा । \newline
56. प॒ञ्च॒रा॒त्रमिति॑ पञ्च - रा॒त्रम् । \newline
57. आ ऽह॑र दहर॒दा ऽह॑रत् । \newline

\textbf{Ghana Paata } \newline

1. य ए॒व मे॒वं ॅयो य ए॒वं ॅवि॒द्वान्. वि॒द्वा ने॒वं ॅयो य ए॒वं ॅवि॒द्वान् । \newline
2. ए॒वं ॅवि॒द्वान्. वि॒द्वा ने॒व मे॒वं ॅवि॒द्वान् प॑ञ्चरा॒त्रेण॑ पञ्चरा॒त्रेण॑ वि॒द्वा ने॒व मे॒वं ॅवि॒द्वान् प॑ञ्चरा॒त्रेण॑ । \newline
3. वि॒द्वान् प॑ञ्चरा॒त्रेण॑ पञ्चरा॒त्रेण॑ वि॒द्वान्. वि॒द्वान् प॑ञ्चरा॒त्रेण॒ यज॑ते॒ यज॑ते पञ्चरा॒त्रेण॑ वि॒द्वान्. वि॒द्वान् प॑ञ्चरा॒त्रेण॒ यज॑ते । \newline
4. प॒ञ्च॒रा॒त्रेण॒ यज॑ते॒ यज॑ते पञ्चरा॒त्रेण॑ पञ्चरा॒त्रेण॒ यज॑ते॒ वि वि यज॑ते पञ्चरा॒त्रेण॑ पञ्चरा॒त्रेण॒ यज॑ते॒ वि । \newline
5. प॒ञ्च॒रा॒त्रेणेति॑ पञ्च - रा॒त्रेण॑ । \newline
6. यज॑ते॒ वि वि यज॑ते॒ यज॑ते॒ वि पा॒प्मना॑ पा॒प्मना॒ वि यज॑ते॒ यज॑ते॒ वि पा॒प्मना᳚ । \newline
7. वि पा॒प्मना॑ पा॒प्मना॒ वि वि पा॒प्मना॒ भ्रातृ॑व्येण॒ भ्रातृ॑व्येण पा॒प्मना॒ वि वि पा॒प्मना॒ भ्रातृ॑व्येण । \newline
8. पा॒प्मना॒ भ्रातृ॑व्येण॒ भ्रातृ॑व्येण पा॒प्मना॑ पा॒प्मना॒ भ्रातृ॑व्ये॒णा भ्रातृ॑व्येण पा॒प्मना॑ पा॒प्मना॒ भ्रातृ॑व्ये॒णा । \newline
9. भ्रातृ॑व्ये॒णा भ्रातृ॑व्येण॒ भ्रातृ॑व्ये॒णा व॑र्तते वर्तत॒ आ भ्रातृ॑व्येण॒ भ्रातृ॑व्ये॒णा व॑र्तते । \newline
10. आ व॑र्तते वर्तत॒ आ व॑र्तते॒ सार्व॑सेनिः॒ सार्व॑सेनिर् वर्तत॒ आ व॑र्तते॒ सार्व॑सेनिः । \newline
11. व॒र्त॒ते॒ सार्व॑सेनिः॒ सार्व॑सेनिर् वर्तते वर्तते॒ सार्व॑सेनिः शौचे॒यः शौ॑चे॒यः सार्व॑सेनिर् वर्तते वर्तते॒ सार्व॑सेनिः शौचे॒यः । \newline
12. सार्व॑सेनिः शौचे॒यः शौ॑चे॒यः सार्व॑सेनिः॒ सार्व॑सेनिः शौचे॒यो॑ ऽकामयता कामयत शौचे॒यः सार्व॑सेनिः॒ सार्व॑सेनिः शौचे॒यो॑ ऽकामयत । \newline
13. सार्व॑सेनि॒रिति॒ सार्व॑ - से॒निः॒ । \newline
14. शौ॒चे॒यो॑ ऽकामयता कामयत शौचे॒यः शौ॑चे॒यो॑ ऽकामयत पशु॒मान् प॑शु॒मा न॑कामयत शौचे॒यः शौ॑चे॒यो॑ ऽकामयत पशु॒मान् । \newline
15. अ॒का॒म॒य॒त॒ प॒शु॒मान् प॑शु॒मा न॑कामयता कामयत पशु॒मान् थ्स्याꣳ॑ स्याम् पशु॒मा न॑कामयता कामयत पशु॒मान् थ्स्या᳚म् । \newline
16. प॒शु॒मान् थ्स्याꣳ॑ स्याम् पशु॒मान् प॑शु॒मान् थ्स्या॒ मितीति॑ स्याम् पशु॒मान् प॑शु॒मान् थ्स्या॒ मिति॑ । \newline
17. प॒शु॒मानिति॑ पशु - मान् । \newline
18. स्या॒ मितीति॑ स्याꣳ स्या॒ मिति॒ स स इति॑ स्याꣳ स्या॒ मिति॒ सः । \newline
19. इति॒ स स इतीति॒ स ए॒त मे॒तꣳ स इतीति॒ स ए॒तम् । \newline
20. स ए॒त मे॒तꣳ स स ए॒तम् प॑ञ्चरा॒त्रम् प॑ञ्चरा॒त्र मे॒तꣳ स स ए॒तम् प॑ञ्चरा॒त्रम् । \newline
21. ए॒तम् प॑ञ्चरा॒त्रम् प॑ञ्चरा॒त्र मे॒त मे॒तम् प॑ञ्चरा॒त्र मा प॑ञ्चरा॒त्र मे॒त मे॒तम् प॑ञ्चरा॒त्र मा । \newline
22. प॒ञ्च॒रा॒त्र मा प॑ञ्चरा॒त्रम् प॑ञ्चरा॒त्र मा ऽह॑र दहर॒दा प॑ञ्चरा॒त्रम् प॑ञ्चरा॒त्र मा ऽह॑रत् । \newline
23. प॒ञ्च॒रा॒त्रमिति॑ पञ्च - रा॒त्रम् । \newline
24. आ ऽह॑र दहर॒दा ऽह॑र॒त् तेन॒ तेना॑ हर॒दा ऽह॑र॒त् तेन॑ । \newline
25. अ॒ह॒र॒त् तेन॒ तेना॑ हर दहर॒त् तेना॑ यजता यजत॒ तेना॑ हर दहर॒त् तेना॑ यजत । \newline
26. तेना॑ यजता यजत॒ तेन॒ तेना॑ यजत॒ तत॒ स्ततो॑ ऽयजत॒ तेन॒ तेना॑ यजत॒ ततः॑ । \newline
27. अ॒य॒ज॒त॒ तत॒ स्ततो॑ ऽयजता यजत॒ ततो॒ वै वै ततो॑ ऽयजता यजत॒ ततो॒ वै । \newline
28. ततो॒ वै वै तत॒ स्ततो॒ वै स स वै तत॒ स्ततो॒ वै सः । \newline
29. वै स स वै वै स स॒हस्रꣳ॑ स॒हस्रꣳ॒॒ स वै वै स स॒हस्र᳚म् । \newline
30. स स॒हस्रꣳ॑ स॒हस्रꣳ॒॒ स स स॒हस्र॑म् प॒शून् प॒शून् थ्स॒हस्रꣳ॒॒ स स स॒हस्र॑म् प॒शून् । \newline
31. स॒हस्र॑म् प॒शून् प॒शून् थ्स॒हस्रꣳ॑ स॒हस्र॑म् प॒शून् प्र प्र प॒शून् थ्स॒हस्रꣳ॑ स॒हस्र॑म् प॒शून् प्र । \newline
32. प॒शून् प्र प्र प॒शून् प॒शून् प्राप्नो॑ दाप्नो॒त् प्र प॒शून् प॒शून् प्राप्नो᳚त् । \newline
33. प्राप्नो॑ दाप्नो॒त् प्र प्राप्नो॒द् यो य आ᳚प्नो॒त् प्र प्राप्नो॒द् यः । \newline
34. आ॒प्नो॒द् यो य आ᳚प्नो दाप्नो॒द् य ए॒व मे॒वं ॅय आ᳚प्नो दाप्नो॒द् य ए॒वम् । \newline
35. य ए॒व मे॒वं ॅयो य ए॒वं ॅवि॒द्वान्. वि॒द्वा ने॒वं ॅयो य ए॒वं ॅवि॒द्वान् । \newline
36. ए॒वं ॅवि॒द्वान्. वि॒द्वा ने॒व मे॒वं ॅवि॒द्वान् प॑ञ्चरा॒त्रेण॑ पञ्चरा॒त्रेण॑ वि॒द्वा ने॒व मे॒वं ॅवि॒द्वान् प॑ञ्चरा॒त्रेण॑ । \newline
37. वि॒द्वान् प॑ञ्चरा॒त्रेण॑ पञ्चरा॒त्रेण॑ वि॒द्वान्. वि॒द्वान् प॑ञ्चरा॒त्रेण॒ यज॑ते॒ यज॑ते पञ्चरा॒त्रेण॑ वि॒द्वान्. वि॒द्वान् प॑ञ्चरा॒त्रेण॒ यज॑ते । \newline
38. प॒ञ्च॒रा॒त्रेण॒ यज॑ते॒ यज॑ते पञ्चरा॒त्रेण॑ पञ्चरा॒त्रेण॒ यज॑ते॒ प्र प्र यज॑ते पञ्चरा॒त्रेण॑ पञ्चरा॒त्रेण॒ यज॑ते॒ प्र । \newline
39. प॒ञ्च॒रा॒त्रेणेति॑ पञ्च - रा॒त्रेण॑ । \newline
40. यज॑ते॒ प्र प्र यज॑ते॒ यज॑ते॒ प्र स॒हस्रꣳ॑ स॒हस्र॒म् प्र यज॑ते॒ यज॑ते॒ प्र स॒हस्र᳚म् । \newline
41. प्र स॒हस्रꣳ॑ स॒हस्र॒म् प्र प्र स॒हस्र॑म् प॒शून् प॒शून् थ्स॒हस्र॒म् प्र प्र स॒हस्र॑म् प॒शून् । \newline
42. स॒हस्र॑म् प॒शून् प॒शून् थ्स॒हस्रꣳ॑ स॒हस्र॑म् प॒शू ना᳚प्नो त्याप्नोति प॒शून् थ्स॒हस्रꣳ॑ स॒हस्र॑म् प॒शू ना᳚प्नोति । \newline
43. प॒शू ना᳚प्नो त्याप्नोति प॒शून् प॒शू ना᳚प्नोति बब॒रो ब॑ब॒र आ᳚प्नोति प॒शून् प॒शू ना᳚प्नोति बब॒रः । \newline
44. आ॒प्नो॒ति॒ ब॒ब॒रो ब॑ब॒र आ᳚प्नो त्याप्नोति बब॒रः प्रावा॑हणिः॒ प्रावा॑हणिर् बब॒र आ᳚प्नो त्याप्नोति बब॒रः प्रावा॑हणिः । \newline
45. ब॒ब॒रः प्रावा॑हणिः॒ प्रावा॑हणिर् बब॒रो ब॑ब॒रः प्रावा॑हणि रकामयता कामयत॒ प्रावा॑हणिर् बब॒रो ब॑ब॒रः प्रावा॑हणि रकामयत । \newline
46. प्रावा॑हणि रकामयता कामयत॒ प्रावा॑हणिः॒ प्रावा॑हणि रकामयत वा॒चो वा॒चो॑ ऽकामयत॒ प्रावा॑हणिः॒ प्रावा॑हणि रकामयत वा॒चः । \newline
47. अ॒का॒म॒य॒त॒ वा॒चो वा॒चो॑ ऽकामयता कामयत वा॒चः प्र॑वदि॒ता प्र॑वदि॒ता वा॒चो॑ ऽकामयता कामयत वा॒चः प्र॑वदि॒ता । \newline
48. वा॒चः प्र॑वदि॒ता प्र॑वदि॒ता वा॒चो वा॒चः प्र॑वदि॒ता स्याꣳ॑ स्याम् प्रवदि॒ता वा॒चो वा॒चः प्र॑वदि॒ता स्या᳚म् । \newline
49. प्र॒व॒दि॒ता स्याꣳ॑ स्याम् प्रवदि॒ता प्र॑वदि॒ता स्या॒ मितीति॑ स्याम् प्रवदि॒ता प्र॑वदि॒ता स्या॒ मिति॑ । \newline
50. प्र॒व॒दि॒तेति॑ प्र - व॒दि॒ता । \newline
51. स्या॒ मितीति॑ स्याꣳ स्या॒ मिति॒ स स इति॑ स्याꣳ स्या॒ मिति॒ सः । \newline
52. इति॒ स स इतीति॒ स ए॒त मे॒तꣳ स इतीति॒ स ए॒तम् । \newline
53. स ए॒त मे॒तꣳ स स ए॒तम् प॑ञ्चरा॒त्रम् प॑ञ्चरा॒त्र मे॒तꣳ स स ए॒तम् प॑ञ्चरा॒त्रम् । \newline
54. ए॒तम् प॑ञ्चरा॒त्रम् प॑ञ्चरा॒त्र मे॒त मे॒तम् प॑ञ्चरा॒त्र मा प॑ञ्चरा॒त्र मे॒त मे॒तम् प॑ञ्चरा॒त्र मा । \newline
55. प॒ञ्च॒रा॒त्र मा प॑ञ्चरा॒त्रम् प॑ञ्चरा॒त्र मा ऽह॑र दहर॒दा प॑ञ्चरा॒त्रम् प॑ञ्चरा॒त्र मा ऽह॑रत् । \newline
56. प॒ञ्च॒रा॒त्रमिति॑ पञ्च - रा॒त्रम् । \newline
57. आ ऽह॑र दहर॒दा ऽह॑र॒त् तेन॒ तेना॑ हर॒दा ऽह॑र॒त् तेन॑ । \newline
\pagebreak
\markright{ TS 7.1.10.3  \hfill https://www.vedavms.in \hfill}

\section{ TS 7.1.10.3 }

\textbf{TS 7.1.10.3 } \newline
\textbf{Samhita Paata} \newline

ऽह॑र॒त् तेना॑यजत॒ ततो॒ वै स वा॒चः प्र॑वदि॒ताऽभ॑व॒द्य ए॒वं ॅवि॒द्वान् प॑ञ्चरा॒त्रेण॒ यज॑ते प्रवदि॒तैव वा॒चो भ॑व॒त्यथो॑ एनं ॅवा॒चस्पति॒-रित्या॑हु॒रना᳚प्त-श्चतूरा॒त्रोऽति॑रिक्तः षड्-रा॒त्रोऽथ॒ वा ए॒ष स॑प्रं॒ति य॒ज्ञो यत् प॑ञ्चरा॒त्रो य ए॒वं ॅवि॒द्वान् प॑ञ्चरा॒त्रेण॒ यज॑ते संप्र॒त्ये॑व य॒ज्ञेन॑ यजते पञ्चरा॒त्रो भ॑वति॒ पञ्च॒ वा ऋ॒तवः॑ संॅवथ्स॒र - [  ] \newline

\textbf{Pada Paata} \newline

अ॒ह॒र॒त् । तेन॑ । अ॒य॒ज॒त॒ । ततः॑ । वै । सः । वा॒चः । प्र॒व॒दि॒तेति॑ प्र - व॒दि॒ता । अ॒भ॒व॒त् । यः । ए॒वम् । वि॒द्वान् । प॒ञ्च॒रा॒त्रेणेति॑ पञ्च - रा॒त्रेण॑ । यज॑ते । प्र॒व॒दि॒तेति॑ प्र - व॒दि॒ता । ए॒व । वा॒चः । भ॒व॒ति॒ । अथो॒ इति॑ । ए॒न॒म् । वा॒चः । पतिः॑ । इति॑ । आ॒हुः॒ । अना᳚प्तः । च॒तू॒रा॒त्र इति॑ चतुः - रा॒त्रः । अति॑रिक्त॒ इत्यति॑ - रि॒क्तः॒ । ष॒ड्रा॒त्र इति॑ षट् - रा॒त्रः । अथ॑ । वै । ए॒षः । स॒प्रं॒तीति॑ सं - प्र॒ति । य॒ज्ञ्ः । यत् । प॒ञ्च॒रा॒त्र इति॑ पञ्च - रा॒त्रः । यः । ए॒वम् । वि॒द्वान् । प॒ञ्च॒रा॒त्रेणेति॑ पञ्च - रा॒त्रेण॑ । यज॑ते । स॒म्प्र॒तीति॑ सं - प्र॒ति । ए॒व । य॒ज्ञेन॑ । य॒ज॒ते॒ । प॒ञ्च॒रा॒त्र इति॑ पञ्च - रा॒त्रः । भ॒व॒ति॒ । पञ्च॑ । वै । ऋ॒तवः॑ । सं॒ॅव॒थ्स॒र इति॑ सं - व॒थ्स॒रः ।  \newline


\textbf{Krama Paata} \newline

अ॒ह॒र॒त् तेन॑ । तेना॑यजत । अ॒य॒ज॒त॒ ततः॑ । ततो॒ वै । वै सः । स वा॒चः । वा॒चः प्र॑वदि॒ता । प्र॒व॒दि॒ताऽभ॑वत् । प्र॒व॒दि॒तेति॑ प्र - व॒दि॒ता । अ॒भ॒व॒द् यः । य ए॒वम् । ए॒वम् ॅवि॒द्वान् । वि॒द्वान् प॑ञ्चरा॒त्रेण॑ । प॒ञ्च॒रा॒त्रेण॒ यज॑ते । प॒ञ्च॒रा॒त्रेणेति॑ पञ्च - रा॒त्रेण॑ । यज॑ते प्रवदि॒ता । प्र॒व॒दि॒तैव । प्र॒व॒दि॒तेति॑ प्र - व॒दि॒ता । ए॒व वा॒चः । वा॒चो भ॑वति । भ॒व॒त्यथो᳚ । अथो॑ एनम् । अथो॒ इत्यथो᳚ । ए॒न॒म् ॅवा॒चः । वा॒चस्पतिः॑ । पति॒रिति॑ । इत्या॑हुः । आ॒हु॒रना᳚प्तः । अना᳚प्तश्चतूरा॒त्रः । च॒तू॒रा॒त्रोऽति॑रिक्तः । च॒तू॒रा॒त्र इति॑ चतुः - रा॒त्रः । अति॑रिक्तष्षड्‍रा॒त्रः । अति॑रिक्त॒ इत्यति॑ - रि॒क्तः॒ । ष॒ड्‍रा॒त्रोऽथ॑ । ष॒ड्‍रा॒त्र इति॑ षट् - रा॒त्रः । अथ॒ वै । वा ए॒षः । ए॒ष स॑म्प्र॒ति । स॒म्प्र॒ति य॒ज्ञ्ः । स॒म्प्र॒तीति॑ सम् - प्र॒ति । य॒ज्ञो यत् । यत् प॑ञ्चरा॒त्रः । प॒ञ्च॒रा॒त्रो यः । प॒ञ्च॒रात्र इति॑ पञ्च - रा॒त्रः । य ए॒वम् । ए॒वम् ॅवि॒द्वान् । वि॒द्वान् प॑ञ्चरा॒त्रेण॑ । प॒ञ्च॒रा॒त्रेण॒ यज॑ते । प॒ञ्च॒रा॒त्रेणेति॑ पञ्च - रा॒त्रेण॑ । यज॑ते सम्प्र॒ति । स॒म्प्र॒त्ये॑व । स॒म्प्र॒तीति॑ सम् - प्र॒ति । ए॒व य॒ज्ञेन॑ । य॒ज्ञेन॑ यजते । य॒ज॒ते॒ प॒ञ्च॒रा॒त्रः । प॒ञ्च॒रा॒त्रो भ॑वति । प॒ञ्च॒रा॒त्र इति॑ पञ्च - रा॒त्रः । भ॒व॒ति॒ पञ्च॑ । पञ्च॒ वै । वा ऋ॒तवः॑ । ऋ॒तवः॑ सम्ॅवथ्स॒रः । स॒म्ॅव॒थ्स॒र ऋ॒तुषु॑ । स॒म्ॅव॒थ्स॒र इति॑ सम् - व॒थ्स॒रः \newline

\textbf{Jatai Paata} \newline

1. अ॒ह॒र॒त् तेन॒ तेना॑ हर दहर॒त् तेन॑ । \newline
2. तेना॑ यजता यजत॒ तेन॒ तेना॑ यजत । \newline
3. अ॒य॒ज॒त॒ तत॒ स्ततो॑ ऽयजता यजत॒ ततः॑ । \newline
4. ततो॒ वै वै तत॒ स्ततो॒ वै । \newline
5. वै स स वै वै सः । \newline
6. स वा॒चो वा॒चः स स वा॒चः । \newline
7. वा॒चः प्र॑वदि॒ता प्र॑वदि॒ता वा॒चो वा॒चः प्र॑वदि॒ता । \newline
8. प्र॒व॒दि॒ता ऽभ॑व दभवत् प्रवदि॒ता प्र॑वदि॒ता ऽभ॑वत् । \newline
9. प्र॒व॒दि॒तेति॑ प्र - व॒दि॒ता । \newline
10. अ॒भ॒व॒द् यो यो॑ ऽभव दभव॒द् यः । \newline
11. य ए॒व मे॒वं ॅयो य ए॒वम् । \newline
12. ए॒वं ॅवि॒द्वान्. वि॒द्वा ने॒व मे॒वं ॅवि॒द्वान् । \newline
13. वि॒द्वान् प॑ञ्चरा॒त्रेण॑ पञ्चरा॒त्रेण॑ वि॒द्वान्. वि॒द्वान् प॑ञ्चरा॒त्रेण॑ । \newline
14. प॒ञ्च॒रा॒त्रेण॒ यज॑ते॒ यज॑ते पञ्चरा॒त्रेण॑ पञ्चरा॒त्रेण॒ यज॑ते । \newline
15. प॒ञ्च॒रा॒त्रेणेति॑ पञ्च - रा॒त्रेण॑ । \newline
16. यज॑ते प्रवदि॒ता प्र॑वदि॒ता यज॑ते॒ यज॑ते प्रवदि॒ता । \newline
17. प्र॒व॒दि॒तै वैव प्र॑वदि॒ता प्र॑वदि॒ तैव । \newline
18. प्र॒व॒दि॒तेति॑ प्र - व॒दि॒ता । \newline
19. ए॒व वा॒चो वा॒च ए॒वैव वा॒चः । \newline
20. वा॒चो भ॑वति भवति वा॒चो वा॒चो भ॑वति । \newline
21. भ॒व॒ त्यथो॒ अथो॑ भवति भव॒ त्यथो᳚ । \newline
22. अथो॑ एन मेन॒ मथो॒ अथो॑ एनम् । \newline
23. अथो॒ इत्यथो᳚ । \newline
24. ए॒नं॒ ॅवा॒चो वा॒च ए॑न मेनं ॅवा॒चः । \newline
25. वा॒च स्पति॒ष् पति॑र् वा॒चो वा॒च स्पतिः॑ । \newline
26. पति॒ रितीति॒ पति॒ष् पति॒ रिति॑ । \newline
27. इत्या॑हु राहु॒ रिती त्या॑हुः । \newline
28. आ॒हु॒ रना॒प्तो ऽना᳚प्त आहु राहु॒ रना᳚प्तः । \newline
29. अना᳚प्त श्चतूरा॒त्र श्च॑तूरा॒त्रो ऽना॒प्तो ऽना᳚प्त श्चतूरा॒त्रः । \newline
30. च॒तू॒रा॒त्रो ऽति॑रि॒क्तो ऽति॑रिक्तश्चतूरा॒त्र श्च॑तूरा॒त्रो ऽति॑रिक्तः । \newline
31. च॒तू॒रा॒त्र इति॑ चतुः - रा॒त्रः । \newline
32. अति॑रिक्त ष्षड्रा॒त्र ष्ष॑ड्रा॒त्रो ऽति॑रि॒क्तो ऽति॑रिक्त ष्षड्रा॒त्रः । \newline
33. अति॑रिक्त॒ इत्यति॑ - रि॒क्तः॒ । \newline
34. ष॒ड्रा॒त्रो ऽथाथ॑ षड्रा॒त्र ष्ष॑ड्रा॒त्रो ऽथ॑ । \newline
35. ष॒ड्रा॒त्र इति॑ षट् - रा॒त्रः । \newline
36. अथ॒ वै वा अथाथ॒ वै । \newline
37. वा ए॒ष ए॒ष वै वा ए॒षः । \newline
38. ए॒ष सं॑प्र॒ति सं॑प्र॒त्ये॑ष ए॒ष सं॑प्र॒ति । \newline
39. सं॒प्र॒ति य॒ज्ञो य॒ज्ञ्ः सं॑प्र॒ति सं॑प्र॒ति य॒ज्ञ्ः । \newline
40. सं॒प्र॒तीति॑ सं - प्र॒ति । \newline
41. य॒ज्ञो यद् यद् य॒ज्ञो य॒ज्ञो यत् । \newline
42. यत् प॑ञ्चरा॒त्रः प॑ञ्चरा॒त्रो यद् यत् प॑ञ्चरा॒त्रः । \newline
43. प॒ञ्च॒रा॒त्रो यो यः प॑ञ्चरा॒त्रः प॑ञ्चरा॒त्रो यः । \newline
44. प॒ञ्च॒रा॒त्र इति॑ पञ्च - रा॒त्रः । \newline
45. य ए॒व मे॒वं ॅयो य ए॒वम् । \newline
46. ए॒वं ॅवि॒द्वान्. वि॒द्वा ने॒व मे॒वं ॅवि॒द्वान् । \newline
47. वि॒द्वान् प॑ञ्चरा॒त्रेण॑ पञ्चरा॒त्रेण॑ वि॒द्वान्. वि॒द्वान् प॑ञ्चरा॒त्रेण॑ । \newline
48. प॒ञ्च॒रा॒त्रेण॒ यज॑ते॒ यज॑ते पञ्चरा॒त्रेण॑ पञ्चरा॒त्रेण॒ यज॑ते । \newline
49. प॒ञ्च॒रा॒त्रेणेति॑ पञ्च - रा॒त्रेण॑ । \newline
50. यज॑ते संप्र॒ति सं॑प्र॒ति यज॑ते॒ यज॑ते संप्र॒ति । \newline
51. सं॒प्र॒ त्ये॑वैव सं॑प्र॒ति सं॑प्र॒ त्ये॑व । \newline
52. स॒म्प्र॒तीति॑ सं - प्र॒ति । \newline
53. ए॒व य॒ज्ञेन॑ य॒ज्ञे नै॒वैव य॒ज्ञेन॑ । \newline
54. य॒ज्ञेन॑ यजते यजते य॒ज्ञेन॑ य॒ज्ञेन॑ यजते । \newline
55. य॒ज॒ते॒ प॒ञ्च॒रा॒त्रः प॑ञ्चरा॒त्रो य॑जते यजते पञ्चरा॒त्रः । \newline
56. प॒ञ्च॒रा॒त्रो भ॑वति भवति पञ्चरा॒त्रः प॑ञ्चरा॒त्रो भ॑वति । \newline
57. प॒ञ्च॒रा॒त्र इति॑ पञ्च - रा॒त्रः । \newline
58. भ॒व॒ति॒ पञ्च॒ पञ्च॑ भवति भवति॒ पञ्च॑ । \newline
59. पञ्च॒ वै वै पञ्च॒ पञ्च॒ वै । \newline
60. वा ऋ॒तव॑ ऋ॒तवो॒ वै वा ऋ॒तवः॑ । \newline
61. ऋ॒तवः॑ संॅवथ्स॒रः सं॑ॅवथ्स॒र ऋ॒तव॑ ऋ॒तवः॑ संॅवथ्स॒रः । \newline
62. सं॒ॅव॒थ्स॒र ऋ॒तुष् वृ॒तुषु॑ संॅवथ्स॒रः सं॑ॅवथ्स॒र ऋ॒तुषु॑ । \newline
63. सं॒ॅव॒थ्स॒र इति॑ सं - व॒थ्स॒रः । \newline

\textbf{Ghana Paata } \newline

1. अ॒ह॒र॒त् तेन॒ तेना॑ हर दहर॒त् तेना॑ यजता यजत॒ तेना॑ हर दहर॒त् तेना॑ यजत । \newline
2. तेना॑ यजता यजत॒ तेन॒ तेना॑ यजत॒ तत॒ स्ततो॑ ऽयजत॒ तेन॒ तेना॑ यजत॒ ततः॑ । \newline
3. अ॒य॒ज॒त॒ तत॒ स्ततो॑ ऽयजता यजत॒ ततो॒ वै वै ततो॑ ऽयजता यजत॒ ततो॒ वै । \newline
4. ततो॒ वै वै तत॒ स्ततो॒ वै स स वै तत॒ स्ततो॒ वै सः । \newline
5. वै स स वै वै स वा॒चो वा॒चः स वै वै स वा॒चः । \newline
6. स वा॒चो वा॒चः स स वा॒चः प्र॑वदि॒ता प्र॑वदि॒ता वा॒चः स स वा॒चः प्र॑वदि॒ता । \newline
7. वा॒चः प्र॑वदि॒ता प्र॑वदि॒ता वा॒चो वा॒चः प्र॑वदि॒ता ऽभ॑व दभवत् प्रवदि॒ता वा॒चो वा॒चः प्र॑वदि॒ता ऽभ॑वत् । \newline
8. प्र॒व॒दि॒ता ऽभ॑व दभवत् प्रवदि॒ता प्र॑वदि॒ता ऽभ॑व॒द् यो यो॑ ऽभवत् प्रवदि॒ता प्र॑वदि॒ता ऽभ॑व॒द् यः । \newline
9. प्र॒व॒दि॒तेति॑ प्र - व॒दि॒ता । \newline
10. अ॒भ॒व॒द् यो यो॑ ऽभव दभव॒द् य ए॒व मे॒वं ॅयो॑ ऽभव दभव॒द् य ए॒वम् । \newline
11. य ए॒व मे॒वं ॅयो य ए॒वं ॅवि॒द्वान्. वि॒द्वा ने॒वं ॅयो य ए॒वं ॅवि॒द्वान् । \newline
12. ए॒वं ॅवि॒द्वान्. वि॒द्वा ने॒व मे॒वं ॅवि॒द्वान् प॑ञ्चरा॒त्रेण॑ पञ्चरा॒त्रेण॑ वि॒द्वा ने॒व मे॒वं ॅवि॒द्वान् प॑ञ्चरा॒त्रेण॑ । \newline
13. वि॒द्वान् प॑ञ्चरा॒त्रेण॑ पञ्चरा॒त्रेण॑ वि॒द्वान्. वि॒द्वान् प॑ञ्चरा॒त्रेण॒ यज॑ते॒ यज॑ते पञ्चरा॒त्रेण॑ वि॒द्वान्. वि॒द्वान् प॑ञ्चरा॒त्रेण॒ यज॑ते । \newline
14. प॒ञ्च॒रा॒त्रेण॒ यज॑ते॒ यज॑ते पञ्चरा॒त्रेण॑ पञ्चरा॒त्रेण॒ यज॑ते प्रवदि॒ता प्र॑वदि॒ता यज॑ते पञ्चरा॒त्रेण॑ पञ्चरा॒त्रेण॒ यज॑ते प्रवदि॒ता । \newline
15. प॒ञ्च॒रा॒त्रेणेति॑ पञ्च - रा॒त्रेण॑ । \newline
16. यज॑ते प्रवदि॒ता प्र॑वदि॒ता यज॑ते॒ यज॑ते प्रवदि॒ तैवैव प्र॑वदि॒ता यज॑ते॒ यज॑ते प्रवदि॒तैव । \newline
17. प्र॒व॒दि॒तैवैव प्र॑वदि॒ता प्र॑वदि॒तैव वा॒चो वा॒च ए॒व प्र॑वदि॒ता प्र॑वदि॒तैव वा॒चः । \newline
18. प्र॒व॒दि॒तेति॑ प्र - व॒दि॒ता । \newline
19. ए॒व वा॒चो वा॒च ए॒वैव वा॒चो भ॑वति भवति वा॒च ए॒वैव वा॒चो भ॑वति । \newline
20. वा॒चो भ॑वति भवति वा॒चो वा॒चो भ॑व॒ त्यथो॒ अथो॑ भवति वा॒चो वा॒चो भ॑व॒ त्यथो᳚ । \newline
21. भ॒व॒ त्यथो॒ अथो॑ भवति भव॒ त्यथो॑ एन मेन॒ मथो॑ भवति भव॒ त्यथो॑ एनम् । \newline
22. अथो॑ एन मेन॒ मथो॒ अथो॑ एनं ॅवा॒चो वा॒च ए॑न॒ मथो॒ अथो॑ एनं ॅवा॒चः । \newline
23. अथो॒ इत्यथो᳚ । \newline
24. ए॒नं॒ ॅवा॒चो वा॒च ए॑न मेनं ॅवा॒च स्पति॒ष् पति॑र् वा॒च ए॑न मेनं ॅवा॒च स्पतिः॑ । \newline
25. वा॒च स्पति॒ष् पति॑र् वा॒चो वा॒च स्पति॒ रितीति॒ पति॑र् वा॒चो वा॒च स्पति॒ रिति॑ । \newline
26. पति॒रितीति॒ पति॒ष् पति॒ रित्या॑हु राहु॒ रिति॒ पति॒ष् पति॒ रित्या॑हुः । \newline
27. इत्या॑हु राहु॒ रिती त्या॑हु॒ रना॒प्तो ऽना᳚प्त आहु॒ रिती त्या॑हु॒ रना᳚प्तः । \newline
28. आ॒हु॒ रना॒प्तो ऽना᳚प्त आहु राहु॒ रना᳚प्त श्चतूरा॒त्र श्च॑तूरा॒त्रो ऽना᳚प्त आहु राहु॒ रना᳚प्त श्चतूरा॒त्रः । \newline
29. अना᳚प्त श्चतूरा॒त्र श्च॑तूरा॒त्रो ऽना॒प्तो ऽना᳚प्त श्चतूरा॒त्रो ऽति॑रि॒क्तो ऽति॑रिक्त श्चतूरा॒त्रो ऽना॒प्तो ऽना᳚प्त श्चतूरा॒त्रो ऽति॑रिक्तः । \newline
30. च॒तू॒रा॒त्रो ऽति॑रि॒क्तो ऽति॑रिक्त श्चतूरा॒त्र श्च॑तूरा॒त्रो ऽति॑रिक्त ष्षड्रा॒त्र ष्ष॑ड्रा॒त्रो ऽति॑रिक्त श्चतूरा॒त्र श्च॑तूरा॒त्रो ऽति॑रिक्त ष्षड्रा॒त्रः । \newline
31. च॒तू॒रा॒त्र इति॑ चतुः - रा॒त्रः । \newline
32. अति॑रिक्त ष्षड्रा॒त्र ष्ष॑ड्रा॒त्रो ऽति॑रि॒क्तो ऽति॑रिक्त ष्षड्रा॒त्रो ऽथाथ॑ षड्रा॒त्रो ऽति॑रि॒क्तो ऽति॑रिक्त ष्षड्रा॒त्रो ऽथ॑ । \newline
33. अति॑रिक्त॒ इत्यति॑ - रि॒क्तः॒ । \newline
34. ष॒ड्रा॒त्रो ऽथाथ॑ षड्रा॒त्र ष्ष॑ड्रा॒त्रो ऽथ॒ वै वा अथ॑ षड्रा॒त्र ष्ष॑ड्रा॒त्रो ऽथ॒ वै । \newline
35. ष॒ड्रा॒त्र इति॑ षट् - रा॒त्रः । \newline
36. अथ॒ वै वा अथाथ॒ वा ए॒ष ए॒ष वा अथाथ॒ वा ए॒षः । \newline
37. वा ए॒ष ए॒ष वै वा ए॒ष सं॑प्र॒ति सं॑प्र॒ त्ये॑ष वै वा ए॒ष सं॑प्र॒ति । \newline
38. ए॒ष सं॑प्र॒ति सं॑प्र॒ त्ये॑ष ए॒ष सं॑प्र॒ति य॒ज्ञो य॒ज्ञ्ः सं॑प्र॒ त्ये॑ष ए॒ष सं॑प्र॒ति य॒ज्ञ्ः । \newline
39. सं॒प्र॒ति य॒ज्ञो य॒ज्ञ्ः सं॑प्र॒ति सं॑प्र॒ति य॒ज्ञो यद् यद् य॒ज्ञ्ः सं॑प्र॒ति सं॑प्र॒ति य॒ज्ञो यत् । \newline
40. सं॒प्र॒तीति॑ सं - प्र॒ति । \newline
41. य॒ज्ञो यद् यद् य॒ज्ञो य॒ज्ञो यत् प॑ञ्चरा॒त्रः प॑ञ्चरा॒त्रो यद् य॒ज्ञो य॒ज्ञो यत् प॑ञ्चरा॒त्रः । \newline
42. यत् प॑ञ्चरा॒त्रः प॑ञ्चरा॒त्रो यद् यत् प॑ञ्चरा॒त्रो यो यः प॑ञ्चरा॒त्रो यद् यत् प॑ञ्चरा॒त्रो यः । \newline
43. प॒ञ्च॒रा॒त्रो यो यः प॑ञ्चरा॒त्रः प॑ञ्चरा॒त्रो य ए॒व मे॒वं ॅयः प॑ञ्चरा॒त्रः प॑ञ्चरा॒त्रो य ए॒वम् । \newline
44. प॒ञ्च॒रा॒त्र इति॑ पञ्च - रा॒त्रः । \newline
45. य ए॒व मे॒वं ॅयो य ए॒वं ॅवि॒द्वान्. वि॒द्वा ने॒वं ॅयो य ए॒वं ॅवि॒द्वान् । \newline
46. ए॒वं ॅवि॒द्वान्. वि॒द्वा ने॒व मे॒वं ॅवि॒द्वान् प॑ञ्चरा॒त्रेण॑ पञ्चरा॒त्रेण॑ वि॒द्वा ने॒व मे॒वं ॅवि॒द्वान् प॑ञ्चरा॒त्रेण॑ । \newline
47. वि॒द्वान् प॑ञ्चरा॒त्रेण॑ पञ्चरा॒त्रेण॑ वि॒द्वान्. वि॒द्वान् प॑ञ्चरा॒त्रेण॒ यज॑ते॒ यज॑ते पञ्चरा॒त्रेण॑ वि॒द्वान्. वि॒द्वान् प॑ञ्चरा॒त्रेण॒ यज॑ते । \newline
48. प॒ञ्च॒रा॒त्रेण॒ यज॑ते॒ यज॑ते पञ्चरा॒त्रेण॑ पञ्चरा॒त्रेण॒ यज॑ते संप्र॒ति सं॑प्र॒ति यज॑ते पञ्चरा॒त्रेण॑ पञ्चरा॒त्रेण॒ यज॑ते संप्र॒ति । \newline
49. प॒ञ्च॒रा॒त्रेणेति॑ पञ्च - रा॒त्रेण॑ । \newline
50. यज॑ते संप्र॒ति सं॑प्र॒ति यज॑ते॒ यज॑ते संप्र॒ त्ये॑वैव सं॑प्र॒ति यज॑ते॒ यज॑ते संप्र॒ त्ये॑व । \newline
51. सं॒प्र॒ त्ये॑वैव सं॑प्र॒ति सं॑प्र॒ त्ये॑व य॒ज्ञेन॑ य॒ज्ञेनै॒व सं॑प्र॒ति सं॑प्र॒ त्ये॑व य॒ज्ञेन॑ । \newline
52. स॒म्प्र॒तीति॑ सं - प्र॒ति । \newline
53. ए॒व य॒ज्ञेन॑ य॒ज्ञेनै॒वैव य॒ज्ञेन॑ यजते यजते य॒ज्ञेनै॒वैव य॒ज्ञेन॑ यजते । \newline
54. य॒ज्ञेन॑ यजते यजते य॒ज्ञेन॑ य॒ज्ञेन॑ यजते पञ्चरा॒त्रः प॑ञ्चरा॒त्रो य॑जते य॒ज्ञेन॑ य॒ज्ञेन॑ यजते पञ्चरा॒त्रः । \newline
55. य॒ज॒ते॒ प॒ञ्च॒रा॒त्रः प॑ञ्चरा॒त्रो य॑जते यजते पञ्चरा॒त्रो भ॑वति भवति पञ्चरा॒त्रो य॑जते यजते पञ्चरा॒त्रो भ॑वति । \newline
56. प॒ञ्च॒रा॒त्रो भ॑वति भवति पञ्चरा॒त्रः प॑ञ्चरा॒त्रो भ॑वति॒ पञ्च॒ पञ्च॑ भवति पञ्चरा॒त्रः प॑ञ्चरा॒त्रो भ॑वति॒ पञ्च॑ । \newline
57. प॒ञ्च॒रा॒त्र इति॑ पञ्च - रा॒त्रः । \newline
58. भ॒व॒ति॒ पञ्च॒ पञ्च॑ भवति भवति॒ पञ्च॒ वै वै पञ्च॑ भवति भवति॒ पञ्च॒ वै । \newline
59. पञ्च॒ वै वै पञ्च॒ पञ्च॒ वा ऋ॒तव॑ ऋ॒तवो॒ वै पञ्च॒ पञ्च॒ वा ऋ॒तवः॑ । \newline
60. वा ऋ॒तव॑ ऋ॒तवो॒ वै वा ऋ॒तवः॑ संॅवथ्स॒रः सं॑ॅवथ्स॒र ऋ॒तवो॒ वै वा ऋ॒तवः॑ संॅवथ्स॒रः । \newline
61. ऋ॒तवः॑ संॅवथ्स॒रः सं॑ॅवथ्स॒र ऋ॒तव॑ ऋ॒तवः॑ संॅवथ्स॒र ऋ॒तुष् वृ॒तुषु॑ संॅवथ्स॒र ऋ॒तव॑ ऋ॒तवः॑ संॅवथ्स॒र ऋ॒तुषु॑ । \newline
62. सं॒ॅव॒थ्स॒र ऋ॒तुष् वृ॒तुषु॑ संॅवथ्स॒रः सं॑ॅवथ्स॒र ऋ॒तुष्वे॒वैव र्‌तुषु॑ संॅवथ्स॒रः सं॑ॅवथ्स॒र ऋ॒तुष्वे॒व । \newline
63. सं॒ॅव॒थ्स॒र इति॑ सं - व॒थ्स॒रः । \newline
\pagebreak
\markright{ TS 7.1.10.4  \hfill https://www.vedavms.in \hfill}

\section{ TS 7.1.10.4 }

\textbf{TS 7.1.10.4 } \newline
\textbf{Samhita Paata} \newline

ऋ॒तुष्वे॒व सं॑ॅवथ्स॒रे प्रति॑ तिष्ठ॒त्यथो॒ पञ्चा᳚क्षरा प॒ङ्क्तिः पाङ्क्तो॑ य॒ज्ञो य॒ज्ञ्मे॒वाव॑ रुन्धे त्रि॒वृद॑ग्निष्टो॒मो भ॑वति॒ तेज॑ ए॒वाव॑ रुन्धे पञ्चद॒शो भ॑वतीन्द्रि॒यमे॒वाव॑ रुन्धे सप्तद॒शो भ॑वत्य॒न्नाद्य॒स्या-व॑रुद्ध्या॒ अथो॒ प्रैव तेन॑ जायते पञ्चविꣳ॒॒शो᳚ ऽग्निष्टो॒मो भ॑वति प्र॒जाप॑ते॒राप्त्यै॑ महाव्र॒तवा॑-न॒न्नाद्य॒स्या-व॑रुद्ध्यै विश्व॒जिथ् सर्व॑पृष्ठो-ऽतिरा॒त्रो भ॑वति॒ सर्व॑स्या॒भिजि॑त्यै ( ) ॥ \newline

\textbf{Pada Paata} \newline

ऋ॒तुषु॑ । ए॒व । सं॒ॅव॒थ्स॒र इति॑ सं - व॒थ्स॒रे । प्रतीति॑ । ति॒ष्ठ॒न्ति॒ । अथो॒ इति॑ । पञ्चा᳚क्ष॒रेति॒ पञ्च॑ - अ॒क्ष॒रा॒ । प॒ङ्क्तिः । पाङ्क्तः॑ । य॒ज्ञ्ः । य॒ज्ञ्म् । ए॒व । अवेति॑ । रु॒न्धे॒ । त्रि॒वृदिति॑ त्रि - वृत् । अ॒ग्नि॒ष्टो॒म इत्य॑ग्नि - स्तो॒मः । भ॒व॒ति॒ । तेजः॑ । ए॒व । अवेति॑ । रु॒न्धे॒ । प॒ञ्च॒द॒श इति॑ पञ्च - द॒शः । भ॒व॒ति॒ । इ॒न्द्रि॒यम् । ए॒व । अवेति॑ । रु॒न्धे॒ । स॒प्त॒द॒श इति॑ सप्त - द॒शः । भ॒व॒ति॒ । अ॒न्नाद्य॒स्येत्य॑न्न - अद्य॑स्य । अव॑रुद्ध्या॒ इत्यव॑ - रु॒द्ध्यै॒ । अथो॒ इति॑ । प्रेति॑ । ए॒व । तेन॑ । जा॒य॒ते॒ । प॒ञ्च॒विꣳ॒॒श इति॑ पञ्च - विꣳ॒॒शः । अ॒ग्नि॒ष्टो॒म इत्य॑ग्नि - स्तो॒मः । भ॒व॒ति॒ । प्र॒जाप॑ते॒रिति॑ प्र॒जा-प॒तेः॒ । आप्त्यै᳚ । म॒हा॒व्र॒तवा॒निति॑ महाव्र॒त-वा॒न् । अ॒न्नाद्य॒स्येत्य॑न्न - अद्य॑स्य । अव॑रुद्ध्या॒ इत्यव॑ - रु॒द्ध्यै॒ । वि॒श्व॒जिदिति॑ विश्व - जित् । सर्व॑पृष्ठ॒ इति॒ सर्व॑ - पृ॒ष्ठः॒ । अ॒ति॒रा॒त्र इत्य॑ति - रा॒त्रः । भ॒व॒ति॒ । सर्व॑स्य । अ॒भिजि॑त्या॒ इत्य॒भि-जि॒त्यै॒ ( ) ॥  \newline


\textbf{Krama Paata} \newline

ऋ॒तुष्वे॒व । ए॒व स॑म्ॅवथ्स॒रे । स॒म्ॅव॒थ्स॒रे प्रति॑ । स॒म्ॅव॒थ्स॒र इति॑ सम् - व॒थ्स॒रे । प्रति॑ तिष्ठति । ति॒ष्ठ॒त्यथो᳚ । अथो॒ पञ्चा᳚क्षरा । अथो॒ इत्यथो᳚ । पञ्चा᳚क्षरा प॒ङ्‍क्तिः । पञ्चा᳚क्ष॒रेति॒ पञ्च॑ - अ॒क्ष॒रा॒ । प॒ङ्‍क्तिः पाङ्‍क्तः॑ । पाङ्‍क्तो॑ य॒ज्ञ्ः । य॒ज्ञो य॒ज्ञ्म् । य॒ज्ञ्मे॒व । ए॒वाव॑ । अव॑ रुन्धे । रु॒न्धे॒ त्रि॒वृत् । त्रि॒वृद॑ग्निष्टो॒मः । त्रि॒वृदिति॑ त्रि - वृत् । अ॒ग्नि॒ष्टो॒मो भ॑वति । अ॒ग्नि॒ष्टो॒म इत्य॑ग्नि - स्तो॒मः । भ॒व॒ति॒ तेजः॑ । तेज॑ ए॒व । ए॒वाव॑ । अव॑ रुन्धे । रु॒न्धे॒ प॒ञ्च॒द॒शः । प॒ञ्च॒द॒शो भ॑वति । प॒ञ्च॒द॒श इति॑ पञ्च - द॒शः । भ॒व॒ती॒न्द्रि॒यम् । इ॒न्द्रि॒यमे॒व । ए॒वाव॑ । अव॑ रुन्धे । रु॒न्धे॒ स॒प्त॒द॒शः । स॒प्त॒द॒शो भ॑वति । स॒प्त॒द॒श इति॑ सप्त - द॒शः । भ॒व॒त्य॒न्नाद्य॑स्य । अ॒न्नाद्य॒स्या,व॑रुद्ध्यै । अ॒न्नाद्य॒स्येत्य॑न्न - अद्य॑स्य । अव॑रुद्ध्या॒ अथो᳚ । अव॑रुद्ध्या॒ इत्यव॑ - रु॒द्ध्यै॒ । अथो॒ प्र । अथो॒ इत्यथो᳚ । प्रैव । ए॒व तेन॑ । तेन॑ जायते । जा॒य॒ते॒ प॒ञ्च॒विꣳ॒॒शः । प॒ञ्च॒विꣳ॒॒शो᳚ऽग्निष्टो॒मः । प॒ञ्च॒विꣳ॒॒श इति॑ पञ्च - विꣳ॒॒शः । अ॒ग्नि॒ष्टो॒मो भ॑वति । अ॒ग्नि॒ष्टो॒म इत्य॑ग्नि - स्तो॒मः । भ॒व॒ति॒ प्र॒जाप॑तेः । प्र॒जाप॑ते॒राप्त्यै᳚ । प्र॒जाप॑ते॒रिति॑ प्र॒जा - प॒तेः॒ । आप्त्यै॑ महाव्र॒तवान्॑ । म॒हा॒व्र॒तवा॑,न॒न्नाद्य॑स्य । म॒हा॒व्र॒तवा॒निति॑ महाव्र॒त - वा॒न्॒ । अ॒न्नाद्य॒स्या,व॑रुद्ध्यै । अ॒न्नाद्य॒स्येत्य॑न्न - अद्य॑स्य । अव॑रुद्ध्यै विश्व॒जित् । अव॑रुद्ध्या॒ इत्यव॑ - रु॒द्ध्यै॒ । वि॒श्व॒जिथ् सर्व॑पृष्ठः । वि॒श्व॒जिदिति॑ विश्व - जित् । सर्व॑पृष्ठोऽतिरा॒त्रः । सर्व॑पृष्ठ॒ इति॒ सर्व॑ - पृ॒ष्ठः॒ । अ॒ति॒रा॒त्रो भ॑वति । अ॒ति॒रा॒त्र इत्य॑ति - रा॒त्रः । भ॒व॒ति॒ सर्व॑स्य । सर्व॑स्या॒भिजि॑त्यै ( ) । अ॒भिजि॑त्या॒ इत्य॒भि - जि॒त्यै॒ । \newline

\textbf{Jatai Paata} \newline

1. ऋ॒तुष्वे॒वैव र्‌तुष् वृ॒तु ष्वे॒व । \newline
2. ए॒व सं॑ॅवथ्स॒रे सं॑ॅवथ्स॒र ए॒वैव सं॑ॅवथ्स॒रे । \newline
3. सं॒ॅव॒थ्स॒रे प्रति॒ प्रति॑ संॅवथ्स॒रे सं॑ॅवथ्स॒रे प्रति॑ । \newline
4. सं॒ॅव॒थ्स॒र इति॑ सं - व॒थ्स॒रे । \newline
5. प्रति॑ तिष्ठति तिष्ठति॒ प्रति॒ प्रति॑ तिष्ठति । \newline
6. ति॒ष्ठ॒ त्यथो॒ अथो॑ तिष्ठति तिष्ठ॒ त्यथो᳚ । \newline
7. अथो॒ पञ्चा᳚क्षरा॒ पञ्चा᳚क्ष॒रा ऽथो॒ अथो॒ पञ्चा᳚क्षरा । \newline
8. अथो॒ इत्यथो᳚ । \newline
9. पञ्चा᳚क्षरा प॒ङ्क्तिः प॒ङ्क्तिः पञ्चा᳚क्षरा॒ पञ्चा᳚क्षरा प॒ङ्क्तिः । \newline
10. पञ्चा᳚क्ष॒रेति॒ पञ्च॑ - अ॒क्ष॒रा॒ । \newline
11. प॒ङ्क्तिः पाङ्क्तः॒ पाङ्क्तः॑ प॒ङ्क्तिः प॒ङ्क्तिः पाङ्क्तः॑ । \newline
12. पाङ्क्तो॑ य॒ज्ञो य॒ज्ञ्ः पाङ्क्तः॒ पाङ्क्तो॑ य॒ज्ञ्ः । \newline
13. य॒ज्ञो य॒ज्ञ्ं ॅय॒ज्ञ्ं ॅय॒ज्ञो य॒ज्ञो य॒ज्ञ्म् । \newline
14. य॒ज्ञ् मे॒वैव य॒ज्ञ्ं ॅय॒ज्ञ् मे॒व । \newline
15. ए॒वावा वै॒वै वाव॑ । \newline
16. अव॑ रुन्धे रु॒न्धे ऽवाव॑ रुन्धे । \newline
17. रु॒न्धे॒ त्रि॒वृत् त्रि॒वृद् रु॑न्धे रुन्धे त्रि॒वृत् । \newline
18. त्रि॒वृ द॑ग्निष्टो॒मो᳚ ऽग्निष्टो॒म स्त्रि॒वृत् त्रि॒वृ द॑ग्निष्टो॒मः । \newline
19. त्रि॒वृदिति॑ त्रि - वृत् । \newline
20. अ॒ग्नि॒ष्टो॒मो भ॑वति भव त्यग्निष्टो॒मो᳚ ऽग्निष्टो॒मो भ॑वति । \newline
21. अ॒ग्नि॒ष्टो॒म इत्य॑ग्नि - स्तो॒मः । \newline
22. भ॒व॒ति॒ तेज॒ स्तेजो॑ भवति भवति॒ तेजः॑ । \newline
23. तेज॑ ए॒वैव तेज॒ स्तेज॑ ए॒व । \newline
24. ए॒वावा वै॒वै वाव॑ । \newline
25. अव॑ रुन्धे रु॒न्धे ऽवाव॑ रुन्धे । \newline
26. रु॒न्धे॒ प॒ञ्च॒द॒शः प॑ञ्चद॒शो रु॑न्धे रुन्धे पञ्चद॒शः । \newline
27. प॒ञ्च॒द॒शो भ॑वति भवति पञ्चद॒शः प॑ञ्चद॒शो भ॑वति । \newline
28. प॒ञ्च॒द॒श इति॑ पञ्च - द॒शः । \newline
29. भ॒व॒ ती॒न्द्रि॒य मि॑न्द्रि॒यम् भ॑वति भव तीन्द्रि॒यम् । \newline
30. इ॒न्द्रि॒य मे॒वैवेन्द्रि॒य मि॑न्द्रि॒य मे॒व । \newline
31. ए॒वावा वै॒वै वाव॑ । \newline
32. अव॑ रुन्धे रु॒न्धे ऽवाव॑ रुन्धे । \newline
33. रु॒न्धे॒ स॒प्त॒द॒शः स॑प्तद॒शो रु॑न्धे रुन्धे सप्तद॒शः । \newline
34. स॒प्त॒द॒शो भ॑वति भवति सप्तद॒शः स॑प्तद॒शो भ॑वति । \newline
35. स॒प्त॒द॒श इति॑ सप्त - द॒शः । \newline
36. भ॒व॒ त्य॒न्नाद्य॑ स्या॒न्नाद्य॑स्य भवति भव त्य॒न्नाद्य॑स्य । \newline
37. अ॒न्नाद्य॒स्या व॑रुद्ध्या॒ अव॑रुद्ध्या अ॒न्नाद्य॑ स्या॒न्नाद्य॒स्या व॑रुद्ध्यै । \newline
38. अ॒न्नाद्य॒स्येत्य॑न्न - अद्य॑स्य । \newline
39. अव॑रुद्ध्या॒ अथो॒ अथो॒ अव॑रुद्ध्या॒ अव॑रुद्ध्या॒ अथो᳚ । \newline
40. अव॑रुद्ध्या॒ इत्यव॑ - रु॒द्ध्यै॒ । \newline
41. अथो॒ प्र प्राथो॒ अथो॒ प्र । \newline
42. अथो॒ इत्यथो᳚ । \newline
43. प्रैवैव प्र प्रैव । \newline
44. ए॒व तेन॒ तेनै॒वैव तेन॑ । \newline
45. तेन॑ जायते जायते॒ तेन॒ तेन॑ जायते । \newline
46. जा॒य॒ते॒ प॒ञ्च॒विꣳ॒॒शः प॑ञ्चविꣳ॒॒शो जा॑यते जायते पञ्चविꣳ॒॒शः । \newline
47. प॒ञ्च॒विꣳ॒॒शो᳚ ऽग्निष्टो॒मो᳚ ऽग्निष्टो॒मः प॑ञ्चविꣳ॒॒शः प॑ञ्चविꣳ॒॒शो᳚ ऽग्निष्टो॒मः । \newline
48. प॒ञ्च॒विꣳ॒॒श इति॑ पञ्च - विꣳ॒॒शः । \newline
49. अ॒ग्नि॒ष्टो॒मो भ॑वति भव त्यग्निष्टो॒मो᳚ ऽग्निष्टो॒मो भ॑वति । \newline
50. अ॒ग्नि॒ष्टो॒म इत्य॑ग्नि - स्तो॒मः । \newline
51. भ॒व॒ति॒ प्र॒जाप॑तेः प्र॒जाप॑तेर् भवति भवति प्र॒जाप॑तेः । \newline
52. प्र॒जाप॑ते॒ राप्त्या॒ आप्त्यै᳚ प्र॒जाप॑तेः प्र॒जाप॑ते॒ राप्त्यै᳚ । \newline
53. प्र॒जाप॑ते॒रिति॑ प्र॒जा - प॒तेः॒ । \newline
54. आप्त्यै॑ महाव्र॒तवा᳚न् महाव्र॒तवा॒ नाप्त्या॒ आप्त्यै॑ महाव्र॒तवान्॑ । \newline
55. म॒हा॒व्र॒तवा॑ न॒न्नाद्य॑स्या॒ न्नाद्य॑स्य महाव्र॒तवा᳚न् महाव्र॒तवा॑ न॒न्नाद्य॑स्य । \newline
56. म॒हा॒व्र॒तवा॒निति॑ महाव्र॒त - वा॒न् । \newline
57. अ॒न्नाद्य॒स्या व॑रुद्ध्या॒ अव॑रुद्ध्या अ॒न्नाद्य॑स्या॒ न्नाद्य॒स्या व॑रुद्ध्यै । \newline
58. अ॒न्नाद्य॒स्येत्य॑न्न - अद्य॑स्य । \newline
59. अव॑रुद्ध्यै विश्व॒जिद् वि॑श्व॒जि दव॑रुद्ध्या॒ अव॑रुद्ध्यै विश्व॒जित् । \newline
60. अव॑रुद्ध्या॒ इत्यव॑ - रु॒द्ध्यै॒ । \newline
61. वि॒श्व॒जिथ् सर्व॑पृष्ठः॒ सर्व॑पृष्ठो विश्व॒जिद् वि॑श्व॒जिथ् सर्व॑पृष्ठः । \newline
62. वि॒श्व॒जिदिति॑ विश्व - जित् । \newline
63. सर्व॑पृष्ठो ऽतिरा॒त्रो॑ ऽतिरा॒त्रः सर्व॑पृष्ठः॒ सर्व॑पृष्ठो ऽतिरा॒त्रः । \newline
64. सर्व॑पृष्ठ॒ इति॒ सर्व॑ - पृ॒ष्ठः॒ । \newline
65. अ॒ति॒रा॒त्रो भ॑वति भव त्यतिरा॒त्रो॑ ऽतिरा॒त्रो भ॑वति । \newline
66. अ॒ति॒रा॒त्र इत्य॑ति - रा॒त्रः । \newline
67. भ॒व॒ति॒ सर्व॑स्य॒ सर्व॑स्य भवति भवति॒ सर्व॑स्य । \newline
68. सर्व॑स्या॒ भिजि॑त्या अ॒भिजि॑त्यै॒ सर्व॑स्य॒ सर्व॑स्या॒ भिजि॑त्यै । \newline
69. अ॒भिजि॑त्या॒ इत्य॒भि - जि॒त्यै॒ । \newline

\textbf{Ghana Paata } \newline

1. ऋ॒तुष्वे॒वैव र्‌तुष् वृ॒तु ष्वे॒व सं॑ॅवथ्स॒रे सं॑ॅवथ्स॒र ए॒व र्‌तुष् वृ॒तु ष्वे॒व सं॑ॅवथ्स॒रे । \newline
2. ए॒व सं॑ॅवथ्स॒रे सं॑ॅवथ्स॒र ए॒वैव सं॑ॅवथ्स॒रे प्रति॒ प्रति॑ संॅवथ्स॒र ए॒वैव सं॑ॅवथ्स॒रे प्रति॑ । \newline
3. सं॒ॅव॒थ्स॒रे प्रति॒ प्रति॑ संॅवथ्स॒रे सं॑ॅवथ्स॒रे प्रति॑ तिष्ठति तिष्ठति॒ प्रति॑ संॅवथ्स॒रे सं॑ॅवथ्स॒रे प्रति॑ तिष्ठति । \newline
4. सं॒ॅव॒थ्स॒र इति॑ सं - व॒थ्स॒रे । \newline
5. प्रति॑ तिष्ठति तिष्ठति॒ प्रति॒ प्रति॑ तिष्ठ॒ त्यथो॒ अथो॑ तिष्ठति॒ प्रति॒ प्रति॑ तिष्ठ॒ त्यथो᳚ । \newline
6. ति॒ष्ठ॒ त्यथो॒ अथो॑ तिष्ठति तिष्ठ॒ त्यथो॒ पञ्चा᳚क्षरा॒ पञ्चा᳚क्ष॒रा ऽथो॑ तिष्ठति तिष्ठ॒ त्यथो॒ पञ्चा᳚क्षरा । \newline
7. अथो॒ पञ्चा᳚क्षरा॒ पञ्चा᳚क्ष॒रा ऽथो॒ अथो॒ पञ्चा᳚क्षरा प॒ङ्क्तिः प॒ङ्क्तिः पञ्चा᳚क्ष॒रा ऽथो॒ अथो॒ पञ्चा᳚क्षरा प॒ङ्क्तिः । \newline
8. अथो॒ इत्यथो᳚ । \newline
9. पञ्चा᳚क्षरा प॒ङ्क्तिः प॒ङ्क्तिः पञ्चा᳚क्षरा॒ पञ्चा᳚क्षरा प॒ङ्क्तिः पाङ्क्तः॒ पाङ्क्तः॑ प॒ङ्क्तिः पञ्चा᳚क्षरा॒ पञ्चा᳚क्षरा प॒ङ्क्तिः पाङ्क्तः॑ । \newline
10. पञ्चा᳚क्ष॒रेति॒ पञ्च॑ - अ॒क्ष॒रा॒ । \newline
11. प॒ङ्क्तिः पाङ्क्तः॒ पाङ्क्तः॑ प॒ङ्क्तिः प॒ङ्क्तिः पाङ्क्तो॑ य॒ज्ञो य॒ज्ञ्ः पाङ्क्तः॑ प॒ङ्क्तिः प॒ङ्क्तिः पाङ्क्तो॑ य॒ज्ञ्ः । \newline
12. पाङ्क्तो॑ य॒ज्ञो य॒ज्ञ्ः पाङ्क्तः॒ पाङ्क्तो॑ य॒ज्ञो य॒ज्ञ्ं ॅय॒ज्ञ्ं ॅय॒ज्ञ्ः पाङ्क्तः॒ पाङ्क्तो॑ य॒ज्ञो य॒ज्ञ्म् । \newline
13. य॒ज्ञो य॒ज्ञ्ं ॅय॒ज्ञ्ं ॅय॒ज्ञो य॒ज्ञो य॒ज्ञ् मे॒वैव य॒ज्ञ्ं ॅय॒ज्ञो य॒ज्ञो य॒ज्ञ् मे॒व । \newline
14. य॒ज्ञ् मे॒वैव य॒ज्ञ्ं ॅय॒ज्ञ् मे॒वावा वै॒व य॒ज्ञ्ं ॅय॒ज्ञ् मे॒वाव॑ । \newline
15. ए॒वावा वै॒वै वाव॑ रुन्धे रु॒न्धे ऽवै॒वै वाव॑ रुन्धे । \newline
16. अव॑ रुन्धे रु॒न्धे ऽवाव॑ रुन्धे त्रि॒वृत् त्रि॒वृद् रु॒न्धे ऽवाव॑ रुन्धे त्रि॒वृत् । \newline
17. रु॒न्धे॒ त्रि॒वृत् त्रि॒वृद् रु॑न्धे रुन्धे त्रि॒वृ द॑ग्निष्टो॒मो᳚ ऽग्निष्टो॒म स्त्रि॒वृद् रु॑न्धे रुन्धे त्रि॒वृ द॑ग्निष्टो॒मः । \newline
18. त्रि॒वृ द॑ग्निष्टो॒मो᳚ ऽग्निष्टो॒म स्त्रि॒वृत् त्रि॒वृ द॑ग्निष्टो॒मो भ॑वति भव त्यग्निष्टो॒म स्त्रि॒वृत् त्रि॒वृ द॑ग्निष्टो॒मो भ॑वति । \newline
19. त्रि॒वृदिति॑ त्रि - वृत् । \newline
20. अ॒ग्नि॒ष्टो॒मो भ॑वति भव त्यग्निष्टो॒मो᳚ ऽग्निष्टो॒मो भ॑वति॒ तेज॒ स्तेजो॑ भव त्यग्निष्टो॒मो᳚ ऽग्निष्टो॒मो भ॑वति॒ तेजः॑ । \newline
21. अ॒ग्नि॒ष्टो॒म इत्य॑ग्नि - स्तो॒मः । \newline
22. भ॒व॒ति॒ तेज॒ स्तेजो॑ भवति भवति॒ तेज॑ ए॒वैव तेजो॑ भवति भवति॒ तेज॑ ए॒व । \newline
23. तेज॑ ए॒वैव तेज॒ स्तेज॑ ए॒वावा वै॒व तेज॒ स्तेज॑ ए॒वाव॑ । \newline
24. ए॒वावा वै॒वै वाव॑ रुन्धे रु॒न्धे ऽवै॒वै वाव॑ रुन्धे । \newline
25. अव॑ रुन्धे रु॒न्धे ऽवाव॑ रुन्धे पञ्चद॒शः प॑ञ्चद॒शो रु॒न्धे ऽवाव॑ रुन्धे पञ्चद॒शः । \newline
26. रु॒न्धे॒ प॒ञ्च॒द॒शः प॑ञ्चद॒शो रु॑न्धे रुन्धे पञ्चद॒शो भ॑वति भवति पञ्चद॒शो रु॑न्धे रुन्धे पञ्चद॒शो भ॑वति । \newline
27. प॒ञ्च॒द॒शो भ॑वति भवति पञ्चद॒शः प॑ञ्चद॒शो भ॑व तीन्द्रि॒य मि॑न्द्रि॒यम् भ॑वति पञ्चद॒शः प॑ञ्चद॒शो भ॑व तीन्द्रि॒यम् । \newline
28. प॒ञ्च॒द॒श इति॑ पञ्च - द॒शः । \newline
29. भ॒व॒ ती॒न्द्रि॒य मि॑न्द्रि॒यम् भ॑वति भव तीन्द्रि॒य मे॒वैवेन्द्रि॒यम् भ॑वति भव तीन्द्रि॒य मे॒व । \newline
30. इ॒न्द्रि॒य मे॒वैवेन्द्रि॒य मि॑न्द्रि॒य मे॒वावा वै॒वेन्द्रि॒य मि॑न्द्रि॒य मे॒वाव॑ । \newline
31. ए॒वावा वै॒वै वाव॑ रुन्धे रु॒न्धे ऽवै॒वै वाव॑ रुन्धे । \newline
32. अव॑ रुन्धे रु॒न्धे ऽवाव॑ रुन्धे सप्तद॒शः स॑प्तद॒शो रु॒न्धे ऽवाव॑ रुन्धे सप्तद॒शः । \newline
33. रु॒न्धे॒ स॒प्त॒द॒शः स॑प्तद॒शो रु॑न्धे रुन्धे सप्तद॒शो भ॑वति भवति सप्तद॒शो रु॑न्धे रुन्धे सप्तद॒शो भ॑वति । \newline
34. स॒प्त॒द॒शो भ॑वति भवति सप्तद॒शः स॑प्तद॒शो भ॑व त्य॒न्नाद्य॑स्या॒ न्नाद्य॑स्य भवति सप्तद॒शः स॑प्तद॒शो भ॑व त्य॒न्नाद्य॑स्य । \newline
35. स॒प्त॒द॒श इति॑ सप्त - द॒शः । \newline
36. भ॒व॒ त्य॒न्नाद्य॑स्या॒ न्नाद्य॑स्य भवति भव त्य॒न्नाद्य॒स्या व॑रुद्ध्या॒ अव॑रुद्ध्या अ॒न्नाद्य॑स्य भवति भव त्य॒न्नाद्य॒स्या व॑रुद्ध्यै । \newline
37. अ॒न्नाद्य॒स्या व॑रुद्ध्या॒ अव॑रुद्ध्या अ॒न्नाद्य॑स्या॒ न्नाद्य॒स्या व॑रुद्ध्या॒ अथो॒ अथो॒ अव॑रुद्ध्या अ॒न्नाद्य॑स्या॒ न्नाद्य॒स्या व॑रुद्ध्या॒ अथो᳚ । \newline
38. अ॒न्नाद्य॒स्येत्य॑न्न - अद्य॑स्य । \newline
39. अव॑रुद्ध्या॒ अथो॒ अथो॒ अव॑रुद्ध्या॒ अव॑रुद्ध्या॒ अथो॒ प्र प्राथो॒ अव॑रुद्ध्या॒ अव॑रुद्ध्या॒ अथो॒ प्र । \newline
40. अव॑रुद्ध्या॒ इत्यव॑ - रु॒द्ध्यै॒ । \newline
41. अथो॒ प्र प्राथो॒ अथो॒ प्रैवैव प्राथो॒ अथो॒ प्रैव । \newline
42. अथो॒ इत्यथो᳚ । \newline
43. प्रैवैव प्र प्रैव तेन॒ तेनै॒व प्र प्रैव तेन॑ । \newline
44. ए॒व तेन॒ तेनै॒वैव तेन॑ जायते जायते॒ तेनै॒वैव तेन॑ जायते । \newline
45. तेन॑ जायते जायते॒ तेन॒ तेन॑ जायते पञ्चविꣳ॒॒शः प॑ञ्चविꣳ॒॒शो जा॑यते॒ तेन॒ तेन॑ जायते पञ्चविꣳ॒॒शः । \newline
46. जा॒य॒ते॒ प॒ञ्च॒विꣳ॒॒शः प॑ञ्चविꣳ॒॒शो जा॑यते जायते पञ्चविꣳ॒॒शो᳚ ऽग्निष्टो॒मो᳚ ऽग्निष्टो॒मः प॑ञ्चविꣳ॒॒शो जा॑यते जायते पञ्चविꣳ॒॒शो᳚ ऽग्निष्टो॒मः । \newline
47. प॒ञ्च॒विꣳ॒॒शो᳚ ऽग्निष्टो॒मो᳚ ऽग्निष्टो॒मः प॑ञ्चविꣳ॒॒शः प॑ञ्चविꣳ॒॒शो᳚ ऽग्निष्टो॒मो भ॑वति भव त्यग्निष्टो॒मः प॑ञ्चविꣳ॒॒शः प॑ञ्चविꣳ॒॒शो᳚ ऽग्निष्टो॒मो भ॑वति । \newline
48. प॒ञ्च॒विꣳ॒॒श इति॑ पञ्च - विꣳ॒॒शः । \newline
49. अ॒ग्नि॒ष्टो॒मो भ॑वति भव त्यग्निष्टो॒मो᳚ ऽग्निष्टो॒मो भ॑वति प्र॒जाप॑तेः प्र॒जाप॑तेर् भव त्यग्निष्टो॒मो᳚ ऽग्निष्टो॒मो भ॑वति प्र॒जाप॑तेः । \newline
50. अ॒ग्नि॒ष्टो॒म इत्य॑ग्नि - स्तो॒मः । \newline
51. भ॒व॒ति॒ प्र॒जाप॑तेः प्र॒जाप॑तेर् भवति भवति प्र॒जाप॑ते॒ राप्त्या॒ आप्त्यै᳚ प्र॒जाप॑तेर् भवति भवति प्र॒जाप॑ते॒ राप्त्यै᳚ । \newline
52. प्र॒जाप॑ते॒ राप्त्या॒ आप्त्यै᳚ प्र॒जाप॑तेः प्र॒जाप॑ते॒ राप्त्यै॑ महाव्र॒तवा᳚न् महाव्र॒तवा॒ नाप्त्यै᳚ प्र॒जाप॑तेः प्र॒जाप॑ते॒ राप्त्यै॑ महाव्र॒तवान्॑ । \newline
53. प्र॒जाप॑ते॒रिति॑ प्र॒जा - प॒तेः॒ । \newline
54. आप्त्यै॑ महाव्र॒तवा᳚न् महाव्र॒तवा॒ नाप्त्या॒ आप्त्यै॑ महाव्र॒तवा॑ न॒न्नाद्य॑स्या॒ न्नाद्य॑स्य महाव्र॒तवा॒ नाप्त्या॒ आप्त्यै॑ महाव्र॒तवा॑ न॒न्नाद्य॑स्य । \newline
55. म॒हा॒व्र॒तवा॑ न॒न्नाद्य॑स्या॒ न्नाद्य॑स्य महाव्र॒तवा᳚न् महाव्र॒तवा॑ न॒न्नाद्य॒स्या व॑रुद्ध्या॒ अव॑रुद्ध्या अ॒न्नाद्य॑स्य महाव्र॒तवा᳚न् महाव्र॒तवा॑ न॒न्नाद्य॒स्या व॑रुद्ध्यै । \newline
56. म॒हा॒व्र॒तवा॒निति॑ महाव्र॒त - वा॒न् । \newline
57. अ॒न्नाद्य॒स्या व॑रुद्ध्या॒ अव॑रुद्ध्या अ॒न्नाद्य॑स्या॒ न्नाद्य॒स्या व॑रुद्ध्यै विश्व॒जिद् वि॑श्व॒जि दव॑रुद्ध्या अ॒न्नाद्य॑स्या॒ न्नाद्य॒स्या व॑रुद्ध्यै विश्व॒जित् । \newline
58. अ॒न्नाद्य॒स्येत्य॑न्न - अद्य॑स्य । \newline
59. अव॑रुद्ध्यै विश्व॒जिद् वि॑श्व॒जि दव॑रुद्ध्या॒ अव॑रुद्ध्यै विश्व॒जिथ् सर्व॑पृष्ठः॒ सर्व॑पृष्ठो विश्व॒जि दव॑रुद्ध्या॒ अव॑रुद्ध्यै विश्व॒जिथ् सर्व॑पृष्ठः । \newline
60. अव॑रुद्ध्या॒ इत्यव॑ - रु॒द्ध्यै॒ । \newline
61. वि॒श्व॒जिथ् सर्व॑पृष्ठः॒ सर्व॑पृष्ठो विश्व॒जिद् वि॑श्व॒जिथ् सर्व॑पृष्ठो ऽतिरा॒त्रो॑ ऽतिरा॒त्रः सर्व॑पृष्ठो विश्व॒जिद् वि॑श्व॒जिथ् सर्व॑पृष्ठो ऽतिरा॒त्रः । \newline
62. वि॒श्व॒जिदिति॑ विश्व - जित् । \newline
63. सर्व॑पृष्ठो ऽतिरा॒त्रो॑ ऽतिरा॒त्रः सर्व॑पृष्ठः॒ सर्व॑पृष्ठो ऽतिरा॒त्रो भ॑वति भव त्यतिरा॒त्रः सर्व॑पृष्ठः॒ सर्व॑पृष्ठो ऽतिरा॒त्रो भ॑वति । \newline
64. सर्व॑पृष्ठ॒ इति॒ सर्व॑ - पृ॒ष्ठः॒ । \newline
65. अ॒ति॒रा॒त्रो भ॑वति भव त्यतिरा॒त्रो॑ ऽतिरा॒त्रो भ॑वति॒ सर्व॑स्य॒ सर्व॑स्य भव त्यतिरा॒त्रो॑ ऽतिरा॒त्रो भ॑वति॒ सर्व॑स्य । \newline
66. अ॒ति॒रा॒त्र इत्य॑ति - रा॒त्रः । \newline
67. भ॒व॒ति॒ सर्व॑स्य॒ सर्व॑स्य भवति भवति॒ सर्व॑स्या॒ भिजि॑त्या अ॒भिजि॑त्यै॒ सर्व॑स्य भवति भवति॒ सर्व॑स्या॒ भिजि॑त्यै । \newline
68. सर्व॑स्या॒ भिजि॑त्या अ॒भिजि॑त्यै॒ सर्व॑स्य॒ सर्व॑स्या॒ भिजि॑त्यै । \newline
69. अ॒भिजि॑त्या॒ इत्य॒भि - जि॒त्यै॒ । \newline
\pagebreak
\markright{ TS 7.1.11.1  \hfill https://www.vedavms.in \hfill}

\section{ TS 7.1.11.1 }

\textbf{TS 7.1.11.1 } \newline
\textbf{Samhita Paata} \newline

दे॒वस्य॑त्वा सवि॒तुः प्र॑स॒वे᳚ऽश्विनो᳚र्बा॒हुभ्यां᳚ पू॒ष्णो हस्ता᳚भ्या॒मा द॑द इ॒माम॑गृभ्णन् रश॒नामृ॒तस्य॒ पूर्व॒ आयु॑षि वि॒दथे॑षु क॒व्या । तया॑ दे॒वाः सु॒तमा ब॑भूवुर् ऋ॒तस्य॒ सामन्थ᳚-स॒रमा॒रप॑न्ती ॥अ॒भि॒धा अ॑सि॒ भुव॑नमसि य॒न्ताऽसि॑ ध॒र्ताऽसि॒ सो᳚ऽग्निं ॅवै᳚श्वान॒रꣳ सप्र॑थसं गच्छ॒ स्वाहा॑कृतः पृथि॒व्यां ॅय॒न्ता राड् य॒न्ताऽसि॒ यम॑नो ध॒र्ताऽसि॑ ध॒रुणः॑ ( ) कृ॒ष्यै त्वा॒ क्षेमा॑य त्वा र॒य्यै त्वा॒ पोषा॑य त्वा पृथि॒व्यै त्वा॒ ऽन्तरि॑क्षाय त्वा दि॒वे त्वा॑ स॒ते त्वाऽस॑ते त्वा॒द्भ्यस्त्वौ-ष॑धीभ्यस्त्वा॒ विश्वे᳚भ्यस्त्वा भू॒तेभ्यः॑ ॥ \newline

\textbf{Pada Paata} \newline

दे॒वस्य॑ । त्वा॒ । स॒वि॒तुः । प्र॒स॒व इति॑ प्र - स॒वे । अ॒श्विनोः᳚ । बा॒हुभ्या॒मिति॑ बा॒हु - भ्या॒म् । पू॒ष्णः । हस्ता᳚भ्याम् । एति॑ । द॒दे॒ । इ॒माम् । अ॒गृ॒भ्ण॒न्न् । र॒श॒नाम् । ऋ॒तस्य॑ । पूर्वे᳚ । आयु॑षि । वि॒दथे॑षु । क॒व्या ॥ तया᳚ । दे॒वाः । सु॒तम् । एति॑ । ब॒भू॒वुः॒ । ऋ॒तस्य॑ । सामन्न्॑ । स॒रम् । आ॒रप॒न्तीत्या᳚ - रप॑न्ती ॥ अ॒भि॒धा इत्य॑भि - धाः । अ॒सि॒ । भुव॑नम् । अ॒सि॒ । य॒न्ता । अ॒सि॒ । ध॒र्ता । अ॒सि॒ । सः । अ॒ग्निम् । वै॒श्वा॒न॒रम् । सप्र॑थस॒मिति॒ स - प्र॒थ॒स॒म् । ग॒च्छ॒ । स्वाहा॑कृत॒ इति॒ स्वाहा᳚ - कृ॒तः॒ । पृ॒थि॒व्याम् । य॒न्ता । राट् । य॒न्ता । अ॒सि॒ । यम॑नः । ध॒र्ता । अ॒सि॒ । ध॒रुणः॑ ( ) । कृ॒ष्यै । त्वा॒ । क्षेमा॑य । त्वा॒ । र॒य्यै । त्वा॒ । पोषा॑य । त्वा॒ । पृ॒थि॒व्यै । त्वा॒ । अ॒न्तरि॑क्षाय । त्वा॒ । दि॒वे । त्वा॒ । स॒ते । त्वा॒ । अस॑ते । त्वा॒ । अ॒द्भ्य इत्य॑त् - भ्यः । त्वा॒ । ओष॑धीभ्य॒ इत्योष॑धि - भ्यः॒ । त्वा॒ । विश्वे᳚भ्यः । त्वा॒ । भू॒तेभ्यः॑ ॥  \newline


\textbf{Krama Paata} \newline

दे॒वस्य॑ त्वा । त्वा॒ स॒वि॒तुः । स॒वि॒तुः प्र॑स॒वे । प्र॒स॒वे᳚ऽश्विनोः᳚ । प्र॒स॒व इति॑ प्र - स॒वे । अ॒श्विनो᳚र् बा॒हुभ्या᳚म् । बा॒हुभ्या᳚म् पू॒ष्णः । बा॒हुभ्या॒मिति॑ बा॒हु - भ्या॒म् । पू॒ष्णो हस्ता᳚भ्याम् । हस्ता᳚भ्या॒मा । आ द॑दे । द॒द॒ इ॒माम् । इ॒माम॑गृभ्णन्न् । अ॒गृ॒भ्ण॒न् र॒श॒नाम् । र॒श॒नामृ॒तस्य॑ । ऋ॒तस्य॒ पूर्वे᳚ । पूर्व॒ आयु॑षि । आयु॑षि वि॒दथे॑षु । वि॒दथे॑षु क॒व्या । क॒व्येति॑ क॒व्या ॥ तया॑ दे॒वाः । दे॒वाः सु॒तम् । सु॒तमा । आ ब॑भूवुः । ब॒भू॒वु॒र्.॒ ऋ॒तस्य॑ । ऋ॒तस्य॒ सामन्न्॑ । साम᳚न्थ् स॒रम् । स॒रमा॒रप॑न्ती । आ॒रप॒न्तीत्या᳚ - रप॑न्ती ॥ अ॒भि॒धा अ॑सि । अ॒भि॒धा इत्य॑भि - धाः । अ॒सि॒ भुव॑नम् । भुव॑नमसि । अ॒सि॒ य॒न्ता । य॒न्ताऽसि॑ । अ॒सि॒ ध॒र्ता । ध॒र्ताऽसि॑ । अ॒सि॒ सः । सो᳚ऽग्निम् । अ॒ग्निम् ॅवै᳚श्वान॒रम् । वै॒श्वा॒न॒रꣳ सप्र॑थसम् । सप्र॑थसम् गच्छ । सप्र॑थस॒मिति॒ स - प्र॒थ॒स॒म् । ग॒च्छ॒ स्वाहा॑कृतः । स्वाहा॑कृतः पृथि॒व्याम् । स्वाहा॑कृत॒ इति॒ स्वाहा᳚ - कृ॒तः॒ । पृ॒थि॒व्याम् ॅय॒न्ता । य॒न्ता राट् । राड्‍ य॒न्ता । य॒न्ताऽसि॑ । अ॒सि॒ यम॑नः । यम॑नो ध॒र्ता । ध॒र्ताऽसि॑ । अ॒सि॒ ध॒रुणः॑ ( ) । ध॒रुणः॑ कृ॒ष्यै । कृ॒ष्यै त्वा᳚ । त्वा॒ क्षेमा॑य । क्षेमा॑य त्वा । त्वा॒ र॒य्यै । र॒य्यै त्वा᳚ । त्वा॒ पोषा॑य । पोषा॑य त्वा । त्वा॒ पृ॒थि॒व्यै । पृ॒थि॒व्यै त्वा᳚ । त्वा॒ऽन्तरि॑क्षाय । अ॒न्तरि॑क्षाय त्वा । त्वा॒ दि॒वे । दि॒वे त्वा᳚ । त्वा॒ स॒ते । स॒ते त्वा᳚ । त्वाऽस॑ते । अस॑ते त्वा । त्वा॒ऽद्भ्यः । अ॒द्भ्यस्त्वा᳚ । अ॒द्भ्य इत्य॑त् - भ्यः । त्वौष॑धीभ्यः । ओष॑धीभ्यस्त्वा । ओष॑धीभ्य॒ इत्योष॑धि - भ्यः॒ । त्वा॒ विश्वे᳚भ्यः । विश्वे᳚भ्यस्त्वा । त्वा॒ भू॒तेभ्यः॑ । भू॒तेभ्य॒ इति॑ भू॒तेभ्यः॑ । \newline

\textbf{Jatai Paata} \newline

1. दे॒वस्य॑ त्वा त्वा दे॒वस्य॑ दे॒वस्य॑ त्वा । \newline
2. त्वा॒ स॒वि॒तुः स॑वि॒तु स्त्वा᳚ त्वा सवि॒तुः । \newline
3. स॒वि॒तुः प्र॑स॒वे प्र॑स॒वे स॑वि॒तुः स॑वि॒तुः प्र॑स॒वे । \newline
4. प्र॒स॒वे᳚ ऽश्विनो॑ र॒श्विनोः᳚ प्रस॒वे प्र॑स॒वे᳚ ऽश्विनोः᳚ । \newline
5. प्र॒स॒व इति॑ प्र - स॒वे । \newline
6. अ॒श्विनो᳚र् बा॒हुभ्या᳚म् बा॒हुभ्या॑ म॒श्विनो॑ र॒श्विनो᳚र् बा॒हुभ्या᳚म् । \newline
7. बा॒हुभ्या᳚म् पू॒ष्णः पू॒ष्णो बा॒हुभ्या᳚म् बा॒हुभ्या᳚म् पू॒ष्णः । \newline
8. बा॒हुभ्या॒मिति॑ बा॒हु - भ्या॒म् । \newline
9. पू॒ष्णो हस्ता᳚भ्याꣳ॒॒ हस्ता᳚भ्याम् पू॒ष्णः पू॒ष्णो हस्ता᳚भ्याम् । \newline
10. हस्ता᳚भ्या॒ मा हस्ता᳚भ्याꣳ॒॒ हस्ता᳚भ्या॒ मा । \newline
11. आ द॑दे दद॒ आ द॑दे । \newline
12. द॒द॒ इ॒मा मि॒माम् द॑दे दद इ॒माम् । \newline
13. इ॒मा म॑गृभ्णन् नगृभ्णन् नि॒मा मि॒मा म॑गृभ्णन्न् । \newline
14. अ॒गृ॒भ्ण॒न् र॒श॒नाꣳ र॑श॒ना म॑गृभ्णन् नगृभ्णन् रश॒नाम् । \newline
15. र॒श॒ना मृ॒तस्य॒ र्‌तस्य॑ रश॒नाꣳ र॑श॒ना मृ॒तस्य॑ । \newline
16. ऋ॒तस्य॒ पूर्वे॒ पूर्व॑ ऋ॒तस्य॒ र्‌तस्य॒ पूर्वे᳚ । \newline
17. पूर्व॒ आयु॒ ष्यायु॑षि॒ पूर्वे॒ पूर्व॒ आयु॑षि । \newline
18. आयु॑षि वि॒दथे॑षु वि॒दथे॒ ष्वायु॒ ष्यायु॑षि वि॒दथे॑षु । \newline
19. वि॒दथे॑षु क॒व्या क॒व्या वि॒दथे॑षु वि॒दथे॑षु क॒व्या । \newline
20. क॒व्येति॑ क॒व्या । \newline
21. तया॑ दे॒वा दे॒वा स्तया॒ तया॑ दे॒वाः । \newline
22. दे॒वाः सु॒तꣳ सु॒तम् दे॒वा दे॒वाः सु॒तम् । \newline
23. सु॒त मा सु॒तꣳ सु॒त मा । \newline
24. आ ब॑भूवुर् बभूवु॒रा ब॑भूवुः । \newline
25. ब॒भू॒वु॒र्॒. ऋ॒तस्य॒ र्‌तस्य॑ बभूवुर् बभूवुर्. ऋ॒तस्य॑ । \newline
26. ऋ॒तस्य॒ साम॒न् थ्साम॑न् नृ॒तस्य॒ र्‌तस्य॒ सामन्न्॑ । \newline
27. सामन्᳚ थ्स॒रꣳ स॒रꣳ साम॒न् थ्सामन्᳚ थ्स॒रम् । \newline
28. स॒र मा॒रप॑ न्त्या॒रप॑न्ती स॒रꣳ स॒र मा॒रप॑न्ती । \newline
29. आ॒रप॒न्तीत्या᳚ - रप॑न्ती । \newline
30. अ॒भि॒धा अ॑स्यस्य भि॒धा अ॑भि॒धा अ॑सि । \newline
31. अ॒भि॒धा इत्य॑भि - धाः । \newline
32. अ॒सि॒ भुव॑न॒म् भुव॑न मस्यसि॒ भुव॑नम् । \newline
33. भुव॑न मस्यसि॒ भुव॑न॒म् भुव॑न मसि । \newline
34. अ॒सि॒ य॒न्ता य॒न्ता ऽस्य॑सि य॒न्ता । \newline
35. य॒न्ता ऽस्य॑सि य॒न्ता य॒न्ता ऽसि॑ । \newline
36. अ॒सि॒ ध॒र्ता ध॒र्ता ऽस्य॑सि ध॒र्ता । \newline
37. ध॒र्ता ऽस्य॑सि ध॒र्ता ध॒र्ता ऽसि॑ । \newline
38. अ॒सि॒ स सो᳚ ऽस्यसि॒ सः । \newline
39. सो᳚ ऽग्नि म॒ग्निꣳ स सो᳚ ऽग्निम् । \newline
40. अ॒ग्निं ॅवै᳚श्वान॒रं ॅवै᳚श्वान॒र म॒ग्नि म॒ग्निं ॅवै᳚श्वान॒रम् । \newline
41. वै॒श्वा॒न॒रꣳ सप्र॑थसꣳ॒॒ सप्र॑थसं ॅवैश्वान॒रं ॅवै᳚श्वान॒रꣳ सप्र॑थसम् । \newline
42. सप्र॑थसम् गच्छ गच्छ॒ सप्र॑थसꣳ॒॒ सप्र॑थसम् गच्छ । \newline
43. सप्र॑थस॒मिति॒ स - प्र॒थ॒स॒म् । \newline
44. ग॒च्छ॒ स्वाहा॑कृतः॒ स्वाहा॑कृतो गच्छ गच्छ॒ स्वाहा॑कृतः । \newline
45. स्वाहा॑कृतः पृथि॒व्याम् पृ॑थि॒व्याꣳ स्वाहा॑कृतः॒ स्वाहा॑कृतः पृथि॒व्याम् । \newline
46. स्वाहा॑कृत॒ इति॒ स्वाहा᳚ - कृ॒तः॒ । \newline
47. पृ॒थि॒व्यां ॅय॒न्ता य॒न्ता पृ॑थि॒व्याम् पृ॑थि॒व्यां ॅय॒न्ता । \newline
48. य॒न्ता राड् राड् य॒न्ता य॒न्ता राट् । \newline
49. राड् य॒न्ता य॒न्ता राड् राड् य॒न्ता । \newline
50. य॒न्ता ऽस्य॑सि य॒न्ता य॒न्ता ऽसि॑ । \newline
51. अ॒सि॒ यम॑नो॒ यम॑नो ऽस्यसि॒ यम॑नः । \newline
52. यम॑नो ध॒र्ता ध॒र्ता यम॑नो॒ यम॑नो ध॒र्ता । \newline
53. ध॒र्ता ऽस्य॑सि ध॒र्ता ध॒र्ता ऽसि॑ । \newline
54. अ॒सि॒ ध॒रुणो॑ ध॒रुणो᳚ ऽस्यसि ध॒रुणः॑ । \newline
55. ध॒रुणः॑ कृ॒ष्यै कृ॒ष्यै ध॒रुणो॑ ध॒रुणः॑ कृ॒ष्यै । \newline
56. कृ॒ष्यै त्वा᳚ त्वा कृ॒ष्यै कृ॒ष्यै त्वा᳚ । \newline
57. त्वा॒ क्षेमा॑य॒ क्षेमा॑य त्वा त्वा॒ क्षेमा॑य । \newline
58. क्षेमा॑य त्वा त्वा॒ क्षेमा॑य॒ क्षेमा॑य त्वा । \newline
59. त्वा॒ र॒य्यै र॒य्यै त्वा᳚ त्वा र॒य्यै । \newline
60. र॒य्यै त्वा᳚ त्वा र॒य्यै र॒य्यै त्वा᳚ । \newline
61. त्वा॒ पोषा॑य॒ पोषा॑य त्वा त्वा॒ पोषा॑य । \newline
62. पोषा॑य त्वा त्वा॒ पोषा॑य॒ पोषा॑य त्वा । \newline
63. त्वा॒ पृ॒थि॒व्यै पृ॑थि॒व्यै त्वा᳚ त्वा पृथि॒व्यै । \newline
64. पृ॒थि॒व्यै त्वा᳚ त्वा पृथि॒व्यै पृ॑थि॒व्यै त्वा᳚ । \newline
65. त्वा॒ ऽन्तरि॑क्षाया॒ न्तरि॑क्षाय त्वा त्वा॒ ऽन्तरि॑क्षाय । \newline
66. अ॒न्तरि॑क्षाय त्वा त्वा॒ ऽन्तरि॑क्षाया॒ न्तरि॑क्षाय त्वा । \newline
67. त्वा॒ दि॒वे दि॒वे त्वा᳚ त्वा दि॒वे । \newline
68. दि॒वे त्वा᳚ त्वा दि॒वे दि॒वे त्वा᳚ । \newline
69. त्वा॒ स॒ते स॒ते त्वा᳚ त्वा स॒ते । \newline
70. स॒ते त्वा᳚ त्वा स॒ते स॒ते त्वा᳚ । \newline
71. त्वा ऽस॒ते ऽस॑ते त्वा॒ त्वा ऽस॑ते । \newline
72. अस॑ते त्वा॒ त्वा ऽस॒ते ऽस॑ते त्वा । \newline
73. त्वा॒ ऽद्भ्यो᳚ ऽद्भ्य स्त्वा᳚ त्वा॒ ऽद्भ्यः । \newline
74. अ॒द्भ्य स्त्वा᳚ त्वा॒ ऽद्भ्यो᳚ ऽद्भ्य स्त्वा᳚ । \newline
75. अ॒द्भ्य इत्य॑त् - भ्यः । \newline
76. त्वौष॑धीभ्य॒ ओष॑धीभ्य स्त्वा॒ त्वौष॑धीभ्यः । \newline
77. ओष॑धीभ्य स्त्वा॒ त्वौष॑धीभ्य॒ ओष॑धीभ्य स्त्वा । \newline
78. ओष॑धीभ्य॒ इत्योष॑धि - भ्यः॒ । \newline
79. त्वा॒ विश्वे᳚भ्यो॒ विश्वे᳚भ्य स्त्वा त्वा॒ विश्वे᳚भ्यः । \newline
80. विश्वे᳚भ्य स्त्वा त्वा॒ विश्वे᳚भ्यो॒ विश्वे᳚भ्य स्त्वा । \newline
81. त्वा॒ भू॒तेभ्यो॑ भू॒तेभ्य॑ स्त्वा त्वा भू॒तेभ्यः॑ । \newline
82. भू॒तेभ्य॒ इति॑ भू॒तेभ्यः॑ । \newline

\textbf{Ghana Paata } \newline

1. दे॒वस्य॑ त्वा त्वा दे॒वस्य॑ दे॒वस्य॑ त्वा सवि॒तुः स॑वि॒तु स्त्वा॑ दे॒वस्य॑ दे॒वस्य॑ त्वा सवि॒तुः । \newline
2. त्वा॒ स॒वि॒तुः स॑वि॒तु स्त्वा᳚ त्वा सवि॒तुः प्र॑स॒वे प्र॑स॒वे स॑वि॒तु स्त्वा᳚ त्वा सवि॒तुः प्र॑स॒वे । \newline
3. स॒वि॒तुः प्र॑स॒वे प्र॑स॒वे स॑वि॒तुः स॑वि॒तुः प्र॑स॒वे᳚ ऽश्विनो॑ र॒श्विनोः᳚ प्रस॒वे स॑वि॒तुः स॑वि॒तुः प्र॑स॒वे᳚ ऽश्विनोः᳚ । \newline
4. प्र॒स॒वे᳚ ऽश्विनो॑ र॒श्विनोः᳚ प्रस॒वे प्र॑स॒वे᳚ ऽश्विनो᳚र् बा॒हुभ्या᳚म् बा॒हुभ्या॑ म॒श्विनोः᳚ प्रस॒वे प्र॑स॒वे᳚ ऽश्विनो᳚र् बा॒हुभ्या᳚म् । \newline
5. प्र॒स॒व इति॑ प्र - स॒वे । \newline
6. अ॒श्विनो᳚र् बा॒हुभ्या᳚म् बा॒हुभ्या॑ म॒श्विनो॑ र॒श्विनो᳚र् बा॒हुभ्या᳚म् पू॒ष्णः पू॒ष्णो बा॒हुभ्या॑ म॒श्विनो॑ र॒श्विनो᳚र् बा॒हुभ्या᳚म् पू॒ष्णः । \newline
7. बा॒हुभ्या᳚म् पू॒ष्णः पू॒ष्णो बा॒हुभ्या᳚म् बा॒हुभ्या᳚म् पू॒ष्णो हस्ता᳚भ्याꣳ॒॒ हस्ता᳚भ्याम् पू॒ष्णो बा॒हुभ्या᳚म् बा॒हुभ्या᳚म् पू॒ष्णो हस्ता᳚भ्याम् । \newline
8. बा॒हुभ्या॒मिति॑ बा॒हु - भ्या॒म् । \newline
9. पू॒ष्णो हस्ता᳚भ्याꣳ॒॒ हस्ता᳚भ्याम् पू॒ष्णः पू॒ष्णो हस्ता᳚भ्या॒ मा हस्ता᳚भ्याम् पू॒ष्णः पू॒ष्णो हस्ता᳚भ्या॒ मा । \newline
10. हस्ता᳚भ्या॒ मा हस्ता᳚भ्याꣳ॒॒ हस्ता᳚भ्या॒ मा द॑दे दद॒ आ हस्ता᳚भ्याꣳ॒॒ हस्ता᳚भ्या॒ मा द॑दे । \newline
11. आ द॑दे दद॒ आ द॑द इ॒मा मि॒माम् द॑द॒ आ द॑द इ॒माम् । \newline
12. द॒द॒ इ॒मा मि॒माम् द॑दे दद इ॒मा म॑गृभ्णन् नगृभ्णन् नि॒माम् द॑दे दद इ॒मा म॑गृभ्णन्न् । \newline
13. इ॒मा म॑गृभ्णन् नगृभ्णन् नि॒मा मि॒मा म॑गृभ्णन् रश॒नाꣳ र॑श॒ना म॑गृभ्णन् नि॒मा मि॒मा म॑गृभ्णन् रश॒नाम् । \newline
14. अ॒गृ॒भ्ण॒न् र॒श॒नाꣳ र॑श॒ना म॑गृभ्णन् नगृभ्णन् रश॒ना मृ॒तस्य॒ र्‌तस्य॑ रश॒ना म॑गृभ्णन् नगृभ्णन् रश॒ना मृ॒तस्य॑ । \newline
15. र॒श॒ना मृ॒तस्य॒ र्‌तस्य॑ रश॒नाꣳ र॑श॒ना मृ॒तस्य॒ पूर्वे॒ पूर्व॑ ऋ॒तस्य॑ रश॒नाꣳ र॑श॒ना मृ॒तस्य॒ पूर्वे᳚ । \newline
16. ऋ॒तस्य॒ पूर्वे॒ पूर्व॑ ऋ॒तस्य॒ र्‌तस्य॒ पूर्व॒ आयु॒ ष्यायु॑षि॒ पूर्व॑ ऋ॒तस्य॒ र्‌तस्य॒ पूर्व॒ आयु॑षि । \newline
17. पूर्व॒ आयु॒ ष्यायु॑षि॒ पूर्वे॒ पूर्व॒ आयु॑षि वि॒दथे॑षु वि॒दथे॒ ष्वायु॑षि॒ पूर्वे॒ पूर्व॒ आयु॑षि वि॒दथे॑षु । \newline
18. आयु॑षि वि॒दथे॑षु वि॒दथे॒ ष्वायु॒ ष्यायु॑षि वि॒दथे॑षु क॒व्या क॒व्या वि॒दथे॒ ष्वायु॒ ष्यायु॑षि वि॒दथे॑षु क॒व्या । \newline
19. वि॒दथे॑षु क॒व्या क॒व्या वि॒दथे॑षु वि॒दथे॑षु क॒व्या । \newline
20. क॒व्येति॑ क॒व्या । \newline
21. तया॑ दे॒वा दे॒वा स्तया॒ तया॑ दे॒वाः सु॒तꣳ सु॒तम् दे॒वा स्तया॒ तया॑ दे॒वाः सु॒तम् । \newline
22. दे॒वाः सु॒तꣳ सु॒तम् दे॒वा दे॒वाः सु॒त मा सु॒तम् दे॒वा दे॒वाः सु॒त मा । \newline
23. सु॒त मा सु॒तꣳ सु॒त मा ब॑भूवुर् बभूवु॒रा सु॒तꣳ सु॒त मा ब॑भूवुः । \newline
24. आ ब॑भूवुर् बभूवु॒रा ब॑भूवुर्. ऋ॒तस्य॒ र्‌तस्य॑ बभूवु॒रा ब॑भूवुर्. ऋ॒तस्य॑ । \newline
25. ब॒भू॒वु॒र्॒. ऋ॒तस्य॒ र्‌तस्य॑ बभूवुर् बभूवुर्. ऋ॒तस्य॒ साम॒न् थ्साम॑न् नृ॒तस्य॑ बभूवुर् बभूवुर्. ऋ॒तस्य॒ सामन्न्॑ । \newline
26. ऋ॒तस्य॒ साम॒न् थ्साम॑न् नृ॒तस्य॒ र्‌तस्य॒ सामन्᳚ थ्स॒रꣳ स॒रꣳ साम॑न् नृ॒तस्य॒ र्‌तस्य॒ सामन्᳚ थ्स॒रम् । \newline
27. सामन्᳚ थ्स॒रꣳ स॒रꣳ साम॒न् थ्सामन्᳚ थ्स॒र मा॒रप॑ न्त्या॒रप॑न्ती स॒रꣳ साम॒न् थ्सामन्᳚ थ्स॒र मा॒रप॑न्ती । \newline
28. स॒र मा॒रप॑ न्त्या॒रप॑न्ती स॒रꣳ स॒र मा॒रप॑न्ती । \newline
29. आ॒रप॒न्तीत्या᳚ - रप॑न्ती । \newline
30. अ॒भि॒धा अ॑स्यस्य भि॒धा अ॑भि॒धा अ॑सि॒ भुव॑न॒म् भुव॑न मस्यभि॒धा अ॑भि॒धा अ॑सि॒ भुव॑नम् । \newline
31. अ॒भि॒धा इत्य॑भि - धाः । \newline
32. अ॒सि॒ भुव॑न॒म् भुव॑न मस्यसि॒ भुव॑न मस्यसि॒ भुव॑न मस्यसि॒ भुव॑न मसि । \newline
33. भुव॑न मस्यसि॒ भुव॑न॒म् भुव॑न मसि य॒न्ता य॒न्ता ऽसि॒ भुव॑न॒म् भुव॑न मसि य॒न्ता । \newline
34. अ॒सि॒ य॒न्ता य॒न्ता ऽस्य॑सि य॒न्ता ऽस्य॑सि य॒न्ता ऽस्य॑सि य॒न्ता ऽसि॑ । \newline
35. य॒न्ता ऽस्य॑सि य॒न्ता य॒न्ता ऽसि॑ ध॒र्ता ध॒र्ता ऽसि॑ य॒न्ता य॒न्ता ऽसि॑ ध॒र्ता । \newline
36. अ॒सि॒ ध॒र्ता ध॒र्ता ऽस्य॑सि ध॒र्ता ऽस्य॑सि ध॒र्ता ऽस्य॑सि ध॒र्ता ऽसि॑ । \newline
37. ध॒र्ता ऽस्य॑सि ध॒र्ता ध॒र्ता ऽसि॒ स सो॑ ऽसि ध॒र्ता ध॒र्ता ऽसि॒ सः । \newline
38. अ॒सि॒ स सो᳚ ऽस्यसि॒ सो᳚ ऽग्नि म॒ग्निꣳ सो᳚ ऽस्यसि॒ सो᳚ ऽग्निम् । \newline
39. सो᳚ ऽग्नि म॒ग्निꣳ स सो᳚ ऽग्निं ॅवै᳚श्वान॒रं ॅवै᳚श्वान॒र म॒ग्निꣳ स सो᳚ ऽग्निं ॅवै᳚श्वान॒रम् । \newline
40. अ॒ग्निं ॅवै᳚श्वान॒रं ॅवै᳚श्वान॒र म॒ग्नि म॒ग्निं ॅवै᳚श्वान॒रꣳ सप्र॑थसꣳ॒॒ सप्र॑थसं ॅवैश्वान॒र म॒ग्नि म॒ग्निं ॅवै᳚श्वान॒रꣳ सप्र॑थसम् । \newline
41. वै॒श्वा॒न॒रꣳ सप्र॑थसꣳ॒॒ सप्र॑थसं ॅवैश्वान॒रं ॅवै᳚श्वान॒रꣳ सप्र॑थसम् गच्छ गच्छ॒ सप्र॑थसं ॅवैश्वान॒रं ॅवै᳚श्वान॒रꣳ सप्र॑थसम् गच्छ । \newline
42. सप्र॑थसम् गच्छ गच्छ॒ सप्र॑थसꣳ॒॒ सप्र॑थसम् गच्छ॒ स्वाहा॑कृतः॒ स्वाहा॑कृतो गच्छ॒ सप्र॑थसꣳ॒॒ सप्र॑थसम् गच्छ॒ स्वाहा॑कृतः । \newline
43. सप्र॑थस॒मिति॒ स - प्र॒थ॒स॒म् । \newline
44. ग॒च्छ॒ स्वाहा॑कृतः॒ स्वाहा॑कृतो गच्छ गच्छ॒ स्वाहा॑कृतः पृथि॒व्याम् पृ॑थि॒व्याꣳ स्वाहा॑कृतो गच्छ गच्छ॒ स्वाहा॑कृतः पृथि॒व्याम् । \newline
45. स्वाहा॑कृतः पृथि॒व्याम् पृ॑थि॒व्याꣳ स्वाहा॑कृतः॒ स्वाहा॑कृतः पृथि॒व्यां ॅय॒न्ता य॒न्ता पृ॑थि॒व्याꣳ स्वाहा॑कृतः॒ स्वाहा॑कृतः पृथि॒व्यां ॅय॒न्ता । \newline
46. स्वाहा॑कृत॒ इति॒ स्वाहा᳚ - कृ॒तः॒ । \newline
47. पृ॒थि॒व्यां ॅय॒न्ता य॒न्ता पृ॑थि॒व्याम् पृ॑थि॒व्यां ॅय॒न्ता राड् राड् य॒न्ता पृ॑थि॒व्याम् पृ॑थि॒व्यां ॅय॒न्ता राट् । \newline
48. य॒न्ता राड् राड् य॒न्ता य॒न्ता राड् य॒न्ता य॒न्ता राड् य॒न्ता य॒न्ता राड् य॒न्ता । \newline
49. राड् य॒न्ता य॒न्ता राड् राड् य॒न्ता ऽस्य॑सि य॒न्ता राड् राड् य॒न्ता ऽसि॑ । \newline
50. य॒न्ता ऽस्य॑सि य॒न्ता य॒न्ता ऽसि॒ यम॑नो॒ यम॑नो ऽसि य॒न्ता य॒न्ता ऽसि॒ यम॑नः । \newline
51. अ॒सि॒ यम॑नो॒ यम॑नो ऽस्यसि॒ यम॑नो ध॒र्ता ध॒र्ता यम॑नो ऽस्यसि॒ यम॑नो ध॒र्ता । \newline
52. यम॑नो ध॒र्ता ध॒र्ता यम॑नो॒ यम॑नो ध॒र्ता ऽस्य॑सि ध॒र्ता यम॑नो॒ यम॑नो ध॒र्ता ऽसि॑ । \newline
53. ध॒र्ता ऽस्य॑सि ध॒र्ता ध॒र्ता ऽसि॑ ध॒रुणो॑ ध॒रुणो॑ ऽसि ध॒र्ता ध॒र्ता ऽसि॑ ध॒रुणः॑ । \newline
54. अ॒सि॒ ध॒रुणो॑ ध॒रुणो᳚ ऽस्यसि ध॒रुणः॑ कृ॒ष्यै कृ॒ष्यै ध॒रुणो᳚ ऽस्यसि ध॒रुणः॑ कृ॒ष्यै । \newline
55. ध॒रुणः॑ कृ॒ष्यै कृ॒ष्यै ध॒रुणो॑ ध॒रुणः॑ कृ॒ष्यै त्वा᳚ त्वा कृ॒ष्यै ध॒रुणो॑ ध॒रुणः॑ कृ॒ष्यै त्वा᳚ । \newline
56. कृ॒ष्यै त्वा᳚ त्वा कृ॒ष्यै कृ॒ष्यै त्वा॒ क्षेमा॑य॒ क्षेमा॑य त्वा कृ॒ष्यै कृ॒ष्यै त्वा॒ क्षेमा॑य । \newline
57. त्वा॒ क्षेमा॑य॒ क्षेमा॑य त्वा त्वा॒ क्षेमा॑य त्वा त्वा॒ क्षेमा॑य त्वा त्वा॒ क्षेमा॑य त्वा । \newline
58. क्षेमा॑य त्वा त्वा॒ क्षेमा॑य॒ क्षेमा॑य त्वा र॒य्यै र॒य्यै त्वा॒ क्षेमा॑य॒ क्षेमा॑य त्वा र॒य्यै । \newline
59. त्वा॒ र॒य्यै र॒य्यै त्वा᳚ त्वा र॒य्यै त्वा᳚ त्वा र॒य्यै त्वा᳚ त्वा र॒य्यै त्वा᳚ । \newline
60. र॒य्यै त्वा᳚ त्वा र॒य्यै र॒य्यै त्वा॒ पोषा॑य॒ पोषा॑य त्वा र॒य्यै र॒य्यै त्वा॒ पोषा॑य । \newline
61. त्वा॒ पोषा॑य॒ पोषा॑य त्वा त्वा॒ पोषा॑य त्वा त्वा॒ पोषा॑य त्वा त्वा॒ पोषा॑य त्वा । \newline
62. पोषा॑य त्वा त्वा॒ पोषा॑य॒ पोषा॑य त्वा पृथि॒व्यै पृ॑थि॒व्यै त्वा॒ पोषा॑य॒ पोषा॑य त्वा पृथि॒व्यै । \newline
63. त्वा॒ पृ॒थि॒व्यै पृ॑थि॒व्यै त्वा᳚ त्वा पृथि॒व्यै त्वा᳚ त्वा पृथि॒व्यै त्वा᳚ त्वा पृथि॒व्यै त्वा᳚ । \newline
64. पृ॒थि॒व्यै त्वा᳚ त्वा पृथि॒व्यै पृ॑थि॒व्यै त्वा॒ ऽन्तरि॑क्षाया॒ न्तरि॑क्षाय त्वा पृथि॒व्यै पृ॑थि॒व्यै त्वा॒ ऽन्तरि॑क्षाय । \newline
65. त्वा॒ ऽन्तरि॑क्षाया॒ न्तरि॑क्षाय त्वा त्वा॒ ऽन्तरि॑क्षाय त्वा त्वा॒ ऽन्तरि॑क्षाय त्वा त्वा॒ ऽन्तरि॑क्षाय त्वा । \newline
66. अ॒न्तरि॑क्षाय त्वा त्वा॒ ऽन्तरि॑क्षाया॒ न्तरि॑क्षाय त्वा दि॒वे दि॒वे त्वा॒ ऽन्तरि॑क्षाया॒ न्तरि॑क्षाय त्वा दि॒वे । \newline
67. त्वा॒ दि॒वे दि॒वे त्वा᳚ त्वा दि॒वे त्वा᳚ त्वा दि॒वे त्वा᳚ त्वा दि॒वे त्वा᳚ । \newline
68. दि॒वे त्वा᳚ त्वा दि॒वे दि॒वे त्वा॑ स॒ते स॒ते त्वा॑ दि॒वे दि॒वे त्वा॑ स॒ते । \newline
69. त्वा॒ स॒ते स॒ते त्वा᳚ त्वा स॒ते त्वा᳚ त्वा स॒ते त्वा᳚ त्वा स॒ते त्वा᳚ । \newline
70. स॒ते त्वा᳚ त्वा स॒ते स॒ते त्वा ऽस॒ते ऽस॑ते त्वा स॒ते स॒ते त्वा ऽस॑ते । \newline
71. त्वा ऽस॒ते ऽस॑ते त्वा॒ त्वा ऽस॑ते त्वा॒ त्वा ऽस॑ते त्वा॒ त्वा ऽस॑ते त्वा । \newline
72. अस॑ते त्वा॒ त्वा ऽस॒ते ऽस॑ते त्वा॒ ऽद्भ्यो᳚ ऽद्भ्य स्त्वा ऽस॒ते ऽस॑ते त्वा॒ ऽद्भ्यः । \newline
73. त्वा॒ ऽद्भ्यो᳚ ऽद्भ्य स्त्वा᳚ त्वा॒ ऽद्भ्य स्त्वा᳚ त्वा॒ ऽद्भ्य स्त्वा᳚ त्वा॒ ऽद्भ्य स्त्वा᳚ । \newline
74. अ॒द्भ्य स्त्वा᳚ त्वा॒ ऽद्भ्यो᳚ ऽद्भ्य स्त्वौष॑धीभ्य॒ ओष॑धीभ्य स्त्वा॒ ऽद्भ्यो᳚ ऽद्भ्य स्त्वौष॑धीभ्यः । \newline
75. अ॒द्भ्य इत्य॑त् - भ्यः । \newline
76. त्वौष॑धीभ्य॒ ओष॑धीभ्य स्त्वा॒ त्वौष॑धीभ्य स्त्वा॒ त्वौष॑धीभ्य स्त्वा॒ त्वौष॑धीभ्य स्त्वा । \newline
77. ओष॑धीभ्य स्त्वा॒ त्वौष॑धीभ्य॒ ओष॑धीभ्य स्त्वा॒ विश्वे᳚भ्यो॒ विश्वे᳚भ्य॒ स्त्वौष॑धीभ्य॒ ओष॑धीभ्य स्त्वा॒ विश्वे᳚भ्यः । \newline
78. ओष॑धीभ्य॒ इत्योष॑धि - भ्यः॒ । \newline
79. त्वा॒ विश्वे᳚भ्यो॒ विश्वे᳚भ्य स्त्वा त्वा॒ विश्वे᳚भ्य स्त्वा त्वा॒ विश्वे᳚भ्य स्त्वा त्वा॒ विश्वे᳚भ्य स्त्वा । \newline
80. विश्वे᳚भ्य स्त्वा त्वा॒ विश्वे᳚भ्यो॒ विश्वे᳚भ्य स्त्वा भू॒तेभ्यो॑ भू॒तेभ्य॑ स्त्वा॒ विश्वे᳚भ्यो॒ विश्वे᳚भ्य स्त्वा भू॒तेभ्यः॑ । \newline
81. त्वा॒ भू॒तेभ्यो॑ भू॒तेभ्य॑ स्त्वा त्वा भू॒तेभ्यः॑ । \newline
82. भू॒तेभ्य॒ इति॑ भू॒तेभ्यः॑ । \newline
\pagebreak
\markright{ TS 7.1.12.1  \hfill https://www.vedavms.in \hfill}

\section{ TS 7.1.12.1 }

\textbf{TS 7.1.12.1 } \newline
\textbf{Samhita Paata} \newline

वि॒भूर्मा॒त्रा प्र॒भूः पि॒त्राश्वो॑ऽसि॒ हयो॒ऽस्यत्यो॑ऽसि॒ नरो॒ऽस्यर्वा॑ऽसि॒ सप्ति॑रसि वा॒ज्य॑सि॒ वृषा॑ऽसि नृ॒मणा॑ असि॒ ययु॒र्नामा᳚स्यादि॒त्यानां॒ पत्वान्वि॑ह्य॒ग्नये॒ स्वाहा॒ स्वाहे᳚न्द्रा॒ग्निभ्याꣳ॒॒ स्वाहा᳚ प्र॒जाप॑तये॒ स्वाहा॒ विश्वे᳚भ्यो दे॒वेभ्यः॒ स्वाहा॒ सर्वा᳚भ्यो दे॒वेता᳚भ्य इ॒ह धृतिः॒ स्वाहे॒ह विधृ॑तिः॒ स्वाहे॒ह रन्तिः॒ स्वाहे॒ ( ) -ह रम॑तिः॒ स्वाहा॒ भूर॑सि भु॒वे त्वा॒ भव्या॑य त्वा भविष्य॒ते त्वा॒ विश्वे᳚भ्यस्त्वा भू॒तेभ्यो॒ देवा॑ आशापाला ए॒तं दे॒वेभ्योऽश्वं॒ मेधा॑य॒ प्रोक्षि॑तं गोपायत ॥ \newline

\textbf{Pada Paata} \newline

वि॒भूरिति॑ वि - भूः । मा॒त्रा । प्र॒भूरिति॑ प्र - भूः । पि॒त्रा । अश्वः॑ । अ॒सि॒ । हयः॑ । अ॒सि॒ । अत्यः॑ । अ॒सि॒ । नरः॑ । अ॒सि॒ । अर्वा᳚ । अ॒सि॒ । सप्तिः॑ । अ॒सि॒ । वा॒जी । अ॒सि॒ । वृषा᳚ । अ॒सि॒ । नृ॒मणा॒ इति॑ नृ - मनाः᳚ । अ॒सि॒ । ययुः॑ । नाम॑ । अ॒सि॒ । आ॒दि॒त्याना᳚म् । पत्व॑ । अन्विति॑ । इ॒हि॒ । अ॒ग्नये᳚ । स्वाहा᳚ । स्वाहा᳚ । इ॒न्द्रा॒ग्निभ्या॒मिती᳚न्द्रा॒ग्नि - भ्या॒म् । स्वाहा᳚ । प्र॒जाप॑तय॒ इति॑ प्र॒जा - प॒त॒ये॒ । स्वाहा᳚ । विश्वे᳚भ्यः । दे॒वेभ्यः॑ । स्वाहा᳚ । सर्वा᳚भ्यः । दे॒वेता᳚भ्यः । इ॒ह । धृतिः॑ । स्वाहा᳚ । इ॒ह । विधृ॑ति॒रिति॒ वि - धृ॒तिः॒ । स्वाहा᳚ । इ॒ह । रन्तिः॑ । स्वाहा᳚ ( ) । इ॒ह । रम॑तिः । स्वाहा᳚ । भूः । अ॒सि॒ । भु॒वे । त्वा॒ । भव्या॑य । त्वा॒ । भ॒वि॒ष्य॒ते । त्वा॒ । विश्वे᳚भ्यः । त्वा॒ । भू॒तेभ्यः॑ । देवाः᳚ । आ॒शा॒पा॒ला॒ इत्या॑शा - पा॒लाः॒ । ए॒तम् । दे॒वेभ्यः॑ । अश्व᳚म् । मेधा॑य । प्रोक्षि॑त॒मिति॒ प्र - उ॒क्षि॒त॒म् । गो॒पा॒य॒त॒ ॥  \newline


\textbf{Krama Paata} \newline

वि॒भूर् मा॒त्रा । वि॒भूरिति॑ वि - भूः । मा॒त्रा प्र॒भूः । प्र॒भूः पि॒त्रा । प्र॒भूरिति॑ प्र - भूः । पि॒त्राऽश्वः॑ । अश्वो॑ऽसि । अ॒सि॒ हयः॑ । हयो॑ऽसि । अ॒स्यत्यः॑ । अत्यो॑ऽसि । अ॒सि॒ नरः॑ । नरो॑ऽसि । अ॒स्यर्वा᳚ । अर्वा॑ऽसि । अ॒सि॒ सप्तिः॑ । सप्ति॑रसि । अ॒सि॒ वा॒जी । वा॒ज्य॑सि । अ॒सि॒ वृषा᳚ । वृषा॑ऽसि । अ॒सि॒ नृ॒मणाः᳚ । नृ॒मणा॑ असि । नृ॒मणा॒ इति॑ नृ - मनाः᳚ । अ॒सि॒ ययुः॑ । ययु॒र् नाम॑ । नामा॑सि । अ॒स्या॒दि॒त्याना᳚म् । आ॒दि॒त्याना॒म् पत्व॑ । पत्वानु॑ । अन्वि॑हि । इ॒ह्य॒ग्नये᳚ । अ॒ग्नये॒ स्वाहा᳚ । स्वाहा॒ स्वाहा᳚ । स्वाहे᳚न्द्रा॒ग्निभ्या᳚म् । इ॒न्द्रा॒ग्निभ्याꣳ॒॒ स्वाहा᳚ । इ॒न्द्रा॒ग्निभ्या॒मिती᳚न्द्रा॒ग्नि - भ्या॒म् । स्वाहा᳚ प्र॒जाप॑तये । प्र॒जाप॑तये॒ स्वाहा᳚ । प्र॒जाप॑तय॒ इति॑ प्र॒जा - प॒त॒ये॒ । स्वाहा॒ विश्वे᳚भ्यः । विश्वे᳚भ्यो दे॒वेभ्यः॑ । दे॒वेभ्यः॒ स्वाहा᳚ । स्वाहा॒ सर्वा᳚भ्यः । सर्वा᳚भ्यो दे॒वता᳚भ्यः । दे॒वता᳚भ्य इ॒ह । इ॒ह धृतिः॑ । धृतिः॒ स्वाहा᳚ । स्वाहे॒ह । इ॒ह विधृ॑तिः । विधृ॑तिः॒ स्वाहा᳚ । विधृ॑ति॒रिति॒ वि - धृ॒तिः॒ । स्वाहे॒ह । इ॒ह रन्तिः॑ । रन्तिः॒ स्वाहा᳚ ( ) । स्वाहे॒ह । इ॒ह रम॑तिः । रम॑तिः॒ स्वाहा᳚ । स्वाहा॒ भूः । भूर॑सि । अ॒सि॒ भु॒वे । भु॒वे त्वा᳚ । त्वा॒ भव्या॑य । भव्या॑य त्वा । त्वा॒ भ॒वि॒ष्य॒ते । भ॒वि॒ष्य॒ते त्वा᳚ । त्वा॒ विश्वे᳚भ्यः । विश्वे᳚भ्यस्त्वा । त्वा॒ भू॒तेभ्यः॑ । भू॒तेभ्यो॒ देवाः᳚ । देवा॑ आशापालाः । आ॒शा॒पा॒ला॒ ए॒तम् । आ॒शा॒पा॒ला॒ इत्या॑शा - पा॒लाः॒ । ए॒तम् दे॒वेभ्यः॑ । दे॒वेभ्योऽश्व᳚म् । अश्व॒म् मेधा॑य । मेधा॑य॒ प्रोक्षि॑तम् । प्रोक्षि॑तम् गोपायत । प्रोक्षि॑त॒मिति॒ प्र - उ॒क्षि॒त॒म् । गो॒पा॒य॒तेति॑ गोपायत । \newline

\textbf{Jatai Paata} \newline

1. वि॒भूर् मा॒त्रा मा॒त्रा वि॒भूर् वि॒भूर् मा॒त्रा । \newline
2. वि॒भूरिति॑ वि - भूः । \newline
3. मा॒त्रा प्र॒भूः प्र॒भूर् मा॒त्रा मा॒त्रा प्र॒भूः । \newline
4. प्र॒भूः पि॒त्रा पि॒त्रा प्र॒भूः प्र॒भूः पि॒त्रा । \newline
5. प्र॒भूरिति॑ प्र - भूः । \newline
6. पि॒त्रा ऽश्वो ऽश्वः॑ पि॒त्रा पि॒त्रा ऽश्वः॑ । \newline
7. अश्वो᳚ ऽस्य॒स्य श्वो ऽश्वो॑ ऽसि । \newline
8. अ॒सि॒ हयो॒ हयो᳚ ऽस्यसि॒ हयः॑ । \newline
9. हयो᳚ ऽस्यसि॒ हयो॒ हयो॑ ऽसि । \newline
10. अ॒स्यत्यो ऽत्यो᳚ ऽस्य॒स्य त्यः॑ । \newline
11. अत्यो᳚ ऽस्य॒स्य त्यो ऽत्यो॑ ऽसि । \newline
12. अ॒सि॒ नरो॒ नरो᳚ ऽस्यसि॒ नरः॑ । \newline
13. नरो᳚ ऽस्यसि॒ नरो॒ नरो॑ ऽसि । \newline
14. अ॒स्यर्वा ऽर्वा᳚ ऽस्य॒ स्यर्वा᳚ । \newline
15. अर्वा᳚ ऽस्य॒ स्यर्वा ऽर्वा॑ ऽसि । \newline
16. अ॒सि॒ सप्तिः॒ सप्ति॑ रस्यसि॒ सप्तिः॑ । \newline
17. सप्ति॑ रस्यसि॒ सप्तिः॒ सप्ति॑ रसि । \newline
18. अ॒सि॒ वा॒जी वा॒ज्य॑ स्यसि वा॒जी । \newline
19. वा॒ज्य॑ स्यसि वा॒जी वा॒ज्य॑सि । \newline
20. अ॒सि॒ वृषा॒ वृषा᳚ ऽस्यसि॒ वृषा᳚ । \newline
21. वृषा᳚ ऽस्यसि॒ वृषा॒ वृषा॑ ऽसि । \newline
22. अ॒सि॒ नृ॒मणा॑ नृ॒मणा॑ अस्यसि नृ॒मणाः᳚ । \newline
23. नृ॒मणा॑ अस्यसि नृ॒मणा॑ नृ॒मणा॑ असि । \newline
24. नृ॒मणा॒ इति॑ नृ - मनाः᳚ । \newline
25. अ॒सि॒ ययु॒र् ययु॑ रस्यसि॒ ययुः॑ । \newline
26. ययु॒र् नाम॒ नाम॒ ययु॒र् ययु॒र् नाम॑ । \newline
27. नामा᳚ स्यसि॒ नाम॒ नामा॑सि । \newline
28. अ॒स्या॒ दि॒त्याना॑ मादि॒त्याना॑ मस्यस्या दि॒त्याना᳚म् । \newline
29. आ॒दि॒त्याना॒म् पत्व॒ पत्वा॑ दि॒त्याना॑ मादि॒त्याना॒म् पत्व॑ । \newline
30. पत्वा न्वनु॒ पत्व॒ पत्वानु॑ । \newline
31. अन्वि॑ ही॒ह्यन् वन् वि॑हि । \newline
32. इ॒ह्य॒ ग्नये॒ ऽग्नय॑ इहीह्य॒ ग्नये᳚ । \newline
33. अ॒ग्नये॒ स्वाहा॒ स्वाहा॒ ऽग्नये॒ ऽग्नये॒ स्वाहा᳚ । \newline
34. स्वाहा॒ स्वाहा᳚ । \newline
35. स्वाहे᳚न्द्रा॒ग्निभ्या॑ मिन्द्रा॒ग्निभ्याꣳ॒॒ स्वाहा॒ स्वाहे᳚न्द्रा॒ग्निभ्या᳚म् । \newline
36. इ॒न्द्रा॒ग्निभ्याꣳ॒॒ स्वाहा॒ स्वाहे᳚न्द्रा॒ग्निभ्या॑ मिन्द्रा॒ग्निभ्याꣳ॒॒ स्वाहा᳚ । \newline
37. इ॒न्द्रा॒ग्निभ्या॒मिती᳚न्द्रा॒ग्नि - भ्या॒म् । \newline
38. स्वाहा᳚ प्र॒जाप॑तये प्र॒जाप॑तये॒ स्वाहा॒ स्वाहा᳚ प्र॒जाप॑तये । \newline
39. प्र॒जाप॑तये॒ स्वाहा॒ स्वाहा᳚ प्र॒जाप॑तये प्र॒जाप॑तये॒ स्वाहा᳚ । \newline
40. प्र॒जाप॑तय॒ इति॑ प्र॒जा - प॒त॒ये॒ । \newline
41. स्वाहा॒ विश्वे᳚भ्यो॒ विश्वे᳚भ्यः॒ स्वाहा॒ स्वाहा॒ विश्वे᳚भ्यः । \newline
42. विश्वे᳚भ्यो दे॒वेभ्यो॑ दे॒वेभ्यो॒ विश्वे᳚भ्यो॒ विश्वे᳚भ्यो दे॒वेभ्यः॑ । \newline
43. दे॒वेभ्यः॒ स्वाहा॒ स्वाहा॑ दे॒वेभ्यो॑ दे॒वेभ्यः॒ स्वाहा᳚ । \newline
44. स्वाहा॒ सर्वा᳚भ्यः॒ सर्वा᳚भ्यः॒ स्वाहा॒ स्वाहा॒ सर्वा᳚भ्यः । \newline
45. सर्वा᳚भ्यो दे॒वेता᳚भ्यो दे॒वेता᳚भ्यः॒ सर्वा᳚भ्यः॒ सर्वा᳚भ्यो दे॒वेता᳚भ्यः । \newline
46. दे॒वेता᳚भ्य इ॒हेह दे॒वेता᳚भ्यो दे॒वेता᳚भ्य इ॒ह । \newline
47. इ॒ह धृति॒र् धृति॑ रि॒हेह धृतिः॑ । \newline
48. धृतिः॒ स्वाहा॒ स्वाहा॒ धृति॒र् धृतिः॒ स्वाहा᳚ । \newline
49. स्वाहे॒ हेह स्वाहा॒ स्वाहे॒ह । \newline
50. इ॒ह विधृ॑ति॒र् विधृ॑ति रि॒हेह विधृ॑तिः । \newline
51. विधृ॑तिः॒ स्वाहा॒ स्वाहा॒ विधृ॑ति॒र् विधृ॑तिः॒ स्वाहा᳚ । \newline
52. विधृ॑ति॒रिति॒ वि - धृ॒तिः॒ । \newline
53. स्वाहे॒ हेह स्वाहा॒ स्वाहे॒ह । \newline
54. इ॒ह रन्ती॒ रन्ति॑ रि॒हेह रन्तिः॑ । \newline
55. रन्तिः॒ स्वाहा॒ स्वाहा॒ रन्ती॒ रन्तिः॒ स्वाहा᳚ । \newline
56. स्वाहे॒ हेह स्वाहा॒ स्वाहे॒ह । \newline
57. इ॒ह रम॑ती॒ रम॑ति रि॒हेह रम॑तिः । \newline
58. रम॑तिः॒ स्वाहा॒ स्वाहा॒ रम॑ती॒ रम॑तिः॒ स्वाहा᳚ । \newline
59. स्वाहा॒ भूर् भूः स्वाहा॒ स्वाहा॒ भूः । \newline
60. भू र॑स्यसि॒ भूर् भूर॑सि । \newline
61. अ॒सि॒ भु॒वे भु॒वे᳚ ऽस्यसि भु॒वे । \newline
62. भु॒वे त्वा᳚ त्वा भु॒वे भु॒वे त्वा᳚ । \newline
63. त्वा॒ भव्या॑य॒ भव्या॑य त्वा त्वा॒ भव्या॑य । \newline
64. भव्या॑य त्वा त्वा॒ भव्या॑य॒ भव्या॑य त्वा । \newline
65. त्वा॒ भ॒वि॒ष्य॒ते भ॑विष्य॒ते त्वा᳚ त्वा भविष्य॒ते । \newline
66. भ॒वि॒ष्य॒ते त्वा᳚ त्वा भविष्य॒ते भ॑विष्य॒ते त्वा᳚ । \newline
67. त्वा॒ विश्वे᳚भ्यो॒ विश्वे᳚भ्य स्त्वा त्वा॒ विश्वे᳚भ्यः । \newline
68. विश्वे᳚भ्य स्त्वा त्वा॒ विश्वे᳚भ्यो॒ विश्वे᳚भ्य स्त्वा । \newline
69. त्वा॒ भू॒तेभ्यो॑ भू॒तेभ्य॑ स्त्वा त्वा भू॒तेभ्यः॑ । \newline
70. भू॒तेभ्यो॒ देवा॒ देवा॑ भू॒तेभ्यो॑ भू॒तेभ्यो॒ देवाः᳚ । \newline
71. देवा॑ आशापाला आशापाला॒ देवा॒ देवा॑ आशापालाः । \newline
72. आ॒शा॒पा॒ला॒ ए॒त मे॒त मा॑शापाला आशापाला ए॒तम् । \newline
73. आ॒शा॒पा॒ला॒ इत्या॑शा - पा॒लाः॒ । \newline
74. ए॒तम् दे॒वेभ्यो॑ दे॒वेभ्य॑ ए॒त मे॒तम् दे॒वेभ्यः॑ । \newline
75. दे॒वेभ्यो ऽश्व॒ मश्व॑म् दे॒वेभ्यो॑ दे॒वेभ्यो ऽश्व᳚म् । \newline
76. अश्व॒म् मेधा॑य॒ मेधा॒या श्व॒ मश्व॒म् मेधा॑य । \newline
77. मेधा॑य॒ प्रोक्षि॑त॒म् प्रोक्षि॑त॒म् मेधा॑य॒ मेधा॑य॒ प्रोक्षि॑तम् । \newline
78. प्रोक्षि॑तम् गोपायत गोपायत॒ प्रोक्षि॑त॒म् प्रोक्षि॑तम् गोपायत । \newline
79. प्रोक्षि॑त॒मिति॒ प्र - उ॒क्षि॒त॒म् । \newline
80. गो॒पा॒य॒तेति॑ गोपायत । \newline

\textbf{Ghana Paata } \newline

1. वि॒भूर् मा॒त्रा मा॒त्रा वि॒भूर् वि॒भूर् मा॒त्रा प्र॒भूः प्र॒भूर् मा॒त्रा वि॒भूर् वि॒भूर् मा॒त्रा प्र॒भूः । \newline
2. वि॒भूरिति॑ वि - भूः । \newline
3. मा॒त्रा प्र॒भूः प्र॒भूर् मा॒त्रा मा॒त्रा प्र॒भूः पि॒त्रा पि॒त्रा प्र॒भूर् मा॒त्रा मा॒त्रा प्र॒भूः पि॒त्रा । \newline
4. प्र॒भूः पि॒त्रा पि॒त्रा प्र॒भूः प्र॒भूः पि॒त्रा ऽश्वो ऽश्वः॑ पि॒त्रा प्र॒भूः प्र॒भूः पि॒त्रा ऽश्वः॑ । \newline
5. प्र॒भूरिति॑ प्र - भूः । \newline
6. पि॒त्रा ऽश्वो ऽश्वः॑ पि॒त्रा पि॒त्रा ऽश्वो᳚ ऽस्य॒स्य श्वः॑ पि॒त्रा पि॒त्रा ऽश्वो॑ ऽसि । \newline
7. अश्वो᳚ ऽस्य॒स्य श्वो ऽश्वो॑ ऽसि॒ हयो॒ हयो॒ ऽस्य श्वो ऽश्वो॑ ऽसि॒ हयः॑ । \newline
8. अ॒सि॒ हयो॒ हयो᳚ ऽस्यसि॒ हयो᳚ ऽस्यसि॒ हयो᳚ ऽस्यसि॒ हयो॑ ऽसि । \newline
9. हयो᳚ ऽस्यसि॒ हयो॒ हयो॒ ऽस्यत्यो ऽत्यो॑ ऽसि॒ हयो॒ हयो॒ ऽस्यत्यः॑ । \newline
10. अ॒स्यत्यो ऽत्यो᳚ ऽस्य॒ स्यत्यो᳚ ऽस्य॒ स्यत्यो᳚ ऽस्य॒ स्यत्यो॑ ऽसि । \newline
11. अत्यो᳚ ऽस्य॒ स्यत्यो ऽत्यो॑ ऽसि॒ नरो॒ नरो॒ ऽस्यत्यो ऽत्यो॑ ऽसि॒ नरः॑ । \newline
12. अ॒सि॒ नरो॒ नरो᳚ ऽस्यसि॒ नरो᳚ ऽस्यसि॒ नरो᳚ ऽस्यसि॒ नरो॑ ऽसि । \newline
13. नरो᳚ ऽस्यसि॒ नरो॒ नरो॒ ऽस्यर्वा ऽर्वा॑ ऽसि॒ नरो॒ नरो॒ ऽस्यर्वा᳚ । \newline
14. अ॒स्यर्वा ऽर्वा᳚ ऽस्य॒ स्यर्वा᳚ ऽस्य॒ स्यर्वा᳚ ऽस्य॒ स्यर्वा॑ ऽसि । \newline
15. अर्वा᳚ ऽस्य॒ स्यर्वा ऽर्वा॑ ऽसि॒ सप्तिः॒ सप्ति॑ र॒स्यर्वा ऽर्वा॑ ऽसि॒ सप्तिः॑ । \newline
16. अ॒सि॒ सप्तिः॒ सप्ति॑ रस्यसि॒ सप्ति॑ रस्यसि॒ सप्ति॑ रस्यसि॒ सप्ति॑रसि । \newline
17. सप्ति॑ रस्यसि॒ सप्तिः॒ सप्ति॑रसि वा॒जी वा॒ज्य॑सि॒ सप्तिः॒ सप्ति॑ रसि वा॒जी । \newline
18. अ॒सि॒ वा॒जी वा॒ज्य॑ स्यसि वा॒ज्य॑ स्यसि वा॒ज्य॑ स्यसि वा॒ज्य॑सि । \newline
19. वा॒ज्य॑ स्यसि वा॒जी वा॒ज्य॑सि॒ वृषा॒ वृषा॑ ऽसि वा॒जी वा॒ज्य॑सि॒ वृषा᳚ । \newline
20. अ॒सि॒ वृषा॒ वृषा᳚ ऽस्यसि॒ वृषा᳚ ऽस्यसि॒ वृषा᳚ ऽस्यसि॒ वृषा॑ ऽसि । \newline
21. वृषा᳚ ऽस्यसि॒ वृषा॒ वृषा॑ ऽसि नृ॒मणा॑ नृ॒मणा॑ असि॒ वृषा॒ वृषा॑ ऽसि नृ॒मणाः᳚ । \newline
22. अ॒सि॒ नृ॒मणा॑ नृ॒मणा॑ अस्यसि नृ॒मणा॑ अस्यसि नृ॒मणा॑ अस्यसि नृ॒मणा॑ असि । \newline
23. नृ॒मणा॑ अस्यसि नृ॒मणा॑ नृ॒मणा॑ असि॒ ययु॒र् ययु॑ रसि नृ॒मणा॑ नृ॒मणा॑ असि॒ ययुः॑ । \newline
24. नृ॒मणा॒ इति॑ नृ - मनाः᳚ । \newline
25. अ॒सि॒ ययु॒र् ययु॑ रस्यसि॒ ययु॒र् नाम॒ नाम॒ ययु॑ रस्यसि॒ ययु॒र् नाम॑ । \newline
26. ययु॒र् नाम॒ नाम॒ ययु॒र् ययु॒र् नामा᳚ स्यसि॒ नाम॒ ययु॒र् ययु॒र् नामा॑सि । \newline
27. नामा᳚ स्यसि॒ नाम॒ नामा᳚ स्यादि॒त्याना॑ मादि॒त्याना॑ मसि॒ नाम॒ नामा᳚ स्यादि॒त्याना᳚म् । \newline
28. अ॒स्या॒दि॒त्याना॑ मादि॒त्याना॑ मस्य स्यादि॒त्याना॒म् पत्व॒ पत्वा॑ दि॒त्याना॑ मस्य स्यादि॒त्याना॒म् पत्व॑ । \newline
29. आ॒दि॒त्याना॒म् पत्व॒ पत्वा॑ दि॒त्याना॑ मादि॒त्याना॒म् पत्वान्वनु॒ पत्वा॑ दि॒त्याना॑ मादि॒त्याना॒म् पत्वानु॑ । \newline
30. पत्वान् वनु॒ पत्व॒ पत्वान् वि॑ही॒ह्यनु॒ पत्व॒ पत्वान् वि॑हि । \newline
31. अन्वि॑ही॒ ह्यन् वन्वि॑ ह्य॒ग्नये॒ ऽग्नय॑ इ॒ह्यन् वन् वि॑ह्य॒ग्नये᳚ । \newline
32. इ॒ह्य॒ ग्नये॒ ऽग्नय॑ इही ह्य॒ग्नये॒ स्वाहा॒ स्वाहा॒ ऽग्नय॑ इही ह्य॒ग्नये॒ स्वाहा᳚ । \newline
33. अ॒ग्नये॒ स्वाहा॒ स्वाहा॒ ऽग्नये॒ ऽग्नये॒ स्वाहा᳚ । \newline
34. स्वाहा॒ स्वाहा᳚ । \newline
35. स्वाहे᳚न्द्रा॒ग्निभ्या॑ मिन्द्रा॒ग्निभ्याꣳ॒॒ स्वाहा॒ स्वाहे᳚न्द्रा॒ग्निभ्याꣳ॒॒ स्वाहा॒ स्वाहे᳚न्द्रा॒ग्निभ्याꣳ॒॒ स्वाहा॒ स्वाहे᳚न्द्रा॒ग्निभ्याꣳ॒॒ स्वाहा᳚ । \newline
36. इ॒न्द्रा॒ग्निभ्याꣳ॒॒ स्वाहा॒ स्वाहे᳚न्द्रा॒ग्निभ्या॑ मिन्द्रा॒ग्निभ्याꣳ॒॒ स्वाहा᳚ प्र॒जाप॑तये प्र॒जाप॑तये॒ 
स्वाहे᳚न्द्रा॒ग्निभ्या॑ मिन्द्रा॒ग्निभ्याꣳ॒॒ स्वाहा᳚ प्र॒जाप॑तये । \newline
37. इ॒न्द्रा॒ग्निभ्या॒मिती᳚न्द्रा॒ग्नि - भ्या॒म् । \newline
38. स्वाहा᳚ प्र॒जाप॑तये प्र॒जाप॑तये॒ स्वाहा॒ स्वाहा᳚ प्र॒जाप॑तये॒ स्वाहा॒ स्वाहा᳚ प्र॒जाप॑तये॒ स्वाहा॒ स्वाहा᳚ प्र॒जाप॑तये॒ स्वाहा᳚ । \newline
39. प्र॒जाप॑तये॒ स्वाहा॒ स्वाहा᳚ प्र॒जाप॑तये प्र॒जाप॑तये॒ स्वाहा॒ विश्वे᳚भ्यो॒ विश्वे᳚भ्यः॒ स्वाहा᳚ प्र॒जाप॑तये प्र॒जाप॑तये॒ स्वाहा॒ विश्वे᳚भ्यः । \newline
40. प्र॒जाप॑तय॒ इति॑ प्र॒जा - प॒त॒ये॒ । \newline
41. स्वाहा॒ विश्वे᳚भ्यो॒ विश्वे᳚भ्यः॒ स्वाहा॒ स्वाहा॒ विश्वे᳚भ्यो दे॒वेभ्यो॑ दे॒वेभ्यो॒ विश्वे᳚भ्यः॒ स्वाहा॒ स्वाहा॒ विश्वे᳚भ्यो दे॒वेभ्यः॑ । \newline
42. विश्वे᳚भ्यो दे॒वेभ्यो॑ दे॒वेभ्यो॒ विश्वे᳚भ्यो॒ विश्वे᳚भ्यो दे॒वेभ्यः॒ स्वाहा॒ स्वाहा॑ दे॒वेभ्यो॒ विश्वे᳚भ्यो॒ विश्वे᳚भ्यो दे॒वेभ्यः॒ स्वाहा᳚ । \newline
43. दे॒वेभ्यः॒ स्वाहा॒ स्वाहा॑ दे॒वेभ्यो॑ दे॒वेभ्यः॒ स्वाहा॒ सर्वा᳚भ्यः॒ सर्वा᳚भ्यः॒ स्वाहा॑ दे॒वेभ्यो॑ दे॒वेभ्यः॒ स्वाहा॒ सर्वा᳚भ्यः । \newline
44. स्वाहा॒ सर्वा᳚भ्यः॒ सर्वा᳚भ्यः॒ स्वाहा॒ स्वाहा॒ सर्वा᳚भ्यो दे॒वेता᳚भ्यो दे॒वेता᳚भ्यः॒ सर्वा᳚भ्यः॒ स्वाहा॒ स्वाहा॒ सर्वा᳚भ्यो दे॒वेता᳚भ्यः । \newline
45. सर्वा᳚भ्यो दे॒वेता᳚भ्यो दे॒वेता᳚भ्यः॒ सर्वा᳚भ्यः॒ सर्वा᳚भ्यो दे॒वेता᳚भ्य इ॒हेह दे॒वेता᳚भ्यः॒ सर्वा᳚भ्यः॒ सर्वा᳚भ्यो दे॒वेता᳚भ्य इ॒ह । \newline
46. दे॒वेता᳚भ्य इ॒हेह दे॒वेता᳚भ्यो दे॒वेता᳚भ्य इ॒ह धृति॒र् धृति॑ रि॒ह दे॒वेता᳚भ्यो दे॒वेता᳚भ्य इ॒ह धृतिः॑ । \newline
47. इ॒ह धृति॒र् धृति॑ रि॒हेह धृतिः॒ स्वाहा॒ स्वाहा॒ धृति॑ रि॒हेह धृतिः॒ स्वाहा᳚ । \newline
48. धृतिः॒ स्वाहा॒ स्वाहा॒ धृति॒र् धृतिः॒ स्वा हे॒हेह स्वाहा॒ धृति॒र् धृतिः॒ स्वाहे॒ह । \newline
49. स्वा हे॒हेह स्वाहा॒ स्वाहे॒ह विधृ॑ति॒र् विधृ॑ति रि॒ह स्वाहा॒ स्वाहे॒ह विधृ॑तिः । \newline
50. इ॒ह विधृ॑ति॒र् विधृ॑ति रि॒हेह विधृ॑तिः॒ स्वाहा॒ स्वाहा॒ विधृ॑ति रि॒हेह विधृ॑तिः॒ स्वाहा᳚ । \newline
51. विधृ॑तिः॒ स्वाहा॒ स्वाहा॒ विधृ॑ति॒र् विधृ॑तिः॒ स्वा हे॒हेह स्वाहा॒ विधृ॑ति॒र् विधृ॑तिः॒ स्वाहे॒ह । \newline
52. विधृ॑ति॒रिति॒ वि - धृ॒तिः॒ । \newline
53. स्वा हे॒हेह स्वाहा॒ स्वाहे॒ह रन्ती॒ रन्ति॑ रि॒ह स्वाहा॒ स्वाहे॒ह रन्तिः॑ । \newline
54. इ॒ह रन्ती॒ रन्ति॑ रि॒हेह रन्तिः॒ स्वाहा॒ स्वाहा॒ रन्ति॑ रि॒हेह रन्तिः॒ स्वाहा᳚ । \newline
55. रन्तिः॒ स्वाहा॒ स्वाहा॒ रन्ती॒ रन्तिः॒ स्वा हे॒हेह स्वाहा॒ रन्ती॒ रन्तिः॒ स्वाहे॒ह । \newline
56. स्वा हे॒हेह स्वाहा॒ स्वाहे॒ह रम॑ती॒ रम॑तिरि॒ह स्वाहा॒ स्वाहे॒ह रम॑तिः । \newline
57. इ॒ह रम॑ती॒ रम॑ति रि॒हेह रम॑तिः॒ स्वाहा॒ स्वाहा॒ रम॑ति रि॒हेह रम॑तिः॒ स्वाहा᳚ । \newline
58. रम॑तिः॒ स्वाहा॒ स्वाहा॒ रम॑ती॒ रम॑तिः॒ स्वाहा॒ भूर् भूः स्वाहा॒ रम॑ती॒ रम॑तिः॒ स्वाहा॒ भूः । \newline
59. स्वाहा॒ भूर् भूः स्वाहा॒ स्वाहा॒ भूर॑स्यसि॒ भूः स्वाहा॒ स्वाहा॒ भूर॑सि । \newline
60. भूर॑स्यसि॒ भूर् भूर॑सि भु॒वे भु॒वे॑ ऽसि॒ भूर् भूर॑सि भु॒वे । \newline
61. अ॒सि॒ भु॒वे भु॒वे᳚ ऽस्यसि भु॒वे त्वा᳚ त्वा भु॒वे᳚ ऽस्यसि भु॒वे त्वा᳚ । \newline
62. भु॒वे त्वा᳚ त्वा भु॒वे भु॒वे त्वा॒ भव्या॑य॒ भव्या॑य त्वा भु॒वे भु॒वे त्वा॒ भव्या॑य । \newline
63. त्वा॒ भव्या॑य॒ भव्या॑य त्वा त्वा॒ भव्या॑य त्वा त्वा॒ भव्या॑य त्वा त्वा॒ भव्या॑य त्वा । \newline
64. भव्या॑य त्वा त्वा॒ भव्या॑य॒ भव्या॑य त्वा भविष्य॒ते भ॑विष्य॒ते त्वा॒ भव्या॑य॒ भव्या॑य त्वा भविष्य॒ते । \newline
65. त्वा॒ भ॒वि॒ष्य॒ते भ॑विष्य॒ते त्वा᳚ त्वा भविष्य॒ते त्वा᳚ त्वा भविष्य॒ते त्वा᳚ त्वा भविष्य॒ते त्वा᳚ । \newline
66. भ॒वि॒ष्य॒ते त्वा᳚ त्वा भविष्य॒ते भ॑विष्य॒ते त्वा॒ विश्वे᳚भ्यो॒ विश्वे᳚भ्य स्त्वा भविष्य॒ते भ॑विष्य॒ते त्वा॒ विश्वे᳚भ्यः । \newline
67. त्वा॒ विश्वे᳚भ्यो॒ विश्वे᳚भ्य स्त्वा त्वा॒ विश्वे᳚भ्य स्त्वा त्वा॒ विश्वे᳚भ्य स्त्वा त्वा॒ विश्वे᳚भ्य स्त्वा । \newline
68. विश्वे᳚भ्य स्त्वा त्वा॒ विश्वे᳚भ्यो॒ विश्वे᳚भ्य स्त्वा भू॒तेभ्यो॑ भू॒तेभ्य॑ स्त्वा॒ विश्वे᳚भ्यो॒ विश्वे᳚भ्य स्त्वा भू॒तेभ्यः॑ । \newline
69. त्वा॒ भू॒तेभ्यो॑ भू॒तेभ्य॑ स्त्वा त्वा भू॒तेभ्यो॒ देवा॒ देवा॑ भू॒तेभ्य॑ स्त्वा त्वा भू॒तेभ्यो॒ देवाः᳚ । \newline
70. भू॒तेभ्यो॒ देवा॒ देवा॑ भू॒तेभ्यो॑ भू॒तेभ्यो॒ देवा॑ आशापाला आशापाला॒ देवा॑ भू॒तेभ्यो॑ भू॒तेभ्यो॒ देवा॑ आशापालाः । \newline
71. देवा॑ आशापाला आशापाला॒ देवा॒ देवा॑ आशापाला ए॒त मे॒त मा॑शापाला॒ देवा॒ देवा॑ आशापाला ए॒तम् । \newline
72. आ॒शा॒पा॒ला॒ ए॒त मे॒त मा॑शापाला आशापाला ए॒तम् दे॒वेभ्यो॑ दे॒वेभ्य॑ ए॒त मा॑शापाला आशापाला ए॒तम् दे॒वेभ्यः॑ । \newline
73. आ॒शा॒पा॒ला॒ इत्या॑शा - पा॒लाः॒ । \newline
74. ए॒तम् दे॒वेभ्यो॑ दे॒वेभ्य॑ ए॒त मे॒तम् दे॒वेभ्यो ऽश्व॒ मश्व॑म् दे॒वेभ्य॑ ए॒त मे॒तम् दे॒वेभ्यो ऽश्व᳚म् । \newline
75. दे॒वेभ्यो ऽश्व॒ मश्व॑म् दे॒वेभ्यो॑ दे॒वेभ्यो ऽश्व॒म् मेधा॑य॒ मेधा॒या श्व॑म् दे॒वेभ्यो॑ दे॒वेभ्यो ऽश्व॒म् मेधा॑य । \newline
76. अश्व॒म् मेधा॑य॒ मेधा॒या श्व॒ मश्व॒म् मेधा॑य॒ प्रोक्षि॑त॒म् प्रोक्षि॑त॒म् मेधा॒या श्व॒ मश्व॒म् मेधा॑य॒ प्रोक्षि॑तम् । \newline
77. मेधा॑य॒ प्रोक्षि॑त॒म् प्रोक्षि॑त॒म् मेधा॑य॒ मेधा॑य॒ प्रोक्षि॑तम् गोपायत गोपायत॒ प्रोक्षि॑त॒म् मेधा॑य॒ मेधा॑य॒ प्रोक्षि॑तम् गोपायत । \newline
78. प्रोक्षि॑तम् गोपायत गोपायत॒ प्रोक्षि॑त॒म् प्रोक्षि॑तम् गोपायत । \newline
79. प्रोक्षि॑त॒मिति॒ प्र - उ॒क्षि॒त॒म् । \newline
80. गो॒पा॒य॒तेति॑ गोपायत । \newline
\pagebreak
\markright{ TS 7.1.13.1  \hfill https://www.vedavms.in \hfill}

\section{ TS 7.1.13.1 }

\textbf{TS 7.1.13.1 } \newline
\textbf{Samhita Paata} \newline

आय॑नाय॒ स्वाहा॒ प्राय॑णाय॒ स्वाहो᳚द्द्रा॒वाय॒ स्वाहोद्द्रु॑ताय॒ स्वाहा॑ शूका॒राय॒ स्वाहा॒ शूकृ॑ताय॒ स्वाहा॒ पला॑यिताय॒ स्वाहा॒ ऽऽपला॑यिताय॒ स्वाहा॒ ऽऽवल्ग॑ते॒ स्वाहा॑ परा॒वल्ग॑ते॒ स्वाहा॑ ऽऽय॒ते स्वाहा᳚ प्रय॒ते स्वाहा॒ सर्व॑स्मै॒ स्वाहा᳚ ॥ \newline

\textbf{Pada Paata} \newline

आय॑ना॒येत्या᳚ - अय॑नाय । स्वाहा᳚ । प्राय॑णा॒येति॑ प्र - अय॑नाय । स्वाहा᳚ । उ॒द्द्रा॒वायेत्यु॑त् - द्रा॒वाय॑ । स्वाहा᳚ । उद्द्रु॑ता॒येत्युत्-द्रु॒ता॒य॒ । स्वाहा᳚ । शू॒का॒रायेति॑ शू-का॒राय॑ । स्वाहा᳚ । शूकृ॑ता॒येति॒ शू-कृ॒ता॒य॒ । स्वाहा᳚ । पला॑यिताय । स्वाहा᳚ । आ॒पला॑यिता॒येत्या᳚ - पला॑यिताय । स्वाहा᳚ । आ॒वल्ग॑त॒ इत्या᳚ - वल्ग॑ते । स्वाहा᳚ । प॒रा॒वल्ग॑त॒ इति॑ परा - वल्ग॑ते । स्वाहा᳚ । आ॒य॒त इत्या᳚ - य॒ते । स्वाहा᳚ । प्र॒य॒त इति॑ प्र - य॒ते । स्वाहा᳚ । सर्व॑स्मै । स्वाहा᳚ ॥  \newline


\textbf{Krama Paata} \newline

आय॑नाय॒ स्वाहा᳚ । आय॑ना॒येत्या᳚ - अय॑नाय । स्वाहा॒ प्राय॑णाय । प्राय॑णाय॒ स्वाहा᳚ । प्राय॑णा॒येति॑ प्र - अय॑नाय । स्वाहो᳚द्द्रा॒वाय॑ । उ॒द्द्रा॒वाय॒ स्वाहाः᳚ । उ॒द्द्रा॒वायेत्यु॑त् - द्रा॒वाय॑ । स्वाहोद्द्रु॑ताय । उद्द्रु॑ताय॒ स्वाहा᳚ । उद्द्रु॑ता॒येत्युत् - द्रु॒ता॒य॒ । स्वाहा॑ शूका॒राय॑ । शू॒का॒राय॒ स्वाहा᳚ । शू॒का॒रायेति॑ शू - का॒राय॑ । स्वाहा॒ शूकृ॑ताय । शूकृ॑ताय॒ स्वाहा᳚ । शूकृ॑ता॒येति॒ शू - कृ॒ता॒य॒ । स्वाहा॒ पला॑यिताय । पला॑यिताय॒ स्वाहा᳚ । स्वाहा॒ऽऽपला॑यिताय । आ॒पला॑यिताय॒ स्वाहा᳚ । आ॒पला॑यिता॒येत्या᳚ - पला॑यिताय । स्वाहा॒ऽऽवल्ग॑ते । आ॒वल्ग॑ते॒ स्वाहा᳚ । आ॒वल्ग॑त॒ इत्या᳚ - वल्ग॑ते । स्वाहा॑ परा॒वल्ग॑ते । प॒रा॒वल्ग॑ते॒ स्वाहा᳚ । प॒रा॒वल्ग॑त॒ इति॑ परा - वल्ग॑ते । स्वाहा॑ऽऽय॒ते । आ॒य॒ते स्वाहा᳚ । आ॒य॒त इत्या᳚ - य॒ते । स्वाहा᳚ प्रय॒ते । प्र॒य॒ते स्वाहा᳚ । प्र॒य॒त इति॑ प्र - य॒ते । स्वाहा॒ सर्व॑स्मै । सर्व॑स्मै॒ स्वाहा᳚ । स्वाहेति॒ स्वाहा᳚ । \newline

\textbf{Jatai Paata} \newline

1. आय॑नाय॒ स्वाहा॒ स्वाहा ऽऽय॑ना॒या य॑नाय॒ स्वाहा᳚ । \newline
2. आय॑ना॒येत्या᳚ - अय॑नाय । \newline
3. स्वाहा॒ प्राय॑णाय॒ प्राय॑णाय॒ स्वाहा॒ स्वाहा॒ प्राय॑णाय । \newline
4. प्राय॑णाय॒ स्वाहा॒ स्वाहा॒ प्राय॑णाय॒ प्राय॑णाय॒ स्वाहा᳚ । \newline
5. प्राय॑णा॒येति॑ प्र - अय॑नाय । \newline
6. स्वाहो᳚द्द्रा॒वायो᳚ द्द्रा॒वाय॒ स्वाहा॒ स्वाहो᳚द्द्रा॒वाय॑ । \newline
7. उ॒द्द्रा॒वाय॒ स्वाहा॒ स्वाहो᳚द्द्रा॒वा यो᳚द्द्रा॒वाय॒ स्वाहा᳚ । \newline
8. उ॒द्द्रा॒वायेत्यु॑त् - द्रा॒वाय॑ । \newline
9. स्वाहोद्द्रु॑ता॒ योद्द्रु॑ताय॒ स्वाहा॒ स्वाहोद्द्रु॑ताय । \newline
10. उद्द्रु॑ताय॒ स्वाहा॒ स्वाहोद्द्रु॑ता॒ योद्द्रु॑ताय॒ स्वाहा᳚ । \newline
11. उद्द्रु॑ता॒येत्युत् - द्रु॒ता॒य॒ । \newline
12. स्वाहा॑ शूका॒राय॑ शूका॒राय॒ स्वाहा॒ स्वाहा॑ शूका॒राय॑ । \newline
13. शू॒का॒राय॒ स्वाहा॒ स्वाहा॑ शूका॒राय॑ शूका॒राय॒ स्वाहा᳚ । \newline
14. शू॒का॒रायेति॑ शू - का॒राय॑ । \newline
15. स्वाहा॒ शूकृ॑ताय॒ शूकृ॑ताय॒ स्वाहा॒ स्वाहा॒ शूकृ॑ताय । \newline
16. शूकृ॑ताय॒ स्वाहा॒ स्वाहा॒ शूकृ॑ताय॒ शूकृ॑ताय॒ स्वाहा᳚ । \newline
17. शूकृ॑ता॒येति॒ शू - कृ॒ता॒य॒ । \newline
18. स्वाहा॒ पला॑यिताय॒ पला॑यिताय॒ स्वाहा॒ स्वाहा॒ पला॑यिताय । \newline
19. पला॑यिताय॒ स्वाहा॒ स्वाहा॒ पला॑यिताय॒ पला॑यिताय॒ स्वाहा᳚ । \newline
20. स्वाहा॒ ऽऽपला॑यिताया॒ पला॑यिताय॒ स्वाहा॒ स्वाहा॒ ऽऽपला॑यिताय । \newline
21. आ॒पला॑यिताय॒ स्वाहा॒ स्वाहा॒ ऽऽपला॑यिताया॒ पला॑यिताय॒ स्वाहा᳚ । \newline
22. आ॒पला॑यिता॒येत्या᳚ - पला॑यिताय । \newline
23. स्वाहा॒ ऽऽवल्ग॑त आ॒वल्ग॑ते॒ स्वाहा॒ स्वाहा॒ ऽऽवल्ग॑ते । \newline
24. आ॒वल्ग॑ते॒ स्वाहा॒ स्वाहा॒ ऽऽवल्ग॑त आ॒वल्ग॑ते॒ स्वाहा᳚ । \newline
25. आ॒वल्ग॑त॒ इत्या᳚ - वल्ग॑ते । \newline
26. स्वाहा॑ परा॒वल्ग॑ते परा॒वल्ग॑ते॒ स्वाहा॒ स्वाहा॑ परा॒वल्ग॑ते । \newline
27. प॒रा॒वल्ग॑ते॒ स्वाहा॒ स्वाहा॑ परा॒वल्ग॑ते परा॒वल्ग॑ते॒ स्वाहा᳚ । \newline
28. प॒रा॒वल्ग॑त॒ इति॑ परा - वल्ग॑ते । \newline
29. स्वाहा॑ ऽऽय॒त आ॑य॒ते स्वाहा॒ स्वाहा॑ ऽऽय॒ते । \newline
30. आ॒य॒ते स्वाहा॒ स्वाहा॑ ऽऽय॒त आ॑य॒ते स्वाहा᳚ । \newline
31. आ॒य॒त इत्या᳚ - य॒ते । \newline
32. स्वाहा᳚ प्रय॒ते प्र॑य॒ते स्वाहा॒ स्वाहा᳚ प्रय॒ते । \newline
33. प्र॒य॒ते स्वाहा॒ स्वाहा᳚ प्रय॒ते प्र॑य॒ते स्वाहा᳚ । \newline
34. प्र॒य॒त इति॑ प्र - य॒ते । \newline
35. स्वाहा॒ सर्व॑स्मै॒ सर्व॑स्मै॒ स्वाहा॒ स्वाहा॒ सर्व॑स्मै । \newline
36. सर्व॑स्मै॒ स्वाहा॒ स्वाहा॒ सर्व॑स्मै॒ सर्व॑स्मै॒ स्वाहा᳚ । \newline
37. स्वाहेति॒ स्वाहा᳚ । \newline

\textbf{Ghana Paata } \newline

1. आय॑नाय॒ स्वाहा॒ स्वाहा ऽऽय॑ना॒या य॑नाय॒ स्वाहा॒ प्राय॑णाय॒ प्राय॑णाय॒ स्वाहा ऽऽय॑ना॒या य॑नाय॒ स्वाहा॒ प्राय॑णाय । \newline
2. आय॑ना॒येत्या᳚ - अय॑नाय । \newline
3. स्वाहा॒ प्राय॑णाय॒ प्राय॑णाय॒ स्वाहा॒ स्वाहा॒ प्राय॑णाय॒ स्वाहा॒ स्वाहा॒ प्राय॑णाय॒ स्वाहा॒ स्वाहा॒ प्राय॑णाय॒ स्वाहा᳚ । \newline
4. प्राय॑णाय॒ स्वाहा॒ स्वाहा॒ प्राय॑णाय॒ प्राय॑णाय॒ स्वाहो᳚द्द्रा॒वा यो᳚द्द्रा॒वाय॒ स्वाहा॒ प्राय॑णाय॒ प्राय॑णाय॒ स्वाहो᳚द्द्रा॒वाय॑ । \newline
5. प्राय॑णा॒येति॑ प्र - अय॑नाय । \newline
6. स्वाहो᳚द्द्रा॒वा यो᳚द्द्रा॒वाय॒ स्वाहा॒ स्वाहो᳚द्द्रा॒वाय॒ स्वाहा॒ स्वाहो᳚द्द्रा॒वाय॒ स्वाहा॒ स्वाहो᳚द्द्रा॒वाय॒ स्वाहा᳚ । \newline
7. उ॒द्द्रा॒वाय॒ स्वाहा॒ स्वाहो᳚द्द्रा॒वा यो᳚द्द्रा॒वाय॒ स्वाहोद्द्रु॑ता॒ योद्द्रु॑ताय॒ स्वाहो᳚द्द्रा॒वा यो᳚द्द्रा॒वाय॒ 
स्वाहोद्द्रु॑ताय । \newline
8. उ॒द्द्रा॒वायेत्यु॑त् - द्रा॒वाय॑ । \newline
9. स्वाहोद्द्रु॑ता॒ योद्द्रु॑ताय॒ स्वाहा॒ स्वाहोद्द्रु॑ताय॒ स्वाहा॒ स्वाहोद्द्रु॑ताय॒ स्वाहा॒ स्वाहोद्द्रु॑ताय॒ स्वाहा᳚ । \newline
10. उद्द्रु॑ताय॒ स्वाहा॒ स्वाहोद्द्रु॑ता॒ योद्द्रु॑ताय॒ स्वाहा॑ शूका॒राय॑ शूका॒राय॒ स्वाहोद्द्रु॑ता॒ योद्द्रु॑ताय॒ स्वाहा॑ शूका॒राय॑ । \newline
11. उद्द्रु॑ता॒येत्युत् - द्रु॒ता॒य॒ । \newline
12. स्वाहा॑ शूका॒राय॑ शूका॒राय॒ स्वाहा॒ स्वाहा॑ शूका॒राय॒ स्वाहा॒ स्वाहा॑ शूका॒राय॒ स्वाहा॒ स्वाहा॑ शूका॒राय॒ स्वाहा᳚ । \newline
13. शू॒का॒राय॒ स्वाहा॒ स्वाहा॑ शूका॒राय॑ शूका॒राय॒ स्वाहा॒ शूकृ॑ताय॒ शूकृ॑ताय॒ स्वाहा॑ शूका॒राय॑ शूका॒राय॒ स्वाहा॒ शूकृ॑ताय । \newline
14. शू॒का॒रायेति॑ शू - का॒राय॑ । \newline
15. स्वाहा॒ शूकृ॑ताय॒ शूकृ॑ताय॒ स्वाहा॒ स्वाहा॒ शूकृ॑ताय॒ स्वाहा॒ स्वाहा॒ शूकृ॑ताय॒ स्वाहा॒ स्वाहा॒ शूकृ॑ताय॒ स्वाहा᳚ । \newline
16. शूकृ॑ताय॒ स्वाहा॒ स्वाहा॒ शूकृ॑ताय॒ शूकृ॑ताय॒ स्वाहा॒ पला॑यिताय॒ पला॑यिताय॒ स्वाहा॒ शूकृ॑ताय॒ शूकृ॑ताय॒ स्वाहा॒ पला॑यिताय । \newline
17. शूकृ॑ता॒येति॒ शू - कृ॒ता॒य॒ । \newline
18. स्वाहा॒ पला॑यिताय॒ पला॑यिताय॒ स्वाहा॒ स्वाहा॒ पला॑यिताय॒ स्वाहा॒ स्वाहा॒ पला॑यिताय॒ स्वाहा॒ स्वाहा॒ पला॑यिताय॒ स्वाहा᳚ । \newline
19. पला॑यिताय॒ स्वाहा॒ स्वाहा॒ पला॑यिताय॒ पला॑यिताय॒ स्वाहा॒ ऽऽपला॑यिताया॒ पला॑यिताय॒ स्वाहा॒ पला॑यिताय॒ पला॑यिताय॒ स्वाहा॒ ऽऽपला॑यिताय । \newline
20. स्वाहा॒ ऽऽपला॑यिताया॒ पला॑यिताय॒ स्वाहा॒ स्वाहा॒ ऽऽपला॑यिताय॒ स्वाहा॒ स्वाहा॒ ऽऽपला॑यिताय॒ स्वाहा॒ स्वाहा॒ ऽऽपला॑यिताय॒ स्वाहा᳚ । \newline
21. आ॒पला॑यिताय॒ स्वाहा॒ स्वाहा॒ ऽऽपला॑यिताया॒ पला॑यिताय॒ स्वाहा॒ ऽऽवल्ग॑त आ॒वल्ग॑ते॒ स्वाहा॒ ऽऽपला॑यिताया॒ पला॑यिताय॒ स्वाहा॒ ऽऽवल्ग॑ते । \newline
22. आ॒पला॑यिता॒येत्या᳚ - पला॑यिताय । \newline
23. स्वाहा॒ ऽऽवल्ग॑त आ॒वल्ग॑ते॒ स्वाहा॒ स्वाहा॒ ऽऽवल्ग॑ते॒ स्वाहा॒ स्वाहा॒ ऽऽवल्ग॑ते॒ स्वाहा॒ स्वाहा॒ ऽऽवल्ग॑ते॒ स्वाहा᳚ । \newline
24. आ॒वल्ग॑ते॒ स्वाहा॒ स्वाहा॒ ऽऽवल्ग॑त आ॒वल्ग॑ते॒ स्वाहा॑ परा॒वल्ग॑ते परा॒वल्ग॑ते॒ स्वाहा॒ ऽऽवल्ग॑त आ॒वल्ग॑ते॒ स्वाहा॑ परा॒वल्ग॑ते । \newline
25. आ॒वल्ग॑त॒ इत्या᳚ - वल्ग॑ते । \newline
26. स्वाहा॑ परा॒वल्ग॑ते परा॒वल्ग॑ते॒ स्वाहा॒ स्वाहा॑ परा॒वल्ग॑ते॒ स्वाहा॒ स्वाहा॑ परा॒वल्ग॑ते॒ स्वाहा॒ स्वाहा॑ परा॒वल्ग॑ते॒ स्वाहा᳚ । \newline
27. प॒रा॒वल्ग॑ते॒ स्वाहा॒ स्वाहा॑ परा॒वल्ग॑ते परा॒वल्ग॑ते॒ स्वाहा॑ ऽऽय॒त आ॑य॒ते स्वाहा॑ परा॒वल्ग॑ते परा॒वल्ग॑ते॒ स्वाहा॑ ऽऽय॒ते । \newline
28. प॒रा॒वल्ग॑त॒ इति॑ परा - वल्ग॑ते । \newline
29. स्वाहा॑ ऽऽय॒त आ॑य॒ते स्वाहा॒ स्वाहा॑ ऽऽय॒ते स्वाहा॒ स्वाहा॑ ऽऽय॒ते स्वाहा॒ स्वाहा॑ ऽऽय॒ते स्वाहा᳚ । \newline
30. आ॒य॒ते स्वाहा॒ स्वाहा॑ ऽऽय॒त आ॑य॒ते स्वाहा᳚ प्रय॒ते प्र॑य॒ते स्वाहा॑ ऽऽय॒त आ॑य॒ते स्वाहा᳚ प्रय॒ते । \newline
31. आ॒य॒त इत्या᳚ - य॒ते । \newline
32. स्वाहा᳚ प्रय॒ते प्र॑य॒ते स्वाहा॒ स्वाहा᳚ प्रय॒ते स्वाहा॒ स्वाहा᳚ प्रय॒ते स्वाहा॒ स्वाहा᳚ प्रय॒ते स्वाहा᳚ । \newline
33. प्र॒य॒ते स्वाहा॒ स्वाहा᳚ प्रय॒ते प्र॑य॒ते स्वाहा॒ सर्व॑स्मै॒ सर्व॑स्मै॒ स्वाहा᳚ प्रय॒ते प्र॑य॒ते स्वाहा॒ सर्व॑स्मै । \newline
34. प्र॒य॒त इति॑ प्र - य॒ते । \newline
35. स्वाहा॒ सर्व॑स्मै॒ सर्व॑स्मै॒ स्वाहा॒ स्वाहा॒ सर्व॑स्मै॒ स्वाहा॒ स्वाहा॒ सर्व॑स्मै॒ स्वाहा॒ स्वाहा॒ सर्व॑स्मै॒ स्वाहा᳚ । \newline
36. सर्व॑स्मै॒ स्वाहा॒ स्वाहा॒ सर्व॑स्मै॒ सर्व॑स्मै॒ स्वाहा᳚ । \newline
37. स्वाहेति॒ स्वाहा᳚ । \newline
\pagebreak
\markright{ TS 7.1.14.1  \hfill https://www.vedavms.in \hfill}

\section{ TS 7.1.14.1 }

\textbf{TS 7.1.14.1 } \newline
\textbf{Samhita Paata} \newline

अ॒ग्नये॒ स्वाहा॒ सोमा॑य॒ स्वाहा॑ वा॒यवे॒ स्वाहा॒ ऽपां मोदा॑य॒ स्वाहा॑ सवि॒त्रे स्वाहा॒ सर॑स्वत्यै॒ स्वाहे-न्द्रा॑य॒ स्वाहा॒ बृह॒स्पत॑ये॒ स्वाहा॑ मि॒त्राय॒ स्वाहा॒ वरु॑णाय॒ स्वाहा॒ सर्व॑स्मै॒ स्वाहा᳚ ॥ \newline

\textbf{Pada Paata} \newline

अ॒ग्नये᳚ । स्वाहा᳚ । सोमा॑य । स्वाहा᳚ । वा॒यवे᳚ । स्वाहा᳚ । अ॒पाम् । मोदा॑य । स्वाहा᳚ । स॒वि॒त्रे । स्वाहा᳚ । सर॑स्वत्यै । स्वाहा᳚ । इन्द्रा॑य । स्वाहा᳚ । बृह॒स्पत॑ये । स्वाहा᳚ । मि॒त्राय॑ । स्वाहा᳚ । वरु॑णाय । स्वाहा᳚ । सर्व॑स्मै । स्वाहा᳚ ॥  \newline


\textbf{Krama Paata} \newline

अ॒ग्नये॒ स्वाहा᳚ । स्वाहा॒ सोमा॑य । सोमा॑य॒ स्वाहा᳚ । स्वाहा॑ वा॒यवे᳚ । वा॒यवे॒ स्वाहा᳚ । स्वाहा॒ऽपाम् । अ॒पाम् मोदा॑य । मोदा॑य॒ स्वाहा᳚ । स्वाहा॑ सवि॒त्रे । स॒वि॒त्रे स्वाहा᳚ । स्वाहा॒ सर॑स्वत्यै । सर॑स्वत्यै॒ स्वाहा᳚ । स्वाहेन्द्रा॑य । इन्द्रा॑य॒ स्वाहा᳚ । स्वाहा॒ बृह॒स्पत॑ये । बृह॒स्पत॑ये॒ स्वाहा᳚ । स्वाहा॑ मि॒त्राय॑ । मि॒त्राय॒ स्वाहा᳚ । स्वाहा॒ वरु॑णाय । वरु॑णाय॒ स्वाहा᳚ । स्वाहा॒ सर्व॑स्मै । सर्व॑स्मै॒ स्वाहा᳚ ।? स्वाहेति॒ स्वाहा᳚ । \newline

\textbf{Jatai Paata} \newline

1. अ॒ग्नये॒ स्वाहा॒ स्वाहा॒ ऽग्नये॒ ऽग्नये॒ स्वाहा᳚ । \newline
2. स्वाहा॒ सोमा॑य॒ सोमा॑य॒ स्वाहा॒ स्वाहा॒ सोमा॑य । \newline
3. सोमा॑य॒ स्वाहा॒ स्वाहा॒ सोमा॑य॒ सोमा॑य॒ स्वाहा᳚ । \newline
4. स्वाहा॑ वा॒यवे॑ वा॒यवे॒ स्वाहा॒ स्वाहा॑ वा॒यवे᳚ । \newline
5. वा॒यवे॒ स्वाहा॒ स्वाहा॑ वा॒यवे॑ वा॒यवे॒ स्वाहा᳚ । \newline
6. स्वाहा॒ ऽपा म॒पाꣳ स्वाहा॒ स्वाहा॒ ऽपाम् । \newline
7. अ॒पाम् मोदा॑य॒ मोदा॑या॒ पा म॒पाम् मोदा॑य । \newline
8. मोदा॑य॒ स्वाहा॒ स्वाहा॒ मोदा॑य॒ मोदा॑य॒ स्वाहा᳚ । \newline
9. स्वाहा॑ सवि॒त्रे स॑वि॒त्रे स्वाहा॒ स्वाहा॑ सवि॒त्रे । \newline
10. स॒वि॒त्रे स्वाहा॒ स्वाहा॑ सवि॒त्रे स॑वि॒त्रे स्वाहा᳚ । \newline
11. स्वाहा॒ सर॑स्वत्यै॒ सर॑स्वत्यै॒ स्वाहा॒ स्वाहा॒ सर॑स्वत्यै । \newline
12. सर॑स्वत्यै॒ स्वाहा॒ स्वाहा॒ सर॑स्वत्यै॒ सर॑स्वत्यै॒ स्वाहा᳚ । \newline
13. स्वाहेन्द्रा॒ येन्द्रा॑य॒ स्वाहा॒ स्वाहेन्द्रा॑य । \newline
14. इन्द्रा॑य॒ स्वाहा॒ स्वाहेन्द्रा॒ येन्द्रा॑य॒ स्वाहा᳚ । \newline
15. स्वाहा॒ बृह॒स्पत॑ये॒ बृह॒स्पत॑ये॒ स्वाहा॒ स्वाहा॒ बृह॒स्पत॑ये । \newline
16. बृह॒स्पत॑ये॒ स्वाहा॒ स्वाहा॒ बृह॒स्पत॑ये॒ बृह॒स्पत॑ये॒ स्वाहा᳚ । \newline
17. स्वाहा॑ मि॒त्राय॑ मि॒त्राय॒ स्वाहा॒ स्वाहा॑ मि॒त्राय॑ । \newline
18. मि॒त्राय॒ स्वाहा॒ स्वाहा॑ मि॒त्राय॑ मि॒त्राय॒ स्वाहा᳚ । \newline
19. स्वाहा॒ वरु॑णाय॒ वरु॑णाय॒ स्वाहा॒ स्वाहा॒ वरु॑णाय । \newline
20. वरु॑णाय॒ स्वाहा॒ स्वाहा॒ वरु॑णाय॒ वरु॑णाय॒ स्वाहा᳚ । \newline
21. स्वाहा॒ सर्व॑स्मै॒ सर्व॑स्मै॒ स्वाहा॒ स्वाहा॒ सर्व॑स्मै । \newline
22. सर्व॑स्मै॒ स्वाहा॒ स्वाहा॒ सर्व॑स्मै॒ सर्व॑स्मै॒ स्वाहा᳚ । \newline
23. स्वाहेति॒ स्वाहा᳚ । \newline

\textbf{Ghana Paata } \newline

1. अ॒ग्नये॒ स्वाहा॒ स्वाहा॒ ऽग्नये॒ ऽग्नये॒ स्वाहा॒ सोमा॑य॒ सोमा॑य॒ स्वाहा॒ ऽग्नये॒ ऽग्नये॒ स्वाहा॒ सोमा॑य । \newline
2. स्वाहा॒ सोमा॑य॒ सोमा॑य॒ स्वाहा॒ स्वाहा॒ सोमा॑य॒ स्वाहा॒ स्वाहा॒ सोमा॑य॒ स्वाहा॒ स्वाहा॒ सोमा॑य॒ स्वाहा᳚ । \newline
3. सोमा॑य॒ स्वाहा॒ स्वाहा॒ सोमा॑य॒ सोमा॑य॒ स्वाहा॑ वा॒यवे॑ वा॒यवे॒ स्वाहा॒ सोमा॑य॒ सोमा॑य॒ स्वाहा॑ वा॒यवे᳚ । \newline
4. स्वाहा॑ वा॒यवे॑ वा॒यवे॒ स्वाहा॒ स्वाहा॑ वा॒यवे॒ स्वाहा॒ स्वाहा॑ वा॒यवे॒ स्वाहा॒ स्वाहा॑ वा॒यवे॒ स्वाहा᳚ । \newline
5. वा॒यवे॒ स्वाहा॒ स्वाहा॑ वा॒यवे॑ वा॒यवे॒ स्वाहा॒ ऽपा म॒पाꣳ स्वाहा॑ वा॒यवे॑ वा॒यवे॒ स्वाहा॒ ऽपाम् । \newline
6. स्वाहा॒ ऽपा म॒पाꣳ स्वाहा॒ स्वाहा॒ ऽपाम् मोदा॑य॒ मोदा॑या॒ पाꣳ स्वाहा॒ स्वाहा॒ ऽपाम् मोदा॑य । \newline
7. अ॒पाम् मोदा॑य॒ मोदा॑या॒ पा म॒पाम् मोदा॑य॒ स्वाहा॒ स्वाहा॒ मोदा॑या॒ पा म॒पाम् मोदा॑य॒ स्वाहा᳚ । \newline
8. मोदा॑य॒ स्वाहा॒ स्वाहा॒ मोदा॑य॒ मोदा॑य॒ स्वाहा॑ सवि॒त्रे स॑वि॒त्रे स्वाहा॒ मोदा॑य॒ मोदा॑य॒ स्वाहा॑ सवि॒त्रे । \newline
9. स्वाहा॑ सवि॒त्रे स॑वि॒त्रे स्वाहा॒ स्वाहा॑ सवि॒त्रे स्वाहा॒ स्वाहा॑ सवि॒त्रे स्वाहा॒ स्वाहा॑ सवि॒त्रे स्वाहा᳚ । \newline
10. स॒वि॒त्रे स्वाहा॒ स्वाहा॑ सवि॒त्रे स॑वि॒त्रे स्वाहा॒ सर॑स्वत्यै॒ सर॑स्वत्यै॒ स्वाहा॑ सवि॒त्रे स॑वि॒त्रे स्वाहा॒ सर॑स्वत्यै । \newline
11. स्वाहा॒ सर॑स्वत्यै॒ सर॑स्वत्यै॒ स्वाहा॒ स्वाहा॒ सर॑स्वत्यै॒ स्वाहा॒ स्वाहा॒ सर॑स्वत्यै॒ स्वाहा॒ स्वाहा॒ सर॑स्वत्यै॒ स्वाहा᳚ । \newline
12. सर॑स्वत्यै॒ स्वाहा॒ स्वाहा॒ सर॑स्वत्यै॒ सर॑स्वत्यै॒ स्वाहेन्द्रा॒ येन्द्रा॑य॒ स्वाहा॒ सर॑स्वत्यै॒ सर॑स्वत्यै॒ स्वाहेन्द्रा॑य । \newline
13. स्वाहेन्द्रा॒ येन्द्रा॑य॒ स्वाहा॒ स्वाहेन्द्रा॑य॒ स्वाहा॒ स्वाहेन्द्रा॑य॒ स्वाहा॒ स्वाहेन्द्रा॑य॒ स्वाहा᳚ । \newline
14. इन्द्रा॑य॒ स्वाहा॒ स्वाहेन्द्रा॒ येन्द्रा॑य॒ स्वाहा॒ बृह॒स्पत॑ये॒ बृह॒स्पत॑ये॒ स्वाहेन्द्रा॒ येन्द्रा॑य॒ स्वाहा॒ बृह॒स्पत॑ये । \newline
15. स्वाहा॒ बृह॒स्पत॑ये॒ बृह॒स्पत॑ये॒ स्वाहा॒ स्वाहा॒ बृह॒स्पत॑ये॒ स्वाहा॒ स्वाहा॒ बृह॒स्पत॑ये॒ स्वाहा॒ स्वाहा॒ बृह॒स्पत॑ये॒ स्वाहा᳚ । \newline
16. बृह॒स्पत॑ये॒ स्वाहा॒ स्वाहा॒ बृह॒स्पत॑ये॒ बृह॒स्पत॑ये॒ स्वाहा॑ मि॒त्राय॑ मि॒त्राय॒ स्वाहा॒ बृह॒स्पत॑ये॒ बृह॒स्पत॑ये॒ स्वाहा॑ मि॒त्राय॑ । \newline
17. स्वाहा॑ मि॒त्राय॑ मि॒त्राय॒ स्वाहा॒ स्वाहा॑ मि॒त्राय॒ स्वाहा॒ स्वाहा॑ मि॒त्राय॒ स्वाहा॒ स्वाहा॑ मि॒त्राय॒ स्वाहा᳚ । \newline
18. मि॒त्राय॒ स्वाहा॒ स्वाहा॑ मि॒त्राय॑ मि॒त्राय॒ स्वाहा॒ वरु॑णाय॒ वरु॑णाय॒ स्वाहा॑ मि॒त्राय॑ मि॒त्राय॒ स्वाहा॒ वरु॑णाय । \newline
19. स्वाहा॒ वरु॑णाय॒ वरु॑णाय॒ स्वाहा॒ स्वाहा॒ वरु॑णाय॒ स्वाहा॒ स्वाहा॒ वरु॑णाय॒ स्वाहा॒ स्वाहा॒ वरु॑णाय॒ स्वाहा᳚ । \newline
20. वरु॑णाय॒ स्वाहा॒ स्वाहा॒ वरु॑णाय॒ वरु॑णाय॒ स्वाहा॒ सर्व॑स्मै॒ सर्व॑स्मै॒ स्वाहा॒ वरु॑णाय॒ वरु॑णाय॒ स्वाहा॒ सर्व॑स्मै । \newline
21. स्वाहा॒ सर्व॑स्मै॒ सर्व॑स्मै॒ स्वाहा॒ स्वाहा॒ सर्व॑स्मै॒ स्वाहा॒ स्वाहा॒ सर्व॑स्मै॒ स्वाहा॒ स्वाहा॒ सर्व॑स्मै॒ स्वाहा᳚ । \newline
22. सर्व॑स्मै॒ स्वाहा॒ स्वाहा॒ सर्व॑स्मै॒ सर्व॑स्मै॒ स्वाहा᳚ । \newline
23. स्वाहेति॒ स्वाहा᳚ । \newline
\pagebreak
\markright{ TS 7.1.15.1  \hfill https://www.vedavms.in \hfill}

\section{ TS 7.1.15.1 }

\textbf{TS 7.1.15.1 } \newline
\textbf{Samhita Paata} \newline

पृ॒थि॒व्यै स्वाहा॒ ऽन्तरि॑क्षाय॒ स्वाहा॑ दि॒वे स्वाहा॒ सूर्या॑य॒ स्वाहा॑ च॒न्द्रम॑से॒ स्वाहा॒ नक्ष॑त्रेभ्यः॒ स्वाहा॒ प्राच्यै॑ दि॒शे स्वाहा॒ दक्षि॑णायै दि॒शे स्वाहा᳚ प्र॒तीच्यै॑ दि॒शे स्वाहो-दी᳚च्यै दि॒शे स्वाहो॒र्द्ध्वायै॑ दि॒शे स्वाहा॑ दि॒ग्भ्यः स्वाहा॑ ऽवान्तरदि॒शाभ्यः॒ स्वाहा॒ समा᳚भ्यः॒ स्वाहा॑ श॒रद्भ्यः॒ स्वाहा॑ ऽहोरा॒त्रेभ्यः॒ स्वाहा᳚ ऽर्द्धमा॒सेभ्यः॒ स्वाहा॒ मासे᳚भ्यः॒ स्वाहा॒र्तुभ्यः॒ स्वाहा॑ संॅवथ्स॒राय॒ स्वाहा॒ सर्व॑स्मै॒ स्वाहा᳚ ॥ \newline

\textbf{Pada Paata} \newline

पृ॒थि॒व्यै । स्वाहा᳚ । अ॒न्तरि॑क्षाय । स्वाहा᳚ । दि॒वे । स्वाहा᳚ । सूर्या॑य । स्वाहा᳚ । च॒न्द्रम॑से । स्वाहा᳚ । नक्ष॑त्रेभ्यः । स्वाहा᳚ । प्राच्यै᳚ । दि॒शे । स्वाहा᳚ । दक्षि॑णायै । दि॒शे । स्वाहा᳚ । प्र॒तीच्यै᳚ । दि॒शे । स्वाहा᳚ । उदी᳚च्यै । दि॒शे । स्वाहा᳚ । ऊ॒र्ध्वायै᳚ । दि॒शे । स्वाहा᳚ । दि॒ग्भ्य इति॑ दिक् - भ्यः । स्वाहा᳚ । अ॒वा॒न्त॒र॒दि॒शाभ्य॒ इत्य॑वान्तर - दि॒शाभ्यः॑ । स्वाहा᳚ । समा᳚भ्यः । स्वाहा᳚ । श॒रद्भ्य॒ इति॑ श॒रत् - भ्यः॒ । स्वाहा᳚ । अ॒हो॒रा॒त्रेभ्य॒ इत्य॑हः - रा॒त्रेभ्यः॑ । स्वाहा᳚ । अ॒द्‌र्ध॒मा॒सेभ्य॒ इत्य॑द्‌र्ध-मा॒सेभ्यः॑ । स्वाहा᳚ । मासे᳚भ्यः । स्वाहा᳚ । ऋ॒तुभ्य॒ इत्यृ॒तु - भ्यः॒ । स्वाहा᳚ । सं॒ॅव॒थ्स॒रायेति॑ सं - व॒थ्स॒राय॑ । स्वाहा᳚ । सर्व॑स्मै । स्वाहा᳚ ॥  \newline


\textbf{Krama Paata} \newline

पृ॒थि॒व्यै स्वाहा᳚ । स्वाहा॒ऽन्तरि॑क्षाय । अ॒न्तरि॑क्षाय॒ स्वाहा᳚ । स्वाहा॑ दि॒वे । दि॒वे स्वाहा᳚ । स्वाहा॒ सूर्या॑य । सूर्या॑य॒ स्वाहा᳚ । स्वाहा॑ च॒न्द्रम॑से । च॒न्द्रम॑से॒ स्वाहा᳚ । स्वाहा॒ नक्ष॑त्रेभ्यः । नक्ष॑त्रेभ्यः॒ स्वाहा᳚ । स्वाहा॒ प्राच्यै᳚ । प्राच्यै॑ दि॒शे । दि॒शे स्वाहा᳚ । स्वाहा॒ दक्षि॑णायै । दक्षि॑णायै दि॒शे । दि॒शे स्वाहा᳚ । स्वाहा᳚ प्र॒तीच्यै᳚ । प्र॒तीच्यै॑ दि॒शे । दि॒शे स्वाहा᳚ । स्वाहोदी᳚च्यै । उदी᳚च्यै दि॒शे । दि॒शे स्वाहा᳚ । स्वाहो॒र्द्ध्वायै᳚ । ऊ॒र्द्ध्वायै॑ दि॒शे । दि॒शे स्वाहा᳚ । स्वाहा॑ दि॒ग्भ्यः । दि॒ग्भ्यः स्वाहा᳚ । दि॒ग्भ्य इति॑ दिक् - भ्यः । स्वाहा॑ऽवान्तरदि॒शाभ्यः॑ । अ॒वा॒न्त॒र॒दि॒शाभ्यः॒ स्वाहा᳚ । अ॒वा॒न्त॒र॒दि॒शाभ्य॒ इत्य॑वान्तर - दि॒शाभ्यः॑ । स्वाहा॒ समा᳚भ्यः । समा᳚भ्यः॒ स्वाहा᳚ । स्वाहा॑ श॒रद्भ्यः॑ । श॒रद्भ्यः॒ स्वाहा᳚ । श॒रद्भ्य॒ इति॑ श॒रत् - भ्यः॒ । स्वाहा॑ऽहोरा॒त्रेभ्यः॑ । अ॒हो॒रा॒त्रेभ्यः॒ स्वाहा᳚ । अ॒हो॒रा॒त्रेभ्य॒ इत्य॑हः - रा॒त्रेभ्यः॑ । स्वाहा᳚ऽर्द्धमा॒सेभ्यः॑ । अ॒र्द्ध॒मा॒सेभ्यः॒ स्वाहा᳚ । अ॒र्द्ध॒मा॒सेभ्य॒ इत्य॑र्द्ध - मा॒सेभ्यः॑ । स्वाहा॒ मासे᳚भ्यः । मासे᳚भ्यः॒ स्वाहा᳚ । स्वाह॒र्तुभ्यः॑ । ऋ॒तुभ्यः॒ स्वाहा᳚ । ऋ॒तुभ्य॒ इत्यृ॒तु - भ्यः॒ । स्वाहा॑ सम्ॅवथ्स॒राय॑ । स॒म्ॅव॒थ्स॒राय॒ स्वाहा᳚ । स॒म्ॅव॒थ्स॒रायेति॑ सम् - व॒थ्स॒राय॑ । स्वाहा॒ सर्व॑स्मै । सर्व॑स्मै॒ स्वाहा᳚ । स्वाहेति॒ स्वाहा᳚ । \newline

\textbf{Jatai Paata} \newline

1. पृ॒थि॒व्यै स्वाहा॒ स्वाहा॑ पृथि॒व्यै पृ॑थि॒व्यै स्वाहा᳚ । \newline
2. स्वाहा॒ ऽन्तरि॑क्षाया॒ न्तरि॑क्षाय॒ स्वाहा॒ स्वाहा॒ ऽन्तरि॑क्षाय । \newline
3. अ॒न्तरि॑क्षाय॒ स्वाहा॒ स्वाहा॒ ऽन्तरि॑क्षाया॒ न्तरि॑क्षाय॒ स्वाहा᳚ । \newline
4. स्वाहा॑ दि॒वे दि॒वे स्वाहा॒ स्वाहा॑ दि॒वे । \newline
5. दि॒वे स्वाहा॒ स्वाहा॑ दि॒वे दि॒वे स्वाहा᳚ । \newline
6. स्वाहा॒ सूर्या॑य॒ सूर्या॑य॒ स्वाहा॒ स्वाहा॒ सूर्या॑य । \newline
7. सूर्या॑य॒ स्वाहा॒ स्वाहा॒ सूर्या॑य॒ सूर्या॑य॒ स्वाहा᳚ । \newline
8. स्वाहा॑ च॒न्द्रम॑से च॒न्द्रम॑से॒ स्वाहा॒ स्वाहा॑ च॒न्द्रम॑से । \newline
9. च॒न्द्रम॑से॒ स्वाहा॒ स्वाहा॑ च॒न्द्रम॑से च॒न्द्रम॑से॒ स्वाहा᳚ । \newline
10. स्वाहा॒ नक्ष॑त्रेभ्यो॒ नक्ष॑त्रेभ्यः॒ स्वाहा॒ स्वाहा॒ नक्ष॑त्रेभ्यः । \newline
11. नक्ष॑त्रेभ्यः॒ स्वाहा॒ स्वाहा॒ नक्ष॑त्रेभ्यो॒ नक्ष॑त्रेभ्यः॒ स्वाहा᳚ । \newline
12. स्वाहा॒ प्राच्यै॒ प्राच्यै॒ स्वाहा॒ स्वाहा॒ प्राच्यै᳚ । \newline
13. प्राच्यै॑ दि॒शे दि॒शे प्राच्यै॒ प्राच्यै॑ दि॒शे । \newline
14. दि॒शे स्वाहा॒ स्वाहा॑ दि॒शे दि॒शे स्वाहा᳚ । \newline
15. स्वाहा॒ दक्षि॑णायै॒ दक्षि॑णायै॒ स्वाहा॒ स्वाहा॒ दक्षि॑णायै । \newline
16. दक्षि॑णायै दि॒शे दि॒शे दक्षि॑णायै॒ दक्षि॑णायै दि॒शे । \newline
17. दि॒शे स्वाहा॒ स्वाहा॑ दि॒शे दि॒शे स्वाहा᳚ । \newline
18. स्वाहा᳚ प्र॒तीच्यै᳚ प्र॒तीच्यै॒ स्वाहा॒ स्वाहा᳚ प्र॒तीच्यै᳚ । \newline
19. प्र॒तीच्यै॑ दि॒शे दि॒शे प्र॒तीच्यै᳚ प्र॒तीच्यै॑ दि॒शे । \newline
20. दि॒शे स्वाहा॒ स्वाहा॑ दि॒शे दि॒शे स्वाहा᳚ । \newline
21. स्वाहोदी᳚च्या॒ उदी᳚च्यै॒ स्वाहा॒ स्वाहोदी᳚च्यै । \newline
22. उदी᳚च्यै दि॒शे दि॒श उदी᳚च्या॒ उदी᳚च्यै दि॒शे । \newline
23. दि॒शे स्वाहा॒ स्वाहा॑ दि॒शे दि॒शे स्वाहा᳚ । \newline
24. स्वाहो॒र्ध्वाया॑ ऊ॒र्ध्वायै॒ स्वाहा॒ स्वाहो॒र्ध्वायै᳚ । \newline
25. ऊ॒र्ध्वायै॑ दि॒शे दि॒श ऊ॒र्ध्वाया॑ ऊ॒र्ध्वायै॑ दि॒शे । \newline
26. दि॒शे स्वाहा॒ स्वाहा॑ दि॒शे दि॒शे स्वाहा᳚ । \newline
27. स्वाहा॑ दि॒ग्भ्यो दि॒ग्भ्यः स्वाहा॒ स्वाहा॑ दि॒ग्भ्यः । \newline
28. दि॒ग्भ्यः स्वाहा॒ स्वाहा॑ दि॒ग्भ्यो दि॒ग्भ्यः स्वाहा᳚ । \newline
29. दि॒ग्भ्य इति॑ दिक् - भ्यः । \newline
30. स्वाहा॑ ऽवान्तरदि॒शाभ्यो॑ ऽवान्तरदि॒शाभ्यः॒ स्वाहा॒ स्वाहा॑ ऽवान्तरदि॒शाभ्यः॑ । \newline
31. अ॒वा॒न्त॒र॒दि॒शाभ्यः॒ स्वाहा॒ स्वाहा॑ ऽवान्तरदि॒शाभ्यो॑ ऽवान्तरदि॒शाभ्यः॒ स्वाहा᳚ । \newline
32. अ॒वा॒न्त॒र॒दि॒शाभ्य॒ इत्य॑वान्तर - दि॒शाभ्यः॑ । \newline
33. स्वाहा॒ समा᳚भ्यः॒ समा᳚भ्यः॒ स्वाहा॒ स्वाहा॒ समा᳚भ्यः । \newline
34. समा᳚भ्यः॒ स्वाहा॒ स्वाहा॒ समा᳚भ्यः॒ समा᳚भ्यः॒ स्वाहा᳚ । \newline
35. स्वाहा॑ श॒रद्भ्यः॑ श॒रद्भ्यः॒ स्वाहा॒ स्वाहा॑ श॒रद्भ्यः॑ । \newline
36. श॒रद्भ्यः॒ स्वाहा॒ स्वाहा॑ श॒रद्भ्यः॑ श॒रद्भ्यः॒ स्वाहा᳚ । \newline
37. श॒रद्भ्य॒ इति॑ श॒रत् - भ्यः॒ । \newline
38. स्वाहा॑ ऽहोरा॒त्रेभ्यो॑ ऽहोरा॒त्रेभ्यः॒ स्वाहा॒ स्वाहा॑ ऽहोरा॒त्रेभ्यः॑ । \newline
39. अ॒हो॒रा॒त्रेभ्यः॒ स्वाहा॒ स्वाहा॑ ऽहोरा॒त्रेभ्यो॑ ऽहोरा॒त्रेभ्यः॒ स्वाहा᳚ । \newline
40. अ॒हो॒रा॒त्रेभ्य॒ इत्य॑हः - रा॒त्रेभ्यः॑ । \newline
41. स्वाहा᳚ ऽर्द्धमा॒सेभ्यो᳚ ऽर्द्धमा॒सेभ्यः॒ स्वाहा॒ स्वाहा᳚ ऽर्द्धमा॒सेभ्यः॑ । \newline
42. अ॒र्द्ध॒मा॒सेभ्यः॒ स्वाहा॒ स्वाहा᳚ ऽर्द्धमा॒सेभ्यो᳚ ऽर्द्धमा॒सेभ्यः॒ स्वाहा᳚ । \newline
43. अ॒र्द्ध॒मा॒सेभ्य॒ इत्य॑र्द्ध - मा॒सेभ्यः॑ । \newline
44. स्वाहा॒ मासे᳚भ्यो॒ मासे᳚भ्यः॒ स्वाहा॒ स्वाहा॒ मासे᳚भ्यः । \newline
45. मासे᳚भ्यः॒ स्वाहा॒ स्वाहा॒ मासे᳚भ्यो॒ मासे᳚भ्यः॒ स्वाहा᳚ । \newline
46. स्वाह॒ र्‌तुभ्य॑ ऋ॒तुभ्यः॒ स्वाहा॒ स्वाह॒ र्‌तुभ्यः॑ । \newline
47. ऋ॒तुभ्यः॒ स्वाहा॒ स्वाह॒ र्‌तुभ्य॑ ऋ॒तुभ्यः॒ स्वाहा᳚ । \newline
48. ऋ॒तुभ्य॒ इत्यृ॒तु - भ्यः॒ । \newline
49. स्वाहा॑ संॅवथ्स॒राय॑ संॅवथ्स॒राय॒ स्वाहा॒ स्वाहा॑ संॅवथ्स॒राय॑ । \newline
50. सं॒ॅव॒थ्स॒राय॒ स्वाहा॒ स्वाहा॑ संॅवथ्स॒राय॑ संॅवथ्स॒राय॒ स्वाहा᳚ । \newline
51. सं॒ॅव॒थ्स॒रायेति॑ सं - व॒थ्स॒राय॑ । \newline
52. स्वाहा॒ सर्व॑स्मै॒ सर्व॑स्मै॒ स्वाहा॒ स्वाहा॒ सर्व॑स्मै । \newline
53. सर्व॑स्मै॒ स्वाहा॒ स्वाहा॒ सर्व॑स्मै॒ सर्व॑स्मै॒ स्वाहा᳚ । \newline
54. स्वाहेति॒ स्वाहा᳚ । \newline

\textbf{Ghana Paata } \newline

1. पृ॒थि॒व्यै स्वाहा॒ स्वाहा॑ पृथि॒व्यै पृ॑थि॒व्यै स्वाहा॒ ऽन्तरि॑क्षाया॒ न्तरि॑क्षाय॒ स्वाहा॑ पृथि॒व्यै पृ॑थि॒व्यै स्वाहा॒ ऽन्तरि॑क्षाय । \newline
2. स्वाहा॒ ऽन्तरि॑क्षाया॒ न्तरि॑क्षाय॒ स्वाहा॒ स्वाहा॒ ऽन्तरि॑क्षाय॒ स्वाहा॒ स्वाहा॒ ऽन्तरि॑क्षाय॒ स्वाहा॒ स्वाहा॒ ऽन्तरि॑क्षाय॒ स्वाहा᳚ । \newline
3. अ॒न्तरि॑क्षाय॒ स्वाहा॒ स्वाहा॒ ऽन्तरि॑क्षाया॒ न्तरि॑क्षाय॒ स्वाहा॑ दि॒वे दि॒वे स्वाहा॒ ऽन्तरि॑क्षाया॒ न्तरि॑क्षाय॒ स्वाहा॑ दि॒वे । \newline
4. स्वाहा॑ दि॒वे दि॒वे स्वाहा॒ स्वाहा॑ दि॒वे स्वाहा॒ स्वाहा॑ दि॒वे स्वाहा॒ स्वाहा॑ दि॒वे स्वाहा᳚ । \newline
5. दि॒वे स्वाहा॒ स्वाहा॑ दि॒वे दि॒वे स्वाहा॒ सूर्या॑य॒ सूर्या॑य॒ स्वाहा॑ दि॒वे दि॒वे स्वाहा॒ सूर्या॑य । \newline
6. स्वाहा॒ सूर्या॑य॒ सूर्या॑य॒ स्वाहा॒ स्वाहा॒ सूर्या॑य॒ स्वाहा॒ स्वाहा॒ सूर्या॑य॒ स्वाहा॒ स्वाहा॒ सूर्या॑य॒ स्वाहा᳚ । \newline
7. सूर्या॑य॒ स्वाहा॒ स्वाहा॒ सूर्या॑य॒ सूर्या॑य॒ स्वाहा॑ च॒न्द्रम॑से च॒न्द्रम॑से॒ स्वाहा॒ सूर्या॑य॒ सूर्या॑य॒ स्वाहा॑ च॒न्द्रम॑से । \newline
8. स्वाहा॑ च॒न्द्रम॑से च॒न्द्रम॑से॒ स्वाहा॒ स्वाहा॑ च॒न्द्रम॑से॒ स्वाहा॒ स्वाहा॑ च॒न्द्रम॑से॒ स्वाहा॒ स्वाहा॑ च॒न्द्रम॑से॒ स्वाहा᳚ । \newline
9. च॒न्द्रम॑से॒ स्वाहा॒ स्वाहा॑ च॒न्द्रम॑से च॒न्द्रम॑से॒ स्वाहा॒ नक्ष॑त्रेभ्यो॒ नक्ष॑त्रेभ्यः॒ स्वाहा॑ च॒न्द्रम॑से च॒न्द्रम॑से॒ स्वाहा॒ नक्ष॑त्रेभ्यः । \newline
10. स्वाहा॒ नक्ष॑त्रेभ्यो॒ नक्ष॑त्रेभ्यः॒ स्वाहा॒ स्वाहा॒ नक्ष॑त्रेभ्यः॒ स्वाहा॒ स्वाहा॒ नक्ष॑त्रेभ्यः॒ स्वाहा॒ स्वाहा॒ नक्ष॑त्रेभ्यः॒ स्वाहा᳚ । \newline
11. नक्ष॑त्रेभ्यः॒ स्वाहा॒ स्वाहा॒ नक्ष॑त्रेभ्यो॒ नक्ष॑त्रेभ्यः॒ स्वाहा॒ प्राच्यै॒ प्राच्यै॒ स्वाहा॒ नक्ष॑त्रेभ्यो॒ नक्ष॑त्रेभ्यः॒ स्वाहा॒ प्राच्यै᳚ । \newline
12. स्वाहा॒ प्राच्यै॒ प्राच्यै॒ स्वाहा॒ स्वाहा॒ प्राच्यै॑ दि॒शे दि॒शे प्राच्यै॒ स्वाहा॒ स्वाहा॒ प्राच्यै॑ दि॒शे । \newline
13. प्राच्यै॑ दि॒शे दि॒शे प्राच्यै॒ प्राच्यै॑ दि॒शे स्वाहा॒ स्वाहा॑ दि॒शे प्राच्यै॒ प्राच्यै॑ दि॒शे स्वाहा᳚ । \newline
14. दि॒शे स्वाहा॒ स्वाहा॑ दि॒शे दि॒शे स्वाहा॒ दक्षि॑णायै॒ दक्षि॑णायै॒ स्वाहा॑ दि॒शे दि॒शे स्वाहा॒ दक्षि॑णायै । \newline
15. स्वाहा॒ दक्षि॑णायै॒ दक्षि॑णायै॒ स्वाहा॒ स्वाहा॒ दक्षि॑णायै दि॒शे दि॒शे दक्षि॑णायै॒ स्वाहा॒ स्वाहा॒ दक्षि॑णायै दि॒शे । \newline
16. दक्षि॑णायै दि॒शे दि॒शे दक्षि॑णायै॒ दक्षि॑णायै दि॒शे स्वाहा॒ स्वाहा॑ दि॒शे दक्षि॑णायै॒ दक्षि॑णायै दि॒शे स्वाहा᳚ । \newline
17. दि॒शे स्वाहा॒ स्वाहा॑ दि॒शे दि॒शे स्वाहा᳚ प्र॒तीच्यै᳚ प्र॒तीच्यै॒ स्वाहा॑ दि॒शे दि॒शे स्वाहा᳚ प्र॒तीच्यै᳚ । \newline
18. स्वाहा᳚ प्र॒तीच्यै᳚ प्र॒तीच्यै॒ स्वाहा॒ स्वाहा᳚ प्र॒तीच्यै॑ दि॒शे दि॒शे प्र॒तीच्यै॒ स्वाहा॒ स्वाहा᳚ प्र॒तीच्यै॑ दि॒शे । \newline
19. प्र॒तीच्यै॑ दि॒शे दि॒शे प्र॒तीच्यै᳚ प्र॒तीच्यै॑ दि॒शे स्वाहा॒ स्वाहा॑ दि॒शे प्र॒तीच्यै᳚ प्र॒तीच्यै॑ दि॒शे स्वाहा᳚ । \newline
20. दि॒शे स्वाहा॒ स्वाहा॑ दि॒शे दि॒शे स्वाहोदी᳚च्या॒ उदी᳚च्यै॒ स्वाहा॑ दि॒शे दि॒शे स्वाहोदी᳚च्यै । \newline
21. स्वाहोदी᳚च्या॒ उदी᳚च्यै॒ स्वाहा॒ स्वाहोदी᳚च्यै दि॒शे दि॒श उदी᳚च्यै॒ स्वाहा॒ स्वाहोदी᳚च्यै दि॒शे । \newline
22. उदी᳚च्यै दि॒शे दि॒श उदी᳚च्या॒ उदी᳚च्यै दि॒शे स्वाहा॒ स्वाहा॑ दि॒श उदी᳚च्या॒ उदी᳚च्यै दि॒शे स्वाहा᳚ । \newline
23. दि॒शे स्वाहा॒ स्वाहा॑ दि॒शे दि॒शे स्वाहो॒र्ध्वाया॑ ऊ॒र्ध्वायै॒ स्वाहा॑ दि॒शे दि॒शे स्वाहो॒र्ध्वायै᳚ । \newline
24. स्वाहो॒र्ध्वाया॑ ऊ॒र्ध्वायै॒ स्वाहा॒ स्वाहो॒र्ध्वायै॑ दि॒शे दि॒श ऊ॒र्ध्वायै॒ स्वाहा॒ स्वाहो॒र्ध्वायै॑ दि॒शे । \newline
25. ऊ॒र्ध्वायै॑ दि॒शे दि॒श ऊ॒र्ध्वाया॑ ऊ॒र्ध्वायै॑ दि॒शे स्वाहा॒ स्वाहा॑ दि॒श ऊ॒र्ध्वाया॑ ऊ॒र्ध्वायै॑ दि॒शे स्वाहा᳚ । \newline
26. दि॒शे स्वाहा॒ स्वाहा॑ दि॒शे दि॒शे स्वाहा॑ दि॒ग्भ्यो दि॒ग्भ्यः स्वाहा॑ दि॒शे दि॒शे स्वाहा॑ दि॒ग्भ्यः । \newline
27. स्वाहा॑ दि॒ग्भ्यो दि॒ग्भ्यः स्वाहा॒ स्वाहा॑ दि॒ग्भ्यः स्वाहा॒ स्वाहा॑ दि॒ग्भ्यः स्वाहा॒ स्वाहा॑ दि॒ग्भ्यः स्वाहा᳚ । \newline
28. दि॒ग्भ्यः स्वाहा॒ स्वाहा॑ दि॒ग्भ्यो दि॒ग्भ्यः स्वाहा॑ ऽवान्तरदि॒शाभ्यो॑ ऽवान्तरदि॒शाभ्यः॒ स्वाहा॑ दि॒ग्भ्यो दि॒ग्भ्यः स्वाहा॑ ऽवान्तरदि॒शाभ्यः॑ । \newline
29. दि॒ग्भ्य इति॑ दिक् - भ्यः । \newline
30. स्वाहा॑ ऽवान्तरदि॒शाभ्यो॑ ऽवान्तरदि॒शाभ्यः॒ स्वाहा॒ स्वाहा॑ ऽवान्तरदि॒शाभ्यः॒ स्वाहा॒ स्वाहा॑ ऽवान्तरदि॒शाभ्यः॒ स्वाहा॒ स्वाहा॑ ऽवान्तरदि॒शाभ्यः॒ स्वाहा᳚ । \newline
31. अ॒वा॒न्त॒र॒दि॒शाभ्यः॒ स्वाहा॒ स्वाहा॑ ऽवान्तरदि॒शाभ्यो॑ ऽवान्तरदि॒शाभ्यः॒ स्वाहा॒ समा᳚भ्यः॒ समा᳚भ्यः॒ स्वाहा॑ ऽवान्तरदि॒शाभ्यो॑ ऽवान्तरदि॒शाभ्यः॒ स्वाहा॒ समा᳚भ्यः । \newline
32. अ॒वा॒न्त॒र॒दि॒शाभ्य॒ इत्य॑वान्तर - दि॒शाभ्यः॑ । \newline
33. स्वाहा॒ समा᳚भ्यः॒ समा᳚भ्यः॒ स्वाहा॒ स्वाहा॒ समा᳚भ्यः॒ स्वाहा॒ स्वाहा॒ समा᳚भ्यः॒ स्वाहा॒ स्वाहा॒ समा᳚भ्यः॒ स्वाहा᳚ । \newline
34. समा᳚भ्यः॒ स्वाहा॒ स्वाहा॒ समा᳚भ्यः॒ समा᳚भ्यः॒ स्वाहा॑ श॒रद्भ्यः॑ श॒रद्भ्यः॒ स्वाहा॒ समा᳚भ्यः॒ समा᳚भ्यः॒ स्वाहा॑ श॒रद्भ्यः॑ । \newline
35. स्वाहा॑ श॒रद्भ्यः॑ श॒रद्भ्यः॒ स्वाहा॒ स्वाहा॑ श॒रद्भ्यः॒ स्वाहा॒ स्वाहा॑ श॒रद्भ्यः॒ स्वाहा॒ स्वाहा॑ श॒रद्भ्यः॒ स्वाहा᳚ । \newline
36. श॒रद्भ्यः॒ स्वाहा॒ स्वाहा॑ श॒रद्भ्यः॑ श॒रद्भ्यः॒ स्वाहा॑ ऽहोरा॒त्रेभ्यो॑ ऽहोरा॒त्रेभ्यः॒ स्वाहा॑ श॒रद्भ्यः॑ श॒रद्भ्यः॒ स्वाहा॑ ऽहोरा॒त्रेभ्यः॑ । \newline
37. श॒रद्भ्य॒ इति॑ श॒रत् - भ्यः॒ । \newline
38. स्वाहा॑ ऽहोरा॒त्रेभ्यो॑ ऽहोरा॒त्रेभ्यः॒ स्वाहा॒ स्वाहा॑ ऽहोरा॒त्रेभ्यः॒ स्वाहा॒ स्वाहा॑ ऽहोरा॒त्रेभ्यः॒ स्वाहा॒ स्वाहा॑ ऽहोरा॒त्रेभ्यः॒ स्वाहा᳚ । \newline
39. अ॒हो॒रा॒त्रेभ्यः॒ स्वाहा॒ स्वाहा॑ ऽहोरा॒त्रेभ्यो॑ ऽहोरा॒त्रेभ्यः॒ स्वाहा᳚ ऽर्द्धमा॒सेभ्यो᳚ ऽर्द्धमा॒सेभ्यः॒ स्वाहा॑ ऽहोरा॒त्रेभ्यो॑ ऽहोरा॒त्रेभ्यः॒ स्वाहा᳚ ऽर्द्धमा॒सेभ्यः॑ । \newline
40. अ॒हो॒रा॒त्रेभ्य॒ इत्य॑हः - रा॒त्रेभ्यः॑ । \newline
41. स्वाहा᳚ ऽर्द्धमा॒सेभ्यो᳚ ऽर्द्धमा॒सेभ्यः॒ स्वाहा॒ स्वाहा᳚ ऽर्द्धमा॒सेभ्यः॒ स्वाहा॒ स्वाहा᳚ ऽर्द्धमा॒सेभ्यः॒ स्वाहा॒ स्वाहा᳚ ऽर्द्धमा॒सेभ्यः॒ स्वाहा᳚ । \newline
42. अ॒र्द्ध॒मा॒सेभ्यः॒ स्वाहा॒ स्वाहा᳚ ऽर्द्धमा॒सेभ्यो᳚ ऽर्द्धमा॒सेभ्यः॒ स्वाहा॒ मासे᳚भ्यो॒ मासे᳚भ्यः॒ स्वाहा᳚ ऽर्द्धमा॒सेभ्यो᳚ ऽर्द्धमा॒सेभ्यः॒ स्वाहा॒ मासे᳚भ्यः । \newline
43. अ॒र्द्ध॒मा॒सेभ्य॒ इत्य॑र्द्ध - मा॒सेभ्यः॑ । \newline
44. स्वाहा॒ मासे᳚भ्यो॒ मासे᳚भ्यः॒ स्वाहा॒ स्वाहा॒ मासे᳚भ्यः॒ स्वाहा॒ स्वाहा॒ मासे᳚भ्यः॒ स्वाहा॒ स्वाहा॒ मासे᳚भ्यः॒ स्वाहा᳚ । \newline
45. मासे᳚भ्यः॒ स्वाहा॒ स्वाहा॒ मासे᳚भ्यो॒ मासे᳚भ्यः॒ स्वाह॒ र्‌तुभ्य॑ ऋ॒तुभ्यः॒ स्वाहा॒ मासे᳚भ्यो॒ मासे᳚भ्यः॒ स्वाह॒ र्‌तुभ्यः॑ । \newline
46. स्वाह॒ र्‌तुभ्य॑ ऋ॒तुभ्यः॒ स्वाहा॒ स्वाह॒ र्‌तुभ्यः॒ स्वाहा॒ स्वाह॒ र्‌तुभ्यः॒ स्वाहा॒ स्वाह॒ र्‌तुभ्यः॒ स्वाहा᳚ । \newline
47. ऋ॒तुभ्यः॒ स्वाहा॒ स्वाह॒ र्‌तुभ्य॑ ऋ॒तुभ्यः॒ स्वाहा॑ संॅवथ्स॒राय॑ संॅवथ्स॒राय॒ स्वाह॒ र्‌तुभ्य॑ ऋ॒तुभ्यः॒ स्वाहा॑ संॅवथ्स॒राय॑ । \newline
48. ऋ॒तुभ्य॒ इत्यृ॒तु - भ्यः॒ । \newline
49. स्वाहा॑ संॅवथ्स॒राय॑ संॅवथ्स॒राय॒ स्वाहा॒ स्वाहा॑ संॅवथ्स॒राय॒ स्वाहा॒ स्वाहा॑ संॅवथ्स॒राय॒ स्वाहा॒ स्वाहा॑ संॅवथ्स॒राय॒ स्वाहा᳚ । \newline
50. सं॒ॅव॒थ्स॒राय॒ स्वाहा॒ स्वाहा॑ संॅवथ्स॒राय॑ संॅवथ्स॒राय॒ स्वाहा॒ सर्व॑स्मै॒ सर्व॑स्मै॒ स्वाहा॑ संॅवथ्स॒राय॑ संॅवथ्स॒राय॒ स्वाहा॒ सर्व॑स्मै । \newline
51. सं॒ॅव॒थ्स॒रायेति॑ सं - व॒थ्स॒राय॑ । \newline
52. स्वाहा॒ सर्व॑स्मै॒ सर्व॑स्मै॒ स्वाहा॒ स्वाहा॒ सर्व॑स्मै॒ स्वाहा॒ स्वाहा॒ सर्व॑स्मै॒ स्वाहा॒ स्वाहा॒ सर्व॑स्मै॒ स्वाहा᳚ । \newline
53. सर्व॑स्मै॒ स्वाहा॒ स्वाहा॒ सर्व॑स्मै॒ सर्व॑स्मै॒ स्वाहा᳚ । \newline
54. स्वाहेति॒ स्वाहा᳚ । \newline
\pagebreak
\markright{ TS 7.1.16.1  \hfill https://www.vedavms.in \hfill}

\section{ TS 7.1.16.1 }

\textbf{TS 7.1.16.1 } \newline
\textbf{Samhita Paata} \newline

अ॒ग्नये॒ स्वाहा॒ सोमा॑य॒ स्वाहा॑ सवि॒त्रे स्वाहा॒ सर॑स्वत्यै॒ स्वाहा॑ पू॒ष्णे स्वाहा॒ बृह॒स्पत॑ये॒ स्वाहा॒ ऽपां मोदा॑य॒ स्वाहा॑ वा॒यवे॒ स्वाहा॑ मि॒त्राय॒ स्वाहा॒ वरु॑णाय॒ स्वाहा॒ सर्व॑स्मै॒ स्वाहा᳚ ॥ \newline

\textbf{Pada Paata} \newline

अ॒ग्नये᳚ । स्वाहा᳚ । सोमा॑य । स्वाहा᳚ । स॒वि॒त्रे । स्वाहा᳚ । सर॑स्वत्यै । स्वाहा᳚ । पू॒ष्णे । स्वाहा᳚ । बृह॒स्पत॑ये । स्वाहा᳚ । अ॒पाम् । मोदा॑य । स्वाहा᳚ । वा॒यवे᳚ । स्वाहा᳚ । मि॒त्राय॑ । स्वाहा᳚ । वरु॑णाय । स्वाहा᳚ । सर्व॑स्मै । स्वाहा᳚ ॥  \newline


\textbf{Krama Paata} \newline

अ॒ग्नये॒ स्वाहा᳚ । स्वाहा॒ सोमा॑य । सोमा॑य॒ स्वाहा᳚ । स्वाहा॑ सवि॒त्रे । स॒वि॒त्रे स्वाहा᳚ । स्वाहा॒ सर॑स्वत्यै । सर॑स्वत्यै॒ स्वाहा᳚ । स्वाहा॑ पू॒ष्णे । पू॒ष्णे स्वाहा᳚ । स्वाहा॒ बृह॒स्पत॑ये । बृह॒स्पत॑ये॒ स्वाहा᳚ । स्वाहा॒ऽपाम् । अ॒पाम् मोदा॑य । मोदा॑य॒ स्वाहा᳚ । स्वाहा॑ वा॒यवे᳚ । वा॒यवे॒ स्वाहा᳚ । स्वाहा॑ मि॒त्राय॑ । मि॒त्राय॒ स्वाहा᳚ । स्वाहा॒ वरु॑णाय । वरु॑णाय॒ स्वाहा᳚ । स्वाहा॒ सर्व॑स्मै । सर्व॑स्मै॒ स्वाहा᳚ । स्वाहेति॒ स्वाहा᳚ । \newline

\textbf{Jatai Paata} \newline

1. अ॒ग्नये॒ स्वाहा॒ स्वाहा॒ ऽग्नये॒ ऽग्नये॒ स्वाहा᳚ । \newline
2. स्वाहा॒ सोमा॑य॒ सोमा॑य॒ स्वाहा॒ स्वाहा॒ सोमा॑य । \newline
3. सोमा॑य॒ स्वाहा॒ स्वाहा॒ सोमा॑य॒ सोमा॑य॒ स्वाहा᳚ । \newline
4. स्वाहा॑ सवि॒त्रे स॑वि॒त्रे स्वाहा॒ स्वाहा॑ सवि॒त्रे । \newline
5. स॒वि॒त्रे स्वाहा॒ स्वाहा॑ सवि॒त्रे स॑वि॒त्रे स्वाहा᳚ । \newline
6. स्वाहा॒ सर॑स्वत्यै॒ सर॑स्वत्यै॒ स्वाहा॒ स्वाहा॒ सर॑स्वत्यै । \newline
7. सर॑स्वत्यै॒ स्वाहा॒ स्वाहा॒ सर॑स्वत्यै॒ सर॑स्वत्यै॒ स्वाहा᳚ । \newline
8. स्वाहा॑ पू॒ष्णे पू॒ष्णे स्वाहा॒ स्वाहा॑ पू॒ष्णे । \newline
9. पू॒ष्णे स्वाहा॒ स्वाहा॑ पू॒ष्णे पू॒ष्णे स्वाहा᳚ । \newline
10. स्वाहा॒ बृह॒स्पत॑ये॒ बृह॒स्पत॑ये॒ स्वाहा॒ स्वाहा॒ बृह॒स्पत॑ये । \newline
11. बृह॒स्पत॑ये॒ स्वाहा॒ स्वाहा॒ बृह॒स्पत॑ये॒ बृह॒स्पत॑ये॒ स्वाहा᳚ । \newline
12. स्वाहा॒ ऽपा म॒पाꣳ स्वाहा॒ स्वाहा॒ ऽपाम् । \newline
13. अ॒पाम् मोदा॑य॒ मोदा॑या॒ पा म॒पाम् मोदा॑य । \newline
14. मोदा॑य॒ स्वाहा॒ स्वाहा॒ मोदा॑य॒ मोदा॑य॒ स्वाहा᳚ । \newline
15. स्वाहा॑ वा॒यवे॑ वा॒यवे॒ स्वाहा॒ स्वाहा॑ वा॒यवे᳚ । \newline
16. वा॒यवे॒ स्वाहा॒ स्वाहा॑ वा॒यवे॑ वा॒यवे॒ स्वाहा᳚ । \newline
17. स्वाहा॑ मि॒त्राय॑ मि॒त्राय॒ स्वाहा॒ स्वाहा॑ मि॒त्राय॑ । \newline
18. मि॒त्राय॒ स्वाहा॒ स्वाहा॑ मि॒त्राय॑ मि॒त्राय॒ स्वाहा᳚ । \newline
19. स्वाहा॒ वरु॑णाय॒ वरु॑णाय॒ स्वाहा॒ स्वाहा॒ वरु॑णाय । \newline
20. वरु॑णाय॒ स्वाहा॒ स्वाहा॒ वरु॑णाय॒ वरु॑णाय॒ स्वाहा᳚ । \newline
21. स्वाहा॒ सर्व॑स्मै॒ सर्व॑स्मै॒ स्वाहा॒ स्वाहा॒ सर्व॑स्मै । \newline
22. सर्व॑स्मै॒ स्वाहा॒ स्वाहा॒ सर्व॑स्मै॒ सर्व॑स्मै॒ स्वाहा᳚ । \newline
23. स्वाहेति॒ स्वाहा᳚ । \newline

\textbf{Ghana Paata } \newline

1. अ॒ग्नये॒ स्वाहा॒ स्वाहा॒ ऽग्नये॒ ऽग्नये॒ स्वाहा॒ सोमा॑य॒ सोमा॑य॒ स्वाहा॒ ऽग्नये॒ ऽग्नये॒ स्वाहा॒ सोमा॑य । \newline
2. स्वाहा॒ सोमा॑य॒ सोमा॑य॒ स्वाहा॒ स्वाहा॒ सोमा॑य॒ स्वाहा॒ स्वाहा॒ सोमा॑य॒ स्वाहा॒ स्वाहा॒ सोमा॑य॒ स्वाहा᳚ । \newline
3. सोमा॑य॒ स्वाहा॒ स्वाहा॒ सोमा॑य॒ सोमा॑य॒ स्वाहा॑ सवि॒त्रे स॑वि॒त्रे स्वाहा॒ सोमा॑य॒ सोमा॑य॒ स्वाहा॑ सवि॒त्रे । \newline
4. स्वाहा॑ सवि॒त्रे स॑वि॒त्रे स्वाहा॒ स्वाहा॑ सवि॒त्रे स्वाहा॒ स्वाहा॑ सवि॒त्रे स्वाहा॒ स्वाहा॑ सवि॒त्रे स्वाहा᳚ । \newline
5. स॒वि॒त्रे स्वाहा॒ स्वाहा॑ सवि॒त्रे स॑वि॒त्रे स्वाहा॒ सर॑स्वत्यै॒ सर॑स्वत्यै॒ स्वाहा॑ सवि॒त्रे स॑वि॒त्रे स्वाहा॒ सर॑स्वत्यै । \newline
6. स्वाहा॒ सर॑स्वत्यै॒ सर॑स्वत्यै॒ स्वाहा॒ स्वाहा॒ सर॑स्वत्यै॒ स्वाहा॒ स्वाहा॒ सर॑स्वत्यै॒ स्वाहा॒ स्वाहा॒ सर॑स्वत्यै॒ स्वाहा᳚ । \newline
7. सर॑स्वत्यै॒ स्वाहा॒ स्वाहा॒ सर॑स्वत्यै॒ सर॑स्वत्यै॒ स्वाहा॑ पू॒ष्णे पू॒ष्णे स्वाहा॒ सर॑स्वत्यै॒ सर॑स्वत्यै॒ स्वाहा॑ पू॒ष्णे । \newline
8. स्वाहा॑ पू॒ष्णे पू॒ष्णे स्वाहा॒ स्वाहा॑ पू॒ष्णे स्वाहा॒ स्वाहा॑ पू॒ष्णे स्वाहा॒ स्वाहा॑ पू॒ष्णे स्वाहा᳚ । \newline
9. पू॒ष्णे स्वाहा॒ स्वाहा॑ पू॒ष्णे पू॒ष्णे स्वाहा॒ बृह॒स्पत॑ये॒ बृह॒स्पत॑ये॒ स्वाहा॑ पू॒ष्णे पू॒ष्णे स्वाहा॒ बृह॒स्पत॑ये । \newline
10. स्वाहा॒ बृह॒स्पत॑ये॒ बृह॒स्पत॑ये॒ स्वाहा॒ स्वाहा॒ बृह॒स्पत॑ये॒ स्वाहा॒ स्वाहा॒ बृह॒स्पत॑ये॒ स्वाहा॒ स्वाहा॒ बृह॒स्पत॑ये॒ स्वाहा᳚ । \newline
11. बृह॒स्पत॑ये॒ स्वाहा॒ स्वाहा॒ बृह॒स्पत॑ये॒ बृह॒स्पत॑ये॒ स्वाहा॒ ऽपा म॒पाꣳ स्वाहा॒ बृह॒स्पत॑ये॒ बृह॒स्पत॑ये॒ स्वाहा॒ ऽपाम् । \newline
12. स्वाहा॒ ऽपा म॒पाꣳ स्वाहा॒ स्वाहा॒ ऽपाम् मोदा॑य॒ मोदा॑या॒ पाꣳ स्वाहा॒ स्वाहा॒ ऽपाम् मोदा॑य । \newline
13. अ॒पाम् मोदा॑य॒ मोदा॑या॒ पा म॒पाम् मोदा॑य॒ स्वाहा॒ स्वाहा॒ मोदा॑या॒ पा म॒पाम् मोदा॑य॒ स्वाहा᳚ । \newline
14. मोदा॑य॒ स्वाहा॒ स्वाहा॒ मोदा॑य॒ मोदा॑य॒ स्वाहा॑ वा॒यवे॑ वा॒यवे॒ स्वाहा॒ मोदा॑य॒ मोदा॑य॒ स्वाहा॑ वा॒यवे᳚ । \newline
15. स्वाहा॑ वा॒यवे॑ वा॒यवे॒ स्वाहा॒ स्वाहा॑ वा॒यवे॒ स्वाहा॒ स्वाहा॑ वा॒यवे॒ स्वाहा॒ स्वाहा॑ वा॒यवे॒ स्वाहा᳚ । \newline
16. वा॒यवे॒ स्वाहा॒ स्वाहा॑ वा॒यवे॑ वा॒यवे॒ स्वाहा॑ मि॒त्राय॑ मि॒त्राय॒ स्वाहा॑ वा॒यवे॑ वा॒यवे॒ स्वाहा॑ मि॒त्राय॑ । \newline
17. स्वाहा॑ मि॒त्राय॑ मि॒त्राय॒ स्वाहा॒ स्वाहा॑ मि॒त्राय॒ स्वाहा॒ स्वाहा॑ मि॒त्राय॒ स्वाहा॒ स्वाहा॑ मि॒त्राय॒ स्वाहा᳚ । \newline
18. मि॒त्राय॒ स्वाहा॒ स्वाहा॑ मि॒त्राय॑ मि॒त्राय॒ स्वाहा॒ वरु॑णाय॒ वरु॑णाय॒ स्वाहा॑ मि॒त्राय॑ मि॒त्राय॒ स्वाहा॒ वरु॑णाय । \newline
19. स्वाहा॒ वरु॑णाय॒ वरु॑णाय॒ स्वाहा॒ स्वाहा॒ वरु॑णाय॒ स्वाहा॒ स्वाहा॒ वरु॑णाय॒ स्वाहा॒ स्वाहा॒ वरु॑णाय॒ स्वाहा᳚ । \newline
20. वरु॑णाय॒ स्वाहा॒ स्वाहा॒ वरु॑णाय॒ वरु॑णाय॒ स्वाहा॒ सर्व॑स्मै॒ सर्व॑स्मै॒ स्वाहा॒ वरु॑णाय॒ वरु॑णाय॒ स्वाहा॒ सर्व॑स्मै । \newline
21. स्वाहा॒ सर्व॑स्मै॒ सर्व॑स्मै॒ स्वाहा॒ स्वाहा॒ सर्व॑स्मै॒ स्वाहा॒ स्वाहा॒ सर्व॑स्मै॒ स्वाहा॒ स्वाहा॒ सर्व॑स्मै॒ स्वाहा᳚ । \newline
22. सर्व॑स्मै॒ स्वाहा॒ स्वाहा॒ सर्व॑स्मै॒ सर्व॑स्मै॒ स्वाहा᳚ । \newline
23. स्वाहेति॒ स्वाहा᳚ । \newline
\pagebreak
\markright{ TS 7.1.17.1  \hfill https://www.vedavms.in \hfill}

\section{ TS 7.1.17.1 }

\textbf{TS 7.1.17.1 } \newline
\textbf{Samhita Paata} \newline

पृ॒थि॒व्यै स्वाहा॒ ऽन्तरि॑क्षाय॒ स्वाहा॑ दि॒वे स्वाहा॒ ऽग्नये॒ स्वाहा॒ सोमा॑य॒ स्वाहा॒ सूर्या॑य॒ स्वाहा॑ च॒न्द्रम॑से॒ स्वाहा ऽह्ने॒ स्वाहा॒ रात्रि॑यै॒ स्वाह॒र्जवे॒ स्वाहा॑ सा॒धवे॒ स्वाहा॑ सुक्षि॒त्यै स्वाहा᳚ क्षु॒धे स्वाहा॑ ऽऽशिति॒म्ने स्वाहा॒ रोगा॑य॒ स्वाहा॑ हि॒माय॒ स्वाहा॑ शी॒ताय॒ स्वाहा॑ ऽऽत॒पाय॒ स्वाहा ऽर॑ण्याय॒ स्वाहा॑ सुव॒र्गाय॒ स्वाहा॑ लो॒काय॒ स्वाहा॒ सर्व॑स्मै॒ स्वाहा᳚ ॥ \newline

\textbf{Pada Paata} \newline

पृ॒थि॒व्यै । स्वाहा᳚ । अ॒न्तरि॑क्षाय । स्वाहा᳚ । दि॒वे । स्वाहा᳚ । अ॒ग्नये᳚ । स्वाहा᳚ । सोमा॑य । स्वाहा᳚ । सूर्या॑य । स्वाहा᳚ । च॒न्द्रम॑से । स्वाहा᳚ । अह्ने᳚ । स्वाहा᳚ । रात्रि॑यै । स्वाहा᳚ । ऋ॒जवे᳚ । स्वाहा᳚ । सा॒धवे᳚ । स्वाहा᳚ । सु॒क्षि॒त्या इति॑ सु - क्षि॒त्यै । स्वाहा᳚ । क्षु॒धे । स्वाहा᳚ । आ॒शि॒ति॒म्ने । स्वाहा᳚ । रोगा॑य । स्वाहा᳚ । हि॒माय॑ । स्वाहा᳚ । शी॒ताय॑ । स्वाहा᳚ । आ॒त॒पायेत्या᳚ - त॒पाय॑ । स्वाहा᳚ । अर॑ण्याय । स्वाहा᳚ । सु॒व॒र्गायेति॑ सुवः - गाय॑ । स्वाहा᳚ । लो॒काय॑ । स्वाहा᳚ । सर्व॑स्मै । स्वाहा᳚ ॥  \newline


\textbf{Krama Paata} \newline

पृ॒थि॒व्यै स्वाहा᳚ । स्वाहा॒ऽन्तरि॑क्षाय । अ॒न्तरि॑क्षाय॒ स्वाहा᳚ । स्वाहा॑ दि॒वे । दि॒वे स्वाहा᳚ । स्वाहा॒ऽग्नये᳚ । अ॒ग्नये॒ स्वाहा᳚ । स्वाहा॒ सोमा॑य । सोमा॑य॒ स्वाहा᳚ । स्वाहा॒ सूर्या॑य । सूर्या॑य॒ स्वाहा᳚ । स्वाहा॑ च॒न्द्रम॑से । च॒न्द्रम॑से॒ स्वाहा᳚ । स्वाहाऽह्ने᳚ । अह्ने॒ स्वाहा᳚ । स्वाहा॒ रात्रि॑यै । रात्रि॑यै॒ स्वाहा᳚ । स्वाह॒र्जवे᳚ । ऋ॒जवे॒ स्वाहा᳚ । स्वाहा॑ सा॒धवे᳚ । सा॒धवे॒ स्वाहा᳚ । स्वाहा॑ सुक्षि॒त्यै । सु॒क्षि॒त्यै स्वाहा᳚ । सु॒क्षि॒त्या इति॑ सु - क्षि॒त्यै । स्वाहा᳚ क्षु॒धे । क्षु॒धे स्वाहा᳚ । स्वाहा॑ऽऽशिति॒म्ने । आ॒शि॒ति॒म्ने स्वाहा᳚ । स्वाहा॒ रोगा॑य । रोगा॑य॒ स्वाहा᳚ । स्वाहा॑ हि॒माय॑ । हि॒माय॒ स्वाहा᳚ । स्वाहा॑ शी॒ताय॑ । शी॒ताय॒ स्वाहा᳚ । स्वाहा॑ऽऽत॒पाय॑ । आ॒त॒पाय॒ स्वाहा᳚ । आ॒त॒पायेत्या᳚ - त॒पाय॑ । स्वाहाऽर॑ण्याय । अर॑ण्याय॒ स्वाहा᳚ । स्वाहा॑ सुव॒र्गाय॑ । सु॒व॒र्गाय॒ स्वाहा᳚ । सु॒व॒र्गायेति॑ सुवः - गाय॑ । स्वाहा॑ लो॒काय॑ । लो॒काय॒ स्वाहा᳚ । स्वाहा॒ सर्व॑स्मै । सर्व॑स्मै॒ स्वाहा᳚ । स्वाहेति॒ स्वाहा᳚ । \newline

\textbf{Jatai Paata} \newline

1. पृ॒थि॒व्यै स्वाहा॒ स्वाहा॑ पृथि॒व्यै पृ॑थि॒व्यै स्वाहा᳚ । \newline
2. स्वाहा॒ ऽन्तरि॑क्षाया॒ न्तरि॑क्षाय॒ स्वाहा॒ स्वाहा॒ ऽन्तरि॑क्षाय । \newline
3. अ॒न्तरि॑क्षाय॒ स्वाहा॒ स्वाहा॒ ऽन्तरि॑क्षाया॒ न्तरि॑क्षाय॒ स्वाहा᳚ । \newline
4. स्वाहा॑ दि॒वे दि॒वे स्वाहा॒ स्वाहा॑ दि॒वे । \newline
5. दि॒वे स्वाहा॒ स्वाहा॑ दि॒वे दि॒वे स्वाहा᳚ । \newline
6. स्वाहा॒ ऽग्नये॒ ऽग्नये॒ स्वाहा॒ स्वाहा॒ ऽग्नये᳚ । \newline
7. अ॒ग्नये॒ स्वाहा॒ स्वाहा॒ ऽग्नये॒ ऽग्नये॒ स्वाहा᳚ । \newline
8. स्वाहा॒ सोमा॑य॒ सोमा॑य॒ स्वाहा॒ स्वाहा॒ सोमा॑य । \newline
9. सोमा॑य॒ स्वाहा॒ स्वाहा॒ सोमा॑य॒ सोमा॑य॒ स्वाहा᳚ । \newline
10. स्वाहा॒ सूर्या॑य॒ सूर्या॑य॒ स्वाहा॒ स्वाहा॒ सूर्या॑य । \newline
11. सूर्या॑य॒ स्वाहा॒ स्वाहा॒ सूर्या॑य॒ सूर्या॑य॒ स्वाहा᳚ । \newline
12. स्वाहा॑ च॒न्द्रम॑से च॒न्द्रम॑से॒ स्वाहा॒ स्वाहा॑ च॒न्द्रम॑से । \newline
13. च॒न्द्रम॑से॒ स्वाहा॒ स्वाहा॑ च॒न्द्रम॑से च॒न्द्रम॑से॒ स्वाहा᳚ । \newline
14. स्वाहा ऽह्ने ऽह्ने॒ स्वाहा॒ स्वाहा ऽह्ने᳚ । \newline
15. अह्ने॒ स्वाहा॒ स्वाहा ऽह्ने ऽह्ने॒ स्वाहा᳚ । \newline
16. स्वाहा॒ रात्रि॑यै॒ रात्रि॑यै॒ स्वाहा॒ स्वाहा॒ रात्रि॑यै । \newline
17. रात्रि॑यै॒ स्वाहा॒ स्वाहा॒ रात्रि॑यै॒ रात्रि॑यै॒ स्वाहा᳚ । \newline
18. स्वाह॒ र्‌जव॑ ऋ॒जवे॒ स्वाहा॒ स्वाह॒ र्‌जवे᳚ । \newline
19. ऋ॒जवे॒ स्वाहा॒ स्वाह॒ र्‌जव॑ ऋ॒जवे॒ स्वाहा᳚ । \newline
20. स्वाहा॑ सा॒धवे॑ सा॒धवे॒ स्वाहा॒ स्वाहा॑ सा॒धवे᳚ । \newline
21. सा॒धवे॒ स्वाहा॒ स्वाहा॑ सा॒धवे॑ सा॒धवे॒ स्वाहा᳚ । \newline
22. स्वाहा॑ सुक्षि॒त्यै सु॑क्षि॒त्यै स्वाहा॒ स्वाहा॑ सुक्षि॒त्यै । \newline
23. सु॒क्षि॒त्यै स्वाहा॒ स्वाहा॑ सुक्षि॒त्यै सु॑क्षि॒त्यै स्वाहा᳚ । \newline
24. सु॒क्षि॒त्या इति॑ सु - क्षि॒त्यै । \newline
25. स्वाहा᳚ क्षु॒धे क्षु॒धे स्वाहा॒ स्वाहा᳚ क्षु॒धे । \newline
26. क्षु॒धे स्वाहा॒ स्वाहा᳚ क्षु॒धे क्षु॒धे स्वाहा᳚ । \newline
27. स्वाहा॑ ऽऽशिति॒म्न आ॑शिति॒म्ने स्वाहा॒ स्वाहा॑ ऽऽशिति॒म्ने । \newline
28. आ॒शि॒ति॒म्ने स्वाहा॒ स्वाहा॑ ऽऽशिति॒म्न आ॑शिति॒म्ने स्वाहा᳚ । \newline
29. स्वाहा॒ रोगा॑य॒ रोगा॑य॒ स्वाहा॒ स्वाहा॒ रोगा॑य । \newline
30. रोगा॑य॒ स्वाहा॒ स्वाहा॒ रोगा॑य॒ रोगा॑य॒ स्वाहा᳚ । \newline
31. स्वाहा॑ हि॒माय॑ हि॒माय॒ स्वाहा॒ स्वाहा॑ हि॒माय॑ । \newline
32. हि॒माय॒ स्वाहा॒ स्वाहा॑ हि॒माय॑ हि॒माय॒ स्वाहा᳚ । \newline
33. स्वाहा॑ शी॒ताय॑ शी॒ताय॒ स्वाहा॒ स्वाहा॑ शी॒ताय॑ । \newline
34. शी॒ताय॒ स्वाहा॒ स्वाहा॑ शी॒ताय॑ शी॒ताय॒ स्वाहा᳚ । \newline
35. स्वाहा॑ ऽऽत॒पाया॑ त॒पाय॒ स्वाहा॒ स्वाहा॑ ऽऽत॒पाय॑ । \newline
36. आ॒त॒पाय॒ स्वाहा॒ स्वाहा॑ ऽऽत॒पाया॑ त॒पाय॒ स्वाहा᳚ । \newline
37. आ॒त॒पायेत्या᳚ - त॒पाय॑ । \newline
38. स्वाहा ऽर॑ण्या॒या र॑ण्याय॒ स्वाहा॒ स्वाहा ऽर॑ण्याय । \newline
39. अर॑ण्याय॒ स्वाहा॒ स्वाहा ऽर॑ण्या॒या र॑ण्याय॒ स्वाहा᳚ । \newline
40. स्वाहा॑ सुव॒र्गाय॑ सुव॒र्गाय॒ स्वाहा॒ स्वाहा॑ सुव॒र्गाय॑ । \newline
41. सु॒व॒र्गाय॒ स्वाहा॒ स्वाहा॑ सुव॒र्गाय॑ सुव॒र्गाय॒ स्वाहा᳚ । \newline
42. सु॒व॒र्गायेति॑ सुवः - गाय॑ । \newline
43. स्वाहा॑ लो॒काय॑ लो॒काय॒ स्वाहा॒ स्वाहा॑ लो॒काय॑ । \newline
44. लो॒काय॒ स्वाहा॒ स्वाहा॑ लो॒काय॑ लो॒काय॒ स्वाहा᳚ । \newline
45. स्वाहा॒ सर्व॑स्मै॒ सर्व॑स्मै॒ स्वाहा॒ स्वाहा॒ सर्व॑स्मै । \newline
46. सर्व॑स्मै॒ स्वाहा॒ स्वाहा॒ सर्व॑स्मै॒ सर्व॑स्मै॒ स्वाहा᳚ । \newline
47. स्वाहेति॒ स्वाहा᳚ । \newline

\textbf{Ghana Paata } \newline

1. पृ॒थि॒व्यै स्वाहा॒ स्वाहा॑ पृथि॒व्यै पृ॑थि॒व्यै स्वाहा॒ ऽन्तरि॑क्षाया॒ न्तरि॑क्षाय॒ स्वाहा॑ पृथि॒व्यै पृ॑थि॒व्यै स्वाहा॒ ऽन्तरि॑क्षाय । \newline
2. स्वाहा॒ ऽन्तरि॑क्षाया॒ न्तरि॑क्षाय॒ स्वाहा॒ स्वाहा॒ ऽन्तरि॑क्षाय॒ स्वाहा॒ स्वाहा॒ ऽन्तरि॑क्षाय॒ स्वाहा॒ स्वाहा॒ ऽन्तरि॑क्षाय॒ स्वाहा᳚ । \newline
3. अ॒न्तरि॑क्षाय॒ स्वाहा॒ स्वाहा॒ ऽन्तरि॑क्षाया॒ न्तरि॑क्षाय॒ स्वाहा॑ दि॒वे दि॒वे स्वाहा॒ ऽन्तरि॑क्षाया॒ न्तरि॑क्षाय॒ स्वाहा॑ दि॒वे । \newline
4. स्वाहा॑ दि॒वे दि॒वे स्वाहा॒ स्वाहा॑ दि॒वे स्वाहा॒ स्वाहा॑ दि॒वे स्वाहा॒ स्वाहा॑ दि॒वे स्वाहा᳚ । \newline
5. दि॒वे स्वाहा॒ स्वाहा॑ दि॒वे दि॒वे स्वाहा॒ ऽग्नये॒ ऽग्नये॒ स्वाहा॑ दि॒वे दि॒वे स्वाहा॒ ऽग्नये᳚ । \newline
6. स्वाहा॒ ऽग्नये॒ ऽग्नये॒ स्वाहा॒ स्वाहा॒ ऽग्नये॒ स्वाहा॒ स्वाहा॒ ऽग्नये॒ स्वाहा॒ स्वाहा॒ ऽग्नये॒ स्वाहा᳚ । \newline
7. अ॒ग्नये॒ स्वाहा॒ स्वाहा॒ ऽग्नये॒ ऽग्नये॒ स्वाहा॒ सोमा॑य॒ सोमा॑य॒ स्वाहा॒ ऽग्नये॒ ऽग्नये॒ स्वाहा॒ सोमा॑य । \newline
8. स्वाहा॒ सोमा॑य॒ सोमा॑य॒ स्वाहा॒ स्वाहा॒ सोमा॑य॒ स्वाहा॒ स्वाहा॒ सोमा॑य॒ स्वाहा॒ स्वाहा॒ सोमा॑य॒ स्वाहा᳚ । \newline
9. सोमा॑य॒ स्वाहा॒ स्वाहा॒ सोमा॑य॒ सोमा॑य॒ स्वाहा॒ सूर्या॑य॒ सूर्या॑य॒ स्वाहा॒ सोमा॑य॒ सोमा॑य॒ स्वाहा॒ सूर्या॑य । \newline
10. स्वाहा॒ सूर्या॑य॒ सूर्या॑य॒ स्वाहा॒ स्वाहा॒ सूर्या॑य॒ स्वाहा॒ स्वाहा॒ सूर्या॑य॒ स्वाहा॒ स्वाहा॒ सूर्या॑य॒ स्वाहा᳚ । \newline
11. सूर्या॑य॒ स्वाहा॒ स्वाहा॒ सूर्या॑य॒ सूर्या॑य॒ स्वाहा॑ च॒न्द्रम॑से च॒न्द्रम॑से॒ स्वाहा॒ सूर्या॑य॒ सूर्या॑य॒ स्वाहा॑ च॒न्द्रम॑से । \newline
12. स्वाहा॑ च॒न्द्रम॑से च॒न्द्रम॑से॒ स्वाहा॒ स्वाहा॑ च॒न्द्रम॑से॒ स्वाहा॒ स्वाहा॑ च॒न्द्रम॑से॒ स्वाहा॒ स्वाहा॑ च॒न्द्रम॑से॒ स्वाहा᳚ । \newline
13. च॒न्द्रम॑से॒ स्वाहा॒ स्वाहा॑ च॒न्द्रम॑से च॒न्द्रम॑से॒ स्वाहा ऽह्ने ऽह्ने॒ स्वाहा॑ च॒न्द्रम॑से च॒न्द्रम॑से॒ स्वाहा ऽह्ने᳚ । \newline
14. स्वाहा ऽह्ने ऽह्ने॒ स्वाहा॒ स्वाहा ऽह्ने॒ स्वाहा॒ स्वाहा ऽह्ने॒ स्वाहा॒ स्वाहा ऽह्ने॒ स्वाहा᳚ । \newline
15. अह्ने॒ स्वाहा॒ स्वाहा ऽह्ने ऽह्ने॒ स्वाहा॒ रात्रि॑यै॒ रात्रि॑यै॒ स्वाहा ऽह्ने ऽह्ने॒ स्वाहा॒ रात्रि॑यै । \newline
16. स्वाहा॒ रात्रि॑यै॒ रात्रि॑यै॒ स्वाहा॒ स्वाहा॒ रात्रि॑यै॒ स्वाहा॒ स्वाहा॒ रात्रि॑यै॒ स्वाहा॒ स्वाहा॒ रात्रि॑यै॒ स्वाहा᳚ । \newline
17. रात्रि॑यै॒ स्वाहा॒ स्वाहा॒ रात्रि॑यै॒ रात्रि॑यै॒ स्वाह॒ र्‌जव॑ ऋ॒जवे॒ स्वाहा॒ रात्रि॑यै॒ रात्रि॑यै॒ स्वाह॒ र्‌जवे᳚ । \newline
18. स्वाह॒ र्‌जव॑ ऋ॒जवे॒ स्वाहा॒ स्वाह॒ र्‌जवे॒ स्वाहा॒ स्वाह॒ र्‌जवे॒ स्वाहा॒ स्वाह॒ र्‌जवे॒ स्वाहा᳚ । \newline
19. ऋ॒जवे॒ स्वाहा॒ स्वाह॒ र्‌जव॑ ऋ॒जवे॒ स्वाहा॑ सा॒धवे॑ सा॒धवे॒ स्वाह॒ र्‌जव॑ ऋ॒जवे॒ स्वाहा॑ सा॒धवे᳚ । \newline
20. स्वाहा॑ सा॒धवे॑ सा॒धवे॒ स्वाहा॒ स्वाहा॑ सा॒धवे॒ स्वाहा॒ स्वाहा॑ सा॒धवे॒ स्वाहा॒ स्वाहा॑ सा॒धवे॒ स्वाहा᳚ । \newline
21. सा॒धवे॒ स्वाहा॒ स्वाहा॑ सा॒धवे॑ सा॒धवे॒ स्वाहा॑ सुक्षि॒त्यै सु॑क्षि॒त्यै स्वाहा॑ सा॒धवे॑ सा॒धवे॒ स्वाहा॑ सुक्षि॒त्यै । \newline
22. स्वाहा॑ सुक्षि॒त्यै सु॑क्षि॒त्यै स्वाहा॒ स्वाहा॑ सुक्षि॒त्यै स्वाहा॒ स्वाहा॑ सुक्षि॒त्यै स्वाहा॒ स्वाहा॑ सुक्षि॒त्यै स्वाहा᳚ । \newline
23. सु॒क्षि॒त्यै स्वाहा॒ स्वाहा॑ सुक्षि॒त्यै सु॑क्षि॒त्यै स्वाहा᳚ क्षु॒धे क्षु॒धे स्वाहा॑ सुक्षि॒त्यै सु॑क्षि॒त्यै स्वाहा᳚ क्षु॒धे । \newline
24. सु॒क्षि॒त्या इति॑ सु - क्षि॒त्यै । \newline
25. स्वाहा᳚ क्षु॒धे क्षु॒धे स्वाहा॒ स्वाहा᳚ क्षु॒धे स्वाहा॒ स्वाहा᳚ क्षु॒धे स्वाहा॒ स्वाहा᳚ क्षु॒धे स्वाहा᳚ । \newline
26. क्षु॒धे स्वाहा॒ स्वाहा᳚ क्षु॒धे क्षु॒धे स्वाहा॑ ऽऽशिति॒म्न आ॑शिति॒म्ने स्वाहा᳚ क्षु॒धे क्षु॒धे स्वाहा॑ ऽऽशिति॒म्ने । \newline
27. स्वाहा॑ ऽऽशिति॒म्न आ॑शिति॒म्ने स्वाहा॒ स्वाहा॑ ऽऽशिति॒म्ने स्वाहा॒ स्वाहा॑ ऽऽशिति॒म्ने स्वाहा॒ स्वाहा॑ ऽऽशिति॒म्ने स्वाहा᳚ । \newline
28. आ॒शि॒ति॒म्ने स्वाहा॒ स्वाहा॑ ऽऽशिति॒म्न आ॑शिति॒म्ने स्वाहा॒ रोगा॑य॒ रोगा॑य॒ स्वाहा॑ ऽऽशिति॒म्न आ॑शिति॒म्ने स्वाहा॒ रोगा॑य । \newline
29. स्वाहा॒ रोगा॑य॒ रोगा॑य॒ स्वाहा॒ स्वाहा॒ रोगा॑य॒ स्वाहा॒ स्वाहा॒ रोगा॑य॒ स्वाहा॒ स्वाहा॒ रोगा॑य॒ स्वाहा᳚ । \newline
30. रोगा॑य॒ स्वाहा॒ स्वाहा॒ रोगा॑य॒ रोगा॑य॒ स्वाहा॑ हि॒माय॑ हि॒माय॒ स्वाहा॒ रोगा॑य॒ रोगा॑य॒ स्वाहा॑ हि॒माय॑ । \newline
31. स्वाहा॑ हि॒माय॑ हि॒माय॒ स्वाहा॒ स्वाहा॑ हि॒माय॒ स्वाहा॒ स्वाहा॑ हि॒माय॒ स्वाहा॒ स्वाहा॑ हि॒माय॒ स्वाहा᳚ । \newline
32. हि॒माय॒ स्वाहा॒ स्वाहा॑ हि॒माय॑ हि॒माय॒ स्वाहा॑ शी॒ताय॑ शी॒ताय॒ स्वाहा॑ हि॒माय॑ हि॒माय॒ स्वाहा॑ शी॒ताय॑ । \newline
33. स्वाहा॑ शी॒ताय॑ शी॒ताय॒ स्वाहा॒ स्वाहा॑ शी॒ताय॒ स्वाहा॒ स्वाहा॑ शी॒ताय॒ स्वाहा॒ स्वाहा॑ शी॒ताय॒ स्वाहा᳚ । \newline
34. शी॒ताय॒ स्वाहा॒ स्वाहा॑ शी॒ताय॑ शी॒ताय॒ स्वाहा॑ ऽऽत॒पाया॑ त॒पाय॒ स्वाहा॑ शी॒ताय॑ शी॒ताय॒ स्वाहा॑ ऽऽत॒पाय॑ । \newline
35. स्वाहा॑ ऽऽत॒पाया॑ त॒पाय॒ स्वाहा॒ स्वाहा॑ ऽऽत॒पाय॒ स्वाहा॒ स्वाहा॑ ऽऽत॒पाय॒ स्वाहा॒ स्वाहा॑ ऽऽत॒पाय॒ स्वाहा᳚ । \newline
36. आ॒त॒पाय॒ स्वाहा॒ स्वाहा॑ ऽऽत॒पाया॑ त॒पाय॒ स्वाहा ऽर॑ण्या॒या र॑ण्याय॒ स्वाहा॑ ऽऽत॒पाया॑ त॒पाय॒ स्वाहा ऽर॑ण्याय । \newline
37. आ॒त॒पायेत्या᳚ - त॒पाय॑ । \newline
38. स्वाहा ऽर॑ण्या॒या र॑ण्याय॒ स्वाहा॒ स्वाहा ऽर॑ण्याय॒ स्वाहा॒ स्वाहा ऽर॑ण्याय॒ स्वाहा॒ स्वाहा ऽर॑ण्याय॒ स्वाहा᳚ । \newline
39. अर॑ण्याय॒ स्वाहा॒ स्वाहा ऽर॑ण्या॒या र॑ण्याय॒ स्वाहा॑ सुव॒र्गाय॑ सुव॒र्गाय॒ स्वाहा ऽर॑ण्या॒या र॑ण्याय॒ स्वाहा॑ सुव॒र्गाय॑ । \newline
40. स्वाहा॑ सुव॒र्गाय॑ सुव॒र्गाय॒ स्वाहा॒ स्वाहा॑ सुव॒र्गाय॒ स्वाहा॒ स्वाहा॑ सुव॒र्गाय॒ स्वाहा॒ स्वाहा॑ सुव॒र्गाय॒ स्वाहा᳚ । \newline
41. सु॒व॒र्गाय॒ स्वाहा॒ स्वाहा॑ सुव॒र्गाय॑ सुव॒र्गाय॒ स्वाहा॑ लो॒काय॑ लो॒काय॒ स्वाहा॑ सुव॒र्गाय॑ सुव॒र्गाय॒ स्वाहा॑ लो॒काय॑ । \newline
42. सु॒व॒र्गायेति॑ सुवः - गाय॑ । \newline
43. स्वाहा॑ लो॒काय॑ लो॒काय॒ स्वाहा॒ स्वाहा॑ लो॒काय॒ स्वाहा॒ स्वाहा॑ लो॒काय॒ स्वाहा॒ स्वाहा॑ लो॒काय॒ स्वाहा᳚ । \newline
44. लो॒काय॒ स्वाहा॒ स्वाहा॑ लो॒काय॑ लो॒काय॒ स्वाहा॒ सर्व॑स्मै॒ सर्व॑स्मै॒ स्वाहा॑ लो॒काय॑ लो॒काय॒ स्वाहा॒ सर्व॑स्मै । \newline
45. स्वाहा॒ सर्व॑स्मै॒ सर्व॑स्मै॒ स्वाहा॒ स्वाहा॒ सर्व॑स्मै॒ स्वाहा॒ स्वाहा॒ सर्व॑स्मै॒ स्वाहा॒ स्वाहा॒ सर्व॑स्मै॒ स्वाहा᳚ । \newline
46. सर्व॑स्मै॒ स्वाहा॒ स्वाहा॒ सर्व॑स्मै॒ सर्व॑स्मै॒ स्वाहा᳚ । \newline
47. स्वाहेति॒ स्वाहा᳚ । \newline
\pagebreak
\markright{ TS 7.1.18.1  \hfill https://www.vedavms.in \hfill}

\section{ TS 7.1.18.1 }

\textbf{TS 7.1.18.1 } \newline
\textbf{Samhita Paata} \newline

भुवो॑ दे॒वानां॒ कर्म॑णा॒ऽपस॒र्तस्य॑ प॒थ्या॑ऽसि॒ वसु॑भि-र्दे॒वेभि॑-र्दे॒वत॑या गाय॒त्रेण॑ त्वा॒ छन्द॑सा युनज्मि वस॒न्तेन॑ त्व॒र्तुना॑ ह॒विषा॑ दीक्षयामि रु॒द्रेभि॑-र्दे॒वेभि॑-र्दे॒वत॑या॒ त्रैष्टु॑भेन त्वा॒ छन्द॑सा युनज्मि ग्री॒ष्मेण॑ त्व॒र्तुना॑ ह॒विषा॑ दीक्षया-म्यादि॒त्येभि॑-र्दे॒वेभि॑-र्दे॒वत॑या॒ जाग॑तेन त्वा॒ छन्द॑सा युनज्मि व॒र्॒.षाभि॑स्त्व॒र्तुना॑ ह॒विषा॑ दीक्षयामि॒ विश्वे॑भि-र्दे॒वेभि॑-र्दे॒वत॒या ऽऽनु॑ष्टुभेन त्वा॒ छन्द॑सा युनज्मि - [  ] \newline

\textbf{Pada Paata} \newline

भुवः॑ । दे॒वाना᳚म् । कर्म॑णा । अ॒पसा᳚ । ऋ॒तस्य॑ । प॒थ्या᳚ । अ॒सि॒ । वसु॑भि॒रिति॒ वसु॑ - भिः॒ । दे॒वेभिः॑ । दे॒वत॑या । गा॒य॒त्रेण॑ । त्वा॒ । छन्द॑सा । यु॒न॒ज्मि॒ । व॒स॒न्तेन॑ । त्वा॒ । ऋ॒तुना᳚ । ह॒विषा᳚ । दी॒क्ष॒या॒मि॒ । रु॒द्रेभिः॑ । दे॒वेभिः॑ । दे॒वत॑या । त्रैष्टु॑भेन । त्वा॒ । छन्द॑सा । यु॒न॒ज्मि॒ । ग्री॒ष्मेण॑ । त्वा॒ । ऋ॒तुना᳚ । ह॒विषा᳚ । दी॒क्ष॒या॒मि॒ । आ॒दि॒त्येभिः॑ । दे॒वेभिः॑ । दे॒वत॑या । जाग॑तेन । त्वा॒ । छन्द॑सा । यु॒न॒ज्मि॒ । व॒र्॒.षाभिः॑ । त्वा॒ । ऋ॒तुना᳚ । ह॒विषा᳚ । दी॒क्ष॒या॒मि॒ । विश्वे॑भिः । दे॒वेभिः॑ । दे॒वत॑या । आनु॑ष्टुभे॒नेत्यानु॑ - स्तु॒भे॒न॒ । त्वा॒ । छन्द॑सा । यु॒न॒ज्मि॒ ।  \newline


\textbf{Krama Paata} \newline

भुवो॑ दे॒वाना᳚म् । दे॒वाना॒म् कर्म॑णा । कर्म॑णा॒ऽपसा᳚ । अ॒पस॒र्तस्य॑ । ऋ॒तस्य॑ प॒थ्या᳚ । पथ्या॑ऽसि । अ॒सि॒ वसु॑भिः । वसु॑भिर् दे॒वेभिः॑ । वसु॑भि॒रिति॒ वसु॑ - भिः॒ । दे॒वेभि॑र् दे॒वत॑या । दे॒वत॑या गाय॒त्रेण॑ । गा॒य॒त्रेण॑ त्वा । त्वा॒ छन्द॑सा । छन्द॑सा युनज्मि । यु॒न॒ज्मि॒ व॒स॒न्तेन॑ । व॒स॒न्तेन॑ त्वा । त्व॒र्तुना᳚ । ऋ॒तुना॑ ह॒विषा᳚ । ह॒विषा॑ दीक्षयामि । दी॒क्ष॒या॒मि॒ रु॒द्रेभिः॑ । रु॒द्रेभि॑र् दे॒वेभिः॑ । दे॒वेभि॑र् दे॒वत॑या । दे॒वत॑या॒ त्रैष्टु॑भेन । त्रैष्टु॑भेन त्वा । त्वा॒ छन्द॑सा । छन्द॑सा युनज्मि । यु॒न॒ज्मि॒ ग्री॒ष्मेण॑ । ग्री॒ष्मेण॑ त्वा । त्व॒र्तुना᳚ । ऋ॒तना॑ ह॒विषा᳚ । ह॒विषा॑ दीक्षयामि । दी॒क्ष॒या॒म्या॒दि॒त्येभिः॑ । आ॒दि॒त्यैभि॑र् दे॒वेभिः॑ । दे॒वेभि॑र् दे॒वत॑या । दे॒वत॑या॒ जाग॑तेन । जाग॑तेन त्वा । त्वा॒ छन्द॑सा । छन्द॑सा युनज्मि । यु॒न॒ज्मि॒ व॒र्॒.षाभिः॑ । व॒र्॒.षाभि॑स्त्वा । त्व॒र्तुना᳚ । ऋ॒तना॑ ह॒विषा᳚ । ह॒विषा॑ दीक्षयामि । दी॒क्ष॒या॒मि॒ विश्वे॑भिः । विश्वे॑भिर् दे॒वेभिः॑ । दे॒वेभि॑र् दे॒वत॑या । दे॒वत॒याऽऽनु॑ष्टुभेन । आनु॑ष्टुभेन त्वा । आनु॑ष्टुभे॒नेत्यानु॑ - स्तु॒भे॒न॒ । त्वा॒ छन्द॑सा । छन्द॑सा युनज्मि ( ) । यु॒न॒ज्मि॒ श॒रदा᳚ \newline

\textbf{Jatai Paata} \newline

1. भुवो॑ दे॒वाना᳚म् दे॒वाना॒म् भुवो॒ भुवो॑ दे॒वाना᳚म् । \newline
2. दे॒वाना॒म् कर्म॑णा॒ कर्म॑णा दे॒वाना᳚म् दे॒वाना॒म् कर्म॑णा । \newline
3. कर्म॑णा॒ ऽपसा॒ ऽपसा॒ कर्म॑णा॒ कर्म॑णा॒ ऽपसा᳚ । \newline
4. अ॒पस॒ र्‌तस्य॒ र्‌तस्या॒ पसा॒ ऽपस॒ र्‌तस्य॑ । \newline
5. ऋ॒तस्य॑ प॒थ्या॑ प॒थ्य॑ र्‌तस्य॒ र्‌तस्य॑ प॒थ्या᳚ । \newline
6. प॒थ्या᳚ ऽस्यसि प॒थ्या॑ प॒थ्या॑ ऽसि । \newline
7. अ॒सि॒ वसु॑भि॒र् वसु॑भि रस्यसि॒ वसु॑भिः । \newline
8. वसु॑भिर् दे॒वेभि॑र् दे॒वेभि॒र् वसु॑भि॒र् वसु॑भिर् दे॒वेभिः॑ । \newline
9. वसु॑भि॒रिति॒ वसु॑ - भिः॒ । \newline
10. दे॒वेभि॑र् दे॒वत॑या दे॒वत॑या दे॒वेभि॑र् दे॒वेभि॑र् दे॒वत॑या । \newline
11. दे॒वत॑या गाय॒त्रेण॑ गाय॒त्रेण॑ दे॒वत॑या दे॒वत॑या गाय॒त्रेण॑ । \newline
12. गा॒य॒त्रेण॑ त्वा त्वा गाय॒त्रेण॑ गाय॒त्रेण॑ त्वा । \newline
13. त्वा॒ छन्द॑सा॒ छन्द॑सा त्वा त्वा॒ छन्द॑सा । \newline
14. छन्द॑सा युनज्मि युनज्मि॒ छन्द॑सा॒ छन्द॑सा युनज्मि । \newline
15. यु॒न॒ज्मि॒ व॒स॒न्तेन॑ वस॒न्तेन॑ युनज्मि युनज्मि वस॒न्तेन॑ । \newline
16. व॒स॒न्तेन॑ त्वा त्वा वस॒न्तेन॑ वस॒न्तेन॑ त्वा । \newline
17. त्व॒ र्‌तुन॒ र्‌तुना᳚ त्वा त्व॒ र्‌तुना᳚ । \newline
18. ऋ॒तुना॑ ह॒विषा॑ ह॒विष॒ र्‌तुन॒ र्‌तुना॑ ह॒विषा᳚ । \newline
19. ह॒विषा॑ दीक्षयामि दीक्षयामि ह॒विषा॑ ह॒विषा॑ दीक्षयामि । \newline
20. दी॒क्ष॒या॒मि॒ रु॒द्रेभी॑ रु॒द्रेभि॑र् दीक्षयामि दीक्षयामि रु॒द्रेभिः॑ । \newline
21. रु॒द्रेभि॑र् दे॒वेभि॑र् दे॒वेभी॑ रु॒द्रेभी॑ रु॒द्रेभि॑र् दे॒वेभिः॑ । \newline
22. दे॒वेभि॑र् दे॒वत॑या दे॒वत॑या दे॒वेभि॑र् दे॒वेभि॑र् दे॒वत॑या । \newline
23. दे॒वत॑या॒ त्रैष्टु॑भेन॒ त्रैष्टु॑भेन दे॒वत॑या दे॒वत॑या॒ त्रैष्टु॑भेन । \newline
24. त्रैष्टु॑भेन त्वा त्वा॒ त्रैष्टु॑भेन॒ त्रैष्टु॑भेन त्वा । \newline
25. त्वा॒ छन्द॑सा॒ छन्द॑सा त्वा त्वा॒ छन्द॑सा । \newline
26. छन्द॑सा युनज्मि युनज्मि॒ छन्द॑सा॒ छन्द॑सा युनज्मि । \newline
27. यु॒न॒ज्मि॒ ग्री॒ष्मेण॑ ग्री॒ष्मेण॑ युनज्मि युनज्मि ग्री॒ष्मेण॑ । \newline
28. ग्री॒ष्मेण॑ त्वा त्वा ग्री॒ष्मेण॑ ग्री॒ष्मेण॑ त्वा । \newline
29. त्व॒ र्‌तुन॒ र्‌तुना᳚ त्वा त्व॒ र्‌तुना᳚ । \newline
30. ऋ॒तुना॑ ह॒विषा॑ ह॒विष॒ र्‌तुन॒ र्‌तुना॑ ह॒विषा᳚ । \newline
31. ह॒विषा॑ दीक्षयामि दीक्षयामि ह॒विषा॑ ह॒विषा॑ दीक्षयामि । \newline
32. दी॒क्ष॒या॒ म्या॒दि॒त्येभि॑ रादि॒त्येभि॑र् दीक्षयामि दीक्षया म्यादि॒त्येभिः॑ । \newline
33. आ॒दि॒त्येभि॑र् दे॒वेभि॑र् दे॒वेभि॑ रादि॒त्येभि॑ रादि॒त्येभि॑र् दे॒वेभिः॑ । \newline
34. दे॒वेभि॑र् दे॒वत॑या दे॒वत॑या दे॒वेभि॑र् दे॒वेभि॑र् दे॒वत॑या । \newline
35. दे॒वत॑या॒ जाग॑तेन॒ जाग॑तेन दे॒वत॑या दे॒वत॑या॒ जाग॑तेन । \newline
36. जाग॑तेन त्वा त्वा॒ जाग॑तेन॒ जाग॑तेन त्वा । \newline
37. त्वा॒ छन्द॑सा॒ छन्द॑सा त्वा त्वा॒ छन्द॑सा । \newline
38. छन्द॑सा युनज्मि युनज्मि॒ छन्द॑सा॒ छन्द॑सा युनज्मि । \newline
39. यु॒न॒ज्मि॒ व॒र्॒.षाभि॑र् व॒र्॒.षाभि॑र् युनज्मि युनज्मि व॒र्॒.षाभिः॑ । \newline
40. व॒र्॒.षाभि॑ स्त्वा त्वा व॒र्॒.षाभि॑र् व॒र्॒.षाभि॑ स्त्वा । \newline
41. त्व॒ र्‌तुन॒ र्‌तुना᳚ त्वा त्व॒ र्‌तुना᳚ । \newline
42. ऋ॒तुना॑ ह॒विषा॑ ह॒विष॒ र्‌तुन॒ र्‌तुना॑ ह॒विषा᳚ । \newline
43. ह॒विषा॑ दीक्षयामि दीक्षयामि ह॒विषा॑ ह॒विषा॑ दीक्षयामि । \newline
44. दी॒क्ष॒या॒मि॒ विश्वे॑भि॒र् विश्वे॑भिर् दीक्षयामि दीक्षयामि॒ विश्वे॑भिः । \newline
45. विश्वे॑भिर् दे॒वेभि॑र् दे॒वेभि॒र् विश्वे॑भि॒र् विश्वे॑भिर् दे॒वेभिः॑ । \newline
46. दे॒वेभि॑र् दे॒वत॑या दे॒वत॑या दे॒वेभि॑र् दे॒वेभि॑र् दे॒वत॑या । \newline
47. दे॒वत॒या ऽऽनु॑ष्टुभे॒ना नु॑ष्टुभेन दे॒वत॑या दे॒वत॒या ऽऽनु॑ष्टुभेन । \newline
48. आनु॑ष्टुभेन त्वा॒ त्वा ऽऽनु॑ष्टुभे॒ना नु॑ष्टुभेन त्वा । \newline
49. आनु॑ष्टुभे॒नेत्यानु॑ - स्तु॒भे॒न॒ । \newline
50. त्वा॒ छन्द॑सा॒ छन्द॑सा त्वा त्वा॒ छन्द॑सा । \newline
51. छन्द॑सा युनज्मि युनज्मि॒ छन्द॑सा॒ छन्द॑सा युनज्मि । \newline
52. यु॒न॒ज्मि॒ श॒रदा॑ श॒रदा॑ युनज्मि युनज्मि श॒रदा᳚ । \newline

\textbf{Ghana Paata } \newline

1. भुवो॑ दे॒वाना᳚म् दे॒वाना॒म् भुवो॒ भुवो॑ दे॒वाना॒म् कर्म॑णा॒ कर्म॑णा दे॒वाना॒म् भुवो॒ भुवो॑ दे॒वाना॒म् कर्म॑णा । \newline
2. दे॒वाना॒म् कर्म॑णा॒ कर्म॑णा दे॒वाना᳚म् दे॒वाना॒म् कर्म॑णा॒ ऽपसा॒ ऽपसा॒ कर्म॑णा दे॒वाना᳚म् दे॒वाना॒म् कर्म॑णा॒ ऽपसा᳚ । \newline
3. कर्म॑णा॒ ऽपसा॒ ऽपसा॒ कर्म॑णा॒ कर्म॑णा॒ ऽपस॒ र्‌तस्य॒ र्‌तस्या॒पसा॒ कर्म॑णा॒ कर्म॑णा॒ ऽपस॒ र्‌तस्य॑ । \newline
4. अ॒पस॒ र्‌तस्य॒ र्‌तस्या॒ पसा॒ ऽपस॒ र्‌तस्य॑ प॒थ्या॑ प॒थ्य॑ र्‌तस्या॒ पसा॒ ऽपस॒ र्‌तस्य॑ प॒थ्या᳚ । \newline
5. ऋ॒तस्य॑ प॒थ्या॑ प॒थ्य॑ र्‌तस्य॒ र्‌तस्य॑ प॒थ्या᳚ ऽस्यसि प॒थ्य॑ र्‌तस्य॒ र्‌तस्य॑ प॒थ्या॑ ऽसि । \newline
6. प॒थ्या᳚ ऽस्यसि प॒थ्या॑ प॒थ्या॑ ऽसि॒ वसु॑भि॒र् वसु॑भि रसि प॒थ्या॑ प॒थ्या॑ ऽसि॒ वसु॑भिः । \newline
7. अ॒सि॒ वसु॑भि॒र् वसु॑भि रस्यसि॒ वसु॑भिर् दे॒वेभि॑र् दे॒वेभि॒र् वसु॑भि रस्यसि॒ वसु॑भिर् दे॒वेभिः॑ । \newline
8. वसु॑भिर् दे॒वेभि॑र् दे॒वेभि॒र् वसु॑भि॒र् वसु॑भिर् दे॒वेभि॑र् दे॒वत॑या दे॒वत॑या दे॒वेभि॒र् वसु॑भि॒र् वसु॑भिर् दे॒वेभि॑र् दे॒वत॑या । \newline
9. वसु॑भि॒रिति॒ वसु॑ - भिः॒ । \newline
10. दे॒वेभि॑र् दे॒वत॑या दे॒वत॑या दे॒वेभि॑र् दे॒वेभि॑र् दे॒वत॑या गाय॒त्रेण॑ गाय॒त्रेण॑ दे॒वत॑या दे॒वेभि॑र् दे॒वेभि॑र् दे॒वत॑या गाय॒त्रेण॑ । \newline
11. दे॒वत॑या गाय॒त्रेण॑ गाय॒त्रेण॑ दे॒वत॑या दे॒वत॑या गाय॒त्रेण॑ त्वा त्वा गाय॒त्रेण॑ दे॒वत॑या दे॒वत॑या गाय॒त्रेण॑ त्वा । \newline
12. गा॒य॒त्रेण॑ त्वा त्वा गाय॒त्रेण॑ गाय॒त्रेण॑ त्वा॒ छन्द॑सा॒ छन्द॑सा त्वा गाय॒त्रेण॑ गाय॒त्रेण॑ त्वा॒ छन्द॑सा । \newline
13. त्वा॒ छन्द॑सा॒ छन्द॑सा त्वा त्वा॒ छन्द॑सा युनज्मि युनज्मि॒ छन्द॑सा त्वा त्वा॒ छन्द॑सा युनज्मि । \newline
14. छन्द॑सा युनज्मि युनज्मि॒ छन्द॑सा॒ छन्द॑सा युनज्मि वस॒न्तेन॑ वस॒न्तेन॑ युनज्मि॒ छन्द॑सा॒ छन्द॑सा युनज्मि वस॒न्तेन॑ । \newline
15. यु॒न॒ज्मि॒ व॒स॒न्तेन॑ वस॒न्तेन॑ युनज्मि युनज्मि वस॒न्तेन॑ त्वा त्वा वस॒न्तेन॑ युनज्मि युनज्मि वस॒न्तेन॑ त्वा । \newline
16. व॒स॒न्तेन॑ त्वा त्वा वस॒न्तेन॑ वस॒न्तेन॑ त्व॒ र्‌तुन॒ र्‌तुना᳚ त्वा वस॒न्तेन॑ वस॒न्तेन॑ त्व॒ र्‌तुना᳚ । \newline
17. त्व॒ र्‌तुन॒ र्‌तुना᳚ त्वा त्व॒ र्‌तुना॑ ह॒विषा॑ ह॒विष॒ र्‌तुना᳚ त्वा त्व॒ र्‌तुना॑ ह॒विषा᳚ । \newline
18. ऋ॒तुना॑ ह॒विषा॑ ह॒विष॒ र्‌तुन॒ र्‌तुना॑ ह॒विषा॑ दीक्षयामि दीक्षयामि ह॒विष॒ र्‌तुन॒ र्‌तुना॑ ह॒विषा॑ दीक्षयामि । \newline
19. ह॒विषा॑ दीक्षयामि दीक्षयामि ह॒विषा॑ ह॒विषा॑ दीक्षयामि रु॒द्रेभी॑ रु॒द्रेभि॑र् दीक्षयामि ह॒विषा॑ ह॒विषा॑ दीक्षयामि रु॒द्रेभिः॑ । \newline
20. दी॒क्ष॒या॒मि॒ रु॒द्रेभी॑ रु॒द्रेभि॑र् दीक्षयामि दीक्षयामि रु॒द्रेभि॑र् दे॒वेभि॑र् दे॒वेभी॑ रु॒द्रेभि॑र् दीक्षयामि दीक्षयामि रु॒द्रेभि॑र् दे॒वेभिः॑ । \newline
21. रु॒द्रेभि॑र् दे॒वेभि॑र् दे॒वेभी॑ रु॒द्रेभी॑ रु॒द्रेभि॑र् दे॒वेभि॑र् दे॒वत॑या दे॒वत॑या दे॒वेभी॑ रु॒द्रेभी॑ रु॒द्रेभि॑र् दे॒वेभि॑र् दे॒वत॑या । \newline
22. दे॒वेभि॑र् दे॒वत॑या दे॒वत॑या दे॒वेभि॑र् दे॒वेभि॑र् दे॒वत॑या॒ त्रैष्टु॑भेन॒ त्रैष्टु॑भेन दे॒वत॑या दे॒वेभि॑र् दे॒वेभि॑र् दे॒वत॑या॒ त्रैष्टु॑भेन । \newline
23. दे॒वत॑या॒ त्रैष्टु॑भेन॒ त्रैष्टु॑भेन दे॒वत॑या दे॒वत॑या॒ त्रैष्टु॑भेन त्वा त्वा॒ त्रैष्टु॑भेन दे॒वत॑या दे॒वत॑या॒ त्रैष्टु॑भेन त्वा । \newline
24. त्रैष्टु॑भेन त्वा त्वा॒ त्रैष्टु॑भेन॒ त्रैष्टु॑भेन त्वा॒ छन्द॑सा॒ छन्द॑सा त्वा॒ त्रैष्टु॑भेन॒ त्रैष्टु॑भेन त्वा॒ छन्द॑सा । \newline
25. त्वा॒ छन्द॑सा॒ छन्द॑सा त्वा त्वा॒ छन्द॑सा युनज्मि युनज्मि॒ छन्द॑सा त्वा त्वा॒ छन्द॑सा युनज्मि । \newline
26. छन्द॑सा युनज्मि युनज्मि॒ छन्द॑सा॒ छन्द॑सा युनज्मि ग्री॒ष्मेण॑ ग्री॒ष्मेण॑ युनज्मि॒ छन्द॑सा॒ छन्द॑सा युनज्मि ग्री॒ष्मेण॑ । \newline
27. यु॒न॒ज्मि॒ ग्री॒ष्मेण॑ ग्री॒ष्मेण॑ युनज्मि युनज्मि ग्री॒ष्मेण॑ त्वा त्वा ग्री॒ष्मेण॑ युनज्मि युनज्मि ग्री॒ष्मेण॑ त्वा । \newline
28. ग्री॒ष्मेण॑ त्वा त्वा ग्री॒ष्मेण॑ ग्री॒ष्मेण॑ त्व॒ र्‌तुन॒ र्‌तुना᳚ त्वा ग्री॒ष्मेण॑ ग्री॒ष्मेण॑ त्व॒ र्‌तुना᳚ । \newline
29. त्व॒ र्‌तुन॒ र्‌तुना᳚ त्वा त्व॒ र्‌तुना॑ ह॒विषा॑ ह॒विष॒ र्‌तुना᳚ त्वा त्व॒ र्‌तुना॑ ह॒विषा᳚ । \newline
30. ऋ॒तुना॑ ह॒विषा॑ ह॒विष॒ र्‌तुन॒ र्‌तुना॑ ह॒विषा॑ दीक्षयामि दीक्षयामि ह॒विष॒ र्‌तुन॒ र्‌तुना॑ ह॒विषा॑ दीक्षयामि । \newline
31. ह॒विषा॑ दीक्षयामि दीक्षयामि ह॒विषा॑ ह॒विषा॑ दीक्षया म्यादि॒त्येभि॑ रादि॒त्येभि॑र् दीक्षयामि ह॒विषा॑ ह॒विषा॑ दीक्षया म्यादि॒त्येभिः॑ । \newline
32. दी॒क्ष॒या॒ म्या॒दि॒त्येभि॑ रादि॒त्येभि॑र् दीक्षयामि दीक्षया म्यादि॒त्येभि॑र् दे॒वेभि॑र् दे॒वेभि॑ रादि॒त्येभि॑र् दीक्षयामि दीक्षया म्यादि॒त्येभि॑र् दे॒वेभिः॑ । \newline
33. आ॒दि॒त्येभि॑र् दे॒वेभि॑र् दे॒वेभि॑ रादि॒त्येभि॑ रादि॒त्येभि॑र् दे॒वेभि॑र् दे॒वत॑या दे॒वत॑या दे॒वेभि॑ रादि॒त्येभि॑ रादि॒त्येभि॑र् दे॒वेभि॑र् दे॒वत॑या । \newline
34. दे॒वेभि॑र् दे॒वत॑या दे॒वत॑या दे॒वेभि॑र् दे॒वेभि॑र् दे॒वत॑या॒ जाग॑तेन॒ जाग॑तेन दे॒वत॑या दे॒वेभि॑र् दे॒वेभि॑र् दे॒वत॑या॒ जाग॑तेन । \newline
35. दे॒वत॑या॒ जाग॑तेन॒ जाग॑तेन दे॒वत॑या दे॒वत॑या॒ जाग॑तेन त्वा त्वा॒ जाग॑तेन दे॒वत॑या दे॒वत॑या॒ जाग॑तेन त्वा । \newline
36. जाग॑तेन त्वा त्वा॒ जाग॑तेन॒ जाग॑तेन त्वा॒ छन्द॑सा॒ छन्द॑सा त्वा॒ जाग॑तेन॒ जाग॑तेन त्वा॒ छन्द॑सा । \newline
37. त्वा॒ छन्द॑सा॒ छन्द॑सा त्वा त्वा॒ छन्द॑सा युनज्मि युनज्मि॒ छन्द॑सा त्वा त्वा॒ छन्द॑सा युनज्मि । \newline
38. छन्द॑सा युनज्मि युनज्मि॒ छन्द॑सा॒ छन्द॑सा युनज्मि व॒र्॒.षाभि॑र् व॒र्॒.षाभि॑र् युनज्मि॒ छन्द॑सा॒ छन्द॑सा युनज्मि व॒र्॒.षाभिः॑ । \newline
39. यु॒न॒ज्मि॒ व॒र्॒.षाभि॑र् व॒र्॒.षाभि॑र् युनज्मि युनज्मि व॒र्॒.षाभि॑ स्त्वा त्वा व॒र्॒.षाभि॑र् युनज्मि युनज्मि व॒र्॒.षाभि॑ स्त्वा । \newline
40. व॒र्॒.षाभि॑ स्त्वा त्वा व॒र्॒.षाभि॑र् व॒र्॒.षाभि॑ स्त्व॒ र्‌तुन॒ र्‌तुना᳚ त्वा व॒र्॒.षाभि॑र् व॒र्॒.षाभि॑ स्त्व॒ र्‌तुना᳚ । \newline
41. त्व॒ र्‌तुन॒ र्‌तुना᳚ त्वा त्व॒ र्‌तुना॑ ह॒विषा॑ ह॒विष॒ र्‌तुना᳚ त्वा त्व॒ र्‌तुना॑ ह॒विषा᳚ । \newline
42. ऋ॒तुना॑ ह॒विषा॑ ह॒विष॒ र्‌तुन॒ र्‌तुना॑ ह॒विषा॑ दीक्षयामि दीक्षयामि ह॒विष॒ र्‌तुन॒ र्‌तुना॑ ह॒विषा॑ दीक्षयामि । \newline
43. ह॒विषा॑ दीक्षयामि दीक्षयामि ह॒विषा॑ ह॒विषा॑ दीक्षयामि॒ विश्वे॑भि॒र् विश्वे॑भिर् दीक्षयामि ह॒विषा॑ ह॒विषा॑ दीक्षयामि॒ विश्वे॑भिः । \newline
44. दी॒क्ष॒या॒मि॒ विश्वे॑भि॒र् विश्वे॑भिर् दीक्षयामि दीक्षयामि॒ विश्वे॑भिर् दे॒वेभि॑र् दे॒वेभि॒र् विश्वे॑भिर् दीक्षयामि दीक्षयामि॒ विश्वे॑भिर् दे॒वेभिः॑ । \newline
45. विश्वे॑भिर् दे॒वेभि॑र् दे॒वेभि॒र् विश्वे॑भि॒र् विश्वे॑भिर् दे॒वेभि॑र् दे॒वत॑या दे॒वत॑या दे॒वेभि॒र् विश्वे॑भि॒र् विश्वे॑भिर् दे॒वेभि॑र् दे॒वत॑या । \newline
46. दे॒वेभि॑र् दे॒वत॑या दे॒वत॑या दे॒वेभि॑र् दे॒वेभि॑र् दे॒वत॒या ऽऽनु॑ष्टुभे॒ना नु॑ष्टुभेन दे॒वत॑या दे॒वेभि॑र् दे॒वेभि॑र् दे॒वत॒या ऽऽनु॑ष्टुभेन । \newline
47. दे॒वत॒या ऽऽनु॑ष्टुभे॒ना नु॑ष्टुभेन दे॒वत॑या दे॒वत॒या ऽऽनु॑ष्टुभेन त्वा॒ त्वा ऽऽनु॑ष्टुभेन दे॒वत॑या दे॒वत॒या ऽऽनु॑ष्टुभेन त्वा । \newline
48. आनु॑ष्टुभेन त्वा॒ त्वा ऽऽनु॑ष्टुभे॒ना नु॑ष्टुभेन त्वा॒ छन्द॑सा॒ छन्द॑सा॒ त्वा ऽऽनु॑ष्टुभे॒ना नु॑ष्टुभेन त्वा॒ छन्द॑सा । \newline
49. आनु॑ष्टुभे॒नेत्यानु॑ - स्तु॒भे॒न॒ । \newline
50. त्वा॒ छन्द॑सा॒ छन्द॑सा त्वा त्वा॒ छन्द॑सा युनज्मि युनज्मि॒ छन्द॑सा त्वा त्वा॒ छन्द॑सा युनज्मि । \newline
51. छन्द॑सा युनज्मि युनज्मि॒ छन्द॑सा॒ छन्द॑सा युनज्मि श॒रदा॑ श॒रदा॑ युनज्मि॒ छन्द॑सा॒ छन्द॑सा युनज्मि श॒रदा᳚ । \newline
52. यु॒न॒ज्मि॒ श॒रदा॑ श॒रदा॑ युनज्मि युनज्मि श॒रदा᳚ त्वा त्वा श॒रदा॑ युनज्मि युनज्मि श॒रदा᳚ त्वा । \newline
\pagebreak
\markright{ TS 7.1.18.2  \hfill https://www.vedavms.in \hfill}

\section{ TS 7.1.18.2 }

\textbf{TS 7.1.18.2 } \newline
\textbf{Samhita Paata} \newline

श॒रदा᳚ त्व॒र्तुना॑ ह॒विषा॑ दीक्षया॒म्यङ्गि॑रोभि-र्दे॒वेभि॑-र्दे॒वत॑या॒ पाङ्क्ते॑न त्वा॒ छन्द॑सा युनज्मि हेमन्तशिशि॒राभ्यां᳚ त्व॒र्तुना॑ ह॒विषा॑ दीक्षया॒म्याऽहं दी॒क्षाम॑रुहमृ॒तस्य॒ पत्नीं᳚ गाय॒त्रेण॒ छन्द॑सा॒ ब्रह्म॑णा च॒र्तꣳ स॒त्ये॑ऽधाꣳ स॒त्यमृ॒ते॑ऽधां ॥म॒ही मू ॒षु >1सु॒त्रामा॑ण >2-मि॒ह धृतिः॒ स्वाहे॒ह विधृ॑तिः॒ स्वाहे॒ह रन्तिः॒ स्वाहे॒ह रम॑तिः॒ स्वाहा᳚ ॥ \newline

\textbf{Pada Paata} \newline

श॒रदा᳚ । त्वा॒ । ऋ॒तुना᳚ । ह॒विषा᳚ । दी॒क्ष॒या॒मि॒ । अङ्गि॑रोभि॒रित्यङ्गि॑रः- भिः॒ । दे॒वेभिः॑ । दे॒वत॑या । पाङ्क्ते॑न । त्वा॒ । छन्द॑सा । यु॒न॒ज्मि॒ । हे॒म॒न्त॒शि॒शि॒राभ्या॒मिति॑ हेमन्त - शि॒शि॒राभ्या᳚म् । त्वा॒ । ऋ॒तुना᳚ । ह॒विषा᳚ । दी॒क्ष॒या॒मि॒ । एति॑ । अ॒हम् । दी॒क्षाम् । अ॒रु॒ह॒म् । ऋ॒तस्य॑ । पत्नी᳚म् । गा॒य॒त्रेण॑ । छन्द॑सा । ब्रह्म॑णा । च॒ । ऋ॒तम् । स॒त्ये । अ॒धा॒म् । स॒त्यम् । ऋ॒ते । अ॒धा॒म् ॥ म॒हीम् । उ॒ । स्विति॑ । सु॒त्रामा॑ण॒मिति॑ सु - त्रामा॑णम् । इ॒ह । धृतिः॑ । स्वाहा᳚ । इ॒ह । विधृ॑ति॒रिति॒ वि - धृ॒तिः॒ । स्वाहा᳚ । इ॒ह । रन्तिः॑ । स्वाहा᳚ । इ॒ह । रम॑तिः । स्वाहा᳚ ॥  \newline


\textbf{Krama Paata} \newline

श॒रदा᳚ त्वा । त्व॒र्तुना᳚ । ऋ॒तुना॑ ह॒विषा᳚ । ह॒विषा॑ दीक्षयामि । दी॒क्ष॒या॒म्यङ्‍गि॑रोभिः । अङ्‍गि॑रोभिर् दे॒वेभिः॑ । अङ्‍गि॑रोभि॒रित्यङ्‍गि॑रः - भिः॒ । दे॒वेभि॑र् दे॒वत॑या । दे॒वत॑या॒ पाङ्‍क्ते॑न । पाङ्‍क्ते॑न त्वा । त्वा॒ छन्द॑सा । छन्द॑सा युनज्मि । यु॒न॒ज्मि॒ हे॒म॒न्त॒शि॒शि॒राभ्या᳚म् । हे॒म॒न्त॒शि॒शि॒राभ्या᳚म् त्वा । हे॒म॒न्त॒शि॒शि॒राभ्या॒मिति॑ हेमन्त - शि॒शि॒राभ्या᳚म् । त्व॒र्तुना᳚ । ऋ॒तुना॑ ह॒विषा᳚ । ह॒विषा॑ दीक्षयामि । दी॒क्ष॒या॒म्या । आ ऽहम् । अ॒हम् दी॒क्षाम् । दी॒क्षाम॑रुहम् । अ॒रु॒ह॒मृ॒तस्य॑ । ऋ॒तस्य॒ पत्नी᳚म् । पत्नी᳚म् गाय॒त्रेण॑ । गा॒य॒त्रेण॒ छन्द॑सा । छन्द॑सा॒ ब्रह्म॑णा । ब्रह्म॑णा च । च॒र्तम् । ऋ॒तꣳ स॒त्ये । स॒त्ये॑ऽधाम् । अ॒धाꣳ॒॒ स॒त्यम् । स॒त्यमृ॒ते । ऋ॒ते॑ऽधाम् । अ॒धा॒मित्य॑धाम् ॥ म॒हीमु॑ । ऊ॒ षु । सु सु॒त्रामा॑णम् । सु॒त्रामा॑णमि॒ह । सु॒त्रामा॑ण॒मिति॑ सु - त्रामा॑णम् । इ॒ह धृतिः॑ । धृतिः॒ स्वाहा᳚ । स्वाहे॒ह । इ॒ह विधृ॑तिः । विधृ॑तिः॒ स्वाहा᳚ । विधृ॑ति॒रिति॒ वि - धृ॒तिः॒ । स्वाहे॒ह । इ॒ह रन्तिः॑ । रन्तिः॒ स्वाहा᳚ । स्वाहे॒ह । इ॒ह रम॑तिः । रम॑तिः॒ स्वाहा᳚ । स्वाहेति॒ स्वाहा᳚ । \newline

\textbf{Jatai Paata} \newline

1. श॒रदा᳚ त्वा त्वा श॒रदा॑ श॒रदा᳚ त्वा । \newline
2. त्व॒ र्‌तुन॒ र्‌तुना᳚ त्वा त्व॒ र्‌तुना᳚ । \newline
3. ऋ॒तुना॑ ह॒विषा॑ ह॒विष॒ र्‌तुन॒ र्‌तुना॑ ह॒विषा᳚ । \newline
4. ह॒विषा॑ दीक्षयामि दीक्षयामि ह॒विषा॑ ह॒विषा॑ दीक्षयामि । \newline
5. दी॒क्ष॒या॒ म्यङ्गि॑रोभि॒ रङ्गि॑रोभिर् दीक्षयामि दीक्षया॒ म्यङ्गि॑रोभिः । \newline
6. अङ्गि॑रोभिर् दे॒वेभि॑र् दे॒वेभि॒ रङ्गि॑रोभि॒ रङ्गि॑रोभिर् दे॒वेभिः॑ । \newline
7. अङ्गि॑रोभि॒रित्यङ्गि॑रः - भिः॒ । \newline
8. दे॒वेभि॑र् दे॒वत॑या दे॒वत॑या दे॒वेभि॑र् दे॒वेभि॑र् दे॒वत॑या । \newline
9. दे॒वत॑या॒ पाङ्क्ते॑न॒ पाङ्क्ते॑न दे॒वत॑या दे॒वत॑या॒ पाङ्क्ते॑न । \newline
10. पाङ्क्ते॑न त्वा त्वा॒ पाङ्क्ते॑न॒ पाङ्क्ते॑न त्वा । \newline
11. त्वा॒ छन्द॑सा॒ छन्द॑सा त्वा त्वा॒ छन्द॑सा । \newline
12. छन्द॑सा युनज्मि युनज्मि॒ छन्द॑सा॒ छन्द॑सा युनज्मि । \newline
13. यु॒न॒ज्मि॒ हे॒म॒न्त॒शि॒शि॒राभ्याꣳ॑ हेमन्तशिशि॒राभ्यां᳚ ॅयुनज्मि युनज्मि हेमन्तशिशि॒राभ्या᳚म् । \newline
14. हे॒म॒न्त॒शि॒शि॒राभ्या᳚म् त्वा त्वा हेमन्तशिशि॒राभ्याꣳ॑ हेमन्तशिशि॒राभ्या᳚म् त्वा । \newline
15. हे॒म॒न्त॒शि॒शि॒राभ्या॒मिति॑ हेमन्त - शि॒शि॒राभ्या᳚म् । \newline
16. त्व॒ र्‌तुन॒ र्‌तुना᳚ त्वा त्व॒ र्‌तुना᳚ । \newline
17. ऋ॒तुना॑ ह॒विषा॑ ह॒विष॒ र्‌तुन॒ र्‌तुना॑ ह॒विषा᳚ । \newline
18. ह॒विषा॑ दीक्षयामि दीक्षयामि ह॒विषा॑ ह॒विषा॑ दीक्षयामि । \newline
19. दी॒क्ष॒या॒म्या दी᳚क्षयामि दीक्षया॒म्या । \newline
20. आ ऽह म॒ह मा ऽहम् । \newline
21. अ॒हम् दी॒क्षाम् दी॒क्षा म॒ह म॒हम् दी॒क्षाम् । \newline
22. दी॒क्षा म॑रुह मरुहम् दी॒क्षाम् दी॒क्षा म॑रुहम् । \newline
23. अ॒रु॒ह॒ मृ॒तस्य॒ र्‌तस्या॑ रुह मरुह मृ॒तस्य॑ । \newline
24. ऋ॒तस्य॒ पत्नी॒म् पत्नी॑ मृ॒तस्य॒ र्‌तस्य॒ पत्नी᳚म् । \newline
25. पत्नी᳚म् गाय॒त्रेण॑ गाय॒त्रेण॒ पत्नी॒म् पत्नी᳚म् गाय॒त्रेण॑ । \newline
26. गा॒य॒त्रेण॒ छन्द॑सा॒ छन्द॑सा गाय॒त्रेण॑ गाय॒त्रेण॒ छन्द॑सा । \newline
27. छन्द॑सा॒ ब्रह्म॑णा॒ ब्रह्म॑णा॒ छन्द॑सा॒ छन्द॑सा॒ ब्रह्म॑णा । \newline
28. ब्रह्म॑णा च च॒ ब्रह्म॑णा॒ ब्रह्म॑णा च । \newline
29. च॒ र्‌त मृ॒तम् च॑ च॒ र्‌तम् । \newline
30. ऋ॒तꣳ स॒त्ये स॒त्य ऋ॒त मृ॒तꣳ स॒त्ये । \newline
31. स॒त्ये॑ ऽधा मधाꣳ स॒त्ये स॒त्ये॑ ऽधाम् । \newline
32. अ॒धाꣳ॒॒ स॒त्यꣳ स॒त्य म॑धा मधाꣳ स॒त्यम् । \newline
33. स॒त्य मृ॒त ऋ॒ते स॒त्यꣳ स॒त्य मृ॒ते । \newline
34. ऋ॒ते॑ ऽधा मधा मृ॒त ऋ॒ते॑ ऽधाम् । \newline
35. अ॒धा॒मित्य॑धाम् । \newline
36. म॒ही मु॑ वु म॒हीम् म॒ही मु॑ । \newline
37. ऊ॒ षु सू॑ षु । \newline
38. सु सु॒त्रामा॑णꣳ सु॒त्रामा॑णꣳ॒॒ सु सु सु॒त्रामा॑णम् । \newline
39. सु॒त्रामा॑ण मि॒हेह सु॒त्रामा॑णꣳ सु॒त्रामा॑ण मि॒ह । \newline
40. सु॒त्रामा॑ण॒मिति॑ सु - त्रामा॑णम् । \newline
41. इ॒ह धृति॒र् धृति॑ रि॒हेह धृतिः॑ । \newline
42. धृतिः॒ स्वाहा॒ स्वाहा॒ धृति॒र् धृतिः॒ स्वाहा᳚ । \newline
43. स्वाहे॒ हेह स्वाहा॒ स्वाहे॒ह । \newline
44. इ॒ह विधृ॑ति॒र् विधृ॑ति रि॒हेह विधृ॑तिः । \newline
45. विधृ॑तिः॒ स्वाहा॒ स्वाहा॒ विधृ॑ति॒र् विधृ॑तिः॒ स्वाहा᳚ । \newline
46. विधृ॑ति॒रिति॒ वि - धृ॒तिः॒ । \newline
47. स्वाहे॒ हेह स्वाहा॒ स्वाहे॒ह । \newline
48. इ॒ह रन्ती॒ रन्ति॑ रि॒हेह रन्तिः॑ । \newline
49. रन्तिः॒ स्वाहा॒ स्वाहा॒ रन्ती॒ रन्तिः॒ स्वाहा᳚ । \newline
50. स्वाहे॒ हेह स्वाहा॒ स्वाहे॒ह । \newline
51. इ॒ह रम॑ती॒ रम॑ति रि॒हेह रम॑तिः । \newline
52. रम॑तिः॒ स्वाहा॒ स्वाहा॒ रम॑ती॒ रम॑तिः॒ स्वाहा᳚ । \newline
53. स्वाहेति॒ स्वाहा᳚ । \newline

\textbf{Ghana Paata } \newline

1. श॒रदा᳚ त्वा त्वा श॒रदा॑ श॒रदा᳚ त्व॒ र्‌तुन॒ र्‌तुना᳚ त्वा श॒रदा॑ श॒रदा᳚ त्व॒ र्‌तुना᳚ । \newline
2. त्व॒ र्‌तुन॒ र्‌तुना᳚ त्वा त्व॒ र्‌तुना॑ ह॒विषा॑ ह॒विष॒ र्‌तुना᳚ त्वा त्व॒ र्‌तुना॑ ह॒विषा᳚ । \newline
3. ऋ॒तुना॑ ह॒विषा॑ ह॒विष॒ र्‌तुन॒ र्‌तुना॑ ह॒विषा॑ दीक्षयामि दीक्षयामि ह॒विष॒ र्‌तुन॒ र्‌तुना॑ ह॒विषा॑ दीक्षयामि । \newline
4. ह॒विषा॑ दीक्षयामि दीक्षयामि ह॒विषा॑ ह॒विषा॑ दीक्षया॒ म्यङ्गि॑रोभि॒ रङ्गि॑रोभिर् दीक्षयामि ह॒विषा॑ ह॒विषा॑ दीक्षया॒ म्यङ्गि॑रोभिः । \newline
5. दी॒क्ष॒या॒ म्यङ्गि॑रोभि॒ रङ्गि॑रोभिर् दीक्षयामि दीक्षया॒ म्यङ्गि॑रोभिर् दे॒वेभि॑र् दे॒वेभि॒ रङ्गि॑रोभिर् दीक्षयामि दीक्षया॒ म्यङ्गि॑रोभिर् दे॒वेभिः॑ । \newline
6. अङ्गि॑रोभिर् दे॒वेभि॑र् दे॒वेभि॒ रङ्गि॑रोभि॒ रङ्गि॑रोभिर् दे॒वेभि॑र् दे॒वत॑या दे॒वत॑या दे॒वेभि॒ रङ्गि॑रोभि॒ रङ्गि॑रोभिर् दे॒वेभि॑र् दे॒वत॑या । \newline
7. अङ्गि॑रोभि॒रित्यङ्गि॑रः - भिः॒ । \newline
8. दे॒वेभि॑र् दे॒वत॑या दे॒वत॑या दे॒वेभि॑र् दे॒वेभि॑र् दे॒वत॑या॒ पाङ्क्ते॑न॒ पाङ्क्ते॑न दे॒वत॑या दे॒वेभि॑र् दे॒वेभि॑र् दे॒वत॑या॒ पाङ्क्ते॑न । \newline
9. दे॒वत॑या॒ पाङ्क्ते॑न॒ पाङ्क्ते॑न दे॒वत॑या दे॒वत॑या॒ पाङ्क्ते॑न त्वा त्वा॒ पाङ्क्ते॑न दे॒वत॑या दे॒वत॑या॒ पाङ्क्ते॑न त्वा । \newline
10. पाङ्क्ते॑न त्वा त्वा॒ पाङ्क्ते॑न॒ पाङ्क्ते॑न त्वा॒ छन्द॑सा॒ छन्द॑सा त्वा॒ पाङ्क्ते॑न॒ पाङ्क्ते॑न त्वा॒ छन्द॑सा । \newline
11. त्वा॒ छन्द॑सा॒ छन्द॑सा त्वा त्वा॒ छन्द॑सा युनज्मि युनज्मि॒ छन्द॑सा त्वा त्वा॒ छन्द॑सा युनज्मि । \newline
12. छन्द॑सा युनज्मि युनज्मि॒ छन्द॑सा॒ छन्द॑सा युनज्मि हेमन्तशिशि॒राभ्याꣳ॑ हेमन्तशिशि॒राभ्यां᳚ ॅयुनज्मि॒ छन्द॑सा॒ छन्द॑सा युनज्मि हेमन्तशिशि॒राभ्या᳚म् । \newline
13. यु॒न॒ज्मि॒ हे॒म॒न्त॒शि॒शि॒राभ्याꣳ॑ हेमन्तशिशि॒राभ्यां᳚ ॅयुनज्मि युनज्मि हेमन्तशिशि॒राभ्या᳚म् त्वा त्वा हेमन्तशिशि॒राभ्यां᳚ ॅयुनज्मि युनज्मि हेमन्तशिशि॒राभ्या᳚म् त्वा । \newline
14. हे॒म॒न्त॒शि॒शि॒राभ्या᳚म् त्वा त्वा हेमन्तशिशि॒राभ्याꣳ॑ हेमन्तशिशि॒राभ्या᳚म् त्व॒ र्‌तुन॒ र्‌तुना᳚ त्वा हेमन्तशिशि॒राभ्याꣳ॑ हेमन्तशिशि॒राभ्या᳚म् त्व॒ र्‌तुना᳚ । \newline
15. हे॒म॒न्त॒शि॒शि॒राभ्या॒मिति॑ हेमन्त - शि॒शि॒राभ्या᳚म् । \newline
16. त्व॒ र्‌तुन॒ र्‌तुना᳚ त्वा त्व॒ र्‌तुना॑ ह॒विषा॑ ह॒विष॒ र्‌तुना᳚ त्वा त्व॒ र्‌तुना॑ ह॒विषा᳚ । \newline
17. ऋ॒तुना॑ ह॒विषा॑ ह॒विष॒ र्‌तुन॒ र्‌तुना॑ ह॒विषा॑ दीक्षयामि दीक्षयामि ह॒विष॒ र्‌तुन॒ र्‌तुना॑ ह॒विषा॑ दीक्षयामि । \newline
18. ह॒विषा॑ दीक्षयामि दीक्षयामि ह॒विषा॑ ह॒विषा॑ दीक्षया॒म्या दी᳚क्षयामि ह॒विषा॑ ह॒विषा॑ दीक्षया॒म्या । \newline
19. दी॒क्ष॒या॒म्या दी᳚क्षयामि दीक्षया॒म्या ऽह म॒ह मा दी᳚क्षयामि दीक्षया॒म्या ऽहम् । \newline
20. आ ऽह म॒ह मा ऽहम् दी॒क्षाम् दी॒क्षा म॒ह मा ऽहम् दी॒क्षाम् । \newline
21. अ॒हम् दी॒क्षाम् दी॒क्षा म॒ह म॒हम् दी॒क्षा म॑रुह मरुहम् दी॒क्षा म॒ह म॒हम् दी॒क्षा म॑रुहम् । \newline
22. दी॒क्षा म॑रुह मरुहम् दी॒क्षाम् दी॒क्षा म॑रुह मृ॒तस्य॒ र्‌तस्या॑ रुहम् दी॒क्षाम् दी॒क्षा म॑रुह मृ॒तस्य॑ । \newline
23. अ॒रु॒ह॒ मृ॒तस्य॒ र्‌तस्या॑रुह मरुह मृ॒तस्य॒ पत्नी॒म् पत्नी॑ मृ॒तस्या॑रुह मरुह मृ॒तस्य॒ पत्नी᳚म् । \newline
24. ऋ॒तस्य॒ पत्नी॒म् पत्नी॑ मृ॒तस्य॒ र्‌तस्य॒ पत्नी᳚म् गाय॒त्रेण॑ गाय॒त्रेण॒ पत्नी॑ मृ॒तस्य॒ र्‌तस्य॒ पत्नी᳚म् गाय॒त्रेण॑ । \newline
25. पत्नी᳚म् गाय॒त्रेण॑ गाय॒त्रेण॒ पत्नी॒म् पत्नी᳚म् गाय॒त्रेण॒ छन्द॑सा॒ छन्द॑सा गाय॒त्रेण॒ पत्नी॒म् पत्नी᳚म् गाय॒त्रेण॒ छन्द॑सा । \newline
26. गा॒य॒त्रेण॒ छन्द॑सा॒ छन्द॑सा गाय॒त्रेण॑ गाय॒त्रेण॒ छन्द॑सा॒ ब्रह्म॑णा॒ ब्रह्म॑णा॒ छन्द॑सा गाय॒त्रेण॑ गाय॒त्रेण॒ छन्द॑सा॒ ब्रह्म॑णा । \newline
27. छन्द॑सा॒ ब्रह्म॑णा॒ ब्रह्म॑णा॒ छन्द॑सा॒ छन्द॑सा॒ ब्रह्म॑णा च च॒ ब्रह्म॑णा॒ छन्द॑सा॒ छन्द॑सा॒ ब्रह्म॑णा च । \newline
28. ब्रह्म॑णा च च॒ ब्रह्म॑णा॒ ब्रह्म॑णा च॒ र्‌त मृ॒तम् च॒ ब्रह्म॑णा॒ ब्रह्म॑णा च॒ र्‌तम् । \newline
29. च॒ र्‌त मृ॒तम् च॑ च॒ र्‌तꣳ स॒त्ये स॒त्य ऋ॒तम् च॑ च॒ र्‌तꣳ स॒त्ये । \newline
30. ऋ॒तꣳ स॒त्ये स॒त्य ऋ॒त मृ॒तꣳ स॒त्ये॑ ऽधा मधाꣳ स॒त्य ऋ॒त मृ॒तꣳ स॒त्ये॑ ऽधाम् । \newline
31. स॒त्ये॑ ऽधा मधाꣳ स॒त्ये स॒त्ये॑ ऽधाꣳ स॒त्यꣳ स॒त्य म॑धाꣳ स॒त्ये स॒त्ये॑ ऽधाꣳ स॒त्यम् । \newline
32. अ॒धाꣳ॒॒ स॒त्यꣳ स॒त्य म॑धा मधाꣳ स॒त्य मृ॒त ऋ॒ते स॒त्य म॑धा मधाꣳ स॒त्य मृ॒ते । \newline
33. स॒त्य मृ॒त ऋ॒ते स॒त्यꣳ स॒त्य मृ॒ते॑ ऽधा मधा मृ॒ते स॒त्यꣳ स॒त्य मृ॒ते॑ ऽधाम् । \newline
34. ऋ॒ते॑ ऽधा मधा मृ॒त ऋ॒ते॑ ऽधाम् । \newline
35. अ॒धा॒मित्य॑धाम् । \newline
36. म॒ही मु॑ वु म॒हीम् म॒ही मू॒ षु सू॑ म॒हीम् म॒ही मू॒ षु । \newline
37. ऊ॒ षु सू॑ षु सु॒त्रामा॑णꣳ सु॒त्रामा॑णꣳ॒॒ सू॑ षु सु॒त्रामा॑णम् । \newline
38. सु सु॒त्रामा॑णꣳ सु॒त्रामा॑णꣳ॒॒ सु सु सु॒त्रामा॑ण मि॒हेह सु॒त्रामा॑णꣳ॒॒ सु सु सु॒त्रामा॑ण मि॒ह । \newline
39. सु॒त्रामा॑ण मि॒हेह सु॒त्रामा॑णꣳ सु॒त्रामा॑ण मि॒ह धृति॒र् धृति॑ रि॒ह सु॒त्रामा॑णꣳ सु॒त्रामा॑ण मि॒ह धृतिः॑ । \newline
40. सु॒त्रामा॑ण॒मिति॑ सु - त्रामा॑णम् । \newline
41. इ॒ह धृति॒र् धृति॑ रि॒हेह धृतिः॒ स्वाहा॒ स्वाहा॒ धृति॑ रि॒हेह धृतिः॒ स्वाहा᳚ । \newline
42. धृतिः॒ स्वाहा॒ स्वाहा॒ धृति॒र् धृतिः॒ स्वा हे॒हेह स्वाहा॒ धृति॒र् धृतिः॒ स्वाहे॒ह । \newline
43. स्वा हे॒हेह स्वाहा॒ स्वाहे॒ह विधृ॑ति॒र् विधृ॑ति रि॒ह स्वाहा॒ स्वाहे॒ह विधृ॑तिः । \newline
44. इ॒ह विधृ॑ति॒र् विधृ॑ति रि॒हेह विधृ॑तिः॒ स्वाहा॒ स्वाहा॒ विधृ॑ति रि॒हेह विधृ॑तिः॒ स्वाहा᳚ । \newline
45. विधृ॑तिः॒ स्वाहा॒ स्वाहा॒ विधृ॑ति॒र् विधृ॑तिः॒ स्वा हे॒हेह स्वाहा॒ विधृ॑ति॒र् विधृ॑तिः॒ स्वाहे॒ह । \newline
46. विधृ॑ति॒रिति॒ वि - धृ॒तिः॒ । \newline
47. स्वा हे॒हेह स्वाहा॒ स्वाहे॒ह रन्ती॒ रन्ति॑रि॒ह स्वाहा॒ स्वाहे॒ह रन्तिः॑ । \newline
48. इ॒ह रन्ती॒ रन्ति॑ रि॒हेह रन्तिः॒ स्वाहा॒ स्वाहा॒ रन्ति॑ रि॒हेह रन्तिः॒ स्वाहा᳚ । \newline
49. रन्तिः॒ स्वाहा॒ स्वाहा॒ रन्ती॒ रन्तिः॒ स्वाहे॒ हेह स्वाहा॒ रन्ती॒ रन्तिः॒ स्वाहे॒ह । \newline
50. स्वाहे॒ हेह स्वाहा॒ स्वाहे॒ह रम॑ती॒ रम॑ तिरि॒ह स्वाहा॒ स्वाहे॒ह रम॑तिः । \newline
51. इ॒ह रम॑ती॒ रम॑ति रि॒हेह रम॑तिः॒ स्वाहा॒ स्वाहा॒ रम॑ति रि॒हेह रम॑तिः॒ स्वाहा᳚ । \newline
52. रम॑तिः॒ स्वाहा॒ स्वाहा॒ रम॑ती॒ रम॑तिः॒ स्वाहा᳚ । \newline
53. स्वाहेति॒ स्वाहा᳚ । \newline
\pagebreak
\markright{ TS 7.1.19.1  \hfill https://www.vedavms.in \hfill}

\section{ TS 7.1.19.1 }

\textbf{TS 7.1.19.1 } \newline
\textbf{Samhita Paata} \newline

ई॒कां॒राय॒ स्वाहें कृ॑ताय॒ स्वाहा॒ क्रन्द॑ते॒ स्वाहा॑ ऽव॒क्रन्द॑ते॒ स्वाहा॒ प्रोथ॑ते॒ स्वाहा᳚ प्र॒प्रोथ॑ते॒ स्वाहा॑ ग॒न्धाय॒ स्वाहा᳚ घ्रा॒ताय॒ स्वाहा᳚ प्रा॒णाय॒ स्वाहा᳚ व्या॒नाय॒ स्वाहा॑ ऽपा॒नाय॒ स्वाहा॑ सन्दी॒यमा॑नाय॒ स्वाहा॒ सन्दि॑ताय॒ स्वाहा॑ विचृ॒त्यमा॑नाय॒ स्वाहा॒ विचृ॑त्ताय॒ स्वाहा॑ पलायि॒ष्यमा॑णाय॒ स्वाहा॒ पला॑यिताय॒ स्वाहो॑परꣳस्य॒ते स्वाहोप॑रताय॒ स्वाहा॑ निवेक्ष्य॒ते स्वाहा॑ निवि॒शमा॑नाय॒ स्वाहा॒ निवि॑ष्टाय॒ स्वाहा॑ निषथ्स्य॒ते स्वाहा॑ नि॒षीद॑ते॒ स्वाहा॒ निष॑ण्णाय॒ स्वाहा॑ - [  ] \newline

\textbf{Pada Paata} \newline

ई॒कां॒रायेती᳚म् - का॒राय॑ । स्वाहा᳚ । ईकृं॑ता॒येती᳚म् - कृ॒ता॒य॒ । स्वाहा᳚ । क्रन्द॑ते । स्वाहा᳚ । अ॒व॒क्रन्द॑त॒ इत्य॑व - क्रन्द॑ते । स्वाहा᳚ । प्रोथ॑ते । स्वाहा᳚ । प्र॒प्रोथ॑त॒ इति॑ प्र - प्रोथ॑ते । स्वाहा᳚ । ग॒न्धाय॑ । स्वाहा᳚ । घ्रा॒ताय॑ । स्वाहा᳚ । प्रा॒णायेति॑ प्र - अ॒नाय॑ । स्वाहा᳚ । व्या॒नायेति॑ वि - अ॒नाय॑ । स्वाहा᳚ । अ॒पा॒नायेत्य॑प - अ॒नाय॑ । स्वाहा᳚ । स॒दीं॒यमा॑ना॒येति॑ सं-दी॒यमा॑नाय । स्वाहा᳚ । संदि॑ता॒येति॒ सं-दि॒ता॒य॒ । स्वाहा᳚ । वि॒चृ॒त्यमा॑ना॒येति॑ वि - चृ॒त्यमा॑नाय । स्वाहा᳚ । विचृ॑त्ता॒येति॒ वि - चृ॒त्ता॒य॒ । स्वाहा᳚ । प॒ला॒यि॒ष्यमा॑णाय । स्वाहा᳚ । पला॑यिताय । स्वाहा᳚ । उ॒प॒रꣳ॒॒स्य॒त इत्यु॑प - रꣳ॒॒स्य॒ते । स्वाहा᳚ । उप॑रता॒येत्युप॑ - र॒ता॒य॒ । स्वाहा᳚ । नि॒वे॒क्ष्य॒त इति॑ नि - वे॒क्ष्य॒ते । स्वाहा᳚ । नि॒वि॒शमा॑ना॒येति॑ नि - वि॒शमा॑नाय । स्वाहा᳚ । निवि॑ष्टा॒येति॒ नि - वि॒ष्टा॒य॒ । स्वाहा᳚ । नि॒ष॒थ्स्य॒त इति॑ नि - स॒थ्स्य॒ते । स्वाहा᳚ । नि॒षीद॑त॒ इति॑ नि - सीद॑ते । स्वाहा᳚ । निष॑ण्णा॒येति॒ नि - स॒न्ना॒य॒ । स्वाहा᳚ ।  \newline


\textbf{Krama Paata} \newline

ई॒ङ्‍का॒राय॒ स्वाहा᳚ । ई॒ङ्‍का॒रायेती᳚म् - का॒राय॑ । स्वाहेङ्‍कृ॑ताय । ईङ्‍कृ॑ताय॒ स्वाहा᳚ । ईङ्‍कृ॑ता॒येतीम् - कृ॒ता॒य॒ । स्वाहा॒ क्रन्द॑ते । क्रन्द॑ते॒ स्वाहा᳚ । स्वाहा॑ऽव॒क्रन्द॑ते । अ॒व॒क्रन्द॑ते॒ स्वाहा᳚ । अ॒व॒क्रन्द॑त॒ इत्य॑व - क्रन्द॑ते । स्वाहा॒ प्रोथ॑ते । प्रोथ॑ते॒ स्वाहा᳚ । स्वाहा᳚ प्र॒प्रोथ॑ते । प्र॒प्रोथ॑ते॒ स्वाहा᳚ । प्र॒प्रोथ॑त॒ इति॑ प्र - प्रोथ॑ते । स्वाहा॑ ग॒न्धाय॑ । ग॒न्धाय॒ स्वाहा᳚ । स्वाहा᳚ घ्रा॒ताय॑ । घ्रा॒ताय॒ स्वाहा᳚ । स्वाहा᳚ प्रा॒णाय॑ । प्रा॒णाय॒ स्वाहा᳚ । प्रा॒णायेति॑ प्र - अ॒नाय॑ । स्वाहा᳚ व्या॒नाय॑ । व्या॒नाय॒ स्वाहा᳚ । व्या॒नायेति॑ वि - अ॒नाय॑ । स्वाहा॑ऽपा॒नाय॑ । अ॒पा॒नाय॒ स्वाहा᳚ । अ॒पा॒नायेत्य॑प - अ॒नाय॑ । स्वाहा॑ सन्दी॒यमा॑नाय । स॒न्दी॒यमा॑नाय॒ स्वाहा᳚ । स॒न्दी॒यमा॑ना॒येति॑ सम् - दी॒यमा॑नाय । स्वाहा॒ सन्दि॑ताय । सन्दि॑ताय॒ स्वाहा᳚ । सन्दि॑ता॒येति॒ सम् - दि॒ता॒य॒ । स्वाहा॑ विचृ॒त्यमा॑नाय । वि॒चृ॒त्यमा॑नाय॒ स्वाहा᳚ । वि॒चृ॒त्यमा॑ना॒येति॑ वि - चृ॒त्यमा॑नाय । स्वाहा॒ विचृ॑त्ताय । विचृ॑त्ताय॒ स्वाहा᳚ । विचृ॑त्ता॒येति॒ वि - चृ॒त्ता॒य॒ । स्वाहा॑ पलायि॒ष्यमा॑णाय । प॒ला॒यि॒ष्यमा॑णाय॒ स्वाहा᳚ । स्वाहा॒ पला॑यिताय । पला॑यिताय॒ स्वाहा᳚ । स्वाहो॑परꣳस्य॒ते । उ॒प॒रꣳ॒॒स्य॒ते स्वाहा᳚ । उ॒प॒रꣳ॒॒स्य॒त इत्यु॑प - रꣳ॒॒स्य॒ते । स्वाहोप॑रताय । उप॑रताय॒ स्वाहा᳚ । उप॑रता॒येत्युप॑ - र॒ता॒य॒ । स्वाहा॑ निवेक्ष्य॒ते । नि॒वे॒क्ष्य॒ते स्वाहा᳚ । नि॒वे॒क्ष्य॒त इति॑ नि - वे॒क्ष्य॒ते । स्वाहा॑ निवि॒शमा॑नाय । नि॒वि॒शमा॑नाय॒ स्वाहा᳚ । नि॒वि॒शमा॑ना॒येति॑ नि - वि॒शमा॑नाय । स्वाहा॒ निवि॑ष्टाय । निवि॑ष्टाय॒ स्वाहा᳚ । निवि॑ष्टा॒येति॒ नि - वि॒ष्टा॒य॒ । स्वाहा॑ निषथ्स्य॒ते । नि॒ष॒थ्स्य॒ते स्वाहा᳚ । नि॒ष॒थ्स्य॒त इति॑ नि - स॒थ्स्य॒ते । स्वाहा॑ नि॒षीद॑ते । नि॒षीद॑ते॒ स्वाहा᳚ । नि॒षीद॑त॒ इति॑ नि - सीद॑ते । स्वाहा॒ निष॑ण्णाय । निष॑ण्णाय॒ स्वाहा᳚ । निष॑ण्णा॒येति॒ नि - स॒न्ना॒य॒ । स्वाहा॑ऽऽसिष्य॒ते \newline

\textbf{Jatai Paata} \newline

1. ई॒ङ्का॒राय॒ स्वाहा॒ स्वाहे᳚ङ् का॒रा ये᳚ङ्का॒राय॒ स्वाहा᳚ । \newline
2. ई॒ङ्का॒रायेती᳚म् - का॒राय॑ । \newline
3. स्वाहेङ्कृ॑ता॒ येङ्कृ॑ताय॒ स्वाहा॒ स्वाहेङ्कृ॑ताय । \newline
4. ईङ्कृ॑ताय॒ स्वाहा॒ स्वाहेङ्कृ॑ता॒ येङ्कृ॑ताय॒ स्वाहा᳚ । \newline
5. ईङ्कृ॑ता॒येतीम् - कृ॒ता॒य॒ । \newline
6. स्वाहा॒ क्रन्द॑ते॒ क्रन्द॑ते॒ स्वाहा॒ स्वाहा॒ क्रन्द॑ते । \newline
7. क्रन्द॑ते॒ स्वाहा॒ स्वाहा॒ क्रन्द॑ते॒ क्रन्द॑ते॒ स्वाहा᳚ । \newline
8. स्वाहा॑ ऽव॒क्रन्द॑ते ऽव॒क्रन्द॑ते॒ स्वाहा॒ स्वाहा॑ ऽव॒क्रन्द॑ते । \newline
9. अ॒व॒क्रन्द॑ते॒ स्वाहा॒ स्वाहा॑ ऽव॒क्रन्द॑ते ऽव॒क्रन्द॑ते॒ स्वाहा᳚ । \newline
10. अ॒व॒क्रन्द॑त॒ इत्य॑व - क्रन्द॑ते । \newline
11. स्वाहा॒ प्रोथ॑ते॒ प्रोथ॑ते॒ स्वाहा॒ स्वाहा॒ प्रोथ॑ते । \newline
12. प्रोथ॑ते॒ स्वाहा॒ स्वाहा॒ प्रोथ॑ते॒ प्रोथ॑ते॒ स्वाहा᳚ । \newline
13. स्वाहा᳚ प्र॒प्रोथ॑ते प्र॒प्रोथ॑ते॒ स्वाहा॒ स्वाहा᳚ प्र॒प्रोथ॑ते । \newline
14. प्र॒प्रोथ॑ते॒ स्वाहा॒ स्वाहा᳚ प्र॒प्रोथ॑ते प्र॒प्रोथ॑ते॒ स्वाहा᳚ । \newline
15. प्र॒प्रोथ॑त॒ इति॑ प्र - प्रोथ॑ते । \newline
16. स्वाहा॑ ग॒न्धाय॑ ग॒न्धाय॒ स्वाहा॒ स्वाहा॑ ग॒न्धाय॑ । \newline
17. ग॒न्धाय॒ स्वाहा॒ स्वाहा॑ ग॒न्धाय॑ ग॒न्धाय॒ स्वाहा᳚ । \newline
18. स्वाहा᳚ घ्रा॒ताय॑ घ्रा॒ताय॒ स्वाहा॒ स्वाहा᳚ घ्रा॒ताय॑ । \newline
19. घ्रा॒ताय॒ स्वाहा॒ स्वाहा᳚ घ्रा॒ताय॑ घ्रा॒ताय॒ स्वाहा᳚ । \newline
20. स्वाहा᳚ प्रा॒णाय॑ प्रा॒णाय॒ स्वाहा॒ स्वाहा᳚ प्रा॒णाय॑ । \newline
21. प्रा॒णाय॒ स्वाहा॒ स्वाहा᳚ प्रा॒णाय॑ प्रा॒णाय॒ स्वाहा᳚ । \newline
22. प्रा॒णायेति॑ प्र - अ॒नाय॑ । \newline
23. स्वाहा᳚ व्या॒नाय॑ व्या॒नाय॒ स्वाहा॒ स्वाहा᳚ व्या॒नाय॑ । \newline
24. व्या॒नाय॒ स्वाहा॒ स्वाहा᳚ व्या॒नाय॑ व्या॒नाय॒ स्वाहा᳚ । \newline
25. व्या॒नायेति॑ वि - अ॒नाय॑ । \newline
26. स्वाहा॑ ऽपा॒नाया॑ पा॒नाय॒ स्वाहा॒ स्वाहा॑ ऽपा॒नाय॑ । \newline
27. अ॒पा॒नाय॒ स्वाहा॒ स्वाहा॑ ऽपा॒नाया॑ पा॒नाय॒ स्वाहा᳚ । \newline
28. अ॒पा॒नायेत्य॑प - अ॒नाय॑ । \newline
29. स्वाहा॑ सन्दी॒यमा॑नाय सन्दी॒यमा॑नाय॒ स्वाहा॒ स्वाहा॑ सन्दी॒यमा॑नाय । \newline
30. स॒न्दी॒यमा॑नाय॒ स्वाहा॒ स्वाहा॑ सन्दी॒यमा॑नाय सन्दी॒यमा॑नाय॒ स्वाहा᳚ । \newline
31. स॒न्दी॒यमा॑ना॒येति॑ सं - दी॒यमा॑नाय । \newline
32. स्वाहा॒ सन्दि॑ताय॒ सन्दि॑ताय॒ स्वाहा॒ स्वाहा॒ सन्दि॑ताय । \newline
33. सन्दि॑ताय॒ स्वाहा॒ स्वाहा॒ सन्दि॑ताय॒ सन्दि॑ताय॒ स्वाहा᳚ । \newline
34. सन्दि॑ता॒येति॒ सं - दि॒ता॒य॒ । \newline
35. स्वाहा॑ विचृ॒त्यमा॑नाय विचृ॒त्यमा॑नाय॒ स्वाहा॒ स्वाहा॑ विचृ॒त्यमा॑नाय । \newline
36. वि॒चृ॒त्यमा॑नाय॒ स्वाहा॒ स्वाहा॑ विचृ॒त्यमा॑नाय विचृ॒त्यमा॑नाय॒ स्वाहा᳚ । \newline
37. वि॒चृ॒त्यमा॑ना॒येति॑ वि - चृ॒त्यमा॑नाय । \newline
38. स्वाहा॒ विचृ॑त्ताय॒ विचृ॑त्ताय॒ स्वाहा॒ स्वाहा॒ विचृ॑त्ताय । \newline
39. विचृ॑त्ताय॒ स्वाहा॒ स्वाहा॒ विचृ॑त्ताय॒ विचृ॑त्ताय॒ स्वाहा᳚ । \newline
40. विचृ॑त्ता॒येति॒ वि - चृ॒त्ता॒य॒ । \newline
41. स्वाहा॑ पलायि॒ष्यमा॑णाय पलायि॒ष्यमा॑णाय॒ स्वाहा॒ स्वाहा॑ पलायि॒ष्यमा॑णाय । \newline
42. प॒ला॒यि॒ष्यमा॑णाय॒ स्वाहा॒ स्वाहा॑ पलायि॒ष्यमा॑णाय पलायि॒ष्यमा॑णाय॒ स्वाहा᳚ । \newline
43. स्वाहा॒ पला॑यिताय॒ पला॑यिताय॒ स्वाहा॒ स्वाहा॒ पला॑यिताय । \newline
44. पला॑यिताय॒ स्वाहा॒ स्वाहा॒ पला॑यिताय॒ पला॑यिताय॒ स्वाहा᳚ । \newline
45. स्वाहो॑ परꣳस्य॒त उ॑परꣳस्य॒ते स्वाहा॒ स्वाहो॑ परꣳस्य॒ते । \newline
46. उ॒प॒रꣳ॒॒स्य॒ते स्वाहा॒ स्वाहो॑ परꣳस्य॒त उ॑परꣳस्य॒ते स्वाहा᳚ । \newline
47. उ॒प॒रꣳ॒॒स्य॒त इत्यु॑प - रꣳ॒॒स्य॒ते । \newline
48. स्वाहोप॑रता॒ योप॑रताय॒ स्वाहा॒ स्वाहोप॑रताय । \newline
49. उप॑रताय॒ स्वाहा॒ स्वाहोप॑रता॒ योप॑रताय॒ स्वाहा᳚ । \newline
50. उप॑रता॒येत्युप॑ - र॒ता॒य॒ । \newline
51. स्वाहा॑ निवेक्ष्य॒ते नि॑वेक्ष्य॒ते स्वाहा॒ स्वाहा॑ निवेक्ष्य॒ते । \newline
52. नि॒वे॒क्ष्य॒ते स्वाहा॒ स्वाहा॑ निवेक्ष्य॒ते नि॑वेक्ष्य॒ते स्वाहा᳚ । \newline
53. नि॒वे॒क्ष्य॒त इति॑ नि - वे॒क्ष्य॒ते । \newline
54. स्वाहा॑ निवि॒शमा॑नाय निवि॒शमा॑नाय॒ स्वाहा॒ स्वाहा॑ निवि॒शमा॑नाय । \newline
55. नि॒वि॒शमा॑नाय॒ स्वाहा॒ स्वाहा॑ निवि॒शमा॑नाय निवि॒शमा॑नाय॒ स्वाहा᳚ । \newline
56. नि॒वि॒शमा॑ना॒येति॑ नि - वि॒शमा॑नाय । \newline
57. स्वाहा॒ निवि॑ष्टाय॒ निवि॑ष्टाय॒ स्वाहा॒ स्वाहा॒ निवि॑ष्टाय । \newline
58. निवि॑ष्टाय॒ स्वाहा॒ स्वाहा॒ निवि॑ष्टाय॒ निवि॑ष्टाय॒ स्वाहा᳚ । \newline
59. निवि॑ष्टा॒येति॒ नि - वि॒ष्टा॒य॒ । \newline
60. स्वाहा॑ निषथ्स्य॒ते नि॑षथ्स्य॒ते स्वाहा॒ स्वाहा॑ निषथ्स्य॒ते । \newline
61. नि॒ष॒थ्स्य॒ते स्वाहा॒ स्वाहा॑ निषथ्स्य॒ते नि॑षथ्स्य॒ते स्वाहा᳚ । \newline
62. नि॒ष॒थ्स्य॒त इति॑ नि - स॒थ्स्य॒ते । \newline
63. स्वाहा॑ नि॒षीद॑ते नि॒षीद॑ते॒ स्वाहा॒ स्वाहा॑ नि॒षीद॑ते । \newline
64. नि॒षीद॑ते॒ स्वाहा॒ स्वाहा॑ नि॒षीद॑ते नि॒षीद॑ते॒ स्वाहा᳚ । \newline
65. नि॒षीद॑त॒ इति॑ नि - सीद॑ते । \newline
66. स्वाहा॒ निष॑ण्णाय॒ निष॑ण्णाय॒ स्वाहा॒ स्वाहा॒ निष॑ण्णाय । \newline
67. निष॑ण्णाय॒ स्वाहा॒ स्वाहा॒ निष॑ण्णाय॒ निष॑ण्णाय॒ स्वाहा᳚ । \newline
68. निष॑ण्णा॒येति॒ नि - स॒न्ना॒य॒ । \newline
69. स्वाहा॑ ऽऽसिष्य॒त आ॑सिष्य॒ते स्वाहा॒ स्वाहा॑ ऽऽसिष्य॒ते । \newline

\textbf{Ghana Paata } \newline

1. ई॒ङ्का॒राय॒ स्वाहा॒ स्वाहे᳚ङ्का॒रा ये᳚ङ्का॒राय॒ स्वाहेङ्कृ॑ता॒ येङ्कृ॑ताय॒ स्वाहे᳚ङ्का॒रा ये᳚ङ्का॒राय॒ स्वाहेङ्कृ॑ताय । \newline
2. ई॒ङ्का॒रायेती᳚म् - का॒राय॑ । \newline
3. स्वाहेङ्कृ॑ता॒ येङ्कृ॑ताय॒ स्वाहा॒ स्वाहेङ्कृ॑ताय॒ स्वाहा॒ स्वाहेङ्कृ॑ताय॒ स्वाहा॒ स्वाहेङ्कृ॑ताय॒ स्वाहा᳚ । \newline
4. ईङ्कृ॑ताय॒ स्वाहा॒ स्वाहेङ्कृ॑ता॒ येङ्कृ॑ताय॒ स्वाहा॒ क्रन्द॑ते॒ क्रन्द॑ते॒ स्वाहेङ्कृ॑ता॒ येङ्कृ॑ताय॒ स्वाहा॒ क्रन्द॑ते । \newline
5. ईङ्कृ॑ता॒येतीम् - कृ॒ता॒य॒ । \newline
6. स्वाहा॒ क्रन्द॑ते॒ क्रन्द॑ते॒ स्वाहा॒ स्वाहा॒ क्रन्द॑ते॒ स्वाहा॒ स्वाहा॒ क्रन्द॑ते॒ स्वाहा॒ स्वाहा॒ क्रन्द॑ते॒ स्वाहा᳚ । \newline
7. क्रन्द॑ते॒ स्वाहा॒ स्वाहा॒ क्रन्द॑ते॒ क्रन्द॑ते॒ स्वाहा॑ ऽव॒क्रन्द॑ते ऽव॒क्रन्द॑ते॒ स्वाहा॒ क्रन्द॑ते॒ क्रन्द॑ते॒ स्वाहा॑ ऽव॒क्रन्द॑ते । \newline
8. स्वाहा॑ ऽव॒क्रन्द॑ते ऽव॒क्रन्द॑ते॒ स्वाहा॒ स्वाहा॑ ऽव॒क्रन्द॑ते॒ स्वाहा॒ स्वाहा॑ ऽव॒क्रन्द॑ते॒ स्वाहा॒ स्वाहा॑ ऽव॒क्रन्द॑ते॒ स्वाहा᳚ । \newline
9. अ॒व॒क्रन्द॑ते॒ स्वाहा॒ स्वाहा॑ ऽव॒क्रन्द॑ते ऽव॒क्रन्द॑ते॒ स्वाहा॒ प्रोथ॑ते॒ प्रोथ॑ते॒ स्वाहा॑ ऽव॒क्रन्द॑ते ऽव॒क्रन्द॑ते॒ स्वाहा॒ प्रोथ॑ते । \newline
10. अ॒व॒क्रन्द॑त॒ इत्य॑व - क्रन्द॑ते । \newline
11. स्वाहा॒ प्रोथ॑ते॒ प्रोथ॑ते॒ स्वाहा॒ स्वाहा॒ प्रोथ॑ते॒ स्वाहा॒ स्वाहा॒ प्रोथ॑ते॒ स्वाहा॒ स्वाहा॒ प्रोथ॑ते॒ स्वाहा᳚ । \newline
12. प्रोथ॑ते॒ स्वाहा॒ स्वाहा॒ प्रोथ॑ते॒ प्रोथ॑ते॒ स्वाहा᳚ प्र॒प्रोथ॑ते प्र॒प्रोथ॑ते॒ स्वाहा॒ प्रोथ॑ते॒ प्रोथ॑ते॒ स्वाहा᳚ प्र॒प्रोथ॑ते । \newline
13. स्वाहा᳚ प्र॒प्रोथ॑ते प्र॒प्रोथ॑ते॒ स्वाहा॒ स्वाहा᳚ प्र॒प्रोथ॑ते॒ स्वाहा॒ स्वाहा᳚ प्र॒प्रोथ॑ते॒ स्वाहा॒ स्वाहा᳚ प्र॒प्रोथ॑ते॒ स्वाहा᳚ । \newline
14. प्र॒प्रोथ॑ते॒ स्वाहा॒ स्वाहा᳚ प्र॒प्रोथ॑ते प्र॒प्रोथ॑ते॒ स्वाहा॑ ग॒न्धाय॑ ग॒न्धाय॒ स्वाहा᳚ प्र॒प्रोथ॑ते प्र॒प्रोथ॑ते॒ स्वाहा॑ ग॒न्धाय॑ । \newline
15. प्र॒प्रोथ॑त॒ इति॑ प्र - प्रोथ॑ते । \newline
16. स्वाहा॑ ग॒न्धाय॑ ग॒न्धाय॒ स्वाहा॒ स्वाहा॑ ग॒न्धाय॒ स्वाहा॒ स्वाहा॑ ग॒न्धाय॒ स्वाहा॒ स्वाहा॑ ग॒न्धाय॒ स्वाहा᳚ । \newline
17. ग॒न्धाय॒ स्वाहा॒ स्वाहा॑ ग॒न्धाय॑ ग॒न्धाय॒ स्वाहा᳚ घ्रा॒ताय॑ घ्रा॒ताय॒ स्वाहा॑ ग॒न्धाय॑ ग॒न्धाय॒ स्वाहा᳚ घ्रा॒ताय॑ । \newline
18. स्वाहा᳚ घ्रा॒ताय॑ घ्रा॒ताय॒ स्वाहा॒ स्वाहा᳚ घ्रा॒ताय॒ स्वाहा॒ स्वाहा᳚ घ्रा॒ताय॒ स्वाहा॒ स्वाहा᳚ घ्रा॒ताय॒ स्वाहा᳚ । \newline
19. घ्रा॒ताय॒ स्वाहा॒ स्वाहा᳚ घ्रा॒ताय॑ घ्रा॒ताय॒ स्वाहा᳚ प्रा॒णाय॑ प्रा॒णाय॒ स्वाहा᳚ घ्रा॒ताय॑ घ्रा॒ताय॒ स्वाहा᳚ प्रा॒णाय॑ । \newline
20. स्वाहा᳚ प्रा॒णाय॑ प्रा॒णाय॒ स्वाहा॒ स्वाहा᳚ प्रा॒णाय॒ स्वाहा॒ स्वाहा᳚ प्रा॒णाय॒ स्वाहा॒ स्वाहा᳚ प्रा॒णाय॒ स्वाहा᳚ । \newline
21. प्रा॒णाय॒ स्वाहा॒ स्वाहा᳚ प्रा॒णाय॑ प्रा॒णाय॒ स्वाहा᳚ व्या॒नाय॑ व्या॒नाय॒ स्वाहा᳚ प्रा॒णाय॑ प्रा॒णाय॒ स्वाहा᳚ व्या॒नाय॑ । \newline
22. प्रा॒णायेति॑ प्र - अ॒नाय॑ । \newline
23. स्वाहा᳚ व्या॒नाय॑ व्या॒नाय॒ स्वाहा॒ स्वाहा᳚ व्या॒नाय॒ स्वाहा॒ स्वाहा᳚ व्या॒नाय॒ स्वाहा॒ स्वाहा᳚ व्या॒नाय॒ स्वाहा᳚ । \newline
24. व्या॒नाय॒ स्वाहा॒ स्वाहा᳚ व्या॒नाय॑ व्या॒नाय॒ स्वाहा॑ ऽपा॒नाया॑ पा॒नाय॒ स्वाहा᳚ व्या॒नाय॑ व्या॒नाय॒ स्वाहा॑ ऽपा॒नाय॑ । \newline
25. व्या॒नायेति॑ वि - अ॒नाय॑ । \newline
26. स्वाहा॑ ऽपा॒नाया॑ पा॒नाय॒ स्वाहा॒ स्वाहा॑ ऽपा॒नाय॒ स्वाहा॒ स्वाहा॑ ऽपा॒नाय॒ स्वाहा॒ स्वाहा॑ ऽपा॒नाय॒ स्वाहा᳚ । \newline
27. अ॒पा॒नाय॒ स्वाहा॒ स्वाहा॑ ऽपा॒नाया॑ पा॒नाय॒ स्वाहा॑ सन्दी॒यमा॑नाय सन्दी॒यमा॑नाय॒ स्वाहा॑ ऽपा॒नाया॑ पा॒नाय॒ स्वाहा॑ सन्दी॒यमा॑नाय । \newline
28. अ॒पा॒नायेत्य॑प - अ॒नाय॑ । \newline
29. स्वाहा॑ सन्दी॒यमा॑नाय सन्दी॒यमा॑नाय॒ स्वाहा॒ स्वाहा॑ सन्दी॒यमा॑नाय॒ स्वाहा॒ स्वाहा॑ सन्दी॒यमा॑नाय॒ स्वाहा॒ स्वाहा॑ सन्दी॒यमा॑नाय॒ स्वाहा᳚ । \newline
30. स॒न्दी॒यमा॑नाय॒ स्वाहा॒ स्वाहा॑ सन्दी॒यमा॑नाय सन्दी॒यमा॑नाय॒ स्वाहा॒ सन्दि॑ताय॒ सन्दि॑ताय॒ स्वाहा॑ सन्दी॒यमा॑नाय सन्दी॒यमा॑नाय॒ स्वाहा॒ सन्दि॑ताय । \newline
31. स॒न्दी॒यमा॑ना॒येति॑ सं - दी॒यमा॑नाय । \newline
32. स्वाहा॒ सन्दि॑ताय॒ सन्दि॑ताय॒ स्वाहा॒ स्वाहा॒ सन्दि॑ताय॒ स्वाहा॒ स्वाहा॒ सन्दि॑ताय॒ स्वाहा॒ स्वाहा॒ सन्दि॑ताय॒ स्वाहा᳚ । \newline
33. सन्दि॑ताय॒ स्वाहा॒ स्वाहा॒ सन्दि॑ताय॒ सन्दि॑ताय॒ स्वाहा॑ विचृ॒त्यमा॑नाय विचृ॒त्यमा॑नाय॒ स्वाहा॒ सन्दि॑ताय॒ सन्दि॑ताय॒ स्वाहा॑ विचृ॒त्यमा॑नाय । \newline
34. सन्दि॑ता॒येति॒ सं - दि॒ता॒य॒ । \newline
35. स्वाहा॑ विचृ॒त्यमा॑नाय विचृ॒त्यमा॑नाय॒ स्वाहा॒ स्वाहा॑ विचृ॒त्यमा॑नाय॒ स्वाहा॒ स्वाहा॑ विचृ॒त्यमा॑नाय॒ स्वाहा॒ स्वाहा॑ विचृ॒त्यमा॑नाय॒ स्वाहा᳚ । \newline
36. वि॒चृ॒त्यमा॑नाय॒ स्वाहा॒ स्वाहा॑ विचृ॒त्यमा॑नाय विचृ॒त्यमा॑नाय॒ स्वाहा॒ विचृ॑त्ताय॒ विचृ॑त्ताय॒ स्वाहा॑ विचृ॒त्यमा॑नाय विचृ॒त्यमा॑नाय॒ स्वाहा॒ विचृ॑त्ताय । \newline
37. वि॒चृ॒त्यमा॑ना॒येति॑ वि - चृ॒त्यमा॑नाय । \newline
38. स्वाहा॒ विचृ॑त्ताय॒ विचृ॑त्ताय॒ स्वाहा॒ स्वाहा॒ विचृ॑त्ताय॒ स्वाहा॒ स्वाहा॒ विचृ॑त्ताय॒ स्वाहा॒ स्वाहा॒ विचृ॑त्ताय॒ स्वाहा᳚ । \newline
39. विचृ॑त्ताय॒ स्वाहा॒ स्वाहा॒ विचृ॑त्ताय॒ विचृ॑त्ताय॒ स्वाहा॑ पलायि॒ष्यमा॑णाय पलायि॒ष्यमा॑णाय॒ स्वाहा॒ विचृ॑त्ताय॒ विचृ॑त्ताय॒ स्वाहा॑ पलायि॒ष्यमा॑णाय । \newline
40. विचृ॑त्ता॒येति॒ वि - चृ॒त्ता॒य॒ । \newline
41. स्वाहा॑ पलायि॒ष्यमा॑णाय पलायि॒ष्यमा॑णाय॒ स्वाहा॒ स्वाहा॑ पलायि॒ष्यमा॑णाय॒ स्वाहा॒ स्वाहा॑ पलायि॒ष्यमा॑णाय॒ स्वाहा॒ स्वाहा॑ पलायि॒ष्यमा॑णाय॒ स्वाहा᳚ । \newline
42. प॒ला॒यि॒ष्यमा॑णाय॒ स्वाहा॒ स्वाहा॑ पलायि॒ष्यमा॑णाय पलायि॒ष्यमा॑णाय॒ स्वाहा॒ पला॑यिताय॒ पला॑यिताय॒ स्वाहा॑ पलायि॒ष्यमा॑णाय पलायि॒ष्यमा॑णाय॒ स्वाहा॒ पला॑यिताय । \newline
43. स्वाहा॒ पला॑यिताय॒ पला॑यिताय॒ स्वाहा॒ स्वाहा॒ पला॑यिताय॒ स्वाहा॒ स्वाहा॒ पला॑यिताय॒ स्वाहा॒ स्वाहा॒ पला॑यिताय॒ स्वाहा᳚ । \newline
44. पला॑यिताय॒ स्वाहा॒ स्वाहा॒ पला॑यिताय॒ पला॑यिताय॒ स्वाहो॑परꣳस्य॒त उ॑परꣳस्य॒ते स्वाहा॒ पला॑यिताय॒ पला॑यिताय॒ स्वाहो॑परꣳस्य॒ते । \newline
45. स्वाहो॑परꣳस्य॒त उ॑परꣳस्य॒ते स्वाहा॒ स्वाहो॑परꣳस्य॒ते स्वाहा॒ स्वाहो॑परꣳस्य॒ते स्वाहा॒ 
स्वाहो॑परꣳस्य॒ते स्वाहा᳚ । \newline
46. उ॒प॒रꣳ॒॒स्य॒ते स्वाहा॒ स्वाहो॑परꣳस्य॒त उ॑परꣳस्य॒ते स्वाहोप॑रता॒ योप॑रताय॒ स्वाहो॑ परꣳस्य॒त उ॑परꣳस्य॒ते स्वाहोप॑रताय । \newline
47. उ॒प॒रꣳ॒॒स्य॒त इत्यु॑प - रꣳ॒॒स्य॒ते । \newline
48. स्वाहोप॑रता॒ योप॑रताय॒ स्वाहा॒ स्वाहोप॑रताय॒ स्वाहा॒ स्वाहोप॑रताय॒ स्वाहा॒ स्वाहोप॑रताय॒ स्वाहा᳚ । \newline
49. उप॑रताय॒ स्वाहा॒ स्वाहोप॑रता॒ योप॑रताय॒ स्वाहा॑ निवेक्ष्य॒ते नि॑वेक्ष्य॒ते स्वाहोप॑रता॒ योप॑रताय॒ स्वाहा॑ निवेक्ष्य॒ते । \newline
50. उप॑रता॒येत्युप॑ - र॒ता॒य॒ । \newline
51. स्वाहा॑ निवेक्ष्य॒ते नि॑वेक्ष्य॒ते स्वाहा॒ स्वाहा॑ निवेक्ष्य॒ते स्वाहा॒ स्वाहा॑ निवेक्ष्य॒ते स्वाहा॒ स्वाहा॑ निवेक्ष्य॒ते स्वाहा᳚ । \newline
52. नि॒वे॒क्ष्य॒ते स्वाहा॒ स्वाहा॑ निवेक्ष्य॒ते नि॑वेक्ष्य॒ते स्वाहा॑ निवि॒शमा॑नाय निवि॒शमा॑नाय॒ स्वाहा॑ निवेक्ष्य॒ते नि॑वेक्ष्य॒ते स्वाहा॑ निवि॒शमा॑नाय । \newline
53. नि॒वे॒क्ष्य॒त इति॑ नि - वे॒क्ष्य॒ते । \newline
54. स्वाहा॑ निवि॒शमा॑नाय निवि॒शमा॑नाय॒ स्वाहा॒ स्वाहा॑ निवि॒शमा॑नाय॒ स्वाहा॒ स्वाहा॑ निवि॒शमा॑नाय॒ स्वाहा॒ स्वाहा॑ निवि॒शमा॑नाय॒ स्वाहा᳚ । \newline
55. नि॒वि॒शमा॑नाय॒ स्वाहा॒ स्वाहा॑ निवि॒शमा॑नाय निवि॒शमा॑नाय॒ स्वाहा॒ निवि॑ष्टाय॒ निवि॑ष्टाय॒ स्वाहा॑ निवि॒शमा॑नाय निवि॒शमा॑नाय॒ स्वाहा॒ निवि॑ष्टाय । \newline
56. नि॒वि॒शमा॑ना॒येति॑ नि - वि॒शमा॑नाय । \newline
57. स्वाहा॒ निवि॑ष्टाय॒ निवि॑ष्टाय॒ स्वाहा॒ स्वाहा॒ निवि॑ष्टाय॒ स्वाहा॒ स्वाहा॒ निवि॑ष्टाय॒ स्वाहा॒ स्वाहा॒ निवि॑ष्टाय॒ स्वाहा᳚ । \newline
58. निवि॑ष्टाय॒ स्वाहा॒ स्वाहा॒ निवि॑ष्टाय॒ निवि॑ष्टाय॒ स्वाहा॑ निषथ्स्य॒ते नि॑षथ्स्य॒ते स्वाहा॒ निवि॑ष्टाय॒ निवि॑ष्टाय॒ स्वाहा॑ निषथ्स्य॒ते । \newline
59. निवि॑ष्टा॒येति॒ नि - वि॒ष्टा॒य॒ । \newline
60. स्वाहा॑ निषथ्स्य॒ते नि॑षथ्स्य॒ते स्वाहा॒ स्वाहा॑ निषथ्स्य॒ते स्वाहा॒ स्वाहा॑ निषथ्स्य॒ते स्वाहा॒ स्वाहा॑ निषथ्स्य॒ते स्वाहा᳚ । \newline
61. नि॒ष॒थ्स्य॒ते स्वाहा॒ स्वाहा॑ निषथ्स्य॒ते नि॑षथ्स्य॒ते स्वाहा॑ नि॒षीद॑ते नि॒षीद॑ते॒ स्वाहा॑ निषथ्स्य॒ते नि॑षथ्स्य॒ते स्वाहा॑ नि॒षीद॑ते । \newline
62. नि॒ष॒थ्स्य॒त इति॑ नि - स॒थ्स्य॒ते । \newline
63. स्वाहा॑ नि॒षीद॑ते नि॒षीद॑ते॒ स्वाहा॒ स्वाहा॑ नि॒षीद॑ते॒ स्वाहा॒ स्वाहा॑ नि॒षीद॑ते॒ स्वाहा॒ स्वाहा॑ नि॒षीद॑ते॒ स्वाहा᳚ । \newline
64. नि॒षीद॑ते॒ स्वाहा॒ स्वाहा॑ नि॒षीद॑ते नि॒षीद॑ते॒ स्वाहा॒ निष॑ण्णाय॒ निष॑ण्णाय॒ स्वाहा॑ नि॒षीद॑ते नि॒षीद॑ते॒ स्वाहा॒ निष॑ण्णाय । \newline
65. नि॒षीद॑त॒ इति॑ नि - सीद॑ते । \newline
66. स्वाहा॒ निष॑ण्णाय॒ निष॑ण्णाय॒ स्वाहा॒ स्वाहा॒ निष॑ण्णाय॒ स्वाहा॒ स्वाहा॒ निष॑ण्णाय॒ स्वाहा॒ स्वाहा॒ निष॑ण्णाय॒ स्वाहा᳚ । \newline
67. निष॑ण्णाय॒ स्वाहा॒ स्वाहा॒ निष॑ण्णाय॒ निष॑ण्णाय॒ स्वाहा॑ ऽऽसिष्य॒त आ॑सिष्य॒ते स्वाहा॒ निष॑ण्णाय॒ निष॑ण्णाय॒ स्वाहा॑ ऽऽसिष्य॒ते । \newline
68. निष॑ण्णा॒येति॒ नि - स॒न्ना॒य॒ । \newline
69. स्वाहा॑ ऽऽसिष्य॒त आ॑सिष्य॒ते स्वाहा॒ स्वाहा॑ ऽऽसिष्य॒ते स्वाहा॒ स्वाहा॑ ऽऽसिष्य॒ते स्वाहा॒ स्वाहा॑ ऽऽसिष्य॒ते स्वाहा᳚ । \newline
\pagebreak
\markright{ TS 7.1.19.2  \hfill https://www.vedavms.in \hfill}

\section{ TS 7.1.19.2 }

\textbf{TS 7.1.19.2 } \newline
\textbf{Samhita Paata} \newline

ऽऽसिष्य॒ते स्वाहा ऽऽसी॑नाय॒ स्वाहा॑ ऽऽसि॒ताय॒ स्वाहा॑ निपथ्स्य॒ते स्वाहा॑ नि॒पद्य॑मानाय॒ स्वाहा॒ निप॑न्नाय॒ स्वाहा॑ शयिष्य॒ते स्वाहा॒ शया॑नाय॒ स्वाहा॑ शयि॒ताय॒ स्वाहा॑ संमीलिष्य॒ते स्वाहा॑ स॒मींल॑ते॒ स्वाहा॒ संमी॑लिताय॒ स्वाहा᳚ स्वफ्स्य॒ते स्वाहा᳚ स्वप॒ते स्वाहा॑ सु॒प्ताय॒ स्वाहा᳚ प्रभोथ्स्य॒ते स्वाहा᳚ प्र॒बुद्ध्य॑मानाय॒ स्वाहा॒ प्रबु॑द्धाय॒ स्वाहा॑ जागरिष्य॒ते स्वाहा॒ जाग्र॑ते॒ स्वाहा॑ जागरि॒ताय॒ स्वाहा॒ शुश्रू॑षमाणाय॒ स्वाहा॑ शृण्व॒ते स्वाहा᳚ श्रु॒ताय॒ स्वाहा॑ वीक्षिष्य॒ते स्वाहा॒ - [  ] \newline

\textbf{Pada Paata} \newline

आ॒सि॒ष्य॒ते । स्वाहा᳚ । आसी॑नाय । स्वाहा᳚ । आ॒सि॒ताय॑ । स्वाहा᳚ । नि॒प॒थ्स्य॒त इति॑ नि - प॒थ्स्य॒ते । स्वाहा᳚ । नि॒पद्य॑माना॒येति॑ नि - पद्य॑मानाय । स्वाहा᳚ । निप॑न्ना॒येति॒ नि - प॒न्ना॒य॒ । स्वाहा᳚ । श॒यि॒ष्य॒ते । स्वाहा᳚ । शया॑नाय । स्वाहा᳚ । श॒यि॒ताय॑ । स्वाहा᳚ । स॒म्मी॒लि॒ष्य॒त इति॑ सं - मी॒लि॒ष्य॒ते । स्वाहा᳚ । स॒म्मील॑त॒ इति॑ सं - मील॑ते । स्वाहा᳚ । सम्मी॑लिता॒येति॒ सं - मी॒लि॒ता॒य॒ । स्वाहा᳚ । स्व॒फ्स्य॒ते । स्वाहा᳚ । स्व॒प॒ते । स्वाहा᳚ । सु॒प्ताय॑ । स्वाहा᳚ । प्र॒भो॒थ्स्य॒त इति॑ प्र - भो॒थ्स्य॒ते । स्वाहा᳚ । प्र॒बुद्ध्य॑माना॒येति॑ प्र - बुद्ध्य॑मानाय । स्वाहा᳚ । प्रबु॑द्धा॒येति॒ प्र - बु॒द्धा॒य॒ । स्वाहा᳚ । जा॒ग॒रि॒ष्य॒ते । स्वाहा᳚ । जाग्र॑ते । स्वाहा᳚ । जा॒ग॒रि॒ताय॑ । स्वाहा᳚ । शुश्रू॑षमाणाय । स्वाहा᳚ । शृ॒ण्व॒ते । स्वाहा᳚ । श्रु॒ताय॑ । स्वाहा᳚ । वी॒क्षि॒ष्य॒त इति॑ वि - ई॒क्षि॒ष्य॒ते । स्वाहा᳚ ।  \newline


\textbf{Krama Paata} \newline

आ॒सि॒ष्य॒ते स्वाहा᳚ । स्वाहाऽऽसी॑नाय । आसी॑नाय॒ स्वाहा᳚ । स्वाहा॑ऽऽसि॒ताय॑ । आ॒सि॒ताय॒ स्वाहा᳚ । स्वाहा॑ निपथ्स्य॒ते । नि॒प॒थ्स्य॒ते स्वाहा᳚ । नि॒प॒थ्स्य॒त इति॑ नि - प॒थ्स्य॒ते । स्वाहा॑ नि॒पद्य॑मानाय । नि॒पद्य॑मानाय॒ स्वाहा᳚ । नि॒पद्य॑माना॒येति॑ नि - पद्य॑मानाय । स्वाहा॒ निप॑न्नाय । निप॑न्नाय॒ स्वाहा᳚ । निप॑न्ना॒येति॒ नि - प॒न्ना॒य॒ । स्वाहा॑ शयिष्य॒ते । श॒यि॒ष्य॒ते स्वाहा᳚ । स्वाहा॒ शया॑नाय । शया॑नाय॒ स्वाहा᳚ । स्वाहा॑ शयि॒ताय॑ । श॒यि॒ताय॒ स्वाहा᳚ । स्वाहा॑ सम्मीलिष्य॒ते । स॒म्मी॒लि॒ष्य॒ते स्वाहा᳚ । स॒म्मी॒लि॒ष्य॒त इति॑ सम् - मी॒लि॒ष्य॒ते । स्वाहा॑ स॒म्मील॑ते । स॒म्मी॑लते॒ स्वाहा᳚ । स॒म्मील॑त॒ इति॑ सम् - मील॑ते । स्वाहा॒ सम्मी॑लिताय । सम्मी॑लिताय॒ स्वाहा᳚ । सम्मी॑लिता॒येति॒ सम् - मी॒लि॒ता॒य॒ । स्वाहा᳚ स्वफ्स्य॒ते । स्व॒फ्स्य॒ते स्वाहा᳚ । स्वाहा᳚ स्वप॒ते । स्व॒प॒ते स्वाहा᳚ । स्वाहा॑ सु॒प्ताय॑ । सु॒प्ताय॒ स्वाहा᳚ । स्वाहा᳚ प्रभोथ्स्य॒ते । प्र॒भो॒थ्स्य॒ते स्वाहा᳚ । प्र॒भो॒थ्स्य॒त इति॑ प्र - भो॒थ्स्य॒ते । स्वाहा᳚ प्र॒बुद्ध्य॑मानाय । प्र॒बुद्ध्य॑मानाय॒ स्वाहा᳚ । प्र॒बुद्ध्य॑माना॒येति॑ प्र - बुद्ध्य॑मानाय । स्वाहा॒ प्रबु॑द्धाय । प्रबु॑द्धाय॒ स्वाहा᳚ । प्रबु॑द्धा॒येति॒ प्र - बु॒द्धा॒य॒ । स्वाहा॑ जागरिष्य॒ते । जा॒ग॒रि॒ष्य॒ते स्वाहा᳚ । स्वाहा॒ जाग्र॑ते । जाग्र॑ते॒ स्वाहा᳚ । स्वाहा॑ जागरि॒ताय॑ । जा॒ग॒रि॒ताय॒ स्वाहा᳚ । स्वाहा॒ शुश्रू॑षमाणाय । शुश्रू॑षमाणाय॒ स्वाहा᳚ । स्वाहा॑ शृण्व॒ते । शृ॒ण्व॒ते स्वाहा᳚ । स्वाहा᳚ श्रु॒ताय॑ । श्रु॒ताय॒ स्वाहा᳚ । स्वाहा॑ वीक्षिष्य॒ते । वी॒क्षि॒ष्य॒ते स्वाहा᳚ । वी॒क्ष॒ष्य॒त इति॑ वि - ई॒क्षि॒ष्य॒ते । स्वाहा॒ वीक्ष॑माणाय \newline

\textbf{Jatai Paata} \newline

1. आ॒सि॒ष्य॒ते स्वाहा॒ स्वाहा॑ ऽऽसिष्य॒त आ॑सिष्य॒ते स्वाहा᳚ । \newline
2. स्वाहा ऽऽसी॑ना॒या सी॑नाय॒ स्वाहा॒ स्वाहा ऽऽसी॑नाय । \newline
3. आसी॑नाय॒ स्वाहा॒ स्वाहा ऽऽसी॑ना॒या सी॑नाय॒ स्वाहा᳚ । \newline
4. स्वाहा॑ ऽऽसि॒ताया॑ सि॒ताय॒ स्वाहा॒ स्वाहा॑ ऽऽसि॒ताय॑ । \newline
5. आ॒सि॒ताय॒ स्वाहा॒ स्वाहा॑ ऽऽसि॒ताया॑ सि॒ताय॒ स्वाहा᳚ । \newline
6. स्वाहा॑ निपथ्स्य॒ते नि॑पथ्स्य॒ते स्वाहा॒ स्वाहा॑ निपथ्स्य॒ते । \newline
7. नि॒प॒थ्स्य॒ते स्वाहा॒ स्वाहा॑ निपथ्स्य॒ते नि॑पथ्स्य॒ते स्वाहा᳚ । \newline
8. नि॒प॒थ्स्य॒त इति॑ नि - प॒थ्स्य॒ते । \newline
9. स्वाहा॑ नि॒पद्य॑मानाय नि॒पद्य॑मानाय॒ स्वाहा॒ स्वाहा॑ नि॒पद्य॑मानाय । \newline
10. नि॒पद्य॑मानाय॒ स्वाहा॒ स्वाहा॑ नि॒पद्य॑मानाय नि॒पद्य॑मानाय॒ स्वाहा᳚ । \newline
11. नि॒पद्य॑माना॒येति॑ नि - पद्य॑मानाय । \newline
12. स्वाहा॒ निप॑न्नाय॒ निप॑न्नाय॒ स्वाहा॒ स्वाहा॒ निप॑न्नाय । \newline
13. निप॑न्नाय॒ स्वाहा॒ स्वाहा॒ निप॑न्नाय॒ निप॑न्नाय॒ स्वाहा᳚ । \newline
14. निप॑न्ना॒येति॒ नि - प॒न्ना॒य॒ । \newline
15. स्वाहा॑ शयिष्य॒ते श॑यिष्य॒ते स्वाहा॒ स्वाहा॑ शयिष्य॒ते । \newline
16. श॒यि॒ष्य॒ते स्वाहा॒ स्वाहा॑ शयिष्य॒ते श॑यिष्य॒ते स्वाहा᳚ । \newline
17. स्वाहा॒ शया॑नाय॒ शया॑नाय॒ स्वाहा॒ स्वाहा॒ शया॑नाय । \newline
18. शया॑नाय॒ स्वाहा॒ स्वाहा॒ शया॑नाय॒ शया॑नाय॒ स्वाहा᳚ । \newline
19. स्वाहा॑ शयि॒ताय॑ शयि॒ताय॒ स्वाहा॒ स्वाहा॑ शयि॒ताय॑ । \newline
20. श॒यि॒ताय॒ स्वाहा॒ स्वाहा॑ शयि॒ताय॑ शयि॒ताय॒ स्वाहा᳚ । \newline
21. स्वाहा॑ सम्मीलिष्य॒ते स॑म्मीलिष्य॒ते स्वाहा॒ स्वाहा॑ सम्मीलिष्य॒ते । \newline
22. स॒म्मी॒लि॒ष्य॒ते स्वाहा॒ स्वाहा॑ सम्मीलिष्य॒ते स॑म्मीलिष्य॒ते स्वाहा᳚ । \newline
23. स॒म्मी॒लि॒ष्य॒त इति॑ सं - मी॒लि॒ष्य॒ते । \newline
24. स्वाहा॑ स॒म्मील॑ते स॒म्मील॑ते॒ स्वाहा॒ स्वाहा॑ स॒म्मील॑ते । \newline
25. स॒म्मील॑ते॒ स्वाहा॒ स्वाहा॑ स॒म्मील॑ते स॒म्मील॑ते॒ स्वाहा᳚ । \newline
26. स॒म्मील॑त॒ इति॑ सं - मील॑ते । \newline
27. स्वाहा॒ सम्मी॑लिताय॒ सम्मी॑लिताय॒ स्वाहा॒ स्वाहा॒ सम्मी॑लिताय । \newline
28. सम्मी॑लिताय॒ स्वाहा॒ स्वाहा॒ सम्मी॑लिताय॒ सम्मी॑लिताय॒ स्वाहा᳚ । \newline
29. सम्मी॑लिता॒येति॒ सं - मी॒लि॒ता॒य॒ । \newline
30. स्वाहा᳚ स्वफ्स्य॒ते स्व॑फ्स्य॒ते स्वाहा॒ स्वाहा᳚ स्वफ्स्य॒ते । \newline
31. स्व॒फ्स्य॒ते स्वाहा॒ स्वाहा᳚ स्वफ्स्य॒ते स्व॑फ्स्य॒ते स्वाहा᳚ । \newline
32. स्वाहा᳚ स्वप॒ते स्व॑प॒ते स्वाहा॒ स्वाहा᳚ स्वप॒ते । \newline
33. स्व॒प॒ते स्वाहा॒ स्वाहा᳚ स्वप॒ते स्व॑प॒ते स्वाहा᳚ । \newline
34. स्वाहा॑ सु॒प्ताय॑ सु॒प्ताय॒ स्वाहा॒ स्वाहा॑ सु॒प्ताय॑ । \newline
35. सु॒प्ताय॒ स्वाहा॒ स्वाहा॑ सु॒प्ताय॑ सु॒प्ताय॒ स्वाहा᳚ । \newline
36. स्वाहा᳚ प्रभोथ्स्य॒ते प्र॑भोथ्स्य॒ते स्वाहा॒ स्वाहा᳚ प्रभोथ्स्य॒ते । \newline
37. प्र॒भो॒थ्स्य॒ते स्वाहा॒ स्वाहा᳚ प्रभोथ्स्य॒ते प्र॑भोथ्स्य॒ते स्वाहा᳚ । \newline
38. प्र॒भो॒थ्स्य॒त इति॑ प्र - भो॒थ्स्य॒ते । \newline
39. स्वाहा᳚ प्र॒बुद्ध्य॑मानाय प्र॒बुद्ध्य॑मानाय॒ स्वाहा॒ स्वाहा᳚ प्र॒बुद्ध्य॑मानाय । \newline
40. प्र॒बुद्ध्य॑मानाय॒ स्वाहा॒ स्वाहा᳚ प्र॒बुद्ध्य॑मानाय प्र॒बुद्ध्य॑मानाय॒ स्वाहा᳚ । \newline
41. प्र॒बुद्ध्य॑माना॒येति॑ प्र - बुद्ध्य॑मानाय । \newline
42. स्वाहा॒ प्रबु॑द्धाय॒ प्रबु॑द्धाय॒ स्वाहा॒ स्वाहा॒ प्रबु॑द्धाय । \newline
43. प्रबु॑द्धाय॒ स्वाहा॒ स्वाहा॒ प्रबु॑द्धाय॒ प्रबु॑द्धाय॒ स्वाहा᳚ । \newline
44. प्रबु॑द्धा॒येति॒ प्र - बु॒द्धा॒य॒ । \newline
45. स्वाहा॑ जागरिष्य॒ते जा॑गरिष्य॒ते स्वाहा॒ स्वाहा॑ जागरिष्य॒ते । \newline
46. जा॒ग॒रि॒ष्य॒ते स्वाहा॒ स्वाहा॑ जागरिष्य॒ते जा॑गरिष्य॒ते स्वाहा᳚ । \newline
47. स्वाहा॒ जाग्र॑ते॒ जाग्र॑ते॒ स्वाहा॒ स्वाहा॒ जाग्र॑ते । \newline
48. जाग्र॑ते॒ स्वाहा॒ स्वाहा॒ जाग्र॑ते॒ जाग्र॑ते॒ स्वाहा᳚ । \newline
49. स्वाहा॑ जागरि॒ताय॑ जागरि॒ताय॒ स्वाहा॒ स्वाहा॑ जागरि॒ताय॑ । \newline
50. जा॒ग॒रि॒ताय॒ स्वाहा॒ स्वाहा॑ जागरि॒ताय॑ जागरि॒ताय॒ स्वाहा᳚ । \newline
51. स्वाहा॒ शुश्रू॑षमाणाय॒ शुश्रू॑षमाणाय॒ स्वाहा॒ स्वाहा॒ शुश्रू॑षमाणाय । \newline
52. शुश्रू॑षमाणाय॒ स्वाहा॒ स्वाहा॒ शुश्रू॑षमाणाय॒ शुश्रू॑षमाणाय॒ स्वाहा᳚ । \newline
53. स्वाहा॑ शृण्व॒ते शृ॑ण्व॒ते स्वाहा॒ स्वाहा॑ शृण्व॒ते । \newline
54. शृ॒ण्व॒ते स्वाहा॒ स्वाहा॑ शृण्व॒ते शृ॑ण्व॒ते स्वाहा᳚ । \newline
55. स्वाहा᳚ श्रु॒ताय॑ श्रु॒ताय॒ स्वाहा॒ स्वाहा᳚ श्रु॒ताय॑ । \newline
56. श्रु॒ताय॒ स्वाहा॒ स्वाहा᳚ श्रु॒ताय॑ श्रु॒ताय॒ स्वाहा᳚ । \newline
57. स्वाहा॑ वीक्षिष्य॒ते वी᳚क्षिष्य॒ते स्वाहा॒ स्वाहा॑ वीक्षिष्य॒ते । \newline
58. वी॒क्षि॒ष्य॒ते स्वाहा॒ स्वाहा॑ वीक्षिष्य॒ते वी᳚क्षिष्य॒ते स्वाहा᳚ । \newline
59. वी॒क्षि॒ष्य॒त इति॑ वि - ई॒क्षि॒ष्य॒ते । \newline
60. स्वाहा॒ वीक्ष॑माणाय॒ वीक्ष॑माणाय॒ स्वाहा॒ स्वाहा॒ वीक्ष॑माणाय । \newline

\textbf{Ghana Paata } \newline

1. आ॒सि॒ष्य॒ते स्वाहा॒ स्वाहा॑ ऽऽसिष्य॒त आ॑सिष्य॒ते स्वाहा ऽऽसी॑ना॒या सी॑नाय॒ स्वाहा॑ ऽऽसिष्य॒त आ॑सिष्य॒ते स्वाहा ऽऽसी॑नाय । \newline
2. स्वाहा ऽऽसी॑ना॒या सी॑नाय॒ स्वाहा॒ स्वाहा ऽऽसी॑नाय॒ स्वाहा॒ स्वाहा ऽऽसी॑नाय॒ स्वाहा॒ स्वाहा ऽऽसी॑नाय॒ स्वाहा᳚ । \newline
3. आसी॑नाय॒ स्वाहा॒ स्वाहा ऽऽसी॑ना॒या सी॑नाय॒ स्वाहा॑ ऽऽसि॒ताया॑ सि॒ताय॒ स्वाहा ऽऽसी॑ना॒या सी॑नाय॒ स्वाहा॑ ऽऽसि॒ताय॑ । \newline
4. स्वाहा॑ ऽऽसि॒ताया॑ सि॒ताय॒ स्वाहा॒ स्वाहा॑ ऽऽसि॒ताय॒ स्वाहा॒ स्वाहा॑ ऽऽसि॒ताय॒ स्वाहा॒ स्वाहा॑ ऽऽसि॒ताय॒ स्वाहा᳚ । \newline
5. आ॒सि॒ताय॒ स्वाहा॒ स्वाहा॑ ऽऽसि॒ताया॑ सि॒ताय॒ स्वाहा॑ निपथ्स्य॒ते नि॑पथ्स्य॒ते स्वाहा॑ ऽऽसि॒ताया॑ सि॒ताय॒ स्वाहा॑ निपथ्स्य॒ते । \newline
6. स्वाहा॑ निपथ्स्य॒ते नि॑पथ्स्य॒ते स्वाहा॒ स्वाहा॑ निपथ्स्य॒ते स्वाहा॒ स्वाहा॑ निपथ्स्य॒ते स्वाहा॒ स्वाहा॑ निपथ्स्य॒ते स्वाहा᳚ । \newline
7. नि॒प॒थ्स्य॒ते स्वाहा॒ स्वाहा॑ निपथ्स्य॒ते नि॑पथ्स्य॒ते स्वाहा॑ नि॒पद्य॑मानाय नि॒पद्य॑मानाय॒ स्वाहा॑ निपथ्स्य॒ते नि॑पथ्स्य॒ते स्वाहा॑ नि॒पद्य॑मानाय । \newline
8. नि॒प॒थ्स्य॒त इति॑ नि - प॒थ्स्य॒ते । \newline
9. स्वाहा॑ नि॒पद्य॑मानाय नि॒पद्य॑मानाय॒ स्वाहा॒ स्वाहा॑ नि॒पद्य॑मानाय॒ स्वाहा॒ स्वाहा॑ नि॒पद्य॑मानाय॒ स्वाहा॒ स्वाहा॑ नि॒पद्य॑मानाय॒ स्वाहा᳚ । \newline
10. नि॒पद्य॑मानाय॒ स्वाहा॒ स्वाहा॑ नि॒पद्य॑मानाय नि॒पद्य॑मानाय॒ स्वाहा॒ निप॑न्नाय॒ निप॑न्नाय॒ स्वाहा॑ नि॒पद्य॑मानाय नि॒पद्य॑मानाय॒ स्वाहा॒ निप॑न्नाय । \newline
11. नि॒पद्य॑माना॒येति॑ नि - पद्य॑मानाय । \newline
12. स्वाहा॒ निप॑न्नाय॒ निप॑न्नाय॒ स्वाहा॒ स्वाहा॒ निप॑न्नाय॒ स्वाहा॒ स्वाहा॒ निप॑न्नाय॒ स्वाहा॒ स्वाहा॒ निप॑न्नाय॒ स्वाहा᳚ । \newline
13. निप॑न्नाय॒ स्वाहा॒ स्वाहा॒ निप॑न्नाय॒ निप॑न्नाय॒ स्वाहा॑ शयिष्य॒ते श॑यिष्य॒ते स्वाहा॒ निप॑न्नाय॒ निप॑न्नाय॒ स्वाहा॑ शयिष्य॒ते । \newline
14. निप॑न्ना॒येति॒ नि - प॒न्ना॒य॒ । \newline
15. स्वाहा॑ शयिष्य॒ते श॑यिष्य॒ते स्वाहा॒ स्वाहा॑ शयिष्य॒ते स्वाहा॒ स्वाहा॑ शयिष्य॒ते स्वाहा॒ स्वाहा॑ शयिष्य॒ते स्वाहा᳚ । \newline
16. श॒यि॒ष्य॒ते स्वाहा॒ स्वाहा॑ शयिष्य॒ते श॑यिष्य॒ते स्वाहा॒ शया॑नाय॒ शया॑नाय॒ स्वाहा॑ शयिष्य॒ते श॑यिष्य॒ते स्वाहा॒ शया॑नाय । \newline
17. स्वाहा॒ शया॑नाय॒ शया॑नाय॒ स्वाहा॒ स्वाहा॒ शया॑नाय॒ स्वाहा॒ स्वाहा॒ शया॑नाय॒ स्वाहा॒ स्वाहा॒ शया॑नाय॒ स्वाहा᳚ । \newline
18. शया॑नाय॒ स्वाहा॒ स्वाहा॒ शया॑नाय॒ शया॑नाय॒ स्वाहा॑ शयि॒ताय॑ शयि॒ताय॒ स्वाहा॒ शया॑नाय॒ शया॑नाय॒ स्वाहा॑ शयि॒ताय॑ । \newline
19. स्वाहा॑ शयि॒ताय॑ शयि॒ताय॒ स्वाहा॒ स्वाहा॑ शयि॒ताय॒ स्वाहा॒ स्वाहा॑ शयि॒ताय॒ स्वाहा॒ स्वाहा॑ शयि॒ताय॒ स्वाहा᳚ । \newline
20. श॒यि॒ताय॒ स्वाहा॒ स्वाहा॑ शयि॒ताय॑ शयि॒ताय॒ स्वाहा॑ सम्मीलिष्य॒ते स॑म्मीलिष्य॒ते स्वाहा॑ शयि॒ताय॑ शयि॒ताय॒ स्वाहा॑ सम्मीलिष्य॒ते । \newline
21. स्वाहा॑ सम्मीलिष्य॒ते स॑म्मीलिष्य॒ते स्वाहा॒ स्वाहा॑ सम्मीलिष्य॒ते स्वाहा॒ स्वाहा॑ सम्मीलिष्य॒ते स्वाहा॒ स्वाहा॑ सम्मीलिष्य॒ते स्वाहा᳚ । \newline
22. स॒म्मी॒लि॒ष्य॒ते स्वाहा॒ स्वाहा॑ सम्मीलिष्य॒ते स॑म्मीलिष्य॒ते स्वाहा॑ स॒म्मील॑ते स॒म्मील॑ते॒ स्वाहा॑ सम्मीलिष्य॒ते स॑म्मीलिष्य॒ते स्वाहा॑ स॒म्मील॑ते । \newline
23. स॒म्मी॒लि॒ष्य॒त इति॑ सं - मी॒लि॒ष्य॒ते । \newline
24. स्वाहा॑ स॒म्मील॑ते स॒म्मील॑ते॒ स्वाहा॒ स्वाहा॑ स॒म्मील॑ते॒ स्वाहा॒ स्वाहा॑ स॒म्मील॑ते॒ स्वाहा॒ स्वाहा॑ स॒म्मील॑ते॒ स्वाहा᳚ । \newline
25. स॒म्मील॑ते॒ स्वाहा॒ स्वाहा॑ स॒म्मील॑ते स॒म्मील॑ते॒ स्वाहा॒ सम्मी॑लिताय॒ सम्मी॑लिताय॒ स्वाहा॑ स॒म्मील॑ते स॒म्मील॑ते॒ स्वाहा॒ सम्मी॑लिताय । \newline
26. स॒म्मील॑त॒ इति॑ सं - मील॑ते । \newline
27. स्वाहा॒ सम्मी॑लिताय॒ सम्मी॑लिताय॒ स्वाहा॒ स्वाहा॒ सम्मी॑लिताय॒ स्वाहा॒ स्वाहा॒ सम्मी॑लिताय॒ स्वाहा॒ स्वाहा॒ सम्मी॑लिताय॒ स्वाहा᳚ । \newline
28. सम्मी॑लिताय॒ स्वाहा॒ स्वाहा॒ सम्मी॑लिताय॒ सम्मी॑लिताय॒ स्वाहा᳚ स्वफ्स्य॒ते स्व॑फ्स्य॒ते स्वाहा॒ सम्मी॑लिताय॒ सम्मी॑लिताय॒ स्वाहा᳚ स्वफ्स्य॒ते । \newline
29. सम्मी॑लिता॒येति॒ सं - मी॒लि॒ता॒य॒ । \newline
30. स्वाहा᳚ स्वफ्स्य॒ते स्व॑फ्स्य॒ते स्वाहा॒ स्वाहा᳚ स्वफ्स्य॒ते स्वाहा॒ स्वाहा᳚ स्वफ्स्य॒ते स्वाहा॒ स्वाहा᳚ स्वफ्स्य॒ते स्वाहा᳚ । \newline
31. स्व॒फ्स्य॒ते स्वाहा॒ स्वाहा᳚ स्वफ्स्य॒ते स्व॑फ्स्य॒ते स्वाहा᳚ स्वप॒ते स्व॑प॒ते स्वाहा᳚ स्वफ्स्य॒ते स्व॑फ्स्य॒ते स्वाहा᳚ स्वप॒ते । \newline
32. स्वाहा᳚ स्वप॒ते स्व॑प॒ते स्वाहा॒ स्वाहा᳚ स्वप॒ते स्वाहा॒ स्वाहा᳚ स्वप॒ते स्वाहा॒ स्वाहा᳚ स्वप॒ते स्वाहा᳚ । \newline
33. स्व॒प॒ते स्वाहा॒ स्वाहा᳚ स्वप॒ते स्व॑प॒ते स्वाहा॑ सु॒प्ताय॑ सु॒प्ताय॒ स्वाहा᳚ स्वप॒ते स्व॑प॒ते स्वाहा॑ सु॒प्ताय॑ । \newline
34. स्वाहा॑ सु॒प्ताय॑ सु॒प्ताय॒ स्वाहा॒ स्वाहा॑ सु॒प्ताय॒ स्वाहा॒ स्वाहा॑ सु॒प्ताय॒ स्वाहा॒ स्वाहा॑ सु॒प्ताय॒ स्वाहा᳚ । \newline
35. सु॒प्ताय॒ स्वाहा॒ स्वाहा॑ सु॒प्ताय॑ सु॒प्ताय॒ स्वाहा᳚ प्रभोथ्स्य॒ते प्र॑भोथ्स्य॒ते स्वाहा॑ सु॒प्ताय॑ सु॒प्ताय॒ स्वाहा᳚ प्रभोथ्स्य॒ते । \newline
36. स्वाहा᳚ प्रभोथ्स्य॒ते प्र॑भोथ्स्य॒ते स्वाहा॒ स्वाहा᳚ प्रभोथ्स्य॒ते स्वाहा॒ स्वाहा᳚ प्रभोथ्स्य॒ते स्वाहा॒ स्वाहा᳚ प्रभोथ्स्य॒ते स्वाहा᳚ । \newline
37. प्र॒भो॒थ्स्य॒ते स्वाहा॒ स्वाहा᳚ प्रभोथ्स्य॒ते प्र॑भोथ्स्य॒ते स्वाहा᳚ प्र॒बुद्ध्य॑मानाय प्र॒बुद्ध्य॑मानाय॒ स्वाहा᳚ प्रभोथ्स्य॒ते प्र॑भोथ्स्य॒ते स्वाहा᳚ प्र॒बुद्ध्य॑मानाय । \newline
38. प्र॒भो॒थ्स्य॒त इति॑ प्र - भो॒थ्स्य॒ते । \newline
39. स्वाहा᳚ प्र॒बुद्ध्य॑मानाय प्र॒बुद्ध्य॑मानाय॒ स्वाहा॒ स्वाहा᳚ प्र॒बुद्ध्य॑मानाय॒ स्वाहा॒ स्वाहा᳚ प्र॒बुद्ध्य॑मानाय॒ स्वाहा॒ स्वाहा᳚ प्र॒बुद्ध्य॑मानाय॒ स्वाहा᳚ । \newline
40. प्र॒बुद्ध्य॑मानाय॒ स्वाहा॒ स्वाहा᳚ प्र॒बुद्ध्य॑मानाय प्र॒बुद्ध्य॑मानाय॒ स्वाहा॒ प्रबु॑द्धाय॒ प्रबु॑द्धाय॒ स्वाहा᳚ प्र॒बुद्ध्य॑मानाय प्र॒बुद्ध्य॑मानाय॒ स्वाहा॒ प्रबु॑द्धाय । \newline
41. प्र॒बुद्ध्य॑माना॒येति॑ प्र - बुद्ध्य॑मानाय । \newline
42. स्वाहा॒ प्रबु॑द्धाय॒ प्रबु॑द्धाय॒ स्वाहा॒ स्वाहा॒ प्रबु॑द्धाय॒ स्वाहा॒ स्वाहा॒ प्रबु॑द्धाय॒ स्वाहा॒ स्वाहा॒ प्रबु॑द्धाय॒ स्वाहा᳚ । \newline
43. प्रबु॑द्धाय॒ स्वाहा॒ स्वाहा॒ प्रबु॑द्धाय॒ प्रबु॑द्धाय॒ स्वाहा॑ जागरिष्य॒ते जा॑गरिष्य॒ते स्वाहा॒ प्रबु॑द्धाय॒ प्रबु॑द्धाय॒ स्वाहा॑ जागरिष्य॒ते । \newline
44. प्रबु॑द्धा॒येति॒ प्र - बु॒द्धा॒य॒ । \newline
45. स्वाहा॑ जागरिष्य॒ते जा॑गरिष्य॒ते स्वाहा॒ स्वाहा॑ जागरिष्य॒ते स्वाहा॒ स्वाहा॑ जागरिष्य॒ते स्वाहा॒ स्वाहा॑ जागरिष्य॒ते स्वाहा᳚ । \newline
46. जा॒ग॒रि॒ष्य॒ते स्वाहा॒ स्वाहा॑ जागरिष्य॒ते जा॑गरिष्य॒ते स्वाहा॒ जाग्र॑ते॒ जाग्र॑ते॒ स्वाहा॑ जागरिष्य॒ते जा॑गरिष्य॒ते स्वाहा॒ जाग्र॑ते । \newline
47. स्वाहा॒ जाग्र॑ते॒ जाग्र॑ते॒ स्वाहा॒ स्वाहा॒ जाग्र॑ते॒ स्वाहा॒ स्वाहा॒ जाग्र॑ते॒ स्वाहा॒ स्वाहा॒ जाग्र॑ते॒ स्वाहा᳚ । \newline
48. जाग्र॑ते॒ स्वाहा॒ स्वाहा॒ जाग्र॑ते॒ जाग्र॑ते॒ स्वाहा॑ जागरि॒ताय॑ जागरि॒ताय॒ स्वाहा॒ जाग्र॑ते॒ जाग्र॑ते॒ स्वाहा॑ जागरि॒ताय॑ । \newline
49. स्वाहा॑ जागरि॒ताय॑ जागरि॒ताय॒ स्वाहा॒ स्वाहा॑ जागरि॒ताय॒ स्वाहा॒ स्वाहा॑ जागरि॒ताय॒ स्वाहा॒ स्वाहा॑ जागरि॒ताय॒ स्वाहा᳚ । \newline
50. जा॒ग॒रि॒ताय॒ स्वाहा॒ स्वाहा॑ जागरि॒ताय॑ जागरि॒ताय॒ स्वाहा॒ शुश्रू॑षमाणाय॒ शुश्रू॑षमाणाय॒ स्वाहा॑ जागरि॒ताय॑ जागरि॒ताय॒ स्वाहा॒ शुश्रू॑षमाणाय । \newline
51. स्वाहा॒ शुश्रू॑षमाणाय॒ शुश्रू॑षमाणाय॒ स्वाहा॒ स्वाहा॒ शुश्रू॑षमाणाय॒ स्वाहा॒ स्वाहा॒ शुश्रू॑षमाणाय॒ स्वाहा॒ स्वाहा॒ शुश्रू॑षमाणाय॒ स्वाहा᳚ । \newline
52. शुश्रू॑षमाणाय॒ स्वाहा॒ स्वाहा॒ शुश्रू॑षमाणाय॒ शुश्रू॑षमाणाय॒ स्वाहा॑ शृण्व॒ते शृ॑ण्व॒ते स्वाहा॒ शुश्रू॑षमाणाय॒ शुश्रू॑षमाणाय॒ स्वाहा॑ शृण्व॒ते । \newline
53. स्वाहा॑ शृण्व॒ते शृ॑ण्व॒ते स्वाहा॒ स्वाहा॑ शृण्व॒ते स्वाहा॒ स्वाहा॑ शृण्व॒ते स्वाहा॒ स्वाहा॑ शृण्व॒ते स्वाहा᳚ । \newline
54. शृ॒ण्व॒ते स्वाहा॒ स्वाहा॑ शृण्व॒ते शृ॑ण्व॒ते स्वाहा᳚ श्रु॒ताय॑ श्रु॒ताय॒ स्वाहा॑ शृण्व॒ते शृ॑ण्व॒ते स्वाहा᳚ श्रु॒ताय॑ । \newline
55. स्वाहा᳚ श्रु॒ताय॑ श्रु॒ताय॒ स्वाहा॒ स्वाहा᳚ श्रु॒ताय॒ स्वाहा॒ स्वाहा᳚ श्रु॒ताय॒ स्वाहा॒ स्वाहा᳚ श्रु॒ताय॒ स्वाहा᳚ । \newline
56. श्रु॒ताय॒ स्वाहा॒ स्वाहा᳚ श्रु॒ताय॑ श्रु॒ताय॒ स्वाहा॑ वीक्षिष्य॒ते वी᳚क्षिष्य॒ते स्वाहा᳚ श्रु॒ताय॑ श्रु॒ताय॒ स्वाहा॑ वीक्षिष्य॒ते । \newline
57. स्वाहा॑ वीक्षिष्य॒ते वी᳚क्षिष्य॒ते स्वाहा॒ स्वाहा॑ वीक्षिष्य॒ते स्वाहा॒ स्वाहा॑ वीक्षिष्य॒ते स्वाहा॒ स्वाहा॑ वीक्षिष्य॒ते स्वाहा᳚ । \newline
58. वी॒क्षि॒ष्य॒ते स्वाहा॒ स्वाहा॑ वीक्षिष्य॒ते वी᳚क्षिष्य॒ते स्वाहा॒ वीक्ष॑माणाय॒ वीक्ष॑माणाय॒ स्वाहा॑ वीक्षिष्य॒ते वी᳚क्षिष्य॒ते स्वाहा॒ वीक्ष॑माणाय । \newline
59. वी॒क्षि॒ष्य॒त इति॑ वि - ई॒क्षि॒ष्य॒ते । \newline
60. स्वाहा॒ वीक्ष॑माणाय॒ वीक्ष॑माणाय॒ स्वाहा॒ स्वाहा॒ वीक्ष॑माणाय॒ स्वाहा॒ स्वाहा॒ वीक्ष॑माणाय॒ स्वाहा॒ स्वाहा॒ वीक्ष॑माणाय॒ स्वाहा᳚ । \newline
\pagebreak
\markright{ TS 7.1.19.3  \hfill https://www.vedavms.in \hfill}

\section{ TS 7.1.19.3 }

\textbf{TS 7.1.19.3 } \newline
\textbf{Samhita Paata} \newline

वीक्ष॑माणाय॒ स्वाहा॒ वीक्षि॑ताय॒ स्वाहा॑ सꣳहास्य॒ते स्वाहा॑ स॒जिंहा॑नाय॒ स्वाहो॒-ज्जिहा॑नाय॒ स्वाहा॑ विवर्थ्स्य॒ते स्वाहा॑ वि॒वर्त॑मानाय॒ स्वाहा॒ विवृ॑त्ताय॒ स्वाहो᳚-त्थास्य॒ते स्वाहो॒त्तिष्ठ॑ते॒ स्वाहोत्थि॑ताय॒ स्वाहा॑ विधविष्य॒ते स्वाहा॑ विधून्वा॒नाय॒ स्वाहा॒ विधू॑ताय॒ स्वाहो᳚-त्क्रꣳस्य॒ते स्वाहो॒त्क्राम॑ते॒ स्वाहोत्क्रा᳚न्ताय॒ स्वाहा॑ चङ्क्रमिष्य॒ते स्वाहा॑ चङ्क्र॒म्यमा॑णाय॒ स्वाहा॑ चङ्क्रमि॒ताय॒ स्वाहा॑ कण्डूयिष्य॒ते स्वाहा॑ कण्डू॒यमा॑नाय॒ स्वाहा॑ कण्डूयि॒ताय॒ स्वाहा॑ निकषिष्य॒ते स्वाहा॑ नि॒कष॑माणाय॒ स्वाहा॒ ( ) निक॑षिताय॒ स्वाहा॒ यदत्ति॒ तस्मै॒ स्वाहा॒ यत् पिब॑ति॒ तस्मै॒ स्वाहा॒ यन्मेह॑ति॒ तस्मै॒ स्वाहा॒ यच्छकृ॑त् क॒रोति॒ तस्मै॒ स्वाहा॒ रेत॑से॒ स्वाहा᳚ प्र॒जाभ्यः॒ स्वाहा᳚ प्र॒जन॑नाय॒ स्वाहा॒ सर्व॑स्मै॒ स्वाहा᳚ ॥ \newline

\textbf{Pada Paata} \newline

वीक्ष॑माणा॒येति॑ वि-ईक्ष॑माणाय । स्वाहा᳚ । वीक्षि॑ता॒येति॒ वि-ई॒क्षि॒ता॒य॒ । स्वाहा᳚ । सꣳ॒॒हा॒स्य॒त इति॑ सं - हा॒स्य॒ते । स्वाहा᳚ । स॒जिंहा॑ना॒येति॑ सं - जिहा॑नाय । स्वाहा᳚ । उ॒ज्जिहा॑ना॒येत्यु॑त्- जिहा॑नाय । स्वाहा᳚ । वि॒व॒र्थ्स्य॒त इति॑ वि - व॒र्थ्स्य॒ते । स्वाहा᳚ । वि॒वर्त॑माना॒येति॑ वि - वर्त॑मानाय । स्वाहा᳚ । विवृ॑त्ता॒येति॒ वि-वृ॒त्ता॒य॒ । स्वाहा᳚ । उ॒त्था॒स्य॒त इत्यु॑त् - स्था॒स्य॒ते । स्वाहा᳚ । उ॒त्तिष्ठ॑त॒ इत्यु॑त्- तिष्ठ॑ते । स्वाहा᳚ । उत्थि॑ता॒येत्युत् - स्थि॒ता॒य॒ । स्वाहा᳚ । वि॒ध॒वि॒ष्य॒त इति॑ वि - ध॒वि॒ष्य॒ते । स्वाहा᳚ । वि॒धू॒न्वा॒नायेति॑ वि - धू॒न्वा॒नाय॑ । स्वाहा᳚ । विधू॑ता॒येति॒ वि - धू॒ता॒य॒ । स्वाहा᳚ । उ॒त्क्रꣳ॒॒स्य॒त इत्यु॑त्- क्रꣳ॒॒स्य॒ते । स्वाहा᳚ । उ॒त्क्राम॑त॒ इत्यु॑त् - क्राम॑ते । स्वाहा᳚ । उत्क्रा᳚न्ता॒येत्युत् - क्रा॒न्ता॒य॒ । स्वाहा᳚ । च॒ङ्क्र॒मि॒ष्य॒ते । स्वाहा᳚ । च॒ङ्क्र॒म्यमा॑णाय । स्वाहा᳚ । च॒ङ्क्र॒मि॒ताय॑ । स्वाहा᳚ । क॒ण्डू॒यि॒ष्य॒ते । स्वाहा᳚ । क॒ण्डू॒यमा॑नाय । स्वाहा᳚ । क॒ण्डू॒यि॒ताय॑ । स्वाहा᳚ । नि॒क॒षि॒ष्य॒त इति॑ नि - क॒षि॒ष्य॒ते । स्वाहा᳚ । नि॒कष॑माणा॒येति॑ नि - कष॑माणाय । स्वाहा᳚ ( ) । निक॑षिता॒येति॒ नि - क॒षि॒ता॒य॒ । स्वाहा᳚ । यत् । अत्ति॑ । तस्मै᳚ । स्वाहा᳚ । यत् । पिब॑ति । तस्मै᳚ । स्वाहा᳚ । यत् । मेह॑ति । तस्मै᳚ । स्वाहा᳚ । यत् । शकृ॑त् । क॒रोति॑ । तस्मै᳚ । स्वाहा᳚ । रेत॑से । स्वाहा᳚ । प्र॒जाभ्य॒ इति॑ प्र - जाभ्यः॑ । स्वाहा᳚ । प्र॒जन॑ना॒येति॑ प्र - जन॑नाय । स्वाहा᳚ । सर्व॑स्मै । स्वाहा᳚ ॥  \newline


\textbf{Krama Paata} \newline

वीक्ष॑माणाय॒ स्वाहा᳚ । वीक्ष॑माणा॒येति॑ वि - ईक्ष॑माणाय । स्वाहा॒ वीक्षि॑ताय । वीक्षि॑ताय॒ स्वाहा᳚ । वीक्षि॑ता॒येति॒ वि - ई॒क्षि॒ता॒य॒ । स्वाहा॑ सꣳहास्य॒ते । सꣳ॒॒हा॒स्य॒ते स्वाहा᳚ । सꣳ॒॒हा॒स्य॒त इति॑ सम् - हा॒स्य॒ते । स्वाहा॑ स॒ञ्जिहा॑नाय । स॒ञ्जिहा॑नाय॒ स्वाहा᳚ । स॒ञ्जिहा॑ना॒येति॑ सम् - जिहा॑नाय । स्वाहो॒ज्जिहा॑नाय । उ॒ज्जिहा॑नाय॒ स्वाहा᳚ । उ॒ज्जिहा॑ना॒येत्यु॑त् - जिहा॑नाय । स्वाहा॑ विवर्थ्स्य॒ते । वि॒व॒र्थ्स्य॒ते स्वाहा᳚ । वि॒व॒र्थ्स्य॒त इति॑ वि - व॒र्थ्स्य॒ते । स्वाहा॑ वि॒वर्त॑मानाय । वि॒वर्त॑मानाय॒ स्वाहा᳚ । वि॒वर्त॑माना॒येति॑ वि - वर्त॑मानाय । स्वाहा॒ विवृ॑त्ताय । विवृ॑त्ताय॒ स्वाहा᳚ । विवृ॑त्ता॒येति॒ वि - वृ॒त्ता॒य॒ । स्वाहो᳚त्थास्य॒ते । उ॒त्था॒स्य॒ते स्वाहा᳚ । उ॒त्था॒स्य॒त इत्यु॑त् - स्था॒स्य॒ते । स्वाहो॒त्तिष्ठ॑ते । उ॒त्तिष्ठ॑ते॒ स्वाहा᳚ । उ॒त्तिष्ठ॑त॒ इत्यु॑त् - तिष्ठ॑ते । स्वाहोत्थि॑ताय । उत्थि॑ताय॒ स्वाहा᳚ । उत्थि॑ता॒येत्युत् - स्थि॒ता॒य॒ । स्वाहा॑ विधविष्य॒ते । वि॒ध॒वि॒ष्य॒ते स्वाहा᳚ । वि॒ध॒वि॒ष्य॒त इति॑ वि - ध॒वि॒ष्य॒ते । स्वाहा॑ विधून्वा॒नाय॑ । वि॒धू॒न्वा॒नाय॒ स्वाहा᳚ । वि॒धू॒न्वा॒नायेति॑ वि - धू॒न्वा॒नाय॑ । स्वाहा॒ विधू॑ताय । विधू॑ताय॒ स्वाहा᳚ । विधू॑ता॒येति॒ वि - धू॒ता॒य॒ । स्वाहो᳚त्क्रꣳस्य॒ते । उ॒त्क्रꣳ॒॒स्य॒ते स्वाहा᳚ । उ॒त्क्रꣳ॒॒स्य॒त इत्यु॑त् - क्रꣳ॒॒स्य॒ते । स्वाहो॒त्क्राम॑ते । उ॒त्क्राम॑ते॒ स्वाहा᳚ । उ॒त्क्राम॑त॒ इत्यु॑त् - क्राम॑ते । स्वाहोत्क्रा᳚न्ताय । उत्क्रा᳚न्ताय॒ स्वाहा᳚ । उत्क्रा᳚न्ता॒येत्युत् - क्रा॒न्ता॒य॒ । स्वाहा॑ चङ्‍क्रमिष्य॒ते । च॒ङ्‍क्र॒मि॒ष्य॒ते स्वाहा᳚ । स्वाहा॑ चङ्‍क्र॒म्यमा॑णाय । च॒ङ्‍क्र॒म्यमा॑णाय॒ स्वाहा᳚ । स्वाहा॑ चङ्‍क्रमि॒ताय॑ । च॒ङ्‍क्र॒मि॒ताय॒ स्वाहा᳚ । स्वाहा॑ कण्डूयिष्य॒ते । क॒ण्डू॒यि॒ष्य॒ते स्वाहा᳚ । स्वाहा॑ कण्डू॒यमा॑नाय । क॒ण्डू॒यमा॑नाय॒ स्वाहा᳚ । स्वाहा॑ कण्डूयि॒ताय॑ । क॒ण्डू॒यि॒ताय॒ स्वाहा᳚ । स्वाहा॑ निकषिष्य॒ते । नि॒क॒षि॒ष्य॒ते स्वाहा᳚ । नि॒क॒षि॒ष्य॒त इति॑ नि - क॒षि॒ष्य॒ते । स्वाहा॑ नि॒कष॑माणाय ( ) । नि॒कष॑माणाय॒ स्वाहा᳚ । नि॒कष॑माणा॒येति॑ नि - कष॑माणाय । स्वाहा॒ निक॑षिताय । निक॑षिताय॒ स्वाहा᳚ । निक॑षिता॒येति॒ नि - क॒षि॒ता॒य॒ । स्वाहा॒ यत् । यदत्ति॑ । अत्ति॒ तस्मै᳚ । तस्मै॒ स्वाहा᳚ । स्वाहा॒ यत् । यत् पिब॑ति । पिब॑ति॒ तस्मै᳚ । तस्मै॒ स्वाहा᳚ । स्वाहा॒ यत् । यन् मेह॑ति । मेह॑ति॒ तस्मै᳚ । तस्मै॒ स्वाहा᳚ । स्वाहा॒ यत् । यच्छकृ॑त् । शकृ॑त् क॒रोति॑ । क॒रोति॒ तस्मै᳚ । तस्मै॒ स्वाहा᳚ । स्वाहा॒ रेत॑से । रेत॑से॒ स्वाहा᳚ । स्वाहा᳚ प्र॒जाभ्यः॑ । प्र॒जाभ्यः॒ स्वाहा᳚ । प्र॒जाभ्य॒ इति॑ प्र - जाभ्यः॑ । स्वाहा᳚ प्र॒जन॑नाय । प्र॒जन॑नाय॒ स्वाहा᳚ । 
प्र॒जन॑ना॒येति॑ प्र - जन॑नाय । स्वाहा॒ सर्व॑स्मै । सर्व॑स्मै॒ स्वाहा᳚ । स्वाहेति॒ स्वाहा᳚ । \newline

\textbf{Jatai Paata} \newline

1. वीक्ष॑माणाय॒ स्वाहा॒ स्वाहा॒ वीक्ष॑माणाय॒ वीक्ष॑माणाय॒ स्वाहा᳚ । \newline
2. वीक्ष॑माणा॒येति॑ वि - ईक्ष॑माणाय । \newline
3. स्वाहा॒ वीक्षि॑ताय॒ वीक्षि॑ताय॒ स्वाहा॒ स्वाहा॒ वीक्षि॑ताय । \newline
4. वीक्षि॑ताय॒ स्वाहा॒ स्वाहा॒ वीक्षि॑ताय॒ वीक्षि॑ताय॒ स्वाहा᳚ । \newline
5. वीक्षि॑ता॒येति॒ वि - ई॒क्षि॒ता॒य॒ । \newline
6. स्वाहा॑ सꣳहास्य॒ते सꣳ॑हास्य॒ते स्वाहा॒ स्वाहा॑ सꣳहास्य॒ते । \newline
7. सꣳ॒॒हा॒स्य॒ते स्वाहा॒ स्वाहा॑ सꣳहास्य॒ते सꣳ॑हास्य॒ते स्वाहा᳚ । \newline
8. सꣳ॒॒हा॒स्य॒त इति॑ सं - हा॒स्य॒ते । \newline
9. स्वाहा॑ स॒ञ्जिहा॑नाय स॒ञ्जिहा॑नाय॒ स्वाहा॒ स्वाहा॑ स॒ञ्जिहा॑नाय । \newline
10. स॒ञ्जिहा॑नाय॒ स्वाहा॒ स्वाहा॑ स॒ञ्जिहा॑नाय स॒ञ्जिहा॑नाय॒ स्वाहा᳚ । \newline
11. स॒ञ्जिहा॑ना॒येति॑ सं - जिहा॑नाय । \newline
12. स्वाहो॒ ज्जिहा॑नायो॒ ज्जिहा॑नाय॒ स्वाहा॒ स्वाहो॒ ज्जिहा॑नाय । \newline
13. उ॒ज्जिहा॑नाय॒ स्वाहा॒ स्वाहो॒ ज्जिहा॑नायो॒ ज्जिहा॑नाय॒ स्वाहा᳚ । \newline
14. उ॒ज्जिहा॑ना॒येत्यु॑त् - जिहा॑नाय । \newline
15. स्वाहा॑ विवर्थ्स्य॒ते वि॑वर्थ्स्य॒ते स्वाहा॒ स्वाहा॑ विवर्थ्स्य॒ते । \newline
16. वि॒व॒र्थ्स्य॒ते स्वाहा॒ स्वाहा॑ विवर्थ्स्य॒ते वि॑वर्थ्स्य॒ते स्वाहा᳚ । \newline
17. वि॒व॒र्थ्स्य॒त इति॑ वि - व॒र्थ्स्य॒ते । \newline
18. स्वाहा॑ वि॒वर्त॑मानाय वि॒वर्त॑मानाय॒ स्वाहा॒ स्वाहा॑ वि॒वर्त॑मानाय । \newline
19. वि॒वर्त॑मानाय॒ स्वाहा॒ स्वाहा॑ वि॒वर्त॑मानाय वि॒वर्त॑मानाय॒ स्वाहा᳚ । \newline
20. वि॒वर्त॑माना॒येति॑ वि - वर्त॑मानाय । \newline
21. स्वाहा॒ विवृ॑त्ताय॒ विवृ॑त्ताय॒ स्वाहा॒ स्वाहा॒ विवृ॑त्ताय । \newline
22. विवृ॑त्ताय॒ स्वाहा॒ स्वाहा॒ विवृ॑त्ताय॒ विवृ॑त्ताय॒ स्वाहा᳚ । \newline
23. विवृ॑त्ता॒येति॒ वि - वृ॒त्ता॒य॒ । \newline
24. स्वाहो᳚ त्थास्य॒त उ॑त्थास्य॒ते स्वाहा॒ स्वाहो᳚ त्थास्य॒ते । \newline
25. उ॒त्था॒स्य॒ते स्वाहा॒ स्वाहो᳚ त्थास्य॒त उ॑त्थास्य॒ते स्वाहा᳚ । \newline
26. उ॒त्था॒स्य॒त इत्यु॑त् - स्था॒स्य॒ते । \newline
27. स्वाहो॒ त्तिष्ठ॑त उ॒त्तिष्ठ॑ते॒ स्वाहा॒ स्वाहो॒ त्तिष्ठ॑ते । \newline
28. उ॒त्तिष्ठ॑ते॒ स्वाहा॒ स्वाहो॒त्तिष्ठ॑त उ॒त्तिष्ठ॑ते॒ स्वाहा᳚ । \newline
29. उ॒त्तिष्ठ॑त॒ इत्यु॑त् - तिष्ठ॑ते । \newline
30. स्वाहोत्थि॑ता॒ योत्थि॑ताय॒ स्वाहा॒ स्वाहोत्थि॑ताय । \newline
31. उत्थि॑ताय॒ स्वाहा॒ स्वाहोत्थि॑ता॒ योत्थि॑ताय॒ स्वाहा᳚ । \newline
32. उत्थि॑ता॒येत्युत् - स्थि॒ता॒य॒ । \newline
33. स्वाहा॑ विधविष्य॒ते वि॑धविष्य॒ते स्वाहा॒ स्वाहा॑ विधविष्य॒ते । \newline
34. वि॒ध॒वि॒ष्य॒ते स्वाहा॒ स्वाहा॑ विधविष्य॒ते वि॑धविष्य॒ते स्वाहा᳚ । \newline
35. वि॒ध॒वि॒ष्य॒त इति॑ वि - ध॒वि॒ष्य॒ते । \newline
36. स्वाहा॑ विधून्वा॒नाय॑ विधून्वा॒नाय॒ स्वाहा॒ स्वाहा॑ विधून्वा॒नाय॑ । \newline
37. वि॒धू॒न्वा॒नाय॒ स्वाहा॒ स्वाहा॑ विधून्वा॒नाय॑ विधून्वा॒नाय॒ स्वाहा᳚ । \newline
38. वि॒धू॒न्वा॒नायेति॑ वि - धू॒न्वा॒नाय॑ । \newline
39. स्वाहा॒ विधू॑ताय॒ विधू॑ताय॒ स्वाहा॒ स्वाहा॒ विधू॑ताय । \newline
40. विधू॑ताय॒ स्वाहा॒ स्वाहा॒ विधू॑ताय॒ विधू॑ताय॒ स्वाहा᳚ । \newline
41. विधू॑ता॒येति॒ वि - धू॒ता॒य॒ । \newline
42. स्वाहो᳚ त्क्रꣳस्य॒त उ॑त्क्रꣳस्य॒ते स्वाहा॒ स्वाहो᳚ त्क्रꣳस्य॒ते । \newline
43. उ॒त्क्रꣳ॒॒स्य॒ते स्वाहा॒ स्वाहो᳚ त्क्रꣳस्य॒त उ॑त्क्रꣳस्य॒ते स्वाहा᳚ । \newline
44. उ॒त्क्रꣳ॒॒स्य॒त इत्यु॑त् - क्रꣳ॒॒स्य॒ते । \newline
45. स्वाहो॒त्क्राम॑त उ॒त्क्राम॑ते॒ स्वाहा॒ स्वाहो॒त्क्राम॑ते । \newline
46. उ॒त्क्राम॑ते॒ स्वाहा॒ स्वाहो॒त्क्राम॑त उ॒त्क्राम॑ते॒ स्वाहा᳚ । \newline
47. उ॒त्क्राम॑त॒ इत्यु॑त् - क्राम॑ते । \newline
48. स्वाहोत्क्रा᳚न्ता॒ योत्क्रा᳚न्ताय॒ स्वाहा॒ स्वाहोत्क्रा᳚न्ताय । \newline
49. उत्क्रा᳚न्ताय॒ स्वाहा॒ स्वाहोत्क्रा᳚न्ता॒ योत्क्रा᳚न्ताय॒ स्वाहा᳚ । \newline
50. उत्क्रा᳚न्ता॒येत्युत् - क्रा॒न्ता॒य॒ । \newline
51. स्वाहा॑ चङ्क्रमिष्य॒ते च॑ङ्क्रमिष्य॒ते स्वाहा॒ स्वाहा॑ चङ्क्रमिष्य॒ते । \newline
52. च॒ङ्क्र॒मि॒ष्य॒ते स्वाहा॒ स्वाहा॑ चङ्क्रमिष्य॒ते च॑ङ्क्रमिष्य॒ते स्वाहा᳚ । \newline
53. स्वाहा॑ चङ्क्र॒म्यमा॑णाय चङ्क्र॒म्यमा॑णाय॒ स्वाहा॒ स्वाहा॑ चङ्क्र॒म्यमा॑णाय । \newline
54. च॒ङ्क्र॒म्यमा॑णाय॒ स्वाहा॒ स्वाहा॑ चङ्क्र॒म्यमा॑णाय चङ्क्र॒म्यमा॑णाय॒ स्वाहा᳚ । \newline
55. स्वाहा॑ चङ्क्रमि॒ताय॑ चङ्क्रमि॒ताय॒ स्वाहा॒ स्वाहा॑ चङ्क्रमि॒ताय॑ । \newline
56. च॒ङ्क्र॒मि॒ताय॒ स्वाहा॒ स्वाहा॑ चङ्क्रमि॒ताय॑ चङ्क्रमि॒ताय॒ स्वाहा᳚ । \newline
57. स्वाहा॑ कण्डूयिष्य॒ते क॑ण्डूयिष्य॒ते स्वाहा॒ स्वाहा॑ कण्डूयिष्य॒ते । \newline
58. क॒ण्डू॒यि॒ष्य॒ते स्वाहा॒ स्वाहा॑ कण्डूयिष्य॒ते क॑ण्डूयिष्य॒ते स्वाहा᳚ । \newline
59. स्वाहा॑ कण्डू॒यमा॑नाय कण्डू॒यमा॑नाय॒ स्वाहा॒ स्वाहा॑ कण्डू॒यमा॑नाय । \newline
60. क॒ण्डू॒यमा॑नाय॒ स्वाहा॒ स्वाहा॑ कण्डू॒यमा॑नाय कण्डू॒यमा॑नाय॒ स्वाहा᳚ । \newline
61. स्वाहा॑ कण्डूयि॒ताय॑ कण्डूयि॒ताय॒ स्वाहा॒ स्वाहा॑ कण्डूयि॒ताय॑ । \newline
62. क॒ण्डू॒यि॒ताय॒ स्वाहा॒ स्वाहा॑ कण्डूयि॒ताय॑ कण्डूयि॒ताय॒ स्वाहा᳚ । \newline
63. स्वाहा॑ निकषिष्य॒ते नि॑कषिष्य॒ते स्वाहा॒ स्वाहा॑ निकषिष्य॒ते । \newline
64. नि॒क॒षि॒ष्य॒ते स्वाहा॒ स्वाहा॑ निकषिष्य॒ते नि॑कषिष्य॒ते स्वाहा᳚ । \newline
65. नि॒क॒षि॒ष्य॒त इति॑ नि - क॒षि॒ष्य॒ते । \newline
66. स्वाहा॑ नि॒कष॑माणाय नि॒कष॑माणाय॒ स्वाहा॒ स्वाहा॑ नि॒कष॑माणाय । \newline
67. नि॒कष॑माणाय॒ स्वाहा॒ स्वाहा॑ नि॒कष॑माणाय नि॒कष॑माणाय॒ स्वाहा᳚ । \newline
68. नि॒कष॑माणा॒येति॑ नि - कष॑माणाय । \newline
69. स्वाहा॒ निक॑षिताय॒ निक॑षिताय॒ स्वाहा॒ स्वाहा॒ निक॑षिताय । \newline
70. निक॑षिताय॒ स्वाहा॒ स्वाहा॒ निक॑षिताय॒ निक॑षिताय॒ स्वाहा᳚ । \newline
71. निक॑षिता॒येति॒ नि - क॒षि॒ता॒य॒ । \newline
72. स्वाहा॒ यद् यथ् स्वाहा॒ स्वाहा॒ यत् । \newline
73. यद त्त्यत्ति॒ यद् यदत्ति॑ । \newline
74. अत्ति॒ तस्मै॒ तस्मा॒ अत्त्यत्ति॒ तस्मै᳚ । \newline
75. तस्मै॒ स्वाहा॒ स्वाहा॒ तस्मै॒ तस्मै॒ स्वाहा᳚ । \newline
76. स्वाहा॒ यद् यथ् स्वाहा॒ स्वाहा॒ यत् । \newline
77. यत् पिब॑ति॒ पिब॑ति॒ यद् यत् पिब॑ति । \newline
78. पिब॑ति॒ तस्मै॒ तस्मै॒ पिब॑ति॒ पिब॑ति॒ तस्मै᳚ । \newline
79. तस्मै॒ स्वाहा॒ स्वाहा॒ तस्मै॒ तस्मै॒ स्वाहा᳚ । \newline
80. स्वाहा॒ यद् यथ् स्वाहा॒ स्वाहा॒ यत् । \newline
81. यन् मेह॑ति॒ मेह॑ति॒ यद् यन् मेह॑ति । \newline
82. मेह॑ति॒ तस्मै॒ तस्मै॒ मेह॑ति॒ मेह॑ति॒ तस्मै᳚ । \newline
83. तस्मै॒ स्वाहा॒ स्वाहा॒ तस्मै॒ तस्मै॒ स्वाहा᳚ । \newline
84. स्वाहा॒ यद् यथ् स्वाहा॒ स्वाहा॒ यत् । \newline
85. यच्छकृ॒च् छकृ॒द् यद् यच्छकृ॑त् । \newline
86. शकृ॑त् क॒रोति॑ क॒रोति॒ शकृ॒च् छकृ॑त् क॒रोति॑ । \newline
87. क॒रोति॒ तस्मै॒ तस्मै॑ क॒रोति॑ क॒रोति॒ तस्मै᳚ । \newline
88. तस्मै॒ स्वाहा॒ स्वाहा॒ तस्मै॒ तस्मै॒ स्वाहा᳚ । \newline
89. स्वाहा॒ रेत॑से॒ रेत॑से॒ स्वाहा॒ स्वाहा॒ रेत॑से । \newline
90. रेत॑से॒ स्वाहा॒ स्वाहा॒ रेत॑से॒ रेत॑से॒ स्वाहा᳚ । \newline
91. स्वाहा᳚ प्र॒जाभ्यः॑ प्र॒जाभ्यः॒ स्वाहा॒ स्वाहा᳚ प्र॒जाभ्यः॑ । \newline
92. प्र॒जाभ्यः॒ स्वाहा॒ स्वाहा᳚ प्र॒जाभ्यः॑ प्र॒जाभ्यः॒ स्वाहा᳚ । \newline
93. प्र॒जाभ्य॒ इति॑ प्र - जाभ्यः॑ । \newline
94. स्वाहा᳚ प्र॒जन॑नाय प्र॒जन॑नाय॒ स्वाहा॒ स्वाहा᳚ प्र॒जन॑नाय । \newline
95. प्र॒जन॑नाय॒ स्वाहा॒ स्वाहा᳚ प्र॒जन॑नाय प्र॒जन॑नाय॒ स्वाहा᳚ । \newline
96. प्र॒जन॑ना॒येति॑ प्र - जन॑नाय । \newline
97. स्वाहा॒ सर्व॑स्मै॒ सर्व॑स्मै॒ स्वाहा॒ स्वाहा॒ सर्व॑स्मै । \newline
98. सर्व॑स्मै॒ स्वाहा॒ स्वाहा॒ सर्व॑स्मै॒ सर्व॑स्मै॒ स्वाहा᳚ । \newline
99. स्वाहेति॒ स्वाहा᳚ । \newline

\textbf{Ghana Paata } \newline

1. वीक्ष॑माणाय॒ स्वाहा॒ स्वाहा॒ वीक्ष॑माणाय॒ वीक्ष॑माणाय॒ स्वाहा॒ वीक्षि॑ताय॒ वीक्षि॑ताय॒ स्वाहा॒ वीक्ष॑माणाय॒ वीक्ष॑माणाय॒ स्वाहा॒ वीक्षि॑ताय । \newline
2. वीक्ष॑माणा॒येति॑ वि - ईक्ष॑माणाय । \newline
3. स्वाहा॒ वीक्षि॑ताय॒ वीक्षि॑ताय॒ स्वाहा॒ स्वाहा॒ वीक्षि॑ताय॒ स्वाहा॒ स्वाहा॒ वीक्षि॑ताय॒ स्वाहा॒ स्वाहा॒ वीक्षि॑ताय॒ स्वाहा᳚ । \newline
4. वीक्षि॑ताय॒ स्वाहा॒ स्वाहा॒ वीक्षि॑ताय॒ वीक्षि॑ताय॒ स्वाहा॑ सꣳहास्य॒ते सꣳ॑हास्य॒ते स्वाहा॒ वीक्षि॑ताय॒ वीक्षि॑ताय॒ स्वाहा॑ सꣳहास्य॒ते । \newline
5. वीक्षि॑ता॒येति॒ वि - ई॒क्षि॒ता॒य॒ । \newline
6. स्वाहा॑ सꣳहास्य॒ते सꣳ॑हास्य॒ते स्वाहा॒ स्वाहा॑ सꣳहास्य॒ते स्वाहा॒ स्वाहा॑ सꣳहास्य॒ते स्वाहा॒ स्वाहा॑ सꣳहास्य॒ते स्वाहा᳚ । \newline
7. सꣳ॒॒हा॒स्य॒ते स्वाहा॒ स्वाहा॑ सꣳहास्य॒ते सꣳ॑हास्य॒ते स्वाहा॑ स॒ञ्जिहा॑नाय स॒ञ्जिहा॑नाय॒ स्वाहा॑ सꣳहास्य॒ते सꣳ॑हास्य॒ते स्वाहा॑ स॒ञ्जिहा॑नाय । \newline
8. सꣳ॒॒हा॒स्य॒त इति॑ सं - हा॒स्य॒ते । \newline
9. स्वाहा॑ स॒ञ्जिहा॑नाय स॒ञ्जिहा॑नाय॒ स्वाहा॒ स्वाहा॑ स॒ञ्जिहा॑नाय॒ स्वाहा॒ स्वाहा॑ स॒ञ्जिहा॑नाय॒ स्वाहा॒ स्वाहा॑ स॒ञ्जिहा॑नाय॒ स्वाहा᳚ । \newline
10. स॒ञ्जिहा॑नाय॒ स्वाहा॒ स्वाहा॑ स॒ञ्जिहा॑नाय स॒ञ्जिहा॑नाय॒ स्वाहो॒ज्जिहा॑ना यो॒ज्जिहा॑नाय॒ स्वाहा॑ स॒ञ्जिहा॑नाय स॒ञ्जिहा॑नाय॒ स्वाहो॒ज्जिहा॑नाय । \newline
11. स॒ञ्जिहा॑ना॒येति॑ सं - जिहा॑नाय । \newline
12. स्वाहो॒ज्जिहा॑ना यो॒ज्जिहा॑नाय॒ स्वाहा॒ स्वाहो॒ज्जिहा॑नाय॒ स्वाहा॒ स्वाहो॒ज्जिहा॑नाय॒ स्वाहा॒ स्वाहो॒ज्जिहा॑नाय॒ स्वाहा᳚ । \newline
13. उ॒ज्जिहा॑नाय॒ स्वाहा॒ स्वाहो॒ज्जिहा॑ना यो॒ज्जिहा॑नाय॒ स्वाहा॑ विवर्थ्स्य॒ते वि॑वर्थ्स्य॒ते स्वाहो॒ज्जिहा॑ना यो॒ज्जिहा॑नाय॒ स्वाहा॑ विवर्थ्स्य॒ते । \newline
14. उ॒ज्जिहा॑ना॒येत्यु॑त् - जिहा॑नाय । \newline
15. स्वाहा॑ विवर्थ्स्य॒ते वि॑वर्थ्स्य॒ते स्वाहा॒ स्वाहा॑ विवर्थ्स्य॒ते स्वाहा॒ स्वाहा॑ विवर्थ्स्य॒ते स्वाहा॒ स्वाहा॑ विवर्थ्स्य॒ते स्वाहा᳚ । \newline
16. वि॒व॒र्थ्स्य॒ते स्वाहा॒ स्वाहा॑ विवर्थ्स्य॒ते वि॑वर्थ्स्य॒ते स्वाहा॑ वि॒वर्त॑मानाय वि॒वर्त॑मानाय॒ स्वाहा॑ विवर्थ्स्य॒ते वि॑वर्थ्स्य॒ते स्वाहा॑ वि॒वर्त॑मानाय । \newline
17. वि॒व॒र्थ्स्य॒त इति॑ वि - व॒र्थ्स्य॒ते । \newline
18. स्वाहा॑ वि॒वर्त॑मानाय वि॒वर्त॑मानाय॒ स्वाहा॒ स्वाहा॑ वि॒वर्त॑मानाय॒ स्वाहा॒ स्वाहा॑ वि॒वर्त॑मानाय॒ स्वाहा॒ स्वाहा॑ वि॒वर्त॑मानाय॒ स्वाहा᳚ । \newline
19. वि॒वर्त॑मानाय॒ स्वाहा॒ स्वाहा॑ वि॒वर्त॑मानाय वि॒वर्त॑मानाय॒ स्वाहा॒ विवृ॑त्ताय॒ विवृ॑त्ताय॒ स्वाहा॑ वि॒वर्त॑मानाय वि॒वर्त॑मानाय॒ स्वाहा॒ विवृ॑त्ताय । \newline
20. वि॒वर्त॑माना॒येति॑ वि - वर्त॑मानाय । \newline
21. स्वाहा॒ विवृ॑त्ताय॒ विवृ॑त्ताय॒ स्वाहा॒ स्वाहा॒ विवृ॑त्ताय॒ स्वाहा॒ स्वाहा॒ विवृ॑त्ताय॒ स्वाहा॒ स्वाहा॒ विवृ॑त्ताय॒ स्वाहा᳚ । \newline
22. विवृ॑त्ताय॒ स्वाहा॒ स्वाहा॒ विवृ॑त्ताय॒ विवृ॑त्ताय॒ स्वाहो᳚त्थास्य॒त उ॑त्थास्य॒ते स्वाहा॒ विवृ॑त्ताय॒ विवृ॑त्ताय॒ स्वाहो᳚त्थास्य॒ते । \newline
23. विवृ॑त्ता॒येति॒ वि - वृ॒त्ता॒य॒ । \newline
24. स्वाहो᳚त्थास्य॒त उ॑त्थास्य॒ते स्वाहा॒ स्वाहो᳚त्थास्य॒ते स्वाहा॒ स्वाहो᳚त्थास्य॒ते स्वाहा॒ स्वाहो᳚त्थास्य॒ते स्वाहा᳚ । \newline
25. उ॒त्था॒स्य॒ते स्वाहा॒ स्वाहो᳚त्थास्य॒त उ॑त्थास्य॒ते स्वाहो॒त्तिष्ठ॑त उ॒त्तिष्ठ॑ते॒ स्वाहो᳚त्थास्य॒त उ॑त्थास्य॒ते स्वाहो॒त्तिष्ठ॑ते । \newline
26. उ॒त्था॒स्य॒त इत्यु॑त् - स्था॒स्य॒ते । \newline
27. स्वाहो॒त्तिष्ठ॑त उ॒त्तिष्ठ॑ते॒ स्वाहा॒ स्वाहो॒त्तिष्ठ॑ते॒ स्वाहा॒ स्वाहो॒त्तिष्ठ॑ते॒ स्वाहा॒ स्वाहो॒त्तिष्ठ॑ते॒ स्वाहा᳚ । \newline
28. उ॒त्तिष्ठ॑ते॒ स्वाहा॒ स्वाहो॒त्तिष्ठ॑त उ॒त्तिष्ठ॑ते॒ स्वाहोत्थि॑ता॒ योत्थि॑ताय॒ स्वाहो॒त्तिष्ठ॑त उ॒त्तिष्ठ॑ते॒ 
स्वाहोत्थि॑ताय । \newline
29. उ॒त्तिष्ठ॑त॒ इत्यु॑त् - तिष्ठ॑ते । \newline
30. स्वाहोत्थि॑ता॒ योत्थि॑ताय॒ स्वाहा॒ स्वाहोत्थि॑ताय॒ स्वाहा॒ स्वाहोत्थि॑ताय॒ स्वाहा॒ स्वाहोत्थि॑ताय॒ स्वाहा᳚ । \newline
31. उत्थि॑ताय॒ स्वाहा॒ स्वाहोत्थि॑ता॒ योत्थि॑ताय॒ स्वाहा॑ विधविष्य॒ते वि॑धविष्य॒ते स्वाहोत्थि॑ता॒ योत्थि॑ताय॒ स्वाहा॑ विधविष्य॒ते । \newline
32. उत्थि॑ता॒येत्युत् - स्थि॒ता॒य॒ । \newline
33. स्वाहा॑ विधविष्य॒ते वि॑धविष्य॒ते स्वाहा॒ स्वाहा॑ विधविष्य॒ते स्वाहा॒ स्वाहा॑ विधविष्य॒ते स्वाहा॒ स्वाहा॑ विधविष्य॒ते स्वाहा᳚ । \newline
34. वि॒ध॒वि॒ष्य॒ते स्वाहा॒ स्वाहा॑ विधविष्य॒ते वि॑धविष्य॒ते स्वाहा॑ विधून्वा॒नाय॑ विधून्वा॒नाय॒ स्वाहा॑ विधविष्य॒ते वि॑धविष्य॒ते स्वाहा॑ विधून्वा॒नाय॑ । \newline
35. वि॒ध॒वि॒ष्य॒त इति॑ वि - ध॒वि॒ष्य॒ते । \newline
36. स्वाहा॑ विधून्वा॒नाय॑ विधून्वा॒नाय॒ स्वाहा॒ स्वाहा॑ विधून्वा॒नाय॒ स्वाहा॒ स्वाहा॑ विधून्वा॒नाय॒ स्वाहा॒ स्वाहा॑ विधून्वा॒नाय॒ स्वाहा᳚ । \newline
37. वि॒धू॒न्वा॒नाय॒ स्वाहा॒ स्वाहा॑ विधून्वा॒नाय॑ विधून्वा॒नाय॒ स्वाहा॒ विधू॑ताय॒ विधू॑ताय॒ स्वाहा॑ विधून्वा॒नाय॑ विधून्वा॒नाय॒ स्वाहा॒ विधू॑ताय । \newline
38. वि॒धू॒न्वा॒नायेति॑ वि - धू॒न्वा॒नाय॑ । \newline
39. स्वाहा॒ विधू॑ताय॒ विधू॑ताय॒ स्वाहा॒ स्वाहा॒ विधू॑ताय॒ स्वाहा॒ स्वाहा॒ विधू॑ताय॒ स्वाहा॒ स्वाहा॒ विधू॑ताय॒ स्वाहा᳚ । \newline
40. विधू॑ताय॒ स्वाहा॒ स्वाहा॒ विधू॑ताय॒ विधू॑ताय॒ स्वाहो᳚त्क्रꣳस्य॒त उ॑त्क्रꣳस्य॒ते स्वाहा॒ विधू॑ताय॒ विधू॑ताय॒ स्वाहो᳚त्क्रꣳस्य॒ते । \newline
41. विधू॑ता॒येति॒ वि - धू॒ता॒य॒ । \newline
42. स्वाहो᳚त्क्रꣳस्य॒त उ॑त्क्रꣳस्य॒ते स्वाहा॒ स्वाहो᳚त्क्रꣳस्य॒ते स्वाहा॒ स्वाहो᳚त्क्रꣳस्य॒ते स्वाहा॒ स्वाहो᳚त्क्रꣳस्य॒ते स्वाहा᳚ । \newline
43. उ॒त्क्रꣳ॒॒स्य॒ते स्वाहा॒ स्वाहो᳚त्क्रꣳस्य॒त उ॑त्क्रꣳस्य॒ते स्वाहो॒त्क्राम॑त उ॒त्क्राम॑ते॒ स्वाहो᳚त्क्रꣳस्य॒त उ॑त्क्रꣳस्य॒ते स्वाहो॒त्क्राम॑ते । \newline
44. उ॒त्क्रꣳ॒॒स्य॒त इत्यु॑त् - क्रꣳ॒॒स्य॒ते । \newline
45. स्वाहो॒त्क्राम॑त उ॒त्क्राम॑ते॒ स्वाहा॒ स्वाहो॒त्क्राम॑ते॒ स्वाहा॒ स्वाहो॒त्क्राम॑ते॒ स्वाहा॒ स्वाहो॒त्क्राम॑ते॒ स्वाहा᳚ । \newline
46. उ॒त्क्राम॑ते॒ स्वाहा॒ स्वाहो॒त्क्राम॑त उ॒त्क्राम॑ते॒ स्वाहोत्क्रा᳚न्ता॒ योत्क्रा᳚न्ताय॒ स्वाहो॒त्क्राम॑त उ॒त्क्राम॑ते॒ 
स्वाहोत्क्रा᳚न्ताय । \newline
47. उ॒त्क्राम॑त॒ इत्यु॑त् - क्राम॑ते । \newline
48. स्वाहोत्क्रा᳚न्ता॒ योत्क्रा᳚न्ताय॒ स्वाहा॒ स्वाहोत्क्रा᳚न्ताय॒ स्वाहा॒ स्वाहोत्क्रा᳚न्ताय॒ स्वाहा॒ स्वाहोत्क्रा᳚न्ताय॒ स्वाहा᳚ । \newline
49. उत्क्रा᳚न्ताय॒ स्वाहा॒ स्वाहोत्क्रा᳚न्ता॒ योत्क्रा᳚न्ताय॒ स्वाहा॑ चङ्क्रमिष्य॒ते च॑ङ्क्रमिष्य॒ते स्वाहोत्क्रा᳚न्ता॒ 
योत्क्रा᳚न्ताय॒ स्वाहा॑ चङ्क्रमिष्य॒ते । \newline
50. उत्क्रा᳚न्ता॒येत्युत् - क्रा॒न्ता॒य॒ । \newline
51. स्वाहा॑ चङ्क्रमिष्य॒ते च॑ङ्क्रमिष्य॒ते स्वाहा॒ स्वाहा॑ चङ्क्रमिष्य॒ते स्वाहा॒ स्वाहा॑ चङ्क्रमिष्य॒ते स्वाहा॒ स्वाहा॑ चङ्क्रमिष्य॒ते स्वाहा᳚ । \newline
52. च॒ङ्क्र॒मि॒ष्य॒ते स्वाहा॒ स्वाहा॑ चङ्क्रमिष्य॒ते च॑ङ्क्रमिष्य॒ते स्वाहा॑ चङ्क्र॒म्यमा॑णाय चङ्क्र॒म्यमा॑णाय॒ स्वाहा॑ चङ्क्रमिष्य॒ते च॑ङ्क्रमिष्य॒ते स्वाहा॑ चङ्क्र॒म्यमा॑णाय । \newline
53. स्वाहा॑ चङ्क्र॒म्यमा॑णाय चङ्क्र॒म्यमा॑णाय॒ स्वाहा॒ स्वाहा॑ चङ्क्र॒म्यमा॑णाय॒ स्वाहा॒ स्वाहा॑ चङ्क्र॒म्यमा॑णाय॒ स्वाहा॒ स्वाहा॑ चङ्क्र॒म्यमा॑णाय॒ स्वाहा᳚ । \newline
54. च॒ङ्क्र॒म्यमा॑णाय॒ स्वाहा॒ स्वाहा॑ चङ्क्र॒म्यमा॑णाय चङ्क्र॒म्यमा॑णाय॒ स्वाहा॑ चङ्क्रमि॒ताय॑ चङ्क्रमि॒ताय॒ स्वाहा॑ चङ्क्र॒म्यमा॑णाय चङ्क्र॒म्यमा॑णाय॒ स्वाहा॑ चङ्क्रमि॒ताय॑ । \newline
55. स्वाहा॑ चङ्क्रमि॒ताय॑ चङ्क्रमि॒ताय॒ स्वाहा॒ स्वाहा॑ चङ्क्रमि॒ताय॒ स्वाहा॒ स्वाहा॑ चङ्क्रमि॒ताय॒ स्वाहा॒ स्वाहा॑ चङ्क्रमि॒ताय॒ स्वाहा᳚ । \newline
56. च॒ङ्क्र॒मि॒ताय॒ स्वाहा॒ स्वाहा॑ चङ्क्रमि॒ताय॑ चङ्क्रमि॒ताय॒ स्वाहा॑ कण्डूयिष्य॒ते क॑ण्डूयिष्य॒ते स्वाहा॑ चङ्क्रमि॒ताय॑ चङ्क्रमि॒ताय॒ स्वाहा॑ कण्डूयिष्य॒ते । \newline
57. स्वाहा॑ कण्डूयिष्य॒ते क॑ण्डूयिष्य॒ते स्वाहा॒ स्वाहा॑ कण्डूयिष्य॒ते स्वाहा॒ स्वाहा॑ कण्डूयिष्य॒ते स्वाहा॒ स्वाहा॑ कण्डूयिष्य॒ते स्वाहा᳚ । \newline
58. क॒ण्डू॒यि॒ष्य॒ते स्वाहा॒ स्वाहा॑ कण्डूयिष्य॒ते क॑ण्डूयिष्य॒ते स्वाहा॑ कण्डू॒यमा॑नाय कण्डू॒यमा॑नाय॒ स्वाहा॑ कण्डूयिष्य॒ते क॑ण्डूयिष्य॒ते स्वाहा॑ कण्डू॒यमा॑नाय । \newline
59. स्वाहा॑ कण्डू॒यमा॑नाय कण्डू॒यमा॑नाय॒ स्वाहा॒ स्वाहा॑ कण्डू॒यमा॑नाय॒ स्वाहा॒ स्वाहा॑ कण्डू॒यमा॑नाय॒ स्वाहा॒ स्वाहा॑ कण्डू॒यमा॑नाय॒ स्वाहा᳚ । \newline
60. क॒ण्डू॒यमा॑नाय॒ स्वाहा॒ स्वाहा॑ कण्डू॒यमा॑नाय कण्डू॒यमा॑नाय॒ स्वाहा॑ कण्डूयि॒ताय॑ कण्डूयि॒ताय॒ स्वाहा॑ कण्डू॒यमा॑नाय कण्डू॒यमा॑नाय॒ स्वाहा॑ कण्डूयि॒ताय॑ । \newline
61. स्वाहा॑ कण्डूयि॒ताय॑ कण्डूयि॒ताय॒ स्वाहा॒ स्वाहा॑ कण्डूयि॒ताय॒ स्वाहा॒ स्वाहा॑ कण्डूयि॒ताय॒ स्वाहा॒ स्वाहा॑ कण्डूयि॒ताय॒ स्वाहा᳚ । \newline
62. क॒ण्डू॒यि॒ताय॒ स्वाहा॒ स्वाहा॑ कण्डूयि॒ताय॑ कण्डूयि॒ताय॒ स्वाहा॑ निकषिष्य॒ते नि॑कषिष्य॒ते स्वाहा॑ कण्डूयि॒ताय॑ कण्डूयि॒ताय॒ स्वाहा॑ निकषिष्य॒ते । \newline
63. स्वाहा॑ निकषिष्य॒ते नि॑कषिष्य॒ते स्वाहा॒ स्वाहा॑ निकषिष्य॒ते स्वाहा॒ स्वाहा॑ निकषिष्य॒ते स्वाहा॒ स्वाहा॑ निकषिष्य॒ते स्वाहा᳚ । \newline
64. नि॒क॒षि॒ष्य॒ते स्वाहा॒ स्वाहा॑ निकषिष्य॒ते नि॑कषिष्य॒ते स्वाहा॑ नि॒कष॑माणाय नि॒कष॑माणाय॒ स्वाहा॑ निकषिष्य॒ते नि॑कषिष्य॒ते स्वाहा॑ नि॒कष॑माणाय । \newline
65. नि॒क॒षि॒ष्य॒त इति॑ नि - क॒षि॒ष्य॒ते । \newline
66. स्वाहा॑ नि॒कष॑माणाय नि॒कष॑माणाय॒ स्वाहा॒ स्वाहा॑ नि॒कष॑माणाय॒ स्वाहा॒ स्वाहा॑ नि॒कष॑माणाय॒ स्वाहा॒ स्वाहा॑ नि॒कष॑माणाय॒ स्वाहा᳚ । \newline
67. नि॒कष॑माणाय॒ स्वाहा॒ स्वाहा॑ नि॒कष॑माणाय नि॒कष॑माणाय॒ स्वाहा॒ निक॑षिताय॒ निक॑षिताय॒ स्वाहा॑ नि॒कष॑माणाय नि॒कष॑माणाय॒ स्वाहा॒ निक॑षिताय । \newline
68. नि॒कष॑माणा॒येति॑ नि - कष॑माणाय । \newline
69. स्वाहा॒ निक॑षिताय॒ निक॑षिताय॒ स्वाहा॒ स्वाहा॒ निक॑षिताय॒ स्वाहा॒ स्वाहा॒ निक॑षिताय॒ स्वाहा॒ स्वाहा॒ निक॑षिताय॒ स्वाहा᳚ । \newline
70. निक॑षिताय॒ स्वाहा॒ स्वाहा॒ निक॑षिताय॒ निक॑षिताय॒ स्वाहा॒ यद् यथ् स्वाहा॒ निक॑षिताय॒ निक॑षिताय॒ स्वाहा॒ यत् । \newline
71. निक॑षिता॒येति॒ नि - क॒षि॒ता॒य॒ । \newline
72. स्वाहा॒ यद् यथ् स्वाहा॒ स्वाहा॒ यदत्त्यत्ति॒ यथ् स्वाहा॒ स्वाहा॒ यदत्ति॑ । \newline
73. यदत्त्यत्ति॒ यद् यदत्ति॒ तस्मै॒ तस्मा॒ अत्ति॒ यद् यदत्ति॒ तस्मै᳚ । \newline
74. अत्ति॒ तस्मै॒ तस्मा॒ अत्त्यत्ति॒ तस्मै॒ स्वाहा॒ स्वाहा॒ तस्मा॒ अत्त्यत्ति॒ तस्मै॒ स्वाहा᳚ । \newline
75. तस्मै॒ स्वाहा॒ स्वाहा॒ तस्मै॒ तस्मै॒ स्वाहा॒ यद् यथ् स्वाहा॒ तस्मै॒ तस्मै॒ स्वाहा॒ यत् । \newline
76. स्वाहा॒ यद् यथ् स्वाहा॒ स्वाहा॒ यत् पिब॑ति॒ पिब॑ति॒ यथ् स्वाहा॒ स्वाहा॒ यत् पिब॑ति । \newline
77. यत् पिब॑ति॒ पिब॑ति॒ यद् यत् पिब॑ति॒ तस्मै॒ तस्मै॒ पिब॑ति॒ यद् यत् पिब॑ति॒ तस्मै᳚ । \newline
78. पिब॑ति॒ तस्मै॒ तस्मै॒ पिब॑ति॒ पिब॑ति॒ तस्मै॒ स्वाहा॒ स्वाहा॒ तस्मै॒ पिब॑ति॒ पिब॑ति॒ तस्मै॒ स्वाहा᳚ । \newline
79. तस्मै॒ स्वाहा॒ स्वाहा॒ तस्मै॒ तस्मै॒ स्वाहा॒ यद् यथ् स्वाहा॒ तस्मै॒ तस्मै॒ स्वाहा॒ यत् । \newline
80. स्वाहा॒ यद् यथ् स्वाहा॒ स्वाहा॒ यन् मेह॑ति॒ मेह॑ति॒ यथ् स्वाहा॒ स्वाहा॒ यन् मेह॑ति । \newline
81. यन् मेह॑ति॒ मेह॑ति॒ यद् यन् मेह॑ति॒ तस्मै॒ तस्मै॒ मेह॑ति॒ यद् यन् मेह॑ति॒ तस्मै᳚ । \newline
82. मेह॑ति॒ तस्मै॒ तस्मै॒ मेह॑ति॒ मेह॑ति॒ तस्मै॒ स्वाहा॒ स्वाहा॒ तस्मै॒ मेह॑ति॒ मेह॑ति॒ तस्मै॒ स्वाहा᳚ । \newline
83. तस्मै॒ स्वाहा॒ स्वाहा॒ तस्मै॒ तस्मै॒ स्वाहा॒ यद् यथ् स्वाहा॒ तस्मै॒ तस्मै॒ स्वाहा॒ यत् । \newline
84. स्वाहा॒ यद् यथ् स्वाहा॒ स्वाहा॒ यच्छकृ॒च् छकृ॒द् यथ् स्वाहा॒ स्वाहा॒ यच्छकृ॑त् । \newline
85. यच्छकृ॒च् छकृ॒द् यद् यच्छकृ॑त् क॒रोति॑ क॒रोति॒ शकृ॒द् यद् यच्छकृ॑त् क॒रोति॑ । \newline
86. शकृ॑त् क॒रोति॑ क॒रोति॒ शकृ॒च् छकृ॑त् क॒रोति॒ तस्मै॒ तस्मै॑ क॒रोति॒ शकृ॒च् छकृ॑त् क॒रोति॒ तस्मै᳚ । \newline
87. क॒रोति॒ तस्मै॒ तस्मै॑ क॒रोति॑ क॒रोति॒ तस्मै॒ स्वाहा॒ स्वाहा॒ तस्मै॑ क॒रोति॑ क॒रोति॒ तस्मै॒ स्वाहा᳚ । \newline
88. तस्मै॒ स्वाहा॒ स्वाहा॒ तस्मै॒ तस्मै॒ स्वाहा॒ रेत॑से॒ रेत॑से॒ स्वाहा॒ तस्मै॒ तस्मै॒ स्वाहा॒ रेत॑से । \newline
89. स्वाहा॒ रेत॑से॒ रेत॑से॒ स्वाहा॒ स्वाहा॒ रेत॑से॒ स्वाहा॒ स्वाहा॒ रेत॑से॒ स्वाहा॒ स्वाहा॒ रेत॑से॒ स्वाहा᳚ । \newline
90. रेत॑से॒ स्वाहा॒ स्वाहा॒ रेत॑से॒ रेत॑से॒ स्वाहा᳚ प्र॒जाभ्यः॑ प्र॒जाभ्यः॒ स्वाहा॒ रेत॑से॒ रेत॑से॒ स्वाहा᳚ प्र॒जाभ्यः॑ । \newline
91. स्वाहा᳚ प्र॒जाभ्यः॑ प्र॒जाभ्यः॒ स्वाहा॒ स्वाहा᳚ प्र॒जाभ्यः॒ स्वाहा॒ स्वाहा᳚ प्र॒जाभ्यः॒ स्वाहा॒ स्वाहा᳚ प्र॒जाभ्यः॒ स्वाहा᳚ । \newline
92. प्र॒जाभ्यः॒ स्वाहा॒ स्वाहा᳚ प्र॒जाभ्यः॑ प्र॒जाभ्यः॒ स्वाहा᳚ प्र॒जन॑नाय प्र॒जन॑नाय॒ स्वाहा᳚ प्र॒जाभ्यः॑ प्र॒जाभ्यः॒ स्वाहा᳚ प्र॒जन॑नाय । \newline
93. प्र॒जाभ्य॒ इति॑ प्र - जाभ्यः॑ । \newline
94. स्वाहा᳚ प्र॒जन॑नाय प्र॒जन॑नाय॒ स्वाहा॒ स्वाहा᳚ प्र॒जन॑नाय॒ स्वाहा॒ स्वाहा᳚ प्र॒जन॑नाय॒ स्वाहा॒ स्वाहा᳚ प्र॒जन॑नाय॒ स्वाहा᳚ । \newline
95. प्र॒जन॑नाय॒ स्वाहा॒ स्वाहा᳚ प्र॒जन॑नाय प्र॒जन॑नाय॒ स्वाहा॒ सर्व॑स्मै॒ सर्व॑स्मै॒ स्वाहा᳚ प्र॒जन॑नाय प्र॒जन॑नाय॒ स्वाहा॒ सर्व॑स्मै । \newline
96. प्र॒जन॑ना॒येति॑ प्र - जन॑नाय । \newline
97. स्वाहा॒ सर्व॑स्मै॒ सर्व॑स्मै॒ स्वाहा॒ स्वाहा॒ सर्व॑स्मै॒ स्वाहा॒ स्वाहा॒ सर्व॑स्मै॒ स्वाहा॒ स्वाहा॒ सर्व॑स्मै॒ स्वाहा᳚ । \newline
98. सर्व॑स्मै॒ स्वाहा॒ स्वाहा॒ सर्व॑स्मै॒ सर्व॑स्मै॒ स्वाहा᳚ । \newline
99. स्वाहेति॒ स्वाहा᳚ । \newline
\pagebreak
\markright{ TS 7.1.20.1  \hfill https://www.vedavms.in \hfill}

\section{ TS 7.1.20.1 }

\textbf{TS 7.1.20.1 } \newline
\textbf{Samhita Paata} \newline

अ॒ग्नये॒ स्वाहा॑ वा॒यवे॒ स्वाहा॒ सूर्या॑य॒ स्वाहा॒-र्तम॑स्यृ॒तस्य॒र्तम॑सि स॒त्यम॑सि स॒त्यस्य॑ स॒त्यम॑स्यृ॒तस्य॒ पन्था॑ असि दे॒वानां᳚ छा॒याऽमृत॑स्य॒ नाम॒ तथ् स॒त्यं ॅयत् त्वं प्र॒जाप॑ति॒रस्यधि॒ यद॑स्मिन् वा॒जिनी॑व॒ शुभः॒ स्पर्द्ध॑न्ते॒ दिवः॒ सूर्ये॑ण॒ विशो॒ऽपो वृ॑णा॒नः प॑वते क॒व्यन् प॒शुं न गो॒पा इर्यः॒ परि॑ज्मा ॥ \newline

\textbf{Pada Paata} \newline

अ॒ग्नये᳚ । स्वाहा᳚ । वा॒यवे᳚ । स्वाहा᳚ । सूर्या॑य । स्वाहा᳚ । ऋ॒तम् । अ॒सि॒ । ऋ॒तस्य॑ । ऋ॒तम् । अ॒सि॒ । स॒त्यम् । अ॒सि॒ । स॒त्यस्य॑ । स॒त्यम् । अ॒सि॒ । ऋ॒तस्य॑ । पन्थाः᳚ । अ॒सि॒ । दे॒वाना᳚म् । छा॒या । अ॒मृत॑स्य । नाम॑ । तत् । स॒त्यम् । यत् । त्वम् । प्र॒जाप॑ति॒रिति॑ प्र॒जा - प॒तिः॒ । असि॑ । अधीति॑ । यत् । अ॒स्मि॒न्न् । वा॒जिनि॑ । इ॒व॒ । शुभः॑ । स्पद्‌र्ध॑न्ते । दिवः॑ । सूर्ये॑ण । विशः॑ । अ॒पः । वृ॒णा॒नः । प॒व॒ते॒ । क॒व्यन्न् । प॒शुम् । न । गो॒पा इति॑ गो - पाः । इर्यः॑ । परि॒ज्मेति॒ परि॑ - ज्मा॒ ॥  \newline


\textbf{Krama Paata} \newline

अ॒ग्नये॒ स्वाहा᳚ । स्वाहा॑ वा॒यवे᳚ । वा॒यवे॒ स्वाहा᳚ । स्वाहा॒ सूर्या॑य । सूर्या॑य॒ स्वाहा᳚ । स्वाह॒र्तम् । ऋ॒तम॑सि । अ॒स्यृ॒र्तस्य॑ । ऋ॒तस्य॒र्तम् । ऋ॒तम॑सि । अ॒सि॒ स॒त्यम् । स॒त्यम॑सि । अ॒सि॒ स॒त्यस्य॑ । स॒त्यस्य॑ स॒त्यम् । स॒त्यम॑सि । अ॒स्यृ॒तस्य॑ । ऋ॒तस्य॒ पन्थाः᳚ । पन्था॑ असि । अ॒सि॒ दे॒वाना᳚म् । दे॒वाना᳚म् छा॒या । छा॒याऽमृत॑स्य । अ॒मृत॑स्य॒ नाम॑ । नाम॒ तत् । तथ् स॒त्यम् । स॒त्यम् ॅयत् । यत् त्वम् । त्वम् प्र॒जाप॑तिः । प्र॒जाप॑ति॒रसि॑ । प्र॒जाप॑ति॒रिति॑ प्र॒जा - प॒तिः॒ । अस्यधि॑ । अधि॒ यत् । यद॑स्मिन्न् । अ॒स्मि॒न् वा॒जिनि॑ । वा॒जिनी॑व । इ॒व॒ शुभः॑ । शुभः॒ स्पर्द्ध॑न्ते । स्पर्द्ध॑न्ते॒ दिवः॑ । दिवः॒ सूर्ये॑ण । सूर्ये॑ण॒ विशः॑ । विशो॒ऽपः । अ॒पो वृ॑णा॒नः । वृ॒णा॒नः प॑वते । प॒व॒ते॒ क॒व्यन्न् । क॒व्यन् प॒शुम् । प॒शुम् न । न गो॒पाः । गो॒पा इर्यः॑ । गो॒पा इति॑ गो - पाः । इर्यः॒ परि॑ज्मा । परि॒ज्मेति॒ परि॑ - ज्मा॒ । \newline

\textbf{Jatai Paata} \newline

1. अ॒ग्नये॒ स्वाहा॒ स्वाहा॒ ऽग्नये॒ ऽग्नये॒ स्वाहा᳚ । \newline
2. स्वाहा॑ वा॒यवे॑ वा॒यवे॒ स्वाहा॒ स्वाहा॑ वा॒यवे᳚ । \newline
3. वा॒यवे॒ स्वाहा॒ स्वाहा॑ वा॒यवे॑ वा॒यवे॒ स्वाहा᳚ । \newline
4. स्वाहा॒ सूर्या॑य॒ सूर्या॑य॒ स्वाहा॒ स्वाहा॒ सूर्या॑य । \newline
5. सूर्या॑य॒ स्वाहा॒ स्वाहा॒ सूर्या॑य॒ सूर्या॑य॒ स्वाहा᳚ । \newline
6. स्वाह॒ र्‌त मृ॒तꣳ स्वाहा॒ स्वाह॒ र्‌तम् । \newline
7. ऋ॒त म॑स्य स्यृ॒त मृ॒त म॑सि । \newline
8. अ॒स्यृ॒ तस्य॒ र्‌तस्या᳚स्य स्यृ॒तस्य॑ । \newline
9. ऋ॒तस्य॒ र्‌त मृ॒त मृ॒तस्य॒ र्‌तस्य॒ र्‌तम् । \newline
10. ऋ॒त म॑स्य स्यृ॒त मृ॒त म॑सि । \newline
11. अ॒सि॒ स॒त्यꣳ स॒त्य म॑स्यसि स॒त्यम् । \newline
12. स॒त्य म॑स्यसि स॒त्यꣳ स॒त्य म॑सि । \newline
13. अ॒सि॒ स॒त्यस्य॑ स॒त्य स्या᳚स्यसि स॒त्यस्य॑ । \newline
14. स॒त्यस्य॑ स॒त्यꣳ स॒त्यꣳ स॒त्यस्य॑ स॒त्यस्य॑ स॒त्यम् । \newline
15. स॒त्य म॑स्यसि स॒त्यꣳ स॒त्य म॑सि । \newline
16. अ॒स्यृ॒तस्य॒ र्‌तस्या᳚स्य स्यृ॒तस्य॑ । \newline
17. ऋ॒तस्य॒ पन्थाः॒ पन्था॑ ऋ॒तस्य॒ र्‌तस्य॒ पन्थाः᳚ । \newline
18. पन्था॑ अस्यसि॒ पन्थाः॒ पन्था॑ असि । \newline
19. अ॒सि॒ दे॒वाना᳚म् दे॒वाना॑ मस्यसि दे॒वाना᳚म् । \newline
20. दे॒वाना᳚म् छा॒या छा॒या दे॒वाना᳚म् दे॒वाना᳚म् छा॒या । \newline
21. छा॒या ऽमृत॑स्या॒ मृत॑स्य छा॒या छा॒या ऽमृत॑स्य । \newline
22. अ॒मृत॑स्य॒ नाम॒ नामा॒ मृत॑स्या॒ मृत॑स्य॒ नाम॑ । \newline
23. नाम॒ तत् तन् नाम॒ नाम॒ तत् । \newline
24. तथ् स॒त्यꣳ स॒त्यम् तत् तथ् स॒त्यम् । \newline
25. स॒त्यं ॅयद् यथ् स॒त्यꣳ स॒त्यं ॅयत् । \newline
26. यत् त्वम् त्वं ॅयद् यत् त्वम् । \newline
27. त्वम् प्र॒जाप॑तिः प्र॒जाप॑ति॒ स्त्वम् त्वम् प्र॒जाप॑तिः । \newline
28. प्र॒जाप॑ति॒ रस्यसि॑ प्र॒जाप॑तिः प्र॒जाप॑ति॒ रसि॑ । \newline
29. प्र॒जाप॑ति॒रिति॑ प्र॒जा - प॒तिः॒ । \newline
30. अस्यध्य ध्यस्य स्यधि॑ । \newline
31. अधि॒ यद् यद ध्यधि॒ यत् । \newline
32. यद॑स्मिन् नस्मि॒न्॒. यद् यद॑स्मिन्न् । \newline
33. अ॒स्मि॒न्॒. वा॒जिनि॑ वा॒जि न्य॑स्मिन् नस्मिन्. वा॒जिनि॑ । \newline
34. वा॒जि नी॑वेव वा॒जिनि॑ वा॒जि नी॑व । \newline
35. इ॒व॒ शुभः॒ शुभ॑ इवेव॒ शुभः॑ । \newline
36. शुभः॒ स्पर्द्ध॑न्ते॒ स्पर्द्ध॑न्ते॒ शुभः॒ शुभः॒ स्पर्द्ध॑न्ते । \newline
37. स्पर्द्ध॑न्ते॒ दिवो॒ दिवः॒ स्पर्द्ध॑न्ते॒ स्पर्द्ध॑न्ते॒ दिवः॑ । \newline
38. दिवः॒ सूर्ये॑ण॒ सूर्ये॑ण॒ दिवो॒ दिवः॒ सूर्ये॑ण । \newline
39. सूर्ये॑ण॒ विशो॒ विशः॒ सूर्ये॑ण॒ सूर्ये॑ण॒ विशः॑ । \newline
40. विशो॒ ऽपो॑ ऽपो विशो॒ विशो॒ ऽपः । \newline
41. अ॒पो वृ॑णा॒नो वृ॑णा॒नो᳚(1॒) ऽपो॑ ऽपो वृ॑णा॒नः । \newline
42. वृ॒णा॒नः प॑वते पवते वृणा॒नो वृ॑णा॒नः प॑वते । \newline
43. प॒व॒ते॒ क॒व्यन् क॒व्यन् प॑वते पवते क॒व्यन्न् । \newline
44. क॒व्यन् प॒शुम् प॒शुम् क॒व्यन् क॒व्यन् प॒शुम् । \newline
45. प॒शुन् न न प॒शुम् प॒शुन् न । \newline
46. न गो॒पा गो॒पा न न गो॒पाः । \newline
47. गो॒पा इर्य॒ इर्यो॑ गो॒पा गो॒पा इर्यः॑ । \newline
48. गो॒पा इति॑ गो - पाः । \newline
49. इर्यः॒ परि॑ज्मा॒ परि॒ज्मेर्य॒ इर्यः॒ परि॑ज्मा । \newline
50. परि॒ज्मेति॒ परि॑ - ज्मा॒ । \newline

\textbf{Ghana Paata } \newline

1. अ॒ग्नये॒ स्वाहा॒ स्वाहा॒ ऽग्नये॒ ऽग्नये॒ स्वाहा॑ वा॒यवे॑ वा॒यवे॒ स्वाहा॒ ऽग्नये॒ ऽग्नये॒ स्वाहा॑ वा॒यवे᳚ । \newline
2. स्वाहा॑ वा॒यवे॑ वा॒यवे॒ स्वाहा॒ स्वाहा॑ वा॒यवे॒ स्वाहा॒ स्वाहा॑ वा॒यवे॒ स्वाहा॒ स्वाहा॑ वा॒यवे॒ स्वाहा᳚ । \newline
3. वा॒यवे॒ स्वाहा॒ स्वाहा॑ वा॒यवे॑ वा॒यवे॒ स्वाहा॒ सूर्या॑य॒ सूर्या॑य॒ स्वाहा॑ वा॒यवे॑ वा॒यवे॒ स्वाहा॒ सूर्या॑य । \newline
4. स्वाहा॒ सूर्या॑य॒ सूर्या॑य॒ स्वाहा॒ स्वाहा॒ सूर्या॑य॒ स्वाहा॒ स्वाहा॒ सूर्या॑य॒ स्वाहा॒ स्वाहा॒ सूर्या॑य॒ स्वाहा᳚ । \newline
5. सूर्या॑य॒ स्वाहा॒ स्वाहा॒ सूर्या॑य॒ सूर्या॑य॒ स्वाह॒ र्‌त मृ॒तꣳ स्वाहा॒ सूर्या॑य॒ सूर्या॑य॒ स्वाह॒ र्‌तम् । \newline
6. स्वाह॒ र्‌त मृ॒तꣳ स्वाहा॒ स्वाह॒ र्‌त म॑स्यस्यृ॒तꣳ स्वाहा॒ स्वाह॒ र्‌त म॑सि । \newline
7. ऋ॒त म॑स्य स्यृ॒त मृ॒त म॑स्यृ॒तस्य॒ र्‌तस्या᳚ स्यृ॒त मृ॒त म॑स्यृ॒तस्य॑ । \newline
8. अ॒स्यृ॒तस्य॒ र्‌तस्या᳚स्य स्यृ॒तस्य॒ र्‌त मृ॒त मृ॒तस्या᳚स्य स्यृ॒तस्य॒ र्‌तम् । \newline
9. ऋ॒तस्य॒ र्‌त मृ॒त मृ॒तस्य॒ र्‌तस्य॒ र्‌त म॑स्य स्यृ॒त मृ॒तस्य॒ र्‌तस्य॒ र्‌त म॑सि । \newline
10. ऋ॒त म॑स्य स्यृ॒त मृ॒त म॑सि स॒त्यꣳ स॒त्य म॑स्यृ॒त मृ॒त म॑सि स॒त्यम् । \newline
11. अ॒सि॒ स॒त्यꣳ स॒त्य म॑स्यसि स॒त्य म॑स्यसि स॒त्य म॑स्यसि स॒त्य म॑सि । \newline
12. स॒त्य म॑स्यसि स॒त्यꣳ स॒त्य म॑सि स॒त्यस्य॑ स॒त्य स्या॑सि स॒त्यꣳ स॒त्य म॑सि स॒त्यस्य॑ । \newline
13. अ॒सि॒ स॒त्यस्य॑ स॒त्य स्या᳚स्यसि स॒त्यस्य॑ स॒त्यꣳ स॒त्यꣳ स॒त्य स्या᳚स्यसि स॒त्यस्य॑ स॒त्यम् । \newline
14. स॒त्यस्य॑ स॒त्यꣳ स॒त्यꣳ स॒त्यस्य॑ स॒त्यस्य॑ स॒त्य म॑स्यसि स॒त्यꣳ स॒त्यस्य॑ स॒त्यस्य॑ स॒त्य म॑सि । \newline
15. स॒त्य म॑स्यसि स॒त्यꣳ स॒त्य म॑स्यृ॒तस्य॒ र्‌तस्या॑सि स॒त्यꣳ स॒त्य म॑स्यृ॒तस्य॑ । \newline
16. अ॒स्यृ॒तस्य॒ र्‌तस्या᳚स्य स्यृ॒तस्य॒ पन्थाः॒ पन्था॑ ऋ॒तस्या᳚स्य स्यृ॒तस्य॒ पन्थाः᳚ । \newline
17. ऋ॒तस्य॒ पन्थाः॒ पन्था॑ ऋ॒तस्य॒ र्‌तस्य॒ पन्था॑ अस्यसि॒ पन्था॑ ऋ॒तस्य॒ र्‌तस्य॒ पन्था॑ असि । \newline
18. पन्था॑ अस्यसि॒ पन्थाः॒ पन्था॑ असि दे॒वाना᳚म् दे॒वाना॑ मसि॒ पन्थाः॒ पन्था॑ असि दे॒वाना᳚म् । \newline
19. अ॒सि॒ दे॒वाना᳚म् दे॒वाना॑ मस्यसि दे॒वाना᳚म् छा॒या छा॒या दे॒वाना॑ मस्यसि दे॒वाना᳚म् छा॒या । \newline
20. दे॒वाना᳚म् छा॒या छा॒या दे॒वाना᳚म् दे॒वाना᳚म् छा॒या ऽमृत॑स्या॒ मृत॑स्य छा॒या दे॒वाना᳚म् दे॒वाना᳚म् छा॒या ऽमृत॑स्य । \newline
21. छा॒या ऽमृत॑स्या॒ मृत॑स्य छा॒या छा॒या ऽमृत॑स्य॒ नाम॒ नामा॒ मृत॑स्य छा॒या छा॒या ऽमृत॑स्य॒ नाम॑ । \newline
22. अ॒मृत॑स्य॒ नाम॒ नामा॒ मृत॑स्या॒ मृत॑स्य॒ नाम॒ तत् तन् नामा॒ मृत॑स्या॒ मृत॑स्य॒ नाम॒ तत् । \newline
23. नाम॒ तत् तन् नाम॒ नाम॒ तथ् स॒त्यꣳ स॒त्यम् तन् नाम॒ नाम॒ तथ् स॒त्यम् । \newline
24. तथ् स॒त्यꣳ स॒त्यम् तत् तथ् स॒त्यं ॅयद् यथ् स॒त्यम् तत् तथ् स॒त्यं ॅयत् । \newline
25. स॒त्यं ॅयद् यथ् स॒त्यꣳ स॒त्यं ॅयत् त्वम् त्वं ॅयथ् स॒त्यꣳ स॒त्यं ॅयत् त्वम् । \newline
26. यत् त्वम् त्वं ॅयद् यत् त्वम् प्र॒जाप॑तिः प्र॒जाप॑ति॒ स्त्वं ॅयद् यत् त्वम् प्र॒जाप॑तिः । \newline
27. त्वम् प्र॒जाप॑तिः प्र॒जाप॑ति॒ स्त्वम् त्वम् प्र॒जाप॑ति॒ रस्यसि॑ प्र॒जाप॑ति॒ स्त्वम् त्वम् प्र॒जाप॑ति॒ रसि॑ । \newline
28. प्र॒जाप॑ति॒ रस्यसि॑ प्र॒जाप॑तिः प्र॒जाप॑ति॒ रस्यध्य ध्यसि॑ प्र॒जाप॑तिः प्र॒जाप॑ति॒ रस्यधि॑ । \newline
29. प्र॒जाप॑ति॒रिति॑ प्र॒जा - प॒तिः॒ । \newline
30. अस्यध्य ध्यस्य स्यधि॒ यद् यद ध्यस्य स्यधि॒ यत् । \newline
31. अधि॒ यद् यद ध्यधि॒ यद॑स्मिन् नस्मि॒न्॒. यद ध्यधि॒ यद॑स्मिन्न् । \newline
32. यद॑स्मिन् नस्मि॒न्॒. यद् यद॑स्मिन्. वा॒जिनि॑ वा॒जि न्य॑स्मि॒न्॒. यद् यद॑स्मिन्. वा॒जिनि॑ । \newline
33. अ॒स्मि॒न्॒. वा॒जिनि॑ वा॒जि न्य॑स्मिन् नस्मिन्. वा॒जिनी॑वेव वा॒जि न्य॑स्मिन् नस्मिन्. वा॒जिनी॑व । \newline
34. वा॒जिनी॑वेव वा॒जिनि॑ वा॒जिनी॑व॒ शुभः॒ शुभ॑ इव वा॒जिनि॑ वा॒जिनी॑व॒ शुभः॑ । \newline
35. इ॒व॒ शुभः॒ शुभ॑ इवेव॒ शुभः॒ स्पर्द्ध॑न्ते॒ स्पर्द्ध॑न्ते॒ शुभ॑ इवेव॒ शुभः॒ स्पर्द्ध॑न्ते । \newline
36. शुभः॒ स्पर्द्ध॑न्ते॒ स्पर्द्ध॑न्ते॒ शुभः॒ शुभः॒ स्पर्द्ध॑न्ते॒ दिवो॒ दिवः॒ स्पर्द्ध॑न्ते॒ शुभः॒ शुभः॒ स्पर्द्ध॑न्ते॒ दिवः॑ । \newline
37. स्पर्द्ध॑न्ते॒ दिवो॒ दिवः॒ स्पर्द्ध॑न्ते॒ स्पर्द्ध॑न्ते॒ दिवः॒ सूर्ये॑ण॒ सूर्ये॑ण॒ दिवः॒ स्पर्द्ध॑न्ते॒ स्पर्द्ध॑न्ते॒ दिवः॒ सूर्ये॑ण । \newline
38. दिवः॒ सूर्ये॑ण॒ सूर्ये॑ण॒ दिवो॒ दिवः॒ सूर्ये॑ण॒ विशो॒ विशः॒ सूर्ये॑ण॒ दिवो॒ दिवः॒ सूर्ये॑ण॒ विशः॑ । \newline
39. सूर्ये॑ण॒ विशो॒ विशः॒ सूर्ये॑ण॒ सूर्ये॑ण॒ विशो॒ ऽपो॑ ऽपो विशः॒ सूर्ये॑ण॒ सूर्ये॑ण॒ विशो॒ ऽपः । \newline
40. विशो॒ ऽपो॑ ऽपो विशो॒ विशो॒ ऽपो वृ॑णा॒नो वृ॑णा॒नो॑ ऽपो विशो॒ विशो॒ ऽपो वृ॑णा॒नः । \newline
41. अ॒पो वृ॑णा॒नो वृ॑णा॒नो᳚(1॒) ऽपो॑ ऽपो वृ॑णा॒नः प॑वते पवते वृणा॒नो᳚(1॒) ऽपो॑ ऽपो वृ॑णा॒नः प॑वते । \newline
42. वृ॒णा॒नः प॑वते पवते वृणा॒नो वृ॑णा॒नः प॑वते क॒व्यन् क॒व्यन् प॑वते वृणा॒नो वृ॑णा॒नः प॑वते क॒व्यन्न् । \newline
43. प॒व॒ते॒ क॒व्यन् क॒व्यन् प॑वते पवते क॒व्यन् प॒शुम् प॒शुम् क॒व्यन् प॑वते पवते क॒व्यन् प॒शुम् । \newline
44. क॒व्यन् प॒शुम् प॒शुम् क॒व्यन् क॒व्यन् प॒शुन् न न प॒शुम् क॒व्यन् क॒व्यन् प॒शुन् न । \newline
45. प॒शुन् न न प॒शुम् प॒शुन् न गो॒पा गो॒पा न प॒शुम् प॒शुन् न गो॒पाः । \newline
46. न गो॒पा गो॒पा न न गो॒पा इर्य॒ इर्यो॑ गो॒पा न न गो॒पा इर्यः॑ । \newline
47. गो॒पा इर्य॒ इर्यो॑ गो॒पा गो॒पा इर्यः॒ परि॑ज्मा॒ परि॒ज् मेर्यो॑ गो॒पा गो॒पा इर्यः॒ परि॑ज्मा । \newline
48. गो॒पा इति॑ गो - पाः । \newline
49. इर्यः॒ परि॑ज्मा॒ परि॒ज् मेर्य॒ इर्यः॒ परि॑ज्मा । \newline
50. परि॒ज्मेति॒ परि॑ - ज्मा॒ । \newline
\pagebreak


\end{document}