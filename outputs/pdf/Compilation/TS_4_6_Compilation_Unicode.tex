\documentclass[17pt]{extarticle}
\usepackage{babel}
\usepackage{fontspec}
\usepackage{polyglossia}
\usepackage{extsizes}

\usepackage{color}   %May be necessary if you want to color links
\usepackage{hyperref}
\hypersetup{
    colorlinks=true, %set true if you want colored links
    linktoc=all,     %set to all if you want both sections and subsections linked
    linkcolor=black,  %choose some color if you want links to stand out
}

\setmainlanguage{sanskrit}
\setotherlanguages{english} %% or other languages
\setlength{\parindent}{0pt}
\pagestyle{myheadings}
\newfontfamily\devanagarifont[Script=Devanagari]{AdishilaVedic}
\renewcommand{\theHsection}{\thepart.section.\thesection}

\newcommand{\VAR}[1]{}
\newcommand{\BLOCK}[1]{}




\begin{document}
\begin{titlepage}
    \begin{center}
 
\begin{sanskrit}
    { \Large
    कृष्ण यजुर्वेदीय तैत्तिरीय संहिता,पद,जटा,घन पाठः 
    }
    \\
    \vspace{2.5cm}
    \mbox{ \Large
    4.6      चतुर्थकाण्डे षष्ठः प्रश्नः - परिषेचन-संस्काराभिधानं   }
\end{sanskrit}
\end{center}

\end{titlepage}
\tableofcontents
\phantomsection
\pagebreak

\markright{ TS 4.6.1.1  \hfill https://www.vedavms.in \hfill}

\section{ TS 4.6.1.1 }

\textbf{TS 4.6.1.1 } \newline
\textbf{Samhita Paata} \newline

अश्म॒न्नूर्जं॒ पर्व॑ते शिश्रिया॒णां ॅवाते॑ प॒र्जन्ये॒ वरु॑णस्य॒ शुष्मे᳚ । अ॒द्भ्य ओष॑धीभ्यो॒ वन॒स्पति॒भ्योऽधि॒ संभृ॑तां॒ तां न॒ इष॒मूर्जं॑ धत्त मरुतः सꣳ ररा॒णाः ॥ अश्मꣳ॑स्ते॒ क्षुद॒मुं ते॒ शुगृ॑च्छतु॒ यं द्वि॒ष्मः ॥ स॒मु॒द्रस्य॑ त्वा॒ऽ*वाक॒याऽग्ने॒ परि॑ व्ययामसि । पा॒व॒को अ॒स्मभ्यꣳ॑ शि॒वो भ॑व ॥ हि॒मस्य॑ त्वा ज॒रायु॒णाऽग्ने॒ परि॑ व्ययामसि । पा॒व॒को अ॒स्मभ्यꣳ॑ शि॒वो भ॑व ॥ उप॒ - [  ] \newline

\textbf{Pada Paata} \newline

अश्मन्न्॑ । ऊर्ज᳚म् । पर्व॑ते । शि॒श्रि॒या॒णाम् । वाते᳚ । प॒र्जन्ये᳚ । वरु॑णस्य । शुष्मे᳚ ॥ अ॒द्भ्य इत्य॑त् - भ्यः । ओष॑धीभ्य॒ इत्योष॑धि - भ्यः॒ । वन॒स्पति॑भ्य॒ इति॒ वन॒स्पति॑ - भ्यः॒ । अधीति॑ । संभृ॑ता॒मिति॒ सं - भृ॒ता॒म् । ताम् । नः॒ । इष᳚म् । ऊर्ज᳚म् । ध॒त्त॒ । म॒रु॒तः॒ । सꣳ॒॒र॒रा॒णा इति॑ सं - र॒रा॒णाः ॥ अश्मन्न्॑ । ते॒ । क्षुत् । अ॒मुम् । ते॒ । शुक् । ऋ॒च्छ॒तु॒ । यम् । द्वि॒ष्मः ॥ स॒मु॒द्रस्य॑ । त्वा॒ । अ॒वाक॑या । अग्ने᳚ । परीति॑ । व्य॒या॒म॒सि॒ ॥ पा॒व॒कः । अ॒स्मभ्य॒मित्य॒स्म - भ्य॒म् । शि॒वः । भ॒व॒ ॥ हि॒मस्य॑ । त्वा॒ । ज॒रायु॑णा । अग्ने᳚ । परीति॑ । व्य॒या॒म॒सि॒ ॥ पा॒व॒कः । अ॒स्मभ्य॒मित्य॒स्म - भ्य॒म् । शि॒वः । भ॒व॒ ॥ उपेति॑ ।  \newline


\textbf{Krama Paata} \newline

अश्म॒न्नूर्ज᳚म् । ऊर्ज॒म् पर्व॑ते । पर्व॑ते शिश्रिया॒णाम् । शि॒श्रि॒या॒णाम् ॅवाते᳚ । वाते॑ प॒र्जन्ये᳚ । प॒र्जन्ये॒ वरु॑णस्य । वरु॑णस्य॒ शुष्मे᳚ । शुष्म॒ इति॒ शुष्मे᳚ ॥ अ॒द्भ्य ओष॑धीभ्यः । अ॒द्भ्य इत्य॑त् - भ्यः । ओष॑धीभ्यो॒ वन॒स्पति॑भ्यः । ओष॑धीभ्य॒ इत्योष॑धि - भ्यः॒ । वन॒स्पति॒भ्योऽधि॑ । वन॒स्पति॑भ्य॒ इति॒ वन॒स्पति॑ - भ्यः॒ । अधि॒ सम्भृ॑ताम् । सम्भृ॑ता॒म् ताम् । सम्भृ॑ता॒मिति॒ सं - भृ॒ता॒म् । ताम् नः॑ । न॒ इष᳚म् । इष॒मूर्ज᳚म् । ऊर्ज॑म् धत्त । ध॒त्त॒ म॒रु॒तः॒ । म॒रु॒तः॒ सꣳ॒॒र॒रा॒णाः । सꣳ॒॒र॒रा॒णा इति॑ सम् - र॒रा॒णाः ॥ अश्मꣳ॑स्ते । ते॒ क्षुत् । क्षुद॒मुम् । अ॒मुम् ते᳚ । ते॒ शुक् । शुगृ॑च्छतु । ऋ॒च्छ॒तु॒ यम् । यम् द्वि॒ष्मः । द्वि॒ष्म इति॑ द्वि॒ष्मः ॥ स॒मु॒द्रस्य॑ त्वा । त्वा॒ऽवाक॑या । अ॒वाक॒याऽग्ने᳚ । अग्ने॒ परि॑ । परि॑ व्ययामसि । व्य॒या॒म॒सीति॑ व्ययामसि ॥ पा॒व॒को अ॒स्मभ्य᳚म् । अ॒स्मभ्यꣳ॑ शि॒वः । अ॒स्मभ्य॒मित्य॒स्म - भ्य॒म् । शि॒वो भ॑व । भ॒वेति॑ भव ॥ हि॒मस्य॑ त्वा । त्वा॒ ज॒रायु॑णा । ज॒रायु॒णाऽग्ने᳚ । अग्ने॒ परि॑ । परि॑ व्ययामसि । व्य॒या॒म॒सीति॑ व्ययामसि ॥ पा॒व॒को अ॒स्मभ्य᳚म् । अ॒स्मभ्यꣳ॑ शि॒वः । अ॒स्मभ्य॒मित्य॒स्म - भ्य॒म् । शि॒वो भ॑व । भ॒वेति॑ भव ॥ उप॒ ज्मन्न् \newline

\textbf{Jatai Paata} \newline

1. अश्म॒न् नूर्ज॒ मूर्ज॒ मश्म॒न् नश्म॒न् नूर्ज᳚म् । \newline
2. ऊर्ज॒म् पर्व॑ते॒ पर्व॑त॒ ऊर्ज॒ मूर्ज॒म् पर्व॑ते । \newline
3. पर्व॑ते शिश्रिया॒णाꣳ शि॑श्रिया॒णाम् पर्व॑ते॒ पर्व॑ते शिश्रिया॒णाम् । \newline
4. शि॒श्रि॒या॒णां ॅवाते॒ वाते॑ शिश्रिया॒णाꣳ शि॑श्रिया॒णां ॅवाते᳚ । \newline
5. वाते॑ प॒र्जन्ये॑ प॒र्जन्ये॒ वाते॒ वाते॑ प॒र्जन्ये᳚ । \newline
6. प॒र्जन्ये॒ वरु॑णस्य॒ वरु॑णस्य प॒र्जन्ये॑ प॒र्जन्ये॒ वरु॑णस्य । \newline
7. वरु॑णस्य॒ शुष्मे॒ शुष्मे॒ वरु॑णस्य॒ वरु॑णस्य॒ शुष्मे᳚ । \newline
8. शुष्म॒ इति॒ शुष्मे᳚ । \newline
9. अ॒द्भ्य ओष॑धीभ्य॒ ओष॑धीभ्यो अ॒द्भ्यो अ॒द्भ्य ओष॑धीभ्यः । \newline
10. अ॒द्भ्य इत्य॑त् - भ्यः । \newline
11. ओष॑धीभ्यो॒ वन॒स्पति॑भ्यो॒ वन॒स्पति॑भ्य॒ ओष॑धीभ्य॒ ओष॑धीभ्यो॒ वन॒स्पति॑भ्यः । \newline
12. ओष॑धीभ्य॒ इत्योष॑धि - भ्यः॒ । \newline
13. वन॒स्पति॒भ्यो ऽध्यधि॒ वन॒स्पति॑भ्यो॒ वन॒स्पति॒भ्यो ऽधि॑ । \newline
14. वन॒स्पति॑भ्य॒ इति॒ वन॒स्पति॑ - भ्यः॒ । \newline
15. अधि॒ संभृ॑ताꣳ॒॒ संभृ॑ता॒ मध्यधि॒ संभृ॑ताम् । \newline
16. संभृ॑ता॒म् ताम् ताꣳ संभृ॑ताꣳ॒॒ संभृ॑ता॒म् ताम् । \newline
17. संभृ॑ता॒मिति॒ सं - भृ॒ता॒म् । \newline
18. ताम् नो॑ न॒ स्ताम् ताम् नः॑ । \newline
19. न॒ इष॒ मिष॑म् नो न॒ इष᳚म् । \newline
20. इष॒ मूर्ज॒ मूर्ज॒ मिष॒ मिष॒ मूर्ज᳚म् । \newline
21. ऊर्ज॑म् धत्त ध॒त्तोर्ज॒ मूर्ज॑म् धत्त । \newline
22. ध॒त्त॒ म॒रु॒तो॒ म॒रु॒तो॒ ध॒त्त॒ ध॒त्त॒ म॒रु॒तः॒ । \newline
23. म॒रु॒तः॒ सꣳ॒॒र॒रा॒णाः सꣳ॑ररा॒णा म॑रुतो मरुतः सꣳररा॒णाः । \newline
24. सꣳ॒॒र॒रा॒णा इति॑ सं - र॒रा॒णाः । \newline
25. अश्मꣳ॑ स्ते ते॒ अश्म॒न् नश्मꣳ॑ स्ते । \newline
26. ते॒ क्षुत् क्षुत् ते॑ ते॒ क्षुत् । \newline
27. क्षुद॒मु म॒मुम् क्षुत् क्षुद॒मुम् । \newline
28. अ॒मुम् ते॑ ते अ॒मु म॒मुम् ते᳚ । \newline
29. ते॒ शुक् छुक् ते॑ ते॒ शुक् । \newline
30. शुग् ऋ॑च्छ त्वृच्छतु॒ शुक् छुगृ॑च्छतु । \newline
31. ऋ॒च्छ॒तु॒ यं ॅय मृ॑च्छ त्वृच्छतु॒ यम् । \newline
32. यम् द्वि॒ष्मो द्वि॒ष्मो यं ॅयम् द्वि॒ष्मः । \newline
33. द्वि॒ष्म इति॑ द्वि॒ष्मः । \newline
34. स॒मु॒द्रस्य॑ त्वा त्वा समु॒द्रस्य॑ समु॒द्रस्य॑ त्वा । \newline
35. त्वा॒ ऽवाक॑या॒ ऽवाक॑या त्वा त्वा॒ ऽवाक॑या । \newline
36. अ॒वाक॒या ऽग्ने ऽग्ने॒ ऽवाक॑या॒ ऽवाक॒या ऽग्ने᳚ । \newline
37. अग्ने॒ परि॒ पर्यग्ने ऽग्ने॒ परि॑ । \newline
38. परि॑ व्ययामसि व्ययामसि॒ परि॒ परि॑ व्ययामसि । \newline
39. व्य॒या॒म॒सीति॑ व्ययामसि । \newline
40. पा॒व॒को अ॒स्मभ्य॑ म॒स्मभ्य॑म् पाव॒कः पा॑व॒को अ॒स्मभ्य᳚म् । \newline
41. अ॒स्मभ्यꣳ॑ शि॒वः शि॒वो अ॒स्मभ्य॑ म॒स्मभ्यꣳ॑ शि॒वः । \newline
42. अ॒स्मभ्य॒मित्य॒स्म - भ्य॒म् । \newline
43. शि॒वो भ॑व भव शि॒वः शि॒वो भ॑व । \newline
44. भ॒वेति॑ भव । \newline
45. हि॒मस्य॑ त्वा त्वा हि॒मस्य॑ हि॒मस्य॑ त्वा । \newline
46. त्वा॒ ज॒रायु॑णा ज॒रायु॑णा त्वा त्वा ज॒रायु॑णा । \newline
47. ज॒रायु॒णा ऽग्ने ऽग्ने॑ ज॒रायु॑णा ज॒रायु॒णा ऽग्ने᳚ । \newline
48. अग्ने॒ परि॒ पर्यग्ने ऽग्ने॒ परि॑ । \newline
49. परि॑ व्ययामसि व्ययामसि॒ परि॒ परि॑ व्ययामसि । \newline
50. व्य॒या॒म॒सीति॑ व्ययामसि । \newline
51. पा॒व॒को अ॒स्मभ्य॑ म॒स्मभ्य॑म् पाव॒कः पा॑व॒को अ॒स्मभ्य᳚म् । \newline
52. अ॒स्मभ्यꣳ॑ शि॒वः शि॒वो अ॒स्मभ्य॑ म॒स्मभ्यꣳ॑ शि॒वः । \newline
53. अ॒स्मभ्य॒मित्य॒स्म - भ्य॒म् । \newline
54. शि॒वो भ॑व भव शि॒वः शि॒वो भ॑व । \newline
55. भ॒वेति॑ भव । \newline
56. उप॒ ज्मन् ज्मन् नुपोप॒ ज्मन्न् । \newline

\textbf{Ghana Paata } \newline

1. अश्म॒न् नूर्ज॒ मूर्ज॒ मश्म॒न् नश्म॒न् नूर्ज॒म् पर्व॑ते॒ पर्व॑त॒ ऊर्ज॒ मश्म॒न् नश्म॒न् नूर्ज॒म् पर्व॑ते । \newline
2. ऊर्ज॒म् पर्व॑ते॒ पर्व॑त॒ ऊर्ज॒ मूर्ज॒म् पर्व॑ते शिश्रिया॒णाꣳ शि॑श्रिया॒णाम् पर्व॑त॒ ऊर्ज॒ मूर्ज॒म् पर्व॑ते शिश्रिया॒णाम् । \newline
3. पर्व॑ते शिश्रिया॒णाꣳ शि॑श्रिया॒णाम् पर्व॑ते॒ पर्व॑ते शिश्रिया॒णां ॅवाते॒ वाते॑ शिश्रिया॒णाम् पर्व॑ते॒ पर्व॑ते शिश्रिया॒णां ॅवाते᳚ । \newline
4. शि॒श्रि॒या॒णां ॅवाते॒ वाते॑ शिश्रिया॒णाꣳ शि॑श्रिया॒णां ॅवाते॑ प॒र्जन्ये॑ प॒र्जन्ये॒ वाते॑ शिश्रिया॒णाꣳ शि॑श्रिया॒णां ॅवाते॑ प॒र्जन्ये᳚ । \newline
5. वाते॑ प॒र्जन्ये॑ प॒र्जन्ये॒ वाते॒ वाते॑ प॒र्जन्ये॒ वरु॑णस्य॒ वरु॑णस्य प॒र्जन्ये॒ वाते॒ वाते॑ प॒र्जन्ये॒ वरु॑णस्य । \newline
6. प॒र्जन्ये॒ वरु॑णस्य॒ वरु॑णस्य प॒र्जन्ये॑ प॒र्जन्ये॒ वरु॑णस्य॒ शुष्मे॒ शुष्मे॒ वरु॑णस्य प॒र्जन्ये॑ प॒र्जन्ये॒ वरु॑णस्य॒ शुष्मे᳚ । \newline
7. वरु॑णस्य॒ शुष्मे॒ शुष्मे॒ वरु॑णस्य॒ वरु॑णस्य॒ शुष्मे᳚ । \newline
8. शुष्म॒ इति॒ शुष्मे᳚ । \newline
9. अ॒द्भ्य ओष॑धीभ्य॒ ओष॑धीभ्यो अ॒द्भ्यो अ॒द्भ्य ओष॑धीभ्यो॒ वन॒स्पति॑भ्यो॒ वन॒स्पति॑भ्य॒ ओष॑धीभ्यो अ॒द्भ्यो अ॒द्भ्य ओष॑धीभ्यो॒ वन॒स्पति॑भ्यः । \newline
10. अ॒द्भ्य इत्य॑त् - भ्यः । \newline
11. ओष॑धीभ्यो॒ वन॒स्पति॑भ्यो॒ वन॒स्पति॑भ्य॒ ओष॑धीभ्य॒ ओष॑धीभ्यो॒ वन॒स्पति॒भ्यो ऽध्यधि॒ वन॒स्पति॑भ्य॒ ओष॑धीभ्य॒ ओष॑धीभ्यो॒ वन॒स्पति॒भ्यो ऽधि॑ । \newline
12. ओष॑धीभ्य॒ इत्योष॑धि - भ्यः॒ । \newline
13. वन॒स्पति॒भ्यो ऽध्यधि॒ वन॒स्पति॑भ्यो॒ वन॒स्पति॒भ्यो ऽधि॒ संभृ॑ताꣳ॒॒ संभृ॑ता॒ मधि॒ वन॒स्पति॑भ्यो॒ वन॒स्पति॒भ्यो ऽधि॒ संभृ॑ताम् । \newline
14. वन॒स्पति॑भ्य॒ इति॒ वन॒स्पति॑ - भ्यः॒ । \newline
15. अधि॒ संभृ॑ताꣳ॒॒ संभृ॑ता॒ मध्यधि॒ संभृ॑ता॒म् ताम् ताꣳ संभृ॑ता॒ मध्यधि॒ संभृ॑ता॒म् ताम् । \newline
16. संभृ॑ता॒म् ताम् ताꣳ संभृ॑ताꣳ॒॒ संभृ॑ता॒म् ताम् नो॑ न॒ स्ताꣳ संभृ॑ताꣳ॒॒ संभृ॑ता॒म् ताम् नः॑ । \newline
17. संभृ॑ता॒मिति॒ सं - भृ॒ता॒म् । \newline
18. ताम् नो॑ न॒ स्ताम् ताम् न॒ इष॒ मिष॑म् न॒ स्ताम् ताम् न॒ इष᳚म् । \newline
19. न॒ इष॒ मिष॑म् नो न॒ इष॒ मूर्ज॒ मूर्ज॒ मिष॑म् नो न॒ इष॒ मूर्ज᳚म् । \newline
20. इष॒ मूर्ज॒ मूर्ज॒ मिष॒ मिष॒ मूर्ज॑म् धत्त ध॒त्तोर्ज॒ मिष॒ मिष॒ मूर्ज॑म् धत्त । \newline
21. ऊर्ज॑म् धत्त ध॒त्तोर्ज॒ मूर्ज॑म् धत्त मरुतो मरुतो ध॒त्तोर्ज॒ मूर्ज॑म् धत्त मरुतः । \newline
22. ध॒त्त॒ म॒रु॒तो॒ म॒रु॒तो॒ ध॒त्त॒ ध॒त्त॒ म॒रु॒तः॒ सꣳ॒॒र॒रा॒णाः सꣳ॑ररा॒णा म॑रुतो धत्त धत्त मरुतः सꣳररा॒णाः । \newline
23. म॒रु॒तः॒ सꣳ॒॒र॒रा॒णाः सꣳ॑ररा॒णा म॑रुतो मरुतः सꣳररा॒णाः । \newline
24. सꣳ॒॒र॒रा॒णा इति॑ सं - र॒रा॒णाः । \newline
25. अश्मꣳ॑ स्ते ते॒ अश्म॒न् नश्मꣳ॑स्ते॒ क्षुत् क्षुत् ते॒ अश्म॒न् नश्मꣳ॑स्ते॒ क्षुत् । \newline
26. ते॒ क्षुत् क्षुत् ते॑ ते॒ क्षुद॒मु म॒मुम् क्षुत् ते॑ ते॒ क्षुद॒मुम् । \newline
27. क्षुद॒मु म॒मुम् क्षुत् क्षुद॒मुम् ते॑ ते अ॒मुम् क्षुत् क्षुद॒मुम् ते᳚ । \newline
28. अ॒मुम् ते॑ ते अ॒मु म॒मुम् ते॒ शुक् छुक् ते॑ अ॒मु म॒मुम् ते॒ शुक् । \newline
29. ते॒ शुक् छुक् ते॑ ते॒ शुगृ॑च्छ त्वृच्छतु॒ शुक् ते॑ ते॒ शुगृ॑च्छतु । \newline
30. शुगृ॑च्छ त्वृच्छतु॒ शुक् छुगृ॑च्छतु॒ यं ॅय मृ॑च्छतु॒ शुग् छुगृ॑च्छतु॒ यम् । \newline
31. ऋ॒च्छ॒तु॒ यं ॅय मृ॑च्छ त्वृच्छतु॒ यम् द्वि॒ष्मो द्वि॒ष्मो य मृ॑च्छ त्वृच्छतु॒ यम् द्वि॒ष्मः । \newline
32. यम् द्वि॒ष्मो द्वि॒ष्मो यं ॅयम् द्वि॒ष्मः । \newline
33. द्वि॒ष्म इति॑ द्वि॒ष्मः । \newline
34. स॒मु॒द्रस्य॑ त्वा त्वा समु॒द्रस्य॑ समु॒द्रस्य॑ त्वा॒ ऽवाक॑या॒ ऽवाक॑या त्वा समु॒द्रस्य॑ समु॒द्रस्य॑ त्वा॒ ऽवाक॑या । \newline
35. त्वा॒ ऽवाक॑या॒ ऽवाक॑या त्वा त्वा॒ ऽवाक॒या ऽग्ने ऽग्ने॒ ऽवाक॑या त्वा त्वा॒ ऽवाक॒या ऽग्ने᳚ । \newline
36. अ॒वाक॒या ऽग्ने ऽग्ने॒ ऽवाक॑या॒ ऽवाक॒या ऽग्ने॒ परि॒ पर्यग्ने॒ ऽवाक॑या॒ ऽवाक॒या ऽग्ने॒ परि॑ । \newline
37. अग्ने॒ परि॒ पर्यग्ने ऽग्ने॒ परि॑ व्ययामसि व्ययामसि॒ पर्यग्ने ऽग्ने॒ परि॑ व्ययामसि । \newline
38. परि॑ व्ययामसि व्ययामसि॒ परि॒ परि॑ व्ययामसि । \newline
39. व्य॒या॒म॒सीति॑ व्ययामसि । \newline
40. पा॒व॒को अ॒स्मभ्य॑ म॒स्मभ्य॑म् पाव॒कः पा॑व॒को अ॒स्मभ्यꣳ॑ शि॒वः शि॒वो अ॒स्मभ्य॑म् पाव॒कः पा॑व॒को अ॒स्मभ्यꣳ॑ शि॒वः । \newline
41. अ॒स्मभ्यꣳ॑ शि॒वः शि॒वो अ॒स्मभ्य॑ म॒स्मभ्यꣳ॑ शि॒वो भ॑व भव शि॒वो अ॒स्मभ्य॑ म॒स्मभ्यꣳ॑ शि॒वो भ॑व । \newline
42. अ॒स्मभ्य॒मित्य॒स्म - भ्य॒म् । \newline
43. शि॒वो भ॑व भव शि॒वः शि॒वो भ॑व । \newline
44. भ॒वेति॑ भव । \newline
45. हि॒मस्य॑ त्वा त्वा हि॒मस्य॑ हि॒मस्य॑ त्वा ज॒रायु॑णा ज॒रायु॑णा त्वा हि॒मस्य॑ हि॒मस्य॑ त्वा ज॒रायु॑णा । \newline
46. त्वा॒ ज॒रायु॑णा ज॒रायु॑णा त्वा त्वा ज॒रायु॒णा ऽग्ने ऽग्ने॑ ज॒रायु॑णा त्वा त्वा ज॒रायु॒णा ऽग्ने᳚ । \newline
47. ज॒रायु॒णा ऽग्ने ऽग्ने॑ ज॒रायु॑णा ज॒रायु॒णा ऽग्ने॒ परि॒ पर्यग्ने॑ ज॒रायु॑णा ज॒रायु॒णा ऽग्ने॒ परि॑ । \newline
48. अग्ने॒ परि॒ पर्यग्ने ऽग्ने॒ परि॑ व्ययामसि व्ययामसि॒ पर्यग्ने ऽग्ने॒ परि॑ व्ययामसि । \newline
49. परि॑ व्ययामसि व्ययामसि॒ परि॒ परि॑ व्ययामसि । \newline
50. व्य॒या॒म॒सीति॑ व्ययामसि । \newline
51. पा॒व॒को अ॒स्मभ्य॑ म॒स्मभ्य॑म् पाव॒कः पा॑व॒को अ॒स्मभ्यꣳ॑ शि॒वः शि॒वो अ॒स्मभ्य॑म् पाव॒कः पा॑व॒को अ॒स्मभ्यꣳ॑ शि॒वः । \newline
52. अ॒स्मभ्यꣳ॑ शि॒वः शि॒वो अ॒स्मभ्य॑ म॒स्मभ्यꣳ॑ शि॒वो भ॑व भव शि॒वो अ॒स्मभ्य॑ म॒स्मभ्यꣳ॑ शि॒वो भ॑व । \newline
53. अ॒स्मभ्य॒मित्य॒स्म - भ्य॒म् । \newline
54. शि॒वो भ॑व भव शि॒वः शि॒वो भ॑व । \newline
55. भ॒वेति॑ भव । \newline
56. उप॒ ज्मन् ज्मन् नुपोप॒ ज्मन् नुपोप॒ ज्मन् नुपोप॒ ज्मन् नुप॑ । \newline
\pagebreak
\markright{ TS 4.6.1.2  \hfill https://www.vedavms.in \hfill}

\section{ TS 4.6.1.2 }

\textbf{TS 4.6.1.2 } \newline
\textbf{Samhita Paata} \newline

-ज्मन्नुप॑ वेत॒सेऽव॑त्तरं न॒दीष्वा । अग्ने॑ पि॒त्तम॒पाम॑सि ॥ मण्डू॑कि॒ ताभि॒रा ग॑हि॒ सेमं नो॑ य॒ज्ञ्ं । पा॒व॒कव॑र्णꣳ शि॒वं कृ॑धि ॥ पा॒व॒क आ चि॒तय॑न्त्या कृ॒पा । क्षाम॑न् रुरु॒च उ॒षसो॒ न भा॒नुना᳚ ॥ तूर्व॒न् न याम॒न्नेत॑शस्य॒ नू रण॒ आ यो घृ॒णे । न त॑तृषा॒णो अ॒जरः॑ ॥ अग्ने॑ पावक रो॒चिषा॑ म॒न्द्रया॑ देव जि॒ह्वया᳚ । आ दे॒वान् - [  ] \newline

\textbf{Pada Paata} \newline

ज्मन्न् । उपेति॑ । वे॒त॒से । अव॑त्तर॒मित्यव॑त्-त॒र॒म् । न॒दीषु॑ । आ ॥ अग्ने᳚ । पि॒त्तम् । अ॒पाम् । अ॒सि॒ ॥ मण्डू॑कि । ताभिः॑ । एति॑ । ग॒हि॒ । सा । इ॒मम् । नः॒ । य॒ज्ञ्म् ॥ पा॒व॒कव॑र्ण॒मिति॑ पाव॒क - व॒र्ण॒म् । शि॒वम् । कृ॒धि॒ ॥ पा॒व॒के । एति॑ । चि॒तय॑न्त्या । कृ॒पा ॥ क्षामन्न्॑ । रु॒रु॒चे । उ॒षसः॑ । न । भा॒नुना᳚ ॥ तूर्वन्न्॑ । न । यामन्न्॑ । एत॑शस्य । नु । रणे᳚ । एति॑ । यः । घृ॒णे ॥ न । त॒तृ॒षा॒णः । अ॒जरः॑ ॥ अग्ने᳚ । पा॒व॒क॒ । रो॒चिषा᳚ । म॒न्द्रया᳚ । दे॒व॒ । जि॒ह्वया᳚ ॥ एति॑ । दे॒वान् ।  \newline


\textbf{Krama Paata} \newline

ज्मन्नुप॑ । उप॑ वेत॒से । वे॒त॒सेऽव॑त्तरम् । अव॑त्तरम् न॒दीषु॑ । अव॑त्तर॒मित्यव॑त् - त॒र॒म् । न॒दीष्वा । एत्या ॥ अग्ने॑ पि॒त्तम् । पि॒त्तम॒पाम् । अ॒पाम॑सि । अ॒सीत्य॑सि ॥ मण्डू॑कि॒ ताभिः॑ । ताभि॒रा । आ ग॑हि । ग॒हि॒ सा । सेमम् । इ॒मम् नः॑ । नो॒ य॒ज्ञ्म् । य॒ज्ञ्मिति॑ य॒ज्ञ्म् ॥ पा॒व॒कव॑र्णꣳ शि॒वम् । पा॒व॒कव॑र्ण॒मिति॑ पाव॒क - व॒र्ण॒म् । शि॒वम् कृ॑धि । कृ॒धीति॑ कृधि ॥ पा॒व॒क आ । आ चि॒तय॑न्त्या । चि॒तय॑न्त्या कृ॒पा । कृ॒पेति॑ कृ॒पा ॥ क्षाम॑न् रुरु॒चे । रु॒रु॒च उ॒षसः॑ । उ॒षसो॒ न । न भा॒नुना᳚ । भा॒नुनेति॑ भा॒नुना᳚ ॥ तूर्व॒न् न । न यामन्न्॑ । याम॒न्नेत॑शस्य । एत॑शस्य॒ नु । नू रणे᳚ । रण॒ आ । आ यः । यो घृ॒णे । घृ॒ण इति॑ घृ॒णे ॥ न त॑तृषा॒णः । त॒तृ॒षा॒णो अ॒जरः॑ । अ॒जर॒ इत्य॒जरः॑ ॥ अग्ने॑ पावक । पा॒व॒क॒ रो॒चिषा᳚ । रो॒चिषा॑ म॒न्द्रया᳚ । म॒न्द्रया॑ देव । दे॒व॒ जि॒ह्वया᳚ । जि॒ह्वयेति॑ जि॒ह्वया᳚ ॥ आ दे॒वान् । दे॒वान्. व॑क्षि \newline

\textbf{Jatai Paata} \newline

1. ज्मन् नुपोप॒ ज्मन् ज्मन् नुप॑ । \newline
2. उप॑ वेत॒से वे॑त॒स उपोप॑ वेत॒से । \newline
3. वे॒त॒से ऽव॑त्तर॒ मव॑त्तरं ॅवेत॒से वे॑त॒से ऽव॑त्तरम् । \newline
4. अव॑त्तरम् न॒दीषु॑ न॒दी ष्वव॑त्तर॒ मव॑त्तरम् न॒दीषु॑ । \newline
5. अव॑त्तर॒मित्यव॑त् - त॒र॒म् । \newline
6. न॒दीष्वा न॒दीषु॑ न॒दीष्वा । \newline
7. एत्या । \newline
8. अग्ने॑ पि॒त्तम् पि॒त्त मग्ने ऽग्ने॑ पि॒त्तम् । \newline
9. पि॒त्त म॒पा म॒पाम् पि॒त्तम् पि॒त्त म॒पाम् । \newline
10. अ॒पा म॑स्य स्य॒पा म॒पा म॑सि । \newline
11. अ॒सीत्य॑सि । \newline
12. मण्डू॑कि॒ ताभि॒ स्ताभि॒र् मण्डू॑कि॒ मण्डू॑कि॒ ताभिः॑ । \newline
13. ताभि॒रा ताभि॒ स्ताभि॒रा । \newline
14. आ ग॑हि ग॒ह्या ग॑हि । \newline
15. ग॒हि॒ सा सा ग॑हि गहि॒ सा । \newline
16. सेम मि॒मꣳ सा सेमम् । \newline
17. इ॒मम् नो॑ न इ॒म मि॒मम् नः॑ । \newline
18. नो॒ य॒ज्ञ्ं ॅय॒ज्ञ्म् नो॑ नो य॒ज्ञ्म् । \newline
19. य॒ज्ञ्मिति॑ य॒ज्ञ्म् । \newline
20. पा॒व॒कव॑र्णꣳ शि॒वꣳ शि॒वम् पा॑व॒कव॑र्णम् पाव॒कव॑र्णꣳ शि॒वम् । \newline
21. पा॒व॒कव॑र्ण॒मिति॑ पाव॒क - व॒र्ण॒म् । \newline
22. शि॒वम् कृ॑धि कृधि शि॒वꣳ शि॒वम् कृ॑धि । \newline
23. कृ॒धीति॑ कृधि । \newline
24. पा॒व॒क आ पा॑व॒के पा॑व॒क आ । \newline
25. आ चि॒तय॑न्त्या चि॒तय॒न्त्या ऽऽचि॒तय॑न्त्या । \newline
26. चि॒तय॑न्त्या कृ॒पा कृ॒पा चि॒तय॑न्त्या चि॒तय॑न्त्या कृ॒पा । \newline
27. कृ॒पेति॑ कृ॒पा । \newline
28. क्षाम॑न् रुरु॒चे रु॑रु॒चे क्षाम॒न् क्षाम॑न् रुरु॒चे । \newline
29. रु॒रु॒च उ॒षस॑ उ॒षसो॑ रुरु॒चे रु॑रु॒च उ॒षसः॑ । \newline
30. उ॒षसो॒ न नोषस॑ उ॒षसो॒ न । \newline
31. न भा॒नुना॑ भा॒नुना॒ न न भा॒नुना᳚ । \newline
32. भा॒नुनेति॑ भा॒नुना᳚ । \newline
33. तूर्व॒न् न न तूर्व॒न् तूर्व॒न् न । \newline
34. न याम॒न्॒. याम॒न् न न यामन्न्॑ । \newline
35. याम॒न् नेत॑श॒ स्यैत॑शस्य॒ याम॒न्॒. याम॒न् नेत॑शस्य । \newline
36. एत॑शस्य॒ नुन् वेत॑श॒ स्यैत॑शस्य॒ नु । \newline
37. नू रणे॒ रणे॒ नु नू रणे᳚ । \newline
38. रण॒ आ रणे॒ रण॒ आ । \newline
39. आ यो य आ यः । \newline
40. यो घृ॒णे घृ॒णे यो यो घृ॒णे । \newline
41. घृ॒ण इति॑ घृ॒णे । \newline
42. न त॑तृषा॒ण स्त॑तृषा॒णो न न त॑तृषा॒णः । \newline
43. त॒तृ॒षा॒णो अ॒जरो॑ अ॒जर॑ स्ततृषा॒ण स्त॑तृषा॒णो अ॒जरः॑ । \newline
44. अ॒जर॒ इत्य॒जरः॑ । \newline
45. अग्ने॑ पावक पाव॒काग्ने ऽग्ने॑ पावक । \newline
46. पा॒व॒क॒ रो॒चिषा॑ रो॒चिषा॑ पावक पावक रो॒चिषा᳚ । \newline
47. रो॒चिषा॑ म॒न्द्रया॑ म॒न्द्रया॑ रो॒चिषा॑ रो॒चिषा॑ म॒न्द्रया᳚ । \newline
48. म॒न्द्रया॑ देव देव म॒न्द्रया॑ म॒न्द्रया॑ देव । \newline
49. दे॒व॒ जि॒ह्वया॑ जि॒ह्वया॑ देव देव जि॒ह्वया᳚ । \newline
50. जि॒ह्वयेति॑ जि॒ह्वया᳚ । \newline
51. आ दे॒वान् दे॒वा ना दे॒वान् । \newline
52. दे॒वान्. व॑क्षि वक्षि दे॒वान् दे॒वान्. व॑क्षि । \newline

\textbf{Ghana Paata } \newline

1. ज्मन् नुपोप॒ ज्मन् ज्मन् नुप॑ वेत॒से वे॑त॒स उप॒ ज्मन् ज्मन् नुप॑ वेत॒से । \newline
2. उप॑ वेत॒से वे॑त॒स उपोप॑ वेत॒से ऽव॑त्तर॒ मव॑त्तरं ॅवेत॒स उपोप॑ वेत॒से ऽव॑त्तरम् । \newline
3. वे॒त॒से ऽव॑त्तर॒ मव॑त्तरं ॅवेत॒से वे॑त॒से ऽव॑त्तरम् न॒दीषु॑ न॒दी ष्वव॑त्तरं ॅवेत॒से वे॑त॒से ऽव॑त्तरम् न॒दीषु॑ । \newline
4. अव॑त्तरम् न॒दीषु॑ न॒दी ष्वव॑त्तर॒ मव॑त्तरम् न॒दीष्वा न॒दी ष्वव॑त्तर॒ मव॑त्तरम् न॒दीष्वा । \newline
5. अव॑त्तर॒मित्यव॑त् - त॒र॒म् । \newline
6. न॒दीष्वा न॒दीषु॑ न॒दीष्वा । \newline
7. एत्या । \newline
8. अग्ने॑ पि॒त्तम् पि॒त्त मग्ने ऽग्ने॑ पि॒त्त म॒पा म॒पाम् पि॒त्त मग्ने ऽग्ने॑ पि॒त्त म॒पाम् । \newline
9. पि॒त्त म॒पा म॒पाम् पि॒त्तम् पि॒त्त म॒पा म॑स्य स्य॒पाम् पि॒त्तम् पि॒त्त म॒पा म॑सि । \newline
10. अ॒पा म॑स्य स्य॒पा म॒पा म॑सि । \newline
11. अ॒सीत्य॑सि । \newline
12. मण्डू॑कि॒ ताभि॒ स्ताभि॒र् मण्डू॑कि॒ मण्डू॑कि॒ ताभि॒रा ताभि॒र् मण्डू॑कि॒ मण्डू॑कि॒ ताभि॒रा । \newline
13. ताभि॒रा ताभि॒ स्ताभि॒रा ग॑हि ग॒ह्या ताभि॒ स्ताभि॒रा ग॑हि । \newline
14. आ ग॑हि ग॒ह्या ग॑हि॒ सा सा ग॒ह्या ग॑हि॒ सा । \newline
15. ग॒हि॒ सा सा ग॑हि गहि॒ सेम मि॒मꣳ सा ग॑हि गहि॒ सेमम् । \newline
16. सेम मि॒मꣳ सा सेमम् नो॑ न इ॒मꣳ सा सेमम् नः॑ । \newline
17. इ॒मम् नो॑ न इ॒म मि॒मम् नो॑ य॒ज्ञ्ं ॅय॒ज्ञ्म् न॑ इ॒म मि॒मम् नो॑ य॒ज्ञ्म् । \newline
18. नो॒ य॒ज्ञ्ं ॅय॒ज्ञ्म् नो॑ नो य॒ज्ञ्म् । \newline
19. य॒ज्ञ्मिति॑ य॒ज्ञ्म् । \newline
20. पा॒व॒कव॑र्णꣳ शि॒वꣳ शि॒वम् पा॑व॒कव॑र्णम् पाव॒कव॑र्णꣳ शि॒वम् कृ॑धि कृधि शि॒वम् पा॑व॒कव॑र्णम् पाव॒कव॑र्णꣳ शि॒वम् कृ॑धि । \newline
21. पा॒व॒कव॑र्ण॒मिति॑ पाव॒क - व॒र्ण॒म् । \newline
22. शि॒वम् कृ॑धि कृधि शि॒वꣳ शि॒वम् कृ॑धि । \newline
23. कृ॒धीति॑ कृधि । \newline
24. पा॒व॒क आ पा॑व॒के पा॑व॒क आ चि॒तय॑न्त्या चि॒तय॒न्त्या ऽऽपा॑व॒के पा॑व॒क आ चि॒तय॑न्त्या । \newline
25. आ चि॒तय॑न्त्या चि॒तय॒न्त्या ऽऽचि॒तय॑न्त्या कृ॒पा कृ॒पा चि॒तय॒न्त्या ऽऽचि॒तय॑न्त्या कृ॒पा । \newline
26. चि॒तय॑न्त्या कृ॒पा कृ॒पा चि॒तय॑न्त्या चि॒तय॑न्त्या कृ॒पा । \newline
27. कृ॒पेति॑ कृ॒पा । \newline
28. क्षाम॑न् रुरु॒चे रु॑रु॒चे क्षाम॒न् क्षाम॑न् रुरु॒च उ॒षस॑ उ॒षसो॑ रुरु॒चे क्षाम॒न् क्षाम॑न् रुरु॒च उ॒षसः॑ । \newline
29. रु॒रु॒च उ॒षस॑ उ॒षसो॑ रुरु॒चे रु॑रु॒च उ॒षसो॒ न नोषसो॑ रुरु॒चे रु॑रु॒च उ॒षसो॒ न । \newline
30. उ॒षसो॒ न नोषस॑ उ॒षसो॒ न भा॒नुना॑ भा॒नुना॒ नोषस॑ उ॒षसो॒ न भा॒नुना᳚ । \newline
31. न भा॒नुना॑ भा॒नुना॒ न न भा॒नुना᳚ । \newline
32. भा॒नुनेति॑ भा॒नुना᳚ । \newline
33. तूर्व॒न् न न तूर्व॒न् तूर्व॒न् न याम॒न्॒. याम॒न् न तूर्व॒न् तूर्व॒न् न यामन्न्॑ । \newline
34. न याम॒न्॒. याम॒न् न न याम॒न् नेत॑श॒ स्यैत॑शस्य॒ याम॒न् न न याम॒न् नेत॑शस्य । \newline
35. याम॒न् नेत॑श॒ स्यैत॑शस्य॒ याम॒न्॒. याम॒न् नेत॑शस्य॒ नुन् वेत॑शस्य॒ याम॒न्॒. याम॒न् नेत॑शस्य॒ नु । \newline
36. एत॑शस्य॒ नुन् वेत॑श॒ स्यैत॑शस्य॒ नू रणे॒ रणे॒न् वेत॑श॒स्यै त॑शस्य॒ नू रणे᳚ । \newline
37. नू रणे॒ रणे॒ नु नू रण॒ आ रणे॒ नु नू रण॒ आ । \newline
38. रण॒ आ रणे॒ रण॒ आ यो य आ रणे॒ रण॒ आ यः । \newline
39. आ यो य आ यो घृ॒णे घृ॒णे य आ यो घृ॒णे । \newline
40. यो घृ॒णे घृ॒णे यो यो घृ॒णे । \newline
41. घृ॒ण इति॑ घृ॒णे । \newline
42. न त॑तृषा॒ण स्त॑तृषा॒णो न न त॑तृषा॒णो अ॒जरो॑ अ॒जर॑ स्ततृषा॒णो न न त॑तृषा॒णो अ॒जरः॑ । \newline
43. त॒तृ॒षा॒णो अ॒जरो॑ अ॒जर॑ स्ततृषा॒ण स्त॑तृषा॒णो अ॒जरः॑ । \newline
44. अ॒जर॒ इत्य॒जरः॑ । \newline
45. अग्ने॑ पावक पाव॒काग्ने ऽग्ने॑ पावक रो॒चिषा॑ रो॒चिषा॑ पाव॒काग्ने ऽग्ने॑ पावक रो॒चिषा᳚ । \newline
46. पा॒व॒क॒ रो॒चिषा॑ रो॒चिषा॑ पावक पावक रो॒चिषा॑ म॒न्द्रया॑ म॒न्द्रया॑ रो॒चिषा॑ पावक पावक रो॒चिषा॑ म॒न्द्रया᳚ । \newline
47. रो॒चिषा॑ म॒न्द्रया॑ म॒न्द्रया॑ रो॒चिषा॑ रो॒चिषा॑ म॒न्द्रया॑ देव देव म॒न्द्रया॑ रो॒चिषा॑ रो॒चिषा॑ म॒न्द्रया॑ देव । \newline
48. म॒न्द्रया॑ देव देव म॒न्द्रया॑ म॒न्द्रया॑ देव जि॒ह्वया॑ जि॒ह्वया॑ देव म॒न्द्रया॑ म॒न्द्रया॑ देव जि॒ह्वया᳚ । \newline
49. दे॒व॒ जि॒ह्वया॑ जि॒ह्वया॑ देव देव जि॒ह्वया᳚ । \newline
50. जि॒ह्वयेति॑ जि॒ह्वया᳚ । \newline
51. आ दे॒वान् दे॒वाना दे॒वान्. व॑क्षि वक्षि दे॒वाना दे॒वान्. व॑क्षि । \newline
52. दे॒वान्. व॑क्षि वक्षि दे॒वान् दे॒वान्. व॑क्षि॒ यक्षि॒ यक्षि॑ वक्षि दे॒वान् दे॒वान्. व॑क्षि॒ यक्षि॑ । \newline
\pagebreak
\markright{ TS 4.6.1.3  \hfill https://www.vedavms.in \hfill}

\section{ TS 4.6.1.3 }

\textbf{TS 4.6.1.3 } \newline
\textbf{Samhita Paata} \newline

व॑क्षि॒ यक्षि॑ च ॥ स नः॑ पावक दीदि॒वोऽग्ने॑ दे॒वाꣳ इ॒हाऽऽ व॑ह । उप॑ य॒ज्ञ्ꣳ ह॒विश्च॑ नः ॥ अ॒पामि॒दं न्यय॑नꣳ समु॒द्रस्य॑ नि॒वेश॑नं । अ॒न्यं ते॑ अ॒स्मत् त॑पन्तु हे॒तयः॑ पाव॒को अ॒स्मभ्यꣳ॑ शि॒वो भ॑व ॥ नम॑स्ते॒ हर॑से शो॒चिषे॒ नम॑स्ते अस्त्व॒र्चिषे᳚ । अ॒न्यं ते॑ अ॒स्मत् त॑पन्तु हे॒तयः॑ पाव॒को अ॒स्मभ्यꣳ॑ शि॒वो भ॑व ॥ नृ॒षदे॒ वड॑ - [  ] \newline

\textbf{Pada Paata} \newline

व॒क्षि॒ । यक्षि॑ । च॒ ॥ सः । नः॒ । पा॒व॒क॒ । दी॒दि॒वः॒ । अग्ने᳚ । दे॒वान् । इ॒ह । एति॑ । व॒ह॒ ॥ उपेति॑ । य॒ज्ञ्म् । ह॒विः । च॒ । नः॒ ॥ अ॒पाम् । इ॒दम् । न्यय॑न॒मिति॑ नि - अय॑नम् । स॒मु॒द्रस्य॑ । नि॒वेश॑न॒मिति॑ नि - वेश॑नम् ॥ अ॒न्यम् । ते॒ । अ॒स्मत् । त॒प॒न्तु॒ । हे॒तयः॑ । पा॒व॒कः । अ॒स्मभ्य॒मित्य॒स्म - भ्य॒म् । शि॒वः । भ॒व॒ ॥ नमः॑ । ते॒ । हर॑से । शो॒चिषे᳚ । नमः॑ । ते॒ । अ॒स्तु॒ । अ॒र्चिषे᳚ ॥ अ॒न्यम् । ते॒ । अ॒स्मत् । त॒प॒न्तु॒ । हे॒तयः॑ । पा॒व॒कः । अ॒स्मभ्य॒मित्य॒स्म - भ्य॒म् । शि॒वः । भ॒व॒ ॥ नृ॒षद॒ इति॑ नृ - सदे᳚ । वट् ।  \newline


\textbf{Krama Paata} \newline

व॒क्षि॒ यक्षि॑ । यक्षि॑ च । चेति॑ च ॥ स नः॑ । नः॒ पा॒व॒क॒ । पा॒व॒क॒ दी॒दि॒वः॒ । दी॒दि॒वोऽग्ने᳚ । अग्ने॑ दे॒वान् । दे॒वाꣳ इ॒ह । इ॒हा । आ व॑ह । व॒हेति॑ वह ॥ उप॑ य॒ज्ञ्म् । य॒ज्ञ्ꣳ ह॒विः । ह॒विश्च॑ । च॒ नः॒ । न॒ इति॑ नः ॥ अ॒पामि॒दम् । इ॒दम् न्यय॑नम् । न्यय॑नꣳ समु॒द्रस्य॑ । न्यय॑न॒मिति॑ नि - अय॑नम् । स॒मु॒द्रस्य॑ नि॒वेश॑नम् । नि॒वेश॑न॒मिति॑ नि - वेश॑नम् ॥ अ॒न्यम् ते᳚ । ते॒ अ॒स्मत् । अ॒स्मत् त॑पन्तु । त॒प॒न्तु॒ हे॒तयः॑ । हे॒तयः॑ पाव॒कः । पा॒व॒को अ॒स्मभ्य᳚म् । अ॒स्मभ्यꣳ॑ शि॒वः । अ॒स्मभ्य॒मित्य॒स्म - भ्य॒म् । शि॒वो भ॑व । भ॒वेति॑ भव ॥ नम॑स्ते । ते॒ हर॑से । हर॑से शो॒चिषे᳚ । शो॒चिषे॒ नमः॑ । नम॑स्ते । ते॒ अ॒स्तु॒ । अ॒स्त्व॒र्चिषे᳚ । अ॒र्चिष॒ इत्य॒र्चिषे᳚ ॥ अ॒न्यम् ते᳚ । ते॒ अ॒स्मत् । अ॒स्मत् त॑पन्तु । त॒प॒न्तु॒ हे॒तयः॑ । हे॒तयः॑ पाव॒कः । पा॒व॒को अ॒स्मभ्य᳚म् । अ॒स्मभ्यꣳ॑ शि॒वः । अ॒स्मभ्य॒मित्य॒स्म - भ्य॒म् । शि॒वो भ॑व । भ॒वेति॑ भव ॥ नृ॒षदे॒ वट् । नृ॒षद॒ इति॑ नृ - सदे᳚ । वड॑फ्सु॒षदे᳚ \newline

\textbf{Jatai Paata} \newline

1. व॒क्षि॒ यक्षि॒ यक्षि॑ वक्षि वक्षि॒ यक्षि॑ । \newline
2. यक्षि॑ च च॒ यक्षि॒ यक्षि॑ च । \newline
3. चेति॑ च । \newline
4. स नो॑ नः॒ स स नः॑ । \newline
5. नः॒ पा॒व॒क॒ पा॒व॒क॒ नो॒ नः॒ पा॒व॒क॒ । \newline
6. पा॒व॒क॒ दी॒दि॒वो॒ दी॒दि॒वः॒ पा॒व॒क॒ पा॒व॒क॒ दी॒दि॒वः॒ । \newline
7. दी॒दि॒वो ऽग्ने ऽग्ने॑ दीदिवो दीदि॒वो ऽग्ने᳚ । \newline
8. अग्ने॑ दे॒वान् दे॒वाꣳ अग्ने ऽग्ने॑ दे॒वान् । \newline
9. दे॒वाꣳ इ॒हे ह दे॒वान् दे॒वाꣳ इ॒ह । \newline
10. इ॒हेहे हा । \newline
11. आ व॑ह व॒हा व॑ह । \newline
12. व॒हेति॑ वह । \newline
13. उप॑ य॒ज्ञ्ं ॅय॒ज्ञ् मुपोप॑ य॒ज्ञ्म् । \newline
14. य॒ज्ञ्ꣳ ह॒विर्. ह॒विर् य॒ज्ञ्ं ॅय॒ज्ञ्ꣳ ह॒विः । \newline
15. ह॒विश्च॑ च ह॒विर्. ह॒विश्च॑ । \newline
16. च॒ नो॒ न॒श्च॒ च॒ नः॒ । \newline
17. न॒ इति॑ नः । \newline
18. अ॒पा मि॒द मि॒द म॒पा म॒पा मि॒दम् । \newline
19. इ॒दम् न्यय॑न॒म् न्यय॑न मि॒द मि॒दम् न्यय॑नम् । \newline
20. न्यय॑नꣳ समु॒द्रस्य॑ समु॒द्रस्य॒ न्यय॑न॒म् न्यय॑नꣳ समु॒द्रस्य॑ । \newline
21. न्यय॑न॒मिति॑ नि - अय॑नम् । \newline
22. स॒मु॒द्रस्य॑ नि॒वेश॑नम् नि॒वेश॑नꣳ समु॒द्रस्य॑ समु॒द्रस्य॑ नि॒वेश॑नम् । \newline
23. नि॒वेश॑न॒मिति॑ नि - वेश॑नम् । \newline
24. अ॒न्यम् ते॑ ते अ॒न्य म॒न्यम् ते᳚ । \newline
25. ते॒ अ॒स्म द॒स्मत् ते॑ ते अ॒स्मत् । \newline
26. अ॒स्मत् त॑पन्तु तप न्त्व॒स्म द॒स्मत् त॑पन्तु । \newline
27. त॒प॒न्तु॒ हे॒तयो॑ हे॒तय॑ स्तपन्तु तपन्तु हे॒तयः॑ । \newline
28. हे॒तयः॑ पाव॒कः पा॑व॒को हे॒तयो॑ हे॒तयः॑ पाव॒कः । \newline
29. पा॒व॒को अ॒स्मभ्य॑ म॒स्मभ्य॑म् पाव॒कः पा॑व॒को अ॒स्मभ्य᳚म् । \newline
30. अ॒स्मभ्यꣳ॑ शि॒वः शि॒वो अ॒स्मभ्य॑ म॒स्मभ्यꣳ॑ शि॒वः । \newline
31. अ॒स्मभ्य॒मित्य॒स्म - भ्य॒म् । \newline
32. शि॒वो भ॑व भव शि॒वः शि॒वो भ॑व । \newline
33. भ॒वेति॑ भव । \newline
34. नम॑ स्ते ते॒ नमो॒ नम॑ स्ते । \newline
35. ते॒ हर॑से॒ हर॑से ते ते॒ हर॑से । \newline
36. हर॑से शो॒चिषे॑ शो॒चिषे॒ हर॑से॒ हर॑से शो॒चिषे᳚ । \newline
37. शो॒चिषे॒ नमो॒ नमः॑ शो॒चिषे॑ शो॒चिषे॒ नमः॑ । \newline
38. नम॑ स्ते ते॒ नमो॒ नम॑ स्ते । \newline
39. ते॒ अ॒स्त्व॒स्तु॒ ते॒ ते॒ अ॒स्तु॒ । \newline
40. अ॒स्त्व॒र्चिषे॑ अ॒र्चिषे॑ अस्त्व स्त्व॒र्चिषे᳚ । \newline
41. अ॒र्चिष॒ इत्य॒र्चिषे᳚ । \newline
42. अ॒न्यम् ते॑ ते अ॒न्य म॒न्यम् ते᳚ । \newline
43. ते॒ अ॒स्म द॒स्मत् ते॑ ते अ॒स्मत् । \newline
44. अ॒स्मत् त॑पन्तु तप न्त्व॒स्म द॒स्मत् त॑पन्तु । \newline
45. त॒प॒न्तु॒ हे॒तयो॑ हे॒तय॑ स्तपन्तु तपन्तु हे॒तयः॑ । \newline
46. हे॒तयः॑ पाव॒कः पा॑व॒को हे॒तयो॑ हे॒तयः॑ पाव॒कः । \newline
47. पा॒व॒को अ॒स्मभ्य॑ म॒स्मभ्य॑म् पाव॒कः पा॑व॒को अ॒स्मभ्य᳚म् । \newline
48. अ॒स्मभ्यꣳ॑ शि॒वः शि॒वो अ॒स्मभ्य॑ म॒स्मभ्यꣳ॑ शि॒वः । \newline
49. अ॒स्मभ्य॒मित्य॒स्म - भ्य॒म् । \newline
50. शि॒वो भ॑व भव शि॒वः शि॒वो भ॑व । \newline
51. भ॒वेति॑ भव । \newline
52. नृ॒षदे॒ वड् वण् णृ॒षदे॑ नृ॒षदे॒ वट् । \newline
53. नृ॒षद॒ इति॑ नृ - सदे᳚ । \newline
54. वड॑फ्सु॒षदे॑ अफ्सु॒षदे॒ वड् वड॑फ्सु॒षदे᳚ । \newline

\textbf{Ghana Paata } \newline

1. व॒क्षि॒ यक्षि॒ यक्षि॑ वक्षि वक्षि॒ यक्षि॑ च च॒ यक्षि॑ वक्षि वक्षि॒ यक्षि॑ च । \newline
2. यक्षि॑ च च॒ यक्षि॒ यक्षि॑ च । \newline
3. चेति॑ च । \newline
4. स नो॑ नः॒ स स नः॑ पावक पावक नः॒ स स नः॑ पावक । \newline
5. नः॒ पा॒व॒क॒ पा॒व॒क॒ नो॒ नः॒ पा॒व॒क॒ दी॒दि॒वो॒ दी॒दि॒वः॒ पा॒व॒क॒ नो॒ नः॒ पा॒व॒क॒ दी॒दि॒वः॒ । \newline
6. पा॒व॒क॒ दी॒दि॒वो॒ दी॒दि॒वः॒ पा॒व॒क॒ पा॒व॒क॒ दी॒दि॒वो ऽग्ने ऽग्ने॑ दीदिवः पावक पावक दीदि॒वो ऽग्ने᳚ । \newline
7. दी॒दि॒वो ऽग्ने ऽग्ने॑ दीदिवो दीदि॒वो ऽग्ने॑ दे॒वान् दे॒वाꣳ अग्ने॑ दीदिवो दीदि॒वो ऽग्ने॑ दे॒वान् । \newline
8. अग्ने॑ दे॒वान् दे॒वाꣳ अग्ने ऽग्ने॑ दे॒वाꣳ इ॒हेह दे॒वाꣳ अग्ने ऽग्ने॑ दे॒वाꣳ इ॒ह । \newline
9. दे॒वाꣳ इ॒हेह दे॒वान् दे॒वाꣳ इ॒हेह दे॒वान् दे॒वाꣳ इ॒हा । \newline
10. इ॒हेहे हा व॑ह व॒हेहे हा व॑ह । \newline
11. आ व॑ह व॒हा व॑ह । \newline
12. व॒हेति॑ वह । \newline
13. उप॑ य॒ज्ञ्ं ॅय॒ज्ञ् मुपोप॑ य॒ज्ञ्ꣳ ह॒विर्. ह॒विर् य॒ज्ञ् मुपोप॑ य॒ज्ञ्ꣳ ह॒विः । \newline
14. य॒ज्ञ्ꣳ ह॒विर्. ह॒विर् य॒ज्ञ्ं ॅय॒ज्ञ्ꣳ ह॒विश्च॑ च ह॒विर् य॒ज्ञ्ं ॅय॒ज्ञ्ꣳ ह॒विश्च॑ । \newline
15. ह॒विश्च॑ च ह॒विर्. ह॒विश्च॑ नो नश्च ह॒विर्. ह॒विश्च॑ नः । \newline
16. च॒ नो॒ न॒श्च॒ च॒ नः॒ । \newline
17. न॒ इति॑ नः । \newline
18. अ॒पा मि॒द मि॒द म॒पा म॒पा मि॒दम् न्यय॑न॒म् न्यय॑न मि॒द म॒पा म॒पा मि॒दम् न्यय॑नम् । \newline
19. इ॒दन् न्यय॑न॒म् न्यय॑न मि॒द मि॒दम् न्यय॑नꣳ समु॒द्रस्य॑ समु॒द्रस्य॒ न्यय॑न मि॒द मि॒दम् न्यय॑नꣳ समु॒द्रस्य॑ । \newline
20. न्यय॑नꣳ समु॒द्रस्य॑ समु॒द्रस्य॒ न्यय॑न॒म् न्यय॑नꣳ समु॒द्रस्य॑ नि॒वेश॑नम् नि॒वेश॑नꣳ समु॒द्रस्य॒ न्यय॑न॒म् न्यय॑नꣳ समु॒द्रस्य॑ नि॒वेश॑नम् । \newline
21. न्यय॑न॒मिति॑ नि - अय॑नम् । \newline
22. स॒मु॒द्रस्य॑ नि॒वेश॑नम् नि॒वेश॑नꣳ समु॒द्रस्य॑ समु॒द्रस्य॑ नि॒वेश॑नम् । \newline
23. नि॒वेश॑न॒मिति॑ नि - वेश॑नम् । \newline
24. अ॒न्यम् ते॑ ते अ॒न्य म॒न्यम् ते॑ अ॒स्म द॒स्मत् ते॑ अ॒न्य म॒न्यम् ते॑ अ॒स्मत् । \newline
25. ते॒ अ॒स्म द॒स्मत् ते॑ ते अ॒स्मत् त॑पन्तु तपन् त्व॒स्मत् ते॑ ते अ॒स्मत् त॑पन्तु । \newline
26. अ॒स्मत् त॑पन्तु तपन् त्व॒स्म द॒स्मत् त॑पन्तु हे॒तयो॑ हे॒तय॑ स्तपन् त्व॒स्म द॒स्मत् त॑पन्तु हे॒तयः॑ । \newline
27. त॒प॒न्तु॒ हे॒तयो॑ हे॒तय॑ स्तपन्तु तपन्तु हे॒तयः॑ पाव॒कः पा॑व॒को हे॒तय॑ स्तपन्तु तपन्तु हे॒तयः॑ पाव॒कः । \newline
28. हे॒तयः॑ पाव॒कः पा॑व॒को हे॒तयो॑ हे॒तयः॑ पाव॒को अ॒स्मभ्य॑ म॒स्मभ्य॑म् पाव॒को हे॒तयो॑ हे॒तयः॑ पाव॒को अ॒स्मभ्य᳚म् । \newline
29. पा॒व॒को अ॒स्मभ्य॑ म॒स्मभ्य॑म् पाव॒कः पा॑व॒को अ॒स्मभ्यꣳ॑ शि॒वः शि॒वो अ॒स्मभ्य॑म् पाव॒कः पा॑व॒को अ॒स्मभ्यꣳ॑ शि॒वः । \newline
30. अ॒स्मभ्यꣳ॑ शि॒वः शि॒वो अ॒स्मभ्य॑ म॒स्मभ्यꣳ॑ शि॒वो भ॑व भव शि॒वो अ॒स्मभ्य॑ म॒स्मभ्यꣳ॑ शि॒वो भ॑व । \newline
31. अ॒स्मभ्य॒मित्य॒स्म - भ्य॒म् । \newline
32. शि॒वो भ॑व भव शि॒वः शि॒वो भ॑व । \newline
33. भ॒वेति॑ भव । \newline
34. नम॑ स्ते ते॒ नमो॒ नम॑ स्ते॒ हर॑से॒ हर॑से ते॒ नमो॒ नम॑ स्ते॒ हर॑से । \newline
35. ते॒ हर॑से॒ हर॑से ते ते॒ हर॑से शो॒चिषे॑ शो॒चिषे॒ हर॑से ते ते॒ हर॑से शो॒चिषे᳚ । \newline
36. हर॑से शो॒चिषे॑ शो॒चिषे॒ हर॑से॒ हर॑से शो॒चिषे॒ नमो॒ नमः॑ शो॒चिषे॒ हर॑से॒ हर॑से शो॒चिषे॒ नमः॑ । \newline
37. शो॒चिषे॒ नमो॒ नमः॑ शो॒चिषे॑ शो॒चिषे॒ नम॑ स्ते ते॒ नमः॑ शो॒चिषे॑ शो॒चिषे॒ नम॑ स्ते । \newline
38. नम॑ स्ते ते॒ नमो॒ नम॑ स्ते अस्त्वस्तु ते॒ नमो॒ नम॑ स्ते अस्तु । \newline
39. ते॒ अ॒स्त्व॒स्तु॒ ते॒ ते॒ अ॒स्त्व॒र्चिषे॑ अ॒र्चिषे॑ अस्तु ते ते अस्त्व॒र्चिषे᳚ । \newline
40. अ॒स्त्व॒र्चिषे॑ अ॒र्चिषे॑ अस्त्व स्त्व॒र्चिषे᳚ । \newline
41. अ॒र्चिष॒ इत्य॒र्चिषे᳚ । \newline
42. अ॒न्यम् ते॑ ते अ॒न्य म॒न्यम् ते॑ अ॒स्म द॒स्मत् ते॑ अ॒न्य म॒न्यम् ते॑ अ॒स्मत् । \newline
43. ते॒ अ॒स्म द॒स्मत् ते॑ ते अ॒स्मत् त॑पन्तु तपन् त्व॒स्मत् ते॑ ते अ॒स्मत् त॑पन्तु । \newline
44. अ॒स्मत् त॑पन्तु तपन् त्व॒स्म द॒स्मत् त॑पन्तु हे॒तयो॑ हे॒तय॑ स्तपन् त्व॒स्म द॒स्मत् त॑पन्तु हे॒तयः॑ । \newline
45. त॒प॒न्तु॒ हे॒तयो॑ हे॒तय॑ स्तपन्तु तपन्तु हे॒तयः॑ पाव॒कः पा॑व॒को हे॒तय॑ स्तपन्तु तपन्तु हे॒तयः॑ पाव॒कः । \newline
46. हे॒तयः॑ पाव॒कः पा॑व॒को हे॒तयो॑ हे॒तयः॑ पाव॒को अ॒स्मभ्य॑ म॒स्मभ्य॑म् पाव॒को हे॒तयो॑ हे॒तयः॑ पाव॒को अ॒स्मभ्य᳚म् । \newline
47. पा॒व॒को अ॒स्मभ्य॑ म॒स्मभ्य॑म् पाव॒कः पा॑व॒को अ॒स्मभ्यꣳ॑ शि॒वः शि॒वो अ॒स्मभ्य॑म् पाव॒कः पा॑व॒को अ॒स्मभ्यꣳ॑ शि॒वः । \newline
48. अ॒स्मभ्यꣳ॑ शि॒वः शि॒वो अ॒स्मभ्य॑ म॒स्मभ्यꣳ॑ शि॒वो भ॑व भव शि॒वो अ॒स्मभ्य॑ म॒स्मभ्यꣳ॑ शि॒वो भ॑व । \newline
49. अ॒स्मभ्य॒मित्य॒स्म - भ्य॒म् । \newline
50. शि॒वो भ॑व भव शि॒वः शि॒वो भ॑व । \newline
51. भ॒वेति॑ भव । \newline
52. नृ॒षदे॒ वड् वण् णृ॒षदे॑ नृ॒षदे॒ वड॑फ्सु॒षदे॑ अफ्सु॒षदे॒ वण् णृ॒षदे॑ नृ॒षदे॒ वड॑फ्सु॒षदे᳚ । \newline
53. नृ॒षद॒ इति॑ नृ - सदे᳚ । \newline
54. वड॑फ्सु॒षदे॑ अफ्सु॒षदे॒ वड् वड॑फ्सु॒षदे॒ वड् वड॑फ्सु॒षदे॒ वड् वड॑फ्सु॒षदे॒ वट् । \newline
\pagebreak
\markright{ TS 4.6.1.4  \hfill https://www.vedavms.in \hfill}

\section{ TS 4.6.1.4 }

\textbf{TS 4.6.1.4 } \newline
\textbf{Samhita Paata} \newline

-फ्सु॒षदे॒ वड् व॑न॒सदे॒ वड् ब॑र्.हि॒षदे॒ वट्थ् सु॑व॒र्विदे॒ वट् ॥ ये दे॒वा दे॒वानां᳚ ॅय॒ज्ञिया॑ य॒ज्ञिया॑नाꣳ संॅवथ् स॒रीण॒मुप॑ भा॒गमास॑ते । अ॒हु॒तादो॑ ह॒विषो॑ य॒ज्ञे अ॒स्मिन्थ् स्व॒यं जु॑हुद्ध्वं॒ मधु॑नो घृ॒तस्य॑ ॥ ये दे॒वा दे॒वेष्वधि॑ देव॒त्वमाय॒न्॒ ये ब्रह्म॑णः पुर ए॒तारो॑ अ॒स्य । येभ्यो॒ नर्ते पव॑ते॒ धाम॒ किं च॒न न ते दि॒वो न पृ॑थि॒व्या अधि॒ स्नुषु॑ ॥ प्रा॒ण॒दा - [  ] \newline

\textbf{Pada Paata} \newline

अ॒फ्सु॒षद॒ इत्य॑फ्सु - सदे᳚ । वट् । व॒न॒सद॒ इति॑ वन - सदे᳚ । वट् । ब॒र्॒.हि॒षद॒ इति॑ बर्.हि - सदे᳚ । वट् । सु॒व॒र्विद॒ इति॑ सुवः - विदे᳚ । वट् ॥ ये । दे॒वाः । दे॒वाना᳚म् । य॒ज्ञियाः᳚ । य॒ज्ञिया॑नाम् । सं॒ॅव॒थ्स॒रीण॒मिति॑ सं - व॒थ्स॒रीण᳚म् । उपेति॑ । भा॒गम् । आस॑ते ॥ अ॒हु॒ताद॒ इत्य॑हुत - अदः॑ । ह॒विषः॑ । य॒ज्ञे । अ॒स्मिन्न् । स्व॒यम् । जु॒हु॒द्ध्व॒म् । मधु॑नः । घृ॒तस्य॑ ॥ ये । दे॒वाः । दे॒वेषू॑ । अधीति॑ । दे॒व॒त्वमिति॑ देव - त्वम् । आयन्न्॑ । ये । ब्रह्म॑णः । पु॒र॒ ए॒तार॒ इति॑ पुरः - ए॒तारः॑ । अ॒स्य ॥ येभ्यः॑ । न । ऋ॒ते । पव॑ते । धाम॑ । किम् । च॒न । न । ते । दि॒वः । न । पृ॒थि॒व्याः । अधीति॑ । स्नुषु॑ ॥ प्रा॒ण॒दा इति॑ प्राण - दाः ।  \newline


\textbf{Krama Paata} \newline

अ॒फ्सु॒षदे॒ वट् । अ॒फ्सु॒षद॒ इत्य॑फ्सु - सदे᳚ । वड् व॑न॒सदे᳚ । व॒न॒सदे॒ वट् । व॒न॒सद॒ इति॑ वन - सदे᳚ । वड् ब॑र्.हि॒षदे᳚ । ब॒र्.॒हि॒षदे॒ वट् । ब॒र्.॒हि॒षद॒ इति॑ बर्.हि - सदे᳚ । वट्थ् सु॑व॒र्विदे᳚ । सु॒व॒र्विदे॒ वट् । सु॒व॒र्विद॒ इति॑ सुवः - विदे᳚ । वडिति॒ वट् ॥ ये दे॒वाः । दे॒वा दे॒वाना᳚म् । दे॒वानां᳚ ॅय॒ज्ञियाः᳚ । य॒ज्ञिया॑ य॒ज्ञिया॑नाम् । य॒ज्ञिया॑नाꣳ सम्ॅवथ्स॒रीण᳚म् । स॒म्ॅव॒थ्स॒रीण॒मुप॑ । स॒म्ॅव॒थ्स॒रीण॒मिति॑ सं - व॒थ्स॒रीण᳚म् । उप॑ भा॒गम् । भा॒गमास॑ते । आस॑त॒ इत्यास॑ते ॥ अ॒हु॒तादो॑ ह॒विषः॑ । अ॒हु॒ताद॒ इत्य॑हुत - अदः॑ । ह॒विषो॑ य॒ज्ञे । य॒ज्ञे अ॒स्मिन्न् । अ॒स्मिन्थ् स्व॒यम् । स्व॒यम् जु॑हुद्ध्वम् । जु॒हु॒द्ध्व॒म् मधु॑नः । मधु॑नो घृ॒तस्य॑ । घृ॒तस्येति॑ घृ॒तस्य॑ ॥ ये दे॒वाः । दे॒वा दे॒वेषु॑ । दे॒वेष्वधि॑ । अधि॑ देव॒त्वम् । दे॒व॒त्वमायन्न्॑ । दे॒व॒त्वमिति॑ देव - त्वम् । आय॒न्॒. ये । ये ब्रह्म॑णः । ब्रह्म॑णः पुरए॒तारः॑ । पु॒र॒ए॒तारो॑ अ॒स्य । पु॒र॒ए॒तार॒ इति॑ पुरः - ए॒तारः॑ । अ॒स्येत्य॒स्य ॥ येभ्यो॒ न । नर्ते । ऋ॒ते पव॑ते । पव॑ते॒ धाम॑ । धाम॒ किम् । किम् च॒न । च॒न न । न ते । ते दि॒वः । दि॒वो न । न पृ॑थि॒व्याः । पृ॒थि॒व्या अधि॑ । अधि॒ स्नुषु॑ । स्नुष्विति॒ स्नुषु॑ ॥ प्रा॒ण॒दा ( ) अ॑पान॒दाः । प्रा॒ण॒दा इति॑ प्राण - दाः \newline

\textbf{Jatai Paata} \newline

1. अ॒फ्सु॒षदे॒ वड् वड॑फ्सु॒षदे॑ अफ्सु॒षदे॒ वट् । \newline
2. अ॒फ्सु॒षद॒ इत्य॑फ्सु - सदे᳚ । \newline
3. वड् व॑न॒सदे॑ वन॒सदे॒ वड् वड् व॑न॒सदे᳚ । \newline
4. व॒न॒सदे॒ वड् वड् व॑न॒सदे॑ वन॒सदे॒ वट् । \newline
5. व॒न॒सद॒ इति॑ वन - सदे᳚ । \newline
6. वड् ब॑र्.हि॒षदे॑ बर्.हि॒षदे॒ वड् वड् ब॑र्.हि॒षदे᳚ । \newline
7. ब॒र्॒.हि॒षदे॒ वड् वड् ब॑र्.हि॒षदे॑ बर्.हि॒षदे॒ वट् । \newline
8. ब॒र्॒.हि॒षद॒ इति॑ बर्.हि - सदे᳚ । \newline
9. वट् थ्सु॑व॒र्विदे॑ सुव॒र्विदे॒ वड् वट् थ्सु॑व॒र्विदे᳚ । \newline
10. सु॒व॒र्विदे॒ वड् वट् थ्सु॑व॒र्विदे॑ सुव॒र्विदे॒ वट् । \newline
11. सु॒व॒र्विद॒ इति॑ सुवः - विदे᳚ । \newline
12. वडिति॒ वट् । \newline
13. ये दे॒वा दे॒वा ये ये दे॒वाः । \newline
14. दे॒वा दे॒वाना᳚म् दे॒वाना᳚म् दे॒वा दे॒वा दे॒वाना᳚म् । \newline
15. दे॒वानां᳚ ॅय॒ज्ञिया॑ य॒ज्ञिया॑ दे॒वाना᳚म् दे॒वानां᳚ ॅय॒ज्ञियाः᳚ । \newline
16. य॒ज्ञिया॑ य॒ज्ञिया॑नां ॅय॒ज्ञिया॑नां ॅय॒ज्ञिया॑ य॒ज्ञिया॑ य॒ज्ञिया॑नाम् । \newline
17. य॒ज्ञिया॑नाꣳ संॅवथ्स॒रीणꣳ॑ संॅवथ्स॒रीणं॑ ॅय॒ज्ञिया॑नां ॅय॒ज्ञिया॑नाꣳ संॅवथ्स॒रीण᳚म् । \newline
18. सं॒ॅव॒थ्स॒रीण॒ मुपोप॑ संॅवथ्स॒रीणꣳ॑ संॅवथ्स॒रीण॒ मुप॑ । \newline
19. सं॒ॅव॒थ्स॒रीण॒मिति॑ सं - व॒थ्स॒रीण᳚म् । \newline
20. उप॑ भा॒गम् भा॒ग मुपोप॑ भा॒गम् । \newline
21. भा॒ग मास॑त॒ आस॑ते भा॒गम् भा॒ग मास॑ते । \newline
22. आस॑त॒ इत्यास॑ते । \newline
23. अ॒हु॒तादो॑ ह॒विषो॑ ह॒विषो॑ ऽहु॒तादो॑ ऽहु॒तादो॑ ह॒विषः॑ । \newline
24. अ॒हु॒ताद॒ इत्य॑हुत - अदः॑ । \newline
25. ह॒विषो॑ य॒ज्ञे य॒ज्ञे ह॒विषो॑ ह॒विषो॑ य॒ज्ञे । \newline
26. य॒ज्ञे अ॒स्मिन् न॒स्मिन्. य॒ज्ञे य॒ज्ञे अ॒स्मिन्न् । \newline
27. अ॒स्मिन् थ्स्व॒यꣳ स्व॒य म॒स्मिन् न॒स्मिन् थ्स्व॒यम् । \newline
28. स्व॒यम् जु॑हुद्ध्वम् जुहुद्ध्वꣳ स्व॒यꣳ स्व॒यम् जु॑हुद्ध्वम् । \newline
29. जु॒हु॒द्ध्व॒म् मधु॑नो॒ मधु॑नो जुहुद्ध्वम् जुहुद्ध्व॒म् मधु॑नः । \newline
30. मधु॑नो घृ॒तस्य॑ घृ॒तस्य॒ मधु॑नो॒ मधु॑नो घृ॒तस्य॑ । \newline
31. घृ॒तस्येति॑ घृ॒तस्य॑ । \newline
32. ये दे॒वा दे॒वा ये ये दे॒वाः । \newline
33. दे॒वा दे॒वेषु॑ दे॒वेषु॑ दे॒वा दे॒वा दे॒वेषु॑ । \newline
34. दे॒वे ष्वध्यधि॑ दे॒वेषु॑ दे॒वे ष्वधि॑ । \newline
35. अधि॑ देव॒त्वम् दे॑व॒त्व मध्यधि॑ देव॒त्वम् । \newline
36. दे॒व॒त्व माय॒न् नाय॑न् देव॒त्वम् दे॑व॒त्व मायन्न्॑ । \newline
37. दे॒व॒त्वमिति॑ देव - त्वम् । \newline
38. आय॒न्॒. ये य आय॒न् नाय॒न्॒. ये । \newline
39. ये ब्रह्म॑णो॒ ब्रह्म॑णो॒ ये ये ब्रह्म॑णः । \newline
40. ब्रह्म॑णः पुरए॒तारः॑ पुरए॒तारो॒ ब्रह्म॑णो॒ ब्रह्म॑णः पुरए॒तारः॑ । \newline
41. पु॒र॒ए॒तारो॑ अ॒स्यास्य पु॑रए॒तारः॑ पुरए॒तारो॑ अ॒स्य । \newline
42. पु॒र॒ए॒तार॒ इति॑ पुरः - ए॒तारः॑ । \newline
43. अ॒स्येत्य॒स्य । \newline
44. येभ्यो॒ न न येभ्यो॒ येभ्यो॒ न । \newline
45. न र्‌त ऋ॒ते न न र्‌ते । \newline
46. ऋ॒ते पव॑ते॒ पव॑त ऋ॒त ऋ॒ते पव॑ते । \newline
47. पव॑ते॒ धाम॒ धाम॒ पव॑ते॒ पव॑ते॒ धाम॑ । \newline
48. धाम॒ किम् किम् धाम॒ धाम॒ किम् । \newline
49. किम् च॒न च॒न किम् किम् च॒न । \newline
50. च॒न न न च॒न च॒न न । \newline
51. न ते ते न न ते । \newline
52. ते दि॒वो दि॒व स्ते ते दि॒वः । \newline
53. दि॒वो न न दि॒वो दि॒वो न । \newline
54. न पृ॑थि॒व्याः पृ॑थि॒व्या न न पृ॑थि॒व्याः । \newline
55. पृ॒थि॒व्या अध्यधि॑ पृथि॒व्याः पृ॑थि॒व्या अधि॑ । \newline
56. अधि॒ स्नुषु॒ स्नु ष्वध्यधि॒ स्नुषु॑ । \newline
57. स्नुष्विति॒ स्नुषु॑ । \newline
58. प्रा॒ण॒दा अ॑पान॒दा अ॑पान॒दाः प्रा॑ण॒दाः प्रा॑ण॒दा अ॑पान॒दाः । \newline
59. प्रा॒ण॒दा इति॑ प्राण - दाः । \newline

\textbf{Ghana Paata } \newline

1. अ॒फ्सु॒षदे॒ वड् वड॑फ्सु॒षदे॑ अफ्सु॒षदे॒ वड् व॑न॒सदे॑ वन॒सदे॒ वड॑फ्सु॒षदे॑ अफ्सु॒षदे॒ वड् व॑न॒सदे᳚ । \newline
2. अ॒फ्सु॒षद॒ इत्य॑फ्सु - सदे᳚ । \newline
3. वड् व॑न॒सदे॑ वन॒सदे॒ वड् वड् व॑न॒सदे॒ वड् वड् व॑न॒सदे॒ वड् वड् व॑न॒सदे॒ वट् । \newline
4. व॒न॒सदे॒ वड् वड् व॑न॒सदे॑ वन॒सदे॒ वड् ब॑र्.हि॒षदे॑ बर्.हि॒षदे॒ वड् व॑न॒सदे॑ वन॒सदे॒ वड् ब॑र्.हि॒षदे᳚ । \newline
5. व॒न॒सद॒ इति॑ वन - सदे᳚ । \newline
6. वड् ब॑र्.हि॒षदे॑ बर्.हि॒षदे॒ वड् वड् ब॑र्.हि॒षदे॒ वड् वड् ब॑र्.हि॒षदे॒ वड् वड् ब॑र्.हि॒षदे॒ वट् । \newline
7. ब॒र्॒.हि॒षदे॒ वड् वड् ब॑र्.हि॒षदे॑ बर्.हि॒षदे॒ वट् थ्सु॑व॒र्विदे॑ सुव॒र्विदे॒ वड् ब॑र्.हि॒षदे॑ बर्.हि॒षदे॒ वट् थ्सु॑व॒र्विदे᳚ । \newline
8. ब॒र्॒.हि॒षद॒ इति॑ बर्.हि - सदे᳚ । \newline
9. वट् थ्सु॑व॒र्विदे॑ सुव॒र्विदे॒ वड् वट् थ्सु॑व॒र्विदे॒ वड् वट् थ्सु॑व॒र्विदे॒ वड् वट् थ्सु॑व॒र्विदे॒ वट् । \newline
10. सु॒व॒र्विदे॒ वड् वट् थ्सु॑व॒र्विदे॑ सुव॒र्विदे॒ वट् । \newline
11. सु॒व॒र्विद॒ इति॑ सुवः - विदे᳚ । \newline
12. वडिति॒ वट् । \newline
13. ये दे॒वा दे॒वा ये ये दे॒वा दे॒वाना᳚म् दे॒वाना᳚म् दे॒वा ये ये दे॒वा दे॒वाना᳚म् । \newline
14. दे॒वा दे॒वाना᳚म् दे॒वाना᳚म् दे॒वा दे॒वा दे॒वानां᳚ ॅय॒ज्ञिया॑ य॒ज्ञिया॑ दे॒वाना᳚म् दे॒वा दे॒वा दे॒वानां᳚ ॅय॒ज्ञियाः᳚ । \newline
15. दे॒वानां᳚ ॅय॒ज्ञिया॑ य॒ज्ञिया॑ दे॒वाना᳚म् दे॒वानां᳚ ॅय॒ज्ञिया॑ य॒ज्ञिया॑नां ॅय॒ज्ञिया॑नां ॅय॒ज्ञिया॑ दे॒वाना᳚म् दे॒वानां᳚ ॅय॒ज्ञिया॑ य॒ज्ञिया॑नाम् । \newline
16. य॒ज्ञिया॑ य॒ज्ञिया॑नां ॅय॒ज्ञिया॑नां ॅय॒ज्ञिया॑ य॒ज्ञिया॑ य॒ज्ञिया॑नाꣳ संॅवथ्स॒रीणꣳ॑ संॅवथ्स॒रीणं॑ ॅय॒ज्ञिया॑नां ॅय॒ज्ञिया॑ य॒ज्ञिया॑ य॒ज्ञिया॑नाꣳ संॅवथ्स॒रीण᳚म् । \newline
17. य॒ज्ञिया॑नाꣳ संॅवथ्स॒रीणꣳ॑ संॅवथ्स॒रीणं॑ ॅय॒ज्ञिया॑नां ॅय॒ज्ञिया॑नाꣳ संॅवथ्स॒रीण॒ मुपोप॑ संॅवथ्स॒रीणं॑ ॅय॒ज्ञिया॑नां ॅय॒ज्ञिया॑नाꣳ संॅवथ्स॒रीण॒ मुप॑ । \newline
18. सं॒ॅव॒थ्स॒रीण॒ मुपोप॑ संॅवथ्स॒रीणꣳ॑ संॅवथ्स॒रीण॒ मुप॑ भा॒गम् भा॒ग मुप॑ संॅवथ्स॒रीणꣳ॑ संॅवथ्स॒रीण॒ मुप॑ भा॒गम् । \newline
19. सं॒ॅव॒थ्स॒रीण॒मिति॑ सं - व॒थ्स॒रीण᳚म् । \newline
20. उप॑ भा॒गम् भा॒ग मुपोप॑ भा॒ग मास॑त॒ आस॑ते भा॒ग मुपोप॑ भा॒ग मास॑ते । \newline
21. भा॒ग मास॑त॒ आस॑ते भा॒गम् भा॒ग मास॑ते । \newline
22. आस॑त॒ इत्यास॑ते । \newline
23. अ॒हु॒तादो॑ ह॒विषो॑ ह॒विषो॑ ऽहु॒तादो॑ ऽहु॒तादो॑ ह॒विषो॑ य॒ज्ञे य॒ज्ञे ह॒विषो॑ ऽहु॒तादो॑ ऽहु॒तादो॑ ह॒विषो॑ य॒ज्ञे । \newline
24. अ॒हु॒ताद॒ इत्य॑हुत - अदः॑ । \newline
25. ह॒विषो॑ य॒ज्ञे य॒ज्ञे ह॒विषो॑ ह॒विषो॑ य॒ज्ञे अ॒स्मिन् न॒स्मिन्. य॒ज्ञे ह॒विषो॑ ह॒विषो॑ य॒ज्ञे अ॒स्मिन्न् । \newline
26. य॒ज्ञे अ॒स्मिन् न॒स्मिन्. य॒ज्ञे य॒ज्ञे अ॒स्मिन् थ्स्व॒यꣳ स्व॒य म॒स्मिन्. य॒ज्ञे य॒ज्ञे अ॒स्मिन् थ्स्व॒यम् । \newline
27. अ॒स्मिन् थ्स्व॒यꣳ स्व॒य म॒स्मिन् न॒स्मिन् थ्स्व॒यम् जु॑हुद्ध्वम् जुहुद्ध्वꣳ स्व॒य म॒स्मिन् न॒स्मिन् थ्स्व॒यम् जु॑हुद्ध्वम् । \newline
28. स्व॒यम् जु॑हुद्ध्वम् जुहुद्ध्वꣳ स्व॒यꣳ स्व॒यम् जु॑हुद्ध्व॒म् मधु॑नो॒ मधु॑नो जुहुद्ध्वꣳ स्व॒यꣳ स्व॒यम् जु॑हुद्ध्व॒म् मधु॑नः । \newline
29. जु॒हु॒द्ध्व॒म् मधु॑नो॒ मधु॑नो जुहुद्ध्वम् जुहुद्ध्व॒म् मधु॑नो घृ॒तस्य॑ घृ॒तस्य॒ मधु॑नो जुहुद्ध्वम् जुहुद्ध्व॒म् मधु॑नो घृ॒तस्य॑ । \newline
30. मधु॑नो घृ॒तस्य॑ घृ॒तस्य॒ मधु॑नो॒ मधु॑नो घृ॒तस्य॑ । \newline
31. घृ॒तस्येति॑ घृ॒तस्य॑ । \newline
32. ये दे॒वा दे॒वा ये ये दे॒वा दे॒वेषु॑ दे॒वेषु॑ दे॒वा ये ये दे॒वा दे॒वेषु॑ । \newline
33. दे॒वा दे॒वेषु॑ दे॒वेषु॑ दे॒वा दे॒वा दे॒वे ष्वध्यधि॑ दे॒वेषु॑ दे॒वा दे॒वा दे॒वेष्वधि॑ । \newline
34. दे॒वे ष्वध्यधि॑ दे॒वेषु॑ दे॒वेष्वधि॑ देव॒त्वम् दे॑व॒त्व मधि॑ दे॒वेषु॑ दे॒वेष्वधि॑ देव॒त्वम् । \newline
35. अधि॑ देव॒त्वम् दे॑व॒त्व मध्यधि॑ देव॒त्व माय॒न् नाय॑न् देव॒त्व मध्यधि॑ देव॒त्व मायन्न्॑ । \newline
36. दे॒व॒त्व माय॒न् नाय॑न् देव॒त्वम् दे॑व॒त्व माय॒न्॒. ये य आय॑न् देव॒त्वम् दे॑व॒त्व माय॒न्॒. ये । \newline
37. दे॒व॒त्वमिति॑ देव - त्वम् । \newline
38. आय॒न्॒. ये य आय॒न् नाय॒न्॒. ये ब्रह्म॑णो॒ ब्रह्म॑णो॒ य आय॒न् नाय॒न्॒. ये ब्रह्म॑णः । \newline
39. ये ब्रह्म॑णो॒ ब्रह्म॑णो॒ ये ये ब्रह्म॑णः पुरए॒तारः॑ पुरए॒तारो॒ ब्रह्म॑णो॒ ये ये ब्रह्म॑णः पुरए॒तारः॑ । \newline
40. ब्रह्म॑णः पुरए॒तारः॑ पुरए॒तारो॒ ब्रह्म॑णो॒ ब्रह्म॑णः पुरए॒तारो॑ अ॒स्यास्य पु॑रए॒तारो॒ ब्रह्म॑णो॒ ब्रह्म॑णः पुरए॒तारो॑ अ॒स्य । \newline
41. पु॒र॒ए॒तारो॑ अ॒स्यास्य पु॑रए॒तारः॑ पुरए॒तारो॑ अ॒स्य । \newline
42. पु॒र॒ए॒तार॒ इति॑ पुरः - ए॒तारः॑ । \newline
43. अ॒स्येत्य॒स्य । \newline
44. येभ्यो॒ न न येभ्यो॒ येभ्यो॒ न र्‌त ऋ॒ते न येभ्यो॒ येभ्यो॒ न र्‌ते । \newline
45. न र्‌त ऋ॒ते न न र्‌ते पव॑ते॒ पव॑त ऋ॒ते न न र्‌ते पव॑ते । \newline
46. ऋ॒ते पव॑ते॒ पव॑त ऋ॒त ऋ॒ते पव॑ते॒ धाम॒ धाम॒ पव॑त ऋ॒त ऋ॒ते पव॑ते॒ धाम॑ । \newline
47. पव॑ते॒ धाम॒ धाम॒ पव॑ते॒ पव॑ते॒ धाम॒ किम् किम् धाम॒ पव॑ते॒ पव॑ते॒ धाम॒ किम् । \newline
48. धाम॒ किम् किम् धाम॒ धाम॒ किम् च॒न च॒न किम् धाम॒ धाम॒ किम् च॒न । \newline
49. किम् च॒न च॒न किम् किम् च॒न न न च॒न किम् किम् च॒न न । \newline
50. च॒न न न च॒न च॒न न ते ते न च॒न च॒न न ते । \newline
51. न ते ते न न ते दि॒वो दि॒व स्ते न न ते दि॒वः । \newline
52. ते दि॒वो दि॒व स्ते ते दि॒वो न न दि॒व स्ते ते दि॒वो न । \newline
53. दि॒वो न न दि॒वो दि॒वो न पृ॑थि॒व्याः पृ॑थि॒व्या न दि॒वो दि॒वो न पृ॑थि॒व्याः । \newline
54. न पृ॑थि॒व्याः पृ॑थि॒व्या न न पृ॑थि॒व्या अध्यधि॑ पृथि॒व्या न न पृ॑थि॒व्या अधि॑ । \newline
55. पृ॒थि॒व्या अध्यधि॑ पृथि॒व्याः पृ॑थि॒व्या अधि॒ स्नुषु॒ स्नुष्वधि॑ पृथि॒व्याः पृ॑थि॒व्या अधि॒ स्नुषु॑ । \newline
56. अधि॒ स्नुषु॒ स्नु ष्वध्यधि॒ स्नुषु॑ । \newline
57. स्नुष्विति॒ स्नुषु॑ । \newline
58. प्रा॒ण॒दा अ॑पान॒दा अ॑पान॒दाः प्रा॑ण॒दाः प्रा॑ण॒दा अ॑पान॒दा व्या॑न॒दा व्या॑न॒दा अ॑पान॒दाः प्रा॑ण॒दाः प्रा॑ण॒दा अ॑पान॒दा व्या॑न॒दाः । \newline
59. प्रा॒ण॒दा इति॑ प्राण - दाः । \newline
\pagebreak
\markright{ TS 4.6.1.5  \hfill https://www.vedavms.in \hfill}

\section{ TS 4.6.1.5 }

\textbf{TS 4.6.1.5 } \newline
\textbf{Samhita Paata} \newline

अ॑पान॒दा व्या॑न॒दाश्च॑क्षु॒र्दा व॑र्चो॒दा व॑रिवो॒दाः । अ॒न्यं ते॑ अ॒स्मत् त॑पन्तु हे॒तयः॑ पाव॒को अ॒स्मभ्यꣳ॑ शि॒वो भ॑व ॥ अ॒ग्निस्ति॒ग्मेन॑ शो॒चिषा॒ यꣳस॒द्विश्वं॒ न्य॑त्रिणं᳚ । अ॒ग्निर्नो॑ वꣳसते र॒यिं ॥ सैनाऽनी॑केन सुवि॒दत्रो॑ अ॒स्मे यष्टा॑ दे॒वाꣳ आय॑जिष्ठः स्व॒स्ति । अद॑ब्धो गो॒पा उ॒त नः॑ पर॒स्पा अग्ने᳚ द्यु॒मदु॒त रे॒वद्-दि॑दीहि ॥ \newline

\textbf{Pada Paata} \newline

अ॒पा॒न॒दा इत्य॑पान - दाः । व्या॒न॒दा इति॑ व्यान - दाः । च॒क्षु॒र्दा इति॑ चक्षुः-दाः । व॒र्चो॒दा इति॑ वर्चः-दाः । व॒रि॒वो॒दा इति॑ वरिवः-दाः ॥ अ॒न्यम् । ते॒ । अ॒स्मत् । त॒प॒न्तु॒ । हे॒तयः॑ । पा॒व॒कः । अ॒स्मभ्य॒मित्य॒स्म - भ्य॒म् । शि॒वः । भ॒व॒ ॥ अ॒ग्निः । ति॒ग्मेन॑ । शो॒चिषा᳚ । यꣳस॑त् । विश्व᳚म् । नीति॑ । अ॒त्रिण᳚म् ॥ अ॒ग्निः । नः॒ । वꣳ॒॒स॒ते॒ । र॒यिम् ॥ सः । ए॒ना । अनी॑केन । सु॒वि॒दत्र॒ इति॑ सु-वि॒दत्रः॑ । अ॒स्मे इति॑ । यष्टा᳚ । दे॒वान् । आय॑जिष्ठ॒ इत्या-य॒जि॒ष्ठः॒ । स्व॒स्ति ॥ अद॑ब्धः । गो॒पा इति॑ गो - पाः । उ॒त । नः॒ । प॒र॒स्पा इति॑ परः - पाः । अग्ने᳚ । द्यु॒मदिति॑ द्यु - मत् । उ॒त । रे॒वत् । दि॒दी॒हि॒ ॥  \newline


\textbf{Krama Paata} \newline

अ॒पा॒न॒दा व्या॑न॒दाः । अ॒पा॒न॒दा इत्य॑पान - दाः । व्या॒न॒दाश्च॑क्षु॒र्दाः । व्या॒न॒दा इति॑ व्यान - दाः । च॒क्षु॒र्दा व॑र्चो॒दाः । च॒क्षु॒र्दा इति॑ चक्षुः - दाः । व॒र्चो॒दा व॑रिवो॒दाः । व॒र्चो॒दा इति॑ वर्चः - दाः । व॒रि॒वो॒दा इति॑ वरिवः - दाः ॥ अ॒न्यम् ते᳚ । ते॒ अ॒स्मत् । अ॒स्मत् त॑पन्तु । त॒प॒न्तु॒ हे॒तयः॑ । हे॒तयः॑ पाव॒कः । पा॒व॒को अ॒स्मभ्य᳚म् । अ॒स्मभ्यꣳ॑ शि॒वः । अ॒स्मभ्य॒मित्य॒स्म - भ्य॒म् । शि॒वो भ॑व । भ॒वेति॑ भव ॥ अ॒ग्निस्ति॒ग्मेन॑ । ति॒ग्मेन॑ शो॒चिषा᳚ । शो॒चिषा॒ यꣳस॑त् । यꣳस॒द् विश्व᳚म् । विश्व॒म् नि । न्य॑त्रिण᳚म् । अ॒त्रिण॒मित्य॒त्रिण᳚म् ॥ अ॒ग्निर् नः॑ । नो॒ वꣳ॒॒स॒ते॒ । वꣳ॒॒स॒ते॒ र॒यिम् । र॒यिमिति॑ र॒यिम् ॥ सैना । ए॒नाऽनी॑केन । अनी॑केन सुवि॒दत्रः॑ । सु॒वि॒दत्रो॑ अ॒स्मे । सु॒वि॒दत्र॒ इति॑ सु - वि॒दत्रः॑ । अ॒स्मे यष्टा᳚ । अ॒स्मे इत्य॒स्मे । यष्टा॑ दे॒वान् । दे॒वाꣳ आय॑जिष्ठः । आय॑जिष्ठः स्व॒स्ति । आय॑जिष्ठ॒ इत्या - य॒जि॒ष्ठः॒ । स्व॒स्तीति॑ स्व॒स्ति ॥ अद॑ब्धो गो॒पाः । गो॒पा उ॒त । गो॒पा इति॑ गो - पाः । उ॒त नः॑ । नः॒ प॒र॒स्पाः । प॒र॒स्पा अग्ने᳚ । प॒र॒स्पा इति॑ परः - पाः । अग्ने᳚ द्यु॒मत् । द्यु॒मदु॒त । द्यु॒मदिति॑ द्यु - मत् । उ॒त रे॒वत् । रे॒वद् दि॑दीहि । 
दि॒दी॒हीति॑ दिदीहि । \newline

\textbf{Jatai Paata} \newline

1. अ॒पा॒न॒दा व्या॑न॒दा व्या॑न॒दा अ॑पान॒दा अ॑पान॒दा व्या॑न॒दाः । \newline
2. अ॒पा॒न॒दा इत्य॑पान - दाः । \newline
3. व्या॒न॒दा श्च॑क्षु॒र्दा श्च॑क्षु॒र्दा व्या॑न॒दा व्या॑न॒दा श्च॑क्षु॒र्दाः । \newline
4. व्या॒न॒दा इति॑ व्यान - दाः । \newline
5. च॒क्षु॒र्दा व॑र्चो॒दा व॑र्चो॒दा श्च॑क्षु॒र्दा श्च॑क्षु॒र्दा व॑र्चो॒दाः । \newline
6. च॒क्षु॒र्दा इति॑ चक्षुः - दाः । \newline
7. व॒र्चो॒दा व॑रिवो॒दा व॑रिवो॒दा व॑र्चो॒दा व॑र्चो॒दा व॑रिवो॒दाः । \newline
8. व॒र्चो॒दा इति॑ वर्चः - दाः । \newline
9. व॒रि॒वो॒दा इति॑ वरिवः - दाः । \newline
10. अ॒न्यम् ते॑ ते अ॒न्य म॒न्यम् ते᳚ । \newline
11. ते॒ अ॒स्म द॒स्मत् ते॑ ते अ॒स्मत् । \newline
12. अ॒स्मत् त॑पन्तु तप न्त्व॒स्म द॒स्मत् त॑पन्तु । \newline
13. त॒प॒न्तु॒ हे॒तयो॑ हे॒तय॑ स्तपन्तु तपन्तु हे॒तयः॑ । \newline
14. हे॒तयः॑ पाव॒कः पा॑व॒को हे॒तयो॑ हे॒तयः॑ पाव॒कः । \newline
15. पा॒व॒को अ॒स्मभ्य॑ म॒स्मभ्य॑म् पाव॒कः पा॑व॒को अ॒स्मभ्य᳚म् । \newline
16. अ॒स्मभ्यꣳ॑ शि॒वः शि॒वो अ॒स्मभ्य॑ म॒स्मभ्यꣳ॑ शि॒वः । \newline
17. अ॒स्मभ्य॒मित्य॒स्म - भ्य॒म् । \newline
18. शि॒वो भ॑व भव शि॒वः शि॒वो भ॑व । \newline
19. भ॒वेति॑ भव । \newline
20. अ॒ग्नि स्ति॒ग्मेन॑ ति॒ग्मेना॒ ग्नि र॒ग्नि स्ति॒ग्मेन॑ । \newline
21. ति॒ग्मेन॑ शो॒चिषा॑ शो॒चिषा॑ ति॒ग्मेन॑ ति॒ग्मेन॑ शो॒चिषा᳚ । \newline
22. शो॒चिषा॒ यꣳस॒द् यꣳस॑ च्छो॒चिषा॑ शो॒चिषा॒ यꣳस॑त् । \newline
23. यꣳस॒द् विश्वं॒ ॅविश्वं॒ ॅयꣳस॒द् यꣳस॒द् विश्व᳚म् । \newline
24. विश्व॒म् नि नि विश्वं॒ ॅविश्व॒म् नि । \newline
25. न्य॑त्रिण॑ म॒त्रिण॒म् नि न्य॑त्रिण᳚म् । \newline
26. अ॒त्रिण॒मित्य॒त्रिण᳚म् । \newline
27. अ॒ग्निर् नो॑ नो अ॒ग्नि र॒ग्निर् नः॑ । \newline
28. नो॒ वꣳ॒॒स॒ते॒ वꣳ॒॒स॒ते॒ नो॒ नो॒ वꣳ॒॒स॒ते॒ । \newline
29. वꣳ॒॒स॒ते॒ र॒यिꣳ र॒यिं ॅवꣳ॑सते वꣳसते र॒यिम् । \newline
30. र॒यिमिति॑ र॒यिम् । \newline
31. सैनैना स सैना । \newline
32. ए॒ना ऽनी॑के॒ना नी॑के नै॒नैना ऽनी॑केन । \newline
33. अनी॑केन सुवि॒दत्रः॑ सुवि॒दत्रो ऽनी॑के॒ना नी॑केन सुवि॒दत्रः॑ । \newline
34. सु॒वि॒दत्रो॑ अ॒स्मे अ॒स्मे सु॑वि॒दत्रः॑ सुवि॒दत्रो॑ अ॒स्मे । \newline
35. सु॒वि॒दत्र॒ इति॑ सु - वि॒दत्रः॑ । \newline
36. अ॒स्मे यष्टा॒ यष्टा॒ ऽस्मे अ॒स्मे यष्टा᳚ । \newline
37. अ॒स्मे इत्य॒स्मे । \newline
38. यष्टा॑ दे॒वान् दे॒वान्. यष्टा॒ यष्टा॑ दे॒वान् । \newline
39. दे॒वाꣳ आय॑जिष्ठ॒ आय॑जिष्ठो दे॒वान् दे॒वाꣳ आय॑जिष्ठः । \newline
40. आय॑जिष्ठः स्व॒स्ति स्व॒ स्त्याय॑जिष्ठ॒ आय॑जिष्ठः स्व॒स्ति । \newline
41. आय॑जिष्ठ॒ इत्या - य॒जि॒ष्ठः॒ । \newline
42. स्व॒स्तीति॑ स्व॒स्ति । \newline
43. अद॑ब्धो गो॒पा गो॒पा अद॑ब्धो॒ अद॑ब्धो गो॒पाः । \newline
44. गो॒पा उ॒तोत गो॒पा गो॒पा उ॒त । \newline
45. गो॒पा इति॑ गो - पाः । \newline
46. उ॒त नो॑ न उ॒तोत नः॑ । \newline
47. नः॒ प॒र॒स्पाः प॑र॒स्पा नो॑ नः पर॒स्पाः । \newline
48. प॒र॒स्पा अग्ने ऽग्ने॑ पर॒स्पाः प॑र॒स्पा अग्ने᳚ । \newline
49. प॒र॒स्पा इति॑ परः - पाः । \newline
50. अग्ने᳚ द्यु॒मद् द्यु॒मदग्ने ऽग्ने᳚ द्यु॒मत् । \newline
51. द्यु॒म दु॒तोत द्यु॒मद् द्यु॒मदु॒त । \newline
52. द्यु॒मदिति॑ द्यु - मत् । \newline
53. उ॒त रे॒वद् रे॒व दु॒तोत रे॒वत् । \newline
54. रे॒वद् दि॑दीहि दिदीहि रे॒वद् रे॒वद् दि॑दीहि । \newline
55. दि॒दी॒हीति॑ दिदीहि । \newline

\textbf{Ghana Paata } \newline

1. अ॒पा॒न॒दा व्या॑न॒दा व्या॑न॒दा अ॑पान॒दा अ॑पान॒दा व्या॑न॒दा श्च॑क्षु॒र्दा श्च॑क्षु॒र्दा व्या॑न॒दा अ॑पान॒दा अ॑पान॒दा व्या॑न॒दा श्च॑क्षु॒र्दाः । \newline
2. अ॒पा॒न॒दा इत्य॑पान - दाः । \newline
3. व्या॒न॒दा श्च॑क्षु॒र्दा श्च॑क्षु॒र्दा व्या॑न॒दा व्या॑न॒दा श्च॑क्षु॒र्दा व॑र्चो॒दा व॑र्चो॒दा श्च॑क्षु॒र्दा व्या॑न॒दा व्या॑न॒दा श्च॑क्षु॒र्दा व॑र्चो॒दाः । \newline
4. व्या॒न॒दा इति॑ व्यान - दाः । \newline
5. च॒क्षु॒र्दा व॑र्चो॒दा व॑र्चो॒दा श्च॑क्षु॒र्दा श्च॑क्षु॒र्दा व॑र्चो॒दा व॑रिवो॒दा व॑रिवो॒दा व॑र्चो॒दा श्च॑क्षु॒र्दा श्च॑क्षु॒र्दा व॑र्चो॒दा व॑रिवो॒दाः । \newline
6. च॒क्षु॒र्दा इति॑ चक्षुः - दाः । \newline
7. व॒र्चो॒दा व॑रिवो॒दा व॑रिवो॒दा व॑र्चो॒दा व॑र्चो॒दा व॑रिवो॒दाः । \newline
8. व॒र्चो॒दा इति॑ वर्चः - दाः । \newline
9. व॒रि॒वो॒दा इति॑ वरिवः - दाः । \newline
10. अ॒न्यम् ते॑ ते अ॒न्य म॒न्यम् ते॑ अ॒स्म द॒स्मत् ते॑ अ॒न्य म॒न्यम् ते॑ अ॒स्मत् । \newline
11. ते॒ अ॒स्म द॒स्मत् ते॑ ते अ॒स्मत् त॑पन्तु तपन् त्व॒स्मत् ते॑ ते अ॒स्मत् त॑पन्तु । \newline
12. अ॒स्मत् त॑पन्तु तपन् त्व॒स्म द॒स्मत् त॑पन्तु हे॒तयो॑ हे॒तय॑ स्तपन् त्व॒स्म द॒स्मत् त॑पन्तु हे॒तयः॑ । \newline
13. त॒प॒न्तु॒ हे॒तयो॑ हे॒तय॑ स्तपन्तु तपन्तु हे॒तयः॑ पाव॒कः पा॑व॒को हे॒तय॑ स्तपन्तु तपन्तु हे॒तयः॑ पाव॒कः । \newline
14. हे॒तयः॑ पाव॒कः पा॑व॒को हे॒तयो॑ हे॒तयः॑ पाव॒को अ॒स्मभ्य॑ म॒स्मभ्य॑म् पाव॒को हे॒तयो॑ हे॒तयः॑ पाव॒को अ॒स्मभ्य᳚म् । \newline
15. पा॒व॒को अ॒स्मभ्य॑ म॒स्मभ्य॑म् पाव॒कः पा॑व॒को अ॒स्मभ्यꣳ॑ शि॒वः शि॒वो अ॒स्मभ्य॑म् पाव॒कः पा॑व॒को अ॒स्मभ्यꣳ॑ शि॒वः । \newline
16. अ॒स्मभ्यꣳ॑ शि॒वः शि॒वो अ॒स्मभ्य॑ म॒स्मभ्यꣳ॑ शि॒वो भ॑व भव शि॒वो अ॒स्मभ्य॑ म॒स्मभ्यꣳ॑ शि॒वो भ॑व । \newline
17. अ॒स्मभ्य॒मित्य॒स्म - भ्य॒म् । \newline
18. शि॒वो भ॑व भव शि॒वः शि॒वो भ॑व । \newline
19. भ॒वेति॑ भव । \newline
20. अ॒ग्नि स्ति॒ग्मेन॑ ति॒ग्मेना॒ग्नि र॒ग्नि स्ति॒ग्मेन॑ शो॒चिषा॑ शो॒चिषा॑ ति॒ग्मेना॒ग्नि र॒ग्नि स्ति॒ग्मेन॑ शो॒चिषा᳚ । \newline
21. ति॒ग्मेन॑ शो॒चिषा॑ शो॒चिषा॑ ति॒ग्मेन॑ ति॒ग्मेन॑ शो॒चिषा॒ यꣳस॒द् यꣳस॑ च्छो॒चिषा॑ ति॒ग्मेन॑ ति॒ग्मेन॑ शो॒चिषा॒ यꣳस॑त् । \newline
22. शो॒चिषा॒ यꣳस॒द् यꣳस॑ च्छो॒चिषा॑ शो॒चिषा॒ यꣳस॒द् विश्वं॒ ॅविश्वं॒ ॅयꣳस॑ च्छो॒चिषा॑ शो॒चिषा॒ यꣳस॒द् विश्व᳚म् । \newline
23. यꣳस॒द् विश्वं॒ ॅविश्वं॒ ॅयꣳस॒द् यꣳस॒द् विश्व॒म् नि नि विश्वं॒ ॅयꣳस॒द् यꣳस॒द् विश्व॒म् नि । \newline
24. विश्व॒म् नि नि विश्वं॒ ॅविश्व॒म् न्य॑त्रिण॑ म॒त्रिण॒म् नि विश्वं॒ ॅविश्व॒म् न्य॑त्रिण᳚म् । \newline
25. न्य॑त्रिण॑ म॒त्रिण॒म् नि न्य॑त्रिण᳚म् । \newline
26. अ॒त्रिण॒मित्य॒त्रिण᳚म् । \newline
27. अ॒ग्निर् नो॑ नो अ॒ग्नि र॒ग्निर् नो॑ वꣳसते वꣳसते नो अ॒ग्नि र॒ग्निर् नो॑ वꣳसते । \newline
28. नो॒ वꣳ॒॒स॒ते॒ वꣳ॒॒स॒ते॒ नो॒ नो॒ वꣳ॒॒स॒ते॒ र॒यिꣳ र॒यिं ॅवꣳ॑सते नो नो वꣳसते र॒यिम् । \newline
29. वꣳ॒॒स॒ते॒ र॒यिꣳ र॒यिं ॅवꣳ॑सते वꣳसते र॒यिम् । \newline
30. र॒यिमिति॑ र॒यिम् । \newline
31. सैनैना स सैना ऽनी॑के॒ना नी॑केनै॒ना स सैना ऽनी॑केन । \newline
32. ए॒ना ऽनी॑के॒ना नी॑केनै॒नैना ऽनी॑केन सुवि॒दत्रः॑ सुवि॒दत्रो ऽनी॑केनै॒नैना ऽनी॑केन सुवि॒दत्रः॑ । \newline
33. अनी॑केन सुवि॒दत्रः॑ सुवि॒दत्रो ऽनी॑के॒ना नी॑केन सुवि॒दत्रो॑ अ॒स्मे अ॒स्मे सु॑वि॒दत्रो ऽनी॑के॒ना नी॑केन सुवि॒दत्रो॑ अ॒स्मे । \newline
34. सु॒वि॒दत्रो॑ अ॒स्मे अ॒स्मे सु॑वि॒दत्रः॑ सुवि॒दत्रो॑ अ॒स्मे यष्टा॒ यष्टा॒ ऽस्मे सु॑वि॒दत्रः॑ सुवि॒दत्रो॑ अ॒स्मे यष्टा᳚ । \newline
35. सु॒वि॒दत्र॒ इति॑ सु - वि॒दत्रः॑ । \newline
36. अ॒स्मे यष्टा॒ यष्टा॒ ऽस्मे अ॒स्मे यष्टा॑ दे॒वान् दे॒वान्. यष्टा॒ ऽस्मे अ॒स्मे यष्टा॑ दे॒वान् । \newline
37. अ॒स्मे इत्य॒स्मे । \newline
38. यष्टा॑ दे॒वान् दे॒वान्. यष्टा॒ यष्टा॑ दे॒वाꣳ आय॑जिष्ठ॒ आय॑जिष्ठो दे॒वान्. यष्टा॒ यष्टा॑ दे॒वाꣳ आय॑जिष्ठः । \newline
39. दे॒वाꣳ आय॑जिष्ठ॒ आय॑जिष्ठो दे॒वान् दे॒वाꣳ आय॑जिष्ठः स्व॒स्ति स्व॒स्त्या य॑जिष्ठो दे॒वान् दे॒वाꣳ आय॑जिष्ठः स्व॒स्ति । \newline
40. आय॑जिष्ठः स्व॒स्ति स्व॒स्त्या य॑जिष्ठ॒ आय॑जिष्ठः स्व॒स्ति । \newline
41. आय॑जिष्ठ॒ इत्या - य॒जि॒ष्ठः॒ । \newline
42. स्व॒स्तीति॑ स्व॒स्ति । \newline
43. अद॑ब्धो गो॒पा गो॒पा अद॑ब्धो॒ अद॑ब्धो गो॒पा उ॒तोत गो॒पा अद॑ब्धो॒ अद॑ब्धो गो॒पा उ॒त । \newline
44. गो॒पा उ॒तोत गो॒पा गो॒पा उ॒त नो॑ न उ॒त गो॒पा गो॒पा उ॒त नः॑ । \newline
45. गो॒पा इति॑ गो - पाः । \newline
46. उ॒त नो॑ न उ॒तोत नः॑ पर॒स्पाः प॑र॒स्पा न॑ उ॒तोत नः॑ पर॒स्पाः । \newline
47. नः॒ प॒र॒स्पाः प॑र॒स्पा नो॑ नः पर॒स्पा अग्ने ऽग्ने॑ पर॒स्पा नो॑ नः पर॒स्पा अग्ने᳚ । \newline
48. प॒र॒स्पा अग्ने ऽग्ने॑ पर॒स्पाः प॑र॒स्पा अग्ने᳚ द्यु॒मद् द्यु॒मदग्ने॑ पर॒स्पाः प॑र॒स्पा अग्ने᳚ द्यु॒मत् । \newline
49. प॒र॒स्पा इति॑ परः - पाः । \newline
50. अग्ने᳚ द्यु॒मद् द्यु॒म दग्ने ऽग्ने᳚ द्यु॒म दु॒तोत द्यु॒म दग्ने ऽग्ने᳚ द्यु॒म दु॒त । \newline
51. द्यु॒म दु॒तोत द्यु॒मद् द्यु॒म दु॒त रे॒वद् रे॒व दु॒त द्यु॒मद् द्यु॒म दु॒त रे॒वत् । \newline
52. द्यु॒मदिति॑ द्यु - मत् । \newline
53. उ॒त रे॒वद् रे॒व दु॒तोत रे॒वद् दि॑दीहि दिदीहि रे॒व दु॒तोत रे॒वद् दि॑दीहि । \newline
54. रे॒वद् दि॑दीहि दिदीहि रे॒वद् रे॒वद् दि॑दीहि । \newline
55. दि॒दी॒हीति॑ दिदीहि । \newline
\pagebreak
\markright{ TS 4.6.2.1  \hfill https://www.vedavms.in \hfill}

\section{ TS 4.6.2.1 }

\textbf{TS 4.6.2.1 } \newline
\textbf{Samhita Paata} \newline

य इ॒मा विश्वा॒ भुव॑नानि॒ जुह्व॒दृषि॒र्॒.होता॑ निष॒सादा॑ पि॒ता नः॑ । स आ॒शिषा॒ द्रवि॑णमि॒च्छमा॑नः परम॒च्छदो॒ वर॒ आ वि॑वेश ॥ वि॒श्वक॑र्मा॒ मन॑सा॒ यद्विहा॑या धा॒ता वि॑धा॒ता प॑र॒मोत स॒न्दृक् । तेषा॑मि॒ष्टानि॒ समि॒षा म॑दन्ति॒ यत्र॑ सप्त॒र्॒.षीन् प॒र एक॑मा॒हुः ॥ यो नः॑ पि॒ता ज॑नि॒ता यो वि॑धा॒ता यो नः॑ स॒तो अ॒भ्या सज्ज॒जान॑ । \newline

\textbf{Pada Paata} \newline

य । इ॒मा । विश्वा᳚ । भुव॑नानि । जुह्व॑त् । ऋषिः॑ । होता᳚ । नि॒ष॒सादेति॑ नि - स॒साद॑ । पि॒ता । नः॒ ॥ सः । आ॒शिषेत्या᳚ -शिषा᳚ । द्रवि॑णम् । इ॒च्छमा॑नः । प॒र॒म॒च्छद॒ इति॑ परम - छदः॑ । वरे᳚ । एति॑ । वि॒वे॒श॒ ॥ वि॒श्वक॒र्मेति॑ वि॒श्व-क॒र्मा॒ । मन॑सा । यत् । विहा॑या॒ इति॒ वि-हा॒याः॒ । धा॒ता । वि॒धा॒तेति॑ वि - धा॒ता । प॒र॒मा । उ॒त । स॒न्दृगिति॑ सं-दृक् ॥ तेषा᳚म् । इ॒ष्टानि॑ । समिति॑ । इ॒षा । म॒द॒न्ति॒ । यत्र॑ । स॒प्त॒र्॒.षीनिति॑ सप्त-ऋ॒षीन् । प॒रः । एक᳚म् । आ॒हुः ॥ यः । नः॒ । पि॒ता । ज॒नि॒ता । यः । वि॒धा॒तेति॑ वि - धा॒ता । यः । नः॒ । स॒तः । अ॒भि । एति॑ । सत् । जजान॑ ॥  \newline


\textbf{Krama Paata} \newline

य इ॒मा । इ॒मा विश्वा᳚ । विश्वा॒ भुव॑नानि । भुव॑नानि॒ जुह्व॑त् । जुह्व॒दृषिः॑ । ऋषि॒र्॒. होता᳚ । होता॑ निष॒साद॑ । नि॒ष॒सादा॑ पि॒ता । नि॒ष॒सादेति॑ नि - स॒साद॑ । पि॒ता नः॑ । न॒ इति॑ नः ॥ स आ॒शिषा᳚ । आ॒शिषा॒ द्रवि॑णम् । आ॒शिषेत्या᳚ - शिषा᳚ । द्रवि॑णमि॒च्छमा॑नः । इ॒च्छमा॑नः परम॒च्छदः॑ । प॒र॒म॒च्छदो॒ वरे᳚ । प॒र॒म॒च्छद॒ इति॑ परम - छदः॑ । वर॒ आ । आ वि॑वेश । वि॒वे॒शेति॑ विवेश ॥ वि॒श्वक॑र्मा॒ मन॑सा । वि॒श्वक॒र्मेति॑ वि॒श्व - क॒र्मा॒ । मन॑सा॒ यत् । यद् विहा॑याः । विहा॑या धा॒ता । विहा॑या॒ इति॒ वि - हा॒याः॒ । धा॒ता वि॑धा॒ता । वि॒धा॒ता प॑र॒मा । वि॒धा॒तेति॑ वि - धा॒ता । प॒र॒मोत । उ॒त स॒न्दृक् । स॒न्दृगिति॑ सम् - दृक् ॥ तेषा॑मि॒ष्टानि॑ । इ॒ष्टानि॒ सम् । समि॒षा । इ॒षा म॑दन्ति । म॒द॒न्ति॒ यत्र॑ । यत्र॑ सप्त॒र्॒.षीन् । स॒प्त॒र्.॒षीन् प॒रः । स॒प्त॒र्॒.षीनिति॑ सप्त - ऋ॒षीन् । प॒र एक᳚म् । एक॑मा॒हुः । आ॒हुरित्या॒हुः ॥ यो नः॑ । नः॒ पि॒ता । पि॒ता ज॑नि॒ता । ज॒नि॒ता यः । यो वि॑धा॒ता । वि॒धा॒ता यः । वि॒धा॒तेति॑ वि - धा॒ता । यो नः॑ । नः॒ स॒तः । स॒तो अ॒भि । अ॒भ्या । आ सत् । सज् ज॒जान॑ । ज॒जानेति॑ ज॒जान॑ । \newline

\textbf{Jatai Paata} \newline

1. य इ॒मेमा यो य इ॒मा । \newline
2. इ॒मा विश्वा॒ विश्वे॒मेमा विश्वा᳚ । \newline
3. विश्वा॒ भुव॑नानि॒ भुव॑नानि॒ विश्वा॒ विश्वा॒ भुव॑नानि । \newline
4. भुव॑नानि॒ जुह्व॒ज् जुह्व॒द् भुव॑नानि॒ भुव॑नानि॒ जुह्व॑त् । \newline
5. जुह्व॒दृषि॒र्॒. ऋषि॒र् जुह्व॒ज् जुह्व॒दृषिः॑ । \newline
6. ऋषि॒र्॒. होता॒ होत र्.षि॒र्॒. ऋषि॒र्॒. होता᳚ । \newline
7. होता॑ निष॒साद॑ निष॒साद॒ होता॒ होता॑ निष॒साद॑ । \newline
8. नि॒ष॒सादा॑ पि॒ता पि॒ता नि॑ष॒साद॑ निष॒सादा॑ पि॒ता । \newline
9. नि॒ष॒सादेति॑ नि - स॒साद॑ । \newline
10. पि॒ता नो॑ नः पि॒ता पि॒ता नः॑ । \newline
11. न॒ इति॑ नः । \newline
12. स आ॒शिषा॒ ऽऽशिषा॒ स स आ॒शिषा᳚ । \newline
13. आ॒शिषा॒ द्रवि॑ण॒म् द्रवि॑ण मा॒शिषा॒ ऽऽशिषा॒ द्रवि॑णम् । \newline
14. आ॒शिषेत्या᳚ - शिषा᳚ । \newline
15. द्रवि॑ण मि॒च्छमा॑न इ॒च्छमा॑नो॒ द्रवि॑ण॒म् द्रवि॑ण मि॒च्छमा॑नः । \newline
16. इ॒च्छमा॑नः परम॒च्छदः॑ परम॒च्छद॑ इ॒च्छमा॑न इ॒च्छमा॑नः परम॒च्छदः॑ । \newline
17. प॒र॒म॒च्छदो॒ वरे॒ वरे॑ परम॒च्छदः॑ परम॒च्छदो॒ वरे᳚ । \newline
18. प॒र॒म॒च्छद॒ इति॑ परम - छदः॑ । \newline
19. वर॒ आ वरे॒ वर॒ आ । \newline
20. आ वि॑वेश विवे॒शा वि॑वेश । \newline
21. वि॒वे॒शेति॑ विवेश । \newline
22. वि॒श्वक॑र्मा॒ मन॑सा॒ मन॑सा वि॒श्वक॑र्मा वि॒श्वक॑र्मा॒ मन॑सा । \newline
23. वि॒श्वक॒र्मेति॑ वि॒श्व - क॒र्मा॒ । \newline
24. मन॑सा॒ यद् यन् मन॑सा॒ मन॑सा॒ यत् । \newline
25. यद् विहा॑या॒ विहा॑या॒ यद् यद् विहा॑याः । \newline
26. विहा॑या धा॒ता धा॒ता विहा॑या॒ विहा॑या धा॒ता । \newline
27. विहा॑या॒ इति॒ वि - हा॒याः॒ । \newline
28. धा॒ता वि॑धा॒ता वि॑धा॒ता धा॒ता धा॒ता वि॑धा॒ता । \newline
29. वि॒धा॒ता प॑र॒मा प॑र॒मा वि॑धा॒ता वि॑धा॒ता प॑र॒मा । \newline
30. वि॒धा॒तेति॑ वि - धा॒ता । \newline
31. प॒र॒मोतोत प॑र॒मा प॑र॒मोत । \newline
32. उ॒त स॒न्दृख् स॒न्दृ गु॒तोत स॒न्दृक् । \newline
33. स॒न्दृगिति॑ सं - दृक् । \newline
34. तेषा॑ मि॒ष्टा नी॒ष्टानि॒ तेषा॒म् तेषा॑ मि॒ष्टानि॑ । \newline
35. इ॒ष्टानि॒ सꣳ स मि॒ष्टा नी॒ष्टानि॒ सम् । \newline
36. स मि॒षेषा सꣳ स मि॒षा । \newline
37. इ॒षा म॑दन्ति मदन्ती॒ षेषा म॑दन्ति । \newline
38. म॒द॒न्ति॒ यत्र॒ यत्र॑ मदन्ति मदन्ति॒ यत्र॑ । \newline
39. यत्र॑ सप्त॒र्॒.षीन् थ्स॑प्त॒र्॒.षीन्. यत्र॒ यत्र॑ सप्त॒र्॒.षीन् । \newline
40. स॒प्त॒र्॒.षीन् प॒रः प॒रः स॑प्त॒र्॒.षीन् थ्स॑प्त॒र्॒.षीन् प॒रः । \newline
41. स॒प्त॒र्॒.षीनिति॑ सप्त - ऋ॒षीन् । \newline
42. प॒र एक॒ मेक॑म् प॒रः प॒र एक᳚म् । \newline
43. एक॑ मा॒हु रा॒हु रेक॒ मेक॑ मा॒हुः । \newline
44. आ॒हुरित्या॒हुः । \newline
45. यो नो॑ नो॒ यो यो नः॑ । \newline
46. नः॒ पि॒ता पि॒ता नो॑ नः पि॒ता । \newline
47. पि॒ता ज॑नि॒ता ज॑नि॒ता पि॒ता पि॒ता ज॑नि॒ता । \newline
48. ज॒नि॒ता यो यो ज॑नि॒ता ज॑नि॒ता यः । \newline
49. यो वि॑धा॒ता वि॑धा॒ता यो यो वि॑धा॒ता । \newline
50. वि॒धा॒ता यो यो वि॑धा॒ता वि॑धा॒ता यः । \newline
51. वि॒धा॒तेति॑ वि - धा॒ता । \newline
52. यो नो॑ नो॒ यो यो नः॑ । \newline
53. नः॒ स॒तः स॒तो नो॑ नः स॒तः । \newline
54. स॒तो अ॒भ्य॑भि स॒तः स॒तो अ॒भि । \newline
55. अ॒भ्या ऽभ्य॑भ्या । \newline
56. आ सथ् सदा सत् । \newline
57. सज् ज॒जान॑ ज॒जान॒ सथ् सज् ज॒जान॑ । \newline
58. ज॒जानेति॑ ज॒जान॑ । \newline

\textbf{Ghana Paata } \newline

1. य इ॒मेमा यो य इ॒मा विश्वा॒ विश्वे॒मा यो य इ॒मा विश्वा᳚ । \newline
2. इ॒मा विश्वा॒ विश्वे॒ मेमा विश्वा॒ भुव॑नानि॒ भुव॑नानि॒ विश्वे॒ मेमा विश्वा॒ भुव॑नानि । \newline
3. विश्वा॒ भुव॑नानि॒ भुव॑नानि॒ विश्वा॒ विश्वा॒ भुव॑नानि॒ जुह्व॒ज् जुह्व॒द् भुव॑नानि॒ विश्वा॒ विश्वा॒ भुव॑नानि॒ जुह्व॑त् । \newline
4. भुव॑नानि॒ जुह्व॒ज् जुह्व॒द् भुव॑नानि॒ भुव॑नानि॒ जुह्व॒ दृषि॒र्॒. ऋषि॒र् जुह्व॒द् भुव॑नानि॒ भुव॑नानि॒ जुह्व॒ दृषिः॑ । \newline
5. जुह्व॒ दृषि॒र्॒. ऋषि॒र् जुह्व॒ज् जुह्व॒ दृषि॒र्॒. होता॒ होत र्.षि॒र् जुह्व॒ज् जुह्व॒ दृषि॒र्॒. होता᳚ । \newline
6. ऋषि॒र्॒. होता॒ होत र्.षि॒र्॒. ऋषि॒र्॒. होता॑ निष॒साद॑ निष॒साद॒ होत र्.षि॒र्॒. ऋषि॒र्॒. होता॑ निष॒साद॑ । \newline
7. होता॑ निष॒साद॑ निष॒साद॒ होता॒ होता॑ निष॒सादा॑ पि॒ता पि॒ता नि॑ष॒साद॒ होता॒ होता॑ निष॒सादा॑ पि॒ता । \newline
8. नि॒ष॒सादा॑ पि॒ता पि॒ता नि॑ष॒साद॑ निष॒सादा॑ पि॒ता नो॑ नः पि॒ता नि॑ष॒साद॑ निष॒सादा॑ पि॒ता नः॑ । \newline
9. नि॒ष॒सादेति॑ नि - स॒साद॑ । \newline
10. पि॒ता नो॑ नः पि॒ता पि॒ता नः॑ । \newline
11. न॒ इति॑ नः । \newline
12. स आ॒शिषा॒ ऽऽशिषा॒ स स आ॒शिषा॒ द्रवि॑ण॒म् द्रवि॑ण मा॒शिषा॒ स स आ॒शिषा॒ द्रवि॑णम् । \newline
13. आ॒शिषा॒ द्रवि॑ण॒म् द्रवि॑ण मा॒शिषा॒ ऽऽशिषा॒ द्रवि॑ण मि॒च्छमा॑न इ॒च्छमा॑नो॒ द्रवि॑ण मा॒शिषा॒ ऽऽशिषा॒ द्रवि॑ण मि॒च्छमा॑नः । \newline
14. आ॒शिषेत्या᳚ - शिषा᳚ । \newline
15. द्रवि॑ण मि॒च्छमा॑न इ॒च्छमा॑नो॒ द्रवि॑ण॒म् द्रवि॑ण मि॒च्छमा॑नः परम॒च्छदः॑ परम॒च्छद॑ इ॒च्छमा॑नो॒ द्रवि॑ण॒म् द्रवि॑ण मि॒च्छमा॑नः परम॒च्छदः॑ । \newline
16. इ॒च्छमा॑नः परम॒च्छदः॑ परम॒च्छद॑ इ॒च्छमा॑न इ॒च्छमा॑नः परम॒च्छदो॒ वरे॒ वरे॑ परम॒च्छद॑ इ॒च्छमा॑न इ॒च्छमा॑नः परम॒च्छदो॒ वरे᳚ । \newline
17. प॒र॒म॒च्छदो॒ वरे॒ वरे॑ परम॒च्छदः॑ परम॒च्छदो॒ वर॒ आ वरे॑ परम॒च्छदः॑ परम॒च्छदो॒ वर॒ आ । \newline
18. प॒र॒म॒च्छद॒ इति॑ परम - छदः॑ । \newline
19. वर॒ आ वरे॒ वर॒ आ वि॑वेश विवे॒शा वरे॒ वर॒ आ वि॑वेश । \newline
20. आ वि॑वेश विवे॒शा वि॑वेश । \newline
21. वि॒वे॒शेति॑ विवेश । \newline
22. वि॒श्वक॑र्मा॒ मन॑सा॒ मन॑सा वि॒श्वक॑र्मा वि॒श्वक॑र्मा॒ मन॑सा॒ यद् यन् मन॑सा वि॒श्वक॑र्मा वि॒श्वक॑र्मा॒ मन॑सा॒ यत् । \newline
23. वि॒श्वक॒र्मेति॑ वि॒श्व - क॒र्मा॒ । \newline
24. मन॑सा॒ यद् यन् मन॑सा॒ मन॑सा॒ यद् विहा॑या॒ विहा॑या॒ यन् मन॑सा॒ मन॑सा॒ यद् विहा॑याः । \newline
25. यद् विहा॑या॒ विहा॑या॒ यद् यद् विहा॑या धा॒ता धा॒ता विहा॑या॒ यद् यद् विहा॑या धा॒ता । \newline
26. विहा॑या धा॒ता धा॒ता विहा॑या॒ विहा॑या धा॒ता वि॑धा॒ता वि॑धा॒ता धा॒ता विहा॑या॒ विहा॑या धा॒ता वि॑धा॒ता । \newline
27. विहा॑या॒ इति॒ वि - हा॒याः॒ । \newline
28. धा॒ता वि॑धा॒ता वि॑धा॒ता धा॒ता धा॒ता वि॑धा॒ता प॑र॒मा प॑र॒मा वि॑धा॒ता धा॒ता धा॒ता वि॑धा॒ता प॑र॒मा । \newline
29. वि॒धा॒ता प॑र॒मा प॑र॒मा वि॑धा॒ता वि॑धा॒ता प॑र॒ मोतोत प॑र॒मा वि॑धा॒ता वि॑धा॒ता प॑र॒मोत । \newline
30. वि॒धा॒तेति॑ वि - धा॒ता । \newline
31. प॒र॒ मोतोत प॑र॒मा प॑र॒मोत स॒न्दृख् स॒न्दृ गु॒त प॑र॒मा प॑र॒मोत स॒न्दृक् । \newline
32. उ॒त स॒न्दृख् स॒न्दृ गु॒तोत स॒न्दृक् । \newline
33. स॒न्दृगिति॑ सं - दृक् । \newline
34. तेषा॑ मि॒ष्टानी॒ष्टानि॒ तेषा॒म् तेषा॑ मि॒ष्टानि॒ सꣳ स मि॒ष्टानि॒ तेषा॒म् तेषा॑ मि॒ष्टानि॒ सम् । \newline
35. इ॒ष्टानि॒ सꣳ स मि॒ष्टानी॒ ष्टानि॒ स मि॒षेषा स मि॒ष्टानी॒ ष्टानि॒ स मि॒षा । \newline
36. स मि॒षेषा सꣳ स मि॒षा म॑दन्ति मदन्ती॒षा सꣳ स मि॒षा म॑दन्ति । \newline
37. इ॒षा म॑दन्ति मदन्ती॒षेषा म॑दन्ति॒ यत्र॒ यत्र॑ मदन्ती॒षेषा म॑दन्ति॒ यत्र॑ । \newline
38. म॒द॒न्ति॒ यत्र॒ यत्र॑ मदन्ति मदन्ति॒ यत्र॑ सप्त॒र्॒.षीन् थ्स॑प्त॒र्॒.षीन्. यत्र॑ मदन्ति मदन्ति॒ यत्र॑ सप्त॒र्॒.षीन् । \newline
39. यत्र॑ सप्त॒र्॒.षीन् थ्स॑प्त॒र्॒.षीन्. यत्र॒ यत्र॑ सप्त॒र्॒.षीन् प॒रः प॒रः स॑प्त॒र्॒.षीन्. यत्र॒ यत्र॑ सप्त॒र्॒.षीन् प॒रः । \newline
40. स॒प्त॒र्॒.षीन् प॒रः प॒रः स॑प्त॒र्॒.षीन् थ्स॑प्त॒र्॒.षीन् प॒र एक॒ मेक॑म् प॒रः स॑प्त॒र्॒.षीन् थ्स॑प्त॒र्॒.षीन् प॒र एक᳚म् । \newline
41. स॒प्त॒र्॒.षीनिति॑ सप्त - ऋ॒षीन् । \newline
42. प॒र एक॒ मेक॑म् प॒रः प॒र एक॑ मा॒हु रा॒हु रेक॑म् प॒रः प॒र एक॑ मा॒हुः । \newline
43. एक॑ मा॒हु रा॒हु रेक॒ मेक॑ मा॒हुः । \newline
44. आ॒हुरित्या॒हुः । \newline
45. यो नो॑ नो॒ यो यो नः॑ पि॒ता पि॒ता नो॒ यो यो नः॑ पि॒ता । \newline
46. नः॒ पि॒ता पि॒ता नो॑ नः पि॒ता ज॑नि॒ता ज॑नि॒ता पि॒ता नो॑ नः पि॒ता ज॑नि॒ता । \newline
47. पि॒ता ज॑नि॒ता ज॑नि॒ता पि॒ता पि॒ता ज॑नि॒ता यो यो ज॑नि॒ता पि॒ता पि॒ता ज॑नि॒ता यः । \newline
48. ज॒नि॒ता यो यो ज॑नि॒ता ज॑नि॒ता यो वि॑धा॒ता वि॑धा॒ता यो ज॑नि॒ता ज॑नि॒ता यो वि॑धा॒ता । \newline
49. यो वि॑धा॒ता वि॑धा॒ता यो यो वि॑धा॒ता यो यो वि॑धा॒ता यो यो वि॑धा॒ता यः । \newline
50. वि॒धा॒ता यो यो वि॑धा॒ता वि॑धा॒ता यो नो॑ नो॒ यो वि॑धा॒ता वि॑धा॒ता यो नः॑ । \newline
51. वि॒धा॒तेति॑ वि - धा॒ता । \newline
52. यो नो॑ नो॒ यो यो नः॑ स॒तः स॒तो नो॒ यो यो नः॑ स॒तः । \newline
53. नः॒ स॒तः स॒तो नो॑ नः स॒तो अ॒भ्य॑भि स॒तो नो॑ नः स॒तो अ॒भि । \newline
54. स॒तो अ॒भ्य॑भि स॒तः स॒तो अ॒भ्या ऽभि स॒तः स॒तो अ॒भ्या । \newline
55. अ॒भ्या ऽभ्य॑भ्या सथ् सदा ऽभ्य॑भ्या सत् । \newline
56. आ सथ् सदा सज् ज॒जान॑ ज॒जान॒ सदा सज् ज॒जान॑ । \newline
57. सज् ज॒जान॑ ज॒जान॒ सथ् सज् ज॒जान॑ । \newline
58. ज॒जानेति॑ ज॒जान॑ । \newline
\pagebreak
\markright{ TS 4.6.2.2  \hfill https://www.vedavms.in \hfill}

\section{ TS 4.6.2.2 }

\textbf{TS 4.6.2.2 } \newline
\textbf{Samhita Paata} \newline

यो दे॒वानां᳚ नाम॒धा एक॑ ए॒व तꣳ स॑प्रं॒श्नं भुव॑ना यन्त्य॒न्या ॥ त आऽय॑जन्त॒ द्रवि॑णꣳ॒॒ सम॑स्मा॒ ऋष॑यः॒ पूर्वे॑ जरि॒तारो॒ न भू॒ना । अ॒सूर्ता॒ सूर्ता॒ रज॑सो वि॒माने॒ ये भू॒तानि॑ स॒मकृ॑ण्वन्नि॒मानि॑ ॥ न तं ॅवि॑दाथ॒ य इ॒दं ज॒जाना॒न्यद् यु॒ष्माक॒मन्त॑रं भवाति । नी॒हा॒रेण॒ प्रावृ॑ता॒ जल्प्या॑ चासु॒तृप॑ उक्थ॒ शास॑श्चरन्ति ॥ प॒रो दि॒वा प॒र ए॒ना - [  ] \newline

\textbf{Pada Paata} \newline

यः । दे॒वाना᳚म् । ना॒म॒धा इति॑ नाम - धाः । एकः॑ । ए॒व । तम् । स॒प्रं॒श्नमिति॑ सं - प्र॒श्नम् । भुव॑ना । य॒न्ति॒ । अ॒न्या ॥ ते । एति॑ । अ॒य॒ज॒न्त॒ । द्रवि॑णम् । समिति॑ । अ॒स्मै॒ । ऋष॑यः । पूर्वे᳚ । ज॒रि॒तारः॑ । न॒ । भू॒ना ॥ अ॒सूर्ता᳚ । सूर्ता᳚ । रज॑सः । वि॒मान॒ इति॑ वि - माने᳚ । ये । भू॒तानि॑ । स॒मकृ॑ण्व॒न्निति॑ सं - अकृ॑ण्वन्न् । इ॒मानि॑ ॥ न । तम् । वि॒दा॒थ॒ । यः । इ॒दम् । ज॒जान॑ । अ॒न्यत् । यु॒ष्माक᳚म् । अन्त॑रम् । भ॒वा॒ति॒ ॥ नी॒हा॒रेण॑ । प्रावृ॑ताः । जल्प्या᳚ । च॒ । अ॒सु॒तृप॒ इत्य॑सु - तृपः॑ । उ॒क्थ॒शास॒ इत्यु॑क्थ - शासः॑ । च॒र॒न्ति॒ ॥ प॒रः । दि॒वा । प॒रः । ए॒ना ।  \newline


\textbf{Krama Paata} \newline

यो दे॒वाना᳚म् । दे॒वाना᳚म् नाम॒धाः । ना॒म॒धा एकः॑ । ना॒म॒धा इति॑ नाम - धाः । एक॑ ए॒व । ए॒व तम् । तꣳ स॑म्प्र॒श्ञम् । स॒म्प्र॒श्ञम् भुव॑ना । स॒म्प्र॒श्ञमिति॑ सम् - प्र॒श्ञम् । भुव॑ना यन्ति । य॒न्त्य॒न्या । अ॒न्येत्य॒न्या ॥ त आ । आऽय॑जन्त । अ॒य॒ज॒न्त॒ द्रवि॑णम् । द्रवि॑णꣳ॒॒ सम् । सम॑स्मै । अ॒स्मा॒ ऋष॑यः । ऋष॑यः॒ पूर्वे᳚ । पूर्वे॑ जरि॒तारः॑ । ज॒रि॒तारो॒ न । न भू॒ना । भू॒नेति॑ भू॒ना ॥ अ॒सूर्ता॒ सूर्ता᳚ । सूर्ता॒ रज॑सः । रज॑सो वि॒माने᳚ । वि॒माने॒ ये । वि॒मान॒ इति॑ वि - माने᳚ । ये भू॒तानि॑ । भू॒तानि॑ स॒मकृ॑ण्वन्न् । स॒मकृ॑ण्वन्नि॒मानि॑ । स॒मकृ॑ण्व॒न्निति॑ सम् - अकृ॑ण्वन्न् । इ॒मानीती॒मानि॑ ॥ न तम् । तं ॅवि॑दाथ । वि॒दा॒थ॒ यः । य इ॒दम् । इ॒दम् ज॒जान॑ । ज॒जाना॒न्यत् । अ॒न्यद् यु॒ष्माक᳚म् । यु॒ष्माक॒मन्त॑रम् । अन्त॑रम् भवाति । भ॒वा॒तीति॑ भवाति ॥ नी॒हा॒रेण॒ प्रावृ॑ताः । प्रावृ॑ता॒ जल्प्या᳚ । जल्प्या॑ च । चा॒सु॒तृपः॑ । अ॒सु॒तृप॑ उक्थ॒शासः॑ । अ॒सु॒तृप॒ इत्य॑सु - तृपः॑ । उ॒क्थ॒शास॑श्चरन्ति । उ॒क्थ॒शास॒ इत्यु॑क्थ - शासः॑ । च॒र॒न्तीति॑ चरन्ति ॥ प॒रो दि॒वा । दि॒वा प॒रः । प॒र ए॒ना । ए॒ना पृ॑थि॒व्या \newline

\textbf{Jatai Paata} \newline

1. यो दे॒वाना᳚म् दे॒वानां॒ ॅयो यो दे॒वाना᳚म् । \newline
2. दे॒वाना᳚म् नाम॒धा ना॑म॒धा दे॒वाना᳚म् दे॒वाना᳚म् नाम॒धाः । \newline
3. ना॒म॒धा एक॒ एको॑ नाम॒धा ना॑म॒धा एकः॑ । \newline
4. ना॒म॒धा इति॑ नाम - धाः । \newline
5. एक॑ ए॒वैवैक॒ एक॑ ए॒व । \newline
6. ए॒व तम् त मे॒वैव तम् । \newline
7. तꣳ सं॑प्र॒श्ञꣳ सं॑प्र॒श्ञम् तम् तꣳ सं॑प्र॒श्ञम् । \newline
8. सं॒प्र॒श्ञम् भुव॑ना॒ भुव॑ना संप्र॒श्ञꣳ सं॑प्र॒श्ञम् भुव॑ना । \newline
9. सं॒प्र॒श्ञमिति॑ सं - प्र॒श्ञम् । \newline
10. भुव॑ना यन्ति यन्ति॒ भुव॑ना॒ भुव॑ना यन्ति । \newline
11. य॒न्त्य॒न्या ऽन्या य॑न्ति यन्त्य॒न्या । \newline
12. अ॒न्येत्य॒न्या । \newline
13. त आ ते त आ । \newline
14. आ ऽय॑जन्ता यज॒न्ता ऽय॑जन्त । \newline
15. अ॒य॒ज॒न्त॒ द्रवि॑ण॒म् द्रवि॑ण मयजन्ता यजन्त॒ द्रवि॑णम् । \newline
16. द्रवि॑णꣳ॒॒ सꣳ सम् द्रवि॑ण॒म् द्रवि॑णꣳ॒॒ सम् । \newline
17. स म॑स्मा अस्मै॒ सꣳ स म॑स्मै । \newline
18. अ॒स्मा॒ ऋष॑य॒ ऋष॑यो अस्मा अस्मा॒ ऋष॑यः । \newline
19. ऋष॑यः॒ पूर्वे॒ पूर्व॒ ऋष॑य॒ ऋष॑यः॒ पूर्वे᳚ । \newline
20. पूर्वे॑ जरि॒तारो॑ जरि॒तारः॒ पूर्वे॒ पूर्वे॑ जरि॒तारः॑ । \newline
21. ज॒रि॒तारो॑ न न जरि॒तारो॑ जरि॒तारो॑ न । \newline
22. न॒ भू॒ना भू॒ना न॑ न भू॒ना । \newline
23. भू॒नेति॑ भू॒ना । \newline
24. अ॒सूर्ता॒ सूर्ता॒ सूर्ता॒ ऽसूर्ता॒ ऽसूर्ता॒ सूर्ता᳚ । \newline
25. सूर्ता॒ रज॑सो॒ रज॑सः॒ सूर्ता॒ सूर्ता॒ रज॑सः । \newline
26. रज॑सो वि॒माने॑ वि॒माने॒ रज॑सो॒ रज॑सो वि॒माने᳚ । \newline
27. वि॒माने॒ ये ये वि॒माने॑ वि॒माने॒ ये । \newline
28. वि॒मान॒ इति॑ वि - माने᳚ । \newline
29. ये भू॒तानि॑ भू॒तानि॒ ये ये भू॒तानि॑ । \newline
30. भू॒तानि॑ स॒मकृ॑ण्वन् थ्स॒मकृ॑ण्वन् भू॒तानि॑ भू॒तानि॑ स॒मकृ॑ण्वन्न् । \newline
31. स॒मकृ॑ण्वन् नि॒मानी॒ मानि॑ स॒मकृ॑ण्वन् थ्स॒मकृ॑ण्वन् नि॒मानि॑ । \newline
32. स॒मकृ॑ण्व॒न्निति॑ सं - अकृ॑ण्वन्न् । \newline
33. इ॒मानीती॒मानि॑ । \newline
34. न तम् तम् न न तम् । \newline
35. तं ॅवि॑दाथ विदाथ॒ तम् तं ॅवि॑दाथ । \newline
36. वि॒दा॒थ॒ यो यो वि॑दाथ विदाथ॒ यः । \newline
37. य इ॒द मि॒दं ॅयो य इ॒दम् । \newline
38. इ॒दम् ज॒जान॑ ज॒जाने॒द मि॒दम् ज॒जान॑ । \newline
39. ज॒जाना॒ न्य द॒न्यज् ज॒जान॑ ज॒जाना॒न्यत् । \newline
40. अ॒न्यद् यु॒ष्माकं॑ ॅयु॒ष्माक॑ म॒न्य द॒न्यद् यु॒ष्माक᳚म् । \newline
41. यु॒ष्माक॒ मन्त॑र॒ मन्त॑रं ॅयु॒ष्माकं॑ ॅयु॒ष्माक॒ मन्त॑रम् । \newline
42. अन्त॑रम् भवाति भवा॒ त्यन्त॑र॒ मन्त॑रम् भवाति । \newline
43. भ॒वा॒तीति॑ भवाति । \newline
44. नी॒हा॒रेण॒ प्रावृ॑ताः॒ प्रावृ॑ता नीहा॒रेण॑ नीहा॒रेण॒ प्रावृ॑ताः । \newline
45. प्रावृ॑ता॒ जल्प्या॒ जल्प्या॒ प्रावृ॑ताः॒ प्रावृ॑ता॒ जल्प्या᳚ । \newline
46. जल्प्या॑ च च॒ जल्प्या॒ जल्प्या॑ च । \newline
47. चा॒सु॒तृपो॑ असु॒तृप॑श्च चासु॒तृपः॑ । \newline
48. अ॒सु॒तृप॑ उक्थ॒शास॑ उक्थ॒शासो॑ असु॒तृपो॑ असु॒तृप॑ उक्थ॒शासः॑ । \newline
49. अ॒सु॒तृप॒ इत्य॑सु - तृपः॑ । \newline
50. उ॒क्थ॒शास॑ श्चरन्ति चर न्त्युक्थ॒शास॑ उक्थ॒शास॑ श्चरन्ति । \newline
51. उ॒क्थ॒शास॒ इत्यु॑क्थ - शासः॑ । \newline
52. च॒र॒न्तीति॑ चरन्ति । \newline
53. प॒रो दि॒वा दि॒वा प॒रः प॒रो दि॒वा । \newline
54. दि॒वा प॒रः प॒रो दि॒वा दि॒वा प॒रः । \newline
55. प॒र ए॒नैना प॒रः प॒र ए॒ना । \newline
56. ए॒ना पृ॑थि॒व्या पृ॑थि॒व्यैनैना पृ॑थि॒व्या । \newline

\textbf{Ghana Paata } \newline

1. यो दे॒वाना᳚म् दे॒वानां॒ ॅयो यो दे॒वाना᳚म् नाम॒धा ना॑म॒धा दे॒वानां॒ ॅयो यो दे॒वाना᳚म् नाम॒धाः । \newline
2. दे॒वाना᳚म् नाम॒धा ना॑म॒धा दे॒वाना᳚म् दे॒वाना᳚म् नाम॒धा एक॒ एको॑ नाम॒धा दे॒वाना᳚म् दे॒वाना᳚म् नाम॒धा एकः॑ । \newline
3. ना॒म॒धा एक॒ एको॑ नाम॒धा ना॑म॒धा एक॑ ए॒वैवैको॑ नाम॒धा ना॑म॒धा एक॑ ए॒व । \newline
4. ना॒म॒धा इति॑ नाम - धाः । \newline
5. एक॑ ए॒वैवैक॒ एक॑ ए॒व तम् त मे॒वैक॒ एक॑ ए॒व तम् । \newline
6. ए॒व तम् त मे॒वैव तꣳ सं॑प्र॒श्ञꣳ सं॑प्र॒श्ञम् त मे॒वैव तꣳ सं॑प्र॒श्ञम् । \newline
7. तꣳ सं॑प्र॒श्ञꣳ सं॑प्र॒श्ञम् तम् तꣳ सं॑प्र॒श्ञम् भुव॑ना॒ भुव॑ना संप्र॒श्ञम् तम् तꣳ सं॑प्र॒श्ञम् भुव॑ना । \newline
8. सं॒प्र॒श्ञम् भुव॑ना॒ भुव॑ना संप्र॒श्ञꣳ सं॑प्र॒श्ञम् भुव॑ना यन्ति यन्ति॒ भुव॑ना संप्र॒श्ञꣳ सं॑प्र॒श्ञम् भुव॑ना यन्ति । \newline
9. सं॒प्र॒श्ञमिति॑ सं - प्र॒श्ञम् । \newline
10. भुव॑ना यन्ति यन्ति॒ भुव॑ना॒ भुव॑ना यन् त्य॒न्या ऽन्या य॑न्ति॒ भुव॑ना॒ भुव॑ना यन् त्य॒न्या । \newline
11. य॒न्त्य॒न्या ऽन्या य॑न्ति यन्त्य॒न्या । \newline
12. अ॒न्येत्य॒न्या । \newline
13. त आ ते त आ ऽय॑जन्ता यज॒न्ता ते त आ ऽय॑जन्त । \newline
14. आ ऽय॑जन्ता यज॒न्ता ऽय॑जन्त॒ द्रवि॑ण॒म् द्रवि॑ण मयज॒न्ता ऽय॑जन्त॒ द्रवि॑णम् । \newline
15. अ॒य॒ज॒न्त॒ द्रवि॑ण॒म् द्रवि॑ण मयजन्ता यजन्त॒ द्रवि॑णꣳ॒॒ सꣳ सम् द्रवि॑ण मयजन्ता यजन्त॒ द्रवि॑णꣳ॒॒ सम् । \newline
16. द्रवि॑णꣳ॒॒ सꣳ सम् द्रवि॑ण॒म् द्रवि॑णꣳ॒॒ स म॑स्मा अस्मै॒ सम् द्रवि॑ण॒म् द्रवि॑णꣳ॒॒ स म॑स्मै । \newline
17. स म॑स्मा अस्मै॒ सꣳ स म॑स्मा॒ ऋष॑य॒ ऋष॑यो अस्मै॒ सꣳ स म॑स्मा॒ ऋष॑यः । \newline
18. अ॒स्मा॒ ऋष॑य॒ ऋष॑यो अस्मा अस्मा॒ ऋष॑यः॒ पूर्वे॒ पूर्व॒ ऋष॑यो अस्मा अस्मा॒ ऋष॑यः॒ पूर्वे᳚ । \newline
19. ऋष॑यः॒ पूर्वे॒ पूर्व॒ ऋष॑य॒ ऋष॑यः॒ पूर्वे॑ जरि॒तारो॑ जरि॒तारः॒ पूर्व॒ ऋष॑य॒ ऋष॑यः॒ पूर्वे॑ जरि॒तारः॑ । \newline
20. पूर्वे॑ जरि॒तारो॑ जरि॒तारः॒ पूर्वे॒ पूर्वे॑ जरि॒तारो॑ न न जरि॒तारः॒ पूर्वे॒ पूर्वे॑ जरि॒तारो॑ न । \newline
21. ज॒रि॒तारो॑ न न जरि॒तारो॑ जरि॒तारो॑ न भू॒ना भू॒ना न॑ जरि॒तारो॑ जरि॒तारो॑ न भू॒ना । \newline
22. न॒ भू॒ना भू॒ना न॑ न भू॒ना । \newline
23. भू॒नेति॑ भू॒ना । \newline
24. अ॒सूर्ता॒ सूर्ता॒ सूर्ता॒ ऽसूर्ता॒ ऽसूर्ता॒ सूर्ता॒ रज॑सो॒ रज॑सः॒ सूर्ता॒ ऽसूर्ता॒ ऽसूर्ता॒ सूर्ता॒ रज॑सः । \newline
25. सूर्ता॒ रज॑सो॒ रज॑सः॒ सूर्ता॒ सूर्ता॒ रज॑सो वि॒माने॑ वि॒माने॒ रज॑सः॒ सूर्ता॒ सूर्ता॒ रज॑सो वि॒माने᳚ । \newline
26. रज॑सो वि॒माने॑ वि॒माने॒ रज॑सो॒ रज॑सो वि॒माने॒ ये ये वि॒माने॒ रज॑सो॒ रज॑सो वि॒माने॒ ये । \newline
27. वि॒माने॒ ये ये वि॒माने॑ वि॒माने॒ ये भू॒तानि॑ भू॒तानि॒ ये वि॒माने॑ वि॒माने॒ ये भू॒तानि॑ । \newline
28. वि॒मान॒ इति॑ वि - माने᳚ । \newline
29. ये भू॒तानि॑ भू॒तानि॒ ये ये भू॒तानि॑ स॒मकृ॑ण्वन् थ्स॒मकृ॑ण्वन् भू॒तानि॒ ये ये भू॒तानि॑ स॒मकृ॑ण्वन्न् । \newline
30. भू॒तानि॑ स॒मकृ॑ण्वन् थ्स॒मकृ॑ण्वन् भू॒तानि॑ भू॒तानि॑ स॒मकृ॑ण्वन् नि॒मानी॒मानि॑ स॒मकृ॑ण्वन् भू॒तानि॑ भू॒तानि॑ स॒मकृ॑ण्वन् नि॒मानि॑ । \newline
31. स॒मकृ॑ण्वन् नि॒मानी॒मानि॑ स॒मकृ॑ण्वन् थ्स॒मकृ॑ण्वन् नि॒मानि॑ । \newline
32. स॒मकृ॑ण्व॒न्निति॑ सं - अकृ॑ण्वन्न् । \newline
33. इ॒मानीती॒मानि॑ । \newline
34. न तम् तम् न न तं ॅवि॑दाथ विदाथ॒ तम् न न तं ॅवि॑दाथ । \newline
35. तं ॅवि॑दाथ विदाथ॒ तम् तं ॅवि॑दाथ॒ यो यो वि॑दाथ॒ तम् तं ॅवि॑दाथ॒ यः । \newline
36. वि॒दा॒थ॒ यो यो वि॑दाथ विदाथ॒ य इ॒द मि॒दं ॅयो वि॑दाथ विदाथ॒ य इ॒दम् । \newline
37. य इ॒द मि॒दं ॅयो य इ॒दम् ज॒जान॑ ज॒जाने॒दं ॅयो य इ॒दम् ज॒जान॑ । \newline
38. इ॒दम् ज॒जान॑ ज॒जाने॒द मि॒दम् ज॒जा ना॒न्य द॒न्यज् ज॒जाने॒द मि॒दम् ज॒जा ना॒न्यत् । \newline
39. ज॒जा ना॒न्य द॒न्यज् ज॒जान॑ ज॒जा ना॒न्यद् यु॒ष्माकं॑ ॅयु॒ष्माक॑ म॒न्यज् ज॒जान॑ ज॒जा ना॒न्यद् यु॒ष्माक᳚म् । \newline
40. अ॒न्यद् यु॒ष्माकं॑ ॅयु॒ष्माक॑ म॒न्य द॒न्यद् यु॒ष्माक॒ मन्त॑र॒ मन्त॑रं ॅयु॒ष्माक॑ म॒न्य द॒न्यद् यु॒ष्माक॒ मन्त॑रम् । \newline
41. यु॒ष्माक॒ मन्त॑र॒ मन्त॑रं ॅयु॒ष्माकं॑ ॅयु॒ष्माक॒ मन्त॑रम् भवाति भवा॒ त्यन्त॑रं ॅयु॒ष्माकं॑ ॅयु॒ष्माक॒ मन्त॑रम् भवाति । \newline
42. अन्त॑रम् भवाति भवा॒ त्यन्त॑र॒ मन्त॑रम् भवाति । \newline
43. भ॒वा॒तीति॑ भवाति । \newline
44. नी॒हा॒रेण॒ प्रावृ॑ताः॒ प्रावृ॑ता नीहा॒रेण॑ नीहा॒रेण॒ प्रावृ॑ता॒ जल्प्या॒ जल्प्या॒ प्रावृ॑ता नीहा॒रेण॑ नीहा॒रेण॒ प्रावृ॑ता॒ जल्प्या᳚ । \newline
45. प्रावृ॑ता॒ जल्प्या॒ जल्प्या॒ प्रावृ॑ताः॒ प्रावृ॑ता॒ जल्प्या॑ च च॒ जल्प्या॒ प्रावृ॑ताः॒ प्रावृ॑ता॒ जल्प्या॑ च । \newline
46. जल्प्या॑ च च॒ जल्प्या॒ जल्प्या॑ चासु॒तृपो॑ असु॒तृप॑श्च॒ जल्प्या॒ जल्प्या॑ चासु॒तृपः॑ । \newline
47. चा॒सु॒तृपो॑ असु॒तृप॑श्च चासु॒तृप॑ उक्थ॒शास॑ उक्थ॒शासो॑ असु॒तृप॑श्च चासु॒तृप॑ उक्थ॒शासः॑ । \newline
48. अ॒सु॒तृप॑ उक्थ॒शास॑ उक्थ॒शासो॑ असु॒तृपो॑ असु॒तृप॑ उक्थ॒शास॑ श्चरन्ति चरन् त्युक्थ॒शासो॑ असु॒तृपो॑ असु॒तृप॑ उक्थ॒शास॑ श्चरन्ति । \newline
49. अ॒सु॒तृप॒ इत्य॑सु - तृपः॑ । \newline
50. उ॒क्थ॒शास॑ श्चरन्ति चरन् त्युक्थ॒शास॑ उक्थ॒शास॑ श्चरन्ति । \newline
51. उ॒क्थ॒शास॒ इत्यु॑क्थ - शासः॑ । \newline
52. च॒र॒न्तीति॑ चरन्ति । \newline
53. प॒रो दि॒वा दि॒वा प॒रः प॒रो दि॒वा प॒रः प॒रो दि॒वा प॒रः प॒रो दि॒वा प॒रः । \newline
54. दि॒वा प॒रः प॒रो दि॒वा दि॒वा प॒र ए॒नैना प॒रो दि॒वा दि॒वा प॒र ए॒ना । \newline
55. प॒र ए॒नैना प॒रः प॒र ए॒ना पृ॑थि॒व्या पृ॑थि॒व्यैना प॒रः प॒र ए॒ना पृ॑थि॒व्या । \newline
56. ए॒ना पृ॑थि॒व्या पृ॑थि॒ व्यैनैना पृ॑थि॒व्या प॒रः प॒रः पृ॑थि॒ व्यैनैना पृ॑थि॒व्या प॒रः । \newline
\pagebreak
\markright{ TS 4.6.2.3  \hfill https://www.vedavms.in \hfill}

\section{ TS 4.6.2.3 }

\textbf{TS 4.6.2.3 } \newline
\textbf{Samhita Paata} \newline

पृ॑थि॒व्या प॒रो दे॒वेभि॒रसु॑रै॒र्गुहा॒ यत् । कꣳ स्वि॒द्गर्भं॑ प्रथ॒मं द॑द्ध्र॒ आपो॒ यत्र॑ दे॒वाः स॒मग॑च्छन्त॒ विश्वे᳚ ॥ तमिद्गर्भं॑ प्रथ॒मं द॑द्ध्र॒ आपो॒ यत्र॑ दे॒वाः स॒मग॑च्छन्त॒ विश्वे᳚ । अ॒जस्य॒ नाभा॒वद्ध्येक॒-मर्पि॑तं॒ ॅयस्मि॑न्नि॒दं ॅविश्वं॒ भुव॑न॒मधि॑ श्रि॒तं ॥ वि॒श्वक॑र्मा॒ ह्यज॑निष्ट दे॒व आदिद्-ग॑न्ध॒र्वो अ॑भवद् द्वि॒तीयः॑ । तृ॒तीयः॑ पि॒ता ज॑नि॒तौष॑धीना - [  ] \newline

\textbf{Pada Paata} \newline

पृ॒थि॒व्या । प॒रः । दे॒वेभिः॑ । असु॑रैः । गुहा᳚ । यत् ॥ कम् । स्वि॒त् । गर्भ᳚म् । प्र॒थ॒मम् । द॒द्ध्रे॒ । आपः॑ । यत्र॑ । दे॒वाः । स॒मग॑च्छ॒न्तेति॑ सं - अग॑च्छन्त । विश्वे᳚ ॥ तम् । इत् । गर्भ᳚म् । प्र॒थ॒मम् । द॒द्ध्रे॒ । आपः॑ । यत्र॑ । दे॒वाः । स॒मग॑च्छ॒न्तेति॑ सं - अग॑च्छन्त । विश्वे᳚ ॥ अ॒जस्य॑ । नाभौ᳚ । अधीति॑ । एक᳚म् । अर्पि॑तम् । यस्मिन्न्॑ । इ॒दम् । विश्व᳚म् । भुव॑नम् । अधीति॑ । श्रि॒तम् ॥ वि॒श्वक॒र्मेति॑ वि॒श्व - क॒र्मा॒ । हि । अज॑निष्ट । दे॒वः । आत् । इत् । ग॒न्ध॒र्वः । अ॒भ॒व॒त् । द्वि॒तीयः॑ ॥ तृ॒तीयः॑ । पि॒ता । ज॒नि॒ता । ओष॑धीनाम् ।  \newline


\textbf{Krama Paata} \newline

पृ॒थि॒व्या प॒रः । प॒रो दे॒वेभिः॑ । दे॒वेभि॒रसु॑रैः । असु॑रै॒र् गुहा᳚ । गुहा॒ यत् । यदिति॒ यत् ॥ कꣳ स्वि॑त् । स्वि॒द् गर्भ᳚म् । गर्भ॑म् प्रथ॒मम् । प्र॒थ॒मम् द॑द्ध्रे । द॒द्ध्र॒ आपः॑ । आपो॒ यत्र॑ । यत्र॑ दे॒वाः । दे॒वाः स॒मग॑च्छन्त । स॒मग॑च्छन्त॒ विश्वे᳚ । स॒मग॑च्छ॒न्तेति॑ सम् - अग॑च्छन्त । विश्व॒ इति॒ विश्वे᳚ ॥ तमित् । इद् गर्भ᳚म् । गर्भ॑म् प्रथ॒मम् । प्र॒थ॒मम् द॑द्ध्रे । द॒द्ध्र॒ आपः॑ । आपो॒ यत्र॑ । यत्र॑ दे॒वाः । दे॒वाः स॒मग॑च्छन्त । स॒मग॑च्छन्त॒ विश्वे᳚ । स॒मग॑च्छ॒न्तेति॑ सम् - अग॑च्छन्त । विश्व॒ इति॒ विश्वे᳚ ॥ अ॒जस्य॒ नाभौ᳚ । नाभा॒वधि॑ । अद्ध्येक᳚म् । एक॒मर्पि॑तम् । अर्पि॑तं॒ ॅयस्मिन्न्॑ । यस्मि॑न्नि॒दम् । इ॒दं ॅविश्व᳚म् । विश्व॒म् भुव॑नम् । भुव॑न॒मधि॑ । अधि॑ श्रि॒तम् । श्रि॒तमिति॑ श्रि॒तम् ॥ वि॒श्वक॑र्मा॒ हि । वि॒श्वक॒र्मेति॑ वि॒श्व - क॒र्मा॒ । ह्यज॑निष्ट । अज॑निष्ट दे॒वः । दे॒व आत् । आदित् । इद् ग॑न्ध॒र्वः । ग॒न्ध॒र्वो अ॑भवत् । अ॒भ॒व॒द् द्वि॒तीयः॑ । द्वि॒तीय॒ इति॑ द्वि॒तीयः॑ ॥ तृ॒तीयः॑ पि॒ता । पि॒ता ज॑नि॒ता । ज॒नि॒तौष॑धीनाम् । ओष॑धीनाम॒पाम् \newline

\textbf{Jatai Paata} \newline

1. पृ॒थि॒व्या प॒रः प॒रः पृ॑थि॒व्या पृ॑थि॒व्या प॒रः । \newline
2. प॒रो दे॒वेभि॑र् दे॒वेभिः॑ प॒रः प॒रो दे॒वेभिः॑ । \newline
3. दे॒वेभि॒ रसु॑रै॒ रसु॑रैर् दे॒वेभि॑र् दे॒वेभि॒ रसु॑रैः । \newline
4. असु॑रै॒र् गुहा॒ गुहा॑ ऽसुरै॒ रसु॑रै॒र् गुहा᳚ । \newline
5. गुहा॒ यद् यद् गुहा॒ गुहा॒ यत् । \newline
6. यदिति॒ यत् । \newline
7. कꣳ स्वि॑थ् स्वि॒त् कम् कꣳ स्वि॑त् । \newline
8. स्वि॒द् गर्भ॒म् गर्भꣳ॑ स्विथ् स्वि॒द् गर्भ᳚म् । \newline
9. गर्भ॑म् प्रथ॒मम् प्र॑थ॒मम् गर्भ॒म् गर्भ॑म् प्रथ॒मम् । \newline
10. प्र॒थ॒मम् द॑द्ध्रे दद्ध्रे प्रथ॒मम् प्र॑थ॒मम् द॑द्ध्रे । \newline
11. द॒द्ध्र॒ आप॒ आपो॑ दद्ध्रे दद्ध्र॒ आपः॑ । \newline
12. आपो॒ यत्र॒ यत्राप॒ आपो॒ यत्र॑ । \newline
13. यत्र॑ दे॒वा दे॒वा यत्र॒ यत्र॑ दे॒वाः । \newline
14. दे॒वाः स॒मग॑च्छन्त स॒मग॑च्छन्त दे॒वा दे॒वाः स॒मग॑च्छन्त । \newline
15. स॒मग॑च्छन्त॒ विश्वे॒ विश्वे॑ स॒मग॑च्छन्त स॒मग॑च्छन्त॒ विश्वे᳚ । \newline
16. स॒मग॑च्छ॒न्तेति॑ सं - अग॑च्छन्त । \newline
17. विश्व॒ इति॒ विश्वे᳚ । \newline
18. त मिदित् तम् त मित् । \newline
19. इद् गर्भ॒म् गर्भ॒ मिदिद् गर्भ᳚म् । \newline
20. गर्भ॑म् प्रथ॒मम् प्र॑थ॒मम् गर्भ॒म् गर्भ॑म् प्रथ॒मम् । \newline
21. प्र॒थ॒मम् द॑द्ध्रे दद्ध्रे प्रथ॒मम् प्र॑थ॒मम् द॑द्ध्रे । \newline
22. द॒द्ध्र॒ आप॒ आपो॑ दद्ध्रे दद्ध्र॒ आपः॑ । \newline
23. आपो॒ यत्र॒ यत्राप॒ आपो॒ यत्र॑ । \newline
24. यत्र॑ दे॒वा दे॒वा यत्र॒ यत्र॑ दे॒वाः । \newline
25. दे॒वाः स॒मग॑च्छन्त स॒मग॑च्छन्त दे॒वा दे॒वाः स॒मग॑च्छन्त । \newline
26. स॒मग॑च्छन्त॒ विश्वे॒ विश्वे॑ स॒मग॑च्छन्त स॒मग॑च्छन्त॒ विश्वे᳚ । \newline
27. स॒मग॑च्छ॒न्तेति॑ सं - अग॑च्छन्त । \newline
28. विश्व॒ इति॒ विश्वे᳚ । \newline
29. अ॒जस्य॒ नाभौ॒ नाभा॑ व॒जस्या॒ जस्य॒ नाभौ᳚ । \newline
30. नाभा॒ वध्यधि॒ नाभौ॒ नाभा॒ वधि॑ । \newline
31. अध्येक॒ मेक॒ मध्य ध्येक᳚म् । \newline
32. एक॒ मर्पि॑त॒ मर्पि॑त॒ मेक॒ मेक॒ मर्पि॑तम् । \newline
33. अर्पि॑तं॒ ॅयस्मि॒न्॒. यस्मि॒न् नर्पि॑त॒ मर्पि॑तं॒ ॅयस्मिन्न्॑ । \newline
34. यस्मि॑न् नि॒द मि॒दं ॅयस्मि॒न्॒. यस्मि॑न् नि॒दम् । \newline
35. इ॒दं ॅविश्वं॒ ॅविश्व॑ मि॒द मि॒दं ॅविश्व᳚म् । \newline
36. विश्व॒म् भुव॑न॒म् भुव॑नं॒ ॅविश्वं॒ ॅविश्व॒म् भुव॑नम् । \newline
37. भुव॑न॒ मध्यधि॒ भुव॑न॒म् भुव॑न॒ मधि॑ । \newline
38. अधि॑ श्रि॒तꣳ श्रि॒त मध्यधि॑ श्रि॒तम् । \newline
39. श्रि॒तमिति॑ श्रि॒तम् । \newline
40. वि॒श्वक॑र्मा॒ हि हि वि॒श्वक॑र्मा वि॒श्वक॑र्मा॒ हि । \newline
41. वि॒श्वक॒र्मेति॑ वि॒श्व - क॒र्मा॒ । \newline
42. ह्यज॑नि॒ष्टा ज॑निष्ट॒ हि ह्यज॑निष्ट । \newline
43. अज॑निष्ट दे॒वो दे॒वो ऽज॑नि॒ष्टा ज॑निष्ट दे॒वः । \newline
44. दे॒व आदाद् दे॒वो दे॒व आत् । \newline
45. आदि दिदा दा दित् । \newline
46. इद् ग॑न्ध॒र्वो ग॑न्ध॒र्व इदिद् ग॑न्ध॒र्वः । \newline
47. ग॒न्ध॒र्वो अ॑भव दभवद् गन्ध॒र्वो ग॑न्ध॒र्वो अ॑भवत् । \newline
48. अ॒भ॒व॒द् द्वि॒तीयो᳚ द्वि॒तीयो॑ अभव दभवद् द्वि॒तीयः॑ । \newline
49. द्वि॒तीय॒ इति॑ द्वि॒तीयः॑ । \newline
50. तृ॒तीयः॑ पि॒ता पि॒ता तृ॒तीय॑ स्तृ॒तीयः॑ पि॒ता । \newline
51. पि॒ता ज॑नि॒ता ज॑नि॒ता पि॒ता पि॒ता ज॑नि॒ता । \newline
52. ज॒नि॒तौष॑धीना॒ मोष॑धीनाम् जनि॒ता ज॑नि॒तौष॑धीनाम् । \newline
53. ओष॑धीना म॒पा म॒पा मोष॑धीना॒ मोष॑धीना म॒पाम् । \newline

\textbf{Ghana Paata } \newline

1. पृ॒थि॒व्या प॒रः प॒रः पृ॑थि॒व्या पृ॑थि॒व्या प॒रो दे॒वेभि॑र् दे॒वेभिः॑ प॒रः पृ॑थि॒व्या पृ॑थि॒व्या प॒रो दे॒वेभिः॑ । \newline
2. प॒रो दे॒वेभि॑र् दे॒वेभिः॑ प॒रः प॒रो दे॒वेभि॒ रसु॑रै॒ रसु॑रैर् दे॒वेभिः॑ प॒रः प॒रो दे॒वेभि॒ रसु॑रैः । \newline
3. दे॒वेभि॒ रसु॑रै॒ रसु॑रैर् दे॒वेभि॑र् दे॒वेभि॒ रसु॑रै॒र् गुहा॒ गुहा॑ ऽसुरैर् दे॒वेभि॑र् दे॒वेभि॒ रसु॑रै॒र् गुहा᳚ । \newline
4. असु॑रै॒र् गुहा॒ गुहा॑ ऽसुरै॒ रसु॑रै॒र् गुहा॒ यद् यद् गुहा॑ ऽसुरै॒ रसु॑रै॒र् गुहा॒ यत् । \newline
5. गुहा॒ यद् यद् गुहा॒ गुहा॒ यत् । \newline
6. यदिति॒ यत् । \newline
7. कꣳ स्वि॑थ् स्वि॒त् कम् कꣳ स्वि॒द् गर्भ॒म् गर्भꣳ॑ स्वि॒त् कम् कꣳ स्वि॒द् गर्भ᳚म् । \newline
8. स्वि॒द् गर्भ॒म् गर्भꣳ॑ स्विथ् स्वि॒द् गर्भ॑म् प्रथ॒मम् प्र॑थ॒मम् गर्भꣳ॑ स्विथ् स्वि॒द् गर्भ॑म् प्रथ॒मम् । \newline
9. गर्भ॑म् प्रथ॒मम् प्र॑थ॒मम् गर्भ॒म् गर्भ॑म् प्रथ॒मम् द॑द्ध्रे दद्ध्रे प्रथ॒मम् गर्भ॒म् गर्भ॑म् प्रथ॒मम् द॑द्ध्रे । \newline
10. प्र॒थ॒मम् द॑द्ध्रे दद्ध्रे प्रथ॒मम् प्र॑थ॒मम् द॑द्ध्र॒ आप॒ आपो॑ दद्ध्रे प्रथ॒मम् प्र॑थ॒मम् द॑द्ध्र॒ आपः॑ । \newline
11. द॒द्ध्र॒ आप॒ आपो॑ दद्ध्रे दद्ध्र॒ आपो॒ यत्र॒ यत्रापो॑ दद्ध्रे दद्ध्र॒ आपो॒ यत्र॑ । \newline
12. आपो॒ यत्र॒ यत्राप॒ आपो॒ यत्र॑ दे॒वा दे॒वा यत्राप॒ आपो॒ यत्र॑ दे॒वाः । \newline
13. यत्र॑ दे॒वा दे॒वा यत्र॒ यत्र॑ दे॒वाः स॒मग॑च्छन्त स॒मग॑च्छन्त दे॒वा यत्र॒ यत्र॑ दे॒वाः स॒मग॑च्छन्त । \newline
14. दे॒वाः स॒मग॑च्छन्त स॒मग॑च्छन्त दे॒वा दे॒वाः स॒मग॑च्छन्त॒ विश्वे॒ विश्वे॑ स॒मग॑च्छन्त दे॒वा दे॒वाः स॒मग॑च्छन्त॒ विश्वे᳚ । \newline
15. स॒मग॑च्छन्त॒ विश्वे॒ विश्वे॑ स॒मग॑च्छन्त स॒मग॑च्छन्त॒ विश्वे᳚ । \newline
16. स॒मग॑च्छ॒न्तेति॑ सं - अग॑च्छन्त । \newline
17. विश्व॒ इति॒ विश्वे᳚ । \newline
18. त मिदित् तम् त मिद् गर्भ॒म् गर्भ॒ मित् तम् त मिद् गर्भ᳚म् । \newline
19. इद् गर्भ॒म् गर्भ॒ मिदिद् गर्भ॑म् प्रथ॒मम् प्र॑थ॒मम् गर्भ॒ मिदिद् गर्भ॑म् प्रथ॒मम् । \newline
20. गर्भ॑म् प्रथ॒मम् प्र॑थ॒मम् गर्भ॒म् गर्भ॑म् प्रथ॒मम् द॑द्ध्रे दद्ध्रे प्रथ॒मम् गर्भ॒म् गर्भ॑म् प्रथ॒मम् द॑द्ध्रे । \newline
21. प्र॒थ॒मम् द॑द्ध्रे दद्ध्रे प्रथ॒मम् प्र॑थ॒मम् द॑द्ध्र॒ आप॒ आपो॑ दद्ध्रे प्रथ॒मम् प्र॑थ॒मम् द॑द्ध्र॒ आपः॑ । \newline
22. द॒द्ध्र॒ आप॒ आपो॑ दद्ध्रे दद्ध्र॒ आपो॒ यत्र॒ यत्रापो॑ दद्ध्रे दद्ध्र॒ आपो॒ यत्र॑ । \newline
23. आपो॒ यत्र॒ यत्राप॒ आपो॒ यत्र॑ दे॒वा दे॒वा यत्राप॒ आपो॒ यत्र॑ दे॒वाः । \newline
24. यत्र॑ दे॒वा दे॒वा यत्र॒ यत्र॑ दे॒वाः स॒मग॑च्छन्त स॒मग॑च्छन्त दे॒वा यत्र॒ यत्र॑ दे॒वाः स॒मग॑च्छन्त । \newline
25. दे॒वाः स॒मग॑च्छन्त स॒मग॑च्छन्त दे॒वा दे॒वाः स॒मग॑च्छन्त॒ विश्वे॒ विश्वे॑ स॒मग॑च्छन्त दे॒वा दे॒वाः स॒मग॑च्छन्त॒ विश्वे᳚ । \newline
26. स॒मग॑च्छन्त॒ विश्वे॒ विश्वे॑ स॒मग॑च्छन्त स॒मग॑च्छन्त॒ विश्वे᳚ । \newline
27. स॒मग॑च्छ॒न्तेति॑ सं - अग॑च्छन्त । \newline
28. विश्व॒ इति॒ विश्वे᳚ । \newline
29. अ॒जस्य॒ नाभौ॒ नाभा॑ व॒जस्या॒ जस्य॒ नाभा॒ वध्यधि॒ नाभा॑ व॒जस्या॒ जस्य॒ नाभा॒ वधि॑ । \newline
30. नाभा॒ वध्यधि॒ नाभौ॒ नाभा॒ वध्येक॒ मेक॒ मधि॒ नाभौ॒ नाभा॒ वध्येक᳚म् । \newline
31. अध्येक॒ मेक॒ मध्यध्येक॒ मर्पि॑त॒ मर्पि॑त॒ मेक॒ मध्यध्येक॒ मर्पि॑तम् । \newline
32. एक॒ मर्पि॑त॒ मर्पि॑त॒ मेक॒ मेक॒ मर्पि॑तं॒ ॅयस्मि॒न्॒. यस्मि॒न् नर्पि॑त॒ मेक॒ मेक॒ मर्पि॑तं॒ ॅयस्मिन्न्॑ । \newline
33. अर्पि॑तं॒ ॅयस्मि॒न्॒. यस्मि॒न् नर्पि॑त॒ मर्पि॑तं॒ ॅयस्मि॑न् नि॒द मि॒दं ॅयस्मि॒न् नर्पि॑त॒ मर्पि॑तं॒ ॅयस्मि॑न् नि॒दम् । \newline
34. यस्मि॑न् नि॒द मि॒दं ॅयस्मि॒न्॒. यस्मि॑न् नि॒दं ॅविश्वं॒ ॅविश्व॑ मि॒दं ॅयस्मि॒न्॒. यस्मि॑न् नि॒दं ॅविश्व᳚म् । \newline
35. इ॒दं ॅविश्वं॒ ॅविश्व॑ मि॒द मि॒दं ॅविश्व॒म् भुव॑न॒म् भुव॑नं॒ ॅविश्व॑ मि॒द मि॒दं ॅविश्व॒म् भुव॑नम् । \newline
36. विश्व॒म् भुव॑न॒म् भुव॑नं॒ ॅविश्वं॒ ॅविश्व॒म् भुव॑न॒ मध्यधि॒ भुव॑नं॒ ॅविश्वं॒ ॅविश्व॒म् भुव॑न॒ मधि॑ । \newline
37. भुव॑न॒ मध्यधि॒ भुव॑न॒म् भुव॑न॒ मधि॑ श्रि॒तꣳ श्रि॒त मधि॒ भुव॑न॒म् भुव॑न॒ मधि॑ श्रि॒तम् । \newline
38. अधि॑ श्रि॒तꣳ श्रि॒त मध्यधि॑ श्रि॒तम् । \newline
39. श्रि॒तमिति॑ श्रि॒तम् । \newline
40. वि॒श्वक॑र्मा॒ हि हि वि॒श्वक॑र्मा वि॒श्वक॑र्मा॒ ह्यज॑नि॒ष्टा ज॑निष्ट॒ हि वि॒श्वक॑र्मा वि॒श्वक॑र्मा॒ ह्यज॑निष्ट । \newline
41. वि॒श्वक॒र्मेति॑ वि॒श्व - क॒र्मा॒ । \newline
42. ह्यज॑नि॒ष्टा ज॑निष्ट॒ हि ह्यज॑निष्ट दे॒वो दे॒वो ऽज॑निष्ट॒ हि ह्यज॑निष्ट दे॒वः । \newline
43. अज॑निष्ट दे॒वो दे॒वो ऽज॑नि॒ष्टा ज॑निष्ट दे॒व आदाद् दे॒वो ऽज॑नि॒ष्टा ज॑निष्ट दे॒व आत् । \newline
44. दे॒व आदाद् दे॒वो दे॒व आदिदि दाद् दे॒वो दे॒व आदित् । \newline
45. आदिदि दादा दिद् ग॑न्ध॒र्वो ग॑न्ध॒र्व इदादा दिद् ग॑न्ध॒र्वः । \newline
46. इद् ग॑न्ध॒र्वो ग॑न्ध॒र्व इदिद् ग॑न्ध॒र्वो अ॑भव दभवद् गन्ध॒र्व इदिद् ग॑न्ध॒र्वो अ॑भवत् । \newline
47. ग॒न्ध॒र्वो अ॑भव दभवद् गन्ध॒र्वो ग॑न्ध॒र्वो अ॑भवद् द्वि॒तीयो᳚ द्वि॒तीयो॑ अभवद् गन्ध॒र्वो ग॑न्ध॒र्वो अ॑भवद् द्वि॒तीयः॑ । \newline
48. अ॒भ॒व॒द् द्वि॒तीयो᳚ द्वि॒तीयो॑ अभव दभवद् द्वि॒तीयः॑ । \newline
49. द्वि॒तीय॒ इति॑ द्वि॒तीयः॑ । \newline
50. तृ॒तीयः॑ पि॒ता पि॒ता तृ॒तीय॑ स्तृ॒तीयः॑ पि॒ता ज॑नि॒ता ज॑नि॒ता पि॒ता तृ॒तीय॑ स्तृ॒तीयः॑ पि॒ता ज॑नि॒ता । \newline
51. पि॒ता ज॑नि॒ता ज॑नि॒ता पि॒ता पि॒ता ज॑नि॒तौष॑धीना॒ मोष॑धीनाम् जनि॒ता पि॒ता पि॒ता ज॑नि॒तौष॑धीनाम् । \newline
52. ज॒नि॒तौष॑धीना॒ मोष॑धीनाम् जनि॒ता ज॑नि॒तौष॑धीना म॒पा म॒पा मोष॑धीनाम् जनि॒ता ज॑नि॒तौष॑धीना म॒पाम् । \newline
53. ओष॑धीना म॒पा म॒पा मोष॑धीना॒ मोष॑धीना म॒पाम् गर्भ॒म् गर्भ॑ म॒पा मोष॑धीना॒ मोष॑धीना म॒पाम् गर्भ᳚म् । \newline
\pagebreak
\markright{ TS 4.6.2.4  \hfill https://www.vedavms.in \hfill}

\section{ TS 4.6.2.4 }

\textbf{TS 4.6.2.4 } \newline
\textbf{Samhita Paata} \newline

-म॒पां गर्भं॒ ॅव्य॑दधात् पुरु॒त्रा ॥ चक्षु॑षः पि॒ता मन॑सा॒ हि धीरो॑ घृ॒तमे॑ने अजन॒न्नन्न॑माने । य॒देदन्ता॒ अद॑दृꣳहन्त॒ पूर्व॒ आदिद् द्यावा॑पृथि॒वी अ॑प्रथेतां ॥ वि॒श्वत॑-श्चक्षुरु॒त वि॒श्वतो॑मुखो वि॒श्वतो॑हस्त उ॒त वि॒श्वत॑स्पात् । सं बा॒हुभ्यां॒ नम॑ति॒ सं पत॑त्रै॒ र्द्यावा॑पृथि॒वी ज॒नय॑न् दे॒व एकः॑ ॥ किꣳ स्वि॑दासी-दधि॒ष्ठान॑-मा॒रंभ॑णं कत॒मथ् स्वि॒त् किमा॑सीत् । यदी॒ भूमिं॑ ज॒नय॑न् - [  ] \newline

\textbf{Pada Paata} \newline

अ॒पाम् । गर्भ᳚म् । वीति॑ । अ॒द॒धा॒त् । पु॒रु॒त्रेति॑ पुरु - त्रा ॥ चक्षु॑षः । पि॒ता । मन॑सा । हि । धीरः॑ । घृ॒तम् । ए॒ने॒ इति॑ । अ॒ज॒न॒त् । नन्न॑माने॒ इति॑ ॥ य॒दा । इत् । अन्ताः᳚ । अद॑दृꣳहन्त । पूर्वे᳚ । आत् । इत् । द्यावा॑पृथि॒वी इति॒ द्यावा᳚ - पृ॒थि॒वी । अ॒प्र॒थे॒ता॒म् ॥ वि॒श्वत॑श्चक्षु॒रिति॑ वि॒श्वतः॑ - च॒क्षुः॒ । उ॒त । वि॒श्वतो॑मुख॒ इति॑ वि॒श्वतः॑ - मु॒खः॒ । वि॒श्वतो॑हस्त॒ इति॑ वि॒श्वतः॑ - ह॒स्तः॒ । उ॒त । वि॒श्वत॑स्पा॒दिति॑ वि॒श्वतः॑-पा॒त् ॥ समिति॑ । बा॒हुभ्या॒मिति॑ बा॒हु-भ्या॒म् । नम॑ति । समिति॑ । पत॑त्रैः । द्यावा॑पृथि॒वी इति॒ द्यावा᳚ - पृ॒थि॒वी । ज॒नयन्न्॑ । दे॒वः । एकः॑ ॥ किम् । स्वि॒त् । आ॒सी॒त् । अ॒धि॒ष्ठान॒मित्य॑धि - स्थान᳚म् । आ॒रंभ॑ण॒मित्या᳚ - रंभ॑णम् । क॒त॒मत् । स्वि॒त् । किम् । आ॒सी॒त् ॥ यदि॑ । भूमि᳚म् । ज॒नयन्न्॑ ।  \newline


\textbf{Krama Paata} \newline

अ॒पाम् गर्भ᳚म् । गर्भं॒ ॅवि । व्य॑दधात् । अ॒द॒धा॒त्॒ पु॒रु॒त्रा । पु॒रु॒त्रेति॑ पुरु - त्रा ॥ चक्षु॑षः पि॒ता । पि॒ता मन॑सा । मन॑सा॒ हि । हि धीरः॑ । धीरो॑ घृ॒तम् । घृ॒तमे॑ने । ए॒ने॒ अ॒ज॒न॒त्॒ । ए॒ने॒ इत्ये॑ने । अ॒ज॒न॒न् नन्न॑माने । नन्न॑माने॒ इति॒ नन्न॑माने ॥ य॒देत् । इ॒दन्ताः᳚ । अन्ता॒ अद॑दृꣳहन्त । अद॑दृꣳहन्त॒ पूर्वे᳚ । पूर्व॒ आत् । आदित् । इद् द्यावा॑पृथि॒वी । द्यावा॑पृथि॒वी अ॑प्रथेताम् । द्यावा॑पृथि॒वी इति॒ द्यावा᳚ - पृ॒थि॒वी । अ॒प्र॒थे॒ता॒मित्य॑प्रथेताम् ॥ वि॒श्वत॑श्चक्षुरु॒त । वि॒श्वत॑श्चक्षु॒रिति॑ वि॒श्वतः॑ - च॒क्षुः॒ । उ॒त वि॒श्वतो॑मुखः । वि॒श्वतो॑मुखो वि॒श्वतो॑हस्तः । वि॒श्वतो॑मुख॒ इति॑ वि॒श्वतः॑ - मु॒खः॒ । वि॒श्वतो॑हस्त उ॒त । वि॒श्वतो॑हस्त॒ इति॑ वि॒श्वतः॑ - ह॒स्तः॒ । उ॒त वि॒श्वत॑स्पात् । वि॒श्वत॑स्पा॒दिति॑ वि॒श्वतः॑ - पा॒त्॒ ॥ सम् बा॒हुभ्या᳚म् । बा॒हुभ्या॒म् नम॑ति । बा॒हुभ्या॒मिति॑ बा॒हु - भ्या॒म् । नम॑ति॒ सम् । सम् पत॑त्रैः । पत॑त्रै॒र् द्यावा॑पृथि॒वी । द्यावा॑पृथि॒वी ज॒नयन्न्॑ । द्यावा॑पृथि॒वी इति॒ द्यावा᳚ - पृ॒थि॒वी । ज॒नय॑न् दे॒वः । दे॒व एकः॑ । एक॒ इत्येकः॑ ॥ किꣳ स्वि॑त् । स्वि॒दा॒सी॒त्॒ । आ॒सी॒द॒धि॒ष्ठान᳚म् । अ॒धि॒ष्ठान॑मा॒रम्भ॑णम् । अ॒धि॒ष्ठान॒मित्य॑धि - स्थान᳚म् । आ॒रम्भ॑णम् कत॒मत् । आ॒रम्भ॑ण॒मित्या᳚ - रम्भ॑णम् । क॒त॒मथ् स्वि॑त् । स्वि॒त् किम् । किमा॑सीत् । आ॒सी॒दित्या॑सीत् ॥ यदी॒भूमि᳚म् । भूमि॑म् ज॒नयन्न्॑ । ज॒नय॑न् वि॒श्वक॑र्मा \newline

\textbf{Jatai Paata} \newline

1. अ॒पाम् गर्भ॒म् गर्भ॑ म॒पा म॒पाम् गर्भ᳚म् । \newline
2. गर्भं॒ ॅवि वि गर्भ॒म् गर्भं॒ ॅवि । \newline
3. व्य॑दधा ददधा॒द् वि व्य॑दधात् । \newline
4. अ॒द॒धा॒त् पु॒रु॒त्रा पु॑रु॒त्रा ऽद॑धा ददधात् पुरु॒त्रा । \newline
5. पु॒रु॒त्रेति॑ पुरु - त्रा । \newline
6. चक्षु॑षः पि॒ता पि॒ता चक्षु॑ष॒ श्चक्षु॑षः पि॒ता । \newline
7. पि॒ता मन॑सा॒ मन॑सा पि॒ता पि॒ता मन॑सा । \newline
8. मन॑सा॒ हि हि मन॑सा॒ मन॑सा॒ हि । \newline
9. हि धीरो॒ धीरो॒ हि हि धीरः॑ । \newline
10. धीरो॑ घृ॒तम् घृ॒तम् धीरो॒ धीरो॑ घृ॒तम् । \newline
11. घृ॒त मे॑ने एने घृ॒तम् घृ॒त मे॑ने । \newline
12. ए॒ने॒ अ॒ज॒न॒ द॒ज॒न॒ दे॒ने॒ ए॒ने॒ अ॒ज॒न॒त् । \newline
13. ए॒ने॒ इत्ये॑ने । \newline
14. अ॒ज॒न॒न् नन्न॑माने॒ नन्न॑माने अजनदजन॒न् नन्न॑माने । \newline
15. नन्न॑माने॒ इति॒ नन्न॑माने । \newline
16. य॒देदिद् य॒दा य॒देत् । \newline
17. इदन्ता॒ अन्ता॒ इदिदन्ताः᳚ । \newline
18. अन्ता॒ अद॑दृꣳह॒न्ता द॑दृꣳह॒न्ता न्ता॒ अन्ता॒ अद॑दृꣳहन्त । \newline
19. अद॑दृꣳहन्त॒ पूर्वे॒ पूर्वे॒ अद॑दृꣳह॒न्ता द॑दृꣳहन्त॒ पूर्वे᳚ । \newline
20. पूर्व॒ आदात् पूर्वे॒ पूर्व॒ आत् । \newline
21. आदि दिदा दादित् । \newline
22. इद् द्यावा॑पृथि॒वी द्यावा॑पृथि॒वी इदिद् द्यावा॑पृथि॒वी । \newline
23. द्यावा॑पृथि॒वी अ॑प्रथेता मप्रथेता॒म् द्यावा॑पृथि॒वी द्यावा॑पृथि॒वी अ॑प्रथेताम् । \newline
24. द्यावा॑पृथि॒वी इति॒ द्यावा᳚ - पृ॒थि॒वी । \newline
25. अ॒प्र॒थे॒ता॒मित्य॑प्रथेताम् । \newline
26. वि॒श्वत॑श्चक्षु रु॒तोत वि॒श्वत॑श्चक्षुर् वि॒श्वत॑श्चक्षु रु॒त । \newline
27. वि॒श्वत॑श्चक्षु॒रिति॑ वि॒श्वतः॑ - च॒क्षुः॒ । \newline
28. उ॒त वि॒श्वतो॑मुखो वि॒श्वतो॑मुख उ॒तोत वि॒श्वतो॑मुखः । \newline
29. वि॒श्वतो॑मुखो वि॒श्वतो॑हस्तो वि॒श्वतो॑हस्तो वि॒श्वतो॑मुखो वि॒श्वतो॑मुखो वि॒श्वतो॑हस्तः । \newline
30. वि॒श्वतो॑मुख॒ इति॑ वि॒श्वतः॑ - मु॒खः॒ । \newline
31. वि॒श्वतो॑हस्त उ॒तोत वि॒श्वतो॑हस्तो वि॒श्वतो॑हस्त उ॒त । \newline
32. वि॒श्वतो॑हस्त॒ इति॑ वि॒श्वतः॑ - ह॒स्तः॒ । \newline
33. उ॒त वि॒श्वत॑स्पाद् वि॒श्वत॑स्पा दु॒तोत वि॒श्वत॑स्पात् । \newline
34. वि॒श्वत॑स्पा॒दिति॑ वि॒श्वतः॑ - पा॒त् । \newline
35. सम् बा॒हुभ्या᳚म् बा॒हुभ्याꣳ॒॒ सꣳ सम् बा॒हुभ्या᳚म् । \newline
36. बा॒हुभ्या॒म् नम॑ति॒ नम॑ति बा॒हुभ्या᳚म् बा॒हुभ्या॒म् नम॑ति । \newline
37. बा॒हुभ्या॒मिति॑ बा॒हु - भ्या॒म् । \newline
38. नम॑ति॒ सꣳ सम् नम॑ति॒ नम॑ति॒ सम् । \newline
39. सम् पत॑त्रैः॒ पत॑त्रैः॒ सꣳ सम् पत॑त्रैः । \newline
40. पत॑त्रै॒र् द्यावा॑पृथि॒वी द्यावा॑पृथि॒वी पत॑त्रैः॒ पत॑त्रै॒र् द्यावा॑पृथि॒वी । \newline
41. द्यावा॑पृथि॒वी ज॒नय॑न् ज॒नय॒न् द्यावा॑पृथि॒वी द्यावा॑पृथि॒वी ज॒नयन्न्॑ । \newline
42. द्यावा॑पृथि॒वी इति॒ द्यावा᳚ - पृ॒थि॒वी । \newline
43. ज॒नय॑न् दे॒वो दे॒वो ज॒नय॑न् ज॒नय॑न् दे॒वः । \newline
44. दे॒व एक॒ एको॑ दे॒वो दे॒व एकः॑ । \newline
45. एक॒ इत्येकः॑ । \newline
46. किꣳ स्वि॑थ् स्वि॒त् किम् किꣳ स्वि॑त् । \newline
47. स्वि॒दा॒सी॒ दा॒सी॒थ् स्वि॒थ् स्वि॒दा॒सी॒त् । \newline
48. आ॒सी॒द॒ धि॒ष्ठान॑ मधि॒ष्ठान॑ मासी दासी दधि॒ष्ठान᳚म् । \newline
49. अ॒धि॒ष्ठान॑ मा॒रंभ॑ण मा॒रंभ॑ण मधि॒ष्ठान॑ मधि॒ष्ठान॑ मा॒रंभ॑णम् । \newline
50. अ॒धि॒ष्ठान॒मित्य॑धि - स्थान᳚म् । \newline
51. आ॒रंभ॑णम् कत॒मत् क॑त॒म दा॒रंभ॑ण मा॒रंभ॑णम् कत॒मत् । \newline
52. आ॒रंभ॑ण॒मित्या᳚ - रंभ॑णम् । \newline
53. क॒त॒मथ् स्वि॑थ् स्वित् कत॒मत् क॑त॒मथ् स्वि॑त् । \newline
54. स्वि॒त् किम् किꣳ स्वि॑थ् स्वि॒त् किम् । \newline
55. कि मा॑सी दासी॒त् किम् कि मा॑सीत् । \newline
56. आ॒सी॒दित्या॑सीत् । \newline
57. यदी॒ भूमि॒म् भूमिं॒ ॅयदि॒ यदी॒ भूमि᳚म् । \newline
58. भूमि॑म् ज॒नय॑न् ज॒नय॒न् भूमि॒म् भूमि॑म् ज॒नयन्न्॑ । \newline
59. ज॒नय॑न् वि॒श्वक॑र्मा वि॒श्वक॑र्मा ज॒नय॑न् ज॒नय॑न् वि॒श्वक॑र्मा । \newline

\textbf{Ghana Paata } \newline

1. अ॒पाम् गर्भ॒म् गर्भ॑ म॒पा म॒पाम् गर्भं॒ ॅवि वि गर्भ॑ म॒पा म॒पाम् गर्भं॒ ॅवि । \newline
2. गर्भं॒ ॅवि वि गर्भ॒म् गर्भं॒ ॅव्य॑दधा ददधा॒द् वि गर्भ॒म् गर्भं॒ ॅव्य॑दधात् । \newline
3. व्य॑दधा ददधा॒द् वि व्य॑दधात् पुरु॒त्रा पु॑रु॒त्रा ऽद॑धा॒द् वि व्य॑दधात् पुरु॒त्रा । \newline
4. अ॒द॒धा॒त् पु॒रु॒त्रा पु॑रु॒त्रा ऽद॑धा ददधात् पुरु॒त्रा । \newline
5. पु॒रु॒त्रेति॑ पुरु - त्रा । \newline
6. चक्षु॑षः पि॒ता पि॒ता चक्षु॑ष॒ श्चक्षु॑षः पि॒ता मन॑सा॒ मन॑सा पि॒ता चक्षु॑ष॒ श्चक्षु॑षः पि॒ता मन॑सा । \newline
7. पि॒ता मन॑सा॒ मन॑सा पि॒ता पि॒ता मन॑सा॒ हि हि मन॑सा पि॒ता पि॒ता मन॑सा॒ हि । \newline
8. मन॑सा॒ हि हि मन॑सा॒ मन॑सा॒ हि धीरो॒ धीरो॒ हि मन॑सा॒ मन॑सा॒ हि धीरः॑ । \newline
9. हि धीरो॒ धीरो॒ हि हि धीरो॑ घृ॒तम् घृ॒तम् धीरो॒ हि हि धीरो॑ घृ॒तम् । \newline
10. धीरो॑ घृ॒तम् घृ॒तम् धीरो॒ धीरो॑ घृ॒त मे॑ने एने घृ॒तम् धीरो॒ धीरो॑ घृ॒त मे॑ने । \newline
11. घृ॒त मे॑ने एने घृ॒तम् घृ॒त मे॑ने अजन दजन देने घृ॒तम् घृ॒त मे॑ने अजनत् । \newline
12. ए॒ने॒ अ॒ज॒न॒ द॒ज॒न॒ दे॒ने॒ ए॒ने॒ अ॒ज॒न॒न् नन्न॑माने॒ नन्न॑माने अजन देने एने अजन॒न् नन्न॑माने । \newline
13. ए॒ने॒ इत्ये॑ने । \newline
14. अ॒ज॒न॒न् नन्न॑माने॒ नन्न॑माने अजन दजन॒न् नन्न॑माने । \newline
15. नन्न॑माने॒ इति॒ नन्न॑माने । \newline
16. य॒देदिद् य॒दा य॒दे दन्ता॒ अन्ता॒ इद् य॒दा य॒दे दन्ताः᳚ । \newline
17. इदन्ता॒ अन्ता॒ इदिदन्ता॒ अद॑दृꣳह॒न्ता द॑दृꣳह॒न्ता न्ता॒ इदिदन्ता॒ अद॑दृꣳहन्त । \newline
18. अन्ता॒ अद॑दृꣳह॒न्ता द॑दृꣳह॒न्ता न्ता॒ अन्ता॒ अद॑दृꣳहन्त॒ पूर्वे॒ पूर्वे॒ अद॑दृꣳह॒न्ता न्ता॒ अन्ता॒ अद॑दृꣳहन्त॒ पूर्वे᳚ । \newline
19. अद॑दृꣳहन्त॒ पूर्वे॒ पूर्वे॒ अद॑दृꣳह॒न्ता द॑दृꣳहन्त॒ पूर्व॒ आदात् पूर्वे॒ अद॑दृꣳह॒न्ता द॑दृꣳहन्त॒ पूर्व॒ आत् । \newline
20. पूर्व॒ आदात् पूर्वे॒ पूर्व॒ आदिदि दात् पूर्वे॒ पूर्व॒ आदित् । \newline
21. आदिदिदा दादिद् द्यावा॑पृथि॒वी द्यावा॑पृथि॒वी इदादा दिद् द्यावा॑पृथि॒वी । \newline
22. इद् द्यावा॑पृथि॒वी द्यावा॑पृथि॒वी इदिद् द्यावा॑पृथि॒वी अ॑प्रथेता मप्रथेता॒म् द्यावा॑पृथि॒वी इदिद् द्यावा॑पृथि॒वी अ॑प्रथेताम् । \newline
23. द्यावा॑पृथि॒वी अ॑प्रथेता मप्रथेता॒म् द्यावा॑पृथि॒वी द्यावा॑पृथि॒वी अ॑प्रथेताम् । \newline
24. द्यावा॑पृथि॒वी इति॒ द्यावा᳚ - पृ॒थि॒वी । \newline
25. अ॒प्र॒थे॒ता॒मित्य॑प्रथेताम् । \newline
26. वि॒श्वत॑श्चक्षु रु॒तोत वि॒श्वत॑श्चक्षुर् वि॒श्वत॑श्चक्षु रु॒त वि॒श्वतो॑मुखो वि॒श्वतो॑मुख उ॒त वि॒श्वत॑श्चक्षुर् वि॒श्वत॑श्चक्षु रु॒त वि॒श्वतो॑मुखः । \newline
27. वि॒श्वत॑श्चक्षु॒रिति॑ वि॒श्वतः॑ - च॒क्षुः॒ । \newline
28. उ॒त वि॒श्वतो॑मुखो वि॒श्वतो॑मुख उ॒तोत वि॒श्वतो॑मुखो वि॒श्वतो॑हस्तो वि॒श्वतो॑हस्तो वि॒श्वतो॑मुख उ॒तोत वि॒श्वतो॑मुखो वि॒श्वतो॑हस्तः । \newline
29. वि॒श्वतो॑मुखो वि॒श्वतो॑हस्तो वि॒श्वतो॑हस्तो वि॒श्वतो॑मुखो वि॒श्वतो॑मुखो वि॒श्वतो॑हस्त उ॒तोत वि॒श्वतो॑हस्तो वि॒श्वतो॑मुखो वि॒श्वतो॑मुखो वि॒श्वतो॑हस्त उ॒त । \newline
30. वि॒श्वतो॑मुख॒ इति॑ वि॒श्वतः॑ - मु॒खः॒ । \newline
31. वि॒श्वतो॑हस्त उ॒तोत वि॒श्वतो॑हस्तो वि॒श्वतो॑हस्त उ॒त वि॒श्वत॑स्पाद् वि॒श्वत॑स्पा दु॒त वि॒श्वतो॑हस्तो वि॒श्वतो॑हस्त उ॒त वि॒श्वत॑स्पात् । \newline
32. वि॒श्वतो॑हस्त॒ इति॑ वि॒श्वतः॑ - ह॒स्तः॒ । \newline
33. उ॒त वि॒श्वत॑स्पाद् वि॒श्वत॑स्पादु॒तोत वि॒श्वत॑स्पात् । \newline
34. वि॒श्वत॑स्पा॒दिति॑ वि॒श्वतः॑ - पा॒त् । \newline
35. सम् बा॒हुभ्या᳚म् बा॒हुभ्याꣳ॒॒ सꣳ सम् बा॒हुभ्या॒म् नम॑ति॒ नम॑ति बा॒हुभ्याꣳ॒॒ सꣳ सम् बा॒हुभ्या॒म् नम॑ति । \newline
36. बा॒हुभ्या॒म् नम॑ति॒ नम॑ति बा॒हुभ्या᳚म् बा॒हुभ्या॒म् नम॑ति॒ सꣳ सम् नम॑ति बा॒हुभ्या᳚म् बा॒हुभ्या॒म् नम॑ति॒ सम् । \newline
37. बा॒हुभ्या॒मिति॑ बा॒हु - भ्या॒म् । \newline
38. नम॑ति॒ सꣳ सम् नम॑ति॒ नम॑ति॒ सम् पत॑त्रैः॒ पत॑त्रैः॒ सम् नम॑ति॒ नम॑ति॒ सम् पत॑त्रैः । \newline
39. सम् पत॑त्रैः॒ पत॑त्रैः॒ सꣳ सम् पत॑त्रै॒र् द्यावा॑पृथि॒वी द्यावा॑पृथि॒वी पत॑त्रैः॒ सꣳ सम् पत॑त्रै॒र् द्यावा॑पृथि॒वी । \newline
40. पत॑त्रै॒र् द्यावा॑पृथि॒वी द्यावा॑पृथि॒वी पत॑त्रैः॒ पत॑त्रै॒र् द्यावा॑पृथि॒वी ज॒नय॑न् ज॒नय॒न् द्यावा॑पृथि॒वी पत॑त्रैः॒ पत॑त्रै॒र् द्यावा॑पृथि॒वी ज॒नयन्न्॑ । \newline
41. द्यावा॑पृथि॒वी ज॒नय॑न् ज॒नय॒न् द्यावा॑पृथि॒वी द्यावा॑पृथि॒वी ज॒नय॑न् दे॒वो दे॒वो ज॒नय॒न् द्यावा॑पृथि॒वी द्यावा॑पृथि॒वी ज॒नय॑न् दे॒वः । \newline
42. द्यावा॑पृथि॒वी इति॒ द्यावा᳚ - पृ॒थि॒वी । \newline
43. ज॒नय॑न् दे॒वो दे॒वो ज॒नय॑न् ज॒नय॑न् दे॒व एक॒ एको॑ दे॒वो ज॒नय॑न् ज॒नय॑न् दे॒व एकः॑ । \newline
44. दे॒व एक॒ एको॑ दे॒वो दे॒व एकः॑ । \newline
45. एक॒ इत्येकः॑ । \newline
46. किꣳ स्वि॑थ् स्वि॒त् किम् किꣳ स्वि॑दासी दासीथ् स्वि॒त् किम् किꣳ स्वि॑दासीत् । \newline
47. स्वि॒दा॒ सी॒दा॒सी॒थ् स्वि॒थ् स्वि॒दा॒सी॒ द॒धि॒ष्ठान॑ मधि॒ष्ठान॑ मासीथ् स्विथ् स्विदासी दधि॒ष्ठान᳚म् । \newline
48. आ॒सी॒ द॒धि॒ष्ठान॑ मधि॒ष्ठान॑ मासी दासी दधि॒ष्ठान॑ मा॒रंभ॑ण मा॒रंभ॑ण मधि॒ष्ठान॑ मासी दासी दधि॒ष्ठान॑ मा॒रंभ॑णम् । \newline
49. अ॒धि॒ष्ठान॑ मा॒रंभ॑ण मा॒रंभ॑ण मधि॒ष्ठान॑ मधि॒ष्ठान॑ मा॒रंभ॑णम् कत॒मत् क॑त॒म दा॒रंभ॑ण मधि॒ष्ठान॑ मधि॒ष्ठान॑ मा॒रंभ॑णम् कत॒मत् । \newline
50. अ॒धि॒ष्ठान॒मित्य॑धि - स्थान᳚म् । \newline
51. आ॒रंभ॑णम् कत॒मत् क॑त॒म दा॒रंभ॑ण मा॒रंभ॑णम् कत॒मथ् स्वि॑थ् स्वित् कत॒म दा॒रंभ॑ण मा॒रंभ॑णम् कत॒मथ् स्वि॑त् । \newline
52. आ॒रंभ॑ण॒मित्या᳚ - रंभ॑णम् । \newline
53. क॒त॒मथ् स्वि॑थ् स्वित् कत॒मत् क॑त॒मथ् स्वि॒त् किम् किꣳ स्वि॑त् कत॒मत् क॑त॒मथ् स्वि॒त् किम् । \newline
54. स्वि॒त् किम् किꣳ स्वि॑थ् स्वि॒त् कि मा॑सी दासी॒त् किꣳ स्वि॑थ् स्वि॒त् कि मा॑सीत् । \newline
55. कि मा॑सी दासी॒त् किम् कि मा॑सीत् । \newline
56. आ॒सी॒दित्या॑सीत् । \newline
57. यदी॒ भूमि॒म् भूमिं॒ ॅयदि॒ यदी॒ भूमि॑म् ज॒नय॑न् ज॒नय॒न् भूमिं॒ ॅयदि॒ यदी॒ भूमि॑म् ज॒नयन्न्॑ । \newline
58. भूमि॑म् ज॒नय॑न् ज॒नय॒न् भूमि॒म् भूमि॑म् ज॒नय॑न् वि॒श्वक॑र्मा वि॒श्वक॑र्मा ज॒नय॒न् भूमि॒म् भूमि॑म् ज॒नय॑न् वि॒श्वक॑र्मा । \newline
59. ज॒नय॑न् वि॒श्वक॑र्मा वि॒श्वक॑र्मा ज॒नय॑न् ज॒नय॑न् वि॒श्वक॑र्मा॒ वि वि वि॒श्वक॑र्मा ज॒नय॑न् ज॒नय॑न् वि॒श्वक॑र्मा॒ वि । \newline
\pagebreak
\markright{ TS 4.6.2.5  \hfill https://www.vedavms.in \hfill}

\section{ TS 4.6.2.5 }

\textbf{TS 4.6.2.5 } \newline
\textbf{Samhita Paata} \newline

वि॒श्वक॑र्मा॒ वि द्यामौर्णो᳚न् महि॒ना वि॒श्वच॑क्षाः ॥ किꣳ स्वि॒द्वनं॒ क उ॒ स वृ॒क्ष आ॑सी॒द्यतो॒ द्यावा॑पृथि॒वी नि॑ष्टत॒क्षुः । मनी॑षिणो॒ मन॑सा पृ॒च्छतेदु॒ तद्यद॒द्ध्यति॑ष्ठ॒द् भुव॑नानि धा॒रयन्न्॑ ॥ या ते॒ धामा॑नि पर॒माणि॒ याऽव॒मा या म॑द्ध्य॒मा वि॑श्वकर्मन्नु॒तेमा । शिक्षा॒ सखि॑भ्यो ह॒विषि॑ स्वधावः स्व॒यं ॅय॑जस्व त॒नुवं॑ जुषा॒णः ॥ वा॒चस्पतिं॑ ॅवि॒श्वक॑र्माणमू॒तये॑ - [  ] \newline

\textbf{Pada Paata} \newline

वि॒श्वक॒र्मेति॑ वि॒श्व - क॒र्मा॒ । वीति॑ । द्याम् । और्णो᳚त् । म॒हि॒ना । वि॒श्वच॑क्षा॒ इति॑ वि॒श्व - च॒क्षाः॒ ॥ किम् । स्वि॒त् । वन᳚म् । कः । उ॒ । सः । वृ॒क्षः । आ॒सी॒त् । यतः॑ । द्यावा॑पृथि॒वी इति॒ द्यावा᳚ - पृ॒थि॒वी । नि॒ष्ट॒त॒क्षुरिति॑ निः - त॒त॒क्षुः ॥ मनी॑षिणः । मन॑सा । पृ॒च्छत॑ । इत् । उ॒ । तत् । यत् । अ॒द्ध्यति॑ष्ठ॒दित्य॑धि - अति॑ष्ठत् । भुव॑नानि । धा॒रयन्न्॑ ॥ या । ते॒ । धामा॑नि । प॒र॒माणि॑ । या । अ॒व॒मा । या । म॒द्ध्य॒मा । वि॒श्व॒क॒र्म॒न्निति॑ विश्व - क॒र्म॒न्न् । उ॒त । इ॒मा ॥ शिक्ष॑ । सखि॑भ्य॒ इति॒ सखि॑ - भ्यः॒ । ह॒विषि॑ । स्व॒धा॒व॒ इति॑ स्वधा - वः॒ । स्व॒यम् । य॒ज॒स्व॒ । त॒नुव᳚म् । जु॒षा॒णः ॥ वा॒चः । पति᳚म् । वि॒श्वक॑र्माण॒मिति॑ वि॒श्व - क॒र्मा॒ण॒म् । ऊ॒तये᳚ ।  \newline


\textbf{Krama Paata} \newline

वि॒श्वक॑र्मा॒ वि । वि॒श्वक॒र्मेति॑ वि॒श्व - क॒र्मा॒ । वि द्याम् । द्यामौर्णो᳚त् । और्णो᳚न् महि॒ना । म॒हि॒ना वि॒श्वच॑क्षाः । वि॒श्वच॑क्षा॒ इति॑ वि॒श्व - च॒क्षाः॒ ॥ किꣳ स्वि॑त् । स्वि॒द् वन᳚म् । वन॒म् कः । क उ॑ । उ॒ सः । स वृ॒क्षः । वृ॒क्ष आ॑सीत् । आ॒सी॒द् यतः॑ । यतो॒ द्यावा॑पृथि॒वी । द्यावा॑पृथि॒वी नि॑ष्टत॒क्षुः । द्यावा॑पृथि॒वी इति॒ द्यावा᳚ - पृ॒थि॒वी । नि॒ष्ट॒त॒क्षुरिति॑ निः - त॒त॒क्षुः ॥ मनी॑षिणो॒ मन॑सा । मन॑सा पृ॒च्छत॑ । पृ॒च्छतेत् । इदु॑ । उ॒ तत् । तद् यत् । यद॒द्ध्यति॑ष्ठत् । अ॒द्ध्यति॑ष्ठ॒द् भुव॑नानि । अ॒द्ध्यति॑ष्ठ॒दित्य॑द्धि - अति॑ष्ठत् । भुव॑नानि धा॒रयन्न्॑ । धा॒रय॒न्निति॑ धा॒रयन्न्॑ ॥ या ते᳚ । ते॒ धामा॑नि । धामा॑नि पर॒माणि॑ । प॒र॒माणि॒ या । याऽव॒मा । अ॒व॒मा या । या म॑द्ध्य॒मा । म॒द्ध्य॒मा वि॑श्वकर्मन्न् । वि॒श्व॒क॒र्म॒न्नु॒त । वि॒श्व॒क॒र्म॒न्निति॑ विश्व - क॒र्म॒न्न्॒ । उ॒तेमा । इ॒मेती॒मा ॥ शिक्षा॒ सखि॑भ्यः । सखि॑भ्यो ह॒विषि॑ । सखि॑भ्य॒ इति॒ सखि॑ - भ्यः॒ । ह॒विषि॑ स्वधावः । स्व॒धा॒वः॒ स्व॒यम् । स्व॒धा॒व॒ इति॑ स्वधा - वः॒ । स्व॒यं ॅय॑जस्व । य॒ज॒स्व॒ त॒नुव᳚म् । त॒नुव॑म् जुषा॒णः । जु॒षा॒ण इति॑ जुषा॒णः ॥ वा॒चस्पति᳚म् । पतिं॑ ॅवि॒श्वक॑र्माणम् । वि॒श्वक॑र्माणमू॒तये᳚ । वि॒श्वक॑र्माण॒मिति॑ वि॒श्व - क॒र्मा॒ण॒म् । ऊ॒तये॑ मनो॒युज᳚म् \newline

\textbf{Jatai Paata} \newline

1. वि॒श्वक॑र्मा॒ वि वि वि॒श्वक॑र्मा वि॒श्वक॑र्मा॒ वि । \newline
2. वि॒श्वक॒र्मेति॑ वि॒श्व - क॒र्मा॒ । \newline
3. वि द्याम् द्यां ॅवि वि द्याम् । \newline
4. द्या मौर्णो॒ दौर्णो॒द् द्याम् द्या मौर्णो᳚त् । \newline
5. और्णो᳚न् महि॒ना म॑हि॒नौर्णो॒ दौर्णो᳚न् महि॒ना । \newline
6. म॒हि॒ना वि॒श्वच॑क्षा वि॒श्वच॑क्षा महि॒ना म॑हि॒ना वि॒श्वच॑क्षाः । \newline
7. वि॒श्वच॑क्षा॒ इति॑ वि॒श्व - च॒क्षाः॒ । \newline
8. किꣳ स्वि॑थ् स्वि॒त् किम् किꣳ स्वि॑त् । \newline
9. स्वि॒द् वनं॒ ॅवनꣳ॑ स्विथ् स्वि॒द् वन᳚म् । \newline
10. वन॒म् कः को वनं॒ ॅवन॒म् कः । \newline
11. क उ॑ वु॒ कः क उ॑ । \newline
12. उ॒ स स उ॑ वु॒ सः । \newline
13. स वृ॒क्षो वृ॒क्षः स स वृ॒क्षः । \newline
14. वृ॒क्ष आ॑सी दासीद् वृ॒क्षो वृ॒क्ष आ॑सीत् । \newline
15. आ॒सी॒द् यतो॒ यत॑ आसी दासी॒द् यतः॑ । \newline
16. यतो॒ द्यावा॑पृथि॒वी द्यावा॑पृथि॒वी यतो॒ यतो॒ द्यावा॑पृथि॒वी । \newline
17. द्यावा॑पृथि॒वी नि॑ष्टत॒क्षुर् नि॑ष्टत॒क्षुर् द्यावा॑पृथि॒वी द्यावा॑पृथि॒वी नि॑ष्टत॒क्षुः । \newline
18. द्यावा॑पृथि॒वी इति॒ द्यावा᳚ - पृ॒थि॒वी । \newline
19. नि॒ष्ट॒त॒क्षुरिति॑ निः - त॒त॒क्षुः । \newline
20. मनी॑षिणो॒ मन॑सा॒ मन॑सा॒ मनी॑षिणो॒ मनी॑षिणो॒ मन॑सा । \newline
21. मन॑सा पृ॒च्छत॑ पृ॒च्छत॒ मन॑सा॒ मन॑सा पृ॒च्छत॑ । \newline
22. पृ॒च्छते दित् पृ॒च्छत॑ पृ॒च्छतेत् । \newline
23. इदु॑ वु॒ विदिदु॑ । \newline
24. उ॒ तत् तदू॒ तत् । \newline
25. तद् यद् यत् तत् तद् यत् । \newline
26. यद॒द्ध्य ति॑ष्ठ द॒द्ध्यति॑ष्ठ॒द् यद् यद॒द्ध्यति॑ष्ठत् । \newline
27. अ॒द्ध्यति॑ष्ठ॒द् भुव॑नानि॒ भुव॑ना न्य॒द्ध्यति॑ष्ठ द॒द्ध्यति॑ष्ठ॒द् भुव॑नानि । \newline
28. अ॒द्ध्यति॑ष्ठ॒दित्य॑धि - अति॑ष्ठत् । \newline
29. भुव॑नानि धा॒रय॑न् धा॒रय॒न् भुव॑नानि॒ भुव॑नानि धा॒रयन्न्॑ । \newline
30. धा॒रय॒न्निति॑ धा॒रयन्न्॑ । \newline
31. या ते॑ ते॒ या या ते᳚ । \newline
32. ते॒ धामा॑नि॒ धामा॑नि ते ते॒ धामा॑नि । \newline
33. धामा॑नि पर॒माणि॑ पर॒माणि॒ धामा॑नि॒ धामा॑नि पर॒माणि॑ । \newline
34. प॒र॒माणि॒ या या प॑र॒माणि॑ पर॒माणि॒ या । \newline
35. या ऽव॒मा ऽव॒मा या या ऽव॒मा । \newline
36. अ॒व॒मा या या ऽव॒मा ऽव॒मा या । \newline
37. या म॑द्ध्य॒मा म॑द्ध्य॒मा या या म॑द्ध्य॒मा । \newline
38. म॒द्ध्य॒मा वि॑श्वकर्मन्. विश्वकर्मन् मद्ध्य॒मा म॑द्ध्य॒मा वि॑श्वकर्मन्न् । \newline
39. वि॒श्व॒क॒र्म॒न् नु॒तोत वि॑श्वकर्मन्. विश्वकर्मन् नु॒त । \newline
40. वि॒श्व॒क॒र्म॒न्निति॑ विश्व - क॒र्म॒न्न् । \newline
41. उ॒ते मेमोतोते मा । \newline
42. इ॒मेती॒मा । \newline
43. शिक्षा॒ सखि॑भ्यः॒ सखि॑भ्यः॒ शिक्ष॒ शिक्षा॒ सखि॑भ्यः । \newline
44. सखि॑भ्यो ह॒विषि॑ ह॒विषि॒ सखि॑भ्यः॒ सखि॑भ्यो ह॒विषि॑ । \newline
45. सखि॑भ्य॒ इति॒ सखि॑ - भ्यः॒ । \newline
46. ह॒विषि॑ स्वधावः स्वधावो ह॒विषि॑ ह॒विषि॑ स्वधावः । \newline
47. स्व॒धा॒वः॒ स्व॒यꣳ स्व॒यꣳ स्व॑धावः स्वधावः स्व॒यम् । \newline
48. स्व॒धा॒व॒ इति॑ स्वधा - वः॒ । \newline
49. स्व॒यं ॅय॑जस्व यजस्व स्व॒यꣳ स्व॒यं ॅय॑जस्व । \newline
50. य॒ज॒स्व॒ त॒नुव॑म् त॒नुवं॑ ॅयजस्व यजस्व त॒नुव᳚म् । \newline
51. त॒नुव॑म् जुषा॒णो जु॑षा॒णस्त॒नुव॑म् त॒नुव॑म् जुषा॒णः । \newline
52. जु॒षा॒ण इति॑ जुषा॒णः । \newline
53. वा॒च स्पति॒म् पतिं॑ ॅवा॒चो वा॒च स्पति᳚म् । \newline
54. पतिं॑ ॅवि॒श्वक॑र्माणं ॅवि॒श्वक॑र्माण॒म् पति॒म् पतिं॑ ॅवि॒श्वक॑र्माणम् । \newline
55. वि॒श्वक॑र्माण मू॒तय॑ ऊ॒तये॑ वि॒श्वक॑र्माणं ॅवि॒श्वक॑र्माण मू॒तये᳚ । \newline
56. वि॒श्वक॑र्माण॒मिति॑ वि॒श्व - क॒र्मा॒ण॒म् । \newline
57. ऊ॒तये॑ मनो॒युज॑म् मनो॒युज॑ मू॒तय॑ ऊ॒तये॑ मनो॒युज᳚म् । \newline

\textbf{Ghana Paata } \newline

1. वि॒श्वक॑र्मा॒ वि वि वि॒श्वक॑र्मा वि॒श्वक॑र्मा॒ वि द्याम् द्यां ॅवि वि॒श्वक॑र्मा वि॒श्वक॑र्मा॒ वि द्याम् । \newline
2. वि॒श्वक॒र्मेति॑ वि॒श्व - क॒र्मा॒ । \newline
3. वि द्याम् द्यां ॅवि वि द्या मौर्णो॒ दौर्णो॒द् द्यां ॅवि वि द्या मौर्णो᳚त् । \newline
4. द्या मौर्णो॒ दौर्णो॒द् द्याम् द्या मौर्णो᳚न् महि॒ना म॑हि॒ नौर्णो॒द् द्याम् द्या मौर्णो᳚न् महि॒ना । \newline
5. और्णो᳚न् महि॒ना म॑हि॒ नौर्णो॒ दौर्णो᳚न् महि॒ना वि॒श्वच॑क्षा वि॒श्वच॑क्षा महि॒ नौर्णो॒ दौर्णो᳚न् महि॒ना वि॒श्वच॑क्षाः । \newline
6. म॒हि॒ना वि॒श्वच॑क्षा वि॒श्वच॑क्षा महि॒ना म॑हि॒ना वि॒श्वच॑क्षाः । \newline
7. वि॒श्वच॑क्षा॒ इति॑ वि॒श्व - च॒क्षाः॒ । \newline
8. किꣳ स्वि॑थ् स्वि॒त् किम् किꣳ स्वि॒द् वनं॒ ॅवनꣳ॑ स्वि॒त् किम् किꣳ स्वि॒द् वन᳚म् । \newline
9. स्वि॒द् वनं॒ ॅवनꣳ॑ स्विथ् स्वि॒द् वन॒म् कः को वनꣳ॑ स्विथ् स्वि॒द् वन॒म् कः । \newline
10. वन॒म् कः को वनं॒ ॅवन॒म् क उ॑ वु॒ को वनं॒ ॅवन॒म् क उ॑ । \newline
11. क उ॑ वु॒ कः क उ॒ स स उ॒ कः क उ॒ सः । \newline
12. उ॒ स स उ॑ वु॒ स वृ॒क्षो वृ॒क्षः स उ॑ वु॒ स वृ॒क्षः । \newline
13. स वृ॒क्षो वृ॒क्षः स स वृ॒क्ष आ॑सी दासीद् वृ॒क्षः स स वृ॒क्ष आ॑सीत् । \newline
14. वृ॒क्ष आ॑सी दासीद् वृ॒क्षो वृ॒क्ष आ॑सी॒द् यतो॒ यत॑ आसीद् वृ॒क्षो वृ॒क्ष आ॑सी॒द् यतः॑ । \newline
15. आ॒सी॒द् यतो॒ यत॑ आसी दासी॒द् यतो॒ द्यावा॑पृथि॒वी द्यावा॑पृथि॒वी यत॑ आसी दासी॒द् यतो॒ द्यावा॑पृथि॒वी । \newline
16. यतो॒ द्यावा॑पृथि॒वी द्यावा॑पृथि॒वी यतो॒ यतो॒ द्यावा॑पृथि॒वी नि॑ष्टत॒क्षुर् नि॑ष्टत॒क्षुर् द्यावा॑पृथि॒वी यतो॒ यतो॒ द्यावा॑पृथि॒वी नि॑ष्टत॒क्षुः । \newline
17. द्यावा॑पृथि॒वी नि॑ष्टत॒क्षुर् नि॑ष्टत॒क्षुर् द्यावा॑पृथि॒वी द्यावा॑पृथि॒वी नि॑ष्टत॒क्षुः । \newline
18. द्यावा॑पृथि॒वी इति॒ द्यावा᳚ - पृ॒थि॒वी । \newline
19. नि॒ष्ट॒त॒क्षुरिति॑ निः - त॒त॒क्षुः । \newline
20. मनी॑षिणो॒ मन॑सा॒ मन॑सा॒ मनी॑षिणो॒ मनी॑षिणो॒ मन॑सा पृ॒च्छत॑ पृ॒च्छत॒ मन॑सा॒ मनी॑षिणो॒ मनी॑षिणो॒ मन॑सा पृ॒च्छत॑ । \newline
21. मन॑सा पृ॒च्छत॑ पृ॒च्छत॒ मन॑सा॒ मन॑सा पृ॒च्छतेदित् पृ॒च्छत॒ मन॑सा॒ मन॑सा पृ॒च्छतेत् । \newline
22. पृ॒च्छतेदित् पृ॒च्छत॑ पृ॒च्छतेदु॑ वु॒ वित् पृ॒च्छत॑ पृ॒च्छतेदु॑ । \newline
23. इदु॑ वु॒ विदिदु॒ तत् तदु॒ विदिदु॒ तत् । \newline
24. उ॒ तत् तदू॒ तद् यद् यत् तदू॒ तद् यत् । \newline
25. तद् यद् यत् तत् तद् य द॒द्ध्यति॑ष्ठ द॒द्ध्यति॑ष्ठ॒द् यत् तत् तद् यद॒द्ध्यति॑ष्ठत् । \newline
26. य द॒द्ध्यति॑ष्ठ द॒द्ध्यति॑ष्ठ॒द् यद् य द॒द्ध्यति॑ष्ठ॒द् भुव॑नानि॒ भुव॑ना न्य॒द्ध्यति॑ष्ठ॒द् यद् य द॒द्ध्यति॑ष्ठ॒द् भुव॑नानि । \newline
27. अ॒द्ध्यति॑ष्ठ॒द् भुव॑नानि॒ भुव॑ना न्य॒द्ध्यति॑ष्ठ द॒द्ध्यति॑ष्ठ॒द् भुव॑नानि धा॒रय॑न् धा॒रय॒न् भुव॑ना न्य॒द्ध्यति॑ष्ठ द॒द्ध्यति॑ष्ठ॒द् भुव॑नानि धा॒रयन्न्॑ । \newline
28. अ॒द्ध्यति॑ष्ठ॒दित्य॑धि - अति॑ष्ठत् । \newline
29. भुव॑नानि धा॒रय॑न् धा॒रय॒न् भुव॑नानि॒ भुव॑नानि धा॒रयन्न्॑ । \newline
30. धा॒रय॒न्निति॑ धा॒रयन्न्॑ । \newline
31. या ते॑ ते॒ या या ते॒ धामा॑नि॒ धामा॑नि ते॒ या या ते॒ धामा॑नि । \newline
32. ते॒ धामा॑नि॒ धामा॑नि ते ते॒ धामा॑नि पर॒माणि॑ पर॒माणि॒ धामा॑नि ते ते॒ धामा॑नि पर॒माणि॑ । \newline
33. धामा॑नि पर॒माणि॑ पर॒माणि॒ धामा॑नि॒ धामा॑नि पर॒माणि॒ या या प॑र॒माणि॒ धामा॑नि॒ धामा॑नि पर॒माणि॒ या । \newline
34. प॒र॒माणि॒ या या प॑र॒माणि॑ पर॒माणि॒ या ऽव॒मा ऽव॒मा या प॑र॒माणि॑ पर॒माणि॒ या ऽव॒मा । \newline
35. या ऽव॒मा ऽव॒मा या या ऽव॒मा या या ऽव॒मा या या ऽव॒मा या । \newline
36. अ॒व॒मा या या ऽव॒मा ऽव॒मा या म॑द्ध्य॒मा म॑द्ध्य॒मा या ऽव॒मा ऽव॒मा या म॑द्ध्य॒मा । \newline
37. या म॑द्ध्य॒मा म॑द्ध्य॒मा या या म॑द्ध्य॒मा वि॑श्वकर्मन्. विश्वकर्मन् मद्ध्य॒मा या या म॑द्ध्य॒मा वि॑श्वकर्मन्न् । \newline
38. म॒द्ध्य॒मा वि॑श्वकर्मन्. विश्वकर्मन् मद्ध्य॒मा म॑द्ध्य॒मा वि॑श्वकर्मन् नु॒तोत वि॑श्वकर्मन् मद्ध्य॒मा म॑द्ध्य॒मा वि॑श्वकर्मन् नु॒त । \newline
39. वि॒श्व॒क॒र्म॒न् नु॒तोत वि॑श्वकर्मन्. विश्वकर्मन् नु॒ते मेमोत वि॑श्वकर्मन्. विश्वकर्मन् नु॒ते मा । \newline
40. वि॒श्व॒क॒र्म॒न्निति॑ विश्व - क॒र्म॒न्न् । \newline
41. उ॒ते मेमो तोते मा । \newline
42. इ॒मेती॒मा । \newline
43. शिक्षा॒ सखि॑भ्यः॒ सखि॑भ्यः॒ शिक्ष॒ शिक्षा॒ सखि॑भ्यो ह॒विषि॑ ह॒विषि॒ सखि॑भ्यः॒ शिक्ष॒ शिक्षा॒ सखि॑भ्यो ह॒विषि॑ । \newline
44. सखि॑भ्यो ह॒विषि॑ ह॒विषि॒ सखि॑भ्यः॒ सखि॑भ्यो ह॒विषि॑ स्वधावः स्वधावो ह॒विषि॒ सखि॑भ्यः॒ सखि॑भ्यो ह॒विषि॑ स्वधावः । \newline
45. सखि॑भ्य॒ इति॒ सखि॑ - भ्यः॒ । \newline
46. ह॒विषि॑ स्वधावः स्वधावो ह॒विषि॑ ह॒विषि॑ स्वधावः स्व॒यꣳ स्व॒यꣳ स्व॑धावो ह॒विषि॑ ह॒विषि॑ स्वधावः स्व॒यम् । \newline
47. स्व॒धा॒वः॒ स्व॒यꣳ स्व॒यꣳ स्व॑धावः स्वधावः स्व॒यं ॅय॑जस्व यजस्व स्व॒यꣳ स्व॑धावः स्वधावः स्व॒यं ॅय॑जस्व । \newline
48. स्व॒धा॒व॒ इति॑ स्वधा - वः॒ । \newline
49. स्व॒यं ॅय॑जस्व यजस्व स्व॒यꣳ स्व॒यं ॅय॑जस्व त॒नुव॑म् त॒नुवं॑ ॅयजस्व स्व॒यꣳ स्व॒यं ॅय॑जस्व त॒नुव᳚म् । \newline
50. य॒ज॒स्व॒ त॒नुव॑म् त॒नुवं॑ ॅयजस्व यजस्व त॒नुव॑म् जुषा॒णो जु॑षा॒ण स्त॒नुवं॑ ॅयजस्व यजस्व त॒नुव॑म् जुषा॒णः । \newline
51. त॒नुव॑म् जुषा॒णो जु॑षा॒ण स्त॒नुव॑म् त॒नुव॑म् जुषा॒णः । \newline
52. जु॒षा॒ण इति॑ जुषा॒णः । \newline
53. वा॒च स्पति॒म् पतिं॑ ॅवा॒चो वा॒च स्पतिं॑ ॅवि॒श्वक॑र्माणं ॅवि॒श्वक॑र्माण॒म् पतिं॑ ॅवा॒चो 
वा॒च स्पतिं॑ ॅवि॒श्वक॑र्माणम् । \newline
54. पतिं॑ ॅवि॒श्वक॑र्माणं ॅवि॒श्वक॑र्माण॒म् पति॒म् पतिं॑ ॅवि॒श्वक॑र्माण मू॒तय॑ ऊ॒तये॑ वि॒श्वक॑र्माण॒म् पति॒म् पतिं॑ ॅवि॒श्वक॑र्माण मू॒तये᳚ । \newline
55. वि॒श्वक॑र्माण मू॒तय॑ ऊ॒तये॑ वि॒श्वक॑र्माणं ॅवि॒श्वक॑र्माण मू॒तये॑ मनो॒युज॑म् मनो॒युज॑ मू॒तये॑ वि॒श्वक॑र्माणं ॅवि॒श्वक॑र्माण मू॒तये॑ मनो॒युज᳚म् । \newline
56. वि॒श्वक॑र्माण॒मिति॑ वि॒श्व - क॒र्मा॒ण॒म् । \newline
57. ऊ॒तये॑ मनो॒युज॑म् मनो॒युज॑ मू॒तय॑ ऊ॒तये॑ मनो॒युजं॒ ॅवाजे॒ वाजे॑ मनो॒युज॑ मू॒तय॑ ऊ॒तये॑ मनो॒युजं॒ ॅवाजे᳚ । \newline
\pagebreak
\markright{ TS 4.6.2.6  \hfill https://www.vedavms.in \hfill}

\section{ TS 4.6.2.6 }

\textbf{TS 4.6.2.6 } \newline
\textbf{Samhita Paata} \newline

मनो॒युजं॒ ॅवाजे॑ अ॒द्या हु॑वेम । स नो॒ नेदि॑ष्ठा॒ हव॑नानि जोषते वि॒श्वश॑भूं॒रव॑से सा॒धुक॑र्मा ॥ विश्व॑कर्मन्. ह॒विषा॑ वावृधा॒नः स्व॒यं ॅय॑जस्व त॒नुवं॑ जुषा॒णः । मुह्य॑न्त्व॒न्ये अ॒भितः॑ स॒पत्ना॑ इ॒हास्माकं॑ म॒घवा॑ सू॒रिर॑स्तु ॥ विश्व॑कर्मन्. ह॒विषा॒ वर्द्ध॑नेन त्रा॒तार॒मिन्द्र॑-मकृणोरव॒द्ध्यं । तस्मै॒ विशः॒ सम॑नमन्त पू॒र्वीर॒यमु॒ग्रो वि॑ह॒व्यो॑ यथाऽस॑त् ॥ स॒मु॒द्राय॑ व॒युना॑य॒ सिन्धू॑नां॒ पत॑ये॒ नमः॑ ( ) । न॒दीनाꣳ॒॒ सर्वा॑सां पि॒त्रे जु॑हु॒ता वि॒श्वक॑र्मणे॒ विश्वाऽहाम॑र्त्यꣳ ह॒विः ॥ \newline

\textbf{Pada Paata} \newline

म॒नो॒युज॒मिति॑ मनः - युज᳚म् । वाजे᳚ । अ॒द्य । हु॒वे॒म॒ ॥ सः । नः॒ । नेदि॑ष्ठा । हव॑नानि । जो॒ष॒ते॒ । वि॒श्वश॑भूं॒रिति॑ वि॒श्व - श॒भूंः॒ । अव॑से । सा॒धुक॒र्मेति॑ सा॒धु - क॒र्मा॒ ॥ विश्व॑कर्म॒न्निति॒ विश्व॑-क॒र्म॒न्न् । ह॒विषा᳚ । वा॒वृ॒धा॒नः । स्व॒यम् । य॒ज॒स्व॒ । त॒नुव᳚म् । जु॒षा॒णः ॥ मुह्य॑न्तु । अ॒न्ये । अ॒भितः॑ । स॒पत्नाः᳚ । इ॒ह । अ॒स्माक᳚म् । म॒घवेति॑ म॒घ - वा॒ । सू॒रिः । अ॒स्तु॒ ॥ विश्व॑कर्म॒न्निति॒ विश्व॑ - क॒र्म॒न्न् । ह॒विषा᳚ । वद्‌र्ध॑नेन । त्रा॒तार᳚म् । इन्द्र᳚म् । अ॒कृ॒णोः॒ । अ॒व॒द्ध्यम् ॥ तस्मै᳚ । विशः॑ । समिति॑ । अ॒न॒म॒न्त॒ । पू॒र्वीः । अ॒यम् । उ॒ग्रः । वि॒ह॒व्य॑ इति॑ वि-ह॒व्यः॑ । यथा᳚ । अस॑त् ॥ स॒मु॒द्राय॑ । व॒युना॑य । सिन्धू॑नाम् । पत॑ये । नमः॑ ( ) ॥ न॒दीना᳚म् । सर्वा॑साम् । पि॒त्रे । जु॒हु॒त । वि॒श्वक॑र्मण॒ इति॑ वि॒श्व - क॒र्म॒ण॒ । विश्वा᳚ । अहा᳚ । अम॑र्त्यम् । ह॒विः ॥  \newline


\textbf{Krama Paata} \newline

म॒नो॒युजं॒ ॅवाजे᳚ । म॒नो॒युज॒मिति॑ मनः - युज᳚म् । वाजे॑ अ॒द्य । अ॒द्या हु॑वेम । हु॒वे॒मेति॑ हुवेम ॥ स नः॑ । नो॒ नेदि॑ष्ठा । नेदि॑ष्ठा॒ हव॑नानि । हव॑नानि जोषते । जो॒ष॒ते॒ वि॒श्वश॑म्भूः । वि॒श्वश॑म्भू॒रव॑से । वि॒श्वश॑म्भू॒रिति॑ वि॒श्व - श॒म्भूः॒ । अव॑से सा॒धुक॑र्मा । सा॒धुक॒र्मेति॑ सा॒धु - क॒र्मा॒ ॥ विश्व॑कर्मन्. ह॒विषा᳚ । विश्व॑कर्म॒न्निति॒ विश्व॑ - क॒र्म॒न्न्॒ । ह॒विषा॑ वावृधा॒नः । वा॒वृ॒धा॒नः स्व॒यम् । स्व॒यं ॅय॑जस्व । य॒ज॒स्व॒ त॒नुव᳚म् । त॒नुव॑म् जुषा॒णः । जु॒षा॒ण इति॑ जुषा॒णः ॥ मुह्य॑न्त्व॒न्ये । अ॒न्ये अ॒भितः॑ । अ॒भितः॑ स॒पत्नाः᳚ । स॒पत्ना॑ इ॒ह । इ॒हास्माक᳚म् । अ॒स्माक॑म् म॒घवा᳚ । म॒घवा॑ सू॒रिः । म॒घवेति॑ म॒घ - वा॒ । सू॒रिर॑स्तु । अ॒स्त्वित्य॑स्तु ॥ विश्व॑कर्मन्. ह॒विषा᳚ । विश्व॑कर्म॒न्निति॒ विश्व॑ - क॒र्म॒न्न्॒ । ह॒विषा॒ वर्द्ध॑नेन । वर्द्ध॑नेन त्रा॒तार᳚म् । त्रा॒तार॒मिन्द्र᳚म् । इन्द्र॑मकृणोः । अ॒कृ॒णो॒र॒व॒द्ध्यम् । अ॒व॒द्ध्यमित्य॑व॒द्ध्यम् ॥ तस्मै॒ विशः॑ । विशः॒ सम् । सम॑नमन्त । अ॒न॒म॒न्त॒ पू॒र्वीः । पू॒र्वीर॒यम् । अ॒यमु॒ग्रः । उ॒ग्रो वि॑ह॒व्यः॑ । वि॒ह॒व्यो॑ यथा᳚ । वि॒ह॒व्य॑ इति॑ वि - ह॒व्यः॑ । यथाऽस॑त् । अस॒दित्यस॑त् ॥ स॒मु॒द्राय॑ व॒युना॑य । व॒युना॑य॒ सिन्धू॑नाम् । सिन्धू॑ना॒म् पत॑ये । पत॑ये॒ नमः॑ ( ) । नम॒ इति॒ नमः॑ ॥ न॒दीनाꣳ॒॒ सर्वा॑साम् । सर्वा॑साम् पि॒त्रे । पि॒त्रे जु॑हु॒त । जु॒हु॒ता वि॒श्वक॑र्मणे । वि॒श्वक॑र्मणे॒ विश्वा᳚ । वि॒श्वक॑र्मण॒ इति॑ वि॒श्व - क॒र्म॒णे॒ । विश्वाऽहा᳚ । अहाऽम॑र्त्यम् । अम॑र्त्यꣳ ह॒विः । ह॒विरिति॑ ह॒विः । \newline

\textbf{Jatai Paata} \newline

1. म॒नो॒युजं॒ ॅवाजे॒ वाजे॑ मनो॒युज॑म् मनो॒युजं॒ ॅवाजे᳚ । \newline
2. म॒नो॒युज॒मिति॑ मनः - युज᳚म् । \newline
3. वाजे॑ अ॒द्याद्य वाजे॒ वाजे॑ अ॒द्य । \newline
4. अ॒द्या हु॑वेम हुवेमा॒द्याद्या हु॑वेम । \newline
5. हु॒वे॒मेति॑ हुवेम । \newline
6. स नो॑ नः॒ स स नः॑ । \newline
7. नो॒ नेदि॑ष्ठा॒ नेदि॑ष्ठा नो नो॒ नेदि॑ष्ठा । \newline
8. नेदि॑ष्ठा॒ हव॑नानि॒ हव॑नानि॒ नेदि॑ष्ठा॒ नेदि॑ष्ठा॒ हव॑नानि । \newline
9. हव॑नानि जोषते जोषते॒ हव॑नानि॒ हव॑नानि जोषते । \newline
10. जो॒ष॒ते॒ वि॒श्वशं॑भूर् वि॒श्वशं॑भूर् जोषते जोषते वि॒श्वशं॑भूः । \newline
11. वि॒श्वशं॑भू॒ रव॒से ऽव॑से वि॒श्वशं॑भूर् वि॒श्वशं॑भू॒ रव॑से । \newline
12. वि॒श्वशं॑भू॒रिति॑ वि॒श्व - शं॒भूः॒ । \newline
13. अव॑से सा॒धुक॑र्मा सा॒धुक॒र्मा ऽव॒से ऽव॑से सा॒धुक॑र्मा । \newline
14. सा॒धुक॒र्मेति॑ सा॒धु - क॒र्मा॒ । \newline
15. विश्व॑कर्मन्. ह॒विषा॑ ह॒विषा॒ विश्व॑कर्म॒न्॒. विश्व॑कर्मन्. ह॒विषा᳚ । \newline
16. विश्व॑कर्म॒न्निति॒ विश्व॑ - क॒र्म॒न्न् । \newline
17. ह॒विषा॑ वावृधा॒नो वा॑वृधा॒नो ह॒विषा॑ ह॒विषा॑ वावृधा॒नः । \newline
18. वा॒वृ॒धा॒नः स्व॒यꣳ स्व॒यं ॅवा॑वृधा॒नो वा॑वृधा॒नः स्व॒यम् । \newline
19. स्व॒यं ॅय॑जस्व यजस्व स्व॒यꣳ स्व॒यं ॅय॑जस्व । \newline
20. य॒ज॒स्व॒ त॒नुव॑म् त॒नुवं॑ ॅयजस्व यजस्व त॒नुव᳚म् । \newline
21. त॒नुव॑म् जुषा॒णो जु॑षा॒ण स्त॒नुव॑म् त॒नुव॑म् जुषा॒णः । \newline
22. जु॒षा॒ण इति॑ जुषा॒णः । \newline
23. मुह्य॑ न्त्व॒न्ये अ॒न्ये मुह्य॑न्तु॒ मुह्य॑ न्त्व॒न्ये । \newline
24. अ॒न्ये अ॒भितो॑ अ॒भितो॑ अ॒न्ये अ॒न्ये अ॒भितः॑ । \newline
25. अ॒भितः॑ स॒पत्नाः᳚ स॒पत्ना॑ अ॒भितो॑ अ॒भितः॑ स॒पत्नाः᳚ । \newline
26. स॒पत्ना॑ इ॒हेह स॒पत्नाः᳚ स॒पत्ना॑ इ॒ह । \newline
27. इ॒हास्माक॑ म॒स्माक॑ मि॒हे हास्माक᳚म् । \newline
28. अ॒स्माक॑म् म॒घवा॑ म॒घवा॒ ऽस्माक॑ म॒स्माक॑म् म॒घवा᳚ । \newline
29. म॒घवा॑ सू॒रिः सू॒रिर् म॒घवा॑ म॒घवा॑ सू॒रिः । \newline
30. म॒घवेति॑ म॒घ - वा॒ । \newline
31. सू॒रिर॑ स्त्वस्तु सू॒रिः सू॒रि र॑स्तु । \newline
32. अ॒स्त्वित्य॑स्तु । \newline
33. विश्व॑कर्मन्. ह॒विषा॑ ह॒विषा॒ विश्व॑कर्म॒न्॒. विश्व॑कर्मन्. ह॒विषा᳚ । \newline
34. विश्व॑कर्म॒न्निति॒ विश्व॑ - क॒र्म॒न्न् । \newline
35. ह॒विषा॒ वर्द्ध॑नेन॒ वर्द्ध॑नेन ह॒विषा॑ ह॒विषा॒ वर्द्ध॑नेन । \newline
36. वर्द्ध॑नेन त्रा॒तार॑म् त्रा॒तारं॒ ॅवर्द्ध॑नेन॒ वर्द्ध॑नेन त्रा॒तार᳚म् । \newline
37. त्रा॒तार॒ मिन्द्र॒ मिन्द्र॑म् त्रा॒तार॑म् त्रा॒तार॒ मिन्द्र᳚म् । \newline
38. इन्द्र॑ मकृणो रकृणो॒ रिन्द्र॒ मिन्द्र॑ मकृणोः । \newline
39. अ॒कृ॒णो॒ र॒व॒द्ध्य म॑व॒द्ध्य म॑कृणो रकृणो रव॒द्ध्यम् । \newline
40. अ॒व॒द्ध्यमित्य॑व॒ध्यम् । \newline
41. तस्मै॒ विशो॒ विश॒ स्तस्मै॒ तस्मै॒ विशः॑ । \newline
42. विशः॒ सꣳ सं ॅविशो॒ विशः॒ सम् । \newline
43. स म॑नमन्ता नमन्त॒ सꣳ स म॑नमन्त । \newline
44. अ॒न॒म॒न्त॒ पू॒र्वीः पू॒र्वी र॑नमन्ता नमन्त पू॒र्वीः । \newline
45. पू॒र्वी र॒य म॒यम् पू॒र्वीः पू॒र्वी र॒यम् । \newline
46. अ॒य मु॒ग्र उ॒ग्रो॑ ऽय म॒य मु॒ग्रः । \newline
47. उ॒ग्रो वि॑ह॒व्यो॑ विह॒व्य॑ उ॒ग्र उ॒ग्रो वि॑ह॒व्यः॑ । \newline
48. वि॒ह॒व्यो॑ यथा॒ यथा॑ विह॒व्यो॑ विह॒व्यो॑ यथा᳚ । \newline
49. वि॒ह॒व्य॑ इति॑ वि - ह॒व्यः॑ । \newline
50. यथा ऽस॒ दस॒द् यथा॒ यथा ऽस॑त् । \newline
51. अस॒दित्यस॑त् । \newline
52. स॒मु॒द्राय॑ व॒युना॑य व॒युना॑य समु॒द्राय॑ समु॒द्राय॑ व॒युना॑य । \newline
53. व॒युना॑य॒ सिन्धू॑नाꣳ॒॒ सिन्धू॑नां ॅव॒युना॑य व॒युना॑य॒ सिन्धू॑नाम् । \newline
54. सिन्धू॑ना॒म् पत॑ये॒ पत॑ये॒ सिन्धू॑नाꣳ॒॒ सिन्धू॑ना॒म् पत॑ये । \newline
55. पत॑ये॒ नमो॒ नम॒ स्पत॑ये॒ पत॑ये॒ नमः॑ । \newline
56. नम॒ इति॒ नमः॑ । \newline
57. न॒दीनाꣳ॒॒ सर्वा॑साꣳ॒॒ सर्वा॑साम् न॒दीना᳚म् न॒दीनाꣳ॒॒ सर्वा॑साम् । \newline
58. सर्वा॑साम् पि॒त्रे पि॒त्रे सर्वा॑साꣳ॒॒ सर्वा॑साम् पि॒त्रे । \newline
59. पि॒त्रे जु॑हु॒त जु॑हु॒त पि॒त्रे पि॒त्रे जु॑हु॒त । \newline
60. जु॒हु॒ता वि॒श्वक॑र्मणे वि॒श्वक॑र्मणे जुहु॒त जु॑हु॒ता वि॒श्वक॑र्मणे । \newline
61. वि॒श्वक॑र्मणे॒ विश्वा॒ विश्वा॑ वि॒श्वक॑र्मणे वि॒श्वक॑र्मणे॒ विश्वा᳚ । \newline
62. वि॒श्वक॑र्मण॒ इति॑ वि॒श्व - क॒र्म॒णे॒ । \newline
63. विश्वा ऽहा ऽहा॒ विश्वा॒ विश्वा ऽहा᳚ । \newline
64. अहा ऽम॑र्त्य॒ मम॑र्त्य॒ महा ऽहा ऽम॑र्त्यम् । \newline
65. अम॑र्त्यꣳ ह॒विर्. ह॒वि रम॑र्त्य॒ मम॑र्त्यꣳ ह॒विः । \newline
66. ह॒विरिति॑ ह॒विः । \newline

\textbf{Ghana Paata } \newline

1. म॒नो॒युजं॒ ॅवाजे॒ वाजे॑ मनो॒युज॑म् मनो॒युजं॒ ॅवाजे॑ अ॒द्याद्य वाजे॑ मनो॒युज॑म् मनो॒युजं॒ ॅवाजे॑ अ॒द्य । \newline
2. म॒नो॒युज॒मिति॑ मनः - युज᳚म् । \newline
3. वाजे॑ अ॒द्याद्य वाजे॒ वाजे॑ अ॒द्या हु॑वेम हुवेमा॒द्य वाजे॒ वाजे॑ अ॒द्या हु॑वेम । \newline
4. अ॒द्या हु॑वेम हुवेमा॒ द्याद्या हु॑वेम । \newline
5. हु॒वे॒मेति॑ हुवेम । \newline
6. स नो॑ नः॒ स स नो॒ नेदि॑ष्ठा॒ नेदि॑ष्ठा नः॒ स स नो॒ नेदि॑ष्ठा । \newline
7. नो॒ नेदि॑ष्ठा॒ नेदि॑ष्ठा नो नो॒ नेदि॑ष्ठा॒ हव॑नानि॒ हव॑नानि॒ नेदि॑ष्ठा नो नो॒ नेदि॑ष्ठा॒ हव॑नानि । \newline
8. नेदि॑ष्ठा॒ हव॑नानि॒ हव॑नानि॒ नेदि॑ष्ठा॒ नेदि॑ष्ठा॒ हव॑नानि जोषते जोषते॒ हव॑नानि॒ नेदि॑ष्ठा॒ नेदि॑ष्ठा॒ हव॑नानि जोषते । \newline
9. हव॑नानि जोषते जोषते॒ हव॑नानि॒ हव॑नानि जोषते वि॒श्वशं॑भूर् वि॒श्वशं॑भूर् जोषते॒ हव॑नानि॒ हव॑नानि जोषते वि॒श्वशं॑भूः । \newline
10. जो॒ष॒ते॒ वि॒श्वशं॑भूर् वि॒श्वशं॑भूर् जोषते जोषते वि॒श्वशं॑भू॒ रव॒से ऽव॑से वि॒श्वशं॑भूर् जोषते जोषते वि॒श्वशं॑भू॒ रव॑से । \newline
11. वि॒श्वशं॑भू॒ रव॒से ऽव॑से वि॒श्वशं॑भूर् वि॒श्वशं॑भू॒ रव॑से सा॒धुक॑र्मा सा॒धुक॒र्मा ऽव॑से वि॒श्वशं॑भूर् वि॒श्वशं॑भू॒ रव॑से सा॒धुक॑र्मा । \newline
12. वि॒श्वशं॑भू॒रिति॑ वि॒श्व - शं॒भूः॒ । \newline
13. अव॑से सा॒धुक॑र्मा सा॒धुक॒र्मा ऽव॒से ऽव॑से सा॒धुक॑र्मा । \newline
14. सा॒धुक॒र्मेति॑ सा॒धु - क॒र्मा॒ । \newline
15. विश्व॑कर्मन्. ह॒विषा॑ ह॒विषा॒ विश्व॑कर्म॒न्॒. विश्व॑कर्मन्. ह॒विषा॑ वावृधा॒नो वा॑वृधा॒नो ह॒विषा॒ विश्व॑कर्म॒न्॒. विश्व॑कर्मन्. ह॒विषा॑ वावृधा॒नः । \newline
16. विश्व॑कर्म॒न्निति॒ विश्व॑ - क॒र्म॒न्न् । \newline
17. ह॒विषा॑ वावृधा॒नो वा॑वृधा॒नो ह॒विषा॑ ह॒विषा॑ वावृधा॒नः स्व॒यꣳ स्व॒यं ॅवा॑वृधा॒नो ह॒विषा॑ ह॒विषा॑ वावृधा॒नः स्व॒यम् । \newline
18. वा॒वृ॒धा॒नः स्व॒यꣳ स्व॒यं ॅवा॑वृधा॒नो वा॑वृधा॒नः स्व॒यं ॅय॑जस्व यजस्व स्व॒यं ॅवा॑वृधा॒नो वा॑वृधा॒नः स्व॒यं ॅय॑जस्व । \newline
19. स्व॒यं ॅय॑जस्व यजस्व स्व॒यꣳ स्व॒यं ॅय॑जस्व त॒नुव॑म् त॒नुवं॑ ॅयजस्व स्व॒यꣳ स्व॒यं ॅय॑जस्व त॒नुव᳚म् । \newline
20. य॒ज॒स्व॒ त॒नुव॑म् त॒नुवं॑ ॅयजस्व यजस्व त॒नुव॑म् जुषा॒णो जु॑षा॒ण स्त॒नुवं॑ ॅयजस्व यजस्व त॒नुव॑म् जुषा॒णः । \newline
21. त॒नुव॑म् जुषा॒णो जु॑षा॒ण स्त॒नुव॑म् त॒नुव॑म् जुषा॒णः । \newline
22. जु॒षा॒ण इति॑ जुषा॒णः । \newline
23. मुह्य॑ न्त्व॒न्ये अ॒न्ये मुह्य॑न्तु॒ मुह्य॑न्त्व॒न्ये अ॒भितो॑ अ॒भितो॑ अ॒न्ये मुह्य॑न्तु॒ मुह्य॑न्त्व॒न्ये अ॒भितः॑ । \newline
24. अ॒न्ये अ॒भितो॑ अ॒भितो॑ अ॒न्ये अ॒न्ये अ॒भितः॑ स॒पत्नाः᳚ स॒पत्ना॑ अ॒भितो॑ अ॒न्ये अ॒न्ये अ॒भितः॑ स॒पत्नाः᳚ । \newline
25. अ॒भितः॑ स॒पत्नाः᳚ स॒पत्ना॑ अ॒भितो॑ अ॒भितः॑ स॒पत्ना॑ इ॒हेह स॒पत्ना॑ अ॒भितो॑ अ॒भितः॑ स॒पत्ना॑ इ॒ह । \newline
26. स॒पत्ना॑ इ॒हेह स॒पत्नाः᳚ स॒पत्ना॑ इ॒हा स्माक॑ म॒स्माक॑ मि॒ह स॒पत्नाः᳚ स॒पत्ना॑ इ॒हा स्माक᳚म् । \newline
27. इ॒हास्माक॑ म॒स्माक॑ मि॒हेहा स्माक॑म् म॒घवा॑ म॒घवा॒ ऽस्माक॑ मि॒हेहा स्माक॑म् म॒घवा᳚ । \newline
28. अ॒स्माक॑म् म॒घवा॑ म॒घवा॒ ऽस्माक॑ म॒स्माक॑म् म॒घवा॑ सू॒रिः सू॒रिर् म॒घवा॒ ऽस्माक॑ म॒स्माक॑म् म॒घवा॑ सू॒रिः । \newline
29. म॒घवा॑ सू॒रिः सू॒रिर् म॒घवा॑ म॒घवा॑ सू॒रि र॑स्त्वस्तु सू॒रिर् म॒घवा॑ म॒घवा॑ सू॒रि र॑स्तु । \newline
30. म॒घवेति॑ म॒घ - वा॒ । \newline
31. सू॒रि र॑स्त्वस्तु सू॒रिः सू॒रि र॑स्तु । \newline
32. अ॒स्त्वित्य॑स्तु । \newline
33. विश्व॑कर्मन्. ह॒विषा॑ ह॒विषा॒ विश्व॑कर्म॒न्॒. विश्व॑कर्मन्. ह॒विषा॒ वर्द्ध॑नेन॒ वर्द्ध॑नेन ह॒विषा॒ विश्व॑कर्म॒न्॒. विश्व॑कर्मन्. ह॒विषा॒ वर्द्ध॑नेन । \newline
34. विश्व॑कर्म॒न्निति॒ विश्व॑ - क॒र्म॒न्न् । \newline
35. ह॒विषा॒ वर्द्ध॑नेन॒ वर्द्ध॑नेन ह॒विषा॑ ह॒विषा॒ वर्द्ध॑नेन त्रा॒तार॑म् त्रा॒तारं॒ ॅवर्द्ध॑नेन ह॒विषा॑ ह॒विषा॒ वर्द्ध॑नेन त्रा॒तार᳚म् । \newline
36. वर्द्ध॑नेन त्रा॒तार॑म् त्रा॒तारं॒ ॅवर्द्ध॑नेन॒ वर्द्ध॑नेन त्रा॒तार॒ मिन्द्र॒ मिन्द्र॑म् त्रा॒तारं॒ ॅवर्द्ध॑नेन॒ वर्द्ध॑नेन त्रा॒तार॒ मिन्द्र᳚म् । \newline
37. त्रा॒तार॒ मिन्द्र॒ मिन्द्र॑म् त्रा॒तार॑म् त्रा॒तार॒ मिन्द्र॑ मकृणो रकृणो॒ रिन्द्र॑म् त्रा॒तार॑म् त्रा॒तार॒ मिन्द्र॑ मकृणोः । \newline
38. इन्द्र॑ मकृणो रकृणो॒ रिन्द्र॒ मिन्द्र॑ मकृणो रव॒द्ध्य म॑व॒द्ध्य म॑कृणो॒ रिन्द्र॒ मिन्द्र॑ मकृणो रव॒द्ध्यम् । \newline
39. अ॒कृ॒णो॒ र॒व॒द्ध्य म॑व॒द्ध्य म॑कृणो रकृणो रव॒द्ध्यम् । \newline
40. अ॒व॒द्ध्यमित्य॑व॒ध्यम् । \newline
41. तस्मै॒ विशो॒ विश॒ स्तस्मै॒ तस्मै॒ विशः॒ सꣳ सं ॅविश॒ स्तस्मै॒ तस्मै॒ विशः॒ सम् । \newline
42. विशः॒ सꣳ सं ॅविशो॒ विशः॒ स म॑नमन्ता नमन्त॒ सं ॅविशो॒ विशः॒ स म॑नमन्त । \newline
43. स म॑नमन्ता नमन्त॒ सꣳ स म॑नमन्त पू॒र्वीः पू॒र्वी र॑नमन्त॒ सꣳ स म॑नमन्त पू॒र्वीः । \newline
44. अ॒न॒म॒न्त॒ पू॒र्वीः पू॒र्वी र॑नमन्ता नमन्त पू॒र्वी र॒य म॒यम् पू॒र्वी र॑नमन्ता नमन्त पू॒र्वीर॒यम् । \newline
45. पू॒र्वी र॒य म॒यम् पू॒र्वीः पू॒र्वीर॒य मु॒ग्र उ॒ग्रो॑ ऽयम् पू॒र्वीः पू॒र्वीर॒य मु॒ग्रः । \newline
46. अ॒य मु॒ग्र उ॒ग्रो॑ ऽय म॒य मु॒ग्रो वि॑ह॒व्यो॑ विह॒व्य॑ उ॒ग्रो॑ ऽय म॒य मु॒ग्रो वि॑ह॒व्यः॑ । \newline
47. उ॒ग्रो वि॑ह॒व्यो॑ विह॒व्य॑ उ॒ग्र उ॒ग्रो वि॑ह॒व्यो॑ यथा॒ यथा॑ विह॒व्य॑ उ॒ग्र उ॒ग्रो वि॑ह॒व्यो॑ यथा᳚ । \newline
48. वि॒ह॒व्यो॑ यथा॒ यथा॑ विह॒व्यो॑ विह॒व्यो॑ यथा ऽस॒ दस॒द् यथा॑ विह॒व्यो॑ विह॒व्यो॑ यथा ऽस॑त् । \newline
49. वि॒ह॒व्य॑ इति॑ वि - ह॒व्यः॑ । \newline
50. यथा ऽस॒ दस॒द् यथा॒ यथा ऽस॑त् । \newline
51. अस॒दित्यस॑त् । \newline
52. स॒मु॒द्राय॑ व॒युना॑य व॒युना॑य समु॒द्राय॑ समु॒द्राय॑ व॒युना॑य॒ सिन्धू॑नाꣳ॒॒ सिन्धू॑नां ॅव॒युना॑य समु॒द्राय॑ समु॒द्राय॑ व॒युना॑य॒ सिन्धू॑नाम् । \newline
53. व॒युना॑य॒ सिन्धू॑नाꣳ॒॒ सिन्धू॑नां ॅव॒युना॑य व॒युना॑य॒ सिन्धू॑ना॒म् पत॑ये॒ पत॑ये॒ सिन्धू॑नां ॅव॒युना॑य व॒युना॑य॒ सिन्धू॑ना॒म् पत॑ये । \newline
54. सिन्धू॑ना॒म् पत॑ये॒ पत॑ये॒ सिन्धू॑नाꣳ॒॒ सिन्धू॑ना॒म् पत॑ये॒ नमो॒ नम॒ स्पत॑ये॒ सिन्धू॑नाꣳ॒॒ सिन्धू॑ना॒म् पत॑ये॒ नमः॑ । \newline
55. पत॑ये॒ नमो॒ नम॒ स्पत॑ये॒ पत॑ये॒ नमः॑ । \newline
56. नम॒ इति॒ नमः॑ । \newline
57. न॒दीनाꣳ॒॒ सर्वा॑साꣳ॒॒ सर्वा॑साम् न॒दीना᳚म् न॒दीनाꣳ॒॒ सर्वा॑साम् पि॒त्रे पि॒त्रे सर्वा॑साम् न॒दीना᳚म् न॒दीनाꣳ॒॒ सर्वा॑साम् पि॒त्रे । \newline
58. सर्वा॑साम् पि॒त्रे पि॒त्रे सर्वा॑साꣳ॒॒ सर्वा॑साम् पि॒त्रे जु॑हु॒त जु॑हु॒त पि॒त्रे सर्वा॑साꣳ॒॒ सर्वा॑साम् पि॒त्रे जु॑हु॒त । \newline
59. पि॒त्रे जु॑हु॒त जु॑हु॒त पि॒त्रे पि॒त्रे जु॑हु॒ता वि॒श्वक॑र्मणे वि॒श्वक॑र्मणे जुहु॒त पि॒त्रे पि॒त्रे जु॑हु॒ता वि॒श्वक॑र्मणे । \newline
60. जु॒हु॒ता वि॒श्वक॑र्मणे वि॒श्वक॑र्मणे जुहु॒त जु॑हु॒ता वि॒श्वक॑र्मणे॒ विश्वा॒ विश्वा॑ वि॒श्वक॑र्मणे जुहु॒त जु॑हु॒ता वि॒श्वक॑र्मणे॒ विश्वा᳚ । \newline
61. वि॒श्वक॑र्मणे॒ विश्वा॒ विश्वा॑ वि॒श्वक॑र्मणे वि॒श्वक॑र्मणे॒ विश्वा ऽहा ऽहा॒ विश्वा॑ वि॒श्वक॑र्मणे वि॒श्वक॑र्मणे॒ विश्वा ऽहा᳚ । \newline
62. वि॒श्वक॑र्मण॒ इति॑ वि॒श्व - क॒र्म॒णे॒ । \newline
63. विश्वा ऽहा ऽहा॒ विश्वा॒ विश्वा ऽहा ऽम॑र्त्य॒ मम॑र्त्य॒ महा॒ विश्वा॒ विश्वा ऽहा ऽम॑र्त्यम् । \newline
64. अहा ऽम॑र्त्य॒ मम॑र्त्य॒ महा ऽहा ऽम॑र्त्यꣳ ह॒विर्. ह॒वि रम॑र्त्य॒ महा ऽहा ऽम॑र्त्यꣳ ह॒विः । \newline
65. अम॑र्त्यꣳ ह॒विर्. ह॒वि रम॑र्त्य॒ मम॑र्त्यꣳ ह॒विः । \newline
66. ह॒विरिति॑ ह॒विः । \newline
\pagebreak
\markright{ TS 4.6.3.1  \hfill https://www.vedavms.in \hfill}

\section{ TS 4.6.3.1 }

\textbf{TS 4.6.3.1 } \newline
\textbf{Samhita Paata} \newline

उदे॑नमुत्त॒रां न॒याग्ने॑ घृतेनाऽऽहुत । रा॒यस्पोषे॑ण॒ सꣳ सृ॑ज प्र॒जया॑ च॒ धने॑न च ॥ इन्द्रे॒मं प्र॑त॒रां कृ॑धि सजा॒ताना॑-मसद्व॒शी । समे॑नं॒ ॅवर्च॑सा सृज दे॒वेभ्यो॑ भाग॒धा अ॑सत् ॥ यस्य॑ कु॒र्मो ह॒विर्गृ॒हे तम॑ग्ने वर्द्धया॒ त्वं । तस्म॑ दे॒वा अधि॑ ब्रवन्न॒यं च॒ ब्रह्म॑ण॒स्पतिः॑ ॥ उदु॑ त्वा॒ विश्वे॑ दे॒वा - [  ] \newline

\textbf{Pada Paata} \newline

उदिति॑ । ए॒न॒म् । उ॒त्त॒रामित्यु॑त् - त॒राम् । न॒य॒ । अग्ने᳚ । घृ॒ते॒न॒ । आ॒हु॒तेत्या᳚ - हु॒त॒ ॥ रा॒यः । पोषे॑ण । समिति॑ । सृ॒ज॒ । प्र॒जयेति॑ प्र - जया᳚ । च॒ । धने॑न । च॒ ॥ इन्द्र॑ । इ॒मम् । प्र॒त॒रामिति॑ प्र-त॒राम् । कृ॒धि॒ । स॒जा॒ताना॒मिति॑ स - जा॒ताना᳚म् । अ॒स॒त् । व॒शी ॥ समिति॑ । ए॒न॒म् । वर्च॑सा । सृ॒ज॒ । दे॒वेभ्यः॑ । भा॒ग॒धा इति॑ भाग - धाः । अ॒स॒त् ॥ यस्य॑ । कु॒र्मः । ह॒विः । गृ॒हे । तम् । अ॒ग्ने॒ । व॒द्‌र्ध॒य॒ । त्वम् ॥ तस्मै᳚ । दे॒वाः । अधीति॑ । ब्र॒व॒न्न् । अ॒यम् । च॒ । ब्रह्म॑णः । पतिः॑ ॥ उदिति॑ । उ॒ । त्वा॒ । विश्वे᳚ । दे॒वाः ।  \newline


\textbf{Krama Paata} \newline

उदे॑नम् । ए॒न॒मु॒त्त॒राम् । उ॒त्त॒राम् न॑य । उ॒त्त॒रामित्यु॑त् - त॒राम् । न॒याग्ने᳚ । अग्ने॑ घृतेन । घृ॒ते॒ना॒हु॒त॒ । आ॒हु॒तेत्या᳚ - हु॒त॒ ॥ रा॒यस्पोषे॑ण । पोषे॑ण॒ सम् । सꣳ सृ॑ज । सृ॒ज॒ प्र॒जया᳚ । प्र॒जया॑ च । प्र॒जयेति॑ प्र - जया᳚ । च॒ धने॑न । धने॑न च । चेति॑ च ॥ इन्द्रे॒मम् । इ॒मम् प्र॑त॒राम् । प्र॒त॒राम् कृ॑धि । प्र॒त॒रामिति॑ प्र - त॒राम् । कृ॒धि॒ स॒जा॒ताना᳚म् । स॒जा॒ताना॑मसत् । स॒जा॒ताना॒मिति॑ स - जा॒ताना᳚म् । अ॒स॒द् व॒शी । व॒शीति॑ व॒शी ॥ समे॑नम् । ए॒नं॒ ॅवर्च॑सा । वर्च॑सा सृज । सृ॒ज॒ दे॒वेभ्यः॑ । दे॒वेभ्यो॑ भाग॒धाः । भा॒ग॒धा अ॑सत् । भा॒ग॒धा इति॑ भाग - धाः । अ॒स॒दित्य॑सत् ॥ यस्य॑ कु॒र्मः । कु॒र्मो ह॒विः । ह॒विर् गृ॒हे । गृ॒हे तम् । तम॑ग्ने । अ॒ग्ने॒ व॒र्द्ध॒य॒ । व॒र्द्ध॒या॒ त्वम् । त्वमिति॒ त्वम् ॥ तस्मै॑ दे॒वाः । दे॒वा अधि॑ । अधि॑ ब्रवन्न् । ब्र॒व॒न्न॒यम् । अ॒यम् च॑ । च॒ ब्रह्म॑णः । ब्रह्म॑ण॒स्पतिः॑ । पति॒रिति॒ पतिः॑ ॥ उदु॑ । उ॒ त्वा॒ । त्वा॒ विश्वे᳚ । विश्वे॑ दे॒वाः । दे॒वा अग्ने᳚ \newline

\textbf{Jatai Paata} \newline

1. उदे॑न मेन॒ मुदुदे॑नम् । \newline
2. ए॒न॒ मु॒त्त॒रा मु॑त्त॒रा मे॑न मेन मुत्त॒राम् । \newline
3. उ॒त्त॒राम् न॑य नयोत्त॒रा मु॑त्त॒राम् न॑य । \newline
4. उ॒त्त॒रामित्यु॑त् - त॒राम् । \newline
5. न॒याग्ने ऽग्ने॑ नय न॒याग्ने᳚ । \newline
6. अग्ने॑ घृतेन घृते॒नाग्ने ऽग्ने॑ घृतेन । \newline
7. घृ॒ते॒ना॒ हु॒ता॒ हु॒त॒ घृ॒ते॒न॒ घृ॒ते॒ना॒ हु॒त॒ । \newline
8. आ॒हु॒तेत्या᳚ - हु॒त॒ । \newline
9. रा॒य स्पोषे॑ण॒ पोषे॑ण रा॒यो रा॒य स्पोषे॑ण । \newline
10. पोषे॑ण॒ सꣳ सम् पोषे॑ण॒ पोषे॑ण॒ सम् । \newline
11. सꣳ सृ॑ज सृज॒ सꣳ सꣳ सृ॑ज । \newline
12. सृ॒ज॒ प्र॒जया᳚ प्र॒जया॑ सृज सृज प्र॒जया᳚ । \newline
13. प्र॒जया॑ च च प्र॒जया᳚ प्र॒जया॑ च । \newline
14. प्र॒जयेति॑ प्र - जया᳚ । \newline
15. च॒ धने॑न॒ धने॑न च च॒ धने॑न । \newline
16. धने॑न च च॒ धने॑न॒ धने॑न च । \newline
17. चेति॑ च । \newline
18. इन्द्रे॒ ममि॒म मिन्द्रे न्द्रे॒मम् । \newline
19. इ॒मम् प्र॑त॒राम् प्र॑त॒रा मि॒म मि॒मम् प्र॑त॒राम् । \newline
20. प्र॒त॒राम् कृ॑धि कृधि प्रत॒राम् प्र॑त॒राम् कृ॑धि । \newline
21. प्र॒त॒रामिति॑ प्र - त॒राम् । \newline
22. कृ॒धि॒ स॒जा॒तानाꣳ॑ सजा॒ताना᳚म् कृधि कृधि सजा॒ताना᳚म् । \newline
23. स॒जा॒ताना॑ मस दसथ् सजा॒तानाꣳ॑ सजा॒ताना॑ मसत् । \newline
24. स॒जा॒ताना॒मिति॑ स - जा॒ताना᳚म् । \newline
25. अ॒स॒द् व॒शी व॒श्य॑स दसद् व॒शी । \newline
26. व॒शीति॑ व॒शी । \newline
27. स मे॑न मेनꣳ॒॒ सꣳ स मे॑नम् । \newline
28. ए॒नं॒ ॅवर्च॑सा॒ वर्च॑सैन मेनं॒ ॅवर्च॑सा । \newline
29. वर्च॑सा सृज सृज॒ वर्च॑सा॒ वर्च॑सा सृज । \newline
30. सृ॒ज॒ दे॒वेभ्यो॑ दे॒वेभ्यः॑ सृज सृज दे॒वेभ्यः॑ । \newline
31. दे॒वेभ्यो॑ भाग॒धा भा॑ग॒धा दे॒वेभ्यो॑ दे॒वेभ्यो॑ भाग॒धाः । \newline
32. भा॒ग॒धा अ॑स दसद् भाग॒धा भा॑ग॒धा अ॑सत् । \newline
33. भा॒ग॒धा इति॑ भाग - धाः । \newline
34. अ॒स॒दित्य॑सत् । \newline
35. यस्य॑ कु॒र्मः कु॒र्मो यस्य॒ यस्य॑ कु॒र्मः । \newline
36. कु॒र्मो ह॒विर्. ह॒विः कु॒र्मः कु॒र्मो ह॒विः । \newline
37. ह॒विर् गृ॒हे गृ॒हे ह॒विर्. ह॒विर् गृ॒हे । \newline
38. गृ॒हे तम् तम् गृ॒हे गृ॒हे तम् । \newline
39. त म॑ग्ने अग्ने॒ तम् त म॑ग्ने । \newline
40. अ॒ग्ने॒ व॒र्द्ध॒य॒ व॒र्द्ध॒या॒ग्ने॒ अ॒ग्ने॒ व॒र्द्ध॒य॒ । \newline
41. व॒र्द्ध॒या॒ त्वम् त्वं ॅव॑र्द्धय वर्द्धया॒ त्वम् । \newline
42. त्वमिति॒ त्वम् । \newline
43. तस्मै॑ दे॒वा दे॒वा स्तस्मै॒ तस्मै॑ दे॒वाः । \newline
44. दे॒वा अध्यधि॑ दे॒वा दे॒वा अधि॑ । \newline
45. अधि॑ ब्रवन् ब्रव॒न् नध्यधि॑ ब्रवन्न् । \newline
46. ब्र॒व॒न् न॒य म॒यम् ब्र॑वन् ब्रवन् न॒यम् । \newline
47. अ॒यम् च॑ चा॒य म॒यम् च॑ । \newline
48. च॒ ब्रह्म॑णो॒ ब्रह्म॑णश्च च॒ ब्रह्म॑णः । \newline
49. ब्रह्म॑ण॒ स्पति॒ष् पति॒र् ब्रह्म॑णो॒ ब्रह्म॑ण॒ स्पतिः॑ । \newline
50. पति॒रिति॒ पतिः॑ । \newline
51. उदु॑ वु॒ वुदुदु॑ । \newline
52. उ॒ त्वा॒ त्व॒ वु॒ त्वा॒ । \newline
53. त्वा॒ विश्वे॒ विश्वे᳚ त्वा त्वा॒ विश्वे᳚ । \newline
54. विश्वे॑ दे॒वा दे॒वा विश्वे॒ विश्वे॑ दे॒वाः । \newline
55. दे॒वा अग्ने ऽग्ने॑ दे॒वा दे॒वा अग्ने᳚ । \newline

\textbf{Ghana Paata } \newline

1. उदे॑न मेन॒ मुदुदे॑न मुत्त॒रा मु॑त्त॒रा मे॑न॒ मुदुदे॑न मुत्त॒राम् । \newline
2. ए॒न॒ मु॒त्त॒रा मु॑त्त॒रा मे॑न मेन मुत्त॒राम् न॑य नयोत्त॒रा मे॑न मेन मुत्त॒राम् न॑य । \newline
3. उ॒त्त॒राम् न॑य नयोत्त॒रा मु॑त्त॒राम् न॒याग्ने ऽग्ने॑ नयोत्त॒रा मु॑त्त॒राम् न॒याग्ने᳚ । \newline
4. उ॒त्त॒रामित्यु॑त् - त॒राम् । \newline
5. न॒याग्ने ऽग्ने॑ नय न॒याग्ने॑ घृतेन घृते॒नाग्ने॑ नय न॒याग्ने॑ घृतेन । \newline
6. अग्ने॑ घृतेन घृते॒नाग्ने ऽग्ने॑ घ्रुते नाहुता हुत घृते॒नाग्ने ऽग्ने॑ घृतेनाहुत । \newline
7. घृ॒ते॒ ना॒हु॒ता॒ हु॒त॒ घृ॒ते॒न॒ घृ॒ते॒ ना॒हु॒त॒ । \newline
8. आ॒हु॒तेत्या᳚ - हु॒त॒ । \newline
9. रा॒य स्पोषे॑ण॒ पोषे॑ण रा॒यो रा॒य स्पोषे॑ण॒ सꣳ सम् पोषे॑ण रा॒यो रा॒य स्पोषे॑ण॒ सम् । \newline
10. पोषे॑ण॒ सꣳ सम् पोषे॑ण॒ पोषे॑ण॒ सꣳ सृ॑ज सृज॒ सम् पोषे॑ण॒ पोषे॑ण॒ सꣳ सृ॑ज । \newline
11. सꣳ सृ॑ज सृज॒ सꣳ सꣳ सृ॑ज प्र॒जया᳚ प्र॒जया॑ सृज॒ सꣳ सꣳ सृ॑ज प्र॒जया᳚ । \newline
12. सृ॒ज॒ प्र॒जया᳚ प्र॒जया॑ सृज सृज प्र॒जया॑ च च प्र॒जया॑ सृज सृज प्र॒जया॑ च । \newline
13. प्र॒जया॑ च च प्र॒जया᳚ प्र॒जया॑ च॒ धने॑न॒ धने॑न च प्र॒जया᳚ प्र॒जया॑ च॒ धने॑न । \newline
14. प्र॒जयेति॑ प्र - जया᳚ । \newline
15. च॒ धने॑न॒ धने॑न च च॒ धने॑न च च॒ धने॑न च च॒ धने॑न च । \newline
16. धने॑न च च॒ धने॑न॒ धने॑न च । \newline
17. चेति॑ च । \newline
18. इन्द्रे॒म मि॒म मिन्द्रेन्द्रे॒ मम् प्र॑त॒राम् प्र॑त॒रा मि॒म मिन्द्रेन्द्रे॒ मम् प्र॑त॒राम् । \newline
19. इ॒मम् प्र॑त॒राम् प्र॑त॒रा मि॒म मि॒मम् प्र॑त॒राम् कृ॑धि कृधि प्रत॒रा मि॒म मि॒मम् प्र॑त॒राम् कृ॑धि । \newline
20. प्र॒त॒राम् कृ॑धि कृधि प्रत॒राम् प्र॑त॒राम् कृ॑धि सजा॒तानाꣳ॑ सजा॒ताना᳚म् कृधि प्रत॒राम् प्र॑त॒राम् कृ॑धि सजा॒ताना᳚म् । \newline
21. प्र॒त॒रामिति॑ प्र - त॒राम् । \newline
22. कृ॒धि॒ स॒जा॒तानाꣳ॑ सजा॒ताना᳚म् कृधि कृधि सजा॒ताना॑ मस दसथ् सजा॒ताना᳚म् कृधि कृधि सजा॒ताना॑ मसत् । \newline
23. स॒जा॒ताना॑ मस दसथ् सजा॒तानाꣳ॑ सजा॒ताना॑ मसद् व॒शी व॒श्य॑सथ् सजा॒तानाꣳ॑ सजा॒ताना॑ मसद् व॒शी । \newline
24. स॒जा॒ताना॒मिति॑ स - जा॒ताना᳚म् । \newline
25. अ॒स॒द् व॒शी व॒श्य॑ सदसद् व॒शी । \newline
26. व॒शीति॑ व॒शी । \newline
27. स मे॑न मेनꣳ॒॒ सꣳ स मे॑नं॒ ॅवर्च॑सा॒ वर्च॑सैनꣳ॒॒ सꣳ स मे॑नं॒ ॅवर्च॑सा । \newline
28. ए॒नं॒ ॅवर्च॑सा॒ वर्च॑सैन मेनं॒ ॅवर्च॑सा सृज सृज॒ वर्च॑ सैन मेनं॒ ॅवर्च॑सा सृज । \newline
29. वर्च॑सा सृज सृज॒ वर्च॑सा॒ वर्च॑सा सृज दे॒वेभ्यो॑ दे॒वेभ्यः॑ सृज॒ वर्च॑सा॒ वर्च॑सा सृज दे॒वेभ्यः॑ । \newline
30. सृ॒ज॒ दे॒वेभ्यो॑ दे॒वेभ्यः॑ सृज सृज दे॒वेभ्यो॑ भाग॒धा भा॑ग॒धा दे॒वेभ्यः॑ सृज सृज दे॒वेभ्यो॑ भाग॒धाः । \newline
31. दे॒वेभ्यो॑ भाग॒धा भा॑ग॒धा दे॒वेभ्यो॑ दे॒वेभ्यो॑ भाग॒धा अ॑स दसद् भाग॒धा दे॒वेभ्यो॑ दे॒वेभ्यो॑ भाग॒धा अ॑सत् । \newline
32. भा॒ग॒धा अ॑स दसद् भाग॒धा भा॑ग॒धा अ॑सत् । \newline
33. भा॒ग॒धा इति॑ भाग - धाः । \newline
34. अ॒स॒दित्य॑सत् । \newline
35. यस्य॑ कु॒र्मः कु॒र्मो यस्य॒ यस्य॑ कु॒र्मो ह॒विर्. ह॒विः कु॒र्मो यस्य॒ यस्य॑ कु॒र्मो ह॒विः । \newline
36. कु॒र्मो ह॒विर्. ह॒विः कु॒र्मः कु॒र्मो ह॒विर् गृ॒हे गृ॒हे ह॒विः कु॒र्मः कु॒र्मो ह॒विर् गृ॒हे । \newline
37. ह॒विर् गृ॒हे गृ॒हे ह॒विर्. ह॒विर् गृ॒हे तम् तम् गृ॒हे ह॒विर्. ह॒विर् गृ॒हे तम् । \newline
38. गृ॒हे तम् तम् गृ॒हे गृ॒हेत म॑ग्ने अग्ने॒ तम् गृ॒हे गृ॒हेत म॑ग्ने । \newline
39. त म॑ग्ने अग्ने॒ तम् त म॑ग्ने वर्द्धय वर्द्धयाग्ने॒ तम् त म॑ग्ने वर्द्धय । \newline
40. अ॒ग्ने॒ व॒र्द्ध॒य॒ व॒र्द्ध॒ या॒ग्ने॒ अ॒ग्ने॒ व॒र्द्ध॒या॒ त्वम् त्वं ॅव॑र्द्ध याग्ने अग्ने वर्द्धया॒ त्वम् । \newline
41. व॒र्द्ध॒या॒ त्वम् त्वं ॅव॑र्द्धय वर्द्धया॒ त्वम् । \newline
42. त्वमिति॒ त्वम् । \newline
43. तस्मै॑ दे॒वा दे॒वा स्तस्मै॒ तस्मै॑ दे॒वा अध्यधि॑ दे॒वा स्तस्मै॒ तस्मै॑ दे॒वा अधि॑ । \newline
44. दे॒वा अध्यधि॑ दे॒वा दे॒वा अधि॑ ब्रवन् ब्रव॒न् नधि॑ दे॒वा दे॒वा अधि॑ ब्रवन्न् । \newline
45. अधि॑ ब्रवन् ब्रव॒न् नध्यधि॑ ब्रवन् न॒य म॒यम् ब्र॑व॒न् नध्यधि॑ ब्रवन् न॒यम् । \newline
46. ब्र॒व॒न् न॒य म॒यम् ब्र॑वन् ब्रवन् न॒यम् च॑ चा॒यम् ब्र॑वन् ब्रवन् न॒यम् च॑ । \newline
47. अ॒यम् च॑ चा॒य म॒यम् च॒ ब्रह्म॑णो॒ ब्रह्म॑ण श्चा॒य म॒यम् च॒ ब्रह्म॑णः । \newline
48. च॒ ब्रह्म॑णो॒ ब्रह्म॑णश्च च॒ ब्रह्म॑ण॒ स्पति॒ष् पति॒र् ब्रह्म॑णश्च च॒ ब्रह्म॑ण॒ स्पतिः॑ । \newline
49. ब्रह्म॑ण॒ स्पति॒ष् पति॒र् ब्रह्म॑णो॒ ब्रह्म॑ण॒ स्पतिः॑ । \newline
50. पति॒रिति॒ पतिः॑ । \newline
51. उदु॑ वु॒ वुदुदु॑ त्वा त्व॒ वुदुदु॑ त्वा । \newline
52. उ॒ त्वा॒ त्व॒ वु॒ त्वा॒ विश्वे॒ विश्वे᳚ त्व वु त्वा॒ विश्वे᳚ । \newline
53. त्वा॒ विश्वे॒ विश्वे᳚ त्वा त्वा॒ विश्वे॑ दे॒वा दे॒वा विश्वे᳚ त्वा त्वा॒ विश्वे॑ दे॒वाः । \newline
54. विश्वे॑ दे॒वा दे॒वा विश्वे॒ विश्वे॑ दे॒वा अग्ने ऽग्ने॑ दे॒वा विश्वे॒ विश्वे॑ दे॒वा अग्ने᳚ । \newline
55. दे॒वा अग्ने ऽग्ने॑ दे॒वा दे॒वा अग्ने॒ भर॑न्तु॒ भर॒न्त्वग्ने॑ दे॒वा दे॒वा अग्ने॒ भर॑न्तु । \newline
\pagebreak
\markright{ TS 4.6.3.2  \hfill https://www.vedavms.in \hfill}

\section{ TS 4.6.3.2 }

\textbf{TS 4.6.3.2 } \newline
\textbf{Samhita Paata} \newline

अग्ने॒ भर॑न्तु॒ चित्ति॑भिः । स नो॑ भव शि॒वत॑मः सु॒प्रती॑को वि॒भाव॑सुः ॥ पञ्च॒ दिशो॒ दैवी᳚र्य॒ज्ञ्म॑वन्तु दे॒वीरपाम॑तिं दुर्म॒तिं बाध॑मानाः । रा॒यस्पोषे॑ य॒ज्ञ्प॑ति-मा॒भज॑न्तीः ॥ रा॒यस्पोषे॒ अधि॑ य॒ज्ञो अ॑स्था॒थ् समि॑द्धे अ॒ग्नावधि॑ मामहा॒नः । उ॒क्थप॑त्र॒ ईड्यो॑ गृभी॒तस्त॒प्तं घ॒र्मं प॑रि॒गृह्या॑यजन्त ॥ ऊ॒र्जा यद्-य॒ज्ञ्मश॑मन्त दे॒वा दैव्या॑य ध॒र्त्रे जोष्ट्रे᳚ । दे॒व॒श्रीः श्रीम॑णाः श॒तप॑याः - [  ] \newline

\textbf{Pada Paata} \newline

अग्ने᳚ । भर॑न्तु । चित्ति॑भि॒रिति॒ चित्ति॑ - भिः॒ ॥ सः । नः॒ । भ॒व॒ । शि॒वत॑म॒ इति॑ शि॒व - त॒मः॒ । सु॒प्रती॑क॒ इति॑ सु - प्रती॑कः । वि॒भाव॑सु॒रिति॑ वि॒भा - व॒सुः॒ ॥ पञ्च॑ । दिशः॑ । दैवीः᳚ । य॒ज्ञ्म् । अ॒व॒न्तु॒ । दे॒वीः । अपेति॑ । अम॑तिम् । दु॒र्म॒तिमिति॑ दुः - म॒तिम् । बाध॑मानाः ॥ रा॒यः । पोषे᳚ । य॒ज्ञ्प॑ति॒मिति॑ य॒ज्ञ् - प॒ति॒म् । आ॒भज॑न्ती॒रित्या᳚ - भज॑न्तीः ॥ रा॒यः । पोषे᳚ । अधीति॑ । य॒ज्ञ्ः । अ॒स्था॒त् । समि॑द्ध॒ इति॒ सं - इ॒द्धे॒ । अ॒ग्नौ । अधीति॑ । मा॒म॒हा॒नः ॥ उ॒क्थप॑त्र॒ इत्यु॒क्थ - प॒त्रः॒ । ईड्यः॑ । गृ॒भी॒तः । त॒प्तम् । घ॒र्मम् । प॒रि॒गृह्येति॑ परि-गृह्य॑ । अ॒य॒ज॒न्त॒ ॥ ऊ॒र्जा । यत् । य॒ज्ञ्म् । अश॑मन्त । दे॒वाः । दैव्या॑य । ध॒र्त्रे । जोष्ट्रे᳚ ॥ दे॒व॒श्रीरिति॑ देव - श्रीः । श्रीम॑णा॒ इति॒ श्री - म॒नाः॒ । श॒तप॑या॒ इति॑ श॒त - प॒याः॒ ।  \newline


\textbf{Krama Paata} \newline

अग्ने॒ भर॑न्तु । भर॑न्तु॒ चित्ति॑भिः । चित्ति॑भि॒रिति॒ चित्ति॑ - भिः॒ ॥ स नः॑ । नो॒ भ॒व॒ । भ॒व॒ शि॒वत॑मः । शि॒वत॑मः सु॒प्रती॑कः । शि॒वत॑म॒ इति॑ शि॒व - त॒मः॒ । सु॒प्रती॑को वि॒भाव॑सुः । सु॒प्रती॑क॒ इति॑ सु - प्रती॑कः । वि॒भाव॑सु॒रिति॑ वि॒भा - व॒सुः॒ ॥ पञ्च॒ दिशः॑ । दिशो॒ दैवीः᳚ । दैवी᳚र् य॒ज्ञ्म् । य॒ज्ञ्म॑वन्तु । अ॒व॒न्तु॒ दे॒वीः । दे॒वीरप॑ । अपाम॑तिम् । अम॑तिम् दुर्म॒तिम् । दु॒र्म॒तिम् बाध॑मानाः । दु॒र्म॒तिमिति॑ दुः - म॒तिम् । बाध॑माना॒ इति॒ बाध॑मानाः ॥ रा॒यस्पोषे᳚ । पोषे॑ य॒ज्ञ्प॑तिम् । य॒ज्ञ्प॑तिमा॒भज॑न्तीः । य॒ज्ञ्प॑ति॒मिति॑ य॒ज्ञ् - प॒ति॒म् । आ॒भज॑न्ती॒रित्या᳚ - भज॑न्तीः ॥ रा॒यस्पोषे᳚ । पोषे॒ अधि॑ । अधि॑ य॒ज्ञ्ः । य॒ज्ञो अ॑स्थात् । अ॒स्था॒थ् समि॑द्धे । समि॑द्धे अ॒ग्नौ । समि॑द्ध॒ इति॒ सम् - इ॒द्धे॒ । अ॒ग्नावधि॑ । अधि॑ मामहा॒नः । मा॒म॒हा॒न इति॑ मामहा॒नः ॥ उ॒क्थप॑त्र॒ ईड्यः॑ । उ॒क्थप॑त्र॒ इत्यु॒क्थ - प॒त्रः॒ । ईड्यो॑ गृभी॒तः । गृ॒भी॒तस्त॒प्तम् । त॒प्तम् घ॒र्मम् । घ॒र्मम् प॑रि॒गृह्य॑ । प॒रि॒गृह्या॑यजन्त । प॒रि॒गृह्येति॑ परि - गृह्य॑ । अ॒य॒ज॒न्तेत्य॑यजन्त ॥ ऊ॒र्जा यत् । यद् य॒ज्ञ्म् । य॒ज्ञ्मश॑मन्त । अश॑मन्त दे॒वाः । दे॒वा दैव्या॑य । दैव्या॑य ध॒र्त्रे । ध॒र्त्रे जोष्ट्रे᳚ । जोष्ट्रे॒ इति॒ जोष्ट्रे᳚ ॥ दे॒व॒श्रीः श्रीम॑णाः । दे॒व॒श्रीरिति॑ देव - श्रीः । श्रीम॑णाः श॒तप॑याः । श्रीम॑णा॒ इति॒ श्री - म॒नाः॒ । श॒तप॑याः परि॒गृह्य॑ । श॒तप॑या॒ इति॑ श॒त - प॒याः॒ \newline

\textbf{Jatai Paata} \newline

1. अग्ने॒ भर॑न्तु॒ भर॒ न्त्वग्ने ऽग्ने॒ भर॑न्तु । \newline
2. भर॑न्तु॒ चित्ति॑भि॒ श्चित्ति॑भि॒र् भर॑न्तु॒ भर॑न्तु॒ चित्ति॑भिः । \newline
3. चित्ति॑भि॒रिति॒ चित्ति॑ - भिः॒ । \newline
4. स नो॑ नः॒ स स नः॑ । \newline
5. नो॒ भ॒व॒ भ॒व॒ नो॒ नो॒ भ॒व॒ । \newline
6. भ॒व॒ शि॒वत॑मः शि॒वत॑मो भव भव शि॒वत॑मः । \newline
7. शि॒वत॑मः सु॒प्रती॑कः सु॒प्रती॑कः शि॒वत॑मः शि॒वत॑मः सु॒प्रती॑कः । \newline
8. शि॒वत॑म॒ इति॑ शि॒व - त॒मः॒ । \newline
9. सु॒प्रती॑को वि॒भाव॑सुर् वि॒भाव॑सुः सु॒प्रती॑कः सु॒प्रती॑को वि॒भाव॑सुः । \newline
10. सु॒प्रती॑क॒ इति॑ सु - प्रती॑कः । \newline
11. वि॒भाव॑सु॒रिति॑ वि॒भा - व॒सुः॒ । \newline
12. पञ्च॒ दिशो॒ दिशः॒ पञ्च॒ पञ्च॒ दिशः॑ । \newline
13. दिशो॒ दैवी॒र् दैवी॒र् दिशो॒ दिशो॒ दैवीः᳚ । \newline
14. दैवी᳚र् य॒ज्ञ्ं ॅय॒ज्ञ्म् दैवी॒र् दैवी᳚र् य॒ज्ञ्म् । \newline
15. य॒ज्ञ् म॑व न्त्ववन्तु य॒ज्ञ्ं ॅय॒ज्ञ् म॑वन्तु । \newline
16. अ॒व॒न्तु॒ दे॒वीर् दे॒वी र॑व न्त्ववन्तु दे॒वीः । \newline
17. दे॒वी रपाप॑ दे॒वीर् दे॒वी रप॑ । \newline
18. अपाम॑ति॒ मम॑ति॒ मपापाम॑तिम् । \newline
19. अम॑तिम् दुर्म॒तिम् दु॑र्म॒ति मम॑ति॒ मम॑तिम् दुर्म॒तिम् । \newline
20. दु॒र्म॒तिम् बाध॑माना॒ बाध॑माना दुर्म॒तिम् दु॑र्म॒तिम् बाध॑मानाः । \newline
21. दु॒र्म॒तिमिति॑ दुः - म॒तिम् । \newline
22. बाध॑माना॒ इति॒ बाध॑मानाः । \newline
23. रा॒य स्पोषे॒ पोषे॑ रा॒यो रा॒य स्पोषे᳚ । \newline
24. पोषे॑ य॒ज्ञ्प॑तिं ॅय॒ज्ञ्प॑ति॒म् पोषे॒ पोषे॑ य॒ज्ञ्प॑तिम् । \newline
25. य॒ज्ञ्प॑ति मा॒भज॑न्ती रा॒भज॑न्तीर् य॒ज्ञ्प॑तिं ॅय॒ज्ञ्प॑ति मा॒भज॑न्तीः । \newline
26. य॒ज्ञ्प॑ति॒मिति॑ य॒ज्ञ् - प॒ति॒म् । \newline
27. आ॒भज॑न्ती॒रित्या᳚ - भज॑न्तीः । \newline
28. रा॒य स्पोषे॒ पोषे॑ रा॒यो रा॒य स्पोषे᳚ । \newline
29. पोषे॒ अध्यधि॒ पोषे॒ पोषे॒ अधि॑ । \newline
30. अधि॑ य॒ज्ञो य॒ज्ञो अध्यधि॑ य॒ज्ञ्ः । \newline
31. य॒ज्ञो अ॑स्था दस्थाद् य॒ज्ञो य॒ज्ञो अ॑स्थात् । \newline
32. अ॒स्था॒थ् समि॑द्धे॒ समि॑द्धे अस्था दस्था॒थ् समि॑द्धे । \newline
33. समि॑द्धे अ॒ग्ना व॒ग्नौ समि॑द्धे॒ समि॑द्धे अ॒ग्नौ । \newline
34. समि॑द्ध॒ इति॒ सं - इ॒द्धे॒ । \newline
35. अ॒ग्ना वध्यध्य॒ग्ना व॒ग्ना वधि॑ । \newline
36. अधि॑ मामहा॒नो मा॑महा॒नो अध्यधि॑ मामहा॒नः । \newline
37. मा॒म॒हा॒न इति॑ मामहा॒नः । \newline
38. उ॒क्थप॑त्र॒ ईड्य॒ ईड्य॑ उ॒क्थप॑त्र उ॒क्थप॑त्र॒ ईड्यः॑ । \newline
39. उ॒क्थप॑त्र॒ इत्यु॒क्थ - प॒त्रः॒ । \newline
40. ईड्यो॑ गृभी॒तो गृ॑भी॒त ईड्य॒ ईड्यो॑ गृभी॒तः । \newline
41. गृ॒भी॒त स्त॒प्तम् त॒प्तम् गृ॑भी॒तो गृ॑भी॒त स्त॒प्तम् । \newline
42. त॒प्तम् घ॒र्मम् घ॒र्मम् त॒प्तम् त॒प्तम् घ॒र्मम् । \newline
43. घ॒र्मम् प॑रि॒गृह्य॑ परि॒गृह्य॑ घ॒र्मम् घ॒र्मम् प॑रि॒गृह्य॑ । \newline
44. प॒रि॒गृह्या॑ यजन्ता यजन्त परि॒गृह्य॑ परि॒गृह्या॑ यजन्त । \newline
45. प॒रि॒गृह्येति॑ परि - गृह्य॑ । \newline
46. अ॒य॒ज॒न्तेत्य॑यजन्त । \newline
47. ऊ॒र्जा यद् यदू॒र्जोर्जा यत् । \newline
48. यद् य॒ज्ञ्ं ॅय॒ज्ञ्ं ॅयद् यद् य॒ज्ञ्म् । \newline
49. य॒ज्ञ् मश॑म॒न्ता श॑मन्त य॒ज्ञ्ं ॅय॒ज्ञ् मश॑मन्त । \newline
50. अश॑मन्त दे॒वा दे॒वा अश॑म॒न्ता श॑मन्त दे॒वाः । \newline
51. दे॒वा दैव्या॑य॒ दैव्या॑य दे॒वा दे॒वा दैव्या॑य । \newline
52. दैव्या॑य ध॒र्त्रे ध॒र्त्रे दैव्या॑य॒ दैव्या॑य ध॒र्त्रे । \newline
53. ध॒र्त्रे जोष्ट्रे॒ जोष्ट्रे॑ ध॒र्त्रे ध॒र्त्रे जोष्ट्रे᳚ । \newline
54. जोष्ट्र॒ इति॒ जोष्ट्रे᳚ । \newline
55. दे॒व॒श्रीः श्रीम॑णाः॒ श्रीम॑णा देव॒श्रीर् दे॑व॒श्रीः श्रीम॑णाः । \newline
56. दे॒व॒श्रीरिति॑ देव - श्रीः । \newline
57. श्रीम॑णाः श॒तप॑याः श॒तप॑याः॒ श्रीम॑णाः॒ श्रीम॑णाः श॒तप॑याः । \newline
58. श्रीम॑णा॒ इति॒ श्री - म॒नाः॒ । \newline
59. श॒तप॑याः परि॒गृह्य॑ परि॒गृह्य॑ श॒तप॑याः श॒तप॑याः परि॒गृह्य॑ । \newline
60. श॒तप॑या॒ इति॑ श॒त - प॒याः॒ । \newline

\textbf{Ghana Paata } \newline

1. अग्ने॒ भर॑न्तु॒ भर॒न्त्वग्ने ऽग्ने॒ भर॑न्तु॒ चित्ति॑भि॒ श्चित्ति॑भि॒र् भर॒न्त्वग्ने ऽग्ने॒ भर॑न्तु॒ चित्ति॑भिः । \newline
2. भर॑न्तु॒ चित्ति॑भि॒ श्चित्ति॑भि॒र् भर॑न्तु॒ भर॑न्तु॒ चित्ति॑भिः । \newline
3. चित्ति॑भि॒रिति॒ चित्ति॑ - भिः॒ । \newline
4. स नो॑ नः॒ स स नो॑ भव भव नः॒ स स नो॑ भव । \newline
5. नो॒ भ॒व॒ भ॒व॒ नो॒ नो॒ भ॒व॒ शि॒वत॑मः शि॒वत॑मो भव नो नो भव शि॒वत॑मः । \newline
6. भ॒व॒ शि॒वत॑मः शि॒वत॑मो भव भव शि॒वत॑मः सु॒प्रती॑कः सु॒प्रती॑कः शि॒वत॑मो भव भव शि॒वत॑मः सु॒प्रती॑कः । \newline
7. शि॒वत॑मः सु॒प्रती॑कः सु॒प्रती॑कः शि॒वत॑मः शि॒वत॑मः सु॒प्रती॑को वि॒भाव॑सुर् वि॒भाव॑सुः सु॒प्रती॑कः शि॒वत॑मः शि॒वत॑मः सु॒प्रती॑को वि॒भाव॑सुः । \newline
8. शि॒वत॑म॒ इति॑ शि॒व - त॒मः॒ । \newline
9. सु॒प्रती॑को वि॒भाव॑सुर् वि॒भाव॑सुः सु॒प्रती॑कः सु॒प्रती॑को वि॒भाव॑सुः । \newline
10. सु॒प्रती॑क॒ इति॑ सु - प्रती॑कः । \newline
11. वि॒भाव॑सु॒रिति॑ वि॒भा - व॒सुः॒ । \newline
12. पञ्च॒ दिशो॒ दिशः॒ पञ्च॒ पञ्च॒ दिशो॒ दैवी॒र् दैवी॒र् दिशः॒ पञ्च॒ पञ्च॒ दिशो॒ दैवीः᳚ । \newline
13. दिशो॒ दैवी॒र् दैवी॒र् दिशो॒ दिशो॒ दैवी᳚र् य॒ज्ञ्ं ॅय॒ज्ञ्म् दैवी॒र् दिशो॒ दिशो॒ दैवी᳚र् य॒ज्ञ्म् । \newline
14. दैवी᳚र् य॒ज्ञ्ं ॅय॒ज्ञ्म् दैवी॒र् दैवी᳚र् य॒ज्ञ् म॑वन्त्ववन्तु य॒ज्ञ्म् दैवी॒र् दैवी᳚र् य॒ज्ञ् म॑वन्तु । \newline
15. य॒ज्ञ् म॑वन्त्ववन्तु य॒ज्ञ्ं ॅय॒ज्ञ् म॑वन्तु दे॒वीर् दे॒वी र॑वन्तु य॒ज्ञ्ं ॅय॒ज्ञ् म॑वन्तु दे॒वीः । \newline
16. अ॒व॒न्तु॒ दे॒वीर् दे॒वी र॑वन् त्ववन्तु दे॒वी रपाप॑ दे॒वी र॑वन् त्ववन्तु दे॒वीरप॑ । \newline
17. दे॒वी रपाप॑ दे॒वीर् दे॒वी रपा म॑ति॒ मम॑ति॒ मप॑ दे॒वीर् दे॒वी रपा म॑तिम् । \newline
18. अपाम॑ति॒ मम॑ति॒ मपा पा म॑तिम् दुर्म॒तिम् दु॑र्म॒ति मम॑ति॒ मपा पाम॑तिम् दुर्म॒तिम् । \newline
19. अम॑तिम् दुर्म॒तिम् दु॑र्म॒ति मम॑ति॒ मम॑तिम् दुर्म॒तिम् बाध॑माना॒ बाध॑माना दुर्म॒ति मम॑ति॒ मम॑तिम् दुर्म॒तिम् बाध॑मानाः । \newline
20. दु॒र्म॒तिम् बाध॑माना॒ बाध॑माना दुर्म॒तिम् दु॑र्म॒तिम् बाध॑मानाः । \newline
21. दु॒र्म॒तिमिति॑ दुः - म॒तिम् । \newline
22. बाध॑माना॒ इति॒ बाध॑मानाः । \newline
23. रा॒य स्पोषे॒ पोषे॑ रा॒यो रा॒य स्पोषे॑ य॒ज्ञ्प॑तिं ॅय॒ज्ञ्प॑ति॒म् पोषे॑ रा॒यो रा॒य स्पोषे॑ य॒ज्ञ्प॑तिम् । \newline
24. पोषे॑ य॒ज्ञ्प॑तिं ॅय॒ज्ञ्प॑ति॒म् पोषे॒ पोषे॑ य॒ज्ञ्प॑ति मा॒भज॑न्ती रा॒भज॑न्तीर् य॒ज्ञ्प॑ति॒म् पोषे॒ पोषे॑ य॒ज्ञ्प॑ति मा॒भज॑न्तीः । \newline
25. य॒ज्ञ्प॑ति मा॒भज॑न्ती रा॒भज॑न्तीर् य॒ज्ञ्प॑तिं ॅय॒ज्ञ्प॑ति मा॒भज॑न्तीः । \newline
26. य॒ज्ञ्प॑ति॒मिति॑ य॒ज्ञ् - प॒ति॒म् । \newline
27. आ॒भज॑न्ती॒रित्या᳚ - भज॑न्तीः । \newline
28. रा॒य स्पोषे॒ पोषे॑ रा॒यो रा॒य स्पोषे॒ अध्यधि॒ पोषे॑ रा॒यो रा॒य स्पोषे॒ अधि॑ । \newline
29. पोषे॒ अध्यधि॒ पोषे॒ पोषे॒ अधि॑ य॒ज्ञो य॒ज्ञो अधि॒ पोषे॒ पोषे॒ अधि॑ य॒ज्ञ्ः । \newline
30. अधि॑ य॒ज्ञो य॒ज्ञो अध्यधि॑ य॒ज्ञो अ॑स्था दस्थाद् य॒ज्ञो अध्यधि॑ य॒ज्ञो अ॑स्थात् । \newline
31. य॒ज्ञो अ॑स्था दस्थाद् य॒ज्ञो य॒ज्ञो अ॑स्था॒थ् समि॑द्धे॒ समि॑द्धे अस्थाद् य॒ज्ञो य॒ज्ञो अ॑स्था॒थ् समि॑द्धे । \newline
32. अ॒स्था॒थ् समि॑द्धे॒ समि॑द्धे अस्था दस्था॒थ् समि॑द्धे अ॒ग्ना व॒ग्नौ समि॑द्धे अस्था दस्था॒थ् समि॑द्धे अ॒ग्नौ । \newline
33. समि॑द्धे अ॒ग्ना व॒ग्नौ समि॑द्धे॒ समि॑द्धे अ॒ग्ना वध्य ध्य॒ग्नौ समि॑द्धे॒ समि॑द्धे अ॒ग्ना वधि॑ । \newline
34. समि॑द्ध॒ इति॒ सं - इ॒द्धे॒ । \newline
35. अ॒ग्ना वध्य ध्य॒ग्ना व॒ग्ना वधि॑ मामहा॒नो मा॑महा॒नो अध्य॒ग्ना व॒ग्ना वधि॑ मामहा॒नः । \newline
36. अधि॑ मामहा॒नो मा॑महा॒नो अध्यधि॑ मामहा॒नः । \newline
37. मा॒म॒हा॒न इति॑ मामहा॒नः । \newline
38. उ॒क्थप॑त्र॒ ईड्य॒ ईड्य॑ उ॒क्थप॑त्र उ॒क्थप॑त्र॒ ईड्यो॑ गृभी॒तो गृ॑भी॒त ईड्य॑ उ॒क्थप॑त्र उ॒क्थप॑त्र॒ ईड्यो॑ गृभी॒तः । \newline
39. उ॒क्थप॑त्र॒ इत्यु॒क्थ - प॒त्रः॒ । \newline
40. ईड्यो॑ गृभी॒तो गृ॑भी॒त ईड्य॒ ईड्यो॑ गृभी॒त स्त॒प्तम् त॒प्तम् गृ॑भी॒त ईड्य॒ ईड्यो॑ गृभी॒त स्त॒प्तम् । \newline
41. गृ॒भी॒त स्त॒प्तम् त॒प्तम् गृ॑भी॒तो गृ॑भी॒त स्त॒प्तम् घ॒र्मम् घ॒र्मम् त॒प्तम् गृ॑भी॒तो गृ॑भी॒त स्त॒प्तम् घ॒र्मम् । \newline
42. त॒प्तम् घ॒र्मम् घ॒र्मम् त॒प्तम् त॒प्तम् घ॒र्मम् प॑रि॒गृह्य॑ परि॒गृह्य॑ घ॒र्मम् त॒प्तम् त॒प्तम् घ॒र्मम् प॑रि॒गृह्य॑ । \newline
43. घ॒र्मम् प॑रि॒गृह्य॑ परि॒गृह्य॑ घ॒र्मम् घ॒र्मम् प॑रि॒गृह्या॑ यजन्ता यजन्त परि॒गृह्य॑ घ॒र्मम् घ॒र्मम् प॑रि॒गृह्या॑ यजन्त । \newline
44. प॒रि॒गृह्या॑ यजन्ता यजन्त परि॒गृह्य॑ परि॒गृह्या॑ यजन्त । \newline
45. प॒रि॒गृह्येति॑ परि - गृह्य॑ । \newline
46. अ॒य॒ज॒न्तेत्य॑यजन्त । \newline
47. ऊ॒र्जा यद् यदू॒र्जोर्जा यद् य॒ज्ञ्ं ॅय॒ज्ञ्ं ॅयदू॒र्जोर्जा यद् य॒ज्ञ्म् । \newline
48. यद् य॒ज्ञ्ं ॅय॒ज्ञ्ं ॅयद् यद् य॒ज्ञ् मश॑म॒न्ता श॑मन्त य॒ज्ञ्ं ॅयद् यद् य॒ज्ञ् मश॑मन्त । \newline
49. य॒ज्ञ् मश॑म॒न्ता श॑मन्त य॒ज्ञ्ं ॅय॒ज्ञ् मश॑मन्त दे॒वा दे॒वा अश॑मन्त य॒ज्ञ्ं ॅय॒ज्ञ् मश॑मन्त दे॒वाः । \newline
50. अश॑मन्त दे॒वा दे॒वा अश॑म॒न्ता श॑मन्त दे॒वा दैव्या॑य॒ दैव्या॑य दे॒वा अश॑म॒न्ता श॑मन्त दे॒वा दैव्या॑य । \newline
51. दे॒वा दैव्या॑य॒ दैव्या॑य दे॒वा दे॒वा दैव्या॑य ध॒र्त्रे ध॒र्त्रे दैव्या॑य दे॒वा दे॒वा दैव्या॑य ध॒र्त्रे । \newline
52. दैव्या॑य ध॒र्त्रे ध॒र्त्रे दैव्या॑य॒ दैव्या॑य ध॒र्त्रे जोष्ट्रे॒ जोष्ट्रे॑ ध॒र्त्रे दैव्या॑य॒ दैव्या॑य ध॒र्त्रे जोष्ट्रे᳚ । \newline
53. ध॒र्त्रे जोष्ट्रे॒ जोष्ट्रे॑ ध॒र्त्रे ध॒र्त्रे जोष्ट्रे᳚ । \newline
54. जोष्ट्र॒ इति॒ जोष्ट्रे᳚ । \newline
55. दे॒व॒श्रीः श्रीम॑णाः॒ श्रीम॑णा देव॒श्रीर् दे॑व॒श्रीः श्रीम॑णाः श॒तप॑याः श॒तप॑याः॒ श्रीम॑णा देव॒श्रीर् दे॑व॒श्रीः श्रीम॑णाः श॒तप॑याः । \newline
56. दे॒व॒श्रीरिति॑ देव - श्रीः । \newline
57. श्रीम॑णाः श॒तप॑याः श॒तप॑याः॒ श्रीम॑णाः॒ श्रीम॑णाः श॒तप॑याः परि॒गृह्य॑ परि॒गृह्य॑ श॒तप॑याः॒ श्रीम॑णाः॒ श्रीम॑णाः श॒तप॑याः परि॒गृह्य॑ । \newline
58. श्रीम॑णा॒ इति॒ श्री - म॒नाः॒ । \newline
59. श॒तप॑याः परि॒गृह्य॑ परि॒गृह्य॑ श॒तप॑याः श॒तप॑याः परि॒गृह्य॑ दे॒वा दे॒वाः प॑रि॒गृह्य॑ श॒तप॑याः श॒तप॑याः परि॒गृह्य॑ दे॒वाः । \newline
60. श॒तप॑या॒ इति॑ श॒त - प॒याः॒ । \newline
\pagebreak
\markright{ TS 4.6.3.3  \hfill https://www.vedavms.in \hfill}

\section{ TS 4.6.3.3 }

\textbf{TS 4.6.3.3 } \newline
\textbf{Samhita Paata} \newline

परि॒गृह्य॑ दे॒वा य॒ज्ञ्मा॑यन्न् ॥ सूर्य॑रश्मि॒र्॒.हरि॑केशः पु॒रस्ता᳚थ् सवि॒ता ज्योति॒रुद॑याꣳ॒॒ अज॑स्रं । तस्य॑ पू॒षा प्र॑स॒वं ॅया॑ति दे॒वः स॒पंश्य॒न् विश्वा॒ भुव॑नानि गो॒पाः ॥ दे॒वा दे॒वेभ्यो॑ अद्ध्व॒र्यन्तो॑ अस्थुर्वी॒तꣳ श॑मि॒त्रे श॑मि॒ता य॒जद्ध्यै᳚ । तु॒रीयो॑ य॒ज्ञो यत्र॑ ह॒व्यमेति॒ ततः॑ पाव॒का आ॒शिषो॑ नो जुषन्तां ॥ वि॒मान॑ ए॒ष दि॒वो मद्ध्य॑ आस्त आपप्रि॒वान् रोद॑सी अ॒न्तरि॑क्षं । स वि॒श्वाची॑र॒भि - [  ] \newline

\textbf{Pada Paata} \newline

प॒रि॒गृह्येति॑ परि - गृह्य॑ । दे॒वाः । य॒ज्ञ्म् । आ॒य॒न्न् ॥ सूर्य॑रश्मि॒रिति॒ सूर्य॑ - र॒श्मिः॒ । हरि॑केश॒ इति॒ हरि॑ - के॒शः॒ । पु॒रस्ता᳚त् । स॒वि॒ता । ज्योतिः॑ । उदिति॑ । अ॒या॒न् । अज॑स्रम् ॥ तस्य॑ । पू॒षा । प्र॒स॒वमिति॑ प्र - स॒वम् । या॒ति॒ । दे॒वः । स॒पंश्य॒न्निति॑ सं - पश्यन्न्॑ । विश्वा᳚ । भुव॑नानि । गो॒पा इति॑ गो - पाः ॥ दे॒वाः । दे॒वेभ्यः॑ । अ॒द्ध्व॒र्यन्तः॑ । अ॒स्थुः॒ । वी॒तम् । श॒मि॒त्रे । श॒मि॒ता । य॒जद्ध्यै᳚ ॥ तु॒रीयः॑ । य॒ज्ञ्ः । यत्र॑ । ह॒व्यम् । एति॑ । ततः॑ । पा॒व॒काः । आ॒शिष॒ इत्या᳚-शिषः॑ । नः॒ । जु॒ष॒न्ता॒म् ॥ वि॒मान॒ इति॑ वि-मानः॑ । ए॒षः । दि॒वः । म॒द्ध्ये᳚ । आ॒स्ते॒ । आ॒प॒प्रि॒वानित्या᳚ - प॒प्रि॒वान् । रोद॑सी॒ इति॑ । अ॒न्तरि॑क्षम् ॥ सः । वि॒श्वाचीः᳚ । अ॒भीति॑ ।  \newline


\textbf{Krama Paata} \newline

प॒रि॒गृह्य॑ दे॒वाः । प॒रि॒गृह्येति॑ परि - गृह्य॑ । दे॒वा य॒ज्ञ्म् । य॒ज्ञ्मा॑यन्न् । आ॒य॒न्नित्या॑यन्न् ॥ सूर्य॑रश्मि॒र्॒. हरि॑केशः । सूर्य॑रश्मि॒रिति॒ सूर्य॑ - र॒श्मिः॒ । हरि॑केशः पु॒रस्ता᳚त् । हरि॑केश॒ इति॒ हरि॑ - के॒शः॒ । पु॒रस्ता᳚थ् सवि॒ता । स॒वि॒ता ज्योतिः॑ । ज्योति॒रुत् । उद॑यान् । अ॒याꣳ॒॒ अज॑स्रम् । अज॑स्र॒मित्यज॑स्रम् ॥ तस्य॑ पू॒षा । पू॒षा प्र॑स॒वम् । प्र॒स॒वं ॅया॑ति । प्र॒स॒वमिति॑ प्र - स॒वम् । या॒ति॒ दे॒वः । दे॒वः स॒म्पश्यन्न्॑ । स॒म्पश्य॒न् विश्वा᳚ । स॒म्पश्य॒न्निति॑ सम् - पश्यन्न्॑ । विश्वा॒ भुव॑नानि । भुव॑नानि गो॒पाः । गो॒पा इति॑ गो - पाः ॥ दे॒वा दे॒वेभ्यः॑ । दे॒वेभ्यो॑ अद्ध्व॒र्यन्तः॑ । अ॒द्ध्व॒र्यन्तो॑ अस्थुः । अ॒स्थु॒र् वी॒तम् । वी॒तꣳ श॑मि॒त्रे । श॒मि॒त्रे श॑मि॒ता । श॒मि॒ता य॒जद्ध्यै᳚ । य॒जद्ध्या॒ इति॑ य॒जद्ध्यै᳚ ॥ तु॒रीयो॑ य॒ज्ञ्ः । य॒ज्ञो यत्र॑ । यत्र॑ ह॒व्यम् । ह॒व्यमेति॑ । एति॒ ततः॑ । ततः॑ पाव॒काः । पा॒व॒का आ॒शिषः॑ । आ॒शिषो॑ नः । आ॒शिष॒ इत्या᳚ - शिषः॑ । नो॒ जु॒ष॒न्ता॒म् । जु॒ष॒न्ता॒मिति॑ जुषन्ताम् ॥ वि॒मान॑ ए॒षः । वि॒मान॒ इति॑ वि - मानः॑ । ए॒ष दि॒वः । दि॒वो मद्ध्ये᳚ । मद्ध्य॑ आस्ते । आ॒स्त॒ आ॒प॒प्रि॒वान् । आ॒प॒प्रि॒वान् रोद॑सी । आ॒प॒प्रि॒वानित्या᳚ - प॒प्रि॒वान् । रोद॑सी अ॒न्तरि॑क्षम् । रोद॑सी॒ इति॒ रोद॑सी । अ॒न्तरि॑क्ष॒मित्य॒न्तरि॑क्षम् ॥ स वि॒श्वाचीः᳚ । वि॒श्वाची॑र॒भि । अ॒भि च॑ष्टे \newline

\textbf{Jatai Paata} \newline

1. प॒रि॒गृह्य॑ दे॒वा दे॒वाः प॑रि॒गृह्य॑ परि॒गृह्य॑ दे॒वाः । \newline
2. प॒रि॒गृह्येति॑ परि - गृह्य॑ । \newline
3. दे॒वा य॒ज्ञ्ं ॅय॒ज्ञ्म् दे॒वा दे॒वा य॒ज्ञ्म् । \newline
4. य॒ज्ञ् मा॑यन् नायन्. य॒ज्ञ्ं ॅय॒ज्ञ् मा॑यन्न् । \newline
5. आ॒य॒न्नित्या॑यन्न् । \newline
6. सूर्य॑रश्मि॒र्॒. हरि॑केशो॒ हरि॑केशः॒ सूर्य॑रश्मिः॒ सूर्य॑रश्मि॒र्॒. हरि॑केशः । \newline
7. सूर्य॑रश्मि॒रिति॒ सूर्य॑ - र॒श्मिः॒ । \newline
8. हरि॑केशः पु॒रस्ता᳚त् पु॒रस्ता॒ द्धरि॑केशो॒ हरि॑केशः पु॒रस्ता᳚त् । \newline
9. हरि॑केश॒ इति॒ हरि॑ - के॒शः॒ । \newline
10. पु॒रस्ता᳚थ् सवि॒ता स॑वि॒ता पु॒रस्ता᳚त् पु॒रस्ता᳚थ् सवि॒ता । \newline
11. स॒वि॒ता ज्योति॒र् ज्योतिः॑ सवि॒ता स॑वि॒ता ज्योतिः॑ । \newline
12. ज्योति॒ रुदुज् ज्योति॒र् ज्योति॒ रुत् । \newline
13. उद॑याꣳ अयाꣳ॒॒ उदु द॑यान् । \newline
14. अ॒याꣳ॒॒ अज॑स्र॒ मज॑स्र मयाꣳ अयाꣳ॒॒ अज॑स्रम् । \newline
15. अज॑स्र॒मित्यज॑स्रम् । \newline
16. तस्य॑ पू॒षा पू॒षा तस्य॒ तस्य॑ पू॒षा । \newline
17. पू॒षा प्र॑स॒वम् प्र॑स॒वम् पू॒षा पू॒षा प्र॑स॒वम् । \newline
18. प्र॒स॒वं ॅया॑ति याति प्रस॒वम् प्र॑स॒वं ॅया॑ति । \newline
19. प्र॒स॒वमिति॑ प्र - स॒वम् । \newline
20. या॒ति॒ दे॒वो दे॒वो या॑ति याति दे॒वः । \newline
21. दे॒वः सं॒पश्यन्᳚ थ्सं॒पश्य॑न् दे॒वो दे॒वः सं॒पश्यन्न्॑ । \newline
22. सं॒पश्य॒न्॒. विश्वा॒ विश्वा॑ सं॒पश्यन्᳚ थ्सं॒पश्य॒न्॒. विश्वा᳚ । \newline
23. सं॒पश्य॒न्निति॑ सं - पश्यन्न्॑ । \newline
24. विश्वा॒ भुव॑नानि॒ भुव॑नानि॒ विश्वा॒ विश्वा॒ भुव॑नानि । \newline
25. भुव॑नानि गो॒पा गो॒पा भुव॑नानि॒ भुव॑नानि गो॒पाः । \newline
26. गो॒पा इति॑ गो - पाः । \newline
27. दे॒वा दे॒वेभ्यो॑ दे॒वेभ्यो॑ दे॒वा दे॒वा दे॒वेभ्यः॑ । \newline
28. दे॒वेभ्यो॑ अद्ध्व॒र्यन्तो॑ अद्ध्व॒र्यन्तो॑ दे॒वेभ्यो॑ दे॒वेभ्यो॑ अद्ध्व॒र्यन्तः॑ । \newline
29. अ॒द्ध्व॒र्यन्तो॑ अस्थु रस्थु रद्ध्व॒र्यन्तो॑ अद्ध्व॒र्यन्तो॑ अस्थुः । \newline
30. अ॒स्थु॒र् वी॒तं ॅवी॒त म॑स्थु रस्थुर् वी॒तम् । \newline
31. वी॒तꣳ श॑मि॒त्रे श॑मि॒त्रे वी॒तं ॅवी॒तꣳ श॑मि॒त्रे । \newline
32. श॒मि॒त्रे श॑मि॒ता श॑मि॒ता श॑मि॒त्रे श॑मि॒त्रे श॑मि॒ता । \newline
33. श॒मि॒ता य॒जद्ध्यै॑ य॒जद्ध्यै॑ शमि॒ता श॑मि॒ता य॒जद्ध्यै᳚ । \newline
34. य॒जद्ध्या॒ इति॑ य॒जद्ध्यै᳚ । \newline
35. तु॒रीयो॑ य॒ज्ञो य॒ज्ञ् स्तु॒रीय॑ स्तु॒रीयो॑ य॒ज्ञ्ः । \newline
36. य॒ज्ञो यत्र॒ यत्र॑ य॒ज्ञो य॒ज्ञो यत्र॑ । \newline
37. यत्र॑ ह॒व्यꣳ ह॒व्यं ॅयत्र॒ यत्र॑ ह॒व्यम् । \newline
38. ह॒व्य मेत्येति॑ ह॒व्यꣳ ह॒व्य मेति॑ । \newline
39. एति॒ तत॒ स्तत॒ एत्येति॒ ततः॑ । \newline
40. ततः॑ पाव॒काः पा॑व॒का स्तत॒ स्ततः॑ पाव॒काः । \newline
41. पा॒व॒का आ॒शिष॑ आ॒शिषः॑ पाव॒काः पा॑व॒का आ॒शिषः॑ । \newline
42. आ॒शिषो॑ नो न आ॒शिष॑ आ॒शिषो॑ नः । \newline
43. आ॒शिष॒ इत्या᳚ - शिषः॑ । \newline
44. नो॒ जु॒ष॒न्ता॒म् जु॒ष॒न्ता॒न्नो॒ नो॒ जु॒ष॒न्ता॒म् । \newline
45. जु॒ष॒न्ता॒मिति॑ जुषन्ताम् । \newline
46. वि॒मान॑ ए॒ष ए॒ष वि॒मानो॑ वि॒मान॑ ए॒षः । \newline
47. वि॒मान॒ इति॑ वि - मानः॑ । \newline
48. ए॒ष दि॒वो दि॒व ए॒ष ए॒ष दि॒वः । \newline
49. दि॒वो मद्ध्ये॒ मद्ध्ये॑ दि॒वो दि॒वो मद्ध्ये᳚ । \newline
50. मद्ध्य॑ आस्त आस्ते॒ मद्ध्ये॒ मद्ध्य॑ आस्ते । \newline
51. आ॒स्त॒ आ॒प॒प्रि॒वा ना॑पप्रि॒वा ना᳚स्त आस्त आपप्रि॒वान् । \newline
52. आ॒प॒प्रि॒वान् रोद॑सी॒ रोद॑सी आपप्रि॒वा ना॑पप्रि॒वान् रोद॑सी । \newline
53. आ॒प॒प्रि॒वानित्या᳚ - प॒प्रि॒वान् । \newline
54. रोद॑सी अ॒न्तरि॑क्ष म॒न्तरि॑क्षꣳ॒॒ रोद॑सी॒ रोद॑सी अ॒न्तरि॑क्षम् । \newline
55. रोद॑सी॒ इति॒ रोद॑सी । \newline
56. अ॒न्तरि॑क्ष॒मित्य॒न्तरि॑क्षम् । \newline
57. स वि॒श्वाची᳚र् वि॒श्वाचीः॒ स स वि॒श्वाचीः᳚ । \newline
58. वि॒श्वाची॑ र॒भ्य॑भि वि॒श्वाची᳚र् वि॒श्वाची॑ र॒भि । \newline
59. अ॒भि च॑ष्टे चष्टे अ॒भ्य॑भि च॑ष्टे । \newline

\textbf{Ghana Paata } \newline

1. प॒रि॒गृह्य॑ दे॒वा दे॒वाः प॑रि॒गृह्य॑ परि॒गृह्य॑ दे॒वा य॒ज्ञ्ं ॅय॒ज्ञ्म् दे॒वाः प॑रि॒गृह्य॑ परि॒गृह्य॑ दे॒वा य॒ज्ञ्म् । \newline
2. प॒रि॒गृह्येति॑ परि - गृह्य॑ । \newline
3. दे॒वा य॒ज्ञ्ं ॅय॒ज्ञ्म् दे॒वा दे॒वा य॒ज्ञ् मा॑यन् नायन्. य॒ज्ञ्म् दे॒वा दे॒वा य॒ज्ञ् मा॑यन्न् । \newline
4. य॒ज्ञ् मा॑यन् नायन्. य॒ज्ञ्ं ॅय॒ज्ञ् मा॑यन्न् । \newline
5. आ॒य॒न्नित्या॑यन्न् । \newline
6. सूर्य॑रश्मि॒र्॒. हरि॑केशो॒ हरि॑केशः॒ सूर्य॑रश्मिः॒ सूर्य॑रश्मि॒र्॒. हरि॑केशः पु॒रस्ता᳚त् पु॒रस्ता॒ द्धरि॑केशः॒ सूर्य॑रश्मिः॒ सूर्य॑रश्मि॒र्॒. हरि॑केशः पु॒रस्ता᳚त् । \newline
7. सूर्य॑रश्मि॒रिति॒ सूर्य॑ - र॒श्मिः॒ । \newline
8. हरि॑केशः पु॒रस्ता᳚त् पु॒रस्ता॒ द्धरि॑केशो॒ हरि॑केशः पु॒रस्ता᳚थ् सवि॒ता स॑वि॒ता पु॒रस्ता॒ द्धरि॑केशो॒ हरि॑केशः पु॒रस्ता᳚थ् सवि॒ता । \newline
9. हरि॑केश॒ इति॒ हरि॑ - के॒शः॒ । \newline
10. पु॒रस्ता᳚थ् सवि॒ता स॑वि॒ता पु॒रस्ता᳚त् पु॒रस्ता᳚थ् सवि॒ता ज्योति॒र् ज्योतिः॑ सवि॒ता पु॒रस्ता᳚त् पु॒रस्ता᳚थ् सवि॒ता ज्योतिः॑ । \newline
11. स॒वि॒ता ज्योति॒र् ज्योतिः॑ सवि॒ता स॑वि॒ता ज्योति॒ रुदुज् ज्योतिः॑ सवि॒ता स॑वि॒ता ज्योति॒रुत् । \newline
12. ज्योति॒ रुदुज् ज्योति॒र् ज्योति॒ रुद॑याꣳ अयाꣳ॒॒ उज् ज्योति॒र् ज्योति॒ रुद॑यान् । \newline
13. उद॑याꣳ अयाꣳ॒॒ उदुद॑याꣳ॒॒ अज॑स्र॒ मज॑स्र मयाꣳ॒॒ उदुद॑याꣳ॒॒ अज॑स्रम् । \newline
14. अ॒याꣳ॒॒ अज॑स्र॒ मज॑स्र मयाꣳ अयाꣳ॒॒ अज॑स्रम् । \newline
15. अज॑स्र॒मित्यज॑स्रम् । \newline
16. तस्य॑ पू॒षा पू॒षा तस्य॒ तस्य॑ पू॒षा प्र॑स॒वम् प्र॑स॒वम् पू॒षा तस्य॒ तस्य॑ पू॒षा प्र॑स॒वम् । \newline
17. पू॒षा प्र॑स॒वम् प्र॑स॒वम् पू॒षा पू॒षा प्र॑स॒वं ॅया॑ति याति प्रस॒वम् पू॒षा पू॒षा प्र॑स॒वं ॅया॑ति । \newline
18. प्र॒स॒वं ॅया॑ति याति प्रस॒वम् प्र॑स॒वं ॅया॑ति दे॒वो दे॒वो या॑ति प्रस॒वम् प्र॑स॒वं ॅया॑ति दे॒वः । \newline
19. प्र॒स॒वमिति॑ प्र - स॒वम् । \newline
20. या॒ति॒ दे॒वो दे॒वो या॑ति याति दे॒वः सं॒पश्यन्᳚ थ्सं॒पश्य॑न् दे॒वो या॑ति याति दे॒वः सं॒पश्यन्न्॑ । \newline
21. दे॒वः सं॒पश्यन्᳚ थ्सं॒पश्य॑न् दे॒वो दे॒वः सं॒पश्य॒न्॒. विश्वा॒ विश्वा॑ सं॒पश्य॑न् दे॒वो दे॒वः सं॒पश्य॒न्॒. विश्वा᳚ । \newline
22. सं॒पश्य॒न्॒. विश्वा॒ विश्वा॑ सं॒पश्यन्᳚ थ्सं॒पश्य॒न्॒. विश्वा॒ भुव॑नानि॒ भुव॑नानि॒ विश्वा॑ सं॒पश्यन्᳚ थ्सं॒पश्य॒न्॒. विश्वा॒ भुव॑नानि । \newline
23. सं॒पश्य॒न्निति॑ सं - पश्यन्न्॑ । \newline
24. विश्वा॒ भुव॑नानि॒ भुव॑नानि॒ विश्वा॒ विश्वा॒ भुव॑नानि गो॒पा गो॒पा भुव॑नानि॒ विश्वा॒ विश्वा॒ भुव॑नानि गो॒पाः । \newline
25. भुव॑नानि गो॒पा गो॒पा भुव॑नानि॒ भुव॑नानि गो॒पाः । \newline
26. गो॒पा इति॑ गो - पाः । \newline
27. दे॒वा दे॒वेभ्यो॑ दे॒वेभ्यो॑ दे॒वा दे॒वा दे॒वेभ्यो॑ अद्ध्व॒र्यन्तो॑ अद्ध्व॒र्यन्तो॑ दे॒वेभ्यो॑ दे॒वा दे॒वा दे॒वेभ्यो॑ अद्ध्व॒र्यन्तः॑ । \newline
28. दे॒वेभ्यो॑ अद्ध्व॒र्यन्तो॑ अद्ध्व॒र्यन्तो॑ दे॒वेभ्यो॑ दे॒वेभ्यो॑ अद्ध्व॒र्यन्तो॑ अस्थु रस्थु रद्ध्व॒र्यन्तो॑ दे॒वेभ्यो॑ दे॒वेभ्यो॑ अद्ध्व॒र्यन्तो॑ अस्थुः । \newline
29. अ॒द्ध्व॒र्यन्तो॑ अस्थु रस्थु रद्ध्व॒र्यन्तो॑ अद्ध्व॒र्यन्तो॑ अस्थुर् वी॒तं ॅवी॒त म॑स्थु रद्ध्व॒र्यन्तो॑ 
अद्ध्व॒र्यन्तो॑ अस्थुर् वी॒तम् । \newline
30. अ॒स्थु॒र् वी॒तं ॅवी॒त म॑स्थु रस्थुर् वी॒तꣳ श॑मि॒त्रे श॑मि॒त्रे वी॒त म॑स्थु रस्थुर् वी॒तꣳ श॑मि॒त्रे । \newline
31. वी॒तꣳ श॑मि॒त्रे श॑मि॒त्रे वी॒तं ॅवी॒तꣳ श॑मि॒त्रे श॑मि॒ता श॑मि॒ता श॑मि॒त्रे वी॒तं ॅवी॒तꣳ श॑मि॒त्रे श॑मि॒ता । \newline
32. श॒मि॒त्रे श॑मि॒ता श॑मि॒ता श॑मि॒त्रे श॑मि॒त्रे श॑मि॒ता य॒जद्ध्यै॑ य॒जद्ध्यै॑ शमि॒ता श॑मि॒त्रे श॑मि॒त्रे श॑मि॒ता य॒जद्ध्यै᳚ । \newline
33. श॒मि॒ता य॒जद्ध्यै॑ य॒जद्ध्यै॑ शमि॒ता श॑मि॒ता य॒जद्ध्यै᳚ । \newline
34. य॒जद्ध्या॒ इति॑ य॒जद्ध्यै᳚ । \newline
35. तु॒रीयो॑ य॒ज्ञो य॒ज्ञ् स्तु॒रीय॑ स्तु॒रीयो॑ य॒ज्ञो यत्र॒ यत्र॑ य॒ज्ञ् स्तु॒रीय॑ स्तु॒रीयो॑ य॒ज्ञो यत्र॑ । \newline
36. य॒ज्ञो यत्र॒ यत्र॑ य॒ज्ञो य॒ज्ञो यत्र॑ ह॒व्यꣳ ह॒व्यं ॅयत्र॑ य॒ज्ञो य॒ज्ञो यत्र॑ ह॒व्यम् । \newline
37. यत्र॑ ह॒व्यꣳ ह॒व्यं ॅयत्र॒ यत्र॑ ह॒व्य मेत्येति॑ ह॒व्यं ॅयत्र॒ यत्र॑ ह॒व्य मेति॑ । \newline
38. ह॒व्य मेत्येति॑ ह॒व्यꣳ ह॒व्य मेति॒ तत॒ स्तत॒ एति॑ ह॒व्यꣳ ह॒व्य मेति॒ ततः॑ । \newline
39. एति॒ तत॒ स्तत॒ एत्येति॒ ततः॑ पाव॒काः पा॑व॒का स्तत॒ एत्येति॒ ततः॑ पाव॒काः । \newline
40. ततः॑ पाव॒काः पा॑व॒का स्तत॒ स्ततः॑ पाव॒का आ॒शिष॑ आ॒शिषः॑ पाव॒का स्तत॒ स्ततः॑ पाव॒का आ॒शिषः॑ । \newline
41. पा॒व॒का आ॒शिष॑ आ॒शिषः॑ पाव॒काः पा॑व॒का आ॒शिषो॑ नो न आ॒शिषः॑ पाव॒काः पा॑व॒का आ॒शिषो॑ नः । \newline
42. आ॒शिषो॑ नो न आ॒शिष॑ आ॒शिषो॑ नो जुषन्ताम् जुषन्ताम् न आ॒शिष॑ आ॒शिषो॑ नो जुषन्ताम् । \newline
43. आ॒शिष॒ इत्या᳚ - शिषः॑ । \newline
44. नो॒ जु॒ष॒न्ता॒म् जु॒ष॒न्ता॒म् नो॒ नो॒ जु॒ष॒न्ता॒म् । \newline
45. जु॒ष॒न्ता॒मिति॑ जुषन्ताम् । \newline
46. वि॒मान॑ ए॒ष ए॒ष वि॒मानो॑ वि॒मान॑ ए॒ष दि॒वो दि॒व ए॒ष वि॒मानो॑ वि॒मान॑ ए॒ष दि॒वः । \newline
47. वि॒मान॒ इति॑ वि - मानः॑ । \newline
48. ए॒ष दि॒वो दि॒व ए॒ष ए॒ष दि॒वो मद्ध्ये॒ मद्ध्ये॑ दि॒व ए॒ष ए॒ष दि॒वो मद्ध्ये᳚ । \newline
49. दि॒वो मद्ध्ये॒ मद्ध्ये॑ दि॒वो दि॒वो मद्ध्य॑ आस्त आस्ते॒ मद्ध्ये॑ दि॒वो दि॒वो मद्ध्य॑ आस्ते । \newline
50. मद्ध्य॑ आस्त आस्ते॒ मद्ध्ये॒ मद्ध्य॑ आस्त आपप्रि॒वाना॑ पप्रि॒वाना᳚स्ते॒ मद्ध्ये॒ मद्ध्य॑ आस्त आपप्रि॒वान् । \newline
51. आ॒स्त॒ आ॒प॒प्रि॒वाना॑ पप्रि॒वाना᳚स्त आस्त आपप्रि॒वान् रोद॑सी॒ रोद॑सी आपप्रि॒वाना᳚ स्त आस्त आपप्रि॒वान् रोद॑सी । \newline
52. आ॒प॒प्रि॒वान् रोद॑सी॒ रोद॑सी आपप्रि॒वाना॑ पप्रि॒वान् रोद॑सी अ॒न्तरि॑क्ष म॒न्तरि॑क्षꣳ॒॒ रोद॑सी 
आपप्रि॒वाना॑ पप्रि॒वान् रोद॑सी अ॒न्तरि॑क्षम् । \newline
53. आ॒प॒प्रि॒वानित्या᳚ - प॒प्रि॒वान् । \newline
54. रोद॑सी अ॒न्तरि॑क्ष म॒न्तरि॑क्षꣳ॒॒ रोद॑सी॒ रोद॑सी अ॒न्तरि॑क्षम् । \newline
55. रोद॑सी॒ इति॒ रोद॑सी । \newline
56. अ॒न्तरि॑क्ष॒मित्य॒न्तरि॑क्षम् । \newline
57. स वि॒श्वाची᳚र् वि॒श्वाचीः॒ स स वि॒श्वाची॑ र॒भ्य॑भि वि॒श्वाचीः॒ स स वि॒श्वाची॑ र॒भि । \newline
58. वि॒श्वाची॑ र॒भ्य॑भि वि॒श्वाची᳚र् वि॒श्वाची॑ र॒भि च॑ष्टे चष्टे अ॒भि वि॒श्वाची᳚र् वि॒श्वाची॑ र॒भि च॑ष्टे । \newline
59. अ॒भि च॑ष्टे चष्टे अ॒भ्य॑भि च॑ष्टे घृ॒ताची᳚र् घृ॒ताची᳚ श्चष्टे अ॒भ्य॑भि च॑ष्टे घृ॒ताचीः᳚ । \newline
\pagebreak
\markright{ TS 4.6.3.4  \hfill https://www.vedavms.in \hfill}

\section{ TS 4.6.3.4 }

\textbf{TS 4.6.3.4 } \newline
\textbf{Samhita Paata} \newline

च॑ष्टे घृ॒ताची॑रन्त॒रा पूर्व॒मप॑रं च के॒तुं ॥ उ॒क्षा स॑मु॒द्रो अ॑रु॒णः सु॑प॒र्णः पूर्व॑स्य॒ योनिं॑ पि॒तुरा वि॑वेश । मद्ध्ये॑ दि॒वो निहि॑तः॒ पृश्नि॒रश्मा॒ वि च॑क्रमे॒ रज॑सः पा॒त्यन्तौ᳚ ॥ इन्द्रं॒ ॅविश्वा॑ अवीवृधन्थ् समु॒द्रव्य॑चसं॒ गिरः॑ ।र॒थीत॑मꣳ रथी॒नां ॅवाजा॑नाꣳ॒॒ सत्प॑तिं॒ पतिं᳚ ॥ सु॒म्न॒हूर्य॒ज्ञो दे॒वाꣳ आ च॑ वक्ष॒द्यक्ष॑द॒ग्निर्दे॒वो दे॒वाꣳ आ च॑ वक्षत् ॥ वाज॑स्य ( ) मा प्रस॒वेनो᳚द्-ग्रा॒भेणो-द॑ग्रभीत् । अथा॑ स॒पत्नाꣳ॒॒ इन्द्रो॑ मे निग्रा॒भेणाध॑राꣳ अकः ॥ उ॒द्ग्रा॒भं च॑ निग्रा॒भं च॒ ब्रह्म॑ दे॒वा अ॑वीवृधन्न् । अथा॑ स॒पत्ना॑निन्द्रा॒ग्नी मे॑ विषू॒चीना॒न् व्य॑स्यतां ॥ \newline

\textbf{Pada Paata} \newline

च॒ष्टे॒ । घृ॒ताचीः᳚ । अ॒न्त॒रा । पूर्व᳚म् । अप॑रम् । च॒ । के॒तुम् ॥ उ॒क्षा । स॒मु॒द्रः । अ॒रु॒णः । सु॒प॒र्ण इति॑ सु - प॒र्णः । पूर्व॑स्य । योनि᳚म् । पि॒तुः । एति॑ । वि॒वे॒श॒ ॥ मद्ध्ये᳚ । दि॒वः । निहि॑त॒ इति॒ नि - हि॒तः॒ । पृश्निः॑ । अश्मा᳚ । वीति॑ । च॒क्र॒मे॒ । रज॑सः । पा॒ति॒ । अन्तौ᳚ ॥ इन्द्र᳚म् । विश्वाः᳚ । अ॒वी॒वृ॒ध॒न्न् । स॒मु॒द्रव्य॑चस॒मिति॑ समु॒द्र-व्य॒च॒स॒म् । गिरः॑ ॥ र॒थीत॑म॒मिति॑ र॒थि - त॒म॒म् । र॒थी॒नाम् । वाजा॑नाम् । सत्प॑ति॒मिति॒ सत् - प॒ति॒म् । पति᳚म् ॥ सु॒म्न॒हूरिति॑ सुम्न - हूः । य॒ज्ञ्ः । दे॒वान् । एति॑ । च॒ । व॒क्ष॒त् । यक्ष॑त् । अ॒ग्निः । दे॒वः । दे॒वान् । एति॑ । च॒ । व॒क्ष॒त् ॥ वाज॑स्य ( ) । मा॒ । प्र॒स॒वेनेति॑ प्र - स॒वेन॑ । उ॒द्ग्रा॒भेणेत्यु॑त् - ग्रा॒भेण॑ । उदिति॑ । अ॒ग्र॒भी॒त् ॥ अथ॑ । स॒पत्नान्॑ । इन्द्रः॑ । मे॒ । नि॒ग्रा॒भेणेति॑ नि - ग्रा॒भेण॑ । अध॑रान् । अ॒कः॒ ॥ उ॒द्ग्रा॒भमित्यु॑त् - ग्रा॒भम् । च॒ । नि॒ग्रा॒भमिति॑ नि - ग्रा॒भम् । च॒ । ब्रह्म॑ । दे॒वाः । अ॒वी॒वृ॒ध॒न्न् ॥ अथ॑ । स॒पत्नान्॑ । इ॒न्द्रा॒ग्नी इती᳚न्द्र - अ॒ग्नी । मे॒ । वि॒षू॒चीनान्॑ । वीति॑ । अ॒स्य॒ता॒म् ॥  \newline


\textbf{Krama Paata} \newline

च॒ष्टे॒ घृ॒ताचीः᳚ । घृ॒ताची॑रन्त॒रा । अ॒न्त॒रा पूर्व᳚म् । पूर्व॒मप॑रम् । अप॑रम् च । च॒ के॒तुम् । के॒तुमिति॑ के॒तुम् ॥ उ॒क्षा स॑मु॒द्रः । स॒मु॒द्रो अ॑रु॒णः । अ॒रु॒णः सु॑प॒र्णः । सु॒प॒र्णः पूर्व॑स्य । सु॒प॒र्ण इति॑ सु - प॒र्णः । पूर्व॑स्य॒ योनि᳚म् । योनि॑म् पि॒तुः । पि॒तुरा । आ वि॑वेश । वि॒वे॒शेति॑ विवेश ॥ मद्ध्ये॑ दि॒वः । दि॒वो निहि॑तः । निहि॑तः॒ पृश्ञिः॑ । निहि॑त॒ इति॒ नि - हि॒तः॒ । पृश्ञि॒रश्मा᳚ । अश्मा॒ वि । वि च॑क्रमे । च॒क्र॒मे॒ रज॑सः । रज॑सः पाति । पा॒त्यन्तौ᳚ । अन्ता॒वित्यन्तौ᳚ ॥ इन्द्रं॒ ॅविश्वाः᳚ । विश्वा॑ अवीवृधन्न् । अ॒वी॒वृ॒ध॒न्थ् स॒मु॒द्रव्य॑चसम् । स॒मु॒द्रव्य॑चस॒ङ्गिरः॑ । स॒मु॒द्रव्य॑चस॒मिति॑ समु॒द्र - व्य॒च॒स॒म् । गिर॒ इति॒ गिरः॑ ॥ र॒थीत॑मꣳ रथी॒नाम् । र॒थीत॑म॒मिति॑ र॒थि - त॒म॒म् । र॒थी॒नां ॅवाजा॑नाम् । वाजा॑नाꣳ॒॒ सत्प॑तिम् । सत्प॑ति॒म् पति᳚म् । सत्प॑ति॒मिति॒ सत् - प॒ति॒म् । पति॒मिति॒ पति᳚म् ॥ सु॒म्न॒हूर् य॒ज्ञ्ः । सु॒म्न॒हूरिति॑ सुम्न - हूः । य॒ज्ञो दे॒वान् । दे॒वाꣳ आ । आ च॑ । च॒ व॒क्ष॒त्॒ । व॒क्ष॒द् यक्ष॑त् । यक्ष॑द॒ग्निः । अ॒ग्निर् दे॒वः । दे॒वो दे॒वान् । दे॒वाꣳ आ । आ च॑ । च॒ व॒क्ष॒त्॒ । व॒क्ष॒दिति॑ वक्षत् ॥ वाज॑स्य मा ( ) । मा॒ प्र॒स॒वेन॑ । प्र॒स॒वेनो᳚द्ग्रा॒भेण॑ । प्र॒स॒वेनेति॑ प्र - स॒वेन॑ । उ॒द्ग्रा॒भेणोत् । उ॒द्ग्रा॒भेणेत्यु॑त् - ग्रा॒भेण॑ । उद॑ग्रभीत् । अ॒ग्र॒भी॒दित्य॑ग्रभीत् ॥ अथा॑ स॒पत्नान्॑ । स॒पत्नाꣳ॒॒ इन्द्रः॑ । इन्द्रो॑ मे । मे॒ नि॒ग्रा॒भेण॑ । नि॒ग्रा॒भेणाध॑रान् । नि॒ग्रा॒भेणेति॑ नि - ग्रा॒भेण॑ । अध॑राꣳ अकः । अ॒क॒रित्य॑कः ॥ उ॒द्ग्रा॒भम् च॑ । उ॒द्ग्रा॒भमित्यु॑त् - ग्रा॒भम् । च॒ नि॒ग्रा॒भम् । नि॒ग्रा॒भम् च॑ । नि॒ग्रा॒भमिति॑ नि - ग्रा॒भम् । च॒ ब्रह्म॑ । ब्रह्म॑ दे॒वाः । दे॒वा अ॑वीवृधन्न् । अ॒वी॒वृ॒ध॒न्नित्य॑वीवृधन्न् ॥ अथा॑ स॒पत्नान्॑ । स॒पत्ना॑निन्द्रा॒ग्नी । इ॒न्द्रा॒ग्नी मे᳚ । इ॒न्द्रा॒ग्नी इती᳚न्द्र - अ॒ग्नी । मे॒ वि॒षू॒चीनान्॑ । वि॒षू॒चीना॒न्॒. वि । व्य॑स्यताम् । अ॒स्य॒ता॒मित्य॑स्यताम् । \newline

\textbf{Jatai Paata} \newline

1. च॒ष्टे॒ घृ॒ताची᳚र् घृ॒ताची᳚ श्चष्टे चष्टे घृ॒ताचीः᳚ । \newline
2. घृ॒ताची॑ रन्त॒रा ऽन्त॒रा घृ॒ताची᳚र् घृ॒ताची॑ रन्त॒रा । \newline
3. अ॒न्त॒रा पूर्व॒म् पूर्व॑ मन्त॒रा ऽन्त॒रा पूर्व᳚म् । \newline
4. पूर्व॒ मप॑र॒ मप॑र॒म् पूर्व॒म् पूर्व॒ मप॑रम् । \newline
5. अप॑रम् च॒ चाप॑र॒ मप॑रम् च । \newline
6. च॒ के॒तुम् के॒तुम् च॑ च के॒तुम् । \newline
7. के॒तुमिति॑ के॒तुम् । \newline
8. उ॒क्षा स॑मु॒द्रः स॑मु॒द्र उ॒क्षोक्षा स॑मु॒द्रः । \newline
9. स॒मु॒द्रो अ॑रु॒णो अ॑रु॒णः स॑मु॒द्रः स॑मु॒द्रो अ॑रु॒णः । \newline
10. अ॒रु॒णः सु॑प॒र्णः सु॑प॒र्णो अ॑रु॒णो अ॑रु॒णः सु॑प॒र्णः । \newline
11. सु॒प॒र्णः पूर्व॑स्य॒ पूर्व॑स्य सुप॒र्णः सु॑प॒र्णः पूर्व॑स्य । \newline
12. सु॒प॒र्ण इति॑ सु - प॒र्णः । \newline
13. पूर्व॑स्य॒ योनिं॒ ॅयोनि॒म् पूर्व॑स्य॒ पूर्व॑स्य॒ योनि᳚म् । \newline
14. योनि॑म् पि॒तुः पि॒तुर् योनिं॒ ॅयोनि॑म् पि॒तुः । \newline
15. पि॒तुरा पि॒तुः पि॒तुरा । \newline
16. आ वि॑वेश विवे॒शा वि॑वेश । \newline
17. वि॒वे॒शेति॑ विवेश । \newline
18. मद्ध्ये॑ दि॒वो दि॒वो मद्ध्ये॒ मद्ध्ये॑ दि॒वः । \newline
19. दि॒वो निहि॑तो॒ निहि॑तो दि॒वो दि॒वो निहि॑तः । \newline
20. निहि॑तः॒ पृश्ञिः॒ पृश्ञि॒र् निहि॑तो॒ निहि॑तः॒ पृश्ञिः॑ । \newline
21. निहि॑त॒ इति॒ नि - हि॒तः॒ । \newline
22. पृश्ञि॒ रश्मा ऽश्मा॒ पृश्ञिः॒ पृश्ञि॒ रश्मा᳚ । \newline
23. अश्मा॒ वि व्यश्मा ऽश्मा॒ वि । \newline
24. वि च॑क्रमे चक्रमे॒ वि वि च॑क्रमे । \newline
25. च॒क्र॒मे॒ रज॑सो॒ रज॑स श्चक्रमे चक्रमे॒ रज॑सः । \newline
26. रज॑सः पाति पाति॒ रज॑सो॒ रज॑सः पाति । \newline
27. पा॒त्यन्ता॒ वन्तौ॑ पाति पा॒त्यन्तौ᳚ । \newline
28. अन्ता॒वित्यन्तौ᳚ । \newline
29. इन्द्रं॒ ॅविश्वा॒ विश्वा॒ इन्द्र॒ मिन्द्रं॒ ॅविश्वाः᳚ । \newline
30. विश्वा॑ अवीवृधन् नवीवृध॒न्॒. विश्वा॒ विश्वा॑ अवीवृधन्न् । \newline
31. अ॒वी॒वृ॒ध॒न् थ्स॒मु॒द्रव्य॑चसꣳ समु॒द्रव्य॑चस मवीवृधन् नवीवृधन् थ्समु॒द्रव्य॑चसम् । \newline
32. स॒मु॒द्रव्य॑चस॒म् गिरो॒ गिरः॑ समु॒द्रव्य॑चसꣳ समु॒द्रव्य॑चस॒म् गिरः॑ । \newline
33. स॒मु॒द्रव्य॑चस॒मिति॑ समु॒द्र - व्य॒च॒स॒म् । \newline
34. गिर॒ इति॒ गिरः॑ । \newline
35. र॒थीत॑मꣳ रथी॒नाꣳ र॑थी॒नाꣳ र॒थीत॑मꣳ र॒थीत॑मꣳ रथी॒नाम् । \newline
36. र॒थीत॑म॒मिति॑ र॒थि - त॒म॒म् । \newline
37. र॒थी॒नां ॅवाजा॑नां॒ ॅवाजा॑नाꣳ रथी॒नाꣳ र॑थी॒नां ॅवाजा॑नाम् । \newline
38. वाजा॑नाꣳ॒॒ सत्प॑तिꣳ॒॒ सत्प॑तिं॒ ॅवाजा॑नां॒ ॅवाजा॑नाꣳ॒॒ सत्प॑तिम् । \newline
39. सत्प॑ति॒म् पति॒म् पतिꣳ॒॒ सत्प॑तिꣳ॒॒ सत्प॑ति॒म् पति᳚म् । \newline
40. सत्प॑ति॒मिति॒ सत् - प॒ति॒म् । \newline
41. पति॒मिति॒ पति᳚म् । \newline
42. सु॒म्न॒हूर् य॒ज्ञो य॒ज्ञ्ः सु॑म्न॒हूः सु॑म्न॒हूर् य॒ज्ञ्ः । \newline
43. सु॒म्न॒हूरिति॑ सुम्न - हूः । \newline
44. य॒ज्ञो दे॒वान् दे॒वान्. य॒ज्ञो य॒ज्ञो दे॒वान् । \newline
45. दे॒वाꣳ आ दे॒वान् दे॒वाꣳ आ । \newline
46. आ च॒ चा च॑ । \newline
47. च॒ व॒क्ष॒द् व॒क्ष॒च् च॒ च॒ व॒क्ष॒त् । \newline
48. व॒क्ष॒द् यक्ष॒द् यक्ष॑द् वक्षद् वक्ष॒द् यक्ष॑त् । \newline
49. यक्ष॑ द॒ग्नि र॒ग्निर् यक्ष॒द् यक्ष॑ द॒ग्निः । \newline
50. अ॒ग्निर् दे॒वो दे॒वो अ॒ग्नि र॒ग्निर् दे॒वः । \newline
51. दे॒वो दे॒वान् दे॒वान् दे॒वो दे॒वो दे॒वान् । \newline
52. दे॒वाꣳ आ दे॒वान् दे॒वाꣳ आ । \newline
53. आ च॒ चा च॑ । \newline
54. च॒ व॒क्ष॒द् व॒क्ष॒च् च॒ च॒ व॒क्ष॒त् । \newline
55. व॒क्ष॒दिति॑ वक्षत् । \newline
56. वाज॑स्य मा मा॒ वाज॑स्य॒ वाज॑स्य मा । \newline
57. मा॒ प्र॒स॒वेन॑ प्रस॒वेन॑ मा मा प्रस॒वेन॑ । \newline
58. प्र॒स॒वे नो᳚द्ग्रा॒भे णो᳚द्ग्रा॒भेण॑ प्रस॒वेन॑ प्रस॒वे नो᳚द्ग्रा॒भेण॑ । \newline
59. प्र॒स॒वेनेति॑ प्र - स॒वेन॑ । \newline
60. उ॒द्ग्रा॒भे णोदुदु॑द्ग्रा॒भे णो᳚द्ग्रा॒भेणोत् । \newline
61. उ॒द्ग्रा॒भेणेत्यु॑त् - ग्रा॒भेण॑ । \newline
62. उद॑ग्रभी दग्रभी॒ दुदु द॑ग्रभीत् । \newline
63. अ॒ग्र॒भी॒दित्य॑ग्रभीत् । \newline
64. अथा॑ स॒पत्ना᳚न् थ्स॒पत्नाꣳ॒॒ अथाथा॑ स॒पत्नान्॑ । \newline
65. स॒पत्नाꣳ॒॒ इन्द्र॒ इन्द्रः॑ स॒पत्ना᳚न् थ्स॒पत्नाꣳ॒॒ इन्द्रः॑ । \newline
66. इन्द्रो॑ मे म॒ इन्द्र॒ इन्द्रो॑ मे । \newline
67. मे॒ नि॒ग्रा॒भेण॑ निग्रा॒भेण॑ मे मे निग्रा॒भेण॑ । \newline
68. नि॒ग्रा॒भेणा ध॑राꣳ॒॒ अध॑रान् निग्रा॒भेण॑ निग्रा॒भेणा ध॑रान् । \newline
69. नि॒ग्रा॒भेणेति॑ नि - ग्रा॒भेण॑ । \newline
70. अध॑राꣳ अक-रक॒-रध॑राꣳ॒॒ अध॑राꣳ अकः । \newline
71. अ॒क॒रित्य॑कः । \newline
72. उ॒द्ग्रा॒भम् च॑ चोद्ग्रा॒भ मु॑द्ग्रा॒भम् च॑ । \newline
73. उ॒द्ग्रा॒भमित्यु॑त् - ग्रा॒भम् । \newline
74. च॒ नि॒ग्रा॒भम् नि॑ग्रा॒भम् च॑ च निग्रा॒भम् । \newline
75. नि॒ग्रा॒भम् च॑ च निग्रा॒भम् नि॑ग्रा॒भम् च॑ । \newline
76. नि॒ग्रा॒भमिति॑ नि - ग्रा॒भम् । \newline
77. च॒ ब्रह्म॒ ब्रह्म॑ च च॒ ब्रह्म॑ । \newline
78. ब्रह्म॑ दे॒वा दे॒वा ब्रह्म॒ ब्रह्म॑ दे॒वाः । \newline
79. दे॒वा अ॑वीवृधन् नवीवृधन् दे॒वा दे॒वा अ॑वीवृधन्न् । \newline
80. अ॒वी॒वृ॒ध॒न्नित्य॑वीवृधन्न् । \newline
81. अथा॑ स॒पत्ना᳚न् थ्स॒पत्ना॒ नथाथा॑ स॒पत्नान्॑ । \newline
82. स॒पत्ना॑ निन्द्रा॒ग्नी इ॑न्द्रा॒ग्नी स॒पत्ना᳚न् थ्स॒पत्ना॑ निन्द्रा॒ग्नी । \newline
83. इ॒न्द्रा॒ग्नी मे॑ म इन्द्रा॒ग्नी इ॑न्द्रा॒ग्नी मे᳚ । \newline
84. इ॒न्द्रा॒ग्नी इती᳚न्द्र - अ॒ग्नी । \newline
85. मे॒ वि॒षू॒चीनान्॑. विषू॒चीना᳚न् मे मे विषू॒चीनान्॑ । \newline
86. वि॒षू॒चीना॒न्॒. वि वि वि॑षू॒चीनान्॑. विषू॒चीना॒न्॒. वि । \newline
87. व्य॑स्यता मस्यतां॒ ॅवि व्य॑स्यताम् । \newline
88. अ॒स्य॒ता॒मित्य॑स्यताम् । \newline

\textbf{Ghana Paata } \newline

1. च॒ष्टे॒ घृ॒ताची᳚र् घृ॒ताची᳚ श्चष्टे चष्टे घृ॒ताची॑ रन्त॒रा ऽन्त॒रा घृ॒ताची᳚ श्चष्टे चष्टे घृ॒ताची॑ रन्त॒रा । \newline
2. घृ॒ताची॑ रन्त॒रा ऽन्त॒रा घृ॒ताची᳚र् घृ॒ताची॑ रन्त॒रा पूर्व॒म् पूर्व॑ मन्त॒रा घृ॒ताची᳚र् घृ॒ताची॑ रन्त॒रा पूर्व᳚म् । \newline
3. अ॒न्त॒रा पूर्व॒म् पूर्व॑ मन्त॒रा ऽन्त॒रा पूर्व॒ मप॑र॒ मप॑र॒म् पूर्व॑ मन्त॒रा ऽन्त॒रा पूर्व॒ मप॑रम् । \newline
4. पूर्व॒ मप॑र॒ मप॑र॒म् पूर्व॒म् पूर्व॒ मप॑रम् च॒ चाप॑र॒म् पूर्व॒म् पूर्व॒ मप॑रम् च । \newline
5. अप॑रम् च॒ चाप॑र॒ मप॑रम् च के॒तुम् के॒तुम् चाप॑र॒ मप॑रम् च के॒तुम् । \newline
6. च॒ के॒तुम् के॒तुम् च॑ च के॒तुम् । \newline
7. के॒तुमिति॑ के॒तुम् । \newline
8. उ॒क्षा स॑मु॒द्रः स॑मु॒द्र उ॒क्षोक्षा स॑मु॒द्रो अ॑रु॒णो अ॑रु॒णः स॑मु॒द्र उ॒क्षोक्षा स॑मु॒द्रो अ॑रु॒णः । \newline
9. स॒मु॒द्रो अ॑रु॒णो अ॑रु॒णः स॑मु॒द्रः स॑मु॒द्रो अ॑रु॒णः सु॑प॒र्णः सु॑प॒र्णो अ॑रु॒णः स॑मु॒द्रः स॑मु॒द्रो अ॑रु॒णः सु॑प॒र्णः । \newline
10. अ॒रु॒णः सु॑प॒र्णः सु॑प॒र्णो अ॑रु॒णो अ॑रु॒णः सु॑प॒र्णः पूर्व॑स्य॒ पूर्व॑स्य सुप॒र्णो अ॑रु॒णो अ॑रु॒णः सु॑प॒र्णः पूर्व॑स्य । \newline
11. सु॒प॒र्णः पूर्व॑स्य॒ पूर्व॑स्य सुप॒र्णः सु॑प॒र्णः पूर्व॑स्य॒ योनिं॒ ॅयोनि॒म् पूर्व॑स्य सुप॒र्णः सु॑प॒र्णः पूर्व॑स्य॒ योनि᳚म् । \newline
12. सु॒प॒र्ण इति॑ सु - प॒र्णः । \newline
13. पूर्व॑स्य॒ योनिं॒ ॅयोनि॒म् पूर्व॑स्य॒ पूर्व॑स्य॒ योनि॑म् पि॒तुः पि॒तुर् योनि॒म् पूर्व॑स्य॒ पूर्व॑स्य॒ योनि॑म् पि॒तुः । \newline
14. योनि॑म् पि॒तुः पि॒तुर् योनिं॒ ॅयोनि॑म् पि॒तुरा पि॒तुर् योनिं॒ ॅयोनि॑म् पि॒तुरा । \newline
15. पि॒तुरा पि॒तुः पि॒तुरा वि॑वेश विवे॒शा पि॒तुः पि॒तुरा वि॑वेश । \newline
16. आ वि॑वेश विवे॒शा वि॑वेश । \newline
17. वि॒वे॒शेति॑ विवेश । \newline
18. मद्ध्ये॑ दि॒वो दि॒वो मद्ध्ये॒ मद्ध्ये॑ दि॒वो निहि॑तो॒ निहि॑तो दि॒वो मद्ध्ये॒ मद्ध्ये॑ दि॒वो निहि॑तः । \newline
19. दि॒वो निहि॑तो॒ निहि॑तो दि॒वो दि॒वो निहि॑तः॒ पृश्ञिः॒ पृश्ञि॒र् निहि॑तो दि॒वो दि॒वो निहि॑तः॒ पृश्ञिः॑ । \newline
20. निहि॑तः॒ पृश्ञिः॒ पृश्ञि॒र् निहि॑तो॒ निहि॑तः॒ पृश्ञि॒ रश्मा ऽश्मा॒ पृश्ञि॒र् निहि॑तो॒ निहि॑तः॒ पृश्ञि॒ रश्मा᳚ । \newline
21. निहि॑त॒ इति॒ नि - हि॒तः॒ । \newline
22. पृश्ञि॒ रश्मा ऽश्मा॒ पृश्ञिः॒ पृश्ञि॒ रश्मा॒ वि व्यश्मा॒ पृश्ञिः॒ पृश्ञि॒ रश्मा॒ वि । \newline
23. अश्मा॒ वि व्यश्मा ऽश्मा॒ वि च॑क्रमे चक्रमे॒ व्यश्मा ऽश्मा॒ वि च॑क्रमे । \newline
24. वि च॑क्रमे चक्रमे॒ वि वि च॑क्रमे॒ रज॑सो॒ रज॑स श्चक्रमे॒ वि वि च॑क्रमे॒ रज॑सः । \newline
25. च॒क्र॒मे॒ रज॑सो॒ रज॑स श्चक्रमे चक्रमे॒ रज॑सः पाति पाति॒ रज॑स श्चक्रमे चक्रमे॒ रज॑सः पाति । \newline
26. रज॑सः पाति पाति॒ रज॑सो॒ रज॑सः पा॒त्यन्ता॒ वन्तौ॑ पाति॒ रज॑सो॒ रज॑सः पा॒त्यन्तौ᳚ । \newline
27. पा॒त्यन्ता॒ वन्तौ॑ पाति पा॒त्यन्तौ᳚ । \newline
28. अन्ता॒वित्यन्तौ᳚ । \newline
29. इन्द्रं॒ ॅविश्वा॒ विश्वा॒ इन्द्र॒ मिन्द्रं॒ ॅविश्वा॑ अवीवृधन् नवीवृध॒न्॒. विश्वा॒ इन्द्र॒ मिन्द्रं॒ ॅविश्वा॑ अवीवृधन्न् । \newline
30. विश्वा॑ अवीवृधन् नवीवृध॒न्॒. विश्वा॒ विश्वा॑ अवीवृधन् थ्समु॒द्रव्य॑चसꣳ समु॒द्रव्य॑चस मवीवृध॒न्॒. विश्वा॒ विश्वा॑ अवीवृधन् थ्समु॒द्रव्य॑चसम् । \newline
31. अ॒वी॒वृ॒ध॒न् थ्स॒मु॒द्रव्य॑चसꣳ समु॒द्रव्य॑चस मवीवृधन् नवीवृधन् थ्समु॒द्रव्य॑चस॒म् गिरो॒ गिरः॑ समु॒द्रव्य॑चस मवीवृधन् नवीवृधन् थ्समु॒द्रव्य॑चस॒म् गिरः॑ । \newline
32. स॒मु॒द्रव्य॑चस॒म् गिरो॒ गिरः॑ समु॒द्रव्य॑चसꣳ समु॒द्रव्य॑चस॒म् गिरः॑ । \newline
33. स॒मु॒द्रव्य॑चस॒मिति॑ समु॒द्र - व्य॒च॒स॒म् । \newline
34. गिर॒ इति॒ गिरः॑ । \newline
35. र॒थीत॑मꣳ रथी॒नाꣳ र॑थी॒नाꣳ र॒थीत॑मꣳ र॒थीत॑मꣳ रथी॒नां ॅवाजा॑नां॒ ॅवाजा॑नाꣳ रथी॒नाꣳ र॒थीत॑मꣳ र॒थीत॑मꣳ रथी॒नां ॅवाजा॑नाम् । \newline
36. र॒थीत॑म॒मिति॑ र॒थि - त॒म॒म् । \newline
37. र॒थी॒नां ॅवाजा॑नां॒ ॅवाजा॑नाꣳ रथी॒नाꣳ र॑थी॒नां ॅवाजा॑नाꣳ॒॒ सत्प॑तिꣳ॒॒ सत्प॑तिं॒ ॅवाजा॑नाꣳ रथी॒नाꣳ र॑थी॒नां ॅवाजा॑नाꣳ॒॒ सत्प॑तिम् । \newline
38. वाजा॑नाꣳ॒॒ सत्प॑तिꣳ॒॒ सत्प॑तिं॒ ॅवाजा॑नां॒ ॅवाजा॑नाꣳ॒॒ सत्प॑ति॒म् पति॒म् पतिꣳ॒॒ सत्प॑तिं॒ ॅवाजा॑नां॒ ॅवाजा॑नाꣳ॒॒ सत्प॑ति॒म् पति᳚म् । \newline
39. सत्प॑ति॒म् पति॒म् पतिꣳ॒॒ सत्प॑तिꣳ॒॒ सत्प॑ति॒म् पति᳚म् । \newline
40. सत्प॑ति॒मिति॒ सत् - प॒ति॒म् । \newline
41. पति॒मिति॒ पति᳚म् । \newline
42. सु॒म्न॒हूर् य॒ज्ञो य॒ज्ञ्ः सु॑म्न॒हूः सु॑म्न॒हूर् य॒ज्ञो दे॒वान् दे॒वान्. य॒ज्ञ्ः सु॑म्न॒हूः सु॑म्न॒हूर् य॒ज्ञो दे॒वान् । \newline
43. सु॒म्न॒हूरिति॑ सुम्न - हूः । \newline
44. य॒ज्ञो दे॒वान् दे॒वान्. य॒ज्ञो य॒ज्ञो दे॒वाꣳ आ दे॒वान्. य॒ज्ञो य॒ज्ञो दे॒वाꣳ आ । \newline
45. दे॒वाꣳ आ दे॒वान् दे॒वाꣳ आ च॒ चा दे॒वान् दे॒वाꣳ आ च॑ । \newline
46. आ च॒ चा च॑ वक्षद् वक्ष॒च् चा च॑ वक्षत् । \newline
47. च॒ व॒क्ष॒द् व॒क्ष॒च् च॒ च॒ व॒क्ष॒द् यक्ष॒द् यक्ष॑द् वक्षच् च च वक्ष॒द् यक्ष॑त् । \newline
48. व॒क्ष॒द् यक्ष॒द् यक्ष॑द् वक्षद् वक्ष॒द् यक्ष॑ द॒ग्नि र॒ग्निर् यक्ष॑द् वक्षद् वक्ष॒द् यक्ष॑ द॒ग्निः । \newline
49. यक्ष॑ द॒ग्नि र॒ग्निर् यक्ष॒द् यक्ष॑ द॒ग्निर् दे॒वो दे॒वो अ॒ग्निर् यक्ष॒द् यक्ष॑ द॒ग्निर् दे॒वः । \newline
50. अ॒ग्निर् दे॒वो दे॒वो अ॒ग्नि र॒ग्निर् दे॒वो दे॒वान् दे॒वान् दे॒वो अ॒ग्नि र॒ग्निर् दे॒वो दे॒वान् । \newline
51. दे॒वो दे॒वान् दे॒वान् दे॒वो दे॒वो दे॒वाꣳ आ दे॒वान् दे॒वो दे॒वो दे॒वाꣳ आ । \newline
52. दे॒वाꣳ आ दे॒वान् दे॒वाꣳ आ च॒ चा दे॒वान् दे॒वाꣳ आ च॑ । \newline
53. आ च॒ चा च॑ वक्षद् वक्ष॒च् चा च॑ वक्षत् । \newline
54. च॒ व॒क्ष॒द् व॒क्ष॒च् च॒ च॒ व॒क्ष॒त् । \newline
55. व॒क्ष॒दिति॑ वक्षत् । \newline
56. वाज॑स्य मा मा॒ वाज॑स्य॒ वाज॑स्य मा प्रस॒वेन॑ प्रस॒वेन॑ मा॒ वाज॑स्य॒ वाज॑स्य मा प्रस॒वेन॑ । \newline
57. मा॒ प्र॒स॒वेन॑ प्रस॒वेन॑ मा मा प्रस॒वे नो᳚द्ग्रा॒भे णो᳚द्ग्रा॒भेण॑ प्रस॒वेन॑ मा मा 
प्रस॒वे नो᳚द्ग्रा॒भेण॑ । \newline
58. प्र॒स॒वे नो᳚द्ग्रा॒भे णो᳚द्ग्रा॒भेण॑ प्रस॒वेन॑ प्रस॒वे नो᳚द्ग्रा॒भे णोदुदु॑द्ग्रा॒भेण॑ प्रस॒वेन॑ 
प्रस॒वे नो᳚द्ग्रा॒भेणोत् । \newline
59. प्र॒स॒वेनेति॑ प्र - स॒वेन॑ । \newline
60. उ॒द्ग्रा॒भे णोदुदु॑द्ग्रा॒भे णो᳚द्ग्रा॒भे णोद॑ग्रभी दग्रभी॒ दुदु॑द्ग्रा॒भे णो᳚द्ग्रा॒भे णोद॑ग्रभीत् । \newline
61. उ॒द्ग्रा॒भेणेत्यु॑त् - ग्रा॒भेण॑ । \newline
62. उद॑ग्रभी दग्रभी॒ दुदु द॑ग्रभीत् । \newline
63. अ॒ग्र॒भी॒दित्य॑ग्रभीत् । \newline
64. अथा॑ स॒पत्ना᳚न् थ्स॒पत्नाꣳ॒॒ अथाथा॑ स॒पत्नाꣳ॒॒ इन्द्र॒ इन्द्रः॑ स॒पत्नाꣳ॒॒ अथाथा॑ स॒पत्नाꣳ॒॒ इन्द्रः॑ । \newline
65. स॒पत्नाꣳ॒॒ इन्द्र॒ इन्द्रः॑ स॒पत्ना᳚न् थ्स॒पत्नाꣳ॒॒ इन्द्रो॑ मे म॒ इन्द्रः॑ स॒पत्ना᳚न् थ्स॒पत्नाꣳ॒॒ इन्द्रो॑ मे । \newline
66. इन्द्रो॑ मे म॒ इन्द्र॒ इन्द्रो॑ मे निग्रा॒भेण॑ निग्रा॒भेण॑ म॒ इन्द्र॒ इन्द्रो॑ मे निग्रा॒भेण॑ । \newline
67. मे॒ नि॒ग्रा॒भेण॑ निग्रा॒भेण॑ मे मे निग्रा॒भेणा ध॑राꣳ॒॒ अध॑रान् निग्रा॒भेण॑ मे मे निग्रा॒भेणा ध॑रान् । \newline
68. नि॒ग्रा॒भेणा ध॑राꣳ॒॒ अध॑रान् निग्रा॒भेण॑ निग्रा॒भेणा ध॑राꣳ अक रक॒ रध॑रान् निग्रा॒भेण॑ निग्रा॒भेणा ध॑राꣳ अकः । \newline
69. नि॒ग्रा॒भेणेति॑ नि - ग्रा॒भेण॑ । \newline
70. अध॑राꣳ अक रक॒ रध॑राꣳ॒॒ अध॑राꣳ अकः । \newline
71. अ॒क॒रित्य॑कः । \newline
72. उ॒द्ग्रा॒भम् च॑ चोद्ग्रा॒भ मु॑द्ग्रा॒भम् च॑ निग्रा॒भम् नि॑ग्रा॒भम् चो᳚द्ग्रा॒भ मु॑द्ग्रा॒भम् च॑ निग्रा॒भम् । \newline
73. उ॒द्ग्रा॒भमित्यु॑त् - ग्रा॒भम् । \newline
74. च॒ नि॒ग्रा॒भम् नि॑ग्रा॒भम् च॑ च निग्रा॒भम् च॑ च निग्रा॒भम् च॑ च निग्रा॒भम् च॑ । \newline
75. नि॒ग्रा॒भम् च॑ च निग्रा॒भम् नि॑ग्रा॒भम् च॒ ब्रह्म॒ ब्रह्म॑ च निग्रा॒भम् नि॑ग्रा॒भम् च॒ ब्रह्म॑ । \newline
76. नि॒ग्रा॒भमिति॑ नि - ग्रा॒भम् । \newline
77. च॒ ब्रह्म॒ ब्रह्म॑ च च॒ ब्रह्म॑ दे॒वा दे॒वा ब्रह्म॑ च च॒ ब्रह्म॑ दे॒वाः । \newline
78. ब्रह्म॑ दे॒वा दे॒वा ब्रह्म॒ ब्रह्म॑ दे॒वा अ॑वीवृधन् नवीवृधन् दे॒वा ब्रह्म॒ ब्रह्म॑ दे॒वा अ॑वीवृधन्न् । \newline
79. दे॒वा अ॑वीवृधन् नवीवृधन् दे॒वा दे॒वा अ॑वीवृधन्न् । \newline
80. अ॒वी॒वृ॒ध॒न्नित्य॑वीवृधन्न् । \newline
81. अथा॑ स॒पत्ना᳚न् थ्स॒पत्ना॒ नथाथा॑ स॒पत्ना॑ निन्द्रा॒ग्नी इ॑न्द्रा॒ग्नी स॒पत्ना॒ नथाथा॑ स॒पत्ना॑ निन्द्रा॒ग्नी । \newline
82. स॒पत्ना॑ निन्द्रा॒ग्नी इ॑न्द्रा॒ग्नी स॒पत्ना᳚न् थ्स॒पत्ना॑ निन्द्रा॒ग्नी मे॑ म इन्द्रा॒ग्नी स॒पत्ना᳚न् थ्स॒पत्ना॑ निन्द्रा॒ग्नी मे᳚ । \newline
83. इ॒न्द्रा॒ग्नी मे॑ म इन्द्रा॒ग्नी इ॑न्द्रा॒ग्नी मे॑ विषू॒चीनान्॑. विषू॒चीना᳚न् म इन्द्रा॒ग्नी इ॑न्द्रा॒ग्नी मे॑ विषू॒चीनान्॑ । \newline
84. इ॒न्द्रा॒ग्नी इती᳚न्द्र - अ॒ग्नी । \newline
85. मे॒ वि॒षू॒चीनान्॑. विषू॒चीना᳚न् मे मे विषू॒चीना॒न्॒. वि वि वि॑षू॒चीना᳚न् मे मे विषू॒चीना॒न्॒. वि । \newline
86. वि॒षू॒चीना॒न्॒. वि वि वि॑षू॒चीनान्॑. विषू॒चीना॒न् व्य॑स्यता मस्यतां॒ ॅवि वि॑षू॒चीनान्॑. विषू॒चीना॒न् व्य॑स्यताम् । \newline
87. व्य॑स्यता मस्यतां॒ ॅवि व्य॑स्यताम् । \newline
88. अ॒स्य॒ता॒मित्य॑स्यताम् । \newline
\pagebreak
\markright{ TS 4.6.4.1  \hfill https://www.vedavms.in \hfill}

\section{ TS 4.6.4.1 }

\textbf{TS 4.6.4.1 } \newline
\textbf{Samhita Paata} \newline

आ॒शुः शिशा॑नो वृष॒भो न यु॒द्ध्मो घ॑नाघ॒नः क्षोभ॑ण-श्चर्.षणी॒नां । सं॒क्रन्द॑नोऽनिमि॒ष ए॑क वी॒रः श॒तꣳ सेना॑ अजयथ् सा॒कमिन्द्रः॑ ॥ सं॒क्रन्द॑नेना निमि॒षेण॑ जि॒ष्णुना॑ युत्का॒रेण॑ दुश्च्यव॒नेन॑ धृ॒ष्णुना᳚ । तदिन्द्रे॑ण जयत॒ तथ् स॑हद्ध्वं॒ ॅयुधो॑ नर॒ इषु॑हस्तेन॒ वृष्णा᳚ ॥ स इषु॑हस्तैः॒ स नि॑षं॒गिभि॑र्व॒शी सꣳस्र॑ष्टा॒ स युध॒ इन्द्रो॑ ग॒णेन॑ । सꣳ॒॒सृ॒ष्ट॒जिथ् सो॑म॒पा बा॑हुश॒र्द्ध्यू᳚र्द्ध्वध॑न्वा॒ प्रति॑हिताभि॒रस्ता᳚ ॥ बृह॑स्पते॒ परि॑ दीया॒- [  ] \newline

\textbf{Pada Paata} \newline

आ॒शुः । शिशा॑नः । वृ॒ष॒भः । न । यु॒द्ध्मः । घ॒ना॒घ॒नः । क्षोभ॑णः । च॒र॒.ष॒णी॒नाम् ॥ सं॒क्रन्द॑न॒ इति॑ सं - क्रन्द॑नः । अ॒नि॒मि॒ष इत्य॑नि - मि॒षः । ए॒क॒वी॒र इत्ये॑क - वी॒रः । श॒तम् । सेनाः᳚ । अ॒ज॒य॒त् । सा॒कम् । इन्द्रः॑ ॥ स॒क्रंन्द॑ने॒नेति॑ सं - क्रन्द॑नेन । अ॒नि॒मि॒षेणेत्य॑नि - मि॒षेण॑ । जि॒ष्णुना᳚ । यु॒त्का॒रेणेति॑ युत् - का॒रेण॑ । दु॒श्च्य॒व॒नेनेति॑ दुः-च्य॒व॒नेन॑ । धृ॒ष्णुना᳚ ॥ तत् । इन्द्रे॑ण । ज॒य॒त॒ । तत् । स॒ह॒द्ध्वं॒ । युधः॑ । न॒रः॒ । इषु॑हस्ते॒नेतीषु॑ - ह॒स्ते॒न॒ । वृष्णा᳚ ॥ सः । इषु॑हस्तै॒रितीषु॑ - ह॒स्तैः॒ । सः । नि॒ष॒ङ्गिभि॒रिति॑ निष॒ङ्गि - भिः॒ । व॒शी । सꣳस्र॒ष्टेति॒ सं - स्र॒ष्टा॒ । सः । युधः॑ । इन्द्रः॑ । ग॒णेन॑ ॥ सꣳ॒॒सृ॒ष्ट॒जिदिति॑ सꣳसृष्ट - जित् । सो॒म॒पा इति॑ सोम - पाः । बा॒हु॒श॒द्‌र्धीति॑ बाहु - श॒द्‌र्धी । ऊ॒द्‌र्ध्वध॒न्वेत्यू॒द्‌र्ध्व - ध॒न्वा॒ । प्रति॑हिताभि॒रिति॒ प्रति॑ - हि॒ता॒भिः॒ । अस्ता᳚ ॥ बृह॑स्पते । परीति॑ । दी॒य॒ ।  \newline


\textbf{Krama Paata} \newline

आ॒शुः शिशा॑नः । शिशा॑नो वृष॒भः । वृ॒ष॒भो न । न यु॒द्ध्मः । यु॒द्ध्मो घ॑नाघ॒नः । घ॒ना॒घ॒नः क्षोभ॑णः । क्षोभ॑णश्चर्.षणी॒नाम् । च॒र्.॒ष॒णी॒नामिति॑ चर्.षणी॒नाम् ॥ स॒ङ्क्रन्द॑नोऽनिमि॒षः । स॒ङ्क्रन्द॑न॒ इति॑ सम् - क्रन्द॑नः । अ॒नि॒मि॒ष ए॑कवी॒रः । अ॒नि॒मि॒ष इत्य॑नि - मि॒षः । ए॒क॒वी॒रः श॒तम् । ए॒क॒वी॒र इत्ये॑क - वी॒रः । श॒तꣳ सेनाः᳚ । सेना॑ अजयत् । अ॒ज॒य॒थ् सा॒कम् । सा॒कमिन्द्रः॑ । इन्द्र॒ इतीन्द्रः॑ ॥ स॒ङ्क्रन्द॑नेनानिमि॒षेण॑ । स॒ङ्क्रन्द॑ने॒नेति॑ सं - क्रन्द॑नेन । अ॒नि॒मि॒षेण॑ जि॒ष्णुना᳚ । अ॒नि॒मि॒षेणेत्य॑नि - मि॒षेण॑ । जि॒ष्णुना॑ युत्का॒रेण॑ । यु॒त्का॒रेण॑ दुश्च्यव॒नेन॑ । यु॒त्का॒रेणेति॑ युत् - का॒रेण॑ । दु॒श्च्य॒व॒नेन॑ धृ॒ष्णुना᳚ । दु॒श्च्य॒व॒नेनेति॑ दुः - च्य॒व॒नेन॑ । धृ॒ष्णुनेति॑ धृ॒ष्णुना᳚ ॥ तदिन्द्रे॑ण । इन्द्रे॑ण जयत । ज॒य॒त॒ तत् । तथ् स॑हद्ध्वम् । स॒ह॒द्ध्वं॒ ॅयुधः॑ । युधो॑ नरः । न॒र॒ इषु॑हस्तेन । इषु॑हस्तेन॒ वृष्णा᳚ । इषु॑हस्ते॒नेतीषु॑ - ह॒स्ते॒न॒ । वृष्णेति॒ वृष्णा᳚ ॥ स इषु॑हस्तैः । इषु॑हस्तैः॒ सः । इषु॑हस्तै॒रितीषु॑ - ह॒स्तैः॒ । स नि॑ष॒ङ्गिभिः॑ । नि॒ष॒ङ्गिभि॑र् व॒शी । नि॒ष॒ङ्गिभि॒रिति॑ निष॒ङ्गि - भिः॒ । व॒शी सꣳस्र॑ष्टा । सꣳस्र॑ष्टा॒ सः । सꣳस्र॒ष्टेति॒ सम् - स्र॒ष्टा॒ । स युधः॑ । युध॒ इन्द्रः॑ । इन्द्रो॑ ग॒णेन॑ । ग॒णेनेति॑ ग॒णेन॑ ॥ सꣳ॒॒सृ॒ष्ट॒जिथ् सो॑म॒पाः । सꣳ॒॒सृ॒ष्ट॒जिदिति॑ सꣳसृष्ट - जित् । सो॒म॒पा बा॑हुश॒र्धी । सो॒म॒पा इति॑ सोम - पाः । बा॒हु॒श॒र्द्ध्यू᳚र्द्ध्वद्ध॑न्वा । बा॒हु॒श॒र्द्धीति॑ बाहु - श॒र्धी । ऊ॒र्द्ध्वध॑न्वा॒ प्रति॑हिताभिः । ऊ॒र्द्ध्वध॒न्वेत्यू॒र्द्ध्व - ध॒न्वा॒ । प्रति॑हिताभि॒रस्ता᳚ । प्रति॑हिताभि॒रिति॒ प्रति॑ - हि॒ता॒भिः॒ । अस्तेत्यस्ता᳚ ॥ बृह॑स्पते॒ परि॑ । परि॑ दीय । दी॒या॒ रथे॑न \newline

\textbf{Jatai Paata} \newline

1. आ॒शुः शिशा॑नः॒ शिशा॑न आ॒शु रा॒शुः शिशा॑नः । \newline
2. शिशा॑नो वृष॒भो वृ॑ष॒भः शिशा॑नः॒ शिशा॑नो वृष॒भः । \newline
3. वृ॒ष॒भो न न वृ॑ष॒भो वृ॑ष॒भो न । \newline
4. न यु॒द्ध्मो यु॒द्ध्मो न न यु॒द्ध्मः । \newline
5. यु॒द्ध्मो घ॑नाघ॒नो घ॑नाघ॒नो यु॒द्ध्मो यु॒द्ध्मो घ॑नाघ॒नः । \newline
6. घ॒ना॒घ॒नः क्षोभ॑णः॒ क्षोभ॑णो घनाघ॒नो घ॑नाघ॒नः क्षोभ॑णः । \newline
7. क्षोभ॑णश्चर्.षणी॒नाम् चर्॑.षणी॒नाम् क्षोभ॑णः॒ क्षोभ॑ण श्चर्.षणी॒नाम् । \newline
8. च॒र्॒.ष॒णी॒नामिति॑ चर्.षणी॒नाम् । \newline
9. स॒ङ्क्रन्द॑नो ऽनिमि॒षो॑ ऽनिमि॒षः स॒ङ्क्रन्द॑नः स॒ङ्क्रन्द॑नो ऽनिमि॒षः । \newline
10. स॒ङ्क्रन्द॑न॒ इति॑ सं - क्रन्द॑नः । \newline
11. अ॒नि॒मि॒ष ए॑कवी॒र ए॑कवी॒रो॑ ऽनिमि॒षो॑ ऽनिमि॒ष ए॑कवी॒रः । \newline
12. अ॒नि॒मि॒ष इत्य॑नि - मि॒षः । \newline
13. ए॒क॒वी॒रः श॒तꣳ श॒त मे॑कवी॒र ए॑कवी॒रः श॒तम् । \newline
14. ए॒क॒वी॒र इत्ये॑क - वी॒रः । \newline
15. श॒तꣳ सेनाः॒ सेनाः᳚ श॒तꣳ श॒तꣳ सेनाः᳚ । \newline
16. सेना॑ अजय दजय॒थ् सेनाः॒ सेना॑ अजयत् । \newline
17. अ॒ज॒य॒थ् सा॒कꣳ सा॒क म॑जय दजयथ् सा॒कम् । \newline
18. सा॒क मिन्द्र॒ इन्द्रः॑ सा॒कꣳ सा॒क मिन्द्रः॑ । \newline
19. इन्द्र॒ इतीन्द्रः॑ । \newline
20. स॒ङ्क्रन्द॑नेना निमि॒षेणा॑ निमि॒षेण॑ स॒ङ्क्रन्द॑नेन स॒ङ्क्रन्द॑नेना निमि॒षेण॑ । \newline
21. स॒ङ्क्रन्द॑ने॒नेति॑ सं - क्रन्द॑नेन । \newline
22. अ॒नि॒मि॒षेण॑ जि॒ष्णुना॑ जि॒ष्णुना॑ ऽनिमि॒षेणा॑ निमि॒षेण॑ जि॒ष्णुना᳚ । \newline
23. अ॒नि॒मि॒षेणेत्य॑नि - मि॒षेण॑ । \newline
24. जि॒ष्णुना॑ युत्का॒रेण॑ युत्का॒रेण॑ जि॒ष्णुना॑ जि॒ष्णुना॑ युत्का॒रेण॑ । \newline
25. यु॒त्का॒रेण॑ दुश्च्यव॒नेन॑ दुश्च्यव॒नेन॑ युत्का॒रेण॑ युत्का॒रेण॑ दुश्च्यव॒नेन॑ । \newline
26. यु॒त्का॒रेणेति॑ युत् - का॒रेण॑ । \newline
27. दु॒श्च्य॒व॒नेन॑ धृ॒ष्णुना॑ धृ॒ष्णुना॑ दुश्च्यव॒नेन॑ दुश्च्यव॒नेन॑ धृ॒ष्णुना᳚ । \newline
28. दु॒श्च्य॒व॒नेनेति॑ दुः - च्य॒व॒नेन॑ । \newline
29. धृ॒ष्णुनेति॑ धृ॒ष्णुना᳚ । \newline
30. तदिन्द्रे॒ णेन्द्रे॑ण॒ तत् तदिन्द्रे॑ण । \newline
31. इन्द्रे॑ण जयत जय॒ तेन्द्रे॒ णेन्द्रे॑ण जयत । \newline
32. ज॒य॒त॒ तत् तज् ज॑यत जयत॒ तत् । \newline
33. तथ् स॑हद्ध्वꣳ सहद्ध्वं॒ तत् तथ् स॑हद्ध्वं । \newline
34. स॒ह॒द्ध्वं॒ ॅयुधो॒ युधः॑ सहद्ध्वꣳ सहद्ध्वं॒ ॅयुधः॑ । \newline
35. युधो॑ नरो नरो॒ युधो॒ युधो॑ नरः । \newline
36. न॒र॒ इषु॑हस्ते॒ने षु॑हस्तेन नरो नर॒ इषु॑हस्तेन । \newline
37. इषु॑हस्तेन॒ वृष्णा॒ वृष्णेषु॑हस्ते॒ने षु॑हस्तेन॒ वृष्णा᳚ । \newline
38. इषु॑हस्ते॒नेतीषु॑ - ह॒स्ते॒न॒ । \newline
39. वृष्णेति॒ वृष्णा᳚ । \newline
40. स इषु॑हस्तै॒ रिषु॑हस्तैः॒ स स इषु॑हस्तैः । \newline
41. इषु॑हस्तैः॒ स स इषु॑हस्तै॒ रिषु॑हस्तैः॒ सः । \newline
42. इषु॑हस्तै॒रितीषु॑ - ह॒स्तैः॒ । \newline
43. स नि॑ष॒ङ्गिभि॑र् निष॒ङ्गिभिः॒ स स नि॑ष॒ङ्गिभिः॑ । \newline
44. नि॒ष॒ङ्गिभि॑र् व॒शी व॒शी नि॑ष॒ङ्गिभि॑र् निष॒ङ्गिभि॑र् व॒शी । \newline
45. नि॒ष॒ङ्गिभि॒रिति॑ निष॒ङ्गि - भिः॒ । \newline
46. व॒शी सꣳस्र॑ष्टा॒ सꣳस्र॑ष्टा व॒शी व॒शी सꣳस्र॑ष्टा । \newline
47. सꣳस्र॑ष्टा॒ स स सꣳस्र॑ष्टा॒ सꣳस्र॑ष्टा॒ सः । \newline
48. सꣳस्र॒ष्टेति॒ सं - स्र॒ष्टा॒ । \newline
49. स युधो॒ युधः॒ स स युधः॑ । \newline
50. युध॒ इन्द्र॒ इन्द्रो॒ युधो॒ युध॒ इन्द्रः॑ । \newline
51. इन्द्रो॑ ग॒णेन॑ ग॒णेनेन्द्र॒ इन्द्रो॑ ग॒णेन॑ । \newline
52. ग॒णेनेति॑ ग॒णेन॑ । \newline
53. सꣳ॒॒सृ॒ष्ट॒जिथ् सो॑म॒पाः सो॑म॒पाः सꣳ॑सृष्ट॒जिथ् सꣳ॑सृष्ट॒जिथ् सो॑म॒पाः । \newline
54. सꣳ॒॒सृ॒ष्ट॒जिदिति॑ सꣳसृष्ट - जित् । \newline
55. सो॒म॒पा बा॑हुश॒र्द्धी बा॑हुश॒र्द्धी सो॑म॒पाः सो॑म॒पा बा॑हुश॒र्द्धी । \newline
56. सो॒म॒पा इति॑ सोम - पाः । \newline
57. बा॒हु॒श॒ र्द्ध्यू᳚र्द्ध्वध॑ न्वो॒र्द्ध्वध॑न्वा बाहुश॒र्द्धी बा॑हुश॒ र्द्ध्यू᳚र्द्ध्वध॑न्वा । \newline
58. बा॒हु॒श॒र्द्धीति॑ बाहु - श॒र्द्धी । \newline
59. ऊ॒र्द्ध्वध॑न्वा॒ प्रति॑हिताभिः॒ प्रति॑हिताभि रू॒र्द्ध्वध॑न् वो॒र्द्ध्वध॑न्वा॒ प्रति॑हिताभिः । \newline
60. ऊ॒र्द्ध्वध॒न्वेत्यू॒र्द्ध्व - ध॒न्वा॒ । \newline
61. प्रति॑हिताभि॒ रस्ता ऽस्ता॒ प्रति॑हिताभिः॒ प्रति॑हिताभि॒ रस्ता᳚ । \newline
62. प्रति॑हिताभि॒रिति॒ प्रति॑ - हि॒ता॒भिः॒ । \newline
63. अस्तेत्यस्ता᳚ । \newline
64. बृह॑स्पते॒ परि॒ परि॒ बृह॑स्पते॒ बृह॑स्पते॒ परि॑ । \newline
65. परि॑ दीय दीय॒ परि॒ परि॑ दीय । \newline
66. दी॒या॒ रथे॑न॒ रथे॑न दीय दीया॒ रथे॑न । \newline

\textbf{Ghana Paata } \newline

1. आ॒शुः शिशा॑नः॒ शिशा॑न आ॒शु रा॒शुः शिशा॑नो वृष॒भो वृ॑ष॒भः शिशा॑न आ॒शु रा॒शुः शिशा॑नो वृष॒भः । \newline
2. शिशा॑नो वृष॒भो वृ॑ष॒भः शिशा॑नः॒ शिशा॑नो वृष॒भो न न वृ॑ष॒भः शिशा॑नः॒ शिशा॑नो वृष॒भो न । \newline
3. वृ॒ष॒भो न न वृ॑ष॒भो वृ॑ष॒भो न यु॒द्ध्मो यु॒द्ध्मो न वृ॑ष॒भो वृ॑ष॒भो न यु॒द्ध्मः । \newline
4. न यु॒द्ध्मो यु॒द्ध्मो न न यु॒द्ध्मो घ॑नाघ॒नो घ॑नाघ॒नो यु॒द्ध्मो न न यु॒द्ध्मो घ॑नाघ॒नः । \newline
5. यु॒द्ध्मो घ॑नाघ॒नो घ॑नाघ॒नो यु॒द्ध्मो यु॒द्ध्मो घ॑नाघ॒नः क्षोभ॑णः॒ क्षोभ॑णो घनाघ॒नो यु॒द्ध्मो यु॒द्ध्मो घ॑नाघ॒नः क्षोभ॑णः । \newline
6. घ॒ना॒घ॒नः क्षोभ॑णः॒ क्षोभ॑णो घनाघ॒नो घ॑नाघ॒नः क्षोभ॑ण श्चर्.षणी॒नाम् चर्॑.षणी॒नाम् क्षोभ॑णो घनाघ॒नो घ॑नाघ॒नः क्षोभ॑ण श्चर्.षणी॒नाम् । \newline
7. क्षोभ॑ण श्चर्.षणी॒नाम् चर्॑.षणी॒नाम् क्षोभ॑णः॒ क्षोभ॑ण श्चर्.षणी॒नाम् । \newline
8. च॒र्॒.ष॒णी॒नामिति॑ चर्.षणी॒नाम् । \newline
9. स॒ङ्क्रन्द॑नो ऽनिमि॒षो॑ ऽनिमि॒षः स॒ङ्क्रन्द॑नः स॒ङ्क्रन्द॑नो ऽनिमि॒ष ए॑कवी॒र ए॑कवी॒रो॑ ऽनिमि॒षः स॒ङ्क्रन्द॑नः स॒ङ्क्रन्द॑नो ऽनिमि॒ष ए॑कवी॒रः । \newline
10. स॒ङ्क्रन्द॑न॒ इति॑ सं - क्रन्द॑नः । \newline
11. अ॒नि॒मि॒ष ए॑कवी॒र ए॑कवी॒रो॑ ऽनिमि॒षो॑ ऽनिमि॒ष ए॑कवी॒रः श॒तꣳ श॒त मे॑कवी॒रो॑ ऽनिमि॒षो॑ ऽनिमि॒ष ए॑कवी॒रः श॒तम् । \newline
12. अ॒नि॒मि॒ष इत्य॑नि - मि॒षः । \newline
13. ए॒क॒वी॒रः श॒तꣳ श॒त मे॑कवी॒र ए॑कवी॒रः श॒तꣳ सेनाः॒ सेनाः᳚ श॒त मे॑कवी॒र ए॑कवी॒रः श॒तꣳ सेनाः᳚ । \newline
14. ए॒क॒वी॒र इत्ये॑क - वी॒रः । \newline
15. श॒तꣳ सेनाः॒ सेनाः᳚ श॒तꣳ श॒तꣳ सेना॑ अजय दजय॒थ् सेनाः᳚ श॒तꣳ श॒तꣳ सेना॑ अजयत् । \newline
16. सेना॑ अजय दजय॒थ् सेनाः॒ सेना॑ अजयथ् सा॒कꣳ सा॒क म॑जय॒थ् सेनाः॒ सेना॑ अजयथ् सा॒कम् । \newline
17. अ॒ज॒य॒थ् सा॒कꣳ सा॒क म॑जय दजयथ् सा॒क मिन्द्र॒ इन्द्रः॑ सा॒क म॑जय दजयथ् सा॒क मिन्द्रः॑ । \newline
18. सा॒क मिन्द्र॒ इन्द्रः॑ सा॒कꣳ सा॒क मिन्द्रः॑ । \newline
19. इन्द्र॒ इतीन्द्रः॑ । \newline
20. स॒ङ्क्रन्द॑नेना निमि॒षेणा॑ निमि॒षेण॑ स॒ङ्क्रन्द॑नेन स॒ङ्क्रन्द॑नेना निमि॒षेण॑ जि॒ष्णुना॑ जि॒ष्णुना॑ ऽनिमि॒षेण॑ स॒ङ्क्रन्द॑नेन स॒ङ्क्रन्द॑नेना निमि॒षेण॑ जि॒ष्णुना᳚ । \newline
21. स॒ङ्क्रन्द॑ने॒नेति॑ सं - क्रन्द॑नेन । \newline
22. अ॒नि॒मि॒षेण॑ जि॒ष्णुना॑ जि॒ष्णुना॑ ऽनिमि॒षेणा॑ निमि॒षेण॑ जि॒ष्णुना॑ युत्का॒रेण॑ युत्का॒रेण॑ जि॒ष्णुना॑ ऽनिमि॒षेणा॑ निमि॒षेण॑ जि॒ष्णुना॑ युत्का॒रेण॑ । \newline
23. अ॒नि॒मि॒षेणेत्य॑नि - मि॒षेण॑ । \newline
24. जि॒ष्णुना॑ युत्का॒रेण॑ युत्का॒रेण॑ जि॒ष्णुना॑ जि॒ष्णुना॑ युत्का॒रेण॑ दुश्च्यव॒नेन॑ दुश्च्यव॒नेन॑ युत्का॒रेण॑ जि॒ष्णुना॑ जि॒ष्णुना॑ युत्का॒रेण॑ दुश्च्यव॒नेन॑ । \newline
25. यु॒त्का॒रेण॑ दुश्च्यव॒नेन॑ दुश्च्यव॒नेन॑ युत्का॒रेण॑ युत्का॒रेण॑ दुश्च्यव॒नेन॑ धृ॒ष्णुना॑ धृ॒ष्णुना॑ दुश्च्यव॒नेन॑ युत्का॒रेण॑ युत्का॒रेण॑ दुश्च्यव॒नेन॑ धृ॒ष्णुना᳚ । \newline
26. यु॒त्का॒रेणेति॑ युत् - का॒रेण॑ । \newline
27. दु॒श्च्य॒व॒नेन॑ धृ॒ष्णुना॑ धृ॒ष्णुना॑ दुश्च्यव॒नेन॑ दुश्च्यव॒नेन॑ धृ॒ष्णुना᳚ । \newline
28. दु॒श्च्य॒व॒नेनेति॑ दुः - च्य॒व॒नेन॑ । \newline
29. धृ॒ष्णुनेति॑ धृ॒ष्णुना᳚ । \newline
30. तदिन्द्रे॒ णेन्द्रे॑ण॒ तत् तदिन्द्रे॑ण जयत जय॒तेन्द्रे॑ण॒ तत् तदिन्द्रे॑ण जयत । \newline
31. इन्द्रे॑ण जयत जय॒तेन्द्रे॒ णेन्द्रे॑ण जयत॒ तत् तज् ज॑य॒तेन्द्रे॒ णेन्द्रे॑ण जयत॒ तत् । \newline
32. ज॒य॒त॒ तत् तज् ज॑यत जयत॒ तथ् स॑हद्ध्वꣳ सहद्ध्वं॒ तज् ज॑यत जयत॒ तथ् स॑हद्ध्वं । \newline
33. तथ् स॑हद्ध्वꣳ सहद्ध्वं॒ तत् तथ् स॑हद्ध्वं॒ ॅयुधो॒ युधः॑ सहद्ध्वं॒ तत् तथ् स॑हद्ध्वं॒ ॅयुधः॑ । \newline
34. स॒ह॒द्ध्वं॒ ॅयुधो॒ युधः॑ सहद्ध्वꣳ सहद्ध्वं॒ ॅयुधो॑ नरो नरो॒ युधः॑ 
सहद्ध्वꣳ सहद्ध्वं॒ ॅयुधो॑ नरः । \newline
35. युधो॑ नरो नरो॒ युधो॒ युधो॑ नर॒ इषु॑हस्ते॒ नेषु॑हस्तेन नरो॒ युधो॒ युधो॑ नर॒ इषु॑हस्तेन । \newline
36. न॒र॒ इषु॑हस्ते॒ नेषु॑हस्तेन नरो नर॒ इषु॑हस्तेन॒ वृष्णा॒ वृष्णे षु॑हस्तेन नरो नर॒ इषु॑हस्तेन॒ वृष्णा᳚ । \newline
37. इषु॑हस्तेन॒ वृष्णा॒ वृष्णे षु॑हस्ते॒ नेषु॑हस्तेन॒ वृष्णा᳚ । \newline
38. इषु॑हस्ते॒नेतीषु॑ - ह॒स्ते॒न॒ । \newline
39. वृष्णेति॒ वृष्णा᳚ । \newline
40. स इषु॑हस्तै॒ रिषु॑हस्तैः॒ स स इषु॑हस्तैः॒ स स इषु॑हस्तैः॒ स स इषु॑हस्तैः॒ सः । \newline
41. इषु॑हस्तैः॒ स स इषु॑हस्तै॒ रिषु॑हस्तैः॒ स नि॑ष॒ङ्गिभि॑र् निष॒ङ्गिभिः॒ स इषु॑हस्तै॒ रिषु॑हस्तैः॒ स नि॑ष॒ङ्गिभिः॑ । \newline
42. इषु॑हस्तै॒रितीषु॑ - ह॒स्तैः॒ । \newline
43. स नि॑ष॒ङ्गिभि॑र् निष॒ङ्गिभिः॒ स स नि॑ष॒ङ्गिभि॑र् व॒शी व॒शी नि॑ष॒ङ्गिभिः॒ स स नि॑ष॒ङ्गिभि॑र् व॒शी । \newline
44. नि॒ष॒ङ्गिभि॑र् व॒शी व॒शी नि॑ष॒ङ्गिभि॑र् निष॒ङ्गिभि॑र् व॒शी सꣳस्र॑ष्टा॒ सꣳस्र॑ष्टा व॒शी नि॑ष॒ङ्गिभि॑र् निष॒ङ्गिभि॑र् व॒शी सꣳस्र॑ष्टा । \newline
45. नि॒ष॒ङ्गिभि॒रिति॑ निष॒ङ्गि - भिः॒ । \newline
46. व॒शी सꣳस्र॑ष्टा॒ सꣳस्र॑ष्टा व॒शी व॒शी सꣳस्र॑ष्टा॒ स स सꣳस्र॑ष्टा व॒शी व॒शी सꣳस्र॑ष्टा॒ सः । \newline
47. सꣳस्र॑ष्टा॒ स स सꣳस्र॑ष्टा॒ सꣳस्र॑ष्टा॒ स युधो॒ युधः॒ स सꣳस्र॑ष्टा॒ सꣳस्र॑ष्टा॒ स युधः॑ । \newline
48. सꣳस्र॒ष्टेति॒ सं - स्र॒ष्टा॒ । \newline
49. स युधो॒ युधः॒ स स युध॒ इन्द्र॒ इन्द्रो॒ युधः॒ स स युध॒ इन्द्रः॑ । \newline
50. युध॒ इन्द्र॒ इन्द्रो॒ युधो॒ युध॒ इन्द्रो॑ ग॒णेन॑ ग॒णेनेन्द्रो॒ युधो॒ युध॒ इन्द्रो॑ ग॒णेन॑ । \newline
51. इन्द्रो॑ ग॒णेन॑ ग॒णेनेन्द्र॒ इन्द्रो॑ ग॒णेन॑ । \newline
52. ग॒णेनेति॑ ग॒णेन॑ । \newline
53. सꣳ॒॒सृ॒ष्ट॒जिथ् सो॑म॒पाः सो॑म॒पाः सꣳ॑सृष्ट॒जिथ् सꣳ॑सृष्ट॒जिथ् सो॑म॒पा बा॑हुश॒र्द्धी बा॑हुश॒र्द्धी सो॑म॒पाः सꣳ॑सृष्ट॒जिथ् सꣳ॑सृष्ट॒जिथ् सो॑म॒पा बा॑हुश॒र्द्धी । \newline
54. सꣳ॒॒सृ॒ष्ट॒जिदिति॑ सꣳसृष्ट - जित् । \newline
55. सो॒म॒पा बा॑हुश॒र्द्धी बा॑हुश॒र्द्धी सो॑म॒पाः सो॑म॒पा बा॑हुश॒र्द्ध्यू᳚र्द्ध्वध॑ न्वो॒र्द्ध्वध॑न्वा बाहुश॒र्द्धी सो॑म॒पाः सो॑म॒पा बा॑हुश॒र्द्ध्यू᳚र्द्ध्वध॑न्वा । \newline
56. सो॒म॒पा इति॑ सोम - पाः । \newline
57. बा॒हु॒श॒र्द्ध्यू᳚र्द्ध्वध॑ न्वो॒र्द्ध्वध॑न्वा बाहुश॒र्द्धी बा॑हुश॒र्द्ध्यू᳚र्द्ध्वध॑न्वा॒ प्रति॑हिताभिः॒ प्रति॑हिताभि रू॒र्द्ध्वध॑न्वा बाहुश॒र्द्धी बा॑हुश॒र्द्ध्यू᳚र्द्ध्वध॑न्वा॒ प्रति॑हिताभिः । \newline
58. बा॒हु॒श॒र्द्धीति॑ बाहु - श॒र्द्धी । \newline
59. ऊ॒र्द्ध्वध॑न्वा॒ प्रति॑हिताभिः॒ प्रति॑हिताभि रू॒र्द्ध्वध॑ न्वो॒र्द्ध्वध॑न्वा॒ प्रति॑हिताभि॒ रस्ता ऽस्ता॒ प्रति॑हिताभि रू॒र्द्ध्वध॑ न्वो॒र्द्ध्वध॑न्वा॒ प्रति॑हिताभि॒ रस्ता᳚ । \newline
60. ऊ॒र्द्ध्वध॒न्वेत्यू॒र्द्ध्व - ध॒न्वा॒ । \newline
61. प्रति॑हिताभि॒ रस्ता ऽस्ता॒ प्रति॑हिताभिः॒ प्रति॑हिताभि॒ रस्ता᳚ । \newline
62. प्रति॑हिताभि॒रिति॒ प्रति॑ - हि॒ता॒भिः॒ । \newline
63. अस्तेत्यस्ता᳚ । \newline
64. बृह॑स्पते॒ परि॒ परि॒ बृह॑स्पते॒ बृह॑स्पते॒ परि॑ दीय दीय॒ परि॒ बृह॑स्पते॒ बृह॑स्पते॒ परि॑ दीय । \newline
65. परि॑ दीय दीय॒ परि॒ परि॑ दीया॒ रथे॑न॒ रथे॑न दीय॒ परि॒ परि॑ दीया॒ रथे॑न । \newline
66. दी॒या॒ रथे॑न॒ रथे॑न दीय दीया॒ रथे॑न रक्षो॒हा र॑क्षो॒हा रथे॑न दीय दीया॒ रथे॑न रक्षो॒हा । \newline
\pagebreak
\markright{ TS 4.6.4.2  \hfill https://www.vedavms.in \hfill}

\section{ TS 4.6.4.2 }

\textbf{TS 4.6.4.2 } \newline
\textbf{Samhita Paata} \newline

रथे॑न रक्षो॒हा ऽमित्राꣳ॑ अप॒ बाध॑मानः । प्र॒भं॒जन्थ् सेनाः᳚ प्रमृ॒णो यु॒धा जय॑न्न॒स्माक॑-मेद्ध्यवि॒ता रथा॑नां ॥ गो॒त्र॒भिदं॑ गो॒विदं॒ ॅवज्र॑बाहुं॒ जय॑न्त॒मज्म॑ प्रमृ॒णन्त॒-मोज॑सा । इ॒मꣳ स॑जाता॒ अनु॑वीर-यद्ध्व॒मिन्द्रꣳ॑ सखा॒योऽनु॒ सꣳ र॑भद्ध्वं ॥ ब॒ल॒वि॒ज्ञा॒यः-स्थवि॑रः॒ प्रवी॑रः॒ सह॑स्वान्. वा॒जी सह॑मान उ॒ग्रः । अ॒भिवी॑रो अ॒भिस॑त्वा सहो॒जा जैत्र॑मिन्द्र॒ रथ॒माति॑ष्ठ गो॒वित् ॥ अ॒भि गो॒त्राणि॒ सह॑सा॒ गाह॑मानोऽदा॒यो- [  ] \newline

\textbf{Pada Paata} \newline

रथे॑न । र॒क्षो॒हेति॑ रक्षः - हा । अ॒मित्रान्॑ । अ॒प॒बाध॑मान॒ इत्य॑प - बाध॑मानः ॥ प्र॒भ॒ञ्जन्निति॑ प्र - भ॒ञ्जन्न् । सेनाः᳚ । प्र॒मृ॒ण इति॑ प्र - मृ॒णः । यु॒धा । जयन्न्॑ । अ॒स्माक᳚म् । ए॒धि॒ । अ॒वि॒ता । रथा॑नाम् ॥ गो॒त्र॒भिद॒मिति॑ गोत्र - भिद᳚म् । गो॒विद॒मिति॑ गो - विद᳚म् । वज्र॑बाहु॒मिति॒ वज्र॑ - बा॒हु॒म् । जय॑न्तम् । अज्म॑ । प्र॒मृ॒णन्त॒मिति॑ प्र - मृ॒णन्त᳚म् । ओज॑सा ॥ इ॒मम् । स॒जा॒ता॒ इति॑ स - जा॒ताः॒ । अन्विति॑ । वी॒र॒य॒द्ध्व॒म् । इन्द्र᳚म् । स॒खा॒यः॒ । अनु॑ । समिति॑ । र॒भ॒द्ध्व॒म् ॥ ब॒ल॒वि॒ज्ञा॒य इति॑ बल - वि॒ज्ञा॒यः । स्थवि॑रः । प्रवी॑र॒ इति॒ प्र - वी॒रः॒ । सह॑स्वान् । वा॒जी । सह॑मानः । उ॒ग्रः ॥ अ॒भिवी॑र॒ इत्य॒भि - वी॒रः॒ । अ॒भिस॒त्वेत्य॒भि - स॒त्वा॒ । स॒हो॒जा इति॑ सहः - जाः । जैत्र᳚म् । इ॒न्द्र॒ । रथ᳚म् । एति॑ । ति॒ष्ठ॒ । गो॒विदिति॑ गो - वित् ॥ अ॒भीति॑ । गो॒त्राणि॑ । सह॑सा । गाह॑मानः । अ॒दा॒यः ।  \newline


\textbf{Krama Paata} \newline

रथे॑न रक्षो॒हा । र॒क्षो॒हाऽमित्रान्॑ । र॒क्षो॒हेति॑ रक्षः - हा । अ॒मित्राꣳ॑ अप॒बाध॑मानः । अ॒प॒बाध॑मान॒ इत्य॑प - बाध॑मानः ॥ प्र॒भ॒ञ्जन्थ् सेनाः᳚ । प्र॒भ॒ञ्जन्निति॑ प्र - भ॒ञ्जन्न् । सेनाः᳚ प्रमृ॒णः । प्र॒मृ॒णो यु॒धा । प्र॒मृ॒ण इति॑ प्र - मृ॒णः । यु॒धा जयन्न्॑ । जय॑न्न॒स्माक᳚म् । अ॒स्माक॑मेधि । ए॒द्ध्य॒वि॒ता । अ॒वि॒ता रथा॑नाम् । रथा॑ना॒मिति॒ रथा॑नाम् ॥ गो॒त्र॒भिद॑म् गो॒विद᳚म् । गो॒त्र॒भिद॒मिति॑ गोत्र - भिद᳚म् । गो॒विदं॒ ॅवज्र॑बाहुम् । गो॒विद॒मिति॑ गो - विद᳚म् । वज्र॑बाहु॒म् जय॑न्तम् । वज्र॑बाहु॒मिति॒ वज्र॑ - बा॒हु॒म् । जय॑न्त॒मज्म॑ । अज्म॑ प्रमृ॒णन्त᳚म् । प्र॒मृ॒णन्त॒मोज॑सा । प्र॒मृ॒णन्त॒मिति॑ प्र - मृ॒णन्त᳚म् । ओज॒सेत्योज॑सा ॥ इ॒मꣳ स॑जाताः । स॒जा॒ता॒ अनु॑ । स॒जा॒ता॒ इति॑ स - जा॒ताः॒ । अनु॑ वीरयद्ध्वम् । वी॒र॒य॒द्ध्व॒मिन्द्र᳚म् । इन्द्रꣳ॑ सखा॒यः । स॒खा॒योऽनु॑ । अनु॒ सम् । सꣳ र॑भद्ध्वम् । र॒भ॒द्ध्व॒मिति॑ रभद्ध्वम् ॥ ब॒ल॒वि॒ज्ञा॒यः स्थवि॑रः । ब॒ल॒वि॒ज्ञा॒य इति॑ बल - वि॒ज्ञा॒यः । स्थवि॑रः॒ प्रवी॑रः । प्रवी॑रः॒ सह॑स्वान् । प्रवी॑र॒ इति॒ प्र - वी॒रः॒ । सह॑स्वान्. वा॒जी । वा॒जी सह॑मानः । सह॑मान उ॒ग्रः । उ॒ग्र इत्यु॒ग्रः ॥ अ॒भिवी॑रो अ॒भिस॑त्वा । अ॒भिवी॑र॒ इत्य॒भि - वी॒रः॒ । अ॒भिस॑त्वा सहो॒जाः । अ॒भिस॒त्वेत्य॒भि - स॒त्वा॒ । स॒हो॒जा जैत्र᳚म् । स॒हो॒जा इति॑ सहः - जाः । जैत्र॑मिन्द्र । इ॒न्द्र॒ रथ᳚म् । रथ॒मा । आ ति॑ष्ठ । ति॒ष्ठ॒ गो॒वित् । गो॒विदिति॑ गो - वित् ॥ अ॒भि गो॒त्राणि॑ । गो॒त्राणि॒ सह॑सा । सह॑सा॒ गाह॑मानः । गाह॑मानोऽदा॒यः । अ॒दा॒यो वी॒रः \newline

\textbf{Jatai Paata} \newline

1. रथे॑न रक्षो॒हा र॑क्षो॒हा रथे॑न॒ रथे॑न रक्षो॒हा । \newline
2. र॒क्षो॒हा ऽमित्राꣳ॑ अ॒मित्रा᳚न् रक्षो॒हा र॑क्षो॒हा ऽमित्रान्॑ । \newline
3. र॒क्षो॒हेति॑ रक्षः - हा । \newline
4. अ॒मित्राꣳ॑ अप॒बाध॑मानो अप॒बाध॑मानो अ॒मित्राꣳ॑ अ॒मित्राꣳ॑ अप॒बाध॑मानः । \newline
5. अ॒प॒बाध॑मान॒ इत्य॑प - बाध॑मानः । \newline
6. प्र॒भ॒ञ्जन् थ्सेनाः॒ सेनाः᳚ प्रभ॒ञ्जन् प्र॑भ॒ञ्जन् थ्सेनाः᳚ । \newline
7. प्र॒भ॒ञ्जन्निति॑ प्र - भ॒ञ्जन्न् । \newline
8. सेनाः᳚ प्रमृ॒णः प्र॑मृ॒णः सेनाः॒ सेनाः᳚ प्रमृ॒णः । \newline
9. प्र॒मृ॒णो यु॒धा यु॒धा प्र॑मृ॒णः प्र॑मृ॒णो यु॒धा । \newline
10. प्र॒मृ॒ण इति॑ प्र - मृ॒णः । \newline
11. यु॒धा जय॒न् जय॑न्. यु॒धा यु॒धा जयन्न्॑ । \newline
12. जय॑न् न॒स्माक॑ म॒स्माक॒म् जय॒न् जय॑न् न॒स्माक᳚म् । \newline
13. अ॒स्माक॑ मेध्ये ध्य॒स्माक॑ म॒स्माक॑ मेधि । \newline
14. ए॒ध्य॒वि॒ता ऽवि॒तै ध्ये᳚ध्यवि॒ता । \newline
15. अ॒वि॒ता रथा॑नाꣳ॒॒ रथा॑ना मवि॒ता ऽवि॒ता रथा॑नाम् । \newline
16. रथा॑ना॒मिति॒ रथा॑नाम् । \newline
17. गो॒त्र॒भिद॑म् गो॒विद॑म् गो॒विद॑म् गोत्र॒भिद॑म् गोत्र॒भिद॑म् गो॒विद᳚म् । \newline
18. गो॒त्र॒भिद॒मिति॑ गोत्र - भिद᳚म् । \newline
19. गो॒विदं॒ ॅवज्र॑बाहुं॒ ॅवज्र॑बाहुम् गो॒विद॑म् गो॒विदं॒ ॅवज्र॑बाहुम् । \newline
20. गो॒विद॒मिति॑ गो - विद᳚म् । \newline
21. वज्र॑बाहु॒म् जय॑न्त॒म् जय॑न्तं॒ ॅवज्र॑बाहुं॒ ॅवज्र॑बाहु॒म् जय॑न्तम् । \newline
22. वज्र॑बाहु॒मिति॒ वज्र॑ - बा॒हु॒म् । \newline
23. जय॑न्त॒ मज्माज्म॒ जय॑न्त॒म् जय॑न्त॒ मज्म॑ । \newline
24. अज्म॑ प्रमृ॒णन्त॑म् प्रमृ॒णन्त॒ मज्माज्म॑ प्रमृ॒णन्त᳚म् । \newline
25. प्र॒मृ॒णन्त॒ मोज॒ सौज॑सा प्रमृ॒णन्त॑म् प्रमृ॒णन्त॒ मोज॑सा । \newline
26. प्र॒मृ॒णन्त॒मिति॑ प्र - मृ॒णन्त᳚म् । \newline
27. ओज॒सेत्योज॑सा । \newline
28. इ॒मꣳ स॑जाताः सजाता इ॒म मि॒मꣳ स॑जाताः । \newline
29. स॒जा॒ता॒ अन्वनु॑ सजाताः सजाता॒ अनु॑ । \newline
30. स॒जा॒ता॒ इति॑ स - जा॒ताः॒ । \newline
31. अनु॑ वीरयद्ध्वं ॅवीरयद्ध्व॒ मन्वनु॑ वीरयद्ध्वम् । \newline
32. वी॒र॒य॒द्ध्व॒ मिन्द्र॒ मिन्द्रं॑ ॅवीरयद्ध्वं ॅवीरयद्ध्व॒ मिन्द्र᳚म् । \newline
33. इन्द्रꣳ॑ सखायः सखाय॒ इन्द्र॒ मिन्द्रꣳ॑ सखायः । \newline
34. स॒खा॒यो ऽन्वनु॑ सखायः सखा॒यो ऽनु॑ । \newline
35. अनु॒ सꣳ स मन्वनु॒ सम् । \newline
36. सꣳ र॑भद्ध्वꣳ रभद्ध्वꣳ॒॒ सꣳ सꣳ र॑भद्ध्वम् । \newline
37. र॒भ॒द्ध्व॒मिति॑ रभद्ध्वम् । \newline
38. ब॒ल॒वि॒ज्ञा॒यः स्थवि॑रः॒ स्थवि॑रो बलविज्ञा॒यो ब॑लविज्ञा॒यः स्थवि॑रः । \newline
39. ब॒ल॒वि॒ज्ञा॒य इति॑ बल - वि॒ज्ञा॒यः । \newline
40. स्थवि॑रः॒ प्रवी॑रः॒ प्रवी॑रः॒ स्थवि॑रः॒ स्थवि॑रः॒ प्रवी॑रः । \newline
41. प्रवी॑रः॒ सह॑स्वा॒न् थ्सह॑स्वा॒न् प्रवी॑रः॒ प्रवी॑रः॒ सह॑स्वान् । \newline
42. प्रवी॑र॒ इति॒ प्र - वी॒रः॒ । \newline
43. सह॑स्वान्. वा॒जी वा॒जी सह॑स्वा॒न् थ्सह॑स्वान्. वा॒जी । \newline
44. वा॒जी सह॑मानः॒ सह॑मानो वा॒जी वा॒जी सह॑मानः । \newline
45. सह॑मान उ॒ग्र उ॒ग्रः सह॑मानः॒ सह॑मान उ॒ग्रः । \newline
46. उ॒ग्र इत्यु॒ग्रः । \newline
47. अ॒भिवी॑रो अ॒भिस॑त्वा॒ ऽभिस॑त्वा॒ ऽभिवी॑रो अ॒भिवी॑रो अ॒भिस॑त्वा । \newline
48. अ॒भिवी॑र॒ इत्य॒भि - वी॒रः॒ । \newline
49. अ॒भिस॑त्वा सहो॒जाः स॑हो॒जा अ॒भिस॑त्वा॒ ऽभिस॑त्वा सहो॒जाः । \newline
50. अ॒भिस॒त्वेत्य॒भि - स॒त्वा॒ । \newline
51. स॒हो॒जा जैत्र॒म् जैत्रꣳ॑ सहो॒जाः स॑हो॒जा जैत्र᳚म् । \newline
52. स॒हो॒जा इति॑ सहः - जाः । \newline
53. जैत्र॑ मिन्द्रेन्द्र॒ जैत्र॒म् जैत्र॑ मिन्द्र । \newline
54. इ॒न्द्र॒ रथꣳ॒॒ रथ॑ मिन्द्रेन्द्र॒ रथ᳚म् । \newline
55. रथ॒ मा रथꣳ॒॒ रथ॒ मा । \newline
56. आ ति॑ष्ठ ति॒ष्ठा ति॑ष्ठ । \newline
57. ति॒ष्ठ॒ गो॒विद् गो॒वित् ति॑ष्ठ तिष्ठ गो॒वित् । \newline
58. गो॒विदिति॑ गो - वित् । \newline
59. अ॒भि गो॒त्राणि॑ गो॒त्रा ण्य॒भ्य॑भि गो॒त्राणि॑ । \newline
60. गो॒त्राणि॒ सह॑सा॒ सह॑सा गो॒त्राणि॑ गो॒त्राणि॒ सह॑सा । \newline
61. सह॑सा॒ गाह॑मानो॒ गाह॑मानः॒ सह॑सा॒ सह॑सा॒ गाह॑मानः । \newline
62. गाह॑मानो ऽदा॒यो अ॑दा॒यो गाह॑मानो॒ गाह॑मानो ऽदा॒यः । \newline
63. अ॒दा॒यो वी॒रो वी॒रो अ॑दा॒यो अ॑दा॒यो वी॒रः । \newline

\textbf{Ghana Paata } \newline

1. रथे॑न रक्षो॒हा र॑क्षो॒हा रथे॑न॒ रथे॑न रक्षो॒हा ऽमित्राꣳ॑ अ॒मित्रा᳚न् रक्षो॒हा रथे॑न॒ रथे॑न रक्षो॒हा ऽमित्रान्॑ । \newline
2. र॒क्षो॒हा ऽमित्राꣳ॑ अ॒मित्रा᳚न् रक्षो॒हा र॑क्षो॒हा ऽमित्राꣳ॑ अप॒बाध॑मानो अप॒बाध॑मानो अ॒मित्रा᳚न् रक्षो॒हा र॑क्षो॒हा ऽमित्राꣳ॑ अप॒बाध॑मानः । \newline
3. र॒क्षो॒हेति॑ रक्षः - हा । \newline
4. अ॒मित्राꣳ॑ अप॒बाध॑मानो अप॒बाध॑मानो अ॒मित्राꣳ॑ अ॒मित्राꣳ॑ अप॒बाध॑मानः । \newline
5. अ॒प॒बाध॑मान॒ इत्य॑प - बाध॑मानः । \newline
6. प्र॒भ॒ञ्जन् थ्सेनाः॒ सेनाः᳚ प्रभ॒ञ्जन् प्र॑भ॒ञ्जन् थ्सेनाः᳚ प्रमृ॒णः प्र॑मृ॒णः सेनाः᳚ प्रभ॒ञ्जन् प्र॑भ॒ञ्जन् थ्सेनाः᳚ प्रमृ॒णः । \newline
7. प्र॒भ॒ञ्जन्निति॑ प्र - भ॒ञ्जन्न् । \newline
8. सेनाः᳚ प्रमृ॒णः प्र॑मृ॒णः सेनाः॒ सेनाः᳚ प्रमृ॒णो यु॒धा यु॒धा प्र॑मृ॒णः सेनाः॒ सेनाः᳚ प्रमृ॒णो यु॒धा । \newline
9. प्र॒मृ॒णो यु॒धा यु॒धा प्र॑मृ॒णः प्र॑मृ॒णो यु॒धा जय॒न् जय॑न्. यु॒धा प्र॑मृ॒णः प्र॑मृ॒णो यु॒धा जयन्न्॑ । \newline
10. प्र॒मृ॒ण इति॑ प्र - मृ॒णः । \newline
11. यु॒धा जय॒न् जय॑न्. यु॒धा यु॒धा जय॑न् न॒स्माक॑ म॒स्माक॒म् जय॑न्. यु॒धा यु॒धा जय॑न् न॒स्माक᳚म् । \newline
12. जय॑न् न॒स्माक॑ म॒स्माक॒म् जय॒न् जय॑न् न॒स्माक॑ मेध्ये ध्य॒स्माक॒म् जय॒न् जय॑न् न॒स्माक॑ मेधि । \newline
13. अ॒स्माक॑ मेध्ये ध्य॒स्माक॑ म॒स्माक॑ मेध्यवि॒ता ऽवि॒तै ध्य॒स्माक॑ म॒स्माक॑ मेध्यवि॒ता । \newline
14. ए॒ध्य॒ वि॒ता ऽवि॒तैध्ये᳚ ध्यवि॒ता रथा॑नाꣳ॒॒ रथा॑ना मवि॒तै ध्ये᳚ध्यवि॒ता रथा॑नाम् । \newline
15. अ॒वि॒ता रथा॑नाꣳ॒॒ रथा॑ना मवि॒ता ऽवि॒ता रथा॑नाम् । \newline
16. रथा॑ना॒मिति॒ रथा॑नाम् । \newline
17. गो॒त्र॒भिद॑म् गो॒विद॑म् गो॒विद॑म् गोत्र॒भिद॑म् गोत्र॒भिद॑म् गो॒विदं॒ ॅवज्र॑बाहुं॒ ॅवज्र॑बाहुम् गो॒विद॑म् गोत्र॒भिद॑म् गोत्र॒भिद॑म् गो॒विदं॒ ॅवज्र॑बाहुम् । \newline
18. गो॒त्र॒भिद॒मिति॑ गोत्र - भिद᳚म् । \newline
19. गो॒विदं॒ ॅवज्र॑बाहुं॒ ॅवज्र॑बाहुम् गो॒विद॑म् गो॒विदं॒ ॅवज्र॑बाहु॒म् जय॑न्त॒म् जय॑न्तं॒ ॅवज्र॑बाहुम् गो॒विद॑म् गो॒विदं॒ ॅवज्र॑बाहु॒म् जय॑न्तम् । \newline
20. गो॒विद॒मिति॑ गो - विद᳚म् । \newline
21. वज्र॑बाहु॒म् जय॑न्त॒म् जय॑न्तं॒ ॅवज्र॑बाहुं॒ ॅवज्र॑बाहु॒म् जय॑न्त॒ मज्माज्म॒ जय॑न्तं॒ ॅवज्र॑बाहुं॒ ॅवज्र॑बाहु॒म् जय॑न्त॒ मज्म॑ । \newline
22. वज्र॑बाहु॒मिति॒ वज्र॑ - बा॒हु॒म् । \newline
23. जय॑न्त॒ मज्माज्म॒ जय॑न्त॒म् जय॑न्त॒ मज्म॑ प्रमृ॒णन्त॑म् प्रमृ॒णन्त॒ मज्म॒ जय॑न्त॒म् जय॑न्त॒ मज्म॑ प्रमृ॒णन्त᳚म् । \newline
24. अज्म॑ प्रमृ॒णन्त॑म् प्रमृ॒णन्त॒ मज्माज्म॑ प्रमृ॒णन्त॒ मोज॒ सौज॑सा प्रमृ॒णन्त॒ मज्मा ज्म॑ प्रमृ॒णन्त॒ मोज॑सा । \newline
25. प्र॒मृ॒णन्त॒ मोज॒ सौज॑सा प्रमृ॒णन्त॑म् प्रमृ॒णन्त॒ मोज॑सा । \newline
26. प्र॒मृ॒णन्त॒मिति॑ प्र - मृ॒णन्त᳚म् । \newline
27. ओज॒सेत्योज॑सा । \newline
28. इ॒मꣳ स॑जाताः सजाता इ॒म मि॒मꣳ स॑जाता॒ अन्वनु॑ सजाता इ॒म मि॒मꣳ स॑जाता॒ अनु॑ । \newline
29. स॒जा॒ता॒ अन्वनु॑ सजाताः सजाता॒ अनु॑ वीरयद्ध्वं ॅवीरयद्ध्व॒ मनु॑ सजाताः सजाता॒ अनु॑ वीरयद्ध्वम् । \newline
30. स॒जा॒ता॒ इति॑ स - जा॒ताः॒ । \newline
31. अनु॑ वीरयद्ध्वं ॅवीरयद्ध्व॒ मन्वनु॑ वीरयद्ध्व॒ मिन्द्र॒ मिन्द्रं॑ ॅवीरयद्ध्व॒ मन्वनु॑ वीरयद्ध्व॒ मिन्द्र᳚म् । \newline
32. वी॒र॒य॒द्ध्व॒ मिन्द्र॒ मिन्द्रं॑ ॅवीरयद्ध्वं ॅवीरयद्ध्व॒ मिन्द्रꣳ॑ सखायः सखाय॒ इन्द्रं॑ ॅवीरयद्ध्वं ॅवीरयद्ध्व॒ मिन्द्रꣳ॑ सखायः । \newline
33. इन्द्रꣳ॑ सखायः सखाय॒ इन्द्र॒ मिन्द्रꣳ॑ सखा॒यो ऽन्वनु॑ सखाय॒ इन्द्र॒ मिन्द्रꣳ॑ सखा॒यो ऽनु॑ । \newline
34. स॒खा॒यो ऽन्वनु॑ सखायः सखा॒यो ऽनु॒ सꣳ स मनु॑ सखायः सखा॒यो ऽनु॒ सम् । \newline
35. अनु॒ सꣳ स मन्वनु॒ सꣳ र॑भद्ध्वꣳ रभद्ध्वꣳ॒॒ स मन्वनु॒ सꣳ र॑भद्ध्वम् । \newline
36. सꣳ र॑भद्ध्वꣳ रभद्ध्वꣳ॒॒ सꣳ सꣳ र॑भद्ध्वम् । \newline
37. र॒भ॒द्ध्व॒मिति॑ रभद्ध्वम् । \newline
38. ब॒ल॒वि॒ज्ञा॒यः स्थवि॑रः॒ स्थवि॑रो बलविज्ञा॒यो ब॑लविज्ञा॒यः स्थवि॑रः॒ प्रवी॑रः॒ प्रवी॑रः॒ स्थवि॑रो बलविज्ञा॒यो ब॑लविज्ञा॒यः स्थवि॑रः॒ प्रवी॑रः । \newline
39. ब॒ल॒वि॒ज्ञा॒य इति॑ बल - वि॒ज्ञा॒यः । \newline
40. स्थवि॑रः॒ प्रवी॑रः॒ प्रवी॑रः॒ स्थवि॑रः॒ स्थवि॑रः॒ प्रवी॑रः॒ सह॑स्वा॒न् थ्सह॑स्वा॒न् प्रवी॑रः॒ स्थवि॑रः॒ स्थवि॑रः॒ प्रवी॑रः॒ सह॑स्वान् । \newline
41. प्रवी॑रः॒ सह॑स्वा॒न् थ्सह॑स्वा॒न् प्रवी॑रः॒ प्रवी॑रः॒ सह॑स्वान्. वा॒जी वा॒जी सह॑स्वा॒न् प्रवी॑रः॒ प्रवी॑रः॒ सह॑स्वान्. वा॒जी । \newline
42. प्रवी॑र॒ इति॒ प्र - वी॒रः॒ । \newline
43. सह॑स्वान्. वा॒जी वा॒जी सह॑स्वा॒न् थ्सह॑स्वान्. वा॒जी सह॑मानः॒ सह॑मानो वा॒जी सह॑स्वा॒न् थ्सह॑स्वान्. वा॒जी सह॑मानः । \newline
44. वा॒जी सह॑मानः॒ सह॑मानो वा॒जी वा॒जी सह॑मान उ॒ग्र उ॒ग्रः सह॑मानो वा॒जी वा॒जी सह॑मान उ॒ग्रः । \newline
45. सह॑मान उ॒ग्र उ॒ग्रः सह॑मानः॒ सह॑मान उ॒ग्रः । \newline
46. उ॒ग्र इत्यु॒ग्रः । \newline
47. अ॒भिवी॑रो अ॒भिस॑त्वा॒ ऽभिस॑त्वा॒ ऽभिवी॑रो अ॒भिवी॑रो अ॒भिस॑त्वा सहो॒जाः स॑हो॒जा अ॒भिस॑त्वा॒ ऽभिवी॑रो अ॒भिवी॑रो अ॒भिस॑त्वा सहो॒जाः । \newline
48. अ॒भिवी॑र॒ इत्य॒भि - वी॒रः॒ । \newline
49. अ॒भिस॑त्वा सहो॒जाः स॑हो॒जा अ॒भिस॑त्वा॒ ऽभिस॑त्वा सहो॒जा जैत्र॒म् जैत्रꣳ॑ सहो॒जा अ॒भिस॑त्वा॒ ऽभिस॑त्वा सहो॒जा जैत्र᳚म् । \newline
50. अ॒भिस॒त्वेत्य॒भि - स॒त्वा॒ । \newline
51. स॒हो॒जा जैत्र॒म् जैत्रꣳ॑ सहो॒जाः स॑हो॒जा जैत्र॑ मिन्द्रेन्द्र॒ जैत्रꣳ॑ सहो॒जाः स॑हो॒जा जैत्र॑ मिन्द्र । \newline
52. स॒हो॒जा इति॑ सहः - जाः । \newline
53. जैत्र॑ मिन्द्रेन्द्र॒ जैत्र॒म् जैत्र॑ मिन्द्र॒ रथꣳ॒॒ रथ॑ मिन्द्र॒ जैत्र॒म् जैत्र॑ मिन्द्र॒ रथ᳚म् । \newline
54. इ॒न्द्र॒ रथꣳ॒॒ रथ॑ मिन्द्रेन्द्र॒ रथ॒ मा रथ॑ मिन्द्रेन्द्र॒ रथ॒ मा । \newline
55. रथ॒ मा रथꣳ॒॒ रथ॒ मा ति॑ष्ठ ति॒ष्ठा रथꣳ॒॒ रथ॒ मा ति॑ष्ठ । \newline
56. आ ति॑ष्ठ ति॒ष्ठा ति॑ष्ठ गो॒विद् गो॒वित् ति॒ष्ठा ति॑ष्ठ गो॒वित् । \newline
57. ति॒ष्ठ॒ गो॒विद् गो॒वित् ति॑ष्ठ तिष्ठ गो॒वित् । \newline
58. गो॒विदिति॑ गो - वित् । \newline
59. अ॒भि गो॒त्राणि॑ गो॒त्राण्य॒भ्य॑भि गो॒त्राणि॒ सह॑सा॒ सह॑सा गो॒त्राण्य॒भ्य॑भि गो॒त्राणि॒ सह॑सा । \newline
60. गो॒त्राणि॒ सह॑सा॒ सह॑सा गो॒त्राणि॑ गो॒त्राणि॒ सह॑सा॒ गाह॑मानो॒ गाह॑मानः॒ सह॑सा गो॒त्राणि॑ गो॒त्राणि॒ सह॑सा॒ गाह॑मानः । \newline
61. सह॑सा॒ गाह॑मानो॒ गाह॑मानः॒ सह॑सा॒ सह॑सा॒ गाह॑मानो ऽदा॒यो अ॑दा॒यो गाह॑मानः॒ सह॑सा॒ सह॑सा॒ गाह॑मानो ऽदा॒यः । \newline
62. गाह॑मानो ऽदा॒यो अ॑दा॒यो गाह॑मानो॒ गाह॑मानो ऽदा॒यो वी॒रो वी॒रो अ॑दा॒यो गाह॑मानो॒ गाह॑मानो ऽदा॒यो वी॒रः । \newline
63. अ॒दा॒यो वी॒रो वी॒रो अ॑दा॒यो अ॑दा॒यो वी॒रः श॒तम॑न्युः श॒तम॑न्युर् वी॒रो अ॑दा॒यो अ॑दा॒यो वी॒रः श॒तम॑न्युः । \newline
\pagebreak
\markright{ TS 4.6.4.3  \hfill https://www.vedavms.in \hfill}

\section{ TS 4.6.4.3 }

\textbf{TS 4.6.4.3 } \newline
\textbf{Samhita Paata} \newline

वी॒रः श॒तम॑न्यु॒रिन्द्रः॑ । दु॒श्च्य॒व॒नः पृ॑तना॒षाड॑ यु॒द्ध्यो᳚-स्माकꣳ॒॒ सेना॑ अवतु॒ प्र यु॒थ्सु ॥ इन्द्र॑ आसां-ने॒ता बृह॒स्पति॒ र्दक्षि॑णा य॒ज्ञ्ः पु॒र ए॑तु॒ सोमः॑ । दे॒व॒से॒नाना॑-मभिभञ्जती॒नां जय॑न्तीनां म॒रुतो॑ य॒न्त्वग्रे᳚ ॥ इन्द्र॑स्य॒ वृष्णो॒ वरु॑णस्य॒ राज्ञ्॑ आदि॒त्यानां᳚ म॒रुताꣳ॒॒ शर्द्ध॑ उ॒ग्रं । म॒हाम॑नसां भुवनच्य॒वानां॒ घोषो॑ दे॒वानां॒ जय॑ता॒ मुद॑स्थात् ॥ अ॒स्माक॒-मिन्द्रः॒ समृ॑तेषु ध्व॒जेष्व॒स्माकं॒ ॅया इष॑व॒स्ता ज॑यन्तु । \newline

\textbf{Pada Paata} \newline

वी॒रः । श॒तम॑न्यु॒रिति॑ श॒त - म॒न्युः॒ । इन्द्रः॑ ॥ दु॒श्च्य॒व॒न इति॑ दुः - च्य॒व॒नः । पृ॒त॒ना॒षाट् । अ॒यु॒द्ध्यः । अ॒स्माक᳚म् । सेनाः᳚ । अ॒व॒तु॒ । प्रेति॑ । यु॒थ्स्विति॑ युत् - सु ॥ इन्द्रः॑ । आ॒सा॒म् । ने॒ता । बृह॒स्पतिः॑ । दक्षि॑णा । य॒ज्ञ्ः । पु॒रः । ए॒तु॒ । सोमः॑ ॥ दे॒व॒से॒नाना॒मिति॑ देव - से॒नाना᳚म् । अ॒भि॒भ॒ञ्ज॒ती॒नामित्य॑भि - भ॒ञ्ज॒ती॒नाम् । जय॑न्तीनाम् । म॒रुतः॑ । य॒न्तु॒ । अग्रे᳚ ॥ इन्द्र॑स्य । वृष्णः॑ । वरु॑णस्य । राज्ञ्ः॑ । आ॒दि॒त्याना᳚म् । म॒रुता᳚म् । शर्द्धः॑ । उ॒ग्रम् ॥ म॒हाम॑नसा॒मिति॑ म॒हा - म॒न॒सा॒म् । भु॒व॒न॒च्य॒वाना॒मिति॑ भुवन - च्य॒वाना᳚म् । घोषः॑ । दे॒वाना᳚म् । जय॑ताम् । उदिति॑ । अ॒स्था॒त् ॥ अ॒स्माक᳚म् । इन्द्रः॑ । समृ॑ते॒ष्विति॒ सं - ऋ॒ते॒षु॒ । ध्व॒जेषु॑ । अ॒स्माक᳚म् । याः । इष॑वः । ता: । ज॒य॒न्तु॒ ॥  \newline


\textbf{Krama Paata} \newline

वी॒रः श॒तम॑न्युः । श॒तम॑न्यु॒रिन्द्रः॑ । श॒तम॑न्यु॒रिति॑ श॒त - म॒न्युः॒ । इन्द्र॒ इतीन्द्रः॑ ॥ दु॒श्च्य॒व॒नः पृ॑तना॒षाट् । दु॒श्च्य॒व॒न इति॑ दुः - च्य॒व॒नः । पृ॒त॒ना॒षाड॑यु॒द्ध्यः । अ॒यु॒द्ध्यो᳚ऽस्माक᳚म् । अ॒स्माकꣳ॒॒ सेनाः᳚ । सेना॑ अवतु । अ॒व॒तु॒ प्र । प्र यु॒थ्सु । यु॒थ्स्विति॑ युत् - सु ॥ इन्द्र॑ आसाम् । आ॒सा॒म् ने॒ता । ने॒ता बृह॒स्पतिः॑ । बृह॒स्पति॒र् दक्षि॑णा । दक्षि॑णा य॒ज्ञ्ः । य॒ज्ञ्ः पु॒रः । पु॒र ए॑तु । ए॒तु॒ सोमः॑ । सोम॒ इति॒ सोमः॑ ॥ दे॒व॒से॒नाना॑मभिभञ्जती॒नाम् । दे॒व॒से॒नाना॒मिति॑ देव - से॒नाना᳚म् । अ॒भि॒भ॒ञ्ज॒ती॒नाम् जय॑न्तीनाम् । अ॒भि॒भ॒ञ्ज॒ती॒नामित्य॑भि - भ॒ञ्ज॒ती॒नाम् । जय॑न्तीनाम् म॒रुतः॑ । म॒रुतो॑ यन्तु । य॒न्त्वग्रे᳚ । अग्र॒ इत्यग्रे᳚ ॥ इन्द्र॑स्य॒ वृष्णः॑ । वृष्णो॒ वरु॑णस्य । वरु॑णस्य॒ राज्ञ्ः॑ । राज्ञ्॑ आदि॒त्याना᳚म् । आ॒दि॒त्याना᳚म् म॒रुता᳚म् । म॒रुताꣳ॒॒ शर्द्धः॑ । शर्द्ध॑ उ॒ग्रम् । उ॒ग्रमित्यु॒ग्रम् ॥ म॒हाम॑नसाम् भुवनच्य॒वाना᳚म् । म॒हाम॑नसा॒मिति॑ म॒हा - म॒न॒सा॒म् । भु॒व॒न॒च्य॒वाना॒म् घोषः॑ । भु॒व॒न॒च्य॒वाना॒मिति॑ भुवन - च्य॒वाना᳚म् । घोषो॑ दे॒वाना᳚म् । दे॒वाना॒म् जय॑ताम् । जय॑ता॒मुत् । उद॑स्थात् । अ॒स्था॒दित्य॑स्थात् ॥ अ॒स्माक॒मिन्द्रः॑ । इन्द्रः॒ समृ॑तेषु । समृ॑तेषु ध्व॒जेषु॑ । समृ॑ते॒ष्विति॒ सम् - ऋ॒ते॒षु॒ । ध्व॒जेष्व॒स्माक᳚म् । अ॒स्माकं॒ ॅयाः । या इष॑वः । इष॑व॒स्ताः । ता ज॑यन्तु । ज॒य॒न्त्विति॑ जयन्तु । \newline

\textbf{Jatai Paata} \newline

1. वी॒रः श॒तम॑न्युः श॒तम॑न्युर् वी॒रो वी॒रः श॒तम॑न्युः । \newline
2. श॒तम॑न्यु॒ रिन्द्र॒ इन्द्रः॑ श॒तम॑न्युः श॒तम॑न्यु॒ रिन्द्रः॑ । \newline
3. श॒तम॑न्यु॒रिति॑ श॒त - म॒न्युः॒ । \newline
4. इन्द्र॒ इतीन्द्रः॑ । \newline
5. दु॒श्च्य॒व॒नः पृ॑तना॒षाट् पृ॑तना॒षाड् दु॑श्च्यव॒नो दु॑श्च्यव॒नः पृ॑तना॒षाट् । \newline
6. दु॒श्च्य॒व॒न इति॑ दुः - च्य॒व॒नः । \newline
7. पृ॒त॒ना॒षा ड॑यु॒द्ध्यो॑ ऽयु॒द्ध्यः पृ॑तना॒षाट् पृ॑तना॒षा ड॑यु॒द्ध्यः । \newline
8. अ॒यु॒द्ध्यो᳚ ऽस्माक॑ म॒स्माक॑ मयु॒द्ध्यो॑ ऽयु॒द्ध्यो᳚ ऽस्माक᳚म् । \newline
9. अ॒स्माकꣳ॒॒ सेनाः॒ सेना॑ अ॒स्माक॑ म॒स्माकꣳ॒॒ सेनाः᳚ । \newline
10. सेना॑ अव त्ववतु॒ सेनाः॒ सेना॑ अवतु । \newline
11. अ॒व॒तु॒ प्र प्राव॑त्ववतु॒ प्र । \newline
12. प्र यु॒थ्सु यु॒थ्सु प्र प्र यु॒थ्सु । \newline
13. यु॒थ्‌स्विति॑ युत् - सु । \newline
14. इन्द्र॑ आसा मासा॒ मिन्द्र॒ इन्द्र॑ आसाम् । \newline
15. आ॒सा॒म् ने॒ता ने॒ता ऽऽसा॑ मासाम् ने॒ता । \newline
16. ने॒ता बृह॒स्पति॒र् बृह॒स्पति॑र् ने॒ता ने॒ता बृह॒स्पतिः॑ । \newline
17. बृह॒स्पति॒र् दक्षि॑णा॒ दक्षि॑णा॒ बृह॒स्पति॒र् बृह॒स्पति॒र् दक्षि॑णा । \newline
18. दक्षि॑णा य॒ज्ञो य॒ज्ञो दक्षि॑णा॒ दक्षि॑णा य॒ज्ञ्ः । \newline
19. य॒ज्ञ्ः पु॒रः पु॒रो य॒ज्ञो य॒ज्ञ्ः पु॒रः । \newline
20. पु॒र ए᳚त्वेतु पु॒रः पु॒र ए॑तु । \newline
21. ए॒तु॒ सोमः॒ सोम॑ एत्वेतु॒ सोमः॑ । \newline
22. सोम॒ इति॒ सोमः॑ । \newline
23. दे॒व॒से॒नाना॑ मभिभञ्जती॒ना म॑भिभञ्जती॒नाम् दे॑वसे॒नाना᳚म् देवसे॒नाना॑ मभिभञ्जती॒नाम् । \newline
24. दे॒व॒से॒नाना॒मिति॑ देव - से॒नाना᳚म् । \newline
25. अ॒भि॒भ॒ञ्ज॒ती॒नाम् जय॑न्तीना॒म् जय॑न्तीना मभिभञ्जती॒ना म॑भिभञ्जती॒नाम् जय॑न्तीनाम् । \newline
26. अ॒भि॒भ॒ञ्ज॒ती॒नामित्य॑भि - भ॒ञ्ज॒ती॒नाम् । \newline
27. जय॑न्तीनाम् म॒रुतो॑ म॒रुतो॒ जय॑न्तीना॒म् जय॑न्तीनाम् म॒रुतः॑ । \newline
28. म॒रुतो॑ यन्तु यन्तु म॒रुतो॑ म॒रुतो॑ यन्तु । \newline
29. य॒न्त्वग्रे॒ अग्रे॑ यन्तु य॒न्त्वग्रे᳚ । \newline
30. अग्र॒ इत्यग्रे᳚ । \newline
31. इन्द्र॑स्य॒ वृष्णो॒ वृष्ण॒ इन्द्र॒स्ये न्द्र॑स्य॒ वृष्णः॑ । \newline
32. वृष्णो॒ वरु॑णस्य॒ वरु॑णस्य॒ वृष्णो॒ वृष्णो॒ वरु॑णस्य । \newline
33. वरु॑णस्य॒ राज्ञो॒ राज्ञो॒ वरु॑णस्य॒ वरु॑णस्य॒ राज्ञ्ः॑ । \newline
34. राज्ञ्॑ आदि॒त्याना॑ मादि॒त्यानाꣳ॒॒ राज्ञो॒ राज्ञ्॑ आदि॒त्याना᳚म् । \newline
35. आ॒दि॒त्याना᳚म् म॒रुता᳚म् म॒रुता॑ मादि॒त्याना॑ मादि॒त्याना᳚म् म॒रुता᳚म् । \newline
36. म॒रुताꣳ॒॒ शर्द्धः॒ शर्द्धो॑ म॒रुता᳚म् म॒रुताꣳ॒॒ शर्द्धः॑ । \newline
37. शर्द्ध॑ उ॒ग्र मु॒ग्रꣳ शर्द्धः॒ शर्द्ध॑ उ॒ग्रम् । \newline
38. उ॒ग्रमित्यु॒ग्रम् । \newline
39. म॒हाम॑नसाम् भुवनच्य॒वाना᳚म् भुवनच्य॒वाना᳚म् म॒हाम॑नसाम् म॒हाम॑नसाम् भुवनच्य॒वाना᳚म् । \newline
40. म॒हाम॑नसा॒मिति॑ म॒हा - म॒न॒सा॒म् । \newline
41. भु॒व॒न॒च्य॒वाना॒म् घोषो॒ घोषो॑ भुवनच्य॒वाना᳚म् भुवनच्य॒वाना॒म् घोषः॑ । \newline
42. भु॒व॒न॒च्य॒वाना॒मिति॑ भुवन - च्य॒वाना᳚म् । \newline
43. घोषो॑ दे॒वाना᳚म् दे॒वाना॒म् घोषो॒ घोषो॑ दे॒वाना᳚म् । \newline
44. दे॒वाना॒म् जय॑ता॒म् जय॑ताम् दे॒वाना᳚म् दे॒वाना॒म् जय॑ताम् । \newline
45. जय॑ता॒ मुदुज् जय॑ता॒म् जय॑ता॒ मुत् । \newline
46. उद॑स्था दस्था॒ दुदु द॑स्थात् । \newline
47. अ॒स्था॒दित्य॑स्थात् । \newline
48. अ॒स्माक॒ मिन्द्र॒ इन्द्रो॒ ऽस्माक॑ म॒स्माक॒ मिन्द्रः॑ । \newline
49. इन्द्रः॒ समृ॑तेषु॒ समृ॑ते॒ ष्विन्द्र॒ इन्द्रः॒ समृ॑तेषु । \newline
50. समृ॑तेषु ध्व॒जेषु॑ ध्व॒जेषु॒ समृ॑तेषु॒ समृ॑तेषु ध्व॒जेषु॑ । \newline
51. समृ॑ते॒ष्विति॒ सं - ऋ॒ते॒षु॒ । \newline
52. ध्व॒जे ष्व॒स्माक॑ म॒स्माक॑म् ध्व॒जेषु॑ ध्व॒जे ष्व॒स्माक᳚म् । \newline
53. अ॒स्माकं॒ ॅया या अ॒स्माक॑ म॒स्माकं॒ ॅयाः । \newline
54. या इष॑व॒ इष॑वो॒ या या इष॑वः । \newline
55. इष॑व॒ स्ता स्ता इष॑व॒ इष॑व॒ स्ताः । \newline
56. ता ज॑यन्तु जयन्तु॒ ता स्ता ज॑यन्तु । \newline
57. ज॒य॒न्त्विति॑ जयन्तु । \newline

\textbf{Ghana Paata } \newline

1. वी॒रः श॒तम॑न्युः श॒तम॑न्युर् वी॒रो वी॒रः श॒तम॑न्यु॒ रिन्द्र॒ इन्द्रः॑ श॒तम॑न्युर् वी॒रो वी॒रः श॒तम॑न्यु॒ रिन्द्रः॑ । \newline
2. श॒तम॑न्यु॒ रिन्द्र॒ इन्द्रः॑ श॒तम॑न्युः श॒तम॑न्यु॒ रिन्द्रः॑ । \newline
3. श॒तम॑न्यु॒रिति॑ श॒त - म॒न्युः॒ । \newline
4. इन्द्र॒ इतीन्द्रः॑ । \newline
5. दु॒श्च्य॒व॒नः पृ॑तना॒षाट् पृ॑तना॒षाड् दु॑श्च्यव॒नो दु॑श्च्यव॒नः पृ॑तना॒षा ड॑यु॒द्ध्यो॑ ऽयु॒द्ध्यः पृ॑तना॒षाड् दु॑श्च्यव॒नो दु॑श्च्यव॒नः पृ॑तना॒षा ड॑यु॒द्ध्यः । \newline
6. दु॒श्च्य॒व॒न इति॑ दुः - च्य॒व॒नः । \newline
7. पृ॒त॒ना॒षा ड॑यु॒द्ध्यो॑ ऽयु॒द्ध्यः पृ॑तना॒षाट् पृ॑तना॒षा ड॑यु॒द्ध्यो᳚ ऽस्माक॑ म॒स्माक॑ मयु॒द्ध्यः पृ॑तना॒षाट् पृ॑तना॒षा ड॑यु॒द्ध्यो᳚ ऽस्माक᳚म् । \newline
8. अ॒यु॒द्ध्यो᳚ ऽस्माक॑ म॒स्माक॑ मयु॒द्ध्यो॑ ऽयु॒द्ध्यो᳚ ऽस्माकꣳ॒॒ सेनाः॒ सेना॑ अ॒स्माक॑ मयु॒द्ध्यो॑ ऽयु॒द्ध्यो᳚ ऽस्माकꣳ॒॒ सेनाः᳚ । \newline
9. अ॒स्माकꣳ॒॒ सेनाः॒ सेना॑ अ॒स्माक॑ म॒स्माकꣳ॒॒ सेना॑ अवत्ववतु॒ सेना॑ अ॒स्माक॑ म॒स्माकꣳ॒॒ सेना॑ अवतु । \newline
10. सेना॑ अवत्ववतु॒ सेनाः॒ सेना॑ अवतु॒ प्र प्राव॑तु॒ सेनाः॒ सेना॑ अवतु॒ प्र । \newline
11. अ॒व॒तु॒ प्र प्राव॑ त्ववतु॒ प्र यु॒थ्सु यु॒थ्सु प्राव॑ त्ववतु॒ प्र यु॒थ्सु । \newline
12. प्र यु॒थ्सु यु॒थ्सु प्र प्र यु॒थ्सु । \newline
13. यु॒थ्स्विति॑ युत् - सु । \newline
14. इन्द्र॑ आसा मासा॒ मिन्द्र॒ इन्द्र॑ आसाम् ने॒ता ने॒ता ऽऽसा॒ मिन्द्र॒ इन्द्र॑ आसाम् ने॒ता । \newline
15. आ॒सा॒म् ने॒ता ने॒ता ऽऽसा॑ मासाम् ने॒ता बृह॒स्पति॒र् बृह॒स्पति॑र् ने॒ता ऽऽसा॑ मासाम् ने॒ता बृह॒स्पतिः॑ । \newline
16. ने॒ता बृह॒स्पति॒र् बृह॒स्पति॑र् ने॒ता ने॒ता बृह॒स्पति॒र् दक्षि॑णा॒ दक्षि॑णा॒ बृह॒स्पति॑र् ने॒ता ने॒ता बृह॒स्पति॒र् दक्षि॑णा । \newline
17. बृह॒स्पति॒र् दक्षि॑णा॒ दक्षि॑णा॒ बृह॒स्पति॒र् बृह॒स्पति॒र् दक्षि॑णा य॒ज्ञो य॒ज्ञो दक्षि॑णा॒ बृह॒स्पति॒र् बृह॒स्पति॒र् दक्षि॑णा य॒ज्ञ्ः । \newline
18. दक्षि॑णा य॒ज्ञो य॒ज्ञो दक्षि॑णा॒ दक्षि॑णा य॒ज्ञ्ः पु॒रः पु॒रो य॒ज्ञो दक्षि॑णा॒ दक्षि॑णा य॒ज्ञ्ः पु॒रः । \newline
19. य॒ज्ञ्ः पु॒रः पु॒रो य॒ज्ञो य॒ज्ञ्ः पु॒र ए᳚त्वेतु पु॒रो य॒ज्ञो य॒ज्ञ्ः पु॒र ए॑तु । \newline
20. पु॒र ए᳚त्वेतु पु॒रः पु॒र ए॑तु॒ सोमः॒ सोम॑ एतु पु॒रः पु॒र ए॑तु॒ सोमः॑ । \newline
21. ए॒तु॒ सोमः॒ सोम॑ एत्वेतु॒ सोमः॑ । \newline
22. सोम॒ इति॒ सोमः॑ । \newline
23. दे॒व॒से॒नाना॑ मभिभञ्जती॒ना म॑भिभञ्जती॒नाम् दे॑वसे॒नाना᳚म् देवसे॒नाना॑ मभिभञ्जती॒नाम् जय॑न्तीना॒म् जय॑न्तीना मभिभञ्जती॒नाम् दे॑वसे॒नाना᳚म् देवसे॒नाना॑ मभिभञ्जती॒नाम् जय॑न्तीनाम् । \newline
24. दे॒व॒से॒नाना॒मिति॑ देव - से॒नाना᳚म् । \newline
25. अ॒भि॒भ॒ञ्ज॒ती॒नाम् जय॑न्तीना॒म् जय॑न्तीना मभिभञ्जती॒ना म॑भिभञ्जती॒नाम् जय॑न्तीनाम् म॒रुतो॑ म॒रुतो॒ जय॑न्तीना मभिभञ्जती॒ना म॑भिभञ्जती॒नाम् जय॑न्तीनाम् म॒रुतः॑ । \newline
26. अ॒भि॒भ॒ञ्ज॒ती॒नामित्य॑भि - भ॒ञ्ज॒ती॒नाम् । \newline
27. जय॑न्तीनाम् म॒रुतो॑ म॒रुतो॒ जय॑न्तीना॒म् जय॑न्तीनाम् म॒रुतो॑ यन्तु यन्तु म॒रुतो॒ जय॑न्तीना॒म् जय॑न्तीनाम् म॒रुतो॑ यन्तु । \newline
28. म॒रुतो॑ यन्तु यन्तु म॒रुतो॑ म॒रुतो॑ य॒न्त्वग्रे॒ अग्रे॑ यन्तु म॒रुतो॑ म॒रुतो॑ य॒न्त्वग्रे᳚ । \newline
29. य॒न्त्वग्रे॒ अग्रे॑ यन्तु य॒न्त्वग्रे᳚ । \newline
30. अग्र॒ इत्यग्रे᳚ । \newline
31. इन्द्र॑स्य॒ वृष्णो॒ वृष्ण॒ इन्द्र॒ स्येन्द्र॑स्य॒ वृष्णो॒ वरु॑णस्य॒ वरु॑णस्य॒ वृष्ण॒ 
इन्द्र॒ स्येन्द्र॑स्य॒ वृष्णो॒ वरु॑णस्य । \newline
32. वृष्णो॒ वरु॑णस्य॒ वरु॑णस्य॒ वृष्णो॒ वृष्णो॒ वरु॑णस्य॒ राज्ञो॒ राज्ञो॒ वरु॑णस्य॒ वृष्णो॒ वृष्णो॒ वरु॑णस्य॒ राज्ञ्ः॑ । \newline
33. वरु॑णस्य॒ राज्ञो॒ राज्ञो॒ वरु॑णस्य॒ वरु॑णस्य॒ राज्ञ्॑ आदि॒त्याना॑ मादि॒त्यानाꣳ॒॒ राज्ञो॒ वरु॑णस्य॒ वरु॑णस्य॒ राज्ञ्॑ आदि॒त्याना᳚म् । \newline
34. राज्ञ्॑ आदि॒त्याना॑ मादि॒त्यानाꣳ॒॒ राज्ञो॒ राज्ञ्॑ आदि॒त्याना᳚म् म॒रुता᳚म् म॒रुता॑ मादि॒त्यानाꣳ॒॒ राज्ञो॒ राज्ञ्॑ आदि॒त्याना᳚म् म॒रुता᳚म् । \newline
35. आ॒दि॒त्याना᳚म् म॒रुता᳚म् म॒रुता॑ मादि॒त्याना॑ मादि॒त्याना᳚म् म॒रुताꣳ॒॒ शर्द्धः॒ शर्द्धो॑ म॒रुता॑ मादि॒त्याना॑ मादि॒त्याना᳚म् म॒रुताꣳ॒॒ शर्द्धः॑ । \newline
36. म॒रुताꣳ॒॒ शर्द्धः॒ शर्द्धो॑ म॒रुता᳚म् म॒रुताꣳ॒॒ शर्द्ध॑ उ॒ग्र मु॒ग्रꣳ शर्द्धो॑ म॒रुता᳚म् म॒रुताꣳ॒॒ शर्द्ध॑ उ॒ग्रम् । \newline
37. शर्द्ध॑ उ॒ग्र मु॒ग्रꣳ शर्द्धः॒ शर्द्ध॑ उ॒ग्रम् । \newline
38. उ॒ग्रमित्यु॒ग्रम् । \newline
39. म॒हाम॑नसाम् भुवनच्य॒वाना᳚म् भुवनच्य॒वाना᳚म् म॒हाम॑नसाम् म॒हाम॑नसाम् भुवनच्य॒वाना॒म् घोषो॒ घोषो॑ भुवनच्य॒वाना᳚म् म॒हाम॑नसाम् म॒हाम॑नसाम् भुवनच्य॒वाना॒म् घोषः॑ । \newline
40. म॒हाम॑नसा॒मिति॑ म॒हा - म॒न॒सा॒म् । \newline
41. भु॒व॒न॒च्य॒वाना॒म् घोषो॒ घोषो॑ भुवनच्य॒वाना᳚म् भुवनच्य॒वाना॒म् घोषो॑ दे॒वाना᳚म् दे॒वाना॒म् घोषो॑ भुवनच्य॒वाना᳚म् भुवनच्य॒वाना॒म् घोषो॑ दे॒वाना᳚म् । \newline
42. भु॒व॒न॒च्य॒वाना॒मिति॑ भुवन - च्य॒वाना᳚म् । \newline
43. घोषो॑ दे॒वाना᳚म् दे॒वाना॒म् घोषो॒ घोषो॑ दे॒वाना॒म् जय॑ता॒म् जय॑ताम् दे॒वाना॒म् घोषो॒ घोषो॑ दे॒वाना॒म् जय॑ताम् । \newline
44. दे॒वाना॒म् जय॑ता॒म् जय॑ताम् दे॒वाना᳚म् दे॒वाना॒म् जय॑ता॒ मुदुज् जय॑ताम् दे॒वाना᳚म् दे॒वाना॒म् जय॑ता॒ मुत् । \newline
45. जय॑ता॒ मुदुज् जय॑ता॒म् जय॑ता॒ मुद॑स्था दस्था॒दुज् जय॑ता॒म् जय॑ता॒ मुद॑स्थात् । \newline
46. उद॑स्था दस्था॒ दुदु द॑स्थात् । \newline
47. अ॒स्था॒दित्य॑स्थात् । \newline
48. अ॒स्माक॒ मिन्द्र॒ इन्द्रो॒ ऽस्माक॑ म॒स्माक॒ मिन्द्रः॒ समृ॑तेषु॒ समृ॑ते॒ ष्विन्द्रो॒ ऽस्माक॑ म॒स्माक॒ मिन्द्रः॒ समृ॑तेषु । \newline
49. इन्द्रः॒ समृ॑तेषु॒ समृ॑ते॒ ष्विन्द्र॒ इन्द्रः॒ समृ॑तेषु ध्व॒जेषु॑ ध्व॒जेषु॒ समृ॑ते॒ ष्विन्द्र॒ इन्द्रः॒ समृ॑तेषु ध्व॒जेषु॑ । \newline
50. समृ॑तेषु ध्व॒जेषु॑ ध्व॒जेषु॒ समृ॑तेषु॒ समृ॑तेषु ध्व॒जे ष्व॒स्माक॑ म॒स्माक॑म् ध्व॒जेषु॒ समृ॑तेषु॒ समृ॑तेषु ध्व॒जे ष्व॒स्माक᳚म् । \newline
51. समृ॑ते॒ष्विति॒ सं - ऋ॒ते॒षु॒ । \newline
52. ध्व॒जे ष्व॒स्माक॑ म॒स्माक॑म् ध्व॒जेषु॑ ध्व॒जे ष्व॒स्माकं॒ ॅया या अ॒स्माक॑म् ध्व॒जेषु॑ ध्व॒जे ष्व॒स्माकं॒ ॅयाः । \newline
53. अ॒स्माकं॒ ॅया या अ॒स्माक॑ म॒स्माकं॒ ॅया इष॑व॒ इष॑वो॒ या अ॒स्माक॑ म॒स्माकं॒ ॅया इष॑वः । \newline
54. या इष॑व॒ इष॑वो॒ या या इष॑व॒ स्ता स्ता इष॑वो॒ या या इष॑व॒ स्ताः । \newline
55. इष॑व॒ स्ता स्ता इष॑व॒ इष॑व॒ स्ता ज॑यन्तु जयन्तु॒ ता इष॑व॒ इष॑व॒ स्ता ज॑यन्तु । \newline
56. ता ज॑यन्तु जयन्तु॒ ता स्ता ज॑यन्तु । \newline
57. ज॒य॒न्त्विति॑ जयन्तु । \newline
\pagebreak
\markright{ TS 4.6.4.4  \hfill https://www.vedavms.in \hfill}

\section{ TS 4.6.4.4 }

\textbf{TS 4.6.4.4 } \newline
\textbf{Samhita Paata} \newline

अ॒स्माकं॑ ॅवी॒रा उत्त॑रे भवन्त्व॒स्मानु॑ देवा अवता॒ हवे॑षु ॥ उद्ध॑र्.षय मघव॒न्ना-यु॑धा॒-न्युथ्सत्व॑नां माम॒कानां॒ महाꣳ॑सि । उद्वृ॑त्रहन् वा॒जिनां॒ ॅवाजि॑ना॒न्युद्-रथा॑नां॒ जय॑तामेतु॒ घोषः॑ ॥ उप॒ प्रेत॒ जय॑ता नरःस्थि॒रा वः॑ सन्तु बा॒हवः॑ । इन्द्रो॑ वः॒ शर्म॑ यच्छ त्वना-धृ॒ष्या यथास॑थ ॥ अव॑सृष्टा॒ परा॑ पत॒ शर॑व्ये॒ ब्रह्म॑ सꣳशिता । गच्छा॒मित्रा॒न् प्र- [  ] \newline

\textbf{Pada Paata} \newline

अ॒स्माक᳚म् । वी॒राः । उत्त॑र॒ इत्युत् - त॒रे॒ । भ॒व॒न्तु॒ । अ॒स्मान् । उ॒ । दे॒वाः॒ । अ॒व॒त॒ । हवे॑षु ॥ उदिति॑ । ह॒र्॒.ष॒य॒ । म॒घ॒व॒न्निति॑ मघ - व॒न्न् । आयु॑धानि । उदिति॑ । सत्व॑नाम् । मा॒म॒काना᳚म् । महाꣳ॑सि ॥ उदिति॑ । वृ॒त्र॒ह॒न्निति॑ वृत्र-ह॒न्न् । वा॒जिना᳚म् । वाजि॑नानि । उदिति॑ । रथा॑नाम् । जय॑ताम् । ए॒तु॒ । घोषः॑ ॥ उप॑ । प्रेति॑ । इ॒त॒ । जय॑त । न॒रः॒ । स्थि॒राः । वः॒ । स॒न्तु॒ । बा॒हवः॑ ॥ इन्द्रः॑ । वः॒ । शर्म॑ । य॒च्छ॒तु॒ । अ॒ना॒धृ॒ष्या इत्य॑ना - धृ॒ष्याः । यथा᳚ । अस॑थ ॥ अव॑सृ॒ष्टेत्यव॑ - सृ॒ष्टा॒ । परेति॑ । प॒त॒ । शर॑व्ये । ब्रह्म॑सꣳशि॒तेति॒ ब्रह्म॑-सꣳ॒॒शि॒ता॒ ॥ गच्छ॑ । अ॒मित्रान्॑ । प्रेति॑ ।  \newline


\textbf{Krama Paata} \newline

अ॒स्माकं॑ ॅवी॒राः । वी॒रा उत्त॑रे । उत्त॑रे भवन्तु । उत्त॑र॒ इत्युत् - त॒रे॒ । भ॒व॒न्त्व॒स्मान् । अ॒स्मानु॑ । उ॒ दे॒वाः॒ । दे॒वा॒ अ॒व॒त॒ । अ॒व॒ता॒ हवे॑षु । हवे॒ष्विति॒ हवे॑षु ॥ उद्ध॑र्.षय । ह॒र्॒.ष॒य॒ म॒घ॒व॒न्न्॒ । म॒घ॒व॒न्नायु॑धानि । म॒घ॒व॒न्निति॑ मघ - व॒न्न्॒ । आयु॑धा॒न्युत् । उथ् सत्व॑नाम् । सत्व॑नाम् माम॒काना᳚म् । मा॒म॒काना॒म् महाꣳ॑सि । महाꣳ॑सीति॒ महाꣳ॑सि ॥ उद् वृ॑त्रहन्न् । वृ॒त्र॒ह॒न्॒. वा॒जिना᳚म् । वृ॒त्र॒ह॒न्निति॑ वृत्र - ह॒न्न्॒ । वा॒जिनां॒ ॅवाजि॑नानि । वाजि॑ना॒न्युत् । उद् रथा॑नाम् । रथा॑ना॒म् जय॑ताम् । जय॑तामेतु । ए॒तु॒ घोषः॑ । घोष॒ इति॒ घोषः॑ ॥ उप॒ प्र । प्रेत॑ । इ॒त॒ जय॑त । जय॑ता नरः । न॒रः॒ स्थि॒राः । स्थि॒रा वः॑ । वः॒ स॒न्तु॒ । स॒न्तु॒ बा॒हवः॑ । बा॒हव॒ इति॑ बा॒हवः॑ ॥ इन्द्रो॑ वः । वः॒ शर्म॑ । शर्म॑ यच्छतु । य॒च्छ॒त्व॒ना॒धृ॒ष्याः । अ॒ना॒धृ॒ष्या यथा᳚ । अ॒ना॒धृ॒ष्या इत्य॑ना - धृ॒ष्याः । यथाऽस॑थ । अस॒थेत्यस॑थ ॥ अव॑सृष्टा॒ परा᳚ । अव॑सृ॒ष्टेत्यव॑ - सृ॒ष्टा॒ । परा॑ पत । प॒त॒ शर॑व्ये । शर॑व्ये॒ ब्रह्म॑सꣳशिता । ब्रह्म॑सꣳशि॒तेति॒ ब्रह्म॑ - सꣳ॒॒शि॒ता॒ ॥ गच्छा॒मित्रान्॑ । अ॒मित्रा॒न् प्र ( ) । प्र वि॑श \newline

\textbf{Jatai Paata} \newline

1. अ॒स्माकं॑ ॅवी॒रा वी॒रा अ॒स्माक॑ म॒स्माकं॑ ॅवी॒राः । \newline
2. वी॒रा उत्त॑र॒ उत्त॑रे वी॒रा वी॒रा उत्त॑रे । \newline
3. उत्त॑रे भवन्तु भव॒न्तूत्त॑र॒ उत्त॑रे भवन्तु । \newline
4. उत्त॑र॒ इत्युत् - त॒रे॒ । \newline
5. भ॒व॒न्त्व॒स्मा न॒स्मान् भ॑वन्तु भव न्त्व॒स्मान् । \newline
6. अ॒स्मा नु॑ वु व॒स्मा न॒स्मा नु॑ । \newline
7. उ॒ दे॒वा॒ दे॒वा॒ उ॒ वु॒ दे॒वाः॒ । \newline
8. दे॒वा॒ अ॒व॒ता॒ व॒त॒ दे॒वा॒ दे॒वा॒ अ॒व॒त॒ । \newline
9. अ॒व॒ता॒ हवे॑षु॒ हवे᳚ ष्ववता वता॒ हवे॑षु । \newline
10. हवे॒ष्विति॒ हवे॑षु । \newline
11. उद्धर्॑.षय हर्.ष॒यो दुद्धर्॑.षय । \newline
12. ह॒र्॒.ष॒य॒ म॒घ॒व॒न् म॒घ॒व॒न्॒. ह॒र्॒.ष॒य॒ ह॒र्॒.ष॒य॒ म॒घ॒व॒न्न् । \newline
13. म॒घ॒व॒न् नायु॑धा॒ न्यायु॑धानि मघवन् मघव॒न् नायु॑धानि । \newline
14. म॒घ॒व॒न्निति॑ मघ - व॒न्न् । \newline
15. आयु॑धा॒ न्युदु दायु॑धा॒ न्यायु॑धा॒ न्युत् । \newline
16. उथ् सत्व॑नाꣳ॒॒ सत्व॑ना॒ मुदुथ् सत्व॑नाम् । \newline
17. सत्व॑नाम् माम॒काना᳚म् माम॒कानाꣳ॒॒ सत्व॑नाꣳ॒॒ सत्व॑नाम् माम॒काना᳚म् । \newline
18. मा॒म॒काना॒म् महाꣳ॑सि॒ महाꣳ॑सि माम॒काना᳚म् माम॒काना॒म् महाꣳ॑सि । \newline
19. महाꣳ॑सीति॒ महाꣳ॑सि । \newline
20. उद् वृ॑त्रहन् वृत्रह॒न् नुदुद् वृ॑त्रहन्न् । \newline
21. वृ॒त्र॒ह॒न्॒. वा॒जिनां᳚ ॅवा॒जिनां᳚ ॅवृत्रहन् वृत्रहन्. वा॒जिना᳚म् । \newline
22. वृ॒त्र॒ह॒न्निति॑ वृत्र - ह॒न्न् । \newline
23. वा॒जिनां॒ ॅवाजि॑नानि॒ वाजि॑नानि वा॒जिनां᳚ ॅवा॒जिनां॒ ॅवाजि॑नानि । \newline
24. वाजि॑ना॒ न्युदुद् वाजि॑नानि॒ वाजि॑ना॒ न्युत् । \newline
25. उद् रथा॑नाꣳ॒॒ रथा॑ना॒ मुदुद् रथा॑नाम् । \newline
26. रथा॑ना॒म् जय॑ता॒म् जय॑ताꣳ॒॒ रथा॑नाꣳ॒॒ रथा॑ना॒म् जय॑ताम् । \newline
27. जय॑ता मेत्वेतु॒ जय॑ता॒म् जय॑ता मेतु । \newline
28. ए॒तु॒ घोषो॒ घोष॑ एत्वेतु॒ घोषः॑ । \newline
29. घोष॒ इति॒ घोषः॑ । \newline
30. उप॒ प्र प्रोपोप॒ प्र । \newline
31. प्रेते॑त॒ प्र प्रेत॑ । \newline
32. इ॒त॒ जय॑त॒ जय॑ते ते त॒ जय॑त । \newline
33. जय॑ता नरो नरो॒ जय॑त॒ जय॑ता नरः । \newline
34. न॒रः॒ स्थि॒राः स्थि॒रा न॑रो नरः स्थि॒राः । \newline
35. स्थि॒रा वो॑ वः स्थि॒राः स्थि॒रा वः॑ । \newline
36. वः॒ स॒न्तु॒ स॒न्तु॒ वो॒ वः॒ स॒न्तु॒ । \newline
37. स॒न्तु॒ बा॒हवो॑ बा॒हवः॑ सन्तु सन्तु बा॒हवः॑ । \newline
38. बा॒हव॒ इति॑ बा॒हवः॑ । \newline
39. इन्द्रो॑ वो व॒ इन्द्र॒ इन्द्रो॑ वः । \newline
40. वः॒ शर्म॒ शर्म॑ वो वः॒ शर्म॑ । \newline
41. शर्म॑ यच्छतु यच्छतु॒ शर्म॒ शर्म॑ यच्छतु । \newline
42. य॒च्छ॒ त्व॒ना॒धृ॒ष्या अ॑नाधृ॒ष्या य॑च्छतु यच्छ त्वनाधृ॒ष्याः । \newline
43. अ॒ना॒धृ॒ष्या यथा॒ यथा॑ ऽनाधृ॒ष्या अ॑नाधृ॒ष्या यथा᳚ । \newline
44. अ॒ना॒धृ॒ष्या इत्य॑ना - धृ॒ष्याः । \newline
45. यथा ऽस॒था स॑थ॒ यथा॒ यथा ऽस॑थ । \newline
46. अस॒थेत्यस॑थ । \newline
47. अव॑सृष्टा॒ परा॒ परा ऽव॑सृ॒ष्टा ऽव॑सृष्टा॒ परा᳚ । \newline
48. अव॑सृ॒ष्टेत्यव॑ - सृ॒ष्टा॒ । \newline
49. परा॑ पत पत॒ परा॒ परा॑ पत । \newline
50. प॒त॒ शर॑व्ये॒ शर॑व्ये पत पत॒ शर॑व्ये । \newline
51. शर॑व्ये॒ ब्रह्म॑सꣳशिता॒ ब्रह्म॑सꣳशिता॒ शर॑व्ये॒ शर॑व्ये॒ ब्रह्म॑सꣳशिता । \newline
52. ब्रह्म॑सꣳशि॒तेति॒ ब्रह्म॑ - सꣳ॒॒शि॒ता॒ । \newline
53. गच्छा॒ मित्रा॑ न॒मित्रा॒न् गच्छ॒ गच्छा॒ मित्रान्॑ । \newline
54. अ॒मित्रा॒न् प्र प्रामित्रा॑ न॒मित्रा॒न् प्र । \newline
55. प्र वि॑श विश॒ प्र प्र वि॑श । \newline

\textbf{Ghana Paata } \newline

1. अ॒स्माकं॑ ॅवी॒रा वी॒रा अ॒स्माक॑ म॒स्माकं॑ ॅवी॒रा उत्त॑र॒ उत्त॑रे वी॒रा अ॒स्माक॑ म॒स्माकं॑ ॅवी॒रा उत्त॑रे । \newline
2. वी॒रा उत्त॑र॒ उत्त॑रे वी॒रा वी॒रा उत्त॑रे भवन्तु भव॒न्तूत्त॑रे वी॒रा वी॒रा उत्त॑रे भवन्तु । \newline
3. उत्त॑रे भवन्तु भव॒न्तूत्त॑र॒ उत्त॑रे भवन्त्व॒स्मा न॒स्मान् भ॑व॒न्तूत्त॑र॒ उत्त॑रे भवन्त्व॒स्मान् । \newline
4. उत्त॑र॒ इत्युत् - त॒रे॒ । \newline
5. भ॒व॒न्त्व॒स्मा न॒स्मान् भ॑वन्तु भवन्त्व॒स्मा नु॑ वु व॒स्मान् भ॑वन्तु भवन्त्व॒स्मा नु॑ । \newline
6. अ॒स्मा नु॑ वु व॒स्मा न॒स्मा नु॑ देवा देवा उ व॒स्मा न॒स्मा नु॑ देवाः । \newline
7. उ॒ दे॒वा॒ दे॒वा॒ उ॒ वु॒ दे॒वा॒ अ॒व॒ता॒ व॒त॒ दे॒वा॒ उ॒ वु॒ दे॒वा॒ अ॒व॒त॒ । \newline
8. दे॒वा॒ अ॒व॒ता॒ व॒त॒ दे॒वा॒ दे॒वा॒ अ॒व॒ता॒ हवे॑षु॒ हवे᳚ष्ववत देवा देवा अवता॒ हवे॑षु । \newline
9. अ॒व॒ता॒ हवे॑षु॒ हवे᳚ष्ववता वता॒ हवे॑षु । \newline
10. हवे॒ष्विति॒ हवे॑षु । \newline
11. उद्धर्॑.षय हर्.ष॒योदु द्धर्॑.षय मघवन् मघवन्. हर्.ष॒यो दुद्धर्॑.षय मघवन्न् । \newline
12. ह॒र्॒.ष॒य॒ म॒घ॒व॒न् म॒घ॒व॒न्॒. ह॒र्॒.ष॒य॒ ह॒र्॒.ष॒य॒ म॒घ॒व॒न् नायु॑धा॒ न्यायु॑धानि मघवन्. हर्.षय हर्.षय मघव॒न् नायु॑धानि । \newline
13. म॒घ॒व॒न् नायु॑धा॒ न्यायु॑धानि मघवन् मघव॒न् नायु॑धा॒ न्युदु दायु॑धानि मघवन् मघव॒न् नायु॑धा॒न्युत् । \newline
14. म॒घ॒व॒न्निति॑ मघ - व॒न्न् । \newline
15. आयु॑धा॒ न्युदु दायु॑धा॒ न्यायु॑धा॒ न्युथ् सत्व॑नाꣳ॒॒ सत्व॑ना॒ मुदायु॑धा॒ न्यायु॑धा॒ न्युथ् सत्व॑नाम् । \newline
16. उथ् सत्व॑नाꣳ॒॒ सत्व॑ना॒ मुदुथ् सत्व॑नाम् माम॒काना᳚म् माम॒कानाꣳ॒॒ सत्व॑ना॒ मुदुथ् सत्व॑नाम् माम॒काना᳚म् । \newline
17. सत्व॑नाम् माम॒काना᳚म् माम॒कानाꣳ॒॒ सत्व॑नाꣳ॒॒ सत्व॑नाम् माम॒काना॒म् महाꣳ॑सि॒ महाꣳ॑सि माम॒कानाꣳ॒॒ सत्व॑नाꣳ॒॒ सत्व॑नाम् माम॒काना॒म् महाꣳ॑सि । \newline
18. मा॒म॒काना॒म् महाꣳ॑सि॒ महाꣳ॑सि माम॒काना᳚म् माम॒काना॒म् महाꣳ॑सि । \newline
19. महाꣳ॑सीति॒ महाꣳ॑सि । \newline
20. उद् वृ॑त्रहन् वृत्रह॒न् नुदुद् वृ॑त्रहन्. वा॒जिनां᳚ ॅवा॒जिनां᳚ ॅवृत्रह॒न् नुदुद् वृ॑त्रहन्. वा॒जिना᳚म् । \newline
21. वृ॒त्र॒ह॒न्॒. वा॒जिनां᳚ ॅवा॒जिनां᳚ ॅवृत्रहन् वृत्रहन्. वा॒जिनां॒ ॅवाजि॑नानि॒ वाजि॑नानि वा॒जिनां᳚ ॅवृत्रहन् वृत्रहन्. वा॒जिनां॒ ॅवाजि॑नानि । \newline
22. वृ॒त्र॒ह॒न्निति॑ वृत्र - ह॒न्न् । \newline
23. वा॒जिनां॒ ॅवाजि॑नानि॒ वाजि॑नानि वा॒जिनां᳚ ॅवा॒जिनां॒ ॅवाजि॑ना॒ न्युदुद् वाजि॑नानि वा॒जिनां᳚ ॅवा॒जिनां॒ ॅवाजि॑ना॒न्युत् । \newline
24. वाजि॑ना॒ न्युदुद् वाजि॑नानि॒ वाजि॑ना॒ न्युद् रथा॑नाꣳ॒॒ रथा॑ना॒ मुद् वाजि॑नानि॒ वाजि॑ना॒ न्युद् रथा॑नाम् । \newline
25. उद् रथा॑नाꣳ॒॒ रथा॑ना॒ मुदुद् रथा॑ना॒म् जय॑ता॒म् जय॑ताꣳ॒॒ रथा॑ना॒ मुदुद् रथा॑ना॒म् जय॑ताम् । \newline
26. रथा॑ना॒म् जय॑ता॒म् जय॑ताꣳ॒॒ रथा॑नाꣳ॒॒ रथा॑ना॒म् जय॑ता मेत्वेतु॒ जय॑ताꣳ॒॒ रथा॑नाꣳ॒॒ रथा॑ना॒म् जय॑ता मेतु । \newline
27. जय॑ता मेत्वेतु॒ जय॑ता॒म् जय॑ता मेतु॒ घोषो॒ घोष॑ एतु॒ जय॑ता॒म् जय॑ता मेतु॒ घोषः॑ । \newline
28. ए॒तु॒ घोषो॒ घोष॑ एत्वेतु॒ घोषः॑ । \newline
29. घोष॒ इति॒ घोषः॑ । \newline
30. उप॒ प्र प्रोपोप॒ प्रेते॑त॒ प्रोपोप॒ प्रेत॑ । \newline
31. प्रेते॑त॒ प्र प्रेत॒ जय॑त॒ जय॑तेत॒ प्र प्रेत॒ जय॑त । \newline
32. इ॒त॒ जय॑त॒ जय॑ते तेत॒ जय॑ता नरो नरो॒ जय॑ते तेत॒ जय॑ता नरः । \newline
33. जय॑ता नरो नरो॒ जय॑त॒ जय॑ता नरः स्थि॒राः स्थि॒रा न॑रो॒ जय॑त॒ जय॑ता नरः स्थि॒राः । \newline
34. न॒रः॒ स्थि॒राः स्थि॒रा न॑रो नरः स्थि॒रा वो॑ वः स्थि॒रा न॑रो नरः स्थि॒रा वः॑ । \newline
35. स्थि॒रा वो॑ वः स्थि॒राः स्थि॒रा वः॑ सन्तु सन्तु वः स्थि॒राः स्थि॒रा वः॑ सन्तु । \newline
36. वः॒ स॒न्तु॒ स॒न्तु॒ वो॒ वः॒ स॒न्तु॒ बा॒हवो॑ बा॒हवः॑ सन्तु वो वः सन्तु बा॒हवः॑ । \newline
37. स॒न्तु॒ बा॒हवो॑ बा॒हवः॑ सन्तु सन्तु बा॒हवः॑ । \newline
38. बा॒हव॒ इति॑ बा॒हवः॑ । \newline
39. इन्द्रो॑ वो व॒ इन्द्र॒ इन्द्रो॑ वः॒ शर्म॒ शर्म॑ व॒ इन्द्र॒ इन्द्रो॑ वः॒ शर्म॑ । \newline
40. वः॒ शर्म॒ शर्म॑ वो वः॒ शर्म॑ यच्छतु यच्छतु॒ शर्म॑ वो वः॒ शर्म॑ यच्छतु । \newline
41. शर्म॑ यच्छतु यच्छतु॒ शर्म॒ शर्म॑ यच्छ त्वनाधृ॒ष्या अ॑नाधृ॒ष्या य॑च्छतु॒ शर्म॒ शर्म॑ यच्छ
त्वनाधृ॒ष्याः । \newline
42. य॒च्छ॒ त्व॒ना॒धृ॒ष्या अ॑नाधृ॒ष्या य॑च्छतु यच्छ त्वनाधृ॒ष्या यथा॒ यथा॑ ऽनाधृ॒ष्या य॑च्छतु यच्छ त्वनाधृ॒ष्या यथा᳚ । \newline
43. अ॒ना॒धृ॒ष्या यथा॒ यथा॑ ऽनाधृ॒ष्या अ॑नाधृ॒ष्या यथा ऽस॒था स॑थ॒ यथा॑ ऽनाधृ॒ष्या अ॑नाधृ॒ष्या यथा ऽस॑थ । \newline
44. अ॒ना॒धृ॒ष्या इत्य॑ना - धृ॒ष्याः । \newline
45. यथा ऽस॒था स॑थ॒ यथा॒ यथा ऽस॑थ । \newline
46. अस॒थेत्यस॑थ । \newline
47. अव॑सृष्टा॒ परा॒ परा ऽव॑सृ॒ष्टा ऽव॑सृष्टा॒ परा॑ पत पत॒ परा व॑सृ॒ष्टा ऽव॑सृष्टा॒ परा॑ पत । \newline
48. अव॑सृ॒ष्टेत्यव॑ - सृ॒ष्टा॒ । \newline
49. परा॑ पत पत॒ परा॒ परा॑ पत॒ शर॑व्ये॒ शर॑व्ये पत॒ परा॒ परा॑ पत॒ शर॑व्ये । \newline
50. प॒त॒ शर॑व्ये॒ शर॑व्ये पत पत॒ शर॑व्ये॒ ब्रह्म॑सꣳशिता॒ ब्रह्म॑सꣳशिता॒ शर॑व्ये पत पत॒ शर॑व्ये॒ ब्रह्म॑सꣳशिता । \newline
51. शर॑व्ये॒ ब्रह्म॑सꣳशिता॒ ब्रह्म॑सꣳशिता॒ शर॑व्ये॒ शर॑व्ये॒ ब्रह्म॑सꣳशिता । \newline
52. ब्रह्म॑सꣳशि॒तेति॒ ब्रह्म॑ - सꣳ॒॒शि॒ता॒ । \newline
53. गच्छा॒मित्रा॑ न॒मित्रा॒न् गच्छ॒ गच्छा॒मित्रा॒न् प्र प्रामित्रा॒न् गच्छ॒ गच्छा॒ मित्रा॒न् प्र । \newline
54. अ॒मित्रा॒न् प्र प्रामित्रा॑ न॒मित्रा॒न् प्र वि॑श विश॒ प्रामित्रा॑ न॒मित्रा॒न् प्र वि॑श । \newline
55. प्र वि॑श विश॒ प्र प्र वि॑श॒ मा मा वि॑श॒ प्र प्र वि॑श॒ मा । \newline
\pagebreak
\markright{ TS 4.6.4.5  \hfill https://www.vedavms.in \hfill}

\section{ TS 4.6.4.5 }

\textbf{TS 4.6.4.5 } \newline
\textbf{Samhita Paata} \newline

वि॑श॒ मैषां॒ कञ्च॒नोच्छि॑षः ॥ मर्मा॑णि ते॒ वर्म॑भिश्छादयामि॒ सोम॑स्त्वा॒ राजा॒ ऽमृते॑ना॒भि-व॑स्तां । उ॒रोर्वरी॑यो॒ वरि॑वस्ते अस्तु॒ जय॑न्तं॒ त्वामनु॑ मदन्तु दे॒वाः ॥ यत्र॑ बा॒णाः सं॒पत॑न्ति कुमा॒रा वि॑शि॒खा इ॑व । इन्द्रो॑ न॒स्तत्र॑ वृत्र॒हा वि॑श्वा॒हा शर्म॑ यच्छतु ॥ \newline

\textbf{Pada Paata} \newline

वि॒श॒ । मा । ए॒षा॒म् । कम् । च॒न । उदिति॑ । शि॒षः॒ ॥ मर्मा॑णि । ते॒ । वर्म॑भि॒रिति॒ वर्म॑ -भिः॒ । छा॒द॒या॒मि॒ । सोमः॑ । त्वा॒ । राजा᳚ । अ॒मृते॑न । अ॒भीति॑ । व॒स्ता॒म् ॥ उ॒रोः । वरी॑यः । वरि॑वः । ते॒ । अ॒स्तु॒ । जय॑न्तम् । त्वाम् । अन्विति॑ । म॒द॒न्तु॒ । दे॒वाः ॥ यत्र॑ । बा॒णाः । स॒पंत॒न्तीति॑ सं - पत॑न्ति । कु॒मा॒राः । वि॒शि॒खा इति॑ वि - शि॒खाः । इ॒व॒ ॥ इन्द्रः॑ । नः॒ । तत्र॑ । वृ॒त्र॒हेति॑ वृत्र - हा । वि॒श्वा॒हेति॑ विश्वा - हा । शर्म॑ । य॒च्छ॒तु॒ ॥  \newline


\textbf{Krama Paata} \newline

वि॒श॒ मा । मैषा᳚म् । ए॒षा॒म् कम् । कम् च॒न । च॒नोत् । उच्छि॑षः । शि॒ष॒ इति॑ शिषः ॥ मर्मा॑णि ते । ते॒ वर्म॑भिः । वर्म॑भि श्छादयामि । वर्म॑भि॒रिति॒ वर्म॑ - भिः॒ । छा॒द॒या॒मि॒ सोमः॑ । सोम॑स्त्वा । त्वा॒ राजा᳚ । राजा॒ऽमृते॑न । अ॒मृते॑ना॒भि । अ॒भि व॑स्ताम् । व॒स्ता॒मिति॑ वस्ताम् ॥ उ॒रोर् वरी॑यः । वरी॑यो॒ वरि॑वः । वरि॑वस्ते । ते॒ अ॒स्तु॒ । अ॒स्तु॒ जय॑न्तम् । जय॑न्त॒म् त्वाम् । त्वामनु॑ । अनु॑ मदन्तु । म॒द॒न्तु॒ दे॒वाः । दे॒वा इति॑ दे॒वाः ॥ यत्र॑ बा॒णाः । बा॒णाः स॒म्पत॑न्ति । स॒म्पत॑न्ति कुमा॒राः । स॒म्पत॒न्तीति॑ सम् - पत॑न्ति । कु॒मा॒रा वि॑शि॒खाः । वि॒शि॒खा इ॑व । वि॒शि॒खा इति॑ वि - शि॒खाः । इ॒वेती॑व ॥ इन्द्रो॑ नः । न॒स्तत्र॑ । तत्र॑ वृत्र॒हा । वृ॒त्र॒हा वि॑श्वा॒हा । वृ॒त्र॒हेति॑ वृत्र - हा । वि॒श्वा॒हा शर्म॑ । वि॒श्वा॒हेति॑ विश्व - हा । शर्म॑ यच्छतु । य॒च्छ॒त्विति॑ यच्छतु । \newline

\textbf{Jatai Paata} \newline

1. वि॒श॒ मा मा वि॑श विश॒ मा । \newline
2. मैषा॑ मेषा॒म् मा मैषा᳚म् । \newline
3. ए॒षा॒म् कम् कमे॑षा मेषा॒म् कम् । \newline
4. कम् च॒न च॒न कम् कम् च॒न । \newline
5. च॒नोदुच् च॒न च॒नोत् । \newline
6. उच्छि॑षः शिष॒ उदु च्छि॑षः । \newline
7. शि॒ष॒ इति॑ शिषः । \newline
8. मर्मा॑णि ते ते॒ मर्मा॑णि॒ मर्मा॑णि ते । \newline
9. ते॒ वर्म॑भि॒र् वर्म॑भि स्ते ते॒ वर्म॑भिः । \newline
10. वर्म॑भि श्छादयामि छादयामि॒ वर्म॑भि॒र् वर्म॑भि श्छादयामि । \newline
11. वर्म॑भि॒रिति॒ वर्म॑ - भिः॒ । \newline
12. छा॒द॒या॒मि॒ सोमः॒ सोम॑ श्छादयामि छादयामि॒ सोमः॑ । \newline
13. सोम॑ स्त्वा त्वा॒ सोमः॒ सोम॑ स्त्वा । \newline
14. त्वा॒ राजा॒ राजा᳚ त्वा त्वा॒ राजा᳚ । \newline
15. राजा॒ ऽमृते॑ना॒ मृते॑न॒ राजा॒ राजा॒ ऽमृते॑न । \newline
16. अ॒मृते॑ना॒ भ्या᳚(1॒)भ्य॑ मृते॑ना॒ मृते॑ना॒भि । \newline
17. अ॒भि व॑स्तां ॅवस्ता म॒भ्य॑भि व॑स्ताम् । \newline
18. व॒स्ता॒मिति॑ वस्ताम् । \newline
19. उ॒रोर् वरी॑यो॒ वरी॑य उ॒रो रु॒रोर् वरी॑यः । \newline
20. वरी॑यो॒ वरि॑वो॒ वरि॑वो॒ वरी॑यो॒ वरी॑यो॒ वरि॑वः । \newline
21. वरि॑व स्ते ते॒ वरि॑वो॒ वरि॑व स्ते । \newline
22. ते॒ अ॒स्त्व॒स्तु॒ ते॒ ते॒ अ॒स्तु॒ । \newline
23. अ॒स्तु॒ जय॑न्त॒म् जय॑न्त मस्त्वस्तु॒ जय॑न्तम् । \newline
24. जय॑न्त॒म् त्वाम् त्वाम् जय॑न्त॒म् जय॑न्त॒म् त्वाम् । \newline
25. त्वा मन्वनु॒ त्वाम् त्वा मनु॑ । \newline
26. अनु॑ मदन्तु मद॒ न्त्वन्वनु॑ मदन्तु । \newline
27. म॒द॒न्तु॒ दे॒वा दे॒वा म॑दन्तु मदन्तु दे॒वाः । \newline
28. दे॒वा इति॑ दे॒वाः । \newline
29. यत्र॑ बा॒णा बा॒णा यत्र॒ यत्र॑ बा॒णाः । \newline
30. बा॒णाः सं॒पत॑न्ति सं॒पत॑न्ति बा॒णा बा॒णाः सं॒पत॑न्ति । \newline
31. सं॒पत॑न्ति कुमा॒राः कु॑मा॒राः सं॒पत॑न्ति सं॒पत॑न्ति कुमा॒राः । \newline
32. सं॒पत॒न्तीति॑ सं - पत॑न्ति । \newline
33. कु॒मा॒रा वि॑शि॒खा वि॑शि॒खाः कु॑मा॒राः कु॑मा॒रा वि॑शि॒खाः । \newline
34. वि॒शि॒खा इ॑वेव विशि॒खा वि॑शि॒खा इ॑व । \newline
35. वि॒शि॒खा इति॑ वि - शि॒खाः । \newline
36. इ॒वेती॑व । \newline
37. इन्द्रो॑ नो न॒ इन्द्र॒ इन्द्रो॑ नः । \newline
38. न॒ स्तत्र॒ तत्र॑ नो न॒ स्तत्र॑ । \newline
39. तत्र॑ वृत्र॒हा वृ॑त्र॒हा तत्र॒ तत्र॑ वृत्र॒हा । \newline
40. वृ॒त्र॒हा वि॑श्वा॒हा वि॑श्वा॒हा वृ॑त्र॒हा वृ॑त्र॒हा वि॑श्वा॒हा । \newline
41. वृ॒त्र॒हेति॑ वृत्र - हा । \newline
42. वि॒श्वा॒हा शर्म॒ शर्म॑ विश्वा॒हा वि॑श्वा॒हा शर्म॑ । \newline
43. वि॒श्वा॒हेति॑ विश्व - हा । \newline
44. शर्म॑ यच्छतु यच्छतु॒ शर्म॒ शर्म॑ यच्छतु । \newline
45. य॒च्छ॒त्विति॑ यच्छतु । \newline

\textbf{Ghana Paata } \newline

1. वि॒श॒ मा मा वि॑श विश॒ मैषा॑ मेषा॒म् मा वि॑श विश॒ मैषा᳚म् । \newline
2. मैषा॑ मेषा॒म् मा मैषा॒म् कम् क मे॑षा॒म् मा मैषा॒म् कम् । \newline
3. ए॒षा॒म् कम् क मे॑षा मेषा॒म् कम् च॒न च॒न क मे॑षा मेषा॒म् कम् च॒न । \newline
4. कम् च॒न च॒न कम् कम् च॒नो दुच् च॒न कम् कम् च॒नोत् । \newline
5. च॒नो दुच् च॒न च॒नोच्छि॑षः शिष॒ उच् च॒न च॒नोच्छि॑षः । \newline
6. उच्छि॑षः शिष॒ उदुच्छि॑षः । \newline
7. शि॒ष॒ इति॑ शिषः । \newline
8. मर्मा॑णि ते ते॒ मर्मा॑णि॒ मर्मा॑णि ते॒ वर्म॑भि॒र् वर्म॑भि स्ते॒ मर्मा॑णि॒ मर्मा॑णि ते॒ वर्म॑भिः । \newline
9. ते॒ वर्म॑भि॒र् वर्म॑भि स्ते ते॒ वर्म॑भि श्छादयामि छादयामि॒ वर्म॑भि स्ते ते॒ वर्म॑भि श्छादयामि । \newline
10. वर्म॑भि श्छादयामि छादयामि॒ वर्म॑भि॒र् वर्म॑भि श्छादयामि॒ सोमः॒ सोम॑ श्छादयामि॒ वर्म॑भि॒र् वर्म॑भि श्छादयामि॒ सोमः॑ । \newline
11. वर्म॑भि॒रिति॒ वर्म॑ - भिः॒ । \newline
12. छा॒द॒या॒मि॒ सोमः॒ सोम॑ श्छादयामि छादयामि॒ सोम॑ स्त्वा त्वा॒ सोम॑ श्छादयामि छादयामि॒ सोम॑ स्त्वा । \newline
13. सोम॑ स्त्वा त्वा॒ सोमः॒ सोम॑ स्त्वा॒ राजा॒ राजा᳚ त्वा॒ सोमः॒ सोम॑ स्त्वा॒ राजा᳚ । \newline
14. त्वा॒ राजा॒ राजा᳚ त्वा त्वा॒ राजा॒ ऽमृते॑ना॒ मृते॑न॒ राजा᳚ त्वा त्वा॒ राजा॒ ऽमृते॑न । \newline
15. राजा॒ ऽमृते॑ना॒ मृते॑न॒ राजा॒ राजा॒ ऽमृते॑ना॒ भ्या᳚(1॒)भ्य॑ मृते॑न॒ राजा॒ राजा॒ ऽमृते॑ना॒भि । \newline
16. अ॒मृते॑ना॒ भ्या᳚(1॒)भ्य॑मृते॑ना॒ मृते॑ना॒भि व॑स्तां ॅवस्ता म॒भ्य॑ मृते॑ना॒ मृते॑ना॒भि व॑स्ताम् । \newline
17. अ॒भि व॑स्तां ॅवस्ता म॒भ्य॑भि व॑स्ताम् । \newline
18. व॒स्ता॒मिति॑ वस्ताम् । \newline
19. उ॒रोर् वरी॑यो॒ वरी॑य उ॒रो रु॒रोर् वरी॑यो॒ वरि॑वो॒ वरि॑वो॒ वरी॑य उ॒रो रु॒रोर् वरी॑यो॒ वरि॑वः । \newline
20. वरी॑यो॒ वरि॑वो॒ वरि॑वो॒ वरी॑यो॒ वरी॑यो॒ वरि॑व स्ते ते॒ वरि॑वो॒ वरी॑यो॒ वरी॑यो॒ वरि॑व स्ते । \newline
21. वरि॑व स्ते ते॒ वरि॑वो॒ वरि॑व स्ते अस्त्वस्तु ते॒ वरि॑वो॒ वरि॑व स्ते अस्तु । \newline
22. ते॒ अ॒स्त्व॒स्तु॒ ते॒ ते॒ अ॒स्तु॒ जय॑न्त॒म् जय॑न्त मस्तु ते ते अस्तु॒ जय॑न्तम् । \newline
23. अ॒स्तु॒ जय॑न्त॒म् जय॑न्त मस्त्वस्तु॒ जय॑न्त॒म् त्वाम् त्वाम् जय॑न्त मस्त्वस्तु॒ जय॑न्त॒म् त्वाम् । \newline
24. जय॑न्त॒म् त्वाम् त्वाम् जय॑न्त॒म् जय॑न्त॒म् त्वा मन्वनु॒ त्वाम् जय॑न्त॒म् जय॑न्त॒म् त्वा मनु॑ । \newline
25. त्वा मन्वनु॒ त्वाम् त्वा मनु॑ मदन्तु मद॒न्त्वनु॒ त्वाम् त्वा मनु॑ मदन्तु । \newline
26. अनु॑ मदन्तु मद॒न् त्वन्वनु॑ मदन्तु दे॒वा दे॒वा म॑द॒न् त्वन्वनु॑ मदन्तु दे॒वाः । \newline
27. म॒द॒न्तु॒ दे॒वा दे॒वा म॑दन्तु मदन्तु दे॒वाः । \newline
28. दे॒वा इति॑ दे॒वाः । \newline
29. यत्र॑ बा॒णा बा॒णा यत्र॒ यत्र॑ बा॒णाः सं॒पत॑न्ति सं॒पत॑न्ति बा॒णा यत्र॒ यत्र॑ बा॒णाः सं॒पत॑न्ति । \newline
30. बा॒णाः सं॒पत॑न्ति सं॒पत॑न्ति बा॒णा बा॒णाः सं॒पत॑न्ति कुमा॒राः कु॑मा॒राः सं॒पत॑न्ति बा॒णा बा॒णाः सं॒पत॑न्ति कुमा॒राः । \newline
31. सं॒पत॑न्ति कुमा॒राः कु॑मा॒राः सं॒पत॑न्ति सं॒पत॑न्ति कुमा॒रा वि॑शि॒खा वि॑शि॒खाः कु॑मा॒राः सं॒पत॑न्ति सं॒पत॑न्ति कुमा॒रा वि॑शि॒खाः । \newline
32. सं॒पत॒न्तीति॑ सं - पत॑न्ति । \newline
33. कु॒मा॒रा वि॑शि॒खा वि॑शि॒खाः कु॑मा॒राः कु॑मा॒रा वि॑शि॒खा इ॑वेव विशि॒खाः कु॑मा॒राः कु॑मा॒रा वि॑शि॒खा इ॑व । \newline
34. वि॒शि॒खा इ॑वेव विशि॒खा वि॑शि॒खा इ॑व । \newline
35. वि॒शि॒खा इति॑ वि - शि॒खाः । \newline
36. इ॒वेती॑व । \newline
37. इन्द्रो॑ नो न॒ इन्द्र॒ इन्द्रो॑ न॒ स्तत्र॒ तत्र॑ न॒ इन्द्र॒ इन्द्रो॑ न॒ स्तत्र॑ । \newline
38. न॒ स्तत्र॒ तत्र॑ नो न॒ स्तत्र॑ वृत्र॒हा वृ॑त्र॒हा तत्र॑ नो न॒ स्तत्र॑ वृत्र॒हा । \newline
39. तत्र॑ वृत्र॒हा वृ॑त्र॒हा तत्र॒ तत्र॑ वृत्र॒हा वि॑श्वा॒हा वि॑श्वा॒हा वृ॑त्र॒हा तत्र॒ तत्र॑ वृत्र॒हा वि॑श्वा॒हा । \newline
40. वृ॒त्र॒हा वि॑श्वा॒हा वि॑श्वा॒हा वृ॑त्र॒हा वृ॑त्र॒हा वि॑श्वा॒हा शर्म॒ शर्म॑ विश्वा॒हा वृ॑त्र॒हा वृ॑त्र॒हा वि॑श्वा॒हा शर्म॑ । \newline
41. वृ॒त्र॒हेति॑ वृत्र - हा । \newline
42. वि॒श्वा॒हा शर्म॒ शर्म॑ विश्वा॒हा वि॑श्वा॒हा शर्म॑ यच्छतु यच्छतु॒ शर्म॑ विश्वा॒हा वि॑श्वा॒हा शर्म॑ यच्छतु । \newline
43. वि॒श्वा॒हेति॑ विश्व - हा । \newline
44. शर्म॑ यच्छतु यच्छतु॒ शर्म॒ शर्म॑ यच्छतु । \newline
45. य॒च्छ॒त्विति॑ यच्छतु । \newline
\pagebreak
\markright{ TS 4.6.5.1  \hfill https://www.vedavms.in \hfill}

\section{ TS 4.6.5.1 }

\textbf{TS 4.6.5.1 } \newline
\textbf{Samhita Paata} \newline

प्राची॒मनु॑ प्र॒दिशं॒ प्रेहि॑ वि॒द्वान॒ग्नेर॑ग्ने पु॒रो अ॑ग्निर्भवे॒ह । विश्वा॒ आशा॒ दीद्या॑नो॒ वि भा॒ह्यूर्जं॑ नो धेहि द्वि॒पदे॒ चतु॑ष्पदे ॥ क्रम॑द्ध्वम॒ग्निना॒ नाक॒मुख्यꣳ॒॒ हस्ते॑षु॒ बिभ्र॑तः । दि॒वः पृ॒ष्ठꣳ सुव॑र्ग॒त्वा मि॒श्रा दे॒वेभि॑राद्ध्वं ॥ पृ॒थि॒व्या अ॒हमुद॒न्तरि॑क्ष॒-माऽरु॑ह-म॒न्तरि॑क्षा॒द् दिव॒माऽरु॑हं ।दि॒वो नाक॑स्य पृ॒ष्ठाथ् सुव॒र्ज्योति॑रगा- [  ] \newline

\textbf{Pada Paata} \newline

प्राची᳚म् । अन्विति॑ । प्र॒दिश॒मिति॑ प्र-दिश᳚म् । प्रेति॑ । इ॒हि॒ । वि॒द्वान् । अ॒ग्नेः । अ॒ग्ने॒ । पु॒रो अ॑ग्नि॒रिति॑ पु॒रः - अ॒ग्निः॒ । भ॒व॒ । इ॒ह ॥ विश्वाः᳚ । आशाः᳚ । दीद्या॑नः । वीति॑ । भा॒हि॒ । ऊर्ज᳚म् । नः॒ । धे॒हि॒ । द्वि॒पद॒ इति॑ द्वि - पदे᳚ । चतु॑ष्पद॒ इति॒ चतुः॑ - प॒दे॒ ॥ क्रम॑द्ध्वम् । अ॒ग्निना᳚ । नाक᳚म् । उख्य᳚म् । हस्ते॑षु । बिभ्र॑तः ॥ दि॒वः । पृ॒ष्ठम् । सुवः॑ । ग॒त्वा । मि॒श्राः । दे॒वेभिः॑ । आ॒द्ध्व॒म् ॥ पृ॒थि॒व्याः । अ॒हम् । उदिति॑ । अ॒न्तरि॑क्षम् । एति॑ । अ॒रु॒ह॒म् । अ॒न्तरि॑क्षात् । दिव᳚म् । एति॑ । अ॒रु॒ह॒म् ॥ दि॒वः । नाक॑स्य । पृ॒ष्ठात् । सुवः॑ । ज्योतिः॑ । अ॒गा॒म् ।  \newline


\textbf{Krama Paata} \newline

प्राची॒मनु॑ । अनु॑ प्र॒दिश᳚म् । प्र॒दिश॒म् प्र । प्र॒दिश॒मिति॑ प्र - दिश᳚म् । प्रेहि॑ । इ॒हि॒ वि॒द्वान् । वि॒द्वान॒ग्नेः । अ॒ग्नेर॑ग्ने । अ॒ग्ने॒ पु॒रोअ॑ग्निः । पु॒रोअ॑ग्निर् भव । पु॒रोअ॑ग्नि॒रिति॑ पु॒रः - अ॒ग्निः॒ । भ॒वे॒ह । इ॒हेती॒ह ॥ विश्वा॒ आशाः᳚ । आशा॒ दीद्या॑नः । दीद्या॑नो॒ वि । वि भा॑हि । भा॒ह्यूर्ज᳚म् । ऊर्ज॑म् नः । नो॒ धे॒हि॒ । धे॒हि॒ द्वि॒पदे᳚ । द्वि॒पदे॒ चतु॑ष्पदे । द्वि॒पद॒ इति॑ द्वि - पदे᳚ । चतु॑ष्पद॒ इति॒ चतुः॑ - प॒दे॒ ॥ क्रम॑द्ध्वम॒ग्निना᳚ । अ॒ग्निना॒ नाक᳚म् । नाक॒मुख्य᳚म् । उख्यꣳ॒॒ हस्ते॑षु । हस्ते॑षु॒ बिभ्र॑तः । बिभ्र॑त॒ इति॒ बिभ्र॑तः ॥ दि॒वः पृ॒ष्ठम् । पृ॒ष्ठꣳ सुवः॑ । सुव॑र् ग॒त्वा । ग॒त्वा मि॒श्राः । मि॒श्रा दे॒वेभिः॑ । दे॒वेभि॑राद्ध्वम् । आ॒द्ध्व॒मित्या᳚द्ध्वम् ॥ पृ॒थि॒व्या अ॒हम् । अ॒हमुत् । उद॒न्तरि॑क्षम् । अ॒न्तरि॑क्ष॒मा । आऽरु॑हम् । अ॒रु॒ह॒म॒न्तरि॑क्षात् । अ॒न्तरि॑क्षा॒द् दिव᳚म् । दिव॒मा । आऽरु॑हम् । अ॒रु॒ह॒मित्य॑रुहम् ॥ दि॒वो नाक॑स्य । नाक॑स्य पृ॒ष्ठात् । पृ॒ष्ठाथ् सुवः॑ । सुव॒र् ज्योतिः॑ । ज्योति॑रगाम् । अ॒गा॒म॒हम् \newline

\textbf{Jatai Paata} \newline

1. प्राची॒ मन्वनु॒ प्राची॒म् प्राची॒ मनु॑ । \newline
2. अनु॑ प्र॒दिश॑म् प्र॒दिश॒ मन्वनु॑ प्र॒दिश᳚म् । \newline
3. प्र॒दिश॒म् प्र प्र प्र॒दिश॑म् प्र॒दिश॒म् प्र । \newline
4. प्र॒दिश॒मिति॑ प्र - दिश᳚म् । \newline
5. प्रेही॑हि॒ प्र प्रेहि॑ । \newline
6. इ॒हि॒ वि॒द्वान्. वि॒द्वा नि॑हीहि वि॒द्वान् । \newline
7. वि॒द्वा न॒ग्ने र॒ग्नेर् वि॒द्वान्. वि॒द्वा न॒ग्नेः । \newline
8. अ॒ग्ने र॑ग्ने अग्ने अ॒ग्ने र॒ग्ने र॑ग्ने । \newline
9. अ॒ग्ने॒ पु॒रोअ॑ग्निः पु॒रोअ॑ग्नि रग्ने अग्ने पु॒रोअ॑ग्निः । \newline
10. पु॒रोअ॑ग्निर् भव भव पु॒रोअ॑ग्निः पु॒रोअ॑ग्निर् भव । \newline
11. पु॒रोअ॑ग्नि॒रिति॑ पु॒रः - अ॒ग्निः॒ । \newline
12. भ॒वे॒ हेह भ॑व भवे॒ह । \newline
13. इ॒हेती॒ह । \newline
14. विश्वा॒ आशा॒ आशा॒ विश्वा॒ विश्वा॒ आशाः᳚ । \newline
15. आशा॒ दीद्या॑नो॒ दीद्या॑न॒ आशा॒ आशा॒ दीद्या॑नः । \newline
16. दीद्या॑नो॒ वि वि दीद्या॑नो॒ दीद्या॑नो॒ वि । \newline
17. वि भा॑हि भाहि॒ वि वि भा॑हि । \newline
18. भा॒ह्यूर्ज॒ मूर्ज॑म् भाहि भा॒ह्यूर्ज᳚म् । \newline
19. ऊर्ज॑म् नो न॒ ऊर्ज॒ मूर्ज॑म् नः । \newline
20. नो॒ धे॒हि॒ धे॒हि॒ नो॒ नो॒ धे॒हि॒ । \newline
21. धे॒हि॒ द्वि॒पदे᳚ द्वि॒पदे॑ धेहि धेहि द्वि॒पदे᳚ । \newline
22. द्वि॒पदे॒ चतु॑ष्पदे॒ चतु॑ष्पदे द्वि॒पदे᳚ द्वि॒पदे॒ चतु॑ष्पदे । \newline
23. द्वि॒पद॒ इति॑ द्वि - पदे᳚ । \newline
24. चतु॑ष्पद॒ इति॒ चतुः॑ - प॒दे॒ । \newline
25. क्रम॑द्ध्व म॒ग्निना॒ ऽग्निना॒ क्रम॑द्ध्व॒म् क्रम॑द्ध्व म॒ग्निना᳚ । \newline
26. अ॒ग्निना॒ नाक॒म् नाक॑ म॒ग्निना॒ ऽग्निना॒ नाक᳚म् । \newline
27. नाक॒ मुख्य॒ मुख्य॒म् नाक॒म् नाक॒ मुख्य᳚म् । \newline
28. उख्यꣳ॒॒ हस्ते॑षु॒ हस्ते॒षूख्य॒ मुख्यꣳ॒॒ हस्ते॑षु । \newline
29. हस्ते॑षु॒ बिभ्र॑तो॒ बिभ्र॑तो॒ हस्ते॑षु॒ हस्ते॑षु॒ बिभ्र॑तः । \newline
30. बिभ्र॑त॒ इति॒ बिभ्र॑तः । \newline
31. दि॒वः पृ॒ष्ठम् पृ॒ष्ठम् दि॒वो दि॒वः पृ॒ष्ठम् । \newline
32. पृ॒ष्ठꣳ सुवः॒ सुवः॑ पृ॒ष्ठम् पृ॒ष्ठꣳ सुवः॑ । \newline
33. सुव॑र् ग॒त्वा ग॒त्वा सुवः॒ सुव॑र् ग॒त्वा । \newline
34. ग॒त्वा मि॒श्रा मि॒श्रा ग॒त्वा ग॒त्वा मि॒श्राः । \newline
35. मि॒श्रा दे॒वेभि॑र् दे॒वेभि॑र् मि॒श्रा मि॒श्रा दे॒वेभिः॑ । \newline
36. दे॒वेभि॑ राद्ध्व माद्ध्वम् दे॒वेभि॑र् दे॒वेभि॑ राद्ध्वम् । \newline
37. आ॒द्ध्व॒मित्या᳚द्ध्वम् । \newline
38. पृ॒थि॒व्या अ॒ह म॒हम् पृ॑थि॒व्याः पृ॑थि॒व्या अ॒हम् । \newline
39. अ॒ह मुदुद॒ह म॒ह मुत् । \newline
40. उद॒न्तरि॑क्ष म॒न्तरि॑क्ष॒ मुदु द॒न्तरि॑क्षम् । \newline
41. अ॒न्तरि॑क्ष॒ मा ऽन्तरि॑क्ष म॒न्तरि॑क्ष॒ मा । \newline
42. आ ऽरु॑ह मरुह॒ मा ऽरु॑हम् । \newline
43. अ॒रु॒ह॒ म॒न्तरि॑क्षा द॒न्तरि॑क्षा दरुह मरुह म॒न्तरि॑क्षात् । \newline
44. अ॒न्तरि॑क्षा॒द् दिव॒म् दिव॑ म॒न्तरि॑क्षा द॒न्तरि॑क्षा॒द् दिव᳚म् । \newline
45. दिव॒मा दिव॒म् दिव॒मा । \newline
46. आ ऽरु॑ह मरुह॒ मा ऽरु॑हम् । \newline
47. अ॒रु॒ह॒मित्य॑रुहम् । \newline
48. दि॒वो नाक॑स्य॒ नाक॑स्य दि॒वो दि॒वो नाक॑स्य । \newline
49. नाक॑स्य पृ॒ष्ठात् पृ॒ष्ठान् नाक॑स्य॒ नाक॑स्य पृ॒ष्ठात् । \newline
50. पृ॒ष्ठाथ् सुवः॒ सुवः॑ पृ॒ष्ठात् पृ॒ष्ठाथ् सुवः॑ । \newline
51. सुव॒र् ज्योति॒र् ज्योतिः॒ सुवः॒ सुव॒र् ज्योतिः॑ । \newline
52. ज्योति॑ रगा मगा॒म् ज्योति॒र् ज्योति॑ रगाम् । \newline
53. अ॒गा॒ म॒ह म॒ह म॑गा मगा म॒हम् । \newline

\textbf{Ghana Paata } \newline

1. प्राची॒ मन्वनु॒ प्राची॒म् प्राची॒ मनु॑ प्र॒दिश॑म् प्र॒दिश॒ मनु॒ प्राची॒म् प्राची॒ मनु॑ प्र॒दिश᳚म् । \newline
2. अनु॑ प्र॒दिश॑म् प्र॒दिश॒ मन्वनु॑ प्र॒दिश॒म् प्र प्र प्र॒दिश॒ मन्वनु॑ प्र॒दिश॒म् प्र । \newline
3. प्र॒दिश॒म् प्र प्र प्र॒दिश॑म् प्र॒दिश॒म् प्रेही॑हि॒ प्र प्र॒दिश॑म् प्र॒दिश॒म् प्रेहि॑ । \newline
4. प्र॒दिश॒मिति॑ प्र - दिश᳚म् । \newline
5. प्रेही॑हि॒ प्र प्रेहि॑ वि॒द्वान्. वि॒द्वा नि॑हि॒ प्र प्रेहि॑ वि॒द्वान् । \newline
6. इ॒हि॒ वि॒द्वान्. वि॒द्वा नि॑हीहि वि॒द्वा न॒ग्ने र॒ग्नेर् वि॒द्वा नि॑हीहि वि॒द्वा न॒ग्नेः । \newline
7. वि॒द्वा न॒ग्ने र॒ग्नेर् वि॒द्वान्. वि॒द्वा न॒ग्ने र॑ग्ने अग्ने अ॒ग्नेर् वि॒द्वान्. वि॒द्वा न॒ग्ने र॑ग्ने । \newline
8. अ॒ग्ने र॑ग्ने अग्ने अ॒ग्ने र॒ग्ने र॑ग्ने पु॒रोअ॑ग्निः पु॒रोअ॑ग्नि रग्ने अ॒ग्ने र॒ग्ने र॑ग्ने पु॒रोअ॑ग्निः । \newline
9. अ॒ग्ने॒ पु॒रोअ॑ग्निः पु॒रोअ॑ग्नि रग्ने अग्ने पु॒रोअ॑ग्निर् भव भव पु॒रोअ॑ग्नि रग्ने अग्ने पु॒रोअ॑ग्निर् भव । \newline
10. पु॒रोअ॑ग्निर् भव भव पु॒रोअ॑ग्निः पु॒रोअ॑ग्निर् भवे॒ हेह भ॑व पु॒रोअ॑ग्निः पु॒रोअ॑ग्निर् भवे॒ह । \newline
11. पु॒रोअ॑ग्नि॒रिति॑ पु॒रः - अ॒ग्निः॒ । \newline
12. भ॒वे॒ हेह भ॑व भवे॒ह । \newline
13. इ॒हेती॒ह । \newline
14. विश्वा॒ आशा॒ आशा॒ विश्वा॒ विश्वा॒ आशा॒ दीद्या॑नो॒ दीद्या॑न॒ आशा॒ विश्वा॒ विश्वा॒ आशा॒ दीद्या॑नः । \newline
15. आशा॒ दीद्या॑नो॒ दीद्या॑न॒ आशा॒ आशा॒ दीद्या॑नो॒ वि वि दीद्या॑न॒ आशा॒ आशा॒ दीद्या॑नो॒ वि । \newline
16. दीद्या॑नो॒ वि वि दीद्या॑नो॒ दीद्या॑नो॒ वि भा॑हि भाहि॒ वि दीद्या॑नो॒ दीद्या॑नो॒ वि भा॑हि । \newline
17. वि भा॑हि भाहि॒ वि वि भा॒ह्यूर्ज॒ मूर्ज॑म् भाहि॒ वि वि भा॒ह्यूर्ज᳚म् । \newline
18. भा॒ह्यूर्ज॒ मूर्ज॑म् भाहि भा॒ह्यूर्ज॑म् नो न॒ ऊर्ज॑म् भाहि भा॒ह्यूर्ज॑म् नः । \newline
19. ऊर्ज॑म् नो न॒ ऊर्ज॒ मूर्ज॑म् नो धेहि धेहि न॒ ऊर्ज॒ मूर्ज॑म् नो धेहि । \newline
20. नो॒ धे॒हि॒ धे॒हि॒ नो॒ नो॒ धे॒हि॒ द्वि॒पदे᳚ द्वि॒पदे॑ धेहि नो नो धेहि द्वि॒पदे᳚ । \newline
21. धे॒हि॒ द्वि॒पदे᳚ द्वि॒पदे॑ धेहि धेहि द्वि॒पदे॒ चतु॑ष्पदे॒ चतु॑ष्पदे द्वि॒पदे॑ धेहि धेहि द्वि॒पदे॒ चतु॑ष्पदे । \newline
22. द्वि॒पदे॒ चतु॑ष्पदे॒ चतु॑ष्पदे द्वि॒पदे᳚ द्वि॒पदे॒ चतु॑ष्पदे । \newline
23. द्वि॒पद॒ इति॑ द्वि - पदे᳚ । \newline
24. चतु॑ष्पद॒ इति॒ चतुः॑ - प॒दे॒ । \newline
25. क्रम॑द्ध्व म॒ग्निना॒ ऽग्निना॒ क्रम॑द्ध्व॒म् क्रम॑द्ध्व म॒ग्निना॒ नाक॒म् नाक॑ म॒ग्निना॒ क्रम॑द्ध्व॒म् क्रम॑द्ध्व म॒ग्निना॒ नाक᳚म् । \newline
26. अ॒ग्निना॒ नाक॒म् नाक॑ म॒ग्निना॒ ऽग्निना॒ नाक॒ मुख्य॒ मुख्य॒म् नाक॑ म॒ग्निना॒ ऽग्निना॒ नाक॒ मुख्य᳚म् । \newline
27. नाक॒ मुख्य॒ मुख्य॒म् नाक॒म् नाक॒ मुख्यꣳ॒॒ हस्ते॑षु॒ हस्ते॒ षूख्य॒म् नाक॒म् नाक॒ मुख्यꣳ॒॒ हस्ते॑षु । \newline
28. उख्यꣳ॒॒ हस्ते॑षु॒ हस्ते॒षूख्य॒ मुख्यꣳ॒॒ हस्ते॑षु॒ बिभ्र॑तो॒ बिभ्र॑तो॒ हस्ते॒षूख्य॒ मुख्यꣳ॒॒ हस्ते॑षु॒ बिभ्र॑तः । \newline
29. हस्ते॑षु॒ बिभ्र॑तो॒ बिभ्र॑तो॒ हस्ते॑षु॒ हस्ते॑षु॒ बिभ्र॑तः । \newline
30. बिभ्र॑त॒ इति॒ बिभ्र॑तः । \newline
31. दि॒वः पृ॒ष्ठम् पृ॒ष्ठम् दि॒वो दि॒वः पृ॒ष्ठꣳ सुवः॒ सुवः॑ पृ॒ष्ठम् दि॒वो दि॒वः पृ॒ष्ठꣳ सुवः॑ । \newline
32. पृ॒ष्ठꣳ सुवः॒ सुवः॑ पृ॒ष्ठम् पृ॒ष्ठꣳ सुव॑र् ग॒त्वा ग॒त्वा सुवः॑ पृ॒ष्ठम् पृ॒ष्ठꣳ सुव॑र् ग॒त्वा । \newline
33. सुव॑र् ग॒त्वा ग॒त्वा सुवः॒ सुव॑र् ग॒त्वा मि॒श्रा मि॒श्रा ग॒त्वा सुवः॒ सुव॑र् ग॒त्वा मि॒श्राः । \newline
34. ग॒त्वा मि॒श्रा मि॒श्रा ग॒त्वा ग॒त्वा मि॒श्रा दे॒वेभि॑र् दे॒वेभि॑र् मि॒श्रा ग॒त्वा ग॒त्वा मि॒श्रा दे॒वेभिः॑ । \newline
35. मि॒श्रा दे॒वेभि॑र् दे॒वेभि॑र् मि॒श्रा मि॒श्रा दे॒वेभि॑ राद्ध्व माद्ध्वम् दे॒वेभि॑र् मि॒श्रा मि॒श्रा दे॒वेभि॑ राद्ध्वम् । \newline
36. दे॒वेभि॑ राद्ध्व माद्ध्वम् दे॒वेभि॑र् दे॒वेभि॑ राद्ध्वम् । \newline
37. आ॒द्ध्व॒मित्या᳚द्ध्वम् । \newline
38. पृ॒थि॒व्या अ॒ह म॒हम् पृ॑थि॒व्याः पृ॑थि॒व्या अ॒ह मुदु द॒हम् पृ॑थि॒व्याः पृ॑थि॒व्या अ॒ह मुत् । \newline
39. अ॒ह मुदु द॒ह म॒ह मुद॒न्तरि॑क्ष म॒न्तरि॑क्ष॒ मुद॒ह म॒ह मुद॒न्तरि॑क्षम् । \newline
40. उद॒न्तरि॑क्ष म॒न्तरि॑क्ष॒ मुदु द॒न्तरि॑क्ष॒ मा ऽन्तरि॑क्ष॒ मुदुद॒न्तरि॑क्ष॒ मा । \newline
41. अ॒न्तरि॑क्ष॒ मा ऽन्तरि॑क्ष म॒न्तरि॑क्ष॒ मा ऽरु॑ह मरुह॒ मा ऽन्तरि॑क्ष म॒न्तरि॑क्ष॒ मा ऽरु॑हम् । \newline
42. आ ऽरु॑ह मरुह॒ मा ऽरु॑ह म॒न्तरि॑क्षा द॒न्तरि॑क्षा दरुह॒ मा ऽरु॑ह म॒न्तरि॑क्षात् । \newline
43. अ॒रु॒ह॒ म॒न्तरि॑क्षा द॒न्तरि॑क्षा दरुह मरुह म॒न्तरि॑क्षा॒द् दिव॒म् दिव॑ म॒न्तरि॑क्षा दरुह मरुह म॒न्तरि॑क्षा॒द् दिव᳚म् । \newline
44. अ॒न्तरि॑क्षा॒द् दिव॒म् दिव॑ म॒न्तरि॑क्षा द॒न्तरि॑क्षा॒द् दिव॒ मा दिव॑ म॒न्तरि॑क्षा द॒न्तरि॑क्षा॒द् दिव॒ मा । \newline
45. दिव॒ मा दिव॒म् दिव॒ मा ऽरु॑ह मरुह॒ मा दिव॒म् दिव॒ मा ऽरु॑हम् । \newline
46. आ ऽरु॑ह मरुह॒ मा ऽरु॑हम् । \newline
47. अ॒रु॒ह॒मित्य॑रुहम् । \newline
48. दि॒वो नाक॑स्य॒ नाक॑स्य दि॒वो दि॒वो नाक॑स्य पृ॒ष्ठात् पृ॒ष्ठान् नाक॑स्य दि॒वो दि॒वो नाक॑स्य पृ॒ष्ठात् । \newline
49. नाक॑स्य पृ॒ष्ठात् पृ॒ष्ठान् नाक॑स्य॒ नाक॑स्य पृ॒ष्ठाथ् सुवः॒ सुवः॑ पृ॒ष्ठान् नाक॑स्य॒ नाक॑स्य पृ॒ष्ठाथ् सुवः॑ । \newline
50. पृ॒ष्ठाथ् सुवः॒ सुवः॑ पृ॒ष्ठात् पृ॒ष्ठाथ् सुव॒र् ज्योति॒र् ज्योतिः॒ सुवः॑ पृ॒ष्ठात् पृ॒ष्ठाथ् सुव॒र् ज्योतिः॑ । \newline
51. सुव॒र् ज्योति॒र् ज्योतिः॒ सुवः॒ सुव॒र् ज्योति॑ रगा मगा॒म् ज्योतिः॒ सुवः॒ सुव॒र् ज्योति॑ रगाम् । \newline
52. ज्योति॑ रगा मगा॒म् ज्योति॒र् ज्योति॑ रगा म॒ह म॒ह म॑गा॒म् ज्योति॒र् ज्योति॑ रगा म॒हम् । \newline
53. अ॒गा॒ म॒ह म॒ह म॑गा मगा म॒हम् । \newline
\pagebreak
\markright{ TS 4.6.5.2  \hfill https://www.vedavms.in \hfill}

\section{ TS 4.6.5.2 }

\textbf{TS 4.6.5.2 } \newline
\textbf{Samhita Paata} \newline

म॒हं ॥ सुव॒र्यन्तो॒ नापे᳚क्षन्त॒ आ द्याꣳ रो॑हन्ति॒ रोद॑सी । य॒ज्ञ्ं ॅये वि॒श्वतो॑धारꣳ॒॒ सुवि॑द्वाꣳसो वितेनि॒रे ॥ अग्ने॒ प्रेहि॑ प्रथ॒मो दे॑वय॒तां चक्षु॑र्दे॒वाना॑मु॒त मर्त्या॑नां । इय॑क्षमाणा॒ भृगु॑भिः स॒जोषाः॒ सुव॑र्यन्तु॒ यज॑मानाः स्व॒स्ति ॥ नक्तो॒षासा॒ सम॑नसा॒ विरू॑पे धा॒पये॑ते॒ शिशु॒मेकꣳ॑ समी॒ची । द्यावा॒ क्षामा॑ रु॒क्मो अ॒न्तर्वि भा॑ति दे॒वा अ॒ग्निं धा॑रयन् द्रविणो॒दाः ॥ अग्ने॑ सहस्राक्ष - [  ] \newline

\textbf{Pada Paata} \newline

अ॒हम् ॥ सुवः॑ । यन्तः॑ । न । अपेति॑ । ई॒क्ष॒न्ते॒ । एति॑ । द्याम् । रो॒ह॒न्ति॒ । रोद॑सी॒ इति॑ ॥ य॒ज्ञ्म् । ये । वि॒श्वतो॑धार॒मिति॑ वि॒श्वतः॑ - धा॒र॒म् । सुवि॑द्वाꣳस॒ इति॒ सु - वि॒द्वाꣳ॒॒सः॒ । वि॒ते॒नि॒र इति॑ वि - ते॒नि॒रे ॥ अग्ने᳚ । प्रेति॑ । इ॒हि॒ । प्र॒थ॒मः । दे॒व॒य॒तामिति॑ देव - य॒ताम् । चक्षुः॑ । दे॒वाना᳚म् । उ॒त । मर्त्या॑नाम् ॥ इय॑क्षमाणाः । भृगु॑भि॒रिति॒ भृगु॑ - भिः॒ । स॒जोषा॒ इति॑ स-जोषाः᳚ । सुवः॑ । य॒न्तु॒ । यज॑मानाः । स्व॒स्ति ॥ नक्तो॒षासा᳚ । सम॑न॒सेति॒ स - म॒न॒सा॒ । विरू॑पे॒ इति॒ वि - रू॒पे॒ । धा॒पये॑ते॒ इति॑ । शिशु᳚म् । एक᳚म् । स॒मी॒ची इति॑ ॥ द्यावा᳚ । क्षाम॑ । रु॒क्मः । अ॒न्तः । वीति॑ । भा॒ति॒ । दे॒वाः । अ॒ग्निम् । धा॒र॒य॒न्न् । द्र॒वि॒णो॒दा इति॑ द्रविणः - दाः ॥ अग्ने᳚ । स॒ह॒स्रा॒क्षेति॑ सहस्र - अ॒क्ष॒ ।  \newline


\textbf{Krama Paata} \newline

अ॒हमित्य॒हम् ॥ सुव॒र् यन्तः॑ । यन्तो॒ न । नाप॑ । अपे᳚क्षन्ते । ई॒क्ष॒न्त॒ आ । आ द्याम् । द्याꣳ रो॑हन्ति । रो॒ह॒न्ति॒ रोद॑सी । रोद॑सी॒ इति॒ रोद॑सी ॥ य॒ज्ञ्ं ॅये । ये वि॒श्वतो॑धारम् । वि॒श्वतो॑धारꣳ॒॒ सुवि॑द्वाꣳसः । वि॒श्वतो॑धार॒मिति॑ वि॒श्वतः॑ - धा॒र॒म् । सुवि॑द्वाꣳसो वितेनि॒रे । सुवि॑द्वाꣳस॒ इति॒ सु - वि॒द्वाꣳ॒॒सः॒ । वि॒ते॒नि॒र इति॑ वि - ते॒नि॒रे ॥ अग्ने॒ प्र । प्रेहि॑ । इ॒हि॒ प्र॒थ॒मः । प्र॒थ॒मो दे॑वय॒ताम् । दे॒व॒य॒ताम् चक्षुः॑ । दे॒व॒य॒तामिति॑ देव - य॒ताम् । चक्षु॑र् दे॒वाना᳚म् । दे॒वाना॑मु॒त । उ॒त मर्त्या॑नाम् । मर्त्या॑ना॒मिति॒ मर्त्या॑नाम् ॥ इय॑क्षमाणा॒ भृगु॑भिः । भृगु॑भिः स॒जोषाः᳚ । भृगु॑भि॒रिति॒ भृगु॑ - भिः॒ । स॒जोषाः॒ सुवः॑ । स॒जोषा॒ इति॑ स - जोषाः᳚ । सुव॑र् यन्तु । य॒न्तु॒ यज॑मानाः । यज॑मानाः स्व॒स्ति । स्व॒स्तीति॑ स्व॒स्ति ॥ नक्तो॒षासा॒ सम॑नसा । सम॑नसा॒ विरू॑पे । सम॑न॒सेति॒ स - म॒न॒सा॒ । विरू॑पे धा॒पये॑ते । विरू॑पे॒ इति॒ वि - रू॒पे॒ । धा॒पये॑ते॒ शिशु᳚म् । धा॒पये॑ते॒ इति॑ धा॒पये॑ते । शिशु॒मेक᳚म् । एकꣳ॑ समी॒ची । स॒मी॒ची इति॑ समी॒ची ॥ द्यावा॒ क्षाम॑ । क्षामा॑ रु॒क्मः । रु॒क्मो अ॒न्तः । अ॒न्तर् वि । वि भा॑ति । भा॒ति॒ दे॒वाः । दे॒वा अ॒ग्निम् । अ॒ग्निम् धा॑रयन्न् । धा॒र॒य॒न् द्र॒वि॒णो॒दाः । द्र॒वि॒णो॒दा इति॑ द्रविणः - दाः ॥ अग्ने॑ सहस्राक्ष । स॒ह॒स्रा॒क्ष॒ श॒त॒मू॒र्द्ध॒न्न्॒ । स॒ह॒स्रा॒क्षेति॑ सहस्र - अ॒क्ष॒ \newline

\textbf{Jatai Paata} \newline

1. अ॒हमित्य॒हम् । \newline
2. सुव॒र् यन्तो॒ यन्तः॒ सुवः॒ सुव॒र् यन्तः॑ । \newline
3. यन्तो॒ न न यन्तो॒ यन्तो॒ न । \newline
4. नापाप॒ न नाप॑ । \newline
5. अपे᳚क्षन्त ईक्षन्ते॒ अपापे᳚क्षन्ते । \newline
6. ई॒क्ष॒न्त॒ एक्ष॑न्त ईक्षन्त॒ आ । \newline
7. आ द्याम् द्या मा द्याम् । \newline
8. द्याꣳ रो॑हन्ति रोहन्ति॒ द्याम् द्याꣳ रो॑हन्ति । \newline
9. रो॒ह॒न्ति॒ रोद॑सी॒ रोद॑सी रोहन्ति रोहन्ति॒ रोद॑सी । \newline
10. रोद॑सी॒ इति॒ रोद॑सी । \newline
11. य॒ज्ञ्ं ॅये ये य॒ज्ञ्ं ॅय॒ज्ञ्ं ॅये । \newline
12. ये वि॒श्वतो॑धारं ॅवि॒श्वतो॑धारं॒ ॅये ये वि॒श्वतो॑धारम् । \newline
13. वि॒श्वतो॑धारꣳ॒॒ सुवि॑द्वाꣳसः॒ सुवि॑द्वाꣳसो वि॒श्वतो॑धारं ॅवि॒श्वतो॑धारꣳ॒॒ सुवि॑द्वाꣳसः । \newline
14. वि॒श्वतो॑धार॒मिति॑ वि॒श्वतः॑ - धा॒र॒म् । \newline
15. सुवि॑द्वाꣳसो वितेनि॒रे वि॑तेनि॒रे सुवि॑द्वाꣳसः॒ सुवि॑द्वाꣳसो वितेनि॒रे । \newline
16. सुवि॑द्वाꣳस॒ इति॒ सु - वि॒द्वाꣳ॒॒सः॒ । \newline
17. वि॒ते॒नि॒र इति॑ वि - ते॒नि॒रे । \newline
18. अग्ने॒ प्र प्राग्ने ऽग्ने॒ प्र । \newline
19. प्रेही॑हि॒ प्र प्रेहि॑ । \newline
20. इ॒हि॒ प्र॒थ॒मः प्र॑थ॒म इ॑हीहि प्रथ॒मः । \newline
21. प्र॒थ॒मो दे॑वय॒ताम् दे॑वय॒ताम् प्र॑थ॒मः प्र॑थ॒मो दे॑वय॒ताम् । \newline
22. दे॒व॒य॒ताम् चक्षु॒ श्चक्षु॑र् देवय॒ताम् दे॑वय॒ताम् चक्षुः॑ । \newline
23. दे॒व॒य॒तामिति॑ देव - य॒ताम् । \newline
24. चक्षु॑र् दे॒वाना᳚म् दे॒वाना॒म् चक्षु॒ श्चक्षु॑र् दे॒वाना᳚म् । \newline
25. दे॒वाना॑ मु॒तोत दे॒वाना᳚म् दे॒वाना॑ मु॒त । \newline
26. उ॒त मर्त्या॑ना॒म् मर्त्या॑ना मु॒तोत मर्त्या॑नाम् । \newline
27. मर्त्या॑ना॒मिति॒ मर्त्या॑नाम् । \newline
28. इय॑क्षमाणा॒ भृगु॑भि॒र् भृगु॑भि॒ रिय॑क्षमाणा॒ इय॑क्षमाणा॒ भृगु॑भिः । \newline
29. भृगु॑भिः स॒जोषाः᳚ स॒जोषा॒ भृगु॑भि॒र् भृगु॑भिः स॒जोषाः᳚ । \newline
30. भृगु॑भि॒रिति॒ भृगु॑ - भिः॒ । \newline
31. स॒जोषाः॒ सुवः॒ सुवः॑ स॒जोषाः᳚ स॒जोषाः॒ सुवः॑ । \newline
32. स॒जोषा॒ इति॑ स - जोषाः᳚ । \newline
33. सुव॑र् यन्तु यन्तु॒ सुवः॒ सुव॑र् यन्तु । \newline
34. य॒न्तु॒ यज॑माना॒ यज॑माना यन्तु यन्तु॒ यज॑मानाः । \newline
35. यज॑मानाः स्व॒स्ति स्व॒स्ति यज॑माना॒ यज॑मानाः स्व॒स्ति । \newline
36. स्व॒स्तीति॑ स्व॒स्ति । \newline
37. नक्तो॒षासा॒ सम॑नसा॒ सम॑नसा॒ नक्तो॒षासा॒ नक्तो॒षासा॒ सम॑नसा । \newline
38. सम॑नसा॒ विरू॑पे॒ विरू॑पे॒ सम॑नसा॒ सम॑नसा॒ विरू॑पे । \newline
39. सम॑न॒सेति॒ स - म॒न॒सा॒ । \newline
40. विरू॑पे धा॒पये॑ते धा॒पये॑ते॒ विरू॑पे॒ विरू॑पे धा॒पये॑ते । \newline
41. विरू॑पे॒ इति॒ वि - रू॒पे॒ । \newline
42. धा॒पये॑ते॒ शिशुꣳ॒॒ शिशु॑म् धा॒पये॑ते धा॒पये॑ते॒ शिशु᳚म् । \newline
43. धा॒पये॑ते॒ इति॑ धा॒पये॑ते । \newline
44. शिशु॒ मेक॒ मेकꣳ॒॒ शिशुꣳ॒॒ शिशु॒ मेक᳚म् । \newline
45. एकꣳ॑ समी॒ची स॑मी॒ची एक॒ मेकꣳ॑ समी॒ची । \newline
46. स॒मी॒ची इति॑ समी॒ची । \newline
47. द्यावा॒ क्षाम॒ क्षाम॒ द्यावा॒ द्यावा॒ क्षाम॑ । \newline
48. क्षामा॑ रु॒क्मो रु॒क्मः क्षाम॒ क्षामा॑ रु॒क्मः । \newline
49. रु॒क्मो अ॒न्त-र॒न्ता रु॒क्मो रु॒क्मो अ॒न्तः । \newline
50. अ॒न्तर् वि व्य॑न्त-र॒न्तर् वि । \newline
51. वि भा॑ति भाति॒ वि वि भा॑ति । \newline
52. भा॒ति॒ दे॒वा दे॒वा भा॑ति भाति दे॒वाः । \newline
53. दे॒वा अ॒ग्नि म॒ग्निम् दे॒वा दे॒वा अ॒ग्निम् । \newline
54. अ॒ग्निम् धा॑रयन् धारयन् न॒ग्नि म॒ग्निम् धा॑रयन्न् । \newline
55. धा॒र॒य॒न् द्र॒वि॒णो॒दा द्र॑विणो॒दा धा॑रयन् धारयन् द्रविणो॒दाः । \newline
56. द्र॒वि॒णो॒दा इति॑ द्रविणः - दाः । \newline
57. अग्ने॑ सहस्राक्ष सहस्रा॒क्षाग्ने ऽग्ने॑ सहस्राक्ष । \newline
58. स॒ह॒स्रा॒क्ष॒ श॒त॒मू॒र्द्ध॒ञ्॒ छ॒त॒मू॒र्द्ध॒न् थ्स॒ह॒स्रा॒क्ष॒ स॒ह॒स्रा॒क्ष॒ श॒त॒मू॒र्द्ध॒न्न् । \newline
59. स॒ह॒स्रा॒क्षेति॑ सहस्र - अ॒क्ष॒ । \newline

\textbf{Ghana Paata } \newline

1. अ॒हमित्य॒हम् । \newline
2. सुव॒र् यन्तो॒ यन्तः॒ सुवः॒ सुव॒र् यन्तो॒ न न यन्तः॒ सुवः॒ सुव॒र् यन्तो॒ न । \newline
3. यन्तो॒ न न यन्तो॒ यन्तो॒ ना पाप॒ न यन्तो॒ यन्तो॒ नाप॑ । \newline
4. ना पाप॒ न ना पे᳚क्षन्त ईक्षन्ते॒ अप॒ न ना पे᳚क्षन्ते । \newline
5. अपे᳚क्षन्त ईक्षन्ते॒ अपापे᳚क्षन्त॒ एक्षन्ते॒ अपापे᳚क्षन्त॒ आ । \newline
6. ई॒क्ष॒न्त॒ एक्ष॑न्त ईक्षन्त॒ आ द्याम् द्या मेक्ष॑न्त ईक्षन्त॒ आ द्याम् । \newline
7. आ द्याम् द्या मा द्याꣳ रो॑हन्ति रोहन्ति॒ द्या मा द्याꣳ रो॑हन्ति । \newline
8. द्याꣳ रो॑हन्ति रोहन्ति॒ द्याम् द्याꣳ रो॑हन्ति॒ रोद॑सी॒ रोद॑सी रोहन्ति॒ द्याम् द्याꣳ रो॑हन्ति॒ रोद॑सी । \newline
9. रो॒ह॒न्ति॒ रोद॑सी॒ रोद॑सी रोहन्ति रोहन्ति॒ रोद॑सी । \newline
10. रोद॑सी॒ इति॒ रोद॑सी । \newline
11. य॒ज्ञ्ं ॅये ये य॒ज्ञ्ं ॅय॒ज्ञ्ं ॅये वि॒श्वतो॑धारं ॅवि॒श्वतो॑धारं॒ ॅये य॒ज्ञ्ं ॅय॒ज्ञ्ं ॅये वि॒श्वतो॑धारम् । \newline
12. ये वि॒श्वतो॑धारं ॅवि॒श्वतो॑धारं॒ ॅये ये वि॒श्वतो॑धारꣳ॒॒ सुवि॑द्वाꣳसः॒ सुवि॑द्वाꣳसो वि॒श्वतो॑धारं॒ ॅये ये वि॒श्वतो॑धारꣳ॒॒ सुवि॑द्वाꣳसः । \newline
13. वि॒श्वतो॑धारꣳ॒॒ सुवि॑द्वाꣳसः॒ सुवि॑द्वाꣳसो वि॒श्वतो॑धारं ॅवि॒श्वतो॑धारꣳ॒॒ सुवि॑द्वाꣳसो वितेनि॒रे वि॑तेनि॒रे सुवि॑द्वाꣳसो वि॒श्वतो॑धारं ॅवि॒श्वतो॑धारꣳ॒॒ सुवि॑द्वाꣳसो वितेनि॒रे । \newline
14. वि॒श्वतो॑धार॒मिति॑ वि॒श्वतः॑ - धा॒र॒म् । \newline
15. सुवि॑द्वाꣳसो वितेनि॒रे वि॑तेनि॒रे सुवि॑द्वाꣳसः॒ सुवि॑द्वाꣳसो वितेनि॒रे । \newline
16. सुवि॑द्वाꣳस॒ इति॒ सु - वि॒द्वाꣳ॒॒सः॒ । \newline
17. वि॒ते॒नि॒र इति॑ वि - ते॒नि॒रे । \newline
18. अग्ने॒ प्र प्राग्ने ऽग्ने॒ प्रेही॑हि॒ प्राग्ने ऽग्ने॒ प्रेहि॑ । \newline
19. प्रेही॑हि॒ प्र प्रेहि॑ प्रथ॒मः प्र॑थ॒म इ॑हि॒ प्र प्रेहि॑ प्रथ॒मः । \newline
20. इ॒हि॒ प्र॒थ॒मः प्र॑थ॒म इ॑हीहि प्रथ॒मो दे॑वय॒ताम् दे॑वय॒ताम् प्र॑थ॒म इ॑हीहि प्रथ॒मो दे॑वय॒ताम् । \newline
21. प्र॒थ॒मो दे॑वय॒ताम् दे॑वय॒ताम् प्र॑थ॒मः प्र॑थ॒मो दे॑वय॒ताम् चक्षु॒ श्चक्षु॑र् देवय॒ताम् प्र॑थ॒मः प्र॑थ॒मो दे॑वय॒ताम् चक्षुः॑ । \newline
22. दे॒व॒य॒ताम् चक्षु॒ श्चक्षु॑र् देवय॒ताम् दे॑वय॒ताम् चक्षु॑र् दे॒वाना᳚म् दे॒वाना॒म् चक्षु॑र् देवय॒ताम् दे॑वय॒ताम् चक्षु॑र् दे॒वाना᳚म् । \newline
23. दे॒व॒य॒तामिति॑ देव - य॒ताम् । \newline
24. चक्षु॑र् दे॒वाना᳚म् दे॒वाना॒म् चक्षु॒ श्चक्षु॑र् दे॒वाना॑ मु॒तोत दे॒वाना॒म् चक्षु॒ श्चक्षु॑र् दे॒वाना॑ मु॒त । \newline
25. दे॒वाना॑ मु॒तोत दे॒वाना᳚म् दे॒वाना॑ मु॒त मर्त्या॑ना॒म् मर्त्या॑ना मु॒त दे॒वाना᳚म् दे॒वाना॑ मु॒त मर्त्या॑नाम् । \newline
26. उ॒त मर्त्या॑ना॒म् मर्त्या॑ना मु॒तोत मर्त्या॑नाम् । \newline
27. मर्त्या॑ना॒मिति॒ मर्त्या॑नाम् । \newline
28. इय॑क्षमाणा॒ भृगु॑भि॒र् भृगु॑भि॒ रिय॑क्षमाणा॒ इय॑क्षमाणा॒ भृगु॑भिः स॒जोषाः᳚ स॒जोषा॒ भृगु॑भि॒ रिय॑क्षमाणा॒ इय॑क्षमाणा॒ भृगु॑भिः स॒जोषाः᳚ । \newline
29. भृगु॑भिः स॒जोषाः᳚ स॒जोषा॒ भृगु॑भि॒र् भृगु॑भिः स॒जोषाः॒ सुवः॒ सुवः॑ स॒जोषा॒ भृगु॑भि॒र् भृगु॑भिः स॒जोषाः॒ सुवः॑ । \newline
30. भृगु॑भि॒रिति॒ भृगु॑ - भिः॒ । \newline
31. स॒जोषाः॒ सुवः॒ सुवः॑ स॒जोषाः᳚ स॒जोषाः॒ सुव॑र् यन्तु यन्तु॒ सुवः॑ स॒जोषाः᳚ स॒जोषाः॒ सुव॑र् यन्तु । \newline
32. स॒जोषा॒ इति॑ स - जोषाः᳚ । \newline
33. सुव॑र् यन्तु यन्तु॒ सुवः॒ सुव॑र् यन्तु॒ यज॑माना॒ यज॑माना यन्तु॒ सुवः॒ सुव॑र् यन्तु॒ यज॑मानाः । \newline
34. य॒न्तु॒ यज॑माना॒ यज॑माना यन्तु यन्तु॒ यज॑मानाः स्व॒स्ति स्व॒स्ति यज॑माना यन्तु यन्तु॒ यज॑मानाः स्व॒स्ति । \newline
35. यज॑मानाः स्व॒स्ति स्व॒स्ति यज॑माना॒ यज॑मानाः स्व॒स्ति । \newline
36. स्व॒स्तीति॑ स्व॒स्ति । \newline
37. नक्तो॒षासा॒ सम॑नसा॒ सम॑नसा॒ नक्तो॒षासा॒ नक्तो॒षासा॒ सम॑नसा॒ विरू॑पे॒ विरू॑पे॒ सम॑नसा॒ नक्तो॒षासा॒ नक्तो॒षासा॒ सम॑नसा॒ विरू॑पे । \newline
38. सम॑नसा॒ विरू॑पे॒ विरू॑पे॒ सम॑नसा॒ सम॑नसा॒ विरू॑पे धा॒पये॑ते धा॒पये॑ते॒ विरू॑पे॒ सम॑नसा॒ सम॑नसा॒ विरू॑पे धा॒पये॑ते । \newline
39. सम॑न॒सेति॒ स - म॒न॒सा॒ । \newline
40. विरू॑पे धा॒पये॑ते धा॒पये॑ते॒ विरू॑पे॒ विरू॑पे धा॒पये॑ते॒ शिशुꣳ॒॒ शिशु॑म् धा॒पये॑ते॒ विरू॑पे॒ विरू॑पे धा॒पये॑ते॒ शिशु᳚म् । \newline
41. विरू॑पे॒ इति॒ वि - रू॒पे॒ । \newline
42. धा॒पये॑ते॒ शिशुꣳ॒॒ शिशु॑म् धा॒पये॑ते धा॒पये॑ते॒ शिशु॒ मेक॒ मेकꣳ॒॒ शिशु॑म् धा॒पये॑ते धा॒पये॑ते॒ शिशु॒ मेक᳚म् । \newline
43. धा॒पये॑ते॒ इति॑ धा॒पये॑ते । \newline
44. शिशु॒ मेक॒ मेकꣳ॒॒ शिशुꣳ॒॒ शिशु॒ मेकꣳ॑ समी॒ची स॑मी॒ची एकꣳ॒॒ शिशुꣳ॒॒ शिशु॒ मेकꣳ॑ समी॒ची । \newline
45. एकꣳ॑ समी॒ची स॑मी॒ची एक॒ मेकꣳ॑ समी॒ची । \newline
46. स॒मी॒ची इति॑ समी॒ची । \newline
47. द्यावा॒ क्षाम॒ क्षाम॒ द्यावा॒ द्यावा॒ क्षामा॑ रु॒क्मो रु॒क्मः क्षाम॒ द्यावा॒ द्यावा॒ क्षामा॑ रु॒क्मः । \newline
48. क्षामा॑ रु॒क्मो रु॒क्मः क्षाम॒ क्षामा॑ रु॒क्मो अ॒न्त र॒न्ता रु॒क्मः क्षाम॒ क्षामा॑ रु॒क्मो अ॒न्तः । \newline
49. रु॒क्मो अ॒न्त र॒न्ता रु॒क्मो रु॒क्मो अ॒न्तर् वि व्य॑न्ता रु॒क्मो रु॒क्मो अ॒न्तर् वि । \newline
50. अ॒न्तर् वि व्य॑न्त र॒न्तर् वि भा॑ति भाति॒ व्य॑न्त र॒न्तर् वि भा॑ति । \newline
51. वि भा॑ति भाति॒ वि वि भा॑ति दे॒वा दे॒वा भा॑ति॒ वि वि भा॑ति दे॒वाः । \newline
52. भा॒ति॒ दे॒वा दे॒वा भा॑ति भाति दे॒वा अ॒ग्नि म॒ग्निम् दे॒वा भा॑ति भाति दे॒वा अ॒ग्निम् । \newline
53. दे॒वा अ॒ग्नि म॒ग्निम् दे॒वा दे॒वा अ॒ग्निम् धा॑रयन् धारयन् न॒ग्निम् दे॒वा दे॒वा अ॒ग्निम् धा॑रयन्न् । \newline
54. अ॒ग्निम् धा॑रयन् धारयन् न॒ग्नि म॒ग्निम् धा॑रयन् द्रविणो॒दा द्र॑विणो॒दा धा॑रयन् न॒ग्नि म॒ग्निम् धा॑रयन् द्रविणो॒दाः । \newline
55. धा॒र॒य॒न् द्र॒वि॒णो॒दा द्र॑विणो॒दा धा॑रयन् धारयन् द्रविणो॒दाः । \newline
56. द्र॒वि॒णो॒दा इति॑ द्रविणः - दाः । \newline
57. अग्ने॑ सहस्राक्ष सहस्रा॒क्षाग्ने ऽग्ने॑ सहस्राक्ष शतमूर्द्धञ् छतमूर्द्धन् थ्सहस्रा॒क्षाग्ने ऽग्ने॑ सहस्राक्ष शतमूर्द्धन्न् । \newline
58. स॒ह॒स्रा॒क्ष॒ श॒त॒मू॒र्द्ध॒ञ्॒ छ॒त॒मू॒र्द्ध॒न् थ्स॒ह॒स्रा॒क्ष॒ स॒ह॒स्रा॒क्ष॒ श॒त॒मू॒र्द्ध॒ञ्॒ छ॒तꣳ श॒तꣳ श॑तमूर्द्धन् थ्सहस्राक्ष सहस्राक्ष शतमूर्द्धञ् छ॒तम् । \newline
59. स॒ह॒स्रा॒क्षेति॑ सहस्र - अ॒क्ष॒ । \newline
\pagebreak
\markright{ TS 4.6.5.3  \hfill https://www.vedavms.in \hfill}

\section{ TS 4.6.5.3 }

\textbf{TS 4.6.5.3 } \newline
\textbf{Samhita Paata} \newline

शतमूर्द्धञ्छ॒तं ते᳚ प्रा॒णाः स॒हस्र॑मपा॒नाः । त्वꣳ सा॑ह॒स्रस्य॑ रा॒य ई॑शिषे॒ तस्मै॑ ते विधेम॒ वाजा॑य॒ स्वाहा᳚ ॥ सु॒प॒र्णो॑ऽसि ग॒रुत्मा᳚न् पृथि॒व्याꣳ सी॑द पृ॒ष्ठे पृ॑थि॒व्याः सी॑द भा॒साऽ*न्तरि॑क्ष॒मा पृ॑ण॒ ज्योति॑षा॒ दिव॒मुत्त॑भान॒ तेज॑सा॒ दिश॒ उद् दृꣳ॑ह ॥ आ॒जुह्वा॑नः सु॒प्रती॑कः पु॒रस्ता॒दग्ने॒ स्वां ॅयोनि॒मा सी॑द सा॒द्ध्या । अ॒स्मिन्थ्-स॒धस्थे॒ अद्ध्युत्त॑रस्मि॒न् विश्वे॑ देवा॒ - [  ] \newline

\textbf{Pada Paata} \newline

श॒त॒मू॒द्‌र्ध॒न्निति॑ शत - मू॒र्ध॒न्न् । श॒तम् । ते॒ । प्रा॒णा इति॑ प्र-अ॒नाः । स॒हस्र᳚म् । अ॒पा॒न इत्य॑प - अ॒नाः ॥ त्वम् । सा॒ह॒स्रस्य॑ । रा॒यः । ई॒शि॒षे॒ । तस्मै᳚ । ते॒ । वि॒धे॒म॒ । वाजा॑य । स्वाहा᳚ ॥ सु॒प॒र्ण॒ इति॑ सु-प॒र्णः । अ॒सि॒ । ग॒रुत्मान्॑ । पृ॒थि॒व्याम् । सी॒द॒ । पृ॒ष्ठे । पृ॒थि॒व्याः । सी॒द॒ । भा॒सा । अ॒न्तरि॑क्षम् । एति॑ । पृ॒ण॒ । ज्योति॑षा । दिव᳚म् । उदिति॑ । स्त॒भा॒न॒ । तेज॑सा । दिशः॑ । उदिति॑ । दृꣳ॒॒ह॒ ॥ आ॒जुह्वा॑न॒ इत्या᳚ - जुह्वा॑नः । सु॒प्रती॑क॒ इति॑ सु - प्रती॑कः । पु॒रस्ता᳚त् । अग्ने᳚ । स्वाम् । योनि᳚म् । एति॑ । सी॒द॒ । सा॒द्ध्या ॥ अ॒स्मिन्न् । स॒धस्थ॒ इति॑ स॒ध - स्थे॒ । अधीति॑ । उत्त॑रस्मि॒न्नित्युत् - त॒र॒स्मि॒न्न् । विश्वे᳚ । दे॒वाः॒ ।  \newline


\textbf{Krama Paata} \newline

श॒त॒मू॒र्द्ध॒ञ्छ॒तम् । श॒त॒मू॒र्द्ध॒न्निति॑ शत - मू॒र्द्ध॒न्न्॒ । श॒तम् ते᳚ । ते॒ प्रा॒णाः । प्रा॒णाः स॒हस्र᳚म् । प्रा॒णा इति॑ प्र - अ॒नाः । स॒हस्र॑मपा॒नाः । अ॒पा॒ना इत्य॑प - अ॒नाः ॥ त्वꣳ सा॑ह॒स्रस्य॑ । सा॒ह॒स्रस्य॑ रा॒यः । रा॒य ई॑शिषे । ई॒शि॒षे॒ तस्मै᳚ । तस्मै॑ ते । ते॒ वि॒धे॒म॒ । वि॒धे॒म॒ वाजा॑य । वाजा॑य॒ स्वाहा᳚ । स्वाहेति॒ स्वाहा᳚ ॥ सु॒प॒र्णो॑ऽसि । सु॒प॒र्ण इति॑ सु - प॒र्णः । अ॒सि॒ ग॒रुत्मान्॑ । ग॒रुत्मा᳚न् पृथि॒व्याम् । पृ॒थि॒व्याꣳ सी॑द । सी॒द॒ पृ॒ष्ठे । पृ॒ष्ठे पृ॑थि॒व्याः । पृ॒थि॒व्याः सी॑द । सी॒द॒ भा॒सा । भा॒साऽन्तरि॑क्षम् । अ॒न्तरि॑क्ष॒मा । आ पृ॑ण । पृ॒ण॒ ज्योति॑षा । ज्योति॑षा॒ दिव᳚म् । दिव॒मुत् । उत् त॑भान । स्त॒भा॒न॒ तेज॑सा । तेज॑सा॒ दिशः॑ । दिश॒ उत् । उद् दृꣳ॑ह । दृꣳ॒॒हेति॑ दृꣳह ॥ आ॒जुह्वा॑नः सु॒प्रती॑कः । आ॒जुह्वा॑न॒ इत्या᳚ - जुह्वा॑नः । सु॒प्रती॑कः पु॒रस्ता᳚त् । सु॒प्रती॑क॒ इति॑ सु - प्रती॑कः । पु॒रस्ता॒दग्ने᳚ । अग्ने॒ स्वाम् । स्वां ॅयोनि᳚म् । योनि॒मा । आ सी॑द । सी॒द॒ सा॒द्ध्या । सा॒द्ध्येति॑ सा॒द्ध्या ॥ अ॒स्मिन्थ् स॒धस्थे᳚ । स॒धस्थे॒ अधि॑ । स॒धस्थ॒ इति॑ स॒ध - स्थे॒ । अद्ध्युत्त॑रस्मिन्न् । उत्त॑रस्मि॒न् विश्वे᳚ । उत्त॑रस्मि॒न्नित्युत् - त॒र॒स्मि॒न्न्॒ । विश्वे॑ देवाः । दे॒वा॒ यज॑मानः \newline

\textbf{Jatai Paata} \newline

1. श॒त॒मू॒र्द्ध॒ञ् छ॒तꣳ श॒तꣳ श॑तमूर्द्धञ् छतमूर्द्धञ् छ॒तम् । \newline
2. श॒त॒मू॒र्द्ध॒न्निति॑ शत - मू॒र्ध॒न्न् । \newline
3. श॒तम् ते॑ ते श॒तꣳ श॒तम् ते᳚ । \newline
4. ते॒ प्रा॒णाः प्रा॒णा स्ते॑ ते प्रा॒णाः । \newline
5. प्रा॒णाः स॒हस्रꣳ॑ स॒हस्र॑म् प्रा॒णाः प्रा॒णाः स॒हस्र᳚म् । \newline
6. प्रा॒णा इति॑ प्र - अ॒नाः । \newline
7. स॒हस्र॑ मपा॒ना अ॑पा॒नाः स॒हस्रꣳ॑ स॒हस्र॑ मपा॒नाः । \newline
8. अ॒पा॒ना इत्य॑प - अ॒नाः । \newline
9. त्वꣳ सा॑ह॒स्रस्य॑ साह॒स्रस्य॒ त्वम् त्वꣳ सा॑ह॒स्रस्य॑ । \newline
10. सा॒ह॒स्रस्य॑ रा॒यो रा॒यः सा॑ह॒स्रस्य॑ साह॒स्रस्य॑ रा॒यः । \newline
11. रा॒य ई॑शिष ईशिषे रा॒यो रा॒य ई॑शिषे । \newline
12. ई॒शि॒षे॒ तस्मै॒ तस्मा॑ ईशिष ईशिषे॒ तस्मै᳚ । \newline
13. तस्मै॑ ते ते॒ तस्मै॒ तस्मै॑ ते । \newline
14. ते॒ वि॒धे॒म॒ वि॒धे॒म॒ ते॒ ते॒ वि॒धे॒म॒ । \newline
15. वि॒धे॒म॒ वाजा॑य॒ वाजा॑य विधेम विधेम॒ वाजा॑य । \newline
16. वाजा॑य॒ स्वाहा॒ स्वाहा॒ वाजा॑य॒ वाजा॑य॒ स्वाहा᳚ । \newline
17. स्वाहेति॒ स्वाहा᳚ । \newline
18. सु॒प॒र्णो᳚ ऽस्यसि सुप॒र्णः सु॑प॒र्णो॑ ऽसि । \newline
19. सु॒प॒र्ण इति॑ सु - प॒र्णः । \newline
20. अ॒सि॒ ग॒रुत्मा᳚न् ग॒रुत्मा॑ नस्यसि ग॒रुत्मान्॑ । \newline
21. ग॒रुत्मा᳚न् पृथि॒व्याम् पृ॑थि॒व्याम् ग॒रुत्मा᳚न् ग॒रुत्मा᳚न् पृथि॒व्याम् । \newline
22. पृ॒थि॒व्याꣳ सी॑द सीद पृथि॒व्याम् पृ॑थि॒व्याꣳ सी॑द । \newline
23. सी॒द॒ पृ॒ष्ठे पृ॒ष्ठे सी॑द सीद पृ॒ष्ठे । \newline
24. पृ॒ष्ठे पृ॑थि॒व्याः पृ॑थि॒व्याः पृ॒ष्ठे पृ॒ष्ठे पृ॑थि॒व्याः । \newline
25. पृ॒थि॒व्याः सी॑द सीद पृथि॒व्याः पृ॑थि॒व्याः सी॑द । \newline
26. सी॒द॒ भा॒सा भा॒सा सी॑द सीद भा॒सा । \newline
27. भा॒सा ऽन्तरि॑क्ष म॒न्तरि॑क्षम् भा॒सा भा॒सा ऽन्तरि॑क्षम् । \newline
28. अ॒न्तरि॑क्ष॒मा ऽन्तरि॑क्ष म॒न्तरि॑क्ष॒मा । \newline
29. आ पृ॑ण पृ॒णा पृ॑ण । \newline
30. पृ॒ण॒ ज्योति॑षा॒ ज्योति॑षा पृण पृण॒ ज्योति॑षा । \newline
31. ज्योति॑षा॒ दिव॒म् दिव॒म् ज्योति॑षा॒ ज्योति॑षा॒ दिव᳚म् । \newline
32. दिव॒ मुदुद् दिव॒म् दिव॒ मुत् । \newline
33. उत् त॑भान स्तभा॒ नोदुत् त॑भान । \newline
34. स्त॒भा॒न॒ तेज॑सा॒ तेज॑सा स्तभान स्तभान॒ तेज॑सा । \newline
35. तेज॑सा॒ दिशो॒ दिश॒ स्तेज॑सा॒ तेज॑सा॒ दिशः॑ । \newline
36. दिश॒ उदुद् दिशो॒ दिश॒ उत् । \newline
37. उद् दृꣳ॑ह दृꣳ॒॒ होदुद् दृꣳ॑ह । \newline
38. दृꣳ॒॒हेति॑ दृꣳह । \newline
39. आ॒जुह्वा॑नः सु॒प्रती॑कः सु॒प्रती॑क आ॒जुह्वा॑न आ॒जुह्वा॑नः सु॒प्रती॑कः । \newline
40. आ॒जुह्वा॑न॒ इत्या᳚ - जुह्वा॑नः । \newline
41. सु॒प्रती॑कः पु॒रस्ता᳚त् पु॒रस्ता᳚थ् सु॒प्रती॑कः सु॒प्रती॑कः पु॒रस्ता᳚त् । \newline
42. सु॒प्रती॑क॒ इति॑ सु - प्रती॑कः । \newline
43. पु॒रस्ता॒ दग्ने ऽग्ने॑ पु॒रस्ता᳚त् पु॒रस्ता॒ दग्ने᳚ । \newline
44. अग्ने॒ स्वाꣳ स्वा मग्ने ऽग्ने॒ स्वाम् । \newline
45. स्वां ॅयोनिं॒ ॅयोनिꣳ॒॒ स्वाꣳ स्वां ॅयोनि᳚म् । \newline
46. योनि॒मा योनिं॒ ॅयोनि॒मा । \newline
47. आ सी॑द सी॒दा सी॑द । \newline
48. सी॒द॒ सा॒द्ध्या सा॒द्ध्या सी॑द सीद सा॒द्ध्या । \newline
49. सा॒द्ध्येति॑ सा॒द्ध्या । \newline
50. अ॒स्मिन् थ्स॒धस्थे॑ स॒धस्थे॑ अ॒स्मिन् न॒स्मिन् थ्स॒धस्थे᳚ । \newline
51. स॒धस्थे॒ अध्यधि॑ स॒धस्थे॑ स॒धस्थे॒ अधि॑ । \newline
52. स॒धस्थ॒ इति॑ स॒ध - स्थे॒ । \newline
53. अध्युत्त॑रस्मि॒न् नुत्त॑रस्मि॒न् नध्यध्युत्त॑रस्मिन्न् । \newline
54. उत्त॑रस्मि॒न्॒. विश्वे॒ विश्व॒ उत्त॑रस्मि॒न् नुत्त॑रस्मि॒न्॒. विश्वे᳚ । \newline
55. उत्त॑रस्मि॒न्नित्युत् - त॒र॒स्मि॒न्न् । \newline
56. विश्वे॑ देवा देवा॒ विश्वे॒ विश्वे॑ देवाः । \newline
57. दे॒वा॒ यज॑मानो॒ यज॑मानो देवा देवा॒ यज॑मानः । \newline

\textbf{Ghana Paata } \newline

1. श॒त॒मू॒र्द्ध॒ञ् छ॒तꣳ श॒तꣳ श॑तमूर्द्धञ् छतमूर्द्धञ् छ॒तम् ते॑ ते श॒तꣳ श॑तमूर्द्धञ् छतमूर्द्धञ् छ॒तम् ते᳚ । \newline
2. श॒त॒मू॒र्द्ध॒न्निति॑ शत - मू॒र्ध॒न्न् । \newline
3. श॒तम् ते॑ ते श॒तꣳ श॒तम् ते᳚ प्रा॒णाः प्रा॒णा स्ते॑ श॒तꣳ श॒तम् ते᳚ प्रा॒णाः । \newline
4. ते॒ प्रा॒णाः प्रा॒णा स्ते॑ ते प्रा॒णाः स॒हस्रꣳ॑ स॒हस्र॑म् प्रा॒णा स्ते॑ ते प्रा॒णाः स॒हस्र᳚म् । \newline
5. प्रा॒णाः स॒हस्रꣳ॑ स॒हस्र॑म् प्रा॒णाः प्रा॒णाः स॒हस्र॑ मपा॒ना अ॑पा॒नाः स॒हस्र॑म् प्रा॒णाः प्रा॒णाः स॒हस्र॑ मपा॒नाः । \newline
6. प्रा॒णा इति॑ प्र - अ॒नाः । \newline
7. स॒हस्र॑ मपा॒ना अ॑पा॒नाः स॒हस्रꣳ॑ स॒हस्र॑ मपा॒नाः । \newline
8. अ॒पा॒ना इत्य॑प - अ॒नाः । \newline
9. त्वꣳ सा॑ह॒स्रस्य॑ साह॒स्रस्य॒ त्वम् त्वꣳ सा॑ह॒स्रस्य॑ रा॒यो रा॒यः सा॑ह॒स्रस्य॒ त्वम् त्वꣳ सा॑ह॒स्रस्य॑ रा॒यः । \newline
10. सा॒ह॒स्रस्य॑ रा॒यो रा॒यः सा॑ह॒स्रस्य॑ साह॒स्रस्य॑ रा॒य ई॑शिष ईशिषे रा॒यः सा॑ह॒स्रस्य॑ साह॒स्रस्य॑ रा॒य ई॑शिषे । \newline
11. रा॒य ई॑शिष ईशिषे रा॒यो रा॒य ई॑शिषे॒ तस्मै॒ तस्मा॑ ईशिषे रा॒यो रा॒य ई॑शिषे॒ तस्मै᳚ । \newline
12. ई॒शि॒षे॒ तस्मै॒ तस्मा॑ ईशिष ईशिषे॒ तस्मै॑ ते ते॒ तस्मा॑ ईशिष ईशिषे॒ तस्मै॑ ते । \newline
13. तस्मै॑ ते ते॒ तस्मै॒ तस्मै॑ ते विधेम विधेम ते॒ तस्मै॒ तस्मै॑ ते विधेम । \newline
14. ते॒ वि॒धे॒म॒ वि॒धे॒म॒ ते॒ ते॒ वि॒धे॒म॒ वाजा॑य॒ वाजा॑य विधेम ते ते विधेम॒ वाजा॑य । \newline
15. वि॒धे॒म॒ वाजा॑य॒ वाजा॑य विधेम विधेम॒ वाजा॑य॒ स्वाहा॒ स्वाहा॒ वाजा॑य विधेम विधेम॒ वाजा॑य॒ स्वाहा᳚ । \newline
16. वाजा॑य॒ स्वाहा॒ स्वाहा॒ वाजा॑य॒ वाजा॑य॒ स्वाहा᳚ । \newline
17. स्वाहेति॒ स्वाहा᳚ । \newline
18. सु॒प॒र्णो᳚ ऽस्यसि सुप॒र्णः सु॑प॒र्णो॑ ऽसि ग॒रुत्मा᳚न् ग॒रुत्मा॑ नसि सुप॒र्णः सु॑प॒र्णो॑ ऽसि ग॒रुत्मान्॑ । \newline
19. सु॒प॒र्ण इति॑ सु - प॒र्णः । \newline
20. अ॒सि॒ ग॒रुत्मा᳚न् ग॒रुत्मा॑ नस्यसि ग॒रुत्मा᳚न् पृथि॒व्याम् पृ॑थि॒व्याम् ग॒रुत्मा॑ नस्यसि ग॒रुत्मा᳚न् पृथि॒व्याम् । \newline
21. ग॒रुत्मा᳚न् पृथि॒व्याम् पृ॑थि॒व्याम् ग॒रुत्मा᳚न् ग॒रुत्मा᳚न् पृथि॒व्याꣳ सी॑द सीद पृथि॒व्याम् ग॒रुत्मा᳚न् ग॒रुत्मा᳚न् पृथि॒व्याꣳ सी॑द । \newline
22. पृ॒थि॒व्याꣳ सी॑द सीद पृथि॒व्याम् पृ॑थि॒व्याꣳ सी॑द पृ॒ष्ठे पृ॒ष्ठे सी॑द पृथि॒व्याम् पृ॑थि॒व्याꣳ सी॑द पृ॒ष्ठे । \newline
23. सी॒द॒ पृ॒ष्ठे पृ॒ष्ठे सी॑द सीद पृ॒ष्ठे पृ॑थि॒व्याः पृ॑थि॒व्याः पृ॒ष्ठे सी॑द सीद पृ॒ष्ठे पृ॑थि॒व्याः । \newline
24. पृ॒ष्ठे पृ॑थि॒व्याः पृ॑थि॒व्याः पृ॒ष्ठे पृ॒ष्ठे पृ॑थि॒व्याः सी॑द सीद पृथि॒व्याः पृ॒ष्ठे पृ॒ष्ठे पृ॑थि॒व्याः सी॑द । \newline
25. पृ॒थि॒व्याः सी॑द सीद पृथि॒व्याः पृ॑थि॒व्याः सी॑द भा॒सा भा॒सा सी॑द पृथि॒व्याः पृ॑थि॒व्याः सी॑द भा॒सा । \newline
26. सी॒द॒ भा॒सा भा॒सा सी॑द सीद भा॒सा ऽन्तरि॑क्ष म॒न्तरि॑क्षम् भा॒सा सी॑द सीद भा॒सा ऽन्तरि॑क्षम् । \newline
27. भा॒सा ऽन्तरि॑क्ष म॒न्तरि॑क्षम् भा॒सा भा॒सा ऽन्तरि॑क्ष॒ मा ऽन्तरि॑क्षम् भा॒सा भा॒सा ऽन्तरि॑क्ष॒ मा । \newline
28. अ॒न्तरि॑क्ष॒ मा ऽन्तरि॑क्ष म॒न्तरि॑क्ष॒ मा पृ॑ण पृ॒णा ऽन्तरि॑क्ष म॒न्तरि॑क्ष॒ मा पृ॑ण । \newline
29. आ पृ॑ण पृ॒णा पृ॑ण॒ ज्योति॑षा॒ ज्योति॑षा पृ॒णा पृ॑ण॒ ज्योति॑षा । \newline
30. पृ॒ण॒ ज्योति॑षा॒ ज्योति॑षा पृण पृण॒ ज्योति॑षा॒ दिव॒म् दिव॒म् ज्योति॑षा पृण पृण॒ ज्योति॑षा॒ दिव᳚म् । \newline
31. ज्योति॑षा॒ दिव॒म् दिव॒म् ज्योति॑षा॒ ज्योति॑षा॒ दिव॒ मुदुद् दिव॒म् ज्योति॑षा॒ ज्योति॑षा॒ दिव॒ मुत् । \newline
32. दिव॒ मुदुद् दिव॒म् दिव॒ मुत् त॑भान स्तभा॒नोद् दिव॒म् दिव॒ मुत् त॑भान । \newline
33. उत् त॑भान स्तभा॒नोदुत् त॑भान॒ तेज॑सा॒ तेज॑सा स्तभा॒नोदुत् त॑भान॒ तेज॑सा । \newline
34. स्त॒भा॒न॒ तेज॑सा॒ तेज॑सा स्तभान स्तभान॒ तेज॑सा॒ दिशो॒ दिश॒ स्तेज॑सा स्तभान स्तभान॒ तेज॑सा॒ दिशः॑ । \newline
35. तेज॑सा॒ दिशो॒ दिश॒ स्तेज॑सा॒ तेज॑सा॒ दिश॒ उदुद् दिश॒ स्तेज॑सा॒ तेज॑सा॒ दिश॒ उत् । \newline
36. दिश॒ उदुद् दिशो॒ दिश॒ उद् दृꣳ॑ह दृꣳ॒॒होद् दिशो॒ दिश॒ उद् दृꣳ॑ह । \newline
37. उद् दृꣳ॑ह दृꣳ॒॒होदुद् दृꣳ॑ह । \newline
38. दृꣳ॒॒हेति॑ दृꣳह । \newline
39. आ॒जुह्वा॑नः सु॒प्रती॑कः सु॒प्रती॑क आ॒जुह्वा॑न आ॒जुह्वा॑नः सु॒प्रती॑कः पु॒रस्ता᳚त् पु॒रस्ता᳚थ् सु॒प्रती॑क आ॒जुह्वा॑न आ॒जुह्वा॑नः सु॒प्रती॑कः पु॒रस्ता᳚त् । \newline
40. आ॒जुह्वा॑न॒ इत्या᳚ - जुह्वा॑नः । \newline
41. सु॒प्रती॑कः पु॒रस्ता᳚त् पु॒रस्ता᳚थ् सु॒प्रती॑कः सु॒प्रती॑कः पु॒रस्ता॒ दग्ने ऽग्ने॑ पु॒रस्ता᳚थ् सु॒प्रती॑कः सु॒प्रती॑कः पु॒रस्ता॒दग्ने᳚ । \newline
42. सु॒प्रती॑क॒ इति॑ सु - प्रती॑कः । \newline
43. पु॒रस्ता॒ दग्ने ऽग्ने॑ पु॒रस्ता᳚त् पु॒रस्ता॒ दग्ने॒ स्वाꣳ स्वा मग्ने॑ पु॒रस्ता᳚त् पु॒रस्ता॒ दग्ने॒ स्वाम् । \newline
44. अग्ने॒ स्वाꣳ स्वा मग्ने ऽग्ने॒ स्वां ॅयोनिं॒ ॅयोनिꣳ॒॒ स्वा मग्ने ऽग्ने॒ स्वां ॅयोनि᳚म् । \newline
45. स्वां ॅयोनिं॒ ॅयोनिꣳ॒॒ स्वाꣳ स्वां ॅयोनि॒ मा योनिꣳ॒॒ स्वाꣳ स्वां ॅयोनि॒ मा । \newline
46. योनि॒ मा योनिं॒ ॅयोनि॒ मा सी॑द सी॒दा योनिं॒ ॅयोनि॒ मा सी॑द । \newline
47. आ सी॑द सी॒दा सी॑द सा॒द्ध्या सा॒द्ध्या सी॒दा सी॑द सा॒द्ध्या । \newline
48. सी॒द॒ सा॒द्ध्या सा॒द्ध्या सी॑द सीद सा॒द्ध्या । \newline
49. सा॒द्ध्येति॑ सा॒द्ध्या । \newline
50. अ॒स्मिन् थ्स॒धस्थे॑ स॒धस्थे॑ अ॒स्मिन् न॒स्मिन् थ्स॒धस्थे॒ अध्यधि॑ स॒धस्थे॑ अ॒स्मिन् न॒स्मिन् थ्स॒धस्थे॒ अधि॑ । \newline
51. स॒धस्थे॒ अध्यधि॑ स॒धस्थे॑ स॒धस्थे॒ अध्युत्त॑रस्मि॒न् नुत्त॑रस्मि॒न् नधि॑ स॒धस्थे॑ स॒धस्थे॒ अध्युत् त॑रस्मिन्न् । \newline
52. स॒धस्थ॒ इति॑ स॒ध - स्थे॒ । \newline
53. अध्युत्त॑रस्मि॒न् नुत्त॑रस्मि॒न् नध्यध्युत् त॑रस्मि॒न्॒. विश्वे॒ विश्व॒ उत्त॑रस्मि॒न् नध्य ध्युत्त॑रस्मि॒न्॒. विश्वे᳚ । \newline
54. उत्त॑रस्मि॒न्॒. विश्वे॒ विश्व॒ उत्त॑रस्मि॒न् नुत्त॑रस्मि॒न्॒. विश्वे॑ देवा देवा॒ विश्व॒ उत्त॑रस्मि॒न् नुत्त॑रस्मि॒न्॒. विश्वे॑ देवाः । \newline
55. उत्त॑रस्मि॒न्नित्युत् - त॒र॒स्मि॒न्न् । \newline
56. विश्वे॑ देवा देवा॒ विश्वे॒ विश्वे॑ देवा॒ यज॑मानो॒ यज॑मानो देवा॒ विश्वे॒ विश्वे॑ देवा॒ यज॑मानः । \newline
57. दे॒वा॒ यज॑मानो॒ यज॑मानो देवा देवा॒ यज॑मानश्च च॒ यज॑मानो देवा देवा॒ यज॑मानश्च । \newline
\pagebreak
\markright{ TS 4.6.5.4  \hfill https://www.vedavms.in \hfill}

\section{ TS 4.6.5.4 }

\textbf{TS 4.6.5.4 } \newline
\textbf{Samhita Paata} \newline

यज॑मानश्च सीदत ॥ प्रेद्धो॑ अग्ने दीदिहि पु॒रो नोऽज॑स्रया सू॒र्म्या॑ यविष्ठ । त्वाꣳ शश्व॑न्त॒ उप॑ यन्ति॒ वाजाः᳚ ॥ वि॒धेम॑ ते पर॒मे जन्म॑न्नग्ने वि॒धेम॒ स्तोमै॒रव॑रे स॒धस्थे᳚ । यस्मा॒द्-योने॑रु॒दारि॑था॒ यजे॒ तं प्रत्वे ह॒वीꣳषि॑ जुहुरे॒ समि॑द्धे ॥ ताꣳ स॑वि॒तुर्वरे᳚ण्यस्य चि॒त्रामाऽहं ॅवृ॑णे सुम॒तिं ॅवि॒श्वज॑न्यां । याम॑स्य॒ कण्वो॒ अदु॑ह॒त् प्रपी॑नाꣳ स॒हस्र॑धारां॒ - [  ] \newline

\textbf{Pada Paata} \newline

यज॑मानः । च॒ । सी॒द॒त॒ ॥ प्रेद्ध॒ इति॒ प्र - इ॒द्धः॒ । अ॒ग्ने॒ । दी॒दि॒हि॒ । पु॒रः । नः॒ । अज॑स्रया । सू॒र्म्या᳚ । य॒वि॒ष्ठ॒ ॥ त्वा॒म् । शश्व॑न्तः । उपेति॑ । य॒न्ति॒ । वाजाः᳚ ॥ वि॒धेम॑ । ते॒ । प॒र॒मे । जन्मन्न्॑ । अ॒ग्ने॒ । वि॒धेम॑ । स्तोमैः᳚ । अव॑रे । स॒धस्थ॒ इति॑ स॒ध - स्थे॒ ॥ यस्मा᳚त् । योनेः᳚ । उ॒दारि॒थेत्यु॑त् - आरि॑थ । यजे᳚ । तम् । प्रेति॑ । त्वे इति॑ । ह॒वीꣳषि॑ । जु॒हु॒रे॒ । समि॑द्ध॒ इति॒ सं - इ॒द्धे॒ ॥ ताम् । स॒वि॒तुः । वरे᳚ण्यस्य । चि॒त्राम् । एति॑ । अ॒हम् । वृ॒णे॒ । सु॒म॒तिमिति॑ सु - म॒तिम् । वि॒श्वज॑न्या॒मिति॑ वि॒श्व - ज॒न्या॒म् ॥ याम् । अ॒स्य॒ । कण्वः॑ । अदु॑हत् । प्रपी॑ना॒मिति॒ प्र-पी॒ना॒म् । स॒हस्र॑धारा॒मिति॑ स॒हस्र॑ - धा॒रा॒म् ।  \newline


\textbf{Krama Paata} \newline

यज॑मानश्च । च॒ सी॒द॒त॒ । सी॒द॒तेति॑ सीदत ॥ प्रेद्धो॑ अग्ने । प्रेद्ध॒ इति॒ प्र - इ॒द्धः॒ । अ॒ग्ने॒ दी॒दि॒हि॒ । दी॒दि॒हि॒ पु॒रः । पु॒रो नः॑ । नोऽज॑स्रया । अज॑स्रया सू॒र्म्या᳚ । सू॒र्म्या॑ यविष्ठ । य॒वि॒ष्ठेति॑ यविष्ठ ॥ त्वाꣳ शश्व॑न्तः । शश्व॑न्त॒ उप॑ । उप॑ यन्ति । य॒न्ति॒ वाजाः᳚ । वाजा॒ इति॒ वाजाः᳚ ॥ वि॒धेम॑ ते । ते॒ प॒र॒मे । प॒र॒मे जन्मन्न्॑ । जन्म॑न्नग्ने । अ॒ग्ने॒ वि॒धेम॑ । वि॒धेम॒ स्तोमैः᳚ । स्तोमै॒रव॑रे । अव॑रे स॒धस्थे᳚ । स॒धस्थ॒ इति॑ स॒ध - स्थे॒ ॥ यस्मा॒द् योनेः᳚ । योने॑रु॒दारि॑थ । उ॒दारि॑था॒ यजे᳚ । उ॒दारि॒थेत्यु॑त् - आरि॑थ । यजे॒ तम् । तम् प्र । प्र त्वे । त्वे ह॒वीꣳषि॑ । त्वे इति॒ त्वे । ह॒वीꣳषि॑ जुहुरे । जु॒हु॒रे॒ समि॑द्धे । समि॑द्ध॒ इति॒ सं - इ॒द्धे॒ ॥ ताꣳ स॑वि॒तुः । स॒वि॒तुर् वरे᳚ण्यस्य । वरे᳚ण्यस्य चि॒त्राम् । चि॒त्रामा । आऽहम् । अ॒हं ॅवृ॑णे । वृ॒णे॒ सु॒म॒तिम् । सु॒म॒तिं ॅवि॒श्वज॑न्याम् । सु॒म॒तिमिति॑ सु - म॒तिम् । वि॒श्वज॑न्या॒मिति॑ वि॒श्व - ज॒न्या॒म् ॥ याम॑स्य । अ॒स्य॒ कण्वः॑ । कण्वो॒ अदु॑हत् । अदु॑ह॒त् प्रपी॑नाम् । प्रपी॑नाꣳ स॒हस्र॑धाराम् । प्रपी॑ना॒मिति॒ प्र - पी॒ना॒म् । स॒हस्र॑धारा॒म् पय॑सा । स॒हस्र॑धारा॒मिति॑ स॒हस्र॑ - धा॒रा॒म् \newline

\textbf{Jatai Paata} \newline

1. यज॑मानश्च च॒ यज॑मानो॒ यज॑मानश्च । \newline
2. च॒ सी॒द॒त॒ सी॒द॒त॒ च॒ च॒ सी॒द॒त॒ । \newline
3. सी॒द॒तेति॑ सीदत । \newline
4. प्रेद्धो॑ अग्ने अग्ने॒ प्रेद्धः॒ प्रेद्धो॑ अग्ने । \newline
5. प्रेद्ध॒ इति॒ प्र - इ॒द्धः॒ । \newline
6. अ॒ग्ने॒ दी॒दि॒हि॒ दी॒दि॒ह्य॒ग्ने॒ अ॒ग्ने॒ दी॒दि॒हि॒ । \newline
7. दी॒दि॒हि॒ पु॒रः पु॒रो दी॑दिहि दीदिहि पु॒रः । \newline
8. पु॒रो नो॑ नः पु॒रः पु॒रो नः॑ । \newline
9. नो ऽज॑स्र॒या ऽज॑स्रया नो॒ नो ऽज॑स्रया । \newline
10. अज॑स्रया सू॒र्म्या॑ सू॒र्म्या ऽज॑स्र॒या ऽज॑स्रया सू॒र्म्या᳚ । \newline
11. सू॒र्म्या॑ यविष्ठ यविष्ठ सू॒र्म्या॑ सू॒र्म्या॑ यविष्ठ । \newline
12. य॒वि॒ष्ठेति॑ यविष्ठ । \newline
13. त्वाꣳ शश्व॑न्तः॒ शश्व॑न्त॒ स्त्वाम् त्वाꣳ शश्व॑न्तः । \newline
14. शश्व॑न्त॒ उपोप॒ शश्व॑न्तः॒ शश्व॑न्त॒ उप॑ । \newline
15. उप॑ यन्ति य॒न्त्युपोप॑ यन्ति । \newline
16. य॒न्ति॒ वाजा॒ वाजा॑ यन्ति यन्ति॒ वाजाः᳚ । \newline
17. वाजा॒ इति॒ वाजाः᳚ । \newline
18. वि॒धेम॑ ते ते वि॒धेम॑ वि॒धेम॑ ते । \newline
19. ते॒ प॒र॒मे प॑र॒मे ते॑ ते पर॒मे । \newline
20. प॒र॒मे जन्म॒न् जन्म॑न् पर॒मे प॑र॒मे जन्मन्न्॑ । \newline
21. जन्म॑न् नग्ने अग्ने॒ जन्म॒न् जन्म॑न् नग्ने । \newline
22. अ॒ग्ने॒ वि॒धेम॑ वि॒धेमा᳚ग्ने अग्ने वि॒धेम॑ । \newline
23. वि॒धेम॒ स्तोमैः॒ स्तोमै᳚र् वि॒धेम॑ वि॒धेम॒ स्तोमैः᳚ । \newline
24. स्तोमै॒ रव॒रे ऽव॑रे॒ स्तोमैः॒ स्तोमै॒ रव॑रे । \newline
25. अव॑रे स॒धस्थे॑ स॒धस्थे ऽव॒रे ऽव॑रे स॒धस्थे᳚ । \newline
26. स॒धस्थ॒ इति॑ स॒ध - स्थे॒ । \newline
27. यस्मा॒द् योने॒र् योने॒र् यस्मा॒द् यस्मा॒द् योनेः᳚ । \newline
28. योने॑ रु॒दारि॑ थो॒दारि॑थ॒ योने॒र् योने॑ रु॒दारि॑थ । \newline
29. उ॒दारि॑था॒ यजे॒ यज॑ उ॒दारि॑ थो॒दारि॑था॒ यजे᳚ । \newline
30. उ॒दारि॒थेत्यु॑त् - आरि॑थ । \newline
31. यजे॒ तम् तं ॅयजे॒ यजे॒ तम् । \newline
32. तम् प्र प्र तम् तम् प्र । \newline
33. प्र त्वे त्वे प्र प्र त्वे । \newline
34. त्वे ह॒वीꣳषि॑ ह॒वीꣳषि॒ त्वे त्वे ह॒वीꣳषि॑ । \newline
35. त्वे इति॒ त्वे । \newline
36. ह॒वीꣳषि॑ जुहुरे जुहुरे ह॒वीꣳषि॑ ह॒वीꣳषि॑ जुहुरे । \newline
37. जु॒हु॒रे॒ समि॑द्धे॒ समि॑द्धे जुहुरे जुहुरे॒ समि॑द्धे । \newline
38. समि॑द्ध॒ इति॒ सं - इ॒द्धे॒ । \newline
39. ताꣳ स॑वि॒तुः स॑वि॒तु स्ताम् ताꣳ स॑वि॒तुः । \newline
40. स॒वि॒तुर् वरे᳚ण्यस्य॒ वरे᳚ण्यस्य सवि॒तुः स॑वि॒तुर् वरे᳚ण्यस्य । \newline
41. वरे᳚ण्यस्य चि॒त्राम् चि॒त्रां ॅवरे᳚ण्यस्य॒ वरे᳚ण्यस्य चि॒त्राम् । \newline
42. चि॒त्रामा चि॒त्राम् चि॒त्रामा । \newline
43. आ ऽह म॒ह मा ऽहम् । \newline
44. अ॒हं ॅवृ॑णे वृणे॒ ऽह म॒हं ॅवृ॑णे । \newline
45. वृ॒णे॒ सु॒म॒तिꣳ सु॑म॒तिं ॅवृ॑णे वृणे सुम॒तिम् । \newline
46. सु॒म॒तिं ॅवि॒श्वज॑न्यां ॅवि॒श्वज॑न्याꣳ सुम॒तिꣳ सु॑म॒तिं ॅवि॒श्वज॑न्याम् । \newline
47. सु॒म॒तिमिति॑ सु - म॒तिम् । \newline
48. वि॒श्वज॑न्या॒मिति॑ वि॒श्व - ज॒न्या॒म् । \newline
49. या म॑स्यास्य॒ यां ॅया म॑स्य । \newline
50. अ॒स्य॒ कण्वः॒ कण्वो॑ अस्यास्य॒ कण्वः॑ । \newline
51. कण्वो॒ अदु॑ह॒ ददु॑ह॒त् कण्वः॒ कण्वो॒ अदु॑हत् । \newline
52. अदु॑ह॒त् प्रपी॑ना॒म् प्रपी॑ना॒ मदु॑ह॒ ददु॑ह॒त् प्रपी॑नाम् । \newline
53. प्रपी॑नाꣳ स॒हस्र॑धाराꣳ स॒हस्र॑धारा॒म् प्रपी॑ना॒म् प्रपी॑नाꣳ स॒हस्र॑धाराम् । \newline
54. प्रपी॑ना॒मिति॒ प्र - पी॒ना॒म् । \newline
55. स॒हस्र॑धारा॒म् पय॑सा॒ पय॑सा स॒हस्र॑धाराꣳ स॒हस्र॑धारा॒म् पय॑सा । \newline
56. स॒हस्र॑धारा॒मिति॑ स॒हस्र॑ - धा॒रा॒म् । \newline

\textbf{Ghana Paata } \newline

1. यज॑मानश्च च॒ यज॑मानो॒ यज॑मानश्च सीदत सीदत च॒ यज॑मानो॒ यज॑मानश्च सीदत । \newline
2. च॒ सी॒द॒त॒ सी॒द॒त॒ च॒ च॒ सी॒द॒त॒ । \newline
3. सी॒द॒तेति॑ सीदत । \newline
4. प्रेद्धो॑ अग्ने अग्ने॒ प्रेद्धः॒ प्रेद्धो॑ अग्ने दीदिहि दीदिह्यग्ने॒ प्रेद्धः॒ प्रेद्धो॑ अग्ने दीदिहि । \newline
5. प्रेद्ध॒ इति॒ प्र - इ॒द्धः॒ । \newline
6. अ॒ग्ने॒ दी॒दि॒हि॒ दी॒दि॒ह्य॒ग्ने॒ अ॒ग्ने॒ दी॒दि॒हि॒ पु॒रः पु॒रो दी॑दिह्यग्ने अग्ने दीदिहि पु॒रः । \newline
7. दी॒दि॒हि॒ पु॒रः पु॒रो दी॑दिहि दीदिहि पु॒रो नो॑ नः पु॒रो दी॑दिहि दीदिहि पु॒रो नः॑ । \newline
8. पु॒रो नो॑ नः पु॒रः पु॒रो नो ऽज॑स्र॒या ऽज॑स्रया नः पु॒रः पु॒रो नो ऽज॑स्रया । \newline
9. नो ऽज॑स्र॒या ऽज॑स्रया नो॒ नो ऽज॑स्रया सू॒र्म्या॑ सू॒र्म्या ऽज॑स्रया नो॒ नो ऽज॑स्रया सू॒र्म्या᳚ । \newline
10. अज॑स्रया सू॒र्म्या॑ सू॒र्म्या ऽज॑स्र॒या ऽज॑स्रया सू॒र्म्या॑ यविष्ठ यविष्ठ सू॒र्म्या ऽज॑स्र॒या ऽज॑स्रया सू॒र्म्या॑ यविष्ठ । \newline
11. सू॒र्म्या॑ यविष्ठ यविष्ठ सू॒र्म्या॑ सू॒र्म्या॑ यविष्ठ । \newline
12. य॒वि॒ष्ठेति॑ यविष्ठ । \newline
13. त्वाꣳ शश्व॑न्तः॒ शश्व॑न्त॒ स्त्वाम् त्वाꣳ शश्व॑न्त॒ उपोप॒ शश्व॑न्त॒ स्त्वाम् त्वाꣳ शश्व॑न्त॒ उप॑ । \newline
14. शश्व॑न्त॒ उपोप॒ शश्व॑न्तः॒ शश्व॑न्त॒ उप॑ यन्ति य॒न्त्युप॒ शश्व॑न्तः॒ शश्व॑न्त॒ उप॑ यन्ति । \newline
15. उप॑ यन्ति य॒न्त्यु पोप॑ यन्ति॒ वाजा॒ वाजा॑ य॒न्त्यु पोप॑ यन्ति॒ वाजाः᳚ । \newline
16. य॒न्ति॒ वाजा॒ वाजा॑ यन्ति यन्ति॒ वाजाः᳚ । \newline
17. वाजा॒ इति॒ वाजाः᳚ । \newline
18. वि॒धेम॑ ते ते वि॒धेम॑ वि॒धेम॑ ते पर॒मे प॑र॒मे ते॑ वि॒धेम॑ वि॒धेम॑ ते पर॒मे । \newline
19. ते॒ प॒र॒मे प॑र॒मे ते॑ ते पर॒मे जन्म॒न् जन्म॑न् पर॒मे ते॑ ते पर॒मे जन्मन्न्॑ । \newline
20. प॒र॒मे जन्म॒न् जन्म॑न् पर॒मे प॑र॒मे जन्म॑न् नग्ने अग्ने॒ जन्म॑न् पर॒मे प॑र॒मे जन्म॑न् नग्ने । \newline
21. जन्म॑न् नग्ने अग्ने॒ जन्म॒न् जन्म॑न् नग्ने वि॒धेम॑ वि॒धेमा᳚ग्ने॒ जन्म॒न् जन्म॑न् नग्ने वि॒धेम॑ । \newline
22. अ॒ग्ने॒ वि॒धेम॑ वि॒धेमा᳚ग्ने अग्ने वि॒धेम॒ स्तोमैः॒ स्तोमै᳚र् वि॒धेमा᳚ग्ने अग्ने वि॒धेम॒ स्तोमैः᳚ । \newline
23. वि॒धेम॒ स्तोमैः॒ स्तोमै᳚र् वि॒धेम॑ वि॒धेम॒ स्तोमै॒ रव॒रे ऽव॑रे॒ स्तोमै᳚र् वि॒धेम॑ वि॒धेम॒ स्तोमै॒ रव॑रे । \newline
24. स्तोमै॒ रव॒रे ऽव॑रे॒ स्तोमैः॒ स्तोमै॒ रव॑रे स॒धस्थे॑ स॒धस्थे ऽव॑रे॒ स्तोमैः॒ स्तोमै॒ रव॑रे स॒धस्थे᳚ । \newline
25. अव॑रे स॒धस्थे॑ स॒धस्थे ऽव॒रे ऽव॑रे स॒धस्थे᳚ । \newline
26. स॒धस्थ॒ इति॑ स॒ध - स्थे॒ । \newline
27. यस्मा॒द् योने॒र् योने॒र् यस्मा॒द् यस्मा॒द् योने॑ रु॒दारि॑थो॒ दारि॑थ॒ योने॒र् यस्मा॒द् यस्मा॒द् योने॑ रु॒दारि॑थ । \newline
28. योने॑ रु॒दारि॑थो॒ दारि॑थ॒ योने॒र् योने॑ रु॒दारि॑था॒ यजे॒ यज॑ उ॒दारि॑थ॒ योने॒र् योने॑ रु॒दारि॑था॒ यजे᳚ । \newline
29. उ॒दारि॑था॒ यजे॒ यज॑ उ॒दारि॑थो॒ दारि॑था॒ यजे॒ तम् तं ॅयज॑ उ॒दारि॑थो॒ दारि॑था॒ यजे॒ तम् । \newline
30. उ॒दारि॒थेत्यु॑त् - आरि॑थ । \newline
31. यजे॒ तम् तं ॅयजे॒ यजे॒ तम् प्र प्र तं ॅयजे॒ यजे॒ तम् प्र । \newline
32. तम् प्र प्र तम् तम् प्र त्वे त्वे प्र तम् तम् प्र त्वे । \newline
33. प्र त्वे त्वे प्र प्र त्वे ह॒वीꣳषि॑ ह॒वीꣳषि॒ त्वे प्र प्र त्वे ह॒वीꣳषि॑ । \newline
34. त्वे ह॒वीꣳषि॑ ह॒वीꣳषि॒ त्वे त्वे ह॒वीꣳषि॑ जुहुरे जुहुरे ह॒वीꣳषि॒ त्वे त्वे ह॒वीꣳषि॑ जुहुरे । \newline
35. त्वे इति॒ त्वे । \newline
36. ह॒वीꣳषि॑ जुहुरे जुहुरे ह॒वीꣳषि॑ ह॒वीꣳषि॑ जुहुरे॒ समि॑द्धे॒ समि॑द्धे जुहुरे ह॒वीꣳषि॑ ह॒वीꣳषि॑ जुहुरे॒ समि॑द्धे । \newline
37. जु॒हु॒रे॒ समि॑द्धे॒ समि॑द्धे जुहुरे जुहुरे॒ समि॑द्धे । \newline
38. समि॑द्ध॒ इति॒ सं - इ॒द्धे॒ । \newline
39. ताꣳ स॑वि॒तुः स॑वि॒तु स्ताम् ताꣳ स॑वि॒तुर् वरे᳚ण्यस्य॒ वरे᳚ण्यस्य सवि॒तु स्ताम् ताꣳ स॑वि॒तुर् वरे᳚ण्यस्य । \newline
40. स॒वि॒तुर् वरे᳚ण्यस्य॒ वरे᳚ण्यस्य सवि॒तुः स॑वि॒तुर् वरे᳚ण्यस्य चि॒त्राम् चि॒त्रां ॅवरे᳚ण्यस्य सवि॒तुः स॑वि॒तुर् वरे᳚ण्यस्य चि॒त्राम् । \newline
41. वरे᳚ण्यस्य चि॒त्राम् चि॒त्रां ॅवरे᳚ण्यस्य॒ वरे᳚ण्यस्य चि॒त्रा मा चि॒त्रां ॅवरे᳚ण्यस्य॒ वरे᳚ण्यस्य चि॒त्रा मा । \newline
42. चि॒त्रा मा चि॒त्राम् चि॒त्रा मा ऽह म॒ह मा चि॒त्राम् चि॒त्रा मा ऽहम् । \newline
43. आ ऽह म॒ह मा ऽहं ॅवृ॑णे वृणे॒ ऽह मा ऽहं ॅवृ॑णे । \newline
44. अ॒हं ॅवृ॑णे वृणे॒ ऽह म॒हं ॅवृ॑णे सुम॒तिꣳ सु॑म॒तिं ॅवृ॑णे॒ ऽह म॒हं ॅवृ॑णे सुम॒तिम् । \newline
45. वृ॒णे॒ सु॒म॒तिꣳ सु॑म॒तिं ॅवृ॑णे वृणे सुम॒तिं ॅवि॒श्वज॑न्यां ॅवि॒श्वज॑न्याꣳ सुम॒तिं ॅवृ॑णे वृणे सुम॒तिं ॅवि॒श्वज॑न्याम् । \newline
46. सु॒म॒तिं ॅवि॒श्वज॑न्यां ॅवि॒श्वज॑न्याꣳ सुम॒तिꣳ सु॑म॒तिं ॅवि॒श्वज॑न्याम् । \newline
47. सु॒म॒तिमिति॑ सु - म॒तिम् । \newline
48. वि॒श्वज॑न्या॒मिति॑ वि॒श्व - ज॒न्या॒म् । \newline
49. या म॑स्यास्य॒ यां ॅया म॑स्य॒ कण्वः॒ कण्वो॑ अस्य॒ यां ॅया म॑स्य॒ कण्वः॑ । \newline
50. अ॒स्य॒ कण्वः॒ कण्वो॑ अस्यास्य॒ कण्वो॒ अदु॑ह॒ ददु॑ह॒त् कण्वो॑ अस्यास्य॒ कण्वो॒ अदु॑हत् । \newline
51. कण्वो॒ अदु॑ह॒ ददु॑ह॒त् कण्वः॒ कण्वो॒ अदु॑ह॒त् प्रपी॑ना॒म् प्रपी॑ना॒ मदु॑ह॒त् कण्वः॒ कण्वो॒ अदु॑ह॒त् प्रपी॑नाम् । \newline
52. अदु॑ह॒त् प्रपी॑ना॒म् प्रपी॑ना॒ मदु॑ह॒ ददु॑ह॒त् प्रपी॑नाꣳ स॒हस्र॑धाराꣳ स॒हस्र॑धारा॒म् प्रपी॑ना॒ मदु॑ह॒ ददु॑ह॒त् प्रपी॑नाꣳ स॒हस्र॑धाराम् । \newline
53. प्रपी॑नाꣳ स॒हस्र॑धाराꣳ स॒हस्र॑धारा॒म् प्रपी॑ना॒म् प्रपी॑नाꣳ स॒हस्र॑धारा॒म् पय॑सा॒ पय॑सा स॒हस्र॑धारा॒म् प्रपी॑ना॒म् प्रपी॑नाꣳ स॒हस्र॑धारा॒म् पय॑सा । \newline
54. प्रपी॑ना॒मिति॒ प्र - पी॒ना॒म् । \newline
55. स॒हस्र॑धारा॒म् पय॑सा॒ पय॑सा स॒हस्र॑धाराꣳ स॒हस्र॑धारा॒म् पय॑सा म॒हीम् म॒हीम् पय॑सा स॒हस्र॑धाराꣳ स॒हस्र॑धारा॒म् पय॑सा म॒हीम् । \newline
56. स॒हस्र॑धारा॒मिति॑ स॒हस्र॑ - धा॒रा॒म् । \newline
\pagebreak
\markright{ TS 4.6.5.5  \hfill https://www.vedavms.in \hfill}

\section{ TS 4.6.5.5 }

\textbf{TS 4.6.5.5 } \newline
\textbf{Samhita Paata} \newline

पय॑सा म॒हीं गां ॥ स॒प्त ते॑ अग्ने स॒मिधः॑ स॒प्त जि॒ह्वाः स॒प्तर्.ष॑यः स॒प्त धाम॑ प्रि॒याणि॑ । स॒प्त होत्राः᳚ सप्त॒धा त्वा॑ यजन्ति स॒प्त योनी॒रा पृ॑णस्वा घृ॒तेन॑ ॥ ई॒दृङ् चा᳚न्या॒दृङ् चै॑ता॒दृङ्च॑ प्रति॒दृङ् च॑ मि॒तश्च॒ संमि॑तश्च॒ सभ॑राः ॥ शु॒क्रज्यो॑तिश्च चि॒त्रज्यो॑तिश्च स॒त्यज्यो॑तिश्च॒ ज्योति॑ष्माꣳश्च स॒त्यश्च॑र्त॒पाश्चात्यꣳ॑हाः ॥ \newline

\textbf{Pada Paata} \newline

पय॑सा । म॒हीम् । गाम् ॥ स॒प्त । ते॒ । अ॒ग्ने॒ । स॒मिध॒ इति॑ सम् - इधः॑ । स॒प्त । जि॒ह्वाः । स॒प्त । ऋष॑यः । स॒प्त । धाम॑ । प्रि॒याणि॑ ॥ स॒प्त । होत्राः᳚ । स॒प्त॒धेति॑ सप्त - धा । त्वा॒ । य॒ज॒न्ति॒ । स॒प्त । योनीः᳚ । एति॑ । पृ॒ण॒स्व॒ । घृ॒तेन॑ ॥ ई॒दृङ् । च॒ । अ॒न्या॒दृङ् । च॒ । ए॒ता॒दृङ् । च॒ । प्र॒ति॒दृङिति॑ प्रति - दृङ् । च॒ । मि॒तः । च॒ । संमि॑त॒ इति॒ सं - मि॒तः॒ । च॒ । सभ॑रा॒ इति॒ स - भ॒राः॒ ॥ शु॒क्रज्यो॑ति॒रिति॑ शु॒क्र-ज्यो॒तिः॒ । च॒ । चि॒त्रज्यो॑ति॒रिति॑ चि॒त्र-ज्यो॒तिः॒ । च॒ । स॒त्यज्यो॑ति॒रिति॑ स॒त्य - ज्यो॒तिः॒ । च॒ । ज्योति॑ष्मान् । च॒ । स॒त्यः । च॒ । ऋ॒त॒पा इत्यृ॑त-पाः । च॒ । अत्यꣳ॑हा॒ इत्यति॑ - अꣳ॒॒हाः॒ ॥  \newline


\textbf{Krama Paata} \newline

पय॑सा म॒हीम् । म॒हीम् गाम् । गामिति॒ गाम् ॥ स॒प्त ते᳚ । ते॒ अ॒ग्ने॒ । अ॒ग्ने॒ स॒मिधः॑ । स॒मिधः॑ स॒प्त । स॒मिध॒ इति॑ सम् - इधः॑ । स॒प्त॒ जि॒ह्वाः । जि॒ह्वाः स॒प्त । स॒प्तर्.ष॑यः । ऋष॑यः स॒प्त । स॒प्त धाम॑ । धाम॑ प्रि॒याणि॑ । प्रि॒याणीति॑ प्रि॒याणि॑ ॥ स॒प्त होत्राः᳚ । होत्राः᳚ सप्त॒धा । स॒प्त॒धा त्वा᳚ । स॒प्त॒धेति॑ सप्त - धा । त्वा॒ य॒ज॒न्ति॒ । य॒ज॒न्ति॒ स॒प्त । स॒प्त योनीः᳚ । योनी॒रा । आ पृ॑णस्व । पृ॒ण॒स्वा॒ घृ॒तेन॑ । घृ॒तेनेति॑ घृ॒तेन॑ ॥ ई॒दृङ् च॑ । चा॒न्या॒दृङ्ङ् । अ॒न्या॒दृङ् च॑ । चै॒ता॒दृङ्ङ् । ए॒ता॒दृङ् च॑ । च॒ प्र॒ति॒दृङ्ङ् । प्र॒ति॒दृङ् च॑ । प्र॒ति॒दृङ्ङिति॑ प्रति - दृङ्ङ् । च॒ मि॒तः । मि॒तश्च॑ । च॒ सम्मि॑तः । सम्मि॑तश्च । सम्मि॑त॒ इति॒ सम् - मि॒तः॒ । च॒ सभ॑राः । सभ॑रा॒ इति॒ स - भ॒राः॒ ॥ शु॒क्रज्यो॑तिश्च । शु॒क्रज्यो॑ति॒रिति॑ शु॒क्र - ज्यो॒तिः॒ । च॒ चि॒त्रज्यो॑तिः । चि॒त्रज्यो॑तिश्च । चि॒त्रज्यो॑ति॒रिति॑ चि॒त्र - ज्यो॒तिः॒ । च॒ स॒त्यज्यो॑तिः । स॒त्यज्यो॑तिश्च । स॒त्यज्यो॑ति॒रिति॑ स॒त्य - ज्यो॒तिः॒ । च॒ ज्योति॑ष्मान् । ज्योति॑ष्माꣳश्च । च॒ स॒त्यः । स॒त्यश्च॑ । च॒र्त॒पाः । ऋ॒त॒पाश्च॑ । ऋ॒त॒पा इत्यृ॑त - पाः । चात्यꣳ॑हाः । अत्यꣳ॑हा॒ इत्यति॑ - अꣳ॒॒हाः॒ । \newline

\textbf{Jatai Paata} \newline

1. पय॑सा म॒हीम् म॒हीम् पय॑सा॒ पय॑सा म॒हीम् । \newline
2. म॒हीम् गाम् गाम् म॒हीम् म॒हीम् गाम् । \newline
3. गामिति॒ गाम् । \newline
4. स॒प्त ते॑ ते स॒प्त स॒प्त ते᳚ । \newline
5. ते॒ अ॒ग्ने॒ ऽग्ने॒ ते॒ ते॒ अ॒ग्ने॒ । \newline
6. अ॒ग्ने॒ स॒मिधः॑ स॒मिधो᳚ ऽग्ने ऽग्ने स॒मिधः॑ । \newline
7. स॒मिधः॑ स॒प्त स॒प्त स॒मिधः॑ स॒मिधः॑ स॒प्त । \newline
8. स॒मिध॒ इति॑ सम् - इधः॑ । \newline
9. स॒प्त जि॒ह्वा जि॒ह्वाः स॒प्त स॒प्त जि॒ह्वाः । \newline
10. जि॒ह्वाः स॒प्त स॒प्त जि॒ह्वा जि॒ह्वाः स॒प्त । \newline
11. स॒प्त र्.ष॑य॒ ऋष॑यः स॒प्त स॒प्त र्.ष॑यः । \newline
12. ऋष॑यः स॒प्त स॒प्त र्.ष॑य॒ ऋष॑यः स॒प्त । \newline
13. स॒प्त धाम॒ धाम॑ स॒प्त स॒प्त धाम॑ । \newline
14. धाम॑ प्रि॒याणि॑ प्रि॒याणि॒ धाम॒ धाम॑ प्रि॒याणि॑ । \newline
15. प्रि॒याणीति॑ प्रि॒याणि॑ । \newline
16. स॒प्त होत्रा॒ होत्राः᳚ स॒प्त स॒प्त होत्राः᳚ । \newline
17. होत्राः᳚ सप्त॒धा स॑प्त॒धा होत्रा॒ होत्राः᳚ सप्त॒धा । \newline
18. स॒प्त॒धा त्वा᳚ त्वा सप्त॒धा स॑प्त॒धा त्वा᳚ । \newline
19. स॒प्त॒धेति॑ सप्त - धा । \newline
20. त्वा॒ य॒ज॒न्ति॒ य॒ज॒न्ति॒ त्वा॒ त्वा॒ य॒ज॒न्ति॒ । \newline
21. य॒ज॒न्ति॒ स॒प्त स॒प्त य॑जन्ति यजन्ति स॒प्त । \newline
22. स॒प्त योनी॒र् योनीः᳚ स॒प्त स॒प्त योनीः᳚ । \newline
23. योनी॒रा योनी॒र् योनी॒रा । \newline
24. आ पृ॑णस्व पृण॒स्वा पृ॑णस्व । \newline
25. पृ॒ण॒स्वा॒ घृ॒तेन॑ घृ॒तेन॑ पृणस्व पृणस्वा घृ॒तेन॑ । \newline
26. घृ॒तेनेति॑ घृ॒तेन॑ । \newline
27. ई॒दृङ् च॑ चे॒दृङ् ङी॒दृङ् च॑ । \newline
28. चा॒न्या॒दृङ् ङ॑न्या॒दृङ् च॑ चान्या॒दृङ् । \newline
29. अ॒न्या॒दृङ् च॑ चान्या॒दृङ् ङ॑न्या॒दृङ् च॑ । \newline
30. चै॒ता॒दृङ् ङे॑ता॒दृङ् च॑ चैता॒दृङ् । \newline
31. ए॒ता॒दृङ् च॑ चैता॒दृङ् ङे॑ता॒दृङ् च॑ । \newline
32. च॒ प्र॒ति॒दृङ् प्र॑ति॒दृङ् च॑ च प्रति॒दृङ् । \newline
33. प्र॒ति॒दृङ् च॑ च प्रति॒दृङ् प्र॑ति॒दृङ् च॑ । \newline
34. प्र॒ति॒दृङ्ङिति॑ प्रति - दृङ् । \newline
35. च॒ मि॒तो मि॒तश्च॑ च मि॒तः । \newline
36. मि॒तश्च॑ च मि॒तो मि॒तश्च॑ । \newline
37. च॒ संमि॑तः॒ संमि॑तश्च च॒ संमि॑तः । \newline
38. संमि॑तश्च च॒ संमि॑तः॒ संमि॑तश्च । \newline
39. संमि॑त॒ इति॒ सं - मि॒तः॒ । \newline
40. च॒ सभ॑राः॒ सभ॑राश्च च॒ सभ॑राः । \newline
41. सभ॑रा॒ इति॒ स - भ॒राः॒ । \newline
42. शु॒क्रज्यो॑तिश्च च शु॒क्रज्यो॑तिः शु॒क्रज्यो॑तिश्च । \newline
43. शु॒क्रज्यो॑ति॒रिति॑ शु॒क्र - ज्यो॒तिः॒ । \newline
44. च॒ चि॒त्रज्यो॑ति श्चि॒त्रज्यो॑तिश्च च चि॒त्रज्यो॑तिः । \newline
45. चि॒त्रज्यो॑तिश्च च चि॒त्रज्यो॑ति श्चि॒त्रज्यो॑तिश्च । \newline
46. चि॒त्रज्यो॑ति॒रिति॑ चि॒त्र - ज्यो॒तिः॒ । \newline
47. च॒ स॒त्यज्यो॑तिः स॒त्यज्यो॑तिश्च च स॒त्यज्यो॑तिः । \newline
48. स॒त्यज्यो॑तिश्च च स॒त्यज्यो॑तिः स॒त्यज्यो॑तिश्च । \newline
49. स॒त्यज्यो॑ति॒रिति॑ स॒त्य - ज्यो॒तिः॒ । \newline
50. च॒ ज्योति॑ष्मा॒न् ज्योति॑ष्माꣳश्च च॒ ज्योति॑ष्मान् । \newline
51. ज्योति॑ष्माꣳश्च च॒ ज्योति॑ष्मा॒न् ज्योति॑ष्माꣳश्च । \newline
52. च॒ स॒त्यः स॒त्यश्च॑ च स॒त्यः । \newline
53. स॒त्यश्च॑ च स॒त्यः स॒त्यश्च॑ । \newline
54. च॒ र्‌त॒पा ऋ॑त॒पाश्च॑ च र्‌त॒पाः । \newline
55. ऋ॒त॒पाश्च॑ च र्‌त॒पा ऋ॑त॒पाश्च॑ । \newline
56. ऋ॒त॒पा इत्यृ॑त - पाः । \newline
57. चात्यꣳ॑हा॒ अत्यꣳ॑हाश्च॒ चात्यꣳ॑हाः । \newline
58. अत्यꣳ॑हा॒ इत्यति॑ - अꣳ॒॒हाः॒ । \newline

\textbf{Ghana Paata } \newline

1. पय॑सा म॒हीम् म॒हीम् पय॑सा॒ पय॑सा म॒हीम् गाम् गाम् म॒हीम् पय॑सा॒ पय॑सा म॒हीम् गाम् । \newline
2. म॒हीम् गाम् गाम् म॒हीम् म॒हीम् गाम् । \newline
3. गामिति॒ गाम् । \newline
4. स॒प्त ते॑ ते स॒प्त स॒प्त ते॑ अग्ने ऽग्ने ते स॒प्त स॒प्त ते॑ अग्ने । \newline
5. ते॒ अ॒ग्ने॒ ऽग्ने॒ ते॒ ते॒ अ॒ग्ने॒ स॒मिधः॑ स॒मिधो᳚ ऽग्ने ते ते अग्ने स॒मिधः॑ । \newline
6. अ॒ग्ने॒ स॒मिधः॑ स॒मिधो᳚ ऽग्ने ऽग्ने स॒मिधः॑ स॒प्त स॒प्त स॒मिधो᳚ ऽग्ने ऽग्ने स॒मिधः॑ स॒प्त । \newline
7. स॒मिधः॑ स॒प्त स॒प्त स॒मिधः॑ स॒मिधः॑ स॒प्त जि॒ह्वा जि॒ह्वाः स॒प्त स॒मिधः॑ स॒मिधः॑ स॒प्त जि॒ह्वाः । \newline
8. स॒मिध॒ इति॑ सम् - इधः॑ । \newline
9. स॒प्त जि॒ह्वा जि॒ह्वाः स॒प्त स॒प्त जि॒ह्वाः स॒प्त स॒प्त जि॒ह्वाः स॒प्त स॒प्त जि॒ह्वाः स॒प्त । \newline
10. जि॒ह्वाः स॒प्त स॒प्त जि॒ह्वा जि॒ह्वाः स॒प्त र्.ष॑य॒ ऋष॑यः स॒प्त जि॒ह्वा जि॒ह्वाः स॒प्त र्.ष॑यः । \newline
11. स॒प्त र्.ष॑य॒ ऋष॑यः स॒प्त स॒प्त र्.ष॑यः स॒प्त स॒प्त र्.ष॑यः स॒प्त स॒प्त र्.ष॑यः स॒प्त । \newline
12. ऋष॑यः स॒प्त स॒प्त र्.ष॑य॒ ऋष॑यः स॒प्त धाम॒ धाम॑ स॒प्त र्.ष॑य॒ ऋष॑यः स॒प्त धाम॑ । \newline
13. स॒प्त धाम॒ धाम॑ स॒प्त स॒प्त धाम॑ प्रि॒याणि॑ प्रि॒याणि॒ धाम॑ स॒प्त स॒प्त धाम॑ प्रि॒याणि॑ । \newline
14. धाम॑ प्रि॒याणि॑ प्रि॒याणि॒ धाम॒ धाम॑ प्रि॒याणि॑ । \newline
15. प्रि॒याणीति॑ प्रि॒याणि॑ । \newline
16. स॒प्त होत्रा॒ होत्राः᳚ स॒प्त स॒प्त होत्राः᳚ सप्त॒धा स॑प्त॒धा होत्राः᳚ स॒प्त स॒प्त होत्राः᳚ सप्त॒धा । \newline
17. होत्राः᳚ सप्त॒धा स॑प्त॒धा होत्रा॒ होत्राः᳚ सप्त॒धा त्वा᳚ त्वा सप्त॒धा होत्रा॒ होत्राः᳚ सप्त॒धा त्वा᳚ । \newline
18. स॒प्त॒धा त्वा᳚ त्वा सप्त॒धा स॑प्त॒धा त्वा॑ यजन्ति यजन्ति त्वा सप्त॒धा स॑प्त॒धा त्वा॑ यजन्ति । \newline
19. स॒प्त॒धेति॑ सप्त - धा । \newline
20. त्वा॒ य॒ज॒न्ति॒ य॒ज॒न्ति॒ त्वा॒ त्वा॒ य॒ज॒न्ति॒ स॒प्त स॒प्त य॑जन्ति त्वा त्वा यजन्ति स॒प्त । \newline
21. य॒ज॒न्ति॒ स॒प्त स॒प्त य॑जन्ति यजन्ति स॒प्त योनी॒र् योनीः᳚ स॒प्त य॑जन्ति यजन्ति स॒प्त योनीः᳚ । \newline
22. स॒प्त योनी॒र् योनीः᳚ स॒प्त स॒प्त योनी॒रा योनीः᳚ स॒प्त स॒प्त योनी॒रा । \newline
23. योनी॒रा योनी॒र् योनी॒रा पृ॑णस्व पृण॒स्वा योनी॒र् योनी॒रा पृ॑णस्व । \newline
24. आ पृ॑णस्व पृण॒स्वा पृ॑णस्वा घृ॒तेन॑ घृ॒तेन॑ पृण॒स्वा पृ॑णस्वा घृ॒तेन॑ । \newline
25. पृ॒ण॒स्वा॒ घृ॒तेन॑ घृ॒तेन॑ पृणस्व पृणस्वा घृ॒तेन॑ । \newline
26. घृ॒तेनेति॑ घृ॒तेन॑ । \newline
27. ई॒दृङ् च॑ चे॒दृङ् ङी॒दृङ् चा᳚न्या॒दृङ् ङ॑न्या॒दृङ् चे॒दृङ् ङी॒दृङ् चा᳚न्या॒दृङ् । \newline
28. चा॒न्या॒दृङ् ङ॑न्या॒दृङ् च॑ चान्या॒दृङ् च॑ चान्या॒दृङ् च॑ चान्या॒दृङ् च॑ । \newline
29. अ॒न्या॒दृङ् च॑ चान्या॒दृङ् ङ॑न्या॒दृङ् चै॑ता॒दृङ् ङे॑ता॒दृङ् चा᳚न्या॒दृङ् ङ॑न्या॒दृङ् चै॑ता॒दृङ् । \newline
30. चै॒ता॒दृङ् ङे॑ता॒दृङ् च॑ चैता॒दृङ् च॑ चैता॒दृङ् च॑ चैता॒दृङ् च॑ । \newline
31. ए॒ता॒दृङ् च॑ चैता॒दृङ् ङे॑ता॒दृङ् च॑ प्रति॒दृङ् प्र॑ति॒दृङ् चै॑ता॒दृङ् ङे॑ता॒दृङ् च॑ प्रति॒दृङ् । \newline
32. च॒ प्र॒ति॒दृङ् प्र॑ति॒दृङ् च॑ च प्रति॒दृङ् च॑ च प्रति॒दृङ् च॑ च प्रति॒दृङ् च॑ । \newline
33. प्र॒ति॒दृङ् च॑ च प्रति॒दृङ् प्र॑ति॒दृङ् च॑ मि॒तो मि॒तश्च॑ प्रति॒दृङ् प्र॑ति॒दृङ् च॑ मि॒तः । \newline
34. प्र॒ति॒दृङ्ङिति॑ प्रति - दृङ् । \newline
35. च॒ मि॒तो मि॒तश्च॑ च मि॒तश्च॑ च मि॒तश्च॑ च मि॒तश्च॑ । \newline
36. मि॒तश्च॑ च मि॒तो मि॒तश्च॒ संमि॑तः॒ संमि॑तश्च मि॒तो मि॒तश्च॒ संमि॑तः । \newline
37. च॒ संमि॑तः॒ संमि॑तश्च च॒ संमि॑तश्च च॒ संमि॑तश्च च॒ संमि॑तश्च । \newline
38. संमि॑तश्च च॒ संमि॑तः॒ संमि॑तश्च॒ सभ॑राः॒ सभ॑राश्च॒ संमि॑तः॒ संमि॑तश्च॒ सभ॑राः । \newline
39. संमि॑त॒ इति॒ सं - मि॒तः॒ । \newline
40. च॒ सभ॑राः॒ सभ॑राश्च च॒ सभ॑राः । \newline
41. सभ॑रा॒ इति॒ स - भ॒राः॒ । \newline
42. शु॒क्रज्यो॑तिश्च च शु॒क्रज्यो॑तिः शु॒क्रज्यो॑तिश्च चि॒त्रज्यो॑ति श्चि॒त्रज्यो॑तिश्च शु॒क्रज्यो॑तिः शु॒क्रज्यो॑तिश्च चि॒त्रज्यो॑तिः । \newline
43. शु॒क्रज्यो॑ति॒रिति॑ शु॒क्र - ज्यो॒तिः॒ । \newline
44. च॒ चि॒त्रज्यो॑ति श्चि॒त्रज्यो॑तिश्च च चि॒त्रज्यो॑तिश्च च चि॒त्रज्यो॑तिश्च च चि॒त्रज्यो॑तिश्च । \newline
45. चि॒त्रज्यो॑तिश्च च चि॒त्रज्यो॑ति श्चि॒त्रज्यो॑तिश्च स॒त्यज्यो॑तिः स॒त्यज्यो॑तिश्च चि॒त्रज्यो॑ति श्चि॒त्रज्यो॑तिश्च स॒त्यज्यो॑तिः । \newline
46. चि॒त्रज्यो॑ति॒रिति॑ चि॒त्र - ज्यो॒तिः॒ । \newline
47. च॒ स॒त्यज्यो॑तिः स॒त्यज्यो॑तिश्च च स॒त्यज्यो॑तिश्च च स॒त्यज्यो॑तिश्च च स॒त्यज्यो॑तिश्च । \newline
48. स॒त्यज्यो॑तिश्च च स॒त्यज्यो॑तिः स॒त्यज्यो॑तिश्च॒ ज्योति॑ष्मा॒न् ज्योति॑ष्माꣳश्च स॒त्यज्यो॑तिः स॒त्यज्यो॑तिश्च॒ ज्योति॑ष्मान् । \newline
49. स॒त्यज्यो॑ति॒रिति॑ स॒त्य - ज्यो॒तिः॒ । \newline
50. च॒ ज्योति॑ष्मा॒न् ज्योति॑ष्माꣳश्च च॒ ज्योति॑ष्माꣳश्च च॒ ज्योति॑ष्माꣳश्च च॒ ज्योति॑ष्माꣳश्च । \newline
51. ज्योति॑ष्माꣳश्च च॒ ज्योति॑ष्मा॒न् ज्योति॑ष्माꣳश्च स॒त्यः स॒त्यश्च॒ ज्योति॑ष्मा॒न् ज्योति॑ष्माꣳश्च स॒त्यः । \newline
52. च॒ स॒त्यः स॒त्यश्च॑ च स॒त्यश्च॑ च स॒त्यश्च॑ च स॒त्यश्च॑ । \newline
53. स॒त्यश्च॑ च स॒त्यः स॒त्यश्च॑ र्‌त॒पा ऋ॑त॒पाश्च॑ स॒त्यः स॒त्यश्च॑ र्‌त॒पाः । \newline
54. च॒ र्‌त॒पा ऋ॑त॒पाश्च॑ च र्‌त॒पाश्च॑ च र्‌त॒पाश्च॑ च र्‌त॒पाश्च॑ । \newline
55. ऋ॒त॒पाश्च॑ च र्‌त॒पा ऋ॑त॒पा श्चात्यꣳ॑हा॒ अत्यꣳ॑हाश्च र्‌त॒पा ऋ॑त॒पा श्चात्यꣳ॑हाः । \newline
56. ऋ॒त॒पा इत्यृ॑त - पाः । \newline
57. चात्यꣳ॑हा॒ अत्यꣳ॑हाश्च॒ चात्यꣳ॑हाः । \newline
58. अत्यꣳ॑हा॒ इत्यति॑ - अꣳ॒॒हाः॒ । \newline
\pagebreak
\markright{ TS 4.6.5.6  \hfill https://www.vedavms.in \hfill}

\section{ TS 4.6.5.6 }

\textbf{TS 4.6.5.6 } \newline
\textbf{Samhita Paata} \newline

ऋ॒त॒जिच्च॑ सत्य॒जिच्च॑ सेन॒जिच्च॑ सु॒षेण॒श्चान्त्य॑मित्रश्च दू॒रेअ॑मित्रश्च ग॒णः ॥ ऋ॒तश्च॑ स॒त्यश्च॑ ध्रु॒वश्च॑ ध॒रुण॑श्च ध॒र्ता च॑ विध॒र्ता च॑ विधार॒यः ॥ ई॒दृक्षा॑स एता॒दृक्षा॑स ऊ॒ षुणः॑ स॒दृक्षा॑सः॒ प्रति॑सदृक्षास॒ एत॑न । मि॒तास॑श्च॒ संमि॑तासश्च न ऊ॒तये॒ सभ॑रसो मरुतो य॒ज्ञे अ॒स्मिन्निन्द्रं॒ दैवी॒र्विशो॑ म॒रुतोऽनु॑वर्त्मानो॒ ( ) यथेन्द्रं॒ दैवी॒र्विशो॑ म॒रुतोऽनु॑वर्त्मान ए॒वमि॒मं ॅयज॑मानं॒ दैवी᳚श्च॒ विशो॒ मानु॑षी॒श्चानु॑वर्त्मानो भवन्तु ॥ \newline

\textbf{Pada Paata} \newline

ऋ॒त॒जिदित्यृ॑त - जित् । च॒ । स॒त्य॒जिदिति॑ सत्य - जित् । च॒ । से॒न॒जिदिति॑ सेन - जित् । च॒ । सु॒षेण॒ इति॑ सु - सेनः॑ । च॒ । अन्त्य॑मित्र॒ इत्यन्ति॑-अ॒मि॒त्रः॒ । च॒ । दू॒रे अ॑मित्र॒ इति॑ दू॒रे - अ॒मि॒त्रः॒ । च॒ । ग॒णः ॥ ऋ॒तः । च॒ । स॒त्यः । च॒ । ध्रु॒वः । च॒ । ध॒रुणः॑ । च॒ । ध॒र्ता । च॒ । वि॒ध॒र्तेति॑ वि - ध॒र्ता । च॒ । वि॒धा॒र॒य इति॑ वि-धा॒र॒यः ॥ ई॒दृक्षा॑सः । ए॒ता॒दृक्षा॑सः । उ॒ । स्विति॑ । नः॒ । स॒दृक्षा॑सः । प्रति॑सदृक्षास॒ इति॒ प्रति॑ - स॒दृ॒क्षा॒सः॒ । एति॑ । इ॒त॒न॒ ॥ मि॒तासः॑ । च॒ । संमि॑तास॒ इति॒ सं - मि॒ता॒सः॒ । च॒ । नः॒ । ऊ॒तये᳚ । सभ॑रस॒ इति॒ स - भ॒र॒सः॒ । म॒रु॒तः॒ । य॒ज्ञे । अ॒स्मिन्न् । इन्द्र᳚म् । दैवीः᳚ । विशः॑ । म॒रुतः॑ । अनु॑वर्त्मान॒ इत्यनु॑-व॒र्त्मा॒नः॒ ( ) । यथा᳚ । इन्द्र᳚म् । दैवीः᳚ । विशः॑ । म॒रुतः॑ । अनु॑वर्त्मान॒ इत्यनु॑ - व॒र्त्मा॒नः॒ । ए॒वम् । इ॒मम् । यज॑मानम् । दैवीः᳚ । च॒ । विशः॑ । मानु॑षीः । च॒ । अनु॑वर्त्मान॒ इत्यनु॑ - व॒र्त्मा॒नः॒ । भ॒व॒न्तु॒ ॥  \newline


\textbf{Krama Paata} \newline

ऋ॒त॒जिच् च॑ । ऋ॒त॒जिदित्यृ॑त - जित् । च॒ स॒त्य॒जित् । स॒त्य॒जिच् च॑ । स॒त्य॒जिदिति॑ सत्य - जित् । च॒ से॒न॒जित् । से॒न॒जिच् च॑ । से॒न॒जिदिति॑ सेन - जित् । च॒ सु॒षेणः॑ । सु॒षेण॑श्च । सु॒षेण॒ इति॑ सु - सेनः॑ । चान्त्य॑मित्रः । अन्त्य॑मित्रश्च । अन्त्य॑मित्र॒ इत्यन्ति॑ - अ॒मि॒त्रः॒ । च॒ दू॒रेअ॑मित्रः । दू॒रेअ॑मित्रश्च । दू॒रेअ॑मित्र॒ इति॑ दू॒रे - अ॒मि॒त्रः॒ । च॒ ग॒णः । ग॒ण इति॑ ग॒णः ॥ ऋ॒तश्च॑ । च॒ स॒त्यः । स॒त्यश्च॑ । च॒ ध्रु॒वः । ध्रु॒वश्च॑ । च॒ ध॒रुणः॑ । ध॒रुण॑श्च । च॒ ध॒र्ता । ध॒र्ता च॑ । च॒ वि॒ध॒र्ता । वि॒ध॒र्ता च॑ । वि॒ध॒र्तेति॑ वि - ध॒र्ता । च॒ वि॒धा॒र॒यः । वि॒धा॒र॒य इति॑ वि - धा॒र॒यः ॥ ई॒दृक्षा॑स एता॒दृक्षा॑सः । ए॒ता॒दृक्षा॑स उ । ऊ॒ षु णः॑ । सु नः॑ । नः॒ स॒दृक्षा॑सः । स॒दृक्षा॑सः॒ प्रति॑सदृक्षासः । प्रति॑सदृक्षास॒ आ । प्रति॑सदृक्षास॒ इति॒ प्रति॑ - स॒दृ॒क्षा॒सः॒ । एत॑न । इ॒त॒नेती॑तन ॥ मि॒तास॑श्च । च॒ सम्मि॑तासः । सम्मि॑तासश्च । सम्मि॑तास॒ इति॒ सम् - मि॒ता॒सः॒ । च॒ नः॒ । न॒ ऊ॒तये᳚ । ऊ॒तये॒ सभ॑रसः । सभ॑रसो मरुतः । सभ॑रस॒ इति॒ स - भ॒र॒सः॒ । म॒रु॒तो॒ य॒ज्ञे । य॒ज्ञे अ॒स्मिन्न् । अ॒स्मिन्निन्द्र᳚म् । इन्द्र॒म् दैवीः᳚ । दैवी॒र् विशः॑ । विशो॑ म॒रुतः॑ । म॒रुतोऽनु॑वर्त्मानः । अनु॑वर्त्मानो॒ यथा᳚ । अनु॑वर्त्मान॒ इत्यनु॑ - व॒र्त्मा॒नः॒ । यथेन्द्र᳚म् । इन्द्र॒म् दैवीः᳚ । दैवी॒र् विशः॑ । विशो॑ म॒रुतः॑ । म॒रुतोऽनु॑वर्त्मानः । अनु॑वर्त्मान ए॒वम् । अनु॑वर्त्मान॒ इत्यनु॑ - व॒र्त्मा॒नः॒ । 
ए॒वमि॒मम् । इ॒मं ॅयज॑मानम् । यज॑मान॒म् दैवीः᳚ । दैवी᳚श्च । च॒ विशः॑ । विशो॒ मानु॑षीः । मानु॑षीश्च । चानु॑वर्त्मानः ( ) । अनु॑वर्त्मानो भवन्तु । अनु॑वर्त्मान॒ इत्यनु॑ - व॒र्त्मा॒नः॒ । भ॒व॒न्त्विति॑ भवन्तु । \newline

\textbf{Jatai Paata} \newline

1. ऋ॒त॒जिच् च॑ च र्‌त॒जि दृ॑त॒जिच् च॑ । \newline
2. ऋ॒त॒जिदित्यृ॑त - जित् । \newline
3. च॒ स॒त्य॒जिथ् स॑त्य॒जिच् च॑ च सत्य॒जित् । \newline
4. स॒त्य॒जिच् च॑ च सत्य॒जिथ् स॑त्य॒जिच् च॑ । \newline
5. स॒त्य॒जिदिति॑ सत्य - जित् । \newline
6. च॒ से॒न॒जिथ् से॑न॒जिच् च॑ च सेन॒जित् । \newline
7. से॒न॒जिच् च॑ च सेन॒जिथ् से॑न॒जिच् च॑ । \newline
8. से॒न॒जिदिति॑ सेन - जित् । \newline
9. च॒ सु॒षेणः॑ सु॒षेण॑श्च च सु॒षेणः॑ । \newline
10. सु॒षेण॑श्च च सु॒षेणः॑ सु॒षेण॑श्च । \newline
11. सु॒षेण॒ इति॑ सु - सेनः॑ । \newline
12. चान्त्य॑मित्रो॒ अन्त्य॑मित्रश्च॒ चान्त्य॑मित्रः । \newline
13. अन्त्य॑मित्रश्च॒ चान्त्य॑मित्रो॒ अन्त्य॑मित्रश्च । \newline
14. अन्त्य॑मित्र॒ इत्यन्ति॑ - अ॒मि॒त्रः॒ । \newline
15. च॒ दू॒रेअ॑मित्रो दू॒रेअ॑मित्रश्च च दू॒रेअ॑मित्रः । \newline
16. दू॒रेअ॑मित्रश्च च दू॒रेअ॑मित्रो दू॒रेअ॑मित्रश्च । \newline
17. दू॒रेअ॑मित्र॒ इति॑ दू॒रे - अ॒मि॒त्रः॒ । \newline
18. च॒ ग॒णो ग॒णश्च॑ च ग॒णः । \newline
19. ग॒ण इति॑ ग॒णः । \newline
20. ऋ॒तश्च॑ च॒ र्‌त ऋ॒तश्च॑ । \newline
21. च॒ स॒त्यः स॒त्यश्च॑ च स॒त्यः । \newline
22. स॒त्यश्च॑ च स॒त्यः स॒त्यश्च॑ । \newline
23. च॒ ध्रु॒वो ध्रु॒वश्च॑ च ध्रु॒वः । \newline
24. ध्रु॒वश्च॑ च ध्रु॒वो ध्रु॒वश्च॑ । \newline
25. च॒ ध॒रुणो॑ ध॒रुण॑श्च च ध॒रुणः॑ । \newline
26. ध॒रुण॑श्च च ध॒रुणो॑ ध॒रुण॑श्च । \newline
27. च॒ ध॒र्ता ध॒र्ता च॑ च ध॒र्ता । \newline
28. ध॒र्ता च॑ च ध॒र्ता ध॒र्ता च॑ । \newline
29. च॒ वि॒ध॒र्ता वि॑ध॒र्ता च॑ च विध॒र्ता । \newline
30. वि॒ध॒र्ता च॑ च विध॒र्ता वि॑ध॒र्ता च॑ । \newline
31. वि॒ध॒र्तेति॑ वि - ध॒र्ता । \newline
32. च॒ वि॒धा॒र॒यो वि॑धार॒यश्च॑ च विधार॒यः । \newline
33. वि॒धा॒र॒य इति॑ वि - धा॒र॒यः । \newline
34. ई॒दृक्षा॑स एता॒दृक्षा॑स एता॒दृक्षा॑स ई॒दृक्षा॑स ई॒दृक्षा॑स एता॒दृक्षा॑सः । \newline
35. ए॒ता॒दृक्षा॑स उ वु वेता॒दृक्षा॑स एता॒दृक्षा॑स उ । \newline
36. ऊ॒ षु णो॑ नः॒ सू॑ षु णः॑ । \newline
37. सु णो॑ नः॒ सु सु णः॑ । \newline
38. नः॒ स॒दृक्षा॑सः स॒दृक्षा॑सो नो नः स॒दृक्षा॑सः । \newline
39. स॒दृक्षा॑सः॒ प्रति॑सदृक्षासः॒ प्रति॑सदृक्षासः स॒दृक्षा॑सः स॒दृक्षा॑सः॒ प्रति॑सदृक्षासः । \newline
40. प्रति॑सदृक्षास॒ आ प्रति॑सदृक्षासः॒ प्रति॑सदृक्षास॒ आ । \newline
41. प्रति॑सदृक्षास॒ इति॒ प्रति॑ - स॒दृ॒क्षा॒सः॒ । \newline
42. एत॑ने त॒नेत॑न । \newline
43. इ॒त॒नेती॑तन । \newline
44. मि॒तास॑श्च च मि॒तासो॑ मि॒तास॑श्च । \newline
45. च॒ संमि॑तासः॒ संमि॑तासश्च च॒ संमि॑तासः । \newline
46. संमि॑तासश्च च॒ संमि॑तासः॒ संमि॑तासश्च । \newline
47. संमि॑तास॒ इति॒ सं - मि॒ता॒सः॒ । \newline
48. च॒ नो॒ न॒श्च॒ च॒ नः॒ । \newline
49. न॒ ऊ॒तय॑ ऊ॒तये॑ नो न ऊ॒तये᳚ । \newline
50. ऊ॒तये॒ सभ॑रसः॒ सभ॑रस ऊ॒तय॑ ऊ॒तये॒ सभ॑रसः । \newline
51. सभ॑रसो मरुतो मरुतः॒ सभ॑रसः॒ सभ॑रसो मरुतः । \newline
52. सभ॑रस॒ इति॒ स - भ॒र॒सः॒ । \newline
53. म॒रु॒तो॒ य॒ज्ञे य॒ज्ञे म॑रुतो मरुतो य॒ज्ञे । \newline
54. य॒ज्ञे अ॒स्मिन् न॒स्मिन्. य॒ज्ञे य॒ज्ञे अ॒स्मिन्न् । \newline
55. अ॒स्मिन् निन्द्र॒ मिन्द्र॑ म॒स्मिन् न॒स्मिन् निन्द्र᳚म् । \newline
56. इन्द्र॒म् दैवी॒र् दैवी॒ रिन्द्र॒ मिन्द्र॒म् दैवीः᳚ । \newline
57. दैवी॒र् विशो॒ विशो॒ दैवी॒र् दैवी॒र् विशः॑ । \newline
58. विशो॑ म॒रुतो॑ म॒रुतो॒ विशो॒ विशो॑ म॒रुतः॑ । \newline
59. म॒रुतो ऽनु॑वर्त्मा॒नो ऽनु॑वर्त्मानो म॒रुतो॑ म॒रुतो ऽनु॑वर्त्मानः । \newline
60. अनु॑वर्त्मानो॒ यथा॒ यथा ऽनु॑वर्त्मा॒नो ऽनु॑वर्त्मानो॒ यथा᳚ । \newline
61. अनु॑वर्त्मान॒ इत्यनु॑ - व॒र्त्मा॒नः॒ । \newline
62. यथेन्द्र॒ मिन्द्रं॒ ॅयथा॒ यथेन्द्र᳚म् । \newline
63. इन्द्र॒म् दैवी॒र् दैवी॒ रिन्द्र॒ मिन्द्र॒म् दैवीः᳚ । \newline
64. दैवी॒र् विशो॒ विशो॒ दैवी॒र् दैवी॒र् विशः॑ । \newline
65. विशो॑ म॒रुतो॑ म॒रुतो॒ विशो॒ विशो॑ म॒रुतः॑ । \newline
66. म॒रुतो ऽनु॑वर्त्मा॒नो ऽनु॑वर्त्मानो म॒रुतो॑ म॒रुतो ऽनु॑वर्त्मानः । \newline
67. अनु॑वर्त्मान ए॒व मे॒व मनु॑वर्त्मा॒नो ऽनु॑वर्त्मान ए॒वम् । \newline
68. अनु॑वर्त्मान॒ इत्यनु॑ - व॒र्त्मा॒नः॒ । \newline
69. ए॒व मि॒म मि॒म मे॒व मे॒व मि॒मम् । \newline
70. इ॒मं ॅयज॑मानं॒ ॅयज॑मान मि॒म मि॒मं ॅयज॑मानम् । \newline
71. यज॑मान॒म् दैवी॒र् दैवी॒र् यज॑मानं॒ ॅयज॑मान॒म् दैवीः᳚ । \newline
72. दैवी᳚श्च च॒ दैवी॒र् दैवी᳚श्च । \newline
73. च॒ विशो॒ विश॑श्च च॒ विशः॑ । \newline
74. विशो॒ मानु॑षी॒र् मानु॑षी॒र् विशो॒ विशो॒ मानु॑षीः । \newline
75. मानु॑षीश्च च॒ मानु॑षी॒र् मानु॑षीश्च । \newline
76. चानु॑वर्त्मा॒नो ऽनु॑वर्त्मानश्च॒ चानु॑वर्त्मानः । \newline
77. अनु॑वर्त्मानो भवन्तु भव॒ न्त्वनु॑वर्त्मा॒नो ऽनु॑वर्त्मानो भवन्तु । \newline
78. अनु॑वर्त्मान॒ इत्यनु॑ - व॒र्त्मा॒नः॒ । \newline
79. भ॒व॒न्त्विति॑ भवन्तु । \newline

\textbf{Ghana Paata } \newline

1. ऋ॒त॒जिच् च॑ च र्‌त॒जि दृ॑त॒जिच् च॑ सत्य॒जिथ् स॑त्य॒जिच् च॑ र्‌त॒जि दृ॑त॒जिच् च॑ सत्य॒जित् । \newline
2. ऋ॒त॒जिदित्यृ॑त - जित् । \newline
3. च॒ स॒त्य॒जिथ् स॑त्य॒जिच् च॑ च सत्य॒जिच् च॑ च सत्य॒जिच् च॑ च सत्य॒जिच् च॑ । \newline
4. स॒त्य॒जिच् च॑ च सत्य॒जिथ् स॑त्य॒जिच् च॑ सेन॒जिथ् से॑न॒जिच् च॑ सत्य॒जिथ् स॑त्य॒जिच् च॑ सेन॒जित् । \newline
5. स॒त्य॒जिदिति॑ सत्य - जित् । \newline
6. च॒ से॒न॒जिथ् से॑न॒जिच् च॑ च सेन॒जिच् च॑ च सेन॒जिच् च॑ च सेन॒जिच् च॑ । \newline
7. से॒न॒जिच् च॑ च सेन॒जिथ् से॑न॒जिच् च॑ सु॒षेणः॑ सु॒षेण॑श्च सेन॒जिथ् से॑न॒जिच् च॑ सु॒षेणः॑ । \newline
8. से॒न॒जिदिति॑ सेन - जित् । \newline
9. च॒ सु॒षेणः॑ सु॒षेण॑श्च च सु॒षेण॑श्च च सु॒षेण॑श्च च सु॒षेण॑श्च । \newline
10. सु॒षेण॑श्च च सु॒षेणः॑ सु॒षेण॒ श्चान्त्य॑मित्रो॒ अन्त्य॑मित्रश्च सु॒षेणः॑ सु॒षेण॒ श्चान्त्य॑मित्रः । \newline
11. सु॒षेण॒ इति॑ सु - सेनः॑ । \newline
12. चान्त्य॑मित्रो॒ अन्त्य॑मित्रश्च॒ चान्त्य॑मित्रश्च॒ चान्त्य॑मित्रश्च॒ चान्त्य॑मित्रश्च । \newline
13. अन्त्य॑मित्रश्च॒ चान्त्य॑मित्रो॒ अन्त्य॑मित्रश्च दू॒रेअ॑मित्रो दू॒रेअ॑मित्र॒ श्चान्त्य॑मित्रो॒ अन्त्य॑मित्रश्च 
दू॒रेअ॑मित्रः । \newline
14. अन्त्य॑मित्र॒ इत्यन्ति॑ - अ॒मि॒त्रः॒ । \newline
15. च॒ दू॒रेअ॑मित्रो दू॒रेअ॑मित्रश्च च दू॒रेअ॑मित्रश्च च दू॒रेअ॑मित्रश्च च दू॒रेअ॑मित्रश्च । \newline
16. दू॒रेअ॑मित्रश्च च दू॒रेअ॑मित्रो दू॒रेअ॑मित्रश्च ग॒णो ग॒णश्च॑ दू॒रेअ॑मित्रो दू॒रेअ॑मित्रश्च ग॒णः । \newline
17. दू॒रेअ॑मित्र॒ इति॑ दू॒रे - अ॒मि॒त्रः॒ । \newline
18. च॒ ग॒णो ग॒णश्च॑ च ग॒णः । \newline
19. ग॒ण इति॑ ग॒णः । \newline
20. ऋ॒तश्च॑ च॒ र्‌त ऋ॒तश्च॑ स॒त्यः स॒त्यश्च॒ र्‌त ऋ॒तश्च॑ स॒त्यः । \newline
21. च॒ स॒त्यः स॒त्यश्च॑ च स॒त्यश्च॑ च स॒त्यश्च॑ च स॒त्यश्च॑ । \newline
22. स॒त्यश्च॑ च स॒त्यः स॒त्यश्च॑ ध्रु॒वो ध्रु॒वश्च॑ स॒त्यः स॒त्यश्च॑ ध्रु॒वः । \newline
23. च॒ ध्रु॒वो ध्रु॒वश्च॑ च ध्रु॒वश्च॑ च ध्रु॒वश्च॑ च ध्रु॒वश्च॑ । \newline
24. ध्रु॒वश्च॑ च ध्रु॒वो ध्रु॒वश्च॑ ध॒रुणो॑ ध॒रुण॑श्च ध्रु॒वो ध्रु॒वश्च॑ ध॒रुणः॑ । \newline
25. च॒ ध॒रुणो॑ ध॒रुण॑श्च च ध॒रुण॑श्च च ध॒रुण॑श्च च ध॒रुण॑श्च । \newline
26. ध॒रुण॑श्च च ध॒रुणो॑ ध॒रुण॑श्च ध॒र्ता ध॒र्ता च॑ ध॒रुणो॑ ध॒रुण॑श्च ध॒र्ता । \newline
27. च॒ ध॒र्ता ध॒र्ता च॑ च ध॒र्ता च॑ च ध॒र्ता च॑ च ध॒र्ता च॑ । \newline
28. ध॒र्ता च॑ च ध॒र्ता ध॒र्ता च॑ विध॒र्ता वि॑ध॒र्ता च॑ ध॒र्ता ध॒र्ता च॑ विध॒र्ता । \newline
29. च॒ वि॒ध॒र्ता वि॑ध॒र्ता च॑ च विध॒र्ता च॑ च विध॒र्ता च॑ च विध॒र्ता च॑ । \newline
30. वि॒ध॒र्ता च॑ च विध॒र्ता वि॑ध॒र्ता च॑ विधार॒यो वि॑धार॒यश्च॑ विध॒र्ता वि॑ध॒र्ता च॑ विधार॒यः । \newline
31. वि॒ध॒र्तेति॑ वि - ध॒र्ता । \newline
32. च॒ वि॒धा॒र॒यो वि॑धार॒यश्च॑ च विधार॒यः । \newline
33. वि॒धा॒र॒य इति॑ वि - धा॒र॒यः । \newline
34. ई॒दृक्षा॑स एता॒दृक्षा॑स एता॒दृक्षा॑स ई॒दृक्षा॑स ई॒दृक्षा॑स एता॒दृक्षा॑स उ वु वेता॒दृक्षा॑स ई॒दृक्षा॑स ई॒दृक्षा॑स एता॒दृक्षा॑स उ । \newline
35. ए॒ता॒दृक्षा॑स उ वु वेता॒दृक्षा॑स एता॒दृक्षा॑स ऊ॒ षु णो॑ नः॒ स्वे॑ता॒दृक्षा॑स एता॒दृक्षा॑स 
ऊ॒ षु णः॑ । \newline
36. ऊ॒ षु णो॑ नः॒ सू॑ षु णः॑ स॒दृक्षा॑सः स॒दृक्षा॑सो नः॒ सू॑ षु णः॑ स॒दृक्षा॑सः । \newline
37. सु नो॑ नः॒ सु सु नः॑ स॒दृक्षा॑सः स॒दृक्षा॑सो नः॒ सु सु नः॑ स॒दृक्षा॑सः । \newline
38. नः॒ स॒दृक्षा॑सः स॒दृक्षा॑सो नो नः स॒दृक्षा॑सः॒ प्रति॑सदृक्षासः॒ प्रति॑सदृक्षासः स॒दृक्षा॑सो नो नः स॒दृक्षा॑सः॒ प्रति॑सदृक्षासः । \newline
39. स॒दृक्षा॑सः॒ प्रति॑सदृक्षासः॒ प्रति॑सदृक्षासः स॒दृक्षा॑सः स॒दृक्षा॑सः॒ प्रति॑सदृक्षास॒ आ प्रति॑सदृक्षासः स॒दृक्षा॑सः स॒दृक्षा॑सः॒ प्रति॑सदृक्षास॒ आ । \newline
40. प्रति॑सदृक्षास॒ आ प्रति॑सदृक्षासः॒ प्रति॑सदृक्षास॒ एत॑नेत॒ना प्रति॑सदृक्षासः॒ प्रति॑सदृक्षास॒ एत॑न । \newline
41. प्रति॑सदृक्षास॒ इति॒ प्रति॑ - स॒दृ॒क्षा॒सः॒ । \newline
42. एत॑ने त॒नेत॑न । \newline
43. इ॒त॒नेती॑तन । \newline
44. मि॒तास॑श्च च मि॒तासो॑ मि॒तास॑श्च॒ संमि॑तासः॒ संमि॑तासश्च मि॒तासो॑ मि॒तास॑श्च॒ संमि॑तासः । \newline
45. च॒ संमि॑तासः॒ संमि॑तासश्च च॒ संमि॑तासश्च च॒ संमि॑तासश्च च॒ संमि॑तासश्च । \newline
46. संमि॑तासश्च च॒ संमि॑तासः॒ संमि॑तासश्च नो नश्च॒ संमि॑तासः॒ संमि॑तासश्च नः । \newline
47. संमि॑तास॒ इति॒ सं - मि॒ता॒सः॒ । \newline
48. च॒ नो॒ न॒श्च॒ च॒ न॒ ऊ॒तय॑ ऊ॒तये॑ नश्च च न ऊ॒तये᳚ । \newline
49. न॒ ऊ॒तय॑ ऊ॒तये॑ नो न ऊ॒तये॒ सभ॑रसः॒ सभ॑रस ऊ॒तये॑ नो न ऊ॒तये॒ सभ॑रसः । \newline
50. ऊ॒तये॒ सभ॑रसः॒ सभ॑रस ऊ॒तय॑ ऊ॒तये॒ सभ॑रसो मरुतो मरुतः॒ सभ॑रस ऊ॒तय॑ ऊ॒तये॒ सभ॑रसो मरुतः । \newline
51. सभ॑रसो मरुतो मरुतः॒ सभ॑रसः॒ सभ॑रसो मरुतो य॒ज्ञे य॒ज्ञे म॑रुतः॒ सभ॑रसः॒ सभ॑रसो मरुतो य॒ज्ञे । \newline
52. सभ॑रस॒ इति॒ स - भ॒र॒सः॒ । \newline
53. म॒रु॒तो॒ य॒ज्ञे य॒ज्ञे म॑रुतो मरुतो य॒ज्ञे अ॒स्मिन् न॒स्मिन्. य॒ज्ञे म॑रुतो मरुतो य॒ज्ञे अ॒स्मिन्न् । \newline
54. य॒ज्ञे अ॒स्मिन् न॒स्मिन्. य॒ज्ञे य॒ज्ञे अ॒स्मिन् निन्द्र॒ मिन्द्र॑ म॒स्मिन्. य॒ज्ञे य॒ज्ञे अ॒स्मिन् निन्द्र᳚म् । \newline
55. अ॒स्मिन् निन्द्र॒ मिन्द्र॑ म॒स्मिन् न॒स्मिन् निन्द्र॒म् दैवी॒र् दैवी॒ रिन्द्र॑ म॒स्मिन् न॒स्मिन् निन्द्र॒म् दैवीः᳚ । \newline
56. इन्द्र॒म् दैवी॒र् दैवी॒ रिन्द्र॒ मिन्द्र॒म् दैवी॒र् विशो॒ विशो॒ दैवी॒ रिन्द्र॒ मिन्द्र॒म् दैवी॒र् विशः॑ । \newline
57. दैवी॒र् विशो॒ विशो॒ दैवी॒र् दैवी॒र् विशो॑ म॒रुतो॑ म॒रुतो॒ विशो॒ दैवी॒र् दैवी॒र् विशो॑ म॒रुतः॑ । \newline
58. विशो॑ म॒रुतो॑ म॒रुतो॒ विशो॒ विशो॑ म॒रुतो ऽनु॑वर्त्मा॒नो ऽनु॑वर्त्मानो म॒रुतो॒ विशो॒ विशो॑ म॒रुतो ऽनु॑वर्त्मानः । \newline
59. म॒रुतो ऽनु॑वर्त्मा॒नो ऽनु॑वर्त्मानो म॒रुतो॑ म॒रुतो ऽनु॑वर्त्मानो॒ यथा॒ यथा ऽनु॑वर्त्मानो म॒रुतो॑ म॒रुतो ऽनु॑वर्त्मानो॒ यथा᳚ । \newline
60. अनु॑वर्त्मानो॒ यथा॒ यथा ऽनु॑वर्त्मा॒नो ऽनु॑वर्त्मानो॒ यथेन्द्र॒ मिन्द्रं॒ ॅयथा ऽनु॑वर्त्मा॒नो ऽनु॑वर्त्मानो॒ यथेन्द्र᳚म् । \newline
61. अनु॑वर्त्मान॒ इत्यनु॑ - व॒र्त्मा॒नः॒ । \newline
62. यथेन्द्र॒ मिन्द्रं॒ ॅयथा॒ यथेन्द्र॒म् दैवी॒र् दैवी॒ रिन्द्रं॒ ॅयथा॒ यथेन्द्र॒म् दैवीः᳚ । \newline
63. इन्द्र॒म् दैवी॒र् दैवी॒ रिन्द्र॒ मिन्द्र॒म् दैवी॒र् विशो॒ विशो॒ दैवी॒ रिन्द्र॒ मिन्द्र॒म् दैवी॒र् विशः॑ । \newline
64. दैवी॒र् विशो॒ विशो॒ दैवी॒र् दैवी॒र् विशो॑ म॒रुतो॑ म॒रुतो॒ विशो॒ दैवी॒र् दैवी॒र् विशो॑ म॒रुतः॑ । \newline
65. विशो॑ म॒रुतो॑ म॒रुतो॒ विशो॒ विशो॑ म॒रुतो ऽनु॑वर्त्मा॒नो ऽनु॑वर्त्मानो म॒रुतो॒ विशो॒ विशो॑ म॒रुतो ऽनु॑वर्त्मानः । \newline
66. म॒रुतो ऽनु॑वर्त्मा॒नो ऽनु॑वर्त्मानो म॒रुतो॑ म॒रुतो ऽनु॑वर्त्मान ए॒व मे॒व मनु॑वर्त्मानो म॒रुतो॑ म॒रुतो ऽनु॑वर्त्मान ए॒वम् । \newline
67. अनु॑वर्त्मान ए॒व मे॒व मनु॑वर्त्मा॒नो ऽनु॑वर्त्मान ए॒व मि॒म मि॒म मे॒व मनु॑वर्त्मा॒नो ऽनु॑वर्त्मान ए॒व मि॒मम् । \newline
68. अनु॑वर्त्मान॒ इत्यनु॑ - व॒र्त्मा॒नः॒ । \newline
69. ए॒व मि॒म मि॒म मे॒व मे॒व मि॒मं ॅयज॑मानं॒ ॅयज॑मान मि॒म मे॒व मे॒व मि॒मं ॅयज॑मानम् । \newline
70. इ॒मं ॅयज॑मानं॒ ॅयज॑मान मि॒म मि॒मं ॅयज॑मान॒म् दैवी॒र् दैवी॒र् यज॑मान मि॒म मि॒मं ॅयज॑मान॒म् दैवीः᳚ । \newline
71. यज॑मान॒म् दैवी॒र् दैवी॒र् यज॑मानं॒ ॅयज॑मान॒म् दैवी᳚श्च च॒ दैवी॒र् यज॑मानं॒ ॅयज॑मान॒म् दैवी᳚श्च । \newline
72. दैवी᳚श्च च॒ दैवी॒र् दैवी᳚श्च॒ विशो॒ विश॑श्च॒ दैवी॒र् दैवी᳚श्च॒ विशः॑ । \newline
73. च॒ विशो॒ विश॑श्च च॒ विशो॒ मानु॑षी॒र् मानु॑षी॒र् विश॑श्च च॒ विशो॒ मानु॑षीः । \newline
74. विशो॒ मानु॑षी॒र् मानु॑षी॒र् विशो॒ विशो॒ मानु॑षीश्च च॒ मानु॑षी॒र् विशो॒ विशो॒ मानु॑षीश्च । \newline
75. मानु॑षीश्च च॒ मानु॑षी॒र् मानु॑षी॒ श्चानु॑वर्त्मा॒नो ऽनु॑वर्त्मानश्च॒ मानु॑षी॒र् मानु॑षी॒ श्चानु॑वर्त्मानः । \newline
76. चानु॑वर्त्मा॒नो ऽनु॑वर्त्मानश्च॒ चानु॑वर्त्मानो भवन्तु भव॒न् त्वनु॑ वर्त्मानश्च॒ चानु॑वर्त्मानो भवन्तु । \newline
77. अनु॑वर्त्मानो भवन्तु भव॒न् त्वनु॑ वर्त्मा॒नो ऽनु॑वर्त्मानो भवन्तु । \newline
78. अनु॑वर्त्मान॒ इत्यनु॑ - व॒र्त्मा॒नः॒ । \newline
79. भ॒व॒न्त्विति॑ भवन्तु । \newline
\pagebreak
\markright{ TS 4.6.6.1  \hfill https://www.vedavms.in \hfill}

\section{ TS 4.6.6.1 }

\textbf{TS 4.6.6.1 } \newline
\textbf{Samhita Paata} \newline

जी॒मूत॑स्येव भवति॒ प्रती॑कं॒ ॅयद्व॒र्मी याति॑ स॒मदा॑मु॒पस्थे᳚ । अना॑विद्धया त॒नुवा॑ जय॒ त्वꣳ स त्वा॒ वर्म॑णो महि॒मा पि॑पर्तु ॥ धन्व॑ना॒ गा धन्व॑ना॒ऽऽजिं ज॑येम॒ धन्व॑ना ती॒व्राः स॒मदो॑ जयेम । धनुः॒ शत्रो॑रपका॒मं कृ॑णोति॒ धन्व॑ना॒ सर्वाः᳚ प्र॒दिशो॑ जयेम ॥ व॒क्ष्यन्ती॒वेदा ग॑नीगन्ति॒ कर्णं॑ प्रि॒यꣳ सखा॑यं परिषस्वजा॒ना । योषे॑व शिङ्क्ते॒ वित॒ताऽधि॒ धन्व॒न् - [  ] \newline

\textbf{Pada Paata} \newline

जी॒मूत॑स्य । इ॒व॒ । भ॒व॒ति॒ । प्रती॑कम् । यत् । व॒र्मी । याति॑ । स॒मदा॒मिति॑ स - मदा᳚म् । उ॒पस्थ॒ इत्यु॒प - स्थे॒ ॥ अना॑विद्ध॒येत्यना᳚ - वि॒द्ध॒या॒ । त॒नुवा᳚ । ज॒य॒ । त्वम् । सः । त्वा॒ । वर्म॑णः । म॒हि॒मा । पि॒प॒र्तु॒ ॥ धन्व॑ना । गाः । धन्व॑ना । आ॒जिम् । ज॒ये॒म॒ । धन्व॑ना । ती॒व्राः । स॒मद॒ इति॑ स - मदः॑ । ज॒ये॒म॒ ॥ धनुः॑ । शत्रोः᳚ । अ॒प॒का॒ममित्य॑प - का॒मम् । कृ॒णो॒ति॒ । धन्व॑ना । सर्वाः᳚ । प्र॒दिश॒ इति॑ प्र - दिशः॑ । ज॒ये॒म॒ ॥ व॒क्ष्यन्ती᳚ । इ॒व॒ । इत् । एति॑ । ग॒नी॒ग॒न्ति॒ । कर्ण᳚म् । प्रि॒यम् । सखा॑यम् । प॒रि॒ष॒स्व॒जा॒नेति॑ परि - स॒स्व॒जा॒ना ॥ योषा᳚ । इ॒व । शि॒ङ्क्ते॒ । वित॒तेति॒ वि - त॒ता॒ । अधीति॑ । धन्वन्न्॑ ।  \newline


\textbf{Krama Paata} \newline

जी॒मूत॑स्येव । इ॒व॒ भ॒व॒ति॒ । भ॒व॒ति॒ प्रती॑कम् । प्रती॑कं॒ ॅयत् । यद् व॒र्मी । व॒र्मी याति॑ । याति॑ स॒मदा᳚म् । स॒मदा॑मु॒पस्थे᳚ । स॒मदा॒मिति॑ स - मदा᳚म् । उ॒पस्थ॒ इत्यु॒प - स्थे॒ ॥ अना॑विद्धया त॒नुवा᳚ । अना॑विद्ध॒येत्यना᳚ - वि॒द्ध॒या॒ । त॒नुवा॑ जय । ज॒य॒ त्वम् । त्वꣳ सः । स त्वा᳚ । त्वा॒ वर्म॑णः । वर्म॑णो महि॒मा । म॒हि॒मा पि॑पर्तु । पि॒प॒र्त्विति॑ पिपर्तु ॥ धन्व॑ना॒ गाः । गा धन्व॑ना । धन्व॑ना॒ऽऽजिम् । आ॒जिम् ज॑येम । ज॒ये॒म॒ धन्व॑ना । धन्व॑ना ती॒व्राः । ती॒व्राः स॒मदः॑ । स॒मदो॑ जयेम । स॒मद॒ इति॑ स - मदः॑ । ज॒ये॒मेति॑ जयेम ॥ धनुः॒ शत्रोः᳚ । शत्रो॑रपका॒मम् । अ॒प॒का॒मम् कृ॑णोति । अ॒प॒का॒ममित्य॑प - का॒मम् । कृ॒णो॒ति॒ धन्व॑ना । धन्व॑ना॒ सर्वाः᳚ । सर्वाः᳚ प्र॒दिशः॑ । प्र॒दिशो॑ जयेम । प्र॒दिश॒ इति॑ प्र - दिशः॑ । ज॒ये॒मेति॑ जयेम ॥ व॒क्ष्यन्ती॑व । इ॒वेत् । इदा । आ ग॑नीगन्ति । ग॒नी॒ग॒न्ति॒ कर्ण᳚म् । कर्ण॑म् प्रि॒यम् । प्रि॒यꣳ सखा॑यम् । सखा॑यम् परिषस्वजा॒ना । प॒रि॒ष॒स्व॒जा॒नेति॑ परि - स॒स्व॒जा॒ना ॥ योषे॑व । इ॒व॒ शि॒ङ्क्ते॒ । शि॒ङ्क्ते॒ वित॑ता । वित॒ताऽधि॑ । वित॒तेति॒ वि - त॒ता॒ । अधि॒ धन्वन्न्॑ । धन्व॒न् ज्या \newline

\textbf{Jatai Paata} \newline

1. जी॒मूत॑स्ये वे व जी॒मूत॑स्य जी॒मूत॑स्ये व । \newline
2. इ॒व॒ भ॒व॒ति॒ भ॒व॒ती॒वे॒व॒ भ॒व॒ति॒ । \newline
3. भ॒व॒ति॒ प्रती॑क॒म् प्रती॑कम् भवति भवति॒ प्रती॑कम् । \newline
4. प्रती॑कं॒ ॅयद् यत् प्रती॑क॒म् प्रती॑कं॒ ॅयत् । \newline
5. यद् व॒र्मी व॒र्मी यद् यद् व॒र्मी । \newline
6. व॒र्मी याति॒ याति॑ व॒र्मी व॒र्मी याति॑ । \newline
7. याति॑ स॒मदाꣳ॑ स॒मदां॒ ॅयाति॒ याति॑ स॒मदा᳚म् । \newline
8. स॒मदा॑ मु॒पस्थ॑ उ॒पस्थे॑ स॒मदाꣳ॑ स॒मदा॑ मु॒पस्थे᳚ । \newline
9. स॒मदा॒मिति॑ स - मदा᳚म् । \newline
10. उ॒पस्थ॒ इत्यु॒प - स्थे॒ । \newline
11. अना॑विद्धया त॒नुवा॑ त॒नुवा ऽना॑विद्ध॒या ऽना॑विद्धया त॒नुवा᳚ । \newline
12. अना॑विद्ध॒येत्यना᳚ - वि॒द्ध॒या॒ । \newline
13. त॒नुवा॑ जय जय त॒नुवा॑ त॒नुवा॑ जय । \newline
14. ज॒य॒ त्वम् त्वम् ज॑य जय॒ त्वम् । \newline
15. त्वꣳ स स त्वम् त्वꣳ सः । \newline
16. स त्वा᳚ त्वा॒ स स त्वा᳚ । \newline
17. त्वा॒ वर्म॑णो॒ वर्म॑ण स्त्वा त्वा॒ वर्म॑णः । \newline
18. वर्म॑णो महि॒मा म॑हि॒मा वर्म॑णो॒ वर्म॑णो महि॒मा । \newline
19. म॒हि॒मा पि॑पर्तु पिपर्तु महि॒मा म॑हि॒मा पि॑पर्तु । \newline
20. पि॒प॒र्त्विति॑ पिपर्तु । \newline
21. धन्व॑ना॒ गा गा धन्व॑ना॒ धन्व॑ना॒ गाः । \newline
22. गा धन्व॑ना॒ धन्व॑ना॒ गा गा धन्व॑ना । \newline
23. धन्व॑ना॒ ऽऽजि मा॒जिम् धन्व॑ना॒ धन्व॑ना॒ ऽऽजिम् । \newline
24. आ॒जिम् ज॑येम जयेमा॒जि मा॒जिम् ज॑येम । \newline
25. ज॒ये॒म॒ धन्व॑ना॒ धन्व॑ना जयेम जयेम॒ धन्व॑ना । \newline
26. धन्व॑ना ती॒व्रा स्ती॒व्रा धन्व॑ना॒ धन्व॑ना ती॒व्राः । \newline
27. ती॒व्राः स॒मदः॑ स॒मद॑ स्ती॒व्रा स्ती॒व्राः स॒मदः॑ । \newline
28. स॒मदो॑ जयेम जयेम स॒मदः॑ स॒मदो॑ जयेम । \newline
29. स॒मद॒ इति॑ स - मदः॑ । \newline
30. ज॒ये॒मेति॑ जयेम । \newline
31. धनुः॒ शत्रोः॒ शत्रो॒र् धनु॒र् धनुः॒ शत्रोः᳚ । \newline
32. शत्रो॑ रपका॒म म॑पका॒मꣳ शत्रोः॒ शत्रो॑ रपका॒मम् । \newline
33. अ॒प॒का॒मम् कृ॑णोति कृणो त्यपका॒म म॑पका॒मम् कृ॑णोति । \newline
34. अ॒प॒का॒ममित्य॑प - का॒मम् । \newline
35. कृ॒णो॒ति॒ धन्व॑ना॒ धन्व॑ना कृणोति कृणोति॒ धन्व॑ना । \newline
36. धन्व॑ना॒ सर्वाः॒ सर्वा॒ धन्व॑ना॒ धन्व॑ना॒ सर्वाः᳚ । \newline
37. सर्वाः᳚ प्र॒दिशः॑ प्र॒दिशः॒ सर्वाः॒ सर्वाः᳚ प्र॒दिशः॑ । \newline
38. प्र॒दिशो॑ जयेम जयेम प्र॒दिशः॑ प्र॒दिशो॑ जयेम । \newline
39. प्र॒दिश॒ इति॑ प्र - दिशः॑ । \newline
40. ज॒ये॒मेति॑ जयेम । \newline
41. व॒क्ष्यन्ती॑वेव व॒क्ष्यन्ती॑ व॒क्ष्यन्ती॑व । \newline
42. इ॒वे दिदि॑ वे॒वेत् । \newline
43. इदेदिदा । \newline
44. आ ग॑नीगन्ति गनीग॒न्त्या ग॑नीगन्ति । \newline
45. ग॒नी॒ग॒न्ति॒ कर्ण॒म् कर्ण॑म् गनीगन्ति गनीगन्ति॒ कर्ण᳚म् । \newline
46. कर्ण॑म् प्रि॒यम् प्रि॒यम् कर्ण॒म् कर्ण॑म् प्रि॒यम् । \newline
47. प्रि॒यꣳ सखा॑यꣳ॒॒ सखा॑यम् प्रि॒यम् प्रि॒यꣳ सखा॑यम् । \newline
48. सखा॑यम् परिषस्वजा॒ना प॑रिषस्वजा॒ना सखा॑यꣳ॒॒ सखा॑यम् परिषस्वजा॒ना । \newline
49. प॒रि॒ष॒स्व॒जा॒नेति॑ परि - स॒स्व॒जा॒ना । \newline
50. योषे॑वेव॒ योषा॒ योषे॑व । \newline
51. इ॒व॒ शि॒ङ्क्ते॒ शि॒ङ्क्त॒ इ॒वे॒व॒ शि॒ङ्क्ते॒ । \newline
52. शि॒ङ्क्ते॒ वित॑ता॒ वित॑ता शिङ्क्ते शिङ्क्ते॒ वित॑ता । \newline
53. वित॒ता ऽध्यधि॒ वित॑ता॒ वित॒ता ऽधि॑ । \newline
54. वित॒तेति॒ वि - त॒ता॒ । \newline
55. अधि॒ धन्व॒न् धन्व॒न् नध्यधि॒ धन्वन्न्॑ । \newline
56. धन्व॒न् ज्या ज्या धन्व॒न् धन्व॒न् ज्या । \newline

\textbf{Ghana Paata } \newline

1. जी॒मूत॑स्ये वेव जी॒मूत॑स्य जी॒मूत॑स्येव भवति भवतीव जी॒मूत॑स्य जी॒मूत॑स्येव भवति । \newline
2. इ॒व॒ भ॒व॒ति॒ भ॒व॒ती॒ वे॒व॒ भ॒व॒ति॒ प्रती॑क॒म् प्रती॑कम् भवती वेव भवति॒ प्रती॑कम् । \newline
3. भ॒व॒ति॒ प्रती॑क॒म् प्रती॑कम् भवति भवति॒ प्रती॑कं॒ ॅयद् यत् प्रती॑कम् भवति भवति॒ प्रती॑कं॒ ॅयत् । \newline
4. प्रती॑कं॒ ॅयद् यत् प्रती॑क॒म् प्रती॑कं॒ ॅयद् व॒र्मी व॒र्मी यत् प्रती॑क॒म् प्रती॑कं॒ ॅयद् व॒र्मी । \newline
5. यद् व॒र्मी व॒र्मी यद् यद् व॒र्मी याति॒ याति॑ व॒र्मी यद् यद् व॒र्मी याति॑ । \newline
6. व॒र्मी याति॒ याति॑ व॒र्मी व॒र्मी याति॑ स॒मदाꣳ॑ स॒मदां॒ ॅयाति॑ व॒र्मी व॒र्मी याति॑ स॒मदा᳚म् । \newline
7. याति॑ स॒मदाꣳ॑ स॒मदां॒ ॅयाति॒ याति॑ स॒मदा॑ मु॒पस्थ॑ उ॒पस्थे॑ स॒मदां॒ ॅयाति॒ याति॑ स॒मदा॑ मु॒पस्थे᳚ । \newline
8. स॒मदा॑ मु॒पस्थ॑ उ॒पस्थे॑ स॒मदाꣳ॑ स॒मदा॑ मु॒पस्थे᳚ । \newline
9. स॒मदा॒मिति॑ स - मदा᳚म् । \newline
10. उ॒पस्थ॒ इत्यु॒प - स्थे॒ । \newline
11. अना॑विद्धया त॒नुवा॑ त॒नुवा ऽना॑विद्ध॒या ऽना॑विद्धया त॒नुवा॑ जय जय त॒नुवा ऽना॑विद्ध॒या ऽना॑विद्धया त॒नुवा॑ जय । \newline
12. अना॑विद्ध॒येत्यना᳚ - वि॒द्ध॒या॒ । \newline
13. त॒नुवा॑ जय जय त॒नुवा॑ त॒नुवा॑ जय॒ त्वम् त्वम् ज॑य त॒नुवा॑ त॒नुवा॑ जय॒ त्वम् । \newline
14. ज॒य॒ त्वम् त्वम् ज॑य जय॒ त्वꣳ स स त्वम् ज॑य जय॒ त्वꣳ सः । \newline
15. त्वꣳ स स त्वम् त्वꣳ स त्वा᳚ त्वा॒ स त्वम् त्वꣳ स त्वा᳚ । \newline
16. स त्वा᳚ त्वा॒ स स त्वा॒ वर्म॑णो॒ वर्म॑ण स्त्वा॒ स स त्वा॒ वर्म॑णः । \newline
17. त्वा॒ वर्म॑णो॒ वर्म॑ण स्त्वा त्वा॒ वर्म॑णो महि॒मा म॑हि॒मा वर्म॑ण स्त्वा त्वा॒ वर्म॑णो महि॒मा । \newline
18. वर्म॑णो महि॒मा म॑हि॒मा वर्म॑णो॒ वर्म॑णो महि॒मा पि॑पर्तु पिपर्तु महि॒मा वर्म॑णो॒ वर्म॑णो महि॒मा पि॑पर्तु । \newline
19. म॒हि॒मा पि॑पर्तु पिपर्तु महि॒मा म॑हि॒मा पि॑पर्तु । \newline
20. पि॒प॒र्त्विति॑ पिपर्तु । \newline
21. धन्व॑ना॒ गा गा धन्व॑ना॒ धन्व॑ना॒ गा धन्व॑ना॒ धन्व॑ना॒ गा धन्व॑ना॒ धन्व॑ना॒ गा धन्व॑ना । \newline
22. गा धन्व॑ना॒ धन्व॑ना॒ गा गा धन्व॑ना॒ ऽऽजि मा॒जिम् धन्व॑ना॒ गा गा धन्व॑ना॒ ऽऽजिम् । \newline
23. धन्व॑ना॒ ऽऽजि मा॒जिम् धन्व॑ना॒ धन्व॑ना॒ ऽऽजिम् ज॑येम जयेमा॒जिम् धन्व॑ना॒ धन्व॑ना॒ ऽऽजिम् ज॑येम । \newline
24. आ॒जिम् ज॑येम जयेमा॒जि मा॒जिम् ज॑येम॒ धन्व॑ना॒ धन्व॑ना जयेमा॒जि मा॒जिम् ज॑येम॒ धन्व॑ना । \newline
25. ज॒ये॒म॒ धन्व॑ना॒ धन्व॑ना जयेम जयेम॒ धन्व॑ना ती॒व्रा स्ती॒व्रा धन्व॑ना जयेम जयेम॒ धन्व॑ना ती॒व्राः । \newline
26. धन्व॑ना ती॒व्रा स्ती॒व्रा धन्व॑ना॒ धन्व॑ना ती॒व्राः स॒मदः॑ स॒मद॑ स्ती॒व्रा धन्व॑ना॒ धन्व॑ना ती॒व्राः स॒मदः॑ । \newline
27. ती॒व्राः स॒मदः॑ स॒मद॑ स्ती॒व्रा स्ती॒व्राः स॒मदो॑ जयेम जयेम स॒मद॑ स्ती॒व्रा स्ती॒व्राः स॒मदो॑ जयेम । \newline
28. स॒मदो॑ जयेम जयेम स॒मदः॑ स॒मदो॑ जयेम । \newline
29. स॒मद॒ इति॑ स - मदः॑ । \newline
30. ज॒ये॒मेति॑ जयेम । \newline
31. धनुः॒ शत्रोः॒ शत्रो॒र् धनु॒र् धनुः॒ शत्रो॑ रपका॒म म॑पका॒मꣳ शत्रो॒र् धनु॒र् धनुः॒ शत्रो॑ रपका॒मम् । \newline
32. शत्रो॑ रपका॒म म॑पका॒मꣳ शत्रोः॒ शत्रो॑ रपका॒मम् कृ॑णोति कृणो त्यपका॒मꣳ शत्रोः॒ शत्रो॑ रपका॒मम् कृ॑णोति । \newline
33. अ॒प॒का॒मम् कृ॑णोति कृणो त्यपका॒म म॑पका॒मम् कृ॑णोति॒ धन्व॑ना॒ धन्व॑ना कृणो त्यपका॒म म॑पका॒मम् कृ॑णोति॒ धन्व॑ना । \newline
34. अ॒प॒का॒ममित्य॑प - का॒मम् । \newline
35. कृ॒णो॒ति॒ धन्व॑ना॒ धन्व॑ना कृणोति कृणोति॒ धन्व॑ना॒ सर्वाः॒ सर्वा॒ धन्व॑ना कृणोति कृणोति॒ धन्व॑ना॒ सर्वाः᳚ । \newline
36. धन्व॑ना॒ सर्वाः॒ सर्वा॒ धन्व॑ना॒ धन्व॑ना॒ सर्वाः᳚ प्र॒दिशः॑ प्र॒दिशः॒ सर्वा॒ धन्व॑ना॒ धन्व॑ना॒ सर्वाः᳚ प्र॒दिशः॑ । \newline
37. सर्वाः᳚ प्र॒दिशः॑ प्र॒दिशः॒ सर्वाः॒ सर्वाः᳚ प्र॒दिशो॑ जयेम जयेम प्र॒दिशः॒ सर्वाः॒ सर्वाः᳚ प्र॒दिशो॑ जयेम । \newline
38. प्र॒दिशो॑ जयेम जयेम प्र॒दिशः॑ प्र॒दिशो॑ जयेम । \newline
39. प्र॒दिश॒ इति॑ प्र - दिशः॑ । \newline
40. ज॒ये॒मेति॑ जयेम । \newline
41. व॒क्ष्यन्ती॑वेव व॒क्ष्यन्ती॑ व॒क्ष्यन्ती॒वेदि दि॑व व॒क्ष्यन्ती॑ व॒क्ष्यन्ती॒वेत् । \newline
42. इ॒वे दिदि॑वे॒ वेदे दि॑वे॒ वेदा । \newline
43. इदे दिदा ग॑नीगन्ति गनीग॒न् त्येदिदा ग॑नीगन्ति । \newline
44. आ ग॑नीगन्ति गनीग॒न्त्या ग॑नीगन्ति॒ कर्ण॒म् कर्ण॑म् गनीग॒न् त्याग॑नीगन्ति॒ कर्ण᳚म् । \newline
45. ग॒नी॒ग॒न्ति॒ कर्ण॒म् कर्ण॑म् गनीगन्ति गनीगन्ति॒ कर्ण॑म् प्रि॒यम् प्रि॒यम् कर्ण॑म् गनीगन्ति गनीगन्ति॒ कर्ण॑म् प्रि॒यम् । \newline
46. कर्ण॑म् प्रि॒यम् प्रि॒यम् कर्ण॒म् कर्ण॑म् प्रि॒यꣳ सखा॑यꣳ॒॒ सखा॑यम् प्रि॒यम् कर्ण॒म् कर्ण॑म् प्रि॒यꣳ सखा॑यम् । \newline
47. प्रि॒यꣳ सखा॑यꣳ॒॒ सखा॑यम् प्रि॒यम् प्रि॒यꣳ सखा॑यम् परिषस्वजा॒ना प॑रिषस्वजा॒ना सखा॑यम् प्रि॒यम् प्रि॒यꣳ सखा॑यम् परिषस्वजा॒ना । \newline
48. सखा॑यम् परिषस्वजा॒ना प॑रिषस्वजा॒ना सखा॑यꣳ॒॒ सखा॑यम् परिषस्वजा॒ना । \newline
49. प॒रि॒ष॒स्व॒जा॒नेति॑ परि - स॒स्व॒जा॒ना । \newline
50. योषे॑वेव॒ योषा॒ योषे॑व शिङ्क्ते शिङ्क्त इव॒ योषा॒ योषे॑व शिङ्क्ते । \newline
51. इ॒व॒ शि॒ङ्क्ते॒ शि॒ङ्क्त॒ इ॒वे॒व॒ शि॒ङ्क्ते॒ वित॑ता॒ वित॑ता शिङ्क्त इवेव शिङ्क्ते॒ वित॑ता । \newline
52. शि॒ङ्क्ते॒ वित॑ता॒ वित॑ता शिङ्क्ते शिङ्क्ते॒ वित॒ता ऽध्यधि॒ वित॑ता शिङ्क्ते शिङ्क्ते॒ वित॒ता ऽधि॑ । \newline
53. वित॒ता ऽध्यधि॒ वित॑ता॒ वित॒ता ऽधि॒ धन्व॒न् धन्व॒न् नधि॒ वित॑ता॒ वित॒ता ऽधि॒ धन्वन्न्॑ । \newline
54. वित॒तेति॒ वि - त॒ता॒ । \newline
55. अधि॒ धन्व॒न् धन्व॒न् नध्यधि॒ धन्व॒न् ज्या ज्या धन्व॒न् नध्यधि॒ धन्व॒न् ज्या । \newline
56. धन्व॒न् ज्या ज्या धन्व॒न् धन्व॒न् ज्या इ॒य मि॒यम् ज्या धन्व॒न् धन्व॒न् ज्या इ॒यम् । \newline
\pagebreak
\markright{ TS 4.6.6.2  \hfill https://www.vedavms.in \hfill}

\section{ TS 4.6.6.2 }

\textbf{TS 4.6.6.2 } \newline
\textbf{Samhita Paata} \newline

ज्या इ॒यꣳ सम॑ने पा॒रय॑न्ती ॥ ते आ॒चर॑न्ती॒ सम॑नेव॒ योषा॑ मा॒तेव॑ पु॒त्रं बि॑भृतामु॒पस्थे᳚ । अप॒ शत्रून्॑ विद्ध्यताꣳ संॅविदा॒ने आर्त्नी॑ इ॒मे वि॑ष्फु॒रन्ती॑ अ॒मित्रान्॑ ॥ ब॒ह्वी॒नां पि॒ता ब॒हुर॑स्य पु॒त्रश्चि॒श्चा कृ॑णोति॒ सम॑नाऽव॒गत्य॑ । इ॒षु॒धिः सङ्काः॒ पृत॑नाश्च॒ सर्वाः᳚ पृ॒ष्ठे निन॑द्धो जयति॒ प्रसू॑तः ॥ रथे॒ तिष्ठ॑न् नयति वा॒जिनः॑ पु॒रो यत्र॑यत्र का॒मय॑ते सुषार॒थिः । अ॒भीशू॑नां महि॒मानं॑ - [  ] \newline

\textbf{Pada Paata} \newline

ज्या । इ॒यम् । सम॑ने । पा॒रय॑न्ती ॥ ते इति॑ । आ॒चर॑न्ती॒ इत्या᳚ - चर॑न्ती । सम॑ना । इ॒व॒ । योषा᳚ । मा॒ता । इ॒व॒ । पु॒त्रम् । बि॒भृ॒ता॒म् । उ॒पस्थ॒ इत्यु॒प - स्थे॒ ॥ अपेति॑ । शत्रून्॑ । वि॒द्ध्य॒ता॒म् । सं॒ॅवि॒दा॒न इति॑ सं - वि॒दा॒ने । आर्त्नी॒ इति॑ । इ॒मे इति । वि॒ष्फु॒रन्ती॒ इति॑ वि-स्फु॒रन्ती᳚ । अ॒मित्रान्॑ ॥ ब॒ह्वी॒नाम् । पि॒ता । ब॒हुः । अ॒स्य॒ । पु॒त्रः । चि॒श्चा । कृ॒णो॒ति॒ । सम॑ना । अ॒व॒गत्येत्यव॑ - गत्य॑ ॥ इ॒षु॒धिरिती॑षु - धिः । सङ्काः᳚ । पृत॑नाः । च॒ । सर्वाः᳚ । पृ॒ष्ठे । निन॑द्ध॒ इति॒ नि - न॒द्धः॒ । ज॒य॒ति॒ । प्रसू॑त॒ इति॒ प्र - सू॒तः॒ ॥ रथे᳚ । तिष्ठन्न्॑ । न॒य॒ति॒ । वा॒जिनः॑ । पु॒रः । यत्र॑य॒त्रेति॒ यत्र॑-य॒त्र॒ । का॒मय॑ते । सु॒षा॒र॒थिरिति॑ सु - सा॒र॒थिः ॥ अ॒भीशू॑नाम् । म॒हि॒मान᳚म् ।  \newline


\textbf{Krama Paata} \newline

ज्या इ॒यम् । इ॒यꣳ सम॑ने । सम॑ने पा॒रय॑न्ती । पा॒रय॒न्तीति॑ पा॒रय॑न्ती ॥ ते आ॒चर॑न्ती । ते इति॒ ते । आ॒चर॑न्ती॒ सम॑ना । आ॒चर॑न्ती॒ इत्या᳚ - चर॑न्ती । सम॑नेव । इ॒व॒ योषा᳚ । योषा॑ मा॒ता । मा॒तेव॑ । इ॒व॒ पु॒त्रम् । पु॒त्रं बि॑भृताम् । बि॒भृ॒ता॒मु॒पस्थे᳚ । उ॒पस्थ॒ इत्यु॒प - स्थे॒ ॥ अप॒ शत्रून्॑ । शत्रून्॑. विद्ध्यताम् । वि॒द्ध्य॒ताꣳ॒॒ स॒म्ॅवि॒दा॒ने । स॒म्ॅवि॒दा॒ने आर्त्नी᳚ । स॒म्ॅवि॒दा॒ने इति॑ सम् - वि॒दा॒ने । आर्त्नी॑ इ॒मे । आर्त्नी॒ इत्यार्त्नी᳚ । इ॒मे वि॑ष्फु॒रन्ती᳚ । इ॒मे इती॒मे । वि॒ष्फु॒रन्ती॑ अ॒मित्रान्॑ । वि॒ष्फु॒रन्ती॒ इति॑ वि - स्फु॒रन्ती᳚ । अ॒मित्रा॒नित्य॒मित्रान्॑ ॥ ब॒ह्वी॒नाम् पि॒ता । पि॒ता ब॒हुः । ब॒हुर॑स्य । अ॒स्य॒ पु॒त्रः । पु॒त्रश्चि॒श्चा । चि॒श्चा कृ॑णोति । कृ॒णो॒ति॒ सम॑ना । सम॑नाऽव॒गत्य॑ । अ॒व॒गत्येत्य॑व - गत्य॑ ॥ इ॒षु॒धिः सङ्काः᳚ । इ॒षु॒धिरिती॑षु - धिः । सङ्काः॒ पृत॑नाः । पृत॑नाश्च । च॒ सर्वाः᳚ । सर्वाः᳚ पृ॒ष्ठे । पृ॒ष्ठे निन॑द्धः । निन॑द्धो जयति । निन॑द्ध॒ इति॒ नि - न॒द्धः॒ । ज॒य॒ति॒ प्रसू॑तः । प्रसू॑त॒ इति॒ प्र - सू॒तः॒ ॥ रथे॒ तिष्ठन्न्॑ । तिष्ठ॑न् नयति । न॒य॒ति॒ वा॒जिनः॑ । वा॒जिनः॑ पु॒रः । पु॒रो यत्र॑यत्र । यत्र॑यत्र का॒मय॑ते । यत्र॑य॒त्रेति॒ यत्र॑ - य॒त्र॒ । का॒मय॑ते सुषार॒थिः । सु॒षा॒र॒थिरिति॑ सु - सा॒र॒थिः ॥ अ॒भीशूना॑म् महि॒मान᳚म् । म॒हि॒मान॑म् पनायत \newline

\textbf{Jatai Paata} \newline

1. ज्या इ॒य मि॒यम् ज्या ज्या इ॒यम् । \newline
2. इ॒यꣳ सम॑ने॒ सम॑न इ॒य मि॒यꣳ सम॑ने । \newline
3. सम॑ने पा॒रय॑न्ती पा॒रय॑न्ती॒ सम॑ने॒ सम॑ने पा॒रय॑न्ती । \newline
4. पा॒रय॒न्तीति॑ पा॒रय॑न्ती । \newline
5. ते आ॒चर॑न्ती आ॒चर॑न्ती॒ ते ते आ॒चर॑न्ती । \newline
6. ते इति॒ ते । \newline
7. आ॒चर॑न्ती॒ सम॑ना॒ सम॑ना॒ ऽऽचर॑न्ती आ॒चर॑न्ती॒ सम॑ना । \newline
8. आ॒चर॑न्ती॒ इत्या᳚ - चर॑न्ती । \newline
9. सम॑नेवेव॒ सम॑ना॒ सम॑नेव । \newline
10. इ॒व॒ योषा॒ योषे॑वेव॒ योषा᳚ । \newline
11. योषा॑ मा॒ता मा॒ता योषा॒ योषा॑ मा॒ता । \newline
12. मा॒तेवे॑व मा॒ता मा॒तेव॑ । \newline
13. इ॒व॒ पु॒त्रम् पु॒त्र मि॑वेव पु॒त्रम् । \newline
14. पु॒त्रम् बि॑भृताम् बिभृताम् पु॒त्रम् पु॒त्रम् बि॑भृताम् । \newline
15. बि॒भृ॒ता॒ मु॒पस्थ॑ उ॒पस्थे॑ बिभृताम् बिभृता मु॒पस्थे᳚ । \newline
16. उ॒पस्थ॒ इत्यु॒प - स्थे॒ । \newline
17. अप॒ शत्रू॒ञ् छत्रू॒ नपाप॒ शत्रून्॑ । \newline
18. शत्रून्॑. विद्ध्यतां ॅविद्ध्यताꣳ॒॒ शत्रू॒ञ् छत्रून्॑. विद्ध्यताम् । \newline
19. वि॒द्ध्य॒ताꣳ॒॒ सं॒ॅवि॒दा॒ने सं॑ॅविदा॒ने वि॑द्ध्यतां ॅविद्ध्यताꣳ संॅविदा॒ने । \newline
20. सं॒ॅवि॒दा॒ने आर्त्नी॒ आर्त्नी॑ संॅविदा॒ने सं॑ॅविदा॒ने आर्त्नी᳚ । \newline
21. सं॒ॅवि॒दा॒ने इति॑ सं - वि॒दा॒ने । \newline
22. आर्त्नी॑ इ॒मे इ॒मे आर्त्नी॒ आर्त्नी॑ इ॒मे । \newline
23. आर्त्नी॒ इत्यार्त्नी᳚ । \newline
24. इ॒मे वि॑ष्फु॒रन्ती॑ विष्फु॒रन्ती॑ इ॒मे इ॒मे वि॑ष्फु॒रन्ती᳚ । \newline
25. इ॒मे इती॒मे । \newline
26. वि॒ष्फु॒रन्ती॑ अ॒मित्रा॑ न॒मित्रान्॑. विष्फु॒रन्ती॑ विष्फु॒रन्ती॑ अ॒मित्रान्॑ । \newline
27. वि॒ष्फु॒रन्ती॒ इति॑ वि - स्फु॒रन्ती᳚ । \newline
28. अ॒मित्रा॒नित्य॒मित्रान्॑ । \newline
29. ब॒ह्वी॒नाम् पि॒ता पि॒ता ब॑ह्वी॒नाम् ब॑ह्वी॒नाम् पि॒ता । \newline
30. पि॒ता ब॒हुर् ब॒हुः पि॒ता पि॒ता ब॒हुः । \newline
31. ब॒हु र॑स्यास्य ब॒हुर् ब॒हुर॑स्य । \newline
32. अ॒स्य॒ पु॒त्रः पु॒त्रो अ॑स्यास्य पु॒त्रः । \newline
33. पु॒त्र श्चि॒श्चा चि॒श्चा पु॒त्रः पु॒त्र श्चि॒श्चा । \newline
34. चि॒श्चा कृ॑णोति कृणोति चि॒श्चा चि॒श्चा कृ॑णोति । \newline
35. कृ॒णो॒ति॒ सम॑ना॒ सम॑ना कृणोति कृणोति॒ सम॑ना । \newline
36. सम॑ना ऽव॒गत्या॑ व॒गत्य॒ सम॑ना॒ सम॑ना ऽव॒गत्य॑ । \newline
37. अ॒व॒गत्येत्य॑व - गत्य॑ । \newline
38. इ॒षु॒धिः सङ्काः॒ सङ्का॑ इषु॒धि रि॑षु॒धिः सङ्काः᳚ । \newline
39. इ॒षु॒धिरिती॑षु - धिः । \newline
40. सङ्काः॒ पृत॑नाः॒ पृत॑नाः॒ सङ्काः॒ सङ्काः॒ पृत॑नाः । \newline
41. पृत॑नाश्च च॒ पृत॑नाः॒ पृत॑नाश्च । \newline
42. च॒ सर्वाः॒ सर्वा᳚श्च च॒ सर्वाः᳚ । \newline
43. सर्वाः᳚ पृ॒ष्ठे पृ॒ष्ठे सर्वाः॒ सर्वाः᳚ पृ॒ष्ठे । \newline
44. पृ॒ष्ठे निन॑द्धो॒ निन॑द्धः पृ॒ष्ठे पृ॒ष्ठे निन॑द्धः । \newline
45. निन॑द्धो जयति जयति॒ निन॑द्धो॒ निन॑द्धो जयति । \newline
46. निन॑द्ध॒ इति॒ नि - न॒द्धः॒ । \newline
47. ज॒य॒ति॒ प्रसू॑तः॒ प्रसू॑तो जयति जयति॒ प्रसू॑तः । \newline
48. प्रसू॑त॒ इति॒ प्र - सू॒तः॒ । \newline
49. रथे॒ तिष्ठꣳ॒॒ स्तिष्ठ॒न् रथे॒ रथे॒ तिष्ठन्न्॑ । \newline
50. तिष्ठ॑न् नयति नयति॒ तिष्ठꣳ॒॒ स्तिष्ठ॑न् नयति । \newline
51. न॒य॒ति॒ वा॒जिनो॑ वा॒जिनो॑ नयति नयति वा॒जिनः॑ । \newline
52. वा॒जिनः॑ पु॒रः पु॒रो वा॒जिनो॑ वा॒जिनः॑ पु॒रः । \newline
53. पु॒रो यत्र॑यत्र॒ यत्र॑यत्र पु॒रः पु॒रो यत्र॑यत्र । \newline
54. यत्र॑यत्र का॒मय॑ते का॒मय॑ते॒ यत्र॑यत्र॒ यत्र॑यत्र का॒मय॑ते । \newline
55. यत्र॑य॒त्रेति॒ यत्र॑ - य॒त्र॒ । \newline
56. का॒मय॑ते सुषार॒थिः सु॑षार॒थिः का॒मय॑ते का॒मय॑ते सुषार॒थिः । \newline
57. सु॒षा॒र॒थिरिति॑ सु - सा॒र॒थिः । \newline
58. अ॒भीशू॑नाम् महि॒मान॑म् महि॒मान॑ म॒भीशू॑ना म॒भीशू॑नाम् महि॒मान᳚म् । \newline
59. म॒हि॒मान॑म् पनायत पनायत महि॒मान॑म् महि॒मान॑म् पनायत । \newline

\textbf{Ghana Paata } \newline

1. ज्या इ॒य मि॒यम् ज्या ज्या इ॒यꣳ सम॑ने॒ सम॑न इ॒यम् ज्या ज्या इ॒यꣳ सम॑ने । \newline
2. इ॒यꣳ सम॑ने॒ सम॑न इ॒य मि॒यꣳ सम॑ने पा॒रय॑न्ती पा॒रय॑न्ती॒ सम॑न इ॒य मि॒यꣳ सम॑ने पा॒रय॑न्ती । \newline
3. सम॑ने पा॒रय॑न्ती पा॒रय॑न्ती॒ सम॑ने॒ सम॑ने पा॒रय॑न्ती । \newline
4. पा॒रय॒न्तीति॑ पा॒रय॑न्ती । \newline
5. ते आ॒चर॑न्ती आ॒चर॑न्ती॒ ते ते आ॒चर॑न्ती॒ सम॑ना॒ सम॑ना॒ ऽऽचर॑न्ती॒ ते ते आ॒चर॑न्ती॒ सम॑ना । \newline
6. ते इति॒ ते । \newline
7. आ॒चर॑न्ती॒ सम॑ना॒ सम॑ना॒ ऽऽचर॑न्ती आ॒चर॑न्ती॒ सम॑नेवेव॒ सम॑ना॒ ऽऽचर॑न्ती आ॒चर॑न्ती॒ सम॑नेव । \newline
8. आ॒चर॑न्ती॒ इत्या᳚ - चर॑न्ती । \newline
9. सम॑नेवेव॒ सम॑ना॒ सम॑नेव॒ योषा॒ योषे॑व॒ सम॑ना॒ सम॑नेव॒ योषा᳚ । \newline
10. इ॒व॒ योषा॒ योषे॑वेव॒ योषा॑ मा॒ता मा॒ता योषे॑वेव॒ योषा॑ मा॒ता । \newline
11. योषा॑ मा॒ता मा॒ता योषा॒ योषा॑ मा॒तेवे॑व मा॒ता योषा॒ योषा॑ मा॒तेव॑ । \newline
12. मा॒तेवे॑व मा॒ता मा॒तेव॑ पु॒त्रम् पु॒त्र मि॑व मा॒ता मा॒तेव॑ पु॒त्रम् । \newline
13. इ॒व॒ पु॒त्रम् पु॒त्र मि॑वेव पु॒त्रम् बि॑भृताम् बिभृताम् पु॒त्र मि॑वेव पु॒त्रम् बि॑भृताम् । \newline
14. पु॒त्रम् बि॑भृताम् बिभृताम् पु॒त्रम् पु॒त्रम् बि॑भृता मु॒पस्थ॑ उ॒पस्थे॑ बिभृताम् पु॒त्रम् पु॒त्रम् बि॑भृता मु॒पस्थे᳚ । \newline
15. बि॒भृ॒ता॒ मु॒पस्थ॑ उ॒पस्थे॑ बिभृताम् बिभृता मु॒पस्थे᳚ । \newline
16. उ॒पस्थ॒ इत्यु॒प - स्थे॒ । \newline
17. अप॒ शत्रू॒ञ् छत्रू॒ नपाप॒ शत्रून्॑. विद्ध्यतां ॅविद्ध्यताꣳ॒॒ शत्रू॒ नपाप॒ शत्रून्॑. विद्ध्यताम् । \newline
18. शत्रून्॑. विद्ध्यतां ॅविद्ध्यताꣳ॒॒ शत्रू॒ञ् छत्रून्॑. विद्ध्यताꣳ संॅविदा॒ने सं॑ॅविदा॒ने वि॑द्ध्यताꣳ॒॒ शत्रू॒ञ् छत्रून्॑. विद्ध्यताꣳ संॅविदा॒ने । \newline
19. वि॒द्ध्य॒ताꣳ॒॒ सं॒ॅवि॒दा॒ने सं॑ॅविदा॒ने वि॑द्ध्यतां ॅविद्ध्यताꣳ संॅविदा॒ने आर्त्नी॒ आर्त्नी॑ संॅविदा॒ने वि॑द्ध्यतां ॅविद्ध्यताꣳ संॅविदा॒ने आर्त्नी᳚ । \newline
20. सं॒ॅवि॒दा॒ने आर्त्नी॒ आर्त्नी॑ संॅविदा॒ने सं॑ॅविदा॒ने आर्त्नी॑ इ॒मे इ॒मे आर्त्नी॑ संॅविदा॒ने सं॑ॅविदा॒ने आर्त्नी॑ इ॒मे । \newline
21. सं॒ॅवि॒दा॒ने इति॑ सं - वि॒दा॒ने । \newline
22. आर्त्नी॑ इ॒मे इ॒मे आर्त्नी॒ आर्त्नी॑ इ॒मे वि॑ष्फु॒रन्ती॑ विष्फु॒रन्ती॑ इ॒मे आर्त्नी॒ आर्त्नी॑ इ॒मे वि॑ष्फु॒रन्ती᳚ । \newline
23. आर्त्नी॒ इत्यार्त्नी᳚ । \newline
24. इ॒मे वि॑ष्फु॒रन्ती॑ विष्फु॒रन्ती॑ इ॒मे इ॒मे वि॑ष्फु॒रन्ती॑ अ॒मित्रा॑ न॒मित्रान्॑. विष्फु॒रन्ती॑ इ॒मे इ॒मे वि॑ष्फु॒रन्ती॑ अ॒मित्रान्॑ । \newline
25. इ॒मे इती॒मे । \newline
26. वि॒ष्फु॒रन्ती॑ अ॒मित्रा॑ न॒मित्रान्॑. विष्फु॒रन्ती॑ विष्फु॒रन्ती॑ अ॒मित्रान्॑ । \newline
27. वि॒ष्फु॒रन्ती॒ इति॑ वि - स्फु॒रन्ती᳚ । \newline
28. अ॒मित्रा॒नित्य॒मित्रान्॑ । \newline
29. ब॒ह्वी॒नाम् पि॒ता पि॒ता ब॑ह्वी॒नाम् ब॑ह्वी॒नाम् पि॒ता ब॒हुर् ब॒हुः पि॒ता ब॑ह्वी॒नाम् ब॑ह्वी॒नाम् पि॒ता ब॒हुः । \newline
30. पि॒ता ब॒हुर् ब॒हुः पि॒ता पि॒ता ब॒हु र॑स्यास्य ब॒हुः पि॒ता पि॒ता ब॒हुर॑स्य । \newline
31. ब॒हुर॑ स्यास्य ब॒हुर् ब॒हुर॑स्य पु॒त्रः पु॒त्रो अ॑स्य ब॒हुर् ब॒हुर॑स्य पु॒त्रः । \newline
32. अ॒स्य॒ पु॒त्रः पु॒त्रो अ॑स्यास्य पु॒त्र श्चि॒श्चा चि॒श्चा पु॒त्रो अ॑स्यास्य पु॒त्र श्चि॒श्चा । \newline
33. पु॒त्र श्चि॒श्चा चि॒श्चा पु॒त्रः पु॒त्र श्चि॒श्चा कृ॑णोति कृणोति चि॒श्चा पु॒त्रः पु॒त्र श्चि॒श्चा कृ॑णोति । \newline
34. चि॒श्चा कृ॑णोति कृणोति चि॒श्चा चि॒श्चा कृ॑णोति॒ सम॑ना॒ सम॑ना कृणोति चि॒श्चा चि॒श्चा कृ॑णोति॒ सम॑ना । \newline
35. कृ॒णो॒ति॒ सम॑ना॒ सम॑ना कृणोति कृणोति॒ सम॑ना ऽव॒गत्या॑ व॒गत्य॒ सम॑ना कृणोति कृणोति॒ सम॑ना ऽव॒गत्य॑ । \newline
36. सम॑ना ऽव॒गत्या॑ व॒गत्य॒ सम॑ना॒ सम॑ना ऽव॒गत्य॑ । \newline
37. अ॒व॒गत्येत्य॑व - गत्य॑ । \newline
38. इ॒षु॒धिः सङ्काः॒ सङ्का॑ इषु॒धि रि॑षु॒धिः सङ्काः॒ पृत॑नाः॒ पृत॑नाः॒ सङ्का॑ इषु॒धि रि॑षु॒धिः सङ्काः॒ पृत॑नाः । \newline
39. इ॒षु॒धिरिती॑षु - धिः । \newline
40. सङ्काः॒ पृत॑नाः॒ पृत॑नाः॒ सङ्काः॒ सङ्काः॒ पृत॑नाश्च च॒ पृत॑नाः॒ सङ्काः॒ सङ्काः॒ पृत॑नाश्च । \newline
41. पृत॑नाश्च च॒ पृत॑नाः॒ पृत॑नाश्च॒ सर्वाः॒ सर्वा᳚श्च॒ पृत॑नाः॒ पृत॑नाश्च॒ सर्वाः᳚ । \newline
42. च॒ सर्वाः॒ सर्वा᳚श्च च॒ सर्वाः᳚ पृ॒ष्ठे पृ॒ष्ठे सर्वा᳚श्च च॒ सर्वाः᳚ पृ॒ष्ठे । \newline
43. सर्वाः᳚ पृ॒ष्ठे पृ॒ष्ठे सर्वाः॒ सर्वाः᳚ पृ॒ष्ठे निन॑द्धो॒ निन॑द्धः पृ॒ष्ठे सर्वाः॒ सर्वाः᳚ पृ॒ष्ठे निन॑द्धः । \newline
44. पृ॒ष्ठे निन॑द्धो॒ निन॑द्धः पृ॒ष्ठे पृ॒ष्ठे निन॑द्धो जयति जयति॒ निन॑द्धः पृ॒ष्ठे पृ॒ष्ठे निन॑द्धो जयति । \newline
45. निन॑द्धो जयति जयति॒ निन॑द्धो॒ निन॑द्धो जयति॒ प्रसू॑तः॒ प्रसू॑तो जयति॒ निन॑द्धो॒ निन॑द्धो जयति॒ प्रसू॑तः । \newline
46. निन॑द्ध॒ इति॒ नि - न॒द्धः॒ । \newline
47. ज॒य॒ति॒ प्रसू॑तः॒ प्रसू॑तो जयति जयति॒ प्रसू॑तः । \newline
48. प्रसू॑त॒ इति॒ प्र - सू॒तः॒ । \newline
49. रथे॒ तिष्ठꣳ॒॒ स्तिष्ठ॒न् रथे॒ रथे॒ तिष्ठ॑न् नयति नयति॒ तिष्ठ॒न् रथे॒ रथे॒ तिष्ठ॑न् नयति । \newline
50. तिष्ठ॑न् नयति नयति॒ तिष्ठꣳ॒॒ स्तिष्ठ॑न् नयति वा॒जिनो॑ वा॒जिनो॑ नयति॒ तिष्ठꣳ॒॒ स्तिष्ठ॑न् नयति वा॒जिनः॑ । \newline
51. न॒य॒ति॒ वा॒जिनो॑ वा॒जिनो॑ नयति नयति वा॒जिनः॑ पु॒रः पु॒रो वा॒जिनो॑ नयति नयति वा॒जिनः॑ पु॒रः । \newline
52. वा॒जिनः॑ पु॒रः पु॒रो वा॒जिनो॑ वा॒जिनः॑ पु॒रो यत्र॑यत्र॒ यत्र॑यत्र पु॒रो वा॒जिनो॑ वा॒जिनः॑ पु॒रो यत्र॑यत्र । \newline
53. पु॒रो यत्र॑यत्र॒ यत्र॑यत्र पु॒रः पु॒रो यत्र॑यत्र का॒मय॑ते का॒मय॑ते॒ यत्र॑यत्र पु॒रः पु॒रो यत्र॑यत्र का॒मय॑ते । \newline
54. यत्र॑यत्र का॒मय॑ते का॒मय॑ते॒ यत्र॑यत्र॒ यत्र॑यत्र का॒मय॑ते सुषार॒थिः सु॑षार॒थिः का॒मय॑ते॒ यत्र॑यत्र॒ यत्र॑यत्र का॒मय॑ते सुषार॒थिः । \newline
55. यत्र॑य॒त्रेति॒ यत्र॑ - य॒त्र॒ । \newline
56. का॒मय॑ते सुषार॒थिः सु॑षार॒थिः का॒मय॑ते का॒मय॑ते सुषार॒थिः । \newline
57. सु॒षा॒र॒थिरिति॑ सु - सा॒र॒थिः । \newline
58. अ॒भीशू॑नाम् महि॒मान॑म् महि॒मान॑ म॒भीशू॑ना म॒भीशू॑नाम् महि॒मान॑म् पनायत पनायत महि॒मान॑ म॒भीशू॑ना म॒भीशू॑नाम् महि॒मान॑म् पनायत । \newline
59. म॒हि॒मान॑म् पनायत पनायत महि॒मान॑म् महि॒मान॑म् पनायत॒ मनो॒ मनः॑ पनायत महि॒मान॑म् महि॒मान॑म् पनायत॒ मनः॑ । \newline
\pagebreak
\markright{ TS 4.6.6.3  \hfill https://www.vedavms.in \hfill}

\section{ TS 4.6.6.3 }

\textbf{TS 4.6.6.3 } \newline
\textbf{Samhita Paata} \newline

पनायत॒ मनः॑ प॒श्चादनु॑ यच्छन्ति र॒श्मयः॑ ॥ ती॒व्रान् घोषा᳚न् कृण्वते॒ वृष॑पाण॒योऽश्वा॒ रथे॑भिः स॒ह वा॒जय॑न्तः । अ॒व॒क्राम॑न्तः॒ प्रप॑दैर॒मित्रा᳚न् क्षि॒णन्ति॒ शत्रूꣳ॒॒रन॑पव्ययन्तः ॥ र॒थ॒वाह॑नꣳ ह॒विर॑स्य॒ नाम॒ यत्राऽऽ*यु॑धं॒ निहि॑तमस्य॒ वर्म॑ । तत्रा॒ रथ॒मुप॑ श॒ग्मꣳ स॑देम वि॒श्वाहा॑ व॒यꣳ सु॑मन॒स्यमा॑नाः ॥ स्वा॒दु॒षꣳ॒॒ सदः॑ पि॒तरो॑ वयो॒धाः कृ॑च्छ्रे॒श्रितः॒ शक्ती॑वन्तो गभी॒राः । चि॒त्रसे॑ना॒ इषु॑बला॒ अमृ॑द्ध्राः स॒तोवी॑रा उ॒रवो᳚ व्रातसा॒हाः ॥ ब्राह्म॑णासः॒ - [  ] \newline

\textbf{Pada Paata} \newline

प॒ना॒य॒त॒ । मनः॑ । प॒श्चात् । अन्विति॑ । य॒च्छ॒न्ति॒ । र॒श्मयः॑ ॥ ती॒व्रान् । घोषान्॑ । कृ॒ण्व॒ते॒ । वृष॑पाणय॒ इति॒ वृष॑-पा॒ण॒यः॒ । अश्वाः᳚ । रथे॑भिः । स॒ह । वा॒जय॑न्तः ॥ अ॒व॒क्राम॑न्त॒ इत्य॑व - क्राम॑न्तः । प्रप॑दै॒रिति॒ प्र - प॒दैः॒ । अ॒मित्रान्॑ । क्षि॒णन्ति॑ । शत्रून्॑ । अन॑पव्ययन्त॒ इत्यन॑प - व्य॒य॒न्तः॒ ॥ र॒थ॒वाह॑न॒मिति॑ रथ-वाह॑नम् । ह॒विः । अ॒स्य॒ । नाम॑ । यत्र॑ । आयु॑धम् । निहि॑त॒मिति॒ नि - हि॒त॒म् । अ॒स्य॒ । वर्म॑ ॥ तत्र॑ । रथ᳚म् । उपेति॑ । श॒ग्मम् । स॒दे॒म॒ । वि॒श्वाहेति॑ विश्वा - अहा᳚ । व॒यम् । सु॒म॒न॒स्यमा॑ना॒ इति॑ सु - म॒न॒स्यमा॑नाः ॥ स्वा॒दु॒षꣳ॒॒सद॒ इति॑ स्वादु - सꣳ॒॒सदः॑ । पि॒तरः॑ । व॒यो॒धा इति॑ वयः - धाः । कृ॒च्छ्रे॒श्रित॒ इति॑ कृच्छ्रे - श्रितः॑ । शक्ती॑वन्त॒ इति॒ शक्ति॑ - व॒न्तः॒ । ग॒भी॒राः ॥ चि॒त्रसे॑ना॒ इति॑ चि॒त्र - से॒नाः॒ । इषु॑बला॒ इतीषु॑ - ब॒लाः॒ । अमृ॑द्ध्राः । स॒तोवी॑रा॒ इति॑ स॒तः - वी॒राः॒ । उ॒रवः॑ । व्रा॒त॒सा॒हा इति॑ व्रात - सा॒हाः ॥ ब्राह्म॑णासः ।  \newline


\textbf{Krama Paata} \newline

प॒ना॒य॒त॒ मनः॑ । मनः॑ प॒श्चात् । प॒श्चादनु॑ । अनु॑ यच्छन्ति । य॒च्छ॒न्ति॒ र॒श्मयः॑ । र॒श्मय॒ इति॑ र॒श्मयः॑ ॥ ती॒व्रान् घोषान्॑ । घोषा᳚न् कृण्वते । कृ॒ण्व॒ते॒ वृष॑पाणयः । वृष॑पाण॒योऽश्वाः᳚ । वृष॑पाणय॒ इति॒ वृष॑ - पा॒ण॒यः॒ । अश्वा॒ रथे॑भिः । रथे॑भिः स॒ह । स॒ह वा॒जय॑न्तः । वा॒जय॑न्त॒ इति॑ वा॒जय॑न्तः ॥ अ॒व॒क्राम॑न्तः॒ प्रप॑दैः । अ॒व॒क्राम॑न्त॒ इत्य॑व - क्राम॑न्तः । प्रप॑दैर॒मित्रान्॑ । प्रप॑दै॒रिति॒ प्र - प॒दैः॒ । अ॒मित्रा᳚न् क्षि॒णन्ति॑ । क्षि॒णन्ति॒ शत्रून्॑ । शत्रूꣳ॒॒रन॑पव्ययन्तः । अन॑पव्ययन्त॒ इत्यन॑प - व्य॒य॒न्तः॒ ॥ र॒थ॒वाह॑नꣳ ह॒विः । र॒थ॒वाह॑न॒मिति॑ रथ - वाह॑नम् । ह॒विर॑स्य । अ॒स्य॒ नाम॑ । नाम॒ यत्र॑ । यत्रायु॑धम् । आयु॑ध॒म् निहि॑तम् । निहि॑तमस्य । निहि॑त॒मिति॒ नि - हि॒त॒म् । अ॒स्य॒ वर्म॑ । वर्मेति॒ वर्म॑ ॥ तत्रा॒ रथ᳚म् । रथ॒मुप॑ । उप॑ श॒ग्मम् । श॒ग्मꣳ स॑देम । स॒दे॒म॒ वि॒श्वाहा᳚ । वि॒श्वाहा॑ व॒यम् । वि॒श्वाहेति॑ विश्वा - अहा᳚ । व॒यꣳ सु॑मन॒स्यमा॑नाः । सु॒म॒न॒स्यमा॑ना॒ इति॑ सु - म॒न॒स्यमा॑नाः ॥ स्वा॒दु॒षꣳ॒॒सदः॑ पि॒तरः॑ । स्वा॒दु॒षꣳ॒॒सद॒ इति॑ स्वादु - सꣳ॒॒सदः॑ । पि॒तरो॑ वयो॒धाः । व॒यो॒धाः कृ॑च्छ्रे॒श्रितः॑ । व॒यो॒धा इति॑ वयः - धाः । कृ॒च्छ्रे॒श्रितः॒ शक्ती॑वन्तः । कृ॒च्छ्रे॒श्रित॒ इति॑ कृच्छ्रे - श्रितः॑ । शक्ती॑वन्तो गभी॒राः । शक्ती॑वन्त॒ इति॒ शक्ति॑ - व॒न्तः॒ । ग॒भी॒रा इति॑ गभी॒राः ॥ चि॒त्रसे॑ना॒ इषु॑बलाः । चि॒त्रसे॑ना॒ इति॑ चि॒त्र - से॒नाः॒ । इषु॑बला॒ अमृ॑द्ध्राः । इषु॑बला॒ इतीषु॑ - ब॒लाः॒ । अमृ॑द्ध्राः स॒तोवी॑राः । स॒तोवी॑रा उ॒रवः॑ । स॒तोवी॑रा॒ इति॑ स॒तः - वी॒राः॒ । उ॒रवो᳚ व्रातसा॒हाः । व्रा॒त॒सा॒हा इति॑ व्रात - सा॒हाः ॥ ब्राह्म॑नासः॒ पित॑रः \newline

\textbf{Jatai Paata} \newline

1. प॒ना॒य॒त॒ मनो॒ मनः॑ पनायत पनायत॒ मनः॑ । \newline
2. मनः॑ प॒श्चात् प॒श्चान् मनो॒ मनः॑ प॒श्चात् । \newline
3. प॒श्चा दन्वनु॑ प॒श्चात् प॒श्चा दनु॑ । \newline
4. अनु॑ यच्छन्ति यच्छ॒ न्त्यन्वनु॑ यच्छन्ति । \newline
5. य॒च्छ॒न्ति॒ र॒श्मयो॑ र॒श्मयो॑ यच्छन्ति यच्छन्ति र॒श्मयः॑ । \newline
6. र॒श्मय॒ इति॑ र॒श्मयः॑ । \newline
7. ती॒व्रान् घोषा॒न् घोषा᳚न् ती॒व्रान् ती॒व्रान् घोषान्॑ । \newline
8. घोषा᳚न् कृण्वते कृण्वते॒ घोषा॒न् घोषा᳚न् कृण्वते । \newline
9. कृ॒ण्व॒ते॒ वृष॑पाणयो॒ वृष॑पाणयः कृण्वते कृण्वते॒ वृष॑पाणयः । \newline
10. वृष॑पाण॒यो ऽश्वा॒ अश्वा॒ वृष॑पाणयो॒ वृष॑पाण॒यो ऽश्वाः᳚ । \newline
11. वृष॑पाणय॒ इति॒ वृष॑ - पा॒ण॒यः॒ । \newline
12. अश्वा॒ रथे॑भी॒ रथे॑भि॒ रश्वा॒ अश्वा॒ रथे॑भिः । \newline
13. रथे॑भिः स॒ह स॒ह रथे॑भी॒ रथे॑भिः स॒ह । \newline
14. स॒ह वा॒जय॑न्तो वा॒जय॑न्तः स॒ह स॒ह वा॒जय॑न्तः । \newline
15. वा॒जय॑न्त॒ इति॑ वा॒जय॑न्तः । \newline
16. अ॒व॒क्राम॑न्तः॒ प्रप॑दैः॒ प्रप॑दै रव॒क्राम॑न्तो ऽव॒क्राम॑न्तः॒ प्रप॑दैः । \newline
17. अ॒व॒क्राम॑न्त॒ इत्य॑व - क्राम॑न्तः । \newline
18. प्रप॑दै र॒मित्रा॑ न॒मित्रा॒न् प्रप॑दैः॒ प्रप॑दै र॒मित्रान्॑ । \newline
19. प्रप॑दै॒रिति॒ प्र - प॒दैः॒ । \newline
20. अ॒मित्रा᳚न् क्षि॒णन्ति॑ क्षि॒ण न्त्य॒मित्रा॑ न॒मित्रा᳚न् क्षि॒णन्ति॑ । \newline
21. क्षि॒णन्ति॒ शत्रू॒ञ् छत्रू᳚न् क्षि॒णन्ति॑ क्षि॒णन्ति॒ शत्रून्॑ । \newline
22. शत्रूꣳ॒॒-रन॑पव्यय॒न्तो ऽन॑पव्ययन्तः॒ शत्रू॒ञ् छत्रूꣳ॒॒-रन॑पव्ययन्तः । \newline
23. अन॑पव्ययन्त॒ इत्यन॑प - व्य॒य॒न्तः॒ । \newline
24. र॒थ॒वाह॑नꣳ ह॒विर्. ह॒वी र॑थ॒वाह॑नꣳ रथ॒वाह॑नꣳ ह॒विः । \newline
25. र॒थ॒वाह॑न॒मिति॑ रथ - वाह॑नम् । \newline
26. ह॒वि र॑स्यास्य ह॒विर्. ह॒विर॑स्य । \newline
27. अ॒स्य॒ नाम॒ नामा᳚स्यास्य॒ नाम॑ । \newline
28. नाम॒ यत्र॒ यत्र॒ नाम॒ नाम॒ यत्र॑ । \newline
29. यत्रायु॑ध॒ मायु॑धं॒ ॅयत्र॒ यत्रायु॑धम् । \newline
30. आयु॑ध॒म् निहि॑त॒म् निहि॑त॒ मायु॑ध॒ मायु॑ध॒म् निहि॑तम् । \newline
31. निहि॑त मस्यास्य॒ निहि॑त॒म् निहि॑त मस्य । \newline
32. निहि॑त॒मिति॒ नि - हि॒त॒म् । \newline
33. अ॒स्य॒ वर्म॒ वर्मा᳚स्यास्य॒ वर्म॑ । \newline
34. वर्मेति॒ वर्म॑ । \newline
35. तत्रा॒ रथꣳ॒॒ रथ॒म् तत्र॒ तत्रा॒ रथ᳚म् । \newline
36. रथ॒ मुपोप॒ रथꣳ॒॒ रथ॒ मुप॑ । \newline
37. उप॑ श॒ग्मꣳ श॒ग्म मुपोप॑ श॒ग्मम् । \newline
38. श॒ग्मꣳ स॑देम सदेम श॒ग्मꣳ श॒ग्मꣳ स॑देम । \newline
39. स॒दे॒म॒ वि॒श्वाहा॑ वि॒श्वाहा॑ सदेम सदेम वि॒श्वाहा᳚ । \newline
40. वि॒श्वाहा॑ व॒यं ॅव॒यं ॅवि॒श्वाहा॑ वि॒श्वाहा॑ व॒यम् । \newline
41. वि॒श्वाहेति॑ विश्वा - अहा᳚ । \newline
42. व॒यꣳ सु॑मन॒स्यमा॑नाः सुमन॒स्यमा॑ना व॒यं ॅव॒यꣳ सु॑मन॒स्यमा॑नाः । \newline
43. सु॒म॒न॒स्यमा॑ना॒ इति॑ सु - म॒न॒स्यमा॑नाः । \newline
44. स्वा॒दु॒षꣳ॒॒सदः॑ पि॒तरः॑ पि॒तरः॑ स्वादुषꣳ॒॒सदः॑ स्वादुषꣳ॒॒सदः॑ पि॒तरः॑ । \newline
45. स्वा॒दु॒षꣳ॒॒सद॒ इति॑ स्वादु - सꣳ॒॒सदः॑ । \newline
46. पि॒तरो॑ वयो॒धा व॑यो॒धाः पि॒तरः॑ पि॒तरो॑ वयो॒धाः । \newline
47. व॒यो॒धाः कृ॑च्छ्रे॒श्रितः॑ कृच्छ्रे॒श्रितो॑ वयो॒धा व॑यो॒धाः कृ॑च्छ्रे॒श्रितः॑ । \newline
48. व॒यो॒धा इति॑ वयः - धाः । \newline
49. कृ॒च्छ्रे॒श्रितः॒ शक्ती॑वन्तः॒ शक्ती॑वन्तः कृच्छ्रे॒श्रितः॑ कृच्छ्रे॒श्रितः॒ शक्ती॑वन्तः । \newline
50. कृ॒च्छ्रे॒श्रित॒ इति॑ कृच्छ्रे - श्रितः॑ । \newline
51. शक्ती॑वन्तो गभी॒रा ग॑भी॒राः शक्ती॑वन्तः॒ शक्ती॑वन्तो गभी॒राः । \newline
52. शक्ती॑वन्त॒ इति॒ शक्ति॑ - व॒न्तः॒ । \newline
53. ग॒भी॒रा इति॑ गभी॒राः । \newline
54. चि॒त्रसे॑ना॒ इषु॑बला॒ इषु॑बला श्चि॒त्रसे॑ना श्चि॒त्रसे॑ना॒ इषु॑बलाः । \newline
55. चि॒त्रसे॑ना॒ इति॑ चि॒त्र - से॒नाः॒ । \newline
56. इषु॑बला॒ अमृ॑द्ध्रा॒ अमृ॑द्ध्रा॒ इषु॑बला॒ इषु॑बला॒ अमृ॑द्ध्राः । \newline
57. इषु॑बला॒ इतीषु॑ - ब॒लाः॒ । \newline
58. अमृ॑द्ध्राः स॒तोवी॑राः स॒तोवी॑रा॒ अमृ॑द्ध्रा॒ अमृ॑द्ध्राः स॒तोवी॑राः । \newline
59. स॒तोवी॑रा उ॒रव॑ उ॒रवः॑ स॒तोवी॑राः स॒तोवी॑रा उ॒रवः॑ । \newline
60. स॒तोवी॑रा॒ इति॑ स॒तः - वी॒राः॒ । \newline
61. उ॒रवो᳚ व्रातसा॒हा व्रा॑तसा॒हा उ॒रव॑ उ॒रवो᳚ व्रातसा॒हाः । \newline
62. व्रा॒त॒सा॒हा इति॑ व्रात - सा॒हाः । \newline
63. ब्राह्म॑णासः॒ पित॑रः॒ पित॑रो॒ ब्राह्म॑णासो॒ ब्राह्म॑णासः॒ पित॑रः । \newline

\textbf{Ghana Paata } \newline

1. प॒ना॒य॒त॒ मनो॒ मनः॑ पनायत पनायत॒ मनः॑ प॒श्चात् प॒श्चान् मनः॑ पनायत पनायत॒ मनः॑ प॒श्चात् । \newline
2. मनः॑ प॒श्चात् प॒श्चान् मनो॒ मनः॑ प॒श्चा दन्वनु॑ प॒श्चान् मनो॒ मनः॑ प॒श्चादनु॑ । \newline
3. प॒श्चा दन्वनु॑ प॒श्चात् प॒श्चा दनु॑ यच्छन्ति यच्छ॒न् त्यनु॑ प॒श्चात् प॒श्चा दनु॑ यच्छन्ति । \newline
4. अनु॑ यच्छन्ति यच्छ॒न् त्यन्वनु॑ यच्छन्ति र॒श्मयो॑ र॒श्मयो॑ यच्छ॒न् त्यन्वनु॑ यच्छन्ति र॒श्मयः॑ । \newline
5. य॒च्छ॒न्ति॒ र॒श्मयो॑ र॒श्मयो॑ यच्छन्ति यच्छन्ति र॒श्मयः॑ । \newline
6. र॒श्मय॒ इति॑ र॒श्मयः॑ । \newline
7. ती॒व्रान् घोषा॒न् घोषा᳚न् ती॒व्रान् ती॒व्रान् घोषा᳚न् कृण्वते कृण्वते॒ घोषा᳚न् ती॒व्रान् ती॒व्रान् घोषा᳚न् कृण्वते । \newline
8. घोषा᳚न् कृण्वते कृण्वते॒ घोषा॒न् घोषा᳚न् कृण्वते॒ वृष॑पाणयो॒ वृष॑पाणयः कृण्वते॒ घोषा॒न् घोषा᳚न् कृण्वते॒ वृष॑पाणयः । \newline
9. कृ॒ण्व॒ते॒ वृष॑पाणयो॒ वृष॑पाणयः कृण्वते कृण्वते॒ वृष॑पाण॒यो ऽश्वा॒ अश्वा॒ वृष॑पाणयः कृण्वते कृण्वते॒ वृष॑पाण॒यो ऽश्वाः᳚ । \newline
10. वृष॑पाण॒यो ऽश्वा॒ अश्वा॒ वृष॑पाणयो॒ वृष॑पाण॒यो ऽश्वा॒ रथे॑भी॒ रथे॑भि॒ रश्वा॒ वृष॑पाणयो॒ वृष॑पाण॒यो ऽश्वा॒ रथे॑भिः । \newline
11. वृष॑पाणय॒ इति॒ वृष॑ - पा॒ण॒यः॒ । \newline
12. अश्वा॒ रथे॑भी॒ रथे॑भि॒ रश्वा॒ अश्वा॒ रथे॑भिः स॒ह स॒ह रथे॑भि॒ रश्वा॒ अश्वा॒ रथे॑भिः स॒ह । \newline
13. रथे॑भिः स॒ह स॒ह रथे॑भी॒ रथे॑भिः स॒ह वा॒जय॑न्तो वा॒जय॑न्तः स॒ह रथे॑भी॒ रथे॑भिः स॒ह वा॒जय॑न्तः । \newline
14. स॒ह वा॒जय॑न्तो वा॒जय॑न्तः स॒ह स॒ह वा॒जय॑न्तः । \newline
15. वा॒जय॑न्त॒ इति॑ वा॒जय॑न्तः । \newline
16. अ॒व॒क्राम॑न्तः॒ प्रप॑दैः॒ प्रप॑दै रव॒क्राम॑न्तो ऽव॒क्राम॑न्तः॒ प्रप॑दै र॒मित्रा॑ न॒मित्रा॒न् प्रप॑दै रव॒क्राम॑न्तो ऽव॒क्राम॑न्तः॒ प्रप॑दै र॒मित्रान्॑ । \newline
17. अ॒व॒क्राम॑न्त॒ इत्य॑व - क्राम॑न्तः । \newline
18. प्रप॑दै र॒मित्रा॑ न॒मित्रा॒न् प्रप॑दैः॒ प्रप॑दै र॒मित्रा᳚न् क्षि॒णन्ति॑ क्षि॒णन्त्य॒ मित्रा॒न् प्रप॑दैः॒ प्रप॑दै र॒मित्रा᳚न् क्षि॒णन्ति॑ । \newline
19. प्रप॑दै॒रिति॒ प्र - प॒दैः॒ । \newline
20. अ॒मित्रा᳚न् क्षि॒णन्ति॑ क्षि॒णन्त्य॒ मित्रा॑ न॒मित्रा᳚न् क्षि॒णन्ति॒ शत्रू॒ञ् छत्रू᳚न् क्षि॒णन्त्य॒ मित्रा॑ न॒मित्रा᳚न् क्षि॒णन्ति॒ शत्रून्॑ । \newline
21. क्षि॒णन्ति॒ शत्रू॒ञ् छत्रू᳚न् क्षि॒णन्ति॑ क्षि॒णन्ति॒ शत्रूꣳ॒॒ रन॑पव्यय॒न्तो ऽन॑पव्ययन्तः॒ शत्रू᳚न् क्षि॒णन्ति॑ क्षि॒णन्ति॒ शत्रूꣳ॒॒ रन॑पव्ययन्तः । \newline
22. शत्रूꣳ॒॒ रन॑पव्यय॒न्तो ऽन॑पव्ययन्तः॒ शत्रू॒ञ् छत्रूꣳ॒॒ रन॑पव्ययन्तः । \newline
23. अन॑पव्ययन्त॒ इत्यन॑प - व्य॒य॒न्तः॒ । \newline
24. र॒थ॒वाह॑नꣳ ह॒विर्. ह॒वी र॑थ॒वाह॑नꣳ रथ॒वाह॑नꣳ ह॒विर॑स्यास्य ह॒वी र॑थ॒वाह॑नꣳ रथ॒वाह॑नꣳ ह॒विर॑स्य । \newline
25. र॒थ॒वाह॑न॒मिति॑ रथ - वाह॑नम् । \newline
26. ह॒विर॑ स्यास्य ह॒विर्. ह॒विर॑स्य॒ नाम॒ नामा᳚स्य ह॒विर्. ह॒विर॑स्य॒ नाम॑ । \newline
27. अ॒स्य॒ नाम॒ नामा᳚ स्यास्य॒ नाम॒ यत्र॒ यत्र॒ नामा᳚ स्यास्य॒ नाम॒ यत्र॑ । \newline
28. नाम॒ यत्र॒ यत्र॒ नाम॒ नाम॒ यत्रा यु॑ध॒ मायु॑धं॒ ॅयत्र॒ नाम॒ नाम॒ यत्रा यु॑धम् । \newline
29. यत्रा यु॑ध॒ मायु॑धं॒ ॅयत्र॒ यत्रा यु॑ध॒म् निहि॑त॒म् निहि॑त॒ मायु॑धं॒ ॅयत्र॒ यत्रा यु॑ध॒म् निहि॑तम् । \newline
30. आयु॑ध॒म् निहि॑त॒म् निहि॑त॒ मायु॑ध॒ मायु॑ध॒म् निहि॑त मस्यास्य॒ निहि॑त॒ मायु॑ध॒ मायु॑ध॒म् निहि॑त मस्य । \newline
31. निहि॑त मस्यास्य॒ निहि॑त॒म् निहि॑त मस्य॒ वर्म॒ वर्मा᳚स्य॒ निहि॑त॒म् निहि॑त मस्य॒ वर्म॑ । \newline
32. निहि॑त॒मिति॒ नि - हि॒त॒म् । \newline
33. अ॒स्य॒ वर्म॒ वर्मा᳚स्यास्य॒ वर्म॑ । \newline
34. वर्मेति॒ वर्म॑ । \newline
35. तत्रा॒ रथꣳ॒॒ रथ॒म् तत्र॒ तत्रा॒ रथ॒ मुपोप॒ रथ॒म् तत्र॒ तत्रा॒ रथ॒ मुप॑ । \newline
36. रथ॒ मुपोप॒ रथꣳ॒॒ रथ॒ मुप॑ श॒ग्मꣳ श॒ग्म मुप॒ रथꣳ॒॒ रथ॒ मुप॑ श॒ग्मम् । \newline
37. उप॑ श॒ग्मꣳ श॒ग्म मुपोप॑ श॒ग्मꣳ स॑देम सदेम श॒ग्म मुपोप॑ श॒ग्मꣳ स॑देम । \newline
38. श॒ग्मꣳ स॑देम सदेम श॒ग्मꣳ श॒ग्मꣳ स॑देम वि॒श्वाहा॑ वि॒श्वाहा॑ सदेम श॒ग्मꣳ श॒ग्मꣳ स॑देम वि॒श्वाहा᳚ । \newline
39. स॒दे॒म॒ वि॒श्वाहा॑ वि॒श्वाहा॑ सदेम सदेम वि॒श्वाहा॑ व॒यं ॅव॒यं ॅवि॒श्वाहा॑ सदेम सदेम वि॒श्वाहा॑ व॒यम् । \newline
40. वि॒श्वाहा॑ व॒यं ॅव॒यं ॅवि॒श्वाहा॑ वि॒श्वाहा॑ व॒यꣳ सु॑मन॒स्यमा॑नाः सुमन॒स्यमा॑ना व॒यं ॅवि॒श्वाहा॑ वि॒श्वाहा॑ व॒यꣳ सु॑मन॒स्यमा॑नाः । \newline
41. वि॒श्वाहेति॑ विश्वा - अहा᳚ । \newline
42. व॒यꣳ सु॑मन॒स्यमा॑नाः सुमन॒स्यमा॑ना व॒यं ॅव॒यꣳ सु॑मन॒स्यमा॑नाः । \newline
43. सु॒म॒न॒स्यमा॑ना॒ इति॑ सु - म॒न॒स्यमा॑नाः । \newline
44. स्वा॒दु॒षꣳ॒॒सदः॑ पि॒तरः॑ पि॒तरः॑ स्वादुषꣳ॒॒सदः॑ स्वादुषꣳ॒॒सदः॑ पि॒तरो॑ वयो॒धा व॑यो॒धाः पि॒तरः॑ स्वादुषꣳ॒॒सदः॑ स्वादुषꣳ॒॒सदः॑ पि॒तरो॑ वयो॒धाः । \newline
45. स्वा॒दु॒षꣳ॒॒सद॒ इति॑ स्वादु - सꣳ॒॒सदः॑ । \newline
46. पि॒तरो॑ वयो॒धा व॑यो॒धाः पि॒तरः॑ पि॒तरो॑ वयो॒धाः कृ॑च्छ्रे॒श्रितः॑ कृच्छ्रे॒श्रितो॑ वयो॒धाः पि॒तरः॑ पि॒तरो॑ वयो॒धाः कृ॑च्छ्रे॒श्रितः॑ । \newline
47. व॒यो॒धाः कृ॑च्छ्रे॒श्रितः॑ कृच्छ्रे॒श्रितो॑ वयो॒धा व॑यो॒धाः कृ॑च्छ्रे॒श्रितः॒ शक्ती॑वन्तः॒ शक्ती॑वन्तः कृच्छ्रे॒श्रितो॑ वयो॒धा व॑यो॒धाः कृ॑च्छ्रे॒श्रितः॒ शक्ती॑वन्तः । \newline
48. व॒यो॒धा इति॑ वयः - धाः । \newline
49. कृ॒च्छ्रे॒श्रितः॒ शक्ती॑वन्तः॒ शक्ती॑वन्तः कृच्छ्रे॒श्रितः॑ कृच्छ्रे॒श्रितः॒ शक्ती॑वन्तो गभी॒रा ग॑भी॒राः शक्ती॑वन्तः कृच्छ्रे॒श्रितः॑ कृच्छ्रे॒श्रितः॒ शक्ती॑वन्तो गभी॒राः । \newline
50. कृ॒च्छ्रे॒श्रित॒ इति॑ कृच्छ्रे - श्रितः॑ । \newline
51. शक्ती॑वन्तो गभी॒रा ग॑भी॒राः शक्ती॑वन्तः॒ शक्ती॑वन्तो गभी॒राः । \newline
52. शक्ती॑वन्त॒ इति॒ शक्ति॑ - व॒न्तः॒ । \newline
53. ग॒भी॒रा इति॑ गभी॒राः । \newline
54. चि॒त्रसे॑ना॒ इषु॑बला॒ इषु॑बला श्चि॒त्रसे॑ना श्चि॒त्रसे॑ना॒ इषु॑बला॒ अमृ॑द्ध्रा॒ अमृ॑द्ध्रा॒ इषु॑बला श्चि॒त्रसे॑ना श्चि॒त्रसे॑ना॒ इषु॑बला॒ अमृ॑द्ध्राः । \newline
55. चि॒त्रसे॑ना॒ इति॑ चि॒त्र - से॒नाः॒ । \newline
56. इषु॑बला॒ अमृ॑द्ध्रा॒ अमृ॑द्ध्रा॒ इषु॑बला॒ इषु॑बला॒ अमृ॑द्ध्राः स॒तोवी॑राः स॒तोवी॑रा॒ अमृ॑द्ध्रा॒ इषु॑बला॒ इषु॑बला॒ अमृ॑द्ध्राः स॒तोवी॑राः । \newline
57. इषु॑बला॒ इतीषु॑ - ब॒लाः॒ । \newline
58. अमृ॑द्ध्राः स॒तोवी॑राः स॒तोवी॑रा॒ अमृ॑द्ध्रा॒ अमृ॑द्ध्राः स॒तोवी॑रा उ॒रव॑ उ॒रवः॑ स॒तोवी॑रा॒ अमृ॑द्ध्रा॒ अमृ॑द्ध्राः स॒तोवी॑रा उ॒रवः॑ । \newline
59. स॒तोवी॑रा उ॒रव॑ उ॒रवः॑ स॒तोवी॑राः स॒तोवी॑रा उ॒रवो᳚ व्रातसा॒हा व्रा॑तसा॒हा उ॒रवः॑ स॒तोवी॑राः स॒तोवी॑रा उ॒रवो᳚ व्रातसा॒हाः । \newline
60. स॒तोवी॑रा॒ इति॑ स॒तः - वी॒राः॒ । \newline
61. उ॒रवो᳚ व्रातसा॒हा व्रा॑तसा॒हा उ॒रव॑ उ॒रवो᳚ व्रातसा॒हाः । \newline
62. व्रा॒त॒सा॒हा इति॑ व्रात - सा॒हाः । \newline
63. ब्राह्म॑णासः॒ पित॑रः॒ पित॑रो॒ ब्राह्म॑णासो॒ ब्राह्म॑णासः॒ पित॑रः॒ सोम्या॑सः॒ सोम्या॑सः॒ पित॑रो॒ ब्राह्म॑णासो॒ ब्राह्म॑णासः॒ पित॑रः॒ सोम्या॑सः । \newline
\pagebreak
\markright{ TS 4.6.6.4  \hfill https://www.vedavms.in \hfill}

\section{ TS 4.6.6.4 }

\textbf{TS 4.6.6.4 } \newline
\textbf{Samhita Paata} \newline

पित॑रः॒ सोम्या॑सः शि॒वे नो॒ द्यावा॑पृथि॒वी अ॑ने॒हसा᳚ ।पू॒षा नः॑ पातु दुरि॒तादृ॑तावृधो॒ रक्षा॒ माकि॑र्नो अ॒घशꣳ॑स ईशत ॥ सु॒प॒र्णं ॅव॑स्ते मृ॒गो अ॑स्या॒ दन्तो॒ गोभिः॒ सन्न॑द्धा पतति॒ प्रसू॑ता । यत्रा॒ नरः॒ सं च॒ वि च॒ द्रव॑न्ति॒ तत्रा॒स्मभ्य॒मिष॑वः॒ शर्म॑ यꣳसन्न् ॥ ऋजी॑ते॒ परि॑ वृङ्ग्धि॒ नोऽश्मा॑ भवतु नस्त॒नूः । सोमो॒ अधि॑ ब्रवीतु॒ नोऽदि॑तिः॒ - [  ] \newline

\textbf{Pada Paata} \newline

पित॑रः । सोम्या॑सः । शि॒वे इति॑ । नः॒ । द्यावा॑पृथि॒वी इति॒ द्यावा᳚ - पृ॒थि॒वी । अ॒ने॒हसा᳚ ॥ पू॒षा । नः॒ । पा॒तु॒ । दु॒रि॒तादिति॑ दुः - इ॒तात् । ऋ॒ता॒वृ॒ध॒ इत्यृ॑त - वृ॒धः॒ । रक्ष॑ । माकिः॑ । नः॒ । अ॒घशꣳ॑स॒ इत्य॒घ - शꣳ॒॒सः॒ । ई॒श॒त॒ ॥ सु॒प॒र्णमिति॑ सु - प॒र्णम् । व॒स्ते॒ । मृ॒गः । अ॒स्याः॒ । दन्तः॑ । गोभिः॑ । सन्न॒द्धेति॒ सं - न॒द्धा॒ । प॒त॒ति॒ । प्रसू॒तेति॒ प्र - सू॒ता॒ ॥ यत्र॑ । नरः॑ । समिति॑ । च॒ । वीति॑ । च॒ । द्रव॑न्ति । तत्र॑ । अ॒स्मभ्य॒मित्य॒स्म - भ्य॒म् । इष॑वः । शर्म॑ । यꣳ॒॒स॒न्न् ॥ ऋजी॑ते । परीति॑ । वृ॒ङ्ग्धि॒ । नः॒ । अश्मा᳚ । भ॒व॒तु॒ । नः॒ । त॒नूः ॥ सोमः॑ । अधीति॑ । ब्र॒वी॒तु॒ । नः॒ । अदि॑तिः ।  \newline


\textbf{Krama Paata} \newline

पित॑रः॒ सोम्या॑सः । सोम्या॑सः शि॒वे । शि॒वे नः॑ । शि॒वे इति॑ शि॒वे । नो॒ द्यावा॑पृथि॒वी । द्यावा॑पृथि॒वी अ॑ने॒हसा᳚ । द्यावा॑पृथि॒वी इति॒ द्यावा᳚ - पृ॒थि॒वी । अ॒ने॒हसेत्य॑ने॒हसा᳚ ॥ पू॒षा नः॑ । नः॒ पा॒तु॒ । पा॒तु॒ दु॒रि॒तात् । दु॒रि॒तादृ॑तावृधः । दु॒रि॒तादिति॑ दुः - इ॒तात् । ऋ॒ता॒वृ॒धो॒ रक्ष॑ । ऋ॒ता॒वृ॒ध॒ इत्यृ॑त - वृ॒धः॒ । रक्षा॒ माकिः॑ । माकि॑र् नः । नो॒ अ॒घशꣳ॑सः । अ॒घशꣳ॑स ईशत । अ॒घशꣳ॑स॒ इत्य॒घ - शꣳ॒॒सः॒ । ई॒श॒तेती॑शत ॥ सु॒प॒र्णं ॅव॑स्ते । सु॒प॒र्णमिति॑ सु - प॒र्णम् । व॒स्ते॒ मृ॒गः । मृ॒गो अ॑स्याः । अ॒स्या॒ दन्तः॑ । दन्तो॒ गोभिः॑ । गोभिः॒ सन्न॑द्धा । सन्न॑द्धा पतति । सन्न॒द्धेति॒ सम् - न॒द्धा॒ । प॒त॒ति॒ प्रसू॑ता । प्रसू॒तेति॒ प्र - सू॒ता॒ ॥ यत्रा॒ नरः॑ । नरः॒ सम् । सम् च॑ । च॒ वि । वि च॑ । च॒ द्रव॑न्ति । द्रव॑न्ति॒ तत्र॑ । तत्रा॒स्मभ्य᳚म् । अ॒स्मभ्य॒मिष॑वः । अ॒स्मभ्य॒मित्य॒स्म - भ्य॒म् । इष॑वः॒ शर्म॑ । शर्म॑ यꣳसन्न् । यꣳ॒॒स॒न्निति॑ यꣳसन्न् ॥ ऋजी॑ते॒ परि॑ । परि॑ वृङ्ग्धि । वृ॒ङ्ग्धि॒ नः॒ । नोऽश्मा᳚ । अश्मा॑ भवतु । भ॒व॒तु॒ नः॒ । न॒स्त॒नूः । त॒नूरिति॑ त॒नूः ॥ सोमो॒ अधि॑ । अधि॑ ब्रवीतु । ब्र॒वी॒तु॒ नः॒ । नोऽदि॑तिः । अदि॑तिः॒ शर्म॑ \newline

\textbf{Jatai Paata} \newline

1. पित॑रः॒ सोम्या॑सः॒ सोम्या॑सः॒ पित॑रः॒ पित॑रः॒ सोम्या॑सः । \newline
2. सोम्या॑सः शि॒वे शि॒वे सोम्या॑सः॒ सोम्या॑सः शि॒वे । \newline
3. शि॒वे नो॑ नः शि॒वे शि॒वे नः॑ । \newline
4. शि॒वे इति॑ शि॒वे । \newline
5. नो॒ द्यावा॑पृथि॒वी द्यावा॑पृथि॒वी नो॑ नो॒ द्यावा॑पृथि॒वी । \newline
6. द्यावा॑पृथि॒वी अ॑ने॒हसा॑ ऽने॒हसा॒ द्यावा॑पृथि॒वी द्यावा॑पृथि॒वी अ॑ने॒हसा᳚ । \newline
7. द्यावा॑पृथि॒वी इति॒ द्यावा᳚ - पृ॒थि॒वी । \newline
8. अ॒ने॒हसेत्य॑ने॒हसा᳚ । \newline
9. पू॒षा नो॑ नः पू॒षा पू॒षा नः॑ । \newline
10. नः॒ पा॒तु॒ पा॒तु॒ नो॒ नः॒ पा॒तु॒ । \newline
11. पा॒तु॒ दु॒रि॒ताद् दु॑रि॒तात् पा॑तु पातु दुरि॒तात् । \newline
12. दु॒रि॒ता दृ॑तावृध ऋतावृधो दुरि॒ताद् दु॑रि॒ता दृ॑तावृधः । \newline
13. दु॒रि॒तादिति॑ दुः - इ॒तात् । \newline
14. ऋ॒ता॒वृ॒धो॒ रक्ष॒ रक्ष॑ र्तावृध ऋतावृधो॒ रक्ष॑ । \newline
15. ऋ॒ता॒वृ॒ध॒ इत्यृ॑त - वृ॒धः॒ । \newline
16. रक्षा॒ माकि॒र् माकी॒ रक्ष॒ रक्षा॒ माकिः॑ । \newline
17. माकि॑र् नो नो॒ माकि॒र् माकि॑र् नः । \newline
18. नो॒ अ॒घशꣳ॑सो॒ ऽघशꣳ॑सो नो नो अ॒घशꣳ॑सः । \newline
19. अ॒घशꣳ॑स ईशते शता॒ घशꣳ॑सो॒ ऽघशꣳ॑स ईशत । \newline
20. अ॒घशꣳ॑स॒ इत्य॒घ - शꣳ॒॒सः॒ । \newline
21. ई॒श॒तेती॑शत । \newline
22. सु॒प॒र्णं ॅव॑स्ते वस्ते सुप॒र्णꣳ सु॑प॒र्णं ॅव॑स्ते । \newline
23. सु॒प॒र्णमिति॑ सु - प॒र्णम् । \newline
24. व॒स्ते॒ मृ॒गो मृ॒गो व॑स्ते वस्ते मृ॒गः । \newline
25. मृ॒गो अ॑स्या अस्या मृ॒गो मृ॒गो अ॑स्याः । \newline
26. अ॒स्या॒ दन्तो॒ दन्तो॑ अस्या अस्या॒ दन्तः॑ । \newline
27. दन्तो॒ गोभि॒र् गोभि॒र् दन्तो॒ दन्तो॒ गोभिः॑ । \newline
28. गोभिः॒ सन्न॑द्धा॒ सन्न॑द्धा॒ गोभि॒र् गोभिः॒ सन्न॑द्धा । \newline
29. सन्न॑द्धा पतति पतति॒ सन्न॑द्धा॒ सन्न॑द्धा पतति । \newline
30. सन्न॒द्धेति॒ सं - न॒द्धा॒ । \newline
31. प॒त॒ति॒ प्रसू॑ता॒ प्रसू॑ता पतति पतति॒ प्रसू॑ता । \newline
32. प्रसू॒तेति॒ प्र - सू॒ता॒ । \newline
33. यत्रा॒ नरो॒ नरो॒ यत्र॒ यत्रा॒ नरः॑ । \newline
34. नरः॒ सꣳ सम् नरो॒ नरः॒ सम् । \newline
35. सम् च॑ च॒ सꣳ सम् च॑ । \newline
36. च॒ वि वि च॑ च॒ वि । \newline
37. वि च॑ च॒ वि वि च॑ । \newline
38. च॒ द्रव॑न्ति॒ द्रव॑न्ति च च॒ द्रव॑न्ति । \newline
39. द्रव॑न्ति॒ तत्र॒ तत्र॒ द्रव॑न्ति॒ द्रव॑न्ति॒ तत्र॑ । \newline
40. तत्रा॒स्मभ्य॑ म॒स्मभ्य॒म् तत्र॒ तत्रा॒स्मभ्य᳚म् । \newline
41. अ॒स्मभ्य॒ मिष॑व॒ इष॑वो अ॒स्मभ्य॑ म॒स्मभ्य॒ मिष॑वः । \newline
42. अ॒स्मभ्य॒मित्य॒स्म - भ्य॒म् । \newline
43. इष॑वः॒ शर्म॒ शर्मेष॑व॒ इष॑वः॒ शर्म॑ । \newline
44. शर्म॑ यꣳसन्. यꣳस॒ञ् छर्म॒ शर्म॑ यꣳसन्न् । \newline
45. यꣳ॒॒स॒न्निति॑ यꣳसन्न् । \newline
46. ऋजी॑ते॒ परि॒ पर्यृजी॑त॒ ऋजी॑ते॒ परि॑ । \newline
47. परि॑ वृङ्ग्धि वृङ्ग्धि॒ परि॒ परि॑ वृङ्ग्धि । \newline
48. वृ॒ङ्ग्धि॒ नो॒ नो॒ वृ॒ङ्ग्धि॒ वृ॒ङ्ग्धि॒ नः॒ । \newline
49. नो ऽश्मा ऽश्मा॑ नो॒ नो ऽश्मा᳚ । \newline
50. अश्मा॑ भवतु भव॒ त्वश्मा ऽश्मा॑ भवतु । \newline
51. भ॒व॒तु॒ नो॒ नो॒ भ॒व॒तु॒ भ॒व॒तु॒ नः॒ । \newline
52. न॒ स्त॒नू स्त॒नूर् नो॑ न स्त॒नूः । \newline
53. त॒नूरिति॑ त॒नूः । \newline
54. सोमो॒ अध्यधि॒ सोमः॒ सोमो॒ अधि॑ । \newline
55. अधि॑ ब्रवीतु ब्रवी॒ त्वध्यधि॑ ब्रवीतु । \newline
56. ब्र॒वी॒तु॒ नो॒ नो॒ ब्र॒वी॒तु॒ ब्र॒वी॒तु॒ नः॒ । \newline
57. नो ऽदि॑ति॒ रदि॑तिर् नो॒ नो ऽदि॑तिः । \newline
58. अदि॑तिः॒ शर्म॒ शर्मादि॑ति॒ रदि॑तिः॒ शर्म॑ । \newline

\textbf{Ghana Paata } \newline

1. पित॑रः॒ सोम्या॑सः॒ सोम्या॑सः॒ पित॑रः॒ पित॑रः॒ सोम्या॑सः शि॒वे शि॒वे सोम्या॑सः॒ पित॑रः॒ पित॑रः॒ सोम्या॑सः शि॒वे । \newline
2. सोम्या॑सः शि॒वे शि॒वे सोम्या॑सः॒ सोम्या॑सः शि॒वे नो॑ नः शि॒वे सोम्या॑सः॒ सोम्या॑सः शि॒वे नः॑ । \newline
3. शि॒वे नो॑ नः शि॒वे शि॒वे नो॒ द्यावा॑पृथि॒वी द्यावा॑पृथि॒वी नः॑ शि॒वे शि॒वे नो॒ द्यावा॑पृथि॒वी । \newline
4. शि॒वे इति॑ शि॒वे । \newline
5. नो॒ द्यावा॑पृथि॒वी द्यावा॑पृथि॒वी नो॑ नो॒ द्यावा॑पृथि॒वी अ॑ने॒हसा॑ ऽने॒हसा॒ द्यावा॑पृथि॒वी नो॑ नो॒ द्यावा॑पृथि॒वी अ॑ने॒हसा᳚ । \newline
6. द्यावा॑पृथि॒वी अ॑ने॒हसा॑ ऽने॒हसा॒ द्यावा॑पृथि॒वी द्यावा॑पृथि॒वी अ॑ने॒हसा᳚ । \newline
7. द्यावा॑पृथि॒वी इति॒ द्यावा᳚ - पृ॒थि॒वी । \newline
8. अ॒ने॒हसेत्य॑ने॒हसा᳚ । \newline
9. पू॒षा नो॑ नः पू॒षा पू॒षा नः॑ पातु पातु नः पू॒षा पू॒षा नः॑ पातु । \newline
10. नः॒ पा॒तु॒ पा॒तु॒ नो॒ नः॒ पा॒तु॒ दु॒रि॒ताद् दु॑रि॒तात् पा॑तु नो नः पातु दुरि॒तात् । \newline
11. पा॒तु॒ दु॒रि॒ताद् दु॑रि॒तात् पा॑तु पातु दुरि॒ता दृ॑तावृध ऋतावृधो दुरि॒तात् पा॑तु पातु दुरि॒ता दृ॑तावृधः । \newline
12. दु॒रि॒ता दृ॑तावृध ऋतावृधो दुरि॒ताद् दु॑रि॒ता दृ॑तावृधो॒ रक्ष॒ रक्ष॑ र्‌तावृधो दुरि॒ताद् दु॑रि॒ता दृ॑तावृधो॒ रक्ष॑ । \newline
13. दु॒रि॒तादिति॑ दुः - इ॒तात् । \newline
14. ऋ॒ता॒वृ॒धो॒ रक्ष॒ रक्ष॑ र्‌तावृध ऋतावृधो॒ रक्षा॒ माकि॒र् माकी॒ रक्ष॑ र्‌तावृध ऋतावृधो॒ रक्षा॒ माकिः॑ । \newline
15. ऋ॒ता॒वृ॒ध॒ इत्यृ॑त - वृ॒धः॒ । \newline
16. रक्षा॒ माकि॒र् माकी॒ रक्ष॒ रक्षा॒ माकि॑र् नो नो॒ माकी॒ रक्ष॒ रक्षा॒ माकि॑र् नः । \newline
17. माकि॑र् नो नो॒ माकि॒र् माकि॑र् नो अ॒घशꣳ॑सो॒ ऽघशꣳ॑सो नो॒ माकि॒र् माकि॑र् नो अ॒घशꣳ॑सः । \newline
18. नो॒ अ॒घशꣳ॑सो॒ ऽघशꣳ॑सो नो नो अ॒घशꣳ॑स ईशतेशता॒ घशꣳ॑सो नो नो अ॒घशꣳ॑स ईशत । \newline
19. अ॒घशꣳ॑स ईशतेशता॒ घशꣳ॑सो॒ ऽघशꣳ॑स ईशत । \newline
20. अ॒घशꣳ॑स॒ इत्य॒घ - शꣳ॒॒सः॒ । \newline
21. ई॒श॒तेती॑शत । \newline
22. सु॒प॒र्णं ॅव॑स्ते वस्ते सुप॒र्णꣳ सु॑प॒र्णं ॅव॑स्ते मृ॒गो मृ॒गो व॑स्ते सुप॒र्णꣳ सु॑प॒र्णं ॅव॑स्ते मृ॒गः । \newline
23. सु॒प॒र्णमिति॑ सु - प॒र्णम् । \newline
24. व॒स्ते॒ मृ॒गो मृ॒गो व॑स्ते वस्ते मृ॒गो अ॑स्या अस्या मृ॒गो व॑स्ते वस्ते मृ॒गो अ॑स्याः । \newline
25. मृ॒गो अ॑स्या अस्या मृ॒गो मृ॒गो अ॑स्या॒ दन्तो॒ दन्तो॑ अस्या मृ॒गो मृ॒गो अ॑स्या॒ दन्तः॑ । \newline
26. अ॒स्या॒ दन्तो॒ दन्तो॑ अस्या अस्या॒ दन्तो॒ गोभि॒र् गोभि॒र् दन्तो॑ अस्या अस्या॒ दन्तो॒ गोभिः॑ । \newline
27. दन्तो॒ गोभि॒र् गोभि॒र् दन्तो॒ दन्तो॒ गोभिः॒ सन्न॑द्धा॒ सन्न॑द्धा॒ गोभि॒र् दन्तो॒ दन्तो॒ गोभिः॒ सन्न॑द्धा । \newline
28. गोभिः॒ सन्न॑द्धा॒ सन्न॑द्धा॒ गोभि॒र् गोभिः॒ सन्न॑द्धा पतति पतति॒ सन्न॑द्धा॒ गोभि॒र् गोभिः॒ सन्न॑द्धा पतति । \newline
29. सन्न॑द्धा पतति पतति॒ सन्न॑द्धा॒ सन्न॑द्धा पतति॒ प्रसू॑ता॒ प्रसू॑ता पतति॒ सन्न॑द्धा॒ सन्न॑द्धा पतति॒ प्रसू॑ता । \newline
30. सन्न॒द्धेति॒ सं - न॒द्धा॒ । \newline
31. प॒त॒ति॒ प्रसू॑ता॒ प्रसू॑ता पतति पतति॒ प्रसू॑ता । \newline
32. प्रसू॒तेति॒ प्र - सू॒ता॒ । \newline
33. यत्रा॒ नरो॒ नरो॒ यत्र॒ यत्रा॒ नरः॒ सꣳ सम् नरो॒ यत्र॒ यत्रा॒ नरः॒ सम् । \newline
34. नरः॒ सꣳ सम् नरो॒ नरः॒ सम् च॑ च॒ सम् नरो॒ नरः॒ सम् च॑ । \newline
35. सम् च॑ च॒ सꣳ सम् च॒ वि वि च॒ सꣳ सम् च॒ वि । \newline
36. च॒ वि वि च॑ च॒ वि च॑ च॒ वि च॑ च॒ वि च॑ । \newline
37. वि च॑ च॒ वि वि च॒ द्रव॑न्ति॒ द्रव॑न्ति च॒ वि वि च॒ द्रव॑न्ति । \newline
38. च॒ द्रव॑न्ति॒ द्रव॑न्ति च च॒ द्रव॑न्ति॒ तत्र॒ तत्र॒ द्रव॑न्ति च च॒ द्रव॑न्ति॒ तत्र॑ । \newline
39. द्रव॑न्ति॒ तत्र॒ तत्र॒ द्रव॑न्ति॒ द्रव॑न्ति॒ तत्रा॒स्मभ्य॑ म॒स्मभ्य॒म् तत्र॒ द्रव॑न्ति॒ द्रव॑न्ति॒ तत्रा॒स्मभ्य᳚म् । \newline
40. तत्रा॒स्मभ्य॑ म॒स्मभ्य॒म् तत्र॒ तत्रा॒स्मभ्य॒ मिष॑व॒ इष॑वो अ॒स्मभ्य॒म् तत्र॒ तत्रा॒स्मभ्य॒ मिष॑वः । \newline
41. अ॒स्मभ्य॒ मिष॑व॒ इष॑वो अ॒स्मभ्य॑ म॒स्मभ्य॒ मिष॑वः॒ शर्म॒ शर्मेष॑वो अ॒स्मभ्य॑ म॒स्मभ्य॒ मिष॑वः॒ शर्म॑ । \newline
42. अ॒स्मभ्य॒मित्य॒स्म - भ्य॒म् । \newline
43. इष॑वः॒ शर्म॒ शर्मेष॑व॒ इष॑वः॒ शर्म॑ यꣳसन्. यꣳस॒ञ् छर्मे ष॑व॒ इष॑वः॒ शर्म॑ यꣳसन्न् । \newline
44. शर्म॑ यꣳसन्. यꣳस॒ञ् छर्म॒ शर्म॑ यꣳसन्न् । \newline
45. यꣳ॒॒स॒न्निति॑ यꣳसन्न् । \newline
46. ऋजी॑ते॒ परि॒ पर्यृजी॑त॒ ऋजी॑ते॒ परि॑ वृङ्ग्धि वृङ्ग्धि॒ पर्यृजी॑त॒ ऋजी॑ते॒ परि॑ वृङ्ग्धि । \newline
47. परि॑ वृङ्ग्धि वृङ्ग्धि॒ परि॒ परि॑ वृङ्ग्धि नो नो वृङ्ग्धि॒ परि॒ परि॑ वृङ्ग्धि नः । \newline
48. वृ॒ङ्ग्धि॒ नो॒ नो॒ वृ॒ङ्ग्धि॒ वृ॒ङ्ग्धि॒ नो ऽश्मा ऽश्मा॑ नो वृङ्ग्धि वृङ्ग्धि॒ नो ऽश्मा᳚ । \newline
49. नो ऽश्मा ऽश्मा॑ नो॒ नो ऽश्मा॑ भवतु भव॒ त्वश्मा॑ नो॒ नो ऽश्मा॑ भवतु । \newline
50. अश्मा॑ भवतु भव॒ त्वश्मा ऽश्मा॑ भवतु नो नो भव॒ त्वश्मा ऽश्मा॑ भवतु नः । \newline
51. भ॒व॒तु॒ नो॒ नो॒ भ॒व॒तु॒ भ॒व॒तु॒ न॒ स्त॒नू स्त॒नूर् नो॑ भवतु भवतु न स्त॒नूः । \newline
52. न॒ स्त॒नू स्त॒नूर् नो॑ न स्त॒नूः । \newline
53. त॒नूरिति॑ त॒नूः । \newline
54. सोमो॒ अध्यधि॒ सोमः॒ सोमो॒ अधि॑ ब्रवीतु ब्रवी॒ त्वधि॒ सोमः॒ सोमो॒ अधि॑ ब्रवीतु । \newline
55. अधि॑ ब्रवीतु ब्रवी॒ त्वध्यधि॑ ब्रवीतु नो नो ब्रवी॒ त्वध्यधि॑ ब्रवीतु नः । \newline
56. ब्र॒वी॒तु॒ नो॒ नो॒ ब्र॒वी॒तु॒ ब्र॒वी॒तु॒ नो ऽदि॑ति॒ रदि॑तिर् नो ब्रवीतु ब्रवीतु॒ नो ऽदि॑तिः । \newline
57. नो ऽदि॑ति॒ रदि॑तिर् नो॒ नो ऽदि॑तिः॒ शर्म॒ शर्मादि॑तिर् नो॒ नो ऽदि॑तिः॒ शर्म॑ । \newline
58. अदि॑तिः॒ शर्म॒ शर्मादि॑ति॒ रदि॑तिः॒ शर्म॑ यच्छतु यच्छतु॒ शर्मादि॑ति॒ रदि॑तिः॒ शर्म॑ यच्छतु । \newline
\pagebreak
\markright{ TS 4.6.6.5  \hfill https://www.vedavms.in \hfill}

\section{ TS 4.6.6.5 }

\textbf{TS 4.6.6.5 } \newline
\textbf{Samhita Paata} \newline

शर्म॑ यच्छतु ॥ आ ज॑ङ्घन्ति॒ सान्वे॑षां ज॒घनाꣳ॒॒ उप॑ जिघ्नते । अश्वा॑जनि॒ प्रचे॑त॒सोऽश्वा᳚न्थ् स॒मथ्सु॑ चोदय ॥ अहि॑रिव भो॒गैः पर्ये॑ति बा॒हुं ज्याया॑ हे॒तिं प॑रि॒बाध॑मानः । ह॒स्त॒घ्नो विश्वा॑ व॒युना॑नि वि॒द्वान् पुमा॒न् पुमाꣳ॑सं॒ परि॑ पातु वि॒श्वतः॑ ॥ वन॑स्पते वी॒ड्व॑ङ्गो॒ हि भू॒या अ॒स्मथ् स॑खा प्र॒तर॑णः सु॒वीरः॑ । गोभिः॒ सन्न॑द्धो असि वी॒डय॑स्वाऽऽस्था॒ता ते॑ जयतु॒ जेत्वा॑नि ॥ दि॒वः पृ॑थि॒व्याः पर्यो- [  ] \newline

\textbf{Pada Paata} \newline

शर्म॑ । य॒च्छ॒तु॒ ॥ एति॑ । ज॒ङ्घ॒न्ति॒ । सानु॑ । ए॒षा॒म् । ज॒घनान्॑ । उपेति॑ । जि॒घ्न॒ते॒ ॥ अश्वा॑ज॒नीत्यश्व॑ - अ॒ज॒नि॒ । प्रचे॑तस॒ इति॒ प्र - चे॒त॒सः॒ । अश्वान्॑ । स॒मथ्स्विति॑ स॒मत् - सु॒ । चो॒द॒य॒ ॥ अहिः॑ । इ॒व॒ । भो॒गैः । परीति॑ । ए॒ति॒ । बा॒हुम् । ज्यायाः᳚ । हे॒तिम् । प॒रि॒बाध॑मान॒ इति॑ परि - बाध॑मानः ॥ ह॒स्त॒घ्न इति॑ हस्त - घ्नः । विश्वा᳚ । व॒युना॑नि । वि॒द्वान् । पुमान्॑ । पुमाꣳ॑सम् । परीति॑ । पा॒तु॒ । वि॒श्वतः॑ ॥ वन॑स्पते । वी॒ड्व॑ङ्ग॒ इति॑ वी॒डु - अ॒ङ्गः॒ । हि । भू॒याः । अ॒स्मथ्स॒खेत्य॒स्मत् - स॒खा॒ । प्र॒तर॑ण॒ इति॑ प्र - तर॑णः । सु॒वीर॒ इति॑ सु - वीरः॑ ॥ गोभिः॑ । सन्न॑द्ध॒ इति॒ सं - न॒द्धः॒ । अ॒सि॒ । वी॒डय॑स्व । आ॒स्था॒तेत्या᳚ - स्था॒ता । ते॒ । ज॒य॒तु॒ । जेत्वा॑नि ॥ दि॒वः । पृ॒थि॒व्याः । परीति॑ ।  \newline


\textbf{Krama Paata} \newline

शर्म॑ यच्छतु । य॒च्छ॒त्विति॑ यच्छतु ॥ आ ज॑ङ्घन्ति । ज॒ङ्घ॒न्ति॒ सानु॑ । सान्वे॑षाम् । ए॒षां॒ ज॒घनान्॑ । ज॒घनाꣳ॒॒ उप॑ । उप॑ जिघ्नते । जि॒घ्न॒त॒ इति॑ जिघ्नते ॥ अश्वा॑जनि॒ प्रचे॑तसः । अश्वा॑ज॒नीत्यश्व॑ - अ॒ज॒नि॒ । प्रचे॑त॒सोऽश्वान्॑ । प्रचे॑तस॒ इति॒ प्र - चे॒त॒सः॒ । अश्वा᳚न्थ् स॒मथ्सु॑ । स॒मथ्सु॑ चोदय । स॒मथ्स्विति॑ स॒मत् - सु॒ । चो॒द॒येति॑ चोदय ॥ अहि॑रिव । इ॒व॒ भो॒गैः । भो॒गैः परि॑ । पर्ये॑ति । ए॒ति॒ बा॒हुम् । बा॒हुम् ज्यायाः᳚ । ज्याया॑ हे॒तिम् । हे॒तिम् प॑रि॒बाध॑मानः । प॒रि॒बाध॑मान॒ इति॑ परि - बाध॑मानः ॥ ह॒स्त॒घ्नो विश्वा᳚ । ह॒स्त॒घ्न इति॑ हस्त - घ्नः । विश्वा॑ व॒युना॑नि । व॒युना॑नि वि॒द्वान् । वि॒द्वान् पुमान्॑ । पुमा॒न् पुमाꣳ॑सम् । पुमाꣳ॑स॒म् परि॑ । परि॑ पातु । पा॒तु॒ वि॒श्वतः॑ । वि॒श्वत॒ इति॑ वि॒श्वतः॑ ॥ वन॑स्पते वी॒ड्व॑ङ्गः । वी॒ड्व॑ङ्गो॒ हि । वी॒ड्व॑ङ्ग॒ इति॑ वी॒डु - अ॒ङ्गः॒ । हि भू॒याः । भू॒या अ॒स्मथ्स॑खा । अ॒स्मथ्स॑खा प्र॒तर॑णः । अ॒स्मथ्स॒खेत्य॒स्मत् - स॒खा॒ । प्र॒तर॑णः सु॒वीरः॑ । प्र॒तर॑ण॒ इति॑ प्र - तर॑णः । सु॒वीर॒ इति॑ सु - वीरः॑ ॥ गोभिः॒ सन्न॑द्धः । सन्न॑द्धो असि । सन्न॑द्ध॒ इति॒ सं - न॒द्धः॒ । अ॒सि॒ वी॒डय॑स्व । वी॒डय॑स्वास्था॒ता । आ॒स्था॒ता ते᳚ । आ॒स्था॒तेत्या᳚ - स्था॒ता । ते॒ ज॒य॒तु॒ । ज॒य॒तु॒ जेत्वा॑नि । जेत्वा॒नीति॒ जेत्वा॑नि ॥ दि॒वः पृ॑थि॒व्याः । पृ॒थि॒व्याः परिः॑ । पर्योजः॑ \newline

\textbf{Jatai Paata} \newline

1. शर्म॑ यच्छतु यच्छतु॒ शर्म॒ शर्म॑ यच्छतु । \newline
2. य॒च्छ॒त्विति॑ यच्छतु । \newline
3. आ ज॑ङ्घन्ति जङ्घ॒न्त्या ज॑ङ्घन्ति । \newline
4. ज॒ङ्घ॒न्ति॒ सानु॒ सानु॑ जङ्घन्ति जङ्घन्ति॒ सानु॑ । \newline
5. सान्वे॑षा मेषाꣳ॒॒ सानु॒ सान्वे॑षाम् । \newline
6. ए॒षा॒म् ज॒घना᳚न् ज॒घनाꣳ॑ एषा मेषाम् ज॒घनान्॑ । \newline
7. ज॒घनाꣳ॒॒ उपोप॑ ज॒घना᳚न् ज॒घनाꣳ॒॒ उप॑ । \newline
8. उप॑ जिघ्नते जिघ्नत॒ उपोप॑ जिघ्नते । \newline
9. जि॒घ्न॒त॒ इति॑ जिघ्नते । \newline
10. अश्वा॑जनि॒ प्रचे॑तसः॒ प्रचे॑त॒सो ऽश्वा॑ज॒ न्यश्वा॑जनि॒ प्रचे॑तसः । \newline
11. अश्वा॑ज॒नीत्यश्व॑ - अ॒ज॒नि॒ । \newline
12. प्रचे॑त॒सो ऽश्वा॒ नश्वा॒न् प्रचे॑तसः॒ प्रचे॑त॒सो ऽश्वान्॑ । \newline
13. प्रचे॑तस॒ इति॒ प्र - चे॒त॒सः॒ । \newline
14. अश्वा᳚न् थ्स॒मथ्सु॑ स॒मथ्स्वश्वा॒ नश्वा᳚न् थ्स॒मथ्सु॑ । \newline
15. स॒मथ्सु॑ चोदय चोदय स॒मथ्सु॑ स॒मथ्सु॑ चोदय । \newline
16. स॒मथ्स्विति॑ स॒मत् - सु॒ । \newline
17. चो॒द॒येति॑ चोदय । \newline
18. अहि॑ रिवे॒ वाहि॒ रहि॑ रिव । \newline
19. इ॒व॒ भो॒गैर् भो॒गै रि॑वेव भो॒गैः । \newline
20. भो॒गैः परि॒ परि॑ भो॒गैर् भो॒गैः परि॑ । \newline
21. पर्ये᳚ त्येति॒ परि॒ पर्ये॑ति । \newline
22. ए॒ति॒ बा॒हुम् बा॒हु मे᳚त्येति बा॒हुम् । \newline
23. बा॒हुम् ज्याया॒ ज्याया॑ बा॒हुम् बा॒हुम् ज्यायाः᳚ । \newline
24. ज्याया॑ हे॒तिꣳ हे॒तिम् ज्याया॒ ज्याया॑ हे॒तिम् । \newline
25. हे॒तिम् प॑रि॒बाध॑मानः परि॒बाध॑मानो हे॒तिꣳ हे॒तिम् प॑रि॒बाध॑मानः । \newline
26. प॒रि॒बाध॑मान॒ इति॑ परि - बाध॑मानः । \newline
27. ह॒स्त॒घ्नो विश्वा॒ विश्वा॑ हस्त॒घ्नो ह॑स्त॒घ्नो विश्वा᳚ । \newline
28. ह॒स्त॒घ्न इति॑ हस्त - घ्नः । \newline
29. विश्वा॑ व॒युना॑नि व॒युना॑नि॒ विश्वा॒ विश्वा॑ व॒युना॑नि । \newline
30. व॒युना॑नि वि॒द्वान्. वि॒द्वान्. व॒युना॑नि व॒युना॑नि वि॒द्वान् । \newline
31. वि॒द्वान् पुमा॒न् पुमान्॑. वि॒द्वान्. वि॒द्वान् पुमान्॑ । \newline
32. पुमा॒न् पुमाꣳ॑स॒म् पुमाꣳ॑स॒म् पुमा॒न् पुमा॒न् पुमाꣳ॑सम् । \newline
33. पुमाꣳ॑स॒म् परि॒ परि॒ पुमाꣳ॑स॒म् पुमाꣳ॑स॒म् परि॑ । \newline
34. परि॑ पातु पातु॒ परि॒ परि॑ पातु । \newline
35. पा॒तु॒ वि॒श्वतो॑ वि॒श्वत॑ स्पातु पातु वि॒श्वतः॑ । \newline
36. वि॒श्वत॒ इति॑ वि॒श्वतः॑ । \newline
37. वन॑स्पते वी॒ड्व॑ङ्गो वी॒ड्व॑ङ्गो॒ वन॑स्पते॒ वन॑स्पते वी॒ड्व॑ङ्गः । \newline
38. वी॒ड्व॑ङ्गो॒ हि हि वी॒ड्व॑ङ्गो वी॒ड्व॑ङ्गो॒ हि । \newline
39. वी॒ड्व॑ङ्ग॒ इति॑ वी॒डु - अ॒ङ्गः॒ । \newline
40. हि भू॒या भू॒या हि हि भू॒याः । \newline
41. भू॒या अ॒स्मथ्स॑खा॒ ऽस्मथ्स॑खा भू॒या भू॒या अ॒स्मथ्स॑खा । \newline
42. अ॒स्मथ्स॑खा प्र॒तर॑णः प्र॒तर॑णो अ॒स्मथ्स॑खा॒ ऽस्मथ्स॑खा प्र॒तर॑णः । \newline
43. अ॒स्मथ्स॒खेत्य॒स्मत् - स॒खा॒ । \newline
44. प्र॒तर॑णः सु॒वीरः॑ सु॒वीरः॑ प्र॒तर॑णः प्र॒तर॑णः सु॒वीरः॑ । \newline
45. प्र॒तर॑ण॒ इति॑ प्र - तर॑णः । \newline
46. सु॒वीर॒ इति॑ सु - वीरः॑ । \newline
47. गोभिः॒ सन्न॑द्धः॒ सन्न॑द्धो॒ गोभि॒र् गोभिः॒ सन्न॑द्धः । \newline
48. सन्न॑द्धो अस्यसि॒ सन्न॑द्धः॒ सन्न॑द्धो असि । \newline
49. सन्न॑द्ध॒ इति॒ सं - न॒द्धः॒ । \newline
50. अ॒सि॒ वी॒डय॑स्व वी॒डय॑ स्वास्यसि वी॒डय॑स्व । \newline
51. वी॒डय॑स्वा स्था॒ता ऽऽस्था॒ता वी॒डय॑स्व वी॒डय॑स्वा स्था॒ता । \newline
52. आ॒स्था॒ता ते॑ त आस्था॒ता ऽऽस्था॒ता ते᳚ । \newline
53. आ॒स्था॒तेत्या᳚ - स्था॒ता । \newline
54. ते॒ ज॒य॒तु॒ ज॒य॒तु॒ ते॒ ते॒ ज॒य॒तु॒ । \newline
55. ज॒य॒तु॒ जेत्वा॑नि॒ जेत्वा॑नि जयतु जयतु॒ जेत्वा॑नि । \newline
56. जेत्वा॒नीति॒ जेत्वा॑नि । \newline
57. दि॒वः पृ॑थि॒व्याः पृ॑थि॒व्या दि॒वो दि॒वः पृ॑थि॒व्याः । \newline
58. पृ॒थि॒व्याः परि॒ परि॑ पृथि॒व्याः पृ॑थि॒व्याः परि॑ । \newline
59. पर्योज॒ ओजः॒ परि॒ पर्योजः॑ । \newline

\textbf{Ghana Paata } \newline

1. शर्म॑ यच्छतु यच्छतु॒ शर्म॒ शर्म॑ यच्छतु । \newline
2. य॒च्छ॒त्विति॑ यच्छतु । \newline
3. आ ज॑ङ्घन्ति जङ्घ॒न्त्या ज॑ङ्घन्ति॒ सानु॒ सानु॑ जङ्घ॒न्त्या ज॑ङ्घन्ति॒ सानु॑ । \newline
4. ज॒ङ्घ॒न्ति॒ सानु॒ सानु॑ जङ्घन्ति जङ्घन्ति॒ सान्वे॑षा मेषाꣳ॒॒ सानु॑ जङ्घन्ति जङ्घन्ति॒ सान्वे॑षाम् । \newline
5. सान्वे॑षा मेषाꣳ॒॒ सानु॒ सान्वे॑षाम् ज॒घना᳚न् ज॒घनाꣳ॑ एषाꣳ॒॒ सानु॒ सान्वे॑षाम् ज॒घनान्॑ । \newline
6. ए॒षा॒म् ज॒घना᳚न् ज॒घनाꣳ॑ एषा मेषाम् ज॒घनाꣳ॒॒ उपोप॑ ज॒घनाꣳ॑ एषा मेषाम् ज॒घनाꣳ॒॒ उप॑ । \newline
7. ज॒घनाꣳ॒॒ उपोप॑ ज॒घना᳚न् ज॒घनाꣳ॒॒ उप॑ जिघ्नते जिघ्नत॒ उप॑ ज॒घना᳚न् ज॒घनाꣳ॒॒ उप॑ जिघ्नते । \newline
8. उप॑ जिघ्नते जिघ्नत॒ उपोप॑ जिघ्नते । \newline
9. जि॒घ्न॒त॒ इति॑ जिघ्नते । \newline
10. अश्वा॑जनि॒ प्रचे॑तसः॒ प्रचे॑त॒सो ऽश्वा॑ज॒ न्यश्वा॑जनि॒ प्रचे॑त॒सो ऽश्वा॒ नश्वा॒न् प्रचे॑त॒सो ऽश्वा॑ज॒ न्यश्वा॑जनि॒ प्रचे॑त॒सो ऽश्वान्॑ । \newline
11. अश्वा॑ज॒नीत्यश्व॑ - अ॒ज॒नि॒ । \newline
12. प्रचे॑त॒सो ऽश्वा॒ नश्वा॒न् प्रचे॑तसः॒ प्रचे॑त॒सो ऽश्वा᳚न् थ्स॒मथ्सु॑ स॒मथ्स्वश्वा॒न् प्रचे॑तसः॒ प्रचे॑त॒सो ऽश्वा᳚न् थ्स॒मथ्सु॑ । \newline
13. प्रचे॑तस॒ इति॒ प्र - चे॒त॒सः॒ । \newline
14. अश्वा᳚न् थ्स॒मथ्सु॑ स॒मथ्स्वश्वा॒ नश्वा᳚न् थ्स॒मथ्सु॑ चोदय चोदय स॒मथ्स्वश्वा॒ नश्वा᳚न् थ्स॒मथ्सु॑ चोदय । \newline
15. स॒मथ्सु॑ चोदय चोदय स॒मथ्सु॑ स॒मथ्सु॑ चोदय । \newline
16. स॒मथ्स्विति॑ स॒मत् - सु॒ । \newline
17. चो॒द॒येति॑ चोदय । \newline
18. अहि॑ रिवे॒ वाहि॒ रहि॑ रिव भो॒गैर् भो॒गै रि॒वाहि॒ रहि॑ रिव भो॒गैः । \newline
19. इ॒व॒ भो॒गैर् भो॒गै रि॑वेव भो॒गैः परि॒ परि॑ भो॒गै रि॑वेव भो॒गैः परि॑ । \newline
20. भो॒गैः परि॒ परि॑ भो॒गैर् भो॒गैः पर्ये᳚त्येति॒ परि॑ भो॒गैर् भो॒गैः पर्ये॑ति । \newline
21. पर्ये᳚त्येति॒ परि॒ पर्ये॑ति बा॒हुम् बा॒हु मे॑ति॒ परि॒ पर्ये॑ति बा॒हुम् । \newline
22. ए॒ति॒ बा॒हुम् बा॒हु मे᳚त्येति बा॒हुम् ज्याया॒ ज्याया॑ बा॒हु मे᳚त्येति बा॒हुम् ज्यायाः᳚ । \newline
23. बा॒हुम् ज्याया॒ ज्याया॑ बा॒हुम् बा॒हुम् ज्याया॑ हे॒तिꣳ हे॒तिम् ज्याया॑ बा॒हुम् बा॒हुम् ज्याया॑ हे॒तिम् । \newline
24. ज्याया॑ हे॒तिꣳ हे॒तिम् ज्याया॒ ज्याया॑ हे॒तिम् प॑रि॒बाध॑मानः परि॒बाध॑मानो हे॒तिम् ज्याया॒ ज्याया॑ हे॒तिम् प॑रि॒बाध॑मानः । \newline
25. हे॒तिम् प॑रि॒बाध॑मानः परि॒बाध॑मानो हे॒तिꣳ हे॒तिम् प॑रि॒बाध॑मानः । \newline
26. प॒रि॒बाध॑मान॒ इति॑ परि - बाध॑मानः । \newline
27. ह॒स्त॒घ्नो विश्वा॒ विश्वा॑ हस्त॒घ्नो ह॑स्त॒घ्नो विश्वा॑ व॒युना॑नि व॒युना॑नि॒ विश्वा॑ हस्त॒घ्नो ह॑स्त॒घ्नो विश्वा॑ व॒युना॑नि । \newline
28. ह॒स्त॒घ्न इति॑ हस्त - घ्नः । \newline
29. विश्वा॑ व॒युना॑नि व॒युना॑नि॒ विश्वा॒ विश्वा॑ व॒युना॑नि वि॒द्वान्. वि॒द्वान्. व॒युना॑नि॒ विश्वा॒ विश्वा॑ व॒युना॑नि वि॒द्वान् । \newline
30. व॒युना॑नि वि॒द्वान्. वि॒द्वान्. व॒युना॑नि व॒युना॑नि वि॒द्वान् पुमा॒न् पुमान्॑. वि॒द्वान्. व॒युना॑नि व॒युना॑नि वि॒द्वान् पुमान्॑ । \newline
31. वि॒द्वान् पुमा॒न् पुमान्॑. वि॒द्वान्. वि॒द्वान् पुमा॒न् पुमाꣳ॑स॒म् पुमाꣳ॑स॒म् पुमान्॑. वि॒द्वान्. वि॒द्वान् पुमा॒न् पुमाꣳ॑सम् । \newline
32. पुमा॒न् पुमाꣳ॑स॒म् पुमाꣳ॑स॒म् पुमा॒न् पुमा॒न् पुमाꣳ॑स॒म् परि॒ परि॒ पुमाꣳ॑स॒म् पुमा॒न् पुमा॒न् पुमाꣳ॑स॒म् परि॑ । \newline
33. पुमाꣳ॑स॒म् परि॒ परि॒ पुमाꣳ॑स॒म् पुमाꣳ॑स॒म् परि॑ पातु पातु॒ परि॒ पुमाꣳ॑स॒म् पुमाꣳ॑स॒म् परि॑ पातु । \newline
34. परि॑ पातु पातु॒ परि॒ परि॑ पातु वि॒श्वतो॑ वि॒श्वत॑ स्पातु॒ परि॒ परि॑ पातु वि॒श्वतः॑ । \newline
35. पा॒तु॒ वि॒श्वतो॑ वि॒श्वत॑ स्पातु पातु वि॒श्वतः॑ । \newline
36. वि॒श्वत॒ इति॑ वि॒श्वतः॑ । \newline
37. वन॑स्पते वी॒ड्व॑ङ्गो वी॒ड्व॑ङ्गो॒ वन॑स्पते॒ वन॑स्पते वी॒ड्व॑ङ्गो॒ हि हि वी॒ड्व॑ङ्गो॒ वन॑स्पते॒ वन॑स्पते वी॒ड्व॑ङ्गो॒ हि । \newline
38. वी॒ड्व॑ङ्गो॒ हि हि वी॒ड्व॑ङ्गो वी॒ड्व॑ङ्गो॒ हि भू॒या भू॒या हि वी॒ड्व॑ङ्गो वी॒ड्व॑ङ्गो॒ हि भू॒याः । \newline
39. वी॒ड्व॑ङ्ग॒ इति॑ वी॒डु - अ॒ङ्गः॒ । \newline
40. हि भू॒या भू॒या हि हि भू॒या अ॒स्मथ्स॑खा॒ ऽस्मथ्स॑खा भू॒या हि हि भू॒या अ॒स्मथ्स॑खा । \newline
41. भू॒या अ॒स्मथ्स॑खा॒ ऽस्मथ्स॑खा भू॒या भू॒या अ॒स्मथ्स॑खा प्र॒तर॑णः प्र॒तर॑णो अ॒स्मथ्स॑खा भू॒या भू॒या अ॒स्मथ्स॑खा प्र॒तर॑णः । \newline
42. अ॒स्मथ्स॑खा प्र॒तर॑णः प्र॒तर॑णो अ॒स्मथ्स॑खा॒ ऽस्मथ्स॑खा प्र॒तर॑णः सु॒वीरः॑ सु॒वीरः॑ प्र॒तर॑णो अ॒स्मथ्स॑खा॒ ऽस्मथ्स॑खा प्र॒तर॑णः सु॒वीरः॑ । \newline
43. अ॒स्मथ्स॒खेत्य॒स्मत् - स॒खा॒ । \newline
44. प्र॒तर॑णः सु॒वीरः॑ सु॒वीरः॑ प्र॒तर॑णः प्र॒तर॑णः सु॒वीरः॑ । \newline
45. प्र॒तर॑ण॒ इति॑ प्र - तर॑णः । \newline
46. सु॒वीर॒ इति॑ सु - वीरः॑ । \newline
47. गोभिः॒ सन्न॑द्धः॒ सन्न॑द्धो॒ गोभि॒र् गोभिः॒ सन्न॑द्धो अस्यसि॒ सन्न॑द्धो॒ गोभि॒र् गोभिः॒ सन्न॑द्धो असि । \newline
48. सन्न॑द्धो अस्यसि॒ सन्न॑द्धः॒ सन्न॑द्धो असि वी॒डय॑स्व वी॒डय॑स्वासि॒ सन्न॑द्धः॒ सन्न॑द्धो असि वी॒डय॑स्व । \newline
49. सन्न॑द्ध॒ इति॒ सं - न॒द्धः॒ । \newline
50. अ॒सि॒ वी॒डय॑स्व वी॒डय॑ स्वास्यसि वी॒डय॑स्वा स्था॒ता ऽऽस्था॒ता वी॒डय॑स्वास्यसि वी॒डय॑स्वा स्था॒ता । \newline
51. वी॒डय॑स्वा स्था॒ता ऽऽस्था॒ता वी॒डय॑स्व वी॒डय॑स्वा स्था॒ता ते॑ त आस्था॒ता वी॒डय॑स्व वी॒डय॑स्वास्था॒ता ते᳚ । \newline
52. आ॒स्था॒ता ते॑ त आस्था॒ता ऽऽस्था॒ता ते॑ जयतु जयतु त आस्था॒ता ऽऽस्था॒ता ते॑ जयतु । \newline
53. आ॒स्था॒तेत्या᳚ - स्था॒ता । \newline
54. ते॒ ज॒य॒तु॒ ज॒य॒तु॒ ते॒ ते॒ ज॒य॒तु॒ जेत्वा॑नि॒ जेत्वा॑नि जयतु ते ते जयतु॒ जेत्वा॑नि । \newline
55. ज॒य॒तु॒ जेत्वा॑नि॒ जेत्वा॑नि जयतु जयतु॒ जेत्वा॑नि । \newline
56. जेत्वा॒नीति॒ जेत्वा॑नि । \newline
57. दि॒वः पृ॑थि॒व्याः पृ॑थि॒व्या दि॒वो दि॒वः पृ॑थि॒व्याः परि॒ परि॑ पृथि॒व्या दि॒वो दि॒वः पृ॑थि॒व्याः परि॑ । \newline
58. पृ॒थि॒व्याः परि॒ परि॑ पृथि॒व्याः पृ॑थि॒व्याः पर्योज॒ ओजः॒ परि॑ पृथि॒व्याः पृ॑थि॒व्याः पर्योजः॑ । \newline
59. पर्योज॒ ओजः॒ परि॒ पर्योज॒ उद्भृ॑त॒ मुद्भृ॑त॒ मोजः॒ परि॒ पर्योज॒ उद्भृ॑तम् । \newline
\pagebreak
\markright{ TS 4.6.6.6  \hfill https://www.vedavms.in \hfill}

\section{ TS 4.6.6.6 }

\textbf{TS 4.6.6.6 } \newline
\textbf{Samhita Paata} \newline

-ज॒ उद्-भृ॑तं॒ ॅवन॒स्पति॑भ्यः॒ पर्याभृ॑तꣳ॒॒ सहः॑ । अ॒पामो॒ज्मानं॒ परि॒ गोभि॒रावृ॑त॒मिन्द्र॑स्य॒ वज्रꣳ॑ ह॒विषा॒ रथं॑ ॅयज ॥ इन्द्र॑स्य॒ वज्रो॑ म॒रुता॒मनी॑कं मि॒त्रस्य॒ गर्भो॒ वरु॑णस्य॒ नाभिः॑ । सेमां नो॑ ह॒व्यदा॑तिं जुषा॒णो देव॑ रथ॒ प्रति॑ ह॒व्या गृ॑भाय ॥ उप॑ श्वासय पृथि॒वीमु॒त द्यां पु॑रु॒त्रा ते॑ मनुतां॒ ॅविष्ठि॑तं॒ जग॑त् । स दु॑न्दुभे स॒जूरिन्द्रे॑ण दे॒वैर्दू॒रा- [  ] \newline

\textbf{Pada Paata} \newline

ओजः॑ । उद्भृ॑त॒मित्युत् - भृ॒त॒म् । वन॒स्पति॑भ्य॒ इति॒ वन॒स्पति॑-भ्यः॒ । परीति॑ । आभृ॑त॒मित्या - भृ॒त॒म् । सहः॑ ॥ अ॒पाम् । ओ॒ज्मान᳚म् । परीति॑ । गोभिः॑ । आवृ॑त॒मित्या-वृ॒त॒म् । इन्द्र॑स्य । वज्र᳚म् । ह॒विषा᳚ । रथ᳚म् । य॒ज॒ ॥ इन्द्र॑स्य । वज्रः॑ । म॒रुता᳚म् । अनी॑कम् । मि॒त्रस्य॑ । गर्भः॑ । वरु॑णस्य । नाभिः॑ ॥ सः । इ॒माम् । नः॒ । ह॒व्यदा॑ति॒मिति॑ ह॒व्य - दा॒ति॒म् । जु॒षा॒णः । देव॑ । र॒थ॒ । प्रतीति॑ । ह॒व्या । गृ॒भा॒य॒ ॥ उपेति॑ । श्वा॒स॒य॒ । पृ॒थि॒वीम् । उ॒त । द्याम् । पु॒रु॒त्रेति॑ पुरु - त्रा । ते॒ । म॒नु॒ता॒म् । विष्ठि॑त॒मिति॒ वि - स्थि॒त॒म् । जग॑त् ॥ सः । दु॒न्दु॒भे॒ । स॒जूरिति॑ स - जूः । इन्द्रे॑ण । दे॒वैः । दू॒रात् ।  \newline


\textbf{Krama Paata} \newline

ओज॒ उद्भृ॑तम् । उद्भृ॑तं॒ ॅवन॒स्पति॑भ्यः । उद्भृ॑त॒मित्युत् - भृ॒त॒म् । वन॒स्पति॑भ्यः॒ परि॑ । वन॒स्पति॑भ्य॒ इति॒ वन॒स्पति॑ - भ्यः॒ । पर्याभृ॑तम् । आभृ॑तꣳ॒॒ सहः॑ । आभृ॑त॒मित्या - भृ॒त॒म् । सह॒ इति॒ सहः॑ ॥ अ॒पामो॒ज्मान᳚म् । ओ॒ज्मान॒म् परि॑ । परि॒ गोभिः॑ । गोभि॒रावृ॑तम् । आवृ॑त॒मिन्द्र॑स्य । आवृ॑त॒मित्या - वृ॒त॒म् । इन्द्र॑स्य॒ वज्र᳚म् । वज्रꣳ॑ ह॒विषा᳚ । ह॒विषा॒ रथ᳚म् । रथं॑ ॅयज । य॒जेति॑ यज ॥ इन्द्र॑स्य॒ वज्रः॑ । वज्रो॑ म॒रुता᳚म् । म॒रुता॒मनी॑कम् । अनी॑कम् मि॒त्रस्य॑ । मि॒त्रस्य॒ गर्भः॑ । गर्भो॒ वरु॑णस्य । वरु॑णस्य॒ नाभिः॑ । नाभि॒रिति॒ नाभिः॑ ॥ सेमाम् । इ॒माम् नः॑ । नो॒ ह॒व्यदा॑तिम् । ह॒व्यदा॑तिम् जुषा॒णः । ह॒व्यदा॑ति॒मिति॑ ह॒व्य - दा॒ति॒म् । जु॒षा॒णो देव॑ । देव॑ रथ । र॒थ॒ प्रति॑ । प्रति॑ ह॒व्या । ह॒व्या गृ॑भाय । गृ॒भा॒येति॑ गृभाय ॥ उप॑ श्वासय । श्वा॒स॒य॒ पृ॒थि॒वीम् । पृ॒थि॒वीमु॒त । उ॒त द्याम् । द्याम् पु॑रु॒त्रा । पु॒रु॒त्रा ते᳚ । पु॒रु॒त्रेति॑ पुरु - त्रा । ते॒ म॒नु॒ता॒म् । म॒नु॒तां॒ ॅविष्ठि॑तम् । विष्ठि॑त॒म् जग॑त् । विष्ठि॑त॒मिति॒ वि - स्थि॒त॒म् । जग॒दिति॒ जग॑त् ॥ स दु॑न्दुभे । दु॒न्दु॒भे॒ स॒जूः । स॒जूरिन्द्रे॑ण । स॒जूरिति॑ स - जूः । इन्द्रे॑ण दे॒वैः । दे॒वैर् दू॒रात् ( ) । दू॒राद् दवी॑यः \newline

\textbf{Jatai Paata} \newline

1. ओज॒ उद्‍भृ॑त॒ मुद्‍भृ॑त॒ मोज॒ ओज॒ उद्‍भृ॑तम् । \newline
2. उद्‍भृ॑तं॒ ॅवन॒स्पति॑भ्यो॒ वन॒स्पति॑भ्य॒ उद्‍भृ॑त॒ मुद्‍भृ॑तं॒ ॅवन॒स्पति॑भ्यः । \newline
3. उद्‍भृ॑त॒मित्युत् - भृ॒त॒म् । \newline
4. वन॒स्पति॑भ्यः॒ परि॒ परि॒ वन॒स्पति॑भ्यो॒ वन॒स्पति॑भ्यः॒ परि॑ । \newline
5. वन॒स्पति॑भ्य॒ इति॒ वन॒स्पति॑ - भ्यः॒ । \newline
6. पर्याभृ॑त॒ माभृ॑त॒म् परि॒ पर्याभृ॑तम् । \newline
7. आभृ॑तꣳ॒॒ सहः॒ सह॒ आभृ॑त॒ माभृ॑तꣳ॒॒ सहः॑ । \newline
8. आभृ॑त॒मित्या - भृ॒त॒म् । \newline
9. सह॒ इति॒ सहः॑ । \newline
10. अ॒पा मो॒ज्मान॑ मो॒ज्मान॑ म॒पा म॒पा मो॒ज्मान᳚म् । \newline
11. ओ॒ज्मान॒म् परि॒ पर्यो॒ज्मान॑ मो॒ज्मान॒म् परि॑ । \newline
12. परि॒ गोभि॒र् गोभिः॒ परि॒ परि॒ गोभिः॑ । \newline
13. गोभि॒ रावृ॑त॒ मावृ॑त॒म् गोभि॒र् गोभि॒ रावृ॑तम् । \newline
14. आवृ॑त॒ मिन्द्र॒ स्येन्द्र॒ स्यावृ॑त॒ मावृ॑त॒ मिन्द्र॑स्य । \newline
15. आवृ॑त॒मित्या - वृ॒त॒म् । \newline
16. इन्द्र॑स्य॒ वज्रं॒ ॅवज्र॒ मिन्द्र॒ स्येन्द्र॑स्य॒ वज्र᳚म् । \newline
17. वज्रꣳ॑ ह॒विषा॑ ह॒विषा॒ वज्रं॒ ॅवज्रꣳ॑ ह॒विषा᳚ । \newline
18. ह॒विषा॒ रथꣳ॒॒ रथꣳ॑ ह॒विषा॑ ह॒विषा॒ रथ᳚म् । \newline
19. रथं॑ ॅयज यज॒ रथꣳ॒॒ रथं॑ ॅयज । \newline
20. य॒जेति॑ यज । \newline
21. इन्द्र॑स्य॒ वज्रो॒ वज्र॒ इन्द्र॒ स्येन्द्र॑स्य॒ वज्रः॑ । \newline
22. वज्रो॑ म॒रुता᳚म् म॒रुतां॒ ॅवज्रो॒ वज्रो॑ म॒रुता᳚म् । \newline
23. म॒रुता॒ मनी॑क॒ मनी॑कम् म॒रुता᳚म् म॒रुता॒ मनी॑कम् । \newline
24. अनी॑कम् मि॒त्रस्य॑ मि॒त्रस्या नी॑क॒ मनी॑कम् मि॒त्रस्य॑ । \newline
25. मि॒त्रस्य॒ गर्भो॒ गर्भो॑ मि॒त्रस्य॑ मि॒त्रस्य॒ गर्भः॑ । \newline
26. गर्भो॒ वरु॑णस्य॒ वरु॑णस्य॒ गर्भो॒ गर्भो॒ वरु॑णस्य । \newline
27. वरु॑णस्य॒ नाभि॒र् नाभि॒र् वरु॑णस्य॒ वरु॑णस्य॒ नाभिः॑ । \newline
28. नाभि॒रिति॒ नाभिः॑ । \newline
29. सेमा मि॒माꣳ स सेमाम् । \newline
30. इ॒माम् नो॑ न इ॒मा मि॒माम् नः॑ । \newline
31. नो॒ ह॒व्यदा॑तिꣳ ह॒व्यदा॑तिम् नो नो ह॒व्यदा॑तिम् । \newline
32. ह॒व्यदा॑तिम् जुषा॒णो जु॑षा॒णो ह॒व्यदा॑तिꣳ ह॒व्यदा॑तिम् जुषा॒णः । \newline
33. ह॒व्यदा॑ति॒मिति॑ ह॒व्य - दा॒ति॒म् । \newline
34. जु॒षा॒णो देव॒ देव॑ जुषा॒णो जु॑षा॒णो देव॑ । \newline
35. देव॑ रथ रथ॒ देव॒ देव॑ रथ । \newline
36. र॒थ॒ प्रति॒ प्रति॑ रथ रथ॒ प्रति॑ । \newline
37. प्रति॑ ह॒व्या ह॒व्या प्रति॒ प्रति॑ ह॒व्या । \newline
38. ह॒व्या गृ॑भाय गृभाय ह॒व्या ह॒व्या गृ॑भाय । \newline
39. गृ॒भा॒येति॑ गृभाय । \newline
40. उप॑ श्वासय श्वास॒योपोप॑ श्वासय । \newline
41. श्वा॒स॒य॒ पृ॒थि॒वीम् पृ॑थि॒वीꣳ श्वा॑सय श्वासय पृथि॒वीम् । \newline
42. पृ॒थि॒वी मु॒तोत पृ॑थि॒वीम् पृ॑थि॒वी मु॒त । \newline
43. उ॒त द्याम् द्या मु॒तोत द्याम् । \newline
44. द्याम् पु॑रु॒त्रा पु॑रु॒त्रा द्याम् द्याम् पु॑रु॒त्रा । \newline
45. पु॒रु॒त्रा ते॑ ते पुरु॒त्रा पु॑रु॒त्रा ते᳚ । \newline
46. पु॒रु॒त्रेति॑ पुरु - त्रा । \newline
47. ते॒ म॒नु॒ता॒म् म॒नु॒ता॒म् ते॒ ते॒ म॒नु॒ता॒म् । \newline
48. म॒नु॒तां॒ ॅविष्ठि॑तं॒ ॅविष्ठि॑तम् मनुताम् मनुतां॒ ॅविष्ठि॑तम् । \newline
49. विष्ठि॑त॒म् जग॒ज् जग॒द् विष्ठि॑तं॒ ॅविष्ठि॑त॒म् जग॑त् । \newline
50. विष्ठि॑त॒मिति॒ वि - स्थि॒त॒म् । \newline
51. जग॒दिति॒ जग॑त् । \newline
52. स दु॑न्दुभे दुन्दुभे॒ स स दु॑न्दुभे । \newline
53. दु॒न्दु॒भे॒ स॒जूः स॒जूर् दु॑न्दुभे दुन्दुभे स॒जूः । \newline
54. स॒जूरिन्द्रे॒ णेन्द्रे॑ण स॒जूः स॒जूरिन्द्रे॑ण । \newline
55. स॒जूरिति॑ स - जूः । \newline
56. इन्द्रे॑ण दे॒वैर् दे॒वैरिन्द्रे॒ णेन्द्रे॑ण दे॒वैः । \newline
57. दे॒वैर् दू॒राद् दू॒राद् दे॒वैर् दे॒वैर् दू॒रात् । \newline
58. दू॒राद् दवी॑यो॒ दवी॑यो दू॒राद् दू॒राद् दवी॑यः । \newline

\textbf{Ghana Paata } \newline

1. ओज॒ उद्भृ॑त॒ मुद्भृ॑त॒ मोज॒ ओज॒ उद्भृ॑तं॒ ॅवन॒स्पति॑भ्यो॒ वन॒स्पति॑भ्य॒ उद्भृ॑त॒ मोज॒ ओज॒ उद्भृ॑तं॒ ॅवन॒स्पति॑भ्यः । \newline
2. उद्भृ॑तं॒ ॅवन॒स्पति॑भ्यो॒ वन॒स्पति॑भ्य॒ उद्भृ॑त॒ मुद्भृ॑तं॒ ॅवन॒स्पति॑भ्यः॒ परि॒ परि॒ वन॒स्पति॑भ्य॒ उद्भृ॑त॒ मुद्भृ॑तं॒ ॅवन॒स्पति॑भ्यः॒ परि॑ । \newline
3. उद्भृ॑त॒मित्युत् - भृ॒त॒म् । \newline
4. वन॒स्पति॑भ्यः॒ परि॒ परि॒ वन॒स्पति॑भ्यो॒ वन॒स्पति॑भ्यः॒ पर्याभृ॑त॒ माभृ॑त॒म् परि॒ वन॒स्पति॑भ्यो॒ वन॒स्पति॑भ्यः॒ पर्याभृ॑तम् । \newline
5. वन॒स्पति॑भ्य॒ इति॒ वन॒स्पति॑ - भ्यः॒ । \newline
6. पर्याभृ॑त॒ माभृ॑त॒म् परि॒ पर्याभृ॑तꣳ॒॒ सहः॒ सह॒ आभृ॑त॒म् परि॒ पर्याभृ॑तꣳ॒॒ सहः॑ । \newline
7. आभृ॑तꣳ॒॒ सहः॒ सह॒ आभृ॑त॒ माभृ॑तꣳ॒॒ सहः॑ । \newline
8. आभृ॑त॒मित्या - भृ॒त॒म् । \newline
9. सह॒ इति॒ सहः॑ । \newline
10. अ॒पा मो॒ज्मान॑ मो॒ज्मान॑ म॒पा म॒पा मो॒ज्मान॒म् परि॒ पर्यो॒ज्मान॑ म॒पा म॒पा मो॒ज्मान॒म् परि॑ । \newline
11. ओ॒ज्मान॒म् परि॒ पर्यो॒ज्मान॑ मो॒ज्मान॒म् परि॒ गोभि॒र् गोभिः॒ पर्यो॒ज्मान॑ मो॒ज्मान॒म् परि॒ गोभिः॑ । \newline
12. परि॒ गोभि॒र् गोभिः॒ परि॒ परि॒ गोभि॒ रावृ॑त॒ मावृ॑त॒म् गोभिः॒ परि॒ परि॒ गोभि॒ रावृ॑तम् । \newline
13. गोभि॒ रावृ॑त॒ मावृ॑त॒म् गोभि॒र् गोभि॒ रावृ॑त॒ मिन्द्र॒ स्येन्द्र॒स्या वृ॑त॒म् गोभि॒र् गोभि॒ रावृ॑त॒ मिन्द्र॑स्य । \newline
14. आवृ॑त॒ मिन्द्र॒ स्येन्द्र॒स्या वृ॑त॒ मावृ॑त॒ मिन्द्र॑स्य॒ वज्रं॒ ॅवज्र॒ मिन्द्र॒स्या वृ॑त॒ मावृ॑त॒ मिन्द्र॑स्य॒ वज्र᳚म् । \newline
15. आवृ॑त॒मित्या - वृ॒त॒म् । \newline
16. इन्द्र॑स्य॒ वज्रं॒ ॅवज्र॒ मिन्द्र॒ स्येन्द्र॑स्य॒ वज्रꣳ॑ ह॒विषा॑ ह॒विषा॒ वज्र॒ मिन्द्र॒ स्येन्द्र॑स्य॒ वज्रꣳ॑ ह॒विषा᳚ । \newline
17. वज्रꣳ॑ ह॒विषा॑ ह॒विषा॒ वज्रं॒ ॅवज्रꣳ॑ ह॒विषा॒ रथꣳ॒॒ रथꣳ॑ ह॒विषा॒ वज्रं॒ ॅवज्रꣳ॑ ह॒विषा॒ रथ᳚म् । \newline
18. ह॒विषा॒ रथꣳ॒॒ रथꣳ॑ ह॒विषा॑ ह॒विषा॒ रथं॑ ॅयज यज॒ रथꣳ॑ ह॒विषा॑ ह॒विषा॒ रथं॑ ॅयज । \newline
19. रथं॑ ॅयज यज॒ रथꣳ॒॒ रथं॑ ॅयज । \newline
20. य॒जेति॑ यज । \newline
21. इन्द्र॑स्य॒ वज्रो॒ वज्र॒ इन्द्र॒ स्येन्द्र॑स्य॒ वज्रो॑ म॒रुता᳚म् म॒रुतां॒ ॅवज्र॒ इन्द्र॒ स्येन्द्र॑स्य॒ वज्रो॑ म॒रुता᳚म् । \newline
22. वज्रो॑ म॒रुता᳚म् म॒रुतां॒ ॅवज्रो॒ वज्रो॑ म॒रुता॒ मनी॑क॒ मनी॑कम् म॒रुतां॒ ॅवज्रो॒ वज्रो॑ म॒रुता॒ मनी॑कम् । \newline
23. म॒रुता॒ मनी॑क॒ मनी॑कम् म॒रुता᳚म् म॒रुता॒ मनी॑कम् मि॒त्रस्य॑ मि॒त्रस्या नी॑कम् म॒रुता᳚म् म॒रुता॒ मनी॑कम् मि॒त्रस्य॑ । \newline
24. अनी॑कम् मि॒त्रस्य॑ मि॒त्रस्या नी॑क॒ मनी॑कम् मि॒त्रस्य॒ गर्भो॒ गर्भो॑ मि॒त्रस्या नी॑क॒ मनी॑कम् मि॒त्रस्य॒ गर्भः॑ । \newline
25. मि॒त्रस्य॒ गर्भो॒ गर्भो॑ मि॒त्रस्य॑ मि॒त्रस्य॒ गर्भो॒ वरु॑णस्य॒ वरु॑णस्य॒ गर्भो॑ मि॒त्रस्य॑ मि॒त्रस्य॒ गर्भो॒ वरु॑णस्य । \newline
26. गर्भो॒ वरु॑णस्य॒ वरु॑णस्य॒ गर्भो॒ गर्भो॒ वरु॑णस्य॒ नाभि॒र् नाभि॒र् वरु॑णस्य॒ गर्भो॒ गर्भो॒ वरु॑णस्य॒ नाभिः॑ । \newline
27. वरु॑णस्य॒ नाभि॒र् नाभि॒र् वरु॑णस्य॒ वरु॑णस्य॒ नाभिः॑ । \newline
28. नाभि॒रिति॒ नाभिः॑ । \newline
29. सेमा मि॒माꣳ स सेमाम् नो॑ न इ॒माꣳ स सेमाम् नः॑ । \newline
30. इ॒माम् नो॑ न इ॒मा मि॒माम् नो॑ ह॒व्यदा॑तिꣳ ह॒व्यदा॑तिम् न इ॒मा मि॒माम् नो॑ ह॒व्यदा॑तिम् । \newline
31. नो॒ ह॒व्यदा॑तिꣳ ह॒व्यदा॑तिम् नो नो ह॒व्यदा॑तिम् जुषा॒णो जु॑षा॒णो ह॒व्यदा॑तिम् नो नो ह॒व्यदा॑तिम् जुषा॒णः । \newline
32. ह॒व्यदा॑तिम् जुषा॒णो जु॑षा॒णो ह॒व्यदा॑तिꣳ ह॒व्यदा॑तिम् जुषा॒णो देव॒ देव॑ जुषा॒णो ह॒व्यदा॑तिꣳ ह॒व्यदा॑तिम् जुषा॒णो देव॑ । \newline
33. ह॒व्यदा॑ति॒मिति॑ ह॒व्य - दा॒ति॒म् । \newline
34. जु॒षा॒णो देव॒ देव॑ जुषा॒णो जु॑षा॒णो देव॑ रथ रथ॒ देव॑ जुषा॒णो जु॑षा॒णो देव॑ रथ । \newline
35. देव॑ रथ रथ॒ देव॒ देव॑ रथ॒ प्रति॒ प्रति॑ रथ॒ देव॒ देव॑ रथ॒ प्रति॑ । \newline
36. र॒थ॒ प्रति॒ प्रति॑ रथ रथ॒ प्रति॑ ह॒व्या ह॒व्या प्रति॑ रथ रथ॒ प्रति॑ ह॒व्या । \newline
37. प्रति॑ ह॒व्या ह॒व्या प्रति॒ प्रति॑ ह॒व्या गृ॑भाय गृभाय ह॒व्या प्रति॒ प्रति॑ ह॒व्या गृ॑भाय । \newline
38. ह॒व्या गृ॑भाय गृभाय ह॒व्या ह॒व्या गृ॑भाय । \newline
39. गृ॒भा॒येति॑ गृभाय । \newline
40. उप॑ श्वासय श्वास॒यो पोप॑ श्वासय पृथि॒वीम् पृ॑थि॒वीꣳ श्वा॑स॒यो पोप॑ श्वासय पृथि॒वीम् । \newline
41. श्वा॒स॒य॒ पृ॒थि॒वीम् पृ॑थि॒वीꣳ श्वा॑सय श्वासय पृथि॒वी मु॒तोत पृ॑थि॒वीꣳ श्वा॑सय श्वासय पृथि॒वी मु॒त । \newline
42. पृ॒थि॒वी मु॒तोत पृ॑थि॒वीम् पृ॑थि॒वी मु॒त द्याम् द्या मु॒त पृ॑थि॒वीम् पृ॑थि॒वी मु॒त द्याम् । \newline
43. उ॒त द्याम् द्या मु॒तोत द्याम् पु॑रु॒त्रा पु॑रु॒त्रा द्या मु॒तोत द्याम् पु॑रु॒त्रा । \newline
44. द्याम् पु॑रु॒त्रा पु॑रु॒त्रा द्याम् द्याम् पु॑रु॒त्रा ते॑ ते पुरु॒त्रा द्याम् द्याम् पु॑रु॒त्रा ते᳚ । \newline
45. पु॒रु॒त्रा ते॑ ते पुरु॒त्रा पु॑रु॒त्रा ते॑ मनुताम् मनुताम् ते पुरु॒त्रा पु॑रु॒त्रा ते॑ मनुताम् । \newline
46. पु॒रु॒त्रेति॑ पुरु - त्रा । \newline
47. ते॒ म॒नु॒ता॒म् म॒नु॒ता॒म् ते॒ ते॒ म॒नु॒तां॒ ॅविष्ठि॑तं॒ ॅविष्ठि॑तम् मनुताम् ते ते मनुतां॒ ॅविष्ठि॑तम् । \newline
48. म॒नु॒तां॒ ॅविष्ठि॑तं॒ ॅविष्ठि॑तम् मनुताम् मनुतां॒ ॅविष्ठि॑त॒म् जग॒ज् जग॒द् विष्ठि॑तम् मनुताम् मनुतां॒ ॅविष्ठि॑त॒म् जग॑त् । \newline
49. विष्ठि॑त॒म् जग॒ज् जग॒द् विष्ठि॑तं॒ ॅविष्ठि॑त॒म् जग॑त् । \newline
50. विष्ठि॑त॒मिति॒ वि - स्थि॒त॒म् । \newline
51. जग॒दिति॒ जग॑त् । \newline
52. स दु॑न्दुभे दुन्दुभे॒ स स दु॑न्दुभे स॒जूः स॒जूर् दु॑न्दुभे॒ स स दु॑न्दुभे स॒जूः । \newline
53. दु॒न्दु॒भे॒ स॒जूः स॒जूर् दु॑न्दुभे दुन्दुभे स॒जू रिन्द्रे॒ णेन्द्रे॑ण स॒जूर् दु॑न्दुभे दुन्दुभे स॒जू रिन्द्रे॑ण । \newline
54. स॒जू रिन्द्रे॒ णेन्द्रे॑ण स॒जूः स॒जू रिन्द्रे॑ण दे॒वैर् दे॒वै रिन्द्रे॑ण स॒जूः स॒जू रिन्द्रे॑ण दे॒वैः । \newline
55. स॒जूरिति॑ स - जूः । \newline
56. इन्द्रे॑ण दे॒वैर् दे॒वै रिन्द्रे॒ णेन्द्रे॑ण दे॒वैर् दू॒राद् दू॒राद् दे॒वै रिन्द्रे॒ णेन्द्रे॑ण दे॒वैर् दू॒रात् । \newline
57. दे॒वैर् दू॒राद् दू॒राद् दे॒वैर् दे॒वैर् दू॒राद् दवी॑यो॒ दवी॑यो दू॒राद् दे॒वैर् दे॒वैर् दू॒राद् दवी॑यः । \newline
58. दू॒राद् दवी॑यो॒ दवी॑यो दू॒राद् दू॒राद् दवी॑यो॒ अपाप॒ दवी॑यो दू॒राद् दू॒राद् दवी॑यो॒ अप॑ । \newline
\pagebreak
\markright{ TS 4.6.6.7  \hfill https://www.vedavms.in \hfill}

\section{ TS 4.6.6.7 }

\textbf{TS 4.6.6.7 } \newline
\textbf{Samhita Paata} \newline

द्दवी॑यो॒ अप॑सेध॒ शत्रून्॑ ॥ आ क्र॑न्दय॒ बल॒मोजो॑ न॒ आ धा॒ निष्ट॑निहि दुरि॒ता बाध॑मानः । अप॑ प्रोथ दुन्दुभे दु॒च्छुनाꣳ॑ इ॒त इन्द्र॑स्य मु॒ष्टिर॑सि वी॒डय॑स्व ॥आऽमूर॑ज प्र॒त्याव॑र्तये॒माः के॑तु॒मद् दु॑न्दु॒भि र्वा॑वदीति । समश्व॑पर्णा॒श्चर॑न्ति नो॒ नरो॒ऽस्माक॑मिन्द्र र॒थिनो॑ जयन्तु ॥ \newline

\textbf{Pada Paata} \newline

दवी॑यः । अपेति॑ । से॒ध॒ । शत्रून्॑ ॥ एति॑ । क्र॒न्द॒य॒ । बल᳚म् । ओजः॑ । नः॒ । एति॑ । धाः॒ । निरिति॑ । स्थ॒नि॒हि॒ । दु॒रि॒तेति॑ दुः - इ॒ता । बाध॑मानः ॥ अपेति॑ । प्रो॒थ॒ । दु॒न्दु॒भे॒ । दु॒च्छुनान्॑ । इ॒तः । इन्द्र॑स्य । मु॒ष्टिः । अ॒सि॒ । वी॒डय॑स्व ॥ एति॑ । अ॒मूः । अ॒ज॒ । प्र॒त्याव॑र्त॒येति॑ प्रति - आव॑र्तय । इ॒माः । के॒तु॒मदिति॑ केतु - मत् । दु॒न्दु॒भिः । वा॒व॒दी॒ति॒ ॥ समिति॑ । अश्व॑पर्णा॒ इत्यश्व॑-प॒र्णाः॒ । चर॑न्ति । नः॒ । नरः॑ । अ॒स्माक᳚म् । इ॒न्द्र॒ । र॒थिनः॑ । ज॒य॒न्तु॒ ॥  \newline


\textbf{Krama Paata} \newline

दवी॑यो॒ अप॑ । अप॑ सेध । से॒ध॒ शत्रून्॑ । शत्रू॒निति॒ शत्रून्॑ ॥ आ क्र॑न्दय । क्र॒न्द॒य॒ बल᳚म् । बल॒मोजः॑ । ओजो॑ नः । न॒ आ । आ धाः᳚ । धा॒ निः । निष्ट॑निहि । स्त॒नि॒हि॒ दु॒रि॒ता । दु॒रि॒ता बाध॑मानः । दु॒रि॒तेति॑ दुः - इ॒ता । बाध॑मान॒ इति॒ बाध॑मानः ॥ अप॑ प्रोथ । प्रो॒थ॒ दु॒न्दु॒भे॒ । दु॒न्दु॒भे॒ दु॒च्छुनान्॑ । दु॒च्छुनाꣳ॑ इ॒तः । इ॒त इन्द्र॑स्य । इन्द्र॑स्य मु॒ष्टिः । मु॒ष्टिर॑सि । अ॒सि॒ वी॒डय॑स्व । वी॒डय॒स्वेति॑ वी॒डय॑स्व ॥ आऽमूः । अ॒मूर॑ज । अ॒ज॒ प्र॒त्याव॑र्तय । प्र॒त्याव॑र्तये॒माः । प्र॒त्याव॑र्त॒येति॑ प्रति - आव॑र्तय । इ॒माः के॑तु॒मत् । के॒तु॒मद् दु॑न्दु॒भिः । के॒तु॒मदिति॑ केतु - मत् । दु॒न्दु॒भिर् वा॑वदीति । वा॒व॒दी॒तीति॑ वावदीति ॥ समश्व॑पर्णाः । अश्व॑पर्णा॒श्चर॑न्ति । अश्व॑पर्णा॒ इत्यश्व॑ - प॒र्णाः॒ । चर॑न्ति नः । नो॒ नरः॑ । नरो॒ऽस्माक᳚म् । अ॒स्माक॑मिन्द्र । इ॒न्द्र॒ र॒थिनः॑ । र॒थिनो॑ जयन्तु । 
ज॒य॒न्त्विति॑ जयन्तु । \newline

\textbf{Jatai Paata} \newline

1. दवी॑यो॒ अपाप॒ दवी॑यो॒ दवी॑यो॒ अप॑ । \newline
2. अप॑ सेध से॒धापाप॑ सेध । \newline
3. से॒ध॒ शत्रू॒ञ् छत्रून्᳚ थ्सेध सेध॒ शत्रून्॑ । \newline
4. शत्रू॒निति॒ शत्रून्॑ । \newline
5. आ क्र॑न्दय क्रन्द॒या क्र॑न्दय । \newline
6. क्र॒न्द॒य॒ बल॒म् बल॑म् क्रन्दय क्रन्दय॒ बल᳚म् । \newline
7. बल॒ मोज॒ ओजो॒ बल॒म् बल॒ मोजः॑ । \newline
8. ओजो॑ नो न॒ ओज॒ ओजो॑ नः । \newline
9. न॒ आ नो॑ न॒ आ । \newline
10. आ धा॑ धा॒ आ धाः᳚ । \newline
11. धा॒ निर् णिर् धा॑ धा॒ निः । \newline
12. निष्ट॑निहि स्थनिहि॒ निर् णिष्ट॑निहि । \newline
13. स्थ॒नि॒हि॒ दु॒रि॒ता दु॑रि॒ता स्थ॑निहि स्थनिहि दुरि॒ता । \newline
14. दु॒रि॒ता बाध॑मानो॒ बाध॑मानो दुरि॒ता दु॑रि॒ता बाध॑मानः । \newline
15. दु॒रि॒तेति॑ दुः - इ॒ता । \newline
16. बाध॑मान॒ इति॒ बाध॑मानः । \newline
17. अप॑ प्रोथ प्रो॒था पाप॑ प्रोथ । \newline
18. प्रो॒थ॒ दु॒न्दु॒भे॒ दु॒न्दु॒भे॒ प्रो॒थ॒ प्रो॒थ॒ दु॒न्दु॒भे॒ । \newline
19. दु॒न्दु॒भे॒ दु॒च्छुना᳚न् दु॒च्छुना᳚न् दुन्दुभे दुन्दुभे दु॒च्छुनान्॑ । \newline
20. दु॒च्छुनाꣳ॑ इ॒त इ॒तो दु॒च्छुना᳚न् दु॒च्छुनाꣳ॑ इ॒तः । \newline
21. इ॒त इन्द्र॒ स्येन्द्र॑स्ये॒ त इ॒त इन्द्र॑स्य । \newline
22. इन्द्र॑स्य मु॒ष्टिर् मु॒ष्टि रिन्द्र॒ स्येन्द्र॑स्य मु॒ष्टिः । \newline
23. मु॒ष्टि र॑स्यसि मु॒ष्टिर् मु॒ष्टि र॑सि । \newline
24. अ॒सि॒ वी॒डय॑स्व वी॒डय॑स्वा स्यसि वी॒डय॑स्व । \newline
25. वी॒डय॒स्वेति॑ वी॒डय॑स्व । \newline
26. आ ऽमू र॒मूरा ऽमूः । \newline
27. अ॒मू र॑जा जा॒ मू र॒मू र॑ज । \newline
28. अ॒ज॒ प्र॒त्याव॑र्तय प्र॒त्याव॑र्तया जाज प्र॒त्याव॑र्तय । \newline
29. प्र॒त्याव॑र्तये॒ मा इ॒माः प्र॒त्याव॑र्तय प्र॒त्याव॑र्तये॒ माः । \newline
30. प्र॒त्याव॑र्त॒येति॑ प्रति - आव॑र्तय । \newline
31. इ॒माः के॑तु॒मत् के॑तु॒ मदि॒मा इ॒माः के॑तु॒मत् । \newline
32. के॒तु॒मद् दु॑न्दु॒भिर् दु॑न्दु॒भिः के॑तु॒मत् के॑तु॒मद् दु॑न्दु॒भिः । \newline
33. के॒तु॒मदिति॑ केतु - मत् । \newline
34. दु॒न्दु॒भिर् वा॑वदीति वावदीति दुन्दु॒भिर् दु॑न्दु॒भिर् वा॑वदीति । \newline
35. वा॒व॒दि॒तीति॑ वावदीति । \newline
36. स मश्व॑पर्णा॒ अश्व॑पर्णाः॒ सꣳ स मश्व॑पर्णाः । \newline
37. अश्व॑पर्णा॒ श्चर॑न्ति॒ चर॒ न्त्यश्व॑पर्णा॒ अश्व॑पर्णा॒ श्चर॑न्ति । \newline
38. अश्व॑पर्णा॒ इत्यश्व॑ - प॒र्णाः॒ । \newline
39. चर॑न्ति नो न॒ श्चर॑न्ति॒ चर॑न्ति नः । \newline
40. नो॒ नरो॒ नरो॑ नो नो॒ नरः॑ । \newline
41. नरो॒ ऽस्माक॑ म॒स्माक॒म् नरो॒ नरो॒ ऽस्माक᳚म् । \newline
42. अ॒स्माक॑ मिन्द्रे न्द्रा॒स्माक॑ म॒स्माक॑ मिन्द्र । \newline
43. इ॒न्द्र॒ र॒थिनो॑ र॒थिन॑ इन्द्रे न्द्र र॒थिनः॑ । \newline
44. र॒थिनो॑ जयन्तु जयन्तु र॒थिनो॑ र॒थिनो॑ जयन्तु । \newline
45. ज॒य॒न्त्विति॑ जयन्तु । \newline

\textbf{Ghana Paata } \newline

1. दवी॑यो॒ अपाप॒ दवी॑यो॒ दवी॑यो॒ अप॑ सेध से॒धाप॒ दवी॑यो॒ दवी॑यो॒ अप॑ सेध । \newline
2. अप॑ सेध से॒धापाप॑ सेध॒ शत्रू॒ञ् छत्रून्᳚ थ्से॒धा पाप॑ सेध॒ शत्रून्॑ । \newline
3. से॒ध॒ शत्रू॒ञ् छत्रून्᳚ थ्सेध सेध॒ शत्रून्॑ । \newline
4. शत्रू॒निति॒ शत्रून्॑ । \newline
5. आ क्र॑न्दय क्रन्द॒या क्र॑न्दय॒ बल॒म् बल॑म् क्रन्द॒या क्र॑न्दय॒ बल᳚म् । \newline
6. क्र॒न्द॒य॒ बल॒म् बल॑म् क्रन्दय क्रन्दय॒ बल॒ मोज॒ ओजो॒ बल॑म् क्रन्दय क्रन्दय॒ बल॒ मोजः॑ । \newline
7. बल॒ मोज॒ ओजो॒ बल॒म् बल॒ मोजो॑ नो न॒ ओजो॒ बल॒म् बल॒ मोजो॑ नः । \newline
8. ओजो॑ नो न॒ ओज॒ ओजो॑ न॒ आ न॒ ओज॒ ओजो॑ न॒ आ । \newline
9. न॒ आ नो॑ न॒ आ धा॑ धा॒ आ नो॑ न॒ आ धाः᳚ । \newline
10. आ धा॑ धा॒ आ धा॒ निर् णिर् धा॒ आ धा॒ निः । \newline
11. धा॒ निर् णिर् धा॑ धा॒ निष्ट॑निहि स्थनिहि॒ निर् धा॑ धा॒ निष्ट॑निहि । \newline
12. निष्ट॑निहि स्थनिहि॒ निर् णिष्ट॑निहि दुरि॒ता दु॑रि॒ता स्थ॑निहि॒ निर् णिष्ट॑निहि दुरि॒ता । \newline
13. स्थ॒नि॒हि॒ दु॒रि॒ता दु॑रि॒ता स्थ॑निहि स्थनिहि दुरि॒ता बाध॑मानो॒ बाध॑मानो दुरि॒ता स्थ॑निहि स्थनिहि दुरि॒ता बाध॑मानः । \newline
14. दु॒रि॒ता बाध॑मानो॒ बाध॑मानो दुरि॒ता दु॑रि॒ता बाध॑मानः । \newline
15. दु॒रि॒तेति॑ दुः - इ॒ता । \newline
16. बाध॑मान॒ इति॒ बाध॑मानः । \newline
17. अप॑ प्रोथ प्रो॒था पाप॑ प्रोथ दुन्दुभे दुन्दुभे प्रो॒था पाप॑ प्रोथ दुन्दुभे । \newline
18. प्रो॒थ॒ दु॒न्दु॒भे॒ दु॒न्दु॒भे॒ प्रो॒थ॒ प्रो॒थ॒ दु॒न्दु॒भे॒ दु॒च्छुना᳚न् दु॒च्छुना᳚न् दुन्दुभे प्रोथ प्रोथ दुन्दुभे दु॒च्छुनान्॑ । \newline
19. दु॒न्दु॒भे॒ दु॒च्छुना᳚न् दु॒च्छुना᳚न् दुन्दुभे दुन्दुभे दु॒च्छुनाꣳ॑ इ॒त इ॒तो दु॒च्छुना᳚न् दुन्दुभे दुन्दुभे दु॒च्छुनाꣳ॑ इ॒तः । \newline
20. दु॒च्छुनाꣳ॑ इ॒त इ॒तो दु॒च्छुना᳚न् दु॒च्छुनाꣳ॑ इ॒त इन्द्र॒ स्येन्द्र॑स्ये॒तो दु॒च्छुना᳚न् दु॒च्छुना(ग्न्)॑ इ॒त इन्द्र॑स्य । \newline
21. इ॒त इन्द्र॒ स्येन्द्र॑स्ये॒त इ॒त इन्द्र॑स्य मु॒ष्टिर् मु॒ष्टि रिन्द्र॑स्ये॒त इ॒त इन्द्र॑स्य मु॒ष्टिः । \newline
22. इन्द्र॑स्य मु॒ष्टिर् मु॒ष्टि रिन्द्र॒ स्येन्द्र॑स्य मु॒ष्टि र॑स्यसि मु॒ष्टि रिन्द्र॒ स्येन्द्र॑स्य मु॒ष्टिर॑सि । \newline
23. मु॒ष्टि र॑स्यसि मु॒ष्टिर् मु॒ष्टिर॑सि वी॒डय॑स्व वी॒डय॑स्वासि मु॒ष्टिर् मु॒ष्टि र॑सि वी॒डय॑स्व । \newline
24. अ॒सि॒ वी॒डय॑स्व वी॒डय॑स्वास्यसि वी॒डय॑स्व । \newline
25. वी॒डय॒स्वेति॑ वी॒डय॑स्व । \newline
26. आ ऽमूर॒ मूरा ऽमूर॑ जाजा॒ मूरा ऽमूर॑ज । \newline
27. अ॒मूर॑ जाजा॒ मू र॒मू र॑ज प्र॒त्याव॑र्तय प्र॒त्याव॑र्तयाजा॒ मू र॒मू र॑ज प्र॒त्याव॑र्तय । \newline
28. अ॒ज॒ प्र॒त्याव॑र्तय प्र॒त्याव॑र्तया जाज प्र॒त्याव॑र्तये॒मा इ॒माः प्र॒त्याव॑र्तया जाज प्र॒त्याव॑र्तये॒माः । \newline
29. प्र॒त्याव॑र्तये॒मा इ॒माः प्र॒त्याव॑र्तय प्र॒त्याव॑र्तये॒माः के॑तु॒मत् के॑तु॒म दि॒माः प्र॒त्याव॑र्तय प्र॒त्याव॑र्तये॒माः के॑तु॒मत् । \newline
30. प्र॒त्याव॑र्त॒येति॑ प्रति - आव॑र्तय । \newline
31. इ॒माः के॑तु॒मत् के॑तु॒म दि॒मा इ॒माः के॑तु॒मद् दु॑न्दु॒भिर् दु॑न्दु॒भिः के॑तु॒म दि॒मा इ॒माः के॑तु॒मद् दु॑न्दु॒भिः । \newline
32. के॒तु॒मद् दु॑न्दु॒भिर् दु॑न्दु॒भिः के॑तु॒मत् के॑तु॒मद् दु॑न्दु॒भिर् वा॑वदीति वावदीति दुन्दु॒भिः के॑तु॒मत् के॑तु॒मद् दु॑न्दु॒भिर् वा॑वदीति । \newline
33. के॒तु॒मदिति॑ केतु - मत् । \newline
34. दु॒न्दु॒भिर् वा॑वदीति वावदीति दुन्दु॒भिर् दु॑न्दु॒भिर् वा॑वदीति । \newline
35. वा॒व॒दि॒तीति॑ वावदीति । \newline
36. स मश्व॑पर्णा॒ अश्व॑पर्णाः॒ सꣳ स मश्व॑पर्णा॒ श्चर॑न्ति॒ चर॒न् त्यश्व॑पर्णाः॒ सꣳ स मश्व॑पर्णा॒ श्चर॑न्ति । \newline
37. अश्व॑पर्णा॒ श्चर॑न्ति॒ चर॒न् त्यश्व॑पर्णा॒ अश्व॑पर्णा॒ श्चर॑न्ति नो न॒श्चर॒न् त्यश्व॑पर्णा॒ अश्व॑पर्णा॒ श्चर॑न्ति नः । \newline
38. अश्व॑पर्णा॒ इत्यश्व॑ - प॒र्णाः॒ । \newline
39. चर॑न्ति नो न॒ श्चर॑न्ति॒ चर॑न्ति नो॒ नरो॒ नरो॑ न॒ श्चर॑न्ति॒ चर॑न्ति नो॒ नरः॑ । \newline
40. नो॒ नरो॒ नरो॑ नो नो॒ नरो॒ ऽस्माक॑ म॒स्माक॒म् नरो॑ नो नो॒ नरो॒ ऽस्माक᳚म् । \newline
41. नरो॒ ऽस्माक॑ म॒स्माक॒म् नरो॒ नरो॒ ऽस्माक॑ मिन्द्रेन्द्रा॒ स्माक॒म् नरो॒ नरो॒ ऽस्माक॑ मिन्द्र । \newline
42. अ॒स्माक॑ मिन्द्रेन्द्रा॒ स्माक॑ म॒स्माक॑ मिन्द्र र॒थिनो॑ र॒थिन॑ इन्द्रा॒ स्माक॑ म॒स्माक॑ मिन्द्र र॒थिनः॑ । \newline
43. इ॒न्द्र॒ र॒थिनो॑ र॒थिन॑ इन्द्रेन्द्र र॒थिनो॑ जयन्तु जयन्तु र॒थिन॑ इन्द्रेन्द्र र॒थिनो॑ जयन्तु । \newline
44. र॒थिनो॑ जयन्तु जयन्तु र॒थिनो॑ र॒थिनो॑ जयन्तु । \newline
45. ज॒य॒न्त्विति॑ जयन्तु । \newline
\pagebreak
\markright{ TS 4.6.7.1  \hfill https://www.vedavms.in \hfill}

\section{ TS 4.6.7.1 }

\textbf{TS 4.6.7.1 } \newline
\textbf{Samhita Paata} \newline

यदक्र॑न्दः प्रथ॒मं जाय॑मान उ॒द्यन्थ् स॑मु॒द्रादु॒त वा॒ पुरी॑षात् । श्ये॒नस्य॑ प॒क्षा ह॑रि॒णस्य॑ बा॒हू उ॑प॒स्तुत्यं॒ महि॑ जा॒तं ते॑ अर्वन्न् ॥ य॒मेन॑ द॒त्तं त्रि॒त ए॑नमायुन॒गिन्द्र॑ एणं प्रथ॒मो अद्ध्य॑तिष्ठत् । ग॒न्ध॒र्वो अ॑स्य रश॒नाम॑-गृभ्णा॒थ् सूरा॒दश्वं॑ ॅवसवो॒ निर॑तष्ट ॥ असि॑ य॒मो अस्या॑दि॒त्यो अ॑र्व॒न्नसि॑ त्रि॒तो गुह्ये॑न व्र॒तेन॑ । असि॒ सोमे॑न स॒मया॒ विपृ॑क्त - [  ] \newline

\textbf{Pada Paata} \newline

यत् । अक्र॑न्दः । प्र॒थ॒मम् । जाय॑मानः । उ॒द्यन्नित्यु॑त्-यन्न् । स॒मु॒द्रात् । उ॒त । वा॒ । पुरी॑षात् ॥ श्ये॒नस्य॑ । प॒क्षा । ह॒रि॒णस्य॑ । बा॒हू इति॑ । उ॒प॒स्तुत्य॒मित्यु॑प - स्तुत्य᳚म् । महि॑ । जा॒तम् । ते॒ । अ॒र्व॒न्न् ॥ य॒मेन॑ । द॒त्तम् । त्रि॒तः । ए॒न॒म् । आ॒यु॒न॒क् । इन्द्रः॑ । ए॒न॒म् । प्र॒थ॒मः । अधीति॑ । अ॒ति॒ष्ठ॒त् ॥ ग॒न्ध॒र्वः । अ॒स्य॒ । र॒श॒नाम् । अ॒गृ॒भ्णा॒त् । सूरा᳚त् । अश्व᳚म् । व॒स॒वः॒ । निरिति॑ । अ॒त॒ष्ट॒ ॥ असि॑ । य॒मः । असि॑ । आ॒दि॒त्यः । अ॒र्व॒न्न् । असि॑ । त्रि॒तः । गुह्ये॑न । व्र॒तेन॑ ॥ असि॑ । सोमे॑न । स॒मया᳚ । विपृ॑क्त॒ इति॒ वि - पृ॒क्तः॒ ।  \newline


\textbf{Krama Paata} \newline

यदक्र॑न्दः । अक्र॑न्दः प्रथ॒मम् । प्र॒थ॒मम् जाय॑मानः । जाय॑मान उ॒द्यन्न् । उ॒द्यन्थ् स॑मु॒द्रात् । उ॒द्यन्नित्यु॑त् - यन्न् । स॒मु॒द्रादु॒त । उ॒त वा᳚ । वा॒ पुरी॑षात् । पुरी॑षा॒दिति॒ पुरी॑षात् ॥ श्ये॒नस्य॑ प॒क्षा । प॒क्षा ह॑रि॒णस्य॑ । ह॒रि॒णस्य॑ बा॒हू । बा॒हू उ॑प॒स्तुत्य᳚म् । बा॒हू इति॑ बा॒हू । उ॒प॒स्तुत्य॒म् महि॑ । उ॒प॒स्तुत्य॒मित्यु॑प - स्तुत्य᳚म् । महि॑ जा॒तम् । जा॒तम् ते᳚ । ते॒ अ॒र्व॒न्न्॒ । अ॒र्व॒न्नित्य॑र्वन्न् ॥ य॒मेन॑ द॒त्तम् । द॒त्तम् त्रि॒तः । त्रि॒त ए॑नम् । ए॒न॒मा॒यु॒न॒क्॒ । आ॒यु॒न॒गिन्द्रः॑ । इन्द्र॑ एणम् । ए॒न॒म् प्र॒थ॒मः । प्र॒थ॒मो अधि॑ । अद्ध्य॑तिष्ठत् । अ॒ति॒ष्ठ॒दित्य॑तिष्ठत् ॥ ग॒न्ध॒र्वो अ॑स्य । अ॒स्य॒ र॒श॒नाम् । र॒श॒नाम॑गृभ्णात् । अ॒गृ॒भ्णा॒थ् सूरा᳚त् । सूरा॒दश्व᳚म् । अश्वं॑ ॅवसवः । व॒स॒वो॒ निः । निर॑तष्ट । अ॒त॒ष्टेत्य॑तष्ट ॥ असि॑ य॒मः । य॒मो असि॑ । अस्या॑दि॒त्यः । आ॒दि॒त्यो अ॑र्वन्न् । अ॒र्व॒न्नसि॑ । असि॑ त्रि॒तः । त्रि॒तो गुह्ये॑न । गुह्ये॑न व्र॒तेन॑ । व्र॒तेनेति॑ व्र॒तेन॑ ॥ असि॒ सोमे॑न । सोमे॑न स॒मया᳚ । स॒मया॒ विपृ॑क्तः । विपृ॑क्त आ॒हुः । विपृ॑क्त॒ इति॒ वि - पृ॒क्तः॒ \newline

\textbf{Jatai Paata} \newline

1. Yअदक्र॑न्दो॒ अक्र॑न्दो॒ यद् यदक्र॑न्दः । \newline
2. अक्र॑न्दः प्रथ॒मम् प्र॑थ॒म मक्र॑न्दो॒ अक्र॑न्दः प्रथ॒मम् । \newline
3. प्र॒थ॒मम् जाय॑मानो॒ जाय॑मानः प्रथ॒मम् प्र॑थ॒मम् जाय॑मानः । \newline
4. जाय॑मान उ॒द्यन् नु॒द्यन् जाय॑मानो॒ जाय॑मान उ॒द्यन्न् । \newline
5. उ॒द्यन् थ्स॑मु॒द्राथ् स॑मु॒द्रा दु॒द्यन् नु॒द्यन् थ्स॑मु॒द्रात् । \newline
6. उ॒द्यन्नित्यु॑त् - यन्न् । \newline
7. स॒मु॒द्रा दु॒तोत स॑मु॒द्राथ् स॑मु॒द्रा दु॒त । \newline
8. उ॒त वा॑ वो॒तोत वा᳚ । \newline
9. वा॒ पुरी॑षा॒त् पुरी॑षाद् वा वा॒ पुरी॑षात् । \newline
10. पुरी॑षा॒दिति॒ पुरी॑षात् । \newline
11. श्ये॒नस्य॑ प॒क्षा प॒क्षा श्ये॒नस्य॑ श्ये॒नस्य॑ प॒क्षा । \newline
12. प॒क्षा ह॑रि॒णस्य॑ हरि॒णस्य॑ प॒क्षा प॒क्षा ह॑रि॒णस्य॑ । \newline
13. ह॒रि॒णस्य॑ बा॒हू बा॒हू ह॑रि॒णस्य॑ हरि॒णस्य॑ बा॒हू । \newline
14. बा॒हू उ॑प॒स्तुत्य॑ मुप॒स्तुत्य॑म् बा॒हू बा॒हू उ॑प॒स्तुत्य᳚म् । \newline
15. बा॒हू इति॑ बा॒हू । \newline
16. उ॒प॒स्तुत्य॒म् महि॒ मह्यु॑प॒स्तुत्य॑ मुप॒स्तुत्य॒म् महि॑ । \newline
17. उ॒प॒स्तुत्य॒मित्यु॑प - स्तुत्य᳚म् । \newline
18. महि॑ जा॒तम् जा॒तम् महि॒ महि॑ जा॒तम् । \newline
19. जा॒तम् ते॑ ते जा॒तम् जा॒तम् ते᳚ । \newline
20. ते॒ अ॒र्व॒न् न॒र्व॒न् ते॒ ते॒ अ॒र्व॒न्न् । \newline
21. अ॒र्व॒न्नित्य॑र्वन्न् । \newline
22. य॒मेन॑ द॒त्तम् द॒त्तं ॅय॒मेन॑ य॒मेन॑ द॒त्तम् । \newline
23. द॒त्तम् त्रि॒त स्त्रि॒तो द॒त्तम् द॒त्तम् त्रि॒तः । \newline
24. त्रि॒त ए॑न मेनम् त्रि॒त स्त्रि॒त ए॑नम् । \newline
25. ए॒न॒ मा॒यु॒न॒ गा॒यु॒न॒ गे॒न॒ मे॒न॒ मा॒यु॒न॒क् । \newline
26. आ॒यु॒न॒ गिन्द्र॒ इन्द्र॑ आयुन गायुन॒ गिन्द्रः॑ । \newline
27. इन्द्र॑ एण मेन॒ मिन्द्र॒ इन्द्र॑ एणम् । \newline
28. ए॒न॒म् प्र॒थ॒मः प्र॑थ॒म ए॑न मेनम् प्रथ॒मः । \newline
29. प्र॒थ॒मो अध्यधि॑ प्रथ॒मः प्र॑थ॒मो अधि॑ । \newline
30. अध्य॑तिष्ठ दतिष्ठ॒ दध्य ध्य॑तिष्ठत् । \newline
31. अ॒ति॒ष्ठ॒दित्य॑तिष्ठत् । \newline
32. ग॒न्ध॒र्वो अ॑स्यास्य गन्ध॒र्वो ग॑न्ध॒र्वो अ॑स्य । \newline
33. अ॒स्य॒ र॒श॒नाꣳ र॑श॒ना म॑स्यास्य रश॒नाम् । \newline
34. र॒श॒ना म॑गृभ्णा दगृभ्णाद् रश॒नाꣳ र॑श॒ना म॑गृभ्णात् । \newline
35. अ॒गृ॒भ्णा॒थ् सूरा॒थ् सूरा॑दगृभ्णा दगृभ्णा॒थ् सूरा᳚त् । \newline
36. सूरा॒दश्व॒ मश्वꣳ॒॒ सूरा॒थ् सूरा॒दश्व᳚म् । \newline
37. अश्वं॑ ॅवसवो वसवो॒ अश्व॒ मश्वं॑ ॅवसवः । \newline
38. व॒स॒वो॒ निर् णिर् व॑सवो वसवो॒ निः । \newline
39. निर॑तष्टा तष्ट॒ निर् णिर॑तष्ट । \newline
40. अ॒त॒ष्टेत्य॑तष्ट । \newline
41. असि॑ य॒मो य॒मो अस्यसि॑ य॒मः । \newline
42. य॒मो अस्यसि॑ य॒मो य॒मो असि॑ । \newline
43. अस्या॑दि॒त्य आ॑दि॒त्यो ऽस्य स्या॑दि॒त्यः । \newline
44. आ॒दि॒त्यो अ॑र्वन् नर्वन् नादि॒त्य आ॑दि॒त्यो अ॑र्वन्न् । \newline
45. अ॒र्व॒न् नस्य स्य॑र्वन् नर्व॒न् नसि॑ । \newline
46. असि॑ त्रि॒त स्त्रि॒तो ऽस्यसि॑ त्रि॒तः । \newline
47. त्रि॒तो गुह्ये॑न॒ गुह्ये॑न त्रि॒त स्त्रि॒तो गुह्ये॑न । \newline
48. गुह्ये॑न व्र॒तेन॑ व्र॒तेन॒ गुह्ये॑न॒ गुह्ये॑न व्र॒तेन॑ । \newline
49. व्र॒तेनेति॑ व्र॒तेन॑ । \newline
50. असि॒ सोमे॑न॒ सोमे॒ना स्यसि॒ सोमे॑न । \newline
51. सोमे॑न स॒मया॑ स॒मया॒ सोमे॑न॒ सोमे॑न स॒मया᳚ । \newline
52. स॒मया॒ विपृ॑क्तो॒ विपृ॑क्तः स॒मया॑ स॒मया॒ विपृ॑क्तः । \newline
53. विपृ॑क्त आ॒हु रा॒हुर् विपृ॑क्तो॒ विपृ॑क्त आ॒हुः । \newline
54. विपृ॑क्त॒ इति॒ वि - पृ॒क्तः॒ । \newline

\textbf{Ghana Paata } \newline

1. यदक्र॑न्दो॒ अक्र॑न्दो॒ यद् यदक्र॑न्दः प्रथ॒मम् प्र॑थ॒म मक्र॑न्दो॒ यद् यदक्र॑न्दः प्रथ॒मम् । \newline
2. अक्र॑न्दः प्रथ॒मम् प्र॑थ॒म मक्र॑न्दो॒ अक्र॑न्दः प्रथ॒मम् जाय॑मानो॒ जाय॑मानः प्रथ॒म 
मक्र॑न्दो॒ अक्र॑न्दः प्रथ॒मम् जाय॑मानः । \newline
3. प्र॒थ॒मम् जाय॑मानो॒ जाय॑मानः प्रथ॒मम् प्र॑थ॒मम् जाय॑मान उ॒द्यन् नु॒द्यन् जाय॑मानः प्रथ॒मम् प्र॑थ॒मम् जाय॑मान उ॒द्यन्न् । \newline
4. जाय॑मान उ॒द्यन् नु॒द्यन् जाय॑मानो॒ जाय॑मान उ॒द्यन् थ्स॑मु॒द्राथ् स॑मु॒द्रा दु॒द्यन् जाय॑मानो॒ जाय॑मान उ॒द्यन् थ्स॑मु॒द्रात् । \newline
5. उ॒द्यन् थ्स॑मु॒द्राथ् स॑मु॒द्रा दु॒द्यन् नु॒द्यन् थ्स॑मु॒द्रा दु॒तोत स॑मु॒द्रा दु॒द्यन् नु॒द्यन् थ्स॑मु॒द्रा दु॒त । \newline
6. उ॒द्यन्नित्यु॑त् - यन्न् । \newline
7. स॒मु॒द्रा दु॒तोत स॑मु॒द्राथ् स॑मु॒द्रा दु॒त वा॑ वो॒त स॑मु॒द्राथ् स॑मु॒द्रा दु॒त वा᳚ । \newline
8. उ॒त वा॑ वो॒तोत वा॒ पुरी॑षा॒त् पुरी॑षाद् वो॒तोत वा॒ पुरी॑षात् । \newline
9. वा॒ पुरी॑षा॒त् पुरी॑षाद् वा वा॒ पुरी॑षात् । \newline
10. पुरी॑षा॒दिति॒ पुरी॑षात् । \newline
11. श्ये॒नस्य॑ प॒क्षा प॒क्षा श्ये॒नस्य॑ श्ये॒नस्य॑ प॒क्षा ह॑रि॒णस्य॑ हरि॒णस्य॑ प॒क्षा श्ये॒नस्य॑ श्ये॒नस्य॑ प॒क्षा ह॑रि॒णस्य॑ । \newline
12. प॒क्षा ह॑रि॒णस्य॑ हरि॒णस्य॑ प॒क्षा प॒क्षा ह॑रि॒णस्य॑ बा॒हू बा॒हू ह॑रि॒णस्य॑ प॒क्षा प॒क्षा ह॑रि॒णस्य॑ बा॒हू । \newline
13. ह॒रि॒णस्य॑ बा॒हू बा॒हू ह॑रि॒णस्य॑ हरि॒णस्य॑ बा॒हू उ॑प॒स्तुत्य॑ मुप॒स्तुत्य॑म् बा॒हू ह॑रि॒णस्य॑ हरि॒णस्य॑ बा॒हू उ॑प॒स्तुत्य᳚म् । \newline
14. बा॒हू उ॑प॒स्तुत्य॑ मुप॒स्तुत्य॑म् बा॒हू बा॒हू उ॑प॒स्तुत्य॒म् महि॒ मह्यु॑प॒स्तुत्य॑म् बा॒हू बा॒हू उ॑प॒स्तुत्य॒म् महि॑ । \newline
15. बा॒हू इति॑ बा॒हू । \newline
16. उ॒प॒स्तुत्य॒म् महि॒ मह्यु॑ प॒स्तुत्य॑ मुप॒स्तुत्य॒म् महि॑ जा॒तम् जा॒तम् मह्यु॑ प॒स्तुत्य॑ मुप॒स्तुत्य॒म् महि॑ जा॒तम् । \newline
17. उ॒प॒स्तुत्य॒मित्यु॑प - स्तुत्य᳚म् । \newline
18. महि॑ जा॒तम् जा॒तम् महि॒ महि॑ जा॒तम् ते॑ ते जा॒तम् महि॒ महि॑ जा॒तम् ते᳚ । \newline
19. जा॒तम् ते॑ ते जा॒तम् जा॒तम् ते॑ अर्वन् नर्वन् ते जा॒तम् जा॒तम् ते॑ अर्वन्न् । \newline
20. ते॒ अ॒र्व॒न् न॒र्व॒न् ते॒ ते॒ अ॒र्व॒न्न् । \newline
21. अ॒र्व॒न्नित्य॑र्वन्न् । \newline
22. य॒मेन॑ द॒त्तम् द॒त्तं ॅय॒मेन॑ य॒मेन॑ द॒त्तम् त्रि॒त स्त्रि॒तो द॒त्तं ॅय॒मेन॑ य॒मेन॑ द॒त्तम् त्रि॒तः । \newline
23. द॒त्तम् त्रि॒त स्त्रि॒तो द॒त्तम् द॒त्तम् त्रि॒त ए॑न मेनम् त्रि॒तो द॒त्तम् द॒त्तम् त्रि॒त ए॑नम् । \newline
24. त्रि॒त ए॑न मेनम् त्रि॒त स्त्रि॒त ए॑न मायुन गायुन गेनम् त्रि॒त स्त्रि॒त ए॑न मायुनक् । \newline
25. ए॒न॒ मा॒यु॒न॒ गा॒यु॒न॒ गे॒न॒ मे॒न॒ मा॒यु॒न॒ गिन्द्र॒ इन्द्र॑ आयुनगेन मेन मायुन॒ गिन्द्रः॑ । \newline
26. आ॒यु॒न॒ गिन्द्र॒ इन्द्र॑ आयुन गायुन॒ गिन्द्र॑ एण मेन॒ मिन्द्र॑ आयुन गायुन॒ गिन्द्र॑ एणम् । \newline
27. इन्द्र॑ एण मेन॒ मिन्द्र॒ इन्द्र॑ एणम् प्रथ॒मः प्र॑थ॒म ए॑न॒ मिन्द्र॒ इन्द्र॑ एणम् प्रथ॒मः । \newline
28. ए॒न॒म् प्र॒थ॒मः प्र॑थ॒म ए॑न मेनम् प्रथ॒मो अध्यधि॑ प्रथ॒म ए॑न मेनम् प्रथ॒मो अधि॑ । \newline
29. प्र॒थ॒मो अध्यधि॑ प्रथ॒मः प्र॑थ॒मो अध्य॑तिष्ठ दतिष्ठ॒ दधि॑ प्रथ॒मः प्र॑थ॒मो अध्य॑तिष्ठत् । \newline
30. अध्य॑तिष्ठ दतिष्ठ॒ दध्यध्य॑ तिष्ठत् । \newline
31. अ॒ति॒ष्ठ॒दित्य॑तिष्ठत् । \newline
32. ग॒न्ध॒र्वो अ॑स्यास्य गन्ध॒र्वो ग॑न्ध॒र्वो अ॑स्य रश॒नाꣳ र॑श॒ना म॑स्य गन्ध॒र्वो ग॑न्ध॒र्वो अ॑स्य रश॒नाम् । \newline
33. अ॒स्य॒ र॒श॒नाꣳ र॑श॒ना म॑स्यास्य रश॒ना म॑गृभ्णा दगृभ्णाद् रश॒ना म॑स्यास्य रश॒ना म॑गृभ्णात् । \newline
34. र॒श॒ना म॑गृभ्णा दगृभ्णाद् रश॒नाꣳ र॑श॒ना म॑गृभ्णा॒थ् सूरा॒थ् सूरा॑द गृभ्णाद् रश॒नाꣳ र॑श॒ना म॑गृभ्णा॒थ् सूरा᳚त् । \newline
35. अ॒गृ॒भ्णा॒थ् सूरा॒थ् सूरा॑ दगृभ्णा दगृभ्णा॒थ् सूरा॒दश्व॒ मश्वꣳ॒॒ सूरा॑ दगृभ्णा दगृभ्णा॒थ् सूरा॒दश्व᳚म् । \newline
36. सूरा॒दश्व॒ मश्वꣳ॒॒ सूरा॒थ् सूरा॒दश्वं॑ ॅवसवो वसवो॒ अश्वꣳ॒॒ सूरा॒थ् सूरा॒दश्वं॑ ॅवसवः । \newline
37. अश्वं॑ ॅवसवो वसवो॒ अश्व॒ मश्वं॑ ॅवसवो॒ निर् णिर् व॑सवो॒ अश्व॒ मश्वं॑ ॅवसवो॒ निः । \newline
38. व॒स॒वो॒ निर् णिर् व॑सवो वसवो॒ निर॑तष्टा तष्ट॒ निर् व॑सवो वसवो॒ निर॑तष्ट । \newline
39. निर॑तष्टा तष्ट॒ निर् णिर॑तष्ट । \newline
40. अ॒त॒ष्टेत्य॑तष्ट । \newline
41. असि॑ य॒मो य॒मो अस्यसि॑ य॒मो अस्यसि॑ य॒मो अस्यसि॑ य॒मो असि॑ । \newline
42. य॒मो अस्यसि॑ य॒मो य॒मो अस्या॑दि॒त्य आ॑दि॒त्यो ऽसि॑ य॒मो य॒मो अस्या॑दि॒त्यः । \newline
43. अस्या॑दि॒त्य आ॑दि॒त्यो ऽस्यस्या॑दि॒त्यो अ॑र्वन् नर्वन् नादि॒त्यो ऽस्यस्या॑ दि॒त्यो अ॑र्वन्न् । \newline
44. आ॒दि॒त्यो अ॑र्वन् नर्वन् नादि॒त्य आ॑दि॒त्यो अ॑र्व॒न् नस्यस्य॑र्वन् नादि॒त्य आ॑दि॒त्यो अ॑र्व॒न् नसि॑ । \newline
45. अ॒र्व॒न् नस्य स्य॑र्वन् नर्व॒न् नसि॑ त्रि॒त स्त्रि॒तो ऽस्य॑र्वन् नर्व॒न् नसि॑ त्रि॒तः । \newline
46. असि॑ त्रि॒त स्त्रि॒तो ऽस्यसि॑ त्रि॒तो गुह्ये॑न॒ गुह्ये॑न त्रि॒तो ऽस्यसि॑ त्रि॒तो गुह्ये॑न । \newline
47. त्रि॒तो गुह्ये॑न॒ गुह्ये॑न त्रि॒त स्त्रि॒तो गुह्ये॑न व्र॒तेन॑ व्र॒तेन॒ गुह्ये॑न त्रि॒त स्त्रि॒तो गुह्ये॑न व्र॒तेन॑ । \newline
48. गुह्ये॑न व्र॒तेन॑ व्र॒तेन॒ गुह्ये॑न॒ गुह्ये॑न व्र॒तेन॑ । \newline
49. व्र॒तेनेति॑ व्र॒तेन॑ । \newline
50. असि॒ सोमे॑न॒ सोमे॒ना स्यसि॒ सोमे॑न स॒मया॑ स॒मया॒ सोमे॒ना स्यसि॒ सोमे॑न स॒मया᳚ । \newline
51. सोमे॑न स॒मया॑ स॒मया॒ सोमे॑न॒ सोमे॑न स॒मया॒ विपृ॑क्तो॒ विपृ॑क्तः स॒मया॒ सोमे॑न॒ सोमे॑न स॒मया॒ विपृ॑क्तः । \newline
52. स॒मया॒ विपृ॑क्तो॒ विपृ॑क्तः स॒मया॑ स॒मया॒ विपृ॑क्त आ॒हु रा॒हुर् विपृ॑क्तः स॒मया॑ स॒मया॒ विपृ॑क्त आ॒हुः । \newline
53. विपृ॑क्त आ॒हु रा॒हुर् विपृ॑क्तो॒ विपृ॑क्त आ॒हु स्ते॑ त आ॒हुर् विपृ॑क्तो॒ विपृ॑क्त आ॒हु स्ते᳚ । \newline
54. विपृ॑क्त॒ इति॒ वि - पृ॒क्तः॒ । \newline
\pagebreak
\markright{ TS 4.6.7.2  \hfill https://www.vedavms.in \hfill}

\section{ TS 4.6.7.2 }

\textbf{TS 4.6.7.2 } \newline
\textbf{Samhita Paata} \newline

आ॒हुस्ते॒ त्रीणि॑ दि॒वि बन्ध॑नानि ॥ त्रीणि॑ त आहुर्दि॒वि बन्ध॑नानि॒ त्रीण्य॒फ्सु त्रीण्य॒न्तः स॑मु॒द्रे । उ॒तेव॑ मे॒ वरु॑णश्छन्थ् स्यर्व॒न्॒. यत्रा॑ त आ॒हुः प॑र॒मं ज॒नित्रं᳚ ॥ इ॒मा ते॑ वाजिन्नव॒मार्ज॑नानी॒मा श॒फानाꣳ॑ सनि॒तुर्नि॒धाना᳚ । अत्रा॑ ते भ॒द्रा र॑श॒ना अ॑पश्यमृ॒तस्य॒ या अ॑भि॒रक्ष॑न्ति गो॒पाः ॥ आ॒त्मानं॑ ते॒ मन॑सा॒ऽऽराद॑जानाम॒वो दि॒वा - [  ] \newline

\textbf{Pada Paata} \newline

आ॒हुः । ते॒ । त्रीणि॑ । दि॒वि । बन्ध॑नानि ॥ त्रीणि॑ । ते॒ । आ॒हुः॒ । दि॒वि । बन्ध॑नानि । त्रीणि॑ । अ॒फ्स्वित्य॑प् - सु । त्रीणि॑ । अ॒न्तः । स॒मु॒द्रे ॥ उ॒त । इ॒व॒ । मे॒ । वरु॑णः । छ॒न्थ्सि॒ । अ॒र्व॒न्न् । यत्र॑ । ते॒ । आ॒हुः । प॒र॒मम् । ज॒नित्र᳚म् ॥ इ॒मा । ते॒ । वा॒जि॒न्न् । अ॒व॒मार्ज॑ना॒नीत्य॑व - मार्ज॑नानि । इ॒मा । श॒फाना᳚म् । स॒नि॒तुः । नि॒धानेति॑ नि - धाना᳚ ॥ अत्र॑ । ते॒ । भ॒द्राः । र॒श॒नाः । अ॒प॒श्य॒म् । ऋ॒तस्य॑ । याः । अ॒भि॒रक्ष॒न्तीय॑भि - रक्ष॑न्ति । गो॒पा इति॑ गो - पाः ॥ आ॒त्मान᳚म् । ते॒ । मन॑सा । आ॒रात् । अ॒जा॒ना॒म् । अ॒वः । दि॒वा ।  \newline


\textbf{Krama Paata} \newline

आ॒हुस्ते᳚ । ते॒ त्रीणि॑ । त्रीणि॑ दि॒वि । दि॒वि बन्ध॑नानि । बन्ध॑ना॒नीति॒ बन्ध॑नानि ॥ त्रीणि॑ ते । त॒ आ॒हुः॒ । आ॒हु॒र् दि॒वि । दि॒वि बन्ध॑नानि । बन्ध॑नानि॒ त्रीणि॑ । त्रीण्य॒फ्सु । अ॒फ्सु त्रीणि॑ । अ॒फ्स्वित्य॑प् - सु । त्रीण्य॒न्तः । अ॒न्तः स॑मु॒द्रे । स॒मु॒द्र इति॑ समु॒द्रे ॥ उ॒तेव॑ । इ॒व॒ मे॒ । मे॒ वरु॑णः । वरु॑ण श्छन्थ्सि । छ॒न्थ्स्य॒र्व॒न्न्॒ । अ॒र्व॒न्.॒ यत्र॑ । यत्रा॑ ते । त॒ आ॒हुः । आ॒हुः प॑र॒मम् । प॒र॒मम् ज॒नित्र᳚म् । ज॒नित्र॒मिति॑ ज॒नित्र᳚म् ॥ इ॒मा ते᳚ । ते॒ वा॒जि॒न्न्॒ । वा॒जि॒न्न॒व॒मार्ज॑नानि । अ॒व॒मार्ज॑नानी॒मा । अ॒व॒मार्ज॑ना॒नीत्य॑व - मार्ज॑नानि । इ॒मा श॒फाना᳚म् । श॒फानाꣳ॑ सनि॒तुः । स॒नि॒तुर् नि॒धाना᳚ । नि॒धानेति॑ नि - धाना᳚ ॥ अत्रा॑ ते । ते॒ भ॒द्राः । भ॒द्रा र॑श॒नाः । र॒श॒ना अ॑पश्यम् । अ॒प॒श्य॒मृ॒तस्य॑ । ऋ॒तस्य॒ याः । या अ॑भि॒रक्ष॑न्ति । अ॒भि॒रक्ष॑न्ति गो॒पाः । अ॒भि॒रक्ष॒न्तीत्य॑भि - रक्ष॑न्ति । गो॒पा इति॑ गो - पाः ॥ आ॒त्मान॑म् ते । ते॒ मन॑सा । मन॑सा॒ऽऽरात् । आ॒राद॑जानाम् । अ॒जा॒ना॒म॒वः । अ॒वो दि॒वा । दि॒वा प॒तय॑न्तम् \newline

\textbf{Jatai Paata} \newline

1. आ॒हु स्ते॑ त आ॒हु रा॒हु स्ते᳚ । \newline
2. ते॒ त्रीणि॒ त्रीणि॑ ते ते॒ त्रीणि॑ । \newline
3. त्रीणि॑ दि॒वि दि॒वि त्रीणि॒ त्रीणि॑ दि॒वि । \newline
4. दि॒वि बन्ध॑नानि॒ बन्ध॑नानि दि॒वि दि॒वि बन्ध॑नानि । \newline
5. बन्ध॑ना॒नीति॒ बन्ध॑नानि । \newline
6. त्रीणि॑ ते ते॒ त्रीणि॒ त्रीणि॑ ते । \newline
7. त॒ आ॒हु॒ रा॒हु॒ स्ते॒ त॒ आ॒हुः॒ । \newline
8. आ॒हु॒र् दि॒वि दि॒व्या॑हु राहुर् दि॒वि । \newline
9. दि॒वि बन्ध॑नानि॒ बन्ध॑नानि दि॒वि दि॒वि बन्ध॑नानि । \newline
10. बन्ध॑नानि॒ त्रीणि॒ त्रीणि॒ बन्ध॑नानि॒ बन्ध॑नानि॒ त्रीणि॑ । \newline
11. त्रीण्य॒फ्स्व॑फ्सु त्रीणि॒ त्रीण्य॒फ्सु । \newline
12. अ॒फ्सु त्रीणि॒ त्रीण्य॒फ्स्व॑फ्सु त्रीणि॑ । \newline
13. अ॒फ्स्वित्य॑प् - सु । \newline
14. त्रीण्य॒न्त-र॒न्त स्त्रीणि॒ त्रीण्य॒न्तः । \newline
15. अ॒न्तः स॑मु॒द्रे स॑मु॒द्रे अ॒न्त-र॒न्तः स॑मु॒द्रे । \newline
16. स॒मु॒द्र इति॑ समु॒द्रे । \newline
17. उ॒तेवे॑ वो॒तोतेव॑ । \newline
18. इ॒व॒ मे॒ म॒ इ॒वे॒ व॒मे॒ । \newline
19. मे॒ वरु॑णो॒ वरु॑णो मे मे॒ वरु॑णः । \newline
20. वरु॑ण श्छन्थ्सि छन्थ्सि॒ वरु॑णो॒ वरु॑ण श्छन्थ्सि । \newline
21. छ॒न्थ्स्य॒र्व॒न् न॒र्व॒न् छ॒न्थ्सि॒ छ॒न्थ्स्य॒र्व॒न्न् । \newline
22. अ॒र्व॒न्॒. यत्र॒ यत्रा᳚र्वन् नर्व॒न्॒. यत्र॑ । \newline
23. यत्रा॑ ते ते॒ यत्र॒ यत्रा॑ ते । \newline
24. त॒ आ॒हु रा॒हु स्ते॑ त आ॒हुः । \newline
25. आ॒हुः प॑र॒मम् प॑र॒म मा॒हु रा॒हुः प॑र॒मम् । \newline
26. प॒र॒मम् ज॒नित्र॑म् ज॒नित्र॑म् पर॒मम् प॑र॒मम् ज॒नित्र᳚म् । \newline
27. ज॒नित्र॒मिति॑ ज॒नित्र᳚म् । \newline
28. इ॒मा ते॑ त इ॒मेमा ते᳚ । \newline
29. ते॒ वा॒जि॒न्॒. वा॒जि॒न् ते॒ ते॒ वा॒जि॒न्न् । \newline
30. वा॒जि॒न् न॒व॒मार्ज॑ना न्यव॒मार्ज॑नानि वाजिन्. वाजिन् नव॒मार्ज॑नानि । \newline
31. अ॒व॒मार्ज॑नानी॒मेमा ऽव॒मार्ज॑ना न्यव॒मार्ज॑नानी॒मा । \newline
32. अ॒व॒मार्ज॑ना॒नीत्य॑व - मार्ज॑नानि । \newline
33. इ॒मा श॒फानाꣳ॑ श॒फाना॑ मि॒मेमा श॒फाना᳚म् । \newline
34. श॒फानाꣳ॑ सनि॒तुः स॑नि॒तुः श॒फानाꣳ॑ श॒फानाꣳ॑ सनि॒तुः । \newline
35. स॒नि॒तुर् नि॒धाना॑ नि॒धाना॑ सनि॒तुः स॑नि॒तुर् नि॒धाना᳚ । \newline
36. नि॒धानेति॑ नि - धाना᳚ । \newline
37. अत्रा॑ ते ते॒ अत्रात्रा॑ ते । \newline
38. ते॒ भ॒द्रा भ॒द्रा स्ते॑ ते भ॒द्राः । \newline
39. भ॒द्रा र॑श॒ना र॑श॒ना भ॒द्रा भ॒द्रा र॑श॒नाः । \newline
40. र॒श॒ना अ॑पश्य मपश्यꣳ रश॒ना र॑श॒ना अ॑पश्यम् । \newline
41. अ॒प॒श्य॒ मृ॒तस्य॒ र्‌तस्या॑ पश्य मपश्य मृ॒तस्य॑ । \newline
42. ऋ॒तस्य॒ या या ऋ॒तस्य॒ र्‌तस्य॒ याः । \newline
43. या अ॑भि॒रक्ष॑ न्त्यभि॒रक्ष॑न्ति॒ या या अ॑भि॒रक्ष॑न्ति । \newline
44. अ॒भि॒रक्ष॑न्ति गो॒पा गो॒पा अ॑भि॒रक्ष॑ न्त्यभि॒रक्ष॑न्ति गो॒पाः । \newline
45. अ॒भि॒रक्ष॒न्तीत्य॑भि - रक्ष॑न्ति । \newline
46. गो॒पा इति॑ गो - पाः । \newline
47. आ॒त्मान॑म् ते त आ॒त्मान॑ मा॒त्मान॑म् ते । \newline
48. ते॒ मन॑सा॒ मन॑सा ते ते॒ मन॑सा । \newline
49. मन॑सा॒ ऽऽरा दा॒रान् मन॑सा॒ मन॑सा॒ ऽऽरात् । \newline
50. आ॒रा द॑जाना मजाना मा॒रा दा॒रा द॑जानाम् । \newline
51. अ॒जा॒ना॒ म॒वो॑ ऽवो अ॑जाना मजाना म॒वः । \newline
52. अ॒वो दि॒वा दि॒वा ऽवो॑ ऽवो दि॒वा । \newline
53. दि॒वा प॒तय॑न्तम् प॒तय॑न्तम् दि॒वा दि॒वा प॒तय॑न्तम् । \newline

\textbf{Ghana Paata } \newline

1. आ॒हु स्ते॑ त आ॒हु रा॒हु स्ते॒ त्रीणि॒ त्रीणि॑ त आ॒हु रा॒हु स्ते॒ त्रीणि॑ । \newline
2. ते॒ त्रीणि॒ त्रीणि॑ ते ते॒ त्रीणि॑ दि॒वि दि॒वि त्रीणि॑ ते ते॒ त्रीणि॑ दि॒वि । \newline
3. त्रीणि॑ दि॒वि दि॒वि त्रीणि॒ त्रीणि॑ दि॒वि बन्ध॑नानि॒ बन्ध॑नानि दि॒वि त्रीणि॒ त्रीणि॑ दि॒वि बन्ध॑नानि । \newline
4. दि॒वि बन्ध॑नानि॒ बन्ध॑नानि दि॒वि दि॒वि बन्ध॑नानि । \newline
5. बन्ध॑ना॒नीति॒ बन्ध॑नानि । \newline
6. त्रीणि॑ ते ते॒ त्रीणि॒ त्रीणि॑ त आहु राहु स्ते॒ त्रीणि॒ त्रीणि॑ त आहुः । \newline
7. त॒ आ॒हु॒ रा॒हु॒ स्ते॒ त॒ आ॒हु॒र् दि॒वि दि॒व्या॑हु स्ते त आहुर् दि॒वि । \newline
8. आ॒हु॒र् दि॒वि दि॒व्या॑हु राहुर् दि॒वि बन्ध॑नानि॒ बन्ध॑नानि दि॒व्या॑हु राहुर् दि॒वि बन्ध॑नानि । \newline
9. दि॒वि बन्ध॑नानि॒ बन्ध॑नानि दि॒वि दि॒वि बन्ध॑नानि॒ त्रीणि॒ त्रीणि॒ बन्ध॑नानि दि॒वि दि॒वि बन्ध॑नानि॒ त्रीणि॑ । \newline
10. बन्ध॑नानि॒ त्रीणि॒ त्रीणि॒ बन्ध॑नानि॒ बन्ध॑नानि॒ त्रीण्य॒फ्स्व॑फ्सु त्रीणि॒ बन्ध॑नानि॒ बन्ध॑नानि॒ त्रीण्य॒फ्सु । \newline
11. त्रीण्य॒फ्स्व॑फ्सु त्रीणि॒ त्रीण्य॒फ्सु त्रीणि॒ त्रीण्य॒फ्सु त्रीणि॒ त्रीण्य॒फ्सु त्रीणि॑ । \newline
12. अ॒फ्सु त्रीणि॒ त्रीण्य॒फ्स्व॑फ्सु त्रीण्य॒न्त र॒न्त स्त्रीण्य॒फ्स्व॑फ्सु त्रीण्य॒न्तः । \newline
13. अ॒फ्स्वित्य॑प् - सु । \newline
14. त्रीण्य॒न्त र॒न्त स्त्रीणि॒ त्रीण्य॒न्तः स॑मु॒द्रे स॑मु॒द्रे अ॒न्त स्त्रीणि॒ त्रीण्य॒न्तः स॑मु॒द्रे । \newline
15. अ॒न्तः स॑मु॒द्रे स॑मु॒द्रे अ॒न्त र॒न्तः स॑मु॒द्रे । \newline
16. स॒मु॒द्र इति॑ समु॒द्रे । \newline
17. उ॒तेवे॑ वो॒तो तेव॑ मे म इवो॒तो तेव॑ मे । \newline
18. इ॒व॒ मे॒ म॒ इ॒वे॒व॒ मे॒ वरु॑णो॒ वरु॑णो म इवेव मे॒ वरु॑णः । \newline
19. मे॒ वरु॑णो॒ वरु॑णो मे मे॒ वरु॑ण श्छन्थ्सि छन्थ्सि॒ वरु॑णो मे मे॒ वरु॑ण श्छन्थ्सि । \newline
20. वरु॑ण श्छन्थ्सि छन्थ्सि॒ वरु॑णो॒ वरु॑ण श्छन्थ्स्यर्वन् नर्वन् छन्थ्सि॒ वरु॑णो॒ वरु॑ण श्छन्थ्स्यर्वन्न् । \newline
21. छ॒न्थ्स्य॒र्व॒न् न॒र्व॒न् छ॒न्थ्सि॒ छ॒न्थ्स्य॒र्व॒न्॒. यत्र॒ यत्रा᳚र्वन् छन्थ्सि छन्थ्स्यर्व॒न्॒. यत्र॑ । \newline
22. अ॒र्व॒न्॒. यत्र॒ यत्रा᳚र्वन् नर्व॒न्॒. यत्रा॑ ते ते॒ यत्रा᳚र्वन् नर्व॒न्॒. यत्रा॑ ते । \newline
23. यत्रा॑ ते ते॒ यत्र॒ यत्रा॑ त आ॒हु रा॒हु स्ते॒ यत्र॒ यत्रा॑ त आ॒हुः । \newline
24. त॒ आ॒हु रा॒हु स्ते॑ त आ॒हुः प॑र॒मम् प॑र॒म मा॒हु स्ते॑ त आ॒हुः प॑र॒मम् । \newline
25. आ॒हुः प॑र॒मम् प॑र॒म मा॒हु रा॒हुः प॑र॒मम् ज॒नित्र॑म् ज॒नित्र॑म् पर॒म मा॒हु रा॒हुः प॑र॒मम् ज॒नित्र᳚म् । \newline
26. प॒र॒मम् ज॒नित्र॑म् ज॒नित्र॑म् पर॒मम् प॑र॒मम् ज॒नित्र᳚म् । \newline
27. ज॒नित्र॒मिति॑ ज॒नित्र᳚म् । \newline
28. इ॒मा ते॑ त इ॒मेमा ते॑ वाजिन्. वाजिन् त इ॒मेमा ते॑ वाजिन्न् । \newline
29. ते॒ वा॒जि॒न्॒. वा॒जि॒न् ते॒ ते॒ वा॒जि॒न् न॒व॒मार्ज॑ना न्यव॒मार्ज॑नानि वाजिन् ते ते वाजिन् नव॒मार्ज॑नानि । \newline
30. वा॒जि॒न् न॒व॒मार्ज॑ना न्यव॒मार्ज॑नानि वाजिन्. वाजिन् नव॒मार्ज॑नानी॒ मेमा ऽव॒मार्ज॑नानि वाजिन्. वाजिन् नव॒मार्ज॑ना नी॒मा । \newline
31. अ॒व॒मार्ज॑नानी॒ मेमा ऽव॒मार्ज॑ना न्यव॒मार्ज॑नानी॒मा श॒फानाꣳ॑ श॒फाना॑ मि॒मा ऽव॒मार्ज॑
नान्यव॒मार्ज॑ना नी॒मा श॒फाना᳚म् । \newline
32. अ॒व॒मार्ज॑ना॒नीत्य॑व - मार्ज॑नानि । \newline
33. इ॒मा श॒फानाꣳ॑ श॒फाना॑ मि॒मेमा श॒फानाꣳ॑ सनि॒तुः स॑नि॒तुः श॒फाना॑ मि॒मेमा श॒फानाꣳ॑ सनि॒तुः । \newline
34. श॒फानाꣳ॑ सनि॒तुः स॑नि॒तुः श॒फानाꣳ॑ श॒फानाꣳ॑ सनि॒तुर् नि॒धाना॑ नि॒धाना॑ सनि॒तुः श॒फानाꣳ॑ श॒फानाꣳ॑ सनि॒तुर् नि॒धाना᳚ । \newline
35. स॒नि॒तुर् नि॒धाना॑ नि॒धाना॑ सनि॒तुः स॑नि॒तुर् नि॒धाना᳚ । \newline
36. नि॒धानेति॑ नि - धाना᳚ । \newline
37. अत्रा॑ ते ते॒ अत्रात्रा॑ ते भ॒द्रा भ॒द्रा स्ते॒ अत्रात्रा॑ ते भ॒द्राः । \newline
38. ते॒ भ॒द्रा भ॒द्रा स्ते॑ ते भ॒द्रा र॑श॒ना र॑श॒ना भ॒द्रा स्ते॑ ते भ॒द्रा र॑श॒नाः । \newline
39. भ॒द्रा र॑श॒ना र॑श॒ना भ॒द्रा भ॒द्रा र॑श॒ना अ॑पश्य मपश्यꣳ रश॒ना भ॒द्रा भ॒द्रा र॑श॒ना अ॑पश्यम् । \newline
40. र॒श॒ना अ॑पश्य मपश्यꣳ रश॒ना र॑श॒ना अ॑पश्य मृ॒तस्य॒ र्‌तस्या॑ पश्यꣳ रश॒ना र॑श॒ना अ॑पश्य मृ॒तस्य॑ । \newline
41. अ॒प॒श्य॒ मृ॒तस्य॒ र्‌तस्या॑ पश्य मपश्य मृ॒तस्य॒ या या ऋ॒तस्या॑ पश्य मपश्य मृ॒तस्य॒ याः । \newline
42. ऋ॒तस्य॒ या या ऋ॒तस्य॒ र्‌तस्य॒ या अ॑भि॒रक्ष॑ न्त्यभि॒रक्ष॑न्ति॒ या ऋ॒तस्य॒ र्‌तस्य॒ या अ॑भि॒रक्ष॑न्ति । \newline
43. या अ॑भि॒रक्ष॑ न्त्यभि॒रक्ष॑न्ति॒ या या अ॑भि॒रक्ष॑न्ति गो॒पा गो॒पा अ॑भि॒रक्ष॑न्ति॒ या या अ॑भि॒रक्ष॑न्ति गो॒पाः । \newline
44. अ॒भि॒रक्ष॑न्ति गो॒पा गो॒पा अ॑भि॒रक्ष॑ न्त्यभि॒रक्ष॑न्ति गो॒पाः । \newline
45. अ॒भि॒रक्ष॒न्तीत्य॑भि - रक्ष॑न्ति । \newline
46. गो॒पा इति॑ गो - पाः । \newline
47. आ॒त्मान॑म् ते त आ॒त्मान॑ मा॒त्मान॑म् ते॒ मन॑सा॒ मन॑सा त आ॒त्मान॑ मा॒त्मान॑म् ते॒ मन॑सा । \newline
48. ते॒ मन॑सा॒ मन॑सा ते ते॒ मन॑सा॒ ऽऽरा दा॒रान् मन॑सा ते ते॒ मन॑सा॒ ऽऽरात् । \newline
49. मन॑सा॒ ऽऽरा दा॒रान् मन॑सा॒ मन॑सा॒ ऽऽरा द॑जाना मजाना मा॒रान् मन॑सा॒ मन॑सा॒ ऽऽरा द॑जानाम् । \newline
50. आ॒रा द॑जाना मजाना मा॒रा दा॒रा द॑जाना म॒वो॑ ऽवो अ॑जाना मा॒रा दा॒रा द॑जाना म॒वः । \newline
51. अ॒जा॒ना॒ म॒वो॑ ऽवो अ॑जाना मजाना म॒वो दि॒वा दि॒वा ऽवो अ॑जाना मजाना म॒वो दि॒वा । \newline
52. अ॒वो दि॒वा दि॒वा ऽवो॑ ऽवो दि॒वा प॒तय॑न्तम् प॒तय॑न्तम् दि॒वा ऽवो॑ ऽवो दि॒वा प॒तय॑न्तम् । \newline
53. दि॒वा प॒तय॑न्तम् प॒तय॑न्तम् दि॒वा दि॒वा प॒तय॑न्तम् पत॒ङ्गम् प॑त॒ङ्गम् प॒तय॑न्तम् दि॒वा दि॒वा प॒तय॑न्तम् पत॒ङ्गम् । \newline
\pagebreak
\markright{ TS 4.6.7.3  \hfill https://www.vedavms.in \hfill}

\section{ TS 4.6.7.3 }

\textbf{TS 4.6.7.3 } \newline
\textbf{Samhita Paata} \newline

प॒तय॑न्तं पत॒ङ्गं । शिरो॑ अपश्यं प॒थिभिः॑ सु॒गेभि॑ररे॒णुभि॒र्जेह॑मानं पत॒त्रि ॥ अत्रा॑ ते रू॒पमु॑त्त॒मम॑पश्यं॒ जिगी॑षमाणमि॒ष आ प॒दे गोः । य॒दा ते॒ मर्तो॒ अनु॒ भोग॒मान॒डादिद् ग्रसि॑ष्ठ॒ ओष॑धीरजीगः ॥ अनु॑ त्वा॒ रथो॒ अनु॒ मर्यो॑ अर्व॒न्ननु॒ गावोऽनु॒ भगः॑ क॒नीनां᳚ । अनु॒ व्राता॑स॒स्तव॑ स॒ख्यमी॑यु॒रनु॑ दे॒वा म॑मिरे वी॒र्यं॑ - [  ] \newline

\textbf{Pada Paata} \newline

प॒तय॑न्तम् । प॒त॒ङ्गम् ॥ शिरः॑ । अ॒प॒श्य॒म् । प॒थिभि॒रिति॑ प॒थि - भिः॒ । सु॒गेभि॒रिति॑ सु - गेभिः॑ । अ॒रे॒णुभि॒रित्य॑रे॒णु - भिः॒ । जेह॑मानम् । प॒त॒त्रि ॥ अत्र॑ । ते॒ । रू॒पम् । उ॒त्त॒ममित्यु॑त् - त॒मम् । अ॒प॒श्य॒म् । जिगी॑षमाणम् । इ॒षः । एति॑ । प॒दे । गोः ॥ य॒दा । ते॒ । मर्तः॑ । अन्विति॑ । भोग᳚म् । आन॑ट् । आत् । इत् । ग्रसि॑ष्ठः । ओष॑धीः । अ॒जी॒गः॒ ॥ अन्विति॑ । त्वा॒ । रथः॑ । अन्विति॑ । मर्यः॑ । अ॒र्व॒न्न् । अन्विति॑ । गावः॑ । अन्विति॑ । भगः॑ । क॒नीना᳚म् ॥ अन्विति॑ । व्राता॑सः । तव॑ । स॒ख्यम् । ई॒युः॒ । अन्विति॑ । दे॒वाः । म॒मि॒रे॒ । वी॒र्य᳚म् ।  \newline


\textbf{Krama Paata} \newline

प॒तय॑न्तम् पत॒ङ्गम् । प॒त॒ङ्गमिति॑ पत॒ङ्गम् ॥ शिरो॑ अपश्यम् । अ॒प॒श्य॒म् प॒थिभिः॑ । प॒थिभिः॑ सु॒गेभिः॑ । प॒थिभि॒रिति॑ प॒थि - भिः॒ । सु॒गेभि॑ररे॒णुभिः॑ । सु॒गेभि॒रिति॑ सु - गेभिः॑ । अ॒रे॒णुभि॒र् जेह॑मानम् । अ॒रे॒णुभि॒रित्य॑रे॒णु - भिः॒ । जेह॑मानम् पत॒त्रि । प॒त॒त्रीति॑ पत॒त्रि ॥ अत्रा॑ ते । ते॒ रू॒पम् । रू॒पमु॑त्त॒मम् । उ॒त्त॒मम॑पश्यम् । उ॒त्त॒ममित्यु॑त् - त॒मम् । अ॒प॒श्य॒म् जिगी॑षमाणम् । जिगी॑षमाणमि॒षः । इ॒ष आ । आ प॒दे । प॒दे गोः । गोरिति॒ गोः ॥ य॒दा ते᳚ । ते॒ मर्तः॑ । मर्तो॒ अनु॑ । अनु॒ भोग᳚म् । भोग॒मान॑ट् । आन॒डात् । आदित् । इद् ग्रसि॑ष्ठः । ग्रसि॑ष्ठ॒ ओष॑धीः । ओष॑धीरजीगः । अ॒जी॒ग॒रित्य॑जीगः ॥ अनु॑ त्वा । त्वा॒ रथः॑ । रथो॒ अनु॑ । अनु॒ मर्यः॑ । मर्यो॑ अर्वन्न् । अ॒र्व॒न्ननु॑ । अनु॒ गावः॑ । गावोऽनु॑ । अनु॒ भगः॑ । भगः॑ क॒नीनाम्᳚ । क॒नीना॒मिति॑ क॒नीनाम्᳚ ॥ अनु॒ व्राता॑सः । व्राता॑स॒स्तव॑ । तव॑ स॒ख्यम् । स॒ख्यमी॑युः । ई॒यु॒रनु॑ । अनु॑ दे॒वाः । दे॒वा म॑मिरे । म॒मि॒रे॒ वी॒र्य᳚म् । वी॒र्य॑म् ते \newline

\textbf{Jatai Paata} \newline

1. प॒तय॑न्तम् पत॒ङ्गम् प॑त॒ङ्गम् प॒तय॑न्तम् प॒तय॑न्तम् पत॒ङ्गम् । \newline
2. प॒त॒ङ्गमिति॑ पत॒ङ्गम् । \newline
3. शिरो॑ अपश्य मपश्यꣳ॒॒ शिरः॒ शिरो॑ अपश्यम् । \newline
4. अ॒प॒श्य॒म् प॒थिभिः॑ प॒थिभि॑ रपश्य मपश्यम् प॒थिभिः॑ । \newline
5. प॒थिभिः॑ सु॒गेभिः॑ सु॒गेभिः॑ प॒थिभिः॑ प॒थिभिः॑ सु॒गेभिः॑ । \newline
6. प॒थिभि॒रिति॑ प॒थि - भिः॒ । \newline
7. सु॒गेभि॑ ररे॒णुभि॑ ररे॒णुभिः॑ सु॒गेभिः॑ सु॒गेभि॑र रे॒णुभिः॑ । \newline
8. सु॒गेभि॒रिति॑ सु - गेभिः॑ । \newline
9. अ॒रे॒णुभि॒र् जेह॑मान॒म् जेह॑मान मरे॒णुभि॑ ररे॒णुभि॒र् जेह॑मानम् । \newline
10. अ॒रे॒णुभि॒रित्य॑रे॒णु - भिः॒ । \newline
11. जेह॑मानम् पत॒त्रि प॑त॒त्रि जेह॑मान॒म् जेह॑मानम् पत॒त्रि । \newline
12. प॒त॒त्रीति॑ पत॒त्रि । \newline
13. अत्रा॑ ते ते॒ अत्रात्रा॑ ते । \newline
14. ते॒ रू॒पꣳ रू॒पम् ते॑ ते रू॒पम् । \newline
15. रू॒प मु॑त्त॒म मु॑त्त॒मꣳ रू॒पꣳ रू॒प मु॑त्त॒मम् । \newline
16. उ॒त्त॒म म॑पश्य मपश्य मुत्त॒म मु॑त्त॒म म॑पश्यम् । \newline
17. उ॒त्त॒ममित्यु॑त् - त॒मम् । \newline
18. अ॒प॒श्य॒म् जिगी॑षमाण॒म् जिगी॑षमाण मपश्य मपश्य॒म् जिगी॑षमाणम् । \newline
19. जिगी॑षमाण मि॒ष इ॒षो जिगी॑षमाण॒म् जिगी॑षमाण मि॒षः । \newline
20. इ॒ष एष इ॒ष आ । \newline
21. आ प॒दे प॒द आ प॒दे । \newline
22. प॒दे गोर् गोः प॒दे प॒दे गोः । \newline
23. गोरिति॒ गोः । \newline
24. य॒दा ते॑ ते य॒दा य॒दा ते᳚ । \newline
25. ते॒ मर्तो॒ मर्त॑ स्ते ते॒ मर्तः॑ । \newline
26. मर्तो॒ अन्वनु॒ मर्तो॒ मर्तो॒ अनु॑ । \newline
27. अनु॒ भोग॒म् भोग॒ मन्वनु॒ भोग᳚म् । \newline
28. भोग॒ मान॒ डान॒ड् भोग॒म् भोग॒ मान॑ट् । \newline
29. आन॒ डादा दान॒ डान॒ डात् । \newline
30. आदिदि दादा दित् । \newline
31. इद् ग्रसि॑ष्ठो॒ ग्रसि॑ष्ठ॒ इदिद् ग्रसि॑ष्ठः । \newline
32. ग्रसि॑ष्ठ॒ ओष॑धी॒ रोष॑धी॒र् ग्रसि॑ष्ठो॒ ग्रसि॑ष्ठ॒ ओष॑धीः । \newline
33. ओष॑धी रजीग-रजीग॒-रोष॑धी॒ रोष॑धी रजीगः । \newline
34. अ॒जी॒ग॒रित्य॑जीगः । \newline
35. अनु॑ त्वा॒ त्वा ऽन्वनु॑ त्वा । \newline
36. त्वा॒ रथो॒ रथ॑ स्त्वा त्वा॒ रथः॑ । \newline
37. रथो॒ अन्वनु॒ रथो॒ रथो॒ अनु॑ । \newline
38. अनु॒ मर्यो॒ मर्यो॒ अन्वनु॒ मर्यः॑ । \newline
39. मर्यो॑ अर्वन् नर्व॒न् मर्यो॒ मर्यो॑ अर्वन्न् । \newline
40. अ॒र्व॒न् नन् वन् व॑र्वन् नर्व॒न् ननु॑ । \newline
41. अनु॒ गावो॒ गावो॒ अन्वनु॒ गावः॑ । \newline
42. गावो ऽन्वनु॒ गावो॒ गावो ऽनु॑ । \newline
43. अनु॒ भगो॒ भगो॒ अन्वनु॒ भगः॑ । \newline
44. भगः॑ क॒नीना᳚म् क॒नीना॒म् भगो॒ भगः॑ क॒नीना᳚म् । \newline
45. क॒नीना॒मिति॑ क॒नीनाम्᳚ । \newline
46. अनु॒ व्राता॑सो॒ व्राता॑सो॒ अन्वनु॒ व्राता॑सः । \newline
47. व्राता॑स॒ स्तव॒ तव॒ व्राता॑सो॒ व्राता॑स॒ स्तव॑ । \newline
48. तव॑ स॒ख्यꣳ स॒ख्यम् तव॒ तव॑ स॒ख्यम् । \newline
49. स॒ख्य मी॑यु रीयुः स॒ख्यꣳ स॒ख्य मी॑युः । \newline
50. ई॒यु॒ रन् वन् वी॑यु रीयु॒ रनु॑ । \newline
51. अनु॑ दे॒वा दे॒वा अन्वनु॑ दे॒वाः । \newline
52. दे॒वा म॑मिरे ममिरे दे॒वा दे॒वा म॑मिरे । \newline
53. म॒मि॒रे॒ वी॒र्यं॑ ॅवी॒र्य॑म् ममिरे ममिरे वी॒र्य᳚म् । \newline
54. वी॒र्य॑म् ते ते वी॒र्यं॑ ॅवी॒र्य॑म् ते । \newline

\textbf{Ghana Paata } \newline

1. प॒तय॑न्तम् पत॒ङ्गम् प॑त॒ङ्गम् प॒तय॑न्तम् प॒तय॑न्तम् पत॒ङ्गम् । \newline
2. प॒त॒ङ्गमिति॑ पत॒ङ्गम् । \newline
3. शिरो॑ अपश्य मपश्यꣳ॒॒ शिरः॒ शिरो॑ अपश्यम् प॒थिभिः॑ प॒थिभि॑ रपश्यꣳ॒॒ शिरः॒ शिरो॑ अपश्यम् प॒थिभिः॑ । \newline
4. अ॒प॒श्य॒म् प॒थिभिः॑ प॒थिभि॑ रपश्य मपश्यम् प॒थिभिः॑ सु॒गेभिः॑ सु॒गेभिः॑ प॒थिभि॑ रपश्य मपश्यम् प॒थिभिः॑ सु॒गेभिः॑ । \newline
5. प॒थिभिः॑ सु॒गेभिः॑ सु॒गेभिः॑ प॒थिभिः॑ प॒थिभिः॑ सु॒गेभि॑ ररे॒णुभि॑ ररे॒णुभिः॑ सु॒गेभिः॑ प॒थिभिः॑ प॒थिभिः॑ सु॒गेभि॑ ररे॒णुभिः॑ । \newline
6. प॒थिभि॒रिति॑ प॒थि - भिः॒ । \newline
7. सु॒गेभि॑ ररे॒णुभि॑ ररे॒णुभिः॑ सु॒गेभिः॑ सु॒गेभि॑ ररे॒णुभि॒र् जेह॑मान॒म् जेह॑मान मरे॒णुभिः॑ सु॒गेभिः॑ सु॒गेभि॑ ररे॒णुभि॒र् जेह॑मानम् । \newline
8. सु॒गेभि॒रिति॑ सु - गेभिः॑ । \newline
9. अ॒रे॒णुभि॒र् जेह॑मान॒म् जेह॑मान मरे॒णुभि॑ ररे॒णुभि॒र् जेह॑मानम् पत॒त्रि प॑त॒त्रि जेह॑मान मरे॒णुभि॑ ररे॒णुभि॒र् जेह॑मानम् पत॒त्रि । \newline
10. अ॒रे॒णुभि॒रित्य॑रे॒णु - भिः॒ । \newline
11. जेह॑मानम् पत॒त्रि प॑त॒त्रि जेह॑मान॒म् जेह॑मानम् पत॒त्रि । \newline
12. प॒त॒त्रीति॑ पत॒त्रि । \newline
13. अत्रा॑ ते ते॒ अत्रात्रा॑ ते रू॒पꣳ रू॒पम् ते॒ अत्रात्रा॑ ते रू॒पम् । \newline
14. ते॒ रू॒पꣳ रू॒पम् ते॑ ते रू॒प मु॑त्त॒म मु॑त्त॒मꣳ रू॒पम् ते॑ ते रू॒प मु॑त्त॒मम् । \newline
15. रू॒प मु॑त्त॒म मु॑त्त॒मꣳ रू॒पꣳ रू॒प मु॑त्त॒म म॑पश्य मपश्य मुत्त॒मꣳ रू॒पꣳ रू॒प मु॑त्त॒म म॑पश्यम् । \newline
16. उ॒त्त॒म म॑पश्य मपश्य मुत्त॒म मु॑त्त॒म म॑पश्य॒म् जिगी॑षमाण॒म् जिगी॑षमाण मपश्य मुत्त॒म मु॑त्त॒म म॑पश्य॒म् जिगी॑षमाणम् । \newline
17. उ॒त्त॒ममित्यु॑त् - त॒मम् । \newline
18. अ॒प॒श्य॒म् जिगी॑षमाण॒म् जिगी॑षमाण मपश्य मपश्य॒म् जिगी॑षमाण मि॒ष इ॒षो जिगी॑षमाण मपश्य मपश्य॒म् जिगी॑षमाण मि॒षः । \newline
19. जिगी॑षमाण मि॒ष इ॒षो जिगी॑षमाण॒म् जिगी॑षमाण मि॒ष एषो जिगी॑षमाण॒म् जिगी॑षमाण मि॒ष आ । \newline
20. इ॒ष एष इ॒ष आ प॒दे प॒द एष इ॒ष आ प॒दे । \newline
21. आ प॒दे प॒द आ प॒दे गोर् गोः प॒द आ प॒दे गोः । \newline
22. प॒दे गोर् गोः प॒दे प॒दे गोः । \newline
23. गोरिति॒ गोः । \newline
24. य॒दा ते॑ ते य॒दा य॒दा ते॒ मर्तो॒ मर्त॑ स्ते य॒दा य॒दा ते॒ मर्तः॑ । \newline
25. ते॒ मर्तो॒ मर्त॑ स्ते ते॒ मर्तो॒ अन्वनु॒ मर्त॑ स्ते ते॒ मर्तो॒ अनु॑ । \newline
26. मर्तो॒ अन्वनु॒ मर्तो॒ मर्तो॒ अनु॒ भोग॒म् भोग॒ मनु॒ मर्तो॒ मर्तो॒ अनु॒ भोग᳚म् । \newline
27. अनु॒ भोग॒म् भोग॒ मन्वनु॒ भोग॒ मान॒ डान॒ड् भोग॒ मन्वनु॒ भोग॒ मान॑ट् । \newline
28. भोग॒ मान॒ डान॒ड् भोग॒म् भोग॒ मान॒ डदा दान॒ड् भोग॒म् भोग॒ मान॒डात् । \newline
29. आन॒ डदा दान॒डा न॒डा दिदिदा दान॒ डान॒ डादित् । \newline
30. आदि दिदा दादिद् ग्रसि॑ष्ठो॒ ग्रसि॑ष्ठ॒ इदा दादिद् ग्रसि॑ष्ठः । \newline
31. इद् ग्रसि॑ष्ठो॒ ग्रसि॑ष्ठ॒ इदिद् ग्रसि॑ष्ठ॒ ओष॑धी॒ रोष॑धी॒र् ग्रसि॑ष्ठ॒ इदिद् ग्रसि॑ष्ठ॒ ओष॑धीः । \newline
32. ग्रसि॑ष्ठ॒ ओष॑धी॒ रोष॑धी॒र् ग्रसि॑ष्ठो॒ ग्रसि॑ष्ठ॒ ओष॑धी रजीग रजीग॒ रोष॑धी॒र् ग्रसि॑ष्ठो॒ ग्रसि॑ष्ठ॒ ओष॑धी रजीगः । \newline
33. ओष॑धी रजीग रजीग॒ रोष॑धी॒ रोष॑धी रजीगः । \newline
34. अ॒जी॒ग॒रित्य॑जीगः । \newline
35. अनु॑ त्वा॒ त्वा ऽन्वनु॑ त्वा॒ रथो॒ रथ॒ स्त्वा ऽन्वनु॑ त्वा॒ रथः॑ । \newline
36. त्वा॒ रथो॒ रथ॑ स्त्वा त्वा॒ रथो॒ अन्वनु॒ रथ॑ स्त्वा त्वा॒ रथो॒ अनु॑ । \newline
37. रथो॒ अन्वनु॒ रथो॒ रथो॒ अनु॒ मर्यो॒ मर्यो ऽनु॒ रथो॒ रथो॒ अनु॒ मर्यः॑ । \newline
38. अनु॒ मर्यो॒ मर्यो॒ अन्वनु॒ मर्यो॑ अर्वन् नर्व॒न् मर्यो॒ अन्वनु॒ मर्यो॑ अर्वन्न् । \newline
39. मर्यो॑ अर्वन् नर्व॒न् मर्यो॒ मर्यो॑ अर्व॒न् नन्वन् व॑र्व॒न् मर्यो॒ मर्यो॑ अर्व॒न् ननु॑ । \newline
40. अ॒र्व॒न् नन्वन् व॑र्वन् नर्व॒न् ननु॒ गावो॒ गावो॒ अन्व॑र्वन् नर्व॒न् ननु॒ गावः॑ । \newline
41. अनु॒ गावो॒ गावो॒ अन्वनु॒ गावो ऽन्वनु॒ गावो॒ अन्वनु॒ गावो ऽनु॑ । \newline
42. गावो ऽन्वनु॒ गावो॒ गावो ऽनु॒ भगो॒ भगो ऽनु॒ गावो॒ गावो ऽनु॒ भगः॑ । \newline
43. अनु॒ भगो॒ भगो॒ अन्वनु॒ भगः॑ क॒नीना᳚म् क॒नीना॒म् भगो॒ अन्वनु॒ भगः॑ क॒नीना᳚म् । \newline
44. भगः॑ क॒नीना᳚म् क॒नीना॒म् भगो॒ भगः॑ क॒नीना᳚म् । \newline
45. क॒नीना॒मिति॑ क॒नीनाम्᳚ । \newline
46. अनु॒ व्राता॑सो॒ व्राता॑सो॒ अन्वनु॒ व्राता॑स॒ स्तव॒ तव॒ व्राता॑सो॒ अन्वनु॒ व्राता॑स॒ स्तव॑ । \newline
47. व्राता॑स॒ स्तव॒ तव॒ व्राता॑सो॒ व्राता॑स॒ स्तव॑ स॒ख्यꣳ स॒ख्यम् तव॒ व्राता॑सो॒ व्राता॑स॒ स्तव॑ स॒ख्यम् । \newline
48. तव॑ स॒ख्यꣳ स॒ख्यम् तव॒ तव॑ स॒ख्य मी॑यु रीयुः स॒ख्यम् तव॒ तव॑ स॒ख्य मी॑युः । \newline
49. स॒ख्य मी॑यु रीयुः स॒ख्यꣳ स॒ख्य मी॑यु॒ रन्वन् वी॑युः स॒ख्यꣳ स॒ख्य मी॑यु॒ रनु॑ । \newline
50. ई॒यु॒ रन्वन् वी॑यु रीयु॒रनु॑ दे॒वा दे॒वा अन्वी॑यु रीयु॒ रनु॑ दे॒वाः । \newline
51. अनु॑ दे॒वा दे॒वा अन्वनु॑ दे॒वा म॑मिरे ममिरे दे॒वा अन्वनु॑ दे॒वा म॑मिरे । \newline
52. दे॒वा म॑मिरे ममिरे दे॒वा दे॒वा म॑मिरे वी॒र्यं॑ ॅवी॒र्य॑म् ममिरे दे॒वा दे॒वा म॑मिरे वी॒र्य᳚म् । \newline
53. म॒मि॒रे॒ वी॒र्यं॑ ॅवी॒र्य॑म् ममिरे ममिरे वी॒र्य॑म् ते ते वी॒र्य॑म् ममिरे ममिरे वी॒र्य॑म् ते । \newline
54. वी॒र्य॑म् ते ते वी॒र्यं॑ ॅवी॒र्य॑म् ते । \newline
\pagebreak
\markright{ TS 4.6.7.4  \hfill https://www.vedavms.in \hfill}

\section{ TS 4.6.7.4 }

\textbf{TS 4.6.7.4 } \newline
\textbf{Samhita Paata} \newline

ते ॥ हिर॑ण्यशृ॒ङ्गोऽयो॑ अस्य॒ पादा॒ मनो॑जवा॒ अव॑र॒ इन्द्र॑ आसीत् । दे॒वा इद॑स्य हवि॒रद्य॑माय॒न्॒. यो अर्व॑न्तं प्रथ॒मो अ॒द्ध्यति॑ष्ठत् ॥ ई॒र्मान्ता॑सः॒ सिलि॑कमद्ध्यमासः॒ सꣳ शूर॑णासो दि॒व्यासो॒ अत्याः᳚ । हꣳ॒॒सा इ॑व श्रेणि॒शो य॑तन्ते॒ यदाक्षि॑षुर्दि॒व्य-मज्म॒मश्वाः᳚ ॥ तव॒ शरी॑रं पतयि॒ष्ण्व॑र्व॒न् तव॑ चि॒त्तं ॅवात॑ इव॒ ध्रजी॑मान् । तव॒ शृङ्गा॑णि॒ विष्ठि॑ता पुरु॒त्राऽर॑ण्येषु॒ जर्भु॑राणा चरन्ति ॥ उप॒ - [  ] \newline

\textbf{Pada Paata} \newline

ते॒ ॥ हिर॑ण्यशृङ्ग॒ इति॒ हिर॑ण्य - शृ॒ङ्गः॒ । अयः॑ । अ॒स्य॒ । पादाः᳚ । मनो॑जवा॒ इति॒ मनः॑ - ज॒वाः॒ । अव॑रः । इन्द्रः॑ । आ॒सी॒त् ॥ दे॒वाः । इत् । अ॒स्य॒ । ह॒वि॒रद्य॒मिति॑ हविः - अद्य᳚म् । आ॒य॒न्न् । यः । अर्व॑न्तम् । प्र॒थ॒मः । अ॒द्ध्यति॑ष्ठ॒दित्य॑धि - अति॑ष्ठत् ॥ ई॒र्मान्ता॑स॒ इती॒र्म-अ॒न्ता॒सः॒ । सिलि॑कमद्ध्यमास॒ इति॒ सिलि॑क-म॒द्ध्य॒मा॒सः॒ । समिति॑ । शूर॑णासः । दि॒व्यासः॑ । अत्याः᳚ ॥ हꣳ॒॒साः । इ॒व॒ । श्रे॒णि॒श इति॑ श्रेणि - शः । य॒त॒न्ते॒ । यत् । आक्षि॑षुः । दि॒व्यम् । अज्म᳚म् । अश्वाः᳚ ॥ तव॑ । शरी॑रम् । प॒त॒यि॒ष्णु । अ॒र्व॒न्न् । तव॑ । चि॒त्तम् । वातः॑ । इ॒व॒ । ध्रजी॑मान् ॥ तव॑ । शृङ्गा॑णि । विष्ठि॒तेति॒ वि - स्थि॒ता॒ । पु॒रु॒त्रेति॑ पुरु - त्रा । अर॑ण्येषु । जर्भु॑राणा । च॒र॒न्ति॒ ॥ उप॑ ।  \newline


\textbf{Krama Paata} \newline

त॒ इति॑ ते ॥ हिर॑ण्यशृ॒ङ्गोऽयः॑ । हिर॑ण्यशृङ्ग॒ इति॒ हिर॑ण्य - शृ॒ङ्गः॒ । अयो॑ अस्य । अ॒स्य॒ पादाः᳚ । पादा॒ मनो॑जवाः । मनो॑जवा॒ अव॑रः । मनो॑जवा॒ इति॒ मनः॑ - ज॒वाः॒ । अव॑र॒ इन्द्रः॑ । इन्द्र॑ आसीत् । आ॒सी॒दित्या॑सीत् ॥ दे॒वा इत् । इद॑स्य । अ॒स्य॒ ह॒वि॒रद्य᳚म् । ह॒वि॒रद्य॑मायन्न् । ह॒वि॒रद्य॒मिति॑ हविः - अद्य᳚म् । आ॒य॒न्.॒ यः । यो अर्व॑न्तम् । अर्व॑न्तम् प्रथ॒मः । प्र॒थ॒मो अ॒द्ध्यति॑ष्ठत् । अ॒द्ध्यति॑ष्ठ॒दित्य॑धि - अति॑ष्ठत् ॥ ई॒र्मान्ता॑सः॒ सिलि॑कमद्ध्यमासः । ई॒र्मान्ता॑स॒ इती॒र्म - अ॒न्ता॒सः॒ । सिलि॑कमद्ध्यमासः॒ सम् । सिलि॑कमद्ध्यमास॒ इति॒ सिलि॑क - म॒द्ध्य॒मा॒सः॒ । सꣳ शूर॑णासः । शूर॑णासो दि॒व्यासः॑ । दि॒व्यासो॒ अत्याः᳚ । अत्या॒ इत्यत्याः᳚ ॥ हꣳ॒॒सा इ॑व । इ॒व॒ श्रे॒णि॒शः । श्रे॒णि॒शो य॑तन्ते । श्रे॒णि॒श इति॑ श्रेणि - शः । य॒त॒न्ते॒ यत् । यदाक्षि॑षुः । आक्षि॑षुर् दि॒व्यम् । दि॒व्यमज्म᳚म् । अज्म॒मश्वाः᳚ । अश्वा॒ इत्यश्वाः᳚ ॥ तव॒ शरी॑रम् । शरी॑रम् पतयि॒ष्णु । प॒त॒यि॒ष्ण्व॑र्वन्न् । अ॒र्व॒न् तव॑ । तव॑ चि॒त्तम् । चि॒त्तं ॅवातः॑ । वात॑ इव । इ॒व॒ ध्रजी॑मान् । ध्रजी॑मा॒निति॒ ध्रजी॑मान् ॥ तव॒ शृङ्गा॑णि । शृङ्गा॑णि॒ विष्ठि॑ता । विष्ठि॑ता पुरु॒त्रा । विष्ठि॒तेति॒ वि - स्थि॒ता॒ । पु॒रु॒त्राऽर॑ण्येषु । पु॒रु॒त्रेति॑ पुरु - त्रा । अर॑ण्येषु॒ जर्भु॑राणा । जर्भु॑राणा चरन्ति । च॒र॒न्तीति॑ चरन्ति ॥ उप॒ ( ) प्र \newline

\textbf{Jatai Paata} \newline

1. त॒ इति॑ ते । \newline
2. हिर॑ण्यशृ॒ङ्गो ऽयो ऽयो॒ हिर॑ण्यशृङ्गो॒ हिर॑ण्यशृ॒ङ्गो ऽयः॑ । \newline
3. हिर॑ण्यशृङ्ग॒ इति॒ हिर॑ण्य - शृ॒ङ्गः॒ । \newline
4. अयो॑ अस्या॒स्या यो ऽयो॑ अस्य । \newline
5. अ॒स्य॒ पादाः॒ पादा॑ अस्यास्य॒ पादाः᳚ । \newline
6. पादा॒ मनो॑जवा॒ मनो॑जवाः॒ पादाः॒ पादा॒ मनो॑जवाः । \newline
7. मनो॑जवा॒ अव॒रो ऽव॑रो॒ मनो॑जवा॒ मनो॑जवा॒ अव॑रः । \newline
8. मनो॑जवा॒ इति॒ मनः॑ - ज॒वाः॒ । \newline
9. अव॑र॒ इन्द्र॒ इन्द्रो ऽव॒रो ऽव॑र॒ इन्द्रः॑ । \newline
10. इन्द्र॑ आसी दासी॒ दिन्द्र॒ इन्द्र॑ आसीत् । \newline
11. आ॒सी॒दित्या॑सीत् । \newline
12. दे॒वा इदिद् दे॒वा दे॒वा इत् । \newline
13. इद॑स्या॒ स्येदि द॑स्य । \newline
14. अ॒स्य॒ ह॒वि॒रद्यꣳ॑ हवि॒रद्य॑ मस्यास्य हवि॒रद्य᳚म् । \newline
15. ह॒वि॒रद्य॑ मायन् नायन्. हवि॒रद्यꣳ॑ हवि॒रद्य॑ मायन्न् । \newline
16. ह॒वि॒रद्य॒मिति॑ हविः - अद्य᳚म् । \newline
17. आ॒य॒न्॒. यो य आ॑यन् नाय॒न्॒. यः । \newline
18. यो अर्व॑न्त॒ मर्व॑न्तं॒ ॅयो यो अर्व॑न्तम् । \newline
19. अर्व॑न्तम् प्रथ॒मः प्र॑थ॒मो अर्व॑न्त॒ मर्व॑न्तम् प्रथ॒मः । \newline
20. प्र॒थ॒मो अ॒द्ध्यति॑ष्ठ द॒द्ध्यति॑ष्ठत् प्रथ॒मः प्र॑थ॒मो अ॒द्ध्यति॑ष्ठत् । \newline
21. अ॒द्ध्यति॑ष्ठ॒दित्य॑धि - अति॑ष्ठत् । \newline
22. ई॒र्मान्ता॑सः॒ सिलि॑कमद्ध्यमासः॒ सिलि॑कमद्ध्यमास ई॒र्मान्ता॑स ई॒र्मान्ता॑सः॒ सिलि॑कमद्ध्यमासः । \newline
23. ई॒र्मान्ता॑स॒ इती॒र्म - अ॒न्ता॒सः॒ । \newline
24. सिलि॑कमद्ध्यमासः॒ सꣳ सꣳ सिलि॑कमद्ध्यमासः॒ सिलि॑कमद्ध्यमासः॒ सम् । \newline
25. सिलि॑कमद्ध्यमास॒ इति॒ सिलि॑क - म॒द्ध्य॒मा॒सः॒ । \newline
26. सꣳ शूर॑णासः॒ शूर॑णासः॒ सꣳ सꣳ शूर॑णासः । \newline
27. शूर॑णासो दि॒व्यासो॑ दि॒व्यासः॒ शूर॑णासः॒ शूर॑णासो दि॒व्यासः॑ । \newline
28. दि॒व्यासो॒ अत्या॒ अत्या॑ दि॒व्यासो॑ दि॒व्यासो॒ अत्याः᳚ । \newline
29. अत्या॒ इत्यत्याः᳚ । \newline
30. हꣳ॒॒सा इ॑वेव हꣳ॒॒सा हꣳ॒॒सा इ॑व । \newline
31. इ॒व॒ श्रे॒णि॒शः श्रे॑णि॒श इ॑वेव श्रेणि॒शः । \newline
32. श्रे॒णि॒शो य॑तन्ते यतन्ते श्रेणि॒शः श्रे॑णि॒शो य॑तन्ते । \newline
33. श्रे॒णि॒श इति॑ श्रेणि - शः । \newline
34. य॒त॒न्ते॒ यद् यद् य॑तन्ते यतन्ते॒ यत् । \newline
35. यदाक्षि॑षु॒ राक्षि॑षु॒र् यद् यदाक्षि॑षुः । \newline
36. आक्षि॑षुर् दि॒व्यम् दि॒व्य माक्षि॑षु॒ राक्षि॑षुर् दि॒व्यम् । \newline
37. दि॒व्य मज्म॒ मज्म॑म् दि॒व्यम् दि॒व्य मज्म᳚म् । \newline
38. अज्म॒ मश्वा॒ अश्वा॒ अज्म॒ मज्म॒ मश्वाः᳚ । \newline
39. अश्वा॒ इत्यश्वाः᳚ । \newline
40. तव॒ शरी॑रꣳ॒॒ शरी॑र॒म् तव॒ तव॒ शरी॑रम् । \newline
41. शरी॑रम् पतयि॒ष्णु प॑तयि॒ष्णु शरी॑रꣳ॒॒ शरी॑रम् पतयि॒ष्णु । \newline
42. प॒त॒यि॒ ष्ण्व॑र्वन् नर्वन् पतयि॒ष्णु प॑तयि॒ ष्ण्व॑र्वन्न् । \newline
43. अ॒र्व॒न् तव॒ तवा᳚र्वन् नर्व॒न् तव॑ । \newline
44. तव॑ चि॒त्तम् चि॒त्तम् तव॒ तव॑ चि॒त्तम् । \newline
45. चि॒त्तं ॅवातो॒ वात॑ श्चि॒त्तम् चि॒त्तं ॅवातः॑ । \newline
46. वात॑ इवे व॒ वातो॒ वात॑ इव । \newline
47. इ॒व॒ ध्रजी॑मा॒न् ध्रजी॑मा निवेव॒ ध्रजी॑मान् । \newline
48. ध्रजी॑मा॒निति॒ ध्रजी॑मान् । \newline
49. तव॒ शृङ्गा॑णि॒ शृङ्गा॑णि॒ तव॒ तव॒ शृङ्गा॑णि । \newline
50. शृङ्गा॑णि॒ विष्ठि॑ता॒ विष्ठि॑ता॒ शृङ्गा॑णि॒ शृङ्गा॑णि॒ विष्ठि॑ता । \newline
51. विष्ठि॑ता पुरु॒त्रा पु॑रु॒त्रा विष्ठि॑ता॒ विष्ठि॑ता पुरु॒त्रा । \newline
52. विष्ठि॒तेति॒ वि - स्थि॒ता॒ । \newline
53. पु॒रु॒त्रा ऽर॑ण्ये॒ ष्वर॑ण्येषु पुरु॒त्रा पु॑रु॒त्रा ऽर॑ण्येषु । \newline
54. पु॒रु॒त्रेति॑ पुरु - त्रा । \newline
55. अर॑ण्येषु॒ जर्भु॑राणा॒ जर्भु॑रा॒णा ऽर॑ण्ये॒ ष्वर॑ण्येषु॒ जर्भु॑राणा । \newline
56. जर्भु॑राणा चरन्ति चरन्ति॒ जर्भु॑राणा॒ जर्भु॑राणा चरन्ति । \newline
57. च॒र॒न्तीति॑ चरन्ति । \newline
58. उप॒ प्र प्रोपोप॒ प्र । \newline

\textbf{Ghana Paata } \newline

1. त॒ इति॑ ते । \newline
2. हिर॑ण्यशृ॒ङ्गो ऽयो ऽयो॒ हिर॑ण्यशृङ्गो॒ हिर॑ण्यशृ॒ङ्गो ऽयो॑ अस्या॒स्यायो॒ हिर॑ण्यशृङ्गो॒ हिर॑ण्यशृ॒ङ्गो ऽयो॑ अस्य । \newline
3. हिर॑ण्यशृङ्ग॒ इति॒ हिर॑ण्य - शृ॒ङ्गः॒ । \newline
4. अयो॑ अस्या॒स्यायो ऽयो॑ अस्य॒ पादाः॒ पादा॑ अ॒स्यायो ऽयो॑ अस्य॒ पादाः᳚ । \newline
5. अ॒स्य॒ पादाः॒ पादा॑ अस्यास्य॒ पादा॒ मनो॑जवा॒ मनो॑जवाः॒ पादा॑ अस्यास्य॒ पादा॒ मनो॑जवाः । \newline
6. पादा॒ मनो॑जवा॒ मनो॑जवाः॒ पादाः॒ पादा॒ मनो॑जवा॒ अव॒रो ऽव॑रो॒ मनो॑जवाः॒ पादाः॒ पादा॒ मनो॑जवा॒ अव॑रः । \newline
7. मनो॑जवा॒ अव॒रो ऽव॑रो॒ मनो॑जवा॒ मनो॑जवा॒ अव॑र॒ इन्द्र॒ इन्द्रो ऽव॑रो॒ मनो॑जवा॒ मनो॑जवा॒ अव॑र॒ इन्द्रः॑ । \newline
8. मनो॑जवा॒ इति॒ मनः॑ - ज॒वाः॒ । \newline
9. अव॑र॒ इन्द्र॒ इन्द्रो ऽव॒रो ऽव॑र॒ इन्द्र॑ आसी दासी॒ दिन्द्रो ऽव॒रो ऽव॑र॒ इन्द्र॑ आसीत् । \newline
10. इन्द्र॑ आसी दासी॒ दिन्द्र॒ इन्द्र॑ आसीत् । \newline
11. आ॒सी॒दित्या॑सीत् । \newline
12. दे॒वा इदिद् दे॒वा दे॒वा इद॑स्या॒ स्येद् दे॒वा दे॒वा इद॑स्य । \newline
13. इद॑स्या॒ स्ये दिद॑स्य हवि॒रद्यꣳ॑ हवि॒रद्य॑ म॒स्ये दिद॑स्य हवि॒रद्य᳚म् । \newline
14. अ॒स्य॒ ह॒वि॒रद्यꣳ॑ हवि॒रद्य॑ मस्यास्य हवि॒रद्य॑ मायन् नायन्. हवि॒रद्य॑ मस्यास्य हवि॒रद्य॑ मायन्न् । \newline
15. ह॒वि॒रद्य॑ मायन् नायन्. हवि॒रद्यꣳ॑ हवि॒रद्य॑ माय॒न्॒. यो य आ॑यन्. हवि॒रद्यꣳ॑ हवि॒रद्य॑ माय॒न्॒. यः । \newline
16. ह॒वि॒रद्य॒मिति॑ हविः - अद्य᳚म् । \newline
17. आ॒य॒न्॒. यो य आ॑यन् नाय॒न्॒. यो अर्व॑न्त॒ मर्व॑न्तं॒ ॅय आ॑यन् नाय॒न्॒. यो अर्व॑न्तम् । \newline
18. यो अर्व॑न्त॒ मर्व॑न्तं॒ ॅयो यो अर्व॑न्तम् प्रथ॒मः प्र॑थ॒मो अर्व॑न्तं॒ ॅयो यो अर्व॑न्तम् प्रथ॒मः । \newline
19. अर्व॑न्तम् प्रथ॒मः प्र॑थ॒मो अर्व॑न्त॒ मर्व॑न्तम् प्रथ॒मो अ॒द्ध्यति॑ष्ठ द॒द्ध्यति॑ष्ठत् प्रथ॒मो 
अर्व॑न्त॒ मर्व॑न्तम् प्रथ॒मो अ॒द्ध्यति॑ष्ठत् । \newline
20. प्र॒थ॒मो अ॒द्ध्यति॑ष्ठ द॒द्ध्यति॑ष्ठत् प्रथ॒मः प्र॑थ॒मो अ॒द्ध्यति॑ष्ठत् । \newline
21. अ॒द्ध्यति॑ष्ठ॒दित्य॑धि - अति॑ष्ठत् । \newline
22. ई॒र्मान्ता॑सः॒ सिलि॑कमद्ध्यमासः॒ सिलि॑कमद्ध्यमास ई॒र्मान्ता॑स ई॒र्मान्ता॑सः॒ सिलि॑कमद्ध्यमासः॒ सꣳ सꣳ सिलि॑कमद्ध्यमास ई॒र्मान्ता॑स ई॒र्मान्ता॑सः॒ सिलि॑कमद्ध्यमासः॒ सम् । \newline
23. ई॒र्मान्ता॑स॒ इती॒र्म - अ॒न्ता॒सः॒ । \newline
24. सिलि॑कमद्ध्यमासः॒ सꣳ सꣳ सिलि॑कमद्ध्यमासः॒ सिलि॑कमद्ध्यमासः॒ सꣳ शूर॑णासः॒ शूर॑णासः॒ सꣳ सिलि॑कमद्ध्यमासः॒ सिलि॑कमद्ध्यमासः॒ सꣳ शूर॑णासः । \newline
25. सिलि॑कमद्ध्यमास॒ इति॒ सिलि॑क - म॒द्ध्य॒मा॒सः॒ । \newline
26. सꣳ शूर॑णासः॒ शूर॑णासः॒ सꣳ सꣳ शूर॑णासो दि॒व्यासो॑ दि॒व्यासः॒ शूर॑णासः॒ सꣳ सꣳ शूर॑णासो दि॒व्यासः॑ । \newline
27. शूर॑णासो दि॒व्यासो॑ दि॒व्यासः॒ शूर॑णासः॒ शूर॑णासो दि॒व्यासो॒ अत्या॒ अत्या॑ दि॒व्यासः॒ शूर॑णासः॒ शूर॑णासो दि॒व्यासो॒ अत्याः᳚ । \newline
28. दि॒व्यासो॒ अत्या॒ अत्या॑ दि॒व्यासो॑ दि॒व्यासो॒ अत्याः᳚ । \newline
29. अत्या॒ इत्यत्याः᳚ । \newline
30. हꣳ॒॒सा इ॑वेव हꣳ॒॒सा हꣳ॒॒सा इ॑व श्रेणि॒शः श्रे॑णि॒श इ॑व हꣳ॒॒सा हꣳ॒॒सा इ॑व श्रेणि॒शः । \newline
31. इ॒व॒ श्रे॒णि॒शः श्रे॑णि॒श इ॑वेव श्रेणि॒शो य॑तन्ते यतन्ते श्रेणि॒श इ॑वेव श्रेणि॒शो य॑तन्ते । \newline
32. श्रे॒णि॒शो य॑तन्ते यतन्ते श्रेणि॒शः श्रे॑णि॒शो य॑तन्ते॒ यद् यद् य॑तन्ते श्रेणि॒शः श्रे॑णि॒शो य॑तन्ते॒ यत् । \newline
33. श्रे॒णि॒श इति॑ श्रेणि - शः । \newline
34. य॒त॒न्ते॒ यद् यद् य॑तन्ते यतन्ते॒ यदाक्षि॑षु॒ राक्षि॑षु॒र् यद् य॑तन्ते यतन्ते॒ यदाक्षि॑षुः । \newline
35. यदाक्षि॑षु॒ राक्षि॑षु॒र् यद् यदाक्षि॑षुर् दि॒व्यम् दि॒व्य माक्षि॑षु॒र् यद् यदाक्षि॑षुर् दि॒व्यम् । \newline
36. आक्षि॑षुर् दि॒व्यम् दि॒व्य माक्षि॑षु॒ राक्षि॑षुर् दि॒व्य मज्म॒ मज्म॑म् दि॒व्य माक्षि॑षु॒ राक्षि॑षुर् दि॒व्य मज्म᳚म् । \newline
37. दि॒व्य मज्म॒ मज्म॑म् दि॒व्यम् दि॒व्य मज्म॒ मश्वा॒ अश्वा॒ अज्म॑म् दि॒व्यम् दि॒व्य मज्म॒ मश्वाः᳚ । \newline
38. अज्म॒ मश्वा॒ अश्वा॒ अज्म॒ मज्म॒ मश्वाः᳚ । \newline
39. अश्वा॒ इत्यश्वाः᳚ । \newline
40. तव॒ शरी॑रꣳ॒॒ शरी॑र॒म् तव॒ तव॒ शरी॑रम् पतयि॒ष्णु प॑तयि॒ष्णु शरी॑र॒म् तव॒ तव॒ शरी॑रम् पतयि॒ष्णु । \newline
41. शरी॑रम् पतयि॒ष्णु प॑तयि॒ष्णु शरी॑रꣳ॒॒ शरी॑रम् पतयि॒ ष्ण्व॑र्वन् नर्वन् पतयि॒ष्णु शरी॑रꣳ॒॒ शरी॑रम् पतयि॒ ष्ण्व॑र्वन्न् । \newline
42. प॒त॒यि॒ ष्ण्व॑र्वन् नर्वन् पतयि॒ष्णु प॑तयि॒ ष्ण्व॑र्व॒न् तव॒ तवा᳚र्वन् पतयि॒ष्णु प॑तयि॒ ष्ण्व॑र्व॒न् तव॑ । \newline
43. अ॒र्व॒न् तव॒ तवा᳚र्वन् नर्व॒न् तव॑ चि॒त्तम् चि॒त्तम् तवा᳚र्वन् नर्व॒न् तव॑ चि॒त्तम् । \newline
44. तव॑ चि॒त्तम् चि॒त्तम् तव॒ तव॑ चि॒त्तं ॅवातो॒ वात॑ श्चि॒त्तम् तव॒ तव॑ चि॒त्तं ॅवातः॑ । \newline
45. चि॒त्तं ॅवातो॒ वात॑ श्चि॒त्तम् चि॒त्तं ॅवात॑ इवेव॒ वात॑ श्चि॒त्तम् चि॒त्तं ॅवात॑ इव । \newline
46. वात॑ इवेव॒ वातो॒ वात॑ इव॒ ध्रजी॑मा॒न् ध्रजी॑मा निव॒ वातो॒ वात॑ इव॒ ध्रजी॑मान् । \newline
47. इ॒व॒ ध्रजी॑मा॒न् ध्रजी॑मा निवेव॒ ध्रजी॑मान् । \newline
48. ध्रजी॑मा॒निति॒ ध्रजी॑मान् । \newline
49. तव॒ शृङ्गा॑णि॒ शृङ्गा॑णि॒ तव॒ तव॒ शृङ्गा॑णि॒ विष्ठि॑ता॒ विष्ठि॑ता॒ शृङ्गा॑णि॒ तव॒ तव॒ शृङ्गा॑णि॒ विष्ठि॑ता । \newline
50. शृङ्गा॑णि॒ विष्ठि॑ता॒ विष्ठि॑ता॒ शृङ्गा॑णि॒ शृङ्गा॑णि॒ विष्ठि॑ता पुरु॒त्रा पु॑रु॒त्रा विष्ठि॑ता॒ शृङ्गा॑णि॒ शृङ्गा॑णि॒ विष्ठि॑ता पुरु॒त्रा । \newline
51. विष्ठि॑ता पुरु॒त्रा पु॑रु॒त्रा विष्ठि॑ता॒ विष्ठि॑ता पुरु॒त्रा ऽर॑ण्ये॒ ष्वर॑ण्येषु पुरु॒त्रा विष्ठि॑ता॒ विष्ठि॑ता पुरु॒त्रा ऽर॑ण्येषु । \newline
52. विष्ठि॒तेति॒ वि - स्थि॒ता॒ । \newline
53. पु॒रु॒त्रा ऽर॑ण्ये॒ ष्वर॑ण्येषु पुरु॒त्रा पु॑रु॒त्रा ऽर॑ण्येषु॒ जर्भु॑राणा॒ जर्भु॑रा॒णा ऽर॑ण्येषु पुरु॒त्रा पु॑रु॒त्रा ऽर॑ण्येषु॒ जर्भु॑राणा । \newline
54. पु॒रु॒त्रेति॑ पुरु - त्रा । \newline
55. अर॑ण्येषु॒ जर्भु॑राणा॒ जर्भु॑रा॒णा ऽर॑ण्ये॒ ष्वर॑ण्येषु॒ जर्भु॑राणा चरन्ति चरन्ति॒ जर्भु॑रा॒णा ऽर॑ण्ये॒
ष्वर॑ण्येषु॒ जर्भु॑राणा चरन्ति । \newline
56. जर्भु॑राणा चरन्ति चरन्ति॒ जर्भु॑राणा॒ जर्भु॑राणा चरन्ति । \newline
57. च॒र॒न्तीति॑ चरन्ति । \newline
58. उप॒ प्र प्रोपोप॒ प्रागा॑ दगा॒त् प्रोपोप॒ प्रागा᳚त् । \newline
\pagebreak
\markright{ TS 4.6.7.5  \hfill https://www.vedavms.in \hfill}

\section{ TS 4.6.7.5 }

\textbf{TS 4.6.7.5 } \newline
\textbf{Samhita Paata} \newline

प्रागा॒च्छस॑नं ॅवा॒ज्यर्वा॑ देव॒द्रीचा॒ मन॑सा॒ दीद्ध्या॑नः । अ॒जः पु॒रो नी॑यते॒ नाभि॑र॒स्यानु॑ प॒श्चात् क॒वयो॑ यन्ति रे॒भाः ॥उप॒ प्रागा᳚त् पर॒मं ॅयथ् स॒धस्थ॒मर्वाꣳ॒॒ अच्छा॑ पि॒तरं॑ मा॒तरं॑ च । अ॒द्या दे॒वान् जुष्ट॑तमो॒ हि ग॒म्या अथाऽऽशा᳚स्ते दा॒शुषे॒ वार्या॑णि ॥ \newline

\textbf{Pada Paata} \newline

प्रेति॑ । अ॒गा॒त् । शस॑नम् । वा॒जी । अर्वा᳚ । दे॒व॒द्रीचेति॑ देव-द्रीचा᳚ । मन॑सा । दीद्ध्या॑नः ॥ अ॒जः । पु॒रः । नी॒य॒ते॒ । नाभिः॑ । अ॒स्य॒ । अन्विति॑ । प॒श्चात् । क॒वयः॑ । य॒न्ति॒ । रे॒भाः ॥ उप॑ । प्रेति॑ । अ॒गा॒त् । प॒र॒मम् । यत् । स॒धस्थ॒मिति॑ स॒ध - स्थ॒म् । अर्वान्॑ । अच्छ॑ । पि॒तर᳚म् । मा॒तर᳚म् । च॒ ॥ अ॒द्य । दे॒वान् । जुष्ट॑तम॒ इति॒ जुष्ट॑ - त॒मः॒ । हि । ग॒म्याः । अथ॑ । एति॑ । शा॒स्ते॒ । दा॒शुषे᳚ । वार्या॑णि ॥  \newline


\textbf{Krama Paata} \newline

प्रागा᳚त् । अ॒गा॒च्छस॑नम् । शस॑नं ॅवा॒जी । वा॒ज्यर्वा᳚ । अर्वा॑ देव॒द्रीचा᳚ । दे॒व॒द्रीचा॒ मन॑सा । दे॒व॒द्रीचेति॑ देव - द्रीचा᳚ । मन॑सा॒ दीद्ध्या॑नः । दीद्ध्या॑न॒ इति॒ दीद्ध्या॑नः ॥ अ॒जः पु॒रः । पु॒रो नी॑यते । नी॒य॒ते॒ नाभिः॑ । नाभि॑रस्य । अ॒स्यानु॑ । अनु॑ प॒श्चात् । प॒श्चात् क॒वयः॑ । क॒वयो॑ यन्ति । य॒न्ति॒ रे॒भाः । रे॒भा इति॑ रे॒भाः ॥ उप॒ प्र । प्रागा᳚त् । अ॒गा॒त् प्॒र॒मम् । प॒र॒मम् ॅयत् । यथ् स॒धस्थ᳚म् । स॒धस्थ॒मर्वान्॑ । स॒धस्थ॒मिति॑ स॒ध - स्थ॒म् । अर्वाꣳ॒॒ अच्छ॑ । अच्छा॑ पि॒तर᳚म् । पि॒तर॑म् मा॒तर᳚म् । मा॒तर॑म् च । चेति॑ च ॥ अ॒द्या दे॒वान् । दे॒वान् जुष्ट॑तमः । जुष्ट॑तमो॒ हि । जुष्ट॑तम॒ इति॒ जुष्ट॑ - त॒मः॒ । हि ग॒म्याः । ग॒म्या अथ॑ । अथा । आ शा᳚स्ते । शा॒स्ते॒ दा॒शुषे᳚ । दा॒शुषे॒ वार्या॑णि । वार्या॒णीति॒ वार्या॑णि । \newline

\textbf{Jatai Paata} \newline

1. प्रागा॑ दगा॒त् प्र प्रागा᳚त् । \newline
2. अ॒गा॒ च्छस॑नꣳ॒॒ शस॑न मगा दगा॒ च्छस॑नम् । \newline
3. शस॑नं ॅवा॒जी वा॒जी शस॑नꣳ॒॒ शस॑नं ॅवा॒जी । \newline
4. वा॒ज्यर्वा ऽर्वा॑ वा॒जी वा॒ज्यर्वा᳚ । \newline
5. अर्वा॑ देव॒द्रीचा॑ देव॒द्रीचा ऽर्वा ऽर्वा॑ देव॒द्रीचा᳚ । \newline
6. दे॒व॒द्रीचा॒ मन॑सा॒ मन॑सा देव॒द्रीचा॑ देव॒द्रीचा॒ मन॑सा । \newline
7. दे॒व॒द्रीचेति॑ देव - द्रीचा᳚ । \newline
8. मन॑सा॒ दीद्ध्या॑नो॒ दीद्ध्या॑नो॒ मन॑सा॒ मन॑सा॒ दीद्ध्या॑नः । \newline
9. दीद्ध्या॑न॒ इति॒ दीद्ध्या॑नः । \newline
10. अ॒जः पु॒रः पु॒रो अ॒जो अ॒जः पु॒रः । \newline
11. पु॒रो नी॑यते नीयते पु॒रः पु॒रो नी॑यते । \newline
12. नी॒य॒ते॒ नाभि॒र् नाभि॑र् नीयते नीयते॒ नाभिः॑ । \newline
13. नाभि॑ रस्यास्य॒ नाभि॒र् नाभि॑ रस्य । \newline
14. अ॒स्या न्वन् व॑स्या॒ स्यानु॑ । \newline
15. अनु॑ प॒श्चात् प॒श्चा दन्वनु॑ प॒श्चात् । \newline
16. प॒श्चात् क॒वयः॑ क॒वयः॑ प॒श्चात् प॒श्चात् क॒वयः॑ । \newline
17. क॒वयो॑ यन्ति यन्ति क॒वयः॑ क॒वयो॑ यन्ति । \newline
18. य॒न्ति॒ रे॒भा रे॒भा य॑न्ति यन्ति रे॒भाः । \newline
19. रे॒भा इति॑ रे॒भाः । \newline
20. उप॒ प्र प्रोपोप॒ प्र । \newline
21. प्रागा॑ दगा॒त् प्र प्रागा᳚त् । \newline
22. अ॒गा॒त् प॒र॒मम् प॑र॒म म॑गा दगात् पर॒मम् । \newline
23. प॒र॒मं ॅयद् यत् प॑र॒मम् प॑र॒मं ॅयत् । \newline
24. यथ् स॒धस्थꣳ॑ स॒धस्थं॒ ॅयद् यथ् स॒धस्थ᳚म् । \newline
25. स॒धस्थ॒ मर्वाꣳ॒॒ अर्वा᳚न् थ्स॒धस्थꣳ॑ स॒धस्थ॒ मर्वान्॑ । \newline
26. स॒धस्थ॒मिति॑ स॒ध - स्थ॒म् । \newline
27. अर्वाꣳ॒॒ अच्छाच्छा र्वाꣳ॒॒ अर्वाꣳ॒॒ अच्छ॑ । \newline
28. अच्छा॑ पि॒तर॑म् पि॒तर॒ मच्छाच्छा॑ पि॒तर᳚म् । \newline
29. पि॒तर॑म् मा॒तर॑म् मा॒तर॑म् पि॒तर॑म् पि॒तर॑म् मा॒तर᳚म् । \newline
30. मा॒तर॑म् च च मा॒तर॑म् मा॒तर॑म् च । \newline
31. चेति॑ च । \newline
32. अ॒द्या दे॒वान् दे॒वा न॒द्याद्या दे॒वान् । \newline
33. दे॒वान् जुष्ट॑तमो॒ जुष्ट॑तमो दे॒वान् दे॒वान् जुष्ट॑तमः । \newline
34. जुष्ट॑तमो॒ हि हि जुष्ट॑तमो॒ जुष्ट॑तमो॒ हि । \newline
35. जुष्ट॑तम॒ इति॒ जुष्ट॑ - त॒मः॒ । \newline
36. हि ग॒म्या ग॒म्या हि हि ग॒म्याः । \newline
37. ग॒म्या अथाथ॑ ग॒म्या ग॒म्या अथ॑ । \newline
38. अथा ऽथाथा । \newline
39. आ शा᳚स्ते शास्त॒ आ शा᳚स्ते । \newline
40. शा॒स्ते॒ दा॒शुषे॑ दा॒शुषे॑ शास्ते शास्ते दा॒शुषे᳚ । \newline
41. दा॒शुषे॒ वार्या॑णि॒ वार्या॑णि दा॒शुषे॑ दा॒शुषे॒ वार्या॑णि । \newline
42. वार्या॒णीति॒ वार्या॑णि । \newline

\textbf{Ghana Paata } \newline

1. प्रागा॑ दगा॒त् प्र प्रागा॒ च्छस॑नꣳ॒॒ शस॑न मगा॒त् प्र प्रागा॒ च्छस॑नम् । \newline
2. अ॒गा॒ च्छस॑नꣳ॒॒ शस॑न मगा दगा॒ च्छस॑नं ॅवा॒जी वा॒जी शस॑न मगा दगा॒ च्छस॑नं ॅवा॒जी । \newline
3. शस॑नं ॅवा॒जी वा॒जी शस॑नꣳ॒॒ शस॑नं ॅवा॒ज्यर्वा ऽर्वा॑ वा॒जी शस॑नꣳ॒॒ शस॑नं ॅवा॒ज्यर्वा᳚ । \newline
4. वा॒ज्यर्वा ऽर्वा॑ वा॒जी वा॒ज्यर्वा॑ देव॒द्रीचा॑ देव॒द्रीचा ऽर्वा॑ वा॒जी वा॒ज्यर्वा॑ देव॒द्रीचा᳚ । \newline
5. अर्वा॑ देव॒द्रीचा॑ देव॒द्रीचा ऽर्वा ऽर्वा॑ देव॒द्रीचा॒ मन॑सा॒ मन॑सा देव॒द्रीचा ऽर्वा ऽर्वा॑ देव॒द्रीचा॒ मन॑सा । \newline
6. दे॒व॒द्रीचा॒ मन॑सा॒ मन॑सा देव॒द्रीचा॑ देव॒द्रीचा॒ मन॑सा॒ दीद्ध्या॑नो॒ दीद्ध्या॑नो॒ मन॑सा देव॒द्रीचा॑ देव॒द्रीचा॒ मन॑सा॒ दीद्ध्या॑नः । \newline
7. दे॒व॒द्रीचेति॑ देव - द्रीचा᳚ । \newline
8. मन॑सा॒ दीद्ध्या॑नो॒ दीद्ध्या॑नो॒ मन॑सा॒ मन॑सा॒ दीद्ध्या॑नः । \newline
9. दीद्ध्या॑न॒ इति॒ दीद्ध्या॑नः । \newline
10. अ॒जः पु॒रः पु॒रो अ॒जो अ॒जः पु॒रो नी॑यते नीयते पु॒रो अ॒जो अ॒जः पु॒रो नी॑यते । \newline
11. पु॒रो नी॑यते नीयते पु॒रः पु॒रो नी॑यते॒ नाभि॒र् नाभि॑र् नीयते पु॒रः पु॒रो नी॑यते॒ नाभिः॑ । \newline
12. नी॒य॒ते॒ नाभि॒र् नाभि॑र् नीयते नीयते॒ नाभि॑ रस्यास्य॒ नाभि॑र् नीयते नीयते॒ नाभि॑ रस्य । \newline
13. नाभि॑ रस्यास्य॒ नाभि॒र् नाभि॑ र॒स्यान्वन् व॑स्य॒ नाभि॒र् नाभि॑ र॒स्यानु॑ । \newline
14. अ॒स्यान्वन् व॑स्या॒ स्यानु॑ प॒श्चात् प॒श्चा दन्व॑स्या॒ स्यानु॑ प॒श्चात् । \newline
15. अनु॑ प॒श्चात् प॒श्चा दन्वनु॑ प॒श्चात् क॒वयः॑ क॒वयः॑ प॒श्चा दन्वनु॑ प॒श्चात् क॒वयः॑ । \newline
16. प॒श्चात् क॒वयः॑ क॒वयः॑ प॒श्चात् प॒श्चात् क॒वयो॑ यन्ति यन्ति क॒वयः॑ प॒श्चात् प॒श्चात् क॒वयो॑ यन्ति । \newline
17. क॒वयो॑ यन्ति यन्ति क॒वयः॑ क॒वयो॑ यन्ति रे॒भा रे॒भा य॑न्ति क॒वयः॑ क॒वयो॑ यन्ति रे॒भाः । \newline
18. य॒न्ति॒ रे॒भा रे॒भा य॑न्ति यन्ति रे॒भाः । \newline
19. रे॒भा इति॑ रे॒भाः । \newline
20. उप॒ प्र प्रोपोप॒ प्रागा॑ दगा॒त् प्रोपोप॒ प्रागा᳚त् । \newline
21. प्रागा॑ दगा॒त् प्र प्रागा᳚त् पर॒मम् प॑र॒म म॑गा॒त् प्र प्रागा᳚त् पर॒मम् । \newline
22. अ॒गा॒त् प॒र॒मम् प॑र॒म म॑गा दगात् पर॒मं ॅयद् यत् प॑र॒म म॑गा दगात् पर॒मं ॅयत् । \newline
23. प॒र॒मं ॅयद् यत् प॑र॒मम् प॑र॒मं ॅयथ् स॒धस्थꣳ॑ स॒धस्थं॒ ॅयत् प॑र॒मम् प॑र॒मं ॅयथ् स॒धस्थ᳚म् । \newline
24. यथ् स॒धस्थꣳ॑ स॒धस्थं॒ ॅयद् यथ् स॒धस्थ॒ मर्वाꣳ॒॒ अर्वा᳚न् थ्स॒धस्थं॒ ॅयद् यथ् स॒धस्थ॒ मर्वान्॑ । \newline
25. स॒धस्थ॒ मर्वाꣳ॒॒ अर्वा᳚न् थ्स॒धस्थꣳ॑ स॒धस्थ॒ मर्वाꣳ॒॒ अच्छा च्छार्वा᳚न् थ्स॒धस्थꣳ॑ स॒धस्थ॒ मर्वाꣳ॒॒ अच्छ॑ । \newline
26. स॒धस्थ॒मिति॑ स॒ध - स्थ॒म् । \newline
27. अर्वाꣳ॒॒ अच्छा च्छार्वाꣳ॒॒ अर्वाꣳ॒॒ अच्छा॑ पि॒तर॑म् पि॒तर॒ मच्छार्वाꣳ॒॒ अर्वाꣳ॒॒ अच्छा॑ पि॒तर᳚म् । \newline
28. अच्छा॑ पि॒तर॑म् पि॒तर॒ मच्छाच्छा॑ पि॒तर॑म् मा॒तर॑म् मा॒तर॑म् पि॒तर॒ मच्छाच्छा॑ पि॒तर॑म् मा॒तर᳚म् । \newline
29. पि॒तर॑म् मा॒तर॑म् मा॒तर॑म् पि॒तर॑म् पि॒तर॑म् मा॒तर॑म् च च मा॒तर॑म् पि॒तर॑म् पि॒तर॑म् मा॒तर॑म् च । \newline
30. मा॒तर॑म् च च मा॒तर॑म् मा॒तर॑म् च । \newline
31. चेति॑ च । \newline
32. अ॒द्या दे॒वान् दे॒वा न॒द्याद्या दे॒वान् जुष्ट॑तमो॒ जुष्ट॑तमो दे॒वा न॒द्याद्या दे॒वान् जुष्ट॑तमः । \newline
33. दे॒वान् जुष्ट॑तमो॒ जुष्ट॑तमो दे॒वान् दे॒वान् जुष्ट॑तमो॒ हि हि जुष्ट॑तमो दे॒वान् दे॒वान् जुष्ट॑तमो॒ हि । \newline
34. जुष्ट॑तमो॒ हि हि जुष्ट॑तमो॒ जुष्ट॑तमो॒ हि ग॒म्या ग॒म्या हि जुष्ट॑तमो॒ जुष्ट॑तमो॒ हि ग॒म्याः । \newline
35. जुष्ट॑तम॒ इति॒ जुष्ट॑ - त॒मः॒ । \newline
36. हि ग॒म्या ग॒म्या हि हि ग॒म्या अथाथ॑ ग॒म्या हि हि ग॒म्या अथ॑ । \newline
37. ग॒म्या अथाथ॑ ग॒म्या ग॒म्या अथा ऽथ॑ ग॒म्या ग॒म्या अथा । \newline
38. अथा ऽथाथा शा᳚स्ते शास्त॒ आ ऽथाथा शा᳚स्ते । \newline
39. आ शा᳚स्ते शास्त॒ आ शा᳚स्ते दा॒शुषे॑ दा॒शुषे॑ शास्त॒ आ शा᳚स्ते दा॒शुषे᳚ । \newline
40. शा॒स्ते॒ दा॒शुषे॑ दा॒शुषे॑ शास्ते शास्ते दा॒शुषे॒ वार्या॑णि॒ वार्या॑णि दा॒शुषे॑ शास्ते शास्ते दा॒शुषे॒ वार्या॑णि । \newline
41. दा॒शुषे॒ वार्या॑णि॒ वार्या॑णि दा॒शुषे॑ दा॒शुषे॒ वार्या॑णि । \newline
42. वार्या॒णीति॒ वार्या॑णि । \newline
\pagebreak
\markright{ TS 4.6.8.1  \hfill https://www.vedavms.in \hfill}

\section{ TS 4.6.8.1 }

\textbf{TS 4.6.8.1 } \newline
\textbf{Samhita Paata} \newline

मा नो॑ मि॒त्रो वरु॑णो अर्य॒माऽऽयुरिन्द्र॑ ऋभु॒क्षा म॒रुतः॒ परि॑ ख्यन्न् । यद्-वा॒जिनो॑ दे॒वजा॑तस्य॒ सप्तेः᳚ प्रव॒क्ष्यामो॑ वि॒दथे॑ वी॒र्या॑णि ॥ यन्नि॒र्णिजा॒ रेक्ण॑सा॒ प्रावृ॑तस्य रा॒तिं गृ॑भी॒तां मु॑ख॒तो नय॑न्ति । सुप्रा॑ङ॒जो मेम्य॑द् वि॒श्वरू॑प इन्द्रापू॒ष्णोः प्रि॒यमप्ये॑ति॒ पाथः॑ ॥ ए॒ष च्छागः॑ पु॒रो अश्वे॑न वा॒जिना॑ पू॒ष्णो भा॒गो नी॑यते वि॒श्वदे᳚व्यः । अ॒भि॒प्रियं॒ ॅयत् पु॑रो॒डाश॒मर्व॑ता॒ त्वष्टे - [  ] \newline

\textbf{Pada Paata} \newline

मा । नः॒ । मि॒त्रः । वरु॑णः । अ॒र्य॒मा । आ॒युः । इन्द्रः॑ । ऋ॒भु॒क्षा इत्यृ॑भु - क्षाः । म॒रुतः॑ । परीति॑ । ख्य॒न्न् ॥ यत् । वा॒जिनः॑ । दे॒वजा॑त॒स्येति॑ दे॒व - जा॒त॒स्य॒ । सप्तेः᳚ । प्र॒व॒क्ष्याम॒ इति॑ प्र - व॒क्ष्यामः॑ । वि॒दथे᳚ । वी॒र्या॑णि ॥ यत् । नि॒र्णिजेति॑ निः - निजा᳚ । रेक्ण॑सा । प्रावृ॑तस्य । रा॒तिम् । गृ॒भी॒ताम् । मु॒ख॒तः । नय॑न्ति ॥ सुप्रा॒ङिति॒ सु - प्रा॒ङ् । अ॒जः । मेम्य॑त् । वि॒श्वरू॑प॒ इति॑ वि॒श्व - रू॒पः॒ । इ॒न्द्रा॒पू॒ष्णोरिती᳚न्द्रा-पू॒ष्णोः । प्रि॒यम् । अपीति॑ । ए॒ति॒ । पाथः॑ ॥ ए॒षः । छागः॑ । पु॒रः । अश्वे॑न । वा॒जिना᳚ । पू॒ष्णः । भा॒गः । नी॒य॒ते॒ । वि॒श्वदे᳚व्य॒ इति॑ वि॒श्व - दे॒व्यः॒ ॥ अ॒भि॒प्रिय॒मित्य॑भि- प्रिय᳚म् । यत् । पु॒रो॒डाश᳚म् । अर्व॑ता । त्वष्टा᳚ । इत् ।  \newline


\textbf{Krama Paata} \newline

मा नः॑ । नो॒ मि॒त्रः । मि॒त्रो वरु॑णः । वरु॑णो अर्य॒मा । अ॒र्य॒माऽऽयुः । आ॒युरिन्द्रः॑ । इन्द्र॑ ऋभु॒क्षाः । ऋ॒भु॒क्षा म॒रुतः॑ । ऋ॒भु॒क्षा इत्यृ॑भु - क्षाः । म॒रुतः॒ परि॑ । परि॑ ख्यन्न् । ख्य॒न्निति॑ ख्यन्न् ॥ यद् वा॒जिनः॑ । वा॒जिनो॑ दे॒वजा॑तस्य । दे॒वजा॑तस्य॒ सप्तेः᳚ । दे॒वजा॑त॒स्येति॑ दे॒व - जा॒त॒स्य॒ । सप्तेः᳚ प्रव॒क्ष्यामः॑ । प्र॒व॒क्ष्यामो॑ वि॒दथे᳚ । प्र॒व॒क्ष्याम॒ इति॑ प्र - व॒क्ष्यामः॑ । वि॒दथे॑ वी॒र्या॑णि । वी॒र्या॑णीति॑ वी॒र्या॑णि ॥ यन् नि॒र्णिजा᳚ । नि॒र्णिजा॒ रेक्ण॑सा । नि॒र्णिजेति॑ निः - निजा᳚ । रेक्ण॑सा॒ प्रावृ॑तस्य । प्रावृ॑तस्य रा॒तिम् । रा॒तिम् गृ॑भी॒ताम् । गृ॒भी॒ताम् मु॑ख॒तः । मु॒ख॒तो नय॑न्ति । नय॒न्तीति॒ नय॑न्ति ॥ सुप्रा॑ङ॒जः । सुप्रा॒ङिति॒ सु - प्रा॒ङ्॒ । अ॒जो मेम्य॑त् । मेम्य॑द् वि॒श्वरू॑पः । वि॒श्वरू॑प इन्द्रापू॒ष्णोः । वि॒श्वरू॑प॒ इति॑ वि॒श्व - रू॒पः॒ । इ॒न्द्रा॒पू॒ष्णोः प्रि॒यम् । इ॒न्द्रा॒पू॒ष्णोरिती᳚न्द्रा - पू॒ष्णोः । प्रि॒यमपि॑ । अप्ये॑ति । ए॒ति॒ पाथः॑ । पाथ॒ इति॒ पाथः॑ ॥ ए॒ष च्छागः॑ । छागः॑ पु॒रः । पु॒रो अश्वे॑न । अश्वे॑न वा॒जिना᳚ । वा॒जिना॑ पू॒ष्णः । पू॒ष्णो भा॒गः । भा॒गो नी॑यते । नी॒य॒ते॒ वि॒श्वदे᳚व्यः । वि॒श्वदे᳚व्य॒ इति॑ वि॒श्व - दे॒व्यः॒ ॥ अ॒भि॒प्रियं॒ ॅयत् । अ॒भि॒प्रिय॒मित्य॑भि - प्रिय᳚म् । यत् पु॑रो॒डाश᳚म् । पु॒रो॒डाश॒मर्व॑ता । अर्व॑ता॒ त्वष्टा᳚ । त्वष्टेत् । इदे॑नम् \newline

\textbf{Jatai Paata} \newline

1. मा नो॑ नो॒ मा मा नः॑ । \newline
2. नो॒ मि॒त्रो मि॒त्रो नो॑ नो मि॒त्रः । \newline
3. मि॒त्रो वरु॑णो॒ वरु॑णो मि॒त्रो मि॒त्रो वरु॑णः । \newline
4. वरु॑णो अर्य॒मा ऽर्य॒मा वरु॑णो॒ वरु॑णो अर्य॒मा । \newline
5. अ॒र्य॒मा ऽऽयु रा॒यु र॑र्य॒मा ऽर्य॒मा ऽऽयुः । \newline
6. आ॒यु रिन्द्र॒ इन्द्र॑ आ॒यु रा॒यु रिन्द्रः॑ । \newline
7. इन्द्र॑ ऋभु॒क्षा ऋ॑भु॒क्षा इन्द्र॒ इन्द्र॑ ऋभु॒क्षाः । \newline
8. ऋ॒भु॒क्षा म॒रुतो॑ म॒रुत॑ ऋभु॒क्षा ऋ॑भु॒क्षा म॒रुतः॑ । \newline
9. ऋ॒भु॒क्षा इत्यृ॑भु - क्षाः । \newline
10. म॒रुतः॒ परि॒ परि॑ म॒रुतो॑ म॒रुतः॒ परि॑ । \newline
11. परि॑ ख्यन् ख्य॒न् परि॒ परि॑ ख्यन्न् । \newline
12. ख्य॒न्निति॑ ख्यन्न् । \newline
13. यद् वा॒जिनो॑ वा॒जिनो॒ यद् यद् वा॒जिनः॑ । \newline
14. वा॒जिनो॑ दे॒वजा॑तस्य दे॒वजा॑तस्य वा॒जिनो॑ वा॒जिनो॑ दे॒वजा॑तस्य । \newline
15. दे॒वजा॑तस्य॒ सप्तेः॒ सप्ते᳚र् दे॒वजा॑तस्य दे॒वजा॑तस्य॒ सप्तेः᳚ । \newline
16. दे॒वजा॑त॒स्येति॑ दे॒व - जा॒त॒स्य॒ । \newline
17. सप्तेः᳚ प्रव॒क्ष्यामः॑ प्रव॒क्ष्यामः॒ सप्तेः॒ सप्तेः᳚ प्रव॒क्ष्यामः॑ । \newline
18. प्र॒व॒क्ष्यामो॑ वि॒दथे॑ वि॒दथे᳚ प्रव॒क्ष्यामः॑ प्रव॒क्ष्यामो॑ वि॒दथे᳚ । \newline
19. प्र॒व॒क्ष्याम॒ इति॑ प्र - व॒क्ष्यामः॑ । \newline
20. वि॒दथे॑ वी॒र्या॑णि वी॒र्या॑णि वि॒दथे॑ वि॒दथे॑ वी॒र्या॑णि । \newline
21. वी॒र्या॑णीति॑ वी॒र्या॑णि । \newline
22. यन् नि॒र्णिजा॑ नि॒र्णिजा॒ यद् यन् नि॒र्णिजा᳚ । \newline
23. नि॒र्णिजा॒ रेक्ण॑सा॒ रेक्ण॑सा नि॒र्णिजा॑ नि॒र्णिजा॒ रेक्ण॑सा । \newline
24. नि॒र्णिजेति॑ निः - निजा᳚ । \newline
25. रेक्ण॑सा॒ प्रावृ॑तस्य॒ प्रावृ॑तस्य॒ रेक्ण॑सा॒ रेक्ण॑सा॒ प्रावृ॑तस्य । \newline
26. प्रावृ॑तस्य रा॒तिꣳ रा॒तिम् प्रावृ॑तस्य॒ प्रावृ॑तस्य रा॒तिम् । \newline
27. रा॒तिम् गृ॑भी॒ताम् गृ॑भी॒ताꣳ रा॒तिꣳ रा॒तिम् गृ॑भी॒ताम् । \newline
28. गृ॒भी॒ताम् मु॑ख॒तो मु॑ख॒तो गृ॑भी॒ताम् गृ॑भी॒ताम् मु॑ख॒तः । \newline
29. मु॒ख॒तो नय॑न्ति॒ नय॑न्ति मुख॒तो मु॑ख॒तो नय॑न्ति । \newline
30. नय॒न्तीति॒ नय॑न्ति । \newline
31. सुप्रा॑ ङ॒जो अ॒जः सुप्रा॒ङ् ख्सुप्रा॑ ङ॒जः । \newline
32. सुप्रा॒ङिति॒ सु - प्रा॒ङ् । \newline
33. अ॒जो मेम्य॒न् मेम्य॑ द॒जो अ॒जो मेम्य॑त् । \newline
34. मेम्य॑द् वि॒श्वरू॑पो वि॒श्वरू॑पो॒ मेम्य॒न् मेम्य॑द् वि॒श्वरू॑पः । \newline
35. वि॒श्वरू॑प इन्द्रापू॒ष्णो रि॑न्द्रापू॒ष्णोर् वि॒श्वरू॑पो वि॒श्वरू॑प इन्द्रापू॒ष्णोः । \newline
36. वि॒श्वरू॑प॒ इति॑ वि॒श्व - रू॒पः॒ । \newline
37. इ॒न्द्रा॒पू॒ष्णोः प्रि॒यम् प्रि॒य मि॑न्द्रापू॒ष्णो रि॑न्द्रापू॒ष्णोः प्रि॒यम् । \newline
38. इ॒न्द्रा॒पू॒ष्णोरिती᳚न्द्रा - पू॒ष्णोः । \newline
39. प्रि॒य मप्यपि॑ प्रि॒यम् प्रि॒य मपि॑ । \newline
40. अप्ये᳚ त्ये॒ त्यप्यप्ये॑ति । \newline
41. ए॒ति॒ पाथः॒ पाथ॑ एत्येति॒ पाथः॑ । \newline
42. पाथ॒ इति॒ पाथः॑ । \newline
43. ए॒ष च्छाग॒ श्छाग॑ ए॒ष ए॒ष च्छागः॑ । \newline
44. छागः॑ पु॒रः पु॒र श्छाग॒ श्छागः॑ पु॒रः । \newline
45. पु॒रो अश्वे॒ना श्वे॑न पु॒रः पु॒रो अश्वे॑न । \newline
46. अश्वे॑न वा॒जिना॑ वा॒जिना ऽश्वे॒ना श्वे॑न वा॒जिना᳚ । \newline
47. वा॒जिना॑ पू॒ष्णः पू॒ष्णो वा॒जिना॑ वा॒जिना॑ पू॒ष्णः । \newline
48. पू॒ष्णो भा॒गो भा॒गः पू॒ष्णः पू॒ष्णो भा॒गः । \newline
49. भा॒गो नी॑यते नीयते भा॒गो भा॒गो नी॑यते । \newline
50. नी॒य॒ते॒ वि॒श्वदे᳚व्यो वि॒श्वदे᳚व्यो नीयते नीयते वि॒श्वदे᳚व्यः । \newline
51. वि॒श्वदे᳚व्य॒ इति॑ वि॒श्व - दे॒व्यः॒ । \newline
52. अ॒भि॒प्रियं॒ ॅयद् यद॑भि॒प्रिय॑ मभि॒प्रियं॒ ॅयत् । \newline
53. अ॒भि॒प्रिय॒मित्य॑भि - प्रिय᳚म् । \newline
54. यत् पु॑रो॒डाश॑म् पुरो॒डाशं॒ ॅयद् यत् पु॑रो॒डाश᳚म् । \newline
55. पु॒रो॒डाश॒ मर्व॒ता ऽर्व॑ता पुरो॒डाश॑म् पुरो॒डाश॒ मर्व॑ता । \newline
56. अर्व॑ता॒ त्वष्टा॒ त्वष्टा ऽर्व॒ता ऽर्व॑ता॒ त्वष्टा᳚ । \newline
57. त्वष्टेदित् त्वष्टा॒ त्वष्टेत् । \newline
58. इदे॑न मेन॒ मिदि दे॑नम् । \newline

\textbf{Ghana Paata } \newline

1. मा नो॑ नो॒ मा मा नो॑ मि॒त्रो मि॒त्रो नो॒ मा मा नो॑ मि॒त्रः । \newline
2. नो॒ मि॒त्रो मि॒त्रो नो॑ नो मि॒त्रो वरु॑णो॒ वरु॑णो मि॒त्रो नो॑ नो मि॒त्रो वरु॑णः । \newline
3. मि॒त्रो वरु॑णो॒ वरु॑णो मि॒त्रो मि॒त्रो वरु॑णो अर्य॒मा ऽर्य॒मा वरु॑णो मि॒त्रो मि॒त्रो वरु॑णो अर्य॒मा । \newline
4. वरु॑णो अर्य॒मा ऽर्य॒मा वरु॑णो॒ वरु॑णो अर्य॒मा ऽऽयु रा॒यु र॑र्य॒मा वरु॑णो॒ वरु॑णो अर्य॒मा ऽऽयुः । \newline
5. अ॒र्य॒मा ऽऽयु रा॒यु र॑र्य॒मा ऽर्य॒मा ऽऽयु रिन्द्र॒ इन्द्र॑ आ॒यु र॑र्य॒मा ऽर्य॒मा ऽऽयु रिन्द्रः॑ । \newline
6. आ॒यु रिन्द्र॒ इन्द्र॑ आ॒यु रा॒यु रिन्द्र॑ ऋभु॒क्षा ऋ॑भु॒क्षा इन्द्र॑ आ॒यु रा॒यु रिन्द्र॑ ऋभु॒क्षाः । \newline
7. इन्द्र॑ ऋभु॒क्षा ऋ॑भु॒क्षा इन्द्र॒ इन्द्र॑ ऋभु॒क्षा म॒रुतो॑ म॒रुत॑ ऋभु॒क्षा इन्द्र॒ इन्द्र॑ ऋभु॒क्षा म॒रुतः॑ । \newline
8. ऋ॒भु॒क्षा म॒रुतो॑ म॒रुत॑ ऋभु॒क्षा ऋ॑भु॒क्षा म॒रुतः॒ परि॒ परि॑ म॒रुत॑ ऋभु॒क्षा ऋ॑भु॒क्षा म॒रुतः॒ परि॑ । \newline
9. ऋ॒भु॒क्षा इत्यृ॑भु - क्षाः । \newline
10. म॒रुतः॒ परि॒ परि॑ म॒रुतो॑ म॒रुतः॒ परि॑ ख्यन् ख्य॒न् परि॑ म॒रुतो॑ म॒रुतः॒ परि॑ ख्यन्न् । \newline
11. परि॑ ख्यन् ख्य॒न् परि॒ परि॑ ख्यन्न् । \newline
12. ख्य॒न्निति॑ ख्यन्न् । \newline
13. यद् वा॒जिनो॑ वा॒जिनो॒ यद् यद् वा॒जिनो॑ दे॒वजा॑तस्य दे॒वजा॑तस्य वा॒जिनो॒ यद् यद् वा॒जिनो॑ दे॒वजा॑तस्य । \newline
14. वा॒जिनो॑ दे॒वजा॑तस्य दे॒वजा॑तस्य वा॒जिनो॑ वा॒जिनो॑ दे॒वजा॑तस्य॒ सप्तेः॒ सप्ते᳚र् दे॒वजा॑तस्य वा॒जिनो॑ वा॒जिनो॑ दे॒वजा॑तस्य॒ सप्तेः᳚ । \newline
15. दे॒वजा॑तस्य॒ सप्तेः॒ सप्ते᳚र् दे॒वजा॑तस्य दे॒वजा॑तस्य॒ सप्तेः᳚ प्रव॒क्ष्यामः॑ प्रव॒क्ष्यामः॒ सप्ते᳚र् दे॒वजा॑तस्य दे॒वजा॑तस्य॒ सप्तेः᳚ प्रव॒क्ष्यामः॑ । \newline
16. दे॒वजा॑त॒स्येति॑ दे॒व - जा॒त॒स्य॒ । \newline
17. सप्तेः᳚ प्रव॒क्ष्यामः॑ प्रव॒क्ष्यामः॒ सप्तेः॒ सप्तेः᳚ प्रव॒क्ष्यामो॑ वि॒दथे॑ वि॒दथे᳚ प्रव॒क्ष्यामः॒ सप्तेः॒ सप्तेः᳚ प्रव॒क्ष्यामो॑ वि॒दथे᳚ । \newline
18. प्र॒व॒क्ष्यामो॑ वि॒दथे॑ वि॒दथे᳚ प्रव॒क्ष्यामः॑ प्रव॒क्ष्यामो॑ वि॒दथे॑ वी॒र्या॑णि वी॒र्या॑णि वि॒दथे᳚ प्रव॒क्ष्यामः॑ प्रव॒क्ष्यामो॑ वि॒दथे॑ वी॒र्या॑णि । \newline
19. प्र॒व॒क्ष्याम॒ इति॑ प्र - व॒क्ष्यामः॑ । \newline
20. वि॒दथे॑ वी॒र्या॑णि वी॒र्या॑णि वि॒दथे॑ वि॒दथे॑ वी॒र्या॑णि । \newline
21. वी॒र्या॑णीति॑ वी॒र्या॑णि । \newline
22. यन् नि॒र्णिजा॑ नि॒र्णिजा॒ यद् यन् नि॒र्णिजा॒ रेक्ण॑सा॒ रेक्ण॑सा नि॒र्णिजा॒ यद् यन् नि॒र्णिजा॒ रेक्ण॑सा । \newline
23. नि॒र्णिजा॒ रेक्ण॑सा॒ रेक्ण॑सा नि॒र्णिजा॑ नि॒र्णिजा॒ रेक्ण॑सा॒ प्रावृ॑तस्य॒ प्रावृ॑तस्य॒ रेक्ण॑सा नि॒र्णिजा॑ नि॒र्णिजा॒ रेक्ण॑सा॒ प्रावृ॑तस्य । \newline
24. नि॒र्णिजेति॑ निः - निजा᳚ । \newline
25. रेक्ण॑सा॒ प्रावृ॑तस्य॒ प्रावृ॑तस्य॒ रेक्ण॑सा॒ रेक्ण॑सा॒ प्रावृ॑तस्य रा॒तिꣳ रा॒तिम् प्रावृ॑तस्य॒ रेक्ण॑सा॒ रेक्ण॑सा॒ प्रावृ॑तस्य रा॒तिम् । \newline
26. प्रावृ॑तस्य रा॒तिꣳ रा॒तिम् प्रावृ॑तस्य॒ प्रावृ॑तस्य रा॒तिम् गृ॑भी॒ताम् गृ॑भी॒ताꣳ रा॒तिम् प्रावृ॑तस्य॒ प्रावृ॑तस्य रा॒तिम् गृ॑भी॒ताम् । \newline
27. रा॒तिम् गृ॑भी॒ताम् गृ॑भी॒ताꣳ रा॒तिꣳ रा॒तिम् गृ॑भी॒ताम् मु॑ख॒तो मु॑ख॒तो गृ॑भी॒ताꣳ रा॒तिꣳ रा॒तिम् गृ॑भी॒ताम् मु॑ख॒तः । \newline
28. गृ॒भी॒ताम् मु॑ख॒तो मु॑ख॒तो गृ॑भी॒ताम् गृ॑भी॒ताम् मु॑ख॒तो नय॑न्ति॒ नय॑न्ति मुख॒तो गृ॑भी॒ताम् गृ॑भी॒ताम् मु॑ख॒तो नय॑न्ति । \newline
29. मु॒ख॒तो नय॑न्ति॒ नय॑न्ति मुख॒तो मु॑ख॒तो नय॑न्ति । \newline
30. नय॒न्तीति॒ नय॑न्ति । \newline
31. सुप्रा॑ ङ॒जो अ॒जः सुप्रा॒ङ् ख्सुप्रा॑ ङ॒जो मेम्य॒न् मेम्य॑ द॒जः सुप्रा॒ङ् ख्सुप्रा॑ ङ॒जो मेम्य॑त् । \newline
32. सुप्रा॒ङिति॒ सु - प्रा॒ङ् । \newline
33. अ॒जो मेम्य॒न् मेम्य॑ द॒जो अ॒जो मेम्य॑द् वि॒श्वरू॑पो वि॒श्वरू॑पो॒ मेम्य॑ द॒जो अ॒जो मेम्य॑द् वि॒श्वरू॑पः । \newline
34. मेम्य॑द् वि॒श्वरू॑पो वि॒श्वरू॑पो॒ मेम्य॒न् मेम्य॑द् वि॒श्वरू॑प इन्द्रापू॒ष्णो रि॑न्द्रापू॒ष्णोर् वि॒श्वरू॑पो॒ मेम्य॒न् मेम्य॑द् वि॒श्वरू॑प इन्द्रापू॒ष्णोः । \newline
35. वि॒श्वरू॑प इन्द्रापू॒ष्णो रि॑न्द्रापू॒ष्णोर् वि॒श्वरू॑पो वि॒श्वरू॑प इन्द्रापू॒ष्णोः प्रि॒यम् प्रि॒य मि॑न्द्रापू॒ष्णोर् वि॒श्वरू॑पो वि॒श्वरू॑प इन्द्रापू॒ष्णोः प्रि॒यम् । \newline
36. वि॒श्वरू॑प॒ इति॑ वि॒श्व - रू॒पः॒ । \newline
37. इ॒न्द्रा॒पू॒ष्णोः प्रि॒यम् प्रि॒य मि॑न्द्रापू॒ष्णो रि॑न्द्रापू॒ष्णोः प्रि॒य मप्यपि॑ प्रि॒य मि॑न्द्रापू॒ष्णो रि॑न्द्रापू॒ष्णोः प्रि॒य मपि॑ । \newline
38. इ॒न्द्रा॒पू॒ष्णोरिती᳚न्द्रा - पू॒ष्णोः । \newline
39. प्रि॒य मप्यपि॑ प्रि॒यम् प्रि॒य मप्ये᳚ त्ये॒त्यपि॑ प्रि॒यम् प्रि॒य मप्ये॑ति । \newline
40. अप्ये᳚ त्ये॒त्यप्य प्ये॑ति॒ पाथः॒ पाथ॑ ए॒त्य प्यप्ये॑ति॒ पाथः॑ । \newline
41. ए॒ति॒ पाथः॒ पाथ॑ एत्येति॒ पाथः॑ । \newline
42. पाथ॒ इति॒ पाथः॑ । \newline
43. ए॒ष च्छाग॒ श्छाग॑ ए॒ष ए॒ष च्छागः॑ पु॒रः पु॒र श्छाग॑ ए॒ष ए॒ष च्छागः॑ पु॒रः । \newline
44. छागः॑ पु॒रः पु॒र श्छाग॒ श्छागः॑ पु॒रो अश्वे॒ना श्वे॑न पु॒र श्छाग॒ श्छागः॑ पु॒रो अश्वे॑न । \newline
45. पु॒रो अश्वे॒ना श्वे॑न पु॒रः पु॒रो अश्वे॑न वा॒जिना॑ वा॒जिना ऽश्वे॑न पु॒रः पु॒रो अश्वे॑न वा॒जिना᳚ । \newline
46. अश्वे॑न वा॒जिना॑ वा॒जिना ऽश्वे॒ना श्वे॑न वा॒जिना॑ पू॒ष्णः पू॒ष्णो वा॒जिना ऽश्वे॒ना श्वे॑न वा॒जिना॑ पू॒ष्णः । \newline
47. वा॒जिना॑ पू॒ष्णः पू॒ष्णो वा॒जिना॑ वा॒जिना॑ पू॒ष्णो भा॒गो भा॒गः पू॒ष्णो वा॒जिना॑ वा॒जिना॑ पू॒ष्णो भा॒गः । \newline
48. पू॒ष्णो भा॒गो भा॒गः पू॒ष्णः पू॒ष्णो भा॒गो नी॑यते नीयते भा॒गः पू॒ष्णः पू॒ष्णो भा॒गो नी॑यते । \newline
49. भा॒गो नी॑यते नीयते भा॒गो भा॒गो नी॑यते वि॒श्वदे᳚व्यो वि॒श्वदे᳚व्यो नीयते भा॒गो भा॒गो नी॑यते वि॒श्वदे᳚व्यः । \newline
50. नी॒य॒ते॒ वि॒श्वदे᳚व्यो वि॒श्वदे᳚व्यो नीयते नीयते वि॒श्वदे᳚व्यः । \newline
51. वि॒श्वदे᳚व्य॒ इति॑ वि॒श्व - दे॒व्यः॒ । \newline
52. अ॒भि॒प्रियं॒ ॅयद् यद॑भि॒प्रिय॑ मभि॒प्रियं॒ ॅयत् पु॑रो॒डाश॑म् पुरो॒डाशं॒ ॅयद॑भि॒प्रिय॑ मभि॒प्रियं॒ ॅयत् पु॑रो॒डाश᳚म् । \newline
53. अ॒भि॒प्रिय॒मित्य॑भि - प्रिय᳚म् । \newline
54. यत् पु॑रो॒डाश॑म् पुरो॒डाशं॒ ॅयद् यत् पु॑रो॒डाश॒ मर्व॒ता ऽर्व॑ता पुरो॒डाशं॒ ॅयद् यत् पु॑रो॒डाश॒ मर्व॑ता । \newline
55. पु॒रो॒डाश॒ मर्व॒ता ऽर्व॑ता पुरो॒डाश॑म् पुरो॒डाश॒ मर्व॑ता॒ त्वष्टा॒ त्वष्टा ऽर्व॑ता पुरो॒डाश॑म् पुरो॒डाश॒ मर्व॑ता॒ त्वष्टा᳚ । \newline
56. अर्व॑ता॒ त्वष्टा॒ त्वष्टा ऽर्व॒ता ऽर्व॑ता॒ त्वष्टेदित् त्वष्टा ऽर्व॒ता ऽर्व॑ता॒ त्वष्टेत् । \newline
57. त्वष्टेदित् त्वष्टा॒ त्वष्टे दे॑न मेन॒ मित् त्वष्टा॒ त्वष्टे दे॑नम् । \newline
58. इदे॑न मेन॒ मिदि दे॑नꣳ सौश्रव॒साय॑ सौश्रव॒सायै॑न॒ मिदिद् ए॑नꣳ सौश्रव॒साय॑ । \newline
\pagebreak
\markright{ TS 4.6.8.2  \hfill https://www.vedavms.in \hfill}

\section{ TS 4.6.8.2 }

\textbf{TS 4.6.8.2 } \newline
\textbf{Samhita Paata} \newline

-दे॑नꣳ सौश्रव॒साय॑ जिन्वति ॥ यद्ध॒विष्य॑मृतु॒शो दे॑व॒यानं॒ त्रिर्मानु॑षाः॒ पर्यश्वं॒ नय॑न्ति । अत्रा॑ पू॒ष्णः प्र॑थ॒मो भा॒ग ए॑ति य॒ज्ञ्ं दे॒वेभ्यः॑ प्रतिवे॒दय॑न्न॒जः ॥ होता᳚ऽद्ध्व॒र्युराव॑या अग्निमि॒न्धो ग्रा॑वग्रा॒भ उ॒त शꣳस्ता॒ सुवि॑प्रः । तेन॑ य॒ज्ञेन॒ स्व॑रंकृतेन॒ स्वि॑ष्टेन व॒क्षणा॒ आ पृ॑णद्ध्वं ॥ यू॒प॒व्र॒स्का उ॒त ये यू॑पवा॒हाश्च॒षालं॒ ॅये अ॑श्वयू॒पाय॒ तक्ष॑ति । ये चार्व॑ते॒ पच॑नꣳ स॒भंर॑न्त्यु॒तो - [  ] \newline

\textbf{Pada Paata} \newline

ए॒न॒म् । सौ॒श्र॒व॒साय॑ । जि॒न्व॒ति॒ ॥ यत् । ह॒विष्य᳚म् । ऋ॒तु॒श इत्यृ॑तु - शः । दे॒व॒यान॒मिति॑ देव - यान᳚म् । त्रिः । मानु॑षाः । परीति॑ । अश्व᳚म् । नय॑न्ति ॥ अत्र॑ । पू॒ष्णः । प्र॒थ॒मः । भा॒गः । ए॒ति॒ । य॒ज्ञ्म् । दे॒वेभ्यः॑ । प्र॒ति॒वे॒दय॒न्निति॑ प्रति - वे॒दयन्न्॑ । अ॒जः ॥ होता᳚ । अ॒द्ध्व॒र्युः । आव॑या॒ इत्या - व॒याः॒ । अ॒ग्नि॒मि॒न्ध इत्य॑ग्निं - इ॒न्धः । ग्रा॒व॒ग्रा॒भ इति॑ ग्राव - ग्रा॒भः । उ॒त । शꣳस्ता᳚ । सुवि॑प्र॒ इति॒ सु - वि॒प्रः॒ ॥ तेन॑ । य॒ज्ञेन॑ । स्व॑रंकृते॒नेति॒ सु - अ॒र॒कृं॒ते॒न॒ । स्वि॑ष्टे॒नेति॒ सु - इ॒ष्टे॒न॒ । व॒क्षणाः᳚ । एति॑ । पृ॒ण॒द्ध्व॒म् ॥ यू॒प॒व्र॒स्का इति॑ यूप - व्र॒स्काः । उ॒त । ये । यू॒प॒वा॒हा इति॑ यूप - वा॒हाः । च॒षाल᳚म् । ये । अ॒श्व॒यू॒पायेत्य॑श्व-यू॒पाय॑ । तक्ष॑ति ॥ ये । च॒ । अर्व॑ते । पच॑नम् । स॒भंर॒न्तीति॑ सं - भर॑न्ति । उ॒तो इति॑ ।  \newline


\textbf{Krama Paata} \newline

ए॒नꣳ॒॒ सौ॒श्र॒व॒साय॑ । सौ॒श्र॒व॒साय॑ जिन्वति । जि॒न्व॒तीति॑ जिन्वति ॥ यद्ध॒विष्य᳚म् । ह॒विष्य॑मृतु॒शः । ऋ॒तु॒शो दे॑व॒यान᳚म् । ऋ॒तु॒श इत्यृ॑तु - शः । दे॒व॒यान॒म् त्रिः । दे॒व॒यान॒मिति॑ देव - यान᳚म् । त्रिर् मानु॑षाः । मानु॑षाः॒ परि॑ । पर्यश्व᳚म् । अश्व॒म् नय॑न्ति । नय॒न्तीति॒ नय॑न्ति ॥ अत्रा॑ पू॒ष्णः । पू॒ष्णः प्र॑थ॒मः । प्र॒थ॒मो भा॒गः । भा॒ग ए॑ति । ए॒ति॒ य॒ज्ञ्म् । य॒ज्ञ्म् दे॒वेभ्यः॑ । दे॒वेभ्यः॑ प्रतिवे॒दयन्न्॑ । प्र॒ति॒वे॒दय॑न्न॒जः । प्र॒ति॒वे॒दय॒न्निति॑ प्रति - वे॒दयन्न्॑ । अ॒ज इत्य॒जः ॥ होता᳚ऽद्ध्व॒र्युः । अ॒द्ध्व॒र्युराव॑याः । आव॑या अग्निमि॒न्धः । आव॑या॒ इत्या - व॒याः॒ । अ॒ग्नि॒मि॒न्धो ग्रा॑वग्रा॒भः । अ॒ग्नि॒मि॒न्ध इत्य॑ग्निम् - इ॒न्धः । ग्रा॒व॒ग्रा॒भ उ॒त । ग्रा॒व॒ग्रा॒भ इति॑ ग्राव - ग्रा॒भः । उ॒त शꣳस्ता᳚ । शꣳस्ता॒ सुवि॑प्रः । सुवि॑प्र॒ इति॒ सु - वि॒प्रः॒ ॥ तेन॑ य॒ज्ञेन॑ । य॒ज्ञेन॒ स्व॑रङ्कृतेन । स्व॑रङ्कृतेन॒ स्वि॑ष्टेन । स्व॑रङ्कृते॒नेति॒ सु - अ॒र॒ङ्कृ॒ते॒न॒ । स्वि॑ष्टेन व॒क्षणाः᳚ । स्वि॑ष्टे॒नेति॒ सु - इ॒ष्टे॒न॒ । व॒क्षणा॒ आ । आ पृ॑णद्ध्वम् । पृ॒ण॒द्ध्व॒मिति॑ पृणद्ध्वम् ॥ यू॒प॒व्र॒स्का उ॒त । यू॒प॒व्र॒स्का इति॑ यूप - व्र॒स्काः । उ॒त ये । ये यू॑पवा॒हाः । यू॒प॒वा॒हाश्च॒षाल᳚म् । यू॒प॒वा॒हा इति॑ यूप - वा॒हाः । च॒षालं॒ ॅये । ये अ॑श्वयू॒पाय॑ । अ॒श्व॒यू॒पाय॒ तक्ष॑ति । अ॒श्व॒यू॒पायेत्य॑श्व - यू॒पाय॑ । तक्ष॒तीति॒ तक्ष॑ति ॥ ये च॑ । चार्व॑ते । अर्व॑ते॒ पच॑नम् । पच॑नꣳ स॒म्भर॑न्ति । स॒म्भर॑न्त्यु॒तो । स॒म्भर॒न्तीति॑ सम् - भर॑न्ति । उ॒तो तेषा᳚म् । उ॒तो इत्यु॒तो \newline

\textbf{Jatai Paata} \newline

1. ए॒नꣳ॒॒ सौ॒श्र॒व॒साय॑ सौश्रव॒सायै॑न मेनꣳ सौश्रव॒साय॑ । \newline
2. सौ॒श्र॒व॒साय॑ जिन्वति जिन्वति सौश्रव॒साय॑ सौश्रव॒साय॑ जिन्वति । \newline
3. जि॒न्व॒तीति॑ जिन्वति । \newline
4. यद्ध॒विष्यꣳ॑ ह॒विष्यं॒ ॅयद् यद्ध॒विष्य᳚म् । \newline
5. ह॒विष्य॑ मृतु॒श ऋ॑तु॒शो ह॒विष्यꣳ॑ ह॒विष्य॑ मृतु॒शः । \newline
6. ऋ॒तु॒शो दे॑व॒यान॑म् देव॒यान॑ मृतु॒श ऋ॑तु॒शो दे॑व॒यान᳚म् । \newline
7. ऋ॒तु॒श इत्यृ॑तु - शः । \newline
8. दे॒व॒यान॒म् त्रिस्त्रिर् दे॑व॒यान॑म् देव॒यान॒म् त्रिः । \newline
9. दे॒व॒यान॒मिति॑ देव - यान᳚म् । \newline
10. त्रिर् मानु॑षा॒ मानु॑षा॒ स्त्रि स्त्रिर् मानु॑षाः । \newline
11. मानु॑षाः॒ परि॒ परि॒ मानु॑षा॒ मानु॑षाः॒ परि॑ । \newline
12. पर्यश्व॒ मश्व॒म् परि॒ पर्यश्व᳚म् । \newline
13. अश्व॒म् नय॑न्ति॒ नय॒ न्त्यश्व॒ मश्व॒म् नय॑न्ति । \newline
14. नय॒न्तीति॒ नय॑न्ति । \newline
15. अत्रा॑ पू॒ष्णः पू॒ष्णो अत्रात्रा॑ पू॒ष्णः । \newline
16. पू॒ष्णः प्र॑थ॒मः प्र॑थ॒मः पू॒ष्णः पू॒ष्णः प्र॑थ॒मः । \newline
17. प्र॒थ॒मो भा॒गो भा॒गः प्र॑थ॒मः प्र॑थ॒मो भा॒गः । \newline
18. भा॒ग ए᳚त्येति भा॒गो भा॒ग ए॑ति । \newline
19. ए॒ति॒ य॒ज्ञ्ं ॅय॒ज्ञ् मे᳚त्येति य॒ज्ञ्म् । \newline
20. य॒ज्ञ्म् दे॒वेभ्यो॑ दे॒वेभ्यो॑ य॒ज्ञ्ं ॅय॒ज्ञ्म् दे॒वेभ्यः॑ । \newline
21. दे॒वेभ्यः॑ प्रतिवे॒दय॑न् प्रतिवे॒दय॑न् दे॒वेभ्यो॑ दे॒वेभ्यः॑ प्रतिवे॒दयन्न्॑ । \newline
22. प्र॒ति॒वे॒दय॑न् न॒जो अ॒जः प्र॑तिवे॒दय॑न् प्रतिवे॒दय॑न् न॒जः । \newline
23. प्र॒ति॒वे॒दय॒न्निति॑ प्रति - वे॒दयन्न्॑ । \newline
24. अ॒ज इत्य॒जः । \newline
25. होता᳚ ऽद्ध्व॒र्यु र॑द्ध्व॒र्युर्. होता॒ होता᳚ ऽद्ध्व॒र्युः । \newline
26. अ॒द्ध्व॒र्यु राव॑या॒ आव॑या अद्ध्व॒र्यु र॑द्ध्व॒र्यु राव॑याः । \newline
27. आव॑या अग्निमि॒न्धो अ॑ग्निमि॒न्ध आव॑या॒ आव॑या अग्निमि॒न्धः । \newline
28. आव॑या॒ इत्या - व॒याः॒ । \newline
29. अ॒ग्नि॒मि॒न्धो ग्रा॑वग्रा॒भो ग्रा॑वग्रा॒भो अ॑ग्निमि॒न्धो अ॑ग्निमि॒न्धो ग्रा॑वग्रा॒भः । \newline
30. अ॒ग्नि॒मि॒न्ध इत्य॑ग्निं - इ॒न्धः । \newline
31. ग्रा॒व॒ग्रा॒भ उ॒तोत ग्रा॑वग्रा॒भो ग्रा॑वग्रा॒भ उ॒त । \newline
32. ग्रा॒व॒ग्रा॒भ इति॑ ग्राव - ग्रा॒भः । \newline
33. उ॒त शꣳस्ता॒ शꣳस्तो॒तोत शꣳस्ता᳚ । \newline
34. शꣳस्ता॒ सुवि॑प्रः॒ सुवि॑प्रः॒ शꣳस्ता॒ शꣳस्ता॒ सुवि॑प्रः । \newline
35. सुवि॑प्र॒ इति॒ सु - वि॒प्रः॒ । \newline
36. तेन॑ य॒ज्ञेन॑ य॒ज्ञेन॒ तेन॒ तेन॑ य॒ज्ञेन॑ । \newline
37. य॒ज्ञेन॒ स्व॑रंकृतेन॒ स्व॑रंकृतेन य॒ज्ञेन॑ य॒ज्ञेन॒ स्व॑रंकृतेन । \newline
38. स्व॑रंकृतेन॒ स्वि॑ष्टेन॒ स्वि॑ष्टेन॒ स्व॑रंकृतेन॒ स्व॑रंकृतेन॒ स्वि॑ष्टेन । \newline
39. स्व॑रंकृते॒नेति॒ सु - अ॒रं॒कृ॒ते॒न॒ । \newline
40. स्वि॑ष्टेन व॒क्षणा॑ व॒क्षणाः॒ स्वि॑ष्टेन॒ स्वि॑ष्टेन व॒क्षणाः᳚ । \newline
41. स्वि॑ष्टे॒नेति॒ सु - इ॒ष्टे॒न॒ । \newline
42. व॒क्षणा॒ आ व॒क्षणा॑ व॒क्षणा॒ आ । \newline
43. आ पृ॑णद्ध्वम् पृणद्ध्व॒ मा पृ॑णद्ध्वम् । \newline
44. पृ॒ण॒द्ध्व॒मिति॑ पृणद्ध्वम् । \newline
45. यू॒प॒व्र॒स्का उ॒तोत यू॑पव्र॒स्का यू॑पव्र॒स्का उ॒त । \newline
46. यू॒प॒व्र॒स्का इति॑ यूप - व्र॒स्काः । \newline
47. उ॒त ये य उ॒तोत ये । \newline
48. ये यू॑पवा॒हा यू॑पवा॒हा ये ये यू॑पवा॒हाः । \newline
49. यू॒प॒वा॒हा श्च॒षाल॑म् च॒षालं॑ ॅयूपवा॒हा यू॑पवा॒हा श्च॒षाल᳚म् । \newline
50. यू॒प॒वा॒हा इति॑ यूप - वा॒हाः । \newline
51. च॒षालं॒ ॅये ये च॒षाल॑म् च॒षालं॒ ॅये । \newline
52. ये अ॑श्वयू॒पाया᳚ श्वयू॒पाय॒ ये ये अ॑श्वयू॒पाय॑ । \newline
53. अ॒श्व॒यू॒पाय॒ तक्ष॑ति॒ तक्ष॑ त्यश्वयू॒पाया᳚ श्वयू॒पाय॒ तक्ष॑ति । \newline
54. अ॒श्व॒यू॒पायेत्य॑श्व - यू॒पाय॑ । \newline
55. तक्ष॒तीति॒ तक्ष॑ति । \newline
56. ये च॑ च॒ ये ये च॑ । \newline
57. चार्व॑ते॒ अर्व॑ते च॒ चार्व॑ते । \newline
58. अर्व॑ते॒ पच॑न॒म् पच॑न॒ मर्व॑ते॒ अर्व॑ते॒ पच॑नम् । \newline
59. पच॑नꣳ सं॒भर॑न्ति सं॒भर॑न्ति॒ पच॑न॒म् पच॑नꣳ सं॒भर॑न्ति । \newline
60. सं॒भर॑ न्त्यु॒तो उ॒तो सं॒भर॑न्ति सं॒भर॑ न्त्यु॒तो । \newline
61. सं॒भर॒न्तीति॑ सं - भर॑न्ति । \newline
62. उ॒तो तेषा॒म् तेषा॑ मु॒तो उ॒तो तेषा᳚म् । \newline
63. उ॒तो इत्यु॒तो । \newline

\textbf{Ghana Paata } \newline

1. ए॒नꣳ॒॒ सौ॒श्र॒व॒साय॑ सौश्रव॒सायै॑न मेनꣳ सौश्रव॒साय॑ जिन्वति जिन्वति सौश्रव॒सायै॑न मेनꣳ सौश्रव॒साय॑ जिन्वति । \newline
2. सौ॒श्र॒व॒साय॑ जिन्वति जिन्वति सौश्रव॒साय॑ सौश्रव॒साय॑ जिन्वति । \newline
3. जि॒न्व॒तीति॑ जिन्वति । \newline
4. यद्ध॒विष्यꣳ॑ ह॒विष्यं॒ ॅयद् यद्ध॒विष्य॑ मृतु॒श ऋ॑तु॒शो ह॒विष्यं॒ ॅयद् यद्ध॒विष्य॑ मृतु॒शः । \newline
5. ह॒विष्य॑ मृतु॒श ऋ॑तु॒शो ह॒विष्यꣳ॑ ह॒विष्य॑ मृतु॒शो दे॑व॒यान॑म् देव॒यान॑ मृतु॒शो ह॒विष्यꣳ॑ ह॒विष्य॑ मृतु॒शो दे॑व॒यान᳚म् । \newline
6. ऋ॒तु॒शो दे॑व॒यान॑म् देव॒यान॑ मृतु॒श ऋ॑तु॒शो दे॑व॒यान॒म् त्रि स्त्रिर् दे॑व॒यान॑ मृतु॒श ऋ॑तु॒शो दे॑व॒यान॒म् त्रिः । \newline
7. ऋ॒तु॒श इत्यृ॑तु - शः । \newline
8. दे॒व॒यान॒म् त्रि स्त्रिर् दे॑व॒यान॑म् देव॒यान॒म् त्रिर् मानु॑षा॒ मानु॑षा॒ स्त्रिर् दे॑व॒यान॑म् देव॒यान॒म् त्रिर् मानु॑षाः । \newline
9. दे॒व॒यान॒मिति॑ देव - यान᳚म् । \newline
10. त्रिर् मानु॑षा॒ मानु॑षा॒ स्त्रि स्त्रिर् मानु॑षाः॒ परि॒ परि॒ मानु॑षा॒ स्त्रि स्त्रिर् मानु॑षाः॒ परि॑ । \newline
11. मानु॑षाः॒ परि॒ परि॒ मानु॑षा॒ मानु॑षाः॒ पर्यश्व॒ मश्व॒म् परि॒ मानु॑षा॒ मानु॑षाः॒ पर्यश्व᳚म् । \newline
12. पर्यश्व॒ मश्व॒म् परि॒ पर्यश्व॒म् नय॑न्ति॒ नय॒न् त्यश्व॒म् परि॒ पर्यश्व॒म् नय॑न्ति । \newline
13. अश्व॒म् नय॑न्ति॒ नय॒न् त्यश्व॒ मश्व॒म् नय॑न्ति । \newline
14. नय॒न्तीति॒ नय॑न्ति । \newline
15. अत्रा॑ पू॒ष्णः पू॒ष्णो अत्रात्रा॑ पू॒ष्णः प्र॑थ॒मः प्र॑थ॒मः पू॒ष्णो अत्रात्रा॑ पू॒ष्णः प्र॑थ॒मः । \newline
16. पू॒ष्णः प्र॑थ॒मः प्र॑थ॒मः पू॒ष्णः पू॒ष्णः प्र॑थ॒मो भा॒गो भा॒गः प्र॑थ॒मः पू॒ष्णः पू॒ष्णः प्र॑थ॒मो भा॒गः । \newline
17. प्र॒थ॒मो भा॒गो भा॒गः प्र॑थ॒मः प्र॑थ॒मो भा॒ग ए᳚त्येति भा॒गः प्र॑थ॒मः प्र॑थ॒मो भा॒ग ए॑ति । \newline
18. भा॒ग ए᳚त्येति भा॒गो भा॒ग ए॑ति य॒ज्ञ्ं ॅय॒ज्ञ् मे॑ति भा॒गो भा॒ग ए॑ति य॒ज्ञ्म् । \newline
19. ए॒ति॒ य॒ज्ञ्ं ॅय॒ज्ञ् मे᳚त्येति य॒ज्ञ्म् दे॒वेभ्यो॑ दे॒वेभ्यो॑ य॒ज्ञ् मे᳚त्येति य॒ज्ञ्म् दे॒वेभ्यः॑ । \newline
20. य॒ज्ञ्म् दे॒वेभ्यो॑ दे॒वेभ्यो॑ य॒ज्ञ्ं ॅय॒ज्ञ्म् दे॒वेभ्यः॑ प्रतिवे॒दय॑न् प्रतिवे॒दय॑न् दे॒वेभ्यो॑ य॒ज्ञ्ं ॅय॒ज्ञ्म् दे॒वेभ्यः॑ प्रतिवे॒दयन्न्॑ । \newline
21. दे॒वेभ्यः॑ प्रतिवे॒दय॑न् प्रतिवे॒दय॑न् दे॒वेभ्यो॑ दे॒वेभ्यः॑ प्रतिवे॒दय॑न् न॒जो अ॒जः प्र॑तिवे॒दय॑न् दे॒वेभ्यो॑ दे॒वेभ्यः॑ प्रतिवे॒दय॑न् न॒जः । \newline
22. प्र॒ति॒वे॒दय॑न् न॒जो अ॒जः प्र॑तिवे॒दय॑न् प्रतिवे॒दय॑न् न॒जः । \newline
23. प्र॒ति॒वे॒दय॒न्निति॑ प्रति - वे॒दयन्न्॑ । \newline
24. अ॒ज इत्य॒जः । \newline
25. होता᳚ ऽद्ध्व॒र्यु र॑द्ध्व॒र्युर्. होता॒ होता᳚ ऽद्ध्व॒र्यु राव॑या॒ आव॑या अद्ध्व॒र्युर्. होता॒ होता᳚ ऽद्ध्व॒र्यु राव॑याः । \newline
26. अ॒द्ध्व॒र्यु राव॑या॒ आव॑या अद्ध्व॒र्यु र॑द्ध्व॒र्यु राव॑या अग्निमि॒न्धो अ॑ग्निमि॒न्ध आव॑या अद्ध्व॒र्यु र॑द्ध्व॒र्यु राव॑या अग्निमि॒न्धः । \newline
27. आव॑या अग्निमि॒न्धो अ॑ग्निमि॒न्ध आव॑या॒ आव॑या अग्निमि॒न्धो ग्रा॑वग्रा॒भो ग्रा॑वग्रा॒भो अ॑ग्निमि॒न्ध आव॑या॒ आव॑या अग्निमि॒न्धो ग्रा॑वग्रा॒भः । \newline
28. आव॑या॒ इत्या - व॒याः॒ । \newline
29. अ॒ग्नि॒मि॒न्धो ग्रा॑वग्रा॒भो ग्रा॑वग्रा॒भो अ॑ग्निमि॒न्धो अ॑ग्निमि॒न्धो ग्रा॑वग्रा॒भ उ॒तोत ग्रा॑वग्रा॒भो 
अ॑ग्निमि॒न्धो अ॑ग्निमि॒न्धो ग्रा॑वग्रा॒भ उ॒त । \newline
30. अ॒ग्नि॒मि॒न्ध इत्य॑ग्निं - इ॒न्धः । \newline
31. ग्रा॒व॒ग्रा॒भ उ॒तोत ग्रा॑वग्रा॒भो ग्रा॑वग्रा॒भ उ॒त शꣳस्ता॒ शꣳस्तो॒त ग्रा॑वग्रा॒भो ग्रा॑वग्रा॒भ उ॒त शꣳस्ता᳚ । \newline
32. ग्रा॒व॒ग्रा॒भ इति॑ ग्राव - ग्रा॒भः । \newline
33. उ॒त शꣳस्ता॒ शꣳस्तो॒ तोत शꣳस्ता॒ सुवि॑प्रः॒ सुवि॑प्रः॒ शꣳस्तो॒ तोत शꣳस्ता॒ सुवि॑प्रः । \newline
34. शꣳस्ता॒ सुवि॑प्रः॒ सुवि॑प्रः॒ शꣳस्ता॒ शꣳस्ता॒ सुवि॑प्रः । \newline
35. सुवि॑प्र॒ इति॒ सु - वि॒प्रः॒ । \newline
36. तेन॑ य॒ज्ञेन॑ य॒ज्ञेन॒ तेन॒ तेन॑ य॒ज्ञेन॒ स्व॑रंकृतेन॒ स्व॑रंकृतेन य॒ज्ञेन॒ तेन॒ तेन॑ य॒ज्ञेन॒ स्व॑रंकृतेन । \newline
37. य॒ज्ञेन॒ स्व॑रंकृतेन॒ स्व॑रंकृतेन य॒ज्ञेन॑ य॒ज्ञेन॒ स्व॑रंकृतेन॒ स्वि॑ष्टेन॒ स्वि॑ष्टेन॒ स्व॑रंकृतेन य॒ज्ञेन॑ य॒ज्ञेन॒ स्व॑रंकृतेन॒ स्वि॑ष्टेन । \newline
38. स्व॑रंकृतेन॒ स्वि॑ष्टेन॒ स्वि॑ष्टेन॒ स्व॑रंकृतेन॒ स्व॑रंकृतेन॒ स्वि॑ष्टेन व॒क्षणा॑ व॒क्षणाः॒ स्वि॑ष्टेन॒ स्व॑रंकृतेन॒ स्व॑रंकृतेन॒ स्वि॑ष्टेन व॒क्षणाः᳚ । \newline
39. स्व॑रंकृते॒नेति॒ सु - अ॒रं॒कृ॒ते॒न॒ । \newline
40. स्वि॑ष्टेन व॒क्षणा॑ व॒क्षणाः॒ स्वि॑ष्टेन॒ स्वि॑ष्टेन व॒क्षणा॒ आ व॒क्षणाः॒ स्वि॑ष्टेन॒ स्वि॑ष्टेन व॒क्षणा॒ आ । \newline
41. स्वि॑ष्टे॒नेति॒ सु - इ॒ष्टे॒न॒ । \newline
42. व॒क्षणा॒ आ व॒क्षणा॑ व॒क्षणा॒ आ पृ॑णद्ध्वम् पृणद्ध्व॒ मा व॒क्षणा॑ व॒क्षणा॒ आ पृ॑णद्ध्वम् । \newline
43. आ पृ॑णद्ध्वम् पृणद्ध्व॒ मा पृ॑णद्ध्वम् । \newline
44. पृ॒ण॒द्ध्व॒मिति॑ पृणद्ध्वम् । \newline
45. यू॒प॒व्र॒स्का उ॒तोत यू॑पव्र॒स्का यू॑पव्र॒स्का उ॒त ये य उ॒त यू॑पव्र॒स्का यू॑पव्र॒स्का उ॒त ये । \newline
46. यू॒प॒व्र॒स्का इति॑ यूप - व्र॒स्काः । \newline
47. उ॒त ये य उ॒तोत ये यू॑पवा॒हा यू॑पवा॒हा य उ॒तोत ये यू॑पवा॒हाः । \newline
48. ये यू॑पवा॒हा यू॑पवा॒हा ये ये यू॑पवा॒हा श्च॒षाल॑म् च॒षालं॑ ॅयूपवा॒हा ये ये यू॑पवा॒हा श्च॒षाल᳚म् । \newline
49. यू॒प॒वा॒हा श्च॒षाल॑म् च॒षालं॑ ॅयूपवा॒हा यू॑पवा॒हा श्च॒षालं॒ ॅये ये च॒षालं॑ ॅयूपवा॒हा यू॑पवा॒हा श्च॒षालं॒ ॅये । \newline
50. यू॒प॒वा॒हा इति॑ यूप - वा॒हाः । \newline
51. च॒षालं॒ ॅये ये च॒षाल॑म् च॒षालं॒ ॅये अ॑श्वयू॒पाया᳚ श्वयू॒पाय॒ ये च॒षाल॑म् च॒षालं॒ ॅये अ॑श्वयू॒पाय॑ । \newline
52. ये अ॑श्वयू॒पाया᳚ श्वयू॒पाय॒ ये ये अ॑श्वयू॒पाय॒ तक्ष॑ति॒ तक्ष॑ त्यश्वयू॒पाय॒ ये ये अ॑श्वयू॒पाय॒ तक्ष॑ति । \newline
53. अ॒श्व॒यू॒पाय॒ तक्ष॑ति॒ तक्ष॑ त्यश्वयू॒पाया᳚ श्वयू॒पाय॒ तक्ष॑ति । \newline
54. अ॒श्व॒यू॒पायेत्य॑श्व - यू॒पाय॑ । \newline
55. तक्ष॒तीति॒ तक्ष॑ति । \newline
56. ये च॑ च॒ ये ये चार्व॑ते॒ अर्व॑ते च॒ ये ये चार्व॑ते । \newline
57. चार्व॑ते॒ अर्व॑ते च॒ चार्व॑ते॒ पच॑न॒म् पच॑न॒ मर्व॑ते च॒ चार्व॑ते॒ पच॑नम् । \newline
58. अर्व॑ते॒ पच॑न॒म् पच॑न॒ मर्व॑ते॒ अर्व॑ते॒ पच॑नꣳ सं॒भर॑न्ति सं॒भर॑न्ति॒ पच॑न॒ मर्व॑ते॒ 
अर्व॑ते॒ पच॑नꣳ सं॒भर॑न्ति । \newline
59. पच॑नꣳ सं॒भर॑न्ति सं॒भर॑न्ति॒ पच॑न॒म् पच॑नꣳ सं॒भर॑न्त्यु॒तो उ॒तो सं॒भर॑न्ति॒ पच॑न॒म् पच॑नꣳ सं॒भर॑न्त्यु॒तो । \newline
60. सं॒भर॑न्त्यु॒तो उ॒तो सं॒भर॑न्ति सं॒भर॑न्त्यु॒तो तेषा॒म् तेषा॑ मु॒तो सं॒भर॑न्ति सं॒भर॑न्त्यु॒तो तेषा᳚म् । \newline
61. सं॒भर॒न्तीति॑ सं - भर॑न्ति । \newline
62. उ॒तो तेषा॒म् तेषा॑ मु॒तो उ॒तो तेषा॑ म॒भिगू᳚र्ति र॒भिगू᳚र्ति॒ स्तेषा॑ मु॒तो उ॒तो तेषा॑ म॒भिगू᳚र्तिः । \newline
63. उ॒तो इत्यु॒तो । \newline
\pagebreak
\markright{ TS 4.6.8.3  \hfill https://www.vedavms.in \hfill}

\section{ TS 4.6.8.3 }

\textbf{TS 4.6.8.3 } \newline
\textbf{Samhita Paata} \newline

तेषा॑-म॒भिगू᳚र्तिर्न इन्वतु ॥ उप॒ प्रागा᳚थ् सु॒मन्मे॑ऽधायि॒ मन्म॑ दे॒वाना॒माशा॒ उप॑ वी॒तपृ॑ष्ठः । अन्वे॑नं॒ ॅविप्रा॒ ऋष॑यो मदन्ति दे॒वानां᳚ पु॒ष्टे च॑कृमा सु॒बन्धुं᳚ ॥ यद्-वा॒जिनो॒ दाम॑ स॒दांन॒मर्व॑तो॒ या शी॑र्.ष॒ण्या॑ रश॒ना रज्जु॑रस्य । यद्वा॑ घास्य॒ प्रभृ॑तमा॒स्ये॑ तृणꣳ॒॒ सर्वा॒ ता ते॒ अपि॑ दे॒वेष्व॑स्तु ॥ यदश्व॑स्य क्र॒विषो॒ - [  ] \newline

\textbf{Pada Paata} \newline

तेषा᳚म् । अ॒भिगू᳚र्ति॒रित्य॒भि - गू॒र्तिः॒ । नः॒ । इ॒न्व॒तु॒ ॥ उप॑ । प्रेति॑ । अ॒गा॒त् । सु॒मदिति॑ सु - मत् । मे॒ । अ॒धा॒यि॒ । मन्म॑ । दे॒वाना᳚म् । आशाः᳚ । उपेति॑ । वी॒तपृ॑ष्ठ॒ इति॑ वी॒त - पृ॒ष्ठः॒ ॥ अन्विति॑ । ए॒न॒म् । विप्राः᳚ । ऋष॑यः । म॒द॒न्ति॒ । दे॒वाना᳚म् । पु॒ष्टे । च॒कृ॒म॒ । सु॒बन्धु॒मिति॑ सु - बन्धु᳚म् ॥ यत् । वा॒जिनः॑ । दाम॑ । स॒दांन॒मिति॑ सं - दान᳚म् । अर्व॑तः । या । शी॒र्॒.ष॒ण्या᳚ । र॒श॒ना । रज्जुः॑ । अ॒स्य॒ ॥ यत् । वा॒ । घ॒ । अ॒स्य॒ । प्रभृ॑त॒मिति॒ प्र - भृ॒त॒म् । आ॒स्ये᳚ । तृण᳚म् । सर्वा᳚ । ता । ते॒ । अपीति॑ । दे॒वेषु॑ । अ॒स्तु॒ ॥ यत् । अश्व॑स्य । क्र॒विषः॑ ।  \newline


\textbf{Krama Paata} \newline

तेषा॑म॒भिगू᳚र्तिः । अ॒भिगू᳚र्तिर् नः । अ॒भिगू᳚र्ति॒रित्य॒भि - गू॒र्तिः॒ । न॒ इ॒न्व॒तु॒ । इ॒न्व॒त्विती᳚न्वतु ॥ उप॒ प्र । प्रागा᳚त् । अ॒गा॒थ् सु॒मत् । सु॒मन् मे᳚ । सु॒मदिति॑ सु - मत् । मे॒ऽधा॒यि॒ । अ॒धा॒यि॒ मन्म॑ । मन्म॑ दे॒वाना᳚म् । दे॒वाना॒माशाः᳚ । आशा॒ उप॑ । उप॑ वी॒तपृ॑ष्ठः । वी॒तपृ॑ष्ठ॒ इति॑ वी॒त - पृ॒ष्ठः॒ ॥ अन्वे॑नम् । ए॒नं॒ ॅविप्राः᳚ । विप्रा॒ ऋष॑यः । ऋष॑यो मदन्ति । म॒द॒न्ति॒ दे॒वाना᳚म् । दे॒वाना᳚म् पु॒ष्टे । पु॒ष्टे च॑कृम । च॒कृ॒मा॒ सु॒बन्धु᳚म् । सु॒बन्धु॒मिति॑ सु - बन्धु᳚म् ॥ यद् वा॒जिनः॑ । वा॒जिनो॒ दाम॑ । दाम॑ स॒न्दान᳚म् । स॒न्दान॒मर्व॑तः । स॒न्दान॒मिति॑ सम् - दान᳚म् । अर्व॑तो॒ या । या शी॑र्.ष॒ण्या᳚ । शी॒र्॒.ष॒ण्या॑ रश॒ना । र॒श॒ना रज्जुः॑ । रज्जु॑रस्य । अ॒स्येत्य॑स्य ॥ यद् वा᳚ । वा॒ घ॒ । घा॒स्य॒ । अ॒स्य॒ प्रभृ॑तम् । प्रभृ॑तमा॒स्ये᳚ । प्रभृ॑त॒मिति॒ प्र - भृ॒त॒म् । आ॒स्ये॑ तृण᳚म् । तृणꣳ॒॒ सर्वा᳚ । सर्वा॒ ता । ता ते᳚ । ते॒ अपि॑ । अपि॑ दे॒वेषु॑ । दे॒वेष्व॑स्तु । अ॒स्त्वित्य॑स्तु ॥ यदश्व॑स्य । अश्व॑स्य क्र॒विषः॑ । क्र॒विषो॒ मक्षि॑का \newline

\textbf{Jatai Paata} \newline

1. तेषा॑ म॒भिगू᳚र्ति र॒भिगू᳚र्ति॒ स्तेषा॒म् तेषा॑ म॒भिगू᳚र्तिः । \newline
2. अ॒भिगू᳚र्तिर् नो नो अ॒भिगू᳚र्ति र॒भिगू᳚र्तिर् नः । \newline
3. अ॒भिगू᳚र्ति॒रित्य॒भि - गू॒र्तिः॒ । \newline
4. न॒ इ॒न्व॒ त्वि॒न्‌व॒तु॒ नो॒ न॒ इ॒न्व॒तु॒ । \newline
5. इ॒न्व॒त्विती᳚न्वतु । \newline
6. उप॒ प्र प्रोपोप॒ प्र । \newline
7. प्रागा॑ दगा॒त् प्र प्रागा᳚त् । \newline
8. अ॒गा॒थ् सु॒मथ् सु॒म द॑गा दगाथ् सु॒मत् । \newline
9. सु॒मन् मे॑ मे सु॒मथ् सु॒मन् मे᳚ । \newline
10. सु॒मदिति॑ सु - मत् । \newline
11. मे॒ ऽधा॒ य्य॒धा॒यि॒ मे॒ मे॒ ऽधा॒यि॒ । \newline
12. अ॒धा॒यि॒ मन्म॒ मन्मा॑धा य्यधायि॒ मन्म॑ । \newline
13. मन्म॑ दे॒वाना᳚म् दे॒वाना॒म् मन्म॒ मन्म॑ दे॒वाना᳚म् । \newline
14. दे॒वाना॒ माशा॒ आशा॑ दे॒वाना᳚म् दे॒वाना॒ माशाः᳚ । \newline
15. आशा॒ उपोपाशा॒ आशा॒ उप॑ । \newline
16. उप॑ वी॒तपृ॑ष्ठो वी॒तपृ॑ष्ठ॒ उपोप॑ वी॒तपृ॑ष्ठः । \newline
17. वी॒तपृ॑ष्ठ॒ इति॑ वी॒त - पृ॒ष्ठः॒ । \newline
18. अन्वे॑न मेन॒ मन्‌वन्‌ वे॑नम् । \newline
19. ए॒नं॒ ॅविप्रा॒ विप्रा॑ एन मेनं॒ ॅविप्राः᳚ । \newline
20. विप्रा॒ ऋष॑य॒ ऋष॑यो॒ विप्रा॒ विप्रा॒ ऋष॑यः । \newline
21. ऋष॑यो मदन्ति मद॒ न्त्यृष॑य॒ ऋष॑यो मदन्ति । \newline
22. म॒द॒न्ति॒ दे॒वाना᳚म् दे॒वाना᳚म् मदन्ति मदन्ति दे॒वाना᳚म् । \newline
23. दे॒वाना᳚म् पु॒ष्टे पु॒ष्टे दे॒वाना᳚म् दे॒वाना᳚म् पु॒ष्टे । \newline
24. पु॒ष्टे च॑कृम चकृम पु॒ष्टे पु॒ष्टे च॑कृम । \newline
25. च॒कृ॒मा॒ सु॒बन्धुꣳ॑ सु॒बन्धु॑म् चकृम चकृमा सु॒बन्धु᳚म् । \newline
26. सु॒बन्धु॒मिति॑ सु - बन्धु᳚म् । \newline
27. यद् वा॒जिनो॑ वा॒जिनो॒ यद् यद् वा॒जिनः॑ । \newline
28. वा॒जिनो॒ दाम॒ दाम॑ वा॒जिनो॑ वा॒जिनो॒ दाम॑ । \newline
29. दाम॑ स॒न्दानꣳ॑ स॒न्दान॒म् दाम॒ दाम॑ स॒न्दान᳚म् । \newline
30. स॒न्दान॒ मर्व॑तो॒ अर्व॑तः स॒न्दानꣳ॑ स॒न्दान॒ मर्व॑तः । \newline
31. स॒न्दान॒मिति॑ सं - दान᳚म् । \newline
32. अर्व॑तो॒ या या ऽर्व॑तो॒ अर्व॑तो॒ या । \newline
33. या शी॑र्.ष॒ण्या॑ शीर्.ष॒ण्या॑ या या शी॑र्.ष॒ण्या᳚ । \newline
34. शी॒र्॒.ष॒ण्या॑ रश॒ना र॑श॒ना शी॑र्.ष॒ण्या॑ शीर्.ष॒ण्या॑ रश॒ना । \newline
35. र॒श॒ना रज्जू॒ रज्जू॑ रश॒ना र॑श॒ना रज्जुः॑ । \newline
36. रज्जु॑ रस्यास्य॒ रज्जू॒ रज्जु॑ रस्य । \newline
37. अ॒स्येत्य॑स्य । \newline
38. यद् वा॑ वा॒ यद् यद् वा᳚ । \newline
39. वा॒ घ॒ घ॒ वा॒ वा॒ घ॒ । \newline
40. घा॒ स्या॒ स्य॒ घ॒ घा॒ स्य॒ । \newline
41. अ॒स्य॒ प्रभृ॑त॒म् प्रभृ॑त मस्यास्य॒ प्रभृ॑तम् । \newline
42. प्रभृ॑त मा॒स्य॑ आ॒स्ये᳚ प्रभृ॑त॒म् प्रभृ॑त मा॒स्ये᳚ । \newline
43. प्रभृ॑त॒मिति॒ प्र - भृ॒त॒म् । \newline
44. आ॒स्ये॑ तृण॒म् तृण॑ मा॒स्य॑ आ॒स्ये॑ तृण᳚म् । \newline
45. तृणꣳ॒॒ सर्वा॒ सर्वा॒ तृण॒म् तृणꣳ॒॒ सर्वा᳚ । \newline
46. सर्वा॒ ता ता सर्वा॒ सर्वा॒ ता । \newline
47. ता ते॑ ते॒ ता ता ते᳚ । \newline
48. ते॒ अप्यपि॑ ते ते॒ अपि॑ । \newline
49. अपि॑ दे॒वेषु॑ दे॒वे ष्वप्यपि॑ दे॒वेषु॑ । \newline
50. दे॒वे ष्व॑स्त्वस्तु दे॒वेषु॑ दे॒वे ष्व॑स्तु । \newline
51. अ॒स्त्वित्य॑स्तु । \newline
52. यदश्व॒स्या श्व॑स्य॒ यद् यदश्व॑स्य । \newline
53. अश्व॑स्य क्र॒विषः॑ क्र॒विषो॒ अश्व॒स्या श्व॑स्य क्र॒विषः॑ । \newline
54. क्र॒विषो॒ मक्षि॑का॒ मक्षि॑का क्र॒विषः॑ क्र॒विषो॒ मक्षि॑का । \newline

\textbf{Ghana Paata } \newline

1. तेषा॑ म॒भिगू᳚र्ति र॒भिगू᳚र्ति॒ स्तेषा॒म् तेषा॑ म॒भिगू᳚र्तिर् नो नो अ॒भिगू᳚र्ति॒ स्तेषा॒म् तेषा॑ म॒भिगू᳚र्तिर् नः । \newline
2. अ॒भिगू᳚र्तिर् नो नो अ॒भिगू᳚र्ति र॒भिगू᳚र्तिर् न इन्व त्विन्वतु नो अ॒भिगू᳚र्ति र॒भिगू᳚र्तिर् न इन्वतु । \newline
3. अ॒भिगू᳚र्ति॒रित्य॒भि - गू॒र्तिः॒ । \newline
4. न॒ इ॒न्व॒ त्वि॒न्व॒तु॒ नो॒ न॒ इ॒न्व॒तु॒ । \newline
5. इ॒न्व॒त्विती᳚न्वतु । \newline
6. उप॒ प्र प्रोपोप॒ प्रागा॑ दगा॒त् प्रोपोप॒ प्रागा᳚त् । \newline
7. प्रागा॑ दगा॒त् प्र प्रागा᳚थ् सु॒मथ् सु॒म द॑गा॒त् प्र प्रागा᳚थ् सु॒मत् । \newline
8. अ॒गा॒थ् सु॒मथ् सु॒म द॑गा दगाथ् सु॒मन् मे॑ मे सु॒म द॑गा दगाथ् सु॒मन् मे᳚ । \newline
9. सु॒मन् मे॑ मे सु॒मथ् सु॒मन् मे॑ ऽधा य्यधायि मे सु॒मथ् सु॒मन् मे॑ ऽधायि । \newline
10. सु॒मदिति॑ सु - मत् । \newline
11. मे॒ ऽधा॒ य्य॒धा॒यि॒ मे॒ मे॒ ऽधा॒यि॒ मन्म॒ मन्मा॑ धायि मे मे ऽधायि॒ मन्म॑ । \newline
12. अ॒धा॒यि॒ मन्म॒ मन्मा॑ धाय्यधायि॒ मन्म॑ दे॒वाना᳚म् दे॒वाना॒म् मन्मा॑धा य्यधायि॒ मन्म॑ दे॒वाना᳚म् । \newline
13. मन्म॑ दे॒वाना᳚म् दे॒वाना॒म् मन्म॒ मन्म॑ दे॒वाना॒ माशा॒ आशा॑ दे॒वाना॒म् मन्म॒ मन्म॑ दे॒वाना॒ माशाः᳚ । \newline
14. दे॒वाना॒ माशा॒ आशा॑ दे॒वाना᳚म् दे॒वाना॒ माशा॒ उपोपाशा॑ दे॒वाना᳚म् दे॒वाना॒ माशा॒ उप॑ । \newline
15. आशा॒ उपोपाशा॒ आशा॒ उप॑ वी॒तपृ॑ष्ठो वी॒तपृ॑ष्ठ॒ उपाशा॒ आशा॒ उप॑ वी॒तपृ॑ष्ठः । \newline
16. उप॑ वी॒तपृ॑ष्ठो वी॒तपृ॑ष्ठ॒ उपोप॑ वी॒तपृ॑ष्ठः । \newline
17. वी॒तपृ॑ष्ठ॒ इति॑ वी॒त - पृ॒ष्ठः॒ । \newline
18. अन्वे॑न मेन॒ मन्वन् वे॑नं॒ ॅविप्रा॒ विप्रा॑ एन॒ मन्वन् वे॑नं॒ ॅविप्राः᳚ । \newline
19. ए॒नं॒ ॅविप्रा॒ विप्रा॑ एन मेनं॒ ॅविप्रा॒ ऋष॑य॒ ऋष॑यो॒ विप्रा॑ एन मेनं॒ ॅविप्रा॒ ऋष॑यः । \newline
20. विप्रा॒ ऋष॑य॒ ऋष॑यो॒ विप्रा॒ विप्रा॒ ऋष॑यो मदन्ति मद॒ न्त्यृष॑यो॒ विप्रा॒ विप्रा॒ ऋष॑यो मदन्ति । \newline
21. ऋष॑यो मदन्ति मद॒ न्त्यृष॑य॒ ऋष॑यो मदन्ति दे॒वाना᳚म् दे॒वाना᳚म् मद॒ न्त्यृष॑य॒ ऋष॑यो मदन्ति दे॒वाना᳚म् । \newline
22. म॒द॒न्ति॒ दे॒वाना᳚म् दे॒वाना᳚म् मदन्ति मदन्ति दे॒वाना᳚म् पु॒ष्टे पु॒ष्टे दे॒वाना᳚म् मदन्ति मदन्ति दे॒वाना᳚म् पु॒ष्टे । \newline
23. दे॒वाना᳚म् पु॒ष्टे पु॒ष्टे दे॒वाना᳚म् दे॒वाना᳚म् पु॒ष्टे च॑कृम चकृम पु॒ष्टे दे॒वाना᳚म् दे॒वाना᳚म् पु॒ष्टे च॑कृम । \newline
24. पु॒ष्टे च॑कृम चकृम पु॒ष्टे पु॒ष्टे च॑कृमा सु॒बन्धुꣳ॑ सु॒बन्धु॑म् चकृम पु॒ष्टे पु॒ष्टे च॑कृमा सु॒बन्धु᳚म् । \newline
25. च॒कृ॒मा॒ सु॒बन्धुꣳ॑ सु॒बन्धु॑म् चकृम चकृमा सु॒बन्धु᳚म् । \newline
26. सु॒बन्धु॒मिति॑ सु - बन्धु᳚म् । \newline
27. यद् वा॒जिनो॑ वा॒जिनो॒ यद् यद् वा॒जिनो॒ दाम॒ दाम॑ वा॒जिनो॒ यद् यद् वा॒जिनो॒ दाम॑ । \newline
28. वा॒जिनो॒ दाम॒ दाम॑ वा॒जिनो॑ वा॒जिनो॒ दाम॑ स॒न्दानꣳ॑ स॒न्दान॒म् दाम॑ वा॒जिनो॑ वा॒जिनो॒ दाम॑ स॒न्दान᳚म् । \newline
29. दाम॑ स॒न्दानꣳ॑ स॒न्दान॒म् दाम॒ दाम॑ स॒न्दान॒ मर्व॑तो॒ अर्व॑तः स॒न्दान॒म् दाम॒ दाम॑ स॒न्दान॒ मर्व॑तः । \newline
30. स॒न्दान॒ मर्व॑तो॒ अर्व॑तः स॒न्दानꣳ॑ स॒न्दान॒ मर्व॑तो॒ या या ऽर्व॑तः स॒न्दानꣳ॑ स॒न्दान॒ मर्व॑तो॒ या । \newline
31. स॒न्दान॒मिति॑ सं - दान᳚म् । \newline
32. अर्व॑तो॒ या या ऽर्व॑तो॒ अर्व॑तो॒ या शी॑र्.ष॒ण्या॑ शीर्.ष॒ण्या॑ या ऽर्व॑तो॒ अर्व॑तो॒ या शी॑र्.ष॒ण्या᳚ । \newline
33. या शी॑र्.ष॒ण्या॑ शीर्.ष॒ण्या॑ या या शी॑र्.ष॒ण्या॑ रश॒ना र॑श॒ना शी॑र्.ष॒ण्या॑ या या शी॑र्.ष॒ण्या॑ रश॒ना । \newline
34. शी॒र्॒.ष॒ण्या॑ रश॒ना र॑श॒ना शी॑र्.ष॒ण्या॑ शीर्.ष॒ण्या॑ रश॒ना रज्जू॒ रज्जू॑ रश॒ना शी॑र्.ष॒ण्या॑ शीर्.ष॒ण्या॑ रश॒ना रज्जुः॑ । \newline
35. र॒श॒ना रज्जू॒ रज्जू॑ रश॒ना र॑श॒ना रज्जु॑ रस्यास्य॒ रज्जू॑ रश॒ना र॑श॒ना रज्जु॑ रस्य । \newline
36. रज्जु॑ रस्यास्य॒ रज्जू॒ रज्जु॑ रस्य । \newline
37. अ॒स्येत्य॑स्य । \newline
38. यद् वा॑ वा॒ यद् यद् वा॑ घ घ वा॒ यद् यद् वा॑ घ । \newline
39. वा॒ घ॒ घ॒ वा॒ वा॒ घा॒ स्या॒स्य॒ घ॒ वा॒ वा॒ घा॒स्य॒ । \newline
40. घा॒ स्या॒स्य॒ घ॒ घा॒स्य॒ प्रभृ॑त॒म् प्रभृ॑त मस्य घ घास्य॒ प्रभृ॑तम् । \newline
41. अ॒स्य॒ प्रभृ॑त॒म् प्रभृ॑त मस्यास्य॒ प्रभृ॑त मा॒स्य॑ आ॒स्ये᳚ प्रभृ॑त मस्यास्य॒ प्रभृ॑त मा॒स्ये᳚ । \newline
42. प्रभृ॑त मा॒स्य॑ आ॒स्ये᳚ प्रभृ॑त॒म् प्रभृ॑त मा॒स्ये॑ तृण॒म् तृण॑ मा॒स्ये᳚ प्रभृ॑त॒म् प्रभृ॑त मा॒स्ये॑ तृण᳚म् । \newline
43. प्रभृ॑त॒मिति॒ प्र - भृ॒त॒म् । \newline
44. आ॒स्ये॑ तृण॒म् तृण॑ मा॒स्य॑ आ॒स्ये॑ तृणꣳ॒॒ सर्वा॒ सर्वा॒ तृण॑ मा॒स्य॑ आ॒स्ये॑ तृणꣳ॒॒ सर्वा᳚ । \newline
45. तृणꣳ॒॒ सर्वा॒ सर्वा॒ तृण॒म् तृणꣳ॒॒ सर्वा॒ ता ता सर्वा॒ तृण॒म् तृणꣳ॒॒ सर्वा॒ ता । \newline
46. सर्वा॒ ता ता सर्वा॒ सर्वा॒ ता ते॑ ते॒ ता सर्वा॒ सर्वा॒ ता ते᳚ । \newline
47. ता ते॑ ते॒ ता ता ते॒ अप्यपि॑ ते॒ ता ता ते॒ अपि॑ । \newline
48. ते॒ अप्यपि॑ ते ते॒ अपि॑ दे॒वेषु॑ दे॒वे ष्वपि॑ ते ते॒ अपि॑ दे॒वेषु॑ । \newline
49. अपि॑ दे॒वेषु॑ दे॒वे ष्वप्यपि॑ दे॒वे ष्व॑स्त्वस्तु दे॒वे ष्वप्यपि॑ दे॒वे ष्व॑स्तु । \newline
50. दे॒वे ष्व॑स्त्वस्तु दे॒वेषु॑ दे॒वे ष्व॑स्तु । \newline
51. अ॒स्त्वित्य॑स्तु । \newline
52. यदश्व॒स्या श्व॑स्य॒ यद् यदश्व॑स्य क्र॒विषः॑ क्र॒विषो॒ अश्व॑स्य॒ यद् यदश्व॑स्य क्र॒विषः॑ । \newline
53. अश्व॑स्य क्र॒विषः॑ क्र॒विषो॒ अश्व॒स्या श्व॑स्य क्र॒विषो॒ मक्षि॑का॒ मक्षि॑का क्र॒विषो॒ अश्व॒स्या श्व॑स्य क्र॒विषो॒ मक्षि॑का । \newline
54. क्र॒विषो॒ मक्षि॑का॒ मक्षि॑का क्र॒विषः॑ क्र॒विषो॒ मक्षि॒का ऽऽशा श॒ मक्षि॑का क्र॒विषः॑ क्र॒विषो॒ मक्षि॒का ऽऽश॑ । \newline
\pagebreak
\markright{ TS 4.6.8.4  \hfill https://www.vedavms.in \hfill}

\section{ TS 4.6.8.4 }

\textbf{TS 4.6.8.4 } \newline
\textbf{Samhita Paata} \newline

मक्षि॒काऽऽश॒ यद्वा॒ स्वरौ॒ स्वधि॑तौ रि॒प्तमस्ति॑ । यद्धस्त॑योः शमि॒तुर्यन्न॒खेषु॒ सर्वा॒ ता ते॒ अपि॑ दे॒वेष्व॑स्तु ॥ यदूव॑द्ध्यमु॒दर॑स्याप॒वाति॒ य आ॒मस्य॑ क्र॒विषो॑ ग॒न्धो अस्ति॑ । सु॒कृ॒ता तच्छ॑मि॒तारः॑ कृण्वन्तू॒त मेधꣳ॑ शृत॒पाकं॑ पचन्तु ॥ यत् ते॒ गात्रा॑द॒ग्निना॑ प॒च्यमा॑नाद॒भि शूलं॒ निह॑तस्याव॒धाव॑ति । मा तद्-भूम्या॒मा श्रि॑ष॒ ( )-न्मा तृणे॑षु दे॒वेभ्य॒स्तदु॒शद्भ्यो॑ रा॒तम॑स्तु ॥ \newline

\textbf{Pada Paata} \newline

मक्षि॑का । आश॑ । यत् । वा॒ । स्वरौ᳚ । स्वधि॑ता॒विति॒ स्व - धि॒तौ॒ । रि॒प्तम् । अस्ति॑ ॥ यत् । हस्त॑योः । श॒मि॒तुः । यत् । न॒खेषु॑ । सर्वा᳚ । ता । ते॒ । अपीति॑ । दे॒वेषु॑ । अ॒स्तु॒ ॥ यत् । ऊव॑द्ध्यम् । उ॒दर॑स्य । अ॒प॒वातीत्य॑प - वाति॑ । यः । आ॒मस्य॑ । क्र॒विषः॑ । ग॒न्धः । अस्ति॑ ॥ सु॒कृ॒तेति॑ सु - कृ॒ता । तत् । श॒मि॒तारः॑ । कृ॒ण्व॒न्तु॒ । उ॒त । मेध᳚म् । शृ॒त॒पाक॒मिति॑ शृत - पाक᳚म् । प॒च॒न्तु॒ ॥ यत् । ते॒ । गात्रा᳚त् । अ॒ग्निना᳚ । प॒च्यमा॑नात् । अ॒भीति॑ । शूल᳚म् । निह॑त॒स्येति॒ नि - ह॒त॒स्य॒ । अ॒व॒धाव॒तीत्य॑व - धाव॑ति ॥ मा । तत् । भूम्या᳚म् । एति॑ । श्रि॒ष॒त् ( ) । मा । तृणे॑षु । दे॒वेभ्यः॑ । तत् । उ॒शद्भ्य॒ इत्यु॒शत् - भ्यः॒ । रा॒तम् । अ॒स्तु॒ ॥  \newline


\textbf{Krama Paata} \newline

मक्षि॒काऽऽश॑ । आश॒ यत् । यद् वा᳚ । वा॒ स्वरौ᳚ । स्वरौ॒ स्वधि॑तौ । स्वधि॑तौ रि॒प्तम् । स्वधि॑ता॒विति॒ स्व - धि॒तौ॒ । रि॒प्तमस्ति॑ । अस्तीत्यस्ति॑ ॥ यद्धस्त॑योः । हस्त॑योः शमि॒तुः । श॒मि॒तुर् यत् । यन् न॒खेषु॑ । न॒खेषु॒ सर्वा᳚ । सर्वा॒ ता । ता ते᳚ । ते॒ अपि॑ । अपि॑ दे॒वेषु॑ । दे॒वेष्व॑स्तु । अ॒स्त्वित्य॑स्तु ॥ यदूव॑द्ध्यम् । ऊव॑द्ध्यमु॒दर॑स्य । उ॒दर॑स्याप॒वाति॑ । अ॒प॒वाति॒ यः । अ॒प॒वातीत्य॑प - वाति॑ । य आ॒मस्य॑ । आ॒मस्य॑ क्र॒विषः॑ । क्र॒विषो॑ ग॒न्धः । ग॒न्धो अस्ति॑ । अस्तीत्यस्ति॑ ॥ सु॒कृ॒ता तत् । सु॒कृ॒तेति॑ सु - कृ॒ता । तच्छ॑मि॒तारः॑ । श॒मि॒तारः॑ कृण्वन्तु । कृ॒ण्व॒न्तू॒त । उ॒त मेध᳚म् । मेधꣳ॑ शृत॒पाक᳚म् । शृ॒त॒पाक॑म् पचन्तु । शृ॒त॒पाक॒मिति॑ शृत - पाक᳚म् । प॒च॒न्त्विति॑ पचन्तु ॥ यत् ते᳚ । ते॒ गात्रा᳚त् । गात्रा॑द॒ग्निना᳚ । अ॒ग्निना॑ प॒च्यमा॑नात् । प॒च्यमा॑नाद॒भि । अ॒भि शूल᳚म् । शूल॒म् निह॑तस्य । निह॑तस्याव॒धाव॑ति । निह॑त॒स्येति॒ नि - ह॒त॒स्य॒ । अ॒व॒धाव॒तीत्य॑व - धाव॑ति ॥ मा तत् । तद् भूम्या᳚म् । भूम्या॒मा । आ श्रि॑षत् ( ) । श्रि॒ष॒न् मा । मा तृणे॑षु । तृणे॑षु दे॒वेभ्यः॑ । दे॒वेभ्य॒स्तत् । तदु॒शद्भ्यः॑ । उ॒शद्भ्यो॑ रा॒तम् । उ॒शद्भ्य॒ इत्यु॒शत् - भ्यः॒ । रा॒तम॑स्तु । अ॒स्त्वित्य॑स्तु । \newline

\textbf{Jatai Paata} \newline

1. मक्षि॒का ऽऽशाश॒ मक्षि॑का॒ मक्षि॒का ऽऽश॑ । \newline
2. आश॒ यद् यदाशाश॒ यत् । \newline
3. यद् वा॑ वा॒ यद् यद् वा᳚ । \newline
4. वा॒ स्वरौ॒ स्वरौ॑ वा वा॒ स्वरौ᳚ । \newline
5. स्वरौ॒ स्वधि॑तौ॒ स्वधि॑तौ॒ स्वरौ॒ स्वरौ॒ स्वधि॑तौ । \newline
6. स्वधि॑तौ रि॒प्तꣳ रि॒प्तꣳ स्वधि॑तौ॒ स्वधि॑तौ रि॒प्तम् । \newline
7. स्वधि॑ता॒विति॒ स्व - धि॒तौ॒ । \newline
8. रि॒प्त मस्त्यस्ति॑ रि॒प्तꣳ रि॒प्त मस्ति॑ । \newline
9. अस्तीत्यस्ति॑ । \newline
10. यद्धस्त॑यो॒र्॒. हस्त॑यो॒र् यद् यद्धस्त॑योः । \newline
11. हस्त॑योः शमि॒तुः श॑मि॒तुर्. हस्त॑यो॒र्॒. हस्त॑योः शमि॒तुः । \newline
12. श॒मि॒तुर् यद् यच्छ॑मि॒तुः श॑मि॒तुर् यत् । \newline
13. यन् न॒खेषु॑ न॒खेषु॒ यद् यन् न॒खेषु॑ । \newline
14. न॒खेषु॒ सर्वा॒ सर्वा॑ न॒खेषु॑ न॒खेषु॒ सर्वा᳚ । \newline
15. सर्वा॒ ता ता सर्वा॒ सर्वा॒ ता । \newline
16. ता ते॑ ते॒ ता ता ते᳚ । \newline
17. ते॒ अप्यपि॑ ते ते॒ अपि॑ । \newline
18. अपि॑ दे॒वेषु॑ दे॒वे ष्वप्यपि॑ दे॒वेषु॑ । \newline
19. दे॒वे ष्व॑स्त्वस्तु दे॒वेषु॑ दे॒वे ष्व॑स्तु । \newline
20. अ॒स्त्वित्य॑स्तु । \newline
21. यदूव॑द्ध्य॒ मूव॑द्ध्यं॒ ॅयद् यदूव॑द्ध्यम् । \newline
22. ऊव॑द्ध्य मु॒दर॑ स्यो॒दर॒ स्योव॑द्ध्य॒ मूव॑द्ध्य मु॒दर॑स्य । \newline
23. उ॒दर॑स्या प॒वा त्य॑प॒वा त्यु॒दर॑ स्यो॒दर॑स्या प॒वाति॑ । \newline
24. अ॒प॒वाति॒ यो यो अ॑प॒वा त्य॑प॒वाति॒ यः । \newline
25. अ॒प॒वातीत्य॑प - वाति॑ । \newline
26. य आ॒मस्या॒ मस्य॒ यो य आ॒मस्य॑ । \newline
27. आ॒मस्य॑ क्र॒विषः॑ क्र॒विष॑ आ॒मस्या॒ मस्य॑ क्र॒विषः॑ । \newline
28. क्र॒विषो॑ ग॒न्धो ग॒न्धः क्र॒विषः॑ क्र॒विषो॑ ग॒न्धः । \newline
29. ग॒न्धो अस्त्यस्ति॑ ग॒न्धो ग॒न्धो अस्ति॑ । \newline
30. अस्तीत्यस्ति॑ । \newline
31. सु॒कृ॒ता तत् तथ् सु॑कृ॒ता सु॑कृ॒ता तत् । \newline
32. सु॒कृ॒तेति॑ सु - कृ॒ता । \newline
33. तच्छ॑मि॒तारः॑ शमि॒तार॒ स्तत् तच्छ॑मि॒तारः॑ । \newline
34. श॒मि॒तारः॑ कृण्वन्तु कृण्वन्तु शमि॒तारः॑ शमि॒तारः॑ कृण्वन्तु । \newline
35. कृ॒ण्व॒न्तू॒तोत कृ॑ण्वन्तु कृण्वन्तू॒त । \newline
36. उ॒त मेध॒म् मेध॑ मु॒तोत मेध᳚म् । \newline
37. मेधꣳ॑ शृत॒पाकꣳ॑ शृत॒पाक॒म् मेध॒म् मेधꣳ॑ शृत॒पाक᳚म् । \newline
38. शृ॒त॒पाक॑म् पचन्तु पचन्तु शृत॒पाकꣳ॑ शृत॒पाक॑म् पचन्तु । \newline
39. शृ॒त॒पाक॒मिति॑ शृत - पाक᳚म् । \newline
40. प॒च॒न्त्विति॑ पचन्तु । \newline
41. यत् ते॑ ते॒ यद् यत् ते᳚ । \newline
42. ते॒ गात्रा॒द् गात्रा᳚त् ते ते॒ गात्रा᳚त् । \newline
43. गात्रा॑ द॒ग्निना॒ ऽग्निना॒ गात्रा॒द् गात्रा॑ द॒ग्निना᳚ । \newline
44. अ॒ग्निना॑ प॒च्यमा॑नात् प॒च्यमा॑ना द॒ग्निना॒ ऽग्निना॑ प॒च्यमा॑नात् । \newline
45. प॒च्यमा॑ना द॒भ्य॑भि प॒च्यमा॑नात् प॒च्यमा॑ना द॒भि । \newline
46. अ॒भि शूलꣳ॒॒ शूल॑ म॒भ्य॑भि शूल᳚म् । \newline
47. शूल॒म् निह॑तस्य॒ निह॑तस्य॒ शूलꣳ॒॒ शूल॒म् निह॑तस्य । \newline
48. निह॑तस्या व॒धाव॑ त्यव॒धाव॑ति॒ निह॑तस्य॒ निह॑तस्या व॒धाव॑ति । \newline
49. निह॑त॒स्येति॒ नि - ह॒त॒स्य॒ । \newline
50. अ॒व॒धाव॒तीत्य॑व - धाव॑ति । \newline
51. मा तत् तन् मा मा तत् । \newline
52. तद् भूम्या॒म् भूम्या॒म् तत् तद् भूम्या᳚म् । \newline
53. भूम्या॒ मा भूम्या॒म् भूम्या॒ मा । \newline
54. आ श्रि॑ष च्छ्रिष॒दा श्रि॑षत् । \newline
55. श्रि॒ष॒न् मा मा श्रि॑ष च्छ्रिष॒न् मा । \newline
56. मा तृणे॑षु॒ तृणे॑षु॒ मा मा तृणे॑षु । \newline
57. तृणे॑षु दे॒वेभ्यो॑ दे॒वेभ्य॒ स्तृणे॑षु॒ तृणे॑षु दे॒वेभ्यः॑ । \newline
58. दे॒वेभ्य॒ स्तत् तद् दे॒वेभ्यो॑ दे॒वेभ्य॒ स्तत् । \newline
59. तदु॒शद्भ्य॑ उ॒शद्भ्य॒ स्तत् तदु॒शद्भ्यः॑ । \newline
60. उ॒शद्भ्यो॑ रा॒तꣳ रा॒त मु॒शद्भ्य॑ उ॒शद्भ्यो॑ रा॒तम् । \newline
61. उ॒शद्भ्य॒ इत्यु॒शत् - भ्यः॒ । \newline
62. रा॒त म॑स्त्वस्तु रा॒तꣳ रा॒त म॑स्तु । \newline
63. अ॒स्त्वित्य॑स्तु । \newline

\textbf{Ghana Paata } \newline

1. मक्षि॒का ऽऽशा श॒ मक्षि॑का॒ मक्षि॒का ऽऽश॒ यद् यदाश॒ मक्षि॑का॒ मक्षि॒का ऽऽश॒ यत् । \newline
2. आश॒ यद् यदाशा श॒ यद् वा॑ वा॒ यदा शा श॒ यद् वा᳚ । \newline
3. यद् वा॑ वा॒ यद् यद् वा॒ स्वरौ॒ स्वरौ॑ वा॒ यद् यद् वा॒ स्वरौ᳚ । \newline
4. वा॒ स्वरौ॒ स्वरौ॑ वा वा॒ स्वरौ॒ स्वधि॑तौ॒ स्वधि॑तौ॒ स्वरौ॑ वा वा॒ स्वरौ॒ स्वधि॑तौ । \newline
5. स्वरौ॒ स्वधि॑तौ॒ स्वधि॑तौ॒ स्वरौ॒ स्वरौ॒ स्वधि॑तौ रि॒प्तꣳ रि॒प्तꣳ स्वधि॑तौ॒ स्वरौ॒ स्वरौ॒ स्वधि॑तौ रि॒प्तम् । \newline
6. स्वधि॑तौ रि॒प्तꣳ रि॒प्तꣳ स्वधि॑तौ॒ स्वधि॑तौ रि॒प्त मस्त्यस्ति॑ रि॒प्तꣳ स्वधि॑तौ॒ स्वधि॑तौ रि॒प्त मस्ति॑ । \newline
7. स्वधि॑ता॒विति॒ स्व - धि॒तौ॒ । \newline
8. रि॒प्त मस्त्यस्ति॑ रि॒प्तꣳ रि॒प्त मस्ति॑ । \newline
9. अस्तीत्यस्ति॑ । \newline
10. यद्धस्त॑यो॒र्॒. हस्त॑यो॒र् यद् यद्धस्त॑योः शमि॒तुः श॑मि॒तुर्. हस्त॑यो॒र् यद् यद्धस्त॑योः शमि॒तुः । \newline
11. हस्त॑योः शमि॒तुः श॑मि॒तुर्. हस्त॑यो॒र्॒. हस्त॑योः शमि॒तुर् यद् यच्छ॑मि॒तुर्. हस्त॑यो॒र्॒. हस्त॑योः शमि॒तुर् यत् । \newline
12. श॒मि॒तुर् यद् यच्छ॑मि॒तुः श॑मि॒तुर् यन् न॒खेषु॑ न॒खेषु॒ यच्छ॑मि॒तुः श॑मि॒तुर् यन् न॒खेषु॑ । \newline
13. यन् न॒खेषु॑ न॒खेषु॒ यद् यन् न॒खेषु॒ सर्वा॒ सर्वा॑ न॒खेषु॒ यद् यन् न॒खेषु॒ सर्वा᳚ । \newline
14. न॒खेषु॒ सर्वा॒ सर्वा॑ न॒खेषु॑ न॒खेषु॒ सर्वा॒ ता ता सर्वा॑ न॒खेषु॑ न॒खेषु॒ सर्वा॒ ता । \newline
15. सर्वा॒ ता ता सर्वा॒ सर्वा॒ ता ते॑ ते॒ ता सर्वा॒ सर्वा॒ ता ते᳚ । \newline
16. ता ते॑ ते॒ ता ता ते॒ अप्यपि॑ ते॒ ता ता ते॒ अपि॑ । \newline
17. ते॒ अप्यपि॑ ते ते॒ अपि॑ दे॒वेषु॑ दे॒वेष्वपि॑ ते ते॒ अपि॑ दे॒वेषु॑ । \newline
18. अपि॑ दे॒वेषु॑ दे॒वे ष्वप्यपि॑ दे॒वे ष्व॑स्त्वस्तु दे॒वे ष्वप्यपि॑ दे॒वे ष्व॑स्तु । \newline
19. दे॒वे ष्व॑स्त्वस्तु दे॒वेषु॑ दे॒वेष्व॑स्तु । \newline
20. अ॒स्त्वित्य॑स्तु । \newline
21. यदूव॑द्ध्य॒ मूव॑द्ध्यं॒ ॅयद् यदूव॑द्ध्य मु॒दर॑ स्यो॒दर॒ स्योव॑द्ध्यं॒ ॅयद् यदूव॑द्ध्य मु॒दर॑स्य । \newline
22. ऊव॑द्ध्य मु॒दर॑ स्यो॒दर॒ स्योव॑द्ध्य॒ मूव॑द्ध्य मु॒दर॑स्या प॒वा त्य॑प॒वा त्यु॒दर॒ स्योव॑द्ध्य॒ मूव॑द्ध्य मु॒दर॑स्या प॒वाति॑ । \newline
23. उ॒दर॑स्या प॒वा त्य॑प॒वा त्यु॒दर॑ स्यो॒दर॑स्या प॒वाति॒ यो यो अ॑प॒वा त्यु॒दर॑ स्यो॒दर॑स्या प॒वाति॒ यः । \newline
24. अ॒प॒वाति॒ यो यो अ॑प॒वा त्य॑प॒वाति॒ य आ॒मस्या॒ मस्य॒ यो अ॑प॒वा त्य॑प॒वाति॒ य आ॒मस्य॑ । \newline
25. अ॒प॒वातीत्य॑प - वाति॑ । \newline
26. य आ॒मस्या॒ मस्य॒ यो य आ॒मस्य॑ क्र॒विषः॑ क्र॒विष॑ आ॒मस्य॒ यो य आ॒मस्य॑ क्र॒विषः॑ । \newline
27. आ॒मस्य॑ क्र॒विषः॑ क्र॒विष॑ आ॒मस्या॒ मस्य॑ क्र॒विषो॑ ग॒न्धो ग॒न्धः क्र॒विष॑ आ॒मस्या॒ मस्य॑ क्र॒विषो॑ ग॒न्धः । \newline
28. क्र॒विषो॑ ग॒न्धो ग॒न्धः क्र॒विषः॑ क्र॒विषो॑ ग॒न्धो अस्त्यस्ति॑ ग॒न्धः क्र॒विषः॑ क्र॒विषो॑ ग॒न्धो अस्ति॑ । \newline
29. ग॒न्धो अस्त्यस्ति॑ ग॒न्धो ग॒न्धो अस्ति॑ । \newline
30. अस्तीत्यस्ति॑ । \newline
31. सु॒कृ॒ता तत् तथ् सु॑कृ॒ता सु॑कृ॒ता तच्छ॑मि॒तारः॑ शमि॒तार॒ स्तथ् सु॑कृ॒ता सु॑कृ॒ता तच्छ॑मि॒तारः॑ । \newline
32. सु॒कृ॒तेति॑ सु - कृ॒ता । \newline
33. तच्छ॑मि॒तारः॑ शमि॒तार॒ स्तत् तच्छ॑मि॒तारः॑ कृण्वन्तु कृण्वन्तु शमि॒तार॒ स्तत् तच्छ॑मि॒तारः॑ कृण्वन्तु । \newline
34. श॒मि॒तारः॑ कृण्वन्तु कृण्वन्तु शमि॒तारः॑ शमि॒तारः॑ कृण्वन् तू॒तोत कृ॑ण्वन्तु शमि॒तारः॑ शमि॒तारः॑ कृण्वन्तू॒त । \newline
35. कृ॒ण्व॒न् तू॒तोत कृ॑ण्वन्तु कृण्वन्तू॒त मेध॒म् मेध॑ मु॒त कृ॑ण्वन्तु कृण्वन्तू॒त मेध᳚म् । \newline
36. उ॒त मेध॒म् मेध॑ मु॒तोत मेधꣳ॑ शृत॒पाकꣳ॑ शृत॒पाक॒म् मेध॑ मु॒तोत मेधꣳ॑ शृत॒पाक᳚म् । \newline
37. मेधꣳ॑ शृत॒पाकꣳ॑ शृत॒पाक॒म् मेध॒म् मेधꣳ॑ शृत॒पाक॑म् पचन्तु पचन्तु शृत॒पाक॒म् मेध॒म् मेधꣳ॑ शृत॒पाक॑म् पचन्तु । \newline
38. शृ॒त॒पाक॑म् पचन्तु पचन्तु शृत॒पाकꣳ॑ शृत॒पाक॑म् पचन्तु । \newline
39. शृ॒त॒पाक॒मिति॑ शृत - पाक᳚म् । \newline
40. प॒च॒न्त्विति॑ पचन्तु । \newline
41. यत् ते॑ ते॒ यद् यत् ते॒ गात्रा॒द् गात्रा᳚त् ते॒ यद् यत् ते॒ गात्रा᳚त् । \newline
42. ते॒ गात्रा॒द् गात्रा᳚त् ते ते॒ गात्रा॑ द॒ग्निना॒ ऽग्निना॒ गात्रा᳚त् ते ते॒ गात्रा॑ द॒ग्निना᳚ । \newline
43. गात्रा॑ द॒ग्निना॒ ऽग्निना॒ गात्रा॒द् गात्रा॑ द॒ग्निना॑ प॒च्यमा॑नात् प॒च्यमा॑ना द॒ग्निना॒ गात्रा॒द् गात्रा॑ द॒ग्निना॑ प॒च्यमा॑नात् । \newline
44. अ॒ग्निना॑ प॒च्यमा॑नात् प॒च्यमा॑ना द॒ग्निना॒ ऽग्निना॑ प॒च्यमा॑ना द॒भ्य॑भि प॒च्यमा॑ना द॒ग्निना॒ ऽग्निना॑ प॒च्यमा॑ना द॒भि । \newline
45. प॒च्यमा॑ना द॒भ्य॑भि प॒च्यमा॑नात् प॒च्यमा॑ना द॒भि शूलꣳ॒॒ शूल॑ म॒भि प॒च्यमा॑नात् प॒च्यमा॑ना द॒भि शूल᳚म् । \newline
46. अ॒भि शूलꣳ॒॒ शूल॑ म॒भ्य॑भि शूल॒म् निह॑तस्य॒ निह॑तस्य॒ शूल॑ म॒भ्य॑भि शूल॒म् निह॑तस्य । \newline
47. शूल॒म् निह॑तस्य॒ निह॑तस्य॒ शूलꣳ॒॒ शूल॒म् निह॑तस्या व॒धाव॑ त्यव॒धाव॑ति॒ निह॑तस्य॒ शूलꣳ॒॒ शूल॒म् निह॑तस्या व॒धाव॑ति । \newline
48. निह॑तस्या व॒धाव॑ त्यव॒धाव॑ति॒ निह॑तस्य॒ निह॑तस्या व॒धाव॑ति । \newline
49. निह॑त॒स्येति॒ नि - ह॒त॒स्य॒ । \newline
50. अ॒व॒धाव॒तीत्य॑व - धाव॑ति । \newline
51. मा तत् तन् मा मा तद् भूम्या॒म् भूम्या॒म् तन् मा मा तद् भूम्या᳚म् । \newline
52. तद् भूम्या॒म् भूम्या॒म् तत् तद् भूम्या॒ मा भूम्या॒म् तत् तद् भूम्या॒ मा । \newline
53. भूम्या॒ मा भूम्या॒म् भूम्या॒ मा श्रि॑ष च्छ्रिष॒दा भूम्या॒म् भूम्या॒ मा श्रि॑षत् । \newline
54. आ श्रि॑ष च्छ्रिष॒दा श्रि॑ष॒न् मा मा श्रि॑ष॒दा श्रि॑ष॒न् मा । \newline
55. श्रि॒ष॒न् मा मा श्रि॑ष च्छ्रिष॒न् मा तृणे॑षु॒ तृणे॑षु॒ मा श्रि॑ष च्छ्रिष॒न् मा तृणे॑षु । \newline
56. मा तृणे॑षु॒ तृणे॑षु॒ मा मा तृणे॑षु दे॒वेभ्यो॑ दे॒वेभ्य॒ स्तृणे॑षु॒ मा मा तृणे॑षु दे॒वेभ्यः॑ । \newline
57. तृणे॑षु दे॒वेभ्यो॑ दे॒वेभ्य॒ स्तृणे॑षु॒ तृणे॑षु दे॒वेभ्य॒ स्तत् तद् दे॒वेभ्य॒ स्तृणे॑षु॒ तृणे॑षु दे॒वेभ्य॒ स्तत् । \newline
58. दे॒वेभ्य॒ स्तत् तद् दे॒वेभ्यो॑ दे॒वेभ्य॒ स्त दु॒शद्भ्य॑ उ॒शद्भ्य॒ स्तद् दे॒वेभ्यो॑ दे॒वेभ्य॒ स्त दु॒शद्भ्यः॑ । \newline
59. तदु॒शद्भ्य॑ उ॒शद्भ्य॒ स्तत् तदु॒शद्भ्यो॑ रा॒तꣳ रा॒त मु॒शद्भ्य॒ स्तत् तदु॒शद्भ्यो॑ रा॒तम् । \newline
60. उ॒शद्भ्यो॑ रा॒तꣳ रा॒त मु॒शद्भ्य॑ उ॒शद्भ्यो॑ रा॒त म॑स्त्वस्तु रा॒त मु॒शद्भ्य॑ उ॒शद्भ्यो॑ रा॒त म॑स्तु । \newline
61. उ॒शद्भ्य॒ इत्यु॒शत् - भ्यः॒ । \newline
62. रा॒त म॑स्त्वस्तु रा॒तꣳ रा॒त म॑स्तु । \newline
63. अ॒स्त्वित्य॑स्तु । \newline
\pagebreak
\markright{ TS 4.6.9.1  \hfill https://www.vedavms.in \hfill}

\section{ TS 4.6.9.1 }

\textbf{TS 4.6.9.1 } \newline
\textbf{Samhita Paata} \newline

ये वा॒जिनं॑ परि॒पश्य॑न्ति प॒क्वं ॅय ई॑मा॒हुः सु॑र॒भिर्निर्.ह॒रेति॑ । ये चार्व॑तो माꣳसभि॒क्षामु॒पास॑त उ॒तो तेषा॑म॒भिगू᳚र्तिर्न इन्वतु ॥ यन्नीक्ष॑णं माꣳ॒॒स्पच॑न्या उ॒खाया॒ या पात्रा॑णि यू॒ष्ण आ॒सेच॑नानि । ऊ॒ष्म॒ण्या॑ऽपि॒धाना॑ चरू॒णाम॒ङ्काः सू॒नाः परि॑ भूष॒न्त्यश्वं᳚ ॥ नि॒क्रम॑णं नि॒षद॑नं ॅवि॒वर्त॑नं॒ ॅयच्च॒ पड्बी॑श॒मर्व॑तः । यच्च॑ प॒पौ यच्च॑ घा॒सिं -[  ] \newline

\textbf{Pada Paata} \newline

ये । वा॒जिन᳚म् । प॒रि॒पश्य॒न्तीति॑ परि-पश्य॑न्ति । प॒क्वम् । ये । ई॒म् । आ॒हुः । सु॒र॒भिः । निरिति॑ । ह॒र॒ । इति॑ ॥ ये । च॒ । अर्व॑तः । माꣳ॒॒स॒भि॒क्षामिति॑ माꣳस - भि॒क्षाम् । उ॒पास॑त॒ इत्यु॑प - आस॑ते । उ॒तो इति॑ । तेषा᳚म् । अ॒भिगू᳚र्ति॒रित्य॒भि - गू॒र्तिः॒ । नः॒ । इ॒न्व॒तु॒ ॥ यत् । नीक्ष॑णम् । माꣳ॒॒स्पच॑न्याः । उ॒खायाः᳚ । या । पात्रा॑णि । यू॒ष्णः । आ॒सेच॑ना॒नीत्या᳚ - सेच॑नानि ॥ ऊ॒ष्म॒ण्या᳚ । अ॒पि॒धानेत्य॑पि - धाना᳚ । च॒रू॒णाम् । अ॒ङ्काः । सू॒नाः । परीति॑ । भू॒ष॒न्ति॒ । अश्व᳚म् ॥ नि॒क्रम॑ण॒मिति॑ नि - क्रम॑णम् । नि॒षद॑न॒मिति॑ नि - सद॑नम् । वि॒वर्त॑न॒मिति॑ वि - वर्त॑नम् । यत् । च॒ । पड्बी॑शम् । अर्व॑तः ॥ यत् । च॒ । प॒पौ । यत् । च॒ । घा॒सिम् ।  \newline


\textbf{Krama Paata} \newline

ये वा॒जिन᳚म् । वा॒जिन॑म् परि॒पश्य॑न्ति । प॒रि॒पश्य॑न्ति प॒क्वम् । प॒रि॒पश्य॒न्तीति॑ परि - पश्य॑न्ति । प॒क्वं ॅये । य ई᳚म् । ई॒मा॒हुः । आ॒हुः सु॑र॒भिः । सु॒र॒भिर् निः । निर्. ह॑र । ह॒रेति॑ । इतीतीति॑ ॥ ये च॑ । चार्व॑तः । अर्व॑तो माꣳसभि॒क्षाम् । माꣳ॒॒स॒भि॒क्षामु॒पास॑ते । माꣳ॒॒स॒भि॒क्षामिति॑ माꣳस - भि॒क्षाम् । उ॒पास॑त उ॒तो । उ॒पास॑त॒ इत्यु॑प - आस॑ते । उ॒तो तेषा᳚म् । उ॒तो इत्यु॒तो । तेषा॑म॒भिगू᳚र्तिः । अ॒भिगू᳚र्तिर् नः । अ॒भिगू᳚र्ति॒रित्य॒भि - गू॒र्तिः॒ । न॒ इ॒न्व॒तु॒ । इ॒न्व॒त्विती᳚न्वतु ॥ यन् नीक्ष॑णम् । नीक्ष॑णम् माꣳ॒॒स्पच॑न्याः । माꣳ॒॒स्पच॑न्या उ॒खायाः᳚ । उ॒खाया॒ या । या पात्रा॑णि । पात्रा॑णि यू॒ष्णः । यू॒ष्ण आ॒सेच॑नानि । आ॒सेच॑ना॒नीत्या᳚ - सेच॑नानि ॥ ऊ॒ष्म॒ण्या॑ऽपि॒धाना᳚ । अ॒पि॒धाना॑ चरू॒णाम् । अ॒पि॒धानेत्य॑पि - धाना᳚ । च॒रू॒णाम॒ङ्काः । अ॒ङ्काः सू॒नाः । सू॒नाः परि॑ । परि॑ भूषन्ति । भू॒ष॒न्त्यश्व᳚म् । अश्व॒मित्यश्व᳚म् ॥ नि॒क्रम॑णम् नि॒षद॑नम् । नि॒क्रम॑ण॒मिति॑ नि - क्रम॑णम् । नि॒षद॑नं ॅवि॒वर्त॑नम् । नि॒षद॑न॒मिति॑ नि - सद॑नम् । वि॒वर्त॑नं॒ ॅयत् । वि॒वर्त॑न॒मिति॑ वि - वर्त॑नम् । यच्च॑ । च॒ पड्बी॑शम् । पड्बी॑श॒मर्व॑तः । अर्व॑त॒ इत्यव॑र्तः ॥ यच्च॑ । च॒ प॒पौ । प॒पौ यत् । यच्च॑ । च॒ घा॒सिम् । घा॒सिम् ज॒घास॑ \newline

\textbf{Jatai Paata} \newline

1. ये वा॒जिनं॑ ॅवा॒जिनं॒ ॅये ये वा॒जिन᳚म् । \newline
2. वा॒जिन॑म् परि॒पश्य॑न्ति परि॒पश्य॑न्ति वा॒जिनं॑ ॅवा॒जिन॑म् परि॒पश्य॑न्ति । \newline
3. प॒रि॒पश्य॑न्ति प॒क्वम् प॒क्वम् प॑रि॒पश्य॑न्ति परि॒पश्य॑न्ति प॒क्वम् । \newline
4. प॒रि॒पश्य॒न्तीति॑ परि - पश्य॑न्ति । \newline
5. प॒क्वं ॅये ये प॒क्वम् प॒क्वं ॅये । \newline
6. य ई॑मीं॒ ॅये य ई᳚म् । \newline
7. ई॒मा॒हु रा॒हु री॑मी मा॒हुः । \newline
8. आ॒हुः सु॑र॒भिः सु॑र॒भि रा॒हु रा॒हुः सु॑र॒भिः । \newline
9. सु॒र॒भिर् निर् णिः सु॑र॒भिः सु॑र॒भिर् निः । \newline
10. निर्. ह॑र हर॒ निर् णिर्. ह॑र । \newline
11. ह॒रेतीति॑ हर ह॒रेति॑ । \newline
12. इतीतीति॑ । \newline
13. ये च॑ च॒ ये ये च॑ । \newline
14. चार्व॑तो॒ अर्व॑तश्च॒ चार्व॑तः । \newline
15. अर्व॑तो माꣳसभि॒क्षाम् माꣳ॑सभि॒क्षा मर्व॑तो॒ अर्व॑तो माꣳसभि॒क्षाम् । \newline
16. माꣳ॒॒स॒भि॒क्षा मु॒पास॑त उ॒पास॑ते माꣳसभि॒क्षाम् माꣳ॑सभि॒क्षा मु॒पास॑ते । \newline
17. माꣳ॒॒स॒भि॒क्षामिति॑ माꣳस - भि॒क्षाम् । \newline
18. उ॒पास॑त उ॒तो उ॒तो उ॒पास॑त उ॒पास॑त उ॒तो । \newline
19. उ॒पास॑त॒ इत्यु॑प - आस॑ते । \newline
20. उ॒तो तेषा॒म् तेषा॑ मु॒तो उ॒तो तेषा᳚म् । \newline
21. उ॒तो इत्यु॒तो । \newline
22. तेषा॑ म॒भिगू᳚र्ति र॒भिगू᳚र्ति॒ स्तेषा॒म् तेषा॑ म॒भिगू᳚र्तिः । \newline
23. अ॒भिगू᳚र्तिर् नो नो अ॒भिगू᳚र्ति र॒भिगू᳚र्तिर् नः । \newline
24. अ॒भिगू᳚र्ति॒रित्य॒भि - गू॒र्तिः॒ । \newline
25. न॒ इ॒न्व॒ त्वि॒न्‌व॒तु॒ नो॒ न॒ इ॒न्व॒तु॒ । \newline
26. इ॒न्व॒त्विती᳚न्वतु । \newline
27. यन् नीक्ष॑ण॒म् नीक्ष॑णं॒ ॅयद् यन् नीक्ष॑णम् । \newline
28. नीक्ष॑णम् माꣳ॒॒स्पच॑न्या माꣳ॒॒स्पच॑न्या॒ नीक्ष॑ण॒म् नीक्ष॑णम् माꣳ॒॒स्पच॑न्याः । \newline
29. माꣳ॒॒स्पच॑न्या उ॒खाया॑ उ॒खाया॑ माꣳ॒॒स्पच॑न्या माꣳ॒॒स्पच॑न्या उ॒खायाः᳚ । \newline
30. उ॒खाया॒ या योखाया॑ उ॒खाया॒ या । \newline
31. या पात्रा॑णि॒ पात्रा॑णि॒ या या पात्रा॑णि । \newline
32. पात्रा॑णि यू॒ष्णो यू॒ष्णः पात्रा॑णि॒ पात्रा॑णि यू॒ष्णः । \newline
33. यू॒ष्ण आ॒सेच॑ना न्या॒सेच॑नानि यू॒ष्णो यू॒ष्ण आ॒सेच॑नानि । \newline
34. आ॒सेच॑ना॒नीत्या᳚ - सेच॑नानि । \newline
35. ऊ॒ष्म॒ण्या॑ ऽपि॒धाना॑ ऽपि॒धा नो᳚ष्म॒ण् यो᳚ष्म॒ण्या॑ ऽपि॒धाना᳚ । \newline
36. अ॒पि॒धाना॑ चरू॒णाम् च॑रू॒णा म॑पि॒धाना॑ ऽपि॒धाना॑ चरू॒णाम् । \newline
37. अ॒पि॒धानेत्य॑पि - धाना᳚ । \newline
38. च॒रू॒णा म॒ङ्का अ॒ङ्का श्च॑रू॒णाम् च॑रू॒णा म॒ङ्काः । \newline
39. अ॒ङ्काः सू॒नाः सू॒ना अ॒ङ्का अ॒ङ्काः सू॒नाः । \newline
40. सू॒नाः परि॒ परि॑ सू॒नाः सू॒नाः परि॑ । \newline
41. परि॑ भूषन्ति भूषन्ति॒ परि॒ परि॑ भूषन्ति । \newline
42. भू॒ष॒ न्त्यश्व॒ मश्व॑म् भूषन्ति भूष॒ न्त्यश्व᳚म् । \newline
43. अश्व॒मित्यश्व᳚म् । \newline
44. नि॒क्रम॑णम् नि॒षद॑नम् नि॒षद॑नम् नि॒क्रम॑णम् नि॒क्रम॑णम् नि॒षद॑नम् । \newline
45. नि॒क्रम॑ण॒मिति॑ नि - क्रम॑णम् । \newline
46. नि॒षद॑नं ॅवि॒वर्त॑नं ॅवि॒वर्त॑नम् नि॒षद॑नम् नि॒षद॑नं ॅवि॒वर्त॑नम् । \newline
47. नि॒षद॑न॒मिति॑ नि - सद॑नम् । \newline
48. वि॒वर्त॑नं॒ ॅयद् यद् वि॒वर्त॑नं ॅवि॒वर्त॑नं॒ ॅयत् । \newline
49. वि॒वर्त॑न॒मिति॑ वि - वर्त॑नम् । \newline
50. यच् च॑ च॒ यद् यच् च॑ । \newline
51. च॒ पड्बी॑श॒म् पड्बी॑शम् च च॒ पड्बी॑शम् । \newline
52. पड्बी॑श॒ मर्व॑तो॒ अर्व॑तः॒ पड्बी॑श॒म् पड्बी॑श॒ मर्व॑तः । \newline
53. अर्व॑त॒ इत्यर्व॑तः । \newline
54. यच् च॑ च॒ यद् यच् च॑ । \newline
55. च॒ प॒पौ प॒पौ च॑ च प॒पौ । \newline
56. प॒पौ यद् यत् प॒पौ प॒पौ यत् । \newline
57. यच् च॑ च॒ यद् यच् च॑ । \newline
58. च॒ घा॒सिम् घा॒सिम् च॑ च घा॒सिम् । \newline
59. घा॒सिम् ज॒घास॑ ज॒घास॑ घा॒सिम् घा॒सिम् ज॒घास॑ । \newline

\textbf{Ghana Paata } \newline

1. ये वा॒जिनं॑ ॅवा॒जिनं॒ ॅये ये वा॒जिन॑म् परि॒पश्य॑न्ति परि॒पश्य॑न्ति वा॒जिनं॒ ॅये ये वा॒जिन॑म् परि॒पश्य॑न्ति । \newline
2. वा॒जिन॑म् परि॒पश्य॑न्ति परि॒पश्य॑न्ति वा॒जिनं॑ ॅवा॒जिन॑म् परि॒पश्य॑न्ति प॒क्वम् प॒क्वम् प॑रि॒पश्य॑न्ति वा॒जिनं॑ ॅवा॒जिन॑म् परि॒पश्य॑न्ति प॒क्वम् । \newline
3. प॒रि॒पश्य॑न्ति प॒क्वम् प॒क्वम् प॑रि॒पश्य॑न्ति परि॒पश्य॑न्ति प॒क्वं ॅये ये प॒क्वम् प॑रि॒पश्य॑न्ति परि॒पश्य॑न्ति प॒क्वं ॅये । \newline
4. प॒रि॒पश्य॒न्तीति॑ परि - पश्य॑न्ति । \newline
5. प॒क्वं ॅये ये प॒क्वम् प॒क्वं ॅय ई॑मीं॒ ॅये प॒क्वम् प॒क्वं ॅय ई᳚म् । \newline
6. य ई॑म् ईं॒ ॅये य ई॑ मा॒हु रा॒हुरीं॒ ॅये य ई॑ मा॒हुः । \newline
7. ई॒ मा॒हु रा॒हु री॑मी मा॒हुः सु॑र॒भिः सु॑र॒भि रा॒हु री॑मी मा॒हुः सु॑र॒भिः । \newline
8. आ॒हुः सु॑र॒भिः सु॑र॒भि रा॒हु रा॒हुः सु॑र॒भिर् निर् णिः सु॑र॒भि रा॒हु रा॒हुः सु॑र॒भिर् निः । \newline
9. सु॒र॒भिर् निर् णिः सु॑र॒भिः सु॑र॒भिर् निर्. ह॑र हर॒ निः सु॑र॒भिः सु॑र॒भिर् निर्. ह॑र । \newline
10. निर्. ह॑र हर॒ निर् णिर्. ह॒रेतीति॑ हर॒ निर् णिर्. ह॒रेति॑ । \newline
11. ह॒रेतीति॑ हर ह॒रेति॑ । \newline
12. इतीतीति॑ । \newline
13. ये च॑ च॒ ये ये चार्व॑तो॒ अर्व॑तश्च॒ ये ये चार्व॑तः । \newline
14. चार्व॑तो॒ अर्व॑तश्च॒ चार्व॑तो माꣳसभि॒क्षाम् माꣳ॑सभि॒क्षा मर्व॑तश्च॒ चार्व॑तो माꣳसभि॒क्षाम् । \newline
15. अर्व॑तो माꣳसभि॒क्षाम् माꣳ॑सभि॒क्षा मर्व॑तो॒ अर्व॑तो माꣳसभि॒क्षा मु॒पास॑त उ॒पास॑ते माꣳसभि॒क्षा मर्व॑तो॒ अर्व॑तो माꣳसभि॒क्षा मु॒पास॑ते । \newline
16. माꣳ॒॒स॒भि॒क्षा मु॒पास॑त उ॒पास॑ते माꣳसभि॒क्षाम् माꣳ॑सभि॒क्षा मु॒पास॑त उ॒तो उ॒तो उ॒पास॑ते माꣳसभि॒क्षाम् माꣳ॑सभि॒क्षा मु॒पास॑त उ॒तो । \newline
17. माꣳ॒॒स॒भि॒क्षामिति॑ माꣳस - भि॒क्षाम् । \newline
18. उ॒पास॑त उ॒तो उ॒तो उ॒पास॑त उ॒पास॑त उ॒तो तेषा॒म् तेषा॑ मु॒तो उ॒पास॑त उ॒पास॑त उ॒तो तेषा᳚म् । \newline
19. उ॒पास॑त॒ इत्यु॑प - आस॑ते । \newline
20. उ॒तो तेषा॒म् तेषा॑ मु॒तो उ॒तो तेषा॑ म॒भिगू᳚र्ति र॒भिगू᳚र्ति॒ स्तेषा॑ मु॒तो उ॒तो तेषा॑ म॒भिगू᳚र्तिः । \newline
21. उ॒तो इत्यु॒तो । \newline
22. तेषा॑ म॒भिगू᳚र्ति र॒भिगू᳚र्ति॒ स्तेषा॒म् तेषा॑ म॒भिगू᳚र्तिर् नो नो अ॒भिगू᳚र्ति॒ स्तेषा॒म् तेषा॑ म॒भिगू᳚र्तिर् नः । \newline
23. अ॒भिगू᳚र्तिर् नो नो अ॒भिगू᳚र्ति र॒भिगू᳚र्तिर् न इन्व त्विन्वतु नो अ॒भिगू᳚र्ति र॒भिगू᳚र्तिर् न इन्वतु । \newline
24. अ॒भिगू᳚र्ति॒रित्य॒भि - गू॒र्तिः॒ । \newline
25. न॒ इ॒न्व॒ त्वि॒न्व॒तु॒ नो॒ न॒ इ॒न्व॒तु॒ । \newline
26. इ॒न्व॒त्विती᳚न्वतु । \newline
27. यन् नीक्ष॑ण॒म् नीक्ष॑णं॒ ॅयद् यन् नीक्ष॑णम् माꣳ॒॒स्पच॑न्या माꣳ॒॒स्पच॑न्या॒ नीक्ष॑णं॒ ॅयद् यन् नीक्ष॑णम् माꣳ॒॒स्पच॑न्याः । \newline
28. नीक्ष॑णम् माꣳ॒॒स्पच॑न्या माꣳ॒॒स्पच॑न्या॒ नीक्ष॑ण॒म् नीक्ष॑णम् माꣳ॒॒स्पच॑न्या उ॒खाया॑ उ॒खाया॑ माꣳ॒॒स्पच॑न्या॒ नीक्ष॑ण॒म् नीक्ष॑णम् माꣳ॒॒स्पच॑न्या उ॒खायाः᳚ । \newline
29. माꣳ॒॒स्पच॑न्या उ॒खाया॑ उ॒खाया॑ माꣳ॒॒स्पच॑न्या माꣳ॒॒स्पच॑न्या उ॒खाया॒ या योखाया॑ माꣳ॒॒स्पच॑न्या माꣳ॒॒स्पच॑न्या उ॒खाया॒ या । \newline
30. उ॒खाया॒ या योखाया॑ उ॒खाया॒ या पात्रा॑णि॒ पात्रा॑णि॒ योखाया॑ उ॒खाया॒ या पात्रा॑णि । \newline
31. या पात्रा॑णि॒ पात्रा॑णि॒ या या पात्रा॑णि यू॒ष्णो यू॒ष्णः पात्रा॑णि॒ या या पात्रा॑णि यू॒ष्णः । \newline
32. पात्रा॑णि यू॒ष्णो यू॒ष्णः पात्रा॑णि॒ पात्रा॑णि यू॒ष्ण आ॒सेच॑ना न्या॒सेच॑नानि यू॒ष्णः पात्रा॑णि॒ पात्रा॑णि यू॒ष्ण आ॒सेच॑नानि । \newline
33. यू॒ष्ण आ॒सेच॑ना न्या॒सेच॑नानि यू॒ष्णो यू॒ष्ण आ॒सेच॑नानि । \newline
34. आ॒सेच॑ना॒नीत्या᳚ - सेच॑नानि । \newline
35. ऊ॒ष्म॒ण्या॑ ऽपि॒धाना॑ ऽपि॒धानो᳚ ष्म॒ण्यो᳚ ष्म॒ण्या॑ ऽपि॒धाना॑ चरू॒णाम् च॑रू॒णा म॑पि॒धानो᳚ ष्म॒ण्यो᳚ ष्म॒ण्या॑ ऽपि॒धाना॑ चरू॒णाम् । \newline
36. अ॒पि॒धाना॑ चरू॒णाम् च॑रू॒णा म॑पि॒धाना॑ ऽपि॒धाना॑ चरू॒णा म॒ङ्का अ॒ङ्का श्च॑रू॒णा म॑पि॒धाना॑ ऽपि॒धाना॑ चरू॒णा म॒ङ्काः । \newline
37. अ॒पि॒धानेत्य॑पि - धाना᳚ । \newline
38. च॒रू॒णा म॒ङ्का अ॒ङ्का श्च॑रू॒णाम् च॑रू॒णा म॒ङ्काः सू॒नाः सू॒ना अ॒ङ्का श्च॑रू॒णाम् च॑रू॒णा म॒ङ्काः सू॒नाः । \newline
39. अ॒ङ्काः सू॒नाः सू॒ना अ॒ङ्का अ॒ङ्काः सू॒नाः परि॒ परि॑ सू॒ना अ॒ङ्का अ॒ङ्काः सू॒नाः परि॑ । \newline
40. सू॒नाः परि॒ परि॑ सू॒नाः सू॒नाः परि॑ भूषन्ति भूषन्ति॒ परि॑ सू॒नाः सू॒नाः परि॑ भूषन्ति । \newline
41. परि॑ भूषन्ति भूषन्ति॒ परि॒ परि॑ भूष॒ न्त्यश्व॒ मश्व॑म् भूषन्ति॒ परि॒ परि॑ भूष॒ न्त्यश्व᳚म् । \newline
42. भू॒ष॒ न्त्यश्व॒ मश्व॑म् भूषन्ति भूष॒ न्त्यश्व᳚म् । \newline
43. अश्व॒मित्यश्व᳚म् । \newline
44. नि॒क्रम॑णम् नि॒षद॑नम् नि॒षद॑नम् नि॒क्रम॑णम् नि॒क्रम॑णम् नि॒षद॑नं ॅवि॒वर्त॑नं ॅवि॒वर्त॑नम् नि॒षद॑नम् नि॒क्रम॑णम् नि॒क्रम॑णम् नि॒षद॑नं ॅवि॒वर्त॑नम् । \newline
45. नि॒क्रम॑ण॒मिति॑ नि - क्रम॑णम् । \newline
46. नि॒षद॑नं ॅवि॒वर्त॑नं ॅवि॒वर्त॑नम् नि॒षद॑नम् नि॒षद॑नं ॅवि॒वर्त॑नं॒ ॅयद् यद् वि॒वर्त॑नम् नि॒षद॑नम् नि॒षद॑नं ॅवि॒वर्त॑नं॒ ॅयत् । \newline
47. नि॒षद॑न॒मिति॑ नि - सद॑नम् । \newline
48. वि॒वर्त॑नं॒ ॅयद् यद् वि॒वर्त॑नं ॅवि॒वर्त॑नं॒ ॅयच् च॑ च॒ यद् वि॒वर्त॑नं ॅवि॒वर्त॑नं॒ ॅयच् च॑ । \newline
49. वि॒वर्त॑न॒मिति॑ वि - वर्त॑नम् । \newline
50. यच् च॑ च॒ यद् यच् च॒ पड्बी॑श॒म् पड्बी॑शम् च॒ यद् यच् च॒ पड्बी॑शम् । \newline
51. च॒ पड्बी॑श॒म् पड्बी॑शम् च च॒ पड्बी॑श॒ मर्व॑तो॒ अर्व॑तः॒ पड्बी॑शम् च च॒ पड्बी॑श॒ मर्व॑तः । \newline
52. पड्बी॑श॒ मर्व॑तो॒ अर्व॑तः॒ पड्बी॑श॒म् पड्बी॑श॒ मर्व॑तः । \newline
53. अर्व॑त॒ इत्यर्व॑तः । \newline
54. यच् च॑ च॒ यद् यच् च॑ प॒पौ प॒पौ च॒ यद् यच् च॑ प॒पौ । \newline
55. च॒ प॒पौ प॒पौ च॑ च प॒पौ यद् यत् प॒पौ च॑ च प॒पौ यत् । \newline
56. प॒पौ यद् यत् प॒पौ प॒पौ यच् च॑ च॒ यत् प॒पौ प॒पौ यच् च॑ । \newline
57. यच् च॑ च॒ यद् यच् च॑ घा॒सिम् घा॒सिम् च॒ यद् यच् च॑ घा॒सिम् । \newline
58. च॒ घा॒सिम् घा॒सिम् च॑ च घा॒सिम् ज॒घास॑ ज॒घास॑ घा॒सिम् च॑ च घा॒सिम् ज॒घास॑ । \newline
59. घा॒सिम् ज॒घास॑ ज॒घास॑ घा॒सिम् घा॒सिम् ज॒घास॒ सर्वा॒ सर्वा॑ ज॒घास॑ घा॒सिम् घा॒सिम् ज॒घास॒ सर्वा᳚ । \newline
\pagebreak
\markright{ TS 4.6.9.2  \hfill https://www.vedavms.in \hfill}

\section{ TS 4.6.9.2 }

\textbf{TS 4.6.9.2 } \newline
\textbf{Samhita Paata} \newline

ज॒घास॒ सर्वा॒ ता ते॒ अपि॑ दे॒वेष्व॑स्तु ॥ मा त्वा॒ऽग्नि-र्द्ध्व॑नयिद्-धू॒मग॑न्धि॒र्मोखा भ्राज॑न्त्य॒भि वि॑क्त॒ जघ्रिः॑ । इ॒ष्टं ॅवी॒तम॒भिगू᳚र्तं॒ ॅवष॑ट्कृतं॒ तं दे॒वासः॒ प्रति॑ गृभ्ण॒न्त्यश्वं᳚ ॥ यदश्वा॑य॒ वास॑ उपस्तृ॒णन्त्य॑धीवा॒सं ॅया हिर॑ण्यान्यस्मै । स॒न्दान॒मर्व॑न्तं॒ पड्बी॑शं प्रि॒या दे॒वेष्वा या॑मयन्ति ॥ यत् ते॑ सा॒दे मह॑सा॒ शूकृ॑तस्य॒ पार्ष्णि॑या वा॒ कश॑या - [  ] \newline

\textbf{Pada Paata} \newline

ज॒घास॑ । सर्वा᳚ । ता । ते॒ । अपीति॑ । दे॒वेषु॑ । अ॒स्तु॒ ॥ मा । त्वा॒ । अ॒ग्निः । ध्व॒न॒यि॒त् । धू॒मग॑न्धि॒रिति॑ धू॒म - ग॒न्धिः॒ । मा । उ॒खा । भ्राज॑न्ति । अ॒भीति॑ । वि॒क्त॒ । जघ्रिः॑ ॥ इ॒ष्टम् । वी॒तम् । अ॒भिगू᳚र्त॒मित्य॒भि - गू॒र्त॒म् । वष॑ट्कृत॒मिति॒ वष॑ट् - कृ॒त॒म् । तम् । दे॒वासः॑ । प्रतीति॑ । गृ॒भ्ण॒न्ति॒ । अश्व᳚म् ॥ यत् । अश्वा॑य । वासः॑ । उ॒प॒स्तृ॒णन्तीत्यु॑प - स्तृ॒णन्ति॑ । अ॒धी॒वा॒समित्य॑धि - वा॒सम् । या । हिर॑ण्यानि । अ॒स्मै॒ ॥ स॒न्दान॒मिति॑ सं - दान᳚म् । अर्व॑न्तम् । पड्बी॑शम् । प्रि॒या । दे॒वेषु॑ । एति॑ । या॒म॒य॒न्ति॒ ॥ यत् । ते॒ । सा॒दे । मह॑सा । शूकृ॑त॒स्येति॒ शू - कृ॒त॒स्य॒ । पार्ष्णि॑या । वा॒ । कश॑या ।  \newline


\textbf{Krama Paata} \newline

ज॒घास॒ सर्वा᳚ । सर्वा॒ ता । ता ते᳚ । ते॒ अपि॑ । अपि॑ दे॒वेषु॑ । दे॒वेष्व॑स्तु । अ॒स्त्वित्य॑स्तु ॥ मा त्वा᳚ । त्वा॒ऽग्निः । अ॒ग्नि ध्व॑नयित् । ध्व॒न॒यि॒द् धू॒मग॑न्धिः । धू॒मग॑न्धि॒र् मा । धू॒मग॑न्धि॒रिति॑ धू॒म - ग॒न्धिः॒ । मोखा । उ॒खा भ्राज॑न्ति । भ्राज॑न्त्य॒भि । अ॒भि वि॑क्त । वि॒क्त॒ जघ्रिः॑ । जघ्रि॒रिति॒ जघ्रिः॑ ॥ इ॒ष्टं ॅवी॒तम् । वी॒तम॒भिगू᳚र्तम् । अ॒भिगू᳚र्तं॒ ॅवष॑ट्कृतम् । अ॒भिगू᳚र्त॒मित्य॒भि - गू॒र्त॒म् । वष॑ट्कृत॒म् तम् । वष॑ट्कृत॒मिति॒ वष॑ट् - कृ॒त॒म् । तम् दे॒वासः॑ । दे॒वासः॒ प्रति॑ । प्रति॑ गृभ्णन्ति । गृ॒भ्ण॒न्त्यश्व᳚म् । अश्व॒मित्यश्व᳚म् ॥ यदश्वा॑य । अश्वा॑य॒ वासः॑ । वास॑ उपस्तृ॒णन्ति॑ । उ॒प॒स्तृ॒णन्त्य॑धीवा॒सम् । उ॒प॒स्तृ॒णन्तीत्यु॑प - स्तृ॒णन्ति॑ । अ॒धी॒वा॒सं ॅया । अ॒धी॒वा॒समित्य॑धि - वा॒सम् । या हिर॑ण्यानि । हिर॑ण्यान्यस्मै । अ॒स्मा॒ इत्य॑स्मै ॥ स॒न्दान॒मर्व॑न्तम् । स॒न्दान॒मिति॑ सम् - दान᳚म् । अर्व॑न्त॒म् पड्बी॑शम् । पड्बी॑शम् प्रि॒या । प्रि॒या दे॒वेषु॑ । दे॒वेष्वा । आ या॑मयन्ति । या॒म॒य॒न्तीति॑ यामयन्ति ॥ यत् ते᳚ । ते॒ सा॒दे । सा॒दे मह॑सा । मह॑सा॒ शूकृ॑तस्य । शूकृ॑तस्य॒ पार्ष्णि॑या । शूकृ॑त॒स्येति॒ शू - कृ॒त॒स्य॒ । पार्ष्णि॑या वा । वा॒ कश॑या । कश॑या वा \newline

\textbf{Jatai Paata} \newline

1. ज॒घास॒ सर्वा॒ सर्वा॑ ज॒घास॑ ज॒घास॒ सर्वा᳚ । \newline
2. सर्वा॒ ता ता सर्वा॒ सर्वा॒ ता । \newline
3. ता ते॑ ते॒ ता ता ते᳚ । \newline
4. ते॒ अप्यपि॑ ते ते॒ अपि॑ । \newline
5. अपि॑ दे॒वेषु॑ दे॒वे ष्वप्यपि॑ दे॒वेषु॑ । \newline
6. दे॒वे ष्व॑स्त्वस्तु दे॒वेषु॑ दे॒वे ष्व॑स्तु । \newline
7. अ॒स्त्वित्य॑स्तु । \newline
8. मा त्वा᳚ त्वा॒ मा मा त्वा᳚ । \newline
9. त्वा॒ ऽग्नि र॒ग्नि स्त्वा᳚ त्वा॒ ऽग्निः । \newline
10. अ॒ग्निर् ध्व॑नयिद् ध्वनयि द॒ग्नि र॒ग्निर् ध्व॑नयित् । \newline
11. ध्व॒न॒यि॒द् धू॒मग॑न्धिर् धू॒मग॑न्धिर् ध्वनयिद् ध्वनयिद् धू॒मग॑न्धिः । \newline
12. धू॒मग॑न्धि॒र् मा मा धू॒मग॑न्धिर् धू॒मग॑न्धि॒र् मा । \newline
13. धू॒मग॑न्धि॒रिति॑ धू॒म - ग॒न्धिः॒ । \newline
14. मोखोखा मा मोखा । \newline
15. उ॒खा भ्राज॑न्ती॒ भ्राज॑ न्त्यु॒खोखा भ्राज॑न्ती । \newline
16. भ्राज॑ न्त्य॒भ्य॑भि भ्राज॑न्ती॒ भ्राज॑ न्त्य॒भि । \newline
17. अ॒भि वि॑क्त विक्ता॒भ्य॑भि वि॑क्त । \newline
18. वि॒क्त॒ जघ्रि॒र् जघ्रि॑र् विक्त विक्त॒ जघ्रिः॑ । \newline
19. जघ्रि॒रिति॒ जघ्रिः॑ । \newline
20. इ॒ष्टं ॅवी॒तं ॅवी॒त मि॒ष्ट मि॒ष्टं ॅवी॒तम् । \newline
21. वी॒त म॒भिगू᳚र्त म॒भिगू᳚र्तं ॅवी॒तं ॅवी॒त म॒भिगू᳚र्तम् । \newline
22. अ॒भिगू᳚र्तं॒ ॅवष॑ट्कृतं॒ ॅवष॑ट्कृत म॒भिगू᳚र्त म॒भिगू᳚र्तं॒ ॅवष॑ट्कृतम् । \newline
23. अ॒भिगू᳚र्त॒मित्य॒भि - गू॒र्त॒म् । \newline
24. वष॑ट्कृत॒म् तम् तं ॅवष॑ट्कृतं॒ ॅवष॑ट्कृत॒म् तम् । \newline
25. वष॑ट्कृत॒मिति॒ वष॑ट् - कृ॒त॒म् । \newline
26. तम् दे॒वासो॑ दे॒वास॒ स्तम् तम् दे॒वासः॑ । \newline
27. दे॒वासः॒ प्रति॒ प्रति॑ दे॒वासो॑ दे॒वासः॒ प्रति॑ । \newline
28. प्रति॑ गृभ्णन्ति गृभ्णन्ति॒ प्रति॒ प्रति॑ गृभ्णन्ति । \newline
29. गृ॒भ्ण॒ न्त्यश्व॒ मश्व॑म् गृभ्णन्ति गृभ्ण॒ न्त्यश्व᳚म् । \newline
30. अश्व॒मित्यश्व᳚म् । \newline
31. यदश्वा॒या श्वा॑य॒ यद् यदश्वा॑य । \newline
32. अश्वा॑य॒ वासो॒ वासो ऽश्वा॒या श्वा॑य॒ वासः॑ । \newline
33. वास॑ उपस्तृ॒ण न्त्यु॑पस्तृ॒णन्ति॒ वासो॒ वास॑ उपस्तृ॒णन्ति॑ । \newline
34. उ॒प॒स्तृ॒ण न्त्य॑धीवा॒स म॑धीवा॒स मु॑पस्तृ॒णन् त्यु॑पस्तृ॒ण न्त्य॑धीवा॒सम् । \newline
35. उ॒प॒स्तृ॒णन्तीत्यु॑प - स्तृ॒णन्ति॑ । \newline
36. अ॒धी॒वा॒सं ॅया या ऽधी॑वा॒स म॑धीवा॒सं ॅया । \newline
37. अ॒धी॒वा॒समित्य॑धि - वा॒सम् । \newline
38. या हिर॑ण्यानि॒ हिर॑ण्यानि॒ या या हिर॑ण्यानि । \newline
39. हिर॑ण्या न्यस्मा अस्मै॒ हिर॑ण्यानि॒ हिर॑ण्या न्यस्मै । \newline
40. अ॒स्मै॒ इत्य॑स्मै । \newline
41. स॒न्दान॒ मर्व॑न्त॒ मर्व॑न्तꣳ स॒न्दानꣳ॑ स॒न्दान॒ मर्व॑न्तम् । \newline
42. स॒न्दान॒मिति॑ सं - दान᳚म् । \newline
43. अर्व॑न्त॒म् पड्बी॑श॒म् पड्बी॑श॒ मर्व॑न्त॒ मर्व॑न्त॒म् पड्बी॑शम् । \newline
44. पड्बी॑शम् प्रि॒या प्रि॒या पड्बी॑श॒म् पड्बी॑शम् प्रि॒या । \newline
45. प्रि॒या दे॒वेषु॑ दे॒वेषु॑ प्रि॒या प्रि॒या दे॒वेषु॑ । \newline
46. दे॒वेष्वा दे॒वेषु॑ दे॒वेष्वा । \newline
47. आ या॑मयन्ति यामय॒न्त्या या॑मयन्ति । \newline
48. या॒म॒य॒न्तीति॑ यामयन्ति । \newline
49. यत् ते॑ ते॒ यद् यत् ते᳚ । \newline
50. ते॒ सा॒दे सा॒दे ते॑ ते सा॒दे । \newline
51. सा॒दे मह॑सा॒ मह॑सा सा॒दे सा॒दे मह॑सा । \newline
52. मह॑सा॒ शूकृ॑तस्य॒ शूकृ॑तस्य॒ मह॑सा॒ मह॑सा॒ शूकृ॑तस्य । \newline
53. शूकृ॑तस्य॒ पार्ष्णि॑या॒ पार्ष्णि॑या॒ शूकृ॑तस्य॒ शूकृ॑तस्य॒ पार्ष्णि॑या । \newline
54. शूकृ॑त॒स्येति॒ शू - कृ॒त॒स्य॒ । \newline
55. पार्ष्णि॑या वा वा॒ पार्ष्णि॑या॒ पार्ष्णि॑या वा । \newline
56. वा॒ कश॑या॒ कश॑या वा वा॒ कश॑या । \newline
57. कश॑या वा वा॒ कश॑या॒ कश॑या वा । \newline

\textbf{Ghana Paata } \newline

1. ज॒घास॒ सर्वा॒ सर्वा॑ ज॒घास॑ ज॒घास॒ सर्वा॒ ता ता सर्वा॑ ज॒घास॑ ज॒घास॒ सर्वा॒ ता । \newline
2. सर्वा॒ ता ता सर्वा॒ सर्वा॒ ता ते॑ ते॒ ता सर्वा॒ सर्वा॒ ता ते᳚ । \newline
3. ता ते॑ ते॒ ता ता ते॒ अप्यपि॑ ते॒ ता ता ते॒ अपि॑ । \newline
4. ते॒ अप्यपि॑ ते ते॒ अपि॑ दे॒वेषु॑ दे॒वे ष्वपि॑ ते ते॒ अपि॑ दे॒वेषु॑ । \newline
5. अपि॑ दे॒वेषु॑ दे॒वे ष्वप्यपि॑ दे॒वे ष्व॑स्त्वस्तु दे॒वे ष्वप्यपि॑ दे॒वेष्व॑स्तु । \newline
6. दे॒वे ष्व॑स्त्वस्तु दे॒वेषु॑ दे॒वे ष्व॑स्तु । \newline
7. अ॒स्त्वित्य॑स्तु । \newline
8. मा त्वा᳚ त्वा॒ मा मा त्वा॒ ऽग्नि र॒ग्नि स्त्वा॒ मा मा त्वा॒ ऽग्निः । \newline
9. त्वा॒ ऽग्नि र॒ग्नि स्त्वा᳚ त्वा॒ ऽग्निर् ध्व॑नयिद् ध्वनयि द॒ग्नि स्त्वा᳚ त्वा॒ ऽग्निर् ध्व॑नयित् । \newline
10. अ॒ग्निर् ध्व॑नयिद् ध्वनयि द॒ग्नि र॒ग्निर् ध्व॑नयिद् धू॒मग॑न्धिर् धू॒मग॑न्धिर् ध्वनयि द॒ग्नि र॒ग्निर् ध्व॑नयिद् धू॒मग॑न्धिः । \newline
11. ध्व॒न॒यि॒द् धू॒मग॑न्धिर् धू॒मग॑न्धिर् ध्वनयिद् ध्वनयिद् धू॒मग॑न्धि॒र् मा मा धू॒मग॑न्धिर् ध्वनयिद् ध्वनयिद् धू॒मग॑न्धि॒र् मा । \newline
12. धू॒मग॑न्धि॒र् मा मा धू॒मग॑न्धिर् धू॒मग॑न्धि॒र् मोखोखा मा धू॒मग॑न्धिर् धू॒मग॑न्धि॒र् मोखा । \newline
13. धू॒मग॑न्धि॒रिति॑ धू॒म - ग॒न्धिः॒ । \newline
14. मोखोखा मा मोखा भ्राज॑न्ती॒ भ्राज॑ न्त्यु॒खा मा मोखा भ्राज॑न्ती । \newline
15. उ॒खा भ्राज॑न्ती॒ भ्राज॑न् त्यु॒खोखा भ्राज॑ न्त्य॒भ्य॑भि भ्राज॑ न्त्यु॒खोखा भ्राज॑ न्त्य॒भि । \newline
16. भ्राज॑ न्त्य॒भ्य॑भि भ्राज॑न्ती॒ भ्राज॑ न्त्य॒भि वि॑क्त विक्ता॒भि भ्राज॑न्ती॒ भ्राज॑ न्त्य॒भि वि॑क्त । \newline
17. अ॒भि वि॑क्त विक्ता॒भ्य॑भि वि॑क्त॒ जघ्रि॒र् जघ्रि॑र् विक्ता॒भ्य॑भि वि॑क्त॒ जघ्रिः॑ । \newline
18. वि॒क्त॒ जघ्रि॒र् जघ्रि॑र् विक्त विक्त॒ जघ्रिः॑ । \newline
19. जघ्रि॒रिति॒ जघ्रिः॑ । \newline
20. इ॒ष्टं ॅवी॒तं ॅवी॒त मि॒ष्ट मि॒ष्टं ॅवी॒त म॒भिगू᳚र्त म॒भिगू᳚र्तं ॅवी॒त मि॒ष्ट मि॒ष्टं ॅवी॒त म॒भिगू᳚र्तम् । \newline
21. वी॒त म॒भिगू᳚र्त म॒भिगू᳚र्तं ॅवी॒तं ॅवी॒त म॒भिगू᳚र्तं॒ ॅवष॑ट्कृतं॒ ॅवष॑ट्कृत म॒भिगू᳚र्तं ॅवी॒तं ॅवी॒त म॒भिगू᳚र्तं॒ ॅवष॑ट्कृतम् । \newline
22. अ॒भिगू᳚र्तं॒ ॅवष॑ट्कृतं॒ ॅवष॑ट्कृत म॒भिगू᳚र्त म॒भिगू᳚र्तं॒ ॅवष॑ट्कृत॒म् तम् तं ॅवष॑ट्कृत म॒भिगू᳚र्त म॒भिगू᳚र्तं॒ ॅवष॑ट्कृत॒म् तम् । \newline
23. अ॒भिगू᳚र्त॒मित्य॒भि - गू॒र्त॒म् । \newline
24. वष॑ट्कृत॒म् तम् तं ॅवष॑ट्कृतं॒ ॅवष॑ट्कृत॒म् तम् दे॒वासो॑ दे॒वास॒ स्तं ॅवष॑ट्कृतं॒ ॅवष॑ट्कृत॒म् तम् दे॒वासः॑ । \newline
25. वष॑ट्कृत॒मिति॒ वष॑ट् - कृ॒त॒म् । \newline
26. तम् दे॒वासो॑ दे॒वास॒ स्तम् तम् दे॒वासः॒ प्रति॒ प्रति॑ दे॒वास॒ स्तम् तम् दे॒वासः॒ प्रति॑ । \newline
27. दे॒वासः॒ प्रति॒ प्रति॑ दे॒वासो॑ दे॒वासः॒ प्रति॑ गृभ्णन्ति गृभ्णन्ति॒ प्रति॑ दे॒वासो॑ दे॒वासः॒ प्रति॑ गृभ्णन्ति । \newline
28. प्रति॑ गृभ्णन्ति गृभ्णन्ति॒ प्रति॒ प्रति॑ गृभ्ण॒ न्त्यश्व॒ मश्व॑म् गृभ्णन्ति॒ प्रति॒ प्रति॑ गृभ्ण॒ न्त्यश्व᳚म् । \newline
29. गृ॒भ्ण॒ न्त्यश्व॒ मश्व॑म् गृभ्णन्ति गृभ्ण॒ न्त्यश्व᳚म् । \newline
30. अश्व॒मित्यश्व᳚म् । \newline
31. यदश्वा॒या श्वा॑य॒ यद् यदश्वा॑य॒ वासो॒ वासो ऽश्वा॑य॒ यद् यदश्वा॑य॒ वासः॑ । \newline
32. अश्वा॑य॒ वासो॒ वासो ऽश्वा॒या श्वा॑य॒ वास॑ उपस्तृ॒ण न्त्यु॑पस्तृ॒णन्ति॒ वासो ऽश्वा॒या श्वा॑य॒ वास॑ उपस्तृ॒णन्ति॑ । \newline
33. वास॑ उपस्तृ॒ण न्त्यु॑पस्तृ॒णन्ति॒ वासो॒ वास॑ उपस्तृ॒ण न्त्य॑धीवा॒स म॑धीवा॒स मु॑पस्तृ॒णन्ति॒ वासो॒ वास॑ उपस्तृ॒ण न्त्य॑धीवा॒सम् । \newline
34. उ॒प॒स्तृ॒ण न्त्य॑धीवा॒स म॑धीवा॒स मु॑पस्तृ॒ण न्त्यु॑पस्तृ॒ण न्त्य॑धीवा॒सं ॅया या ऽधी॑वा॒स मु॑पस्तृ॒ण न्त्यु॑पस्तृ॒ण न्त्य॑धीवा॒सं ॅया । \newline
35. उ॒प॒स्तृ॒णन्तीत्यु॑प - स्तृ॒णन्ति॑ । \newline
36. अ॒धी॒वा॒सं ॅया या ऽधी॑वा॒स म॑धीवा॒सं ॅया हिर॑ण्यानि॒ हिर॑ण्यानि॒ या ऽधी॑वा॒स म॑धीवा॒सं ॅया हिर॑ण्यानि । \newline
37. अ॒धी॒वा॒समित्य॑धि - वा॒सम् । \newline
38. या हिर॑ण्यानि॒ हिर॑ण्यानि॒ या या हिर॑ण्या न्यस्मा अस्मै॒ हिर॑ण्यानि॒ या या हिर॑ण्या न्यस्मै । \newline
39. हिर॑ण्या न्यस्मा अस्मै॒ हिर॑ण्यानि॒ हिर॑ण्या न्यस्मै । \newline
40. अ॒स्मा॒ इत्य॑स्मै । \newline
41. स॒न्दान॒ मर्व॑न्त॒ मर्व॑न्तꣳ स॒न्दानꣳ॑ स॒न्दान॒ मर्व॑न्त॒म् पड्बी॑श॒म् पड्बी॑श॒ मर्व॑न्तꣳ स॒न्दानꣳ॑ स॒न्दान॒ मर्व॑न्त॒म् पड्बी॑शम् । \newline
42. स॒न्दान॒मिति॑ सं - दान᳚म् । \newline
43. अर्व॑न्त॒म् पड्बी॑श॒म् पड्बी॑श॒ मर्व॑न्त॒ मर्व॑न्त॒म् पड्बी॑शम् प्रि॒या प्रि॒या पड्बी॑श॒ मर्व॑न्त॒ मर्व॑न्त॒म् पड्बी॑शम् प्रि॒या । \newline
44. पड्बी॑शम् प्रि॒या प्रि॒या पड्बी॑श॒म् पड्बी॑शम् प्रि॒या दे॒वेषु॑ दे॒वेषु॑ प्रि॒या पड्बी॑श॒म् पड्बी॑शम् प्रि॒या दे॒वेषु॑ । \newline
45. प्रि॒या दे॒वेषु॑ दे॒वेषु॑ प्रि॒या प्रि॒या दे॒वेष्वा दे॒वेषु॑ प्रि॒या प्रि॒या दे॒वेष्वा । \newline
46. दे॒वेष्वा दे॒वेषु॑ दे॒वेष्वा या॑मयन्ति यामय॒न्त्या दे॒वेषु॑ दे॒वेष्वा या॑मयन्ति । \newline
47. आ या॑मयन्ति यामय॒न्त्या या॑मयन्ति । \newline
48. या॒म॒य॒न्तीति॑ यामयन्ति । \newline
49. यत् ते॑ ते॒ यद् यत् ते॑ सा॒दे सा॒दे ते॒ यद् यत् ते॑ सा॒दे । \newline
50. ते॒ सा॒दे सा॒दे ते॑ ते सा॒दे मह॑सा॒ मह॑सा सा॒दे ते॑ ते सा॒दे मह॑सा । \newline
51. सा॒दे मह॑सा॒ मह॑सा सा॒दे सा॒दे मह॑सा॒ शूकृ॑तस्य॒ शूकृ॑तस्य॒ मह॑सा सा॒दे सा॒दे मह॑सा॒ शूकृ॑तस्य । \newline
52. मह॑सा॒ शूकृ॑तस्य॒ शूकृ॑तस्य॒ मह॑सा॒ मह॑सा॒ शूकृ॑तस्य॒ पार्ष्णि॑या॒ पार्ष्णि॑या॒ शूकृ॑तस्य॒ मह॑सा॒ मह॑सा॒ शूकृ॑तस्य॒ पार्ष्णि॑या । \newline
53. शूकृ॑तस्य॒ पार्ष्णि॑या॒ पार्ष्णि॑या॒ शूकृ॑तस्य॒ शूकृ॑तस्य॒ पार्ष्णि॑या वा वा॒ पार्ष्णि॑या॒ शूकृ॑तस्य॒ शूकृ॑तस्य॒ पार्ष्णि॑या वा । \newline
54. शूकृ॑त॒स्येति॒ शू - कृ॒त॒स्य॒ । \newline
55. पार्ष्णि॑या वा वा॒ पार्ष्णि॑या॒ पार्ष्णि॑या वा॒ कश॑या॒ कश॑या वा॒ पार्ष्णि॑या॒ पार्ष्णि॑या वा॒ कश॑या । \newline
56. वा॒ कश॑या॒ कश॑या वा वा॒ कश॑या वा वा॒ कश॑या वा वा॒ कश॑या वा । \newline
57. कश॑या वा वा॒ कश॑या॒ कश॑या वा तु॒तोद॑ तु॒तोद॑ वा॒ कश॑या॒ कश॑या वा तु॒तोद॑ । \newline
\pagebreak
\markright{ TS 4.6.9.3  \hfill https://www.vedavms.in \hfill}

\section{ TS 4.6.9.3 }

\textbf{TS 4.6.9.3 } \newline
\textbf{Samhita Paata} \newline

वा तु॒तोद॑ । स्रु॒चेव॒ ता ह॒विषो॑ अद्ध्व॒रेषु॒ सर्वा॒ ता ते॒ ब्रह्म॑णा सूदयामि ॥ चतु॑स्त्रिꣳशद्-वा॒जिनो॑ दे॒वब॑न्धो॒-र्वङ्क्री॒-रश्व॑स्य॒ स्वधि॑तिः॒ समे॑ति । अच्छि॑द्रा॒ गात्रा॑ व॒युना॑ कृणोत॒ परु॑ष्परुरनु॒घुष्या॒ वि श॑स्त ॥ एक॒स्त्वष्टु॒रश्व॑स्या विश॒स्ता द्वा य॒न्तारा॑ भवत॒स्तथ॒र्तुः । या ते॒ गात्रा॑णामृतु॒था कृ॒णोमि॒ ताता॒ पिण्डा॑नां॒ प्र जु॑होम्य॒ग्नौ ॥ मा त्वा॑ तपत् - [  ] \newline

\textbf{Pada Paata} \newline

वा॒ । तु॒तोद॑ ॥ स्रु॒चा । इ॒व॒ । ता । ह॒विषः॑ । अ॒द्ध्व॒रेषु॑ । सर्वा᳚ । ता । ते॒ । ब्रह्म॑णा । सू॒द॒या॒मि॒ ॥ चतु॑स्त्रिꣳश॒दिति॒ चतुः॑ - त्रिꣳ॒॒श॒त् । वा॒जिनः॑ । दे॒वब॑न्धो॒रिति॑ दे॒व - ब॒न्धोः॒ । वङ्क्रीः᳚ । अश्व॑स्य । स्वधि॑ति॒रिति॒ स्व - धि॒तिः॒ । समिति॑ । ए॒ति॒ ॥ अच्छि॑द्रा । गात्रा᳚ । व॒युना᳚ । कृ॒णो॒त॒ । परु॑ष्परु॒रिति॒ परुः॑ - प॒रुः॒ । अ॒नु॒घुष्येत्य॑नु-घुष्य॑ । वीति॑ । श॒स्त॒ ॥ एकः॑ । त्वष्टुः॑ । अश्व॑स्य । वि॒श॒स्तेति॑ वि - श॒स्ता । द्वा । य॒न्तारा᳚ । भ॒व॒तः॒ । तथा᳚ । ऋ॒तुः ॥ या । ते॒ । गात्रा॑णाम् । ऋ॒तु॒थेत्यृ॑तु - था । कृ॒णोमि॑ । तातेति॒ ता - ता॒ । पिण्डा॑नाम् । प्रेति॑ । जु॒हो॒मि॒ । अ॒ग्नौ ॥ मा । त्वा॒ । त॒प॒त् ।  \newline


\textbf{Krama Paata} \newline

वा॒ तू॒तोद॑ । तु॒तोदेति॑ तु॒तोद॑ ॥ स्रु॒चेव॑ । इ॒व॒ ता । ता ह॒विषः॑ । ह॒विषो॑ अद्ध्व॒रेषु॑ । अ॒द्ध्व॒रेषु॒ सर्वा᳚ । सर्वा॒ ता । ता ते᳚ । ते॒ ब्रह्म॑णा । ब्रह्म॑णा सूदयामि । सू॒द॒या॒मीति॑ सूदयामि ॥ चतु॑स्त्रिꣳशद् वा॒जिनः॑ । चतु॑स्त्रिꣳश॒दिति॒ चतुः॑ - त्रिꣳ॒॒श॒त्॒ । वा॒जिनो॑ दे॒वब॑न्धोः । दे॒वब॑न्धो॒र् वङ्क्रीः᳚ । दे॒वब॑न्धो॒रिति॑ दे॒व - ब॒न्धोः॒ । वङ्क्री॒रश्व॑स्य । अश्व॑स्य॒ स्वधि॑तिः । स्वधि॑तिः॒ सम् । स्वधि॑ति॒रिति॒ स्व - धि॒तिः॒ । समे॑ति । ए॒तीत्ये॑ति ॥ अच्छि॑द्रा॒ गात्रा᳚ । गात्रा॑ व॒युना᳚ । व॒युना॑ कृणोत । कृ॒णो॒त॒ परु॑ष्परुः । परु॑ष्परुरनु॒घुष्य॑ । परु॑ष्परु॒रिति॒ परुः॑ - प॒रुः॒ । अ॒नु॒घुष्या॒ वि । अ॒नु॒घुष्येत्य॑नु - घुष्य॑ । वि श॑स्त । 
श॒स्तेति॑ शस्त ॥ एक॒स्त्वष्टुः॑ । त्वष्टु॒रश्व॑स्य । अश्व॑स्या विश॒स्ता । वि॒श॒स्ता द्वा । वि॒श॒स्तेति॑ वि - श॒स्ता । द्वा य॒न्तारा᳚ । य॒न्तारा॑ भवतः । भ॒व॒त॒स्तथा᳚ । तथ॒र्तुः । ऋ॒तुरित्यृ॒तुः ॥ या ते᳚ । ते॒ गात्रा॑णाम् । गात्रा॑णामृतु॒था । ऋ॒तु॒था कृ॒णोमि॑ । ऋ॒तु॒थेत्यृ॑तु - था । कृ॒णोमि॒ ताता᳚ । ताता॒ पिण्डा॑नाम् । तातेति॒ ता - ता॒ । पिण्डा॑ना॒म् प्र । प्र जु॑होमि । जु॒हो॒म्य॒ग्नौ । अ॒ग्नावित्य॒ग्नौ ॥ मा त्वा᳚ । त्वा॒ त॒प॒त्॒ । त॒प॒त् प्रि॒यः \newline

\textbf{Jatai Paata} \newline

1. वा॒ तु॒तोद॑ तु॒तोद॑ वा वा तु॒तोद॑ । \newline
2. तु॒तोदेति॑ तु॒तोद॑ । \newline
3. स्रु॒चेवे॑ व स्रु॒चा स्रु॒चेव॑ । \newline
4. इ॒व॒ ता तेवे॑व॒ ता । \newline
5. ता ह॒विषो॑ ह॒विष॒ स्ता ता ह॒विषः॑ । \newline
6. ह॒विषो॑ अद्ध्व॒रे ष्व॑द्ध्व॒रेषु॑ ह॒विषो॑ ह॒विषो॑ अद्ध्व॒रेषु॑ । \newline
7. अ॒द्ध्व॒रेषु॒ सर्वा॒ सर्वा᳚ ऽद्ध्व॒रे ष्व॑द्ध्व॒रेषु॒ सर्वा᳚ । \newline
8. सर्वा॒ ता ता सर्वा॒ सर्वा॒ ता । \newline
9. ता ते॑ ते॒ ता ता ते᳚ । \newline
10. ते॒ ब्रह्म॑णा॒ ब्रह्म॑णा ते ते॒ ब्रह्म॑णा । \newline
11. ब्रह्म॑णा सूदयामि सूदयामि॒ ब्रह्म॑णा॒ ब्रह्म॑णा सूदयामि । \newline
12. सू॒द॒या॒मीति॑ सूदयामि । \newline
13. चतु॑स्त्रिꣳशद् वा॒जिनो॑ वा॒जिन॒ श्चतु॑स्त्रिꣳश॒च् चतु॑स्त्रिꣳशद् वा॒जिनः॑ । \newline
14. चतु॑स्त्रिꣳश॒दिति॒ चतुः॑ - त्रिꣳ॒॒श॒त् । \newline
15. वा॒जिनो॑ दे॒वब॑न्धोर् दे॒वब॑न्धोर् वा॒जिनो॑ वा॒जिनो॑ दे॒वब॑न्धोः । \newline
16. दे॒वब॑न्धो॒र् वङ्क्री॒र् वङ्क्री᳚र् दे॒वब॑न्धोर् दे॒वब॑न्धो॒र् वङ्क्रीः᳚ । \newline
17. दे॒वब॑न्धो॒रिति॑ दे॒व - ब॒न्धोः॒ । \newline
18. वङ्क्री॒ रश्व॒स्या श्व॑स्य॒ वङ्क्री॒र् वङ्क्री॒ रश्व॑स्य । \newline
19. अश्व॑स्य॒ स्वधि॑तिः॒ स्वधि॑ति॒ रश्व॒स्या श्व॑स्य॒ स्वधि॑तिः । \newline
20. स्वधि॑तिः॒ सꣳ सꣳ स्वधि॑तिः॒ स्वधि॑तिः॒ सम् । \newline
21. स्वधि॑ति॒रिति॒ स्व - धि॒तिः॒ । \newline
22. स मे᳚त्येति॒ सꣳ स मे॑ति । \newline
23. ए॒तीत्ये॑ति । \newline
24. अच्छि॑द्रा॒ गात्रा॒ गात्रा ऽच्छि॒द्रा ऽच्छि॑द्रा॒ गात्रा᳚ । \newline
25. गात्रा॑ व॒युना॑ व॒युना॒ गात्रा॒ गात्रा॑ व॒युना᳚ । \newline
26. व॒युना॑ कृणोत कृणोत व॒युना॑ व॒युना॑ कृणोत । \newline
27. कृ॒णो॒त॒ परु॑ष्परुः॒ परु॑ष्परुः कृणोत कृणोत॒ परु॑ष्परुः । \newline
28. परु॑ष्परु रनु॒घुष्या॑ नु॒घुष्य॒ परु॑ष्परुः॒ परु॑ष्परु रनु॒घुष्य॑ । \newline
29. परु॑ष्परु॒रिति॒ परुः॑ - प॒रुः॒ । \newline
30. अ॒नु॒घुष्या॒ वि व्य॑नु॒घुष्या॑ नु॒घुष्या॒ वि । \newline
31. अ॒नु॒घुष्येत्य॑नु - घुष्य॑ । \newline
32. वि श॑स्त शस्त॒ वि वि श॑स्त । \newline
33. श॒स्तेति॑ शस्त । \newline
34. एक॒ स्त्वष्टु॒ स्त्वष्टु॒ रेक॒ एक॒ स्त्वष्टुः॑ । \newline
35. त्वष्टु॒ रश्व॒स्या श्व॑स्य॒ त्वष्टु॒ स्त्वष्टु॒ रश्व॑स्य । \newline
36. अश्व॑स्या विश॒स्ता वि॑श॒स्ता ऽश्व॒स्या श्व॑स्या विश॒स्ता । \newline
37. वि॒श॒स्ता द्वा द्वा वि॑श॒स्ता वि॑श॒स्ता द्वा । \newline
38. वि॒श॒स्तेति॑ वि - श॒स्ता । \newline
39. द्वा य॒न्तारा॑ य॒न्तारा॒ द्वा द्वा य॒न्तारा᳚ । \newline
40. य॒न्तारा॑ भवतो भवतो य॒न्तारा॑ य॒न्तारा॑ भवतः । \newline
41. भ॒व॒त॒ स्तथा॒ तथा॑ भवतो भवत॒ स्तथा᳚ । \newline
42. तथ॒ र्‌तुर्. ऋ॒तु स्तथा॒ तथ॒ र्‌तुः । \newline
43. ऋ॒तुरित्यृ॒तुः । \newline
44. या ते॑ ते॒ या या ते᳚ । \newline
45. ते॒ गात्रा॑णा॒म् गात्रा॑णाम् ते ते॒ गात्रा॑णाम् । \newline
46. गात्रा॑णा मृतु॒थ र्‌तु॒था गात्रा॑णा॒म् गात्रा॑णा मृतु॒था । \newline
47. ऋ॒तु॒था कृ॒णोमि॑ कृ॒णो म्यृ॑तु॒थ र्‌तु॒था कृ॒णोमि॑ । \newline
48. ऋ॒तु॒थेत्यृ॑तु - था । \newline
49. कृ॒णोमि॒ ताता॒ ताता॑ कृ॒णोमि॑ कृ॒णोमि॒ ताता᳚ । \newline
50. ताता॒ पिण्डा॑ना॒म् पिण्डा॑ना॒म् ताता॒ ताता॒ पिण्डा॑नाम् । \newline
51. तातेति॒ ता - ता॒ । \newline
52. पिण्डा॑ना॒म् प्र प्र पिण्डा॑ना॒म् पिण्डा॑ना॒म् प्र । \newline
53. प्र जु॑होमि जुहोमि॒ प्र प्र जु॑होमि । \newline
54. जु॒हो॒ म्य॒ग्ना व॒ग्नौ जु॑होमि जुहो म्य॒ग्नौ । \newline
55. अ॒ग्नावित्य॒ग्नौ । \newline
56. मा त्वा᳚ त्वा॒ मा मा त्वा᳚ । \newline
57. त्वा॒ त॒प॒त् त॒प॒त् त्वा॒ त्वा॒ त॒प॒त् । \newline
58. त॒प॒त् प्रि॒यः प्रि॒य स्त॑पत् तपत् प्रि॒यः । \newline

\textbf{Ghana Paata } \newline

1. वा॒ तु॒तोद॑ तु॒तोद॑ वा वा तु॒तोद॑ । \newline
2. तु॒तोदेति॑ तु॒तोद॑ । \newline
3. स्रु॒चेवे॑व स्रु॒चा स्रु॒चेव॒ ता तेव॑ स्रु॒चा स्रु॒चेव॒ ता । \newline
4. इ॒व॒ ता तेवे॑ व॒ ता ह॒विषो॑ ह॒विष॒ स्तेवे॑व॒ ता ह॒विषः॑ । \newline
5. ता ह॒विषो॑ ह॒विष॒ स्ता ता ह॒विषो॑ अद्ध्व॒रे ष्व॑द्ध्व॒रेषु॑ ह॒विष॒ स्ता ता ह॒विषो॑ अद्ध्व॒रेषु॑ । \newline
6. ह॒विषो॑ अद्ध्व॒रे ष्व॑द्ध्व॒रेषु॑ ह॒विषो॑ ह॒विषो॑ अद्ध्व॒रेषु॒ सर्वा॒ सर्वा᳚ ऽद्ध्व॒रेषु॑ ह॒विषो॑ ह॒विषो॑ अद्ध्व॒रेषु॒ सर्वा᳚ । \newline
7. अ॒द्ध्व॒रेषु॒ सर्वा॒ सर्वा᳚ ऽद्ध्व॒रे ष्व॑द्ध्व॒रेषु॒ सर्वा॒ ता ता सर्वा᳚ ऽद्ध्व॒रे ष्व॑द्ध्व॒रेषु॒ सर्वा॒ ता । \newline
8. सर्वा॒ ता ता सर्वा॒ सर्वा॒ ता ते॑ ते॒ ता सर्वा॒ सर्वा॒ ता ते᳚ । \newline
9. ता ते॑ ते॒ ता ता ते॒ ब्रह्म॑णा॒ ब्रह्म॑णा ते॒ ता ता ते॒ ब्रह्म॑णा । \newline
10. ते॒ ब्रह्म॑णा॒ ब्रह्म॑णा ते ते॒ ब्रह्म॑णा सूदयामि सूदयामि॒ ब्रह्म॑णा ते ते॒ ब्रह्म॑णा सूदयामि । \newline
11. ब्रह्म॑णा सूदयामि सूदयामि॒ ब्रह्म॑णा॒ ब्रह्म॑णा सूदयामि । \newline
12. सू॒द॒या॒मीति॑ सूदयामि । \newline
13. चतु॑स्त्रिꣳशद् वा॒जिनो॑ वा॒जिन॒ श्चतु॑स्त्रिꣳश॒च् चतु॑स्त्रिꣳशद् वा॒जिनो॑ दे॒वब॑न्धोर् दे॒वब॑न्धोर् वा॒जिन॒ श्चतु॑स्त्रिꣳश॒च् चतु॑स्त्रिꣳशद् वा॒जिनो॑ दे॒वब॑न्धोः । \newline
14. चतु॑स्त्रिꣳश॒दिति॒ चतुः॑ - त्रिꣳ॒॒श॒त् । \newline
15. वा॒जिनो॑ दे॒वब॑न्धोर् दे॒वब॑न्धोर् वा॒जिनो॑ वा॒जिनो॑ दे॒वब॑न्धो॒र् वङ्क्री॒र् वङ्क्री᳚र् दे॒वब॑न्धोर् वा॒जिनो॑ वा॒जिनो॑ दे॒वब॑न्धो॒र् वङ्क्रीः᳚ । \newline
16. दे॒वब॑न्धो॒र् वङ्क्री॒र् वङ्क्री᳚र् दे॒वब॑न्धोर् दे॒वब॑न्धो॒र् वङ्क्री॒ रश्व॒स्या श्व॑स्य॒ वङ्क्री᳚र् दे॒वब॑न्धोर् दे॒वब॑न्धो॒र् वङ्क्री॒ रश्व॑स्य । \newline
17. दे॒वब॑न्धो॒रिति॑ दे॒व - ब॒न्धोः॒ । \newline
18. वङ्क्री॒ रश्व॒स्या श्व॑स्य॒ वङ्क्री॒र् वङ्क्री॒ रश्व॑स्य॒ स्वधि॑तिः॒ स्वधि॑ति॒ रश्व॑स्य॒ वङ्क्री॒र् वङ्क्री॒ रश्व॑स्य॒ स्वधि॑तिः । \newline
19. अश्व॑स्य॒ स्वधि॑तिः॒ स्वधि॑ति॒ रश्व॒स्या श्व॑स्य॒ स्वधि॑तिः॒ सꣳ सꣳ स्वधि॑ति॒ रश्व॒स्या श्व॑स्य॒ स्वधि॑तिः॒ सम् । \newline
20. स्वधि॑तिः॒ सꣳ सꣳ स्वधि॑तिः॒ स्वधि॑तिः॒ स मे᳚त्येति॒ सꣳ स्वधि॑तिः॒ स्वधि॑तिः॒ स मे॑ति । \newline
21. स्वधि॑ति॒रिति॒ स्व - धि॒तिः॒ । \newline
22. स मे᳚त्येति॒ सꣳ स मे॑ति । \newline
23. ए॒तीत्ये॑ति । \newline
24. अच्छि॑द्रा॒ गात्रा॒ गात्रा ऽच्छि॒द्रा ऽच्छि॑द्रा॒ गात्रा॑ व॒युना॑ व॒युना॒ गात्रा ऽच्छि॒द्रा ऽच्छि॑द्रा॒ गात्रा॑ व॒युना᳚ । \newline
25. गात्रा॑ व॒युना॑ व॒युना॒ गात्रा॒ गात्रा॑ व॒युना॑ कृणोत कृणोत व॒युना॒ गात्रा॒ गात्रा॑ व॒युना॑ कृणोत । \newline
26. व॒युना॑ कृणोत कृणोत व॒युना॑ व॒युना॑ कृणोत॒ परु॑ष्परुः॒ परु॑ष्परुः कृणोत व॒युना॑ व॒युना॑ कृणोत॒ परु॑ष्परुः । \newline
27. कृ॒णो॒त॒ परु॑ष्परुः॒ परु॑ष्परुः कृणोत कृणोत॒ परु॑ष्परु रनु॒घुष्या॑ नु॒घुष्य॒ परु॑ष्परुः कृणोत कृणोत॒ परु॑ष्परु रनु॒घुष्य॑ । \newline
28. परु॑ष्परु रनु॒घुष्या॑ नु॒घुष्य॒ परु॑ष्परुः॒ परु॑ष्परु रनु॒घुष्या॒ वि व्य॑नु॒घुष्य॒ परु॑ष्परुः॒ परु॑ष्परु रनु॒घुष्या॒ वि । \newline
29. परु॑ष्परु॒रिति॒ परुः॑ - प॒रुः॒ । \newline
30. अ॒नु॒घुष्या॒ वि व्य॑नु॒घुष्या॑ नु॒घुष्या॒ वि श॑स्त शस्त॒ व्य॑नु॒घुष्या॑ नु॒घुष्या॒ वि श॑स्त । \newline
31. अ॒नु॒घुष्येत्य॑नु - घुष्य॑ । \newline
32. वि श॑स्त शस्त॒ वि वि श॑स्त । \newline
33. श॒स्तेति॑ शस्त । \newline
34. एक॒ स्त्वष्टु॒ स्त्वष्टु॒ रेक॒ एक॒ स्त्वष्टु॒ रश्व॒स्या श्व॑स्य॒ त्वष्टु॒ रेक॒ एक॒ स्त्वष्टु॒ रश्व॑स्य । \newline
35. त्वष्टु॒ रश्व॒स्या श्व॑स्य॒ त्वष्टु॒ स्त्वष्टु॒ रश्व॑स्या विश॒स्ता वि॑श॒स्ता ऽश्व॑स्य॒ त्वष्टु॒ स्त्वष्टु॒ रश्व॑स्या विश॒स्ता । \newline
36. अश्व॑स्या विश॒स्ता वि॑श॒स्ता ऽश्व॒स्या श्व॑स्या विश॒स्ता द्वा द्वा वि॑श॒स्ता ऽश्व॒स्या श्व॑स्या विश॒स्ता द्वा । \newline
37. वि॒श॒स्ता द्वा द्वा वि॑श॒स्ता वि॑श॒स्ता द्वा य॒न्तारा॑ य॒न्तारा॒ द्वा वि॑श॒स्ता वि॑श॒स्ता द्वा य॒न्तारा᳚ । \newline
38. वि॒श॒स्तेति॑ वि - श॒स्ता । \newline
39. द्वा य॒न्तारा॑ य॒न्तारा॒ द्वा द्वा य॒न्तारा॑ भवतो भवतो य॒न्तारा॒ द्वा द्वा य॒न्तारा॑ भवतः । \newline
40. य॒न्तारा॑ भवतो भवतो य॒न्तारा॑ य॒न्तारा॑ भवत॒ स्तथा॒ तथा॑ भवतो य॒न्तारा॑ य॒न्तारा॑ भवत॒ स्तथा᳚ । \newline
41. भ॒व॒त॒ स्तथा॒ तथा॑ भवतो भवत॒ स्तथ॒ र्‌तुर्. ऋ॒तु स्तथा॑ भवतो भवत॒ स्तथ॒ र्‌तुः । \newline
42. तथ॒ र्‌तुर्. ऋ॒तु स्तथा॒ तथ॒ र्‌तुः । \newline
43. ऋ॒तुरित्यृ॒तुः । \newline
44. या ते॑ ते॒ या या ते॒ गात्रा॑णा॒म् गात्रा॑णाम् ते॒ या या ते॒ गात्रा॑णाम् । \newline
45. ते॒ गात्रा॑णा॒म् गात्रा॑णाम् ते ते॒ गात्रा॑णा मृतु॒थ र्‌तु॒था गात्रा॑णाम् ते ते॒ गात्रा॑णा मृतु॒था । \newline
46. गात्रा॑णा मृतु॒थ र्‌तु॒था गात्रा॑णा॒म् गात्रा॑णा मृतु॒था कृ॒णोमि॑ कृ॒णो म्यृ॑तु॒था गात्रा॑णा॒म् गात्रा॑णा मृतु॒था कृ॒णोमि॑ । \newline
47. ऋ॒तु॒था कृ॒णोमि॑ कृ॒णो म्यृ॑तु॒थ र्‌तु॒था कृ॒णोमि॒ ताता॒ ताता॑ कृ॒णो म्यृ॑तु॒थ र्‌तु॒था कृ॒णोमि॒ ताता᳚ । \newline
48. ऋ॒तु॒थेत्यृ॑तु - था । \newline
49. कृ॒णोमि॒ ताता॒ ताता॑ कृ॒णोमि॑ कृ॒णोमि॒ ताता॒ पिण्डा॑ना॒म् पिण्डा॑ना॒म् ताता॑ कृ॒णोमि॑ कृ॒णोमि॒ ताता॒ पिण्डा॑नाम् । \newline
50. ताता॒ पिण्डा॑ना॒म् पिण्डा॑ना॒म् ताता॒ ताता॒ पिण्डा॑ना॒म् प्र प्र पिण्डा॑ना॒म् ताता॒ ताता॒ पिण्डा॑ना॒म् प्र । \newline
51. तातेति॒ ता - ता॒ । \newline
52. पिण्डा॑ना॒म् प्र प्र पिण्डा॑ना॒म् पिण्डा॑ना॒म् प्र जु॑होमि जुहोमि॒ प्र पिण्डा॑ना॒म् पिण्डा॑ना॒म् प्र जु॑होमि । \newline
53. प्र जु॑होमि जुहोमि॒ प्र प्र जु॑होम्य॒ग्ना व॒ग्नौ जु॑होमि॒ प्र प्र जु॑होम्य॒ग्नौ । \newline
54. जु॒हो॒ म्य॒ग्ना व॒ग्नौ जु॑होमि जुहो म्य॒ग्नौ । \newline
55. अ॒ग्नावित्य॒ग्नौ । \newline
56. मा त्वा᳚ त्वा॒ मा मा त्वा॑ तपत् तपत् त्वा॒ मा मा त्वा॑ तपत् । \newline
57. त्वा॒ त॒प॒त् त॒प॒त् त्वा॒ त्वा॒ त॒प॒त् प्रि॒यः प्रि॒य स्त॑पत् त्वा त्वा तपत् प्रि॒यः । \newline
58. त॒प॒त् प्रि॒यः प्रि॒य स्त॑पत् तपत् प्रि॒य आ॒त्मा ऽऽत्मा प्रि॒य स्त॑पत् तपत् प्रि॒य आ॒त्मा । \newline
\pagebreak
\markright{ TS 4.6.9.4  \hfill https://www.vedavms.in \hfill}

\section{ TS 4.6.9.4 }

\textbf{TS 4.6.9.4 } \newline
\textbf{Samhita Paata} \newline

प्रि॒य आ॒त्माऽपि॒यन्तं॒ मा स्वधि॑तिस्त॒नुव॒ आ ति॑ष्ठिपत् ते । मा ते॑ गृ॒द्ध्नु-र॑विश॒स्ताऽति॒हाय॑ छि॒द्रा गात्रा᳚ण्य॒सिना॒ मिथू॑ कः ॥ न वा उ॑वे॒तन्म्रि॑यसे॒ न रि॑ष्यसि दे॒वाꣳ इदे॑षि प॒थिभिः॑ सु॒गेभिः॑ । हरी॑ ते॒ युञ्जा॒ पृष॑ती अभूता॒मुपा᳚स्थाद्-वा॒जी धु॒रि रास॑भस्य ॥ सु॒गव्यं॑ नो वा॒जी स्वश्वि॑यं पुꣳ॒॒सः पु॒त्राꣳ उ॒त वि॑श्वा॒पुषꣳ॑ र॒यिं ( ) । अ॒ना॒गा॒स्त्वं नो॒ अदि॑तिः कृणोतु क्ष॒त्रं नो॒ अश्वो॑ वनताꣳ ह॒विष्मान्॑ ॥ \newline

\textbf{Pada Paata} \newline

प्रि॒यः । आ॒त्मा । अ॒पि॒यन्त॒मित्य॑पि - यन्त᳚म् । मा । स्वधि॑ति॒रिति॒ स्व - धि॒तिः॒ । त॒नुवः॑ । एति॑ । ति॒ष्ठि॒प॒त् । ते॒ ॥ मा । ते॒ । गृ॒द्ध्नुः । अ॒वि॒श॒स्तेत्य॑वि - श॒स्ता । अ॒ति॒हायेत्य॑ति-हाय॑ । छि॒द्रा । गात्रा॑णि । अ॒सिना᳚ । मिथु॑ । कः॒ ॥ न । वै । उ॒ । ए॒तत् । म्रि॒य॒से॒ । न । रि॒ष्य॒सि॒ । दे॒वान् । इत् । ए॒षि॒ । प॒थिभि॒रिति॑ प॒थि - भिः॒ । सु॒गेभि॒रिति॑ सु - गेभिः॑ ॥ हरी॒ इति॑ । ते॒ । युञ्जा᳚ । पृष॑ती॒ इति॑ । अ॒भू॒ता॒म् । उपेति॑ । अ॒स्था॒त् । वा॒जी । धु॒रि । रास॑भस्य ॥ सु॒गव्य॒मिति॑ सु - गव्य᳚म् । नः॒ । वा॒जी । स्वश्वि॑य॒मिति॑ सु - अश्वि॑यम् । पुꣳ॒॒सः । पु॒त्रान् । उ॒त । वि॒श्वा॒पुष॒मिति॑ विश्व -पुष᳚म् । र॒यिम् ( ) ॥ अ॒ना॒गा॒स्त्वमित्य॑नागाः - त्वम् । नः॒ । अदि॑तिः । कृ॒णो॒तु॒ । क्ष॒त्रम् । नः॒ । अश्वः॑ । व॒न॒ता॒म् । ह॒विष्मान्॑ ॥  \newline


\textbf{Krama Paata} \newline

प्रि॒य आ॒त्मा । आ॒त्माऽपि॒यन्त᳚म् । अ॒पि॒यन्त॒म् मा । अ॒पि॒यन्त॒मित्य॑पि - यन्त᳚म् । मा स्वधि॑तिः । स्वधि॑ति स्त॒नुवः॑ । स्वधि॑ति॒रिति॒ स्व - धि॒तिः॒ । त॒नुव॒ आ । आ ति॑ष्ठिपत् । ति॒ष्ठि॒प॒त् ते॒ । त॒ इति॑ ते ॥ मा ते᳚ । ते॒ गृ॒द्ध्नुः । गृ॒द्ध्नुर॑विश॒स्ता । अ॒वि॒श॒स्ताऽति॒हाय॑ । अ॒वि॒श॒स्तेत्य॑वि - श॒स्ता । अ॒ति॒हाय॑ छि॒द्रा । अ॒ति॒हायेत्य॑ति - हाय॑ । छि॒द्रा गात्रा॑णि । गात्रा᳚ण्य॒सिना᳚ । अ॒सिना॒ मिथु॑ । मिथू॑ कः । क॒ इति॑ कः ॥ न वै । वा उ॑ । उ॒वे॒तत् । ए॒तन् म्रि॑यसे । म्रि॒य॒से॒ न । न रि॑ष्यसि । रि॒ष्य॒सि॒ दे॒वान् । दे॒वाꣳ इत् । इदे॑षि । ए॒षि॒ प॒थिभिः॑ । प॒थिभिः॑ सु॒गेभिः॑ । प॒थिभि॒रिति॑ प॒थि - भिः॒ । सु॒गेभि॒रिति॑ सु - गेभिः॑ ॥ हरी॑ ते । हरी॒ इति॒ हरी᳚ । ते॒ युञ्जा᳚ । युञ्जा॒ पृष॑ती । पृष॑ती अभूताम् । पृष॑ती॒ इति॒ पृष॑ती । अ॒भू॒ता॒मुप॑ । उपा᳚स्थात् । अ॒स्था॒द् वा॒जी । वा॒जी धु॒रि । धु॒रि रास॑भस्य । रास॑भ॒स्येति॒ रास॑भस्य ॥ सु॒गव्य॑म् नः । सु॒गव्य॒मिति॑ सु - गव्य᳚म् । नो॒ वा॒जी । वा॒जी स्वश्वि॑यम् । स्वश्वि॑यम् पुꣳ॒॒सः । स्वश्वि॑य॒मिति॑ सु - अश्वि॑यम् । पुꣳ॒॒सः पु॒त्रान् । पु॒त्राꣳ उ॒त । उ॒त वि॑श्वा॒पुष᳚म् । वि॒श्वा॒पुषꣳ॑ र॒यिम् ( ) । वि॒श्वा॒पुष॒मिति॑ विश्व - पुष᳚म् । र॒यिमिति॑ र॒यिम् ॥ अ॒ना॒गा॒स्त्वम् नः॑ । अ॒ना॒गा॒स्त्वमित्य॑नागाः - त्वम् । नो॒ अदि॑तिः । अदि॑तिः कृणोतु । कृ॒णो॒तु॒ क्ष॒त्रम् । क्ष॒त्रम् नः॑ । नो॒ अश्वः॑ । अश्वो॑ वनताम् । व॒न॒ताꣳ॒॒ ह॒विष्मान्॑ । ह॒विष्मा॒निति॑ ह॒विष्मान्॑ । \newline

\textbf{Jatai Paata} \newline

1. प्रि॒य आ॒त्मा ऽऽत्मा प्रि॒यः प्रि॒य आ॒त्मा । \newline
2. आ॒त्मा ऽपि॒यन्त॑ मपि॒यन्त॑ मा॒त्मा ऽऽत्मा ऽपि॒यन्त᳚म् । \newline
3. अ॒पि॒यन्त॒म् मा मा ऽपि॒यन्त॑ मपि॒यन्त॒म् मा । \newline
4. अ॒पि॒यन्त॒मित्य॑पि - यन्त᳚म् । \newline
5. मा स्वधि॑तिः॒ स्वधि॑ति॒र् मा मा स्वधि॑तिः । \newline
6. स्वधि॑ति स्त॒नुव॑ स्त॒नुवः॒ स्वधि॑तिः॒ स्वधि॑ति स्त॒नुवः॑ । \newline
7. स्वधि॑ति॒रिति॒ स्व - धि॒तिः॒ । \newline
8. त॒नुव॒ आ त॒नुव॑ स्त॒नुव॒ आ । \newline
9. आ ति॑ष्ठिपत् तिष्ठिप॒दा ति॑ष्ठिपत् । \newline
10. ति॒ष्ठि॒प॒त् ते॒ ते॒ ति॒ष्ठि॒प॒त् ति॒ष्ठि॒प॒त् ते॒ । \newline
11. त॒ इति॑ ते । \newline
12. मा ते॑ ते॒ मा मा ते᳚ । \newline
13. ते॒ गृ॒द्ध्नुर् गृ॒द्ध्नु स्ते॑ ते गृ॒द्ध्नुः । \newline
14. गृ॒द्ध्नु र॑विश॒स्ता ऽवि॑श॒स्ता गृ॒द्ध्नुर् गृ॒द्ध्नु र॑विश॒स्ता । \newline
15. अ॒वि॒श॒स्ता ऽति॒हाया॑ ति॒हाया॑ विश॒स्ता ऽवि॑श॒स्ता ऽति॒हाय॑ । \newline
16. अ॒वि॒श॒स्तेत्य॑वि - श॒स्ता । \newline
17. अ॒ति॒हाय॑ छि॒द्रा छि॒द्रा ऽति॒हाया॑ ति॒हाय॑ छि॒द्रा । \newline
18. अ॒ति॒हायेत्य॑ति - हाय॑ । \newline
19. छि॒द्रा गात्रा॑णि॒ गात्रा॑णि छि॒द्रा छि॒द्रा गात्रा॑णि । \newline
20. गात्रा᳚ ण्य॒सिना॒ ऽसिना॒ गात्रा॑णि॒ गात्रा᳚ ण्य॒सिना᳚ । \newline
21. अ॒सिना॒ मिथु॒ मिथ्व॒सिना॒ ऽसिना॒ मिथु॑ । \newline
22. मिथू॑ कः क॒र् मिथु॒ मिथू॑ कः । \newline
23. क॒रिति॑ कः । \newline
24. न वै वै न न वै । \newline
25. वा उ॑ वु॒ वै वा उ॑ । \newline
26. उ॒ वे॒त दे॒तदु॑ वु वे॒तत् । \newline
27. ए॒तन् म्रि॑यसे म्रियस ए॒त दे॒तन् म्रि॑यसे । \newline
28. म्रि॒य॒से॒ न न म्रि॑यसे म्रियसे॒ न । \newline
29. न रि॑ष्यसि रिष्यसि॒ न न रि॑ष्यसि । \newline
30. रि॒ष्य॒सि॒ दे॒वान् दे॒वान् रि॑ष्यसि रिष्यसि दे॒वान् । \newline
31. दे॒वाꣳ इदिद् दे॒वान् दे॒वाꣳ इत् । \newline
32. इदे᳚ ष्ये॒षी दिदे॑षि । \newline
33. ए॒षि॒ प॒थिभिः॑ प॒थिभि॑ रेष्येषि प॒थिभिः॑ । \newline
34. प॒थिभिः॑ सु॒गेभिः॑ सु॒गेभिः॑ प॒थिभिः॑ प॒थिभिः॑ सु॒गेभिः॑ । \newline
35. प॒थिभि॒रिति॑ प॒थि - भिः॒ । \newline
36. सु॒गेभि॒रिति॑ सु - गेभिः॑ । \newline
37. हरी॑ ते ते॒ हरी॒ हरी॑ ते । \newline
38. हरी॒ इति॒ हरी᳚ । \newline
39. ते॒ युञ्जा॒ युञ्जा॑ ते ते॒ युञ्जा᳚ । \newline
40. युञ्जा॒ पृष॑ती॒ पृष॑ती॒ युञ्जा॒ युञ्जा॒ पृष॑ती । \newline
41. पृष॑ती अभूता मभूता॒म् पृष॑ती॒ पृष॑ती अभूताम् । \newline
42. पृष॑ती॒ इति॒ पृष॑ती । \newline
43. अ॒भू॒ता॒ मुपोपा॑भूता मभूता॒ मुप॑ । \newline
44. उपा᳚स्था दस्था॒ दुपोपा᳚ स्थात् । \newline
45. अ॒स्था॒द् वा॒जी वा॒ज्य॑स्था दस्थाद् वा॒जी । \newline
46. वा॒जी धु॒रि धु॒रि वा॒जी वा॒जी धु॒रि । \newline
47. धु॒रि रास॑भस्य॒ रास॑भस्य धु॒रि धु॒रि रास॑भस्य । \newline
48. रास॑भ॒स्येति॒ रास॑भस्य । \newline
49. सु॒गव्य॑म् नो नः सु॒गव्यꣳ॑ सु॒गव्य॑म् नः । \newline
50. सु॒गव्य॒मिति॑ सु - गव्य᳚म् । \newline
51. नो॒ वा॒जी वा॒जी नो॑ नो वा॒जी । \newline
52. वा॒जी स्वश्वि॑यꣳ॒॒ स्वश्वि॑यं ॅवा॒जी वा॒जी स्वश्वि॑यम् । \newline
53. स्वश्वि॑यम् पुꣳ॒॒सः पुꣳ॒॒सः स्वश्वि॑यꣳ॒॒ स्वश्वि॑यम् पुꣳ॒॒सः । \newline
54. स्वश्वि॑य॒मिति॑ सु - अश्वि॑यम् । \newline
55. पुꣳ॒॒सः पु॒त्रान् पु॒त्रान् पुꣳ॒॒सः पुꣳ॒॒सः पु॒त्रान् । \newline
56. पु॒त्राꣳ उ॒तोत पु॒त्रान् पु॒त्राꣳ उ॒त । \newline
57. उ॒त वि॑श्वा॒पुषं॑ ॅविश्वा॒पुष॑ मु॒तोत वि॑श्वा॒पुष᳚म् । \newline
58. वि॒श्वा॒पुषꣳ॑ र॒यिꣳ र॒यिं ॅवि॑श्वा॒पुषं॑ ॅविश्वा॒पुषꣳ॑ र॒यिम् । \newline
59. वि॒श्वा॒पुष॒मिति॑ विश्व - पुष᳚म् । \newline
60. र॒यिमिति॑ र॒यिम् । \newline
61. अ॒ना॒गा॒स्त्वम् नो॑ नो ऽनागा॒स्त्व म॑नागा॒स्त्वम् नः॑ । \newline
62. अ॒ना॒गा॒स्त्वमित्य॑नागाः - त्वम् । \newline
63. नो॒ अदि॑ति॒ रदि॑तिर् नो नो॒ अदि॑तिः । \newline
64. अदि॑तिः कृणोतु कृणो॒ त्वदि॑ति॒ रदि॑तिः कृणोतु । \newline
65. कृ॒णो॒तु॒ क्ष॒त्रम् क्ष॒त्रम् कृ॑णोतु कृणोतु क्ष॒त्रम् । \newline
66. क्ष॒त्रम् नो॑ नः क्ष॒त्रम् क्ष॒त्रम् नः॑ । \newline
67. नो॒ अश्वो॒ अश्वो॑ नो नो॒ अश्वः॑ । \newline
68. अश्वो॑ वनतां ॅवनता॒ मश्वो॒ अश्वो॑ वनताम् । \newline
69. व॒न॒ताꣳ॒॒ ह॒विष्मान्॑. ह॒विष्मान्॑. वनतां ॅवनताꣳ ह॒विष्मान्॑ । \newline
70. ह॒विष्मा॒निति॑ ह॒विष्मान्॑ । \newline

\textbf{Ghana Paata } \newline

1. प्रि॒य आ॒त्मा ऽऽत्मा प्रि॒यः प्रि॒य आ॒त्मा ऽपि॒यन्त॑ मपि॒यन्त॑ मा॒त्मा प्रि॒यः प्रि॒य आ॒त्मा ऽपि॒यन्त᳚म् । \newline
2. आ॒त्मा ऽपि॒यन्त॑ मपि॒यन्त॑ मा॒त्मा ऽऽत्मा ऽपि॒यन्त॒म् मा मा ऽपि॒यन्त॑ मा॒त्मा ऽऽत्मा ऽपि॒यन्त॒म् मा । \newline
3. अ॒पि॒यन्त॒म् मा मा ऽपि॒यन्त॑ मपि॒यन्त॒म् मा स्वधि॑तिः॒ स्वधि॑ति॒र् मा ऽपि॒यन्त॑ मपि॒यन्त॒म् मा स्वधि॑तिः । \newline
4. अ॒पि॒यन्त॒मित्य॑पि - यन्त᳚म् । \newline
5. मा स्वधि॑तिः॒ स्वधि॑ति॒र् मा मा स्वधि॑ति स्त॒नुव॑ स्त॒नुवः॒ स्वधि॑ति॒र् मा मा स्वधि॑ति स्त॒नुवः॑ । \newline
6. स्वधि॑ति स्त॒नुव॑ स्त॒नुवः॒ स्वधि॑तिः॒ स्वधि॑ति स्त॒नुव॒ आ त॒नुवः॒ स्वधि॑तिः॒ स्वधि॑ति स्त॒नुव॒ आ । \newline
7. स्वधि॑ति॒रिति॒ स्व - धि॒तिः॒ । \newline
8. त॒नुव॒ आ त॒नुव॑ स्त॒नुव॒ आ ति॑ष्ठिपत् तिष्ठिप॒दा त॒नुव॑ स्त॒नुव॒ आ ति॑ष्ठिपत् । \newline
9. आ ति॑ष्ठिपत् तिष्ठिप॒दा ति॑ष्ठिपत् ते ते तिष्ठिप॒दा ति॑ष्ठिपत् ते । \newline
10. ति॒ष्ठि॒प॒त् ते॒ ते॒ ति॒ष्ठि॒प॒त् ति॒ष्ठि॒प॒त् ते॒ । \newline
11. त॒ इति॑ ते । \newline
12. मा ते॑ ते॒ मा मा ते॑ गृ॒द्ध्नुर् गृ॒द्ध्नु स्ते॒ मा मा ते॑ गृ॒द्ध्नुः । \newline
13. ते॒ गृ॒द्ध्नुर् गृ॒द्ध्नु स्ते॑ ते गृ॒द्ध्नु र॑विश॒स्ता ऽवि॑श॒स्ता गृ॒द्ध्नु स्ते॑ ते गृ॒द्ध्नु र॑विश॒स्ता । \newline
14. गृ॒द्ध्नु र॑विश॒स्ता ऽवि॑श॒स्ता गृ॒द्ध्नुर् गृ॒द्ध्नु र॑विश॒स्ता ऽति॒हाया॑ ति॒हाया॑ विश॒स्ता गृ॒द्ध्नुर् गृ॒द्ध्नु र॑विश॒स्ता ऽति॒हाय॑ । \newline
15. अ॒वि॒श॒स्ता ऽति॒हाया॑ ति॒हाया॑ विश॒स्ता ऽवि॑श॒स्ता ऽति॒हाय॑ छि॒द्रा छि॒द्रा ऽति॒हाया॑ विश॒स्ता ऽवि॑श॒स्ता ऽति॒हाय॑ छि॒द्रा । \newline
16. अ॒वि॒श॒स्तेत्य॑वि - श॒स्ता । \newline
17. अ॒ति॒हाय॑ छि॒द्रा छि॒द्रा ऽति॒हाया॑ ति॒हाय॑ छि॒द्रा गात्रा॑णि॒ गात्रा॑णि छि॒द्रा ऽति॒हाया॑ ति॒हाय॑ छि॒द्रा गात्रा॑णि । \newline
18. अ॒ति॒हायेत्य॑ति - हाय॑ । \newline
19. छि॒द्रा गात्रा॑णि॒ गात्रा॑णि छि॒द्रा छि॒द्रा गात्रा᳚ ण्य॒सिना॒ ऽसिना॒ गात्रा॑णि छि॒द्रा छि॒द्रा गात्रा᳚ ण्य॒सिना᳚ । \newline
20. गात्रा᳚ ण्य॒सिना॒ ऽसिना॒ गात्रा॑णि॒ गात्रा᳚ ण्य॒सिना॒ मिथु॒ मिथ्व॒सिना॒ गात्रा॑णि॒ गात्रा᳚ ण्य॒सिना॒ मिथु॑ । \newline
21. अ॒सिना॒ मिथु॒ मिथ्व॒सिना॒ ऽसिना॒ मिथू॑ कः क॒र् मिथ्व॒सिना॒ ऽसिना॒ मिथू॑ कः । \newline
22. मिथू॑ कः क॒र् मिथु॒ मिथू॑ कः । \newline
23. क॒रिति॑ कः । \newline
24. न वै वै न न वा उ॑ वु॒ वै न न वा उ॑ । \newline
25. वा उ॑ वु॒ वै वा उ॑ वे॒त दे॒त दु॒ वै वा उ॑ वे॒तत् । \newline
26. उ॒ वे॒त दे॒तदु॑ वु वे॒तन् म्रि॑यसे म्रियस ए॒तदु॑ वु वे॒तन् म्रि॑यसे । \newline
27. ए॒तन् म्रि॑यसे म्रियस ए॒त दे॒तन् म्रि॑यसे॒ न न म्रि॑यस ए॒त दे॒तन् म्रि॑यसे॒ न । \newline
28. म्रि॒य॒से॒ न न म्रि॑यसे म्रियसे॒ न रि॑ष्यसि रिष्यसि॒ न म्रि॑यसे म्रियसे॒ न रि॑ष्यसि । \newline
29. न रि॑ष्यसि रिष्यसि॒ न न रि॑ष्यसि दे॒वान् दे॒वान् रि॑ष्यसि॒ न न रि॑ष्यसि दे॒वान् । \newline
30. रि॒ष्य॒सि॒ दे॒वान् दे॒वान् रि॑ष्यसि रिष्यसि दे॒वाꣳ इदिद् दे॒वान् रि॑ष्यसि रिष्यसि दे॒वाꣳ इत् । \newline
31. दे॒वाꣳ इदिद् दे॒वान् दे॒वाꣳ इदे᳚ष्ये॒षीद् दे॒वान् दे॒वाꣳ इदे॑षि । \newline
32. इदे᳚ष्ये॒षी दिदे॑षि प॒थिभिः॑ प॒थिभि॑ रे॒षीदि दे॑षि प॒थिभिः॑ । \newline
33. ए॒षि॒ प॒थिभिः॑ प॒थिभि॑रे ष्येषि प॒थिभिः॑ सु॒गेभिः॑ सु॒गेभिः॑ प॒थिभि॑ रेष्येषि प॒थिभिः॑ सु॒गेभिः॑ । \newline
34. प॒थिभिः॑ सु॒गेभिः॑ सु॒गेभिः॑ प॒थिभिः॑ प॒थिभिः॑ सु॒गेभिः॑ । \newline
35. प॒थिभि॒रिति॑ प॒थि - भिः॒ । \newline
36. सु॒गेभि॒रिति॑ सु - गेभिः॑ । \newline
37. हरी॑ ते ते॒ हरी॒ हरी॑ ते॒ युञ्जा॒ युञ्जा॑ ते॒ हरी॒ हरी॑ ते॒ युञ्जा᳚ । \newline
38. हरी॒ इति॒ हरी᳚ । \newline
39. ते॒ युञ्जा॒ युञ्जा॑ ते ते॒ युञ्जा॒ पृष॑ती॒ पृष॑ती॒ युञ्जा॑ ते ते॒ युञ्जा॒ पृष॑ती । \newline
40. युञ्जा॒ पृष॑ती॒ पृष॑ती॒ युञ्जा॒ युञ्जा॒ पृष॑ती अभूता मभूता॒म् पृष॑ती॒ युञ्जा॒ युञ्जा॒ पृष॑ती अभूताम् । \newline
41. पृष॑ती अभूता मभूता॒म् पृष॑ती॒ पृष॑ती अभूता॒ मुपोपा॑ भूता॒म् पृष॑ती॒ पृष॑ती अभूता॒ मुप॑ । \newline
42. पृष॑ती॒ इति॒ पृष॑ती । \newline
43. अ॒भू॒ता॒ मुपोपा॑ भूता मभूता॒ मुपा᳚स्था दस्था॒ दुपा॑भूता मभूता॒ मुपा᳚स्थात् । \newline
44. उपा᳚ स्था दस्था॒ दुपोपा᳚ स्थाद् वा॒जी वा॒ज्य॑ स्था॒ दुपोपा᳚ स्थाद् वा॒जी । \newline
45. अ॒स्था॒द् वा॒जी वा॒ज्य॑स्था दस्थाद् वा॒जी धु॒रि धु॒रि वा॒ज्य॑ स्था दस्थाद् वा॒जी धु॒रि । \newline
46. वा॒जी धु॒रि धु॒रि वा॒जी वा॒जी धु॒रि रास॑भस्य॒ रास॑भस्य धु॒रि वा॒जी वा॒जी धु॒रि रास॑भस्य । \newline
47. धु॒रि रास॑भस्य॒ रास॑भस्य धु॒रि धु॒रि रास॑भस्य । \newline
48. रास॑भ॒स्येति॒ रास॑भस्य । \newline
49. सु॒गव्य॑म् नो नः सु॒गव्यꣳ॑ सु॒गव्य॑म् नो वा॒जी वा॒जी नः॑ सु॒गव्यꣳ॑ सु॒गव्य॑म् नो वा॒जी । \newline
50. सु॒गव्य॒मिति॑ सु - गव्य᳚म् । \newline
51. नो॒ वा॒जी वा॒जी नो॑ नो वा॒जी स्वश्वि॑यꣳ॒॒ स्वश्वि॑यं ॅवा॒जी नो॑ नो वा॒जी स्वश्वि॑यम् । \newline
52. वा॒जी स्वश्वि॑यꣳ॒॒ स्वश्वि॑यं ॅवा॒जी वा॒जी स्वश्वि॑यम् पुꣳ॒॒सः पुꣳ॒॒सः स्वश्वि॑यं ॅवा॒जी वा॒जी स्वश्वि॑यम् पुꣳ॒॒सः । \newline
53. स्वश्वि॑यम् पुꣳ॒॒सः पुꣳ॒॒सः स्वश्वि॑यꣳ॒॒ स्वश्वि॑यम् पुꣳ॒॒सः पु॒त्रान् पु॒त्रान् पुꣳ॒॒सः स्वश्वि॑यꣳ॒॒ स्वश्वि॑यम् पुꣳ॒॒सः पु॒त्रान् । \newline
54. स्वश्वि॑य॒मिति॑ सु - अश्वि॑यम् । \newline
55. पुꣳ॒॒सः पु॒त्रान् पु॒त्रान् पुꣳ॒॒सः पुꣳ॒॒सः पु॒त्राꣳ उ॒तोत पु॒त्रान् पुꣳ॒॒सः पुꣳ॒॒सः पु॒त्राꣳ उ॒त । \newline
56. पु॒त्राꣳ उ॒तोत पु॒त्रान् पु॒त्राꣳ उ॒त वि॑श्वा॒पुषं॑ ॅविश्वा॒पुष॑ मु॒त पु॒त्रान् पु॒त्राꣳ उ॒त वि॑श्वा॒पुष᳚म् । \newline
57. उ॒त वि॑श्वा॒पुषं॑ ॅविश्वा॒पुष॑ मु॒तोत वि॑श्वा॒पुषꣳ॑ र॒यिꣳ र॒यिं ॅवि॑श्वा॒पुष॑ मु॒तोत वि॑श्वा॒पुषꣳ॑ र॒यिम् । \newline
58. वि॒श्वा॒पुषꣳ॑ र॒यिꣳ र॒यिं ॅवि॑श्वा॒पुषं॑ ॅविश्वा॒पुषꣳ॑ र॒यिम् । \newline
59. वि॒श्वा॒पुष॒मिति॑ विश्व - पुष᳚म् । \newline
60. र॒यिमिति॑ र॒यिम् । \newline
61. अ॒ना॒गा॒स्त्वम् नो॑ नो ऽनागा॒स्त्व म॑नागा॒स्त्वम् नो॒ अदि॑ति॒ रदि॑तिर् नो ऽनागा॒स्त्व म॑नागा॒स्त्वम् नो॒ अदि॑तिः । \newline
62. अ॒ना॒गा॒स्त्वमित्य॑नागाः - त्वम् । \newline
63. नो॒ अदि॑ति॒ रदि॑तिर् नो नो॒ अदि॑तिः कृणोतु कृणो॒ त्वदि॑तिर् नो नो॒ अदि॑तिः कृणोतु । \newline
64. अदि॑तिः कृणोतु कृणो॒ त्वदि॑ति॒ रदि॑तिः कृणोतु क्ष॒त्रम् क्ष॒त्रम् कृ॑णो॒ त्वदि॑ति॒ रदि॑तिः कृणोतु क्ष॒त्रम् । \newline
65. कृ॒णो॒तु॒ क्ष॒त्रम् क्ष॒त्रम् कृ॑णोतु कृणोतु क्ष॒त्रम् नो॑ नः क्ष॒त्रम् कृ॑णोतु कृणोतु क्ष॒त्रम् नः॑ । \newline
66. क्ष॒त्रम् नो॑ नः क्ष॒त्रम् क्ष॒त्रम् नो॒ अश्वो॒ अश्वो॑ नः क्ष॒त्रम् क्ष॒त्रम् नो॒ अश्वः॑ । \newline
67. नो॒ अश्वो॒ अश्वो॑ नो नो॒ अश्वो॑ वनतां ॅवनता॒ मश्वो॑ नो नो॒ अश्वो॑ वनताम् । \newline
68. अश्वो॑ वनतां ॅवनता॒ मश्वो॒ अश्वो॑ वनताꣳ ह॒विष्मान्॑. ह॒विष्मान्॑. वनता॒ मश्वो॒ अश्वो॑ वनताꣳ ह॒विष्मान्॑ । \newline
69. व॒न॒ताꣳ॒॒ ह॒विष्मान्॑. ह॒विष्मान्॑. वनतां ॅवनताꣳ ह॒विष्मान्॑ । \newline
70. ह॒विष्मा॒निति॑ ह॒विष्मान्॑ । \newline
\pagebreak


\end{document}