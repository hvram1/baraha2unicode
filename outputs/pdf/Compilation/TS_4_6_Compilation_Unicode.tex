\documentclass[17pt]{extarticle}
\usepackage{babel}
\usepackage{fontspec}
\usepackage{polyglossia}
\usepackage{extsizes}

\usepackage{color}   %May be necessary if you want to color links
\usepackage{hyperref}
\hypersetup{
    colorlinks=true, %set true if you want colored links
    linktoc=all,     %set to all if you want both sections and subsections linked
    linkcolor=black,  %choose some color if you want links to stand out
}

\setmainlanguage{sanskrit}
\setotherlanguages{english} %% or other languages
\setlength{\parindent}{0pt}
\pagestyle{myheadings}
\newfontfamily\devanagarifont[Script=Devanagari]{AdishilaVedic}
\renewcommand{\theHsection}{\thepart.section.\thesection}

\newcommand{\VAR}[1]{}
\newcommand{\BLOCK}[1]{}




\begin{document}
\begin{titlepage}
    \begin{center}
 
\begin{sanskrit}
    { \Large
    कृष्ण यजुर्वेदीय तैत्तिरीय संहिता,पद,जटा,घन पाठः 
    }
    \\
    \vspace{2.5cm}
    \mbox{ \Large
    4.6      चतुर्थकाण्डे षष्ठः प्रश्नः - परिषेचन-संस्काराभिधानं   }
\end{sanskrit}
\end{center}

\end{titlepage}
\tableofcontents
\phantomsection
\pagebreak

\markright{ TS 4.6.1.1  \hfill https://www.vedavms.in \hfill}

\section{ TS 4.6.1.1 }

\textbf{TS 4.6.1.1 } \newline
\textbf{Samhita Paata} \newline

अश्म॒न्नूर्जं॒ पर्व॑ते शिश्रिया॒णां ॅवाते॑ प॒र्जन्ये॒ वरु॑णस्य॒ शुष्मे᳚ । अ॒द्भ्य ओष॑धीभ्यो॒ वन॒स्पति॒भ्योऽधि॒ संभृ॑तां॒ तां न॒ इष॒मूर्जं॑ धत्त मरुतः सꣳ ररा॒णाः ॥ अश्मꣳ॑स्ते॒ क्षुद॒मुं ते॒ शुगृ॑च्छतु॒ यं द्वि॒ष्मः ॥ स॒मु॒द्रस्य॑ त्वा॒ऽ*वाक॒याऽग्ने॒ परि॑ व्ययामसि । पा॒व॒को अ॒स्मभ्यꣳ॑ शि॒वो भ॑व ॥ हि॒मस्य॑ त्वा ज॒रायु॒णाऽग्ने॒ परि॑ व्ययामसि । पा॒व॒को अ॒स्मभ्यꣳ॑ शि॒वो भ॑व ॥ उप॒ - [  ] \newline

\textbf{Pada Paata} \newline

अश्मन्न्॑ । ऊर्ज᳚म् । पर्व॑ते । शि॒श्रि॒या॒णाम् । वाते᳚ । प॒र्जन्ये᳚ । वरु॑णस्य । शुष्मे᳚ ॥ अ॒द्भ्य इत्य॑त् - भ्यः । ओष॑धीभ्य॒ इत्योष॑धि - भ्यः॒ । वन॒स्पति॑भ्य॒ इति॒ वन॒स्पति॑ - भ्यः॒ । अधीति॑ । संभृ॑ता॒मिति॒ सं - भृ॒ता॒म् । ताम् । नः॒ । इष᳚म् । ऊर्ज᳚म् । ध॒त्त॒ । म॒रु॒तः॒ । सꣳ॒॒र॒रा॒णा इति॑ सं - र॒रा॒णाः ॥ अश्मन्न्॑ । ते॒ । क्षुत् । अ॒मुम् । ते॒ । शुक् । ऋ॒च्छ॒तु॒ । यम् । द्वि॒ष्मः ॥ स॒मु॒द्रस्य॑ । त्वा॒ । अ॒वाक॑या । अग्ने᳚ । परीति॑ । व्य॒या॒म॒सि॒ ॥ पा॒व॒कः । अ॒स्मभ्य॒मित्य॒स्म - भ्य॒म् । शि॒वः । भ॒व॒ ॥ हि॒मस्य॑ । त्वा॒ । ज॒रायु॑णा । अग्ने᳚ । परीति॑ । व्य॒या॒म॒सि॒ ॥ पा॒व॒कः । अ॒स्मभ्य॒मित्य॒स्म - भ्य॒म् । शि॒वः । भ॒व॒ ॥ उपेति॑ ।  \newline




\markright{ TS 4.6.1.2  \hfill https://www.vedavms.in \hfill}

\section{ TS 4.6.1.2 }

\textbf{TS 4.6.1.2 } \newline
\textbf{Samhita Paata} \newline

-ज्मन्नुप॑ वेत॒सेऽव॑त्तरं न॒दीष्वा । अग्ने॑ पि॒त्तम॒पाम॑सि ॥ मण्डू॑कि॒ ताभि॒रा ग॑हि॒ सेमं नो॑ य॒ज्ञ्ं । पा॒व॒कव॑र्णꣳ शि॒वं कृ॑धि ॥ पा॒व॒क आ चि॒तय॑न्त्या कृ॒पा । क्षाम॑न् रुरु॒च उ॒षसो॒ न भा॒नुना᳚ ॥ तूर्व॒न् न याम॒न्नेत॑शस्य॒ नू रण॒ आ यो घृ॒णे । न त॑तृषा॒णो अ॒जरः॑ ॥ अग्ने॑ पावक रो॒चिषा॑ म॒न्द्रया॑ देव जि॒ह्वया᳚ । आ दे॒वान् - [  ] \newline

\textbf{Pada Paata} \newline

ज्मन्न् । उपेति॑ । वे॒त॒से । अव॑त्तर॒मित्यव॑त्-त॒र॒म् । न॒दीषु॑ । आ ॥ अग्ने᳚ । पि॒त्तम् । अ॒पाम् । अ॒सि॒ ॥ मण्डू॑कि । ताभिः॑ । एति॑ । ग॒हि॒ । सा । इ॒मम् । नः॒ । य॒ज्ञ्म् ॥ पा॒व॒कव॑र्ण॒मिति॑ पाव॒क - व॒र्ण॒म् । शि॒वम् । कृ॒धि॒ ॥ पा॒व॒के । एति॑ । चि॒तय॑न्त्या । कृ॒पा ॥ क्षामन्न्॑ । रु॒रु॒चे । उ॒षसः॑ । न । भा॒नुना᳚ ॥ तूर्वन्न्॑ । न । यामन्न्॑ । एत॑शस्य । नु । रणे᳚ । एति॑ । यः । घृ॒णे ॥ न । त॒तृ॒षा॒णः । अ॒जरः॑ ॥ अग्ने᳚ । पा॒व॒क॒ । रो॒चिषा᳚ । म॒न्द्रया᳚ । दे॒व॒ । जि॒ह्वया᳚ ॥ एति॑ । दे॒वान् ।  \newline




\markright{ TS 4.6.1.3  \hfill https://www.vedavms.in \hfill}

\section{ TS 4.6.1.3 }

\textbf{TS 4.6.1.3 } \newline
\textbf{Samhita Paata} \newline

व॑क्षि॒ यक्षि॑ च ॥ स नः॑ पावक दीदि॒वोऽग्ने॑ दे॒वाꣳ इ॒हाऽऽ व॑ह । उप॑ य॒ज्ञ्ꣳ ह॒विश्च॑ नः ॥ अ॒पामि॒दं न्यय॑नꣳ समु॒द्रस्य॑ नि॒वेश॑नं । अ॒न्यं ते॑ अ॒स्मत् त॑पन्तु हे॒तयः॑ पाव॒को अ॒स्मभ्यꣳ॑ शि॒वो भ॑व ॥ नम॑स्ते॒ हर॑से शो॒चिषे॒ नम॑स्ते अस्त्व॒र्चिषे᳚ । अ॒न्यं ते॑ अ॒स्मत् त॑पन्तु हे॒तयः॑ पाव॒को अ॒स्मभ्यꣳ॑ शि॒वो भ॑व ॥ नृ॒षदे॒ वड॑ - [  ] \newline

\textbf{Pada Paata} \newline

व॒क्षि॒ । यक्षि॑ । च॒ ॥ सः । नः॒ । पा॒व॒क॒ । दी॒दि॒वः॒ । अग्ने᳚ । दे॒वान् । इ॒ह । एति॑ । व॒ह॒ ॥ उपेति॑ । य॒ज्ञ्म् । ह॒विः । च॒ । नः॒ ॥ अ॒पाम् । इ॒दम् । न्यय॑न॒मिति॑ नि - अय॑नम् । स॒मु॒द्रस्य॑ । नि॒वेश॑न॒मिति॑ नि - वेश॑नम् ॥ अ॒न्यम् । ते॒ । अ॒स्मत् । त॒प॒न्तु॒ । हे॒तयः॑ । पा॒व॒कः । अ॒स्मभ्य॒मित्य॒स्म - भ्य॒म् । शि॒वः । भ॒व॒ ॥ नमः॑ । ते॒ । हर॑से । शो॒चिषे᳚ । नमः॑ । ते॒ । अ॒स्तु॒ । अ॒र्चिषे᳚ ॥ अ॒न्यम् । ते॒ । अ॒स्मत् । त॒प॒न्तु॒ । हे॒तयः॑ । पा॒व॒कः । अ॒स्मभ्य॒मित्य॒स्म - भ्य॒म् । शि॒वः । भ॒व॒ ॥ नृ॒षद॒ इति॑ नृ - सदे᳚ । वट् ।  \newline




\markright{ TS 4.6.1.4  \hfill https://www.vedavms.in \hfill}

\section{ TS 4.6.1.4 }

\textbf{TS 4.6.1.4 } \newline
\textbf{Samhita Paata} \newline

-फ्सु॒षदे॒ वड् व॑न॒सदे॒ वड् ब॑र्.हि॒षदे॒ वट्थ् सु॑व॒र्विदे॒ वट् ॥ ये दे॒वा दे॒वानां᳚ ॅय॒ज्ञिया॑ य॒ज्ञिया॑नाꣳ संॅवथ् स॒रीण॒मुप॑ भा॒गमास॑ते । अ॒हु॒तादो॑ ह॒विषो॑ य॒ज्ञे अ॒स्मिन्थ् स्व॒यं जु॑हुद्ध्वं॒ मधु॑नो घृ॒तस्य॑ ॥ ये दे॒वा दे॒वेष्वधि॑ देव॒त्वमाय॒न्॒ ये ब्रह्म॑णः पुर ए॒तारो॑ अ॒स्य । येभ्यो॒ नर्ते पव॑ते॒ धाम॒ किं च॒न न ते दि॒वो न पृ॑थि॒व्या अधि॒ स्नुषु॑ ॥ प्रा॒ण॒दा - [  ] \newline

\textbf{Pada Paata} \newline

अ॒फ्सु॒षद॒ इत्य॑फ्सु - सदे᳚ । वट् । व॒न॒सद॒ इति॑ वन - सदे᳚ । वट् । ब॒र्॒.हि॒षद॒ इति॑ बर्.हि - सदे᳚ । वट् । सु॒व॒र्विद॒ इति॑ सुवः - विदे᳚ । वट् ॥ ये । दे॒वाः । दे॒वाना᳚म् । य॒ज्ञियाः᳚ । य॒ज्ञिया॑नाम् । सं॒ॅव॒थ्स॒रीण॒मिति॑ सं - व॒थ्स॒रीण᳚म् । उपेति॑ । भा॒गम् । आस॑ते ॥ अ॒हु॒ताद॒ इत्य॑हुत - अदः॑ । ह॒विषः॑ । य॒ज्ञे । अ॒स्मिन्न् । स्व॒यम् । जु॒हु॒द्ध्व॒म् । मधु॑नः । घृ॒तस्य॑ ॥ ये । दे॒वाः । दे॒वेषू॑ । अधीति॑ । दे॒व॒त्वमिति॑ देव - त्वम् । आयन्न्॑ । ये । ब्रह्म॑णः । पु॒र॒ ए॒तार॒ इति॑ पुरः - ए॒तारः॑ । अ॒स्य ॥ येभ्यः॑ । न । ऋ॒ते । पव॑ते । धाम॑ । किम् । च॒न । न । ते । दि॒वः । न । पृ॒थि॒व्याः । अधीति॑ । स्नुषु॑ ॥ प्रा॒ण॒दा इति॑ प्राण - दाः ।  \newline




\markright{ TS 4.6.1.5  \hfill https://www.vedavms.in \hfill}

\section{ TS 4.6.1.5 }

\textbf{TS 4.6.1.5 } \newline
\textbf{Samhita Paata} \newline

अ॑पान॒दा व्या॑न॒दाश्च॑क्षु॒र्दा व॑र्चो॒दा व॑रिवो॒दाः । अ॒न्यं ते॑ अ॒स्मत् त॑पन्तु हे॒तयः॑ पाव॒को अ॒स्मभ्यꣳ॑ शि॒वो भ॑व ॥ अ॒ग्निस्ति॒ग्मेन॑ शो॒चिषा॒ यꣳस॒द्विश्वं॒ न्य॑त्रिणं᳚ । अ॒ग्निर्नो॑ वꣳसते र॒यिं ॥ सैनाऽनी॑केन सुवि॒दत्रो॑ अ॒स्मे यष्टा॑ दे॒वाꣳ आय॑जिष्ठः स्व॒स्ति । अद॑ब्धो गो॒पा उ॒त नः॑ पर॒स्पा अग्ने᳚ द्यु॒मदु॒त रे॒वद्-दि॑दीहि ॥ \newline

\textbf{Pada Paata} \newline

अ॒पा॒न॒दा इत्य॑पान - दाः । व्या॒न॒दा इति॑ व्यान - दाः । च॒क्षु॒र्दा इति॑ चक्षुः-दाः । व॒र्चो॒दा इति॑ वर्चः-दाः । व॒रि॒वो॒दा इति॑ वरिवः-दाः ॥ अ॒न्यम् । ते॒ । अ॒स्मत् । त॒प॒न्तु॒ । हे॒तयः॑ । पा॒व॒कः । अ॒स्मभ्य॒मित्य॒स्म - भ्य॒म् । शि॒वः । भ॒व॒ ॥ अ॒ग्निः । ति॒ग्मेन॑ । शो॒चिषा᳚ । यꣳस॑त् । विश्व᳚म् । नीति॑ । अ॒त्रिण᳚म् ॥ अ॒ग्निः । नः॒ । वꣳ॒॒स॒ते॒ । र॒यिम् ॥ सः । ए॒ना । अनी॑केन । सु॒वि॒दत्र॒ इति॑ सु-वि॒दत्रः॑ । अ॒स्मे इति॑ । यष्टा᳚ । दे॒वान् । आय॑जिष्ठ॒ इत्या-य॒जि॒ष्ठः॒ । स्व॒स्ति ॥ अद॑ब्धः । गो॒पा इति॑ गो - पाः । उ॒त । नः॒ । प॒र॒स्पा इति॑ परः - पाः । अग्ने᳚ । द्यु॒मदिति॑ द्यु - मत् । उ॒त । रे॒वत् । दि॒दी॒हि॒ ॥  \newline




\markright{ TS 4.6.2.1  \hfill https://www.vedavms.in \hfill}

\section{ TS 4.6.2.1 }

\textbf{TS 4.6.2.1 } \newline
\textbf{Samhita Paata} \newline

य इ॒मा विश्वा॒ भुव॑नानि॒ जुह्व॒दृषि॒र्॒.होता॑ निष॒सादा॑ पि॒ता नः॑ । स आ॒शिषा॒ द्रवि॑णमि॒च्छमा॑नः परम॒च्छदो॒ वर॒ आ वि॑वेश ॥ वि॒श्वक॑र्मा॒ मन॑सा॒ यद्विहा॑या धा॒ता वि॑धा॒ता प॑र॒मोत स॒न्दृक् । तेषा॑मि॒ष्टानि॒ समि॒षा म॑दन्ति॒ यत्र॑ सप्त॒र्॒.षीन् प॒र एक॑मा॒हुः ॥ यो नः॑ पि॒ता ज॑नि॒ता यो वि॑धा॒ता यो नः॑ स॒तो अ॒भ्या सज्ज॒जान॑ । \newline

\textbf{Pada Paata} \newline

य । इ॒मा । विश्वा᳚ । भुव॑नानि । जुह्व॑त् । ऋषिः॑ । होता᳚ । नि॒ष॒सादेति॑ नि - स॒साद॑ । पि॒ता । नः॒ ॥ सः । आ॒शिषेत्या᳚ -शिषा᳚ । द्रवि॑णम् । इ॒च्छमा॑नः । प॒र॒म॒च्छद॒ इति॑ परम - छदः॑ । वरे᳚ । एति॑ । वि॒वे॒श॒ ॥ वि॒श्वक॒र्मेति॑ वि॒श्व-क॒र्मा॒ । मन॑सा । यत् । विहा॑या॒ इति॒ वि-हा॒याः॒ । धा॒ता । वि॒धा॒तेति॑ वि - धा॒ता । प॒र॒मा । उ॒त । स॒न्दृगिति॑ सं-दृक् ॥ तेषा᳚म् । इ॒ष्टानि॑ । समिति॑ । इ॒षा । म॒द॒न्ति॒ । यत्र॑ । स॒प्त॒र्॒.षीनिति॑ सप्त-ऋ॒षीन् । प॒रः । एक᳚म् । आ॒हुः ॥ यः । नः॒ । पि॒ता । ज॒नि॒ता । यः । वि॒धा॒तेति॑ वि - धा॒ता । यः । नः॒ । स॒तः । अ॒भि । एति॑ । सत् । जजान॑ ॥  \newline




\markright{ TS 4.6.2.2  \hfill https://www.vedavms.in \hfill}

\section{ TS 4.6.2.2 }

\textbf{TS 4.6.2.2 } \newline
\textbf{Samhita Paata} \newline

यो दे॒वानां᳚ नाम॒धा एक॑ ए॒व तꣳ स॑प्रं॒श्नं भुव॑ना यन्त्य॒न्या ॥ त आऽय॑जन्त॒ द्रवि॑णꣳ॒॒ सम॑स्मा॒ ऋष॑यः॒ पूर्वे॑ जरि॒तारो॒ न भू॒ना । अ॒सूर्ता॒ सूर्ता॒ रज॑सो वि॒माने॒ ये भू॒तानि॑ स॒मकृ॑ण्वन्नि॒मानि॑ ॥ न तं ॅवि॑दाथ॒ य इ॒दं ज॒जाना॒न्यद् यु॒ष्माक॒मन्त॑रं भवाति । नी॒हा॒रेण॒ प्रावृ॑ता॒ जल्प्या॑ चासु॒तृप॑ उक्थ॒ शास॑श्चरन्ति ॥ प॒रो दि॒वा प॒र ए॒ना - [  ] \newline

\textbf{Pada Paata} \newline

यः । दे॒वाना᳚म् । ना॒म॒धा इति॑ नाम - धाः । एकः॑ । ए॒व । तम् । स॒प्रं॒श्नमिति॑ सं - प्र॒श्नम् । भुव॑ना । य॒न्ति॒ । अ॒न्या ॥ ते । एति॑ । अ॒य॒ज॒न्त॒ । द्रवि॑णम् । समिति॑ । अ॒स्मै॒ । ऋष॑यः । पूर्वे᳚ । ज॒रि॒तारः॑ । न॒ । भू॒ना ॥ अ॒सूर्ता᳚ । सूर्ता᳚ । रज॑सः । वि॒मान॒ इति॑ वि - माने᳚ । ये । भू॒तानि॑ । स॒मकृ॑ण्व॒न्निति॑ सं - अकृ॑ण्वन्न् । इ॒मानि॑ ॥ न । तम् । वि॒दा॒थ॒ । यः । इ॒दम् । ज॒जान॑ । अ॒न्यत् । यु॒ष्माक᳚म् । अन्त॑रम् । भ॒वा॒ति॒ ॥ नी॒हा॒रेण॑ । प्रावृ॑ताः । जल्प्या᳚ । च॒ । अ॒सु॒तृप॒ इत्य॑सु - तृपः॑ । उ॒क्थ॒शास॒ इत्यु॑क्थ - शासः॑ । च॒र॒न्ति॒ ॥ प॒रः । दि॒वा । प॒रः । ए॒ना ।  \newline




\markright{ TS 4.6.2.3  \hfill https://www.vedavms.in \hfill}

\section{ TS 4.6.2.3 }

\textbf{TS 4.6.2.3 } \newline
\textbf{Samhita Paata} \newline

पृ॑थि॒व्या प॒रो दे॒वेभि॒रसु॑रै॒र्गुहा॒ यत् । कꣳ स्वि॒द्गर्भं॑ प्रथ॒मं द॑द्ध्र॒ आपो॒ यत्र॑ दे॒वाः स॒मग॑च्छन्त॒ विश्वे᳚ ॥ तमिद्गर्भं॑ प्रथ॒मं द॑द्ध्र॒ आपो॒ यत्र॑ दे॒वाः स॒मग॑च्छन्त॒ विश्वे᳚ । अ॒जस्य॒ नाभा॒वद्ध्येक॒-मर्पि॑तं॒ ॅयस्मि॑न्नि॒दं ॅविश्वं॒ भुव॑न॒मधि॑ श्रि॒तं ॥ वि॒श्वक॑र्मा॒ ह्यज॑निष्ट दे॒व आदिद्-ग॑न्ध॒र्वो अ॑भवद् द्वि॒तीयः॑ । तृ॒तीयः॑ पि॒ता ज॑नि॒तौष॑धीना - [  ] \newline

\textbf{Pada Paata} \newline

पृ॒थि॒व्या । प॒रः । दे॒वेभिः॑ । असु॑रैः । गुहा᳚ । यत् ॥ कम् । स्वि॒त् । गर्भ᳚म् । प्र॒थ॒मम् । द॒द्ध्रे॒ । आपः॑ । यत्र॑ । दे॒वाः । स॒मग॑च्छ॒न्तेति॑ सं - अग॑च्छन्त । विश्वे᳚ ॥ तम् । इत् । गर्भ᳚म् । प्र॒थ॒मम् । द॒द्ध्रे॒ । आपः॑ । यत्र॑ । दे॒वाः । स॒मग॑च्छ॒न्तेति॑ सं - अग॑च्छन्त । विश्वे᳚ ॥ अ॒जस्य॑ । नाभौ᳚ । अधीति॑ । एक᳚म् । अर्पि॑तम् । यस्मिन्न्॑ । इ॒दम् । विश्व᳚म् । भुव॑नम् । अधीति॑ । श्रि॒तम् ॥ वि॒श्वक॒र्मेति॑ वि॒श्व - क॒र्मा॒ । हि । अज॑निष्ट । दे॒वः । आत् । इत् । ग॒न्ध॒र्वः । अ॒भ॒व॒त् । द्वि॒तीयः॑ ॥ तृ॒तीयः॑ । पि॒ता । ज॒नि॒ता । ओष॑धीनाम् ।  \newline




\markright{ TS 4.6.2.4  \hfill https://www.vedavms.in \hfill}

\section{ TS 4.6.2.4 }

\textbf{TS 4.6.2.4 } \newline
\textbf{Samhita Paata} \newline

-म॒पां गर्भं॒ ॅव्य॑दधात् पुरु॒त्रा ॥ चक्षु॑षः पि॒ता मन॑सा॒ हि धीरो॑ घृ॒तमे॑ने अजन॒न्नन्न॑माने । य॒देदन्ता॒ अद॑दृꣳहन्त॒ पूर्व॒ आदिद् द्यावा॑पृथि॒वी अ॑प्रथेतां ॥ वि॒श्वत॑-श्चक्षुरु॒त वि॒श्वतो॑मुखो वि॒श्वतो॑हस्त उ॒त वि॒श्वत॑स्पात् । सं बा॒हुभ्यां॒ नम॑ति॒ सं पत॑त्रै॒ र्द्यावा॑पृथि॒वी ज॒नय॑न् दे॒व एकः॑ ॥ किꣳ स्वि॑दासी-दधि॒ष्ठान॑-मा॒रंभ॑णं कत॒मथ् स्वि॒त् किमा॑सीत् । यदी॒ भूमिं॑ ज॒नय॑न् - [  ] \newline

\textbf{Pada Paata} \newline

अ॒पाम् । गर्भ᳚म् । वीति॑ । अ॒द॒धा॒त् । पु॒रु॒त्रेति॑ पुरु - त्रा ॥ चक्षु॑षः । पि॒ता । मन॑सा । हि । धीरः॑ । घृ॒तम् । ए॒ने॒ इति॑ । अ॒ज॒न॒त् । नन्न॑माने॒ इति॑ ॥ य॒दा । इत् । अन्ताः᳚ । अद॑दृꣳहन्त । पूर्वे᳚ । आत् । इत् । द्यावा॑पृथि॒वी इति॒ द्यावा᳚ - पृ॒थि॒वी । अ॒प्र॒थे॒ता॒म् ॥ वि॒श्वत॑श्चक्षु॒रिति॑ वि॒श्वतः॑ - च॒क्षुः॒ । उ॒त । वि॒श्वतो॑मुख॒ इति॑ वि॒श्वतः॑ - मु॒खः॒ । वि॒श्वतो॑हस्त॒ इति॑ वि॒श्वतः॑ - ह॒स्तः॒ । उ॒त । वि॒श्वत॑स्पा॒दिति॑ वि॒श्वतः॑-पा॒त् ॥ समिति॑ । बा॒हुभ्या॒मिति॑ बा॒हु-भ्या॒म् । नम॑ति । समिति॑ । पत॑त्रैः । द्यावा॑पृथि॒वी इति॒ द्यावा᳚ - पृ॒थि॒वी । ज॒नयन्न्॑ । दे॒वः । एकः॑ ॥ किम् । स्वि॒त् । आ॒सी॒त् । अ॒धि॒ष्ठान॒मित्य॑धि - स्थान᳚म् । आ॒रंभ॑ण॒मित्या᳚ - रंभ॑णम् । क॒त॒मत् । स्वि॒त् । किम् । आ॒सी॒त् ॥ यदि॑ । भूमि᳚म् । ज॒नयन्न्॑ ।  \newline




\markright{ TS 4.6.2.5  \hfill https://www.vedavms.in \hfill}

\section{ TS 4.6.2.5 }

\textbf{TS 4.6.2.5 } \newline
\textbf{Samhita Paata} \newline

वि॒श्वक॑र्मा॒ वि द्यामौर्णो᳚न् महि॒ना वि॒श्वच॑क्षाः ॥ किꣳ स्वि॒द्वनं॒ क उ॒ स वृ॒क्ष आ॑सी॒द्यतो॒ द्यावा॑पृथि॒वी नि॑ष्टत॒क्षुः । मनी॑षिणो॒ मन॑सा पृ॒च्छतेदु॒ तद्यद॒द्ध्यति॑ष्ठ॒द् भुव॑नानि धा॒रयन्न्॑ ॥ या ते॒ धामा॑नि पर॒माणि॒ याऽव॒मा या म॑द्ध्य॒मा वि॑श्वकर्मन्नु॒तेमा । शिक्षा॒ सखि॑भ्यो ह॒विषि॑ स्वधावः स्व॒यं ॅय॑जस्व त॒नुवं॑ जुषा॒णः ॥ वा॒चस्पतिं॑ ॅवि॒श्वक॑र्माणमू॒तये॑ - [  ] \newline

\textbf{Pada Paata} \newline

वि॒श्वक॒र्मेति॑ वि॒श्व - क॒र्मा॒ । वीति॑ । द्याम् । और्णो᳚त् । म॒हि॒ना । वि॒श्वच॑क्षा॒ इति॑ वि॒श्व - च॒क्षाः॒ ॥ किम् । स्वि॒त् । वन᳚म् । कः । उ॒ । सः । वृ॒क्षः । आ॒सी॒त् । यतः॑ । द्यावा॑पृथि॒वी इति॒ द्यावा᳚ - पृ॒थि॒वी । नि॒ष्ट॒त॒क्षुरिति॑ निः - त॒त॒क्षुः ॥ मनी॑षिणः । मन॑सा । पृ॒च्छत॑ । इत् । उ॒ । तत् । यत् । अ॒द्ध्यति॑ष्ठ॒दित्य॑धि - अति॑ष्ठत् । भुव॑नानि । धा॒रयन्न्॑ ॥ या । ते॒ । धामा॑नि । प॒र॒माणि॑ । या । अ॒व॒मा । या । म॒द्ध्य॒मा । वि॒श्व॒क॒र्म॒न्निति॑ विश्व - क॒र्म॒न्न् । उ॒त । इ॒मा ॥ शिक्ष॑ । सखि॑भ्य॒ इति॒ सखि॑ - भ्यः॒ । ह॒विषि॑ । स्व॒धा॒व॒ इति॑ स्वधा - वः॒ । स्व॒यम् । य॒ज॒स्व॒ । त॒नुव᳚म् । जु॒षा॒णः ॥ वा॒चः । पति᳚म् । वि॒श्वक॑र्माण॒मिति॑ वि॒श्व - क॒र्मा॒ण॒म् । ऊ॒तये᳚ ।  \newline




\markright{ TS 4.6.2.6  \hfill https://www.vedavms.in \hfill}

\section{ TS 4.6.2.6 }

\textbf{TS 4.6.2.6 } \newline
\textbf{Samhita Paata} \newline

मनो॒युजं॒ ॅवाजे॑ अ॒द्या हु॑वेम । स नो॒ नेदि॑ष्ठा॒ हव॑नानि जोषते वि॒श्वश॑भूं॒रव॑से सा॒धुक॑र्मा ॥ विश्व॑कर्मन्. ह॒विषा॑ वावृधा॒नः स्व॒यं ॅय॑जस्व त॒नुवं॑ जुषा॒णः । मुह्य॑न्त्व॒न्ये अ॒भितः॑ स॒पत्ना॑ इ॒हास्माकं॑ म॒घवा॑ सू॒रिर॑स्तु ॥ विश्व॑कर्मन्. ह॒विषा॒ वर्द्ध॑नेन त्रा॒तार॒मिन्द्र॑-मकृणोरव॒द्ध्यं । तस्मै॒ विशः॒ सम॑नमन्त पू॒र्वीर॒यमु॒ग्रो वि॑ह॒व्यो॑ यथाऽस॑त् ॥ स॒मु॒द्राय॑ व॒युना॑य॒ सिन्धू॑नां॒ पत॑ये॒ नमः॑ ( ) । न॒दीनाꣳ॒॒ सर्वा॑सां पि॒त्रे जु॑हु॒ता वि॒श्वक॑र्मणे॒ विश्वाऽहाम॑र्त्यꣳ ह॒विः ॥ \newline

\textbf{Pada Paata} \newline

म॒नो॒युज॒मिति॑ मनः - युज᳚म् । वाजे᳚ । अ॒द्य । हु॒वे॒म॒ ॥ सः । नः॒ । नेदि॑ष्ठा । हव॑नानि । जो॒ष॒ते॒ । वि॒श्वश॑भूं॒रिति॑ वि॒श्व - श॒भूंः॒ । अव॑से । सा॒धुक॒र्मेति॑ सा॒धु - क॒र्मा॒ ॥ विश्व॑कर्म॒न्निति॒ विश्व॑-क॒र्म॒न्न् । ह॒विषा᳚ । वा॒वृ॒धा॒नः । स्व॒यम् । य॒ज॒स्व॒ । त॒नुव᳚म् । जु॒षा॒णः ॥ मुह्य॑न्तु । अ॒न्ये । अ॒भितः॑ । स॒पत्नाः᳚ । इ॒ह । अ॒स्माक᳚म् । म॒घवेति॑ म॒घ - वा॒ । सू॒रिः । अ॒स्तु॒ ॥ विश्व॑कर्म॒न्निति॒ विश्व॑ - क॒र्म॒न्न् । ह॒विषा᳚ । वद्‌र्ध॑नेन । त्रा॒तार᳚म् । इन्द्र᳚म् । अ॒कृ॒णोः॒ । अ॒व॒द्ध्यम् ॥ तस्मै᳚ । विशः॑ । समिति॑ । अ॒न॒म॒न्त॒ । पू॒र्वीः । अ॒यम् । उ॒ग्रः । वि॒ह॒व्य॑ इति॑ वि-ह॒व्यः॑ । यथा᳚ । अस॑त् ॥ स॒मु॒द्राय॑ । व॒युना॑य । सिन्धू॑नाम् । पत॑ये । नमः॑ ( ) ॥ न॒दीना᳚म् । सर्वा॑साम् । पि॒त्रे । जु॒हु॒त । वि॒श्वक॑र्मण॒ इति॑ वि॒श्व - क॒र्म॒ण॒ । विश्वा᳚ । अहा᳚ । अम॑र्त्यम् । ह॒विः ॥  \newline




\markright{ TS 4.6.3.1  \hfill https://www.vedavms.in \hfill}

\section{ TS 4.6.3.1 }

\textbf{TS 4.6.3.1 } \newline
\textbf{Samhita Paata} \newline

उदे॑नमुत्त॒रां न॒याग्ने॑ घृतेनाऽऽहुत । रा॒यस्पोषे॑ण॒ सꣳ सृ॑ज प्र॒जया॑ च॒ धने॑न च ॥ इन्द्रे॒मं प्र॑त॒रां कृ॑धि सजा॒ताना॑-मसद्व॒शी । समे॑नं॒ ॅवर्च॑सा सृज दे॒वेभ्यो॑ भाग॒धा अ॑सत् ॥ यस्य॑ कु॒र्मो ह॒विर्गृ॒हे तम॑ग्ने वर्द्धया॒ त्वं । तस्म॑ दे॒वा अधि॑ ब्रवन्न॒यं च॒ ब्रह्म॑ण॒स्पतिः॑ ॥ उदु॑ त्वा॒ विश्वे॑ दे॒वा - [  ] \newline

\textbf{Pada Paata} \newline

उदिति॑ । ए॒न॒म् । उ॒त्त॒रामित्यु॑त् - त॒राम् । न॒य॒ । अग्ने᳚ । घृ॒ते॒न॒ । आ॒हु॒तेत्या᳚ - हु॒त॒ ॥ रा॒यः । पोषे॑ण । समिति॑ । सृ॒ज॒ । प्र॒जयेति॑ प्र - जया᳚ । च॒ । धने॑न । च॒ ॥ इन्द्र॑ । इ॒मम् । प्र॒त॒रामिति॑ प्र-त॒राम् । कृ॒धि॒ । स॒जा॒ताना॒मिति॑ स - जा॒ताना᳚म् । अ॒स॒त् । व॒शी ॥ समिति॑ । ए॒न॒म् । वर्च॑सा । सृ॒ज॒ । दे॒वेभ्यः॑ । भा॒ग॒धा इति॑ भाग - धाः । अ॒स॒त् ॥ यस्य॑ । कु॒र्मः । ह॒विः । गृ॒हे । तम् । अ॒ग्ने॒ । व॒द्‌र्ध॒य॒ । त्वम् ॥ तस्मै᳚ । दे॒वाः । अधीति॑ । ब्र॒व॒न्न् । अ॒यम् । च॒ । ब्रह्म॑णः । पतिः॑ ॥ उदिति॑ । उ॒ । त्वा॒ । विश्वे᳚ । दे॒वाः ।  \newline




\markright{ TS 4.6.3.2  \hfill https://www.vedavms.in \hfill}

\section{ TS 4.6.3.2 }

\textbf{TS 4.6.3.2 } \newline
\textbf{Samhita Paata} \newline

अग्ने॒ भर॑न्तु॒ चित्ति॑भिः । स नो॑ भव शि॒वत॑मः सु॒प्रती॑को वि॒भाव॑सुः ॥ पञ्च॒ दिशो॒ दैवी᳚र्य॒ज्ञ्म॑वन्तु दे॒वीरपाम॑तिं दुर्म॒तिं बाध॑मानाः । रा॒यस्पोषे॑ य॒ज्ञ्प॑ति-मा॒भज॑न्तीः ॥ रा॒यस्पोषे॒ अधि॑ य॒ज्ञो अ॑स्था॒थ् समि॑द्धे अ॒ग्नावधि॑ मामहा॒नः । उ॒क्थप॑त्र॒ ईड्यो॑ गृभी॒तस्त॒प्तं घ॒र्मं प॑रि॒गृह्या॑यजन्त ॥ ऊ॒र्जा यद्-य॒ज्ञ्मश॑मन्त दे॒वा दैव्या॑य ध॒र्त्रे जोष्ट्रे᳚ । दे॒व॒श्रीः श्रीम॑णाः श॒तप॑याः - [  ] \newline

\textbf{Pada Paata} \newline

अग्ने᳚ । भर॑न्तु । चित्ति॑भि॒रिति॒ चित्ति॑ - भिः॒ ॥ सः । नः॒ । भ॒व॒ । शि॒वत॑म॒ इति॑ शि॒व - त॒मः॒ । सु॒प्रती॑क॒ इति॑ सु - प्रती॑कः । वि॒भाव॑सु॒रिति॑ वि॒भा - व॒सुः॒ ॥ पञ्च॑ । दिशः॑ । दैवीः᳚ । य॒ज्ञ्म् । अ॒व॒न्तु॒ । दे॒वीः । अपेति॑ । अम॑तिम् । दु॒र्म॒तिमिति॑ दुः - म॒तिम् । बाध॑मानाः ॥ रा॒यः । पोषे᳚ । य॒ज्ञ्प॑ति॒मिति॑ य॒ज्ञ् - प॒ति॒म् । आ॒भज॑न्ती॒रित्या᳚ - भज॑न्तीः ॥ रा॒यः । पोषे᳚ । अधीति॑ । य॒ज्ञ्ः । अ॒स्था॒त् । समि॑द्ध॒ इति॒ सं - इ॒द्धे॒ । अ॒ग्नौ । अधीति॑ । मा॒म॒हा॒नः ॥ उ॒क्थप॑त्र॒ इत्यु॒क्थ - प॒त्रः॒ । ईड्यः॑ । गृ॒भी॒तः । त॒प्तम् । घ॒र्मम् । प॒रि॒गृह्येति॑ परि-गृह्य॑ । अ॒य॒ज॒न्त॒ ॥ ऊ॒र्जा । यत् । य॒ज्ञ्म् । अश॑मन्त । दे॒वाः । दैव्या॑य । ध॒र्त्रे । जोष्ट्रे᳚ ॥ दे॒व॒श्रीरिति॑ देव - श्रीः । श्रीम॑णा॒ इति॒ श्री - म॒नाः॒ । श॒तप॑या॒ इति॑ श॒त - प॒याः॒ ।  \newline




\markright{ TS 4.6.3.3  \hfill https://www.vedavms.in \hfill}

\section{ TS 4.6.3.3 }

\textbf{TS 4.6.3.3 } \newline
\textbf{Samhita Paata} \newline

परि॒गृह्य॑ दे॒वा य॒ज्ञ्मा॑यन्न् ॥ सूर्य॑रश्मि॒र्॒.हरि॑केशः पु॒रस्ता᳚थ् सवि॒ता ज्योति॒रुद॑याꣳ॒॒ अज॑स्रं । तस्य॑ पू॒षा प्र॑स॒वं ॅया॑ति दे॒वः स॒पंश्य॒न् विश्वा॒ भुव॑नानि गो॒पाः ॥ दे॒वा दे॒वेभ्यो॑ अद्ध्व॒र्यन्तो॑ अस्थुर्वी॒तꣳ श॑मि॒त्रे श॑मि॒ता य॒जद्ध्यै᳚ । तु॒रीयो॑ य॒ज्ञो यत्र॑ ह॒व्यमेति॒ ततः॑ पाव॒का आ॒शिषो॑ नो जुषन्तां ॥ वि॒मान॑ ए॒ष दि॒वो मद्ध्य॑ आस्त आपप्रि॒वान् रोद॑सी अ॒न्तरि॑क्षं । स वि॒श्वाची॑र॒भि - [  ] \newline

\textbf{Pada Paata} \newline

प॒रि॒गृह्येति॑ परि - गृह्य॑ । दे॒वाः । य॒ज्ञ्म् । आ॒य॒न्न् ॥ सूर्य॑रश्मि॒रिति॒ सूर्य॑ - र॒श्मिः॒ । हरि॑केश॒ इति॒ हरि॑ - के॒शः॒ । पु॒रस्ता᳚त् । स॒वि॒ता । ज्योतिः॑ । उदिति॑ । अ॒या॒न् । अज॑स्रम् ॥ तस्य॑ । पू॒षा । प्र॒स॒वमिति॑ प्र - स॒वम् । या॒ति॒ । दे॒वः । स॒पंश्य॒न्निति॑ सं - पश्यन्न्॑ । विश्वा᳚ । भुव॑नानि । गो॒पा इति॑ गो - पाः ॥ दे॒वाः । दे॒वेभ्यः॑ । अ॒द्ध्व॒र्यन्तः॑ । अ॒स्थुः॒ । वी॒तम् । श॒मि॒त्रे । श॒मि॒ता । य॒जद्ध्यै᳚ ॥ तु॒रीयः॑ । य॒ज्ञ्ः । यत्र॑ । ह॒व्यम् । एति॑ । ततः॑ । पा॒व॒काः । आ॒शिष॒ इत्या᳚-शिषः॑ । नः॒ । जु॒ष॒न्ता॒म् ॥ वि॒मान॒ इति॑ वि-मानः॑ । ए॒षः । दि॒वः । म॒द्ध्ये᳚ । आ॒स्ते॒ । आ॒प॒प्रि॒वानित्या᳚ - प॒प्रि॒वान् । रोद॑सी॒ इति॑ । अ॒न्तरि॑क्षम् ॥ सः । वि॒श्वाचीः᳚ । अ॒भीति॑ ।  \newline




\markright{ TS 4.6.3.4  \hfill https://www.vedavms.in \hfill}

\section{ TS 4.6.3.4 }

\textbf{TS 4.6.3.4 } \newline
\textbf{Samhita Paata} \newline

च॑ष्टे घृ॒ताची॑रन्त॒रा पूर्व॒मप॑रं च के॒तुं ॥ उ॒क्षा स॑मु॒द्रो अ॑रु॒णः सु॑प॒र्णः पूर्व॑स्य॒ योनिं॑ पि॒तुरा वि॑वेश । मद्ध्ये॑ दि॒वो निहि॑तः॒ पृश्नि॒रश्मा॒ वि च॑क्रमे॒ रज॑सः पा॒त्यन्तौ᳚ ॥ इन्द्रं॒ ॅविश्वा॑ अवीवृधन्थ् समु॒द्रव्य॑चसं॒ गिरः॑ ।र॒थीत॑मꣳ रथी॒नां ॅवाजा॑नाꣳ॒॒ सत्प॑तिं॒ पतिं᳚ ॥ सु॒म्न॒हूर्य॒ज्ञो दे॒वाꣳ आ च॑ वक्ष॒द्यक्ष॑द॒ग्निर्दे॒वो दे॒वाꣳ आ च॑ वक्षत् ॥ वाज॑स्य ( ) मा प्रस॒वेनो᳚द्-ग्रा॒भेणो-द॑ग्रभीत् । अथा॑ स॒पत्नाꣳ॒॒ इन्द्रो॑ मे निग्रा॒भेणाध॑राꣳ अकः ॥ उ॒द्ग्रा॒भं च॑ निग्रा॒भं च॒ ब्रह्म॑ दे॒वा अ॑वीवृधन्न् । अथा॑ स॒पत्ना॑निन्द्रा॒ग्नी मे॑ विषू॒चीना॒न् व्य॑स्यतां ॥ \newline

\textbf{Pada Paata} \newline

च॒ष्टे॒ । घृ॒ताचीः᳚ । अ॒न्त॒रा । पूर्व᳚म् । अप॑रम् । च॒ । के॒तुम् ॥ उ॒क्षा । स॒मु॒द्रः । अ॒रु॒णः । सु॒प॒र्ण इति॑ सु - प॒र्णः । पूर्व॑स्य । योनि᳚म् । पि॒तुः । एति॑ । वि॒वे॒श॒ ॥ मद्ध्ये᳚ । दि॒वः । निहि॑त॒ इति॒ नि - हि॒तः॒ । पृश्निः॑ । अश्मा᳚ । वीति॑ । च॒क्र॒मे॒ । रज॑सः । पा॒ति॒ । अन्तौ᳚ ॥ इन्द्र᳚म् । विश्वाः᳚ । अ॒वी॒वृ॒ध॒न्न् । स॒मु॒द्रव्य॑चस॒मिति॑ समु॒द्र-व्य॒च॒स॒म् । गिरः॑ ॥ र॒थीत॑म॒मिति॑ र॒थि - त॒म॒म् । र॒थी॒नाम् । वाजा॑नाम् । सत्प॑ति॒मिति॒ सत् - प॒ति॒म् । पति᳚म् ॥ सु॒म्न॒हूरिति॑ सुम्न - हूः । य॒ज्ञ्ः । दे॒वान् । एति॑ । च॒ । व॒क्ष॒त् । यक्ष॑त् । अ॒ग्निः । दे॒वः । दे॒वान् । एति॑ । च॒ । व॒क्ष॒त् ॥ वाज॑स्य ( ) । मा॒ । प्र॒स॒वेनेति॑ प्र - स॒वेन॑ । उ॒द्ग्रा॒भेणेत्यु॑त् - ग्रा॒भेण॑ । उदिति॑ । अ॒ग्र॒भी॒त् ॥ अथ॑ । स॒पत्नान्॑ । इन्द्रः॑ । मे॒ । नि॒ग्रा॒भेणेति॑ नि - ग्रा॒भेण॑ । अध॑रान् । अ॒कः॒ ॥ उ॒द्ग्रा॒भमित्यु॑त् - ग्रा॒भम् । च॒ । नि॒ग्रा॒भमिति॑ नि - ग्रा॒भम् । च॒ । ब्रह्म॑ । दे॒वाः । अ॒वी॒वृ॒ध॒न्न् ॥ अथ॑ । स॒पत्नान्॑ । इ॒न्द्रा॒ग्नी इती᳚न्द्र - अ॒ग्नी । मे॒ । वि॒षू॒चीनान्॑ । वीति॑ । अ॒स्य॒ता॒म् ॥  \newline




\markright{ TS 4.6.4.1  \hfill https://www.vedavms.in \hfill}

\section{ TS 4.6.4.1 }

\textbf{TS 4.6.4.1 } \newline
\textbf{Samhita Paata} \newline

आ॒शुः शिशा॑नो वृष॒भो न यु॒द्ध्मो घ॑नाघ॒नः क्षोभ॑ण-श्चर्.षणी॒नां । सं॒क्रन्द॑नोऽनिमि॒ष ए॑क वी॒रः श॒तꣳ सेना॑ अजयथ् सा॒कमिन्द्रः॑ ॥ सं॒क्रन्द॑नेना निमि॒षेण॑ जि॒ष्णुना॑ युत्का॒रेण॑ दुश्च्यव॒नेन॑ धृ॒ष्णुना᳚ । तदिन्द्रे॑ण जयत॒ तथ् स॑हद्ध्वं॒ ॅयुधो॑ नर॒ इषु॑हस्तेन॒ वृष्णा᳚ ॥ स इषु॑हस्तैः॒ स नि॑षं॒गिभि॑र्व॒शी सꣳस्र॑ष्टा॒ स युध॒ इन्द्रो॑ ग॒णेन॑ । सꣳ॒॒सृ॒ष्ट॒जिथ् सो॑म॒पा बा॑हुश॒र्द्ध्यू᳚र्द्ध्वध॑न्वा॒ प्रति॑हिताभि॒रस्ता᳚ ॥ बृह॑स्पते॒ परि॑ दीया॒- [  ] \newline

\textbf{Pada Paata} \newline

आ॒शुः । शिशा॑नः । वृ॒ष॒भः । न । यु॒द्ध्मः । घ॒ना॒घ॒नः । क्षोभ॑णः । च॒र॒.ष॒णी॒नाम् ॥ सं॒क्रन्द॑न॒ इति॑ सं - क्रन्द॑नः । अ॒नि॒मि॒ष इत्य॑नि - मि॒षः । ए॒क॒वी॒र इत्ये॑क - वी॒रः । श॒तम् । सेनाः᳚ । अ॒ज॒य॒त् । सा॒कम् । इन्द्रः॑ ॥ स॒क्रंन्द॑ने॒नेति॑ सं - क्रन्द॑नेन । अ॒नि॒मि॒षेणेत्य॑नि - मि॒षेण॑ । जि॒ष्णुना᳚ । यु॒त्का॒रेणेति॑ युत् - का॒रेण॑ । दु॒श्च्य॒व॒नेनेति॑ दुः-च्य॒व॒नेन॑ । धृ॒ष्णुना᳚ ॥ तत् । इन्द्रे॑ण । ज॒य॒त॒ । तत् । स॒ह॒द्ध्वं॒ । युधः॑ । न॒रः॒ । इषु॑हस्ते॒नेतीषु॑ - ह॒स्ते॒न॒ । वृष्णा᳚ ॥ सः । इषु॑हस्तै॒रितीषु॑ - ह॒स्तैः॒ । सः । नि॒ष॒ङ्गिभि॒रिति॑ निष॒ङ्गि - भिः॒ । व॒शी । सꣳस्र॒ष्टेति॒ सं - स्र॒ष्टा॒ । सः । युधः॑ । इन्द्रः॑ । ग॒णेन॑ ॥ सꣳ॒॒सृ॒ष्ट॒जिदिति॑ सꣳसृष्ट - जित् । सो॒म॒पा इति॑ सोम - पाः । बा॒हु॒श॒द्‌र्धीति॑ बाहु - श॒द्‌र्धी । ऊ॒द्‌र्ध्वध॒न्वेत्यू॒द्‌र्ध्व - ध॒न्वा॒ । प्रति॑हिताभि॒रिति॒ प्रति॑ - हि॒ता॒भिः॒ । अस्ता᳚ ॥ बृह॑स्पते । परीति॑ । दी॒य॒ ।  \newline




\markright{ TS 4.6.4.2  \hfill https://www.vedavms.in \hfill}

\section{ TS 4.6.4.2 }

\textbf{TS 4.6.4.2 } \newline
\textbf{Samhita Paata} \newline

रथे॑न रक्षो॒हा ऽमित्राꣳ॑ अप॒ बाध॑मानः । प्र॒भं॒जन्थ् सेनाः᳚ प्रमृ॒णो यु॒धा जय॑न्न॒स्माक॑-मेद्ध्यवि॒ता रथा॑नां ॥ गो॒त्र॒भिदं॑ गो॒विदं॒ ॅवज्र॑बाहुं॒ जय॑न्त॒मज्म॑ प्रमृ॒णन्त॒-मोज॑सा । इ॒मꣳ स॑जाता॒ अनु॑वीर-यद्ध्व॒मिन्द्रꣳ॑ सखा॒योऽनु॒ सꣳ र॑भद्ध्वं ॥ ब॒ल॒वि॒ज्ञा॒यः-स्थवि॑रः॒ प्रवी॑रः॒ सह॑स्वान्. वा॒जी सह॑मान उ॒ग्रः । अ॒भिवी॑रो अ॒भिस॑त्वा सहो॒जा जैत्र॑मिन्द्र॒ रथ॒माति॑ष्ठ गो॒वित् ॥ अ॒भि गो॒त्राणि॒ सह॑सा॒ गाह॑मानोऽदा॒यो- [  ] \newline

\textbf{Pada Paata} \newline

रथे॑न । र॒क्षो॒हेति॑ रक्षः - हा । अ॒मित्रान्॑ । अ॒प॒बाध॑मान॒ इत्य॑प - बाध॑मानः ॥ प्र॒भ॒ञ्जन्निति॑ प्र - भ॒ञ्जन्न् । सेनाः᳚ । प्र॒मृ॒ण इति॑ प्र - मृ॒णः । यु॒धा । जयन्न्॑ । अ॒स्माक᳚म् । ए॒धि॒ । अ॒वि॒ता । रथा॑नाम् ॥ गो॒त्र॒भिद॒मिति॑ गोत्र - भिद᳚म् । गो॒विद॒मिति॑ गो - विद᳚म् । वज्र॑बाहु॒मिति॒ वज्र॑ - बा॒हु॒म् । जय॑न्तम् । अज्म॑ । प्र॒मृ॒णन्त॒मिति॑ प्र - मृ॒णन्त᳚म् । ओज॑सा ॥ इ॒मम् । स॒जा॒ता॒ इति॑ स - जा॒ताः॒ । अन्विति॑ । वी॒र॒य॒द्ध्व॒म् । इन्द्र᳚म् । स॒खा॒यः॒ । अनु॑ । समिति॑ । र॒भ॒द्ध्व॒म् ॥ ब॒ल॒वि॒ज्ञा॒य इति॑ बल - वि॒ज्ञा॒यः । स्थवि॑रः । प्रवी॑र॒ इति॒ प्र - वी॒रः॒ । सह॑स्वान् । वा॒जी । सह॑मानः । उ॒ग्रः ॥ अ॒भिवी॑र॒ इत्य॒भि - वी॒रः॒ । अ॒भिस॒त्वेत्य॒भि - स॒त्वा॒ । स॒हो॒जा इति॑ सहः - जाः । जैत्र᳚म् । इ॒न्द्र॒ । रथ᳚म् । एति॑ । ति॒ष्ठ॒ । गो॒विदिति॑ गो - वित् ॥ अ॒भीति॑ । गो॒त्राणि॑ । सह॑सा । गाह॑मानः । अ॒दा॒यः ।  \newline




\markright{ TS 4.6.4.3  \hfill https://www.vedavms.in \hfill}

\section{ TS 4.6.4.3 }

\textbf{TS 4.6.4.3 } \newline
\textbf{Samhita Paata} \newline

वी॒रः श॒तम॑न्यु॒रिन्द्रः॑ । दु॒श्च्य॒व॒नः पृ॑तना॒षाड॑ यु॒द्ध्यो᳚-स्माकꣳ॒॒ सेना॑ अवतु॒ प्र यु॒थ्सु ॥ इन्द्र॑ आसां-ने॒ता बृह॒स्पति॒ र्दक्षि॑णा य॒ज्ञ्ः पु॒र ए॑तु॒ सोमः॑ । दे॒व॒से॒नाना॑-मभिभञ्जती॒नां जय॑न्तीनां म॒रुतो॑ य॒न्त्वग्रे᳚ ॥ इन्द्र॑स्य॒ वृष्णो॒ वरु॑णस्य॒ राज्ञ्॑ आदि॒त्यानां᳚ म॒रुताꣳ॒॒ शर्द्ध॑ उ॒ग्रं । म॒हाम॑नसां भुवनच्य॒वानां॒ घोषो॑ दे॒वानां॒ जय॑ता॒ मुद॑स्थात् ॥ अ॒स्माक॒-मिन्द्रः॒ समृ॑तेषु ध्व॒जेष्व॒स्माकं॒ ॅया इष॑व॒स्ता ज॑यन्तु । \newline

\textbf{Pada Paata} \newline

वी॒रः । श॒तम॑न्यु॒रिति॑ श॒त - म॒न्युः॒ । इन्द्रः॑ ॥ दु॒श्च्य॒व॒न इति॑ दुः - च्य॒व॒नः । पृ॒त॒ना॒षाट् । अ॒यु॒द्ध्यः । अ॒स्माक᳚म् । सेनाः᳚ । अ॒व॒तु॒ । प्रेति॑ । यु॒थ्स्विति॑ युत् - सु ॥ इन्द्रः॑ । आ॒सा॒म् । ने॒ता । बृह॒स्पतिः॑ । दक्षि॑णा । य॒ज्ञ्ः । पु॒रः । ए॒तु॒ । सोमः॑ ॥ दे॒व॒से॒नाना॒मिति॑ देव - से॒नाना᳚म् । अ॒भि॒भ॒ञ्ज॒ती॒नामित्य॑भि - भ॒ञ्ज॒ती॒नाम् । जय॑न्तीनाम् । म॒रुतः॑ । य॒न्तु॒ । अग्रे᳚ ॥ इन्द्र॑स्य । वृष्णः॑ । वरु॑णस्य । राज्ञ्ः॑ । आ॒दि॒त्याना᳚म् । म॒रुता᳚म् । शर्द्धः॑ । उ॒ग्रम् ॥ म॒हाम॑नसा॒मिति॑ म॒हा - म॒न॒सा॒म् । भु॒व॒न॒च्य॒वाना॒मिति॑ भुवन - च्य॒वाना᳚म् । घोषः॑ । दे॒वाना᳚म् । जय॑ताम् । उदिति॑ । अ॒स्था॒त् ॥ अ॒स्माक᳚म् । इन्द्रः॑ । समृ॑ते॒ष्विति॒ सं - ऋ॒ते॒षु॒ । ध्व॒जेषु॑ । अ॒स्माक᳚म् । याः । इष॑वः । ता: । ज॒य॒न्तु॒ ॥  \newline




\markright{ TS 4.6.4.4  \hfill https://www.vedavms.in \hfill}

\section{ TS 4.6.4.4 }

\textbf{TS 4.6.4.4 } \newline
\textbf{Samhita Paata} \newline

अ॒स्माकं॑ ॅवी॒रा उत्त॑रे भवन्त्व॒स्मानु॑ देवा अवता॒ हवे॑षु ॥ उद्ध॑र्.षय मघव॒न्ना-यु॑धा॒-न्युथ्सत्व॑नां माम॒कानां॒ महाꣳ॑सि । उद्वृ॑त्रहन् वा॒जिनां॒ ॅवाजि॑ना॒न्युद्-रथा॑नां॒ जय॑तामेतु॒ घोषः॑ ॥ उप॒ प्रेत॒ जय॑ता नरःस्थि॒रा वः॑ सन्तु बा॒हवः॑ । इन्द्रो॑ वः॒ शर्म॑ यच्छ त्वना-धृ॒ष्या यथास॑थ ॥ अव॑सृष्टा॒ परा॑ पत॒ शर॑व्ये॒ ब्रह्म॑ सꣳशिता । गच्छा॒मित्रा॒न् प्र- [  ] \newline

\textbf{Pada Paata} \newline

अ॒स्माक᳚म् । वी॒राः । उत्त॑र॒ इत्युत् - त॒रे॒ । भ॒व॒न्तु॒ । अ॒स्मान् । उ॒ । दे॒वाः॒ । अ॒व॒त॒ । हवे॑षु ॥ उदिति॑ । ह॒र्॒.ष॒य॒ । म॒घ॒व॒न्निति॑ मघ - व॒न्न् । आयु॑धानि । उदिति॑ । सत्व॑नाम् । मा॒म॒काना᳚म् । महाꣳ॑सि ॥ उदिति॑ । वृ॒त्र॒ह॒न्निति॑ वृत्र-ह॒न्न् । वा॒जिना᳚म् । वाजि॑नानि । उदिति॑ । रथा॑नाम् । जय॑ताम् । ए॒तु॒ । घोषः॑ ॥ उप॑ । प्रेति॑ । इ॒त॒ । जय॑त । न॒रः॒ । स्थि॒राः । वः॒ । स॒न्तु॒ । बा॒हवः॑ ॥ इन्द्रः॑ । वः॒ । शर्म॑ । य॒च्छ॒तु॒ । अ॒ना॒धृ॒ष्या इत्य॑ना - धृ॒ष्याः । यथा᳚ । अस॑थ ॥ अव॑सृ॒ष्टेत्यव॑ - सृ॒ष्टा॒ । परेति॑ । प॒त॒ । शर॑व्ये । ब्रह्म॑सꣳशि॒तेति॒ ब्रह्म॑-सꣳ॒॒शि॒ता॒ ॥ गच्छ॑ । अ॒मित्रान्॑ । प्रेति॑ ।  \newline




\markright{ TS 4.6.4.5  \hfill https://www.vedavms.in \hfill}

\section{ TS 4.6.4.5 }

\textbf{TS 4.6.4.5 } \newline
\textbf{Samhita Paata} \newline

वि॑श॒ मैषां॒ कञ्च॒नोच्छि॑षः ॥ मर्मा॑णि ते॒ वर्म॑भिश्छादयामि॒ सोम॑स्त्वा॒ राजा॒ ऽमृते॑ना॒भि-व॑स्तां । उ॒रोर्वरी॑यो॒ वरि॑वस्ते अस्तु॒ जय॑न्तं॒ त्वामनु॑ मदन्तु दे॒वाः ॥ यत्र॑ बा॒णाः सं॒पत॑न्ति कुमा॒रा वि॑शि॒खा इ॑व । इन्द्रो॑ न॒स्तत्र॑ वृत्र॒हा वि॑श्वा॒हा शर्म॑ यच्छतु ॥ \newline

\textbf{Pada Paata} \newline

वि॒श॒ । मा । ए॒षा॒म् । कम् । च॒न । उदिति॑ । शि॒षः॒ ॥ मर्मा॑णि । ते॒ । वर्म॑भि॒रिति॒ वर्म॑ -भिः॒ । छा॒द॒या॒मि॒ । सोमः॑ । त्वा॒ । राजा᳚ । अ॒मृते॑न । अ॒भीति॑ । व॒स्ता॒म् ॥ उ॒रोः । वरी॑यः । वरि॑वः । ते॒ । अ॒स्तु॒ । जय॑न्तम् । त्वाम् । अन्विति॑ । म॒द॒न्तु॒ । दे॒वाः ॥ यत्र॑ । बा॒णाः । स॒पंत॒न्तीति॑ सं - पत॑न्ति । कु॒मा॒राः । वि॒शि॒खा इति॑ वि - शि॒खाः । इ॒व॒ ॥ इन्द्रः॑ । नः॒ । तत्र॑ । वृ॒त्र॒हेति॑ वृत्र - हा । वि॒श्वा॒हेति॑ विश्वा - हा । शर्म॑ । य॒च्छ॒तु॒ ॥  \newline




\markright{ TS 4.6.5.1  \hfill https://www.vedavms.in \hfill}

\section{ TS 4.6.5.1 }

\textbf{TS 4.6.5.1 } \newline
\textbf{Samhita Paata} \newline

प्राची॒मनु॑ प्र॒दिशं॒ प्रेहि॑ वि॒द्वान॒ग्नेर॑ग्ने पु॒रो अ॑ग्निर्भवे॒ह । विश्वा॒ आशा॒ दीद्या॑नो॒ वि भा॒ह्यूर्जं॑ नो धेहि द्वि॒पदे॒ चतु॑ष्पदे ॥ क्रम॑द्ध्वम॒ग्निना॒ नाक॒मुख्यꣳ॒॒ हस्ते॑षु॒ बिभ्र॑तः । दि॒वः पृ॒ष्ठꣳ सुव॑र्ग॒त्वा मि॒श्रा दे॒वेभि॑राद्ध्वं ॥ पृ॒थि॒व्या अ॒हमुद॒न्तरि॑क्ष॒-माऽरु॑ह-म॒न्तरि॑क्षा॒द् दिव॒माऽरु॑हं ।दि॒वो नाक॑स्य पृ॒ष्ठाथ् सुव॒र्ज्योति॑रगा- [  ] \newline

\textbf{Pada Paata} \newline

प्राची᳚म् । अन्विति॑ । प्र॒दिश॒मिति॑ प्र-दिश᳚म् । प्रेति॑ । इ॒हि॒ । वि॒द्वान् । अ॒ग्नेः । अ॒ग्ने॒ । पु॒रो अ॑ग्नि॒रिति॑ पु॒रः - अ॒ग्निः॒ । भ॒व॒ । इ॒ह ॥ विश्वाः᳚ । आशाः᳚ । दीद्या॑नः । वीति॑ । भा॒हि॒ । ऊर्ज᳚म् । नः॒ । धे॒हि॒ । द्वि॒पद॒ इति॑ द्वि - पदे᳚ । चतु॑ष्पद॒ इति॒ चतुः॑ - प॒दे॒ ॥ क्रम॑द्ध्वम् । अ॒ग्निना᳚ । नाक᳚म् । उख्य᳚म् । हस्ते॑षु । बिभ्र॑तः ॥ दि॒वः । पृ॒ष्ठम् । सुवः॑ । ग॒त्वा । मि॒श्राः । दे॒वेभिः॑ । आ॒द्ध्व॒म् ॥ पृ॒थि॒व्याः । अ॒हम् । उदिति॑ । अ॒न्तरि॑क्षम् । एति॑ । अ॒रु॒ह॒म् । अ॒न्तरि॑क्षात् । दिव᳚म् । एति॑ । अ॒रु॒ह॒म् ॥ दि॒वः । नाक॑स्य । पृ॒ष्ठात् । सुवः॑ । ज्योतिः॑ । अ॒गा॒म् ।  \newline




\markright{ TS 4.6.5.2  \hfill https://www.vedavms.in \hfill}

\section{ TS 4.6.5.2 }

\textbf{TS 4.6.5.2 } \newline
\textbf{Samhita Paata} \newline

म॒हं ॥ सुव॒र्यन्तो॒ नापे᳚क्षन्त॒ आ द्याꣳ रो॑हन्ति॒ रोद॑सी । य॒ज्ञ्ं ॅये वि॒श्वतो॑धारꣳ॒॒ सुवि॑द्वाꣳसो वितेनि॒रे ॥ अग्ने॒ प्रेहि॑ प्रथ॒मो दे॑वय॒तां चक्षु॑र्दे॒वाना॑मु॒त मर्त्या॑नां । इय॑क्षमाणा॒ भृगु॑भिः स॒जोषाः॒ सुव॑र्यन्तु॒ यज॑मानाः स्व॒स्ति ॥ नक्तो॒षासा॒ सम॑नसा॒ विरू॑पे धा॒पये॑ते॒ शिशु॒मेकꣳ॑ समी॒ची । द्यावा॒ क्षामा॑ रु॒क्मो अ॒न्तर्वि भा॑ति दे॒वा अ॒ग्निं धा॑रयन् द्रविणो॒दाः ॥ अग्ने॑ सहस्राक्ष - [  ] \newline

\textbf{Pada Paata} \newline

अ॒हम् ॥ सुवः॑ । यन्तः॑ । न । अपेति॑ । ई॒क्ष॒न्ते॒ । एति॑ । द्याम् । रो॒ह॒न्ति॒ । रोद॑सी॒ इति॑ ॥ य॒ज्ञ्म् । ये । वि॒श्वतो॑धार॒मिति॑ वि॒श्वतः॑ - धा॒र॒म् । सुवि॑द्वाꣳस॒ इति॒ सु - वि॒द्वाꣳ॒॒सः॒ । वि॒ते॒नि॒र इति॑ वि - ते॒नि॒रे ॥ अग्ने᳚ । प्रेति॑ । इ॒हि॒ । प्र॒थ॒मः । दे॒व॒य॒तामिति॑ देव - य॒ताम् । चक्षुः॑ । दे॒वाना᳚म् । उ॒त । मर्त्या॑नाम् ॥ इय॑क्षमाणाः । भृगु॑भि॒रिति॒ भृगु॑ - भिः॒ । स॒जोषा॒ इति॑ स-जोषाः᳚ । सुवः॑ । य॒न्तु॒ । यज॑मानाः । स्व॒स्ति ॥ नक्तो॒षासा᳚ । सम॑न॒सेति॒ स - म॒न॒सा॒ । विरू॑पे॒ इति॒ वि - रू॒पे॒ । धा॒पये॑ते॒ इति॑ । शिशु᳚म् । एक᳚म् । स॒मी॒ची इति॑ ॥ द्यावा᳚ । क्षाम॑ । रु॒क्मः । अ॒न्तः । वीति॑ । भा॒ति॒ । दे॒वाः । अ॒ग्निम् । धा॒र॒य॒न्न् । द्र॒वि॒णो॒दा इति॑ द्रविणः - दाः ॥ अग्ने᳚ । स॒ह॒स्रा॒क्षेति॑ सहस्र - अ॒क्ष॒ ।  \newline




\markright{ TS 4.6.5.3  \hfill https://www.vedavms.in \hfill}

\section{ TS 4.6.5.3 }

\textbf{TS 4.6.5.3 } \newline
\textbf{Samhita Paata} \newline

शतमूर्द्धञ्छ॒तं ते᳚ प्रा॒णाः स॒हस्र॑मपा॒नाः । त्वꣳ सा॑ह॒स्रस्य॑ रा॒य ई॑शिषे॒ तस्मै॑ ते विधेम॒ वाजा॑य॒ स्वाहा᳚ ॥ सु॒प॒र्णो॑ऽसि ग॒रुत्मा᳚न् पृथि॒व्याꣳ सी॑द पृ॒ष्ठे पृ॑थि॒व्याः सी॑द भा॒साऽ*न्तरि॑क्ष॒मा पृ॑ण॒ ज्योति॑षा॒ दिव॒मुत्त॑भान॒ तेज॑सा॒ दिश॒ उद् दृꣳ॑ह ॥ आ॒जुह्वा॑नः सु॒प्रती॑कः पु॒रस्ता॒दग्ने॒ स्वां ॅयोनि॒मा सी॑द सा॒द्ध्या । अ॒स्मिन्थ्-स॒धस्थे॒ अद्ध्युत्त॑रस्मि॒न् विश्वे॑ देवा॒ - [  ] \newline

\textbf{Pada Paata} \newline

श॒त॒मू॒द्‌र्ध॒न्निति॑ शत - मू॒र्ध॒न्न् । श॒तम् । ते॒ । प्रा॒णा इति॑ प्र-अ॒नाः । स॒हस्र᳚म् । अ॒पा॒न इत्य॑प - अ॒नाः ॥ त्वम् । सा॒ह॒स्रस्य॑ । रा॒यः । ई॒शि॒षे॒ । तस्मै᳚ । ते॒ । वि॒धे॒म॒ । वाजा॑य । स्वाहा᳚ ॥ सु॒प॒र्ण॒ इति॑ सु-प॒र्णः । अ॒सि॒ । ग॒रुत्मान्॑ । पृ॒थि॒व्याम् । सी॒द॒ । पृ॒ष्ठे । पृ॒थि॒व्याः । सी॒द॒ । भा॒सा । अ॒न्तरि॑क्षम् । एति॑ । पृ॒ण॒ । ज्योति॑षा । दिव᳚म् । उदिति॑ । स्त॒भा॒न॒ । तेज॑सा । दिशः॑ । उदिति॑ । दृꣳ॒॒ह॒ ॥ आ॒जुह्वा॑न॒ इत्या᳚ - जुह्वा॑नः । सु॒प्रती॑क॒ इति॑ सु - प्रती॑कः । पु॒रस्ता᳚त् । अग्ने᳚ । स्वाम् । योनि᳚म् । एति॑ । सी॒द॒ । सा॒द्ध्या ॥ अ॒स्मिन्न् । स॒धस्थ॒ इति॑ स॒ध - स्थे॒ । अधीति॑ । उत्त॑रस्मि॒न्नित्युत् - त॒र॒स्मि॒न्न् । विश्वे᳚ । दे॒वाः॒ ।  \newline




\markright{ TS 4.6.5.4  \hfill https://www.vedavms.in \hfill}

\section{ TS 4.6.5.4 }

\textbf{TS 4.6.5.4 } \newline
\textbf{Samhita Paata} \newline

यज॑मानश्च सीदत ॥ प्रेद्धो॑ अग्ने दीदिहि पु॒रो नोऽज॑स्रया सू॒र्म्या॑ यविष्ठ । त्वाꣳ शश्व॑न्त॒ उप॑ यन्ति॒ वाजाः᳚ ॥ वि॒धेम॑ ते पर॒मे जन्म॑न्नग्ने वि॒धेम॒ स्तोमै॒रव॑रे स॒धस्थे᳚ । यस्मा॒द्-योने॑रु॒दारि॑था॒ यजे॒ तं प्रत्वे ह॒वीꣳषि॑ जुहुरे॒ समि॑द्धे ॥ ताꣳ स॑वि॒तुर्वरे᳚ण्यस्य चि॒त्रामाऽहं ॅवृ॑णे सुम॒तिं ॅवि॒श्वज॑न्यां । याम॑स्य॒ कण्वो॒ अदु॑ह॒त् प्रपी॑नाꣳ स॒हस्र॑धारां॒ - [  ] \newline

\textbf{Pada Paata} \newline

यज॑मानः । च॒ । सी॒द॒त॒ ॥ प्रेद्ध॒ इति॒ प्र - इ॒द्धः॒ । अ॒ग्ने॒ । दी॒दि॒हि॒ । पु॒रः । नः॒ । अज॑स्रया । सू॒र्म्या᳚ । य॒वि॒ष्ठ॒ ॥ त्वा॒म् । शश्व॑न्तः । उपेति॑ । य॒न्ति॒ । वाजाः᳚ ॥ वि॒धेम॑ । ते॒ । प॒र॒मे । जन्मन्न्॑ । अ॒ग्ने॒ । वि॒धेम॑ । स्तोमैः᳚ । अव॑रे । स॒धस्थ॒ इति॑ स॒ध - स्थे॒ ॥ यस्मा᳚त् । योनेः᳚ । उ॒दारि॒थेत्यु॑त् - आरि॑थ । यजे᳚ । तम् । प्रेति॑ । त्वे इति॑ । ह॒वीꣳषि॑ । जु॒हु॒रे॒ । समि॑द्ध॒ इति॒ सं - इ॒द्धे॒ ॥ ताम् । स॒वि॒तुः । वरे᳚ण्यस्य । चि॒त्राम् । एति॑ । अ॒हम् । वृ॒णे॒ । सु॒म॒तिमिति॑ सु - म॒तिम् । वि॒श्वज॑न्या॒मिति॑ वि॒श्व - ज॒न्या॒म् ॥ याम् । अ॒स्य॒ । कण्वः॑ । अदु॑हत् । प्रपी॑ना॒मिति॒ प्र-पी॒ना॒म् । स॒हस्र॑धारा॒मिति॑ स॒हस्र॑ - धा॒रा॒म् ।  \newline




\markright{ TS 4.6.5.5  \hfill https://www.vedavms.in \hfill}

\section{ TS 4.6.5.5 }

\textbf{TS 4.6.5.5 } \newline
\textbf{Samhita Paata} \newline

पय॑सा म॒हीं गां ॥ स॒प्त ते॑ अग्ने स॒मिधः॑ स॒प्त जि॒ह्वाः स॒प्तर्.ष॑यः स॒प्त धाम॑ प्रि॒याणि॑ । स॒प्त होत्राः᳚ सप्त॒धा त्वा॑ यजन्ति स॒प्त योनी॒रा पृ॑णस्वा घृ॒तेन॑ ॥ ई॒दृङ् चा᳚न्या॒दृङ् चै॑ता॒दृङ्च॑ प्रति॒दृङ् च॑ मि॒तश्च॒ संमि॑तश्च॒ सभ॑राः ॥ शु॒क्रज्यो॑तिश्च चि॒त्रज्यो॑तिश्च स॒त्यज्यो॑तिश्च॒ ज्योति॑ष्माꣳश्च स॒त्यश्च॑र्त॒पाश्चात्यꣳ॑हाः ॥ \newline

\textbf{Pada Paata} \newline

पय॑सा । म॒हीम् । गाम् ॥ स॒प्त । ते॒ । अ॒ग्ने॒ । स॒मिध॒ इति॑ सम् - इधः॑ । स॒प्त । जि॒ह्वाः । स॒प्त । ऋष॑यः । स॒प्त । धाम॑ । प्रि॒याणि॑ ॥ स॒प्त । होत्राः᳚ । स॒प्त॒धेति॑ सप्त - धा । त्वा॒ । य॒ज॒न्ति॒ । स॒प्त । योनीः᳚ । एति॑ । पृ॒ण॒स्व॒ । घृ॒तेन॑ ॥ ई॒दृङ् । च॒ । अ॒न्या॒दृङ् । च॒ । ए॒ता॒दृङ् । च॒ । प्र॒ति॒दृङिति॑ प्रति - दृङ् । च॒ । मि॒तः । च॒ । संमि॑त॒ इति॒ सं - मि॒तः॒ । च॒ । सभ॑रा॒ इति॒ स - भ॒राः॒ ॥ शु॒क्रज्यो॑ति॒रिति॑ शु॒क्र-ज्यो॒तिः॒ । च॒ । चि॒त्रज्यो॑ति॒रिति॑ चि॒त्र-ज्यो॒तिः॒ । च॒ । स॒त्यज्यो॑ति॒रिति॑ स॒त्य - ज्यो॒तिः॒ । च॒ । ज्योति॑ष्मान् । च॒ । स॒त्यः । च॒ । ऋ॒त॒पा इत्यृ॑त-पाः । च॒ । अत्यꣳ॑हा॒ इत्यति॑ - अꣳ॒॒हाः॒ ॥  \newline




\markright{ TS 4.6.5.6  \hfill https://www.vedavms.in \hfill}

\section{ TS 4.6.5.6 }

\textbf{TS 4.6.5.6 } \newline
\textbf{Samhita Paata} \newline

ऋ॒त॒जिच्च॑ सत्य॒जिच्च॑ सेन॒जिच्च॑ सु॒षेण॒श्चान्त्य॑मित्रश्च दू॒रेअ॑मित्रश्च ग॒णः ॥ ऋ॒तश्च॑ स॒त्यश्च॑ ध्रु॒वश्च॑ ध॒रुण॑श्च ध॒र्ता च॑ विध॒र्ता च॑ विधार॒यः ॥ ई॒दृक्षा॑स एता॒दृक्षा॑स ऊ॒ षुणः॑ स॒दृक्षा॑सः॒ प्रति॑सदृक्षास॒ एत॑न । मि॒तास॑श्च॒ संमि॑तासश्च न ऊ॒तये॒ सभ॑रसो मरुतो य॒ज्ञे अ॒स्मिन्निन्द्रं॒ दैवी॒र्विशो॑ म॒रुतोऽनु॑वर्त्मानो॒ ( ) यथेन्द्रं॒ दैवी॒र्विशो॑ म॒रुतोऽनु॑वर्त्मान ए॒वमि॒मं ॅयज॑मानं॒ दैवी᳚श्च॒ विशो॒ मानु॑षी॒श्चानु॑वर्त्मानो भवन्तु ॥ \newline

\textbf{Pada Paata} \newline

ऋ॒त॒जिदित्यृ॑त - जित् । च॒ । स॒त्य॒जिदिति॑ सत्य - जित् । च॒ । से॒न॒जिदिति॑ सेन - जित् । च॒ । सु॒षेण॒ इति॑ सु - सेनः॑ । च॒ । अन्त्य॑मित्र॒ इत्यन्ति॑-अ॒मि॒त्रः॒ । च॒ । दू॒रे अ॑मित्र॒ इति॑ दू॒रे - अ॒मि॒त्रः॒ । च॒ । ग॒णः ॥ ऋ॒तः । च॒ । स॒त्यः । च॒ । ध्रु॒वः । च॒ । ध॒रुणः॑ । च॒ । ध॒र्ता । च॒ । वि॒ध॒र्तेति॑ वि - ध॒र्ता । च॒ । वि॒धा॒र॒य इति॑ वि-धा॒र॒यः ॥ ई॒दृक्षा॑सः । ए॒ता॒दृक्षा॑सः । उ॒ । स्विति॑ । नः॒ । स॒दृक्षा॑सः । प्रति॑सदृक्षास॒ इति॒ प्रति॑ - स॒दृ॒क्षा॒सः॒ । एति॑ । इ॒त॒न॒ ॥ मि॒तासः॑ । च॒ । संमि॑तास॒ इति॒ सं - मि॒ता॒सः॒ । च॒ । नः॒ । ऊ॒तये᳚ । सभ॑रस॒ इति॒ स - भ॒र॒सः॒ । म॒रु॒तः॒ । य॒ज्ञे । अ॒स्मिन्न् । इन्द्र᳚म् । दैवीः᳚ । विशः॑ । म॒रुतः॑ । अनु॑वर्त्मान॒ इत्यनु॑-व॒र्त्मा॒नः॒ ( ) । यथा᳚ । इन्द्र᳚म् । दैवीः᳚ । विशः॑ । म॒रुतः॑ । अनु॑वर्त्मान॒ इत्यनु॑ - व॒र्त्मा॒नः॒ । ए॒वम् । इ॒मम् । यज॑मानम् । दैवीः᳚ । च॒ । विशः॑ । मानु॑षीः । च॒ । अनु॑वर्त्मान॒ इत्यनु॑ - व॒र्त्मा॒नः॒ । भ॒व॒न्तु॒ ॥  \newline




\markright{ TS 4.6.6.1  \hfill https://www.vedavms.in \hfill}

\section{ TS 4.6.6.1 }

\textbf{TS 4.6.6.1 } \newline
\textbf{Samhita Paata} \newline

जी॒मूत॑स्येव भवति॒ प्रती॑कं॒ ॅयद्व॒र्मी याति॑ स॒मदा॑मु॒पस्थे᳚ । अना॑विद्धया त॒नुवा॑ जय॒ त्वꣳ स त्वा॒ वर्म॑णो महि॒मा पि॑पर्तु ॥ धन्व॑ना॒ गा धन्व॑ना॒ऽऽजिं ज॑येम॒ धन्व॑ना ती॒व्राः स॒मदो॑ जयेम । धनुः॒ शत्रो॑रपका॒मं कृ॑णोति॒ धन्व॑ना॒ सर्वाः᳚ प्र॒दिशो॑ जयेम ॥ व॒क्ष्यन्ती॒वेदा ग॑नीगन्ति॒ कर्णं॑ प्रि॒यꣳ सखा॑यं परिषस्वजा॒ना । योषे॑व शिङ्क्ते॒ वित॒ताऽधि॒ धन्व॒न् - [  ] \newline

\textbf{Pada Paata} \newline

जी॒मूत॑स्य । इ॒व॒ । भ॒व॒ति॒ । प्रती॑कम् । यत् । व॒र्मी । याति॑ । स॒मदा॒मिति॑ स - मदा᳚म् । उ॒पस्थ॒ इत्यु॒प - स्थे॒ ॥ अना॑विद्ध॒येत्यना᳚ - वि॒द्ध॒या॒ । त॒नुवा᳚ । ज॒य॒ । त्वम् । सः । त्वा॒ । वर्म॑णः । म॒हि॒मा । पि॒प॒र्तु॒ ॥ धन्व॑ना । गाः । धन्व॑ना । आ॒जिम् । ज॒ये॒म॒ । धन्व॑ना । ती॒व्राः । स॒मद॒ इति॑ स - मदः॑ । ज॒ये॒म॒ ॥ धनुः॑ । शत्रोः᳚ । अ॒प॒का॒ममित्य॑प - का॒मम् । कृ॒णो॒ति॒ । धन्व॑ना । सर्वाः᳚ । प्र॒दिश॒ इति॑ प्र - दिशः॑ । ज॒ये॒म॒ ॥ व॒क्ष्यन्ती᳚ । इ॒व॒ । इत् । एति॑ । ग॒नी॒ग॒न्ति॒ । कर्ण᳚म् । प्रि॒यम् । सखा॑यम् । प॒रि॒ष॒स्व॒जा॒नेति॑ परि - स॒स्व॒जा॒ना ॥ योषा᳚ । इ॒व । शि॒ङ्क्ते॒ । वित॒तेति॒ वि - त॒ता॒ । अधीति॑ । धन्वन्न्॑ ।  \newline




\markright{ TS 4.6.6.2  \hfill https://www.vedavms.in \hfill}

\section{ TS 4.6.6.2 }

\textbf{TS 4.6.6.2 } \newline
\textbf{Samhita Paata} \newline

ज्या इ॒यꣳ सम॑ने पा॒रय॑न्ती ॥ ते आ॒चर॑न्ती॒ सम॑नेव॒ योषा॑ मा॒तेव॑ पु॒त्रं बि॑भृतामु॒पस्थे᳚ । अप॒ शत्रून्॑ विद्ध्यताꣳ संॅविदा॒ने आर्त्नी॑ इ॒मे वि॑ष्फु॒रन्ती॑ अ॒मित्रान्॑ ॥ ब॒ह्वी॒नां पि॒ता ब॒हुर॑स्य पु॒त्रश्चि॒श्चा कृ॑णोति॒ सम॑नाऽव॒गत्य॑ । इ॒षु॒धिः सङ्काः॒ पृत॑नाश्च॒ सर्वाः᳚ पृ॒ष्ठे निन॑द्धो जयति॒ प्रसू॑तः ॥ रथे॒ तिष्ठ॑न् नयति वा॒जिनः॑ पु॒रो यत्र॑यत्र का॒मय॑ते सुषार॒थिः । अ॒भीशू॑नां महि॒मानं॑ - [  ] \newline

\textbf{Pada Paata} \newline

ज्या । इ॒यम् । सम॑ने । पा॒रय॑न्ती ॥ ते इति॑ । आ॒चर॑न्ती॒ इत्या᳚ - चर॑न्ती । सम॑ना । इ॒व॒ । योषा᳚ । मा॒ता । इ॒व॒ । पु॒त्रम् । बि॒भृ॒ता॒म् । उ॒पस्थ॒ इत्यु॒प - स्थे॒ ॥ अपेति॑ । शत्रून्॑ । वि॒द्ध्य॒ता॒म् । सं॒ॅवि॒दा॒न इति॑ सं - वि॒दा॒ने । आर्त्नी॒ इति॑ । इ॒मे इति । वि॒ष्फु॒रन्ती॒ इति॑ वि-स्फु॒रन्ती᳚ । अ॒मित्रान्॑ ॥ ब॒ह्वी॒नाम् । पि॒ता । ब॒हुः । अ॒स्य॒ । पु॒त्रः । चि॒श्चा । कृ॒णो॒ति॒ । सम॑ना । अ॒व॒गत्येत्यव॑ - गत्य॑ ॥ इ॒षु॒धिरिती॑षु - धिः । सङ्काः᳚ । पृत॑नाः । च॒ । सर्वाः᳚ । पृ॒ष्ठे । निन॑द्ध॒ इति॒ नि - न॒द्धः॒ । ज॒य॒ति॒ । प्रसू॑त॒ इति॒ प्र - सू॒तः॒ ॥ रथे᳚ । तिष्ठन्न्॑ । न॒य॒ति॒ । वा॒जिनः॑ । पु॒रः । यत्र॑य॒त्रेति॒ यत्र॑-य॒त्र॒ । का॒मय॑ते । सु॒षा॒र॒थिरिति॑ सु - सा॒र॒थिः ॥ अ॒भीशू॑नाम् । म॒हि॒मान᳚म् ।  \newline




\markright{ TS 4.6.6.3  \hfill https://www.vedavms.in \hfill}

\section{ TS 4.6.6.3 }

\textbf{TS 4.6.6.3 } \newline
\textbf{Samhita Paata} \newline

पनायत॒ मनः॑ प॒श्चादनु॑ यच्छन्ति र॒श्मयः॑ ॥ ती॒व्रान् घोषा᳚न् कृण्वते॒ वृष॑पाण॒योऽश्वा॒ रथे॑भिः स॒ह वा॒जय॑न्तः । अ॒व॒क्राम॑न्तः॒ प्रप॑दैर॒मित्रा᳚न् क्षि॒णन्ति॒ शत्रूꣳ॒॒रन॑पव्ययन्तः ॥ र॒थ॒वाह॑नꣳ ह॒विर॑स्य॒ नाम॒ यत्राऽऽ*यु॑धं॒ निहि॑तमस्य॒ वर्म॑ । तत्रा॒ रथ॒मुप॑ श॒ग्मꣳ स॑देम वि॒श्वाहा॑ व॒यꣳ सु॑मन॒स्यमा॑नाः ॥ स्वा॒दु॒षꣳ॒॒ सदः॑ पि॒तरो॑ वयो॒धाः कृ॑च्छ्रे॒श्रितः॒ शक्ती॑वन्तो गभी॒राः । चि॒त्रसे॑ना॒ इषु॑बला॒ अमृ॑द्ध्राः स॒तोवी॑रा उ॒रवो᳚ व्रातसा॒हाः ॥ ब्राह्म॑णासः॒ - [  ] \newline

\textbf{Pada Paata} \newline

प॒ना॒य॒त॒ । मनः॑ । प॒श्चात् । अन्विति॑ । य॒च्छ॒न्ति॒ । र॒श्मयः॑ ॥ ती॒व्रान् । घोषान्॑ । कृ॒ण्व॒ते॒ । वृष॑पाणय॒ इति॒ वृष॑-पा॒ण॒यः॒ । अश्वाः᳚ । रथे॑भिः । स॒ह । वा॒जय॑न्तः ॥ अ॒व॒क्राम॑न्त॒ इत्य॑व - क्राम॑न्तः । प्रप॑दै॒रिति॒ प्र - प॒दैः॒ । अ॒मित्रान्॑ । क्षि॒णन्ति॑ । शत्रून्॑ । अन॑पव्ययन्त॒ इत्यन॑प - व्य॒य॒न्तः॒ ॥ र॒थ॒वाह॑न॒मिति॑ रथ-वाह॑नम् । ह॒विः । अ॒स्य॒ । नाम॑ । यत्र॑ । आयु॑धम् । निहि॑त॒मिति॒ नि - हि॒त॒म् । अ॒स्य॒ । वर्म॑ ॥ तत्र॑ । रथ᳚म् । उपेति॑ । श॒ग्मम् । स॒दे॒म॒ । वि॒श्वाहेति॑ विश्वा - अहा᳚ । व॒यम् । सु॒म॒न॒स्यमा॑ना॒ इति॑ सु - म॒न॒स्यमा॑नाः ॥ स्वा॒दु॒षꣳ॒॒सद॒ इति॑ स्वादु - सꣳ॒॒सदः॑ । पि॒तरः॑ । व॒यो॒धा इति॑ वयः - धाः । कृ॒च्छ्रे॒श्रित॒ इति॑ कृच्छ्रे - श्रितः॑ । शक्ती॑वन्त॒ इति॒ शक्ति॑ - व॒न्तः॒ । ग॒भी॒राः ॥ चि॒त्रसे॑ना॒ इति॑ चि॒त्र - से॒नाः॒ । इषु॑बला॒ इतीषु॑ - ब॒लाः॒ । अमृ॑द्ध्राः । स॒तोवी॑रा॒ इति॑ स॒तः - वी॒राः॒ । उ॒रवः॑ । व्रा॒त॒सा॒हा इति॑ व्रात - सा॒हाः ॥ ब्राह्म॑णासः ।  \newline




\markright{ TS 4.6.6.4  \hfill https://www.vedavms.in \hfill}

\section{ TS 4.6.6.4 }

\textbf{TS 4.6.6.4 } \newline
\textbf{Samhita Paata} \newline

पित॑रः॒ सोम्या॑सः शि॒वे नो॒ द्यावा॑पृथि॒वी अ॑ने॒हसा᳚ ।पू॒षा नः॑ पातु दुरि॒तादृ॑तावृधो॒ रक्षा॒ माकि॑र्नो अ॒घशꣳ॑स ईशत ॥ सु॒प॒र्णं ॅव॑स्ते मृ॒गो अ॑स्या॒ दन्तो॒ गोभिः॒ सन्न॑द्धा पतति॒ प्रसू॑ता । यत्रा॒ नरः॒ सं च॒ वि च॒ द्रव॑न्ति॒ तत्रा॒स्मभ्य॒मिष॑वः॒ शर्म॑ यꣳसन्न् ॥ ऋजी॑ते॒ परि॑ वृङ्ग्धि॒ नोऽश्मा॑ भवतु नस्त॒नूः । सोमो॒ अधि॑ ब्रवीतु॒ नोऽदि॑तिः॒ - [  ] \newline

\textbf{Pada Paata} \newline

पित॑रः । सोम्या॑सः । शि॒वे इति॑ । नः॒ । द्यावा॑पृथि॒वी इति॒ द्यावा᳚ - पृ॒थि॒वी । अ॒ने॒हसा᳚ ॥ पू॒षा । नः॒ । पा॒तु॒ । दु॒रि॒तादिति॑ दुः - इ॒तात् । ऋ॒ता॒वृ॒ध॒ इत्यृ॑त - वृ॒धः॒ । रक्ष॑ । माकिः॑ । नः॒ । अ॒घशꣳ॑स॒ इत्य॒घ - शꣳ॒॒सः॒ । ई॒श॒त॒ ॥ सु॒प॒र्णमिति॑ सु - प॒र्णम् । व॒स्ते॒ । मृ॒गः । अ॒स्याः॒ । दन्तः॑ । गोभिः॑ । सन्न॒द्धेति॒ सं - न॒द्धा॒ । प॒त॒ति॒ । प्रसू॒तेति॒ प्र - सू॒ता॒ ॥ यत्र॑ । नरः॑ । समिति॑ । च॒ । वीति॑ । च॒ । द्रव॑न्ति । तत्र॑ । अ॒स्मभ्य॒मित्य॒स्म - भ्य॒म् । इष॑वः । शर्म॑ । यꣳ॒॒स॒न्न् ॥ ऋजी॑ते । परीति॑ । वृ॒ङ्ग्धि॒ । नः॒ । अश्मा᳚ । भ॒व॒तु॒ । नः॒ । त॒नूः ॥ सोमः॑ । अधीति॑ । ब्र॒वी॒तु॒ । नः॒ । अदि॑तिः ।  \newline




\markright{ TS 4.6.6.5  \hfill https://www.vedavms.in \hfill}

\section{ TS 4.6.6.5 }

\textbf{TS 4.6.6.5 } \newline
\textbf{Samhita Paata} \newline

शर्म॑ यच्छतु ॥ आ ज॑ङ्घन्ति॒ सान्वे॑षां ज॒घनाꣳ॒॒ उप॑ जिघ्नते । अश्वा॑जनि॒ प्रचे॑त॒सोऽश्वा᳚न्थ् स॒मथ्सु॑ चोदय ॥ अहि॑रिव भो॒गैः पर्ये॑ति बा॒हुं ज्याया॑ हे॒तिं प॑रि॒बाध॑मानः । ह॒स्त॒घ्नो विश्वा॑ व॒युना॑नि वि॒द्वान् पुमा॒न् पुमाꣳ॑सं॒ परि॑ पातु वि॒श्वतः॑ ॥ वन॑स्पते वी॒ड्व॑ङ्गो॒ हि भू॒या अ॒स्मथ् स॑खा प्र॒तर॑णः सु॒वीरः॑ । गोभिः॒ सन्न॑द्धो असि वी॒डय॑स्वाऽऽस्था॒ता ते॑ जयतु॒ जेत्वा॑नि ॥ दि॒वः पृ॑थि॒व्याः पर्यो- [  ] \newline

\textbf{Pada Paata} \newline

शर्म॑ । य॒च्छ॒तु॒ ॥ एति॑ । ज॒ङ्घ॒न्ति॒ । सानु॑ । ए॒षा॒म् । ज॒घनान्॑ । उपेति॑ । जि॒घ्न॒ते॒ ॥ अश्वा॑ज॒नीत्यश्व॑ - अ॒ज॒नि॒ । प्रचे॑तस॒ इति॒ प्र - चे॒त॒सः॒ । अश्वान्॑ । स॒मथ्स्विति॑ स॒मत् - सु॒ । चो॒द॒य॒ ॥ अहिः॑ । इ॒व॒ । भो॒गैः । परीति॑ । ए॒ति॒ । बा॒हुम् । ज्यायाः᳚ । हे॒तिम् । प॒रि॒बाध॑मान॒ इति॑ परि - बाध॑मानः ॥ ह॒स्त॒घ्न इति॑ हस्त - घ्नः । विश्वा᳚ । व॒युना॑नि । वि॒द्वान् । पुमान्॑ । पुमाꣳ॑सम् । परीति॑ । पा॒तु॒ । वि॒श्वतः॑ ॥ वन॑स्पते । वी॒ड्व॑ङ्ग॒ इति॑ वी॒डु - अ॒ङ्गः॒ । हि । भू॒याः । अ॒स्मथ्स॒खेत्य॒स्मत् - स॒खा॒ । प्र॒तर॑ण॒ इति॑ प्र - तर॑णः । सु॒वीर॒ इति॑ सु - वीरः॑ ॥ गोभिः॑ । सन्न॑द्ध॒ इति॒ सं - न॒द्धः॒ । अ॒सि॒ । वी॒डय॑स्व । आ॒स्था॒तेत्या᳚ - स्था॒ता । ते॒ । ज॒य॒तु॒ । जेत्वा॑नि ॥ दि॒वः । पृ॒थि॒व्याः । परीति॑ ।  \newline




\markright{ TS 4.6.6.6  \hfill https://www.vedavms.in \hfill}

\section{ TS 4.6.6.6 }

\textbf{TS 4.6.6.6 } \newline
\textbf{Samhita Paata} \newline

-ज॒ उद्-भृ॑तं॒ ॅवन॒स्पति॑भ्यः॒ पर्याभृ॑तꣳ॒॒ सहः॑ । अ॒पामो॒ज्मानं॒ परि॒ गोभि॒रावृ॑त॒मिन्द्र॑स्य॒ वज्रꣳ॑ ह॒विषा॒ रथं॑ ॅयज ॥ इन्द्र॑स्य॒ वज्रो॑ म॒रुता॒मनी॑कं मि॒त्रस्य॒ गर्भो॒ वरु॑णस्य॒ नाभिः॑ । सेमां नो॑ ह॒व्यदा॑तिं जुषा॒णो देव॑ रथ॒ प्रति॑ ह॒व्या गृ॑भाय ॥ उप॑ श्वासय पृथि॒वीमु॒त द्यां पु॑रु॒त्रा ते॑ मनुतां॒ ॅविष्ठि॑तं॒ जग॑त् । स दु॑न्दुभे स॒जूरिन्द्रे॑ण दे॒वैर्दू॒रा- [  ] \newline

\textbf{Pada Paata} \newline

ओजः॑ । उद्भृ॑त॒मित्युत् - भृ॒त॒म् । वन॒स्पति॑भ्य॒ इति॒ वन॒स्पति॑-भ्यः॒ । परीति॑ । आभृ॑त॒मित्या - भृ॒त॒म् । सहः॑ ॥ अ॒पाम् । ओ॒ज्मान᳚म् । परीति॑ । गोभिः॑ । आवृ॑त॒मित्या-वृ॒त॒म् । इन्द्र॑स्य । वज्र᳚म् । ह॒विषा᳚ । रथ᳚म् । य॒ज॒ ॥ इन्द्र॑स्य । वज्रः॑ । म॒रुता᳚म् । अनी॑कम् । मि॒त्रस्य॑ । गर्भः॑ । वरु॑णस्य । नाभिः॑ ॥ सः । इ॒माम् । नः॒ । ह॒व्यदा॑ति॒मिति॑ ह॒व्य - दा॒ति॒म् । जु॒षा॒णः । देव॑ । र॒थ॒ । प्रतीति॑ । ह॒व्या । गृ॒भा॒य॒ ॥ उपेति॑ । श्वा॒स॒य॒ । पृ॒थि॒वीम् । उ॒त । द्याम् । पु॒रु॒त्रेति॑ पुरु - त्रा । ते॒ । म॒नु॒ता॒म् । विष्ठि॑त॒मिति॒ वि - स्थि॒त॒म् । जग॑त् ॥ सः । दु॒न्दु॒भे॒ । स॒जूरिति॑ स - जूः । इन्द्रे॑ण । दे॒वैः । दू॒रात् ।  \newline




\markright{ TS 4.6.6.7  \hfill https://www.vedavms.in \hfill}

\section{ TS 4.6.6.7 }

\textbf{TS 4.6.6.7 } \newline
\textbf{Samhita Paata} \newline

द्दवी॑यो॒ अप॑सेध॒ शत्रून्॑ ॥ आ क्र॑न्दय॒ बल॒मोजो॑ न॒ आ धा॒ निष्ट॑निहि दुरि॒ता बाध॑मानः । अप॑ प्रोथ दुन्दुभे दु॒च्छुनाꣳ॑ इ॒त इन्द्र॑स्य मु॒ष्टिर॑सि वी॒डय॑स्व ॥आऽमूर॑ज प्र॒त्याव॑र्तये॒माः के॑तु॒मद् दु॑न्दु॒भि र्वा॑वदीति । समश्व॑पर्णा॒श्चर॑न्ति नो॒ नरो॒ऽस्माक॑मिन्द्र र॒थिनो॑ जयन्तु ॥ \newline

\textbf{Pada Paata} \newline

दवी॑यः । अपेति॑ । से॒ध॒ । शत्रून्॑ ॥ एति॑ । क्र॒न्द॒य॒ । बल᳚म् । ओजः॑ । नः॒ । एति॑ । धाः॒ । निरिति॑ । स्थ॒नि॒हि॒ । दु॒रि॒तेति॑ दुः - इ॒ता । बाध॑मानः ॥ अपेति॑ । प्रो॒थ॒ । दु॒न्दु॒भे॒ । दु॒च्छुनान्॑ । इ॒तः । इन्द्र॑स्य । मु॒ष्टिः । अ॒सि॒ । वी॒डय॑स्व ॥ एति॑ । अ॒मूः । अ॒ज॒ । प्र॒त्याव॑र्त॒येति॑ प्रति - आव॑र्तय । इ॒माः । के॒तु॒मदिति॑ केतु - मत् । दु॒न्दु॒भिः । वा॒व॒दी॒ति॒ ॥ समिति॑ । अश्व॑पर्णा॒ इत्यश्व॑-प॒र्णाः॒ । चर॑न्ति । नः॒ । नरः॑ । अ॒स्माक᳚म् । इ॒न्द्र॒ । र॒थिनः॑ । ज॒य॒न्तु॒ ॥  \newline




\markright{ TS 4.6.7.1  \hfill https://www.vedavms.in \hfill}

\section{ TS 4.6.7.1 }

\textbf{TS 4.6.7.1 } \newline
\textbf{Samhita Paata} \newline

यदक्र॑न्दः प्रथ॒मं जाय॑मान उ॒द्यन्थ् स॑मु॒द्रादु॒त वा॒ पुरी॑षात् । श्ये॒नस्य॑ प॒क्षा ह॑रि॒णस्य॑ बा॒हू उ॑प॒स्तुत्यं॒ महि॑ जा॒तं ते॑ अर्वन्न् ॥ य॒मेन॑ द॒त्तं त्रि॒त ए॑नमायुन॒गिन्द्र॑ एणं प्रथ॒मो अद्ध्य॑तिष्ठत् । ग॒न्ध॒र्वो अ॑स्य रश॒नाम॑-गृभ्णा॒थ् सूरा॒दश्वं॑ ॅवसवो॒ निर॑तष्ट ॥ असि॑ य॒मो अस्या॑दि॒त्यो अ॑र्व॒न्नसि॑ त्रि॒तो गुह्ये॑न व्र॒तेन॑ । असि॒ सोमे॑न स॒मया॒ विपृ॑क्त - [  ] \newline

\textbf{Pada Paata} \newline

यत् । अक्र॑न्दः । प्र॒थ॒मम् । जाय॑मानः । उ॒द्यन्नित्यु॑त्-यन्न् । स॒मु॒द्रात् । उ॒त । वा॒ । पुरी॑षात् ॥ श्ये॒नस्य॑ । प॒क्षा । ह॒रि॒णस्य॑ । बा॒हू इति॑ । उ॒प॒स्तुत्य॒मित्यु॑प - स्तुत्य᳚म् । महि॑ । जा॒तम् । ते॒ । अ॒र्व॒न्न् ॥ य॒मेन॑ । द॒त्तम् । त्रि॒तः । ए॒न॒म् । आ॒यु॒न॒क् । इन्द्रः॑ । ए॒न॒म् । प्र॒थ॒मः । अधीति॑ । अ॒ति॒ष्ठ॒त् ॥ ग॒न्ध॒र्वः । अ॒स्य॒ । र॒श॒नाम् । अ॒गृ॒भ्णा॒त् । सूरा᳚त् । अश्व᳚म् । व॒स॒वः॒ । निरिति॑ । अ॒त॒ष्ट॒ ॥ असि॑ । य॒मः । असि॑ । आ॒दि॒त्यः । अ॒र्व॒न्न् । असि॑ । त्रि॒तः । गुह्ये॑न । व्र॒तेन॑ ॥ असि॑ । सोमे॑न । स॒मया᳚ । विपृ॑क्त॒ इति॒ वि - पृ॒क्तः॒ ।  \newline




\markright{ TS 4.6.7.2  \hfill https://www.vedavms.in \hfill}

\section{ TS 4.6.7.2 }

\textbf{TS 4.6.7.2 } \newline
\textbf{Samhita Paata} \newline

आ॒हुस्ते॒ त्रीणि॑ दि॒वि बन्ध॑नानि ॥ त्रीणि॑ त आहुर्दि॒वि बन्ध॑नानि॒ त्रीण्य॒फ्सु त्रीण्य॒न्तः स॑मु॒द्रे । उ॒तेव॑ मे॒ वरु॑णश्छन्थ् स्यर्व॒न्॒. यत्रा॑ त आ॒हुः प॑र॒मं ज॒नित्रं᳚ ॥ इ॒मा ते॑ वाजिन्नव॒मार्ज॑नानी॒मा श॒फानाꣳ॑ सनि॒तुर्नि॒धाना᳚ । अत्रा॑ ते भ॒द्रा र॑श॒ना अ॑पश्यमृ॒तस्य॒ या अ॑भि॒रक्ष॑न्ति गो॒पाः ॥ आ॒त्मानं॑ ते॒ मन॑सा॒ऽऽराद॑जानाम॒वो दि॒वा - [  ] \newline

\textbf{Pada Paata} \newline

आ॒हुः । ते॒ । त्रीणि॑ । दि॒वि । बन्ध॑नानि ॥ त्रीणि॑ । ते॒ । आ॒हुः॒ । दि॒वि । बन्ध॑नानि । त्रीणि॑ । अ॒फ्स्वित्य॑प् - सु । त्रीणि॑ । अ॒न्तः । स॒मु॒द्रे ॥ उ॒त । इ॒व॒ । मे॒ । वरु॑णः । छ॒न्थ्सि॒ । अ॒र्व॒न्न् । यत्र॑ । ते॒ । आ॒हुः । प॒र॒मम् । ज॒नित्र᳚म् ॥ इ॒मा । ते॒ । वा॒जि॒न्न् । अ॒व॒मार्ज॑ना॒नीत्य॑व - मार्ज॑नानि । इ॒मा । श॒फाना᳚म् । स॒नि॒तुः । नि॒धानेति॑ नि - धाना᳚ ॥ अत्र॑ । ते॒ । भ॒द्राः । र॒श॒नाः । अ॒प॒श्य॒म् । ऋ॒तस्य॑ । याः । अ॒भि॒रक्ष॒न्तीय॑भि - रक्ष॑न्ति । गो॒पा इति॑ गो - पाः ॥ आ॒त्मान᳚म् । ते॒ । मन॑सा । आ॒रात् । अ॒जा॒ना॒म् । अ॒वः । दि॒वा ।  \newline




\markright{ TS 4.6.7.3  \hfill https://www.vedavms.in \hfill}

\section{ TS 4.6.7.3 }

\textbf{TS 4.6.7.3 } \newline
\textbf{Samhita Paata} \newline

प॒तय॑न्तं पत॒ङ्गं । शिरो॑ अपश्यं प॒थिभिः॑ सु॒गेभि॑ररे॒णुभि॒र्जेह॑मानं पत॒त्रि ॥ अत्रा॑ ते रू॒पमु॑त्त॒मम॑पश्यं॒ जिगी॑षमाणमि॒ष आ प॒दे गोः । य॒दा ते॒ मर्तो॒ अनु॒ भोग॒मान॒डादिद् ग्रसि॑ष्ठ॒ ओष॑धीरजीगः ॥ अनु॑ त्वा॒ रथो॒ अनु॒ मर्यो॑ अर्व॒न्ननु॒ गावोऽनु॒ भगः॑ क॒नीनां᳚ । अनु॒ व्राता॑स॒स्तव॑ स॒ख्यमी॑यु॒रनु॑ दे॒वा म॑मिरे वी॒र्यं॑ - [  ] \newline

\textbf{Pada Paata} \newline

प॒तय॑न्तम् । प॒त॒ङ्गम् ॥ शिरः॑ । अ॒प॒श्य॒म् । प॒थिभि॒रिति॑ प॒थि - भिः॒ । सु॒गेभि॒रिति॑ सु - गेभिः॑ । अ॒रे॒णुभि॒रित्य॑रे॒णु - भिः॒ । जेह॑मानम् । प॒त॒त्रि ॥ अत्र॑ । ते॒ । रू॒पम् । उ॒त्त॒ममित्यु॑त् - त॒मम् । अ॒प॒श्य॒म् । जिगी॑षमाणम् । इ॒षः । एति॑ । प॒दे । गोः ॥ य॒दा । ते॒ । मर्तः॑ । अन्विति॑ । भोग᳚म् । आन॑ट् । आत् । इत् । ग्रसि॑ष्ठः । ओष॑धीः । अ॒जी॒गः॒ ॥ अन्विति॑ । त्वा॒ । रथः॑ । अन्विति॑ । मर्यः॑ । अ॒र्व॒न्न् । अन्विति॑ । गावः॑ । अन्विति॑ । भगः॑ । क॒नीना᳚म् ॥ अन्विति॑ । व्राता॑सः । तव॑ । स॒ख्यम् । ई॒युः॒ । अन्विति॑ । दे॒वाः । म॒मि॒रे॒ । वी॒र्य᳚म् ।  \newline




\markright{ TS 4.6.7.4  \hfill https://www.vedavms.in \hfill}

\section{ TS 4.6.7.4 }

\textbf{TS 4.6.7.4 } \newline
\textbf{Samhita Paata} \newline

ते ॥ हिर॑ण्यशृ॒ङ्गोऽयो॑ अस्य॒ पादा॒ मनो॑जवा॒ अव॑र॒ इन्द्र॑ आसीत् । दे॒वा इद॑स्य हवि॒रद्य॑माय॒न्॒. यो अर्व॑न्तं प्रथ॒मो अ॒द्ध्यति॑ष्ठत् ॥ ई॒र्मान्ता॑सः॒ सिलि॑कमद्ध्यमासः॒ सꣳ शूर॑णासो दि॒व्यासो॒ अत्याः᳚ । हꣳ॒॒सा इ॑व श्रेणि॒शो य॑तन्ते॒ यदाक्षि॑षुर्दि॒व्य-मज्म॒मश्वाः᳚ ॥ तव॒ शरी॑रं पतयि॒ष्ण्व॑र्व॒न् तव॑ चि॒त्तं ॅवात॑ इव॒ ध्रजी॑मान् । तव॒ शृङ्गा॑णि॒ विष्ठि॑ता पुरु॒त्राऽर॑ण्येषु॒ जर्भु॑राणा चरन्ति ॥ उप॒ - [  ] \newline

\textbf{Pada Paata} \newline

ते॒ ॥ हिर॑ण्यशृङ्ग॒ इति॒ हिर॑ण्य - शृ॒ङ्गः॒ । अयः॑ । अ॒स्य॒ । पादाः᳚ । मनो॑जवा॒ इति॒ मनः॑ - ज॒वाः॒ । अव॑रः । इन्द्रः॑ । आ॒सी॒त् ॥ दे॒वाः । इत् । अ॒स्य॒ । ह॒वि॒रद्य॒मिति॑ हविः - अद्य᳚म् । आ॒य॒न्न् । यः । अर्व॑न्तम् । प्र॒थ॒मः । अ॒द्ध्यति॑ष्ठ॒दित्य॑धि - अति॑ष्ठत् ॥ ई॒र्मान्ता॑स॒ इती॒र्म-अ॒न्ता॒सः॒ । सिलि॑कमद्ध्यमास॒ इति॒ सिलि॑क-म॒द्ध्य॒मा॒सः॒ । समिति॑ । शूर॑णासः । दि॒व्यासः॑ । अत्याः᳚ ॥ हꣳ॒॒साः । इ॒व॒ । श्रे॒णि॒श इति॑ श्रेणि - शः । य॒त॒न्ते॒ । यत् । आक्षि॑षुः । दि॒व्यम् । अज्म᳚म् । अश्वाः᳚ ॥ तव॑ । शरी॑रम् । प॒त॒यि॒ष्णु । अ॒र्व॒न्न् । तव॑ । चि॒त्तम् । वातः॑ । इ॒व॒ । ध्रजी॑मान् ॥ तव॑ । शृङ्गा॑णि । विष्ठि॒तेति॒ वि - स्थि॒ता॒ । पु॒रु॒त्रेति॑ पुरु - त्रा । अर॑ण्येषु । जर्भु॑राणा । च॒र॒न्ति॒ ॥ उप॑ ।  \newline




\markright{ TS 4.6.7.5  \hfill https://www.vedavms.in \hfill}

\section{ TS 4.6.7.5 }

\textbf{TS 4.6.7.5 } \newline
\textbf{Samhita Paata} \newline

प्रागा॒च्छस॑नं ॅवा॒ज्यर्वा॑ देव॒द्रीचा॒ मन॑सा॒ दीद्ध्या॑नः । अ॒जः पु॒रो नी॑यते॒ नाभि॑र॒स्यानु॑ प॒श्चात् क॒वयो॑ यन्ति रे॒भाः ॥उप॒ प्रागा᳚त् पर॒मं ॅयथ् स॒धस्थ॒मर्वाꣳ॒॒ अच्छा॑ पि॒तरं॑ मा॒तरं॑ च । अ॒द्या दे॒वान् जुष्ट॑तमो॒ हि ग॒म्या अथाऽऽशा᳚स्ते दा॒शुषे॒ वार्या॑णि ॥ \newline

\textbf{Pada Paata} \newline

प्रेति॑ । अ॒गा॒त् । शस॑नम् । वा॒जी । अर्वा᳚ । दे॒व॒द्रीचेति॑ देव-द्रीचा᳚ । मन॑सा । दीद्ध्या॑नः ॥ अ॒जः । पु॒रः । नी॒य॒ते॒ । नाभिः॑ । अ॒स्य॒ । अन्विति॑ । प॒श्चात् । क॒वयः॑ । य॒न्ति॒ । रे॒भाः ॥ उप॑ । प्रेति॑ । अ॒गा॒त् । प॒र॒मम् । यत् । स॒धस्थ॒मिति॑ स॒ध - स्थ॒म् । अर्वान्॑ । अच्छ॑ । पि॒तर᳚म् । मा॒तर᳚म् । च॒ ॥ अ॒द्य । दे॒वान् । जुष्ट॑तम॒ इति॒ जुष्ट॑ - त॒मः॒ । हि । ग॒म्याः । अथ॑ । एति॑ । शा॒स्ते॒ । दा॒शुषे᳚ । वार्या॑णि ॥  \newline




\markright{ TS 4.6.8.1  \hfill https://www.vedavms.in \hfill}

\section{ TS 4.6.8.1 }

\textbf{TS 4.6.8.1 } \newline
\textbf{Samhita Paata} \newline

मा नो॑ मि॒त्रो वरु॑णो अर्य॒माऽऽयुरिन्द्र॑ ऋभु॒क्षा म॒रुतः॒ परि॑ ख्यन्न् । यद्-वा॒जिनो॑ दे॒वजा॑तस्य॒ सप्तेः᳚ प्रव॒क्ष्यामो॑ वि॒दथे॑ वी॒र्या॑णि ॥ यन्नि॒र्णिजा॒ रेक्ण॑सा॒ प्रावृ॑तस्य रा॒तिं गृ॑भी॒तां मु॑ख॒तो नय॑न्ति । सुप्रा॑ङ॒जो मेम्य॑द् वि॒श्वरू॑प इन्द्रापू॒ष्णोः प्रि॒यमप्ये॑ति॒ पाथः॑ ॥ ए॒ष च्छागः॑ पु॒रो अश्वे॑न वा॒जिना॑ पू॒ष्णो भा॒गो नी॑यते वि॒श्वदे᳚व्यः । अ॒भि॒प्रियं॒ ॅयत् पु॑रो॒डाश॒मर्व॑ता॒ त्वष्टे - [  ] \newline

\textbf{Pada Paata} \newline

मा । नः॒ । मि॒त्रः । वरु॑णः । अ॒र्य॒मा । आ॒युः । इन्द्रः॑ । ऋ॒भु॒क्षा इत्यृ॑भु - क्षाः । म॒रुतः॑ । परीति॑ । ख्य॒न्न् ॥ यत् । वा॒जिनः॑ । दे॒वजा॑त॒स्येति॑ दे॒व - जा॒त॒स्य॒ । सप्तेः᳚ । प्र॒व॒क्ष्याम॒ इति॑ प्र - व॒क्ष्यामः॑ । वि॒दथे᳚ । वी॒र्या॑णि ॥ यत् । नि॒र्णिजेति॑ निः - निजा᳚ । रेक्ण॑सा । प्रावृ॑तस्य । रा॒तिम् । गृ॒भी॒ताम् । मु॒ख॒तः । नय॑न्ति ॥ सुप्रा॒ङिति॒ सु - प्रा॒ङ् । अ॒जः । मेम्य॑त् । वि॒श्वरू॑प॒ इति॑ वि॒श्व - रू॒पः॒ । इ॒न्द्रा॒पू॒ष्णोरिती᳚न्द्रा-पू॒ष्णोः । प्रि॒यम् । अपीति॑ । ए॒ति॒ । पाथः॑ ॥ ए॒षः । छागः॑ । पु॒रः । अश्वे॑न । वा॒जिना᳚ । पू॒ष्णः । भा॒गः । नी॒य॒ते॒ । वि॒श्वदे᳚व्य॒ इति॑ वि॒श्व - दे॒व्यः॒ ॥ अ॒भि॒प्रिय॒मित्य॑भि- प्रिय᳚म् । यत् । पु॒रो॒डाश᳚म् । अर्व॑ता । त्वष्टा᳚ । इत् ।  \newline




\markright{ TS 4.6.8.2  \hfill https://www.vedavms.in \hfill}

\section{ TS 4.6.8.2 }

\textbf{TS 4.6.8.2 } \newline
\textbf{Samhita Paata} \newline

-दे॑नꣳ सौश्रव॒साय॑ जिन्वति ॥ यद्ध॒विष्य॑मृतु॒शो दे॑व॒यानं॒ त्रिर्मानु॑षाः॒ पर्यश्वं॒ नय॑न्ति । अत्रा॑ पू॒ष्णः प्र॑थ॒मो भा॒ग ए॑ति य॒ज्ञ्ं दे॒वेभ्यः॑ प्रतिवे॒दय॑न्न॒जः ॥ होता᳚ऽद्ध्व॒र्युराव॑या अग्निमि॒न्धो ग्रा॑वग्रा॒भ उ॒त शꣳस्ता॒ सुवि॑प्रः । तेन॑ य॒ज्ञेन॒ स्व॑रंकृतेन॒ स्वि॑ष्टेन व॒क्षणा॒ आ पृ॑णद्ध्वं ॥ यू॒प॒व्र॒स्का उ॒त ये यू॑पवा॒हाश्च॒षालं॒ ॅये अ॑श्वयू॒पाय॒ तक्ष॑ति । ये चार्व॑ते॒ पच॑नꣳ स॒भंर॑न्त्यु॒तो - [  ] \newline

\textbf{Pada Paata} \newline

ए॒न॒म् । सौ॒श्र॒व॒साय॑ । जि॒न्व॒ति॒ ॥ यत् । ह॒विष्य᳚म् । ऋ॒तु॒श इत्यृ॑तु - शः । दे॒व॒यान॒मिति॑ देव - यान᳚म् । त्रिः । मानु॑षाः । परीति॑ । अश्व᳚म् । नय॑न्ति ॥ अत्र॑ । पू॒ष्णः । प्र॒थ॒मः । भा॒गः । ए॒ति॒ । य॒ज्ञ्म् । दे॒वेभ्यः॑ । प्र॒ति॒वे॒दय॒न्निति॑ प्रति - वे॒दयन्न्॑ । अ॒जः ॥ होता᳚ । अ॒द्ध्व॒र्युः । आव॑या॒ इत्या - व॒याः॒ । अ॒ग्नि॒मि॒न्ध इत्य॑ग्निं - इ॒न्धः । ग्रा॒व॒ग्रा॒भ इति॑ ग्राव - ग्रा॒भः । उ॒त । शꣳस्ता᳚ । सुवि॑प्र॒ इति॒ सु - वि॒प्रः॒ ॥ तेन॑ । य॒ज्ञेन॑ । स्व॑रंकृते॒नेति॒ सु - अ॒र॒कृं॒ते॒न॒ । स्वि॑ष्टे॒नेति॒ सु - इ॒ष्टे॒न॒ । व॒क्षणाः᳚ । एति॑ । पृ॒ण॒द्ध्व॒म् ॥ यू॒प॒व्र॒स्का इति॑ यूप - व्र॒स्काः । उ॒त । ये । यू॒प॒वा॒हा इति॑ यूप - वा॒हाः । च॒षाल᳚म् । ये । अ॒श्व॒यू॒पायेत्य॑श्व-यू॒पाय॑ । तक्ष॑ति ॥ ये । च॒ । अर्व॑ते । पच॑नम् । स॒भंर॒न्तीति॑ सं - भर॑न्ति । उ॒तो इति॑ ।  \newline




\markright{ TS 4.6.8.3  \hfill https://www.vedavms.in \hfill}

\section{ TS 4.6.8.3 }

\textbf{TS 4.6.8.3 } \newline
\textbf{Samhita Paata} \newline

तेषा॑-म॒भिगू᳚र्तिर्न इन्वतु ॥ उप॒ प्रागा᳚थ् सु॒मन्मे॑ऽधायि॒ मन्म॑ दे॒वाना॒माशा॒ उप॑ वी॒तपृ॑ष्ठः । अन्वे॑नं॒ ॅविप्रा॒ ऋष॑यो मदन्ति दे॒वानां᳚ पु॒ष्टे च॑कृमा सु॒बन्धुं᳚ ॥ यद्-वा॒जिनो॒ दाम॑ स॒दांन॒मर्व॑तो॒ या शी॑र्.ष॒ण्या॑ रश॒ना रज्जु॑रस्य । यद्वा॑ घास्य॒ प्रभृ॑तमा॒स्ये॑ तृणꣳ॒॒ सर्वा॒ ता ते॒ अपि॑ दे॒वेष्व॑स्तु ॥ यदश्व॑स्य क्र॒विषो॒ - [  ] \newline

\textbf{Pada Paata} \newline

तेषा᳚म् । अ॒भिगू᳚र्ति॒रित्य॒भि - गू॒र्तिः॒ । नः॒ । इ॒न्व॒तु॒ ॥ उप॑ । प्रेति॑ । अ॒गा॒त् । सु॒मदिति॑ सु - मत् । मे॒ । अ॒धा॒यि॒ । मन्म॑ । दे॒वाना᳚म् । आशाः᳚ । उपेति॑ । वी॒तपृ॑ष्ठ॒ इति॑ वी॒त - पृ॒ष्ठः॒ ॥ अन्विति॑ । ए॒न॒म् । विप्राः᳚ । ऋष॑यः । म॒द॒न्ति॒ । दे॒वाना᳚म् । पु॒ष्टे । च॒कृ॒म॒ । सु॒बन्धु॒मिति॑ सु - बन्धु᳚म् ॥ यत् । वा॒जिनः॑ । दाम॑ । स॒दांन॒मिति॑ सं - दान᳚म् । अर्व॑तः । या । शी॒र्॒.ष॒ण्या᳚ । र॒श॒ना । रज्जुः॑ । अ॒स्य॒ ॥ यत् । वा॒ । घ॒ । अ॒स्य॒ । प्रभृ॑त॒मिति॒ प्र - भृ॒त॒म् । आ॒स्ये᳚ । तृण᳚म् । सर्वा᳚ । ता । ते॒ । अपीति॑ । दे॒वेषु॑ । अ॒स्तु॒ ॥ यत् । अश्व॑स्य । क्र॒विषः॑ ।  \newline




\markright{ TS 4.6.8.4  \hfill https://www.vedavms.in \hfill}

\section{ TS 4.6.8.4 }

\textbf{TS 4.6.8.4 } \newline
\textbf{Samhita Paata} \newline

मक्षि॒काऽऽश॒ यद्वा॒ स्वरौ॒ स्वधि॑तौ रि॒प्तमस्ति॑ । यद्धस्त॑योः शमि॒तुर्यन्न॒खेषु॒ सर्वा॒ ता ते॒ अपि॑ दे॒वेष्व॑स्तु ॥ यदूव॑द्ध्यमु॒दर॑स्याप॒वाति॒ य आ॒मस्य॑ क्र॒विषो॑ ग॒न्धो अस्ति॑ । सु॒कृ॒ता तच्छ॑मि॒तारः॑ कृण्वन्तू॒त मेधꣳ॑ शृत॒पाकं॑ पचन्तु ॥ यत् ते॒ गात्रा॑द॒ग्निना॑ प॒च्यमा॑नाद॒भि शूलं॒ निह॑तस्याव॒धाव॑ति । मा तद्-भूम्या॒मा श्रि॑ष॒ ( )-न्मा तृणे॑षु दे॒वेभ्य॒स्तदु॒शद्भ्यो॑ रा॒तम॑स्तु ॥ \newline

\textbf{Pada Paata} \newline

मक्षि॑का । आश॑ । यत् । वा॒ । स्वरौ᳚ । स्वधि॑ता॒विति॒ स्व - धि॒तौ॒ । रि॒प्तम् । अस्ति॑ ॥ यत् । हस्त॑योः । श॒मि॒तुः । यत् । न॒खेषु॑ । सर्वा᳚ । ता । ते॒ । अपीति॑ । दे॒वेषु॑ । अ॒स्तु॒ ॥ यत् । ऊव॑द्ध्यम् । उ॒दर॑स्य । अ॒प॒वातीत्य॑प - वाति॑ । यः । आ॒मस्य॑ । क्र॒विषः॑ । ग॒न्धः । अस्ति॑ ॥ सु॒कृ॒तेति॑ सु - कृ॒ता । तत् । श॒मि॒तारः॑ । कृ॒ण्व॒न्तु॒ । उ॒त । मेध᳚म् । शृ॒त॒पाक॒मिति॑ शृत - पाक᳚म् । प॒च॒न्तु॒ ॥ यत् । ते॒ । गात्रा᳚त् । अ॒ग्निना᳚ । प॒च्यमा॑नात् । अ॒भीति॑ । शूल᳚म् । निह॑त॒स्येति॒ नि - ह॒त॒स्य॒ । अ॒व॒धाव॒तीत्य॑व - धाव॑ति ॥ मा । तत् । भूम्या᳚म् । एति॑ । श्रि॒ष॒त् ( ) । मा । तृणे॑षु । दे॒वेभ्यः॑ । तत् । उ॒शद्भ्य॒ इत्यु॒शत् - भ्यः॒ । रा॒तम् । अ॒स्तु॒ ॥  \newline




\markright{ TS 4.6.9.1  \hfill https://www.vedavms.in \hfill}

\section{ TS 4.6.9.1 }

\textbf{TS 4.6.9.1 } \newline
\textbf{Samhita Paata} \newline

ये वा॒जिनं॑ परि॒पश्य॑न्ति प॒क्वं ॅय ई॑मा॒हुः सु॑र॒भिर्निर्.ह॒रेति॑ । ये चार्व॑तो माꣳसभि॒क्षामु॒पास॑त उ॒तो तेषा॑म॒भिगू᳚र्तिर्न इन्वतु ॥ यन्नीक्ष॑णं माꣳ॒॒स्पच॑न्या उ॒खाया॒ या पात्रा॑णि यू॒ष्ण आ॒सेच॑नानि । ऊ॒ष्म॒ण्या॑ऽपि॒धाना॑ चरू॒णाम॒ङ्काः सू॒नाः परि॑ भूष॒न्त्यश्वं᳚ ॥ नि॒क्रम॑णं नि॒षद॑नं ॅवि॒वर्त॑नं॒ ॅयच्च॒ पड्बी॑श॒मर्व॑तः । यच्च॑ प॒पौ यच्च॑ घा॒सिं -[  ] \newline

\textbf{Pada Paata} \newline

ये । वा॒जिन᳚म् । प॒रि॒पश्य॒न्तीति॑ परि-पश्य॑न्ति । प॒क्वम् । ये । ई॒म् । आ॒हुः । सु॒र॒भिः । निरिति॑ । ह॒र॒ । इति॑ ॥ ये । च॒ । अर्व॑तः । माꣳ॒॒स॒भि॒क्षामिति॑ माꣳस - भि॒क्षाम् । उ॒पास॑त॒ इत्यु॑प - आस॑ते । उ॒तो इति॑ । तेषा᳚म् । अ॒भिगू᳚र्ति॒रित्य॒भि - गू॒र्तिः॒ । नः॒ । इ॒न्व॒तु॒ ॥ यत् । नीक्ष॑णम् । माꣳ॒॒स्पच॑न्याः । उ॒खायाः᳚ । या । पात्रा॑णि । यू॒ष्णः । आ॒सेच॑ना॒नीत्या᳚ - सेच॑नानि ॥ ऊ॒ष्म॒ण्या᳚ । अ॒पि॒धानेत्य॑पि - धाना᳚ । च॒रू॒णाम् । अ॒ङ्काः । सू॒नाः । परीति॑ । भू॒ष॒न्ति॒ । अश्व᳚म् ॥ नि॒क्रम॑ण॒मिति॑ नि - क्रम॑णम् । नि॒षद॑न॒मिति॑ नि - सद॑नम् । वि॒वर्त॑न॒मिति॑ वि - वर्त॑नम् । यत् । च॒ । पड्बी॑शम् । अर्व॑तः ॥ यत् । च॒ । प॒पौ । यत् । च॒ । घा॒सिम् ।  \newline




\markright{ TS 4.6.9.2  \hfill https://www.vedavms.in \hfill}

\section{ TS 4.6.9.2 }

\textbf{TS 4.6.9.2 } \newline
\textbf{Samhita Paata} \newline

ज॒घास॒ सर्वा॒ ता ते॒ अपि॑ दे॒वेष्व॑स्तु ॥ मा त्वा॒ऽग्नि-र्द्ध्व॑नयिद्-धू॒मग॑न्धि॒र्मोखा भ्राज॑न्त्य॒भि वि॑क्त॒ जघ्रिः॑ । इ॒ष्टं ॅवी॒तम॒भिगू᳚र्तं॒ ॅवष॑ट्कृतं॒ तं दे॒वासः॒ प्रति॑ गृभ्ण॒न्त्यश्वं᳚ ॥ यदश्वा॑य॒ वास॑ उपस्तृ॒णन्त्य॑धीवा॒सं ॅया हिर॑ण्यान्यस्मै । स॒न्दान॒मर्व॑न्तं॒ पड्बी॑शं प्रि॒या दे॒वेष्वा या॑मयन्ति ॥ यत् ते॑ सा॒दे मह॑सा॒ शूकृ॑तस्य॒ पार्ष्णि॑या वा॒ कश॑या - [  ] \newline

\textbf{Pada Paata} \newline

ज॒घास॑ । सर्वा᳚ । ता । ते॒ । अपीति॑ । दे॒वेषु॑ । अ॒स्तु॒ ॥ मा । त्वा॒ । अ॒ग्निः । ध्व॒न॒यि॒त् । धू॒मग॑न्धि॒रिति॑ धू॒म - ग॒न्धिः॒ । मा । उ॒खा । भ्राज॑न्ति । अ॒भीति॑ । वि॒क्त॒ । जघ्रिः॑ ॥ इ॒ष्टम् । वी॒तम् । अ॒भिगू᳚र्त॒मित्य॒भि - गू॒र्त॒म् । वष॑ट्कृत॒मिति॒ वष॑ट् - कृ॒त॒म् । तम् । दे॒वासः॑ । प्रतीति॑ । गृ॒भ्ण॒न्ति॒ । अश्व᳚म् ॥ यत् । अश्वा॑य । वासः॑ । उ॒प॒स्तृ॒णन्तीत्यु॑प - स्तृ॒णन्ति॑ । अ॒धी॒वा॒समित्य॑धि - वा॒सम् । या । हिर॑ण्यानि । अ॒स्मै॒ ॥ स॒न्दान॒मिति॑ सं - दान᳚म् । अर्व॑न्तम् । पड्बी॑शम् । प्रि॒या । दे॒वेषु॑ । एति॑ । या॒म॒य॒न्ति॒ ॥ यत् । ते॒ । सा॒दे । मह॑सा । शूकृ॑त॒स्येति॒ शू - कृ॒त॒स्य॒ । पार्ष्णि॑या । वा॒ । कश॑या ।  \newline




\markright{ TS 4.6.9.3  \hfill https://www.vedavms.in \hfill}

\section{ TS 4.6.9.3 }

\textbf{TS 4.6.9.3 } \newline
\textbf{Samhita Paata} \newline

वा तु॒तोद॑ । स्रु॒चेव॒ ता ह॒विषो॑ अद्ध्व॒रेषु॒ सर्वा॒ ता ते॒ ब्रह्म॑णा सूदयामि ॥ चतु॑स्त्रिꣳशद्-वा॒जिनो॑ दे॒वब॑न्धो॒-र्वङ्क्री॒-रश्व॑स्य॒ स्वधि॑तिः॒ समे॑ति । अच्छि॑द्रा॒ गात्रा॑ व॒युना॑ कृणोत॒ परु॑ष्परुरनु॒घुष्या॒ वि श॑स्त ॥ एक॒स्त्वष्टु॒रश्व॑स्या विश॒स्ता द्वा य॒न्तारा॑ भवत॒स्तथ॒र्तुः । या ते॒ गात्रा॑णामृतु॒था कृ॒णोमि॒ ताता॒ पिण्डा॑नां॒ प्र जु॑होम्य॒ग्नौ ॥ मा त्वा॑ तपत् - [  ] \newline

\textbf{Pada Paata} \newline

वा॒ । तु॒तोद॑ ॥ स्रु॒चा । इ॒व॒ । ता । ह॒विषः॑ । अ॒द्ध्व॒रेषु॑ । सर्वा᳚ । ता । ते॒ । ब्रह्म॑णा । सू॒द॒या॒मि॒ ॥ चतु॑स्त्रिꣳश॒दिति॒ चतुः॑ - त्रिꣳ॒॒श॒त् । वा॒जिनः॑ । दे॒वब॑न्धो॒रिति॑ दे॒व - ब॒न्धोः॒ । वङ्क्रीः᳚ । अश्व॑स्य । स्वधि॑ति॒रिति॒ स्व - धि॒तिः॒ । समिति॑ । ए॒ति॒ ॥ अच्छि॑द्रा । गात्रा᳚ । व॒युना᳚ । कृ॒णो॒त॒ । परु॑ष्परु॒रिति॒ परुः॑ - प॒रुः॒ । अ॒नु॒घुष्येत्य॑नु-घुष्य॑ । वीति॑ । श॒स्त॒ ॥ एकः॑ । त्वष्टुः॑ । अश्व॑स्य । वि॒श॒स्तेति॑ वि - श॒स्ता । द्वा । य॒न्तारा᳚ । भ॒व॒तः॒ । तथा᳚ । ऋ॒तुः ॥ या । ते॒ । गात्रा॑णाम् । ऋ॒तु॒थेत्यृ॑तु - था । कृ॒णोमि॑ । तातेति॒ ता - ता॒ । पिण्डा॑नाम् । प्रेति॑ । जु॒हो॒मि॒ । अ॒ग्नौ ॥ मा । त्वा॒ । त॒प॒त् ।  \newline




\markright{ TS 4.6.9.4  \hfill https://www.vedavms.in \hfill}

\section{ TS 4.6.9.4 }

\textbf{TS 4.6.9.4 } \newline
\textbf{Samhita Paata} \newline

प्रि॒य आ॒त्माऽपि॒यन्तं॒ मा स्वधि॑तिस्त॒नुव॒ आ ति॑ष्ठिपत् ते । मा ते॑ गृ॒द्ध्नु-र॑विश॒स्ताऽति॒हाय॑ छि॒द्रा गात्रा᳚ण्य॒सिना॒ मिथू॑ कः ॥ न वा उ॑वे॒तन्म्रि॑यसे॒ न रि॑ष्यसि दे॒वाꣳ इदे॑षि प॒थिभिः॑ सु॒गेभिः॑ । हरी॑ ते॒ युञ्जा॒ पृष॑ती अभूता॒मुपा᳚स्थाद्-वा॒जी धु॒रि रास॑भस्य ॥ सु॒गव्यं॑ नो वा॒जी स्वश्वि॑यं पुꣳ॒॒सः पु॒त्राꣳ उ॒त वि॑श्वा॒पुषꣳ॑ र॒यिं ( ) । अ॒ना॒गा॒स्त्वं नो॒ अदि॑तिः कृणोतु क्ष॒त्रं नो॒ अश्वो॑ वनताꣳ ह॒विष्मान्॑ ॥ \newline

\textbf{Pada Paata} \newline

प्रि॒यः । आ॒त्मा । अ॒पि॒यन्त॒मित्य॑पि - यन्त᳚म् । मा । स्वधि॑ति॒रिति॒ स्व - धि॒तिः॒ । त॒नुवः॑ । एति॑ । ति॒ष्ठि॒प॒त् । ते॒ ॥ मा । ते॒ । गृ॒द्ध्नुः । अ॒वि॒श॒स्तेत्य॑वि - श॒स्ता । अ॒ति॒हायेत्य॑ति-हाय॑ । छि॒द्रा । गात्रा॑णि । अ॒सिना᳚ । मिथु॑ । कः॒ ॥ न । वै । उ॒ । ए॒तत् । म्रि॒य॒से॒ । न । रि॒ष्य॒सि॒ । दे॒वान् । इत् । ए॒षि॒ । प॒थिभि॒रिति॑ प॒थि - भिः॒ । सु॒गेभि॒रिति॑ सु - गेभिः॑ ॥ हरी॒ इति॑ । ते॒ । युञ्जा᳚ । पृष॑ती॒ इति॑ । अ॒भू॒ता॒म् । उपेति॑ । अ॒स्था॒त् । वा॒जी । धु॒रि । रास॑भस्य ॥ सु॒गव्य॒मिति॑ सु - गव्य᳚म् । नः॒ । वा॒जी । स्वश्वि॑य॒मिति॑ सु - अश्वि॑यम् । पुꣳ॒॒सः । पु॒त्रान् । उ॒त । वि॒श्वा॒पुष॒मिति॑ विश्व -पुष᳚म् । र॒यिम् ( ) ॥ अ॒ना॒गा॒स्त्वमित्य॑नागाः - त्वम् । नः॒ । अदि॑तिः । कृ॒णो॒तु॒ । क्ष॒त्रम् । नः॒ । अश्वः॑ । व॒न॒ता॒म् । ह॒विष्मान्॑ ॥  \newline






\end{document}