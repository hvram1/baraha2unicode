\documentclass[17pt]{extarticle}
\usepackage{babel}
\usepackage{fontspec}
\usepackage{polyglossia}
\usepackage{extsizes}



\setmainlanguage{sanskrit}
\setotherlanguages{english} %% or other languages
\setlength{\parindent}{0pt}
\pagestyle{myheadings}
\newfontfamily\devanagarifont[Script=Devanagari]{AdishilaVedic}


\newcommand{\VAR}[1]{}
\newcommand{\BLOCK}[1]{}




\begin{document}
\begin{titlepage}
    \begin{center}
 
\begin{sanskrit}
    { \Huge
    कृष्ण यजुर्वेदीय तैत्तिरीय संहिता,पद,जटा,घन पाठः 
    }
    \\
    \vspace{2.5cm}
    \mbox{ \Huge
    1.4     प्रथमकाण्डे चतुर्त्थः प्रश्नः-(सुत्यादिने कर्तव्या ग्रहाः)   }
\end{sanskrit}
\end{center}

\end{titlepage}
\tableofcontents
\pagebreak

\markright{ TS 1.4.1.1  \hfill https://www.vedavms.in \hfill}
\addcontentsline{toc}{section}{ TS 1.4.1.1 }
\section*{ TS 1.4.1.1 }

\textbf{TS 1.4.1.1 } \newline
\textbf{Samhita Paata} \newline

आ द॑दे॒ ग्रावा᳚ऽ*स्यद्ध्वर॒कृद् दे॒वेभ्यो॑ गंभी॒रमि॒म- म॑द्ध्व॒रं कृ॑द्ध्युत्त॒मेन॑ प॒विनेन्द्रा॑य॒ सोमꣳ॒॒ सुषु॑तं॒ मधु॑मन्तं॒ पय॑स्वन्तं ॅवृष्टि॒वनि॒मिन्द्रा॑य त्वा वृत्र॒घ्न इन्द्रा॑य त्वा वृत्र॒तुर॒ इन्द्रा॑य त्वाऽभिमाति॒घ्न इन्द्रा॑य त्वाऽऽदि॒त्यव॑त॒ इन्द्रा॑य त्वा वि॒श्वदे᳚व्यावते श्वा॒त्राः स्थ॑ वृत्र॒तुरो॒ राधो॑गूर्ता अ॒मृत॑स्य॒ पत्नी॒स्ता दे॑वीर् देव॒त्रेमं ॅय॒ज्ञ्ं ध॒त्तोप॑हूताः॒ सोम॑स्य पिब॒तोप॑हूतो यु॒ष्माकꣳ॒॒ - [ ] \newline

\textbf{Pada Paata} \newline

एति॑ । द॒दे॒ । ग्रावा᳚ । अ॒सि॒ । अ॒द्ध्व॒र॒कृदित्य॑द्ध्वर - कृत् । दे॒वेभ्यः॑ । ग॒भीं॒रम् । इ॒मम् । अ॒द्ध्व॒रम् । कृ॒धि॒ । उ॒त्त॒मेनेत्यु॑त् - त॒मेन॑ । प॒विना᳚ । इन्द्रा॑य । सोम᳚म् । सुषु॑त॒मिति॒ सु - सु॒त॒म् । मधु॑मन्त॒मिति॒ मधु॑ - म॒न्त॒म् । पय॑स्वन्तम् । वृ॒ष्टि॒वनि॒मिति॑ वृष्टि - वनि᳚म् । इन्द्रा॑य । त्वा॒ । वृ॒त्र॒घ्न इति॑ वृत्र - घ्ने । इन्द्रा॑य । त्वा॒ । वृ॒त्र॒तुर॒ इति॑ वृत्र - तुरे᳚ । इन्द्रा॑य । त्वा॒ । अ॒भि॒मा॒ति॒घ्न इत्य॑भिमाति - घ्ने । इन्द्रा॑य । त्वा॒ । आ॒दि॒त्यव॑त॒ इत्या॑दि॒त्य - व॒ते॒ । इन्द्रा॑य । त्वा॒ । वि॒श्वदे᳚व्यावत॒ इति॑ वि॒श्वदे᳚व्य - व॒ते॒ । श्वा॒त्राः । स्थ॒ । वृ॒त्र॒तुर॒ इति॑ वृत्र - तुरः॑ । राधो॑गूर्ता॒ इति॒ राधः॑ - गू॒र्ताः॒ । अ॒मृत॑स्य । पत्नीः᳚ । ताः । दे॒वीः॒ । दे॒व॒त्रेति॑ देव - त्रा । इ॒मम् । य॒ज्ञ्म् । ध॒त्त॒ । उप॑हूता॒ इत्युप॑ - हू॒ताः॒ । सोम॑स्य । पि॒ब॒त॒ । उप॑हूत॒ इत्युप॑-हू॒तः॒ । यु॒ष्माक᳚म् ।  \newline


\textbf{Krama Paata} \newline

आ द॑दे । द॒दे॒ ग्रावा᳚ । ग्रावा॑ऽसि । अ॒स्य॒द्ध्व॒र॒कृत् । अ॒द्ध्व॒र॒कृद् दे॒वेभ्यः॑ । अ॒ध्व॒र॒कृदित्य॑द्ध्वर - कृत् । दे॒वेभ्यो॑ गम्भी॒रम् । ग॒म्भी॒रमि॒मम् । इ॒मम॑द्ध्व॒रम् । अ॒द्ध्व॒रम् कृ॑धि । कृ॒ध्यु॒त्त॒मेन॑ । उ॒त्त॒मेन॑ प॒विना᳚ । उ॒त्त॒मेनेत्यु॑त् - त॒मेन॑ । प॒विनेन्द्रा॑य । इन्द्रा॑य॒ सोम᳚म् । सोमꣳ॒॒ सुषु॑तम् । सुषु॑त॒म् मधु॑मन्तम् । सुषु॑त॒मिति॒ सु - सु॒त॒म् । मधु॑मन्त॒म् पय॑स्वन्तम् । मधु॑मन्त॒मिति॒ मधु॑ - म॒न्त॒म् । पय॑स्वन्तं ॅवृष्टि॒वनि᳚म् । वृ॒ष्टि॒वनि॒मिन्द्रा॑य । वृ॒ष्टि॒वनि॒मिति॑ वृष्टि - वनि᳚म् । इन्द्रा॑य त्वा । त्वा॒ वृ॒त्र॒घ्ने । वृ॒त्र॒घ्न इन्द्रा॑य । वृ॒त्र॒घ्न इति॑ वृत्र - घ्ने । इन्द्रा॑य त्वा । त्वा॒ वृ॒त्र॒तुरे᳚ । वृ॒त्र॒तुर॒ इन्द्रा॑य । वृ॒त्र॒तुर॒ इति॑ वृत्र - तुरे᳚ । इन्द्रा॑य त्वा । त्वा॒ऽभि॒मा॒ति॒घ्ने । अ॒भि॒मा॒ति॒घ्न इन्द्रा॑य । अ॒भि॒मा॒ति॒घ्न इत्य॑भिमाति - घ्ने । इन्द्रा॑य त्वा । त्वा॒ ऽऽदि॒त्यव॑ते । आ॒दि॒त्यव॑त॒ इन्द्रा॑य । आ॒दि॒त्यव॑त॒ इत्या॑दि॒त्य - व॒ते॒ । इन्द्रा॑य त्वा । त्वा॒ वि॒श्वदे᳚व्यावते । वि॒श्वदे᳚व्यावते श्वा॒त्राः । वि॒श्वदे᳚व्यावत॒ इति॑ वि॒श्वदे᳚व्य - व॒ते॒ । श्वा॒त्राः स्थ॑ । स्थ॒ वृ॒त्र॒तुरः॑ । वृ॒त्र॒तुरो॒ राधो॑गूर्ताः । वृ॒त्र॒तुर॒ इति॑ वृत्र - तुरः॑ । राधो॑गूर्ता अ॒मृत॑स्य । राधो॑गूर्ता॒ इति॒ राधः॑ - गू॒र्ताः॒ । अ॒मृत॑स्य॒ पत्नीः᳚ । पत्नी॒स्ताः । ता दे॑वीः । दे॒वी॒र् दे॒व॒त्रा । दे॒व॒त्रेमम् । दे॒व॒त्रेति॑ देव - त्रा । इ॒मं ॅय॒ज्ञ्म् । य॒ज्ञ्म् ध॑त्त । ध॒त्तोप॑हूताः । उप॑हूताः॒ सोम॑स्य । उप॑हूता॒ इत्युप॑ - हू॒ताः॒ । सोम॑स्य पिबत । पि॒ब॒तोप॑हूतः । उप॑हूतो यु॒ष्माक᳚म् । उप॑हूत॒ इत्युप॑ - हू॒तः॒ । यु॒ष्माकꣳ॒॒ सोमः॑ \newline

\textbf{Jatai Paata} \newline

1. आ द॑दे दद॒ आ द॑दे । \newline
2. द॒दे॒ ग्रावा॒ ग्रावा॑ ददे ददे॒ ग्रावा᳚ । \newline
3. ग्रावा᳚ ऽस्यसि॒ ग्रावा॒ ग्रावा॑ ऽसि । \newline
4. अ॒स्य॒ द्ध्व॒र॒कृ द॑द्ध्वर॒कृ द॑स्य स्यद्ध्वर॒कृत् । \newline
5. अ॒द्ध्व॒र॒कृद् दे॒वेभ्यो॑ दे॒वेभ्यो॑ अद्ध्वर॒कृद॑ द्ध्वर॒कृद् दे॒वेभ्यः॑ । \newline
6. अ॒द्ध्व॒र॒कृदित्य॑द्ध्वर - कृत् । \newline
7. दे॒वेभ्यो॑ गंभी॒रम् गं॑भी॒रम् दे॒वेभ्यो॑ दे॒वेभ्यो॑ गंभी॒रम् । \newline
8. गं॒भी॒र मि॒म मि॒मम् गं॑भी॒रम् गं॑भी॒र मि॒मम् । \newline
9. इ॒म म॑द्ध्व॒र म॑द्ध्व॒र मि॒म मि॒म म॑द्ध्व॒रम् । \newline
10. अ॒द्ध्व॒रम् कृ॑धि कृध्यद्ध्व॒र म॑द्ध्व॒रम् कृ॑धि । \newline
11. कृ॒ध्यु॒त्त॒ मेनो᳚त्त॒मेन॑ कृधि कृध्युत्त॒मेन॑ । \newline
12. उ॒त्त॒मेन॑ प॒विना॑ प॒वि नो᳚त्त॒मेनो᳚त्त॒मेन॑ प॒विना᳚ । \newline
13. उ॒त्त॒मेनेत्यु॑त् - त॒मेन॑ । \newline
14. प॒विनेन्द्रा॒ये न्द्रा॑य प॒विना॑ प॒विनेन्द्रा॑य । \newline
15. इन्द्रा॑य॒ सोम॒(ग्म्॒) सोम॒ मिन्द्रा॒ये न्द्रा॑य॒ सोम᳚म् । \newline
16. सोम॒(ग्म्॒) सुषु॑त॒(ग्म्॒) सुषु॑त॒(ग्म्॒) सोम॒(ग्म्॒) सोम॒(ग्म्॒) सुषु॑तम् । \newline
17. सुषु॑त॒म् मधु॑मन्त॒म् मधु॑मन्त॒(ग्म्॒) सुषु॑त॒(ग्म्॒) सुषु॑त॒म् मधु॑मन्तम् । \newline
18. सुषु॑त॒मिति॒ सु - सु॒त॒म् । \newline
19. मधु॑मन्त॒म् पय॑स्वन्त॒म् पय॑स्वन्त॒म् मधु॑मन्त॒म् मधु॑मन्त॒म् पय॑स्वन्तम् । \newline
20. मधु॑मन्त॒मिति॒ मधु॑ - म॒न्त॒म् । \newline
21. पय॑स्वन्तं ॅवृष्टि॒वनिं॑ ॅवृष्टि॒वनि॒म् पय॑स्वन्त॒म् पय॑स्वन्तं ॅवृष्टि॒वनि᳚म् । \newline
22. वृ॒ष्टि॒वनि॒ मिन्द्रा॒ये न्द्रा॑य वृष्टि॒वनिं॑ ॅवृष्टि॒वनि॒ मिन्द्रा॑य । \newline
23. वृ॒ष्टि॒वनि॒मिति॑ वृष्टि - वनि᳚म् । \newline
24. इन्द्रा॑य त्वा॒ त्वेन्द्रा॒ये न्द्रा॑य त्वा । \newline
25. त्वा॒ वृ॒त्र॒घ्ने वृ॑त्र॒घ्ने त्वा᳚ त्वा वृत्र॒घ्ने । \newline
26. वृ॒त्र॒घ्न इन्द्रा॒ये न्द्रा॑य वृत्र॒घ्ने वृ॑त्र॒घ्न इन्द्रा॑य । \newline
27. वृ॒त्र॒घ्न इति॑ वृत्र - घ्ने । \newline
28. इन्द्रा॑य त्वा॒ त्वेन्द्रा॒ये न्द्रा॑य त्वा । \newline
29. त्वा॒ वृ॒त्र॒तुरे॑ वृत्र॒तुरे᳚ त्वा त्वा वृत्र॒तुरे᳚ । \newline
30. वृ॒त्र॒तुर॒ इन्द्रा॒ये न्द्रा॑य वृत्र॒तुरे॑ वृत्र॒तुर॒ इन्द्रा॑य । \newline
31. वृ॒त्र॒तुर॒ इति॑ वृत्र - तुरे᳚ । \newline
32. इन्द्रा॑य त्वा॒ त्वेन्द्रा॒ये न्द्रा॑य त्वा । \newline
33. त्वा॒ ऽभि॒मा॒ति॒घ्ने॑ ऽभिमाति॒घ्ने त्वा᳚ त्वा ऽभिमाति॒घ्ने । \newline
34. अ॒भि॒मा॒ति॒घ्न इन्द्रा॒ये न्द्रा॑याभिमाति॒घ्ने॑ ऽभिमाति॒घ्न इन्द्रा॑य । \newline
35. अ॒भि॒मा॒ति॒घ्न इत्य॑भिमाति - घ्ने । \newline
36. इन्द्रा॑य त्वा॒ त्वेन्द्रा॒ये न्द्रा॑य त्वा । \newline
37. त्वा॒ ऽऽदि॒त्यव॑त आदि॒त्यव॑ते त्वा त्वा ऽऽदि॒त्यव॑ते । \newline
38. आ॒दि॒त्यव॑त॒ इन्द्रा॒ये न्द्रा॑यादि॒त्यव॑त आदि॒त्यव॑त॒ इन्द्रा॑य । \newline
39. आ॒दि॒त्यव॑त॒ इत्या॑दि॒त्य - व॒ते॒ । \newline
40. इन्द्रा॑य त्वा॒ त्वेन्द्रा॒ये न्द्रा॑य त्वा । \newline
41. त्वा॒ वि॒श्वदे᳚व्यावते वि॒श्वदे᳚व्यावते त्वा त्वा वि॒श्वदे᳚व्यावते । \newline
42. वि॒श्वदे᳚व्यावते श्वा॒त्राः श्वा॒त्रा वि॒श्वदे᳚व्यावते वि॒श्वदे᳚व्यावते श्वा॒त्राः । \newline
43. वि॒श्वदे᳚व्यावत॒ इति॑ वि॒श्वदे᳚व्य - व॒ते॒ । \newline
44. श्वा॒त्राः स्थ॑ स्थ श्वा॒त्राः श्वा॒त्राः स्थ॑ । \newline
45. स्थ॒ वृ॒त्र॒तुरो॑ वृत्र॒तुरः॑ स्थ स्थ वृत्र॒तुरः॑ । \newline
46. वृ॒त्र॒तुरो॒ राधो॑गूर्ता॒ राधो॑गूर्ता वृत्र॒तुरो॑ वृत्र॒तुरो॒ राधो॑गूर्ताः । \newline
47. वृ॒त्र॒तुर॒ इति॑ वृत्र - तुरः॑ । \newline
48. राधो॑गूर्ता अ॒मृत॑स्या॒मृत॑स्य॒ राधो॑गूर्ता॒ राधो॑गूर्ता अ॒मृत॑स्य । \newline
49. राधो॑गूर्ता॒ इति॒ राधः॑ - गू॒र्ताः॒ । \newline
50. अ॒मृत॑स्य॒ पत्नीः॒ पत्नी॑ र॒मृत॑स्या॒मृत॑स्य॒ पत्नीः᳚ । \newline
51. पत्नी॒ स्ता स्ताः पत्नीः॒ पत्नी॒ स्ताः । \newline
52. ता दे॑वीर् देवी॒ स्ता स्ता दे॑वीः । \newline
53. दे॒वी॒र् दे॒व॒त्रा दे॑व॒त्रा दे॑वीर् देवीर् देव॒त्रा । \newline
54. दे॒व॒त्रेम मि॒मम् दे॑व॒त्रा दे॑व॒त्रेमम् । \newline
55. दे॒व॒त्रेति॑ देव - त्रा । \newline
56. इ॒मं ॅय॒ज्ञ्ं ॅय॒ज्ञ् मि॒म मि॒मं ॅय॒ज्ञ्म् । \newline
57. य॒ज्ञ्म् ध॑त्त धत्त य॒ज्ञ्ं ॅय॒ज्ञ्म् ध॑त्त । \newline
58. ध॒त्तोप॑हूता॒ उप॑हूता धत्त ध॒त्तोप॑हूताः । \newline
59. उप॑हूताः॒ सोम॑स्य॒ सोम॒स्योप॑हूता॒ उप॑हूताः॒ सोम॑स्य । \newline
60. उप॑हूता॒ इत्युप॑ - हू॒ताः॒ । \newline
61. सोम॑स्य पिबत पिबत॒ सोम॑स्य॒ सोम॑स्य पिबत । \newline
62. पि॒ब॒तोप॑हूत॒ उप॑हूतः पिबत पिब॒तोप॑हूतः । \newline
63. उप॑हूतो यु॒ष्माकं॑ ॅयु॒ष्माक॒ मुप॑हूत॒ उप॑हूतो यु॒ष्माक᳚म् । \newline
64. उप॑हूत॒ इत्युप॑ - हू॒तः॒ । \newline
65. यु॒ष्माक॒(ग्म्॒) सोमः॒ सोमो॑ यु॒ष्माकं॑ ॅयु॒ष्माक॒(ग्म्॒) सोमः॑ । \newline

\textbf{Ghana Paata } \newline

1. आ द॑दे दद॒ आ द॑दे॒ ग्रावा॒ ग्रावा॑ दद॒ आ द॑दे॒ ग्रावा᳚ । \newline
2. द॒दे॒ ग्रावा॒ ग्रावा॑ ददे ददे॒ ग्रावा᳚ ऽस्यसि॒ ग्रावा॑ ददे ददे॒ ग्रावा॑ ऽसि । \newline
3. ग्रावा᳚ ऽस्यसि॒ ग्रावा॒ ग्रावा᳚ ऽस्यद्ध्वर॒कृ द॑द्ध्वर॒कृद॑सि॒ ग्रावा॒ ग्रावा᳚ ऽस्यद्ध्वर॒कृत् । \newline
4. अ॒स्य॒ द्ध्व॒र॒कृद॑ द्ध्वर॒कृ द॑स्यस्य द्ध्वर॒कृद् दे॒वेभ्यो॑ दे॒वेभ्यो॑ अद्ध्वर॒कृ द॑स्यस्य द्ध्वर॒कृद् दे॒वेभ्यः॑ । \newline
5. अ॒द्ध्व॒र॒कृद् दे॒वेभ्यो॑ दे॒वेभ्यो॑ अद्ध्वर॒कृद॑ द्ध्वर॒कृद् दे॒वेभ्यो॑ गंभी॒रम् गं॑भी॒रम् दे॒वेभ्यो॑ अद्ध्वर॒कृद॑ द्ध्वर॒कृद् दे॒वेभ्यो॑ गंभी॒रम् । \newline
6. अ॒द्ध्व॒र॒कृदित्य॑द्ध्वर - कृत् । \newline
7. दे॒वेभ्यो॑ गंभी॒रम् गं॑भी॒रम् दे॒वेभ्यो॑ दे॒वेभ्यो॑ गंभी॒र मि॒म मि॒मम् गं॑भी॒रम् दे॒वेभ्यो॑ दे॒वेभ्यो॑ गंभी॒र मि॒मम् । \newline
8. गं॒भी॒र मि॒म मि॒मम् गं॑भी॒रम् गं॑भी॒र मि॒म म॑द्ध्व॒र म॑द्ध्व॒र मि॒मम् गं॑भी॒रम् गं॑भी॒र मि॒म म॑द्ध्व॒रम् । \newline
9. इ॒म म॑द्ध्व॒र म॑द्ध्व॒र मि॒म मि॒म म॑द्ध्व॒रम् कृ॑धि कृध्यद्ध्व॒र मि॒म मि॒म म॑द्ध्व॒रम् कृ॑धि । \newline
10. अ॒द्ध्व॒रम् कृ॑धि कृध्यद्ध्व॒र म॑द्ध्व॒रम् कृ॑ध्युत्त॒ मेनो᳚त्त॒मेन॑ कृध्यद्ध्व॒र म॑द्ध्व॒रम् कृ॑ध्युत्त॒मेन॑ । \newline
11. कृ॒ध्यु॒त्त॒ मेनो᳚त्त॒मेन॑ कृधि कृध्युत्त॒मेन॑ प॒विना॑ प॒विनो᳚ त्त॒मेन॑ कृधि कृध्युत्त॒मेन॑ प॒विना᳚ । \newline
12. उ॒त्त॒मेन॑ प॒विना॑ प॒विनो᳚ त्त॒मेनो᳚ त्त॒मेन॑ प॒विनेन्द्रा॒ये न्द्रा॑य प॒विनो᳚ त्त॒मेनो᳚ त्त॒मेन॑ प॒विनेन्द्रा॑य । \newline
13. उ॒त्त॒मेनेत्यु॑त् - त॒मेन॑ । \newline
14. प॒विनेन्द्रा॒ये न्द्रा॑य प॒विना॑ प॒विनेन्द्रा॑य॒ सोम॒(ग्म्॒) सोम॒ मिन्द्रा॑य प॒विना॑ प॒विनेन्द्रा॑य॒ सोम᳚म् । \newline
15. इन्द्रा॑य॒ सोम॒(ग्म्॒) सोम॒ मिन्द्रा॒ये न्द्रा॑य॒ सोम॒(ग्म्॒) सुषु॑त॒(ग्म्॒) सुषु॑त॒(ग्म्॒) सोम॒ मिन्द्रा॒ये न्द्रा॑य॒ सोम॒(ग्म्॒) सुषु॑तम् । \newline
16. सोम॒(ग्म्॒) सुषु॑त॒(ग्म्॒) सुषु॑त॒(ग्म्॒) सोम॒(ग्म्॒) सोम॒(ग्म्॒) सुषु॑त॒म् मधु॑मन्त॒म् मधु॑मन्त॒(ग्म्॒) सुषु॑त॒(ग्म्॒) सोम॒(ग्म्॒) सोम॒(ग्म्॒) सुषु॑त॒म् मधु॑मन्तम् । \newline
17. सुषु॑त॒म् मधु॑मन्त॒म् मधु॑मन्त॒(ग्म्॒) सुषु॑त॒(ग्म्॒) सुषु॑त॒म् मधु॑मन्त॒म् पय॑स्वन्त॒म् पय॑स्वन्त॒म् मधु॑मन्त॒(ग्म्॒) सुषु॑त॒(ग्म्॒) सुषु॑त॒म् मधु॑मन्त॒म् पय॑स्वन्तम् । \newline
18. सुषु॑त॒मिति॒ सु - सु॒त॒म् । \newline
19. मधु॑मन्त॒म् पय॑स्वन्त॒म् पय॑स्वन्त॒म् मधु॑मन्त॒म् मधु॑मन्त॒म् पय॑स्वन्तं ॅवृष्टि॒वनिं॑ ॅवृष्टि॒वनि॒म् पय॑स्वन्त॒म् मधु॑मन्त॒म् मधु॑मन्त॒म् पय॑स्वन्तं ॅवृष्टि॒वनि᳚म् । \newline
20. मधु॑मन्त॒मिति॒ मधु॑ - म॒न्त॒म् । \newline
21. पय॑स्वन्तं ॅवृष्टि॒वनिं॑ ॅवृष्टि॒वनि॒म् पय॑स्वन्त॒म् पय॑स्वन्तं ॅवृष्टि॒वनि॒ मिन्द्रा॒ये न्द्रा॑य वृष्टि॒वनि॒म् पय॑स्वन्त॒म् पय॑स्वन्तं ॅवृष्टि॒वनि॒ मिन्द्रा॑य । \newline
22. वृ॒ष्टि॒वनि॒ मिन्द्रा॒ये न्द्रा॑य वृष्टि॒वनिं॑ ॅवृष्टि॒वनि॒ मिन्द्रा॑य त्वा॒ त्वेन्द्रा॑य वृष्टि॒वनिं॑ ॅवृष्टि॒वनि॒ मिन्द्रा॑य त्वा । \newline
23. वृ॒ष्टि॒वनि॒मिति॑ वृष्टि - वनि᳚म् । \newline
24. इन्द्रा॑य त्वा॒ त्वेन्द्रा॒ये न्द्रा॑य त्वा वृत्र॒घ्ने वृ॑त्र॒घ्ने त्वेन्द्रा॒ये न्द्रा॑य त्वा वृत्र॒घ्ने । \newline
25. त्वा॒ वृ॒त्र॒घ्ने वृ॑त्र॒घ्ने त्वा᳚ त्वा वृत्र॒घ्न इन्द्रा॒ये न्द्रा॑य वृत्र॒घ्ने त्वा᳚ त्वा वृत्र॒घ्न इन्द्रा॑य । \newline
26. वृ॒त्र॒घ्न इन्द्रा॒ये न्द्रा॑य वृत्र॒घ्ने वृ॑त्र॒घ्न इन्द्रा॑य त्वा॒ त्वेन्द्रा॑य वृत्र॒घ्ने वृ॑त्र॒घ्न इन्द्रा॑य त्वा । \newline
27. वृ॒त्र॒घ्न इति॑ वृत्र - घ्ने । \newline
28. इन्द्रा॑य त्वा॒ त्वेन्द्रा॒ये न्द्रा॑य त्वा वृत्र॒तुरे॑ वृत्र॒तुरे॒ त्वेन्द्रा॒ये न्द्रा॑य त्वा वृत्र॒तुरे᳚ । \newline
29. त्वा॒ वृ॒त्र॒तुरे॑ वृत्र॒तुरे᳚ त्वा त्वा वृत्र॒तुर॒ इन्द्रा॒ये न्द्रा॑य वृत्र॒तुरे᳚ त्वा त्वा वृत्र॒तुर॒ इन्द्रा॑य । \newline
30. वृ॒त्र॒तुर॒ इन्द्रा॒ये न्द्रा॑य वृत्र॒तुरे॑ वृत्र॒तुर॒ इन्द्रा॑य त्वा॒ त्वेन्द्रा॑य वृत्र॒तुरे॑ वृत्र॒तुर॒ इन्द्रा॑य त्वा । \newline
31. वृ॒त्र॒तुर॒ इति॑ वृत्र - तुरे᳚ । \newline
32. इन्द्रा॑य त्वा॒ त्वेन्द्रा॒ये न्द्रा॑य त्वा ऽभिमाति॒घ्ने॑ ऽभिमाति॒घ्ने त्वेन्द्रा॒ये न्द्रा॑य त्वा ऽभिमाति॒घ्ने । \newline
33. त्वा॒ ऽभि॒मा॒ति॒घ्ने॑ ऽभिमाति॒घ्ने त्वा᳚ त्वा ऽभिमाति॒घ्न इन्द्रा॒ये न्द्रा॑याभिमाति॒घ्ने त्वा᳚ त्वा ऽभिमाति॒घ्न इन्द्रा॑य । \newline
34. अ॒भि॒मा॒ति॒घ्न इन्द्रा॒ये न्द्रा॑याभिमाति॒घ्ने॑ ऽभिमाति॒घ्न इन्द्रा॑य त्वा॒ त्वेन्द्रा॑याभिमाति॒घ्ने॑ ऽभिमाति॒घ्न इन्द्रा॑य त्वा । \newline
35. अ॒भि॒मा॒ति॒घ्न इत्य॑भिमाति - घ्ने । \newline
36. इन्द्रा॑य त्वा॒ त्वेन्द्रा॒ये न्द्रा॑य त्वा ऽऽदि॒त्यव॑त आदि॒त्यव॑ते॒ त्वेन्द्रा॒ये न्द्रा॑य त्वा ऽऽदि॒त्यव॑ते । \newline
37. त्वा॒ ऽऽदि॒त्यव॑त आदि॒त्यव॑ते त्वा त्वा ऽऽदि॒त्यव॑त॒ इन्द्रा॒ये न्द्रा॑यादि॒त्यव॑ते त्वा त्वा ऽऽदि॒त्यव॑त॒ इन्द्रा॑य । \newline
38. आ॒दि॒त्यव॑त॒ इन्द्रा॒ये न्द्रा॑यादि॒त्यव॑त आदि॒त्यव॑त॒ इन्द्रा॑य त्वा॒ त्वेन्द्रा॑यादि॒त्यव॑त आदि॒त्यव॑त॒ इन्द्रा॑य त्वा । \newline
39. आ॒दि॒त्यव॑त॒ इत्या॑दि॒त्य - व॒ते॒ । \newline
40. इन्द्रा॑य त्वा॒ त्वेन्द्रा॒ये न्द्रा॑य त्वा वि॒श्वदे᳚व्यावते वि॒श्वदे᳚व्यावते॒ त्वेन्द्रा॒ये न्द्रा॑य त्वा वि॒श्वदे᳚व्यावते । \newline
41. त्वा॒ वि॒श्वदे᳚व्यावते वि॒श्वदे᳚व्यावते त्वा त्वा वि॒श्वदे᳚व्यावते श्वा॒त्राः श्वा॒त्रा वि॒श्वदे᳚व्यावते त्वा त्वा वि॒श्वदे᳚व्यावते श्वा॒त्राः । \newline
42. वि॒श्वदे᳚व्यावते श्वा॒त्राः श्वा॒त्रा वि॒श्वदे᳚व्यावते वि॒श्वदे᳚व्यावते श्वा॒त्राः स्थ॑ स्थ श्वा॒त्रा वि॒श्वदे᳚व्यावते वि॒श्वदे᳚व्यावते श्वा॒त्राः स्थ॑ । \newline
43. वि॒श्वदे᳚व्यावत॒ इति॑ वि॒श्वदे᳚व्य - व॒ते॒ । \newline
44. श्वा॒त्राः स्थ॑ स्थ श्वा॒त्राः श्वा॒त्राः स्थ॑ वृत्र॒तुरो॑ वृत्र॒तुरः॑ स्थ श्वा॒त्राः श्वा॒त्राः स्थ॑ वृत्र॒तुरः॑ । \newline
45. स्थ॒ वृ॒त्र॒तुरो॑ वृत्र॒तुरः॑ स्थ स्थ वृत्र॒तुरो॒ राधो॑गूर्ता॒ राधो॑गूर्ता वृत्र॒तुरः॑ स्थ स्थ वृत्र॒तुरो॒ राधो॑गूर्ताः । \newline
46. वृ॒त्र॒तुरो॒ राधो॑गूर्ता॒ राधो॑गूर्ता वृत्र॒तुरो॑ वृत्र॒तुरो॒ राधो॑गूर्ता अ॒मृत॑ स्या॒मृत॑स्य॒ राधो॑गूर्ता वृत्र॒तुरो॑ वृत्र॒तुरो॒ राधो॑गूर्ता अ॒मृत॑स्य । \newline
47. वृ॒त्र॒तुर॒ इति॑ वृत्र - तुरः॑ । \newline
48. राधो॑गूर्ता अ॒मृत॑ स्या॒मृत॑स्य॒ राधो॑गूर्ता॒ राधो॑गूर्ता अ॒मृत॑स्य॒ पत्नीः॒ पत्नी॑ र॒मृत॑स्य॒ राधो॑गूर्ता॒ राधो॑गूर्ता अ॒मृत॑स्य॒ पत्नीः᳚ । \newline
49. राधो॑गूर्ता॒ इति॒ राधः॑ - गू॒र्ताः॒ । \newline
50. अ॒मृत॑स्य॒ पत्नीः॒ पत्नी॑ र॒मृत॑ स्या॒मृत॑स्य॒ पत्नी॒ स्ता स्ताः पत्नी॑ र॒मृत॑ स्या॒मृत॑स्य॒ पत्नी॒ स्ताः । \newline
51. पत्नी॒ स्ता स्ताः पत्नीः॒ पत्नी॒ स्ता दे॑वीर् देवी॒ स्ताः पत्नीः॒ पत्नी॒ स्ता दे॑वीः । \newline
52. ता दे॑वीर् देवी॒ स्ता स्ता दे॑वीर् देव॒त्रा दे॑व॒त्रा दे॑वी॒ स्ता स्ता दे॑वीर् देव॒त्रा । \newline
53. दे॒वी॒र् दे॒व॒त्रा दे॑व॒त्रा दे॑वीर् देवीर् देव॒त्रेम मि॒मम् दे॑व॒त्रा दे॑वीर् देवीर् देव॒त्रेमम् । \newline
54. दे॒व॒त्रेम मि॒मम् दे॑व॒त्रा दे॑व॒त्रेमं ॅय॒ज्ञ्ं ॅय॒ज्ञ् मि॒मम् दे॑व॒त्रा दे॑व॒त्रेमं ॅय॒ज्ञ्म् । \newline
55. दे॒व॒त्रेति॑ देव - त्रा । \newline
56. इ॒मं ॅय॒ज्ञ्ं ॅय॒ज्ञ् मि॒म मि॒मं ॅय॒ज्ञ्म् ध॑त्त धत्त य॒ज्ञ् मि॒म मि॒मं ॅय॒ज्ञ्म् ध॑त्त । \newline
57. य॒ज्ञ्म् ध॑त्त धत्त य॒ज्ञ्ं ॅय॒ज्ञ्म् ध॒त्तोप॑हूता॒ उप॑हूता धत्त य॒ज्ञ्ं ॅय॒ज्ञ्म् ध॒त्तोप॑हूताः । \newline
58. ध॒त्तोप॑हूता॒ उप॑हूता धत्त ध॒त्तोप॑हूताः॒ सोम॑स्य॒ सोम॒स्योप॑हूता धत्त ध॒त्तोप॑हूताः॒ सोम॑स्य । \newline
59. उप॑हूताः॒ सोम॑स्य॒ सोम॒स्योप॑हूता॒ उप॑हूताः॒ सोम॑स्य पिबत पिबत॒ सोम॒स्योप॑हूता॒ उप॑हूताः॒ सोम॑स्य पिबत । \newline
60. उप॑हूता॒ इत्युप॑ - हू॒ताः॒ । \newline
61. सोम॑स्य पिबत पिबत॒ सोम॑स्य॒ सोम॑स्य पिब॒तोप॑हूत॒ उप॑हूतः पिबत॒ सोम॑स्य॒ सोम॑स्य पिब॒तोप॑हूतः । \newline
62. पि॒ब॒तोप॑हूत॒ उप॑हूतः पिबत पिब॒तोप॑हूतो यु॒ष्माकं॑ ॅयु॒ष्माक॒ मुप॑हूतः पिबत पिब॒तोप॑हूतो यु॒ष्माक᳚म् । \newline
63. उप॑हूतो यु॒ष्माकं॑ ॅयु॒ष्माक॒ मुप॑हूत॒ उप॑हूतो यु॒ष्माक॒(ग्म्॒) सोमः॒ सोमो॑ यु॒ष्माक॒ मुप॑हूत॒ उप॑हूतो यु॒ष्माक॒(ग्म्॒) सोमः॑ । \newline
64. उप॑हूत॒ इत्युप॑ - हू॒तः॒ । \newline
65. यु॒ष्माक॒(ग्म्॒) सोमः॒ सोमो॑ यु॒ष्माकं॑ ॅयु॒ष्माक॒(ग्म्॒) सोमः॑ पिबतु पिबतु॒ सोमो॑ यु॒ष्माकं॑ ॅयु॒ष्माक॒(ग्म्॒) सोमः॑ पिबतु । \newline
\pagebreak
\markright{ TS 1.4.1.2  \hfill https://www.vedavms.in \hfill}
\addcontentsline{toc}{section}{ TS 1.4.1.2 }
\section*{ TS 1.4.1.2 }

\textbf{TS 1.4.1.2 } \newline
\textbf{Samhita Paata} \newline

सोमः॑ पिबतु॒ यत्ते॑ सोम दि॒वि ज्योति॒र्यत् पृ॑थि॒व्यां ॅयदु॒राव॒न्तरि॑क्षे॒ तेना॒स्मै यज॑मानायो॒रु रा॒या कृ॒द्ध्यधि॑ दा॒त्रे वो॑चो॒ धिष॑णे वी॒डू स॒ती वी॑डयेथा॒-मूर्जं॑ दधाथा॒मूर्जं॑ मे धत्तं॒ मा वाꣳ॑ हिꣳसिषं॒ मा मा॑ हिꣳसिष्टं॒ प्रागपा॒गुद॑गध॒राक्तास्त्वा॒ दिश॒ आ धा॑व॒न्त्वंब॒ नि ष्व॑र । यत्ते॑ ( ) सो॒माऽदा᳚भ्यं॒ नाम॒ जागृ॑वि॒ तस्मै॑ ते सोम॒ सोमा॑य॒ स्वाहा᳚ ॥ \newline

\textbf{Pada Paata} \newline

सोमः॑ । पि॒ब॒तु॒ । यत् । ते॒ । सो॒म॒ । दि॒वि । ज्योतिः॑ । यत् । पृ॒थि॒व्याम् । यत् । उ॒रौ । अ॒न्तरि॑क्षे । तेन॑ । अ॒स्मै । यज॑मानाय । उ॒रु । रा॒या । कृ॒धि॒ । अधीति॑ । दा॒त्रे । वो॒चः॒ । धिष॑णे॒ इति॑ । वी॒डू इति॑ । स॒ती इति॑ । वी॒ड॒ये॒था॒म् । ऊर्ज᳚म् । द॒धा॒था॒म् । ऊर्ज᳚म् । मे॒ । ध॒त्त॒म् । मा । वा॒म् । हिꣳ॒॒॒सि॒ष॒म् । मा । मा॒ । हिꣳ॒॒सि॒ष्ट॒म् । प्राक् । अपा᳚क् । उद॑क् । अ॒ध॒राक् । ताः । त्वा॒ । दिशः॑ । एति॑ । धा॒व॒न्तु॒ । अबं॑ । नीति॑ । स्व॒र॒ ॥ यत् । ते॒ ( ) । सो॒म॒ । अदा᳚भ्यम् । नाम॑ । जागृ॑वि । तस्मै᳚ । ते॒ । सो॒म॒ । सोमा॑य । स्वाहा᳚ ॥  \newline


\textbf{Krama Paata} \newline

सोमः॑ पिबतु । पि॒ब॒तु॒ यत् । यत् ते᳚ । ते॒ सो॒म॒ । सो॒म॒ दि॒वि । दि॒वि ज्योतिः॑ । ज्योति॒र् यत् । यत् पृ॑थि॒व्याम् । पृ॒थि॒व्यां ॅयत् । यदु॒रौ । उ॒राव॒न्तरि॑क्षे । अ॒न्तरि॑क्षे॒ तेन॑ । तेना॒स्मै । अ॒स्मै यज॑मानाय । यज॑मानायो॒रु । उ॒रु रा॒या । रा॒या कृ॑धि । कृ॒ध्यधि॑ । अधि॑ दा॒त्रे । दा॒त्रे वो॑चः । वो॒चो॒ धिष॑णे । धिष॑णे वी॒डू । धिष॑णे॒ इति॒ धिष॑णे । वी॒डू स॒ती । वी॒डू इति॑ वी॒डू । स॒ती वी॑डयेथाम् । स॒ती इति॑ स॒ती । वी॒ड॒ये॒था॒मूर्ज᳚म् । ऊर्ज॑म् दधाथाम् । द॒धा॒था॒मूर्ज᳚म् । ऊर्ज॑म् मे । मे॒ ध॒त्त॒म् । ध॒त्त॒म् मा । मा वा᳚म् । वाꣳ॒॒ हिꣳ॒॒सि॒ष॒म् । हिꣳ॒॒सि॒ष॒म् मा । मा मा᳚ । मा॒ हिꣳ॒॒सि॒ष्ट॒म् । हिꣳ॒॒सि॒ष्ट॒म् प्राक् । प्रागपा᳚क् । अपा॒गुद॑क् । उद॑गध॒राक् । अ॒ध॒राक्ताः । ता स्त्वा᳚ । त्वा॒ दिशः॑ । दिश॒ आ । आ धा॑वन्तु । धा॒व॒न्त्वम्ब॑ । अम्ब॒ नि । नि ष्व॑र । स्व॒रेति॑ स्वर ॥ यत् ते᳚ ( ) । ते॒ सो॒म॒ । सो॒मादा᳚भ्यम् । अदा᳚भ्य॒म् नाम॑ । नाम॒ जागृ॑वि । जागृ॑वि॒ तस्मै᳚ । तस्मै॑ ते । ते॒ सो॒म॒ । सो॒म॒ सोमा॑य । सोमा॑य॒ स्वाहा᳚ । स्वाहेति॒ स्वाहा᳚ । \newline

\textbf{Jatai Paata} \newline

1. सोमः॑ पिबतु पिबतु॒ सोमः॒ सोमः॑ पिबतु । \newline
2. पि॒ब॒तु॒ यद् यत् पि॑बतु पिबतु॒ यत् । \newline
3. यत् ते॑ ते॒ यद् यत् ते᳚ । \newline
4. ते॒ सो॒म॒ सो॒म॒ ते॒ ते॒ सो॒म॒ । \newline
5. सो॒म॒ दि॒वि दि॒वि सो॑म सोम दि॒वि । \newline
6. दि॒वि ज्योति॒र् ज्योति॑र् दि॒वि दि॒वि ज्योतिः॑ । \newline
7. ज्योति॒र् यद् यज् ज्योति॒र् ज्योति॒र् यत् । \newline
8. यत् पृ॑थि॒व्याम् पृ॑थि॒व्यां ॅयद् यत् पृ॑थि॒व्याम् । \newline
9. पृ॒थि॒व्यां ॅयद् यत् पृ॑थि॒व्याम् पृ॑थि॒व्यां ॅयत् । \newline
10. यदु॒रा वु॒रौ यद् यदु॒रौ । \newline
11. उ॒रा व॒न्तरि॑क्षे॒ ऽन्तरि॑क्ष उ॒रा वु॒रा व॒न्तरि॑क्षे । \newline
12. अ॒न्तरि॑क्षे॒ तेन॒ तेना॒न्तरि॑क्षे॒ ऽन्तरि॑क्षे॒ तेन॑ । \newline
13. तेना॒स्मा अ॒स्मै तेन॒ तेना॒स्मै । \newline
14. अ॒स्मै यज॑मानाय॒ यज॑मानाया॒स्मा अ॒स्मै यज॑मानाय । \newline
15. यज॑मानायो॒रू॑रु यज॑मानाय॒ यज॑मानायो॒रु । \newline
16. उ॒रु रा॒या रा॒योरू॑रु रा॒या । \newline
17. रा॒या कृ॑धि कृधि रा॒या रा॒या कृ॑धि । \newline
18. कृ॒ध्यध्यधि॑ कृधि कृ॒ध्यधि॑ । \newline
19. अधि॑ दा॒त्रे दा॒त्रे ऽध्यधि॑ दा॒त्रे । \newline
20. दा॒त्रे वो॑चो वोचो दा॒त्रे दा॒त्रे वो॑चः । \newline
21. वो॒चो॒ धिष॑णे॒ धिष॑णे वोचो वोचो॒ धिष॑णे । \newline
22. धिष॑णे वी॒डू वी॒डू धिष॑णे॒ धिष॑णे वी॒डू । \newline
23. धिष॑णे॒ इति॒ धिष॑णे । \newline
24. वी॒डू स॒ती स॒ती वी॒डू वी॒डू स॒ती । \newline
25. वी॒डू इति॑ वी॒डू । \newline
26. स॒ती वी॑डयेथां ॅवीडयेथाꣳ स॒ती स॒ती वी॑डयेथाम् । \newline
27. स॒ती इति॑ स॒ती । \newline
28. वी॒ड॒ये॒था॒ मूर्ज॒ मूर्जं॑ ॅवीडयेथां ॅवीडयेथा॒ मूर्ज᳚म् । \newline
29. ऊर्ज॑म् दधाथाम् दधाथा॒ मूर्ज॒ मूर्ज॑म् दधाथाम् । \newline
30. द॒धा॒था॒ मूर्ज॒ मूर्ज॑म् दधाथाम् दधाथा॒ मूर्ज᳚म् । \newline
31. ऊर्ज॑म् मे म॒ ऊर्ज॒ मूर्ज॑म् मे । \newline
32. मे॒ ध॒त्त॒म् ध॒त्त॒म् मे॒ मे॒ ध॒त्त॒म् । \newline
33. ध॒त्त॒म् मा मा ध॑त्तम् धत्त॒म् मा । \newline
34. मा वां᳚ ॅवा॒म् मा मा वा᳚म् । \newline
35. वा॒(ग्म्॒) हि॒(ग्म्॒)सि॒ष॒(ग्म्॒) हि॒(ग्म्॒)सि॒षं॒ ॅवां॒ ॅवा॒(ग्म्॒) हि॒(ग्म्॒)सि॒ष॒म् । \newline
36. हि॒(ग्म्॒)सि॒ष॒म् मा मा हि(ग्म्॑)सिषꣳ हिꣳसिष॒म् मा । \newline
37. मा मा॑ मा॒ मा मा मा᳚ । \newline
38. मा॒ हि॒(ग्म्॒)सि॒ष्ट॒(ग्म्॒) हि॒(ग्म्॒)सि॒ष्ट॒म् मा॒ मा॒ हि॒(ग्म्॒)सि॒ष्ट॒म् । \newline
39. हि॒(ग्म्॒)सि॒ष्ट॒म् प्राक् प्राग्घि(ग्म्॑)सिष्टꣳ हिꣳसिष्ट॒म् प्राक् । \newline
40. प्रागपा॒गपा॒क् प्राक् प्रागपा᳚क् । \newline
41. अपा॒ गुद॒ गुद॒ गपा॒ गपा॒ गुद॑क् । \newline
42. उद॑ गध॒रा ग॑ध॒रा गुद॒ गुद॑ गध॒राक् । \newline
43. अ॒ध॒राक् तास्ता अ॑ध॒रा ग॑ध॒राक् ताः । \newline
44. ता स्त्वा᳚ त्वा॒ ता स्ता स्त्वा᳚ । \newline
45. त्वा॒ दिशो॒ दिश॑ स्त्वा त्वा॒ दिशः॑ । \newline
46. दिश॒ आ दिशो॒ दिश॒ आ । \newline
47. आ धा॑वन्तु धाव॒न्त्वा धा॑वन्तु । \newline
48. धा॒व॒न्त्वंबांब॑ धावन्तु धाव॒न्त्वंब॑ । \newline
49. अंब॒ नि न्यंबांब॒ नि । \newline
50. नि ष्व॑र स्वर॒ नि नि ष्व॑र । \newline
51. स्व॒रेति॑ स्वर । \newline
52. यत् ते॑ ते॒ यद् यत् ते᳚ । \newline
53. ते॒ सो॒म॒ सो॒म॒ ते॒ ते॒ सो॒म॒ । \newline
54. सो॒मादा᳚भ्य॒ मदा᳚भ्यꣳ सोम सो॒मादा᳚भ्यम् । \newline
55. अदा᳚भ्य॒ न्नाम॒ नामादा᳚भ्य॒ मदा᳚भ्य॒ न्नाम॑ । \newline
56. नाम॒ जागृ॑वि॒ जागृ॑वि॒ नाम॒ नाम॒ जागृ॑वि । \newline
57. जागृ॑वि॒ तस्मै॒ तस्मै॒ जागृ॑वि॒ जागृ॑वि॒ तस्मै᳚ । \newline
58. तस्मै॑ ते ते॒ तस्मै॒ तस्मै॑ ते । \newline
59. ते॒ सो॒म॒ सो॒म॒ ते॒ ते॒ सो॒म॒ । \newline
60. सो॒म॒ सोमा॑य॒ सोमा॑य सोम सोम॒ सोमा॑य । \newline
61. सोमा॑य॒ स्वाहा॒ स्वाहा॒ सोमा॑य॒ सोमा॑य॒ स्वाहा᳚ । \newline
62. स्वाहेति॒ स्वाहा᳚ । \newline

\textbf{Ghana Paata } \newline

1. सोमः॑ पिबतु पिबतु॒ सोमः॒ सोमः॑ पिबतु॒ यद् यत् पि॑बतु॒ सोमः॒ सोमः॑ पिबतु॒ यत् । \newline
2. पि॒ब॒तु॒ यद् यत् पि॑बतु पिबतु॒ यत् ते॑ ते॒ यत् पि॑बतु पिबतु॒ यत् ते᳚ । \newline
3. यत् ते॑ ते॒ यद् यत् ते॑ सोम सोम ते॒ यद् यत् ते॑ सोम । \newline
4. ते॒ सो॒म॒ सो॒म॒ ते॒ ते॒ सो॒म॒ दि॒वि दि॒वि सो॑म ते ते सोम दि॒वि । \newline
5. सो॒म॒ दि॒वि दि॒वि सो॑म सोम दि॒वि ज्योति॒र् ज्योति॑र् दि॒वि सो॑म सोम दि॒वि ज्योतिः॑ । \newline
6. दि॒वि ज्योति॒र् ज्योति॑र् दि॒वि दि॒वि ज्योति॒र् यद् यज् ज्योति॑र् दि॒वि दि॒वि ज्योति॒र् यत् । \newline
7. ज्योति॒र् यद् यज् ज्योति॒र् ज्योति॒र् यत् पृ॑थि॒व्याम् पृ॑थि॒व्यां ॅयज् ज्योति॒र् ज्योति॒र् यत् पृ॑थि॒व्याम् । \newline
8. यत् पृ॑थि॒व्याम् पृ॑थि॒व्यां ॅयद् यत् पृ॑थि॒व्यां ॅयद् यत् पृ॑थि॒व्यां ॅयद् यत् पृ॑थि॒व्यां ॅयत् । \newline
9. पृ॒थि॒व्यां ॅयद् यत् पृ॑थि॒व्याम् पृ॑थि॒व्यां ॅयदु॒रा वु॒रौ यत् पृ॑थि॒व्याम् पृ॑थि॒व्यां ॅयदु॒रौ । \newline
10. यदु॒रा वु॒रौ यद् यदु॒रा व॒न्तरि॑क्षे॒ ऽन्तरि॑क्ष उ॒रौ यद् यदु॒रा व॒न्तरि॑क्षे । \newline
11. उ॒रा व॒न्तरि॑क्षे॒ ऽन्तरि॑क्ष उ॒रा वु॒रा व॒न्तरि॑क्षे॒ तेन॒ तेना॒न्तरि॑क्ष उ॒रा वु॒रा व॒न्तरि॑क्षे॒ तेन॑ । \newline
12. अ॒न्तरि॑क्षे॒ तेन॒ तेना॒न्तरि॑क्षे॒ ऽन्तरि॑क्षे॒ तेना॒स्मा अ॒स्मै तेना॒न्तरि॑क्षे॒ ऽन्तरि॑क्षे॒ तेना॒स्मै । \newline
13. तेना॒स्मा अ॒स्मै तेन॒ तेना॒स्मै यज॑मानाय॒ यज॑मानाया॒स्मै तेन॒ तेना॒स्मै यज॑मानाय । \newline
14. अ॒स्मै यज॑मानाय॒ यज॑मानाया॒स्मा अ॒स्मै यज॑माना यो॒रू॑रु यज॑मानाया॒स्मा अ॒स्मै यज॑मानायो॒रु । \newline
15. यज॑मानायो॒रू॑रु यज॑मानाय॒ यज॑मानायो॒रु रा॒या रा॒योरु यज॑मानाय॒ यज॑मानायो॒रु रा॒या । \newline
16. उ॒रु रा॒या रा॒योरू॑रु रा॒या कृ॑धि कृधि रा॒योरू॑रु रा॒या कृ॑धि । \newline
17. रा॒या कृ॑धि कृधि रा॒या रा॒या कृ॒ध्यध्यधि॑ कृधि रा॒या रा॒या कृ॒ध्यधि॑ । \newline
18. कृ॒ध्यध्यधि॑ कृधि कृ॒ध्यधि॑ दा॒त्रे दा॒त्रे ऽधि॑ कृधि कृ॒ध्यधि॑ दा॒त्रे । \newline
19. अधि॑ दा॒त्रे दा॒त्रे ऽध्यधि॑ दा॒त्रे वो॑चो वोचो दा॒त्रे ऽध्यधि॑ दा॒त्रे वो॑चः । \newline
20. दा॒त्रे वो॑चो वोचो दा॒त्रे दा॒त्रे वो॑चो॒ धिष॑णे॒ धिष॑णे वोचो दा॒त्रे दा॒त्रे वो॑चो॒ धिष॑णे । \newline
21. वो॒चो॒ धिष॑णे॒ धिष॑णे वोचो वोचो॒ धिष॑णे वी॒डू वी॒डू धिष॑णे वोचो वोचो॒ धिष॑णे वी॒डू । \newline
22. धिष॑णे वी॒डू वी॒डू धिष॑णे॒ धिष॑णे वी॒डू स॒ती स॒ती वी॒डू धिष॑णे॒ धिष॑णे वी॒डू स॒ती । \newline
23. धिष॑णे॒ इति॒ धिष॑णे । \newline
24. वी॒डू स॒ती स॒ती वी॒डू वी॒डू स॒ती वी॑डयेथां ॅवीडयेथाꣳ स॒ती वी॒डू वी॒डू स॒ती वी॑डयेथाम् । \newline
25. वी॒डू इति॑ वी॒डू । \newline
26. स॒ती वी॑डयेथां ॅवीडयेथाꣳ स॒ती स॒ती वी॑डयेथा॒ मूर्ज॒ मूर्जं॑ ॅवीडयेथाꣳ स॒ती स॒ती वी॑डयेथा॒ मूर्ज᳚म् । \newline
27. स॒ती इति॑ स॒ती । \newline
28. वी॒ड॒ये॒था॒ मूर्ज॒ मूर्जं॑ ॅवीडयेथां ॅवीडयेथा॒ मूर्ज॑म् दधाथाम् दधाथा॒ मूर्जं॑ ॅवीडयेथां ॅवीडयेथा॒ मूर्ज॑म् दधाथाम् । \newline
29. ऊर्ज॑म् दधाथाम् दधाथा॒ मूर्ज॒ मूर्ज॑म् दधाथा॒ मूर्ज॒ मूर्ज॑म् दधाथा॒ मूर्ज॒ मूर्ज॑म् दधाथा॒ मूर्ज᳚म् । \newline
30. द॒धा॒था॒ मूर्ज॒ मूर्ज॑म् दधाथाम् दधाथा॒ मूर्ज॑म् मे म॒ ऊर्ज॑म् दधाथाम् दधाथा॒ मूर्ज॑म् मे । \newline
31. ऊर्ज॑म् मे म॒ ऊर्ज॒ मूर्ज॑म् मे धत्तम् धत्तम् म॒ ऊर्ज॒ मूर्ज॑म् मे धत्तम् । \newline
32. मे॒ ध॒त्त॒म् ध॒त्त॒म् मे॒ मे॒ ध॒त्त॒म् मा मा ध॑त्तम् मे मे धत्त॒म् मा । \newline
33. ध॒त्त॒म् मा मा ध॑त्तम् धत्त॒म् मा वां᳚ ॅवा॒म् मा ध॑त्तम् धत्त॒म् मा वा᳚म् । \newline
34. मा वां᳚ ॅवा॒म् मा मा वा(ग्म्॑) हिꣳसिषꣳ हिꣳसिषं ॅवा॒म् मा मा वा(ग्म्॑) हिꣳसिषम् । \newline
35. वा॒(ग्म्॒) हिꣳ॒॒सि॒ष॒(ग्म्॒) हिꣳ॒॒सि॒षं॒ ॅवां॒ ॅवा॒(ग्म्॒) हिꣳ॒॒सि॒ष॒म् मा मा हि(ग्म्॑)सिषं ॅवां ॅवाꣳहिꣳसिष॒म् मा । \newline
36. हिꣳ॒॒सि॒ष॒म् मा मा हि(ग्म्॑)सिषꣳ हिꣳसिष॒म् मा मा॑ मा॒ मा हि(ग्म्॑)सिषꣳ हिꣳसिष॒म् मा मा᳚ । \newline
37. मा मा॑ मा॒ मा मा मा॑ हिꣳसिष्टꣳ हिꣳसिष्टम् मा॒ मा मा मा॑ हिꣳसिष्टम् । \newline
38. मा॒ हिꣳ॒॒सि॒ष्ट॒(ग्म्॒) हिꣳ॒॒सि॒ष्ट॒म् मा॒ मा॒ हिꣳ॒॒सि॒ष्ट॒म् प्राक् प्राग्घि(ग्म्॑)सिष्टम् मा मा हिꣳसिष्ट॒म् प्राक् । \newline
39. हिꣳ॒॒सि॒ष्ट॒म् प्राक् प्राग्घि(ग्म्॑)सिष्टꣳ हिꣳसिष्ट॒म् प्रागपा॒गपा॒क् प्राग्घि(ग्म्॑)सिष्टꣳ हिꣳसिष्ट॒म् प्रागपा᳚क् । \newline
40. प्रा गपा॒ गपा॒क् प्राक् प्रा गपा॒ गुद॒ गुद॒ गपा॒क् प्राक् प्रा गपा॒ गुद॑क् । \newline
41. अपा॒ गुद॒ गुद॒ गपा॒ गपा॒ गुद॑ गध॒रा ग॑ध॒रा गुद॒ गपा॒ गपा॒ गुद॑ गध॒राक् । \newline
42. उद॑ गध॒रा ग॑ध॒रा गुद॒ गुद॑ गध॒राक् ता स्ता अ॑ध॒रा गुद॒ गुद॑ गध॒राक् ताः । \newline
43. अ॒ध॒राक् ता स्ता अ॑ध॒ राग॑ध॒राक् ता स्त्वा᳚ त्वा॒ ता अ॑ध॒ राग॑ध॒राक् ता स्त्वा᳚ । \newline
44. ता स्त्वा᳚ त्वा॒ ता स्ता स्त्वा॒ दिशो॒ दिश॑ स्त्वा॒ ता स्ता स्त्वा॒ दिशः॑ । \newline
45. त्वा॒ दिशो॒ दिश॑ स्त्वा त्वा॒ दिश॒ आ दिश॑ स्त्वा त्वा॒ दिश॒ आ । \newline
46. दिश॒ आ दिशो॒ दिश॒ आ धा॑वन्तु धाव॒न्त्वा दिशो॒ दिश॒ आ धा॑वन्तु । \newline
47. आ धा॑वन्तु धाव॒न्त्वा धा॑व॒न् त्वम्बांब॑ धाव॒न्त्वा धा॑व॒न्त्वम्ब॑ । \newline
48. धा॒व॒न्त्वम्बांब॑ धावन्तु धाव॒न्त्वंब॒ नि न्यंब॑ धावन्तु धाव॒न्त्वंब॒ नि । \newline
49. अंब॒ नि न्यम्बांब॒ नि ष्व॑र स्वर॒ न्यम्बांब॒ नि ष्व॑र । \newline
50. नि ष्व॑र स्वर॒ नि नि ष्व॑र । \newline
51. स्व॒रेति॑ स्वर । \newline
52. यत् ते॑ ते॒ यद् यत् ते॑ सोम सोम ते॒ यद् यत् ते॑ सोम । \newline
53. ते॒ सो॒म॒ सो॒म॒ ते॒ ते॒ सो॒मादा᳚भ्य॒ मदा᳚भ्यꣳ सोम ते ते सो॒मादा᳚भ्यम् । \newline
54. सो॒मादा᳚भ्य॒ मदा᳚भ्यꣳ सोम सो॒मादा᳚भ्य॒म् नाम॒ नामादा᳚भ्यꣳ सोम सो॒मादा᳚भ्य॒म् नाम॑ । \newline
55. अदा᳚भ्य॒म् नाम॒ नामादा᳚भ्य॒ मदा᳚भ्य॒म् नाम॒ जागृ॑वि॒ जागृ॑वि॒ नामादा᳚भ्य॒ मदा᳚भ्य॒म् नाम॒ जागृ॑वि । \newline
56. नाम॒ जागृ॑वि॒ जागृ॑वि॒ नाम॒ नाम॒ जागृ॑वि॒ तस्मै॒ तस्मै॒ जागृ॑वि॒ नाम॒ नाम॒ जागृ॑वि॒ तस्मै᳚ । \newline
57. जागृ॑वि॒ तस्मै॒ तस्मै॒ जागृ॑वि॒ जागृ॑वि॒ तस्मै॑ ते ते॒ तस्मै॒ जागृ॑वि॒ जागृ॑वि॒ तस्मै॑ ते । \newline
58. तस्मै॑ ते ते॒ तस्मै॒ तस्मै॑ ते सोम सोम ते॒ तस्मै॒ तस्मै॑ ते सोम । \newline
59. ते॒ सो॒म॒ सो॒म॒ ते॒ ते॒ सो॒म॒ सोमा॑य॒ सोमा॑य सोम ते ते सोम॒ सोमा॑य । \newline
60. सो॒म॒ सोमा॑य॒ सोमा॑य सोम सोम॒ सोमा॑य॒ स्वाहा॒ स्वाहा॒ सोमा॑य सोम सोम॒ सोमा॑य॒ स्वाहा᳚ । \newline
61. सोमा॑य॒ स्वाहा॒ स्वाहा॒ सोमा॑य॒ सोमा॑य॒ स्वाहा᳚ । \newline
62. स्वाहेति॒ स्वाहा᳚ । \newline
\pagebreak
\markright{ TS 1.4.2.1  \hfill https://www.vedavms.in \hfill}
\addcontentsline{toc}{section}{ TS 1.4.2.1 }
\section*{ TS 1.4.2.1 }

\textbf{TS 1.4.2.1 } \newline
\textbf{Samhita Paata} \newline

वा॒चस्पत॑ये पवस्व वाजि॒न् वृषा॒ वृष्णो॑ अꣳ॒॒शुभ्यां॒ गभ॑स्तिपूतो दे॒वो दे॒वानां᳚ प॒वित्र॑मसि॒ येषां᳚ भा॒गोऽसि॒ तेभ्य॑स्त्वा॒ स्वांकृ॑तोऽसि॒ मधु॑मतीर् न॒ इष॑स्कृधि॒ विश्वे᳚भ्यस्त्वेन्द्रि॒येभ्यो॑ दि॒व्येभ्यः॒ पार्त्थि॑वेभ्यो॒ मन॑स्त्वा ऽष्टू॒र्व॑न्तरि॑क्ष॒-मन्वि॑हि॒ स्वाहा᳚ त्वा सुभ॒वः सूर्या॑य दे॒वेभ्य॑स्त्वा मरीचि॒पेभ्य॑ ए॒ष ते॒ योनिः॑ प्रा॒णाय॑ त्वा ॥ \newline

\textbf{Pada Paata} \newline

वा॒चः । पत॑ये । प॒व॒स्व॒ । वा॒जि॒न्न् । वृषा᳚ । वृष्णः॑ । अꣳ॒॒शुभ्या॒मित्यꣳ॒॒शु - भ्या॒म् । गभ॑स्तिपूत॒ इति॒ गभ॑स्ति - पू॒तः॒ । दे॒वः । दे॒वाना᳚म् । प॒वित्र᳚म् । अ॒सि॒ । येषा᳚म् । भा॒गः । असि॑ । तेभ्यः॑ । त्वा॒ । स्वाङ्कृ॑तः । अ॒सि॒ । मधु॑मती॒रिति॒ मधु॑ - म॒तीः॒ । नः॒ । इषः॑ । कृ॒धि॒ । विश्वे᳚भ्यः । त्वा॒ । इ॒न्द्रि॒येभ्यः॑ । दि॒व्येभ्यः॑ । पार्त्थि॑वेभ्यः । मनः॑ । त्वा॒ । अ॒ष्टु॒ । उ॒रु । अ॒न्तरि॑क्षम् । अन्विति॑ । इ॒हि॒ । स्वाहा᳚ । त्वा॒ । सु॒भ॒व॒ इति॑ सु - भ॒वः॒ । सूर्या॑य । दे॒वेभ्यः॑ । त्वा॒ । म॒री॒चि॒पेभ्य॒ इति॑ मरीचि - पेभ्यः॑ । ए॒षः । ते॒ । योनिः॑ । प्रा॒णायेति॑ प्र - अ॒नाय॑ । त्वा॒ ॥  \newline


\textbf{Krama Paata} \newline

वा॒चस्पत॑ये । पत॑ये पवस्व । प॒व॒स्व॒ वा॒जि॒न्न्॒ । वा॒जि॒न् वृषा᳚ । वृषा॒ वृष्णः॑ । वृष्णो॑ अꣳ॒॒शुभ्या᳚म् । अꣳ॒॒शुभ्या॒म् गभ॑स्तिपूतः । अꣳ॒॒शुभ्या॒मित्यꣳ॒॒शु - भ्या॒म् । गभ॑स्तिपूतो दे॒वः । गभ॑स्तिपूत॒ इति॒ गभ॑स्ति - पू॒तः॒ । दे॒वो दे॒वाना᳚म् । दे॒वाना᳚म् प॒वित्र᳚म् । प॒वित्र॑मसि । अ॒सि॒ येषा᳚म् । येषा᳚म् भा॒गः । भा॒गोऽसि॑ । असि॒ तेभ्यः॑ । तेभ्य॑स्त्वा । त्वा॒ स्वाङ्कृ॑तः । स्वाङ्कृ॑तोऽसि । अ॒सि॒ मधु॑मतीः । मधु॑मतीर् नः । मधु॑मती॒रिति॒ मधु॑ - म॒तीः॒ । न॒ इषः॑ । इष॑स्कृधि । कृ॒धि॒ विश्वे᳚भ्यः । विश्वे᳚भ्यस्त्वा । त्वे॒न्द्रि॒येभ्यः॑ । इ॒न्द्रि॒येभ्यो॑ दि॒व्येभ्यः॑ । दि॒व्येभ्यः॒ पार्त्थि॑वेभ्यः । पार्त्थि॑वेभ्यो॒ मनः॑ । मन॑स्त्वा । त्वा॒ऽष्टु॒ । अ॒ष्टू॒रु । उ॒र्व॑न्तरि॑क्षम् । अ॒न्तरि॑क्ष॒मनु॑ । अन्वि॑हि । इ॒हि॒ स्वाहा᳚ । स्वाहा᳚ त्वा । त्वा॒ सु॒भ॒वः॒ । सु॒भ॒वः॒ सूर्या॑य । सु॒भ॒व॒ इति॑ सु - भ॒वः॒ । सूर्या॑य दे॒वेभ्यः॑ । दे॒वेभ्य॑स्त्वा । त्वा॒ म॒री॒चि॒पेभ्यः॑ । म॒री॒चि॒पेभ्य॑ ए॒षः । म॒री॒चि॒पेभ्य॒ इति॑ मरीचि - पेभ्यः॑ । ए॒ष ते᳚ । ते॒ योनिः॑ । योनिः॑ प्रा॒णाय॑ । प्रा॒णाय॑ त्वा । प्रा॒णायेति॑ प्र - अ॒नाय॑ । त्वेति॑ त्वा । \newline

\textbf{Jatai Paata} \newline

1. वा॒च स्पत॑ये॒ पत॑ये वा॒चो वा॒च स्पत॑ये । \newline
2. पत॑ये पवस्व पवस्व॒ पत॑ये॒ पत॑ये पवस्व । \newline
3. प॒व॒स्व॒ वा॒जि॒न्॒. वा॒जि॒न् प॒व॒स्व॒ प॒व॒स्व॒ वा॒जि॒न्न् । \newline
4. वा॒जि॒न् वृषा॒ वृषा॑ वाजिन्. वाजि॒न् वृषा᳚ । \newline
5. वृषा॒ वृष्णो॒ वृष्णो॒ वृषा॒ वृषा॒ वृष्णः॑ । \newline
6. वृष्णो॑ अ॒(ग्म्॒)शुभ्या॑ म॒(ग्म्॒)शुभ्यां॒ ॅवृष्णो॒ वृष्णो॑ अ॒(ग्म्॒)शुभ्या᳚म् । \newline
7. अ॒(ग्म्॒)शुभ्या॒म् गभ॑स्तिपूतो॒ गभ॑स्तिपूतो॒(ग्म्॒) शुभ्या॑ म॒(ग्म्॒)शुभ्या॒म् गभ॑स्तिपूतः । \newline
8. अ॒(ग्म्॒)शुभ्या॒मित्य॒(ग्म्॒)शु - भ्या॒म् । \newline
9. गभ॑स्तिपूतो दे॒वो दे॒वो गभ॑स्तिपूतो॒ गभ॑स्तिपूतो दे॒वः । \newline
10. गभ॑स्तिपूत॒ इति॒ गभ॑स्ति - पू॒तः॒ । \newline
11. दे॒वो दे॒वाना᳚म् दे॒वाना᳚म् दे॒वो दे॒वो दे॒वाना᳚म् । \newline
12. दे॒वाना᳚म् प॒वित्र॑म् प॒वित्र॑म् दे॒वाना᳚म् दे॒वाना᳚म् प॒वित्र᳚म् । \newline
13. प॒वित्र॑ मस्यसि प॒वित्र॑म् प॒वित्र॑ मसि । \newline
14. अ॒सि॒ येषां॒ ॅयेषा॑ मस्यसि॒ येषा᳚म् । \newline
15. येषा᳚म् भा॒गो भा॒गो येषां॒ ॅयेषा᳚म् भा॒गः । \newline
16. भा॒गो ऽस्यसि॑ भा॒गो भा॒गो ऽसि॑ । \newline
17. असि॒ तेभ्य॒ स्तेभ्यो ऽस्यसि॒ तेभ्यः॑ । \newline
18. तेभ्य॑ स्त्वा त्वा॒ तेभ्य॒ स्तेभ्य॑ स्त्वा । \newline
19. त्वा॒ स्वाङ्कृ॑तः॒ स्वाङ्कृ॑त स्त्वा त्वा॒ स्वाङ्कृ॑तः । \newline
20. स्वाङ्कृ॑तो ऽस्यसि॒ स्वाङ्कृ॑तः॒ स्वाङ्कृ॑तो ऽसि । \newline
21. अ॒सि॒ मधु॑मती॒र् मधु॑मती रस्यसि॒ मधु॑मतीः । \newline
22. मधु॑मतीर् नो नो॒ मधु॑मती॒र् मधु॑मतीर् नः । \newline
23. मधु॑मती॒रिति॒ मधु॑ - म॒तीः॒ । \newline
24. न॒ इष॒ इषो॑ नो न॒ इषः॑ । \newline
25. इष॑स्कृधि कृ॒धीष॒ इष॑स्कृधि । \newline
26. कृ॒धि॒ विश्वे᳚भ्यो॒ विश्वे᳚भ्य स्कृधि कृधि॒ विश्वे᳚भ्यः । \newline
27. विश्वे᳚भ्य स्त्वा त्वा॒ विश्वे᳚भ्यो॒ विश्वे᳚भ्य स्त्वा । \newline
28. त्वे॒न्द्रि॒येभ्य॑ इन्द्रि॒येभ्य॑ स्त्वा त्वेन्द्रि॒येभ्यः॑ । \newline
29. इ॒न्द्रि॒येभ्यो॑ दि॒व्येभ्यो॑ दि॒व्येभ्य॑ इन्द्रि॒येभ्य॑ इन्द्रि॒येभ्यो॑ दि॒व्येभ्यः॑ । \newline
30. दि॒व्येभ्यः॒ पार्थि॑वेभ्यः॒ पार्थि॑वेभ्यो दि॒व्येभ्यो॑ दि॒व्येभ्यः॒ पार्थि॑वेभ्यः । \newline
31. पार्थि॑वेभ्यो॒ मनो॒ मनः॒ पार्थि॑वेभ्यः॒ पार्थि॑वेभ्यो॒ मनः॑ । \newline
32. मन॑ स्त्वा त्वा॒ मनो॒ मन॑ स्त्वा । \newline
33. त्वा॒ ऽष्ट्व॒ष्टु॒ त्वा॒ त्वा॒ ऽष्टु॒ । \newline
34. अ॒ष्टू॒रू᳚(1॒)र्व॑ष्ट्वष्टू॑रु । \newline
35. उ॒र्व॑न्तरि॑क्ष म॒न्तरि॑क्ष मु॒रू᳚(1॒)र्व॑न्तरि॑क्षम् । \newline
36. अ॒न्तरि॑क्ष॒ मन्वन्व॒न्तरि॑क्ष म॒न्तरि॑क्ष॒ मनु॑ । \newline
37. अन्वि॑ही॒ह्यन्वन्वि॑हि । \newline
38. इ॒हि॒ स्वाहा॒ स्वाहे॑हीहि॒ स्वाहा᳚ । \newline
39. स्वाहा᳚ त्वा त्वा॒ स्वाहा॒ स्वाहा᳚ त्वा । \newline
40. त्वा॒ सु॒भ॒वः॒ सु॒भ॒व॒ स्त्वा॒ त्वा॒ सु॒भ॒वः॒ । \newline
41. सु॒भ॒वः॒ सूर्या॑य॒ सूर्या॑य सुभवः सुभवः॒ सूर्या॑य । \newline
42. सु॒भ॒व॒ इति॑ सु - भ॒वः॒ । \newline
43. सूर्या॑य दे॒वेभ्यो॑ दे॒वेभ्यः॒ सूर्या॑य॒ सूर्या॑य दे॒वेभ्यः॑ । \newline
44. दे॒वेभ्य॑ स्त्वा त्वा दे॒वेभ्यो॑ दे॒वेभ्य॑ स्त्वा । \newline
45. त्वा॒ म॒री॒चि॒पेभ्यो॑ मरीचि॒पेभ्य॑ स्त्वा त्वा मरीचि॒पेभ्यः॑ । \newline
46. म॒री॒चि॒पेभ्य॑ ए॒ष ए॒ष म॑रीचि॒पेभ्यो॑ मरीचि॒पेभ्य॑ ए॒षः । \newline
47. म॒री॒चि॒पेभ्य॒ इति॑ मरीचि - पेभ्यः॑ । \newline
48. ए॒ष ते॑ त ए॒ष ए॒ष ते᳚ । \newline
49. ते॒ योनि॒र् योनि॑ स्ते ते॒ योनिः॑ । \newline
50. योनिः॑ प्रा॒णाय॑ प्रा॒णाय॒ योनि॒र् योनिः॑ प्रा॒णाय॑ । \newline
51. प्रा॒णाय॑ त्वा त्वा प्रा॒णाय॑ प्रा॒णाय॑ त्वा । \newline
52. प्रा॒णायेति॑ प्र - अ॒नाय॑ । \newline
53. त्वेति॑ त्वा । \newline

\textbf{Ghana Paata } \newline

1. वा॒च स्पत॑ये॒ पत॑ये वा॒चो वा॒च स्पत॑ये पवस्व पवस्व॒ पत॑ये वा॒चो वा॒च स्पत॑ये पवस्व । \newline
2. पत॑ये पवस्व पवस्व॒ पत॑ये॒ पत॑ये पवस्व वाजिन्. वाजिन् पवस्व॒ पत॑ये॒ पत॑ये पवस्व वाजिन्न् । \newline
3. प॒व॒स्व॒ वा॒जि॒न्॒. वा॒जि॒न् प॒व॒स्व॒ प॒व॒स्व॒ वा॒जि॒न् वृषा॒ वृषा॑ वाजिन् पवस्व पवस्व वाजि॒न् वृषा᳚ । \newline
4. वा॒जि॒न् वृषा॒ वृषा॑ वाजिन्. वाजि॒न् वृषा॒ वृष्णो॒ वृष्णो॒ वृषा॑ वाजिन्. वाजि॒न् वृषा॒ वृष्णः॑ । \newline
5. वृषा॒ वृष्णो॒ वृष्णो॒ वृषा॒ वृषा॒ वृष्णो॑ अꣳ॒॒शुभ्या॑ मꣳ॒॒शुभ्यां॒ ॅवृष्णो॒ वृषा॒ वृषा॒ वृष्णो॑ अꣳ॒॒शुभ्या᳚म् । \newline
6. वृष्णो॑ अꣳ॒॒शुभ्या॑ मꣳ॒॒शुभ्यां॒ ॅवृष्णो॒ वृष्णो॑ अꣳ॒॒शुभ्या॒म् गभ॑स्तिपूतो॒ गभ॑स्तिपूतो॒(ग्म्॒) शुभ्यां॒ ॅवृष्णो॒ वृष्णो॑ अꣳ॒॒शुभ्या॒म् गभ॑स्तिपूतः । \newline
7. अꣳ॒॒शुभ्या॒म् गभ॑स्तिपूतो॒ गभ॑स्तिपूतो॒(ग्म्॒) शुभ्या॑ मꣳ॒॒शुभ्या॒म् गभ॑स्तिपूतो दे॒वो दे॒वो गभ॑स्तिपूतो॒(ग्म्॒) शुभ्या॑ मꣳ॒॒शुभ्या॒म् गभ॑स्तिपूतो दे॒वः । \newline
8. अꣳ॒॒शुभ्या॒मित्यꣳ॒॒शु - भ्या॒म् । \newline
9. गभ॑स्तिपूतो दे॒वो दे॒वो गभ॑स्तिपूतो॒ गभ॑स्तिपूतो दे॒वो दे॒वाना᳚म् दे॒वाना᳚म् दे॒वो गभ॑स्तिपूतो॒ गभ॑स्तिपूतो दे॒वो दे॒वाना᳚म् । \newline
10. गभ॑स्तिपूत॒ इति॒ गभ॑स्ति - पू॒तः॒ । \newline
11. दे॒वो दे॒वाना᳚म् दे॒वाना᳚म् दे॒वो दे॒वो दे॒वाना᳚म् प॒वित्र॑म् प॒वित्र॑म् दे॒वाना᳚म् दे॒वो दे॒वो दे॒वाना᳚म् प॒वित्र᳚म् । \newline
12. दे॒वाना᳚म् प॒वित्र॑म् प॒वित्र॑म् दे॒वाना᳚म् दे॒वाना᳚म् प॒वित्र॑ मस्यसि प॒वित्र॑म् दे॒वाना᳚म् दे॒वाना᳚म् प॒वित्र॑ मसि । \newline
13. प॒वित्र॑ मस्यसि प॒वित्र॑म् प॒वित्र॑ मसि॒ येषां॒ ॅयेषा॑ मसि प॒वित्र॑म् प॒वित्र॑ मसि॒ येषा᳚म् । \newline
14. अ॒सि॒ येषां॒ ॅयेषा॑ मस्यसि॒ येषा᳚म् भा॒गो भा॒गो येषा॑ मस्यसि॒ येषा᳚म् भा॒गः । \newline
15. येषा᳚म् भा॒गो भा॒गो येषां॒ ॅयेषा᳚म् भा॒गो ऽस्यसि॑ भा॒गो येषां॒ ॅयेषा᳚म् भा॒गो ऽसि॑ । \newline
16. भा॒गो ऽस्यसि॑ भा॒गो भा॒गो ऽसि॒ तेभ्य॒ स्तेभ्यो ऽसि॑ भा॒गो भा॒गो ऽसि॒ तेभ्यः॑ । \newline
17. असि॒ तेभ्य॒ स्तेभ्यो ऽस्यसि॒ तेभ्य॑ स्त्वा त्वा॒ तेभ्यो ऽस्यसि॒ तेभ्य॑ स्त्वा । \newline
18. तेभ्य॑ स्त्वा त्वा॒ तेभ्य॒ स्तेभ्य॑ स्त्वा॒ स्वाङ्कृ॑तः॒ स्वाङ्कृ॑त स्त्वा॒ तेभ्य॒ स्तेभ्य॑ स्त्वा॒ स्वाङ्कृ॑तः । \newline
19. त्वा॒ स्वाङ्कृ॑तः॒ स्वाङ्कृ॑त स्त्वा त्वा॒ स्वाङ्कृ॑तो ऽस्यसि॒ स्वाङ्कृ॑त स्त्वा त्वा॒ स्वाङ्कृ॑तो ऽसि । \newline
20. स्वाङ्कृ॑तो ऽस्यसि॒ स्वाङ्कृ॑तः॒ स्वाङ्कृ॑तो ऽसि॒ मधु॑मती॒र् मधु॑मतीरसि॒ स्वाङ्कृ॑तः॒ स्वाङ्कृ॑तो ऽसि॒ मधु॑मतीः । \newline
21. अ॒सि॒ मधु॑मती॒र् मधु॑मतीरस्यसि॒ मधु॑मतीर् नो नो॒ मधु॑मती रस्यसि॒ मधु॑मतीर् नः । \newline
22. मधु॑मतीर् नो नो॒ मधु॑मती॒र् मधु॑मतीर् न॒ इष॒ इषो॑ नो॒ मधु॑मती॒र् मधु॑मतीर् न॒ इषः॑ । \newline
23. मधु॑मती॒रिति॒ मधु॑ - म॒तीः॒ । \newline
24. न॒ इष॒ इषो॑ नो न॒ इष॑ स्कृधि कृ॒धीषो॑ नो न॒ इष॑ स्कृधि । \newline
25. इष॑ स्कृधि कृ॒धीष॒ इष॑ स्कृधि॒ विश्वे᳚भ्यो॒ विश्वे᳚भ्य स्कृ॒धीष॒ इष॑ स्कृधि॒ विश्वे᳚भ्यः । \newline
26. कृ॒धि॒ विश्वे᳚भ्यो॒ विश्वे᳚भ्य स्कृधि कृधि॒ विश्वे᳚भ्य स्त्वा त्वा॒ विश्वे᳚भ्य स्कृधि कृधि॒ विश्वे᳚भ्य स्त्वा । \newline
27. विश्वे᳚भ्य स्त्वा त्वा॒ विश्वे᳚भ्यो॒ विश्वे᳚भ्य स्त्वेन्द्रि॒येभ्य॑ इन्द्रि॒येभ्य॑ स्त्वा॒ विश्वे᳚भ्यो॒ विश्वे᳚भ्य स्त्वेन्द्रि॒येभ्यः॑ । \newline
28. त्वे॒न्द्रि॒येभ्य॑ इन्द्रि॒येभ्य॑ स्त्वा त्वेन्द्रि॒येभ्यो॑ दि॒व्येभ्यो॑ दि॒व्येभ्य॑ इन्द्रि॒येभ्य॑ स्त्वा त्वेन्द्रि॒येभ्यो॑ दि॒व्येभ्यः॑ । \newline
29. इ॒न्द्रि॒येभ्यो॑ दि॒व्येभ्यो॑ दि॒व्येभ्य॑ इन्द्रि॒येभ्य॑ इन्द्रि॒येभ्यो॑ दि॒व्येभ्यः॒ पार्थि॑वेभ्यः॒ पार्थि॑वेभ्यो दि॒व्येभ्य॑ इन्द्रि॒येभ्य॑ इन्द्रि॒येभ्यो॑ दि॒व्येभ्यः॒ पार्थि॑वेभ्यः । \newline
30. दि॒व्येभ्यः॒ पार्थि॑वेभ्यः॒ पार्थि॑वेभ्यो दि॒व्येभ्यो॑ दि॒व्येभ्यः॒ पार्थि॑वेभ्यो॒ मनो॒ मनः॒ पार्थि॑वेभ्यो दि॒व्येभ्यो॑ दि॒व्येभ्यः॒ पार्थि॑वेभ्यो॒ मनः॑ । \newline
31. पार्थि॑वेभ्यो॒ मनो॒ मनः॒ पार्थि॑वेभ्यः॒ पार्थि॑वेभ्यो॒ मन॑ स्त्वा त्वा॒ मनः॒ पार्थि॑वेभ्यः॒ पार्थि॑वेभ्यो॒ मन॑ स्त्वा । \newline
32. मन॑ स्त्वा त्वा॒ मनो॒ मन॑ स्त्वा ऽष्ट्वष्टु त्वा॒ मनो॒ मन॑ स्त्वा ऽष्टु । \newline
33. त्वा॒ ऽष्ट्व॒ष्टु॒ त्वा॒ त्वा॒ ऽष्टू॒रू᳚(1॒) र्व॑ष्टु त्वा त्वा ऽष्टू॑रु । \newline
34. अ॒ष्टू॒रू᳚(1॒)र्व॑ष्ट्वष्टू॒र्व॑न्तरि॑क्ष म॒न्तरि॑क्ष मु॒र्व॑ष्ट्वष्टू॒र्व॑न्तरि॑क्षम् । \newline
35. उ॒र्व॑न्तरि॑क्ष म॒न्तरि॑क्ष मु॒रू᳚(1॒)र्व॑न्तरि॑क्ष॒ मन्वन्व॒न्तरि॑क्ष मु॒रू᳚(1॒)र्व॑न्तरि॑क्ष॒ मनु॑ । \newline
36. अ॒न्तरि॑क्ष॒ मन्वन्व॒न्तरि॑क्ष म॒न्तरि॑क्ष॒ मन्वि॑ही॒ ह्यन्व॒न्तरि॑क्ष म॒न्तरि॑क्ष॒ मन्वि॑हि । \newline
37. अन्वि॑ही॒ ह्यन्वन्वि॑हि॒ स्वाहा॒ स्वाहे॒ ह्यन्वन्वि॑हि॒ स्वाहा᳚ । \newline
38. इ॒हि॒ स्वाहा॒ स्वाहे॑हीहि॒ स्वाहा᳚ त्वा त्वा॒ स्वाहे॑हीहि॒ स्वाहा᳚ त्वा । \newline
39. स्वाहा᳚ त्वा त्वा॒ स्वाहा॒ स्वाहा᳚ त्वा सुभवः सुभव स्त्वा॒ स्वाहा॒ स्वाहा᳚ त्वा सुभवः । \newline
40. त्वा॒ सु॒भ॒वः॒ सु॒भ॒व॒ स्त्वा॒ त्वा॒ सु॒भ॒वः॒ सूर्या॑य॒ सूर्या॑य सुभव स्त्वा त्वा सुभवः॒ सूर्या॑य । \newline
41. सु॒भ॒वः॒ सूर्या॑य॒ सूर्या॑य सुभवः सुभवः॒ सूर्या॑य दे॒वेभ्यो॑ दे॒वेभ्यः॒ सूर्या॑य सुभवः सुभवः॒ सूर्या॑य दे॒वेभ्यः॑ । \newline
42. सु॒भ॒व॒ इति॑ सु - भ॒वः॒ । \newline
43. सूर्या॑य दे॒वेभ्यो॑ दे॒वेभ्यः॒ सूर्या॑य॒ सूर्या॑य दे॒वेभ्य॑ स्त्वा त्वा दे॒वेभ्यः॒ सूर्या॑य॒ सूर्या॑य दे॒वेभ्य॑ स्त्वा । \newline
44. दे॒वेभ्य॑ स्त्वा त्वा दे॒वेभ्यो॑ दे॒वेभ्य॑ स्त्वा मरीचि॒पेभ्यो॑ मरीचि॒पेभ्य॑ स्त्वा दे॒वेभ्यो॑ दे॒वेभ्य॑ स्त्वा मरीचि॒पेभ्यः॑ । \newline
45. त्वा॒ म॒री॒चि॒पेभ्यो॑ मरीचि॒पेभ्य॑ स्त्वा त्वा मरीचि॒पेभ्य॑ ए॒ष ए॒ष म॑रीचि॒पेभ्य॑ स्त्वा त्वा मरीचि॒पेभ्य॑ ए॒षः । \newline
46. म॒री॒चि॒पेभ्य॑ ए॒ष ए॒ष म॑रीचि॒पेभ्यो॑ मरीचि॒पेभ्य॑ ए॒ष ते॑ त ए॒ष म॑रीचि॒पेभ्यो॑ मरीचि॒पेभ्य॑ ए॒ष ते᳚ । \newline
47. म॒री॒चि॒पेभ्य॒ इति॑ मरीचि - पेभ्यः॑ । \newline
48. ए॒ष ते॑ त ए॒ष ए॒ष ते॒ योनि॒र् योनि॑ स्त ए॒ष ए॒ष ते॒ योनिः॑ । \newline
49. ते॒ योनि॒र् योनि॑ स्ते ते॒ योनिः॑ प्रा॒णाय॑ प्रा॒णाय॒ योनि॑ स्ते ते॒ योनिः॑ प्रा॒णाय॑ । \newline
50. योनिः॑ प्रा॒णाय॑ प्रा॒णाय॒ योनि॒र् योनिः॑ प्रा॒णाय॑ त्वा त्वा प्रा॒णाय॒ योनि॒र् योनिः॑ प्रा॒णाय॑ त्वा । \newline
51. प्रा॒णाय॑ त्वा त्वा प्रा॒णाय॑ प्रा॒णाय॑ त्वा । \newline
52. प्रा॒णायेति॑ प्र - अ॒नाय॑ । \newline
53. त्वेति॑ त्वा । \newline
\pagebreak
\markright{ TS 1.4.3.1  \hfill https://www.vedavms.in \hfill}
\addcontentsline{toc}{section}{ TS 1.4.3.1 }
\section*{ TS 1.4.3.1 }

\textbf{TS 1.4.3.1 } \newline
\textbf{Samhita Paata} \newline

उ॒प॒या॒मगृ॑हीतो ऽस्य॒न्तर्य॑च्छ मघवन् पा॒हि सोम॑मुरु॒ष्य रायः॒ समिषो॑ यजस्वा॒ऽन्तस्ते॑ दधामि॒ द्यावा॑पृथि॒वी अ॒न्तरु॒र्व॑न्तरि॑क्षꣳ स॒जोषा॑ दे॒वैरव॑रैः॒ परै᳚श्चाऽन्तर्या॒मे म॑घवन् मादयस्व॒ स्वांकृ॑तोऽसि॒ मधु॑मतीर्न॒ इष॑स्कृधि॒ विश्वे᳚भ्यस्त्वेन्द्रि॒येभ्यो॑ दि॒व्येभ्यः॒ पार्त्थि॑वेभ्यो॒ मन॑स्त्वाऽष्टू॒र्व॑न्तरि॑क्ष॒मन्वि॑हि॒ स्वाहा᳚ त्वा सुभवः॒ सूर्या॑य दे॒वेभ्य॑ ( ) स्त्वा मरीचि॒पेभ्य॑ ए॒ष ते॒ \newline

\textbf{Pada Paata} \newline

उ॒प॒या॒मगृ॑हीत॒ इत्यु॑पया॒म - गृ॒ही॒तः॒ । अ॒सि॒ । अ॒न्तः । य॒च्छ॒ । म॒घ॒व॒न्निति॑ मघ - व॒न्न् । पा॒हि । सोम᳚म् । उ॒रु॒ष्य । रायः॑ । समिति॑ । इषः॑ । य॒ज॒स्व॒ । अ॒न्तः । ते॒ । द॒धा॒मि॒ । द्यावा॑पृथि॒वी इति॒ द्यावा᳚ - पृ॒थि॒वी । अ॒न्तः । उ॒रु । अ॒न्तरि॑क्षम् । स॒जोषा॒ इति॑ स - जोषाः᳚ । दे॒वैः । अव॑रैः । परैः᳚ । च॒ । अ॒न्त॒र्या॒म इत्य॑न्तः-या॒मे । म॒घ॒व॒न्निति॑ मघ - व॒न्न् । मा॒द॒य॒स्व॒ । स्वाङ्कृ॑तः । अ॒सि॒ । मधु॑मती॒रिति॒ मधु॑ - म॒तीः॒ । नः॒ । इषः॑ । कृ॒धि॒ । विश्वे᳚भ्यः । त्वा॒ । इ॒न्द्रि॒येभ्यः॑ । दि॒व्येभ्यः॑ । पार्त्थि॑वेभ्यः । मनः॑ । त्वा॒ । अ॒ष्टु॒ । उ॒रु । अ॒न्तरि॑क्षम् । अन्विति॑ । इ॒हि॒ । स्वाहा᳚ । त्वा॒ । सु॒भ॒व॒ इति॑ सु - भ॒वः॒ । सूर्या॑य । दे॒वेभ्यः॑( ) । त्वा॒ । म॒री॒चि॒पेभ्य॒ इति॑ मरीचि - पेभ्यः॑ । ए॒षः । ते॒ । योनिः॑ । अ॒पा॒नायेत्य॑प - अ॒नाय॑ । त्वा॒ ॥  \newline


\textbf{Krama Paata} \newline

उ॒प॒या॒मगृ॑हीतोऽसि । उ॒प॒या॒मगृ॑हीत॒ इत्यु॑पया॒म - गृ॒ही॒तः॒ । अ॒स्य॒न्तः । अ॒न्तर्. य॑च्छ । य॒च्छ॒ म॒घ॒व॒न्न्॒ । म॒घ॒व॒न् पा॒हि । म॒घ॒न॒न्निति॑ मघ - व॒न्न्॒ । पा॒हि सोम᳚म् । सोम॑मुरु॒ष्य । उ॒रु॒ष्य रायः॑ । रायः॒ सम् । समिषः॑ । इषो॑ यजस्व । य॒ज॒स्वा॒न्तः । अ॒न्तस्ते᳚ । ते॒ द॒धा॒मि॒ । द॒धा॒मि॒ द्यावा॑पृथि॒वी । द्यावा॑पृथि॒वी अ॒न्तः । द्यावा॑पृथि॒वी इति॒ द्यावा᳚ - पृ॒थि॒वी । अ॒न्तरु॒रु । उ॒र्व॑न्तरि॑क्षम् । अ॒न्तरि॑क्षꣳ स॒जोषाः᳚ । स॒जोषा॑ दे॒वैः । स॒जोषा॒ इति॑ स - जोषा᳚ । दे॒वैरव॑रैः । अव॑रैः॒ परैः᳚ । परै᳚श्च । चा॒न्त॒र्या॒मे । अ॒न्त॒र्या॒मे म॑घवन्न् । अ॒न्त॒र्या॒म इत्य॑न्तः - या॒मे । म॒घ॒व॒न् मा॒द॒य॒स्व॒ । म॒घ॒व॒न्निति॑ मघ - व॒न्न्॒ । मा॒द॒य॒स्व॒ स्वाङ्कृ॑तः । स्वाङ्कृ॑तोऽसि । अ॒सि॒ मधु॑मतीः । मधु॑मतीर् नः । मधु॑मती॒रिति॒ मधु॑ - म॒तीः॒ । न॒ इषः॑ । इष॑स्कृधि । कृ॒धि॒ विश्वे᳚भ्यः । विश्वे᳚भ्यस्त्वा । त्वे॒न्द्रि॒येभ्यः॑ । इ॒न्द्रि॒येभ्यो॑ दि॒व्येभ्यः॑ । दि॒व्येभ्यः॒ पार्त्थि॑वेभ्यः । पार्त्थि॑वेभ्यो॒ मनः॑ । मन॑स्त्वा । त्वा॒ऽष्टु॒ । अ॒ष्टू॒रु । उ॒र्व॑न्तरि॑क्षम् । अ॒न्तरि॑क्ष॒मनु॑ । अन्वि॑हि । इ॒हि॒ स्वाहा᳚ । स्वाहा᳚ त्वा । त्वा॒ सु॒भ॒वः॒ । सु॒भ॒वः॒ सूर्या॑य । सु॒भ॒व॒ इति॑ सु - भ॒वः॒ । सूर्या॑य दे॒वेभ्यः॑ ( ) । दे॒वेभ्य॑स्त्वा । त्वा॒ म॒री॒चि॒पेभ्यः॑ । म॒री॒चि॒पेभ्य॑ ए॒षः । म॒री॒चि॒पेभ्य॒ इति॑ मरीचि - पेभ्यः॑ । ए॒ष ते᳚ । ते॒ योनिः॑ । योनि॑रपा॒नाय॑ । 
अ॒पा॒नाय॑ त्वा । अ॒पा॒नायेत्य॑प - अ॒नाय॑ । त्वेति॑ त्वा । \newline

\textbf{Jatai Paata} \newline

1. उ॒प॒या॒मगृ॑हीतो ऽस्यस्युपया॒मगृ॑हीत उपया॒मगृ॑हीतो ऽसि । \newline
2. उ॒प॒या॒मगृ॑हीत॒ इत्यु॑पया॒म - गृ॒ही॒तः॒ । \newline
3. अ॒स्य॒न्त र॒न्त र॑स्यस्य॒न्तः । \newline
4. अ॒न्तर् य॑च्छ यच्छा॒न्त र॒न्तर् य॑च्छ । \newline
5. य॒च्छ॒ म॒घ॒व॒न् म॒घ॒व॒न्॒. य॒च्छ॒ य॒च्छ॒ म॒घ॒व॒न्न् । \newline
6. म॒घ॒व॒न् पा॒हि पा॒हि म॑घवन् मघवन् पा॒हि । \newline
7. म॒घ॒व॒न्निति॑ मघ - व॒न्न् । \newline
8. पा॒हि सोम॒(ग्म्॒) सोम॑म् पा॒हि पा॒हि सोम᳚म् । \newline
9. सोम॑ मुरु॒ष्योरु॒ष्य सोम॒(ग्म्॒) सोम॑ मुरु॒ष्य । \newline
10. उ॒रु॒ष्य रायो॒ राय॑ उरु॒ष्योरु॒ष्य रायः॑ । \newline
11. रायः॒ सꣳ सꣳ रायो॒ रायः॒ सम् । \newline
12. स मिष॒ इषः॒ सꣳ स मिषः॑ । \newline
13. इषो॑ यजस्व यज॒स्वेष॒ इषो॑ यजस्व । \newline
14. य॒ज॒स्वा॒न्त र॒न्तर् य॑जस्व यजस्वा॒न्तः । \newline
15. अ॒न्त स्ते॑ ते॒ ऽन्त र॒न्त स्ते᳚ । \newline
16. ते॒ द॒धा॒मि॒ द॒धा॒मि॒ ते॒ ते॒ द॒धा॒मि॒ । \newline
17. द॒धा॒मि॒ द्यावा॑पृथि॒वी द्यावा॑पृथि॒वी द॑धामि दधामि॒ द्यावा॑पृथि॒वी । \newline
18. द्यावा॑पृथि॒वी अ॒न्त र॒न्तर् द्यावा॑पृथि॒वी द्यावा॑पृथि॒वी अ॒न्तः । \newline
19. द्यावा॑पृथि॒वी इति॒ द्यावा᳚ - पृ॒थि॒वी । \newline
20. अ॒न्त रु॒रू᳚(1॒)र्व॑न्त र॒न्त रु॒रु । \newline
21. उ॒र्व॑न्तरि॑क्ष म॒न्तरि॑क्ष मु॒रू᳚(1॒)र्व॑न्तरि॑क्षम् । \newline
22. अ॒न्तरि॑क्षꣳ स॒जोषाः᳚ स॒जोषा॑ अ॒न्तरि॑क्ष म॒न्तरि॑क्षꣳ स॒जोषाः᳚ । \newline
23. स॒जोषा॑ दे॒वैर् दे॒वैः स॒जोषाः᳚ स॒जोषा॑ दे॒वैः । \newline
24. स॒जोषा॒ इति॑ स - जोषाः᳚ । \newline
25. दे॒वै रव॑रै॒ रव॑रैर् दे॒वैर् दे॒वैरव॑रैः । \newline
26. अव॑रैः॒ परैः॒ परै॒ रव॑रै॒ रव॑रैः॒ परैः᳚ । \newline
27. परै᳚श्च च॒ परैः॒ परै᳚श्च । \newline
28. चा॒न्त॒र्या॒मे᳚ ऽन्तर्या॒मे च॑ चान्तर्या॒मे । \newline
29. अ॒न्त॒र्या॒मे म॑घवन् मघवन् नन्तर्या॒मे᳚ ऽन्तर्या॒मे म॑घवन्न् । \newline
30. अ॒न्त॒र्या॒म इत्य॑न्तः - या॒मे । \newline
31. म॒घ॒व॒न् मा॒द॒य॒स्व॒ मा॒द॒य॒स्व॒ म॒घ॒व॒न् म॒घ॒व॒न् मा॒द॒य॒स्व॒ । \newline
32. म॒घ॒व॒न्निति॑ मघ - व॒न्न् । \newline
33. मा॒द॒य॒स्व॒ स्वाङ्कृ॑तः॒ स्वाङ्कृ॑तो मादयस्व मादयस्व॒ स्वाङ्कृ॑तः । \newline
34. स्वाङ्कृ॑तो ऽस्यसि॒ स्वाङ्कृ॑तः॒ स्वाङ्कृ॑तो ऽसि । \newline
35. अ॒सि॒ मधु॑मती॒र् मधु॑मती रस्यसि॒ मधु॑मतीः । \newline
36. मधु॑मतीर् नो नो॒ मधु॑मती॒र् मधु॑मतीर् नः । \newline
37. मधु॑मती॒रिति॒ मधु॑ - म॒तीः॒ । \newline
38. न॒ इष॒ इषो॑ नो न॒ इषः॑ । \newline
39. इष॑स्कृधि कृ॒धीष॒ इष॑स्कृधि । \newline
40. कृ॒धि॒ विश्वे᳚भ्यो॒ विश्वे᳚भ्यस्कृधि कृधि॒ विश्वे᳚भ्यः । \newline
41. विश्वे᳚भ्य स्त्वा त्वा॒ विश्वे᳚भ्यो॒ विश्वे᳚भ्य स्त्वा । \newline
42. त्वे॒न्द्रि॒येभ्य॑ इन्द्रि॒येभ्य॑ स्त्वा त्वेन्द्रि॒येभ्यः॑ । \newline
43. इ॒न्द्रि॒येभ्यो॑ दि॒व्येभ्यो॑ दि॒व्येभ्य॑ इन्द्रि॒येभ्य॑ इन्द्रि॒येभ्यो॑ दि॒व्येभ्यः॑ । \newline
44. दि॒व्येभ्यः॒ पार्थि॑वेभ्यः॒ पार्थि॑वेभ्यो दि॒व्येभ्यो॑ दि॒व्येभ्यः॒ पार्थि॑वेभ्यः । \newline
45. पार्थि॑वेभ्यो॒ मनो॒ मनः॒ पार्थि॑वेभ्यः॒ पार्थि॑वेभ्यो॒ मनः॑ । \newline
46. मन॑ स्त्वा त्वा॒ मनो॒ मन॑ स्त्वा । \newline
47. त्वा॒ ऽष्ट्व॒ष्टु॒ त्वा॒ त्वा॒ ऽष्टु॒ । \newline
48. अ॒ष्टू॒रू᳚(1॒)र्व॑ष्ट्वष्टू॑रु । \newline
49. उ॒र्व॑न्तरि॑क्ष म॒न्तरि॑क्ष मु॒रू᳚(1॒)र्व॑न्तरि॑क्षम् । \newline
50. अ॒न्तरि॑क्ष॒ मन्वन्व॒न्तरि॑क्ष म॒न्तरि॑क्ष॒ मनु॑ । \newline
51. अन्वि॑ ही॒ह्यन्वन् वि॑हि । \newline
52. इ॒हि॒ स्वाहा॒ स्वाहे॑हीहि॒ स्वाहा᳚ । \newline
53. स्वाहा᳚ त्वा त्वा॒ स्वाहा॒ स्वाहा᳚ त्वा । \newline
54. त्वा॒ सु॒भ॒वः॒ सु॒भ॒व॒ स्त्वा॒ त्वा॒ सु॒भ॒वः॒ । \newline
55. सु॒भ॒वः॒ सूर्या॑य॒ सूर्या॑य सुभवः सुभवः॒ सूर्या॑य । \newline
56. सु॒भ॒व॒ इति॑ सु - भ॒वः॒ । \newline
57. सूर्या॑य दे॒वेभ्यो॑ दे॒वेभ्यः॒ सूर्या॑य॒ सूर्या॑य दे॒वेभ्यः॑ । \newline
58. दे॒वेभ्य॑ स्त्वा त्वा दे॒वेभ्यो॑ दे॒वेभ्य॑ स्त्वा । \newline
59. त्वा॒ म॒री॒चि॒पेभ्यो॑ मरीचि॒पेभ्य॑ स्त्वा त्वा मरीचि॒पेभ्यः॑ । \newline
60. म॒री॒चि॒पेभ्य॑ ए॒ष ए॒ष म॑रीचि॒पेभ्यो॑ मरीचि॒पेभ्य॑ ए॒षः । \newline
61. म॒री॒चि॒पेभ्य॒ इति॑ मरीचि - पेभ्यः॑ । \newline
62. ए॒ष ते॑ त ए॒ष ए॒ष ते᳚ । \newline
63. ते॒ योनि॒र् योनि॑ स्ते ते॒ योनिः॑ । \newline
64. योनि॑ रपा॒नाया॑पा॒नाय॒ योनि॒र् योनि॑ रपा॒नाय॑ । \newline
65. अ॒पा॒नाय॑ त्वा त्वा ऽपा॒नाया॑पा॒नाय॑ त्वा । \newline
66. अ॒पा॒नायेत्य॑प - अ॒नाय॑ । \newline
67. त्वेति॑ त्वा । \newline

\textbf{Ghana Paata } \newline

1. उ॒प॒या॒मगृ॑हीतो ऽस्यस्युपया॒मगृ॑हीत उपया॒मगृ॑हीतो ऽस्य॒न्त र॒न्त र॑स्युपया॒मगृ॑हीत उपया॒मगृ॑हीतो ऽस्य॒न्तः । \newline
2. उ॒प॒या॒मगृ॑हीत॒ इत्यु॑पया॒म - गृ॒ही॒तः॒ । \newline
3. अ॒स्य॒न्त र॒न्त र॑स्य स्य॒न्तर् य॑च्छ यच्छा॒न्त र॑स्य स्य॒न्तर् य॑च्छ । \newline
4. अ॒न्तर् य॑च्छ यच्छा॒न्त र॒न्तर् य॑च्छ मघवन् मघवन्. यच्छा॒न्त र॒न्तर् य॑च्छ मघवन्न् । \newline
5. य॒च्छ॒ म॒घ॒व॒न् म॒घ॒व॒न्॒. य॒च्छ॒ य॒च्छ॒ म॒घ॒व॒न् पा॒हि पा॒हि म॑घवन्. यच्छ यच्छ मघवन् पा॒हि । \newline
6. म॒घ॒व॒न् पा॒हि पा॒हि म॑घवन् मघवन् पा॒हि सोम॒(ग्म्॒) सोम॑म् पा॒हि म॑घवन् मघवन् पा॒हि सोम᳚म् । \newline
7. म॒घ॒व॒न्निति॑ मघ - व॒न्न् । \newline
8. पा॒हि सोम॒(ग्म्॒) सोम॑म् पा॒हि पा॒हि सोम॑ मुरु॒ ष्योरु॒ष्य सोम॑म् पा॒हि पा॒हि सोम॑ मुरु॒ष्य । \newline
9. सोम॑ मुरु॒ष्योरु॒ष्य सोम॒(ग्म्॒) सोम॑ मुरु॒ष्य रायो॒ राय॑ उरु॒ष्य सोम॒(ग्म्॒) सोम॑ मुरु॒ष्य रायः॑ । \newline
10. उ॒रु॒ष्य रायो॒ राय॑ उरु॒ष्योरु॒ष्य रायः॒ सꣳ सꣳ राय॑ उरु॒ष्योरु॒ष्य रायः॒ सम् । \newline
11. रायः॒ सꣳ सꣳ रायो॒ रायः॒ स मिष॒ इषः॒ सꣳ रायो॒ रायः॒ स मिषः॑ । \newline
12. स मिष॒ इषः॒ सꣳ स मिषो॑ यजस्व यज॒स्वे षः॒ सꣳ स मिषो॑ यजस्व । \newline
13. इषो॑ यजस्व यज॒स्वे ष॒ इषो॑ यजस्वा॒न्त र॒न्तर् य॑ज॒स्वे ष॒ इषो॑ यजस्वा॒न्तः । \newline
14. य॒ज॒स्वा॒न्त र॒न्तर् य॑जस्व यजस्वा॒न्त स्ते॑ ते॒ ऽन्तर् य॑जस्व यजस्वा॒न्त स्ते᳚ । \newline
15. अ॒न्त स्ते॑ ते॒ ऽन्त र॒न्त स्ते॑ दधामि दधामि ते॒ ऽन्त र॒न्त स्ते॑ दधामि । \newline
16. ते॒ द॒धा॒मि॒ द॒धा॒मि॒ ते॒ ते॒ द॒धा॒मि॒ द्यावा॑पृथि॒वी द्यावा॑पृथि॒वी द॑धामि ते ते दधामि॒ द्यावा॑पृथि॒वी । \newline
17. द॒धा॒मि॒ द्यावा॑पृथि॒वी द्यावा॑पृथि॒वी द॑धामि दधामि॒ द्यावा॑पृथि॒वी अ॒न्त र॒न्तर् द्यावा॑पृथि॒वी द॑धामि दधामि॒ द्यावा॑पृथि॒वी अ॒न्तः । \newline
18. द्यावा॑पृथि॒वी अ॒न्त र॒न्तर् द्यावा॑पृथि॒वी द्यावा॑पृथि॒वी अ॒न्त रु॒रू᳚(1॒)र्व॑न्तर् द्यावा॑पृथि॒वी द्यावा॑पृथि॒वी अ॒न्त रु॒रु । \newline
19. द्यावा॑पृथि॒वी इति॒ द्यावा᳚ - पृ॒थि॒वी । \newline
20. अ॒न्त रु॒रू᳚(1॒)र्व॑न्त र॒न्त रु॒र्व॑न्तरि॑क्ष म॒न्तरि॑क्ष मु॒र्व॑न्त र॒न्त रु॒र्व॑न्तरि॑क्षम् । \newline
21. उ॒र्व॑न्तरि॑क्ष म॒न्तरि॑क्ष मु॒रू᳚(1॒)र्व॑न्तरि॑क्षꣳ स॒जोषाः᳚ स॒जोषा॑ अ॒न्तरि॑क्ष मु॒रू᳚(1॒)र्व॑न्तरि॑क्षꣳ स॒जोषाः᳚ । \newline
22. अ॒न्तरि॑क्षꣳ स॒जोषाः᳚ स॒जोषा॑ अ॒न्तरि॑क्ष म॒न्तरि॑क्षꣳ स॒जोषा॑ दे॒वैर् दे॒वैः स॒जोषा॑ अ॒न्तरि॑क्ष म॒न्तरि॑क्षꣳ स॒जोषा॑ दे॒वैः । \newline
23. स॒जोषा॑ दे॒वैर् दे॒वैः स॒जोषाः᳚ स॒जोषा॑ दे॒वै रव॑रै॒ रव॑रैर् दे॒वैः स॒जोषाः᳚ स॒जोषा॑ दे॒वै रव॑रैः । \newline
24. स॒जोषा॒ इति॑ स - जोषाः᳚ । \newline
25. दे॒वै रव॑रै॒ रव॑रैर् दे॒वैर् दे॒वैरव॑रैः॒ परैः॒ परै॒ रव॑रैर् दे॒वैर् दे॒वै रव॑रैः॒ परैः᳚ । \newline
26. अव॑रैः॒ परैः॒ परै॒ रव॑रै॒ रव॑रैः॒ परै᳚श्च च॒ परै॒ रव॑रै॒ रव॑रैः॒ परै᳚श्च । \newline
27. परै᳚श्च च॒ परैः॒ परै᳚ श्चान्तर्या॒मे᳚ ऽन्तर्या॒मे च॒ परैः॒ परै᳚ श्चान्तर्या॒मे । \newline
28. चा॒न्त॒र्या॒मे᳚ ऽन्तर्या॒मे च॑ चान्तर्या॒मे म॑घवन् मघवन् नन्तर्या॒मे च॑ चान्तर्या॒मे म॑घवन्न् । \newline
29. अ॒न्त॒र्या॒मे म॑घवन् मघवन् नन्तर्या॒मे᳚ ऽन्तर्या॒मे म॑घवन् मादयस्व मादयस्व मघवन् नन्तर्या॒मे᳚ ऽन्तर्या॒मे म॑घवन् मादयस्व । \newline
30. अ॒न्त॒र्या॒म इत्य॑न्तः - या॒मे । \newline
31. म॒घ॒व॒न् मा॒द॒य॒स्व॒ मा॒द॒य॒स्व॒ म॒घ॒व॒न् म॒घ॒व॒न् मा॒द॒य॒स्व॒ स्वाङ्कृ॑तः॒ स्वाङ्कृ॑तो मादयस्व मघवन् मघवन् मादयस्व॒ स्वाङ्कृ॑तः । \newline
32. म॒घ॒व॒न्निति॑ मघ - व॒न्न् । \newline
33. मा॒द॒य॒स्व॒ स्वाङ्कृ॑तः॒ स्वाङ्कृ॑तो मादयस्व मादयस्व॒ स्वाङ्कृ॑तो ऽस्यसि॒ स्वाङ्कृ॑तो मादयस्व मादयस्व॒ स्वाङ्कृ॑तो ऽसि । \newline
34. स्वाङ्कृ॑तो ऽस्यसि॒ स्वाङ्कृ॑तः॒ स्वाङ्कृ॑तो ऽसि॒ मधु॑मती॒र् मधु॑मतीरसि॒ स्वाङ्कृ॑तः॒ स्वाङ्कृ॑तो ऽसि॒ मधु॑मतीः । \newline
35. अ॒सि॒ मधु॑मती॒र् मधु॑मती रस्यसि॒ मधु॑मतीर् नो नो॒ मधु॑मती रस्यसि॒ मधु॑मतीर् नः । \newline
36. मधु॑मतीर् नो नो॒ मधु॑मती॒र् मधु॑मतीर् न॒ इष॒ इषो॑ नो॒ मधु॑मती॒र् मधु॑मतीर् न॒ इषः॑ । \newline
37. मधु॑मती॒रिति॒ मधु॑ - म॒तीः॒ । \newline
38. न॒ इष॒ इषो॑ नो न॒ इष॑ स्कृधि कृ॒धीषो॑ नो न॒ इष॑ स्कृधि । \newline
39. इष॑ स्कृधि कृ॒धीष॒ इष॑ स्कृधि॒ विश्वे᳚भ्यो॒ विश्वे᳚भ्य स्कृ॒धीष॒ इष॑ स्कृधि॒ विश्वे᳚भ्यः । \newline
40. कृ॒धि॒ विश्वे᳚भ्यो॒ विश्वे᳚भ्य स्कृधि कृधि॒ विश्वे᳚भ्य स्त्वा त्वा॒ विश्वे᳚भ्य स्कृधि कृधि॒ विश्वे᳚भ्य स्त्वा । \newline
41. विश्वे᳚भ्य स्त्वा त्वा॒ विश्वे᳚भ्यो॒ विश्वे᳚भ्य स्त्वेन्द्रि॒येभ्य॑ इन्द्रि॒येभ्य॑ स्त्वा॒ विश्वे᳚भ्यो॒ विश्वे᳚भ्य स्त्वेन्द्रि॒येभ्यः॑ । \newline
42. त्वे॒न्द्रि॒येभ्य॑ इन्द्रि॒येभ्य॑ स्त्वा त्वेन्द्रि॒येभ्यो॑ दि॒व्येभ्यो॑ दि॒व्येभ्य॑ इन्द्रि॒येभ्य॑ स्त्वा त्वेन्द्रि॒येभ्यो॑ दि॒व्येभ्यः॑ । \newline
43. इ॒न्द्रि॒येभ्यो॑ दि॒व्येभ्यो॑ दि॒व्येभ्य॑ इन्द्रि॒येभ्य॑ इन्द्रि॒येभ्यो॑ दि॒व्येभ्यः॒ पार्थि॑वेभ्यः॒ पार्थि॑वेभ्यो दि॒व्येभ्य॑ इन्द्रि॒येभ्य॑ इन्द्रि॒येभ्यो॑ दि॒व्येभ्यः॒ पार्थि॑वेभ्यः । \newline
44. दि॒व्येभ्यः॒ पार्थि॑वेभ्यः॒ पार्थि॑वेभ्यो दि॒व्येभ्यो॑ दि॒व्येभ्यः॒ पार्थि॑वेभ्यो॒ मनो॒ मनः॒ पार्थि॑वेभ्यो दि॒व्येभ्यो॑ दि॒व्येभ्यः॒ पार्थि॑वेभ्यो॒ मनः॑ । \newline
45. पार्थि॑वेभ्यो॒ मनो॒ मनः॒ पार्थि॑वेभ्यः॒ पार्थि॑वेभ्यो॒ मन॑ स्त्वा त्वा॒ मनः॒ पार्थि॑वेभ्यः॒ पार्थि॑वेभ्यो॒ मन॑ स्त्वा । \newline
46. मन॑ स्त्वा त्वा॒ मनो॒ मन॑ स्त्वा ऽष्ट्वष्टु त्वा॒ मनो॒ मन॑ स्त्वा ऽष्टु । \newline
47. त्वा॒ ऽष्ट्व॒ष्टु॒ त्वा॒ त्वा॒ ऽष्टू॒रू᳚(1॒)र्व॑ष्टु त्वा त्वा ऽष्टू॑रु । \newline
48. अ॒ष्टू॒रू᳚(1॒)र्व॑ष्ट्वष्टू॒ र्व॑न्तरि॑क्ष म॒न्तरि॑क्ष मु॒र्व॑ष्ट्वष्टू॒ र्व॑न्तरि॑क्षम् । \newline
49. उ॒र्व॑न्तरि॑क्ष म॒न्तरि॑क्ष मु॒रू᳚(1॒)र्व॑न्तरि॑क्ष॒ मन्वन्व॒न्तरि॑क्ष मु॒रू᳚(1॒)र्व॑न्तरि॑क्ष॒ मनु॑ । \newline
50. अ॒न्तरि॑क्ष॒ मन्वन् व॒न्तरि॑क्ष म॒न्तरि॑क्ष॒ मन्वि॑ ही॒ह्यन्व॒न्तरि॑क्ष म॒न्तरि॑क्ष॒ मन्वि॑हि । \newline
51. अन्वि॑ ही॒ह्य न्वन्वि॑हि॒ स्वाहा॒ स्वाहे॒ह्य न्वन्वि॑हि॒ स्वाहा᳚ । \newline
52. इ॒हि॒ स्वाहा॒ स्वाहे॑हीहि॒ स्वाहा᳚ त्वा त्वा॒ स्वाहे॑हीहि॒ स्वाहा᳚ त्वा । \newline
53. स्वाहा᳚ त्वा त्वा॒ स्वाहा॒ स्वाहा᳚ त्वा सुभवः सुभव स्त्वा॒ स्वाहा॒ स्वाहा᳚ त्वा सुभवः । \newline
54. त्वा॒ सु॒भ॒वः॒ सु॒भ॒व॒ स्त्वा॒ त्वा॒ सु॒भ॒वः॒ सूर्या॑य॒ सूर्या॑य सुभव स्त्वा त्वा सुभवः॒ सूर्या॑य । \newline
55. सु॒भ॒वः॒ सूर्या॑य॒ सूर्या॑य सुभवः सुभवः॒ सूर्या॑य दे॒वेभ्यो॑ दे॒वेभ्यः॒ सूर्या॑य सुभवः सुभवः॒ सूर्या॑य दे॒वेभ्यः॑ । \newline
56. सु॒भ॒व॒ इति॑ सु - भ॒वः॒ । \newline
57. सूर्या॑य दे॒वेभ्यो॑ दे॒वेभ्यः॒ सूर्या॑य॒ सूर्या॑य दे॒वेभ्य॑ स्त्वा त्वा दे॒वेभ्यः॒ सूर्या॑य॒ सूर्या॑य दे॒वेभ्य॑ स्त्वा । \newline
58. दे॒वेभ्य॑ स्त्वा त्वा दे॒वेभ्यो॑ दे॒वेभ्य॑ स्त्वा मरीचि॒पेभ्यो॑ मरीचि॒पेभ्य॑ स्त्वा दे॒वेभ्यो॑ दे॒वेभ्य॑ स्त्वा मरीचि॒पेभ्यः॑ । \newline
59. त्वा॒ म॒री॒चि॒पेभ्यो॑ मरीचि॒पेभ्य॑ स्त्वा त्वा मरीचि॒पेभ्य॑ ए॒ष ए॒ष म॑रीचि॒पेभ्य॑ स्त्वा त्वा मरीचि॒पेभ्य॑ ए॒षः । \newline
60. म॒री॒चि॒पेभ्य॑ ए॒ष ए॒ष म॑रीचि॒पेभ्यो॑ मरीचि॒पेभ्य॑ ए॒ष ते॑ त ए॒ष म॑रीचि॒पेभ्यो॑ मरीचि॒पेभ्य॑ ए॒ष ते᳚ । \newline
61. म॒री॒चि॒पेभ्य॒ इति॑ मरीचि - पेभ्यः॑ । \newline
62. ए॒ष ते॑ त ए॒ष ए॒ष ते॒ योनि॒र् योनि॑ स्त ए॒ष ए॒ष ते॒ योनिः॑ । \newline
63. ते॒ योनि॒र् योनि॑ स्ते ते॒ योनि॑ रपा॒ना या॑पा॒नाय॒ योनि॑ स्ते ते॒ योनि॑ रपा॒नाय॑ । \newline
64. योनि॑ रपा॒ना या॑पा॒नाय॒ योनि॒र् योनि॑ रपा॒नाय॑ त्वा त्वा ऽपा॒नाय॒ योनि॒र् योनि॑ रपा॒नाय॑ त्वा । \newline
65. अ॒पा॒नाय॑ त्वा त्वा ऽपा॒ना या॑पा॒नाय॑ त्वा । \newline
66. अ॒पा॒नायेत्य॑प - अ॒नाय॑ । \newline
67. त्वेति॑ त्वा । \newline
\pagebreak
\markright{ TS 1.4.4.1  \hfill https://www.vedavms.in \hfill}
\addcontentsline{toc}{section}{ TS 1.4.4.1 }
\section*{ TS 1.4.4.1 }

\textbf{TS 1.4.4.1 } \newline
\textbf{Samhita Paata} \newline

आ वा॑यो भूष शुचिपा॒ उप॑ नः स॒हस्रं॑ ते नि॒युतो॑ विश्ववार । उपो॑ ते॒ अन्धो॒ मद्य॑मयामि॒ यस्य॑ देव दधि॒षे पू᳚र्व॒पेयं᳚ ॥ उ॒प॒या॒मगृ॑हीतोऽसि वा॒यवे॒ त्वेन्द्र॑वायू इ॒मे सु॒ताः । उप॒ प्रयो॑भि॒रा ग॑त॒मिन्द॑वो वामु॒शन्ति॒ हि ॥ उ॒प॒या॒मगृ॑हीतो-ऽसीन्द्रवा॒युभ्यां᳚ त्वै॒ष ते॒ योनिः॑ स॒जोषा᳚भ्यां त्वा ॥ \newline

\textbf{Pada Paata} \newline

एति॑ । वा॒यो॒ इति॑ । भू॒ष॒ । शु॒चि॒पा॒ इति॑ शुचि - पाः॒ । उपेति॑ । नः॒ । स॒हस्र᳚म् । ते॒ । नि॒युत॒ इति॑ नि - युतः॑ । वि॒श्व॒वा॒रेति॑ विश्व - वा॒र॒ ॥ उपो॒ इति॑ । ते॒ । अन्धः॑ । मद्य᳚म् । अ॒या॒मि॒ । यस्य॑ । दे॒व॒ । द॒धि॒षे । पू॒र्व॒पेय॒मिति॑ पूर्व - पेय᳚म् ॥ उ॒प॒या॒मगृ॑हीत॒ इत्यु॑पया॒म - गृ॒ही॒तः॒ । अ॒सि॒ । वा॒यवे᳚ । त्वा॒ । इन्द्र॑वायू॒ इतीन्द्र॑ - वा॒यू॒ । इ॒मे । सु॒ताः ॥ उपेति॑ । प्रयो॑भि॒रिति॒ प्रयः॑ - भिः॒ । एति॑ । ग॒त॒म् । इन्द॑वः । वा॒म् । उ॒शन्ति॑ । हि ॥ उ॒प॒या॒मगृ॑हीत॒ इत्यु॑पया॒म - गृ॒ही॒तः॒ । अ॒सि॒ । इ॒न्द्र॒वा॒युभ्या॒मिती᳚न्द्रवा॒यु - भ्या॒म् । त्वा॒ । ए॒षः । ते॒ । योनिः॑ । स॒जोषा᳚भ्या॒मिति॑ स - जोषा᳚भ्याम् । त्वा॒ ॥  \newline


\textbf{Krama Paata} \newline

आ वा॑यो । वा॒यो॒ भू॒ष॒ । वा॒यो॒ इति॑ वायो । भू॒ष॒ शु॒चि॒पाः॒ । शु॒चि॒पा॒ उप॑ । शु॒चि॒पा॒ इति॑ शुचि - पाः॒ । उप॑ नः । नः॒ स॒हस्र᳚म् । स॒हस्र॑म् ते । ते॒ नि॒युतः॑ । नि॒युतो॑ विश्ववार । नि॒युत॒ इति॑ नि - युतः॑ । वि॒श्व॒वा॒रेति॑ विश्व - वा॒र॒ ॥ उपो॑ ते । उपो॒ इत्युपो᳚ । ते॒ अन्धः॑ । अन्धो॒ मद्य᳚म् । मद्य॑मयामि । अ॒या॒मि॒ यस्य॑ । यस्य॑ देव । दे॒व॒ द॒धि॒षे । द॒धि॒षे पू᳚र्व॒पेय᳚म् । पू॒र्व॒पेय॒मिति॑ पूर्व - पेय᳚म् ॥ उ॒प॒या॒मगृ॑हीतोऽसि । उ॒प॒या॒मगृ॑हीत॒ इत्यु॑पया॒म - गृ॒ही॒तः॒ । अ॒सि॒ वा॒यवे᳚ । वा॒यवे᳚ त्वा । त्वेन्द्र॑वायू । इन्द्र॑वायू इ॒मे । इन्द्र॑वायू॒ इतीन्द्र॑ - वा॒यू॒ । इ॒मे सु॒ताः । सु॒ता इति॑ सु॒ताः ॥ उप॒ प्रयो॑भिः । प्रयो॑भि॒रा । प्रयो॑भि॒रिति॒ प्रयः॑ - भिः॒ । आ ग॑तम् । ग॒त॒मिन्द॑वः । इन्द॑वो वाम् । वा॒मु॒शन्ति॑ । उ॒शन्ति॒ हि । हीति॒ हि ॥ उ॒प॒या॒मगृ॑हीतोऽसि । उ॒प॒या॒मगृ॑हीत॒ इत्यु॑पया॒म - गृ॒ही॒तः॒ । अ॒सी॒न्द्र॒वा॒युभ्या᳚म् । इ॒न्द्र॒वा॒युभ्या᳚म् त्वा । इ॒न्द्र॒वा॒युभ्या॒मिती᳚न्द्रवा॒यु - भ्या॒म् । त्वै॒षः । ए॒ष ते᳚ । ते॒ योनिः॑ । योनिः॑ स॒जोषा᳚भ्याम् । स॒जोषा᳚भ्याम् त्वा । स॒जोषा᳚भ्या॒मिति॑ स - जोषा᳚भ्याम् । 
त्वेति॑ त्वा । \newline

\textbf{Jatai Paata} \newline

1. आ वा॑यो वायो॒ आ वा॑यो । \newline
2. वा॒यो॒ भू॒ष॒ भू॒ष॒ वा॒यो॒ वा॒यो॒ भू॒ष॒ । \newline
3. वा॒यो॒ इति॑ वायो । \newline
4. भू॒ष॒ शु॒चि॒पाः॒ शु॒चि॒पा॒ भू॒ष॒ भू॒ष॒ शु॒चि॒पाः॒ । \newline
5. शु॒चि॒पा॒ उपोप॑ शुचिपाः शुचिपा॒ उप॑ । \newline
6. शु॒चि॒पा॒ इति॑ शुचि - पाः॒ । \newline
7. उप॑ नो न॒ उपोप॑ नः । \newline
8. नः॒ स॒हस्र(ग्म्॑) स॒हस्र॑न्नो नः स॒हस्र᳚म् । \newline
9. स॒हस्र॑म् ते ते स॒हस्र(ग्म्॑) स॒हस्र॑म् ते । \newline
10. ते॒ नि॒युतो॑ नि॒युत॑ स्ते ते नि॒युतः॑ । \newline
11. नि॒युतो॑ विश्ववार विश्ववार नि॒युतो॑ नि॒युतो॑ विश्ववार । \newline
12. नि॒युत॒ इति॑ नि - युतः॑ । \newline
13. वि॒श्व॒वा॒रेति॑ विश्व - वा॒र॒ । \newline
14. उपो॑ ते त॒ उपो॒ उपो॑ ते । \newline
15. उपो॒ इत्युपो᳚ । \newline
16. ते॒ अन्धो ऽन्ध॑ स्ते ते॒ अन्धः॑ । \newline
17. अन्धो॒ मद्य॒म् मद्य॒ मन्धो ऽन्धो॒ मद्य᳚म् । \newline
18. मद्य॑ मयाम्ययामि॒ मद्य॒म् मद्य॑ मयामि । \newline
19. अ॒या॒मि॒ यस्य॒ यस्या॑या म्ययामि॒ यस्य॑ । \newline
20. यस्य॑ देव देव॒ यस्य॒ यस्य॑ देव । \newline
21. दे॒व॒ द॒धि॒षे द॑धि॒षे दे॑व देव दधि॒षे । \newline
22. द॒धि॒षे पू᳚र्व॒पेय॑म् पूर्व॒पेय॑म् दधि॒षे द॑धि॒षे पू᳚र्व॒पेय᳚म् । \newline
23. पू॒र्व॒पेय॒मिति॑ पूर्व - पेय᳚म् । \newline
24. उ॒प॒या॒मगृ॑हीतो ऽस्यस्युपया॒मगृ॑हीत उपया॒मगृ॑हीतो ऽसि । \newline
25. उ॒प॒या॒मगृ॑हीत॒ इत्यु॑पया॒म - गृ॒ही॒तः॒ । \newline
26. अ॒सि॒ वा॒यवे॑ वा॒यवे᳚ ऽस्यसि वा॒यवे᳚ । \newline
27. वा॒यवे᳚ त्वा त्वा वा॒यवे॑ वा॒यवे᳚ त्वा । \newline
28. त्वेन्द्र॑वायू॒ इन्द्र॑वायू त्वा॒ त्वेन्द्र॑वायू । \newline
29. इन्द्र॑वायू इ॒म इ॒म इन्द्र॑वायू॒ इन्द्र॑वायू इ॒मे । \newline
30. इन्द्र॑वायू॒ इतीन्द्र॑ - वा॒यू॒ । \newline
31. इ॒मे सु॒ताः सु॒ता इ॒म इ॒मे सु॒ताः । \newline
32. सु॒ता इति॑ सु॒ताः । \newline
33. उप॒ प्रयो॑भिः॒ प्रयो॑भि॒ रुपोप॒ प्रयो॑भिः । \newline
34. प्रयो॑भि॒रा प्रयो॑भिः॒ प्रयो॑भि॒रा । \newline
35. प्रयो॑भि॒रिति॒ प्रयः॑ - भिः॒ । \newline
36. आ ग॑तम् गत॒ मा ग॑तम् । \newline
37. ग॒त॒ मिन्द॑व॒ इन्द॑वो गतम् गत॒ मिन्द॑वः । \newline
38. इन्द॑वो वां ॅवा॒ मिन्द॑व॒ इन्द॑वो वाम् । \newline
39. वा॒ मु॒शन्त्यु॒शन्ति॑ वां ॅवा मु॒शन्ति॑ । \newline
40. उ॒शन्ति॒ हि ह्यु॑शन्त्यु॒शन्ति॒ हि । \newline
41. हीति॒ हि । \newline
42. उ॒प॒या॒मगृ॑हीतो ऽस्यस्युपया॒मगृ॑हीत उपया॒मगृ॑हीतो ऽसि । \newline
43. उ॒प॒या॒मगृ॑हीत॒ इत्यु॑पया॒म - गृ॒ही॒तः॒ । \newline
44. अ॒सी॒न्द्र॒वा॒युभ्या॑ मिन्द्रवा॒युभ्या॑ मस्यसीन्द्रवा॒युभ्या᳚म् । \newline
45. इ॒न्द्र॒वा॒युभ्या᳚म् त्वा त्वेन्द्रवा॒युभ्या॑ मिन्द्रवा॒युभ्या᳚म् त्वा । \newline
46. इ॒न्द्र॒वा॒युभ्या॒मिती᳚न्द्रवा॒यु - भ्या॒म् । \newline
47. त्वै॒ष ए॒ष त्वा᳚ त्वै॒षः । \newline
48. ए॒ष ते॑ त ए॒ष ए॒ष ते᳚ । \newline
49. ते॒ योनि॒र् योनि॑ स्ते ते॒ योनिः॑ । \newline
50. योनिः॑ स॒जोषा᳚भ्याꣳ स॒जोषा᳚भ्यां॒ ॅयोनि॒र् योनिः॑ स॒जोषा᳚भ्याम् । \newline
51. स॒जोषा᳚भ्याम् त्वा त्वा स॒जोषा᳚भ्याꣳ स॒जोषा᳚भ्याम् त्वा । \newline
52. स॒जोषा᳚भ्या॒मिति॑ स - जोषा᳚भ्याम् । \newline
53. त्वेति॑ त्वा । \newline

\textbf{Ghana Paata } \newline

1. आ वा॑यो वायो॒ आ वा॑यो भूष भूष वायो॒ आ वा॑यो भूष । \newline
2. वा॒यो॒ भू॒ष॒ भू॒ष॒ वा॒यो॒ वा॒यो॒ भू॒ष॒ शु॒चि॒पाः॒ शु॒चि॒पा॒ भू॒ष॒ वा॒यो॒ वा॒यो॒ भू॒ष॒ शु॒चि॒पाः॒ । \newline
3. वा॒यो॒ इति॑ वायो । \newline
4. भू॒ष॒ शु॒चि॒पाः॒ शु॒चि॒पा॒ भू॒ष॒ भू॒ष॒ शु॒चि॒पा॒ उपोप॑ शुचिपा भूष भूष शुचिपा॒ उप॑ । \newline
5. शु॒चि॒पा॒ उपोप॑ शुचिपाः शुचिपा॒ उप॑ नो न॒ उप॑ शुचिपाः शुचिपा॒ उप॑ नः । \newline
6. शु॒चि॒पा॒ इति॑ शुचि - पाः॒ । \newline
7. उप॑ नो न॒ उपोप॑ नः स॒हस्र(ग्म्॑) स॒हस्र॑म् न॒ उपोप॑ नः स॒हस्र᳚म् । \newline
8. नः॒ स॒हस्र(ग्म्॑) स॒हस्र॑म् नो नः स॒हस्र॑म् ते ते स॒हस्र॑म् नो नः स॒हस्र॑म् ते । \newline
9. स॒हस्र॑म् ते ते स॒हस्र(ग्म्॑) स॒हस्र॑म् ते नि॒युतो॑ नि॒युत॑ स्ते स॒हस्र(ग्म्॑) स॒हस्र॑म् ते नि॒युतः॑ । \newline
10. ते॒ नि॒युतो॑ नि॒युत॑ स्ते ते नि॒युतो॑ विश्ववार विश्ववार नि॒युत॑ स्ते ते नि॒युतो॑ विश्ववार । \newline
11. नि॒युतो॑ विश्ववार विश्ववार नि॒युतो॑ नि॒युतो॑ विश्ववार । \newline
12. नि॒युत॒ इति॑ नि - युतः॑ । \newline
13. वि॒श्व॒वा॒रेति॑ विश्व - वा॒र॒ । \newline
14. उपो॑ ते त॒ उपो॒ उपो॑ ते॒ अन्धो ऽन्ध॑ स्त॒ उपो॒ उपो॑ ते॒ अन्धः॑ । \newline
15. उपो॒ इत्युपो᳚ । \newline
16. ते॒ अन्धो ऽन्ध॑ स्ते ते॒ अन्धो॒ मद्य॒म् मद्य॒ मन्ध॑ स्ते ते॒ अन्धो॒ मद्य᳚म् । \newline
17. अन्धो॒ मद्य॒म् मद्य॒ मन्धो ऽन्धो॒ मद्य॑ मया म्ययामि॒ मद्य॒ मन्धो ऽन्धो॒ मद्य॑ मयामि । \newline
18. मद्य॑ मया म्ययामि॒ मद्य॒म् मद्य॑ मयामि॒ यस्य॒ यस्या॑यामि॒ मद्य॒म् मद्य॑ मयामि॒ यस्य॑ । \newline
19. अ॒या॒मि॒ यस्य॒ यस्या॑ याम्ययामि॒ यस्य॑ देव देव॒ यस्या॑ याम्ययामि॒ यस्य॑ देव । \newline
20. यस्य॑ देव देव॒ यस्य॒ यस्य॑ देव दधि॒षे द॑धि॒षे दे॑व॒ यस्य॒ यस्य॑ देव दधि॒षे । \newline
21. दे॒व॒ द॒धि॒षे द॑धि॒षे दे॑व देव दधि॒षे पू᳚र्व॒पेय॑म् पूर्व॒पेय॑म् दधि॒षे दे॑व देव दधि॒षे पू᳚र्व॒पेय᳚म् । \newline
22. द॒धि॒षे पू᳚र्व॒पेय॑म् पूर्व॒पेय॑म् दधि॒षे द॑धि॒षे पू᳚र्व॒पेय᳚म् । \newline
23. पू॒र्व॒पेय॒मिति॑ पूर्व - पेय᳚म् । \newline
24. उ॒प॒या॒मगृ॑हीतो ऽस्यस्युपया॒मगृ॑हीत उपया॒मगृ॑हीतो ऽसि वा॒यवे॑ वा॒यवे᳚ ऽस्युपया॒मगृ॑हीत उपया॒मगृ॑हीतो ऽसि वा॒यवे᳚ । \newline
25. उ॒प॒या॒मगृ॑हीत॒ इत्यु॑पया॒म - गृ॒ही॒तः॒ । \newline
26. अ॒सि॒ वा॒यवे॑ वा॒यवे᳚ ऽस्यसि वा॒यवे᳚ त्वा त्वा वा॒यवे᳚ ऽस्यसि वा॒यवे᳚ त्वा । \newline
27. वा॒यवे᳚ त्वा त्वा वा॒यवे॑ वा॒यवे॒ त्वेन्द्र॑वायू॒ इन्द्र॑वायू त्वा वा॒यवे॑ वा॒यवे॒ त्वेन्द्र॑वायू । \newline
28. त्वेन्द्र॑वायू॒ इन्द्र॑वायू त्वा॒ त्वेन्द्र॑वायू इ॒म इ॒म इन्द्र॑वायू त्वा॒ त्वेन्द्र॑वायू इ॒मे । \newline
29. इन्द्र॑वायू इ॒म इ॒म इन्द्र॑वायू॒ इन्द्र॑वायू इ॒मे सु॒ताः सु॒ता इ॒म इन्द्र॑वायू॒ इन्द्र॑वायू इ॒मे सु॒ताः । \newline
30. इन्द्र॑वायू॒ इतीन्द्र॑ - वा॒यू॒ । \newline
31. इ॒मे सु॒ताः सु॒ता इ॒म इ॒मे सु॒ताः । \newline
32. सु॒ता इति॑ सु॒ताः । \newline
33. उप॒ प्रयो॑भिः॒ प्रयो॑भि॒ रुपोप॒ प्रयो॑भि॒रा प्रयो॑भि॒ रुपोप॒ प्रयो॑भि॒रा । \newline
34. प्रयो॑भि॒रा प्रयो॑भिः॒ प्रयो॑भि॒रा ग॑तम् गत॒ मा प्रयो॑भिः॒ प्रयो॑भि॒रा ग॑तम् । \newline
35. प्रयो॑भि॒रिति॒ प्रयः॑ - भिः॒ । \newline
36. आ ग॑तम् गत॒ मा ग॑त॒ मिन्द॑व॒ इन्द॑वो गत॒ मा ग॑त॒ मिन्द॑वः । \newline
37. ग॒त॒ मिन्द॑व॒ इन्द॑वो गतम् गत॒ मिन्द॑वो वां ॅवा॒ मिन्द॑वो गतम् गत॒ मिन्द॑वो वाम् । \newline
38. इन्द॑वो वां ॅवा॒ मिन्द॑व॒ इन्द॑वो वा मु॒श न्त्यु॒शन्ति॑ वा॒ मिन्द॑व॒ इन्द॑वो वा मु॒शन्ति॑ । \newline
39. वा॒ मु॒श न्त्यु॒शन्ति॑ वां ॅवा मु॒शन्ति॒ हि ह्यु॑शन्ति॑ वां ॅवा मु॒शन्ति॒ हि । \newline
40. उ॒शन्ति॒ हि ह्यु॑श न्त्यु॒शन्ति॒ हि । \newline
41. हीति॒ हि । \newline
42. उ॒प॒या॒मगृ॑हीतो ऽस्यस्युपया॒मगृ॑हीत उपया॒मगृ॑हीतो ऽसीन्द्रवा॒युभ्या॑ मिन्द्रवा॒युभ्या॑ मस्युपया॒मगृ॑हीत उपया॒मगृ॑हीतो ऽसीन्द्रवा॒युभ्या᳚म् । \newline
43. उ॒प॒या॒मगृ॑हीत॒ इत्यु॑पया॒म - गृ॒ही॒तः॒ । \newline
44. अ॒सी॒न्द्र॒वा॒युभ्या॑ मिन्द्रवा॒युभ्या॑ मस्यसीन्द्रवा॒युभ्या᳚म् त्वा त्वेन्द्रवा॒युभ्या॑ मस्यसीन्द्रवा॒युभ्या᳚म् त्वा । \newline
45. इ॒न्द्र॒वा॒युभ्या᳚म् त्वा त्वेन्द्रवा॒युभ्या॑ मिन्द्रवा॒युभ्या᳚म् त्वै॒ष ए॒ष त्वे᳚न्द्रवा॒युभ्या॑ मिन्द्रवा॒युभ्या᳚म् त्वै॒षः । \newline
46. इ॒न्द्र॒वा॒युभ्या॒मिती᳚न्द्रवा॒यु - भ्या॒म् । \newline
47. त्वै॒ष ए॒ष त्वा᳚ त्वै॒ष ते॑ त ए॒ष त्वा᳚ त्वै॒ष ते᳚ । \newline
48. ए॒ष ते॑ त ए॒ष ए॒ष ते॒ योनि॒र् योनि॑ स्त ए॒ष ए॒ष ते॒ योनिः॑ । \newline
49. ते॒ योनि॒र् योनि॑ स्ते ते॒ योनिः॑ स॒जोषा᳚भ्याꣳ स॒जोषा᳚भ्यां॒ ॅयोनि॑ स्ते ते॒ योनिः॑ स॒जोषा᳚भ्याम् । \newline
50. योनिः॑ स॒जोषा᳚भ्याꣳ स॒जोषा᳚भ्यां॒ ॅयोनि॒र् योनिः॑ स॒जोषा᳚भ्याम् त्वा त्वा स॒जोषा᳚भ्यां॒ ॅयोनि॒र् योनिः॑ 
स॒जोषा᳚भ्याम् त्वा । \newline
51. स॒जोषा᳚भ्याम् त्वा त्वा स॒जोषा᳚भ्याꣳ स॒जोषा᳚भ्याम् त्वा । \newline
52. स॒जोषा᳚भ्या॒मिति॑ स - जोषा᳚भ्याम् । \newline
53. त्वेति॑ त्वा । \newline
\pagebreak
\markright{ TS 1.4.5.1  \hfill https://www.vedavms.in \hfill}
\addcontentsline{toc}{section}{ TS 1.4.5.1 }
\section*{ TS 1.4.5.1 }

\textbf{TS 1.4.5.1 } \newline
\textbf{Samhita Paata} \newline

अ॒यं ॅवां᳚ मित्रावरुणा सु॒तः सोम॑ ऋतावृधा । ममेदि॒ह श्रु॑तꣳ॒॒ हवं᳚ । उ॒प॒या॒मगृ॑हीतोऽसि मि॒त्रावरु॑णाभ्यां त्वै॒ष ते॒ योनि॑र्. ऋता॒युभ्यां᳚ त्वा ॥ \newline

\textbf{Pada Paata} \newline

अ॒यम् । वा॒म् । मि॒त्रा॒व॒रु॒णेति॑ मित्रा - व॒रु॒णा॒ । सु॒तः । सोमः॑ । ऋ॒ता॒वृ॒धेत्यृ॑त - वृ॒धा॒ ॥ मम॑ । इत् । इ॒ह । श्रु॒त॒म् । हव᳚म् ॥ उ॒प॒या॒मगृ॑हीत॒ इत्यु॑पया॒म - गृ॒हीतः॒ । अ॒सि॒ । मि॒त्रावरु॑णाभ्या॒मिति॑ मि॒त्रा - वरु॑णाभ्याम् । त्वा॒ । ए॒षः । ते॒ । योनिः॑ । ऋ॒ता॒युभ्या॒मित्यृ॑ता॒यु - भ्या॒म् । त्वा॒ ॥  \newline


\textbf{Krama Paata} \newline

अ॒यं ॅवा᳚म् । वा॒म् मि॒त्रा॒व॒रु॒णा॒ । मि॒त्रा॒व॒रु॒णा॒ सु॒तः । मि॒त्रा॒व॒रु॒णेति॑ मित्रा - व॒रु॒णा॒ । सु॒तः सोमः॑ । सोम॑ ऋतावृधा । ऋ॒ता॒वृ॒धेत्यृ॑त - वृ॒धा॒ ॥ ममेत् । इदि॒ह । इ॒ह श्रु॑तम् । श्रु॒तꣳ॒॒ हव᳚म् । हव॒मिति॒ हव᳚म् ॥ उ॒प॒या॒मगृ॑हीतोऽसि । उ॒प॒या॒मगृ॑हीत॒ इत्यु॑पया॒म - गृ॒ही॒तः॒ । अ॒सि॒ मि॒त्रावरु॑णाभ्याम् । मि॒त्रावरु॑णाभ्याम् त्वा । मि॒त्रावरु॑णाभ्या॒मिति॑ मि॒त्रा - वरु॑णाभ्याम् । त्वै॒षः । ए॒ष ते᳚ । ते॒ योनिः॑ । योनि॑र्. ऋता॒युभ्या᳚म् । ऋ॒ता॒युभ्या᳚म् त्वा । ऋ॒ता॒युभ्या॒मित्यृ॑ता॒यु - भ्या॒म् । त्वेति॑ त्वा । \newline

\textbf{Jatai Paata} \newline

1. अ॒यं ॅवां᳚ ॅवा म॒य म॒यं ॅवा᳚म् । \newline
2. वा॒म् मि॒त्रा॒व॒रु॒णा॒ मि॒त्रा॒व॒रु॒णा॒ वां॒ ॅवा॒म् मि॒त्रा॒व॒रु॒णा॒ । \newline
3. मि॒त्रा॒व॒रु॒णा॒ सु॒तः सु॒तो मि॑त्रावरुणा मित्रावरुणा सु॒तः । \newline
4. मि॒त्रा॒व॒रु॒णेति॑ मित्रा - व॒रु॒णा॒ । \newline
5. सु॒तः सोमः॒ सोमः॑ सु॒तः सु॒तः सोमः॑ । \newline
6. सोम॑ ऋतावृधर्तावृधा॒ सोमः॒ सोम॑ ऋतावृधा । \newline
7. ऋ॒ता॒वृ॒धेत्यृ॑त - वृ॒धा॒ । \newline
8. ममे दिन् मम॒ ममे त् । \newline
9. इदि॒हे हे दिदि॒ह । \newline
10. इ॒ह श्रु॑तꣳ श्रुत मि॒हेह श्रु॑तम् । \newline
11. श्रु॒त॒(ग्म्॒) हव॒(ग्म्॒) हव(ग्ग्॑) श्रुतꣳ श्रुत॒(ग्म्॒) हव᳚म् । \newline
12. हव॒मिति॒ हव᳚म् । \newline
13. उ॒प॒या॒मगृ॑हीतो ऽस्यस्युपया॒मगृ॑हीत उपया॒मगृ॑हीतो ऽसि । \newline
14. उ॒प॒या॒मगृ॑हीत॒ इत्यु॑पया॒म - गृ॒ही॒तः॒ । \newline
15. अ॒सि॒ मि॒त्रावरु॑णाभ्याम् मि॒त्रावरु॑णाभ्या मस्यसि मि॒त्रावरु॑णाभ्याम् । \newline
16. मि॒त्रावरु॑णाभ्याम् त्वा त्वा मि॒त्रावरु॑णाभ्याम् मि॒त्रावरु॑णाभ्याम् त्वा । \newline
17. मि॒त्रावरु॑णाभ्या॒मिति॑ मि॒त्रा - वरु॑णाभ्याम् । \newline
18. त्वै॒ष ए॒ष त्वा᳚ त्वै॒षः । \newline
19. ए॒ष ते॑ त ए॒ष ए॒ष ते᳚ । \newline
20. ते॒ योनि॒र् योनि॑ स्ते ते॒ योनिः॑ । \newline
21. योनि॑र्. ऋता॒युभ्या॑ मृता॒युभ्यां॒ ॅयोनि॒र् योनि॑र्. ऋता॒युभ्या᳚म् । \newline
22. ऋ॒ता॒युभ्या᳚म् त्वा त्वर्ता॒युभ्या॑ मृता॒युभ्या᳚म् त्वा । \newline
23. ऋ॒ता॒युभ्या॒मित्यृ॑ता॒यु - भ्या॒म् । \newline
24. त्वेति॑ त्वा । \newline

\textbf{Ghana Paata } \newline

1. अ॒यं ॅवां᳚ ॅवा म॒य म॒यं ॅवा᳚म् मित्रावरुणा मित्रावरुणा वा म॒य म॒यं ॅवा᳚म् मित्रावरुणा । \newline
2. वा॒म् मि॒त्रा॒व॒रु॒णा॒ मि॒त्रा॒व॒रु॒णा॒ वां॒ ॅवा॒म् मि॒त्रा॒व॒रु॒णा॒ सु॒तः सु॒तो मि॑त्रावरुणा वां ॅवाम् मित्रावरुणा सु॒तः । \newline
3. मि॒त्रा॒व॒रु॒णा॒ सु॒तः सु॒तो मि॑त्रावरुणा मित्रावरुणा सु॒तः सोमः॒ सोमः॑ सु॒तो मि॑त्रावरुणा मित्रावरुणा सु॒तः सोमः॑ । \newline
4. मि॒त्रा॒व॒रु॒णेति॑ मित्रा - व॒रु॒णा॒ । \newline
5. सु॒तः सोमः॒ सोमः॑ सु॒तः सु॒तः सोम॑ ऋतावृधर्तावृधा॒ सोमः॑ सु॒तः सु॒तः सोम॑ ऋतावृधा । \newline
6. सोम॑ ऋतावृधर्तावृधा॒ सोमः॒ सोम॑ ऋतावृधा । \newline
7. ऋ॒ता॒वृ॒धेत्यृ॑त - वृ॒धा॒ । \newline
8. ममे दिन् मम॒ ममे दि॒हे हेन् मम॒ ममे दि॒ह । \newline
9. इदि॒हे हे दिदि॒ह श्रु॑तꣳ श्रुत मि॒हे दिदि॒ह श्रु॑तम् । \newline
10. इ॒ह श्रु॑तꣳ श्रुत मि॒हे ह श्रु॑त॒(ग्म्॒) हव॒(ग्म्॒) हव(ग्ग्॑) श्रुत मि॒हे ह श्रु॑त॒(ग्म्॒) हव᳚म् । \newline
11. श्रु॒त॒(ग्म्॒) हव॒(ग्म्॒) हव(ग्ग्॑) श्रुतꣳ श्रुत॒(ग्म्॒) हव᳚म् । \newline
12. हव॒मिति॒ हव᳚म् । \newline
13. उ॒प॒या॒मगृ॑हीतो ऽस्यस्युपया॒मगृ॑हीत उपया॒मगृ॑हीतो ऽसि मि॒त्रावरु॑णाभ्याम् मि॒त्रावरु॑णाभ्या मस्युपया॒मगृ॑हीत उपया॒मगृ॑हीतो ऽसि मि॒त्रावरु॑णाभ्याम् । \newline
14. उ॒प॒या॒मगृ॑हीत॒ इत्यु॑पया॒म - गृ॒ही॒तः॒ । \newline
15. अ॒सि॒ मि॒त्रावरु॑णाभ्याम् मि॒त्रावरु॑णाभ्या मस्यसि मि॒त्रावरु॑णाभ्याम् त्वा त्वा मि॒त्रावरु॑णाभ्या मस्यसि मि॒त्रावरु॑णाभ्याम् त्वा । \newline
16. मि॒त्रावरु॑णाभ्याम् त्वा त्वा मि॒त्रावरु॑णाभ्याम् मि॒त्रावरु॑णाभ्याम् त्वै॒ष ए॒ष त्वा॑ मि॒त्रावरु॑णाभ्याम् मि॒त्रावरु॑णाभ्याम् त्वै॒षः । \newline
17. मि॒त्रावरु॑णाभ्या॒मिति॑ मि॒त्रा - वरु॑णाभ्याम् । \newline
18. त्वै॒ष ए॒ष त्वा᳚ त्वै॒ष ते॑ त ए॒ष त्वा᳚ त्वै॒ष ते᳚ । \newline
19. ए॒ष ते॑ त ए॒ष ए॒ष ते॒ योनि॒र् योनि॑ स्त ए॒ष ए॒ष ते॒ योनिः॑ । \newline
20. ते॒ योनि॒र् योनि॑ स्ते ते॒ योनि॑र्. ऋता॒युभ्या॑ मृता॒युभ्यां॒ ॅयोनि॑ स्ते ते॒ योनि॑र्. ऋता॒युभ्या᳚म् । \newline
21. योनि॑र्. ऋता॒युभ्या॑ मृता॒युभ्यां॒ ॅयोनि॒र् योनि॑र्. ऋता॒युभ्या᳚म् त्वा त्वर्ता॒युभ्यां॒ ॅयोनि॒र् योनि॑र्. ऋता॒युभ्या᳚म् त्वा । \newline
22. ऋ॒ता॒युभ्या᳚म् त्वा त्वर्ता॒युभ्या॑ मृता॒युभ्या᳚म् त्वा । \newline
23. ऋ॒ता॒युभ्या॒मित्यृ॑ता॒यु - भ्या॒म् । \newline
24. त्वेति॑ त्वा । \newline
\pagebreak
\markright{ TS 1.4.6.1  \hfill https://www.vedavms.in \hfill}
\addcontentsline{toc}{section}{ TS 1.4.6.1 }
\section*{ TS 1.4.6.1 }

\textbf{TS 1.4.6.1 } \newline
\textbf{Samhita Paata} \newline

या वां॒ कशा॒ मधु॑म॒त्यश्वि॑ना सू॒नृता॑वती । तया॑ य॒ज्ञ्ं मि॑मिक्षतं । उ॒प॒या॒मगृ॑हीतो-ऽस्य॒श्विभ्यां᳚ त्वै॒ष ते॒ योनि॒र्माद्ध्वी᳚भ्यां त्वा ॥ \newline

\textbf{Pada Paata} \newline

या । वा॒म् । कशा᳚ । मधु॑म॒तीति॒ मधु॑ - म॒ती॒ । अश्वि॑ना । सू॒नृता॑व॒तीति॑ सू॒नृता᳚ - व॒ती॒ ॥ तया᳚ । य॒ज्ञ्म् । मि॒मि॒क्ष॒त॒म् ॥ उ॒प॒या॒मगृ॑हीत॒ इत्यु॑पया॒म - गृ॒ही॒तः॒ । अ॒सि॒ । अ॒श्विभ्या॒मित्य॒श्वि - भ्या॒म् । त्वा॒ । ए॒षः । ते॒ । योनिः॑ । माद्ध्वी᳚भ्याम् । त्वा॒ ॥  \newline


\textbf{Krama Paata} \newline

या वा᳚म् । वा॒म् कशा᳚ । कशा॒ मधु॑मती । मधु॑म॒त्यश्वि॑ना । मधु॑म॒तीति॒ मधु॑ - म॒ती॒ । अश्वि॑ना सू॒नृता॑वती । सू॒नृता॑व॒तीति॑ सू॒नृता᳚ - व॒ती॒ ॥ तया॑ य॒ज्ञ्म् । य॒ज्ञ्म् मि॑मिक्षतम् । मि॒मि॒क्ष॒त॒मिति॑ मिमिक्षतम् ॥ उ॒प॒या॒मगृ॑हीतोऽसि । उ॒प॒या॒मगृ॑हीत॒ इत्यु॑पया॒म - गृ॒ही॒तः॒ । अ॒स्य॒श्विभ्या᳚म् । अ॒श्विभ्या᳚म् त्वा । अ॒श्विभ्या॒मित्य॒श्वि - भ्या॒म् । त्वै॒षः । ए॒ष ते᳚ । ते॒ योनिः॑ । योनि॒र् माद्ध्वी᳚भ्याम् । माद्ध्वी᳚भ्याम् त्वा । त्वेति॑ त्वा । \newline

\textbf{Jatai Paata} \newline

1. या वां᳚ ॅवां॒ ॅया या वा᳚म् । \newline
2. वा॒म् कशा॒ कशा॑ वां ॅवा॒म् कशा᳚ । \newline
3. कशा॒ मधु॑मती॒ मधु॑मती॒ कशा॒ कशा॒ मधु॑मती । \newline
4. मधु॑म॒त्यश्वि॒ना ऽश्वि॑ना॒ मधु॑मती॒ मधु॑म॒त्यश्वि॑ना । \newline
5. मधु॑म॒तीति॒ मधु॑ - म॒ती॒ । \newline
6. अश्वि॑ना सू॒नृता॑वती सू॒नृता॑व॒त्यश्वि॒ना ऽश्वि॑ना सू॒नृता॑वती । \newline
7. सू॒नृता॑व॒तीति॑ सू॒नृता᳚ - व॒ती॒ । \newline
8. तया॑ य॒ज्ञ्ं ॅय॒ज्ञ्म् तया॒ तया॑ य॒ज्ञ्म् । \newline
9. य॒ज्ञ्म् मि॑मिक्षतम् मिमिक्षतं ॅय॒ज्ञ्ं ॅय॒ज्ञ्म् मि॑मिक्षतम् । \newline
10. मि॒मि॒क्ष॒त॒मिति॑ मिमिक्षतम् । \newline
11. उ॒प॒या॒मगृ॑हीतो ऽस्यस्युपया॒मगृ॑हीत उपया॒मगृ॑हीतो ऽसि । \newline
12. उ॒प॒या॒मगृ॑हीत॒ इत्यु॑पया॒म - गृ॒ही॒तः॒ । \newline
13. अ॒स्य॒श्विभ्या॑ म॒श्विभ्या॑ मस्यस्य॒श्विभ्या᳚म् । \newline
14. अ॒श्विभ्या᳚म् त्वा त्वा॒ ऽश्विभ्या॑ म॒श्विभ्या᳚म् त्वा । \newline
15. अ॒श्विभ्या॒मित्य॒श्वि - भ्या॒म् । \newline
16. त्वै॒ष ए॒ष त्वा᳚ त्वै॒षः । \newline
17. ए॒ष ते॑ त ए॒ष ए॒ष ते᳚ । \newline
18. ते॒ योनि॒र् योनि॑ स्ते ते॒ योनिः॑ । \newline
19. योनि॒र् माद्ध्वी᳚भ्या॒म् माद्ध्वी᳚भ्यां॒ ॅयोनि॒र् योनि॒र् माद्ध्वी᳚भ्याम् । \newline
20. माद्ध्वी᳚भ्याम् त्वा त्वा॒ माद्ध्वी᳚भ्या॒म् माद्ध्वी᳚भ्याम् त्वा । \newline
21. त्वेति॑ त्वा । \newline

\textbf{Ghana Paata } \newline

1. या वां᳚ ॅवां॒ ॅया या वा॒म् कशा॒ कशा॑ वां॒ ॅया या वा॒म् कशा᳚ । \newline
2. वा॒म् कशा॒ कशा॑ वां ॅवा॒म् कशा॒ मधु॑मती॒ मधु॑मती॒ कशा॑ वां ॅवा॒म् कशा॒ मधु॑मती । \newline
3. कशा॒ मधु॑मती॒ मधु॑मती॒ कशा॒ कशा॒ मधु॑म॒त्यश्वि॒ना ऽश्वि॑ना॒ मधु॑मती॒ कशा॒ कशा॒ मधु॑म॒त्यश्वि॑ना । \newline
4. मधु॑म॒त्यश्वि॒ना ऽश्वि॑ना॒ मधु॑मती॒ मधु॑म॒त्यश्वि॑ना सू॒नृता॑वती सू॒नृता॑व॒त्यश्वि॑ना॒ मधु॑मती॒ मधु॑म॒त्यश्वि॑ना सू॒नृता॑वती । \newline
5. मधु॑म॒तीति॒ मधु॑ - म॒ती॒ । \newline
6. अश्वि॑ना सू॒नृता॑वती सू॒नृता॑व॒त्यश्वि॒ना ऽश्वि॑ना सू॒नृता॑वती । \newline
7. सू॒नृता॑व॒तीति॑ सू॒नृता᳚ - व॒ती॒ । \newline
8. तया॑ य॒ज्ञ्ं ॅय॒ज्ञ्म् तया॒ तया॑ य॒ज्ञ्म् मि॑मिक्षतम् मिमिक्षतं ॅय॒ज्ञ्म् तया॒ तया॑ य॒ज्ञ्म् मि॑मिक्षतम् । \newline
9. य॒ज्ञ्म् मि॑मिक्षतम् मिमिक्षतं ॅय॒ज्ञ्ं ॅय॒ज्ञ्म् मि॑मिक्षतम् । \newline
10. मि॒मि॒क्ष॒त॒मिति॑ मिमिक्षतम् । \newline
11. उ॒प॒या॒मगृ॑हीतो ऽस्यस्युपया॒मगृ॑हीत उपया॒मगृ॑हीतो ऽस्य॒श्विभ्या॑ म॒श्विभ्या॑ मस्युपया॒मगृ॑हीत उपया॒मगृ॑हीतो ऽस्य॒श्विभ्या᳚म् । \newline
12. उ॒प॒या॒मगृ॑हीत॒ इत्यु॑पया॒म - गृ॒ही॒तः॒ । \newline
13. अ॒स्य॒श्विभ्या॑ म॒श्विभ्या॑ मस्यस्य॒श्विभ्या᳚म् त्वा त्वा॒ ऽश्विभ्या॑ मस्यस्य॒श्विभ्या᳚म् त्वा । \newline
14. अ॒श्विभ्या᳚म् त्वा त्वा॒ ऽश्विभ्या॑ म॒श्विभ्या᳚म् त्वै॒ष ए॒ष त्वा॒ ऽश्विभ्या॑ म॒श्विभ्या᳚म् त्वै॒षः । \newline
15. अ॒श्विभ्या॒मित्य॒श्वि - भ्या॒म् । \newline
16. त्वै॒ष ए॒ष त्वा᳚ त्वै॒ष ते॑ त ए॒ष त्वा᳚ त्वै॒ष ते᳚ । \newline
17. ए॒ष ते॑ त ए॒ष ए॒ष ते॒ योनि॒र् योनि॑ स्त ए॒ष ए॒ष ते॒ योनिः॑ । \newline
18. ते॒ योनि॒र् योनि॑ स्ते ते॒ योनि॒र् माद्ध्वी᳚भ्या॒म् माद्ध्वी᳚भ्यां॒ ॅयोनि॑ स्ते ते॒ योनि॒र् माद्ध्वी᳚भ्याम् । \newline
19. योनि॒र् माद्ध्वी᳚भ्या॒म् माद्ध्वी᳚भ्यां॒ ॅयोनि॒र् योनि॒र् माद्ध्वी᳚भ्याम् त्वा त्वा॒ माद्ध्वी᳚भ्यां॒ ॅयोनि॒र् योनि॒र् माद्ध्वी᳚भ्याम् त्वा । \newline
20. माद्ध्वी᳚भ्याम् त्वा त्वा॒ माद्ध्वी᳚भ्या॒म् माद्ध्वी᳚भ्याम् त्वा । \newline
21. त्वेति॑ त्वा । \newline
\pagebreak
\markright{ TS 1.4.7.1  \hfill https://www.vedavms.in \hfill}
\addcontentsline{toc}{section}{ TS 1.4.7.1 }
\section*{ TS 1.4.7.1 }

\textbf{TS 1.4.7.1 } \newline
\textbf{Samhita Paata} \newline

प्रा॒त॒र्युजौ॒ वि मु॑च्येथा॒-मश्वि॑ना॒वेह ग॑च्छतं । अ॒स्य सोम॑स्य पी॒तये᳚ ॥ उ॒प॒या॒मगृ॑हीतो-ऽस्य॒श्विभ्यां᳚ त्वै॒ष ते॒ योनि॑र॒श्विभ्यां᳚ त्वा ॥꣡꣡꣡꣡꣡꣡꣡꣡꣡꣡꣡꣡꣡꣡꣡꣡꣡꣡ \newline

\textbf{Pada Paata} \newline

प्रा॒त॒र्युजा॒विति॑ प्रातः - युजौ᳚ । वीति॑ । मु॒च्ये॒था॒म् । अश्वि॑नौ । एति॑ । इ॒ह । ग॒च्छ॒त॒म् ॥ अ॒स्य । सोम॑स्य । पी॒तये᳚ ॥ उ॒प॒या॒मगृ॑हीत॒ इत्यु॑पया॒म - गृ॒ह॒तः॒ । अ॒सि॒ । अ॒श्विभ्या॒मित्य॒श्वि - भ्या॒म् । त्वा॒ । ए॒षः । ते॒ । योनिः॑ । अ॒श्विभ्या॒मित्य॒श्वि - भ्या॒म् । त्वा॒ ॥  \newline


\textbf{Krama Paata} \newline

प्रा॒त॒र्युजौ॒ वि । प्रा॒त॒र्युजा॒विति॑ प्रातः - युजौ᳚ । वि मु॑च्येथाम् । मु॒च्ये॒था॒मश्वि॑नौ । अश्वि॑ना॒वा । एह । इ॒ह ग॑च्छतम् । ग॒च्छ॒त॒मिति॑ गच्छतम् ॥ अ॒स्य सोम॑स्य । सोम॑स्य पी॒तये᳚ । पी॒तय॒ इति॑ पी॒तये᳚ ॥ उ॒प॒या॒मगृ॑हीतोऽसि । उ॒प॒या॒मगृ॑हीत॒ इत्यु॑पया॒म - गृ॒ही॒तः॒ । अ॒स्य॒श्विभ्या᳚म् । अ॒श्विभ्या᳚म् त्वा । अ॒श्विभ्या॒मित्य॒श्वि - भ्या॒म् । त्वै॒षः । ए॒ष ते᳚ । ते॒ योनिः॑ । योनि॑र॒श्विभ्या᳚म् । अ॒श्विभ्या᳚म् त्वा । अ॒श्विभ्या॒मित्य॒श्वि - भ्या॒म् । त्वेति॑ त्वा । \newline

\textbf{Jatai Paata} \newline

1. प्रा॒त॒र्युजौ॒ वि वि प्रा॑त॒र्युजौ᳚ प्रात॒र्युजौ॒ वि । \newline
2. प्रा॒त॒र्युजा॒विति॑ प्रातः - युजौ᳚ । \newline
3. वि मु॑च्येथाम् मुच्येथां॒ ॅवि वि मु॑च्येथाम् । \newline
4. मु॒च्ये॒था॒ मश्वि॑ना॒ वश्वि॑नौ मुच्येथाम् मुच्येथा॒ मश्वि॑नौ । \newline
5. अश्वि॑ना॒ वा ऽश्वि॑ना॒ वश्वि॑ना॒ वा । \newline
6. एहे हेह । \newline
7. इ॒ह ग॑च्छतम् गच्छत मि॒हे ह ग॑च्छतम् । \newline
8. ग॒च्छ॒त॒मिति॑ गच्छतम् । \newline
9. अ॒स्य सोम॑स्य॒ सोम॑स्या॒स्यास्य सोम॑स्य । \newline
10. सोम॑स्य पी॒तये॑ पी॒तये॒ सोम॑स्य॒ सोम॑स्य पी॒तये᳚ । \newline
11. पी॒तय॒ इति॑ पी॒तये᳚ । \newline
12. उ॒प॒या॒मगृ॑हीतो ऽस्यस्युपया॒मगृ॑हीत उपया॒मगृ॑हीतो ऽसि । \newline
13. उ॒प॒या॒मगृ॑हीत॒ इत्यु॑पया॒म - गृ॒ह॒तः॒ । \newline
14. अ॒स्य॒श्विभ्या॑ म॒श्विभ्या॑ मस्यस्य॒श्विभ्या᳚म् । \newline
15. अ॒श्विभ्या᳚म् त्वा त्वा॒ ऽश्विभ्या॑ म॒श्विभ्या᳚म् त्वा । \newline
16. अ॒श्विभ्या॒मित्य॒श्वि - भ्या॒म् । \newline
17. त्वै॒ष ए॒ष त्वा᳚ त्वै॒षः । \newline
18. ए॒ष ते॑ त ए॒ष ए॒ष ते᳚ । \newline
19. ते॒ योनि॒र् योनि॑ स्ते ते॒ योनिः॑ । \newline
20. योनि॑ र॒श्विभ्या॑ म॒श्विभ्यां॒ ॅयोनि॒र् योनि॑ र॒श्विभ्या᳚म् । \newline
21. अ॒श्विभ्या᳚म् त्वा त्वा॒ ऽश्विभ्या॑ म॒श्विभ्या᳚म् त्वा । \newline
22. अ॒श्विभ्या॒मित्य॒श्वि - भ्या॒म् । \newline
23. त्वेति॑ त्वा । \newline

\textbf{Ghana Paata } \newline

1. प्रा॒त॒र्युजौ॒ वि वि प्रा॑त॒र्युजौ᳚ प्रात॒र्युजौ॒ वि मु॑च्येथाम् मुच्येथां॒ ॅवि प्रा॑त॒र्युजौ᳚ प्रात॒र्युजौ॒ वि मु॑च्येथाम् । \newline
2. प्रा॒त॒र्युजा॒विति॑ प्रातः - युजौ᳚ । \newline
3. वि मु॑च्येथाम् मुच्येथां॒ ॅवि वि मु॑च्येथा॒ मश्वि॑ना॒ वश्वि॑नौ मुच्येथां॒ ॅवि वि मु॑च्येथा॒ मश्वि॑नौ । \newline
4. मु॒च्ये॒था॒ मश्वि॑ना॒ वश्वि॑नौ मुच्येथाम् मुच्येथा॒ मश्वि॑ना॒ वा ऽश्वि॑नौ मुच्येथाम् मुच्येथा॒ मश्वि॑ना॒ वा । \newline
5. अश्वि॑ना॒ वा ऽश्वि॑ना॒ वश्वि॑ना॒ वेहे हा ऽश्वि॑ना॒ वश्वि॑ना॒ वेह । \newline
6. एहे हेह ग॑च्छतम् गच्छत मि॒हेह ग॑च्छतम् । \newline
7. इ॒ह ग॑च्छतम् गच्छत मि॒हे ह ग॑च्छतम् । \newline
8. ग॒च्छ॒त॒मिति॑ गच्छतम् । \newline
9. अ॒स्य सोम॑स्य॒ सोम॑स्या॒स्यास्य सोम॑स्य पी॒तये॑ पी॒तये॒ सोम॑ स्या॒स्यास्य सोम॑स्य पी॒तये᳚ । \newline
10. सोम॑स्य पी॒तये॑ पी॒तये॒ सोम॑स्य॒ सोम॑स्य पी॒तये᳚ । \newline
11. पी॒तय॒ इति॑ पी॒तये᳚ । \newline
12. उ॒प॒या॒मगृ॑हीतो ऽस्य स्युपया॒मगृ॑हीत उपया॒मगृ॑हीतो ऽस्य॒श्विभ्या॑ म॒श्विभ्या॑ मस्युपया॒मगृ॑हीत उपया॒मगृ॑हीतो ऽस्य॒श्विभ्या᳚म् । \newline
13. उ॒प॒या॒मगृ॑हीत॒ इत्यु॑पया॒म - गृ॒ह॒तः॒ । \newline
14. अ॒स्य॒श्विभ्या॑ म॒श्विभ्या॑ मस्यस्य॒श्विभ्या᳚म् त्वा त्वा॒ ऽश्विभ्या॑ मस्यस्य॒श्विभ्या᳚म् त्वा । \newline
15. अ॒श्विभ्या᳚म् त्वा त्वा॒ ऽश्विभ्या॑ म॒श्विभ्या᳚म् त्वै॒ष ए॒ष त्वा॒ ऽश्विभ्या॑ म॒श्विभ्या᳚म् त्वै॒षः । \newline
16. अ॒श्विभ्या॒मित्य॒श्वि - भ्या॒म् । \newline
17. त्वै॒ष ए॒ष त्वा᳚ त्वै॒ष ते॑ त ए॒ष त्वा᳚ त्वै॒ष ते᳚ । \newline
18. ए॒ष ते॑ त ए॒ष ए॒ष ते॒ योनि॒र् योनि॑ स्त ए॒ष ए॒ष ते॒ योनिः॑ । \newline
19. ते॒ योनि॒र् योनि॑ स्ते ते॒ योनि॑ र॒श्विभ्या॑ म॒श्विभ्यां॒ ॅयोनि॑ स्ते ते॒ योनि॑ र॒श्विभ्या᳚म् । \newline
20. योनि॑ र॒श्विभ्या॑ म॒श्विभ्यां॒ ॅयोनि॒र् योनि॑ र॒श्विभ्या᳚म् त्वा त्वा॒ ऽश्विभ्यां॒ ॅयोनि॒र् योनि॑ र॒श्विभ्या᳚म् त्वा । \newline
21. अ॒श्विभ्या᳚म् त्वा त्वा॒ ऽश्विभ्या॑ म॒श्विभ्या᳚म् त्वा । \newline
22. अ॒श्विभ्या॒मित्य॒श्वि - भ्या॒म् । \newline
23. त्वेति॑ त्वा । \newline
\pagebreak
\markright{ TS 1.4.8.1  \hfill https://www.vedavms.in \hfill}
\addcontentsline{toc}{section}{ TS 1.4.8.1 }
\section*{ TS 1.4.8.1 }

\textbf{TS 1.4.8.1 } \newline
\textbf{Samhita Paata} \newline

अ॒यं ॅवे॒नश्चो॑दय॒त् पृश्ञि॑गर्भा॒ ज्योति॑र्जरायू॒ रज॑सो वि॒माने᳚ । इ॒मम॒पाꣳ सं॑ग॒मे सूर्य॑स्य॒ शिशुं॒ न विप्रा॑ म॒तिभी॑ रिहन्ति ॥ उ॒प॒या॒मगृ॑हीतो-ऽसि॒ शण्डा॑य त्वै॒ष ते॒ योनि॑र् वी॒रतां᳚ पाहि ॥ \newline

\textbf{Pada Paata} \newline

अ॒यम् । वे॒नः । चो॒द॒य॒त्॒ । पृश्नि॑गर्भा॒ इति॒ पृश्नि॑ - ग॒र्भाः॒ । ज्योति॑र्जरायु॒रिति॒ ज्योतिः॑ - ज॒रा॒युः॒ । रज॑सः । वि॒मान॒ इति॑ वि - माने᳚ ॥ इ॒मम् । अ॒पाम् । स॒गं॒म इति॑ सं - ग॒मे । सूर्य॑स्य । शिशु᳚म् । न । विप्राः᳚ । म॒तिभि॒रिति॑ म॒ति - भिः॒ । रि॒ह॒न्ति॒ ॥ उ॒प॒या॒मगृ॑हीत॒ इत्यु॑पया॒म - गृही॒तः॒ । अ॒सि॒ । शण्डा॑य । त्वा॒ । ए॒षः । ते॒ । योनिः॑ । वी॒रता᳚म् । पा॒हि॒ ॥  \newline


\textbf{Krama Paata} \newline

अ॒यं ॅवे॒नः । वे॒नश्चो॑दयत् । चो॒द॒य॒त्,पृश्ञि॑गर्भाः । पृश्ञि॑गर्भा॒ ज्योति॑र्जरायुः । पृश्ञि॑गर्भा॒ इति॒ पृश्ञि॑ - ग॒र्भाः॒ । ज्योति॑र्जरायू॒ रज॑सः । ज्योति॑र्जरायु॒रिति॒ ज्योतिः॑ - ज॒रा॒युः॒ । रज॑सो वि॒माने᳚ । वि॒मान॒ इति॑ वि - माने᳚ ॥ इ॒मम॒पाम् । अ॒पाꣳ स॑ङ्ग॒मे । स॒ङ्ग॒मे सूर्य॑स्य । स॒ङ्ग॒म इति॑ सम् - ग॒मे । सूर्य॑स्य॒ शिशु᳚म् । शिशु॒न्न । न विप्राः᳚ । विप्रा॑ म॒तिभिः॑ । म॒तिभी॑ रिहन्ति । म॒तिभि॒रिति॑ म॒ति - भिः॒ । रि॒ह॒न्तीति॑ रिहन्ति ॥ उ॒प॒या॒मगृ॑हीतोऽसि । उ॒प॒या॒मगृ॑हीत॒ इत्यु॑पया॒म - गृ॒ही॒तः॒ । अ॒सि॒ शण्डा॑य । शण्डा॑य त्वा । त्वै॒षः । ए॒ष ते᳚ । ते॒ योनिः॑ । योनि॑र् वी॒रता᳚म् । वी॒रता᳚म् पाहि । पा॒हीति॑ पाहि । \newline

\textbf{Jatai Paata} \newline

1. अ॒यं ॅवे॒नो वे॒नो॑ ऽय म॒यं ॅवे॒नः । \newline
2. वे॒न श्चो॑दयच् चोदयद् वे॒नो वे॒न श्चो॑दयत् । \newline
3. चो॒द॒य॒त् पृश्ञि॑गर्भाः॒ पृश्ञि॑गर्भा श्चोदयच् चोदय॒त् पृश्ञि॑गर्भाः । \newline
4. पृश्ञि॑गर्भा॒ ज्योति॑र्जरायु॒र् ज्योति॑र्जरायुः॒ पृश्ञि॑गर्भाः॒ पृश्ञि॑गर्भा॒ ज्योति॑र्जरायुः । \newline
5. पृश्ञि॑गर्भा॒ इति॒ पृश्ञि॑ - ग॒र्भाः॒ । \newline
6. ज्योति॑र्जरायू॒ रज॑सो॒ रज॑सो॒ ज्योति॑र्जरायु॒र् ज्योति॑र्जरायू॒ रज॑सः । \newline
7. ज्योति॑र्जरायु॒रिति॒ ज्योतिः॑ - ज॒रा॒युः॒ । \newline
8. रज॑सो वि॒माने॑ वि॒माने॒ रज॑सो॒ रज॑सो वि॒माने᳚ । \newline
9. वि॒मान॒ इति॑ वि - माने᳚ । \newline
10. इ॒म म॒पा म॒पा मि॒म मि॒म म॒पाम् । \newline
11. अ॒पाꣳ स॑ङ्ग॒मे स॑ङ्ग॒मे॑ ऽपा म॒पाꣳ स॑ङ्ग॒मे । \newline
12. स॒ङ्ग॒मे सूर्य॑स्य॒ सूर्य॑स्य सङ्ग॒मे स॑ङ्ग॒मे सूर्य॑स्य । \newline
13. सं॒ग॒म इति॑ सं - ग॒मे । \newline
14. सूर्य॑स्य॒ शिशु॒(ग्म्॒) शिशु॒(ग्म्॒) सूर्य॑स्य॒ सूर्य॑स्य॒ शिशु᳚म् । \newline
15. शिशु॒न्न न शिशु॒(ग्म्॒) शिशु॒न्न । \newline
16. न विप्रा॒ विप्रा॒ न न विप्राः᳚ । \newline
17. विप्रा॑ म॒तिभि॑र् म॒तिभि॒र् विप्रा॒ विप्रा॑ म॒तिभिः॑ । \newline
18. म॒तिभी॑ रिहन्ति रिहन्ति म॒तिभि॑र् म॒तिभी॑ रिहन्ति । \newline
19. म॒तिभि॒रिति॑ म॒ति - भिः॒ । \newline
20. रि॒ह॒न्तीति॑ रिहन्ति । \newline
21. उ॒प॒या॒मगृ॑हीतो ऽस्यस्युपया॒मगृ॑हीत उपया॒मगृ॑हीतो ऽसि । \newline
22. उ॒प॒या॒मगृ॑हीत॒ इत्यु॑पया॒म - गृ॒ही॒तः॒ । \newline
23. अ॒सि॒ शण्डा॑य॒ शण्डा॑यास्यसि॒ शण्डा॑य । \newline
24. शण्डा॑य त्वा त्वा॒ शण्डा॑य॒ शण्डा॑य त्वा । \newline
25. त्वै॒ष ए॒ष त्वा᳚ त्वै॒षः । \newline
26. ए॒ष ते॑ त ए॒ष ए॒ष ते᳚ । \newline
27. ते॒ योनि॒र् योनि॑ स्ते ते॒ योनिः॑ । \newline
28. योनि॑र् वी॒रतां᳚ ॅवी॒रतां॒ ॅयोनि॒र् योनि॑र् वी॒रता᳚म् । \newline
29. वी॒रता᳚म् पाहि पाहि वी॒रतां᳚ ॅवी॒रता᳚म् पाहि । \newline
30. पा॒हीति॑ पाहि । \newline

\textbf{Ghana Paata } \newline

1. अ॒यं ॅवे॒नो वे॒नो॑ ऽय म॒यं ॅवे॒न श्चो॑दयच् चोदयद् वे॒नो॑ ऽय म॒यं ॅवे॒न श्चो॑दयत् । \newline
2. वे॒न श्चो॑दयच् चोदयद् वे॒नो वे॒न श्चो॑दय॒त् पृश्ञि॑गर्भाः॒ पृश्ञि॑गर्भा श्चोदयद् वे॒नो वे॒न श्चो॑दय॒त् पृश्ञि॑गर्भाः । \newline
3. चो॒द॒य॒त् पृश्ञि॑गर्भाः॒ पृश्ञि॑गर्भा श्चोदयच् चोदय॒त् पृश्ञि॑गर्भा॒ ज्योति॑र्ज रायु॒र् ज्योति॑र्जरायुः॒ पृश्ञि॑गर्भा श्चोदयच् चोदय॒त् पृश्ञि॑गर्भा॒ ज्योति॑र्जरायुः । \newline
4. पृश्ञि॑गर्भा॒ ज्योति॑र्जरायु॒र् ज्योति॑र्जरायुः॒ पृश्ञि॑गर्भाः॒ पृश्ञि॑गर्भा॒ ज्योति॑र्जरायू॒ रज॑सो॒ रज॑सो॒ ज्योति॑र्जरायुः॒ पृश्ञि॑गर्भाः॒ पृश्ञि॑गर्भा॒ ज्योति॑र्जरायू॒ रज॑सः । \newline
5. पृश्ञि॑गर्भा॒ इति॒ पृश्ञि॑ - ग॒र्भाः॒ । \newline
6. ज्योति॑र्जरायू॒ रज॑सो॒ रज॑सो॒ ज्योति॑र्जरायु॒र् ज्योति॑र्जरायू॒ रज॑सो वि॒माने॑ वि॒माने॒ रज॑सो॒ ज्योति॑र्जरायु॒र् ज्योति॑र्जरायू॒ रज॑सो वि॒माने᳚ । \newline
7. ज्योति॑र्जरायु॒रिति॒ ज्योतिः॑ - ज॒रा॒युः॒ । \newline
8. रज॑सो वि॒माने॑ वि॒माने॒ रज॑सो॒ रज॑सो वि॒माने᳚ । \newline
9. वि॒मान॒ इति॑ वि - माने᳚ । \newline
10. इ॒म म॒पा म॒पा मि॒म मि॒म म॒पाꣳ स॑ङ्ग॒मे स॑ङ्ग॒मे॑ ऽपा मि॒म मि॒म म॒पाꣳ स॑ङ्ग॒मे । \newline
11. अ॒पाꣳ स॑ङ्ग॒मे स॑ङ्ग॒मे॑ ऽपा म॒पाꣳ स॑ङ्ग॒मे सूर्य॑स्य॒ सूर्य॑स्य सङ्ग॒मे॑ ऽपा म॒पाꣳ स॑ङ्ग॒मे सूर्य॑स्य । \newline
12. स॒ङ्ग॒मे सूर्य॑स्य॒ सूर्य॑स्य सङ्ग॒मे स॑ङ्ग॒मे सूर्य॑स्य॒ शिशुꣳ॒॒ शिशु॒(ग्म्॒) सूर्य॑स्य सङ्ग॒मे स॑ङ्ग॒मे सूर्य॑स्य॒ शिशु᳚म् । \newline
13. सं॒ग॒म इति॑ सं - ग॒मे । \newline
14. सूर्य॑स्य॒ शिशुꣳ॒॒ शिशु॒(ग्म्॒) सूर्य॑स्य॒ सूर्य॑स्य॒ शिशु॒म् न न शिशु॒(ग्म्॒) सूर्य॑स्य॒ सूर्य॑स्य॒ शिशु॒म् न । \newline
15. शिशु॒म् न न शिशुꣳ॒॒ शिशु॒म् न विप्रा॒ विप्रा॒ न शिशुꣳ॒॒ शिशु॒म् न विप्राः᳚ । \newline
16. न विप्रा॒ विप्रा॒ न न विप्रा॑ म॒तिभि॑र् म॒तिभि॒र् विप्रा॒ न न विप्रा॑ म॒तिभिः॑ । \newline
17. विप्रा॑ म॒तिभि॑र् म॒तिभि॒र् विप्रा॒ विप्रा॑ म॒तिभी॑ रिहन्ति रिहन्ति म॒तिभि॒र् विप्रा॒ विप्रा॑ म॒तिभी॑ रिहन्ति । \newline
18. म॒तिभी॑ रिहन्ति रिहन्ति म॒तिभि॑र् म॒तिभी॑ रिहन्ति । \newline
19. म॒तिभि॒रिति॑ म॒ति - भिः॒ । \newline
20. रि॒ह॒न्तीति॑ रिहन्ति । \newline
21. उ॒प॒या॒मगृ॑हीतो ऽस्य स्युपया॒मगृ॑हीत उपया॒मगृ॑हीतो ऽसि॒ शण्डा॑य॒ शण्डा॑ यास्युपया॒मगृ॑हीत उपया॒मगृ॑हीतो ऽसि॒ शण्डा॑य । \newline
22. उ॒प॒या॒मगृ॑हीत॒ इत्यु॑पया॒म - गृ॒ही॒तः॒ । \newline
23. अ॒सि॒ शण्डा॑य॒ शण्डा॑ यास्यसि॒ शण्डा॑य त्वा त्वा॒ शण्डा॑ यास्यसि॒ शण्डा॑य त्वा । \newline
24. शण्डा॑य त्वा त्वा॒ शण्डा॑य॒ शण्डा॑य त्वै॒ष ए॒ष त्वा॒ शण्डा॑य॒ शण्डा॑य त्वै॒षः । \newline
25. त्वै॒ष ए॒ष त्वा᳚ त्वै॒ष ते॑ त ए॒ष त्वा᳚ त्वै॒ष ते᳚ । \newline
26. ए॒ष ते॑ त ए॒ष ए॒ष ते॒ योनि॒र् योनि॑ स्त ए॒ष ए॒ष ते॒ योनिः॑ । \newline
27. ते॒ योनि॒र् योनि॑ स्ते ते॒ योनि॑र् वी॒रतां᳚ ॅवी॒रतां॒ ॅयोनि॑ स्ते ते॒ योनि॑र् वी॒रता᳚म् । \newline
28. योनि॑र् वी॒रतां᳚ ॅवी॒रतां॒ ॅयोनि॒र् योनि॑र् वी॒रता᳚म् पाहि पाहि वी॒रतां॒ ॅयोनि॒र् योनि॑र् वी॒रता᳚म् पाहि । \newline
29. वी॒रता᳚म् पाहि पाहि वी॒रतां᳚ ॅवी॒रता᳚म् पाहि । \newline
30. पा॒हीति॑ पाहि । \newline
\pagebreak
\markright{ TS 1.4.9.1  \hfill https://www.vedavms.in \hfill}
\addcontentsline{toc}{section}{ TS 1.4.9.1 }
\section*{ TS 1.4.9.1 }

\textbf{TS 1.4.9.1 } \newline
\textbf{Samhita Paata} \newline

तं प्र॒त्नथा॑ पू॒र्वथा॑ वि॒श्वथे॒मथा᳚ ज्ये॒ष्ठता॑तिं बर्.हि॒षदꣳ॑ सुव॒र्विदं॑ प्रतीची॒नं ॅवृ॒जनं॑ दोहसे गि॒राऽऽशुं जय॑न्त॒मनु॒ यासु॒ वर्द्ध॑से । उ॒प॒या॒मगृ॑हीतो-ऽसि॒ मर्का॑य त्वै॒ष ते॒ योनिः॑ प्र॒जाः पा॑हि ॥ \newline

\textbf{Pada Paata} \newline

तम् । प्र॒त्नथा᳚ । पू॒र्वथा᳚ । वि॒श्वथा᳚ । इ॒मथा᳚ । ज्ये॒ष्ठता॑ति॒मिति॑ ज्ये॒ष्ठ - ता॒ति॒म् । ब॒र्॒.हि॒षद॒मिति॑ बर्.हि - सद᳚म् । सु॒व॒र्विद॒मिति॑ सुवः - विद᳚म् । प्र॒ती॒ची॒नम् । वृ॒जन᳚म् । दो॒ह॒से॒ । गि॒रा । आ॒शुम् । जय॑न्तम् । अन्विति॑ । यासु॑ । वर्ध॑से ॥ उ॒प॒या॒मगृ॑हीत॒ इत्यु॑पया॒म - गृ॒ही॒तः॒ । अ॒सि॒ । मर्का॑य । त्वा॒ । ए॒षः । ते॒ । योनिः॑ । प्र॒जा इति॑ प्र - जाः । पा॒हि॒ ॥  \newline


\textbf{Krama Paata} \newline

तम् प्र॒त्नथा᳚ । प्र॒त्नथा॑ पू॒र्वथा᳚ । पू॒र्वथा॑ वि॒श्वथा᳚ । वि॒श्वथे॒मथा᳚ । इ॒मथा᳚ ज्ये॒ष्ठता॑तिम् । ज्ये॒ष्ठता॑तिम् बर्.हि॒षद᳚म् । ज्ये॒ष्ठता॑ति॒मिति॑ ज्ये॒ष्ठ - ता॒ति॒म् । ब॒र्॒.हि॒षदꣳ॑ सुव॒र्विद᳚म् । ब॒र्.॒हि॒षद॒मिति॑ बर्.हि - सद᳚म् । सु॒व॒र्विद॑म् प्रतीची॒नम् । सु॒व॒र्विद॒मिति॑ सुवः - विद᳚म् । प्र॒ती॒ची॒नं ॅवृ॒जन᳚म् । वृ॒जन॑म् दोहसे । दो॒ह॒से॒ गि॒रा । गि॒राऽऽशुम् । आ॒शुम् जय॑न्तम् । जय॑न्त॒मनु॑ । अनु॒ यासु॑ । यासु॒ वर्द्ध॑से । वर्द्ध॑स॒ इति॒ वर्द्ध॑से ॥ उ॒प॒या॒मगृ॑हीतोऽसि । उ॒प॒या॒मगृ॑हीत॒ इत्यु॑पया॒म - गृ॒ही॒तः॒ । अ॒सि॒ मर्का॑य । मर्का॑य त्वा । त्वै॒षः । ए॒ष ते᳚ । ते॒ योनिः॑ । योनिः॑ प्र॒जाः । प्र॒जाः पा॑हि । प्र॒जा इति॑ प्र - जाः । पा॒हीति॑ पाहि । \newline

\textbf{Jatai Paata} \newline

1. तम् प्र॒त्नथा᳚ प्र॒त्नथा॒ तम् तम् प्र॒त्नथा᳚ । \newline
2. प्र॒त्नथा॑ पू॒र्वथा॑ पू॒र्वथा᳚ प्र॒त्नथा᳚ प्र॒त्नथा॑ पू॒र्वथा᳚ । \newline
3. पू॒र्वथा॑ वि॒श्वथा॑ वि॒श्वथा॑ पू॒र्वथा॑ पू॒र्वथा॑ वि॒श्वथा᳚ । \newline
4. वि॒श्वथे॒मथे॒मथा॑ वि॒श्वथा॑ वि॒श्वथे॒मथा᳚ । \newline
5. इ॒मथा᳚ ज्ये॒ष्ठता॑तिम् ज्ये॒ष्ठता॑ति मि॒मथे॒मथा᳚ ज्ये॒ष्ठता॑तिम् । \newline
6. ज्ये॒ष्ठता॑तिम् बर्.हि॒षद॑म् बर्.हि॒षद॑म् ज्ये॒ष्ठता॑तिम् ज्ये॒ष्ठता॑तिम् बर्.हि॒षद᳚म् । \newline
7. ज्ये॒ष्ठता॑ति॒मिति॑ ज्ये॒ष्ठ - ता॒ति॒म् । \newline
8. ब॒र्॒.हि॒षद(ग्म्॑) सुव॒र्विद(ग्म्॑) सुव॒र्विद॑म् बर्.हि॒षद॑म् बर्.हि॒षद(ग्म्॑) सुव॒र्विद᳚म् । \newline
9. ब॒र्॒.हि॒षद॒मिति॑ बर्.हि - सद᳚म् । \newline
10. सु॒व॒र्विद॑म् प्रतीची॒नम् प्र॑तीची॒नꣳ सु॑व॒र्विद(ग्म्॑) सुव॒र्विद॑म् प्रतीची॒नम् । \newline
11. सु॒व॒र्विद॒मिति॑ सुवः - विद᳚म् । \newline
12. प्र॒ती॒ची॒नं ॅवृ॒जनं॑ ॅवृ॒जन॑म् प्रतीची॒नम् प्र॑तीची॒नं ॅवृ॒जन᳚म् । \newline
13. वृ॒जन॑म् दोहसे दोहसे वृ॒जनं॑ ॅवृ॒जन॑म् दोहसे । \newline
14. दो॒ह॒से॒ गि॒रा गि॒रा दो॑हसे दोहसे गि॒रा । \newline
15. गि॒रा ऽऽशु मा॒शुम् गि॒रा गि॒रा ऽऽशुम् । \newline
16. आ॒शुम् जय॑न्त॒म् जय॑न्त मा॒शु मा॒शुम् जय॑न्तम् । \newline
17. जय॑न्त॒ मन्वनु॒ जय॑न्त॒म् जय॑न्त॒ मनु॑ । \newline
18. अनु॒ यासु॒ यास्वन्वनु॒ यासु॑ । \newline
19. यासु॒ वर्द्ध॑से॒ वर्द्ध॑से॒ यासु॒ यासु॒ वर्द्ध॑से । \newline
20. वर्द्ध॑स॒ इति॒ वर्द्ध॑से । \newline
21. उ॒प॒या॒मगृ॑हीतो ऽस्यस्युपया॒मगृ॑हीत उपया॒मगृ॑हीतो ऽसि । \newline
22. उ॒प॒या॒मगृ॑हीत॒ इत्यु॑पया॒म - गृ॒ही॒तः॒ । \newline
23. अ॒सि॒ मर्का॑य॒ मर्का॑यास्यसि॒ मर्का॑य । \newline
24. मर्का॑य त्वा त्वा॒ मर्का॑य॒ मर्का॑य त्वा । \newline
25. त्वै॒ष ए॒ष त्वा᳚ त्वै॒षः । \newline
26. ए॒ष ते॑ त ए॒ष ए॒ष ते᳚ । \newline
27. ते॒ योनि॒र् योनि॑ स्ते ते॒ योनिः॑ । \newline
28. योनिः॑ प्र॒जाः प्र॒जा योनि॒र् योनिः॑ प्र॒जाः । \newline
29. प्र॒जाः पा॑हि पाहि प्र॒जाः प्र॒जाः पा॑हि । \newline
30. प्र॒जा इति॑ प्र - जाः । \newline
31. पा॒हीति॑ पाहि । \newline

\textbf{Ghana Paata } \newline

1. तम् प्र॒त्नथा᳚ प्र॒त्नथा॒ तम् तम् प्र॒त्नथा॑ पू॒र्वथा॑ पू॒र्वथा᳚ प्र॒त्नथा॒ तम् तम् प्र॒त्नथा॑ पू॒र्वथा᳚ । \newline
2. प्र॒त्नथा॑ पू॒र्वथा॑ पू॒र्वथा᳚ प्र॒त्नथा᳚ प्र॒त्नथा॑ पू॒र्वथा॑ वि॒श्वथा॑ वि॒श्वथा॑ पू॒र्वथा᳚ प्र॒त्नथा᳚ प्र॒त्नथा॑ पू॒र्वथा॑ वि॒श्वथा᳚ । \newline
3. पू॒र्वथा॑ वि॒श्वथा॑ वि॒श्वथा॑ पू॒र्वथा॑ पू॒र्वथा॑ वि॒श्वथे॒मथे॒मथा॑ वि॒श्वथा॑ पू॒र्वथा॑ पू॒र्वथा॑ वि॒श्वथे॒मथा᳚ । \newline
4. वि॒श्वथे॒मथे॒मथा॑ वि॒श्वथा॑ वि॒श्वथे॒मथा᳚ ज्ये॒ष्ठता॑तिम् ज्ये॒ष्ठता॑ति मि॒मथा॑ वि॒श्वथा॑ वि॒श्वथे॒मथा᳚ ज्ये॒ष्ठता॑तिम् । \newline
5. इ॒मथा᳚ ज्ये॒ष्ठता॑तिम् ज्ये॒ष्ठता॑ति मि॒मथे॒मथा᳚ ज्ये॒ष्ठता॑तिम् बर्.हि॒षद॑म् बर्.हि॒षद॑म् ज्ये॒ष्ठता॑ति मि॒मथे॒मथा᳚ ज्ये॒ष्ठता॑तिम् बर्.हि॒षद᳚म् । \newline
6. ज्ये॒ष्ठता॑तिम् बर्.हि॒षद॑म् बर्.हि॒षद॑म् ज्ये॒ष्ठता॑तिम् ज्ये॒ष्ठता॑तिम् बर्.हि॒षद(ग्म्॑) सुव॒र्विद(ग्म्॑) सुव॒र्विद॑म् बर्.हि॒षद॑म् ज्ये॒ष्ठता॑तिम् ज्ये॒ष्ठता॑तिम् बर्.हि॒षद(ग्म्॑) सुव॒र्विद᳚म् । \newline
7. ज्ये॒ष्ठता॑ति॒मिति॑ ज्ये॒ष्ठ - ता॒ति॒म् । \newline
8. ब॒र्॒.हि॒षद(ग्म्॑) सुव॒र्विद(ग्म्॑) सुव॒र्विद॑म् बर्.हि॒षद॑म् बर्.हि॒षद(ग्म्॑) सुव॒र्विद॑म् प्रतीची॒नम् प्र॑तीची॒नꣳ सु॑व॒र्विद॑म् बर्.हि॒षद॑म् बर्.हि॒षद(ग्म्॑) सुव॒र्विद॑म् प्रतीची॒नम् । \newline
9. ब॒र्॒.हि॒षद॒मिति॑ बर्.हि - सद᳚म् । \newline
10. सु॒व॒र्विद॑म् प्रतीची॒नम् प्र॑तीची॒नꣳ सु॑व॒र्विद(ग्म्॑) सुव॒र्विद॑म् प्रतीची॒नं ॅवृ॒जनं॑ ॅवृ॒जन॑म् प्रतीची॒नꣳ सु॑व॒र्विद(ग्म्॑) सुव॒र्विद॑म् प्रतीची॒नं ॅवृ॒जन᳚म् । \newline
11. सु॒व॒र्विद॒मिति॑ सुवः - विद᳚म् । \newline
12. प्र॒ती॒ची॒नं ॅवृ॒जनं॑ ॅवृ॒जन॑म् प्रतीची॒नम् प्र॑तीची॒नं ॅवृ॒जन॑म् दोहसे दोहसे वृ॒जन॑म् प्रतीची॒नम् प्र॑तीची॒नं ॅवृ॒जन॑म् दोहसे । \newline
13. वृ॒जन॑म् दोहसे दोहसे वृ॒जनं॑ ॅवृ॒जन॑म् दोहसे गि॒रा गि॒रा दो॑हसे वृ॒जनं॑ ॅवृ॒जन॑म् दोहसे गि॒रा । \newline
14. दो॒ह॒से॒ गि॒रा गि॒रा दो॑हसे दोहसे गि॒रा ऽऽशु मा॒शुम् गि॒रा दो॑हसे दोहसे गि॒रा ऽऽशुम् । \newline
15. गि॒रा ऽऽशु मा॒शुम् गि॒रा गि॒रा ऽऽशुम् जय॑न्त॒म् जय॑न्त मा॒शुम् गि॒रा गि॒रा ऽऽशुम् जय॑न्तम् । \newline
16. आ॒शुम् जय॑न्त॒म् जय॑न्त मा॒शु मा॒शुम् जय॑न्त॒ मन्वनु॒ जय॑न्त मा॒शु मा॒शुम् जय॑न्त॒ मनु॑ । \newline
17. जय॑न्त॒ मन्वनु॒ जय॑न्त॒म् जय॑न्त॒ मनु॒ यासु॒ यास्वनु॒ जय॑न्त॒म् जय॑न्त॒ मनु॒ यासु॑ । \newline
18. अनु॒ यासु॒ यास्वन्वनु॒ यासु॒ वर्द्ध॑से॒ वर्द्ध॑से॒ यास्वन्वनु॒ यासु॒ वर्द्ध॑से । \newline
19. यासु॒ वर्द्ध॑से॒ वर्द्ध॑से॒ यासु॒ यासु॒ वर्द्ध॑से । \newline
20. वर्द्ध॑स॒ इति॒ वर्द्ध॑से । \newline
21. उ॒प॒या॒मगृ॑हीतो ऽस्यस्युपया॒मगृ॑हीत उपया॒मगृ॑हीतो ऽसि॒ मर्का॑य॒ मर्का॑यास्युपया॒मगृ॑हीत उपया॒मगृ॑हीतो ऽसि॒ मर्का॑य । \newline
22. उ॒प॒या॒मगृ॑हीत॒ इत्यु॑पया॒म - गृ॒ही॒तः॒ । \newline
23. अ॒सि॒ मर्का॑य॒ मर्का॑यास्यसि॒ मर्का॑य त्वा त्वा॒ मर्का॑यास्यसि॒ मर्का॑य त्वा । \newline
24. मर्का॑य त्वा त्वा॒ मर्का॑य॒ मर्का॑य त्वै॒ष ए॒ष त्वा॒ मर्का॑य॒ मर्का॑य त्वै॒षः । \newline
25. त्वै॒ष ए॒ष त्वा᳚ त्वै॒ष ते॑ त ए॒ष त्वा᳚ त्वै॒ष ते᳚ । \newline
26. ए॒ष ते॑ त ए॒ष ए॒ष ते॒ योनि॒र् योनि॑ स्त ए॒ष ए॒ष ते॒ योनिः॑ । \newline
27. ते॒ योनि॒र् योनि॑ स्ते ते॒ योनिः॑ प्र॒जाः प्र॒जा योनि॑ स्ते ते॒ योनिः॑ प्र॒जाः । \newline
28. योनिः॑ प्र॒जाः प्र॒जा योनि॒र् योनिः॑ प्र॒जाः पा॑हि पाहि प्र॒जा योनि॒र् योनिः॑ प्र॒जाः पा॑हि । \newline
29. प्र॒जाः पा॑हि पाहि प्र॒जाः प्र॒जाः पा॑हि । \newline
30. प्र॒जा इति॑ प्र - जाः । \newline
31. पा॒हीति॑ पाहि । \newline
\pagebreak
\markright{ TS 1.4.10.1  \hfill https://www.vedavms.in \hfill}
\addcontentsline{toc}{section}{ TS 1.4.10.1 }
\section*{ TS 1.4.10.1 }

\textbf{TS 1.4.10.1 } \newline
\textbf{Samhita Paata} \newline

ये दे॑वा दि॒व्येका॑दश॒ स्थ पृ॑थि॒व्यामद्ध्येका॑दश॒ स्थाऽफ्सु॒षदो॑ महि॒नैका॑दश॒ स्थ ते दे॑वा य॒ज्ञ्मि॒मं जु॑षद्ध्व-मुपया॒मगृ॑हीतो-ऽस्याग्रय॒णो॑ऽसि॒ स्वा᳚ग्रयणो॒ जिन्व॑ य॒ज्ञ्ं जिन्व॑ य॒ज्ञ्प॑तिम॒भि सव॑ना पाहि॒ विष्णु॒स्त्वां पा॑तु॒ विशं॒ त्वं पा॑हीन्द्रि॒येणै॒ष ते॒ योनि॒र् विश्वे᳚भ्यस्त्वा दे॒वेभ्यः॑ ॥ \newline

\textbf{Pada Paata} \newline

ये । दे॒वाः॒ । दि॒वि । एका॑दश । स्थ । पृ॒थि॒व्याम् । अधीति॑ । एका॑दश । स्थ । अ॒फ्सु॒षद॒ इत्य॑फ्सु - सदः॑ । म॒हि॒ना । एका॑दश । स्थ । ते । दे॒वाः॒ । य॒ज्ञ्म् । इ॒मम् । जु॒ष॒द्ध्व॒म् । उ॒प॒या॒मगृ॑हीत॒ इत्यु॑पया॒म - गृ॒ही॒तः॒ । अ॒सि॒ । आ॒ग्र॒य॒णः । अ॒सि॒ । स्वा᳚ग्रयण॒ इति॒ सु - आ॒ग्र॒य॒णः॒ । जिन्व॑ । य॒ज्ञ्म् । जिन्व॑ । य॒ज्ञ्प॑ति॒मिति॑ य॒ज्ञ् - प॒ति॒म् । अ॒भीति॑ । सव॑ना । पा॒हि॒ । विष्णुः॑ । त्वाम् । पा॒तु॒ । विश᳚म् । त्वम् । पा॒हि॒ । इ॒न्द्रि॒येण॑ । ए॒षः । ते॒ । योनिः॑ । विश्वे᳚भ्यः । त्वा॒ । दे॒वेभ्यः॑ ॥  \newline


\textbf{Krama Paata} \newline

ये दे॑वाः । दे॒वा॒ दि॒वि । दि॒व्येका॑दश । एका॑दश॒ स्थ । स्थ पृ॑थि॒व्याम् । पृ॒थि॒व्यामधि॑ । अद्ध्येका॑दश । एका॑दश॒ स्थ । स्थाफ्सु॒षदः॑ । अ॒फ्सु॒षदो॑ महि॒ना । अ॒फ्सु॒षद॒ इत्य॑फ्सु - सदः॑ । म॒हि॒नैका॑दश । एका॑दश॒ स्थ । स्थ ते । ते दे॑वाः । दे॒वा॒ य॒ज्ञ्म् । य॒ज्ञ्मि॒मम् । इ॒मम् जु॑षद्ध्वम् । जु॒ष॒द्ध्व॒मु॒प॒या॒मगृ॑हीतः । उ॒प॒या॒मगृ॑हीतोऽसि । उ॒प॒या॒मगृ॑हीत॒ इत्यु॑पया॒म - गृ॒ही॒तः॒ । अ॒स्या॒ग्र॒य॒णः । आ॒ग्र॒य॒णो॑ऽसि । अ॒सि॒ स्वा᳚ग्रयणः । स्वा᳚ग्रयणो॒ जिन्व॑ । स्वा᳚ग्रयण॒ इति॒ सु - आ॒ग्र॒य॒णः॒ । जिन्व॑ य॒ज्ञ्म् । य॒ज्ञ्म् जिन्व॑ । जिन्व॑ य॒ज्ञ्प॑तिम् । य॒ज्ञ्प॑तिम॒भि । य॒ज्ञ्प॑ति॒मिति॑ य॒ज्ञ् - प॒ति॒म् । अ॒भि सव॑ना । सव॑ना पाहि । पा॒हि॒ विष्णुः॑ । विष्णु॒स्त्वाम् । त्वाम् पा॑तु । पा॒तु॒ विश᳚म् । विश॒म् त्वम् । त्वम् पा॑हि । पा॒ही॒न्द्रि॒येण॑ । इ॒न्द्रि॒येणै॒षः । ए॒ष ते᳚ । ते॒ योनिः॑ । योनि॒र् विश्वे᳚भ्यः । विश्वे᳚भ्यस्त्वा । त्वा॒ दे॒वेभ्यः॑ । दे॒वेभ्य॒ इति॑ दे॒वेभ्यः॑ । \newline

\textbf{Jatai Paata} \newline

1. ये दे॑वा देवा॒ ये ये दे॑वाः । \newline
2. दे॒वा॒ दि॒वि दि॒वि दे॑वा देवा दि॒वि । \newline
3. दि॒व्येका॑ द॒शैका॑दश दि॒वि दि॒व्येका॑दश । \newline
4. एका॑दश॒ स्थ स्थैका॑ द॒शैका॑दश॒ स्थ । \newline
5. स्थ पृ॑थि॒व्याम् पृ॑थि॒व्याꣳ स्थ स्थ पृ॑थि॒व्याम् । \newline
6. पृ॒थि॒व्या मध्यधि॑ पृथि॒व्याम् पृ॑थि॒व्या मधि॑ । \newline
7. अध्येका॑ द॒शैका॑द॒शाध्य ध्येका॑दश । \newline
8. एका॑दश॒ स्थ स्थैका॑ द॒शैका॑दश॒ स्थ । \newline
9. स्थाफ्सु॒षदो᳚ ऽफ्सु॒षदः॒ स्थ स्थाफ्सु॒षदः॑ । \newline
10. अ॒फ्सु॒षदो॑ महि॒ना म॑हि॒ना ऽफ्सु॒षदो᳚ ऽफ्सु॒षदो॑ महि॒ना । \newline
11. अ॒फ्सु॒षद॒ इत्य॑फ्सु - सदः॑ । \newline
12. म॒हि॒ नैका॑ द॒शैका॑दश महि॒ना म॑हि॒नैका॑दश । \newline
13. एका॑दश॒ स्थ स्थैका॑ द॒शैका॑दश॒ स्थ । \newline
14. स्थ ते ते स्थ स्थ ते । \newline
15. ते दे॑वा देवा॒स्ते ते दे॑वाः । \newline
16. दे॒वा॒ य॒ज्ञ्ं ॅय॒ज्ञ्म् दे॑वा देवा य॒ज्ञ्म् । \newline
17. य॒ज्ञ् मि॒म मि॒मं ॅय॒ज्ञ्ं ॅय॒ज्ञ् मि॒मम् । \newline
18. इ॒मम् जु॑षद्ध्वम् जुषद्ध्व मि॒म मि॒मम् जु॑षद्ध्वम् । \newline
19. जु॒ष॒द्ध्व॒ मु॒प॒या॒मगृ॑हीत उपया॒मगृ॑हीतो जुषद्ध्वम् जुषद्ध्व मुपया॒मगृ॑हीतः । \newline
20. उ॒प॒या॒मगृ॑हीतो ऽस्यस्युपया॒मगृ॑हीत उपया॒मगृ॑हीतो ऽसि । \newline
21. उ॒प॒या॒मगृ॑हीत॒ इत्यु॑पया॒म - गृ॒ही॒तः॒ । \newline
22. अ॒स्या॒ग्र॒य॒ण आ᳚ग्रय॒णो᳚ ऽस्यस्याग्रय॒णः । \newline
23. आ॒ग्र॒य॒णो᳚ ऽस्यस्याग्रय॒ण आ᳚ग्रय॒णो॑ ऽसि । \newline
24. अ॒सि॒ स्वा᳚ग्रयणः॒ स्वा᳚ग्रयणो ऽस्यसि॒ स्वा᳚ग्रयणः । \newline
25. स्वा᳚ग्रयणो॒ जिन्व॒ जिन्व॒ स्वा᳚ग्रयणः॒ स्वा᳚ग्रयणो॒ जिन्व॑ । \newline
26. स्वा᳚ग्रयण॒ इति॒ सु - आ॒ग्र॒य॒णः॒ । \newline
27. जिन्व॑ य॒ज्ञ्ं ॅय॒ज्ञ्म् जिन्व॒ जिन्व॑ य॒ज्ञ्म् । \newline
28. य॒ज्ञ्म् जिन्व॒ जिन्व॑ य॒ज्ञ्ं ॅय॒ज्ञ्म् जिन्व॑ । \newline
29. जिन्व॑ य॒ज्ञ्प॑तिं ॅय॒ज्ञ्प॑ति॒म् जिन्व॒ जिन्व॑ य॒ज्ञ्प॑तिम् । \newline
30. य॒ज्ञ्प॑ति म॒भ्य॑भि य॒ज्ञ्प॑तिं ॅय॒ज्ञ्प॑ति म॒भि । \newline
31. य॒ज्ञ्प॑ति॒मिति॑ य॒ज्ञ् - प॒ति॒म् । \newline
32. अ॒भि सव॑ना॒ सव॑ना॒ ऽभ्य॑भि सव॑ना । \newline
33. सव॑ना पाहि पाहि॒ सव॑ना॒ सव॑ना पाहि । \newline
34. पा॒हि॒ विष्णु॒र् विष्णुः॑ पाहि पाहि॒ विष्णुः॑ । \newline
35. विष्णु॒ स्त्वाम् त्वां ॅविष्णु॒र् विष्णु॒ स्त्वाम् । \newline
36. त्वाम् पा॑तु पातु॒ त्वाम् त्वाम् पा॑तु । \newline
37. पा॒तु॒ विशं॒ ॅविश॑म् पातु पातु॒ विश᳚म् । \newline
38. विश॒म् त्वम् त्वं ॅविशं॒ ॅविश॒म् त्वम् । \newline
39. त्वम् पा॑हि पाहि॒ त्वम् त्वम् पा॑हि । \newline
40. पा॒ही॒न्द्रि॒येणे᳚ न्द्रि॒येण॑ पाहि पाहीन्द्रि॒येण॑ । \newline
41. इ॒न्द्रि॒येणै॒ष ए॒ष इ॑न्द्रि॒येणे᳚ न्द्रि॒येणै॒षः । \newline
42. ए॒ष ते॑ त ए॒ष ए॒ष ते᳚ । \newline
43. ते॒ योनि॒र् योनि॑ स्ते ते॒ योनिः॑ । \newline
44. योनि॒र् विश्वे᳚भ्यो॒ विश्वे᳚भ्यो॒ योनि॒र् योनि॒र् विश्वे᳚भ्यः । \newline
45. विश्वे᳚भ्य स्त्वा त्वा॒ विश्वे᳚भ्यो॒ विश्वे᳚भ्य स्त्वा । \newline
46. त्वा॒ दे॒वेभ्यो॑ दे॒वेभ्य॑ स्त्वा त्वा दे॒वेभ्यः॑ । \newline
47. दे॒वेभ्य॒ इति॑ दे॒वेभ्यः॑ । \newline

\textbf{Ghana Paata } \newline

1. ये दे॑वा देवा॒ ये ये दे॑वा दि॒वि दि॒वि दे॑वा॒ ये ये दे॑वा दि॒वि । \newline
2. दे॒वा॒ दि॒वि दि॒वि दे॑वा देवा दि॒व्येका॑ द॒शैका॑दश दि॒वि दे॑वा देवा दि॒व्येका॑दश । \newline
3. दि॒व्ये का॑द॒शैका॑दश दि॒वि दि॒व्येका॑दश॒ स्थ स्थैका॑दश दि॒वि दि॒व्येका॑दश॒ स्थ । \newline
4. एका॑दश॒ स्थ स्थैका॑ द॒शैका॑दश॒ स्थ पृ॑थि॒व्याम् पृ॑थि॒व्याꣳ स्थैका॑ द॒शैका॑दश॒ स्थ पृ॑थि॒व्याम् । \newline
5. स्थ पृ॑थि॒व्याम् पृ॑थि॒व्याꣳ स्थ स्थ पृ॑थि॒व्या मध्यधि॑ पृथि॒व्याꣳ स्थ स्थ पृ॑थि॒व्या मधि॑ । \newline
6. पृ॒थि॒व्या मध्यधि॑ पृथि॒व्याम् पृ॑थि॒व्या मध्येका॑ द॒शै का॑द॒शाधि॑ पृथि॒व्याम् पृ॑थि॒व्या मध्येका॑दश । \newline
7. अध्येका॑ द॒शै का॑द॒ शाध्य ध्येका॑दश॒ स्थ स्थै का॑द॒शा ध्यध्येका॑दश॒ स्थ । \newline
8. एका॑दश॒ स्थ स्थै का॑द॒शैका॑दश॒ स्थाफ्सु॒षदो᳚ ऽफ्सु॒षदः॒ स्थै का॑द॒शैका॑दश॒ स्थाफ्सु॒षदः॑ । \newline
9. स्थाफ्सु॒षदो᳚ ऽफ्सु॒षदः॒ स्थ स्थाफ्सु॒षदो॑ महि॒ना म॑हि॒ना ऽफ्सु॒षदः॒ स्थ स्थाफ्सु॒षदो॑ महि॒ना । \newline
10. अ॒फ्सु॒षदो॑ महि॒ना म॑हि॒ना ऽफ्सु॒षदो᳚ ऽफ्सु॒षदो॑ महि॒ नैका॑ द॒शैका॑दश महि॒ना ऽफ्सु॒षदो᳚ ऽफ्सु॒षदो॑ महि॒नैका॑दश । \newline
11. अ॒फ्सु॒षद॒ इत्य॑फ्सु - सदः॑ । \newline
12. म॒हि॒ नैका॑ द॒शैका॑दश महि॒ना म॑हि॒नैका॑दश॒ स्थ स्थैका॑दश महि॒ना म॑हि॒नैका॑दश॒ स्थ । \newline
13. एका॑दश॒ स्थ स्थैका॑ द॒शैका॑दश॒ स्थ ते ते स्थैका॑ द॒शैका॑दश॒ स्थ ते । \newline
14. स्थ ते ते स्थ स्थ ते दे॑वा देवा॒ स्ते स्थ स्थ ते दे॑वाः । \newline
15. ते दे॑वा देवा॒ स्ते ते दे॑वा य॒ज्ञ्ं ॅय॒ज्ञ्म् दे॑वा॒ स्ते ते दे॑वा य॒ज्ञ्म् । \newline
16. दे॒वा॒ य॒ज्ञ्ं ॅय॒ज्ञ्म् दे॑वा देवा य॒ज्ञ् मि॒म मि॒मं ॅय॒ज्ञ्म् दे॑वा देवा य॒ज्ञ् मि॒मम् । \newline
17. य॒ज्ञ् मि॒म मि॒मं ॅय॒ज्ञ्ं ॅय॒ज्ञ् मि॒मम् जु॑षद्ध्वम् जुषद्ध्व मि॒मं ॅय॒ज्ञ्ं ॅय॒ज्ञ् मि॒मम् जु॑षद्ध्वम् । \newline
18. इ॒मम् जु॑षद्ध्वम् जुषद्ध्व मि॒म मि॒मम् जु॑षद्ध्व मुपया॒मगृ॑हीत उपया॒मगृ॑हीतो जुषद्ध्व मि॒म मि॒मम् जु॑षद्ध्व मुपया॒मगृ॑हीतः । \newline
19. जु॒ष॒द्ध्व॒ मु॒प॒या॒मगृ॑हीत उपया॒मगृ॑हीतो जुषद्ध्वम् जुषद्ध्व मुपया॒मगृ॑हीतो ऽस्यस्युपया॒मगृ॑हीतो जुषद्ध्वम् जुषद्ध्व मुपया॒मगृ॑हीतो ऽसि । \newline
20. उ॒प॒या॒मगृ॑हीतो ऽस्यस्युपया॒मगृ॑हीत उपया॒मगृ॑हीतो ऽस्याग्रय॒ण आ᳚ग्रय॒णो᳚ ऽस्युपया॒मगृ॑हीत उपया॒मगृ॑हीतो ऽस्याग्रय॒णः । \newline
21. उ॒प॒या॒मगृ॑हीत॒ इत्यु॑पया॒म - गृ॒ही॒तः॒ । \newline
22. अ॒स्या॒ग्र॒य॒ण आ᳚ग्रय॒णो᳚ ऽस्यस्याग्रय॒णो᳚ ऽस्यस्याग्रय॒णो᳚ ऽस्यस्याग्रय॒णो॑ ऽसि । \newline
23. आ॒ग्र॒य॒णो᳚ ऽस्यस्याग्रय॒ण आ᳚ग्रय॒णो॑ ऽसि॒ स्वा᳚ग्रयणः॒ स्वा᳚ग्रयणो ऽस्याग्रय॒ण आ᳚ग्रय॒णो॑ ऽसि॒ स्वा᳚ग्रयणः । \newline
24. अ॒सि॒ स्वा᳚ग्रयणः॒ स्वा᳚ग्रयणो ऽस्यसि॒ स्वा᳚ग्रयणो॒ जिन्व॒ जिन्व॒ स्वा᳚ग्रयणो ऽस्यसि॒ स्वा᳚ग्रयणो॒ जिन्व॑ । \newline
25. स्वा᳚ग्रयणो॒ जिन्व॒ जिन्व॒ स्वा᳚ग्रयणः॒ स्वा᳚ग्रयणो॒ जिन्व॑ य॒ज्ञ्ं ॅय॒ज्ञ्म् जिन्व॒ स्वा᳚ग्रयणः॒ स्वा᳚ग्रयणो॒ जिन्व॑ य॒ज्ञ्म् । \newline
26. स्वा᳚ग्रयण॒ इति॒ सु - आ॒ग्र॒य॒णः॒ । \newline
27. जिन्व॑ य॒ज्ञ्ं ॅय॒ज्ञ्म् जिन्व॒ जिन्व॑ य॒ज्ञ्म् जिन्व॒ जिन्व॑ य॒ज्ञ्म् जिन्व॒ जिन्व॑ य॒ज्ञ्म् जिन्व॑ । \newline
28. य॒ज्ञ्म् जिन्व॒ जिन्व॑ य॒ज्ञ्ं ॅय॒ज्ञ्म् जिन्व॑ य॒ज्ञ्प॑तिं ॅय॒ज्ञ्प॑ति॒म् जिन्व॑ य॒ज्ञ्ं ॅय॒ज्ञ्म् जिन्व॑ य॒ज्ञ्प॑तिम् । \newline
29. जिन्व॑ य॒ज्ञ्प॑तिं ॅय॒ज्ञ्प॑ति॒म् जिन्व॒ जिन्व॑ य॒ज्ञ्प॑ति म॒भ्य॑भि य॒ज्ञ्प॑ति॒म् जिन्व॒ जिन्व॑ य॒ज्ञ्प॑ति म॒भि । \newline
30. य॒ज्ञ्प॑ति म॒भ्य॑भि य॒ज्ञ्प॑तिं ॅय॒ज्ञ्प॑ति म॒भि सव॑ना॒ सव॑ना॒ ऽभि य॒ज्ञ्प॑तिं ॅय॒ज्ञ्प॑ति म॒भि सव॑ना । \newline
31. य॒ज्ञ्प॑ति॒मिति॑ य॒ज्ञ् - प॒ति॒म् । \newline
32. अ॒भि सव॑ना॒ सव॑ना॒ ऽभ्य॑भि सव॑ना पाहि पाहि॒ सव॑ना॒ ऽभ्य॑भि सव॑ना पाहि । \newline
33. सव॑ना पाहि पाहि॒ सव॑ना॒ सव॑ना पाहि॒ विष्णु॒र् विष्णुः॑ पाहि॒ सव॑ना॒ सव॑ना पाहि॒ विष्णुः॑ । \newline
34. पा॒हि॒ विष्णु॒र् विष्णुः॑ पाहि पाहि॒ विष्णु॒ स्त्वाम् त्वां ॅविष्णुः॑ पाहि पाहि॒ विष्णु॒ स्त्वाम् । \newline
35. विष्णु॒ स्त्वाम् त्वां ॅविष्णु॒र् विष्णु॒ स्त्वाम् पा॑तु पातु॒ त्वां ॅविष्णु॒र् विष्णु॒ स्त्वाम् पा॑तु । \newline
36. त्वाम् पा॑तु पातु॒ त्वाम् त्वाम् पा॑तु॒ विशं॒ ॅविश॑म् पातु॒ त्वाम् त्वाम् पा॑तु॒ विश᳚म् । \newline
37. पा॒तु॒ विशं॒ ॅविश॑म् पातु पातु॒ विश॒म् त्वम् त्वं ॅविश॑म् पातु पातु॒ विश॒म् त्वम् । \newline
38. विश॒म् त्वम् त्वं ॅविशं॒ ॅविश॒म् त्वम् पा॑हि पाहि॒ त्वं ॅविशं॒ ॅविश॒म् त्वम् पा॑हि । \newline
39. त्वम् पा॑हि पाहि॒ त्वम् त्वम् पा॑हीन्द्रि॒येणे᳚ न्द्रि॒येण॑ पाहि॒ त्वम् त्वम् पा॑हीन्द्रि॒येण॑ । \newline
40. पा॒ही॒न्द्रि॒येणे᳚ न्द्रि॒येण॑ पाहि पाहीन्द्रि॒येणै॒ष ए॒ष इ॑न्द्रि॒येण॑ पाहि पाहीन्द्रि॒येणै॒षः । \newline
41. इ॒न्द्रि॒येणै॒ष ए॒ष इ॑न्द्रि॒येणे᳚ न्द्रि॒येणै॒ष ते॑ त ए॒ष इ॑न्द्रि॒येणे᳚ न्द्रि॒येणै॒ष ते᳚ । \newline
42. ए॒ष ते॑ त ए॒ष ए॒ष ते॒ योनि॒र् योनि॑ स्त ए॒ष ए॒ष ते॒ योनिः॑ । \newline
43. ते॒ योनि॒र् योनि॑ स्ते ते॒ योनि॒र् विश्वे᳚भ्यो॒ विश्वे᳚भ्यो॒ योनि॑ स्ते ते॒ योनि॒र् विश्वे᳚भ्यः । \newline
44. योनि॒र् विश्वे᳚भ्यो॒ विश्वे᳚भ्यो॒ योनि॒र् योनि॒र् विश्वे᳚भ्य स्त्वा त्वा॒ विश्वे᳚भ्यो॒ योनि॒र् योनि॒र् विश्वे᳚भ्य स्त्वा । \newline
45. विश्वे᳚भ्य स्त्वा त्वा॒ विश्वे᳚भ्यो॒ विश्वे᳚भ्य स्त्वा दे॒वेभ्यो॑ दे॒वेभ्य॑ स्त्वा॒ विश्वे᳚भ्यो॒ विश्वे᳚भ्य स्त्वा दे॒वेभ्यः॑ । \newline
46. त्वा॒ दे॒वेभ्यो॑ दे॒वेभ्य॑ स्त्वा त्वा दे॒वेभ्यः॑ । \newline
47. दे॒वेभ्य॒ इति॑ दे॒वेभ्यः॑ । \newline
\pagebreak
\markright{ TS 1.4.11.1  \hfill https://www.vedavms.in \hfill}
\addcontentsline{toc}{section}{ TS 1.4.11.1 }
\section*{ TS 1.4.11.1 }

\textbf{TS 1.4.11.1 } \newline
\textbf{Samhita Paata} \newline

त्रिꣳ॒॒शत्त्रय॑श्च ग॒णिनो॑ रु॒जन्तो॒ दिवꣳ॑ रु॒द्राः पृ॑थि॒वीं च॑ सचन्ते । ए॒का॒द॒शासो॑ अफ्सु॒षदः॑ सु॒तꣳ सोमं॑ जुषन्ताꣳ॒॒ सव॑नाय॒ विश्वे᳚ ॥ उ॒प॒या॒मगृ॑हीतो -ऽस्याग्रय॒णो॑ऽसि॒ स्वा᳚ग्रयणो॒ जिन्व॑ य॒ज्ञ्ं जिन्व॑ य॒ज्ञ्प॑तिम॒भि सव॑ना पाहि॒ विष्णु॒स्त्वां पा॑तु॒ विशं॒ त्वं पा॑हीन्द्रि॒येणै॒ष ते॒ योनि॒र् विश्वे᳚भ्यस्त्वा दे॒वेभ्यः॑ ॥ \newline

\textbf{Pada Paata} \newline

त्रिꣳ॒॒शत् । त्रयः॑ । च॒ । ग॒णिनः॑ । रु॒जन्तः॑ । दिव᳚म् । रु॒द्राः । पृ॒थि॒वीम् । च॒ । स॒च॒न्ते॒ ॥ ए॒का॒द॒शासः॑ । अ॒फ्सु॒षद॒ इत्य॑फ्सु-सदः॑ । सु॒तम् । सोम᳚म् । जु॒ष॒न्ता॒म् । सव॑नाय । विश्वे᳚ ॥ उ॒प॒या॒मगृ॑हीत॒ इत्यु॑पया॒म - गृ॒ही॒तः॒ । अ॒सि॒ । आ॒ग्र॒य॒णः । अ॒सि॒ । स्वा᳚ग्रयण॒ इति॒ सु - आ॒ग्र॒य॒णः॒ । जिन्व॑ । य॒ज्ञ्म् । जिन्व॑ । य॒ज्ञ्प॑ति॒मिति॑ य॒ज्ञ् - प॒ति॒म् । अ॒भीति॑ । सव॑ना । पा॒हि॒ । विष्णुः॑ । त्वाम् । पा॒तु॒ । विश᳚म् । त्वम् । पा॒हि॒ । इ॒न्द्रि॒येण॑ । ए॒षः । ते॒ । योनिः॑ । विश्वे᳚भ्यः । त्वा॒ । दे॒वेभ्यः॑ ॥  \newline


\textbf{Krama Paata} \newline

त्रिꣳ॒॒शत् त्रयः॑ । त्रय॑श्च । च॒ ग॒णिनः॑ । ग॒णिनो॑ रु॒जन्तः॑ । रु॒जन्तो॒ दिव᳚म् । दिवꣳ॑ रु॒द्राः । रु॒द्राः पृ॑थि॒वीम् । पृ॒थि॒वीम् च॑ । च॒ स॒च॒न्ते॒ । स॒च॒न्त॒ इति॑ सचन्ते ॥ ए॒का॒द॒शासो॑ अफ्सु॒षदः॑ । अ॒फ्सु॒षदः॑ सु॒तम् । अ॒फ्सु॒षद॒ इत्य॑फ्सु - सदः॑ । सु॒तꣳ सोम᳚म् । सोम॑म् जुषन्ताम् । जु॒ष॒न्ताꣳ॒॒ सव॑नाय । सव॑नाय॒ विश्वे᳚ । विश्व॒ इति॒ विश्वे᳚ ॥ उ॒प॒या॒मगृ॑हीतोऽसि । उ॒प॒या॒मगृ॑हीत॒ इत्यु॑पया॒म - गृ॒ही॒तः॒ । अ॒स्या॒ग्र॒य॒णः । आ॒ग्र॒य॒णो॑ऽसि । अ॒सि॒ स्वा᳚ग्रयणः । स्वा᳚ग्रयणो॒ जिन्व॑ । स्वा᳚ग्रयण॒ इति॒ सु - आ॒ग्र॒य॒णः॒ । जिन्व॑ य॒ज्ञ्म् । य॒ज्ञ्म् जिन्व॑ । जिन्व॑ य॒ज्ञ्प॑तिम् । य॒ज्ञ्प॑तिम॒भि । य॒ज्ञ्प॑ति॒मिति॑ य॒ज्ञ् - प॒ति॒म् । अ॒भि सव॑ना । सव॑ना पाहि । पा॒हि॒ विष्णुः॑ । विष्णु॒स्त्वाम् । त्वाम् पा॑तु । पा॒तु॒ विश᳚म् । विश॒म् त्वम् । त्वम् पा॑हि । पा॒ही॒न्द्रि॒येण॑ । इ॒न्द्रि॒येणै॒षः । ए॒ष ते᳚ । ते॒ योनिः॑ । योनि॒र्,विश्वे᳚भ्यः । विश्वे᳚भ्यस्त्वा । त्वा॒ दे॒वेभ्यः॑ । दे॒वेभ्य॒ इति॑ दे॒वेभ्यः॑ । \newline

\textbf{Jatai Paata} \newline

1. त्रि॒(ग्म्॒)शत् त्रय॒ स्त्रय॑ स्त्रि॒(ग्म्॒)शत् त्रि॒(ग्म्॒)शत् त्रयः॑ । \newline
2. त्रय॑श्च च॒ त्रय॒ स्त्रय॑श्च । \newline
3. च॒ ग॒णिनो॑ ग॒णिन॑श्च च ग॒णिनः॑ । \newline
4. ग॒णिनो॑ रु॒जन्तो॑ रु॒जन्तो॑ ग॒णिनो॑ ग॒णिनो॑ रु॒जन्तः॑ । \newline
5. रु॒जन्तो॒ दिव॒म् दिव(ग्म्॑) रु॒जन्तो॑ रु॒जन्तो॒ दिव᳚म् । \newline
6. दिव(ग्म्॑) रु॒द्रा रु॒द्रा दिव॒म् दिव(ग्म्॑) रु॒द्राः । \newline
7. रु॒द्राः पृ॑थि॒वीम् पृ॑थि॒वीꣳ रु॒द्रा रु॒द्राः पृ॑थि॒वीम् । \newline
8. पृ॒थि॒वीम् च॑ च पृथि॒वीम् पृ॑थि॒वीम् च॑ । \newline
9. च॒ स॒च॒न्ते॒ स॒च॒न्ते॒ च॒ च॒ स॒च॒न्ते॒ । \newline
10. स॒च॒न्त॒ इति॑ सचन्ते । \newline
11. ए॒का॒द॒शासो॑ अफ्सु॒षदो᳚ ऽफ्सु॒षद॑ एकाद॒शास॑ एकाद॒शासो॑ अफ्सु॒षदः॑ । \newline
12. अ॒फ्सु॒षदः॑ सु॒तꣳ सु॒त म॑फ्सु॒षदो᳚ ऽफ्सु॒षदः॑ सु॒तम् । \newline
13. अ॒फ्सु॒षद॒ इत्य॑फ्सु - सदः॑ । \newline
14. सु॒तꣳ सोम॒(ग्म्॒) सोम(ग्म्॑) सु॒तꣳ सु॒तꣳ सोम᳚म् । \newline
15. सोम॑म् जुषन्ताम् जुषन्ता॒(ग्म्॒) सोम॒(ग्म्॒) सोम॑म् जुषन्ताम् । \newline
16. जु॒ष॒न्ता॒(ग्म्॒) सव॑नाय॒ सव॑नाय जुषन्ताम् जुषन्ता॒(ग्म्॒) सव॑नाय । \newline
17. सव॑नाय॒ विश्वे॒ विश्वे॒ सव॑नाय॒ सव॑नाय॒ विश्वे᳚ । \newline
18. विश्व॒ इति॒ विश्वे᳚ । \newline
19. उ॒प॒या॒मगृ॑हीतो ऽस्यस्युपया॒मगृ॑हीत उपया॒मगृ॑हीतो ऽसि । \newline
20. उ॒प॒या॒मगृ॑हीत॒ इत्यु॑पया॒म - गृ॒ही॒तः॒ । \newline
21. अ॒स्या॒ग्र॒य॒ण आ᳚ग्रय॒णो᳚ ऽस्यस्याग्रय॒णः । \newline
22. आ॒ग्र॒य॒णो᳚ ऽस्यस्याग्रय॒ण आ᳚ग्रय॒णो॑ ऽसि । \newline
23. अ॒सि॒ स्वा᳚ग्रयणः॒ स्वा᳚ग्रयणो ऽस्यसि॒ स्वा᳚ग्रयणः । \newline
24. स्वा᳚ग्रयणो॒ जिन्व॒ जिन्व॒ स्वा᳚ग्रयणः॒ स्वा᳚ग्रयणो॒ जिन्व॑ । \newline
25. स्वा᳚ग्रयण॒ इति॒ सु - आ॒ग्र॒य॒णः॒ । \newline
26. जिन्व॑ य॒ज्ञ्ं ॅय॒ज्ञ्म् जिन्व॒ जिन्व॑ य॒ज्ञ्म् । \newline
27. य॒ज्ञ्म् जिन्व॒ जिन्व॑ य॒ज्ञ्ं ॅय॒ज्ञ्म् जिन्व॑ । \newline
28. जिन्व॑ य॒ज्ञ्प॑तिं ॅय॒ज्ञ्प॑ति॒म् जिन्व॒ जिन्व॑ य॒ज्ञ्प॑तिम् । \newline
29. य॒ज्ञ्प॑ति म॒भ्य॑भि य॒ज्ञ्प॑तिं ॅय॒ज्ञ्प॑ति म॒भि । \newline
30. य॒ज्ञ्प॑ति॒मिति॑ य॒ज्ञ् - प॒ति॒म् । \newline
31. अ॒भि सव॑ना॒ सव॑ना॒ ऽभ्य॑भि सव॑ना । \newline
32. सव॑ना पाहि पाहि॒ सव॑ना॒ सव॑ना पाहि । \newline
33. पा॒हि॒ विष्णु॒र् विष्णुः॑ पाहि पाहि॒ विष्णुः॑ । \newline
34. विष्णु॒ स्त्वाम् त्वां ॅविष्णु॒र् विष्णु॒ स्त्वाम् । \newline
35. त्वाम् पा॑तु पातु॒ त्वाम् त्वाम् पा॑तु । \newline
36. पा॒तु॒ विशं॒ ॅविश॑म् पातु पातु॒ विश᳚म् । \newline
37. विश॒म् त्वम् त्वं ॅविशं॒ ॅविश॒म् त्वम् । \newline
38. त्वम् पा॑हि पाहि॒ त्वम् त्वम् पा॑हि । \newline
39. पा॒ही॒न्द्रि॒येणे᳚ न्द्रि॒येण॑ पाहि पाहीन्द्रि॒येण॑ । \newline
40. इ॒न्द्रि॒येणै॒ष ए॒ष इ॑न्द्रि॒येणे᳚ न्द्रि॒येणै॒षः । \newline
41. ए॒ष ते॑ त ए॒ष ए॒ष ते᳚ । \newline
42. ते॒ योनि॒र् योनि॑ स्ते ते॒ योनिः॑ । \newline
43. योनि॒र् विश्वे᳚भ्यो॒ विश्वे᳚भ्यो॒ योनि॒र् योनि॒र् विश्वे᳚भ्यः । \newline
44. विश्वे᳚भ्य स्त्वा त्वा॒ विश्वे᳚भ्यो॒ विश्वे᳚भ्य स्त्वा । \newline
45. त्वा॒ दे॒वेभ्यो॑ दे॒वेभ्य॑ स्त्वा त्वा दे॒वेभ्यः॑ । \newline
46. दे॒वेभ्य॒ इति॑ दे॒वेभ्यः॑ । \newline

\textbf{Ghana Paata } \newline

1. त्रिꣳ॒॒शत् त्रय॒ स्त्रय॑ स्त्रिꣳ॒॒शत् त्रिꣳ॒॒शत् त्रय॑श्च च॒ त्रय॑ स्त्रिꣳ॒॒शत् त्रिꣳ॒॒शत् त्रय॑श्च । \newline
2. त्रय॑श्च च॒ त्रय॒ स्त्रय॑श्च ग॒णिनो॑ ग॒णिन॑श्च॒ त्रय॒ स्त्रय॑श्च ग॒णिनः॑ । \newline
3. च॒ ग॒णिनो॑ ग॒णिन॑श्च च ग॒णिनो॑ रु॒जन्तो॑ रु॒जन्तो॑ ग॒णिन॑श्च च ग॒णिनो॑ रु॒जन्तः॑ । \newline
4. ग॒णिनो॑ रु॒जन्तो॑ रु॒जन्तो॑ ग॒णिनो॑ ग॒णिनो॑ रु॒जन्तो॒ दिव॒म् दिव(ग्म्॑) रु॒जन्तो॑ ग॒णिनो॑ ग॒णिनो॑ रु॒जन्तो॒ दिव᳚म् । \newline
5. रु॒जन्तो॒ दिव॒म् दिव(ग्म्॑) रु॒जन्तो॑ रु॒जन्तो॒ दिव(ग्म्॑) रु॒द्रा रु॒द्रा दिव(ग्म्॑) रु॒जन्तो॑ रु॒जन्तो॒ दिव(ग्म्॑) रु॒द्राः । \newline
6. दिव(ग्म्॑) रु॒द्रा रु॒द्रा दिव॒म् दिव(ग्म्॑) रु॒द्राः पृ॑थि॒वीम् पृ॑थि॒वीꣳ रु॒द्रा दिव॒म् दिव(ग्म्॑) रु॒द्राः पृ॑थि॒वीम् । \newline
7. रु॒द्राः पृ॑थि॒वीम् पृ॑थि॒वीꣳ रु॒द्रा रु॒द्राः पृ॑थि॒वीम् च॑ च पृथि॒वीꣳ रु॒द्रा रु॒द्राः पृ॑थि॒वीम् च॑ । \newline
8. पृ॒थि॒वीम् च॑ च पृथि॒वीम् पृ॑थि॒वीम् च॑ सचन्ते सचन्ते च पृथि॒वीम् पृ॑थि॒वीम् च॑ सचन्ते । \newline
9. च॒ स॒च॒न्ते॒ स॒च॒न्ते॒ च॒ च॒ स॒च॒न्ते॒ । \newline
10. स॒च॒न्त॒ इति॑ सचन्ते । \newline
11. ए॒का॒द॒शासो॑ अफ्सु॒षदो᳚ ऽफ्सु॒षद॑ एकाद॒शास॑ एकाद॒शासो॑ अफ्सु॒षदः॑ सु॒तꣳ सु॒त म॑फ्सु॒षद॑ एकाद॒शास॑ एकाद॒शासो॑ अफ्सु॒षदः॑ सु॒तम् । \newline
12. अ॒फ्सु॒षदः॑ सु॒तꣳ सु॒त म॑फ्सु॒षदो᳚ ऽफ्सु॒षदः॑ सु॒तꣳ सोम॒(ग्म्॒) सोम(ग्म्॑) सु॒त म॑फ्सु॒षदो᳚ ऽफ्सु॒षदः॑ सु॒तꣳ सोम᳚म् । \newline
13. अ॒फ्सु॒षद॒ इत्य॑फ्सु - सदः॑ । \newline
14. सु॒तꣳ सोम॒(ग्म्॒) सोम(ग्म्॑) सु॒तꣳ सु॒तꣳ सोम॑म् जुषन्ताम् जुषन्ता॒(ग्म्॒) सोम(ग्म्॑) सु॒तꣳ सु॒तꣳ सोम॑म् जुषन्ताम् । \newline
15. सोम॑म् जुषन्ताम् जुषन्ता॒(ग्म्॒) सोम॒(ग्म्॒) सोम॑म् जुषन्ता॒(ग्म्॒) सव॑नाय॒ सव॑नाय जुषन्ता॒(ग्म्॒) सोम॒(ग्म्॒) सोम॑म् जुषन्ता॒(ग्म्॒) सव॑नाय । \newline
16. जु॒ष॒न्ता॒(ग्म्॒) सव॑नाय॒ सव॑नाय जुषन्ताम् जुषन्ता॒(ग्म्॒) सव॑नाय॒ विश्वे॒ विश्वे॒ सव॑नाय जुषन्ताम् जुषन्ता॒(ग्म्॒) सव॑नाय॒ विश्वे᳚ । \newline
17. सव॑नाय॒ विश्वे॒ विश्वे॒ सव॑नाय॒ सव॑नाय॒ विश्वे᳚ । \newline
18. विश्व॒ इति॒ विश्वे᳚ । \newline
19. उ॒प॒या॒मगृ॑हीतो ऽस्य स्युपया॒मगृ॑हीत उपया॒मगृ॑हीतो ऽस्याग्रय॒ण आ᳚ग्रय॒णो᳚ ऽस्युपया॒मगृ॑हीत उपया॒मगृ॑हीतो ऽस्याग्रय॒णः । \newline
20. उ॒प॒या॒मगृ॑हीत॒ इत्यु॑पया॒म - गृ॒ही॒तः॒ । \newline
21. अ॒स्या॒ग्र॒य॒ण आ᳚ग्रय॒णो᳚ ऽस्यस्याग्रय॒णो᳚ ऽस्यस्याग्रय॒णो᳚ ऽस्यस्याग्रय॒णो॑ ऽसि । \newline
22. आ॒ग्र॒य॒णो᳚ ऽस्यस्याग्रय॒ण आ᳚ग्रय॒णो॑ ऽसि॒ स्वा᳚ग्रयणः॒ स्वा᳚ग्रयणो ऽस्याग्रय॒ण आ᳚ग्रय॒णो॑ ऽसि॒ स्वा᳚ग्रयणः । \newline
23. अ॒सि॒ स्वा᳚ग्रयणः॒ स्वा᳚ग्रयणो ऽस्यसि॒ स्वा᳚ग्रयणो॒ जिन्व॒ जिन्व॒ स्वा᳚ग्रयणो ऽस्यसि॒ स्वा᳚ग्रयणो॒ जिन्व॑ । \newline
24. स्वा᳚ग्रयणो॒ जिन्व॒ जिन्व॒ स्वा᳚ग्रयणः॒ स्वा᳚ग्रयणो॒ जिन्व॑ य॒ज्ञ्ं ॅय॒ज्ञ्म् जिन्व॒ स्वा᳚ग्रयणः॒ स्वा᳚ग्रयणो॒ जिन्व॑ य॒ज्ञ्म् । \newline
25. स्वा᳚ग्रयण॒ इति॒ सु - आ॒ग्र॒य॒णः॒ । \newline
26. जिन्व॑ य॒ज्ञ्ं ॅय॒ज्ञ्म् जिन्व॒ जिन्व॑ य॒ज्ञ्म् जिन्व॒ जिन्व॑ य॒ज्ञ्म् जिन्व॒ जिन्व॑ य॒ज्ञ्म् जिन्व॑ । \newline
27. य॒ज्ञ्म् जिन्व॒ जिन्व॑ य॒ज्ञ्ं ॅय॒ज्ञ्म् जिन्व॑ य॒ज्ञ्प॑तिं ॅय॒ज्ञ्प॑ति॒म् जिन्व॑ य॒ज्ञ्ं ॅय॒ज्ञ्म् जिन्व॑ य॒ज्ञ्प॑तिम् । \newline
28. जिन्व॑ य॒ज्ञ्प॑तिं ॅय॒ज्ञ्प॑ति॒म् जिन्व॒ जिन्व॑ य॒ज्ञ्प॑ति म॒भ्य॑भि य॒ज्ञ्प॑ति॒म् जिन्व॒ जिन्व॑ य॒ज्ञ्प॑ति म॒भि । \newline
29. य॒ज्ञ्प॑ति म॒भ्य॑भि य॒ज्ञ्प॑तिं ॅय॒ज्ञ्प॑ति म॒भि सव॑ना॒ सव॑ना॒ ऽभि य॒ज्ञ्प॑तिं ॅय॒ज्ञ्प॑ति म॒भि सव॑ना । \newline
30. य॒ज्ञ्प॑ति॒मिति॑ य॒ज्ञ् - प॒ति॒म् । \newline
31. अ॒भि सव॑ना॒ सव॑ना॒ ऽभ्य॑भि सव॑ना पाहि पाहि॒ सव॑ना॒ ऽभ्य॑भि सव॑ना पाहि । \newline
32. सव॑ना पाहि पाहि॒ सव॑ना॒ सव॑ना पाहि॒ विष्णु॒र् विष्णुः॑ पाहि॒ सव॑ना॒ सव॑ना पाहि॒ विष्णुः॑ । \newline
33. पा॒हि॒ विष्णु॒र् विष्णुः॑ पाहि पाहि॒ विष्णु॒ स्त्वाम् त्वां ॅविष्णुः॑ पाहि पाहि॒ विष्णु॒ स्त्वाम् । \newline
34. विष्णु॒ स्त्वाम् त्वां ॅविष्णु॒र् विष्णु॒ स्त्वाम् पा॑तु पातु॒ त्वां ॅविष्णु॒र् विष्णु॒ स्त्वाम् पा॑तु । \newline
35. त्वाम् पा॑तु पातु॒ त्वाम् त्वाम् पा॑तु॒ विशं॒ ॅविश॑म् पातु॒ त्वाम् त्वाम् पा॑तु॒ विश᳚म् । \newline
36. पा॒तु॒ विशं॒ ॅविश॑म् पातु पातु॒ विश॒म् त्वम् त्वं ॅविश॑म् पातु पातु॒ विश॒म् त्वम् । \newline
37. विश॒म् त्वम् त्वं ॅविशं॒ ॅविश॒म् त्वम् पा॑हि पाहि॒ त्वं ॅविशं॒ ॅविश॒म् त्वम् पा॑हि । \newline
38. त्वम् पा॑हि पाहि॒ त्वम् त्वम् पा॑हीन्द्रि॒येणे᳚ न्द्रि॒येण॑ पाहि॒ त्वम् त्वम् पा॑हीन्द्रि॒येण॑ । \newline
39. पा॒ही॒न्द्रि॒येणे᳚ न्द्रि॒येण॑ पाहि पाहीन्द्रि॒येणै॒ष ए॒ष इ॑न्द्रि॒येण॑ पाहि पाहीन्द्रि॒येणै॒षः । \newline
40. इ॒न्द्रि॒येणै॒ष ए॒ष इ॑न्द्रि॒येणे᳚ न्द्रि॒येणै॒ष ते॑ त ए॒ष इ॑न्द्रि॒येणे᳚ न्द्रि॒येणै॒ष ते᳚ । \newline
41. ए॒ष ते॑ त ए॒ष ए॒ष ते॒ योनि॒र् योनि॑ स्त ए॒ष ए॒ष ते॒ योनिः॑ । \newline
42. ते॒ योनि॒र् योनि॑ स्ते ते॒ योनि॒र् विश्वे᳚भ्यो॒ विश्वे᳚भ्यो॒ योनि॑ स्ते ते॒ योनि॒र् विश्वे᳚भ्यः । \newline
43. योनि॒र् विश्वे᳚भ्यो॒ विश्वे᳚भ्यो॒ योनि॒र् योनि॒र् विश्वे᳚भ्य स्त्वा त्वा॒ विश्वे᳚भ्यो॒ योनि॒र् योनि॒र् विश्वे᳚भ्य स्त्वा । \newline
44. विश्वे᳚भ्य स्त्वा त्वा॒ विश्वे᳚भ्यो॒ विश्वे᳚भ्य स्त्वा दे॒वेभ्यो॑ दे॒वेभ्य॑ स्त्वा॒ विश्वे᳚भ्यो॒ विश्वे᳚भ्य स्त्वा दे॒वेभ्यः॑ । \newline
45. त्वा॒ दे॒वेभ्यो॑ दे॒वेभ्य॑ स्त्वा त्वा दे॒वेभ्यः॑ । \newline
46. दे॒वेभ्य॒ इति॑ दे॒वेभ्यः॑ । \newline
\pagebreak
\markright{ TS 1.4.12.1  \hfill https://www.vedavms.in \hfill}
\addcontentsline{toc}{section}{ TS 1.4.12.1 }
\section*{ TS 1.4.12.1 }

\textbf{TS 1.4.12.1 } \newline
\textbf{Samhita Paata} \newline

उ॒प॒या॒मगृ॑हीतो॒-ऽसीन्द्रा॑य त्वा बृ॒हद्व॑ते॒ वय॑स्वत उक्था॒युवे॒ यत्त॑ इन्द्र बृ॒हद्वय॒स्तस्मै᳚ त्वा॒ विष्ण॑वे त्वै॒ष ते॒ योनि॒रिन्द्रा॑य त्वोक्था॒युवे᳚ ॥ \newline

\textbf{Pada Paata} \newline

उ॒प॒या॒मगृ॑हीत॒ इत्यु॑पया॒म - गृ॒ही॒तः॒ । अ॒सि॒ । इन्द्रा॑य । त्वा॒ । बृ॒हद्व॑त॒ इति॑ बृ॒हत् - व॒ते॒ । वय॑स्वते । उ॒क्था॒युव॒ इत्यु॑क्थ - युवे᳚ । यत् । ते॒ । इ॒न्द्र॒ । बृ॒हत् । वयः॑ । तस्मै᳚ । त्वा॒ । विष्ण॑वे । त्वा॒ । ए॒षः । ते॒ । योनिः॑ । इन्द्रा॑य । त्वा॒ । उ॒क्था॒युव॒ इत्यु॑क्थ - युवे᳚ ॥  \newline


\textbf{Krama Paata} \newline

उ॒प॒या॒मगृ॑हीतोऽसि । उ॒प॒या॒मगृ॑हीत॒ इत्यु॑पया॒म - गृ॒ही॒तः॒ । अ॒सीन्द्रा॑य । इन्द्रा॑य त्वा । त्वा॒ बृ॒हद्व॑ते । बृ॒हद्व॑ते॒ वय॑स्वते । बृ॒हद्व॑त॒ इति॑ बृ॒हत् - व॒ते॒ । वय॑स्वत उक्था॒युवे᳚ । उ॒क्था॒युवे॒ यत् । उ॒क्था॒युव॒ इत्यु॑क्थ - युवे᳚ । यत्ते᳚ । त॒ इ॒न्द्र॒ । इ॒न्द्र॒ बृ॒हत् । बृ॒हद् वयः॑ । वय॒स्तस्मै᳚ । तस्मै᳚ त्वा । त्वा॒ विष्ण॑वे । विष्ण॑वे त्वा । त्वै॒षः । ए॒ष ते᳚ । ते॒ योनिः॑ । योनि॒रिन्द्रा॑य । इन्द्रा॑य त्वा । त्वो॒क्था॒युवे᳚ । उ॒क्था॒युव॒ इत्यु॑क्थ - युवे᳚ । \newline

\textbf{Jatai Paata} \newline

1. उ॒प॒या॒मगृ॑हीतो ऽस्यस्युपया॒मगृ॑हीत उपया॒मगृ॑हीतो ऽसि । \newline
2. उ॒प॒या॒मगृ॑हीत॒ इत्यु॑पया॒म - गृ॒ही॒तः॒ । \newline
3. अ॒सीन्द्रा॒ये न्द्रा॑यास्य॒सीन्द्रा॑य । \newline
4. इन्द्रा॑य त्वा॒ त्वेन्द्रा॒ये न्द्रा॑य त्वा । \newline
5. त्वा॒ बृ॒हद्व॑ते बृ॒हद्व॑ते त्वा त्वा बृ॒हद्व॑ते । \newline
6. बृ॒हद्व॑ते॒ वय॑स्वते॒ वय॑स्वते बृ॒हद्व॑ते बृ॒हद्व॑ते॒ वय॑स्वते । \newline
7. बृ॒हद्व॑त॒ इति॑ बृ॒हत् - व॒ते॒ । \newline
8. वय॑स्वत उक्था॒युव॑ उक्था॒युवे॒ वय॑स्वते॒ वय॑स्वत उक्था॒युवे᳚ । \newline
9. उ॒क्था॒युवे॒ यद् यदु॑क्था॒युव॑ उक्था॒युवे॒ यत् । \newline
10. उ॒क्था॒युव॒ इत्यु॑क्थ - युवे᳚ । \newline
11. यत् ते॑ ते॒ यद् यत् ते᳚ । \newline
12. त॒ इ॒न्द्रे॒ न्द्र॒ ते॒ त॒ इ॒न्द्र॒ । \newline
13. इ॒न्द्र॒ बृ॒हद् बृ॒हदि॑न्द्रे न्द्र बृ॒हत् । \newline
14. बृ॒हद् वयो॒ वयो॑ बृ॒हद् बृ॒हद् वयः॑ । \newline
15. वय॒ स्तस्मै॒ तस्मै॒ वयो॒ वय॒ स्तस्मै᳚ । \newline
16. तस्मै᳚ त्वा त्वा॒ तस्मै॒ तस्मै᳚ त्वा । \newline
17. त्वा॒ विष्ण॑वे॒ विष्ण॑वे त्वा त्वा॒ विष्ण॑वे । \newline
18. विष्ण॑वे त्वा त्वा॒ विष्ण॑वे॒ विष्ण॑वे त्वा । \newline
19. त्वै॒ष ए॒ष त्वा᳚ त्वै॒षः । \newline
20. ए॒ष ते॑ त ए॒ष ए॒ष ते᳚ । \newline
21. ते॒ योनि॒र् योनि॑ स्ते ते॒ योनिः॑ । \newline
22. योनि॒ रिन्द्रा॒ये न्द्रा॑य॒ योनि॒र् योनि॒ रिन्द्रा॑य । \newline
23. इन्द्रा॑य त्वा॒ त्वेन्द्रा॒ये न्द्रा॑य त्वा । \newline
24. त्वो॒क्था॒युव॑ उक्था॒युवे᳚ त्वा त्वोक्था॒युवे᳚ । \newline
25. उ॒क्था॒युव॒ इत्यु॑क्थ - युवे᳚ । \newline

\textbf{Ghana Paata } \newline

1. उ॒प॒या॒मगृ॑हीतो ऽस्यस्युपया॒मगृ॑हीत उपया॒मगृ॑हीतो॒ ऽसीन्द्रा॒ये न्द्रा॑यास्युपया॒मगृ॑हीत उपया॒मगृ॑हीतो॒ ऽसीन्द्रा॑य । \newline
2. उ॒प॒या॒मगृ॑हीत॒ इत्यु॑पया॒म - गृ॒ही॒तः॒ । \newline
3. अ॒सीन्द्रा॒ये न्द्रा॑यास्य॒सीन्द्रा॑य त्वा॒ त्वेन्द्रा॑यास्य॒सीन्द्रा॑य त्वा । \newline
4. इन्द्रा॑य त्वा॒ त्वेन्द्रा॒ये न्द्रा॑य त्वा बृ॒हद्व॑ते बृ॒हद्व॑ते॒ त्वेन्द्रा॒ये न्द्रा॑य त्वा बृ॒हद्व॑ते । \newline
5. त्वा॒ बृ॒हद्व॑ते बृ॒हद्व॑ते त्वा त्वा बृ॒हद्व॑ते॒ वय॑स्वते॒ वय॑स्वते बृ॒हद्व॑ते त्वा त्वा बृ॒हद्व॑ते॒ वय॑स्वते । \newline
6. बृ॒हद्व॑ते॒ वय॑स्वते॒ वय॑स्वते बृ॒हद्व॑ते बृ॒हद्व॑ते॒ वय॑स्वत उक्था॒युव॑ उक्था॒युवे॒ वय॑स्वते बृ॒हद्व॑ते बृ॒हद्व॑ते॒ वय॑स्वत उक्था॒युवे᳚ । \newline
7. बृ॒हद्व॑त॒ इति॑ बृ॒हत् - व॒ते॒ । \newline
8. वय॑स्वत उक्था॒युव॑ उक्था॒युवे॒ वय॑स्वते॒ वय॑स्वत उक्था॒युवे॒ यद् यदु॑क्था॒युवे॒ वय॑स्वते॒ वय॑स्वत उक्था॒युवे॒ यत् । \newline
9. उ॒क्था॒युवे॒ यद् यदु॑क्था॒युव॑ उक्था॒युवे॒ यत् ते॑ ते॒ यदु॑क्था॒युव॑ उक्था॒युवे॒ यत् ते᳚ । \newline
10. उ॒क्था॒युव॒ इत्यु॑क्थ - युवे᳚ । \newline
11. यत् ते॑ ते॒ यद् यत् त॑ इन्द्रे न्द्र ते॒ यद् यत् त॑ इन्द्र । \newline
12. त॒ इ॒न्द्रे॒ न्द्र॒ ते॒ त॒ इ॒न्द्र॒ बृ॒हद् बृ॒हदि॑न्द्र ते त इन्द्र बृ॒हत् । \newline
13. इ॒न्द्र॒ बृ॒हद् बृ॒हदि॑न्द्रे न्द्र बृ॒हद् वयो॒ वयो॑ बृ॒हदि॑न्द्रे न्द्र बृ॒हद् वयः॑ । \newline
14. बृ॒हद् वयो॒ वयो॑ बृ॒हद् बृ॒हद् वय॒ स्तस्मै॒ तस्मै॒ वयो॑ बृ॒हद् बृ॒हद् वय॒ स्तस्मै᳚ । \newline
15. वय॒ स्तस्मै॒ तस्मै॒ वयो॒ वय॒ स्तस्मै᳚ त्वा त्वा॒ तस्मै॒ वयो॒ वय॒ स्तस्मै᳚ त्वा । \newline
16. तस्मै᳚ त्वा त्वा॒ तस्मै॒ तस्मै᳚ त्वा॒ विष्ण॑वे॒ विष्ण॑वे त्वा॒ तस्मै॒ तस्मै᳚ त्वा॒ विष्ण॑वे । \newline
17. त्वा॒ विष्ण॑वे॒ विष्ण॑वे त्वा त्वा॒ विष्ण॑वे त्वा त्वा॒ विष्ण॑वे त्वा त्वा॒ विष्ण॑वे त्वा । \newline
18. विष्ण॑वे त्वा त्वा॒ विष्ण॑वे॒ विष्ण॑वे त्वै॒ष ए॒ष त्वा॒ विष्ण॑वे॒ विष्ण॑वे त्वै॒षः । \newline
19. त्वै॒ष ए॒ष त्वा᳚ त्वै॒ष ते॑ त ए॒ष त्वा᳚ त्वै॒ष ते᳚ । \newline
20. ए॒ष ते॑ त ए॒ष ए॒ष ते॒ योनि॒र् योनि॑ स्त ए॒ष ए॒ष ते॒ योनिः॑ । \newline
21. ते॒ योनि॒र् योनि॑ स्ते ते॒ योनि॒ रिन्द्रा॒ये न्द्रा॑य॒ योनि॑ स्ते ते॒ योनि॒ रिन्द्रा॑य । \newline
22. योनि॒ रिन्द्रा॒ये न्द्रा॑य॒ योनि॒र् योनि॒ रिन्द्रा॑य त्वा॒ त्वेन्द्रा॑य॒ योनि॒र् योनि॒ रिन्द्रा॑य त्वा । \newline
23. इन्द्रा॑य त्वा॒ त्वेन्द्रा॒ये न्द्रा॑य त्वोक्था॒युव॑ उक्था॒युवे॒ त्वेन्द्रा॒ये न्द्रा॑य त्वोक्था॒युवे᳚ । \newline
24. त्वो॒क्था॒युव॑ उक्था॒युवे᳚ त्वा त्वोक्था॒युवे᳚ । \newline
25. उ॒क्था॒युव॒ इत्यु॑क्थ - युवे᳚ । \newline
\pagebreak
\markright{ TS 1.4.13.1  \hfill https://www.vedavms.in \hfill}
\addcontentsline{toc}{section}{ TS 1.4.13.1 }
\section*{ TS 1.4.13.1 }

\textbf{TS 1.4.13.1 } \newline
\textbf{Samhita Paata} \newline

मू॒र्द्धानं॑ दि॒वो अ॑र॒तिं पृ॑थि॒व्या वै᳚श्वान॒रमृ॒ताय॑ जा॒तम॒ग्निं । क॒विꣳ स॒म्राज॒-मति॑थिं॒ जना॑नामा॒सन्ना पात्रं॑ जनयन्त दे॒वाः ॥ उ॒प॒या॒मगृ॑हीतो-ऽस्य॒ग्नये᳚ त्वा वैश्वान॒राय॑ ध्रु॒वो॑ऽसि ध्रु॒वक्षि॑तिर् ध्रु॒वाणां᳚ ध्रु॒वत॒मोऽच्यु॑ताना-मच्युत॒क्षित्त॑म ए॒ष ते॒ योनि॑र॒ग्नये᳚ त्वा वैश्वान॒राय॑ ॥ \newline

\textbf{Pada Paata} \newline

मू॒र्धान᳚म् । दि॒वः । अ॒र॒तिम् । पृ॒थि॒व्याः । वै॒श्वा॒न॒रम् । ऋ॒ताय॑ । जा॒तम् । अ॒ग्निम् ॥ क॒विम् । स॒म्राज॒मिति॑ सं - राज᳚म् । अति॑थिम् । जना॑नाम् । आ॒सन्न् । एति॑ । पात्र᳚म् । ज॒न॒य॒न्त॒ । दे॒वाः ॥ उ॒प॒या॒मगृ॑हीत॒ इत्यु॑पया॒म - गृ॒ही॒तः॒ । अ॒सि॒ । अ॒ग्नये᳚ । त्वा॒ । वै॒श्वा॒न॒राय॑ । ध्रु॒वः । अ॒सि॒ । ध्रु॒वक्षि॑ति॒रिति॑ ध्रु॒व-क्षि॒तिः॒ । ध्रु॒वाणा᳚म् । ध्रु॒वत॑म॒ इति॑ ध्रु॒व - त॒मः॒ । अच्यु॑तानाम् । अ॒च्यु॒त॒क्षित्त॑म॒ इत्य॑च्युत॒क्षित् - त॒मः॒ । ए॒षः । ते॒ । योनिः॑ । अ॒ग्नये᳚ । त्वा॒ । वै॒श्वा॒न॒राय॑ ॥  \newline


\textbf{Krama Paata} \newline

मू॒र्द्धान॑म् दि॒वः । दि॒वो अ॑र॒तिम् । अ॒र॒तिम् पृ॑थि॒व्याः । पृ॒थि॒व्या वै᳚श्वान॒रम् । वै॒श्वा॒न॒रमृ॒ताय॑ । ऋ॒ताय॑ जा॒तम् । जा॒तम॒ग्निम् । अ॒ग्निमित्य॒ग्निम् ॥ क॒विꣳ स॒म्राज᳚म् । स॒म्राज॒मति॑थिम् । स॒म्राज॒मिति॑ सम् - राज᳚म् । अति॑थि॒म् जना॑नाम् । जना॑नामा॒सन्न् । आ॒सन्ना । आ पात्र᳚म् । पात्र॑म् जनयन्त । ज॒न॒य॒न्त॒ दे॒वाः । दे॒वा इति॑ दे॒वाः ॥ उ॒प॒या॒मगृ॑हीतोऽसि । उ॒प॒या॒मगृ॑हीत॒ इत्यु॑पया॒म - गृ॒ही॒तः॒ । अ॒स्य॒ग्नये᳚ । अ॒ग्नये᳚ त्वा । त्वा॒ वै॒श्वा॒न॒राय॑ । वै॒श्वा॒न॒राय॑ ध्रु॒वः । ध्रु॒वो॑ऽसि । अ॒सि॒ ध्रु॒वक्षि॑तिः । ध्रु॒वक्षि॑तिर् ध्रु॒वाणा᳚म् । ध्रु॒वक्षि॑ति॒रिति॑ ध्रु॒व - क्षि॒तिः॒ । ध्रु॒वाणा᳚म् ध्रु॒वत॑मः । ध्रु॒वत॒मोऽच्यु॑तानाम् । ध्रु॒वत॑म॒ इति॑ ध्रु॒व - त॒मः॒ । अच्यु॑तानामच्युत॒क्षित्त॑मः । अ॒च्यु॒त॒क्षित्त॑म ए॒षः । अ॒च्यु॒त॒क्षित्त॑म॒ इत्य॑च्युत॒क्षित् - त॒मः॒ । ए॒ष ते᳚ । ते॒ योनिः॑ । योनि॑र॒ग्नये᳚ । अ॒ग्नये᳚ त्वा । त्वा॒ वै॒श्वा॒न॒राय॑ । वै॒श्वा॒न॒रायेति॑ वैश्वान॒राय॑ । \newline

\textbf{Jatai Paata} \newline

1. मू॒र्द्धान॑म् दि॒वो दि॒वो मू॒र्द्धान॑म् मू॒र्द्धान॑म् दि॒वः । \newline
2. दि॒वो अ॑र॒ति म॑र॒तिम् दि॒वो दि॒वो अ॑र॒तिम् । \newline
3. अ॒र॒तिम् पृ॑थि॒व्याः पृ॑थि॒व्या अ॑र॒ति म॑र॒तिम् पृ॑थि॒व्याः । \newline
4. पृ॒थि॒व्या वै᳚श्वान॒रं ॅवै᳚श्वान॒रम् पृ॑थि॒व्याः पृ॑थि॒व्या वै᳚श्वान॒रम् । \newline
5. वै॒श्वा॒न॒र मृ॒ताय॒ र्ताय॑ वैश्वान॒रं ॅवै᳚श्वान॒र मृ॒ताय॑ । \newline
6. ऋ॒ताय॑ जा॒तम् जा॒त मृ॒ताय॒ र्ताय॑ जा॒तम् । \newline
7. जा॒त म॒ग्नि म॒ग्निम् जा॒तम् जा॒त म॒ग्निम् । \newline
8. अ॒ग्निमित्य॒ग्निम् । \newline
9. क॒विꣳ स॒म्राज(ग्म्॑) स॒म्राज॑म् क॒विम् क॒विꣳ स॒म्राज᳚म् । \newline
10. स॒म्राज॒ मति॑थि॒ मति॑थिꣳ स॒म्राज(ग्म्॑) स॒म्राज॒ मति॑थिम् । \newline
11. स॒म्राज॒मिति॑ सं - राज᳚म् । \newline
12. अति॑थि॒म् जना॑ना॒म् जना॑ना॒ मति॑थि॒ मति॑थि॒म् जना॑नाम् । \newline
13. जना॑ना मा॒सन् ना॒सन् जना॑ना॒म् जना॑ना मा॒सन्न् । \newline
14. आ॒सन् ना ऽऽसन् ना॒सन् ना । \newline
15. आ पात्र॒म् पात्र॒ मा पात्र᳚म् । \newline
16. पात्र॑म् जनयन्त जनयन्त॒ पात्र॒म् पात्र॑म् जनयन्त । \newline
17. ज॒न॒य॒न्त॒ दे॒वा दे॒वा ज॑नयन्त जनयन्त दे॒वाः । \newline
18. दे॒वा इति॑ दे॒वाः । \newline
19. उ॒प॒या॒मगृ॑हीतो ऽस्यस्युपया॒मगृ॑हीत उपया॒मगृ॑हीतो ऽसि । \newline
20. उ॒प॒या॒मगृ॑हीत॒ इत्यु॑पया॒म - गृ॒ही॒तः॒ । \newline
21. अ॒स्य॒ग्नये॒ ऽग्नये᳚ ऽस्यस्य॒ग्नये᳚ । \newline
22. अ॒ग्नये᳚ त्वा त्वा॒ ऽग्नये॒ ऽग्नये᳚ त्वा । \newline
23. त्वा॒ वै॒श्वा॒न॒राय॑ वैश्वान॒राय॑ त्वा त्वा वैश्वान॒राय॑ । \newline
24. वै॒श्वा॒न॒राय॑ ध्रु॒वो ध्रु॒वो वै᳚श्वान॒राय॑ वैश्वान॒राय॑ ध्रु॒वः । \newline
25. ध्रु॒वो᳚ ऽस्यसि ध्रु॒वो ध्रु॒वो॑ ऽसि । \newline
26. अ॒सि॒ ध्रु॒वक्षि॑तिर् ध्रु॒वक्षि॑ति रस्यसि ध्रु॒वक्षि॑तिः । \newline
27. ध्रु॒वक्षि॑तिर् ध्रु॒वाणा᳚म् ध्रु॒वाणा᳚म् ध्रु॒वक्षि॑तिर् ध्रु॒वक्षि॑तिर् ध्रु॒वाणा᳚म् । \newline
28. ध्रु॒वक्षि॑ति॒रिति॑ ध्रु॒व - क्षि॒तिः॒ । \newline
29. ध्रु॒वाणा᳚म् ध्रु॒वत॑मो ध्रु॒वत॑मो ध्रु॒वाणा᳚म् ध्रु॒वाणा᳚म् ध्रु॒वत॑मः । \newline
30. ध्रु॒वत॒मो ऽच्यु॑ताना॒ मच्यु॑तानाम् ध्रु॒वत॑मो ध्रु॒वत॒मो ऽच्यु॑तानाम् । \newline
31. ध्रु॒वत॑म॒ इति॑ ध्रु॒व - त॒मः॒ । \newline
32. अच्यु॑ताना मच्युत॒क्षित्त॑मो ऽच्युत॒क्षित्त॒मो ऽच्यु॑ताना॒ मच्यु॑ताना मच्युत॒क्षित्त॑मः । \newline
33. अ॒च्यु॒त॒क्षित्त॑म ए॒ष ए॒षो᳚च्युत॒क्षित्त॑मो ऽच्युत॒क्षित्त॑म ए॒षः । \newline
34. अ॒च्यु॒त॒क्षित्त॑म॒ इत्य॑च्युत॒क्षित् - त॒मः॒ । \newline
35. ए॒ष ते॑ त ए॒ष ए॒ष ते᳚ । \newline
36. ते॒ योनि॒र् योनि॑ स्ते ते॒ योनिः॑ । \newline
37. योनि॑ र॒ग्नये॒ ऽग्नये॒ योनि॒र् योनि॑ र॒ग्नये᳚ । \newline
38. अ॒ग्नये᳚ त्वा त्वा॒ ऽग्नये॒ ऽग्नये᳚ त्वा । \newline
39. त्वा॒ वै॒श्वा॒न॒राय॑ वैश्वान॒राय॑ त्वा त्वा वैश्वान॒राय॑ । \newline
40. वै॒श्वा॒न॒रायेति॑ वैश्वान॒राय॑ । \newline

\textbf{Ghana Paata } \newline

1. मू॒र्द्धान॑म् दि॒वो दि॒वो मू॒र्द्धान॑म् मू॒र्द्धान॑म् दि॒वो अ॑र॒ति म॑र॒तिम् दि॒वो मू॒र्द्धान॑म् मू॒र्द्धान॑म् दि॒वो अ॑र॒तिम् । \newline
2. दि॒वो अ॑र॒ति म॑र॒तिम् दि॒वो दि॒वो अ॑र॒तिम् पृ॑थि॒व्याः पृ॑थि॒व्या अ॑र॒तिम् दि॒वो दि॒वो अ॑र॒तिम् पृ॑थि॒व्याः । \newline
3. अ॒र॒तिम् पृ॑थि॒व्याः पृ॑थि॒व्या अ॑र॒ति म॑र॒तिम् पृ॑थि॒व्या वै᳚श्वान॒रं ॅवै᳚श्वान॒रम् पृ॑थि॒व्या अ॑र॒ति म॑र॒तिम् पृ॑थि॒व्या वै᳚श्वान॒रम् । \newline
4. पृ॒थि॒व्या वै᳚श्वान॒रं ॅवै᳚श्वान॒रम् पृ॑थि॒व्याः पृ॑थि॒व्या वै᳚श्वान॒र मृ॒ताय॒ र्ताय॑ वैश्वान॒रम् पृ॑थि॒व्याः पृ॑थि॒व्या वै᳚श्वान॒र मृ॒ताय॑ । \newline
5. वै॒श्वा॒न॒र मृ॒ताय॒ र्ताय॑ वैश्वान॒रं ॅवै᳚श्वान॒र मृ॒ताय॑ जा॒तम् जा॒त मृ॒ताय॑ वैश्वान॒रं ॅवै᳚श्वान॒र मृ॒ताय॑ जा॒तम् । \newline
6. ऋ॒ताय॑ जा॒तम् जा॒त मृ॒ताय॒ र्ताय॑ जा॒त म॒ग्नि म॒ग्निम् जा॒त मृ॒ताय॒ र्ताय॑ जा॒त म॒ग्निम् । \newline
7. जा॒त म॒ग्नि म॒ग्निम् जा॒तम् जा॒त म॒ग्निम् । \newline
8. अ॒ग्निमित्य॒ग्निम् । \newline
9. क॒विꣳ स॒म्राज(ग्म्॑) स॒म्राज॑म् क॒विम् क॒विꣳ स॒म्राज॒ मति॑थि॒ मति॑थिꣳ स॒म्राज॑म् क॒विम् क॒विꣳ स॒म्राज॒ मति॑थिम् । \newline
10. स॒म्राज॒ मति॑थि॒ मति॑थिꣳ स॒म्राज(ग्म्॑) स॒म्राज॒ मति॑थि॒म् जना॑ना॒म् जना॑ना॒ मति॑थिꣳ स॒म्राज(ग्म्॑) स॒म्राज॒ मति॑थि॒म् जना॑नाम् । \newline
11. स॒म्राज॒मिति॑ सं - राज᳚म् । \newline
12. अति॑थि॒म् जना॑ना॒म् जना॑ना॒ मति॑थि॒ मति॑थि॒म् जना॑ना मा॒सन् ना॒सन् जना॑ना॒ मति॑थि॒ मति॑थि॒म् जना॑ना मा॒सन्न् । \newline
13. जना॑ना मा॒सन् ना॒सन् जना॑ना॒म् जना॑ना मा॒सन् ना ऽऽसन् जना॑ना॒म् जना॑ना मा॒सन् ना । \newline
14. आ॒सन् ना ऽऽसन् ना॒सन् ना पात्र॒म् पात्र॒ मा ऽऽसन् ना॒सन् ना पात्र᳚म् । \newline
15. आ पात्र॒म् पात्र॒ मा पात्र॑म् जनयन्त जनयन्त॒ पात्र॒ मा पात्र॑म् जनयन्त । \newline
16. पात्र॑म् जनयन्त जनयन्त॒ पात्र॒म् पात्र॑म् जनयन्त दे॒वा दे॒वा ज॑नयन्त॒ पात्र॒म् पात्र॑म् जनयन्त दे॒वाः । \newline
17. ज॒न॒य॒न्त॒ दे॒वा दे॒वा ज॑नयन्त जनयन्त दे॒वाः । \newline
18. दे॒वा इति॑ दे॒वाः । \newline
19. उ॒प॒या॒मगृ॑हीतो ऽस्यस्युपया॒मगृ॑हीत उपया॒मगृ॑हीतो ऽस्य॒ग्नये॒ ऽग्नये᳚ ऽस्युपया॒मगृ॑हीत उपया॒मगृ॑हीतो ऽस्य॒ग्नये᳚ । \newline
20. उ॒प॒या॒मगृ॑हीत॒ इत्यु॑पया॒म - गृ॒ही॒तः॒ । \newline
21. अ॒स्य॒ग्नये॒ ऽग्नये᳚ ऽस्यस्य॒ग्नये᳚ त्वा त्वा॒ ऽग्नये᳚ ऽस्यस्य॒ग्नये᳚ त्वा । \newline
22. अ॒ग्नये᳚ त्वा त्वा॒ ऽग्नये॒ ऽग्नये᳚ त्वा वैश्वान॒राय॑ वैश्वान॒राय॑ त्वा॒ ऽग्नये॒ ऽग्नये᳚ त्वा वैश्वान॒राय॑ । \newline
23. त्वा॒ वै॒श्वा॒न॒राय॑ वैश्वान॒राय॑ त्वा त्वा वैश्वान॒राय॑ ध्रु॒वो ध्रु॒वो वै᳚श्वान॒राय॑ त्वा त्वा वैश्वान॒राय॑ ध्रु॒वः । \newline
24. वै॒श्वा॒न॒राय॑ ध्रु॒वो ध्रु॒वो वै᳚श्वान॒राय॑ वैश्वान॒राय॑ ध्रु॒वो᳚ ऽस्यसि ध्रु॒वो वै᳚श्वान॒राय॑ वैश्वान॒राय॑ ध्रु॒वो॑ ऽसि । \newline
25. ध्रु॒वो᳚ ऽस्यसि ध्रु॒वो ध्रु॒वो॑ ऽसि ध्रु॒वक्षि॑तिर् ध्रु॒वक्षि॑ति रसि ध्रु॒वो ध्रु॒वो॑ ऽसि ध्रु॒वक्षि॑तिः । \newline
26. अ॒सि॒ ध्रु॒वक्षि॑तिर् ध्रु॒वक्षि॑ति रस्यसि ध्रु॒वक्षि॑तिर् ध्रु॒वाणा᳚म् ध्रु॒वाणा᳚म् ध्रु॒वक्षि॑ति रस्यसि ध्रु॒वक्षि॑तिर् ध्रु॒वाणा᳚म् । \newline
27. ध्रु॒वक्षि॑तिर् ध्रु॒वाणा᳚म् ध्रु॒वाणा᳚म् ध्रु॒वक्षि॑तिर् ध्रु॒वक्षि॑तिर् ध्रु॒वाणा᳚म् ध्रु॒वत॑मो ध्रु॒वत॑मो ध्रु॒वाणा᳚म् ध्रु॒वक्षि॑तिर् ध्रु॒वक्षि॑तिर् ध्रु॒वाणा᳚म् ध्रु॒वत॑मः । \newline
28. ध्रु॒वक्षि॑ति॒रिति॑ ध्रु॒व - क्षि॒तिः॒ । \newline
29. ध्रु॒वाणा᳚म् ध्रु॒वत॑मो ध्रु॒वत॑मो ध्रु॒वाणा᳚म् ध्रु॒वाणा᳚म् ध्रु॒वत॒मो ऽच्यु॑ताना॒ मच्यु॑तानाम् ध्रु॒वत॑मो ध्रु॒वाणा᳚म् ध्रु॒वाणा᳚म् ध्रु॒वत॒मो ऽच्यु॑तानाम् । \newline
30. ध्रु॒वत॒मो ऽच्यु॑ताना॒ मच्यु॑तानाम् ध्रु॒वत॑मो ध्रु॒वत॒मो ऽच्यु॑ताना मच्युत॒क्षित्त॑मो ऽच्युत॒क्षित्त॒मो ऽच्यु॑तानाम् ध्रु॒वत॑मो ध्रु॒वत॒मो ऽच्यु॑ताना मच्युत॒क्षित्त॑मः । \newline
31. ध्रु॒वत॑म॒ इति॑ ध्रु॒व - त॒मः॒ । \newline
32. अच्यु॑ताना मच्युत॒क्षित्त॑मो ऽच्युत॒क्षित्त॒मो ऽच्यु॑ताना॒ मच्यु॑ताना मच्युत॒क्षित्त॑म ए॒ष ए॒षो᳚च्युत॒क्षित्त॒मो ऽच्यु॑ताना॒ मच्यु॑ताना मच्युत॒क्षित्त॑म ए॒षः । \newline
33. अ॒च्यु॒त॒क्षित्त॑म ए॒ष ए॒षो᳚च्युत॒क्षित्त॑मो ऽच्युत॒क्षित्त॑म ए॒ष ते॑ त ए॒षो᳚च्युत॒क्षित्त॑मो ऽच्युत॒क्षित्त॑म ए॒ष ते᳚ । \newline
34. अ॒च्यु॒त॒क्षित्त॑म॒ इत्य॑च्युत॒क्षित् - त॒मः॒ । \newline
35. ए॒ष ते॑ त ए॒ष ए॒ष ते॒ योनि॒र् योनि॑ स्त ए॒ष ए॒ष ते॒ योनिः॑ । \newline
36. ते॒ योनि॒र् योनि॑ स्ते ते॒ योनि॑ र॒ग्नये॒ ऽग्नये॒ योनि॑ स्ते ते॒ योनि॑ र॒ग्नये᳚ । \newline
37. योनि॑ र॒ग्नये॒ ऽग्नये॒ योनि॒र् योनि॑ र॒ग्नये᳚ त्वा त्वा॒ ऽग्नये॒ योनि॒र् योनि॑ र॒ग्नये᳚ त्वा । \newline
38. अ॒ग्नये᳚ त्वा त्वा॒ ऽग्नये॒ ऽग्नये᳚ त्वा वैश्वान॒राय॑ वैश्वान॒राय॑ त्वा॒ ऽग्नये॒ ऽग्नये᳚ त्वा वैश्वान॒राय॑ । \newline
39. त्वा॒ वै॒श्वा॒न॒राय॑ वैश्वान॒राय॑ त्वा त्वा वैश्वान॒राय॑ । \newline
40. वै॒श्वा॒न॒रायेति॑ वैश्वान॒राय॑ । \newline
\pagebreak
\markright{ TS 1.4.14.1  \hfill https://www.vedavms.in \hfill}
\addcontentsline{toc}{section}{ TS 1.4.14.1 }
\section*{ TS 1.4.14.1 }

\textbf{TS 1.4.14.1 } \newline
\textbf{Samhita Paata} \newline

मधु॑श्च॒ माध॑वश्च शु॒क्रश्च॒ शुचि॑श्च॒ नभ॑श्च नभ॒स्य॑श्चे॒षश्चो॒र्जश्च॒ सह॑श्च सह॒स्य॑श्च॒ तप॑श्च तप॒स्य॑श्चो-पया॒मगृ॑हीतोऽसि सꣳ॒॒सर्पो᳚- ऽस्यꣳहस्प॒त्याय॑ त्वा ॥ \newline

\textbf{Pada Paata} \newline

मधुः॑ । च॒ । माध॑वः । च॒ । शु॒क्रः । च॒ । शुचिः॑ । च॒ । नभः॑ । च॒ । न॒भ॒स्यः॑ । च॒ । इ॒षः । च॒ । ऊ॒र्जः । च॒ । सहः॑ । च॒ । स॒ह॒स्यः॑ । च॒ । तपः॑ । च॒ । त॒प॒स्यः॑ । च॒ । उ॒प॒या॒मगृ॑हीत॒ इत्यु॑पया॒म-गृ॒ही॒तः॒ । अ॒सि॒ । सꣳ॒॒सर्प॒ इति॑ सं - सर्पः॑ । अ॒सि॒ । अꣳ॒॒ह॒स्प॒त्यायेत्यꣳ॑हः - प॒त्याय॑ । त्वा॒ ॥  \newline


\textbf{Krama Paata} \newline

मधु॑श्च । च॒ माध॑वः । माध॑वश्च । च॒ शु॒क्रः । शु॒क्रश्च॑ । च॒ शुचिः॑ । शुचि॑श्च । च॒ नभः॑ । नभ॑श्च । च॒ न॒भ॒स्यः॑ । न॒भ॒स्य॑श्च । चे॒षः । इ॒षश्च॑ । चो॒र्जः । ऊ॒र्जश्च॑ । च॒ सहः॑ । सह॑श्च । च॒ स॒ह॒स्यः॑ । स॒ह॒स्य॑श्च । च॒ तपः॑ । तप॑श्च । च॒ त॒प॒स्यः॑ । त॒प॒स्य॑श्च । चो॒प॒या॒मगृ॑हीतः । उ॒प॒या॒मगृ॑हीतोऽसि । उ॒प॒या॒मगृ॑हीत॒ इत्यु॑पया॒म - गृ॒ही॒तः॒ । अ॒सि॒ सꣳ॒॒सर्पः॑ । सꣳ॒॒सर्पो॑ऽसि । सꣳ॒॒सर्प॒ इति॑ सम् - सर्पः॑ । अ॒स्यꣳ॒॒ह॒स्प॒त्याय॑ । अꣳ॒॒ह॒स्प॒त्याय॑ त्वा । अꣳ॒॒ह॒स्प॒त्यायेत्यꣳ॑हः - प॒त्याय॑ । त्वेति॑ त्वा । \newline

\textbf{Jatai Paata} \newline

1. मधु॑श्च च॒ मधु॒र् मधु॑श्च । \newline
2. च॒ माध॑वो॒ माध॑वश्च च॒ माध॑वः । \newline
3. माध॑वश्च च॒ माध॑वो॒ माध॑वश्च । \newline
4. च॒ शु॒क्रः शु॒क्रश्च॑ च शु॒क्रः । \newline
5. शु॒क्रश्च॑ च शु॒क्रः शु॒क्रश्च॑ । \newline
6. च॒ शुचिः॒ शुचि॑श्च च॒ शुचिः॑ । \newline
7. शुचि॑श्च च॒ शुचिः॒ शुचि॑श्च । \newline
8. च॒ नभो॒ नभ॑श्च च॒ नभः॑ । \newline
9. नभ॑श्च च॒ नभो॒ नभ॑श्च । \newline
10. च॒ न॒भ॒स्यो॑ नभ॒स्य॑श्च च नभ॒स्यः॑ । \newline
11. न॒भ॒स्य॑श्च च नभ॒स्यो॑ नभ॒स्य॑श्च । \newline
12. चे॒ ष इ॒षश्च॑ चे॒ षः । \newline
13. इ॒षश्च॑ चे॒ ष इ॒षश्च॑ । \newline
14. चो॒र्ज ऊ॒र्जश्च॑ चो॒र्जः । \newline
15. ऊ॒र्जश्च॑ चो॒र्ज ऊ॒र्जश्च॑ । \newline
16. च॒ सहः॒ सह॑श्च च॒ सहः॑ । \newline
17. सह॑श्च च॒ सहः॒ सह॑श्च । \newline
18. च॒ स॒ह॒स्यः॑ सह॒स्य॑श्च च सह॒स्यः॑ । \newline
19. स॒ह॒स्य॑श्च च सह॒स्यः॑ सह॒स्य॑श्च । \newline
20. च॒ तप॒ स्तप॑श्च च॒ तपः॑ । \newline
21. तप॑श्च च॒ तप॒ स्तप॑श्च । \newline
22. च॒ त॒प॒स्य॑ स्तप॒स्य॑श्च च तप॒स्यः॑ । \newline
23. त॒प॒स्य॑श्च च तप॒स्य॑ स्तप॒स्य॑श्च । \newline
24. चो॒प॒या॒मगृ॑हीत उपया॒मगृ॑हीतश्च चोपया॒मगृ॑हीतः । \newline
25. उ॒प॒या॒मगृ॑हीतो ऽस्यस्युपया॒मगृ॑हीत उपया॒मगृ॑हीतो ऽसि । \newline
26. उ॒प॒या॒मगृ॑हीत॒ इत्यु॑पया॒म - गृ॒ही॒तः॒ । \newline
27. अ॒सि॒ स॒(ग्म्॒)सर्पः॑ स॒(ग्म्॒)सर्पो᳚ ऽस्यसि स॒(ग्म्॒)सर्पः॑ । \newline
28. स॒(ग्म्॒)सर्पो᳚ ऽस्यसि स॒(ग्म्॒)सर्पः॑ स॒(ग्म्॒)सर्पो॑ ऽसि । \newline
29. स॒(ग्म्॒)सर्प॒ इति॑ सं - सर्पः॑ । \newline
30. अ॒स्य॒(ग्म्॒)ह॒स्प॒ त्याया(ग्म्॑)हस्प॒ त्याया᳚ स्यस्यꣳहस्प॒त्याय॑ । \newline
31. अ॒(ग्म्॒)ह॒स्प॒त्याय॑ त्वा त्वाऽ ꣳहस्प॒त्या या(ग्म्॑)हस्प॒त्याय॑ त्वा । \newline
32. अ॒(ग्म्॒)ह॒स्प॒त्यायेत्य(ग्म्॑)हः - प॒त्याय॑ । \newline
33. त्वेति॑ त्वा । \newline

\textbf{Ghana Paata } \newline

1. मधु॑श्च च॒ मधु॒र् मधु॑श्च॒ माध॑वो॒ माध॑वश्च॒ मधु॒र् मधु॑श्च॒ माध॑वः । \newline
2. च॒ माध॑वो॒ माध॑वश्च च॒ माध॑वश्च च॒ माध॑वश्च च॒ माध॑वश्च । \newline
3. माध॑वश्च च॒ माध॑वो॒ माध॑वश्च शु॒क्रः शु॒क्रश्च॒ माध॑वो॒ माध॑वश्च शु॒क्रः । \newline
4. च॒ शु॒क्रः शु॒क्रश्च॑ च शु॒क्रश्च॑ च शु॒क्रश्च॑ च शु॒क्रश्च॑ । \newline
5. शु॒क्रश्च॑ च शु॒क्रः शु॒क्रश्च॒ शुचिः॒ शुचि॑श्च शु॒क्रः शु॒क्रश्च॒ शुचिः॑ । \newline
6. च॒ शुचिः॒ शुचि॑श्च च॒ शुचि॑श्च च॒ शुचि॑श्च च॒ शुचि॑श्च । \newline
7. शुचि॑श्च च॒ शुचिः॒ शुचि॑श्च॒ नभो॒ नभ॑श्च॒ शुचिः॒ शुचि॑श्च॒ नभः॑ । \newline
8. च॒ नभो॒ नभ॑श्च च॒ नभ॑श्च च॒ नभ॑श्च च॒ नभ॑श्च । \newline
9. नभ॑श्च च॒ नभो॒ नभ॑श्च नभ॒स्यो॑ नभ॒स्य॑श्च॒ नभो॒ नभ॑श्च नभ॒स्यः॑ । \newline
10. च॒ न॒भ॒स्यो॑ नभ॒स्य॑श्च च नभ॒स्य॑श्च च नभ॒स्य॑श्च च नभ॒स्य॑श्च । \newline
11. न॒भ॒स्य॑श्च च नभ॒स्यो॑ नभ॒स्य॑श्चे॒ ष इ॒षश्च॑ नभ॒स्यो॑ नभ॒स्य॑श्चे॒ षः । \newline
12. चे॒ ष इ॒षश्च॑ चे॒ षश्च॑ चे॒ षश्च॑ चे॒ षश्च॑ । \newline
13. इ॒षश्च॑ चे॒ ष इ॒षश्चो॒र्ज ऊ॒र्जश्चे॒ ष इ॒षश्चो॒र्जः । \newline
14. चो॒र्ज ऊ॒र्जश्च॑ चो॒र्जश्च॑ चो॒र्जश्च॑ चो॒र्जश्च॑ । \newline
15. ऊ॒र्जश्च॑ चो॒र्ज ऊ॒र्जश्च॒ सहः॒ सह॑श्चो॒र्ज ऊ॒र्जश्च॒ सहः॑ । \newline
16. च॒ सहः॒ सह॑श्च च॒ सह॑श्च च॒ सह॑श्च च॒ सह॑श्च । \newline
17. सह॑श्च च॒ सहः॒ सह॑श्च सह॒स्यः॑ सह॒स्य॑श्च॒ सहः॒ सह॑श्च सह॒स्यः॑ । \newline
18. च॒ स॒ह॒स्यः॑ सह॒स्य॑श्च च सह॒स्य॑श्च च सह॒स्य॑श्च च सह॒स्य॑श्च । \newline
19. स॒ह॒स्य॑श्च च सह॒स्यः॑ सह॒स्य॑श्च॒ तप॒ स्तप॑श्च सह॒स्यः॑ सह॒स्य॑श्च॒ तपः॑ । \newline
20. च॒ तप॒ स्तप॑श्च च॒ तप॑श्च च॒ तप॑श्च च॒ तप॑श्च । \newline
21. तप॑श्च च॒ तप॒ स्तप॑श्च तप॒स्य॑ स्तप॒स्य॑श्च॒ तप॒ स्तप॑श्च तप॒स्यः॑ । \newline
22. च॒ त॒प॒स्य॑ स्तप॒स्य॑श्च च तप॒स्य॑श्च च तप॒स्य॑श्च च तप॒स्य॑श्च । \newline
23. त॒प॒स्य॑श्च च तप॒स्य॑ स्तप॒स्य॑ चोपया॒मगृ॑हीत उपया॒मगृ॑हीतश्च तप॒स्य॑ स्तप॒स्य॑ चोपया॒मगृ॑हीतः । \newline
24. चो॒प॒या॒मगृ॑हीत उपया॒मगृ॑हीतश्च चोपया॒मगृ॑हीतो ऽस्यस्युपया॒मगृ॑हीतश्च चोपया॒मगृ॑हीतो ऽसि । \newline
25. उ॒प॒या॒मगृ॑हीतो ऽस्यस्युपया॒मगृ॑हीत उपया॒मगृ॑हीतो ऽसि सꣳ॒॒सर्पः॑ सꣳ॒॒सर्पो᳚ ऽस्युपया॒मगृ॑हीत उपया॒मगृ॑हीतो ऽसि सꣳ॒॒सर्पः॑ । \newline
26. उ॒प॒या॒मगृ॑हीत॒ इत्यु॑पया॒म - गृ॒ही॒तः॒ । \newline
27. अ॒सि॒ सꣳ॒॒सर्पः॑ सꣳ॒॒सर्पो᳚ ऽस्यसि सꣳ॒॒सर्पो᳚ ऽस्यसि सꣳ॒॒सर्पो᳚ ऽस्यसि सꣳ॒॒सर्पो॑ ऽसि । \newline
28. सꣳ॒॒सर्पो᳚ ऽस्यसि सꣳ॒॒सर्पः॑ सꣳ॒॒सर्पो᳚ ऽस्यꣳहस्प॒त्या या(ग्म्॑)हस्प॒त्या या॑सि सꣳ॒॒सर्पः॑ सꣳ॒॒सर्पो᳚ ऽस्यꣳहस्प॒त्याय॑ । \newline
29. सꣳ॒॒सर्प॒ इति॑ सं - सर्पः॑ । \newline
30. अ॒स्यꣳ॒॒ह॒स्प॒त्या या(ग्म्॑)हस्प॒त्याया᳚ स्यस्यꣳहस्प॒त्याय॑त्वा त्वा ऽꣳहस्प॒त्याया᳚ स्यस्यꣳहस्प॒त्याय॑ त्वा । \newline
31. अꣳ॒॒ह॒स्प॒त्याय॑ त्वा त्वा ऽꣳहस्प॒त्या या(ग्म्॑)हस्प॒त्याय॑ त्वा । \newline
32. अꣳ॒॒ह॒स्प॒त्यायेत्य(ग्म्॑)हः - प॒त्याय॑ । \newline
33. त्वेति॑ त्वा । \newline
\pagebreak
\markright{ TS 1.4.15.1  \hfill https://www.vedavms.in \hfill}
\addcontentsline{toc}{section}{ TS 1.4.15.1 }
\section*{ TS 1.4.15.1 }

\textbf{TS 1.4.15.1 } \newline
\textbf{Samhita Paata} \newline

इन्द्रा᳚ग्नी॒ आ ग॑तꣳ सु॒तं गी॒र्भिर् नभो॒ वरे᳚ण्यं । अ॒स्य पा॑तं धि॒येषि॒ता ॥ उ॒प॒या॒मगृ॑हीतो-ऽसीन्द्रा॒ग्निभ्यां᳚ त्वै॒ष ते॒ योनि॑रिन्द्रा॒ग्निभ्यां᳚ त्वा ॥ \newline

\textbf{Pada Paata} \newline

इन्द्रा᳚ग्नी॒ इतीन्द्र॑ - अ॒ग्नी॒ । एति॑ । ग॒त॒म् । सु॒तम् । गी॒र्भिः । नभः॑ । वरे᳚ण्यम् ॥ अ॒स्य । पा॒त॒म् । धि॒या । इ॒षि॒ता ॥ उ॒प॒या॒मगृ॑हीत॒ इत्यु॑पया॒म - गृ॒ही॒तः॒ । अ॒सि॒ । इ॒न्द्रा॒ग्निभ्या॒मिती᳚न्द्रा॒ग्नि - भ्या॒म् । त्वा॒ । ए॒षः । ते॒ । योनिः॑ । इ॒न्द्रा॒ग्निभ्या॒मिती᳚न्द्रा॒ग्नि - भ्या॒म् । त्वा॒ ॥  \newline


\textbf{Krama Paata} \newline

इन्द्रा᳚ग्नी॒ आ । इन्दा᳚ग्नी॒ इतीन्द्र॑ - अ॒ग्नी॒ । आ ग॑तम् । ग॒तꣳ॒॒ सु॒तम् । सु॒तम् गी॒र्भिः । गी॒र्भिर्,नभः॑ । नभो॒ वरे᳚ण्यम् । वरे᳚ण्य॒मिति॒ वरे᳚ण्यम् ॥ अ॒स्य पा॑तम् । पा॒त॒म् धि॒या । धि॒येषि॒ता । इ॒षि॒तेती॑षि॒ता ॥ उ॒प॒या॒मगृ॑हीतोऽसि । उ॒प॒या॒मगृ॑हीत॒ इत्यु॑पया॒म - गृ॒ही॒तः॒ । अ॒सी॒न्द्रा॒ग्निभ्या᳚म् । इ॒न्द्रा॒ग्निभ्या᳚म् त्वा । इ॒न्द्रा॒ग्निभ्या॒मिती᳚न्द्रा॒ग्नि - भ्या॒म् । त्वै॒षः । ए॒ष ते᳚ । ते॒ योनिः॑ । योनि॑रिन्द्रा॒ग्निभ्या᳚म् । इ॒न्द्रा॒ग्निभ्या᳚म् त्वा । इ॒न्द्रा॒ग्निभ्या॒मिती᳚न्द्रा॒ग्नि - भ्या॒म् । त्वेति॑ त्वा । \newline

\textbf{Jatai Paata} \newline

1. इन्द्रा᳚ग्नी॒ एन्द्रा᳚ग्नी॒ इन्द्रा᳚ग्नी॒ आ । \newline
2. इन्द्रा᳚ग्नी॒ इतीन्द्र॑ - अ॒ग्नी॒ । \newline
3. आ ग॑तम् गत॒ मा ग॑तम् । \newline
4. ग॒त॒(ग्म्॒) सु॒तꣳ सु॒तम् ग॑तम् गतꣳ सु॒तम् । \newline
5. सु॒तम् गी॒र्भिर् गी॒र्भिः सु॒तꣳ सु॒तम् गी॒र्भिः । \newline
6. गी॒र्भिर् नभो॒ नभो॑ गी॒र्भिर् गी॒र्भिर् नभः॑ । \newline
7. नभो॒ वरे᳚ण्यं॒ ॅवरे᳚ण्य॒म् नभो॒ नभो॒ वरे᳚ण्यम् । \newline
8. वरे᳚ण्य॒मिति॒ वरे᳚ण्यम् । \newline
9. अ॒स्य पा॑तम् पात म॒स्यास्य पा॑तम् । \newline
10. पा॒त॒म् धि॒या धि॒या पा॑तम् पातम् धि॒या । \newline
11. धि॒येषि॒तेषि॒ता धि॒या धि॒येषि॒ता । \newline
12. इ॒षि॒तेती॑षि॒ता । \newline
13. उ॒प॒या॒मगृ॑हीतो ऽस्यस्युपया॒मगृ॑हीत उपया॒मगृ॑हीतो ऽसि । \newline
14. उ॒प॒या॒मगृ॑हीत॒ इत्यु॑पया॒म - गृ॒ही॒तः॒ । \newline
15. अ॒सी॒न्द्रा॒ग्निभ्या॑ मिन्द्रा॒ग्निभ्या॑ मस्यसीन्द्रा॒ग्निभ्या᳚म् । \newline
16. इ॒न्द्रा॒ग्निभ्या᳚म् त्वा त्वेन्द्रा॒ग्निभ्या॑ मिन्द्रा॒ग्निभ्या᳚म् त्वा । \newline
17. इ॒न्द्रा॒ग्निभ्या॒मिती᳚न्द्रा॒ग्नि - भ्या॒म् । \newline
18. त्वै॒ष ए॒ष त्वा᳚ त्वै॒षः । \newline
19. ए॒ष ते॑ त ए॒ष ए॒ष ते᳚ । \newline
20. ते॒ योनि॒र् योनि॑ स्ते ते॒ योनिः॑ । \newline
21. योनि॑रिन्द्रा॒ग्निभ्या॑ मिन्द्रा॒ग्निभ्यां॒ ॅयोनि॒र् योनि॑रिन्द्रा॒ग्निभ्या᳚म् । \newline
22. इ॒न्द्रा॒ग्निभ्या᳚म् त्वा त्वेन्द्रा॒ग्निभ्या॑ मिन्द्रा॒ग्निभ्या᳚म् त्वा । \newline
23. इ॒न्द्रा॒ग्निभ्या॒मिती᳚न्द्रा॒ग्नि - भ्या॒म् । \newline
24. त्वेति॑ त्वा । \newline

\textbf{Ghana Paata } \newline

1. इन्द्रा᳚ग्नी॒ एन्द्रा᳚ग्नी॒ इन्द्रा᳚ग्नी॒ आ ग॑तम् गत॒ मेन्द्रा᳚ग्नी॒ इन्द्रा᳚ग्नी॒ आ ग॑तम् । \newline
2. इन्द्रा᳚ग्नी॒ इतीन्द्र॑ - अ॒ग्नी॒ । \newline
3. आ ग॑तम् गत॒ मा ग॑तꣳ सु॒तꣳ सु॒तम् ग॑त॒ मा ग॑तꣳ सु॒तम् । \newline
4. ग॒त॒(ग्म्॒) सु॒तꣳ सु॒तम् ग॑तम् गतꣳ सु॒तम् गी॒र्भिर् गी॒र्भिः सु॒तम् ग॑तम् गतꣳ सु॒तम् गी॒र्भिः । \newline
5. सु॒तम् गी॒र्भिर् गी॒र्भिः सु॒तꣳ सु॒तम् गी॒र्भिर् नभो॒ नभो॑ गी॒र्भिः सु॒तꣳ सु॒तम् गी॒र्भिर् नभः॑ । \newline
6. गी॒र्भिर् नभो॒ नभो॑ गी॒र्भिर् गी॒र्भिर् नभो॒ वरे᳚ण्यं॒ ॅवरे᳚ण्य॒म् नभो॑ गी॒र्भिर् गी॒र्भिर् नभो॒ वरे᳚ण्यम् । \newline
7. नभो॒ वरे᳚ण्यं॒ ॅवरे᳚ण्य॒म् नभो॒ नभो॒ वरे᳚ण्यम् । \newline
8. वरे᳚ण्य॒मिति॒ वरे᳚ण्यम् । \newline
9. अ॒स्य पा॑तम् पात म॒स्यास्य पा॑तम् धि॒या धि॒या पा॑त म॒स्यास्य पा॑तम् धि॒या । \newline
10. पा॒त॒म् धि॒या धि॒या पा॑तम् पातम् धि॒येषि॒तेषि॒ता धि॒या पा॑तम् पातम् धि॒येषि॒ता । \newline
11. धि॒येषि॒तेषि॒ता धि॒या धि॒येषि॒ता । \newline
12. इ॒षि॒तेती॑षि॒ता । \newline
13. उ॒प॒या॒मगृ॑हीतो ऽस्य स्युपया॒मगृ॑हीत उपया॒मगृ॑हीतो ऽसीन्द्रा॒ग्निभ्या॑ मिन्द्रा॒ग्निभ्या॑ मस्युपया॒मगृ॑हीत उपया॒मगृ॑हीतो ऽसीन्द्रा॒ग्निभ्या᳚म् । \newline
14. उ॒प॒या॒मगृ॑हीत॒ इत्यु॑पया॒म - गृ॒ही॒तः॒ । \newline
15. अ॒सी॒न्द्रा॒ग्निभ्या॑ मिन्द्रा॒ग्निभ्या॑ मस्य सीन्द्रा॒ग्निभ्या᳚म् त्वा त्वेन्द्रा॒ग्निभ्या॑ मस्य सीन्द्रा॒ग्निभ्या᳚म् त्वा । \newline
16. इ॒न्द्रा॒ग्निभ्या᳚म् त्वा त्वेन्द्रा॒ग्निभ्या॑ मिन्द्रा॒ग्निभ्या᳚म् त्वै॒ष ए॒ष त्वे᳚न्द्रा॒ग्निभ्या॑ मिन्द्रा॒ग्निभ्या᳚म् त्वै॒षः । \newline
17. इ॒न्द्रा॒ग्निभ्या॒मिती᳚न्द्रा॒ग्नि - भ्या॒म् । \newline
18. त्वै॒ष ए॒ष त्वा᳚ त्वै॒ष ते॑ त ए॒ष त्वा᳚ त्वै॒ष ते᳚ । \newline
19. ए॒ष ते॑ त ए॒ष ए॒ष ते॒ योनि॒र् योनि॑ स्त ए॒ष ए॒ष ते॒ योनिः॑ । \newline
20. ते॒ योनि॒र् योनि॑ स्ते ते॒ योनि॑ रिन्द्रा॒ग्निभ्या॑ मिन्द्रा॒ग्निभ्यां॒ ॅयोनि॑ स्ते ते॒ योनि॑ रिन्द्रा॒ग्निभ्या᳚म् । \newline
21. योनि॑ रिन्द्रा॒ग्निभ्या॑ मिन्द्रा॒ग्निभ्यां॒ ॅयोनि॒र् योनि॑ रिन्द्रा॒ग्निभ्या᳚म् त्वा त्वेन्द्रा॒ग्निभ्यां॒ ॅयोनि॒र् योनि॑ रिन्द्रा॒ग्निभ्या᳚म् त्वा । \newline
22. इ॒न्द्रा॒ग्निभ्या᳚म् त्वा त्वेन्द्रा॒ग्निभ्या॑ मिन्द्रा॒ग्निभ्या᳚म् त्वा । \newline
23. इ॒न्द्रा॒ग्निभ्या॒मिती᳚न्द्रा॒ग्नि - भ्या॒म् । \newline
24. त्वेति॑ त्वा । \newline
\pagebreak
\markright{ TS 1.4.16.1  \hfill https://www.vedavms.in \hfill}
\addcontentsline{toc}{section}{ TS 1.4.16.1 }
\section*{ TS 1.4.16.1 }

\textbf{TS 1.4.16.1 } \newline
\textbf{Samhita Paata} \newline

ओमा॑सश्चर्.षणीधृतो॒ विश्वे॑ देवास॒ आ ग॑त । दा॒श्वाꣳसो॑ दा॒शुषः॑ सु॒तं ॥ उ॒प॒या॒मगृ॑हीतो-ऽसि॒ विश्वे᳚भ्यस्त्वा दे॒वेभ्य॑ ए॒ष ते॒ योनि॒र् विश्वे᳚भ्यस्त्वा दे॒वेभ्यः॑ ॥ \newline

\textbf{Pada Paata} \newline

ओमा॑सः । च॒र्॒.ष॒णी॒धृ॒त॒ इति॑ चर्.षणि - धृ॒तः॒ । विश्वे᳚ । दे॒वा॒सः॒ । एति॑ । ग॒त॒ ॥ दा॒श्वाꣳसः॑ । दा॒शुषः॑ । सु॒तम् ॥ उ॒प॒या॒मगृ॑हीत॒ इत्यु॑पया॒म - गृ॒ही॒तः॒ । अ॒सि॒ । विश्वे᳚भ्यः । त्वा॒ । दे॒वेभ्यः॑ । ए॒षः । ते॒ । योनिः॑ । विश्वे᳚भ्यः । त्वा॒ । दे॒वेभ्यः॑ ॥  \newline


\textbf{Krama Paata} \newline

ओमा॑सश्चर्.षणीधृतः । च॒र्.॒ष॒णी॒धृ॒तो॒ विश्वे᳚ । च॒र्॒.ष॒णी॒धृ॒त॒ इति॑ चर्.षणि - धृ॒तः॒ । विश्वे॑ देवासः । दे॒वा॒स॒ आ । आ ग॑त । ग॒तेति॑ गत ॥ दा॒श्वाꣳसो॑ दा॒शुषः॑ । दा॒शुषः॑ सु॒तम् । सु॒तमिति॑ सु॒तम् ॥ उ॒प॒या॒मगृ॑हीतोऽसि । उ॒प॒या॒मगृ॑हीत॒ इत्यु॑पया॒म - गृ॒ही॒तः॒ । अ॒सि॒ विश्वे᳚भ्यः । विश्वे᳚भ्यस्त्वा । त्वा॒ दे॒वेभ्यः॑ । दे॒वेभ्य॑ ए॒षः । ए॒ष ते᳚ । ते॒ योनिः॑ । योनि॒र् विश्वे᳚भ्यः । विश्वे᳚भ्यस्त्वा । त्वा॒ दे॒वेभ्यः॑ । दे॒वेभ्य॒ इति॑ दे॒वेभ्यः॑ । \newline

\textbf{Jatai Paata} \newline

1. ओमा॑स श्चर्.षणीधृत श्चर्.षणीधृत॒ ओमा॑स॒ ओमा॑स श्चर्.षणीधृतः । \newline
2. च॒र्॒.ष॒णी॒धृ॒तो॒ विश्वे॒ विश्वे॑ चर्.षणीधृत श्चर्.षणीधृतो॒ विश्वे᳚ । \newline
3. च॒र्॒.ष॒णी॒धृ॒त॒ इति॑ चर्.षणि - धृ॒तः॒ । \newline
4. विश्वे॑ देवासो देवासो॒ विश्वे॒ विश्वे॑ देवासः । \newline
5. दे॒वा॒स॒ आ दे॑वासो देवास॒ आ । \newline
6. आ ग॑त ग॒ता ग॑त । \newline
7. ग॒तेति॑ गत । \newline
8. दा॒श्वाꣳसो॑ दा॒शुषो॑ दा॒शुषो॑ दा॒श्वाꣳसो॑ दा॒श्वाꣳसो॑ दा॒शुषः॑ । \newline
9. दा॒शुषः॑ सु॒तꣳ सु॒तम् दा॒शुषो॑ दा॒शुषः॑ सु॒तम् । \newline
10. सु॒तमिति॑ सु॒तम् । \newline
11. उ॒प॒या॒मगृ॑हीतो ऽस्यस्युपया॒मगृ॑हीत उपया॒मगृ॑हीतो ऽसि । \newline
12. उ॒प॒या॒मगृ॑हीत॒ इत्यु॑पया॒म - गृ॒ही॒तः॒ । \newline
13. अ॒सि॒ विश्वे᳚भ्यो॒ विश्वे᳚भ्यो ऽस्यसि॒ विश्वे᳚भ्यः । \newline
14. विश्वे᳚भ्य स्त्वा त्वा॒ विश्वे᳚भ्यो॒ विश्वे᳚भ्य स्त्वा । \newline
15. त्वा॒ दे॒वेभ्यो॑ दे॒वेभ्य॑ स्त्वा त्वा दे॒वेभ्यः॑ । \newline
16. दे॒वेभ्य॑ ए॒ष ए॒ष दे॒वेभ्यो॑ दे॒वेभ्य॑ ए॒षः । \newline
17. ए॒ष ते॑ त ए॒ष ए॒ष ते᳚ । \newline
18. ते॒ योनि॒र् योनि॑ स्ते ते॒ योनिः॑ । \newline
19. योनि॒र् विश्वे᳚भ्यो॒ विश्वे᳚भ्यो॒ योनि॒र् योनि॒र् विश्वे᳚भ्यः । \newline
20. विश्वे᳚भ्य स्त्वा त्वा॒ विश्वे᳚भ्यो॒ विश्वे᳚भ्य स्त्वा । \newline
21. त्वा॒ दे॒वेभ्यो॑ दे॒वेभ्य॑ स्त्वा त्वा दे॒वेभ्यः॑ । \newline
22. दे॒वेभ्य॒ इति॑ दे॒वेभ्यः॑ । \newline

\textbf{Ghana Paata } \newline

1. ओमा॑स श्चर्.षणीधृत श्चर्.षणीधृत॒ ओमा॑स॒ ओमा॑स श्चर्.षणीधृतो॒ विश्वे॒ विश्वे॑ चर्.षणीधृत॒ ओमा॑स॒ ओमा॑स श्चर्.षणीधृतो॒ विश्वे᳚ । \newline
2. च॒र्॒.ष॒णी॒धृ॒तो॒ विश्वे॒ विश्वे॑ चर्.षणीधृत श्चर्.षणीधृतो॒ विश्वे॑ देवासो देवासो॒ विश्वे॑ चर्.षणीधृत श्चर्.षणीधृतो॒ विश्वे॑ देवासः । \newline
3. च॒र्॒.ष॒णी॒धृ॒त॒ इति॑ चर्.षणि - धृ॒तः॒ । \newline
4. विश्वे॑ देवासो देवासो॒ विश्वे॒ विश्वे॑ देवास॒ आ दे॑वासो॒ विश्वे॒ विश्वे॑ देवास॒ आ । \newline
5. दे॒वा॒स॒ आ दे॑वासो देवास॒ आ ग॑त ग॒ता दे॑वासो देवास॒ आ ग॑त । \newline
6. आ ग॑त ग॒ता ग॑त । \newline
7. ग॒तेति॑ गत । \newline
8. दा॒श्वाꣳसो॑ दा॒शुषो॑ दा॒शुषो॑ दा॒श्वाꣳसो॑ दा॒श्वाꣳसो॑ दा॒शुषः॑ सु॒तꣳ सु॒तम् दा॒शुषो॑ दा॒श्वाꣳसो॑ दा॒श्वाꣳसो॑ दा॒शुषः॑ सु॒तम् । \newline
9. दा॒शुषः॑ सु॒तꣳ सु॒तम् दा॒शुषो॑ दा॒शुषः॑ सु॒तम् । \newline
10. सु॒तमिति॑ सु॒तम् । \newline
11. उ॒प॒या॒मगृ॑हीतो ऽस्यस्युपया॒मगृ॑हीत उपया॒मगृ॑हीतो ऽसि॒ विश्वे᳚भ्यो॒ विश्वे᳚भ्यो ऽस्युपया॒मगृ॑हीत उपया॒मगृ॑हीतो ऽसि॒ विश्वे᳚भ्यः । \newline
12. उ॒प॒या॒मगृ॑हीत॒ इत्यु॑पया॒म - गृ॒ही॒तः॒ । \newline
13. अ॒सि॒ विश्वे᳚भ्यो॒ विश्वे᳚भ्यो ऽस्यसि॒ विश्वे᳚भ्य स्त्वा त्वा॒ विश्वे᳚भ्यो ऽस्यसि॒ विश्वे᳚भ्य स्त्वा । \newline
14. विश्वे᳚भ्य स्त्वा त्वा॒ विश्वे᳚भ्यो॒ विश्वे᳚भ्य स्त्वा दे॒वेभ्यो॑ दे॒वेभ्य॑ स्त्वा॒ विश्वे᳚भ्यो॒ विश्वे᳚भ्य स्त्वा दे॒वेभ्यः॑ । \newline
15. त्वा॒ दे॒वेभ्यो॑ दे॒वेभ्य॑ स्त्वा त्वा दे॒वेभ्य॑ ए॒ष ए॒ष दे॒वेभ्य॑ स्त्वा त्वा दे॒वेभ्य॑ ए॒षः । \newline
16. दे॒वेभ्य॑ ए॒ष ए॒ष दे॒वेभ्यो॑ दे॒वेभ्य॑ ए॒ष ते॑ त ए॒ष दे॒वेभ्यो॑ दे॒वेभ्य॑ ए॒ष ते᳚ । \newline
17. ए॒ष ते॑ त ए॒ष ए॒ष ते॒ योनि॒र् योनि॑ स्त ए॒ष ए॒ष ते॒ योनिः॑ । \newline
18. ते॒ योनि॒र् योनि॑ स्ते ते॒ योनि॒र् विश्वे᳚भ्यो॒ विश्वे᳚भ्यो॒ योनि॑ स्ते ते॒ योनि॒र् विश्वे᳚भ्यः । \newline
19. योनि॒र् विश्वे᳚भ्यो॒ विश्वे᳚भ्यो॒ योनि॒र् योनि॒र् विश्वे᳚भ्य स्त्वा त्वा॒ विश्वे᳚भ्यो॒ योनि॒र् योनि॒र् विश्वे᳚भ्य स्त्वा । \newline
20. विश्वे᳚भ्य स्त्वा त्वा॒ विश्वे᳚भ्यो॒ विश्वे᳚भ्य स्त्वा दे॒वेभ्यो॑ दे॒वेभ्य॑ स्त्वा॒ विश्वे᳚भ्यो॒ विश्वे᳚भ्य स्त्वा दे॒वेभ्यः॑ । \newline
21. त्वा॒ दे॒वेभ्यो॑ दे॒वेभ्य॑ स्त्वा त्वा दे॒वेभ्यः॑ । \newline
22. दे॒वेभ्य॒ इति॑ दे॒वेभ्यः॑ । \newline
\pagebreak
\markright{ TS 1.4.17.1  \hfill https://www.vedavms.in \hfill}
\addcontentsline{toc}{section}{ TS 1.4.17.1 }
\section*{ TS 1.4.17.1 }

\textbf{TS 1.4.17.1 } \newline
\textbf{Samhita Paata} \newline

म॒रुत्व॑न्तं ॅवृष॒भं ॅवा॑वृधा॒नमक॑वारिं दि॒व्यꣳ शा॒समिन्द्रं᳚ । वि॒श्वा॒साह॒मव॑से॒ नूत॑नायो॒ग्रꣳ स॑हो॒दामि॒ह तꣳ हु॑वेम ॥ उ॒प॒या॒मगृ॑हीतो॒-ऽसीन्द्रा॑य त्वा म॒रुत्व॑त ए॒ष ते॒ योनि॒रिन्द्रा॑य त्वा म॒रुत्व॑ते ॥ \newline

\textbf{Pada Paata} \newline

म॒रुत्व॑न्तम् । वृ॒ष॒भम् । वा॒वृ॒धा॒नम् । अक॑वारि॒मित्यक॑वा - अ॒रि॒म् । दि॒व्यम् । शा॒सम् । इन्द्र᳚म् ॥ वि॒श्वा॒साह॒मिति॑ विश्व-साह᳚म् । अव॑से । नूत॑नाय । उ॒ग्रम् । स॒हो॒दामिति॑ सहः - दाम् । इ॒ह । तम् । हु॒वे॒म॒ ॥ उ॒प॒या॒मगृ॑हीत॒ इत्यु॑पया॒म - गृ॒ही॒तः॒ । अ॒सि॒ । इन्द्रा॑य । त्वा॒ । म॒रुत्व॑ते । ए॒षः । ते॒ । योनिः॑ । इन्द्रा॑य । त्वा॒ । म॒रुत्व॑ते ॥  \newline


\textbf{Krama Paata} \newline

म॒रुत्व॑न्तं ॅवृष॒भम् । वृ॒ष॒भं ॅवा॑वृधा॒नम् । वा॒वृ॒धा॒नमक॑वारिम् । अक॑वारिम् दि॒व्यम् । अक॑वारि॒मित्यक॑वा - अ॒रि॒म् । दि॒व्यꣳ शा॒सम् । शा॒समिन्द्र᳚म् । इन्द्र॒मितीन्द्र᳚म् ॥ वि॒श्वा॒साह॒मव॑से । वि॒श्वा॒साह॒मिति॑ विश्व - साह᳚म् । अव॑से॒ नूत॑नाय । नूत॑नायो॒ग्रम् । उ॒ग्रꣳ स॑हो॒दाम् । स॒हो॒दामि॒ह । स॒हो॒दामिति॑ सहः - दाम् । इ॒ह तम् । तꣳ हु॑वेम । हु॒वे॒मेति॑ हुवेम । उ॒प॒या॒मगृ॑हीतोऽसि । उ॒प॒या॒मगृ॑हीत॒ इत्यु॑पया॒म - गृ॒ही॒तः॒ । अ॒सीन्द्रा॑य । इन्द्रा॑य त्वा । त्वा॒ म॒रुत्व॑ते । म॒रुत्व॑त ए॒षः । ए॒ष ते᳚ । ते॒ योनिः॑ । योनि॒रिन्द्रा॑य । इन्द्रा॑य त्वा । त्वा॒ म॒रुत्व॑ते । म॒रुत्व॑त॒ इति॑ म॒रुत्व॑ते । \newline

\textbf{Jatai Paata} \newline

1. म॒रुत्व॑न्तं ॅवृष॒भं ॅवृ॑ष॒भम् म॒रुत्व॑न्तम् म॒रुत्व॑न्तं ॅवृष॒भम् । \newline
2. वृ॒ष॒भं ॅवा॑वृधा॒नं ॅवा॑वृधा॒नं ॅवृ॑ष॒भं ॅवृ॑ष॒भं ॅवा॑वृधा॒नम् । \newline
3. वा॒वृ॒धा॒न मक॑वारि॒ मक॑वारिं ॅवावृधा॒नं ॅवा॑वृधा॒न मक॑वारिम् । \newline
4. अक॑वारिम् दि॒व्यम् दि॒व्य मक॑वारि॒ मक॑वारिम् दि॒व्यम् । \newline
5. अक॑वारि॒मित्यक॑वा - अ॒रि॒म् । \newline
6. दि॒व्यꣳ शा॒सꣳ शा॒सम् दि॒व्यम् दि॒व्यꣳ शा॒सम् । \newline
7. शा॒स मिन्द्र॒ मिन्द्र(ग्म्॑) शा॒सꣳ शा॒स मिन्द्र᳚म् । \newline
8. इन्द्र॒मितीन्द्र᳚म् । \newline
9. वि॒श्वा॒साह॒ मव॒से ऽव॑से विश्वा॒साहं॑ ॅविश्वा॒साह॒ मव॑से । \newline
10. वि॒श्वा॒साह॒मिति॑ विश्व - साह᳚म् । \newline
11. अव॑से॒ नूत॑नाय॒ नूत॑ना॒याव॒से ऽव॑से॒ नूत॑नाय । \newline
12. नूत॑नायो॒ग्र मु॒ग्रम् नूत॑नाय॒ नूत॑नायो॒ग्रम् । \newline
13. उ॒ग्रꣳ स॑हो॒दाꣳ स॑हो॒दा मु॒ग्र मु॒ग्रꣳ स॑हो॒दाम् । \newline
14. स॒हो॒दा मि॒हे ह स॑हो॒दाꣳ स॑हो॒दा मि॒ह । \newline
15. स॒हो॒दामिति॑ सहः - दाम् । \newline
16. इ॒ह तम् त मि॒हे ह तम् । \newline
17. तꣳ हु॑वेम हुवेम॒ तम् तꣳ हु॑वेम । \newline
18. हु॒वे॒मेति॑ हुवेम । \newline
19. उ॒प॒या॒मगृ॑हीतो ऽस्यस्युपया॒मगृ॑हीत उपया॒मगृ॑हीतो ऽसि । \newline
20. उ॒प॒या॒मगृ॑हीत॒ इत्यु॑पया॒म - गृ॒ही॒तः॒ । \newline
21. अ॒सीन्द्रा॒ये न्द्रा॑यास्य॒सीन्द्रा॑य । \newline
22. इन्द्रा॑य त्वा॒ त्वेन्द्रा॒ये न्द्रा॑य त्वा । \newline
23. त्वा॒ म॒रुत्व॑ते म॒रुत्व॑ते त्वा त्वा म॒रुत्व॑ते । \newline
24. म॒रुत्व॑त ए॒ष ए॒ष म॒रुत्व॑ते म॒रुत्व॑त ए॒षः । \newline
25. ए॒ष ते॑ त ए॒ष ए॒ष ते᳚ । \newline
26. ते॒ योनि॒र् योनि॑ स्ते ते॒ योनिः॑ । \newline
27. योनि॒ रिन्द्रा॒ये न्द्रा॑य॒ योनि॒र् योनि॒ रिन्द्रा॑य । \newline
28. इन्द्रा॑य त्वा॒ त्वेन्द्रा॒ये न्द्रा॑य त्वा । \newline
29. त्वा॒ म॒रुत्व॑ते म॒रुत्व॑ते त्वा त्वा म॒रुत्व॑ते । \newline
30. म॒रुत्व॑त॒ इति॑ म॒रुत्व॑ते । \newline

\textbf{Ghana Paata } \newline

1. म॒रुत्व॑न्तं ॅवृष॒भं ॅवृ॑ष॒भम् म॒रुत्व॑न्तम् म॒रुत्व॑न्तं ॅवृष॒भं ॅवा॑वृधा॒नं ॅवा॑वृधा॒नं ॅवृ॑ष॒भम् म॒रुत्व॑न्तम् म॒रुत्व॑न्तं ॅवृष॒भं ॅवा॑वृधा॒नम् । \newline
2. वृ॒ष॒भं ॅवा॑वृधा॒नं ॅवा॑वृधा॒नं ॅवृ॑ष॒भं ॅवृ॑ष॒भं ॅवा॑वृधा॒न मक॑वारि॒ मक॑वारिं ॅवावृधा॒नं ॅवृ॑ष॒भं ॅवृ॑ष॒भं ॅवा॑वृधा॒न मक॑वारिम् । \newline
3. वा॒वृ॒धा॒न मक॑वारि॒ मक॑वारिं ॅवावृधा॒नं ॅवा॑वृधा॒न मक॑वारिम् दि॒व्यम् दि॒व्य मक॑वारिं ॅवावृधा॒नं ॅवा॑वृधा॒न मक॑वारिम् दि॒व्यम् । \newline
4. अक॑वारिम् दि॒व्यम् दि॒व्य मक॑वारि॒ मक॑वारिम् दि॒व्यꣳ शा॒सꣳ शा॒सम् दि॒व्य मक॑वारि॒ मक॑वारिम् दि॒व्यꣳ शा॒सम् । \newline
5. अक॑वारि॒मित्यक॑वा - अ॒रि॒म् । \newline
6. दि॒व्यꣳ शा॒सꣳ शा॒सम् दि॒व्यम् दि॒व्यꣳ शा॒स मिन्द्र॒ मिन्द्र(ग्म्॑) शा॒सम् दि॒व्यम् दि॒व्यꣳ शा॒स मिन्द्र᳚म् । \newline
7. शा॒स मिन्द्र॒ मिन्द्र(ग्म्॑) शा॒सꣳ शा॒स मिन्द्र᳚म् । \newline
8. इन्द्र॒मितीन्द्र᳚म् । \newline
9. वि॒श्वा॒साह॒ मव॒से ऽव॑से विश्वा॒साहं॑ ॅविश्वा॒साह॒ मव॑से॒ नूत॑नाय॒ नूत॑ना॒याव॑से विश्वा॒साहं॑ ॅविश्वा॒साह॒ मव॑से॒ नूत॑नाय । \newline
10. वि॒श्वा॒साह॒मिति॑ विश्व - साह᳚म् । \newline
11. अव॑से॒ नूत॑नाय॒ नूत॑ना॒याव॒से ऽव॑से॒ नूत॑नायो॒ग्र मु॒ग्रम् नूत॑ना॒याव॒से ऽव॑से॒ नूत॑नायो॒ग्रम् । \newline
12. नूत॑नायो॒ग्र मु॒ग्रम् नूत॑नाय॒ नूत॑नायो॒ग्रꣳ स॑हो॒दाꣳ स॑हो॒दा मु॒ग्रम् नूत॑नाय॒ नूत॑नायो॒ग्रꣳ स॑हो॒दाम् । \newline
13. उ॒ग्रꣳ स॑हो॒दाꣳ स॑हो॒दा मु॒ग्र मु॒ग्रꣳ स॑हो॒दा मि॒हे ह स॑हो॒दा मु॒ग्र मु॒ग्रꣳ स॑हो॒दा मि॒ह । \newline
14. स॒हो॒दा मि॒हे ह स॑हो॒दाꣳ स॑हो॒दा मि॒ह तम् त मि॒ह स॑हो॒दाꣳ स॑हो॒दा मि॒ह तम् । \newline
15. स॒हो॒दामिति॑ सहः - दाम् । \newline
16. इ॒ह तम् त मि॒हे ह तꣳ हु॑वेम हुवेम॒ त मि॒हे ह तꣳ हु॑वेम । \newline
17. तꣳ हु॑वेम हुवेम॒ तम् तꣳ हु॑वेम । \newline
18. हु॒वे॒मेति॑ हुवेम । \newline
19. उ॒प॒या॒मगृ॑हीतो ऽस्य स्युपया॒मगृ॑हीत उपया॒मगृ॑हीतो॒ ऽसीन्द्रा॒ये न्द्रा॑ यास्युपया॒मगृ॑हीत उपया॒मगृ॑हीतो॒ ऽसीन्द्रा॑य । \newline
20. उ॒प॒या॒मगृ॑हीत॒ इत्यु॑पया॒म - गृ॒ही॒तः॒ । \newline
21. अ॒सीन्द्रा॒ये न्द्रा॑ यास्य॒सीन्द्रा॑य त्वा॒ त्वे न्द्रा॑यास्य॒सीन्द्रा॑य त्वा । \newline
22. इन्द्रा॑य त्वा॒ त्वेन्द्रा॒ये न्द्रा॑य त्वा म॒रुत्व॑ते म॒रुत्व॑ते॒ त्वेन्द्रा॒ये न्द्रा॑य त्वा म॒रुत्व॑ते । \newline
23. त्वा॒ म॒रुत्व॑ते म॒रुत्व॑ते त्वा त्वा म॒रुत्व॑त ए॒ष ए॒ष म॒रुत्व॑ते त्वा त्वा म॒रुत्व॑त ए॒षः । \newline
24. म॒रुत्व॑त ए॒ष ए॒ष म॒रुत्व॑ते म॒रुत्व॑त ए॒ष ते॑ त ए॒ष म॒रुत्व॑ते म॒रुत्व॑त ए॒ष ते᳚ । \newline
25. ए॒ष ते॑ त ए॒ष ए॒ष ते॒ योनि॒र् योनि॑ स्त ए॒ष ए॒ष ते॒ योनिः॑ । \newline
26. ते॒ योनि॒र् योनि॑ स्ते ते॒ योनि॒ रिन्द्रा॒ये न्द्रा॑य॒ योनि॑ स्ते ते॒ योनि॒ रिन्द्रा॑य । \newline
27. योनि॒ रिन्द्रा॒ये न्द्रा॑य॒ योनि॒र् योनि॒ रिन्द्रा॑य त्वा॒ त्वेन्द्रा॑य॒ योनि॒र् योनि॒ रिन्द्रा॑य त्वा । \newline
28. इन्द्रा॑य त्वा॒ त्वेन्द्रा॒ये न्द्रा॑य त्वा म॒रुत्व॑ते म॒रुत्व॑ते॒ त्वेन्द्रा॒ये न्द्रा॑य त्वा म॒रुत्व॑ते । \newline
29. त्वा॒ म॒रुत्व॑ते म॒रुत्व॑ते त्वा त्वा म॒रुत्व॑ते । \newline
30. म॒रुत्व॑त॒ इति॑ म॒रुत्व॑ते । \newline
\pagebreak
\markright{ TS 1.4.18.1  \hfill https://www.vedavms.in \hfill}
\addcontentsline{toc}{section}{ TS 1.4.18.1 }
\section*{ TS 1.4.18.1 }

\textbf{TS 1.4.18.1 } \newline
\textbf{Samhita Paata} \newline

इन्द्र॑ मरुत्व इ॒ह पा॑हि॒ सोमं॒ ॅयथा॑ शार्या॒ते अपि॑बः सु॒तस्य॑ । तव॒ प्रणी॑ती॒ तव॑ शूर॒ शर्म॒न्ना-वि॑वासन्ति क॒वयः॑ सुय॒ज्ञाः ॥ उ॒प॒या॒मगृ॑हीतो॒-ऽसीन्द्रा॑य त्वा म॒रुत्व॑त ए॒ष ते॒ योनि॒रिन्द्रा॑य त्वा म॒रुत्व॑ते ॥ \newline

\textbf{Pada Paata} \newline

इन्द्र॑ । म॒रु॒त्वः॒ । इ॒ह । पा॒हि॒ । सोम᳚म् । यथा᳚ । शा॒र्या॒ते । अपि॑बः । सु॒तस्य॑ ॥ तव॑ । प्रणी॒तीति॒ प्र - नी॒ती॒ । तव॑ । शू॒र॒ । शर्मन्न्॑ । एति॑ । वि॒वा॒स॒न्ति॒ । क॒वयः॑ । सु॒य॒ज्ञा इति॑ सु - य॒ज्ञाः ॥ उ॒प॒या॒मगृ॑हीत॒ इत्यु॑पया॒म - गृ॒ही॒तः॒ । अ॒सि॒ । इन्द्रा॑य । त्वा॒ । म॒रुत्व॑ते । ए॒षः । ते॒ । योनिः॑ । इन्द्रा॑य । त्वा॒ । म॒रुत्व॑ते ॥  \newline


\textbf{Krama Paata} \newline

इन्द्र॑ मरुत्वः । म॒रु॒त्व॒ इ॒ह । इ॒ह पा॑हि । पा॒हि॒ सोम᳚म् । सोमं॒ ॅयथा᳚ । यथा॑ शार्या॒ते । शा॒र्या॒ते अपि॑बः । अपि॑बः सु॒तस्य॑ । सु॒तस्येति॑ सु॒तस्य॑ ॥ तव॒ प्रणी॑ती । प्रणी॑ती॒ तव॑ । प्रणी॒तीति॒ प्र - नी॒ती॒ । तव॑ शूर । शू॒र॒ शर्मन्न्॑ । शर्म॒न्ना । आ वि॑वासन्ति । वि॒वा॒स॒न्ति॒ क॒वयः॑ । क॒वयः॑ सुय॒ज्ञाः । सु॒य॒ज्ञा इति॑ सु - य॒ज्ञाः ॥ उ॒प॒या॒मगृ॑हीतोऽसि । उ॒प॒या॒मगृ॑हीत॒ इत्यु॑पया॒म - गृ॒ही॒तः॒ । अ॒सीन्द्रा॑य । इन्द्रा॑य त्वा । त्वा॒ म॒रुत्व॑ते । म॒रुत्व॑त ए॒षः । ए॒ष ते᳚ । ते॒ योनिः॑ । योनि॒रिन्द्रा॑य । इन्द्रा॑य त्वा । त्वा॒ म॒रुत्व॑ते । म॒रुत्व॑त॒ इति॑ म॒रुत्व॑ते । \newline

\textbf{Jatai Paata} \newline

1. इन्द्र॑ मरुत्वो मरुत्व॒ इन्द्रे न्द्र॑ मरुत्वः । \newline
2. म॒रु॒त्व॒ इ॒हे ह म॑रुत्वो मरुत्व इ॒ह । \newline
3. इ॒ह पा॑हि पाही॒हे ह पा॑हि । \newline
4. पा॒हि॒ सोम॒(ग्म्॒) सोम॑म् पाहि पाहि॒ सोम᳚म् । \newline
5. सोमं॒ ॅयथा॒ यथा॒ सोम॒(ग्म्॒) सोमं॒ ॅयथा᳚ । \newline
6. यथा॑ शार्या॒ते शा᳚र्या॒ते यथा॒ यथा॑ शार्या॒ते । \newline
7. शा॒र्या॒ते अपि॒बो ऽपि॑बः शार्या॒ते शा᳚र्या॒ते अपि॑बः । \newline
8. अपि॑बः सु॒तस्य॑ सु॒तस्यापि॒बो ऽपि॑बः सु॒तस्य॑ । \newline
9. सु॒तस्येति॑ सु॒तस्य॑ । \newline
10. तव॒ प्रणी॑ती॒ प्रणी॑ती॒ तव॒ तव॒ प्रणी॑ती । \newline
11. प्रणी॑ती॒ तव॒ तव॒ प्रणी॑ती॒ प्रणी॑ती॒ तव॑ । \newline
12. प्रणी॒तीति॒ प्र - नी॒ती॒ । \newline
13. तव॑ शूर शूर॒ तव॒ तव॑ शूर । \newline
14. शू॒र॒ शर्म॒ञ् छर्म॑ञ् छूर शूर॒ शर्मन्न्॑ । \newline
15. शर्म॒न् ना शर्म॒ञ् छर्म॒न् ना । \newline
16. आ वि॑वासन्ति विवास॒न्त्या वि॑वासन्ति । \newline
17. वि॒वा॒स॒न्ति॒ क॒वयः॑ क॒वयो॑ विवासन्ति विवासन्ति क॒वयः॑ । \newline
18. क॒वयः॑ सुय॒ज्ञाः सु॑य॒ज्ञाः क॒वयः॑ क॒वयः॑ सुय॒ज्ञाः । \newline
19. सु॒य॒ज्ञा इति॑ सु - य॒ज्ञाः । \newline
20. उ॒प॒या॒मगृ॑हीतो ऽस्यस्युपया॒मगृ॑हीत उपया॒मगृ॑हीतो ऽसि । \newline
21. उ॒प॒या॒मगृ॑हीत॒ इत्यु॑पया॒म - गृ॒ही॒तः॒ । \newline
22. अ॒सीन्द्रा॒ये न्द्रा॑यास्य॒सीन्द्रा॑य । \newline
23. इन्द्रा॑य त्वा॒ त्वेन्द्रा॒ये न्द्रा॑य त्वा । \newline
24. त्वा॒ म॒रुत्व॑ते म॒रुत्व॑ते त्वा त्वा म॒रुत्व॑ते । \newline
25. म॒रुत्व॑त ए॒ष ए॒ष म॒रुत्व॑ते म॒रुत्व॑त ए॒षः । \newline
26. ए॒ष ते॑ त ए॒ष ए॒ष ते᳚ । \newline
27. ते॒ योनि॒र् योनि॑ स्ते ते॒ योनिः॑ । \newline
28. योनि॒ रिन्द्रा॒ये न्द्रा॑य॒ योनि॒र् योनि॒ रिन्द्रा॑य । \newline
29. इन्द्रा॑य त्वा॒ त्वेन्द्रा॒ये न्द्रा॑य त्वा । \newline
30. त्वा॒ म॒रुत्व॑ते म॒रुत्व॑ते त्वा त्वा म॒रुत्व॑ते । \newline
31. म॒रुत्व॑त॒ इति॑ म॒रुत्व॑ते । \newline

\textbf{Ghana Paata } \newline

1. इन्द्र॑ मरुत्वो मरुत्व॒ इन्द्रे न्द्र॑ मरुत्व इ॒हे ह म॑रुत्व॒ इन्द्रे न्द्र॑ मरुत्व इ॒ह । \newline
2. म॒रु॒त्व॒ इ॒हे ह म॑रुत्वो मरुत्व इ॒ह पा॑हि पाही॒ह म॑रुत्वो मरुत्व इ॒ह पा॑हि । \newline
3. इ॒ह पा॑हि पाही॒हे ह पा॑हि॒ सोम॒(ग्म्॒) सोम॑म् पाही॒हे ह पा॑हि॒ सोम᳚म् । \newline
4. पा॒हि॒ सोम॒(ग्म्॒) सोम॑म् पाहि पाहि॒ सोमं॒ ॅयथा॒ यथा॒ सोम॑म् पाहि पाहि॒ सोमं॒ ॅयथा᳚ । \newline
5. सोमं॒ ॅयथा॒ यथा॒ सोम॒(ग्म्॒) सोमं॒ ॅयथा॑ शार्या॒ते शा᳚र्या॒ते यथा॒ सोम॒(ग्म्॒) सोमं॒ ॅयथा॑ शार्या॒ते । \newline
6. यथा॑ शार्या॒ते शा᳚र्या॒ते यथा॒ यथा॑ शार्या॒ते अपि॒बो ऽपि॑बः शार्या॒ते यथा॒ यथा॑ शार्या॒ते अपि॑बः । \newline
7. शा॒र्या॒ते अपि॒बो ऽपि॑बः शार्या॒ते शा᳚र्या॒ते अपि॑बः सु॒तस्य॑ सु॒तस्यापि॑बः शार्या॒ते शा᳚र्या॒ते अपि॑बः सु॒तस्य॑ । \newline
8. अपि॑बः सु॒तस्य॑ सु॒तस्यापि॒बो ऽपि॑बः सु॒तस्य॑ । \newline
9. सु॒तस्येति॑ सु॒तस्य॑ । \newline
10. तव॒ प्रणी॑ती॒ प्रणी॑ती॒ तव॒ तव॒ प्रणी॑ती॒ तव॒ तव॒ प्रणी॑ती॒ तव॒ तव॒ प्रणी॑ती॒ तव॑ । \newline
11. प्रणी॑ती॒ तव॒ तव॒ प्रणी॑ती॒ प्रणी॑ती॒ तव॑ शूर शूर॒ तव॒ प्रणी॑ती॒ प्रणी॑ती॒ तव॑ शूर । \newline
12. प्रणी॒तीति॒ प्र - नी॒ती॒ । \newline
13. तव॑ शूर शूर॒ तव॒ तव॑ शूर॒ शर्म॒ञ् छर्म॑ञ् छूर॒ तव॒ तव॑ शूर॒ शर्मन्न्॑ । \newline
14. शू॒र॒ शर्म॒ञ् छर्म॑ञ् छूर शूर॒ शर्म॒न् ना शर्म॑ञ् छूर शूर॒ शर्म॒न् ना । \newline
15. शर्म॒न् ना शर्म॒ञ् छर्म॒न् ना वि॑वासन्ति विवास॒न्त्या शर्म॒ञ् छर्म॒न् ना वि॑वासन्ति । \newline
16. आ वि॑वासन्ति विवास॒न्त्या वि॑वासन्ति क॒वयः॑ क॒वयो॑ विवास॒न्त्या वि॑वासन्ति क॒वयः॑ । \newline
17. वि॒वा॒स॒न्ति॒ क॒वयः॑ क॒वयो॑ विवासन्ति विवासन्ति क॒वयः॑ सुय॒ज्ञाः सु॑य॒ज्ञाः क॒वयो॑ विवासन्ति विवासन्ति क॒वयः॑ सुय॒ज्ञाः । \newline
18. क॒वयः॑ सुय॒ज्ञाः सु॑य॒ज्ञाः क॒वयः॑ क॒वयः॑ सुय॒ज्ञाः । \newline
19. सु॒य॒ज्ञा इति॑ सु - य॒ज्ञाः । \newline
20. उ॒प॒या॒मगृ॑हीतो ऽस्य स्युपया॒मगृ॑हीत उपया॒मगृ॑हीतो॒ ऽसीन्द्रा॒ये न्द्रा॑या स्युपया॒मगृ॑हीत उपया॒मगृ॑हीतो॒ ऽसीन्द्रा॑य । \newline
21. उ॒प॒या॒मगृ॑हीत॒ इत्यु॑पया॒म - गृ॒ही॒तः॒ । \newline
22. अ॒सीन्द्रा॒ये न्द्रा॑या स्य॒सीन्द्रा॑य त्वा॒ त्वेन्द्रा॑या स्य॒सीन्द्रा॑य त्वा । \newline
23. इन्द्रा॑य त्वा॒ त्वेन्द्रा॒ये न्द्रा॑य त्वा म॒रुत्व॑ते म॒रुत्व॑ते॒ त्वेन्द्रा॒ये न्द्रा॑य त्वा म॒रुत्व॑ते । \newline
24. त्वा॒ म॒रुत्व॑ते म॒रुत्व॑ते त्वा त्वा म॒रुत्व॑त ए॒ष ए॒ष म॒रुत्व॑ते त्वा त्वा म॒रुत्व॑त ए॒षः । \newline
25. म॒रुत्व॑त ए॒ष ए॒ष म॒रुत्व॑ते म॒रुत्व॑त ए॒ष ते॑ त ए॒ष म॒रुत्व॑ते म॒रुत्व॑त ए॒ष ते᳚ । \newline
26. ए॒ष ते॑ त ए॒ष ए॒ष ते॒ योनि॒र् योनि॑ स्त ए॒ष ए॒ष ते॒ योनिः॑ । \newline
27. ते॒ योनि॒र् योनि॑ स्ते ते॒ योनि॒ रिन्द्रा॒ये न्द्रा॑य॒ योनि॑ स्ते ते॒ योनि॒ रिन्द्रा॑य । \newline
28. योनि॒ रिन्द्रा॒ये न्द्रा॑य॒ योनि॒र् योनि॒ रिन्द्रा॑य त्वा॒ त्वेन्द्रा॑य॒ योनि॒र् योनि॒ रिन्द्रा॑य त्वा । \newline
29. इन्द्रा॑य त्वा॒ त्वेन्द्रा॒ये न्द्रा॑य त्वा म॒रुत्व॑ते म॒रुत्व॑ते॒ त्वेन्द्रा॒ये न्द्रा॑य त्वा म॒रुत्व॑ते । \newline
30. त्वा॒ म॒रुत्व॑ते म॒रुत्व॑ते त्वा त्वा म॒रुत्व॑ते । \newline
31. म॒रुत्व॑त॒ इति॑ म॒रुत्व॑ते । \newline
\pagebreak
\markright{ TS 1.4.19.1  \hfill https://www.vedavms.in \hfill}
\addcontentsline{toc}{section}{ TS 1.4.19.1 }
\section*{ TS 1.4.19.1 }

\textbf{TS 1.4.19.1 } \newline
\textbf{Samhita Paata} \newline

म॒रुत्वाꣳ॑ इन्द्र वृष॒भो रणा॑य॒ पिबा॒ सोम॑मनुष्व॒धं मदा॑य । आ सि॑ञ्चस्व ज॒ठरे॒ मद्ध्व॑ ऊ॒र्मिं त्वꣳ राजा॑ऽसि प्र॒दिवः॑ सु॒तानां᳚ ॥ उ॒प॒या॒मगृ॑हीतो॒-ऽसीन्द्रा॑य त्वा म॒रुत्व॑त ए॒ष ते॒ योनि॒रिन्द्रा॑य त्वा म॒रुत्व॑ते ॥ \newline

\textbf{Pada Paata} \newline

म॒रुत्वान्॑ । इ॒न्द्र॒ । वृ॒ष॒भः । रणा॑य । पिब॑ । सोम᳚म् । अ॒नु॒ष्व॒धमित्य॑नु - स्व॒धम् । मदा॑य ॥ एति॑ । सि॒ञ्च॒स्व॒ । ज॒ठरे᳚ । मद्ध्वः॑ । ऊ॒र्मिम् । त्वम् । राजा᳚ । अ॒सि॒ । प्र॒दिव॒ इति॑ प्र - दिवः॑ । सु॒ताना᳚म् ॥ उ॒प॒या॒मगृ॑हीत॒ इत्यु॑पया॒म - गृ॒ही॒तः॒ । अ॒सि॒ । इन्द्रा॑य । त्वा॒ । म॒रुत्व॑ते । ए॒षः । ते॒ । योनिः॑ । इन्द्रा॑य । त्वा॒ । म॒रुत्व॑ते ॥  \newline


\textbf{Krama Paata} \newline

म॒रुत्वाꣳ॑ इन्द्र । इ॒न्द्र॒ वृ॒ष॒भः । वृ॒ष॒भो रणा॑य । रणा॑य॒ पिब॑ । पिबा॒ सोम᳚म् । सोम॑मनुष्व॒धम् । अ॒नु॒ष्व॒धम् मदा॑य । अ॒नु॒ष्व॒धमित्य॑नु - स्व॒धम् । मदा॒येति॒ मदा॑य ॥ आ सि॑ञ्चस्व । सि॒ञ्च॒स्व॒ ज॒ठरे᳚ । ज॒ठरे॒ मद्ध्वः॑ । मद्ध्व॑ ऊ॒र्मिम् । ऊ॒र्मिम् त्वम् । त्वꣳ राजा᳚ । राजा॑ऽसि । अ॒सि॒ प्र॒दिवः॑ । प्र॒दिवः॑ सु॒ताना᳚म् । प्र॒दिव॒ इति॑ प्र - दिवः॑ । सु॒ताना॒मिति॑ सु॒ताना᳚म् ॥ उ॒प॒या॒मगृ॑हीतोऽसि । उ॒प॒या॒मगृ॑हीत॒ इत्यु॑पया॒म - गृ॒ही॒तः॒ । अ॒सीन्द्रा॑य । इन्द्रा॑य त्वा । त्वा॒ म॒रुत्व॑ते । म॒रुत्व॑त ए॒षः । ए॒ष ते᳚ । ते॒ योनिः॑ । योनि॒रिन्द्रा॑य । इन्द्रा॑य त्वा । त्वा॒ म॒रुत्व॑ते । म॒रुत्व॑त॒ इति॑ म॒रुत्व॑ते । \newline

\textbf{Jatai Paata} \newline

1. म॒रुत्वा(ग्म्॑) इन्द्रेन्द्र म॒रुत्वा᳚न् म॒रुत्वा(ग्म्॑) इन्द्र । \newline
2. इ॒न्द्र॒ वृ॒ष॒भो वृ॑ष॒भ इ॑न्द्रेन्द्र वृष॒भः । \newline
3. वृ॒ष॒भो रणा॑य॒ रणा॑य वृष॒भो वृ॑ष॒भो रणा॑य । \newline
4. रणा॑य॒ पिब॒ पिब॒ रणा॑य॒ रणा॑य॒ पिब॑ । \newline
5. पिबा॒ सोम॒(ग्म्॒) सोम॒म् पिब॒ पिबा॒ सोम᳚म् । \newline
6. सोम॑ मनुष्व॒ध म॑नुष्व॒धꣳ सोम॒(ग्म्॒) सोम॑ मनुष्व॒धम् । \newline
7. अ॒नु॒ष्व॒धम् मदा॑य॒ मदा॑यानुष्व॒ध म॑नुष्व॒धम् मदा॑य । \newline
8. अ॒नु॒ष्व॒धमित्य॑नु - स्व॒धम् । \newline
9. मदा॒येति॒ मदा॑य । \newline
10. आ सि॑ञ्चस्व सिञ्च॒स्वा सि॑ञ्चस्व । \newline
11. सि॒ञ्च॒स्व॒ ज॒ठरे॑ ज॒ठरे॑ सिञ्चस्व सिञ्चस्व ज॒ठरे᳚ । \newline
12. ज॒ठरे॒ मद्ध्वो॒ मद्ध्वो॑ ज॒ठरे॑ ज॒ठरे॒ मद्ध्वः॑ । \newline
13. मद्ध्व॑ ऊ॒र्मि मू॒र्मिम् मद्ध्वो॒ मद्ध्व॑ ऊ॒र्मिम् । \newline
14. ऊ॒र्मिम् त्वम् त्व मू॒र्मि मू॒र्मिम् त्वम् । \newline
15. त्वꣳ राजा॒ राजा॒ त्वम् त्वꣳ राजा᳚ । \newline
16. राजा᳚ ऽस्यसि॒ राजा॒ राजा॑ ऽसि । \newline
17. अ॒सि॒ प्र॒दिवः॑ प्र॒दिवो᳚ ऽस्यसि प्र॒दिवः॑ । \newline
18. प्र॒दिवः॑ सु॒ताना(ग्म्॑) सु॒ताना᳚म् प्र॒दिवः॑ प्र॒दिवः॑ सु॒ताना᳚म् । \newline
19. प्र॒दिव॒ इति॑ प्र - दिवः॑ । \newline
20. सु॒ताना॒मिति॑ सु॒ताना᳚म् । \newline
21. उ॒प॒या॒मगृ॑हीतो ऽस्यस्युपया॒मगृ॑हीत उपया॒मगृ॑हीतो ऽसि । \newline
22. उ॒प॒या॒मगृ॑हीत॒ इत्यु॑पया॒म - गृ॒ही॒तः॒ । \newline
23. अ॒सीन्द्रा॒ये न्द्रा॑यास्य॒सीन्द्रा॑य । \newline
24. इन्द्रा॑य त्वा॒ त्वेन्द्रा॒ये न्द्रा॑य त्वा । \newline
25. त्वा॒ म॒रुत्व॑ते म॒रुत्व॑ते त्वा त्वा म॒रुत्व॑ते । \newline
26. म॒रुत्व॑त ए॒ष ए॒ष म॒रुत्व॑ते म॒रुत्व॑त ए॒षः । \newline
27. ए॒ष ते॑ त ए॒ष ए॒ष ते᳚ । \newline
28. ते॒ योनि॒र् योनि॑ स्ते ते॒ योनिः॑ । \newline
29. योनि॒ रिन्द्रा॒ये न्द्रा॑य॒ योनि॒र् योनि॒ रिन्द्रा॑य । \newline
30. इन्द्रा॑य त्वा॒ त्वेन्द्रा॒ये न्द्रा॑य त्वा । \newline
31. त्वा॒ म॒रुत्व॑ते म॒रुत्व॑ते त्वा त्वा म॒रुत्व॑ते । \newline
32. म॒रुत्व॑त॒ इति॑ म॒रुत्व॑ते । \newline

\textbf{Ghana Paata } \newline

1. म॒रुत्वा(ग्म्॑) इन्द्रे न्द्र म॒रुत्वा᳚न् म॒रुत्वा(ग्म्॑) इन्द्र वृष॒भो वृ॑ष॒भ इ॑न्द्र म॒रुत्वा᳚न् म॒रुत्वा(ग्म्॑) इन्द्र वृष॒भः । \newline
2. इ॒न्द्र॒ वृ॒ष॒भो वृ॑ष॒भ इ॑न्द्रे न्द्र वृष॒भो रणा॑य॒ रणा॑य वृष॒भ इ॑न्द्रे न्द्र वृष॒भो रणा॑य । \newline
3. वृ॒ष॒भो रणा॑य॒ रणा॑य वृष॒भो वृ॑ष॒भो रणा॑य॒ पिब॒ पिब॒ रणा॑य वृष॒भो वृ॑ष॒भो रणा॑य॒ पिब॑ । \newline
4. रणा॑य॒ पिब॒ पिब॒ रणा॑य॒ रणा॑य॒ पिबा॒ सोम॒(ग्म्॒) सोम॒म् पिब॒ रणा॑य॒ रणा॑य॒ पिबा॒ सोम᳚म् । \newline
5. पिबा॒ सोम॒(ग्म्॒) सोम॒म् पिब॒ पिबा॒ सोम॑ मनुष्व॒ध म॑नुष्व॒धꣳ सोम॒म् पिब॒ पिबा॒ सोम॑ मनुष्व॒धम् । \newline
6. सोम॑ मनुष्व॒ध म॑नुष्व॒धꣳ सोम॒(ग्म्॒) सोम॑ मनुष्व॒धम् मदा॑य॒ मदा॑यानुष्व॒धꣳ सोम॒(ग्म्॒) सोम॑ मनुष्व॒धम् मदा॑य । \newline
7. अ॒नु॒ष्व॒धम् मदा॑य॒ मदा॑यानुष्व॒ध म॑नुष्व॒धम् मदा॑य । \newline
8. अ॒नु॒ष्व॒धमित्य॑नु - स्व॒धम् । \newline
9. मदा॒येति॒ मदा॑य । \newline
10. आ सि॑ञ्चस्व सिञ्च॒स्वा सि॑ञ्चस्व ज॒ठरे॑ ज॒ठरे॑ सिञ्च॒स्वा सि॑ञ्चस्व ज॒ठरे᳚ । \newline
11. सि॒ञ्च॒स्व॒ ज॒ठरे॑ ज॒ठरे॑ सिञ्चस्व सिञ्चस्व ज॒ठरे॒ मद्ध्वो॒ मद्ध्वो॑ ज॒ठरे॑ सिञ्चस्व सिञ्चस्व ज॒ठरे॒ मद्ध्वः॑ । \newline
12. ज॒ठरे॒ मद्ध्वो॒ मद्ध्वो॑ ज॒ठरे॑ ज॒ठरे॒ मद्ध्व॑ ऊ॒र्मि मू॒र्मिम् मद्ध्वो॑ ज॒ठरे॑ ज॒ठरे॒ मद्ध्व॑ ऊ॒र्मिम् । \newline
13. मद्ध्व॑ ऊ॒र्मि मू॒र्मिम् मद्ध्वो॒ मद्ध्व॑ ऊ॒र्मिम् त्वम् त्व मू॒र्मिम् मद्ध्वो॒ मद्ध्व॑ ऊ॒र्मिम् त्वम् । \newline
14. ऊ॒र्मिम् त्वम् त्व मू॒र्मि मू॒र्मिम् त्वꣳ राजा॒ राजा॒ त्व मू॒र्मि मू॒र्मिम् त्वꣳ राजा᳚ । \newline
15. त्वꣳ राजा॒ राजा॒ त्वम् त्वꣳ राजा᳚ ऽस्यसि॒ राजा॒ त्वम् त्वꣳ राजा॑ ऽसि । \newline
16. राजा᳚ ऽस्यसि॒ राजा॒ राजा॑ ऽसि प्र॒दिवः॑ प्र॒दिवो॑ ऽसि॒ राजा॒ राजा॑ ऽसि प्र॒दिवः॑ । \newline
17. अ॒सि॒ प्र॒दिवः॑ प्र॒दिवो᳚ ऽस्यसि प्र॒दिवः॑ सु॒ताना(ग्म्॑) सु॒ताना᳚म् प्र॒दिवो᳚ ऽस्यसि प्र॒दिवः॑ सु॒ताना᳚म् । \newline
18. प्र॒दिवः॑ सु॒ताना(ग्म्॑) सु॒ताना᳚म् प्र॒दिवः॑ प्र॒दिवः॑ सु॒ताना᳚म् । \newline
19. प्र॒दिव॒ इति॑ प्र - दिवः॑ । \newline
20. सु॒ताना॒मिति॑ सु॒ताना᳚म् । \newline
21. उ॒प॒या॒मगृ॑हीतो ऽस्यस्युपया॒मगृ॑हीत उपया॒मगृ॑हीतो॒ ऽसीन्द्रा॒ये न्द्रा॑या स्युपया॒मगृ॑हीत उपया॒मगृ॑हीतो॒ ऽसीन्द्रा॑य । \newline
22. उ॒प॒या॒मगृ॑हीत॒ इत्यु॑पया॒म - गृ॒ही॒तः॒ । \newline
23. अ॒सीन्द्रा॒ये न्द्रा॑या स्य॒सीन्द्रा॑य त्वा॒ त्वेन्द्रा॑या स्य॒सीन्द्रा॑य त्वा । \newline
24. इन्द्रा॑य त्वा॒ त्वेन्द्रा॒ये न्द्रा॑य त्वा म॒रुत्व॑ते म॒रुत्व॑ते॒ त्वेन्द्रा॒ये न्द्रा॑य त्वा म॒रुत्व॑ते । \newline
25. त्वा॒ म॒रुत्व॑ते म॒रुत्व॑ते त्वा त्वा म॒रुत्व॑त ए॒ष ए॒ष म॒रुत्व॑ते त्वा त्वा म॒रुत्व॑त ए॒षः । \newline
26. म॒रुत्व॑त ए॒ष ए॒ष म॒रुत्व॑ते म॒रुत्व॑त ए॒ष ते॑ त ए॒ष म॒रुत्व॑ते म॒रुत्व॑त ए॒ष ते᳚ । \newline
27. ए॒ष ते॑ त ए॒ष ए॒ष ते॒ योनि॒र् योनि॑ स्त ए॒ष ए॒ष ते॒ योनिः॑ । \newline
28. ते॒ योनि॒र् योनि॑ स्ते ते॒ योनि॒ रिन्द्रा॒ये न्द्रा॑य॒ योनि॑ स्ते ते॒ योनि॒ रिन्द्रा॑य । \newline
29. योनि॒ रिन्द्रा॒ये न्द्रा॑य॒ योनि॒र् योनि॒ रिन्द्रा॑य त्वा॒ त्वेन्द्रा॑य॒ योनि॒र् योनि॒ रिन्द्रा॑य त्वा । \newline
30. इन्द्रा॑य त्वा॒ त्वेन्द्रा॒ये न्द्रा॑य त्वा म॒रुत्व॑ते म॒रुत्व॑ते॒ त्वेन्द्रा॒ये न्द्रा॑य त्वा म॒रुत्व॑ते । \newline
31. त्वा॒ म॒रुत्व॑ते म॒रुत्व॑ते त्वा त्वा म॒रुत्व॑ते । \newline
32. म॒रुत्व॑त॒ इति॑ म॒रुत्व॑ते । \newline
\pagebreak
\markright{ TS 1.4.20.1  \hfill https://www.vedavms.in \hfill}
\addcontentsline{toc}{section}{ TS 1.4.20.1 }
\section*{ TS 1.4.20.1 }

\textbf{TS 1.4.20.1 } \newline
\textbf{Samhita Paata} \newline

म॒हाꣳ इन्द्रो॒ य ओज॑सा प॒र्जन्यो॑ वृष्टि॒माꣳ इ॑व । स्तोमै᳚र्व॒थ्सस्य॑ वावृधे ॥ उ॒प॒या॒मगृ॑हीतो-ऽसि महे॒न्द्राय॑ त्वै॒ष ते॒ योनि॑र् महे॒न्द्राय॑ त्वा ॥ \newline

\textbf{Pada Paata} \newline

म॒हान् । इन्द्रः॑ । यः । ओज॑सा । प॒र्जन्यः॑ । वृ॒ष्टि॒मानिति॑ वृष्टि-मान् । इ॒व॒ ॥ स्तोमैः᳚ । व॒थ्सस्य॑ । वा॒वृ॒धे॒ ॥ उ॒प॒या॒मगृ॑हीत॒ इत्यु॑पया॒म - गृ॒ही॒तः॒ । अ॒सि॒ । म॒हे॒न्द्रायेति॑ महा - इ॒न्द्राय॑ । त्वा॒ । ए॒षः । ते॒ । योनिः॑ । म॒हे॒न्द्रायेति॑ महा - इ॒न्द्राय॑ । त्वा॒ ॥ 21(19)  \newline


\textbf{Krama Paata} \newline

म॒हाꣳ इन्द्रः॑ । इन्द्रो॒ यः । य ओज॑सा । ओज॑सा प॒र्जन्यः॑ । प॒र्जन्यो॑ वृष्टि॒मान् । वृ॒ष्टि॒माꣳ इ॑व । वृ॒ष्टि॒मानिति॑ वृष्टि - मान् । इ॒वेती॑व ॥ स्तोमै᳚र्,व॒थ्सस्य॑ । व॒थ्सस्य॑ वावृधे । वा॒वृ॒ध॒ इति॑ वावृधे ॥ उ॒प॒या॒मगृ॑हीतोऽसि । उ॒प॒या॒मगृ॑हीत॒ इत्यु॑पया॒म - गृ॒ही॒तः॒ । अ॒सि॒ म॒हे॒न्द्राय॑ । म॒हे॒न्द्राय॑ त्वा । म॒हे॒न्द्रायेति॑ महा - इ॒न्द्राय॑ । त्वै॒षः । ए॒ष ते᳚ । ते॒ योनिः॑ । योनि॑र् महे॒न्द्राय॑ । म॒हे॒न्द्राय॑ त्वा । म॒हे॒न्द्रायेति॑ महा - इ॒न्द्राय॑ । त्वेति॑ त्वा । \newline

\textbf{Jatai Paata} \newline

1. म॒हाꣳ इन्द्र॒ इन्द्रो॑ म॒हान् म॒हाꣳ इन्द्रः॑ । \newline
2. इन्द्रो॒ यो य इन्द्र॒ इन्द्रो॒ यः । \newline
3. य ओज॒सौज॑सा॒ यो य ओज॑सा । \newline
4. ओज॑सा प॒र्जन्यः॑ प॒र्जन्य॒ ओज॒सौज॑सा प॒र्जन्यः॑ । \newline
5. प॒र्जन्यो॑ वृष्टि॒मान् वृ॑ष्टि॒मान् प॒र्जन्यः॑ प॒र्जन्यो॑ वृष्टि॒मान् । \newline
6. वृ॒ष्टि॒माꣳ इ॑वे व वृष्टि॒मान् वृ॑ष्टि॒माꣳ इ॑व । \newline
7. वृ॒ष्टि॒मानिति॑ वृष्टि - मान् । \newline
8. इ॒वेती॑व । \newline
9. स्तोमै᳚र् व॒थ्सस्य॑ व॒थ्सस्य॒ स्तोमैः॒ स्तोमै᳚र् व॒थ्सस्य॑ । \newline
10. व॒थ्सस्य॑ वावृधे वावृधे व॒थ्सस्य॑ व॒थ्सस्य॑ वावृधे । \newline
11. वा॒वृ॒ध॒ इति॑ वावृधे । \newline
12. उ॒प॒या॒मगृ॑हीतो ऽस्यस्युपया॒मगृ॑हीत उपया॒मगृ॑हीतो ऽसि । \newline
13. उ॒प॒या॒मगृ॑हीत॒ इत्यु॑पया॒म - गृ॒ही॒तः॒ । \newline
14. अ॒सि॒ म॒हे॒न्द्राय॑ महे॒न्द्राया᳚स्यसि महे॒न्द्राय॑ । \newline
15. म॒हे॒न्द्राय॑ त्वा त्वा महे॒न्द्राय॑ महे॒न्द्राय॑ त्वा । \newline
16. म॒हे॒न्द्रायेति॑ महा - इ॒न्द्राय॑ । \newline
17. त्वै॒ष ए॒ष त्वा᳚ त्वै॒षः । \newline
18. ए॒ष ते॑ त ए॒ष ए॒ष ते᳚ । \newline
19. ते॒ योनि॒र् योनि॑ स्ते ते॒ योनिः॑ । \newline
20. योनि॑र् महे॒न्द्राय॑ महे॒न्द्राय॒ योनि॒र् योनि॑र् महे॒न्द्राय॑ । \newline
21. म॒हे॒न्द्राय॑ त्वा त्वा महे॒न्द्राय॑ महे॒न्द्राय॑ त्वा । \newline
22. म॒हे॒न्द्रायेति॑ महा - इ॒न्द्राय॑ । \newline
23. त्वेति॑ त्वा । \newline

\textbf{Ghana Paata } \newline

1. म॒हाꣳ इन्द्र॒ इन्द्रो॑ म॒हान् म॒हाꣳ इन्द्रो॒ यो य इन्द्रो॑ म॒हान् म॒हाꣳ इन्द्रो॒ यः । \newline
2. इन्द्रो॒ यो य इन्द्र॒ इन्द्रो॒ य ओज॒ सौज॑सा॒ य इन्द्र॒ इन्द्रो॒ य ओज॑सा । \newline
3. य ओज॒ सौज॑सा॒ यो य ओज॑सा प॒र्जन्यः॑ प॒र्जन्य॒ ओज॑सा॒ यो य ओज॑सा प॒र्जन्यः॑ । \newline
4. ओज॑सा प॒र्जन्यः॑ प॒र्जन्य॒ ओज॒ सौज॑सा प॒र्जन्यो॑ वृष्टि॒मान् वृ॑ष्टि॒मान् प॒र्जन्य॒ ओज॒ सौज॑सा प॒र्जन्यो॑ वृष्टि॒मान् । \newline
5. प॒र्जन्यो॑ वृष्टि॒मान् वृ॑ष्टि॒मान् प॒र्जन्यः॑ प॒र्जन्यो॑ वृष्टि॒माꣳ इ॑वे व वृष्टि॒मान् प॒र्जन्यः॑ प॒र्जन्यो॑ वृष्टि॒माꣳ इ॑व । \newline
6. वृ॒ष्टि॒माꣳ इ॑वे व वृष्टि॒मान् वृ॑ष्टि॒माꣳ इ॑व । \newline
7. वृ॒ष्टि॒मानिति॑ वृष्टि - मान् । \newline
8. इ॒वेती॑व । \newline
9. स्तोमै᳚र् व॒थ्सस्य॑ व॒थ्सस्य॒ स्तोमैः॒ स्तोमै᳚र् व॒थ्सस्य॑ वावृधे वावृधे व॒थ्सस्य॒ स्तोमैः॒ स्तोमै᳚र् व॒थ्सस्य॑ वावृधे । \newline
10. व॒थ्सस्य॑ वावृधे वावृधे व॒थ्सस्य॑ व॒थ्सस्य॑ वावृधे । \newline
11. वा॒वृ॒ध॒ इति॑ वावृधे । \newline
12. उ॒प॒या॒मगृ॑हीतो ऽस्य स्युपया॒मगृ॑हीत उपया॒मगृ॑हीतो ऽसि महे॒न्द्राय॑ महे॒न्द्राया᳚ स्युपया॒मगृ॑हीत उपया॒मगृ॑हीतो ऽसि महे॒न्द्राय॑ । \newline
13. उ॒प॒या॒मगृ॑हीत॒ इत्यु॑पया॒म - गृ॒ही॒तः॒ । \newline
14. अ॒सि॒ म॒हे॒न्द्राय॑ महे॒न्द्राया᳚स्यसि महे॒न्द्राय॑ त्वा त्वा महे॒न्द्राया᳚स्यसि महे॒न्द्राय॑ त्वा । \newline
15. म॒हे॒न्द्राय॑ त्वा त्वा महे॒न्द्राय॑ महे॒न्द्राय॑ त्वै॒ष ए॒ष त्वा॑ महे॒न्द्राय॑ महे॒न्द्राय॑ त्वै॒षः । \newline
16. म॒हे॒न्द्रायेति॑ महा - इ॒न्द्राय॑ । \newline
17. त्वै॒ष ए॒ष त्वा᳚ त्वै॒ष ते॑ त ए॒ष त्वा᳚ त्वै॒ष ते᳚ । \newline
18. ए॒ष ते॑ त ए॒ष ए॒ष ते॒ योनि॒र् योनि॑ स्त ए॒ष ए॒ष ते॒ योनिः॑ । \newline
19. ते॒ योनि॒र् योनि॑ स्ते ते॒ योनि॑र् महे॒न्द्राय॑ महे॒न्द्राय॒ योनि॑ स्ते ते॒ योनि॑र् महे॒न्द्राय॑ । \newline
20. योनि॑र् महे॒न्द्राय॑ महे॒न्द्राय॒ योनि॒र् योनि॑र् महे॒न्द्राय॑ त्वा त्वा महे॒न्द्राय॒ योनि॒र् योनि॑र् महे॒न्द्राय॑ त्वा । \newline
21. म॒हे॒न्द्राय॑ त्वा त्वा महे॒न्द्राय॑ महे॒न्द्राय॑ त्वा । \newline
22. म॒हे॒न्द्रायेति॑ महा - इ॒न्द्राय॑ । \newline
23. त्वेति॑ त्वा । \newline
\pagebreak
\markright{ TS 1.4.21.1  \hfill https://www.vedavms.in \hfill}
\addcontentsline{toc}{section}{ TS 1.4.21.1 }
\section*{ TS 1.4.21.1 }

\textbf{TS 1.4.21.1 } \newline
\textbf{Samhita Paata} \newline

म॒हाꣳ इन्द्रो॑ नृ॒वदा च॑र्.षणि॒प्रा उ॒त द्वि॒बर्.हा॑ अमि॒नः सहो॑भिः । अ॒स्म॒द्रिय॑ग्वावृधे वी॒र्या॑यो॒रुः पृ॒थुः सुकृ॑तः क॒र्तृभि॑र्भूत् ॥ उ॒प॒या॒मगृ॑हीतोऽसि महे॒न्द्राय॑ त्वै॒ष ते॒ योनि॑र् महे॒न्द्राय॑ त्वा ॥ \newline

\textbf{Pada Paata} \newline

म॒हान् । इन्द्रः॑ । नृ॒वदिति॑ नृ - वत् । एति॑ । च॒र्॒.ष॒णि॒प्रा इति॑ चर्.षणि - प्राः । उ॒त । द्वि॒बर्.हा॒ इति॑ द्वि - बर्.हाः᳚ । अ॒मि॒नः । सहो॑भि॒रिति॒ सहः॑ - भिः॒ ॥ अ॒स्म॒द्रिय॒गित्य॑स्म - द्रिय॑क् । वा॒वृ॒धे॒ । वी॒र्या॑य । उ॒रुः । पृ॒थुः । सुकृ॑त॒ इति॒ सु - कृ॒तः॒ । क॒र्तृभि॒रिति॑ क॒र्तृ - भिः॒ । भू॒त् ॥ उ॒प॒या॒मगृ॑हीत॒ इत्यु॑पया॒म - गृ॒ही॒तः॒ । अ॒सि॒ । म॒हे॒न्द्रायेति॑ महा - इ॒न्द्राय॑ । त्वा॒ । ए॒षः । ते॒ । योनिः॑ । म॒हे॒न्द्रायेति॑ महा - इ॒न्द्राय॑ । त्वा॒ ॥  \newline


\textbf{Krama Paata} \newline

म॒हाꣳ इन्द्रः॑ । इन्द्रो॑ नृ॒वत् । नृ॒वदा । नृ॒वदिति॑ नृ - वत् । आ च॑र्.षणि॒प्राः । च॒र्.॒ष॒णि॒प्रा उ॒त । च॒र्.॒ष॒णि॒प्रा इति॑ चर्.षणि - प्राः । उ॒त द्वि॒बर्.हाः᳚ । द्वि॒बर्.हा॑ अमि॒नः । द्वि॒बर्.हा॒ इति॑ द्वि - बर्.हाः᳚ । अ॒मि॒नः सहो॑भिः । सहो॑भि॒रिति॒ सहः॑ - भिः॒ ॥ अ॒स्म॒द्रिय॑ग् वावृधे । अ॒स्म॒द्रिय॒गित्य॑स्म - द्रिय॑क् । वा॒वृ॒धे॒ वी॒र्या॑य । वी॒र्या॑यो॒रुः । उ॒रुः पृ॒थुः । पृ॒थुः सुकृ॑तः । सुकृ॑तः क॒र्तृभिः॑ । सुकृ॑त॒ इति॒ सु - कृ॒तः॒ । क॒र्तृभि॑र् भूत् । क॒र्तृभि॒रिति॑ क॒र्तृ - भिः॒ । भू॒दिति॑ भूत् ॥ उ॒प॒या॒मगृ॑हीतोऽसि । उ॒प॒या॒मगृ॑हीत॒ इत्यु॑पया॒म - गृ॒ही॒तः॒ । अ॒सि॒ म॒हे॒न्द्राय॑ । म॒हे॒न्द्राय॑ त्वा । म॒हे॒न्द्रायेति॑ महा - इ॒न्द्राय॑ । त्वै॒षः । ए॒ष ते᳚ । ते॒ योनिः॑ । योनि॑र् महे॒न्द्राय॑ । म॒हे॒न्द्राय॑ त्वा । म॒हे॒न्द्रायेति॑ महा - इ॒न्द्राय॑ । त्वेति॑ त्वा । \newline

\textbf{Jatai Paata} \newline

1. म॒हाꣳ इन्द्र॒ इन्द्रो॑ म॒हान् म॒हाꣳ इन्द्रः॑ । \newline
2. इन्द्रो॑ नृ॒वन् नृ॒वदिन्द्र॒ इन्द्रो॑ नृ॒वत् । \newline
3. नृ॒वदा नृ॒वन् नृ॒वदा । \newline
4. नृ॒वदिति॑ नृ - वत् । \newline
5. आ च॑र्.षणि॒प्रा श्च॑र्.षणि॒प्रा आ च॑र्.षणि॒प्राः । \newline
6. च॒र्॒.ष॒णि॒प्रा उ॒तोत च॑र्.षणि॒प्रा श्च॑र्.षणि॒प्रा उ॒त । \newline
7. च॒र्॒.ष॒णि॒प्रा इति॑ चर्.षणि - प्राः । \newline
8. उ॒त द्वि॒बर्.हा᳚ द्वि॒बर्.हा॑ उ॒तोत द्वि॒बर्.हाः᳚ । \newline
9. द्वि॒बर्.हा॑ अमि॒नो॑ ऽमि॒नो द्वि॒बर्.हा᳚ द्वि॒बर्.हा॑ अमि॒नः । \newline
10. द्वि॒बर्.हा॒ इति॑ द्वि - बर्.हाः᳚ । \newline
11. अ॒मि॒नः सहो॑भिः॒ सहो॑भि रमि॒नो॑ ऽमि॒नः सहो॑भिः । \newline
12. सहो॑भि॒रिति॒ सहः॑ - भिः॒ । \newline
13. अ॒स्म॒द्रिय॑ग् वावृधे वावृधे ऽस्म॒द्रिय॑गस्म॒द्रिय॑ग् वावृधे । \newline
14. अ॒स्म॒द्रिय॒गित्य॑स्म - द्रिय॑क् । \newline
15. वा॒वृ॒धे॒ वी॒र्या॑य वी॒र्या॑य वावृधे वावृधे वी॒र्या॑य । \newline
16. वी॒र्या॑यो॒ रुरु॒रुर् वी॒र्या॑य वी॒र्या॑यो॒रुः । \newline
17. उ॒रुः पृ॒थुः पृ॒थु रु॒रु रु॒रुः पृ॒थुः । \newline
18. पृ॒थुः सुकृ॑तः॒ सुकृ॑तः पृ॒थुः पृ॒थुः सुकृ॑तः । \newline
19. सुकृ॑तः क॒र्तृभिः॑ क॒र्तृभिः॒ सुकृ॑तः॒ सुकृ॑तः क॒र्तृभिः॑ । \newline
20. सुकृ॑त॒ इति॒ सु - कृ॒तः॒ । \newline
21. क॒र्तृभि॑र् भूद् भूत् क॒र्तृभिः॑ क॒र्तृभि॑र् भूत् । \newline
22. क॒र्तृभि॒रिति॑ क॒र्तृ - भिः॒ । \newline
23. भू॒दिति॑ भूत् । \newline
24. उ॒प॒या॒मगृ॑हीतो ऽस्यस्युपया॒मगृ॑हीत उपया॒मगृ॑हीतो ऽसि । \newline
25. उ॒प॒या॒मगृ॑हीत॒ इत्यु॑पया॒म - गृ॒ही॒तः॒ । \newline
26. अ॒सि॒ म॒हे॒न्द्राय॑ महे॒न्द्राया᳚स्यसि महे॒न्द्राय॑ । \newline
27. म॒हे॒न्द्राय॑ त्वा त्वा महे॒न्द्राय॑ महे॒न्द्राय॑ त्वा । \newline
28. म॒हे॒न्द्रायेति॑ महा - इ॒न्द्राय॑ । \newline
29. त्वै॒ष ए॒ष त्वा᳚ त्वै॒षः । \newline
30. ए॒ष ते॑ त ए॒ष ए॒ष ते᳚ । \newline
31. ते॒ योनि॒र् योनि॑ स्ते ते॒ योनिः॑ । \newline
32. योनि॑र् महे॒न्द्राय॑ महे॒न्द्राय॒ योनि॒र् योनि॑र् महे॒न्द्राय॑ । \newline
33. म॒हे॒न्द्राय॑ त्वा त्वा महे॒न्द्राय॑ महे॒न्द्राय॑ त्वा । \newline
34. म॒हे॒न्द्रायेति॑ महा - इ॒न्द्राय॑ । \newline
35. त्वेति॑ त्वा । \newline

\textbf{Ghana Paata } \newline

1. म॒हाꣳ इन्द्र॒ इन्द्रो॑ म॒हान् म॒हाꣳ इन्द्रो॑ नृ॒वन् नृ॒वदिन्द्रो॑ म॒हान् म॒हाꣳ इन्द्रो॑ नृ॒वत् । \newline
2. इन्द्रो॑ नृ॒वन् नृ॒वदिन्द्र॒ इन्द्रो॑ नृ॒वदा नृ॒वदिन्द्र॒ इन्द्रो॑ नृ॒वदा । \newline
3. नृ॒वदा नृ॒वन् नृ॒वदा च॑र्.षणि॒प्रा श्च॑र्.षणि॒प्रा आ नृ॒वन् नृ॒वदा च॑र्.षणि॒प्राः । \newline
4. नृ॒वदिति॑ नृ - वत् । \newline
5. आ च॑र्.षणि॒प्रा श्च॑र्.षणि॒प्रा आ च॑र्.षणि॒प्रा उ॒तोत च॑र्.षणि॒प्रा आ च॑र्.षणि॒प्रा उ॒त । \newline
6. च॒र्॒.ष॒णि॒प्रा उ॒तोत च॑र्.षणि॒प्रा श्च॑र्.षणि॒प्रा उ॒त द्वि॒बर्.हा᳚ द्वि॒बर्.हा॑ उ॒त च॑र्.षणि॒प्रा श्च॑र्.षणि॒प्रा उ॒त द्वि॒बर्.हाः᳚ । \newline
7. च॒र्॒.ष॒णि॒प्रा इति॑ चर्.षणि - प्राः । \newline
8. उ॒त द्वि॒बर्.हा᳚ द्वि॒बर्.हा॑ उ॒तोत द्वि॒बर्.हा॑ अमि॒नो॑ ऽमि॒नो द्वि॒बर्.हा॑ उ॒तोत द्वि॒बर्.हा॑ अमि॒नः । \newline
9. द्वि॒बर्.हा॑ अमि॒नो॑ ऽमि॒नो द्वि॒बर्.हा᳚ द्वि॒बर्.हा॑ अमि॒नः सहो॑भिः॒ सहो॑भि रमि॒नो द्वि॒बर्.हा᳚ द्वि॒बर्.हा॑ अमि॒नः सहो॑भिः । \newline
10. द्वि॒बर्.हा॒ इति॑ द्वि - बर्.हाः᳚ । \newline
11. अ॒मि॒नः सहो॑भिः॒ सहो॑भि रमि॒नो॑ ऽमि॒नः सहो॑भिः । \newline
12. सहो॑भि॒रिति॒ सहः॑ - भिः॒ । \newline
13. अ॒स्म॒द्रिय॑ग् वावृधे वावृधे ऽस्म॒द्रिय॑ गस्म॒द्रिय॑ग् वावृधे वी॒र्या॑य वी॒र्या॑य वावृधे ऽस्म॒द्रिय॑ गस्म॒द्रिय॑ग् वावृधे वी॒र्या॑य । \newline
14. अ॒स्म॒द्रिय॒गित्य॑स्म - द्रिय॑क् । \newline
15. वा॒वृ॒धे॒ वी॒र्या॑य वी॒र्या॑य वावृधे वावृधे वी॒र्या॑यो॒ रुरु॒रुर् वी॒र्या॑य वावृधे वावृधे वी॒र्या॑यो॒रुः । \newline
16. वी॒र्या॑यो॒ रुरु॒रुर् वी॒र्या॑य वी॒र्या॑यो॒रुः पृ॒थुः पृ॒थु रु॒रुर् वी॒र्या॑य वी॒र्या॑यो॒रुः पृ॒थुः । \newline
17. उ॒रुः पृ॒थुः पृ॒थु रु॒रु रु॒रुः पृ॒थुः सुकृ॑तः॒ सुकृ॑तः पृ॒थु रु॒रु रु॒रुः पृ॒थुः सुकृ॑तः । \newline
18. पृ॒थुः सुकृ॑तः॒ सुकृ॑तः पृ॒थुः पृ॒थुः सुकृ॑तः क॒र्तृभिः॑ क॒र्तृभिः॒ सुकृ॑तः पृ॒थुः पृ॒थुः सुकृ॑तः क॒र्तृभिः॑ । \newline
19. सुकृ॑तः क॒र्तृभिः॑ क॒र्तृभिः॒ सुकृ॑तः॒ सुकृ॑तः क॒र्तृभि॑र् भूद् भूत् क॒र्तृभिः॒ सुकृ॑तः॒ सुकृ॑तः क॒र्तृभि॑र् भूत् । \newline
20. सुकृ॑त॒ इति॒ सु - कृ॒तः॒ । \newline
21. क॒र्तृभि॑र् भूद् भूत् क॒र्तृभिः॑ क॒र्तृभि॑र् भूत् । \newline
22. क॒र्तृभि॒रिति॑ क॒र्तृ - भिः॒ । \newline
23. भू॒दिति॑ भूत् । \newline
24. उ॒प॒या॒मगृ॑हीतो ऽस्य स्युपया॒मगृ॑हीत उपया॒मगृ॑हीतो ऽसि महे॒न्द्राय॑ महे॒न्द्राया᳚ स्युपया॒मगृ॑हीत उपया॒मगृ॑हीतो ऽसि महे॒न्द्राय॑ । \newline
25. उ॒प॒या॒मगृ॑हीत॒ इत्यु॑पया॒म - गृ॒ही॒तः॒ । \newline
26. अ॒सि॒ म॒हे॒न्द्राय॑ महे॒न्द्रा या᳚स्यसि महे॒न्द्राय॑ त्वा त्वा महे॒न्द्रा या᳚स्यसि महे॒न्द्राय॑ त्वा । \newline
27. म॒हे॒न्द्राय॑ त्वा त्वा महे॒न्द्राय॑ महे॒न्द्राय॑ त्वै॒ष ए॒ष त्वा॑ महे॒न्द्राय॑ महे॒न्द्राय॑ त्वै॒षः । \newline
28. म॒हे॒न्द्रायेति॑ महा - इ॒न्द्राय॑ । \newline
29. त्वै॒ष ए॒ष त्वा᳚ त्वै॒ष ते॑ त ए॒ष त्वा᳚ त्वै॒ष ते᳚ । \newline
30. ए॒ष ते॑ त ए॒ष ए॒ष ते॒ योनि॒र् योनि॑ स्त ए॒ष ए॒ष ते॒ योनिः॑ । \newline
31. ते॒ योनि॒र् योनि॑ स्ते ते॒ योनि॑र् महे॒न्द्राय॑ महे॒न्द्राय॒ योनि॑ स्ते ते॒ योनि॑र् महे॒न्द्राय॑ । \newline
32. योनि॑र् महे॒न्द्राय॑ महे॒न्द्राय॒ योनि॒र् योनि॑र् महे॒न्द्राय॑ त्वा त्वा महे॒न्द्राय॒ योनि॒र् योनि॑र् महे॒न्द्राय॑ त्वा । \newline
33. म॒हे॒न्द्राय॑ त्वा त्वा महे॒न्द्राय॑ महे॒न्द्राय॑ त्वा । \newline
34. म॒हे॒न्द्रायेति॑ महा - इ॒न्द्राय॑ । \newline
35. त्वेति॑ त्वा । \newline
\pagebreak
\markright{ TS 1.4.22.1  \hfill https://www.vedavms.in \hfill}
\addcontentsline{toc}{section}{ TS 1.4.22.1 }
\section*{ TS 1.4.22.1 }

\textbf{TS 1.4.22.1 } \newline
\textbf{Samhita Paata} \newline

क॒दा च॒न स्त॒रीर॑सि॒ नेन्द्र॑ सश्चसि दा॒शुषे᳚ । उपो॒पेन्नु म॑घव॒न् भूय॒ इन्नु ते॒ दानं॑ दे॒वस्य॑ पृच्यते ॥ उ॒प॒या॒मगृ॑हीतोऽस्या-दि॒त्येभ्य॑स्त्वा ॥ क॒दा च॒न प्र यु॑च्छस्यु॒भे नि पा॑सि॒ जन्म॑नी । तुरी॑यादित्य॒ सव॑नं त इन्द्रि॒यमा त॑स्थाव॒मृतं॑ दि॒वि ॥ य॒ज्ञो दे॒वानां॒ प्रत्ये॑ति सु॒म्नमादि॑त्यासो॒ भव॑ता मृड॒यन्तः॑ । आ वो॒ ( ) ऽर्वाची॑ सुम॒तिर् व॑वृत्यादꣳ॒॒हो-श्चि॒द्या व॑रिवो॒वित्त॒राऽस॑त् ॥ विव॑स्व आदित्यै॒ष ते॑ सोमपी॒थस्तेन॑ मन्दस्व॒ तेन॑ तृप्य तृ॒प्यास्म॑ ते व॒यं त॑र्पयि॒तारो॒ या दि॒व्या वृष्टि॒स्तया᳚ त्वा श्रीणामि ॥ \newline

\textbf{Pada Paata} \newline

क॒दा । च॒न । स्त॒रीः । अ॒सि॒ । न । इ॒न्द्र॒ । स॒श्च॒सि॒ । दा॒शुषे᳚ ॥ उपो॒पेत्युप॑ - उ॒प॒ । इत् । नु । म॒घ॒व॒न्निति॑ मघ - व॒न्न्॒ । भूयः॑ । इत् । नु । ते॒ । दान᳚म् । दे॒वस्य॑ । पृ॒च्य॒ते॒ ॥ उ॒प॒या॒मगृ॑हीत॒ इत्यु॑पया॒म - गृ॒ही॒तः॒ । अ॒सि॒ । आ॒दि॒त्येभ्यः॑ । त्वा॒ ॥ क॒दा । च॒न । प्रेति॑ । यु॒च्छ॒सि॒ । उ॒भे इति॑ । नीति॑ । पा॒सि॒ । जन्म॑नी॒ इति॑ ॥ तुरी॑य । आ॒दि॒त्य॒ । सव॑नम् । ते॒ । इ॒न्द्रि॒यम् । एति॑ । त॒स्थौ॒ । अ॒मृत᳚म् । दि॒वि ॥ य॒ज्ञ्ः । दे॒वाना᳚म् । प्रतीति॑ । ए॒ति॒ । सु॒म्नम् । आदि॑त्यासः । भव॑त । मृ॒ड॒यन्तः॑ ॥ एति॑ । वः॒ ( ) । अ॒र्वाची᳚ । सु॒म॒तिरिति॑ सु - म॒तिः । व॒वृ॒त्या॒त् । अꣳ॒॒होः । चि॒त् । या । व॒रि॒वो॒वित्त॒रेति॑ वरिवो॒वित् - त॒रा॒ । अस॑त् ॥ विव॑स्वः । आ॒दि॒त्य॒ । ए॒षः । ते॒ । सो॒म॒पी॒थ इति॑ सोम - पी॒थः । तेन॑ । म॒न्द॒स्व॒ । तेन॑ । तृ॒प्य॒ । तृ॒प्यास्म॑ । ते॒ । व॒यम् । त॒र्प॒यि॒तारः॑ । या । दि॒व्या । वृष्टिः॑ । तया᳚ । त्वा॒ । श्री॒णा॒मि॒ ॥  \newline


\textbf{Krama Paata} \newline

क॒दा च॒न । च॒न स्त॒रीः । स्त॒रीर॑सि । अ॒सि॒ न । नेन्द्र॑ । इ॒न्द्र॒ स॒श्च॒सि॒ । स॒श्च॒सि॒ दा॒शुषे᳚ । दा॒शुष॒ इति॑ दा॒शुषे᳚ ॥ उपो॒पेत् । उपो॒पेत्युप॑ - उ॒प॒ ॥ इन्नु । नु म॑घवन्न् । म॒घ॒व॒न्,भूयः॑ । म॒घ॒व॒न्निति॑ मघ - व॒न्न्॒ । भूय॒ इत् । इन्नु । नु ते᳚ । ते॒ दान᳚म् । दान॑म् दे॒वस्य॑ । दे॒वस्य॑ पृच्यते । पृ॒च्य॒त॒ इति॑ पृच्यते ॥ उ॒प॒या॒मगृ॑हीतोऽसि । उ॒प॒या॒मगृ॑हीत॒ इत्यु॑पया॒म - गृ॒ही॒तः॒ । अ॒स्या॒दि॒त्येभ्यः॑ । आ॒दि॒त्येभ्य॑स्त्वा । त्वेति॑ त्वा ॥ क॒दा च॒न । च॒न प्र । प्र यु॑च्छसि । यु॒च्छ॒स्यु॒भे । उ॒भे नि । उ॒भे इत्यु॒भे । नि पा॑सि । पा॒सि॒ जन्म॑नी । जन्म॑नी॒ इति॒ जन्म॑नी ॥ तुरी॑यादित्य । आ॒दि॒त्य॒ सव॑नम् । सव॑नम् ते । त॒ इ॒न्द्रि॒यम् । इ॒न्द्रि॒यमा । आ त॑स्थौ । त॒स्था॒व॒मृत᳚म् । अ॒मृत॑म् दि॒वि । दि॒वीति॑ दि॒वि ॥ य॒ज्ञो दे॒वाना᳚म् । दे॒वाना॒म् प्रति॑ । प्रत्ये॑ति । ए॒ति॒ सु॒म्नम् । सु॒म्नमादि॑त्यासः । आदि॑त्यासो॒ भव॑त । भव॑ता मृड॒यन्तः॑ । मृ॒ड॒यन्त॒ इति॑ मृड॒यन्तः॑ ॥ आ वः॑ । वो॒ऽर्वाची᳚ । अ॒र्वाची॑ सुम॒तिः । सु॒म॒तिर् व॑वृत्यात् । सु॒म॒तिरिति॑ सु - म॒तिः । व॒वृ॒त्या॒दꣳ॒॒होः । अꣳ॒॒होश्चि॑त् । चि॒द्या । या व॑रिवो॒वित्त॑रा । व॒रि॒वो॒वित्त॒रा ऽस॑त् । व॒रि॒वो॒वित्त॒रेति॑ वरिवो॒वित् - त॒रा॒ । अस॒दित्यस॑त् ॥ विव॑स्व आदित्य । आ॒दि॒त्यै॒षः । ए॒ष ते᳚ । ते॒ सो॒म॒पी॒थः । सो॒म॒पी॒थस्तेन॑ । सो॒म॒पी॒थ इति॑ सोम - पी॒थः । तेन॑ मन्दस्व । म॒न्द॒स्व॒ तेन॑ । तेन॑ तृप्य । तृ॒प्य॒ तृ॒प्यास्म॑ । तृ॒प्यास्म॑ ते । ते॒ व॒यम् । व॒यम् त॑र्पयि॒तारः॑ । त॒र्प॒यि॒तारो॒ या । या दि॒व्या । दि॒व्या वृष्टिः॑ । वृष्टि॒स्तया᳚ । तया᳚ त्वा । त्वा॒ श्री॒णा॒मि॒ । श्री॒णा॒मीति॑ श्रीणामि । \newline

\textbf{Jatai Paata} \newline

1. क॒दा च॒न च॒न क॒दा क॒दा च॒न । \newline
2. च॒न स्त॒रीः स्त॒री श्च॒न च॒न स्त॒रीः । \newline
3. स्त॒री र॑स्यसि स्त॒रीः स्त॒री र॑सि । \newline
4. अ॒सि॒ न नास्य॑सि॒ न । \newline
5. ने न्द्रे᳚ न्द्र॒ न ने न्द्र॑ । \newline
6. इ॒न्द्र॒ स॒श्च॒सि॒ स॒श्च॒सी॒न्द्रे॒न्द्र॒ स॒श्च॒सि॒ । \newline
7. स॒श्च॒सि॒ दा॒शुषे॑ दा॒शुषे॑ सश्चसि सश्चसि दा॒शुषे᳚ । \newline
8. दा॒शुष॒ इति॑ दा॒शुषे᳚ । \newline
9. उपो॒पे दिदुपो॒पो पो॒पेत् । \newline
10. उपो॒पेत्युप॑ - उ॒प॒ । \newline
11. इन्नु न्विदिन्नु । \newline
12. नु म॑घवन् मघव॒न् नु नु म॑घवन्न् । \newline
13. म॒घ॒व॒न् भूयो॒ भूयो॑ मघवन् मघव॒न् भूयः॑ । \newline
14. म॒घ॒व॒न्निति॑ मघ - व॒न्न्॒ । \newline
15. भूय॒ इदिद् भूयो॒ भूय॒ इत् । \newline
16. इन् नु न्विदिन् नु । \newline
17. नु ते॑ ते॒ नु नु ते᳚ । \newline
18. ते॒ दान॒म् दान॑म् ते ते॒ दान᳚म् । \newline
19. दान॑म् दे॒वस्य॑ दे॒वस्य॒ दान॒म् दान॑म् दे॒वस्य॑ । \newline
20. दे॒वस्य॑ पृच्यते पृच्यते दे॒वस्य॑ दे॒वस्य॑ पृच्यते । \newline
21. पृ॒च्य॒त॒ इति॑ पृच्यते । \newline
22. उ॒प॒या॒मगृ॑हीतो ऽस्यस्युपया॒मगृ॑हीत उपया॒मगृ॑हीतो ऽसि । \newline
23. उ॒प॒या॒मगृ॑हीत॒ इत्यु॑पया॒म - गृ॒ही॒तः॒ । \newline
24. अ॒स्या॒दि॒त्येभ्य॑ आदि॒त्येभ्यो᳚ ऽस्यस्यादि॒त्येभ्यः॑ । \newline
25. आ॒दि॒त्येभ्य॑ स्त्वा त्वा ऽऽदि॒त्येभ्य॑ आदि॒त्येभ्य॑ स्त्वा । \newline
26. त्वेति॑ त्वा । \newline
27. क॒दा च॒न च॒न क॒दा क॒दा च॒न । \newline
28. च॒न प्र प्र च॒न च॒न प्र । \newline
29. प्र यु॑च्छसि युच्छसि॒ प्र प्र यु॑च्छसि । \newline
30. यु॒च्छ॒स्यु॒भे उ॒भे यु॑च्छसि युच्छस्यु॒भे । \newline
31. उ॒भे निन्यु॑भे उ॒भे नि । \newline
32. उ॒भे इत्यु॒भे । \newline
33. नि पा॑सि पासि॒ नि नि पा॑सि । \newline
34. पा॒सि॒ जन्म॑नी॒ जन्म॑नी पासि पासि॒ जन्म॑नी । \newline
35. जन्म॑नी॒ इति॒ जन्म॑नी । \newline
36. तुरी॑यादित्यादित्य॒ तुरी॑य॒ तुरी॑यादित्य । \newline
37. आ॒दि॒त्य॒ सव॑न॒(ग्म्॒) सव॑न मादित्यादित्य॒ सव॑नम् । \newline
38. सव॑नम् ते ते॒ सव॑न॒(ग्म्॒) सव॑नम् ते । \newline
39. त॒ इ॒न्द्रि॒य मि॑न्द्रि॒यम् ते॑ त इन्द्रि॒यम् । \newline
40. इ॒न्द्रि॒य मेन्द्रि॒य मि॑न्द्रि॒य मा । \newline
41. आ त॑स्थौ तस्था॒ वा त॑स्थौ । \newline
42. त॒स्था॒ व॒मृत॑ म॒मृत॑म् तस्थौ तस्था व॒मृत᳚म् । \newline
43. अ॒मृत॑म् दि॒वि दि॒व्य॑मृत॑ म॒मृत॑म् दि॒वि । \newline
44. दि॒वीति॑ दि॒वि । \newline
45. य॒ज्ञो दे॒वाना᳚म् दे॒वानां᳚ ॅय॒ज्ञो य॒ज्ञो दे॒वाना᳚म् । \newline
46. दे॒वाना॒म् प्रति॒ प्रति॑ दे॒वाना᳚म् दे॒वाना॒म् प्रति॑ । \newline
47. प्रत्ये᳚त्येति॒ प्रति॒ प्रत्ये॑ति । \newline
48. ए॒ति॒ सु॒म्नꣳ सु॒म्न मे᳚त्येति सु॒म्नम् । \newline
49. सु॒म्न मादि॑त्यास॒ आदि॑त्यासः सु॒म्नꣳ सु॒म्न मादि॑त्यासः । \newline
50. आदि॑त्यासो॒ भव॑त॒ भव॒तादि॑त्यास॒ आदि॑त्यासो॒ भव॑त । \newline
51. भव॑ता मृड॒यन्तो॑ मृड॒यन्तो॒ भव॑त॒ भव॑ता मृड॒यन्तः॑ । \newline
52. मृ॒ड॒यन्त॒ इति॑ मृड॒यन्तः॑ । \newline
53. आ वो॑ व॒ आ वः॑ । \newline
54. वो॒ ऽर्वाच्य॒र्वाची॑ वो वो॒ ऽर्वाची᳚ । \newline
55. अ॒र्वाची॑ सुम॒तिः सु॑म॒ति र॒र्वाच्य॒र्वाची॑ सुम॒तिः । \newline
56. सु॒म॒तिर् व॑वृत्याद् ववृत्याथ् सु॑म॒तिः सु॑म॒तिर् व॑वृत्यात् । \newline
57. सु॒म॒तिरिति॑ सु - म॒तिः । \newline
58. व॒वृ॒त्या॒ द॒(ग्म्॒)हो र॒(ग्म्॒)होर् व॑वृत्याद् ववृत्या द॒(ग्म्॒)होः । \newline
59. अ॒(ग्म्॒)हो श्चि॑च् चिद॒(ग्म्॒)हो र॒(ग्म्॒)हो श्चि॑त् । \newline
60. चि॒द् या या चि॑च् चि॒द् या । \newline
61. या व॑रिवो॒वित्त॑रा वरिवो॒वित्त॑रा॒ या या व॑रिवो॒वित्त॑रा । \newline
62. व॒रि॒वो॒वित्त॒रा ऽस॒दस॑द् वरिवो॒वित्त॑रा वरिवो॒वित्त॒रा ऽस॑त् । \newline
63. व॒रि॒वो॒वित्त॒रेति॑ वरिवो॒वित् - त॒रा॒ । \newline
64. अस॒दित्यस॑त् । \newline
65. विव॑स्व आदित्यादित्य॒ विव॑स्वो॒ विव॑स्व आदित्य । \newline
66. आ॒दि॒त्यै॒ष ए॒ष आ॑दित्यादित्यै॒षः । \newline
67. ए॒ष ते॑ त ए॒ष ए॒ष ते᳚ । \newline
68. ते॒ सो॒म॒पी॒थः सो॑मपी॒थ स्ते॑ ते सोमपी॒थः । \newline
69. सो॒म॒पी॒थ स्तेन॒ तेन॑ सोमपी॒थः सो॑मपी॒थ स्तेन॑ । \newline
70. सो॒म॒पी॒थ इति॑ सोम - पी॒थः । \newline
71. तेन॑ मन्दस्व मन्दस्व॒ तेन॒ तेन॑ मन्दस्व । \newline
72. म॒न्द॒स्व॒ तेन॒ तेन॑ मन्दस्व मन्दस्व॒ तेन॑ । \newline
73. तेन॑ तृप्य तृप्य॒ तेन॒ तेन॑ तृप्य । \newline
74. तृ॒प्य॒ तृ॒प्यास्म॑ तृ॒प्यास्म॑ तृप्य तृप्य तृ॒प्यास्म॑ । \newline
75. तृ॒प्यास्म॑ ते ते तृ॒प्यास्म॑ तृ॒प्यास्म॑ ते । \newline
76. ते॒ व॒यं ॅव॒यम् ते॑ ते व॒यम् । \newline
77. व॒यम् त॑र्पयि॒तार॑ स्तर्पयि॒तारो॑ व॒यं ॅव॒यम् त॑र्पयि॒तारः॑ । \newline
78. त॒र्प॒यि॒तारो॒ या या त॑र्पयि॒तार॑ स्तर्पयि॒तारो॒ या । \newline
79. या दि॒व्या दि॒व्या या या दि॒व्या । \newline
80. दि॒व्या वृष्टि॒र् वृष्टि॑र् दि॒व्या दि॒व्या वृष्टिः॑ । \newline
81. वृष्टि॒ स्तया॒ तया॒ वृष्टि॒र् वृष्टि॒ स्तया᳚ । \newline
82. तया᳚ त्वा त्वा॒ तया॒ तया᳚ त्वा । \newline
83. त्वा॒ श्री॒णा॒मि॒ श्री॒णा॒मि॒ त्वा॒ त्वा॒ श्री॒णा॒मि॒ । \newline
84. श्री॒णा॒मीति॑ श्रीणामि । \newline

\textbf{Ghana Paata } \newline

1. क॒दा च॒न च॒न क॒दा क॒दा च॒न स्त॒रीः स्त॒री श्च॒न क॒दा क॒दा च॒न स्त॒रीः । \newline
2. च॒न स्त॒रीः स्त॒री श्च॒न च॒न स्त॒री र॑स्यसि स्त॒रीश्च॒न च॒न स्त॒ रीर॑सि । \newline
3. स्त॒री र॑स्यसि स्त॒रीः स्त॒री र॑सि॒ न नासि॑ स्त॒रीः स्त॒री र॑सि॒ न । \newline
4. अ॒सि॒ न नास्य॑सि॒ ने न्द्रे᳚ न्द्र॒ नास्य॑सि॒ ने न्द्र॑ । \newline
5. ने न्द्रे᳚ न्द्र॒ न ने न्द्र॑ सश्चसि सश्चसीन्द्र॒ न ने न्द्र॑ सश्चसि । \newline
6. इ॒न्द्र॒ स॒श्च॒सि॒ स॒श्च॒सी॒न्द्रे॒ न्द्र॒ स॒श्च॒सि॒ दा॒शुषे॑ दा॒शुषे॑ सश्चसीन्द्रे न्द्र सश्चसि दा॒शुषे᳚ । \newline
7. स॒श्च॒सि॒ दा॒शुषे॑ दा॒शुषे॑ सश्चसि सश्चसि दा॒शुषे᳚ । \newline
8. दा॒शुष॒ इति॑ दा॒शुषे᳚ । \newline
9. उपो॒पे दिदुपो॒ पोपो॒पेन् नु न्विदुपो॒ पोपो॒पेन् नु । \newline
10. उपो॒पेत्युप॑ - उ॒प॒ । \newline
11. इन् नु न्विदिन् नु म॑घवन् मघव॒न् न्विदिन् नु म॑घवन्न् । \newline
12. नु म॑घवन् मघव॒न् नु नु म॑घव॒न् भूयो॒ भूयो॑ मघव॒न् नु नु म॑घव॒न् भूयः॑ । \newline
13. म॒घ॒व॒न् भूयो॒ भूयो॑ मघवन् मघव॒न् भूय॒ इदिद् भूयो॑ मघवन् मघव॒न् भूय॒ इत् । \newline
14. म॒घ॒व॒न्निति॑ मघ - व॒न्न्॒ । \newline
15. भूय॒ इदिद् भूयो॒ भूय॒ इन् नु न्विद् भूयो॒ भूय॒ इन् नु । \newline
16. इन् नु न्विदिन् नु ते॑ ते॒ न्विदिन् नु ते᳚ । \newline
17. नु ते॑ ते॒ नु नु ते॒ दान॒म् दान॑म् ते॒ नु नु ते॒ दान᳚म् । \newline
18. ते॒ दान॒म् दान॑म् ते ते॒ दान॑म् दे॒वस्य॑ दे॒वस्य॒ दान॑म् ते ते॒ दान॑म् दे॒वस्य॑ । \newline
19. दान॑म् दे॒वस्य॑ दे॒वस्य॒ दान॒म् दान॑म् दे॒वस्य॑ पृच्यते पृच्यते दे॒वस्य॒ दान॒म् दान॑म् दे॒वस्य॑ पृच्यते । \newline
20. दे॒वस्य॑ पृच्यते पृच्यते दे॒वस्य॑ दे॒वस्य॑ पृच्यते । \newline
21. पृ॒च्य॒त॒ इति॑ पृच्यते । \newline
22. उ॒प॒या॒मगृ॑हीतो ऽस्यस्युपया॒मगृ॑हीत उपया॒मगृ॑हीतो ऽस्यादि॒त्येभ्य॑ आदि॒त्येभ्यो᳚ ऽस्युपया॒मगृ॑हीत उपया॒मगृ॑हीतो ऽस्यादि॒त्येभ्यः॑ । \newline
23. उ॒प॒या॒मगृ॑हीत॒ इत्यु॑पया॒म - गृ॒ही॒तः॒ । \newline
24. अ॒स्या॒दि॒त्येभ्य॑ आदि॒त्येभ्यो᳚ ऽस्य स्यादि॒त्येभ्य॑ स्त्वा त्वा ऽऽदि॒त्येभ्यो᳚ ऽस्य स्यादि॒त्येभ्य॑ स्त्वा । \newline
25. आ॒दि॒त्येभ्य॑ स्त्वा त्वा ऽऽदि॒त्येभ्य॑ आदि॒त्येभ्य॑ स्त्वा । \newline
26. त्वेति॑ त्वा । \newline
27. क॒दा च॒न च॒न क॒दा क॒दा च॒न प्र प्र च॒न क॒दा क॒दा च॒न प्र । \newline
28. च॒न प्र प्र च॒न च॒न प्र यु॑च्छसि युच्छसि॒ प्र च॒न च॒न प्र यु॑च्छसि । \newline
29. प्र यु॑च्छसि युच्छसि॒ प्र प्र यु॑च्छ स्यु॒भे उ॒भे यु॑च्छसि॒ प्र प्र यु॑च्छ स्यु॒भे । \newline
30. यु॒च्छ॒ स्यु॒भे उ॒भे यु॑च्छसि युच्छ स्यु॒भे नि न्यु॑भे यु॑च्छसि युच्छ स्यु॒भे नि । \newline
31. उ॒भे नि न्यु॑भे उ॒भे नि पा॑सि पासि॒ न्यु॑भे उ॒भे नि पा॑सि । \newline
32. उ॒भे इत्यु॒भे । \newline
33. नि पा॑सि पासि॒ नि नि पा॑सि॒ जन्म॑नी॒ जन्म॑नी पासि॒ नि नि पा॑सि॒ जन्म॑नी । \newline
34. पा॒सि॒ जन्म॑नी॒ जन्म॑नी पासि पासि॒ जन्म॑नी । \newline
35. जन्म॑नी॒ इति॒ जन्म॑नी । \newline
36. तुरी॑यादित्यादित्य॒ तुरी॑य॒ तुरी॑यादित्य॒ सव॑न॒(ग्म्॒) सव॑न मादित्य॒ तुरी॑य॒ तुरी॑यादित्य॒ सव॑नम् । \newline
37. आ॒दि॒त्य॒ सव॑न॒(ग्म्॒) सव॑न मादित्यादित्य॒ सव॑नम् ते ते॒ सव॑न मादित्यादित्य॒ सव॑नम् ते । \newline
38. सव॑नम् ते ते॒ सव॑न॒(ग्म्॒) सव॑नम् त इन्द्रि॒य मि॑न्द्रि॒यम् ते॒ सव॑न॒(ग्म्॒) सव॑नम् त इन्द्रि॒यम् । \newline
39. त॒ इ॒न्द्रि॒य मि॑न्द्रि॒यम् ते॑ त इन्द्रि॒य मेन्द्रि॒यम् ते॑ त इन्द्रि॒य मा । \newline
40. इ॒न्द्रि॒य मेन्द्रि॒य मि॑न्द्रि॒य मा त॑स्थौ तस्था॒ वेन्द्रि॒य मि॑न्द्रि॒य मा त॑स्थौ । \newline
41. आ त॑स्थौ तस्था॒ वा त॑स्था व॒मृत॑ म॒मृत॑म् तस्था॒ वा त॑स्था व॒मृत᳚म् । \newline
42. त॒स्था॒ व॒मृत॑ म॒मृत॑म् तस्थौ तस्था व॒मृत॑म् दि॒वि दि॒व्य॑मृत॑म् तस्थौ तस्था व॒मृत॑म् दि॒वि । \newline
43. अ॒मृत॑म् दि॒वि दि॒व्य॑मृत॑ म॒मृत॑म् दि॒वि । \newline
44. दि॒वीति॑ दि॒वि । \newline
45. य॒ज्ञो दे॒वाना᳚म् दे॒वानां᳚ ॅय॒ज्ञो य॒ज्ञो दे॒वाना॒म् प्रति॒ प्रति॑ दे॒वानां᳚ ॅय॒ज्ञो य॒ज्ञो दे॒वाना॒म् प्रति॑ । \newline
46. दे॒वाना॒म् प्रति॒ प्रति॑ दे॒वाना᳚म् दे॒वाना॒म् प्रत्ये᳚त्येति॒ प्रति॑ दे॒वाना᳚म् दे॒वाना॒म् प्रत्ये॑ति । \newline
47. प्रत्ये᳚त्येति॒ प्रति॒ प्रत्ये॑ति सु॒म्नꣳ सु॒म्न मे॑ति॒ प्रति॒ प्रत्ये॑ति सु॒म्नम् । \newline
48. ए॒ति॒ सु॒म्नꣳ सु॒म्न मे᳚त्येति सु॒म्न मादि॑त्यास॒ आदि॑त्यासः सु॒म्न मे᳚त्येति सु॒म्न मादि॑त्यासः । \newline
49. सु॒म्न मादि॑त्यास॒ आदि॑त्यासः सु॒म्नꣳ सु॒म्न मादि॑त्यासो॒ भव॑त॒ भव॒तादि॑त्यासः सु॒म्नꣳ सु॒म्न मादि॑त्यासो॒ भव॑त । \newline
50. आदि॑त्यासो॒ भव॑त॒ भव॒तादि॑त्यास॒ आदि॑त्यासो॒ भव॑ता मृड॒यन्तो॑ मृड॒यन्तो॒ भव॒तादि॑त्यास॒ आदि॑त्यासो॒ भव॑ता मृड॒यन्तः॑ । \newline
51. भव॑ता मृड॒यन्तो॑ मृड॒यन्तो॒ भव॑त॒ भव॑ता मृड॒यन्तः॑ । \newline
52. मृ॒ड॒यन्त॒ इति॑ मृड॒यन्तः॑ । \newline
53. आ वो॑ व॒ आ वो॒ ऽर्वाच्य॒र्वाची॑ व॒ आ वो॒ ऽर्वाची᳚ । \newline
54. वो॒ ऽर्वाच्य॒र्वाची॑ वो वो॒ ऽर्वाची॑ सुम॒तिः सु॑म॒ति र॒र्वाची॑ वो वो॒ ऽर्वाची॑ सुम॒तिः । \newline
55. अ॒र्वाची॑ सुम॒तिः सु॑म॒ति र॒र्वाच्य॒र्वाची॑ सुम॒तिर् व॑वृत्याद् ववृत्याथ् सुम॒ति र॒र्वाच्य॒र्वाची॑ सुम॒तिर् व॑वृत्यात् । \newline
56. सु॒म॒तिर् व॑वृत्याद् ववृत्याथ् सुम॒तिः सु॑म॒तिर् व॑वृत्या दꣳ॒॒हो रꣳ॒॒होर् व॑वृत्याथ् सुम॒तिः सु॑म॒तिर् व॑वृत्या दꣳ॒॒होः । \newline
57. सु॒म॒तिरिति॑ सु - म॒तिः । \newline
58. व॒वृ॒त्या॒ दꣳ॒॒हो रꣳ॒॒होर् व॑वृत्याद् ववृत्या दꣳ॒॒हो श्चि॑च् चि दꣳ॒॒होर् व॑वृत्याद् ववृत्या दꣳ॒॒होश्चि॑त् । \newline
59. अꣳ॒॒हो श्चि॑च् चि दꣳ॒॒हो रꣳ॒॒हो श्चि॒द् या या चि॑ दꣳ॒॒हो रꣳ॒॒हो श्चि॒द् या । \newline
60. चि॒द् या या चि॑च् चि॒द् या व॑रिवो॒वित्त॑रा वरिवो॒वित्त॑रा॒ या चि॑च् चि॒द् या व॑रिवो॒वित्त॑रा । \newline
61. या व॑रिवो॒वित्त॑रा वरिवो॒वित्त॑रा॒ या या व॑रिवो॒वित्त॒रा ऽस॒ दस॑द् वरिवो॒वित्त॑रा॒ या या व॑रिवो॒वित्त॒रा ऽस॑त् । \newline
62. व॒रि॒वो॒वित्त॒रा ऽस॒ दस॑द् वरिवो॒वित्त॑रा वरिवो॒वित्त॒रा ऽस॑त् । \newline
63. व॒रि॒वो॒वित्त॒रेति॑ वरिवो॒वित् - त॒रा॒ । \newline
64. अस॒दित्यस॑त् । \newline
65. विव॑स्व आदित्यादित्य॒ विव॑स्वो॒ विव॑स्व आदित्यै॒ष ए॒ष आ॑दित्य॒ विव॑स्वो॒ विव॑स्व आदित्यै॒षः । \newline
66. आ॒दि॒त्यै॒ष ए॒ष आ॑दि त्यादित्यै॒ष ते॑ त ए॒ष आ॑दित्यादित्यै॒ष ते᳚ । \newline
67. ए॒ष ते॑ त ए॒ष ए॒ष ते॑ सोमपी॒थः सो॑मपी॒थ स्त॑ ए॒ष ए॒ष ते॑ सोमपी॒थः । \newline
68. ते॒ सो॒म॒पी॒थः सो॑मपी॒थ स्ते॑ ते सोमपी॒थ स्तेन॒ तेन॑ सोमपी॒थ स्ते॑ ते सोमपी॒थ स्तेन॑ । \newline
69. सो॒म॒पी॒थ स्तेन॒ तेन॑ सोमपी॒थः सो॑मपी॒थ स्तेन॑ मन्दस्व मन्दस्व॒ तेन॑ सोमपी॒थः सो॑मपी॒थ 
स्तेन॑ मन्दस्व । \newline
70. सो॒म॒पी॒थ इति॑ सोम - पी॒थः । \newline
71. तेन॑ मन्दस्व मन्दस्व॒ तेन॒ तेन॑ मन्दस्व॒ तेन॒ तेन॑ मन्दस्व॒ तेन॒ तेन॑ मन्दस्व॒ तेन॑ । \newline
72. म॒न्द॒स्व॒ तेन॒ तेन॑ मन्दस्व मन्दस्व॒ तेन॑ तृप्य तृप्य॒ तेन॑ मन्दस्व मन्दस्व॒ तेन॑ तृप्य । \newline
73. तेन॑ तृप्य तृप्य॒ तेन॒ तेन॑ तृप्य तृ॒प्यास्म॑ तृ॒प्यास्म॑ तृप्य॒ तेन॒ तेन॑ तृप्य तृ॒प्यास्म॑ । \newline
74. तृ॒प्य॒ तृ॒प्यास्म॑ तृ॒प्यास्म॑ तृप्य तृप्य तृ॒प्यास्म॑ ते ते तृ॒प्यास्म॑ तृप्य तृप्य तृ॒प्यास्म॑ ते । \newline
75. तृ॒प्यास्म॑ ते ते तृ॒प्यास्म॑ तृ॒प्यास्म॑ ते व॒यं ॅव॒यम् ते॑ तृ॒प्यास्म॑ तृ॒प्यास्म॑ ते व॒यम् । \newline
76. ते॒ व॒यं ॅव॒यम् ते॑ ते व॒यम् त॑र्पयि॒तार॑ स्तर्पयि॒तारो॑ व॒यम् ते॑ ते व॒यम् त॑र्पयि॒तारः॑ । \newline
77. व॒यम् त॑र्पयि॒तार॑ स्तर्पयि॒तारो॑ व॒यं ॅव॒यम् त॑र्पयि॒तारो॒ या या त॑र्पयि॒तारो॑ व॒यं ॅव॒यम् त॑र्पयि॒तारो॒ या । \newline
78. त॒र्प॒यि॒तारो॒ या या त॑र्पयि॒तार॑ स्तर्पयि॒तारो॒ या दि॒व्या दि॒व्या या त॑र्पयि॒तार॑ स्तर्पयि॒तारो॒ या दि॒व्या । \newline
79. या दि॒व्या दि॒व्या या या दि॒व्या वृष्टि॒र् वृष्टि॑र् दि॒व्या या या दि॒व्या वृष्टिः॑ । \newline
80. दि॒व्या वृष्टि॒र् वृष्टि॑र् दि॒व्या दि॒व्या वृष्टि॒ स्तया॒ तया॒ वृष्टि॑र् दि॒व्या दि॒व्या वृष्टि॒ स्तया᳚ । \newline
81. वृष्टि॒ स्तया॒ तया॒ वृष्टि॒र् वृष्टि॒ स्तया᳚ त्वा त्वा॒ तया॒ वृष्टि॒र् वृष्टि॒ स्तया᳚ त्वा । \newline
82. तया᳚ त्वा त्वा॒ तया॒ तया᳚ त्वा श्रीणामि श्रीणामि त्वा॒ तया॒ तया᳚ त्वा श्रीणामि । \newline
83. त्वा॒ श्री॒णा॒मि॒ श्री॒णा॒मि॒ त्वा॒ त्वा॒ श्री॒णा॒मि॒ । \newline
84. श्री॒णा॒मीति॑ श्रीणामि । \newline
\pagebreak
\markright{ TS 1.4.23.1  \hfill https://www.vedavms.in \hfill}
\addcontentsline{toc}{section}{ TS 1.4.23.1 }
\section*{ TS 1.4.23.1 }

\textbf{TS 1.4.23.1 } \newline
\textbf{Samhita Paata} \newline

वा॒मम॒द्य स॑वितर्वा॒ममु॒ श्वो दि॒वेदि॑वे वा॒मम॒स्मभ्यꣳ॑ सावीः ॥ वा॒मस्य॒ हि क्षय॑स्य देव॒ भूरे॑र॒या धि॒या वा॑म॒भाजः॑ स्याम ॥ उ॒प॒या॒मगृ॑हीतो-ऽसि दे॒वाय॑ त्वा सवि॒त्रे ॥ \newline

\textbf{Pada Paata} \newline

वा॒मम् । अ॒द्य । स॒वि॒तः॒ । वा॒मम् । उ॒ । श्वः । दि॒वेदि॑व॒ इति॑ दि॒वे - दि॒वे॒ । वा॒मम् । अ॒स्मभ्य॒मित्य॒स्म - भ्य॒म् । सा॒वीः॒ ॥ वा॒मस्य॑ । हि । क्षय॑स्य । दे॒व॒ । भूरेः᳚ । अ॒या । धि॒या । वा॒म॒भाज॒ इति॑ वाम - भाजः॑ । स्या॒म॒ ॥ उ॒प॒या॒मगृ॑हीत॒ इत्यु॑पया॒म - गृ॒ही॒तः॒ । अ॒सि॒ । दे॒वाय॑ । त्वा॒ । स॒वि॒त्रे ॥  \newline


\textbf{Krama Paata} \newline

वा॒मम॒द्य । अ॒द्य स॑वितः । स॒वि॒त॒र्,वा॒मम् । वा॒ममु॑ । उ॒ श्वः । श्वो दि॒वेदि॑वे । दि॒वेदि॑वे वा॒मम् । दि॒वेदि॑व॒ इति॑ दि॒वे - दि॒वे॒ । वा॒मम॒स्मभ्य᳚म् । अ॒स्मभ्यꣳ॑ सावीः । अ॒स्मभ्य॒मित्य॒स्म - भ्य॒म् । सा॒वी॒रिति॑ सावीः ॥ वा॒मस्य॒ हि । हि क्षय॑स्य । क्षय॑स्य देव । दे॒व॒ भूरेः᳚ । भूरे॑र॒या । अ॒या धि॒या । धि॒या वा॑म॒भाजः॑ । वा॒म॒भाजः॑ स्याम । वा॒म॒भाज॒ इति॑ वाम - भाजः॑ । स्या॒मेति॑ स्याम ॥ उ॒प॒या॒मगृ॑हीतोऽसि । उ॒प॒या॒मगृ॑हीत॒ इत्यु॑पया॒म - गृ॒ही॒तः॒ । अ॒सि॒ दे॒वाय॑ । दे॒वाय॑ त्वा । त्वा॒ स॒वि॒त्रे । स॒वि॒त्र इति॑ सवि॒त्रे । \newline

\textbf{Jatai Paata} \newline

1. वा॒म म॒द्याद्य वा॒मं ॅवा॒म म॒द्य । \newline
2. अ॒द्य स॑वितः सवित र॒द्याद्य स॑वितः । \newline
3. स॒वि॒त॒र् वा॒मं ॅवा॒मꣳ स॑वितः सवितर् वा॒मम् । \newline
4. वा॒म मु॑ वु वा॒मं ॅवा॒म मु॑ । \newline
5. उ॒ श्वः श्व उ॑ वु॒ श्वः । \newline
6. श्वो दि॒वेदि॑वे दि॒वेदि॑वे॒ श्वः श्वो दि॒वेदि॑वे । \newline
7. दि॒वेदि॑वे वा॒मं ॅवा॒मम् दि॒वेदि॑वे दि॒वेदि॑वे वा॒मम् । \newline
8. दि॒वेदि॑व॒ इति॑ दि॒वे - दि॒वे॒ । \newline
9. वा॒म म॒स्मभ्य॑ म॒स्मभ्यं॑ ॅवा॒मं ॅवा॒म म॒स्मभ्य᳚म् । \newline
10. अ॒स्मभ्य(ग्म्॑) सावीः सावी र॒स्मभ्य॑ म॒स्मभ्य(ग्म्॑) सावीः । \newline
11. अ॒स्मभ्य॒मित्य॒स्म - भ्य॒म् । \newline
12. सा॒वी॒रिति॑ सावीः । \newline
13. वा॒मस्य॒ हि हि वा॒मस्य॑ वा॒मस्य॒ हि । \newline
14. हि क्षय॑स्य॒ क्षय॑स्य॒ हि हि क्षय॑स्य । \newline
15. क्षय॑स्य देव देव॒ क्षय॑स्य॒ क्षय॑स्य देव । \newline
16. दे॒व॒ भूरे॒र् भूरे᳚र् देव देव॒ भूरेः᳚ । \newline
17. भूरे॑र॒या ऽया भूरे॒र् भूरे॑र॒या । \newline
18. अ॒या धि॒या धि॒या ऽया ऽया धि॒या । \newline
19. धि॒या वा॑म॒भाजो॑ वाम॒भाजो॑ धि॒या धि॒या वा॑म॒भाजः॑ । \newline
20. वा॒म॒भाजः॑ स्याम स्याम वाम॒भाजो॑ वाम॒भाजः॑ स्याम । \newline
21. वा॒म॒भाज॒ इति॑ वाम - भाजः॑ । \newline
22. स्या॒मेति॑ स्याम । \newline
23. उ॒प॒या॒मगृ॑हीतो ऽस्यस्युपया॒मगृ॑हीत उपया॒मगृ॑हीतो ऽसि । \newline
24. उ॒प॒या॒मगृ॑हीत॒ इत्यु॑पया॒म - गृ॒ही॒तः॒ । \newline
25. अ॒सि॒ दे॒वाय॑ दे॒वाया᳚स्यसि दे॒वाय॑ । \newline
26. दे॒वाय॑ त्वा त्वा दे॒वाय॑ दे॒वाय॑ त्वा । \newline
27. त्वा॒ स॒वि॒त्रे स॑वि॒त्रे त्वा᳚ त्वा सवि॒त्रे । \newline
28. स॒वि॒त्र इति॑ सवि॒त्रे । \newline

\textbf{Ghana Paata } \newline

1. वा॒म म॒द्याद्य वा॒मं ॅवा॒म म॒द्य स॑वितः सवित र॒द्य वा॒मं ॅवा॒म म॒द्य स॑वितः । \newline
2. अ॒द्य स॑वितः सवित र॒द्याद्य स॑वितर् वा॒मं ॅवा॒मꣳ स॑वित र॒द्याद्य स॑वितर् वा॒मम् । \newline
3. स॒वि॒त॒र् वा॒मं ॅवा॒मꣳ स॑वितः सवितर् वा॒म मु॑ वु वा॒मꣳ स॑वितः सवितर् वा॒म मु॑ । \newline
4. वा॒म मु॑ वु वा॒मं ॅवा॒म मु॒ श्वः श्व उ॑ वा॒मं ॅवा॒म मु॒ श्वः । \newline
5. उ॒ श्वः श्व उ॑ वु॒ श्वो दि॒वेदि॑वे दि॒वेदि॑वे॒ श्व उ॑ वु॒ श्वो दि॒वेदि॑वे । \newline
6. श्वो दि॒वेदि॑वे दि॒वेदि॑वे॒ श्वः श्वो दि॒वेदि॑वे वा॒मं ॅवा॒मम् दि॒वेदि॑वे॒ श्वः श्वो दि॒वेदि॑वे वा॒मम् । \newline
7. दि॒वेदि॑वे वा॒मं ॅवा॒मम् दि॒वेदि॑वे दि॒वेदि॑वे वा॒म म॒स्मभ्य॑ म॒स्मभ्यं॑ ॅवा॒मम् दि॒वेदि॑वे दि॒वेदि॑वे वा॒म म॒स्मभ्य᳚म् । \newline
8. दि॒वेदि॑व॒ इति॑ दि॒वे - दि॒वे॒ । \newline
9. वा॒म म॒स्मभ्य॑ म॒स्मभ्यं॑ ॅवा॒मं ॅवा॒म म॒स्मभ्य(ग्म्॑) सावीः सावी र॒स्मभ्यं॑ ॅवा॒मं ॅवा॒म म॒स्मभ्य(ग्म्॑) सावीः । \newline
10. अ॒स्मभ्य(ग्म्॑) सावीः सावी र॒स्मभ्य॑ म॒स्मभ्य(ग्म्॑) सावीः । \newline
11. अ॒स्मभ्य॒मित्य॒स्म - भ्य॒म् । \newline
12. सा॒वी॒रिति॑ सावीः । \newline
13. वा॒मस्य॒ हि हि वा॒मस्य॑ वा॒मस्य॒ हि क्षय॑स्य॒ क्षय॑स्य॒ हि वा॒मस्य॑ वा॒मस्य॒ हि क्षय॑स्य । \newline
14. हि क्षय॑स्य॒ क्षय॑स्य॒ हि हि क्षय॑स्य देव देव॒ क्षय॑स्य॒ हि हि क्षय॑स्य देव । \newline
15. क्षय॑स्य देव देव॒ क्षय॑स्य॒ क्षय॑स्य देव॒ भूरे॒र् भूरे᳚र् देव॒ क्षय॑स्य॒ क्षय॑स्य देव॒ भूरेः᳚ । \newline
16. दे॒व॒ भूरे॒र् भूरे᳚र् देव देव॒ भूरे॑र॒या ऽया भूरे᳚र् देव देव॒ भूरे॑र॒या । \newline
17. भूरे॑र॒या ऽया भूरे॒र् भूरे॑र॒या धि॒या धि॒या ऽया भूरे॒र् भूरे॑र॒या धि॒या । \newline
18. अ॒या धि॒या धि॒या ऽया ऽया धि॒या वा॑म॒भाजो॑ वाम॒भाजो॑ धि॒या ऽया ऽया धि॒या वा॑म॒भाजः॑ । \newline
19. धि॒या वा॑म॒भाजो॑ वाम॒भाजो॑ धि॒या धि॒या वा॑म॒भाजः॑ स्याम स्याम वाम॒भाजो॑ धि॒या धि॒या वा॑म॒भाजः॑ स्याम । \newline
20. वा॒म॒भाजः॑ स्याम स्याम वाम॒भाजो॑ वाम॒भाजः॑ स्याम । \newline
21. वा॒म॒भाज॒ इति॑ वाम - भाजः॑ । \newline
22. स्या॒मेति॑ स्याम । \newline
23. उ॒प॒या॒मगृ॑हीतो ऽस्यस्युपया॒मगृ॑हीत उपया॒मगृ॑हीतो ऽसि दे॒वाय॑ दे॒वा या᳚स्युपया॒मगृ॑हीत उपया॒मगृ॑हीतो ऽसि दे॒वाय॑ । \newline
24. उ॒प॒या॒मगृ॑हीत॒ इत्यु॑पया॒म - गृ॒ही॒तः॒ । \newline
25. अ॒सि॒ दे॒वाय॑ दे॒वा या᳚स्यसि दे॒वाय॑ त्वा त्वा दे॒वा या᳚स्यसि दे॒वाय॑ त्वा । \newline
26. दे॒वाय॑ त्वा त्वा दे॒वाय॑ दे॒वाय॑ त्वा सवि॒त्रे स॑वि॒त्रे त्वा॑ दे॒वाय॑ दे॒वाय॑ त्वा सवि॒त्रे । \newline
27. त्वा॒ स॒वि॒त्रे स॑वि॒त्रे त्वा᳚ त्वा सवि॒त्रे । \newline
28. स॒वि॒त्र इति॑ सवि॒त्रे । \newline
\pagebreak
\markright{ TS 1.4.24.1  \hfill https://www.vedavms.in \hfill}
\addcontentsline{toc}{section}{ TS 1.4.24.1 }
\section*{ TS 1.4.24.1 }

\textbf{TS 1.4.24.1 } \newline
\textbf{Samhita Paata} \newline

अद॑ब्धेभिः सवितः पा॒युभि॒ष्ट्वꣳ शि॒वेभि॑र॒द्य परि॑पाहि नो॒ गयं᳚ । हिर॑ण्यजिह्वः सुवि॒ताय॒ नव्य॑से॒ रक्षा॒ माकि॑र्नो अ॒घशꣳ॑स ईशत ॥ उ॒प॒या॒मगृ॑हीतोऽसि दे॒वाय॑ त्वा सवि॒त्रे ॥ \newline

\textbf{Pada Paata} \newline

अद॑ब्धेभिः । स॒वि॒तः॒ । पा॒युभि॒रिति॑ पा॒यु - भिः॒ । त्वम् । शि॒वेभिः॑ । अ॒द्य । परीति॑ । पा॒हि॒ । नः॒ । गय᳚म् ॥ हिर॑ण्यजिह्व॒ इति॒ हिर॑ण्य - जि॒ह्वः॒ । सु॒वि॒ताय॑ । नव्य॑से । रक्ष॑ । माकिः॑ । नः॒ । अ॒घशꣳ॑स॒ इत्य॒घ - शꣳ॒॒सः॒ । ई॒श॒त॒ ॥ उ॒प॒या॒मगृ॑हीत॒ इत्यु॑पया॒म - गृ॒ही॒तः॒ । अ॒सि॒ । दे॒वाय॑ । त्वा॒ । स॒वि॒त्रे ॥  \newline


\textbf{Krama Paata} \newline

अद॑ब्धेभिः सवितः । स॒वि॒तः॒ पा॒युभिः॑ । पा॒युभि॒ष्ट्वम् । पा॒युभि॒रिति॑ पा॒यु - भिः॒ । त्वꣳ शि॒वेभिः॑ । शि॒वेभि॑र॒द्य । अ॒द्य परि॑ । परि॑ पाहि । पा॒हि॒ नः॒ । नो॒ गय᳚म् । गय॒मिति॒ गय᳚म् । हिर॑ण्यजिह्वः सुवि॒ताय॑ । हिर॑ण्यजिह्व॒ इति॒ हिर॑ण्य - जि॒ह्वः॒ । सु॒वि॒ताय॒ नव्य॑से । नव्य॑से॒ रक्ष॑ । रक्षा॒ माकिः॑ । माकि॑र्नः । नो॒ अ॒घशꣳ॑सः । अ॒घशꣳ॑स ईशत । अ॒घशꣳ॑स॒ इत्य॒घ - शꣳ॒॒सः॒ । ई॒श॒तेती॑शत ॥ उ॒प॒या॒मगृ॑हीतोऽसि । उ॒प॒या॒मगृ॑हीत॒ इत्यु॑पया॒म - गृ॒ही॒तः॒ । अ॒सि॒ दे॒वाय॑ । दे॒वाय॑ त्वा । त्वा॒ स॒वि॒त्रे । स॒वि॒त्र इति॑ सवि॒त्रे । \newline

\textbf{Jatai Paata} \newline

1. अद॑ब्धेभिः सवितः सवित॒ रद॑ब्धेभि॒ रद॑ब्धेभिः सवितः । \newline
2. स॒वि॒तः॒ पा॒युभिः॑ पा॒युभिः॑ सवितः सवितः पा॒युभिः॑ । \newline
3. पा॒युभि॒ ष्ट्वम् त्वम् पा॒युभिः॑ पा॒युभि॒ ष्ट्वम् । \newline
4. पा॒युभि॒रिति॑ पा॒यु - भिः॒ । \newline
5. त्वꣳ शि॒वेभिः॑ शि॒वेभि॒ स्त्वम् त्वꣳ शि॒वेभिः॑ । \newline
6. शि॒वेभि॑र॒द्याद्य शि॒वेभिः॑ शि॒वेभि॑र॒द्य । \newline
7. अ॒द्य परि॒ पर्य॒द्याद्य परि॑ । \newline
8. परि॑ पाहि पाहि॒ परि॒ परि॑ पाहि । \newline
9. पा॒हि॒ नो॒ नः॒ पा॒हि॒ पा॒हि॒ नः॒ । \newline
10. नो॒ गय॒म् गय॑न्नो नो॒ गय᳚म् । \newline
11. गय॒मिति॒ गय᳚म् । \newline
12. हिर॑ण्यजिह्वः सुवि॒ताय॑ सुवि॒ताय॒ हिर॑ण्यजिह्वो॒ हिर॑ण्यजिह्वः सुवि॒ताय॑ । \newline
13. हिर॑ण्यजिह्व॒ इति॒ हिर॑ण्य - जि॒ह्वः॒ । \newline
14. सु॒वि॒ताय॒ नव्य॑से॒ नव्य॑से सुवि॒ताय॑ सुवि॒ताय॒ नव्य॑से । \newline
15. नव्य॑से॒ रक्ष॒ रक्ष॒ नव्य॑से॒ नव्य॑से॒ रक्ष॑ । \newline
16. रक्षा॒ माकि॒र् माकी॒ रक्ष॒ रक्षा॒ माकिः॑ । \newline
17. माकि॑र् नो नो॒ माकि॒र् माकि॑र् नः । \newline
18. नो॒ अ॒घश(ग्म्॑)सो॒ ऽघश(ग्म्॑)सो नो नो अ॒घश(ग्म्॑)सः । \newline
19. अ॒घश(ग्म्॑)स ईशतेशता॒घश(ग्म्॑)सो॒ ऽघश(ग्म्॑)स ईशत । \newline
20. अ॒घश(ग्म्॑)स॒ इत्य॒घ - श॒(ग्म्॒)सः॒ । \newline
21. ई॒श॒तेती॑शत । \newline
22. उ॒प॒या॒मगृ॑हीतो ऽस्यस्युपया॒मगृ॑हीत उपया॒मगृ॑हीतो ऽसि । \newline
23. उ॒प॒या॒मगृ॑हीत॒ इत्यु॑पया॒म - गृ॒ही॒तः॒ । \newline
24. अ॒सि॒ दे॒वाय॑ दे॒वाया᳚स्यसि दे॒वाय॑ । \newline
25. दे॒वाय॑ त्वा त्वा दे॒वाय॑ दे॒वाय॑ त्वा । \newline
26. त्वा॒ स॒वि॒त्रे स॑वि॒त्रे त्वा᳚ त्वा सवि॒त्रे । \newline
27. स॒वि॒त्र इति॑ सवि॒त्रे । \newline

\textbf{Ghana Paata } \newline

1. अद॑ब्धेभिः सवितः सवित॒ रद॑ब्धेभि॒ रद॑ब्धेभिः सवितः पा॒युभिः॑ पा॒युभिः॑ सवित॒ रद॑ब्धेभि॒ रद॑ब्धेभिः सवितः पा॒युभिः॑ । \newline
2. स॒वि॒तः॒ पा॒युभिः॑ पा॒युभिः॑ सवितः सवितः पा॒युभि॒ ष्ट्वम् त्वम् पा॒युभिः॑ सवितः सवितः पा॒युभि॒ ष्ट्वम् । \newline
3. पा॒युभि॒ ष्ट्वम् त्वम् पा॒युभिः॑ पा॒युभि॒ ष्ट्वꣳ शि॒वेभिः॑ शि॒वेभि॒ स्त्वम् पा॒युभिः॑ पा॒युभि॒ ष्ट्वꣳ शि॒वेभिः॑ । \newline
4. पा॒युभि॒रिति॑ पा॒यु - भिः॒ । \newline
5. त्वꣳ शि॒वेभिः॑ शि॒वेभि॒ स्त्वम् त्वꣳ शि॒वेभि॑ र॒द्याद्य शि॒वेभि॒ स्त्वम् त्वꣳ शि॒वेभि॑र॒द्य । \newline
6. शि॒वेभि॑ र॒द्याद्य शि॒वेभिः॑ शि॒वेभि॑ र॒द्य परि॒ पर्य॒द्य शि॒वेभिः॑ शि॒वेभि॑ र॒द्य परि॑ । \newline
7. अ॒द्य परि॒ पर्य॒द्याद्य परि॑ पाहि पाहि॒ पर्य॒द्याद्य परि॑ पाहि । \newline
8. परि॑ पाहि पाहि॒ परि॒ परि॑ पाहि नो नः पाहि॒ परि॒ परि॑ पाहि नः । \newline
9. पा॒हि॒ नो॒ नः॒ पा॒हि॒ पा॒हि॒ नो॒ गय॒म् गय॑म् नः पाहि पाहि नो॒ गय᳚म् । \newline
10. नो॒ गय॒म् गय॑म् नो नो॒ गय᳚म् । \newline
11. गय॒मिति॒ गय᳚म् । \newline
12. हिर॑ण्यजिह्वः सुवि॒ताय॑ सुवि॒ताय॒ हिर॑ण्यजिह्वो॒ हिर॑ण्यजिह्वः सुवि॒ताय॒ नव्य॑से॒ नव्य॑से सुवि॒ताय॒ हिर॑ण्यजिह्वो॒ हिर॑ण्यजिह्वः सुवि॒ताय॒ नव्य॑से । \newline
13. हिर॑ण्यजिह्व॒ इति॒ हिर॑ण्य - जि॒ह्वः॒ । \newline
14. सु॒वि॒ताय॒ नव्य॑से॒ नव्य॑से सुवि॒ताय॑ सुवि॒ताय॒ नव्य॑से॒ रक्ष॒ रक्ष॒ नव्य॑से सुवि॒ताय॑ सुवि॒ताय॒ नव्य॑से॒ रक्ष॑ । \newline
15. नव्य॑से॒ रक्ष॒ रक्ष॒ नव्य॑से॒ नव्य॑से॒ रक्षा॒ माकि॒र् माकी॒ रक्ष॒ नव्य॑से॒ नव्य॑से॒ रक्षा॒ माकिः॑ । \newline
16. रक्षा॒ माकि॒र् माकी॒ रक्ष॒ रक्षा॒ माकि॑र् नो नो॒ माकी॒ रक्ष॒ रक्षा॒ माकि॑र् नः । \newline
17. माकि॑र् नो नो॒ माकि॒र् माकि॑र् नो अ॒घश(ग्म्॑)सो॒ ऽघश(ग्म्॑)सो नो॒ माकि॒र् माकि॑र् नो अ॒घश(ग्म्॑)सः । \newline
18. नो॒ अ॒घश(ग्म्॑)सो॒ ऽघश(ग्म्॑)सो नो नो अ॒घश(ग्म्॑)स ईश तेशता॒घश(ग्म्॑)सो नो नो अ॒घश(ग्म्॑)स ईशत । \newline
19. अ॒घश(ग्म्॑)स ईश तेशता॒घश(ग्म्॑)सो॒ ऽघश(ग्म्॑)स ईशत । \newline
20. अ॒घश(ग्म्॑)स॒ इत्य॒घ - शꣳ॒॒सः॒ । \newline
21. ई॒श॒तेती॑शत । \newline
22. उ॒प॒या॒मगृ॑हीतो ऽस्यस्युपया॒मगृ॑हीत उपया॒मगृ॑हीतो ऽसि दे॒वाय॑ दे॒वा या᳚स्युपया॒मगृ॑हीत उपया॒मगृ॑हीतो ऽसि दे॒वाय॑ । \newline
23. उ॒प॒या॒मगृ॑हीत॒ इत्यु॑पया॒म - गृ॒ही॒तः॒ । \newline
24. अ॒सि॒ दे॒वाय॑ दे॒वाया᳚स्यसि दे॒वाय॑ त्वा त्वा दे॒वाया᳚स्यसि दे॒वाय॑ त्वा । \newline
25. दे॒वाय॑ त्वा त्वा दे॒वाय॑ दे॒वाय॑ त्वा सवि॒त्रे स॑वि॒त्रे त्वा॑ दे॒वाय॑ दे॒वाय॑ त्वा सवि॒त्रे । \newline
26. त्वा॒ स॒वि॒त्रे स॑वि॒त्रे त्वा᳚ त्वा सवि॒त्रे । \newline
27. स॒वि॒त्र इति॑ सवि॒त्रे । \newline
\pagebreak
\markright{ TS 1.4.25.1  \hfill https://www.vedavms.in \hfill}
\addcontentsline{toc}{section}{ TS 1.4.25.1 }
\section*{ TS 1.4.25.1 }

\textbf{TS 1.4.25.1 } \newline
\textbf{Samhita Paata} \newline

हिर॑ण्यपाणिमू॒तये॑ सवि॒तार॒मुप॑ ह्वये । स चेत्ता॑ दे॒वता॑ प॒दं ॥ उ॒प॒या॒मगृ॑हीतो-ऽसि दे॒वाय॑ त्वा सवि॒त्रे ॥ \newline

\textbf{Pada Paata} \newline

हिर॑ण्यपाणि॒मिति॒ हिर॑ण्य - पा॒णि॒म् । ऊ॒तये᳚ । स॒वि॒तार᳚म् । उपेति॑ । ह्व॒ये॒ ॥ सः । चेत्ता᳚ । दे॒वता᳚ । प॒दम् ॥ उ॒प॒या॒मगृ॑हीत॒ इत्यु॑पया॒म - गृ॒ही॒तः॒ । अ॒सि॒ । दे॒वाय॑ । त्वा॒ । स॒वि॒त्रे ॥  \newline


\textbf{Krama Paata} \newline

हिर॑ण्यपाणिमू॒तये᳚ । हिर॑ण्यपाणि॒मिति॒ हिर॑ण्य - पा॒णि॒म् । ऊ॒तये॑ सवि॒तार᳚म् । स॒वि॒तार॒मुप॑ । उप॑ ह्वये । ह्व॒य॒ इति॑ ह्वये ॥ स चेत्ता᳚ । चेत्ता॑ दे॒वता᳚ । दे॒वता॑ प॒दम् । प॒दमिति॑ प॒दम् ॥ उ॒प॒या॒मगृ॑हीतोऽसि । उ॒प॒या॒मगृ॑हीत॒ इत्यु॑पया॒म - गृ॒ही॒तः॒ । अ॒सि॒ दे॒वाय॑ । दे॒वाय॑ त्वा । त्वा॒ स॒वि॒त्रे । स॒वि॒त्र इति॑ सवि॒त्रे । \newline

\textbf{Jatai Paata} \newline

1. हिर॑ण्यपाणि मू॒तय॑ ऊ॒तये॒ हिर॑ण्यपाणि॒(ग्म्॒) हिर॑ण्यपाणि मू॒तये᳚ । \newline
2. हिर॑ण्यपाणि॒मिति॒ हिर॑ण्य - पा॒णि॒म् । \newline
3. ऊ॒तये॑ सवि॒तार(ग्म्॑) सवि॒तार॑ मू॒तय॑ ऊ॒तये॑ सवि॒तार᳚म् । \newline
4. स॒वि॒तार॒ मुपोप॑ सवि॒तार(ग्म्॑) सवि॒तार॒ मुप॑ । \newline
5. उप॑ ह्वये ह्वय॒ उपोप॑ ह्वये । \newline
6. ह्व॒य॒ इति॑ ह्वये । \newline
7. स चेत्ता॒ चेत्ता॒ स स चेत्ता᳚ । \newline
8. चेत्ता॑ दे॒वता॑ दे॒वता॒ चेत्ता॒ चेत्ता॑ दे॒वता᳚ । \newline
9. दे॒वता॑ प॒दम् प॒दम् दे॒वता॑ दे॒वता॑ प॒दम् । \newline
10. प॒दमिति॑ प॒दम् । \newline
11. उ॒प॒या॒मगृ॑हीतो ऽस्यस्युपया॒मगृ॑हीत उपया॒मगृ॑हीतो ऽसि । \newline
12. उ॒प॒या॒मगृ॑हीत॒ इत्यु॑पया॒म - गृ॒ही॒तः॒ । \newline
13. अ॒सि॒ दे॒वाय॑ दे॒वाया᳚स्यसि दे॒वाय॑ । \newline
14. दे॒वाय॑ त्वा त्वा दे॒वाय॑ दे॒वाय॑ त्वा । \newline
15. त्वा॒ स॒वि॒त्रे स॑वि॒त्रे त्वा᳚ त्वा सवि॒त्रे । \newline
16. स॒वि॒त्र इति॑ सवि॒त्रे । \newline

\textbf{Ghana Paata } \newline

1. हिर॑ण्यपाणि मू॒तय॑ ऊ॒तये॒ हिर॑ण्यपाणिꣳ॒॒ हिर॑ण्यपाणि मू॒तये॑ सवि॒तार(ग्म्॑) सवि॒तार॑ मू॒तये॒ हिर॑ण्यपाणिꣳ॒॒ हिर॑ण्यपाणि मू॒तये॑ सवि॒तार᳚म् । \newline
2. हिर॑ण्यपाणि॒मिति॒ हिर॑ण्य - पा॒णि॒म् । \newline
3. ऊ॒तये॑ सवि॒तार(ग्म्॑) सवि॒तार॑ मू॒तय॑ ऊ॒तये॑ सवि॒तार॒ मुपोप॑ सवि॒तार॑ मू॒तय॑ ऊ॒तये॑ सवि॒तार॒ मुप॑ । \newline
4. स॒वि॒तार॒ मुपोप॑ सवि॒तार(ग्म्॑) सवि॒तार॒ मुप॑ ह्वये ह्वय॒ उप॑ सवि॒तार(ग्म्॑) सवि॒तार॒ मुप॑ ह्वये । \newline
5. उप॑ ह्वये ह्वय॒ उपोप॑ ह्वये । \newline
6. ह्व॒य॒ इति॑ ह्वये । \newline
7. स चेत्ता॒ चेत्ता॒ स स चेत्ता॑ दे॒वता॑ दे॒वता॒ चेत्ता॒ स स चेत्ता॑ दे॒वता᳚ । \newline
8. चेत्ता॑ दे॒वता॑ दे॒वता॒ चेत्ता॒ चेत्ता॑ दे॒वता॑ प॒दम् प॒दम् दे॒वता॒ चेत्ता॒ चेत्ता॑ दे॒वता॑ प॒दम् । \newline
9. दे॒वता॑ प॒दम् प॒दम् दे॒वता॑ दे॒वता॑ प॒दम् । \newline
10. प॒दमिति॑ प॒दम् । \newline
11. उ॒प॒या॒मगृ॑हीतो ऽस्यस्युपया॒मगृ॑हीत उपया॒मगृ॑हीतो ऽसि दे॒वाय॑ दे॒वा या᳚स्युपया॒मगृ॑हीत उपया॒मगृ॑हीतो ऽसि दे॒वाय॑ । \newline
12. उ॒प॒या॒मगृ॑हीत॒ इत्यु॑पया॒म - गृ॒ही॒तः॒ । \newline
13. अ॒सि॒ दे॒वाय॑ दे॒वाया᳚स्यसि दे॒वाय॑ त्वा त्वा दे॒वाया᳚स्यसि दे॒वाय॑ त्वा । \newline
14. दे॒वाय॑ त्वा त्वा दे॒वाय॑ दे॒वाय॑ त्वा सवि॒त्रे स॑वि॒त्रे त्वा॑ दे॒वाय॑ दे॒वाय॑ त्वा सवि॒त्रे । \newline
15. त्वा॒ स॒वि॒त्रे स॑वि॒त्रे त्वा᳚ त्वा सवि॒त्रे । \newline
16. स॒वि॒त्र इति॑ सवि॒त्रे । \newline
\pagebreak
\markright{ TS 1.4.26.1  \hfill https://www.vedavms.in \hfill}
\addcontentsline{toc}{section}{ TS 1.4.26.1 }
\section*{ TS 1.4.26.1 }

\textbf{TS 1.4.26.1 } \newline
\textbf{Samhita Paata} \newline

सु॒शर्मा॑ऽसि सुप्रतिष्ठा॒नो बृ॒हदु॒क्षे नम॑ ए॒ष ते॒ योनि॒र् विश्वे᳚भ्यस्त्वा दे॒वेभ्यः॑ ॥ \newline

\textbf{Pada Paata} \newline

सु॒शर्मेति॑ सु - शर्मा᳚ । अ॒सि॒ । सु॒प्र॒ति॒ष्ठा॒न इति॑ सु - प्र॒ति॒ष्ठा॒नः । बृ॒हत् । उ॒क्षे । नमः॑ । ए॒षः । ते॒ । योनिः॑ । विश्वे᳚भ्यः । त्वा॒ । दे॒वेभ्यः॑ ॥  \newline


\textbf{Krama Paata} \newline

सु॒शर्मा॑ऽसि । सु॒शर्.मेति॑ सु - शर्मा᳚ । अ॒सि॒ सु॒प्र॒ति॒ष्ठा॒नः । सु॒प्र॒ति॒ष्ठा॒नो बृ॒हत् । सु॒प्र॒ति॒ष्ठा॒न इति॑ सु - प्र॒ति॒ष्ठा॒नः । बृ॒हदु॒क्षे । उ॒क्षे नमः॑ । नम॑ ए॒षः । ए॒ष ते᳚ । ते॒ योनिः॑ । योनि॒र् विश्वे᳚भ्यः । विश्वे᳚भ्यस्त्वा । त्वा॒ दे॒वेभ्यः॑ । दे॒वेभ्य॒ इति॑ दे॒वेभ्यः॑ । \newline

\textbf{Jatai Paata} \newline

1. सु॒शर्मा᳚ ऽस्यसि सु॒शर्मा॑ सु॒शर्मा॑ ऽसि । \newline
2. सु॒शर्मेति॑ सु - शर्मा᳚ । \newline
3. अ॒सि॒ सु॒प्र॒ति॒ष्ठा॒नः सु॑प्रतिष्ठा॒नो᳚ ऽस्यसि सुप्रतिष्ठा॒नः । \newline
4. सु॒प्र॒ति॒ष्ठा॒नो बृ॒हद् बृ॒हथ् सु॑प्रतिष्ठा॒नः सु॑प्रतिष्ठा॒नो बृ॒हत् । \newline
5. सु॒प्र॒ति॒ष्ठा॒न इति॑ सु - प्र॒ति॒ष्ठा॒नः । \newline
6. बृ॒हदु॒क्ष उ॒क्षे बृ॒हद् बृ॒हदु॒क्षे । \newline
7. उ॒क्षे नमो॒ नम॑ उ॒क्ष उ॒क्षे नमः॑ । \newline
8. नम॑ ए॒ष ए॒ष नमो॒ नम॑ ए॒षः । \newline
9. ए॒ष ते॑ त ए॒ष ए॒ष ते᳚ । \newline
10. ते॒ योनि॒र् योनि॑ स्ते ते॒ योनिः॑ । \newline
11. योनि॒र् विश्वे᳚भ्यो॒ विश्वे᳚भ्यो॒ योनि॒र् योनि॒र् विश्वे᳚भ्यः । \newline
12. विश्वे᳚भ्य स्त्वा त्वा॒ विश्वे᳚भ्यो॒ विश्वे᳚भ्य स्त्वा । \newline
13. त्वा॒ दे॒वेभ्यो॑ दे॒वेभ्य॑ स्त्वा त्वा दे॒वेभ्यः॑ । \newline
14. दे॒वेभ्य॒ इति॑ दे॒वेभ्यः॑ । \newline

\textbf{Ghana Paata } \newline

1. सु॒शर्मा᳚ ऽस्यसि सु॒शर्मा॑ सु॒शर्मा॑ ऽसि सुप्रतिष्ठा॒नः सु॑प्रतिष्ठा॒नो॑ ऽसि सु॒शर्मा॑ सु॒शर्मा॑ ऽसि सुप्रतिष्ठा॒नः । \newline
2. सु॒शर्मेति॑ सु - शर्मा᳚ । \newline
3. अ॒सि॒ सु॒प्र॒ति॒ष्ठा॒नः सु॑प्रतिष्ठा॒नो᳚ ऽस्यसि सुप्रतिष्ठा॒नो बृ॒हद् बृ॒हथ् सु॑प्रतिष्ठा॒नो᳚ ऽस्यसि सुप्रतिष्ठा॒नो बृ॒हत् । \newline
4. सु॒प्र॒ति॒ष्ठा॒नो बृ॒हद् बृ॒हथ् सु॑प्रतिष्ठा॒नः सु॑प्रतिष्ठा॒नो बृ॒हदु॒क्ष उ॒क्षे बृ॒हथ् सु॑प्रतिष्ठा॒नः सु॑प्रतिष्ठा॒नो बृ॒हदु॒क्षे । \newline
5. सु॒प्र॒ति॒ष्ठा॒न इति॑ सु - प्र॒ति॒ष्ठा॒नः । \newline
6. बृ॒हदु॒क्ष उ॒क्षे बृ॒हद् बृ॒हदु॒क्षे नमो॒ नम॑ उ॒क्षे बृ॒हद् बृ॒हदु॒क्षे नमः॑ । \newline
7. उ॒क्षे नमो॒ नम॑ उ॒क्ष उ॒क्षे नम॑ ए॒ष ए॒ष नम॑ उ॒क्ष उ॒क्षे नम॑ ए॒षः । \newline
8. नम॑ ए॒ष ए॒ष नमो॒ नम॑ ए॒ष ते॑ त ए॒ष नमो॒ नम॑ ए॒ष ते᳚ । \newline
9. ए॒ष ते॑ त ए॒ष ए॒ष ते॒ योनि॒र् योनि॑ स्त ए॒ष ए॒ष ते॒ योनिः॑ । \newline
10. ते॒ योनि॒र् योनि॑ स्ते ते॒ योनि॒र् विश्वे᳚भ्यो॒ विश्वे᳚भ्यो॒ योनि॑ स्ते ते॒ योनि॒र् विश्वे᳚भ्यः । \newline
11. योनि॒र् विश्वे᳚भ्यो॒ विश्वे᳚भ्यो॒ योनि॒र् योनि॒र् विश्वे᳚भ्य स्त्वा त्वा॒ विश्वे᳚भ्यो॒ योनि॒र् योनि॒र् विश्वे᳚भ्य स्त्वा । \newline
12. विश्वे᳚भ्य स्त्वा त्वा॒ विश्वे᳚भ्यो॒ विश्वे᳚भ्य स्त्वा दे॒वेभ्यो॑ दे॒वेभ्य॑ स्त्वा॒ विश्वे᳚भ्यो॒ विश्वे᳚भ्य स्त्वा दे॒वेभ्यः॑ । \newline
13. त्वा॒ दे॒वेभ्यो॑ दे॒वेभ्य॑ स्त्वा त्वा दे॒वेभ्यः॑ । \newline
14. दे॒वेभ्य॒ इति॑ दे॒वेभ्यः॑ । \newline
\pagebreak
\markright{ TS 1.4.27.1  \hfill https://www.vedavms.in \hfill}
\addcontentsline{toc}{section}{ TS 1.4.27.1 }
\section*{ TS 1.4.27.1 }

\textbf{TS 1.4.27.1 } \newline
\textbf{Samhita Paata} \newline

बृह॒स्पति॑सुतस्य त इन्दो इन्द्रि॒याव॑तः॒ पत्नी॑वन्तं॒ ग्रहं॑ गृह्णा॒म्यग्ना(3)इ पत्नी॒वा(3) स्स॒जूर्दे॒वेन॒ त्वष्ट्रा॒ सोमं॑ पिब॒ स्वाहा᳚ ॥ \newline

\textbf{Pada Paata} \newline

बृह॒स्पति॑सुत॒स्येति॒ बृह॒स्पति॑ - सु॒त॒स्य॒ । ते॒ । इ॒न्दो॒ इति॑ । इ॒न्द्रि॒याव॑त॒ इती᳚न्द्रि॒य - व॒तः॒ । पत्नी॑वन्त॒मिति॒ पत्नी᳚ - व॒न्त॒म् । ग्रह᳚म् । गृ॒ह्णा॒मि॒ । अग्ना(3) इ । पत्नी॒वा(3) इति॒ पत्नी᳚ - वा(3)ः । स॒जूरिति॑ स - जूः । दे॒वेन॑ । त्वष्ट्रा᳚ । सोम᳚म् । पि॒ब॒ । स्वाहा᳚ ॥  \newline


\textbf{Krama Paata} \newline

बृह॒स्पति॑सुतस्य ते । बृह॒स्पति॑सुत॒स्येति॒ बृह॒स्पति॑ - सु॒त॒स्य॒ । त॒ इ॒न्दो॒ । इ॒न्दो॒ इ॒न्द्रि॒याव॑तः । इ॒न्दो॒ इती᳚न्दो । इ॒न्द्रि॒याव॑तः॒ पत्नी॑वन्तम् । इ॒न्द्रि॒याव॑त॒ इती᳚न्द्रि॒य - व॒तः॒ । पत्नी॑वन्त॒म् ग्रह᳚म् । पत्नी॑वन्त॒मिति॒ पत्नी᳚ - व॒न्त॒म् । गह॑म् गृह्णामि । गृ॒ह्णा॒म्यग्ना(3)इ । अग्ना(3)इ पत्नी॒वा(3)ः । पत्नी॒वा(3)ः स॒जूः । पत्नी॒वा(3) इति॒ पत्नी᳚ - वा(3)ः । स॒जूर् दे॒वेन॑ । स॒जूरिति॑ स - जूः । दे॒वेन॒ त्वष्ट्रा᳚ । त्वष्ट्रा॒ सोम᳚म् । सोम॑म् पिब । पि॒ब॒ स्वाहा᳚ । स्वाहेति॒ स्वाहा᳚ । \newline

\textbf{Jatai Paata} \newline

1. बृह॒स्पति॑सुतस्य ते ते॒ बृह॒स्पति॑सुतस्य॒ बृह॒स्पति॑सुतस्य ते । \newline
2. बृह॒स्पति॑सुत॒स्येति॒ बृह॒स्पति॑ - सु॒त॒स्य॒ । \newline
3. त॒ इ॒न्दो॒ इ॒न्दो॒ ते॒ त॒ इ॒न्दो॒ । \newline
4. इ॒न्दो॒ इ॒न्द्रि॒याव॑त इन्द्रि॒याव॑त इन्दो इन्दो इन्द्रि॒याव॑तः । \newline
5. इ॒न्दो॒ इती᳚न्दो । \newline
6. इ॒न्द्रि॒याव॑तः॒ पत्नी॑वन्त॒म् पत्नी॑वन्त मिन्द्रि॒याव॑त इन्द्रि॒याव॑तः॒ पत्नी॑वन्तम् । \newline
7. इ॒न्द्रि॒याव॑त॒ इती᳚न्द्रि॒य - व॒तः॒ । \newline
8. पत्नी॑वन्त॒म् ग्रह॒म् ग्रह॒म् पत्नी॑वन्त॒म् पत्नी॑वन्त॒म् ग्रह᳚म् । \newline
9. पत्नी॑वन्त॒मिति॒ पत्नी᳚ - व॒न्त॒म् । \newline
10. ग्रह॑म् गृह्णामि गृह्णामि॒ ग्रह॒म् ग्रह॑म् गृह्णामि । \newline
11. गृ॒ह्णा॒म्यग्ना(3) अग्ना(3)इ गृ॑ह्णामि गृह्णा॒म्यग्ना(3)इ । \newline
12. अग्ना(3)इ पत्नी॒वा(3)ः पत्नी॒वा(3) अग्ना(3) अग्ना(3)इ पत्नी॒वा(3)ः । \newline
13. पत्नी॒वा(3)ः स॒जूः स॒जूः पत्नी॒वा(3)ः पत्नी॒वा(3)ः स॒जूः । \newline
14. पत्नी॒वा(3) इति॒ पत्नी᳚ - वा(3)ः । \newline
15. स॒जूर् दे॒वेन॑ दे॒वेन॑ स॒जूः स॒जूर् दे॒वेन॑ । \newline
16. स॒जूरिति॑ स - जूः । \newline
17. दे॒वेन॒ त्वष्ट्रा॒ त्वष्ट्रा॑ दे॒वेन॑ दे॒वेन॒ त्वष्ट्रा᳚ । \newline
18. त्वष्ट्रा॒ सोम॒(ग्म्॒) सोम॒म् त्वष्ट्रा॒ त्वष्ट्रा॒ सोम᳚म् । \newline
19. सोम॑म् पिब पिब॒ सोम॒(ग्म्॒) सोम॑म् पिब । \newline
20. पि॒ब॒ स्वाहा॒ स्वाहा॑ पिब पिब॒ स्वाहा᳚ । \newline
21. स्वाहेति॒ स्वाहा᳚ । \newline

\textbf{Ghana Paata } \newline

1. बृह॒स्पति॑सुतस्य ते ते॒ बृह॒स्पति॑सुतस्य॒ बृह॒स्पति॑सुतस्य त इन्दो इन्दो ते॒ बृह॒स्पति॑सुतस्य॒ बृह॒स्पति॑सुतस्य त इन्दो । \newline
2. बृह॒स्पति॑सुत॒स्येति॒ बृह॒स्पति॑ - सु॒त॒स्य॒ । \newline
3. त॒ इ॒न्दो॒ इ॒न्दो॒ ते॒ त॒ इ॒न्दो॒ इ॒न्द्रि॒याव॑त इन्द्रि॒याव॑त इन्दो ते त इन्दो इन्द्रि॒याव॑तः । \newline
4. इ॒न्दो॒ इ॒न्द्रि॒याव॑त इन्द्रि॒याव॑त इन्दो इन्दो इन्द्रि॒याव॑तः॒ पत्नी॑वन्त॒म् पत्नी॑वन्त मिन्द्रि॒याव॑त इन्दो इन्दो इन्द्रि॒याव॑तः॒ पत्नी॑वन्तम् । \newline
5. इ॒न्दो॒ इती᳚न्दो । \newline
6. इ॒न्द्रि॒याव॑तः॒ पत्नी॑वन्त॒म् पत्नी॑वन्त मिन्द्रि॒याव॑त इन्द्रि॒याव॑तः॒ पत्नी॑वन्त॒म् ग्रह॒म् ग्रह॒म् पत्नी॑वन्त मिन्द्रि॒याव॑त इन्द्रि॒याव॑तः॒ पत्नी॑वन्त॒म् ग्रह᳚म् । \newline
7. इ॒न्द्रि॒याव॑त॒ इती᳚न्द्रि॒य - व॒तः॒ । \newline
8. पत्नी॑वन्त॒म् ग्रह॒म् ग्रह॒म् पत्नी॑वन्त॒म् पत्नी॑वन्त॒म् ग्रह॑म् गृह्णामि गृह्णामि॒ ग्रह॒म् पत्नी॑वन्त॒म् पत्नी॑वन्त॒म् ग्रह॑म् गृह्णामि । \newline
9. पत्नी॑वन्त॒मिति॒ पत्नी᳚ - व॒न्त॒म् । \newline
10. ग्रह॑म् गृह्णामि गृह्णामि॒ ग्रह॒म् ग्रह॑म् गृह्णा॒म्यग्ना(3) अग्ना(3)इ गृ॑ह्णामि॒ ग्रह॒म् ग्रह॑म् गृह्णा॒म्यग्ना(3)इ । \newline
11. गृ॒ह्णा॒म्यग्ना(3) अग्ना(3)इ गृ॑ह्णामि गृह्णा॒म्यग्ना(3)इ पत्नी॒वा(3)ः पत्नी॒वा(3) अग्ना(3)इ गृ॑ह्णामि गृह्णा॒म्यग्ना(3)इ पत्नी॒वा(3)ः । \newline
12. अग्ना(3)इ पत्नी॒वा(3)ः पत्नी॒वा(3) अग्ना(3) अग्ना(3)इ पत्नी॒वा(3)ः स॒जूः स॒जूः पत्नी॒वा(3) अग्ना(3) अग्ना(3)इ पत्नी॒वा(3)ः स॒जूः । \newline
13. पत्नी॒वा(3)ः स॒जूः स॒जूः पत्नी॒वा(3)ः पत्नी॒वा(3)ः स॒जूर् दे॒वेन॑ दे॒वेन॑ स॒जूः पत्नी॒वा(3)ः पत्नी॒वा(3)ः स॒जूर् दे॒वेन॑ । \newline
14. पत्नी॒वा(3) इति॒ पत्नी᳚ - वा(3)ः । \newline
15. स॒जूर् दे॒वेन॑ दे॒वेन॑ स॒जूः स॒जूर् दे॒वेन॒ त्वष्ट्रा॒ त्वष्ट्रा॑ दे॒वेन॑ स॒जूः स॒जूर् दे॒वेन॒ त्वष्ट्रा᳚ । \newline
16. स॒जूरिति॑ स - जूः । \newline
17. दे॒वेन॒ त्वष्ट्रा॒ त्वष्ट्रा॑ दे॒वेन॑ दे॒वेन॒ त्वष्ट्रा॒ सोम॒(ग्म्॒) सोम॒म् त्वष्ट्रा॑ दे॒वेन॑ दे॒वेन॒ त्वष्ट्रा॒ सोम᳚म् । \newline
18. त्वष्ट्रा॒ सोम॒(ग्म्॒) सोम॒म् त्वष्ट्रा॒ त्वष्ट्रा॒ सोम॑म् पिब पिब॒ सोम॒म् त्वष्ट्रा॒ त्वष्ट्रा॒ सोम॑म् पिब । \newline
19. सोम॑म् पिब पिब॒ सोम॒(ग्म्॒) सोम॑म् पिब॒ स्वाहा॒ स्वाहा॑ पिब॒ सोम॒(ग्म्॒) सोम॑म् पिब॒ स्वाहा᳚ । \newline
20. पि॒ब॒ स्वाहा॒ स्वाहा॑ पिब पिब॒ स्वाहा᳚ । \newline
21. स्वाहेति॒ स्वाहा᳚ । \newline
\pagebreak
\markright{ TS 1.4.28.1  \hfill https://www.vedavms.in \hfill}
\addcontentsline{toc}{section}{ TS 1.4.28.1 }
\section*{ TS 1.4.28.1 }

\textbf{TS 1.4.28.1 } \newline
\textbf{Samhita Paata} \newline

हरि॑रसि हारियोज॒नो हर्योः᳚ स्था॒ता वज्र॑स्य भ॒र्ता पृश्ञेः᳚ प्रे॒ता तस्य॑ ते देव सोमे॒ष्टय॑जुषः स्तु॒तस्तो॑मस्य श॒स्तोक्थ॑स्य॒ हरि॑वन्तं॒ ग्रहं॑ गृह्णामि ह॒रीः स्थ॒ हर्यो᳚र्द्धा॒नाः स॒हसो॑मा॒ इन्द्रा॑य॒ स्वाहा᳚ ॥ \newline

\textbf{Pada Paata} \newline

हरिः॑ । अ॒सि॒ । हा॒रि॒यो॒ज॒न इति॑ हारि - यो॒ज॒नः । हर्योः᳚ । स्था॒ता । वज्र॑स्य । भ॒र्ता । पृश्नेः᳚ । प्रे॒ता । तस्य॑ । ते॒ । दे॒व॒ । सो॒म॒ । इ॒ष्टय॑जुष॒ इती॒ष्ट - य॒जु॒षः॒ । स्तु॒तस्तो॑म॒स्येति॑ स्तु॒त - स्तो॒म॒स्य॒ । श॒स्तोक्थ॒स्येति॑ श॒स्त - उ॒क्थ॒स्य॒ । हरि॑वन्त॒मिति॒ हरि॑ - व॒न्त॒म् । ग्रह᳚म् । गृ॒ह्णा॒मि॒ । ह॒रीः । स्थ॒ । हर्योः᳚ । धा॒नाः । स॒हसो॑मा॒ इति॑ स॒ह - सो॒माः॒ । इन्द्रा॑य । स्वाहा᳚ ॥  \newline


\textbf{Krama Paata} \newline

हरि॑रसि । अ॒सि॒ हा॒रि॒यो॒ज॒नः । हा॒रि॒यो॒ज॒नो हर्योः᳚ । हा॒रि॒यो॒ज॒न इति॑ हारि - यो॒ज॒नः । हर्योः᳚ स्था॒ता । स्था॒ता वज्र॑स्य । वज्र॑स्य भ॒र्ता । भ॒र्ता पृश्ञेः᳚ । पृश्ञेः᳚ प्रे॒ता । प्रे॒ता तस्य॑ । तस्य॑ ते । ते॒ दे॒व॒ । दे॒व॒ सो॒म॒ । सो॒मे॒ष्टय॑जुषः । इ॒ष्टय॑जुषः स्तु॒तस्तो॑मस्य । इ॒ष्टय॑जुष॒ इती॒ष्ट - य॒जु॒षः॒ । स्तु॒तस्तो॑मस्य श॒स्तोक्थ॑स्य । स्तु॒तस्तो॑म॒स्येति॑ स्तु॒त - स्तो॒म॒स्य॒ । श॒स्तोक्थ॑स्य॒ हरि॑वन्तम् । श॒स्तोक्थ॒स्येति॑ श॒स्त - उ॒क्थ॒स्य॒ । हरि॑वन्त॒म् ग्रह᳚म् । हरि॑वन्त॒मिति॒ हरि॑ - व॒न्त॒म् । ग्रह॑म् गृह्णामि । गृ॒ह्णा॒मि॒ ह॒रीः । ह॒रीः स्थ॑ । स्थ॒ हर्योः᳚ । हर्यो᳚र् धा॒नाः । धा॒नाः स॒हसो॑माः । स॒हसो॑मा॒ इन्द्रा॑य । स॒हसो॑मा॒ इति॑ स॒ह - सो॒माः॒ । इन्द्रा॑य॒ स्वाहा᳚ । स्वाहेति॒ स्वाहा᳚ । \newline

\textbf{Jatai Paata} \newline

1. हरि॑ रस्यसि॒ हरि॒र्॒. हरि॑रसि । \newline
2. अ॒सि॒ हा॒रि॒यो॒ज॒नो हा॑रियोज॒नो᳚ ऽस्यसि हारियोज॒नः । \newline
3. हा॒रि॒यो॒ज॒नो हर्यो॒र्॒. हर्योर्॑. हारियोज॒नो हा॑रियोज॒नो हर्योः᳚ । \newline
4. हा॒रि॒यो॒ज॒न इति॑ हारि - यो॒ज॒नः । \newline
5. हर्योः᳚ स्था॒ता स्था॒ता हर्यो॒र्॒. हर्योः᳚ स्था॒ता । \newline
6. स्था॒ता वज्र॑स्य॒ वज्र॑स्य स्था॒ता स्था॒ता वज्र॑स्य । \newline
7. वज्र॑स्य भ॒र्ता भ॒र्ता वज्र॑स्य॒ वज्र॑स्य भ॒र्ता । \newline
8. भ॒र्ता पृश्ञेः॒ पृश्ञे᳚र् भ॒र्ता भ॒र्ता पृश्ञेः᳚ । \newline
9. पृश्ञेः᳚ प्रे॒ता प्रे॒ता पृश्ञेः॒ पृश्ञेः᳚ प्रे॒ता । \newline
10. प्रे॒ता तस्य॒ तस्य॑ प्रे॒ता प्रे॒ता तस्य॑ । \newline
11. तस्य॑ ते ते॒ तस्य॒ तस्य॑ ते । \newline
12. ते॒ दे॒व॒ दे॒व॒ ते॒ ते॒ दे॒व॒ । \newline
13. दे॒व॒ सो॒म॒ सो॒म॒ दे॒व॒ दे॒व॒ सो॒म॒ । \newline
14. सो॒मे॒ ष्टय॑जुष इ॒ष्टय॑जुषः सोम सोमे॒ ष्टय॑जुषः । \newline
15. इ॒ष्टय॑जुषः स्तु॒तस्तो॑मस्य स्तु॒तस्तो॑मस्ये॒ ष्टय॑जुष इ॒ष्टय॑जुषः स्तु॒तस्तो॑मस्य । \newline
16. इ॒ष्टय॑जुष॒ इती॒ष्ट - य॒जु॒षः॒ । \newline
17. स्तु॒तस्तो॑मस्य श॒स्तोक्थ॑स्य श॒स्तोक्थ॑स्य स्तु॒तस्तो॑मस्य स्तु॒तस्तो॑मस्य श॒स्तोक्थ॑स्य । \newline
18. स्तु॒तस्तो॑म॒स्येति॑ स्तु॒त - स्तो॒म॒स्य॒ । \newline
19. श॒स्तोक्थ॑स्य॒ हरि॑वन्त॒(ग्म्॒) हरि॑वन्तꣳ श॒स्तोक्थ॑स्य श॒स्तोक्थ॑स्य॒ हरि॑वन्तम् । \newline
20. श॒स्तोक्थ॒स्येति॑ श॒स्त - उ॒क्थ॒स्य॒ । \newline
21. हरि॑वन्त॒म् ग्रह॒म् ग्रह॒(ग्म्॒) हरि॑वन्त॒(ग्म्॒) हरि॑वन्त॒म् ग्रह᳚म् । \newline
22. हरि॑वन्त॒मिति॒ हरि॑ - व॒न्त॒म् । \newline
23. ग्रह॑म् गृह्णामि गृह्णामि॒ ग्रह॒म् ग्रह॑म् गृह्णामि । \newline
24. गृ॒ह्णा॒मि॒ ह॒रीर्. ह॒रीर् गृ॑ह्णामि गृह्णामि ह॒रीः । \newline
25. ह॒रीः स्थ॑ स्थ ह॒रीर्. ह॒रीः स्थ॑ । \newline
26. स्थ॒ हर्यो॒र्॒. हर्योः᳚ स्थ स्थ॒ हर्योः᳚ । \newline
27. हर्यो᳚र् धा॒ना धा॒ना हर्यो॒र्॒. हर्यो᳚र् धा॒नाः । \newline
28. धा॒नाः स॒हसो॑माः स॒हसो॑मा धा॒ना धा॒नाः स॒हसो॑माः । \newline
29. स॒हसो॑मा॒ इन्द्रा॒ये न्द्रा॑य स॒हसो॑माः स॒हसो॑मा॒ इन्द्रा॑य । \newline
30. स॒हसो॑मा॒ इति॑ स॒ह - सो॒माः॒ । \newline
31. इन्द्रा॑य॒ स्वाहा॒ स्वाहेन्द्रा॒ये न्द्रा॑य॒ स्वाहा᳚ । \newline
32. स्वाहेति॒ स्वाहा᳚ । \newline

\textbf{Ghana Paata } \newline

1. हरि॑रस्यसि॒ हरि॒र्॒. हरि॑रसि हारियोज॒नो हा॑रियोज॒नो॑ ऽसि॒ हरि॒र्॒. हरि॑रसि हारियोज॒नः । \newline
2. अ॒सि॒ हा॒रि॒यो॒ज॒नो हा॑रियोज॒नो᳚ ऽस्यसि हारियोज॒नो हर्यो॒र्॒. हर्योर्॑. हारियोज॒नो᳚ ऽस्यसि हारियोज॒नो हर्योः᳚ । \newline
3. हा॒रि॒यो॒ज॒नो हर्यो॒र्॒. हर्योर्॑. हारियोज॒नो हा॑रियोज॒नो हर्योः᳚ स्था॒ता स्था॒ता हर्योर्॑. हारियोज॒नो हा॑रियोज॒नो हर्योः᳚ स्था॒ता । \newline
4. हा॒रि॒यो॒ज॒न इति॑ हारि - यो॒ज॒नः । \newline
5. हर्योः᳚ स्था॒ता स्था॒ता हर्यो॒र्॒. हर्योः᳚ स्था॒ता वज्र॑स्य॒ वज्र॑स्य स्था॒ता हर्यो॒र्॒. हर्योः᳚ स्था॒ता वज्र॑स्य । \newline
6. स्था॒ता वज्र॑स्य॒ वज्र॑स्य स्था॒ता स्था॒ता वज्र॑स्य भ॒र्ता भ॒र्ता वज्र॑स्य स्था॒ता स्था॒ता वज्र॑स्य भ॒र्ता । \newline
7. वज्र॑स्य भ॒र्ता भ॒र्ता वज्र॑स्य॒ वज्र॑स्य भ॒र्ता पृश्ञेः॒ पृश्ञे᳚र् भ॒र्ता वज्र॑स्य॒ वज्र॑स्य भ॒र्ता पृश्ञेः᳚ । \newline
8. भ॒र्ता पृश्ञेः॒ पृश्ञे᳚र् भ॒र्ता भ॒र्ता पृश्ञेः᳚ प्रे॒ता प्रे॒ता पृश्ञे᳚र् भ॒र्ता भ॒र्ता पृश्ञेः᳚ प्रे॒ता । \newline
9. पृश्ञेः᳚ प्रे॒ता प्रे॒ता पृश्ञेः॒ पृश्ञेः᳚ प्रे॒ता तस्य॒ तस्य॑ प्रे॒ता पृश्ञेः॒ पृश्ञेः᳚ प्रे॒ता तस्य॑ । \newline
10. प्रे॒ता तस्य॒ तस्य॑ प्रे॒ता प्रे॒ता तस्य॑ ते ते॒ तस्य॑ प्रे॒ता प्रे॒ता तस्य॑ ते । \newline
11. तस्य॑ ते ते॒ तस्य॒ तस्य॑ ते देव देव ते॒ तस्य॒ तस्य॑ ते देव । \newline
12. ते॒ दे॒व॒ दे॒व॒ ते॒ ते॒ दे॒व॒ सो॒म॒ सो॒म॒ दे॒व॒ ते॒ ते॒ दे॒व॒ सो॒म॒ । \newline
13. दे॒व॒ सो॒म॒ सो॒म॒ दे॒व॒ दे॒व॒ सो॒मे॒ ष्टय॑जुष इ॒ष्टय॑जुषः सोम देव देव सोमे॒ ष्टय॑जुषः । \newline
14. सो॒मे॒ ष्टय॑जुष इ॒ष्टय॑जुषः सोम सोमे॒ ष्टय॑जुषः स्तु॒तस्तो॑मस्य स्तु॒तस्तो॑मस्ये॒ ष्टय॑जुषः सोम सोमे॒ ष्टय॑जुषः स्तु॒तस्तो॑मस्य । \newline
15. इ॒ष्टय॑जुषः स्तु॒त स्तो॑मस्य स्तु॒तस्तो॑मस्ये॒ ष्टय॑जुष इ॒ष्टय॑जुषः स्तु॒तस्तो॑मस्य श॒स्तोक्थ॑स्य श॒ स्तोक्थ॑स्य स्तु॒तस्तो॑मस्ये॒ ष्टय॑जुष इ॒ष्टय॑जुषः स्तु॒तस्तो॑मस्य श॒स्तोक्थ॑स्य । \newline
16. इ॒ष्टय॑जुष॒ इती॒ष्ट - य॒जु॒षः॒ । \newline
17. स्तु॒तस्तो॑मस्य श॒स्तोक्थ॑स्य श॒स्तोक्थ॑स्य स्तु॒तस्तो॑मस्य स्तु॒तस्तो॑मस्य श॒तोक्थ॑स्य॒ हरि॑वन्त॒(ग्म्॒) हरि॑वन्तꣳ श॒स्तोक्थ॑स्य स्तु॒तस्तो॑मस्य स्तु॒तस्तो॑मस्य श॒स्तोक्थ॑स्य॒ हरि॑वन्तम् । \newline
18. स्तु॒तस्तो॑म॒स्येति॑ स्तु॒त - स्तो॒म॒स्य॒ । \newline
19. श॒स्तोक्थ॑स्य॒ हरि॑वन्त॒(ग्म्॒) हरि॑वन्तꣳ श॒स्तोक्थ॑स्य श॒स्तोक्थ॑स्य॒ हरि॑वन्त॒म् ग्रह॒म् ग्रह॒(ग्म्॒) हरि॑वन्तꣳ श॒स्तोक्थ॑स्य श॒स्तोक्थ॑स्य॒ हरि॑वन्त॒म् ग्रह᳚म् । \newline
20. श॒स्तोक्थ॒स्येति॑ श॒स्त - उ॒क्थ॒स्य॒ । \newline
21. हरि॑वन्त॒म् ग्रह॒म् ग्रह॒(ग्म्॒) हरि॑वन्त॒(ग्म्॒) हरि॑वन्त॒म् ग्रह॑म् गृह्णामि गृह्णामि॒ ग्रह॒(ग्म्॒) हरि॑वन्त॒(ग्म्॒) हरि॑वन्त॒म् ग्रह॑म् गृह्णामि । \newline
22. हरि॑वन्त॒मिति॒ हरि॑ - व॒न्त॒म् । \newline
23. ग्रह॑म् गृह्णामि गृह्णामि॒ ग्रह॒म् ग्रह॑म् गृह्णामि ह॒रीर्. ह॒रीर् गृ॑ह्णामि॒ ग्रह॒म् ग्रह॑म् गृह्णामि ह॒रीः । \newline
24. गृ॒ह्णा॒मि॒ ह॒रीर्. ह॒रीर् गृ॑ह्णामि गृह्णामि ह॒रीः स्थ॑ स्थ ह॒रीर् गृ॑ह्णामि गृह्णामि ह॒रीः स्थ॑ । \newline
25. ह॒रीः स्थ॑ स्थ ह॒रीर्. ह॒रीः स्थ॒ हर्यो॒र्॒. हर्योः᳚ स्थ ह॒रीर्. ह॒रीः स्थ॒ हर्योः᳚ । \newline
26. स्थ॒ हर्यो॒र्॒. हर्योः᳚ स्थ स्थ॒ हर्यो᳚र् धा॒ना धा॒ना हर्योः᳚ स्थ स्थ॒ हर्यो᳚र् धा॒नाः । \newline
27. हर्यो᳚र् धा॒ना धा॒ना हर्यो॒र्॒. हर्यो᳚र् धा॒नाः स॒हसो॑माः स॒हसो॑मा धा॒ना हर्यो॒र्॒. हर्यो᳚र् धा॒नाः स॒हसो॑माः । \newline
28. धा॒नाः स॒हसो॑माः स॒हसो॑मा धा॒ना धा॒नाः स॒हसो॑मा॒ इन्द्रा॒ये न्द्रा॑य स॒हसो॑मा धा॒ना धा॒नाः स॒हसो॑मा॒ इन्द्रा॑य । \newline
29. स॒हसो॑मा॒ इन्द्रा॒ये न्द्रा॑य स॒हसो॑माः स॒हसो॑मा॒ इन्द्रा॑य॒ स्वाहा॒ स्वाहेन्द्रा॑य स॒हसो॑माः स॒हसो॑मा॒ इन्द्रा॑य॒ स्वाहा᳚ । \newline
30. स॒हसो॑मा॒ इति॑ स॒ह - सो॒माः॒ । \newline
31. इन्द्रा॑य॒ स्वाहा॒ स्वाहेन्द्रा॒ये न्द्रा॑य॒ स्वाहा᳚ । \newline
32. स्वाहेति॒ स्वाहा᳚ । \newline
\pagebreak
\markright{ TS 1.4.29.1  \hfill https://www.vedavms.in \hfill}
\addcontentsline{toc}{section}{ TS 1.4.29.1 }
\section*{ TS 1.4.29.1 }

\textbf{TS 1.4.29.1 } \newline
\textbf{Samhita Paata} \newline

अग्न॒ आयूꣳ॑षि पवस॒ आ सु॒वोर्ज॒मिषं॑ च नः । आ॒रे बा॑धस्व दु॒च्छुनां᳚ ॥ उ॒प॒या॒मगृ॑हीतो-ऽस्य॒ग्नये᳚ त्वा॒ तेज॑स्वत ए॒ष ते॒ योनि॑र॒ग्नये᳚ त्वा॒ तेज॑स्वते ॥ \newline

\textbf{Pada Paata} \newline

अग्ने᳚ । आयूꣳ॑षि । प॒व॒से॒ । एति॑ । सु॒व॒ । ऊर्ज᳚म् । इष᳚म् । च॒ । नः॒ ॥ आ॒रे । बा॒ध॒स्व॒ । दु॒च्छुना᳚म् ॥ उ॒प॒या॒मगृ॑हीत॒ इत्यु॑पया॒म - गृ॒ही॒तः॒ । अ॒सि॒ । अ॒ग्नये᳚ । त्वा॒ । तेज॑स्वते । ए॒षः । ते॒ । योनिः॑ । अ॒ग्नये᳚ । त्वा॒ । तेज॑स्वते ॥  \newline


\textbf{Krama Paata} \newline

अग्न॒ आयूꣳ॑षि । आयूꣳ॑षि पवसे । प॒व॒स॒ आ । आ सु॑व । सु॒वोर्ज᳚म् । ऊर्ज॒मिष᳚म् । इष॑म् च । च॒ नः॒ । न॒ इति॑ नः ॥ आ॒रे बा॑धस्व । बा॒ध॒स्व॒ दु॒च्छुना᳚म् । दु॒च्छुना॒मिति॑ दु॒च्छुना᳚म् ॥ उ॒प॒या॒मगृ॑हीतोऽसि । उ॒प॒या॒मगृ॑हीत॒ इत्यु॑पया॒म - गृ॒ही॒तः॒ । अ॒स्य॒ग्नये᳚ । अ॒ग्नये᳚ त्वा । त्वा॒ तेज॑स्वते । तेज॑स्वत ए॒षः । ए॒ष ते᳚ । ते॒ योनिः॑ । योनि॑र॒ग्नये᳚ । अ॒ग्नये᳚ त्वा । त्वा॒ तेज॑स्वते । तेज॑स्वत॒ इति॒ तेज॑स्वते । \newline

\textbf{Jatai Paata} \newline

1. अग्न॒ आयू॒(ग्ग्॒) ष्यायू॒(ग्ग्॒) ष्यग्ने ऽग्न॒ आयू(ग्म्॑)षि । \newline
2. आयू(ग्म्॑)षि पवसे पवस॒ आयू॒(ग्ग्॒) ष्यायू(ग्म्॑)षि पवसे । \newline
3. प॒व॒स॒ आ प॑वसे पवस॒ आ । \newline
4. आ सु॑व सु॒वा सु॑व । \newline
5. सु॒वोर्ज॒ मूर्ज(ग्म्॑) सुव सु॒वोर्ज᳚म् । \newline
6. ऊर्ज॒ मिष॒ मिष॒ मूर्ज॒ मूर्ज॒ मिष᳚म् । \newline
7. इष॑म् च॒ चेष॒ मिष॑म् च । \newline
8. च॒ नो॒ न॒श्च॒ च॒ नः॒ । \newline
9. न॒ इति॑ नः । \newline
10. आ॒रे बा॑धस्व बाधस्वा॒र आ॒रे बा॑धस्व । \newline
11. बा॒ध॒स्व॒ दु॒च्छुना᳚म् दु॒च्छुना᳚म् बाधस्व बाधस्व दु॒च्छुना᳚म् । \newline
12. दु॒च्छुना॒मिति॑ दु॒च्छुना᳚म् । \newline
13. उ॒प॒या॒मगृ॑हीतो ऽस्यस्युपया॒मगृ॑हीत उपया॒मगृ॑हीतो ऽसि । \newline
14. उ॒प॒या॒मगृ॑हीत॒ इत्यु॑पया॒म - गृ॒ही॒तः॒ । \newline
15. अ॒स्य॒ग्नये॒ ऽग्नये᳚ ऽस्यस्य॒ग्नये᳚ । \newline
16. अ॒ग्नये᳚ त्वा त्वा॒ ऽग्नये॒ ऽग्नये᳚ त्वा । \newline
17. त्वा॒ तेज॑स्वते॒ तेज॑स्वते त्वा त्वा॒ तेज॑स्वते । \newline
18. तेज॑स्वत ए॒ष ए॒ष तेज॑स्वते॒ तेज॑स्वत ए॒षः । \newline
19. ए॒ष ते॑ त ए॒ष ए॒ष ते᳚ । \newline
20. ते॒ योनि॒र् योनि॑ स्ते ते॒ योनिः॑ । \newline
21. योनि॑ र॒ग्नये॒ ऽग्नये॒ योनि॒र् योनि॑ र॒ग्नये᳚ । \newline
22. अ॒ग्नये᳚ त्वा त्वा॒ ऽग्नये॒ ऽग्नये᳚ त्वा । \newline
23. त्वा॒ तेज॑स्वते॒ तेज॑स्वते त्वा त्वा॒ तेज॑स्वते । \newline
24. तेज॑स्वत॒ इति॒ तेज॑स्वते । \newline

\textbf{Ghana Paata } \newline

1. अग्न॒ आयू॒(ग्ग्॒) ष्यायू॒(ग्ग्॒) ष्यग्ने ऽग्न॒ आयू(ग्म्॑)षि पवसे पवस॒ आयू॒(ग्ग्॒) ष्यग्ने ऽग्न॒ आयू(ग्म्॑)षि पवसे । \newline
2. आयू(ग्म्॑)षि पवसे पवस॒ आयू॒(ग्ग्॒) ष्यायू(ग्म्॑)षि पवस॒ आ प॑वस॒ आयू॒(ग्ग्॒) ष्यायू(ग्म्॑)षि पवस॒ आ । \newline
3. प॒व॒स॒ आ प॑वसे पवस॒ आ सु॑व सु॒वा प॑वसे पवस॒ आ सु॑व । \newline
4. आ सु॑व सु॒वा सु॒वोर्ज॒ मूर्ज(ग्म्॑) सु॒वा सु॒वोर्ज᳚म् । \newline
5. सु॒वोर्ज॒ मूर्ज(ग्म्॑) सुव सु॒वोर्ज॒ मिष॒ मिष॒ मूर्ज(ग्म्॑) सुव सु॒वोर्ज॒ मिष᳚म् । \newline
6. ऊर्ज॒ मिष॒ मिष॒ मूर्ज॒ मूर्ज॒ मिष॑म् च॒ चे ष॒ मूर्ज॒ मूर्ज॒ मिष॑म् च । \newline
7. इष॑म् च॒ चे ष॒ मिष॑म् च नो न॒श्चे ष॒ मिष॑म् च नः । \newline
8. च॒ नो॒ न॒श्च॒ च॒ नः॒ । \newline
9. न॒ इति॑ नः । \newline
10. आ॒रे बा॑धस्व बाधस्वा॒र आ॒रे बा॑धस्व दु॒च्छुना᳚म् दु॒च्छुना᳚म् बाधस्वा॒र आ॒रे बा॑धस्व दु॒च्छुना᳚म् । \newline
11. बा॒ध॒स्व॒ दु॒च्छुना᳚म् दु॒च्छुना᳚म् बाधस्व बाधस्व दु॒च्छुना᳚म् । \newline
12. दु॒च्छुना॒मिति॑ दु॒च्छुना᳚म् । \newline
13. उ॒प॒या॒मगृ॑हीतो ऽस्यस्युपया॒मगृ॑हीत उपया॒मगृ॑हीतो ऽस्य॒ग्नये॒ ऽग्नये᳚ ऽस्युपया॒मगृ॑हीत उपया॒मगृ॑हीतो ऽस्य॒ग्नये᳚ । \newline
14. उ॒प॒या॒मगृ॑हीत॒ इत्यु॑पया॒म - गृ॒ही॒तः॒ । \newline
15. अ॒स्य॒ग्नये॒ ऽग्नये᳚ ऽस्य स्य॒ग्नये᳚ त्वा त्वा॒ ऽग्नये᳚ ऽस्य स्य॒ग्नये᳚ त्वा । \newline
16. अ॒ग्नये᳚ त्वा त्वा॒ ऽग्नये॒ ऽग्नये᳚ त्वा॒ तेज॑स्वते॒ तेज॑स्वते त्वा॒ ऽग्नये॒ ऽग्नये᳚ त्वा॒ तेज॑स्वते । \newline
17. त्वा॒ तेज॑स्वते॒ तेज॑स्वते त्वा त्वा॒ तेज॑स्वत ए॒ष ए॒ष तेज॑स्वते त्वा त्वा॒ तेज॑स्वत ए॒षः । \newline
18. तेज॑स्वत ए॒ष ए॒ष तेज॑स्वते॒ तेज॑स्वत ए॒ष ते॑ त ए॒ष तेज॑स्वते॒ तेज॑स्वत ए॒ष ते᳚ । \newline
19. ए॒ष ते॑ त ए॒ष ए॒ष ते॒ योनि॒र् योनि॑ स्त ए॒ष ए॒ष ते॒ योनिः॑ । \newline
20. ते॒ योनि॒र् योनि॑ स्ते ते॒ योनि॑ र॒ग्नये॒ ऽग्नये॒ योनि॑ स्ते ते॒ योनि॑ र॒ग्नये᳚ । \newline
21. योनि॑ र॒ग्नये॒ ऽग्नये॒ योनि॒र् योनि॑ र॒ग्नये᳚ त्वा त्वा॒ ऽग्नये॒ योनि॒र् योनि॑ र॒ग्नये᳚ त्वा । \newline
22. अ॒ग्नये᳚ त्वा त्वा॒ ऽग्नये॒ ऽग्नये᳚ त्वा॒ तेज॑स्वते॒ तेज॑स्वते त्वा॒ ऽग्नये॒ ऽग्नये᳚ त्वा॒ तेज॑स्वते । \newline
23. त्वा॒ तेज॑स्वते॒ तेज॑स्वते त्वा त्वा॒ तेज॑स्वते । \newline
24. तेज॑स्वत॒ इति॒ तेज॑स्वते । \newline
\pagebreak
\markright{ TS 1.4.30.1  \hfill https://www.vedavms.in \hfill}
\addcontentsline{toc}{section}{ TS 1.4.30.1 }
\section*{ TS 1.4.30.1 }

\textbf{TS 1.4.30.1 } \newline
\textbf{Samhita Paata} \newline

उ॒त्तिष्ठ॒न्नोज॑सा स॒ह पी॒त्वा शिप्रे॑ अवेपयः । सोम॑मिन्द्र च॒मू सु॒तं ॥ उ॒प॒या॒मगृ॑हीतो॒-ऽसीन्द्रा॑य॒ त्वौज॑स्वत ए॒ष ते॒ योनि॒रिन्द्रा॑य॒ त्वौज॑स्वते ॥ \newline

\textbf{Pada Paata} \newline

उ॒त्तिष्ठ॒न्नित्यु॑त् - तिष्ठन्न्॑ । ओज॑सा । स॒ह । पी॒त्वा । शिप्रे॒ इति॑ । अ॒वे॒प॒यः॒ ॥ सोम᳚म् । इ॒न्द्र॒ । च॒मू इति॑ । सु॒तम् ॥ उ॒प॒या॒मगृ॑हीत॒ इत्यु॑पया॒म - गृ॒ही॒तः॒ । अ॒सि॒ । इन्द्रा॑य । त्वा॒ । ओज॑स्वते । ए॒षः । ते॒ । योनिः॑ । इन्द्रा॑य । त्वा॒ । ओज॑स्वते ॥ 31(21)  \newline


\textbf{Krama Paata} \newline

उ॒त्तिष्ठ॒न्नोज॑सा । उ॒त्तिष्ठ॒न्नित्यु॑त् - तिष्ठन्न्॑ । ओज॑सा स॒ह । स॒ह पी॒त्वा । पी॒त्वा शिप्रे᳚ । शिप्रे॑ अवेपयः । शिप्रे॒ इति॒ शिप्रे᳚ । अ॒वे॒प॒य॒ इत्य॑वेपयः ॥ सोम॑मिन्द्र । इ॒न्द्र॒ च॒मू । च॒मू सु॒तम् । च॒मू इति॑ च॒मू । सु॒तमिति॑ सु॒तम् ॥ उ॒प॒या॒मगृ॑हीतोऽसि । उ॒प॒या॒मगृ॑हीत॒ इत्यु॑पया॒म - गृ॒ही॒तः॒ । अ॒सीन्द्रा॑य । इन्द्रा॑य त्वा । त्वौज॑स्वते । ओज॑स्वत ए॒षः । ए॒ष ते᳚ । ते॒ योनिः॑ । योनि॒रिन्द्रा॑य । इन्द्रा॑य त्वा । त्वौज॑स्वते । ओज॑स्वत॒ इत्योज॑स्वते । \newline

\textbf{Jatai Paata} \newline

1. उ॒त्तिष्ठ॒न् नोज॒सौज॑सो॒त्तिष्ठ॑न् नु॒त्तिष्ठ॒न् नोज॑सा । \newline
2. उ॒त्तिष्ठ॒न्नित्यु॑त् - तिष्ठन्न्॑ । \newline
3. ओज॑सा स॒ह स॒हौज॒सौज॑सा स॒ह । \newline
4. स॒ह पी॒त्वा पी॒त्वा स॒ह स॒ह पी॒त्वा । \newline
5. पी॒त्वा शिप्रे॒ शिप्रे॑ पी॒त्वा पी॒त्वा शिप्रे᳚ । \newline
6. शिप्रे॑ अवेपयो ऽवेपयः॒ शिप्रे॒ शिप्रे॑ अवेपयः । \newline
7. शिप्रे॒ इति॒ शिप्रे᳚ । \newline
8. अ॒वे॒प॒य॒ इत्य॑वेपयः । \newline
9. सोम॑ मिन्द्रेन्द्र॒ सोम॒(ग्म्॒) सोम॑ मिन्द्र । \newline
10. इ॒न्द्र॒ च॒मू च॒मू इ॑न्द्रेन्द्र च॒मू । \newline
11. च॒मू सु॒तꣳ सु॒तम् च॒मू च॒मू सु॒तम् । \newline
12. च॒मू इति॑ च॒मू । \newline
13. सु॒तमिति॑ सु॒तम् । \newline
14. उ॒प॒या॒मगृ॑हीतो ऽस्यस्युपया॒मगृ॑हीत उपया॒मगृ॑हीतो ऽसि । \newline
15. उ॒प॒या॒मगृ॑हीत॒ इत्यु॑पया॒म - गृ॒ही॒तः॒ । \newline
16. अ॒सीन्द्रा॒ये न्द्रा॑यास्य॒सीन्द्रा॑य । \newline
17. इन्द्रा॑य त्वा॒ त्वेन्द्रा॒ये न्द्रा॑य त्वा । \newline
18. त्वौज॑स्वत॒ ओज॑स्वते त्वा॒ त्वौज॑स्वते । \newline
19. ओज॑स्वत ए॒ष ए॒ष ओज॑स्वत॒ ओज॑स्वत ए॒षः । \newline
20. ए॒ष ते॑ त ए॒ष ए॒ष ते᳚ । \newline
21. ते॒ योनि॒र् योनि॑ स्ते ते॒ योनिः॑ । \newline
22. योनि॒ रिन्द्रा॒ये न्द्रा॑य॒ योनि॒र् योनि॒ रिन्द्रा॑य । \newline
23. इन्द्रा॑य त्वा॒ त्वेन्द्रा॒ये न्द्रा॑य त्वा । \newline
24. त्वौज॑स्वत॒ ओज॑स्वते त्वा॒ त्वौज॑स्वते । \newline
25. ओज॑स्वत॒ इत्योज॑स्वते । \newline

\textbf{Ghana Paata } \newline

1. उ॒त्तिष्ठ॒न् नोज॒ सौज॑सो॒त्तिष्ठ॑न् नु॒त्तिष्ठ॒न् नोज॑सा स॒ह स॒हौज॑सो॒त्तिष्ठ॑न् नु॒त्तिष्ठ॒न् नोज॑सा स॒ह । \newline
2. उ॒त्तिष्ठ॒न्नित्यु॑त् - तिष्ठन्न्॑ । \newline
3. ओज॑सा स॒ह स॒हौ ज॒सौज॑सा स॒ह पी॒त्वा पी॒त्वा स॒हौ ज॒सौज॑सा स॒ह पी॒त्वा । \newline
4. स॒ह पी॒त्वा पी॒त्वा स॒ह स॒ह पी॒त्वा शिप्रे॒ शिप्रे॑ पी॒त्वा स॒ह स॒ह पी॒त्वा शिप्रे᳚ । \newline
5. पी॒त्वा शिप्रे॒ शिप्रे॑ पी॒त्वा पी॒त्वा शिप्रे॑ अवेपयो ऽवेपयः॒ शिप्रे॑ पी॒त्वा पी॒त्वा शिप्रे॑ अवेपयः । \newline
6. शिप्रे॑ अवेपयो ऽवेपयः॒ शिप्रे॒ शिप्रे॑ अवेपयः । \newline
7. शिप्रे॒ इति॒ शिप्रे᳚ । \newline
8. अ॒वे॒प॒य॒ इत्य॑वेपयः । \newline
9. सोम॑ मिन्द्रे न्द्र॒ सोम॒(ग्म्॒) सोम॑ मिन्द्र च॒मू च॒मू इ॑न्द्र॒ सोम॒(ग्म्॒) सोम॑ मिन्द्र च॒मू । \newline
10. इ॒न्द्र॒ च॒मू च॒मू इ॑न्द्रे न्द्र च॒मू सु॒तꣳ सु॒तम् च॒मू इ॑न्द्रे न्द्र च॒मू सु॒तम् । \newline
11. च॒मू सु॒तꣳ सु॒तम् च॒मू च॒मू सु॒तम् । \newline
12. च॒मू इति॑ च॒मू । \newline
13. सु॒तमिति॑ सु॒तम् । \newline
14. उ॒प॒या॒मगृ॑हीतो ऽस्यस्युपया॒मगृ॑हीत उपया॒मगृ॑हीतो॒ ऽसीन्द्रा॒ये न्द्रा॑या स्युपया॒मगृ॑हीत उपया॒मगृ॑हीतो॒ ऽसीन्द्रा॑य । \newline
15. उ॒प॒या॒मगृ॑हीत॒ इत्यु॑पया॒म - गृ॒ही॒तः॒ । \newline
16. अ॒सीन्द्रा॒ये न्द्रा॑ यास्य॒ सीन्द्रा॑य त्वा॒ त्वेन्द्रा॑ यास्य॒ सीन्द्रा॑य त्वा । \newline
17. इन्द्रा॑य त्वा॒ त्वेन्द्रा॒ये न्द्रा॑य॒ त्वौज॑स्वत॒ ओज॑स्वते॒ त्वेन्द्रा॒ये न्द्रा॑य॒ त्वौज॑स्वते । \newline
18. त्वौज॑स्वत॒ ओज॑स्वते त्वा॒ त्वौज॑स्वत ए॒ष ए॒ष ओज॑स्वते त्वा॒ त्वौज॑स्वत ए॒षः । \newline
19. ओज॑स्वत ए॒ष ए॒ष ओज॑स्वत॒ ओज॑स्वत ए॒ष ते॑ त ए॒ष ओज॑स्वत॒ ओज॑स्वत ए॒ष ते᳚ । \newline
20. ए॒ष ते॑ त ए॒ष ए॒ष ते॒ योनि॒र् योनि॑ स्त ए॒ष ए॒ष ते॒ योनिः॑ । \newline
21. ते॒ योनि॒र् योनि॑ स्ते ते॒ योनि॒ रिन्द्रा॒ये न्द्रा॑य॒ योनि॑ स्ते ते॒ योनि॒ रिन्द्रा॑य । \newline
22. योनि॒ रिन्द्रा॒ये न्द्रा॑य॒ योनि॒र् योनि॒ रिन्द्रा॑य त्वा॒ त्वेन्द्रा॑य॒ योनि॒र् योनि॒ रिन्द्रा॑य त्वा । \newline
23. इन्द्रा॑य त्वा॒ त्वेन्द्रा॒ये न्द्रा॑य॒ त्वौज॑स्वत॒ ओज॑स्वते॒ त्वेन्द्रा॒ये न्द्रा॑य॒ त्वौज॑स्वते । \newline
24. त्वौज॑स्वत॒ ओज॑स्वते त्वा॒ त्वौज॑स्वते । \newline
25. ओज॑स्वत॒ इत्योज॑स्वते । \newline
\pagebreak
\markright{ TS 1.4.31.1  \hfill https://www.vedavms.in \hfill}
\addcontentsline{toc}{section}{ TS 1.4.31.1 }
\section*{ TS 1.4.31.1 }

\textbf{TS 1.4.31.1 } \newline
\textbf{Samhita Paata} \newline

त॒रणि॑र् वि॒श्वद॑र्.शतो ज्योति॒ष्कृद॑सि सूर्य । विश्व॒मा भा॑सि रोच॒नं ॥ उ॒प॒या॒मगृ॑हीतो-ऽसि॒ सूर्या॑य त्वा॒ भ्राज॑स्वत ए॒ष ते॒ योनिः॒ सूर्या॑य त्वा॒ भ्राज॑स्वते ॥ \newline

\textbf{Pada Paata} \newline

त॒रणिः॑ । वि॒श्वद॑र्.शत॒ इति॑ वि॒श्व - द॒र्.॒श॒तः॒ । ज्यो॒ति॒ष्कृदिति॑ ज्योतिः - कृत् । अ॒सि॒ । सू॒र्य॒ ॥ विश्व᳚म् । एति॑ । भा॒सि॒ । रो॒च॒नम् ॥ उ॒प॒या॒मगृ॑हीत॒ इत्यु॑पया॒म - गृ॒ही॒तः॒ । अ॒सि॒ । सूर्या॑य । त्वा॒ । भ्राज॑स्वते । ए॒षः । ते॒ । योनिः॑ । सूर्या॑य । त्वा॒ । भ्राज॑स्वते ॥  \newline


\textbf{Krama Paata} \newline

त॒रणि॑र् वि॒श्वद॑र्.शतः । वि॒श्वद॑र्.शतो ज्योति॒ष्कृत् । वि॒श्वद॑र्.शत॒ इति॑ वि॒श्व - द॒र्.॒श॒तः॒ । ज्यो॒ति॒ष्कृद॑सि । ज्यो॒ति॒ष्कृदिति॑ ज्योतिः - कृत् । अ॒सि॒ सू॒र्य॒ । सू॒र्येति॑ सूर्य ॥ विश्व॒मा । आ भा॑सि । भा॒सि॒ रो॒च॒नम् । रो॒च॒नमिति॑ रोच॒नम् ॥ उ॒प॒या॒मगृ॑हीतोऽसि । उ॒प॒या॒मगृ॑हीत॒ इत्यु॑पया॒म - गृ॒ही॒तः॒ । अ॒सि॒ सूर्या॑य । सूर्या॑य त्वा । त्वा॒ भ्राज॑स्वते । भ्राज॑स्वत ए॒षः । ए॒ष ते᳚ । ते॒ योनिः॑ । योनिः॒ सूर्या॑य । सूर्या॑य त्वा । त्वा॒ भ्राज॑स्वते । भ्राज॑स्वत॒ इति॒ भ्राज॑स्वते । \newline

\textbf{Jatai Paata} \newline

1. त॒रणि॑र् वि॒श्वद॑र्.शतो वि॒श्वद॑र्.शत स्त॒रणि॑ स्त॒रणि॑र् वि॒श्वद॑र्.शतः । \newline
2. वि॒श्वद॑र्.शतो ज्योति॒ष्कृज् ज्यो॑ति॒ष्कृद् वि॒श्वद॑र्.शतो वि॒श्वद॑र्.शतो ज्योति॒ष्कृत् । \newline
3. वि॒श्वद॑र्.शत॒ इति॑ वि॒श्व - द॒र्॒.श॒तः॒ । \newline
4. ज्यो॒ति॒ष्कृ द॑स्यसि ज्योति॒ष्कृज् ज्यो॑ति॒ष्कृद॑सि । \newline
5. ज्यो॒ति॒ष्कृदिति॑ ज्योतिः - कृत् । \newline
6. अ॒सि॒ सू॒र्य॒ सू॒र्या॒स्य॒सि॒ सू॒र्य॒ । \newline
7. सू॒र्येति॑ सूर्य । \newline
8. विश्व॒ मा विश्वं॒ ॅविश्व॒ मा । \newline
9. आ भा॑सि भा॒स्या भा॑सि । \newline
10. भा॒सि॒ रो॒च॒नꣳ रो॑च॒नम् भा॑सि भासि रोच॒नम् । \newline
11. रो॒च॒नमिति॑ रोच॒नम् । \newline
12. उ॒प॒या॒मगृ॑हीतो ऽस्यस्युपया॒मगृ॑हीत उपया॒मगृ॑हीतो ऽसि । \newline
13. उ॒प॒या॒मगृ॑हीत॒ इत्यु॑पया॒म - गृ॒ही॒तः॒ । \newline
14. अ॒सि॒ सूर्या॑य॒ सूर्या॑यास्यसि॒ सूर्या॑य । \newline
15. सूर्या॑य त्वा त्वा॒ सूर्या॑य॒ सूर्या॑य त्वा । \newline
16. त्वा॒ भ्राज॑स्वते॒ भ्राज॑स्वते त्वा त्वा॒ भ्राज॑स्वते । \newline
17. भ्राज॑स्वत ए॒ष ए॒ष भ्राज॑स्वते॒ भ्राज॑स्वत ए॒षः । \newline
18. ए॒ष ते॑ त ए॒ष ए॒ष ते᳚ । \newline
19. ते॒ योनि॒र् योनि॑ स्ते ते॒ योनिः॑ । \newline
20. योनिः॒ सूर्या॑य॒ सूर्या॑य॒ योनि॒र् योनिः॒ सूर्या॑य । \newline
21. सूर्या॑य त्वा त्वा॒ सूर्या॑य॒ सूर्या॑य त्वा । \newline
22. त्वा॒ भ्राज॑स्वते॒ भ्राज॑स्वते त्वा त्वा॒ भ्राज॑स्वते । \newline
23. भ्राज॑स्वत॒ इति॒ भ्राज॑स्वते । \newline

\textbf{Ghana Paata } \newline

1. त॒रणि॑र् वि॒श्वद॑र्.शतो वि॒श्वद॑र्.शत स्त॒रणि॑ स्त॒रणि॑र् वि॒श्वद॑र्.शतो ज्योति॒ष्कृज् ज्यो॑ति॒ष्कृद् वि॒श्वद॑र्.शत स्त॒रणि॑ स्त॒रणि॑र् वि॒श्वद॑र्.शतो ज्योति॒ष्कृत् । \newline
2. वि॒श्वद॑र्.शतो ज्योति॒ष्कृज् ज्यो॑ति॒ष्कृद् वि॒श्वद॑र्.शतो वि॒श्वद॑र्.शतो ज्योति॒ष्कृ द॑स्यसि ज्योति॒ष्कृद् वि॒श्वद॑र्.शतो वि॒श्वद॑र्.शतो ज्योति॒ष्कृद॑सि । \newline
3. वि॒श्वद॑र्.शत॒ इति॑ वि॒श्व - द॒र्॒.श॒तः॒ । \newline
4. ज्यो॒ति॒ष्कृ द॑स्यसि ज्योति॒ष्कृज् ज्यो॑ति॒ष्कृद॑सि सूर्य सूर्यासि ज्योति॒ष्कृज् ज्यो॑ति॒ष्कृद॑सि सूर्य । \newline
5. ज्यो॒ति॒ष्कृदिति॑ ज्योतिः - कृत् । \newline
6. अ॒सि॒ सू॒र्य॒ सू॒र्या॒ स्य॒सि॒ सू॒र्य॒ । \newline
7. सू॒र्येति॑ सूर्य । \newline
8. विश्व॒ मा विश्वं॒ ॅविश्व॒ मा भा॑सि भा॒स्या विश्वं॒ ॅविश्व॒ मा भा॑सि । \newline
9. आ भा॑सि भा॒स्या भा॑सि रोच॒नꣳ रो॑च॒नम् भा॒स्या भा॑सि रोच॒नम् । \newline
10. भा॒सि॒ रो॒च॒नꣳ रो॑च॒नम् भा॑सि भासि रोच॒नम् । \newline
11. रो॒च॒नमिति॑ रोच॒नम् । \newline
12. उ॒प॒या॒मगृ॑हीतो ऽस्यस्युपया॒मगृ॑हीत उपया॒मगृ॑हीतो ऽसि॒ सूर्या॑य॒ सूर्या॑या स्युपया॒मगृ॑हीत उपया॒मगृ॑हीतो ऽसि॒ सूर्या॑य । \newline
13. उ॒प॒या॒मगृ॑हीत॒ इत्यु॑पया॒म - गृ॒ही॒तः॒ । \newline
14. अ॒सि॒ सूर्या॑य॒ सूर्या॑ यास्यसि॒ सूर्या॑य त्वा त्वा॒ सूर्या॑ यास्यसि॒ सूर्या॑य त्वा । \newline
15. सूर्या॑य त्वा त्वा॒ सूर्या॑य॒ सूर्या॑य त्वा॒ भ्राज॑स्वते॒ भ्राज॑स्वते त्वा॒ सूर्या॑य॒ सूर्या॑य त्वा॒ भ्राज॑स्वते । \newline
16. त्वा॒ भ्राज॑स्वते॒ भ्राज॑स्वते त्वा त्वा॒ भ्राज॑स्वत ए॒ष ए॒ष भ्राज॑स्वते त्वा त्वा॒ भ्राज॑स्वत ए॒षः । \newline
17. भ्राज॑स्वत ए॒ष ए॒ष भ्राज॑स्वते॒ भ्राज॑स्वत ए॒ष ते॑ त ए॒ष भ्राज॑स्वते॒ भ्राज॑स्वत ए॒ष ते᳚ । \newline
18. ए॒ष ते॑ त ए॒ष ए॒ष ते॒ योनि॒र् योनि॑ स्त ए॒ष ए॒ष ते॒ योनिः॑ । \newline
19. ते॒ योनि॒र् योनि॑ स्ते ते॒ योनिः॒ सूर्या॑य॒ सूर्या॑य॒ योनि॑ स्ते ते॒ योनिः॒ सूर्या॑य । \newline
20. योनिः॒ सूर्या॑य॒ सूर्या॑य॒ योनि॒र् योनिः॒ सूर्या॑य त्वा त्वा॒ सूर्या॑य॒ योनि॒र् योनिः॒ सूर्या॑य त्वा । \newline
21. सूर्या॑य त्वा त्वा॒ सूर्या॑य॒ सूर्या॑य त्वा॒ भ्राज॑स्वते॒ भ्राज॑स्वते त्वा॒ सूर्या॑य॒ सूर्या॑य त्वा॒ भ्राज॑स्वते । \newline
22. त्वा॒ भ्राज॑स्वते॒ भ्राज॑स्वते त्वा त्वा॒ भ्राज॑स्वते । \newline
23. भ्राज॑स्वत॒ इति॒ भ्राज॑स्वते । \newline
\pagebreak
\markright{ TS 1.4.32.1  \hfill https://www.vedavms.in \hfill}
\addcontentsline{toc}{section}{ TS 1.4.32.1 }
\section*{ TS 1.4.32.1 }

\textbf{TS 1.4.32.1 } \newline
\textbf{Samhita Paata} \newline

आ प्या॑यस्व मदिन्तम॒ सोम॒ विश्वा॑भि-रू॒तिभिः॑ । भवा॑ नः स॒प्रथ॑स्तमः ॥ \newline

\textbf{Pada Paata} \newline

एति॑ । प्या॒य॒स्व॒ । म॒दि॒न्त॒म॒ । सोम॑ । विश्वा॑भिः । ऊ॒तिभि॒रित्यू॒ति - भिः॒ ॥ भव॑ । नः॒ । स॒प्रथ॑स्तम॒ इति॑ स॒प्रथः॑ - त॒मः॒ ॥  \newline


\textbf{Krama Paata} \newline

आ प्या॑यस्व । प्या॒य॒स्व॒ म॒दि॒न्त॒म॒ । म॒दि॒न्त॒म॒ सोम॑ । सोम॒ विश्वा॑भिः । विश्वा॑भिरू॒तिभिः॑ । ऊ॒तिभि॒रित्यू॒ति - भिः॒ ॥ भवा॑ नः । नः॒ स॒प्रथ॑स्तमः । स॒प्रथ॑स्तम॒ इति॑ स॒प्रथः॑ - त॒मः॒ । \newline

\textbf{Jatai Paata} \newline

1. आ प्या॑यस्व प्याय॒स्वा प्या॑यस्व । \newline
2. प्या॒य॒स्व॒ म॒दि॒न्त॒म॒ म॒दि॒न्त॒म॒ प्या॒य॒स्व॒ प्या॒य॒स्व॒ म॒दि॒न्त॒म॒ । \newline
3. म॒दि॒न्त॒म॒ सोम॒ सोम॑ मदिन्तम मदिन्तम॒ सोम॑ । \newline
4. सोम॒ विश्वा॑भि॒र् विश्वा॑भिः॒ सोम॒ सोम॒ विश्वा॑भिः । \newline
5. विश्वा॑भि रू॒तिभि॑ रू॒तिभि॒र् विश्वा॑भि॒र् विश्वा॑भि रू॒तिभिः॑ । \newline
6. ऊ॒तिभि॒रित्यू॒ति - भिः॒ । \newline
7. भवा॑ नो नो॒ भव॒ भवा॑ नः । \newline
8. नः॒ स॒प्रथ॑स्तमः स॒प्रथ॑स्तमो नो नः स॒प्रथ॑स्तमः । \newline
9. स॒प्रथ॑स्तम॒ इति॑ स॒प्रथः॑ - त॒मः॒ । \newline

\textbf{Ghana Paata } \newline

1. आ प्या॑यस्व प्याय॒स्वा प्या॑यस्व मदिन्तम मदिन्तम प्याय॒स्वा प्या॑यस्व मदिन्तम । \newline
2. प्या॒य॒स्व॒ म॒दि॒न्त॒म॒ म॒दि॒न्त॒म॒ प्या॒य॒स्व॒ प्या॒य॒स्व॒ म॒दि॒न्त॒म॒ सोम॒ सोम॑ मदिन्तम प्यायस्व प्यायस्व मदिन्तम॒ सोम॑ । \newline
3. म॒दि॒न्त॒म॒ सोम॒ सोम॑ मदिन्तम मदिन्तम॒ सोम॒ विश्वा॑भि॒र् विश्वा॑भिः॒ सोम॑ मदिन्तम मदिन्तम॒ सोम॒ विश्वा॑भिः । \newline
4. सोम॒ विश्वा॑भि॒र् विश्वा॑भिः॒ सोम॒ सोम॒ विश्वा॑भि रू॒तिभि॑ रू॒तिभि॒र् विश्वा॑भिः॒ सोम॒ सोम॒ विश्वा॑भि रू॒तिभिः॑ । \newline
5. विश्वा॑भि रू॒तिभि॑ रू॒तिभि॒र् विश्वा॑भि॒र् विश्वा॑भि रू॒तिभिः॑ । \newline
6. ऊ॒तिभि॒रित्यू॒ति - भिः॒ । \newline
7. भवा॑ नो नो॒ भव॒ भवा॑ नः स॒प्रथ॑स्तमः स॒प्रथ॑स्तमो नो॒ भव॒ भवा॑ नः स॒प्रथ॑स्तमः । \newline
8. नः॒ स॒प्रथ॑स्तमः स॒प्रथ॑स्तमो नो नः स॒प्रथ॑स्तमः । \newline
9. स॒प्रथ॑स्तम॒ इति॑ स॒प्रथः॑ - त॒मः॒ । \newline
\pagebreak
\markright{ TS 1.4.33.1  \hfill https://www.vedavms.in \hfill}
\addcontentsline{toc}{section}{ TS 1.4.33.1 }
\section*{ TS 1.4.33.1 }

\textbf{TS 1.4.33.1 } \newline
\textbf{Samhita Paata} \newline

ई॒युष्टे ये पूर्व॑तरा॒मप॑श्यन् व्यु॒च्छन्ती॑मु॒षसं॒ मर्त्या॑सः । अ॒स्माभि॑रू॒ नु प्र॑ति॒चक्ष्या॑ऽभू॒दो ते य॑न्ति॒ ये अ॑प॒रीषु॒ पश्यान्॑ ॥ \newline

\textbf{Pada Paata} \newline

ई॒युः । ते । ये । पूर्व॑तरा॒मिति॒ पूर्व॑ - त॒रा॒म् । अप॑श्यन्न् । व्यु॒च्छन्ती॒मिति॑ वि - उ॒च्छन्ती᳚म् । उ॒षस᳚म् । मर्त्या॑सः ॥ अ॒स्माभिः॑ । उ॒ । नु । प्र॒ति॒चक्ष्येति॑ प्रति - चक्ष्या᳚ । अ॒भू॒त् । ओ इति॑ । ते । य॒न्ति॒ । ये । अ॒प॒रीषु॑ । पश्यान्॑ ॥  \newline


\textbf{Krama Paata} \newline

ई॒युष्टे । ते ये । ये पूर्व॑तराम् । पूर्व॑तरा॒मप॑श्यन्न् । पूर्व॑तरा॒मिति॒ पूर्व॑ - त॒रा॒म् । अप॑श्यन्,व्यु॒च्छन्ती᳚म् । व्यु॒च्छन्ती॑मु॒षस᳚म् । व्यु॒च्छन्ती॒मिति॑ वि - उ॒च्छन्ती᳚म् । उ॒षस॒म् मर्त्या॑सः । मर्त्या॑स॒ इति॒ मर्त्या॑सः ॥ अ॒स्माभि॑रु । ऊ॒ नु । नु प्र॑ति॒चक्ष्या᳚ । प्र॒ति॒चक्ष्या॑ऽभूत् । प्र॒ति॒चक्ष्येति॑ प्रति - चक्ष्या᳚ । अ॒भू॒दो । ओ ते । ओ इत्यो । ते य॑न्ति । य॒न्ति॒ ये । ये अ॑प॒रीषु॑ । अ॒प॒रीषु॒ पश्यान्॑ । पश्या॒निति॒ पश्यान्॑ । \newline

\textbf{Jatai Paata} \newline

1. ई॒युष्टे त ई॒यु री॒युष्टे । \newline
2. ते ये ये ते ते ये । \newline
3. ये पूर्व॑तरा॒म् पूर्व॑तरां॒ ॅये ये पूर्व॑तराम् । \newline
4. पूर्व॑तरा॒ मप॑श्य॒न् नप॑श्य॒न् पूर्व॑तरा॒म् पूर्व॑तरा॒ मप॑श्यन्न् । \newline
5. पूर्व॑तरा॒मिति॒ पूर्व॑ - त॒रा॒म् । \newline
6. अप॑श्यन् व्यु॒च्छन्तीं᳚ ॅव्यु॒च्छन्ती॒ मप॑श्य॒न् नप॑श्यन् व्यु॒च्छन्ती᳚म् । \newline
7. व्यु॒च्छन्ती॑ मु॒षस॑ मु॒षसं॑ ॅव्यु॒च्छन्तीं᳚ ॅव्यु॒च्छन्ती॑ मु॒षस᳚म् । \newline
8. व्यु॒च्छन्ती॒मिति॑ वि - उ॒च्छन्ती᳚म् । \newline
9. उ॒षस॒म् मर्त्या॑सो॒ मर्त्या॑स उ॒षस॑ मु॒षस॒म् मर्त्या॑सः । \newline
10. मर्त्या॑स॒ इति॒ मर्त्या॑सः । \newline
11. अ॒स्माभि॑रु वु व॒स्माभि॑र॒स्माभि॑रु । \newline
12. ऊ॒ नु नू नु । \newline
13. नु प्र॑ति॒चक्ष्या᳚ प्रति॒चक्ष्या॒ नु नु प्र॑ति॒चक्ष्या᳚ । \newline
14. प्र॒ति॒चक्ष्या॑ ऽभूदभूत् प्रति॒चक्ष्या᳚ प्रति॒चक्ष्या॑ ऽभूत् । \newline
15. प्र॒ति॒चक्ष्येति॑ प्रति - चक्ष्या᳚ । \newline
16. अ॒भू॒दो ओ अ॑भूदभू॒दो । \newline
17. ओ ते त ओ ओ ते । \newline
18. ओ इत्यो । \newline
19. ते य॑न्ति यन्ति॒ ते ते य॑न्ति । \newline
20. य॒न्ति॒ ये ये य॑न्ति यन्ति॒ ये । \newline
21. ये अ॑प॒री ष्व॑प॒रीषु॒ ये ये अ॑प॒रीषु॑ । \newline
22. अ॒प॒रीषु॒ पश्या॒न् पश्या॑ नप॒री ष्व॑प॒रीषु॒ पश्यान्॑ । \newline
23. पश्या॒निति॒ पश्यान्॑ । \newline

\textbf{Ghana Paata } \newline

1. ई॒युष्टे त ई॒यु री॒युष्टे ये ये त ई॒यु री॒युष्टे ये । \newline
2. ते ये ये ते ते ये पूर्व॑तरा॒म् पूर्व॑तरां॒ ॅये ते ते ये पूर्व॑तराम् । \newline
3. ये पूर्व॑तरा॒म् पूर्व॑तरां॒ ॅये ये पूर्व॑तरा॒ मप॑श्य॒न् नप॑श्य॒न् पूर्व॑तरां॒ ॅये ये पूर्व॑तरा॒ मप॑श्यन्न् । \newline
4. पूर्व॑तरा॒ मप॑श्य॒न् नप॑श्य॒न् पूर्व॑तरा॒म् पूर्व॑तरा॒ मप॑श्यन् व्यु॒च्छन्तीं᳚ ॅव्यु॒च्छन्ती॒ मप॑श्य॒न् पूर्व॑तरा॒म् पूर्व॑तरा॒ मप॑श्यन् व्यु॒च्छन्ती᳚म् । \newline
5. पूर्व॑तरा॒मिति॒ पूर्व॑ - त॒रा॒म् । \newline
6. अप॑श्यन् व्यु॒च्छन्तीं᳚ ॅव्यु॒च्छन्ती॒ मप॑श्य॒न् नप॑श्यन् व्यु॒च्छन्ती॑ मु॒षस॑ मु॒षसं॑ ॅव्यु॒च्छन्ती॒ मप॑श्य॒न् नप॑श्यन् व्यु॒च्छन्ती॑ मु॒षस᳚म् । \newline
7. व्यु॒च्छन्ती॑ मु॒षस॑ मु॒षसं॑ ॅव्यु॒च्छन्तीं᳚ ॅव्यु॒च्छन्ती॑ मु॒षस॒म् मर्त्या॑सो॒ मर्त्या॑स उ॒षसं॑ ॅव्यु॒च्छन्तीं᳚ ॅव्यु॒च्छन्ती॑ मु॒षस॒म् मर्त्या॑सः । \newline
8. व्यु॒च्छन्ती॒मिति॑ वि - उ॒च्छन्ती᳚म् । \newline
9. उ॒षस॒म् मर्त्या॑सो॒ मर्त्या॑स उ॒षस॑ मु॒षस॒म् मर्त्या॑सः । \newline
10. मर्त्या॑स॒ इति॒ मर्त्या॑सः । \newline
11. अ॒स्माभि॑ रु वु व॒स्माभि॑ र॒स्माभि॑ रू॒ नु नु॑न्व॒स्माभि॑ र॒स्माभि॑ रू॒ नु । \newline
12. ऊ॒ नु नू नु प्र॑ति॒चक्ष्या᳚ प्रति॒चक्ष्या॒ नू नु प्र॑ति॒चक्ष्या᳚ । \newline
13. नु प्र॑ति॒चक्ष्या᳚ प्रति॒चक्ष्या॒ नु नु प्र॑ति॒चक्ष्या॑ ऽभूदभूत् प्रति॒चक्ष्या॒ नु नु प्र॑ति॒चक्ष्या॑ ऽभूत् । \newline
14. प्र॒ति॒चक्ष्या॑ ऽभूदभूत् प्रति॒चक्ष्या᳚ प्रति॒चक्ष्या॑ ऽभू॒दो ओ अ॑भूत् प्रति॒चक्ष्या᳚ प्रति॒चक्ष्या॑ ऽभू॒दो । \newline
15. प्र॒ति॒चक्ष्येति॑ प्रति - चक्ष्या᳚ । \newline
16. अ॒भू॒दो ओ अ॑भू दभू॒दो ते त ओ अ॑भू दभू॒दो ते । \newline
17. ओ ते त ओ ओ ते य॑न्ति यन्ति॒ त ओ ओ ते य॑न्ति । \newline
18. ओ इत्यो । \newline
19. ते य॑न्ति यन्ति॒ ते ते य॑न्ति॒ ये ये य॑न्ति॒ ते ते य॑न्ति॒ ये । \newline
20. य॒न्ति॒ ये ये य॑न्ति यन्ति॒ ये अ॑प॒री ष्व॑प॒रीषु॒ ये य॑न्ति यन्ति॒ ये अ॑प॒रीषु॑ । \newline
21. ये अ॑प॒री ष्व॑प॒रीषु॒ ये ये अ॑प॒रीषु॒ पश्या॒न् पश्या॑ नप॒रीषु॒ ये ये अ॑प॒रीषु॒ पश्यान्॑ । \newline
22. अ॒प॒रीषु॒ पश्या॒न् पश्या॑ नप॒री ष्व॑प॒रीषु॒ पश्यान्॑ । \newline
23. पश्या॒निति॒ पश्यान्॑ । \newline
\pagebreak
\markright{ TS 1.4.34.1  \hfill https://www.vedavms.in \hfill}
\addcontentsline{toc}{section}{ TS 1.4.34.1 }
\section*{ TS 1.4.34.1 }

\textbf{TS 1.4.34.1 } \newline
\textbf{Samhita Paata} \newline

ज्योति॑ष्मतीं त्वा सादयामि ज्योति॒ष्कृतं॑ त्वा सादयामि ज्योति॒र्विदं॑ त्वा सादयामि॒ भास्व॑तीं त्वा सादयामि॒ ज्वल॑न्तीं त्वा सादयामि मल्मला॒भव॑न्तीं त्वा सादयामि॒ दीप्य॑मानां त्वा सादयामि॒ रोच॑मानां त्वा सादया॒म्यज॑स्रां त्वा सादयामि बृ॒हज्ज्यो॑तिषं त्वा सादयामि बो॒धय॑न्तीं त्वा सादयामि॒ जाग्र॑तीं त्वा सादयामि ॥ \newline

\textbf{Pada Paata} \newline

ज्योति॑ष्मतीम् । त्वा॒ । सा॒द॒या॒मि॒ । ज्यो॒ति॒ष्कृत॒मिति॑ ज्योतिः-कृत᳚म् । त्वा॒ । सा॒द॒या॒मि॒ । ज्यो॒ति॒र्विद॒मिति॑ ज्योतिः - विद᳚म् । त्वा॒ । सा॒द॒या॒मि॒ । भास्व॑तीम् । त्वा॒ । सा॒द॒या॒मि॒ । ज्वल॑न्तीम् । त्वा॒ । सा॒द॒या॒मि॒ । म॒ल्म॒ला॒भव॑न्ती॒मिति॑ मल्मला - भव॑न्तीम् । त्वा॒ । सा॒द॒या॒मि॒ । दीप्य॑मानाम् । त्वा॒ । सा॒द॒या॒मि॒ । रोच॑मानाम् । त्वा॒ । सा॒द॒या॒मि॒ । अज॑स्राम् । त्वा॒ । सा॒द॒या॒मि॒ । बृ॒हज्ज्यो॑तिष॒मिति॑ बृ॒हत् - ज्यो॒ति॒ष॒म् । त्वा॒ । सा॒द॒या॒मि॒ । बो॒धय॑न्तीम् । त्वा॒ । सा॒द॒या॒मि॒ । जाग्र॑तीम् । त्वा॒ । सा॒द॒या॒मि॒ ॥  \newline


\textbf{Krama Paata} \newline

ज्योति॑ष्मतीम् त्वा । त्वा॒ सा॒द॒या॒मि॒ । सा॒द॒या॒मि॒ ज्यो॒ति॒ष्कृत᳚म् । ज्यो॒ति॒ष्कृत॑म् त्वा । ज्यो॒ति॒ष्कृत॒मिति॑ ज्योतिः - कृत᳚म् । त्वा॒ सा॒द॒या॒मि॒ । सा॒द॒या॒मि॒ ज्यो॒ति॒र्विद᳚म् । ज्यो॒ति॒र्विद॑म् त्वा । ज्यो॒ति॒र्विद॒मिति॑ ज्योतिः - विद᳚म् । त्वा॒ सा॒द॒या॒मि॒ । सा॒द॒या॒मि॒ भास्व॑तीम् । भास्व॑तीम् त्वा । त्वा॒ सा॒द॒या॒मि॒ । सा॒द॒या॒मि॒ ज्वल॑न्तीम् । ज्वल॑न्तीम् त्वा । त्वा॒ सा॒द॒या॒मि॒ । सा॒द॒या॒मि॒ म॒ल्म॒ला॒भव॑न्तीम् । म॒ल्म॒ला॒भव॑न्तीम् त्वा । म॒ल्म॒ला॒भव॑न्ती॒मिति॑ मल्मला - भव॑न्तीम् । त्वा॒ सा॒द॒या॒मि॒ । सा॒द॒या॒मि॒ दीप्य॑मानाम् । दीप्य॑मानाम् त्वा । त्वा॒ सा॒द॒या॒मि॒ । सा॒द॒या॒मि॒ रोच॑मानाम् । रोच॑मानाम् त्वा । त्वा॒ सा॒द॒या॒मि॒ । सा॒द॒या॒म्यज॑स्राम् । अज॑स्राम् त्वा । त्वा॒ सा॒द॒या॒मि॒ । सा॒द॒या॒मि॒ बृ॒हज्ज्यो॑तिषम् । बृ॒हज्ज्यो॑तिषम् त्वा । बृ॒हज्ज्यो॑तिष॒मिति॑ बृ॒हत् - ज्यो॒ति॒ष॒म् । त्वा॒ सा॒द॒या॒मि॒ । सा॒द॒या॒मि॒ बो॒धय॑न्तीम् । बो॒धय॑न्तीम् त्वा । त्वा॒ सा॒द॒या॒मि॒ । सा॒द॒या॒मि॒ जाग्र॑तीम् । जाग्र॑तीम् त्वा । त्वा॒ सा॒द॒या॒मि॒ । सा॒द॒या॒मीति॑ सादयामि । \newline

\textbf{Jatai Paata} \newline

1. ज्योति॑ष्मतीम् त्वा त्वा॒ ज्योति॑ष्मती॒म् ज्योति॑ष्मतीम् त्वा । \newline
2. त्वा॒ सा॒द॒या॒मि॒ सा॒द॒या॒मि॒ त्वा॒ त्वा॒ सा॒द॒या॒मि॒ । \newline
3. सा॒द॒या॒मि॒ ज्यो॒ति॒ष्कृत॑म् ज्योति॒ष्कृत(ग्म्॑) सादयामि सादयामि ज्योति॒ष्कृत᳚म् । \newline
4. ज्यो॒ति॒ष्कृत॑म् त्वा त्वा ज्योति॒ष्कृत॑म् ज्योति॒ष्कृत॑म् त्वा । \newline
5. ज्यो॒ति॒ष्कृत॒मिति॑ ज्योतिः - कृत᳚म् । \newline
6. त्वा॒ सा॒द॒या॒मि॒ सा॒द॒या॒मि॒ त्वा॒ त्वा॒ सा॒द॒या॒मि॒ । \newline
7. सा॒द॒या॒मि॒ ज्यो॒ति॒र्विद॑म् ज्योति॒र्विद(ग्म्॑) सादयामि सादयामि ज्योति॒र्विद᳚म् । \newline
8. ज्यो॒ति॒र्विद॑म् त्वा त्वा ज्योति॒र्विद॑म् ज्योति॒र्विद॑म् त्वा । \newline
9. ज्यो॒ति॒र्विद॒मिति॑ ज्योतिः - विद᳚म् । \newline
10. त्वा॒ सा॒द॒या॒मि॒ सा॒द॒या॒मि॒ त्वा॒ त्वा॒ सा॒द॒या॒मि॒ । \newline
11. सा॒द॒या॒मि॒ भास्व॑ती॒म् भास्व॑तीꣳ सादयामि सादयामि॒ भास्व॑तीम् । \newline
12. भास्व॑तीम् त्वा त्वा॒ भास्व॑ती॒म् भास्व॑तीम् त्वा । \newline
13. त्वा॒ सा॒द॒या॒मि॒ सा॒द॒या॒मि॒ त्वा॒ त्वा॒ सा॒द॒या॒मि॒ । \newline
14. सा॒द॒या॒मि॒ ज्वल॑न्ती॒म् ज्वल॑न्तीꣳ सादयामि सादयामि॒ ज्वल॑न्तीम् । \newline
15. ज्वल॑न्तीम् त्वा त्वा॒ ज्वल॑न्ती॒म् ज्वल॑न्तीम् त्वा । \newline
16. त्वा॒ सा॒द॒या॒मि॒ सा॒द॒या॒मि॒ त्वा॒ त्वा॒ सा॒द॒या॒मि॒ । \newline
17. सा॒द॒या॒मि॒ म॒ल्म॒ला॒भव॑न्तीम् मल्मला॒भव॑न्तीꣳ सादयामि सादयामि मल्मला॒भव॑न्तीम् । \newline
18. म॒ल्म॒ला॒भव॑न्तीम् त्वा त्वा मल्मला॒भव॑न्तीम् मल्मला॒भव॑न्तीम् त्वा । \newline
19. म॒ल्म॒ला॒भव॑न्ती॒मिति॑ मल्मला - भव॑न्तीम् । \newline
20. त्वा॒ सा॒द॒या॒मि॒ सा॒द॒या॒मि॒ त्वा॒ त्वा॒ सा॒द॒या॒मि॒ । \newline
21. सा॒द॒या॒मि॒ दीप्य॑माना॒म् दीप्य॑मानाꣳ सादयामि सादयामि॒ दीप्य॑मानाम् । \newline
22. दीप्य॑मानाम् त्वा त्वा॒ दीप्य॑माना॒म् दीप्य॑मानाम् त्वा । \newline
23. त्वा॒ सा॒द॒या॒मि॒ सा॒द॒या॒मि॒ त्वा॒ त्वा॒ सा॒द॒या॒मि॒ । \newline
24. सा॒द॒या॒मि॒ रोच॑माना॒(ग्म्॒) रोच॑मानाꣳ सादयामि सादयामि॒ रोच॑मानाम् । \newline
25. रोच॑मानाम् त्वा त्वा॒ रोच॑माना॒(ग्म्॒) रोच॑मानाम् त्वा । \newline
26. त्वा॒ सा॒द॒या॒मि॒ सा॒द॒या॒मि॒ त्वा॒ त्वा॒ सा॒द॒या॒मि॒ । \newline
27. सा॒द॒या॒म्यज॑स्रा॒ मज॑स्राꣳ सादयामि सादया॒म्यज॑स्राम् । \newline
28. अज॑स्राम् त्वा॒ त्वा ऽज॑स्रा॒ मज॑स्राम् त्वा । \newline
29. त्वा॒ सा॒द॒या॒मि॒ सा॒द॒या॒मि॒ त्वा॒ त्वा॒ सा॒द॒या॒मि॒ । \newline
30. सा॒द॒या॒मि॒ बृ॒हज्ज्यो॑तिषम् बृ॒हज्ज्यो॑तिषꣳ सादयामि सादयामि बृ॒हज्ज्यो॑तिषम् । \newline
31. बृ॒हज्ज्यो॑तिषम् त्वा त्वा बृ॒हज्ज्यो॑तिषम् बृ॒हज्ज्यो॑तिषम् त्वा । \newline
32. बृ॒हज्ज्यो॑तिष॒मिति॑ बृ॒हत् - ज्यो॒ति॒ष॒म् । \newline
33. त्वा॒ सा॒द॒या॒मि॒ सा॒द॒या॒मि॒ त्वा॒ त्वा॒ सा॒द॒या॒मि॒ । \newline
34. सा॒द॒या॒मि॒ बो॒धय॑न्तीम् बो॒धय॑न्तीꣳ सादयामि सादयामि बो॒धय॑न्तीम् । \newline
35. बो॒धय॑न्तीम् त्वा त्वा बो॒धय॑न्तीम् बो॒धय॑न्तीम् त्वा । \newline
36. त्वा॒ सा॒द॒या॒मि॒ सा॒द॒या॒मि॒ त्वा॒ त्वा॒ सा॒द॒या॒मि॒ । \newline
37. सा॒द॒या॒मि॒ जाग्र॑ती॒म् जाग्र॑तीꣳ सादयामि सादयामि॒ जाग्र॑तीम् । \newline
38. जाग्र॑तीम् त्वा त्वा॒ जाग्र॑ती॒म् जाग्र॑तीम् त्वा । \newline
39. त्वा॒ सा॒द॒या॒मि॒ सा॒द॒या॒मि॒ त्वा॒ त्वा॒ सा॒द॒या॒मि॒ । \newline
40. सा॒द॒या॒मीति॑ सादयामि । \newline

\textbf{Ghana Paata } \newline

1. ज्योति॑ष्मतीम् त्वा त्वा॒ ज्योति॑ष्मती॒म् ज्योति॑ष्मतीम् त्वा सादयामि सादयामि त्वा॒ ज्योति॑ष्मती॒म् ज्योति॑ष्मतीम् त्वा सादयामि । \newline
2. त्वा॒ सा॒द॒या॒मि॒ सा॒द॒या॒मि॒ त्वा॒ त्वा॒ सा॒द॒या॒मि॒ ज्यो॒ति॒ष्कृत॑म् ज्योति॒ष्कृत(ग्म्॑) सादयामि त्वा त्वा सादयामि ज्योति॒ष्कृत᳚म् । \newline
3. सा॒द॒या॒मि॒ ज्यो॒ति॒ष्कृत॑म् ज्योति॒ष्कृत(ग्म्॑) सादयामि सादयामि ज्योति॒ष्कृत॑म् त्वा त्वा ज्योति॒ष्कृत(ग्म्॑) सादयामि सादयामि ज्योति॒ष्कृत॑म् त्वा । \newline
4. ज्यो॒ति॒ष्कृत॑म् त्वा त्वा ज्योति॒ष्कृत॑म् ज्योति॒ष्कृत॑म् त्वा सादयामि सादयामि त्वा ज्योति॒ष्कृत॑म् ज्योति॒ष्कृत॑म् त्वा सादयामि । \newline
5. ज्यो॒ति॒ष्कृत॒मिति॑ ज्योतिः - कृत᳚म् । \newline
6. त्वा॒ सा॒द॒या॒मि॒ सा॒द॒या॒मि॒ त्वा॒ त्वा॒ सा॒द॒या॒मि॒ ज्यो॒ति॒र्विद॑म् ज्योति॒र्विद(ग्म्॑) सादयामि त्वा त्वा सादयामि ज्योति॒र्विद᳚म् । \newline
7. सा॒द॒या॒मि॒ ज्यो॒ति॒र्विद॑म् ज्योति॒र्विद(ग्म्॑) सादयामि सादयामि ज्योति॒र्विद॑म् त्वा त्वा ज्योति॒र्विद(ग्म्॑) सादयामि सादयामि ज्योति॒र्विद॑म् त्वा । \newline
8. ज्यो॒ति॒र्विद॑म् त्वा त्वा ज्योति॒र्विद॑म् ज्योति॒र्विद॑म् त्वा सादयामि सादयामि त्वा ज्योति॒र्विद॑म् ज्योति॒र्विद॑म् त्वा सादयामि । \newline
9. ज्यो॒ति॒र्विद॒मिति॑ ज्योतिः - विद᳚म् । \newline
10. त्वा॒ सा॒द॒या॒मि॒ सा॒द॒या॒मि॒ त्वा॒ त्वा॒ सा॒द॒या॒मि॒ भास्व॑ती॒म् भास्व॑तीꣳ सादयामि त्वा त्वा सादयामि॒ भास्व॑तीम् । \newline
11. सा॒द॒या॒मि॒ भास्व॑ती॒म् भास्व॑तीꣳ सादयामि सादयामि॒ भास्व॑तीम् त्वा त्वा॒ भास्व॑तीꣳ सादयामि सादयामि॒ भास्व॑तीम् त्वा । \newline
12. भास्व॑तीम् त्वा त्वा॒ भास्व॑ती॒म् भास्व॑तीम् त्वा सादयामि सादयामि त्वा॒ भास्व॑ती॒म् भास्व॑तीम् त्वा सादयामि । \newline
13. त्वा॒ सा॒द॒या॒मि॒ सा॒द॒या॒मि॒ त्वा॒ त्वा॒ सा॒द॒या॒मि॒ ज्वल॑न्ती॒म् ज्वल॑न्तीꣳ सादयामि त्वा त्वा सादयामि॒ ज्वल॑न्तीम् । \newline
14. सा॒द॒या॒मि॒ ज्वल॑न्ती॒म् ज्वल॑न्तीꣳ सादयामि सादयामि॒ ज्वल॑न्तीम् त्वा त्वा॒ ज्वल॑न्तीꣳ सादयामि सादयामि॒ ज्वल॑न्तीम् त्वा । \newline
15. ज्वल॑न्तीम् त्वा त्वा॒ ज्वल॑न्ती॒म् ज्वल॑न्तीम् त्वा सादयामि सादयामि त्वा॒ ज्वल॑न्ती॒म् ज्वल॑न्तीम् त्वा सादयामि । \newline
16. त्वा॒ सा॒द॒या॒मि॒ सा॒द॒या॒मि॒ त्वा॒ त्वा॒ सा॒द॒या॒मि॒ म॒ल्म॒ला॒भव॑न्तीम् मल्मला॒भव॑न्तीꣳ सादयामि त्वा त्वा सादयामि मल्मला॒भव॑न्तीम् । \newline
17. सा॒द॒या॒मि॒ म॒ल्म॒ला॒भव॑न्तीम् मल्मला॒भव॑न्तीꣳ सादयामि सादयामि मल्मला॒भव॑न्तीम् त्वा त्वा मल्मला॒भव॑न्तीꣳ सादयामि सादयामि मल्मला॒भव॑न्तीम् त्वा । \newline
18. म॒ल्म॒ला॒भव॑न्तीम् त्वा त्वा मल्मला॒भव॑न्तीम् मल्मला॒भव॑न्तीम् त्वा सादयामि सादयामि त्वा मल्मला॒भव॑न्तीम् मल्मला॒भव॑न्तीम् त्वा सादयामि । \newline
19. म॒ल्म॒ला॒भव॑न्ती॒मिति॑ मल्मला - भव॑न्तीम् । \newline
20. त्वा॒ सा॒द॒या॒मि॒ सा॒द॒या॒मि॒ त्वा॒ त्वा॒ सा॒द॒या॒मि॒ दीप्य॑माना॒म् दीप्य॑मानाꣳ सादयामि त्वा त्वा सादयामि॒ दीप्य॑मानाम् । \newline
21. सा॒द॒या॒मि॒ दीप्य॑माना॒म् दीप्य॑मानाꣳ सादयामि सादयामि॒ दीप्य॑मानाम् त्वा त्वा॒ दीप्य॑मानाꣳ सादयामि सादयामि॒ दीप्य॑मानाम् त्वा । \newline
22. दीप्य॑मानाम् त्वा त्वा॒ दीप्य॑माना॒म् दीप्य॑मानाम् त्वा सादयामि सादयामि त्वा॒ दीप्य॑माना॒म् दीप्य॑मानाम् त्वा सादयामि । \newline
23. त्वा॒ सा॒द॒या॒मि॒ सा॒द॒या॒मि॒ त्वा॒ त्वा॒ सा॒द॒या॒मि॒ रोच॑माना॒(ग्म्॒) रोच॑मानाꣳ सादयामि त्वा त्वा सादयामि॒ रोच॑मानाम् । \newline
24. सा॒द॒या॒मि॒ रोच॑माना॒(ग्म्॒) रोच॑मानाꣳ सादयामि सादयामि॒ रोच॑मानाम् त्वा त्वा॒ रोच॑मानाꣳ सादयामि सादयामि॒ रोच॑मानाम् त्वा । \newline
25. रोच॑मानाम् त्वा त्वा॒ रोच॑माना॒(ग्म्॒) रोच॑मानाम् त्वा सादयामि सादयामि त्वा॒ रोच॑माना॒(ग्म्॒) रोच॑मानाम् त्वा सादयामि । \newline
26. त्वा॒ सा॒द॒या॒मि॒ सा॒द॒या॒मि॒ त्वा॒ त्वा॒ सा॒द॒या॒म्यज॑स्रा॒ मज॑स्राꣳ सादयामि त्वा त्वा सादया॒म्यज॑स्राम् । \newline
27. सा॒द॒या॒ म्यज॑स्रा॒ मज॑स्राꣳ सादयामि सादया॒ म्यज॑स्राम् त्वा॒ त्वा ऽज॑स्राꣳ सादयामि सादया॒ म्यज॑स्राम् त्वा । \newline
28. अज॑स्राम् त्वा॒ त्वा ऽज॑स्रा॒ मज॑स्राम् त्वा सादयामि सादयामि॒ त्वा ऽज॑स्रा॒ मज॑स्राम् त्वा सादयामि । \newline
29. त्वा॒ सा॒द॒या॒मि॒ सा॒द॒या॒मि॒ त्वा॒ त्वा॒ सा॒द॒या॒मि॒ बृ॒हज्ज्यो॑तिषम् बृ॒हज्ज्यो॑तिषꣳ सादयामि त्वा त्वा सादयामि बृ॒हज्ज्यो॑तिषम् । \newline
30. सा॒द॒या॒मि॒ बृ॒हज्ज्यो॑तिषम् बृ॒हज्ज्यो॑तिषꣳ सादयामि सादयामि बृ॒हज्ज्यो॑तिषम् त्वा त्वा बृ॒हज्ज्यो॑तिषꣳ सादयामि सादयामि बृ॒हज्ज्यो॑तिषम् त्वा । \newline
31. बृ॒हज्ज्यो॑तिषम् त्वा त्वा बृ॒हज्ज्यो॑तिषम् बृ॒हज्ज्यो॑तिषम् त्वा सादयामि सादयामि त्वा बृ॒हज्ज्यो॑तिषम् बृ॒हज्ज्यो॑तिषम् त्वा सादयामि । \newline
32. बृ॒हज्ज्यो॑तिष॒मिति॑ बृ॒हत् - ज्यो॒ति॒ष॒म् । \newline
33. त्वा॒ सा॒द॒या॒मि॒ सा॒द॒या॒मि॒ त्वा॒ त्वा॒ सा॒द॒या॒मि॒ बो॒धय॑न्तीम् बो॒धय॑न्तीꣳ सादयामि त्वा त्वा सादयामि बो॒धय॑न्तीम् । \newline
34. सा॒द॒या॒मि॒ बो॒धय॑न्तीम् बो॒धय॑न्तीꣳ सादयामि सादयामि बो॒धय॑न्तीम् त्वा त्वा बो॒धय॑न्तीꣳ सादयामि सादयामि बो॒धय॑न्तीम् त्वा । \newline
35. बो॒धय॑न्तीम् त्वा त्वा बो॒धय॑न्तीम् बो॒धय॑न्तीम् त्वा सादयामि सादयामि त्वा बो॒धय॑न्तीम् बो॒धय॑न्तीम् त्वा सादयामि । \newline
36. त्वा॒ सा॒द॒या॒मि॒ सा॒द॒या॒मि॒ त्वा॒ त्वा॒ सा॒द॒या॒मि॒ जाग्र॑ती॒म् जाग्र॑तीꣳ सादयामि त्वा त्वा सादयामि॒ जाग्र॑तीम् । \newline
37. सा॒द॒या॒मि॒ जाग्र॑ती॒म् जाग्र॑तीꣳ सादयामि सादयामि॒ जाग्र॑तीम् त्वा त्वा॒ जाग्र॑तीꣳ सादयामि सादयामि॒ जाग्र॑तीम् त्वा । \newline
38. जाग्र॑तीम् त्वा त्वा॒ जाग्र॑ती॒म् जाग्र॑तीम् त्वा सादयामि सादयामि त्वा॒ जाग्र॑ती॒म् जाग्र॑तीम् त्वा सादयामि । \newline
39. त्वा॒ सा॒द॒या॒मि॒ सा॒द॒या॒मि॒ त्वा॒ त्वा॒ सा॒द॒या॒मि॒ । \newline
40. सा॒द॒या॒मीति॑ सादयामि । \newline
\pagebreak
\markright{ TS 1.4.35.1  \hfill https://www.vedavms.in \hfill}
\addcontentsline{toc}{section}{ TS 1.4.35.1 }
\section*{ TS 1.4.35.1 }

\textbf{TS 1.4.35.1 } \newline
\textbf{Samhita Paata} \newline

प्र॒या॒साय॒ स्वाहा॑ ऽऽया॒साय॒ स्वाहा॑ विया॒साय॒ स्वाहा॑ संॅया॒साय॒ स्वाहो᳚द्या॒साय॒ स्वाहा॑ऽवया॒साय॒ स्वाहा॑ शु॒चे स्वाहा॒ शोका॑य॒ स्वाहा॑ तप्य॒त्वै स्वाहा॒ तप॑ते॒ स्वाहा᳚ ब्रह्मह॒त्यायै॒ स्वाहा॒ सर्व॑स्मै॒ स्वाहा᳚ ॥ \newline

\textbf{Pada Paata} \newline

प्र॒या॒सायेति॑ प्र - या॒साय॑ । स्वाहा᳚ । आ॒या॒सायेत्या᳚ - या॒साय॑ । स्वाहा᳚ । वि॒या॒सायेति॑ वि - या॒साय॑ । स्वाहा᳚ । स॒म्ॅया॒सायेति॑ सं - या॒साय॑ । स्वाहा᳚ । उ॒द्या॒सायेत्यु॑त् - या॒साय॑ । स्वाहा᳚ । अ॒व॒या॒सायेत्य॑व - या॒साय॑ । स्वाहा᳚ । शु॒चे । स्वाहा᳚ । शोका॑य । स्वाहा᳚ । त॒प्य॒त्वै । स्वाहा᳚ । तप॑ते । स्वाहा᳚ । ब्र॒ह्म॒ह॒त्याया॒ इति॑ ब्रह्म - ह॒त्यायै᳚ । स्वाहा᳚ । सर्व॑स्मै । स्वाहा᳚ ॥  \newline


\textbf{Krama Paata} \newline

प्र॒या॒साय॒ स्वाहा᳚ । प्र॒या॒सायेति॑ प्र - या॒साय॑ । स्वाहा॑ ऽऽया॒साय॑ । आ॒या॒साय॒ स्वाहा᳚ । आ॒या॒सायेत्या᳚ - या॒साय॑ । स्वाहा॑ विया॒साय॑ । वि॒या॒साय॒ स्वाहा᳚ । वि॒या॒सायेति॑ वि - या॒साय॑ । स्वाहा॑ सम्ॅया॒साय॑ । स॒म्ॅया॒साय॒ स्वाहा᳚ । स॒म्ॅया॒सायेति॑ सम् - या॒साय॑ । स्वाहो᳚द्या॒साय॑ । उ॒द्या॒साय॒ स्वाहा᳚ । उ॒द्या॒सायेत्यु॑त् - या॒साय॑ । स्वाहा॑ऽवया॒साय॑ । अ॒व॒या॒साय॒ स्वाहा᳚ । अ॒व॒या॒सायेत्य॑व - या॒साय॑ । स्वाहा॑ शु॒चे । शु॒चे स्वाहा᳚ । स्वाहा॒ शोका॑य । शोका॑य॒ स्वाहा᳚ । स्वाहा॑ तप्य॒त्वै । त॒प्य॒त्वै स्वाहा᳚ । स्वाहा॒ तप॑ते । तप॑ते॒ स्वाहा᳚ । स्वाहा᳚ ब्रह्मह॒त्यायै᳚ । ब्र॒ह्म॒ह॒त्यायै॒ स्वाहा᳚ । ब्र॒ह्म॒ह॒त्याया॒ इति॑ ब्रह्म - ह॒त्यायै᳚ । स्वाहा॒ सर्व॑स्मै । सर्व॑स्मै॒ स्वाहा᳚ । स्वाहेति॒ स्वाहा᳚ । \newline

\textbf{Jatai Paata} \newline

1. प्र॒या॒साय॒ स्वाहा॒ स्वाहा᳚ प्रया॒साय॑ प्रया॒साय॒ स्वाहा᳚ । \newline
2. प्र॒या॒सायेति॑ प्र - या॒साय॑ । \newline
3. स्वाहा॑ ऽऽया॒साया॑या॒साय॒ स्वाहा॒ स्वाहा॑ ऽऽया॒साय॑ । \newline
4. आ॒या॒साय॒ स्वाहा॒ स्वाहा॑ ऽऽया॒साया॑या॒साय॒ स्वाहा᳚ । \newline
5. आ॒या॒सायेत्या᳚ - या॒साय॑ । \newline
6. स्वाहा॑ विया॒साय॑ विया॒साय॒ स्वाहा॒ स्वाहा॑ विया॒साय॑ । \newline
7. वि॒या॒साय॒ स्वाहा॒ स्वाहा॑ विया॒साय॑ विया॒साय॒ स्वाहा᳚ । \newline
8. वि॒या॒सायेति॑ वि - या॒साय॑ । \newline
9. स्वाहा॑ सम्ॅया॒साय॑ सम्ॅया॒साय॒ स्वाहा॒ स्वाहा॑ सम्ॅया॒साय॑ । \newline
10. स॒म्ॅया॒साय॒ स्वाहा॒ स्वाहा॑ सम्ॅया॒साय॑ सम्ॅया॒साय॒ स्वाहा᳚ । \newline
11. स॒म्ॅया॒सायेति॑ सं - या॒साय॑ । \newline
12. स्वाहो᳚द्या॒ सायो᳚द्या॒साय॒ स्वाहा॒ स्वाहो᳚द्या॒साय॑ । \newline
13. उ॒द्या॒साय॒ स्वाहा॒ स्वाहो᳚द्या॒ सायो᳚द्या॒साय॒ स्वाहा᳚ । \newline
14. उ॒द्या॒सायेत्यु॑त् - या॒साय॑ । \newline
15. स्वाहा॑ ऽवया॒सा या॑वया॒साय॒ स्वाहा॒ स्वाहा॑ ऽवया॒साय॑ । \newline
16. अ॒व॒या॒साय॒ स्वाहा॒ स्वाहा॑ ऽवया॒सा या॑वया॒साय॒ स्वाहा᳚ । \newline
17. अ॒व॒या॒सायेत्य॑व - या॒साय॑ । \newline
18. स्वाहा॑ शु॒चे शु॒चे स्वाहा॒ स्वाहा॑ शु॒चे । \newline
19. शु॒चे स्वाहा॒ स्वाहा॑ शु॒चे शु॒चे स्वाहा᳚ । \newline
20. स्वाहा॒ शोका॑य॒ शोका॑य॒ स्वाहा॒ स्वाहा॒ शोका॑य । \newline
21. शोका॑य॒ स्वाहा॒ स्वाहा॒ शोका॑य॒ शोका॑य॒ स्वाहा᳚ । \newline
22. स्वाहा॑ तप्य॒त्वै त॑प्य॒त्वै स्वाहा॒ स्वाहा॑ तप्य॒त्वै । \newline
23. त॒प्य॒त्वै स्वाहा॒ स्वाहा॑ तप्य॒त्वै त॑प्य॒त्वै स्वाहा᳚ । \newline
24. स्वाहा॒ तप॑ते॒ तप॑ते॒ स्वाहा॒ स्वाहा॒ तप॑ते । \newline
25. तप॑ते॒ स्वाहा॒ स्वाहा॒ तप॑ते॒ तप॑ते॒ स्वाहा᳚ । \newline
26. स्वाहा᳚ ब्रह्मह॒त्यायै᳚ ब्रह्मह॒त्यायै॒ स्वाहा॒ स्वाहा᳚ ब्रह्मह॒त्यायै᳚ । \newline
27. ब्र॒ह्म॒ह॒त्यायै॒ स्वाहा॒ स्वाहा᳚ ब्रह्मह॒त्यायै᳚ ब्रह्मह॒त्यायै॒ स्वाहा᳚ । \newline
28. ब्र॒ह्म॒ह॒त्याया॒ इति॑ ब्रह्म - ह॒त्यायै᳚ । \newline
29. स्वाहा॒ सर्व॑स्मै॒ सर्व॑स्मै॒ स्वाहा॒ स्वाहा॒ सर्व॑स्मै । \newline
30. सर्व॑स्मै॒ स्वाहा॒ स्वाहा॒ सर्व॑स्मै॒ सर्व॑स्मै॒ स्वाहा᳚ । \newline
31. स्वाहेति॒ स्वाहा᳚ । \newline

\textbf{Ghana Paata } \newline

1. प्र॒या॒साय॒ स्वाहा॒ स्वाहा᳚ प्रया॒साय॑ प्रया॒साय॒ स्वाहा॑ ऽऽया॒सा या॑या॒साय॒ स्वाहा᳚ प्रया॒साय॑ प्रया॒साय॒ स्वाहा॑ ऽऽया॒साय॑ । \newline
2. प्र॒या॒सायेति॑ प्र - या॒साय॑ । \newline
3. स्वाहा॑ ऽऽया॒सा या॑या॒साय॒ स्वाहा॒ स्वाहा॑ ऽऽया॒साय॒ स्वाहा॒ स्वाहा॑ ऽऽया॒साय॒ स्वाहा॒ स्वाहा॑ ऽऽया॒साय॒ स्वाहा᳚ । \newline
4. आ॒या॒साय॒ स्वाहा॒ स्वाहा॑ ऽऽया॒सा या॑या॒साय॒ स्वाहा॑ विया॒साय॑ विया॒साय॒ स्वाहा॑ ऽऽया॒सा या॑या॒साय॒ स्वाहा॑ विया॒साय॑ । \newline
5. आ॒या॒सायेत्या᳚ - या॒साय॑ । \newline
6. स्वाहा॑ विया॒साय॑ विया॒साय॒ स्वाहा॒ स्वाहा॑ विया॒साय॒ स्वाहा॒ स्वाहा॑ विया॒साय॒ स्वाहा॒ स्वाहा॑ विया॒साय॒ स्वाहा᳚ । \newline
7. वि॒या॒साय॒ स्वाहा॒ स्वाहा॑ विया॒साय॑ विया॒साय॒ स्वाहा॑ सम्ॅया॒साय॑ सम्ॅया॒साय॒ स्वाहा॑ विया॒साय॑ विया॒साय॒ स्वाहा॑ सम्ॅया॒साय॑ । \newline
8. वि॒या॒सायेति॑ वि - या॒साय॑ । \newline
9. स्वाहा॑ सम्ॅया॒साय॑ सम्ॅया॒साय॒ स्वाहा॒ स्वाहा॑ सम्ॅया॒साय॒ स्वाहा॒ स्वाहा॑ सम्ॅया॒साय॒ स्वाहा॒ स्वाहा॑ सम्ॅया॒साय॒ स्वाहा᳚ । \newline
10. स॒म्ॅया॒साय॒ स्वाहा॒ स्वाहा॑ सम्ॅया॒साय॑ सम्ॅया॒साय॒ स्वाहो᳚द्या॒सायो᳚द्या॒साय॒ स्वाहा॑ सम्ॅया॒साय॑ सम्ॅया॒साय॒ स्वाहो᳚द्या॒साय॑ । \newline
11. स॒म्ॅया॒सायेति॑ सं - या॒साय॑ । \newline
12. स्वाहो᳚द्या॒सायो᳚द्या॒साय॒ स्वाहा॒ स्वाहो᳚द्या॒साय॒ स्वाहा॒ स्वाहो᳚द्या॒साय॒ स्वाहा॒ स्वाहो᳚द्या॒साय॒ स्वाहा᳚ । \newline
13. उ॒द्या॒साय॒ स्वाहा॒ स्वाहो᳚द्या॒सा यो᳚द्या॒साय॒ स्वाहा॑ ऽवया॒सा या॑वया॒साय॒ स्वा हो᳚द्या॒ सायो᳚द्या॒साय॒ स्वाहा॑ ऽवया॒साय॑ । \newline
14. उ॒द्या॒सायेत्यु॑त् - या॒साय॑ । \newline
15. स्वाहा॑ ऽवया॒सा या॑वया॒साय॒ स्वाहा॒ स्वाहा॑ ऽवया॒साय॒ स्वाहा॒ स्वाहा॑ ऽवया॒साय॒ स्वाहा॒ स्वाहा॑ ऽवया॒साय॒ स्वाहा᳚ । \newline
16. अ॒व॒या॒साय॒ स्वाहा॒ स्वाहा॑ ऽवया॒सा या॑वया॒साय॒ स्वाहा॑ शु॒चे शु॒चे स्वाहा॑ ऽवया॒सा या॑वया॒साय॒ स्वाहा॑ शु॒चे । \newline
17. अ॒व॒या॒सायेत्य॑व - या॒साय॑ । \newline
18. स्वाहा॑ शु॒चे शु॒चे स्वाहा॒ स्वाहा॑ शु॒चे स्वाहा॒ स्वाहा॑ शु॒चे स्वाहा॒ स्वाहा॑ शु॒चे स्वाहा᳚ । \newline
19. शु॒चे स्वाहा॒ स्वाहा॑ शु॒चे शु॒चे स्वाहा॒ शोका॑य॒ शोका॑य॒ स्वाहा॑ शु॒चे शु॒चे स्वाहा॒ शोका॑य । \newline
20. स्वाहा॒ शोका॑य॒ शोका॑य॒ स्वाहा॒ स्वाहा॒ शोका॑य॒ स्वाहा॒ स्वाहा॒ शोका॑य॒ स्वाहा॒ स्वाहा॒ शोका॑य॒ स्वाहा᳚ । \newline
21. शोका॑य॒ स्वाहा॒ स्वाहा॒ शोका॑य॒ शोका॑य॒ स्वाहा॑ तप्य॒त्वै त॑प्य॒त्वै स्वाहा॒ शोका॑य॒ शोका॑य॒ स्वाहा॑ तप्य॒त्वै । \newline
22. स्वाहा॑ तप्य॒त्वै त॑प्य॒त्वै स्वाहा॒ स्वाहा॑ तप्य॒त्वै स्वाहा॒ स्वाहा॑ तप्य॒त्वै स्वाहा॒ स्वाहा॑ तप्य॒त्वै स्वाहा᳚ । \newline
23. त॒प्य॒त्वै स्वाहा॒ स्वाहा॑ तप्य॒त्वै त॑प्य॒त्वै स्वाहा॒ तप॑ते॒ तप॑ते॒ स्वाहा॑ तप्य॒त्वै त॑प्य॒त्वै स्वाहा॒ तप॑ते । \newline
24. स्वाहा॒ तप॑ते॒ तप॑ते॒ स्वाहा॒ स्वाहा॒ तप॑ते॒ स्वाहा॒ स्वाहा॒ तप॑ते॒ स्वाहा॒ स्वाहा॒ तप॑ते॒ स्वाहा᳚ । \newline
25. तप॑ते॒ स्वाहा॒ स्वाहा॒ तप॑ते॒ तप॑ते॒ स्वाहा᳚ ब्रह्मह॒त्यायै᳚ ब्रह्मह॒त्यायै॒ स्वाहा॒ तप॑ते॒ तप॑ते॒ स्वाहा᳚ ब्रह्मह॒त्यायै᳚ । \newline
26. स्वाहा᳚ ब्रह्मह॒त्यायै᳚ ब्रह्मह॒त्यायै॒ स्वाहा॒ स्वाहा᳚ ब्रह्मह॒त्यायै॒ स्वाहा॒ स्वाहा᳚ ब्रह्मह॒त्यायै॒ स्वाहा॒ स्वाहा᳚ ब्रह्मह॒त्यायै॒ स्वाहा᳚ । \newline
27. ब्र॒ह्म॒ह॒त्यायै॒ स्वाहा॒ स्वाहा᳚ ब्रह्मह॒त्यायै᳚ ब्रह्मह॒त्यायै॒ स्वाहा॒ सर्व॑स्मै॒ सर्व॑स्मै॒ स्वाहा᳚ 
ब्रह्मह॒त्यायै᳚ ब्रह्मह॒त्यायै॒ स्वाहा॒ सर्व॑स्मै । \newline
28. ब्र॒ह्म॒ह॒त्याया॒ इति॑ ब्रह्म - ह॒त्यायै᳚ । \newline
29. स्वाहा॒ सर्व॑स्मै॒ सर्व॑स्मै॒ स्वाहा॒ स्वाहा॒ सर्व॑स्मै॒ स्वाहा॒ स्वाहा॒ सर्व॑स्मै॒ स्वाहा॒ स्वाहा॒ सर्व॑स्मै॒ स्वाहा᳚ । \newline
30. सर्व॑स्मै॒ स्वाहा॒ स्वाहा॒ सर्व॑स्मै॒ सर्व॑स्मै॒ स्वाहा᳚ । \newline
31. स्वाहेति॒ स्वाहा᳚ । \newline
\pagebreak
\markright{ TS 1.4.36.1  \hfill https://www.vedavms.in \hfill}
\addcontentsline{toc}{section}{ TS 1.4.36.1 }
\section*{ TS 1.4.36.1 }

\textbf{TS 1.4.36.1 } \newline

\textbf{Pada Paata} \newline

चि॒त्तम् । स॒तां॒नेनेति॑ सं - ता॒नेन॑ । भ॒वम् । य॒क्ना । रु॒द्रम् । तनि॑म्ना । प॒शु॒पति॒मिति॑ पशु - पति᳚म् । स्थू॒ल॒हृ॒द॒येनेति॑ स्थूल - हृ॒द॒येन॑ । अ॒ग्निम् । हृद॑येन । रु॒द्रम् । लोहि॑तेन । श॒र्वम् । मत॑स्नाभ्याम् । म॒हा॒दे॒वमिति॑ महा - दे॒वम् । अ॒न्तः पा᳚र्श्वे॒नेत्य॒न्तः - पा॒र्श्वे॒न॒ । ओ॒षि॒ष्ठ॒हन॒मित्यो॑षिष्ठ - हन᳚म् । शि॒ङ्गी॒नि॒को॒श्या᳚भ्या॒मिति॑ शिङ्गी - नि॒को॒श्या᳚भ्याम् ॥  \newline


\textbf{Krama Paata} \newline

चि॒त्तꣳ स॑न्ता॒नेन॑ । स॒न्ता॒नेन॑ भ॒वम् । स॒न्ता॒नेनेति॑ सम् - ता॒नेन॑ । भ॒वं ॅय॒क्ना । य॒क्ना रु॒द्रम् । रु॒द्रम् तनि॑म्ना । तनि॑म्ना पशु॒पति᳚म् । प॒शु॒पतिꣳ॑ स्थूलहृद॒येन॑ । प॒शु॒पति॒मिति॑ पशु - पति᳚म् । स्थू॒ल॒हृ॒द॒येना॒ग्निम् । स्थू॒ल॒हृ॒द॒येनेति॑ स्थूल - हृ॒द॒येन॑ । अ॒ग्निꣳ हृद॑येन । हृद॑येन रु॒द्रम् । रु॒द्रं ॅलोहि॑तेन । लोहि॑तेन श॒र्वम् । श॒र्वम् मत॑स्नाभ्याम् । मत॑स्नाभ्याम् महादे॒वम् । म॒हा॒दे॒वम॒न्तःपा᳚र्श्वेन । म॒हा॒दे॒वमिति॑महा - दे॒वम् । अ॒न्तःपा᳚र्श्वेनौषिष्ठ॒हन᳚म् । अ॒न्तःपा᳚र्श्वे॒नेत्य॒न्तः - पा॒र्श्वे॒न॒ । ओ॒षि॒ष्ठ॒हनꣳ॑ शिङ्गीनिको॒श्या᳚भ्याम् । ओ॒षि॒ष्ठ॒हन॒मित्यो॑षिष्ठ - हन᳚म् । शि॒ङ्गी॒नि॒को॒श्या᳚भ्या॒मिति॑ शिङ्गी - नि॒को॒श्या᳚भ्याम् । \newline

\textbf{Jatai Paata} \newline

1. चि॒त्तꣳ स॑न्ता॒नेन॑ सन्ता॒नेन॑ चि॒त्तम् चि॒त्तꣳ स॑न्ता॒नेन॑ । \newline
2. स॒न्ता॒नेन॑ भ॒वम् भ॒वꣳ स॑न्ता॒नेन॑ सन्ता॒नेन॑ भ॒वम् । \newline
3. स॒न्ता॒नेनेति॑ सं - ता॒नेन॑ । \newline
4. भ॒वं ॅय॒क्ना य॒क्ना भ॒वम् भ॒वं ॅय॒क्ना । \newline
5. य॒क्ना रु॒द्रꣳ रु॒द्रं ॅय॒क्ना य॒क्ना रु॒द्रम् । \newline
6. रु॒द्रम् तनि॑म्ना॒ तनि॑म्ना रु॒द्रꣳ रु॒द्रम् तनि॑म्ना । \newline
7. तनि॑म्ना पशु॒पति॑म् पशु॒पति॒म् तनि॑म्ना॒ तनि॑म्ना पशु॒पति᳚म् । \newline
8. प॒शु॒पति(ग्ग्॑) स्थूलहृद॒येन॑ स्थूलहृद॒येन॑ पशु॒पति॑म् पशु॒पति(ग्ग्॑) स्थूलहृद॒येन॑ । \newline
9. प॒शु॒पति॒मिति॑ पशु - पति᳚म् । \newline
10. स्थू॒ल॒हृ॒द॒येना॒ग्नि म॒ग्निꣳ स्थू॑लहृद॒येन॑ स्थूलहृद॒येना॒ग्निम् । \newline
11. स्थू॒ल॒हृ॒द॒येनेति॑ स्थूल - हृ॒द॒येन॑ । \newline
12. अ॒ग्निꣳ हृद॑येन॒ हृद॑येना॒ग्नि म॒ग्निꣳ हृद॑येन । \newline
13. हृद॑येन रु॒द्रꣳ रु॒द्रꣳ हृद॑येन॒ हृद॑येन रु॒द्रम् । \newline
14. रु॒द्रम् ॅलोहि॑तेन॒ लोहि॑तेन रु॒द्रꣳ रु॒द्रम् ॅलोहि॑तेन । \newline
15. लोहि॑तेन श॒र्वꣳ श॒र्वम् ॅलोहि॑तेन॒ लोहि॑तेन श॒र्वम् । \newline
16. श॒र्वम् मत॑स्नाभ्या॒म् मत॑स्नाभ्याꣳ श॒र्वꣳ श॒र्वम् मत॑स्नाभ्याम् । \newline
17. मत॑स्नाभ्याम् महादे॒वम् म॑हादे॒वम् मत॑स्नाभ्या॒म् मत॑स्नाभ्याम् महादे॒वम् । \newline
18. म॒हा॒दे॒व म॒न्तःपा᳚र्श्वे ना॒न्तःपा᳚र्श्वेन महादे॒वम् म॑हादे॒व म॒न्तःपा᳚र्श्वेन । \newline
19. म॒हा॒दे॒वमिति॑ महा - दे॒वम् । \newline
20. अ॒न्तःपा᳚र्श्वे नौषिष्ठ॒हन॑ मोषिष्ठ॒हन॑ म॒न्तःपा᳚र्श्वे ना॒न्तःपा᳚र्श्वे नौषिष्ठ॒हन᳚म् । \newline
21. अ॒न्तःपा᳚र्श्वे॒नेत्य॒न्तः - पा॒र्श्वे॒न॒ । \newline
22. ओ॒षि॒ष्ठ॒हन(ग्म्॑) शिङ्गीनिको॒श्या᳚भ्याꣳ शिङ्गीनिको॒श्या᳚भ्या मोषिष्ठ॒हन॑ मोषिष्ठ॒हन(ग्म्॑) शिङ्गीनिको॒श्या᳚भ्याम् । \newline
23. ओ॒षि॒ष्ठ॒हन॒मित्यो॑षिष्ठ - हन᳚म् । \newline
24. शि॒ङ्गी॒नि॒को॒श्या᳚भ्या॒मिति॑ शिङ्गी - नि॒को॒श्या᳚भ्याम् । \newline

\textbf{Ghana Paata } \newline

1. चि॒त्तꣳ स॑न्ता॒नेन॑ सन्ता॒नेन॑ चि॒त्तम् चि॒त्तꣳ स॑न्ता॒नेन॑ भ॒वम् भ॒वꣳ स॑न्ता॒नेन॑ चि॒त्तम् चि॒त्तꣳ स॑न्ता॒नेन॑ भ॒वम् । \newline
2. स॒न्ता॒नेन॑ भ॒वम् भ॒वꣳ स॑न्ता॒नेन॑ सन्ता॒नेन॑ भ॒वं ॅय॒क्ना य॒क्ना भ॒वꣳ स॑न्ता॒नेन॑ सन्ता॒नेन॑ भ॒वं ॅय॒क्ना । \newline
3. स॒न्ता॒नेनेति॑ सं - ता॒नेन॑ । \newline
4. भ॒वं ॅय॒क्ना य॒क्ना भ॒वम् भ॒वं ॅय॒क्ना रु॒द्रꣳ रु॒द्रं ॅय॒क्ना भ॒वम् भ॒वं ॅय॒क्ना रु॒द्रम् । \newline
5. य॒क्ना रु॒द्रꣳ रु॒द्रं ॅय॒क्ना य॒क्ना रु॒द्रम् तनि॑म्ना॒ तनि॑म्ना रु॒द्रं ॅय॒क्ना य॒क्ना रु॒द्रम् तनि॑म्ना । \newline
6. रु॒द्रम् तनि॑म्ना॒ तनि॑म्ना रु॒द्रꣳ रु॒द्रम् तनि॑म्ना पशु॒पति॑म् पशु॒पति॒म् तनि॑म्ना रु॒द्रꣳ रु॒द्रम् तनि॑म्ना पशु॒पति᳚म् । \newline
7. तनि॑म्ना पशु॒पति॑म् पशु॒पति॒म् तनि॑म्ना॒ तनि॑म्ना पशु॒पति(ग्ग्॑) स्थूलहृद॒येन॑ स्थूलहृद॒येन॑ पशु॒पति॒म् तनि॑म्ना॒ तनि॑म्ना पशु॒पति(ग्ग्॑) स्थूलहृद॒येन॑ । \newline
8. प॒शु॒पति(ग्ग्॑) स्थूलहृद॒येन॑ स्थूलहृद॒येन॑ पशु॒पति॑म् पशु॒पति(ग्ग्॑) स्थूलहृद॒येना॒ग्नि म॒ग्निꣳ स्थू॑लहृद॒येन॑ पशु॒पति॑म् पशु॒पति(ग्ग्॑) स्थूलहृद॒येना॒ग्निम् । \newline
9. प॒शु॒पति॒मिति॑ पशु - पति᳚म् । \newline
10. स्थू॒ल॒हृ॒द॒येना॒ग्नि म॒ग्निꣳ स्थू॑लहृद॒येन॑ स्थूलहृद॒ये ना॒ग्निꣳ हृद॑येन॒ हृद॑येना॒ग्निꣳ स्थू॑लहृद॒येन॑ स्थूलहृद॒ये ना॒ग्निꣳ हृद॑येन । \newline
11. स्थू॒ल॒हृ॒द॒येनेति॑ स्थूल - हृ॒द॒येन॑ । \newline
12. अ॒ग्निꣳ हृद॑येन॒ हृद॑येना॒ग्नि म॒ग्निꣳ हृद॑येन रु॒द्रꣳ रु॒द्रꣳ हृद॑येना॒ग्नि म॒ग्निꣳ हृद॑येन रु॒द्रम् । \newline
13. हृद॑येन रु॒द्रꣳ रु॒द्रꣳ हृद॑येन॒ हृद॑येन रु॒द्रम् ॅलोहि॑तेन॒ लोहि॑तेन रु॒द्रꣳ हृद॑येन॒ हृद॑येन रु॒द्रम् ॅलोहि॑तेन । \newline
14. रु॒द्रम् ॅलोहि॑तेन॒ लोहि॑तेन रु॒द्रꣳ रु॒द्रम् ॅलोहि॑तेन श॒र्वꣳ श॒र्वम् ॅलोहि॑तेन रु॒द्रꣳ रु॒द्रम् ॅलोहि॑तेन श॒र्वम् । \newline
15. लोहि॑तेन श॒र्वꣳ श॒र्वम् ॅलोहि॑तेन॒ लोहि॑तेन श॒र्वम् मत॑स्नाभ्या॒म् मत॑स्नाभ्याꣳ श॒र्वम् ॅलोहि॑तेन॒ लोहि॑तेन श॒र्वम् मत॑स्नाभ्याम् । \newline
16. श॒र्वम् मत॑स्नाभ्या॒म् मत॑स्नाभ्याꣳ श॒र्वꣳ श॒र्वम् मत॑स्नाभ्याम् महादे॒वम् म॑हादे॒वम् मत॑स्नाभ्याꣳ श॒र्वꣳ श॒र्वम् मत॑स्नाभ्याम् महादे॒वम् । \newline
17. मत॑स्नाभ्याम् महादे॒वम् म॑हादे॒वम् मत॑स्नाभ्या॒म् मत॑स्नाभ्याम् महादे॒व म॒न्तःपा᳚र्श्वेना॒न्तःपा᳚र्श्वेन महादे॒वम् मत॑स्नाभ्या॒म् मत॑स्नाभ्याम् महादे॒व म॒न्तःपा᳚र्श्वेन । \newline
18. म॒हा॒दे॒व म॒न्तःपा᳚र्श्वे ना॒न्तःपा᳚र्श्वेन महादे॒वम् म॑हादे॒व म॒न्तःपा᳚र्श्वे नौषिष्ठ॒हन॑ मोषिष्ठ॒हन॑ म॒न्तःपा᳚र्श्वेन महादे॒वम् म॑हादे॒व म॒न्तःपा᳚र्श्वे नौषिष्ठ॒हन᳚म् । \newline
19. म॒हा॒दे॒वमिति॑ महा - दे॒वम् । \newline
20. अ॒न्तःपा᳚र्श्वे नौषिष्ठ॒हन॑ मोषिष्ठ॒हन॑ म॒न्तःपा᳚र्श्वे ना॒न्तःपा᳚र्श्वे नौषिष्ठ॒हन(ग्म्॑) शिङ्गीनिको॒श्या᳚भ्याꣳ शिङ्गीनिको॒श्या᳚भ्या मोषिष्ठ॒हन॑ म॒न्तःपा᳚र्श्वे ना॒न्तःपा᳚र्श्वे नौषिष्ठ॒हन(ग्म्॑) शिङ्गीनिको॒श्या᳚भ्याम् । \newline
21. अ॒न्तःपा᳚र्श्वे॒नेत्य॒न्तः - पा॒र्श्वे॒न॒ । \newline
22. ओ॒षि॒ष्ठ॒हन(ग्म्॑) शिङ्गीनिको॒श्या᳚भ्याꣳ शिङ्गीनिको॒श्या᳚भ्या मोषिष्ठ॒हन॑ मोषिष्ठ॒हन(ग्म्॑) शिङ्गीनिको॒श्या᳚भ्याम् । \newline
23. ओ॒षि॒ष्ठ॒हन॒मित्यो॑षिष्ठ - हन᳚म् । \newline
24. शि॒ङ्गी॒नि॒को॒श्या᳚भ्या॒मिति॑ शिङ्गी - नि॒को॒श्या᳚भ्याम् । \newline
\pagebreak
\markright{ TS 1.4.37.1  \hfill https://www.vedavms.in \hfill}
\addcontentsline{toc}{section}{ TS 1.4.37.1 }
\section*{ TS 1.4.37.1 }

\textbf{TS 1.4.37.1 } \newline
\textbf{Samhita Paata} \newline

आ ति॑ष्ठ वृत्रह॒न् रथं॑ ॅयु॒क्ता ते॒ ब्रह्म॑णा॒ हरी᳚ । अ॒र्वा॒चीनꣳ॒॒ सु ते॒ मनो॒ ग्रावा॑ कृणोतु व॒ग्नुना᳚ ॥उ॒प॒या॒मगृ॑हीतो॒ऽसीन्द्रा॑य त्वा षोड॒शिन॑ ए॒ष ते॒ योनि॒रिन्द्रा॑य त्वा षोड॒शिने᳚ ॥ \newline

\textbf{Pada Paata} \newline

एति॑ । ति॒ष्ठ॒ । वृ॒त्र॒ह॒न्निति॑ वृत्र - ह॒न्न् । रथ᳚म् । यु॒क्ता । ते॒ । ब्रह्म॑णा । हरी॒ इति॑ ॥ अ॒र्वा॒चीन᳚म् । स्विति॑ । ते॒ । मनः॑ । ग्रावा᳚ । कृ॒णो॒तु॒ । व॒ग्नुना᳚ ॥ उ॒प॒या॒मगृ॑हीत॒ इत्यु॑पया॒म - गृ॒ही॒तः॒ । अ॒सि॒ । इन्द्रा॑य । त्वा॒ । षो॒ड॒शिने᳚ । ए॒षः । ते॒ । योनिः॑ । इन्द्रा॑य । त्वा॒ । षो॒ड॒शिने᳚ ॥  \newline


\textbf{Krama Paata} \newline

आ ति॑ष्ठ । ति॒ष्ठ॒ वृ॒त्र॒ह॒न्न्॒ । वृ॒त्र॒ह॒न् रथ᳚म् । वृ॒त्र॒ह॒न्निति॑ वृत्र - ह॒न्न्॒ । रथं॑ ॅयु॒क्ता । यु॒क्ता ते᳚ । ते॒ ब्रह्म॑णा । ब्रह्म॑णा॒ हरी᳚ । हरी॒ इति॒ हरी᳚ ॥ अ॒र्वा॒चीनꣳ॒॒ सु । सु ते᳚ । ते॒ मनः॑ । मनो॒ ग्रावा᳚ । ग्रावा॑ कृणोतु । कृ॒णो॒तु॒ व॒ग्नुना᳚ । व॒ग्नुनेति॑ व॒ग्नुना᳚ ॥ उ॒प॒या॒मगृ॑हीतोऽसि । उ॒प॒या॒मगृ॑हीत॒ इत्यु॑पया॒म - गृ॒ही॒तः॒ । अ॒सीन्द्रा॑य । इन्द्रा॑य त्वा । त्वा॒ षो॒ड॒शिने᳚ । षो॒ड॒शिन॑ ए॒षः । ए॒ष ते᳚ । ते॒ योनिः॑ । योनि॒रिन्द्रा॑य । इन्द्रा॑य त्वा । त्वा॒ षो॒ड॒शिने᳚ । षो॒ड॒शिन॒ इति॑ षोड॒शिने᳚ । \newline

\textbf{Jatai Paata} \newline

1. आ ति॑ष्ठ ति॒ष्ठा ति॑ष्ठ । \newline
2. ति॒ष्ठ॒ वृ॒त्र॒ह॒न्॒ वृ॒त्र॒ह॒न् ति॒ष्ठ॒ ति॒ष्ठ॒ वृ॒त्र॒ह॒न्न् । \newline
3. वृ॒त्र॒ह॒न् रथ॒(ग्म्॒) रथं॑ ॅवृत्रहन् वृत्रह॒न् रथ᳚म् । \newline
4. वृ॒त्र॒ह॒न्निति॑ वृत्र - ह॒न्न् । \newline
5. रथं॑ ॅयु॒क्ता यु॒क्ता रथ॒(ग्म्॒) रथं॑ ॅयु॒क्ता । \newline
6. यु॒क्ता ते॑ ते यु॒क्ता यु॒क्ता ते᳚ । \newline
7. ते॒ ब्रह्म॑णा॒ ब्रह्म॑णा ते ते॒ ब्रह्म॑णा । \newline
8. ब्रह्म॑णा॒ हरी॒ हरी॒ ब्रह्म॑णा॒ ब्रह्म॑णा॒ हरी᳚ । \newline
9. हरी॒ इति॒ हरी᳚ । \newline
10. अ॒र्वा॒चीन॒(ग्म्॒) सु स्व॑र्वा॒चीन॑ मर्वा॒चीन॒(ग्म्॒) सु । \newline
11. सु ते॑ ते॒ सु सु ते᳚ । \newline
12. ते॒ मनो॒ मन॑ स्ते ते॒ मनः॑ । \newline
13. मनो॒ ग्रावा॒ ग्रावा॒ मनो॒ मनो॒ ग्रावा᳚ । \newline
14. ग्रावा॑ कृणोतु कृणोतु॒ ग्रावा॒ ग्रावा॑ कृणोतु । \newline
15. कृ॒णो॒तु॒ व॒ग्नुना॑ व॒ग्नुना॑ कृणोतु कृणोतु व॒ग्नुना᳚ । \newline
16. व॒ग्नुनेति॑ व॒ग्नुना᳚ । \newline
17. उ॒प॒या॒मगृ॑हीतो ऽस्यस्युपया॒मगृ॑हीत उपया॒मगृ॑हीतो ऽसि । \newline
18. उ॒प॒या॒मगृ॑हीत॒ इत्यु॑पया॒म - गृ॒ही॒तः॒ । \newline
19. अ॒सीन्द्रा॒ये न्द्रा॑यास्य॒सीन्द्रा॑य । \newline
20. इन्द्रा॑य त्वा॒ त्वेन्द्रा॒ये न्द्रा॑य त्वा । \newline
21. त्वा॒ षो॒ड॒शिने॑ षोड॒शिने᳚ त्वा त्वा षोड॒शिने᳚ । \newline
22. षो॒ड॒शिन॑ ए॒ष ए॒ष षो॑ड॒शिने॑ षोड॒शिन॑ ए॒षः । \newline
23. ए॒ष ते॑ त ए॒ष ए॒ष ते᳚ । \newline
24. ते॒ योनि॒र् योनि॑ स्ते ते॒ योनिः॑ । \newline
25. योनि॒ रिन्द्रा॒ये न्द्रा॑य॒ योनि॒र् योनि॒ रिन्द्रा॑य । \newline
26. इन्द्रा॑य त्वा॒ त्वेन्द्रा॒ये न्द्रा॑य त्वा । \newline
27. त्वा॒ षो॒ड॒शिने॑ षोड॒शिने᳚ त्वा त्वा षोड॒शिने᳚ । \newline
28. षो॒ड॒शिन॒ इति॑ षोड॒शिने᳚ । \newline

\textbf{Ghana Paata } \newline

1. आ ति॑ष्ठ ति॒ष्ठा ति॑ष्ठ वृत्रहन् वृत्रहन् ति॒ष्ठा ति॑ष्ठ वृत्रहन्न् । \newline
2. ति॒ष्ठ॒ वृ॒त्र॒ह॒न्॒ वृ॒त्र॒ह॒न् ति॒ष्ठ॒ ति॒ष्ठ॒ वृ॒त्र॒ह॒न् रथ॒(ग्म्॒) रथं॑ ॅवृत्रहन् तिष्ठ तिष्ठ वृत्रह॒न् रथ᳚म् । \newline
3. वृ॒त्र॒ह॒न् रथ॒(ग्म्॒) रथं॑ ॅवृत्रहन् वृत्रह॒न् रथं॑ ॅयु॒क्ता यु॒क्ता रथं॑ ॅवृत्रहन् वृत्रह॒न् रथं॑ ॅयु॒क्ता । \newline
4. वृ॒त्र॒ह॒न्निति॑ वृत्र - ह॒न्न् । \newline
5. रथं॑ ॅयु॒क्ता यु॒क्ता रथ॒(ग्म्॒) रथं॑ ॅयु॒क्ता ते॑ ते यु॒क्ता रथ॒(ग्म्॒) रथं॑ ॅयु॒क्ता ते᳚ । \newline
6. यु॒क्ता ते॑ ते यु॒क्ता यु॒क्ता ते॒ ब्रह्म॑णा॒ ब्रह्म॑णा ते यु॒क्ता यु॒क्ता ते॒ ब्रह्म॑णा । \newline
7. ते॒ ब्रह्म॑णा॒ ब्रह्म॑णा ते ते॒ ब्रह्म॑णा॒ हरी॒ हरी॒ ब्रह्म॑णा ते ते॒ ब्रह्म॑णा॒ हरी᳚ । \newline
8. ब्रह्म॑णा॒ हरी॒ हरी॒ ब्रह्म॑णा॒ ब्रह्म॑णा॒ हरी᳚ । \newline
9. हरी॒ इति॒ हरी᳚ । \newline
10. अ॒र्वा॒चीन॒(ग्म्॒) सु स्व॑र्वा॒चीन॑ मर्वा॒चीन॒(ग्म्॒) सु ते॑ ते॒ स्व॑र्वा॒चीन॑ मर्वा॒चीन॒(ग्म्॒) सु ते᳚ । \newline
11. सु ते॑ ते॒ सु सु ते॒ मनो॒ मन॑ स्ते॒ सु सु ते॒ मनः॑ । \newline
12. ते॒ मनो॒ मन॑ स्ते ते॒ मनो॒ ग्रावा॒ ग्रावा॒ मन॑ स्ते ते॒ मनो॒ ग्रावा᳚ । \newline
13. मनो॒ ग्रावा॒ ग्रावा॒ मनो॒ मनो॒ ग्रावा॑ कृणोतु कृणोतु॒ ग्रावा॒ मनो॒ मनो॒ ग्रावा॑ कृणोतु । \newline
14. ग्रावा॑ कृणोतु कृणोतु॒ ग्रावा॒ ग्रावा॑ कृणोतु व॒ग्नुना॑ व॒ग्नुना॑ कृणोतु॒ ग्रावा॒ ग्रावा॑ कृणोतु व॒ग्नुना᳚ । \newline
15. कृ॒णो॒तु॒ व॒ग्नुना॑ व॒ग्नुना॑ कृणोतु कृणोतु व॒ग्नुना᳚ । \newline
16. व॒ग्नुनेति॑ व॒ग्नुना᳚ । \newline
17. उ॒प॒या॒मगृ॑हीतो ऽस्य स्युपया॒मगृ॑हीत उपया॒मगृ॑हीतो॒ ऽसीन्द्रा॒ये न्द्रा॑ यास्युपया॒मगृ॑हीत उपया॒मगृ॑हीतो॒ ऽसीन्द्रा॑य । \newline
18. उ॒प॒या॒मगृ॑हीत॒ इत्यु॑पया॒म - गृ॒ही॒तः॒ । \newline
19. अ॒सीन्द्रा॒ये न्द्रा॑ यास्य॒सीन्द्रा॑य त्वा॒ त्वेन्द्रा॑ यास्य॒सीन्द्रा॑य त्वा । \newline
20. इन्द्रा॑य त्वा॒ त्वेन्द्रा॒ये न्द्रा॑य त्वा षोड॒शिने॑ षोड॒शिने॒ त्वेन्द्रा॒ये न्द्रा॑य त्वा षोड॒शिने᳚ । \newline
21. त्वा॒ षो॒ड॒शिने॑ षोड॒शिने᳚ त्वा त्वा षोड॒शिन॑ ए॒ष ए॒ष षो॑ड॒शिने᳚ त्वा त्वा षोड॒शिन॑ ए॒षः । \newline
22. षो॒ड॒शिन॑ ए॒ष ए॒ष षो॑ड॒शिने॑ षोड॒शिन॑ ए॒ष ते॑ त ए॒ष षो॑ड॒शिने॑ षोड॒शिन॑ ए॒ष ते᳚ । \newline
23. ए॒ष ते॑ त ए॒ष ए॒ष ते॒ योनि॒र् योनि॑ स्त ए॒ष ए॒ष ते॒ योनिः॑ । \newline
24. ते॒ योनि॒र् योनि॑ स्ते ते॒ योनि॒ रिन्द्रा॒ये न्द्रा॑य॒ योनि॑ स्ते ते॒ योनि॒ रिन्द्रा॑य । \newline
25. योनि॒ रिन्द्रा॒ये न्द्रा॑य॒ योनि॒र् योनि॒ रिन्द्रा॑य त्वा॒ त्वेन्द्रा॑य॒ योनि॒र् योनि॒ रिन्द्रा॑य त्वा । \newline
26. इन्द्रा॑य त्वा॒ त्वेन्द्रा॒ये न्द्रा॑य त्वा षोड॒शिने॑ षोड॒शिने॒ त्वेन्द्रा॒ये न्द्रा॑य त्वा षोड॒शिने᳚ । \newline
27. त्वा॒ षो॒ड॒शिने॑ षोड॒शिने᳚ त्वा त्वा षोड॒शिने᳚ । \newline
28. षो॒ड॒शिन॒ इति॑ षोड॒शिने᳚ । \newline
\pagebreak
\markright{ TS 1.4.38.1  \hfill https://www.vedavms.in \hfill}
\addcontentsline{toc}{section}{ TS 1.4.38.1 }
\section*{ TS 1.4.38.1 }

\textbf{TS 1.4.38.1 } \newline
\textbf{Samhita Paata} \newline

इन्द्र॒मिद्धरी॑ वह॒तो-ऽप्र॑तिधृष्टशवस॒-मृषी॑णां च स्तु॒तीरुप॑ य॒ज्ञ्ं च॒ मानु॑षाणां ॥ उ॒प॒या॒मगृ॑हीतो॒-ऽसीन्द्रा॑य त्वा षोड॒शिन॑ ए॒ष ते॒ योनि॒रिन्द्रा॑य त्वा षोड॒शिने᳚ ॥ \newline

\textbf{Pada Paata} \newline

इन्द्र᳚म् । इत् । हरी॒ इति॑ । व॒ह॒तः॒ । अप्र॑तिधृष्टशवस॒मित्यप्र॑तिधृष्ट - श॒व॒स॒म् । ऋषी॑णाम् । च॒ । स्तु॒तीः । उपेति॑ । य॒ज्ञ्म् । च॒ । मानु॑षाणाम् ॥ उ॒प॒या॒मगृ॑हीत॒ इत्यु॑पया॒म - गृ॒ही॒तः॒ । अ॒सि॒ । इन्द्रा॑य । त्वा॒ । षो॒ड॒शिने᳚ । ए॒षः । ते॒ । योनिः॑ । इन्द्रा॑य । त्वा॒ । षो॒ड॒शिने᳚ ॥  \newline


\textbf{Krama Paata} \newline

इन्द्र॒मित् । इद्धरी᳚ । हरी॑ वहतः । हरी॒ इति॒ हरी᳚ ॥ व॒ह॒तो ऽप्र॑तिधृष्टशवसम् । 
अप्र॑तिधृष्टशवस॒ मृषी॑णाम् । अप्र॑तिधृष्टशवस॒मित्यप्र॑तिधृष्ट - श॒व॒स॒म् । ऋषी॑णाम् च । च॒ स्तु॒तीः । स्तु॒तीरुप॑ । उप॑ य॒ज्ञ्म् । य॒ज्ञ्म् च॑ । च॒ मानु॑षाणाम् । मानु॑षाणा॒मिति॒ मानु॑षाणाम् ॥ उ॒प॒या॒मगृ॑हीतोऽसि । उ॒प॒या॒मगृ॑हीत॒ इत्यु॑पया॒म - गृ॒ही॒तः॒ । अ॒सीन्द्रा॑य । इन्द्रा॑य त्वा । त्वा॒ षो॒ड॒शिने᳚ । षो॒ड॒शिन॑ ए॒षः । ए॒ष ते᳚ । ते॒ योनिः॑ । योनि॒रिन्द्रा॑य । इन्द्रा॑य त्वा । त्वा॒ षो॒ड॒शिने᳚ । षो॒ड॒शिन॒ इति॑ षोड॒शिने᳚ । \newline

\textbf{Jatai Paata} \newline

1. इन्द्र॒ मिदिदिन्द्र॒ मिन्द्र॒ मित् । \newline
2. इद्धरी॒ हरी॒ इदिद्धरी᳚ । \newline
3. हरी॑ वहतो वहतो॒ हरी॒ हरी॑ वहतः । \newline
4. हरी॒ इति॒ हरी᳚ । \newline
5. व॒ह॒तो ऽप्र॑तिधृष्टशवस॒ मप्र॑तिधृष्टशवसं ॅवहतो वह॒तो ऽप्र॑तिधृष्टशवसम् । \newline
6. अप्र॑तिधृष्टशवस॒ मृषी॑णा॒ मृषी॑णा॒ मप्र॑तिधृष्टशवस॒ मप्र॑तिधृष्टशवस॒ मृषी॑णाम् । \newline
7. अप्र॑तिधृष्टशवस॒मित्यप्र॑तिधृष्ट - श॒व॒स॒म् । \newline
8. ऋषी॑णाम् च॒ चर्.षी॑णा॒ मृषी॑णाम् च । \newline
9. च॒ स्तु॒तीः स्तु॒तीश्च॑ च स्तु॒तीः । \newline
10. स्तु॒ती रुपोप॑ स्तु॒तीः स्तु॒तीरुप॑ । \newline
11. उप॑ य॒ज्ञ्ं ॅय॒ज्ञ् मुपोप॑ य॒ज्ञ्म् । \newline
12. य॒ज्ञ्म् च॑ च य॒ज्ञ्ं ॅय॒ज्ञ्म् च॑ । \newline
13. च॒ मानु॑षाणा॒म् मानु॑षाणाम् च च॒ मानु॑षाणाम् । \newline
14. मानु॑षाणा॒मिति॒ मानु॑षाणाम् । \newline
15. उ॒प॒या॒मगृ॑हीतो ऽस्यस्युपया॒मगृ॑हीत उपया॒मगृ॑हीतो ऽसि । \newline
16. उ॒प॒या॒मगृ॑हीत॒ इत्यु॑पया॒म - गृ॒ही॒तः॒ । \newline
17. अ॒सीन्द्रा॒ये न्द्रा॑यास्य॒सीन्द्रा॑य । \newline
18. इन्द्रा॑य त्वा॒ त्वेन्द्रा॒ये न्द्रा॑य त्वा । \newline
19. त्वा॒ षो॒ड॒शिने॑ षोड॒शिने᳚ त्वा त्वा षोड॒शिने᳚ । \newline
20. षो॒ड॒शिन॑ ए॒ष ए॒ष षो॑ड॒शिने॑ षोड॒शिन॑ ए॒षः । \newline
21. ए॒ष ते॑ त ए॒ष ए॒ष ते᳚ । \newline
22. ते॒ योनि॒र् योनि॑ स्ते ते॒ योनिः॑ । \newline
23. योनि॒ रिन्द्रा॒ये न्द्रा॑य॒ योनि॒र् योनि॒ रिन्द्रा॑य । \newline
24. इन्द्रा॑य त्वा॒ त्वेन्द्रा॒ये न्द्रा॑य त्वा । \newline
25. त्वा॒ षो॒ड॒शिने॑ षोड॒शिने᳚ त्वा त्वा षोड॒शिने᳚ । \newline
26. षो॒ड॒शिन॒ इति॑ षोड॒शिने᳚ । \newline

\textbf{Ghana Paata } \newline

1. इन्द्र॒ मिदिदिन्द्र॒ मिन्द्र॒ मिद्धरी॒ हरी॒ इदिन्द्र॒ मिन्द्र॒ मिद्धरी᳚ । \newline
2. इद्धरी॒ हरी॒ इदिद्धरी॑ वहतो वहतो॒ हरी॒ इदिद्धरी॑ वहतः । \newline
3. हरी॑ वहतो वहतो॒ हरी॒ हरी॑ वह॒तो ऽप्र॑तिधृष्टशवस॒ मप्र॑तिधृष्टशवसं ॅवहतो॒ हरी॒ हरी॑ वह॒तो ऽप्र॑तिधृष्टशवसम् । \newline
4. हरी॒ इति॒ हरी᳚ । \newline
5. व॒ह॒तो ऽप्र॑तिधृष्टशवस॒ मप्र॑तिधृष्टशवसं ॅवहतो वह॒तो ऽप्र॑तिधृष्टशवस॒ मृषी॑णा॒ मृषी॑णा॒ मप्र॑तिधृष्टशवसं ॅवहतो वह॒तो ऽप्र॑तिधृष्टशवस॒ मृषी॑णाम् । \newline
6. अप्र॑तिधृष्टशवस॒ मृषी॑णा॒ मृषी॑णा॒ मप्र॑तिधृष्टशवस॒ मप्र॑तिधृष्टशवस॒ मृषी॑णाम् च॒ च र्.षी॑णा॒ मप्र॑तिधृष्टशवस॒ मप्र॑तिधृष्टशवस॒ मृषी॑णाम् च । \newline
7. अप्र॑तिधृष्टशवस॒मित्यप्र॑तिधृष्ट - श॒व॒स॒म् । \newline
8. ऋषी॑णाम् च॒ च र्.षी॑णा॒ मृषी॑णाम् च स्तु॒तीः स्तु॒तीश्च र्.षी॑णा॒ मृषी॑णाम् च स्तु॒तीः । \newline
9. च॒ स्तु॒तीः स्तु॒तीश्च॑ च स्तु॒ती रुपोप॑ स्तु॒तीश्च॑ च स्तु॒ती रुप॑ । \newline
10. स्तु॒ती रुपोप॑ स्तु॒तीः स्तु॒तीरुप॑ य॒ज्ञ्ं ॅय॒ज्ञ् मुप॑ स्तु॒तीः स्तु॒तीरुप॑ य॒ज्ञ्म् । \newline
11. उप॑ य॒ज्ञ्ं ॅय॒ज्ञ् मुपोप॑ य॒ज्ञ्म् च॑ च य॒ज्ञ् मुपोप॑ य॒ज्ञ्म् च॑ । \newline
12. य॒ज्ञ्म् च॑ च य॒ज्ञ्ं ॅय॒ज्ञ्म् च॒ मानु॑षाणा॒म् मानु॑षाणाम् च य॒ज्ञ्ं ॅय॒ज्ञ्म् च॒ मानु॑षाणाम् । \newline
13. च॒ मानु॑षाणा॒म् मानु॑षाणाम् च च॒ मानु॑षाणाम् । \newline
14. मानु॑षाणा॒मिति॒ मानु॑षाणाम् । \newline
15. उ॒प॒या॒मगृ॑हीतो ऽस्य स्युपया॒मगृ॑हीत उपया॒मगृ॑हीतो॒ ऽसीन्द्रा॒ये न्द्रा॑ यास्युपया॒मगृ॑हीत उपया॒मगृ॑हीतो॒ ऽसीन्द्रा॑य । \newline
16. उ॒प॒या॒मगृ॑हीत॒ इत्यु॑पया॒म - गृ॒ही॒तः॒ । \newline
17. अ॒सीन्द्रा॒ये न्द्रा॑ यास्य॒सीन्द्रा॑य त्वा॒ त्वेन्द्रा॑ यास्य॒सीन्द्रा॑य त्वा । \newline
18. इन्द्रा॑य त्वा॒ त्वेन्द्रा॒ये न्द्रा॑य त्वा षोड॒शिने॑ षोड॒शिने॒ त्वेन्द्रा॒ये न्द्रा॑य त्वा षोड॒शिने᳚ । \newline
19. त्वा॒ षो॒ड॒शिने॑ षोड॒शिने᳚ त्वा त्वा षोड॒शिन॑ ए॒ष ए॒ष षो॑ड॒शिने᳚ त्वा त्वा षोड॒शिन॑ ए॒षः । \newline
20. षो॒ड॒शिन॑ ए॒ष ए॒ष षो॑ड॒शिने॑ षोड॒शिन॑ ए॒ष ते॑ त ए॒ष षो॑ड॒शिने॑ षोड॒शिन॑ ए॒ष ते᳚ । \newline
21. ए॒ष ते॑ त ए॒ष ए॒ष ते॒ योनि॒र् योनि॑ स्त ए॒ष ए॒ष ते॒ योनिः॑ । \newline
22. ते॒ योनि॒र् योनि॑ स्ते ते॒ योनि॒ रिन्द्रा॒ये न्द्रा॑य॒ योनि॑ स्ते ते॒ योनि॒ रिन्द्रा॑य । \newline
23. योनि॒ रिन्द्रा॒ये न्द्रा॑य॒ योनि॒र् योनि॒ रिन्द्रा॑य त्वा॒ त्वेन्द्रा॑य॒ योनि॒र् योनि॒ रिन्द्रा॑य त्वा । \newline
24. इन्द्रा॑य त्वा॒ त्वेन्द्रा॒ये न्द्रा॑य त्वा षोड॒शिने॑ षोड॒शिने॒ त्वेन्द्रा॒ये न्द्रा॑य त्वा षोड॒शिने᳚ । \newline
25. त्वा॒ षो॒ड॒शिने॑ षोड॒शिने᳚ त्वा त्वा षोड॒शिने᳚ । \newline
26. षो॒ड॒शिन॒ इति॑ षोड॒शिने᳚ । \newline
\pagebreak
\markright{ TS 1.4.39.1  \hfill https://www.vedavms.in \hfill}
\addcontentsline{toc}{section}{ TS 1.4.39.1 }
\section*{ TS 1.4.39.1 }

\textbf{TS 1.4.39.1 } \newline
\textbf{Samhita Paata} \newline

असा॑वि॒ सोम॑ इन्द्र ते॒ शवि॑ष्ठ धृष्ण॒वा ग॑हि । आ त्वा॑ पृणक्त्विन्द्रि॒यꣳ रजः॒ सूर्यं॒ न र॒श्मिभिः॑ ॥ उ॒प॒या॒मगृ॑हीतो॒ऽसीन्द्रा॑य त्वा षोड॒शिन॑ ए॒ष ते॒ योनि॒रिन्द्रा॑य त्वा षोड॒शिने᳚ ॥ \newline

\textbf{Pada Paata} \newline

असा॑वि । सोमः॑ । इ॒न्द्र॒ । ते॒ । शवि॑ष्ठ । धृ॒ष्णो॒ । एति॑ । ग॒हि॒ ॥ एति॑ । त्वा॒ । पृ॒ण॒क्‌तु॒ । इ॒न्द्रि॒यम् । रजः॑ । सूर्य᳚म् । न । र॒श्मिभि॒रिति॑ र॒श्मि - भिः॒ ॥ उ॒प॒या॒मगृ॑हीत॒ इत्यु॑पया॒म - गृ॒ही॒तः॒ । अ॒सि॒ । इन्द्रा॑य । त्वा॒ । षो॒ड॒शिने᳚ । ए॒षः । ते॒ । योनिः॑ । इन्द्रा॑य । त्वा॒ । षो॒ड॒शिने᳚ ॥  \newline


\textbf{Krama Paata} \newline

असा॑वि॒ सोमः॑ । सोम॑ इन्द्र । इ॒न्द्र॒ ते॒ । ते॒ शवि॑ष्ठ । शवि॑ष्ठ धृष्णो । धृ॒ष्ण॒वा । आ ग॑हि । ग॒हीति॑ गहि ॥ आ त्वा᳚ । त्वा॒ पृ॒ण॒क्तु॒ । पृ॒ण॒क्त्वि॒न्द्रि॒यम् । इ॒न्द्रि॒यꣳ रजः॑ । रजः॒ सूर्य᳚म् । सूर्य॒न्न । न र॒श्मिभिः॑ । र॒श्मिभि॒रिति॑ र॒श्मि - भिः॒ ॥ उ॒प॒या॒मगृ॑हीतोऽसि । उ॒प॒या॒मगृ॑हीत॒ इत्यु॑पया॒म - गृ॒ही॒तः॒ । अ॒सीन्द्रा॑य । इन्द्रा॑य त्वा । त्वा॒ षो॒ड॒शिने᳚ । षो॒ड॒शिन॑ ए॒षः । ए॒ष ते᳚ । ते॒ योनिः॑ । योनि॒रिन्द्रा॑य । इन्द्रा॑य त्वा । त्वा॒ षो॒ड॒शिने᳚ । षो॒ड॒शिन॒ इति॑ षोड॒शिने᳚ । \newline

\textbf{Jatai Paata} \newline

1. असा॑वि॒ सोमः॒ सोमो ऽसा॒व्यसा॑वि॒ सोमः॑ । \newline
2. सोम॑ इन्द्रे न्द्र॒ सोमः॒ सोम॑ इन्द्र । \newline
3. इ॒न्द्र॒ ते॒ त॒ इ॒न्द्रे॒ न्द्र॒ ते॒ । \newline
4. ते॒ शवि॑ष्ठ॒ शवि॑ष्ठ ते ते॒ शवि॑ष्ठ । \newline
5. शवि॑ष्ठ धृष्णो धृष्णो॒ शवि॑ष्ठ॒ शवि॑ष्ठ धृष्णो । \newline
6. धृ॒ष्णवा धृ॑ष्णो धृ॒ष्णवा । \newline
7. आ ग॑हि ग॒ह्या ग॑हि । \newline
8. ग॒हीति॑ गहि । \newline
9. आ त्वा॒ त्वा ऽऽत्वा᳚ । \newline
10. त्वा॒ पृ॒ण॒क्तु॒ पृ॒ण॒क्तु॒ त्वा॒ त्वा॒ पृ॒ण॒क्तु॒ । \newline
11. पृ॒ण॒क्त्वि॒न्द्रि॒य मि॑न्द्रि॒यम् पृ॑णक्तु पृणक्त्विन्द्रि॒यम् । \newline
12. इ॒न्द्रि॒यꣳ रजो॒ रज॑ इन्द्रि॒य मि॑न्द्रि॒यꣳ रजः॑ । \newline
13. रजः॒ सूर्य॒(ग्म्॒) सूर्य॒(ग्म्॒) रजो॒ रजः॒ सूर्य᳚म् । \newline
14. सूर्य॒न्न न सूर्य॒(ग्म्॒) सूर्य॒न्न । \newline
15. न र॒श्मिभी॑ र॒श्मिभि॒र् न न र॒श्मिभिः॑ । \newline
16. र॒श्मिभि॒रिति॑ र॒श्मि - भिः॒ । \newline
17. उ॒प॒या॒मगृ॑हीतो ऽस्यस्युपया॒मगृ॑हीत उपया॒मगृ॑हीतो ऽसि । \newline
18. उ॒प॒या॒मगृ॑हीत॒ इत्यु॑पया॒म - गृ॒ही॒तः॒ । \newline
19. अ॒सीन्द्रा॒ये न्द्रा॑यास्य॒सीन्द्रा॑य । \newline
20. इन्द्रा॑य त्वा॒ त्वेन्द्रा॒ये न्द्रा॑य त्वा । \newline
21. त्वा॒ षो॒ड॒शिने॑ षोड॒शिने᳚ त्वा त्वा षोड॒शिने᳚ । \newline
22. षो॒ड॒शिन॑ ए॒ष ए॒ष षो॑ड॒शिने॑ षोड॒शिन॑ ए॒षः । \newline
23. ए॒ष ते॑ त ए॒ष ए॒ष ते᳚ । \newline
24. ते॒ योनि॒र् योनि॑ स्ते ते॒ योनिः॑ । \newline
25. योनि॒ रिन्द्रा॒ये न्द्रा॑य॒ योनि॒र् योनि॒ रिन्द्रा॑य । \newline
26. इन्द्रा॑य त्वा॒ त्वेन्द्रा॒ये न्द्रा॑य त्वा । \newline
27. त्वा॒ षो॒ड॒शिने॑ षोड॒शिने᳚ त्वा त्वा षोड॒शिने᳚ । \newline
28. षो॒ड॒शिन॒ इति॑ षोड॒शिने᳚ । \newline

\textbf{Ghana Paata } \newline

1. असा॑वि॒ सोमः॒ सोमो ऽसा॒व्यसा॑वि॒ सोम॑ इन्द्रे न्द्र॒ सोमो ऽसा॒व्यसा॑वि॒ सोम॑ इन्द्र । \newline
2. सोम॑ इन्द्रे न्द्र॒ सोमः॒ सोम॑ इन्द्र ते त इन्द्र॒ सोमः॒ सोम॑ इन्द्र ते । \newline
3. इ॒न्द्र॒ ते॒ त॒ इ॒न्द्रे॒ न्द्र॒ ते॒ शवि॑ष्ठ॒ शवि॑ष्ठ त इन्द्रे न्द्र ते॒ शवि॑ष्ठ । \newline
4. ते॒ शवि॑ष्ठ॒ शवि॑ष्ठ ते ते॒ शवि॑ष्ठ धृष्णो धृष्णो॒ शवि॑ष्ठ ते ते॒ शवि॑ष्ठ धृष्णो । \newline
5. शवि॑ष्ठ धृष्णो धृष्णो॒ शवि॑ष्ठ॒ शवि॑ष्ठ धृ॒ष्णवा धृ॑ष्णो॒ शवि॑ष्ठ॒ शवि॑ष्ठ धृ॒ष्णवा । \newline
6. धृ॒ष्णवा धृ॑ष्णो धृ॒ष्णवा ग॑हि ग॒ह्या धृ॑ष्णो धृ॒ष्णवा ग॑हि । \newline
7. आ ग॑हि ग॒ह्या ग॑हि । \newline
8. ग॒हीति॑ गहि । \newline
9. आ त्वा॒ त्वा ऽऽत्वा॑ पृणक्तु पृणक्तु॒ त्वा ऽऽत्वा॑ पृणक्तु । \newline
10. त्वा॒ पृ॒ण॒क्तु॒ पृ॒ण॒क्तु॒ त्वा॒ त्वा॒ पृ॒ण॒क्त्वि॒न्द्रि॒य मि॑न्द्रि॒यम् पृ॑णक्तु त्वा त्वा पृणक्त्विन्द्रि॒यम् । \newline
11. पृ॒ण॒क्त्वि॒न्द्रि॒य मि॑न्द्रि॒यम् पृ॑णक्तु पृणक्त्विन्द्रि॒यꣳ रजो॒ रज॑ इन्द्रि॒यम् पृ॑णक्तु पृणक्त्विन्द्रि॒यꣳ रजः॑ । \newline
12. इ॒न्द्रि॒यꣳ रजो॒ रज॑ इन्द्रि॒य मि॑न्द्रि॒यꣳ रजः॒ सूर्य॒(ग्म्॒) सूर्य॒(ग्म्॒) रज॑ इन्द्रि॒य मि॑न्द्रि॒यꣳ रजः॒ सूर्य᳚म् । \newline
13. रजः॒ सूर्य॒(ग्म्॒) सूर्य॒(ग्म्॒) रजो॒ रजः॒ सूर्य॒म् न न सूर्य॒(ग्म्॒) रजो॒ रजः॒ सूर्य॒म् न । \newline
14. सूर्य॒म् न न सूर्य॒(ग्म्॒) सूर्य॒म् न र॒श्मिभी॑ र॒श्मिभि॒र् न सूर्य॒(ग्म्॒) सूर्य॒म् न र॒श्मिभिः॑ । \newline
15. न र॒श्मिभी॑ र॒श्मिभि॒र् न न र॒श्मिभिः॑ । \newline
16. र॒श्मिभि॒रिति॑ र॒श्मि - भिः॒ । \newline
17. उ॒प॒या॒मगृ॑हीतो ऽस्यस्युपया॒मगृ॑हीत उपया॒मगृ॑हीतो॒ ऽसीन्द्रा॒ये न्द्रा॑यास्युपया॒मगृ॑हीत उपया॒मगृ॑हीतो॒ ऽसीन्द्रा॑य । \newline
18. उ॒प॒या॒मगृ॑हीत॒ इत्यु॑पया॒म - गृ॒ही॒तः॒ । \newline
19. अ॒सीन्द्रा॒ये न्द्रा॑ यास्य॒सीन्द्रा॑य त्वा॒ त्वेन्द्रा॑ यास्य॒सीन्द्रा॑य त्वा । \newline
20. इन्द्रा॑य त्वा॒ त्वेन्द्रा॒ये न्द्रा॑य त्वा षोड॒शिने॑ षोड॒शिने॒ त्वेन्द्रा॒ये न्द्रा॑य त्वा षोड॒शिने᳚ । \newline
21. त्वा॒ षो॒ड॒शिने॑ षोड॒शिने᳚ त्वा त्वा षोड॒शिन॑ ए॒ष ए॒ष षो॑ड॒शिने᳚ त्वा त्वा षोड॒शिन॑ ए॒षः । \newline
22. षो॒ड॒शिन॑ ए॒ष ए॒ष षो॑ड॒शिने॑ षोड॒शिन॑ ए॒ष ते॑ त ए॒ष षो॑ड॒शिने॑ षोड॒शिन॑ ए॒ष ते᳚ । \newline
23. ए॒ष ते॑ त ए॒ष ए॒ष ते॒ योनि॒र् योनि॑ स्त ए॒ष ए॒ष ते॒ योनिः॑ । \newline
24. ते॒ योनि॒र् योनि॑ स्ते ते॒ योनि॒ रिन्द्रा॒ये न्द्रा॑य॒ योनि॑ स्ते ते॒ योनि॒ रिन्द्रा॑य । \newline
25. योनि॒ रिन्द्रा॒ये न्द्रा॑य॒ योनि॒र् योनि॒ रिन्द्रा॑य त्वा॒ त्वेन्द्रा॑य॒ योनि॒र् योनि॒ रिन्द्रा॑य त्वा । \newline
26. इन्द्रा॑य त्वा॒ त्वेन्द्रा॒ये न्द्रा॑य त्वा षोड॒शिने॑ षोड॒शिने॒ त्वेन्द्रा॒ये न्द्रा॑य त्वा षोड॒शिने᳚ । \newline
27. त्वा॒ षो॒ड॒शिने॑ षोड॒शिने᳚ त्वा त्वा षोड॒शिने᳚ । \newline
28. षो॒ड॒शिन॒ इति॑ षोड॒शिने᳚ । \newline
\pagebreak
\markright{ TS 1.4.40.1  \hfill https://www.vedavms.in \hfill}
\addcontentsline{toc}{section}{ TS 1.4.40.1 }
\section*{ TS 1.4.40.1 }

\textbf{TS 1.4.40.1 } \newline
\textbf{Samhita Paata} \newline

सर्व॑स्य प्रति॒शीव॑री॒ भूमि॑स्त्वो॒पस्थ॒ आऽधि॑त । स्यो॒नाऽस्मै॑ सु॒षदा॑ भव॒ यच्छा᳚ऽस्मै शर्म॑ स॒प्रथाः᳚ ॥ उ॒प॒या॒मगृ॑हीतो॒ऽसीन्द्रा॑य त्वा षोड॒शिन॑ ए॒ष ते॒ योनि॒रिन्द्रा॑य त्वा षोड॒शिने᳚ ॥ \newline

\textbf{Pada Paata} \newline

सर्व॑स्य । प्र॒ति॒शीव॒रीति॑ प्रति - शीव॑री । भूमिः॑ । त्वा॒ । उ॒पस्थ॒ इत्यु॒प - स्थे॒ । एति॑ । अ॒धि॒त॒ ॥ स्यो॒ना । अ॒स्मै॒ । सु॒षदेति॑ सु - सदा᳚ । भ॒व॒ । यच्छ॑ । अ॒स्मै॒ । शर्म॑ । स॒प्रथा॒ इति॑ स-प्रथाः᳚ ॥ उ॒प॒या॒मगृ॑हीत॒ इत्यु॑पया॒म - गृ॒ही॒तः॒ । अ॒सि॒ । इन्द्रा॑य । त्वा॒ । षो॒ड॒शिने᳚ । ए॒षः । ते॒ । योनिः॑ । इन्द्रा॑य । त्वा॒ । षो॒ड॒शिने᳚ ॥  \newline


\textbf{Krama Paata} \newline

सर्व॑स्य प्रति॒शीव॑री । प्र॒ति॒शीव॑री॒ भूमिः॑ । प्र॒ति॒शीव॒रीति॑ प्रति - शीव॑री । भूमि॑स्त्वा । त्वो॒पस्थे᳚ । उ॒पस्थ॒ आ । उ॒पस्थ॒ इत्यु॒प - स्थे॒ । आऽधि॑त । अ॒धि॒तेत्य॑धित ॥ स्यो॒नाऽस्मै᳚ । अ॒स्मै॒ सु॒षदा᳚ । सु॒षदा॑ भव । सु॒षदेति॑ सु - सदा᳚ । भ॒व॒ यच्छ॑ । यच्छा᳚स्मै । अ॒स्मै॒ शर्म॑ । शर्म॑ स॒प्रथाः᳚ । स॒प्रथा॒ इति॑ स - प्रथाः᳚ ॥ उ॒प॒या॒मगृ॑हीतोऽसि । उ॒प॒या॒मगृ॑हीत॒ इत्यु॑पया॒म - गृ॒ही॒तः॒ । अ॒सीन्द्रा॑य । इन्द्रा॑य त्वा । त्वा॒ षो॒ड॒शिने᳚ । षो॒ड॒शिन॑ ए॒षः । ए॒ष ते᳚ । ते॒ योनिः॑ । योनि॒रिन्द्रा॑य । इन्द्रा॑य त्वा । त्वा॒ षो॒ड॒शिने᳚ । षो॒ड॒शिन॒ इति॑ षोड॒शिने᳚ । \newline

\textbf{Jatai Paata} \newline

1. सर्व॑स्य प्रति॒शीव॑री प्रति॒शीव॑री॒ सर्व॑स्य॒ सर्व॑स्य प्रति॒शीव॑री । \newline
2. प्र॒ति॒शीव॑री॒ भूमि॒र् भूमिः॑ प्रति॒शीव॑री प्रति॒शीव॑री॒ भूमिः॑ । \newline
3. प्र॒ति॒शीव॒रीति॑ प्रति - शीव॑री । \newline
4. भूमि॑ स्त्वा त्वा॒ भूमि॒र् भूमि॑ स्त्वा । \newline
5. त्वो॒पस्थ॑ उ॒पस्थे᳚ त्वा त्वो॒पस्थे᳚ । \newline
6. उ॒पस्थ॒ ओपस्थ॑ उ॒पस्थ॒ आ । \newline
7. उ॒पस्थ॒ इत्यु॒प - स्थे॒ । \newline
8. आ ऽधि॑ताधि॒ता ऽधि॑त । \newline
9. अ॒धि॒तेत्य॑धित । \newline
10. स्यो॒ना ऽस्मा॑ अस्मै स्यो॒ना स्यो॒ना ऽस्मै᳚ । \newline
11. अ॒स्मै॒ सु॒षदा॑ सु॒षदा᳚ ऽस्मा अस्मै सु॒षदा᳚ । \newline
12. सु॒षदा॑ भव भव सु॒षदा॑ सु॒षदा॑ भव । \newline
13. सु॒षदेति॑ सु - सदा᳚ । \newline
14. भ॒व॒ यच्छ॒ यच्छ॑ भव भव॒ यच्छ॑ । \newline
15. यच्छा᳚स्मा अस्मै॒ यच्छ॒ यच्छा᳚स्मै । \newline
16. अ॒स्मै॒ शर्म॒ शर्मा᳚स्मा अस्मै॒ शर्म॑ । \newline
17. शर्म॑ स॒प्रथाः᳚ स॒प्रथाः॒ शर्म॒ शर्म॑ स॒प्रथाः᳚ । \newline
18. स॒प्रथा॒ इति॑ स - प्रथाः᳚ । \newline
19. उ॒प॒या॒मगृ॑हीतो ऽस्यस्युपया॒मगृ॑हीत उपया॒मगृ॑हीतो ऽसि । \newline
20. उ॒प॒या॒मगृ॑हीत॒ इत्यु॑पया॒म - गृ॒ही॒तः॒ । \newline
21. अ॒सीन्द्रा॒ये न्द्रा॑यास्य॒सीन्द्रा॑य । \newline
22. इन्द्रा॑य त्वा॒ त्वेन्द्रा॒ये न्द्रा॑य त्वा । \newline
23. त्वा॒ षो॒ड॒शिने॑ षोड॒शिने᳚ त्वा त्वा षोड॒शिने᳚ । \newline
24. षो॒ड॒शिन॑ ए॒ष ए॒ष षो॑ड॒शिने॑ षोड॒शिन॑ ए॒षः । \newline
25. ए॒ष ते॑ त ए॒ष ए॒ष ते᳚ । \newline
26. ते॒ योनि॒र् योनि॑ स्ते ते॒ योनिः॑ । \newline
27. योनि॒ रिन्द्रा॒ये न्द्रा॑य॒ योनि॒र् योनि॒ रिन्द्रा॑य । \newline
28. इन्द्रा॑य त्वा॒ त्वेन्द्रा॒ये न्द्रा॑य त्वा । \newline
29. त्वा॒ षो॒ड॒शिने॑ षोड॒शिने᳚ त्वा त्वा षोड॒शिने᳚ । \newline
30. षो॒ड॒शिन॒ इति॑ षोड॒शिने᳚ । \newline

\textbf{Ghana Paata } \newline

1. सर्व॑स्य प्रति॒शीव॑री प्रति॒शीव॑री॒ सर्व॑स्य॒ सर्व॑स्य प्रति॒शीव॑री॒ भूमि॒र् भूमिः॑ प्रति॒शीव॑री॒ सर्व॑स्य॒ सर्व॑स्य प्रति॒शीव॑री॒ भूमिः॑ । \newline
2. प्र॒ति॒शीव॑री॒ भूमि॒र् भूमिः॑ प्रति॒शीव॑री प्रति॒शीव॑री॒ भूमि॑ स्त्वा त्वा॒ भूमिः॑ प्रति॒शीव॑री प्रति॒शीव॑री॒ भूमि॑ स्त्वा । \newline
3. प्र॒ति॒शीव॒रीति॑ प्रति - शीव॑री । \newline
4. भूमि॑ स्त्वा त्वा॒ भूमि॒र् भूमि॑ स्त्वो॒पस्थ॑ उ॒पस्थे᳚ त्वा॒ भूमि॒र् भूमि॑ स्त्वो॒पस्थे᳚ । \newline
5. त्वो॒पस्थ॑ उ॒पस्थे᳚ त्वा त्वो॒पस्थ॒ ओपस्थे᳚ त्वा त्वो॒पस्थ॒ आ । \newline
6. उ॒पस्थ॒ ओपस्थ॑ उ॒पस्थ॒ आ ऽधि॑ताधि॒तोपस्थ॑ उ॒पस्थ॒ आ ऽधि॑त । \newline
7. उ॒पस्थ॒ इत्यु॒प - स्थे॒ । \newline
8. आ ऽधि॑ताधि॒ता ऽधि॑त । \newline
9. अ॒धि॒तेत्य॑धित । \newline
10. स्यो॒ना ऽस्मा॑ अस्मै स्यो॒ना स्यो॒ना ऽस्मै॑ सु॒षदा॑ सु॒षदा᳚ ऽस्मै स्यो॒ना स्यो॒ना ऽस्मै॑ सु॒षदा᳚ । \newline
11. अ॒स्मै॒ सु॒षदा॑ सु॒षदा᳚ ऽस्मा अस्मै सु॒षदा॑ भव भव सु॒षदा᳚ ऽस्मा अस्मै सु॒षदा॑ भव । \newline
12. सु॒षदा॑ भव भव सु॒षदा॑ सु॒षदा॑ भव॒ यच्छ॒ यच्छ॑ भव सु॒षदा॑ सु॒षदा॑ भव॒ यच्छ॑ । \newline
13. सु॒षदेति॑ सु - सदा᳚ । \newline
14. भ॒व॒ यच्छ॒ यच्छ॑ भव भव॒ यच्छा᳚स्मा अस्मै॒ यच्छ॑ भव भव॒ यच्छा᳚स्मै । \newline
15. यच्छा᳚स्मा अस्मै॒ यच्छ॒ यच्छा᳚स्मै॒ शर्म॒ शर्मा᳚स्मै॒ यच्छ॒ यच्छा᳚स्मै॒ शर्म॑ । \newline
16. अ॒स्मै॒ शर्म॒ शर्मा᳚स्मा अस्मै॒ शर्म॑ स॒प्रथाः᳚ स॒प्रथाः॒ शर्मा᳚स्मा अस्मै॒ शर्म॑ स॒प्रथाः᳚ । \newline
17. शर्म॑ स॒प्रथाः᳚ स॒प्रथाः॒ शर्म॒ शर्म॑ स॒प्रथाः᳚ । \newline
18. स॒प्रथा॒ इति॑ स - प्रथाः᳚ । \newline
19. उ॒प॒या॒मगृ॑हीतो ऽस्य स्युपया॒मगृ॑हीत उपया॒मगृ॑हीतो॒ ऽसीन्द्रा॒ये न्द्रा॑ यास्युपया॒मगृ॑हीत उपया॒मगृ॑हीतो॒ ऽसीन्द्रा॑य । \newline
20. उ॒प॒या॒मगृ॑हीत॒ इत्यु॑पया॒म - गृ॒ही॒तः॒ । \newline
21. अ॒सीन्द्रा॒ये न्द्रा॑ यास्य॒सीन्द्रा॑य त्वा॒ त्वेन्द्रा॑ यास्य॒सीन्द्रा॑य त्वा । \newline
22. इन्द्रा॑य त्वा॒ त्वेन्द्रा॒ये न्द्रा॑य त्वा षोड॒शिने॑ षोड॒शिने॒ त्वेन्द्रा॒ये न्द्रा॑य त्वा षोड॒शिने᳚ । \newline
23. त्वा॒ षो॒ड॒शिने॑ षोड॒शिने᳚ त्वा त्वा षोड॒शिन॑ ए॒ष ए॒ष षो॑ड॒शिने᳚ त्वा त्वा षोड॒शिन॑ ए॒षः । \newline
24. षो॒ड॒शिन॑ ए॒ष ए॒ष षो॑ड॒शिने॑ षोड॒शिन॑ ए॒ष ते॑ त ए॒ष षो॑ड॒शिने॑ षोड॒शिन॑ ए॒ष ते᳚ । \newline
25. ए॒ष ते॑ त ए॒ष ए॒ष ते॒ योनि॒र् योनि॑ स्त ए॒ष ए॒ष ते॒ योनिः॑ । \newline
26. ते॒ योनि॒र् योनि॑ स्ते ते॒ योनि॒ रिन्द्रा॒ये न्द्रा॑य॒ योनि॑ स्ते ते॒ योनि॒ रिन्द्रा॑य । \newline
27. योनि॒ रिन्द्रा॒ये न्द्रा॑य॒ योनि॒र् योनि॒ रिन्द्रा॑य त्वा॒ त्वेन्द्रा॑य॒ योनि॒र् योनि॒ रिन्द्रा॑य त्वा । \newline
28. इन्द्रा॑य त्वा॒ त्वेन्द्रा॒ये न्द्रा॑य त्वा षोड॒शिने॑ षोड॒शिने॒ त्वेन्द्रा॒ये न्द्रा॑य त्वा षोड॒शिने᳚ । \newline
29. त्वा॒ षो॒ड॒शिने॑ षोड॒शिने᳚ त्वा त्वा षोड॒शिने᳚ । \newline
30. षो॒ड॒शिन॒ इति॑ षोड॒शिने᳚ । \newline
\pagebreak
\markright{ TS 1.4.41.1  \hfill https://www.vedavms.in \hfill}
\addcontentsline{toc}{section}{ TS 1.4.41.1 }
\section*{ TS 1.4.41.1 }

\textbf{TS 1.4.41.1 } \newline
\textbf{Samhita Paata} \newline

म॒हाꣳ इन्द्रो॒ वज्र॑बाहुः षोड॒शी शर्म॑ यच्छतु । स्व॒स्ति नो॑ म॒घवा॑ करोतु॒ हन्तु॑ पा॒प्मानं॒ ॅयो᳚ऽस्मान् द्वेष्टि॑ ॥ उ॒प॒या॒मगृ॑हीतो॒ऽसीन्द्रा॑य त्वा षोड॒शिन॑ ए॒ष ते॒ योनि॒रिन्द्रा॑य त्वा षोड॒शिने᳚ ॥ \newline

\textbf{Pada Paata} \newline

म॒हान् । इन्द्रः॑ । वज्र॑बाहु॒रिति॒ वज्र॑ - बा॒हुः॒ । षो॒ड॒शी । शर्म॑ । य॒च्छ॒तु॒ ॥ स्व॒स्ति । नः॒ । म॒घवेति॑ म॒घ - वा॒ । क॒रो॒तु॒ । हन्तु॑ । पा॒प्मान᳚म् । यः । अ॒स्मान् । द्वेष्टि॑ ॥ उ॒प॒या॒मगृ॑हीत॒ इत्यु॑पया॒म - गृ॒ही॒तः॒ । अ॒सि॒ । इन्द्रा॑य । त्वा॒ । षो॒ड॒शिने᳚ । ए॒षः । ते॒ । योनिः॑ । इन्द्रा॑य । त्वा॒ । षो॒ड॒शिने᳚ ॥  \newline


\textbf{Krama Paata} \newline

म॒हाꣳ इन्द्रः॑ । इन्द्रो॒ वज्र॑बाहुः । वज्र॑बाहुः षोड॒शी । वज्र॑बाहु॒रिति॒ वज्र॑ - बा॒हुः॒ । षो॒ड॒शी शर्म॑ । शर्म॑ यच्छतु । य॒च्छ॒त्विति॑ यच्छतु ॥ स्व॒स्ति नः॑ । नो॒ म॒घवा᳚ । म॒घवा॑ करोतु । म॒घवेति॑ म॒घ - वा॒ । क॒रो॒तु॒ हन्तु॑ । हन्तु॑ पा॒प्मान᳚म् । पा॒प्मानं॒ ॅयः । यो᳚ऽस्मान् । अ॒स्मान् द्वेष्टि॑ । द्वेष्टीति॒ द्वेष्टि॑ ॥ उ॒प॒या॒मगृ॑हीतोऽसि । उ॒प॒या॒मगृ॑हीत॒ इत्यु॑पया॒म - गृ॒ही॒तः॒ । अ॒सीन्द्रा॑य । इन्द्रा॑य त्वा । त्वा॒ षो॒ड॒शिने᳚ । षो॒ड॒शिन॑ ए॒षः । ए॒ष ते᳚ । ते॒ योनिः॑ । योनि॒रिन्द्रा॑य । इन्द्रा॑य त्वा । त्वा॒ षो॒ड॒शिने᳚ । षो॒ड॒शिन॒ इति॑ षोड॒शिने᳚ । \newline

\textbf{Jatai Paata} \newline

1. म॒हाꣳ इन्द्र॒ इन्द्रो॑ म॒हान् म॒हाꣳ इन्द्रः॑ । \newline
2. इन्द्रो॒ वज्र॑बाहु॒र् वज्र॑बाहु॒ रिन्द्र॒ इन्द्रो॒ वज्र॑बाहुः । \newline
3. वज्र॑बाहु ष्षोड॒शी षो॑ड॒शी वज्र॑बाहु॒र् वज्र॑बाहु ष्षोड॒शी । \newline
4. वज्र॑बाहु॒रिति॒ वज्र॑ - बा॒हुः॒ । \newline
5. षो॒ड॒शी शर्म॒ शर्म॑ षोड॒शी षो॑ड॒शी शर्म॑ । \newline
6. शर्म॑ यच्छतु यच्छतु॒ शर्म॒ शर्म॑ यच्छतु । \newline
7. य॒च्छ॒त्विति॑ यच्छतु । \newline
8. स्व॒स्ति नो॑ नः स्व॒स्ति स्व॒स्ति नः॑ । \newline
9. नो॒ म॒घवा॑ म॒घवा॑ नो नो म॒घवा᳚ । \newline
10. म॒घवा॑ करोतु करोतु म॒घवा॑ म॒घवा॑ करोतु । \newline
11. म॒घवेति॑ म॒घ - वा॒ । \newline
12. क॒रो॒तु॒ हन्तु॒ हन्तु॑ करोतु करोतु॒ हन्तु॑ । \newline
13. हन्तु॑ पा॒प्मान॑म् पा॒प्मान॒(ग्म्॒) हन्तु॒ हन्तु॑ पा॒प्मान᳚म् । \newline
14. पा॒प्मानं॒ ॅयो यः पा॒प्मान॑म् पा॒प्मानं॒ ॅयः । \newline
15. यो᳚ ऽस्मा न॒स्मान्. यो यो᳚ ऽस्मान् । \newline
16. अ॒स्मान् द्वेष्टि॒ द्वेष्ट्य॒स्मा न॒स्मान् द्वेष्टि॑ । \newline
17. द्वेष्टीति॒ द्वेष्टि॑ । \newline
18. उ॒प॒या॒मगृ॑हीतो ऽस्यस्युपया॒मगृ॑हीत उपया॒मगृ॑हीतो ऽसि । \newline
19. उ॒प॒या॒मगृ॑हीत॒ इत्यु॑पया॒म - गृ॒ही॒तः॒ । \newline
20. अ॒सीन्द्रा॒ये न्द्रा॑यास्य॒सीन्द्रा॑य । \newline
21. इन्द्रा॑य त्वा॒ त्वेन्द्रा॒ये न्द्रा॑य त्वा । \newline
22. त्वा॒ षो॒ड॒शिने॑ षोड॒शिने᳚ त्वा त्वा षोड॒शिने᳚ । \newline
23. षो॒ड॒शिन॑ ए॒ष ए॒ष षो॑ड॒शिने॑ षोड॒शिन॑ ए॒षः । \newline
24. ए॒ष ते॑ त ए॒ष ए॒ष ते᳚ । \newline
25. ते॒ योनि॒र् योनि॑ स्ते ते॒ योनिः॑ । \newline
26. योनि॒ रिन्द्रा॒ये न्द्रा॑य॒ योनि॒र् योनि॒ रिन्द्रा॑य । \newline
27. इन्द्रा॑य त्वा॒ त्वेन्द्रा॒ये न्द्रा॑य त्वा । \newline
28. त्वा॒ षो॒ड॒शिने॑ षोड॒शिने᳚ त्वा त्वा षोड॒शिने᳚ । \newline
29. षो॒ड॒शिन॒ इति॑ षोड॒शिने᳚ । \newline

\textbf{Ghana Paata } \newline

1. म॒हाꣳ इन्द्र॒ इन्द्रो॑ म॒हान् म॒हाꣳ इन्द्रो॒ वज्र॑बाहु॒र् वज्र॑बाहु॒ रिन्द्रो॑ म॒हान् म॒हाꣳ इन्द्रो॒ वज्र॑बाहुः । \newline
2. इन्द्रो॒ वज्र॑बाहु॒र् वज्र॑बाहु॒ रिन्द्र॒ इन्द्रो॒ वज्र॑बाहु ष्षोड॒शी षो॑ड॒शी वज्र॑बाहु॒ रिन्द्र॒ इन्द्रो॒ वज्र॑बाहु ष्षोड॒शी । \newline
3. वज्र॑बाहु ष्षोड॒शी षो॑ड॒शी वज्र॑बाहु॒र् वज्र॑बाहु ष्षोड॒शी शर्म॒ शर्म॑ षोड॒शी वज्र॑बाहु॒र् वज्र॑बाहु ष्षोड॒शी शर्म॑ । \newline
4. वज्र॑बाहु॒रिति॒ वज्र॑ - बा॒हुः॒ । \newline
5. षो॒ड॒शी शर्म॒ शर्म॑ षोड॒शी षो॑ड॒शी शर्म॑ यच्छतु यच्छतु॒ शर्म॑ षोड॒शी षो॑ड॒शी शर्म॑ यच्छतु । \newline
6. शर्म॑ यच्छतु यच्छतु॒ शर्म॒ शर्म॑ यच्छतु । \newline
7. य॒च्छ॒त्विति॑ यच्छतु । \newline
8. स्व॒स्ति नो॑ नः स्व॒स्ति स्व॒स्ति नो॑ म॒घवा॑ म॒घवा॑ नः स्व॒स्ति स्व॒स्ति नो॑ म॒घवा᳚ । \newline
9. नो॒ म॒घवा॑ म॒घवा॑ नो नो म॒घवा॑ करोतु करोतु म॒घवा॑ नो नो म॒घवा॑ करोतु । \newline
10. म॒घवा॑ करोतु करोतु म॒घवा॑ म॒घवा॑ करोतु॒ हन्तु॒ हन्तु॑ करोतु म॒घवा॑ म॒घवा॑ करोतु॒ हन्तु॑ । \newline
11. म॒घवेति॑ म॒घ - वा॒ । \newline
12. क॒रो॒तु॒ हन्तु॒ हन्तु॑ करोतु करोतु॒ हन्तु॑ पा॒प्मान॑म् पा॒प्मान॒(ग्म्॒) हन्तु॑ करोतु करोतु॒ हन्तु॑ पा॒प्मान᳚म् । \newline
13. हन्तु॑ पा॒प्मान॑म् पा॒प्मान॒(ग्म्॒) हन्तु॒ हन्तु॑ पा॒प्मानं॒ ॅयो यः पा॒प्मान॒(ग्म्॒) हन्तु॒ हन्तु॑ पा॒प्मानं॒ ॅयः । \newline
14. पा॒प्मानं॒ ॅयो यः पा॒प्मान॑म् पा॒प्मानं॒ ॅयो᳚ ऽस्मा न॒स्मान्. यः पा॒प्मान॑म् पा॒प्मानं॒ ॅयो᳚ ऽस्मान् । \newline
15. यो᳚ ऽस्मा न॒स्मान्. यो यो᳚ ऽस्मान् द्वेष्टि॒ द्वेष्ट्य॒स्मान् यो यो᳚ ऽस्मान् द्वेष्टि॑ । \newline
16. अ॒स्मान् द्वेष्टि॒ द्वेष्ट्य॒स्मा न॒स्मान् द्वेष्टि॑ । \newline
17. द्वेष्टीति॒ द्वेष्टि॑ । \newline
18. उ॒प॒या॒मगृ॑हीतो ऽस्य स्युपया॒मगृ॑हीत उपया॒मगृ॑हीतो॒ ऽसीन्द्रा॒ये न्द्रा॑ यास्युपया॒मगृ॑हीत उपया॒मगृ॑हीतो॒ ऽसीन्द्रा॑य । \newline
19. उ॒प॒या॒मगृ॑हीत॒ इत्यु॑पया॒म - गृ॒ही॒तः॒ । \newline
20. अ॒सीन्द्रा॒ये न्द्रा॑ यास्य॒सीन्द्रा॑य त्वा॒ त्वेन्द्रा॑ यास्य॒सीन्द्रा॑य त्वा । \newline
21. इन्द्रा॑य त्वा॒ त्वेन्द्रा॒ये न्द्रा॑य त्वा षोड॒शिने॑ षोड॒शिने॒ त्वेन्द्रा॒ये न्द्रा॑य त्वा षोड॒शिने᳚ । \newline
22. त्वा॒ षो॒ड॒शिने॑ षोड॒शिने᳚ त्वा त्वा षोड॒शिन॑ ए॒ष ए॒ष षो॑ड॒शिने᳚ त्वा त्वा षोड॒शिन॑ ए॒षः । \newline
23. षो॒ड॒शिन॑ ए॒ष ए॒ष षो॑ड॒शिने॑ षोड॒शिन॑ ए॒ष ते॑ त ए॒ष षो॑ड॒शिने॑ षोड॒शिन॑ ए॒ष ते᳚ । \newline
24. ए॒ष ते॑ त ए॒ष ए॒ष ते॒ योनि॒र् योनि॑ स्त ए॒ष ए॒ष ते॒ योनिः॑ । \newline
25. ते॒ योनि॒र् योनि॑ स्ते ते॒ योनि॒ रिन्द्रा॒ये न्द्रा॑य॒ योनि॑ स्ते ते॒ योनि॒ रिन्द्रा॑य । \newline
26. योनि॒ रिन्द्रा॒ये न्द्रा॑य॒ योनि॒र् योनि॒ रिन्द्रा॑य त्वा॒ त्वेन्द्रा॑य॒ योनि॒र् योनि॒ रिन्द्रा॑य त्वा । \newline
27. इन्द्रा॑य त्वा॒ त्वेन्द्रा॒ये न्द्रा॑य त्वा षोड॒शिने॑ षोड॒शिने॒ त्वेन्द्रा॒ये न्द्रा॑य त्वा षोड॒शिने᳚ । \newline
28. त्वा॒ षो॒ड॒शिने॑ षोड॒शिने᳚ त्वा त्वा षोड॒शिने᳚ । \newline
29. षो॒ड॒शिन॒ इति॑ षोड॒शिने᳚ । \newline
\pagebreak
\markright{ TS 1.4.42.1  \hfill https://www.vedavms.in \hfill}
\addcontentsline{toc}{section}{ TS 1.4.42.1 }
\section*{ TS 1.4.42.1 }

\textbf{TS 1.4.42.1 } \newline
\textbf{Samhita Paata} \newline

स॒जोषा॑ इन्द्र॒ सग॑णो म॒रुद्भिः॒ सोमं॑ पिब वृत्रहञ्छूर वि॒द्वान् । ज॒हि शत्रूꣳ॒॒ रप॒ मृधो॑ नुद॒स्वाऽथाभ॑यं कृणुहि वि॒श्वतो॑ नः ॥उ॒प॒या॒मगृ॑हीतो॒ऽसीन्द्रा॑य त्वा षोड॒शिन॑ ए॒ष ते॒ योनि॒रिन्द्रा॑य त्वा षोड॒शिने᳚ ॥ \newline

\textbf{Pada Paata} \newline

स॒जोषा॒ इति॑ स - जोषाः᳚ । इ॒न्द्र॒ । सग॑ण॒ इति॒ स - ग॒णः॒ । म॒रुद्भि॒रिति॑ म॒रुत् - भिः॒ । सोम᳚म् । पि॒ब॒ । वृ॒त्र॒ह॒न्निति॑ वृत्र - ह॒न्न् । शू॒र॒ । वि॒द्वान् ॥ ज॒हि । शत्रून्॑ । अपेति॑ । मृधः॑ । नु॒द॒स्व॒ । अथ॑ । अभ॑यम् । कृ॒णु॒हि॒ । वि॒श्वतः॑ । नः॒ ॥ उ॒प॒या॒मगृ॑हीत॒ इत्यु॑पया॒म - गृ॒ही॒तः॒ । अ॒सि॒ । इन्द्रा॑य । त्वा॒ । षो॒ड॒शिने᳚ । ए॒षः । ते॒ । योनिः॑ । इन्द्रा॑य । त्वा॒ । षो॒ड॒शिने᳚ ॥  \newline


\textbf{Krama Paata} \newline

स॒जोषा॑ इन्द्र । स॒जोषा॒ इति॑ स - जोषाः᳚ । इ॒न्द्र॒ सग॑णः । सग॑णो म॒रुद्भिः॑ । सग॑ण॒ इति॒ स - ग॒णः॒ । म॒रुद्भिः॒ सोम᳚म् । म॒रुद्भि॒रिति॑ म॒रुत् - भिः॒ । सोम॑म् पिब । पि॒ब॒ वृ॒त्र॒ह॒न्न्॒ । वृ॒त्र॒ह॒ञ्छू॒र॒ । वृ॒त्र॒ह॒न्निति॑ वृत्र - ह॒न्न्॒ । शू॒र॒ वि॒द्वान् । वि॒द्वानिति॑ वि॒द्वान् ॥ ज॒हि शत्रून्॑ । शत्रूꣳ॒॒रप॑ । अप॒ मृधः॑ । मृधो॑ नुदस्व । नु॒द॒स्वाथ॑ । अथाभ॑यम् । अभ॑यम् कृणुहि । कृ॒णु॒हि॒ वि॒श्वतः॑ । वि॒श्वतो॑ नः । न॒ इति॑ नः ॥ उ॒प॒या॒मगृ॑हीतोऽसि । उ॒प॒या॒मगृ॑हीत॒ इत्यु॑पया॒म - गृ॒ही॒तः॒ । अ॒सीन्द्रा॑य । इन्द्रा॑य त्वा । त्वा॒ षो॒ड॒शिने᳚ । षो॒ड॒शिन॑ ए॒षः । ए॒ष ते᳚ । ते॒ योनिः॑ । योनि॒रिन्द्रा॑य । इन्द्रा॑य त्वा । त्वा॒ षो॒ड॒शिने᳚ । षो॒ड॒शिन॒ इति॑ षोड॒शिने᳚ । \newline

\textbf{Jatai Paata} \newline

1. स॒जोषा॑ इन्द्रेन्द्र स॒जोषाः᳚ स॒जोषा॑ इन्द्र । \newline
2. स॒जोषा॒ इति॑ स - जोषाः᳚ । \newline
3. इ॒न्द्र॒ सग॑णः॒ सग॑ण इन्द्रेन्द्र॒ सग॑णः । \newline
4. सग॑णो म॒रुद्भि॑र् म॒रुद्भिः॒ सग॑णः॒ सग॑णो म॒रुद्भिः॑ । \newline
5. सग॑ण॒ इति॒ स - ग॒णः॒ । \newline
6. म॒रुद्भिः॒ सोम॒(ग्म्॒) सोम॑म् म॒रुद्भि॑र् म॒रुद्भिः॒ सोम᳚म् । \newline
7. म॒रुद्भि॒रिति॑ म॒रुत् - भिः॒ । \newline
8. सोम॑म् पिब पिब॒ सोम॒(ग्म्॒) सोम॑म् पिब । \newline
9. पि॒ब॒ वृ॒त्र॒ह॒न्॒ वृ॒त्र॒ह॒न् पि॒ब॒ पि॒ब॒ वृ॒त्र॒ह॒न्न् । \newline
10. वृ॒त्र॒ह॒ञ् छू॒र॒ शू॒र॒ वृ॒त्र॒ह॒न्॒ वृ॒त्र॒ह॒ञ् छू॒र॒ । \newline
11. वृ॒त्र॒ह॒न्निति॑ वृत्र - ह॒न्न् । \newline
12. शू॒र॒ वि॒द्वान्. वि॒द्वाञ् छू॑र शूर वि॒द्वान् । \newline
13. वि॒द्वानिति॑ वि॒द्वान् । \newline
14. ज॒हि शत्रू॒ञ् छत्रू᳚न् ज॒हि ज॒हि शत्रून्॑ । \newline
15. शत्रू॒(ग्म्॒) रपाप॒ शत्रू॒ञ् छत्रूꣳ॒ रप॑ । \newline
16. अप॒ मृधो॒ मृधो ऽपाप॒ मृधः॑ । \newline
17. मृधो॑ नुदस्व नुदस्व॒ मृधो॒ मृधो॑ नुदस्व । \newline
18. नु॒द॒स्वाथाथ॑ नुदस्व नुद॒स्वाथ॑ । \newline
19. अथाभ॑य॒ मभ॑य॒ मथाथाभ॑यम् । \newline
20. अभ॑यम् कृणुहि कृणु॒ह्यभ॑य॒ मभ॑यम् कृणुहि । \newline
21. कृ॒णु॒हि॒ वि॒श्वतो॑ वि॒श्वत॑स्कृणुहि कृणुहि वि॒श्वतः॑ । \newline
22. वि॒श्वतो॑ नो नो वि॒श्वतो॑ वि॒श्वतो॑ नः । \newline
23. न॒ इति॑ नः॒ । \newline
24. उ॒प॒या॒मगृ॑हीतो ऽस्यस्युपया॒मगृ॑हीत उपया॒मगृ॑हीतो ऽसि । \newline
25. उ॒प॒या॒मगृ॑हीत॒ इत्यु॑पया॒म - गृ॒ही॒तः॒ । \newline
26. अ॒सीन्द्रा॒ये न्द्रा॑यास्य॒सीन्द्रा॑य । \newline
27. इन्द्रा॑य त्वा॒ त्वेन्द्रा॒ये न्द्रा॑य त्वा । \newline
28. त्वा॒ षो॒ड॒शिने॑ षोड॒शिने᳚ त्वा त्वा षोड॒शिने᳚ । \newline
29. षो॒ड॒शिन॑ ए॒ष ए॒ष षो॑ड॒शिने॑ षोड॒शिन॑ ए॒षः । \newline
30. ए॒ष ते॑ त ए॒ष ए॒ष ते᳚ । \newline
31. ते॒ योनि॒र् योनि॑ स्ते ते॒ योनिः॑ । \newline
32. योनि॒ रिन्द्रा॒ये न्द्रा॑य॒ योनि॒र् योनि॒ रिन्द्रा॑य । \newline
33. इन्द्रा॑य त्वा॒ त्वेन्द्रा॒ये न्द्रा॑य त्वा । \newline
34. त्वा॒ षो॒ड॒शिने॑ षोड॒शिने᳚ त्वा त्वा षोड॒शिने᳚ । \newline
35. षो॒ड॒शिन॒ इति॑ षोड॒शिने᳚ । \newline

\textbf{Ghana Paata } \newline

1. स॒जोषा॑ इन्द्रे न्द्र स॒जोषाः᳚ स॒जोषा॑ इन्द्र॒ सग॑णः॒ सग॑ण इन्द्र स॒जोषाः᳚ स॒जोषा॑ इन्द्र॒ सग॑णः । \newline
2. स॒जोषा॒ इति॑ स - जोषाः᳚ । \newline
3. इ॒न्द्र॒ सग॑णः॒ सग॑ण इन्द्रे न्द्र॒ सग॑णो म॒रुद्भि॑र् म॒रुद्भिः॒ सग॑ण इन्द्रे न्द्र॒ सग॑णो म॒रुद्भिः॑ । \newline
4. सग॑णो म॒रुद्भि॑र् म॒रुद्भिः॒ सग॑णः॒ सग॑णो म॒रुद्भिः॒ सोम॒(ग्म्॒) सोम॑म् म॒रुद्भिः॒ सग॑णः॒ सग॑णो म॒रुद्भिः॒ सोम᳚म् । \newline
5. सग॑ण॒ इति॒ स - ग॒णः॒ । \newline
6. म॒रुद्भिः॒ सोम॒(ग्म्॒) सोम॑म् म॒रुद्भि॑र् म॒रुद्भिः॒ सोम॑म् पिब पिब॒ सोम॑म् म॒रुद्भि॑र् म॒रुद्भिः॒ सोम॑म् पिब । \newline
7. म॒रुद्भि॒रिति॑ म॒रुत् - भिः॒ । \newline
8. सोम॑म् पिब पिब॒ सोम॒(ग्म्॒) सोम॑म् पिब वृत्रहन् वृत्रहन् पिब॒ सोम॒(ग्म्॒) सोम॑म् पिब वृत्रहन्न् । \newline
9. पि॒ब॒ वृ॒त्र॒ह॒न्॒ वृ॒त्र॒ह॒न् पि॒ब॒ पि॒ब॒ वृ॒त्र॒ह॒ञ् छू॒र॒ शू॒र॒ वृ॒त्र॒ह॒न् पि॒ब॒ पि॒ब॒ वृ॒त्र॒ह॒ञ् छू॒र॒ । \newline
10. वृ॒त्र॒ह॒ञ् छू॒र॒ शू॒र॒ वृ॒त्र॒ह॒न्॒ वृ॒त्र॒ह॒ञ् छू॒र॒ वि॒द्वान् वि॒द्वाञ् छू॑र वृत्रहन् वृत्रहञ् छूर वि॒द्वान् । \newline
11. वृ॒त्र॒ह॒न्निति॑ वृत्र - ह॒न्न् । \newline
12. शू॒र॒ वि॒द्वान्. वि॒द्वाञ् छू॑र शूर वि॒द्वान् । \newline
13. वि॒द्वानिति॑ वि॒द्वान् । \newline
14. ज॒हि शत्रू॒ञ् छत्रू᳚न् ज॒हि ज॒हि शत्रूꣳ॒॒ रपाप॒ शत्रू᳚न् ज॒हि ज॒हि शत्रूꣳ॒॒ रप॑ । \newline
15. शत्रू॒(ग्म्॒) रपाप॒ शत्रू॒ञ् छत्रूꣳ॒॒ रप॒ मृधो॒ मृधो ऽप॒ शत्रू॒ञ् छत्रूꣳ॒॒ रप॒ मृधः॑ । \newline
16. अप॒ मृधो॒ मृधो ऽपाप॒ मृधो॑ नुदस्व नुदस्व॒ मृधो ऽपाप॒ मृधो॑ नुदस्व । \newline
17. मृधो॑ नुदस्व नुदस्व॒ मृधो॒ मृधो॑ नुद॒ स्वाथाथ॑ नुदस्व॒ मृधो॒ मृधो॑ नुद॒स्वाथ॑ । \newline
18. नु॒द॒ स्वाथाथ॑ नुदस्व नुद॒ स्वाथाभ॑य॒ मभ॑य॒ मथ॑ नुदस्व नुद॒ स्वाथाभ॑यम् । \newline
19. अथाभ॑य॒ मभ॑य॒ मथा थाभ॑यम् कृणुहि कृणु॒ह्यभ॑य॒ मथा थाभ॑यम् कृणुहि । \newline
20. अभ॑यम् कृणुहि कृणु॒ह्यभ॑य॒ मभ॑यम् कृणुहि वि॒श्वतो॑ वि॒श्वत॑ स्कृणु॒ह्यभ॑य॒ मभ॑यम् कृणुहि वि॒श्वतः॑ । \newline
21. कृ॒णु॒हि॒ वि॒श्वतो॑ वि॒श्वत॑ स्कृणुहि कृणुहि वि॒श्वतो॑ नो नो वि॒श्वत॑ स्कृणुहि कृणुहि वि॒श्वतो॑ नः । \newline
22. वि॒श्वतो॑ नो नो वि॒श्वतो॑ वि॒श्वतो॑ नः । \newline
23. न॒ इति॑ नः॒ । \newline
24. उ॒प॒या॒मगृ॑हीतो ऽस्य स्युपया॒मगृ॑हीत उपया॒मगृ॑हीतो॒ ऽसीन्द्रा॒ये न्द्रा॑ यास्युपया॒मगृ॑हीत उपया॒मगृ॑हीतो॒ ऽसीन्द्रा॑य । \newline
25. उ॒प॒या॒मगृ॑हीत॒ इत्यु॑पया॒म - गृ॒ही॒तः॒ । \newline
26. अ॒सीन्द्रा॒ये न्द्रा॑ यास्य॒सीन्द्रा॑य त्वा॒ त्वेन्द्रा॑ यास्य॒सीन्द्रा॑य त्वा । \newline
27. इन्द्रा॑य त्वा॒ त्वेन्द्रा॒ये न्द्रा॑य त्वा षोड॒शिने॑ षोड॒शिने॒ त्वेन्द्रा॒ये न्द्रा॑य त्वा षोड॒शिने᳚ । \newline
28. त्वा॒ षो॒ड॒शिने॑ षोड॒शिने᳚ त्वा त्वा षोड॒शिन॑ ए॒ष ए॒ष षो॑ड॒शिने᳚ त्वा त्वा षोड॒शिन॑ ए॒षः । \newline
29. षो॒ड॒शिन॑ ए॒ष ए॒ष षो॑ड॒शिने॑ षोड॒शिन॑ ए॒ष ते॑ त ए॒ष षो॑ड॒शिने॑ षोड॒शिन॑ ए॒ष ते᳚ । \newline
30. ए॒ष ते॑ त ए॒ष ए॒ष ते॒ योनि॒र् योनि॑ स्त ए॒ष ए॒ष ते॒ योनिः॑ । \newline
31. ते॒ योनि॒र् योनि॑ स्ते ते॒ योनि॒ रिन्द्रा॒ये न्द्रा॑य॒ योनि॑ स्ते ते॒ योनि॒ रिन्द्रा॑य । \newline
32. योनि॒ रिन्द्रा॒ये न्द्रा॑य॒ योनि॒र् योनि॒ रिन्द्रा॑य त्वा॒ त्वेन्द्रा॑य॒ योनि॒र् योनि॒ रिन्द्रा॑य त्वा । \newline
33. इन्द्रा॑य त्वा॒ त्वेन्द्रा॒ये न्द्रा॑य त्वा षोड॒शिने॑ षोड॒शिने॒ त्वेन्द्रा॒ये न्द्रा॑य त्वा षोड॒शिने᳚ । \newline
34. त्वा॒ षो॒ड॒शिने॑ षोड॒शिने᳚ त्वा त्वा षोड॒शिने᳚ । \newline
35. षो॒ड॒शिन॒ इति॑ षोड॒शिने᳚ । \newline
\pagebreak
\markright{ TS 1.4.43.1  \hfill https://www.vedavms.in \hfill}
\addcontentsline{toc}{section}{ TS 1.4.43.1 }
\section*{ TS 1.4.43.1 }

\textbf{TS 1.4.43.1 } \newline
\textbf{Samhita Paata} \newline

उदु॒ त्यं जा॒तवे॑दसं दे॒वं ॅव॑हन्ति के॒तवः॑ । दृ॒शे विश्वा॑य॒ सूर्यं᳚ ॥ चि॒त्रं दे॒वाना॒-मुद॑गा॒दनी॑कं॒ चक्षु॑र् मि॒त्रस्य॒ वरु॑णस्या॒ऽग्नेः । आऽप्रा॒ द्यावा॑पृथि॒वी अ॒न्तरि॑क्षꣳ॒॒ सूर्य॑ आ॒त्मा जग॑तस्त॒स्थुष॑श्च ॥ अग्ने॒ नय॑ सु॒पथा॑ रा॒ये अ॒स्मान्. विश्वा॑नि देव व॒युना॑नि वि॒द्वान् । यु॒यो॒द्ध्य॑स्म-ज्जु॑हुरा॒ण मेनो॒ भूयि॑ष्ठां ते॒ नम॑उक्तिं विधेम ॥ दिवं॑ गच्छ॒ सुवः॑ पत रू॒पेण॑ - [ ] \newline

\textbf{Pada Paata} \newline

उदिति॑ । उ॒ । त्यम् । जा॒तवे॑दस॒मिति॑ जा॒त - वे॒द॒स॒म् । दे॒वम् । व॒ह॒न्ति॒ । के॒तवः॑ ॥ दृ॒शे । विश्वा॑य । सूर्य᳚म् ॥ चि॒त्रम् । दे॒वाना᳚म् । उदिति॑ । अ॒गा॒त् । अनी॑कम् । चक्षुः॑ । मि॒त्रस्य॑ । वरु॑णस्य । अ॒ग्नेः ॥ ऐति॑ । अ॒प्राः॒ । द्यावा॑पृथि॒वी इति॒ द्यावा᳚ - पृ॒थि॒वी । अ॒न्तरि॑क्षम् । सूर्यः॑ । आ॒त्मा । जग॑तः । त॒स्थुषः॑ । च॒ ॥ अग्ने᳚ । नय॑ । सु॒पथेति॑ सु - पथा᳚ । रा॒ये । अ॒स्मान् । विश्वा॑नि । दे॒व॒ । व॒युना॑नि । वि॒द्वान् ॥ यु॒यो॒धि । अ॒स्मत् । जु॒हु॒रा॒णम् । एनः॑ । भूयि॑ष्ठाम् । ते॒ । नम॑उक्ति॒मिति॒ नमः॑ - उ॒क्ति॒म् । वि॒धे॒म॒ ॥ दिव᳚म् । ग॒च्छ॒ । सुवः॑ । प॒त॒ । रू॒पेण॑ ।  \newline


\textbf{Krama Paata} \newline

उदु॑ । उ॒ त्यम् । त्यम् जा॒तवे॑दसम् । जा॒तवे॑दसम् दे॒वम् । जा॒तवे॑दस॒मिति॑ जा॒त - वे॒द॒स॒म् । दे॒वं ॅव॑हन्ति । व॒ह॒न्ति॒ के॒तवः॑ । के॒तव॒ इति॑ के॒तवः॑ ॥ दृ॒शे विश्वा॑य । विश्वा॑य॒ सूर्य᳚म् । सूर्य॒मिति॒ सूर्य᳚म् ॥ चि॒त्रम् दे॒वाना᳚म् । दे॒वाना॒मुत् । उद॑गात् । अ॒गा॒दनी॑कम् । अनी॑क॒म् चक्षुः॑ । चक्षु॑र् मि॒त्रस्य॑ । मि॒त्रस्य॒ वरु॑णस्य । वरु॑णस्या॒ग्नेः । अ॒ग्नेरित्य॒ग्नेः ॥ आऽप्राः᳚ । अ॒प्रा॒ द्यावा॑पृथि॒वी । द्यावा॑पृथि॒वी अ॒न्तरि॑क्षम् । द्यावा॑पृथि॒वी इति॒ द्यावा᳚ - पृ॒थि॒वी । अ॒न्तरि॑क्षꣳ॒॒ सूर्यः॑ । सूर्य॑ आ॒त्मा । आ॒त्मा जग॑तः । जग॑तस्त॒स्थुषः॑ । त॒स्थुष॑श्च । चेति॑ च ॥ अग्ने॒ नय॑ । नय॑ सु॒पथा᳚ । सु॒पथा॑ रा॒ये । सु॒पथेति॑ सु - पथा᳚ । रा॒ये अ॒स्मान् । अ॒स्मान्. विश्वा॑नि । विश्वा॑नि देव । दे॒व॒ व॒युना॑नि । व॒युना॑नि वि॒द्वान् । वि॒द्वानिति॑ वि॒द्वान् ॥ यु॒यो॒द्ध्य॑स्मत् । अ॒स्मज्जु॑हुरा॒णम् । जु॒हु॒रा॒णमेनः॑ । एनो॒ भूयि॑ष्ठाम् । भूयि॑ष्ठान्ते । ते॒ नम॑उक्तिम् । नम॑उक्तिं ॅविधेम । नम॑उक्ति॒मिति॒ नमः॑ - उ॒क्ति॒म् । वि॒धे॒मेति॑ विधेम ॥ दिव॑म् गच्छ । ग॒च्छ॒ सुवः॑ । सुवः॑ पत । प॒त॒ रू॒पेण॑ । रू॒पेण॑ वः \newline

\textbf{Jatai Paata} \newline

1. उदु॑ वु॒ वुदुदु॑ । \newline
2. उ॒ त्यम् त्यमु॑वु॒ त्यम् । \newline
3. त्यम् जा॒तवे॑दसम् जा॒तवे॑दस॒म् त्यम् त्यम् जा॒तवे॑दसम् । \newline
4. जा॒तवे॑दसम् दे॒वम् दे॒वम् जा॒तवे॑दसम् जा॒तवे॑दसम् दे॒वम् । \newline
5. जा॒तवे॑दस॒मिति॑ जा॒त - वे॒द॒स॒म् । \newline
6. दे॒वं ॅव॑हन्ति वहन्ति दे॒वम् दे॒वं ॅव॑हन्ति । \newline
7. व॒ह॒न्ति॒ के॒तवः॑ के॒तवो॑ वहन्ति वहन्ति के॒तवः॑ । \newline
8. के॒तव॒ इति॑ के॒तवः॑ । \newline
9. दृ॒शे विश्वा॑य॒ विश्वा॑य दृ॒शे दृ॒शे विश्वा॑य । \newline
10. विश्वा॑य॒ सूर्य॒(ग्म्॒) सूर्यं॒ ॅविश्वा॑य॒ विश्वा॑य॒ सूर्य᳚म् । \newline
11. सूर्य॒मिति॒ सूर्य᳚म् । \newline
12. चि॒त्रम् दे॒वाना᳚म् दे॒वाना᳚म् चि॒त्रम् चि॒त्रम् दे॒वाना᳚म् । \newline
13. दे॒वाना॒ मुदुद् दे॒वाना᳚म् दे॒वाना॒ मुत् । \newline
14. उद॑गा दगा॒ दुदु द॑गात् । \newline
15. अ॒गा॒दनी॑क॒ मनी॑क मगादगा॒ दनी॑कम् । \newline
16. अनी॑क॒म् चक्षु॒ श्चक्षु॒ रनी॑क॒ मनी॑क॒म् चक्षुः॑ । \newline
17. चक्षु॑र् मि॒त्रस्य॑ मि॒त्रस्य॒ चक्षु॒ श्चक्षु॑र् मि॒त्रस्य॑ । \newline
18. मि॒त्रस्य॒ वरु॑णस्य॒ वरु॑णस्य मि॒त्रस्य॑ मि॒त्रस्य॒ वरु॑णस्य । \newline
19. वरु॑णस्या॒ग्ने र॒ग्नेर् वरु॑णस्य॒ वरु॑णस्या॒ग्नेः । \newline
20. अ॒ग्नेरित्य॒ग्नेः । \newline
21. आ ऽप्रा॑ अप्रा॒ आ ऽप्राः᳚ । \newline
22. अ॒प्रा॒ द्यावा॑पृथि॒वी द्यावा॑पृथि॒वी अ॑प्रा अप्रा॒ द्यावा॑पृथि॒वी । \newline
23. द्यावा॑पृथि॒वी अ॒न्तरि॑क्ष म॒न्तरि॑क्ष॒म् द्यावा॑पृथि॒वी द्यावा॑पृथि॒वी अ॒न्तरि॑क्षम् । \newline
24. द्यावा॑पृथि॒वी इति॒ द्यावा᳚ - पृ॒थि॒वी । \newline
25. अ॒न्तरि॑क्ष॒(ग्म्॒) सूर्यः॒ सूर्यो॒ ऽन्तरि॑क्ष म॒न्तरि॑क्ष॒(ग्म्॒) सूर्यः॑ । \newline
26. सूर्य॑ आ॒त्मा ऽऽत्मा सूर्यः॒ सूर्य॑ आ॒त्मा । \newline
27. आ॒त्मा जग॑तो॒ जग॑त आ॒त्मा ऽऽत्मा जग॑तः । \newline
28. जग॑त स्त॒स्थुष॑ स्त॒स्थुषो॒ जग॑तो॒ जग॑त स्त॒स्थुषः॑ । \newline
29. त॒स्थुष॑श्च च त॒स्थुष॑ स्त॒स्थुष॑श्च । \newline
30. चेति॑ च । \newline
31. अग्ने॒ नय॒ नयाग्ने ऽग्ने॒ नय॑ । \newline
32. नय॑ सु॒पथा॑ सु॒पथा॒ नय॒ नय॑ सु॒पथा᳚ । \newline
33. सु॒पथा॑ रा॒ये रा॒ये सु॒पथा॑ सु॒पथा॑ रा॒ये । \newline
34. सु॒पथेति॑ सु - पथा᳚ । \newline
35. रा॒ये अ॒स्मा न॒स्मान् रा॒ये रा॒ये अ॒स्मान् । \newline
36. अ॒स्मान्. विश्वा॑नि॒ विश्वा᳚न्य॒स्मा न॒स्मान्. विश्वा॑नि । \newline
37. विश्वा॑नि देव देव॒ विश्वा॑नि॒ विश्वा॑नि देव । \newline
38. दे॒व॒ व॒युना॑नि व॒युना॑नि देव देव व॒युना॑नि । \newline
39. व॒युना॑नि वि॒द्वान्. वि॒द्वान्. व॒युना॑नि व॒युना॑नि वि॒द्वान् । \newline
40. वि॒द्वानिति॑ वि॒द्वान् । \newline
41. यु॒यो॒ध्य॑स्म द॒स्मद् यु॑यो॒धि यु॑यो॒ध्य॑स्मत् । \newline
42. अ॒स्मज् जु॑हुरा॒णम् जु॑हुरा॒ण म॒स्मद॒स्मज् जु॑हुरा॒णम् । \newline
43. जु॒हु॒रा॒ण मेन॒ एनो॑ जुहुरा॒णम् जु॑हुरा॒ण मेनः॑ । \newline
44. एनो॒ भूयि॑ष्ठा॒म् भूयि॑ष्ठा॒ मेन॒ एनो॒ भूयि॑ष्ठाम् । \newline
45. भूयि॑ष्ठाम् ते ते॒ भूयि॑ष्ठा॒म् भूयि॑ष्ठाम् ते । \newline
46. ते॒ नम॑उक्ति॒ न्नम॑उक्तिम् ते ते॒ नम॑उक्तिम् । \newline
47. नम॑उक्तिं ॅविधेम विधेम॒ नम॑उक्ति॒ न्नम॑उक्तिं ॅविधेम । \newline
48. नम॑उक्ति॒मिति॒ नमः॑ - उ॒क्ति॒म् । \newline
49. वि॒धे॒मेति॑ विधेम । \newline
50. दिव॑म् गच्छ गच्छ॒ दिव॒म् दिव॑म् गच्छ । \newline
51. ग॒च्छ॒ सुवः॒ सुव॑र् गच्छ गच्छ॒ सुवः॑ । \newline
52. सुवः॑ पत पत॒ सुवः॒ सुवः॑ पत । \newline
53. प॒त॒ रू॒पेण॑ रू॒पेण॑ पत पत रू॒पेण॑ । \newline
54. रू॒पेण॑ वो वो रू॒पेण॑ रू॒पेण॑ वः । \newline

\textbf{Ghana Paata } \newline

1. उदु॑ वु॒ वुदुदु॒ त्यम् त्य मु॒ वुदुदु॒ त्यम् । \newline
2. उ॒ त्यम् त्य मु॑ वु॒ त्यम् जा॒तवे॑दसम् जा॒तवे॑दस॒म् त्य मु॑ वु॒ त्यम् जा॒तवे॑दसम् । \newline
3. त्यम् जा॒तवे॑दसम् जा॒तवे॑दस॒म् त्यम् त्यम् जा॒तवे॑दसम् दे॒वम् दे॒वम् जा॒तवे॑दस॒म् त्यम् त्यम् जा॒तवे॑दसम् दे॒वम् । \newline
4. जा॒तवे॑दसम् दे॒वम् दे॒वम् जा॒तवे॑दसम् जा॒तवे॑दसम् दे॒वं ॅव॑हन्ति वहन्ति दे॒वम् जा॒तवे॑दसम् जा॒तवे॑दसम् दे॒वं ॅव॑हन्ति । \newline
5. जा॒तवे॑दस॒मिति॑ जा॒त - वे॒द॒स॒म् । \newline
6. दे॒वं ॅव॑हन्ति वहन्ति दे॒वम् दे॒वं ॅव॑हन्ति के॒तवः॑ के॒तवो॑ वहन्ति दे॒वम् दे॒वं ॅव॑हन्ति के॒तवः॑ । \newline
7. व॒ह॒न्ति॒ के॒तवः॑ के॒तवो॑ वहन्ति वहन्ति के॒तवः॑ । \newline
8. के॒तव॒ इति॑ के॒तवः॑ । \newline
9. दृ॒शे विश्वा॑य॒ विश्वा॑य दृ॒शे दृ॒शे विश्वा॑य॒ सूर्य॒(ग्म्॒) सूर्यं॒ ॅविश्वा॑य दृ॒शे दृ॒शे विश्वा॑य॒ सूर्य᳚म् । \newline
10. विश्वा॑य॒ सूर्य॒(ग्म्॒) सूर्यं॒ ॅविश्वा॑य॒ विश्वा॑य॒ सूर्य᳚म् । \newline
11. सूर्य॒मिति॒ सूर्य᳚म् । \newline
12. चि॒त्रम् दे॒वाना᳚म् दे॒वाना᳚म् चि॒त्रम् चि॒त्रम् दे॒वाना॒ मुदुद् दे॒वाना᳚म् चि॒त्रम् चि॒त्रम् दे॒वाना॒ मुत् । \newline
13. दे॒वाना॒ मुदुद् दे॒वाना᳚म् दे॒वाना॒ मुद॑ गाद गा॒दुद् दे॒वाना᳚म् दे॒वाना॒ मुद॑गात् । \newline
14. उद॑ गादगा॒ दुदुद॑ गा॒दनी॑क॒ मनी॑क मगा॒ दुदुद॑ गा॒दनी॑कम् । \newline
15. अ॒गा॒ दनी॑क॒ मनी॑क मगा दगा॒ दनी॑क॒म् चक्षु॒ श्चक्षु॒ रनी॑क मगा दगा॒ दनी॑क॒म् चक्षुः॑ । \newline
16. अनी॑क॒म् चक्षु॒ श्चक्षु॒ रनी॑क॒ मनी॑क॒म् चक्षु॑र् मि॒त्रस्य॑ मि॒त्रस्य॒ चक्षु॒ रनी॑क॒ मनी॑क॒म् चक्षु॑र् मि॒त्रस्य॑ । \newline
17. चक्षु॑र् मि॒त्रस्य॑ मि॒त्रस्य॒ चक्षु॒ श्चक्षु॑र् मि॒त्रस्य॒ वरु॑णस्य॒ वरु॑णस्य मि॒त्रस्य॒ चक्षु॒ श्चक्षु॑र् मि॒त्रस्य॒ वरु॑णस्य । \newline
18. मि॒त्रस्य॒ वरु॑णस्य॒ वरु॑णस्य मि॒त्रस्य॑ मि॒त्रस्य॒ वरु॑ण स्या॒ग्ने र॒ग्नेर् वरु॑णस्य मि॒त्रस्य॑ मि॒त्रस्य॒ वरु॑णस्या॒ग्नेः । \newline
19. वरु॑ण स्या॒ग्ने र॒ग्नेर् वरु॑णस्य॒ वरु॑ण स्या॒ग्नेः । \newline
20. अ॒ग्नेरित्य॒ग्नेः । \newline
21. आ ऽप्रा॑ अप्रा॒ आ ऽप्रा॒ द्यावा॑पृथि॒वी द्यावा॑पृथि॒वी अ॑प्रा॒ आ ऽप्रा॒ द्यावा॑पृथि॒वी । \newline
22. अ॒प्रा॒ द्यावा॑पृथि॒वी द्यावा॑पृथि॒वी अ॑प्रा अप्रा॒ द्यावा॑पृथि॒वी अ॒न्तरि॑क्ष म॒न्तरि॑क्ष॒म् द्यावा॑पृथि॒वी अ॑प्रा अप्रा॒ द्यावा॑पृथि॒वी अ॒न्तरि॑क्षम् । \newline
23. द्यावा॑पृथि॒वी अ॒न्तरि॑क्ष म॒न्तरि॑क्ष॒म् द्यावा॑पृथि॒वी द्यावा॑पृथि॒वी अ॒न्तरि॑क्ष॒(ग्म्॒) सूर्यः॒ सूर्यो॒ ऽन्तरि॑क्ष॒म् द्यावा॑पृथि॒वी द्यावा॑पृथि॒वी अ॒न्तरि॑क्ष॒(ग्म्॒) सूर्यः॑ । \newline
24. द्यावा॑पृथि॒वी इति॒ द्यावा᳚ - पृ॒थि॒वी । \newline
25. अ॒न्तरि॑क्ष॒(ग्म्॒) सूर्यः॒ सूर्यो॒ ऽन्तरि॑क्ष म॒न्तरि॑क्ष॒(ग्म्॒) सूर्य॑ आ॒त्मा ऽऽत्मा सूर्यो॒ ऽन्तरि॑क्ष म॒न्तरि॑क्ष॒(ग्म्॒) सूर्य॑ आ॒त्मा । \newline
26. सूर्य॑ आ॒त्मा ऽऽत्मा सूर्यः॒ सूर्य॑ आ॒त्मा जग॑तो॒ जग॑त आ॒त्मा सूर्यः॒ सूर्य॑ आ॒त्मा जग॑तः । \newline
27. आ॒त्मा जग॑तो॒ जग॑त आ॒त्मा ऽऽत्मा जग॑त स्त॒ स्थुष॑ स्त॒ स्थुषो॒ जग॑त आ॒त्मा ऽऽत्मा जग॑त स्त॒ स्थुषः॑ । \newline
28. जग॑त स्त॒ स्थुष॑ स्त॒ स्थुषो॒ जग॑तो॒ जग॑त स्त॒ स्थुष॑श्च च त॒ स्थुषो॒ जग॑तो॒ जग॑त स्त॒ स्थुष॑श्च । \newline
29. त॒ स्थुष॑श्च च त॒ स्थुष॑ स्त॒ स्थुष॑श्च । \newline
30. चेति॑ च । \newline
31. अग्ने॒ नय॒ नयाग्ने ऽग्ने॒ नय॑ सु॒पथा॑ सु॒पथा॒ नयाग्ने ऽग्ने॒ नय॑ सु॒पथा᳚ । \newline
32. नय॑ सु॒पथा॑ सु॒पथा॒ नय॒ नय॑ सु॒पथा॑ रा॒ये रा॒ये सु॒पथा॒ नय॒ नय॑ सु॒पथा॑ रा॒ये । \newline
33. सु॒पथा॑ रा॒ये रा॒ये सु॒पथा॑ सु॒पथा॑ रा॒ये अ॒स्मा न॒स्मान् रा॒ये सु॒पथा॑ सु॒पथा॑ रा॒ये अ॒स्मान् । \newline
34. सु॒पथेति॑ सु - पथा᳚ । \newline
35. रा॒ये अ॒स्मा न॒स्मान् रा॒ये रा॒ये अ॒स्मान्. विश्वा॑नि॒ विश्वा᳚ न्य॒स्मान् रा॒ये रा॒ये अ॒स्मान्. विश्वा॑नि । \newline
36. अ॒स्मान्. विश्वा॑नि॒ विश्वा᳚ न्य॒स्मा न॒स्मान्. विश्वा॑नि देव देव॒ विश्वा᳚ न्य॒स्मा न॒स्मान्. विश्वा॑नि देव । \newline
37. विश्वा॑नि देव देव॒ विश्वा॑नि॒ विश्वा॑नि देव व॒युना॑नि व॒युना॑नि देव॒ विश्वा॑नि॒ विश्वा॑नि देव व॒युना॑नि । \newline
38. दे॒व॒ व॒युना॑नि व॒युना॑नि देव देव व॒युना॑नि वि॒द्वान्. वि॒द्वान्. व॒युना॑नि देव देव व॒युना॑नि वि॒द्वान् । \newline
39. व॒युना॑नि वि॒द्वान्. वि॒द्वान्. व॒युना॑नि व॒युना॑नि वि॒द्वान् । \newline
40. वि॒द्वानिति॑ वि॒द्वान् । \newline
41. यु॒यो॒ध्य॑ स्मद॒स्मद् यु॑यो॒धि यु॑यो॒ ध्य॑स्मज् जु॑हुरा॒णम् जु॑हुरा॒ण म॒स्मद् यु॑यो॒धि यु॑यो॒ ध्य॑स्मज् जु॑हुरा॒णम् । \newline
42. अ॒स्मज् जु॑हुरा॒णम् जु॑हुरा॒ण म॒स्म द॒स्मज् जु॑हुरा॒ण मेन॒ एनो॑ जुहुरा॒ण म॒स्म द॒स्मज् जु॑हुरा॒ण मेनः॑ । \newline
43. जु॒हु॒रा॒ण मेन॒ एनो॑ जुहुरा॒णम् जु॑हुरा॒ण मेनो॒ भूयि॑ष्ठा॒म् भूयि॑ष्ठा॒ मेनो॑ जुहुरा॒णम् जु॑हुरा॒ण मेनो॒ भूयि॑ष्ठाम् । \newline
44. एनो॒ भूयि॑ष्ठा॒म् भूयि॑ष्ठा॒ मेन॒ एनो॒ भूयि॑ष्ठाम् ते ते॒ भूयि॑ष्ठा॒ मेन॒ एनो॒ भूयि॑ष्ठाम् ते । \newline
45. भूयि॑ष्ठाम् ते ते॒ भूयि॑ष्ठा॒म् भूयि॑ष्ठाम् ते॒ नम॑उक्ति॒म् नम॑उक्तिम् ते॒ भूयि॑ष्ठा॒म् भूयि॑ष्ठाम् ते॒ नम॑उक्तिम् । \newline
46. ते॒ नम॑उक्ति॒म् नम॑उक्तिम् ते ते॒ नम॑उक्तिं ॅविधेम विधेम॒ नम॑उक्तिम् ते ते॒ नम॑उक्तिं ॅविधेम । \newline
47. नम॑उक्तिं ॅविधेम विधेम॒ नम॑उक्ति॒म् नम॑उक्तिं ॅविधेम । \newline
48. नम॑उक्ति॒मिति॒ नमः॑ - उ॒क्ति॒म् । \newline
49. वि॒धे॒मेति॑ विधेम । \newline
50. दिव॑म् गच्छ गच्छ॒ दिव॒म् दिव॑म् गच्छ॒ सुवः॒ सुव॑र् गच्छ॒ दिव॒म् दिव॑म् गच्छ॒ सुवः॑ । \newline
51. ग॒च्छ॒ सुवः॒ सुव॑र् गच्छ गच्छ॒ सुवः॑ पत पत॒ सुव॑र् गच्छ गच्छ॒ सुवः॑ पत । \newline
52. सुवः॑ पत पत॒ सुवः॒ सुवः॑ पत रू॒पेण॑ रू॒पेण॑ पत॒ सुवः॒ सुवः॑ पत रू॒पेण॑ । \newline
53. प॒त॒ रू॒पेण॑ रू॒पेण॑ पत पत रू॒पेण॑ वो वो रू॒पेण॑ पत पत रू॒पेण॑ वः । \newline
54. रू॒पेण॑ वो वो रू॒पेण॑ रू॒पेण॑ वो रू॒पꣳ रू॒पं ॅवो॑ रू॒पेण॑ रू॒पेण॑ वो रू॒पम् । \newline
\pagebreak
\markright{ TS 1.4.43.2  \hfill https://www.vedavms.in \hfill}
\addcontentsline{toc}{section}{ TS 1.4.43.2 }
\section*{ TS 1.4.43.2 }

\textbf{TS 1.4.43.2 } \newline
\textbf{Samhita Paata} \newline

वो रू॒पम॒भ्यैमि॒ वय॑सा॒ वयः॑ । तु॒थो वो॑ वि॒श्ववे॑दा॒ वि भ॑जतु॒ वर्.षि॑ष्ठे॒ अधि॒ नाके᳚ ॥ ए॒तत्ते॑ अग्ने॒ राध॒ ऐति॒ सोम॑च्युतं॒ तन्मि॒त्रस्य॑ प॒था न॑य॒र्तस्य॑ प॒था प्रेत॑ च॒न्द्रद॑क्षिणा य॒ज्ञ्स्य॑ प॒था सु॑वि॒ता नय॑न्तीर् ब्राह्म॒णम॒द्य रा᳚द्ध्यास॒मृषि॑मार्.षे॒यं पि॑तृ॒मन्तं॑ पैतृम॒त्यꣳ सु॒धातु॑दक्षिणं॒ ॅवि सुवः॒ पश्य॒ व्य॑न्तरि॑क्षं॒ ॅयत॑स्व सद॒स्यै॑ ( ) र॒स्मद्दा᳚त्रा देव॒त्रा ग॑च्छत॒ मधु॑मतीः प्रदा॒तार॒मा वि॑श॒ताऽन॑वहाया॒ऽस्मान् दे॑व॒याने॑न प॒थेत॑ सु॒कृतां᳚ ॅलो॒के सी॑द॒त तन्नः॑ सꣳस्कृ॒तं ॥ \newline

\textbf{Pada Paata} \newline

वः॒ । रू॒पम् । अ॒भि । एति॑ । ए॒मि॒ । वय॑सा । वयः॑ ॥ तु॒थः । वः॒ । वि॒श्ववे॑दा॒ इति॑ वि॒श्व - वे॒दाः॒ । वीति॑ । भ॒ज॒तु॒ । वर्.षि॑ष्ठे । अधीति॑ । नाके᳚ ॥ ए॒तत् । ते॒ । अ॒ग्ने॒ । राधः॑ । एति॑ । ए॒ति॒ । सोम॑च्युत॒मिति॒ सोम॑ - च्यु॒त॒म् । तत् । मि॒त्रस्य॑ । प॒था । न॒य॒ । ऋ॒तस्य॑ । प॒था । प्रेति॑ । इ॒त॒ । च॒न्द्रद॑क्षिणा॒ इति॑ च॒न्द्र - द॒क्षि॒णाः॒ । य॒ज्ञ्स्य॑ । प॒था । सु॒वि॒ता । नय॑न्तीः । ब्रा॒ह्म॒णम् । अ॒द्य । रा॒द्ध्या॒स॒म् । ऋषि᳚म् । आ॒र्.॒षे॒यम् । पि॒तृ॒मन्त॒मिति॑ पितृ - मन्त᳚म् । पै॒तृ॒म॒त्यमिति॑ पैतृ - म॒त्यम् । सु॒धातु॑दक्षिण॒मिति॑ सु॒धातु॑ - द॒क्षि॒ण॒म् । वीति॑ । सुवः॑ । पश्य॑ । वीति॑ । अ॒न्तरि॑क्षम् । यत॑स्व । स॒द॒स्यैः᳚ ( ) । अ॒स्मद्दा᳚त्रा॒ इत्य॒स्मत् - दा॒त्राः॒ । दे॒व॒त्रेति॑ देव - त्रा । ग॒च्छ॒त॒ । मधु॑मती॒रिति॒ मधु॑ - म॒तीः॒ । प्र॒दा॒तार॒मिति॑ प्र - दा॒तार᳚म् । एति॑ । वि॒श॒त॒ । अन॑वहा॒येत्यन॑व - हा॒य॒ । अ॒स्मान् । दे॒व॒याने॒नेति॑ देव - याने॑न । प॒था । इ॒त॒ । सु॒कृता॒मिति॑ सु - कृता᳚म् । लो॒के । सी॒द॒त॒ । तत् । नः॒ । सꣳ॒॒स्कृ॒तम् ॥  \newline


\textbf{Krama Paata} \newline

वो॒ रू॒पम् । रू॒पम॒भि । अ॒भ्या । ऐमि॑ । ए॒मि॒ वय॑सा । वय॑सा॒ वयः॑ । वय॒ इति॒ वयः॑ ॥ तु॒थो वः॑ । वो॒ वि॒श्ववे॑दाः । वि॒श्ववे॑दा॒ वि । वि॒श्ववे॑दा॒ इति॑ वि॒श्व - वे॒दाः॒ । वि भ॑जतु । भ॒ज॒तु॒ वर्.षि॑ष्ठे । वर्.षि॑ष्ठे॒ अधि॑ । अधि॒ नाके᳚ । नाक॒ इति॒ नाके᳚ ॥ ए॒तत्ते᳚ । ते॒ अ॒ग्ने॒ । अ॒ग्ने॒ राधः॑ । राध॒ आ । ऐति॑ । ए॒ति॒ सोम॑च्युतम् । सोम॑च्युत॒म् तत् । सोम॑च्युत॒मिति॒ सोम॑ - च्यु॒त॒म् । तन्मि॒त्रस्य॑ । मि॒त्रस्य॑ प॒था । प॒था न॑य । न॒य॒र्तस्य॑ । ऋ॒तस्य॑ प॒था । प॒था प्र । प्रेत॑ । इ॒त॒ च॒न्द्रद॑क्षिणाः । च॒न्द्रद॑क्षिणा य॒ज्ञ्स्य॑ । च॒न्द्रद॑क्षिणा॒ इति॑ च॒न्द्र - द॒क्षि॒णाः॒ । य॒ज्ञ्स्य॑ प॒था । प॒था सु॑वि॒ता । सु॒वि॒ता नय॑न्तीः । नय॑न्तीर् ब्राह्म॒णम् । ब्रा॒ह्म॒णम॒द्य । अ॒द्य रा᳚द्ध्यासम् । रा॒द्ध्या॒स॒मृषि᳚म् । ऋषि॑मार्.षे॒यम् । आ॒र्॒.षे॒यम् पि॑तृ॒मन्त᳚म् । पि॒तृ॒मन्त॑म् पैतृम॒त्यम् । पि॒तृ॒मन्त॒मिति॑ पितृ - मन्त᳚म् । पै॒तृ॒म॒त्यꣳ सु॒धातु॑दक्षिणम् । पै॒तृ॒म॒त्यमिति॑ पैतृ - म॒त्यम् । सु॒धातु॑दक्षिणं॒ ॅवि । सु॒धातु॑दक्षिण॒मिति॑ सु॒धातु॑ - द॒क्षि॒ण॒म् । वि सुवः॑ । सुवः॒ पश्य॑ । पश्य॒ वि । व्य॑न्तरि॑क्षम् । अ॒न्तरि॑क्षं॒ ॅयत॑स्व । यत॑स्व सद॒स्यैः᳚ । स॒द॒स्यै॑र॒स्मद्दा᳚त्राः । अ॒स्मद्दा᳚त्रा देव॒त्रा । अ॒स्मद्दा᳚त्रा॒ इत्य॒स्मत् - दा॒त्राः॒ । दे॒व॒त्रा ग॑च्छत । दे॒व॒त्रेति॑ देव - त्रा । ग॒च्छ॒त॒ मधु॑मतीः । मधु॑मतीः प्रदा॒तार᳚म् । मधु॑मती॒रिति॒ मधु॑ - म॒तीः॒ । प्र॒दा॒तार॒मा । प्र॒दा॒तार॒मिति॑ प्र - दा॒तार᳚म् । आ वि॑शत । वि॒श॒तान॑वहाय । अन॑वहाया॒स्मान् । अन॑वहा॒येत्यन॑व - हा॒य॒ । अ॒स्मान् दे॑व॒याने॑न । दे॒व॒याने॑न प॒था । दे॒व॒याने॒नेति॑ देव - याने॑न । प॒थेत॑ । इ॒त॒ सु॒कृता᳚म् । सु॒कृतां᳚ ॅलो॒के । सु॒कृता॒मिति॑ सु - कृता᳚म् । लो॒के सी॑दत । सी॒द॒त॒ तत् । तन्नः॑ । नः॒ सꣳ॒॒स्कृ॒तम् । सꣳ॒॒स्कृ॒तमिति॑ सꣳस्कृ॒तम् । \newline

\textbf{Jatai Paata} \newline

1. वो॒ रू॒पꣳ रू॒पं ॅवो॑ वो रू॒पम् । \newline
2. रू॒प म॒भ्य॑भि रू॒पꣳ रू॒प म॒भि । \newline
3. अ॒भ्या ऽभ्य॑भ्या । \newline
4. ऐम्ये॒म्यैमि॑ । \newline
5. ए॒मि॒ वय॑सा॒ वय॑सैम्येमि॒ वय॑सा । \newline
6. वय॑सा॒ वयो॒ वयो॒ वय॑सा॒ वय॑सा॒ वयः॑ । \newline
7. वय॒ इति॒ वयः॑ । \newline
8. तु॒थो वो॑ व स्तु॒थ स्तु॒थो वः॑ । \newline
9. वो॒ वि॒श्ववे॑दा वि॒श्ववे॑दा वो वो वि॒श्ववे॑दाः । \newline
10. वि॒श्ववे॑दा॒ वि वि वि॒श्ववे॑दा वि॒श्ववे॑दा॒ वि । \newline
11. वि॒श्ववे॑दा॒ इति॑ वि॒श्व - वे॒दाः॒ । \newline
12. वि भ॑जतु भजतु॒ वि वि भ॑जतु । \newline
13. भ॒ज॒तु॒ वर्.षि॑ष्ठे॒ वर्.षि॑ष्ठे भजतु भजतु॒ वर्.षि॑ष्ठे । \newline
14. वर्.षि॑ष्ठे॒ अध्यधि॒ वर्.षि॑ष्ठे॒ वर्.षि॑ष्ठे॒ अधि॑ । \newline
15. अधि॒ नाके॒ नाके ऽध्यधि॒ नाके᳚ । \newline
16. नाक॒ इति॒ नाके᳚ । \newline
17. ए॒तत् ते॑ त ए॒तदे॒तत् ते᳚ । \newline
18. ते॒ अ॒ग्ने॒ ऽग्ने॒ ते॒ ते॒ अ॒ग्ने॒ । \newline
19. अ॒ग्ने॒ राधो॒ राधो᳚ ऽग्ने ऽग्ने॒ राधः॑ । \newline
20. राध॒ आ राधो॒ राध॒ आ । \newline
21. ऐत्ये॒त्यैति॑ । \newline
22. ए॒ति॒ सोम॑च्युत॒(ग्म्॒) सोम॑च्युत मेत्येति॒ सोम॑च्युतम् । \newline
23. सोम॑च्युत॒म् तत् तथ् सोम॑च्युत॒(ग्म्॒) सोम॑च्युत॒म् तत् । \newline
24. सोम॑च्युत॒मिति॒ सोम॑ - च्यु॒त॒म् । \newline
25. तन् मि॒त्रस्य॑ मि॒त्रस्य॒ तत् तन् मि॒त्रस्य॑ । \newline
26. मि॒त्रस्य॑ प॒था प॒था मि॒त्रस्य॑ मि॒त्रस्य॑ प॒था । \newline
27. प॒था न॑य नय प॒था प॒था न॑य । \newline
28. न॒य॒ र्तस्य॒ र्तस्य॑ नय नय॒ र्तस्य॑ । \newline
29. ऋ॒तस्य॑ प॒था प॒थर्तस्य॒ र्तस्य॑ प॒था । \newline
30. प॒था प्र प्र प॒था प॒था प्र । \newline
31. प्रे ते॑ त॒ प्र प्रे त॑ । \newline
32. इ॒त॒ च॒न्द्रद॑क्षिणा श्च॒न्द्रद॑क्षिणा इते त च॒न्द्रद॑क्षिणाः । \newline
33. च॒न्द्रद॑क्षिणा य॒ज्ञ्स्य॑ य॒ज्ञ्स्य॑ च॒न्द्रद॑क्षिणा श्च॒न्द्रद॑क्षिणा य॒ज्ञ्स्य॑ । \newline
34. च॒न्द्रद॑क्षिणा॒ इति॑ च॒न्द्र - द॒क्षि॒णाः॒ । \newline
35. य॒ज्ञ्स्य॑ प॒था प॒था य॒ज्ञ्स्य॑ य॒ज्ञ्स्य॑ प॒था । \newline
36. प॒था सु॑वि॒ता सु॑वि॒ता प॒था प॒था सु॑वि॒ता । \newline
37. सु॒वि॒ता नय॑न्ती॒र् नय॑न्तीः सुवि॒ता सु॑वि॒ता नय॑न्तीः । \newline
38. नय॑न्तीर् ब्राह्म॒णम् ब्रा᳚ह्म॒ण न्नय॑न्ती॒र् नय॑न्तीर् ब्राह्म॒णम् । \newline
39. ब्रा॒ह्म॒ण म॒द्याद्य ब्रा᳚ह्म॒णम् ब्रा᳚ह्म॒ण म॒द्य । \newline
40. अ॒द्य रा᳚द्ध्यासꣳ राद्ध्यास म॒द्याद्य रा᳚द्ध्यासम् । \newline
41. रा॒द्ध्या॒स॒ मृषि॒ मृषि(ग्म्॑) राद्ध्यासꣳ राद्ध्यास॒ मृषि᳚म् । \newline
42. ऋषि॑ मार्.षे॒य मा॑र्.षे॒य मृषि॒ मृषि॑ मार्.षे॒यम् । \newline
43. आ॒र्॒.षे॒यम् पि॑तृ॒मन्त॑म् पितृ॒मन्त॑ मार्.षे॒य मा॑र्.षे॒यम् पि॑तृ॒मन्त᳚म् । \newline
44. पि॒तृ॒मन्त॑म् पैतृम॒त्यम् पै॑तृम॒त्यम् पि॑तृ॒मन्त॑म् पितृ॒मन्त॑म् पैतृम॒त्यम् । \newline
45. पि॒तृ॒मन्त॒मिति॑ पितृ - मन्त᳚म् । \newline
46. पै॒तृ॒म॒त्यꣳ सु॒धातु॑दक्षिणꣳ सु॒धातु॑दक्षिणम् पैतृम॒त्यम् पै॑तृम॒त्यꣳ सु॒धातु॑दक्षिणम् । \newline
47. पै॒तृ॒म॒त्यमिति॑ पैतृ - म॒त्यम् । \newline
48. सु॒धातु॑दक्षिणं॒ ॅवि वि सु॒धातु॑दक्षिणꣳ सु॒धातु॑दक्षिणं॒ ॅवि । \newline
49. सु॒धातु॑दक्षिण॒मिति॑ सु॒धातु॑ - द॒क्षि॒ण॒म् । \newline
50. वि सुवः॒ सुव॒र् वि वि सुवः॑ । \newline
51. सुवः॒ पश्य॒ पश्य॒ सुवः॒ सुवः॒ पश्य॑ । \newline
52. पश्य॒ वि वि पश्य॒ पश्य॒ वि । \newline
53. व्य॑न्तरि॑क्ष म॒न्तरि॑क्षं॒ ॅवि व्य॑न्तरि॑क्षम् । \newline
54. अ॒न्तरि॑क्षं॒ ॅयत॑स्व॒ यत॑स्वा॒न्तरि॑क्ष म॒न्तरि॑क्षं॒ ॅयत॑स्व । \newline
55. यत॑स्व सद॒स्यैः᳚ सद॒स्यै᳚र् यत॑स्व॒ यत॑स्व सद॒स्यैः᳚ । \newline
56. स॒द॒स्यै॑ र॒स्मद्दा᳚त्रा अ॒स्मद्दा᳚त्राः सद॒स्यैः᳚ सद॒स्यै॑ र॒स्मद्दा᳚त्राः । \newline
57. अ॒स्मद्दा᳚त्रा देव॒त्रा दे॑व॒त्रा ऽस्मद्दा᳚त्रा अ॒स्मद्दा᳚त्रा देव॒त्रा । \newline
58. अ॒स्मद्दा᳚त्रा॒ इत्य॒स्मत् - दा॒त्राः॒ । \newline
59. दे॒व॒त्रा ग॑च्छत गच्छत देव॒त्रा दे॑व॒त्रा ग॑च्छत । \newline
60. दे॒व॒त्रेति॑ देव - त्रा । \newline
61. ग॒च्छ॒त॒ मधु॑मती॒र् मधु॑मतीर् गच्छत गच्छत॒ मधु॑मतीः । \newline
62. मधु॑मतीः प्रदा॒तार॑म् प्रदा॒तार॒म् मधु॑मती॒र् मधु॑मतीः प्रदा॒तार᳚म् । \newline
63. मधु॑मती॒रिति॒ मधु॑ - म॒तीः॒ । \newline
64. प्र॒दा॒तार॒ मा प्र॑दा॒तार॑म् प्रदा॒तार॒ मा । \newline
65. प्र॒दा॒तार॒मिति॑ प्र - दा॒तार᳚म् । \newline
66. आ वि॑शत विश॒ता वि॑शत । \newline
67. वि॒श॒तान॑वहा॒यान॑वहाय विशत विश॒तान॑वहाय । \newline
68. अन॑वहाया॒स्मा न॒स्मा नन॑वहा॒यान॑वहाया॒स्मान् । \newline
69. अन॑वहा॒येत्यन॑व - हा॒य॒ । \newline
70. अ॒स्मान् दे॑व॒याने॑न देव॒या ने॑ना॒स्मा न॒स्मान् दे॑व॒याने॑न । \newline
71. दे॒व॒याने॑न प॒था प॒था दे॑व॒याने॑न देव॒याने॑न प॒था । \newline
72. दे॒व॒याने॒नेति॑ देव - याने॑न । \newline
73. प॒थेते॑ त प॒था प॒थेत॑ । \newline
74. इ॒त॒ सु॒कृता(ग्म्॑) सु॒कृता॑ मिते त सु॒कृता᳚म् । \newline
75. सु॒कृता᳚म् ॅलो॒के लो॒के सु॒कृता(ग्म्॑) सु॒कृता᳚म् ॅलो॒के । \newline
76. सु॒कृता॒मिति॑ सु - कृता᳚म् । \newline
77. लो॒के सी॑दत सीदत लो॒के लो॒के सी॑दत । \newline
78. सी॒द॒त॒ तत् तथ् सी॑दत सीदत॒ तत् । \newline
79. तन् नो॑ न॒ स्तत् तन् नः॑ । \newline
80. नः॒ स॒(ग्ग्॒)स्कृ॒तꣳ स(ग्ग्॑)स्कृ॒तन्नो॑ नः सꣳस्कृ॒तम् । \newline
81. स॒(ग्ग्॒)स्कृ॒तमिति॑ सꣳस्कृ॒तम् । \newline

\textbf{Ghana Paata } \newline

1. वो॒ रू॒पꣳ रू॒पं ॅवो॑ वो रू॒प म॒भ्य॑भि रू॒पं ॅवो॑ वो रू॒प म॒भि । \newline
2. रू॒प म॒भ्य॑भि रू॒पꣳ रू॒प म॒भ्या ऽभि रू॒पꣳ रू॒प म॒भ्या । \newline
3. अ॒भ्या ऽभ्य॑ भ्यैम्ये॒म्या ऽभ्य॑ भ्यैमि॑ । \newline
4. ऐम्ये॒म्यैमि॒ वय॑सा॒ वय॑ सै॒म्यैमि॒ वय॑सा । \newline
5. ए॒मि॒ वय॑सा॒ वय॑ सैम्येमि॒ वय॑सा॒ वयो॒ वयो॒ वय॑ सैम्येमि॒ वय॑सा॒ वयः॑ । \newline
6. वय॑सा॒ वयो॒ वयो॒ वय॑सा॒ वय॑सा॒ वयः॑ । \newline
7. वय॒ इति॒ वयः॑ । \newline
8. तु॒थो वो॑ व स्तु॒थ स्तु॒थो वो॑ वि॒श्ववे॑दा वि॒श्ववे॑दा व स्तु॒थ स्तु॒थो वो॑ वि॒श्ववे॑दाः । \newline
9. वो॒ वि॒श्ववे॑दा वि॒श्ववे॑दा वो वो वि॒श्ववे॑दा॒ वि वि वि॒श्ववे॑दा वो वो वि॒श्ववे॑दा॒ वि । \newline
10. वि॒श्ववे॑दा॒ वि वि वि॒श्ववे॑दा वि॒श्ववे॑दा॒ वि भ॑जतु भजतु॒ वि वि॒श्ववे॑दा वि॒श्ववे॑दा॒ वि भ॑जतु । \newline
11. वि॒श्ववे॑दा॒ इति॑ वि॒श्व - वे॒दाः॒ । \newline
12. वि भ॑जतु भजतु॒ वि वि भ॑जतु॒ वर्.षि॑ष्ठे॒ वर्.षि॑ष्ठे भजतु॒ वि वि भ॑जतु॒ वर्.षि॑ष्ठे । \newline
13. भ॒ज॒तु॒ वर्.षि॑ष्ठे॒ वर्.षि॑ष्ठे भजतु भजतु॒ वर्.षि॑ष्ठे॒ अध्यधि॒ वर्.षि॑ष्ठे भजतु भजतु॒ वर्.षि॑ष्ठे॒ अधि॑ । \newline
14. वर्.षि॑ष्ठे॒ अध्यधि॒ वर्.षि॑ष्ठे॒ वर्.षि॑ष्ठे॒ अधि॒ नाके॒ नाके ऽधि॒ वर्.षि॑ष्ठे॒ वर्.षि॑ष्ठे॒ अधि॒ नाके᳚ । \newline
15. अधि॒ नाके॒ नाके ऽध्यधि॒ नाके᳚ । \newline
16. नाक॒ इति॒ नाके᳚ । \newline
17. ए॒तत् ते॑ त ए॒त दे॒तत् ते॑ अग्ने ऽग्ने त ए॒त दे॒तत् ते॑ अग्ने । \newline
18. ते॒ अ॒ग्ने॒ ऽग्ने॒ ते॒ ते॒ अ॒ग्ने॒ राधो॒ राधो᳚ ऽग्ने ते ते अग्ने॒ राधः॑ । \newline
19. अ॒ग्ने॒ राधो॒ राधो᳚ ऽग्ने ऽग्ने॒ राध॒ आ राधो᳚ ऽग्ने ऽग्ने॒ राध॒ आ । \newline
20. राध॒ आ राधो॒ राध॒ ऐत्ये॒त्या राधो॒ राध॒ ऐति॑ । \newline
21. ऐत्ये॒त्यैति॒ सोम॑च्युत॒(ग्म्॒) सोम॑च्युत मे॒त्यैति॒ सोम॑च्युतम् । \newline
22. ए॒ति॒ सोम॑च्युत॒(ग्म्॒) सोम॑च्युत मेत्येति॒ सोम॑च्युत॒म् तत् तथ् सोम॑च्युत मेत्येति॒ सोम॑च्युत॒म् तत् । \newline
23. सोम॑च्युत॒म् तत् तथ् सोम॑च्युत॒(ग्म्॒) सोम॑च्युत॒म् तन् मि॒त्रस्य॑ मि॒त्रस्य॒ तथ् सोम॑च्युत॒(ग्म्॒) सोम॑च्युत॒म् तन् मि॒त्रस्य॑ । \newline
24. सोम॑च्युत॒मिति॒ सोम॑ - च्यु॒त॒म् । \newline
25. तन् मि॒त्रस्य॑ मि॒त्रस्य॒ तत् तन् मि॒त्रस्य॑ प॒था प॒था मि॒त्रस्य॒ तत् तन् मि॒त्रस्य॑ प॒था । \newline
26. मि॒त्रस्य॑ प॒था प॒था मि॒त्रस्य॑ मि॒त्रस्य॑ प॒था न॑य नय प॒था मि॒त्रस्य॑ मि॒त्रस्य॑ प॒था न॑य । \newline
27. प॒था न॑य नय प॒था प॒था न॑य॒ र्तस्य॒ र्तस्य॑ नय प॒था प॒था न॑य॒ र्तस्य॑ । \newline
28. न॒य॒ र्तस्य॒ र्तस्य॑ नय नय॒ र्तस्य॑ प॒था प॒थर्तस्य॑ नय नय॒ र्तस्य॑ प॒था । \newline
29. ऋ॒तस्य॑ प॒था प॒थर्तस्य॒ र्तस्य॑ प॒था प्र प्र प॒थर्तस्य॒ र्तस्य॑ प॒था प्र । \newline
30. प॒था प्र प्र प॒था प॒था प्रे ते॑ त॒ प्र प॒था प॒था प्रे त॑ । \newline
31. प्रे ते॑ त॒ प्र प्रे त॑ च॒न्द्रद॑क्षिणा श्च॒न्द्रद॑क्षिणा इत॒ प्र प्रे त॑ च॒न्द्रद॑क्षिणाः । \newline
32. इ॒त॒ च॒न्द्रद॑क्षिणा श्च॒न्द्रद॑क्षिणा इते त च॒न्द्रद॑क्षिणा य॒ज्ञ्स्य॑ य॒ज्ञ्स्य॑ च॒न्द्रद॑क्षिणा इते त च॒न्द्रद॑क्षिणा य॒ज्ञ्स्य॑ । \newline
33. च॒न्द्रद॑क्षिणा य॒ज्ञ्स्य॑ य॒ज्ञ्स्य॑ च॒न्द्रद॑क्षिणा श्च॒न्द्रद॑क्षिणा य॒ज्ञ्स्य॑ प॒था प॒था य॒ज्ञ्स्य॑ च॒न्द्रद॑क्षिणा श्च॒न्द्रद॑क्षिणा य॒ज्ञ्स्य॑ प॒था । \newline
34. च॒न्द्रद॑क्षिणा॒ इति॑ च॒न्द्र - द॒क्षि॒णाः॒ । \newline
35. य॒ज्ञ्स्य॑ प॒था प॒था य॒ज्ञ्स्य॑ य॒ज्ञ्स्य॑ प॒था सु॑वि॒ता सु॑वि॒ता प॒था य॒ज्ञ्स्य॑ य॒ज्ञ्स्य॑ प॒था सु॑वि॒ता । \newline
36. प॒था सु॑वि॒ता सु॑वि॒ता प॒था प॒था सु॑वि॒ता नय॑न्ती॒र् नय॑न्तीः सुवि॒ता प॒था प॒था सु॑वि॒ता नय॑न्तीः । \newline
37. सु॒वि॒ता नय॑न्ती॒र् नय॑न्तीः सुवि॒ता सु॑वि॒ता नय॑न्तीर् ब्राह्म॒णम् ब्रा᳚ह्म॒णम् नय॑न्तीः सुवि॒ता सु॑वि॒ता 
नय॑न्तीर् ब्राह्म॒णम् । \newline
38. नय॑न्तीर् ब्राह्म॒णम् ब्रा᳚ह्म॒णम् नय॑न्ती॒र् नय॑न्तीर् ब्राह्म॒ण म॒द्याद्य ब्रा᳚ह्म॒णम् नय॑न्ती॒र् नय॑न्तीर् ब्राह्म॒ण म॒द्य । \newline
39. ब्रा॒ह्म॒ण म॒द्याद्य ब्रा᳚ह्म॒णम् ब्रा᳚ह्म॒ण म॒द्य रा᳚द्ध्यासꣳ राद्ध्यास म॒द्य ब्रा᳚ह्म॒णम् ब्रा᳚ह्म॒ण म॒द्य रा᳚द्ध्यासम् । \newline
40. अ॒द्य रा᳚द्ध्यासꣳ राद्ध्यास म॒द्याद्य रा᳚द्ध्यास॒ मृषि॒ मृषि(ग्म्॑) राद्ध्यास म॒द्याद्य रा᳚द्ध्यास॒ मृषि᳚म् । \newline
41. रा॒द्ध्या॒स॒ मृषि॒ मृषि(ग्म्॑) राद्ध्यासꣳ राद्ध्यास॒ मृषि॑ मार्.षे॒य मा॑र्.षे॒य मृषि(ग्म्॑) राद्ध्यासꣳ राद्ध्यास॒ मृषि॑ मार्.षे॒यम् । \newline
42. ऋषि॑ मार्.षे॒य मा॑र्.षे॒य मृषि॒ मृषि॑ मार्.षे॒यम् पि॑तृ॒मन्त॑म् पितृ॒मन्त॑ मार्.षे॒य मृषि॒ मृषि॑ मार्.षे॒यम् पि॑तृ॒मन्त᳚म् । \newline
43. आ॒र्॒.षे॒यम् पि॑तृ॒मन्त॑म् पितृ॒मन्त॑ मार्.षे॒य मा॑र्.षे॒यम् पि॑तृ॒मन्त॑म् पैतृम॒त्यम् पै॑तृम॒त्यम् पि॑तृ॒मन्त॑ मार्.षे॒य मा॑र्.षे॒यम् पि॑तृ॒मन्त॑म् पैतृम॒त्यम् । \newline
44. पि॒तृ॒मन्त॑म् पैतृम॒त्यम् पै॑तृम॒त्यम् पि॑तृ॒मन्त॑म् पितृ॒मन्त॑म् पैतृम॒त्यꣳ सु॒धातु॑दक्षिणꣳ सु॒धातु॑दक्षिणम् पैतृम॒त्यम् पि॑तृ॒मन्त॑म् पितृ॒मन्त॑म् पैतृम॒त्यꣳ सु॒धातु॑दक्षिणम् । \newline
45. पि॒तृ॒मन्त॒मिति॑ पितृ - मन्त᳚म् । \newline
46. पै॒तृ॒म॒त्यꣳ सु॒धातु॑दक्षिणꣳ सु॒धातु॑दक्षिणम् पैतृम॒त्यम् पै॑तृम॒त्यꣳ सु॒धातु॑दक्षिणं॒ ॅवि वि सु॒धातु॑दक्षिणम् पैतृम॒त्यम् पै॑तृम॒त्यꣳ सु॒धातु॑दक्षिणं॒ ॅवि । \newline
47. पै॒तृ॒म॒त्यमिति॑ पैतृ - म॒त्यम् । \newline
48. सु॒धातु॑दक्षिणं॒ ॅवि वि सु॒धातु॑दक्षिणꣳ सु॒धातु॑दक्षिणं॒ ॅवि सुवः॒ सुव॒र् वि सु॒धातु॑दक्षिणꣳ सु॒धातु॑दक्षिणं॒ ॅवि सुवः॑ । \newline
49. सु॒धातु॑दक्षिण॒मिति॑ सु॒धातु॑ - द॒क्षि॒ण॒म् । \newline
50. वि सुवः॒ सुव॒र् वि वि सुवः॒ पश्य॒ पश्य॒ सुव॒र् वि वि सुवः॒ पश्य॑ । \newline
51. सुवः॒ पश्य॒ पश्य॒ सुवः॒ सुवः॒ पश्य॒ वि वि पश्य॒ सुवः॒ सुवः॒ पश्य॒ वि । \newline
52. पश्य॒ वि वि पश्य॒ पश्य॒ व्य॑न्तरि॑क्ष म॒न्तरि॑क्षं॒ ॅवि पश्य॒ पश्य॒ व्य॑न्तरि॑क्षम् । \newline
53. व्य॑न्तरि॑क्ष म॒न्तरि॑क्षं॒ ॅवि व्य॑न्तरि॑क्षं॒ ॅयत॑स्व॒ यत॑स्वा॒न्तरि॑क्षं॒ ॅवि व्य॑न्तरि॑क्षं॒ ॅयत॑स्व । \newline
54. अ॒न्तरि॑क्षं॒ ॅयत॑स्व॒ यत॑स्वा॒न्तरि॑क्ष म॒न्तरि॑क्षं॒ ॅयत॑स्व सद॒स्यैः᳚ सद॒स्यै᳚र् यत॑स्वा॒न्तरि॑क्ष म॒न्तरि॑क्षं॒ ॅयत॑स्व सद॒स्यैः᳚ । \newline
55. यत॑स्व सद॒स्यैः᳚ सद॒स्यै᳚र् यत॑स्व॒ यत॑स्व सद॒स्यै॑ र॒स्मद्दा᳚त्रा अ॒स्मद्दा᳚त्राः सद॒स्यै᳚र् यत॑स्व॒ यत॑स्व सद॒स्यै॑ र॒स्मद्दा᳚त्राः । \newline
56. स॒द॒स्यै॑ र॒स्मद्दा᳚त्रा अ॒स्मद्दा᳚त्राः सद॒स्यैः᳚ सद॒स्यै॑ र॒स्मद्दा᳚त्रा देव॒त्रा दे॑व॒त्रा ऽस्मद्दा᳚त्राः सद॒स्यैः᳚ सद॒स्यै॑ र॒स्मद्दा᳚त्रा देव॒त्रा । \newline
57. अ॒स्मद्दा᳚त्रा देव॒त्रा दे॑व॒त्रा ऽस्मद्दा᳚त्रा अ॒स्मद्दा᳚त्रा देव॒त्रा ग॑च्छत गच्छत देव॒त्रा ऽस्मद्दा᳚त्रा अ॒स्मद्दा᳚त्रा देव॒त्रा ग॑च्छत । \newline
58. अ॒स्मद्दा᳚त्रा॒ इत्य॒स्मत् - दा॒त्राः॒ । \newline
59. दे॒व॒त्रा ग॑च्छत गच्छत देव॒त्रा दे॑व॒त्रा ग॑च्छत॒ मधु॑मती॒र् मधु॑मतीर् गच्छत देव॒त्रा दे॑व॒त्रा ग॑च्छत॒ मधु॑मतीः । \newline
60. दे॒व॒त्रेति॑ देव - त्रा । \newline
61. ग॒च्छ॒त॒ मधु॑मती॒र् मधु॑मतीर् गच्छत गच्छत॒ मधु॑मतीः प्रदा॒तार॑म् प्रदा॒तार॒म् मधु॑मतीर् गच्छत गच्छत॒ मधु॑मतीः प्रदा॒तार᳚म् । \newline
62. मधु॑मतीः प्रदा॒तार॑म् प्रदा॒तार॒म् मधु॑मती॒र् मधु॑मतीः प्रदा॒तार॒ मा प्र॑दा॒तार॒म् मधु॑मती॒र् मधु॑मतीः प्रदा॒तार॒ मा । \newline
63. मधु॑मती॒रिति॒ मधु॑ - म॒तीः॒ । \newline
64. प्र॒दा॒तार॒ मा प्र॑दा॒तार॑म् प्रदा॒तार॒ मा वि॑शत विश॒ता प्र॑दा॒तार॑म् प्रदा॒तार॒ मा वि॑शत । \newline
65. प्र॒दा॒तार॒मिति॑ प्र - दा॒तार᳚म् । \newline
66. आ वि॑शत विश॒ता वि॑श॒ता न॑वहा॒या न॑वहाय विश॒ता वि॑श॒ता न॑वहाय । \newline
67. वि॒श॒ता न॑वहा॒या न॑वहाय विशत विश॒ता न॑वहाया॒स्मा न॒स्मा नन॑वहाय विशत विश॒ता न॑वहाया॒स्मान् । \newline
68. अन॑वहाया॒स्मा न॒स्मा नन॑वहा॒या न॑वहाया॒स्मान् दे॑व॒याने॑न देव॒याने॑ना॒स्मा नन॑वहा॒या न॑वहाया॒स्मान् दे॑व॒याने॑न । \newline
69. अन॑वहा॒येत्यन॑व - हा॒य॒ । \newline
70. अ॒स्मान् दे॑व॒याने॑न देव॒याने॑ना॒स्मा न॒स्मान् दे॑व॒याने॑न प॒था प॒था दे॑व॒याने॑ना॒स्मा न॒स्मान् दे॑व॒याने॑न प॒था । \newline
71. दे॒व॒याने॑न प॒था प॒था दे॑व॒याने॑न देव॒याने॑न प॒थेते॑ त प॒था दे॑व॒याने॑न देव॒याने॑न प॒थेत॑ । \newline
72. दे॒व॒याने॒नेति॑ देव - याने॑न । \newline
73. प॒थेते॑ त प॒था प॒थेत॑ सु॒कृता(ग्म्॑) सु॒कृता॑ मित प॒था प॒थेत॑ सु॒कृता᳚म् । \newline
74. इ॒त॒ सु॒कृता(ग्म्॑) सु॒कृता॑ मिते त सु॒कृता᳚म् ॅलो॒के लो॒के सु॒कृता॑ मिते त सु॒कृता᳚म् ॅलो॒के । \newline
75. सु॒कृता᳚म् ॅलो॒के लो॒के सु॒कृता(ग्म्॑) सु॒कृता᳚म् ॅलो॒के सी॑दत सीदत लो॒के सु॒कृता(ग्म्॑) सु॒कृता᳚म् ॅलो॒के सी॑दत । \newline
76. सु॒कृता॒मिति॑ सु - कृता᳚म् । \newline
77. लो॒के सी॑दत सीदत लो॒के लो॒के सी॑दत॒ तत् तथ् सी॑दत लो॒के लो॒के सी॑दत॒ तत् । \newline
78. सी॒द॒त॒ तत् तथ् सी॑दत सीदत॒ तन् नो॑ न॒ स्तथ् सी॑दत सीदत॒ तन् नः॑ । \newline
79. तन् नो॑ न॒ स्तत् तन् नः॑ सꣳस्कृ॒तꣳ स(ग्ग्॑)स्कृ॒तम् न॒ स्तत् तन् नः॑ सꣳस्कृ॒तम् । \newline
80. नः॒ स॒(ग्ग्॒)स्कृ॒तꣳ स(ग्ग्॑)स्कृ॒तम् नो॑ नः सꣳस्कृ॒तम् । \newline
81. स॒(ग्ग्॒)स्कृ॒तमिति॑ सꣳस्कृ॒तम् । \newline
\pagebreak
\markright{ TS 1.4.44.1  \hfill https://www.vedavms.in \hfill}
\addcontentsline{toc}{section}{ TS 1.4.44.1 }
\section*{ TS 1.4.44.1 }

\textbf{TS 1.4.44.1 } \newline
\textbf{Samhita Paata} \newline

धा॒ता रा॒तिः स॑वि॒तेदं जु॑षन्तां प्र॒जाप॑तिर् निधि॒पति॑र्नो अ॒ग्निः । त्वष्टा॒ विष्णुः॑ प्र॒जया॑ सꣳररा॒णो यज॑मानाय॒ द्रवि॑णं दधातु ॥ समि॑न्द्र णो॒ मन॑सा नेषि॒ गोभिः॒ सꣳ सू॒रिभि॑र्मघव॒न्थ् सꣳ स्व॒स्त्या । सं ब्रह्म॑णा दे॒वकृ॑तं॒ ॅयदस्ति॒ सं दे॒वानाꣳ॑ सुम॒त्या य॒ज्ञिया॑नां ॥ सं ॅवर्च॑सा॒ पय॑सा॒ सं त॒नूभि॒-रग॑न्महि॒ मन॑सा॒ सꣳ शि॒वेन॑ ॥ त्वष्टा॑ नो॒ अत्र॒ वरि॑वः कृणो॒ - [ ] \newline

\textbf{Pada Paata} \newline

ध॒ता । रा॒तिः । स॒वि॒ता । इ॒दम् । जु॒ष॒न्ता॒म् । प्र॒जाप॑ति॒रिति॑ प्र॒जा - प॒तिः॒ । नि॒धि॒पति॒रिति॑ निधि - पतिः॑ । नः॒ । अ॒ग्निः ॥ त्वष्टा᳚ । विष्णुः॑ । प्र॒जयेति॑ प्र - जया᳚ । सꣳ॒॒र॒रा॒ण इति॑ सं-र॒रा॒णः । यज॑मानाय । द्रवि॑णम् । द॒धा॒तु॒ ॥ समिति॑ । इ॒न्द्र॒ । नः॒ । मन॑सा । ने॒षि॒ । गोभिः॑ । समिति॑ । सू॒रिभि॒रिति॑ सू॒रि - भिः॒ । म॒घ॒व॒न्निति॑ मघ - व॒न्न् । समिति॑ । स्व॒स्त्या ॥ समिति॑ । ब्रह्म॑णा । दे॒वकृ॑त॒मिति॑ दे॒व - कृ॒त॒म् । यत् । अस्ति॑ । समिति॑ । दे॒वाना᳚म् । सु॒म॒त्येति॑ सु - म॒त्या । य॒ज्ञिया॑नाम् ॥ समिति॑ । वर्च॑सा । पय॑सा । समिति॑ । त॒नूभिः॑ । अग॑न्महि । मन॑सा । समिति॑ । शि॒वेन॑ ॥ त्वष्टा᳚ । नः॒ । अत्र॑ । वरि॑वः । कृ॒णो॒तु॒ ।  \newline


\textbf{Krama Paata} \newline

धा॒ता रा॒तिः । रा॒तिः स॑वि॒ता । स॒वि॒ते॒दम् । इ॒दम् जु॑षन्ताम् । जु॒ष॒न्ता॒म् प्र॒जाप॑तिः । प्र॒जाप॑तिर्,निधि॒पतिः॑ । प्र॒जाप॑ति॒रिति॑ प्र॒जा - प॒तिः॒ । नि॒धि॒पति॑र् नः । नि॒धि॒पति॒रिति॑ निधि - पतिः॑ । नो॒ अ॒ग्निः । अ॒ग्निरित्य॒ग्निः ॥ त्वष्टा॒ विष्णुः॑ । विष्णुः॑ प्र॒जया᳚ । प्र॒जया॑ सꣳररा॒णः । प्र॒जयेति॑ प्र - जया᳚ । सꣳ॒॒र॒रा॒णो यज॑मानाय । सꣳ॒॒र॒रा॒ण इति॑ सं - र॒रा॒णः । यज॑मानाय॒ द्रवि॑णम् । द्रवि॑णं दधातु । द॒धा॒त्विति॑ दधातु ॥ समि॑न्द्र । इ॒न्द्र॒णः॒ । नो॒ मन॑सा । मन॑सा नेषि । ने॒षि॒ गोभिः॑ । गोभिः॒ सम् । सꣳ सू॒रिभिः॑ । सू॒रिभि॑र् मघवन्न् । सू॒रिभि॒रिति॑ सू॒रि - भिः॒ । म॒घ॒व॒न्थ् सम् । म॒घ॒व॒न्निति॑ मघ - व॒न्न्॒ । सꣳ स्व॒स्त्या । स्व॒स्त्येति॑ स्व॒स्त्या ॥ सम् ब्रह्म॑णा । ब्रह्म॑णा दे॒वकृ॑तम् । दे॒वकृ॑तं॒ ॅयत् । दे॒वकृ॑त॒मिति॑ दे॒व - कृ॒त॒म् । यदस्ति॑ । अस्ति॒ सम् । सम् दे॒वाना᳚म् । दे॒वानाꣳ॑ सुम॒त्या । सु॒म॒त्या य॒ज्ञिया॑नाम् । सु॒म॒त्येति॑ सु - म॒त्या । य॒ज्ञिया॑ना॒मिति॑ य॒ज्ञिया॑नाम् ॥ सं ॅवर्च॑सा । वर्च॑सा॒ पय॑सा । पय॑सा॒ सम् । सम् त॒नूभिः॑ । त॒नूभि॒रग॑न्महि । अग॑न्महि॒ मन॑सा । मन॑सा॒ सम् । सꣳ शि॒वेन॑ । शि॒वेनेति॑ शि॒वेन॑ ॥ त्वष्टा॑ नः । नो॒ अत्र॑ । अत्र॒ वरि॑वः । वरि॑वः कृणोतु । कृ॒णो॒त्वनु॑ \newline

\textbf{Jatai Paata} \newline

1. धा॒ता रा॒ती रा॒तिर् धा॒ता धा॒ता रा॒तिः । \newline
2. रा॒तिः स॑वि॒ता स॑वि॒ता रा॒ती रा॒तिः स॑वि॒ता । \newline
3. स॒वि॒तेद मि॒दꣳ स॑वि॒ता स॑वि॒तेदम् । \newline
4. इ॒दम् जु॑षन्ताम् जुषन्ता मि॒द मि॒दम् जु॑षन्ताम् । \newline
5. जु॒ष॒न्ता॒म् प्र॒जाप॑तिः प्र॒जाप॑तिर् जुषन्ताम् जुषन्ताम् प्र॒जाप॑तिः । \newline
6. प्र॒जाप॑तिर् निधि॒पति॑र् निधि॒पतिः॑ प्र॒जाप॑तिः प्र॒जाप॑तिर् निधि॒पतिः॑ । \newline
7. प्र॒जाप॑ति॒रिति॑ प्र॒जा - प॒तिः॒ । \newline
8. नि॒धि॒पति॑र् नो नो निधि॒पति॑र् निधि॒पति॑र् नः । \newline
9. नि॒धि॒पति॒रिति॑ निधि - पतिः॑ । \newline
10. नो॒ अ॒ग्नि र॒ग्निर् नो॑ नो अ॒ग्निः । \newline
11. अ॒ग्निरित्य॒ग्निः । \newline
12. त्वष्टा॒ विष्णु॒र् विष्णु॒ स्त्वष्टा॒ त्वष्टा॒ विष्णुः॑ । \newline
13. विष्णुः॑ प्र॒जया᳚ प्र॒जया॒ विष्णु॒र् विष्णुः॑ प्र॒जया᳚ । \newline
14. प्र॒जया॑ सꣳररा॒णः स(ग्म्॑)ररा॒णः प्र॒जया᳚ प्र॒जया॑ सꣳररा॒णः । \newline
15. प्र॒जयेति॑ प्र - जया᳚ । \newline
16. स॒(ग्म्॒)र॒रा॒णो यज॑मानाय॒ यज॑मानाय सꣳररा॒णः स(ग्म्॑)ररा॒णो यज॑मानाय । \newline
17. स॒(ग्म्॒)र॒रा॒ण इति॑ सं - र॒रा॒णः । \newline
18. यज॑मानाय॒ द्रवि॑ण॒म् द्रवि॑णं॒ ॅयज॑मानाय॒ यज॑मानाय॒ द्रवि॑णम् । \newline
19. द्रवि॑णम् दधातु दधातु॒ द्रवि॑ण॒म् द्रवि॑णम् दधातु । \newline
20. द॒धा॒त्विति॑ दधातु । \newline
21. समि॑न्द्रेन्द्र॒ सꣳ समि॑न्द्र । \newline
22. इ॒न्द्र॒ णो॒ न॒ इ॒न्द्रे॒न्द्र॒ णः॒ । \newline
23. नो॒ मन॑सा॒ मन॑सा नो नो॒ मन॑सा । \newline
24. मन॑सा नेषि नेषि॒ मन॑सा॒ मन॑सा नेषि । \newline
25. ने॒षि॒ गोभि॒र् गोभि॑र् नेषि नेषि॒ गोभिः॑ । \newline
26. गोभिः॒ सꣳ सम् गोभि॒र् गोभिः॒ सम् । \newline
27. सꣳ सू॒रिभिः॑ सू॒रिभिः॒ सꣳ सꣳ सू॒रिभिः॑ । \newline
28. सू॒रिभि॑र् मघवन् मघवन् थ्सू॒रिभिः॑ सू॒रिभि॑र् मघवन्न् । \newline
29. सू॒रिभि॒रिति॑ सू॒रि - भिः॒ । \newline
30. म॒घ॒व॒न् थ्सꣳ सम् म॑घवन् मघव॒न् थ्सम् । \newline
31. म॒घ॒व॒न्निति॑ मघ - व॒न्न् । \newline
32. सꣳ स्व॒स्त्या स्व॒स्त्या सꣳ सꣳ स्व॒स्त्या । \newline
33. स्व॒स्त्येति॑ स्व॒स्त्या । \newline
34. सम् ब्रह्म॑णा॒ ब्रह्म॑णा॒ सꣳ सम् ब्रह्म॑णा । \newline
35. ब्रह्म॑णा दे॒वकृ॑तम् दे॒वकृ॑त॒म् ब्रह्म॑णा॒ ब्रह्म॑णा दे॒वकृ॑तम् । \newline
36. दे॒वकृ॑तं॒ ॅयद् यद् दे॒वकृ॑तम् दे॒वकृ॑तं॒ ॅयत् । \newline
37. दे॒वकृ॑त॒मिति॑ दे॒व - कृ॒त॒म् । \newline
38. यदस्त्यस्ति॒ यद् यदस्ति॑ । \newline
39. अस्ति॒ सꣳ स मस्त्यस्ति॒ सम् । \newline
40. सम् दे॒वाना᳚म् दे॒वाना॒(ग्म्॒) सꣳ सम् दे॒वाना᳚म् । \newline
41. दे॒वाना(ग्म्॑) सुम॒त्या सु॑म॒त्या दे॒वाना᳚म् दे॒वाना(ग्म्॑) सुम॒त्या । \newline
42. सु॒म॒त्या य॒ज्ञिया॑नां ॅय॒ज्ञिया॑नाꣳ सुम॒त्या सु॑म॒त्या य॒ज्ञिया॑नाम् । \newline
43. सु॒म॒त्येति॑ सु - म॒त्या । \newline
44. य॒ज्ञिया॑ना॒मिति॑ य॒ज्ञिया॑नाम् । \newline
45. सं ॅवर्च॑सा॒ वर्च॑सा॒ सꣳ सं ॅवर्च॑सा । \newline
46. वर्च॑सा॒ पय॑सा॒ पय॑सा॒ वर्च॑सा॒ वर्च॑सा॒ पय॑सा । \newline
47. पय॑सा॒ सꣳ सम् पय॑सा॒ पय॑सा॒ सम् । \newline
48. सम् त॒नूभि॑ स्त॒नूभिः॒ सꣳ सम् त॒नूभिः॑ । \newline
49. त॒नूभि॒ रग॑न्म॒ह्यग॑न्महि त॒नूभि॑ स्त॒नूभि॒ रग॑न्महि । \newline
50. अग॑न्महि॒ मन॑सा॒ मन॒सा ऽग॑न्म॒ह्यग॑न्महि॒ मन॑सा । \newline
51. मन॑सा॒ सꣳ सम् मन॑सा॒ मन॑सा॒ सम् । \newline
52. सꣳशि॒वेन॑ शि॒वेन॒ सꣳ सꣳ शि॒वेन॑ । \newline
53. शि॒वेनेति॑ शि॒वेन॑ । \newline
54. त्वष्टा॑ नो न॒ स्त्वष्टा॒ त्वष्टा॑ नः । \newline
55. नो॒ अत्रात्र॑ नो नो॒ अत्र॑ । \newline
56. अत्र॒ वरि॑वो॒ वरि॑वो॒ अत्रात्र॒ वरि॑वः । \newline
57. वरि॑वः कृणोतु कृणोतु॒ वरि॑वो॒ वरि॑वः कृणोतु । \newline
58. कृ॒णो॒त्वन्वनु॑ कृणोतु कृणो॒त्वनु॑ । \newline

\textbf{Ghana Paata } \newline

1. धा॒ता रा॒ती रा॒तिर् धा॒ता धा॒ता रा॒तिः स॑वि॒ता स॑वि॒ता रा॒तिर् धा॒ता धा॒ता रा॒तिः स॑वि॒ता । \newline
2. रा॒तिः स॑वि॒ता स॑वि॒ता रा॒ती रा॒तिः स॑वि॒तेद मि॒दꣳ स॑वि॒ता रा॒ती रा॒तिः स॑वि॒तेदम् । \newline
3. स॒वि॒तेद मि॒दꣳ स॑वि॒ता स॑वि॒तेदम् जु॑षन्ताम् जुषन्ता मि॒दꣳ स॑वि॒ता स॑वि॒तेदम् जु॑षन्ताम् । \newline
4. इ॒दम् जु॑षन्ताम् जुषन्ता मि॒द मि॒दम् जु॑षन्ताम् प्र॒जाप॑तिः प्र॒जाप॑तिर् जुषन्ता मि॒द मि॒दम् जु॑षन्ताम् प्र॒जाप॑तिः । \newline
5. जु॒ष॒न्ता॒म् प्र॒जाप॑तिः प्र॒जाप॑तिर् जुषन्ताम् जुषन्ताम् प्र॒जाप॑तिर् निधि॒पति॑र् निधि॒पतिः॑ प्र॒जाप॑तिर् जुषन्ताम् जुषन्ताम् प्र॒जाप॑तिर् निधि॒पतिः॑ । \newline
6. प्र॒जाप॑तिर् निधि॒पति॑र् निधि॒पतिः॑ प्र॒जाप॑तिः प्र॒जाप॑तिर् निधि॒पति॑र् नो नो निधि॒पतिः॑ प्र॒जाप॑तिः प्र॒जाप॑तिर् निधि॒पति॑र् नः । \newline
7. प्र॒जाप॑ति॒रिति॑ प्र॒जा - प॒तिः॒ । \newline
8. नि॒धि॒पति॑र् नो नो निधि॒पति॑र् निधि॒पति॑र् नो अ॒ग्नि र॒ग्निर् नो॑ निधि॒पति॑र् निधि॒पति॑र् नो अ॒ग्निः । \newline
9. नि॒धि॒पति॒रिति॑ निधि - पतिः॑ । \newline
10. नो॒ अ॒ग्नि र॒ग्निर् नो॑ नो अ॒ग्निः । \newline
11. अ॒ग्निरित्य॒ग्निः । \newline
12. त्वष्टा॒ विष्णु॒र् विष्णु॒ स्त्वष्टा॒ त्वष्टा॒ विष्णुः॑ प्र॒जया᳚ प्र॒जया॒ विष्णु॒ स्त्वष्टा॒ त्वष्टा॒ विष्णुः॑ प्र॒जया᳚ । \newline
13. विष्णुः॑ प्र॒जया᳚ प्र॒जया॒ विष्णु॒र् विष्णुः॑ प्र॒जया॑ सꣳररा॒णः स(ग्म्॑)ररा॒णः प्र॒जया॒ विष्णु॒र् विष्णुः॑ प्र॒जया॑ सꣳररा॒णः । \newline
14. प्र॒जया॑ सꣳररा॒णः स(ग्म्॑)ररा॒णः प्र॒जया᳚ प्र॒जया॑ सꣳररा॒णो यज॑मानाय॒ यज॑मानाय सꣳररा॒णः प्र॒जया᳚ प्र॒जया॑ सꣳररा॒णो यज॑मानाय । \newline
15. प्र॒जयेति॑ प्र - जया᳚ । \newline
16. सꣳ॒॒र॒रा॒णो यज॑मानाय॒ यज॑मानाय सꣳररा॒णः स(ग्म्॑)ररा॒णो यज॑मानाय॒ द्रवि॑ण॒म् द्रवि॑णं॒ ॅयज॑मानाय सꣳररा॒णः स(ग्म्॑)ररा॒णो यज॑मानाय॒ द्रवि॑णम् । \newline
17. सꣳ॒॒र॒रा॒ण इति॑ सं - र॒रा॒णः । \newline
18. यज॑मानाय॒ द्रवि॑ण॒म् द्रवि॑णं॒ ॅयज॑मानाय॒ यज॑मानाय॒ द्रवि॑णम् दधातु दधातु॒ द्रवि॑णं॒ ॅयज॑मानाय॒ यज॑मानाय॒ द्रवि॑णम् दधातु । \newline
19. द्रवि॑णम् दधातु दधातु॒ द्रवि॑ण॒म् द्रवि॑णम् दधातु । \newline
20. द॒धा॒त्विति॑ दधातु । \newline
21. स मि॑न्द्रे न्द्र॒ सꣳ स मि॑न्द्र णो न इन्द्र॒ सꣳ स मि॑न्द्र णः । \newline
22. इ॒न्द्र॒ णो॒ न॒ इ॒न्द्रे॒ न्द्र॒ णो॒ मन॑सा॒ मन॑सा न इन्द्रे न्द्र णो॒ मन॑सा । \newline
23. नो॒ मन॑सा॒ मन॑सा नो नो॒ मन॑सा नेषि नेषि॒ मन॑सा नो नो॒ मन॑सा नेषि । \newline
24. मन॑सा नेषि नेषि॒ मन॑सा॒ मन॑सा नेषि॒ गोभि॒र् गोभि॑र् नेषि॒ मन॑सा॒ मन॑सा नेषि॒ गोभिः॑ । \newline
25. ने॒षि॒ गोभि॒र् गोभि॑र् नेषि नेषि॒ गोभिः॒ सꣳ सम् गोभि॑र् नेषि नेषि॒ गोभिः॒ सम् । \newline
26. गोभिः॒ सꣳ सम् गोभि॒र् गोभिः॒ सꣳ सू॒रिभिः॑ सू॒रिभिः॒ सम् गोभि॒र् गोभिः॒ सꣳ सू॒रिभिः॑ । \newline
27. सꣳ सू॒रिभिः॑ सू॒रिभिः॒ सꣳ सꣳ सू॒रिभि॑र् मघवन् मघवन् थ्सू॒रिभिः॒ सꣳ सꣳ सू॒रिभि॑र् मघवन्न् । \newline
28. सू॒रिभि॑र् मघवन् मघवन् थ्सू॒रिभिः॑ सू॒रिभि॑र् मघव॒न् थ्सꣳ सम् म॑घवन् थ्सू॒रिभिः॑ सू॒रिभि॑र् मघव॒न् थ्सम् । \newline
29. सू॒रिभि॒रिति॑ सू॒रि - भिः॒ । \newline
30. म॒घ॒व॒न् थ्सꣳ सम् म॑घवन् मघव॒न् थ्सꣳ स्व॒स्त्या स्व॒स्त्या सम् म॑घवन् मघव॒न् थ्सꣳ स्व॒स्त्या । \newline
31. म॒घ॒व॒न्निति॑ मघ - व॒न्न् । \newline
32. सꣳ स्व॒स्त्या स्व॒स्त्या सꣳ सꣳ स्व॒स्त्या । \newline
33. स्व॒स्त्येति॑ स्व॒स्त्या । \newline
34. सम् ब्रह्म॑णा॒ ब्रह्म॑णा॒ सꣳ सम् ब्रह्म॑णा दे॒वकृ॑तम् दे॒वकृ॑त॒म् ब्रह्म॑णा॒ सꣳ सम् ब्रह्म॑णा दे॒वकृ॑तम् । \newline
35. ब्रह्म॑णा दे॒वकृ॑तम् दे॒वकृ॑त॒म् ब्रह्म॑णा॒ ब्रह्म॑णा दे॒वकृ॑तं॒ ॅयद् यद् दे॒वकृ॑त॒म् ब्रह्म॑णा॒ ब्रह्म॑णा दे॒वकृ॑तं॒ ॅयत् । \newline
36. दे॒वकृ॑तं॒ ॅयद् यद् दे॒वकृ॑तम् दे॒वकृ॑तं॒ ॅयदस्त्यस्ति॒ यद् दे॒वकृ॑तम् दे॒वकृ॑तं॒ ॅयदस्ति॑ । \newline
37. दे॒वकृ॑त॒मिति॑ दे॒व - कृ॒त॒म् । \newline
38. यदस्त्यस्ति॒ यद् यदस्ति॒ सꣳ समस्ति॒ यद् यदस्ति॒ सम् । \newline
39. अस्ति॒ सꣳ समस्त्यस्ति॒ सम् दे॒वाना᳚म् दे॒वाना॒(ग्म्॒) सम स्त्यस्ति॒ सम् दे॒वाना᳚म् । \newline
40. सम् दे॒वाना᳚म् दे॒वाना॒(ग्म्॒) सꣳ सम् दे॒वाना(ग्म्॑) सुम॒त्या सु॑म॒त्या दे॒वाना॒(ग्म्॒) सꣳ सम् दे॒वाना(ग्म्॑) सुम॒त्या । \newline
41. दे॒वाना(ग्म्॑) सुम॒त्या सु॑म॒त्या दे॒वाना᳚म् दे॒वाना(ग्म्॑) सुम॒त्या य॒ज्ञिया॑नां ॅय॒ज्ञिया॑नाꣳ सुम॒त्या दे॒वाना᳚म् दे॒वाना(ग्म्॑) सुम॒त्या य॒ज्ञिया॑नाम् । \newline
42. सु॒म॒त्या य॒ज्ञिया॑नां ॅय॒ज्ञिया॑नाꣳ सुम॒त्या सु॑म॒त्या य॒ज्ञिया॑नाम् । \newline
43. सु॒म॒त्येति॑ सु - म॒त्या । \newline
44. य॒ज्ञिया॑ना॒मिति॑ य॒ज्ञिया॑नाम् । \newline
45. सं ॅवर्च॑सा॒ वर्च॑सा॒ सꣳ सं ॅवर्च॑सा॒ पय॑सा॒ पय॑सा॒ वर्च॑सा॒ सꣳ सं ॅवर्च॑सा॒ पय॑सा । \newline
46. वर्च॑सा॒ पय॑सा॒ पय॑सा॒ वर्च॑सा॒ वर्च॑सा॒ पय॑सा॒ सꣳ सम् पय॑सा॒ वर्च॑सा॒ वर्च॑सा॒ पय॑सा॒ सम् । \newline
47. पय॑सा॒ सꣳ सम् पय॑सा॒ पय॑सा॒ सम् त॒नूभि॑ स्त॒नूभिः॒ सम् पय॑सा॒ पय॑सा॒ सम् त॒नूभिः॑ । \newline
48. सम् त॒नूभि॑ स्त॒नूभिः॒ सꣳ सम् त॒नूभि॒ रग॑न्म॒ ह्यग॑न्महि त॒नूभिः॒ सꣳ सम् त॒नूभि॒ रग॑न्महि । \newline
49. त॒नूभि॒ रग॑न्म॒ ह्यग॑न्महि त॒नूभि॑ स्त॒नूभि॒ रग॑न्महि॒ मन॑सा॒ मन॒सा ऽग॑न्महि त॒नूभि॑ स्त॒नूभि॒ रग॑न्महि॒ मन॑सा । \newline
50. अग॑न्महि॒ मन॑सा॒ मन॒सा ऽग॑न्म॒ ह्यग॑न्महि॒ मन॑सा॒ सꣳ सम् मन॒सा ऽग॑न्म॒ ह्यग॑न्महि॒ मन॑सा॒ सम् । \newline
51. मन॑सा॒ सꣳ सम् मन॑सा॒ मन॑सा॒ सꣳ शि॒वेन॑ शि॒वेन॒ सम् मन॑सा॒ मन॑सा॒ सꣳ शि॒वेन॑ । \newline
52. सꣳ शि॒वेन॑ शि॒वेन॒ सꣳ सꣳ शि॒वेन॑ । \newline
53. शि॒वेनेति॑ शि॒वेन॑ । \newline
54. त्वष्टा॑ नो न॒ स्त्वष्टा॒ त्वष्टा॑ नो॒ अत्रात्र॑ न॒ स्त्वष्टा॒ त्वष्टा॑ नो॒ अत्र॑ । \newline
55. नो॒ अत्रात्र॑ नो नो॒ अत्र॒ वरि॑वो॒ वरि॑वो॒ अत्र॑ नो नो॒ अत्र॒ वरि॑वः । \newline
56. अत्र॒ वरि॑वो॒ वरि॑वो॒ अत्रात्र॒ वरि॑वः कृणोतु कृणोतु॒ वरि॑वो॒ अत्रात्र॒ वरि॑वः कृणोतु । \newline
57. वरि॑वः कृणोतु कृणोतु॒ वरि॑वो॒ वरि॑वः कृणो॒त्वन्वनु॑ कृणोतु॒ वरि॑वो॒ वरि॑वः कृणो॒त्वनु॑ । \newline
58. कृ॒णो॒त्वन्वनु॑ कृणोतु कृणो॒त्वनु॑ मार्ष्टु मा॒र्ष्ट्वनु॑ कृणोतु कृणो॒त्वनु॑ मार्ष्टु । \newline
\pagebreak
\markright{ TS 1.4.44.2  \hfill https://www.vedavms.in \hfill}
\addcontentsline{toc}{section}{ TS 1.4.44.2 }
\section*{ TS 1.4.44.2 }

\textbf{TS 1.4.44.2 } \newline
\textbf{Samhita Paata} \newline

त्वनु॑ मार्ष्टु त॒नुवो॒ यद्विलि॑ष्टं ॥ यद॒द्य त्वा᳚ प्रय॒ति य॒ज्ञे अ॒स्मिन्नग्ने॒ होता॑र॒मवृ॑णीमही॒ह । ऋध॑गया॒डृध॑गु॒ताऽश॑मिष्ठाः प्रजा॒नन्. य॒ज्ञ्मुप॑ याहि वि॒द्वान् ॥ स्व॒गा वो॑ देवाः॒ सद॑नमकर्म॒ य आ॑ज॒ग्म सव॑ने॒दं जु॑षा॒णाः । ज॒क्षि॒वाꣳसः॑ पपि॒वाꣳस॑श्च॒ विश्वे॒ऽस्मे ध॑त्त वसवो॒ वसू॑नि ॥ यानाऽव॑ह उश॒तो दे॑व दे॒वान् तान्-[ ] \newline

\textbf{Pada Paata} \newline

अन्विति॑ । मा॒र्ष्टु॒ । त॒नुवः॑ । यत् । विलि॑ष्ट॒मिति॒ वि - लि॒ष्ट॒म् ॥ यत् । अ॒द्य । त्वा॒ । प्र॒य॒तीति॑ प्र - य॒ति । य॒ज्ञे । अ॒स्मिन्न् । अग्ने᳚ । होता॑रम् । अवृ॑णीमहि । इ॒ह ॥ ऋध॑क् । अ॒या॒ट् । ऋध॑क् । उ॒त । अश॑मिष्ठाः । प्र॒जा॒नन्निति॑ प्र - जा॒नन्न् । य॒ज्ञ्म् । उपेति॑ । या॒हि॒ । वि॒द्वान् ॥ स्व॒गेति॑ स्व - गा । वः॒ । दे॒वाः॒ । सद॑नम् । अ॒क॒र्म॒ । ये । आ॒ज॒ग्मेत्या᳚ - ज॒ग्म । सव॑ना । इ॒दम् । जु॒षा॒णाः ॥ ज॒क्षि॒वाꣳसः॑ । प॒पि॒वाꣳसः॑ । च॒ । विश्वे᳚ । अ॒स्मे इति॑ । ध॒त्त॒ । व॒स॒वः॒ । वसू॑नि ॥ यान् । एति॑ । अव॑हः । उ॒श॒तः । दे॒व॒ । दे॒वान् । तान् ।  \newline


\textbf{Krama Paata} \newline

अनु॑ मार्ष्टु । मा॒र्ष्टु॒ त॒नुवः॑ । त॒नुवो॒ यत् । यद् विलि॑ष्टम् । विलि॑ष्ट॒मिति॒ वि - लि॒ष्ट॒म् ॥ यद॒द्य । अ॒द्य त्वा᳚ । त्वा॒ प्र॒य॒ति । प्र॒य॒ति य॒ज्ञे । प्र॒य॒तीति॑ प्र - य॒ति । य॒ज्ञे अ॒स्मिन्न् । अ॒स्मिन्नग्ने᳚ । अग्ने॒ होता॑रम् । होता॑र॒मवृ॑णीमहि । अवृ॑णीमही॒ह । इ॒हेती॒ह ॥ ऋध॑ गयाट् । अ॒या॒डृध॑क् । ऋध॑गु॒त । उ॒ताश॑मिष्ठाः । अश॑मिष्ठाः प्रजा॒नन्न् । प्र॒जा॒नन्. य॒ज्ञ्म् । प्र॒जा॒नन्निति॑ प्र - जा॒नन्न् । य॒ज्ञ्मुप॑ । उप॑ याहि । या॒हि॒ वि॒द्वान् । वि॒द्वानिति॑ वि॒द्वान् ॥ स्व॒गा वः॑ । स्व॒गेति॑ स्व - गा । वो॒ दे॒वाः॒ । दे॒वाः॒ सद॑नम् । सद॑नमकर्म । अ॒क॒र्म॒ ये । य आ॑ज॒ग्म । आ॒ज॒ग्म सव॑ना । आ॒ज॒ग्मेत्या᳚ - ज॒ग्म । सव॑ने॒दम् । इ॒दम् जु॑षा॒णाः । जु॒षा॒णा इति॑ जुषा॒णाः ॥ ज॒क्षि॒वाꣳसः॑ पपि॒वाꣳसः॑ । प॒पि॒वाꣳस॑श्च । च॒ विश्वे᳚ । विश्वे॒ ऽस्मे । अ॒स्मे ध॑त्त । अ॒स्मे इत्य॒स्मे । ध॒त्त॒ व॒स॒वः॒ । व॒स॒वो॒ वसू॑नि । वसू॒नीति॒ वसू॑नि ॥ याना । आऽव॑हः । अव॑ह उश॒तः । उ॒श॒तो दे॑व । दे॒व॒ दे॒वान् । दे॒वान् तान् । तान् प्र \newline

\textbf{Jatai Paata} \newline

1. अनु॑ मार्ष्टु मा॒र्ष्ट्वन्वनु॑ मार्ष्टु । \newline
2. मा॒र्ष्टु॒ त॒नुव॑ स्त॒नुवो॑ मार्ष्टु मार्ष्टु त॒नुवः॑ । \newline
3. त॒नुवो॒ यद् यत् त॒नुव॑ स्त॒नुवो॒ यत् । \newline
4. यद् विलि॑ष्टं॒ ॅविलि॑ष्टं॒ ॅयद् यद् विलि॑ष्टम् । \newline
5. विलि॑ष्ट॒मिति॒ वि - लि॒ष्ट॒म् । \newline
6. यद॒द्याद्य यद् यद॒द्य । \newline
7. अ॒द्य त्वा᳚ त्वा॒ ऽद्याद्य त्वा᳚ । \newline
8. त्वा॒ प्र॒य॒ति प्र॑य॒ति त्वा᳚ त्वा प्रय॒ति । \newline
9. प्र॒य॒ति य॒ज्ञे य॒ज्ञे प्र॑य॒ति प्र॑य॒ति य॒ज्ञे । \newline
10. प्र॒यतीति॑ प्र - य॒ति । \newline
11. य॒ज्ञे अ॒स्मिन् न॒स्मिन्. य॒ज्ञे य॒ज्ञे अ॒स्मिन्न् । \newline
12. अ॒स्मिन् नग्ने ऽग्ने॑ अ॒स्मिन् न॒स्मिन् नग्ने᳚ । \newline
13. अग्ने॒ होता॑र॒(ग्म्॒) होता॑र॒ मग्ने ऽग्ने॒ होता॑रम् । \newline
14. होता॑र॒ मवृ॑णीम॒ह्यवृ॑णीमहि॒ होता॑र॒(ग्म्॒) होता॑र॒ मवृ॑णीमहि । \newline
15. अवृ॑णीमही॒हे हावृ॑णी म॒ह्यवृ॑णीमही॒ह । \newline
16. इ॒हेती॒ह । \newline
17. ऋध॑ गयाडया॒ डृध॒ गृध॑गयाट् । \newline
18. अ॒या॒ डृध॒ गृध॑ गया डया॒डृध॑क् । \newline
19. ऋध॑गु॒तोत र्ध॒गृध॑गु॒त । \newline
20. उ॒ताश॑मिष्ठा॒ अश॑मिष्ठा उ॒तोताश॑मिष्ठाः । \newline
21. अश॑मिष्ठाः प्रजा॒नन् प्र॑जा॒नन् नश॑मिष्ठा॒ अश॑मिष्ठाः प्रजा॒नन्न् । \newline
22. प्र॒जा॒नन्. य॒ज्ञ्ं ॅय॒ज्ञ्म् प्र॑जा॒नन् प्र॑जा॒नन्. य॒ज्ञ्म् । \newline
23. प्र॒जा॒नन्निति॑ प्र - जा॒नन्न् । \newline
24. य॒ज्ञ् मुपोप॑ य॒ज्ञ्ं ॅय॒ज्ञ् मुप॑ । \newline
25. उप॑ याहि या॒ह्युपोप॑ याहि । \newline
26. या॒हि॒ वि॒द्वान्. वि॒द्वान्. या॑हि याहि वि॒द्वान् । \newline
27. वि॒द्वानिति॑ वि॒द्वान् । \newline
28. स्व॒गा वो॑ वः स्व॒गा स्व॒गा वः॑ । \newline
29. स्व॒गेति॑ स्व - गा । \newline
30. वो॒ दे॒वा॒ दे॒वा॒ वो॒ वो॒ दे॒वाः॒ । \newline
31. दे॒वाः॒ सद॑न॒(ग्म्॒) सद॑नम् देवा देवाः॒ सद॑नम् । \newline
32. सद॑न मकर्माकर्म॒ सद॑न॒(ग्म्॒) सद॑न मकर्म । \newline
33. अ॒क॒र्म॒ ये ये अ॑कर्माकर्म॒ ये । \newline
34. य आ॑ज॒ग्माज॒ग्म ये य आ॑ज॒ग्म । \newline
35. आ॒ज॒ग्म सव॑ना॒ सव॑ना ऽऽज॒ग्माज॒ग्म सव॑ना । \newline
36. आ॒ज॒ग्मेत्या᳚ - ज॒ग्म । \newline
37. सव॑ने॒द मि॒दꣳ सव॑ना॒ सव॑ने॒दम् । \newline
38. इ॒दम् जु॑षा॒णा जु॑षा॒णा इ॒द मि॒दम् जु॑षा॒णाः । \newline
39. जु॒षा॒णा इति॑ जुषा॒णाः । \newline
40. ज॒क्षि॒वाꣳसः॑ पपि॒वाꣳसः॑ पपि॒वाꣳसो॑ जक्षि॒वाꣳसो॑ जक्षि॒वाꣳसः॑ पपि॒वाꣳसः॑ । \newline
41. प॒पि॒वाꣳस॑श्च च पपि॒वाꣳसः॑ पपि॒वाꣳस॑श्च । \newline
42. च॒ विश्वे॒ विश्वे॑ च च॒ विश्वे᳚ । \newline
43. विश्वे॒ ऽस्मे अ॒स्मे विश्वे॒ विश्वे॒ ऽस्मे । \newline
44. अ॒स्मे ध॑त्त धत्ता॒स्मे अ॒स्मे ध॑त्त । \newline
45. अ॒स्मे इत्य॒स्मे । \newline
46. ध॒त्त॒ व॒स॒वो॒ व॒स॒वो॒ ध॒त्त॒ ध॒त्त॒ व॒स॒वः॒ । \newline
47. व॒स॒वो॒ वसू॑नि॒ वसू॑नि वसवो वसवो॒ वसू॑नि । \newline
48. वसू॒नीति॒ वसू॑नि । \newline
49. या ना यान्. या ना । \newline
50. आ ऽव॒हो ऽव॑ह॒ आ ऽव॑हः । \newline
51. अव॑ह उश॒त उ॑श॒तो ऽव॒हो ऽव॑ह उश॒तः । \newline
52. उ॒श॒तो दे॑व देवोश॒त उ॑श॒तो दे॑व । \newline
53. दे॒व॒ दे॒वान् दे॒वान् दे॑व देव दे॒वान् । \newline
54. दे॒वान् ताꣳस् तान् दे॒वान् दे॒वान् तान् । \newline
55. तान् प्र प्र ताꣳ स्तान् प्र । \newline

\textbf{Ghana Paata } \newline

1. अनु॑ मार्ष्टु मा॒र्ष्ट्वन्वनु॑ मार्ष्टु त॒नुव॑ स्त॒नुवो॑ मा॒र्ष्ट्वन्वनु॑ मार्ष्टु त॒नुवः॑ । \newline
2. मा॒र्ष्टु॒ त॒नुव॑ स्त॒नुवो॑ मार्ष्टु मार्ष्टु त॒नुवो॒ यद् यत् त॒नुवो॑ मार्ष्टु मार्ष्टु त॒नुवो॒ यत् । \newline
3. त॒नुवो॒ यद् यत् त॒नुव॑ स्त॒नुवो॒ यद् विलि॑ष्टं॒ ॅविलि॑ष्टं॒ ॅयत् त॒नुव॑ स्त॒नुवो॒ यद् विलि॑ष्टम् । \newline
4. यद् विलि॑ष्टं॒ ॅविलि॑ष्टं॒ ॅयद् यद् विलि॑ष्टम् । \newline
5. विलि॑ष्ट॒मिति॒ वि - लि॒ष्ट॒म् । \newline
6. यद॒द्याद्य यद् यद॒द्य त्वा᳚ त्वा॒ ऽद्य यद् यद॒द्य त्वा᳚ । \newline
7. अ॒द्य त्वा᳚ त्वा॒ ऽद्याद्य त्वा᳚ प्रय॒ति प्र॑य॒ति त्वा॒ ऽद्याद्य त्वा᳚ प्रय॒ति । \newline
8. त्वा॒ प्र॒य॒ति प्र॑य॒ति त्वा᳚ त्वा प्रय॒ति य॒ज्ञे य॒ज्ञे प्र॑य॒ति त्वा᳚ त्वा प्रय॒ति य॒ज्ञे । \newline
9. प्र॒य॒ति य॒ज्ञे य॒ज्ञे प्र॑य॒ति प्र॑य॒ति य॒ज्ञे अ॒स्मिन् न॒स्मिन्. य॒ज्ञे प्र॑य॒ति प्र॑य॒ति य॒ज्ञे अ॒स्मिन्न् । \newline
10. प्र॒यतीति॑ प्र - य॒ति । \newline
11. य॒ज्ञे अ॒स्मिन् न॒स्मिन्. य॒ज्ञे य॒ज्ञे अ॒स्मिन् नग्ने ऽग्ने॑ अ॒स्मिन्. य॒ज्ञे य॒ज्ञे अ॒स्मिन् नग्ने᳚ । \newline
12. अ॒स्मिन् नग्ने ऽग्ने॑ अ॒स्मिन् न॒स्मिन् नग्ने॒ होता॑र॒(ग्म्॒) होता॑र॒ मग्ने॑ अ॒स्मिन् न॒स्मिन् नग्ने॒ होता॑रम् । \newline
13. अग्ने॒ होता॑र॒(ग्म्॒) होता॑र॒ मग्ने ऽग्ने॒ होता॑र॒ मवृ॑णीम॒ ह्यवृ॑णीमहि॒ होता॑र॒ मग्ने ऽग्ने॒ होता॑र॒ मवृ॑णीमहि । \newline
14. होता॑र॒ मवृ॑णीम॒ ह्यवृ॑णीमहि॒ होता॑र॒(ग्म्॒) होता॑र॒ मवृ॑णीमही॒हे हावृ॑णीमहि॒ होता॑र॒(ग्म्॒) होता॑र॒ मवृ॑णीमही॒ह । \newline
15. अवृ॑णीमही॒हे हावृ॑णीम॒ ह्यवृ॑णीमही॒ह । \newline
16. इ॒हेती॒ह । \newline
17. ऋध॑ गया डया॒ डृध॒ गृ ध॑ गया॒ डृध॒ गृध॑ गया॒ डृध॒ गृध॑ गया॒डृध॑क् । \newline
18. अ॒या॒ डृध॒ गृध॑ गया डया॒ डृध॑गु॒तोत र्ध॑गया डया॒ डृध॑गु॒त । \newline
19. ऋध॑गु॒तोत र्ध॒गृध॑ गु॒ताश॑मिष्ठा॒ अश॑मिष्ठा उ॒त र्ध॒गृध॑ गु॒ताश॑मिष्ठाः । \newline
20. उ॒ताश॑मिष्ठा॒ अश॑मिष्ठा उ॒तोताश॑मिष्ठाः प्रजा॒नन् प्र॑जा॒नन् नश॑मिष्ठा उ॒तोताश॑मिष्ठाः प्रजा॒नन्न् । \newline
21. अश॑मिष्ठाः प्रजा॒नन् प्र॑जा॒नन् नश॑मिष्ठा॒ अश॑मिष्ठाः प्रजा॒नन्. य॒ज्ञ्ं ॅय॒ज्ञ्म् प्र॑जा॒नन् नश॑मिष्ठा॒ अश॑मिष्ठाः प्रजा॒नन्. य॒ज्ञ्म् । \newline
22. प्र॒जा॒नन्. य॒ज्ञ्ं ॅय॒ज्ञ्म् प्र॑जा॒नन् प्र॑जा॒नन्. य॒ज्ञ् मुपोप॑ य॒ज्ञ्म् प्र॑जा॒नन् प्र॑जा॒नन्. य॒ज्ञ् मुप॑ । \newline
23. प्र॒जा॒नन्निति॑ प्र - जा॒नन्न् । \newline
24. य॒ज्ञ् मुपोप॑ य॒ज्ञ्ं ॅय॒ज्ञ् मुप॑ याहि या॒ह्युप॑ य॒ज्ञ्ं ॅय॒ज्ञ् मुप॑ याहि । \newline
25. उप॑ याहि या॒ह्युपोप॑ याहि वि॒द्वान्. वि॒द्वान्. या॒ह्युपोप॑ याहि वि॒द्वान् । \newline
26. या॒हि॒ वि॒द्वान्. वि॒द्वान्. या॑हि याहि वि॒द्वान् । \newline
27. वि॒द्वानिति॑ वि॒द्वान् । \newline
28. स्व॒गा वो॑ वः स्व॒गा स्व॒गा वो॑ देवा देवा वः स्व॒गा स्व॒गा वो॑ देवाः । \newline
29. स्व॒गेति॑ स्व - गा । \newline
30. वो॒ दे॒वा॒ दे॒वा॒ वो॒ वो॒ दे॒वाः॒ सद॑न॒(ग्म्॒) सद॑नम् देवा वो वो देवाः॒ सद॑नम् । \newline
31. दे॒वाः॒ सद॑न॒(ग्म्॒) सद॑नम् देवा देवाः॒ सद॑न मकर्माकर्म॒ सद॑नम् देवा देवाः॒ सद॑न मकर्म । \newline
32. सद॑न मकर्माकर्म॒ सद॑न॒(ग्म्॒) सद॑न मकर्म॒ ये ये अ॑कर्म॒ सद॑न॒(ग्म्॒) सद॑न मकर्म॒ ये । \newline
33. अ॒क॒र्म॒ ये ये अ॑कर्माकर्म॒ य आ॑ज॒ग् माज॒ग्म ये अ॑कर्माकर्म॒ य आ॑ज॒ग्म । \newline
34. य आ॑ज॒ग्माज॒ग्म ये य आ॑ज॒ग्म सव॑ना॒ सव॑ना ऽऽज॒ग्म ये य आ॑ज॒ग्म सव॑ना । \newline
35. आ॒ज॒ग्म सव॑ना॒ सव॑ना ऽऽज॒ग्माज॒ग्म सव॑ने॒द मि॒दꣳ सव॑ना ऽऽज॒ग्माज॒ग्म सव॑ने॒दम् । \newline
36. आ॒ज॒ग्मेत्या᳚ - ज॒ग्म । \newline
37. सव॑ने॒द मि॒दꣳ सव॑ना॒ सव॑ने॒दम् जु॑षा॒णा जु॑षा॒णा इ॒दꣳ सव॑ना॒ सव॑ने॒दम् जु॑षा॒णाः । \newline
38. इ॒दम् जु॑षा॒णा जु॑षा॒णा इ॒द मि॒दम् जु॑षा॒णाः । \newline
39. जु॒षा॒णा इति॑ जुषा॒णाः । \newline
40. ज॒क्षि॒वाꣳसः॑ पपि॒वाꣳसः॑ पपि॒वाꣳसो॑ जक्षि॒वाꣳसो॑ जक्षि॒वाꣳसः॑ पपि॒वाꣳस॑श्च च पपि॒वाꣳसो॑ जक्षि॒वाꣳसो॑ जक्षि॒वाꣳसः॑ पपि॒वाꣳस॑श्च । \newline
41. प॒पि॒वाꣳस॑श्च च पपि॒वाꣳसः॑ पपि॒वाꣳस॑श्च॒ विश्वे॒ विश्वे॑ च पपि॒वाꣳसः॑ पपि॒वाꣳस॑श्च॒ विश्वे᳚ । \newline
42. च॒ विश्वे॒ विश्वे॑ च च॒ विश्वे॒ ऽस्मे अ॒स्मे विश्वे॑ च च॒ विश्वे॒ ऽस्मे । \newline
43. विश्वे॒ ऽस्मे अ॒स्मे विश्वे॒ विश्वे॒ ऽस्मे ध॑त्त धत्ता॒स्मे विश्वे॒ विश्वे॒ ऽस्मे ध॑त्त । \newline
44. अ॒स्मे ध॑त्त धत्ता॒स्मे अ॒स्मे ध॑त्त वसवो वसवो धत्ता॒स्मे अ॒स्मे ध॑त्त वसवः । \newline
45. अ॒स्मे इत्य॒स्मे । \newline
46. ध॒त्त॒ व॒स॒वो॒ व॒स॒वो॒ ध॒त्त॒ ध॒त्त॒ व॒स॒वो॒ वसू॑नि॒ वसू॑नि वसवो धत्त धत्त वसवो॒ वसू॑नि । \newline
47. व॒स॒वो॒ वसू॑नि॒ वसू॑नि वसवो वसवो॒ वसू॑नि । \newline
48. वसू॒नीति॒ वसू॑नि । \newline
49. या ना यान्. या ना ऽव॒हो ऽव॑ह॒ आ यान्. या ना ऽव॑हः । \newline
50. आ ऽव॒हो ऽव॑ह॒ आ ऽव॑ह उश॒त उ॑श॒तो ऽव॑ह॒ आ ऽव॑ह उश॒तः । \newline
51. अव॑ह उश॒त उ॑श॒तो ऽव॒हो ऽव॑ह उश॒तो दे॑व देवोश॒तो ऽव॒हो ऽव॑ह उश॒तो दे॑व । \newline
52. उ॒श॒तो दे॑व देवोश॒त उ॑श॒तो दे॑व दे॒वान् दे॒वान् दे॑वोश॒त उ॑श॒तो दे॑व दे॒वान् । \newline
53. दे॒व॒ दे॒वान् दे॒वान् दे॑व देव दे॒वान् ताꣳ स्तान् दे॒वान् दे॑व देव दे॒वान् तान् । \newline
54. दे॒वान् ताꣳ स्तान् दे॒वान् दे॒वान् तान् प्र प्र तान् दे॒वान् दे॒वान् तान् प्र । \newline
55. तान् प्र प्र ताꣳ स्तान् प्रेर॑येरय॒ प्र ताꣳ स्तान् प्रेर॑य । \newline
\pagebreak
\markright{ TS 1.4.44.3  \hfill https://www.vedavms.in \hfill}
\addcontentsline{toc}{section}{ TS 1.4.44.3 }
\section*{ TS 1.4.44.3 }

\textbf{TS 1.4.44.3 } \newline
\textbf{Samhita Paata} \newline

प्रेर॑य॒ स्वे अ॑ग्ने स॒धस्थे᳚ । वह॑माना॒ भर॑माणा ह॒वीꣳषि॒ वसुं॑ घ॒र्मं दिव॒मा ति॑ष्ठ॒तानु॑ । यज्ञ्॑ य॒ज्ञ्ं ग॑च्छ य॒ज्ञ्प॑तिं गच्छ॒ स्वां ॅयोनिं॑ गच्छ॒ स्वाहै॒ष ते॑ य॒ज्ञो य॑ज्ञ्पते स॒हसू᳚क्तवाकः सु॒वीरः॒ स्वाहा॒ देवा॑ गातुविदो गा॒तुं ॅवि॒त्त्वा गा॒तुमि॑त॒ मन॑सस्पत इ॒मं नो॑ देव दे॒वेषु॑ य॒ज्ञ्ꣳ स्वाहा॑ वा॒चि स्वाहा॒ वाते॑ धाः ॥ \newline

\textbf{Pada Paata} \newline

प्रेति॑ । ई॒र॒य॒ । स्वे । अ॒ग्ने॒ । स॒धस्थ॒ इति॑ स॒ध - स्थे॒ ॥ वह॑मानाः । भर॑माणाः । ह॒वीꣳषि॑ । वसु᳚म् । घ॒र्मम् । दिव᳚म् । एति॑ । ति॒ष्ठ॒त॒ । अनु॑ ॥ यज्ञ्॑ । य॒ज्ञ्म् । ग॒च्छ॒ । य॒ज्ञ्प॑ति॒मिति॑ य॒ज्ञ् - प॒ति॒म् । ग॒च्छ॒ । स्वाम् । योनि᳚म् । ग॒च्छ॒ । स्वाहा᳚ । ए॒षः । ते॒ । य॒ज्ञ्ः । य॒ज्ञ्॒प॒त॒ इति॑ यज्ञ् - प॒ते॒ । स॒हसू᳚क्तवाक॒ इति॑ स॒हस᳚क्त - वा॒कः॒ । सु॒वीर॒ इति॑ सु - वीरः॑ । स्वाहा᳚ । देवाः᳚ । गा॒तु॒वि॒द॒ इति॑ गातु - वि॒दः॒ । गा॒तुम् । वि॒त्त्वा । गा॒तुम् । इ॒त॒ । मन॑सः । प॒ते॒ । इ॒मम् । नः॒ । दे॒व॒ । दे॒वेषु॑ । य॒ज्ञ्म् । स्वाहा᳚ । वा॒चि । स्वाहा᳚ । वाते᳚ । धाः॒ ॥(कृ॒णो॒तु॒ - तान॒ - ष्टाच॑त्वारिꣳशच्च )(आ44 )  \newline


\textbf{Krama Paata} \newline

प्रेर॑य । ई॒र॒य॒ स्वे । स्वे अ॑ग्ने । अ॒ग्ने॒ स॒धस्थे᳚ । स॒धस्थ॒ इति॑ स॒ध - स्थे॒ ॥ वह॑माना॒ भर॑माणाः । भर॑माणा ह॒वीꣳषि॑ । ह॒वीꣳषि॒ वसु᳚म् । वसु॑म् घ॒र्मम् । घ॒र्मम् दिव᳚म् । दिव॒मा । आ ति॑ष्ठत । ति॒ष्ठ॒तानु॑ । अन्वित्यनु॑ ॥ यज्ञ्॑ य॒ज्ञ्म् । य॒ज्ञ्म् ग॑च्छ । ग॒च्छ॒ य॒ज्ञ्प॑तिम् । य॒ज्ञ्प॑तिम् गच्छ । य॒ज्ञ्प॑ति॒मिति॑ य॒ज्ञ् - प॒ति॒म् । ग॒च्छ॒ स्वाम् । स्वां ॅयोनि᳚म् । योनि॑म् गच्छ । ग॒च्छ॒ स्वाहा᳚ । स्वाहै॒षः । ए॒ष ते᳚ । ते॒ य॒ज्ञ्ः । य॒ज्ञो य॑ज्ञ्पते । य॒ज्ञ्॒प॒ते॒ स॒हसू᳚क्तवाकः । य॒ज्ञ्॒प॒त॒ इति॑ यज्ञ् - प॒ते॒ । स॒हसू᳚क्तवाकः सु॒वीरः॑ । स॒हसू᳚क्तवाक॒ इति॑ स॒हसू᳚क्त - वा॒कः॒ । सु॒वीरः॒ स्वाहा᳚ । सु॒वीर॒ इति॑ सु - वीरः॑ । स्वाहा॒ देवाः᳚ । देवा॑ गातुविदः । गा॒तु॒वि॒दो॒ गा॒तुम् । गा॒तु॒वि॒द॒ इति॑ गातु - वि॒दः॒ । गा॒तुं ॅवि॒त्वा । वि॒त्वा गा॒तुम् । गा॒तुमि॑त । इ॒त॒ मन॑सः । मन॑सस्पते । प॒त॒ इ॒मम् । इ॒मम् नः॑ । नो॒ दे॒व॒ । दे॒व॒ दे॒वेषु॑ । दे॒वेषु॑ य॒ज्ञ्म् । य॒ज्ञ्ꣳ स्वाहा᳚ । स्वाहा॑ वा॒चि । वा॒चि स्वाहा᳚ । स्वाहा॒ वाते᳚ । वाते॑ धाः । धा॒ इति॑ धाः । \newline

\textbf{Jatai Paata} \newline

1. प्रेर॑येरय॒ प्र प्रेर॑य । \newline
2. ई॒र॒य॒ स्वे स्व ई॑रयेरय॒ स्वे । \newline
3. स्वे अ॑ग्ने अग्ने॒ स्वे स्वे अ॑ग्ने । \newline
4. अ॒ग्ने॒ स॒धस्थे॑ स॒धस्थे᳚ ऽग्ने अग्ने स॒धस्थे᳚ । \newline
5. स॒धस्थ॒ इति॑ स॒ध - स्थे॒ । \newline
6. वह॑माना॒ भर॑माणा॒ भर॑माणा॒ वह॑माना॒ वह॑माना॒ भर॑माणाः । \newline
7. भर॑माणा ह॒वीꣳषि॑ ह॒वीꣳषि॒ भर॑माणा॒ भर॑माणा ह॒वीꣳषि॑ । \newline
8. ह॒वीꣳषि॒ वसुं॒ ॅवसु(ग्म्॑) ह॒वीꣳषि॑ ह॒वीꣳषि॒ वसु᳚म् । \newline
9. वसु॑म् घ॒र्मम् घ॒र्मं ॅवसुं॒ ॅवसु॑म् घ॒र्मम् । \newline
10. घ॒र्मम् दिव॒म् दिव॑म् घ॒र्मम् घ॒र्मम् दिव᳚म् । \newline
11. दिव॒ मा दिव॒म् दिव॒ मा । \newline
12. आ ति॑ष्ठत तिष्ठ॒ता ति॑ष्ठत । \newline
13. ति॒ष्ठ॒तान्वनु॑ तिष्ठत तिष्ठ॒तानु॑ । \newline
14. अन्वित्यनु॑ । \newline
15. यज्ञ्॑ य॒ज्ञ्ं ॅय॒ज्ञ्ं ॅयज्ञ्॒ यज्ञ्॑ य॒ज्ञ्म् । \newline
16. य॒ज्ञ्म् ग॑च्छ गच्छ य॒ज्ञ्ं ॅय॒ज्ञ्म् ग॑च्छ । \newline
17. ग॒च्छ॒ य॒ज्ञ्प॑तिं ॅय॒ज्ञ्प॑तिम् गच्छ गच्छ य॒ज्ञ्प॑तिम् । \newline
18. य॒ज्ञ्प॑तिम् गच्छ गच्छ य॒ज्ञ्प॑तिं ॅय॒ज्ञ्प॑तिम् गच्छ । \newline
19. य॒ज्ञ्प॑ति॒मिति॑ य॒ज्ञ् - प॒ति॒म् । \newline
20. ग॒च्छ॒ स्वाꣳ स्वाम् ग॑च्छ गच्छ॒ स्वाम् । \newline
21. स्वां ॅयोनिं॒ ॅयोनि॒(ग्ग्॒) स्वाꣳ स्वां ॅयोनि᳚म् । \newline
22. योनि॑म् गच्छ गच्छ॒ योनिं॒ ॅयोनि॑म् गच्छ । \newline
23. ग॒च्छ॒ स्वाहा॒ स्वाहा॑ गच्छ गच्छ॒ स्वाहा᳚ । \newline
24. स्वाहै॒ष ए॒ष स्वाहा॒ स्वाहै॒षः । \newline
25. ए॒ष ते॑ त ए॒ष ए॒ष ते᳚ । \newline
26. ते॒ य॒ज्ञो य॒ज्ञ् स्ते॑ ते य॒ज्ञ्ः । \newline
27. य॒ज्ञो य॑ज्ञ्पते यज्ञ्पते य॒ज्ञो य॒ज्ञो य॑ज्ञ्पते । \newline
28. य॒ज्ञ्॒प॒ते॒ स॒हसू᳚क्तवाकः स॒हसू᳚क्तवाको यज्ञ्पते यज्ञ्पते स॒हसू᳚क्तवाकः । \newline
29. य॒ज्ञ्॒प॒त॒ इति॑ यज्ञ् - प॒ते॒ । \newline
30. स॒हसू᳚क्तवाकः सु॒वीरः॑ सु॒वीरः॑ स॒हसू᳚क्तवाकः स॒हसू᳚क्तवाकः सु॒वीरः॑ । \newline
31. स॒हसू᳚क्तवाक॒ इति॑ स॒हसू᳚क्त - वा॒कः॒ । \newline
32. सु॒वीरः॒ स्वाहा॒ स्वाहा॑ सु॒वीरः॑ सु॒वीरः॒ स्वाहा᳚ । \newline
33. सु॒वीर॒ इति॑ सु - वीरः॑ । \newline
34. स्वाहा॒ देवा॒ देवाः॒ स्वाहा॒ स्वाहा॒ देवाः᳚ । \newline
35. देवा॑ गातुविदो गातुविदो॒ देवा॒ देवा॑ गातुविदः । \newline
36. गा॒तु॒वि॒दो॒ गा॒तुम् गा॒तुम् गा॑तुविदो गातुविदो गा॒तुम् । \newline
37. गा॒तु॒वि॒द॒ इति॑ गातु - वि॒दः॒ । \newline
38. गा॒तुं ॅवि॒त्त्वा वि॒त्त्वा गा॒तुम् गा॒तुं ॅवि॒त्त्वा । \newline
39. वि॒त्त्वा गा॒तुम् गा॒तुं ॅवि॒त्त्वा वि॒त्त्वा गा॒तुम् । \newline
40. गा॒तु मि॑ते त गा॒तुम् गा॒तु मि॑त । \newline
41. इ॒त॒ मन॑सो॒ मन॑स इते त॒ मन॑सः । \newline
42. मन॑स स्पते पते॒ मन॑सो॒ मन॑स स्पते । \newline
43. प॒त॒ इ॒म मि॒मम् प॑ते पत इ॒मम् । \newline
44. इ॒मन्नो॑ न इ॒म मि॒मन्नः॑ । \newline
45. नो॒ दे॒व॒ दे॒व॒ नो॒ नो॒ दे॒व॒ । \newline
46. दे॒व॒ दे॒वेषु॑ दे॒वेषु॑ देव देव दे॒वेषु॑ । \newline
47. दे॒वेषु॑ य॒ज्ञ्ं ॅय॒ज्ञ्म् दे॒वेषु॑ दे॒वेषु॑ य॒ज्ञ्म् । \newline
48. य॒ज्ञ्ꣳ स्वाहा॒ स्वाहा॑ य॒ज्ञ्ं ॅय॒ज्ञ्ꣳ स्वाहा᳚ । \newline
49. स्वाहा॑ वा॒चि वा॒चि स्वाहा॒ स्वाहा॑ वा॒चि । \newline
50. वा॒चि स्वाहा॒ स्वाहा॑ वा॒चि वा॒चि स्वाहा᳚ । \newline
51. स्वाहा॒ वाते॒ वाते॒ स्वाहा॒ स्वाहा॒ वाते᳚ । \newline
52. वाते॑ धा धा॒ वाते॒ वाते॑ धाः । \newline
53. धा॒ इति॑ धाः । \newline

\textbf{Ghana Paata } \newline

1. प्रेर॑ येरय॒ प्र प्रेर॑य॒ स्वे स्व ई॑रय॒ प्र प्रेर॑य॒ स्वे । \newline
2. ई॒र॒य॒ स्वे स्व ई॑रयेरय॒ स्वे अ॑ग्ने अग्ने॒ स्व ई॑रयेरय॒ स्वे अ॑ग्ने । \newline
3. स्वे अ॑ग्ने अग्ने॒ स्वे स्वे अ॑ग्ने स॒धस्थे॑ स॒धस्थे᳚ ऽग्ने॒ स्वे स्वे अ॑ग्ने स॒धस्थे᳚ । \newline
4. अ॒ग्ने॒ स॒धस्थे॑ स॒धस्थे᳚ ऽग्ने अग्ने स॒धस्थे᳚ । \newline
5. स॒धस्थ॒ इति॑ स॒ध - स्थे॒ । \newline
6. वह॑माना॒ भर॑माणा॒ भर॑माणा॒ वह॑माना॒ वह॑माना॒ भर॑माणा ह॒वीꣳषि॑ ह॒वीꣳषि॒ भर॑माणा॒ वह॑माना॒ वह॑माना॒ भर॑माणा ह॒वीꣳषि॑ । \newline
7. भर॑माणा ह॒वीꣳषि॑ ह॒वीꣳषि॒ भर॑माणा॒ भर॑माणा ह॒वीꣳषि॒ वसुं॒ ॅवसु(ग्म्॑) ह॒वीꣳषि॒ भर॑माणा॒ भर॑माणा ह॒वीꣳषि॒ वसु᳚म् । \newline
8. ह॒वीꣳषि॒ वसुं॒ ॅवसु(ग्म्॑) ह॒वीꣳषि॑ ह॒वीꣳषि॒ वसु॑म् घ॒र्मम् घ॒र्मं ॅवसु(ग्म्॑) ह॒वीꣳषि॑ ह॒वीꣳषि॒ वसु॑म् घ॒र्मम् । \newline
9. वसु॑म् घ॒र्मम् घ॒र्मं ॅवसुं॒ ॅवसु॑म् घ॒र्मम् दिव॒म् दिव॑म् घ॒र्मं ॅवसुं॒ ॅवसु॑म् घ॒र्मम् दिव᳚म् । \newline
10. घ॒र्मम् दिव॒म् दिव॑म् घ॒र्मम् घ॒र्मम् दिव॒ मा दिव॑म् घ॒र्मम् घ॒र्मम् दिव॒ मा । \newline
11. दिव॒ मा दिव॒म् दिव॒ मा ति॑ष्ठत तिष्ठ॒ता दिव॒म् दिव॒ मा ति॑ष्ठत । \newline
12. आ ति॑ष्ठत तिष्ठ॒ता ति॑ष्ठ॒तान्वनु॑ तिष्ठ॒ता ति॑ष्ठ॒तानु॑ । \newline
13. ति॒ष्ठ॒तान्वनु॑ तिष्ठत तिष्ठ॒तानु॑ । \newline
14. अन्वित्यनु॑ । \newline
15. यज्ञ्॑ य॒ज्ञ्ं ॅय॒ज्ञ्ं ॅयज्ञ्॒ यज्ञ्॑ य॒ज्ञ्म् ग॑च्छ गच्छ य॒ज्ञ्ं ॅयज्ञ्॒ यज्ञ्॑ य॒ज्ञ्म् ग॑च्छ । \newline
16. य॒ज्ञ्म् ग॑च्छ गच्छ य॒ज्ञ्ं ॅय॒ज्ञ्म् ग॑च्छ य॒ज्ञ्प॑तिं ॅय॒ज्ञ्प॑तिम् गच्छ य॒ज्ञ्ं ॅय॒ज्ञ्म् ग॑च्छ य॒ज्ञ्प॑तिम् । \newline
17. ग॒च्छ॒ य॒ज्ञ्प॑तिं ॅय॒ज्ञ्प॑तिम् गच्छ गच्छ य॒ज्ञ्प॑तिम् गच्छ गच्छ य॒ज्ञ्प॑तिम् गच्छ गच्छ य॒ज्ञ्प॑तिम् गच्छ । \newline
18. य॒ज्ञ्प॑तिम् गच्छ गच्छ य॒ज्ञ्प॑तिं ॅय॒ज्ञ्प॑तिम् गच्छ॒ स्वाꣳ स्वाम् ग॑च्छ य॒ज्ञ्प॑तिं ॅय॒ज्ञ्प॑तिम् गच्छ॒ स्वाम् । \newline
19. य॒ज्ञ्प॑ति॒मिति॑ य॒ज्ञ् - प॒ति॒म् । \newline
20. ग॒च्छ॒ स्वाꣳ स्वाम् ग॑च्छ गच्छ॒ स्वां ॅयोनिं॒ ॅयोनि॒(ग्ग्॒) स्वाम् ग॑च्छ गच्छ॒ स्वां ॅयोनि᳚म् । \newline
21. स्वां ॅयोनिं॒ ॅयोनि॒(ग्ग्॒) स्वाꣳ स्वां ॅयोनि॑म् गच्छ गच्छ॒ योनि॒(ग्ग्॒) स्वाꣳ स्वां ॅयोनि॑म् गच्छ । \newline
22. योनि॑म् गच्छ गच्छ॒ योनिं॒ ॅयोनि॑म् गच्छ॒ स्वाहा॒ स्वाहा॑ गच्छ॒ योनिं॒ ॅयोनि॑म् गच्छ॒ स्वाहा᳚ । \newline
23. ग॒च्छ॒ स्वाहा॒ स्वाहा॑ गच्छ गच्छ॒ स्वाहै॒ष ए॒ष स्वाहा॑ गच्छ गच्छ॒ स्वाहै॒षः । \newline
24. स्वाहै॒ष ए॒ष स्वाहा॒ स्वाहै॒ष ते॑ त ए॒ष स्वाहा॒ स्वाहै॒ष ते᳚ । \newline
25. ए॒ष ते॑ त ए॒ष ए॒ष ते॑ य॒ज्ञो य॒ज्ञ् स्त॑ ए॒ष ए॒ष ते॑ य॒ज्ञ्ः । \newline
26. ते॒ य॒ज्ञो य॒ज्ञ् स्ते॑ ते य॒ज्ञो य॑ज्ञ्पते यज्ञ्पते य॒ज्ञ् स्ते॑ ते य॒ज्ञो य॑ज्ञ्पते । \newline
27. य॒ज्ञो य॑ज्ञ्पते यज्ञ्पते य॒ज्ञो य॒ज्ञो य॑ज्ञ्पते स॒हसू᳚क्तवाकः स॒हसू᳚क्तवाको यज्ञ्पते य॒ज्ञो य॒ज्ञो य॑ज्ञ्पते स॒हसू᳚क्तवाकः । \newline
28. य॒ज्ञ्॒प॒ते॒ स॒हसू᳚क्तवाकः स॒हसू᳚क्तवाको यज्ञ्पते यज्ञ्पते स॒हसू᳚क्तवाकः सु॒वीरः॑ सु॒वीरः॑ स॒हसू᳚क्तवाको यज्ञ्पते यज्ञ्पते स॒हसू᳚क्तवाकः सु॒वीरः॑ । \newline
29. य॒ज्ञ्॒प॒त॒ इति॑ यज्ञ् - प॒ते॒ । \newline
30. स॒हसू᳚क्तवाकः सु॒वीरः॑ सु॒वीरः॑ स॒हसू᳚क्तवाकः स॒हसू᳚क्तवाकः सु॒वीरः॒ स्वाहा॒ स्वाहा॑ सु॒वीरः॑ स॒हसू᳚क्तवाकः स॒हसू᳚क्तवाकः सु॒वीरः॒ स्वाहा᳚ । \newline
31. स॒हसू᳚क्तवाक॒ इति॑ स॒हसू᳚क्त - वा॒कः॒ । \newline
32. सु॒वीरः॒ स्वाहा॒ स्वाहा॑ सु॒वीरः॑ सु॒वीरः॒ स्वाहा॒ देवा॒ देवाः॒ स्वाहा॑ सु॒वीरः॑ सु॒वीरः॒ स्वाहा॒ देवाः᳚ । \newline
33. सु॒वीर॒ इति॑ सु - वीरः॑ । \newline
34. स्वाहा॒ देवा॒ देवाः॒ स्वाहा॒ स्वाहा॒ देवा॑ गातुविदो गातुविदो॒ देवाः॒ स्वाहा॒ स्वाहा॒ देवा॑ गातुविदः । \newline
35. देवा॑ गातुविदो गातुविदो॒ देवा॒ देवा॑ गातुविदो गा॒तुम् गा॒तुम् गा॑तुविदो॒ देवा॒ देवा॑ गातुविदो गा॒तुम् । \newline
36. गा॒तु॒वि॒दो॒ गा॒तुम् गा॒तुम् गा॑तुविदो गातुविदो गा॒तुं ॅवि॒त्त्वा वि॒त्त्वा गा॒तुम् गा॑तुविदो गातुविदो गा॒तुं ॅवि॒त्त्वा । \newline
37. गा॒तु॒वि॒द॒ इति॑ गातु - वि॒दः॒ । \newline
38. गा॒तुं ॅवि॒त्त्वा वि॒त्त्वा गा॒तुम् गा॒तुं ॅवि॒त्त्वा गा॒तुम् गा॒तुं ॅवि॒त्त्वा गा॒तुम् गा॒तुं ॅवि॒त्त्वा गा॒तुम् । \newline
39. वि॒त्त्वा गा॒तुम् गा॒तुं ॅवि॒त्त्वा वि॒त्त्वा गा॒तु मि॑ते त गा॒तुं ॅवि॒त्त्वा वि॒त्त्वा गा॒तु मि॑त । \newline
40. गा॒तु मि॑ते त गा॒तुम् गा॒तु मि॑त॒ मन॑सो॒ मन॑स इत गा॒तुम् गा॒तु मि॑त॒ मन॑सः । \newline
41. इ॒त॒ मन॑सो॒ मन॑स इते त॒ मन॑स स्पते पते॒ मन॑स इते त॒ मन॑स स्पते । \newline
42. मन॑स स्पते पते॒ मन॑सो॒ मन॑स स्पत इ॒म मि॒मम् प॑ते॒ मन॑सो॒ मन॑स स्पत इ॒मम् । \newline
43. प॒त॒ इ॒म मि॒मम् प॑ते पत इ॒मम् नो॑ न इ॒मम् प॑ते पत इ॒मम् नः॑ । \newline
44. इ॒मम् नो॑ न इ॒म मि॒मम् नो॑ देव देव न इ॒म मि॒मम् नो॑ देव । \newline
45. नो॒ दे॒व॒ दे॒व॒ नो॒ नो॒ दे॒व॒ दे॒वेषु॑ दे॒वेषु॑ देव नो नो देव दे॒वेषु॑ । \newline
46. दे॒व॒ दे॒वेषु॑ दे॒वेषु॑ देव देव दे॒वेषु॑ य॒ज्ञ्ं ॅय॒ज्ञ्म् दे॒वेषु॑ देव देव दे॒वेषु॑ य॒ज्ञ्म् । \newline
47. दे॒वेषु॑ य॒ज्ञ्ं ॅय॒ज्ञ्म् दे॒वेषु॑ दे॒वेषु॑ य॒ज्ञ्ꣳ स्वाहा॒ स्वाहा॑ य॒ज्ञ्म् दे॒वेषु॑ दे॒वेषु॑ य॒ज्ञ्ꣳ स्वाहा᳚ । \newline
48. य॒ज्ञ्ꣳ स्वाहा॒ स्वाहा॑ य॒ज्ञ्ं ॅय॒ज्ञ्ꣳ स्वाहा॑ वा॒चि वा॒चि स्वाहा॑ य॒ज्ञ्ं ॅय॒ज्ञ्ꣳ स्वाहा॑ वा॒चि । \newline
49. स्वाहा॑ वा॒चि वा॒चि स्वाहा॒ स्वाहा॑ वा॒चि स्वाहा॒ स्वाहा॑ वा॒चि स्वाहा॒ स्वाहा॑ वा॒चि स्वाहा᳚ । \newline
50. वा॒चि स्वाहा॒ स्वाहा॑ वा॒चि वा॒चि स्वाहा॒ वाते॒ वाते॒ स्वाहा॑ वा॒चि वा॒चि स्वाहा॒ वाते᳚ । \newline
51. स्वाहा॒ वाते॒ वाते॒ स्वाहा॒ स्वाहा॒ वाते॑ धा धा॒ वाते॒ स्वाहा॒ स्वाहा॒ वाते॑ धाः । \newline
52. वाते॑ धा धा॒ वाते॒ वाते॑ धाः । \newline
53. धा॒ इति॑ धाः । \newline
\pagebreak
\markright{ TS 1.4.45.1  \hfill https://www.vedavms.in \hfill}
\addcontentsline{toc}{section}{ TS 1.4.45.1 }
\section*{ TS 1.4.45.1 }

\textbf{TS 1.4.45.1 } \newline
\textbf{Samhita Paata} \newline

उ॒रुꣳ हि राजा॒ वरु॑णश्च॒कार॒ सूर्या॑य॒ पन्था॒-मन्वे॑त॒वा उ॑ । अ॒पदे॒ पादा॒ प्रति॑धातवे-ऽकरु॒ता-ऽप॑व॒क्ता हृ॑दया॒विध॑श्चित् ॥ श॒तं ते॑ राजन् भि॒षजः॑ स॒हस्र॑मु॒र्वी गं॑भी॒रा सु॑म॒तिष्टे॑ अस्तु । बाध॑स्व॒ द्वेषो॒ निर्.ऋ॑तिं परा॒चैः कृ॒तं चि॒देनः॒ प्र मु॑मुग्द्ध्य॒स्मत् ॥ अ॒भिष्ठि॑तो॒ वरु॑णस्य॒ पाशो॒ऽग्नेरनी॑कम॒प आ वि॑वेश । अपा᳚न्नपात् प्रति॒रक्ष॑न्नसु॒र्यं॑ दमे॑दमे - [ ] \newline

\textbf{Pada Paata} \newline

उ॒रुम् । हि । राजा᳚ । वरु॑णः । च॒कार॑ । सूर्या॑य । पन्था᳚म् । अन्वे॑त॒वा इत्यनु॑ - ए॒त॒वै । उ॒ ॥ अ॒पदे᳚ । पादा᳚ । प्रति॑धातव॒ इति॒ प्रति॑-धा॒त॒वे॒ । अ॒कः॒ । उ॒त । अ॒प॒व॒क्तेत्य॑प - व॒क्ता । हृ॒द॒या॒विध॒ इति॑ हृदय - विधः॑ । चि॒त् ॥ श॒तम् । ते॒ । रा॒ज॒न्न् । भि॒षजः॑ । स॒हस्र᳚म् । उ॒र्वी । ग॒भीं॒रा । सु॒म॒तिरिति॑ सु - म॒तिः । ते॒ । अ॒स्तु॒ ॥ बाध॑स्व । द्वेषः॑ । निर्.ऋ॑ति॒मिति॒ निः - ऋ॒ति॒म् । प॒रा॒चैः । कृ॒तम् । चि॒त् । एनः॑ । प्रेति॑ । मु॒मु॒ग्धि॒ । अ॒स्मत् ॥ अ॒भिष्ठि॑त॒ इत्य॒भि - स्थि॒तः॒ । वरु॑णस्य । पाशः॑ । अ॒ग्नेः । अनी॑कम् । अ॒पः । एति॑ । वि॒वे॒श॒ ॥ अपा᳚म् । न॒पा॒त् । प्र॒ति॒रक्ष॒न्निति॑ प्रति - रक्षन्न्॑ । अ॒सु॒र्य᳚म् । दमे॑दम॒ इति॒ दमे᳚ - द॒मे॒ ।  \newline


\textbf{Krama Paata} \newline

उ॒रुꣳ हि । हि राजा᳚ । राजा॒ वरु॑णः । वरु॑णश्च॒कार॑ । च॒कार॒ सूर्या॑य । सूर्या॑य॒ पन्था᳚म् । पन्था॒मन्वे॑त॒वै । अन्वे॑त॒वा उ॑ । अन्वे॑त॒वा इत्यनु॑ - ए॒त॒वै । उ॒वित्यु॑ ॥ अ॒पदे॒ पादा᳚ । पादा॒ प्रति॑धातवे । प्रति॑धातवेऽकः । प्रति॑धातव॒ इति॒ प्रति॑ - धा॒त॒वे॒ । अ॒क॒रु॒त । उ॒ताप॑व॒क्ता । अ॒प॒व॒क्ता हृ॑दया॒विधः॑ । अ॒प॒व॒क्तेत्य॑प - व॒क्ता । हृ॒द॒या॒विध॑श्चित् । हृ॒द॒या॒विध॒ इति॑ हृदय - विधः॑ । चि॒दिति॑ चित् ॥ श॒तम् ते᳚ । ते॒ रा॒ज॒न्न्॒ । रा॒ज॒न् भि॒षजः॑ । भि॒षजः॑ स॒हस्र᳚म् । स॒हस्र॑मु॒र्वी । उ॒र्वी ग॑म्भी॒रा । ग॒म्भी॒रा सु॑म॒तिः । सु॒म॒तिष्टे᳚ । सु॒म॒तिरिति॑ सु - म॒तिः । ते॒ अ॒स्तु॒ । अ॒स्त्वित्य॑स्तु ॥ बाध॑स्व॒ द्वेषः॑ । द्वेषो॒ निर्.ऋ॑तिम् । निर्.ऋ॑तिम् परा॒चैः । निर्.ऋ॑ति॒मिति॒ निः - ऋ॒ति॒म् । प॒रा॒चैः कृ॒तम् । कृ॒तम् चि॑त् । चि॒देनः॑ । एनः॒ प्र । प्र मु॑मुग्धि । मु॒मु॒ग्ध्य॒स्मत् । अ॒स्मदित्य॒स्मत् ॥ अ॒भिष्ठि॑तो॒ वरु॑णस्य । अ॒भिष्ठि॑त॒ इत्य॒भि - स्थि॒तः॒ । वरु॑णस्य॒ पाशः॑ । पाशो॒ऽग्नेः । अ॒ग्नेरनी॑कम् । अनी॑कम॒पः । अ॒प आ । आ वि॑वेश । वि॒वे॒शेति॑ विवेश ॥ अपा᳚म् नपात् । न॒पा॒त् प्र॒ति॒रक्षन्न्॑ । प्र॒ति॒रक्ष॑न्नसु॒र्य᳚म् । प्र॒ति॒रक्ष॒न्निति॑ प्रति - रक्षन्न्॑ । अ॒सु॒र्य॑म् दमे॑दमे । दमे॑दमे स॒मिध᳚म् । दमे॑दम॒ इति॒ दमे᳚ - द॒मे॒ \newline

\textbf{Jatai Paata} \newline

1. उ॒रुꣳ हि ह्यु॑रु मु॒रुꣳ हि । \newline
2. हि राजा॒ राजा॒ हि हि राजा᳚ । \newline
3. राजा॒ वरु॑णो॒ वरु॑णो॒ राजा॒ राजा॒ वरु॑णः । \newline
4. वरु॑ण श्च॒कार॑ च॒कार॒ वरु॑णो॒ वरु॑ण श्च॒कार॑ । \newline
5. च॒कार॒ सूर्या॑य॒ सूर्या॑य च॒कार॑ च॒कार॒ सूर्या॑य । \newline
6. सूर्या॑य॒ पन्था॒म् पन्था॒(ग्म्॒) सूर्या॑य॒ सूर्या॑य॒ पन्था᳚म् । \newline
7. पन्था॒ मन्वे॑त॒वा अन्वे॑त॒वै पन्था॒म् पन्था॒ मन्वे॑त॒वै । \newline
8. अन्वे॑त॒वा उ॑ वु॒ वन्वे॑त॒वा अन्वे॑त॒वा उ॑ । \newline
9. अन्वे॑त॒वा इत्यनु॑ - ए॒त॒वै । \newline
10. उ॒वित्यु॑ । \newline
11. अ॒पदे॒ पादा॒ पादा॒ ऽपदे॒ ऽपदे॒ पादा᳚ । \newline
12. पादा॒ प्रति॑धातवे॒ प्रति॑धातवे॒ पादा॒ पादा॒ प्रति॑धातवे । \newline
13. प्रति॑धातवे ऽक रकः॒ प्रति॑धातवे॒ प्रति॑धातवे ऽकः । \newline
14. प्रति॑धातव॒ इति॒ प्रति॑ - धा॒त॒वे॒ । \newline
15. अ॒क॒ रु॒तोताक॑ रक रु॒त । \newline
16. उ॒ताप॑व॒क्ता ऽप॑व॒क्तो तोताप॑व॒क्ता । \newline
17. अ॒प॒व॒क्ता हृ॑दया॒विधो॑ हृदया॒विधो॑ ऽपव॒क्ता ऽप॑व॒क्ता हृ॑दया॒विधः॑ । \newline
18. अ॒प॒व॒क्तेत्य॑प - व॒क्ता । \newline
19. हृ॒द॒या॒विध॑श्चिच् चिद्धृदया॒विधो॑ हृदया॒विध॑श्चित् । \newline
20. हृ॒द॒या॒विध॒ इति॑ हृदय - विधः॑ । \newline
21. चि॒दिति॑ चित् । \newline
22. श॒तम् ते॑ ते श॒तꣳ श॒तम् ते᳚ । \newline
23. ते॒ रा॒ज॒न् रा॒ज॒न् ते॒ ते॒ रा॒ज॒न्न् । \newline
24. रा॒ज॒न् भि॒षजो॑ भि॒षजो॑ राजन् राजन् भि॒षजः॑ । \newline
25. भि॒षजः॑ स॒हस्र(ग्म्॑) स॒हस्र॑म् भि॒षजो॑ भि॒षजः॑ स॒हस्र᳚म् । \newline
26. स॒हस्र॑ मु॒र्व्यु॑र्वी स॒हस्र(ग्म्॑) स॒हस्र॑ मु॒र्वी । \newline
27. उ॒र्वी गं॑भी॒रा गं॑भी॒रोर्व्यु॑र्वी गं॑भी॒रा । \newline
28. गं॒भी॒रा सु॑म॒तिः सु॑म॒तिर् गं॑भी॒रा गं॑भी॒रा सु॑म॒तिः । \newline
29. सु॒म॒तिष्टे॑ ते सुम॒तिः सु॑म॒तिष्टे᳚ । \newline
30. सु॒म॒तिरिति॑ सु - म॒तिः । \newline
31. ते॒ अ॒स्त्व॒स्तु॒ ते॒ ते॒ अ॒स्तु॒ । \newline
32. अ॒स्त्वित्य॑स्तु । \newline
33. बाध॑स्व॒ द्वेषो॒ द्वेषो॒ बाध॑स्व॒ बाध॑स्व॒ द्वेषः॑ । \newline
34. द्वेषो॒ निर्.ऋ॑ति॒ न्निर्.ऋ॑ति॒म् द्वेषो॒ द्वेषो॒ निर्.ऋ॑तिम् । \newline
35. निर्.ऋ॑तिम् परा॒चैः प॑रा॒चैर् निर्.ऋ॑ति॒ न्निर्.ऋ॑तिम् परा॒चैः । \newline
36. निर्.ऋ॑ति॒मिति॒ निः - ऋ॒ति॒म् । \newline
37. प॒रा॒चैः कृ॒तम् कृ॒तम् प॑रा॒चैः प॑रा॒चैः कृ॒तम् । \newline
38. कृ॒तम् चि॑च् चित् कृ॒तम् कृ॒तम् चि॑त् । \newline
39. चि॒देन॒ एन॑श्चिच् चि॒देनः॑ । \newline
40. एनः॒ प्र प्रैन॒ एनः॒ प्र । \newline
41. प्र मु॑मुग्धि मुमुग्धि॒ प्र प्र मु॑मुग्धि । \newline
42. मु॒मु॒ग्ध्य॒स्म द॒स्मन् मु॑मुग्धि मुमुग्ध्य॒स्मत् । \newline
43. अ॒स्मदित्य॒स्मत् । \newline
44. अ॒भिष्ठि॑तो॒ वरु॑णस्य॒ वरु॑णस्या॒भिष्ठि॑तो॒ ऽभिष्ठि॑तो॒ वरु॑णस्य । \newline
45. अ॒भिष्ठि॑त॒ इत्य॒भि - स्थि॒तः॒ । \newline
46. वरु॑णस्य॒ पाशः॒ पाशो॒ वरु॑णस्य॒ वरु॑णस्य॒ पाशः॑ । \newline
47. पाशो॒ ऽग्नेर॒ग्नेः पाशः॒ पाशो॒ ऽग्नेः । \newline
48. अ॒ग्ने रनी॑क॒ मनी॑क म॒ग्ने र॒ग्ने रनी॑कम् । \newline
49. अनी॑क म॒पो॑ ऽपो ऽनी॑क॒ मनी॑क म॒पः । \newline
50. अ॒प आ ऽपो॑ ऽप आ । \newline
51. आ वि॑वेश विवे॒शा वि॑वेश । \newline
52. वि॒वे॒शेति॑ विवेश । \newline
53. अपा᳚न्नपान् नपा॒दपा॒ मपा᳚न्नपात् । \newline
54. न॒पा॒त् प्र॒ति॒रक्ष॑न् प्रति॒रक्ष॑न् नपान् नपात् प्रति॒रक्षन्न्॑ । \newline
55. प्र॒ति॒रक्ष॑न् नसु॒र्य॑ मसु॒र्य॑म् प्रति॒रक्ष॑न् प्रति॒रक्ष॑न् नसु॒र्य᳚म् । \newline
56. प्र॒ति॒रक्ष॒न्निति॑ प्रति - रक्षन्न्॑ । \newline
57. अ॒सु॒र्य॑म् दमे॑दमे॒ दमे॑दमे ऽसु॒र्य॑ मसु॒र्य॑म् दमे॑दमे । \newline
58. दमे॑दमे स॒मिध(ग्म्॑) स॒मिध॒म् दमे॑दमे॒ दमे॑दमे स॒मिध᳚म् । \newline
59. दमे॑दम॒ इति॒ दमे᳚ - द॒मे॒ । \newline

\textbf{Ghana Paata } \newline

1. उ॒रुꣳ हि ह्यु॑रु मु॒रुꣳ हि राजा॒ राजा॒ ह्यु॑रु मु॒रुꣳ हि राजा᳚ । \newline
2. हि राजा॒ राजा॒ हि हि राजा॒ वरु॑णो॒ वरु॑णो॒ राजा॒ हि हि राजा॒ वरु॑णः । \newline
3. राजा॒ वरु॑णो॒ वरु॑णो॒ राजा॒ राजा॒ वरु॑ण श्च॒कार॑ च॒कार॒ वरु॑णो॒ राजा॒ राजा॒ वरु॑ण श्च॒कार॑ । \newline
4. वरु॑ण श्च॒कार॑ च॒कार॒ वरु॑णो॒ वरु॑ण श्च॒कार॒ सूर्या॑य॒ सूर्या॑य च॒कार॒ वरु॑णो॒ वरु॑ण श्च॒कार॒ सूर्या॑य । \newline
5. च॒कार॒ सूर्या॑य॒ सूर्या॑य च॒कार॑ च॒कार॒ सूर्या॑य॒ पन्था॒म् पन्था॒(ग्म्॒) सूर्या॑य च॒कार॑ च॒कार॒ सूर्या॑य॒ पन्था᳚म् । \newline
6. सूर्या॑य॒ पन्था॒म् पन्था॒(ग्म्॒) सूर्या॑य॒ सूर्या॑य॒ पन्था॒ मन्वे॑त॒वा अन्वे॑त॒वै पन्था॒(ग्म्॒) सूर्या॑य॒ सूर्या॑य॒ पन्था॒ मन्वे॑त॒वै । \newline
7. पन्था॒ मन्वे॑त॒वा अन्वे॑त॒वै पन्था॒म् पन्था॒ मन्वे॑त॒वा उ॑ वु॒ वन्वे॑तवै॒ पन्था॒म् पन्था॒ मन्वे॑त॒वा उ॑ । \newline
8. अन्वे॑त॒वा उ॑ वु॒ वन्वे॑त॒वा अन्वे॑त॒वा उ॑ । \newline
9. अन्वे॑त॒वा इत्यनु॑ - ए॒त॒वै । \newline
10. उ॒वित्यु॑ । \newline
11. अ॒पदे॒ पादा॒ पादा॒ ऽपदे॒ ऽपदे॒ पादा॒ प्रति॑धातवे॒ प्रति॑धातवे॒ पादा॒ ऽपदे॒ ऽपदे॒ पादा॒ प्रति॑धातवे । \newline
12. पादा॒ प्रति॑धातवे॒ प्रति॑धातवे॒ पादा॒ पादा॒ प्रति॑धातवे ऽक रकः॒ प्रति॑धातवे॒ पादा॒ पादा॒ प्रति॑धातवे ऽकः । \newline
13. प्रति॑धातवे ऽक रकः॒ प्रति॑धातवे॒ प्रति॑धातवे ऽक रु॒तोताकः॒ प्रति॑धातवे॒ प्रति॑धातवे ऽक रु॒त । \newline
14. प्रति॑धातव॒ इति॒ प्रति॑ - धा॒त॒वे॒ । \newline
15. अ॒क॒ रु॒तोताक॑ रक रु॒ताप॑व॒क्ता ऽप॑व॒क्तोताक॑ रक रु॒ताप॑व॒क्ता । \newline
16. उ॒ताप॑व॒क्ता ऽप॑व॒क्तोतोताप॑व॒क्ता हृ॑दया॒विधो॑ हृदया॒विधो॑ ऽपव॒क्तोतोताप॑व॒क्ता हृ॑दया॒विधः॑ । \newline
17. अ॒प॒व॒क्ता हृ॑दया॒विधो॑ हृदया॒विधो॑ ऽपव॒क्ता ऽप॑व॒क्ता हृ॑दया॒विध॑ श्चिच् चिद्धृदया॒विधो॑ ऽपव॒क्ता ऽप॑व॒क्ता हृ॑दया॒विध॑ श्चित् । \newline
18. अ॒प॒व॒क्तेत्य॑प - व॒क्ता । \newline
19. हृ॒द॒या॒विध॑ श्चिच् चिद्धृदया॒विधो॑ हृदया॒विध॑ श्चित् । \newline
20. हृ॒द॒या॒विध॒ इति॑ हृदय - विधः॑ । \newline
21. चि॒दिति॑ चित् । \newline
22. श॒तम् ते॑ ते श॒तꣳ श॒तम् ते॑ राजन् राजन् ते श॒तꣳ श॒तम् ते॑ राजन्न् । \newline
23. ते॒ रा॒ज॒न् रा॒ज॒न् ते॒ ते॒ रा॒ज॒न् भि॒षजो॑ भि॒षजो॑ राजन् ते ते राजन् भि॒षजः॑ । \newline
24. रा॒ज॒न् भि॒षजो॑ भि॒षजो॑ राजन् राजन् भि॒षजः॑ स॒हस्र(ग्म्॑) स॒हस्र॑म् भि॒षजो॑ राजन् राजन् भि॒षजः॑ स॒हस्र᳚म् । \newline
25. भि॒षजः॑ स॒हस्र(ग्म्॑) स॒हस्र॑म् भि॒षजो॑ भि॒षजः॑ स॒हस्र॑ मु॒र्व्यु॑र्वी स॒हस्र॑म् भि॒षजो॑ भि॒षजः॑ स॒हस्र॑ मु॒र्वी । \newline
26. स॒हस्र॑ मु॒र्व्यु॑र्वी स॒हस्र(ग्म्॑) स॒हस्र॑ मु॒र्वी गं॑भी॒रा गं॑भी॒रोर्वी स॒हस्र(ग्म्॑) स॒हस्र॑ मु॒र्वी गं॑भी॒रा । \newline
27. उ॒र्वी गं॑भी॒रा गं॑भी॒ रोर्व्यु॑र्वी गं॑भी॒रा सु॑म॒तिः सु॑म॒तिर् गं॑भी॒ रोर्व्यु॑र्वी गं॑भी॒रा सु॑म॒तिः । \newline
28. गं॒भी॒रा सु॑म॒तिः सु॑म॒तिर् गं॑भी॒रा गं॑भी॒रा सु॑म॒तिष्टे॑ ते सुम॒तिर् गं॑भी॒रा गं॑भी॒रा सु॑म॒तिष्टे᳚ । \newline
29. सु॒म॒तिष्टे॑ ते सुम॒तिः सु॑म॒तिष्टे॑ अस्त्वस्तु ते सुम॒तिः सु॑म॒तिष्टे॑ अस्तु । \newline
30. सु॒म॒तिरिति॑ सु - म॒तिः । \newline
31. ते॒ अ॒स्त्व॒स्तु॒ ते॒ ते॒ अ॒स्तु॒ । \newline
32. अ॒स्त्वित्य॑स्तु । \newline
33. बाध॑स्व॒ द्वेषो॒ द्वेषो॒ बाध॑स्व॒ बाध॑स्व॒ द्वेषो॒ निर्.ऋ॑ति॒म् निर्.ऋ॑ति॒म् द्वेषो॒ बाध॑स्व॒ बाध॑स्व॒ द्वेषो॒ निर्.ऋ॑तिम् । \newline
34. द्वेषो॒ निर्.ऋ॑ति॒म् निर्.ऋ॑ति॒म् द्वेषो॒ द्वेषो॒ निर्.ऋ॑तिम् परा॒चैः प॑रा॒चैर् निर्.ऋ॑ति॒म् द्वेषो॒ द्वेषो॒ निर्.ऋ॑तिम् परा॒चैः । \newline
35. निर्.ऋ॑तिम् परा॒चैः प॑रा॒चैर् निर्.ऋ॑ति॒म् निर्.ऋ॑तिम् परा॒चैः कृ॒तम् कृ॒तम् प॑रा॒चैर् निर्.ऋ॑ति॒म् निर्.ऋ॑तिम् परा॒चैः कृ॒तम् । \newline
36. निर्.ऋ॑ति॒मिति॒ निः - ऋ॒ति॒म् । \newline
37. प॒रा॒चैः कृ॒तम् कृ॒तम् प॑रा॒चैः प॑रा॒चैः कृ॒तम् चि॑च् चित् कृ॒तम् प॑रा॒चैः प॑रा॒चैः कृ॒तम् चि॑त् । \newline
38. कृ॒तम् चि॑च् चित् कृ॒तम् कृ॒तम् चि॒देन॒ एन॑श्चित् कृ॒तम् कृ॒तम् चि॒देनः॑ । \newline
39. चि॒देन॒ एन॑ श्चिच् चि॒देनः॒ प्र प्रैन॑ श्चिच् चि॒देनः॒ प्र । \newline
40. एनः॒ प्र प्रैन॒ एनः॒ प्र मु॑मुग्धि मुमुग्धि॒ प्रैन॒ एनः॒ प्र मु॑मुग्धि । \newline
41. प्र मु॑मुग्धि मुमुग्धि॒ प्र प्र मु॑मुग् ध्य॒स्म द॒स्मन् मु॑मुग्धि॒ प्र प्र मु॑मुग्ध्य॒स्मत् । \newline
42. मु॒मु॒ग् ध्य॒स्म द॒स्मन् मु॑मुग्धि मुमुग् ध्य॒स्मत् । \newline
43. अ॒स्मदित्य॒स्मत् । \newline
44. अ॒भिष्ठि॑तो॒ वरु॑णस्य॒ वरु॑ण स्या॒भिष्ठि॑तो॒ ऽभिष्ठि॑तो॒ वरु॑णस्य॒ पाशः॒ पाशो॒ वरु॑ण स्या॒भिष्ठि॑तो॒ ऽभिष्ठि॑तो॒ वरु॑णस्य॒ पाशः॑ । \newline
45. अ॒भिष्ठि॑त॒ इत्य॒भि - स्थि॒तः॒ । \newline
46. वरु॑णस्य॒ पाशः॒ पाशो॒ वरु॑णस्य॒ वरु॑णस्य॒ पाशो॒ ऽग्नेर॒ग्नेः पाशो॒ वरु॑णस्य॒ वरु॑णस्य॒ पाशो॒ ऽग्नेः । \newline
47. पाशो॒ ऽग्नेर॒ग्नेः पाशः॒ पाशो॒ ऽग्नेरनी॑क॒ मनी॑क म॒ग्नेः पाशः॒ पाशो॒ ऽग्नेरनी॑कम् । \newline
48. अ॒ग्ने रनी॑क॒ मनी॑क म॒ग्ने र॒ग्ने रनी॑क म॒पो॑ ऽपो ऽनी॑क म॒ग्ने र॒ग्ने रनी॑क म॒पः । \newline
49. अनी॑क म॒पो॑ ऽपो ऽनी॑क॒ मनी॑क म॒प आ ऽपो ऽनी॑क॒ मनी॑क म॒प आ । \newline
50. अ॒प आ ऽपो॑ ऽप आ वि॑वेश विवे॒शा ऽपो॑ ऽप आ वि॑वेश । \newline
51. आ वि॑वेश विवे॒शा वि॑वेश । \newline
52. वि॒वे॒शेति॑ विवेश । \newline
53. अपा᳚म् नपान् नपा॒दपा॒ मपा᳚म् नपात् प्रति॒रक्ष॑न् प्रति॒रक्ष॑न् नपा॒दपा॒ मपा᳚म् नपात् प्रति॒रक्षन्न्॑ । \newline
54. न॒पा॒त् प्र॒ति॒रक्ष॑न् प्रति॒रक्ष॑न् नपान् नपात् प्रति॒रक्ष॑न् नसु॒र्य॑ मसु॒र्य॑म् प्रति॒रक्ष॑न् नपान् नपात् प्रति॒रक्ष॑न् नसु॒र्य᳚म् । \newline
55. प्र॒ति॒रक्ष॑न् नसु॒र्य॑ मसु॒र्य॑म् प्रति॒रक्ष॑न् प्रति॒रक्ष॑न् नसु॒र्य॑म् दमे॑दमे॒ दमे॑दमे ऽसु॒र्य॑म् प्रति॒रक्ष॑न् प्रति॒रक्ष॑न् नसु॒र्य॑म् दमे॑दमे । \newline
56. प्र॒ति॒रक्ष॒न्निति॑ प्रति - रक्षन्न्॑ । \newline
57. अ॒सु॒र्य॑म् दमे॑दमे॒ दमे॑दमे ऽसु॒र्य॑ मसु॒र्य॑म् दमे॑दमे स॒मिध(ग्म्॑) स॒मिध॒म् दमे॑दमे ऽसु॒र्य॑ मसु॒र्य॑म् दमे॑दमे स॒मिध᳚म् । \newline
58. दमे॑दमे स॒मिध(ग्म्॑) स॒मिध॒म् दमे॑दमे॒ दमे॑दमे स॒मिधं॑ ॅयक्षि यक्षि स॒मिध॒म् दमे॑दमे॒ दमे॑दमे स॒मिधं॑ ॅयक्षि । \newline
59. दमे॑दम॒ इति॒ दमे᳚ - द॒मे॒ । \newline
\pagebreak
\markright{ TS 1.4.45.2  \hfill https://www.vedavms.in \hfill}
\addcontentsline{toc}{section}{ TS 1.4.45.2 }
\section*{ TS 1.4.45.2 }

\textbf{TS 1.4.45.2 } \newline
\textbf{Samhita Paata} \newline

स॒मिधं॑ ॅयक्ष्यग्ने ॥ प्रति॑ ते जि॒ह्वा घृ॒तमुच्च॑रण्येथ् समु॒द्रे ते॒ हृद॑यम॒फ्स्व॑न्तः । सं त्वा॑ विश॒न्त्वोष॑धी-रु॒ताऽऽपो॑ य॒ज्ञ्स्य॑ त्वा यज्ञ्पते ह॒विर्भिः॑ ॥ सू॒क्त॒वा॒के न॑मोवा॒के वि॑धे॒माऽव॑भृथ निचङ्कुण निचे॒रुर॑सि निचङ्कु॒णाऽव॑ दे॒वैर् दे॒वकृ॑त॒मेनो॑ऽया॒डव॒ मर्त्यै॒र् मर्त्य॑कृतमु॒रोरा नो॑ देव रि॒षस्पा॑हि सुमि॒त्रा न॒ आप॒ ओष॑धयः - [ ] \newline

\textbf{Pada Paata} \newline

स॒मिध॒मिति॑ सं - इध᳚म् । य॒क्षि॒ । अ॒ग्ने॒ ॥ प्रतीति॑ । ते॒ । जि॒ह्वा । घृ॒तम् । उदिति॑ । च॒र॒ण्ये॒त् । स॒मु॒द्रे । ते॒ । हृद॑यम् । अ॒फ्स्वित्य॑प् - सु । अ॒न्तः ॥ समिति॑ । त्वा॒ । वि॒श॒न्तु॒ । ओष॑धीः । उ॒त । आपः॑ । य॒ज्ञ्स्य॑ । त्वा॒ । य॒ज्ञ्॒प॒त॒ इति॑ यज्ञ् - प॒ते॒ । ह॒विर्भि॒रिति॑ ह॒विः - भिः॒ ॥ सू॒क्त॒वा॒क इति॑ सूक्त - वा॒के । न॒मो॒वा॒क इति॑ नमः - वा॒के । वि॒धे॒म॒ । अव॑भृ॒थेत्यव॑ - भृ॒थ॒ । नि॒च॒ङ्कु॒णेति॑ नि - च॒ङ्कु॒ण॒ । नि॒चे॒रुरिति॑ नि - चे॒रुः । अ॒सि॒ । नि॒च॒ङ्कु॒णेति॑ नि - च॒ङ्कु॒ण॒ । अवेति॑ । दे॒वैः । दे॒वकृ॑त॒मिति॑ दे॒व - कृ॒त॒म् । एनः॑ । अ॒या॒ट् । अवेति॑ । मर्त्यैः᳚ । मर्त्य॑कृत॒मिति॒ मर्त्य॑ - कृ॒त॒म् । उ॒रोः । एति॑ । नः॒ । दे॒व॒ । रि॒षः । पा॒हि॒ । सु॒मि॒त्रा इति॑ सु - मि॒त्राः । नः॒ । आपः॑ । ओष॑धयः ।  \newline


\textbf{Krama Paata} \newline

स॒मिधं॑ ॅयक्षि । स॒मिध॒मिति॑ सम् - इध᳚म् । य॒क्ष्य॒ग्ने॒ । अ॒ग्न॒ इत्य॑ग्ने ॥ प्रति॑ ते । ते॒ जि॒ह्वा । जि॒ह्वा घृ॒तम् । घृ॒तमुत् । उच्च॑रण्येत् । च॒र॒ण्ये॒थ् स॒मु॒द्रे । स॒मु॒द्रे ते᳚ । ते॒ हृद॑यम् । हृद॑यम॒फ्सु । अ॒फ्स्व॑न्तः । अ॒फ्स्वित्य॑प् - सु । अ॒न्तरित्य॒न्तः ॥ सम् त्वा᳚ । त्वा॒ वि॒श॒न्तु॒ । वि॒श॒न्त्वोष॑धीः । ओष॑धीरु॒त । उ॒तापः॑ । आपो॑ य॒ज्ञ्स्य॑ । य॒ज्ञ्स्य॑ त्वा । त्वा॒ य॒ज्ञ्॒प॒ते॒ । य॒ज्ञ्॒प॒ते॒ ह॒विर्भिः॑ । य॒ज्ञ्॒प॒त॒ इति॑ यज्ञ् - प॒ते॒ । ह॒विर्भि॒रिति॑ ह॒विः - भिः॒ ॥ सू॒क्त॒वा॒के न॑मोवा॒के । सू॒क्त॒वा॒क इति॑ सूक्त - वा॒के । न॒मो॒वा॒के वि॑धेम । न॒मो॒वा॒क इति॑ नमः - वा॒के । वि॒धे॒माव॑भृथ । अव॑भृथ निचङ्कुण । अव॑भृ॒थेत्यव॑ - भृ॒थ॒ । नि॒च॒ङ्कु॒ण॒ नि॒चे॒रुः । नि॒च॒ङ्कु॒णेति॑ नि - च॒ङ्कु॒ण॒ । नि॒चे॒रुर॑सि । नि॒चे॒रुरिति॑ नि - चे॒रुः । अ॒सि॒ नि॒च॒ङ्कु॒ण॒ । नि॒च॒ङ्कु॒णाव॑ । नि॒च॒ङ्कु॒णेति॑ नि - च॒ङ्कु॒ण॒ । अव॑ दे॒वैः । दे॒वैर् दे॒वकृ॑तम् । दे॒वकृ॑त॒मेनः॑ । दे॒वकृ॑त॒मिति॑ दे॒व - कृ॒त॒म् । एनो॑ऽयाट् । अ॒या॒डव॑ । अव॒ मर्त्यैः᳚ । मर्त्यै॒र् मर्त्य॑कृतम् । मर्त्य॑कृतमु॒रोः । मर्त्य॑कृत॒मिति॒ मर्त्य॑ - कृ॒त॒म् । उ॒रोरा । आ नः॑ । नो॒ दे॒व॒ । दे॒व॒ रि॒षः । रि॒षस्पा॑हि । पा॒हि॒ सु॒मि॒त्राः । सु॒मि॒त्रा नः॑ । सु॒मि॒त्रा इति॑ सु - मि॒त्राः । न॒ आपः॑ । आप॒ ओष॑धयः । ओष॑धयः सन्तु \newline

\textbf{Jatai Paata} \newline

1. स॒मिधं॑ ॅयक्षि यक्षि स॒मिध(ग्म्॑) स॒मिधं॑ ॅयक्षि । \newline
2. स॒मिध॒मिति॑ सं - इध᳚म् । \newline
3. य॒क्ष्य॒ग्ने॒ ऽग्ने॒ य॒क्षि॒ य॒क्ष्य॒ग्ने॒ । \newline
4. अ॒ग्न॒ इत्य॑ग्ने । \newline
5. प्रति॑ ते ते॒ प्रति॒ प्रति॑ ते । \newline
6. ते॒ जि॒ह्वा जि॒ह्वा ते॑ ते जि॒ह्वा । \newline
7. जि॒ह्वा घृ॒तम् घृ॒तम् जि॒ह्वा जि॒ह्वा घृ॒तम् । \newline
8. घृ॒त मुदुद् घृ॒तम् घृ॒त मुत् । \newline
9. उच् च॑रण्येच् चरण्ये॒दुदुच् च॑रण्येत् । \newline
10. च॒र॒ण्ये॒थ् स॒मु॒द्रे स॑मु॒द्रे च॑रण्येच् चरण्येथ् समु॒द्रे । \newline
11. स॒मु॒द्रे ते॑ ते समु॒द्रे स॑मु॒द्रे ते᳚ । \newline
12. ते॒ हृद॑य॒(ग्म्॒) हृद॑यम् ते ते॒ हृद॑यम् । \newline
13. हृद॑य म॒फ्स्व॑फ्सु हृद॑य॒(ग्म्॒) हृद॑य म॒फ्सु । \newline
14. अ॒फ्स्व॑न्त र॒न्त र॒फ्स्वा᳚(1॒)फ्स्व॑न्तः । \newline
15. अ॒फ्स्वित्य॑प् - सु । \newline
16. अ॒न्तरित्य॒न्तः । \newline
17. सम् त्वा᳚ त्वा॒ सꣳ सम् त्वा᳚ । \newline
18. त्वा॒ वि॒श॒न्तु॒ वि॒श॒न्तु॒ त्वा॒ त्वा॒ वि॒श॒न्तु॒ । \newline
19. वि॒श॒न्त्वोष॑धी॒रोष॑धीर् विशन्तु विश॒न्त्वोष॑धीः । \newline
20. ओष॑धी रु॒तो तौष॑धी॒ रोष॑धीरु॒त । \newline
21. उ॒ताप॒ आप॑ उ॒तोतापः॑ । \newline
22. आपो॑ य॒ज्ञ्स्य॑ य॒ज्ञ्स्याप॒ आपो॑ य॒ज्ञ्स्य॑ । \newline
23. य॒ज्ञ्स्य॑ त्वा त्वा य॒ज्ञ्स्य॑ य॒ज्ञ्स्य॑ त्वा । \newline
24. त्वा॒ य॒ज्ञ्॒प॒ते॒ य॒ज्ञ्॒प॒ते॒ त्वा॒ त्वा॒ य॒ज्ञ्॒प॒ते॒ । \newline
25. य॒ज्ञ्॒प॒ते॒ ह॒विर्भि॑र्. ह॒विर्भि॑र् यज्ञ्पते यज्ञ्पते ह॒विर्भिः॑ । \newline
26. य॒ज्ञ्॒प॒त॒ इति॑ यज्ञ् - प॒ते॒ । \newline
27. ह॒विर्भि॒रिति॑ ह॒विः - भिः॒ । \newline
28. सू॒क्त॒वा॒के न॑मोवा॒के न॑मोवा॒के सू᳚क्तवा॒के सू᳚क्तवा॒के न॑मोवा॒के । \newline
29. सू॒क्त॒वा॒क इति॑ सूक्त - वा॒के । \newline
30. न॒मो॒वा॒के वि॑धेम विधेम नमोवा॒के न॑मोवा॒के वि॑धेम । \newline
31. न॒मो॒वा॒क इति॑ नमः - वा॒के । \newline
32. वि॒धे॒माव॑भृ॒थाव॑भृथ विधेम विधे॒माव॑भृथ । \newline
33. अव॑भृथ निचङ्कुण निचङ्कु॒णा व॑भृ॒थाव॑भृथ निचङ्कुण । \newline
34. अव॑भृ॒थेत्यव॑ - भृ॒थ॒ । \newline
35. नि॒च॒ङ्कु॒ण॒ नि॒चे॒रुर् नि॑चे॒रुर् नि॑चङ्कुण निचङ्कुण निचे॒रुः । \newline
36. नि॒च॒ङ्कु॒णेति॑ नि - च॒ङ्कु॒ण॒ । \newline
37. नि॒चे॒रु र॑स्यसि निचे॒रुर् नि॑चे॒रुर॑सि । \newline
38. नि॒चे॒रुरिति॑ नि - चे॒रुः । \newline
39. अ॒सि॒ नि॒च॒ङ्कु॒ण॒ नि॒च॒ङ्कु॒णा॒स्य॒सि॒ नि॒च॒ङ्कु॒ण॒ । \newline
40. नि॒च॒ङ्कु॒णावाव॑ निचङ्कुण निचङ्कु॒णाव॑ । \newline
41. नि॒च॒ङ्कु॒णेति॑ नि - च॒ङ्कु॒ण॒ । \newline
42. अव॑ दे॒वैर् दे॒वै रवाव॑ दे॒वैः । \newline
43. दे॒वैर् दे॒वकृ॑तम् दे॒वकृ॑तम् दे॒वैर् दे॒वैर् दे॒वकृ॑तम् । \newline
44. दे॒वकृ॑त॒ मेन॒ एनो॑ दे॒वकृ॑तम् दे॒वकृ॑त॒ मेनः॑ । \newline
45. दे॒वकृ॑त॒मिति॑ दे॒व - कृ॒त॒म् । \newline
46. एनो॑ ऽयाडया॒डेन॒ एनो॑ ऽयाट् । \newline
47. अ॒या॒ डवावा॑ याडया॒ डव॑ । \newline
48. अव॒ मर्त्यै॒र् मर्त्यै॒ रवाव॒ मर्त्यैः᳚ । \newline
49. मर्त्यै॒र् मर्त्य॑कृत॒म् मर्त्य॑कृत॒म् मर्त्यै॒र् मर्त्यै॒र् मर्त्य॑कृतम् । \newline
50. मर्त्य॑कृत मु॒रोरु॒रोर् मर्त्य॑कृत॒म् मर्त्य॑कृत मु॒रोः । \newline
51. मर्त्य॑कृत॒मिति॒ मर्त्य॑ - कृ॒त॒म् । \newline
52. उ॒रो रोरो रु॒रोरा । \newline
53. आ नो॑ न॒ आ नः॑ । \newline
54. नो॒ दे॒व॒ दे॒व॒ नो॒ नो॒ दे॒व॒ । \newline
55. दे॒व॒ रि॒षो रि॒षो दे॑व देव रि॒षः । \newline
56. रि॒षश्पा॑हि पाहि रि॒षो रि॒षश्पा॑हि । \newline
57. पा॒हि॒ सु॒मि॒त्राः सु॑मि॒त्राः पा॑हि पाहि सुमि॒त्राः । \newline
58. सु॒मि॒त्रा नो॑ नः सुमि॒त्राः सु॑मि॒त्रा नः॑ । \newline
59. सु॒मि॒त्रा इति॑ सु - मि॒त्राः । \newline
60. न॒ आप॒ आपो॑ नो न॒ आपः॑ । \newline
61. आप॒ ओष॑धय॒ ओष॑धय॒ आप॒ आप॒ ओष॑धयः । \newline
62. ओष॑धयः सन्तु स॒न्त्वोष॑धय॒ ओष॑धयः सन्तु । \newline

\textbf{Ghana Paata } \newline

1. स॒मिधं॑ ॅयक्षि यक्षि स॒मिध(ग्म्॑) स॒मिधं॑ ॅयक्ष्यग्ने ऽग्ने यक्षि स॒मिध(ग्म्॑) स॒मिधं॑ ॅयक्ष्यग्ने । \newline
2. स॒मिध॒मिति॑ सं - इध᳚म् । \newline
3. य॒क्ष्य॒ग्ने॒ ऽग्ने॒ य॒क्षि॒ य॒क्ष्य॒ग्ने॒ । \newline
4. अ॒ग्न॒ इत्य॑ग्ने । \newline
5. प्रति॑ ते ते॒ प्रति॒ प्रति॑ ते जि॒ह्वा जि॒ह्वा ते॒ प्रति॒ प्रति॑ ते जि॒ह्वा । \newline
6. ते॒ जि॒ह्वा जि॒ह्वा ते॑ ते जि॒ह्वा घृ॒तम् घृ॒तम् जि॒ह्वा ते॑ ते जि॒ह्वा घृ॒तम् । \newline
7. जि॒ह्वा घृ॒तम् घृ॒तम् जि॒ह्वा जि॒ह्वा घृ॒त मुदुद् घृ॒तम् जि॒ह्वा जि॒ह्वा घृ॒त मुत् । \newline
8. घृ॒त मुदुद् घृ॒तम् घृ॒त मुच् च॑रण्येच् चरण्ये॒दुद् घृ॒तम् घृ॒त मुच् च॑रण्येत् । \newline
9. उच् च॑रण्येच् चरण्ये॒ दुदुच् च॑रण्येथ् समु॒द्रे स॑मु॒द्रे च॑रण्ये॒ दुदुच् च॑रण्येथ् समु॒द्रे । \newline
10. च॒र॒ण्ये॒थ् स॒मु॒द्रे स॑मु॒द्रे च॑रण्येच् चरण्येथ् समु॒द्रे ते॑ ते समु॒द्रे च॑रण्येच् चरण्येथ् समु॒द्रे ते᳚ । \newline
11. स॒मु॒द्रे ते॑ ते समु॒द्रे स॑मु॒द्रे ते॒ हृद॑य॒(ग्म्॒) हृद॑यम् ते समु॒द्रे स॑मु॒द्रे ते॒ हृद॑यम् । \newline
12. ते॒ हृद॑य॒(ग्म्॒) हृद॑यम् ते ते॒ हृद॑य म॒फ्स्व॑फ्सु हृद॑यम् ते ते॒ हृद॑य म॒फ्सु । \newline
13. हृद॑य म॒फ्स्व॑फ्सु हृद॑य॒(ग्म्॒) हृद॑य म॒फ्स्व॑न्त र॒न्त र॒फ्सु हृद॑य॒(ग्म्॒) हृद॑य म॒फ्स्व॑न्तः । \newline
14. अ॒फ्स्व॑न्त र॒न्त र॒फ्स्वा᳚(1॒)फ्स्व॑न्तः । \newline
15. अ॒फ्स्वित्य॑प् - सु । \newline
16. अ॒न्तरित्य॒न्तः । \newline
17. सम् त्वा᳚ त्वा॒ सꣳ सम् त्वा॑ विशन्तु विशन्तु त्वा॒ सꣳ सम् त्वा॑ विशन्तु । \newline
18. त्वा॒ वि॒श॒न्तु॒ वि॒श॒न्तु॒ त्वा॒ त्वा॒ वि॒श॒ न्त्वोष॑धी॒ रोष॑धीर् विशन्तु त्वा त्वा विश॒न्त्वोष॑धीः । \newline
19. वि॒श॒न् त्वोष॑धी॒ रोष॑धीर् विशन्तु विश॒न् त्वोष॑धी रु॒तोतौष॑धीर् विशन्तु विश॒ न्त्वोष॑धीरु॒त । \newline
20. ओष॑धी रु॒तो तौष॑धी॒ रोष॑धी रु॒ताप॒ आप॑ उ॒तौष॑धी॒ रोष॑धी रु॒तापः॑ । \newline
21. उ॒ताप॒ आप॑ उ॒तोतापो॑ य॒ज्ञ्स्य॑ य॒ज्ञ्स्याप॑ उ॒तोतापो॑ य॒ज्ञ्स्य॑ । \newline
22. आपो॑ य॒ज्ञ्स्य॑ य॒ज्ञ्स्याप॒ आपो॑ य॒ज्ञ्स्य॑ त्वा त्वा य॒ज्ञ्स्याप॒ आपो॑ य॒ज्ञ्स्य॑ त्वा । \newline
23. य॒ज्ञ्स्य॑ त्वा त्वा य॒ज्ञ्स्य॑ य॒ज्ञ्स्य॑ त्वा यज्ञ्पते यज्ञ्पते त्वा य॒ज्ञ्स्य॑ य॒ज्ञ्स्य॑ त्वा यज्ञ्पते । \newline
24. त्वा॒ य॒ज्ञ्॒प॒ते॒ य॒ज्ञ्॒प॒ते॒ त्वा॒ त्वा॒ य॒ज्ञ्॒प॒ते॒ ह॒विर्भि॑र्. ह॒विर्भि॑र् यज्ञ्पते त्वा त्वा यज्ञ्पते ह॒विर्भिः॑ । \newline
25. य॒ज्ञ्॒प॒ते॒ ह॒विर्भि॑र्. ह॒विर्भि॑र् यज्ञ्पते यज्ञ्पते ह॒विर्भिः॑ । \newline
26. य॒ज्ञ्॒प॒त॒ इति॑ यज्ञ् - प॒ते॒ । \newline
27. ह॒विर्भि॒रिति॑ ह॒विः - भिः॒ । \newline
28. सू॒क्त॒वा॒के न॑मोवा॒के न॑मोवा॒के सू᳚क्तवा॒के सू᳚क्तवा॒के न॑मोवा॒के वि॑धेम विधेम नमोवा॒के सू᳚क्तवा॒के सू᳚क्तवा॒के न॑मोवा॒के वि॑धेम । \newline
29. सू॒क्त॒वा॒क इति॑ सूक्त - वा॒के । \newline
30. न॒मो॒वा॒के वि॑धेम विधेम नमोवा॒के न॑मोवा॒के वि॑धे॒ माव॑भृ॒ थाव॑भृथ विधेम नमोवा॒के न॑मोवा॒के वि॑धे॒माव॑भृथ । \newline
31. न॒मो॒वा॒क इति॑ नमः - वा॒के । \newline
32. वि॒धे॒ माव॑भृ॒ थाव॑भृथ विधेम विधे॒माव॑भृथ निचङ्कुण निचङ्कु॒णाव॑भृथ विधेम विधे॒माव॑भृथ निचङ्कुण । \newline
33. अव॑भृथ निचङ्कुण निचङ्कु॒ णाव॑भृ॒ थाव॑भृथ निचङ्कुण निचे॒रुर् नि॑चे॒रुर् नि॑चङ्कु॒ णाव॑ भृ॒थाव॑भृथ निचङ्कुण निचे॒रुः । \newline
34. अव॑भृ॒थेत्यव॑ - भृ॒थ॒ । \newline
35. नि॒च॒ङ्कु॒ण॒ नि॒चे॒रुर् नि॑चे॒रुर् नि॑चङ्कुण निचङ्कुण निचे॒रुर॑स्यसि निचे॒रुर् नि॑चङ्कुण निचङ्कुण निचे॒रुर॑सि । \newline
36. नि॒च॒ङ्कु॒णेति॑ नि - च॒ङ्कु॒ण॒ । \newline
37. नि॒चे॒ रुर॑स्यसि निचे॒रुर् नि॑चे॒ रुर॑सि निचङ्कुण निचङ्कुणासि निचे॒रुर् नि॑चे॒ रुर॑सि निचङ्कुण । \newline
38. नि॒चे॒रुरिति॑ नि - चे॒रुः । \newline
39. अ॒सि॒ नि॒च॒ङ्कु॒ण॒ नि॒च॒ङ्कु॒ णा॒स्य॒सि॒ नि॒च॒ङ्कु॒णावाव॑ निचङ्कु णास्यसि निचङ्कु॒णाव॑ । \newline
40. नि॒च॒ङ्कु॒णावाव॑ निचङ्कुण निचङ्कु॒णाव॑ दे॒वैर् दे॒वैरव॑ निचङ्कुण निचङ्कु॒णाव॑ दे॒वैः । \newline
41. नि॒च॒ङ्कु॒णेति॑ नि - च॒ङ्कु॒ण॒ । \newline
42. अव॑ दे॒वैर् दे॒वैरवाव॑ दे॒वैर् दे॒वकृ॑तम् दे॒वकृ॑तम् दे॒वैरवाव॑ दे॒वैर् दे॒वकृ॑तम् । \newline
43. दे॒वैर् दे॒वकृ॑तम् दे॒वकृ॑तम् दे॒वैर् दे॒वैर् दे॒वकृ॑त॒ मेन॒ एनो॑ दे॒वकृ॑तम् दे॒वैर् दे॒वैर् दे॒वकृ॑त॒ मेनः॑ । \newline
44. दे॒वकृ॑त॒ मेन॒ एनो॑ दे॒वकृ॑तम् दे॒वकृ॑त॒ मेनो॑ ऽया डया॒डेनो॑ दे॒वकृ॑तम् दे॒वकृ॑त॒ मेनो॑ ऽयाट् । \newline
45. दे॒वकृ॑त॒मिति॑ दे॒व - कृ॒त॒म् । \newline
46. एनो॑ ऽया डया॒डेन॒ एनो॑ ऽया॒ड वावा॑ या॒डेन॒ एनो॑ ऽया॒डव॑ । \newline
47. अ॒या॒ डवावा॑ याडया॒ डव॒ मर्त्यै॒र् मर्त्यै॒ रवा॑ याडया॒ डव॒ मर्त्यैः᳚ । \newline
48. अव॒ मर्त्यै॒र् मर्त्यै॒रवाव॒ मर्त्यै॒र् मर्त्य॑कृत॒म् मर्त्य॑कृत॒म् मर्त्यै॒रवाव॒ मर्त्यै॒र् मर्त्य॑कृतम् । \newline
49. मर्त्यै॒र् मर्त्य॑कृत॒म् मर्त्य॑कृत॒म् मर्त्यै॒र् मर्त्यै॒र् मर्त्य॑कृत मु॒रोरु॒रोर् मर्त्य॑कृत॒म् मर्त्यै॒र् मर्त्यै॒र् मर्त्य॑कृत मु॒रोः । \newline
50. मर्त्य॑कृत मु॒रो रु॒रोर् मर्त्य॑कृत॒म् मर्त्य॑कृत मु॒रो रोरोर् मर्त्य॑कृत॒म् मर्त्य॑कृत मु॒रोरा । \newline
51. मर्त्य॑कृत॒मिति॒ मर्त्य॑ - कृ॒त॒म् । \newline
52. उ॒रो रोरो रु॒रोरा नो॑ न॒ ओरो रु॒रोरा नः॑ । \newline
53. आ नो॑ न॒ आ नो॑ देव देव न॒ आ नो॑ देव । \newline
54. नो॒ दे॒व॒ दे॒व॒ नो॒ नो॒ दे॒व॒ रि॒षो रि॒षो दे॑व नो नो देव रि॒षः । \newline
55. दे॒व॒ रि॒षो रि॒षो दे॑व देव रि॒ष श्पा॑हि पाहि रि॒षो दे॑व देव रि॒ष श्पा॑हि । \newline
56. रि॒ष श्पा॑हि पाहि रि॒षो रि॒ष श्पा॑हि सुमि॒त्राः सु॑मि॒त्राः पा॑हि रि॒षो रि॒ष श्पा॑हि सुमि॒त्राः । \newline
57. पा॒हि॒ सु॒मि॒त्राः सु॑मि॒त्राः पा॑हि पाहि सुमि॒त्रा नो॑ नः सुमि॒त्राः पा॑हि पाहि सुमि॒त्रा नः॑ । \newline
58. सु॒मि॒त्रा नो॑ नः सुमि॒त्राः सु॑मि॒त्रा न॒ आप॒ आपो॑ नः सुमि॒त्राः सु॑मि॒त्रा न॒ आपः॑ । \newline
59. सु॒मि॒त्रा इति॑ सु - मि॒त्राः । \newline
60. न॒ आप॒ आपो॑ नो न॒ आप॒ ओष॑धय॒ ओष॑धय॒ आपो॑ नो न॒ आप॒ ओष॑धयः । \newline
61. आप॒ ओष॑धय॒ ओष॑धय॒ आप॒ आप॒ ओष॑धयः सन्तु स॒न्त्वोष॑धय॒ आप॒ आप॒ ओष॑धयः सन्तु । \newline
62. ओष॑धयः सन्तु स॒न्त्वोष॑धय॒ ओष॑धयः सन्तु दुर्मि॒त्रा दु॑र्मि॒त्राः स॒न्त्वोष॑धय॒ ओष॑धयः सन्तु दुर्मि॒त्राः । \newline
\pagebreak
\markright{ TS 1.4.45.3  \hfill https://www.vedavms.in \hfill}
\addcontentsline{toc}{section}{ TS 1.4.45.3 }
\section*{ TS 1.4.45.3 }

\textbf{TS 1.4.45.3 } \newline
\textbf{Samhita Paata} \newline

सन्तु दुर्मि॒त्रास्तस्मै॑ भूयासु॒र् यो᳚ऽस्मान् द्वेष्टि॒ यं च॑ व॒यं द्वि॒ष्मो देवी॑राप ए॒ष वो॒ गर्भ॒स्तं ॅवः॒ सुप्री॑तꣳ॒॒ सुभृ॑त-मकर्म दे॒वेषु॑ नः सु॒कृतो᳚ ब्रूता॒त् प्रति॑युतो॒ वरु॑णस्य॒ पाशः॒ प्रत्य॑स्तो॒ वरु॑णस्य॒ पाश॒ एधो᳚ऽस्येधिषी॒महि॑ स॒मिद॑सि॒ तेजो॑ऽसि तेजो॒ मयि॑ धेह्य॒पो अन्व॑चारिषꣳ॒॒ रसे॑न॒ सम॑सृक्ष्महि । पय॑स्वाꣳ अग्न॒ आ ( ) ऽग॑मं॒ तं मा॒ सꣳ सृ॑ज॒ वर्च॑सा ॥ \newline

\textbf{Pada Paata} \newline

स॒न्तु॒ । दु॒र्मि॒त्रा इति॑ दुः-मि॒त्राः । तस्मै᳚ । भू॒या॒सुः॒ । यः । अ॒स्मान् । द्वेष्टि॑ । यम् । च॒ । व॒यम् । द्वि॒ष्मः । देवीः᳚ । आ॒पः॒ । ए॒षः । वः॒ । गर्भः॑ । तम् । वः॒ । सुप्री॑त॒मिति॒ सु - प्री॒त॒म् । सुभृ॑त॒मिति॒ सु - भृ॒त॒म् । अ॒क॒र्म॒ । दे॒वेषु॑ । नः॒ । सु॒कृत॒ इति॑ सु-कृतः॑ । ब्रू॒ता॒त् । प्रति॑युत॒ इति॒ प्रति॑ - यु॒तः॒ । वरु॑णस्य । पाशः॑ । प्रत्य॑स्त॒ इति॒ प्रति॑ - अ॒स्तः॒ । वरु॑णस्य । पाशः॑ । एधः॑ । अ॒सि॒ । ए॒धि॒षी॒महि॑ । स॒मिदिति॑ सम् - इत् । अ॒सि॒ । तेजः॑ । अ॒सि॒ । तेजः॑ । मयि॑ । धे॒ह॒ । अ॒पः । अन्विति॑ । अ॒चा॒रि॒ष॒म् । रसे॑न । समिति॑ । अ॒सृ॒क्ष्म॒हि॒ । पय॑स्वान् । अ॒ग्ने॒ । एति॑ ( ) । अ॒ग॒म॒म् । तम् । मा॒ । समिति॑ । सृ॒ज॒ । वर्च॑सा ॥  \newline


\textbf{Krama Paata} \newline

स॒न्तु॒ दु॒र्मि॒त्राः । दु॒र्मि॒त्रास्तस्मै᳚ । दु॒र्मि॒त्रा इति॑ दुः - मि॒त्राः । तस्मै॑ भूयासुः । भू॒या॒सु॒र् यः । 
यो᳚ऽस्मान् । अ॒स्मान् द्वेष्टि॑ । द्वेष्टि॒ यम् । यम् च॑ । च॒ व॒यम् । व॒यम् द्वि॒ष्मः । द्वि॒ष्मो देवीः᳚ । देवी॑रापः । आ॒प॒ ए॒षः । ए॒ष वः॑ । वो॒ गर्भः॑ । गर्भ॒स्तम् । तं ॅवः॑ । वः॒ सुप्री॑तम् । सुप्री॑तꣳ॒॒ सुभृ॑तम् । सुप्री॑त॒मिति॒ सु - प्री॒त॒म् । सुभृ॑तमकर्म । सुभृ॑त॒मिति॒ सु - भृ॒त॒म् । अ॒क॒र्म॒ दे॒वेषु॑ । दे॒वेषु॑ नः । नः॒ सु॒कृतः॑ । सु॒कृतो᳚ ब्रूतात् । सु॒कृत॒ इति॑ सु - कृतः॑ । ब्रू॒ता॒त्,प्रति॑युतः । प्रति॑युतो॒ वरु॑णस्य । प्रति॑युत॒ इति॒ प्रति॑ - यु॒तः॒ । वरु॑णस्य॒ पाशः॑ । पाशः॒ प्रत्य॑स्तः । प्रत्य॑स्तो॒ वरु॑णस्य । प्रत्य॑स्त॒ इति॒ प्रति॑ - अ॒स्तः॒ । वरु॑णस्य॒ पाशः॑ । पाश॒ एधः॑ । एधो॑ऽसि । अ॒स्ये॒धि॒षी॒महि॑ । ए॒धि॒षी॒महि॑ स॒मित् । स॒मिद॑सि । स॒मिदिति॑ सम् - इत् । अ॒सि॒ तेजः॑ । तेजो॑ऽसि । अ॒सि॒ तेजः॑ । तेजो॒ मयि॑ । मयि॑ धेहि । धे॒ह्य॒पः । अ॒पो अनु॑ । अन्व॑चारिषम् । अ॒चा॒रि॒षꣳ॒॒ रसे॑न । रसे॑न॒ सम् । सम॑सृक्ष्महि । अ॒सृ॒क्ष्म॒हीत्य॑सृक्ष्महि ॥ पय॑स्वाꣳ अग्ने । अ॒ग्न॒ आ ( ) । आऽग॑मम् । अ॒ग॒म॒म् तम् । तम् मा᳚ । मा॒ सम् । सꣳ सृ॑ज । सृ॒ज॒ वर्च॑सा । वर्च॒सेति॒ वर्च॑सा । \newline

\textbf{Jatai Paata} \newline

1. स॒न्तु॒ दु॒र्मि॒त्रा दु॑र्मि॒त्राः स॑न्तु सन्तु दुर्मि॒त्राः । \newline
2. दु॒र्मि॒त्रा स्तस्मै॒ तस्मै॑ दुर्मि॒त्रा दु॑र्मि॒त्रा स्तस्मै᳚ । \newline
3. दु॒र्मि॒त्रा इति॑ दुः - मि॒त्राः । \newline
4. तस्मै॑ भूयासुर् भूयासु॒ स्तस्मै॒ तस्मै॑ भूयासुः । \newline
5. भू॒या॒सु॒र् यो यो भू॑यासुर् भूयासु॒र् यः । \newline
6. यो᳚ ऽस्मा न॒स्मान्. यो यो᳚ ऽस्मान् । \newline
7. अ॒स्मान् द्वेष्टि॒ द्वेष्ट्य॒स्मा न॒स्मान् द्वेष्टि॑ । \newline
8. द्वेष्टि॒ यं ॅयम् द्वेष्टि॒ द्वेष्टि॒ यम् । \newline
9. यम् च॑ च॒ यं ॅयम् च॑ । \newline
10. च॒ व॒यं ॅव॒यम् च॑ च व॒यम् । \newline
11. व॒यम् द्वि॒ष्मो द्वि॒ष्मो व॒यं ॅव॒यम् द्वि॒ष्मः । \newline
12. द्वि॒ष्मो देवी॒र् देवी᳚र् द्वि॒ष्मो द्वि॒ष्मो देवीः᳚ । \newline
13. देवी॑राप आपो॒ देवी॒र् देवी॑रापः । \newline
14. आ॒प॒ ए॒ष ए॒ष आ॑प आप ए॒षः । \newline
15. ए॒ष वो॑ व ए॒ष ए॒ष वः॑ । \newline
16. वो॒ गर्भो॒ गर्भो॑ वो वो॒ गर्भः॑ । \newline
17. गर्भ॒ स्तम् तम् गर्भो॒ गर्भ॒ स्तम् । \newline
18. तं ॅवो॑ व॒ स्तम् तं ॅवः॑ । \newline
19. वः॒ सुप्री॑त॒(ग्म्॒) सुप्री॑तं ॅवो वः॒ सुप्री॑तम् । \newline
20. सुप्री॑त॒(ग्म्॒) सुभृ॑त॒(ग्म्॒) सुभृ॑त॒(ग्म्॒) सुप्री॑त॒(ग्म्॒) सुप्री॑त॒(ग्म्॒) सुभृ॑तम् । \newline
21. सुप्री॑त॒मिति॒ सु - प्री॒त॒म् । \newline
22. सुभृ॑त मकर्माकर्म॒ सुभृ॑त॒(ग्म्॒) सुभृ॑त मकर्म । \newline
23. सुभृ॑त॒मिति॒ सु - भृ॒त॒म् । \newline
24. अ॒क॒र्म॒ दे॒वेषु॑ दे॒वेष्व॑कर्माकर्म दे॒वेषु॑ । \newline
25. दे॒वेषु॑ नो नो दे॒वेषु॑ दे॒वेषु॑ नः । \newline
26. नः॒ सु॒कृतः॑ सु॒कृतो॑ नो नः सु॒कृतः॑ । \newline
27. सु॒कृतो᳚ ब्रूताद् ब्रूताथ् सु॒कृतः॑ सु॒कृतो᳚ ब्रूतात् । \newline
28. सु॒कृत॒ इति॑ सु - कृतः॑ । \newline
29. ब्रू॒ता॒त् प्रति॑युतः॒ प्रति॑युतो ब्रूताद् ब्रूता॒त् प्रति॑युतः । \newline
30. प्रति॑युतो॒ वरु॑णस्य॒ वरु॑णस्य॒ प्रति॑युतः॒ प्रति॑युतो॒ वरु॑णस्य । \newline
31. प्रति॑युत॒ इति॒ प्रति॑ - यु॒तः॒ । \newline
32. वरु॑णस्य॒ पाशः॒ पाशो॒ वरु॑णस्य॒ वरु॑णस्य॒ पाशः॑ । \newline
33. पाशः॒ प्रत्य॑स्तः॒ प्रत्य॑स्तः॒ पाशः॒ पाशः॒ प्रत्य॑स्तः । \newline
34. प्रत्य॑स्तो॒ वरु॑णस्य॒ वरु॑णस्य॒ प्रत्य॑स्तः॒ प्रत्य॑स्तो॒ वरु॑णस्य । \newline
35. प्रत्य॑स्त॒ इति॒ प्रति॑ - अ॒स्तः॒ । \newline
36. वरु॑णस्य॒ पाशः॒ पाशो॒ वरु॑णस्य॒ वरु॑णस्य॒ पाशः॑ । \newline
37. पाश॒ एध॒ एधः॒ पाशः॒ पाश॒ एधः॑ । \newline
38. एधो᳚ ऽस्य॒स्येध॒ एधो॑ ऽसि । \newline
39. अ॒स्ये॒धि॒षी॒ मह्ये॑धिषी॒ मह्य॑स्य स्येधिषी॒महि॑ । \newline
40. ए॒धि॒षी॒महि॑ स॒मिथ् स॒मिदे॑धिषी॒ मह्ये॑धिषी॒महि॑ स॒मित् । \newline
41. स॒मि द॑स्यसि स॒मिथ् स॒मिद॑सि । \newline
42. स॒मिदिति॑ सम् - इत् । \newline
43. अ॒सि॒ तेज॒ स्तेजो᳚ ऽस्यसि॒ तेजः॑ । \newline
44. तेजो᳚ ऽस्यसि॒ तेज॒ स्तेजो॑ ऽसि । \newline
45. अ॒सि॒ तेज॒ स्तेजो᳚ ऽस्यसि॒ तेजः॑ । \newline
46. तेजो॒ मयि॒ मयि॒ तेज॒ स्तेजो॒ मयि॑ । \newline
47. मयि॑ धेहि धेहि॒ मयि॒ मयि॑ धेहि । \newline
48. धे॒ह्य॒पो॑ ऽपो धे॑हि धेह्य॒पः । \newline
49. अ॒पो अन्वन्व॒पो॑ ऽपो अनु॑ । \newline
50. अन्व॑चारिष मचारिष॒ मन्वन्व॑चारिषम् । \newline
51. अ॒चा॒रि॒ष॒(ग्म्॒) रसे॑न॒ रसे॑नाचारिष मचारिष॒(ग्म्॒) रसे॑न । \newline
52. रसे॑न॒ सꣳ सꣳ रसे॑न॒ रसे॑न॒ सम् । \newline
53. स म॑सृक्ष्मह्यसृक्ष्महि॒ सꣳ स म॑सृक्ष्महि । \newline
54. अ॒सृ॒क्ष्म॒हीत्य॑सृक्ष्महि । \newline
55. पय॑स्वाꣳ अग्ने ऽग्ने॒ पय॑स्वा॒न् पय॑स्वाꣳ अग्ने । \newline
56. अ॒ग्न॒ आ ऽग्ने᳚ ऽग्न॒ आ । \newline
57. आ ऽग॑म मगम॒ मा ऽग॑मम् । \newline
58. अ॒ग॒म॒म् तम् त म॑गम मगम॒म् तम् । \newline
59. तम् मा॑ मा॒ तम् तम् मा᳚ । \newline
60. मा॒ सꣳ सम् मा॑ मा॒ सम् । \newline
61. सꣳ सृ॑ज सृज॒ सꣳ सꣳ सृ॑ज । \newline
62. सृ॒ज॒ वर्च॑सा॒ वर्च॑सा सृज सृज॒ वर्च॑सा । \newline
63. वर्च॒सेति॒ वर्च॑सा । \newline

\textbf{Ghana Paata } \newline

1. स॒न्तु॒ दु॒र्मि॒त्रा दु॑र्मि॒त्राः स॑न्तु सन्तु दुर्मि॒त्रा स्तस्मै॒ तस्मै॑ दुर्मि॒त्राः स॑न्तु सन्तु दुर्मि॒त्रा स्तस्मै᳚ । \newline
2. दु॒र्मि॒त्रा स्तस्मै॒ तस्मै॑ दुर्मि॒त्रा दु॑र्मि॒त्रा स्तस्मै॑ भूयासुर् भूयासु॒ स्तस्मै॑ दुर्मि॒त्रा दु॑र्मि॒त्रा स्तस्मै॑ भूयासुः । \newline
3. दु॒र्मि॒त्रा इति॑ दुः - मि॒त्राः । \newline
4. तस्मै॑ भूयासुर् भूयासु॒ स्तस्मै॒ तस्मै॑ भूयासु॒र् यो यो भू॑यासु॒ स्तस्मै॒ तस्मै॑ भूयासु॒र् यः । \newline
5. भू॒या॒सु॒र् यो यो भू॑यासुर् भूयासु॒र् यो᳚ ऽस्मा न॒स्मान्. यो भू॑यासुर् भूयासु॒र् यो᳚ ऽस्मान् । \newline
6. यो᳚ ऽस्मा न॒स्मान्. यो यो᳚ ऽस्मान् द्वेष्टि॒ द्वेष्ट्य॒स्मान्. यो यो᳚ ऽस्मान् द्वेष्टि॑ । \newline
7. अ॒स्मान् द्वेष्टि॒ द्वेष्ट्य॒स्मा न॒स्मान् द्वेष्टि॒ यं ॅयम् द्वेष्ट्य॒स्मा न॒स्मान् द्वेष्टि॒ यम् । \newline
8. द्वेष्टि॒ यं ॅयम् द्वेष्टि॒ द्वेष्टि॒ यम् च॑ च॒ यम् द्वेष्टि॒ द्वेष्टि॒ यम् च॑ । \newline
9. यम् च॑ च॒ यं ॅयम् च॑ व॒यं ॅव॒यम् च॒ यं ॅयम् च॑ व॒यम् । \newline
10. च॒ व॒यं ॅव॒यम् च॑ च व॒यम् द्वि॒ष्मो द्वि॒ष्मो व॒यम् च॑ च व॒यम् द्वि॒ष्मः । \newline
11. व॒यम् द्वि॒ष्मो द्वि॒ष्मो व॒यं ॅव॒यम् द्वि॒ष्मो देवी॒र् देवी᳚र् द्वि॒ष्मो व॒यं ॅव॒यम् द्वि॒ष्मो देवीः᳚ । \newline
12. द्वि॒ष्मो देवी॒र् देवी᳚र् द्वि॒ष्मो द्वि॒ष्मो देवी॑राप आपो॒ देवी᳚र् द्वि॒ष्मो द्वि॒ष्मो देवी॑रापः । \newline
13. देवी॑राप आपो॒ देवी॒र् देवी॑राप ए॒ष ए॒ष आ॑पो॒ देवी॒र् देवी॑राप ए॒षः । \newline
14. आ॒प॒ ए॒ष ए॒ष आ॑प आप ए॒ष वो॑ व ए॒ष आ॑प आप ए॒ष वः॑ । \newline
15. ए॒ष वो॑ व ए॒ष ए॒ष वो॒ गर्भो॒ गर्भो॑ व ए॒ष ए॒ष वो॒ गर्भः॑ । \newline
16. वो॒ गर्भो॒ गर्भो॑ वो वो॒ गर्भ॒ स्तम् तम् गर्भो॑ वो वो॒ गर्भ॒ स्तम् । \newline
17. गर्भ॒ स्तम् तम् गर्भो॒ गर्भ॒ स्तं ॅवो॑ व॒ स्तम् गर्भो॒ गर्भ॒ स्तं ॅवः॑ । \newline
18. तं ॅवो॑ व॒ स्तम् तं ॅवः॒ सुप्री॑त॒(ग्म्॒) सुप्री॑तं ॅव॒ स्तम् तं ॅवः॒ सुप्री॑तम् । \newline
19. वः॒ सुप्री॑त॒(ग्म्॒) सुप्री॑तं ॅवो वः॒ सुप्री॑त॒(ग्म्॒) सुभृ॑त॒(ग्म्॒) सुभृ॑त॒(ग्म्॒) सुप्री॑तं ॅवो वः॒ सुप्री॑त॒(ग्म्॒) सुभृ॑तम् । \newline
20. सुप्री॑त॒(ग्म्॒) सुभृ॑त॒(ग्म्॒) सुभृ॑त॒(ग्म्॒) सुप्री॑त॒(ग्म्॒) सुप्री॑त॒(ग्म्॒) सुभृ॑त मकर्माकर्म॒ सुभृ॑त॒(ग्म्॒) सुप्री॑त॒(ग्म्॒) सुप्री॑त॒(ग्म्॒) सुभृ॑त मकर्म । \newline
21. सुप्री॑त॒मिति॒ सु - प्री॒त॒म् । \newline
22. सुभृ॑त मकर्माकर्म॒ सुभृ॑त॒(ग्म्॒) सुभृ॑त मकर्म दे॒वेषु॑ दे॒वेष्व॑कर्म॒ सुभृ॑त॒(ग्म्॒) सुभृ॑त मकर्म दे॒वेषु॑ । \newline
23. सुभृ॑त॒मिति॒ सु - भृ॒त॒म् । \newline
24. अ॒क॒र्म॒ दे॒वेषु॑ दे॒वे ष्व॑कर्माकर्म दे॒वेषु॑ नो नो दे॒वे ष्व॑कर्माकर्म दे॒वेषु॑ नः । \newline
25. दे॒वेषु॑ नो नो दे॒वेषु॑ दे॒वेषु॑ नः सु॒कृतः॑ सु॒कृतो॑ नो दे॒वेषु॑ दे॒वेषु॑ नः सु॒कृतः॑ । \newline
26. नः॒ सु॒कृतः॑ सु॒कृतो॑ नो नः सु॒कृतो᳚ ब्रूताद् ब्रूताथ् सु॒कृतो॑ नो नः सु॒कृतो᳚ ब्रूतात् । \newline
27. सु॒कृतो᳚ ब्रूताद् ब्रूताथ् सु॒कृतः॑ सु॒कृतो᳚ ब्रूता॒त् प्रति॑युतः॒ प्रति॑युतो ब्रूताथ् सु॒कृतः॑ सु॒कृतो᳚ ब्रूता॒त् प्रति॑युतः । \newline
28. सु॒कृत॒ इति॑ सु - कृतः॑ । \newline
29. ब्रू॒ता॒त् प्रति॑युतः॒ प्रति॑युतो ब्रूताद् ब्रूता॒त् प्रति॑युतो॒ वरु॑णस्य॒ वरु॑णस्य॒ प्रति॑युतो ब्रूताद् ब्रूता॒त् प्रति॑युतो॒ वरु॑णस्य । \newline
30. प्रति॑युतो॒ वरु॑णस्य॒ वरु॑णस्य॒ प्रति॑युतः॒ प्रति॑युतो॒ वरु॑णस्य॒ पाशः॒ पाशो॒ वरु॑णस्य॒ प्रति॑युतः॒ प्रति॑युतो॒ वरु॑णस्य॒ पाशः॑ । \newline
31. प्रति॑युत॒ इति॒ प्रति॑ - यु॒तः॒ । \newline
32. वरु॑णस्य॒ पाशः॒ पाशो॒ वरु॑णस्य॒ वरु॑णस्य॒ पाशः॒ प्रत्य॑स्तः॒ प्रत्य॑स्तः॒ पाशो॒ वरु॑णस्य॒ वरु॑णस्य॒ पाशः॒ प्रत्य॑स्तः । \newline
33. पाशः॒ प्रत्य॑स्तः॒ प्रत्य॑स्तः॒ पाशः॒ पाशः॒ प्रत्य॑स्तो॒ वरु॑णस्य॒ वरु॑णस्य॒ प्रत्य॑स्तः॒ पाशः॒ पाशः॒ 
प्रत्य॑स्तो॒ वरु॑णस्य । \newline
34. प्रत्य॑स्तो॒ वरु॑णस्य॒ वरु॑णस्य॒ प्रत्य॑स्तः॒ प्रत्य॑स्तो॒ वरु॑णस्य॒ पाशः॒ पाशो॒ वरु॑णस्य॒ प्रत्य॑स्तः॒ प्रत्य॑स्तो॒ वरु॑णस्य॒ पाशः॑ । \newline
35. प्रत्य॑स्त॒ इति॒ प्रति॑ - अ॒स्तः॒ । \newline
36. वरु॑णस्य॒ पाशः॒ पाशो॒ वरु॑णस्य॒ वरु॑णस्य॒ पाश॒ एध॒ एधः॒ पाशो॒ वरु॑णस्य॒ वरु॑णस्य॒ पाश॒ एधः॑ । \newline
37. पाश॒ एध॒ एधः॒ पाशः॒ पाश॒ एधो᳚ ऽस्य॒स्येधः॒ पाशः॒ पाश॒ एधो॑ ऽसि । \newline
38. एधो᳚ ऽस्य॒स्येध॒ एधो᳚ ऽस्येधिषी॒म ह्ये॑धिषी॒म ह्य॒स्येध॒ एधो᳚ ऽस्येधिषी॒महि॑ । \newline
39. अ॒स्ये॒धि॒षी॒ मह्ये॑धिषी॒ मह्य॑स्य स्येधिषी॒महि॑ स॒मिथ् स॒मिदे॑धिषी॒ मह्य॑स्य स्येधिषी॒महि॑ स॒मित् । \newline
40. ए॒धि॒षी॒महि॑ स॒मिथ् स॒मिदे॑धिषी॒ मह्ये॑धिषी॒महि॑ स॒मिद॑स्यसि स॒मिदे॑धिषी॒ मह्ये॑धिषी॒महि॑ स॒मिद॑सि । \newline
41. स॒मिद॑स्यसि स॒मिथ् स॒मिद॑सि॒ तेज॒ स्तेजो॑ ऽसि स॒मिथ् स॒मिद॑सि॒ तेजः॑ । \newline
42. स॒मिदिति॑ सम् - इत् । \newline
43. अ॒सि॒ तेज॒ स्तेजो᳚ ऽस्यसि॒ तेजो᳚ ऽस्यसि॒ तेजो᳚ ऽस्यसि॒ तेजो॑ ऽसि । \newline
44. तेजो᳚ ऽस्यसि॒ तेज॒ स्तेजो॑ ऽसि॒ तेज॒ स्तेजो॑ ऽसि॒ तेज॒ स्तेजो॑ ऽसि॒ तेजः॑ । \newline
45. अ॒सि॒ तेज॒ स्तेजो᳚ ऽस्यसि॒ तेजो॒ मयि॒ मयि॒ तेजो᳚ ऽस्यसि॒ तेजो॒ मयि॑ । \newline
46. तेजो॒ मयि॒ मयि॒ तेज॒ स्तेजो॒ मयि॑ धेहि धेहि॒ मयि॒ तेज॒ स्तेजो॒ मयि॑ धेहि । \newline
47. मयि॑ धेहि धेहि॒ मयि॒ मयि॑ धेह्य॒पो॑ ऽपो धे॑हि॒ मयि॒ मयि॑ धेह्य॒पः । \newline
48. धे॒ह्य॒पो॑ ऽपो धे॑हि धेह्य॒पो अन्वन्व॒पो धे॑हि धेह्य॒पो अनु॑ । \newline
49. अ॒पो अन्वन्व॒पो॑ ऽपो अन्व॑चारिष मचारिष॒ मन्व॒पो॑ ऽपो अन्व॑चारिषम् । \newline
50. अन्व॑चारिष मचारिष॒ मन्वन्व॑चारिष॒(ग्म्॒) रसे॑न॒ रसे॑नाचारिष॒ मन्वन्व॑चारिष॒(ग्म्॒) रसे॑न । \newline
51. अ॒चा॒रि॒ष॒(ग्म्॒) रसे॑न॒ रसे॑नाचारिष मचारिष॒(ग्म्॒) रसे॑न॒ सꣳ सꣳ रसे॑नाचारिष मचारिष॒(ग्म्॒) रसे॑न॒ सम् । \newline
52. रसे॑न॒ सꣳ सꣳ रसे॑न॒ रसे॑न॒ स म॑सृक्ष्म ह्यसृक्ष्महि॒ सꣳ रसे॑न॒ रसे॑न॒ स म॑सृक्ष्महि । \newline
53. स म॑सृक्ष्म ह्यसृक्ष्महि॒ सꣳ स म॑सृक्ष्महि । \newline
54. अ॒सृ॒क्ष्म॒हीत्य॑सृक्ष्महि । \newline
55. पय॑स्वाꣳ अग्ने ऽग्ने॒ पय॑स्वा॒न् पय॑स्वाꣳ अग्न॒ आ ऽग्ने॒ पय॑स्वा॒न् पय॑स्वाꣳ अग्न॒ आ । \newline
56. अ॒ग्न॒ आ ऽग्ने᳚ ऽग्न॒ आ ऽग॑म मगम॒ मा ऽग्ने᳚ ऽग्न॒ आ ऽग॑मम् । \newline
57. आ ऽग॑म मगम॒ मा ऽग॑म॒म् तम् त म॑गम॒ मा ऽग॑म॒म् तम् । \newline
58. अ॒ग॒म॒म् तम् त म॑गम मगम॒म् तम् मा॑ मा॒ त म॑गम मगम॒म् तम् मा᳚ । \newline
59. तम् मा॑ मा॒ तम् तम् मा॒ सꣳ सम् मा॒ तम् तम् मा॒ सम् । \newline
60. मा॒ सꣳ सम् मा॑ मा॒ सꣳ सृ॑ज सृज॒ सम् मा॑ मा॒ सꣳ सृ॑ज । \newline
61. सꣳ सृ॑ज सृज॒ सꣳ सꣳ सृ॑ज॒ वर्च॑सा॒ वर्च॑सा सृज॒ सꣳ सꣳ सृ॑ज॒ वर्च॑सा । \newline
62. सृ॒ज॒ वर्च॑सा॒ वर्च॑सा सृज सृज॒ वर्च॑सा । \newline
63. वर्च॒सेति॒ वर्च॑सा । \newline
\pagebreak
\markright{ TS 1.4.46.1  \hfill https://www.vedavms.in \hfill}
\addcontentsline{toc}{section}{ TS 1.4.46.1 }
\section*{ TS 1.4.46.1 }

\textbf{TS 1.4.46.1 } \newline
\textbf{Samhita Paata} \newline

यस्त्वा॑ हृ॒दा की॒रिणा॒ मन्य॑मा॒नो ऽम॑र्त्यं॒ मर्त्यो॒ जोह॑वीमि । जात॑वेदो॒ यशो॑ अ॒स्मासु॑ धेहि प्र॒जाभि॑रग्ने अमृत॒त्वम॑श्यां ॥ यस्मै॒ त्वꣳ सु॒कृते॑ जातवेद॒ उ लो॒कम॑ग्ने कृ॒णवः॑ स्यो॒नं । अ॒श्विनꣳ॒॒ स पु॒त्रिणं॑ वी॒रव॑न्तं॒ गोम॑न्तꣳ र॒यिं न॑शते स्व॒स्ति ॥ त्वे सु पु॑त्र शव॒सोऽवृ॑त्र॒न् काम॑कातयः । न त्वामि॒न्द्राति॑ रिच्यते ॥ उ॒क्थौ॑क्थे॒ सोम॒ इन्द्रं॑ ममाद नी॒थेनी॑थे म॒घवा॑नꣳ - [ ] \newline

\textbf{Pada Paata} \newline

यः । त्वा॒ । हृ॒दा । की॒रिणा᳚ । मन्य॑मानः । अम॑र्त्यम् । मर्त्यः॑ । जोह॑वीमि ॥ जात॑वेद॒ इति॒ जात॑ - वे॒दः॒ । यशः॑ । अ॒स्मासु॑ । धे॒हि॒ । प्र॒जाभि॒रिति॑ प्र - जाभिः॑ । अ॒ग्ने॒ । अ॒मृ॒त॒त्वमित्य॑मृत - त्वम् । अ॒श्या॒म् ॥ यस्मै᳚ । त्वम् । सु॒कृत॒ इति॑ सु - कृते᳚ । जा॒त॒वे॒द॒ इति॑ जात-वे॒दः॒ । उ । लो॒कम् । अ॒ग्ने॒ । कृ॒णवः॑ । स्यो॒नम् ॥ अ॒श्विन᳚म् । सः । पु॒त्रिण᳚म् । वी॒रव॑न्त॒मिति॑ वी॒र - व॒न्त॒म् । गोम॑न्त॒मिति॒ गो - म॒न्त॒म् । र॒यिम् । न॒श॒ते॒ । स्व॒स्ति ॥ त्वे इति॑ । स्विति॑ । पु॒त्र॒ । श॒व॒सः॒ । अवृ॑त्रन्न् । काम॑कातय॒ इति॒ काम॑ - का॒त॒यः॒ ॥ न । त्वाम् । इ॒न्द्र॒ । अतीति॑ । रि॒च्य॒ते॒ ॥ उ॒क्थ‌उ॑क्थ॒ इत्यु॒क्थे - उ॒क्थे॒ । सोमः॑ । इन्द्र᳚म् । म॒मा॒द॒ । नी॒थेनी॑थ॒ इति॑ नी॒थे - नी॒थे॒ । म॒घवा॑न॒मिति॑ म॒घ - वा॒न॒म् ।  \newline


\textbf{Krama Paata} \newline

यस्त्वा᳚ । त्वा॒ हृ॒दा । हृ॒दा की॒रिणा᳚ । की॒रिणा॒ मन्य॑मानः । मन्य॑मा॒नोऽम॑र्त्यम् । अम॑र्त्य॒म् मर्त्यः॑ । मर्त्यो॒ जोह॑वीमि । जोह॑वी॒मीति॒ जोह॑वीमि ॥ जात॑वेदो॒ यशः॑ । जात॑वेद॒ इति॒ जात॑ - वे॒दः॒ । यशो॑ अ॒स्मासु॑ । अ॒स्मासु॑ धेहि । धे॒हि॒ प्र॒जाभिः॑ । प्र॒जाभि॑रग्ने । प्र॒जाभि॒रिति॑ प्र - जाभिः॑ । अ॒ग्ने॒ अ॒मृ॒त॒त्वम् । अ॒मृ॒त॒त्वम॑श्याम् । अ॒मृ॒त॒त्वमित्य॑मृत - त्वम् । अ॒श्या॒मित्य॑श्याम् ॥ यस्मै॒ त्वम् । त्वꣳ सु॒कृते᳚ । सु॒कृते॑ जातवेदः । सु॒कृत॒ इति॑ सु - कृते᳚ । जा॒त॒वे॒द॒ उ । जा॒त॒वे॒द॒ इति॑ जात - वे॒दः॒ । उ लो॒कम् । लो॒कम॑ग्ने । अ॒ग्ने॒ कृ॒णवः॑ । कृ॒णवः॑ स्यो॒नम् । स्यो॒नमिति॑ स्यो॒नम् ॥ अ॒श्विनꣳ॒॒ सः । स पु॒त्रिण᳚म् । पु॒त्रिणं॑ ॅवी॒रव॑न्तम् । वी॒रव॑न्त॒म् गोम॑न्तम् । वी॒रव॑न्त॒मिति॑ वी॒र - व॒न्त॒म् । गोम॑न्तꣳ र॒यिम् । गोम॑न्त॒मिति॒ गो - म॒न्त॒म् । र॒यिम् न॑शते । न॒श॒ते॒ स्व॒स्ति । स्व॒स्तीति॑ स्व॒स्ति ॥ त्वे सु । त्वे इति॒ त्वे । सु पु॑त्र । पु॒त्र॒ श॒व॒सः॒ । श॒व॒सोऽवृ॑त्रन्न् । अवृ॑त्र॒न् काम॑कातयः । काम॑कातय॒ इति॒ काम॑ - का॒त॒यः॒ ॥ न त्वाम् । त्वामि॑न्द्र । इ॒न्द्राति॑ । अति॑ रिच्यते । रि॒च्य॒त॒ इति॑ रिच्यते ॥ उ॒क्थ,उ॑क्थे॒ सोमः॑ । उ॒क्थ,उ॑क्थ॒ इत्यु॒क्थे - उ॒क्थे॒ । सोम॒ इन्द्र᳚म् । इन्द्र॑म् ममाद । म॒मा॒द॒ नी॒थेनी॑थे । नी॒थेनी॑थे म॒घवा॑नम् । नी॒थेनी॑थ॒ इति॑ नी॒थे - नी॒थे॒ । म॒घवा॑नꣳ सु॒तासः॑ । म॒घवा॑न॒मिति॑ म॒घ - वा॒न॒म् \newline

\textbf{Jatai Paata} \newline

1. य स्त्वा᳚ त्वा॒ यो य स्त्वा᳚ । \newline
2. त्वा॒ हृ॒दा हृ॒दा त्वा᳚ त्वा हृ॒दा । \newline
3. हृ॒दा की॒रिणा॑ की॒रिणा॑ हृ॒दा हृ॒दा की॒रिणा᳚ । \newline
4. की॒रिणा॒ मन्य॑मानो॒ मन्य॑मानः की॒रिणा॑ की॒रिणा॒ मन्य॑मानः । \newline
5. मन्य॑मा॒नो ऽम॑र्त्य॒ मम॑र्त्य॒म् मन्य॑मानो॒ मन्य॑मा॒नो ऽम॑र्त्यम् । \newline
6. अम॑र्त्य॒म् मर्त्यो॒ मर्त्यो॒ अम॑र्त्य॒ मम॑र्त्य॒म् मर्त्यः॑ । \newline
7. मर्त्यो॒ जोह॑वीमि॒ जोह॑वीमि॒ मर्त्यो॒ मर्त्यो॒ जोह॑वीमि । \newline
8. जोह॑वी॒मीति॒ जोह॑वीमि । \newline
9. जात॑वेदो॒ यशो॒ यशो॒ जात॑वेदो॒ जात॑वेदो॒ यशः॑ । \newline
10. जात॑वेद॒ इति॒ जात॑ - वे॒दः॒ । \newline
11. यशो॑ अ॒स्मास्व॒स्मासु॒ यशो॒ यशो॑ अ॒स्मासु॑ । \newline
12. अ॒स्मासु॑ धेहि धेह्य॒स्मास्व॒स्मासु॑ धेहि । \newline
13. धे॒हि॒ प्र॒जाभिः॑ प्र॒जाभि॑र् धेहि धेहि प्र॒जाभिः॑ । \newline
14. प्र॒जाभि॑रग्ने अग्ने प्र॒जाभिः॑ प्र॒जाभि॑रग्ने । \newline
15. प्र॒जाभि॒रिति॑ प्र - जाभिः॑ । \newline
16. अ॒ग्ने॒ अ॒मृ॒त॒त्व म॑मृत॒त्व म॑ग्ने अग्ने अमृत॒त्वम् । \newline
17. अ॒मृ॒त॒त्व म॑श्या मश्या ममृत॒त्व म॑मृत॒त्व म॑श्याम् । \newline
18. अ॒मृ॒त॒त्वमित्य॑मृत - त्वम् । \newline
19. अ॒श्या॒मित्य॑श्याम् । \newline
20. यस्मै॒ त्वम् त्वं ॅयस्मै॒ यस्मै॒ त्वम् । \newline
21. त्वꣳ सु॒कृते॑ सु॒कृते॒ त्वम् त्वꣳ सु॒कृते᳚ । \newline
22. सु॒कृते॑ जातवेदो जातवेदः सु॒कृते॑ सु॒कृते॑ जातवेदः । \newline
23. सु॒कृत॒ इति॑ सु - कृते᳚ । \newline
24. जा॒त॒वे॒द॒ उ वु जा॑तवेदो जातवेद॒ उ । \newline
25. जा॒त॒वे॒द॒ इति॑ जात - वे॒दः॒ । \newline
26. उ लो॒कम् ॅलो॒क मु वु लो॒कम् । \newline
27. लो॒क म॑ग्ने अग्ने लो॒कम् ॅलो॒क म॑ग्ने । \newline
28. अ॒ग्ने॒ कृ॒णवः॑ कृ॒णवो॑ अग्ने अग्ने कृ॒णवः॑ । \newline
29. कृ॒णवः॑ स्यो॒नꣳ स्यो॒नम् कृ॒णवः॑ कृ॒णवः॑ स्यो॒नम् । \newline
30. स्यो॒नमिति॑ स्यो॒नम् । \newline
31. अ॒श्विन॒(ग्म्॒) स सो अ॒श्विन॑ म॒श्विन॒(ग्म्॒) सः । \newline
32. स पु॒त्रिण॑म् पु॒त्रिण॒(ग्म्॒) स स पु॒त्रिण᳚म् । \newline
33. पु॒त्रिणं॑ ॅवी॒रव॑न्तं ॅवी॒रव॑न्तम् पु॒त्रिण॑म् पु॒त्रिणं॑ ॅवी॒रव॑न्तम् । \newline
34. वी॒रव॑न्त॒म् गोम॑न्त॒म् गोम॑न्तं ॅवी॒रव॑न्तं ॅवी॒रव॑न्त॒म् गोम॑न्तम् । \newline
35. वी॒रव॑न्त॒मिति॑ वी॒र - व॒न्त॒म् । \newline
36. गोम॑न्तꣳ र॒यिꣳ र॒यिम् गोम॑न्त॒म् गोम॑न्तꣳ र॒यिम् । \newline
37. गोम॑न्त॒मिति॒ गो - म॒न्त॒म् । \newline
38. र॒यिन्न॑शते नशते र॒यिꣳ र॒यिन्न॑शते । \newline
39. न॒श॒ते॒ स्व॒स्ति स्व॒स्ति न॑शते नशते स्व॒स्ति । \newline
40. स्व॒स्तीति॑ स्व॒स्ति । \newline
41. त्वे सु सु त्वे त्वे सु । \newline
42. त्वे इति॒ त्वे । \newline
43. सु पु॑त्र पुत्र॒ सु सु पु॑त्र । \newline
44. पु॒त्र॒ श॒व॒सः॒ श॒व॒सः॒ पु॒त्र॒ पु॒त्र॒ श॒व॒सः॒ । \newline
45. श॒व॒सो ऽवृ॑त्र॒न् नवृ॑त्रञ् छवसः शव॒सो ऽवृ॑त्रन्न् । \newline
46. अवृ॑त्र॒न् काम॑कातयः॒ काम॑कात॒यो ऽवृ॑त्र॒न् नवृ॑त्र॒न् काम॑कातयः । \newline
47. काम॑कातय॒ इति॒ काम॑ - का॒त॒यः॒ । \newline
48. न त्वाम् त्वान्न न त्वाम् । \newline
49. त्वा मि॑न्द्रे न्द्र॒ त्वाम् त्वा मि॑न्द्र । \newline
50. इ॒न्द्रात्यती᳚न्द्रे॒ न्द्राति॑ । \newline
51. अति॑ रिच्यते रिच्यते॒ अत्यति॑ रिच्यते । \newline
52. रि॒च्य॒त॒ इति॑ रिच्यते । \newline
53. उ॒क्थ‌उ॑क्थे॒ सोमः॒ सोम॑ उ॒क्थ‌उ॑क्थ उ॒क्थ‌उ॑क्थे॒ सोमः॑ । \newline
54. उ॒क्थ‌उ॑क्थ॒ इत्यु॒क्थे - उ॒क्थे॒ । \newline
55. सोम॒ इन्द्र॒ मिन्द्र॒(ग्म्॒) सोमः॒ सोम॒ इन्द्र᳚म् । \newline
56. इन्द्र॑म् ममाद ममा॒देन्द्र॒ मिन्द्र॑म् ममाद । \newline
57. म॒मा॒द॒ नी॒थेनी॑थे नी॒थेनी॑थे ममाद ममाद नी॒थेनी॑थे । \newline
58. नी॒थेनी॑थे म॒घवा॑नम् म॒घवा॑नम् नी॒थेनी॑थे नी॒थेनी॑थे म॒घवा॑नम् । \newline
59. नी॒थेनी॑थ॒ इति॑ नी॒थे - नी॒थे॒ । \newline
60. म॒घवा॑नꣳ सु॒तासः॑ सु॒तासो॑ म॒घवा॑नम् म॒घवा॑नꣳ सु॒तासः॑ । \newline
61. म॒घवा॑न॒मिति॑ म॒घ - वा॒न॒म् । \newline

\textbf{Ghana Paata } \newline

1. य स्त्वा᳚ त्वा॒ यो य स्त्वा॑ हृ॒दा हृ॒दा त्वा॒ यो य स्त्वा॑ हृ॒दा । \newline
2. त्वा॒ हृ॒दा हृ॒दा त्वा᳚ त्वा हृ॒दा की॒रिणा॑ की॒रिणा॑ हृ॒दा त्वा᳚ त्वा हृ॒दा की॒रिणा᳚ । \newline
3. हृ॒दा की॒रिणा॑ की॒रिणा॑ हृ॒दा हृ॒दा की॒रिणा॒ मन्य॑मानो॒ मन्य॑मानः की॒रिणा॑ हृ॒दा हृ॒दा की॒रिणा॒ मन्य॑मानः । \newline
4. की॒रिणा॒ मन्य॑मानो॒ मन्य॑मानः की॒रिणा॑ की॒रिणा॒ मन्य॑मा॒नो ऽम॑र्त्य॒ मम॑र्त्य॒म् मन्य॑मानः की॒रिणा॑ की॒रिणा॒ मन्य॑मा॒नो ऽम॑र्त्यम् । \newline
5. मन्य॑मा॒नो ऽम॑र्त्य॒ मम॑र्त्य॒म् मन्य॑मानो॒ मन्य॑मा॒नो ऽम॑र्त्य॒म् मर्त्यो॒ मर्त्यो॒ अम॑र्त्य॒म् मन्य॑मानो॒ मन्य॑मा॒नो ऽम॑र्त्य॒म् मर्त्यः॑ । \newline
6. अम॑र्त्य॒म् मर्त्यो॒ मर्त्यो॒ अम॑र्त्य॒ मम॑र्त्य॒म् मर्त्यो॒ जोह॑वीमि॒ जोह॑वीमि॒ मर्त्यो॒ अम॑र्त्य॒ मम॑र्त्य॒म् मर्त्यो॒ जोह॑वीमि । \newline
7. मर्त्यो॒ जोह॑वीमि॒ जोह॑वीमि॒ मर्त्यो॒ मर्त्यो॒ जोह॑वीमि । \newline
8. जोह॑वी॒मीति॒ जोह॑वीमि । \newline
9. जात॑वेदो॒ यशो॒ यशो॒ जात॑वेदो॒ जात॑वेदो॒ यशो॑ अ॒स्मा स्व॒स्मासु॒ यशो॒ जात॑वेदो॒ जात॑वेदो॒ यशो॑ अ॒स्मासु॑ । \newline
10. जात॑वेद॒ इति॒ जात॑ - वे॒दः॒ । \newline
11. यशो॑ अ॒स्मा स्व॒स्मासु॒ यशो॒ यशो॑ अ॒स्मासु॑ धेहि धेह्य॒स्मासु॒ यशो॒ यशो॑ अ॒स्मासु॑ धेहि । \newline
12. अ॒स्मासु॑ धेहि धेह्य॒ स्मास्व॒स्मासु॑ धेहि प्र॒जाभिः॑ प्र॒जाभि॑र् धेह्य॒ स्मास्व॒स्मासु॑ धेहि प्र॒जाभिः॑ । \newline
13. धे॒हि॒ प्र॒जाभिः॑ प्र॒जाभि॑र् धेहि धेहि प्र॒जाभि॑रग्ने अग्ने प्र॒जाभि॑र् धेहि धेहि प्र॒जाभि॑रग्ने । \newline
14. प्र॒जाभि॑रग्ने अग्ने प्र॒जाभिः॑ प्र॒जाभि॑रग्ने अमृत॒त्व म॑मृत॒त्व म॑ग्ने प्र॒जाभिः॑ प्र॒जाभि॑रग्ने अमृत॒त्वम् । \newline
15. प्र॒जाभि॒रिति॑ प्र - जाभिः॑ । \newline
16. अ॒ग्ने॒ अ॒मृ॒त॒त्व म॑मृत॒त्व म॑ग्ने अग्ने अमृत॒त्व म॑श्या मश्या ममृत॒त्व म॑ग्ने अग्ने अमृत॒त्व म॑श्याम् । \newline
17. अ॒मृ॒त॒त्व म॑श्या मश्या ममृत॒त्व म॑मृत॒त्व म॑श्याम् । \newline
18. अ॒मृ॒त॒त्वमित्य॑मृत - त्वम् । \newline
19. अ॒श्या॒मित्य॑श्याम् । \newline
20. यस्मै॒ त्वम् त्वं ॅयस्मै॒ यस्मै॒ त्वꣳ सु॒कृते॑ सु॒कृते॒ त्वं ॅयस्मै॒ यस्मै॒ त्वꣳ सु॒कृते᳚ । \newline
21. त्वꣳ सु॒कृते॑ सु॒कृते॒ त्वम् त्वꣳ सु॒कृते॑ जातवेदो जातवेदः सु॒कृते॒ त्वम् त्वꣳ सु॒कृते॑ जातवेदः । \newline
22. सु॒कृते॑ जातवेदो जातवेदः सु॒कृते॑ सु॒कृते॑ जातवेद॒ उ वु जा॑तवेदः सु॒कृते॑ सु॒कृते॑ जातवेद॒ उ । \newline
23. सु॒कृत॒ इति॑ सु - कृते᳚ । \newline
24. जा॒त॒वे॒द॒ उ वु जा॑तवेदो जातवेद॒ उ लो॒कम् ॅलो॒क मु जा॑तवेदो जातवेद॒ उ लो॒कम् । \newline
25. जा॒त॒वे॒द॒ इति॑ जात - वे॒दः॒ । \newline
26. उ लो॒कम् ॅलो॒क मु वु लो॒क म॑ग्ने अग्ने लो॒क मु वु लो॒क म॑ग्ने । \newline
27. लो॒क म॑ग्ने अग्ने लो॒कम् ॅलो॒क म॑ग्ने कृ॒णवः॑ कृ॒णवो॑ अग्ने लो॒कम् ॅलो॒क म॑ग्ने कृ॒णवः॑ । \newline
28. अ॒ग्ने॒ कृ॒णवः॑ कृ॒णवो॑ अग्ने अग्ने कृ॒णवः॑ स्यो॒नꣳ स्यो॒नम् कृ॒णवो॑ अग्ने अग्ने कृ॒णवः॑ स्यो॒नम् । \newline
29. कृ॒णवः॑ स्यो॒नꣳ स्यो॒नम् कृ॒णवः॑ कृ॒णवः॑ स्यो॒नम् । \newline
30. स्यो॒नमिति॑ स्यो॒नम् । \newline
31. अ॒श्विन॒(ग्म्॒) स सो अ॒श्विन॑ म॒श्विन॒(ग्म्॒) स पु॒त्रिण॑म् पु॒त्रिणꣳ॒॒ सो अ॒श्विन॑ म॒श्विन॒(ग्म्॒) स पु॒त्रिण᳚म् । \newline
32. स पु॒त्रिण॑म् पु॒त्रिण॒(ग्म्॒) स स पु॒त्रिणं॑ ॅवी॒रव॑न्तं ॅवी॒रव॑न्तम् पु॒त्रिण॒(ग्म्॒) स स पु॒त्रिणं॑ ॅवी॒रव॑न्तम् । \newline
33. पु॒त्रिणं॑ ॅवी॒रव॑न्तं ॅवी॒रव॑न्तम् पु॒त्रिण॑म् पु॒त्रिणं॑ ॅवी॒रव॑न्त॒म् गोम॑न्त॒म् गोम॑न्तं ॅवी॒रव॑न्तम् पु॒त्रिण॑म् पु॒त्रिणं॑ ॅवी॒रव॑न्त॒म् गोम॑न्तम् । \newline
34. वी॒रव॑न्त॒म् गोम॑न्त॒म् गोम॑न्तं ॅवी॒रव॑न्तं ॅवी॒रव॑न्त॒म् गोम॑न्तꣳ र॒यिꣳ र॒यिम् गोम॑न्तं ॅवी॒रव॑न्तं ॅवी॒रव॑न्त॒म् गोम॑न्तꣳ र॒यिम् । \newline
35. वी॒रव॑न्त॒मिति॑ वी॒र - व॒न्त॒म् । \newline
36. गोम॑न्तꣳ र॒यिꣳ र॒यिम् गोम॑न्त॒म् गोम॑न्तꣳ र॒यिम् न॑शते नशते र॒यिम् गोम॑न्त॒म् गोम॑न्तꣳ र॒यिम् न॑शते । \newline
37. गोम॑न्त॒मिति॒ गो - म॒न्त॒म् । \newline
38. र॒यिम् न॑शते नशते र॒यिꣳ र॒यिम् न॑शते स्व॒स्ति स्व॒स्ति न॑शते र॒यिꣳ र॒यिम् न॑शते स्व॒स्ति । \newline
39. न॒श॒ते॒ स्व॒स्ति स्व॒स्ति न॑शते नशते स्व॒स्ति । \newline
40. स्व॒स्तीति॑ स्व॒स्ति । \newline
41. त्वे सु सु त्वे त्वे सु पु॑त्र पुत्र॒ सु त्वे त्वे सु पु॑त्र । \newline
42. त्वे इति॒ त्वे । \newline
43. सु पु॑त्र पुत्र॒ सु सु पु॑त्र शवसः शवसः पुत्र॒ सु सु पु॑त्र शवसः । \newline
44. पु॒त्र॒ श॒व॒सः॒ श॒व॒सः॒ पु॒त्र॒ पु॒त्र॒ श॒व॒सो ऽवृ॑त्र॒न् नवृ॑त्रञ् छवसः पुत्र पुत्र शव॒सो ऽवृ॑त्रन्न् । \newline
45. श॒व॒सो ऽवृ॑त्र॒न् नवृ॑त्रञ् छवसः शव॒सो ऽवृ॑त्र॒न् काम॑कातयः॒ काम॑कात॒यो ऽवृ॑त्रञ् छवसः शव॒सो ऽवृ॑त्र॒न् काम॑कातयः । \newline
46. अवृ॑त्र॒न् काम॑कातयः॒ काम॑कात॒यो ऽवृ॑त्र॒न् नवृ॑त्र॒न् काम॑कातयः । \newline
47. काम॑कातय॒ इति॒ काम॑ - का॒त॒यः॒ । \newline
48. न त्वाम् त्वाम् न न त्वा मि॑न्द्रे न्द्र॒ त्वाम् न न त्वा मि॑न्द्र । \newline
49. त्वा मि॑न्द्रे न्द्र॒ त्वाम् त्वा मि॒न्द्रा त्यती᳚न्द्र॒ त्वाम् त्वा मि॒न्द्राति॑ । \newline
50. इ॒न्द्रा त्यती᳚न्द्रे॒ न्द्राति॑ रिच्यते रिच्यते॒ अती᳚न्द्रे॒ न्द्राति॑ रिच्यते । \newline
51. अति॑ रिच्यते रिच्यते॒ अत्यति॑ रिच्यते । \newline
52. रि॒च्य॒त॒ इति॑ रिच्यते । \newline
53. उ॒क्थ‍उ॑क्थे॒ सोमः॒ सोम॑ उ॒क्थ‍उ॑क्थ उ॒क्थ‍उ॑क्थे॒ सोम॒ इन्द्र॒ मिन्द्र॒(ग्म्॒) सोम॑ उ॒क्थ‍उ॑क्थ उ॒क्थ‍उ॑क्थे॒ सोम॒ इन्द्र᳚म् । \newline
54. उ॒क्थ‌उ॑क्थ॒ इत्यु॒क्थे - उ॒क्थे॒ । \newline
55. सोम॒ इन्द्र॒ मिन्द्र॒(ग्म्॒) सोमः॒ सोम॒ इन्द्र॑म् ममाद ममा॒दे न्द्र॒(ग्म्॒) सोमः॒ सोम॒ इन्द्र॑म् ममाद । \newline
56. इन्द्र॑म् ममाद ममा॒दे न्द्र॒ मिन्द्र॑म् ममाद नी॒थेनी॑थे नी॒थेनी॑थे ममा॒दे न्द्र॒ मिन्द्र॑म् ममाद नी॒थेनी॑थे । \newline
57. म॒मा॒द॒ नी॒थेनी॑थे नी॒थेनी॑थे ममाद ममाद नी॒थेनी॑थे म॒घवा॑नम् म॒घवा॑नम् नी॒थेनी॑थे ममाद ममाद नी॒थेनी॑थे म॒घवा॑नम् । \newline
58. नी॒थेनी॑थे म॒घवा॑नम् म॒घवा॑नम् नी॒थेनी॑थे नी॒थेनी॑थे म॒घवा॑नꣳ सु॒तासः॑ सु॒तासो॑ म॒घवा॑नम् नी॒थेनी॑थे नी॒थेनी॑थे म॒घवा॑नꣳ सु॒तासः॑ । \newline
59. नी॒थेनी॑थ॒ इति॑ नी॒थे - नी॒थे॒ । \newline
60. म॒घवा॑नꣳ सु॒तासः॑ सु॒तासो॑ म॒घवा॑नम् म॒घवा॑नꣳ सु॒तासः॑ । \newline
61. म॒घवा॑न॒मिति॑ म॒घ - वा॒न॒म् । \newline
\pagebreak
\markright{ TS 1.4.46.2  \hfill https://www.vedavms.in \hfill}
\addcontentsline{toc}{section}{ TS 1.4.46.2 }
\section*{ TS 1.4.46.2 }

\textbf{TS 1.4.46.2 } \newline
\textbf{Samhita Paata} \newline

सु॒तासः॑ । यदीꣳ॑ स॒बाधः॑ पि॒तरं॒ न पु॒त्राः स॑मा॒नद॑क्षा॒ अव॑से॒ हव॑न्ते ॥ अग्ने॒ रसे॑न॒ तेज॑सा॒ जात॑वेदो॒ वि रो॑चसे । र॒क्षो॒हाऽमी॑व॒चात॑नः ॥ अ॒पो अन्व॑चारिषꣳ॒॒ रसे॑न॒ सम॑सृक्ष्महि । पय॑स्वाꣳ अग्न॒ आऽग॑मं॒ तं मा॒ सꣳ सृ॑ज॒ वर्च॑सा ॥वसु॒र् वसु॑पति॒र्॒. हिक॒मस्य॑ग्ने वि॒भाव॑सुः । स्याम॑ ते सुम॒तावपि॑ ॥ त्वाम॑ग्ने॒ वसु॑पतिं॒ ॅवसू॑नाम॒भि प्र म॑न्दे-[ ] \newline

\textbf{Pada Paata} \newline

सु॒तासः॑ ॥ यत् । ई॒म् । स॒बाध॒ इति॑ स - बाधः॑ । पि॒तर᳚म् । न । पु॒त्राः । स॒मा॒नद॑क्षा॒ इति॑ समा॒न-द॒क्षाः॒ । अव॑से । हव॑न्ते ॥ अग्ने᳚ । रसे॑न । तेज॑सा । जात॑वेद॒ इति॒ जात॑ - वे॒दः॒ । वीति॑ । रो॒च॒से॒ ॥ र॒क्षो॒हेति॑ रक्षः - हा । अ॒मी॒व॒चात॑न॒ इत्य॑मीव - चात॑नः ॥ अ॒पः । अन्विति॑ । अ॒चा॒रि॒ष॒म् । रसे॑न । समिति॑ । अ॒सृ॒क्ष्म॒हि॒ ॥ पय॑स्वान् । अ॒ग्ने॒ । एति॑ । अ॒ग॒म॒म् । तम् । मा॒ । समिति॑ । सृ॒ज॒ । वर्च॑सा ॥ वसुः॑ । वसु॑पति॒रिति॒ वसु॑ - प॒तिः॒ । हिक᳚म् । असि॑ । अ॒ग्ने॒ । वि॒भाव॑सु॒रिति॑ वि॒भा - व॒सुः॒ ॥ स्याम॑ । ते॒ । सु॒म॒ताविति॑ सु-म॒तौ । अपि॑ ॥ त्वाम् । अ॒ग्ने॒ । वसु॑पति॒मिति॒ वसु॑ - प॒ति॒म् । वसू॑नाम् । अ॒भि । प्रेति॑ । म॒न्दे॒ ।  \newline


\textbf{Krama Paata} \newline

सु॒तास॒ इति॑ सु॒तासः॑ ॥ यदी᳚म् । ईꣳ॒॒ स॒बाधः॑ । स॒बाधः॑ पि॒तर᳚म् । स॒बाध॒ इति॑ स - बाधः॑ । पि॒तर॒न्न । न पु॒त्राः । पु॒त्राः स॑मा॒नद॑क्षाः । स॒मा॒नद॑क्षा॒ अव॑से । स॒मा॒नद॑क्षा॒ इति॑ समा॒न - द॒क्षाः॒ । अव॑से॒ हव॑न्ते । हव॑न्त॒ इति॒ हव॑न्ते ॥ अग्ने॒ रसे॑न । रसे॑न॒ तेज॑सा । तेज॑सा॒ जात॑वेदः । जात॑वेदो॒ वि । जात॑वेद॒ इति॒ जात॑ - वे॒दः॒ । वि रो॑चसे । रो॒च॒स॒ इति॑ रोचसे ॥ र॒क्षो॒हा ऽमी॑व॒चात॑नः । र॒क्षो॒हेति॑ रक्षः - हा । अ॒मी॒व॒चात॑न॒ इत्य॑मीव - चात॑नः ॥ अ॒पो अनु॑ । अन्व॑चारिषम् । अ॒चा॒रि॒षꣳ॒॒ रसे॑न । रसे॑न॒ सम् । सम॑सृक्ष्महि । अ॒सृ॒क्ष्म॒हीत्य॑सृक्ष्महि ॥ पय॑स्वाꣳ अग्ने । अ॒ग्न॒ आ । आऽग॑मम् । अ॒ग॒म॒न्तम् । तम् मा᳚ । मा॒ सम् । सꣳ सृ॑ज । सृ॒ज॒ वर्च॑सा । वर्च॒सेति॒ वर्च॑सा ॥ वसु॒र् वसु॑पतिः । वसु॑पति॒र्॒. हिक᳚म् । वसु॑पति॒रिति॒ वसु॑ - प॒तिः॒ । हिक॒मसि॑ । अस्य॑ग्ने । अ॒ग्ने॒ वि॒भाव॑सुः । वि॒भाव॑सु॒रिति॑ वि॒भा - व॒सुः॒ ॥ स्याम॑ ते । ते॒ सु॒म॒तौ । सु॒म॒तावपि॑ । सु॒म॒ताविति॑ सु - म॒तौ । अपीत्यपि॑ ॥ त्वाम॑ग्ने । अ॒ग्ने॒ वसु॑पतिम् । वसु॑पतिं॒ ॅवसू॑नाम् । वसु॑पति॒मिति॒ वसु॑ - प॒ति॒म् । वसू॑नाम॒भि । अ॒भि प्र । प्र म॑न्दे । म॒न्दे॒ अ॒द्ध्व॒रेषु॑ \newline

\textbf{Jatai Paata} \newline

1. सु॒तास॒ इति॑ सु॒तासः॑ । \newline
2. यदी॑ मीं॒ ॅयद् यदी᳚म् । \newline
3. ई॒(ग्म्॒) स॒बाधः॑ स॒बाध॑ ईमीꣳ स॒बाधः॑ । \newline
4. स॒बाधः॑ पि॒तर॑म् पि॒तर(ग्म्॑) स॒बाधः॑ स॒बाधः॑ पि॒तर᳚म् । \newline
5. स॒बाध॒ इति॑ स - बाधः॑ । \newline
6. पि॒तर॒म् न न पि॒तर॑म् पि॒तर॒म् न । \newline
7. न पु॒त्राः पु॒त्रा न न पु॒त्राः । \newline
8. पु॒त्राः स॑मा॒नद॑क्षाः समा॒नद॑क्षाः पु॒त्राः पु॒त्राः स॑मा॒नद॑क्षाः । \newline
9. स॒मा॒नद॑क्षा॒ अव॒से ऽव॑से समा॒नद॑क्षाः समा॒नद॑क्षा॒ अव॑से । \newline
10. स॒मा॒नद॑क्षा॒ इति॑ समा॒न - द॒क्षाः॒ । \newline
11. अव॑से॒ हव॑न्ते॒ हव॒न्ते ऽव॒से ऽव॑से॒ हव॑न्ते । \newline
12. हव॑न्त॒ इति॒ हव॑न्ते । \newline
13. अग्ने॒ रसे॑न॒ रसे॒नाग्ने ऽग्ने॒ रसे॑न । \newline
14. रसे॑न॒ तेज॑सा॒ तेज॑सा॒ रसे॑न॒ रसे॑न॒ तेज॑सा । \newline
15. तेज॑सा॒ जात॑वेदो॒ जात॑वेद॒ स्तेज॑सा॒ तेज॑सा॒ जात॑वेदः । \newline
16. जात॑वेदो॒ वि वि जात॑वेदो॒ जात॑वेदो॒ वि । \newline
17. जात॑वेद॒ इति॒ जात॑ - वे॒दः॒ । \newline
18. वि रो॑चसे रोचसे॒ वि वि रो॑चसे । \newline
19. रो॒च॒स॒ इति॑ रोचसे । \newline
20. र॒क्षो॒हा ऽमी॑व॒चात॑नो अमीव॒चात॑नो रक्षो॒हा र॑क्षो॒हा ऽमी॑व॒चात॑नः । \newline
21. र॒क्षो॒हेति॑ रक्षः - हा । \newline
22. अ॒मी॒व॒चात॑न॒ इत्य॑मीव - चात॑नः । \newline
23. अ॒पो अन्वन्व॒पो॑ ऽपो अनु॑ । \newline
24. अन्व॑चारिष मचारिष॒ मन्वन्व॑चारिषम् । \newline
25. अ॒चा॒रि॒ष॒(ग्म्॒) रसे॑न॒ रसे॑नाचारिष मचारिष॒(ग्म्॒) रसे॑न । \newline
26. रसे॑न॒ सꣳ सꣳ रसे॑न॒ रसे॑न॒ सम् । \newline
27. स म॑सृक्ष्मह्यसृक्ष्महि॒ सꣳ स म॑सृक्ष्महि । \newline
28. अ॒सृ॒क्ष्म॒हीत्य॑सृक्ष्महि । \newline
29. पय॑स्वाꣳ अग्ने ऽग्ने॒ पय॑स्वा॒न् पय॑स्वाꣳ अग्ने । \newline
30. अ॒ग्न॒ आ ऽग्ने᳚ ऽग्न॒ आ । \newline
31. आ ऽग॑म मगम॒ मा ऽग॑मम् । \newline
32. अ॒ग॒म॒म् तम् त म॑गम मगम॒म् तम् । \newline
33. तम् मा॑ मा॒ तम् तम् मा᳚ । \newline
34. मा॒ सꣳ सम् मा॑ मा॒ सम् । \newline
35. सꣳ सृ॑ज सृज॒ सꣳ सꣳ सृ॑ज । \newline
36. सृ॒ज॒ वर्च॑सा॒ वर्च॑सा सृज सृज॒ वर्च॑सा । \newline
37. वर्च॒सेति॒ वर्च॑सा । \newline
38. वसु॒र् वसु॑पति॒र् वसु॑पति॒र् वसु॒र् वसु॒र् वसु॑पतिः । \newline
39. वसु॑पति॒र्॒. हिक॒(ग्म्॒) हिकं॒ ॅवसु॑पति॒र् वसु॑पति॒र्॒. हिक᳚म् । \newline
40. वसु॑पति॒रिति॒ वसु॑ - प॒तिः॒ । \newline
41. हिक॒ मस्यसि॒ हिक॒(ग्म्॒) हिक॒ मसि॑ । \newline
42. अस्य॑ग्ने अ॒ग्ने ऽस्यस्य॑ग्ने । \newline
43. अ॒ग्ने॒ वि॒भाव॑सुर् वि॒भाव॑सु रग्ने अग्ने वि॒भाव॑सुः । \newline
44. वि॒भाव॑सु॒रिति॑ वि॒भा - व॒सुः॒ । \newline
45. स्याम॑ ते ते॒ स्याम॒ स्याम॑ ते । \newline
46. ते॒ सु॒म॒तौ सु॑म॒तौ ते॑ ते सुम॒तौ । \newline
47. सु॒म॒ता वप्यपि॑ सुम॒तौ सु॑म॒ता वपि॑ । \newline
48. सु॒म॒ताविति॑ सु - म॒तौ । \newline
49. अपीत्यपि॑ । \newline
50. त्वा म॑ग्ने अग्ने॒ त्वाम् त्वा म॑ग्ने । \newline
51. अ॒ग्ने॒ वसु॑पतिं॒ ॅवसु॑पति मग्ने अग्ने॒ वसु॑पतिम् । \newline
52. वसु॑पतिं॒ ॅवसू॑नां॒ ॅवसू॑नां॒ ॅवसु॑पतिं॒ ॅवसु॑पतिं॒ ॅवसू॑नाम् । \newline
53. वसु॑पति॒मिति॒ वसु॑ - प॒ति॒म् । \newline
54. वसू॑ना म॒भ्य॑भि वसू॑नां॒ ॅवसू॑ना म॒भि । \newline
55. अ॒भि प्र प्राभ्य॑भि प्र । \newline
56. प्र म॑न्दे मन्दे॒ प्र प्र म॑न्दे । \newline
57. म॒न्दे॒ अ॒द्ध्व॒रे ष्व॑द्ध्व॒रेषु॑ मन्दे मन्दे अद्ध्व॒रेषु॑ । \newline

\textbf{Ghana Paata } \newline

1. सु॒तास॒ इति॑ सु॒तासः॑ । \newline
2. यदी॑ मीं॒ ॅयद् यदी(ग्म्॑) स॒बाधः॑ स॒बाध॑ ईं॒ ॅयद् यदी(ग्म्॑) स॒बाधः॑ । \newline
3. ई॒(ग्म्॒) स॒बाधः॑ स॒बाध॑ ई मीꣳ स॒बाधः॑ पि॒तर॑म् पि॒तर(ग्म्॑) स॒बाध॑ ई मीꣳ स॒बाधः॑ पि॒तर᳚म् । \newline
4. स॒बाधः॑ पि॒तर॑म् पि॒तर(ग्म्॑) स॒बाधः॑ स॒बाधः॑ पि॒तर॒म् न न पि॒तर(ग्म्॑) स॒बाधः॑ स॒बाधः॑ पि॒तर॒म् न । \newline
5. स॒बाध॒ इति॑ स - बाधः॑ । \newline
6. पि॒तर॒म् न न पि॒तर॑म् पि॒तर॒म् न पु॒त्राः पु॒त्रा न पि॒तर॑म् पि॒तर॒म् न पु॒त्राः । \newline
7. न पु॒त्राः पु॒त्रा न न पु॒त्राः स॑मा॒नद॑क्षाः समा॒नद॑क्षाः पु॒त्रा न न पु॒त्राः स॑मा॒नद॑क्षाः । \newline
8. पु॒त्राः स॑मा॒नद॑क्षाः समा॒नद॑क्षाः पु॒त्राः पु॒त्राः स॑मा॒नद॑क्षा॒ अव॒से ऽव॑से समा॒नद॑क्षाः पु॒त्राः पु॒त्राः स॑मा॒नद॑क्षा॒ अव॑से । \newline
9. स॒मा॒नद॑क्षा॒ अव॒से ऽव॑से समा॒नद॑क्षाः समा॒नद॑क्षा॒ अव॑से॒ हव॑न्ते॒ हव॒न्ते ऽव॑से समा॒नद॑क्षाः समा॒नद॑क्षा॒ अव॑से॒ हव॑न्ते । \newline
10. स॒मा॒नद॑क्षा॒ इति॑ समा॒न - द॒क्षाः॒ । \newline
11. अव॑से॒ हव॑न्ते॒ हव॒न्ते ऽव॒से ऽव॑से॒ हव॑न्ते । \newline
12. हव॑न्त॒ इति॒ हव॑न्ते । \newline
13. अग्ने॒ रसे॑न॒ रसे॒नाग्ने ऽग्ने॒ रसे॑न॒ तेज॑सा॒ तेज॑सा॒ रसे॒नाग्ने ऽग्ने॒ रसे॑न॒ तेज॑सा । \newline
14. रसे॑न॒ तेज॑सा॒ तेज॑सा॒ रसे॑न॒ रसे॑न॒ तेज॑सा॒ जात॑वेदो॒ जात॑वेद॒ स्तेज॑सा॒ रसे॑न॒ रसे॑न॒ तेज॑सा॒ जात॑वेदः । \newline
15. तेज॑सा॒ जात॑वेदो॒ जात॑वेद॒ स्तेज॑सा॒ तेज॑सा॒ जात॑वेदो॒ वि वि जात॑वेद॒ स्तेज॑सा॒ तेज॑सा॒ जात॑वेदो॒ वि । \newline
16. जात॑वेदो॒ वि वि जात॑वेदो॒ जात॑वेदो॒ वि रो॑चसे रोचसे॒ वि जात॑वेदो॒ जात॑वेदो॒ वि रो॑चसे । \newline
17. जात॑वेद॒ इति॒ जात॑ - वे॒दः॒ । \newline
18. वि रो॑चसे रोचसे॒ वि वि रो॑चसे । \newline
19. रो॒च॒स॒ इति॑ रोचसे । \newline
20. र॒क्षो॒हा ऽमी॑व॒चात॑नो अमीव॒चात॑नो रक्षो॒हा र॑क्षो॒हा ऽमी॑व॒चात॑नः । \newline
21. र॒क्षो॒हेति॑ रक्षः - हा । \newline
22. अ॒मी॒व॒चात॑न॒ इत्य॑मीव - चात॑नः । \newline
23. अ॒पो अन्वन्व॒पो॑ ऽपो अन्व॑चारिष मचारिष॒ मन्व॒पो॑ ऽपो अन्व॑चारिषम् । \newline
24. अन्व॑चारिष मचारिष॒ मन्व न्व॑चारिष॒(ग्म्॒) रसे॑न॒ रसे॑नाचारिष॒ मन्व न्व॑चारिष॒(ग्म्॒) रसे॑न । \newline
25. अ॒चा॒रि॒ष॒(ग्म्॒) रसे॑न॒ रसे॑नाचारिष मचारिष॒(ग्म्॒) रसे॑न॒ सꣳ सꣳ रसे॑नाचारिष मचारिष॒(ग्म्॒) रसे॑न॒ सम् । \newline
26. रसे॑न॒ सꣳ सꣳ रसे॑न॒ रसे॑न॒ स म॑सृक्ष्म ह्यसृक्ष्महि॒ सꣳ रसे॑न॒ रसे॑न॒ स म॑सृक्ष्महि । \newline
27. स म॑सृक्ष् मह्यसृक्ष्महि॒ सꣳ स म॑सृक्ष्महि । \newline
28. अ॒सृ॒क्ष्म॒हीत्य॑सृक्ष्महि । \newline
29. पय॑स्वाꣳ अग्ने ऽग्ने॒ पय॑स्वा॒न् पय॑स्वाꣳ अग्न॒ आ ऽग्ने॒ पय॑स्वा॒न् पय॑स्वाꣳ अग्न॒ आ । \newline
30. अ॒ग्न॒ आ ऽग्ने᳚ ऽग्न॒ आ ऽग॑म मगम॒ मा ऽग्ने᳚ ऽग्न॒ आ ऽग॑मम् । \newline
31. आ ऽग॑म मगम॒ मा ऽग॑म॒म् तम् त म॑गम॒ मा ऽग॑म॒म् तम् । \newline
32. अ॒ग॒म॒म् तम् त म॑गम मगम॒म् तम् मा॑ मा॒ त म॑गम मगम॒म् तम् मा᳚ । \newline
33. तम् मा॑ मा॒ तम् तम् मा॒ सꣳ सम् मा॒ तम् तम् मा॒ सम् । \newline
34. मा॒ सꣳ सम् मा॑ मा॒ सꣳ सृ॑ज सृज॒ सम् मा॑ मा॒ सꣳ सृ॑ज । \newline
35. सꣳ सृ॑ज सृज॒ सꣳ सꣳ सृ॑ज॒ वर्च॑सा॒ वर्च॑सा सृज॒ सꣳ सꣳ सृ॑ज॒ वर्च॑सा । \newline
36. सृ॒ज॒ वर्च॑सा॒ वर्च॑सा सृज सृज॒ वर्च॑सा । \newline
37. वर्च॒सेति॒ वर्च॑सा । \newline
38. वसु॒र् वसु॑पति॒र् वसु॑पति॒र् वसु॒र् वसु॒र् वसु॑पति॒र्॒. हिकꣳ॒॒ हिकं॒ ॅवसु॑पति॒र् वसु॒र् वसु॒र् वसु॑पति॒र्॒. हिक᳚म् । \newline
39. वसु॑पति॒र्॒. हिकꣳ॒॒ हिकं॒ ॅवसु॑पति॒र् वसु॑पति॒र्॒. हिक॒ मस्यसि॒ हिकं॒ ॅवसु॑पति॒र् वसु॑पति॒र्॒. हिक॒ मसि॑ । \newline
40. वसु॑पति॒रिति॒ वसु॑ - प॒तिः॒ । \newline
41. हिक॒ मस्यसि॒ हिकꣳ॒॒ हिक॒ मस्य॑ग्ने अ॒ग्ने ऽसि॒ हिकꣳ॒॒ हिक॒ मस्य॑ग्ने । \newline
42. अस्य॑ग्ने अ॒ग्ने ऽस्यस्य॑ग्ने वि॒भाव॑सुर् वि॒भाव॑सुर॒ग्ने ऽस्यस्य॑ग्ने वि॒भाव॑सुः । \newline
43. अ॒ग्ने॒ वि॒भाव॑सुर् वि॒भाव॑सु रग्ने अग्ने वि॒भाव॑सुः । \newline
44. वि॒भाव॑सु॒रिति॑ वि॒भा - व॒सुः॒ । \newline
45. स्याम॑ ते ते॒ स्याम॒ स्याम॑ ते सुम॒तौ सु॑म॒तौ ते॒ स्याम॒ स्याम॑ ते सुम॒तौ । \newline
46. ते॒ सु॒म॒तौ सु॑म॒तौ ते॑ ते सुम॒ता वप्यपि॑ सुम॒तौ ते॑ ते सुम॒ता वपि॑ । \newline
47. सु॒म॒ता वप्यपि॑ सुम॒तौ सु॑म॒ता वपि॑ । \newline
48. सु॒म॒ताविति॑ सु - म॒तौ । \newline
49. अपीत्यपि॑ । \newline
50. त्वा म॑ग्ने अग्ने॒ त्वाम् त्वा म॑ग्ने॒ वसु॑पतिं॒ ॅवसु॑पति मग्ने॒ त्वाम् त्वा म॑ग्ने॒ वसु॑पतिम् । \newline
51. अ॒ग्ने॒ वसु॑पतिं॒ ॅवसु॑पति मग्ने अग्ने॒ वसु॑पतिं॒ ॅवसू॑नां॒ ॅवसू॑नां॒ ॅवसु॑पति मग्ने अग्ने॒ वसु॑पतिं॒ ॅवसू॑नाम् । \newline
52. वसु॑पतिं॒ ॅवसू॑नां॒ ॅवसू॑नां॒ ॅवसु॑पतिं॒ ॅवसु॑पतिं॒ ॅवसू॑ना म॒भ्य॑भि वसू॑नां॒ ॅवसु॑पतिं॒ ॅवसु॑पतिं॒ ॅवसू॑ना म॒भि । \newline
53. वसु॑पति॒मिति॒ वसु॑ - प॒ति॒म् । \newline
54. वसू॑ना म॒भ्य॑भि वसू॑नां॒ ॅवसू॑ना म॒भि प्र प्राभि वसू॑नां॒ ॅवसू॑ना म॒भि प्र । \newline
55. अ॒भि प्र प्राभ्य॑भि प्र म॑न्दे मन्दे॒ प्राभ्य॑भि प्र म॑न्दे । \newline
56. प्र म॑न्दे मन्दे॒ प्र प्र म॑न्दे अद्ध्व॒रे ष्व॑द्ध्व॒रेषु॑ मन्दे॒ प्र प्र म॑न्दे अद्ध्व॒रेषु॑ । \newline
57. म॒न्दे॒ अ॒द्ध्व॒रे ष्व॑द्ध्व॒रेषु॑ मन्दे मन्दे अद्ध्व॒रेषु॑ राजन् राजन् नद्ध्व॒रेषु॑ मन्दे मन्दे अद्ध्व॒रेषु॑ राजन्न् । \newline
\pagebreak
\markright{ TS 1.4.46.3  \hfill https://www.vedavms.in \hfill}
\addcontentsline{toc}{section}{ TS 1.4.46.3 }
\section*{ TS 1.4.46.3 }

\textbf{TS 1.4.46.3 } \newline
\textbf{Samhita Paata} \newline

अद्ध्व॒रेषु॑ राजन्न्  । त्वया॒ वाजं॑ ॅवाज॒यन्तो॑ जयेमा॒-ऽभिष्या॑म पृथ्सु॒तीर् मर्त्या॑नां । त्वाम॑ग्ने वाज॒सात॑मं॒ ॅविप्रा॑ वर्द्धन्ति॒ सुष्टु॑तं । स नो॑ रास्व सु॒वीर्यं᳚ ॥ अ॒यं नो॑ अ॒ग्निर्वरि॑वः कृणोत्व॒यं मृधः॑ पु॒र ए॑तु प्रभि॒न्दन्न्  ॥ अ॒यꣳ शत्रू᳚ञ्जयतु॒ जर्.हृ॑षाणो॒ऽयं ॅवाजं॑ जयतु॒ वाज॑सातौ ॥ अ॒ग्निना॒ऽग्निः समि॑द्ध्यते क॒विर् गृ॒हप॑ति॒र् युवा᳚ । ह॒व्य॒वाड्-जु॒ह्वा᳚स्यः ॥ त्वꣳ ह्य॑ग्ने ( ) अ॒ग्निना॒ विप्रो॒ विप्रे॑ण॒ सन्थ्स॒ता । सखा॒ सख्या॑ समि॒द्ध्यसे᳚ ॥उद॑ग्ने॒ शुच॑य॒स्तव॒>1, वि ज्योति॑षा>2 ॥ \newline

\textbf{Pada Paata} \newline

अ॒द्ध्व॒रेषु॑ । रा॒ज॒न्न् ॥ त्वया᳚ । वाज᳚म् । वा॒ज॒यन्त॒ इति॑ वाज - यन्तः॑ । ज॒ये॒म॒ । अ॒भीति॑ । स्या॒म॒ । पृ॒थ्सु॒तीः । मर्त्या॑नाम् ॥ त्वाम् । अ॒ग्ने॒ । वा॒ज॒सात॑म॒मिति॑ वाज - सात॑मम् । विप्राः᳚ । व॒र्ध॒न्ति॒ । सुष्टु॑त॒मिति॒ सु - स्तु॒त॒म् ॥ सः । नः॒ । रा॒स्व॒ । सु॒वीर्य॒मिति॑ सु - वीर्य᳚म् ॥ अ॒यम् । नः॒ । अ॒ग्निः । वरि॑वः । कृ॒णो॒तू॒ । अ॒यम् । मृधः॑ । पु॒रः । ए॒तु॒ । प्र॒भि॒न्दन्निति॑ प्र - भि॒न्दन्न् ॥ अ॒यम् । शत्रून्॑ । ज॒य॒तु॒ । जर्.हृ॑षाणः । अ॒यम् । वाज᳚म् । ज॒य॒तु॒ । वाज॑सात॒विति॒ वाज॑-सा॒तौ॒ ॥ अ॒ग्निना᳚ । अ॒ग्निः । समिति॑ । इ॒द्ध्य॒ते॒ । क॒विः । गृ॒हप॑ति॒रिति॑ गृ॒ह - प॒तिः॒ । युवा᳚ ॥ ह॒व्य॒वाडिति॑ हव्य - वाट् । जु॒ह्वा᳚स्य॒ इति॑ जु॒हु - आ॒स्यः॒ ॥ त्वम् । हि । अ॒ग्ने॒ ( ) । अ॒ग्निना᳚ । विप्रः॑ । विप्रे॑ण । सन्न् । स॒ता ॥ सखा᳚ । सख्या᳚ । स॒मि॒द्ध्यस॒ इति॑ सम् - इ॒द्ध्यसे᳚ ॥ उदिति॑ । अ॒ग्ने॒ । शुच॑यः । तव॑ । वीति॑ । ज्योति॑षा ॥  \newline


\textbf{Krama Paata} \newline

अ॒द्ध्व॒रेषु॑ राजन्न् । रा॒ज॒न्निति॑ राजन्न् ॥ त्वया॒ वाज᳚म् । वाजं॑ ॅवाज॒यन्तः॑ । वा॒ज॒यन्तो॑ जयेम । वा॒ज॒यन्त॒ इति॑ वाज - यन्तः॑ । ज॒ये॒मा॒भि । अ॒भिष्या॑म । स्या॒म॒ पृ॒थ्सु॒तीः । पृ॒थ्सु॒तीर्,मर्त्या॑नाम् । मर्त्या॑ना॒मिति॒ मर्त्या॑नाम् ॥ त्वाम॑ग्ने । अ॒ग्ने॒ वा॒ज॒सात॑मम् । वा॒ज॒सात॑मं॒ ॅविप्राः᳚ । वा॒ज॒सात॑म॒मिति॑ वाज - सात॑मम् । विप्रा॑ वर्द्धन्ति । व॒र्द्ध॒न्ति॒ सुष्टु॑तम् । सुष्टु॑त॒मिति॒ सु - स्तु॒त॒म् ॥ स नः॑ । नो॒ रा॒स्व॒ । रा॒स्व॒ सु॒वीर्य᳚म् । सु॒वीर्य॒मिति॑ सु - वीर्य᳚म् ॥ अ॒यम् नः॑ । नो॒ अ॒ग्निः । अ॒ग्निर्,वरि॑वः । वरि॑वः कृणोतु । कृ॒णो॒त्व॒यम् । अ॒यम् मृधः॑ । मृधः॑ पु॒रः । पु॒र ए॑तु । ए॒तु॒ प्र॒भि॒न्दन्न् । प्र॒भि॒न्दन्निति॑ प्र - भि॒न्दन्न् ॥ अ॒यꣳ शत्रून्॑ । शत्रू᳚न् जयतु । ज॒य॒तु॒ जर्.हृ॑षाणः । जर्.हृ॑षाणो॒ऽयम् । अ॒यं ॅवाज᳚म् । वाज॑म् जयतु । ज॒य॒तु॒ वाज॑सातौ । वाज॑साता॒विति॒ वाज॑ - सा॒तौ॒ ॥ अ॒ग्निना॒ऽग्निः । अ॒ग्निः सम् । समि॑द्ध्यते । इ॒द्ध्य॒ते॒ क॒विः । क॒विर् गृ॒हप॑तिः । गृ॒हप॑ति॒र् युवा᳚ । गृ॒हप॑ति॒रिति॑ गृ॒ह - प॒तिः॒ । युवेति॒ युवा᳚ ॥ ह॒व्य॒वाड्,जु॒ह्वा᳚स्यः । ह॒व्य॒वाडिति॑ हव्य - वाट् । जु॒ह्वा᳚स्य॒ इति॑ जु॒हु - आ॒स्यः॒ ॥ त्वꣳ हि । ह्य॑ग्ने ( ) । अ॒ग्ने॒ अ॒ग्निना᳚ । अ॒ग्निना॒ विप्रः॑ । विप्रो॒ विप्रे॑ण । विप्रे॑ण॒ सन्न् । सन्थ् स॒ता । स॒तेति॑ स॒ता ॥ सखा॒ सख्या᳚ । सख्या॑ समि॒द्ध्यसे᳚ । स॒मि॒द्ध्यस॒ इति॑ सं - इ॒द्ध्यसे᳚ ॥ उद॑ग्ने । अ॒ग्ने॒ शुच॑यः । शुच॑य॒स्तव॑ । तव॒ वि । वि ज्योति॑षा । ज्योति॒षेति॒ ज्योति॑षा । \newline

\textbf{Jatai Paata} \newline

1. अ॒द्ध्व॒रेषु॑ राजन् राजन् नद्ध्व॒रे ष्व॑द्ध्व॒रेषु॑ राजन्न् । \newline
2. रा॒ज॒न्निति॑ राजन्न् । \newline
3. त्वया॒ वाजं॒ ॅवाज॒म् त्वया॒ त्वया॒ वाज᳚म् । \newline
4. वाजं॑ ॅवाज॒यन्तो॑ वाज॒यन्तो॒ वाजं॒ ॅवाजं॑ ॅवाज॒यन्तः॑ । \newline
5. वा॒ज॒यन्तो॑ जयेम जयेम वाज॒यन्तो॑ वाज॒यन्तो॑ जयेम । \newline
6. वा॒ज॒यन्त॒ इति॑ वाज - यन्तः॑ । \newline
7. ज॒ये॒मा॒भ्य॑भि ज॑येम जयेमा॒भि । \newline
8. अ॒भि ष्या॑म स्यामा॒भ्य॑भि ष्या॑म । \newline
9. स्या॒म॒ पृ॒थ्सु॒तीः पृ॑थ्सु॒तीः स्या॑म स्याम पृथ्सु॒तीः । \newline
10. पृ॒थ्सु॒तीर् मर्त्या॑ना॒म् मर्त्या॑नाम् पृथ्सु॒तीः पृ॑थ्सु॒तीर् मर्त्या॑नाम् । \newline
11. मर्त्या॑ना॒मिति॒ मर्त्या॑नाम् । \newline
12. त्वा म॑ग्ने अग्ने॒ त्वाम् त्वा म॑ग्ने । \newline
13. अ॒ग्ने॒ वा॒ज॒सात॑मं ॅवाज॒सात॑म मग्ने अग्ने वाज॒सात॑मम् । \newline
14. वा॒ज॒सात॑मं॒ ॅविप्रा॒ विप्रा॑ वाज॒सात॑मं ॅवाज॒सात॑मं॒ ॅविप्राः᳚ । \newline
15. वा॒ज॒सात॑म॒मिति॑ वाज - सात॑मम् । \newline
16. विप्रा॑ वर्द्धन्ति वर्द्धन्ति॒ विप्रा॒ विप्रा॑ वर्द्धन्ति । \newline
17. व॒र्द्ध॒न्ति॒ सुष्टु॑त॒(ग्म्॒) सुष्टु॑तं ॅवर्द्धन्ति वर्द्धन्ति॒ सुष्टु॑तम् । \newline
18. सुष्टु॑त॒मिति॒ सु - स्तु॒त॒म् । \newline
19. स नो॑ नः॒ स स नः॑ । \newline
20. नो॒ रा॒स्व॒ रा॒स्व॒ नो॒ नो॒ रा॒स्व॒ । \newline
21. रा॒स्व॒ सु॒वीर्य(ग्म्॑) सु॒वीर्य(ग्म्॑) रास्व रास्व सु॒वीर्य᳚म् । \newline
22. सु॒वीर्य॒मिति॑ सु - वीर्य᳚म् । \newline
23. अ॒यन्नो॑ नो॒ ऽय म॒यन्नः॑ । \newline
24. नो॒ अ॒ग्नि र॒ग्निर् नो॑ नो अ॒ग्निः । \newline
25. अ॒ग्निर् वरि॑वो॒ वरि॑वो॒ ऽग्निर॒ग्निर् वरि॑वः । \newline
26. वरि॑वः कृणोतु कृणोतु॒ वरि॑वो॒ वरि॑वः कृणोतु । \newline
27. कृ॒णो॒त्व॒य म॒यम् कृ॑णोतु कृणोत्व॒यम् । \newline
28. अ॒यम् मृधो॒ मृधो॒ ऽय म॒यम् मृधः॑ । \newline
29. मृधः॑ पु॒रः पु॒रो मृधो॒ मृधः॑ पु॒रः । \newline
30. पु॒र ए᳚त्वेतु पु॒रः पु॒र ए॑तु । \newline
31. ए॒तु॒ प्र॒भि॒न्दन् प्र॑भि॒न्दन् ने᳚त्वेतु प्रभि॒न्दन्न् । \newline
32. प्र॒भि॒न्दन्निति॑ प्र - भि॒न्दन्न् । \newline
33. अ॒यꣳ शत्रू॒ञ् छत्रू॑ न॒य म॒यꣳ शत्रून्॑ । \newline
34. शत्रू᳚न् जयतु जयतु॒ शत्रू॒ञ् छत्रू᳚न् जयतु । \newline
35. ज॒य॒तु॒ जर्.हृ॑षाणो॒ जर्.हृ॑षाणो जयतु जयतु॒ जर्.हृ॑षाणः । \newline
36. जर्.हृ॑षाणो॒ ऽय म॒यम् जर्.हृ॑षाणो॒ जर्.हृ॑षाणो॒ ऽयम् । \newline
37. अ॒यं ॅवाजं॒ ॅवाज॑ म॒य म॒यं ॅवाज᳚म् । \newline
38. वाज॑म् जयतु जयतु॒ वाजं॒ ॅवाज॑म् जयतु । \newline
39. ज॒य॒तु॒ वाज॑सातौ॒ वाज॑सातौ जयतु जयतु॒ वाज॑सातौ । \newline
40. वाज॑सात॒विति॒ वाज॑ - सा॒तौ॒ । \newline
41. अ॒ग्निना॒ ऽग्निर॒ग्नि र॒ग्निना॒ ऽग्निना॒ ऽग्निः । \newline
42. अ॒ग्निः सꣳ स म॒ग्निर॒ग्निः सम् । \newline
43. स मि॑द्ध्यत इद्ध्यते॒ सꣳ स मि॑द्ध्यते । \newline
44. इ॒द्ध्य॒ते॒ क॒विः क॒वि रि॑द्ध्यत इद्ध्यते क॒विः । \newline
45. क॒विर् गृ॒हप॑तिर् गृ॒हप॑तिः क॒विः क॒विर् गृ॒हप॑तिः । \newline
46. गृ॒हप॑ति॒र् युवा॒ युवा॑ गृ॒हप॑तिर् गृ॒हप॑ति॒र् युवा᳚ । \newline
47. गृ॒हप॑ति॒रिति॑ गृ॒ह - प॒तिः॒ । \newline
48. युवेति॒ युवा᳚ । \newline
49. ह॒व्य॒वाड् जु॒ह्वा᳚स्यो जु॒ह्वा᳚स्यो हव्य॒वाड्ढ॑व्य॒वाड् जु॒ह्वा᳚स्यः । \newline
50. ह॒व्य॒वाडिति॑ हव्य - वाट् । \newline
51. जु॒ह्वा᳚स्य॒ इति॑ जु॒हु - आ॒स्यः॒ । \newline
52. त्वꣳहि हि त्वम् त्वꣳहि । \newline
53. ह्य॑ग्ने अग्ने॒ हि ह्य॑ग्ने । \newline
54. अ॒ग्ने॒ अ॒ग्निना॒ ऽग्निना᳚ ऽग्ने अग्ने अ॒ग्निना᳚ । \newline
55. अ॒ग्निना॒ विप्रो॒ विप्रो॑ अ॒ग्निना॒ ऽग्निना॒ विप्रः॑ । \newline
56. विप्रो॒ विप्रे॑ण॒ विप्रे॑ण॒ विप्रो॒ विप्रो॒ विप्रे॑ण । \newline
57. विप्रे॑ण॒ सन् थ्सन्. विप्रे॑ण॒ विप्रे॑ण॒ सन्न् । \newline
58. सन् थ्स॒ता स॒ता सन् थ्सन् थ्स॒ता । \newline
59. स॒तेति॑ स॒ता । \newline
60. सखा॒ सख्या॒ सख्या॒ सखा॒ सखा॒ सख्या᳚ । \newline
61. सख्या॑ समि॒द्ध्यसे॑ समि॒द्ध्यसे॒ सख्या॒ सख्या॑ समि॒द्ध्यसे᳚ । \newline
62. स॒मि॒द्ध्यस॒ इति॑ सम् - इ॒द्ध्यसे᳚ । \newline
63. उद॑ग्ने अग्न॒ उदुद॑ग्ने । \newline
64. अ॒ग्ने॒ शुच॑यः॒ शुच॑यो अग्ने अग्ने॒ शुच॑यः । \newline
65. शुच॑य॒ स्तव॒ तव॒ शुच॑यः॒ शुच॑य॒ स्तव॑ । \newline
66. तव॒ वि वि तव॒ तव॒ वि । \newline
67. वि ज्योति॑षा॒ ज्योति॑षा॒ वि वि ज्योति॑षा । \newline
68. ज्योति॒षेति॒ ज्योति॑षा । \newline

\textbf{Ghana Paata } \newline

1. अ॒द्ध्व॒रेषु॑ राजन् राजन् नद्ध्व॒रे ष्व॑द्ध्व॒रेषु॑ राजन्न् । \newline
2. रा॒ज॒न्निति॑ राजन्न् । \newline
3. त्वया॒ वाजं॒ ॅवाज॒म् त्वया॒ त्वया॒ वाजं॑ ॅवाज॒यन्तो॑ वाज॒यन्तो॒ वाज॒म् त्वया॒ त्वया॒ वाजं॑ ॅवाज॒यन्तः॑ । \newline
4. वाजं॑ ॅवाज॒यन्तो॑ वाज॒यन्तो॒ वाजं॒ ॅवाजं॑ ॅवाज॒यन्तो॑ जयेम जयेम वाज॒यन्तो॒ वाजं॒ ॅवाजं॑ ॅवाज॒यन्तो॑ जयेम । \newline
5. वा॒ज॒यन्तो॑ जयेम जयेम वाज॒यन्तो॑ वाज॒यन्तो॑ जयेमा॒भ्य॑भि ज॑येम वाज॒यन्तो॑ वाज॒यन्तो॑ जयेमा॒भि । \newline
6. वा॒ज॒यन्त॒ इति॑ वाज - यन्तः॑ । \newline
7. ज॒ये॒मा॒भ्य॑भि ज॑येम जयेमा॒भि ष्या॑म स्यामा॒भि ज॑येम जयेमा॒भि ष्या॑म । \newline
8. अ॒भि ष्या॑म स्यामा॒भ्य॑भि ष्या॑म पृथ्सु॒तीः पृ॑थ्सु॒तीः स्या॑मा॒भ्य॑भि ष्या॑म पृथ्सु॒तीः । \newline
9. स्या॒म॒ पृ॒थ्सु॒तीः पृ॑थ्सु॒तीः स्या॑म स्याम पृथ्सु॒तीर् मर्त्या॑ना॒म् मर्त्या॑नाम् पृथ्सु॒तीः स्या॑म स्याम पृथ्सु॒तीर् मर्त्या॑नाम् । \newline
10. पृ॒थ्सु॒तीर् मर्त्या॑ना॒म् मर्त्या॑नाम् पृथ्सु॒तीः पृ॑थ्सु॒तीर् मर्त्या॑नाम् । \newline
11. मर्त्या॑ना॒मिति॒ मर्त्या॑नाम् । \newline
12. त्वा म॑ग्ने अग्ने॒ त्वाम् त्वा म॑ग्ने वाज॒सात॑मं ॅवाज॒सात॑म मग्ने॒ त्वाम् त्वा म॑ग्ने वाज॒सात॑मम् । \newline
13. अ॒ग्ने॒ वा॒ज॒सात॑मं ॅवाज॒सात॑म मग्ने अग्ने वाज॒सात॑मं॒ ॅविप्रा॒ विप्रा॑ वाज॒सात॑म मग्ने अग्ने वाज॒सात॑मं॒ ॅविप्राः᳚ । \newline
14. वा॒ज॒सात॑मं॒ ॅविप्रा॒ विप्रा॑ वाज॒सात॑मं ॅवाज॒सात॑मं॒ ॅविप्रा॑ वर्द्धन्ति वर्द्धन्ति॒ विप्रा॑ वाज॒सात॑मं ॅवाज॒सात॑मं॒ ॅविप्रा॑ वर्द्धन्ति । \newline
15. वा॒ज॒सात॑म॒मिति॑ वाज - सात॑मम् । \newline
16. विप्रा॑ वर्द्धन्ति वर्द्धन्ति॒ विप्रा॒ विप्रा॑ वर्द्धन्ति॒ सुष्टु॑त॒(ग्म्॒) सुष्टु॑तं ॅवर्द्धन्ति॒ विप्रा॒ विप्रा॑ वर्द्धन्ति॒ सुष्टु॑तम् । \newline
17. व॒र्द्ध॒न्ति॒ सुष्टु॑त॒(ग्म्॒) सुष्टु॑तं ॅवर्द्धन्ति वर्द्धन्ति॒ सुष्टु॑तम् । \newline
18. सुष्टु॑त॒मिति॒ सु - स्तु॒त॒म् । \newline
19. स नो॑ नः॒ स स नो॑ रास्व रास्व नः॒ स स नो॑ रास्व । \newline
20. नो॒ रा॒स्व॒ रा॒स्व॒ नो॒ नो॒ रा॒स्व॒ सु॒वीर्य(ग्म्॑) सु॒वीर्य(ग्म्॑) रास्व नो नो रास्व सु॒वीर्य᳚म् । \newline
21. रा॒स्व॒ सु॒वीर्य(ग्म्॑) सु॒वीर्य(ग्म्॑) रास्व रास्व सु॒वीर्य᳚म् । \newline
22. सु॒वीर्य॒मिति॑ सु - वीर्य᳚म् । \newline
23. अ॒यम् नो॑ नो॒ ऽय म॒यम् नो॑ अ॒ग्नि र॒ग्निर् नो॒ ऽय म॒यम् नो॑ अ॒ग्निः । \newline
24. नो॒ अ॒ग्नि र॒ग्निर् नो॑ नो अ॒ग्निर् वरि॑वो॒ वरि॑वो॒ ऽग्निर् नो॑ नो अ॒ग्निर् वरि॑वः । \newline
25. अ॒ग्निर् वरि॑वो॒ वरि॑वो॒ ऽग्नि र॒ग्निर् वरि॑वः कृणोतु कृणोतु॒ वरि॑वो॒ ऽग्नि र॒ग्निर् वरि॑वः कृणोतु । \newline
26. वरि॑वः कृणोतु कृणोतु॒ वरि॑वो॒ वरि॑वः कृणोत्व॒य म॒यम् कृ॑णोतु॒ वरि॑वो॒ वरि॑वः कृणोत्व॒यम् । \newline
27. कृ॒णो॒त्व॒य म॒यम् कृ॑णोतु कृणोत्व॒यम् मृधो॒ मृधो॒ ऽयम् कृ॑णोतु कृणोत्व॒यम् मृधः॑ । \newline
28. अ॒यम् मृधो॒ मृधो॒ ऽय म॒यम् मृधः॑ पु॒रः पु॒रो मृधो॒ ऽय म॒यम् मृधः॑ पु॒रः । \newline
29. मृधः॑ पु॒रः पु॒रो मृधो॒ मृधः॑ पु॒र ए᳚त्वेतु पु॒रो मृधो॒ मृधः॑ पु॒र ए॑तु । \newline
30. पु॒र ए᳚त्वेतु पु॒रः पु॒र ए॑तु प्रभि॒न्दन् प्र॑भि॒न्दन् ने॑तु पु॒रः पु॒र ए॑तु प्रभि॒न्दन्न् । \newline
31. ए॒तु॒ प्र॒भि॒न्दन् प्र॑भि॒न्दन् ने᳚त्वेतु प्रभि॒न्दन्न् । \newline
32. प्र॒भि॒न्दन्निति॑ प्र - भि॒न्दन्न् । \newline
33. अ॒यꣳ शत्रू॒ञ् छत्रू॑ न॒य म॒यꣳ शत्रू᳚न् जयतु जयतु॒ शत्रू॑ न॒य म॒यꣳ शत्रू᳚न् जयतु । \newline
34. शत्रू᳚न् जयतु जयतु॒ शत्रू॒ञ् छत्रू᳚न् जयतु॒ जर्.हृ॑षाणो॒ जर्.हृ॑षाणो जयतु॒ शत्रू॒ञ् छत्रू᳚न् जयतु॒ जर्.हृ॑षाणः । \newline
35. ज॒य॒तु॒ जर्.हृ॑षाणो॒ जर्.हृ॑षाणो जयतु जयतु॒ जर्.हृ॑षाणो॒ ऽय म॒यम् जर्.हृ॑षाणो जयतु जयतु॒ जर्.हृ॑षाणो॒ ऽयम् । \newline
36. जर्.हृ॑षाणो॒ ऽय म॒यम् जर्.हृ॑षाणो॒ जर्.हृ॑षाणो॒ ऽयं ॅवाजं॒ ॅवाज॑ म॒यम् जर्.हृ॑षाणो॒ जर्.हृ॑षाणो॒ ऽयं ॅवाज᳚म् । \newline
37. अ॒यं ॅवाजं॒ ॅवाज॑ म॒य म॒यं ॅवाज॑म् जयतु जयतु॒ वाज॑ म॒य म॒यं ॅवाज॑म् जयतु । \newline
38. वाज॑म् जयतु जयतु॒ वाजं॒ ॅवाज॑म् जयतु॒ वाज॑सातौ॒ वाज॑सातौ जयतु॒ वाजं॒ ॅवाज॑म् जयतु॒ वाज॑सातौ । \newline
39. ज॒य॒तु॒ वाज॑सातौ॒ वाज॑सातौ जयतु जयतु॒ वाज॑सातौ । \newline
40. वाज॑सात॒विति॒ वाज॑ - सा॒तौ॒ । \newline
41. अ॒ग्निना॒ ऽग्नि र॒ग्नि र॒ग्निना॒ ऽग्निना॒ ऽग्निः सꣳ स म॒ग्नि र॒ग्निना॒ ऽग्निना॒ ऽग्निः सम् । \newline
42. अ॒ग्निः सꣳ स म॒ग्नि र॒ग्निः स मि॑द्ध्यत इद्ध्यते॒ स म॒ग्नि र॒ग्निः स मि॑द्ध्यते । \newline
43. स मि॑द्ध्यत इद्ध्यते॒ सꣳ स मि॑द्ध्यते क॒विः क॒वि रि॑द्ध्यते॒ सꣳ स मि॑द्ध्यते क॒विः । \newline
44. इ॒द्ध्य॒ते॒ क॒विः क॒वि रि॑द्ध्यत इद्ध्यते क॒विर् गृ॒हप॑तिर् गृ॒हप॑तिः क॒वि रि॑द्ध्यत इद्ध्यते क॒विर् गृ॒हप॑तिः । \newline
45. क॒विर् गृ॒हप॑तिर् गृ॒हप॑तिः क॒विः क॒विर् गृ॒हप॑ति॒र् युवा॒ युवा॑ गृ॒हप॑तिः क॒विः क॒विर् गृ॒हप॑ति॒र् युवा᳚ । \newline
46. गृ॒हप॑ति॒र् युवा॒ युवा॑ गृ॒हप॑तिर् गृ॒हप॑ति॒र् युवा᳚ । \newline
47. गृ॒हप॑ति॒रिति॑ गृ॒ह - प॒तिः॒ । \newline
48. युवेति॒ युवा᳚ । \newline
49. ह॒व्य॒वाड् जु॒ह्वा᳚स्यो जु॒ह्वा᳚स्यो हव्य॒वाड्ढ॑व्य॒वाड् जु॒ह्वा᳚स्यः । \newline
50. ह॒व्य॒वाडिति॑ हव्य - वाट् । \newline
51. जु॒ह्वा᳚स्य॒ इति॑ जु॒हु - आ॒स्यः॒ । \newline
52. त्वꣳहि हि त्वम् त्वꣳ ह्य॑ग्ने अग्ने॒ हि त्वम् त्वꣳ ह्य॑ग्ने । \newline
53. ह्य॑ग्ने अग्ने॒ हि ह्य॑ग्ने अ॒ग्निना॒ ऽग्निना᳚ ऽग्ने॒ हि ह्य॑ग्ने अ॒ग्निना᳚ । \newline
54. अ॒ग्ने॒ अ॒ग्निना॒ ऽग्निना᳚ ऽग्ने अग्ने अ॒ग्निना॒ विप्रो॒ विप्रो॑ अ॒ग्निना᳚ ऽग्ने अग्ने अ॒ग्निना॒ विप्रः॑ । \newline
55. अ॒ग्निना॒ विप्रो॒ विप्रो॑ अ॒ग्निना॒ ऽग्निना॒ विप्रो॒ विप्रे॑ण॒ विप्रे॑ण॒ विप्रो॑ अ॒ग्निना॒ ऽग्निना॒ विप्रो॒ विप्रे॑ण । \newline
56. विप्रो॒ विप्रे॑ण॒ विप्रे॑ण॒ विप्रो॒ विप्रो॒ विप्रे॑ण॒ सन् थ्सन्. विप्रे॑ण॒ विप्रो॒ विप्रो॒ विप्रे॑ण॒ सन्न् । \newline
57. विप्रे॑ण॒ सन् थ्सन्. विप्रे॑ण॒ विप्रे॑ण॒ सन् थ्स॒ता स॒ता सन्. विप्रे॑ण॒ विप्रे॑ण॒ सन् थ्स॒ता । \newline
58. सन् थ्स॒ता स॒ता सन् थ्सन् थ्स॒ता । \newline
59. स॒तेति॑ स॒ता । \newline
60. सखा॒ सख्या॒ सख्या॒ सखा॒ सखा॒ सख्या॑ समि॒द्ध्यसे॑ समि॒द्ध्यसे॒ सख्या॒ सखा॒ सखा॒ सख्या॑ समि॒द्ध्यसे᳚ । \newline
61. सख्या॑ समि॒द्ध्यसे॑ समि॒द्ध्यसे॒ सख्या॒ सख्या॑ समि॒द्ध्यसे᳚ । \newline
62. स॒मि॒द्ध्यस॒ इति॑ सम् - इ॒द्ध्यसे᳚ । \newline
63. उद॑ग्ने अग्न॒ उदुद॑ग्ने॒ शुच॑यः॒ शुच॑यो अग्न॒ उदुद॑ग्ने॒ शुच॑यः । \newline
64. अ॒ग्ने॒ शुच॑यः॒ शुच॑यो अग्ने अग्ने॒ शुच॑य॒ स्तव॒ तव॒ शुच॑यो अग्ने अग्ने॒ शुच॑य॒ स्तव॑ । \newline
65. शुच॑य॒ स्तव॒ तव॒ शुच॑यः॒ शुच॑य॒ स्तव॒ वि वि तव॒ शुच॑यः॒ शुच॑य॒ स्तव॒ वि । \newline
66. तव॒ वि वि तव॒ तव॒ वि ज्योति॑षा॒ ज्योति॑षा॒ वि तव॒ तव॒ वि ज्योति॑षा । \newline
67. वि ज्योति॑षा॒ ज्योति॑षा॒ वि वि ज्योति॑षा । \newline
68. ज्योति॒षेति॒ ज्योति॑षा । \newline
\pagebreak


\end{document}