\documentclass[17pt]{extarticle}
\usepackage{babel}
\usepackage{fontspec}
\usepackage{polyglossia}
\usepackage{extsizes}

\usepackage{color}   %May be necessary if you want to color links
\usepackage{hyperref}
\hypersetup{
    colorlinks=true, %set true if you want colored links
    linktoc=all,     %set to all if you want both sections and subsections linked
    linkcolor=black,  %choose some color if you want links to stand out
}

\setmainlanguage{sanskrit}
\setotherlanguages{english} %% or other languages
\setlength{\parindent}{0pt}
\pagestyle{myheadings}
\newfontfamily\devanagarifont[Script=Devanagari]{AdishilaVedic}
\renewcommand{\theHsection}{\thepart.section.\thesection}

\newcommand{\VAR}[1]{}
\newcommand{\BLOCK}[1]{}




\begin{document}
\begin{titlepage}
    \begin{center}
 
\begin{sanskrit}
    { \Large
    कृष्ण यजुर्वेदीय तैत्तिरीय संहिता,पद,जटा,घन पाठः 
    }
    \\
    \vspace{2.5cm}
    \mbox{ \Large
    7.4      सप्तमकाण्डे चतुर्थः प्रश्नः - सत्रकर्मनिरूपणं   }
\end{sanskrit}
\end{center}

\end{titlepage}
\tableofcontents
\phantomsection
\pagebreak

\markright{ TS 7.4.1.1  \hfill https://www.vedavms.in \hfill}

\section{ TS 7.4.1.1 }

\textbf{TS 7.4.1.1 } \newline
\textbf{Samhita Paata} \newline

बृह॒स्पति॑रकामयत॒ श्रन्मे॑ दे॒वा दधी॑र॒न् गच्छे॑यं पुरो॒धामिति॒ स ए॒तं च॑तुर्विꣳशतिरा॒त्र-म॑पश्य॒त् तमाऽह॑र॒त् तेना॑यजत॒ ततो॒ वै तस्मै॒ श्रद्दे॒वा अद॑ध॒ताग॑च्छत् पुरो॒धां ॅय ए॒वं ॅवि॒द्वाꣳस॑-श्चतुर्विꣳशतिरा॒त्रमास॑ते॒ श्रदे᳚भ्यो मनु॒ष्या॑ दधते॒ गच्छ॑न्ति पुरो॒धां ज्योति॒र्गौरायु॒रिति॑ त्र्य॒हा भ॑वन्ती॒यं ॅवाव ज्योति॑र॒न्तरि॑क्षं॒ गौर॒सावायु॑ - [  ] \newline

\textbf{Pada Paata} \newline

बृह॒स्पतिः॑ । अ॒का॒म॒य॒त॒ । श्रत् । मे॒ । दे॒वाः । दधी॑रन्न् । गच्छे॑यम् । पु॒रो॒धामिति॑ पुरः - धाम् । इति॑ । सः । ए॒तम् । च॒तु॒र्विꣳ॒॒श॒ति॒रा॒त्रमिति॑ चतुर्विꣳशति - रा॒त्रम् । अ॒प॒श्य॒त् । तम् । एति॑ । अ॒ह॒र॒त् । तेन॑ । अ॒य॒ज॒त॒ । ततः॑ । वै । तस्मै᳚ । श्रत् । दे॒वाः । अद॑धत । अग॑च्छत् । पु॒रो॒धामिति॑ पुरः - धाम् । ये । ए॒वम् । वि॒द्वाꣳसः॑ । च॒तु॒र्विꣳ॒॒श॒ति॒रा॒त्रमिति॑ चतुर्विꣳशति - रा॒त्रम् । आस॑ते । श्रत् । ए॒भ्यः॒ । म॒नु॒ष्याः᳚ । द॒ध॒ते॒ । गच्छ॑न्ति । पु॒रो॒धामिति॑ पुरः-धाम् । ज्योतिः॑ । गौः । आयुः॑ । इति॑ । त्र्य॒हा इति॑ त्रि - अ॒हाः । भ॒व॒न्ति॒ । इ॒यम् । वाव । ज्योतिः॑ । अ॒न्तरि॑क्षम् । गौः । अ॒सौ । आयुः॑ ।  \newline


\textbf{Krama Paata} \newline

बृह॒स्पति॑रकामयत । अ॒का॒म॒य॒त॒ श्रत् । श्रन् मे᳚ । मे॒ दे॒वाः । दे॒वा दधी॑रन्न् । दधी॑र॒न् गच्छे॑यम् । गच्छे॑यम् पुरो॒धाम् । पु॒रो॒धामिति॑ । पु॒रो॒धामिति॑ पुरः - धाम् । इति॒ सः । स ए॒तम् । ए॒तम् च॑तुर्विꣳशतिरा॒त्रम् । च॒तु॒र्विꣳ॒॒श॒ति॒रा॒त्रम॑पश्यत् । च॒तु॒र्विꣳ॒॒श॒ति॒रा॒त्रमिति॑ चतुर्विꣳशति - रा॒त्रम् । अ॒प॒श्य॒त् तम् । तमा । 
आऽह॑रत् । अ॒ह॒र॒त् तेन॑ । तेना॑यजत । अ॒य॒ज॒त॒ ततः॑ । ततो॒ वै । वै तस्मै᳚ । तस्मै॒ श्रत् । श्रद् दे॒वाः । दे॒वा अद॑धत । अद॑ध॒ताग॑च्छत् । अग॑च्छत् पुरो॒धाम् । पु॒रो॒धाम् ॅये । पु॒रो॒धामिति॑ पुरः - धाम् । य ए॒वम् । ए॒वम् ॅवि॒द्वाꣳसः॑ । वि॒द्वाꣳस॑श्चतुर्विꣳशतिरा॒त्रम् । च॒तु॒र्विꣳ॒॒श॒ति॒रा॒त्रमास॑ते । च॒तु॒र्विꣳ॒॒श॒ति॒रा॒त्रमिति॑ चतुर्विꣳशति - रा॒त्रम् । आस॑ते॒ श्रत् । श्रदे᳚भ्यः । ए॒भ्यो॒ म॒नु॒ष्याः᳚ । म॒नु॒ष्या॑ दधते । द॒ध॒ते॒ गच्छ॑न्ति । गच्छ॑न्ति पुरो॒धाम् । पु॒रो॒धाम् ज्योतिः॑ । पु॒रो॒धामिति॑ पुरः - धाम् । ज्योति॒र् गौः । गौरायुः॑ । आयु॒रिति॑ । इति॑ त्र्य॒हाः । त्र्य॒हा भ॑वन्ति । त्र्य॒हा इति॑ त्रि - अ॒हाः । भ॒व॒न्ती॒यम् । इ॒यम् ॅवाव । वाव ज्योतिः॑ । ज्योति॑र॒न्तरि॑क्षम् । अ॒न्तरि॑क्ष॒म् गौः । गौर॒सौ । अ॒सावायुः॑ । आयु॑रि॒मान् \newline

\textbf{Jatai Paata} \newline

1. बृह॒स्पति॑ रकामयता कामयत॒ बृह॒स्पति॒र् बृह॒स्पति॑ रकामयत । \newline
2. अ॒का॒म॒य॒त॒ श्रच्छ्रद॑कामयता कामयत॒ श्रत् । \newline
3. श्रन् मे॑ मे॒ श्रच्छ्रन् मे᳚ । \newline
4. मे॒ दे॒वा दे॒वा मे॑ मे दे॒वाः । \newline
5. दे॒वा दधी॑र॒न् दधी॑रन् दे॒वा दे॒वा दधी॑रन्न् । \newline
6. दधी॑र॒न् गच्छे॑य॒म् गच्छे॑य॒म् दधी॑र॒न् दधी॑र॒न् गच्छे॑यम् । \newline
7. गच्छे॑यम् पुरो॒धाम् पु॑रो॒धाम् गच्छे॑य॒म् गच्छे॑यम् पुरो॒धाम् । \newline
8. पु॒रो॒धा मितीति॑ पुरो॒धाम् पु॑रो॒धा मिति॑ । \newline
9. पु॒रो॒धामिति॑ पुरः - धाम् । \newline
10. इति॒ स स इतीति॒ सः । \newline
11. स ए॒त मे॒तꣳ स स ए॒तम् । \newline
12. ए॒तम् च॑तुर्विꣳशतिरा॒त्रम् च॑तुर्विꣳशतिरा॒त्र मे॒त मे॒तम् च॑तुर्विꣳशतिरा॒त्रम् । \newline
13. च॒तु॒र्विꣳ॒॒श॒ति॒रा॒त्र म॑पश्य दपश्यच् चतुर्विꣳशतिरा॒त्रम् च॑तुर्विꣳशतिरा॒त्र म॑पश्यत् । \newline
14. च॒तु॒र्विꣳ॒॒श॒ति॒रा॒त्रमिति॑ चतुर्विꣳशति - रा॒त्रम् । \newline
15. अ॒प॒श्य॒त् तम् त म॑पश्य दपश्य॒त् तम् । \newline
16. त मा तम् त मा । \newline
17. आ ऽह॑र दहर॒दा ऽह॑रत् । \newline
18. अ॒ह॒र॒त् तेन॒ तेना॑ हर दहर॒त् तेन॑ । \newline
19. तेना॑ यजता यजत॒ तेन॒ तेना॑ यजत । \newline
20. अ॒य॒ज॒त॒ तत॒ स्ततो॑ ऽयजता यजत॒ ततः॑ । \newline
21. ततो॒ वै वै तत॒ स्ततो॒ वै । \newline
22. वै तस्मै॒ तस्मै॒ वै वै तस्मै᳚ । \newline
23. तस्मै॒ श्रच्छ्रत् तस्मै॒ तस्मै॒ श्रत् । \newline
24. श्रद् दे॒वा दे॒वाः श्रच्छ्रद् दे॒वाः । \newline
25. दे॒वा अद॑ध॒ता द॑धत दे॒वा दे॒वा अद॑धत । \newline
26. अद॑ध॒ता ग॑च्छ॒ दग॑च्छ॒ दद॑ध॒ता द॑ध॒ता ग॑च्छत् । \newline
27. अग॑च्छत् पुरो॒धाम् पु॑रो॒धा मग॑च्छ॒ दग॑च्छत् पुरो॒धाम् । \newline
28. पु॒रो॒धां ॅये ये पु॑रो॒धाम् पु॑रो॒धां ॅये । \newline
29. पु॒रो॒धामिति॑ पुरः - धाम् । \newline
30. य ए॒व मे॒वं ॅये य ए॒वम् । \newline
31. ए॒वं ॅवि॒द्वाꣳसो॑ वि॒द्वाꣳस॑ ए॒व मे॒वं ॅवि॒द्वाꣳसः॑ । \newline
32. वि॒द्वाꣳस॑ श्चतुर्विꣳशतिरा॒त्रम् च॑तुर्विꣳशतिरा॒त्रं ॅवि॒द्वाꣳसो॑ वि॒द्वाꣳस॑ श्चतुर्विꣳशतिरा॒त्रम् । \newline
33. च॒तु॒र्विꣳ॒॒श॒ति॒रा॒त्र मास॑त॒ आस॑ते चतुर्विꣳशतिरा॒त्रम् च॑तुर्विꣳशतिरा॒त्र मास॑ते । \newline
34. च॒तु॒र्विꣳ॒॒श॒ति॒रा॒त्रमिति॑ चतुर्विꣳशति - रा॒त्रम् । \newline
35. आस॑ते॒ श्रच्छ्रदास॑त॒ आस॑ते॒ श्रत् । \newline
36. श्रदे᳚भ्य एभ्यः॒ श्रच्छ्र दे᳚भ्यः । \newline
37. ए॒भ्यो॒ म॒नु॒ष्या॑ मनु॒ष्या॑ एभ्य एभ्यो मनु॒ष्याः᳚ । \newline
38. म॒नु॒ष्या॑ दधते दधते मनु॒ष्या॑ मनु॒ष्या॑ दधते । \newline
39. द॒ध॒ते॒ गच्छ॑न्ति॒ गच्छ॑न्ति दधते दधते॒ गच्छ॑न्ति । \newline
40. गच्छ॑न्ति पुरो॒धाम् पु॑रो॒धाम् गच्छ॑न्ति॒ गच्छ॑न्ति पुरो॒धाम् । \newline
41. पु॒रो॒धाम् ज्योति॒र् ज्योतिः॑ पुरो॒धाम् पु॑रो॒धाम् ज्योतिः॑ । \newline
42. पु॒रो॒धामिति॑ पुरः - धाम् । \newline
43. ज्योति॒र् गौर् गौर् ज्योति॒र् ज्योति॒र् गौः । \newline
44. गौ रायु॒ रायु॒र् गौर् गौ रायुः॑ । \newline
45. आयु॒ रिती त्यायु॒ रायु॒ रिति॑ । \newline
46. इति॑ त्र्य॒हा स्त्र्य॒हा इतीति॑ त्र्य॒हाः । \newline
47. त्र्य॒हा भ॑वन्ति भवन्ति त्र्य॒हा स्त्र्य॒हा भ॑वन्ति । \newline
48. त्र्य॒हा इति॑ त्रि - अ॒हाः । \newline
49. भ॒व॒न्ती॒य मि॒यम् भ॑वन्ति भवन्ती॒यम् । \newline
50. इ॒यं ॅवाव वावेय मि॒यं ॅवाव । \newline
51. वाव ज्योति॒र् ज्योति॒र् वाव वाव ज्योतिः॑ । \newline
52. ज्योति॑ र॒न्तरि॑क्ष म॒न्तरि॑क्ष॒म् ज्योति॒र् ज्योति॑ र॒न्तरि॑क्षम् । \newline
53. अ॒न्तरि॑क्ष॒म् गौर् गौ र॒न्तरि॑क्ष म॒न्तरि॑क्ष॒म् गौः । \newline
54. गौ र॒सा व॒सौ गौर् गौ र॒सौ । \newline
55. अ॒सा वायु॒ रायु॑ र॒सा व॒सा वायुः॑ । \newline
56. आयु॑ रि॒मा नि॒मा नायु॒ रायु॑ रि॒मान् । \newline

\textbf{Ghana Paata } \newline

1. बृह॒स्पति॑ रकामयता कामयत॒ बृह॒स्पति॒र् बृह॒स्पति॑ रकामयत॒ श्रच्छ्रद॑कामयत॒ बृह॒स्पति॒र् बृह॒स्पति॑ रकामयत॒ श्रत् । \newline
2. अ॒का॒म॒य॒त॒ श्रच्छ्रद॑कामयता कामयत॒ श्रन् मे॑ मे॒ श्रद॑कामयता कामयत॒ श्रन् मे᳚ । \newline
3. श्रन् मे॑ मे॒ श्रच् छ्रन् मे॑ दे॒वा दे॒वा मे॒ श्रच् छ्रन् मे॑ दे॒वाः । \newline
4. मे॒ दे॒वा दे॒वा मे॑ मे दे॒वा दधी॑र॒न् दधी॑रन् दे॒वा मे॑ मे दे॒वा दधी॑रन्न् । \newline
5. दे॒वा दधी॑र॒न् दधी॑रन् दे॒वा दे॒वा दधी॑र॒न् गच्छे॑य॒म् गच्छे॑य॒म् दधी॑रन् दे॒वा दे॒वा दधी॑र॒न् गच्छे॑यम् । \newline
6. दधी॑र॒न् गच्छे॑य॒म् गच्छे॑य॒म् दधी॑र॒न् दधी॑र॒न् गच्छे॑यम् पुरो॒धाम् पु॑रो॒धाम् गच्छे॑य॒म् दधी॑र॒न् दधी॑र॒न् गच्छे॑यम् पुरो॒धाम् । \newline
7. गच्छे॑यम् पुरो॒धाम् पु॑रो॒धाम् गच्छे॑य॒म् गच्छे॑यम् पुरो॒धा मितीति॑ पुरो॒धाम् गच्छे॑य॒म् गच्छे॑यम् पुरो॒धा मिति॑ । \newline
8. पु॒रो॒धा मितीति॑ पुरो॒धाम् पु॑रो॒धा मिति॒ स स इति॑ पुरो॒धाम् पु॑रो॒धा मिति॒ सः । \newline
9. पु॒रो॒धामिति॑ पुरः - धाम् । \newline
10. इति॒ स स इतीति॒ स ए॒त मे॒तꣳ स इतीति॒ स ए॒तम् । \newline
11. स ए॒त मे॒तꣳ स स ए॒तम् च॑तुर्विꣳशतिरा॒त्रम् च॑तुर्विꣳशतिरा॒त्र मे॒तꣳ स स ए॒तम् च॑तुर्विꣳशतिरा॒त्रम् । \newline
12. ए॒तम् च॑तुर्विꣳशतिरा॒त्रम् च॑तुर्विꣳशतिरा॒त्र मे॒त मे॒तम् च॑तुर्विꣳशतिरा॒त्र म॑पश्य दपश्यच् चतुर्विꣳशतिरा॒त्र मे॒त मे॒तम् च॑तुर्विꣳशतिरा॒त्र म॑पश्यत् । \newline
13. च॒तु॒र्विꣳ॒॒श॒ति॒रा॒त्र म॑पश्य दपश्यच् चतुर्विꣳशतिरा॒त्रम् च॑तुर्विꣳशतिरा॒त्र म॑पश्य॒त् तम् तम॑पश्यच् चतुर्विꣳशतिरा॒त्रम् च॑तुर्विꣳशतिरा॒त्र म॑पश्य॒त् तम् । \newline
14. च॒तु॒र्विꣳ॒॒श॒ति॒रा॒त्रमिति॑ चतुर्विꣳशति - रा॒त्रम् । \newline
15. अ॒प॒श्य॒त् तम् त म॑पश्य दपश्य॒त् त मा त म॑पश्य दपश्य॒त् त मा । \newline
16. त मा तम् त मा ऽह॑र दहर॒दा तम् त मा ऽह॑रत् । \newline
17. आ ऽह॑र दहर॒दा ऽह॑र॒त् तेन॒ तेना॑ हर॒दा ऽह॑र॒त् तेन॑ । \newline
18. अ॒ह॒र॒त् तेन॒ तेना॑ हर दहर॒त् तेना॑ यजता यजत॒ तेना॑ हर दहर॒त् तेना॑ यजत । \newline
19. तेना॑ यजता यजत॒ तेन॒ तेना॑ यजत॒ तत॒ स्ततो॑ ऽयजत॒ तेन॒ तेना॑ यजत॒ ततः॑ । \newline
20. अ॒य॒ज॒त॒ तत॒ स्ततो॑ ऽयजता यजत॒ ततो॒ वै वै ततो॑ ऽयजता यजत॒ ततो॒ वै । \newline
21. ततो॒ वै वै तत॒ स्ततो॒ वै तस्मै॒ तस्मै॒ वै तत॒ स्ततो॒ वै तस्मै᳚ । \newline
22. वै तस्मै॒ तस्मै॒ वै वै तस्मै॒ श्रच् छ्रत् तस्मै॒ वै वै तस्मै॒ श्रत् । \newline
23. तस्मै॒ श्रच् छ्रत् तस्मै॒ तस्मै॒ श्रद् दे॒वा दे॒वाः श्रत् तस्मै॒ तस्मै॒ श्रद् दे॒वाः । \newline
24. श्रद् दे॒वा दे॒वाः श्रच् छ्रद् दे॒वा अद॑ध॒ता द॑धत दे॒वाः श्रच् छ्रद् दे॒वा अद॑धत । \newline
25. दे॒वा अद॑ध॒ता द॑धत दे॒वा दे॒वा अद॑ध॒ता ग॑च्छ॒ दग॑च्छ॒ दद॑धत दे॒वा दे॒वा अद॑ध॒ता ग॑च्छत् । \newline
26. अद॑ध॒ता ग॑च्छ॒ दग॑च्छ॒ दद॑ध॒ता द॑ध॒ता ग॑च्छत् पुरो॒धाम् पु॑रो॒धा मग॑च्छ॒ दद॑ध॒ता द॑ध॒ता ग॑च्छत् पुरो॒धाम् । \newline
27. अग॑च्छत् पुरो॒धाम् पु॑रो॒धा मग॑च्छ॒ दग॑च्छत् पुरो॒धां ॅये ये पु॑रो॒धा मग॑च्छ॒ दग॑च्छत् पुरो॒धां ॅये । \newline
28. पु॒रो॒धां ॅये ये पु॑रो॒धाम् पु॑रो॒धां ॅय ए॒व मे॒वं ॅये पु॑रो॒धाम् पु॑रो॒धां ॅय ए॒वम् । \newline
29. पु॒रो॒धामिति॑ पुरः - धाम् । \newline
30. य ए॒व मे॒वं ॅये य ए॒वं ॅवि॒द्वाꣳसो॑ वि॒द्वाꣳस॑ ए॒वं ॅये य ए॒वं ॅवि॒द्वाꣳसः॑ । \newline
31. ए॒वं ॅवि॒द्वाꣳसो॑ वि॒द्वाꣳस॑ ए॒व मे॒वं ॅवि॒द्वाꣳस॑ श्चतुर्विꣳशतिरा॒त्रम् च॑तुर्विꣳशतिरा॒त्रं ॅवि॒द्वाꣳस॑ ए॒व मे॒वं ॅवि॒द्वाꣳस॑ श्चतुर्विꣳशतिरा॒त्रम् । \newline
32. वि॒द्वाꣳस॑ श्चतुर्विꣳशतिरा॒त्रम् च॑तुर्विꣳशतिरा॒त्रं ॅवि॒द्वाꣳसो॑ वि॒द्वाꣳस॑ श्चतुर्विꣳशतिरा॒त्र मास॑त॒ आस॑ते चतुर्विꣳशतिरा॒त्रं ॅवि॒द्वाꣳसो॑ वि॒द्वाꣳस॑ श्चतुर्विꣳशतिरा॒त्र मास॑ते । \newline
33. च॒तु॒र्विꣳ॒॒श॒ति॒रा॒त्र मास॑त॒ आस॑ते चतुर्विꣳशतिरा॒त्रम् च॑तुर्विꣳशतिरा॒त्र मास॑ते॒ श्रच् छ्रदास॑ते चतुर्विꣳशतिरा॒त्रम् च॑तुर्विꣳशतिरा॒त्र मास॑ते॒ श्रत् । \newline
34. च॒तु॒र्विꣳ॒॒श॒ति॒रा॒त्रमिति॑ चतुर्विꣳशति - रा॒त्रम् । \newline
35. आस॑ते॒ श्रच् छ्रदास॑त॒ आस॑ते॒ श्रदे᳚भ्य एभ्यः॒ श्रदास॑त॒ आस॑ते॒ श्रदे᳚भ्यः । \newline
36. श्रदे᳚भ्य एभ्यः॒ श्रच् छ्रदे᳚भ्यो मनु॒ष्या॑ मनु॒ष्या॑ एभ्यः॒ श्रच् छ्रदे᳚भ्यो मनु॒ष्याः᳚ । \newline
37. ए॒भ्यो॒ म॒नु॒ष्या॑ मनु॒ष्या॑ एभ्य एभ्यो मनु॒ष्या॑ दधते दधते मनु॒ष्या॑ एभ्य एभ्यो मनु॒ष्या॑ दधते । \newline
38. म॒नु॒ष्या॑ दधते दधते मनु॒ष्या॑ मनु॒ष्या॑ दधते॒ गच्छ॑न्ति॒ गच्छ॑न्ति दधते मनु॒ष्या॑ मनु॒ष्या॑ दधते॒ गच्छ॑न्ति । \newline
39. द॒ध॒ते॒ गच्छ॑न्ति॒ गच्छ॑न्ति दधते दधते॒ गच्छ॑न्ति पुरो॒धाम् पु॑रो॒धाम् गच्छ॑न्ति दधते दधते॒ गच्छ॑न्ति पुरो॒धाम् । \newline
40. गच्छ॑न्ति पुरो॒धाम् पु॑रो॒धाम् गच्छ॑न्ति॒ गच्छ॑न्ति पुरो॒धाम् ज्योति॒र् ज्योतिः॑ पुरो॒धाम् गच्छ॑न्ति॒ गच्छ॑न्ति पुरो॒धाम् ज्योतिः॑ । \newline
41. पु॒रो॒धाम् ज्योति॒र् ज्योतिः॑ पुरो॒धाम् पु॑रो॒धाम् ज्योति॒र् गौर् गौर् ज्योतिः॑ पुरो॒धाम् पु॑रो॒धाम् ज्योति॒र् गौः । \newline
42. पु॒रो॒धामिति॑ पुरः - धाम् । \newline
43. ज्योति॒र् गौर् गौर् ज्योति॒र् ज्योति॒र् गौ रायु॒ रायु॒र् गौर् ज्योति॒र् ज्योति॒र् गौ रायुः॑ । \newline
44. गौ रायु॒ रायु॒र् गौर् गौ रायु॒ रिती त्यायु॒र् गौर् गौ रायु॒ रिति॑ । \newline
45. आयु॒रिती त्यायु॒ रायु॒ रिति॑ त्र्य॒हा स्त्र्य॒हा इत्यायु॒ रायु॒ रिति॑ त्र्य॒हाः । \newline
46. इति॑ त्र्य॒हा स्त्र्य॒हा इतीति॑ त्र्य॒हा भ॑वन्ति भवन्ति त्र्य॒हा इतीति॑ त्र्य॒हा भ॑वन्ति । \newline
47. त्र्य॒हा भ॑वन्ति भवन्ति त्र्य॒हा स्त्र्य॒हा भ॑वन्ती॒य मि॒यम् भ॑वन्ति त्र्य॒हा स्त्र्य॒हा भ॑वन्ती॒यम् । \newline
48. त्र्य॒हा इति॑ त्रि - अ॒हाः । \newline
49. भ॒व॒न्ती॒य मि॒यम् भ॑वन्ति भवन्ती॒यं ॅवाव वावेयम् भ॑वन्ति भवन्ती॒यं ॅवाव । \newline
50. इ॒यं ॅवाव वावेय मि॒यं ॅवाव ज्योति॒र् ज्योति॒र् वावेय मि॒यं ॅवाव ज्योतिः॑ । \newline
51. वाव ज्योति॒र् ज्योति॒र् वाव वाव ज्योति॑ र॒न्तरि॑क्ष म॒न्तरि॑क्ष॒म् ज्योति॒र् वाव वाव ज्योति॑ र॒न्तरि॑क्षम् । \newline
52. ज्योति॑ र॒न्तरि॑क्ष म॒न्तरि॑क्ष॒म् ज्योति॒र् ज्योति॑ र॒न्तरि॑क्ष॒म् गौर् गौ र॒न्तरि॑क्ष॒म् ज्योति॒र् ज्योति॑ र॒न्तरि॑क्ष॒म् गौः । \newline
53. अ॒न्तरि॑क्ष॒म् गौर् गौ र॒न्तरि॑क्ष म॒न्तरि॑क्ष॒म् गौ र॒सा व॒सौ गौ र॒न्तरि॑क्ष म॒न्तरि॑क्ष॒म् गौ र॒सौ । \newline
54. गौ र॒सा व॒सौ गौर् गौ र॒सा वायु॒ रायु॑ र॒सौ गौर् गौ र॒सा वायुः॑ । \newline
55. अ॒सा वायु॒ रायु॑ र॒सा व॒सा वायु॑ रि॒मा नि॒मा नायु॑ र॒सा व॒सा वायु॑ रि॒मान् । \newline
56. आयु॑ रि॒मा नि॒मा नायु॒ रायु॑ रि॒मा ने॒वैवेमा नायु॒ रायु॑ रि॒मा ने॒व । \newline
\pagebreak
\markright{ TS 7.4.1.2  \hfill https://www.vedavms.in \hfill}

\section{ TS 7.4.1.2 }

\textbf{TS 7.4.1.2 } \newline
\textbf{Samhita Paata} \newline

-रि॒माने॒व लो॒कान॒भ्यारो॑हन्त्यभि पू॒र्वं त्र्य॒हा भ॑वन्त्यभिपू॒र्वमे॒व सु॑व॒र्गं ॅलो॒कम॒भ्यारो॑ह॒न्त्यस॑त्रं॒ ॅवा ए॒तद्-यद॑छन्दो॒मं ॅयच्छ॑न्दो॒मा भव॑न्ति॒ तेन॑ स॒त्रं दे॒वता॑ ए॒व पृ॒ष्ठैरव॑ रुन्धते प॒शूञ्छ॑न्दो॒मैरोजो॒ वै वी॒र्यं॑ पृ॒ष्ठानि॑ प॒शवः॑ छन्दो॒मा ओज॑स्ये॒व वी॒र्ये॑ प॒शुषु॒ प्रति॑ तिष्ठन्ति बृहद्-रथन्त॒राभ्यां᳚ ॅयन्ती॒यं ॅवाव र॑थन्त॒रम॒सौ बृ॒हदा॒भ्यामे॒व - [  ] \newline

\textbf{Pada Paata} \newline

इ॒मान् । ए॒व । लो॒कान् । अ॒भ्यारो॑ह॒न्तीत्य॑भि - आरो॑हन्ति । अ॒भि॒पू॒र्वमित्य॑भि - पू॒र्वम् । त्र्य॒हा इति॑ त्रि - अ॒हाः । भ॒व॒न्ति॒ । अ॒भि॒पू॒र्वमित्य॑भि - पू॒र्वम् । ए॒व । सु॒व॒र्गमिति॑ सुवः - गम् । लो॒कम् । अ॒भ्यारो॑ह॒न्तीत्य॑भि - आरो॑हन्ति । अस॑त्रम् । वै । ए॒तत् । यत् । अ॒छ॒न्दो॒ममित्य॑छन्दः-मम् । यत् । छ॒न्दो॒मा इति॑ छन्दः - माः । भव॑न्ति । तेन॑ । स॒त्रम् । दे॒वताः᳚ । ए॒व । पृ॒ष्ठैः । अवेति॑ । रु॒न्ध॒ते॒ । प॒शून् । छ॒न्दो॒मैरिति॑ छन्दः-मैः । ओजः॑ । वै । वी॒र्य᳚म् । पृ॒ष्ठानि॑ । प॒शवः॑ । छ॒न्दो॒मा इति॑ छन्दः - माः । ओज॑सि । ए॒व । वी॒र्ये᳚ । प॒शुषु॑ । प्रतीति॑ । ति॒ष्ठ॒न्ति॒ । बृ॒ह॒द्र॒थ॒न्त॒राभ्या॒मिति॑ बृहत् - र॒थ॒न्त॒राभ्या᳚म् । य॒न्ति॒ । इ॒यम् । वाव । र॒थ॒न्त॒रमिति॑ रथं - त॒रम् । अ॒सौ । बृ॒हत् । आ॒भ्याम् । ए॒व ।  \newline


\textbf{Krama Paata} \newline

इ॒माने॒व । ए॒व लो॒कान् । लो॒कान॒भ्यारो॑हन्ति । अ॒भ्यारो॑हन्त्यभिपू॒र्वम् । अ॒भ्यारो॑ह॒न्तीत्य॑भि - आरो॑हन्ति । अ॒भि॒पू॒र्वम् त्र्य॒हाः । अ॒भि॒पू॒र्वमित्य॑भि - पू॒र्वम् । त्र्य॒हा भ॑वन्ति । त्र्य॒हा इति॑ त्रि - अ॒हाः । भ॒व॒न्त्य॒भि॒पू॒र्वम् । अ॒भि॒पू॒र्वमे॒व । अ॒भि॒पू॒र्वमित्य॑भि - पू॒र्वम् । ए॒व सु॑व॒र्गम् । सु॒व॒र्गम् ॅलो॒कम् । सु॒व॒र्गमिति॑ सुवः - गम् । लो॒कम॒भ्यारो॑हन्ति । अ॒भ्यारो॑ह॒न्त्यस॑त्रम् । अ॒भ्यारो॑ह॒न्तीत्य॑भि - आरो॑हन्ति । अस॑त्र॒म् ॅवै । वा ए॒तत् । ए॒तद् यत् । यद॑छन्दो॒मम् । अ॒छ॒न्दो॒मम् ॅयत् । अ॒छ॒न्दो॒ममित्य॑छन्दः - मम् । यच् छ॑न्दो॒माः । छ॒न्दो॒मा भव॑न्ति । छ॒न्दो॒मा इति॑ छन्दः - माः । भव॑न्ति॒ तेन॑ । तेन॑ स॒त्रम् । स॒त्रम् दे॒वताः᳚ । दे॒वता॑ ए॒व । ए॒व पृ॒ष्ठैः । पृ॒ष्ठैरव॑ । अव॑ रुन्धते । रु॒न्ध॒ते॒ प॒शून् । प॒शून् छ॑न्दो॒मैः । छ॒न्दो॒मैरोजः॑ । छ॒न्दो॒मैरिति॑ छन्दः - मैः । ओजो॒ वै । वै वी॒र्य᳚म् । वी॒र्य॑म् पृ॒ष्ठानि॑ । पृ॒ष्ठानि॑ प॒शवः॑ । प॒शव॑श्छन्दो॒माः । छ॒न्दो॒मा ओज॑सि । छ॒न्दो॒मा इति॑ छन्दः - माः । ओज॑स्ये॒व । ए॒व वी॒र्ये᳚ । वी॒र्ये॑ प॒शुषु॑ । प॒शुषु॒ प्रति॑ । प्रति॑ तिष्ठन्ति । ति॒ष्ठ॒न्ति॒ बृ॒ह॒द्‍र॒थ॒न्त॒राभ्या᳚म् । बृ॒ह॒द्‍र॒थ॒न्त॒राभ्या᳚म् ॅयन्ति । बृ॒ह॒द्‍र॒थ॒न्त॒राभ्या॒मिति॑ बृहत् - र॒थ॒न्त॒राभ्या᳚म् । य॒न्ती॒यम् । इ॒यम् ॅवाव । वाव र॑थन्त॒रम् । र॒थ॒न्त॒रम॒सौ । र॒थ॒न्त॒रमिति॑ रथम् - त॒रम् । अ॒सौ बृ॒हत् । बृ॒हदा॒भ्याम् । आ॒भ्यामे॒व ( ) । ए॒व य॑न्ति \newline

\textbf{Jatai Paata} \newline

1. इ॒माने॒वैवे मानि॒मा ने॒व । \newline
2. ए॒व लो॒कान् ॅलो॒का ने॒वैव लो॒कान् । \newline
3. लो॒का न॒भ्यारो॑ह न्त्य॒भ्यारो॑हन्ति लो॒कान् ॅलो॒का न॒भ्यारो॑हन्ति । \newline
4. अ॒भ्यारो॑ह न्त्यभिपू॒र्व म॑भिपू॒र्व म॒भ्यारो॑ह न्त्य॒भ्यारो॑हन् त्यभिपू॒र्वम् । \newline
5. अ॒भ्यारो॑ह॒न्तीत्य॑भि - आरो॑हन्ति । \newline
6. अ॒भि॒पू॒र्वम् त्र्य॒हा स्त्र्य॒हा अ॑भिपू॒र्व म॑भिपू॒र्वम् त्र्य॒हाः । \newline
7. अ॒भि॒पू॒र्वमित्य॑भि - पू॒र्वम् । \newline
8. त्र्य॒हा भ॑वन्ति भवन्ति त्र्य॒हा स्त्र्य॒हा भ॑वन्ति । \newline
9. त्र्य॒हा इति॑ त्रि - अ॒हाः । \newline
10. भ॒व॒ न्त्य॒भि॒पू॒र्व म॑भिपू॒र्वम् भ॑वन्ति भव न्त्यभिपू॒र्वम् । \newline
11. अ॒भि॒पू॒र्व मे॒वै वाभि॑पू॒र्व म॑भिपू॒र्व मे॒व । \newline
12. अ॒भि॒पू॒र्वमित्य॑भि - पू॒र्वम् । \newline
13. ए॒व सु॑व॒र्गꣳ सु॑व॒र्ग मे॒वैव सु॑व॒र्गम् । \newline
14. सु॒व॒र्गम् ॅलो॒कम् ॅलो॒कꣳ सु॑व॒र्गꣳ सु॑व॒र्गम् ॅलो॒कम् । \newline
15. सु॒व॒र्गमिति॑ सुवः - गम् । \newline
16. लो॒क म॒भ्यारो॑ह न्त्य॒भ्यारो॑हन्ति लो॒कम् ॅलो॒क म॒भ्यारो॑हन्ति । \newline
17. अ॒भ्यारो॑ह॒ न्त्यस॑त्र॒ मस॑त्र म॒भ्यारो॑हन् त्य॒भ्यारो॑ह॒ न्त्यस॑त्रम् । \newline
18. अ॒भ्यारो॑ह॒न्तीत्य॑भि - आरो॑हन्ति । \newline
19. अस॑त्रं॒ ॅवै वा अस॑त्र॒ मस॑त्रं॒ ॅवै । \newline
20. वा ए॒त दे॒तद् वै वा ए॒तत् । \newline
21. ए॒तद् यद् यदे॒त दे॒तद् यत् । \newline
22. यद॑छन्दो॒म म॑छन्दो॒मं ॅयद् यद॑छन्दो॒मम् । \newline
23. अ॒छ॒न्दो॒मं ॅयद् यद॑छन्दो॒म म॑छन्दो॒मं ॅयत् । \newline
24. अ॒छ॒न्दो॒ममित्य॑छन्दः - मम् । \newline
25. यच् छ॑न्दो॒मा श्छ॑न्दो॒मा यद् यच् छ॑न्दो॒माः । \newline
26. छ॒न्दो॒मा भव॑न्ति॒ भव॑न्ति छन्दो॒मा श्छ॑न्दो॒मा भव॑न्ति । \newline
27. छ॒न्दो॒मा इति॑ छन्दः - माः । \newline
28. भव॑न्ति॒ तेन॒ तेन॒ भव॑न्ति॒ भव॑न्ति॒ तेन॑ । \newline
29. तेन॑ स॒त्रꣳ स॒त्रम् तेन॒ तेन॑ स॒त्रम् । \newline
30. स॒त्रम् दे॒वता॑ दे॒वताः᳚ स॒त्रꣳ स॒त्रम् दे॒वताः᳚ । \newline
31. दे॒वता॑ ए॒वैव दे॒वता॑ दे॒वता॑ ए॒व । \newline
32. ए॒व पृ॒ष्ठैः पृ॒ष्ठै रे॒वैव पृ॒ष्ठैः । \newline
33. पृ॒ष्ठै रवाव॑ पृ॒ष्ठैः पृ॒ष्ठै रव॑ । \newline
34. अव॑ रुन्धते रुन्ध॒ते ऽवाव॑ रुन्धते । \newline
35. रु॒न्ध॒ते॒ प॒शून् प॒शून् रु॑न्धते रुन्धते प॒शून् । \newline
36. प॒शून् छ॑न्दो॒मै श्छ॑न्दो॒मैः प॒शून् प॒शून् छ॑न्दो॒मैः । \newline
37. छ॒न्दो॒मै रोज॒ ओज॑ श्छन्दो॒मै श्छ॑न्दो॒मै रोजः॑ । \newline
38. छ॒न्दो॒मैरिति॑ छन्दः - मैः । \newline
39. ओजो॒ वै वा ओज॒ ओजो॒ वै । \newline
40. वै वी॒र्यं॑ ॅवी॒र्यं॑ ॅवै वै वी॒र्य᳚म् । \newline
41. वी॒र्य॑म् पृ॒ष्ठानि॑ पृ॒ष्ठानि॑ वी॒र्यं॑ ॅवी॒र्य॑म् पृ॒ष्ठानि॑ । \newline
42. पृ॒ष्ठानि॑ प॒शवः॑ प॒शवः॑ पृ॒ष्ठानि॑ पृ॒ष्ठानि॑ प॒शवः॑ । \newline
43. प॒शव॑ श्छन्दो॒मा श्छ॑न्दो॒माः प॒शवः॑ प॒शव॑ श्छन्दो॒माः । \newline
44. छ॒न्दो॒मा ओज॒ स्योज॑सि छन्दो॒मा श्छ॑न्दो॒मा ओज॑सि । \newline
45. छ॒न्दो॒मा इति॑ छन्दः - माः । \newline
46. ओज॑ स्ये॒वैवौज॒ स्योज॑ स्ये॒व । \newline
47. ए॒व वी॒र्ये॑ वी॒र्य॑ ए॒वैव वी॒र्ये᳚ । \newline
48. वी॒र्ये॑ प॒शुषु॑ प॒शुषु॑ वी॒र्ये॑ वी॒र्ये॑ प॒शुषु॑ । \newline
49. प॒शुषु॒ प्रति॒ प्रति॑ प॒शुषु॑ प॒शुषु॒ प्रति॑ । \newline
50. प्रति॑ तिष्ठन्ति तिष्ठन्ति॒ प्रति॒ प्रति॑ तिष्ठन्ति । \newline
51. ति॒ष्ठ॒न्ति॒ बृ॒ह॒द्र॒थ॒न्त॒राभ्या᳚म् बृहद्रथन्त॒राभ्या᳚म् तिष्ठन्ति तिष्ठन्ति बृहद्रथन्त॒राभ्या᳚म् । \newline
52. बृ॒ह॒द्र॒थ॒न्त॒राभ्यां᳚ ॅयन्ति यन्ति बृहद्रथन्त॒राभ्या᳚म् बृहद्रथन्त॒राभ्यां᳚ ॅयन्ति । \newline
53. बृ॒ह॒द्र॒थ॒न्त॒राभ्या॒मिति॑ बृहत् - र॒थ॒न्त॒राभ्या᳚म् । \newline
54. य॒न्ती॒य मि॒यं ॅय॑न्ति यन्ती॒यम् । \newline
55. इ॒यं ॅवाव वावेय मि॒यं ॅवाव । \newline
56. वाव र॑थन्त॒रꣳ र॑थन्त॒रं ॅवाव वाव र॑थन्त॒रम् । \newline
57. र॒थ॒न्त॒र म॒सा व॒सौ र॑थन्त॒रꣳ र॑थन्त॒र म॒सौ । \newline
58. र॒थ॒न्त॒रमिति॑ रथं - त॒रम् । \newline
59. अ॒सौ बृ॒हद् बृ॒ह द॒सा व॒सौ बृ॒हत् । \newline
60. बृ॒ह दा॒भ्या मा॒भ्याम् बृ॒हद् बृ॒ह दा॒भ्याम् । \newline
61. आ॒भ्या मे॒वै वाभ्या मा॒भ्या मे॒व । \newline
62. ए॒व य॑न्ति यन्त्ये॒वैव य॑न्ति । \newline

\textbf{Ghana Paata } \newline

1. इ॒मा ने॒वैवेमा नि॒मा ने॒व लो॒कान् ॅलो॒का ने॒वेमा नि॒मा ने॒व लो॒कान् । \newline
2. ए॒व लो॒कान् ॅलो॒का ने॒वैव लो॒का न॒भ्यारो॑ह न्त्य॒भ्यारो॑हन्ति लो॒का ने॒वैव लो॒का न॒भ्यारो॑हन्ति । \newline
3. लो॒का न॒भ्यारो॑ह न्त्य॒भ्यारो॑हन्ति लो॒कान् ॅलो॒का न॒भ्यारो॑ह न्त्यभिपू॒र्व म॑भिपू॒र्व म॒भ्यारो॑हन्ति लो॒कान् ॅलो॒का न॒भ्यारो॑ह न्त्यभिपू॒र्वम् । \newline
4. अ॒भ्यारो॑ह न्त्यभिपू॒र्व म॑भिपू॒र्व म॒भ्यारो॑ह न्त्य॒भ्यारो॑ह न्त्यभिपू॒र्वम् त्र्य॒हा स्त्र्य॒हा अ॑भिपू॒र्व म॒भ्यारो॑ह न्त्य॒भ्यारो॑ह न्त्यभिपू॒र्वम् त्र्य॒हाः । \newline
5. अ॒भ्यारो॑ह॒न्तीत्य॑भि - आरो॑हन्ति । \newline
6. अ॒भि॒पू॒र्वम् त्र्य॒हा स्त्र्य॒हा अ॑भिपू॒र्व म॑भिपू॒र्वम् त्र्य॒हा भ॑वन्ति भवन्ति त्र्य॒हा अ॑भिपू॒र्व म॑भिपू॒र्वम् त्र्य॒हा भ॑वन्ति । \newline
7. अ॒भि॒पू॒र्वमित्य॑भि - पू॒र्वम् । \newline
8. त्र्य॒हा भ॑वन्ति भवन्ति त्र्य॒हा स्त्र्य॒हा भ॑व न्त्यभिपू॒र्व म॑भिपू॒र्वम् भ॑वन्ति त्र्य॒हा स्त्र्य॒हा भ॑व
न्त्यभिपू॒र्वम् । \newline
9. त्र्य॒हा इति॑ त्रि - अ॒हाः । \newline
10. भ॒व॒ न्त्य॒भि॒पू॒र्व म॑भिपू॒र्वम् भ॑वन्ति भव न्त्यभिपू॒र्व मे॒वै वाभि॑पू॒र्वम् भ॑वन्ति भव न्त्यभिपू॒र्व मे॒व । \newline
11. अ॒भि॒पू॒र्व मे॒वै वाभि॑पू॒र्व म॑भिपू॒र्व मे॒व सु॑व॒र्गꣳ सु॑व॒र्ग मे॒वा भि॑पू॒र्व म॑भिपू॒र्व मे॒व सु॑व॒र्गम् । \newline
12. अ॒भि॒पू॒र्वमित्य॑भि - पू॒र्वम् । \newline
13. ए॒व सु॑व॒र्गꣳ सु॑व॒र्ग मे॒वैव सु॑व॒र्गम् ॅलो॒कम् ॅलो॒कꣳ सु॑व॒र्ग मे॒वैव सु॑व॒र्गम् ॅलो॒कम् । \newline
14. सु॒व॒र्गम् ॅलो॒कम् ॅलो॒कꣳ सु॑व॒र्गꣳ सु॑व॒र्गम् ॅलो॒क म॒भ्यारो॑ह न्त्य॒भ्यारो॑हन्ति लो॒कꣳ सु॑व॒र्गꣳ सु॑व॒र्गम् ॅलो॒क म॒भ्यारो॑हन्ति । \newline
15. सु॒व॒र्गमिति॑ सुवः - गम् । \newline
16. लो॒क म॒भ्यारो॑ह न्त्य॒भ्यारो॑हन्ति लो॒कम् ॅलो॒क म॒भ्यारो॑ह॒ न्त्यस॑त्र॒ मस॑त्र म॒भ्यारो॑हन्ति लो॒कम् ॅलो॒क म॒भ्यारो॑ह॒ न्त्यस॑त्रम् । \newline
17. अ॒भ्यारो॑ह॒ न्त्यस॑त्र॒ मस॑त्र म॒भ्यारो॑ह न्त्य॒भ्यारो॑ह॒ न्त्यस॑त्रं॒ ॅवै वा अस॑त्र म॒भ्यारो॑ह न्त्य॒भ्यारो॑ह॒ न्त्यस॑त्रं॒ ॅवै । \newline
18. अ॒भ्यारो॑ह॒न्तीत्य॑भि - आरो॑हन्ति । \newline
19. अस॑त्रं॒ ॅवै वा अस॑त्र॒ मस॑त्रं॒ ॅवा ए॒त दे॒तद् वा अस॑त्र॒ मस॑त्रं॒ ॅवा ए॒तत् । \newline
20. वा ए॒त दे॒तद् वै वा ए॒तद् यद् यदे॒तद् वै वा ए॒तद् यत् । \newline
21. ए॒तद् यद् यदे॒त दे॒तद् यद॑छन्दो॒म म॑छन्दो॒मं ॅयदे॒त दे॒तद् यद॑छन्दो॒मम् । \newline
22. यद॑छन्दो॒म म॑छन्दो॒मं ॅयद् यद॑छन्दो॒मं ॅयद् यद॑छन्दो॒मं ॅयद् यद॑छन्दो॒मं ॅयत् । \newline
23. अ॒छ॒न्दो॒मं ॅयद् यद॑छन्दो॒म म॑छन्दो॒मं ॅयच् छ॑न्दो॒मा श्छ॑न्दो॒मा यद॑छन्दो॒म म॑छन्दो॒मं ॅयच् छ॑न्दो॒माः । \newline
24. अ॒छ॒न्दो॒ममित्य॑छन्दः - मम् । \newline
25. यच् छ॑न्दो॒मा श्छ॑न्दो॒मा यद् यच् छ॑न्दो॒मा भव॑न्ति॒ भव॑न्ति छन्दो॒मा यद् यच् छ॑न्दो॒मा भव॑न्ति । \newline
26. छ॒न्दो॒मा भव॑न्ति॒ भव॑न्ति छन्दो॒मा श्छ॑न्दो॒मा भव॑न्ति॒ तेन॒ तेन॒ भव॑न्ति छन्दो॒मा श्छ॑न्दो॒मा भव॑न्ति॒ तेन॑ । \newline
27. छ॒न्दो॒मा इति॑ छन्दः - माः । \newline
28. भव॑न्ति॒ तेन॒ तेन॒ भव॑न्ति॒ भव॑न्ति॒ तेन॑ स॒त्रꣳ स॒त्रम् तेन॒ भव॑न्ति॒ भव॑न्ति॒ तेन॑ स॒त्रम् । \newline
29. तेन॑ स॒त्रꣳ स॒त्रम् तेन॒ तेन॑ स॒त्रम् दे॒वता॑ दे॒वताः᳚ स॒त्रम् तेन॒ तेन॑ स॒त्रम् दे॒वताः᳚ । \newline
30. स॒त्रम् दे॒वता॑ दे॒वताः᳚ स॒त्रꣳ स॒त्रम् दे॒वता॑ ए॒वैव दे॒वताः᳚ स॒त्रꣳ स॒त्रम् दे॒वता॑ ए॒व । \newline
31. दे॒वता॑ ए॒वैव दे॒वता॑ दे॒वता॑ ए॒व पृ॒ष्ठैः पृ॒ष्ठै रे॒व दे॒वता॑ दे॒वता॑ ए॒व पृ॒ष्ठैः । \newline
32. ए॒व पृ॒ष्ठैः पृ॒ष्ठै रे॒वैव पृ॒ष्ठै रवाव॑ पृ॒ष्ठै रे॒वैव पृ॒ष्ठै रव॑ । \newline
33. पृ॒ष्ठै रवाव॑ पृ॒ष्ठैः पृ॒ष्ठै रव॑ रुन्धते रुन्ध॒ते ऽव॑ पृ॒ष्ठैः पृ॒ष्ठै रव॑ रुन्धते । \newline
34. अव॑ रुन्धते रुन्ध॒ते ऽवाव॑ रुन्धते प॒शून् प॒शून् रु॑न्ध॒ते ऽवाव॑ रुन्धते प॒शून् । \newline
35. रु॒न्ध॒ते॒ प॒शून् प॒शून् रु॑न्धते रुन्धते प॒शून् छ॑न्दो॒मै श्छ॑न्दो॒मैः प॒शून् रु॑न्धते रुन्धते प॒शून् छ॑न्दो॒मैः । \newline
36. प॒शून् छ॑न्दो॒मै श्छ॑न्दो॒मैः प॒शून् प॒शून् छ॑न्दो॒मै रोज॒ ओज॑ श्छन्दो॒मैः प॒शून् प॒शून् छ॑न्दो॒मै रोजः॑ । \newline
37. छ॒न्दो॒मै रोज॒ ओज॑ श्छन्दो॒मै श्छ॑न्दो॒मै रोजो॒ वै वा ओज॑ श्छन्दो॒मै श्छ॑न्दो॒मै रोजो॒ वै । \newline
38. छ॒न्दो॒मैरिति॑ छन्दः - मैः । \newline
39. ओजो॒ वै वा ओज॒ ओजो॒ वै वी॒र्यं॑ ॅवी॒र्यं॑ ॅवा ओज॒ ओजो॒ वै वी॒र्य᳚म् । \newline
40. वै वी॒र्यं॑ ॅवी॒र्यं॑ ॅवै वै वी॒र्य॑म् पृ॒ष्ठानि॑ पृ॒ष्ठानि॑ वी॒र्यं॑ ॅवै वै वी॒र्य॑म् पृ॒ष्ठानि॑ । \newline
41. वी॒र्य॑म् पृ॒ष्ठानि॑ पृ॒ष्ठानि॑ वी॒र्यं॑ ॅवी॒र्य॑म् पृ॒ष्ठानि॑ प॒शवः॑ प॒शवः॑ पृ॒ष्ठानि॑ वी॒र्यं॑ ॅवी॒र्य॑म् पृ॒ष्ठानि॑ प॒शवः॑ । \newline
42. पृ॒ष्ठानि॑ प॒शवः॑ प॒शवः॑ पृ॒ष्ठानि॑ पृ॒ष्ठानि॑ प॒शव॑ श्छन्दो॒मा श्छ॑न्दो॒माः प॒शवः॑ पृ॒ष्ठानि॑ पृ॒ष्ठानि॑ प॒शव॑ श्छन्दो॒माः । \newline
43. प॒शव॑ श्छन्दो॒मा श्छ॑न्दो॒माः प॒शवः॑ प॒शव॑ श्छन्दो॒मा ओज॒ स्योज॑सि छन्दो॒माः प॒शवः॑ प॒शव॑ श्छन्दो॒मा ओज॑सि । \newline
44. छ॒न्दो॒मा ओज॒ स्योज॑सि छन्दो॒मा श्छ॑न्दो॒मा ओज॑ स्ये॒वैवौज॑सि छन्दो॒मा श्छ॑न्दो॒मा ओज॑ स्ये॒व । \newline
45. छ॒न्दो॒मा इति॑ छन्दः - माः । \newline
46. ओज॑ स्ये॒वैवौज॒ स्योज॑ स्ये॒व वी॒र्ये॑ वी॒र्य॑ ए॒वौज॒ स्योज॑ स्ये॒व वी॒र्ये᳚ । \newline
47. ए॒व वी॒र्ये॑ वी॒र्य॑ ए॒वैव वी॒र्ये॑ प॒शुषु॑ प॒शुषु॑ वी॒र्य॑ ए॒वैव वी॒र्ये॑ प॒शुषु॑ । \newline
48. वी॒र्ये॑ प॒शुषु॑ प॒शुषु॑ वी॒र्ये॑ वी॒र्ये॑ प॒शुषु॒ प्रति॒ प्रति॑ प॒शुषु॑ वी॒र्ये॑ वी॒र्ये॑ प॒शुषु॒ प्रति॑ । \newline
49. प॒शुषु॒ प्रति॒ प्रति॑ प॒शुषु॑ प॒शुषु॒ प्रति॑ तिष्ठन्ति तिष्ठन्ति॒ प्रति॑ प॒शुषु॑ प॒शुषु॒ प्रति॑ तिष्ठन्ति । \newline
50. प्रति॑ तिष्ठन्ति तिष्ठन्ति॒ प्रति॒ प्रति॑ तिष्ठन्ति बृहद्रथन्त॒राभ्या᳚म् बृहद्रथन्त॒राभ्या᳚म् तिष्ठन्ति॒ प्रति॒ प्रति॑ तिष्ठन्ति बृहद्रथन्त॒राभ्या᳚म् । \newline
51. ति॒ष्ठ॒न्ति॒ बृ॒ह॒द्र॒थ॒न्त॒राभ्या᳚म् बृहद्रथन्त॒राभ्या᳚म् तिष्ठन्ति तिष्ठन्ति बृहद्रथन्त॒राभ्यां᳚ ॅयन्ति यन्ति बृहद्रथन्त॒राभ्या᳚म् तिष्ठन्ति तिष्ठन्ति बृहद्रथन्त॒राभ्यां᳚ ॅयन्ति । \newline
52. बृ॒ह॒द्र॒थ॒न्त॒राभ्यां᳚ ॅयन्ति यन्ति बृहद्रथन्त॒राभ्या᳚म् बृहद्रथन्त॒राभ्यां᳚ ॅयन्ती॒य मि॒यं ॅय॑न्ति बृहद्रथन्त॒राभ्या᳚म् बृहद्रथन्त॒राभ्यां᳚ ॅयन्ती॒यम् । \newline
53. बृ॒ह॒द्र॒थ॒न्त॒राभ्या॒मिति॑ बृहत् - र॒थ॒न्त॒राभ्या᳚म् । \newline
54. य॒न्ती॒य मि॒यं ॅय॑न्ति यन्ती॒यं ॅवाव वावेयं ॅय॑न्ति यन्ती॒यं ॅवाव । \newline
55. इ॒यं ॅवाव वावेय मि॒यं ॅवाव र॑थन्त॒रꣳ र॑थन्त॒रं ॅवावेय मि॒यं ॅवाव र॑थन्त॒रम् । \newline
56. वाव र॑थन्त॒रꣳ र॑थन्त॒रं ॅवाव वाव र॑थन्त॒र म॒सा व॒सौ र॑थन्त॒रं ॅवाव वाव र॑थन्त॒र म॒सौ । \newline
57. र॒थ॒न्त॒र म॒सा व॒सौ र॑थन्त॒रꣳ र॑थन्त॒र म॒सौ बृ॒हद् बृ॒ह द॒सौ र॑थन्त॒रꣳ र॑थन्त॒र म॒सौ बृ॒हत् । \newline
58. र॒थ॒न्त॒रमिति॑ रथं - त॒रम् । \newline
59. अ॒सौ बृ॒हद् बृ॒ह द॒सा व॒सौ बृ॒ह दा॒भ्या मा॒भ्याम् बृ॒ह द॒सा व॒सौ बृ॒ह दा॒भ्याम् । \newline
60. बृ॒ह दा॒भ्या मा॒भ्याम् बृ॒हद् बृ॒ह दा॒भ्या मे॒वै वाभ्याम् बृ॒हद् बृ॒ह दा॒भ्या मे॒व । \newline
61. आ॒भ्या मे॒वै वाभ्या मा॒भ्या मे॒व य॑न्ति यन्त्ये॒ वाभ्या मा॒भ्या मे॒व य॑न्ति । \newline
62. ए॒व य॑न्ति यन्त्ये॒वैव य॒न्त्यथो॒ अथो॑ यन्त्ये॒ वैव य॒न्त्यथो᳚ । \newline
\pagebreak
\markright{ TS 7.4.1.3  \hfill https://www.vedavms.in \hfill}

\section{ TS 7.4.1.3 }

\textbf{TS 7.4.1.3 } \newline
\textbf{Samhita Paata} \newline

य॒न्त्यथो॑ अ॒नयो॑रे॒व प्रति॑ तिष्ठन्त्ये॒ते वै य॒ज्ञ्स्या᳚ञ्ज॒साय॑नी स्रु॒ती ताभ्या॑मे॒व सु॑व॒र्गं ॅलो॒कं ॅय॑न्ति चतुर्विꣳशतिरा॒त्रो भ॑वति॒ चतु॑र्विꣳशतिरर्द्धमा॒साः सं॑ॅवथ्स॒रः सं॑ॅवथ्स॒रः सु॑व॒र्गो लो॒कः सं॑ॅवथ्स॒र ए॒व सु॑व॒र्गे लो॒के प्रति॑ तिष्ठ॒न्त्यथो॒ चतु॑र्विꣳशत्यक्षरा गाय॒त्री गा॑य॒त्री ब्र॑ह्मवर्च॒सं गा॑यत्रि॒यैव ब्र॑ह्मवर्च॒समव॑ रुन्धते ऽतिरा॒त्राव॒भितो॑ भवतो ब्रह्मवर्च॒सस्य॒ परि॑गृहीत्यै ॥ \newline

\textbf{Pada Paata} \newline

य॒न्ति॒ । अथो॒ इति॑ । अ॒नयोः᳚ । ए॒व । प्रतीति॑ । ति॒ष्ठ॒न्ति॒ । ए॒ते इति॑ । वै । य॒ज्ञ्स्य॑ । अ॒ञ्ज॒साय॑नी॒ इत्य॑ञ्जसा - अय॑नी । स्रु॒ती इति॑ । ताभ्या᳚म् । ए॒व । सु॒व॒र्गमिति॑ सुवः - गम् । लो॒कम् । य॒न्ति॒ । च॒तु॒र्विꣳ॒॒श॒ति॒रा॒त्र इति॑ चतुर्विꣳशति - रा॒त्रः । भ॒व॒ति॒ । चतु॑र्विꣳशति॒रिति॒ चतुः॑ - विꣳ॒॒श॒तिः॒ । अ॒द्‌र्ध॒मा॒सा इत्य॑द्‌र्ध - मा॒साः । सं॒ॅव॒थ्स॒र इति॑ सं - व॒थ्स॒रः । सं॒ॅव॒थ्स॒र इति॑ सं - व॒थ्स॒रः । सु॒व॒र्ग इति॑ सुवः - गः । लो॒कः । सं॒ॅव॒थ्स॒र इति॑ सं - व॒थ्स॒रे । ए॒व । सु॒व॒र्ग इति॑ सुवः - गे । लो॒के । प्रतीति॑ । ति॒ष्ठ॒न्ति॒ । अथो॒ इति॑ । चतु॑र्विꣳशत्यक्ष॒रेति॒ चतु॑र्विꣳशति - अ॒क्ष॒रा॒ । गा॒य॒त्री । गा॒य॒त्री । ब्र॒ह्म॒व॒र्च॒समिति॑ ब्रह्म - व॒र्च॒सम् । गा॒य॒त्रि॒या । ए॒व । ब्र॒ह्म॒व॒र्च॒समिति॑ ब्रह्म - व॒र्च॒सम् । अवेति॑ । रु॒न्ध॒ते॒ । अ॒ति॒रा॒त्रावित्य॑ति - रा॒त्रौ । अ॒भितः॑ । भ॒व॒तः॒ । ब्र॒ह्म॒व॒र्च॒सस्येति॑ ब्रह्म - व॒र्च॒सस्य॑ । परि॑गृहीत्या॒ इति॒ परि॑ - गृ॒ही॒त्यै॒ ॥  \newline


\textbf{Krama Paata} \newline

य॒न्त्यथो᳚ । अथो॑ अ॒नयोः᳚ । अथो॒ इत्यथो᳚ । अ॒नयो॑रे॒व । ए॒व प्रति॑ । प्रति॑ तिष्ठन्ति । ति॒ष्ठ॒न्त्ये॒ते । ए॒ते वै । ए॒ते इत्ये॒ते । वै य॒ज्ञ्स्य॑ । य॒ज्ञ्स्या᳚ञ्ज॒साय॑नी । अ॒ञ्ज॒साय॑नी स्रु॒ती । अ॒ञ्ज॒साय॑नी॒ इत्य॑ञ्जसा - अय॑नी । स्रु॒ती ताभ्या᳚म् । स्रु॒ती इति॑ स्रु॒ती । ताभ्या॑मे॒व । ए॒व सु॑व॒र्गम् । सु॒व॒र्गम् ॅलो॒कम् । सु॒व॒र्गमिति॑ सुवः - गम् । लो॒कम् ॅय॑न्ति । य॒न्ति॒ च॒तु॒र्विꣳ॒॒श॒ति॒रा॒त्रः । च॒तु॒र्विꣳ॒॒श॒ति॒रा॒त्रो भ॑वति । च॒तु॒र्विꣳ॒॒श॒ति॒रा॒त्र इति॑ चतुर्विꣳशति - रा॒त्रः । भ॒व॒ति॒ चतु॑र्विꣳशतिः । चतु॑र्विꣳशतिरर्द्धमा॒साः । चतु॑र्विꣳशति॒रिति॒ चतुः॑ - विꣳ॒॒श॒तिः॒ । अ॒र्द्ध॒मा॒साः स॑म्ॅवथ्स॒रः । अ॒र्द्ध॒मा॒सा इत्य॑र्द्ध - मा॒साः । स॒म्ॅव॒थ्स॒रः स॑म्ॅवथ्स॒रः । स॒म्ॅव॒थ्स॒र इति॑ सम् - व॒थ्स॒रः । स॒म्ॅव॒थ्स॒रः सु॑व॒र्गः । स॒म्ॅव॒थ्स॒र इति॑ सम् - व॒थ्स॒रः । सु॒व॒र्गो लो॒कः । सु॒व॒र्ग इति॑ सुवः - गः । लो॒कः स॑म्ॅवथ्स॒रे । स॒म्ॅव॒थ्स॒र ए॒व । स॒म्ॅव॒थ्स॒र इति॑ सम् - व॒थ्स॒रे । ए॒व सु॑व॒र्गे । सु॒व॒र्गे लो॒के । सु॒व॒र्ग इति॑ सुवः - गे । लो॒के प्रति॑ । प्रति॑ तिष्ठन्ति । ति॒ष्ठ॒न्त्यथो᳚ । अथो॒ चतु॑र्विꣳशत्यक्षरा । अथो॒ इत्यथो᳚ । चतु॑र्विꣳशत्यक्षरा गाय॒त्री । चतु॑र्विꣳशत्यक्ष॒रेति॒ चतु॑र्विꣳशति - अ॒क्ष॒रा॒ । गा॒य॒त्री गा॑य॒त्री । गा॒य॒त्री ब्र॑ह्मवर्च॒सम् । ब्र॒ह्म॒व॒र्च॒सम् गा॑यत्रि॒या । ब्र॒ह्म॒व॒र्च॒समिति॑ ब्रह्म - व॒र्च॒सम् । गा॒य॒त्रि॒यैव । ए॒व ब्र॑ह्मवर्च॒सम् । ब्र॒ह्म॒व॒र्च॒समव॑ । ब्र॒ह्म॒व॒र्च॒समिति॑ ब्रह्म - व॒र्च॒सम् । अव॑ रुन्धते । रु॒न्ध॒ते॒ऽति॒रा॒त्रौ । अ॒ति॒रा॒त्राव॒भितः॑ । अ॒ति॒रा॒त्रावित्य॑ति - रा॒त्रौ । अ॒भितो॑ भवतः । भ॒व॒तो॒ ब्र॒ह्म॒व॒र्च॒सस्य॑ । ब्र॒ह्म॒व॒र्च॒सस्य॒ परि॑गृहीत्यै । ब्र॒ह्म॒व॒र्च॒सस्येति॑ ब्रह्म - व॒र्च॒सस्य॑ । परि॑गृहित्या॒ इति॒ परि॑ - गृ॒ही॒त्यै॒ । \newline

\textbf{Jatai Paata} \newline

1. य॒न् त्यथो॒ अथो॑ यन्ति य॒न्त्यथो᳚ । \newline
2. अथो॑ अ॒नयो॑ र॒नयो॒ रथो॒ अथो॑ अ॒नयोः᳚ । \newline
3. अथो॒ इत्यथो᳚ । \newline
4. अ॒नयो॑ रे॒वै वानयो॑ र॒नयो॑ रे॒व । \newline
5. ए॒व प्रति॒ प्रत्ये॒वैव प्रति॑ । \newline
6. प्रति॑ तिष्ठन्ति तिष्ठन्ति॒ प्रति॒ प्रति॑ तिष्ठन्ति । \newline
7. ति॒ष्ठ॒न्त्ये॒ते ए॒ते ति॑ष्ठन्ति तिष्ठन्त्ये॒ते । \newline
8. ए॒ते वै वा ए॒ते ए॒ते वै । \newline
9. ए॒ते इत्ये॒ते । \newline
10. वै य॒ज्ञ्स्य॑ य॒ज्ञ्स्य॒ वै वै य॒ज्ञ्स्य॑ । \newline
11. य॒ज्ञ्स्या᳚ ञ्ज॒साय॑नी अञ्ज॒साय॑नी य॒ज्ञ्स्य॑ य॒ज्ञ्स्या᳚ ञ्ज॒साय॑नी । \newline
12. अ॒ञ्ज॒साय॑नी स्रु॒ती स्रु॒ती अ॑ञ्ज॒साय॑नी अञ्ज॒साय॑नी स्रु॒ती । \newline
13. अ॒ञ्ज॒साय॑नी॒ इत्य॑ञ्जसा - अय॑नी । \newline
14. स्रु॒ती ताभ्या॒म् ताभ्याꣳ॑ स्रु॒ती स्रु॒ती ताभ्या᳚म् । \newline
15. स्रु॒ती इति॑ स्रु॒ती । \newline
16. ताभ्या॑ मे॒वैव ताभ्या॒म् ताभ्या॑ मे॒व । \newline
17. ए॒व सु॑व॒र्गꣳ सु॑व॒र्ग मे॒वैव सु॑व॒र्गम् । \newline
18. सु॒व॒र्गम् ॅलो॒कम् ॅलो॒कꣳ सु॑व॒र्गꣳ सु॑व॒र्गम् ॅलो॒कम् । \newline
19. सु॒व॒र्गमिति॑ सुवः - गम् । \newline
20. लो॒कं ॅय॑न्ति यन्ति लो॒कम् ॅलो॒कं ॅय॑न्ति । \newline
21. य॒न्ति॒ च॒तु॒र्विꣳ॒॒श॒ति॒रा॒त्र श्च॑तुर्विꣳशतिरा॒त्रो य॑न्ति यन्ति चतुर्विꣳशतिरा॒त्रः । \newline
22. च॒तु॒र्विꣳ॒॒श॒ति॒रा॒त्रो भ॑वति भवति चतुर्विꣳशतिरा॒त्र श्च॑तुर्विꣳशतिरा॒त्रो भ॑वति । \newline
23. च॒तु॒र्विꣳ॒॒श॒ति॒रा॒त्र इति॑ चतुर्विꣳशति - रा॒त्रः । \newline
24. भ॒व॒ति॒ चतु॑र्विꣳशति॒ श्चतु॑र्विꣳशतिर् भवति भवति॒ चतु॑र्विꣳशतिः । \newline
25. चतु॑र्विꣳशति रर्द्धमा॒सा अ॑र्द्धमा॒सा श्चतु॑र्विꣳशति॒ श्चतु॑र्विꣳशति रर्द्धमा॒साः । \newline
26. चतु॑र्विꣳशति॒रिति॒ चतुः॑ - विꣳ॒॒श॒तिः॒ । \newline
27. अ॒र्द्ध॒मा॒साः सं॑ॅवथ्स॒रः सं॑ॅवथ्स॒रो᳚ ऽर्द्धमा॒सा अ॑र्द्धमा॒साः सं॑ॅवथ्स॒रः । \newline
28. अ॒र्द्ध॒मा॒सा इत्य॑र्द्ध - मा॒साः । \newline
29. सं॒ॅव॒थ्स॒रः सं॑ॅवथ्स॒रः । \newline
30. सं॒ॅव॒थ्स॒र इति॑ सं - व॒थ्स॒रः । \newline
31. सं॒ॅव॒थ्स॒रः सु॑व॒र्गः सु॑व॒र्गः सं॑ॅवथ्स॒रः सं॑ॅवथ्स॒रः सु॑व॒र्गः । \newline
32. सं॒ॅव॒थ्स॒र इति॑ सं - व॒थ्स॒रः । \newline
33. सु॒व॒र्गो लो॒को लो॒कः सु॑व॒र्गः सु॑व॒र्गो लो॒कः । \newline
34. सु॒व॒र्ग इति॑ सुवः - गः । \newline
35. लो॒कः सं॑ॅवथ्स॒रे सं॑ॅवथ्स॒रे लो॒को लो॒कः सं॑ॅवथ्स॒रे । \newline
36. सं॒ॅव॒थ्स॒र ए॒वैव सं॑ॅवथ्स॒रे सं॑ॅवथ्स॒र ए॒व । \newline
37. सं॒ॅव॒थ्स॒र इति॑ सं - व॒थ्स॒रे । \newline
38. ए॒व सु॑व॒र्गे सु॑व॒र्ग ए॒वैव सु॑व॒र्गे । \newline
39. सु॒व॒र्गे लो॒के लो॒के सु॑व॒र्गे सु॑व॒र्गे लो॒के । \newline
40. सु॒व॒र्ग इति॑ सुवः - गे । \newline
41. लो॒के प्रति॒ प्रति॑ लो॒के लो॒के प्रति॑ । \newline
42. प्रति॑ तिष्ठन्ति तिष्ठन्ति॒ प्रति॒ प्रति॑ तिष्ठन्ति । \newline
43. ति॒ष्ठ॒न्त्यथो॒ अथो॑ तिष्ठन्ति तिष्ठ॒न्त्यथो᳚ । \newline
44. अथो॒ चतु॑र्विꣳशत्यक्षरा॒ चतु॑र्विꣳशत्यक्ष॒रा ऽथो॒ अथो॒ चतु॑र्विꣳशत्यक्षरा । \newline
45. अथो॒ इत्यथो᳚ । \newline
46. चतु॑र्विꣳशत्यक्षरा गाय॒त्री गा॑य॒त्री चतु॑र्विꣳशत्यक्षरा॒ चतु॑र्विꣳशत्यक्षरा गाय॒त्री । \newline
47. चतु॑र्विꣳशत्यक्ष॒रेति॒ चतु॑र्विꣳशति - अ॒क्ष॒रा॒ । \newline
48. गा॒य॒त्री गा॑य॒त्री । \newline
49. गा॒य॒त्री ब्र॑ह्मवर्च॒सम् ब्र॑ह्मवर्च॒सम् गा॑य॒त्री गा॑य॒त्री ब्र॑ह्मवर्च॒सम् । \newline
50. ब्र॒ह्म॒व॒र्च॒सम् गा॑यत्रि॒या गा॑यत्रि॒या ब्र॑ह्मवर्च॒सम् ब्र॑ह्मवर्च॒सम् गा॑यत्रि॒या । \newline
51. ब्र॒ह्म॒व॒र्च॒समिति॑ ब्रह्म - व॒र्च॒सम् । \newline
52. गा॒य॒त्रि॒यैवैव गा॑यत्रि॒या गा॑यत्रि॒यैव । \newline
53. ए॒व ब्र॑ह्मवर्च॒सम् ब्र॑ह्मवर्च॒स मे॒वैव ब्र॑ह्मवर्च॒सम् । \newline
54. ब्र॒ह्म॒व॒र्च॒स मवाव॑ ब्रह्मवर्च॒सम् ब्र॑ह्मवर्च॒स मव॑ । \newline
55. ब्र॒ह्म॒व॒र्च॒समिति॑ ब्रह्म - व॒र्च॒सम् । \newline
56. अव॑ रुन्धते रुन्ध॒ते ऽवाव॑ रुन्धते । \newline
57. रु॒न्ध॒ते॒ ऽति॒रा॒त्रा व॑तिरा॒त्रौ रु॑न्धते रुन्धते ऽतिरा॒त्रौ । \newline
58. अ॒ति॒रा॒त्रा व॒भितो॒ ऽभितो॑ ऽतिरा॒त्रा व॑तिरा॒त्रा व॒भितः॑ । \newline
59. अ॒ति॒रा॒त्रावित्य॑ति - रा॒त्रौ । \newline
60. अ॒भितो॑ भवतो भवतो॒ ऽभितो॒ ऽभितो॑ भवतः । \newline
61. भ॒व॒तो॒ ब्र॒ह्म॒व॒र्च॒सस्य॑ ब्रह्मवर्च॒सस्य॑ भवतो भवतो ब्रह्मवर्च॒सस्य॑ । \newline
62. ब्र॒ह्म॒व॒र्च॒सस्य॒ परि॑गृहीत्यै॒ परि॑गृहीत्यै ब्रह्मवर्च॒सस्य॑ ब्रह्मवर्च॒सस्य॒ परि॑गृहीत्यै । \newline
63. ब्र॒ह्म॒व॒र्च॒सस्येति॑ ब्रह्म - व॒र्च॒सस्य॑ । \newline
64. परि॑गृहीत्या॒ इति॒ परि॑ - गृ॒ही॒त्यै॒ । \newline

\textbf{Ghana Paata } \newline

1. य॒न्त्यथो॒ अथो॑ यन्ति य॒न्त्यथो॑ अ॒नयो॑ र॒नयो॒ रथो॑ यन्ति य॒न्त्यथो॑ अ॒नयोः᳚ । \newline
2. अथो॑ अ॒नयो॑ र॒नयो॒ रथो॒ अथो॑ अ॒नयो॑ रे॒वै वानयो॒ रथो॒ अथो॑ अ॒नयो॑ रे॒व । \newline
3. अथो॒ इत्यथो᳚ । \newline
4. अ॒नयो॑ रे॒वै वानयो॑ र॒नयो॑ रे॒व प्रति॒ प्रत्ये॒वा नयो॑ र॒नयो॑ रे॒व प्रति॑ । \newline
5. ए॒व प्रति॒ प्रत्ये॒वैव प्रति॑ तिष्ठन्ति तिष्ठन्ति॒ प्रत्ये॒वैव प्रति॑ तिष्ठन्ति । \newline
6. प्रति॑ तिष्ठन्ति तिष्ठन्ति॒ प्रति॒ प्रति॑ तिष्ठ न्त्ये॒ते ए॒ते ति॑ष्ठन्ति॒ प्रति॒ प्रति॑ तिष्ठ न्त्ये॒ते । \newline
7. ति॒ष्ठ॒ न्त्ये॒ते ए॒ते ति॑ष्ठन्ति तिष्ठ न्त्ये॒ते वै वा ए॒ते ति॑ष्ठन्ति तिष्ठ न्त्ये॒ते वै । \newline
8. ए॒ते वै वा ए॒ते ए॒ते वै य॒ज्ञ्स्य॑ य॒ज्ञ्स्य॒ वा ए॒ते ए॒ते वै य॒ज्ञ्स्य॑ । \newline
9. ए॒ते इत्ये॒ते । \newline
10. वै य॒ज्ञ्स्य॑ य॒ज्ञ्स्य॒ वै वै य॒ज्ञ्स्या᳚ ञ्ज॒साय॑नी अञ्ज॒साय॑नी य॒ज्ञ्स्य॒ वै वै य॒ज्ञ्स्या᳚ ञ्ज॒साय॑नी । \newline
11. य॒ज्ञ्स्या᳚ ञ्ज॒साय॑नी अञ्ज॒साय॑नी य॒ज्ञ्स्य॑ य॒ज्ञ्स्या᳚ ञ्ज॒साय॑नी स्रु॒ती स्रु॒ती अ॑ञ्ज॒साय॑नी य॒ज्ञ्स्य॑ य॒ज्ञ्स्या᳚ ञ्ज॒साय॑नी स्रु॒ती । \newline
12. अ॒ञ्ज॒साय॑नी स्रु॒ती स्रु॒ती अ॑ञ्ज॒साय॑नी अञ्ज॒साय॑नी स्रु॒ती ताभ्या॒म् ताभ्याꣳ॑ स्रु॒ती अ॑ञ्ज॒साय॑नी अञ्ज॒साय॑नी स्रु॒ती ताभ्या᳚म् । \newline
13. अ॒ञ्ज॒साय॑नी॒ इत्य॑ञ्जसा - अय॑नी । \newline
14. स्रु॒ती ताभ्या॒म् ताभ्याꣳ॑ स्रु॒ती स्रु॒ती ताभ्या॑ मे॒वैव ताभ्याꣳ॑ स्रु॒ती स्रु॒ती ताभ्या॑ मे॒व । \newline
15. स्रु॒ती इति॑ स्रु॒ती । \newline
16. ताभ्या॑ मे॒वैव ताभ्या॒म् ताभ्या॑ मे॒व सु॑व॒र्गꣳ सु॑व॒र्ग मे॒व ताभ्या॒म् ताभ्या॑ मे॒व सु॑व॒र्गम् । \newline
17. ए॒व सु॑व॒र्गꣳ सु॑व॒र्ग मे॒वैव सु॑व॒र्गम् ॅलो॒कम् ॅलो॒कꣳ सु॑व॒र्ग मे॒वैव सु॑व॒र्गम् ॅलो॒कम् । \newline
18. सु॒व॒र्गम् ॅलो॒कम् ॅलो॒कꣳ सु॑व॒र्गꣳ सु॑व॒र्गम् ॅलो॒कं ॅय॑न्ति यन्ति लो॒कꣳ सु॑व॒र्गꣳ सु॑व॒र्गम् ॅलो॒कं ॅय॑न्ति । \newline
19. सु॒व॒र्गमिति॑ सुवः - गम् । \newline
20. लो॒कं ॅय॑न्ति यन्ति लो॒कम् ॅलो॒कं ॅय॑न्ति चतुर्विꣳशतिरा॒त्र श्च॑तुर्विꣳशतिरा॒त्रो य॑न्ति लो॒कम् ॅलो॒कं ॅय॑न्ति चतुर्विꣳशतिरा॒त्रः । \newline
21. य॒न्ति॒ च॒तु॒र्विꣳ॒॒श॒ति॒रा॒त्र श्च॑तुर्विꣳशतिरा॒त्रो य॑न्ति यन्ति चतुर्विꣳशतिरा॒त्रो भ॑वति भवति चतुर्विꣳशतिरा॒त्रो य॑न्ति यन्ति चतुर्विꣳशतिरा॒त्रो भ॑वति । \newline
22. च॒तु॒र्विꣳ॒॒श॒ति॒रा॒त्रो भ॑वति भवति चतुर्विꣳशतिरा॒त्र श्च॑तुर्विꣳशतिरा॒त्रो भ॑वति॒ चतु॑र्विꣳशति॒ श्चतु॑र्विꣳशतिर् भवति चतुर्विꣳशतिरा॒त्र श्च॑तुर्विꣳशतिरा॒त्रो भ॑वति॒ चतु॑र्विꣳशतिः । \newline
23. च॒तु॒र्विꣳ॒॒श॒ति॒रा॒त्र इति॑ चतुर्विꣳशति - रा॒त्रः । \newline
24. भ॒व॒ति॒ चतु॑र्विꣳशति॒ श्चतु॑र्विꣳशतिर् भवति भवति॒ चतु॑र्विꣳशति रर्द्धमा॒सा अ॑र्द्धमा॒सा श्चतु॑र्विꣳशतिर् भवति भवति॒ चतु॑र्विꣳशति रर्द्धमा॒साः । \newline
25. चतु॑र्विꣳशति रर्द्धमा॒सा अ॑र्द्धमा॒सा श्चतु॑र्विꣳशति॒ श्चतु॑र्विꣳशति रर्द्धमा॒साः सं॑ॅवथ्स॒रः सं॑ॅवथ्स॒रो᳚ ऽर्द्धमा॒सा श्चतु॑र्विꣳशति॒ श्चतु॑र्विꣳशति रर्द्धमा॒साः सं॑ॅवथ्स॒रः । \newline
26. चतु॑र्विꣳशति॒रिति॒ चतुः॑ - विꣳ॒॒श॒तिः॒ । \newline
27. अ॒र्द्ध॒मा॒साः सं॑ॅवथ्स॒रः सं॑ॅवथ्स॒रो᳚ ऽर्द्धमा॒सा अ॑र्द्धमा॒साः सं॑ॅवथ्स॒रः । \newline
28. अ॒र्द्ध॒मा॒सा इत्य॑र्द्ध - मा॒साः । \newline
29. सं॒ॅव॒थ्स॒रः सं॑ॅवथ्स॒रः । \newline
30. सं॒ॅव॒थ्स॒र इति॑ सं - व॒थ्स॒रः । \newline
31. सं॒ॅव॒थ्स॒रः सु॑व॒र्गः सु॑व॒र्गः सं॑ॅवथ्स॒रः सं॑ॅवथ्स॒रः सु॑व॒र्गो लो॒को लो॒कः सु॑व॒र्गः सं॑ॅवथ्स॒रः सं॑ॅवथ्स॒रः सु॑व॒र्गो लो॒कः । \newline
32. सं॒ॅव॒थ्स॒र इति॑ सं - व॒थ्स॒रः । \newline
33. सु॒व॒र्गो लो॒को लो॒कः सु॑व॒र्गः सु॑व॒र्गो लो॒कः सं॑ॅवथ्स॒रे सं॑ॅवथ्स॒रे लो॒कः सु॑व॒र्गः सु॑व॒र्गो लो॒कः सं॑ॅवथ्स॒रे । \newline
34. सु॒व॒र्ग इति॑ सुवः - गः । \newline
35. लो॒कः सं॑ॅवथ्स॒रे सं॑ॅवथ्स॒रे लो॒को लो॒कः सं॑ॅवथ्स॒र ए॒वैव सं॑ॅवथ्स॒रे लो॒को लो॒कः सं॑ॅवथ्स॒र ए॒व । \newline
36. सं॒ॅव॒थ्स॒र ए॒वैव सं॑ॅवथ्स॒रे सं॑ॅवथ्स॒र ए॒व सु॑व॒र्गे सु॑व॒र्ग ए॒व सं॑ॅवथ्स॒रे सं॑ॅवथ्स॒र ए॒व सु॑व॒र्गे । \newline
37. सं॒ॅव॒थ्स॒र इति॑ सं - व॒थ्स॒रे । \newline
38. ए॒व सु॑व॒र्गे सु॑व॒र्ग ए॒वैव सु॑व॒र्गे लो॒के लो॒के सु॑व॒र्ग ए॒वैव सु॑व॒र्गे लो॒के । \newline
39. सु॒व॒र्गे लो॒के लो॒के सु॑व॒र्गे सु॑व॒र्गे लो॒के प्रति॒ प्रति॑ लो॒के सु॑व॒र्गे सु॑व॒र्गे लो॒के प्रति॑ । \newline
40. सु॒व॒र्ग इति॑ सुवः - गे । \newline
41. लो॒के प्रति॒ प्रति॑ लो॒के लो॒के प्रति॑ तिष्ठन्ति तिष्ठन्ति॒ प्रति॑ लो॒के लो॒के प्रति॑ तिष्ठन्ति । \newline
42. प्रति॑ तिष्ठन्ति तिष्ठन्ति॒ प्रति॒ प्रति॑ तिष्ठ॒ न्त्यथो॒ अथो॑ तिष्ठन्ति॒ प्रति॒ प्रति॑ तिष्ठ॒ न्त्यथो᳚ । \newline
43. ति॒ष्ठ॒ न्त्यथो॒ अथो॑ तिष्ठन्ति तिष्ठ॒ न्त्यथो॒ चतु॑र्विꣳशत्यक्षरा॒ चतु॑र्विꣳशत्यक्ष॒रा ऽथो॑ तिष्ठन्ति तिष्ठ॒ न्त्यथो॒ चतु॑र्विꣳशत्यक्षरा । \newline
44. अथो॒ चतु॑र्विꣳशत्यक्षरा॒ चतु॑र्विꣳशत्यक्ष॒रा ऽथो॒ अथो॒ चतु॑र्विꣳशत्यक्षरा गाय॒त्री गा॑य॒त्री चतु॑र्विꣳशत्यक्ष॒रा ऽथो॒ अथो॒ चतु॑र्विꣳशत्यक्षरा गाय॒त्री । \newline
45. अथो॒ इत्यथो᳚ । \newline
46. चतु॑र्विꣳशत्यक्षरा गाय॒त्री गा॑य॒त्री चतु॑र्विꣳशत्यक्षरा॒ चतु॑र्विꣳशत्यक्षरा गाय॒त्री । \newline
47. चतु॑र्विꣳशत्यक्ष॒रेति॒ चतु॑र्विꣳशति - अ॒क्ष॒रा॒ । \newline
48. गा॒य॒त्री गा॑य॒त्री । \newline
49. गा॒य॒त्री ब्र॑ह्मवर्च॒सम् ब्र॑ह्मवर्च॒सम् गा॑य॒त्री गा॑य॒त्री ब्र॑ह्मवर्च॒सम् गा॑यत्रि॒या गा॑यत्रि॒या ब्र॑ह्मवर्च॒सम् गा॑य॒त्री गा॑य॒त्री ब्र॑ह्मवर्च॒सम् गा॑यत्रि॒या । \newline
50. ब्र॒ह्म॒व॒र्च॒सम् गा॑यत्रि॒या गा॑यत्रि॒या ब्र॑ह्मवर्च॒सम् ब्र॑ह्मवर्च॒सम् गा॑यत्रि॒यैवैव गा॑यत्रि॒या ब्र॑ह्मवर्च॒सम् ब्र॑ह्मवर्च॒सम् गा॑यत्रि॒यैव । \newline
51. ब्र॒ह्म॒व॒र्च॒समिति॑ ब्रह्म - व॒र्च॒सम् । \newline
52. गा॒य॒त्रि॒यैवैव गा॑यत्रि॒या गा॑यत्रि॒यैव ब्र॑ह्मवर्च॒सम् ब्र॑ह्मवर्च॒स मे॒व गा॑यत्रि॒या गा॑यत्रि॒यैव ब्र॑ह्मवर्च॒सम् । \newline
53. ए॒व ब्र॑ह्मवर्च॒सम् ब्र॑ह्मवर्च॒स मे॒वैव ब्र॑ह्मवर्च॒स मवाव॑ ब्रह्मवर्च॒स मे॒वैव ब्र॑ह्मवर्च॒स मव॑ । \newline
54. ब्र॒ह्म॒व॒र्च॒स मवाव॑ ब्रह्मवर्च॒सम् ब्र॑ह्मवर्च॒स मव॑ रुन्धते रुन्ध॒ते ऽव॑ ब्रह्मवर्च॒सम् ब्र॑ह्मवर्च॒स मव॑ रुन्धते । \newline
55. ब्र॒ह्म॒व॒र्च॒समिति॑ ब्रह्म - व॒र्च॒सम् । \newline
56. अव॑ रुन्धते रुन्ध॒ते ऽवाव॑ रुन्धते ऽतिरा॒त्रा व॑तिरा॒त्रौ रु॑न्ध॒ते ऽवाव॑ रुन्धते ऽतिरा॒त्रौ । \newline
57. रु॒न्ध॒ते॒ ऽति॒रा॒त्रा व॑तिरा॒त्रौ रु॑न्धते रुन्धते ऽतिरा॒त्रा व॒भितो॒ ऽभितो॑ ऽतिरा॒त्रौ रु॑न्धते रुन्धते ऽतिरा॒त्रा व॒भितः॑ । \newline
58. अ॒ति॒रा॒त्रा व॒भितो॒ ऽभितो॑ ऽतिरा॒त्रा व॑तिरा॒त्रा व॒भितो॑ भवतो भवतो॒ ऽभितो॑ ऽतिरा॒त्रा व॑तिरा॒त्रा व॒भितो॑ भवतः । \newline
59. अ॒ति॒रा॒त्रावित्य॑ति - रा॒त्रौ । \newline
60. अ॒भितो॑ भवतो भवतो॒ ऽभितो॒ ऽभितो॑ भवतो ब्रह्मवर्च॒सस्य॑ ब्रह्मवर्च॒सस्य॑ भवतो॒ ऽभितो॒ ऽभितो॑ भवतो ब्रह्मवर्च॒सस्य॑ । \newline
61. भ॒व॒तो॒ ब्र॒ह्म॒व॒र्च॒सस्य॑ ब्रह्मवर्च॒सस्य॑ भवतो भवतो ब्रह्मवर्च॒सस्य॒ परि॑गृहीत्यै॒ परि॑गृहीत्यै ब्रह्मवर्च॒सस्य॑ भवतो भवतो ब्रह्मवर्च॒सस्य॒ परि॑गृहीत्यै । \newline
62. ब्र॒ह्म॒व॒र्च॒सस्य॒ परि॑गृहीत्यै॒ परि॑गृहीत्यै ब्रह्मवर्च॒सस्य॑ ब्रह्मवर्च॒सस्य॒ परि॑गृहीत्यै । \newline
63. ब्र॒ह्म॒व॒र्च॒सस्येति॑ ब्रह्म - व॒र्च॒सस्य॑ । \newline
64. परि॑गृहीत्या॒ इति॒ परि॑ - गृ॒ही॒त्यै॒ । \newline
\pagebreak
\markright{ TS 7.4.2.1  \hfill https://www.vedavms.in \hfill}

\section{ TS 7.4.2.1 }

\textbf{TS 7.4.2.1 } \newline
\textbf{Samhita Paata} \newline

यथा॒ वै म॑नुष्या॑ ए॒वं दे॒वा अग्र॑ आस॒न् ते॑ऽकामय॒न्ताव॑र्तिं पा॒प्मानं॑ मृ॒त्युम॑प॒हत्य॒ दैवीꣳ॑ सꣳ॒॒सदं॑ गच्छे॒मेति॒ त ए॒तं च॑तुर्विꣳशतिरा॒त्र-म॑पश्य॒न् तमाऽह॑र॒न् तेना॑यजन्त॒ ततो॒ वै तेऽव॑र्तिं पा॒प्मानं॑ मृ॒त्युम॑प॒हत्य॒ दैवीꣳ॑ सꣳ॒॒सद॑मगच्छ॒न॒. य ए॒वं ॅवि॒द्वाꣳस॑-श्चतुर्विꣳशतिरा॒त्र-मास॒तेऽव॑र्तिमे॒व पा॒प्मान॑-मप॒हत्य॒ श्रियं॑ गच्छन्ति॒ श्रीर्.हि म॑नु॒ष्य॑स्य॒ - [  ] \newline

\textbf{Pada Paata} \newline

यथा᳚ । वै । म॒नु॒ष्याः᳚ । ए॒वम् । दे॒वाः । अग्रे᳚ । आ॒स॒न्न् । ते । अ॒का॒म॒य॒न्त॒ । अव॑र्तिम् । पा॒प्मान᳚म् । मृ॒त्युम् । अ॒प॒हत्येत्य॑प-हत्य॑ । दैवी᳚म् । सꣳ॒॒सद॒मिति॑ सं - सद᳚म् । ग॒च्छे॒म॒ । इति॑ । ते । ए॒तम् । च॒तु॒र्विꣳ॒॒श॒ति॒रा॒त्रमिति॑ चतुर्विꣳशति - रा॒त्रम् । अ॒प॒श्य॒न्न् । तम् । एति॑ । अ॒ह॒र॒न्न् । तेन॑ । अ॒य॒ज॒न्त॒ । ततः॑ । वै । ते । अव॑र्तिम् । पा॒प्मान᳚म् । मृ॒त्युम् । अ॒प॒हत्येत्य॑प- हत्य॑ । दैवी᳚म् । सꣳ॒॒सद॒मिति॑ सं - सद᳚म् । अ॒ग॒च्छ॒न्न् । ये । ए॒वम् । वि॒द्वाꣳसः॑ । च॒तु॒र्विꣳ॒॒श॒ति॒रा॒त्रमिति॑ चतुर्विꣳशति - रा॒त्रम् । आस॑ते । अव॑र्तिम् । ए॒व । पा॒प्मान᳚म् । अ॒प॒हत्येत्य॑प- हत्य॑ । श्रिय᳚म् । ग॒च्छ॒न्ति॒ । श्रीः । हि । म॒नु॒ष्य॑स्य ।  \newline


\textbf{Krama Paata} \newline

यथा॒ वै । वै म॑नु॒ष्याः᳚ । म॒नु॒ष्या॑ ए॒वम् । ए॒वम् दे॒वाः । दे॒वा अग्रे᳚ । अग्र॑ आसन्न् । आ॒स॒न् ते । ते॑ऽकामयन्त । अ॒का॒म॒य॒न्ताव॑र्तिम् । अव॑र्तिम् पा॒प्मान᳚म् । पा॒प्मान॑म् मृ॒त्युम् । मृ॒त्युम॑प॒हत्य॑ । अ॒प॒हत्य॒ दैवी᳚म् । अ॒प॒हत्येत्य॑प - हत्य॑ । दैवीꣳ॑ सꣳ॒॒सद᳚म् । सꣳ॒॒सद॑म् गच्छेम । सꣳ॒॒सद॒मिति॑ सम् - सद᳚म् । ग॒च्छे॒मेति॑ । इति॒ ते । त ए॒तम् । ए॒तम् च॑तुर्विꣳशतिरा॒त्रम् । च॒तु॒र्विꣳ॒॒श॒ति॒रा॒त्रम॑पश्यन्न् । च॒तु॒र्विꣳ॒॒श॒ति॒रा॒त्रमिति॑ चतुर्विꣳशति - रा॒त्रम् । अ॒प॒श्य॒न् तम् । तमा । आऽह॑रन्न् । अ॒ह॒र॒न् तेन॑ । तेना॑यजन्त । अ॒य॒ज॒न्त॒ ततः॑ । ततो॒ वै । वै ते । तेऽव॑र्तिम् । अव॑र्तिम् पा॒प्मान᳚म् । पा॒प्मान॑म् मृ॒त्युम् । मृ॒त्युम॑प॒हत्य॑ । अ॒प॒हत्य॒ दैवी᳚म् । अ॒प॒हत्येत्य॑प - हत्य॑ । दैवीꣳ॑ सꣳ॒॒सद᳚म् । सꣳ॒॒सद॑मगच्छन्न् । सꣳ॒॒सद॒मिति॑ सम् - सद᳚म् । अ॒ग॒च्छ॒न्॒. ये । य ए॒वम् । ए॒वम् ॅवि॒द्वाꣳसः॑ । वि॒द्वाꣳस॑श्चतुर्विꣳशतिरा॒त्रम् । च॒तु॒र्विꣳ॒॒श॒ति॒रा॒त्रमास॑ते । च॒तु॒र्विꣳ॒॒श॒ति॒रा॒त्रमिति॑ चतुर्विꣳशति - रा॒त्रम् । आस॒तेऽव॑र्तिम् । अव॑र्तिमे॒व । ए॒व पा॒प्मान᳚म् । पा॒प्मान॑मप॒हत्य॑ । अ॒प॒हत्य॒ श्रिय᳚म् । अ॒प॒हत्येत्य॑प - हत्य॑ । श्रिय॑म् गच्छन्ति । ग॒च्छ॒न्ति॒ श्रीः । श्रीर्. हि । हि म॑नु॒ष्य॑स्य । म॒नु॒ष्य॑स्य॒ दैवी᳚ \newline

\textbf{Jatai Paata} \newline

1. यथा॒ वै वै यथा॒ यथा॒ वै । \newline
2. वै म॑नु॒ष्या॑ मनु॒ष्या॑ वै वै म॑नु॒ष्याः᳚ । \newline
3. म॒नु॒ष्या॑ ए॒व मे॒वम् म॑नु॒ष्या॑ मनु॒ष्या॑ ए॒वम् । \newline
4. ए॒वम् दे॒वा दे॒वा ए॒व मे॒वम् दे॒वाः । \newline
5. दे॒वा अग्रे ऽग्रे॑ दे॒वा दे॒वा अग्रे᳚ । \newline
6. अग्र॑ आसन् नास॒न् नग्रे ऽग्र॑ आसन्न् । \newline
7. आ॒स॒न् ते त आ॑सन् नास॒न् ते । \newline
8. ते॑ ऽकामयन्ता कामयन्त॒ ते ते॑ ऽकामयन्त । \newline
9. अ॒का॒म॒य॒न्ता व॑र्ति॒ मव॑र्ति मकामयन्ता कामय॒न्ता व॑र्तिम् । \newline
10. अव॑र्तिम् पा॒प्मान॑म् पा॒प्मान॒ मव॑र्ति॒ मव॑र्तिम् पा॒प्मान᳚म् । \newline
11. पा॒प्मान॑म् मृ॒त्युम् मृ॒त्युम् पा॒प्मान॑म् पा॒प्मान॑म् मृ॒त्युम् । \newline
12. मृ॒त्यु म॑प॒हत्या॑ प॒हत्य॑ मृ॒त्युम् मृ॒त्यु म॑प॒हत्य॑ । \newline
13. अ॒प॒हत्य॒ दैवी॒म् दैवी॑ मप॒हत्या॑ प॒हत्य॒ दैवी᳚म् । \newline
14. अ॒प॒हत्येत्य॑प - हत्य॑ । \newline
15. दैवीꣳ॑ सꣳ॒॒सदꣳ॑ सꣳ॒॒सद॒म् दैवी॒म् दैवीꣳ॑ सꣳ॒॒सद᳚म् । \newline
16. सꣳ॒॒सद॑म् गच्छेम गच्छेम सꣳ॒॒सदꣳ॑ सꣳ॒॒सद॑म् गच्छेम । \newline
17. सꣳ॒॒सद॒मिति॑ सं - सद᳚म् । \newline
18. ग॒च्छे॒मेतीति॑ गच्छेम गच्छे॒मेति॑ । \newline
19. इति॒ ते त इतीति॒ ते । \newline
20. त ए॒त मे॒तम् ते त ए॒तम् । \newline
21. ए॒तम् च॑तुर्विꣳशतिरा॒त्रम् च॑तुर्विꣳशतिरा॒त्र मे॒त मे॒तम् च॑तुर्विꣳशतिरा॒त्रम् । \newline
22. च॒तु॒र्विꣳ॒॒श॒ति॒रा॒त्र म॑पश्यन् नपश्यꣳ श्चतुर्विꣳशतिरा॒त्रम् च॑तुर्विꣳशतिरा॒त्र म॑पश्यन्न् । \newline
23. च॒तु॒र्विꣳ॒॒श॒ति॒रा॒त्रमिति॑ चतुर्विꣳशति - रा॒त्रम् । \newline
24. अ॒प॒श्य॒न् तम् त म॑पश्यन् नपश्य॒न् तम् । \newline
25. त मा तम् त मा । \newline
26. आ ऽह॑रन् नहर॒न् ना ऽह॑रन्न् । \newline
27. अ॒ह॒र॒न् तेन॒ तेना॑ हरन् नहर॒न् तेन॑ । \newline
28. तेना॑ यजन्ता यजन्त॒ तेन॒ तेना॑ यजन्त । \newline
29. अ॒य॒ज॒न्त॒ तत॒ स्ततो॑ ऽयजन्ता यजन्त॒ ततः॑ । \newline
30. ततो॒ वै वै तत॒ स्ततो॒ वै । \newline
31. वै ते ते वै वै ते । \newline
32. ते ऽव॑र्ति॒ मव॑र्ति॒म् ते ते ऽव॑र्तिम् । \newline
33. अव॑र्तिम् पा॒प्मान॑म् पा॒प्मान॒ मव॑र्ति॒ मव॑र्तिम् पा॒प्मान᳚म् । \newline
34. पा॒प्मान॑म् मृ॒त्युम् मृ॒त्युम् पा॒प्मान॑म् पा॒प्मान॑म् मृ॒त्युम् । \newline
35. मृ॒त्यु म॑प॒हत्या॑ प॒हत्य॑ मृ॒त्युम् मृ॒त्यु म॑प॒हत्य॑ । \newline
36. अ॒प॒हत्य॒ दैवी॒म् दैवी॑ मप॒हत्या॑ प॒हत्य॒ दैवी᳚म् । \newline
37. अ॒प॒हत्येत्य॑प - हत्य॑ । \newline
38. दैवीꣳ॑ सꣳ॒॒सदꣳ॑ सꣳ॒॒सद॒म् दैवी॒म् दैवीꣳ॑ सꣳ॒॒सद᳚म् । \newline
39. सꣳ॒॒सद॑ मगच्छन् नगच्छन् थ्सꣳ॒॒सदꣳ॑ सꣳ॒॒सद॑ मगच्छन्न् । \newline
40. सꣳ॒॒सद॒मिति॑ सं - सद᳚म् । \newline
41. अ॒ग॒च्छ॒न्॒. ये ये॑ ऽगच्छन् नगच्छ॒न्॒. ये । \newline
42. य ए॒व मे॒वं ॅये य ए॒वम् । \newline
43. ए॒वं ॅवि॒द्वाꣳसो॑ वि॒द्वाꣳस॑ ए॒व मे॒वं ॅवि॒द्वाꣳसः॑ । \newline
44. वि॒द्वाꣳस॑ श्चतुर्विꣳशतिरा॒त्रम् च॑तुर्विꣳशतिरा॒त्रं ॅवि॒द्वाꣳसो॑ वि॒द्वाꣳस॑ श्चतुर्विꣳशतिरा॒त्रम् । \newline
45. च॒तु॒र्विꣳ॒॒श॒ति॒रा॒त्र मास॑त॒ आस॑ते चतुर्विꣳशतिरा॒त्रम् च॑तुर्विꣳशतिरा॒त्र मास॑ते । \newline
46. च॒तु॒र्विꣳ॒॒श॒ति॒रा॒त्रमिति॑ चतुर्विꣳशति - रा॒त्रम् । \newline
47. आस॒ते ऽव॑र्ति॒ मव॑र्ति॒ मास॑त॒ आस॒ते ऽव॑र्तिम् । \newline
48. अव॑र्ति मे॒वै वाव॑र्ति॒ मव॑र्ति मे॒व । \newline
49. ए॒व पा॒प्मान॑म् पा॒प्मान॑ मे॒वैव पा॒प्मान᳚म् । \newline
50. पा॒प्मान॑ मप॒हत्या॑ प॒हत्य॑ पा॒प्मान॑म् पा॒प्मान॑ मप॒हत्य॑ । \newline
51. अ॒प॒हत्य॒ श्रियꣳ॒॒ श्रिय॑ मप॒हत्या॑ प॒हत्य॒ श्रिय᳚म् । \newline
52. अ॒प॒हत्येत्य॑प - हत्य॑ । \newline
53. श्रिय॑म् गच्छन्ति गच्छन्ति॒ श्रियꣳ॒॒ श्रिय॑म् गच्छन्ति । \newline
54. ग॒च्छ॒न्ति॒ श्रीः श्रीर् ग॑च्छन्ति गच्छन्ति॒ श्रीः । \newline
55. श्रीर्. हि हि श्रीः श्रीर्. हि । \newline
56. हि म॑नु॒ष्य॑स्य मनु॒ष्य॑स्य॒ हि हि म॑नु॒ष्य॑स्य । \newline
57. म॒नु॒ष्य॑स्य॒ दैवी॒ दैवी॑ मनु॒ष्य॑स्य मनु॒ष्य॑स्य॒ दैवी᳚ । \newline

\textbf{Ghana Paata } \newline

1. यथा॒ वै वै यथा॒ यथा॒ वै म॑नु॒ष्या॑ मनु॒ष्या॑ वै यथा॒ यथा॒ वै म॑नु॒ष्याः᳚ । \newline
2. वै म॑नु॒ष्या॑ मनु॒ष्या॑ वै वै म॑नु॒ष्या॑ ए॒व मे॒वम् म॑नु॒ष्या॑ वै वै म॑नु॒ष्या॑ ए॒वम् । \newline
3. म॒नु॒ष्या॑ ए॒व मे॒वम् म॑नु॒ष्या॑ मनु॒ष्या॑ ए॒वम् दे॒वा दे॒वा ए॒वम् म॑नु॒ष्या॑ मनु॒ष्या॑ ए॒वम् दे॒वाः । \newline
4. ए॒वम् दे॒वा दे॒वा ए॒व मे॒वम् दे॒वा अग्रे ऽग्रे॑ दे॒वा ए॒व मे॒वम् दे॒वा अग्रे᳚ । \newline
5. दे॒वा अग्रे ऽग्रे॑ दे॒वा दे॒वा अग्र॑ आसन् नास॒न् नग्रे॑ दे॒वा दे॒वा अग्र॑ आसन्न् । \newline
6. अग्र॑ आसन् नास॒न् नग्रे ऽग्र॑ आस॒न् ते त आ॑स॒न् नग्रे ऽग्र॑ आस॒न् ते । \newline
7. आ॒स॒न् ते त आ॑सन् नास॒न् ते॑ ऽकामयन्ता कामयन्त॒ त आ॑सन् नास॒न् ते॑ ऽकामयन्त । \newline
8. ते॑ ऽकामयन्ता कामयन्त॒ ते ते॑ ऽकामय॒न्ता व॑र्ति॒ मव॑र्ति मकामयन्त॒ ते ते॑ ऽकामय॒न्ता व॑र्तिम् । \newline
9. अ॒का॒म॒य॒न्ता व॑र्ति॒ मव॑र्ति मकामयन्ता कामय॒न्ता व॑र्तिम् पा॒प्मान॑म् पा॒प्मान॒ मव॑र्ति मकामयन्ता कामय॒न्ता व॑र्तिम् पा॒प्मान᳚म् । \newline
10. अव॑र्तिम् पा॒प्मान॑म् पा॒प्मान॒ मव॑र्ति॒ मव॑र्तिम् पा॒प्मान॑म् मृ॒त्युम् मृ॒त्युम् पा॒प्मान॒ मव॑र्ति॒ मव॑र्तिम् पा॒प्मान॑म् मृ॒त्युम् । \newline
11. पा॒प्मान॑म् मृ॒त्युम् मृ॒त्युम् पा॒प्मान॑म् पा॒प्मान॑म् मृ॒त्यु म॑प॒हत्या॑ प॒हत्य॑ मृ॒त्युम् पा॒प्मान॑म् पा॒प्मान॑म् मृ॒त्यु म॑प॒हत्य॑ । \newline
12. मृ॒त्यु म॑प॒हत्या॑ प॒हत्य॑ मृ॒त्युम् मृ॒त्यु म॑प॒हत्य॒ दैवी॒म् दैवी॑ मप॒हत्य॑ मृ॒त्युम् मृ॒त्यु म॑प॒हत्य॒ दैवी᳚म् । \newline
13. अ॒प॒हत्य॒ दैवी॒म् दैवी॑ मप॒हत्या॑ प॒हत्य॒ दैवीꣳ॑ सꣳ॒॒सदꣳ॑ सꣳ॒॒सद॒म् दैवी॑ मप॒हत्या॑ प॒हत्य॒ दैवीꣳ॑ सꣳ॒॒सद᳚म् । \newline
14. अ॒प॒हत्येत्य॑प - हत्य॑ । \newline
15. दैवीꣳ॑ सꣳ॒॒सदꣳ॑ सꣳ॒॒सद॒म् दैवी॒म् दैवीꣳ॑ सꣳ॒॒सद॑म् गच्छेम गच्छेम सꣳ॒॒सद॒म् दैवी॒म् दैवीꣳ॑ सꣳ॒॒सद॑म् गच्छेम । \newline
16. सꣳ॒॒सद॑म् गच्छेम गच्छेम सꣳ॒॒सदꣳ॑ सꣳ॒॒सद॑म् गच्छे॒मेतीति॑ गच्छेम सꣳ॒॒सदꣳ॑ सꣳ॒॒सद॑म् गच्छे॒मेति॑ । \newline
17. सꣳ॒॒सद॒मिति॑ सं - सद᳚म् । \newline
18. ग॒च्छे॒मेतीति॑ गच्छेम गच्छे॒मेति॒ ते त इति॑ गच्छेम गच्छे॒मेति॒ ते । \newline
19. इति॒ ते त इतीति॒ त ए॒त मे॒तम् त इतीति॒ त ए॒तम् । \newline
20. त ए॒त मे॒तम् ते त ए॒तम् च॑तुर्विꣳशतिरा॒त्रम् च॑तुर्विꣳशतिरा॒त्र मे॒तम् ते त ए॒तम् च॑तुर्विꣳशतिरा॒त्रम् । \newline
21. ए॒तम् च॑तुर्विꣳशतिरा॒त्रम् च॑तुर्विꣳशतिरा॒त्र मे॒त मे॒तम् च॑तुर्विꣳशतिरा॒त्र म॑पश्यन् नपश्यꣳ श्चतुर्विꣳशतिरा॒त्र मे॒त मे॒तम् च॑तुर्विꣳशतिरा॒त्र म॑पश्यन्न् । \newline
22. च॒तु॒र्विꣳ॒॒श॒ति॒रा॒त्र म॑पश्यन् नपश्यꣳ श्चतुर्विꣳशतिरा॒त्रम् च॑तुर्विꣳशतिरा॒त्र म॑पश्य॒न् तम् त म॑पश्यꣳश् चतुर्विꣳशतिरा॒त्रम् च॑तुर्विꣳशतिरा॒त्र म॑पश्य॒न् तम् । \newline
23. च॒तु॒र्विꣳ॒॒श॒ति॒रा॒त्रमिति॑ चतुर्विꣳशति - रा॒त्रम् । \newline
24. अ॒प॒श्य॒न् तम् त म॑पश्यन् नपश्य॒न् त मा त म॑पश्यन् नपश्य॒न् त मा । \newline
25. त मा तम् त मा ऽह॑रन् नहर॒न् ना तम् त मा ऽह॑रन्न् । \newline
26. आ ऽह॑रन् नहर॒न् ना ऽह॑र॒न् तेन॒ तेना॑ हर॒न् ना ऽह॑र॒न् तेन॑ । \newline
27. अ॒ह॒र॒न् तेन॒ तेना॑ हरन् नहर॒न् तेना॑ यजन्ता यजन्त॒ तेना॑ हरन् नहर॒न् तेना॑ यजन्त । \newline
28. तेना॑ यजन्ता यजन्त॒ तेन॒ तेना॑ यजन्त॒ तत॒ स्ततो॑ ऽयजन्त॒ तेन॒ तेना॑ यजन्त॒ ततः॑ । \newline
29. अ॒य॒ज॒न्त॒ तत॒ स्ततो॑ ऽयजन्ता यजन्त॒ ततो॒ वै वै ततो॑ ऽयजन्ता यजन्त॒ ततो॒ वै । \newline
30. ततो॒ वै वै तत॒ स्ततो॒ वै ते ते वै तत॒ स्ततो॒ वै ते । \newline
31. वै ते ते वै वै ते ऽव॑र्ति॒ मव॑र्ति॒म् ते वै वै ते ऽव॑र्तिम् । \newline
32. ते ऽव॑र्ति॒ मव॑र्ति॒म् ते ते ऽव॑र्तिम् पा॒प्मान॑म् पा॒प्मान॒ मव॑र्ति॒म् ते ते ऽव॑र्तिम् पा॒प्मान᳚म् । \newline
33. अव॑र्तिम् पा॒प्मान॑म् पा॒प्मान॒ मव॑र्ति॒ मव॑र्तिम् पा॒प्मान॑म् मृ॒त्युम् मृ॒त्युम् पा॒प्मान॒ मव॑र्ति॒ मव॑र्तिम् पा॒प्मान॑म् मृ॒त्युम् । \newline
34. पा॒प्मान॑म् मृ॒त्युम् मृ॒त्युम् पा॒प्मान॑म् पा॒प्मान॑म् मृ॒त्यु म॑प॒हत्या॑ प॒हत्य॑ मृ॒त्युम् पा॒प्मान॑म् पा॒प्मान॑म् मृ॒त्यु म॑प॒हत्य॑ । \newline
35. मृ॒त्यु म॑प॒हत्या॑ प॒हत्य॑ मृ॒त्युम् मृ॒त्यु म॑प॒हत्य॒ दैवी॒म् दैवी॑ मप॒हत्य॑ मृ॒त्युम् मृ॒त्यु म॑प॒हत्य॒ दैवी᳚म् । \newline
36. अ॒प॒हत्य॒ दैवी॒म् दैवी॑ मप॒हत्या॑ प॒हत्य॒ दैवीꣳ॑ सꣳ॒॒सदꣳ॑ सꣳ॒॒सद॒म् दैवी॑ मप॒हत्या॑ प॒हत्य॒ दैवीꣳ॑ सꣳ॒॒सद᳚म् । \newline
37. अ॒प॒हत्येत्य॑प - हत्य॑ । \newline
38. दैवीꣳ॑ सꣳ॒॒सदꣳ॑ सꣳ॒॒सद॒म् दैवी॒म् दैवीꣳ॑ सꣳ॒॒सद॑ मगच्छन् नगच्छन् थ्सꣳ॒॒सद॒म् दैवी॒म् दैवीꣳ॑ सꣳ॒॒सद॑ मगच्छन्न् । \newline
39. सꣳ॒॒सद॑ मगच्छन् नगच्छन् थ्सꣳ॒॒सदꣳ॑ सꣳ॒॒सद॑ मगच्छ॒न्॒. ये ये॑ ऽगच्छन् थ्सꣳ॒॒सदꣳ॑ सꣳ॒॒सद॑ मगच्छ॒न्॒. ये । \newline
40. सꣳ॒॒सद॒मिति॑ सं - सद᳚म् । \newline
41. अ॒ग॒च्छ॒न्॒. ये ये॑ ऽगच्छन् नगच्छ॒न्॒. य ए॒व मे॒वं ॅये॑ ऽगच्छन् नगच्छ॒न्॒. य ए॒वम् । \newline
42. य ए॒व मे॒वं ॅये य ए॒वं ॅवि॒द्वाꣳसो॑ वि॒द्वाꣳस॑ ए॒वं ॅये य ए॒वं ॅवि॒द्वाꣳसः॑ । \newline
43. ए॒वं ॅवि॒द्वाꣳसो॑ वि॒द्वाꣳस॑ ए॒व मे॒वं ॅवि॒द्वाꣳस॑ श्चतुर्विꣳशतिरा॒त्रम् च॑तुर्विꣳशतिरा॒त्रं ॅवि॒द्वाꣳस॑ ए॒व मे॒वं ॅवि॒द्वाꣳस॑ श्चतुर्विꣳशतिरा॒त्रम् । \newline
44. वि॒द्वाꣳस॑ श्चतुर्विꣳशतिरा॒त्रम् च॑तुर्विꣳशतिरा॒त्रं ॅवि॒द्वाꣳसो॑ वि॒द्वाꣳस॑ श्चतुर्विꣳशतिरा॒त्र मास॑त॒ आस॑ते चतुर्विꣳशतिरा॒त्रं ॅवि॒द्वाꣳसो॑ वि॒द्वाꣳस॑ श्चतुर्विꣳशतिरा॒त्र मास॑ते । \newline
45. च॒तु॒र्विꣳ॒॒श॒ति॒रा॒त्र मास॑त॒ आस॑ते चतुर्विꣳशतिरा॒त्रम् च॑तुर्विꣳशतिरा॒त्र मास॒ते ऽव॑र्ति॒ मव॑र्ति॒ मास॑ते चतुर्विꣳशतिरा॒त्रम् च॑तुर्विꣳशतिरा॒त्र मास॒ते ऽव॑र्तिम् । \newline
46. च॒तु॒र्विꣳ॒॒श॒ति॒रा॒त्रमिति॑ चतुर्विꣳशति - रा॒त्रम् । \newline
47. आस॒ते ऽव॑र्ति॒ मव॑र्ति॒ मास॑त॒ आस॒ते ऽव॑र्ति मे॒वै वाव॑र्ति॒ मास॑त॒ आस॒ते ऽव॑र्ति मे॒व । \newline
48. अव॑र्ति मे॒वै वाव॑र्ति॒ मव॑र्ति मे॒व पा॒प्मान॑म् पा॒प्मान॑ मे॒वा व॑र्ति॒ मव॑र्ति मे॒व पा॒प्मान᳚म् । \newline
49. ए॒व पा॒प्मान॑म् पा॒प्मान॑ मे॒वैव पा॒प्मान॑ मप॒हत्या॑ प॒हत्य॑ पा॒प्मान॑ मे॒वैव पा॒प्मान॑ मप॒हत्य॑ । \newline
50. पा॒प्मान॑ मप॒हत्या॑ प॒हत्य॑ पा॒प्मान॑म् पा॒प्मान॑ मप॒हत्य॒ श्रियꣳ॒॒ श्रिय॑ मप॒हत्य॑ पा॒प्मान॑म् पा॒प्मान॑ मप॒हत्य॒ श्रिय᳚म् । \newline
51. अ॒प॒हत्य॒ श्रियꣳ॒॒ श्रिय॑ मप॒हत्या॑ प॒हत्य॒ श्रिय॑म् गच्छन्ति गच्छन्ति॒ श्रिय॑ मप॒हत्या॑ प॒हत्य॒ श्रिय॑म् गच्छन्ति । \newline
52. अ॒प॒हत्येत्य॑प - हत्य॑ । \newline
53. श्रिय॑म् गच्छन्ति गच्छन्ति॒ श्रियꣳ॒॒ श्रिय॑म् गच्छन्ति॒ श्रीः श्रीर् ग॑च्छन्ति॒ श्रियꣳ॒॒ श्रिय॑म् गच्छन्ति॒ श्रीः । \newline
54. ग॒च्छ॒न्ति॒ श्रीः श्रीर् ग॑च्छन्ति गच्छन्ति॒ श्रीर्. हि हि श्रीर् ग॑च्छन्ति गच्छन्ति॒ श्रीर्. हि । \newline
55. श्रीर्. हि हि श्रीः श्रीर्. हि म॑नु॒ष्य॑स्य मनु॒ष्य॑स्य॒ हि श्रीः श्रीर्. हि म॑नु॒ष्य॑स्य । \newline
56. हि म॑नु॒ष्य॑स्य मनु॒ष्य॑स्य॒ हि हि म॑नु॒ष्य॑स्य॒ दैवी॒ दैवी॑ मनु॒ष्य॑स्य॒ हि हि म॑नु॒ष्य॑स्य॒ दैवी᳚ । \newline
57. म॒नु॒ष्य॑स्य॒ दैवी॒ दैवी॑ मनु॒ष्य॑स्य मनु॒ष्य॑स्य॒ दैवी॑ सꣳ॒॒सथ् सꣳ॒॒सद् दैवी॑ मनु॒ष्य॑स्य मनु॒ष्य॑स्य॒ दैवी॑ सꣳ॒॒सत् । \newline
\pagebreak
\markright{ TS 7.4.2.2  \hfill https://www.vedavms.in \hfill}

\section{ TS 7.4.2.2 }

\textbf{TS 7.4.2.2 } \newline
\textbf{Samhita Paata} \newline

दैवी॑ सꣳ॒॒सज्ज्योति॑रतिरा॒त्रो भ॑वति सुव॒र्गस्य॑ लो॒कस्यानु॑ख्यात्यै॒ पृष्ठ्यः॑ षड॒हो भ॑वति॒ षड् वा ऋ॒तवः॑ संॅवथ्स॒रस्तं मासा॑ अर्द्धमा॒सा ऋ॒तवः॑ प्र॒विश्य॒ दैवीꣳ॑ सꣳ॒॒सद॑मगच्छ॒न॒. य ए॒वं ॅवि॒द्वाꣳस॑-श्चतुर्विꣳशतिरा॒त्रमास॑ते संॅवथ्स॒रमे॒व प्र॒विश्य॒ वस्य॑सीꣳ सꣳ॒॒सदं॑ गच्छन्ति॒ त्रय॑स्त्रयस्त्रिꣳ॒॒शा अ॒वस्ता᳚द्-भवन्ति॒ त्रय॑स्त्रयस्त्रिꣳ॒॒शाः प॒रस्ता᳚त् त्रयस्त्रिꣳ॒॒शैरे॒वोभ॒यतो ऽव॑र्तिं पा॒प्मान॑मप॒हत्य॒ दैवीꣳ॑ सꣳ॒॒सदं॑ मद्ध्य॒तो - [  ] \newline

\textbf{Pada Paata} \newline

दैवी᳚ । सꣳ॒॒सदिति॑ सं - सत् । ज्योतिः॑ । अ॒ति॒रा॒त्र इत्य॑ति - रा॒त्रः । भ॒व॒ति॒ । सु॒व॒र्गस्येति॑ सुवः - गस्य॑ । लो॒कस्य॑ । अनु॑ख्यात्या॒ इत्यनु॑ -ख्या॒त्यै॒ । पृष्ठ्यः॑ । ष॒ड॒ह इति॑ षट् - अ॒हः । भ॒व॒ति॒ । षट् । वै । ऋ॒तवः॑ । सं॒ॅव॒थ्स॒र इति॑ सं - व॒थ्स॒रः । तम् । मासाः᳚ । अ॒द्‌र्ध॒मा॒सा इत्य॑द्‌र्ध - मा॒साः । ऋ॒तवः॑ । प्र॒विश्येति॑ प्र - विश्य॑ । दैवी᳚म् । सꣳ॒॒सद॒मिति॑ सं - सद᳚म् । अ॒ग॒च्छ॒न्न् । ये । ए॒वम् । वि॒द्वाꣳसः॑ । च॒तु॒र्विꣳ॒॒श॒ति॒रा॒त्रमिति॑ चतुर्विꣳशति - रा॒त्रम् । आस॑ते । सं॒ॅव॒थ्स॒रमिति॑ सं - व॒थ्स॒रम् । ए॒व । प्र॒विश्येति॑ प्र - विश्य॑ । वस्य॑सीम् । सꣳ॒॒सद॒मिति॑ सं - सद᳚म् । ग॒च्छ॒न्ति॒ । त्रयः॑ । त्र॒य॒स्त्रिꣳ॒॒शा इति॑ त्रयः - त्रिꣳ॒॒शाः । अ॒वस्ता᳚त् । भ॒व॒न्ति॒ । त्रयः॑ । त्र॒य॒स्त्रिꣳ॒॒शा इति॑ त्रय - त्रिꣳ॒॒शाः । प॒रस्ता᳚त् । त्र॒य॒स्त्रिꣳ॒॒शैरिति॑ त्रयः - त्रिꣳ॒॒शैः । ए॒व । उ॒भ॒यतः॑ । अव॑र्तिम् । पा॒प्मान᳚म् । अ॒प॒हत्येत्य॑प - हत्य॑ । दैवी᳚म् । सꣳ॒॒सद॒मिति॑ सं - सद᳚म् । म॒द्ध्य॒तः ।  \newline


\textbf{Krama Paata} \newline

दैवी॑ सꣳ॒॒सत् । सꣳ॒॒सज् ज्योतिः॑ । सꣳ॒॒सदिति॑ सम् - सत् । ज्योति॑रतिरा॒त्रः । अ॒ति॒रा॒त्रो भ॑वति । अ॒ति॒रा॒त्र इत्य॑ति - रा॒त्रः । भ॒व॒ति॒ सु॒व॒र्गस्य॑ । सु॒व॒र्गस्य॑ लो॒कस्य॑ । सु॒व॒र्गस्येति॑ सुवः - गस्य॑ । लो॒कस्यानु॑ख्यात्यै । अनु॑ख्यात्यै॒ पृष्ठ्‍यः॑ । अनु॑ख्यात्या॒ इत्यनु॑ - ख्या॒त्यै॒ । पृष्ठ्‍यः॑ षड॒हः । ष॒ड॒हो भ॑वति । ष॒ड॒ह इति॑ षट् - अ॒हः । भ॒व॒ति॒ षट् । षड् वै । वा ऋ॒तवः॑ । ऋ॒तवः॑ सम्ॅवथ्स॒रः । स॒म्ॅव॒थ्स॒रस्तम् । स॒म्ॅव॒थ्स॒र इति॑ सम् - व॒थ्स॒रः । तम् मासाः᳚ । मासा॑ अर्द्धमा॒साः । अ॒र्द्ध॒मा॒सा ऋ॒तवः॑ । अ॒र्द्ध॒मा॒सा इत्य॑र्द्ध - मा॒साः । ऋ॒तवः॑ प्र॒विश्य॑ । प्र॒विश्य॒ दैवी᳚म् । प्र॒विश्येति॑ प्र - विश्य॑ । दैवीꣳ॑ सꣳ॒॒सद᳚म् । सꣳ॒॒सद॑मगच्छन्न् । सꣳ॒॒सद॒मिति॑ सम् - सद᳚म् । अ॒ग॒च्छ॒न्॒. ये । य ए॒वम् । ए॒वम् ॅवि॒द्वाꣳसः॑ । वि॒द्वाꣳस॑श्चतुर्विꣳशतिरा॒त्रम् । च॒तु॒र्विꣳ॒॒श॒ति॒रा॒त्रमास॑ते । च॒तु॒र्विꣳ॒॒श॒ति॒रा॒त्रमिति॑ चतुर्विꣳशति - रा॒त्रम् । आस॑ते सम्ॅवथ्स॒रम् । स॒म्ॅव॒थ्स॒रमे॒व । स॒म्ॅव॒थ्स॒रमिति॑ सम् - व॒थ्स॒रम् । ए॒व प्र॒विश्य॑ । प्र॒विश्य॒ वस्य॑सीम् । प्र॒विश्येति॑ प्र - विश्य॑ । वस्य॑सीꣳ सꣳ॒॒सद᳚म् । सꣳ॒॒सद॑म् गच्छन्ति । सꣳ॒॒सद॒मिति॑ सम् - सद᳚म् । ग॒च्छ॒न्ति॒ त्रयः॑ । त्रय॑स्त्रयस्त्रिꣳ॒॒शाः । त्र॒य॒स्त्रिꣳ॒॒शा अ॒वस्ता᳚त् । त्र॒य॒स्त्रिꣳ॒॒शा इति॑ त्रयः - त्रिꣳ॒॒शाः । अ॒वस्ता᳚द् भवन्ति । भ॒व॒न्ति॒ त्रयः॑ । त्रय॑स्त्रयस्त्रिꣳ॒॒शाः । त्र॒य॒स्त्रिꣳ॒॒शाः प॒रस्ता᳚त् । त्र॒य॒स्त्रिꣳ॒॒शा इति॑ त्रयः - त्रिꣳ॒॒शाः । प॒रस्ता᳚त् त्रयस्त्रिꣳ॒॒शैः । त्र॒य॒स्त्रिꣳ॒॒शैरे॒व । त्र॒य॒स्त्रिꣳ॒॒शैरिति॑ त्रयः - त्रिꣳ॒॒शैः । ए॒वोभ॒यतः॑ । उ॒भ॒यतोऽव॑र्तिम् । अव॑र्तिम् पा॒प्मान᳚म् । पा॒प्मान॑मप॒हत्य॑ । अ॒प॒हत्य॒ दैवी᳚म् । अ॒प॒हत्येत्य॑प - हत्य॑ । दैवीꣳ॑ सꣳ॒॒सद᳚म् । सꣳ॒॒सद॑म् मद्ध्य॒तः । सꣳ॒॒सद॒मिति॑ सम् - सद᳚म् । म॒द्ध्य॒तो ग॑च्छन्ति \newline

\textbf{Jatai Paata} \newline

1. दैवी॑ सꣳ॒॒सथ् सꣳ॒॒सद् दैवी॒ दैवी॑ सꣳ॒॒सत् । \newline
2. सꣳ॒॒सज् ज्योति॒र् ज्योतिः॑ सꣳ॒॒सथ् सꣳ॒॒सज् ज्योतिः॑ । \newline
3. सꣳ॒॒सदिति॑ सं - सत् । \newline
4. ज्योति॑ रतिरा॒त्रो॑ ऽतिरा॒त्रो ज्योति॒र् ज्योति॑ रतिरा॒त्रः । \newline
5. अ॒ति॒रा॒त्रो भ॑वति भव त्यतिरा॒त्रो॑ ऽतिरा॒त्रो भ॑वति । \newline
6. अ॒ति॒रा॒त्र इत्य॑ति - रा॒त्रः । \newline
7. भ॒व॒ति॒ सु॒व॒र्गस्य॑ सुव॒र्गस्य॑ भवति भवति सुव॒र्गस्य॑ । \newline
8. सु॒व॒र्गस्य॑ लो॒कस्य॑ लो॒कस्य॑ सुव॒र्गस्य॑ सुव॒र्गस्य॑ लो॒कस्य॑ । \newline
9. सु॒व॒र्गस्येति॑ सुवः - गस्य॑ । \newline
10. लो॒कस्या नु॑ख्यात्या॒ अनु॑ख्यात्यै लो॒कस्य॑ लो॒कस्या नु॑ख्यात्यै । \newline
11. अनु॑ख्यात्यै॒ पृष्ठ्यः॒ पृष्ठ्यो ऽनु॑ख्यात्या॒ अनु॑ख्यात्यै॒ पृष्ठ्यः॑ । \newline
12. अनु॑ख्यात्या॒ इत्यनु॑ - ख्या॒त्यै॒ । \newline
13. पृष्ठ्य॑ ष्षड॒ह ष्ष॑ड॒हः पृष्ठ्यः॒ पृष्ठ्य॑ ष्षड॒हः । \newline
14. ष॒ड॒हो भ॑वति भवति षड॒ह ष्ष॑ड॒हो भ॑वति । \newline
15. ष॒ड॒ह इति॑ षट् - अ॒हः । \newline
16. भ॒व॒ति॒ षट् थ्षड् भ॑वति भवति॒ षट् । \newline
17. षड् वै वै षट् थ्षड् वै । \newline
18. वा ऋ॒तव॑ ऋ॒तवो॒ वै वा ऋ॒तवः॑ । \newline
19. ऋ॒तवः॑ संॅवथ्स॒रः सं॑ॅवथ्स॒र ऋ॒तव॑ ऋ॒तवः॑ संॅवथ्स॒रः । \newline
20. सं॒ॅव॒थ्स॒र स्तम् तꣳ सं॑ॅवथ्स॒रः सं॑ॅवथ्स॒र स्तम् । \newline
21. सं॒ॅव॒थ्स॒र इति॑ सं - व॒थ्स॒रः । \newline
22. तम् मासा॒ मासा॒ स्तम् तम् मासाः᳚ । \newline
23. मासा॑ अर्द्धमा॒सा अ॑र्द्धमा॒सा मासा॒ मासा॑ अर्द्धमा॒साः । \newline
24. अ॒र्द्ध॒मा॒सा ऋ॒तव॑ ऋ॒तवो᳚ ऽर्द्धमा॒सा अ॑र्द्धमा॒सा ऋ॒तवः॑ । \newline
25. अ॒र्द्ध॒मा॒सा इत्य॑र्द्ध - मा॒साः । \newline
26. ऋ॒तवः॑ प्र॒विश्य॑ प्र॒विश्य॒ र्‌तव॑ ऋ॒तवः॑ प्र॒विश्य॑ । \newline
27. प्र॒विश्य॒ दैवी॒म् दैवी᳚म् प्र॒विश्य॑ प्र॒विश्य॒ दैवी᳚म् । \newline
28. प्र॒विश्येति॑ प्र - विश्य॑ । \newline
29. दैवीꣳ॑ सꣳ॒॒सदꣳ॑ सꣳ॒॒सद॒म् दैवी॒म् दैवीꣳ॑ सꣳ॒॒सद᳚म् । \newline
30. सꣳ॒॒सद॑ मगच्छन् नगच्छन् थ्सꣳ॒॒सदꣳ॑ सꣳ॒॒सद॑ मगच्छन्न् । \newline
31. सꣳ॒॒सद॒मिति॑ सं - सद᳚म् । \newline
32. अ॒ग॒च्छ॒न्॒. ये ये॑ ऽगच्छन् नगच्छ॒न्॒. ये । \newline
33. य ए॒व मे॒वं ॅये य ए॒वम् । \newline
34. ए॒वं ॅवि॒द्वाꣳसो॑ वि॒द्वाꣳस॑ ए॒व मे॒वं ॅवि॒द्वाꣳसः॑ । \newline
35. वि॒द्वाꣳस॑ श्चतुर्विꣳशतिरा॒त्रम् च॑तुर्विꣳशतिरा॒त्रं ॅवि॒द्वाꣳसो॑ वि॒द्वाꣳस॑ श्चतुर्विꣳशतिरा॒त्रम् । \newline
36. च॒तु॒र्विꣳ॒॒श॒ति॒रा॒त्र मास॑त॒ आस॑ते चतुर्विꣳशतिरा॒त्रम् च॑तुर्विꣳशतिरा॒त्र मास॑ते । \newline
37. च॒तु॒र्विꣳ॒॒श॒ति॒रा॒त्रमिति॑ चतुर्विꣳशति - रा॒त्रम् । \newline
38. आस॑ते संॅवथ्स॒रꣳ सं॑ॅवथ्स॒र मास॑त॒ आस॑ते संॅवथ्स॒रम् । \newline
39. सं॒ॅव॒थ्स॒र मे॒वैव सं॑ॅवथ्स॒रꣳ सं॑ॅवथ्स॒र मे॒व । \newline
40. सं॒ॅव॒थ्स॒रमिति॑ सं - व॒थ्स॒रम् । \newline
41. ए॒व प्र॒विश्य॑ प्र॒विश्यै॒वैव प्र॒विश्य॑ । \newline
42. प्र॒विश्य॒ वस्य॑सीं॒ ॅवस्य॑सीम् प्र॒विश्य॑ प्र॒विश्य॒ वस्य॑सीम् । \newline
43. प्र॒विश्येति॑ प्र - विश्य॑ । \newline
44. वस्य॑सीꣳ सꣳ॒॒सदꣳ॑ सꣳ॒॒सदं॒ ॅवस्य॑सीं॒ ॅवस्य॑सीꣳ सꣳ॒॒सद᳚म् । \newline
45. सꣳ॒॒सद॑म् गच्छन्ति गच्छन्ति सꣳ॒॒सदꣳ॑ सꣳ॒॒सद॑म् गच्छन्ति । \newline
46. सꣳ॒॒सद॒मिति॑ सं - सद᳚म् । \newline
47. ग॒च्छ॒न्ति॒ त्रय॒ स्त्रयो॑ गच्छन्ति गच्छन्ति॒ त्रयः॑ । \newline
48. त्रय॑ स्त्रयस्त्रिꣳ॒॒शा स्त्र॑यस्त्रिꣳ॒॒शा स्त्रय॒ स्त्रय॑ स्त्रयस्त्रिꣳ॒॒शाः । \newline
49. त्र॒य॒स्त्रिꣳ॒॒शा अ॒वस्ता॑ द॒वस्ता᳚त् त्रयस्त्रिꣳ॒॒शा स्त्र॑यस्त्रिꣳ॒॒शा अ॒वस्ता᳚त् । \newline
50. त्र॒य॒स्त्रिꣳ॒॒शा इति॑ त्रयः - त्रिꣳ॒॒शाः । \newline
51. अ॒वस्ता᳚द् भवन्ति भवन्त्य॒वस्ता॑ द॒वस्ता᳚द् भवन्ति । \newline
52. भ॒व॒न्ति॒ त्रय॒ स्त्रयो॑ भवन्ति भवन्ति॒ त्रयः॑ । \newline
53. त्रय॑ स्त्रयस्त्रिꣳ॒॒शा स्त्र॑यस्त्रिꣳ॒॒शा स्त्रय॒ स्त्रय॑ स्त्रयस्त्रिꣳ॒॒शाः । \newline
54. त्र॒य॒स्त्रिꣳ॒॒शाः प॒रस्ता᳚त् प॒रस्ता᳚त् त्रयस्त्रिꣳ॒॒शा स्त्र॑यस्त्रिꣳ॒॒शाः प॒रस्ता᳚त् । \newline
55. त्र॒य॒स्त्रिꣳ॒॒शा इति॑ त्रयः - त्रिꣳ॒॒शाः । \newline
56. प॒रस्ता᳚त् त्रयस्त्रिꣳ॒॒शै स्त्र॑यस्त्रिꣳ॒॒शैः प॒रस्ता᳚त् प॒रस्ता᳚त् त्रयस्त्रिꣳ॒॒शैः । \newline
57. त्र॒य॒स्त्रिꣳ॒॒शै रे॒वैव त्र॑यस्त्रिꣳ॒॒शै स्त्र॑य स्त्रिꣳ॒॒शैरे॒व । \newline
58. त्र॒य॒स्त्रिꣳ॒॒शैरिति॑ त्रयः - त्रिꣳ॒॒शैः । \newline
59. ए॒वोभ॒यत॑ उभ॒यत॑ ए॒वैवोभ॒यतः॑ । \newline
60. उ॒भ॒यतो ऽव॑र्ति॒ मव॑र्ति मुभ॒यत॑ उभ॒यतो ऽव॑र्तिम् । \newline
61. अव॑र्तिम् पा॒प्मान॑म् पा॒प्मान॒ मव॑र्ति॒ मव॑र्तिम् पा॒प्मान᳚म् । \newline
62. पा॒प्मान॑ मप॒हत्या॑ प॒हत्य॑ पा॒प्मान॑म् पा॒प्मान॑ मप॒हत्य॑ । \newline
63. अ॒प॒हत्य॒ दैवी॒म् दैवी॑ मप॒हत्या॑ प॒हत्य॒ दैवी᳚म् । \newline
64. अ॒प॒हत्येत्य॑प - हत्य॑ । \newline
65. दैवीꣳ॑ सꣳ॒॒सदꣳ॑ सꣳ॒॒सद॒म् दैवी॒म् दैवीꣳ॑ सꣳ॒॒सद᳚म् । \newline
66. सꣳ॒॒सद॑म् मद्ध्य॒तो म॑द्ध्य॒तः सꣳ॒॒सदꣳ॑ सꣳ॒॒सद॑म् मद्ध्य॒तः । \newline
67. सꣳ॒॒सद॒मिति॑ सं - सद᳚म् । \newline
68. म॒द्ध्य॒तो ग॑च्छन्ति गच्छन्ति मद्ध्य॒तो म॑द्ध्य॒तो ग॑च्छन्ति । \newline

\textbf{Ghana Paata } \newline

1. दैवी॑ सꣳ॒॒सथ् सꣳ॒॒सद् दैवी॒ दैवी॑ सꣳ॒॒सज् ज्योति॒र् ज्योतिः॑ सꣳ॒॒सद् दैवी॒ दैवी॑ सꣳ॒॒सज् ज्योतिः॑ । \newline
2. सꣳ॒॒सज् ज्योति॒र् ज्योतिः॑ सꣳ॒॒सथ् सꣳ॒॒सज् ज्योति॑ रतिरा॒त्रो॑ ऽतिरा॒त्रो ज्योतिः॑ सꣳ॒॒सथ् सꣳ॒॒सज् ज्योति॑ रतिरा॒त्रः । \newline
3. सꣳ॒॒सदिति॑ सं - सत् । \newline
4. ज्योति॑ रतिरा॒त्रो॑ ऽतिरा॒त्रो ज्योति॒र् ज्योति॑ रतिरा॒त्रो भ॑वति भव त्यतिरा॒त्रो ज्योति॒र् ज्योति॑ रतिरा॒त्रो भ॑वति । \newline
5. अ॒ति॒रा॒त्रो भ॑वति भव त्यतिरा॒त्रो॑ ऽतिरा॒त्रो भ॑वति सुव॒र्गस्य॑ सुव॒र्गस्य॑ भव त्यतिरा॒त्रो॑ ऽतिरा॒त्रो भ॑वति सुव॒र्गस्य॑ । \newline
6. अ॒ति॒रा॒त्र इत्य॑ति - रा॒त्रः । \newline
7. भ॒व॒ति॒ सु॒व॒र्गस्य॑ सुव॒र्गस्य॑ भवति भवति सुव॒र्गस्य॑ लो॒कस्य॑ लो॒कस्य॑ सुव॒र्गस्य॑ भवति भवति सुव॒र्गस्य॑ लो॒कस्य॑ । \newline
8. सु॒व॒र्गस्य॑ लो॒कस्य॑ लो॒कस्य॑ सुव॒र्गस्य॑ सुव॒र्गस्य॑ लो॒कस्या नु॑ख्यात्या॒ अनु॑ख्यात्यै लो॒कस्य॑ सुव॒र्गस्य॑ सुव॒र्गस्य॑ लो॒कस्या नु॑ख्यात्यै । \newline
9. सु॒व॒र्गस्येति॑ सुवः - गस्य॑ । \newline
10. लो॒कस्या नु॑ख्यात्या॒ अनु॑ख्यात्यै लो॒कस्य॑ लो॒कस्या नु॑ख्यात्यै॒ पृष्ठ्यः॒ पृष्ठ्यो ऽनु॑ख्यात्यै लो॒कस्य॑ लो॒कस्या नु॑ख्यात्यै॒ पृष्ठ्यः॑ । \newline
11. अनु॑ख्यात्यै॒ पृष्ठ्यः॒ पृष्ठ्यो ऽनु॑ख्यात्या॒ अनु॑ख्यात्यै॒ पृष्ठ्य॑ ष्षड॒ह ष्ष॑ड॒हः पृष्ठ्यो ऽनु॑ख्यात्या॒ अनु॑ख्यात्यै॒ पृष्ठ्य॑ ष्षड॒हः । \newline
12. अनु॑ख्यात्या॒ इत्यनु॑ - ख्या॒त्यै॒ । \newline
13. पृष्ठ्य॑ ष्षड॒ह ष्ष॑ड॒हः पृष्ठ्यः॒ पृष्ठ्य॑ ष्षड॒हो भ॑वति भवति षड॒हः पृष्ठ्यः॒ पृष्ठ्य॑ ष्षड॒हो भ॑वति । \newline
14. ष॒ड॒हो भ॑वति भवति षड॒ह ष्ष॑ड॒हो भ॑वति॒ षट् थ्षड् भ॑वति षड॒ह ष्ष॑ड॒हो भ॑वति॒ षट् । \newline
15. ष॒ड॒ह इति॑ षट् - अ॒हः । \newline
16. भ॒व॒ति॒ षट् थ्षड् भ॑वति भवति॒ षड् वै वै षड् भ॑वति भवति॒ षड् वै । \newline
17. षड् वै वै षट् थ्षड् वा ऋ॒तव॑ ऋ॒तवो॒ वै षट् थ्षड् वा ऋ॒तवः॑ । \newline
18. वा ऋ॒तव॑ ऋ॒तवो॒ वै वा ऋ॒तवः॑ संॅवथ्स॒रः सं॑ॅवथ्स॒र ऋ॒तवो॒ वै वा ऋ॒तवः॑ संॅवथ्स॒रः । \newline
19. ऋ॒तवः॑ संॅवथ्स॒रः सं॑ॅवथ्स॒र ऋ॒तव॑ ऋ॒तवः॑ संॅवथ्स॒र स्तम् तꣳ सं॑ॅवथ्स॒र ऋ॒तव॑ ऋ॒तवः॑ संॅवथ्स॒र स्तम् । \newline
20. सं॒ॅव॒थ्स॒र स्तम् तꣳ सं॑ॅवथ्स॒रः सं॑ॅवथ्स॒र स्तम् मासा॒ मासा॒ स्तꣳ सं॑ॅवथ्स॒रः सं॑ॅवथ्स॒र स्तम् मासाः᳚ । \newline
21. सं॒ॅव॒थ्स॒र इति॑ सं - व॒थ्स॒रः । \newline
22. तम् मासा॒ मासा॒ स्तम् तम् मासा॑ अर्द्धमा॒सा अ॑र्द्धमा॒सा मासा॒ स्तम् तम् मासा॑ अर्द्धमा॒साः । \newline
23. मासा॑ अर्द्धमा॒सा अ॑र्द्धमा॒सा मासा॒ मासा॑ अर्द्धमा॒सा ऋ॒तव॑ ऋ॒तवो᳚ ऽर्द्धमा॒सा मासा॒ मासा॑ अर्द्धमा॒सा ऋ॒तवः॑ । \newline
24. अ॒र्द्ध॒मा॒सा ऋ॒तव॑ ऋ॒तवो᳚ ऽर्द्धमा॒सा अ॑र्द्धमा॒सा ऋ॒तवः॑ प्र॒विश्य॑ प्र॒विश्य॒ र्‌तवो᳚ ऽर्द्धमा॒सा अ॑र्द्धमा॒सा ऋ॒तवः॑ प्र॒विश्य॑ । \newline
25. अ॒र्द्ध॒मा॒सा इत्य॑र्द्ध - मा॒साः । \newline
26. ऋ॒तवः॑ प्र॒विश्य॑ प्र॒विश्य॒ र्‌तव॑ ऋ॒तवः॑ प्र॒विश्य॒ दैवी॒म् दैवी᳚म् प्र॒विश्य॒ र्‌तव॑ ऋ॒तवः॑ प्र॒विश्य॒ दैवी᳚म् । \newline
27. प्र॒विश्य॒ दैवी॒म् दैवी᳚म् प्र॒विश्य॑ प्र॒विश्य॒ दैवीꣳ॑ सꣳ॒॒सदꣳ॑ सꣳ॒॒सद॒म् दैवी᳚म् प्र॒विश्य॑ प्र॒विश्य॒ दैवीꣳ॑ सꣳ॒॒सद᳚म् । \newline
28. प्र॒विश्येति॑ प्र - विश्य॑ । \newline
29. दैवीꣳ॑ सꣳ॒॒सदꣳ॑ सꣳ॒॒सद॒म् दैवी॒म् दैवीꣳ॑ सꣳ॒॒सद॑ मगच्छन् नगच्छन् थ्सꣳ॒॒सद॒म् दैवी॒म् दैवीꣳ॑ सꣳ॒॒सद॑ मगच्छन्न् । \newline
30. सꣳ॒॒सद॑ मगच्छन् नगच्छन् थ्सꣳ॒॒सदꣳ॑ सꣳ॒॒सद॑ मगच्छ॒न्॒. ये ये॑ ऽगच्छन् थ्सꣳ॒॒सदꣳ॑ सꣳ॒॒सद॑ मगच्छ॒न्॒. ये । \newline
31. सꣳ॒॒सद॒मिति॑ सं - सद᳚म् । \newline
32. अ॒ग॒च्छ॒न्॒. ये ये॑ ऽगच्छन् नगच्छ॒न्॒. य ए॒व मे॒वं ॅये॑ ऽगच्छन् नगच्छ॒न्॒. य ए॒वम् । \newline
33. य ए॒व मे॒वं ॅये य ए॒वं ॅवि॒द्वाꣳसो॑ वि॒द्वाꣳस॑ ए॒वं ॅये य ए॒वं ॅवि॒द्वाꣳसः॑ । \newline
34. ए॒वं ॅवि॒द्वाꣳसो॑ वि॒द्वाꣳस॑ ए॒व मे॒वं ॅवि॒द्वाꣳस॑ श्चतुर्विꣳशतिरा॒त्रम् च॑तुर्विꣳशतिरा॒त्रं ॅवि॒द्वाꣳस॑ ए॒व मे॒वं ॅवि॒द्वाꣳस॑ श्चतुर्विꣳशतिरा॒त्रम् । \newline
35. वि॒द्वाꣳस॑ श्चतुर्विꣳशतिरा॒त्रम् च॑तुर्विꣳशतिरा॒त्रं ॅवि॒द्वाꣳसो॑ वि॒द्वाꣳस॑ श्चतुर्विꣳशतिरा॒त्र मास॑त॒ आस॑ते चतुर्विꣳशतिरा॒त्रं ॅवि॒द्वाꣳसो॑ वि॒द्वाꣳस॑ श्चतुर्विꣳशतिरा॒त्र मास॑ते । \newline
36. च॒तु॒र्विꣳ॒॒श॒ति॒रा॒त्र मास॑त॒ आस॑ते चतुर्विꣳशतिरा॒त्रम् च॑तुर्विꣳशतिरा॒त्र मास॑ते संॅवथ्स॒रꣳ सं॑ॅवथ्स॒र मास॑ते चतुर्विꣳशतिरा॒त्रम् च॑तुर्विꣳशतिरा॒त्र मास॑ते संॅवथ्स॒रम् । \newline
37. च॒तु॒र्विꣳ॒॒श॒ति॒रा॒त्रमिति॑ चतुर्विꣳशति - रा॒त्रम् । \newline
38. आस॑ते संॅवथ्स॒रꣳ सं॑ॅवथ्स॒र मास॑त॒ आस॑ते संॅवथ्स॒र मे॒वैव सं॑ॅवथ्स॒र मास॑त॒ आस॑ते संॅवथ्स॒र मे॒व । \newline
39. सं॒ॅव॒थ्स॒र मे॒वैव सं॑ॅवथ्स॒रꣳ सं॑ॅवथ्स॒र मे॒व प्र॒विश्य॑ प्र॒विश्यै॒व सं॑ॅवथ्स॒रꣳ सं॑ॅवथ्स॒र मे॒व प्र॒विश्य॑ । \newline
40. सं॒ॅव॒थ्स॒रमिति॑ सं - व॒थ्स॒रम् । \newline
41. ए॒व प्र॒विश्य॑ प्र॒विश्यै॒वैव प्र॒विश्य॒ वस्य॑सीं॒ ॅवस्य॑सीम् प्र॒विश्यै॒वैव प्र॒विश्य॒ वस्य॑सीम् । \newline
42. प्र॒विश्य॒ वस्य॑सीं॒ ॅवस्य॑सीम् प्र॒विश्य॑ प्र॒विश्य॒ वस्य॑सीꣳ सꣳ॒॒सदꣳ॑ सꣳ॒॒सदं॒ ॅवस्य॑सीम् प्र॒विश्य॑ प्र॒विश्य॒ वस्य॑सीꣳ सꣳ॒॒सद᳚म् । \newline
43. प्र॒विश्येति॑ प्र - विश्य॑ । \newline
44. वस्य॑सीꣳ सꣳ॒॒सदꣳ॑ सꣳ॒॒सदं॒ ॅवस्य॑सीं॒ ॅवस्य॑सीꣳ सꣳ॒॒सद॑म् गच्छन्ति गच्छन्ति सꣳ॒॒सदं॒ ॅवस्य॑सीं॒ ॅवस्य॑सीꣳ सꣳ॒॒सद॑म् गच्छन्ति । \newline
45. सꣳ॒॒सद॑म् गच्छन्ति गच्छन्ति सꣳ॒॒सदꣳ॑ सꣳ॒॒सद॑म् गच्छन्ति॒ त्रय॒ स्त्रयो॑ गच्छन्ति सꣳ॒॒सदꣳ॑ सꣳ॒॒सद॑म् गच्छन्ति॒ त्रयः॑ । \newline
46. सꣳ॒॒सद॒मिति॑ सं - सद᳚म् । \newline
47. ग॒च्छ॒न्ति॒ त्रय॒ स्त्रयो॑ गच्छन्ति गच्छन्ति॒ त्रय॑ स्त्रयस्त्रिꣳ॒॒शा स्त्र॑यस्त्रिꣳ॒॒शा स्त्रयो॑ गच्छन्ति गच्छन्ति॒ त्रय॑ स्त्रयस्त्रिꣳ॒॒शाः । \newline
48. त्रय॑ स्त्रयस्त्रिꣳ॒॒शा स्त्र॑यस्त्रिꣳ॒॒शा स्त्रय॒ स्त्रय॑ स्त्रयस्त्रिꣳ॒॒शा अ॒वस्ता॑ द॒वस्ता᳚त् त्रयस्त्रिꣳ॒॒शा स्त्रय॒स्त्रय॑ स्त्रयस्त्रिꣳ॒॒शा अ॒वस्ता᳚त् । \newline
49. त्र॒य॒स्त्रिꣳ॒॒शा अ॒वस्ता॑ द॒वस्ता᳚त् त्रयस्त्रिꣳ॒॒शा स्त्र॑यस्त्रिꣳ॒॒शा अ॒वस्ता᳚द् भवन्ति भव न्त्य॒वस्ता᳚त् त्रयस्त्रिꣳ॒॒शा स्त्र॑यस्त्रिꣳ॒॒शा अ॒वस्ता᳚द् भवन्ति । \newline
50. त्र॒य॒स्त्रिꣳ॒॒शा इति॑ त्रयः - त्रिꣳ॒॒शाः । \newline
51. अ॒वस्ता᳚द् भवन्ति भव न्त्य॒वस्ता॑ द॒वस्ता᳚द् भवन्ति॒ त्रय॒ स्त्रयो॑ भव न्त्य॒वस्ता॑ द॒वस्ता᳚द् भवन्ति॒ त्रयः॑ । \newline
52. भ॒व॒न्ति॒ त्रय॒ स्त्रयो॑ भवन्ति भवन्ति॒ त्रय॑ स्त्रयस्त्रिꣳ॒॒शा स्त्र॑यस्त्रिꣳ॒॒शा स्त्रयो॑ भवन्ति भवन्ति॒ त्रय॑ स्त्रयस्त्रिꣳ॒॒शाः । \newline
53. त्रय॑ स्त्रयस्त्रिꣳ॒॒शा स्त्र॑यस्त्रिꣳ॒॒शा स्त्रय॒ स्त्रय॑ स्त्रयस्त्रिꣳ॒॒शाः प॒रस्ता᳚त् प॒रस्ता᳚त् त्रयस्त्रिꣳ॒॒शा स्त्रय॒ स्त्रय॑ स्त्रयस्त्रिꣳ॒॒शाः प॒रस्ता᳚त् । \newline
54. त्र॒य॒स्त्रिꣳ॒॒शाः प॒रस्ता᳚त् प॒रस्ता᳚त् त्रयस्त्रिꣳ॒॒शा स्त्र॑यस्त्रिꣳ॒॒शाः प॒रस्ता᳚त् त्रयस्त्रिꣳ॒॒शै स्त्र॑यस्त्रिꣳ॒॒शैः प॒रस्ता᳚त् त्रयस्त्रिꣳ॒॒शा स्त्र॑यस्त्रिꣳ॒॒शाः प॒रस्ता᳚त् त्रयस्त्रिꣳ॒॒शैः । \newline
55. त्र॒य॒स्त्रिꣳ॒॒शा इति॑ त्रयः - त्रिꣳ॒॒शाः । \newline
56. प॒रस्ता᳚त् त्रयस्त्रिꣳ॒॒शै स्त्र॑यस्त्रिꣳ॒॒शैः प॒रस्ता᳚त् प॒रस्ता᳚त् त्रयस्त्रिꣳ॒॒शै रे॒वैव त्र॑यस्त्रिꣳ॒॒शैः प॒रस्ता᳚त् प॒रस्ता᳚त् त्रयस्त्रिꣳ॒॒शै रे॒व । \newline
57. त्र॒य॒स्त्रिꣳ॒॒शै रे॒वैव त्र॑यस्त्रिꣳ॒॒शै स्त्र॑यस्त्रिꣳ॒॒शै रे॒वोभ॒यत॑ उभ॒यत॑ ए॒व त्र॑यस्त्रिꣳ॒॒शै स्त्र॑यस्त्रिꣳ॒॒शै रे॒वोभ॒यतः॑ । \newline
58. त्र॒य॒स्त्रिꣳ॒॒शैरिति॑ त्रयः - त्रिꣳ॒॒शैः । \newline
59. ए॒वो भ॒यत॑ उभ॒यत॑ ए॒वैवोभ॒यतो ऽव॑र्ति॒ मव॑र्ति मुभ॒यत॑ ए॒वैवोभ॒यतो ऽव॑र्तिम् । \newline
60. उ॒भ॒यतो ऽव॑र्ति॒ मव॑र्ति मुभ॒यत॑ उभ॒यतो ऽव॑र्तिम् पा॒प्मान॑म् पा॒प्मान॒ मव॑र्ति मुभ॒यत॑ उभ॒यतो ऽव॑र्तिम् पा॒प्मान᳚म् । \newline
61. अव॑र्तिम् पा॒प्मान॑म् पा॒प्मान॒ मव॑र्ति॒ मव॑र्तिम् पा॒प्मान॑ मप॒हत्या॑ प॒हत्य॑ पा॒प्मान॒ मव॑र्ति॒ मव॑र्तिम् पा॒प्मान॑ मप॒हत्य॑ । \newline
62. पा॒प्मान॑ मप॒हत्या॑ प॒हत्य॑ पा॒प्मान॑म् पा॒प्मान॑ मप॒हत्य॒ दैवी॒म् दैवी॑ मप॒हत्य॑ पा॒प्मान॑म् पा॒प्मान॑ मप॒हत्य॒ दैवी᳚म् । \newline
63. अ॒प॒हत्य॒ दैवी॒म् दैवी॑ मप॒हत्या॑ प॒हत्य॒ दैवीꣳ॑ सꣳ॒॒सदꣳ॑ सꣳ॒॒सद॒म् दैवी॑ मप॒हत्या॑ प॒हत्य॒ दैवीꣳ॑ सꣳ॒॒सद᳚म् । \newline
64. अ॒प॒हत्येत्य॑प - हत्य॑ । \newline
65. दैवीꣳ॑ सꣳ॒॒सदꣳ॑ सꣳ॒॒सद॒म् दैवी॒म् दैवीꣳ॑ सꣳ॒॒सद॑म् मद्ध्य॒तो म॑द्ध्य॒तः सꣳ॒॒सद॒म् दैवी॒म् दैवीꣳ॑ सꣳ॒॒सद॑म् मद्ध्य॒तः । \newline
66. सꣳ॒॒सद॑म् मद्ध्य॒तो म॑द्ध्य॒तः सꣳ॒॒सदꣳ॑ सꣳ॒॒सद॑म् मद्ध्य॒तो ग॑च्छन्ति गच्छन्ति मद्ध्य॒तः सꣳ॒॒सदꣳ॑ सꣳ॒॒सद॑म् मद्ध्य॒तो ग॑च्छन्ति । \newline
67. सꣳ॒॒सद॒मिति॑ सं - सद᳚म् । \newline
68. म॒द्ध्य॒तो ग॑च्छन्ति गच्छन्ति मद्ध्य॒तो म॑द्ध्य॒तो ग॑च्छन्ति पृ॒ष्ठानि॑ पृ॒ष्ठानि॑ गच्छन्ति मद्ध्य॒तो म॑द्ध्य॒तो ग॑च्छन्ति पृ॒ष्ठानि॑ । \newline
\pagebreak
\markright{ TS 7.4.2.3  \hfill https://www.vedavms.in \hfill}

\section{ TS 7.4.2.3 }

\textbf{TS 7.4.2.3 } \newline
\textbf{Samhita Paata} \newline

ग॑च्छन्ति पृ॒ष्ठानि॒ हि दैवी॑ सꣳ॒॒सज्जा॒मि वा ए॒तत् कु॑र्वन्ति॒ यत् त्रय॑स्त्रयस्त्रिꣳ॒॒शा अ॒न्वञ्चो॒ मद्ध्येऽनि॑रुक्तो भवति॒ तेनाजा᳚म्यू॒र्द्ध्वानि॑ पृ॒ष्ठानि॑ भवन्त्यू॒र्द्ध्वाः छ॑न्दो॒मा उ॒भाभ्याꣳ॑ रू॒पाभ्याꣳ॑ सुव॒र्गं ॅलो॒कं ॅय॒न्त्यस॑त्रं॒ ॅवा ए॒तद्-यद॑छन्दो॒मं ॅयच्छ॑न्दो॒मा भव॑न्ति॒ तेन॑ स॒त्रं दे॒वता॑ ए॒व पृ॒ष्ठैरव॑ रुन्धते प॒शूञ्छ॑न्दो॒मैरोजो॒ वै वी॒र्यं॑ पृ॒ष्ठानि॑ प॒शवः॑- [  ] \newline

\textbf{Pada Paata} \newline

ग॒च्छ॒न्ति॒ । पृ॒ष्ठानि॑ । हि । दैवी᳚ । सꣳ॒॒सदिति॑ सं - सत् । जा॒मि । वै । ए॒तत् । कु॒र्व॒न्ति॒ । यत् । त्रयः॑ । त्र॒य॒स्त्रिꣳ॒॒शा इति॑ त्रयः-स्त्रिꣳ॒॒शाः । अ॒न्वञ्चः॑ । मद्ध्ये᳚ । अनि॑रुक्त॒ इत्यनिः॑ - उ॒क्तः॒ । भ॒व॒ति॒ । तेन॑ । अजा॑मि । ऊ॒द्‌र्ध्वानि॑ । पृ॒ष्ठानि॑ । भ॒व॒न्ति॒ । ऊ॒द्‌र्ध्वाः । छ॒न्दो॒मा इति॑ छन्दः - माः । उ॒भाभ्या᳚म् । रू॒पाभ्या᳚म् । सु॒व॒र्गमिति॑ सुवः - गम् । लो॒कम् । य॒न्ति॒ । अस॑त्रम् । वै । ए॒तत् । यत् । अ॒छ॒न्दो॒ममित्य॑छन्दः-मम् । यत् । छ॒न्दो॒मा इति॑ छन्दः- माः । भव॑न्ति । तेन॑ । स॒त्रम् । दे॒वताः᳚ । ए॒व । पृ॒ष्ठैः । अवेति॑ । रु॒न्ध॒ते॒ । प॒शून् । छ॒न्दो॒मैरिति॑ छन्दः - मैः । ओजः॑ । वै । वी॒र्य᳚म् । पृ॒ष्ठानि॑ । प॒शवः॑ ।  \newline


\textbf{Krama Paata} \newline

ग॒च्छ॒न्ति॒ पृ॒ष्ठानि॑ । पृ॒ष्ठानि॒ हि । हि दैवी᳚ । दैवी॑ सꣳ॒॒सत् । सꣳ॒॒सज् जा॒मि । सꣳ॒॒सदिति॑ सम् - सत् । जा॒मि वै । वा ए॒तत् । ए॒तत् कु॑र्वन्ति । कु॒र्व॒न्ति॒ यत् । यत् त्रयः॑ । त्रय॑स्त्रयस्त्रिꣳ॒॒शाः । त्र॒य॒स्त्रिꣳ॒॒शा अ॒न्वञ्चः॑ । त्र॒य॒स्त्रिꣳ॒॒शा इति॑ त्रयः - त्रिꣳ॒॒शाः । अ॒न्वञ्चो॒ मद्ध्ये᳚ । मद्ध्येऽनि॑रुक्तः । अनि॑रुक्तो भवति । अनि॑रुक्त॒ इत्यनिः॑ - उ॒क्तः॒ । भ॒व॒ति॒ तेन॑ । तेनाजा॑मि । अजा᳚म्यू॒र्द्ध्वानि॑ । ऊ॒र्द्ध्वानि॑ पृ॒ष्ठानि॑ । पृ॒ष्ठानि॑ भवन्ति । भ॒व॒न्त्यू॒र्द्ध्वाः । ऊ॒र्द्ध्वाश्छ॑न्दो॒माः । छ॒न्दो॒मा उ॒भाभ्या᳚म् । छ॒न्दो॒मा इति॑ छन्दः - माः । उ॒भाभ्याꣳ॑ रू॒पाभ्या᳚म् । रू॒पाभ्याꣳ॑ सुव॒र्गम् । सु॒व॒र्गम् ॅलो॒कम् । सु॒व॒र्गमिति॑ सुवः - गम् । लो॒कम् ॅय॑न्ति । य॒न्त्यस॑त्रम् । अस॑त्र॒म् ॅवै । वा ए॒तत् । ए॒तद् यत् । यद॑छन्दो॒मम् । अ॒छ॒न्दो॒मम् ॅयत् । अ॒छ॒न्दो॒ममित्य॑छन्दः - मम् । यच् छ॑न्दो॒माः । छ॒न्दो॒मा भव॑न्ति । छ॒न्दो॒मा इति॑ छन्दः - माः । भव॑न्ति॒ तेन॑ । तेन॑ स॒त्रम् । स॒त्रम् दे॒वताः᳚ । दे॒वता॑ ए॒व । ए॒व पृ॒ष्ठैः । पृ॒ष्ठैरव॑ । अव॑ रुन्धते । रु॒न्ध॒ते॒ प॒शून् । प॒शूञ्छ॑न्दो॒मैः । छ॒न्दो॒मैरोजः॑ । छ॒न्दो॒मैरिति॑ छन्दः - मैः । ओजो॒ वै । वै वी॒र्य᳚म् । वी॒र्य॑म् पृ॒ष्ठानि॑ । पृ॒ष्ठानि॑ प॒शवः॑ । प॒शव॑श्छन्दो॒माः \newline

\textbf{Jatai Paata} \newline

1. ग॒च्छ॒न्ति॒ पृ॒ष्ठानि॑ पृ॒ष्ठानि॑ गच्छन्ति गच्छन्ति पृ॒ष्ठानि॑ । \newline
2. पृ॒ष्ठानि॒ हि हि पृ॒ष्ठानि॑ पृ॒ष्ठानि॒ हि । \newline
3. हि दैवी॒ दैवी॒ हि हि दैवी᳚ । \newline
4. दैवी॑ सꣳ॒॒सथ् सꣳ॒॒सद् दैवी॒ दैवी॑ सꣳ॒॒सत् । \newline
5. सꣳ॒॒सज् जा॒मि जा॒मि सꣳ॒॒सथ् सꣳ॒॒सज् जा॒मि । \newline
6. सꣳ॒॒सदिति॑ सं - सत् । \newline
7. जा॒मि वै वै जा॒मि जा॒मि वै । \newline
8. वा ए॒त दे॒तद् वै वा ए॒तत् । \newline
9. ए॒तत् कु॑र्वन्ति कुर्वन्त्ये॒त दे॒तत् कु॑र्वन्ति । \newline
10. कु॒र्व॒न्ति॒ यद् यत् कु॑र्वन्ति कुर्वन्ति॒ यत् । \newline
11. यत् त्रय॒ स्त्रयो॒ यद् यत् त्रयः॑ । \newline
12. त्रय॑ स्त्रयस्त्रिꣳ॒॒शा स्त्र॑यस्त्रिꣳ॒॒शा स्त्रय॒ स्त्रय॑ स्त्रयस्त्रिꣳ॒॒शाः । \newline
13. त्र॒य॒स्त्रिꣳ॒॒शा अ॒न्वञ्चो॒ ऽन्वञ्च॑ स्त्रयस्त्रिꣳ॒॒शा स्त्र॑यस्त्रिꣳ॒॒शा अ॒न्वञ्चः॑ । \newline
14. त्र॒य॒स्त्रिꣳ॒॒शा इति॑ त्रयः - त्रिꣳ॒॒शाः । \newline
15. अ॒न्वञ्चो॒ मद्ध्ये॒ मद्ध्ये॒ ऽन्वञ्चो॒ ऽन्वञ्चो॒ मद्ध्ये᳚ । \newline
16. मद्ध्ये ऽनि॑रु॒क्तो ऽनि॑रुक्तो॒ मद्ध्ये॒ मद्ध्ये ऽनि॑रुक्तः । \newline
17. अनि॑रुक्तो भवति भव॒ त्यनि॑रु॒क्तो ऽनि॑रुक्तो भवति । \newline
18. अनि॑रुक्त॒ इत्यनिः॑ - उ॒क्तः॒ । \newline
19. भ॒व॒ति॒ तेन॒ तेन॑ भवति भवति॒ तेन॑ । \newline
20. तेना जा॒ म्यजा॑मि॒ तेन॒ तेना जा॑मि । \newline
21. अजा᳚ म्यू॒र्द्ध्वा न्यू॒र्द्ध्वा न्यजा॒ म्यजा᳚ म्यू॒र्द्ध्वानि॑ । \newline
22. ऊ॒र्द्ध्वानि॑ पृ॒ष्ठानि॑ पृ॒ष्ठा न्यू॒र्द्ध्वा न्यू॒र्द्ध्वानि॑ पृ॒ष्ठानि॑ । \newline
23. पृ॒ष्ठानि॑ भवन्ति भवन्ति पृ॒ष्ठानि॑ पृ॒ष्ठानि॑ भवन्ति । \newline
24. भ॒व॒न्त्यू॒र्द्ध्वा ऊ॒र्द्ध्वा भ॑वन्ति भवन्त्यू॒र्द्ध्वाः । \newline
25. ऊ॒र्द्ध्वा श्छ॑न्दो॒मा श्छ॑न्दो॒मा ऊ॒र्द्ध्वा ऊ॒र्द्ध्वा श्छ॑न्दो॒माः । \newline
26. छ॒न्दो॒मा उ॒भाभ्या॑ मु॒भाभ्या᳚म् छन्दो॒मा श्छ॑न्दो॒मा उ॒भाभ्या᳚म् । \newline
27. छ॒न्दो॒मा इति॑ छन्दः - माः । \newline
28. उ॒भाभ्याꣳ॑ रू॒पाभ्याꣳ॑ रू॒पाभ्या॑ मु॒भाभ्या॑ मु॒भाभ्याꣳ॑ रू॒पाभ्या᳚म् । \newline
29. रू॒पाभ्याꣳ॑ सुव॒र्गꣳ सु॑व॒र्गꣳ रू॒पाभ्याꣳ॑ रू॒पाभ्याꣳ॑ सुव॒र्गम् । \newline
30. सु॒व॒र्गम् ॅलो॒कम् ॅलो॒कꣳ सु॑व॒र्गꣳ सु॑व॒र्गम् ॅलो॒कम् । \newline
31. सु॒व॒र्गमिति॑ सुवः - गम् । \newline
32. लो॒कं ॅय॑न्ति यन्ति लो॒कम् ॅलो॒कं ॅय॑न्ति । \newline
33. य॒न्त्यस॑त्र॒ मस॑त्रं ॅयन्ति य॒न्त्यस॑त्रम् । \newline
34. अस॑त्रं॒ ॅवै वा अस॑त्र॒ मस॑त्रं॒ ॅवै । \newline
35. वा ए॒त दे॒तद् वै वा ए॒तत् । \newline
36. ए॒तद् यद् यदे॒त दे॒तद् यत् । \newline
37. यद॑छन्दो॒म म॑छन्दो॒मं ॅयद् यद॑छन्दो॒मम् । \newline
38. अ॒छ॒न्दो॒मं ॅयद् यद॑छन्दो॒म म॑छन्दो॒मं ॅयत् । \newline
39. अ॒छ॒न्दो॒ममित्य॑छन्दः - मम् । \newline
40. यच् छ॑न्दो॒मा श्छ॑न्दो॒मा यद् यच् छ॑न्दो॒माः । \newline
41. छ॒न्दो॒मा भव॑न्ति॒ भव॑न्ति छन्दो॒मा श्छ॑न्दो॒मा भव॑न्ति । \newline
42. छ॒न्दो॒मा इति॑ छन्दः - माः । \newline
43. भव॑न्ति॒ तेन॒ तेन॒ भव॑न्ति॒ भव॑न्ति॒ तेन॑ । \newline
44. तेन॑ स॒त्रꣳ स॒त्रम् तेन॒ तेन॑ स॒त्रम् । \newline
45. स॒त्रम् दे॒वता॑ दे॒वताः᳚ स॒त्रꣳ स॒त्रम् दे॒वताः᳚ । \newline
46. दे॒वता॑ ए॒वैव दे॒वता॑ दे॒वता॑ ए॒व । \newline
47. ए॒व पृ॒ष्ठैः पृ॒ष्ठै रे॒वैव पृ॒ष्ठैः । \newline
48. पृ॒ष्ठै रवाव॑ पृ॒ष्ठैः पृ॒ष्ठै रव॑ । \newline
49. अव॑ रुन्धते रुन्ध॒ते ऽवाव॑ रुन्धते । \newline
50. रु॒न्ध॒ते॒ प॒शून् प॒शून् रु॑न्धते रुन्धते प॒शून् । \newline
51. प॒शून् छ॑न्दो॒मै श्छ॑न्दो॒मैः प॒शून् प॒शून् छ॑न्दो॒मैः । \newline
52. छ॒न्दो॒मै रोज॒ ओज॑ श्छन्दो॒मै श्छ॑न्दो॒मै रोजः॑ । \newline
53. छ॒न्दो॒मैरिति॑ छन्दः - मैः । \newline
54. ओजो॒ वै वा ओज॒ ओजो॒ वै । \newline
55. वै वी॒र्यं॑ ॅवी॒र्यं॑ ॅवै वै वी॒र्य᳚म् । \newline
56. वी॒र्य॑म् पृ॒ष्ठानि॑ पृ॒ष्ठानि॑ वी॒र्यं॑ ॅवी॒र्य॑म् पृ॒ष्ठानि॑ । \newline
57. पृ॒ष्ठानि॑ प॒शवः॑ प॒शवः॑ पृ॒ष्ठानि॑ पृ॒ष्ठानि॑ प॒शवः॑ । \newline
58. प॒शव॑ श्छन्दो॒मा श्छ॑न्दो॒माः प॒शवः॑ प॒शव॑ श्छन्दो॒माः । \newline

\textbf{Ghana Paata } \newline

1. ग॒च्छ॒न्ति॒ पृ॒ष्ठानि॑ पृ॒ष्ठानि॑ गच्छन्ति गच्छन्ति पृ॒ष्ठानि॒ हि हि पृ॒ष्ठानि॑ गच्छन्ति गच्छन्ति पृ॒ष्ठानि॒ हि । \newline
2. पृ॒ष्ठानि॒ हि हि पृ॒ष्ठानि॑ पृ॒ष्ठानि॒ हि दैवी॒ दैवी॒ हि पृ॒ष्ठानि॑ पृ॒ष्ठानि॒ हि दैवी᳚ । \newline
3. हि दैवी॒ दैवी॒ हि हि दैवी॑ सꣳ॒॒सथ् सꣳ॒॒सद् दैवी॒ हि हि दैवी॑ सꣳ॒॒सत् । \newline
4. दैवी॑ सꣳ॒॒सथ् सꣳ॒॒सद् दैवी॒ दैवी॑ सꣳ॒॒सज् जा॒मि जा॒मि सꣳ॒॒सद् दैवी॒ दैवी॑ सꣳ॒॒सज् जा॒मि । \newline
5. सꣳ॒॒सज् जा॒मि जा॒मि सꣳ॒॒सथ् सꣳ॒॒सज् जा॒मि वै वै जा॒मि सꣳ॒॒सथ् सꣳ॒॒सज् जा॒मि वै । \newline
6. सꣳ॒॒सदिति॑ सं - सत् । \newline
7. जा॒मि वै वै जा॒मि जा॒मि वा ए॒त दे॒तद् वै जा॒मि जा॒मि वा ए॒तत् । \newline
8. वा ए॒त दे॒तद् वै वा ए॒तत् कु॑र्वन्ति कुर्व न्त्ये॒तद् वै वा ए॒तत् कु॑र्वन्ति । \newline
9. ए॒तत् कु॑र्वन्ति कुर्व न्त्ये॒त दे॒तत् कु॑र्वन्ति॒ यद् यत् कु॑र्व न्त्ये॒त दे॒तत् कु॑र्वन्ति॒ यत् । \newline
10. कु॒र्व॒न्ति॒ यद् यत् कु॑र्वन्ति कुर्वन्ति॒ यत् त्रय॒ स्त्रयो॒ यत् कु॑र्वन्ति कुर्वन्ति॒ यत् त्रयः॑ । \newline
11. यत् त्रय॒ स्त्रयो॒ यद् यत् त्रय॑ स्त्रयस्त्रिꣳ॒॒शा स्त्र॑यस्त्रिꣳ॒॒शा स्त्रयो॒ यद् यत् त्रय॑ स्त्रयस्त्रिꣳ॒॒शाः । \newline
12. त्रय॑ स्त्रयस्त्रिꣳ॒॒शा स्त्र॑यस्त्रिꣳ॒॒शा स्त्रय॒ स्त्रय॑ स्त्रयस्त्रिꣳ॒॒शा अ॒न्वञ्चो॒ ऽन्वञ्च॑ स्त्रयस्त्रिꣳ॒॒शा स्त्रय॒ स्त्रय॑ स्त्रयस्त्रिꣳ॒॒शा अ॒न्वञ्चः॑ । \newline
13. त्र॒य॒स्त्रिꣳ॒॒शा अ॒न्वञ्चो॒ ऽन्वञ्च॑ स्त्रयस्त्रिꣳ॒॒शा स्त्र॑यस्त्रिꣳ॒॒शा अ॒न्वञ्चो॒ मद्ध्ये॒ मद्ध्ये॒ ऽन्वञ्च॑ स्त्रयस्त्रिꣳ॒॒शा स्त्र॑यस्त्रिꣳ॒॒शा अ॒न्वञ्चो॒ मद्ध्ये᳚ । \newline
14. त्र॒य॒स्त्रिꣳ॒॒शा इति॑ त्रयः - त्रिꣳ॒॒शाः । \newline
15. अ॒न्वञ्चो॒ मद्ध्ये॒ मद्ध्ये॒ ऽन्वञ्चो॒ ऽन्वञ्चो॒ मद्ध्ये ऽनि॑रु॒क्तो ऽनि॑रुक्तो॒ मद्ध्ये॒ ऽन्वञ्चो॒ ऽन्वञ्चो॒ मद्ध्ये ऽनि॑रुक्तः । \newline
16. मद्ध्ये ऽनि॑रु॒क्तो ऽनि॑रुक्तो॒ मद्ध्ये॒ मद्ध्ये ऽनि॑रुक्तो भवति भव॒ त्यनि॑रुक्तो॒ मद्ध्ये॒ मद्ध्ये ऽनि॑रुक्तो भवति । \newline
17. अनि॑रुक्तो भवति भव॒ त्यनि॑रु॒क्तो ऽनि॑रुक्तो भवति॒ तेन॒ तेन॑ भव॒ त्यनि॑रु॒क्तो ऽनि॑रुक्तो भवति॒ तेन॑ । \newline
18. अनि॑रुक्त॒ इत्यनिः॑ - उ॒क्तः॒ । \newline
19. भ॒व॒ति॒ तेन॒ तेन॑ भवति भवति॒ तेना जा॒म्य जा॑मि॒ तेन॑ भवति भवति॒ तेना जा॑मि । \newline
20. तेना जा॒म्य जा॑मि॒ तेन॒ तेना जा᳚म्यू॒र्द्ध्वा न्यू॒र्द्ध्वा न्यजा॑मि॒ तेन॒ तेना जा᳚म्यू॒र्द्ध्वानि॑ । \newline
21. अजा᳚म्यू॒र्द्ध्वा न्यू॒र्द्ध्वा न्यजा॒ म्यजा᳚ म्यू॒र्द्ध्वानि॑ पृ॒ष्ठानि॑ पृ॒ष्ठा न्यू॒र्द्ध्वा न्यजा॒ म्यजा᳚ म्यू॒र्द्ध्वानि॑ पृ॒ष्ठानि॑ । \newline
22. ऊ॒र्द्ध्वानि॑ पृ॒ष्ठानि॑ पृ॒ष्ठा न्यू॒र्द्ध्वा न्यू॒र्द्ध्वानि॑ पृ॒ष्ठानि॑ भवन्ति भवन्ति पृ॒ष्ठा न्यू॒र्द्ध्वा न्यू॒र्द्ध्वानि॑ पृ॒ष्ठानि॑ भवन्ति । \newline
23. पृ॒ष्ठानि॑ भवन्ति भवन्ति पृ॒ष्ठानि॑ पृ॒ष्ठानि॑ भव न्त्यू॒र्द्ध्वा ऊ॒र्द्ध्वा भ॑वन्ति पृ॒ष्ठानि॑ पृ॒ष्ठानि॑ भव न्त्यू॒र्द्ध्वाः । \newline
24. भ॒व॒ न्त्यू॒र्द्ध्वा ऊ॒र्द्ध्वा भ॑वन्ति भव न्त्यू॒र्द्ध्वा श्छ॑न्दो॒मा श्छ॑न्दो॒मा ऊ॒र्द्ध्वा भ॑वन्ति भव न्त्यू॒र्द्ध्वा श्छ॑न्दो॒माः । \newline
25. ऊ॒र्द्ध्वा श्छ॑न्दो॒मा श्छ॑न्दो॒मा ऊ॒र्द्ध्वा ऊ॒र्द्ध्वा श्छ॑न्दो॒मा उ॒भाभ्या॑ मु॒भाभ्या᳚म् छन्दो॒मा ऊ॒र्द्ध्वा ऊ॒र्द्ध्वा श्छ॑न्दो॒मा उ॒भाभ्या᳚म् । \newline
26. छ॒न्दो॒मा उ॒भाभ्या॑ मु॒भाभ्या᳚म् छन्दो॒मा श्छ॑न्दो॒मा उ॒भाभ्याꣳ॑ रू॒पाभ्याꣳ॑ रू॒पाभ्या॑ मु॒भाभ्या᳚म् छन्दो॒मा श्छ॑न्दो॒मा उ॒भाभ्याꣳ॑ रू॒पाभ्या᳚म् । \newline
27. छ॒न्दो॒मा इति॑ छन्दः - माः । \newline
28. उ॒भाभ्याꣳ॑ रू॒पाभ्याꣳ॑ रू॒पाभ्या॑ मु॒भाभ्या॑ मु॒भाभ्याꣳ॑ रू॒पाभ्याꣳ॑ सुव॒र्गꣳ सु॑व॒र्गꣳ रू॒पाभ्या॑ मु॒भाभ्या॑ मु॒भाभ्याꣳ॑ रू॒पाभ्याꣳ॑ सुव॒र्गम् । \newline
29. रू॒पाभ्याꣳ॑ सुव॒र्गꣳ सु॑व॒र्गꣳ रू॒पाभ्याꣳ॑ रू॒पाभ्याꣳ॑ सुव॒र्गम् ॅलो॒कम् ॅलो॒कꣳ सु॑व॒र्गꣳ रू॒पाभ्याꣳ॑ रू॒पाभ्याꣳ॑ सुव॒र्गम् ॅलो॒कम् । \newline
30. सु॒व॒र्गम् ॅलो॒कम् ॅलो॒कꣳ सु॑व॒र्गꣳ सु॑व॒र्गम् ॅलो॒कं ॅय॑न्ति यन्ति लो॒कꣳ सु॑व॒र्गꣳ सु॑व॒र्गम् ॅलो॒कं ॅय॑न्ति । \newline
31. सु॒व॒र्गमिति॑ सुवः - गम् । \newline
32. लो॒कं ॅय॑न्ति यन्ति लो॒कम् ॅलो॒कं ॅय॒न्त्यस॑त्र॒ मस॑त्रं ॅयन्ति लो॒कम् ॅलो॒कं ॅय॒न्त्यस॑त्रम् । \newline
33. य॒न्त्यस॑त्र॒ मस॑त्रं ॅयन्ति य॒न्त्यस॑त्रं॒ ॅवै वा अस॑त्रं ॅयन्ति य॒न्त्यस॑त्रं॒ ॅवै । \newline
34. अस॑त्रं॒ ॅवै वा अस॑त्र॒ मस॑त्रं॒ ॅवा ए॒त दे॒तद् वा अस॑त्र॒ मस॑त्रं॒ ॅवा ए॒तत् । \newline
35. वा ए॒त दे॒तद् वै वा ए॒तद् यद् यदे॒तद् वै वा ए॒तद् यत् । \newline
36. ए॒तद् यद् यदे॒त दे॒तद् यद॑छन्दो॒म म॑छन्दो॒मं ॅयदे॒त दे॒तद् यद॑छन्दो॒मम् । \newline
37. यद॑छन्दो॒म म॑छन्दो॒मं ॅयद् यद॑छन्दो॒मं ॅयद् यद॑छन्दो॒मं ॅयद् यद॑छन्दो॒मं ॅयत् । \newline
38. अ॒छ॒न्दो॒मं ॅयद् यद॑छन्दो॒म म॑छन्दो॒मं ॅयच् छ॑न्दो॒मा श्छ॑न्दो॒मा यद॑छन्दो॒म म॑छन्दो॒मं ॅयच् छ॑न्दो॒माः । \newline
39. अ॒छ॒न्दो॒ममित्य॑छन्दः - मम् । \newline
40. यच् छ॑न्दो॒मा श्छ॑न्दो॒मा यद् यच् छ॑न्दो॒मा भव॑न्ति॒ भव॑न्ति छन्दो॒मा यद् यच् छ॑न्दो॒मा भव॑न्ति । \newline
41. छ॒न्दो॒मा भव॑न्ति॒ भव॑न्ति छन्दो॒मा श्छ॑न्दो॒मा भव॑न्ति॒ तेन॒ तेन॒ भव॑न्ति छन्दो॒मा श्छ॑न्दो॒मा भव॑न्ति॒ तेन॑ । \newline
42. छ॒न्दो॒मा इति॑ छन्दः - माः । \newline
43. भव॑न्ति॒ तेन॒ तेन॒ भव॑न्ति॒ भव॑न्ति॒ तेन॑ स॒त्रꣳ स॒त्रम् तेन॒ भव॑न्ति॒ भव॑न्ति॒ तेन॑ स॒त्रम् । \newline
44. तेन॑ स॒त्रꣳ स॒त्रम् तेन॒ तेन॑ स॒त्रम् दे॒वता॑ दे॒वताः᳚ स॒त्रम् तेन॒ तेन॑ स॒त्रम् दे॒वताः᳚ । \newline
45. स॒त्रम् दे॒वता॑ दे॒वताः᳚ स॒त्रꣳ स॒त्रम् दे॒वता॑ ए॒वैव दे॒वताः᳚ स॒त्रꣳ स॒त्रम् दे॒वता॑ ए॒व । \newline
46. दे॒वता॑ ए॒वैव दे॒वता॑ दे॒वता॑ ए॒व पृ॒ष्ठैः पृ॒ष्ठै रे॒व दे॒वता॑ दे॒वता॑ ए॒व पृ॒ष्ठैः । \newline
47. ए॒व पृ॒ष्ठैः पृ॒ष्ठै रे॒वैव पृ॒ष्ठै रवाव॑ पृ॒ष्ठै रे॒वैव पृ॒ष्ठै रव॑ । \newline
48. पृ॒ष्ठै रवाव॑ पृ॒ष्ठैः पृ॒ष्ठै रव॑ रुन्धते रुन्ध॒ते ऽव॑ पृ॒ष्ठैः पृ॒ष्ठै रव॑ रुन्धते । \newline
49. अव॑ रुन्धते रुन्ध॒ते ऽवाव॑ रुन्धते प॒शून् प॒शून् रु॑न्ध॒ते ऽवाव॑ रुन्धते प॒शून् । \newline
50. रु॒न्ध॒ते॒ प॒शून् प॒शून् रु॑न्धते रुन्धते प॒शून् छ॑न्दो॒मै श्छ॑न्दो॒मैः प॒शून् रु॑न्धते रुन्धते प॒शून् छ॑न्दो॒मैः । \newline
51. प॒शून् छ॑न्दो॒मै श्छ॑न्दो॒मैः प॒शून् प॒शून् छ॑न्दो॒मै रोज॒ ओज॑ श्छन्दो॒मैः प॒शून् प॒शून् छ॑न्दो॒मै रोजः॑ । \newline
52. छ॒न्दो॒मै रोज॒ ओज॑ श्छन्दो॒मै श्छ॑न्दो॒मै रोजो॒ वै वा ओज॑ श्छन्दो॒मै श्छ॑न्दो॒मै रोजो॒ वै । \newline
53. छ॒न्दो॒मैरिति॑ छन्दः - मैः । \newline
54. ओजो॒ वै वा ओज॒ ओजो॒ वै वी॒र्यं॑ ॅवी॒र्यं॑ ॅवा ओज॒ ओजो॒ वै वी॒र्य᳚म् । \newline
55. वै वी॒र्यं॑ ॅवी॒र्यं॑ ॅवै वै वी॒र्य॑म् पृ॒ष्ठानि॑ पृ॒ष्ठानि॑ वी॒र्यं॑ ॅवै वै वी॒र्य॑म् पृ॒ष्ठानि॑ । \newline
56. वी॒र्य॑म् पृ॒ष्ठानि॑ पृ॒ष्ठानि॑ वी॒र्यं॑ ॅवी॒र्य॑म् पृ॒ष्ठानि॑ प॒शवः॑ प॒शवः॑ पृ॒ष्ठानि॑ वी॒र्यं॑ ॅवी॒र्य॑म् पृ॒ष्ठानि॑ प॒शवः॑ । \newline
57. पृ॒ष्ठानि॑ प॒शवः॑ प॒शवः॑ पृ॒ष्ठानि॑ पृ॒ष्ठानि॑ प॒शव॑ श्छन्दो॒मा श्छ॑न्दो॒माः प॒शवः॑ पृ॒ष्ठानि॑ पृ॒ष्ठानि॑ प॒शव॑ श्छन्दो॒माः । \newline
58. प॒शव॑ श्छन्दो॒मा श्छ॑न्दो॒माः प॒शवः॑ प॒शव॑ श्छन्दो॒मा ओज॒ स्योज॑सि छन्दो॒माः प॒शवः॑ प॒शव॑ श्छन्दो॒मा ओज॑सि । \newline
\pagebreak
\markright{ TS 7.4.2.4  \hfill https://www.vedavms.in \hfill}

\section{ TS 7.4.2.4 }

\textbf{TS 7.4.2.4 } \newline
\textbf{Samhita Paata} \newline

छन्दो॒मा ओज॑स्ये॒व वी॒र्ये॑ प॒शुषु॒ प्रति॑ तिष्ठन्ति॒ त्रय॑स्त्रयस्त्रिꣳ॒॒शा अ॒वस्ता᳚द्-भवन्ति॒ त्रय॑स्त्रयस्त्रिꣳ॒॒शाः प॒रस्ता॒न्मद्ध्ये॑ पृ॒ष्ठान्युरो॒ वै त्र॑यस्त्रिꣳ॒॒शा आ॒त्मा पृ॒ष्ठान्या॒त्मन॑ ए॒व तद्-यज॑मानाः॒ शर्म॑ नह्य॒न्तेऽना᳚र्त्यै बृहद्-रथन्त॒राभ्यां᳚ ॅयन्ती॒यं ॅवाव र॑थन्त॒रम॒सौ बृ॒हदा॒भ्यामे॒व य॒न्त्यथो॑ अ॒नयो॑रे॒व प्रति॑ तिष्ठन्त्ये॒ते वै य॒ज्ञ्स्या᳚ञ्ज॒साय॑नी स्रु॒ती ताभ्या॑मे॒व - [  ] \newline

\textbf{Pada Paata} \newline

छ॒न्दो॒मा इति॑ छन्दः - माः । ओज॑सि । ए॒व । वी॒र्ये᳚ । प॒शुषु॑ । प्रतीति॑ । ति॒ष्ठ॒न्ति॒ । त्रयः॑ । त्र॒य॒स्त्रिꣳ॒॒शा इति॑ त्रयः - त्रिꣳ॒॒शाः । अ॒वस्ता᳚त् । भ॒व॒न्ति॒ । त्रयः॑ । त्र॒य॒स्त्रिꣳ॒॒शा इति॑ त्रयः - त्रिꣳ॒॒शाः । प॒रस्ता᳚त् । मद्ध्ये᳚ । पृ॒ष्ठानि॑ । उरः॑ । वै । त्र॒य॒स्त्रिꣳ॒॒शा इति॑ त्रयः - त्रिꣳ॒॒शाः । आ॒त्मा । पृ॒ष्ठानि॑ । आ॒त्मने᳚ । ए॒व । तत् । यज॑मानाः । शर्म॑ । न॒ह्य॒न्ते॒ । अना᳚र्त्यै । बृ॒ह॒द्र॒थ॒न्त॒राभ्या॒मिति॑ बृहत् - र॒थ॒न्त॒राभ्या᳚म् । य॒न्ती॒ । इ॒यम् । वाव । र॒थ॒न्त॒रमिति॑ रथं - त॒रम् । अ॒सौ । बृ॒हत् । आ॒भ्याम् । ए॒व । य॒न्ति॒ । अथो॒ इति॑ । अ॒नयोः᳚ । ए॒व । प्रतीति॑ । ति॒ष्ठ॒न्ति॒ । ए॒ते इति॑ । वै । य॒ज्ञ्स्य॑ । अ॒ञ्ज॒साय॑नी॒ इत्य॑ञ्जसा - अय॑नी । स्रु॒ती इति॑ । ताभ्या᳚म् । ए॒व ।  \newline


\textbf{Krama Paata} \newline

छ॒न्दो॒मा ओज॑सि । छ॒न्दो॒मा इति॑ छन्दः - माः । ओज॑स्ये॒व । ए॒व वी॒र्ये᳚ । वी॒र्ये॑ प॒शुषु॑ । प॒शुषु॒ प्रति॑ । प्रति॑ तिष्ठन्ति । ति॒ष्ठ॒न्ति॒ त्रयः॑ । त्रय॑स्त्रयस्त्रिꣳ॒॒शाः । त्र॒य॒स्त्रिꣳ॒॒शा अ॒वस्ता᳚त् । त्र॒य॒स्त्रिꣳ॒॒शा इति॑ त्रयः - त्रिꣳ॒॒शाः । अ॒वस्ता᳚द् भवन्ति । भ॒व॒न्ति॒ त्रयः॑ । त्रय॑स्त्रयस्त्रिꣳ॒॒शाः । त्र॒य॒स्त्रिꣳ॒॒शाः प॒रस्ता᳚त् । त्र॒य॒स्त्रिꣳ॒॒शा इति॑ त्रयः - त्रिꣳ॒॒शाः । प॒रस्ता॒न् मद्ध्ये᳚ । मद्ध्ये॑ पृ॒ष्ठानि॑ । पृ॒ष्ठान्युरः॑ । उरो॒ वै । वै त्र॑यस्त्रिꣳ॒॒शाः । त्र॒य॒स्त्रिꣳ॒॒शा आ॒त्मा । त्र॒य॒स्त्रिꣳ॒॒शा इति॑ त्रयः - त्रिꣳ॒॒शाः । आ॒त्मा पृ॒ष्ठानि॑ । पृ॒ष्ठान्या॒त्मने᳚ । आ॒त्मन॑ ए॒व । ए॒व तत् । तद् यज॑मानाः । यज॑मानाः॒ शर्म॑ । शर्म॑ नह्यन्ते । न॒ह्य॒न्तेऽना᳚र्त्यै । अना᳚र्त्यै बृहद्‍रथन्त॒राभ्या᳚म् । बृ॒ह॒द्‍र॒थ॒न्त॒राभ्या᳚म् ॅयन्ति । बृ॒ह॒द्‍र॒थ॒न्त॒राभ्या॒मिति॑ बृहत् - र॒थ॒न्त॒राभ्या᳚म् । य॒न्ती॒यम् । इ॒यम् ॅवाव । वाव र॑थन्त॒रम् । र॒थ॒न्त॒रम॒सौ । र॒थ॒न्त॒रमिति॑ रथम् - त॒रम् । अ॒सौ बृ॒हत् । बृ॒हदा॒भ्याम् । आ॒भ्यामे॒व । ए॒व य॑न्ति । य॒न्त्यथो᳚ । अथो॑ अ॒नयोः᳚ । अथो॒ इत्यथो᳚ । अ॒नयो॑रे॒व । ए॒व प्रति॑ । प्रति॑ तिष्ठन्ति । ति॒ष्ठ॒न्त्ये॒ते । ए॒ते वै । ए॒ते इत्ये॒ते । वै य॒ज्ञ्स्य॑ । य॒ज्ञ्स्या᳚ञ्ज॒साय॑नी । अ॒ञ्ज॒साय॑नी स्रु॒ती । अ॒ञ्ज॒साय॑नी॒ इत्य॑ञ्जसा - अय॑नी । स्रु॒ती ताभ्या᳚म् । 
स्रु॒ती इति॑ स्रु॒ती । ताभ्या॑मे॒व । ए॒व सु॑व॒र्गम् \newline

\textbf{Jatai Paata} \newline

1. छ॒न्दो॒मा ओज॒स्यो ज॑सि छन्दो॒मा श्छ॑न्दो॒मा ओज॑सि । \newline
2. छ॒न्दो॒मा इति॑ छन्दः - माः । \newline
3. ओज॑ स्ये॒वैवौज॒ स्योज॑स्ये॒व । \newline
4. ए॒व वी॒र्ये॑ वी॒र्य॑ ए॒वैव वी॒र्ये᳚ । \newline
5. वी॒र्ये॑ प॒शुषु॑ प॒शुषु॑ वी॒र्ये॑ वी॒र्ये॑ प॒शुषु॑ । \newline
6. प॒शुषु॒ प्रति॒ प्रति॑ प॒शुषु॑ प॒शुषु॒ प्रति॑ । \newline
7. प्रति॑ तिष्ठन्ति तिष्ठन्ति॒ प्रति॒ प्रति॑ तिष्ठन्ति । \newline
8. ति॒ष्ठ॒न्ति॒ त्रय॒ स्त्रय॑ स्तिष्ठन्ति तिष्ठन्ति॒ त्रयः॑ । \newline
9. त्रय॑ स्त्रयस्त्रिꣳ॒॒शा स्त्र॑यस्त्रिꣳ॒॒शा स्त्रय॒ स्त्रय॑ स्त्रयस्त्रिꣳ॒॒शाः । \newline
10. त्र॒य॒स्त्रिꣳ॒॒शा अ॒वस्ता॑ द॒वस्ता᳚त् त्रयस्त्रिꣳ॒॒शा स्त्र॑यस्त्रिꣳ॒॒शा अ॒वस्ता᳚त् । \newline
11. त्र॒य॒स्त्रिꣳ॒॒शा इति॑ त्रयः - त्रिꣳ॒॒शाः । \newline
12. अ॒वस्ता᳚द् भवन्ति भवन्त्य॒वस्ता॑ द॒वस्ता᳚द् भवन्ति । \newline
13. भ॒व॒न्ति॒ त्रय॒ स्त्रयो॑ भवन्ति भवन्ति॒ त्रयः॑ । \newline
14. त्रय॑ स्त्रयस्त्रिꣳ॒॒शा स्त्र॑यस्त्रिꣳ॒॒शा स्त्रय॒ स्त्रय॑ स्त्रयस्त्रिꣳ॒॒शाः । \newline
15. त्र॒य॒स्त्रिꣳ॒॒शाः प॒रस्ता᳚त् प॒रस्ता᳚त् त्रयस्त्रिꣳ॒॒शा स्त्र॑य स्त्रिꣳ॒॒शाः प॒रस्ता᳚त् । \newline
16. त्र॒य॒स्त्रिꣳ॒॒शा इति॑ त्रयः - त्रिꣳ॒॒शाः । \newline
17. प॒रस्ता॒न् मद्ध्ये॒ मद्ध्ये॑ प॒रस्ता᳚त् प॒रस्ता॒न् मद्ध्ये᳚ । \newline
18. मद्ध्ये॑ पृ॒ष्ठानि॑ पृ॒ष्ठानि॒ मद्ध्ये॒ मद्ध्ये॑ पृ॒ष्ठानि॑ । \newline
19. पृ॒ष्ठा न्युर॒ उरः॑ पृ॒ष्ठानि॑ पृ॒ष्ठा न्युरः॑ । \newline
20. उरो॒ वै वा उर॒ उरो॒ वै । \newline
21. वै त्र॑यस्त्रिꣳ॒॒शा स्त्र॑य स्त्रिꣳ॒॒शा वै वै त्र॑य स्त्रिꣳ॒॒शाः । \newline
22. त्र॒य॒स्त्रिꣳ॒॒शा आ॒त्मा ऽऽत्मा त्र॑यस्त्रिꣳ॒॒शा स्त्र॑यस्त्रिꣳ॒॒शा आ॒त्मा । \newline
23. त्र॒य॒स्त्रिꣳ॒॒शा इति॑ त्रयः - त्रिꣳ॒॒शाः । \newline
24. आ॒त्मा पृ॒ष्ठानि॑ पृ॒ष्ठा न्या॒त्मा ऽऽत्मा पृ॒ष्ठानि॑ । \newline
25. पृ॒ष्ठा न्या॒त्मन॑ आ॒त्मने॑ पृ॒ष्ठानि॑ पृ॒ष्ठा न्या॒त्मने᳚ । \newline
26. आ॒त्मन॑ ए॒वै वात्मन॑ आ॒त्मन॑ ए॒व । \newline
27. ए॒व तत् तदे॒वैव तत् । \newline
28. तद् यज॑माना॒ यज॑माना॒ स्तत् तद् यज॑मानाः । \newline
29. यज॑मानाः॒ शर्म॒ शर्म॒ यज॑माना॒ यज॑मानाः॒ शर्म॑ । \newline
30. शर्म॑ नह्यन्ते नह्यन्ते॒ शर्म॒ शर्म॑ नह्यन्ते । \newline
31. न॒ह्य॒न्ते ऽना᳚र्त्या॒ अना᳚र्त्यै नह्यन्ते नह्य॒न्ते ऽना᳚र्त्यै । \newline
32. अना᳚र्त्यै बृहद्रथन्त॒राभ्या᳚म् बृहद्रथन्त॒राभ्या॒ मना᳚र्त्या॒ अना᳚र्त्यै बृहद्रथन्त॒राभ्या᳚म् । \newline
33. बृ॒ह॒द्र॒थ॒न्त॒राभ्यां᳚ ॅयन्ति यन्ति बृहद्रथन्त॒राभ्या᳚म् बृहद्रथन्त॒राभ्यां᳚ ॅयन्ति । \newline
34. बृ॒ह॒द्र॒थ॒न्त॒राभ्या॒मिति॑ बृहत् - र॒थ॒न्त॒राभ्या᳚म् । \newline
35. य॒न्ती॒य मि॒यं ॅय॑न्ति यन्ती॒यम् । \newline
36. इ॒यं ॅवाव वावेय मि॒यं ॅवाव । \newline
37. वाव र॑थन्त॒रꣳ र॑थन्त॒रं ॅवाव वाव र॑थन्त॒रम् । \newline
38. र॒थ॒न्त॒र म॒सा व॒सौ र॑थन्त॒रꣳ र॑थन्त॒र म॒सौ । \newline
39. र॒थ॒न्त॒रमिति॑ रथं - त॒रम् । \newline
40. अ॒सौ बृ॒हद् बृ॒ह द॒सा व॒सौ बृ॒हत् । \newline
41. बृ॒ह दा॒भ्या मा॒भ्याम् बृ॒हद् बृ॒ह दा॒भ्याम् । \newline
42. आ॒भ्या मे॒वै वाभ्या मा॒भ्या मे॒व । \newline
43. ए॒व य॑न्ति यन्त्ये॒वैव य॑न्ति । \newline
44. य॒न्त्यथो॒ अथो॑ यन्ति य॒न्त्यथो᳚ । \newline
45. अथो॑ अ॒नयो॑ र॒नयो॒ रथो॒ अथो॑ अ॒नयोः᳚ । \newline
46. अथो॒ इत्यथो᳚ । \newline
47. अ॒नयो॑ रे॒वै वानयो॑ र॒नयो॑ रे॒व । \newline
48. ए॒व प्रति॒ प्रत्ये॒वैव प्रति॑ । \newline
49. प्रति॑ तिष्ठन्ति तिष्ठन्ति॒ प्रति॒ प्रति॑ तिष्ठन्ति । \newline
50. ति॒ष्ठ॒न्त्ये॒ते ए॒ते ति॑ष्ठन्ति तिष्ठन्त्ये॒ते । \newline
51. ए॒ते वै वा ए॒ते ए॒ते वै । \newline
52. ए॒ते इत्ये॒ते । \newline
53. वै य॒ज्ञ्स्य॑ य॒ज्ञ्स्य॒ वै वै य॒ज्ञ्स्य॑ । \newline
54. य॒ज्ञ्स्या᳚ ञ्ज॒साय॑नी अञ्ज॒साय॑नी य॒ज्ञ्स्य॑ य॒ज्ञ्स्या᳚ ञ्ज॒साय॑नी । \newline
55. अ॒ञ्ज॒साय॑नी स्रु॒ती स्रु॒ती अ॑ञ्ज॒साय॑नी अञ्ज॒साय॑नी स्रु॒ती । \newline
56. अ॒ञ्ज॒साय॑नी॒ इत्य॑ञ्जसा - अय॑नी । \newline
57. स्रु॒ती ताभ्या॒म् ताभ्याꣳ॑ स्रु॒ती स्रु॒ती ताभ्या᳚म् । \newline
58. स्रु॒ती इति॑ स्रु॒ती । \newline
59. ताभ्या॑ मे॒वैव ताभ्या॒म् ताभ्या॑ मे॒व । \newline
60. ए॒व सु॑व॒र्गꣳ सु॑व॒र्ग मे॒वैव सु॑व॒र्गम् । \newline

\textbf{Ghana Paata } \newline

1. छ॒न्दो॒मा ओज॒ स्योज॑सि छन्दो॒मा श्छ॑न्दो॒मा ओज॑ स्ये॒वैवौज॑सि छन्दो॒मा श्छ॑न्दो॒मा ओज॑ स्ये॒व । \newline
2. छ॒न्दो॒मा इति॑ छन्दः - माः । \newline
3. ओज॑ स्ये॒वैवौज॒ स्योज॑ स्ये॒व वी॒र्ये॑ वी॒र्य॑ ए॒वौज॒ स्योज॑ स्ये॒व वी॒र्ये᳚ । \newline
4. ए॒व वी॒र्ये॑ वी॒र्य॑ ए॒वैव वी॒र्ये॑ प॒शुषु॑ प॒शुषु॑ वी॒र्य॑ ए॒वैव वी॒र्ये॑ प॒शुषु॑ । \newline
5. वी॒र्ये॑ प॒शुषु॑ प॒शुषु॑ वी॒र्ये॑ वी॒र्ये॑ प॒शुषु॒ प्रति॒ प्रति॑ प॒शुषु॑ वी॒र्ये॑ वी॒र्ये॑ प॒शुषु॒ प्रति॑ । \newline
6. प॒शुषु॒ प्रति॒ प्रति॑ प॒शुषु॑ प॒शुषु॒ प्रति॑ तिष्ठन्ति तिष्ठन्ति॒ प्रति॑ प॒शुषु॑ प॒शुषु॒ प्रति॑ तिष्ठन्ति । \newline
7. प्रति॑ तिष्ठन्ति तिष्ठन्ति॒ प्रति॒ प्रति॑ तिष्ठन्ति॒ त्रय॒ स्त्रय॑ स्तिष्ठन्ति॒ प्रति॒ प्रति॑ तिष्ठन्ति॒ त्रयः॑ । \newline
8. ति॒ष्ठ॒न्ति॒ त्रय॒ स्त्रय॑ स्तिष्ठन्ति तिष्ठन्ति॒ त्रय॑ स्त्रयस्त्रिꣳ॒॒शा स्त्र॑यस्त्रिꣳ॒॒शा स्त्रय॑ स्तिष्ठन्ति तिष्ठन्ति॒ त्रय॑ स्त्रयस्त्रिꣳ॒॒शाः । \newline
9. त्रय॑ स्त्रयस्त्रिꣳ॒॒शा स्त्र॑यस्त्रिꣳ॒॒शा स्त्रय॒ स्त्रय॑ स्त्रयस्त्रिꣳ॒॒शा अ॒वस्ता॑ द॒वस्ता᳚त् त्रयस्त्रिꣳ॒॒शा स्त्रय॒ स्त्रय॑ स्त्रयस्त्रिꣳ॒॒शा अ॒वस्ता᳚त् । \newline
10. त्र॒य॒स्त्रिꣳ॒॒शा अ॒वस्ता॑ द॒वस्ता᳚त् त्रयस्त्रिꣳ॒॒शा स्त्र॑यस्त्रिꣳ॒॒शा अ॒वस्ता᳚द् भवन्ति भव न्त्य॒वस्ता᳚त् त्रयस्त्रिꣳ॒॒शा स्त्र॑यस्त्रिꣳ॒॒शा अ॒वस्ता᳚द् भवन्ति । \newline
11. त्र॒य॒स्त्रिꣳ॒॒शा इति॑ त्रयः - त्रिꣳ॒॒शाः । \newline
12. अ॒वस्ता᳚द् भवन्ति भव न्त्य॒वस्ता॑ द॒वस्ता᳚द् भवन्ति॒ त्रय॒ स्त्रयो॑ भव न्त्य॒वस्ता॑ द॒वस्ता᳚द् भवन्ति॒ त्रयः॑ । \newline
13. भ॒व॒न्ति॒ त्रय॒ स्त्रयो॑ भवन्ति भवन्ति॒ त्रय॑ स्त्रयस्त्रिꣳ॒॒शा स्त्र॑यस्त्रिꣳ॒॒शा स्त्रयो॑ भवन्ति भवन्ति॒ त्रय॑ स्त्रयस्त्रिꣳ॒॒शाः । \newline
14. त्रय॑ स्त्रयस्त्रिꣳ॒॒शा स्त्र॑यस्त्रिꣳ॒॒शा स्त्रय॒ स्त्रय॑ स्त्रयस्त्रिꣳ॒॒शाः प॒रस्ता᳚त् प॒रस्ता᳚त् त्रयस्त्रिꣳ॒॒शा स्त्रय॒ स्त्रय॑ स्त्रयस्त्रिꣳ॒॒शाः प॒रस्ता᳚त् । \newline
15. त्र॒य॒स्त्रिꣳ॒॒शाः प॒रस्ता᳚त् प॒रस्ता᳚त् त्रयस्त्रिꣳ॒॒शा स्त्र॑यस्त्रिꣳ॒॒शाः प॒रस्ता॒न् मद्ध्ये॒ मद्ध्ये॑ प॒रस्ता᳚त् त्रयस्त्रिꣳ॒॒शा स्त्र॑यस्त्रिꣳ॒॒शाः प॒रस्ता॒न् मद्ध्ये᳚ । \newline
16. त्र॒य॒स्त्रिꣳ॒॒शा इति॑ त्रयः - त्रिꣳ॒॒शाः । \newline
17. प॒रस्ता॒न् मद्ध्ये॒ मद्ध्ये॑ प॒रस्ता᳚त् प॒रस्ता॒न् मद्ध्ये॑ पृ॒ष्ठानि॑ पृ॒ष्ठानि॒ मद्ध्ये॑ प॒रस्ता᳚त् प॒रस्ता॒न् मद्ध्ये॑ पृ॒ष्ठानि॑ । \newline
18. मद्ध्ये॑ पृ॒ष्ठानि॑ पृ॒ष्ठानि॒ मद्ध्ये॒ मद्ध्ये॑ पृ॒ष्ठा न्युर॒ उरः॑ पृ॒ष्ठानि॒ मद्ध्ये॒ मद्ध्ये॑ पृ॒ष्ठा न्युरः॑ । \newline
19. पृ॒ष्ठा न्युर॒ उरः॑ पृ॒ष्ठानि॑ पृ॒ष्ठा न्युरो॒ वै वा उरः॑ पृ॒ष्ठानि॑ पृ॒ष्ठा न्युरो॒ वै । \newline
20. उरो॒ वै वा उर॒ उरो॒ वै त्र॑यस्त्रिꣳ॒॒शा स्त्र॑यस्त्रिꣳ॒॒शा वा उर॒ उरो॒ वै त्र॑यस्त्रिꣳ॒॒शाः । \newline
21. वै त्र॑यस्त्रिꣳ॒॒शा स्त्र॑यस्त्रिꣳ॒॒शा वै वै त्र॑यस्त्रिꣳ॒॒शा आ॒त्मा ऽऽत्मा त्र॑यस्त्रिꣳ॒॒शा वै वै त्र॑यस्त्रिꣳ॒॒शा आ॒त्मा । \newline
22. त्र॒य॒स्त्रिꣳ॒॒शा आ॒त्मा ऽऽत्मा त्र॑यस्त्रिꣳ॒॒शा स्त्र॑यस्त्रिꣳ॒॒शा आ॒त्मा पृ॒ष्ठानि॑ पृ॒ष्ठा न्या॒त्मा त्र॑यस्त्रिꣳ॒॒शा स्त्र॑यस्त्रिꣳ॒॒शा आ॒त्मा पृ॒ष्ठानि॑ । \newline
23. त्र॒य॒स्त्रिꣳ॒॒शा इति॑ त्रयः - त्रिꣳ॒॒शाः । \newline
24. आ॒त्मा पृ॒ष्ठानि॑ पृ॒ष्ठा न्या॒त्मा ऽऽत्मा पृ॒ष्ठा न्या॒त्मन॑ आ॒त्मने॑ पृ॒ष्ठा न्या॒त्मा ऽऽत्मा पृ॒ष्ठा न्या॒त्मने᳚ । \newline
25. पृ॒ष्ठा न्या॒त्मन॑ आ॒त्मने॑ पृ॒ष्ठानि॑ पृ॒ष्ठा न्या॒त्मन॑ ए॒वै वात्मने॑ पृ॒ष्ठानि॑ पृ॒ष्ठा न्या॒त्मन॑ ए॒व । \newline
26. आ॒त्मन॑ ए॒वै वात्मन॑ आ॒त्मन॑ ए॒व तत् तदे॒वात्मन॑ आ॒त्मन॑ ए॒व तत् । \newline
27. ए॒व तत् तदे॒वैव तद् यज॑माना॒ यज॑माना॒ स्तदे॒वैव तद् यज॑मानाः । \newline
28. तद् यज॑माना॒ यज॑माना॒ स्तत् तद् यज॑मानाः॒ शर्म॒ शर्म॒ यज॑माना॒ स्तत् तद् यज॑मानाः॒ शर्म॑ । \newline
29. यज॑मानाः॒ शर्म॒ शर्म॒ यज॑माना॒ यज॑मानाः॒ शर्म॑ नह्यन्ते नह्यन्ते॒ शर्म॒ यज॑माना॒ यज॑मानाः॒ शर्म॑ नह्यन्ते । \newline
30. शर्म॑ नह्यन्ते नह्यन्ते॒ शर्म॒ शर्म॑ नह्य॒न्ते ऽना᳚र्त्या॒ अना᳚र्त्यै नह्यन्ते॒ शर्म॒ शर्म॑ नह्य॒न्ते ऽना᳚र्त्यै । \newline
31. न॒ह्य॒न्ते ऽना᳚र्त्या॒ अना᳚र्त्यै नह्यन्ते नह्य॒न्ते ऽना᳚र्त्यै बृहद्रथन्त॒राभ्या᳚म् बृहद्रथन्त॒राभ्या॒ मना᳚र्त्यै नह्यन्ते नह्य॒न्ते ऽना᳚र्त्यै बृहद्रथन्त॒राभ्या᳚म् । \newline
32. अना᳚र्त्यै बृहद्रथन्त॒राभ्या᳚म् बृहद्रथन्त॒राभ्या॒ मना᳚र्त्या॒ अना᳚र्त्यै बृहद्रथन्त॒राभ्यां᳚ ॅयन्ति यन्ति बृहद्रथन्त॒राभ्या॒ मना᳚र्त्या॒ अना᳚र्त्यै बृहद्रथन्त॒राभ्यां᳚ ॅयन्ति । \newline
33. बृ॒ह॒द्र॒थ॒न्त॒राभ्यां᳚ ॅयन्ति यन्ति बृहद्रथन्त॒राभ्या᳚म् बृहद्रथन्त॒राभ्यां᳚ ॅयन्ती॒य मि॒यं ॅय॑न्ति बृहद्रथन्त॒राभ्या᳚म् बृहद्रथन्त॒राभ्यां᳚ ॅयन्ती॒यम् । \newline
34. बृ॒ह॒द्र॒थ॒न्त॒राभ्या॒मिति॑ बृहत् - र॒थ॒न्त॒राभ्या᳚म् । \newline
35. य॒न्ती॒य मि॒यं ॅय॑न्ति यन्ती॒यं ॅवाव वावेयं ॅय॑न्ति यन्ती॒यं ॅवाव । \newline
36. इ॒यं ॅवाव वावेय मि॒यं ॅवाव र॑थन्त॒रꣳ र॑थन्त॒रं ॅवावेय मि॒यं ॅवाव र॑थन्त॒रम् । \newline
37. वाव र॑थन्त॒रꣳ र॑थन्त॒रं ॅवाव वाव र॑थन्त॒र म॒सा व॒सौ र॑थन्त॒रं ॅवाव वाव र॑थन्त॒र म॒सौ । \newline
38. र॒थ॒न्त॒र म॒सा व॒सौ र॑थन्त॒रꣳ र॑थन्त॒र म॒सौ बृ॒हद् बृ॒ह द॒सौ र॑थन्त॒रꣳ र॑थन्त॒र म॒सौ बृ॒हत् । \newline
39. र॒थ॒न्त॒रमिति॑ रथं - त॒रम् । \newline
40. अ॒सौ बृ॒हद् बृ॒ह द॒सा व॒सौ बृ॒ह दा॒भ्या मा॒भ्याम् बृ॒ह द॒सा व॒सौ बृ॒ह दा॒भ्याम् । \newline
41. बृ॒ह दा॒भ्या मा॒भ्याम् बृ॒हद् बृ॒ह दा॒भ्या मे॒वै वाभ्याम् बृ॒हद् बृ॒ह दा॒भ्या मे॒व । \newline
42. आ॒भ्या मे॒वै वाभ्या मा॒भ्या मे॒व य॑न्ति यन्त्ये॒वाभ्या मा॒भ्या मे॒व य॑न्ति । \newline
43. ए॒व य॑न्ति यन्त्ये॒वैव य॒न्त्यथो॒ अथो॑ यन्त्ये॒वैव य॒न्त्यथो᳚ । \newline
44. य॒न्त्यथो॒ अथो॑ यन्ति य॒न्त्यथो॑ अ॒नयो॑ र॒नयो॒ रथो॑ यन्ति य॒न्त्यथो॑ अ॒नयोः᳚ । \newline
45. अथो॑ अ॒नयो॑ र॒नयो॒ रथो॒ अथो॑ अ॒नयो॑ रे॒वै वानयो॒ रथो॒ अथो॑ अ॒नयो॑रे॒व । \newline
46. अथो॒ इत्यथो᳚ । \newline
47. अ॒नयो॑ रे॒वै वानयो॑ र॒नयो॑ रे॒व प्रति॒ प्रत्ये॒ वानयो॑ र॒नयो॑ रे॒व प्रति॑ । \newline
48. ए॒व प्रति॒ प्रत्ये॒वैव प्रति॑ तिष्ठन्ति तिष्ठन्ति॒ प्रत्ये॒वैव प्रति॑ तिष्ठन्ति । \newline
49. प्रति॑ तिष्ठन्ति तिष्ठन्ति॒ प्रति॒ प्रति॑ तिष्ठन्त्ये॒ते ए॒ते ति॑ष्ठन्ति॒ प्रति॒ प्रति॑ तिष्ठन्त्ये॒ते । \newline
50. ति॒ष्ठ॒ न्त्ये॒ते ए॒ते ति॑ष्ठन्ति तिष्ठ न्त्ये॒ते वै वा ए॒ते ति॑ष्ठन्ति तिष्ठ न्त्ये॒ते वै । \newline
51. ए॒ते वै वा ए॒ते ए॒ते वै य॒ज्ञ्स्य॑ य॒ज्ञ्स्य॒ वा ए॒ते ए॒ते वै य॒ज्ञ्स्य॑ । \newline
52. ए॒ते इत्ये॒ते । \newline
53. वै य॒ज्ञ्स्य॑ य॒ज्ञ्स्य॒ वै वै य॒ज्ञ्स्या᳚ ञ्ज॒साय॑नी अञ्ज॒साय॑नी य॒ज्ञ्स्य॒ वै वै य॒ज्ञ्स्या᳚ ञ्ज॒साय॑नी । \newline
54. य॒ज्ञ्स्या᳚ ञ्ज॒साय॑नी अञ्ज॒साय॑नी य॒ज्ञ्स्य॑ य॒ज्ञ्स्या᳚ ञ्ज॒साय॑नी स्रु॒ती स्रु॒ती अ॑ञ्ज॒साय॑नी य॒ज्ञ्स्य॑ य॒ज्ञ्स्या᳚ ञ्ज॒साय॑नी स्रु॒ती । \newline
55. अ॒ञ्ज॒साय॑नी स्रु॒ती स्रु॒ती अ॑ञ्ज॒साय॑नी अञ्ज॒साय॑नी स्रु॒ती ताभ्या॒म् ताभ्याꣳ॑ स्रु॒ती अ॑ञ्ज॒साय॑नी अञ्ज॒साय॑नी स्रु॒ती ताभ्या᳚म् । \newline
56. अ॒ञ्ज॒साय॑नी॒ इत्य॑ञ्जसा - अय॑नी । \newline
57. स्रु॒ती ताभ्या॒म् ताभ्याꣳ॑ स्रु॒ती स्रु॒ती ताभ्या॑ मे॒वैव ताभ्याꣳ॑ स्रु॒ती स्रु॒ती ताभ्या॑ मे॒व । \newline
58. स्रु॒ती इति॑ स्रु॒ती । \newline
59. ताभ्या॑ मे॒वैव ताभ्या॒म् ताभ्या॑ मे॒व सु॑व॒र्गꣳ सु॑व॒र्ग मे॒व ताभ्या॒म् ताभ्या॑ मे॒व सु॑व॒र्गम् । \newline
60. ए॒व सु॑व॒र्गꣳ सु॑व॒र्ग मे॒वैव सु॑व॒र्गम् ॅलो॒कम् ॅलो॒कꣳ सु॑व॒र्ग मे॒वैव सु॑व॒र्गम् ॅलो॒कम् । \newline
\pagebreak
\markright{ TS 7.4.2.5  \hfill https://www.vedavms.in \hfill}

\section{ TS 7.4.2.5 }

\textbf{TS 7.4.2.5 } \newline
\textbf{Samhita Paata} \newline

सु॑व॒र्गं ॅलो॒कं ॅय॑न्ति॒ परा᳚ञ्चो॒ वा ए॒ते सु॑व॒र्गं ॅलो॒कम॒भ्यारो॑हन्ति॒ ये प॑रा॒चीना॑नि पृ॒ष्ठान्यु॑प॒यन्ति॑ प्र॒त्यङ् ष॑ड॒हो भ॑वति प्र॒त्यव॑रूढ्या॒ अथो॒ प्रति॑ष्ठित्या उ॒भयो᳚र्लो॒कयोर॑ ऋ॒द्ध्वोत् ति॑ष्ठन्ति त्रि॒वृतोऽधि॑ त्रि॒वृत॒मुप॑ यन्ति॒ स्तोमा॑नाꣳ॒॒ संप॑त्त्यै प्रभ॒वाय॒ ज्योति॑रग्निष्टो॒मो भ॑वत्य॒यं ॅवाव स क्षयो॒ऽस्मादे॒व तेन॒ क्षया॒न्न य॑न्ति चतुर्विꣳशतिरा॒त्रो भ॑वति॒ चतु॑र्विꣳशतिरर्द्धमा॒साः सं॑ॅवथ्स॒रः ( ) सं॑ॅवथ्स॒रः सु॑व॒र्गो लो॒कः सं॑ॅवथ्स॒र ए॒व सु॑व॒र्गे लो॒के प्रति॑ तिष्ठ॒न्त्यथो॒ चतु॑र्विꣳशत्यक्षरा गाय॒त्री गा॑य॒त्री ब्र॑ह्मवर्च॒सं गा॑यत्रि॒यैव ब्र॑ह्मवर्च॒समव॑ रुन्धते ऽतिरा॒त्राव॒भितो॑ भवतो ब्रह्मवर्च॒सस्य॒ परि॑गृहीत्यै ॥ \newline

\textbf{Pada Paata} \newline

सु॒व॒र्गमिति॑ सुवः - गम् । लो॒कम् । य॒न्ति॒ । परा᳚ञ्चः । वै । ए॒ते । सु॒व॒र्गमिति॑ सुवः-गम् । लो॒कम् । अ॒भ्यारो॑ह॒न्तीत्य॑भि - आरो॑हन्ति । ये । प॒रा॒चीना॑नि । पृ॒ष्ठानि॑ । उ॒प॒यन्तीत्यु॑प - यन्ति॑ । प्र॒त्यङ् । ष॒ड॒ह इति॑ षट् - अ॒हः । भ॒व॒ति॒ । प्र॒त्यव॑रूढ्या॒ इति॑ प्रति - अव॑रूढ्यै । अथो॒ इति॑ । प्रति॑ष्ठित्या॒ इति॒ प्रति॑ - स्थि॒त्यै॒ । उ॒भयोः᳚ । लो॒कयोः᳚ । ऋ॒द्ध्वा । उदिति॑ । ति॒ष्ठ॒न्ति॒ । त्रि॒वृत॒ इति॑ त्रि - वृतः॑ । अधीति॑ । त्रि॒वृत॒मिति॑ त्रि - वृत᳚म् । उपेति॑ । य॒न्ति॒ । स्तोमा॑नाम् । संप॑त्त्या॒ इति॒ सं - प॒त्त्यै॒ । प्र॒भ॒वायेति॑ प्र - भ॒वाय॑ । ज्योतिः॑ । अ॒ग्नि॒ष्टो॒म इत्य॑ग्नि-स्तो॒मः । भ॒व॒ति॒ । अ॒यम् । वाव । सः । क्षयः॑ । अ॒स्मात् । ए॒व । तेन॑ । क्षया᳚त् । न । य॒न्ति॒ । च॒तु॒र्विꣳ॒॒श॒ति॒रा॒त्र इति॑ चतुर्विꣳशति-रा॒त्रः । भ॒व॒ति॒ । चतु॑र्विꣳशति॒रिति॒ चतुः॑-विꣳ॒॒श॒तिः॒ । अ॒द्‌र्ध॒मा॒सा इत्य॑द्‌र्ध - मा॒साः । सं॒ॅव॒थ्स॒र इति॑ सं - व॒थ्स॒रः ( ) । सं॒ॅव॒थ्स॒र इति॑ सं - व॒थ्स॒रः । सु॒व॒र्ग इति॑ सुवः - गः । लो॒कः । सं॒ॅव॒थ्स॒र इति॑ सं-व॒थ्स॒रे । ए॒व । सु॒व॒र्ग इति॑ सुवः - गे । लो॒के । प्रतीति॑ । ति॒ष्ठ॒न्ति॒ । अथो॒ इति॑ । चतु॑र्विꣳशत्यक्ष॒रेति॒ चत॑र्विꣳशति- अ॒क्ष॒रा॒ । गा॒य॒त्री । गा॒य॒त्री । ब्र॒ह्म॒व॒र्च॒समिति॑ ब्रह्म - व॒र्च॒सम् । गा॒य॒त्रि॒या । ए॒व । ब्र॒ह्म॒व॒र्च॒समिति॑ ब्रह्म - व॒र्च॒सम् । अवेति॑ । रु॒न्ध॒ते॒ । अ॒ति॒रा॒त्रावित्य॑ति - रा॒त्रौ । अ॒भितः॑ । भ॒व॒तः॒ । ब्र॒ह्म॒व॒र्च॒स्येति॑ ब्रह्म - व॒र्च॒सस्य॑ । परि॑गृहीत्या॒ इति॒ परि॑ - गृ॒ही॒त्यै॒ ॥  \newline


\textbf{Krama Paata} \newline

सु॒व॒र्गम् ॅलो॒कम् । सु॒व॒र्गमिति॑ सुवः - गम् । लो॒कम् ॅय॑न्ति । य॒न्ति॒ परा᳚ञ्चः । परा᳚ञ्चो॒ वै । वा ए॒ते । ए॒ते सु॑व॒र्गम् । सु॒व॒र्गम् ॅलो॒कम् । सु॒व॒र्गमिति॑ सुवः - गम् । लो॒कम॒भ्यारो॑हन्ति । अ॒भ्यारो॑हन्ति॒ ये । अ॒भ्यारो॑ह॒न्तीत्य॑भि - आरो॑हन्ति । ये प॑रा॒चीना॑नि । प॒रा॒चीना॑नि पृ॒ष्ठानि॑ । पृ॒ष्ठान्यु॑प॒यन्ति॑ । उ॒प॒यन्ति॑ प्र॒त्यङ्‍ङ् । उ॒प॒यन्तीत्यु॑प - यन्ति॑ । प्र॒त्यङ्.‍ क्ष॑ड॒हः । ष॒ड॒हो भ॑वति । ष॒ड॒ह इति॑ षट् - अ॒हः । भ॒व॒ति॒ प्र॒त्यव॑रूढ्‍यै । प्र॒त्यव॑रूढ्‍या॒ अथो᳚ । प्र॒त्यव॑रूढ्‍या॒ इति॑ प्रति - अव॑रूढ्‍यै । अथो॒ प्रति॑ष्ठित्यै । अथो॒ इत्यथो᳚ । प्रति॑ष्ठित्या उ॒भयोः᳚ । प्रति॑ष्ठित्या॒ इति॒ प्रति॑ - स्थि॒त्यै॒ । उ॒भयो᳚र् लो॒कयोः᳚ । लो॒कयोर्॑. ऋ॒द्ध्वा । ऋ॒द्ध्वोत् । उत् ति॑ष्ठन्ति । ति॒ष्ठ॒न्ति॒ त्रि॒वृतः॑ । त्रि॒वृतोऽधि॑ । त्रि॒वृत॒ इति॑ त्रि - वृतः॑ । अधि॑ त्रि॒वृत᳚म् । त्रि॒वृत॒मुप॑ । त्रि॒वृत॒मिति॑ त्रि - वृत᳚म् । उप॑ यन्ति । य॒न्ति॒ स्तोमा॑नाम् । स्तोमा॑नाꣳ॒॒ सम्प॑त्यै । सम्प॑त्यै प्रभ॒वाय॑ । सम्प॑त्या॒ इति॒ सम् - प॒त्यै॒ । प्र॒भ॒वाय॒ ज्योतिः॑ । प्र॒भ॒वायेति॑ प्र - भ॒वाय॑ । ज्योति॑रग्निष्टो॒मः । अ॒ग्नि॒ष्टो॒मो भ॑वति । अ॒ग्नि॒ष्टो॒म इत्य॑ग्नि - स्तो॒मः । भ॒व॒त्य॒यम् । अ॒यम् ॅवाव । वाव सः । स क्षयः॑ । क्षयो॒ऽस्मात् । अ॒स्मादे॒व । ए॒व तेन॑ । तेन॒ क्षया᳚त् । क्षया॒न् न । न य॑न्ति । य॒न्ति॒ च॒तु॒र्विꣳ॒॒श॒ति॒रा॒त्रः । च॒तु॒र्विꣳ॒॒श॒ति॒रा॒त्रो भ॑वति । च॒तु॒र्विꣳ॒॒श॒ति॒रा॒त्र इति॑ चतुर्विꣳशति - रा॒त्रः । भ॒व॒ति॒ चतु॑र्विꣳशतिः । चतु॑र्विꣳशतिरर्द्धमा॒साः । चतु॑र्विꣳशति॒रिति॒ चतुः॑ - विꣳ॒॒श॒तिः॒ । अ॒र्द्ध॒मा॒साः स॑म्ॅवथ्स॒रः ( ) । अ॒र्द्ध॒मा॒सा इत्य॑र्द्ध - मा॒साः । स॒म्ॅव॒थ्स॒रः स॑म्ॅवथ्स॒रः । स॒म्ॅव॒थ्स॒र इति॑ सम् - व॒थ्स॒रः । स॒म्ॅव॒थ्स॒रः सु॑व॒र्गः । स॒म्ॅव॒थ्स॒र इति॑ सम् - व॒थ्स॒रः । सु॒व॒र्गो लो॒कः । सु॒व॒र्ग इति॑ सुवः - गः । लो॒कः स॑म्ॅवथ्स॒रे । स॒म्ॅव॒थ्स॒र ए॒व । स॒म्ॅव॒थ्स॒र इति॑ सम् - व॒थ्स॒रे । ए॒व सु॑व॒र्गे । सु॒व॒र्गे लो॒के । सु॒व॒र्ग इति॑ सुवः - गे । लो॒के प्रति॑ । प्रति॑ तिष्ठन्ति । ति॒ष्ठ॒न्त्यथो᳚ । अथो॒ चतु॑र्विꣳशत्यक्षरा । अथो॒ इत्यथो᳚ । चतु॑र्विꣳशत्यक्षरा गाय॒त्री । चतु॑र्विꣳशत्यक्ष॒रेति॒ चतु॑र्विꣳशति - अ॒क्ष॒रा॒ । गा॒य॒त्री गा॑य॒त्री । गा॒य॒त्री ब्र॑ह्मवर्च॒सम् । ब्र॒ह्म॒व॒र्च॒सम् गा॑यत्रि॒या । ब्र॒ह्म॒व॒र्च॒समिति॑ ब्रह्म - व॒र्च॒सम् । गा॒य॒त्रि॒यैव । ए॒व ब्र॑ह्मवर्च॒सम् । ब्र॒ह्म॒व॒र्च॒समव॑ । ब्र॒ह्म॒व॒र्च॒समिति॑ ब्रह्म - व॒र्च॒सम् । अव॑ रुन्धते । रु॒न्ध॒ते॒ऽति॒रा॒त्रौ । अ॒ति॒रा॒त्राव॒भितः॑ । अ॒ति॒रा॒त्रावित्य॑ति - रा॒त्रौ । अ॒भितो॑ भवतः । भ॒व॒तो॒ ब्र॒ह्म॒व॒र्च॒सस्य॑ । ब्र॒ह्म॒व॒र्च॒सस्य॒ परि॑गृहीत्यै । ब्र॒ह्म॒व॒र्च॒सस्येति॑ ब्रह्म - व॒र्च॒सस्य॑ । परि॑गृहीत्या॒ इति॒ परि॑ - गृ॒ही॒त्यै॒ । \newline

\textbf{Jatai Paata} \newline

1. सु॒व॒र्गम् ॅलो॒कम् ॅलो॒कꣳ सु॑व॒र्गꣳ सु॑व॒र्गम् ॅलो॒कम् । \newline
2. सु॒व॒र्गमिति॑ सुवः - गम् । \newline
3. लो॒कं ॅय॑न्ति यन्ति लो॒कम् ॅलो॒कं ॅय॑न्ति । \newline
4. य॒न्ति॒ परा᳚ञ्चः॒ परा᳚ञ्चो यन्ति यन्ति॒ परा᳚ञ्चः । \newline
5. परा᳚ञ्चो॒ वै वै परा᳚ञ्चः॒ परा᳚ञ्चो॒ वै । \newline
6. वा ए॒त ए॒ते वै वा ए॒ते । \newline
7. ए॒ते सु॑व॒र्गꣳ सु॑व॒र्ग मे॒त ए॒ते सु॑व॒र्गम् । \newline
8. सु॒व॒र्गम् ॅलो॒कम् ॅलो॒कꣳ सु॑व॒र्गꣳ सु॑व॒र्गम् ॅलो॒कम् । \newline
9. सु॒व॒र्गमिति॑ सुवः - गम् । \newline
10. लो॒क म॒भ्यारो॑ह न्त्य॒भ्यारो॑हन्ति लो॒कम् ॅलो॒क म॒भ्यारो॑हन्ति । \newline
11. अ॒भ्यारो॑हन्ति॒ ये ये᳚ ऽभ्यारो॑ह न्त्य॒भ्यारो॑हन्ति॒ ये । \newline
12. अ॒भ्यारो॑ह॒न्तीत्य॑भि - आरो॑हन्ति । \newline
13. ये प॑रा॒चीना॑नि परा॒चीना॑नि॒ ये ये प॑रा॒चीना॑नि । \newline
14. प॒रा॒चीना॑नि पृ॒ष्ठानि॑ पृ॒ष्ठानि॑ परा॒चीना॑नि परा॒चीना॑नि पृ॒ष्ठानि॑ । \newline
15. पृ॒ष्ठा न्यु॑प॒यन् त्यु॑प॒यन्ति॑ पृ॒ष्ठानि॑ पृ॒ष्ठा न्यु॑प॒यन्ति॑ । \newline
16. उ॒प॒यन्ति॑ प्र॒त्यङ् प्र॒त्यङ् ङु॑प॒य न्त्यु॑प॒यन्ति॑ प्र॒त्यङ् । \newline
17. उ॒प॒यन्तीत्यु॑प - यन्ति॑ । \newline
18. प्र॒त्यङ् ख्ष॑ड॒ह ष्ष॑ड॒हः प्र॒त्यङ् प्र॒त्यङ् ख्ष॑ड॒हः । \newline
19. ष॒ड॒हो भ॑वति भवति षड॒ह ष्ष॑ड॒हो भ॑वति । \newline
20. ष॒ड॒ह इति॑ षट् - अ॒हः । \newline
21. भ॒व॒ति॒ प्र॒त्यव॑रूढ्यै प्र॒त्यव॑रूढ्यै भवति भवति प्र॒त्यव॑रूढ्यै । \newline
22. प्र॒त्यव॑रूढ्या॒ अथो॒ अथो᳚ प्र॒त्यव॑रूढ्यै प्र॒त्यव॑रूढ्या॒ अथो᳚ । \newline
23. प्र॒त्यव॑रूढ्या॒ इति॑ प्रति - अव॑रूढ्यै । \newline
24. अथो॒ प्रति॑ष्ठित्यै॒ प्रति॑ष्ठित्या॒ अथो॒ अथो॒ प्रति॑ष्ठित्यै । \newline
25. अथो॒ इत्यथो᳚ । \newline
26. प्रति॑ष्ठित्या उ॒भयो॑ रु॒भयोः॒ प्रति॑ष्ठित्यै॒ प्रति॑ष्ठित्या उ॒भयोः᳚ । \newline
27. प्रति॑ष्ठित्या॒ इति॒ प्रति॑ - स्थि॒त्यै॒ । \newline
28. उ॒भयो᳚र् लो॒कयो᳚र् लो॒कयो॑ रु॒भयो॑ रु॒भयो᳚र् लो॒कयोः᳚ । \newline
29. लो॒कयोर्॑. ऋ॒द्ध्व र्‌द्ध्वा लो॒कयो᳚र् लो॒कयोर्॑. ऋ॒द्ध्वा । \newline
30. ऋ॒द्ध्वोदु दृ॒द्ध्व र्‌द्ध्वोत् । \newline
31. उत् ति॑ष्ठन्ति तिष्ठ॒ न्त्युदुत् ति॑ष्ठन्ति । \newline
32. ति॒ष्ठ॒न्ति॒ त्रि॒वृत॑ स्त्रि॒वृत॑ स्तिष्ठन्ति तिष्ठन्ति त्रि॒वृतः॑ । \newline
33. त्रि॒वृतो ऽध्यधि॑ त्रि॒वृत॑ स्त्रि॒वृतो ऽधि॑ । \newline
34. त्रि॒वृत॒ इति॑ त्रि - वृतः॑ । \newline
35. अधि॑ त्रि॒वृत॑म् त्रि॒वृत॒ मध्यधि॑ त्रि॒वृत᳚म् । \newline
36. त्रि॒वृत॒ मुपोप॑ त्रि॒वृत॑म् त्रि॒वृत॒ मुप॑ । \newline
37. त्रि॒वृत॒मिति॑ त्रि - वृत᳚म् । \newline
38. उप॑ यन्ति य॒न्त्युपोप॑ यन्ति । \newline
39. य॒न्ति॒ स्तोमा॑नाꣳ॒॒ स्तोमा॑नां ॅयन्ति यन्ति॒ स्तोमा॑नाम् । \newline
40. स्तोमा॑नाꣳ॒॒ संप॑त्त्यै॒ संप॑त्त्यै॒ स्तोमा॑नाꣳ॒॒ स्तोमा॑नाꣳ॒॒ संप॑त्त्यै । \newline
41. संप॑त्त्यै प्रभ॒वाय॑ प्रभ॒वाय॒ संप॑त्त्यै॒ संप॑त्त्यै प्रभ॒वाय॑ । \newline
42. संप॑त्त्या॒ इति॒ सं - प॒त्त्यै॒ । \newline
43. प्र॒भ॒वाय॒ ज्योति॒र् ज्योतिः॑ प्रभ॒वाय॑ प्रभ॒वाय॒ ज्योतिः॑ । \newline
44. प्र॒भ॒वायेति॑ प्र - भ॒वाय॑ । \newline
45. ज्योति॑ रग्निष्टो॒मो᳚ ऽग्निष्टो॒मो ज्योति॒र् ज्योति॑ रग्निष्टो॒मः । \newline
46. अ॒ग्नि॒ष्टो॒मो भ॑वति भव त्यग्निष्टो॒मो᳚ ऽग्निष्टो॒मो भ॑वति । \newline
47. अ॒ग्नि॒ष्टो॒म इत्य॑ग्नि - स्तो॒मः । \newline
48. भ॒व॒ त्य॒य म॒यम् भ॑वति भव त्य॒यम् । \newline
49. अ॒यं ॅवाव वावाय म॒यं ॅवाव । \newline
50. वाव स स वाव वाव सः । \newline
51. स क्षयः॒ क्षयः॒ स स क्षयः॑ । \newline
52. क्षयो॒ ऽस्मा द॒स्मात् क्षयः॒ क्षयो॒ ऽस्मात् । \newline
53. अ॒स्मा दे॒वै वास्मा द॒स्मा दे॒व । \newline
54. ए॒व तेन॒ तेनै॒वैव तेन॑ । \newline
55. तेन॒ क्षया॒त् क्षया॒त् तेन॒ तेन॒ क्षया᳚त् । \newline
56. क्षया॒न् न न क्षया॒त् क्षया॒न् न । \newline
57. न य॑न्ति यन्ति॒ न न य॑न्ति । \newline
58. य॒न्ति॒ च॒तु॒र्विꣳ॒॒श॒ति॒रा॒त्र श्च॑तुर्विꣳशतिरा॒त्रो य॑न्ति यन्ति चतुर्विꣳशतिरा॒त्रः । \newline
59. च॒तु॒र्विꣳ॒॒श॒ति॒रा॒त्रो भ॑वति भवति चतुर्विꣳशतिरा॒त्र श्च॑तुर्विꣳशतिरा॒त्रो भ॑वति । \newline
60. च॒तु॒र्विꣳ॒॒श॒ति॒रा॒त्र इति॑ चतुर्विꣳशति - रा॒त्रः । \newline
61. भ॒व॒ति॒ चतु॑र्विꣳशति॒ श्चतु॑र्विꣳशतिर् भवति भवति॒ चतु॑र्विꣳशतिः । \newline
62. चतु॑र्विꣳशति रर्द्धमा॒सा अ॑र्द्धमा॒सा श्चतु॑र्विꣳशति॒ श्चतु॑र्विꣳशति रर्द्धमा॒साः । \newline
63. चतु॑र्विꣳशति॒रिति॒ चतुः॑ - विꣳ॒॒श॒तिः॒ । \newline
64. अ॒र्द्ध॒मा॒साः सं॑ॅवथ्स॒रः सं॑ॅवथ्स॒रो᳚ ऽर्द्धमा॒सा अ॑र्द्धमा॒साः सं॑ॅवथ्स॒रः । \newline
65. अ॒र्द्ध॒मा॒सा इत्य॑र्द्ध - मा॒साः । \newline
66. सं॒ॅव॒थ्स॒रः सं॑ॅवथ्स॒रः । \newline
67. सं॒ॅव॒थ्स॒र इति॑ सं - व॒थ्स॒रः । \newline
68. सं॒ॅव॒थ्स॒रः सु॑व॒र्गः सु॑व॒र्गः सं॑ॅवथ्स॒रः सं॑ॅवथ्स॒रः सु॑व॒र्गः । \newline
69. सं॒ॅव॒थ्स॒र इति॑ सं - व॒थ्स॒रः । \newline
70. सु॒व॒र्गो लो॒को लो॒कः सु॑व॒र्गः सु॑व॒र्गो लो॒कः । \newline
71. सु॒व॒र्ग इति॑ सुवः - गः । \newline
72. लो॒कः सं॑ॅवथ्स॒रे सं॑ॅवथ्स॒रे लो॒को लो॒कः सं॑ॅवथ्स॒रे । \newline
73. सं॒ॅव॒थ्स॒र ए॒वैव सं॑ॅवथ्स॒रे सं॑ॅवथ्स॒र ए॒व । \newline
74. सं॒ॅव॒थ्स॒र इति॑ सं - व॒थ्स॒रे । \newline
75. ए॒व सु॑व॒र्गे सु॑व॒र्ग ए॒वैव सु॑व॒र्गे । \newline
76. सु॒व॒र्गे लो॒के लो॒के सु॑व॒र्गे सु॑व॒र्गे लो॒के । \newline
77. सु॒व॒र्ग इति॑ सुवः - गे । \newline
78. लो॒के प्रति॒ प्रति॑ लो॒के लो॒के प्रति॑ । \newline
79. प्रति॑ तिष्ठन्ति तिष्ठन्ति॒ प्रति॒ प्रति॑ तिष्ठन्ति । \newline
80. ति॒ष्ठ॒न्त्यथो॒ अथो॑ तिष्ठन्ति तिष्ठ॒न्त्यथो᳚ । \newline
81. अथो॒ चतु॑र्विꣳशत्यक्षरा॒ चतु॑र्विꣳशत्यक्ष॒रा ऽथो॒ अथो॒ चतु॑र्विꣳशत्यक्षरा । \newline
82. अथो॒ इत्यथो᳚ । \newline
83. चतु॑र्विꣳशत्यक्षरा गाय॒त्री गा॑य॒त्री चतु॑र्विꣳशत्यक्षरा॒ चतु॑र्विꣳशत्यक्षरा गाय॒त्री । \newline
84. चतु॑र्विꣳशत्यक्ष॒रेति॒ चतु॑र्विꣳशति - अ॒क्ष॒रा॒ । \newline
85. गा॒य॒त्री गा॑य॒त्री । \newline
86. गा॒य॒त्री ब्र॑ह्मवर्च॒सम् ब्र॑ह्मवर्च॒सम् गा॑य॒त्री गा॑य॒त्री ब्र॑ह्मवर्च॒सम् । \newline
87. ब्र॒ह्म॒व॒र्च॒सम् गा॑यत्रि॒या गा॑यत्रि॒या ब्र॑ह्मवर्च॒सम् ब्र॑ह्मवर्च॒सम् गा॑यत्रि॒या । \newline
88. ब्र॒ह्म॒व॒र्च॒समिति॑ ब्रह्म - व॒र्च॒सम् । \newline
89. गा॒य॒त्रि॒ यैवैव गा॑यत्रि॒या गा॑यत्रि॒ यैव । \newline
90. ए॒व ब्र॑ह्मवर्च॒सम् ब्र॑ह्मवर्च॒स मे॒वैव ब्र॑ह्मवर्च॒सम् । \newline
91. ब्र॒ह्म॒व॒र्च॒स मवाव॑ ब्रह्मवर्च॒सम् ब्र॑ह्मवर्च॒स मव॑ । \newline
92. ब्र॒ह्म॒व॒र्च॒समिति॑ ब्रह्म - व॒र्च॒सम् । \newline
93. अव॑ रुन्धते रुन्ध॒ते ऽवाव॑ रुन्धते । \newline
94. रु॒न्ध॒ते॒ ऽति॒रा॒त्रा व॑तिरा॒त्रौ रु॑न्धते रुन्धते ऽतिरा॒त्रौ । \newline
95. अ॒ति॒रा॒त्रा व॒भितो॒ ऽभितो॑ ऽतिरा॒त्रा व॑तिरा॒त्रा व॒भितः॑ । \newline
96. अ॒ति॒रा॒त्रावित्य॑ति - रा॒त्रौ । \newline
97. अ॒भितो॑ भवतो भवतो॒ ऽभितो॒ ऽभितो॑ भवतः । \newline
98. भ॒व॒तो॒ ब्र॒ह्म॒व॒र्च॒सस्य॑ ब्रह्मवर्च॒सस्य॑ भवतो भवतो ब्रह्मवर्च॒सस्य॑ । \newline
99. ब्र॒ह्म॒व॒र्च॒सस्य॒ परि॑गृहीत्यै॒ परि॑गृहीत्यै ब्रह्मवर्च॒सस्य॑ ब्रह्मवर्च॒सस्य॒ परि॑गृहीत्यै । \newline
100. ब्र॒ह्म॒व॒र्च॒स्येति॑ ब्रह्म - व॒र्च॒सस्य॑ । \newline
101. परि॑गृहीत्या॒ इति॒ परि॑ - गृ॒ही॒त्यै॒ । \newline

\textbf{Ghana Paata } \newline

1. सु॒व॒र्गम् ॅलो॒कम् ॅलो॒कꣳ सु॑व॒र्गꣳ सु॑व॒र्गम् ॅलो॒कं ॅय॑न्ति यन्ति लो॒कꣳ सु॑व॒र्गꣳ सु॑व॒र्गम् ॅलो॒कं ॅय॑न्ति । \newline
2. सु॒व॒र्गमिति॑ सुवः - गम् । \newline
3. लो॒कं ॅय॑न्ति यन्ति लो॒कम् ॅलो॒कं ॅय॑न्ति॒ परा᳚ञ्चः॒ परा᳚ञ्चो यन्ति लो॒कम् ॅलो॒कं ॅय॑न्ति॒ परा᳚ञ्चः । \newline
4. य॒न्ति॒ परा᳚ञ्चः॒ परा᳚ञ्चो यन्ति यन्ति॒ परा᳚ञ्चो॒ वै वै परा᳚ञ्चो यन्ति यन्ति॒ परा᳚ञ्चो॒ वै । \newline
5. परा᳚ञ्चो॒ वै वै परा᳚ञ्चः॒ परा᳚ञ्चो॒ वा ए॒त ए॒ते वै परा᳚ञ्चः॒ परा᳚ञ्चो॒ वा ए॒ते । \newline
6. वा ए॒त ए॒ते वै वा ए॒ते सु॑व॒र्गꣳ सु॑व॒र्ग मे॒ते वै वा ए॒ते सु॑व॒र्गम् । \newline
7. ए॒ते सु॑व॒र्गꣳ सु॑व॒र्ग मे॒त ए॒ते सु॑व॒र्गम् ॅलो॒कम् ॅलो॒कꣳ सु॑व॒र्ग मे॒त ए॒ते सु॑व॒र्गम् ॅलो॒कम् । \newline
8. सु॒व॒र्गम् ॅलो॒कम् ॅलो॒कꣳ सु॑व॒र्गꣳ सु॑व॒र्गम् ॅलो॒क म॒भ्यारो॑ह न्त्य॒भ्यारो॑हन्ति लो॒कꣳ सु॑व॒र्गꣳ सु॑व॒र्गम् ॅलो॒क म॒भ्यारो॑हन्ति । \newline
9. सु॒व॒र्गमिति॑ सुवः - गम् । \newline
10. लो॒क म॒भ्यारो॑ह न्त्य॒भ्यारो॑हन्ति लो॒कम् ॅलो॒क म॒भ्यारो॑हन्ति॒ ये ये᳚ ऽभ्यारो॑हन्ति लो॒कम् ॅलो॒क म॒भ्यारो॑हन्ति॒ ये । \newline
11. अ॒भ्यारो॑हन्ति॒ ये ये᳚ ऽभ्यारो॑ह न्त्य॒भ्यारो॑हन्ति॒ ये प॑रा॒चीना॑नि परा॒चीना॑नि॒ ये᳚ ऽभ्यारो॑ह न्त्य॒भ्यारो॑हन्ति॒ ये प॑रा॒चीना॑नि । \newline
12. अ॒भ्यारो॑ह॒न्तीत्य॑भि - आरो॑हन्ति । \newline
13. ये प॑रा॒चीना॑नि परा॒चीना॑नि॒ ये ये प॑रा॒चीना॑नि पृ॒ष्ठानि॑ पृ॒ष्ठानि॑ परा॒चीना॑नि॒ ये ये प॑रा॒चीना॑नि पृ॒ष्ठानि॑ । \newline
14. प॒रा॒चीना॑नि पृ॒ष्ठानि॑ पृ॒ष्ठानि॑ परा॒चीना॑नि परा॒चीना॑नि पृ॒ष्ठा न्यु॑प॒य न्त्यु॑प॒यन्ति॑ पृ॒ष्ठानि॑ परा॒चीना॑नि परा॒चीना॑नि पृ॒ष्ठा न्यु॑प॒यन्ति॑ । \newline
15. पृ॒ष्ठा न्यु॑प॒य न्त्यु॑प॒यन्ति॑ पृ॒ष्ठानि॑ पृ॒ष्ठा न्यु॑प॒यन्ति॑ प्र॒त्यङ् प्र॒त्यङ् ङु॑प॒यन्ति॑ पृ॒ष्ठानि॑ पृ॒ष्ठा न्यु॑प॒यन्ति॑ प्र॒त्यङ् । \newline
16. उ॒प॒यन्ति॑ प्र॒त्यङ् प्र॒त्यङ् ङु॑प॒य न्त्यु॑प॒यन्ति॑ प्र॒त्यङ् ख्ष॑ड॒ह ष्ष॑ड॒हः प्र॒त्यङ् ङु॑प॒य न्त्यु॑प॒यन्ति॑ प्र॒त्यङ् ख्ष॑ड॒हः । \newline
17. उ॒प॒यन्तीत्यु॑प - यन्ति॑ । \newline
18. प्र॒त्यङ् ख्ष॑ड॒ह ष्ष॑ड॒हः प्र॒त्यङ् प्र॒त्यङ् ख्ष॑ड॒हो भ॑वति भवति षड॒हः प्र॒त्यङ् प्र॒त्यङ् ख्ष॑ड॒हो भ॑वति । \newline
19. ष॒ड॒हो भ॑वति भवति षड॒ह ष्ष॑ड॒हो भ॑वति प्र॒त्यव॑रूढ्यै प्र॒त्यव॑रूढ्यै भवति षड॒ह ष्ष॑ड॒हो भ॑वति प्र॒त्यव॑रूढ्यै । \newline
20. ष॒ड॒ह इति॑ षट् - अ॒हः । \newline
21. भ॒व॒ति॒ प्र॒त्यव॑रूढ्यै प्र॒त्यव॑रूढ्यै भवति भवति प्र॒त्यव॑रूढ्या॒ अथो॒ अथो᳚ प्र॒त्यव॑रूढ्यै भवति भवति प्र॒त्यव॑रूढ्या॒ अथो᳚ । \newline
22. प्र॒त्यव॑रूढ्या॒ अथो॒ अथो᳚ प्र॒त्यव॑रूढ्यै प्र॒त्यव॑रूढ्या॒ अथो॒ प्रति॑ष्ठित्यै॒ प्रति॑ष्ठित्या॒ अथो᳚ प्र॒त्यव॑रूढ्यै प्र॒त्यव॑रूढ्या॒ अथो॒ प्रति॑ष्ठित्यै । \newline
23. प्र॒त्यव॑रूढ्या॒ इति॑ प्रति - अव॑रूढ्यै । \newline
24. अथो॒ प्रति॑ष्ठित्यै॒ प्रति॑ष्ठित्या॒ अथो॒ अथो॒ प्रति॑ष्ठित्या उ॒भयो॑ रु॒भयोः॒ प्रति॑ष्ठित्या॒ अथो॒ अथो॒ प्रति॑ष्ठित्या उ॒भयोः᳚ । \newline
25. अथो॒ इत्यथो᳚ । \newline
26. प्रति॑ष्ठित्या उ॒भयो॑ रु॒भयोः॒ प्रति॑ष्ठित्यै॒ प्रति॑ष्ठित्या उ॒भयो᳚र् लो॒कयो᳚र् लो॒कयो॑ रु॒भयोः॒ प्रति॑ष्ठित्यै॒ प्रति॑ष्ठित्या उ॒भयो᳚र् लो॒कयोः᳚ । \newline
27. प्रति॑ष्ठित्या॒ इति॒ प्रति॑ - स्थि॒त्यै॒ । \newline
28. उ॒भयो᳚र् लो॒कयो᳚र् लो॒कयो॑ रु॒भयो॑ रु॒भयो᳚र् लो॒कयोर्॑. ऋ॒द्ध्व र्‌द्ध्वा लो॒कयो॑ रु॒भयो॑ रु॒भयो᳚र् लो॒कयोर्॑. ऋ॒द्ध्वा । \newline
29. लो॒कयोर्॑. ऋ॒द्ध्व र्‌द्ध्वा लो॒कयो᳚र् लो॒कयोर्॑. ऋ॒द्ध्वो दुदृ॒द्ध्वा लो॒कयो᳚र् लो॒कयोर्॑. ऋ॒द्ध्वोत् । \newline
30. ऋ॒द्ध्वो दुदृ॒द्ध्व र्‌द्ध्वोत् ति॑ष्ठन्ति तिष्ठ॒ न्त्युदृ॒द्ध्व र्‌द्ध्वोत् ति॑ष्ठन्ति । \newline
31. उत् ति॑ष्ठन्ति तिष्ठ॒ न्त्युदुत् ति॑ष्ठन्ति त्रि॒वृत॑ स्त्रि॒वृत॑ स्तिष्ठ॒ न्त्युदुत् ति॑ष्ठन्ति त्रि॒वृतः॑ । \newline
32. ति॒ष्ठ॒न्ति॒ त्रि॒वृत॑ स्त्रि॒वृत॑ स्तिष्ठन्ति तिष्ठन्ति त्रि॒वृतो ऽध्यधि॑ त्रि॒वृत॑ स्तिष्ठन्ति तिष्ठन्ति त्रि॒वृतो ऽधि॑ । \newline
33. त्रि॒वृतो ऽध्यधि॑ त्रि॒वृत॑ स्त्रि॒वृतो ऽधि॑ त्रि॒वृत॑म् त्रि॒वृत॒ मधि॑ त्रि॒वृत॑ स्त्रि॒वृतो ऽधि॑ त्रि॒वृत᳚म् । \newline
34. त्रि॒वृत॒ इति॑ त्रि - वृतः॑ । \newline
35. अधि॑ त्रि॒वृत॑म् त्रि॒वृत॒ मध्यधि॑ त्रि॒वृत॒ मुपोप॑ त्रि॒वृत॒ मध्यधि॑ त्रि॒वृत॒ मुप॑ । \newline
36. त्रि॒वृत॒ मुपोप॑ त्रि॒वृत॑म् त्रि॒वृत॒ मुप॑ यन्ति य॒न्त्युप॑ त्रि॒वृत॑म् त्रि॒वृत॒ मुप॑ यन्ति । \newline
37. त्रि॒वृत॒मिति॑ त्रि - वृत᳚म् । \newline
38. उप॑ यन्ति य॒न्त्युपोप॑ यन्ति॒ स्तोमा॑नाꣳ॒॒ स्तोमा॑नां ॅय॒न्त्युपोप॑ यन्ति॒ स्तोमा॑नाम् । \newline
39. य॒न्ति॒ स्तोमा॑नाꣳ॒॒ स्तोमा॑नां ॅयन्ति यन्ति॒ स्तोमा॑नाꣳ॒॒ संप॑त्त्यै॒ संप॑त्त्यै॒ स्तोमा॑नां ॅयन्ति यन्ति॒ स्तोमा॑नाꣳ॒॒ संप॑त्त्यै । \newline
40. स्तोमा॑नाꣳ॒॒ संप॑त्त्यै॒ संप॑त्त्यै॒ स्तोमा॑नाꣳ॒॒ स्तोमा॑नाꣳ॒॒ संप॑त्त्यै प्रभ॒वाय॑ प्रभ॒वाय॒ संप॑त्त्यै॒ स्तोमा॑नाꣳ॒॒ स्तोमा॑नाꣳ॒॒ संप॑त्त्यै प्रभ॒वाय॑ । \newline
41. संप॑त्त्यै प्रभ॒वाय॑ प्रभ॒वाय॒ संप॑त्त्यै॒ संप॑त्त्यै प्रभ॒वाय॒ ज्योति॒र् ज्योतिः॑ प्रभ॒वाय॒ संप॑त्त्यै॒ संप॑त्त्यै प्रभ॒वाय॒ ज्योतिः॑ । \newline
42. संप॑त्त्या॒ इति॒ सं - प॒त्त्यै॒ । \newline
43. प्र॒भ॒वाय॒ ज्योति॒र् ज्योतिः॑ प्रभ॒वाय॑ प्रभ॒वाय॒ ज्योति॑ रग्निष्टो॒मो᳚ ऽग्निष्टो॒मो ज्योतिः॑ प्रभ॒वाय॑ प्रभ॒वाय॒ ज्योति॑ रग्निष्टो॒मः । \newline
44. प्र॒भ॒वायेति॑ प्र - भ॒वाय॑ । \newline
45. ज्योति॑ रग्निष्टो॒मो᳚ ऽग्निष्टो॒मो ज्योति॒र् ज्योति॑ रग्निष्टो॒मो भ॑वति भव त्यग्निष्टो॒मो ज्योति॒र् ज्योति॑ रग्निष्टो॒मो भ॑वति । \newline
46. अ॒ग्नि॒ष्टो॒मो भ॑वति भव त्यग्निष्टो॒मो᳚ ऽग्निष्टो॒मो भ॑व त्य॒य म॒यम् भ॑व त्यग्निष्टो॒मो᳚ ऽग्निष्टो॒मो भ॑व त्य॒यम् । \newline
47. अ॒ग्नि॒ष्टो॒म इत्य॑ग्नि - स्तो॒मः । \newline
48. भ॒व॒ त्य॒य म॒यम् भ॑वति भव त्य॒यं ॅवाव वावायम् भ॑वति भव त्य॒यं ॅवाव । \newline
49. अ॒यं ॅवाव वावाय म॒यं ॅवाव स स वावाय म॒यं ॅवाव सः । \newline
50. वाव स स वाव वाव स क्षयः॒ क्षयः॒ स वाव वाव स क्षयः॑ । \newline
51. स क्षयः॒ क्षयः॒ स स क्षयो॒ ऽस्मा द॒स्मात् क्षयः॒ स स क्षयो॒ ऽस्मात् । \newline
52. क्षयो॒ ऽस्मा द॒स्मात् क्षयः॒ क्षयो॒ ऽस्मा दे॒वै वास्मात् क्षयः॒ क्षयो॒ ऽस्मा दे॒व । \newline
53. अ॒स्मा दे॒वै वास्मा द॒स्मा दे॒व तेन॒ तेनै॒वास्मा द॒स्मा दे॒व तेन॑ । \newline
54. ए॒व तेन॒ तेनै॒वैव तेन॒ क्षया॒त् क्षया॒त् तेनै॒वैव तेन॒ क्षया᳚त् । \newline
55. तेन॒ क्षया॒त् क्षया॒त् तेन॒ तेन॒ क्षया॒न् न न क्षया॒त् तेन॒ तेन॒ क्षया॒न् न । \newline
56. क्षया॒न् न न क्षया॒त् क्षया॒न् न य॑न्ति यन्ति॒ न क्षया॒त् क्षया॒न् न य॑न्ति । \newline
57. न य॑न्ति यन्ति॒ न न य॑न्ति चतुर्विꣳशतिरा॒त्र श्च॑तुर्विꣳशतिरा॒त्रो य॑न्ति॒ न न य॑न्ति चतुर्विꣳशतिरा॒त्रः । \newline
58. य॒न्ति॒ च॒तु॒र्विꣳ॒॒श॒ति॒रा॒त्र श्च॑तुर्विꣳशतिरा॒त्रो य॑न्ति यन्ति चतुर्विꣳशतिरा॒त्रो भ॑वति भवति चतुर्विꣳशतिरा॒त्रो य॑न्ति यन्ति चतुर्विꣳशतिरा॒त्रो भ॑वति । \newline
59. च॒तु॒र्विꣳ॒॒श॒ति॒रा॒त्रो भ॑वति भवति चतुर्विꣳशतिरा॒त्र श्च॑तुर्विꣳशतिरा॒त्रो भ॑वति॒ चतु॑र्विꣳशति॒ श्चतु॑र्विꣳशतिर् भवति चतुर्विꣳशतिरा॒त्र श्च॑तुर्विꣳशतिरा॒त्रो भ॑वति॒ चतु॑र्विꣳशतिः । \newline
60. च॒तु॒र्विꣳ॒॒श॒ति॒रा॒त्र इति॑ चतुर्विꣳशति - रा॒त्रः । \newline
61. भ॒व॒ति॒ चतु॑र्विꣳशति॒ श्चतु॑र्विꣳशतिर् भवति भवति॒ चतु॑र्विꣳशति रर्द्धमा॒सा अ॑र्द्धमा॒सा श्चतु॑र्विꣳशतिर् भवति भवति॒ चतु॑र्विꣳशति रर्द्धमा॒साः । \newline
62. चतु॑र्विꣳशति रर्द्धमा॒सा अ॑र्द्धमा॒सा श्चतु॑र्विꣳशति॒ श्चतु॑र्विꣳशति रर्द्धमा॒साः सं॑ॅवथ्स॒रः सं॑ॅवथ्स॒रो᳚ ऽर्द्धमा॒सा श्चतु॑र्विꣳशति॒ श्चतु॑र्विꣳशति रर्द्धमा॒साः सं॑ॅवथ्स॒रः । \newline
63. चतु॑र्विꣳशति॒रिति॒ चतुः॑ - विꣳ॒॒श॒तिः॒ । \newline
64. अ॒र्द्ध॒मा॒साः सं॑ॅवथ्स॒रः सं॑ॅवथ्स॒रो᳚ ऽर्द्धमा॒सा अ॑र्द्धमा॒साः सं॑ॅवथ्स॒रः । \newline
65. अ॒र्द्ध॒मा॒सा इत्य॑र्द्ध - मा॒साः । \newline
66. सं॒ॅव॒थ्स॒रः सं॑ॅवथ्स॒रः । \newline
67. सं॒ॅव॒थ्स॒र इति॑ सं - व॒थ्स॒रः । \newline
68. सं॒ॅव॒थ्स॒रः सु॑व॒र्गः सु॑व॒र्गः सं॑ॅवथ्स॒रः सं॑ॅवथ्स॒रः सु॑व॒र्गो लो॒को लो॒कः सु॑व॒र्गः सं॑ॅवथ्स॒रः सं॑ॅवथ्स॒रः सु॑व॒र्गो लो॒कः । \newline
69. सं॒ॅव॒थ्स॒र इति॑ सं - व॒थ्स॒रः । \newline
70. सु॒व॒र्गो लो॒को लो॒कः सु॑व॒र्गः सु॑व॒र्गो लो॒कः सं॑ॅवथ्स॒रे सं॑ॅवथ्स॒रे लो॒कः सु॑व॒र्गः सु॑व॒र्गो लो॒कः सं॑ॅवथ्स॒रे । \newline
71. सु॒व॒र्ग इति॑ सुवः - गः । \newline
72. लो॒कः सं॑ॅवथ्स॒रे सं॑ॅवथ्स॒रे लो॒को लो॒कः सं॑ॅवथ्स॒र ए॒वैव सं॑ॅवथ्स॒रे लो॒को लो॒कः सं॑ॅवथ्स॒र ए॒व । \newline
73. सं॒ॅव॒थ्स॒र ए॒वैव सं॑ॅवथ्स॒रे सं॑ॅवथ्स॒र ए॒व सु॑व॒र्गे सु॑व॒र्ग ए॒व सं॑ॅवथ्स॒रे सं॑ॅवथ्स॒र ए॒व सु॑व॒र्गे । \newline
74. सं॒ॅव॒थ्स॒र इति॑ सं - व॒थ्स॒रे । \newline
75. ए॒व सु॑व॒र्गे सु॑व॒र्ग ए॒वैव सु॑व॒र्गे लो॒के लो॒के सु॑व॒र्ग ए॒वैव सु॑व॒र्गे लो॒के । \newline
76. सु॒व॒र्गे लो॒के लो॒के सु॑व॒र्गे सु॑व॒र्गे लो॒के प्रति॒ प्रति॑ लो॒के सु॑व॒र्गे सु॑व॒र्गे लो॒के प्रति॑ । \newline
77. सु॒व॒र्ग इति॑ सुवः - गे । \newline
78. लो॒के प्रति॒ प्रति॑ लो॒के लो॒के प्रति॑ तिष्ठन्ति तिष्ठन्ति॒ प्रति॑ लो॒के लो॒के प्रति॑ तिष्ठन्ति । \newline
79. प्रति॑ तिष्ठन्ति तिष्ठन्ति॒ प्रति॒ प्रति॑ तिष्ठ॒ न्त्यथो॒ अथो॑ तिष्ठन्ति॒ प्रति॒ प्रति॑ तिष्ठ॒ न्त्यथो᳚ । \newline
80. ति॒ष्ठ॒ न्त्यथो॒ अथो॑ तिष्ठन्ति तिष्ठ॒ न्त्यथो॒ चतु॑र्विꣳशत्यक्षरा॒ चतु॑र्विꣳशत्यक्ष॒रा ऽथो॑ तिष्ठन्ति तिष्ठ॒ न्त्यथो॒ चतु॑र्विꣳशत्यक्षरा । \newline
81. अथो॒ चतु॑र्विꣳशत्यक्षरा॒ चतु॑र्विꣳशत्यक्ष॒रा ऽथो॒ अथो॒ चतु॑र्विꣳशत्यक्षरा गाय॒त्री गा॑य॒त्री चतु॑र्विꣳशत्यक्ष॒रा ऽथो॒ अथो॒ चतु॑र्विꣳशत्यक्षरा गाय॒त्री । \newline
82. अथो॒ इत्यथो᳚ । \newline
83. चतु॑र्विꣳशत्यक्षरा गाय॒त्री गा॑य॒त्री चतु॑र्विꣳशत्यक्षरा॒ चतु॑र्विꣳशत्यक्षरा गाय॒त्री । \newline
84. चतु॑र्विꣳशत्यक्ष॒रेति॒ चतु॑र्विꣳशति - अ॒क्ष॒रा॒ । \newline
85. गा॒य॒त्री गा॑य॒त्री । \newline
86. गा॒य॒त्री ब्र॑ह्मवर्च॒सम् ब्र॑ह्मवर्च॒सम् गा॑य॒त्री गा॑य॒त्री ब्र॑ह्मवर्च॒सम् गा॑यत्रि॒या गा॑यत्रि॒या ब्र॑ह्मवर्च॒सम् गा॑य॒त्री गा॑य॒त्री ब्र॑ह्मवर्च॒सम् गा॑यत्रि॒या । \newline
87. ब्र॒ह्म॒व॒र्च॒सम् गा॑यत्रि॒या गा॑यत्रि॒या ब्र॑ह्मवर्च॒सम् ब्र॑ह्मवर्च॒सम् गा॑यत्रि॒यैवैव गा॑यत्रि॒या ब्र॑ह्मवर्च॒सम् ब्र॑ह्मवर्च॒सम् गा॑यत्रि॒यैव । \newline
88. ब्र॒ह्म॒व॒र्च॒समिति॑ ब्रह्म - व॒र्च॒सम् । \newline
89. गा॒य॒त्रि॒यैवैव गा॑यत्रि॒या गा॑यत्रि॒यैव ब्र॑ह्मवर्च॒सम् ब्र॑ह्मवर्च॒स मे॒व गा॑यत्रि॒या गा॑यत्रि॒यैव ब्र॑ह्मवर्च॒सम् । \newline
90. ए॒व ब्र॑ह्मवर्च॒सम् ब्र॑ह्मवर्च॒स मे॒वैव ब्र॑ह्मवर्च॒स मवाव॑ ब्रह्मवर्च॒स मे॒वैव ब्र॑ह्मवर्च॒स मव॑ । \newline
91. ब्र॒ह्म॒व॒र्च॒स मवाव॑ ब्रह्मवर्च॒सम् ब्र॑ह्मवर्च॒स मव॑ रुन्धते रुन्ध॒ते ऽव॑ ब्रह्मवर्च॒सम् ब्र॑ह्मवर्च॒स मव॑ रुन्धते । \newline
92. ब्र॒ह्म॒व॒र्च॒समिति॑ ब्रह्म - व॒र्च॒सम् । \newline
93. अव॑ रुन्धते रुन्ध॒ते ऽवाव॑ रुन्धते ऽतिरा॒त्रा व॑तिरा॒त्रौ रु॑न्ध॒ते ऽवाव॑ रुन्धते ऽतिरा॒त्रौ । \newline
94. रु॒न्ध॒ते॒ ऽति॒रा॒त्रा व॑तिरा॒त्रौ रु॑न्धते रुन्धते ऽतिरा॒त्रा व॒भितो॒ ऽभितो॑ ऽतिरा॒त्रौ रु॑न्धते रुन्धते ऽतिरा॒त्रा व॒भितः॑ । \newline
95. अ॒ति॒रा॒त्रा व॒भितो॒ ऽभितो॑ ऽतिरा॒त्रा व॑तिरा॒त्रा व॒भितो॑ भवतो भवतो॒ ऽभितो॑ ऽतिरा॒त्रा व॑तिरा॒त्रा व॒भितो॑ भवतः । \newline
96. अ॒ति॒रा॒त्रावित्य॑ति - रा॒त्रौ । \newline
97. अ॒भितो॑ भवतो भवतो॒ ऽभितो॒ ऽभितो॑ भवतो ब्रह्मवर्च॒सस्य॑ ब्रह्मवर्च॒सस्य॑ भवतो॒ ऽभितो॒ ऽभितो॑ भवतो ब्रह्मवर्च॒सस्य॑ । \newline
98. भ॒व॒तो॒ ब्र॒ह्म॒व॒र्च॒सस्य॑ ब्रह्मवर्च॒सस्य॑ भवतो भवतो ब्रह्मवर्च॒सस्य॒ परि॑गृहीत्यै॒ परि॑गृहीत्यै ब्रह्मवर्च॒सस्य॑ भवतो भवतो ब्रह्मवर्च॒सस्य॒ परि॑गृहीत्यै । \newline
99. ब्र॒ह्म॒व॒र्च॒सस्य॒ परि॑गृहीत्यै॒ परि॑गृहीत्यै ब्रह्मवर्च॒सस्य॑ ब्रह्मवर्च॒सस्य॒ परि॑गृहीत्यै । \newline
100. ब्र॒ह्म॒व॒र्च॒स्येति॑ ब्रह्म - व॒र्च॒सस्य॑ । \newline
101. परि॑गृहीत्या॒ इति॒ परि॑ - गृ॒ही॒त्यै॒ । \newline
\pagebreak
\markright{ TS 7.4.3.1  \hfill https://www.vedavms.in \hfill}

\section{ TS 7.4.3.1 }

\textbf{TS 7.4.3.1 } \newline
\textbf{Samhita Paata} \newline

ऋ॒क्षा वा इ॒यम॑लो॒मका॑ऽऽसी॒थ् साऽका॑मय॒तौष॑धीभि॒-र्वन॒स्पति॑भिः॒ प्र जा॑ये॒येति॒ सैतास्त्रिꣳ॒॒शतꣳ॒॒ रात्री॑रपश्य॒त् ततो॒ वा इ॒यमोष॑धीभि॒-र्वन॒स्पति॑भिः॒ प्राजा॑यत॒ ये प्र॒जाका॑माः प॒शुका॑माः॒ स्युस्त ए॒ता आ॑सीर॒न् प्रैव जा॑यन्ते प्र॒जया॑ प॒शुभि॑रि॒यं ॅवा अ॑क्षुद्ध्य॒थ् सैतां ॅवि॒राज॑मपश्य॒त् तामा॒त्मन् धि॒त्वाऽन्नाद्य॒मवा॑ रु॒न्धौष॑धी॒ - [  ] \newline

\textbf{Pada Paata} \newline

ऋ॒क्षा । वा । इ॒यम् । अ॒लो॒मका᳚ । आ॒सी॒त् । सा । अ॒का॒म॒य॒त॒ । ओष॑धीभि॒रित्योष॑धि-भिः॒ । वन॒स्पति॑भि॒रिति॒ वन॒स्पति॑-भिः॒ । प्रेति॑ । जा॒ये॒य॒ । इति॑ । सा । ए॒ताः । त्रिꣳ॒॒शत᳚म् । रात्रीः᳚ । अ॒प॒श्य॒त् । ततः॑ । वै । इ॒यम् । ओष॑धीभि॒रित्योष॑धि - भिः॒ । वन॒स्पति॑भि॒रिति॒ वन॒स्पति॑ -भिः॒ । प्रेति॑ । अ॒जा॒य॒त॒ । ये । प्र॒जाका॑मा॒ इति॑ प्र॒जा - का॒माः॒ । प॒शुका॑मा॒ इति॑ प॒शु - का॒माः॒ । स्युः । ते । ए॒ताः । आ॒सी॒र॒न्न् । प्रेति॑ । ए॒व । जा॒य॒न्ते॒ । प्र॒जयेति॑ प्र-जया᳚ । प॒शुभि॒रिति॑ प॒शु-भिः॒ । इ॒यम् । वै । अ॒क्षु॒द्ध्य॒त् । सा । ए॒ताम् । वि॒राज॒मिति॑ वि - राज᳚म् । अ॒प॒श्य॒त् । ताम् । आ॒त्मन्न् । धि॒त्वा । अ॒न्नाद्य॒मित्य॑न्न - अद्य᳚म् । अवेति॑ । अ॒रु॒न्ध॒ । ओष॑धीः ।  \newline


\textbf{Krama Paata} \newline

ऋ॒क्षा वै । वा इ॒यम् । इ॒यम॑लो॒मका᳚ । अ॒लो॒मका॑ऽऽसीत् । आ॒सी॒थ् सा । साऽका॑मयत । अ॒का॒म॒य॒तौष॑धीभिः । ओष॑धीभि॒र् वन॒स्पति॑भिः । ओष॑धीभि॒रित्योष॑धि - भिः॒ । वन॒स्पति॑भिः॒ प्र । वन॒स्पति॑भि॒रिति॒ वन॒स्पति॑ - भिः॒ । प्र जा॑येय । जा॒ये॒येति॑ । इति॒ सा । सैताः । ए॒तास्त्रिꣳ॒॒शत᳚म् । त्रिꣳ॒॒शतꣳ॒॒ रात्रीः᳚ । रात्री॑रपश्यत् । अ॒प॒श्य॒त् ततः॑ । ततो॒ वै । वा इ॒यम् । इ॒यमोष॑धीभिः । ओष॑धीभि॒र् वन॒स्पति॑भिः । ओष॑धीभि॒रित्योष॑धि - भिः॒ । वन॒स्पति॑भिः॒ प्र । वन॒स्पति॑भि॒रिति॒ वन॒स्पति॑ - भिः॒ । प्राजा॑यत । अ॒जा॒य॒त॒ ये । ये प्र॒जाका॑माः । प्र॒जाका॑माः प॒शुका॑माः । प्र॒जाका॑मा॒ इति॑ प्र॒जा - का॒माः॒ । प॒शुका॑माः॒ स्युः । प॒शुका॑मा॒ इति॑ प॒शु - का॒माः॒ । स्युस्ते । त ए॒ताः । ए॒ता आ॑सीरन्न् । आ॒सी॒र॒न् प्र । प्रैव । ए॒व जा॑यन्ते । जा॒य॒न्ते॒ प्र॒जया᳚ । प्र॒जया॑ प॒शुभिः॑ । प्र॒जयेति॑ प्र - जया᳚ । प॒शुभि॑रि॒यम् । प॒शुभि॒रिति॑ प॒शु - भिः॒ । इ॒यम् ॅवै । वा अ॑क्षुद्ध्यत् । अ॒क्षु॒द्ध्य॒थ् सा । सैताम् । ए॒ताम् ॅवि॒राज᳚म् । वि॒राज॑मपश्यत् । वि॒राज॒मिति॑ वि - राज᳚म् । अ॒प॒श्य॒त् ताम् । तामा॒त्मन्न् । आ॒त्मन् धि॒त्वा । धि॒त्वाऽन्नाद्य᳚म् । अ॒न्नाद्य॒मव॑ । अ॒न्नाद्य॒मित्य॑न्न - अद्य᳚म् । अवा॑रुन्ध । अ॒रु॒न्धौष॑धीः । ओष॑धी॒र् वन॒स्पतीन्॑ \newline

\textbf{Jatai Paata} \newline

1. ऋ॒क्षा वै वा ऋ॒क्ष र्क्षा वै । \newline
2. वा इ॒य मि॒यं ॅवै वा इ॒यम् । \newline
3. इ॒य म॑लो॒मका॑ ऽलो॒मके॒य मि॒य म॑लो॒मका᳚ । \newline
4. अ॒लो॒मका॑ ऽऽसी दासी दलो॒मका॑ ऽलो॒मका॑ ऽऽसीत् । \newline
5. आ॒सी॒थ् सा सा ऽऽसी॑ दासी॒थ् सा । \newline
6. सा ऽका॑मयता कामयत॒ सा सा ऽका॑मयत । \newline
7. अ॒का॒म॒य॒तौ ष॑धीभि॒ रोष॑धीभि रकामयता कामय॒तौ ष॑धीभिः । \newline
8. ओष॑धीभि॒र् वन॒स्पति॑भि॒र् वन॒स्पति॑भि॒ रोष॑धीभि॒ रोष॑धीभि॒र् वन॒स्पति॑भिः । \newline
9. ओष॑धीभि॒रित्योष॑धि - भिः॒ । \newline
10. वन॒स्पति॑भिः॒ प्र प्र वण॒स्पति॑भि॒र् वन॒स्पति॑भिः॒ प्र । \newline
11. वन॒स्पति॑भि॒रिति॒ वन॒स्पति॑ - भिः॒ । \newline
12. प्र जा॑येय जायेय॒ प्र प्र जा॑येय । \newline
13. जा॒ये॒येतीति॑ जायेय जाये॒येति॑ । \newline
14. इति॒ सा सेतीति॒ सा । \newline
15. सैता ए॒ताः सा सैताः । \newline
16. ए॒ता स्त्रिꣳ॒॒शत॑म् त्रिꣳ॒॒शत॑ मे॒ता ए॒ता स्त्रिꣳ॒॒शत᳚म् । \newline
17. त्रिꣳ॒॒शतꣳ॒॒ रात्री॒ रात्री᳚ स्त्रिꣳ॒॒शत॑म् त्रिꣳ॒॒शतꣳ॒॒ रात्रीः᳚ । \newline
18. रात्री॑ रपश्य दपश्य॒द् रात्री॒ रात्री॑ रपश्यत् । \newline
19. अ॒प॒श्य॒त् तत॒ स्ततो॑ ऽपश्य दपश्य॒त् ततः॑ । \newline
20. ततो॒ वै वै तत॒ स्ततो॒ वै । \newline
21. वा इ॒य मि॒यं ॅवै वा इ॒यम् । \newline
22. इ॒य मोष॑धीभि॒ रोष॑धीभि रि॒य मि॒य मोष॑धीभिः । \newline
23. ओष॑धीभि॒र् वन॒स्पति॑भि॒र् वन॒स्पति॑भि॒ रोष॑धीभि॒ रोष॑धीभि॒र् वन॒स्पति॑भिः । \newline
24. ओष॑धीभि॒रित्योष॑धि - भिः॒ । \newline
25. वन॒स्पति॑भिः॒ प्र प्र वण॒स्पति॑भि॒र् वन॒स्पति॑भिः॒ प्र । \newline
26. वन॒स्पति॑भि॒रिति॒ वन॒स्पति॑ - भिः॒ । \newline
27. प्रा जा॑यता जायत॒ प्र प्रा जा॑यत । \newline
28. अ॒जा॒य॒त॒ ये ये॑ ऽजायता जायत॒ ये । \newline
29. ये प्र॒जाका॑माः प्र॒जाका॑मा॒ ये ये प्र॒जाका॑माः । \newline
30. प्र॒जाका॑माः प॒शुका॑माः प॒शुका॑माः प्र॒जाका॑माः प्र॒जाका॑माः प॒शुका॑माः । \newline
31. प्र॒जाका॑मा॒ इति॑ प्र॒जा - का॒माः॒ । \newline
32. प॒शुका॑माः॒ स्युः स्युः प॒शुका॑माः प॒शुका॑माः॒ स्युः । \newline
33. प॒शुका॑मा॒ इति॑ प॒शु - का॒माः॒ । \newline
34. स्यु स्ते ते स्युः स्यु स्ते । \newline
35. त ए॒ता ए॒ता स्ते त ए॒ताः । \newline
36. ए॒ता आ॑सीरन् नासीरन् ने॒ता ए॒ता आ॑सीरन्न् । \newline
37. आ॒सी॒र॒न् प्र प्रा सी॑रन् नासीर॒न् प्र । \newline
38. प्रैवैव प्र प्रैव । \newline
39. ए॒व जा॑यन्ते जायन्त ए॒वैव जा॑यन्ते । \newline
40. जा॒य॒न्ते॒ प्र॒जया᳚ प्र॒जया॑ जायन्ते जायन्ते प्र॒जया᳚ । \newline
41. प्र॒जया॑ प॒शुभिः॑ प॒शुभिः॑ प्र॒जया᳚ प्र॒जया॑ प॒शुभिः॑ । \newline
42. प्र॒जयेति॑ प्र - जया᳚ । \newline
43. प॒शुभि॑ रि॒य मि॒यम् प॒शुभिः॑ प॒शुभि॑ रि॒यम् । \newline
44. प॒शुभि॒रिति॑ प॒शु - भिः॒ । \newline
45. इ॒यं ॅवै वा इ॒य मि॒यं ॅवै । \newline
46. वा अ॑क्षुद्ध्य दक्षुद्ध्य॒द् वै वा अ॑क्षुद्ध्यत् । \newline
47. अ॒क्षु॒द्ध्य॒थ् सा सा ऽक्षु॑द्ध्य दक्षुद्ध्य॒थ् सा । \newline
48. सैता मे॒ताꣳ सा सैताम् । \newline
49. ए॒तां ॅवि॒राजं॑ ॅवि॒राज॑ मे॒ता मे॒तां ॅवि॒राज᳚म् । \newline
50. वि॒राज॑ मपश्य दपश्यद् वि॒राजं॑ ॅवि॒राज॑ मपश्यत् । \newline
51. वि॒राज॒मिति॑ वि - राज᳚म् । \newline
52. अ॒प॒श्य॒त् ताम् ता म॑पश्य दपश्य॒त् ताम् । \newline
53. ता मा॒त्मन् ना॒त्मन् ताम् ता मा॒त्मन्न् । \newline
54. आ॒त्मन् धि॒त्वा धि॒त्वा ऽऽत्मन् ना॒त्मन् धि॒त्वा । \newline
55. धि॒त्वा ऽन्नाद्य॑ म॒न्नाद्य॑म् धि॒त्वा धि॒त्वा ऽन्नाद्य᳚म् । \newline
56. अ॒न्नाद्य॒ मवा वा॒न्नाद्य॑ म॒न्नाद्य॒ मव॑ । \newline
57. अ॒न्नाद्य॒मित्य॑न्न - अद्य᳚म् । \newline
58. अवा॑ रुन्धा रु॒न्धा वावा॑ रुन्ध । \newline
59. अ॒रु॒न्धौष॑धी॒ रोष॑धी ररुन्धा रु॒न्धौष॑धीः । \newline
60. ओष॑धी॒र् वन॒स्पती॒न्॒. वन॒स्पती॒ नोष॑धी॒ रोष॑धी॒र् वन॒स्पतीन्॑ । \newline

\textbf{Ghana Paata } \newline

1. ऋ॒क्षा वै वा ऋ॒क्ष र्‌क्षा वा इ॒य मि॒यं ॅवा ऋ॒क्ष र्‌क्षा वा इ॒यम् । \newline
2. वा इ॒य मि॒यं ॅवै वा इ॒य म॑लो॒मका॑ ऽलो॒मके॒यं ॅवै वा इ॒य म॑लो॒मका᳚ । \newline
3. इ॒य म॑लो॒मका॑ ऽलो॒मके॒य मि॒य म॑लो॒मका॑ ऽऽसी दासी दलो॒मके॒य मि॒य म॑लो॒मका॑ ऽऽसीत् । \newline
4. अ॒लो॒मका॑ ऽऽसी दासी दलो॒मका॑ ऽलो॒मका॑ ऽऽसी॒थ् सा सा ऽऽसी॑ दलो॒मका॑ ऽलो॒मका॑ ऽऽसी॒थ् सा । \newline
5. आ॒सी॒थ् सा सा ऽऽसी॑ दासी॒थ् सा ऽका॑मयता कामयत॒ सा ऽऽसी॑ दासी॒थ् सा ऽका॑मयत । \newline
6. सा ऽका॑मयता कामयत॒ सा सा ऽका॑मय॒ तौष॑धीभि॒ रोष॑धीभि रकामयत॒ सा सा ऽका॑मय॒ तौष॑धीभिः । \newline
7. अ॒का॒म॒य॒ तौष॑धीभि॒ रोष॑धीभि रकामयता कामय॒ तौष॑धीभि॒र् वन॒स्पति॑भि॒र् वन॒स्पति॑भि॒ रोष॑धीभि रकामयता कामय॒ तौष॑धीभि॒र् वन॒स्पति॑भिः । \newline
8. ओष॑धीभि॒र् वन॒स्पति॑भि॒र् वन॒स्पति॑भि॒ रोष॑धीभि॒ रोष॑धीभि॒र् वन॒स्पति॑भिः॒ प्र प्र वण॒स्पति॑भि॒ रोष॑धीभि॒ रोष॑धीभि॒र् वन॒स्पति॑भिः॒ प्र । \newline
9. ओष॑धीभि॒रित्योष॑धि - भिः॒ । \newline
10. वन॒स्पति॑भिः॒ प्र प्र वण॒स्पति॑भि॒र् वन॒स्पति॑भिः॒ प्र जा॑येय जायेय॒ प्र वण॒स्पति॑भि॒र् वन॒स्पति॑भिः॒ प्र जा॑येय । \newline
11. वन॒स्पति॑भि॒रिति॒ वन॒स्पति॑ - भिः॒ । \newline
12. प्र जा॑येय जायेय॒ प्र प्र जा॑ये॒येतीति॑ जायेय॒ प्र प्र जा॑ये॒येति॑ । \newline
13. जा॒ये॒येतीति॑ जायेय जाये॒येति॒ सा सेति॑ जायेय जाये॒येति॒ सा । \newline
14. इति॒ सा सेतीति॒ सैता ए॒ताः सेतीति॒ सैताः । \newline
15. सैता ए॒ताः सा सैता स्त्रिꣳ॒॒शत॑म् त्रिꣳ॒॒शत॑ मे॒ताः सा सैता स्त्रिꣳ॒॒शत᳚म् । \newline
16. ए॒ता स्त्रिꣳ॒॒शत॑म् त्रिꣳ॒॒शत॑ मे॒ता ए॒ता स्त्रिꣳ॒॒शतꣳ॒॒ रात्री॒ रात्री᳚ स्त्रिꣳ॒॒शत॑ मे॒ता ए॒ता स्त्रिꣳ॒॒शतꣳ॒॒ रात्रीः᳚ । \newline
17. त्रिꣳ॒॒शतꣳ॒॒ रात्री॒ रात्री᳚ स्त्रिꣳ॒॒शत॑म् त्रिꣳ॒॒शतꣳ॒॒ रात्री॑ रपश्य दपश्य॒द् रात्री᳚ स्त्रिꣳ॒॒शत॑म् त्रिꣳ॒॒शतꣳ॒॒ रात्री॑ रपश्यत् । \newline
18. रात्री॑ रपश्य दपश्य॒द् रात्री॒ रात्री॑ रपश्य॒त् तत॒ स्ततो॑ ऽपश्य॒द् रात्री॒ रात्री॑ रपश्य॒त् ततः॑ । \newline
19. अ॒प॒श्य॒त् तत॒ स्ततो॑ ऽपश्य दपश्य॒त् ततो॒ वै वै ततो॑ ऽपश्य दपश्य॒त् ततो॒ वै । \newline
20. ततो॒ वै वै तत॒ स्ततो॒ वा इ॒य मि॒यं ॅवै तत॒ स्ततो॒ वा इ॒यम् । \newline
21. वा इ॒य मि॒यं ॅवै वा इ॒य मोष॑धीभि॒ रोष॑धीभि रि॒यं ॅवै वा इ॒य मोष॑धीभिः । \newline
22. इ॒य मोष॑धीभि॒ रोष॑धीभि रि॒य मि॒य मोष॑धीभि॒र् वन॒स्पति॑भि॒र् वन॒स्पति॑भि॒ रोष॑धीभि रि॒य मि॒य मोष॑धीभि॒र् वन॒स्पति॑भिः । \newline
23. ओष॑धीभि॒र् वन॒स्पति॑भि॒र् वन॒स्पति॑भि॒ रोष॑धीभि॒ रोष॑धीभि॒र् वन॒स्पति॑भिः॒ प्र प्र वण॒स्पति॑भि॒ रोष॑धीभि॒ रोष॑धीभि॒र् वन॒स्पति॑भिः॒ प्र । \newline
24. ओष॑धीभि॒रित्योष॑धि - भिः॒ । \newline
25. वन॒स्पति॑भिः॒ प्र प्र वण॒स्पति॑भि॒र् वन॒स्पति॑भिः॒ प्राजा॑यता जायत॒ प्र वण॒स्पति॑भि॒र् वन॒स्पति॑भिः॒ प्राजा॑यत । \newline
26. वन॒स्पति॑भि॒रिति॒ वन॒स्पति॑ - भिः॒ । \newline
27. प्राजा॑यता जायत॒ प्र प्राजा॑यत॒ ये ये॑ ऽजायत॒ प्र प्राजा॑यत॒ ये । \newline
28. अ॒जा॒य॒त॒ ये ये॑ ऽजायता जायत॒ ये प्र॒जाका॑माः प्र॒जाका॑मा॒ ये॑ ऽजायता जायत॒ ये प्र॒जाका॑माः । \newline
29. ये प्र॒जाका॑माः प्र॒जाका॑मा॒ ये ये प्र॒जाका॑माः प॒शुका॑माः प॒शुका॑माः प्र॒जाका॑मा॒ ये ये प्र॒जाका॑माः प॒शुका॑माः । \newline
30. प्र॒जाका॑माः प॒शुका॑माः प॒शुका॑माः प्र॒जाका॑माः प्र॒जाका॑माः प॒शुका॑माः॒ स्युः स्युः प॒शुका॑माः प्र॒जाका॑माः प्र॒जाका॑माः प॒शुका॑माः॒ स्युः । \newline
31. प्र॒जाका॑मा॒ इति॑ प्र॒जा - का॒माः॒ । \newline
32. प॒शुका॑माः॒ स्युः स्युः प॒शुका॑माः प॒शुका॑माः॒ स्यु स्ते ते स्युः प॒शुका॑माः प॒शुका॑माः॒ स्यु स्ते । \newline
33. प॒शुका॑मा॒ इति॑ प॒शु - का॒माः॒ । \newline
34. स्यु स्ते ते स्युः स्यु स्त ए॒ता ए॒ता स्ते स्युः स्यु स्त ए॒ताः । \newline
35. त ए॒ता ए॒ता स्ते त ए॒ता आ॑सीरन् नासीरन् ने॒ता स्ते त ए॒ता आ॑सीरन्न् । \newline
36. ए॒ता आ॑सीरन् नासीरन् ने॒ता ए॒ता आ॑सीर॒न् प्र प्रासी॑रन् ने॒ता ए॒ता आ॑सीर॒न् प्र । \newline
37. आ॒सी॒र॒न् प्र प्रासी॑रन् नासीर॒न् प्रैवैव प्रासी॑रन् नासीर॒न् प्रैव । \newline
38. प्रैवैव प्र प्रैव जा॑यन्ते जायन्त ए॒व प्र प्रैव जा॑यन्ते । \newline
39. ए॒व जा॑यन्ते जायन्त ए॒वैव जा॑यन्ते प्र॒जया᳚ प्र॒जया॑ जायन्त ए॒वैव जा॑यन्ते प्र॒जया᳚ । \newline
40. जा॒य॒न्ते॒ प्र॒जया᳚ प्र॒जया॑ जायन्ते जायन्ते प्र॒जया॑ प॒शुभिः॑ प॒शुभिः॑ प्र॒जया॑ जायन्ते जायन्ते प्र॒जया॑ प॒शुभिः॑ । \newline
41. प्र॒जया॑ प॒शुभिः॑ प॒शुभिः॑ प्र॒जया᳚ प्र॒जया॑ प॒शुभि॑ रि॒य मि॒यम् प॒शुभिः॑ प्र॒जया᳚ प्र॒जया॑ प॒शुभि॑ रि॒यम् । \newline
42. प्र॒जयेति॑ प्र - जया᳚ । \newline
43. प॒शुभि॑ रि॒य मि॒यम् प॒शुभिः॑ प॒शुभि॑ रि॒यं ॅवै वा इ॒यम् प॒शुभिः॑ प॒शुभि॑ रि॒यं ॅवै । \newline
44. प॒शुभि॒रिति॑ प॒शु - भिः॒ । \newline
45. इ॒यं ॅवै वा इ॒य मि॒यं ॅवा अ॑क्षुद्ध्य दक्षुद्ध्य॒द् वा इ॒य मि॒यं ॅवा अ॑क्षुद्ध्यत् । \newline
46. वा अ॑क्षुद्ध्य दक्षुद्ध्य॒द् वै वा अ॑क्षुद्ध्य॒थ् सा सा ऽक्षु॑द्ध्य॒द् वै वा अ॑क्षुद्ध्य॒थ् सा । \newline
47. अ॒क्षु॒द्ध्य॒थ् सा सा ऽक्षु॑द्ध्य दक्षुद्ध्य॒थ् सैता मे॒ताꣳ सा ऽक्षु॑द्ध्य दक्षुद्ध्य॒थ् सैताम् । \newline
48. सैता मे॒ताꣳ सा सैतां ॅवि॒राजं॑ ॅवि॒राज॑ मे॒ताꣳ सा सैतां ॅवि॒राज᳚म् । \newline
49. ए॒तां ॅवि॒राजं॑ ॅवि॒राज॑ मे॒ता मे॒तां ॅवि॒राज॑ मपश्य दपश्यद् वि॒राज॑ मे॒ता मे॒तां ॅवि॒राज॑ मपश्यत् । \newline
50. वि॒राज॑ मपश्य दपश्यद् वि॒राजं॑ ॅवि॒राज॑ मपश्य॒त् ताम् ता म॑पश्यद् वि॒राजं॑ ॅवि॒राज॑ मपश्य॒त् ताम् । \newline
51. वि॒राज॒मिति॑ वि - राज᳚म् । \newline
52. अ॒प॒श्य॒त् ताम् ता म॑पश्य दपश्य॒त् ता मा॒त्मन् ना॒त्मन् ता म॑पश्य दपश्य॒त् ता मा॒त्मन्न् । \newline
53. ता मा॒त्मन् ना॒त्मन् ताम् ता मा॒त्मन् धि॒त्वा धि॒त्वा ऽऽत्मन् ताम् ता मा॒त्मन् धि॒त्वा । \newline
54. आ॒त्मन् धि॒त्वा धि॒त्वा ऽऽत्मन् ना॒त्मन् धि॒त्वा ऽन्नाद्य॑ म॒न्नाद्य॑म् धि॒त्वा ऽऽत्मन् ना॒त्मन् धि॒त्वा ऽन्नाद्य᳚म् । \newline
55. धि॒त्वा ऽन्नाद्य॑ म॒न्नाद्य॑म् धि॒त्वा धि॒त्वा ऽन्नाद्य॒ मवा वा॒न्नाद्य॑म् धि॒त्वा धि॒त्वा ऽन्नाद्य॒ मव॑ । \newline
56. अ॒न्नाद्य॒ मवा वा॒न्नाद्य॑ म॒न्नाद्य॒ मवा॑ रुन्धा रु॒न्धा वा॒न्नाद्य॑ म॒न्नाद्य॒ मवा॑रुन्ध । \newline
57. अ॒न्नाद्य॒मित्य॑न्न - अद्य᳚म् । \newline
58. अवा॑ रुन्धा रु॒न्धा वावा॑ रु॒न्धौष॑धी॒ रोष॑धी ररु॒न्धा वावा॑ रु॒न्धौष॑धीः । \newline
59. अ॒रु॒न्धौष॑धी॒ रोष॑धी ररुन्धा रु॒न्धौष॑धी॒र् वन॒स्पती॒न्॒. वन॒स्पती॒ नोष॑धी ररुन्धा 
रु॒न्धौष॑धी॒र् वन॒स्पतीन्॑ । \newline
60. ओष॑धी॒र् वन॒स्पती॒न्॒. वन॒स्पती॒ नोष॑धी॒ रोष॑धी॒र् वन॒स्पती᳚न् प्र॒जाम् प्र॒जां ॅवन॒स्पती॒ नोष॑धी॒ रोष॑धी॒र् वन॒स्पती᳚न् प्र॒जाम् । \newline
\pagebreak
\markright{ TS 7.4.3.2  \hfill https://www.vedavms.in \hfill}

\section{ TS 7.4.3.2 }

\textbf{TS 7.4.3.2 } \newline
\textbf{Samhita Paata} \newline

-र्वन॒स्पती᳚न् प्र॒जां प॒शून् तेना॑वर्द्धत॒ सा जे॒मानं॑ महि॒मान॑-मगच्छ॒द्य ए॒वं ॅवि॒द्वाꣳस॑ ए॒ता आस॑ते वि॒राज॑मे॒वाऽऽ*त्मन् धि॒त्वाऽन्नाद्य॒मव॑ रुन्धते॒ वर्द्ध॑न्ते प्र॒जया॑ प॒शुभि॑र्जे॒मानं॑ महि॒मानं॑ गच्छन्ति॒ ज्योति॑रतिरा॒त्रो भ॑वति सुव॒र्गस्य॑ लो॒कस्यानु॑ख्यात्यै॒ पृष्ठ्यः॑ षड॒हो भ॑वति॒ षड् वा ऋ॒तवः॒ षट् पृ॒ष्ठानि॑ पृ॒ष्ठैरे॒वर्तून॒न्वा-रो॑हन्त्यृ॒तुभिः॑ संॅवथ्स॒रं ते सं॑ॅवथ्स॒र ए॒व - [  ] \newline

\textbf{Pada Paata} \newline

वन॒स्पतीन्॑ । प्र॒जामिति॑ प्र - जाम् । प॒शून् । तेन॑ । अ॒व॒द्‌र्ध॒त॒ । सा । जे॒मान᳚म् । म॒हि॒मान᳚म् । अ॒ग॒च्छ॒त् । ये । ए॒वम् । वि॒द्वाꣳसः॑ । ए॒ताः । आस॑ते । वि॒राज॒मिति॑ वि-राज᳚म् । ए॒व । आ॒त्मन्न् । धि॒त्वा । अ॒न्नाद्य॒मित्य॑न्न - अद्य᳚म् । अवेति॑ । रु॒न्ध॒ते॒ । वद्‌र्ध॑न्ते । प्र॒जयेति॑ प्र - जया᳚ । प॒शुभि॒रिति॑ प॒शु - भिः॒ । जे॒मान᳚म् । म॒हि॒मान᳚म् । ग॒च्छ॒न्ति॒ । ज्योतिः॑ । अ॒ति॒रा॒त्र इत्य॑ति - रा॒त्रः । भ॒व॒ति॒ । सु॒व॒र्गस्येति॑ सुवः - गस्य॑ । लो॒कस्य॑ । अनु॑ख्यात्या॒ इत्यनु॑ - ख्या॒त्यै॒ । पृष्ठ्यः॑ । ष॒ड॒ह इति॑ षट् - अ॒हः । भ॒व॒ति॒ । षट् । वै । ऋ॒तवः॑ । षट् । पृ॒ष्ठानि॑ । पृ॒ष्ठैः । ए॒व । ऋ॒तून् । अ॒न्वारो॑ह॒न्तीत्य॑नु - आरो॑हन्ति । ऋ॒तुभि॒रित्यृ॒तु - भिः॒ । सं॒ॅव॒थ्स॒रमिति॑ सं - व॒थ्स॒रम् । ते । सं॒ॅव॒थ्स॒र इति॑ सं - व॒थ्स॒रे । ए॒व ।  \newline


\textbf{Krama Paata} \newline

वन॒स्पती᳚न् प्र॒जाम् । प्र॒जाम् प॒शून् । प्र॒जामिति॑ प्र - जाम् । प॒शून् तेन॑ । तेना॑वर्द्धत । अ॒व॒र्द्ध॒त॒ सा । सा जे॒मान᳚म् । जे॒मान॑म् महि॒मान᳚म् । म॒हि॒मान॑मगच्छत् । अ॒ग॒च्छ॒द् ये । य ए॒वम् । ए॒वम् ॅवि॒द्वाꣳसः॑ । वि॒द्वाꣳस॑ ए॒ताः । ए॒ता आस॑ते । आस॑ते वि॒राज᳚म् । वि॒राज॑मे॒व । वि॒राज॒मिति॑ वि - राज᳚म् । ए॒वात्मन्न् । आ॒त्मन् धि॒त्वा । धि॒त्वाऽन्नाद्य᳚म् । अ॒न्नाद्य॒मव॑ । अ॒न्नाद्य॒मित्य॑न्न - अद्य᳚म् । अव॑ रुन्धते । रु॒न्ध॒ते॒ वर्द्ध॑न्ते । वर्द्ध॑न्ते प्र॒जया᳚ । प्र॒जया॑ प॒शुभिः॑ । प्र॒जयेति॑ प्र - जया᳚ । प॒शुभि॑र् जे॒मान᳚म् । प॒शुभि॒रिति॑ प॒शु - भिः॒ । जे॒मान॑म् महि॒मान᳚म् । म॒हि॒मान॑म् गच्छन्ति । ग॒च्छ॒न्ति॒ ज्योतिः॑ । ज्योति॑रतिरा॒त्रः । अ॒ति॒रा॒त्रो भ॑वति । अ॒ति॒रा॒त्र इत्य॑ति - रा॒त्रः । भ॒व॒ति॒ सु॒व॒र्गस्य॑ । सु॒व॒र्गस्य॑ लो॒कस्य॑ । सु॒व॒र्गस्येति॑ सुवः - गस्य॑ । लो॒कस्यानु॑ख्यात्यै । अनु॑ख्यात्यै॒ पृष्ठ्‍यः॑ । अनु॑ख्यात्या॒ इत्यनु॑ - ख्या॒त्यै॒ । पृष्ठ्‍यः॑ षड॒हः । ष॒ड॒हो भ॑वति । ष॒ड॒ह इति॑ षट् - अ॒हः । भ॒व॒ति॒ षट् । षड् वै । वा ऋ॒तवः॑ । ऋ॒तवः॒ षट् । षट् पृ॒ष्ठानि॑ । पृ॒ष्ठानि॑ पृ॒ष्ठैः । पृ॒ष्ठैरे॒व । ए॒वर्तून् । ऋ॒तून॒न्वारो॑हन्ति । अ॒न्वारो॑हन्त्यृ॒तुभिः॑ । अ॒न्वारो॑ह॒न्तीत्य॑नु - आरो॑हन्ति । ऋ॒तुभिः॑ सम्ॅवथ्स॒रम् । ऋ॒तुभि॒रित्यृ॒तु - भिः॒ । स॒म्ॅव॒थ्स॒रम् ते । स॒म्ॅव॒थ्स॒रमिति॑ सम् - व॒थ्स॒रम् । ते स॑म्ॅवथ्स॒रे । स॒म्ॅव॒थ्स॒र ए॒व । स॒म्ॅव॒थ्स॒र इति॑ सम् - व॒थ्स॒रे । ए॒व प्रति॑ \newline

\textbf{Jatai Paata} \newline

1. वन॒स्पती᳚न् प्र॒जाम् प्र॒जां ॅवन॒स्पती॒न्॒. वन॒स्पती᳚न् प्र॒जाम् । \newline
2. प्र॒जाम् प॒शून् प॒शून् प्र॒जाम् प्र॒जाम् प॒शून् । \newline
3. प्र॒जामिति॑ प्र - जाम् । \newline
4. प॒शून् तेन॒ तेन॑ प॒शून् प॒शून् तेन॑ । \newline
5. तेना॑ वर्द्धता वर्द्धत॒ तेन॒ तेना॑ वर्द्धत । \newline
6. अ॒व॒र्द्ध॒त॒ सा सा ऽव॑र्द्धता वर्द्धत॒ सा । \newline
7. सा जे॒मान॑म् जे॒मानꣳ॒॒ सा सा जे॒मान᳚म् । \newline
8. जे॒मान॑म् महि॒मान॑म् महि॒मान॑म् जे॒मान॑म् जे॒मान॑म् महि॒मान᳚म् । \newline
9. म॒हि॒मान॑ मगच्छ दगच्छन् महि॒मान॑म् महि॒मान॑ मगच्छत् । \newline
10. अ॒ग॒च्छ॒द् ये ये॑ ऽगच्छ दगच्छ॒द् ये । \newline
11. य ए॒व मे॒वं ॅये य ए॒वम् । \newline
12. ए॒वं ॅवि॒द्वाꣳसो॑ वि॒द्वाꣳस॑ ए॒व मे॒वं ॅवि॒द्वाꣳसः॑ । \newline
13. वि॒द्वाꣳस॑ ए॒ता ए॒ता वि॒द्वाꣳसो॑ वि॒द्वाꣳस॑ ए॒ताः । \newline
14. ए॒ता आस॑त॒ आस॑त ए॒ता ए॒ता आस॑ते । \newline
15. आस॑ते वि॒राजं॑ ॅवि॒राज॒ मास॑त॒ आस॑ते वि॒राज᳚म् । \newline
16. वि॒राज॑ मे॒वैव वि॒राजं॑ ॅवि॒राज॑ मे॒व । \newline
17. वि॒राज॒मिति॑ वि - राज᳚म् । \newline
18. ए॒वात्मन् ना॒त्मन् ने॒वै वात्मन्न् । \newline
19. आ॒त्मन् धि॒त्वा धि॒त्वा ऽऽत्मन् ना॒त्मन् धि॒त्वा । \newline
20. धि॒त्वा ऽन्नाद्य॑ म॒न्नाद्य॑म् धि॒त्वा धि॒त्वा ऽन्नाद्य᳚म् । \newline
21. अ॒न्नाद्य॒ मवा वा॒न्नाद्य॑ म॒न्नाद्य॒ मव॑ । \newline
22. अ॒न्नाद्य॒मित्य॑न्न - अद्य᳚म् । \newline
23. अव॑ रुन्धते रुन्ध॒ते ऽवाव॑ रुन्धते । \newline
24. रु॒न्ध॒ते॒ वर्द्ध॑न्ते॒ वर्द्ध॑न्ते रुन्धते रुन्धते॒ वर्द्ध॑न्ते । \newline
25. वर्द्ध॑न्ते प्र॒जया᳚ प्र॒जया॒ वर्द्ध॑न्ते॒ वर्द्ध॑न्ते प्र॒जया᳚ । \newline
26. प्र॒जया॑ प॒शुभिः॑ प॒शुभिः॑ प्र॒जया᳚ प्र॒जया॑ प॒शुभिः॑ । \newline
27. प्र॒जयेति॑ प्र - जया᳚ । \newline
28. प॒शुभि॑र् जे॒मान॑म् जे॒मान॑म् प॒शुभिः॑ प॒शुभि॑र् जे॒मान᳚म् । \newline
29. प॒शुभि॒रिति॑ प॒शु - भिः॒ । \newline
30. जे॒मान॑म् महि॒मान॑म् महि॒मान॑म् जे॒मान॑म् जे॒मान॑म् महि॒मान᳚म् । \newline
31. म॒हि॒मान॑म् गच्छन्ति गच्छन्ति महि॒मान॑म् महि॒मान॑म् गच्छन्ति । \newline
32. ग॒च्छ॒न्ति॒ ज्योति॒र् ज्योति॑र् गच्छन्ति गच्छन्ति॒ ज्योतिः॑ । \newline
33. ज्योति॑ रतिरा॒त्रो॑ ऽतिरा॒त्रो ज्योति॒र् ज्योति॑ रतिरा॒त्रः । \newline
34. अ॒ति॒रा॒त्रो भ॑वति भव त्यतिरा॒त्रो॑ ऽतिरा॒त्रो भ॑वति । \newline
35. अ॒ति॒रा॒त्र इत्य॑ति - रा॒त्रः । \newline
36. भ॒व॒ति॒ सु॒व॒र्गस्य॑ सुव॒र्गस्य॑ भवति भवति सुव॒र्गस्य॑ । \newline
37. सु॒व॒र्गस्य॑ लो॒कस्य॑ लो॒कस्य॑ सुव॒र्गस्य॑ सुव॒र्गस्य॑ लो॒कस्य॑ । \newline
38. सु॒व॒र्गस्येति॑ सुवः - गस्य॑ । \newline
39. लो॒कस्या नु॑ख्यात्या॒ अनु॑ख्यात्यै लो॒कस्य॑ लो॒कस्या नु॑ख्यात्यै । \newline
40. अनु॑ख्यात्यै॒ पृष्ठ्यः॒ पृष्ठ्यो ऽनु॑ख्यात्या॒ अनु॑ख्यात्यै॒ पृष्ठ्यः॑ । \newline
41. अनु॑ख्यात्या॒ इत्यनु॑ - ख्या॒त्यै॒ । \newline
42. पृष्ठ्य॑ ष्षड॒ह ष्ष॑ड॒हः पृष्ठ्यः॒ पृष्ठ्य॑ ष्षड॒हः । \newline
43. ष॒ड॒हो भ॑वति भवति षड॒ह ष्ष॑ड॒हो भ॑वति । \newline
44. ष॒ड॒ह इति॑ षट् - अ॒हः । \newline
45. भ॒व॒ति॒ षट् थ्षड् भ॑वति भवति॒ षट् । \newline
46. षड् वै वै षट् थ्षड् वै । \newline
47. वा ऋ॒तव॑ ऋ॒तवो॒ वै वा ऋ॒तवः॑ । \newline
48. ऋ॒तव॒ ष्षट् थ्षडृ॒तव॑ ऋ॒तव॒ ष्षट् । \newline
49. षट् पृ॒ष्ठानि॑ पृ॒ष्ठानि॒ षट् थ्षट् पृ॒ष्ठानि॑ । \newline
50. पृ॒ष्ठानि॑ पृ॒ष्ठैः पृ॒ष्ठैः पृ॒ष्ठानि॑ पृ॒ष्ठानि॑ पृ॒ष्ठैः । \newline
51. पृ॒ष्ठै रे॒वैव पृ॒ष्ठैः पृ॒ष्ठै रे॒व । \newline
52. ए॒व र्‌तू नृ॒तूने॒वैव र्‌तून् । \newline
53. ऋ॒तू न॒भ्यारो॑ह न्त्य॒भ्यारो॑ह न्त्यृ॒तू नृ॒तून॒ भ्यारो॑हन्ति । \newline
54. अ॒भ्यारो॑ह न्त्यृ॒तुभिर्॑. ऋ॒तुभि॑ र॒भ्यारो॑ह न्त्य॒भ्यारो॑ह न्त्यृ॒तुभिः॑ । \newline
55. अ॒न्वारो॑ह॒न्तीत्य॑नु - आरो॑हन्ति । \newline
56. ऋ॒तुभिः॑ संॅवथ्स॒रꣳ सं॑ॅवथ्स॒र मृ॒तुभिर्॑. ऋ॒तुभिः॑ संॅवथ्स॒रम् । \newline
57. ऋ॒तुभि॒रित्यृ॒तु - भिः॒ । \newline
58. सं॒ॅव॒थ्स॒रम् ते ते सं॑ॅवथ्स॒रꣳ सं॑ॅवथ्स॒रम् ते । \newline
59. सं॒ॅव॒थ्स॒रमिति॑ सं - व॒थ्स॒रम् । \newline
60. ते सं॑ॅवथ्स॒रे सं॑ॅवथ्स॒रे ते ते सं॑ॅवथ्स॒रे । \newline
61. सं॒ॅव॒थ्स॒र ए॒वैव सं॑ॅवथ्स॒रे सं॑ॅवथ्स॒र ए॒व । \newline
62. सं॒ॅव॒थ्स॒र इति॑ सं - व॒थ्स॒रे । \newline
63. ए॒व प्रति॒ प्रत्ये॒वैव प्रति॑ । \newline

\textbf{Ghana Paata } \newline

1. वन॒स्पती᳚न् प्र॒जाम् प्र॒जां ॅवन॒स्पती॒न्॒. वन॒स्पती᳚न् प्र॒जाम् प॒शून् प॒शून् प्र॒जां ॅवन॒स्पती॒न्॒. वन॒स्पती᳚न् प्र॒जाम् प॒शून् । \newline
2. प्र॒जाम् प॒शून् प॒शून् प्र॒जाम् प्र॒जाम् प॒शून् तेन॒ तेन॑ प॒शून् प्र॒जाम् प्र॒जाम् प॒शून् तेन॑ । \newline
3. प्र॒जामिति॑ प्र - जाम् । \newline
4. प॒शून् तेन॒ तेन॑ प॒शून् प॒शून् तेना॑ वर्द्धता वर्द्धत॒ तेन॑ प॒शून् प॒शून् तेना॑ वर्द्धत । \newline
5. तेना॑ वर्द्धता वर्द्धत॒ तेन॒ तेना॑ वर्द्धत॒ सा सा ऽव॑र्द्धत॒ तेन॒ तेना॑ वर्द्धत॒ सा । \newline
6. अ॒व॒र्द्ध॒त॒ सा सा ऽव॑र्द्धता वर्द्धत॒ सा जे॒मान॑म् जे॒मानꣳ॒॒ सा ऽव॑र्द्धता वर्द्धत॒ सा जे॒मान᳚म् । \newline
7. सा जे॒मान॑म् जे॒मानꣳ॒॒ सा सा जे॒मान॑म् महि॒मान॑म् महि॒मान॑म् जे॒मानꣳ॒॒ सा सा जे॒मान॑म् महि॒मान᳚म् । \newline
8. जे॒मान॑म् महि॒मान॑म् महि॒मान॑म् जे॒मान॑म् जे॒मान॑म् महि॒मान॑ मगच्छ दगच्छन् महि॒मान॑म् जे॒मान॑म् जे॒मान॑म् महि॒मान॑ मगच्छत् । \newline
9. म॒हि॒मान॑ मगच्छ दगच्छन् महि॒मान॑म् महि॒मान॑ मगच्छ॒द् ये ये॑ ऽगच्छन् महि॒मान॑म् महि॒मान॑ मगच्छ॒द् ये । \newline
10. अ॒ग॒च्छ॒द् ये ये॑ ऽगच्छ दगच्छ॒द् य ए॒व मे॒वं ॅये॑ ऽगच्छ दगच्छ॒द् य ए॒वम् । \newline
11. य ए॒व मे॒वं ॅये य ए॒वं ॅवि॒द्वाꣳसो॑ वि॒द्वाꣳस॑ ए॒वं ॅये य ए॒वं ॅवि॒द्वाꣳसः॑ । \newline
12. ए॒वं ॅवि॒द्वाꣳसो॑ वि॒द्वाꣳस॑ ए॒व मे॒वं ॅवि॒द्वाꣳस॑ ए॒ता ए॒ता वि॒द्वाꣳस॑ ए॒व मे॒वं ॅवि॒द्वाꣳस॑ ए॒ताः । \newline
13. वि॒द्वाꣳस॑ ए॒ता ए॒ता वि॒द्वाꣳसो॑ वि॒द्वाꣳस॑ ए॒ता आस॑त॒ आस॑त ए॒ता वि॒द्वाꣳसो॑ वि॒द्वाꣳस॑ ए॒ता आस॑ते । \newline
14. ए॒ता आस॑त॒ आस॑त ए॒ता ए॒ता आस॑ते वि॒राजं॑ ॅवि॒राज॒ मास॑त ए॒ता ए॒ता आस॑ते वि॒राज᳚म् । \newline
15. आस॑ते वि॒राजं॑ ॅवि॒राज॒ मास॑त॒ आस॑ते वि॒राज॑ मे॒वैव वि॒राज॒ मास॑त॒ आस॑ते वि॒राज॑ मे॒व । \newline
16. वि॒राज॑ मे॒वैव वि॒राजं॑ ॅवि॒राज॑ मे॒वात्मन् ना॒त्मन् ने॒व वि॒राजं॑ ॅवि॒राज॑ मे॒वात्मन्न् । \newline
17. वि॒राज॒मिति॑ वि - राज᳚म् । \newline
18. ए॒वात्मन् ना॒त्मन् ने॒वै वात्मन् धि॒त्वा धि॒त्वा ऽऽत्मन् ने॒वै वात्मन् धि॒त्वा । \newline
19. आ॒त्मन् धि॒त्वा धि॒त्वा ऽऽत्मन् ना॒त्मन् धि॒त्वा ऽन्नाद्य॑ म॒न्नाद्य॑म् धि॒त्वा ऽऽत्मन् ना॒त्मन् धि॒त्वा ऽन्नाद्य᳚म् । \newline
20. धि॒त्वा ऽन्नाद्य॑ म॒न्नाद्य॑म् धि॒त्वा धि॒त्वा ऽन्नाद्य॒ मवा वा॒न्नाद्य॑म् धि॒त्वा धि॒त्वा ऽन्नाद्य॒ मव॑ । \newline
21. अ॒न्नाद्य॒ मवा वा॒न्नाद्य॑ म॒न्नाद्य॒ मव॑ रुन्धते रुन्ध॒ते ऽवा॒न्नाद्य॑ म॒न्नाद्य॒ मव॑ रुन्धते । \newline
22. अ॒न्नाद्य॒मित्य॑न्न - अद्य᳚म् । \newline
23. अव॑ रुन्धते रुन्ध॒ते ऽवाव॑ रुन्धते॒ वर्द्ध॑न्ते॒ वर्द्ध॑न्ते रुन्ध॒ते ऽवाव॑ रुन्धते॒ वर्द्ध॑न्ते । \newline
24. रु॒न्ध॒ते॒ वर्द्ध॑न्ते॒ वर्द्ध॑न्ते रुन्धते रुन्धते॒ वर्द्ध॑न्ते प्र॒जया᳚ प्र॒जया॒ वर्द्ध॑न्ते रुन्धते रुन्धते॒ वर्द्ध॑न्ते प्र॒जया᳚ । \newline
25. वर्द्ध॑न्ते प्र॒जया᳚ प्र॒जया॒ वर्द्ध॑न्ते॒ वर्द्ध॑न्ते प्र॒जया॑ प॒शुभिः॑ प॒शुभिः॑ प्र॒जया॒ वर्द्ध॑न्ते॒ वर्द्ध॑न्ते प्र॒जया॑ प॒शुभिः॑ । \newline
26. प्र॒जया॑ प॒शुभिः॑ प॒शुभिः॑ प्र॒जया᳚ प्र॒जया॑ प॒शुभि॑र् जे॒मान॑म् जे॒मान॑म् प॒शुभिः॑ प्र॒जया᳚ प्र॒जया॑ प॒शुभि॑र् जे॒मान᳚म् । \newline
27. प्र॒जयेति॑ प्र - जया᳚ । \newline
28. प॒शुभि॑र् जे॒मान॑म् जे॒मान॑म् प॒शुभिः॑ प॒शुभि॑र् जे॒मान॑म् महि॒मान॑म् महि॒मान॑म् जे॒मान॑म् प॒शुभिः॑ प॒शुभि॑र् जे॒मान॑म् महि॒मान᳚म् । \newline
29. प॒शुभि॒रिति॑ प॒शु - भिः॒ । \newline
30. जे॒मान॑म् महि॒मान॑म् महि॒मान॑म् जे॒मान॑म् जे॒मान॑म् महि॒मान॑म् गच्छन्ति गच्छन्ति महि॒मान॑म् जे॒मान॑म् जे॒मान॑म् महि॒मान॑म् गच्छन्ति । \newline
31. म॒हि॒मान॑म् गच्छन्ति गच्छन्ति महि॒मान॑म् महि॒मान॑म् गच्छन्ति॒ ज्योति॒र् ज्योति॑र् गच्छन्ति महि॒मान॑म् महि॒मान॑म् गच्छन्ति॒ ज्योतिः॑ । \newline
32. ग॒च्छ॒न्ति॒ ज्योति॒र् ज्योति॑र् गच्छन्ति गच्छन्ति॒ ज्योति॑ रतिरा॒त्रो॑ ऽतिरा॒त्रो ज्योति॑र् गच्छन्ति गच्छन्ति॒ ज्योति॑ रतिरा॒त्रः । \newline
33. ज्योति॑ रतिरा॒त्रो॑ ऽतिरा॒त्रो ज्योति॒र् ज्योति॑ रतिरा॒त्रो भ॑वति भव त्यतिरा॒त्रो ज्योति॒र् ज्योति॑ रतिरा॒त्रो भ॑वति । \newline
34. अ॒ति॒रा॒त्रो भ॑वति भव त्यतिरा॒त्रो॑ ऽतिरा॒त्रो भ॑वति सुव॒र्गस्य॑ सुव॒र्गस्य॑ भव त्यतिरा॒त्रो॑ ऽतिरा॒त्रो भ॑वति सुव॒र्गस्य॑ । \newline
35. अ॒ति॒रा॒त्र इत्य॑ति - रा॒त्रः । \newline
36. भ॒व॒ति॒ सु॒व॒र्गस्य॑ सुव॒र्गस्य॑ भवति भवति सुव॒र्गस्य॑ लो॒कस्य॑ लो॒कस्य॑ सुव॒र्गस्य॑ भवति भवति सुव॒र्गस्य॑ लो॒कस्य॑ । \newline
37. सु॒व॒र्गस्य॑ लो॒कस्य॑ लो॒कस्य॑ सुव॒र्गस्य॑ सुव॒र्गस्य॑ लो॒कस्या नु॑ख्यात्या॒ अनु॑ख्यात्यै लो॒कस्य॑ सुव॒र्गस्य॑ सुव॒र्गस्य॑ लो॒कस्या नु॑ख्यात्यै । \newline
38. सु॒व॒र्गस्येति॑ सुवः - गस्य॑ । \newline
39. लो॒कस्या नु॑ख्यात्या॒ अनु॑ख्यात्यै लो॒कस्य॑ लो॒कस्या नु॑ख्यात्यै॒ पृष्ठ्यः॒ पृष्ठ्यो ऽनु॑ख्यात्यै लो॒कस्य॑ लो॒कस्या नु॑ख्यात्यै॒ पृष्ठ्यः॑ । \newline
40. अनु॑ख्यात्यै॒ पृष्ठ्यः॒ पृष्ठ्यो ऽनु॑ख्यात्या॒ अनु॑ख्यात्यै॒ पृष्ठ्य॑ ष्षड॒ह ष्ष॑ड॒हः पृष्ठ्यो ऽनु॑ख्यात्या॒ अनु॑ख्यात्यै॒ पृष्ठ्य॑ ष्षड॒हः । \newline
41. अनु॑ख्यात्या॒ इत्यनु॑ - ख्या॒त्यै॒ । \newline
42. पृष्ठ्य॑ ष्षड॒ह ष्ष॑ड॒हः पृष्ठ्यः॒ पृष्ठ्य॑ ष्षड॒हो भ॑वति भवति षड॒हः पृष्ठ्यः॒ पृष्ठ्य॑ ष्षड॒हो भ॑वति । \newline
43. ष॒ड॒हो भ॑वति भवति षड॒ह ष्ष॑ड॒हो भ॑वति॒ षट् थ्षड् भ॑वति षड॒ह ष्ष॑ड॒हो भ॑वति॒ षट् । \newline
44. ष॒ड॒ह इति॑ षट् - अ॒हः । \newline
45. भ॒व॒ति॒ षट् थ्षड् भ॑वति भवति॒ षड् वै वै षड् भ॑वति भवति॒ षड् वै । \newline
46. षड् वै वै षट् थ्षड् वा ऋ॒तव॑ ऋ॒तवो॒ वै षट् थ्षड् वा ऋ॒तवः॑ । \newline
47. वा ऋ॒तव॑ ऋ॒तवो॒ वै वा ऋ॒तव॒ ष्षट् थ्षडृ॒तवो॒ वै वा ऋ॒तव॒ ष्षट् । \newline
48. ऋ॒तव॒ ष्षट् थ्षडृ॒तव॑ ऋ॒तव॒ ष्षट् पृ॒ष्ठानि॑ पृ॒ष्ठानि॒ षडृ॒तव॑ ऋ॒तव॒ ष्षट् पृ॒ष्ठानि॑ । \newline
49. षट् पृ॒ष्ठानि॑ पृ॒ष्ठानि॒ षट् थ्षट् पृ॒ष्ठानि॑ पृ॒ष्ठैः पृ॒ष्ठैः पृ॒ष्ठानि॒ षट् थ्षट् पृ॒ष्ठानि॑ पृ॒ष्ठैः । \newline
50. पृ॒ष्ठानि॑ पृ॒ष्ठैः पृ॒ष्ठैः पृ॒ष्ठानि॑ पृ॒ष्ठानि॑ पृ॒ष्ठै रे॒वैव पृ॒ष्ठैः पृ॒ष्ठानि॑ पृ॒ष्ठानि॑ पृ॒ष्ठै रे॒व । \newline
51. पृ॒ष्ठै रे॒वैव पृ॒ष्ठैः पृ॒ष्ठै रे॒व र्‌तू नृ॒तू ने॒व पृ॒ष्ठैः पृ॒ष्ठै रे॒व र्‌तून् । \newline
52. ए॒व र्‌तू नृ॒तू ने॒वैव र्‌तून॒भ्यारो॑ह न्त्य॒भ्यारो॑ह न्त्यृ॒तू ने॒वैव र्‌तून॒भ्यारो॑हन्ति । \newline
53. ऋ॒तू न॒भ्यारो॑ह न्त्य॒भ्यारो॑ह न्त्यृ॒तू नृ॒तू न॒भ्यारो॑ह न्त्यृ॒तुभिर्॑. ऋ॒तुभि॑ र॒भ्यारो॑ह न्त्यृ॒तू नृ॒तू न॒भ्यारो॑ह न्त्यृ॒तुभिः॑ । \newline
54. अ॒भ्यारो॑ह न्त्यृ॒तुभिर्॑. ऋ॒तुभि॑ र॒भ्यारो॑ह न्त्य॒भ्यारो॑ह न्त्यृ॒तुभिः॑ संॅवथ्स॒रꣳ सं॑ॅवथ्स॒र मृ॒तुभि॑ र॒भ्यारो॑ह न्त्य॒भ्यारो॑ह न्त्यृ॒तुभिः॑ संॅवथ्स॒रम् । \newline
55. अ॒न्वारो॑ह॒न्तीत्य॑नु - आरो॑हन्ति । \newline
56. ऋ॒तुभिः॑ संॅवथ्स॒रꣳ सं॑ॅवथ्स॒र मृ॒तुभिर्॑. ऋ॒तुभिः॑ संॅवथ्स॒रम् ते ते सं॑ॅवथ्स॒र मृ॒तुभिर्॑. ऋ॒तुभिः॑ संॅवथ्स॒रम् ते । \newline
57. ऋ॒तुभि॒रित्यृ॒तु - भिः॒ । \newline
58. सं॒ॅव॒थ्स॒रम् ते ते सं॑ॅवथ्स॒रꣳ सं॑ॅवथ्स॒रम् ते सं॑ॅवथ्स॒रे सं॑ॅवथ्स॒रे ते सं॑ॅवथ्स॒रꣳ सं॑ॅवथ्स॒रम् ते सं॑ॅवथ्स॒रे । \newline
59. सं॒ॅव॒थ्स॒रमिति॑ सं - व॒थ्स॒रम् । \newline
60. ते सं॑ॅवथ्स॒रे सं॑ॅवथ्स॒रे ते ते सं॑ॅवथ्स॒र ए॒वैव सं॑ॅवथ्स॒रे ते ते सं॑ॅवथ्स॒र ए॒व । \newline
61. सं॒ॅव॒थ्स॒र ए॒वैव सं॑ॅवथ्स॒रे सं॑ॅवथ्स॒र ए॒व प्रति॒ प्रत्ये॒व सं॑ॅवथ्स॒रे सं॑ॅवथ्स॒र ए॒व प्रति॑ । \newline
62. सं॒ॅव॒थ्स॒र इति॑ सं - व॒थ्स॒रे । \newline
63. ए॒व प्रति॒ प्रत्ये॒वैव प्रति॑ तिष्ठन्ति तिष्ठन्ति॒ प्रत्ये॒वैव प्रति॑ तिष्ठन्ति । \newline
\pagebreak
\markright{ TS 7.4.3.3  \hfill https://www.vedavms.in \hfill}

\section{ TS 7.4.3.3 }

\textbf{TS 7.4.3.3 } \newline
\textbf{Samhita Paata} \newline

प्रति॑ तिष्ठन्ति त्रयस्त्रिꣳ॒॒शात् त्र॑यस्त्रिꣳ॒॒शमुप॑ यन्ति य॒ज्ञ्स्य॒ संत॑त्या॒ अथो᳚ प्र॒जाप॑ति॒र्वै त्र॑यस्त्रिꣳ॒॒शः प्र॒जाप॑तिमे॒वाऽऽ*र॑भन्ते॒ प्रति॑ष्ठित्यै त्रिण॒वो भ॑वति॒ विजि॑त्या एकविꣳ॒॒शो भ॑वति॒ प्रति॑ष्ठित्या॒ अथो॒ रुच॑मे॒वाऽऽ*त्मन् द॑धते त्रि॒वृद॑ग्नि॒ष्टुद्-भ॑वति पा॒प्मान॑मे॒व तेन॒ निर्द॑ह॒न्तेऽथो॒ तेजो॒ वै त्रि॒वृत् तेज॑ ए॒वाऽऽत्मन् द॑धते पञ्चद॒श इ॑न्द्रस्तो॒मो भ॑वतीन्द्रि॒यमे॒वाव॑ - [  ] \newline

\textbf{Pada Paata} \newline

प्रतीति॑ । ति॒ष्ठ॒न्ति॒ । त्र॒य॒स्त्रिꣳ॒॒शादिति॑ त्रयः - त्रिꣳ॒॒शात् । त्र॒य॒स्त्रिꣳ॒॒शमिति॑ त्रयः - त्रिꣳ॒॒शम् । उपेति॑ । य॒न्ति॒ । य॒ज्ञ्स्य॑ । संत॑त्या॒ इति॒ सं - त॒त्यै॒ । अथो॒ इति॑ । प्र॒जाप॑ति॒रिति॑ प्र॒जा-प॒तिः॒ । वै । त्र॒य॒स्त्रिꣳ॒॒श इति॑ त्रयः-त्रिꣳ॒॒शः । प्र॒जाप॑ति॒मिति॑ प्र॒जा-प॒ति॒म् । ए॒व । एति॑ । र॒भ॒न्ते॒ । प्रति॑ष्ठित्या॒ इति॒ प्रति॑ - स्थि॒त्यै॒ । त्रि॒ण॒व इति॑ त्रि - न॒वः । भ॒व॒ति॒ । विजि॑त्या॒ इति॒ वि - जि॒त्यै॒ । ए॒क॒विꣳ॒॒श इत्ये॑क - विꣳ॒॒शः । भ॒व॒ति॒ । प्रति॑ष्ठित्या॒ इति॒ प्रति॑ - स्थि॒त्यै॒ । अथो॒ इति॑ । रुच᳚म् । ए॒व । आ॒त्मन्न् । द॒ध॒ते॒ । त्रि॒वृदिति॑ त्रि - वृत् । अ॒ग्नि॒ष्टुदित्य॑ग्नि - स्तुत् । भ॒व॒ति॒ । पा॒प्मान᳚म् । ए॒व । तेन॑ । निरिति॑ । द॒ह॒न्ते॒ । अथो॒ इति॑ । तेजः॑ । वै । त्रि॒वृदिति॑ त्रि-वृत् । तेजः॑ । ए॒व । आ॒त्मन्न् । द॒ध॒ते॒ । प॒ञ्च॒द॒श इति॑ पञ्च - द॒शः । इ॒न्द्र॒स्तो॒म इती᳚न्द्र - स्तो॒मः । भ॒व॒ति॒ । इ॒न्द्रि॒यम् । ए॒व । अवेति॑ ।  \newline


\textbf{Krama Paata} \newline

प्रति॑ तिष्ठन्ति । ति॒ष्ठ॒न्ति॒ त्र॒य॒स्त्रिꣳ॒॒शात् । त्र॒य॒स्त्रिꣳ॒॒शात् त्र॑यस्त्रिꣳ॒॒शम् । त्र॒य॒स्त्रिꣳ॒॒शादिति॑ त्रयः - त्रिꣳ॒॒शात् । त्र॒य॒स्त्रिꣳ॒॒शमुप॑ । त्र॒य॒स्त्रिꣳ॒॒शमिति॑ त्रयः - त्रिꣳ॒॒शम् । उप॑ यन्ति । य॒न्ति॒ य॒ज्ञ्स्य॑ । य॒ज्ञ्स्य॒ सन्त॑त्यै । सन्त॑त्या॒ अथो᳚ । सन्त॑त्या॒ इति॒ सम् - त॒त्यै॒ । अथो᳚ प्र॒जाप॑तिः । अथो॒ इत्यथो᳚ । प्र॒जाप॑ति॒र् वै । प्र॒जाप॑ति॒रिति॑ प्र॒जा - प॒तिः॒ । वै त्र॑यस्त्रिꣳ॒॒शः । त्र॒य॒स्त्रिꣳ॒॒शः प्र॒जाप॑तिम् । त्र॒य॒स्त्रिꣳ॒॒श इति॑ त्रयः - त्रिꣳ॒॒शः । प्र॒जाप॑तिमे॒व । प्र॒जाप॑ति॒मिति॑ प्र॒जा - प॒ति॒म् । ए॒वा । आ र॑भन्ते । र॒भ॒न्ते॒ प्रति॑ष्ठित्यै । प्रति॑ष्ठित्यै त्रिण॒वः । प्रति॑ष्ठित्या॒ इति॒ प्रति॑ - स्थि॒त्यै॒ । त्रि॒ण॒वो भ॑वति । त्रि॒ण॒व इति॑ त्रि - न॒वः । भ॒व॒ति॒ विजि॑त्यै । विजि॑त्या एकविꣳ॒॒शः । विजि॑त्या॒ इति॒ वि - जि॒त्यै॒ । ए॒क॒विꣳ॒॒शो भ॑वति । ए॒क॒विꣳ॒॒श इत्ये॑क - विꣳ॒॒शः । भ॒व॒ति॒ प्रति॑ष्ठित्यै । प्रति॑ष्ठित्या॒ अथो᳚ । प्रति॑ष्ठित्या॒ इति॒ प्रति॑ - स्थि॒त्यै॒ । अथो॒ रुच᳚म् । अथो॒ इत्यथो᳚ । रुच॑मे॒व । ए॒वात्मन्न् । आ॒त्मन् द॑धते । द॒ध॒ते॒ त्रि॒वृत् । त्रि॒वृद॑ग्नि॒ष्टुत् । त्रि॒वृदिति॑ त्रि - वृत् । अ॒ग्नि॒ष्टुद् भ॑वति । अ॒ग्नि॒ष्टुदित्य॑ग्नि - स्तुत् । भ॒व॒ति॒ पा॒प्मान᳚म् । पा॒प्मान॑मे॒व । ए॒व तेन॑ । तेन॒ निः । निर् द॑हन्ते । द॒ह॒न्तेऽथो᳚ । अथो॒ तेजः॑ । अथो॒ इत्यथो᳚ । तेजो॒ वै । वै त्रि॒वृत् । त्रि॒वृत् तेजः॑ । त्रि॒वृदिति॑ त्रि - वृत् । तेज॑ ए॒व । ए॒वात्मन्न् । आ॒त्मन् द॑धते । द॒ध॒ते॒ प॒ञ्च॒द॒शः । प॒ञ्च॒द॒श इ॑न्द्रस्तो॒मः । प॒ञ्च॒द॒श इति॑ पञ्च - द॒शः । इ॒न्द्र॒स्तो॒मो भ॑वति । इ॒न्द्र॒स्तो॒म इती᳚न्द्र - स्तो॒मः । भ॒व॒ती॒न्द्रि॒यम् । इ॒न्द्रि॒यमे॒व । ए॒वाव॑ । अव॑ रुन्धते \newline

\textbf{Jatai Paata} \newline

1. प्रति॑ तिष्ठन्ति तिष्ठन्ति॒ प्रति॒ प्रति॑ तिष्ठन्ति । \newline
2. ति॒ष्ठ॒न्ति॒ त्र॒य॒स्त्रिꣳ॒॒शात् त्र॑यस्त्रिꣳ॒॒शात् ति॑ष्ठन्ति तिष्ठन्ति त्रयस्त्रिꣳ॒॒शात् । \newline
3. त्र॒य॒स्त्रिꣳ॒॒शात् त्र॑यस्त्रिꣳ॒॒शम् त्र॑यस्त्रिꣳ॒॒शम् त्र॑यस्त्रिꣳ॒॒शात् त्र॑यस्त्रिꣳ॒॒शात् त्र॑यस्त्रिꣳ॒॒शम् । \newline
4. त्र॒य॒स्त्रिꣳ॒॒शादिति॑ त्रयः - त्रिꣳ॒॒शात् । \newline
5. त्र॒य॒स्त्रिꣳ॒॒श मुपोप॑ त्रयस्त्रिꣳ॒॒शम् त्र॑यस्त्रिꣳ॒॒श मुप॑ । \newline
6. त्र॒य॒स्त्रिꣳ॒॒शमिति॑ त्रयः - त्रिꣳ॒॒शम् । \newline
7. उप॑ यन्ति य॒न्त्युपोप॑ यन्ति । \newline
8. य॒न्ति॒ य॒ज्ञ्स्य॑ य॒ज्ञ्स्य॑ यन्ति यन्ति य॒ज्ञ्स्य॑ । \newline
9. य॒ज्ञ्स्य॒ सन्त॑त्यै॒ सन्त॑त्यै य॒ज्ञ्स्य॑ य॒ज्ञ्स्य॒ सन्त॑त्यै । \newline
10. सन्त॑त्या॒ अथो॒ अथो॒ सन्त॑त्यै॒ सन्त॑त्या॒ अथो᳚ । \newline
11. सन्त॑त्या॒ इति॒ सं - त॒त्यै॒ । \newline
12. अथो᳚ प्र॒जाप॑तिः प्र॒जाप॑ति॒रथो॒ अथो᳚ प्र॒जाप॑तिः । \newline
13. अथो॒ इत्यथो᳚ । \newline
14. प्र॒जाप॑ति॒र् वै वै प्र॒जाप॑तिः प्र॒जाप॑ति॒र् वै । \newline
15. प्र॒जाप॑ति॒रिति॑ प्र॒जा - प॒तिः॒ । \newline
16. वै त्र॑यस्त्रिꣳ॒॒श स्त्र॑यस्त्रिꣳ॒॒शो वै वै त्र॑यस्त्रिꣳ॒॒शः । \newline
17. त्र॒य॒स्त्रिꣳ॒॒शः प्र॒जाप॑तिम् प्र॒जाप॑तिम् त्रयस्त्रिꣳ॒॒श स्त्र॑यस्त्रिꣳ॒॒शः प्र॒जाप॑तिम् । \newline
18. त्र॒य॒स्त्रिꣳ॒॒श इति॑ त्रयः - त्रिꣳ॒॒शः । \newline
19. प्र॒जाप॑ति मे॒वैव प्र॒जाप॑तिम् प्र॒जाप॑ति मे॒व । \newline
20. प्र॒जाप॑ति॒मिति॑ प्र॒जा - प॒ति॒म् । \newline
21. ए॒वै वैवा । \newline
22. आ र॑भन्ते रभन्त॒ आ र॑भन्ते । \newline
23. र॒भ॒न्ते॒ प्रति॑ष्ठित्यै॒ प्रति॑ष्ठित्यै रभन्ते रभन्ते॒ प्रति॑ष्ठित्यै । \newline
24. प्रति॑ष्ठित्यै त्रिण॒व स्त्रि॑ण॒वः प्रति॑ष्ठित्यै॒ प्रति॑ष्ठित्यै त्रिण॒वः । \newline
25. प्रति॑ष्ठित्या॒ इति॒ प्रति॑ - स्थि॒त्यै॒ । \newline
26. त्रि॒ण॒वो भ॑वति भवति त्रिण॒व स्त्रि॑ण॒वो भ॑वति । \newline
27. त्रि॒ण॒व इति॑ त्रि - न॒वः । \newline
28. भ॒व॒ति॒ विजि॑त्यै॒ विजि॑त्यै भवति भवति॒ विजि॑त्यै । \newline
29. विजि॑त्या एकविꣳ॒॒श ए॑कविꣳ॒॒शो विजि॑त्यै॒ विजि॑त्या एकविꣳ॒॒शः । \newline
30. विजि॑त्या॒ इति॒ वि - जि॒त्यै॒ । \newline
31. ए॒क॒विꣳ॒॒शो भ॑वति भव त्येकविꣳ॒॒श ए॑कविꣳ॒॒शो भ॑वति । \newline
32. ए॒क॒विꣳ॒॒श इत्ये॑क - विꣳ॒॒शः । \newline
33. भ॒व॒ति॒ प्रति॑ष्ठित्यै॒ प्रति॑ष्ठित्यै भवति भवति॒ प्रति॑ष्ठित्यै । \newline
34. प्रति॑ष्ठित्या॒ अथो॒ अथो॒ प्रति॑ष्ठित्यै॒ प्रति॑ष्ठित्या॒ अथो᳚ । \newline
35. प्रति॑ष्ठित्या॒ इति॒ प्रति॑ - स्थि॒त्यै॒ । \newline
36. अथो॒ रुचꣳ॒॒ रुच॒ मथो॒ अथो॒ रुच᳚म् । \newline
37. अथो॒ इत्यथो᳚ । \newline
38. रुच॑ मे॒वैव रुचꣳ॒॒ रुच॑ मे॒व । \newline
39. ए॒वात्मन् ना॒त्मन् ने॒वै वात्मन्न् । \newline
40. आ॒त्मन् द॑धते दधत आ॒त्मन् ना॒त्मन् द॑धते । \newline
41. द॒ध॒ते॒ त्रि॒वृत् त्रि॒वृद् द॑धते दधते त्रि॒वृत् । \newline
42. त्रि॒वृ द॑ग्नि॒ष्टु द॑ग्नि॒ष्टुत् त्रि॒वृत् त्रि॒वृ द॑ग्नि॒ष्टुत् । \newline
43. त्रि॒वृदिति॑ त्रि - वृत् । \newline
44. अ॒ग्नि॒ष्टुद् भ॑वति भव त्यग्नि॒ष्टु द॑ग्नि॒ष्टुद् भ॑वति । \newline
45. अ॒ग्नि॒ष्टुदित्य॑ग्नि - स्तुत् । \newline
46. भ॒व॒ति॒ पा॒प्मान॑म् पा॒प्मान॑म् भवति भवति पा॒प्मान᳚म् । \newline
47. पा॒प्मान॑ मे॒वैव पा॒प्मान॑म् पा॒प्मान॑ मे॒व । \newline
48. ए॒व तेन॒ तेनै॒वैव तेन॑ । \newline
49. तेन॒ निर् णिष् टेन॒ तेन॒ निः । \newline
50. निर् द॑हन्ते दहन्ते॒ निर् णिर् द॑हन्ते । \newline
51. द॒ह॒न्ते ऽथो॒ अथो॑ दहन्ते दह॒न्ते ऽथो᳚ । \newline
52. अथो॒ तेज॒ स्तेजो ऽथो॒ अथो॒ तेजः॑ । \newline
53. अथो॒ इत्यथो᳚ । \newline
54. तेजो॒ वै वै तेज॒ स्तेजो॒ वै । \newline
55. वै त्रि॒वृत् त्रि॒वृद् वै वै त्रि॒वृत् । \newline
56. त्रि॒वृत् तेज॒ स्तेज॑ स्त्रि॒वृत् त्रि॒वृत् तेजः॑ । \newline
57. त्रि॒वृदिति॑ त्रि - वृत् । \newline
58. तेज॑ ए॒वैव तेज॒ स्तेज॑ ए॒व । \newline
59. ए॒वात्मन् ना॒त्मन् ने॒वै वात्मन्न् । \newline
60. आ॒त्मन् द॑धते दधत आ॒त्मन् ना॒त्मन् द॑धते । \newline
61. द॒ध॒ते॒ प॒ञ्च॒द॒शः प॑ञ्चद॒शो द॑धते दधते पञ्चद॒शः । \newline
62. प॒ञ्च॒द॒श इ॑न्द्रस्तो॒म इ॑न्द्रस्तो॒मः प॑ञ्चद॒शः प॑ञ्चद॒श इ॑न्द्रस्तो॒मः । \newline
63. प॒ञ्च॒द॒श इति॑ पञ्च - द॒शः । \newline
64. इ॒न्द्र॒स्तो॒मो भ॑वति भवतीन्द्रस्तो॒म इ॑न्द्रस्तो॒मो भ॑वति । \newline
65. इ॒न्द्र॒स्तो॒म इती᳚न्द्र - स्तो॒मः । \newline
66. भ॒व॒ती॒न्द्रि॒य मि॑न्द्रि॒यम् भ॑वति भवतीन्द्रि॒यम् । \newline
67. इ॒न्द्रि॒य मे॒वैवेन्द्रि॒य मि॑न्द्रि॒य मे॒व । \newline
68. ए॒वावा वै॒वै वाव॑ । \newline
69. अव॑ रुन्धते रुन्ध॒ते ऽवाव॑ रुन्धते । \newline

\textbf{Ghana Paata } \newline

1. प्रति॑ तिष्ठन्ति तिष्ठन्ति॒ प्रति॒ प्रति॑ तिष्ठन्ति त्रयस्त्रिꣳ॒॒शात् त्र॑यस्त्रिꣳ॒॒शात् ति॑ष्ठन्ति॒ प्रति॒ प्रति॑ तिष्ठन्ति त्रयस्त्रिꣳ॒॒शात् । \newline
2. ति॒ष्ठ॒न्ति॒ त्र॒य॒स्त्रिꣳ॒॒शात् त्र॑यस्त्रिꣳ॒॒शात् ति॑ष्ठन्ति तिष्ठन्ति त्रयस्त्रिꣳ॒॒शात् त्र॑यस्त्रिꣳ॒॒शम् त्र॑यस्त्रिꣳ॒॒शम् त्र॑यस्त्रिꣳ॒॒शात् ति॑ष्ठन्ति तिष्ठन्ति त्रयस्त्रिꣳ॒॒शात् त्र॑यस्त्रिꣳ॒॒शम् । \newline
3. त्र॒य॒स्त्रिꣳ॒॒शात् त्र॑यस्त्रिꣳ॒॒शम् त्र॑यस्त्रिꣳ॒॒शम् त्र॑यस्त्रिꣳ॒॒शात् त्र॑यस्त्रिꣳ॒॒शात् त्र॑यस्त्रिꣳ॒॒श मुपोप॑ त्रयस्त्रिꣳ॒॒शम् त्र॑यस्त्रिꣳ॒॒शात् त्र॑यस्त्रिꣳ॒॒शात् त्र॑यस्त्रिꣳ॒॒श मुप॑ । \newline
4. त्र॒य॒स्त्रिꣳ॒॒शादिति॑ त्रयः - त्रिꣳ॒॒शात् । \newline
5. त्र॒य॒स्त्रिꣳ॒॒श मुपोप॑ त्रयस्त्रिꣳ॒॒शम् त्र॑यस्त्रिꣳ॒॒श मुप॑ यन्ति य॒न्त्युप॑ त्रयस्त्रिꣳ॒॒शम् त्र॑यस्त्रिꣳ॒॒श मुप॑ यन्ति । \newline
6. त्र॒य॒स्त्रिꣳ॒॒शमिति॑ त्रयः - त्रिꣳ॒॒शम् । \newline
7. उप॑ यन्ति य॒न्त्युपोप॑ यन्ति य॒ज्ञ्स्य॑ य॒ज्ञ्स्य॑ य॒न्त्युपोप॑ यन्ति य॒ज्ञ्स्य॑ । \newline
8. य॒न्ति॒ य॒ज्ञ्स्य॑ य॒ज्ञ्स्य॑ यन्ति यन्ति य॒ज्ञ्स्य॒ सन्त॑त्यै॒ सन्त॑त्यै य॒ज्ञ्स्य॑ यन्ति यन्ति य॒ज्ञ्स्य॒ सन्त॑त्यै । \newline
9. य॒ज्ञ्स्य॒ सन्त॑त्यै॒ सन्त॑त्यै य॒ज्ञ्स्य॑ य॒ज्ञ्स्य॒ सन्त॑त्या॒ अथो॒ अथो॒ सन्त॑त्यै य॒ज्ञ्स्य॑ य॒ज्ञ्स्य॒ सन्त॑त्या॒ अथो᳚ । \newline
10. सन्त॑त्या॒ अथो॒ अथो॒ सन्त॑त्यै॒ सन्त॑त्या॒ अथो᳚ प्र॒जाप॑तिः प्र॒जाप॑ति॒ रथो॒ सन्त॑त्यै॒ सन्त॑त्या॒ अथो᳚ प्र॒जाप॑तिः । \newline
11. सन्त॑त्या॒ इति॒ सं - त॒त्यै॒ । \newline
12. अथो᳚ प्र॒जाप॑तिः प्र॒जाप॑ति॒ रथो॒ अथो᳚ प्र॒जाप॑ति॒र् वै वै प्र॒जाप॑ति॒ रथो॒ अथो᳚ प्र॒जाप॑ति॒र् वै । \newline
13. अथो॒ इत्यथो᳚ । \newline
14. प्र॒जाप॑ति॒र् वै वै प्र॒जाप॑तिः प्र॒जाप॑ति॒र् वै त्र॑यस्त्रिꣳ॒॒श स्त्र॑यस्त्रिꣳ॒॒शो वै प्र॒जाप॑तिः प्र॒जाप॑ति॒र् वै त्र॑यस्त्रिꣳ॒॒शः । \newline
15. प्र॒जाप॑ति॒रिति॑ प्र॒जा - प॒तिः॒ । \newline
16. वै त्र॑यस्त्रिꣳ॒॒श स्त्र॑यस्त्रिꣳ॒॒शो वै वै त्र॑यस्त्रिꣳ॒॒शः प्र॒जाप॑तिम् प्र॒जाप॑तिम् त्रयस्त्रिꣳ॒॒शो वै वै त्र॑यस्त्रिꣳ॒॒शः प्र॒जाप॑तिम् । \newline
17. त्र॒य॒स्त्रिꣳ॒॒शः प्र॒जाप॑तिम् प्र॒जाप॑तिम् त्रयस्त्रिꣳ॒॒श स्त्र॑यस्त्रिꣳ॒॒शः प्र॒जाप॑ति मे॒वैव प्र॒जाप॑तिम् त्रयस्त्रिꣳ॒॒श स्त्र॑यस्त्रिꣳ॒॒शः प्र॒जाप॑ति मे॒व । \newline
18. त्र॒य॒स्त्रिꣳ॒॒श इति॑ त्रयः - त्रिꣳ॒॒शः । \newline
19. प्र॒जाप॑ति मे॒वैव प्र॒जाप॑तिम् प्र॒जाप॑ति मे॒वैव प्र॒जाप॑तिम् प्र॒जाप॑ति मे॒वा । \newline
20. प्र॒जाप॑ति॒मिति॑ प्र॒जा - प॒ति॒म् । \newline
21. ए॒वै वैवा र॑भन्ते रभन्त॒ ऐवैवा र॑भन्ते । \newline
22. आ र॑भन्ते रभन्त॒ आ र॑भन्ते॒ प्रति॑ष्ठित्यै॒ प्रति॑ष्ठित्यै रभन्त॒ आ र॑भन्ते॒ प्रति॑ष्ठित्यै । \newline
23. र॒भ॒न्ते॒ प्रति॑ष्ठित्यै॒ प्रति॑ष्ठित्यै रभन्ते रभन्ते॒ प्रति॑ष्ठित्यै त्रिण॒व स्त्रि॑ण॒वः प्रति॑ष्ठित्यै रभन्ते रभन्ते॒ प्रति॑ष्ठित्यै त्रिण॒वः । \newline
24. प्रति॑ष्ठित्यै त्रिण॒व स्त्रि॑ण॒वः प्रति॑ष्ठित्यै॒ प्रति॑ष्ठित्यै त्रिण॒वो भ॑वति भवति त्रिण॒वः प्रति॑ष्ठित्यै॒ प्रति॑ष्ठित्यै त्रिण॒वो भ॑वति । \newline
25. प्रति॑ष्ठित्या॒ इति॒ प्रति॑ - स्थि॒त्यै॒ । \newline
26. त्रि॒ण॒वो भ॑वति भवति त्रिण॒व स्त्रि॑ण॒वो भ॑वति॒ विजि॑त्यै॒ विजि॑त्यै भवति त्रिण॒व स्त्रि॑ण॒वो भ॑वति॒ विजि॑त्यै । \newline
27. त्रि॒ण॒व इति॑ त्रि - न॒वः । \newline
28. भ॒व॒ति॒ विजि॑त्यै॒ विजि॑त्यै भवति भवति॒ विजि॑त्या एकविꣳ॒॒श ए॑कविꣳ॒॒शो विजि॑त्यै भवति भवति॒ विजि॑त्या एकविꣳ॒॒शः । \newline
29. विजि॑त्या एकविꣳ॒॒श ए॑कविꣳ॒॒शो विजि॑त्यै॒ विजि॑त्या एकविꣳ॒॒शो भ॑वति भव त्येकविꣳ॒॒शो विजि॑त्यै॒ विजि॑त्या एकविꣳ॒॒शो भ॑वति । \newline
30. विजि॑त्या॒ इति॒ वि - जि॒त्यै॒ । \newline
31. ए॒क॒विꣳ॒॒शो भ॑वति भव त्येकविꣳ॒॒श ए॑कविꣳ॒॒शो भ॑वति॒ प्रति॑ष्ठित्यै॒ प्रति॑ष्ठित्यै भव त्येकविꣳ॒॒श ए॑कविꣳ॒॒शो भ॑वति॒ प्रति॑ष्ठित्यै । \newline
32. ए॒क॒विꣳ॒॒श इत्ये॑क - विꣳ॒॒शः । \newline
33. भ॒व॒ति॒ प्रति॑ष्ठित्यै॒ प्रति॑ष्ठित्यै भवति भवति॒ प्रति॑ष्ठित्या॒ अथो॒ अथो॒ प्रति॑ष्ठित्यै भवति भवति॒ प्रति॑ष्ठित्या॒ अथो᳚ । \newline
34. प्रति॑ष्ठित्या॒ अथो॒ अथो॒ प्रति॑ष्ठित्यै॒ प्रति॑ष्ठित्या॒ अथो॒ रुचꣳ॒॒ रुच॒ मथो॒ प्रति॑ष्ठित्यै॒ प्रति॑ष्ठित्या॒ अथो॒ रुच᳚म् । \newline
35. प्रति॑ष्ठित्या॒ इति॒ प्रति॑ - स्थि॒त्यै॒ । \newline
36. अथो॒ रुचꣳ॒॒ रुच॒ मथो॒ अथो॒ रुच॑ मे॒वैव रुच॒ मथो॒ अथो॒ रुच॑ मे॒व । \newline
37. अथो॒ इत्यथो᳚ । \newline
38. रुच॑ मे॒वैव रुचꣳ॒॒ रुच॑ मे॒वात्मन् ना॒त्मन् ने॒व रुचꣳ॒॒ रुच॑ मे॒वात्मन्न् । \newline
39. ए॒वात्मन् ना॒त्मन् ने॒वै वात्मन् द॑धते दधत आ॒त्मन् ने॒वै वात्मन् द॑धते । \newline
40. आ॒त्मन् द॑धते दधत आ॒त्मन् ना॒त्मन् द॑धते त्रि॒वृत् त्रि॒वृद् द॑धत आ॒त्मन् ना॒त्मन् द॑धते त्रि॒वृत् । \newline
41. द॒ध॒ते॒ त्रि॒वृत् त्रि॒वृद् द॑धते दधते त्रि॒वृ द॑ग्नि॒ष्टु द॑ग्नि॒ष्टुत् त्रि॒वृद् द॑धते दधते त्रि॒वृ द॑ग्नि॒ष्टुत् । \newline
42. त्रि॒वृ द॑ग्नि॒ष्टु द॑ग्नि॒ष्टुत् त्रि॒वृत् त्रि॒वृ द॑ग्नि॒ष्टुद् भ॑वति भव त्यग्नि॒ष्टुत् त्रि॒वृत् त्रि॒वृ द॑ग्नि॒ष्टुद् भ॑वति । \newline
43. त्रि॒वृदिति॑ त्रि - वृत् । \newline
44. अ॒ग्नि॒ष्टुद् भ॑वति भव त्यग्नि॒ष्टु द॑ग्नि॒ष्टुद् भ॑वति पा॒प्मान॑म् पा॒प्मान॑म् भव त्यग्नि॒ष्टु द॑ग्नि॒ष्टुद् भ॑वति पा॒प्मान᳚म् । \newline
45. अ॒ग्नि॒ष्टुदित्य॑ग्नि - स्तुत् । \newline
46. भ॒व॒ति॒ पा॒प्मान॑म् पा॒प्मान॑म् भवति भवति पा॒प्मान॑ मे॒वैव पा॒प्मान॑म् भवति भवति पा॒प्मान॑ मे॒व । \newline
47. पा॒प्मान॑ मे॒वैव पा॒प्मान॑म् पा॒प्मान॑ मे॒व तेन॒ तेनै॒व पा॒प्मान॑म् पा॒प्मान॑ मे॒व तेन॑ । \newline
48. ए॒व तेन॒ तेनै॒वैव तेन॒ निर् णिष् टेनै॒वैव तेन॒ निः । \newline
49. तेन॒ निर् णिष् टेन॒ तेन॒ निर् द॑हन्ते दहन्ते॒ निष् टेन॒ तेन॒ निर् द॑हन्ते । \newline
50. निर् द॑हन्ते दहन्ते॒ निर् णिर् द॑ह॒न्ते ऽथो॒ अथो॑ दहन्ते॒ निर् णिर् द॑ह॒न्ते ऽथो᳚ । \newline
51. द॒ह॒न्ते ऽथो॒ अथो॑ दहन्ते दह॒न्ते ऽथो॒ तेज॒ स्तेजो ऽथो॑ दहन्ते दह॒न्ते ऽथो॒ तेजः॑ । \newline
52. अथो॒ तेज॒ स्तेजो ऽथो॒ अथो॒ तेजो॒ वै वै तेजो ऽथो॒ अथो॒ तेजो॒ वै । \newline
53. अथो॒ इत्यथो᳚ । \newline
54. तेजो॒ वै वै तेज॒ स्तेजो॒ वै त्रि॒वृत् त्रि॒वृद् वै तेज॒ स्तेजो॒ वै त्रि॒वृत् । \newline
55. वै त्रि॒वृत् त्रि॒वृद् वै वै त्रि॒वृत् तेज॒ स्तेज॑ स्त्रि॒वृद् वै वै त्रि॒वृत् तेजः॑ । \newline
56. त्रि॒वृत् तेज॒ स्तेज॑ स्त्रि॒वृत् त्रि॒वृत् तेज॑ ए॒वैव तेज॑ स्त्रि॒वृत् त्रि॒वृत् तेज॑ ए॒व । \newline
57. त्रि॒वृदिति॑ त्रि - वृत् । \newline
58. तेज॑ ए॒वैव तेज॒ स्तेज॑ ए॒वात्मन् ना॒त्मन् ने॒व तेज॒ स्तेज॑ ए॒वात्मन्न् । \newline
59. ए॒वात्मन् ना॒त्मन् ने॒वै वात्मन् द॑धते दधत आ॒त्मन् ने॒वै वात्मन् द॑धते । \newline
60. आ॒त्मन् द॑धते दधत आ॒त्मन् ना॒त्मन् द॑धते पञ्चद॒शः प॑ञ्चद॒शो द॑धत आ॒त्मन् ना॒त्मन् द॑धते पञ्चद॒शः । \newline
61. द॒ध॒ते॒ प॒ञ्च॒द॒शः प॑ञ्चद॒शो द॑धते दधते पञ्चद॒श इ॑न्द्रस्तो॒म इ॑न्द्रस्तो॒मः प॑ञ्चद॒शो द॑धते दधते पञ्चद॒श इ॑न्द्रस्तो॒मः । \newline
62. प॒ञ्च॒द॒श इ॑न्द्रस्तो॒म इ॑न्द्रस्तो॒मः प॑ञ्चद॒शः प॑ञ्चद॒श इ॑न्द्रस्तो॒मो भ॑वति भवतीन्द्रस्तो॒मः प॑ञ्चद॒शः प॑ञ्चद॒श इ॑न्द्रस्तो॒मो भ॑वति । \newline
63. प॒ञ्च॒द॒श इति॑ पञ्च - द॒शः । \newline
64. इ॒न्द्र॒स्तो॒मो भ॑वति भवतीन्द्रस्तो॒म इ॑न्द्रस्तो॒मो भ॑वतीन्द्रि॒य मि॑न्द्रि॒यम् भ॑वतीन्द्रस्तो॒म इ॑न्द्रस्तो॒मो 
भ॑वतीन्द्रि॒यम् । \newline
65. इ॒न्द्र॒स्तो॒म इती᳚न्द्र - स्तो॒मः । \newline
66. भ॒व॒ती॒न्द्रि॒य मि॑न्द्रि॒यम् भ॑वति भवतीन्द्रि॒य मे॒वैवेन्द्रि॒यम् भ॑वति भवतीन्द्रि॒य मे॒व । \newline
67. इ॒न्द्रि॒य मे॒वैवेन्द्रि॒य मि॑न्द्रि॒य मे॒वा वावै॒वेन्द्रि॒य मि॑न्द्रि॒य मे॒वाव॑ । \newline
68. ए॒वावा वै॒वै वाव॑ रुन्धते रुन्ध॒ते ऽवै॒वै वाव॑ रुन्धते । \newline
69. अव॑ रुन्धते रुन्ध॒ते ऽवाव॑ रुन्धते सप्तद॒शः स॑प्तद॒शो रु॑न्ध॒ते ऽवाव॑ रुन्धते सप्तद॒शः । \newline
\pagebreak
\markright{ TS 7.4.3.4  \hfill https://www.vedavms.in \hfill}

\section{ TS 7.4.3.4 }

\textbf{TS 7.4.3.4 } \newline
\textbf{Samhita Paata} \newline

रुन्धते सप्तद॒शो भ॑वत्य॒न्नाद्य॒स्या व॑रुद्ध्या॒ अथो॒ प्रैव तेन॑ जायन्त एकविꣳ॒॒शो भ॑वति॒ प्रति॑ष्ठित्या॒ अथो॒ रुच॑मे॒वाऽऽ*त्मन् द॑धते चतुर्विꣳ॒॒शो भ॑वति॒ अतु॑र्विꣳशतिरर्द्धमा॒साः सं॑ॅवथ्स॒रः सं॑ॅवथ्स॒रः सु॑व॒र्गो लो॒कः सं॑ॅवथ्स॒र ए॒व सु॑व॒र्गे लो॒के प्रति॑ तिष्ठ॒न्त्यथो॑ ए॒ष वै वि॑षू॒वान् वि॑षू॒वन्तो॑ भवन्ति॒ य ए॒वं ॅवि॒द्वाꣳस॑ ए॒ता आस॑ते चतुर्विꣳ॒॒शात् पृ॒ष्ठान्युप॑ यन्ति संॅवथ्स॒र ए॒व प्र॑ति॒ष्ठाय॑ - [  ] \newline

\textbf{Pada Paata} \newline

रु॒न्ध॒ते॒ । स॒प्त॒द॒श इति॑ सप्त - द॒शः । भ॒व॒ति॒ । अ॒न्नाद्य॒स्येत्य॑न्न- अद्य॑स्य । अव॑रुद्ध्या॒ इत्यव॑-रु॒द्ध्यै॒ । अथो॒ इति॑ । प्रेति॑ । ए॒व । तेन॑ । जा॒य॒न्ते॒ । ए॒क॒विꣳ॒॒श इत्ये॑क - विꣳ॒॒शः । भ॒व॒ति॒ । प्रति॑ष्ठित्या॒ इति॒ प्रति॑ - स्थि॒त्यै॒ । अथो॒ इति॑ । रुच᳚म् । ए॒व । आ॒त्मन्न् । द॒ध॒ते॒ । च॒तु॒र्विꣳ॒॒श इति॑ चतुः - विꣳ॒॒शः । भ॒व॒ति॒ । चतु॑र्विꣳशति॒रिति॒ चतुः॑ - विꣳ॒॒श॒तिः॒ । अ॒द्‌र्ध॒मा॒सा इत्य॑द्‌र्ध - मा॒साः । सं॒ॅव॒थ्स॒र इति॑ सं - व॒थ्स॒रः । सं॒ॅव॒थ्स॒र इति॑ सं - व॒थ्स॒रः । सु॒व॒र्ग इति॑ सुवः - गः । लो॒कः । सं॒ॅव॒थ्स॒र इति॑ सं - व॒थ्स॒रे । ए॒व । सु॒व॒र्ग इति॑ सुवः - गे । लो॒के । प्रतीति॑ । ति॒ष्ठ॒न्ति॒ । अथो॒ इति॑ । ए॒षः । वै । वि॒षू॒वानिति॑ विषु - वान् । वि॒षू॒वन्त॒ इति॑ विषु - वन्तः॑ । भ॒व॒न्ति॒ । ये । ए॒वम् । वि॒द्वाꣳसः॑ । ए॒ताः । आस॑ते । च॒तु॒र्विꣳ॒॒शादिति॑ चतुः - विꣳ॒॒शात् । पृ॒ष्ठानि॑ । उपेति॑ । य॒न्ति॒ । सं॒ॅव॒थ्स॒र इति॑ सं - व॒थ्स॒रे । ए॒व । प्र॒ति॒ष्ठायेति॑ प्रति-स्थाय॑ ।  \newline


\textbf{Krama Paata} \newline

रु॒न्ध॒ते॒ स॒प्त॒द॒शः । स॒प्त॒द॒शो भ॑वति । स॒प्त॒द॒श इति॑ सप्त - द॒शः । भ॒व॒त्य॒न्नाद्य॑स्य । अ॒न्नाद्य॒स्याव॑रुद्ध्यै । अ॒न्नाद्य॒स्येत्य॑न्न - अद्य॑स्य । अव॑रुद्ध्या॒ अथो᳚ । अव॑रुद्ध्या॒ इत्यव॑ - रु॒द्ध्यै॒ । अथो॒ प्र । अथो॒ इत्यथो᳚ । प्रैव । ए॒व तेन॑ । तेन॑ जायन्ते । जा॒य॒न्त॒ ए॒क॒विꣳ॒॒शः । ए॒क॒विꣳ॒॒शो भ॑वति । ए॒क॒विꣳ॒॒श इत्ये॑क - विꣳ॒॒शः । भ॒व॒ति॒ प्रति॑ष्ठित्यै । प्रति॑ष्ठित्या॒ अथो᳚ । प्रति॑ष्ठित्या॒ इति॒ प्रति॑ - स्थि॒त्यै॒ । अथो॒ रुच᳚म् । अथो॒ इत्यथो᳚ । रुच॑मे॒व । ए॒वात्मन्न् । आ॒त्मन् द॑धते । द॒ध॒ते॒ च॒तु॒र्विꣳ॒॒शः । च॒तु॒र्विꣳ॒॒शो भ॑वति । 
च॒तु॒र्विꣳ॒॒श इति॑ चतुः - विꣳ॒॒शः । भ॒व॒ति॒ चतु॑र्विꣳशतिः । चतु॑र्विꣳशतिरर्द्धमा॒साः । चतु॑र्विꣳशति॒रिति॒ चतुः॑ - विꣳ॒॒श॒तिः॒ । अ॒र्द्ध॒मा॒साः स॑म्ॅवथ्स॒रः । अ॒र्द्ध॒मा॒सा इत्य॑र्द्ध - मा॒साः । स॒म्ॅव॒थ्स॒रः स॑म्ॅवथ्स॒रः । स॒म्ॅव॒थ्स॒र इति॑ सम् - व॒थ्स॒रः । स॒म्ॅव॒थ्स॒रः सु॑व॒र्गः । स॒म्ॅव॒थ्स॒र इति॑ सम् - व॒थ्स॒रः । सु॒व॒र्गो लो॒कः । सु॒व॒र्ग इति॑ सुवः - गः । लो॒कः स॑म्ॅवथ्स॒रे । स॒म्ॅव॒थ्स॒र ए॒व । स॒म्ॅव॒थ्स॒र इति॑ सम् - व॒थ्स॒रे । ए॒व सु॑व॒र्गे । सु॒व॒र्गे लो॒के । सु॒व॒र्ग इति॑ सुवः - गे । लो॒के प्रति॑ । प्रति॑ तिष्ठन्ति । ति॒ष्ठ॒न्त्यथो᳚ । अथो॑ ए॒षः । अथो॒ इत्यथो᳚ । ए॒ष वै । वै वि॑षू॒वान् । वि॒षू॒वान्. वि॑षू॒वन्तः॑ । वि॒षू॒वानिति॑ विषु - वान् । वि॒षू॒वन्तो॑ भवन्ति । वि॒षू॒वन्त॒ इति॑ विषु - वन्तः॑ । भ॒व॒न्ति॒ ये । य ए॒वम् । ए॒वम् ॅवि॒द्वाꣳसः॑ । वि॒द्वाꣳस॑ ए॒ताः । ए॒ता आस॑ते । आस॑ते चतुर्विꣳ॒॒शात् । च॒तु॒र्विꣳ॒॒शात् पृ॒ष्ठानि॑ । च॒तु॒र्विꣳ॒॒शादिति॑ चतुः - विꣳ॒॒शात् । पृ॒ष्ठान्युप॑ । उप॑ यन्ति । य॒न्ति॒ स॒म्ॅव॒थ्स॒रे । स॒म्ॅव॒थ्स॒र ए॒व । स॒म्ॅव॒थ्स॒र इति॑ सम् - व॒थ्स॒रे । ए॒व प्र॑ति॒ष्ठाय॑ । प्र॒ति॒ष्ठाय॑ दे॒वताः᳚ । प्र॒ति॒ष्ठायेति॑ प्रति - स्थाय॑ \newline

\textbf{Jatai Paata} \newline

1. रु॒न्ध॒ते॒ स॒प्त॒द॒शः स॑प्तद॒शो रु॑न्धते रुन्धते सप्तद॒शः । \newline
2. स॒प्त॒द॒शो भ॑वति भवति सप्तद॒शः स॑प्तद॒शो भ॑वति । \newline
3. स॒प्त॒द॒श इति॑ सप्त - द॒शः । \newline
4. भ॒व॒ त्य॒न्नाद्य॑स्या॒ न्नाद्य॑स्य भवति भव त्य॒न्नाद्य॑स्य । \newline
5. अ॒न्नाद्य॒स्या व॑रुद्ध्या॒ अव॑रुद्ध्या अ॒न्नाद्य॑ स्या॒न्नाद्य॒स्या व॑रुद्ध्यै । \newline
6. अ॒न्नाद्य॒स्येत्य॑न्न - अद्य॑स्य । \newline
7. अव॑रुद्ध्या॒ अथो॒ अथो॒ अव॑रुद्ध्या॒ अव॑रुद्ध्या॒ अथो᳚ । \newline
8. अव॑रुद्ध्या॒ इत्यव॑ - रु॒द्ध्यै॒ । \newline
9. अथो॒ प्र प्राथो॒ अथो॒ प्र । \newline
10. अथो॒ इत्यथो᳚ । \newline
11. प्रैवैव प्र प्रैव । \newline
12. ए॒व तेन॒ तेनै॒वैव तेन॑ । \newline
13. तेन॑ जायन्ते जायन्ते॒ तेन॒ तेन॑ जायन्ते । \newline
14. जा॒य॒न्त॒ ए॒क॒विꣳ॒॒श ए॑कविꣳ॒॒शो जा॑यन्ते जायन्त एकविꣳ॒॒शः । \newline
15. ए॒क॒विꣳ॒॒शो भ॑वति भव त्येकविꣳ॒॒श ए॑कविꣳ॒॒शो भ॑वति । \newline
16. ए॒क॒विꣳ॒॒श इत्ये॑क - विꣳ॒॒शः । \newline
17. भ॒व॒ति॒ प्रति॑ष्ठित्यै॒ प्रति॑ष्ठित्यै भवति भवति॒ प्रति॑ष्ठित्यै । \newline
18. प्रति॑ष्ठित्या॒ अथो॒ अथो॒ प्रति॑ष्ठित्यै॒ प्रति॑ष्ठित्या॒ अथो᳚ । \newline
19. प्रति॑ष्ठित्या॒ इति॒ प्रति॑ - स्थि॒त्यै॒ । \newline
20. अथो॒ रुचꣳ॒॒ रुच॒ मथो॒ अथो॒ रुच᳚म् । \newline
21. अथो॒ इत्यथो᳚ । \newline
22. रुच॑ मे॒वैव रुचꣳ॒॒ रुच॑ मे॒व । \newline
23. ए॒वात्मन् ना॒त्मन् ने॒वै वात्मन्न् । \newline
24. आ॒त्मन् द॑धते दधत आ॒त्मन् ना॒त्मन् द॑धते । \newline
25. द॒ध॒ते॒ च॒तु॒र्विꣳ॒॒श श्च॑तुर्विꣳ॒॒शो द॑धते दधते चतुर्विꣳ॒॒शः । \newline
26. च॒तु॒र्विꣳ॒॒शो भ॑वति भवति चतुर्विꣳ॒॒श श्च॑तुर्विꣳ॒॒शो भ॑वति । \newline
27. च॒तु॒र्विꣳ॒॒श इति॑ चतुः - विꣳ॒॒शः । \newline
28. भ॒व॒ति॒ चतु॑र्विꣳशति॒ श्चतु॑र्विꣳशतिर् भवति भवति॒ चतु॑र्विꣳशतिः । \newline
29. चतु॑र्विꣳशति रर्द्धमा॒सा अ॑र्द्धमा॒सा श्चतु॑र्विꣳशति॒ श्चतु॑र्विꣳशति रर्द्धमा॒साः । \newline
30. चतु॑र्विꣳशति॒रिति॒ चतुः॑ - विꣳ॒॒श॒तिः॒ । \newline
31. अ॒र्द्ध॒मा॒साः सं॑ॅवथ्स॒रः सं॑ॅवथ्स॒रो᳚ ऽर्द्धमा॒सा अ॑र्द्धमा॒साः सं॑ॅवथ्स॒रः । \newline
32. अ॒र्द्ध॒मा॒सा इत्य॑र्द्ध - मा॒साः । \newline
33. सं॒ॅव॒थ्स॒रः सं॑ॅवथ्स॒रः । \newline
34. सं॒ॅव॒थ्स॒र इति॑ सं - व॒थ्स॒रः । \newline
35. सं॒ॅव॒थ्स॒रः सु॑व॒र्गः सु॑व॒र्गः सं॑ॅवथ्स॒रः सं॑ॅवथ्स॒रः सु॑व॒र्गः । \newline
36. सं॒ॅव॒थ्स॒र इति॑ सं - व॒थ्स॒रः । \newline
37. सु॒व॒र्गो लो॒को लो॒कः सु॑व॒र्गः सु॑व॒र्गो लो॒कः । \newline
38. सु॒व॒र्ग इति॑ सुवः - गः । \newline
39. लो॒कः सं॑ॅवथ्स॒रे सं॑ॅवथ्स॒रे लो॒को लो॒कः सं॑ॅवथ्स॒रे । \newline
40. सं॒ॅव॒थ्स॒र ए॒वैव सं॑ॅवथ्स॒रे सं॑ॅवथ्स॒र ए॒व । \newline
41. सं॒ॅव॒थ्स॒र इति॑ सं - व॒थ्स॒रे । \newline
42. ए॒व सु॑व॒र्गे सु॑व॒र्ग ए॒वैव सु॑व॒र्गे । \newline
43. सु॒व॒र्गे लो॒के लो॒के सु॑व॒र्गे सु॑व॒र्गे लो॒के । \newline
44. सु॒व॒र्ग इति॑ सुवः - गे । \newline
45. लो॒के प्रति॒ प्रति॑ लो॒के लो॒के प्रति॑ । \newline
46. प्रति॑ तिष्ठन्ति तिष्ठन्ति॒ प्रति॒ प्रति॑ तिष्ठन्ति । \newline
47. ति॒ष्ठ॒न्त्यथो॒ अथो॑ तिष्ठन्ति तिष्ठ॒न्त्यथो᳚ । \newline
48. अथो॑ ए॒ष ए॒षो ऽथो॒ अथो॑ ए॒षः । \newline
49. अथो॒ इत्यथो᳚ । \newline
50. ए॒ष वै वा ए॒ष ए॒ष वै । \newline
51. वै वि॑षू॒वान्. वि॑षू॒वान्. वै वै वि॑षू॒वान् । \newline
52. वि॒षू॒वान्. वि॑षू॒वन्तो॑ विषू॒वन्तो॑ विषू॒वान्. वि॑षू॒वान्. वि॑षू॒वन्तः॑ । \newline
53. वि॒षू॒वानिति॑ विषु - वान् । \newline
54. वि॒षू॒वन्तो॑ भवन्ति भवन्ति विषू॒वन्तो॑ विषू॒वन्तो॑ भवन्ति । \newline
55. वि॒षू॒वन्त॒ इति॑ विषु - वन्तः॑ । \newline
56. भ॒व॒न्ति॒ ये ये भ॑वन्ति भवन्ति॒ ये । \newline
57. य ए॒व मे॒वं ॅये य ए॒वम् । \newline
58. ए॒वं ॅवि॒द्वाꣳसो॑ वि॒द्वाꣳस॑ ए॒व मे॒वं ॅवि॒द्वाꣳसः॑ । \newline
59. वि॒द्वाꣳस॑ ए॒ता ए॒ता वि॒द्वाꣳसो॑ वि॒द्वाꣳस॑ ए॒ताः । \newline
60. ए॒ता आस॑त॒ आस॑त ए॒ता ए॒ता आस॑ते । \newline
61. आस॑ते चतुर्विꣳ॒॒शाच् च॑तुर्विꣳ॒॒शा दास॑त॒ आस॑ते चतुर्विꣳ॒॒शात् । \newline
62. च॒तु॒र्विꣳ॒॒शात् पृ॒ष्ठानि॑ पृ॒ष्ठानि॑ चतुर्विꣳ॒॒शाच् च॑तुर्विꣳ॒॒शात् पृ॒ष्ठानि॑ । \newline
63. च॒तु॒र्विꣳ॒॒शादिति॑ चतुः - विꣳ॒॒शात् । \newline
64. पृ॒ष्ठा न्युपोप॑ पृ॒ष्ठानि॑ पृ॒ष्ठा न्युप॑ । \newline
65. उप॑ यन्ति य॒न्त्युपोप॑ यन्ति । \newline
66. य॒न्ति॒ सं॒ॅव॒थ्स॒रे सं॑ॅवथ्स॒रे य॑न्ति यन्ति संॅवथ्स॒रे । \newline
67. सं॒ॅव॒थ्स॒र ए॒वैव सं॑ॅवथ्स॒रे सं॑ॅवथ्स॒र ए॒व । \newline
68. सं॒ॅव॒थ्स॒र इति॑ सं - व॒थ्स॒रे । \newline
69. ए॒व प्र॑ति॒ष्ठाय॑ प्रति॒ष्ठायै॒वैव प्र॑ति॒ष्ठाय॑ । \newline
70. प्र॒ति॒ष्ठाय॑ दे॒वता॑ दे॒वताः᳚ प्रति॒ष्ठाय॑ प्रति॒ष्ठाय॑ दे॒वताः᳚ । \newline
71. प्र॒ति॒ष्ठायेति॑ प्रति - स्थाय॑ । \newline

\textbf{Ghana Paata } \newline

1. रु॒न्ध॒ते॒ स॒प्त॒द॒शः स॑प्तद॒शो रु॑न्धते रुन्धते सप्तद॒शो भ॑वति भवति सप्तद॒शो रु॑न्धते रुन्धते सप्तद॒शो भ॑वति । \newline
2. स॒प्त॒द॒शो भ॑वति भवति सप्तद॒शः स॑प्तद॒शो भ॑व त्य॒न्नाद्य॑स्या॒ न्नाद्य॑स्य भवति सप्तद॒शः स॑प्तद॒शो भ॑व त्य॒न्नाद्य॑स्य । \newline
3. स॒प्त॒द॒श इति॑ सप्त - द॒शः । \newline
4. भ॒व॒ त्य॒न्नाद्य॑स्या॒ न्नाद्य॑स्य भवति भव त्य॒न्नाद्य॒स्या व॑रुद्ध्या॒ अव॑रुद्ध्या अ॒न्नाद्य॑स्य भवति भव त्य॒न्नाद्य॒स्या व॑रुद्ध्यै । \newline
5. अ॒न्नाद्य॒स्या व॑रुद्ध्या॒ अव॑रुद्ध्या अ॒न्नाद्य॑स्या॒ न्नाद्य॒स्या व॑रुद्ध्या॒ अथो॒ अथो॒ अव॑रुद्ध्या अ॒न्नाद्य॑स्या॒ न्नाद्य॒स्या व॑रुद्ध्या॒ अथो᳚ । \newline
6. अ॒न्नाद्य॒स्येत्य॑न्न - अद्य॑स्य । \newline
7. अव॑रुद्ध्या॒ अथो॒ अथो॒ अव॑रुद्ध्या॒ अव॑रुद्ध्या॒ अथो॒ प्र प्राथो॒ अव॑रुद्ध्या॒ अव॑रुद्ध्या॒ अथो॒ प्र । \newline
8. अव॑रुद्ध्या॒ इत्यव॑ - रु॒द्ध्यै॒ । \newline
9. अथो॒ प्र प्राथो॒ अथो॒ प्रैवैव प्राथो॒ अथो॒ प्रैव । \newline
10. अथो॒ इत्यथो᳚ । \newline
11. प्रैवैव प्र प्रैव तेन॒ तेनै॒व प्र प्रैव तेन॑ । \newline
12. ए॒व तेन॒ तेनै॒वैव तेन॑ जायन्ते जायन्ते॒ तेनै॒वैव तेन॑ जायन्ते । \newline
13. तेन॑ जायन्ते जायन्ते॒ तेन॒ तेन॑ जायन्त एकविꣳ॒॒श ए॑कविꣳ॒॒शो जा॑यन्ते॒ तेन॒ तेन॑ जायन्त एकविꣳ॒॒शः । \newline
14. जा॒य॒न्त॒ ए॒क॒विꣳ॒॒श ए॑कविꣳ॒॒शो जा॑यन्ते जायन्त एकविꣳ॒॒शो भ॑वति भव त्येकविꣳ॒॒शो जा॑यन्ते जायन्त एकविꣳ॒॒शो भ॑वति । \newline
15. ए॒क॒विꣳ॒॒शो भ॑वति भव त्येकविꣳ॒॒श ए॑कविꣳ॒॒शो भ॑वति॒ प्रति॑ष्ठित्यै॒ प्रति॑ष्ठित्यै भव त्येकविꣳ॒॒श ए॑कविꣳ॒॒शो भ॑वति॒ प्रति॑ष्ठित्यै । \newline
16. ए॒क॒विꣳ॒॒श इत्ये॑क - विꣳ॒॒शः । \newline
17. भ॒व॒ति॒ प्रति॑ष्ठित्यै॒ प्रति॑ष्ठित्यै भवति भवति॒ प्रति॑ष्ठित्या॒ अथो॒ अथो॒ प्रति॑ष्ठित्यै भवति भवति॒ प्रति॑ष्ठित्या॒ अथो᳚ । \newline
18. प्रति॑ष्ठित्या॒ अथो॒ अथो॒ प्रति॑ष्ठित्यै॒ प्रति॑ष्ठित्या॒ अथो॒ रुचꣳ॒॒ रुच॒ मथो॒ प्रति॑ष्ठित्यै॒ प्रति॑ष्ठित्या॒ अथो॒ रुच᳚म् । \newline
19. प्रति॑ष्ठित्या॒ इति॒ प्रति॑ - स्थि॒त्यै॒ । \newline
20. अथो॒ रुचꣳ॒॒ रुच॒ मथो॒ अथो॒ रुच॑ मे॒वैव रुच॒ मथो॒ अथो॒ रुच॑ मे॒व । \newline
21. अथो॒ इत्यथो᳚ । \newline
22. रुच॑ मे॒वैव रुचꣳ॒॒ रुच॑ मे॒वात्मन् ना॒त्मन् ने॒व रुचꣳ॒॒ रुच॑ मे॒वात्मन्न् । \newline
23. ए॒वात्मन् ना॒त्मन् ने॒वै वात्मन् द॑धते दधत आ॒त्मन् ने॒वै वात्मन् द॑धते । \newline
24. आ॒त्मन् द॑धते दधत आ॒त्मन् ना॒त्मन् द॑धते चतुर्विꣳ॒॒श श्च॑तुर्विꣳ॒॒शो द॑धत आ॒त्मन् ना॒त्मन् द॑धते चतुर्विꣳ॒॒शः । \newline
25. द॒ध॒ते॒ च॒तु॒र्विꣳ॒॒श श्च॑तुर्विꣳ॒॒शो द॑धते दधते चतुर्विꣳ॒॒शो भ॑वति भवति चतुर्विꣳ॒॒शो द॑धते दधते चतुर्विꣳ॒॒शो भ॑वति । \newline
26. च॒तु॒र्विꣳ॒॒शो भ॑वति भवति चतुर्विꣳ॒॒श श्च॑तुर्विꣳ॒॒शो भ॑वति॒ चतु॑र्विꣳशति॒ श्चतु॑र्विꣳशतिर् भवति चतुर्विꣳ॒॒श श्च॑तुर्विꣳ॒॒शो भ॑वति॒ चतु॑र्विꣳशतिः । \newline
27. च॒तु॒र्विꣳ॒॒श इति॑ चतुः - विꣳ॒॒शः । \newline
28. भ॒व॒ति॒ चतु॑र्विꣳशति॒ श्चतु॑र्विꣳशतिर् भवति भवति॒ चतु॑र्विꣳशति रर्द्धमा॒सा अ॑र्द्धमा॒सा श्चतु॑र्विꣳशतिर् भवति भवति॒ चतु॑र्विꣳशति रर्द्धमा॒साः । \newline
29. चतु॑र्विꣳशति रर्द्धमा॒सा अ॑र्द्धमा॒सा श्चतु॑र्विꣳशति॒ श्चतु॑र्विꣳशति रर्द्धमा॒साः सं॑ॅवथ्स॒रः सं॑ॅवथ्स॒रो᳚ ऽर्द्धमा॒सा श्चतु॑र्विꣳशति॒ श्चतु॑र्विꣳशति रर्द्धमा॒साः सं॑ॅवथ्स॒रः । \newline
30. चतु॑र्विꣳशति॒रिति॒ चतुः॑ - विꣳ॒॒श॒तिः॒ । \newline
31. अ॒र्द्ध॒मा॒साः सं॑ॅवथ्स॒रः सं॑ॅवथ्स॒रो᳚ ऽर्द्धमा॒सा अ॑र्द्धमा॒साः सं॑ॅवथ्स॒रः । \newline
32. अ॒र्द्ध॒मा॒सा इत्य॑र्द्ध - मा॒साः । \newline
33. सं॒ॅव॒थ्स॒रः सं॑ॅवथ्स॒रः । \newline
34. सं॒ॅव॒थ्स॒र इति॑ सं - व॒थ्स॒रः । \newline
35. सं॒ॅव॒थ्स॒रः सु॑व॒र्गः सु॑व॒र्गः सं॑ॅवथ्स॒रः सं॑ॅवथ्स॒रः सु॑व॒र्गो लो॒को लो॒कः सु॑व॒र्गः सं॑ॅवथ्स॒रः सं॑ॅवथ्स॒रः सु॑व॒र्गो लो॒कः । \newline
36. सं॒ॅव॒थ्स॒र इति॑ सं - व॒थ्स॒रः । \newline
37. सु॒व॒र्गो लो॒को लो॒कः सु॑व॒र्गः सु॑व॒र्गो लो॒कः सं॑ॅवथ्स॒रे सं॑ॅवथ्स॒रे लो॒कः सु॑व॒र्गः सु॑व॒र्गो लो॒कः सं॑ॅवथ्स॒रे । \newline
38. सु॒व॒र्ग इति॑ सुवः - गः । \newline
39. लो॒कः सं॑ॅवथ्स॒रे सं॑ॅवथ्स॒रे लो॒को लो॒कः सं॑ॅवथ्स॒र ए॒वैव सं॑ॅवथ्स॒रे लो॒को लो॒कः सं॑ॅवथ्स॒र ए॒व । \newline
40. सं॒ॅव॒थ्स॒र ए॒वैव सं॑ॅवथ्स॒रे सं॑ॅवथ्स॒र ए॒व सु॑व॒र्गे सु॑व॒र्ग ए॒व सं॑ॅवथ्स॒रे सं॑ॅवथ्स॒र ए॒व सु॑व॒र्गे । \newline
41. सं॒ॅव॒थ्स॒र इति॑ सं - व॒थ्स॒रे । \newline
42. ए॒व सु॑व॒र्गे सु॑व॒र्ग ए॒वैव सु॑व॒र्गे लो॒के लो॒के सु॑व॒र्ग ए॒वैव सु॑व॒र्गे लो॒के । \newline
43. सु॒व॒र्गे लो॒के लो॒के सु॑व॒र्गे सु॑व॒र्गे लो॒के प्रति॒ प्रति॑ लो॒के सु॑व॒र्गे सु॑व॒र्गे लो॒के प्रति॑ । \newline
44. सु॒व॒र्ग इति॑ सुवः - गे । \newline
45. लो॒के प्रति॒ प्रति॑ लो॒के लो॒के प्रति॑ तिष्ठन्ति तिष्ठन्ति॒ प्रति॑ लो॒के लो॒के प्रति॑ तिष्ठन्ति । \newline
46. प्रति॑ तिष्ठन्ति तिष्ठन्ति॒ प्रति॒ प्रति॑ तिष्ठ॒ न्त्यथो॒ अथो॑ तिष्ठन्ति॒ प्रति॒ प्रति॑ तिष्ठ॒ न्त्यथो᳚ । \newline
47. ति॒ष्ठ॒ न्त्यथो॒ अथो॑ तिष्ठन्ति तिष्ठ॒ न्त्यथो॑ ए॒ष ए॒षो ऽथो॑ तिष्ठन्ति तिष्ठ॒ न्त्यथो॑ ए॒षः । \newline
48. अथो॑ ए॒ष ए॒षो ऽथो॒ अथो॑ ए॒ष वै वा ए॒षो ऽथो॒ अथो॑ ए॒ष वै । \newline
49. अथो॒ इत्यथो᳚ । \newline
50. ए॒ष वै वा ए॒ष ए॒ष वै वि॑षू॒वान्. वि॑षू॒वान्. वा ए॒ष ए॒ष वै वि॑षू॒वान् । \newline
51. वै वि॑षू॒वान्. वि॑षू॒वान्. वै वै वि॑षू॒वान्. वि॑षू॒वन्तो॑ विषू॒वन्तो॑ विषू॒वान्. वै वै वि॑षू॒वान्. वि॑षू॒वन्तः॑ । \newline
52. वि॒षू॒वान्. वि॑षू॒वन्तो॑ विषू॒वन्तो॑ विषू॒वान्. वि॑षू॒वान्. वि॑षू॒वन्तो॑ भवन्ति भवन्ति विषू॒वन्तो॑ विषू॒वान्. वि॑षू॒वान्. वि॑षू॒वन्तो॑ भवन्ति । \newline
53. वि॒षू॒वानिति॑ विषु - वान् । \newline
54. वि॒षू॒वन्तो॑ भवन्ति भवन्ति विषू॒वन्तो॑ विषू॒वन्तो॑ भवन्ति॒ ये ये भ॑वन्ति विषू॒वन्तो॑ विषू॒वन्तो॑ भवन्ति॒ ये । \newline
55. वि॒षू॒वन्त॒ इति॑ विषु - वन्तः॑ । \newline
56. भ॒व॒न्ति॒ ये ये भ॑वन्ति भवन्ति॒ य ए॒व मे॒वं ॅये भ॑वन्ति भवन्ति॒ य ए॒वम् । \newline
57. य ए॒व मे॒वं ॅये य ए॒वं ॅवि॒द्वाꣳसो॑ वि॒द्वाꣳस॑ ए॒वं ॅये य ए॒वं ॅवि॒द्वाꣳसः॑ । \newline
58. ए॒वं ॅवि॒द्वाꣳसो॑ वि॒द्वाꣳस॑ ए॒व मे॒वं ॅवि॒द्वाꣳस॑ ए॒ता ए॒ता वि॒द्वाꣳस॑ ए॒व मे॒वं ॅवि॒द्वाꣳस॑ ए॒ताः । \newline
59. वि॒द्वाꣳस॑ ए॒ता ए॒ता वि॒द्वाꣳसो॑ वि॒द्वाꣳस॑ ए॒ता आस॑त॒ आस॑त ए॒ता वि॒द्वाꣳसो॑ वि॒द्वाꣳस॑ ए॒ता आस॑ते । \newline
60. ए॒ता आस॑त॒ आस॑त ए॒ता ए॒ता आस॑ते चतुर्विꣳ॒॒शाच् च॑तुर्विꣳ॒॒शा दास॑त ए॒ता ए॒ता आस॑ते चतुर्विꣳ॒॒शात् । \newline
61. आस॑ते चतुर्विꣳ॒॒शाच् च॑तुर्विꣳ॒॒शा दास॑त॒ आस॑ते चतुर्विꣳ॒॒शात् पृ॒ष्ठानि॑ पृ॒ष्ठानि॑ चतुर्विꣳ॒॒शा दास॑त॒ आस॑ते चतुर्विꣳ॒॒शात् पृ॒ष्ठानि॑ । \newline
62. च॒तु॒र्विꣳ॒॒शात् पृ॒ष्ठानि॑ पृ॒ष्ठानि॑ चतुर्विꣳ॒॒शाच् च॑तुर्विꣳ॒॒शात् पृ॒ष्ठा न्युपोप॑ पृ॒ष्ठानि॑ चतुर्विꣳ॒॒शाच् च॑तुर्विꣳ॒॒शात् पृ॒ष्ठा न्युप॑ । \newline
63. च॒तु॒र्विꣳ॒॒शादिति॑ चतुः - विꣳ॒॒शात् । \newline
64. पृ॒ष्ठा न्युपोप॑ पृ॒ष्ठानि॑ पृ॒ष्ठा न्युप॑ यन्ति य॒न्त्युप॑ पृ॒ष्ठानि॑ पृ॒ष्ठा न्युप॑ यन्ति । \newline
65. उप॑ यन्ति य॒न्त्युपोप॑ यन्ति संॅवथ्स॒रे सं॑ॅवथ्स॒रे य॒न्त्युपोप॑ यन्ति संॅवथ्स॒रे । \newline
66. य॒न्ति॒ सं॒ॅव॒थ्स॒रे सं॑ॅवथ्स॒रे य॑न्ति यन्ति संॅवथ्स॒र ए॒वैव सं॑ॅवथ्स॒रे य॑न्ति यन्ति संॅवथ्स॒र ए॒व । \newline
67. सं॒ॅव॒थ्स॒र ए॒वैव सं॑ॅवथ्स॒रे सं॑ॅवथ्स॒र ए॒व प्र॑ति॒ष्ठाय॑ प्रति॒ष्ठायै॒व सं॑ॅवथ्स॒रे सं॑ॅवथ्स॒र ए॒व प्र॑ति॒ष्ठाय॑ । \newline
68. सं॒ॅव॒थ्स॒र इति॑ सं - व॒थ्स॒रे । \newline
69. ए॒व प्र॑ति॒ष्ठाय॑ प्रति॒ष्ठायै॒वैव प्र॑ति॒ष्ठाय॑ दे॒वता॑ दे॒वताः᳚ प्रति॒ष्ठायै॒वैव प्र॑ति॒ष्ठाय॑ दे॒वताः᳚ । \newline
70. प्र॒ति॒ष्ठाय॑ दे॒वता॑ दे॒वताः᳚ प्रति॒ष्ठाय॑ प्रति॒ष्ठाय॑ दे॒वता॑ अ॒भ्यारो॑ह न्त्य॒भ्यारो॑हन्ति दे॒वताः᳚ प्रति॒ष्ठाय॑ प्रति॒ष्ठाय॑ दे॒वता॑ अ॒भ्यारो॑हन्ति । \newline
71. प्र॒ति॒ष्ठायेति॑ प्रति - स्थाय॑ । \newline
\pagebreak
\markright{ TS 7.4.3.5  \hfill https://www.vedavms.in \hfill}

\section{ TS 7.4.3.5 }

\textbf{TS 7.4.3.5 } \newline
\textbf{Samhita Paata} \newline

दे॒वता॑ अ॒भ्यारो॑हन्ति त्रयस्त्रिꣳ॒॒शात् त्र॑यस्त्रिꣳ॒॒शमुप॑ यन्ति॒ त्रय॑स्त्रिꣳश॒द्वै दे॒वता॑ दे॒वता᳚स्वे॒व प्रति॑ तिष्ठन्ति त्रिण॒वो भ॑वती॒मे वै लो॒कास्त्रि॑ण॒व ए॒ष्वे॑व लो॒केषु॒ प्रति॑ तिष्ठन्ति॒ द्वावे॑कविꣳ॒॒शौ भ॑वतः॒ प्रति॑ष्ठित्या॒ अथो॒ रुच॑मे॒वाऽऽ*त्मन् द॑धते ब॒हवः॑ षोड॒शिनो॑ भवन्ति॒ तस्मा᳚द्-ब॒हवः॑ प्र॒जासु॒ वृषा॑णो॒ यदे॒ते स्तोमा॒ व्यति॑षक्ता॒ भव॑न्ति॒ तस्मा॑दि॒य -मोष॑धीभि॒-र्वन॒स्पति॑भि॒-र्व्यति॑षक्ता॒ - [  ] \newline

\textbf{Pada Paata} \newline

दे॒वताः᳚ । अ॒भ्यारो॑ह॒न्तीत्य॑भि - आरो॑हन्ति । त्र॒य॒स्त्रिꣳ॒॒शादिति॑ त्रयः - त्रिꣳ॒॒शात् । त्र॒य॒स्त्रिꣳ॒॒शमिति॑ त्रय - त्रिꣳ॒॒शम् । उपेति॑ । य॒न्ति॒ । त्रय॑स्त्रिꣳश॒दिति॒ त्रयः॑ - त्रिꣳ॒॒श॒त् । वै । दे॒वताः᳚ । दे॒वता॑सु । ए॒व । प्रतीति॑ । ति॒ष्ठ॒न्ति॒ । त्रि॒ण॒व इति॑ त्रि - न॒वः । भ॒व॒ति॒ । इ॒मे । वै । लो॒काः । त्रि॒ण॒व इति॑ त्रि - न॒वः । ए॒षु । ए॒व । लो॒केषु॑ । प्रतीति॑ । ति॒ष्ठ॒न्ति॒ । द्वौ । ए॒क॒विꣳ॒॒शावित्ये॑क - विꣳ॒॒शौ । भ॒व॒तः॒ । प्रति॑ष्ठित्या॒ इति॒ प्रति॑ - स्थि॒त्यै॒ । अथो॒ इति॑ । रुच᳚म् । ए॒व । आ॒त्मन्न् । द॒ध॒ते॒ । ब॒हवः॑ । षो॒ड॒शिनः॑ । भ॒व॒न्ति॒ । तस्मा᳚त् । ब॒हवः॑ । प्र॒जास्विति॑ प्र - जासु॑ । वृषा॑णः । यत् । ए॒ते । स्तोमाः᳚ । व्यति॑षक्ता॒ इति॑ वि - अति॑षक्ताः । भव॑न्ति । तस्मा᳚त् । इ॒यम् । ओष॑धीभि॒रित्योष॑धि - भिः॒ । वन॒स्पति॑भि॒रिति॒ वन॒स्पति॑ - भिः॒ । व्यति॑ष॒क्तेति॑ वि - अति॑षक्ता ।  \newline


\textbf{Krama Paata} \newline

दे॒वता॑ अ॒भ्यारो॑हन्ति । अ॒भ्यारो॑हन्ति त्रयस्त्रिꣳ॒॒शात् । अ॒भ्यारो॑ह॒न्तीत्य॑भि - आरो॑हन्ति । त्र॒य॒स्त्रिꣳ॒॒शात् 
त्र॑यस्त्रिꣳ॒॒शम् । त्र॒य॒स्त्रिꣳ॒॒शादिति॑ त्रयः - त्रिꣳ॒॒शात् । त्र॒य॒स्त्रिꣳ॒॒शमुप॑ । त्र॒य॒स्त्रिꣳ॒॒शमिति॑ त्रयः - त्रिꣳ॒॒शम् । उप॑ यन्ति । य॒न्ति॒ त्रय॑स्त्रिꣳशत् । त्रय॑स्त्रिꣳश॒द् वै । त्रय॑स्त्रिꣳश॒दिति॒ त्रयः॑ - त्रिꣳ॒॒श॒त्॒ । वै दे॒वताः᳚ । दे॒वता॑ दे॒वता॑सु । दे॒वता᳚स्वे॒व । ए॒व प्रति॑ । प्रति॑ तिष्ठन्ति । ति॒ष्ठ॒न्ति॒ त्रि॒ण॒वः । त्रि॒ण॒वो भ॑वति । त्रि॒ण॒व इति॑ त्रि - न॒वः । भ॒व॒ती॒मे । इ॒मे वै । वै लो॒काः । लो॒कास्त्रि॑ण॒वः । त्रि॒ण॒व ए॒षु । त्रि॒ण॒व इति॑ त्रि - न॒वः । ए॒ष्वे॑व । ए॒व लो॒केषु॑ । लो॒केषु॒ प्रति॑ । प्रति॑ तिष्ठन्ति । ति॒ष्ठ॒न्ति॒ द्वौ । द्वावे॑कविꣳ॒॒शौ । ए॒क॒विꣳ॒॒शौ भ॑वतः । ए॒क॒विꣳ॒॒शावित्ये॑क - विꣳ॒॒शौ । भ॒व॒तः॒ प्रति॑ष्ठित्यै । प्रति॑ष्ठित्या॒ अथो᳚ । प्रति॑ष्ठित्या॒ इति॒ प्रति॑ - स्थि॒त्यै॒ । अथो॒ रुच᳚म् । अथो॒ इत्यथो᳚ । रुच॑मे॒व । ए॒वात्मन्न् । आ॒त्मन् द॑धते । द॒ध॒ते॒ ब॒हवः॑ । ब॒हवः॑ षोड॒शिनः॑ । षो॒ड॒शिनो॑ भवन्ति । भ॒व॒न्ति॒ तस्मा᳚त् । तस्मा᳚द् ब॒हवः॑ । ब॒हवः॑ प्र॒जासु॑ । प्र॒जासु॒ वृषा॑णः । प्र॒जास्विति॑ प्र - जासु॑ । वृषा॑णो॒ यत् । यदे॒ते । ए॒ते स्तोमाः᳚ । स्तोमा॒ व्यति॑षक्ताः । व्यति॑षक्ता॒ भव॑न्ति । व्यति॑षक्ता॒ इति॑ वि - अति॑षक्ताः । भव॑न्ति॒ तस्मा᳚त् । तस्मा॑दि॒यम् । इ॒यमोष॑धीभिः । ओष॑धीभि॒र् वन॒स्पति॑भिः । ओष॑धीभि॒रित्योष॑धि - भिः॒ । वन॒स्पति॑भि॒र् व्यति॑षक्ता ( ) । वन॒स्पति॑भि॒रिति॒ वन॒स्पति॑ - भिः॒ । व्यति॑षक्ता॒ व्यति॑षज्यन्ते । व्यति॑ष॒क्तेति॑ वि - अति॑षक्ता \newline

\textbf{Jatai Paata} \newline

1. दे॒वता॑ अ॒भ्यारो॑ह न्त्य॒भ्यारो॑हन्ति दे॒वता॑ दे॒वता॑ अ॒भ्यारो॑हन्ति । \newline
2. अ॒भ्यारो॑हन्ति त्रयस्त्रिꣳ॒॒शात् त्र॑यस्त्रिꣳ॒॒शा द॒भ्यारो॑हन् त्य॒भ्यारो॑हन्ति त्रयस्त्रिꣳ॒॒शात् । \newline
3. अ॒भ्यारो॑ह॒न्तीत्य॑भि - आरो॑हन्ति । \newline
4. त्र॒य॒स्त्रिꣳ॒॒शात् त्र॑यस्त्रिꣳ॒॒शम् त्र॑यस्त्रिꣳ॒॒शम् त्र॑यस्त्रिꣳ॒॒शात् त्र॑यस्त्रिꣳ॒॒शात् त्र॑यस्त्रिꣳ॒॒शम् । \newline
5. त्र॒य॒स्त्रिꣳ॒॒शादिति॑ त्रयः - त्रिꣳ॒॒शात् । \newline
6. त्र॒य॒स्त्रिꣳ॒॒श मुपोप॑ त्रयस्त्रिꣳ॒॒शम् त्र॑यस्त्रिꣳ॒॒श मुप॑ । \newline
7. त्र॒य॒स्त्रिꣳ॒॒शमिति॑ त्रय - त्रिꣳ॒॒शम् । \newline
8. उप॑ यन्ति य॒न्त्युपोप॑ यन्ति । \newline
9. य॒न्ति॒ त्रय॑स्त्रिꣳश॒त् त्रय॑स्त्रिꣳशद् यन्ति यन्ति॒ त्रय॑स्त्रिꣳशत् । \newline
10. त्रय॑स्त्रिꣳश॒द् वै वै त्रय॑स्त्रिꣳश॒त् त्रय॑स्त्रिꣳश॒द् वै । \newline
11. त्रय॑स्त्रिꣳश॒दिति॒ त्रयः॑ - त्रिꣳ॒॒श॒त् । \newline
12. वै दे॒वता॑ दे॒वता॒ वै वै दे॒वताः᳚ । \newline
13. दे॒वता॑ दे॒वता॑सु दे॒वता॑सु दे॒वता॑ दे॒वता॑ दे॒वता॑सु । \newline
14. दे॒वता᳚ स्वे॒वैव दे॒वता॑सु दे॒वता᳚ स्वे॒व । \newline
15. ए॒व प्रति॒ प्रत्ये॒वैव प्रति॑ । \newline
16. प्रति॑ तिष्ठन्ति तिष्ठन्ति॒ प्रति॒ प्रति॑ तिष्ठन्ति । \newline
17. ति॒ष्ठ॒न्ति॒ त्रि॒ण॒व स्त्रि॑ण॒व स्ति॑ष्ठन्ति तिष्ठन्ति त्रिण॒वः । \newline
18. त्रि॒ण॒वो भ॑वति भवति त्रिण॒व स्त्रि॑ण॒वो भ॑वति । \newline
19. त्रि॒ण॒व इति॑ त्रि - न॒वः । \newline
20. भ॒व॒ती॒म इ॒मे भ॑वति भवती॒मे । \newline
21. इ॒मे वै वा इ॒म इ॒मे वै । \newline
22. वै लो॒का लो॒का वै वै लो॒काः । \newline
23. लो॒का स्त्रि॑ण॒व स्त्रि॑ण॒वो लो॒का लो॒का स्त्रि॑ण॒वः । \newline
24. त्रि॒ण॒व ए॒ष्वे॑षु त्रि॑ण॒व स्त्रि॑ण॒व ए॒षु । \newline
25. त्रि॒ण॒व इति॑ त्रि - न॒वः । \newline
26. ए॒ष्वे॑ वैवै ष्वे᳚(1॒)ष्वे॑व । \newline
27. ए॒व लो॒केषु॑ लो॒के ष्वे॒वैव लो॒केषु॑ । \newline
28. लो॒केषु॒ प्रति॒ प्रति॑ लो॒केषु॑ लो॒केषु॒ प्रति॑ । \newline
29. प्रति॑ तिष्ठन्ति तिष्ठन्ति॒ प्रति॒ प्रति॑ तिष्ठन्ति । \newline
30. ति॒ष्ठ॒न्ति॒ द्वौ द्वौ ति॑ष्ठन्ति तिष्ठन्ति॒ द्वौ । \newline
31. द्वा वे॑कविꣳ॒॒शा वे॑कविꣳ॒॒शौ द्वौ द्वा वे॑कविꣳ॒॒शौ । \newline
32. ए॒क॒विꣳ॒॒शौ भ॑वतो भवत एकविꣳ॒॒शा वे॑कविꣳ॒॒शौ भ॑वतः । \newline
33. ए॒क॒विꣳ॒॒शावित्ये॑क - विꣳ॒॒शौ । \newline
34. भ॒व॒तः॒ प्रति॑ष्ठित्यै॒ प्रति॑ष्ठित्यै भवतो भवतः॒ प्रति॑ष्ठित्यै । \newline
35. प्रति॑ष्ठित्या॒ अथो॒ अथो॒ प्रति॑ष्ठित्यै॒ प्रति॑ष्ठित्या॒ अथो᳚ । \newline
36. प्रति॑ष्ठित्या॒ इति॒ प्रति॑ - स्थि॒त्यै॒ । \newline
37. अथो॒ रुचꣳ॒॒ रुच॒ मथो॒ अथो॒ रुच᳚म् । \newline
38. अथो॒ इत्यथो᳚ । \newline
39. रुच॑ मे॒वैव रुचꣳ॒॒ रुच॑ मे॒व । \newline
40. ए॒वात्मन् ना॒त्मन् ने॒वै वात्मन्न् । \newline
41. आ॒त्मन् द॑धते दधत आ॒त्मन् ना॒त्मन् द॑धते । \newline
42. द॒ध॒ते॒ ब॒हवो॑ ब॒हवो॑ दधते दधते ब॒हवः॑ । \newline
43. ब॒हव॑ ष्षोड॒शिन॑ ष्षोड॒शिनो॑ ब॒हवो॑ ब॒हव॑ ष्षोड॒शिनः॑ । \newline
44. षो॒ड॒शिनो॑ भवन्ति भवन्ति षोड॒शिन॑ ष्षोड॒शिनो॑ भवन्ति । \newline
45. भ॒व॒न्ति॒ तस्मा॒त् तस्मा᳚द् भवन्ति भवन्ति॒ तस्मा᳚त् । \newline
46. तस्मा᳚द् ब॒हवो॑ ब॒हव॒ स्तस्मा॒त् तस्मा᳚द् ब॒हवः॑ । \newline
47. ब॒हवः॑ प्र॒जासु॑ प्र॒जासु॑ ब॒हवो॑ ब॒हवः॑ प्र॒जासु॑ । \newline
48. प्र॒जासु॒ वृषा॑णो॒ वृषा॑णः प्र॒जासु॑ प्र॒जासु॒ वृषा॑णः । \newline
49. प्र॒जास्विति॑ प्र - जासु॑ । \newline
50. वृषा॑णो॒ यद् यद् वृषा॑णो॒ वृषा॑णो॒ यत् । \newline
51. यदे॒त ए॒ते यद् यदे॒ते । \newline
52. ए॒ते स्तोमाः॒ स्तोमा॑ ए॒त ए॒ते स्तोमाः᳚ । \newline
53. स्तोमा॒ व्यति॑षक्ता॒ व्यति॑षक्ताः॒ स्तोमाः॒ स्तोमा॒ व्यति॑षक्ताः । \newline
54. व्यति॑षक्ता॒ भव॑न्ति॒ भव॑न्ति॒ व्यति॑षक्ता॒ व्यति॑षक्ता॒ भव॑न्ति । \newline
55. व्यति॑षक्ता॒ इति॑ वि - अति॑षक्ताः । \newline
56. भव॑न्ति॒ तस्मा॒त् तस्मा॒द् भव॑न्ति॒ भव॑न्ति॒ तस्मा᳚त् । \newline
57. तस्मा॑ दि॒य मि॒यम् तस्मा॒त् तस्मा॑ दि॒यम् । \newline
58. इ॒य मोष॑धीभि॒ रोष॑धीभि रि॒य मि॒य मोष॑धीभिः । \newline
59. ओष॑धीभि॒र् वन॒स्पति॑भि॒र् वन॒स्पति॑भि॒ रोष॑धीभि॒ रोष॑धीभि॒र् वन॒स्पति॑भिः । \newline
60. ओष॑धीभि॒रित्योष॑धि - भिः॒ । \newline
61. वन॒स्पति॑भि॒र् व्यति॑षक्ता॒ व्यति॑षक्ता॒ वन॒स्पति॑भि॒र् वन॒स्पति॑भि॒र् व्यति॑षक्ता । \newline
62. वन॒स्पति॑भि॒रिति॒ वन॒स्पति॑ - भिः॒ । \newline
63. व्यति॑षक्ता॒ व्यति॑षज्यन्ते॒ व्यति॑षज्यन्ते॒ व्यति॑षक्ता॒ व्यति॑षक्ता॒ व्यति॑षज्यन्ते । \newline
64. व्यति॑ष॒क्तेति॑ वि - अति॑षक्ता । \newline

\textbf{Ghana Paata } \newline

1. दे॒वता॑ अ॒भ्यारो॑ह न्त्य॒भ्यारो॑हन्ति दे॒वता॑ दे॒वता॑ अ॒भ्यारो॑हन्ति त्रयस्त्रिꣳ॒॒शात् त्र॑यस्त्रिꣳ॒॒शा द॒भ्यारो॑हन्ति दे॒वता॑ दे॒वता॑ अ॒भ्यारो॑हन्ति त्रयस्त्रिꣳ॒॒शात् । \newline
2. अ॒भ्यारो॑हन्ति त्रयस्त्रिꣳ॒॒शात् त्र॑यस्त्रिꣳ॒॒शा द॒भ्यारो॑ह न्त्य॒भ्यारो॑हन्ति त्रयस्त्रिꣳ॒॒शात् त्र॑यस्त्रिꣳ॒॒शम् त्र॑यस्त्रिꣳ॒॒शम् त्र॑यस्त्रिꣳ॒॒शा द॒भ्यारो॑ह न्त्य॒भ्यारो॑हन्ति त्रयस्त्रिꣳ॒॒शात् त्र॑यस्त्रिꣳ॒॒शम् । \newline
3. अ॒भ्यारो॑ह॒न्तीत्य॑भि - आरो॑हन्ति । \newline
4. त्र॒य॒स्त्रिꣳ॒॒शात् त्र॑यस्त्रिꣳ॒॒शम् त्र॑यस्त्रिꣳ॒॒शम् त्र॑यस्त्रिꣳ॒॒शात् त्र॑यस्त्रिꣳ॒॒शात् त्र॑यस्त्रिꣳ॒॒श मुपोप॑ त्रयस्त्रिꣳ॒॒शम् त्र॑यस्त्रिꣳ॒॒शात् त्र॑यस्त्रिꣳ॒॒शात् त्र॑यस्त्रिꣳ॒॒श मुप॑ । \newline
5. त्र॒य॒स्त्रिꣳ॒॒शादिति॑ त्रयः - त्रिꣳ॒॒शात् । \newline
6. त्र॒य॒स्त्रिꣳ॒॒श मुपोप॑ त्रयस्त्रिꣳ॒॒शम् त्र॑यस्त्रिꣳ॒॒श मुप॑ यन्ति य॒न्त्युप॑ त्रयस्त्रिꣳ॒॒शम् त्र॑यस्त्रिꣳ॒॒श मुप॑ यन्ति । \newline
7. त्र॒य॒स्त्रिꣳ॒॒शमिति॑ त्रय - त्रिꣳ॒॒शम् । \newline
8. उप॑ यन्ति य॒न्त्युपोप॑ यन्ति॒ त्रय॑स्त्रिꣳश॒त् त्रय॑स्त्रिꣳशद् य॒न्त्युपोप॑ यन्ति॒ त्रय॑स्त्रिꣳशत् । \newline
9. य॒न्ति॒ त्रय॑स्त्रिꣳश॒त् त्रय॑स्त्रिꣳशद् यन्ति यन्ति॒ त्रय॑स्त्रिꣳश॒द् वै वै त्रय॑स्त्रिꣳशद् यन्ति यन्ति॒ त्रय॑स्त्रिꣳश॒द् वै । \newline
10. त्रय॑स्त्रिꣳश॒द् वै वै त्रय॑स्त्रिꣳश॒त् त्रय॑स्त्रिꣳश॒द् वै दे॒वता॑ दे॒वता॒ वै त्रय॑स्त्रिꣳश॒त् त्रय॑स्त्रिꣳश॒द् वै दे॒वताः᳚ । \newline
11. त्रय॑स्त्रिꣳश॒दिति॒ त्रयः॑ - त्रिꣳ॒॒श॒त् । \newline
12. वै दे॒वता॑ दे॒वता॒ वै वै दे॒वता॑ दे॒वता॑सु दे॒वता॑सु दे॒वता॒ वै वै दे॒वता॑ दे॒वता॑सु । \newline
13. दे॒वता॑ दे॒वता॑सु दे॒वता॑सु दे॒वता॑ दे॒वता॑ दे॒वता᳚ स्वे॒वैव दे॒वता॑सु दे॒वता॑ दे॒वता॑ दे॒वता᳚ स्वे॒व । \newline
14. दे॒वता᳚ स्वे॒वैव दे॒वता॑सु दे॒वता᳚ स्वे॒व प्रति॒ प्रत्ये॒व दे॒वता॑सु दे॒वता᳚ स्वे॒व प्रति॑ । \newline
15. ए॒व प्रति॒ प्रत्ये॒ वैव प्रति॑ तिष्ठन्ति तिष्ठन्ति॒ प्रत्ये॒ वैव प्रति॑ तिष्ठन्ति । \newline
16. प्रति॑ तिष्ठन्ति तिष्ठन्ति॒ प्रति॒ प्रति॑ तिष्ठन्ति त्रिण॒व स्त्रि॑ण॒व स्ति॑ष्ठन्ति॒ प्रति॒ प्रति॑ तिष्ठन्ति त्रिण॒वः । \newline
17. ति॒ष्ठ॒न्ति॒ त्रि॒ण॒व स्त्रि॑ण॒व स्ति॑ष्ठन्ति तिष्ठन्ति त्रिण॒वो भ॑वति भवति त्रिण॒व स्ति॑ष्ठन्ति तिष्ठन्ति त्रिण॒वो भ॑वति । \newline
18. त्रि॒ण॒वो भ॑वति भवति त्रिण॒व स्त्रि॑ण॒वो भ॑वती॒म इ॒मे भ॑वति त्रिण॒व स्त्रि॑ण॒वो भ॑वती॒मे । \newline
19. त्रि॒ण॒व इति॑ त्रि - न॒वः । \newline
20. भ॒व॒ती॒म इ॒मे भ॑वति भवती॒मे वै वा इ॒मे भ॑वति भवती॒मे वै । \newline
21. इ॒मे वै वा इ॒म इ॒मे वै लो॒का लो॒का वा इ॒म इ॒मे वै लो॒काः । \newline
22. वै लो॒का लो॒का वै वै लो॒का स्त्रि॑ण॒व स्त्रि॑ण॒वो लो॒का वै वै लो॒का स्त्रि॑ण॒वः । \newline
23. लो॒का स्त्रि॑ण॒व स्त्रि॑ण॒वो लो॒का लो॒का स्त्रि॑ण॒व ए॒ष्वे॑षु त्रि॑ण॒वो लो॒का लो॒का स्त्रि॑ण॒व ए॒षु । \newline
24. त्रि॒ण॒व ए॒ष्वे॑षु त्रि॑ण॒व स्त्रि॑ण॒व ए॒ष्वे॑ वैवैषु त्रि॑ण॒व स्त्रि॑ण॒व ए॒ष्वे॑व । \newline
25. त्रि॒ण॒व इति॑ त्रि - न॒वः । \newline
26. ए॒ष्वे॑ वैवैष्वे᳚(1॒) ष्वे॑व लो॒केषु॑ लो॒केष्वे॒ वैष्वे᳚(1॒)ष्वे॑व लो॒केषु॑ । \newline
27. ए॒व लो॒केषु॑ लो॒के ष्वे॒वैव लो॒केषु॒ प्रति॒ प्रति॑ लो॒के ष्वे॒वैव लो॒केषु॒ प्रति॑ । \newline
28. लो॒केषु॒ प्रति॒ प्रति॑ लो॒केषु॑ लो॒केषु॒ प्रति॑ तिष्ठन्ति तिष्ठन्ति॒ प्रति॑ लो॒केषु॑ लो॒केषु॒ प्रति॑ तिष्ठन्ति । \newline
29. प्रति॑ तिष्ठन्ति तिष्ठन्ति॒ प्रति॒ प्रति॑ तिष्ठन्ति॒ द्वौ द्वौ ति॑ष्ठन्ति॒ प्रति॒ प्रति॑ तिष्ठन्ति॒ द्वौ । \newline
30. ति॒ष्ठ॒न्ति॒ द्वौ द्वौ ति॑ष्ठन्ति तिष्ठन्ति॒ द्वा वे॑कविꣳ॒॒शा वे॑कविꣳ॒॒शौ द्वौ ति॑ष्ठन्ति तिष्ठन्ति॒ द्वा वे॑कविꣳ॒॒शौ । \newline
31. द्वा वे॑कविꣳ॒॒शा वे॑कविꣳ॒॒शौ द्वौ द्वा वे॑कविꣳ॒॒शौ भ॑वतो भवत एकविꣳ॒॒शौ द्वौ द्वा वे॑कविꣳ॒॒शौ भ॑वतः । \newline
32. ए॒क॒विꣳ॒॒शौ भ॑वतो भवत एकविꣳ॒॒शा वे॑कविꣳ॒॒शौ भ॑वतः॒ प्रति॑ष्ठित्यै॒ प्रति॑ष्ठित्यै भवत एकविꣳ॒॒शा वे॑कविꣳ॒॒शौ भ॑वतः॒ प्रति॑ष्ठित्यै । \newline
33. ए॒क॒विꣳ॒॒शावित्ये॑क - विꣳ॒॒शौ । \newline
34. भ॒व॒तः॒ प्रति॑ष्ठित्यै॒ प्रति॑ष्ठित्यै भवतो भवतः॒ प्रति॑ष्ठित्या॒ अथो॒ अथो॒ प्रति॑ष्ठित्यै भवतो भवतः॒ प्रति॑ष्ठित्या॒ अथो᳚ । \newline
35. प्रति॑ष्ठित्या॒ अथो॒ अथो॒ प्रति॑ष्ठित्यै॒ प्रति॑ष्ठित्या॒ अथो॒ रुचꣳ॒॒ रुच॒ मथो॒ प्रति॑ष्ठित्यै॒ प्रति॑ष्ठित्या॒ अथो॒ रुच᳚म् । \newline
36. प्रति॑ष्ठित्या॒ इति॒ प्रति॑ - स्थि॒त्यै॒ । \newline
37. अथो॒ रुचꣳ॒॒ रुच॒ मथो॒ अथो॒ रुच॑ मे॒वैव रुच॒ मथो॒ अथो॒ रुच॑ मे॒व । \newline
38. अथो॒ इत्यथो᳚ । \newline
39. रुच॑ मे॒वैव रुचꣳ॒॒ रुच॑ मे॒वात्मन् ना॒त्मन् ने॒व रुचꣳ॒॒ रुच॑ मे॒वात्मन्न् । \newline
40. ए॒वात्मन् ना॒त्मन् ने॒वै वात्मन् द॑धते दधत आ॒त्मन् ने॒वै वात्मन् द॑धते । \newline
41. आ॒त्मन् द॑धते दधत आ॒त्मन् ना॒त्मन् द॑धते ब॒हवो॑ ब॒हवो॑ दधत आ॒त्मन् ना॒त्मन् द॑धते ब॒हवः॑ । \newline
42. द॒ध॒ते॒ ब॒हवो॑ ब॒हवो॑ दधते दधते ब॒हव॑ ष्षोड॒शिन॑ ष्षोड॒शिनो॑ ब॒हवो॑ दधते दधते ब॒हव॑ ष्षोड॒शिनः॑ । \newline
43. ब॒हव॑ ष्षोड॒शिन॑ ष्षोड॒शिनो॑ ब॒हवो॑ ब॒हव॑ ष्षोड॒शिनो॑ भवन्ति भवन्ति षोड॒शिनो॑ ब॒हवो॑ ब॒हव॑ ष्षोड॒शिनो॑ भवन्ति । \newline
44. षो॒ड॒शिनो॑ भवन्ति भवन्ति षोड॒शिन॑ ष्षोड॒शिनो॑ भवन्ति॒ तस्मा॒त् तस्मा᳚द् भवन्ति षोड॒शिन॑ ष्षोड॒शिनो॑ भवन्ति॒ तस्मा᳚त् । \newline
45. भ॒व॒न्ति॒ तस्मा॒त् तस्मा᳚द् भवन्ति भवन्ति॒ तस्मा᳚द् ब॒हवो॑ ब॒हव॒ स्तस्मा᳚द् भवन्ति भवन्ति॒ तस्मा᳚द् ब॒हवः॑ । \newline
46. तस्मा᳚द् ब॒हवो॑ ब॒हव॒ स्तस्मा॒त् तस्मा᳚द् ब॒हवः॑ प्र॒जासु॑ प्र॒जासु॑ ब॒हव॒ स्तस्मा॒त् तस्मा᳚द् ब॒हवः॑ प्र॒जासु॑ । \newline
47. ब॒हवः॑ प्र॒जासु॑ प्र॒जासु॑ ब॒हवो॑ ब॒हवः॑ प्र॒जासु॒ वृषा॑णो॒ वृषा॑णः प्र॒जासु॑ ब॒हवो॑ ब॒हवः॑ प्र॒जासु॒ वृषा॑णः । \newline
48. प्र॒जासु॒ वृषा॑णो॒ वृषा॑णः प्र॒जासु॑ प्र॒जासु॒ वृषा॑णो॒ यद् यद् वृषा॑णः प्र॒जासु॑ प्र॒जासु॒ वृषा॑णो॒ यत् । \newline
49. प्र॒जास्विति॑ प्र - जासु॑ । \newline
50. वृषा॑णो॒ यद् यद् वृषा॑णो॒ वृषा॑णो॒ यदे॒त ए॒ते यद् वृषा॑णो॒ वृषा॑णो॒ यदे॒ते । \newline
51. यदे॒त ए॒ते यद् यदे॒ते स्तोमाः॒ स्तोमा॑ ए॒ते यद् यदे॒ते स्तोमाः᳚ । \newline
52. ए॒ते स्तोमाः॒ स्तोमा॑ ए॒त ए॒ते स्तोमा॒ व्यति॑षक्ता॒ व्यति॑षक्ताः॒ स्तोमा॑ ए॒त ए॒ते स्तोमा॒ व्यति॑षक्ताः । \newline
53. स्तोमा॒ व्यति॑षक्ता॒ व्यति॑षक्ताः॒ स्तोमाः॒ स्तोमा॒ व्यति॑षक्ता॒ भव॑न्ति॒ भव॑न्ति॒ व्यति॑षक्ताः॒ स्तोमाः॒ स्तोमा॒ व्यति॑षक्ता॒ भव॑न्ति । \newline
54. व्यति॑षक्ता॒ भव॑न्ति॒ भव॑न्ति॒ व्यति॑षक्ता॒ व्यति॑षक्ता॒ भव॑न्ति॒ तस्मा॒त् तस्मा॒द् भव॑न्ति॒ व्यति॑षक्ता॒ व्यति॑षक्ता॒ भव॑न्ति॒ तस्मा᳚त् । \newline
55. व्यति॑षक्ता॒ इति॑ वि - अति॑षक्ताः । \newline
56. भव॑न्ति॒ तस्मा॒त् तस्मा॒द् भव॑न्ति॒ भव॑न्ति॒ तस्मा॑ दि॒य मि॒यम् तस्मा॒द् भव॑न्ति॒ भव॑न्ति॒ तस्मा॑ दि॒यम् । \newline
57. तस्मा॑ दि॒य मि॒यम् तस्मा॒त् तस्मा॑ दि॒य मोष॑धीभि॒ रोष॑धीभि रि॒यम् तस्मा॒त् तस्मा॑ दि॒य मोष॑धीभिः । \newline
58. इ॒य मोष॑धीभि॒ रोष॑धीभि रि॒य मि॒य मोष॑धीभि॒र् वन॒स्पति॑भि॒र् वन॒स्पति॑भि॒ रोष॑धीभि रि॒य मि॒य मोष॑धीभि॒र् वन॒स्पति॑भिः । \newline
59. ओष॑धीभि॒र् वन॒स्पति॑भि॒र् वन॒स्पति॑भि॒ रोष॑धीभि॒ रोष॑धीभि॒र् वन॒स्पति॑भि॒र् व्यति॑षक्ता॒ व्यति॑षक्ता॒ वन॒स्पति॑भि॒ रोष॑धीभि॒ रोष॑धीभि॒र् वन॒स्पति॑भि॒र् व्यति॑षक्ता । \newline
60. ओष॑धीभि॒रित्योष॑धि - भिः॒ । \newline
61. वन॒स्पति॑भि॒र् व्यति॑षक्ता॒ व्यति॑षक्ता॒ वन॒स्पति॑भि॒र् वन॒स्पति॑भि॒र् व्यति॑षक्ता॒ व्यति॑षज्यन्ते॒ व्यति॑षज्यन्ते॒ व्यति॑षक्ता॒ वन॒स्पति॑भि॒र् वन॒स्पति॑भि॒र् व्यति॑षक्ता॒ व्यति॑षज्यन्ते । \newline
62. वन॒स्पति॑भि॒रिति॒ वन॒स्पति॑ - भिः॒ । \newline
63. व्यति॑षक्ता॒ व्यति॑षज्यन्ते॒ व्यति॑षज्यन्ते॒ व्यति॑षक्ता॒ व्यति॑षक्ता॒ व्यति॑षज्यन्ते प्र॒जया᳚ प्र॒जया॒ व्यति॑षज्यन्ते॒ व्यति॑षक्ता॒ व्यति॑षक्ता॒ व्यति॑षज्यन्ते प्र॒जया᳚ । \newline
64. व्यति॑ष॒क्तेति॑ वि - अति॑षक्ता । \newline
\pagebreak
\markright{ TS 7.4.3.6  \hfill https://www.vedavms.in \hfill}

\section{ TS 7.4.3.6 }

\textbf{TS 7.4.3.6 } \newline
\textbf{Samhita Paata} \newline

व्यति॑षज्यन्ते प्र॒जया॑ प॒शुभि॒र्य ए॒वं ॅवि॒द्वाꣳस॑ ए॒ता आस॒ते ऽक्लृ॑प्ता॒ वा ए॒ते सु॑व॒र्गं ॅलो॒कं ॅय॑न्त्युच्चाव॒चान् हि स्तोमा॑नुप॒यन्ति॒ यदे॒त ऊ॒र्द्ध्वाः क्लृ॒प्ताः स्तोमा॒ भव॑न्ति क्लृ॒प्ता ए॒व सु॑व॒र्गं ॅलो॒कं ॅय॑न्त्यु॒भयो॑रे॒भ्यो लो॒कयोः᳚ कल्पते त्रिꣳ॒॒शदे॒तास्त्रिꣳ॒॒शद॑क्षरा वि॒राडन्नं॑ वि॒राड् वि॒राजै॒वान्नाद्य॒मव॑ रुन्धते ऽतिरा॒त्राव॒भितो॑ भवतो॒ ऽन्नाद्य॑स्य॒ परि॑गृहीत्यै ॥ \newline

\textbf{Pada Paata} \newline

व्यति॑षज्यन्त॒ इति॑ वि - अति॑षज्यन्ते । प्र॒जयेति॑ प्र - जया᳚ । प॒शुभि॒रिति॑ प॒शु - भिः॒ । ये । ए॒वम् । वि॒द्वाꣳसः॑ । ए॒ताः । आस॑ते । अक्लृ॑प्ताः । वै । ए॒ते । सु॒व॒र्गमिति॑ सुवः - गम् । लो॒कम् । य॒न्ति॒ । उ॒च्चा॒व॒चान् । हि । स्तोमान्॑ । उ॒प॒यन्तीत्यु॑प - यन्ति॑ । यत् । ए॒ते । ऊ॒द्‌र्ध्वाः । क्लृ॒प्ताः । स्तोमाः᳚ । भव॑न्ति । क्लृ॒प्ताः । ए॒व । सु॒व॒र्गमिति॑ सुवः - गम् । लो॒कम् । य॒न्ति॒ । उ॒भयोः᳚ । ए॒भ्यः॒ । लो॒कयोः᳚ । क॒ल्प॒ते॒ । त्रिꣳ॒॒शत् । ए॒ताः । त्रिꣳ॒॒शद॑क्ष॒रेति॑ त्रिꣳ॒॒शत् - अ॒क्ष॒रा॒ । वि॒राडिति॑ वि - राट् । अन्न᳚म् । वि॒राडिति॑ वि - राट् । वि॒राजेति॑ वि - राजा᳚ । ए॒व । अ॒न्नाद्य॒मित्य॑न्न-अद्य᳚म् । अवेति॑ । रु॒न्ध॒ते॒ । अ॒ति॒रा॒त्रावित्य॑ति - रा॒त्रौ । अ॒भितः॑ । भ॒व॒तः॒ । अ॒न्नाद्य॒स्येत्य॑न्न - अद्य॑स्य । परि॑गृहीत्या॒ इति॒ परि॑ - गृ॒ही॒त्यै॒ ॥  \newline


\textbf{Krama Paata} \newline

व्यति॑षज्यन्ते प्र॒जया᳚ । व्यति॑षज्यन्त॒ इति॑ वि - अति॑षज्यन्ते । प्र॒जया॑ प॒शुभिः॑ । प्र॒जयेति॑ प्र - जया᳚ । 
प॒शुभि॒र् ये । प॒शुभि॒रिति॑ प॒शु - भिः॒ । य ए॒वम् । ए॒वम् ॅवि॒द्वाꣳसः॑ । वि॒द्वाꣳस॑ ए॒ताः । ए॒ता आस॑ते । आस॒तेऽक्लृ॑प्ताः । अक्लृ॑प्ता॒ वै । वा ए॒ते । ए॒ते सु॑व॒र्गम् । सु॒व॒र्गम् ॅलो॒कम् । सु॒व॒र्गमिति॑ सुवः - गम् । लो॒कम् ॅय॑न्ति । य॒न्त्यु॒च्चा॒व॒चान् । उ॒च्चा॒व॒चान्. हि । हि स्तोमान्॑ । स्तोमा॑नुप॒यन्ति॑ । उ॒प॒यन्ति॒ यत् । उ॒प॒यन्तीत्यु॑प - यन्ति॑ । यदे॒ते । ए॒त ऊ॒र्द्ध्वाः । ऊ॒र्द्ध्वाः क्लृ॒प्ताः । क्लृ॒प्ताः स्तोमाः᳚ । स्तोमा॒ भव॑न्ति । भव॑न्ति क्लृ॒प्ताः । क्लृ॒प्ता ए॒व । ए॒व सु॑व॒र्गम् । सु॒व॒र्गम् ॅलो॒कम् । सु॒व॒र्गमिति॑ सुवः - गम् । लो॒कम् ॅय॑न्ति । य॒न्त्यु॒भयोः᳚ । उ॒भयो॑रेभ्यः । ए॒भ्यो॒ लो॒कयोः᳚ । लो॒कयोः᳚ कल्पते । क॒ल्प॒ते॒ त्रिꣳ॒॒शत् । त्रिꣳ॒॒शदे॒ताः । ए॒तास्त्रिꣳ॒॒शद॑क्षरा । त्रिꣳ॒॒शद॑क्षरा वि॒राट् । त्रिꣳ॒॒शद॑क्ष॒रेति॑ त्रिꣳ॒॒शत् - अ॒क्ष॒रा॒ । वि॒राडन्न᳚म् । वि॒राडिति॑ वि - राट् । अन्न॑म् ॅवि॒राट् । वि॒राड् वि॒राजा᳚ । वि॒राडिति॑ वि - राट् । वि॒राजै॒व । वि॒राजेति॑ वि - राजा᳚ । ए॒वान्नाद्य᳚म् । अ॒न्नाद्य॒मव॑ । अ॒न्नाद्य॒मित्य॑न्न - अद्य᳚म् । अव॑ रुन्धते । रु॒न्ध॒ते॒ऽति॒रा॒त्रौ । अ॒ति॒रा॒त्राव॒भितः॑ । अ॒ति॒रा॒त्रावित्य॑ति - रा॒त्रौ । अ॒भितो॑ भवतः । भ॒व॒तो॒ऽन्नाद्य॑स्य । अ॒न्नाद्य॑स्य॒ परि॑गृहीत्यै । अ॒न्नाद्य॒स्येत्य॑न्न - अद्य॑स्य । परि॑गृहीत्या॒ इति॒ परि॑ - गृ॒ही॒त्यै॒ । \newline

\textbf{Jatai Paata} \newline

1. व्यति॑षज्यन्ते प्र॒जया᳚ प्र॒जया॒ व्यति॑षज्यन्ते॒ व्यति॑षज्यन्ते प्र॒जया᳚ । \newline
2. व्यति॑षज्यन्त॒ इति॑ वि - अति॑षज्यन्ते । \newline
3. प्र॒जया॑ प॒शुभिः॑ प॒शुभिः॑ प्र॒जया᳚ प्र॒जया॑ प॒शुभिः॑ । \newline
4. प्र॒जयेति॑ प्र - जया᳚ । \newline
5. प॒शुभि॒र् ये ये प॒शुभिः॑ प॒शुभि॒र् ये । \newline
6. प॒शुभि॒रिति॑ प॒शु - भिः॒ । \newline
7. य ए॒व मे॒वं ॅये य ए॒वम् । \newline
8. ए॒वं ॅवि॒द्वाꣳसो॑ वि॒द्वाꣳस॑ ए॒व मे॒वं ॅवि॒द्वाꣳसः॑ । \newline
9. वि॒द्वाꣳस॑ ए॒ता ए॒ता वि॒द्वाꣳसो॑ वि॒द्वाꣳस॑ ए॒ताः । \newline
10. ए॒ता आस॑त॒ आस॑त ए॒ता ए॒ता आस॑ते । \newline
11. आस॒ते ऽक्लृ॑प्ता॒ अक्लृ॑प्ता॒ आस॑त॒ आस॒ते ऽक्लृ॑प्ताः । \newline
12. अक्लृ॑प्ता॒ वै वा अक्लृ॑प्ता॒ अक्लृ॑प्ता॒ वै । \newline
13. वा ए॒त ए॒ते वै वा ए॒ते । \newline
14. ए॒ते सु॑व॒र्गꣳ सु॑व॒र्ग मे॒त ए॒ते सु॑व॒र्गम् । \newline
15. सु॒व॒र्गम् ॅलो॒कम् ॅलो॒कꣳ सु॑व॒र्गꣳ सु॑व॒र्गम् ॅलो॒कम् । \newline
16. सु॒व॒र्गमिति॑ सुवः - गम् । \newline
17. लो॒कं ॅय॑न्ति यन्ति लो॒कम् ॅलो॒कं ॅय॑न्ति । \newline
18. य॒न्त्यु॒च्चा॒व॒चा नु॑च्चाव॒चान्. य॑न्ति यन्त्युच्चाव॒चान् । \newline
19. उ॒च्चा॒व॒चान्. हि ह्यु॑च्चाव॒चा नु॑च्चाव॒चान्. हि । \newline
20. हि स्तोमा॒न् थ्स्तोमा॒न्॒. हि हि स्तोमान्॑ । \newline
21. स्तोमा॑ नुप॒य न्त्यु॑प॒यन्ति॒ स्तोमा॒न् थ्स्तोमा॑ नुप॒यन्ति॑ । \newline
22. उ॒प॒यन्ति॒ यद् यदु॑प॒य न्त्यु॑प॒यन्ति॒ यत् । \newline
23. उ॒प॒यन्तीत्यु॑प - यन्ति॑ । \newline
24. यदे॒त ए॒ते यद् यदे॒ते । \newline
25. ए॒त ऊ॒र्द्ध्वा ऊ॒र्द्ध्वा ए॒त ए॒त ऊ॒र्द्ध्वाः । \newline
26. ऊ॒र्द्ध्वाः क्लृ॒प्ताः क्लृ॒प्ता ऊ॒र्द्ध्वा ऊ॒र्द्ध्वाः क्लृ॒प्ताः । \newline
27. क्लृ॒प्ताः स्तोमाः॒ स्तोमाः᳚ क्लृ॒प्ताः क्लृ॒प्ताः स्तोमाः᳚ । \newline
28. स्तोमा॒ भव॑न्ति॒ भव॑न्ति॒ स्तोमाः॒ स्तोमा॒ भव॑न्ति । \newline
29. भव॑न्ति क्लृ॒प्ताः क्लृ॒प्ता भव॑न्ति॒ भव॑न्ति क्लृ॒प्ताः । \newline
30. क्लृ॒प्ता ए॒वैव क्लृ॒प्ताः क्लृ॒प्ता ए॒व । \newline
31. ए॒व सु॑व॒र्गꣳ सु॑व॒र्ग मे॒वैव सु॑व॒र्गम् । \newline
32. सु॒व॒र्गम् ॅलो॒कम् ॅलो॒कꣳ सु॑व॒र्गꣳ सु॑व॒र्गम् ॅलो॒कम् । \newline
33. सु॒व॒र्गमिति॑ सुवः - गम् । \newline
34. लो॒कं ॅय॑न्ति यन्ति लो॒कम् ॅलो॒कं ॅय॑न्ति । \newline
35. य॒न्त्यु॒भयो॑ रु॒भयो᳚र् यन्ति यन्त्यु॒भयोः᳚ । \newline
36. उ॒भयो॑ रेभ्य एभ्य उ॒भयो॑ रु॒भयो॑ रेभ्यः । \newline
37. ए॒भ्यो॒ लो॒कयो᳚र् लो॒कयो॑ रेभ्य एभ्यो लो॒कयोः᳚ । \newline
38. लो॒कयोः᳚ कल्पते कल्पते लो॒कयो᳚र् लो॒कयोः᳚ कल्पते । \newline
39. क॒ल्प॒ते॒ त्रिꣳ॒॒शत् त्रिꣳ॒॒शत् क॑ल्पते कल्पते त्रिꣳ॒॒शत् । \newline
40. त्रिꣳ॒॒श दे॒ता ए॒ता स्त्रिꣳ॒॒शत् त्रिꣳ॒॒श दे॒ताः । \newline
41. ए॒ता स्त्रिꣳ॒॒शद॑क्षरा त्रिꣳ॒॒शद॑क्षरै॒ता ए॒ता स्त्रिꣳ॒॒शद॑क्षरा । \newline
42. त्रिꣳ॒॒शद॑क्षरा वि॒राड् वि॒राट् त्रिꣳ॒॒शद॑क्षरा त्रिꣳ॒॒शद॑क्षरा वि॒राट् । \newline
43. त्रिꣳ॒॒शद॑क्ष॒रेति॑ त्रिꣳ॒॒शत् - अ॒क्ष॒रा॒ । \newline
44. वि॒रा डन्न॒ मन्नं॑ ॅवि॒राड् वि॒रा डन्न᳚म् । \newline
45. वि॒राडिति॑ वि - राट् । \newline
46. अन्नं॑ ॅवि॒राड् वि॒रा डन्न॒ मन्नं॑ ॅवि॒राट् । \newline
47. वि॒राड् वि॒राजा॑ वि॒राजा॑ वि॒राड् वि॒राड् वि॒राजा᳚ । \newline
48. वि॒राडिति॑ वि - राट् । \newline
49. वि॒राजै॒ वैव वि॒राजा॑ वि॒रा जै॒व । \newline
50. वि॒राजेति॑ वि - राजा᳚ । \newline
51. ए॒वा न्नाद्य॑ म॒न्नाद्य॑ मे॒वै वान्नाद्य᳚म् । \newline
52. अ॒न्नाद्य॒ मवा वा॒न्नाद्य॑ म॒न्नाद्य॒ मव॑ । \newline
53. अ॒न्नाद्य॒मित्य॑न्न - अद्य᳚म् । \newline
54. अव॑ रुन्धते रुन्ध॒ते ऽवाव॑ रुन्धते । \newline
55. रु॒न्ध॒ते॒ ऽति॒रा॒त्रा व॑तिरा॒त्रौ रु॑न्धते रुन्धते ऽतिरा॒त्रौ । \newline
56. अ॒ति॒रा॒त्रा व॒भितो॒ ऽभितो॑ ऽतिरा॒त्रा व॑तिरा॒त्रा व॒भितः॑ । \newline
57. अ॒ति॒रा॒त्रावित्य॑ति - रा॒त्रौ । \newline
58. अ॒भितो॑ भवतो भवतो॒ ऽभितो॒ ऽभितो॑ भवतः । \newline
59. भ॒व॒तो॒ ऽन्नाद्य॑स्या॒ न्नाद्य॑स्य भवतो भवतो॒ ऽन्नाद्य॑स्य । \newline
60. अ॒न्नाद्य॑स्य॒ परि॑गृहीत्यै॒ परि॑गृहीत्या अ॒न्नाद्य॑स्या॒ न्नाद्य॑स्य॒ परि॑गृहीत्यै । \newline
61. अ॒न्नाद्य॒स्येत्य॑न्न - अद्य॑स्य । \newline
62. परि॑गृहीत्या॒ इति॒ परि॑ - गृ॒ही॒त्यै॒ । \newline

\textbf{Ghana Paata } \newline

1. व्यति॑षज्यन्ते प्र॒जया᳚ प्र॒जया॒ व्यति॑षज्यन्ते॒ व्यति॑षज्यन्ते प्र॒जया॑ प॒शुभिः॑ प॒शुभिः॑ प्र॒जया॒ व्यति॑षज्यन्ते॒ व्यति॑षज्यन्ते प्र॒जया॑ प॒शुभिः॑ । \newline
2. व्यति॑षज्यन्त॒ इति॑ वि - अति॑षज्यन्ते । \newline
3. प्र॒जया॑ प॒शुभिः॑ प॒शुभिः॑ प्र॒जया᳚ प्र॒जया॑ प॒शुभि॒र् ये ये प॒शुभिः॑ प्र॒जया᳚ प्र॒जया॑ प॒शुभि॒र् ये । \newline
4. प्र॒जयेति॑ प्र - जया᳚ । \newline
5. प॒शुभि॒र् ये ये प॒शुभिः॑ प॒शुभि॒र् य ए॒व मे॒वं ॅये प॒शुभिः॑ प॒शुभि॒र् य ए॒वम् । \newline
6. प॒शुभि॒रिति॑ प॒शु - भिः॒ । \newline
7. य ए॒व मे॒वं ॅये य ए॒वं ॅवि॒द्वाꣳसो॑ वि॒द्वाꣳस॑ ए॒वं ॅये य ए॒वं ॅवि॒द्वाꣳसः॑ । \newline
8. ए॒वं ॅवि॒द्वाꣳसो॑ वि॒द्वाꣳस॑ ए॒व मे॒वं ॅवि॒द्वाꣳस॑ ए॒ता ए॒ता वि॒द्वाꣳस॑ ए॒व मे॒वं ॅवि॒द्वाꣳस॑ ए॒ताः । \newline
9. वि॒द्वाꣳस॑ ए॒ता ए॒ता वि॒द्वाꣳसो॑ वि॒द्वाꣳस॑ ए॒ता आस॑त॒ आस॑त ए॒ता वि॒द्वाꣳसो॑ वि॒द्वाꣳस॑ ए॒ता आस॑ते । \newline
10. ए॒ता आस॑त॒ आस॑त ए॒ता ए॒ता आस॒ते ऽक्लृ॑प्ता॒ अक्लृ॑प्ता॒ आस॑त ए॒ता ए॒ता आस॒ते ऽक्लृ॑प्ताः । \newline
11. आस॒ते ऽक्लृ॑प्ता॒ अक्लृ॑प्ता॒ आस॑त॒ आस॒ते ऽक्लृ॑प्ता॒ वै वा अक्लृ॑प्ता॒ आस॑त॒ आस॒ते ऽक्लृ॑प्ता॒ वै । \newline
12. अक्लृ॑प्ता॒ वै वा अक्लृ॑प्ता॒ अक्लृ॑प्ता॒ वा ए॒त ए॒ते वा अक्लृ॑प्ता॒ अक्लृ॑प्ता॒ वा ए॒ते । \newline
13. वा ए॒त ए॒ते वै वा ए॒ते सु॑व॒र्गꣳ सु॑व॒र्ग मे॒ते वै वा ए॒ते सु॑व॒र्गम् । \newline
14. ए॒ते सु॑व॒र्गꣳ सु॑व॒र्ग मे॒त ए॒ते सु॑व॒र्गम् ॅलो॒कम् ॅलो॒कꣳ सु॑व॒र्ग मे॒त ए॒ते सु॑व॒र्गम् ॅलो॒कम् । \newline
15. सु॒व॒र्गम् ॅलो॒कम् ॅलो॒कꣳ सु॑व॒र्गꣳ सु॑व॒र्गम् ॅलो॒कं ॅय॑न्ति यन्ति लो॒कꣳ सु॑व॒र्गꣳ सु॑व॒र्गम् ॅलो॒कं ॅय॑न्ति । \newline
16. सु॒व॒र्गमिति॑ सुवः - गम् । \newline
17. लो॒कं ॅय॑न्ति यन्ति लो॒कम् ॅलो॒कं ॅय॑न्त्युच्चाव॒चा नु॑च्चाव॒चान्. य॑न्ति लो॒कम् ॅलो॒कं ॅय॑न्त्युच्चाव॒चान् । \newline
18. य॒न्त्यु॒च्चा॒व॒चा नु॑च्चाव॒चान्. य॑न्ति यन्त्युच्चाव॒चान्. हि ह्यु॑च्चाव॒चान्. य॑न्ति यन्त्युच्चाव॒चान्. हि । \newline
19. उ॒च्चा॒व॒चान्. हि ह्यु॑च्चाव॒चा नु॑च्चाव॒चान्. हि स्तोमा॒न् थ्स्तोमा॒न्॒. ह्यु॑च्चाव॒चा नु॑च्चाव॒चान्. हि स्तोमान्॑ । \newline
20. हि स्तोमा॒न् थ्स्तोमा॒न्॒. हि हि स्तोमा॑ नुप॒य न्त्यु॑प॒यन्ति॒ स्तोमा॒न्॒. हि हि स्तोमा॑ नुप॒यन्ति॑ । \newline
21. स्तोमा॑ नुप॒य न्त्यु॑प॒यन्ति॒ स्तोमा॒न् थ्स्तोमा॑ नुप॒यन्ति॒ यद् यदु॑प॒यन्ति॒ स्तोमा॒न् थ्स्तोमा॑ नुप॒यन्ति॒ यत् । \newline
22. उ॒प॒यन्ति॒ यद् यदु॑प॒य न्त्यु॑प॒यन्ति॒ यदे॒त ए॒ते यदु॑प॒य न्त्यु॑प॒यन्ति॒ यदे॒ते । \newline
23. उ॒प॒यन्तीत्यु॑प - यन्ति॑ । \newline
24. यदे॒त ए॒ते यद् यदे॒त ऊ॒र्द्ध्वा ऊ॒र्द्ध्वा ए॒ते यद् यदे॒त ऊ॒र्द्ध्वाः । \newline
25. ए॒त ऊ॒र्द्ध्वा ऊ॒र्द्ध्वा ए॒त ए॒त ऊ॒र्द्ध्वाः क्लृ॒प्ताः क्लृ॒प्ता ऊ॒र्द्ध्वा ए॒त ए॒त ऊ॒र्द्ध्वाः क्लृ॒प्ताः । \newline
26. ऊ॒र्द्ध्वाः क्लृ॒प्ताः क्लृ॒प्ता ऊ॒र्द्ध्वा ऊ॒र्द्ध्वाः क्लृ॒प्ताः स्तोमाः॒ स्तोमाः᳚ क्लृ॒प्ता ऊ॒र्द्ध्वा ऊ॒र्द्ध्वाः क्लृ॒प्ताः स्तोमाः᳚ । \newline
27. क्लृ॒प्ताः स्तोमाः॒ स्तोमाः᳚ क्लृ॒प्ताः क्लृ॒प्ताः स्तोमा॒ भव॑न्ति॒ भव॑न्ति॒ स्तोमाः᳚ क्लृ॒प्ताः क्लृ॒प्ताः स्तोमा॒ भव॑न्ति । \newline
28. स्तोमा॒ भव॑न्ति॒ भव॑न्ति॒ स्तोमाः॒ स्तोमा॒ भव॑न्ति क्लृ॒प्ताः क्लृ॒प्ता भव॑न्ति॒ स्तोमाः॒ स्तोमा॒ भव॑न्ति क्लृ॒प्ताः । \newline
29. भव॑न्ति क्लृ॒प्ताः क्लृ॒प्ता भव॑न्ति॒ भव॑न्ति क्लृ॒प्ता ए॒वैव क्लृ॒प्ता भव॑न्ति॒ भव॑न्ति क्लृ॒प्ता ए॒व । \newline
30. क्लृ॒प्ता ए॒वैव क्लृ॒प्ताः क्लृ॒प्ता ए॒व सु॑व॒र्गꣳ सु॑व॒र्ग मे॒व क्लृ॒प्ताः क्लृ॒प्ता ए॒व सु॑व॒र्गम् । \newline
31. ए॒व सु॑व॒र्गꣳ सु॑व॒र्ग मे॒वैव सु॑व॒र्गम् ॅलो॒कम् ॅलो॒कꣳ सु॑व॒र्ग मे॒वैव सु॑व॒र्गम् ॅलो॒कम् । \newline
32. सु॒व॒र्गम् ॅलो॒कम् ॅलो॒कꣳ सु॑व॒र्गꣳ सु॑व॒र्गम् ॅलो॒कं ॅय॑न्ति यन्ति लो॒कꣳ सु॑व॒र्गꣳ सु॑व॒र्गम् ॅलो॒कं ॅय॑न्ति । \newline
33. सु॒व॒र्गमिति॑ सुवः - गम् । \newline
34. लो॒कं ॅय॑न्ति यन्ति लो॒कम् ॅलो॒कं ॅय॑न्त्यु॒भयो॑ रु॒भयो᳚र् यन्ति लो॒कम् ॅलो॒कं ॅय॑न्त्यु॒भयोः᳚ । \newline
35. य॒न्त्यु॒भयो॑ रु॒भयो᳚र् यन्ति यन्त्यु॒भयो॑ रेभ्य एभ्य उ॒भयो᳚र् यन्ति यन्त्यु॒भयो॑ रेभ्यः । \newline
36. उ॒भयो॑ रेभ्य एभ्य उ॒भयो॑ रु॒भयो॑ रेभ्यो लो॒कयो᳚र् लो॒कयो॑ रेभ्य उ॒भयो॑ रु॒भयो॑ रेभ्यो लो॒कयोः᳚ । \newline
37. ए॒भ्यो॒ लो॒कयो᳚र् लो॒कयो॑ रेभ्य एभ्यो लो॒कयोः᳚ कल्पते कल्पते लो॒कयो॑ रेभ्य एभ्यो लो॒कयोः᳚ कल्पते । \newline
38. लो॒कयोः᳚ कल्पते कल्पते लो॒कयो᳚र् लो॒कयोः᳚ कल्पते त्रिꣳ॒॒शत् त्रिꣳ॒॒शत् क॑ल्पते लो॒कयो᳚र् लो॒कयोः᳚ कल्पते त्रिꣳ॒॒शत् । \newline
39. क॒ल्प॒ते॒ त्रिꣳ॒॒शत् त्रिꣳ॒॒शत् क॑ल्पते कल्पते त्रिꣳ॒॒श दे॒ता ए॒ता स्त्रिꣳ॒॒शत् क॑ल्पते कल्पते त्रिꣳ॒॒श दे॒ताः । \newline
40. त्रिꣳ॒॒श दे॒ता ए॒ता स्त्रिꣳ॒॒शत् त्रिꣳ॒॒श दे॒ता स्त्रिꣳ॒॒शद॑क्षरा त्रिꣳ॒॒शद॑क्ष रै॒ता स्त्रिꣳ॒॒शत् त्रिꣳ॒॒श दे॒ता स्त्रिꣳ॒॒शद॑क्षरा । \newline
41. ए॒ता स्त्रिꣳ॒॒शद॑क्षरा त्रिꣳ॒॒शद॑क्ष रै॒ता ए॒ता स्त्रिꣳ॒॒शद॑क्षरा वि॒राड् वि॒राट् त्रिꣳ॒॒शद॑क्षरै॒ता ए॒ता स्त्रिꣳ॒॒शद॑क्षरा वि॒राट् । \newline
42. त्रिꣳ॒॒शद॑क्षरा वि॒राड् वि॒राट् त्रिꣳ॒॒शद॑क्षरा त्रिꣳ॒॒शद॑क्षरा वि॒रा डन्न॒ मन्नं॑ ॅवि॒राट् त्रिꣳ॒॒शद॑क्षरा त्रिꣳ॒॒शद॑क्षरा वि॒रा डन्न᳚म् । \newline
43. त्रिꣳ॒॒शद॑क्ष॒रेति॑ त्रिꣳ॒॒शत् - अ॒क्ष॒रा॒ । \newline
44. वि॒रा डन्न॒ मन्नं॑ ॅवि॒राड् वि॒रा डन्नं॑ ॅवि॒राड् वि॒रा डन्नं॑ ॅवि॒राड् वि॒रा डन्नं॑ ॅवि॒राट् । \newline
45. वि॒राडिति॑ वि - राट् । \newline
46. अन्नं॑ ॅवि॒राड् वि॒रा डन्न॒ मन्नं॑ ॅवि॒राड् वि॒राजा॑ वि॒राजा॑ वि॒रा डन्न॒ मन्नं॑ ॅवि॒राड् वि॒राजा᳚ । \newline
47. वि॒राड् वि॒राजा॑ वि॒राजा॑ वि॒राड् वि॒राड् वि॒राजै॒वैव वि॒राजा॑ वि॒राड् वि॒राड् वि॒रा जै॒व । \newline
48. वि॒राडिति॑ वि - राट् । \newline
49. वि॒राजै॒वैव वि॒राजा॑ वि॒राजै॒ वान्नाद्य॑ म॒न्नाद्य॑ मे॒व वि॒राजा॑ वि॒राजै॒ वान्नाद्य᳚म् । \newline
50. वि॒राजेति॑ वि - राजा᳚ । \newline
51. ए॒वान्नाद्य॑ म॒न्नाद्य॑ मे॒वै वान्नाद्य॒ मवा वा॒न्नाद्य॑ मे॒वै वान्नाद्य॒ मव॑ । \newline
52. अ॒न्नाद्य॒ मवा वा॒न्नाद्य॑ म॒न्नाद्य॒ मव॑ रुन्धते रुन्ध॒ते ऽवा॒न्नाद्य॑ म॒न्नाद्य॒ मव॑ रुन्धते । \newline
53. अ॒न्नाद्य॒मित्य॑न्न - अद्य᳚म् । \newline
54. अव॑ रुन्धते रुन्ध॒ते ऽवाव॑ रुन्धते ऽतिरा॒त्रा व॑तिरा॒त्रौ रु॑न्ध॒ते ऽवाव॑ रुन्धते ऽतिरा॒त्रौ । \newline
55. रु॒न्ध॒ते॒ ऽति॒रा॒त्रा व॑तिरा॒त्रौ रु॑न्धते रुन्धते ऽतिरा॒त्रा व॒भितो॒ ऽभितो॑ ऽतिरा॒त्रौ रु॑न्धते रुन्धते ऽतिरा॒त्रा व॒भितः॑ । \newline
56. अ॒ति॒रा॒त्रा व॒भितो॒ ऽभितो॑ ऽतिरा॒त्रा व॑तिरा॒त्रा व॒भितो॑ भवतो भवतो॒ ऽभितो॑ ऽतिरा॒त्रा व॑तिरा॒त्रा व॒भितो॑ भवतः । \newline
57. अ॒ति॒रा॒त्रावित्य॑ति - रा॒त्रौ । \newline
58. अ॒भितो॑ भवतो भवतो॒ ऽभितो॒ ऽभितो॑ भवतो॒ ऽन्नाद्य॑स्या॒ न्नाद्य॑स्य भवतो॒ ऽभितो॒ ऽभितो॑ भवतो॒ ऽन्नाद्य॑स्य । \newline
59. भ॒व॒तो॒ ऽन्नाद्य॑स्या॒ न्नाद्य॑स्य भवतो भवतो॒ ऽन्नाद्य॑स्य॒ परि॑गृहीत्यै॒ परि॑गृहीत्या अ॒न्नाद्य॑स्य भवतो भवतो॒ ऽन्नाद्य॑स्य॒ परि॑गृहीत्यै । \newline
60. अ॒न्नाद्य॑स्य॒ परि॑गृहीत्यै॒ परि॑गृहीत्या अ॒न्नाद्य॑स्या॒ न्नाद्य॑स्य॒ परि॑गृहीत्यै । \newline
61. अ॒न्नाद्य॒स्येत्य॑न्न - अद्य॑स्य । \newline
62. परि॑गृहीत्या॒ इति॒ परि॑ - गृ॒ही॒त्यै॒ । \newline
\pagebreak
\markright{ TS 7.4.4.1  \hfill https://www.vedavms.in \hfill}

\section{ TS 7.4.4.1 }

\textbf{TS 7.4.4.1 } \newline
\textbf{Samhita Paata} \newline

प्र॒जाप॑तिः सुव॒र्गं ॅलो॒कमै॒त् तं दे॒वा येन॑येन॒ छन्द॒साऽनु॒ प्रायु॑ञ्जत॒ तेन॒ नाऽऽ*प्नु॑व॒न् त ए॒ता द्वात्रिꣳ॑शतꣳ॒॒ रात्री॑रपश्य॒न् द्वात्रिꣳ॑शदक्षरा ऽनु॒ष्टुगा-नु॑ष्टुभः प्र॒जाप॑तिः॒ स्वेनै॒व छन्द॑सा प्र॒जाप॑ति-मा॒प्त्वा ऽभ्या॒रुह्य॑ सुव॒र्गं ॅलो॒कमा॑य॒न॒. य ए॒वं ॅवि॒द्वाꣳस॑ ए॒ता आस॑ते॒ द्वात्रिꣳ॑शदे॒ता द्वात्रिꣳ॑शदक्षरा ऽनु॒ष्टुगा-नु॑ष्टुभः प्र॒जाप॑तिः॒ स्वेनै॒व छन्द॑सा प्र॒जाप॑तिमा॒प्त्वा श्रियं॑ गच्छन्ति॒ - [  ] \newline

\textbf{Pada Paata} \newline

प्र॒जाप॑ति॒रिति॑ प्र॒जा - प॒तिः॒ । सु॒व॒र्गमिति॑ सुवः - गम् । लो॒कम् । ऐ॒त् । तम् । दे॒वाः । येन॑ये॒नेति॒ येन॑ - ये॒न॒ । छन्द॑सा । अनु॑ । प्रेति॑ । अयु॑ञ्जत । तेन॑ । न । आ॒प्नु॒व॒न्न् । ते । ए॒ताः । द्वात्रिꣳ॑शतम् । रात्रीः᳚ । अ॒प॒श्य॒न्न् । द्वात्रिꣳ॑शदक्ष॒रेति॒ द्वात्रिꣳ॑शत् - अ॒क्ष॒रा॒ । अ॒नु॒ष्टुगित्य॑नु - स्तुक् । आनु॑ष्टुभ॒ इत्यानु॑ - स्तु॒भः॒ । प्र॒जाप॑ति॒रिति॑ प्र॒जा-प॒तिः॒ । स्वेन॑ । ए॒व । छन्द॑सा । प्र॒जाप॑ति॒मिति॑ प्र॒जा - प॒ति॒म् । आ॒प्त्वा । अ॒भ्या॒रुह्येत्य॑भि - आ॒रुह्य॑ । सु॒व॒र्गमिति॑ सुवः - गम् । लो॒कम् । आ॒य॒न्न् । ये । ए॒वम् । वि॒द्वाꣳसः॑ । ए॒ताः । आस॑ते । द्वात्रिꣳ॑शत् । ए॒ताः । द्वात्रिꣳ॑शदक्ष॒रेति॒ द्वात्रिꣳ॑शत् - अ॒क्ष॒रा॒ । अ॒नु॒ष्टुगित्य॑नु - स्तुक् । आनु॑ष्टुभ॒ इत्यानु॑ - स्तु॒भः॒ । प्र॒जाप॑ति॒रिति॑ प्र॒जा-प॒तिः॒ । स्वेन॑ । ए॒व । छन्द॑सा । प्र॒जाप॑ति॒मिति॑ प्र॒जा -प॒ति॒म् । आ॒प्त्वा । श्रिय᳚म् । ग॒च्छ॒न्ति॒ ।  \newline


\textbf{Krama Paata} \newline

प्र॒जाप॑तिः सुव॒र्गम् । प्र॒जाप॑ति॒रिति॑ प्र॒जा - प॒तिः॒ । सु॒व॒र्गम् ॅलो॒कम् । सु॒व॒र्गमिति॑ सुवः - गम् । लो॒कमै᳚त् । ऐ॒त् तम् । तम् दे॒वाः । दे॒वा येन॑येन । येन॑येन॒ छन्द॑सा । येन॑ये॒नेति॒ येन॑ - ये॒न॒ । छन्द॒साऽनु॑ । अनु॒ प्र । प्रायु॑ञ्जत । अयु॑ञ्जत॒ तेन॑ । तेन॒ न । नाप्नु॑वन्न् । आ॒प्नु॒व॒न् ते । त ए॒ताः । ए॒ता द्वात्रिꣳ॑शतम् । द्वात्रिꣳ॑शतꣳ॒॒ रात्रीः᳚ । रात्री॑रपश्यन्न् । अ॒प॒श्य॒न् द्वात्रिꣳ॑शदक्षरा । द्वात्रिꣳ॑शदक्षराऽनु॒ष्टुक् । द्वात्रिꣳ॑शदक्ष॒रेति॒ द्वात्रिꣳ॑शत् - अ॒क्ष॒रा॒ । अ॒नु॒ष्टुगानु॑ष्टुभः । अ॒नु॒ष्टुगित्य॑नु - स्तुक् । आनु॑ष्टुभः प्र॒जाप॑तिः । आनु॑ष्टुभ॒ इत्यानु॑ - स्तु॒भः॒ । प्र॒जाप॑तिः॒ स्वेन॑ । प्र॒जाप॑ति॒रिति॑ प्र॒जा - प॒तिः॒ । स्वेनै॒व । ए॒व छन्द॑सा । छन्द॑सा प्र॒जाप॑तिम् । प्र॒जाप॑तिमा॒प्त्वा । प्र॒जाप॑ति॒मिति॑ प्र॒जा - प॒ति॒म् । आ॒प्त्याऽभ्या॒रुह्य॑ । अ॒भ्या॒रुह्य॑ सुव॒र्गम् । अ॒भ्या॒रुह्येत्य॑भि - आ॒रुह्य॑ । सु॒व॒र्गम् ॅलो॒कम् । सु॒व॒र्गमिति॑ सुवः - गम् । लो॒कमा॑यन्न् । आ॒य॒न्॒. ये । य ए॒वम् । ए॒वम् ॅवि॒द्वाꣳसः॑ । वि॒द्वाꣳस॑ ए॒ताः । ए॒ता आस॑ते । आस॑ते॒ द्वात्रिꣳ॑शत् । द्वात्रिꣳ॑शदे॒ताः । ए॒ता द्वात्रिꣳ॑शदक्षरा । द्वात्रिꣳ॑शदक्षराऽनु॒ष्टुक् । द्वात्रिꣳ॑शदक्ष॒रेति॒ द्वात्रिꣳ॑शत् - अ॒क्ष॒रा॒ । अ॒नु॒ष्टुगानु॑ष्टुभः । अ॒नु॒ष्टुगित्य॑नु - स्तुक् । आनु॑ष्टुभः प्र॒जाप॑तिः । आनु॑ष्टुभ॒ इत्यानु॑ - स्तु॒भः॒ । प्र॒जाप॑तिः॒ स्वेन॑ । प्र॒जाप॑ति॒रिति॑ प्र॒जा - प॒तिः॒ । स्वेनै॒व । ए॒व छन्द॑सा । छन्द॑सा प्र॒जाप॑तिम् । प्र॒जाप॑तिमा॒प्त्वा । प्र॒जाप॑ति॒मिति॑ प्र॒जा - प॒ति॒म् । आ॒प्त्वा श्रिय᳚म् । श्रिय॑म् गच्छन्ति । ग॒च्छ॒न्ति॒ श्रीः \newline

\textbf{Jatai Paata} \newline

1. प्र॒जाप॑तिः सुव॒र्गꣳ सु॑व॒र्गम् प्र॒जाप॑तिः प्र॒जाप॑तिः सुव॒र्गम् । \newline
2. प्र॒जाप॑ति॒रिति॑ प्र॒जा - प॒तिः॒ । \newline
3. सु॒व॒र्गम् ॅलो॒कम् ॅलो॒कꣳ सु॑व॒र्गꣳ सु॑व॒र्गम् ॅलो॒कम् । \newline
4. सु॒व॒र्गमिति॑ सुवः - गम् । \newline
5. लो॒क मै॑दै ल्लो॒कम् ॅलो॒क मै᳚त् । \newline
6. ऐ॒त् तम् त मै॑दै॒त् तम् । \newline
7. तम् दे॒वा दे॒वा स्तम् तम् दे॒वाः । \newline
8. दे॒वा येन॑येन॒ येन॑येन दे॒वा दे॒वा येन॑येन । \newline
9. येन॑येन॒ छन्द॑सा॒ छन्द॑सा॒ येन॑येन॒ येन॑येन॒ छन्द॑सा । \newline
10. येन॑ये॒नेति॒ येन॑ - ये॒न॒ । \newline
11. छन्द॒सा ऽन्वनु॒ च्छन्द॑सा॒ छन्द॒सा ऽनु॑ । \newline
12. अनु॒ प्र प्राण्वनु॒ प्र । \newline
13. प्रायु॑ञ्ज॒ता यु॑ञ्जत॒ प्र प्रायु॑ञ्जत । \newline
14. अयु॑ञ्जत॒ तेन॒ तेना यु॑ञ्ज॒ता यु॑ञ्जत॒ तेन॑ । \newline
15. तेन॒ न न तेन॒ तेन॒ न । \newline
16. नाप्नु॑वन् नाप्नुव॒न् न नाप्नु॑वन्न् । \newline
17. आ॒प्नु॒व॒न् ते त आ᳚प्नुवन् नाप्नुव॒न् ते । \newline
18. त ए॒ता ए॒ता स्ते त ए॒ताः । \newline
19. ए॒ता द्वात्रिꣳ॑शत॒म् द्वात्रिꣳ॑शत मे॒ता ए॒ता द्वात्रिꣳ॑शतम् । \newline
20. द्वात्रिꣳ॑शतꣳ॒॒ रात्री॒ रात्री॒र् द्वात्रिꣳ॑शत॒म् द्वात्रिꣳ॑शतꣳ॒॒ रात्रीः᳚ । \newline
21. रात्री॑ रपश्यन् नपश्य॒न् रात्री॒ रात्री॑ रपश्यन्न् । \newline
22. अ॒प॒श्य॒न् द्वात्रिꣳ॑शदक्षरा॒ द्वात्रिꣳ॑शदक्षरा ऽपश्यन् नपश्य॒न् द्वात्रिꣳ॑शदक्षरा । \newline
23. द्वात्रिꣳ॑शदक्षरा ऽनु॒ष्टु ग॑नु॒ष्टुग् द्वात्रिꣳ॑शदक्षरा॒ द्वात्रिꣳ॑शदक्षरा ऽनु॒ष्टुक् । \newline
24. द्वात्रिꣳ॑शदक्ष॒रेति॒ द्वात्रिꣳ॑शत् - अ॒क्ष॒रा॒ । \newline
25. अ॒नु॒ष्टु गानु॑ष्टुभ॒ आनु॑ष्टुभो ऽनु॒ष्टु ग॑नु॒ष्टु गानु॑ष्टुभः । \newline
26. अ॒नु॒ष्टुगित्य॑नु - स्तुक् । \newline
27. आनु॑ष्टुभः प्र॒जाप॑तिः प्र॒जाप॑ति॒ रानु॑ष्टुभ॒ आनु॑ष्टुभः प्र॒जाप॑तिः । \newline
28. आनु॑ष्टुभ॒ इत्यानु॑ - स्तु॒भः॒ । \newline
29. प्र॒जाप॑तिः॒ स्वेन॒ स्वेन॑ प्र॒जाप॑तिः प्र॒जाप॑तिः॒ स्वेन॑ । \newline
30. प्र॒जाप॑ति॒रिति॑ प्र॒जा - प॒तिः॒ । \newline
31. स्वेनै॒वैव स्वेन॒ स्वेनै॒व । \newline
32. ए॒व छन्द॑सा॒ छन्द॑ सै॒वैव छन्द॑सा । \newline
33. छन्द॑सा प्र॒जाप॑तिम् प्र॒जाप॑ति॒म् छन्द॑सा॒ छन्द॑सा प्र॒जाप॑तिम् । \newline
34. प्र॒जाप॑ति मा॒प्त्वा ऽऽप्त्वा प्र॒जाप॑तिम् प्र॒जाप॑ति मा॒प्त्वा । \newline
35. प्र॒जाप॑ति॒मिति॑ प्र॒जा - प॒ति॒म् । \newline
36. आ॒प्त्वा ऽभ्या॒रुह्या᳚ भ्या॒रुह्या॒ प्त्वा ऽऽप्त्वा ऽभ्या॒रुह्य॑ । \newline
37. अ॒भ्या॒रुह्य॑ सुव॒र्गꣳ सु॑व॒र्ग म॑भ्या॒रुह्या᳚ भ्या॒रुह्य॑ सुव॒र्गम् । \newline
38. अ॒भ्या॒रुह्येत्य॑भि - आ॒रुह्य॑ । \newline
39. सु॒व॒र्गम् ॅलो॒कम् ॅलो॒कꣳ सु॑व॒र्गꣳ सु॑व॒र्गम् ॅलो॒कम् । \newline
40. सु॒व॒र्गमिति॑ सुवः - गम् । \newline
41. लो॒क मा॑यन् नायन् ॅलो॒कम् ॅलो॒क मा॑यन्न् । \newline
42. आ॒य॒न्॒. ये य आ॑यन् नाय॒न्॒. ये । \newline
43. य ए॒व मे॒वं ॅये य ए॒वम् । \newline
44. ए॒वं ॅवि॒द्वाꣳसो॑ वि॒द्वाꣳस॑ ए॒व मे॒वं ॅवि॒द्वाꣳसः॑ । \newline
45. वि॒द्वाꣳस॑ ए॒ता ए॒ता वि॒द्वाꣳसो॑ वि॒द्वाꣳस॑ ए॒ताः । \newline
46. ए॒ता आस॑त॒ आस॑त ए॒ता ए॒ता आस॑ते । \newline
47. आस॑ते॒ द्वात्रिꣳ॑श॒द् द्वात्रिꣳ॑श॒ दास॑त॒ आस॑ते॒ द्वात्रिꣳ॑शत् । \newline
48. द्वात्रिꣳ॑श दे॒ता ए॒ता द्वात्रिꣳ॑श॒द् द्वात्रिꣳ॑श दे॒ताः । \newline
49. ए॒ता द्वात्रिꣳ॑शदक्षरा॒ द्वात्रिꣳ॑शदक्ष रै॒ता ए॒ता द्वात्रिꣳ॑शदक्षरा । \newline
50. द्वात्रिꣳ॑शदक्षरा ऽनु॒ष्टु ग॑नु॒ष्टुग् द्वात्रिꣳ॑शदक्षरा॒ द्वात्रिꣳ॑शदक्षरा ऽनु॒ष्टुक् । \newline
51. द्वात्रिꣳ॑शदक्ष॒रेति॒ द्वात्रिꣳ॑शत् - अ॒क्ष॒रा॒ । \newline
52. अ॒नु॒ष्टु गानु॑ष्टुभ॒ आनु॑ष्टुभो ऽनु॒ष्टु ग॑नु॒ष्टु गानु॑ष्टुभः । \newline
53. अ॒नु॒ष्टुगित्य॑नु - स्तुक् । \newline
54. आनु॑ष्टुभः प्र॒जाप॑तिः प्र॒जाप॑ति॒ रानु॑ष्टुभ॒ आनु॑ष्टुभः प्र॒जाप॑तिः । \newline
55. आनु॑ष्टुभ॒ इत्यानु॑ - स्तु॒भः॒ । \newline
56. प्र॒जाप॑तिः॒ स्वेन॒ स्वेन॑ प्र॒जाप॑तिः प्र॒जाप॑तिः॒ स्वेन॑ । \newline
57. प्र॒जाप॑ति॒रिति॑ प्र॒जा - प॒तिः॒ । \newline
58. स्वेनै॒वैव स्वेन॒ स्वेनै॒व । \newline
59. ए॒व छन्द॑सा॒ छन्द॑सै॒वैव छन्द॑सा । \newline
60. छन्द॑सा प्र॒जाप॑तिम् प्र॒जाप॑ति॒म् छन्द॑सा॒ छन्द॑सा प्र॒जाप॑तिम् । \newline
61. प्र॒जाप॑ति मा॒प्त्वा ऽऽप्त्वा प्र॒जाप॑तिम् प्र॒जाप॑ति मा॒प्त्वा । \newline
62. प्र॒जाप॑ति॒मिति॑ प्र॒जा - प॒ति॒म् । \newline
63. आ॒प्त्वा श्रियꣳ॒॒ श्रिय॑ मा॒प्त्वा ऽऽप्त्वा श्रिय᳚म् । \newline
64. श्रिय॑म् गच्छन्ति गच्छन्ति॒ श्रियꣳ॒॒ श्रिय॑म् गच्छन्ति । \newline
65. ग॒च्छ॒न्ति॒ श्रीः श्रीर् ग॑च्छन्ति गच्छन्ति॒ श्रीः । \newline

\textbf{Ghana Paata } \newline

1. प्र॒जाप॑तिः सुव॒र्गꣳ सु॑व॒र्गम् प्र॒जाप॑तिः प्र॒जाप॑तिः सुव॒र्गम् ॅलो॒कम् ॅलो॒कꣳ सु॑व॒र्गम् प्र॒जाप॑तिः प्र॒जाप॑तिः सुव॒र्गम् ॅलो॒कम् । \newline
2. प्र॒जाप॑ति॒रिति॑ प्र॒जा - प॒तिः॒ । \newline
3. सु॒व॒र्गम् ॅलो॒कम् ॅलो॒कꣳ सु॑व॒र्गꣳ सु॑व॒र्गम् ॅलो॒क मै॑दै ल्लो॒कꣳ सु॑व॒र्गꣳ सु॑व॒र्गम् ॅलो॒क मै᳚त् । \newline
4. सु॒व॒र्गमिति॑ सुवः - गम् । \newline
5. लो॒क मै॑दै ल्लो॒कम् ॅलो॒क मै॒त् तम् त मै᳚ल्लो॒कम् ॅलो॒क मै॒त् तम् । \newline
6. ऐ॒त् तम् तमै॑दै॒त् तम् दे॒वा दे॒वा स्त मै॑दै॒त् तम् दे॒वाः । \newline
7. तम् दे॒वा दे॒वा स्तम् तम् दे॒वा येन॑येन॒ येन॑येन दे॒वा स्तम् तम् दे॒वा येन॑येन । \newline
8. दे॒वा येन॑येन॒ येन॑येन दे॒वा दे॒वा येन॑येन॒ छन्द॑सा॒ छन्द॑सा॒ येन॑येन दे॒वा दे॒वा येन॑येन॒ छन्द॑सा । \newline
9. येन॑येन॒ छन्द॑सा॒ छन्द॑सा॒ येन॑येन॒ येन॑येन॒ छन्द॒सा ऽन्वनु॒ च्छन्द॑सा॒ येन॑येन॒ येन॑येन॒ छन्द॒सा ऽनु॑ । \newline
10. येन॑ये॒नेति॒ येन॑ - ये॒न॒ । \newline
11. छन्द॒सा ऽन्वनु॒ च्छन्द॑सा॒ छन्द॒सा ऽनु॒ प्र प्राणु॒ च्छन्द॑सा॒ छन्द॒सा ऽनु॒ प्र । \newline
12. अनु॒ प्र प्राण्वनु॒ प्रायु॑ञ्ज॒ता यु॑ञ्जत॒ प्राण्वनु॒ प्रायु॑ञ्जत । \newline
13. प्रायु॑ञ्ज॒ता यु॑ञ्जत॒ प्र प्रायु॑ञ्जत॒ तेन॒ तेना यु॑ञ्जत॒ प्र प्रायु॑ञ्जत॒ तेन॑ । \newline
14. अयु॑ञ्जत॒ तेन॒ तेना यु॑ञ्ज॒ता यु॑ञ्जत॒ तेन॒ न न तेना यु॑ञ्ज॒ता यु॑ञ्जत॒ तेन॒ न । \newline
15. तेन॒ न न तेन॒ तेन॒ नाप्नु॑वन् नाप्नुव॒न् न तेन॒ तेन॒ नाप्नु॑वन्न् । \newline
16. नाप्नु॑वन् नाप्नुव॒न् न नाप्नु॑व॒न् ते त आ᳚प्नुव॒न् न नाप्नु॑व॒न् ते । \newline
17. आ॒प्नु॒व॒न् ते त आ᳚प्नुवन् नाप्नुव॒न् त ए॒ता ए॒ता स्त आ᳚प्नुवन् नाप्नुव॒न् त ए॒ताः । \newline
18. त ए॒ता ए॒ता स्ते त ए॒ता द्वात्रिꣳ॑शत॒म् द्वात्रिꣳ॑शत मे॒ता स्ते त ए॒ता द्वात्रिꣳ॑शतम् । \newline
19. ए॒ता द्वात्रिꣳ॑शत॒म् द्वात्रिꣳ॑शत मे॒ता ए॒ता द्वात्रिꣳ॑शतꣳ॒॒ रात्री॒ रात्री॒र् द्वात्रिꣳ॑शत मे॒ता ए॒ता द्वात्रिꣳ॑शतꣳ॒॒ रात्रीः᳚ । \newline
20. द्वात्रिꣳ॑शतꣳ॒॒ रात्री॒ रात्री॒र् द्वात्रिꣳ॑शत॒म् द्वात्रिꣳ॑शतꣳ॒॒ रात्री॑ रपश्यन् नपश्य॒न् रात्री॒र् द्वात्रिꣳ॑शत॒म् द्वात्रिꣳ॑शतꣳ॒॒ रात्री॑ रपश्यन्न् । \newline
21. रात्री॑ रपश्यन् नपश्य॒न् रात्री॒ रात्री॑ रपश्य॒न् द्वात्रिꣳ॑शदक्षरा॒ द्वात्रिꣳ॑शदक्षरा ऽपश्य॒न् रात्री॒ रात्री॑ रपश्य॒न् द्वात्रिꣳ॑शदक्षरा । \newline
22. अ॒प॒श्य॒न् द्वात्रिꣳ॑शदक्षरा॒ द्वात्रिꣳ॑शदक्षरा ऽपश्यन् नपश्य॒न् द्वात्रिꣳ॑शदक्षरा ऽनु॒ष्टु ग॑नु॒ष्टुग् द्वात्रिꣳ॑शदक्षरा ऽपश्यन् नपश्य॒न् द्वात्रिꣳ॑शदक्षरा ऽनु॒ष्टुक् । \newline
23. द्वात्रिꣳ॑शदक्षरा ऽनु॒ष्टु ग॑नु॒ष्टुग् द्वात्रिꣳ॑शदक्षरा॒ द्वात्रिꣳ॑शदक्षरा ऽनु॒ष्टु गानु॑ष्टुभ॒ आनु॑ष्टुभो ऽनु॒ष्टुग् द्वात्रिꣳ॑शदक्षरा॒ द्वात्रिꣳ॑शदक्षरा ऽनु॒ष्टु गानु॑ष्टुभः । \newline
24. द्वात्रिꣳ॑शदक्ष॒रेति॒ द्वात्रिꣳ॑शत् - अ॒क्ष॒रा॒ । \newline
25. अ॒नु॒ष्टु गानु॑ष्टुभ॒ आनु॑ष्टुभो ऽनु॒ष्टु ग॑नु॒ष्टु गानु॑ष्टुभः प्र॒जाप॑तिः प्र॒जाप॑ति॒ रानु॑ष्टुभो ऽनु॒ष्टु ग॑नु॒ष्टु गानु॑ष्टुभः प्र॒जाप॑तिः । \newline
26. अ॒नु॒ष्टुगित्य॑नु - स्तुक् । \newline
27. आनु॑ष्टुभः प्र॒जाप॑तिः प्र॒जाप॑ति॒ रानु॑ष्टुभ॒ आनु॑ष्टुभः प्र॒जाप॑तिः॒ स्वेन॒ स्वेन॑ प्र॒जाप॑ति॒ रानु॑ष्टुभ॒ आनु॑ष्टुभः प्र॒जाप॑तिः॒ स्वेन॑ । \newline
28. आनु॑ष्टुभ॒ इत्यानु॑ - स्तु॒भः॒ । \newline
29. प्र॒जाप॑तिः॒ स्वेन॒ स्वेन॑ प्र॒जाप॑तिः प्र॒जाप॑तिः॒ स्वेनै॒वैव स्वेन॑ प्र॒जाप॑तिः प्र॒जाप॑तिः॒ स्वेनै॒व । \newline
30. प्र॒जाप॑ति॒रिति॑ प्र॒जा - प॒तिः॒ । \newline
31. स्वेनै॒वैव स्वेन॒ स्वेनै॒व छन्द॑सा॒ छन्द॑सै॒व स्वेन॒ स्वेनै॒व छन्द॑सा । \newline
32. ए॒व छन्द॑सा॒ छन्द॑सै॒वैव छन्द॑सा प्र॒जाप॑तिम् प्र॒जाप॑ति॒म् छन्द॑सै॒वैव छन्द॑सा प्र॒जाप॑तिम् । \newline
33. छन्द॑सा प्र॒जाप॑तिम् प्र॒जाप॑ति॒म् छन्द॑सा॒ छन्द॑सा प्र॒जाप॑ति मा॒प्त्वा ऽऽप्त्वा प्र॒जाप॑ति॒म् छन्द॑सा॒ छन्द॑सा प्र॒जाप॑ति मा॒प्त्वा । \newline
34. प्र॒जाप॑ति मा॒प्त्वा ऽऽप्त्वा प्र॒जाप॑तिम् प्र॒जाप॑ति मा॒प्त्वा ऽभ्या॒रुह्या᳚ भ्या॒रुह्या॒ प्त्वा प्र॒जाप॑तिम् प्र॒जाप॑ति मा॒प्त्वा ऽभ्या॒रुह्य॑ । \newline
35. प्र॒जाप॑ति॒मिति॑ प्र॒जा - प॒ति॒म् । \newline
36. आ॒प्त्वा ऽभ्या॒रुह्या᳚ भ्या॒रुह्या॒ प्त्वा ऽऽप्त्वा ऽभ्या॒रुह्य॑ सुव॒र्गꣳ सु॑व॒र्ग म॑भ्या॒ रुह्या॒प्त्वा ऽऽप्त्वा ऽभ्या॒रुह्य॑ सुव॒र्गम् । \newline
37. अ॒भ्या॒रुह्य॑ सुव॒र्गꣳ सु॑व॒र्ग म॑भ्या॒रुह्या᳚ भ्या॒रुह्य॑ सुव॒र्गम् ॅलो॒कम् ॅलो॒कꣳ सु॑व॒र्ग म॑भ्या॒रुह्या᳚ भ्या॒रुह्य॑ सुव॒र्गम् ॅलो॒कम् । \newline
38. अ॒भ्या॒रुह्येत्य॑भि - आ॒रुह्य॑ । \newline
39. सु॒व॒र्गम् ॅलो॒कम् ॅलो॒कꣳ सु॑व॒र्गꣳ सु॑व॒र्गम् ॅलो॒क मा॑यन् नायन् ॅलो॒कꣳ सु॑व॒र्गꣳ सु॑व॒र्गम् ॅलो॒क मा॑यन्न् । \newline
40. सु॒व॒र्गमिति॑ सुवः - गम् । \newline
41. लो॒क मा॑यन् नायन् ॅलो॒कम् ॅलो॒क मा॑य॒न्॒. ये य आ॑यन् ॅलो॒कम् ॅलो॒क मा॑य॒न्॒. ये । \newline
42. आ॒य॒न्॒. ये य आ॑यन् नाय॒न्॒. य ए॒व मे॒वं ॅय आ॑यन् नाय॒न्॒. य ए॒वम् । \newline
43. य ए॒व मे॒वं ॅये य ए॒वं ॅवि॒द्वाꣳसो॑ वि॒द्वाꣳस॑ ए॒वं ॅये य ए॒वं ॅवि॒द्वाꣳसः॑ । \newline
44. ए॒वं ॅवि॒द्वाꣳसो॑ वि॒द्वाꣳस॑ ए॒व मे॒वं ॅवि॒द्वाꣳस॑ ए॒ता ए॒ता वि॒द्वाꣳस॑ ए॒व मे॒वं ॅवि॒द्वाꣳस॑ ए॒ताः । \newline
45. वि॒द्वाꣳस॑ ए॒ता ए॒ता वि॒द्वाꣳसो॑ वि॒द्वाꣳस॑ ए॒ता आस॑त॒ आस॑त ए॒ता वि॒द्वाꣳसो॑ वि॒द्वाꣳस॑ ए॒ता आस॑ते । \newline
46. ए॒ता आस॑त॒ आस॑त ए॒ता ए॒ता आस॑ते॒ द्वात्रिꣳ॑श॒द् द्वात्रिꣳ॑श॒ दास॑त ए॒ता ए॒ता आस॑ते॒ द्वात्रिꣳ॑शत् । \newline
47. आस॑ते॒ द्वात्रिꣳ॑श॒द् द्वात्रिꣳ॑श॒ दास॑त॒ आस॑ते॒ द्वात्रिꣳ॑श दे॒ता ए॒ता द्वात्रिꣳ॑श॒ दास॑त॒ आस॑ते॒ द्वात्रिꣳ॑श दे॒ताः । \newline
48. द्वात्रिꣳ॑श दे॒ता ए॒ता द्वात्रिꣳ॑श॒द् द्वात्रिꣳ॑श दे॒ता द्वात्रिꣳ॑शदक्षरा॒ द्वात्रिꣳ॑शदक्ष रै॒ता द्वात्रिꣳ॑श॒द् द्वात्रिꣳ॑श दे॒ता द्वात्रिꣳ॑शदक्षरा । \newline
49. ए॒ता द्वात्रिꣳ॑शदक्षरा॒ द्वात्रिꣳ॑शदक्ष रै॒ता ए॒ता द्वात्रिꣳ॑शदक्षरा ऽनु॒ष्टु ग॑नु॒ष्टुग् द्वात्रिꣳ॑शदक्ष रै॒ता ए॒ता द्वात्रिꣳ॑शदक्षरा ऽनु॒ष्टुक् । \newline
50. द्वात्रिꣳ॑शदक्षरा ऽनु॒ष्टु ग॑नु॒ष्टुग् द्वात्रिꣳ॑शदक्षरा॒ द्वात्रिꣳ॑शदक्षरा ऽनु॒ष्टु गानु॑ष्टुभ॒ आनु॑ष्टुभो ऽनु॒ष्टुग् द्वात्रिꣳ॑शदक्षरा॒ द्वात्रिꣳ॑शदक्षरा ऽनु॒ष्टु गानु॑ष्टुभः । \newline
51. द्वात्रिꣳ॑शदक्ष॒रेति॒ द्वात्रिꣳ॑शत् - अ॒क्ष॒रा॒ । \newline
52. अ॒नु॒ष्टु गानु॑ष्टुभ॒ आनु॑ष्टुभो ऽनु॒ष्टु ग॑नु॒ष्टु गानु॑ष्टुभः प्र॒जाप॑तिः प्र॒जाप॑ति॒ रानु॑ष्टुभो ऽनु॒ष्टु ग॑नु॒ष्टु गानु॑ष्टुभः प्र॒जाप॑तिः । \newline
53. अ॒नु॒ष्टुगित्य॑नु - स्तुक् । \newline
54. आनु॑ष्टुभः प्र॒जाप॑तिः प्र॒जाप॑ति॒ रानु॑ष्टुभ॒ आनु॑ष्टुभः प्र॒जाप॑तिः॒ स्वेन॒ स्वेन॑ प्र॒जाप॑ति॒ रानु॑ष्टुभ॒ आनु॑ष्टुभः प्र॒जाप॑तिः॒ स्वेन॑ । \newline
55. आनु॑ष्टुभ॒ इत्यानु॑ - स्तु॒भः॒ । \newline
56. प्र॒जाप॑तिः॒ स्वेन॒ स्वेन॑ प्र॒जाप॑तिः प्र॒जाप॑तिः॒ स्वेनै॒वैव स्वेन॑ प्र॒जाप॑तिः प्र॒जाप॑तिः॒ स्वेनै॒व । \newline
57. प्र॒जाप॑ति॒रिति॑ प्र॒जा - प॒तिः॒ । \newline
58. स्वेनै॒वैव स्वेन॒ स्वेनै॒व छन्द॑सा॒ छन्द॑सै॒व स्वेन॒ स्वेनै॒व छन्द॑सा । \newline
59. ए॒व छन्द॑सा॒ छन्द॑सै॒वैव छन्द॑सा प्र॒जाप॑तिम् प्र॒जाप॑ति॒म् छन्द॑सै॒वैव छन्द॑सा प्र॒जाप॑तिम् । \newline
60. छन्द॑सा प्र॒जाप॑तिम् प्र॒जाप॑ति॒म् छन्द॑सा॒ छन्द॑सा प्र॒जाप॑ति मा॒प्त्वा ऽऽप्त्वा प्र॒जाप॑ति॒म् छन्द॑सा॒ छन्द॑सा प्र॒जाप॑ति मा॒प्त्वा । \newline
61. प्र॒जाप॑ति मा॒प्त्वा ऽऽप्त्वा प्र॒जाप॑तिम् प्र॒जाप॑ति मा॒प्त्वा श्रियꣳ॒॒ श्रिय॑ मा॒प्त्वा प्र॒जाप॑तिम् प्र॒जाप॑ति मा॒प्त्वा श्रिय᳚म् । \newline
62. प्र॒जाप॑ति॒मिति॑ प्र॒जा - प॒ति॒म् । \newline
63. आ॒प्त्वा श्रियꣳ॒॒ श्रिय॑ मा॒प्त्वा ऽऽप्त्वा श्रिय॑म् गच्छन्ति गच्छन्ति॒ श्रिय॑ मा॒प्त्वा ऽऽप्त्वा श्रिय॑म् गच्छन्ति । \newline
64. श्रिय॑म् गच्छन्ति गच्छन्ति॒ श्रियꣳ॒॒ श्रिय॑म् गच्छन्ति॒ श्रीः श्रीर् ग॑च्छन्ति॒ श्रियꣳ॒॒ श्रिय॑म् गच्छन्ति॒ श्रीः । \newline
65. ग॒च्छ॒न्ति॒ श्रीः श्रीर् ग॑च्छन्ति गच्छन्ति॒ श्रीर्. हि हि श्रीर् ग॑च्छन्ति गच्छन्ति॒ श्रीर्. हि । \newline
\pagebreak
\markright{ TS 7.4.4.2  \hfill https://www.vedavms.in \hfill}

\section{ TS 7.4.4.2 }

\textbf{TS 7.4.4.2 } \newline
\textbf{Samhita Paata} \newline

श्रीर्.हि म॑नु॒ष्य॑स्य सुव॒र्गो लो॒को द्वात्रिꣳ॑शदे॒ता द्वात्रिꣳ॑शदक्षरा-ऽनु॒ष्टुग्-वाग॑नु॒ष्टुफ् सर्वा॑मे॒व वाच॑माप्नुवन्ति॒ सर्वे॑ वा॒चो व॑दि॒तारो॑ भवन्ति॒ सर्वे॒ हि श्रियं॒ गच्छ॑न्ति॒ ज्योति॒र्गौरायु॒रिति॑ त्र्य॒हा भ॑वन्ती॒यं ॅवाव ज्योति॑र॒न्तरि॑क्षं॒ गौर॒सावायु॑रि॒माने॒व लो॒कान॒भ्यारो॑हन्त्यभिपू॒र्वं त्र्य॒हा भ॑वन्त्यभिपू॒र्वमे॒व सु॑व॒र्गं ॅलो॒कम॒भ्यारो॑हन्ति बृहद्-रथन्त॒राभ्यां᳚ ॅयन्ती॒ - [  ] \newline

\textbf{Pada Paata} \newline

श्रीः । हि । म॒नु॒ष्य॑स्य । सु॒व॒र्ग इति॑ सुवः-गः । लो॒कः । द्वात्रिꣳ॑शत् । ए॒ताः । द्वात्रिꣳ॑शदक्ष॒रेति॒ द्वात्रिꣳ॑शत् - अ॒क्ष॒रा॒ । अ॒नु॒ष्टुगित्य॑नु - स्तुक् । वाक् । अ॒नु॒ष्टुबित्य॑नु - स्तुप् । सर्वा᳚म् । ए॒व । वाच᳚म् । आ॒प्नु॒व॒न्ति॒ । सर्वे᳚ । वा॒चः । व॒दि॒तारः॑ । भ॒व॒न्ति॒ । सर्वे᳚ । हि । श्रिय᳚म् । गच्छ॑न्ति । ज्योतिः॑ । गौः । आयुः॑ । इति॑ । त्र्य॒हा इति॑ त्रि - अ॒हाः । भ॒व॒न्ति॒ । इ॒यम् । वाव । ज्योतिः॑ । अ॒न्तरि॑क्षम् । गौः । अ॒सौ । आयुः॑ । इ॒मान् । ए॒व । लो॒कान् । अ॒भ्यारो॑ह॒न्तीत्य॑भि - आरो॑हन्ति । अ॒भि॒पू॒र्वमित्य॑भि - पू॒र्वम् । त्र्य॒हा इति॑ त्रि-अ॒हाः । भ॒व॒न्ति॒ । अ॒भि॒पू॒र्वमित्य॑भि - पू॒र्वम् । ए॒व । सु॒व॒र्गमिति॑ सुवः - गम् । लो॒कम् । अ॒भ्यारो॑ह॒न्तीत्य॑भि-आरो॑हन्ति । बृ॒ह॒द्र॒थ॒न्त॒राभ्या॒मिति॑ बृहत् - र॒थ॒न्त॒राभ्या᳚म् । य॒न्ति॒ ।  \newline


\textbf{Krama Paata} \newline

श्रीर्. हि । हि म॑नु॒ष्य॑स्य । म॒नु॒ष्य॑स्य सुव॒र्गः । सु॒व॒र्गो लो॒कः । सु॒व॒र्ग इति॑ सुवः - गः । लो॒को द्वात्रिꣳ॑शत् । द्वात्रिꣳ॑शदे॒ताः । ए॒ता द्वात्रिꣳ॑शदक्षरा । द्वात्रिꣳ॑शदक्षराऽनु॒ष्टुक् । द्वात्रिꣳ॑शदक्ष॒रेति॒ द्वात्रिꣳ॑शत् - अ॒क्ष॒रा॒ । अ॒नु॒ष्टुग् वाक् । अ॒नु॒ष्टुगित्य॑नु - स्तुक् । वाग॑नु॒ष्टुप् । अ॒नु॒ष्टुफ् सर्वा᳚म् । अ॒नु॒ष्टुबित्य॑नु - स्तुप् । सर्वा॑मे॒व । ए॒व वाच᳚म् । वाच॑माप्नुवन्ति । आ॒प्नु॒व॒न्ति॒ सर्वे᳚ । सर्वे॑ वा॒चः । वा॒चो व॑दि॒तारः॑ । व॒दि॒तारो॑ भवन्ति । भ॒व॒न्ति॒ सर्वे᳚ । सर्वे॒ हि । हि श्रिय᳚म् । श्रिय॒म् गच्छ॑न्ति । गच्छ॑न्ति॒ ज्योतिः॑ । ज्योति॒र् गौः । गौरायुः॑ । आयु॒रिति॑ । इति॑ त्र्य॒हाः । त्र्य॒हा भ॑वन्ति । त्र्य॒हा इति॑ त्रि - अ॒हाः । भ॒व॒न्ती॒यम् । इ॒यम् ॅवाव । वाव ज्योतिः॑ । ज्योति॑र॒न्तरि॑क्षम् । अ॒न्तरि॑क्ष॒म् गौः । गौर॒सौ । अ॒सावायुः॑ । आयु॑रि॒मान् । इ॒माने॒व । ए॒व लो॒कान् । लो॒कान॒भ्यारो॑हन्ति । अ॒भ्यारो॑हन्त्यभिपू॒र्वम् । अ॒भ्यारो॑ह॒न्तीत्य॑भि - आरो॑हन्ति । अ॒भि॒पू॒र्वम् त्र्य॒हाः । अ॒भि॒पू॒र्वमित्य॑भि - पू॒र्वम् । त्र्य॒हा भ॑वन्ति । त्र्य॒हा इति॑ त्रि - अ॒हाः । भ॒व॒न्त्य॒भि॒पू॒र्वम् । अ॒भि॒पू॒र्वमे॒व । अ॒भि॒पू॒र्वमित्य॑भि - पू॒र्वम् । ए॒व सु॑व॒र्गम् । सु॒व॒र्गम् ॅलो॒कम् । सु॒व॒र्गमिति॑ सुवः - गम् । लो॒कम॒भ्यारो॑हन्ति । अ॒भ्यारो॑हन्ति बृहद्‍रथन्त॒राभ्या᳚म् । अ॒भ्यारो॑ह॒न्तीत्य॑भि - आरो॑हन्ति । बृ॒ह॒द्‍र॒थ॒न्त॒राभ्या᳚म् ॅयन्ति । बृ॒ह॒द्‍र॒थ॒न्त॒राभ्या॒मिति॑ बृहत् - र॒थ॒न्त॒राभ्या᳚म् । य॒न्ती॒यम् \newline

\textbf{Jatai Paata} \newline

1. श्रीर्. हि हि श्रीः श्रीर्. हि । \newline
2. हि म॑नु॒ष्य॑स्य मनु॒ष्य॑स्य॒ हि हि म॑नु॒ष्य॑स्य । \newline
3. म॒नु॒ष्य॑स्य सुव॒र्गः सु॑व॒र्गो म॑नु॒ष्य॑स्य मनु॒ष्य॑स्य सुव॒र्गः । \newline
4. सु॒व॒र्गो लो॒को लो॒कः सु॑व॒र्गः सु॑व॒र्गो लो॒कः । \newline
5. सु॒व॒र्ग इति॑ सुवः - गः । \newline
6. लो॒को द्वात्रिꣳ॑श॒द् द्वात्रिꣳ॑श ल्लो॒को लो॒को द्वात्रिꣳ॑शत् । \newline
7. द्वात्रिꣳ॑श दे॒ता ए॒ता द्वात्रिꣳ॑श॒द् द्वात्रिꣳ॑श दे॒ताः । \newline
8. ए॒ता द्वात्रिꣳ॑शदक्षरा॒ द्वात्रिꣳ॑शदक्ष रै॒ता ए॒ता द्वात्रिꣳ॑शदक्षरा । \newline
9. द्वात्रिꣳ॑शदक्षरा ऽनु॒ष्टु ग॑नु॒ष्टुग् द्वात्रिꣳ॑शदक्षरा॒ द्वात्रिꣳ॑शदक्षरा ऽनु॒ष्टुक् । \newline
10. द्वात्रिꣳ॑शदक्ष॒रेति॒ द्वात्रिꣳ॑शत् - अ॒क्ष॒रा॒ । \newline
11. अ॒नु॒ष्टुग् वाग् वाग॑नु॒ष्टु ग॑नु॒ष्टुग् वाक् । \newline
12. अ॒नु॒ष्टुगित्य॑नु - स्तुक् । \newline
13. वाग॑नु॒ष्टु ब॑नु॒ष्टुब् वाग् वाग॑नु॒ष्टुप् । \newline
14. अ॒नु॒ष्टुफ् सर्वाꣳ॒॒ सर्वा॑ मनु॒ष्टु ब॑नु॒ष्टुफ् सर्वा᳚म् । \newline
15. अ॒नु॒ष्टुबित्य॑नु - स्तुप् । \newline
16. सर्वा॑ मे॒वैव सर्वाꣳ॒॒ सर्वा॑ मे॒व । \newline
17. ए॒व वाचं॒ ॅवाच॑ मे॒वैव वाच᳚म् । \newline
18. वाच॑ माप्नुवन् त्याप्नुवन्ति॒ वाचं॒ ॅवाच॑ माप्नुवन्ति । \newline
19. आ॒प्नु॒व॒न्ति॒ सर्वे॒ सर्व॑ आप्नुव न्त्याप्नुवन्ति॒ सर्वे᳚ । \newline
20. सर्वे॑ वा॒चो वा॒चः सर्वे॒ सर्वे॑ वा॒चः । \newline
21. वा॒चो व॑दि॒तारो॑ वदि॒तारो॑ वा॒चो वा॒चो व॑दि॒तारः॑ । \newline
22. व॒दि॒तारो॑ भवन्ति भवन्ति वदि॒तारो॑ वदि॒तारो॑ भवन्ति । \newline
23. भ॒व॒न्ति॒ सर्वे॒ सर्वे॑ भवन्ति भवन्ति॒ सर्वे᳚ । \newline
24. सर्वे॒ हि हि सर्वे॒ सर्वे॒ हि । \newline
25. हि श्रियꣳ॒॒ श्रियꣳ॒॒ हि हि श्रिय᳚म् । \newline
26. श्रिय॒म् गच्छ॑न्ति॒ गच्छ॑न्ति॒ श्रियꣳ॒॒ श्रिय॒म् गच्छ॑न्ति । \newline
27. गच्छ॑न्ति॒ ज्योति॒र् ज्योति॒र् गच्छ॑न्ति॒ गच्छ॑न्ति॒ ज्योतिः॑ । \newline
28. ज्योति॒र् गौर् गौर् ज्योति॒र् ज्योति॒र् गौः । \newline
29. गौ रायु॒ रायु॒र् गौर् गौ रायुः॑ । \newline
30. आयु॒ रिती त्यायु॒ रायु॒ रिति॑ । \newline
31. इति॑ त्र्य॒हा स्त्र्य॒हा इतीति॑ त्र्य॒हाः । \newline
32. त्र्य॒हा भ॑वन्ति भवन्ति त्र्य॒हा स्त्र्य॒हा भ॑वन्ति । \newline
33. त्र्य॒हा इति॑ त्रि - अ॒हाः । \newline
34. भ॒व॒न्ती॒य मि॒यम् भ॑वन्ति भवन्ती॒यम् । \newline
35. इ॒यं ॅवाव वावेय मि॒यं ॅवाव । \newline
36. वाव ज्योति॒र् ज्योति॒र् वाव वाव ज्योतिः॑ । \newline
37. ज्योति॑ र॒न्तरि॑क्ष म॒न्तरि॑क्ष॒म् ज्योति॒र् ज्योति॑ र॒न्तरि॑क्षम् । \newline
38. अ॒न्तरि॑क्ष॒म् गौर् गौ र॒न्तरि॑क्ष म॒न्तरि॑क्ष॒म् गौः । \newline
39. गौ र॒सा व॒सौ गौर् गौ र॒सौ । \newline
40. अ॒सा वायु॒ रायु॑ र॒सा व॒सा वायुः॑ । \newline
41. आयु॑ रि॒मा नि॒मा नायु॒ रायु॑ रि॒मान् । \newline
42. इ॒मा ने॒वैवे मा नि॒मा ने॒व । \newline
43. ए॒व लो॒कान् ॅलो॒का ने॒वैव लो॒कान् । \newline
44. लो॒का न॒भ्यारो॑ह न्त्य॒भ्यारो॑हन्ति लो॒कान् ॅलो॒का न॒भ्यारो॑हन्ति । \newline
45. अ॒भ्यारो॑ह न्त्यभिपू॒र्व म॑भिपू॒र्व म॒भ्यारो॑ह न्त्य॒भ्यारो॑ह न्त्यभिपू॒र्वम् । \newline
46. अ॒भ्यारो॑ह॒न्तीत्य॑भि - आरो॑हन्ति । \newline
47. अ॒भि॒पू॒र्वम् त्र्य॒हा स्त्र्य॒हा अ॑भिपू॒र्व म॑भिपू॒र्वम् त्र्य॒हाः । \newline
48. अ॒भि॒पू॒र्वमित्य॑भि - पू॒र्वम् । \newline
49. त्र्य॒हा भ॑वन्ति भवन्ति त्र्य॒हा स्त्र्य॒हा भ॑वन्ति । \newline
50. त्र्य॒हा इति॑ त्रि - अ॒हाः । \newline
51. भ॒व॒ न्त्य॒भि॒पू॒र्व म॑भिपू॒र्वम् भ॑वन्ति भव न्त्यभिपू॒र्वम् । \newline
52. अ॒भि॒पू॒र्व मे॒वै वाभि॑पू॒र्व म॑भिपू॒र्व मे॒व । \newline
53. अ॒भि॒पू॒र्वमित्य॑भि - पू॒र्वम् । \newline
54. ए॒व सु॑व॒र्गꣳ सु॑व॒र्ग मे॒वैव सु॑व॒र्गम् । \newline
55. सु॒व॒र्गम् ॅलो॒कम् ॅलो॒कꣳ सु॑व॒र्गꣳ सु॑व॒र्गम् ॅलो॒कम् । \newline
56. सु॒व॒र्गमिति॑ सुवः - गम् । \newline
57. लो॒क म॒भ्यारो॑ह न्त्य॒भ्यारो॑हन्ति लो॒कम् ॅलो॒क म॒भ्यारो॑हन्ति । \newline
58. अ॒भ्यारो॑हन्ति बृहद्रथन्त॒राभ्या᳚म् बृहद्रथन्त॒राभ्या॑ म॒भ्यारो॑ह न्त्य॒भ्यारो॑हन्ति बृहद्रथन्त॒राभ्या᳚म् । \newline
59. अ॒भ्यारो॑ह॒न्तीत्य॑भि - आरो॑हन्ति । \newline
60. बृ॒ह॒द्र॒थ॒न्त॒राभ्यां᳚ ॅयन्ति यन्ति बृहद्रथन्त॒राभ्या᳚म् बृहद्रथन्त॒राभ्यां᳚ ॅयन्ति । \newline
61. बृ॒ह॒द्र॒थ॒न्त॒राभ्या॒मिति॑ बृहत् - र॒थ॒न्त॒राभ्या᳚म् । \newline
62. य॒न्ती॒य मि॒यं ॅय॑न्ति यन्ती॒यम् । \newline

\textbf{Ghana Paata } \newline

1. श्रीर्. हि हि श्रीः श्रीर्. हि म॑नु॒ष्य॑स्य मनु॒ष्य॑स्य॒ हि श्रीः श्रीर्. हि म॑नु॒ष्य॑स्य । \newline
2. हि म॑नु॒ष्य॑स्य मनु॒ष्य॑स्य॒ हि हि म॑नु॒ष्य॑स्य सुव॒र्गः सु॑व॒र्गो म॑नु॒ष्य॑स्य॒ हि हि म॑नु॒ष्य॑स्य सुव॒र्गः । \newline
3. म॒नु॒ष्य॑स्य सुव॒र्गः सु॑व॒र्गो म॑नु॒ष्य॑स्य मनु॒ष्य॑स्य सुव॒र्गो लो॒को लो॒कः सु॑व॒र्गो म॑नु॒ष्य॑स्य मनु॒ष्य॑स्य सुव॒र्गो लो॒कः । \newline
4. सु॒व॒र्गो लो॒को लो॒कः सु॑व॒र्गः सु॑व॒र्गो लो॒को द्वात्रिꣳ॑श॒द् द्वात्रिꣳ॑श ल्लो॒कः सु॑व॒र्गः सु॑व॒र्गो लो॒को द्वात्रिꣳ॑शत् । \newline
5. सु॒व॒र्ग इति॑ सुवः - गः । \newline
6. लो॒को द्वात्रिꣳ॑श॒द् द्वात्रिꣳ॑श ल्लो॒को लो॒को द्वात्रिꣳ॑श दे॒ता ए॒ता द्वात्रिꣳ॑श ल्लो॒को लो॒को द्वात्रिꣳ॑श दे॒ताः । \newline
7. द्वात्रिꣳ॑श दे॒ता ए॒ता द्वात्रिꣳ॑श॒द् द्वात्रिꣳ॑श दे॒ता द्वात्रिꣳ॑शदक्षरा॒ द्वात्रिꣳ॑शदक्ष रै॒ता द्वात्रिꣳ॑श॒द् द्वात्रिꣳ॑श दे॒ता द्वात्रिꣳ॑शदक्षरा । \newline
8. ए॒ता द्वात्रिꣳ॑शदक्षरा॒ द्वात्रिꣳ॑शदक्ष रै॒ता ए॒ता द्वात्रिꣳ॑शदक्षरा ऽनु॒ष्टु ग॑नु॒ष्टुग् द्वात्रिꣳ॑शदक्ष रै॒ता ए॒ता द्वात्रिꣳ॑शदक्षरा ऽनु॒ष्टुक् । \newline
9. द्वात्रिꣳ॑शदक्षरा ऽनु॒ष्टु ग॑नु॒ष्टुग् द्वात्रिꣳ॑शदक्षरा॒ द्वात्रिꣳ॑शदक्षरा ऽनु॒ष्टुग् वाग् वाग॑नु॒ष्टुग् द्वात्रिꣳ॑शदक्षरा॒ द्वात्रिꣳ॑शदक्षरा ऽनु॒ष्टुग् वाक् । \newline
10. द्वात्रिꣳ॑शदक्ष॒रेति॒ द्वात्रिꣳ॑शत् - अ॒क्ष॒रा॒ । \newline
11. अ॒नु॒ष्टुग् वाग् वाग॑नु॒ष्टु ग॑नु॒ष्टुग् वाग॑नु॒ष्टु ब॑नु॒ष्टुब् वाग॑नु॒ष्टु ग॑नु॒ष्टुग् वाग॑नु॒ष्टुप् । \newline
12. अ॒नु॒ष्टुगित्य॑नु - स्तुक् । \newline
13. वाग॑नु॒ष्टु ब॑नु॒ष्टुब् वाग् वाग॑नु॒ष्टुफ् सर्वाꣳ॒॒ सर्वा॑ मनु॒ष्टुब् वाग् वाग॑नु॒ष्टुफ् सर्वा᳚म् । \newline
14. अ॒नु॒ष्टुफ् सर्वाꣳ॒॒ सर्वा॑ मनु॒ष्टु ब॑नु॒ष्टुफ् सर्वा॑ मे॒वैव सर्वा॑ मनु॒ष्टु ब॑नु॒ष्टुफ् सर्वा॑ मे॒व । \newline
15. अ॒नु॒ष्टुबित्य॑नु - स्तुप् । \newline
16. सर्वा॑ मे॒वैव सर्वाꣳ॒॒ सर्वा॑ मे॒व वाचं॒ ॅवाच॑ मे॒व सर्वाꣳ॒॒ सर्वा॑ मे॒व वाच᳚म् । \newline
17. ए॒व वाचं॒ ॅवाच॑ मे॒वैव वाच॑ माप्नुव न्त्याप्नुवन्ति॒ वाच॑ मे॒वैव वाच॑ माप्नुवन्ति । \newline
18. वाच॑ माप्नुव न्त्याप्नुवन्ति॒ वाचं॒ ॅवाच॑ माप्नुवन्ति॒ सर्वे॒ सर्व॑ आप्नुवन्ति॒ वाचं॒ ॅवाच॑ माप्नुवन्ति॒ सर्वे᳚ । \newline
19. आ॒प्नु॒व॒न्ति॒ सर्वे॒ सर्व॑ आप्नुव न्त्याप्नुवन्ति॒ सर्वे॑ वा॒चो वा॒चः सर्व॑ आप्नुव न्त्याप्नुवन्ति॒ सर्वे॑ वा॒चः । \newline
20. सर्वे॑ वा॒चो वा॒चः सर्वे॒ सर्वे॑ वा॒चो व॑दि॒तारो॑ वदि॒तारो॑ वा॒चः सर्वे॒ सर्वे॑ वा॒चो व॑दि॒तारः॑ । \newline
21. वा॒चो व॑दि॒तारो॑ वदि॒तारो॑ वा॒चो वा॒चो व॑दि॒तारो॑ भवन्ति भवन्ति वदि॒तारो॑ वा॒चो वा॒चो व॑दि॒तारो॑ भवन्ति । \newline
22. व॒दि॒तारो॑ भवन्ति भवन्ति वदि॒तारो॑ वदि॒तारो॑ भवन्ति॒ सर्वे॒ सर्वे॑ भवन्ति वदि॒तारो॑ वदि॒तारो॑ भवन्ति॒ सर्वे᳚ । \newline
23. भ॒व॒न्ति॒ सर्वे॒ सर्वे॑ भवन्ति भवन्ति॒ सर्वे॒ हि हि सर्वे॑ भवन्ति भवन्ति॒ सर्वे॒ हि । \newline
24. सर्वे॒ हि हि सर्वे॒ सर्वे॒ हि श्रियꣳ॒॒ श्रियꣳ॒॒ हि सर्वे॒ सर्वे॒ हि श्रिय᳚म् । \newline
25. हि श्रियꣳ॒॒ श्रियꣳ॒॒ हि हि श्रिय॒म् गच्छ॑न्ति॒ गच्छ॑न्ति॒ श्रियꣳ॒॒ हि हि श्रिय॒म् गच्छ॑न्ति । \newline
26. श्रिय॒म् गच्छ॑न्ति॒ गच्छ॑न्ति॒ श्रियꣳ॒॒ श्रिय॒म् गच्छ॑न्ति॒ ज्योति॒र् ज्योति॒र् गच्छ॑न्ति॒ श्रियꣳ॒॒ श्रिय॒म् गच्छ॑न्ति॒ ज्योतिः॑ । \newline
27. गच्छ॑न्ति॒ ज्योति॒र् ज्योति॒र् गच्छ॑न्ति॒ गच्छ॑न्ति॒ ज्योति॒र् गौर् गौर् ज्योति॒र् गच्छ॑न्ति॒ गच्छ॑न्ति॒ ज्योति॒र् गौः । \newline
28. ज्योति॒र् गौर् गौर् ज्योति॒र् ज्योति॒र् गौ रायु॒ रायु॒र् गौर् ज्योति॒र् ज्योति॒र् गौ रायुः॑ । \newline
29. गौ रायु॒ रायु॒र् गौर् गौ रायु॒ रिती त्यायु॒र् गौर् गौ रायु॒ रिति॑ । \newline
30. आयु॒ रिती त्यायु॒ रायु॒ रिति॑ त्र्य॒हा स्त्र्य॒हा इत्यायु॒ रायु॒ रिति॑ त्र्य॒हाः । \newline
31. इति॑ त्र्य॒हा स्त्र्य॒हा इतीति॑ त्र्य॒हा भ॑वन्ति भवन्ति त्र्य॒हा इतीति॑ त्र्य॒हा भ॑वन्ति । \newline
32. त्र्य॒हा भ॑वन्ति भवन्ति त्र्य॒हा स्त्र्य॒हा भ॑वन्ती॒य मि॒यम् भ॑वन्ति त्र्य॒हा स्त्र्य॒हा भ॑वन्ती॒यम् । \newline
33. त्र्य॒हा इति॑ त्रि - अ॒हाः । \newline
34. भ॒व॒न्ती॒य मि॒यम् भ॑वन्ति भवन्ती॒यं ॅवाव वावेयम् भ॑वन्ति भवन्ती॒यं ॅवाव । \newline
35. इ॒यं ॅवाव वावेय मि॒यं ॅवाव ज्योति॒र् ज्योति॒र् वावेय मि॒यं ॅवाव ज्योतिः॑ । \newline
36. वाव ज्योति॒र् ज्योति॒र् वाव वाव ज्योति॑ र॒न्तरि॑क्ष म॒न्तरि॑क्ष॒म् ज्योति॒र् वाव वाव ज्योति॑ र॒न्तरि॑क्षम् । \newline
37. ज्योति॑ र॒न्तरि॑क्ष म॒न्तरि॑क्ष॒म् ज्योति॒र् ज्योति॑ र॒न्तरि॑क्ष॒म् गौर् गौ र॒न्तरि॑क्ष॒म् ज्योति॒र् ज्योति॑ र॒न्तरि॑क्ष॒म् गौः । \newline
38. अ॒न्तरि॑क्ष॒म् गौर् गौ र॒न्तरि॑क्ष म॒न्तरि॑क्ष॒म् गौर॒सा व॒सौ गौ र॒न्तरि॑क्ष म॒न्तरि॑क्ष॒म् गौर॒सौ । \newline
39. गौ र॒सा व॒सौ गौर् गौ र॒सा वायु॒ रायु॑ र॒सौ गौर् गौ र॒सा वायुः॑ । \newline
40. अ॒सा वायु॒ रायु॑ र॒सा व॒सा वायु॑ रि॒मा नि॒मा नायु॑ र॒सा व॒सा वायु॑ रि॒मान् । \newline
41. आयु॑ रि॒मा नि॒मा नायु॒ रायु॑ रि॒मा ने॒वैवेमा नायु॒ रायु॑ रि॒मा ने॒व । \newline
42. इ॒मा ने॒वैवेमा नि॒मा ने॒व लो॒कान् ॅलो॒का ने॒वेमा नि॒मा ने॒व लो॒कान् । \newline
43. ए॒व लो॒कान् ॅलो॒का ने॒वैव लो॒का न॒भ्यारो॑ह न्त्य॒भ्यारो॑हन्ति लो॒का ने॒वैव लो॒का न॒भ्यारो॑हन्ति । \newline
44. लो॒का न॒भ्यारो॑ह न्त्य॒भ्यारो॑हन्ति लो॒कान् ॅलो॒का न॒भ्यारो॑ह न्त्यभिपू॒र्व म॑भिपू॒र्व म॒भ्यारो॑हन्ति लो॒कान् ॅलो॒का न॒भ्यारो॑ह न्त्यभिपू॒र्वम् । \newline
45. अ॒भ्यारो॑ह न्त्यभिपू॒र्व म॑भिपू॒र्व म॒भ्यारो॑ह न्त्य॒भ्यारो॑ह न्त्यभिपू॒र्वम् त्र्य॒हा स्त्र्य॒हा अ॑भिपू॒र्व म॒भ्यारो॑ह न्त्य॒भ्यारो॑ह न्त्यभिपू॒र्वम् त्र्य॒हाः । \newline
46. अ॒भ्यारो॑ह॒न्तीत्य॑भि - आरो॑हन्ति । \newline
47. अ॒भि॒पू॒र्वम् त्र्य॒हा स्त्र्य॒हा अ॑भिपू॒र्व म॑भिपू॒र्वम् त्र्य॒हा भ॑वन्ति भवन्ति त्र्य॒हा अ॑भिपू॒र्व म॑भिपू॒र्वम् त्र्य॒हा भ॑वन्ति । \newline
48. अ॒भि॒पू॒र्वमित्य॑भि - पू॒र्वम् । \newline
49. त्र्य॒हा भ॑वन्ति भवन्ति त्र्य॒हा स्त्र्य॒हा भ॑व न्त्यभिपू॒र्व म॑भिपू॒र्वम् भ॑वन्ति त्र्य॒हा स्त्र्य॒हा भ॑व न्त्यभिपू॒र्वम् । \newline
50. त्र्य॒हा इति॑ त्रि - अ॒हाः । \newline
51. भ॒व॒ न्त्य॒भि॒पू॒र्व म॑भिपू॒र्वम् भ॑वन्ति भव न्त्यभिपू॒र्व मे॒वैवाभि॑पू॒र्वम् भ॑वन्ति भव न्त्यभिपू॒र्व मे॒व । \newline
52. अ॒भि॒पू॒र्व मे॒वैवाभि॑पू॒र्व म॑भिपू॒र्व मे॒व सु॑व॒र्गꣳ सु॑व॒र्ग मे॒वा भि॑पू॒र्व म॑भिपू॒र्व मे॒व सु॑व॒र्गम् । \newline
53. अ॒भि॒पू॒र्वमित्य॑भि - पू॒र्वम् । \newline
54. ए॒व सु॑व॒र्गꣳ सु॑व॒र्ग मे॒वैव सु॑व॒र्गम् ॅलो॒कम् ॅलो॒कꣳ सु॑व॒र्ग मे॒वैव सु॑व॒र्गम् ॅलो॒कम् । \newline
55. सु॒व॒र्गम् ॅलो॒कम् ॅलो॒कꣳ सु॑व॒र्गꣳ सु॑व॒र्गम् ॅलो॒क म॒भ्यारो॑ह न्त्य॒भ्यारो॑हन्ति लो॒कꣳ सु॑व॒र्गꣳ सु॑व॒र्गम् ॅलो॒क म॒भ्यारो॑हन्ति । \newline
56. सु॒व॒र्गमिति॑ सुवः - गम् । \newline
57. लो॒क म॒भ्यारो॑ह न्त्य॒भ्यारो॑हन्ति लो॒कम् ॅलो॒क म॒भ्यारो॑हन्ति बृहद्रथन्त॒राभ्या᳚म् बृहद्रथन्त॒राभ्या॑ म॒भ्यारो॑हन्ति लो॒कम् ॅलो॒क म॒भ्यारो॑हन्ति बृहद्रथन्त॒राभ्या᳚म् । \newline
58. अ॒भ्यारो॑हन्ति बृहद्रथन्त॒राभ्या᳚म् बृहद्रथन्त॒राभ्या॑ म॒भ्यारो॑ह न्त्य॒भ्यारो॑हन्ति बृहद्रथन्त॒राभ्यां᳚ ॅयन्ति यन्ति बृहद्रथन्त॒राभ्या॑ म॒भ्यारो॑ह न्त्य॒भ्यारो॑हन्ति बृहद्रथन्त॒राभ्यां᳚ ॅयन्ति । \newline
59. अ॒भ्यारो॑ह॒न्तीत्य॑भि - आरो॑हन्ति । \newline
60. बृ॒ह॒द्र॒थ॒न्त॒राभ्यां᳚ ॅयन्ति यन्ति बृहद्रथन्त॒राभ्या᳚म् बृहद्रथन्त॒राभ्यां᳚ ॅयन्ती॒य मि॒यं ॅय॑न्ति बृहद्रथन्त॒राभ्या᳚म् बृहद्रथन्त॒राभ्यां᳚ ॅयन्ती॒यम् । \newline
61. बृ॒ह॒द्र॒थ॒न्त॒राभ्या॒मिति॑ बृहत् - र॒थ॒न्त॒राभ्या᳚म् । \newline
62. य॒न्ती॒य मि॒यं ॅय॑न्ति यन्ती॒यं ॅवाव वावेयं ॅय॑न्ति यन्ती॒यं ॅवाव । \newline
\pagebreak
\markright{ TS 7.4.4.3  \hfill https://www.vedavms.in \hfill}

\section{ TS 7.4.4.3 }

\textbf{TS 7.4.4.3 } \newline
\textbf{Samhita Paata} \newline

-यं ॅवाव र॑थन्त॒रम॒सौ बृ॒हदा॒भ्यामे॒व य॒न्त्यथो॑ अ॒नयो॑रे॒व प्रति॑ तिष्ठन्त्ये॒ते वै य॒ज्ञ्स्या᳚ञ्ज॒साय॑नी स्रु॒ती ताभ्या॑मे॒व सु॑व॒र्गं ॅलो॒कं ॅय॑न्ति॒ परा᳚ञ्चो॒ वा ए॒ते सु॑व॒र्गं ॅलो॒कम॒भ्यारो॑हन्ति॒ ये परा॑चस्त्र्य॒हानु॑प॒यन्ति॑ प्र॒त्यङ् त्र्य॒हो भ॑वति प्र॒त्यव॑रूढ्या॒ अथो॒ प्रति॑ष्ठित्या उ॒भयो᳚र्लो॒कयोर्॑. ऋ॒द्ध्वोत् ति॑ष्ठन्ति॒ द्वात्रिꣳ॑शदे॒तास्तासां॒ ॅया स्त्रिꣳ॒॒शत् त्रिꣳ॒॒शद॑क्षरा ( ) वि॒राडन्नं॑ ॅवि॒राड् वि॒राजै॒वाऽन्नाद्य॒मव॑ रुन्धते॒ ये द्वे अ॑होरा॒त्रे ए॒व ते उ॒भाभ्याꣳ॑ रू॒पाभ्याꣳ॑ सुव॒र्गं ॅलो॒कं ॅय॑न्त्यतिरा॒त्राव॒भितो॑ भवतः॒ परि॑गृहीत्यै ॥ \newline

\textbf{Pada Paata} \newline

इ॒यम् । वाव । र॒थ॒न्त॒रमिति॑ रथं - त॒रम् । अ॒सौ । बृ॒हत् । आ॒भ्याम् । ए॒व । य॒न्ति॒ । अथो॒ इति॑ । अ॒नयोः᳚ । ए॒व । प्रतीति॑ । ति॒ष्ठ॒न्ति॒ । ए॒ते इति॑ । वै । य॒ज्ञ्स्य॑ । अ॒ञ्ज॒साय॑नी॒ इत्य॑ञ्जसा - अय॑नी । स्रु॒ती इति॑ । ताभ्या᳚म् । ए॒व । सु॒व॒र्गमिति॑ सुवः - गम् । लो॒कम् । य॒न्ति॒ । परा᳚ञ्चः । वै । ए॒ते । सु॒व॒र्गमिति॑ सुवः - गम् । लो॒कम् । अ॒भ्यारो॑ह॒न्तीत्य॑भि - आरो॑हन्ति । ये । परा॑चः । त्र्य॒हानिति॑ त्रि - अ॒हान् । उ॒प॒यन्तीत्यु॑प - यन्ति॑ । प्र॒त्यङ् । त्र्य॒हा इति॑ त्रि - अ॒हः । भ॒व॒ति॒ । प्र॒त्यव॑रूढ्या॒ इति॑ प्रति - अव॑रूढ्यै । अथो॒ इति॑ । प्रति॑ष्ठित्या॒ इति॒ प्रति॑ - स्थि॒त्यै॒ । उ॒भयोः᳚ । लो॒कयोः᳚ । ऋ॒द्ध्वा । उदिति॑ । ति॒ष्ठ॒न्ति॒ । द्वात्रिꣳ॑शत् । ए॒ताः । तासा᳚म् । याः । त्रिꣳ॒॒शत् । त्रिꣳ॒॒शद॑क्ष॒रेति॑ त्रिꣳ॒॒शत् - अ॒क्ष॒रा॒ ( ) । वि॒राडिति॑ वि - राट् । अन्न᳚म् । वि॒राडिति॑ वि - राट् । वि॒राजेति॑ वि - राजा᳚ । ए॒व । अ॒न्नाद्य॒मित्य॑न्न - अद्य᳚म् । अवेति॑ । रु॒न्ध॒ते॒ । ये इति॑ । द्वे इति॑ । अ॒हो॒रा॒त्रे इत्य॑हः - रा॒त्रे । ए॒व । ते इति॑ । उ॒भाभ्या᳚म् । रू॒पाभ्या᳚म् । सु॒व॒र्गमिति॑ सुवः - गम् । लो॒कम् । य॒न्ति॒ । अ॒ति॒रा॒त्रावित्य॑ति - रा॒त्रौ । अ॒भितः॑ । भ॒व॒तः॒ । परि॑गृहीत्या॒ इति॒ परि॑ - गृ॒ही॒त्यै॒ ॥  \newline


\textbf{Krama Paata} \newline

इ॒यम् ॅवाव । वाव र॑थन्त॒रम् । र॒थ॒न्त॒रम॒सौ । र॒थ॒न्त॒रमिति॑ रथम् - त॒रम् । अ॒सौ बृ॒हत् । बृ॒हदा॒भ्याम् । आ॒भ्यामे॒व । ए॒व य॑न्ति । य॒न्त्यथो᳚ । अथो॑ अ॒नयोः᳚ । अथो॒ इत्यथो᳚ । अ॒नयो॑रे॒व । ए॒व प्रति॑ । प्रति॑ तिष्ठन्ति । ति॒ष्ठ॒न्त्ये॒ते । ए॒ते वै । ए॒ते इत्ये॒ते । वै य॒ज्ञ्स्य॑ । य॒ज्ञ्स्या᳚ञ्ज॒साय॑नी । अ॒ञ्ज॒साय॑नी स्रु॒ती । अ॒ञ्ज॒साय॑नी॒ इत्य॑ञ्जसा - अय॑नी । स्रु॒ती ताभ्या᳚म् । स्रु॒ती इति॑ स्रु॒ती । ताभ्या॑मे॒व । ए॒व सु॑व॒र्गम् । सु॒व॒र्गम् ॅलो॒कम् । सु॒व॒र्गमिति॑ सुवः - गम् । लो॒कम् ॅय॑न्ति । य॒न्ति॒ परा᳚ञ्चः । परा᳚ञ्चो॒ वै । वा ए॒ते । ए॒ते सु॑व॒र्गम् । सु॒व॒र्गम् ॅलो॒कम् । सु॒व॒र्गमिति॑ सुवः - गम् । लो॒कम॒भ्यारो॑हन्ति । अ॒भ्यारो॑हन्ति॒ ये । अ॒भ्यारो॑ह॒न्तीत्य॑भि - आरो॑हन्ति । ये परा॑चः । परा॑चः त्र्य॒हान् । त्र्य॒हानु॑प॒यन्ति॑ । त्र्य॒हानिति॑ त्रि - अ॒हान् । उ॒प॒यन्ति॑ प्र॒त्यङ्‍ङ् । उ॒प॒यन्तीत्यु॑प - यन्ति॑ । प्र॒त्यङ्‍ त्र्य॒हः । त्र्य॒हो भ॑वति । त्र्य॒ह इति॑ त्रि - अ॒हः । भ॒व॒ति॒ प्र॒त्यव॑रूढ्‍यै । प्र॒त्यव॑रूढ्‍या॒ अथो᳚ । प्र॒त्यव॑रूढ्‍या॒ इति॑ प्रति - अव॑रूढ्‍यै । अथो॒ प्रति॑ष्ठित्यै । अथो॒ इत्यथो᳚ । प्रति॑ष्ठित्या उ॒भयोः᳚ । प्रति॑ष्ठित्या॒ इति॒ प्रति॑ - स्थि॒त्यै॒ । उ॒भयो᳚र् लो॒कयोः᳚ । लो॒कयोर्॑. ऋ॒द्ध्वा । ऋ॒द्ध्वोत् । उत् ति॑ष्ठन्ति । ति॒ष्ठ॒न्ति॒ द्वात्रिꣳ॑शत् । द्वात्रिꣳ॑शदे॒ताः । ए॒तास्तासा᳚म् । तासा॒म् ॅयाः । यास्त्रिꣳ॒॒शत् । त्रिꣳ॒॒शत् त्रिꣳ॒॒शद॑क्षरा ( ) । त्रिꣳ॒॒शद॑क्षरा वि॒राट् । त्रिꣳ॒॒शद॑क्ष॒रेति॑ त्रिꣳ॒॒शत् - अ॒क्ष॒रा॒ । वि॒राडन्न᳚म् । वि॒राडिति॑ वि - राट् । अन्न॑म् ॅवि॒राट् । वि॒राड् वि॒राजा᳚ । वि॒राडिति॑ वि - राट् । वि॒राजै॒व । वि॒राजेति॑ वि - राजा᳚ । ए॒वाऽन्नाद्य᳚म् । अ॒न्नाद्य॒मव॑ । अ॒न्नाद्य॒मित्य॑न्न - अद्य᳚म् । अव॑ रुन्धते । रु॒न्ध॒ते॒ ये । ये द्वे । ये इति॒ ये । द्वे अ॑होरा॒त्रे । द्वे इति॒ द्वे । अ॒हो॒रा॒त्रे ए॒व । अ॒हो॒रा॒त्रे इत्य॑हः - रा॒त्रे । ए॒व ते । ते उ॒भाभ्या᳚म् । ते इति॒ ते । उ॒भाभ्याꣳ॑ रू॒पाभ्या᳚म् । रू॒पाभ्याꣳ॑ सुव॒र्गम् । सु॒व॒र्गम् ॅलो॒कम् । सु॒व॒र्गमिति॑ सुवः - गम् । लो॒कम् ॅय॑न्ति । य॒न्त्य॒ति॒रा॒त्रौ । अ॒ति॒रा॒त्राव॒भितः॑ । अ॒ति॒रा॒त्रावित्य॑ति - रा॒त्रौ । अ॒भितो॑ भवतः । भ॒व॒तः॒ परि॑गृहीत्यै । परि॑गृहीत्या॒ इति॒ परि॑ - गृ॒ही॒त्यै॒ । \newline

\textbf{Jatai Paata} \newline

1. इ॒यं ॅवाव वावेय मि॒यं ॅवाव । \newline
2. वाव र॑थन्त॒रꣳ र॑थन्त॒रं ॅवाव वाव र॑थन्त॒रम् । \newline
3. र॒थ॒न्त॒र म॒सा व॒सौ र॑थन्त॒रꣳ र॑थन्त॒र म॒सौ । \newline
4. र॒थ॒न्त॒रमिति॑ रथं - त॒रम् । \newline
5. अ॒सौ बृ॒हद् बृ॒ह द॒सा व॒सौ बृ॒हत् । \newline
6. बृ॒ह दा॒भ्या मा॒भ्याम् बृ॒हद् बृ॒ह दा॒भ्याम् । \newline
7. आ॒भ्या मे॒वै वाभ्या मा॒भ्या मे॒व । \newline
8. ए॒व य॑न्ति यन् त्ये॒वैव य॑न्ति । \newline
9. य॒न्त्यथो॒ अथो॑ यन्ति य॒न्त्यथो᳚ । \newline
10. अथो॑ अ॒नयो॑ र॒नयो॒ रथो॒ अथो॑ अ॒नयोः᳚ । \newline
11. अथो॒ इत्यथो᳚ । \newline
12. अ॒नयो॑ रे॒वै वानयो॑ र॒नयो॑ रे॒व । \newline
13. ए॒व प्रति॒ प्रत्ये॒वैव प्रति॑ । \newline
14. प्रति॑ तिष्ठन्ति तिष्ठन्ति॒ प्रति॒ प्रति॑ तिष्ठन्ति । \newline
15. ति॒ष्ठ॒न्त्ये॒ते ए॒ते ति॑ष्ठन्ति तिष्ठन्त्ये॒ते । \newline
16. ए॒ते वै वा ए॒ते ए॒ते वै । \newline
17. ए॒ते इत्ये॒ते । \newline
18. वै य॒ज्ञ्स्य॑ य॒ज्ञ्स्य॒ वै वै य॒ज्ञ्स्य॑ । \newline
19. य॒ज्ञ्स्या᳚ ञ्ज॒साय॑नी अञ्ज॒साय॑नी य॒ज्ञ्स्य॑ य॒ज्ञ्स्या᳚ ञ्ज॒साय॑नी । \newline
20. अ॒ञ्ज॒साय॑नी स्रु॒ती स्रु॒ती अ॑ञ्ज॒साय॑नी अञ्ज॒साय॑नी स्रु॒ती । \newline
21. अ॒ञ्ज॒साय॑नी॒ इत्य॑ञ्जसा - अय॑नी । \newline
22. स्रु॒ती ताभ्या॒म् ताभ्याꣳ॑ स्रु॒ती स्रु॒ती ताभ्या᳚म् । \newline
23. स्रु॒ती इति॑ स्रु॒ती । \newline
24. ताभ्या॑ मे॒वैव ताभ्या॒म् ताभ्या॑ मे॒व । \newline
25. ए॒व सु॑व॒र्गꣳ सु॑व॒र्ग मे॒वैव सु॑व॒र्गम् । \newline
26. सु॒व॒र्गम् ॅलो॒कम् ॅलो॒कꣳ सु॑व॒र्गꣳ सु॑व॒र्गम् ॅलो॒कम् । \newline
27. सु॒व॒र्गमिति॑ सुवः - गम् । \newline
28. लो॒कं ॅय॑न्ति यन्ति लो॒कम् ॅलो॒कं ॅय॑न्ति । \newline
29. य॒न्ति॒ परा᳚ञ्चः॒ परा᳚ञ्चो यन्ति यन्ति॒ परा᳚ञ्चः । \newline
30. परा᳚ञ्चो॒ वै वै परा᳚ञ्चः॒ परा᳚ञ्चो॒ वै । \newline
31. वा ए॒त ए॒ते वै वा ए॒ते । \newline
32. ए॒ते सु॑व॒र्गꣳ सु॑व॒र्ग मे॒त ए॒ते सु॑व॒र्गम् । \newline
33. सु॒व॒र्गम् ॅलो॒कम् ॅलो॒कꣳ सु॑व॒र्गꣳ सु॑व॒र्गम् ॅलो॒कम् । \newline
34. सु॒व॒र्गमिति॑ सुवः - गम् । \newline
35. लो॒क म॒भ्यारो॑ह न्त्य॒भ्यारो॑हन्ति लो॒कम् ॅलो॒क म॒भ्यारो॑हन्ति । \newline
36. अ॒भ्यारो॑हन्ति॒ ये ये᳚ ऽभ्यारो॑ह न्त्य॒भ्यारो॑हन्ति॒ ये । \newline
37. अ॒भ्यारो॑ह॒न्तीत्य॑भि - आरो॑हन्ति । \newline
38. ये परा॑चः॒ परा॑चो॒ ये ये परा॑चः । \newline
39. परा॑च स्त्र्य॒हान् त्र्य॒हान् परा॑चः॒ परा॑च स्त्र्य॒हान् । \newline
40. त्र्य॒हा नु॑प॒यन् त्यु॑प॒यन्ति॑ त्र्य॒हान् त्र्य॒हा नु॑प॒यन्ति॑ । \newline
41. त्र्य॒हानिति॑ त्रि - अ॒हान् । \newline
42. उ॒प॒यन्ति॑ प्र॒त्यङ् प्र॒त्यङ् ङु॑प॒यन् त्यु॑प॒यन्ति॑ प्र॒त्यङ् । \newline
43. उ॒प॒यन्तीत्यु॑प - यन्ति॑ । \newline
44. प्र॒त्यङ् त्र्य॒ह स्त्र्य॒हः प्र॒त्यङ् प्र॒त्यङ् त्र्य॒हः । \newline
45. त्र्य॒हो भ॑वति भवति त्र्य॒ह स्त्र्य॒हो भ॑वति । \newline
46. त्र्य॒ह इति॑ त्रि - अ॒हः । \newline
47. भ॒व॒ति॒ प्र॒त्यव॑रूढ्यै प्र॒त्यव॑रूढ्यै भवति भवति प्र॒त्यव॑रूढ्यै । \newline
48. प्र॒त्यव॑रूढ्या॒ अथो॒ अथो᳚ प्र॒त्यव॑रूढ्यै प्र॒त्यव॑रूढ्या॒ अथो᳚ । \newline
49. प्र॒त्यव॑रूढ्या॒ इति॑ प्रति - अव॑रूढ्यै । \newline
50. अथो॒ प्रति॑ष्ठित्यै॒ प्रति॑ष्ठित्या॒ अथो॒ अथो॒ प्रति॑ष्ठित्यै । \newline
51. अथो॒ इत्यथो᳚ । \newline
52. प्रति॑ष्ठित्या उ॒भयो॑ रु॒भयोः॒ प्रति॑ष्ठित्यै॒ प्रति॑ष्ठित्या उ॒भयोः᳚ । \newline
53. प्रति॑ष्ठित्या॒ इति॒ प्रति॑ - स्थि॒त्यै॒ । \newline
54. उ॒भयो᳚र् लो॒कयो᳚र् लो॒कयो॑ रु॒भयो॑ रु॒भयो᳚र् लो॒कयोः᳚ । \newline
55. लो॒कयोर्॑. ऋ॒द्ध्व र्‌द्ध्वा लो॒कयो᳚र् लो॒कयोर्॑. ऋ॒द्ध्वा । \newline
56. ऋ॒द्ध्वोदु दृ॒द्ध्व र्‌द्ध्वोत् । \newline
57. उत् ति॑ष्ठन्ति तिष्ठ॒न् त्युदुत् ति॑ष्ठन्ति । \newline
58. ति॒ष्ठ॒न्ति॒ द्वात्रिꣳ॑श॒द् द्वात्रिꣳ॑शत् तिष्ठन्ति तिष्ठन्ति॒ द्वात्रिꣳ॑शत् । \newline
59. द्वात्रिꣳ॑श दे॒ता ए॒ता द्वात्रिꣳ॑श॒द् द्वात्रिꣳ॑श दे॒ताः । \newline
60. ए॒ता स्तासा॒म् तासा॑ मे॒ता ए॒ता स्तासा᳚म् । \newline
61. तासां॒ ॅया या स्तासा॒म् तासां॒ ॅयाः । \newline
62. या स्त्रिꣳ॒॒शत् त्रिꣳ॒॒शद् या या स्त्रिꣳ॒॒शत् । \newline
63. त्रिꣳ॒॒शत् त्रिꣳ॒॒शद॑क्षरा त्रिꣳ॒॒शद॑क्षरा त्रिꣳ॒॒शत् त्रिꣳ॒॒शत् त्रिꣳ॒॒शद॑क्षरा । \newline
64. त्रिꣳ॒॒शद॑क्षरा वि॒राड् वि॒राट् त्रिꣳ॒॒शद॑क्षरा त्रिꣳ॒॒शद॑क्षरा वि॒राट् । \newline
65. त्रिꣳ॒॒शद॑क्ष॒रेति॑ त्रिꣳ॒॒शत् - अ॒क्ष॒रा॒ । \newline
66. वि॒रा डन्न॒ मन्नं॑ ॅवि॒राड् वि॒रा डन्न᳚म् । \newline
67. वि॒राडिति॑ वि - राट् । \newline
68. अन्नं॑ ॅवि॒राड् वि॒रा डन्न॒ मन्नं॑ ॅवि॒राट् । \newline
69. वि॒राड् वि॒राजा॑ वि॒राजा॑ वि॒राड् वि॒राड् वि॒राजा᳚ । \newline
70. वि॒राडिति॑ वि - राट् । \newline
71. वि॒राजै॒वैव वि॒राजा॑ वि॒रा जै॒व । \newline
72. वि॒राजेति॑ वि - राजा᳚ । \newline
73. ए॒वान्नाद्य॑ म॒न्नाद्य॑ मे॒वै वान्नाद्य᳚म् । \newline
74. अ॒न्नाद्य॒ मवा वा॒न्नाद्य॑ म॒न्नाद्य॒ मव॑ । \newline
75. अ॒न्नाद्य॒मित्य॑न्न - अद्य᳚म् । \newline
76. अव॑ रुन्धते रुन्ध॒ते ऽवाव॑ रुन्धते । \newline
77. रु॒न्ध॒ते॒ ये ये रु॑न्धते रुन्धते॒ ये । \newline
78. ये द्वे द्वे ये ये द्वे । \newline
79. ये इति॒ ये । \newline
80. द्वे अ॑होरा॒त्रे अ॑होरा॒त्रे द्वे द्वे अ॑होरा॒त्रे । \newline
81. द्वे इति॒ द्वे । \newline
82. अ॒हो॒रा॒त्रे ए॒वै वाहो॑रा॒त्रे अ॑होरा॒त्रे ए॒व । \newline
83. अ॒हो॒रा॒त्रे इत्य॑हः - रा॒त्रे । \newline
84. ए॒व ते ते ए॒वैव ते । \newline
85. ते उ॒भाभ्या॑ मु॒भाभ्या॒म् ते ते उ॒भाभ्या᳚म् । \newline
86. ते इति॒ ते । \newline
87. उ॒भाभ्याꣳ॑ रू॒पाभ्याꣳ॑ रू॒पाभ्या॑ मु॒भाभ्या॑ मु॒भाभ्याꣳ॑ रू॒पाभ्या᳚म् । \newline
88. रू॒पाभ्याꣳ॑ सुव॒र्गꣳ सु॑व॒र्गꣳ रू॒पाभ्याꣳ॑ रू॒पाभ्याꣳ॑ सुव॒र्गम् । \newline
89. सु॒व॒र्गम् ॅलो॒कम् ॅलो॒कꣳ सु॑व॒र्गꣳ सु॑व॒र्गम् ॅलो॒कम् । \newline
90. सु॒व॒र्गमिति॑ सुवः - गम् । \newline
91. लो॒कं ॅय॑न्ति यन्ति लो॒कम् ॅलो॒कं ॅय॑न्ति । \newline
92. य॒न्त्य॒ति॒रा॒त्रा व॑तिरा॒त्रौ य॑न्ति यन्त्यतिरा॒त्रौ । \newline
93. अ॒ति॒रा॒त्रा व॒भितो॒ ऽभितो॑ ऽतिरा॒त्रा व॑तिरा॒त्रा व॒भितः॑ । \newline
94. अ॒ति॒रा॒त्रावित्य॑ति - रा॒त्रौ । \newline
95. अ॒भितो॑ भवतो भवतो॒ ऽभितो॒ ऽभितो॑ भवतः । \newline
96. भ॒व॒तः॒ परि॑गृहीत्यै॒ परि॑गृहीत्यै भवतो भवतः॒ परि॑गृहीत्यै । \newline
97. परि॑गृहीत्या॒ इति॒ परि॑ - गृ॒ही॒त्यै॒ । \newline

\textbf{Ghana Paata } \newline

1. इ॒यं ॅवाव वावेय मि॒यं ॅवाव र॑थन्त॒रꣳ र॑थन्त॒रं ॅवावेय मि॒यं ॅवाव र॑थन्त॒रम् । \newline
2. वाव र॑थन्त॒रꣳ र॑थन्त॒रं ॅवाव वाव र॑थन्त॒र म॒सा व॒सौ र॑थन्त॒रं ॅवाव वाव र॑थन्त॒र म॒सौ । \newline
3. र॒थ॒न्त॒र म॒सा व॒सौ र॑थन्त॒रꣳ र॑थन्त॒र म॒सौ बृ॒हद् बृ॒ह द॒सौ र॑थन्त॒रꣳ र॑थन्त॒र म॒सौ बृ॒हत् । \newline
4. र॒थ॒न्त॒रमिति॑ रथं - त॒रम् । \newline
5. अ॒सौ बृ॒हद् बृ॒ह द॒सा व॒सौ बृ॒हदा॒भ्या मा॒भ्याम् बृ॒ह द॒सा व॒सौ बृ॒हदा॒भ्याम् । \newline
6. बृ॒ह दा॒भ्या मा॒भ्याम् बृ॒हद् बृ॒ह दा॒भ्या मे॒वैवाभ्याम् बृ॒हद् बृ॒ह दा॒भ्या मे॒व । \newline
7. आ॒भ्या मे॒वैवाभ्या मा॒भ्या मे॒व य॑न्ति यन्त्ये॒ वाभ्या मा॒भ्या मे॒व य॑न्ति । \newline
8. ए॒व य॑न्ति यन्त्ये॒वैव य॒न्त्यथो॒ अथो॑ यन्त्ये॒वैव य॒न्त्यथो᳚ । \newline
9. य॒न्त्यथो॒ अथो॑ यन्ति य॒न्त्यथो॑ अ॒नयो॑ र॒नयो॒ रथो॑ यन्ति य॒न्त्यथो॑ अ॒नयोः᳚ । \newline
10. अथो॑ अ॒नयो॑ र॒नयो॒ रथो॒ अथो॑ अ॒नयो॑ रे॒वै वानयो॒ रथो॒ अथो॑ अ॒नयो॑ रे॒व । \newline
11. अथो॒ इत्यथो᳚ । \newline
12. अ॒नयो॑ रे॒वै वानयो॑ र॒नयो॑ रे॒व प्रति॒ प्रत्ये॒वानयो॑ र॒नयो॑ रे॒व प्रति॑ । \newline
13. ए॒व प्रति॒ प्रत्ये॒वैव प्रति॑ तिष्ठन्ति तिष्ठन्ति॒ प्रत्ये॒वैव प्रति॑ तिष्ठन्ति । \newline
14. प्रति॑ तिष्ठन्ति तिष्ठन्ति॒ प्रति॒ प्रति॑ तिष्ठन्त्ये॒ते ए॒ते ति॑ष्ठन्ति॒ प्रति॒ प्रति॑ तिष्ठन्त्ये॒ते । \newline
15. ति॒ष्ठ॒ न्त्ये॒ते ए॒ते ति॑ष्ठन्ति तिष्ठ न्त्ये॒ते वै वा ए॒ते ति॑ष्ठन्ति तिष्ठ न्त्ये॒ते वै । \newline
16. ए॒ते वै वा ए॒ते ए॒ते वै य॒ज्ञ्स्य॑ य॒ज्ञ्स्य॒ वा ए॒ते ए॒ते वै य॒ज्ञ्स्य॑ । \newline
17. ए॒ते इत्ये॒ते । \newline
18. वै य॒ज्ञ्स्य॑ य॒ज्ञ्स्य॒ वै वै य॒ज्ञ्स्या᳚ ञ्ज॒साय॑नी अञ्ज॒साय॑नी य॒ज्ञ्स्य॒ वै वै य॒ज्ञ्स्या᳚ ञ्ज॒साय॑नी । \newline
19. य॒ज्ञ्स्या᳚ ञ्ज॒साय॑नी अञ्ज॒साय॑नी य॒ज्ञ्स्य॑ य॒ज्ञ्स्या᳚ ञ्ज॒साय॑नी स्रु॒ती स्रु॒ती अ॑ञ्ज॒साय॑नी य॒ज्ञ्स्य॑ य॒ज्ञ्स्या᳚ ञ्ज॒साय॑नी स्रु॒ती । \newline
20. अ॒ञ्ज॒साय॑नी स्रु॒ती स्रु॒ती अ॑ञ्ज॒साय॑नी अञ्ज॒साय॑नी स्रु॒ती ताभ्या॒म् ताभ्याꣳ॑ स्रु॒ती अ॑ञ्ज॒साय॑नी अञ्ज॒साय॑नी स्रु॒ती ताभ्या᳚म् । \newline
21. अ॒ञ्ज॒साय॑नी॒ इत्य॑ञ्जसा - अय॑नी । \newline
22. स्रु॒ती ताभ्या॒म् ताभ्याꣳ॑ स्रु॒ती स्रु॒ती ताभ्या॑ मे॒वैव ताभ्याꣳ॑ स्रु॒ती स्रु॒ती ताभ्या॑ मे॒व । \newline
23. स्रु॒ती इति॑ स्रु॒ती । \newline
24. ताभ्या॑ मे॒वैव ताभ्या॒म् ताभ्या॑ मे॒व सु॑व॒र्गꣳ सु॑व॒र्ग मे॒व ताभ्या॒म् ताभ्या॑ मे॒व सु॑व॒र्गम् । \newline
25. ए॒व सु॑व॒र्गꣳ सु॑व॒र्ग मे॒वैव सु॑व॒र्गम् ॅलो॒कम् ॅलो॒कꣳ सु॑व॒र्ग मे॒वैव सु॑व॒र्गम् ॅलो॒कम् । \newline
26. सु॒व॒र्गम् ॅलो॒कम् ॅलो॒कꣳ सु॑व॒र्गꣳ सु॑व॒र्गम् ॅलो॒कं ॅय॑न्ति यन्ति लो॒कꣳ सु॑व॒र्गꣳ सु॑व॒र्गम् ॅलो॒कं ॅय॑न्ति । \newline
27. सु॒व॒र्गमिति॑ सुवः - गम् । \newline
28. लो॒कं ॅय॑न्ति यन्ति लो॒कम् ॅलो॒कं ॅय॑न्ति॒ परा᳚ञ्चः॒ परा᳚ञ्चो यन्ति लो॒कम् ॅलो॒कं ॅय॑न्ति॒ परा᳚ञ्चः । \newline
29. य॒न्ति॒ परा᳚ञ्चः॒ परा᳚ञ्चो यन्ति यन्ति॒ परा᳚ञ्चो॒ वै वै परा᳚ञ्चो यन्ति यन्ति॒ परा᳚ञ्चो॒ वै । \newline
30. परा᳚ञ्चो॒ वै वै परा᳚ञ्चः॒ परा᳚ञ्चो॒ वा ए॒त ए॒ते वै परा᳚ञ्चः॒ परा᳚ञ्चो॒ वा ए॒ते । \newline
31. वा ए॒त ए॒ते वै वा ए॒ते सु॑व॒र्गꣳ सु॑व॒र्ग मे॒ते वै वा ए॒ते सु॑व॒र्गम् । \newline
32. ए॒ते सु॑व॒र्गꣳ सु॑व॒र्ग मे॒त ए॒ते सु॑व॒र्गम् ॅलो॒कम् ॅलो॒कꣳ सु॑व॒र्ग मे॒त ए॒ते सु॑व॒र्गम् ॅलो॒कम् । \newline
33. सु॒व॒र्गम् ॅलो॒कम् ॅलो॒कꣳ सु॑व॒र्गꣳ सु॑व॒र्गम् ॅलो॒क म॒भ्यारो॑ह न्त्य॒भ्यारो॑हन्ति लो॒कꣳ सु॑व॒र्गꣳ सु॑व॒र्गम् ॅलो॒क म॒भ्यारो॑हन्ति । \newline
34. सु॒व॒र्गमिति॑ सुवः - गम् । \newline
35. लो॒क म॒भ्यारो॑ह न्त्य॒भ्यारो॑हन्ति लो॒कम् ॅलो॒क म॒भ्यारो॑हन्ति॒ ये ये᳚ ऽभ्यारो॑हन्ति लो॒कम् ॅलो॒क म॒भ्यारो॑हन्ति॒ ये । \newline
36. अ॒भ्यारो॑हन्ति॒ ये ये᳚ ऽभ्यारो॑ह न्त्य॒भ्यारो॑हन्ति॒ ये परा॑चः॒ परा॑चो॒ ये᳚ ऽभ्यारो॑ह न्त्य॒भ्यारो॑हन्ति॒ ये परा॑चः । \newline
37. अ॒भ्यारो॑ह॒न्तीत्य॑भि - आरो॑हन्ति । \newline
38. ये परा॑चः॒ परा॑चो॒ ये ये परा॑च स्त्र्य॒हान् त्र्य॒हान् परा॑चो॒ ये ये परा॑च स्त्र्य॒हान् । \newline
39. परा॑च स्त्र्य॒हान् त्र्य॒हान् परा॑चः॒ परा॑च स्त्र्य॒हा नु॑प॒य न्त्यु॑प॒यन्ति॑ त्र्य॒हान् परा॑चः॒ परा॑च स्त्र्य॒हा नु॑प॒यन्ति॑ । \newline
40. त्र्य॒हा नु॑प॒य न्त्यु॑प॒यन्ति॑ त्र्य॒हान् त्र्य॒हा नु॑प॒यन्ति॑ प्र॒त्यङ् प्र॒त्यङ् ङु॑प॒यन्ति॑ त्र्य॒हान् त्र्य॒हा नु॑प॒यन्ति॑ प्र॒त्यङ् । \newline
41. त्र्य॒हानिति॑ त्रि - अ॒हान् । \newline
42. उ॒प॒यन्ति॑ प्र॒त्यङ् प्र॒त्यङ् ङु॑प॒य न्त्यु॑प॒यन्ति॑ प्र॒त्यङ् त्र्य॒ह स्त्र्य॒हः प्र॒त्यङ्
ङु॑प॒य न्त्यु॑प॒यन्ति॑ प्र॒त्यङ् त्र्य॒हः । \newline
43. उ॒प॒यन्तीत्यु॑प - यन्ति॑ । \newline
44. प्र॒त्यङ् त्र्य॒ह स्त्र्य॒हः प्र॒त्यङ् प्र॒त्यङ् त्र्य॒हो भ॑वति भवति त्र्य॒हः प्र॒त्यङ् प्र॒त्यङ् त्र्य॒हो भ॑वति । \newline
45. त्र्य॒हो भ॑वति भवति त्र्य॒ह स्त्र्य॒हो भ॑वति प्र॒त्यव॑रूढ्यै प्र॒त्यव॑रूढ्यै भवति त्र्य॒ह स्त्र्य॒हो भ॑वति प्र॒त्यव॑रूढ्यै । \newline
46. त्र्य॒ह इति॑ त्रि - अ॒हः । \newline
47. भ॒व॒ति॒ प्र॒त्यव॑रूढ्यै प्र॒त्यव॑रूढ्यै भवति भवति प्र॒त्यव॑रूढ्या॒ अथो॒ अथो᳚ प्र॒त्यव॑रूढ्यै भवति भवति प्र॒त्यव॑रूढ्या॒ अथो᳚ । \newline
48. प्र॒त्यव॑रूढ्या॒ अथो॒ अथो᳚ प्र॒त्यव॑रूढ्यै प्र॒त्यव॑रूढ्या॒ अथो॒ प्रति॑ष्ठित्यै॒ प्रति॑ष्ठित्या॒ अथो᳚ प्र॒त्यव॑रूढ्यै प्र॒त्यव॑रूढ्या॒ अथो॒ प्रति॑ष्ठित्यै । \newline
49. प्र॒त्यव॑रूढ्या॒ इति॑ प्रति - अव॑रूढ्यै । \newline
50. अथो॒ प्रति॑ष्ठित्यै॒ प्रति॑ष्ठित्या॒ अथो॒ अथो॒ प्रति॑ष्ठित्या उ॒भयो॑ रु॒भयोः॒ प्रति॑ष्ठित्या॒ अथो॒ अथो॒ प्रति॑ष्ठित्या उ॒भयोः᳚ । \newline
51. अथो॒ इत्यथो᳚ । \newline
52. प्रति॑ष्ठित्या उ॒भयो॑ रु॒भयोः॒ प्रति॑ष्ठित्यै॒ प्रति॑ष्ठित्या उ॒भयो᳚र् लो॒कयो᳚र् लो॒कयो॑ रु॒भयोः॒ प्रति॑ष्ठित्यै॒ प्रति॑ष्ठित्या उ॒भयो᳚र् लो॒कयोः᳚ । \newline
53. प्रति॑ष्ठित्या॒ इति॒ प्रति॑ - स्थि॒त्यै॒ । \newline
54. उ॒भयो᳚र् लो॒कयो᳚र् लो॒कयो॑ रु॒भयो॑ रु॒भयो᳚र् लो॒कयोर्॑. ऋ॒द्ध्व र्‌द्ध्वा लो॒कयो॑ रु॒भयो॑ रु॒भयो᳚र् लो॒कयोर्॑. ऋ॒द्ध्वा । \newline
55. लो॒कयोर्॑. ऋ॒द्ध्व र्‌द्ध्वा लो॒कयो᳚र् लो॒कयोर्॑. ऋ॒द्ध्वोदु दृ॒द्ध्वा लो॒कयो᳚र् लो॒कयोर्॑. ऋ॒द्ध्वोत् । \newline
56. ऋ॒द्ध्वोदु दृ॒द्ध्व र्‌द्ध्वोत् ति॑ष्ठन्ति तिष्ठ॒ न्त्युदृ॒द्ध्व र्‌द्ध्वोत् ति॑ष्ठन्ति । \newline
57. उत् ति॑ष्ठन्ति तिष्ठ॒ न्त्युदुत् ति॑ष्ठन्ति॒ द्वात्रिꣳ॑श॒द् द्वात्रिꣳ॑शत् तिष्ठ॒ न्त्युदुत् ति॑ष्ठन्ति॒ द्वात्रिꣳ॑शत् । \newline
58. ति॒ष्ठ॒न्ति॒ द्वात्रिꣳ॑श॒द् द्वात्रिꣳ॑शत् तिष्ठन्ति तिष्ठन्ति॒ द्वात्रिꣳ॑श दे॒ता ए॒ता द्वात्रिꣳ॑शत् तिष्ठन्ति तिष्ठन्ति॒ द्वात्रिꣳ॑श दे॒ताः । \newline
59. द्वात्रिꣳ॑श दे॒ता ए॒ता द्वात्रिꣳ॑श॒द् द्वात्रिꣳ॑श दे॒ता स्तासा॒म् तासा॑ मे॒ता द्वात्रिꣳ॑श॒द् द्वात्रिꣳ॑श दे॒ता स्तासा᳚म् । \newline
60. ए॒ता स्तासा॒म् तासा॑ मे॒ता ए॒ता स्तासां॒ ॅया या स्तासा॑ मे॒ता ए॒ता स्तासां॒ ॅयाः । \newline
61. तासां॒ ॅया या स्तासा॒म् तासां॒ ॅया स्त्रिꣳ॒॒शत् त्रिꣳ॒॒शद् या स्तासा॒म् तासां॒ ॅया स्त्रिꣳ॒॒शत् । \newline
62. या स्त्रिꣳ॒॒शत् त्रिꣳ॒॒शद् या या स्त्रिꣳ॒॒शत् त्रिꣳ॒॒शद॑क्षरा त्रिꣳ॒॒शद॑क्षरा त्रिꣳ॒॒शद् या या स्त्रिꣳ॒॒शत् त्रिꣳ॒॒शद॑क्षरा । \newline
63. त्रिꣳ॒॒शत् त्रिꣳ॒॒शद॑क्षरा त्रिꣳ॒॒शद॑क्षरा त्रिꣳ॒॒शत् त्रिꣳ॒॒शत् त्रिꣳ॒॒शद॑क्षरा वि॒राड् वि॒राट् त्रिꣳ॒॒शद॑क्षरा त्रिꣳ॒॒शत् त्रिꣳ॒॒शत् त्रिꣳ॒॒शद॑क्षरा वि॒राट् । \newline
64. त्रिꣳ॒॒शद॑क्षरा वि॒राड् वि॒राट् त्रिꣳ॒॒शद॑क्षरा त्रिꣳ॒॒शद॑क्षरा वि॒रा डन्न॒ मन्नं॑ ॅवि॒राट् त्रिꣳ॒॒शद॑क्षरा त्रिꣳ॒॒शद॑क्षरा वि॒रा डन्न᳚म् । \newline
65. त्रिꣳ॒॒शद॑क्ष॒रेति॑ त्रिꣳ॒॒शत् - अ॒क्ष॒रा॒ । \newline
66. वि॒रा डन्न॒ मन्नं॑ ॅवि॒राड् वि॒रा डन्नं॑ ॅवि॒राड् वि॒रा डन्नं॑ ॅवि॒राड् वि॒रा डन्नं॑ ॅवि॒राट् । \newline
67. वि॒राडिति॑ वि - राट् । \newline
68. अन्नं॑ ॅवि॒राड् वि॒रा डन्न॒ मन्नं॑ ॅवि॒राड् वि॒राजा॑ वि॒राजा॑ वि॒रा डन्न॒ मन्नं॑ ॅवि॒राड् वि॒राजा᳚ । \newline
69. वि॒राड् वि॒राजा॑ वि॒राजा॑ वि॒राड् वि॒राड् वि॒राजै॒वैव वि॒राजा॑ वि॒राड् वि॒राड् वि॒राजै॒व । \newline
70. वि॒राडिति॑ वि - राट् । \newline
71. वि॒राजै॒वैव वि॒राजा॑ वि॒राजै॒ वान्नाद्य॑ म॒न्नाद्य॑ मे॒व वि॒राजा॑ वि॒राजै॒ वान्नाद्य᳚म् । \newline
72. वि॒राजेति॑ वि - राजा᳚ । \newline
73. ए॒वान्नाद्य॑ म॒न्नाद्य॑ मे॒वै वान्नाद्य॒ मवावा॒न्नाद्य॑ मे॒वै वान्नाद्य॒ मव॑ । \newline
74. अ॒न्नाद्य॒ मवा वा॒न्नाद्य॑ म॒न्नाद्य॒ मव॑ रुन्धते रुन्ध॒ते ऽवा॒न्नाद्य॑ म॒न्नाद्य॒ मव॑ रुन्धते । \newline
75. अ॒न्नाद्य॒मित्य॑न्न - अद्य᳚म् । \newline
76. अव॑ रुन्धते रुन्ध॒ते ऽवाव॑ रुन्धते॒ ये ये रु॑न्ध॒ते ऽवाव॑ रुन्धते॒ ये । \newline
77. रु॒न्ध॒ते॒ ये ये रु॑न्धते रुन्धते॒ ये द्वे द्वे ये रु॑न्धते रुन्धते॒ ये द्वे । \newline
78. ये द्वे द्वे ये ये द्वे अ॑होरा॒त्रे अ॑होरा॒त्रे द्वे ये ये द्वे अ॑होरा॒त्रे । \newline
79. ये इति॒ ये । \newline
80. द्वे अ॑होरा॒त्रे अ॑होरा॒त्रे द्वे द्वे अ॑होरा॒त्रे ए॒वै वाहो॑रा॒त्रे द्वे द्वे अ॑होरा॒त्रे ए॒व । \newline
81. द्वे इति॒ द्वे । \newline
82. अ॒हो॒रा॒त्रे ए॒वै वाहो॑रा॒त्रे अ॑होरा॒त्रे ए॒व ते ते ए॒वा हो॑रा॒त्रे अ॑होरा॒त्रे ए॒व ते । \newline
83. अ॒हो॒रा॒त्रे इत्य॑हः - रा॒त्रे । \newline
84. ए॒व ते ते ए॒वैव ते उ॒भाभ्या॑ मु॒भाभ्या॒म् ते ए॒वैव ते उ॒भाभ्या᳚म् । \newline
85. ते उ॒भाभ्या॑ मु॒भाभ्या॒म् ते ते उ॒भाभ्याꣳ॑ रू॒पाभ्याꣳ॑ रू॒पाभ्या॑ मु॒भाभ्या॒म् ते ते उ॒भाभ्याꣳ॑ रू॒पाभ्या᳚म् । \newline
86. ते इति॒ ते । \newline
87. उ॒भाभ्याꣳ॑ रू॒पाभ्याꣳ॑ रू॒पाभ्या॑ मु॒भाभ्या॑ मु॒भाभ्याꣳ॑ रू॒पाभ्याꣳ॑ सुव॒र्गꣳ सु॑व॒र्गꣳ रू॒पाभ्या॑ मु॒भाभ्या॑ मु॒भाभ्याꣳ॑ रू॒पाभ्याꣳ॑ सुव॒र्गम् । \newline
88. रू॒पाभ्याꣳ॑ सुव॒र्गꣳ सु॑व॒र्गꣳ रू॒पाभ्याꣳ॑ रू॒पाभ्याꣳ॑ सुव॒र्गम् ॅलो॒कम् ॅलो॒कꣳ सु॑व॒र्गꣳ रू॒पाभ्याꣳ॑ रू॒पाभ्याꣳ॑ सुव॒र्गम् ॅलो॒कम् । \newline
89. सु॒व॒र्गम् ॅलो॒कम् ॅलो॒कꣳ सु॑व॒र्गꣳ सु॑व॒र्गम् ॅलो॒कं ॅय॑न्ति यन्ति लो॒कꣳ सु॑व॒र्गꣳ सु॑व॒र्गम् ॅलो॒कं ॅय॑न्ति । \newline
90. सु॒व॒र्गमिति॑ सुवः - गम् । \newline
91. लो॒कं ॅय॑न्ति यन्ति लो॒कम् ॅलो॒कं ॅय॑न्त्यतिरा॒त्रा व॑तिरा॒त्रौ य॑न्ति लो॒कम् ॅलो॒कं ॅय॑न्त्यतिरा॒त्रौ । \newline
92. य॒न्त्य॒ति॒रा॒त्रा व॑तिरा॒त्रौ य॑न्ति यन्त्यतिरा॒त्रा व॒भितो॒ ऽभितो॑ ऽतिरा॒त्रौ य॑न्ति यन्त्यतिरा॒त्रा व॒भितः॑ । \newline
93. अ॒ति॒रा॒त्रा व॒भितो॒ ऽभितो॑ ऽतिरा॒त्रा व॑तिरा॒त्रा व॒भितो॑ भवतो भवतो॒ ऽभितो॑ ऽतिरा॒त्रा व॑तिरा॒त्रा व॒भितो॑ भवतः । \newline
94. अ॒ति॒रा॒त्रावित्य॑ति - रा॒त्रौ । \newline
95. अ॒भितो॑ भवतो भवतो॒ ऽभितो॒ ऽभितो॑ भवतः॒ परि॑गृहीत्यै॒ परि॑गृहीत्यै भवतो॒ ऽभितो॒ ऽभितो॑ भवतः॒ परि॑गृहीत्यै । \newline
96. भ॒व॒तः॒ परि॑गृहीत्यै॒ परि॑गृहीत्यै भवतो भवतः॒ परि॑गृहीत्यै । \newline
97. परि॑गृहीत्या॒ इति॒ परि॑ - गृ॒ही॒त्यै॒ । \newline
\pagebreak
\markright{ TS 7.4.5.1  \hfill https://www.vedavms.in \hfill}

\section{ TS 7.4.5.1 }

\textbf{TS 7.4.5.1 } \newline
\textbf{Samhita Paata} \newline

द्वे वाव दे॑वस॒त्रे द्वा॑दशा॒हश्चै॒व त्र॑यस्त्रिꣳशद॒हश्च॒ य ए॒वं ॅवि॒द्वाꣳस॑स्त्रयस्त्रिꣳशद॒हमास॑ते सा॒क्षादे॒व दे॒वता॑ अ॒भ्यारो॑हन्ति॒ यथा॒ खलु॒ वै श्रेया॑न॒भ्यारू॑ढः का॒मय॑ते॒ तथा॑ करोति॒ यद्य॑व॒विद्ध्य॑ति॒ पापी॑यान् भवति॒ यदि॒ नाव॒विद्ध्य॑ति स॒दृङ् य ए॒वं ॅवि॒द्वाꣳस॑स्त्रयस्त्रिꣳ- शद॒हमास॑ते॒ वि पा॒प्मना॒ भ्रातृ॑व्ये॒णाऽऽ* व॑र्तन्ते ऽह॒र्भाजो॒ वा ए॒ता दे॒वा अग्र॒ आऽह॑र॒ - [  ] \newline

\textbf{Pada Paata} \newline

द्वे इति॑ । वाव । दे॒व॒स॒त्रे इति॑ देव - स॒त्रे । द्वा॒द॒शा॒ह इति॑ द्वादश-अ॒हः । च॒ । ए॒व । त्र॒य॒स्त्रिꣳ॒॒श॒द॒ह इति॑ त्रयस्त्रिꣳशत्-अ॒हः । च॒ । ये । ए॒वम् । वि॒द्वाꣳसः॑ । त्र॒य॒स्त्रिꣳ॒॒श॒द॒हमिति॑ त्रयस्त्रिꣳशत्-अ॒हम् । आस॑ते । सा॒क्षादिति॑ स - आ॒क्षात् । ए॒व । दे॒वताः᳚ । अ॒भ्यारो॑ह॒न्तीत्य॑भि - आरो॑हन्ति । यथा᳚ । खलु॑ । वै । श्रेयान्॑ । अ॒भ्यारू॑ढ॒ इत्य॑भि - आरू॑ढः । का॒मय॑ते । तथा᳚ । क॒रो॒ति॒ । यदि॑ । अ॒व॒विद्ध्य॒तीत्य॑व - विद्ध्य॑ति । पापी॑यान् । भ॒व॒ति॒ । यदि॑ । न । अ॒व॒विद्ध्य॒तीत्य॑व - विद्ध्य॑ति । स॒दृङ्ङिति॑ स - दृङ् । ये । ए॒वम् । वि॒द्वाꣳसः॑ । त्र॒य॒स्त्रिꣳ॒॒श॒द॒हमिति॑ त्रयस्त्रिꣳशत् - अ॒हम् । आस॑ते । वीति॑ । पा॒प्मना᳚ । भ्रातृ॑व्येण । एति॑ । व॒र्त॒न्ते॒ । अ॒ह॒र्भाज॒ इत्य॑हः - भाजः॑ । वै । ए॒ताः । दे॒वाः । अग्रे᳚ । एति॑ । अ॒ह॒र॒न्न् ।  \newline


\textbf{Krama Paata} \newline

द्वे वाव । द्वे इति॒ द्वे । वाव दे॑वस॒त्रे । दे॒व॒स॒त्रे द्वा॑दशा॒हः । दे॒व॒स॒त्रे इति॑ देव - स॒त्रे । द्वा॒द॒शा॒हश्च॑ । द्वा॒द॒शा॒ह इति॑ द्वादश - अ॒हः । चै॒व । ए॒व त्र॑यस्त्रिꣳशद॒हः । त्र॒य॒स्त्रिꣳ॒॒श॒द॒हश्च॑ । त्र॒य॒स्त्रिꣳ॒॒श॒द॒ह इति॑ त्रयस्त्रिꣳशत् - अ॒हः । च॒ ये । य ए॒वम् । ए॒वम् ॅवि॒द्वाꣳसः॑ । वि॒द्वाꣳस॑स्त्रयस्त्रिꣳशद॒हम् । त्र॒य॒स्त्रिꣳ॒॒श॒द॒हमास॑ते । त्र॒य॒स्त्रिꣳ॒॒श॒द॒हमिति॑ त्रयस्त्रिꣳशत् - अ॒हम् । आस॑ते सा॒क्षात् । सा॒क्षादे॒व । सा॒क्षादिति॑ स - अ॒क्षात् । ए॒व दे॒वताः᳚ । दे॒वता॑ अ॒भ्यारो॑हन्ति । अ॒भ्यारो॑हन्ति॒ यथा᳚ । अ॒भ्यारो॑ह॒न्तीत्य॑भि - आरो॑हन्ति । यथा॒ खलु॑ । खलु॒ वै । वै श्रेयान्॑ । श्रेया॑न॒भ्यारू॑ढः । अ॒भ्यारू॑ढः का॒मय॑ते । अ॒भ्यारू॑ढ॒ इत्य॑भि - आरू॑ढः । का॒मय॑ते॒ तथा᳚ । तथा॑ करोति । क॒रो॒ति॒ यदि॑ । यद्य॑व॒विद्ध्य॑ति । अ॒व॒विद्ध्य॑ति॒ पापी॑यान् । अ॒व॒विद्ध्य॒तीत्य॑व - विद्ध्य॑ति । पापी॑यान् भवति । भ॒व॒ति॒ यदि॑ । यदि॒ न । नाव॒विद्ध्य॑ति । अ॒व॒विद्ध्य॑ति स॒दृङ्‍ङ् । अ॒व॒विद्ध्य॒तीत्य॑व - विद्ध्य॑ति । स॒दृङ्‍ ये । स॒दृङ्ङिति॑ स - दृङ्‍ । य ए॒वम् । ए॒वम् ॅवि॒द्वाꣳसः॑ । वि॒द्वाꣳस॑स्त्रयस्त्रिꣳशद॒हम् । त्र॒य॒स्त्रिꣳ॒॒श॒द॒हमास॑ते । त्र॒य॒स्त्रिꣳ॒॒श॒द॒हमिति॑ त्रयस्त्रिꣳशत् - अ॒हम् । आस॑ते॒ वि । वि पा॒प्मना᳚ । पा॒प्मना॒ भ्रातृ॑व्येण । भ्रातृ॑व्ये॒णा । आ व॑र्तन्ते । व॒र्त॒न्ते॒ऽह॒र्भाजः॑ । अ॒ह॒र्भाजो॒ वै । अ॒ह॒र्भाज॒ इत्य॑हः - भाजः॑ । वा ए॒ताः । ए॒ता दे॒वाः । दे॒वा अग्रे᳚ । अग्र॒ आ । आऽह॑रन्न् । अ॒ह॒र॒न्नहः॑ \newline

\textbf{Jatai Paata} \newline

1. द्वे वाव वाव द्वे द्वे वाव । \newline
2. द्वे इति॒ द्वे । \newline
3. वाव दे॑वस॒त्रे दे॑वस॒त्रे वाव वाव दे॑वस॒त्रे । \newline
4. दे॒व॒स॒त्रे द्वा॑दशा॒हो द्वा॑दशा॒हो दे॑वस॒त्रे दे॑वस॒त्रे द्वा॑दशा॒हः । \newline
5. दे॒व॒स॒त्रे इति॑ देव - स॒त्रे । \newline
6. द्वा॒द॒शा॒ह श्च॑ च द्वादशा॒हो द्वा॑दशा॒ह श्च॑ । \newline
7. द्वा॒द॒शा॒ह इति॑ द्वादश - अ॒हः । \newline
8. चै॒वैव च॑ चै॒व । \newline
9. ए॒व त्र॑यस्त्रिꣳशद॒ह स्त्र॑यस्त्रिꣳशद॒ह ए॒वैव त्र॑यस्त्रिꣳशद॒हः । \newline
10. त्र॒य॒स्त्रिꣳ॒॒श॒द॒ह श्च॑ च त्रयस्त्रिꣳशद॒ह स्त्र॑यस्त्रिꣳशद॒ह श्च॑ । \newline
11. त्र॒य॒स्त्रिꣳ॒॒श॒द॒ह इति॑ त्रयस्त्रिꣳशत् - अ॒हः । \newline
12. च॒ ये ये च॑ च॒ ये । \newline
13. य ए॒व मे॒वं ॅये य ए॒वम् । \newline
14. ए॒वं ॅवि॒द्वाꣳसो॑ वि॒द्वाꣳस॑ ए॒व मे॒वं ॅवि॒द्वाꣳसः॑ । \newline
15. वि॒द्वाꣳस॑ स्त्रयस्त्रिꣳशद॒हम् त्र॑यस्त्रिꣳशद॒हं ॅवि॒द्वाꣳसो॑ वि॒द्वाꣳस॑ स्त्रयस्त्रिꣳशद॒हम् । \newline
16. त्र॒य॒स्त्रिꣳ॒॒श॒द॒ह मास॑त॒ आस॑ते त्रयस्त्रिꣳशद॒हम् त्र॑यस्त्रिꣳशद॒ह मास॑ते । \newline
17. त्र॒य॒स्त्रिꣳ॒॒श॒द॒हमिति॑ त्रयस्त्रिꣳशत् - अ॒हम् । \newline
18. आस॑ते सा॒क्षाथ् सा॒क्षा दास॑त॒ आस॑ते सा॒क्षात् । \newline
19. सा॒क्षा दे॒वैव सा॒क्षाथ् सा॒क्षा दे॒व । \newline
20. सा॒क्षादिति॑ स - अ॒क्षात् । \newline
21. ए॒व दे॒वता॑ दे॒वता॑ ए॒वैव दे॒वताः᳚ । \newline
22. दे॒वता॑ अ॒भ्यारो॑ह न्त्य॒भ्यारो॑हन्ति दे॒वता॑ दे॒वता॑ अ॒भ्यारो॑हन्ति । \newline
23. अ॒भ्यारो॑हन्ति॒ यथा॒ यथा॒ ऽभ्यारो॑ह न्त्य॒भ्यारो॑हन्ति॒ यथा᳚ । \newline
24. अ॒भ्यारो॑ह॒न्तीत्य॑भि - आरो॑हन्ति । \newline
25. यथा॒ खलु॒ खलु॒ यथा॒ यथा॒ खलु॑ । \newline
26. खलु॒ वै वै खलु॒ खलु॒ वै । \newline
27. वै श्रेया॒ञ् छ्रेया॒न्॒. वै वै श्रेयान्॑ । \newline
28. श्रेया॑ न॒भ्यारू॑ढो॒ ऽभ्यारू॑ढः॒ श्रेया॒ञ् छ्रेया॑ न॒भ्यारू॑ढः । \newline
29. अ॒भ्यारू॑ढः का॒मय॑ते का॒मय॑ते॒ ऽभ्यारू॑ढो॒ ऽभ्यारू॑ढः का॒मय॑ते । \newline
30. अ॒भ्यारू॑ढ॒ इत्य॑भि - आरू॑ढः । \newline
31. का॒मय॑ते॒ तथा॒ तथा॑ का॒मय॑ते का॒मय॑ते॒ तथा᳚ । \newline
32. तथा॑ करोति करोति॒ तथा॒ तथा॑ करोति । \newline
33. क॒रो॒ति॒ यदि॒ यदि॑ करोति करोति॒ यदि॑ । \newline
34. यद्य॑व॒विद्ध्य॑ त्यव॒विद्ध्य॑ति॒ यदि॒ यद्य॑व॒विद्ध्य॑ति । \newline
35. अ॒व॒विद्ध्य॑ति॒ पापी॑या॒न् पापी॑या नव॒विद्ध्य॑ त्यव॒विद्ध्य॑ति॒ पापी॑यान् । \newline
36. अ॒व॒विद्ध्य॒तीत्य॑व - विद्ध्य॑ति । \newline
37. पापी॑यान् भवति भवति॒ पापी॑या॒न् पापी॑यान् भवति । \newline
38. भ॒व॒ति॒ यदि॒ यदि॑ भवति भवति॒ यदि॑ । \newline
39. यदि॒ न न यदि॒ यदि॒ न । \newline
40. नाव॒विद्ध्य॑ त्यव॒विद्ध्य॑ति॒ न नाव॒विद्ध्य॑ति । \newline
41. अ॒व॒विद्ध्य॑ति स॒दृङ् ख्स॒दृङ् ङ॑व॒विद्ध्य॑ त्यव॒विद्ध्य॑ति स॒दृङ् । \newline
42. अ॒व॒विद्ध्य॒तीत्य॑व - विद्ध्य॑ति । \newline
43. स॒दृङ् ये ये स॒दृङ् ख्स॒दृङ् ये । \newline
44. स॒दृङ्ङिति॑ स - दृङ् । \newline
45. य ए॒व मे॒वं ॅये य ए॒वम् । \newline
46. ए॒वं ॅवि॒द्वाꣳसो॑ वि॒द्वाꣳस॑ ए॒व मे॒वं ॅवि॒द्वाꣳसः॑ । \newline
47. वि॒द्वाꣳस॑ स्त्रयस्त्रिꣳशद॒हम् त्र॑यस्त्रिꣳशद॒हं ॅवि॒द्वाꣳसो॑ वि॒द्वाꣳस॑ स्त्रयस्त्रिꣳशद॒हम् । \newline
48. त्र॒य॒स्त्रिꣳ॒॒श॒द॒ह मास॑त॒ आस॑ते त्रयस्त्रिꣳशद॒हम् त्र॑यस्त्रिꣳशद॒ह मास॑ते । \newline
49. त्र॒य॒स्त्रिꣳ॒॒श॒द॒हमिति॑ त्रयस्त्रिꣳशत् - अ॒हम् । \newline
50. आस॑ते॒ वि व्यास॑त॒ आस॑ते॒ वि । \newline
51. वि पा॒प्मना॑ पा॒प्मना॒ वि वि पा॒प्मना᳚ । \newline
52. पा॒प्मना॒ भ्रातृ॑व्येण॒ भ्रातृ॑व्येण पा॒प्मना॑ पा॒प्मना॒ भ्रातृ॑व्येण । \newline
53. भ्रातृ॑व्ये॒णा भ्रातृ॑व्येण॒ भ्रातृ॑व्ये॒णा । \newline
54. आ व॑र्तन्ते वर्तन्त॒ आ व॑र्तन्ते । \newline
55. व॒र्त॒न्ते॒ ऽह॒र्भाजो॑ ऽह॒र्भाजो॑ वर्तन्ते वर्तन्ते ऽह॒र्भाजः॑ । \newline
56. अ॒ह॒र्भाजो॒ वै वा अ॑ह॒र्भाजो॑ ऽह॒र्भाजो॒ वै । \newline
57. अ॒ह॒र्भाज॒ इत्य॑हः - भाजः॑ । \newline
58. वा ए॒ता ए॒ता वै वा ए॒ताः । \newline
59. ए॒ता दे॒वा दे॒वा ए॒ता ए॒ता दे॒वाः । \newline
60. दे॒वा अग्रे ऽग्रे॑ दे॒वा दे॒वा अग्रे᳚ । \newline
61. अग्र॒ आ ऽग्रे ऽग्र॒ आ । \newline
62. आ ऽह॑रन् नहर॒न् ना ऽह॑रन्न् । \newline
63. अ॒ह॒र॒न् नह॒ रह॑ रहरन् नहर॒न् नहः॑ । \newline

\textbf{Ghana Paata } \newline

1. द्वे वाव वाव द्वे द्वे वाव दे॑वस॒त्रे दे॑वस॒त्रे वाव द्वे द्वे वाव दे॑वस॒त्रे । \newline
2. द्वे इति॒ द्वे । \newline
3. वाव दे॑वस॒त्रे दे॑वस॒त्रे वाव वाव दे॑वस॒त्रे द्वा॑दशा॒हो द्वा॑दशा॒हो दे॑वस॒त्रे वाव वाव दे॑वस॒त्रे द्वा॑दशा॒हः । \newline
4. दे॒व॒स॒त्रे द्वा॑दशा॒हो द्वा॑दशा॒हो दे॑वस॒त्रे दे॑वस॒त्रे द्वा॑दशा॒ह श्च॑ च द्वादशा॒हो दे॑वस॒त्रे दे॑वस॒त्रे द्वा॑दशा॒ह श्च॑ । \newline
5. दे॒व॒स॒त्रे इति॑ देव - स॒त्रे । \newline
6. द्वा॒द॒शा॒ह श्च॑ च द्वादशा॒हो द्वा॑दशा॒ह श्चै॒वैव च॑ द्वादशा॒हो द्वा॑दशा॒ह श्चै॒व । \newline
7. द्वा॒द॒शा॒ह इति॑ द्वादश - अ॒हः । \newline
8. चै॒वैव च॑ चै॒व त्र॑यस्त्रिꣳशद॒ह स्त्र॑यस्त्रिꣳशद॒ह ए॒व च॑ चै॒व त्र॑यस्त्रिꣳशद॒हः । \newline
9. ए॒व त्र॑यस्त्रिꣳशद॒ह स्त्र॑यस्त्रिꣳशद॒ह ए॒वैव त्र॑यस्त्रिꣳशद॒ह श्च॑ च त्रयस्त्रिꣳशद॒ह ए॒वैव त्र॑यस्त्रिꣳशद॒ह श्च॑ । \newline
10. त्र॒य॒स्त्रिꣳ॒॒श॒द॒ह श्च॑ च त्रयस्त्रिꣳशद॒ह स्त्र॑यस्त्रिꣳशद॒ह श्च॒ ये ये च॑ त्रयस्त्रिꣳशद॒ह स्त्र॑यस्त्रिꣳशद॒ह श्च॒ ये । \newline
11. त्र॒य॒स्त्रिꣳ॒॒श॒द॒ह इति॑ त्रयस्त्रिꣳशत् - अ॒हः । \newline
12. च॒ ये ये च॑ च॒ य ए॒व मे॒वं ॅये च॑ च॒ य ए॒वम् । \newline
13. य ए॒व मे॒वं ॅये य ए॒वं ॅवि॒द्वाꣳसो॑ वि॒द्वाꣳस॑ ए॒वं ॅये य ए॒वं ॅवि॒द्वाꣳसः॑ । \newline
14. ए॒वं ॅवि॒द्वाꣳसो॑ वि॒द्वाꣳस॑ ए॒व मे॒वं ॅवि॒द्वाꣳस॑ स्त्रयस्त्रिꣳशद॒हम् त्र॑यस्त्रिꣳशद॒हं ॅवि॒द्वाꣳस॑ ए॒व मे॒वं ॅवि॒द्वाꣳस॑ स्त्रयस्त्रिꣳशद॒हम् । \newline
15. वि॒द्वाꣳस॑ स्त्रयस्त्रिꣳशद॒हम् त्र॑यस्त्रिꣳशद॒हं ॅवि॒द्वाꣳसो॑ वि॒द्वाꣳस॑ स्त्रयस्त्रिꣳशद॒ह मास॑त॒ आस॑ते त्रयस्त्रिꣳशद॒हं ॅवि॒द्वाꣳसो॑ वि॒द्वाꣳस॑ स्त्रयस्त्रिꣳशद॒ह मास॑ते । \newline
16. त्र॒य॒स्त्रिꣳ॒॒श॒द॒ह मास॑त॒ आस॑ते त्रयस्त्रिꣳशद॒हम् त्र॑यस्त्रिꣳशद॒ह मास॑ते सा॒क्षाथ् सा॒क्षा दास॑ते त्रयस्त्रिꣳशद॒हम् त्र॑यस्त्रिꣳशद॒ह मास॑ते सा॒क्षात् । \newline
17. त्र॒य॒स्त्रिꣳ॒॒श॒द॒हमिति॑ त्रयस्त्रिꣳशत् - अ॒हम् । \newline
18. आस॑ते सा॒क्षाथ् सा॒क्षा दास॑त॒ आस॑ते सा॒क्षा दे॒वैव सा॒क्षा दास॑त॒ आस॑ते सा॒क्षा दे॒व । \newline
19. सा॒क्षा दे॒वैव सा॒क्षाथ् सा॒क्षा दे॒व दे॒वता॑ दे॒वता॑ ए॒व सा॒क्षाथ् सा॒क्षा दे॒व दे॒वताः᳚ । \newline
20. सा॒क्षादिति॑ स - अ॒क्षात् । \newline
21. ए॒व दे॒वता॑ दे॒वता॑ ए॒वैव दे॒वता॑ अ॒भ्यारो॑ह न्त्य॒भ्यारो॑हन्ति दे॒वता॑ ए॒वैव दे॒वता॑ अ॒भ्यारो॑हन्ति । \newline
22. दे॒वता॑ अ॒भ्यारो॑ह न्त्य॒भ्यारो॑हन्ति दे॒वता॑ दे॒वता॑ अ॒भ्यारो॑हन्ति॒ यथा॒ यथा॒ ऽभ्यारो॑हन्ति दे॒वता॑ दे॒वता॑ अ॒भ्यारो॑हन्ति॒ यथा᳚ । \newline
23. अ॒भ्यारो॑हन्ति॒ यथा॒ यथा॒ ऽभ्यारो॑ह न्त्य॒भ्यारो॑हन्ति॒ यथा॒ खलु॒ खलु॒ यथा॒ ऽभ्यारो॑ह न्त्य॒भ्यारो॑हन्ति॒ यथा॒ खलु॑ । \newline
24. अ॒भ्यारो॑ह॒न्तीत्य॑भि - आरो॑हन्ति । \newline
25. यथा॒ खलु॒ खलु॒ यथा॒ यथा॒ खलु॒ वै वै खलु॒ यथा॒ यथा॒ खलु॒ वै । \newline
26. खलु॒ वै वै खलु॒ खलु॒ वै श्रेया॒ञ् छ्रेया॒न्॒. वै खलु॒ खलु॒ वै श्रेयान्॑ । \newline
27. वै श्रेया॒ञ् छ्रेया॒न्॒. वै वै श्रेया॑ न॒भ्यारू॑ढो॒ ऽभ्यारू॑ढः॒ श्रेया॒न्॒. वै वै श्रेया॑ न॒भ्यारू॑ढः । \newline
28. श्रेया॑ न॒भ्यारू॑ढो॒ ऽभ्यारू॑ढः॒ श्रेया॒ञ् छ्रेया॑ न॒भ्यारू॑ढः का॒मय॑ते का॒मय॑ते॒ ऽभ्यारू॑ढः॒ श्रेया॒ञ् छ्रेया॑ न॒भ्यारू॑ढः का॒मय॑ते । \newline
29. अ॒भ्यारू॑ढः का॒मय॑ते का॒मय॑ते॒ ऽभ्यारू॑ढो॒ ऽभ्यारू॑ढः का॒मय॑ते॒ तथा॒ तथा॑ का॒मय॑ते॒ ऽभ्यारू॑ढो॒ ऽभ्यारू॑ढः का॒मय॑ते॒ तथा᳚ । \newline
30. अ॒भ्यारू॑ढ॒ इत्य॑भि - आरू॑ढः । \newline
31. का॒मय॑ते॒ तथा॒ तथा॑ का॒मय॑ते का॒मय॑ते॒ तथा॑ करोति करोति॒ तथा॑ का॒मय॑ते का॒मय॑ते॒ तथा॑ करोति । \newline
32. तथा॑ करोति करोति॒ तथा॒ तथा॑ करोति॒ यदि॒ यदि॑ करोति॒ तथा॒ तथा॑ करोति॒ यदि॑ । \newline
33. क॒रो॒ति॒ यदि॒ यदि॑ करोति करोति॒ यद्य॑व॒विद्ध्य॑ त्यव॒विद्ध्य॑ति॒ यदि॑ करोति करोति॒ यद्य॑व॒विद्ध्य॑ति । \newline
34. यद्य॑व॒विद्ध्य॑ त्यव॒विद्ध्य॑ति॒ यदि॒ यद्य॑व॒विद्ध्य॑ति॒ पापी॑या॒न् पापी॑या नव॒विद्ध्य॑ति॒ यदि॒ यद्य॑व॒विद्ध्य॑ति॒ पापी॑यान् । \newline
35. अ॒व॒विद्ध्य॑ति॒ पापी॑या॒न् पापी॑या नव॒विद्ध्य॑ त्यव॒विद्ध्य॑ति॒ पापी॑यान् भवति भवति॒ पापी॑या नव॒विद्ध्य॑ त्यव॒विद्ध्य॑ति॒ पापी॑यान् भवति । \newline
36. अ॒व॒विद्ध्य॒तीत्य॑व - विद्ध्य॑ति । \newline
37. पापी॑यान् भवति भवति॒ पापी॑या॒न् पापी॑यान् भवति॒ यदि॒ यदि॑ भवति॒ पापी॑या॒न् पापी॑यान् भवति॒ यदि॑ । \newline
38. भ॒व॒ति॒ यदि॒ यदि॑ भवति भवति॒ यदि॒ न न यदि॑ भवति भवति॒ यदि॒ न । \newline
39. यदि॒ न न यदि॒ यदि॒ नाव॒विद्ध्य॑ त्यव॒विद्ध्य॑ति॒ न यदि॒ यदि॒ नाव॒विद्ध्य॑ति । \newline
40. नाव॒विद्ध्य॑ त्यव॒विद्ध्य॑ति॒ न नाव॒विद्ध्य॑ति स॒दृङ् ख्स॒दृङ् ङ॑व॒विद्ध्य॑ति॒ न नाव॒विद्ध्य॑ति स॒दृङ् । \newline
41. अ॒व॒विद्ध्य॑ति स॒दृङ् ख्स॒दृङ् ङ॑व॒विद्ध्य॑ त्यव॒विद्ध्य॑ति स॒दृङ् ये ये स॒दृङ् ङ॑व॒विद्ध्य॑ त्यव॒विद्ध्य॑ति स॒दृङ् ये । \newline
42. अ॒व॒विद्ध्य॒तीत्य॑व - विद्ध्य॑ति । \newline
43. स॒दृङ् ये ये स॒दृङ् ख्स॒दृङ् य ए॒व मे॒वं ॅये स॒दृङ् ख्स॒दृङ् य ए॒वम् । \newline
44. स॒दृङ्ङिति॑ स - दृङ् । \newline
45. य ए॒व मे॒वं ॅये य ए॒वं ॅवि॒द्वाꣳसो॑ वि॒द्वाꣳस॑ ए॒वं ॅये य ए॒वं ॅवि॒द्वाꣳसः॑ । \newline
46. ए॒वं ॅवि॒द्वाꣳसो॑ वि॒द्वाꣳस॑ ए॒व मे॒वं ॅवि॒द्वाꣳस॑ स्त्रयस्त्रिꣳशद॒हम् त्र॑यस्त्रिꣳशद॒हं ॅवि॒द्वाꣳस॑ ए॒व मे॒वं ॅवि॒द्वाꣳस॑ स्त्रयस्त्रिꣳशद॒हम् । \newline
47. वि॒द्वाꣳस॑ स्त्रयस्त्रिꣳशद॒हम् त्र॑यस्त्रिꣳशद॒हं ॅवि॒द्वाꣳसो॑ वि॒द्वाꣳस॑ स्त्रयस्त्रिꣳशद॒ह मास॑त॒ आस॑ते त्रयस्त्रिꣳशद॒हं ॅवि॒द्वाꣳसो॑ वि॒द्वाꣳस॑ स्त्रयस्त्रिꣳशद॒ह मास॑ते । \newline
48. त्र॒य॒स्त्रिꣳ॒॒श॒द॒ह मास॑त॒ आस॑ते त्रयस्त्रिꣳशद॒हम् त्र॑यस्त्रिꣳशद॒ह मास॑ते॒ वि व्यास॑ते त्रयस्त्रिꣳशद॒हम् त्र॑यस्त्रिꣳशद॒ह मास॑ते॒ वि । \newline
49. त्र॒य॒स्त्रिꣳ॒॒श॒द॒हमिति॑ त्रयस्त्रिꣳशत् - अ॒हम् । \newline
50. आस॑ते॒ वि व्यास॑त॒ आस॑ते॒ वि पा॒प्मना॑ पा॒प्मना॒ व्यास॑त॒ आस॑ते॒ वि पा॒प्मना᳚ । \newline
51. वि पा॒प्मना॑ पा॒प्मना॒ वि वि पा॒प्मना॒ भ्रातृ॑व्येण॒ भ्रातृ॑व्येण पा॒प्मना॒ वि वि पा॒प्मना॒ भ्रातृ॑व्येण । \newline
52. पा॒प्मना॒ भ्रातृ॑व्येण॒ भ्रातृ॑व्येण पा॒प्मना॑ पा॒प्मना॒ भ्रातृ॑व्ये॒णा भ्रातृ॑व्येण पा॒प्मना॑ पा॒प्मना॒ भ्रातृ॑व्ये॒णा । \newline
53. भ्रातृ॑व्ये॒णा भ्रातृ॑व्येण॒ भ्रातृ॑व्ये॒णा व॑र्तन्ते वर्तन्त॒ आ भ्रातृ॑व्येण॒ भ्रातृ॑व्ये॒णा व॑र्तन्ते । \newline
54. आ व॑र्तन्ते वर्तन्त॒ आ व॑र्तन्ते ऽह॒र्भाजो॑ ऽह॒र्भाजो॑ वर्तन्त॒ आ व॑र्तन्ते ऽह॒र्भाजः॑ । \newline
55. व॒र्त॒न्ते॒ ऽह॒र्भाजो॑ ऽह॒र्भाजो॑ वर्तन्ते वर्तन्ते ऽह॒र्भाजो॒ वै वा अ॑ह॒र्भाजो॑ वर्तन्ते वर्तन्ते ऽह॒र्भाजो॒ वै । \newline
56. अ॒ह॒र्भाजो॒ वै वा अ॑ह॒र्भाजो॑ ऽह॒र्भाजो॒ वा ए॒ता ए॒ता वा अ॑ह॒र्भाजो॑ ऽह॒र्भाजो॒ वा ए॒ताः । \newline
57. अ॒ह॒र्भाज॒ इत्य॑हः - भाजः॑ । \newline
58. वा ए॒ता ए॒ता वै वा ए॒ता दे॒वा दे॒वा ए॒ता वै वा ए॒ता दे॒वाः । \newline
59. ए॒ता दे॒वा दे॒वा ए॒ता ए॒ता दे॒वा अग्रे ऽग्रे॑ दे॒वा ए॒ता ए॒ता दे॒वा अग्रे᳚ । \newline
60. दे॒वा अग्रे ऽग्रे॑ दे॒वा दे॒वा अग्र॒ आ ऽग्रे॑ दे॒वा दे॒वा अग्र॒ आ । \newline
61. अग्र॒ आ ऽग्रे ऽग्र॒ आ ऽह॑रन् नहर॒न् ना ऽग्रे ऽग्र॒ आ ऽह॑रन्न् । \newline
62. आ ऽह॑रन् नहर॒न् ना ऽह॑र॒न् नह॒ रह॑ रहर॒न् ना ऽह॑र॒न् नहः॑ । \newline
63. अ॒ह॒र॒न् नह॒ रह॑ रहरन् नहर॒न् नह॒ रेक॒ एको ऽह॑ रहरन् नहर॒न् नह॒ रेकः॑ । \newline
\pagebreak
\markright{ TS 7.4.5.2  \hfill https://www.vedavms.in \hfill}

\section{ TS 7.4.5.2 }

\textbf{TS 7.4.5.2 } \newline
\textbf{Samhita Paata} \newline

-न्नह॒रेको ऽभ॑ज॒ता-ह॒रेक॒स्ताभि॒र्वैते प्र॒बाहु॑गार्द्ध्नुव॒न॒. य ए॒वं ॅवि॒द्वाꣳस॑स्त्रयस्त्रिꣳशद॒हमास॑ते॒ सर्व॑ ए॒व प्र॒बाहु॑गृद्ध्नुवन्ति॒ सर्वे॒ ग्राम॑णीयं॒ प्राप्नु॑वन्ति पञ्चा॒हा भ॑वन्ति॒ पञ्च॒ वा ऋ॒तवः॑ संॅवथ्स॒र ऋ॒तुष्वे॒व सं॑ॅवथ्स॒रे प्रति॑ तिष्ठ॒न्त्यथो॒ पञ्चा᳚क्षरा प॒ङ्क्तिः पाङ्क्तो॑ य॒ज्ञो य॒ज्ञ्मे॒वाव॑ रुन्धते॒ त्रीण्या᳚श्वि॒नानि॑ भवन्ति॒ त्रय॑ इ॒मे लो॒का ए॒- [  ] \newline

\textbf{Pada Paata} \newline

अहः॑ । एकः॑ । अभ॑जत । अहः॑ । एकः॑ । ताभिः॑ । वै । ते । प्र॒बाहु॒गिति॑ प्र - बाहु॑क् । आ॒द्‌र्ध्नु॒व॒न्न् । ये । ए॒वम् । वि॒द्वाꣳसः॑ । त्र॒य॒स्त्रिꣳ॒॒श॒द॒हमिति॑ त्रयस्त्रिꣳशत् - अ॒हम् । आस॑ते । सर्वे᳚ । ए॒व । प्र॒बाहु॒गिति॑ प्र - बाहु॑क् । ऋ॒द्ध्नु॒व॒न्ति॒ । सर्वे᳚ । ग्राम॑णीय॒मिति॒ ग्राम॑ - नी॒य॒म् । प्रेति॑ । आ॒प्नु॒व॒न्ति॒ । प॒ञ्चा॒हा इति॑ पञ्च-अ॒हाः । भ॒व॒न्ति॒ । पञ्च॑ । वै । ऋ॒तवः॑ । सं॒ॅव॒थ्स॒र इति॑ सं - व॒थ्स॒रः । ऋ॒तुषु॑ । ए॒व । सं॒ॅव॒थ्स॒र इति॑ सं-व॒थ्स॒रे । प्रतीति॑ । ति॒ष्ठ॒न्ति॒ । अथो॒ इति॑ । पञ्चा᳚क्ष॒रेति॒ पञ्च॑-अ॒क्ष॒रा॒ । प॒ङ्क्तिः । पाङ्क्तः॑ । य॒ज्ञ्ः । य॒ज्ञ्म् । ए॒व । अवेति॑ । रु॒न्ध॒ते॒ । त्रीणि॑ । आ॒श्वि॒नानि॑ । भ॒व॒न्ति॒ । त्रयः॑ । इ॒मे । लो॒काः । ए॒षु ।  \newline


\textbf{Krama Paata} \newline

अह॒रेकः॑ । एकोऽभ॑जत । अभ॑ज॒ताहः॑ । अह॒रेकः॑ । एक॒स्ताभिः॑ । ताभि॒र् वै । वै ते । ते प्र॒बाहु॑क् । प्र॒बाहु॑गार्द्ध्नुवन्न् । प्र॒बाहु॒गिति॑ प्र - बाहु॑क् । आ॒र्द्ध्नु॒व॒न्॒. ये । य ए॒वम् । ए॒वम् ॅवि॒द्वाꣳसः॑ । वि॒द्वाꣳस॑स्त्रयस्त्रिꣳशद॒हम् । त्र॒य॒स्त्रिꣳ॒॒श॒द॒हमास॑ते । त्र॒य॒स्त्रिꣳ॒॒श॒द॒हमिति॑ त्रयस्त्रिꣳशत् - अ॒हम् । आस॑ते॒ सर्वे᳚ । सर्व॑ ए॒व । ए॒व प्र॒बाहु॑क् । प्र॒बाहु॑गृद्ध्नुवन्ति । प्र॒बाहु॒गिति॑ प्र - बाहु॑क् । ऋ॒द्ध्नु॒व॒न्ति॒ सर्वे᳚ । सर्वे॒ ग्राम॑णीयम् । ग्राम॑णीय॒म् प्र । ग्राम॑णीय॒मिति॒ ग्राम॑ - नी॒य॒म् । प्राप्नु॑वन्ति । आ॒प्नु॒व॒न्ति॒ प॒ञ्चा॒हाः । प॒ञ्चा॒हा भ॑वन्ति । प॒ञ्चा॒हा इति॑ पञ्च - अ॒हाः । भ॒व॒न्ति॒ पञ्च॑ । पञ्च॒ वै । वा ऋ॒तवः॑ । ऋ॒तवः॑ सम्ॅवथ्स॒रः । स॒म्ॅव॒थ्स॒र ऋ॒तुषु॑ । स॒म्ॅव॒थ्स॒र इति॑ सम् - व॒थ्स॒रः । ऋ॒तुष्वे॒व । ए॒व स॑म्ॅवथ्स॒रे । स॒म्ॅव॒थ्स॒रे प्रति॑ । स॒म्ॅव॒थ्स॒र इति॑ सम् - व॒थ्स॒रे । प्रति॑ तिष्ठन्ति । ति॒ष्ठ॒न्त्यथो᳚ । अथो॒ पञ्चा᳚क्षरा । अथो॒ इत्यथो᳚ । पञ्चा᳚क्षरा प॒ङ्‍क्तिः । पञ्चा᳚क्ष॒रेति॒ पञ्च॑ - अ॒क्ष॒रा॒ । प॒ङ्‍क्तिः पाङ्‍क्तः॑ । पाङ्‍क्तो॑ य॒ज्ञ्ः । य॒ज्ञो य॒ज्ञ्म् । य॒ज्ञ्मे॒व । ए॒वाव॑ । अव॑ रुन्धते । रु॒न्ध॒ते॒ त्रीणि॑ । त्रीण्या᳚श्वि॒नानि॑ । आ॒श्वि॒नानि॑ भवन्ति । भ॒व॒न्ति॒ त्रयः॑ । त्रय॑ इ॒मे । इ॒मे लो॒काः । लो॒का ए॒षु । ए॒ष्वे॑व \newline

\textbf{Jatai Paata} \newline

1. अह॒ रेक॒ एको ऽह॒ रह॒ रेकः॑ । \newline
2. एको ऽभ॑ज॒ता भ॑ज॒ तैक॒ एको ऽभ॑जत । \newline
3. अभ॑ज॒ ताह॒ रह॒ रभ॑ज॒ता भ॑ज॒ताहः॑ । \newline
4. अह॒ रेक॒ एको ऽह॒ रह॒ रेकः॑ । \newline
5. एक॒ स्ताभि॒ स्ताभि॒ रेक॒ एक॒ स्ताभिः॑ । \newline
6. ताभि॒र् वै वै ताभि॒ स्ताभि॒र् वै । \newline
7. वै ते ते वै वै ते । \newline
8. ते प्र॒बाहु॑क् प्र॒बाहु॒क् ते ते प्र॒बाहु॑क् । \newline
9. प्र॒बाहु॑ गार्द्ध्नुवन् नार्द्ध्नुवन् प्र॒बाहु॑क् प्र॒बाहु॑ गार्द्ध्नुवन्न् । \newline
10. प्र॒बाहु॒गिति॑ प्र - बाहु॑क् । \newline
11. आ॒र्द्ध्नु॒व॒न्॒. ये य आ᳚र्द्ध्नुवन् नार्द्ध्नुव॒न्॒. ये । \newline
12. य ए॒व मे॒वं ॅये य ए॒वम् । \newline
13. ए॒वं ॅवि॒द्वाꣳसो॑ वि॒द्वाꣳस॑ ए॒व मे॒वं ॅवि॒द्वाꣳसः॑ । \newline
14. वि॒द्वाꣳस॑स्त्रयस्त्रिꣳशद॒हम् त्र॑यस्त्रिꣳशद॒हं ॅवि॒द्वाꣳसो॑ वि॒द्वाꣳस॑स्त्रयस्त्रिꣳशद॒हम् । \newline
15. त्र॒य॒स्त्रिꣳ॒॒श॒द॒ह मास॑त॒ आस॑ते त्रयस्त्रिꣳशद॒हम् त्र॑यस्त्रिꣳशद॒ह मास॑ते । \newline
16. त्र॒य॒स्त्रिꣳ॒॒श॒द॒हमिति॑ त्रयस्त्रिꣳशत् - अ॒हम् । \newline
17. आस॑ते॒ सर्वे॒ सर्व॒ आस॑त॒ आस॑ते॒ सर्वे᳚ । \newline
18. सर्व॑ ए॒वैव सर्वे॒ सर्व॑ ए॒व । \newline
19. ए॒व प्र॒बाहु॑क् प्र॒बाहु॑ गे॒वैव प्र॒बाहु॑क् । \newline
20. प्र॒बाहु॑ गृद्ध्नुव न्त्यृद्ध्नुवन्ति प्र॒बाहु॑क् प्र॒बाहु॑ गृद्ध्नुवन्ति । \newline
21. प्र॒बाहु॒गिति॑ प्र - बाहु॑क् । \newline
22. ऋ॒द्ध्नु॒व॒न्ति॒ सर्वे॒ सर्व॑ ऋद्ध्नुव न्त्यृद्ध्नुवन्ति॒ सर्वे᳚ । \newline
23. सर्वे॒ ग्राम॑णीय॒म् ग्राम॑णीयꣳ॒॒ सर्वे॒ सर्वे॒ ग्राम॑णीयम् । \newline
24. ग्राम॑णीय॒म् प्र प्र ग्राम॑णीय॒म् ग्राम॑णीय॒म् प्र । \newline
25. ग्राम॑णीय॒मिति॒ ग्राम॑ - नी॒य॒म् । \newline
26. प्रा प्नु॑व न्त्याप्नुवन्ति॒ प्र प्रा प्नु॑वन्ति । \newline
27. आ॒प्नु॒व॒न्ति॒ प॒ञ्चा॒हाः प॑ञ्चा॒हा आ᳚प्नुव न्त्याप्नुवन्ति पञ्चा॒हाः । \newline
28. प॒ञ्चा॒हा भ॑वन्ति भवन्ति पञ्चा॒हाः प॑ञ्चा॒हा भ॑वन्ति । \newline
29. प॒ञ्चा॒हा इति॑ पञ्च - अ॒हाः । \newline
30. भ॒व॒न्ति॒ पञ्च॒ पञ्च॑ भवन्ति भवन्ति॒ पञ्च॑ । \newline
31. पञ्च॒ वै वै पञ्च॒ पञ्च॒ वै । \newline
32. वा ऋ॒तव॑ ऋ॒तवो॒ वै वा ऋ॒तवः॑ । \newline
33. ऋ॒तवः॑ संॅवथ्स॒रः सं॑ॅवथ्स॒र ऋ॒तव॑ ऋ॒तवः॑ संॅवथ्स॒रः । \newline
34. सं॒ॅव॒थ्स॒र ऋ॒तुष् वृ॒तुषु॑ संॅवथ्स॒रः सं॑ॅवथ्स॒र ऋ॒तुषु॑ । \newline
35. सं॒ॅव॒थ्स॒र इति॑ सं - व॒थ्स॒रः । \newline
36. ऋ॒तु ष्वे॒वैव र्‌तुष् वृ॒तु ष्वे॒व । \newline
37. ए॒व सं॑ॅवथ्स॒रे सं॑ॅवथ्स॒र ए॒वैव सं॑ॅवथ्स॒रे । \newline
38. सं॒ॅव॒थ्स॒रे प्रति॒ प्रति॑ संॅवथ्स॒रे सं॑ॅवथ्स॒रे प्रति॑ । \newline
39. सं॒ॅव॒थ्स॒र इति॑ सं - व॒थ्स॒रे । \newline
40. प्रति॑ तिष्ठन्ति तिष्ठन्ति॒ प्रति॒ प्रति॑ तिष्ठन्ति । \newline
41. ति॒ष्ठ॒ न्त्यथो॒ अथो॑ तिष्ठन्ति तिष्ठ॒ न्त्यथो᳚ । \newline
42. अथो॒ पञ्चा᳚क्षरा॒ पञ्चा᳚क्ष॒रा ऽथो॒ अथो॒ पञ्चा᳚क्षरा । \newline
43. अथो॒ इत्यथो᳚ । \newline
44. पञ्चा᳚क्षरा प॒ङ्क्तिः प॒ङ्क्तिः पञ्चा᳚क्षरा॒ पञ्चा᳚क्षरा प॒ङ्क्तिः । \newline
45. पञ्चा᳚क्ष॒रेति॒ पञ्च॑ - अ॒क्ष॒रा॒ । \newline
46. प॒ङ्क्तिः पाङ्क्तः॒ पाङ्क्तः॑ प॒ङ्क्तिः प॒ङ्क्तिः पाङ्क्तः॑ । \newline
47. पाङ्क्तो॑ य॒ज्ञो य॒ज्ञ्ः पाङ्क्तः॒ पाङ्क्तो॑ य॒ज्ञ्ः । \newline
48. य॒ज्ञो य॒ज्ञ्ं ॅय॒ज्ञ्ं ॅय॒ज्ञो य॒ज्ञो य॒ज्ञ्म् । \newline
49. य॒ज्ञ् मे॒वैव य॒ज्ञ्ं ॅय॒ज्ञ् मे॒व । \newline
50. ए॒वावा वै॒वै वाव॑ । \newline
51. अव॑ रुन्धते रुन्ध॒ते ऽवाव॑ रुन्धते । \newline
52. रु॒न्ध॒ते॒ त्रीणि॒ त्रीणि॑ रुन्धते रुन्धते॒ त्रीणि॑ । \newline
53. त्रीण्या᳚ श्वि॒ना न्या᳚श्वि॒नानि॒ त्रीणि॒ त्रीण्या᳚ श्वि॒नानि॑ । \newline
54. आ॒श्वि॒नानि॑ भवन्ति भव न्त्याश्वि॒ना न्या᳚श्वि॒नानि॑ भवन्ति । \newline
55. भ॒व॒न्ति॒ त्रय॒ स्त्रयो॑ भवन्ति भवन्ति॒ त्रयः॑ । \newline
56. त्रय॑ इ॒म इ॒मे त्रय॒ स्त्रय॑ इ॒मे । \newline
57. इ॒मे लो॒का लो॒का इ॒म इ॒मे लो॒काः । \newline
58. लो॒का ए॒ष्वे॑षु लो॒का लो॒का ए॒षु । \newline
59. ए॒ष्वे॑ वैवै ष्वे᳚(1॒)ष्वे॑व । \newline

\textbf{Ghana Paata } \newline

1. अह॒ रेक॒ एको ऽह॒ रह॒ रेको ऽभ॑ज॒ता भ॑ज॒ तैको ऽह॒ रह॒ रेको ऽभ॑जत । \newline
2. एको ऽभ॑ज॒ता भ॑ज॒तैक॒ एको ऽभ॑ज॒ ताह॒ रह॒ रभ॑ज॒तैक॒ एको ऽभ॑ज॒ ताहः॑ । \newline
3. अभ॑ज॒ ताह॒ रह॒ रभ॑ज॒ता भ॑ज॒ ताह॒ रेक॒ एको ऽह॒ रभ॑ज॒ता भ॑ज॒ ताह॒ रेकः॑ । \newline
4. अह॒ रेक॒ एको ऽह॒ रह॒ रेक॒ स्ताभि॒ स्ताभि॒ रेको ऽह॒ रह॒ रेक॒ स्ताभिः॑ । \newline
5. एक॒ स्ताभि॒ स्ताभि॒ रेक॒ एक॒ स्ताभि॒र् वै वै ताभि॒ रेक॒ एक॒ स्ताभि॒र् वै । \newline
6. ताभि॒र् वै वै ताभि॒ स्ताभि॒र् वै ते ते वै ताभि॒ स्ताभि॒र् वै ते । \newline
7. वै ते ते वै वै ते प्र॒बाहु॑क् प्र॒बाहु॒क् ते वै वै ते प्र॒बाहु॑क् । \newline
8. ते प्र॒बाहु॑क् प्र॒बाहु॒क् ते ते प्र॒बाहु॑ गार्द्ध्नुवन् नार्द्ध्नुवन् प्र॒बाहु॒क् ते ते प्र॒बाहु॑ गार्द्ध्नुवन्न् । \newline
9. प्र॒बाहु॑ गार्द्ध्नुवन् नार्द्ध्नुवन् प्र॒बाहु॑क् प्र॒बाहु॑ गार्द्ध्नुव॒न्॒. ये य आ᳚र्द्ध्नुवन् प्र॒बाहु॑क् प्र॒बाहु॑ गार्द्ध्नुव॒न्॒. ये । \newline
10. प्र॒बाहु॒गिति॑ प्र - बाहु॑क् । \newline
11. आ॒र्द्ध्नु॒व॒न्॒. ये य आ᳚र्द्ध्नुवन् नार्द्ध्नुव॒न्॒. य ए॒व मे॒वं ॅय आ᳚र्द्ध्नुवन् नार्द्ध्नुव॒न्॒. य ए॒वम् । \newline
12. य ए॒व मे॒वं ॅये य ए॒वं ॅवि॒द्वाꣳसो॑ वि॒द्वाꣳस॑ ए॒वं ॅये य ए॒वं ॅवि॒द्वाꣳसः॑ । \newline
13. ए॒वं ॅवि॒द्वाꣳसो॑ वि॒द्वाꣳस॑ ए॒व मे॒वं ॅवि॒द्वाꣳस॑ स्त्रयस्त्रिꣳशद॒हम् त्र॑यस्त्रिꣳशद॒हं ॅवि॒द्वाꣳस॑ ए॒व मे॒वं ॅवि॒द्वाꣳस॑ स्त्रयस्त्रिꣳशद॒हम् । \newline
14. वि॒द्वाꣳस॑ स्त्रयस्त्रिꣳशद॒हम् त्र॑यस्त्रिꣳशद॒हं ॅवि॒द्वाꣳसो॑ वि॒द्वाꣳस॑ स्त्रयस्त्रिꣳशद॒ह मास॑त॒ आस॑ते त्रयस्त्रिꣳशद॒हं ॅवि॒द्वाꣳसो॑ वि॒द्वाꣳस॑ स्त्रयस्त्रिꣳशद॒ह मास॑ते । \newline
15. त्र॒य॒स्त्रिꣳ॒॒श॒द॒ह मास॑त॒ आस॑ते त्रयस्त्रिꣳशद॒हम् त्र॑यस्त्रिꣳशद॒ह मास॑ते॒ सर्वे॒ सर्व॒ आस॑ते त्रयस्त्रिꣳशद॒हम् त्र॑यस्त्रिꣳशद॒ह मास॑ते॒ सर्वे᳚ । \newline
16. त्र॒य॒स्त्रिꣳ॒॒श॒द॒हमिति॑ त्रयस्त्रिꣳशत् - अ॒हम् । \newline
17. आस॑ते॒ सर्वे॒ सर्व॒ आस॑त॒ आस॑ते॒ सर्व॑ ए॒वैव सर्व॒ आस॑त॒ आस॑ते॒ सर्व॑ ए॒व । \newline
18. सर्व॑ ए॒वैव सर्वे॒ सर्व॑ ए॒व प्र॒बाहु॑क् प्र॒बाहु॑ गे॒व सर्वे॒ सर्व॑ ए॒व प्र॒बाहु॑क् । \newline
19. ए॒व प्र॒बाहु॑क् प्र॒बाहु॑ गे॒वैव प्र॒बाहु॑ गृद्ध्नुव न्त्यृद्ध्नुवन्ति प्र॒बाहु॑ गे॒वैव प्र॒बाहु॑ गृद्ध्नुवन्ति । \newline
20. प्र॒बाहु॑ गृद्ध्नुव न्त्यृद्ध्नुवन्ति प्र॒बाहु॑क् प्र॒बाहु॑ गृद्ध्नुवन्ति॒ सर्वे॒ सर्व॑ ऋद्ध्नुवन्ति प्र॒बाहु॑क् प्र॒बाहु॑ गृद्ध्नुवन्ति॒ सर्वे᳚ । \newline
21. प्र॒बाहु॒गिति॑ प्र - बाहु॑क् । \newline
22. ऋ॒द्ध्नु॒व॒न्ति॒ सर्वे॒ सर्व॑ ऋद्ध्नुव न्त्यृद्ध्नुवन्ति॒ सर्वे॒ ग्राम॑णीय॒म् ग्राम॑णीयꣳ॒॒ सर्व॑ ऋद्ध्नुव न्त्यृद्ध्नुवन्ति॒ सर्वे॒ ग्राम॑णीयम् । \newline
23. सर्वे॒ ग्राम॑णीय॒म् ग्राम॑णीयꣳ॒॒ सर्वे॒ सर्वे॒ ग्राम॑णीय॒म् प्र प्र ग्राम॑णीयꣳ॒॒ सर्वे॒ सर्वे॒ ग्राम॑णीय॒म् प्र । \newline
24. ग्राम॑णीय॒म् प्र प्र ग्राम॑णीय॒म् ग्राम॑णीय॒म् प्राप्नु॑व न्त्याप्नुवन्ति॒ प्र ग्राम॑णीय॒म् ग्राम॑णीय॒म् 
प्राप्नु॑वन्ति । \newline
25. ग्राम॑णीय॒मिति॒ ग्राम॑ - नी॒य॒म् । \newline
26. प्राप्नु॑व न्त्याप्नुवन्ति॒ प्र प्राप्नु॑वन्ति पञ्चा॒हाः प॑ञ्चा॒हा आ᳚प्नुवन्ति॒ प्र प्राप्नु॑वन्ति पञ्चा॒हाः । \newline
27. आ॒प्नु॒व॒न्ति॒ प॒ञ्चा॒हाः प॑ञ्चा॒हा आ᳚प्नुव न्त्याप्नुवन्ति पञ्चा॒हा भ॑वन्ति भवन्ति पञ्चा॒हा आ᳚प्नुव न्त्याप्नुवन्ति पञ्चा॒हा भ॑वन्ति । \newline
28. प॒ञ्चा॒हा भ॑वन्ति भवन्ति पञ्चा॒हाः प॑ञ्चा॒हा भ॑वन्ति॒ पञ्च॒ पञ्च॑ भवन्ति पञ्चा॒हाः प॑ञ्चा॒हा भ॑वन्ति॒ पञ्च॑ । \newline
29. प॒ञ्चा॒हा इति॑ पञ्च - अ॒हाः । \newline
30. भ॒व॒न्ति॒ पञ्च॒ पञ्च॑ भवन्ति भवन्ति॒ पञ्च॒ वै वै पञ्च॑ भवन्ति भवन्ति॒ पञ्च॒ वै । \newline
31. पञ्च॒ वै वै पञ्च॒ पञ्च॒ वा ऋ॒तव॑ ऋ॒तवो॒ वै पञ्च॒ पञ्च॒ वा ऋ॒तवः॑ । \newline
32. वा ऋ॒तव॑ ऋ॒तवो॒ वै वा ऋ॒तवः॑ संॅवथ्स॒रः सं॑ॅवथ्स॒र ऋ॒तवो॒ वै वा ऋ॒तवः॑ संॅवथ्स॒रः । \newline
33. ऋ॒तवः॑ संॅवथ्स॒रः सं॑ॅवथ्स॒र ऋ॒तव॑ ऋ॒तवः॑ संॅवथ्स॒र ऋ॒तुष् वृ॒तुषु॑ संॅवथ्स॒र ऋ॒तव॑ ऋ॒तवः॑ संॅवथ्स॒र ऋ॒तुषु॑ । \newline
34. सं॒ॅव॒थ्स॒र ऋ॒तुष् वृ॒तुषु॑ संॅवथ्स॒रः सं॑ॅवथ्स॒र ऋ॒तु ष्वे॒वैव र्‌तुषु॑ संॅवथ्स॒रः सं॑ॅवथ्स॒र ऋ॒तुष्वे॒व । \newline
35. सं॒ॅव॒थ्स॒र इति॑ सं - व॒थ्स॒रः । \newline
36. ऋ॒तु ष्वे॒वैव र्‌तुष् वृ॒तु ष्वे॒व सं॑ॅवथ्स॒रे सं॑ॅवथ्स॒र ए॒व र्‌तुष्व् ऋ॒तु ष्वे॒व सं॑ॅवथ्स॒रे । \newline
37. ए॒व सं॑ॅवथ्स॒रे सं॑ॅवथ्स॒र ए॒वैव सं॑ॅवथ्स॒रे प्रति॒ प्रति॑ संॅवथ्स॒र ए॒वैव सं॑ॅवथ्स॒रे प्रति॑ । \newline
38. सं॒ॅव॒थ्स॒रे प्रति॒ प्रति॑ संॅवथ्स॒रे सं॑ॅवथ्स॒रे प्रति॑ तिष्ठन्ति तिष्ठन्ति॒ प्रति॑ संॅवथ्स॒रे सं॑ॅवथ्स॒रे प्रति॑ तिष्ठन्ति । \newline
39. सं॒ॅव॒थ्स॒र इति॑ सं - व॒थ्स॒रे । \newline
40. प्रति॑ तिष्ठन्ति तिष्ठन्ति॒ प्रति॒ प्रति॑ तिष्ठ॒ न्त्यथो॒ अथो॑ तिष्ठन्ति॒ प्रति॒ प्रति॑ तिष्ठ॒ न्त्यथो᳚ । \newline
41. ति॒ष्ठ॒ न्त्यथो॒ अथो॑ तिष्ठन्ति तिष्ठ॒ न्त्यथो॒ पञ्चा᳚क्षरा॒ पञ्चा᳚क्ष॒रा ऽथो॑ तिष्ठन्ति तिष्ठ॒ न्त्यथो॒ पञ्चा᳚क्षरा । \newline
42. अथो॒ पञ्चा᳚क्षरा॒ पञ्चा᳚क्ष॒रा ऽथो॒ अथो॒ पञ्चा᳚क्षरा प॒ङ्क्तिः प॒ङ्क्तिः पञ्चा᳚क्ष॒रा ऽथो॒ अथो॒ पञ्चा᳚क्षरा प॒ङ्क्तिः । \newline
43. अथो॒ इत्यथो᳚ । \newline
44. पञ्चा᳚क्षरा प॒ङ्क्तिः प॒ङ्क्तिः पञ्चा᳚क्षरा॒ पञ्चा᳚क्षरा प॒ङ्क्तिः पाङ्क्तः॒ पाङ्क्तः॑ प॒ङ्क्तिः पञ्चा᳚क्षरा॒ पञ्चा᳚क्षरा प॒ङ्क्तिः पाङ्क्तः॑ । \newline
45. पञ्चा᳚क्ष॒रेति॒ पञ्च॑ - अ॒क्ष॒रा॒ । \newline
46. प॒ङ्क्तिः पाङ्क्तः॒ पाङ्क्तः॑ प॒ङ्क्तिः प॒ङ्क्तिः पाङ्क्तो॑ य॒ज्ञो य॒ज्ञ्ः पाङ्क्तः॑ प॒ङ्क्तिः प॒ङ्क्तिः पाङ्क्तो॑ य॒ज्ञ्ः । \newline
47. पाङ्क्तो॑ य॒ज्ञो य॒ज्ञ्ः पाङ्क्तः॒ पाङ्क्तो॑ य॒ज्ञो य॒ज्ञ्ं ॅय॒ज्ञ्ं ॅय॒ज्ञ्ः पाङ्क्तः॒ पाङ्क्तो॑ य॒ज्ञो य॒ज्ञ्म् । \newline
48. य॒ज्ञो य॒ज्ञ्ं ॅय॒ज्ञ्ं ॅय॒ज्ञो य॒ज्ञो य॒ज्ञ् मे॒वैव य॒ज्ञ्ं ॅय॒ज्ञो य॒ज्ञो य॒ज्ञ् मे॒व । \newline
49. य॒ज्ञ् मे॒वैव य॒ज्ञ्ं ॅय॒ज्ञ् मे॒वावा वै॒व य॒ज्ञ्ं ॅय॒ज्ञ् मे॒वाव॑ । \newline
50. ए॒वावा वै॒वै वाव॑ रुन्धते रुन्ध॒ते ऽवै॒वै वाव॑ रुन्धते । \newline
51. अव॑ रुन्धते रुन्ध॒ते ऽवाव॑ रुन्धते॒ त्रीणि॒ त्रीणि॑ रुन्ध॒ते ऽवाव॑ रुन्धते॒ त्रीणि॑ । \newline
52. रु॒न्ध॒ते॒ त्रीणि॒ त्रीणि॑ रुन्धते रुन्धते॒ त्रीण्या᳚ श्वि॒ना न्या᳚श्वि॒नानि॒ त्रीणि॑ रुन्धते रुन्धते॒ त्रीण्या᳚श्वि॒नानि॑ । \newline
53. त्रीण्या᳚ श्वि॒ना न्या᳚श्वि॒नानि॒ त्रीणि॒ त्रीण्या᳚ श्वि॒नानि॑ भवन्ति भव न्त्याश्वि॒नानि॒ त्रीणि॒ त्रीण्या᳚ श्वि॒नानि॑ भवन्ति । \newline
54. आ॒श्वि॒नानि॑ भवन्ति भव न्त्याश्वि॒ना न्या᳚श्वि॒नानि॑ भवन्ति॒ त्रय॒ स्त्रयो॑ भव न्त्याश्वि॒ना न्या᳚श्वि॒नानि॑ भवन्ति॒ त्रयः॑ । \newline
55. भ॒व॒न्ति॒ त्रय॒ स्त्रयो॑ भवन्ति भवन्ति॒ त्रय॑ इ॒म इ॒मे त्रयो॑ भवन्ति भवन्ति॒ त्रय॑ इ॒मे । \newline
56. त्रय॑ इ॒म इ॒मे त्रय॒ स्त्रय॑ इ॒मे लो॒का लो॒का इ॒मे त्रय॒ स्त्रय॑ इ॒मे लो॒काः । \newline
57. इ॒मे लो॒का लो॒का इ॒म इ॒मे लो॒का ए॒ष्वे॑षु लो॒का इ॒म इ॒मे लो॒का ए॒षु । \newline
58. लो॒का ए॒ष्वे॑षु लो॒का लो॒का ए॒ष्वे॑वैवैषु लो॒का लो॒का ए॒ष्वे॑व । \newline
59. ए॒ष्वे॑ वैवै ष्वे᳚(1॒)ष्वे॑व लो॒केषु॑ लो॒के ष्वे॒वैष्वे᳚(1॒)ष्वे॑व लो॒केषु॑ । \newline
\pagebreak
\markright{ TS 7.4.5.3  \hfill https://www.vedavms.in \hfill}

\section{ TS 7.4.5.3 }

\textbf{TS 7.4.5.3 } \newline
\textbf{Samhita Paata} \newline

-ष्वे॑व लो॒केषु॒ प्रति॑ तिष्ठ॒न्त्यथो॒ त्रीणि॒ वै य॒ज्ञ्स्ये᳚न्द्रि॒याणि॒ तान्ये॒वाव॑ रुन्धते विश्व॒जिद्-भ॑वत्य॒न्नाद्य॒स्या व॑रुद्ध्यै॒ सर्व॑पृष्ठो भवति॒ सर्व॑स्या॒भिजि॑त्यै॒ वाग्वै द्वा॑दशा॒हो यत् पु॒रस्ता᳚द् द्वादशा॒ह-मु॑पे॒युरना᳚प्तां॒ ॅवाच॒-मुपे॑यु-रुप॒दासु॑कैषां॒ ॅवाख् स्या॑-दु॒परि॑ष्टाद् द्वादशा॒हमुप॑ यन्त्या॒प्तामे॒व वाच॒मुप॑ यन्ति॒ तस्मा॑-दु॒परि॑ष्टाद्-वा॒चा व॑दामो ऽवान्त॒रं - [  ] \newline

\textbf{Pada Paata} \newline

ए॒व । लो॒केषु॑ । प्रतीति॑ । ति॒ष्ठ॒न्ति॒ । अथो॒ इति॑ । त्रीणि॑ । वै । य॒ज्ञ्स्य॑ । इ॒न्द्रि॒याणि॑ । तानि॑ । ए॒व । अवेति॑ । रु॒न्ध॒ते॒ । वि॒श्व॒जिदिति॑ विश्व - जित् । भ॒व॒ति॒ । अ॒न्नाद्य॒स्येत्य॑न्न - अद्य॑स्य । अव॑रुद्ध्या॒ इत्यव॑ - रु॒द्ध्यै॒ । सर्व॑पृष्ठ॒ इति॒ सर्व॑ - पृ॒ष्ठः॒ । भ॒व॒ति॒ । सर्व॑स्य । अ॒भिजि॑त्या॒ इत्य॒भि - जि॒त्यै॒ । वाक् । वै । द्वा॒द॒शा॒ह इति॑ द्वादश - अ॒हः । यत् । पु॒रस्ता᳚त् । द्वा॒द॒शा॒हमिति द्वादश - अ॒हम् । उ॒पे॒युरित्यु॑प-इ॒युः । अना᳚प्ताम् । वाच᳚म् । उपेति॑ । इ॒युः॒ । उ॒प॒दासु॒केत्यु॑प - दासु॑का । ए॒षा॒म् । वाक् । स्या॒त् । उ॒परि॑ष्टात् । द्वा॒द॒शा॒हमिति॑ द्वादश - अ॒हम् । उपेति॑ । य॒न्ति॒ । आ॒प्ताम् । ए॒व । वाच᳚म् । उपेति॑ । य॒न्ति॒ । तस्मा᳚त् । उ॒परि॑ष्टात् । वा॒चा । व॒दा॒मः॒ । अ॒वा॒न्त॒रमित्य॑व - अ॒न्त॒रम् ।  \newline


\textbf{Krama Paata} \newline

ए॒व लो॒केषु॑ । लो॒केषु॒ प्रति॑ । प्रति॑ तिष्ठन्ति । ति॒ष्ठ॒न्त्यथो᳚ । अथो॒ त्रीणि॑ । अथो॒ इत्यथो᳚ । त्रीणि॒ वै । वै य॒ज्ञ्स्य॑ । य॒ज्ञ्स्ये᳚न्द्रि॒याणि॑ । इ॒न्द्रि॒याणि॒ तानि॑ । तान्ये॒व । ए॒वाव॑ । अव॑ रुन्धते । रु॒न्ध॒ते॒ वि॒श्व॒जित् । वि॒श्व॒जिद् भ॑वति । वि॒श्व॒जिदिति॑ विश्व - जित् । भ॒व॒त्य॒न्नाद्य॑स्य । अ॒न्नाद्य॒स्याव॑रुद्ध्यै । अ॒न्नाद्य॒स्येत्य॑न्न - अद्य॑स्य । अव॑रुद्ध्यै॒ सर्व॑पृष्ठः । अव॑रुद्ध्या॒ इत्यव॑ - रु॒द्ध्यै॒ । सर्व॑पृष्ठो भवति । सर्व॑पृष्ठ॒ इति॒ सर्व॑ - पृ॒ष्ठः॒ । भ॒व॒ति॒ सर्व॑स्य । सर्व॑स्या॒भिजि॑त्यै । अ॒भिजि॑त्यै॒ वाक् । अ॒भिजि॑त्या॒ इत्य॒भि - जि॒त्यै॒ । वाग् वै । वै द्वा॑दशा॒हः । द्वा॒द॒शा॒हो यत् । द्वा॒द॒शा॒ह इति॑ द्वादश - अ॒हः । यत् पु॒रस्ता᳚त् । पु॒रस्ता᳚द् द्वादशा॒हम् । द्वा॒द॒शा॒हमु॑पे॒युः । द्वा॒द॒शा॒हमिति॑ द्वादश - अ॒हम् । उ॒पे॒युरना᳚प्ताम् । उ॒पे॒युरित्यु॑प - इ॒युः । अना᳚प्ता॒म् ॅवाच᳚म् । वाच॒मुप॑ । उपे॑युः । इ॒यु॒रु॒प॒दासु॑का । उ॒प॒दासु॑कैषाम् । उ॒प॒दासु॒केत्यु॑प - दासु॑का । ए॒षा॒म् ॅवाक् । वाख् स्या᳚त् । स्या॒दु॒परि॑ष्टात् । उ॒परि॑ष्टाद् द्वादशा॒हम् । द्वा॒द॒शा॒हमुप॑ । द्वा॒द॒शा॒हमिति॑ द्वादश - अ॒हम् । उप॑ यन्ति । य॒न्त्या॒प्ताम् । आ॒प्तामे॒व । ए॒व वाच᳚म् । वाच॒मुप॑ । उप॑ यन्ति । य॒न्ति॒ तस्मा᳚त् । तस्मा॑दु॒परि॑ष्टात् । उ॒परि॑ष्टाद् वा॒चा । वा॒चा व॑दामः । व॒दा॒मो॒ऽवा॒न्त॒रम् । अ॒वा॒न्त॒रम् ॅवै । अ॒वा॒न्त॒रमित्य॑व - अ॒न्त॒रम् \newline

\textbf{Jatai Paata} \newline

1. ए॒व लो॒केषु॑ लो॒के ष्वे॒वैव लो॒केषु॑ । \newline
2. लो॒केषु॒ प्रति॒ प्रति॑ लो॒केषु॑ लो॒केषु॒ प्रति॑ । \newline
3. प्रति॑ तिष्ठन्ति तिष्ठन्ति॒ प्रति॒ प्रति॑ तिष्ठन्ति । \newline
4. ति॒ष्ठ॒ न्त्यथो॒ अथो॑ तिष्ठन्ति तिष्ठ॒ न्त्यथो᳚ । \newline
5. अथो॒ त्रीणि॒ त्रीण्यथो॒ अथो॒ त्रीणि॑ । \newline
6. अथो॒ इत्यथो᳚ । \newline
7. त्रीणि॒ वै वै त्रीणि॒ त्रीणि॒ वै । \newline
8. वै य॒ज्ञ्स्य॑ य॒ज्ञ्स्य॒ वै वै य॒ज्ञ्स्य॑ । \newline
9. य॒ज्ञ् स्ये᳚न्द्रि॒याणी᳚ न्द्रि॒याणि॑ य॒ज्ञ्स्य॑ य॒ज्ञ् स्ये᳚न्द्रि॒याणि॑ । \newline
10. इ॒न्द्रि॒याणि॒ तानि॒ तानी᳚न्द्रि॒याणी᳚ न्द्रि॒याणि॒ तानि॑ । \newline
11. तान्ये॒वैव तानि॒ तान्ये॒व । \newline
12. ए॒वावा वै॒वै वाव॑ । \newline
13. अव॑ रुन्धते रुन्ध॒ते ऽवाव॑ रुन्धते । \newline
14. रु॒न्ध॒ते॒ वि॒श्व॒जिद् वि॑श्व॒जिद् रु॑न्धते रुन्धते विश्व॒जित् । \newline
15. वि॒श्व॒जिद् भ॑वति भवति विश्व॒जिद् वि॑श्व॒जिद् भ॑वति । \newline
16. वि॒श्व॒जिदिति॑ विश्व - जित् । \newline
17. भ॒व॒ त्य॒न्नाद्य॑स्या॒ न्नाद्य॑स्य भवति भव त्य॒न्नाद्य॑स्य । \newline
18. अ॒न्नाद्य॒स्या व॑रुद्ध्या॒ अव॑रुद्ध्या अ॒न्नाद्य॑स्या॒ न्नाद्य॒स्या व॑रुद्ध्यै । \newline
19. अ॒न्नाद्य॒स्येत्य॑न्न - अद्य॑स्य । \newline
20. अव॑रुद्ध्यै॒ सर्व॑पृष्ठः॒ सर्व॑पृ॒ष्ठो ऽव॑रुद्ध्या॒ अव॑रुद्ध्यै॒ सर्व॑पृष्ठः । \newline
21. अव॑रुद्ध्या॒ इत्यव॑ - रु॒द्ध्यै॒ । \newline
22. सर्व॑पृष्ठो भवति भवति॒ सर्व॑पृष्ठः॒ सर्व॑पृष्ठो भवति । \newline
23. सर्व॑पृष्ठ॒ इति॒ सर्व॑ - पृ॒ष्ठः॒ । \newline
24. भ॒व॒ति॒ सर्व॑स्य॒ सर्व॑स्य भवति भवति॒ सर्व॑स्य । \newline
25. सर्व॑स्या॒ भिजि॑त्या अ॒भिजि॑त्यै॒ सर्व॑स्य॒ सर्व॑स्या॒ भिजि॑त्यै । \newline
26. अ॒भिजि॑त्यै॒ वाग् वाग॒ भिजि॑त्या अ॒भिजि॑त्यै॒ वाक् । \newline
27. अ॒भिजि॑त्या॒ इत्य॒भि - जि॒त्यै॒ । \newline
28. वाग् वै वै वाग् वाग् वै । \newline
29. वै द्वा॑दशा॒हो द्वा॑दशा॒हो वै वै द्वा॑दशा॒हः । \newline
30. द्वा॒द॒शा॒हो यद् यद् द्वा॑दशा॒हो द्वा॑दशा॒हो यत् । \newline
31. द्वा॒द॒शा॒ह इति॑ द्वादश - अ॒हः । \newline
32. यत् पु॒रस्ता᳚त् पु॒रस्ता॒द् यद् यत् पु॒रस्ता᳚त् । \newline
33. पु॒रस्ता᳚द् द्वादशा॒हम् द्वा॑दशा॒हम् पु॒रस्ता᳚त् पु॒रस्ता᳚द् द्वादशा॒हम् । \newline
34. द्वा॒द॒शा॒ह मु॑पे॒यु रु॑पे॒युर् द्वा॑दशा॒हम् द्वा॑दशा॒ह मु॑पे॒युः । \newline
35. द्वा॒द॒शा॒हमिति॑ द्वादश - अ॒हम् । \newline
36. उ॒पे॒यु रना᳚प्ता॒ मना᳚प्ता मुपे॒यु रु॑पे॒यु रना᳚प्ताम् । \newline
37. उ॒पे॒युरित्यु॑प - इ॒युः । \newline
38. अना᳚प्तां॒ ॅवाचं॒ ॅवाच॒ मना᳚प्ता॒ मना᳚प्तां॒ ॅवाच᳚म् । \newline
39. वाच॒ मुपोप॒ वाचं॒ ॅवाच॒ मुप॑ । \newline
40. उपे॑यु रियु॒ रुपोपे॑युः । \newline
41. इ॒यु॒ रु॒प॒दासु॑ कोप॒दासु॑केयु रियु रुप॒दासु॑का । \newline
42. उ॒प॒दासु॑ कैषा मेषा मुप॒दासु॑को प॒दासु॑ कैषाम् । \newline
43. उ॒प॒दासु॒केत्यु॑प - दासु॑का । \newline
44. ए॒षां॒ ॅवाग् वागे॑षा मेषां॒ ॅवाक् । \newline
45. वाख् स्या᳚थ् स्या॒द् वाग् वाख् स्या᳚त् । \newline
46. स्या॒ दु॒परि॑ष्टा दु॒परि॑ष्टाथ् स्याथ् स्या दु॒परि॑ष्टात् । \newline
47. उ॒परि॑ष्टाद् द्वादशा॒हम् द्वा॑दशा॒ह मु॒परि॑ष्टा दु॒परि॑ष्टाद् द्वादशा॒हम् । \newline
48. द्वा॒द॒शा॒ह मुपोप॑ द्वादशा॒हम् द्वा॑दशा॒ह मुप॑ । \newline
49. द्वा॒द॒शा॒हमिति॑ द्वादश - अ॒हम् । \newline
50. उप॑ यन्ति य॒न्त्युपोप॑ यन्ति । \newline
51. य॒न्त्या॒प्ता मा॒प्तां ॅय॑न्ति यन्त्या॒प्ताम् । \newline
52. आ॒प्ता मे॒वै वाप्ता मा॒प्ता मे॒व । \newline
53. ए॒व वाचं॒ ॅवाच॑ मे॒वैव वाच᳚म् । \newline
54. वाच॒ मुपोप॒ वाचं॒ ॅवाच॒ मुप॑ । \newline
55. उप॑ यन्ति य॒न्त्युपोप॑ यन्ति । \newline
56. य॒न्ति॒ तस्मा॒त् तस्मा᳚द् यन्ति यन्ति॒ तस्मा᳚त् । \newline
57. तस्मा॑ दु॒परि॑ष्टा दु॒परि॑ष्टा॒त् तस्मा॒त् तस्मा॑ दु॒परि॑ष्टात् । \newline
58. उ॒परि॑ष्टाद् वा॒चा वा॒चोपरि॑ष्टा दु॒परि॑ष्टाद् वा॒चा । \newline
59. वा॒चा व॑दामो वदामो वा॒चा वा॒चा व॑दामः । \newline
60. व॒दा॒मो॒ ऽवा॒न्त॒र म॑वान्त॒रं ॅव॑दामो वदामो ऽवान्त॒रम् । \newline
61. अ॒वा॒न्त॒रं ॅवै वा अ॑वान्त॒र म॑वान्त॒रं ॅवै । \newline
62. अ॒वा॒न्त॒रमित्य॑व - अ॒न्त॒रम् । \newline

\textbf{Ghana Paata } \newline

1. ए॒व लो॒केषु॑ लो॒के ष्वे॒वैव लो॒केषु॒ प्रति॒ प्रति॑ लो॒के ष्वे॒वैव लो॒केषु॒ प्रति॑ । \newline
2. लो॒केषु॒ प्रति॒ प्रति॑ लो॒केषु॑ लो॒केषु॒ प्रति॑ तिष्ठन्ति तिष्ठन्ति॒ प्रति॑ लो॒केषु॑ लो॒केषु॒ प्रति॑ तिष्ठन्ति । \newline
3. प्रति॑ तिष्ठन्ति तिष्ठन्ति॒ प्रति॒ प्रति॑ तिष्ठ॒ न्त्यथो॒ अथो॑ तिष्ठन्ति॒ प्रति॒ प्रति॑ तिष्ठ॒ न्त्यथो᳚ । \newline
4. ति॒ष्ठ॒ न्त्यथो॒ अथो॑ तिष्ठन्ति तिष्ठ॒ न्त्यथो॒ त्रीणि॒ त्रीण्यथो॑ तिष्ठन्ति तिष्ठ॒ न्त्यथो॒ त्रीणि॑ । \newline
5. अथो॒ त्रीणि॒ त्रीण्यथो॒ अथो॒ त्रीणि॒ वै वै त्रीण्यथो॒ अथो॒ त्रीणि॒ वै । \newline
6. अथो॒ इत्यथो᳚ । \newline
7. त्रीणि॒ वै वै त्रीणि॒ त्रीणि॒ वै य॒ज्ञ्स्य॑ य॒ज्ञ्स्य॒ वै त्रीणि॒ त्रीणि॒ वै य॒ज्ञ्स्य॑ । \newline
8. वै य॒ज्ञ्स्य॑ य॒ज्ञ्स्य॒ वै वै य॒ज्ञ् स्ये᳚न्द्रि॒याणी᳚ न्द्रि॒याणि॑ य॒ज्ञ्स्य॒ वै वै य॒ज्ञ् स्ये᳚न्द्रि॒याणि॑ । \newline
9. य॒ज्ञ्स्ये᳚न्द्रि॒याणी᳚ न्द्रि॒याणि॑ य॒ज्ञ्स्य॑ य॒ज्ञ् स्ये᳚न्द्रि॒याणि॒ तानि॒ तानी᳚ न्द्रि॒याणि॑ य॒ज्ञ्स्य॑ य॒ज्ञ्
स्ये᳚न्द्रि॒याणि॒ तानि॑ । \newline
10. इ॒न्द्रि॒याणि॒ तानि॒ तानी᳚न्द्रि॒याणी᳚ न्द्रि॒याणि॒ तान्ये॒वैव तानी᳚न्द्रि॒याणी᳚ न्द्रि॒याणि॒ तान्ये॒व । \newline
11. तान्ये॒वैव तानि॒ तान्ये॒वावा वै॒व तानि॒ तान्ये॒वाव॑ । \newline
12. ए॒वावा वै॒वै वाव॑ रुन्धते रुन्ध॒ते ऽवै॒वै वाव॑ रुन्धते । \newline
13. अव॑ रुन्धते रुन्ध॒ते ऽवाव॑ रुन्धते विश्व॒जिद् वि॑श्व॒जिद् रु॑न्ध॒ते ऽवाव॑ रुन्धते विश्व॒जित् । \newline
14. रु॒न्ध॒ते॒ वि॒श्व॒जिद् वि॑श्व॒जिद् रु॑न्धते रुन्धते विश्व॒जिद् भ॑वति भवति विश्व॒जिद् रु॑न्धते रुन्धते विश्व॒जिद् भ॑वति । \newline
15. वि॒श्व॒जिद् भ॑वति भवति विश्व॒जिद् वि॑श्व॒जिद् भ॑व त्य॒न्नाद्य॑स्या॒ न्नाद्य॑स्य भवति विश्व॒जिद् वि॑श्व॒जिद् भ॑व त्य॒न्नाद्य॑स्य । \newline
16. वि॒श्व॒जिदिति॑ विश्व - जित् । \newline
17. भ॒व॒ त्य॒न्नाद्य॑स्या॒ न्नाद्य॑स्य भवति भव त्य॒न्नाद्य॒स्या व॑रुद्ध्या॒ अव॑रुद्ध्या अ॒न्नाद्य॑स्य भवति भव त्य॒न्नाद्य॒स्या व॑रुद्ध्यै । \newline
18. अ॒न्नाद्य॒स्या व॑रुद्ध्या॒ अव॑रुद्ध्या अ॒न्नाद्य॑स्या॒ न्नाद्य॒स्या व॑रुद्ध्यै॒ सर्व॑पृष्ठः॒ सर्व॑पृ॒ष्ठो ऽव॑रुद्ध्या अ॒न्नाद्य॑स्या॒ न्नाद्य॒स्या व॑रुद्ध्यै॒ सर्व॑पृष्ठः । \newline
19. अ॒न्नाद्य॒स्येत्य॑न्न - अद्य॑स्य । \newline
20. अव॑रुद्ध्यै॒ सर्व॑पृष्ठः॒ सर्व॑पृ॒ष्ठो ऽव॑रुद्ध्या॒ अव॑रुद्ध्यै॒ सर्व॑पृष्ठो भवति भवति॒ सर्व॑पृ॒ष्ठो ऽव॑रुद्ध्या॒ अव॑रुद्ध्यै॒ सर्व॑पृष्ठो भवति । \newline
21. अव॑रुद्ध्या॒ इत्यव॑ - रु॒द्ध्यै॒ । \newline
22. सर्व॑पृष्ठो भवति भवति॒ सर्व॑पृष्ठः॒ सर्व॑पृष्ठो भवति॒ सर्व॑स्य॒ सर्व॑स्य भवति॒ सर्व॑पृष्ठः॒ सर्व॑पृष्ठो भवति॒ सर्व॑स्य । \newline
23. सर्व॑पृष्ठ॒ इति॒ सर्व॑ - पृ॒ष्ठः॒ । \newline
24. भ॒व॒ति॒ सर्व॑स्य॒ सर्व॑स्य भवति भवति॒ सर्व॑स्या॒ भिजि॑त्या अ॒भिजि॑त्यै॒ सर्व॑स्य भवति भवति॒ सर्व॑स्या॒ भिजि॑त्यै । \newline
25. सर्व॑स्या॒ भिजि॑त्या अ॒भिजि॑त्यै॒ सर्व॑स्य॒ सर्व॑स्या॒ भिजि॑त्यै॒ वाग् वाग॒भिजि॑त्यै॒ सर्व॑स्य॒ सर्व॑स्या॒ भिजि॑त्यै॒ वाक् । \newline
26. अ॒भिजि॑त्यै॒ वाग् वाग॒भिजि॑त्या अ॒भिजि॑त्यै॒ वाग् वै वै वाग॒भिजि॑त्या अ॒भिजि॑त्यै॒ वाग् वै । \newline
27. अ॒भिजि॑त्या॒ इत्य॒भि - जि॒त्यै॒ । \newline
28. वाग् वै वै वाग् वाग् वै द्वा॑दशा॒हो द्वा॑दशा॒हो वै वाग् वाग् वै द्वा॑दशा॒हः । \newline
29. वै द्वा॑दशा॒हो द्वा॑दशा॒हो वै वै द्वा॑दशा॒हो यद् यद् द्वा॑दशा॒हो वै वै द्वा॑दशा॒हो यत् । \newline
30. द्वा॒द॒शा॒हो यद् यद् द्वा॑दशा॒हो द्वा॑दशा॒हो यत् पु॒रस्ता᳚त् पु॒रस्ता॒द् यद् द्वा॑दशा॒हो द्वा॑दशा॒हो यत् पु॒रस्ता᳚त् । \newline
31. द्वा॒द॒शा॒ह इति॑ द्वादश - अ॒हः । \newline
32. यत् पु॒रस्ता᳚त् पु॒रस्ता॒द् यद् यत् पु॒रस्ता᳚द् द्वादशा॒हम् द्वा॑दशा॒हम् पु॒रस्ता॒द् यद् यत् पु॒रस्ता᳚द् द्वादशा॒हम् । \newline
33. पु॒रस्ता᳚द् द्वादशा॒हम् द्वा॑दशा॒हम् पु॒रस्ता᳚त् पु॒रस्ता᳚द् द्वादशा॒ह मु॑पे॒यु रु॑पे॒युर् द्वा॑दशा॒हम् पु॒रस्ता᳚त् पु॒रस्ता᳚द् द्वादशा॒ह मु॑पे॒युः । \newline
34. द्वा॒द॒शा॒ह मु॑पे॒यु रु॑पे॒युर् द्वा॑दशा॒हम् द्वा॑दशा॒ह मु॑पे॒यु रना᳚प्ता॒ मना᳚प्ता मुपे॒युर् द्वा॑दशा॒हम् द्वा॑दशा॒ह मु॑पे॒यु रना᳚प्ताम् । \newline
35. द्वा॒द॒शा॒हमिति॑ द्वादश - अ॒हम् । \newline
36. उ॒पे॒यु रना᳚प्ता॒ मना᳚प्ता मुपे॒यु रु॑पे॒यु रना᳚प्तां॒ ॅवाचं॒ ॅवाच॒ मना᳚प्ता मुपे॒यु रु॑पे॒यु रना᳚प्तां॒ ॅवाच᳚म् । \newline
37. उ॒पे॒युरित्यु॑प - इ॒युः । \newline
38. अना᳚प्तां॒ ॅवाचं॒ ॅवाच॒ मना᳚प्ता॒ मना᳚प्तां॒ ॅवाच॒ मुपोप॒ वाच॒ मना᳚प्ता॒ मना᳚प्तां॒ ॅवाच॒ मुप॑ । \newline
39. वाच॒ मुपोप॒ वाचं॒ ॅवाच॒ मुपे॑यु रियु॒ रुप॒ वाचं॒ ॅवाच॒ मुपे॑युः । \newline
40. उपे॑यु रियु॒ रुपोपे॑यु रुप॒दासु॑ कोप॒दासु॑केयु॒ रुपोपे॑यु रुप॒दासु॑का । \newline
41. इ॒यु॒ रु॒प॒दासु॑को प॒दासु॑केयु रियु रुप॒दासु॑कैषा मेषा मुप॒दासु॑केयु रियु रुप॒दासु॑कैषाम् । \newline
42. उ॒प॒दासु॑कैषा मेषा मुप॒दासु॑ कोप॒दासु॑कैषां॒ ॅवाग् वागे॑षा मुप॒दासु॑ कोप॒दासु॑कैषां॒ ॅवाक् । \newline
43. उ॒प॒दासु॒केत्यु॑प - दासु॑का । \newline
44. ए॒षां॒ ॅवाग् वागे॑षा मेषां॒ ॅवाख् स्या᳚थ् स्या॒द् वागे॑षा मेषां॒ ॅवाख् स्या᳚त् । \newline
45. वाख् स्या᳚थ् स्या॒द् वाग् वाख् स्या॑ दु॒परि॑ष्टा दु॒परि॑ष्टाथ् स्या॒द् वाग् वाख् स्या॑ दु॒परि॑ष्टात् । \newline
46. स्या॒ दु॒परि॑ष्टा दु॒परि॑ष्टा थ्स्याथ् स्या दु॒परि॑ष्टाद् द्वादशा॒हम् द्वा॑दशा॒ह मु॒परि॑ष्टा थ्स्याथ् स्या दु॒परि॑ष्टाद् द्वादशा॒हम् । \newline
47. उ॒परि॑ष्टाद् द्वादशा॒हम् द्वा॑दशा॒ह मु॒परि॑ष्टा दु॒परि॑ष्टाद् द्वादशा॒ह मुपोप॑ द्वादशा॒ह मु॒परि॑ष्टा दु॒परि॑ष्टाद् द्वादशा॒ह मुप॑ । \newline
48. द्वा॒द॒शा॒ह मुपोप॑ द्वादशा॒हम् द्वा॑दशा॒ह मुप॑ यन्ति य॒न्त्युप॑ द्वादशा॒हम् द्वा॑दशा॒ह मुप॑ यन्ति । \newline
49. द्वा॒द॒शा॒हमिति॑ द्वादश - अ॒हम् । \newline
50. उप॑ यन्ति य॒न्त्युपोप॑ यन्त्या॒प्ता मा॒प्तां ॅय॒न्त्युपोप॑ यन्त्या॒प्ताम् । \newline
51. य॒न्त्या॒प्ता मा॒प्तां ॅय॑न्ति यन्त्या॒प्ता मे॒वैवाप्तां ॅय॑न्ति यन्त्या॒प्ता मे॒व । \newline
52. आ॒प्ता मे॒वै वाप्ता मा॒प्ता मे॒व वाचं॒ ॅवाच॑ मे॒वाप्ता मा॒प्ता मे॒व वाच᳚म् । \newline
53. ए॒व वाचं॒ ॅवाच॑ मे॒वैव वाच॒ मुपोप॒ वाच॑ मे॒वैव वाच॒ मुप॑ । \newline
54. वाच॒ मुपोप॒ वाचं॒ ॅवाच॒ मुप॑ यन्ति य॒न्त्युप॒ वाचं॒ ॅवाच॒ मुप॑ यन्ति । \newline
55. उप॑ यन्ति य॒न्त्युपोप॑ यन्ति॒ तस्मा॒त् तस्मा᳚द् य॒न्त्युपोप॑ यन्ति॒ तस्मा᳚त् । \newline
56. य॒न्ति॒ तस्मा॒त् तस्मा᳚द् यन्ति यन्ति॒ तस्मा॑ दु॒परि॑ष्टा दु॒परि॑ष्टा॒त् तस्मा᳚द् यन्ति यन्ति॒ तस्मा॑ दु॒परि॑ष्टात् । \newline
57. तस्मा॑ दु॒परि॑ष्टा दु॒परि॑ष्टा॒त् तस्मा॒त् तस्मा॑ दु॒परि॑ष्टाद् वा॒चा वा॒चोपरि॑ष्टा॒त् तस्मा॒त् तस्मा॑ दु॒परि॑ष्टाद् वा॒चा । \newline
58. उ॒परि॑ष्टाद् वा॒चा वा॒चोपरि॑ष्टा दु॒परि॑ष्टाद् वा॒चा व॑दामो वदामो वा॒चोपरि॑ष्टा दु॒परि॑ष्टाद् वा॒चा व॑दामः । \newline
59. वा॒चा व॑दामो वदामो वा॒चा वा॒चा व॑दामो ऽवान्त॒र म॑वान्त॒रं ॅव॑दामो वा॒चा वा॒चा व॑दामो ऽवान्त॒रम् । \newline
60. व॒दा॒मो॒ ऽवा॒न्त॒र म॑वान्त॒रं ॅव॑दामो वदामो ऽवान्त॒रं ॅवै वा अ॑वान्त॒रं ॅव॑दामो वदामो ऽवान्त॒रं ॅवै । \newline
61. अ॒वा॒न्त॒रं ॅवै वा अ॑वान्त॒र म॑वान्त॒रं ॅवै द॑शरा॒त्रेण॑ दशरा॒त्रेण॒ वा अ॑वान्त॒र म॑वान्त॒रं ॅवै द॑शरा॒त्रेण॑ । \newline
62. अ॒वा॒न्त॒रमित्य॑व - अ॒न्त॒रम् । \newline
\pagebreak
\markright{ TS 7.4.5.4  \hfill https://www.vedavms.in \hfill}

\section{ TS 7.4.5.4 }

\textbf{TS 7.4.5.4 } \newline
\textbf{Samhita Paata} \newline

ॅवै द॑शरा॒त्रेण॑ प्र॒जाप॑तिः प्र॒जा अ॑सृजत॒ यद्-द॑शरा॒त्रो भव॑ति प्र॒जा ए॒व तद् यज॑मानाः सृजन्त ए॒ताꣳ ह॒ वा उ॑द॒ङ्कः शौ᳚ल्बाय॒नः स॒त्रस्यर्द्धि॑मुवाच॒ यद् द॑शरा॒त्रो यद् द॑शरा॒त्रो भव॑ति स॒त्रस्यर्द्ध्या॒ अथो॒ यदे॒व पूर्वे॒ष्वह॑स्सु॒ विलो॑म क्रि॒यते॒ तस्यै॒वैषा शान्ति॑द्र्व्यनी॒का वा ए॒ता रात्र॑यो॒ यज॑माना विश्व॒जिथ् स॒हाति॑रा॒त्रेण॒ पूर्वाः॒ षोड॑श स॒हा ( ) ति॑रा॒त्रेणोत्त॑राः॒ षोड॑श॒ य ए॒वं ॅवि॒द्वाꣳस॑स्त्रयस्त्रिꣳशद॒हमास॑त॒ ऐषां᳚ द्व्यनी॒का प्र॒जा जा॑यते ऽतिरा॒त्राव॒भितो॑ भवतः॒ परि॑गृहीत्यै ॥ \newline

\textbf{Pada Paata} \newline

वै । द॒श॒रा॒त्रेणेति॑ दश - रा॒त्रेण॑ । प्र॒जाप॑ति॒रिति॑ प्र॒जा - प॒तिः॒ । प्र॒जा इति॑ प्र - जाः । अ॒सृ॒ज॒त॒ । यत् । द॒श॒रा॒त्र इति॑ दश - रा॒त्रः । भव॑ति । प्र॒जा इति॑ प्र - जाः । ए॒व । तत् । यज॑मानाः । सृ॒ज॒न्ते॒ । ए॒ताम् । ह॒ । वै । उ॒द॒ङ्कः । शौ॒ल्बा॒य॒नः । स॒त्रस्य॑ । ऋद्धि᳚म् । उ॒वा॒च॒ । यत् । द॒श॒रा॒त्र इति॑ दश - रा॒त्रः । यत् । द॒श॒रा॒त्र इति॑ दश - रा॒त्रः । भव॑ति । स॒त्रस्य॑ । ऋद्ध्यै᳚ । अथो॒ इति॑ । यत् । ए॒व । पूर्वे॑षु । अह॒स्स्वित्यहः॑ - सु॒ । विलो॒मेति॒ वि - लो॒म॒ । क्रि॒यते᳚ । तस्य॑ । ए॒व । ए॒षा । शान्तिः॑ । द्व्य॒नी॒का इति॑ द्वि - अ॒नी॒काः । वै । ए॒ताः । रात्र॑यः । यज॑मानाः । वि॒श्व॒जिदिति॑ विश्व - जित् । स॒ह । अ॒ति॒रा॒त्रेणेत्य॑ति - रा॒त्रेण॑ । पूर्वाः᳚ । षोड॑श । स॒ह ( ) । अ॒ति॒रा॒त्रेणेत्य॑ति - रा॒त्रेण॑ । उत्त॑रा॒ इत्युत् - त॒राः॒ । षोड॑श । ये । ए॒वम् । वि॒द्वाꣳसः॑ । त्र॒य॒स्त्रिꣳ॒॒श॒द॒हमिति॑ त्रयस्त्रिꣳशत् - अ॒हम् । आस॑ते । एति॑ । ए॒षा॒म् । द्व्य॒नी॒केति॑ द्वि - अ॒नी॒का । प्र॒जेति॑ प्र - जा । जा॒य॒ते॒ । अ॒ति॒रा॒त्रावित्य॑ति - रा॒त्रौ । अ॒भितः॑ । भ॒व॒तः॒ । परि॑गृहीत्या॒ इति॒ परि॑ - गृ॒ही॒त्यै॒ ॥  \newline


\textbf{Krama Paata} \newline

वै द॑शरा॒त्रेण॑ । द॒श॒रा॒त्रेण॑ प्र॒जाप॑तिः । द॒श॒रा॒त्रेणेति॑ दश - रा॒त्रेण॑ । प्र॒जाप॑तिः प्र॒जाः । प्र॒जाप॑ति॒रिति॑ प्र॒जा - प॒तिः॒ । प्र॒जा अ॑सृजत । प्र॒जा इति॑ प्र - जाः । अ॒सृ॒ज॒त॒ यत् । यद् द॑शरा॒त्रः । द॒श॒रा॒त्रो भव॑ति । द॒श॒रा॒त्र इति॑ दश - रा॒त्रः । भव॑ति प्र॒जाः । प्र॒जा ए॒व । प्र॒जा इति॑ प्र - जाः । ए॒व तत् । तद् यज॑मानाः । यज॑मानाः सृजन्ते । सृ॒ज॒न्त॒ ए॒ताम् । ए॒ताꣳ ह॑ । ह॒ वै । वा उ॑द॒ङ्‍कः । उ॒द॒ङ्‍कः शौ᳚ल्बाय॒नः । शौ॒ल्बा॒य॒नः स॒त्रस्य॑ । स॒त्रस्यर्द्धि᳚म् । ऋद्धि॑मुवाच । उ॒वा॒च॒ यत् । यद् द॑शरा॒त्रः । द॒श॒रा॒त्रो यत् । द॒श॒रा॒त्र इति॑ दश - रा॒त्रः । यद् द॑शरा॒त्रः । द॒श॒रा॒त्रो भव॑ति । द॒श॒रा॒त्र इति॑ दश - रा॒त्रः । भव॑ति स॒त्रस्य॑ । स॒त्रस्यर्द्ध्यै᳚ । ऋद्ध्या॒ अथो᳚ । अथो॒ यत् । अथो॒ इत्यथो᳚ । यदे॒व । ए॒व पूर्वे॑षु । पूर्वे॒ष्वह॑स्सु । अह॑स्सु॒ विलो॑म । अह॒स्स्वित्यहः॑ - सु॒ । विलो॑म क्रि॒यते᳚ । विलो॒मेति॒ वि - लो॒म॒ । क्रि॒यते॒ तस्य॑ । तस्यै॒व । ए॒वैषा । ए॒षा शान्तिः॑ । शान्ति॑र् द्व्यनी॒काः । द्व्य॒नी॒का वै । द्व्य॒नी॒का इति॑ द्वि - अ॒नी॒काः । वा ए॒ताः । ए॒ता रात्र॑यः । रात्र॑यो॒ यज॑मानाः । यज॑माना विश्व॒जित् । वि॒श्व॒जिथ् स॒ह । वि॒श्व॒जिदिति॑ विश्व - जित् । स॒हाति॑रा॒त्रेण॑ । अ॒ति॒रा॒त्रेण॒ पूर्वाः᳚ । अ॒ति॒रा॒त्रेणेत्य॑ति - रा॒त्रेण॑ । पूर्वाः॒ षोड॑श । षोड॑श स॒ह ( ) । स॒हाति॑रा॒त्रेण॑ । अ॒ति॒रा॒त्रेणोत्त॑राः । अ॒ति॒रा॒त्रेणेत्य॑ति - रा॒त्रेण॑ । उत्त॑राः॒ षोड॑श । उत्त॑रा॒ इत्युत् - त॒राः॒ । षोड॑श॒ ये । य ए॒वम् । ए॒वम् ॅवि॒द्वाꣳसः॑ । वि॒द्वाꣳस॑स्त्रयस्त्रिꣳशद॒हम् । त्र॒य॒स्त्रिꣳ॒॒श॒द॒हमास॑ते । त्र॒य॒स्त्रिꣳ॒॒श॒द॒हमिति॑ त्रयस्त्रिꣳशत् - अ॒हम् । आस॑त॒ आ । ऐषा᳚म् । ए॒षा॒म् द्व्य॒नी॒का । द्व्य॒नी॒का प्र॒जाः । द्व्य॒नी॒केति॑ द्वि - अ॒नी॒का । प्र॒जा जा॑यते । प्र॒जेति॑ प्र - जाः । जा॒य॒ते॒ऽति॒रा॒त्रौ । अ॒ति॒रा॒त्राव॒भितः॑ । अ॒ति॒रा॒त्रावित्य॑ति - रा॒त्रौ । अ॒भितो॑ भवतः । भ॒व॒तः॒ परि॑गृहीत्यै । परि॑गृहीत्या॒ इति॒ परि॑ - गृ॒ही॒त्यै॒ । \newline

\textbf{Jatai Paata} \newline

1. वै द॑शरा॒त्रेण॑ दशरा॒त्रेण॒ वै वै द॑शरा॒त्रेण॑ । \newline
2. द॒श॒रा॒त्रेण॑ प्र॒जाप॑तिः प्र॒जाप॑तिर् दशरा॒त्रेण॑ दशरा॒त्रेण॑ प्र॒जाप॑तिः । \newline
3. द॒श॒रा॒त्रेणेति॑ दश - रा॒त्रेण॑ । \newline
4. प्र॒जाप॑तिः प्र॒जाः प्र॒जाः प्र॒जाप॑तिः प्र॒जाप॑तिः प्र॒जाः । \newline
5. प्र॒जाप॑ति॒रिति॑ प्र॒जा - प॒तिः॒ । \newline
6. प्र॒जा अ॑सृजता सृजत प्र॒जाः प्र॒जा अ॑सृजत । \newline
7. प्र॒जा इति॑ प्र - जाः । \newline
8. अ॒सृ॒ज॒त॒ यद् यद॑सृजता सृजत॒ यत् । \newline
9. यद् द॑शरा॒त्रो द॑शरा॒त्रो यद् यद् द॑शरा॒त्रः । \newline
10. द॒श॒रा॒त्रो भव॑ति॒ भव॑ति दशरा॒त्रो द॑शरा॒त्रो भव॑ति । \newline
11. द॒श॒रा॒त्र इति॑ दश - रा॒त्रः । \newline
12. भव॑ति प्र॒जाः प्र॒जा भव॑ति॒ भव॑ति प्र॒जाः । \newline
13. प्र॒जा ए॒वैव प्र॒जाः प्र॒जा ए॒व । \newline
14. प्र॒जा इति॑ प्र - जाः । \newline
15. ए॒व तत् तदे॒वैव तत् । \newline
16. तद् यज॑माना॒ यज॑माना॒ स्तत् तद् यज॑मानाः । \newline
17. यज॑मानाः सृजन्ते सृजन्ते॒ यज॑माना॒ यज॑मानाः सृजन्ते । \newline
18. सृ॒ज॒न्त॒ ए॒ता मे॒ताꣳ सृ॑जन्ते सृजन्त ए॒ताम् । \newline
19. ए॒ताꣳ ह॑ है॒ता मे॒ताꣳ ह॑ । \newline
20. ह॒ वै वै ह॑ ह॒ वै । \newline
21. वा उ॑द॒ङ्क उ॑द॒ङ्को वै वा उ॑द॒ङ्कः । \newline
22. उ॒द॒ङ्कः शौ᳚ल्बाय॒नः शौ᳚ल्बाय॒न उ॑द॒ङ्क उ॑द॒ङ्कः शौ᳚ल्बाय॒नः । \newline
23. शौ॒ल्बा॒य॒नः स॒त्रस्य॑ स॒त्रस्य॑ शौल्बाय॒नः शौ᳚ल्बाय॒नः स॒त्रस्य॑ । \newline
24. स॒त्रस्य र्‌द्धि॒ मृद्धिꣳ॑ स॒त्रस्य॑ स॒त्रस्य र्‌द्धि᳚म् । \newline
25. ऋद्धि॑ मुवा चोवा॒च र्‌द्धि॒ मृद्धि॑ मुवाच । \newline
26. उ॒वा॒च॒ यद् यदु॑वा चोवाच॒ यत् । \newline
27. यद् द॑शरा॒त्रो द॑शरा॒त्रो यद् यद् द॑शरा॒त्रः । \newline
28. द॒श॒रा॒त्रो यद् यद् द॑शरा॒त्रो द॑शरा॒त्रो यत् । \newline
29. द॒श॒रा॒त्र इति॑ दश - रा॒त्रः । \newline
30. यद् द॑शरा॒त्रो द॑शरा॒त्रो यद् यद् द॑शरा॒त्रः । \newline
31. द॒श॒रा॒त्रो भव॑ति॒ भव॑ति दशरा॒त्रो द॑शरा॒त्रो भव॑ति । \newline
32. द॒श॒रा॒त्र इति॑ दश - रा॒त्रः । \newline
33. भव॑ति स॒त्रस्य॑ स॒त्रस्य॒ भव॑ति॒ भव॑ति स॒त्रस्य॑ । \newline
34. स॒त्रस्य र्‌द्ध्या॒ ऋद्ध्यै॑ स॒त्रस्य॑ स॒त्रस्य र्‌द्ध्यै᳚ । \newline
35. ऋद्ध्या॒ अथो॒ अथो॒ ऋद्ध्या॒ ऋद्ध्या॒ अथो᳚ । \newline
36. अथो॒ यद् यदथो॒ अथो॒ यत् । \newline
37. अथो॒ इत्यथो᳚ । \newline
38. यदे॒वैव यद् यदे॒व । \newline
39. ए॒व पूर्वे॑षु॒ पूर्वे᳚ ष्वे॒वैव पूर्वे॑षु । \newline
40. पूर्वे॒ ष्वह॒ स्स्वह॑स्सु॒ पूर्वे॑षु॒ पूर्वे॒ ष्वह॑स्सु । \newline
41. अह॑स्सु॒ विलो॑म॒ विलो॒मा ह॒ स्स्वह॑स्सु॒ विलो॑म । \newline
42. अह॒स्स्वित्यहः॑ - सु॒ । \newline
43. विलो॑म क्रि॒यते᳚ क्रि॒यते॒ विलो॑म॒ विलो॑म क्रि॒यते᳚ । \newline
44. विलो॒मेति॒ वि - लो॒म॒ । \newline
45. क्रि॒यते॒ तस्य॒ तस्य॑ क्रि॒यते᳚ क्रि॒यते॒ तस्य॑ । \newline
46. तस्यै॒वैव तस्य॒ तस्यै॒व । \newline
47. ए॒वै षैषै वैवैषा । \newline
48. ए॒षा शान्तिः॒ शान्ति॑ रे॒षैषा शान्तिः॑ । \newline
49. शान्ति॑र् द्व्यनी॒का द्व्य॑नी॒काः शान्तिः॒ शान्ति॑र् द्व्यनी॒काः । \newline
50. द्व्य॒नी॒का वै वै द्व्य॑नी॒का द्व्य॑नी॒का वै । \newline
51. द्व्य॒नी॒का इति॑ द्वि - अ॒नी॒काः । \newline
52. वा ए॒ता ए॒ता वै वा ए॒ताः । \newline
53. ए॒ता रात्र॑यो॒ रात्र॑य ए॒ता ए॒ता रात्र॑यः । \newline
54. रात्र॑यो॒ यज॑माना॒ यज॑माना॒ रात्र॑यो॒ रात्र॑यो॒ यज॑मानाः । \newline
55. यज॑माना विश्व॒जिद् वि॑श्व॒जिद् यज॑माना॒ यज॑माना विश्व॒जित् । \newline
56. वि॒श्व॒जिथ् स॒ह स॒ह वि॑श्व॒जिद् वि॑श्व॒जिथ् स॒ह । \newline
57. वि॒श्व॒जिदिति॑ विश्व - जित् । \newline
58. स॒हा ति॑रा॒त्रेणा॑ तिरा॒त्रेण॑ स॒ह स॒हा ति॑रा॒त्रेण॑ । \newline
59. अ॒ति॒रा॒त्रेण॒ पूर्वाः॒ पूर्वा॑ अतिरा॒त्रेणा॑ तिरा॒त्रेण॒ पूर्वाः᳚ । \newline
60. अ॒ति॒रा॒त्रेणेत्य॑ति - रा॒त्रेण॑ । \newline
61. पूर्वा॒ ष्षोड॑श॒ षोड॑श॒ पूर्वाः॒ पूर्वा॒ ष्षोड॑श । \newline
62. षोड॑श स॒ह स॒ह षोड॑श॒ षोड॑श स॒ह । \newline
63. स॒हा ति॑रा॒त्रेणा॑ तिरा॒त्रेण॑ स॒ह स॒हा ति॑रा॒त्रेण॑ । \newline
64. अ॒ति॒रा॒त्रे णोत्त॑रा॒ उत्त॑रा अतिरा॒त्रेणा॑ तिरा॒त्रे णोत्त॑राः । \newline
65. अ॒ति॒रा॒त्रेणेत्य॑ति - रा॒त्रेण॑ । \newline
66. उत्त॑रा॒ ष्षोड॑श॒ षोड॒शो त्त॑रा॒ उत्त॑रा॒ ष्षोड॑श । \newline
67. उत्त॑रा॒ इत्युत् - त॒राः॒ । \newline
68. षोड॑श॒ ये ये षोड॑श॒ षोड॑श॒ ये । \newline
69. य ए॒व मे॒वं ॅये य ए॒वम् । \newline
70. ए॒वं ॅवि॒द्वाꣳसो॑ वि॒द्वाꣳस॑ ए॒व मे॒वं ॅवि॒द्वाꣳसः॑ । \newline
71. वि॒द्वाꣳस॑ स्त्रयस्त्रिꣳशद॒हम् त्र॑यस्त्रिꣳशद॒हं ॅवि॒द्वाꣳसो॑ वि॒द्वाꣳस॑ स्त्रयस्त्रिꣳशद॒हम् । \newline
72. त्र॒य॒स्त्रिꣳ॒॒श॒द॒ह मास॑त॒ आस॑ते त्रयस्त्रिꣳशद॒हम् त्र॑यस्त्रिꣳशद॒ह मास॑ते । \newline
73. त्र॒य॒स्त्रिꣳ॒॒श॒द॒हमिति॑ त्रयस्त्रिꣳशत् - अ॒हम् । \newline
74. आस॑त॒ आ ऽऽस॑त॒ आस॑त॒ आ । \newline
75. ऐषा॑ मेषा॒ मैषा᳚म् । \newline
76. ए॒षा॒म् द्व्य॒नी॒का द्व्य॑नी॒कैषा॑ मेषाम् द्व्यनी॒का । \newline
77. द्व्य॒नी॒का प्र॒जा प्र॒जा द्व्य॑नी॒का द्व्य॑नी॒का प्र॒जा । \newline
78. द्व्य॒नी॒केति॑ द्वि - अ॒नी॒का । \newline
79. प्र॒जा जा॑यते जायते प्र॒जा प्र॒जा जा॑यते । \newline
80. प्र॒जेति॑ प्र - जा । \newline
81. जा॒य॒ते॒ ऽति॒रा॒त्रा व॑तिरा॒त्रौ जा॑यते जायते ऽतिरा॒त्रौ । \newline
82. अ॒ति॒रा॒त्रा व॒भितो॒ ऽभितो॑ ऽतिरा॒त्रा व॑तिरा॒त्रा व॒भितः॑ । \newline
83. अ॒ति॒रा॒त्रावित्य॑ति - रा॒त्रौ । \newline
84. अ॒भितो॑ भवतो भवतो॒ ऽभितो॒ ऽभितो॑ भवतः । \newline
85. भ॒व॒तः॒ परि॑गृहीत्यै॒ परि॑गृहीत्यै भवतो भवतः॒ परि॑गृहीत्यै । \newline
86. परि॑गृहीत्या॒ इति॒ परि॑ - गृ॒ही॒त्यै॒ । \newline

\textbf{Ghana Paata } \newline

1. वै द॑शरा॒त्रेण॑ दशरा॒त्रेण॒ वै वै द॑शरा॒त्रेण॑ प्र॒जाप॑तिः प्र॒जाप॑तिर् दशरा॒त्रेण॒ वै वै द॑शरा॒त्रेण॑ प्र॒जाप॑तिः । \newline
2. द॒श॒रा॒त्रेण॑ प्र॒जाप॑तिः प्र॒जाप॑तिर् दशरा॒त्रेण॑ दशरा॒त्रेण॑ प्र॒जाप॑तिः प्र॒जाः प्र॒जाः प्र॒जाप॑तिर् दशरा॒त्रेण॑ दशरा॒त्रेण॑ प्र॒जाप॑तिः प्र॒जाः । \newline
3. द॒श॒रा॒त्रेणेति॑ दश - रा॒त्रेण॑ । \newline
4. प्र॒जाप॑तिः प्र॒जाः प्र॒जाः प्र॒जाप॑तिः प्र॒जाप॑तिः प्र॒जा अ॑सृजता सृजत प्र॒जाः प्र॒जाप॑तिः प्र॒जाप॑तिः प्र॒जा अ॑सृजत । \newline
5. प्र॒जाप॑ति॒रिति॑ प्र॒जा - प॒तिः॒ । \newline
6. प्र॒जा अ॑सृजता सृजत प्र॒जाः प्र॒जा अ॑सृजत॒ यद् यद॑सृजत प्र॒जाः प्र॒जा अ॑सृजत॒ यत् । \newline
7. प्र॒जा इति॑ प्र - जाः । \newline
8. अ॒सृ॒ज॒त॒ यद् यद॑सृजता सृजत॒ यद् द॑शरा॒त्रो द॑शरा॒त्रो यद॑सृजता सृजत॒ यद् द॑शरा॒त्रः । \newline
9. यद् द॑शरा॒त्रो द॑शरा॒त्रो यद् यद् द॑शरा॒त्रो भव॑ति॒ भव॑ति दशरा॒त्रो यद् यद् द॑शरा॒त्रो भव॑ति । \newline
10. द॒श॒रा॒त्रो भव॑ति॒ भव॑ति दशरा॒त्रो द॑शरा॒त्रो भव॑ति प्र॒जाः प्र॒जा भव॑ति दशरा॒त्रो द॑शरा॒त्रो भव॑ति प्र॒जाः । \newline
11. द॒श॒रा॒त्र इति॑ दश - रा॒त्रः । \newline
12. भव॑ति प्र॒जाः प्र॒जा भव॑ति॒ भव॑ति प्र॒जा ए॒वैव प्र॒जा भव॑ति॒ भव॑ति प्र॒जा ए॒व । \newline
13. प्र॒जा ए॒वैव प्र॒जाः प्र॒जा ए॒व तत् तदे॒व प्र॒जाः प्र॒जा ए॒व तत् । \newline
14. प्र॒जा इति॑ प्र - जाः । \newline
15. ए॒व तत् तदे॒वैव तद् यज॑माना॒ यज॑माना॒ स्तदे॒वैव तद् यज॑मानाः । \newline
16. तद् यज॑माना॒ यज॑माना॒ स्तत् तद् यज॑मानाः सृजन्ते सृजन्ते॒ यज॑माना॒ स्तत् तद् यज॑मानाः सृजन्ते । \newline
17. यज॑मानाः सृजन्ते सृजन्ते॒ यज॑माना॒ यज॑मानाः सृजन्त ए॒ता मे॒ताꣳ सृ॑जन्ते॒ यज॑माना॒ यज॑मानाः सृजन्त ए॒ताम् । \newline
18. सृ॒ज॒न्त॒ ए॒ता मे॒ताꣳ सृ॑जन्ते सृजन्त ए॒ताꣳ ह॑ है॒ताꣳ सृ॑जन्ते सृजन्त ए॒ताꣳ ह॑ । \newline
19. ए॒ताꣳ ह॑ है॒ता मे॒ताꣳ ह॒ वै वै है॒ता मे॒ताꣳ ह॒ वै । \newline
20. ह॒ वै वै ह॑ ह॒ वा उ॑द॒ङ्क उ॑द॒ङ्को वै ह॑ ह॒ वा उ॑द॒ङ्कः । \newline
21. वा उ॑द॒ङ्क उ॑द॒ङ्को वै वा उ॑द॒ङ्कः शौ᳚ल्बाय॒नः शौ᳚ल्बाय॒न उ॑द॒ङ्को वै वा उ॑द॒ङ्कः शौ᳚ल्बाय॒नः । \newline
22. उ॒द॒ङ्कः शौ᳚ल्बाय॒नः शौ᳚ल्बाय॒न उ॑द॒ङ्क उ॑द॒ङ्कः शौ᳚ल्बाय॒नः स॒त्रस्य॑ स॒त्रस्य॑ शौल्बाय॒न उ॑द॒ङ्क उ॑द॒ङ्कः शौ᳚ल्बाय॒नः स॒त्रस्य॑ । \newline
23. शौ॒ल्बा॒य॒नः स॒त्रस्य॑ स॒त्रस्य॑ शौल्बाय॒नः शौ᳚ल्बाय॒नः स॒त्रस्य र्‌द्धि॒ मृद्धिꣳ॑ स॒त्रस्य॑ शौल्बाय॒नः शौ᳚ल्बाय॒नः स॒त्रस्य र्‌द्धि᳚म् । \newline
24. स॒त्रस्य र्‌द्धि॒ मृद्धिꣳ॑ स॒त्रस्य॑ स॒त्रस्य र्‌द्धि॑ मुवा चोवा॒च र्‌द्धिꣳ॑ स॒त्रस्य॑ स॒त्रस्य र्‌द्धि॑ मुवाच । \newline
25. ऋद्धि॑ मुवा चोवा॒च र्‌द्धि॒ मृद्धि॑ मुवाच॒ यद् यदु॑वा॒च र्‌द्धि॒ मृद्धि॑ मुवाच॒ यत् । \newline
26. उ॒वा॒च॒ यद् यदु॑वा चोवाच॒ यद् द॑शरा॒त्रो द॑शरा॒त्रो यदु॑वा चोवाच॒ यद् द॑शरा॒त्रः । \newline
27. यद् द॑शरा॒त्रो द॑शरा॒त्रो यद् यद् द॑शरा॒त्रो यद् यद् द॑शरा॒त्रो यद् यद् द॑शरा॒त्रो यत् । \newline
28. द॒श॒रा॒त्रो यद् यद् द॑शरा॒त्रो द॑शरा॒त्रो यद् द॑शरा॒त्रो द॑शरा॒त्रो यद् द॑शरा॒त्रो द॑शरा॒त्रो यद् द॑शरा॒त्रः । \newline
29. द॒श॒रा॒त्र इति॑ दश - रा॒त्रः । \newline
30. यद् द॑शरा॒त्रो द॑शरा॒त्रो यद् यद् द॑शरा॒त्रो भव॑ति॒ भव॑ति दशरा॒त्रो यद् यद् द॑शरा॒त्रो भव॑ति । \newline
31. द॒श॒रा॒त्रो भव॑ति॒ भव॑ति दशरा॒त्रो द॑शरा॒त्रो भव॑ति स॒त्रस्य॑ स॒त्रस्य॒ भव॑ति दशरा॒त्रो द॑शरा॒त्रो भव॑ति स॒त्रस्य॑ । \newline
32. द॒श॒रा॒त्र इति॑ दश - रा॒त्रः । \newline
33. भव॑ति स॒त्रस्य॑ स॒त्रस्य॒ भव॑ति॒ भव॑ति स॒त्रस्य र्‌द्ध्या॒ ऋद्ध्यै॑ स॒त्रस्य॒ भव॑ति॒ भव॑ति स॒त्रस्य र्‌द्ध्यै᳚ । \newline
34. स॒त्रस्य र्‌द्ध्या॒ ऋद्ध्यै॑ स॒त्रस्य॑ स॒त्रस्य र्‌द्ध्या॒ अथो॒ अथो॒ ऋद्ध्यै॑ स॒त्रस्य॑ स॒त्रस्य र्‌द्ध्या॒ अथो᳚ । \newline
35. ऋद्ध्या॒ अथो॒ अथो॒ ऋद्ध्या॒ ऋद्ध्या॒ अथो॒ यद् यदथो॒ ऋद्ध्या॒ ऋद्ध्या॒ अथो॒ यत् । \newline
36. अथो॒ यद् यदथो॒ अथो॒ यदे॒वैव यदथो॒ अथो॒ यदे॒व । \newline
37. अथो॒ इत्यथो᳚ । \newline
38. यदे॒वैव यद् यदे॒व पूर्वे॑षु॒ पूर्वे᳚ष्वे॒व यद् यदे॒व पूर्वे॑षु । \newline
39. ए॒व पूर्वे॑षु॒ पूर्वे᳚ ष्वे॒वैव पूर्वे॒ ष्वह॒ स्स्वह॑स्सु॒ पूर्वे᳚ ष्वे॒वैव पूर्वे॒ ष्वह॑स्सु । \newline
40. पूर्वे॒ ष्वह॒ स्स्वह॑स्सु॒ पूर्वे॑षु॒ पूर्वे॒ ष्वह॑स्सु॒ विलो॑म॒ विलो॒मा ह॑स्सु॒ पूर्वे॑षु॒ पूर्वे॒ ष्वह॑स्सु॒ विलो॑म । \newline
41. अह॑स्सु॒ विलो॑म॒ विलो॒मा ह॒ स्स्वह॑स्सु॒ विलो॑म क्रि॒यते᳚ क्रि॒यते॒ विलो॒मा ह॒ स्स्वह॑स्सु॒ विलो॑म क्रि॒यते᳚ । \newline
42. अह॒स्स्वित्यहः॑ - सु॒ । \newline
43. विलो॑म क्रि॒यते᳚ क्रि॒यते॒ विलो॑म॒ विलो॑म क्रि॒यते॒ तस्य॒ तस्य॑ क्रि॒यते॒ विलो॑म॒ विलो॑म क्रि॒यते॒ तस्य॑ । \newline
44. विलो॒मेति॒ वि - लो॒म॒ । \newline
45. क्रि॒यते॒ तस्य॒ तस्य॑ क्रि॒यते᳚ क्रि॒यते॒ तस्यै॒वैव तस्य॑ क्रि॒यते᳚ क्रि॒यते॒ तस्यै॒व । \newline
46. तस्यै॒वैव तस्य॒ तस्यै॒ वैषैषैव तस्य॒ तस्यै॒ वैषा । \newline
47. ए॒वै षैषै वैवैषा शान्तिः॒ शान्ति॑ रे॒षैवैवैषा शान्तिः॑ । \newline
48. ए॒षा शान्तिः॒ शान्ति॑ रे॒षैषा शान्ति॑र् द्व्यनी॒का द्व्य॑नी॒काः शान्ति॑ रे॒षैषा शान्ति॑र् द्व्यनी॒काः । \newline
49. शान्ति॑र् द्व्यनी॒का द्व्य॑नी॒काः शान्तिः॒ शान्ति॑र् द्व्यनी॒का वै वै द्व्य॑नी॒काः शान्तिः॒ शान्ति॑र् द्व्यनी॒का वै । \newline
50. द्व्य॒नी॒का वै वै द्व्य॑नी॒का द्व्य॑नी॒का वा ए॒ता ए॒ता वै द्व्य॑नी॒का द्व्य॑नी॒का वा ए॒ताः । \newline
51. द्व्य॒नी॒का इति॑ द्वि - अ॒नी॒काः । \newline
52. वा ए॒ता ए॒ता वै वा ए॒ता रात्र॑यो॒ रात्र॑य ए॒ता वै वा ए॒ता रात्र॑यः । \newline
53. ए॒ता रात्र॑यो॒ रात्र॑य ए॒ता ए॒ता रात्र॑यो॒ यज॑माना॒ यज॑माना॒ रात्र॑य ए॒ता ए॒ता रात्र॑यो॒ यज॑मानाः । \newline
54. रात्र॑यो॒ यज॑माना॒ यज॑माना॒ रात्र॑यो॒ रात्र॑यो॒ यज॑माना विश्व॒जिद् वि॑श्व॒जिद् यज॑माना॒ रात्र॑यो॒ रात्र॑यो॒ यज॑माना विश्व॒जित् । \newline
55. यज॑माना विश्व॒जिद् वि॑श्व॒जिद् यज॑माना॒ यज॑माना विश्व॒जिथ् स॒ह स॒ह वि॑श्व॒जिद् यज॑माना॒ यज॑माना विश्व॒जिथ् स॒ह । \newline
56. वि॒श्व॒जिथ् स॒ह स॒ह वि॑श्व॒जिद् वि॑श्व॒जिथ् स॒हा ति॑रा॒त्रेणा॑ तिरा॒त्रेण॑ स॒ह वि॑श्व॒जिद् वि॑श्व॒जिथ् स॒हा ति॑रा॒त्रेण॑ । \newline
57. वि॒श्व॒जिदिति॑ विश्व - जित् । \newline
58. स॒हा ति॑रा॒त्रेणा॑ तिरा॒त्रेण॑ स॒ह स॒हा ति॑रा॒त्रेण॒ पूर्वाः॒ पूर्वा॑ अतिरा॒त्रेण॑ स॒ह स॒हा
ति॑रा॒त्रेण॒ पूर्वाः᳚ । \newline
59. अ॒ति॒रा॒त्रेण॒ पूर्वाः॒ पूर्वा॑ अतिरा॒त्रेणा॑ तिरा॒त्रेण॒ पूर्वा॒ ष्षोड॑श॒ षोड॑श॒ पूर्वा॑ अतिरा॒त्रेणा॑ तिरा॒त्रेण॒ पूर्वा॒ ष्षोड॑श । \newline
60. अ॒ति॒रा॒त्रेणेत्य॑ति - रा॒त्रेण॑ । \newline
61. पूर्वा॒ ष्षोड॑श॒ षोड॑श॒ पूर्वाः॒ पूर्वा॒ ष्षोड॑श स॒ह स॒ह षोड॑श॒ पूर्वाः॒ पूर्वा॒ ष्षोड॑श स॒ह । \newline
62. षोड॑श स॒ह स॒ह षोड॑श॒ षोड॑श स॒हा ति॑रा॒त्रेणा॑ तिरा॒त्रेण॑ स॒ह षोड॑श॒ षोड॑श स॒हा ति॑रा॒त्रेण॑ । \newline
63. स॒हा ति॑रा॒त्रेणा॑ तिरा॒त्रेण॑ स॒ह स॒हा ति॑रा॒त्रे णोत्त॑रा॒ उत्त॑रा अतिरा॒त्रेण॑ स॒ह स॒हा ति॑रा॒त्रे णोत्त॑राः । \newline
64. अ॒ति॒रा॒त्रे णोत्त॑रा॒ उत्त॑रा अतिरा॒त्रेणा॑ तिरा॒त्रे णोत्त॑रा॒ ष्षोड॑श॒ षोड॒ शोत्त॑रा अतिरा॒त्रेणा॑ तिरा॒त्रे णोत्त॑रा॒ ष्षोड॑श । \newline
65. अ॒ति॒रा॒त्रेणेत्य॑ति - रा॒त्रेण॑ । \newline
66. उत्त॑रा॒ ष्षोड॑श॒ षोड॒ शोत्त॑रा॒ उत्त॑रा॒ ष्षोड॑श॒ ये ये षोड॒ शोत्त॑रा॒ उत्त॑रा॒ ष्षोड॑श॒ ये । \newline
67. उत्त॑रा॒ इत्युत् - त॒राः॒ । \newline
68. षोड॑श॒ ये ये षोड॑श॒ षोड॑श॒ य ए॒व मे॒वं ॅये षोड॑श॒ षोड॑श॒ य ए॒वम् । \newline
69. य ए॒व मे॒वं ॅये य ए॒वं ॅवि॒द्वाꣳसो॑ वि॒द्वाꣳस॑ ए॒वं ॅये य ए॒वं ॅवि॒द्वाꣳसः॑ । \newline
70. ए॒वं ॅवि॒द्वाꣳसो॑ वि॒द्वाꣳस॑ ए॒व मे॒वं ॅवि॒द्वाꣳस॑ स्त्रयस्त्रिꣳशद॒हम् त्र॑यस्त्रिꣳशद॒हं ॅवि॒द्वाꣳस॑ ए॒व मे॒वं ॅवि॒द्वाꣳस॑ स्त्रयस्त्रिꣳशद॒हम् । \newline
71. वि॒द्वाꣳस॑ स्त्रयस्त्रिꣳशद॒हम् त्र॑यस्त्रिꣳशद॒हं ॅवि॒द्वाꣳसो॑ वि॒द्वाꣳस॑ स्त्रयस्त्रिꣳशद॒ह मास॑त॒ आस॑ते त्रयस्त्रिꣳशद॒हं ॅवि॒द्वाꣳसो॑ वि॒द्वाꣳस॑ स्त्रयस्त्रिꣳशद॒ह मास॑ते । \newline
72. त्र॒य॒स्त्रिꣳ॒॒श॒द॒ह मास॑त॒ आस॑ते त्रयस्त्रिꣳशद॒हम् त्र॑यस्त्रिꣳशद॒ह मास॑त॒ आ ऽऽस॑ते त्रयस्त्रिꣳशद॒हम् त्र॑यस्त्रिꣳशद॒ह मास॑त॒ आ । \newline
73. त्र॒य॒स्त्रिꣳ॒॒श॒द॒हमिति॑ त्रयस्त्रिꣳशत् - अ॒हम् । \newline
74. आस॑त॒ आ ऽऽस॑त॒ आस॑त॒ ऐषा॑ मेषा॒ मा ऽऽस॑त॒ आस॑त॒ ऐषा᳚म् । \newline
75. ऐषा॑ मेषा॒ मैषा᳚म् द्व्यनी॒का द्व्य॑नी॒कैषा॒ मैषा᳚म् द्व्यनी॒का । \newline
76. ए॒षा॒म् द्व्य॒नी॒का द्व्य॑नी॒कैषा॑ मेषाम् द्व्यनी॒का प्र॒जा प्र॒जा द्व्य॑नी॒कैषा॑ मेषाम् द्व्यनी॒का प्र॒जा । \newline
77. द्व्य॒नी॒का प्र॒जा प्र॒जा द्व्य॑नी॒का द्व्य॑नी॒का प्र॒जा जा॑यते जायते प्र॒जा द्व्य॑नी॒का द्व्य॑नी॒का प्र॒जा जा॑यते । \newline
78. द्व्य॒नी॒केति॑ द्वि - अ॒नी॒का । \newline
79. प्र॒जा जा॑यते जायते प्र॒जा प्र॒जा जा॑यते ऽतिरा॒त्रा व॑तिरा॒त्रौ जा॑यते प्र॒जा प्र॒जा जा॑यते ऽतिरा॒त्रौ । \newline
80. प्र॒जेति॑ प्र - जा । \newline
81. जा॒य॒ते॒ ऽति॒रा॒त्रा व॑तिरा॒त्रौ जा॑यते जायते ऽतिरा॒त्रा व॒भितो॒ ऽभितो॑ ऽतिरा॒त्रौ जा॑यते जायते ऽतिरा॒त्रा व॒भितः॑ । \newline
82. अ॒ति॒रा॒त्रा व॒भितो॒ ऽभितो॑ ऽतिरा॒त्रा व॑तिरा॒त्रा व॒भितो॑ भवतो भवतो॒ ऽभितो॑ ऽतिरा॒त्रा व॑तिरा॒त्रा व॒भितो॑ भवतः । \newline
83. अ॒ति॒रा॒त्रावित्य॑ति - रा॒त्रौ । \newline
84. अ॒भितो॑ भवतो भवतो॒ ऽभितो॒ ऽभितो॑ भवतः॒ परि॑गृहीत्यै॒ परि॑गृहीत्यै भवतो॒ ऽभितो॒ ऽभितो॑ भवतः॒ परि॑गृहीत्यै । \newline
85. भ॒व॒तः॒ परि॑गृहीत्यै॒ परि॑गृहीत्यै भवतो भवतः॒ परि॑गृहीत्यै । \newline
86. परि॑गृहीत्या॒ इति॒ परि॑ - गृ॒ही॒त्यै॒ । \newline
\pagebreak
\markright{ TS 7.4.6.1  \hfill https://www.vedavms.in \hfill}

\section{ TS 7.4.6.1 }

\textbf{TS 7.4.6.1 } \newline
\textbf{Samhita Paata} \newline

आ॒दि॒त्या अ॑कामयन्त सुव॒र्गं ॅलो॒कमि॑या॒मेति॒ ते सु॑व॒र्गं ॅलो॒कं न प्राजा॑न॒न्न सु॑व॒र्गं ॅलो॒कमा॑य॒न् त ए॒तꣳ ष॑ट्त्रिꣳशद्रा॒त्र-म॑पश्य॒न् तमाऽह॑र॒न् तेना॑यजन्त॒ ततो॒ वै ते सु॑व॒र्गं ॅलो॒कं प्राजा॑नन्थ् सुव॒र्गं ॅलो॒कमा॑य॒न्॒. य ए॒वं ॅवि॒द्वाꣳसः॑ षट्त्रिꣳशद् रा॒त्रमास॑ते सुव॒र्गमे॒व लोकं प्र जा॑नन्ति सुव॒र्गं ॅलो॒कं ॅय॑न्ति॒ ज्योति॑रतिरा॒त्रो - [  ] \newline

\textbf{Pada Paata} \newline

आ॒दि॒त्याः । अ॒का॒म॒य॒न्त॒ । सु॒व॒र्गमिति॑ सुवः - गम् । लो॒कम् । इ॒या॒म॒ । इति॑ । ते । सु॒व॒र्गमिति॑ सुवः - गम् । लो॒कम् । न । प्रेति॑ । अ॒जा॒न॒न्न् । न । सु॒व॒र्गमिति॑ सुवः-गम् । लो॒कम् । आ॒य॒न्न् । ते । ए॒तम् । ष॒ट्त्रिꣳ॒॒श॒द्रा॒त्रमिति॑ षट्त्रिꣳशत् - रा॒त्रम् । अ॒प॒श्य॒न्न् । तम् । एति॑ । अ॒ह॒र॒न्न् । तेन॑ । अ॒य॒ज॒न्त॒ । ततः॑ । वै । ते । सु॒व॒र्गमिति॑ सुवः - गम् । लो॒कम् । प्रेति॑ । अ॒जा॒न॒न्न् । सु॒व॒र्गमिति॑ सुवः-गम् । लो॒कम् । आ॒य॒न्न् । ये । ए॒वम् । वि॒द्वाꣳसः॑ । ष॒ट्त्रिꣳ॒॒श॒द्रा॒त्रमिति॑ षट्त्रिꣳशत् - रा॒त्रम् । आस॑ते । सु॒व॒र्गमिति॑ सुवः - गम् । ए॒व । लो॒कम् । प्रेति॑ । जा॒न॒न्ति॒ । सु॒व॒र्गमिति॑ सुवः - गम् । लो॒कम् । य॒न्ति॒ । ज्योतिः॑ । अ॒ति॒रा॒त्र इत्य॑ति - रा॒त्रः ।  \newline


\textbf{Krama Paata} \newline

आ॒दि॒त्या अ॑कामयन्त । अ॒का॒म॒य॒न्त॒ सु॒व॒र्गम् । सु॒व॒र्गम् ॅलो॒कम् । सु॒व॒र्गमिति॑ सुवः - गम् । लो॒कमि॑याम । इ॒या॒मेति॑ । इति॒ ते । ते सु॑व॒र्गम् । सु॒व॒र्गम् ॅलो॒कम् । सु॒व॒र्गमिति॑ सुवः - गम् । लो॒कम् न । न प्र । प्राजा॑नन्न् । अ॒जा॒न॒न् न । न सु॑व॒र्गम् । सु॒व॒र्गम् ॅलो॒कम् । सु॒व॒र्गमिति॑ सुवः - गम् । लो॒कमा॑यन्न् । आ॒य॒न् ते । त ए॒तम् । ए॒तꣳ ष॑ट्त्रिꣳशद्‍रा॒त्रम् । ष॒ट्त्रिꣳ॒॒श॒द्‍रा॒त्रम॑पश्यन्न् । ष॒ट्त्रिꣳ॒॒श॒द्‍रा॒त्रमिति॑ षट्त्रिꣳशत् - रा॒त्रम् । अ॒प॒श्य॒न् तम् । तमा । आऽह॑रन्न् । अ॒ह॒र॒न् तेन॑ । तेना॑यजन्त । अ॒य॒ज॒न्त॒ ततः॑ । ततो॒ वै । वै ते । ते सु॑व॒र्गम् । सु॒व॒र्गम् ॅलो॒कम् । सु॒व॒र्गमिति॑ सुवः - गम् । लो॒कम् प्र । प्राजा॑नन्न् । अ॒जा॒न॒न्थ् सु॒व॒र्गम् । सु॒व॒र्गम् ॅलो॒कम् । सु॒व॒र्गमिति॑ सुवः - गम् । लो॒कमा॑यन्न् । आ॒य॒न्॒. ये । य ए॒वम् । ए॒वम् ॅवि॒द्वाꣳसः॑ । वि॒द्वाꣳसः॑ षट्त्रिꣳशद्‍रा॒त्रम् । ष॒ट्त्रिꣳ॒॒श॒द्‍रा॒त्रमास॑ते । ष॒ट्त्रिꣳ॒॒श॒द्‍रा॒त्रमिति॑ षट्त्रिꣳशत् - रा॒त्रम् । आस॑ते सुव॒र्गम् । सु॒व॒र्गमे॒व । सु॒व॒र्गमिति॑ सुवः - गम् । ए॒व लो॒कम् । लो॒कम् प्र । प्र जा॑नन्ति । जा॒न॒न्ति॒ सु॒व॒र्गम् । सु॒व॒र्गम् ॅलो॒कम् । सु॒व॒र्गमिति॑ सुवः - गम् । लो॒कम् ॅय॑न्ति । य॒न्ति॒ ज्योतिः॑ । ज्योति॑रतिरा॒त्रः । अ॒ति॒रा॒त्रो भ॑वति । अ॒ति॒रा॒त्र इत्य॑ति - रा॒त्रः \newline

\textbf{Jatai Paata} \newline

1. आ॒दि॒त्या अ॑कामयन्ता कामयन्ता दि॒त्या आ॑दि॒त्या अ॑कामयन्त । \newline
2. अ॒का॒म॒य॒न्त॒ सु॒व॒र्गꣳ सु॑व॒र्ग म॑कामयन्ता कामयन्त सुव॒र्गम् । \newline
3. सु॒व॒र्गम् ॅलो॒कम् ॅलो॒कꣳ सु॑व॒र्गꣳ सु॑व॒र्गम् ॅलो॒कम् । \newline
4. सु॒व॒र्गमिति॑ सुवः - गम् । \newline
5. लो॒क मि॑या मेयाम लो॒कम् ॅलो॒क मि॑याम । \newline
6. इ॒या॒ मेतीती॑ यामे या॒मेति॑ । \newline
7. इति॒ ते त इतीति॒ ते । \newline
8. ते सु॑व॒र्गꣳ सु॑व॒र्गम् ते ते सु॑व॒र्गम् । \newline
9. सु॒व॒र्गम् ॅलो॒कम् ॅलो॒कꣳ सु॑व॒र्गꣳ सु॑व॒र्गम् ॅलो॒कम् । \newline
10. सु॒व॒र्गमिति॑ सुवः - गम् । \newline
11. लो॒कन् न न लो॒कम् ॅलो॒कन् न । \newline
12. न प्र प्र ण न प्र । \newline
13. प्रा जा॑नन् नजान॒न् प्र प्रा जा॑नन्न् । \newline
14. अ॒जा॒न॒न् न नाजा॑नन् नजान॒न् न । \newline
15. न सु॑व॒र्गꣳ सु॑व॒र्गन् न न सु॑व॒र्गम् । \newline
16. सु॒व॒र्गम् ॅलो॒कम् ॅलो॒कꣳ सु॑व॒र्गꣳ सु॑व॒र्गम् ॅलो॒कम् । \newline
17. सु॒व॒र्गमिति॑ सुवः - गम् । \newline
18. लो॒क मा॑यन् नायन् ॅलो॒कम् ॅलो॒क मा॑यन्न् । \newline
19. आ॒य॒न् ते त आ॑यन् नाय॒न् ते । \newline
20. त ए॒त मे॒तम् ते त ए॒तम् । \newline
21. ए॒तꣳ ष॑ट्त्रिꣳशद्रा॒त्रꣳ ष॑ट्त्रिꣳशद्रा॒त्र मे॒त मे॒तꣳ ष॑ट्त्रिꣳशद्रा॒त्रम् । \newline
22. ष॒ट्त्रिꣳ॒॒श॒द्रा॒त्र म॑पश्यन् नपश्यन् थ्षट्त्रिꣳशद्रा॒त्रꣳ ष॑ट्त्रिꣳशद्रा॒त्र म॑पश्यन्न् । \newline
23. ष॒ट्त्रिꣳ॒॒श॒द्रा॒त्रमिति॑ षट्त्रिꣳशत् - रा॒त्रम् । \newline
24. अ॒प॒श्य॒न् तम् त म॑पश्यन् नपश्य॒न् तम् । \newline
25. त मा तम् त मा । \newline
26. आ ऽह॑रन् नहर॒न् ना ऽह॑रन्न् । \newline
27. अ॒ह॒र॒न् तेन॒ तेना॑ हरन् नहर॒न् तेन॑ । \newline
28. तेना॑ यजन्ता यजन्त॒ तेन॒ तेना॑ यजन्त । \newline
29. अ॒य॒ज॒न्त॒ तत॒ स्ततो॑ ऽयजन्ता यजन्त॒ ततः॑ । \newline
30. ततो॒ वै वै तत॒ स्ततो॒ वै । \newline
31. वै ते ते वै वै ते । \newline
32. ते सु॑व॒र्गꣳ सु॑व॒र्गम् ते ते सु॑व॒र्गम् । \newline
33. सु॒व॒र्गम् ॅलो॒कम् ॅलो॒कꣳ सु॑व॒र्गꣳ सु॑व॒र्गम् ॅलो॒कम् । \newline
34. सु॒व॒र्गमिति॑ सुवः - गम् । \newline
35. लो॒कम् प्र प्र लो॒कम् ॅलो॒कम् प्र । \newline
36. प्रा जा॑नन् नजान॒न् प्र प्रा जा॑नन्न् । \newline
37. अ॒जा॒न॒न् थ्सु॒व॒र्गꣳ सु॑व॒र्ग म॑जानन् नजानन् थ्सुव॒र्गम् । \newline
38. सु॒व॒र्गम् ॅलो॒कम् ॅलो॒कꣳ सु॑व॒र्गꣳ सु॑व॒र्गम् ॅलो॒कम् । \newline
39. सु॒व॒र्गमिति॑ सुवः - गम् । \newline
40. लो॒क मा॑यन् नायन् ॅलो॒कम् ॅलो॒क मा॑यन्न् । \newline
41. आ॒य॒न्॒. ये य आ॑यन् नाय॒न्॒. ये । \newline
42. य ए॒व मे॒वं ॅये य ए॒वम् । \newline
43. ए॒वं ॅवि॒द्वाꣳसो॑ वि॒द्वाꣳस॑ ए॒व मे॒वं ॅवि॒द्वाꣳसः॑ । \newline
44. वि॒द्वाꣳस॑ ष्षट्त्रिꣳशद्रा॒त्रꣳ ष॑ट्त्रिꣳशद्रा॒त्रं ॅवि॒द्वाꣳसो॑ वि॒द्वाꣳस॑ 
ष्षट्त्रिꣳशद्रा॒त्रम् । \newline
45. ष॒ट्त्रिꣳ॒॒श॒द्रा॒त्र मास॑त॒ आस॑ते षट्त्रिꣳशद्रा॒त्रꣳ ष॑ट्त्रिꣳशद्रा॒त्र मास॑ते । \newline
46. ष॒ट्त्रिꣳ॒॒श॒द्रा॒त्रमिति॑ षट्त्रिꣳशत् - रा॒त्रम् । \newline
47. आस॑ते सुव॒र्गꣳ सु॑व॒र्ग मास॑त॒ आस॑ते सुव॒र्गम् । \newline
48. सु॒व॒र्ग मे॒वैव सु॑व॒र्गꣳ सु॑व॒र्ग मे॒व । \newline
49. सु॒व॒र्गमिति॑ सुवः - गम् । \newline
50. ए॒व लो॒कम् ॅलो॒क मे॒वैव लो॒कम् । \newline
51. लो॒कम् प्र प्र लो॒कम् ॅलो॒कम् प्र । \newline
52. प्र जा॑नन्ति जानन्ति॒ प्र प्र जा॑नन्ति । \newline
53. जा॒न॒न्ति॒ सु॒व॒र्गꣳ सु॑व॒र्गम् जा॑नन्ति जानन्ति सुव॒र्गम् । \newline
54. सु॒व॒र्गम् ॅलो॒कम् ॅलो॒कꣳ सु॑व॒र्गꣳ सु॑व॒र्गम् ॅलो॒कम् । \newline
55. सु॒व॒र्गमिति॑ सुवः - गम् । \newline
56. लो॒कं ॅय॑न्ति यन्ति लो॒कम् ॅलो॒कं ॅय॑न्ति । \newline
57. य॒न्ति॒ ज्योति॒र् ज्योति॑र् यन्ति यन्ति॒ ज्योतिः॑ । \newline
58. ज्योति॑ रतिरा॒त्रो॑ ऽतिरा॒त्रो ज्योति॒र् ज्योति॑ रतिरा॒त्रः । \newline
59. अ॒ति॒रा॒त्रो भ॑वति भव त्यतिरा॒त्रो॑ ऽतिरा॒त्रो भ॑वति । \newline
60. अ॒ति॒रा॒त्र इत्य॑ति - रा॒त्रः । \newline

\textbf{Ghana Paata } \newline

1. आ॒दि॒त्या अ॑कामयन्ता कामयन्ता दि॒त्या आ॑दि॒त्या अ॑कामयन्त सुव॒र्गꣳ सु॑व॒र्ग म॑कामयन्ता दि॒त्या आ॑दि॒त्या अ॑कामयन्त सुव॒र्गम् । \newline
2. अ॒का॒म॒य॒न्त॒ सु॒व॒र्गꣳ सु॑व॒र्ग म॑कामयन्ता कामयन्त सुव॒र्गम् ॅलो॒कम् ॅलो॒कꣳ सु॑व॒र्ग म॑कामयन्ता कामयन्त सुव॒र्गम् ॅलो॒कम् । \newline
3. सु॒व॒र्गम् ॅलो॒कम् ॅलो॒कꣳ सु॑व॒र्गꣳ सु॑व॒र्गम् ॅलो॒क मि॑या मेयाम लो॒कꣳ सु॑व॒र्गꣳ सु॑व॒र्गम् ॅलो॒क मि॑याम । \newline
4. सु॒व॒र्गमिति॑ सुवः - गम् । \newline
5. लो॒क मि॑यामे याम लो॒कम् ॅलो॒क मि॑या ॒मेतीती॑ याम लो॒कम् ॅलो॒क मि॑या॒मेति॑ । \newline
6. इ॒या॒मेतीती॑ यामे या॒मेति॒ ते त इती॑यामे या॒मेति॒ ते । \newline
7. इति॒ ते त इतीति॒ ते सु॑व॒र्गꣳ सु॑व॒र्गम् त इतीति॒ ते सु॑व॒र्गम् । \newline
8. ते सु॑व॒र्गꣳ सु॑व॒र्गम् ते ते सु॑व॒र्गम् ॅलो॒कम् ॅलो॒कꣳ सु॑व॒र्गम् ते ते सु॑व॒र्गम् ॅलो॒कम् । \newline
9. सु॒व॒र्गम् ॅलो॒कम् ॅलो॒कꣳ सु॑व॒र्गꣳ सु॑व॒र्गम् ॅलो॒कन् न न लो॒कꣳ सु॑व॒र्गꣳ सु॑व॒र्गम् ॅलो॒कन् न । \newline
10. सु॒व॒र्गमिति॑ सुवः - गम् । \newline
11. लो॒कन् न न लो॒कम् ॅलो॒कन् न प्र प्र ण लो॒कम् ॅलो॒कन् न प्र । \newline
12. न प्र प्र ण न प्राजा॑नन् नजान॒न् प्र ण न प्राजा॑नन्न् । \newline
13. प्राजा॑नन् नजान॒न् प्र प्राजा॑न॒न् न नाजा॑न॒न् प्र प्राजा॑न॒न् न । \newline
14. अ॒जा॒न॒न् न नाजा॑नन् नजान॒न् न सु॑व॒र्गꣳ सु॑व॒र्गन् नाजा॑नन् नजान॒न् न सु॑व॒र्गम् । \newline
15. न सु॑व॒र्गꣳ सु॑व॒र्गन् न न सु॑व॒र्गम् ॅलो॒कम् ॅलो॒कꣳ सु॑व॒र्गन् न न सु॑व॒र्गम् ॅलो॒कम् । \newline
16. सु॒व॒र्गम् ॅलो॒कम् ॅलो॒कꣳ सु॑व॒र्गꣳ सु॑व॒र्गम् ॅलो॒क मा॑यन् नायन् ॅलो॒कꣳ सु॑व॒र्गꣳ सु॑व॒र्गम् ॅलो॒क मा॑यन्न् । \newline
17. सु॒व॒र्गमिति॑ सुवः - गम् । \newline
18. लो॒क मा॑यन् नायन् ॅलो॒कम् ॅलो॒क मा॑य॒न् ते त आ॑यन् ॅलो॒कम् ॅलो॒क मा॑य॒न् ते । \newline
19. आ॒य॒न् ते त आ॑यन् नाय॒न् त ए॒त मे॒तम् त आ॑यन् नाय॒न् त ए॒तम् । \newline
20. त ए॒त मे॒तम् ते त ए॒तꣳ ष॑ट्त्रिꣳशद्रा॒त्रꣳ ष॑ट्त्रिꣳशद्रा॒त्र मे॒तम् ते त ए॒तꣳ ष॑ट्त्रिꣳशद्रा॒त्रम् । \newline
21. ए॒तꣳ ष॑ट्त्रिꣳशद्रा॒त्रꣳ ष॑ट्त्रिꣳशद्रा॒त्र मे॒त मे॒तꣳ ष॑ट्त्रिꣳशद्रा॒त्र म॑पश्यन् नपश्यन् थ्षट्त्रिꣳशद्रा॒त्र मे॒त मे॒तꣳ ष॑ट्त्रिꣳशद्रा॒त्र म॑पश्यन्न् । \newline
22. ष॒ट्त्रिꣳ॒॒श॒द्रा॒त्र म॑पश्यन् नपश्यन् थ्षट्त्रिꣳशद्रा॒त्रꣳ ष॑ट्त्रिꣳशद्रा॒त्र म॑पश्य॒न् तम् त म॑पश्यन् थ्षट्त्रिꣳशद्रा॒त्रꣳ ष॑ट्त्रिꣳशद्रा॒त्र म॑पश्य॒न् तम् । \newline
23. ष॒ट्त्रिꣳ॒॒श॒द्रा॒त्रमिति॑ षट्त्रिꣳशत् - रा॒त्रम् । \newline
24. अ॒प॒श्य॒न् तम् त म॑पश्यन् नपश्य॒न् त मा त म॑पश्यन् नपश्य॒न् त मा । \newline
25. त मा तम् त मा ऽह॑रन् नहर॒न् ना तम् त मा ऽह॑रन्न् । \newline
26. आ ऽह॑रन् नहर॒न् ना ऽह॑र॒न् तेन॒ तेना॑ हर॒न् ना ऽह॑र॒न् तेन॑ । \newline
27. अ॒ह॒र॒न् तेन॒ तेना॑ हरन् नहर॒न् तेना॑ यजन्ता यजन्त॒ तेना॑ हरन् नहर॒न् तेना॑ यजन्त । \newline
28. तेना॑ यजन्ता यजन्त॒ तेन॒ तेना॑ यजन्त॒ तत॒ स्ततो॑ ऽयजन्त॒ तेन॒ तेना॑ यजन्त॒ ततः॑ । \newline
29. अ॒य॒ज॒न्त॒ तत॒ स्ततो॑ ऽयजन्ता यजन्त॒ ततो॒ वै वै ततो॑ ऽयजन्ता यजन्त॒ ततो॒ वै । \newline
30. ततो॒ वै वै तत॒ स्ततो॒ वै ते ते वै तत॒ स्ततो॒ वै ते । \newline
31. वै ते ते वै वै ते सु॑व॒र्गꣳ सु॑व॒र्गम् ते वै वै ते सु॑व॒र्गम् । \newline
32. ते सु॑व॒र्गꣳ सु॑व॒र्गम् ते ते सु॑व॒र्गम् ॅलो॒कम् ॅलो॒कꣳ सु॑व॒र्गम् ते ते सु॑व॒र्गम् ॅलो॒कम् । \newline
33. सु॒व॒र्गम् ॅलो॒कम् ॅलो॒कꣳ सु॑व॒र्गꣳ सु॑व॒र्गम् ॅलो॒कम् प्र प्र लो॒कꣳ सु॑व॒र्गꣳ सु॑व॒र्गम् ॅलो॒कम् प्र । \newline
34. सु॒व॒र्गमिति॑ सुवः - गम् । \newline
35. लो॒कम् प्र प्र लो॒कम् ॅलो॒कम् प्रा जा॑नन् नजान॒न् प्र लो॒कम् ॅलो॒कम् प्रा जा॑नन्न् । \newline
36. प्रा जा॑नन् नजान॒न् प्र प्रा जा॑नन् थ्सुव॒र्गꣳ सु॑व॒र्ग म॑जान॒न् प्र प्रा जा॑नन् थ्सुव॒र्गम् । \newline
37. अ॒जा॒न॒न् थ्सु॒व॒र्गꣳ सु॑व॒र्ग म॑जानन् नजानन् थ्सुव॒र्गम् ॅलो॒कम् ॅलो॒कꣳ सु॑व॒र्ग म॑जानन् नजानन् थ्सुव॒र्गम् ॅलो॒कम् । \newline
38. सु॒व॒र्गम् ॅलो॒कम् ॅलो॒कꣳ सु॑व॒र्गꣳ सु॑व॒र्गम् ॅलो॒क मा॑यन् नायन् ॅलो॒कꣳ सु॑व॒र्गꣳ सु॑व॒र्गम् ॅलो॒क मा॑यन्न् । \newline
39. सु॒व॒र्गमिति॑ सुवः - गम् । \newline
40. लो॒क मा॑यन् नायन् ॅलो॒कम् ॅलो॒क मा॑य॒न्॒. ये य आ॑यन् ॅलो॒कम् ॅलो॒क मा॑य॒न्॒. ये । \newline
41. आ॒य॒न्॒. ये य आ॑यन् नाय॒न्॒. य ए॒व मे॒वं ॅय आ॑यन् नाय॒न्॒. य ए॒वम् । \newline
42. य ए॒व मे॒वं ॅये य ए॒वं ॅवि॒द्वाꣳसो॑ वि॒द्वाꣳस॑ ए॒वं ॅये य ए॒वं ॅवि॒द्वाꣳसः॑ । \newline
43. ए॒वं ॅवि॒द्वाꣳसो॑ वि॒द्वाꣳस॑ ए॒व मे॒वं ॅवि॒द्वाꣳस॑ ष्षट्त्रिꣳशद्रा॒त्रꣳ ष॑ट्त्रिꣳशद्रा॒त्रं ॅवि॒द्वाꣳस॑ ए॒व मे॒वं ॅवि॒द्वाꣳस॑ ष्षट्त्रिꣳशद्रा॒त्रम् । \newline
44. वि॒द्वाꣳस॑ ष्षट्त्रिꣳशद्रा॒त्रꣳ ष॑ट्त्रिꣳशद्रा॒त्रं ॅवि॒द्वाꣳसो॑ वि॒द्वाꣳस॑ ष्षट्त्रिꣳशद्रा॒त्र मास॑त॒ आस॑ते षट्त्रिꣳशद्रा॒त्रं ॅवि॒द्वाꣳसो॑ वि॒द्वाꣳस॑ ष्षट्त्रिꣳशद्रा॒त्र मास॑ते । \newline
45. ष॒ट्त्रिꣳ॒॒श॒द्रा॒त्र मास॑त॒ आस॑ते षट्त्रिꣳशद्रा॒त्रꣳ ष॑ट्त्रिꣳशद्रा॒त्र मास॑ते सुव॒र्गꣳ सु॑व॒र्ग मास॑ते षट्त्रिꣳशद्रा॒त्रꣳ ष॑ट्त्रिꣳशद्रा॒त्र मास॑ते सुव॒र्गम् । \newline
46. ष॒ट्त्रिꣳ॒॒श॒द्रा॒त्रमिति॑ षट्त्रिꣳशत् - रा॒त्रम् । \newline
47. आस॑ते सुव॒र्गꣳ सु॑व॒र्ग मास॑त॒ आस॑ते सुव॒र्ग मे॒वैव सु॑व॒र्ग मास॑त॒ आस॑ते सुव॒र्ग मे॒व । \newline
48. सु॒व॒र्ग मे॒वैव सु॑व॒र्गꣳ सु॑व॒र्ग मे॒व लो॒कम् ॅलो॒क मे॒व सु॑व॒र्गꣳ सु॑व॒र्ग मे॒व लो॒कम् । \newline
49. सु॒व॒र्गमिति॑ सुवः - गम् । \newline
50. ए॒व लो॒कम् ॅलो॒क मे॒वैव लो॒कम् प्र प्र लो॒क मे॒वैव लो॒कम् प्र । \newline
51. लो॒कम् प्र प्र लो॒कम् ॅलो॒कम् प्र जा॑नन्ति जानन्ति॒ प्र लो॒कम् ॅलो॒कम् प्र जा॑नन्ति । \newline
52. प्र जा॑नन्ति जानन्ति॒ प्र प्र जा॑नन्ति सुव॒र्गꣳ सु॑व॒र्गम् जा॑नन्ति॒ प्र प्र जा॑नन्ति सुव॒र्गम् । \newline
53. जा॒न॒न्ति॒ सु॒व॒र्गꣳ सु॑व॒र्गम् जा॑नन्ति जानन्ति सुव॒र्गम् ॅलो॒कम् ॅलो॒कꣳ सु॑व॒र्गम् जा॑नन्ति जानन्ति सुव॒र्गम् ॅलो॒कम् । \newline
54. सु॒व॒र्गम् ॅलो॒कम् ॅलो॒कꣳ सु॑व॒र्गꣳ सु॑व॒र्गम् ॅलो॒कं ॅय॑न्ति यन्ति लो॒कꣳ सु॑व॒र्गꣳ सु॑व॒र्गम् ॅलो॒कं ॅय॑न्ति । \newline
55. सु॒व॒र्गमिति॑ सुवः - गम् । \newline
56. लो॒कं ॅय॑न्ति यन्ति लो॒कम् ॅलो॒कं ॅय॑न्ति॒ ज्योति॒र् ज्योति॑र् यन्ति लो॒कम् ॅलो॒कं ॅय॑न्ति॒ ज्योतिः॑ । \newline
57. य॒न्ति॒ ज्योति॒र् ज्योति॑र् यन्ति यन्ति॒ ज्योति॑ रतिरा॒त्रो॑ ऽतिरा॒त्रो ज्योति॑र् यन्ति यन्ति॒ ज्योति॑ रतिरा॒त्रः । \newline
58. ज्योति॑ रतिरा॒त्रो॑ ऽतिरा॒त्रो ज्योति॒र् ज्योति॑ रतिरा॒त्रो भ॑वति भव त्यतिरा॒त्रो ज्योति॒र् ज्योति॑ रतिरा॒त्रो भ॑वति । \newline
59. अ॒ति॒रा॒त्रो भ॑वति भव त्यतिरा॒त्रो॑ ऽतिरा॒त्रो भ॑वति॒ ज्योति॒र् ज्योति॑र् भव त्यतिरा॒त्रो॑ ऽतिरा॒त्रो भ॑वति॒ ज्योतिः॑ । \newline
60. अ॒ति॒रा॒त्र इत्य॑ति - रा॒त्रः । \newline
\pagebreak
\markright{ TS 7.4.6.2  \hfill https://www.vedavms.in \hfill}

\section{ TS 7.4.6.2 }

\textbf{TS 7.4.6.2 } \newline
\textbf{Samhita Paata} \newline

भ॑वति॒ ज्योति॑रे॒व पु॒रस्ता᳚द् दधते सुव॒र्गस्य॑ लो॒कस्यानु॑ख्यात्यै षड॒हा भ॑वन्ति॒ षड् वा ऋ॒तव॑ ऋ॒तुष्वे॒व प्रति॑ तिष्ठन्ति च॒त्वारो॑ भवन्ति॒ चत॑स्रो॒ दिशो॑ दि॒क्ष्वे॑व प्रति॑ तिष्ठ॒न्त्यस॑त्रं॒ ॅवा ए॒तद्-यद॑छन्दो॒मं ॅयच्छ॑न्दो॒मा भव॑न्ति॒ तेन॑ स॒त्रं दे॒वता॑ ए॒व पृ॒ष्ठैरव॑ रुन्धते प॒शूञ्छ॑न्दो॒मैरोजो॒ वै वी॒र्यं॑ पृ॒ष्ठानि॑ प॒शवः॑ छन्दो॒मा ओज॑स्ये॒व - [  ] \newline

\textbf{Pada Paata} \newline

भ॒व॒ति॒ । ज्योतिः॑ । ए॒व । पु॒रस्ता᳚त् । द॒ध॒ते॒ । सु॒व॒र्गस्येति॑ सुवः - गस्य॑ । लो॒कस्य॑ । अनु॑ख्यात्या॒ इत्यनु॑ - ख्या॒त्यै॒ । ष॒ड॒हा इति॑ षट् - अ॒हाः । भ॒व॒न्ति॒ । षट् । वै । ऋ॒तवः॑ । ऋ॒तुषु॑ । ए॒व । प्रतीति॑ । ति॒ष्ठ॒न्ति॒ । च॒त्वारः॑ । भ॒व॒न्ति॒ । चत॑स्रः । दिशः॑ । दि॒क्षु । ए॒व । प्रतीति॑ । ति॒ष्ठ॒न्ति॒ । अस॑त्रम् । वै । ए॒तत् । यत् । अ॒छ॒न्दो॒ममित्य॑छन्दः - मम् । यत् । छ॒न्दो॒मा इति॑ छन्दः - माः । भव॑न्ति । तेन॑ । स॒त्रम् । दे॒वताः᳚ । ए॒व । पृ॒ष्ठैः । अवेति॑ । रु॒न्ध॒ते॒ । प॒शून् । छ॒न्दो॒मैरिति॑ छन्दः - मैः । ओजः॑ । वै । वी॒र्य᳚म् । पृ॒ष्ठानि॑ । प॒शवः॑ । छ॒न्दो॒मा इति॑ छन्दः - माः । ओज॑सि । ए॒व ।  \newline


\textbf{Krama Paata} \newline

भ॒व॒ति॒ ज्योतिः॑ । ज्योति॑रे॒व । ए॒व पु॒रस्ता᳚त् । पु॒रस्ता᳚द् दधते । द॒ध॒ते॒ सु॒व॒र्गस्य॑ । सु॒व॒र्गस्य॑ लो॒कस्य॑ । सु॒व॒र्गस्येति॑ सुवः - गस्य॑ । लो॒कस्यानु॑ख्यात्यै । अनु॑ख्यात्यै षड॒हाः । अनु॑ख्यात्या॒ इत्यनु॑ - ख्या॒त्यै॒ । ष॒ड॒हा भ॑वन्ति । ष॒ड॒हा इति॑ षट् - अ॒हाः । भ॒व॒न्ति॒ षट् । षड् वै । वा ऋ॒तवः॑ । ऋ॒तव॑ ऋ॒तुषु॑ । ऋ॒तुष्वे॒व । ए॒व प्रति॑ । प्रति॑ तिष्ठन्ति । ति॒ष्ठ॒न्ति॒ च॒त्वारः॑ । च॒त्वारो॑ भवन्ति । भ॒व॒न्ति॒ चत॑स्रः । चत॑स्रो॒ दिशः॑ । दिशो॑ दि॒क्षु । दि॒क्ष्वे॑व । ए॒व प्रति॑ । प्रति॑ तिष्ठ॒न्ति । ति॒ष्ठ॒न्त्यस॑त्रम् । अस॑त्र॒म् ॅवै । वा ए॒तत् । ए॒तद् यत् । यद॑छन्दो॒मम् । अ॒छ॒न्दो॒मम् ॅयत् । अ॒छ॒न्दो॒ममित्य॑छन्दः - मम् । यच् छ॑न्दो॒माः । छ॒न्दो॒मा भव॑न्ति । छ॒न्दो॒मा इति॑ छन्दः - माः । भव॑न्ति॒ तेन॑ । तेन॑ स॒त्रम् । स॒त्रम् दे॒वताः᳚ । दे॒वता॑ ए॒व । ए॒व पृ॒ष्ठैः । पृ॒ष्ठैरव॑ । अव॑ रुन्धते । रु॒न्ध॒ते॒ प॒शून् । प॒शून् छ॑न्दो॒मैः । छ॒न्दो॒मैरोजः॑ । छ॒न्दो॒मैरिति॑ छन्दः - मैः । ओजो॒ वै । वै वी॒र्य᳚म् । वी॒र्य॑म् पृ॒ष्ठानि॑ । पृ॒ष्ठानि॑ प॒शवः॑ । प॒शव॑श्छन्दो॒माः । छ॒न्दो॒मा ओज॑सि । छ॒न्दो॒मा इति॑ छन्दः - माः । ओज॑स्ये॒व ( ) । ए॒व वी॒र्ये᳚ \newline

\textbf{Jatai Paata} \newline

1. भ॒व॒ति॒ ज्योति॒र् ज्योति॑र् भवति भवति॒ ज्योतिः॑ । \newline
2. ज्योति॑ रे॒वैव ज्योति॒र् ज्योति॑ रे॒व । \newline
3. ए॒व पु॒रस्ता᳚त् पु॒रस्ता॑ दे॒वैव पु॒रस्ता᳚त् । \newline
4. पु॒रस्ता᳚द् दधते दधते पु॒रस्ता᳚त् पु॒रस्ता᳚द् दधते । \newline
5. द॒ध॒ते॒ सु॒व॒र्गस्य॑ सुव॒र्गस्य॑ दधते दधते सुव॒र्गस्य॑ । \newline
6. सु॒व॒र्गस्य॑ लो॒कस्य॑ लो॒कस्य॑ सुव॒र्गस्य॑ सुव॒र्गस्य॑ लो॒कस्य॑ । \newline
7. सु॒व॒र्गस्येति॑ सुवः - गस्य॑ । \newline
8. लो॒कस्या नु॑ख्यात्या॒ अनु॑ख्यात्यै लो॒कस्य॑ लो॒कस्या नु॑ख्यात्यै । \newline
9. अनु॑ख्यात्यै षड॒हा ष्ष॑ड॒हा अनु॑ख्यात्या॒ अनु॑ख्यात्यै षड॒हाः । \newline
10. अनु॑ख्यात्या॒ इत्यनु॑ - ख्या॒त्यै॒ । \newline
11. ष॒ड॒हा भ॑वन्ति भवन्ति षड॒हा ष्ष॑ड॒हा भ॑वन्ति । \newline
12. ष॒ड॒हा इति॑ षट् - अ॒हाः । \newline
13. भ॒व॒न्ति॒ षट् थ्षड् भ॑वन्ति भवन्ति॒ षट् । \newline
14. षड् वै वै षट् थ्षड् वै । \newline
15. वा ऋ॒तव॑ ऋ॒तवो॒ वै वा ऋ॒तवः॑ । \newline
16. ऋ॒तव॑ ऋ॒तुष् वृ॒तुष् वृ॒तव॑ ऋ॒तव॑ ऋ॒तुषु॑ । \newline
17. ऋ॒तु ष्वे॒वैव र्‌तुष् वृ॒तु ष्वे॒व । \newline
18. ए॒व प्रति॒ प्रत्ये॒वैव प्रति॑ । \newline
19. प्रति॑ तिष्ठन्ति तिष्ठन्ति॒ प्रति॒ प्रति॑ तिष्ठन्ति । \newline
20. ति॒ष्ठ॒न्ति॒ च॒त्वार॑ श्च॒त्वार॑ स्तिष्ठन्ति तिष्ठन्ति च॒त्वारः॑ । \newline
21. च॒त्वारो॑ भवन्ति भवन्ति च॒त्वार॑ श्च॒त्वारो॑ भवन्ति । \newline
22. भ॒व॒न्ति॒ चत॑स्र॒ श्चत॑स्रो भवन्ति भवन्ति॒ चत॑स्रः । \newline
23. चत॑स्रो॒ दिशो॒ दिश॒ श्चत॑स्र॒ श्चत॑स्रो॒ दिशः॑ । \newline
24. दिशो॑ दि॒क्षु दि॒क्षु दिशो॒ दिशो॑ दि॒क्षु । \newline
25. दि॒क्ष्वे॑ वैव दि॒क्षु दि॒क्ष्वे॑व । \newline
26. ए॒व प्रति॒ प्रत्ये॒वैव प्रति॑ । \newline
27. प्रति॑ तिष्ठन्ति तिष्ठन्ति॒ प्रति॒ प्रति॑ तिष्ठन्ति । \newline
28. ति॒ष्ठ॒न् त्यस॑त्र॒ मस॑त्रम् तिष्ठन्ति तिष्ठ॒न् त्यस॑त्रम् । \newline
29. अस॑त्रं॒ ॅवै वा अस॑त्र॒ मस॑त्रं॒ ॅवै । \newline
30. वा ए॒त दे॒तद् वै वा ए॒तत् । \newline
31. ए॒तद् यद् यदे॒त दे॒तद् यत् । \newline
32. यद॑छन्दो॒म म॑छन्दो॒मं ॅयद् यद॑छन्दो॒मम् । \newline
33. अ॒छ॒न्दो॒मं ॅयद् यद॑छन्दो॒म म॑छन्दो॒मं ॅयत् । \newline
34. अ॒छ॒न्दो॒ममित्य॑छन्दः - मम् । \newline
35. यच् छ॑न्दो॒मा श्छ॑न्दो॒मा यद् यच् छ॑न्दो॒माः । \newline
36. छ॒न्दो॒मा भव॑न्ति॒ भव॑न्ति छन्दो॒मा श्छ॑न्दो॒मा भव॑न्ति । \newline
37. छ॒न्दो॒मा इति॑ छन्दः - माः । \newline
38. भव॑न्ति॒ तेन॒ तेन॒ भव॑न्ति॒ भव॑न्ति॒ तेन॑ । \newline
39. तेन॑ स॒त्रꣳ स॒त्रम् तेन॒ तेन॑ स॒त्रम् । \newline
40. स॒त्रम् दे॒वता॑ दे॒वताः᳚ स॒त्रꣳ स॒त्रम् दे॒वताः᳚ । \newline
41. दे॒वता॑ ए॒वैव दे॒वता॑ दे॒वता॑ ए॒व । \newline
42. ए॒व पृ॒ष्ठैः पृ॒ष्ठै रे॒वैव पृ॒ष्ठैः । \newline
43. पृ॒ष्ठै रवाव॑ पृ॒ष्ठैः पृ॒ष्ठै रव॑ । \newline
44. अव॑ रुन्धते रुन्ध॒ते ऽवाव॑ रुन्धते । \newline
45. रु॒न्ध॒ते॒ प॒शून् प॒शून् रु॑न्धते रुन्धते प॒शून् । \newline
46. प॒शून् छ॑न्दो॒मै श्छ॑न्दो॒मैः प॒शून् प॒शून् छ॑न्दो॒मैः । \newline
47. छ॒न्दो॒मै रोज॒ ओज॑ श्छन्दो॒मै श्छ॑न्दो॒मै रोजः॑ । \newline
48. छ॒न्दो॒मैरिति॑ छन्दः - मैः । \newline
49. ओजो॒ वै वा ओज॒ ओजो॒ वै । \newline
50. वै वी॒र्यं॑ ॅवी॒र्यं॑ ॅवै वै वी॒र्य᳚म् । \newline
51. वी॒र्य॑म् पृ॒ष्ठानि॑ पृ॒ष्ठानि॑ वी॒र्यं॑ ॅवी॒र्य॑म् पृ॒ष्ठानि॑ । \newline
52. पृ॒ष्ठानि॑ प॒शवः॑ प॒शवः॑ पृ॒ष्ठानि॑ पृ॒ष्ठानि॑ प॒शवः॑ । \newline
53. प॒शव॑ श्छन्दो॒मा श्छ॑न्दो॒माः प॒शवः॑ प॒शव॑ श्छन्दो॒माः । \newline
54. छ॒न्दो॒मा ओज॒ स्योज॑सि छन्दो॒मा श्छ॑न्दो॒मा ओज॑सि । \newline
55. छ॒न्दो॒मा इति॑ छन्दः - माः । \newline
56. ओज॑ स्ये॒वै वौज॒ स्योज॑ स्ये॒व । \newline
57. ए॒व वी॒र्ये॑ वी॒र्य॑ ए॒वैव वी॒र्ये᳚ । \newline

\textbf{Ghana Paata } \newline

1. भ॒व॒ति॒ ज्योति॒र् ज्योति॑र् भवति भवति॒ ज्योति॑ रे॒वैव ज्योति॑र् भवति भवति॒ ज्योति॑ रे॒व । \newline
2. ज्योति॑ रे॒वैव ज्योति॒र् ज्योति॑ रे॒व पु॒रस्ता᳚त् पु॒रस्ता॑ दे॒व ज्योति॒र् ज्योति॑ रे॒व पु॒रस्ता᳚त् । \newline
3. ए॒व पु॒रस्ता᳚त् पु॒रस्ता॑ दे॒वैव पु॒रस्ता᳚द् दधते दधते पु॒रस्ता॑ दे॒वैव पु॒रस्ता᳚द् दधते । \newline
4. पु॒रस्ता᳚द् दधते दधते पु॒रस्ता᳚त् पु॒रस्ता᳚द् दधते सुव॒र्गस्य॑ सुव॒र्गस्य॑ दधते पु॒रस्ता᳚त् पु॒रस्ता᳚द् दधते सुव॒र्गस्य॑ । \newline
5. द॒ध॒ते॒ सु॒व॒र्गस्य॑ सुव॒र्गस्य॑ दधते दधते सुव॒र्गस्य॑ लो॒कस्य॑ लो॒कस्य॑ सुव॒र्गस्य॑ दधते दधते सुव॒र्गस्य॑ लो॒कस्य॑ । \newline
6. सु॒व॒र्गस्य॑ लो॒कस्य॑ लो॒कस्य॑ सुव॒र्गस्य॑ सुव॒र्गस्य॑ लो॒कस्या नु॑ख्यात्या॒ अनु॑ख्यात्यै लो॒कस्य॑ सुव॒र्गस्य॑ सुव॒र्गस्य॑ लो॒कस्या नु॑ख्यात्यै । \newline
7. सु॒व॒र्गस्येति॑ सुवः - गस्य॑ । \newline
8. लो॒कस्या नु॑ख्यात्या॒ अनु॑ख्यात्यै लो॒कस्य॑ लो॒कस्या नु॑ख्यात्यै षड॒हा ष्ष॑ड॒हा अनु॑ख्यात्यै लो॒कस्य॑ लो॒कस्या नु॑ख्यात्यै षड॒हाः । \newline
9. अनु॑ख्यात्यै षड॒हा ष्ष॑ड॒हा अनु॑ख्यात्या॒ अनु॑ख्यात्यै षड॒हा भ॑वन्ति भवन्ति षड॒हा अनु॑ख्यात्या॒ अनु॑ख्यात्यै षड॒हा भ॑वन्ति । \newline
10. अनु॑ख्यात्या॒ इत्यनु॑ - ख्या॒त्यै॒ । \newline
11. ष॒ड॒हा भ॑वन्ति भवन्ति षड॒हा ष्ष॑ड॒हा भ॑वन्ति॒ षट् थ्षड् भ॑वन्ति षड॒हा ष्ष॑ड॒हा भ॑वन्ति॒ षट् । \newline
12. ष॒ड॒हा इति॑ षट् - अ॒हाः । \newline
13. भ॒व॒न्ति॒ षट् थ्षड् भ॑वन्ति भवन्ति॒ षड् वै वै षड् भ॑वन्ति भवन्ति॒ षड् वै । \newline
14. षड् वै वै षट् थ्षड् वा ऋ॒तव॑ ऋ॒तवो॒ वै षट् थ्षड् वा ऋ॒तवः॑ । \newline
15. वा ऋ॒तव॑ ऋ॒तवो॒ वै वा ऋ॒तव॑ ऋ॒तुष् वृ॒तुष् वृ॒तवो॒ वै वा ऋ॒तव॑ ऋ॒तुषु॑ । \newline
16. ऋ॒तव॑ ऋ॒तुष् वृ॒तुष् वृ॒तव॑ ऋ॒तव॑ ऋ॒तुष् वे॒वैव र्‌तुष् वृ॒तव॑ ऋ॒तव॑ ऋ॒तु ष्वे॒व । \newline
17. ऋ॒तु ष्वे॒वैव र्‌तुष् वृ॒तु ष्वे॒व प्रति॒ प्रत्ये॒व र्‌तुष् वृ॒तु ष्वे॒व प्रति॑ । \newline
18. ए॒व प्रति॒ प्रत्ये॒वैव प्रति॑ तिष्ठन्ति तिष्ठन्ति॒ प्रत्ये॒वैव प्रति॑ तिष्ठन्ति । \newline
19. प्रति॑ तिष्ठन्ति तिष्ठन्ति॒ प्रति॒ प्रति॑ तिष्ठन्ति च॒त्वार॑ श्च॒त्वार॑ स्तिष्ठन्ति॒ प्रति॒ प्रति॑ तिष्ठन्ति च॒त्वारः॑ । \newline
20. ति॒ष्ठ॒न्ति॒ च॒त्वार॑ श्च॒त्वार॑ स्तिष्ठन्ति तिष्ठन्ति च॒त्वारो॑ भवन्ति भवन्ति च॒त्वार॑ स्तिष्ठन्ति तिष्ठन्ति च॒त्वारो॑ भवन्ति । \newline
21. च॒त्वारो॑ भवन्ति भवन्ति च॒त्वार॑ श्च॒त्वारो॑ भवन्ति॒ चत॑स्र॒ श्चत॑स्रो भवन्ति च॒त्वार॑ श्च॒त्वारो॑ भवन्ति॒ चत॑स्रः । \newline
22. भ॒व॒न्ति॒ चत॑स्र॒ श्चत॑स्रो भवन्ति भवन्ति॒ चत॑स्रो॒ दिशो॒ दिश॒ श्चत॑स्रो भवन्ति भवन्ति॒ चत॑स्रो॒ दिशः॑ । \newline
23. चत॑स्रो॒ दिशो॒ दिश॒ श्चत॑स्र॒ श्चत॑स्रो॒ दिशो॑ दि॒क्षु दि॒क्षु दिश॒ श्चत॑स्र॒ श्चत॑स्रो॒ दिशो॑ दि॒क्षु । \newline
24. दिशो॑ दि॒क्षु दि॒क्षु दिशो॒ दिशो॑ दि॒क्ष्वे॑वैव दि॒क्षु दिशो॒ दिशो॑ दि॒क्ष्वे॑व । \newline
25. दि॒क्ष्वे॑वैव दि॒क्षु दि॒क्ष्वे॑व प्रति॒ प्रत्ये॒व दि॒क्षु दि॒क्ष्वे॑व प्रति॑ । \newline
26. ए॒व प्रति॒ प्रत्ये॒वैव प्रति॑ तिष्ठन्ति तिष्ठन्ति॒ प्रत्ये॒वैव प्रति॑ तिष्ठन्ति । \newline
27. प्रति॑ तिष्ठन्ति तिष्ठन्ति॒ प्रति॒ प्रति॑ तिष्ठ॒ न्त्यस॑त्र॒ मस॑त्रम् तिष्ठन्ति॒ प्रति॒ प्रति॑ तिष्ठ॒ न्त्यस॑त्रम् । \newline
28. ति॒ष्ठ॒ न्त्यस॑त्र॒ मस॑त्रम् तिष्ठन्ति तिष्ठ॒ न्त्यस॑त्रं॒ ॅवै वा अस॑त्रम् तिष्ठन्ति तिष्ठ॒ न्त्यस॑त्रं॒ ॅवै । \newline
29. अस॑त्रं॒ ॅवै वा अस॑त्र॒ मस॑त्रं॒ ॅवा ए॒त दे॒तद् वा अस॑त्र॒ मस॑त्रं॒ ॅवा ए॒तत् । \newline
30. वा ए॒त दे॒तद् वै वा ए॒तद् यद् यदे॒तद् वै वा ए॒तद् यत् । \newline
31. ए॒तद् यद् यदे॒त दे॒तद् यद॑छन्दो॒म म॑छन्दो॒मं ॅयदे॒त दे॒तद् यद॑छन्दो॒मम् । \newline
32. यद॑छन्दो॒म म॑छन्दो॒मं ॅयद् यद॑छन्दो॒मं ॅयद् यद॑छन्दो॒मं ॅयद् यद॑छन्दो॒मं ॅयत् । \newline
33. अ॒छ॒न्दो॒मं ॅयद् यद॑छन्दो॒म म॑छन्दो॒मं ॅयच् छ॑न्दो॒मा श्छ॑न्दो॒मा यद॑छन्दो॒म म॑छन्दो॒मं ॅयच् छ॑न्दो॒माः । \newline
34. अ॒छ॒न्दो॒ममित्य॑छन्दः - मम् । \newline
35. यच् छ॑न्दो॒मा श्छ॑न्दो॒मा यद् यच् छ॑न्दो॒मा भव॑न्ति॒ भव॑न्ति छन्दो॒मा यद् यच् छ॑न्दो॒मा भव॑न्ति । \newline
36. छ॒न्दो॒मा भव॑न्ति॒ भव॑न्ति छन्दो॒मा श्छ॑न्दो॒मा भव॑न्ति॒ तेन॒ तेन॒ भव॑न्ति छन्दो॒मा श्छ॑न्दो॒मा भव॑न्ति॒ तेन॑ । \newline
37. छ॒न्दो॒मा इति॑ छन्दः - माः । \newline
38. भव॑न्ति॒ तेन॒ तेन॒ भव॑न्ति॒ भव॑न्ति॒ तेन॑ स॒त्रꣳ स॒त्रम् तेन॒ भव॑न्ति॒ भव॑न्ति॒ तेन॑ स॒त्रम् । \newline
39. तेन॑ स॒त्रꣳ स॒त्रम् तेन॒ तेन॑ स॒त्रम् दे॒वता॑ दे॒वताः᳚ स॒त्रम् तेन॒ तेन॑ स॒त्रम् दे॒वताः᳚ । \newline
40. स॒त्रम् दे॒वता॑ दे॒वताः᳚ स॒त्रꣳ स॒त्रम् दे॒वता॑ ए॒वैव दे॒वताः᳚ स॒त्रꣳ स॒त्रम् दे॒वता॑ ए॒व । \newline
41. दे॒वता॑ ए॒वैव दे॒वता॑ दे॒वता॑ ए॒व पृ॒ष्ठैः पृ॒ष्ठै रे॒व दे॒वता॑ दे॒वता॑ ए॒व पृ॒ष्ठैः । \newline
42. ए॒व पृ॒ष्ठैः पृ॒ष्ठै रे॒वैव पृ॒ष्ठै रवाव॑ पृ॒ष्ठै रे॒वैव पृ॒ष्ठै रव॑ । \newline
43. पृ॒ष्ठै रवाव॑ पृ॒ष्ठैः पृ॒ष्ठै रव॑ रुन्धते रुन्ध॒ते ऽव॑ पृ॒ष्ठैः पृ॒ष्ठै रव॑ रुन्धते । \newline
44. अव॑ रुन्धते रुन्ध॒ते ऽवाव॑ रुन्धते प॒शून् प॒शून् रु॑न्ध॒ते ऽवाव॑ रुन्धते प॒शून् । \newline
45. रु॒न्ध॒ते॒ प॒शून् प॒शून् रु॑न्धते रुन्धते प॒शून् छ॑न्दो॒मै श्छ॑न्दो॒मैः प॒शून् रु॑न्धते रुन्धते प॒शून् छ॑न्दो॒मैः । \newline
46. प॒शून् छ॑न्दो॒मै श्छ॑न्दो॒मैः प॒शून् प॒शून् छ॑न्दो॒मै रोज॒ ओज॑ श्छन्दो॒मैः प॒शून् प॒शून् छ॑न्दो॒मै रोजः॑ । \newline
47. छ॒न्दो॒मै रोज॒ ओज॑ श्छन्दो॒मै श्छ॑न्दो॒मै रोजो॒ वै वा ओज॑ श्छन्दो॒मै श्छ॑न्दो॒मै रोजो॒ वै । \newline
48. छ॒न्दो॒मैरिति॑ छन्दः - मैः । \newline
49. ओजो॒ वै वा ओज॒ ओजो॒ वै वी॒र्यं॑ ॅवी॒र्यं॑ ॅवा ओज॒ ओजो॒ वै वी॒र्य᳚म् । \newline
50. वै वी॒र्यं॑ ॅवी॒र्यं॑ ॅवै वै वी॒र्य॑म् पृ॒ष्ठानि॑ पृ॒ष्ठानि॑ वी॒र्यं॑ ॅवै वै वी॒र्य॑म् पृ॒ष्ठानि॑ । \newline
51. वी॒र्य॑म् पृ॒ष्ठानि॑ पृ॒ष्ठानि॑ वी॒र्यं॑ ॅवी॒र्य॑म् पृ॒ष्ठानि॑ प॒शवः॑ प॒शवः॑ पृ॒ष्ठानि॑ वी॒र्यं॑ ॅवी॒र्य॑म् पृ॒ष्ठानि॑ प॒शवः॑ । \newline
52. पृ॒ष्ठानि॑ प॒शवः॑ प॒शवः॑ पृ॒ष्ठानि॑ पृ॒ष्ठानि॑ प॒शव॑ श्छन्दो॒मा श्छ॑न्दो॒माः प॒शवः॑ पृ॒ष्ठानि॑ पृ॒ष्ठानि॑ प॒शव॑ श्छन्दो॒माः । \newline
53. प॒शव॑ श्छन्दो॒मा श्छ॑न्दो॒माः प॒शवः॑ प॒शव॑ श्छन्दो॒मा ओज॒ स्योज॑सि छन्दो॒माः प॒शवः॑ प॒शव॑ श्छन्दो॒मा ओज॑सि । \newline
54. छ॒न्दो॒मा ओज॒ स्योज॑सि छन्दो॒मा श्छ॑न्दो॒मा ओज॑ स्ये॒वै वौज॑सि छन्दो॒मा श्छ॑न्दो॒मा ओज॑ स्ये॒व । \newline
55. छ॒न्दो॒मा इति॑ छन्दः - माः । \newline
56. ओज॑ स्ये॒वै वौज॒ स्योज॑स्ये॒व वी॒र्ये॑ वी॒र्य॑ ए॒वौज॒ स्योज॑ स्ये॒व वी॒र्ये᳚ । \newline
57. ए॒व वी॒र्ये॑ वी॒र्य॑ ए॒वैव वी॒र्ये॑ प॒शुषु॑ प॒शुषु॑ वी॒र्य॑ ए॒वैव वी॒र्ये॑ प॒शुषु॑ । \newline
\pagebreak
\markright{ TS 7.4.6.3  \hfill https://www.vedavms.in \hfill}

\section{ TS 7.4.6.3 }

\textbf{TS 7.4.6.3 } \newline
\textbf{Samhita Paata} \newline

वी॒र्ये॑ प॒शुषु॒ प्रति॑ तिष्ठन्ति षट्-त्रिꣳशद् रा॒त्रो भ॑वति॒ षट्त्रिꣳ॑शदक्षरा बृह॒ती बार्.ह॑ताः प॒शवो॑ बृह॒त्यैव प॒शूनव॑ रुन्धते बृह॒ती छन्द॑साꣳ॒॒ स्वारा᳚ज्यमाश्नुता-श्नु॒वते॒ स्वारा᳚ज्यं॒ ॅय ए॒वं ॅवि॒द्वाꣳसः॑ षट्त्रिꣳशद् रा॒त्रमास॑ते सुव॒र्गमे॒व लो॒कं ॅय॑न्त्यतिरा॒त्राव॒भितो॑ भवतः सुव॒र्गस्य॑ लो॒कस्य॒ परि॑गृहीत्यै ॥ \newline

\textbf{Pada Paata} \newline

वी॒र्ये᳚ । प॒शुषु॑ । प्रतीति॑ । ति॒ष्ठ॒न्ति॒ । ष॒ट्त्रिꣳ॒॒श॒द्रा॒त्र इति॑ षट्त्रिꣳशत्-रा॒त्रः । भ॒व॒ति॒ । षट्त्रिꣳ॑शदक्ष॒रेति॒ षट्त्रिꣳ॑शत्-अ॒क्ष॒रा॒ । बृ॒ह॒ती । बार्.ह॑ताः । प॒शवः॑ । बृ॒ह॒त्या । ए॒व । प॒शून् । अवेति॑ । रु॒न्ध॒ते॒ । बृ॒ह॒ती । छन्द॑साम् । स्वारा᳚ज्य॒मिति॒ स्व - रा॒ज्य॒म् । आ॒श्नु॒त॒ । अ॒श्नु॒वते᳚ । स्वारा᳚ज्य॒मिति॒ स्व - रा॒ज्य॒म् । ये । ए॒वम् । वि॒द्वाꣳसः॑ । ष॒ट्त्रिꣳ॒॒श॒द्रा॒त्रमिति॑ षट्त्रिꣳशत् - रा॒त्रम् । आस॑ते । सु॒व॒र्गमिति॑ सुवः - गम् । ए॒व । लो॒कम् । य॒न्ति॒ । अ॒ति॒रा॒त्रावित्य॑ति - रा॒त्रौ । अ॒भितः॑ । भ॒व॒तः॒ । सु॒व॒र्गस्येति॑ सुवः - गस्य॑ । लो॒कस्य॑ । परि॑गृहीत्या॒ इति॒ परि॑ - गृ॒ही॒त्यै॒ ॥  \newline


\textbf{Krama Paata} \newline

वी॒र्ये॑ प॒शुषु॑ । प॒शुषु॒ प्रति॑ । प्रति॑ तिष्ठन्ति । ति॒ष्ठ॒न्ति॒ ष॒ट्‌त्रिꣳ॒॒श॒द्‍रा॒त्रः । ष॒ट्‌त्रिꣳ॒॒श॒द्‍रा॒त्रो भ॑वति । ष॒ट्‌त्रिꣳ॒॒श॒द्‌रा॒त्र इति॑ षट्‌त्रिꣳशत् - रा॒त्रः । भ॒व॒ति॒ षट्‌त्रिꣳ॑शदक्षरा । षट्‌त्रिꣳ॑शदक्षरा बृह॒ती । षट्‌त्रिꣳ॑शदक्ष॒रेति॒ षट्त्रिꣳ॑शत् - अ॒क्ष॒रा॒ । बृ॒ह॒ती बार्.ह॑ताः । बार्.ह॑ताः प॒शवः॑ । प॒शवो॑ बृह॒त्या । बृ॒ह॒त्यैव । ए॒व प॒शून् । प॒शूनव॑ । अव॑ रुन्धते । रु॒न्ध॒ते॒ बृ॒ह॒ती । बृ॒ह॒ती छन्द॑साम् । छन्द॑साꣳ॒॒ स्वारा᳚ज्यम् । स्वारा᳚ज्यमाश्ञुत । स्वारा᳚ज्य॒मिति॒ स्व - रा॒ज्य॒म् । आ॒श्ञु॒ता॒श्ञु॒वते᳚ । अ॒श्ञु॒वते॒ स्वारा᳚ज्यम् । स्वारा᳚ज्य॒म् ॅये । स्वारा᳚ज्य॒मिति॒ स्व - रा॒ज्य॒म् । य ए॒वम् । ए॒वम् ॅवि॒द्वाꣳसः॑ । वि॒द्वाꣳसः॑ षट्‌त्रिꣳशद्‍रा॒त्रम् । ष॒ट्‌त्रिꣳ॒॒श॒द्‍रा॒त्रमास॑ते । ष॒ट्‌त्रिꣳ॒॒श॒द्‍रा॒त्रमिति॑ षट्‌त्रिꣳशत् - रा॒त्रम् । आस॑ते सुव॒र्गम् । सु॒व॒र्गमे॒व । सु॒व॒र्गमिति॑ सुवः - गम् । ए॒व लो॒कम् । लो॒कम् ॅय॑न्ति । य॒न्त्य॒ति॒रा॒त्रौ । अ॒ति॒रा॒त्राव॒भितः॑ । अ॒ति॒रा॒त्रावित्य॑ति - रा॒त्रौ । अ॒भितो॑ भवतः । भ॒व॒तः॒ सु॒व॒र्गस्य॑ । सु॒व॒र्गस्य॑ लो॒कस्य॑ । सु॒व॒र्गस्येति॑ सुवः - गस्य॑ । लो॒कस्य॒ परि॑गृहीत्यै । परि॑गृहीत्या॒ इति॒ परि॑ - गृ॒ही॒त्यै॒ । \newline

\textbf{Jatai Paata} \newline

1. वी॒र्ये॑ प॒शुषु॑ प॒शुषु॑ वी॒र्ये॑ वी॒र्ये॑ प॒शुषु॑ । \newline
2. प॒शुषु॒ प्रति॒ प्रति॑ प॒शुषु॑ प॒शुषु॒ प्रति॑ । \newline
3. प्रति॑ तिष्ठन्ति तिष्ठन्ति॒ प्रति॒ प्रति॑ तिष्ठन्ति । \newline
4. ति॒ष्ठ॒न्ति॒ ष॒ट्त्रिꣳ॒॒श॒द्रा॒त्र ष्ष॑ट्त्रिꣳशद्रा॒त्र स्ति॑ष्ठन्ति तिष्ठन्ति षट्त्रिꣳशद्रा॒त्रः । \newline
5. ष॒ट्त्रिꣳ॒॒श॒द्रा॒त्रो भ॑वति भवति षट्त्रिꣳशद्रा॒त्र ष्ष॑ट्त्रिꣳशद्रा॒त्रो भ॑वति । \newline
6. ष॒ट्त्रिꣳ॒॒श॒द्रा॒त्र इति॑ षट्त्रिꣳशत् - रा॒त्रः । \newline
7. भ॒व॒ति॒ षट्त्रिꣳ॑शदक्षरा॒ षट्त्रिꣳ॑शदक्षरा भवति भवति॒ षट्त्रिꣳ॑शदक्षरा । \newline
8. षट्त्रिꣳ॑शदक्षरा बृह॒ती बृ॑ह॒ती षट्त्रिꣳ॑शदक्षरा॒ षट्त्रिꣳ॑शदक्षरा बृह॒ती । \newline
9. षट्त्रिꣳ॑शदक्ष॒रेति॒ षट्त्रिꣳ॑शत् - अ॒क्ष॒रा॒ । \newline
10. बृ॒ह॒ती बार्.ह॑ता॒ बार्.ह॑ता बृह॒ती बृ॑ह॒ती बार्.ह॑ताः । \newline
11. बार्.ह॑ताः प॒शवः॑ प॒शवो॒ बार्.ह॑ता॒ बार्.ह॑ताः प॒शवः॑ । \newline
12. प॒शवो॑ बृह॒त्या बृ॑ह॒त्या प॒शवः॑ प॒शवो॑ बृह॒त्या । \newline
13. बृ॒ह॒त्यैवैव बृ॑ह॒त्या बृ॑ह॒ त्यैव । \newline
14. ए॒व प॒शून् प॒शूने॒वैव प॒शून् । \newline
15. प॒शू नवाव॑ प॒शून् प॒शूनव॑ । \newline
16. अव॑ रुन्धते रुन्ध॒ते ऽवाव॑ रुन्धते । \newline
17. रु॒न्ध॒ते॒ बृ॒ह॒ती बृ॑ह॒ती रु॑न्धते रुन्धते बृह॒ती । \newline
18. बृ॒ह॒ती छन्द॑सा॒म् छन्द॑साम् बृह॒ती बृ॑ह॒ती छन्द॑साम् । \newline
19. छन्द॑साꣳ॒॒ स्वारा᳚ज्यꣳ॒॒ स्वारा᳚ज्य॒म् छन्द॑सा॒म् छन्द॑साꣳ॒॒ स्वारा᳚ज्यम् । \newline
20. स्वारा᳚ज्य माश्ञुता श्ञुत॒ स्वारा᳚ज्यꣳ॒॒ स्वारा᳚ज्य माश्ञुत । \newline
21. स्वारा᳚ज्य॒मिति॒ स्व - रा॒ज्य॒म् । \newline
22. आ॒श्ञु॒ता॒ श्ञु॒वते᳚ ऽश्ञु॒वत॑ आश्ञुता श्ञुता श्ञु॒वते᳚ । \newline
23. अ॒श्ञु॒वते॒ स्वारा᳚ज्यꣳ॒॒ स्वारा᳚ज्य मश्ञु॒वते᳚ ऽश्ञु॒वते॒ स्वारा᳚ज्यम् । \newline
24. स्वारा᳚ज्यं॒ ॅये ये स्वारा᳚ज्यꣳ॒॒ स्वारा᳚ज्यं॒ ॅये । \newline
25. स्वारा᳚ज्य॒मिति॒ स्व - रा॒ज्य॒म् । \newline
26. य ए॒व मे॒वं ॅये य ए॒वम् । \newline
27. ए॒वं ॅवि॒द्वाꣳसो॑ वि॒द्वाꣳस॑ ए॒व मे॒वं ॅवि॒द्वाꣳसः॑ । \newline
28. वि॒द्वाꣳस॑ ष्षट्त्रिꣳशद्रा॒त्रꣳ ष॑ट्त्रिꣳशद्रा॒त्रं ॅवि॒द्वाꣳसो॑ वि॒द्वाꣳस॑ष्षट्त्रिꣳशद्रा॒त्रम् । \newline
29. ष॒ट्त्रिꣳ॒॒श॒द्रा॒त्र मास॑त॒ आस॑ते षट्त्रिꣳशद्रा॒त्रꣳ ष॑ट्त्रिꣳशद्रा॒त्र मास॑ते । \newline
30. ष॒ट्त्रिꣳ॒॒श॒द्रा॒त्रमिति॑ षट्त्रिꣳशत् - रा॒त्रम् । \newline
31. आस॑ते सुव॒र्गꣳ सु॑व॒र्ग मास॑त॒ आस॑ते सुव॒र्गम् । \newline
32. सु॒व॒र्ग मे॒वैव सु॑व॒र्गꣳ सु॑व॒र्ग मे॒व । \newline
33. सु॒व॒र्गमिति॑ सुवः - गम् । \newline
34. ए॒व लो॒कम् ॅलो॒क मे॒वैव लो॒कम् । \newline
35. लो॒कं ॅय॑न्ति यन्ति लो॒कम् ॅलो॒कं ॅय॑न्ति । \newline
36. य॒न्त्य॒ति॒रा॒त्रा व॑तिरा॒त्रौ य॑न्ति यन्त्यतिरा॒त्रौ । \newline
37. अ॒ति॒रा॒त्रा व॒भितो॒ ऽभितो॑ ऽतिरा॒त्रा व॑तिरा॒त्रा व॒भितः॑ । \newline
38. अ॒ति॒रा॒त्रावित्य॑ति - रा॒त्रौ । \newline
39. अ॒भितो॑ भवतो भवतो॒ ऽभितो॒ ऽभितो॑ भवतः । \newline
40. भ॒व॒तः॒ सु॒व॒र्गस्य॑ सुव॒र्गस्य॑ भवतो भवतः सुव॒र्गस्य॑ । \newline
41. सु॒व॒र्गस्य॑ लो॒कस्य॑ लो॒कस्य॑ सुव॒र्गस्य॑ सुव॒र्गस्य॑ लो॒कस्य॑ । \newline
42. सु॒व॒र्गस्येति॑ सुवः - गस्य॑ । \newline
43. लो॒कस्य॒ परि॑गृहीत्यै॒ परि॑गृहीत्यै लो॒कस्य॑ लो॒कस्य॒ परि॑गृहीत्यै । \newline
44. परि॑गृहीत्या॒ इति॒ परि॑ - गृ॒ही॒त्यै॒ । \newline

\textbf{Ghana Paata } \newline

1. वी॒र्ये॑ प॒शुषु॑ प॒शुषु॑ वी॒र्ये॑ वी॒र्ये॑ प॒शुषु॒ प्रति॒ प्रति॑ प॒शुषु॑ वी॒र्ये॑ वी॒र्ये॑ प॒शुषु॒ प्रति॑ । \newline
2. प॒शुषु॒ प्रति॒ प्रति॑ प॒शुषु॑ प॒शुषु॒ प्रति॑ तिष्ठन्ति तिष्ठन्ति॒ प्रति॑ प॒शुषु॑ प॒शुषु॒ प्रति॑ तिष्ठन्ति । \newline
3. प्रति॑ तिष्ठन्ति तिष्ठन्ति॒ प्रति॒ प्रति॑ तिष्ठन्ति षट्त्रिꣳशद्रा॒त्र ष्ष॑ट्त्रिꣳशद्रा॒त्र स्ति॑ष्ठन्ति॒ प्रति॒ प्रति॑ तिष्ठन्ति षट्त्रिꣳशद्रा॒त्रः । \newline
4. ति॒ष्ठ॒न्ति॒ ष॒ट्त्रिꣳ॒॒श॒द्रा॒त्र ष्ष॑ट्त्रिꣳशद्रा॒त्र स्ति॑ष्ठन्ति तिष्ठन्ति षट्त्रिꣳशद्रा॒त्रो भ॑वति भवति षट्त्रिꣳशद्रा॒त्र स्ति॑ष्ठन्ति तिष्ठन्ति षट्त्रिꣳशद्रा॒त्रो भ॑वति । \newline
5. ष॒ट्त्रिꣳ॒॒श॒द्रा॒त्रो भ॑वति भवति षट्त्रिꣳशद्रा॒त्र ष्ष॑ट्त्रिꣳशद्रा॒त्रो भ॑वति॒ षट्त्रिꣳ॑शदक्षरा॒ षट्त्रिꣳ॑शदक्षरा भवति षट्त्रिꣳशद्रा॒त्र ष्ष॑ट्त्रिꣳशद्रा॒त्रो भ॑वति॒ षट्त्रिꣳ॑शदक्षरा । \newline
6. ष॒ट्त्रिꣳ॒॒श॒द्रा॒त्र इति॑ षट्त्रिꣳशत् - रा॒त्रः । \newline
7. भ॒व॒ति॒ षट्त्रिꣳ॑शदक्षरा॒ षट्त्रिꣳ॑शदक्षरा भवति भवति॒ षट्त्रिꣳ॑शदक्षरा बृह॒ती बृ॑ह॒ती षट्त्रिꣳ॑शदक्षरा भवति भवति॒ षट्त्रिꣳ॑शदक्षरा बृह॒ती । \newline
8. षट्त्रिꣳ॑शदक्षरा बृह॒ती बृ॑ह॒ती षट्त्रिꣳ॑शदक्षरा॒ षट्त्रिꣳ॑शदक्षरा बृह॒ती बार्.ह॑ता॒ बार्.ह॑ता बृह॒ती षट्त्रिꣳ॑शदक्षरा॒ षट्त्रिꣳ॑शदक्षरा बृह॒ती बार्.ह॑ताः । \newline
9. षट्त्रिꣳ॑शदक्ष॒रेति॒ षट्त्रिꣳ॑शत् - अ॒क्ष॒रा॒ । \newline
10. बृ॒ह॒ती बार्.ह॑ता॒ बार्.ह॑ता बृह॒ती बृ॑ह॒ती बार्.ह॑ताः प॒शवः॑ प॒शवो॒ बार्.ह॑ता बृह॒ती बृ॑ह॒ती बार्.ह॑ताः प॒शवः॑ । \newline
11. बार्.ह॑ताः प॒शवः॑ प॒शवो॒ बार्.ह॑ता॒ बार्.ह॑ताः प॒शवो॑ बृह॒त्या बृ॑ह॒त्या प॒शवो॒ बार्.ह॑ता॒ बार्.ह॑ताः प॒शवो॑ बृह॒त्या । \newline
12. प॒शवो॑ बृह॒त्या बृ॑ह॒त्या प॒शवः॑ प॒शवो॑ बृह॒त्यैवैव बृ॑ह॒त्या प॒शवः॑ प॒शवो॑ बृह॒त्यैव । \newline
13. बृ॒ह॒त्यैवैव बृ॑ह॒त्या बृ॑ह॒त्यैव प॒शून् प॒शू ने॒व बृ॑ह॒त्या बृ॑ह॒त्यैव प॒शून् । \newline
14. ए॒व प॒शून् प॒शूने॒वैव प॒शू नवाव॑ प॒शूने॒वैव प॒शूनव॑ । \newline
15. प॒शूनवाव॑ प॒शून् प॒शूनव॑ रुन्धते रुन्ध॒ते ऽव॑ प॒शून् प॒शूनव॑ रुन्धते । \newline
16. अव॑ रुन्धते रुन्ध॒ते ऽवाव॑ रुन्धते बृह॒ती बृ॑ह॒ती रु॑न्ध॒ते ऽवाव॑ रुन्धते बृह॒ती । \newline
17. रु॒न्ध॒ते॒ बृ॒ह॒ती बृ॑ह॒ती रु॑न्धते रुन्धते बृह॒ती छन्द॑सा॒म् छन्द॑साम् बृह॒ती रु॑न्धते रुन्धते बृह॒ती छन्द॑साम् । \newline
18. बृ॒ह॒ती छन्द॑सा॒म् छन्द॑साम् बृह॒ती बृ॑ह॒ती छन्द॑साꣳ॒॒ स्वारा᳚ज्यꣳ॒॒ स्वारा᳚ज्य॒म् छन्द॑साम् बृह॒ती बृ॑ह॒ती छन्द॑साꣳ॒॒ स्वारा᳚ज्यम् । \newline
19. छन्द॑साꣳ॒॒ स्वारा᳚ज्यꣳ॒॒ स्वारा᳚ज्य॒म् छन्द॑सा॒म् छन्द॑साꣳ॒॒ स्वारा᳚ज्य माश्ञुता श्ञुत॒ स्वारा᳚ज्य॒म् छन्द॑सा॒म् छन्द॑साꣳ॒॒ स्वारा᳚ज्य माश्ञुत । \newline
20. स्वारा᳚ज्य माश्ञुता श्ञुत॒ स्वारा᳚ज्यꣳ॒॒ स्वारा᳚ज्य माश्ञुता श्ञु॒वते᳚ ऽश्ञु॒वत॑ आश्ञुत॒ स्वारा᳚ज्यꣳ॒॒ स्वारा᳚ज्य माश्ञुता श्ञु॒वते᳚ । \newline
21. स्वारा᳚ज्य॒मिति॒ स्व - रा॒ज्य॒म् । \newline
22. आ॒श्ञु॒ता॒ श्ञु॒वते᳚ ऽश्ञु॒वत॑ आश्ञुता श्ञुता श्ञु॒वते॒ स्वारा᳚ज्यꣳ॒॒ स्वारा᳚ज्य मश्ञु॒वत॑ आश्ञुता श्ञुता श्ञु॒वते॒ स्वारा᳚ज्यम् । \newline
23. अ॒श्ञु॒वते॒ स्वारा᳚ज्यꣳ॒॒ स्वारा᳚ज्य मश्ञु॒वते᳚ ऽश्ञु॒वते॒ स्वारा᳚ज्यं॒ ॅये ये स्वारा᳚ज्य मश्ञु॒वते᳚ ऽश्ञु॒वते॒ स्वारा᳚ज्यं॒ ॅये । \newline
24. स्वारा᳚ज्यं॒ ॅये ये स्वारा᳚ज्यꣳ॒॒ स्वारा᳚ज्यं॒ ॅय ए॒व मे॒वं ॅये स्वारा᳚ज्यꣳ॒॒ स्वारा᳚ज्यं॒ ॅय ए॒वम् । \newline
25. स्वारा᳚ज्य॒मिति॒ स्व - रा॒ज्य॒म् । \newline
26. य ए॒व मे॒वं ॅये य ए॒वं ॅवि॒द्वाꣳसो॑ वि॒द्वाꣳस॑ ए॒वं ॅये य ए॒वं ॅवि॒द्वाꣳसः॑ । \newline
27. ए॒वं ॅवि॒द्वाꣳसो॑ वि॒द्वाꣳस॑ ए॒व मे॒वं ॅवि॒द्वाꣳस॑ ष्षट्त्रिꣳशद्रा॒त्रꣳ ष॑ट्त्रिꣳशद्रा॒त्रं ॅवि॒द्वाꣳस॑ ए॒व मे॒वं ॅवि॒द्वाꣳस॑ ष्षट्त्रिꣳशद्रा॒त्रम् । \newline
28. वि॒द्वाꣳस॑ ष्षट्त्रिꣳशद्रा॒त्रꣳ ष॑ट्त्रिꣳशद्रा॒त्रं ॅवि॒द्वाꣳसो॑ वि॒द्वाꣳस॑ ष्षट्त्रिꣳशद्रा॒त्र मास॑त॒ आस॑ते षट्त्रिꣳशद्रा॒त्रं ॅवि॒द्वाꣳसो॑ वि॒द्वाꣳस॑ ष्षट्त्रिꣳशद्रा॒त्र मास॑ते । \newline
29. ष॒ट्त्रिꣳ॒॒श॒द्रा॒त्र मास॑त॒ आस॑ते षट्त्रिꣳशद्रा॒त्रꣳ ष॑ट्त्रिꣳशद्रा॒त्र मास॑ते सुव॒र्गꣳ सु॑व॒र्ग मास॑ते षट्त्रिꣳशद्रा॒त्रꣳ ष॑ट्त्रिꣳशद्रा॒त्र मास॑ते सुव॒र्गम् । \newline
30. ष॒ट्त्रिꣳ॒॒श॒द्रा॒त्रमिति॑ षट्त्रिꣳशत् - रा॒त्रम् । \newline
31. आस॑ते सुव॒र्गꣳ सु॑व॒र्ग मास॑त॒ आस॑ते सुव॒र्ग मे॒वैव सु॑व॒र्ग मास॑त॒ आस॑ते सुव॒र्ग मे॒व । \newline
32. सु॒व॒र्ग मे॒वैव सु॑व॒र्गꣳ सु॑व॒र्ग मे॒व लो॒कम् ॅलो॒क मे॒व सु॑व॒र्गꣳ सु॑व॒र्ग मे॒व लो॒कम् । \newline
33. सु॒व॒र्गमिति॑ सुवः - गम् । \newline
34. ए॒व लो॒कम् ॅलो॒क मे॒वैव लो॒कं ॅय॑न्ति यन्ति लो॒क मे॒वैव लो॒कं ॅय॑न्ति । \newline
35. लो॒कं ॅय॑न्ति यन्ति लो॒कम् ॅलो॒कं ॅय॑न्त्यतिरा॒त्रा व॑तिरा॒त्रौ य॑न्ति लो॒कम् ॅलो॒कं ॅय॑न्त्यतिरा॒त्रौ । \newline
36. य॒न्त्य॒ति॒रा॒त्रा व॑तिरा॒त्रौ य॑न्ति यन्त्यतिरा॒त्रा व॒भितो॒ ऽभितो॑ ऽतिरा॒त्रौ य॑न्ति यन्त्यतिरा॒त्रा व॒भितः॑ । \newline
37. अ॒ति॒रा॒त्रा व॒भितो॒ ऽभितो॑ ऽतिरा॒त्रा व॑तिरा॒त्रा व॒भितो॑ भवतो भवतो॒ ऽभितो॑ ऽतिरा॒त्रा व॑तिरा॒त्रा व॒भितो॑ भवतः । \newline
38. अ॒ति॒रा॒त्रावित्य॑ति - रा॒त्रौ । \newline
39. अ॒भितो॑ भवतो भवतो॒ ऽभितो॒ ऽभितो॑ भवतः सुव॒र्गस्य॑ सुव॒र्गस्य॑ भवतो॒ ऽभितो॒ ऽभितो॑ भवतः सुव॒र्गस्य॑ । \newline
40. भ॒व॒तः॒ सु॒व॒र्गस्य॑ सुव॒र्गस्य॑ भवतो भवतः सुव॒र्गस्य॑ लो॒कस्य॑ लो॒कस्य॑ सुव॒र्गस्य॑ भवतो भवतः सुव॒र्गस्य॑ लो॒कस्य॑ । \newline
41. सु॒व॒र्गस्य॑ लो॒कस्य॑ लो॒कस्य॑ सुव॒र्गस्य॑ सुव॒र्गस्य॑ लो॒कस्य॒ परि॑गृहीत्यै॒ परि॑गृहीत्यै लो॒कस्य॑ सुव॒र्गस्य॑ सुव॒र्गस्य॑ लो॒कस्य॒ परि॑गृहीत्यै । \newline
42. सु॒व॒र्गस्येति॑ सुवः - गस्य॑ । \newline
43. लो॒कस्य॒ परि॑गृहीत्यै॒ परि॑गृहीत्यै लो॒कस्य॑ लो॒कस्य॒ परि॑गृहीत्यै । \newline
44. परि॑गृहीत्या॒ इति॒ परि॑ - गृ॒ही॒त्यै॒ । \newline
\pagebreak
\markright{ TS 7.4.7.1  \hfill https://www.vedavms.in \hfill}

\section{ TS 7.4.7.1 }

\textbf{TS 7.4.7.1 } \newline
\textbf{Samhita Paata} \newline

वसि॑ष्ठो ह॒तपु॑त्रोऽकामयत वि॒न्देय॑ प्र॒जाम॒भि सा॑दा॒सान् भ॑वेय॒मिति॒ स ए॒तमे॑कस्मा-न्नपञ्चा॒श-म॑पश्य॒त् तमाऽह॑र॒त् तेना॑यजत॒ ततो॒ वै सोऽवि॑न्दत प्र॒जाम॒भि सौ॑दा॒सान॑भव॒द्य ए॒वं ॅवि॒द्वाꣳस॑ एकस्मा-न्नपञ्चा॒शमास॑ते वि॒न्दन्ते᳚ प्र॒जाम॒भि भ्रातृ॑व्यान् भवन्ति॒ त्रय॑स्त्रि॒वृतो᳚ऽग्निष्टो॒मा भ॑वन्ति॒ वज्र॑स्यै॒व मुखꣳ॒॒ सꣳ श्य॑न्ति॒ दश॑ पञ्चद॒शा भ॑वन्ति पञ्चद॒शो वज्रो॒ - [  ] \newline

\textbf{Pada Paata} \newline

वसि॑ष्ठः । ह॒तपु॑त्र॒ इति॑ ह॒त - पु॒त्रः॒ । अ॒का॒म॒य॒त॒ । वि॒न्देय॑ । प्र॒जामिति॑ प्र-जाम् । अ॒भीति॑ । सौ॒दा॒सान् । भ॒वे॒य॒म् । इति॑ । सः । ए॒तम् । ए॒क॒स्मा॒न्‌न॒प॒ञ्चा॒शमित्ये॑कस्मात् - न॒प॒ञ्चा॒शम् । अ॒प॒श्य॒त् । तम् । एति॑ । अ॒ह॒र॒त् । तेन॑ । अ॒य॒ज॒त॒ । ततः॑ । वै । सः । अवि॑न्दत । प्र॒जामिति॑ प्र-जाम् । अ॒भीति॑ । सौ॒दा॒सान् । अ॒भ॒व॒त् । ये । ए॒वम् । वि॒द्वाꣳसः॑ । ए॒क॒स्मा॒न्न॒प॒ञ्चा॒शमित्ये॑कस्मात्-न॒प॒ञ्चा॒शम् । आस॑ते । वि॒न्दन्ते᳚ । प्र॒जामिति॑ प्र - जाम् । अ॒भीति॑ । भ्रातृ॑व्यान् । भ॒व॒न्ति॒ । त्रयः॑ । त्रि॒वृत॒ इति॑ त्रि - वृतः॑ । अ॒ग्नि॒ष्टो॒मा इत्य॑ग्नि - स्तो॒माः । भ॒व॒न्ति॒ । वज्र॑स्य । ए॒व । मुख᳚म् । समिति॑ । श्य॒न्ति॒ । दश॑ । प॒ञ्च॒द॒शा इति॑ पञ्च - द॒शाः । भ॒व॒न्ति॒ । प॒ञ्च॒द॒श इति॑ पञ्च - द॒शः । वज्रः॑ ।  \newline


\textbf{Krama Paata} \newline

वसि॑ष्ठो ह॒तपु॑त्रः । ह॒तपु॑त्रोऽकामयत । ह॒तपु॑त्र॒ इति॑ ह॒त - पु॒त्रः॒ । अ॒का॒म॒य॒त॒ वि॒न्देय॑ । वि॒न्देय॑ प्र॒जाम् । प्र॒जाम॒भि । प्र॒जामिति॑ प्र - जाम् । अ॒भि सौ॑दा॒सान् । सौ॒दा॒सान् भ॑वेयम् । भ॒वे॒य॒मिति॑ । इति॒ सः । स ए॒तम् । ए॒तमे॑कस्मान्नपञ्चा॒शम् । ए॒क॒स्मा॒न्न॒प॒ञ्चा॒शम॑पश्यत् । ए॒क॒स्मा॒न्न॒प॒ञ्चा॒शमित्ये॑कस्मात् - न॒प॒ञ्चा॒शम् । अ॒प॒श्य॒त् तम् । तमा । आऽह॑रत् । अ॒ह॒र॒त् तेन॑ । तेना॑यजत । अ॒य॒ज॒त॒ ततः॑ । ततो॒ वै । वै सः । सोऽवि॑न्दत । अवि॑न्दत प्र॒जाम् । प्र॒जाम॒भि । प्र॒जामिति॑ प्र - जाम् । अ॒भि सौ॑दा॒सान् । सौ॒दा॒सान॑भवत् । अ॒भ॒व॒द् ये । य ए॒वम् । ए॒वम् ॅवि॒द्वाꣳसः॑ । वि॒द्वाꣳस॑ एकस्मान्नपञ्चा॒शम् । ए॒क॒स्मा॒न्न॒प॒ञ्चा॒शमास॑ते । ए॒क॒स्मा॒न्न॒प॒ञ्चा॒शमित्ये॑कस्मात् - न॒प॒ञ्चा॒शम् । आस॑ते वि॒न्दन्ते᳚ । वि॒न्दन्ते᳚ प्र॒जाम् । प्र॒जाम॒भि । प्र॒जामिति॑ प्र - जाम् । अ॒भि भ्रातृ॑व्यान् । भ्रातृ॑व्यान् भवन्ति । भ॒व॒न्ति॒ त्रयः॑ । त्रय॑स्त्रि॒वृतः॑ । त्रि॒वृतो᳚ऽग्निष्टो॒माः । त्रि॒वृत॒ इति॑ त्रि - वृतः॑ । अ॒ग्नि॒ष्टो॒मा भ॑वन्ति । अ॒ग्नि॒ष्टो॒मा इत्य॑ग्नि - स्तो॒माः । भ॒व॒न्ति॒ वज्र॑स्य । वज्र॑स्यै॒व । ए॒व मुख᳚म् । मुखꣳ॒॒ सम् । सꣳ श्य॑न्ति । श्य॒न्ति॒ दश॑ । दश॑ पञ्चद॒शाः । प॒ञ्च॒द॒शा भ॑वन्ति । प॒ञ्च॒द॒शा इति॑ पञ्च - द॒शाः । भ॒व॒न्ति॒ प॒ञ्च॒द॒शः । प॒ञ्च॒द॒शो वज्रः॑ । प॒ञ्च॒द॒श इति॑ पञ्च - द॒शः । वज्रो॒ वज्र᳚म् \newline

\textbf{Jatai Paata} \newline

1. वसि॑ष्ठो ह॒तपु॑त्रो ह॒तपु॑त्रो॒ वसि॑ष्ठो॒ वसि॑ष्ठो ह॒तपु॑त्रः । \newline
2. ह॒तपु॑त्रो ऽकामयता कामयत ह॒तपु॑त्रो ह॒तपु॑त्रो ऽकामयत । \newline
3. ह॒तपु॑त्र॒ इति॑ ह॒त - पु॒त्रः॒ । \newline
4. अ॒का॒म॒य॒त॒ वि॒न्देय॑ वि॒न्देया॑ कामयता कामयत वि॒न्देय॑ । \newline
5. वि॒न्देय॑ प्र॒जाम् प्र॒जां ॅवि॒न्देय॑ वि॒न्देय॑ प्र॒जाम् । \newline
6. प्र॒जा म॒भ्य॑भि प्र॒जाम् प्र॒जा म॒भि । \newline
7. प्र॒जामिति॑ प्र - जाम् । \newline
8. अ॒भि सौ॑दा॒सान् थ्सौ॑दा॒सान॒ भ्य॑भि सौ॑दा॒सान् । \newline
9. सौ॒दा॒सान् भ॑वेयम् भवेयꣳ सौदा॒सान् थ्सौ॑दा॒सान् भ॑वेयम् । \newline
10. भ॒वे॒य॒ मितीति॑ भवेयम् भवेय॒ मिति॑ । \newline
11. इति॒ स स इतीति॒ सः । \newline
12. स ए॒त मे॒तꣳ स स ए॒तम् । \newline
13. ए॒त मे॑कस्मान्नपञ्चा॒श मे॑कस्मान्नपञ्चा॒श मे॒त मे॒त मे॑कस्मान्नपञ्चा॒शम् । \newline
14. ए॒क॒स्मा॒न्न॒प॒ञ्चा॒श म॑पश्य दपश्य देकस्मान्नपञ्चा॒श मे॑कस्मान्नपञ्चा॒श म॑पश्यत् । \newline
15. ए॒क॒स्मा॒न्न॒प॒ञ्चा॒शमित्ये॑कस्मात् - न॒प॒ञ्चा॒शम् । \newline
16. अ॒प॒श्य॒त् तम् त म॑पश्य दपश्य॒त् तम् । \newline
17. त मा तम् त मा । \newline
18. आ ऽह॑र दहर॒दा ऽह॑रत् । \newline
19. अ॒ह॒र॒त् तेन॒ तेना॑ हर दहर॒त् तेन॑ । \newline
20. तेना॑ यजता यजत॒ तेन॒ तेना॑ यजत । \newline
21. अ॒य॒ज॒त॒ तत॒ स्ततो॑ ऽयजता यजत॒ ततः॑ । \newline
22. ततो॒ वै वै तत॒ स्ततो॒ वै । \newline
23. वै स स वै वै सः । \newline
24. सो ऽवि॑न्द॒ता वि॑न्दत॒ स सो ऽवि॑न्दत । \newline
25. अवि॑न्दत प्र॒जाम् प्र॒जा मवि॑न्द॒ता वि॑न्दत प्र॒जाम् । \newline
26. प्र॒जा म॒भ्य॑भि प्र॒जाम् प्र॒जा म॒भि । \newline
27. प्र॒जामिति॑ प्र - जाम् । \newline
28. अ॒भि सौ॑दा॒सान् थ्सौ॑दा॒सान॒ भ्य॑भि सौ॑दा॒सान् । \newline
29. सौ॒दा॒सा न॑भव दभवथ् सौदा॒सान् थ्सौ॑दा॒सा न॑भवत् । \newline
30. अ॒भ॒व॒द् ये ये॑ ऽभव दभव॒द् ये । \newline
31. य ए॒व मे॒वं ॅये य ए॒वम् । \newline
32. ए॒वं ॅवि॒द्वाꣳसो॑ वि॒द्वाꣳस॑ ए॒व मे॒वं ॅवि॒द्वाꣳसः॑ । \newline
33. वि॒द्वाꣳस॑ एकस्मान्नपञ्चा॒श मे॑कस्मान्नपञ्चा॒शं ॅवि॒द्वाꣳसो॑ वि॒द्वाꣳस॑ एकस्मान्नपञ्चा॒शम् । \newline
34. ए॒क॒स्मा॒न्न॒प॒ञ्चा॒श मास॑त॒ आस॑त एकस्मान्नपञ्चा॒श मे॑कस्मान्नपञ्चा॒श मास॑ते । \newline
35. ए॒क॒स्मा॒न्न॒प॒ञ्चा॒शमित्ये॑कस्मात् - न॒प॒ञ्चा॒शम् । \newline
36. आस॑ते वि॒न्दन्ते॑ वि॒न्दन्त॒ आस॑त॒ आस॑ते वि॒न्दन्ते᳚ । \newline
37. वि॒न्दन्ते᳚ प्र॒जाम् प्र॒जां ॅवि॒न्दन्ते॑ वि॒न्दन्ते᳚ प्र॒जाम् । \newline
38. प्र॒जा म॒भ्य॑भि प्र॒जाम् प्र॒जा म॒भि । \newline
39. प्र॒जामिति॑ प्र - जाम् । \newline
40. अ॒भि भ्रातृ॑व्या॒न् भ्रातृ॑व्यान॒ भ्य॑भि भ्रातृ॑व्यान् । \newline
41. भ्रातृ॑व्यान् भवन्ति भवन्ति॒ भ्रातृ॑व्या॒न् भ्रातृ॑व्यान् भवन्ति । \newline
42. भ॒व॒न्ति॒ त्रय॒ स्त्रयो॑ भवन्ति भवन्ति॒ त्रयः॑ । \newline
43. त्रय॑ स्त्रि॒वृत॑ स्त्रि॒वृत॒ स्त्रय॒ स्त्रय॑ स्त्रि॒वृतः॑ । \newline
44. त्रि॒वृतो᳚ ऽग्निष्टो॒मा अ॑ग्निष्टो॒मा स्त्रि॒वृत॑ स्त्रि॒वृतो᳚ ऽग्निष्टो॒माः । \newline
45. त्रि॒वृत॒ इति॑ त्रि - वृतः॑ । \newline
46. अ॒ग्नि॒ष्टो॒मा भ॑वन्ति भव न्त्यग्निष्टो॒मा अ॑ग्निष्टो॒मा भ॑वन्ति । \newline
47. अ॒ग्नि॒ष्टो॒मा इत्य॑ग्नि - स्तो॒माः । \newline
48. भ॒व॒न्ति॒ वज्र॑स्य॒ वज्र॑स्य भवन्ति भवन्ति॒ वज्र॑स्य । \newline
49. वज्र॑ स्यै॒वैव वज्र॑स्य॒ वज्र॑ स्यै॒व । \newline
50. ए॒व मुख॒म् मुख॑ मे॒वैव मुख᳚म् । \newline
51. मुखꣳ॒॒ सꣳ सम् मुख॒म् मुखꣳ॒॒ सम् । \newline
52. सꣳ श्य॑न्ति श्यन्ति॒ सꣳ सꣳ श्य॑न्ति । \newline
53. श्य॒न्ति॒ दश॒ दश॑ श्यन्ति श्यन्ति॒ दश॑ । \newline
54. दश॑ पञ्चद॒शाः प॑ञ्चद॒शा दश॒ दश॑ पञ्चद॒शाः । \newline
55. प॒ञ्च॒द॒शा भ॑वन्ति भवन्ति पञ्चद॒शाः प॑ञ्चद॒शा भ॑वन्ति । \newline
56. प॒ञ्च॒द॒शा इति॑ पञ्च - द॒शाः । \newline
57. भ॒व॒न्ति॒ प॒ञ्च॒द॒शः प॑ञ्चद॒शो भ॑वन्ति भवन्ति पञ्चद॒शः । \newline
58. प॒ञ्च॒द॒शो वज्रो॒ वज्रः॑ पञ्चद॒शः प॑ञ्चद॒शो वज्रः॑ । \newline
59. प॒ञ्च॒द॒श इति॑ पञ्च - द॒शः । \newline
60. वज्रो॒ वज्रं॒ ॅवज्रं॒ ॅवज्रो॒ वज्रो॒ वज्र᳚म् । \newline

\textbf{Ghana Paata } \newline

1. वसि॑ष्ठो ह॒तपु॑त्रो ह॒तपु॑त्रो॒ वसि॑ष्ठो॒ वसि॑ष्ठो ह॒तपु॑त्रो ऽकामयता कामयत ह॒तपु॑त्रो॒ वसि॑ष्ठो॒ वसि॑ष्ठो ह॒तपु॑त्रो ऽकामयत । \newline
2. ह॒तपु॑त्रो ऽकामयता कामयत ह॒तपु॑त्रो ह॒तपु॑त्रो ऽकामयत वि॒न्देय॑ वि॒न्देया॑ कामयत ह॒तपु॑त्रो ह॒तपु॑त्रो ऽकामयत वि॒न्देय॑ । \newline
3. ह॒तपु॑त्र॒ इति॑ ह॒त - पु॒त्रः॒ । \newline
4. अ॒का॒म॒य॒त॒ वि॒न्देय॑ वि॒न्देया॑ कामयता कामयत वि॒न्देय॑ प्र॒जाम् प्र॒जां ॅवि॒न्देया॑ कामयता कामयत वि॒न्देय॑ प्र॒जाम् । \newline
5. वि॒न्देय॑ प्र॒जाम् प्र॒जां ॅवि॒न्देय॑ वि॒न्देय॑ प्र॒जा म॒भ्य॑भि प्र॒जां ॅवि॒न्देय॑ वि॒न्देय॑ प्र॒जा म॒भि । \newline
6. प्र॒जा म॒भ्य॑भि प्र॒जाम् प्र॒जा म॒भि सौ॑दा॒सान् थ्सौ॑दा॒सा न॒भि प्र॒जाम् प्र॒जा म॒भि सौ॑दा॒सान् । \newline
7. प्र॒जामिति॑ प्र - जाम् । \newline
8. अ॒भि सौ॑दा॒सान् थ्सौ॑दा॒सा न॒भ्य॑भि सौ॑दा॒सान् भ॑वेयम् भवेयꣳ सौदा॒सा न॒भ्य॑भि सौ॑दा॒सान् भ॑वेयम् । \newline
9. सौ॒दा॒सान् भ॑वेयम् भवेयꣳ सौदा॒सान् थ्सौ॑दा॒सान् भ॑वेय॒ मितीति॑ भवेयꣳ सौदा॒सान् थ्सौ॑दा॒सान् भ॑वेय॒ मिति॑ । \newline
10. भ॒वे॒य॒ मितीति॑ भवेयम् भवेय॒ मिति॒ स स इति॑ भवेयम् भवेय॒ मिति॒ सः । \newline
11. इति॒ स स इतीति॒ स ए॒त मे॒तꣳ स इतीति॒ स ए॒तम् । \newline
12. स ए॒त मे॒तꣳ स स ए॒त मे॑कस्मान्नपञ्चा॒श मे॑कस्मान्नपञ्चा॒श मे॒तꣳ स स ए॒त मे॑कस्मान्नपञ्चा॒शम् । \newline
13. ए॒त मे॑कस्मान्नपञ्चा॒श मे॑कस्मान्नपञ्चा॒श मे॒त मे॒त मे॑कस्मान्नपञ्चा॒श म॑पश्य दपश्य देकस्मान्नपञ्चा॒श मे॒त मे॒त मे॑कस्मान्नपञ्चा॒श म॑पश्यत् । \newline
14. ए॒क॒स्मा॒न्न॒प॒ञ्चा॒श म॑पश्य दपश्य देकस्मान्नपञ्चा॒श मे॑कस्मान्नपञ्चा॒श म॑पश्य॒त् तम् त म॑पश्य देकस्मान्नपञ्चा॒श मे॑कस्मान्नपञ्चा॒श म॑पश्य॒त् तम् । \newline
15. ए॒क॒स्मा॒न्न॒प॒ञ्चा॒शमित्ये॑कस्मात् - न॒प॒ञ्चा॒शम् । \newline
16. अ॒प॒श्य॒त् तम् त म॑पश्य दपश्य॒त् त मा त म॑पश्य दपश्य॒त् त मा । \newline
17. त मा तम् त मा ऽह॑र दहर॒दा तम् त मा ऽह॑रत् । \newline
18. आ ऽह॑र दहर॒दा ऽह॑र॒त् तेन॒ तेना॑ हर॒दा ऽह॑र॒त् तेन॑ । \newline
19. अ॒ह॒र॒त् तेन॒ तेना॑ हर दहर॒त् तेना॑ यजता यजत॒ तेना॑ हर दहर॒त् तेना॑ यजत । \newline
20. तेना॑ यजता यजत॒ तेन॒ तेना॑ यजत॒ तत॒ स्ततो॑ ऽयजत॒ तेन॒ तेना॑ यजत॒ ततः॑ । \newline
21. अ॒य॒ज॒त॒ तत॒ स्ततो॑ ऽयजता यजत॒ ततो॒ वै वै ततो॑ ऽयजता यजत॒ ततो॒ वै । \newline
22. ततो॒ वै वै तत॒ स्ततो॒ वै स स वै तत॒ स्ततो॒ वै सः । \newline
23. वै स स वै वै सो ऽवि॑न्द॒ता वि॑न्दत॒ स वै वै सो ऽवि॑न्दत । \newline
24. सो ऽवि॑न्द॒ता वि॑न्दत॒ स सो ऽवि॑न्दत प्र॒जाम् प्र॒जा मवि॑न्दत॒ स सो ऽवि॑न्दत प्र॒जाम् । \newline
25. अवि॑न्दत प्र॒जाम् प्र॒जा मवि॑न्द॒ता वि॑न्दत प्र॒जा म॒भ्य॑भि प्र॒जा मवि॑न्द॒ता वि॑न्दत प्र॒जा म॒भि । \newline
26. प्र॒जा म॒भ्य॑भि प्र॒जाम् प्र॒जा म॒भि सौ॑दा॒सान् थ्सौ॑दा॒सा न॒भि प्र॒जाम् प्र॒जा म॒भि सौ॑दा॒सान् । \newline
27. प्र॒जामिति॑ प्र - जाम् । \newline
28. अ॒भि सौ॑दा॒सान् थ्सौ॑दा॒सा न॒भ्य॑भि सौ॑दा॒सा न॑भव दभवथ् सौदा॒सा न॒भ्य॑भि सौ॑दा॒सा न॑भवत् । \newline
29. सौ॒दा॒सा न॑भव दभवथ् सौदा॒सान् थ्सौ॑दा॒सा न॑भव॒द् ये ये॑ ऽभवथ् सौदा॒सान् थ्सौ॑दा॒सा न॑भव॒द् ये । \newline
30. अ॒भ॒व॒द् ये ये॑ ऽभव दभव॒द् य ए॒व मे॒वं ॅये॑ ऽभव दभव॒द् य ए॒वम् । \newline
31. य ए॒व मे॒वं ॅये य ए॒वं ॅवि॒द्वाꣳसो॑ वि॒द्वाꣳस॑ ए॒वं ॅये य ए॒वं ॅवि॒द्वाꣳसः॑ । \newline
32. ए॒वं ॅवि॒द्वाꣳसो॑ वि॒द्वाꣳस॑ ए॒व मे॒वं ॅवि॒द्वाꣳस॑ एकस्मान्नपञ्चा॒श मे॑कस्मान्नपञ्चा॒शं ॅवि॒द्वाꣳस॑ ए॒व मे॒वं ॅवि॒द्वाꣳस॑ एकस्मान्नपञ्चा॒शम् । \newline
33. वि॒द्वाꣳस॑ एकस्मान्नपञ्चा॒श मे॑कस्मान्नपञ्चा॒शं ॅवि॒द्वाꣳसो॑ वि॒द्वाꣳस॑ एकस्मान्नपञ्चा॒श मास॑त॒ आस॑त एकस्मान्नपञ्चा॒शं ॅवि॒द्वाꣳसो॑ वि॒द्वाꣳस॑ एकस्मान्नपञ्चा॒श मास॑ते । \newline
34. ए॒क॒स्मा॒न्न॒प॒ञ्चा॒श मास॑त॒ आस॑त एकस्मान्नपञ्चा॒श मे॑कस्मान्नपञ्चा॒श मास॑ते वि॒न्दन्ते॑ वि॒न्दन्त॒ आस॑त एकस्मान्नपञ्चा॒श मे॑कस्मान्नपञ्चा॒श मास॑ते वि॒न्दन्ते᳚ । \newline
35. ए॒क॒स्मा॒न्न॒प॒ञ्चा॒शमित्ये॑कस्मात् - न॒प॒ञ्चा॒शम् । \newline
36. आस॑ते वि॒न्दन्ते॑ वि॒न्दन्त॒ आस॑त॒ आस॑ते वि॒न्दन्ते᳚ प्र॒जाम् प्र॒जां ॅवि॒न्दन्त॒ आस॑त॒ आस॑ते वि॒न्दन्ते᳚ प्र॒जाम् । \newline
37. वि॒न्दन्ते᳚ प्र॒जाम् प्र॒जां ॅवि॒न्दन्ते॑ वि॒न्दन्ते᳚ प्र॒जा म॒भ्य॑भि प्र॒जां ॅवि॒न्दन्ते॑ वि॒न्दन्ते᳚ प्र॒जा म॒भि । \newline
38. प्र॒जा म॒भ्य॑भि प्र॒जाम् प्र॒जा म॒भि भ्रातृ॑व्या॒न् भ्रातृ॑व्या न॒भि प्र॒जाम् प्र॒जा म॒भि भ्रातृ॑व्यान् । \newline
39. प्र॒जामिति॑ प्र - जाम् । \newline
40. अ॒भि भ्रातृ॑व्या॒न् भ्रातृ॑व्या न॒भ्य॑भि भ्रातृ॑व्यान् भवन्ति भवन्ति॒ भ्रातृ॑व्या न॒भ्य॑भि भ्रातृ॑व्यान् भवन्ति । \newline
41. भ्रातृ॑व्यान् भवन्ति भवन्ति॒ भ्रातृ॑व्या॒न् भ्रातृ॑व्यान् भवन्ति॒ त्रय॒ स्त्रयो॑ भवन्ति॒ भ्रातृ॑व्या॒न् भ्रातृ॑व्यान् भवन्ति॒ त्रयः॑ । \newline
42. भ॒व॒न्ति॒ त्रय॒ स्त्रयो॑ भवन्ति भवन्ति॒ त्रय॑ स्त्रि॒वृत॑ स्त्रि॒वृत॒ स्त्रयो॑ भवन्ति भवन्ति॒ त्रय॑ स्त्रि॒वृतः॑ । \newline
43. त्रय॑ स्त्रि॒वृत॑ स्त्रि॒वृत॒ स्त्रय॒ स्त्रय॑ स्त्रि॒वृतो᳚ ऽग्निष्टो॒मा अ॑ग्निष्टो॒मा स्त्रि॒वृत॒ स्त्रय॒ स्त्रय॑ स्त्रि॒वृतो᳚ ऽग्निष्टो॒माः । \newline
44. त्रि॒वृतो᳚ ऽग्निष्टो॒मा अ॑ग्निष्टो॒मा स्त्रि॒वृत॑ स्त्रि॒वृतो᳚ ऽग्निष्टो॒मा भ॑वन्ति भव न्त्यग्निष्टो॒मा स्त्रि॒वृत॑ स्त्रि॒वृतो᳚ ऽग्निष्टो॒मा भ॑वन्ति । \newline
45. त्रि॒वृत॒ इति॑ त्रि - वृतः॑ । \newline
46. अ॒ग्नि॒ष्टो॒मा भ॑वन्ति भव न्त्यग्निष्टो॒मा अ॑ग्निष्टो॒मा भ॑वन्ति॒ वज्र॑स्य॒ वज्र॑स्य भव न्त्यग्निष्टो॒मा अ॑ग्निष्टो॒मा भ॑वन्ति॒ वज्र॑स्य । \newline
47. अ॒ग्नि॒ष्टो॒मा इत्य॑ग्नि - स्तो॒माः । \newline
48. भ॒व॒न्ति॒ वज्र॑स्य॒ वज्र॑स्य भवन्ति भवन्ति॒ वज्र॑स्यै॒वैव वज्र॑स्य भवन्ति भवन्ति॒ वज्र॑स्यै॒व । \newline
49. वज्र॑स्यै॒वैव वज्र॑स्य॒ वज्र॑स्यै॒व मुख॒म् मुख॑ मे॒व वज्र॑स्य॒ वज्र॑स्यै॒व मुख᳚म् । \newline
50. ए॒व मुख॒म् मुख॑ मे॒वैव मुखꣳ॒॒ सꣳ सम् मुख॑ मे॒वैव मुखꣳ॒॒ सम् । \newline
51. मुखꣳ॒॒ सꣳ सम् मुख॒म् मुखꣳ॒॒ सꣳ श्य॑न्ति श्यन्ति॒ सम् मुख॒म् मुखꣳ॒॒ सꣳ श्य॑न्ति । \newline
52. सꣳ श्य॑न्ति श्यन्ति॒ सꣳ सꣳ श्य॑न्ति॒ दश॒ दश॑ श्यन्ति॒ सꣳ सꣳ श्य॑न्ति॒ दश॑ । \newline
53. श्य॒न्ति॒ दश॒ दश॑ श्यन्ति श्यन्ति॒ दश॑ पञ्चद॒शाः प॑ञ्चद॒शा दश॑ श्यन्ति श्यन्ति॒ दश॑ पञ्चद॒शाः । \newline
54. दश॑ पञ्चद॒शाः प॑ञ्चद॒शा दश॒ दश॑ पञ्चद॒शा भ॑वन्ति भवन्ति पञ्चद॒शा दश॒ दश॑ पञ्चद॒शा भ॑वन्ति । \newline
55. प॒ञ्च॒द॒शा भ॑वन्ति भवन्ति पञ्चद॒शाः प॑ञ्चद॒शा भ॑वन्ति पञ्चद॒शः प॑ञ्चद॒शो भ॑वन्ति पञ्चद॒शाः प॑ञ्चद॒शा भ॑वन्ति पञ्चद॒शः । \newline
56. प॒ञ्च॒द॒शा इति॑ पञ्च - द॒शाः । \newline
57. भ॒व॒न्ति॒ प॒ञ्च॒द॒शः प॑ञ्चद॒शो भ॑वन्ति भवन्ति पञ्चद॒शो वज्रो॒ वज्रः॑ पञ्चद॒शो भ॑वन्ति भवन्ति पञ्चद॒शो वज्रः॑ । \newline
58. प॒ञ्च॒द॒शो वज्रो॒ वज्रः॑ पञ्चद॒शः प॑ञ्चद॒शो वज्रो॒ वज्रं॒ ॅवज्रं॒ ॅवज्रः॑ पञ्चद॒शः प॑ञ्चद॒शो वज्रो॒ वज्र᳚म् । \newline
59. प॒ञ्च॒द॒श इति॑ पञ्च - द॒शः । \newline
60. वज्रो॒ वज्रं॒ ॅवज्रं॒ ॅवज्रो॒ वज्रो॒ वज्र॑ मे॒वैव वज्रं॒ ॅवज्रो॒ वज्रो॒ वज्र॑ मे॒व । \newline
\pagebreak
\markright{ TS 7.4.7.2  \hfill https://www.vedavms.in \hfill}

\section{ TS 7.4.7.2 }

\textbf{TS 7.4.7.2 } \newline
\textbf{Samhita Paata} \newline

वज्र॑मे॒व भ्रातृ॑व्येभ्यः॒ प्र ह॑रन्ति षोडशि॒म॑द्-दश॒ममह॑-र्भवति॒ वज्र॑ ए॒व वी॒र्यं॑ दधति॒ द्वाद॑श सप्तद॒शा भ॑वन्त्य॒न्नाद्य॒स्या व॑रुद्ध्या॒ अथो॒ प्रैव तैर्जा॑यन्ते॒ पृष्ठ्यः॑ षड॒हो भ॑वति॒ षड्वा ऋ॒तवः॒ षट् पृ॒ष्ठानि॑ पृ॒ष्ठैरे॒वर्तून॒न्वारो॑हन्त्यृ॒तृभिः॑ संॅवथ्स॒रं ते सं॑ॅवथ्स॒र ए॒व प्रति॑ तिष्ठन्ति॒ द्वाद॑शैकविꣳ॒॒शा भ॑वन्ति॒ प्रति॑ष्ठित्या॒ अथो॒ रुच॑मे॒वाऽऽ*त्मन्- [  ] \newline

\textbf{Pada Paata} \newline

वज्र᳚म् । ए॒व । भ्रातृ॑व्येभ्यः । प्रेति॑ । ह॒र॒न्ति॒ । षो॒ड॒शि॒मदिति॑ षोडशि - मत् । द॒श॒मम् । अहः॑ । भ॒व॒ति॒ । वज्रे᳚ । ए॒व । वी॒र्य᳚म् । द॒ध॒ति॒ । द्वाद॑श । स॒प्त॒द॒शा इति॑ सप्त - द॒शाः । भ॒व॒न्ति॒ । अ॒न्नाद्य॒स्येत्य॑न्न- अद्य॑स्य । अव॑रुद्ध्या॒ इत्यव॑-रु॒द्ध्यै॒ । अथो॒ इति॑ । प्रेति॑ । ए॒व । तैः । जा॒य॒न्ते॒ । पृष्ठ्यः॑ । ष॒ड॒ह इति॑ षट् - अ॒हः । भ॒व॒ति॒ । षट् । वै । ऋ॒तवः॑ । षट् । पृ॒ष्ठानि॑ । पृ॒ष्ठैः । ए॒व । ऋ॒तून् । अ॒न्वारो॑ह॒न्तीत्य॑नु-आरो॑हन्ति । ऋ॒तुभि॒रित्यृ॒तु - भिः॒ । सं॒ॅव॒थ्स॒रमिति॑ सं - व॒थ्स॒रम् । ते । सं॒ॅव॒थ्स॒र इति॑ सं - व॒थ्स॒रे । ए॒व । प्रतीति॑ । ति॒ष्ठ॒न्ति॒ । द्वाद॑श । ए॒क॒विꣳ॒॒शा इत्ये॑क - विꣳ॒॒शाः । भ॒व॒न्ति॒ । प्रति॑ष्ठित्या॒ इति॒ प्रति॑ - स्थि॒त्यै॒ । अथो॒ इति॑ । रुच᳚म् । ए॒व । आ॒त्मन्न् ।  \newline


\textbf{Krama Paata} \newline

वज्र॑मे॒व । ए॒व भ्रातृ॑व्येभ्यः । भ्रातृ॑व्येभ्यः॒ प्र । प्र ह॑रन्ति । ह॒र॒न्ति॒ षो॒ड॒शि॒मत् । षो॒ड॒शि॒मद् द॑श॒मम् । षो॒ड॒शि॒मदिति॑ षोडशि - मत् । द॒श॒ममहः॑ । अह॑र् भवति । भ॒व॒ति॒ वज्रे᳚ । वज्र॑ ए॒व । ए॒व वी॒र्य᳚म् । वी॒र्य॑म् दधति । द॒ध॒ति॒ द्वाद॑श । द्वाद॑श सप्तद॒शाः । स॒प्त॒द॒शा भ॑वन्ति । स॒प्त॒द॒शा इति॑ सप्त - द॒शाः । भ॒व॒न्त्य॒न्नाद्य॑स्य । अ॒न्नाद्य॒स्याव॑रुद्ध्यै । अ॒न्नाद्य॒स्येत्य॑न्न - अद्य॑स्य । अव॑रुद्ध्या॒ अथो᳚ । अव॑रुद्ध्या॒ इत्यव॑ - रु॒द्ध्यै॒ । अथो॒ प्र । अथो॒ इत्यथो᳚ । प्रैव । ए॒व तैः । तैर् जा॑यन्ते । जा॒य॒न्ते॒ पृष्ठ्‍यः॑ । पृष्ठ्‍यः॑ षड॒हः । ष॒ड॒हो भ॑वति । ष॒ड॒ह इति॑ षट् - अ॒हः । भ॒व॒ति॒ षट् । षड् वै । वा ऋ॒तवः॑ । ऋ॒तवः॒ षट् । षट् पृ॒ष्ठानि॑ । पृ॒ष्ठानि॑ पृ॒ष्ठैः । पृ॒ष्ठैरे॒व । ए॒वर्तून् । ऋ॒तून॒न्वारो॑हन्ति । अ॒न्वारो॑हन्त्यृ॒तुभिः॑ । अ॒न्वारो॑ह॒न्तीत्य॑नु - आरो॑हन्ति । ऋ॒तुभिः॑ सम्ॅवथ्स॒रम् । ऋ॒तुभि॒रित्यृ॒तु - भिः॒ । स॒म्ॅव॒थ्स॒रम् ते । स॒म्ॅव॒थ्स॒रमिति॑ सम् - व॒थ्स॒रम् । ते स॑म्ॅवथ्स॒रे । स॒म्ॅव॒थ्स॒र ए॒व । स॒म्ॅव॒थ्स॒र इति॑ सम् - व॒थ्स॒रे । ए॒व प्रति॑ । प्रति॑ तिष्ठन्ति । ति॒ष्ठ॒न्ति॒ द्वाद॑श । द्वाद॑शैकविꣳ॒॒शाः । ए॒क॒विꣳ॒॒शा भ॑वन्ति । ए॒क॒विꣳ॒॒शा इत्ये॑क - विꣳ॒॒शाः । भ॒व॒न्ति॒ प्रति॑ष्ठित्यै । प्रति॑ष्ठित्या॒ अथो᳚ । प्रति॑ष्ठित्या॒ इति॒ प्रति॑ - स्थि॒त्यै॒ । अथो॒ रुच᳚म् । अथो॒ इत्यथो᳚ । रुच॑मे॒व । ए॒वात्मन्न् । आ॒त्मन् द॑धते \newline

\textbf{Jatai Paata} \newline

1. वज्र॑ मे॒वैव वज्रं॒ ॅवज्र॑ मे॒व । \newline
2. ए॒व भ्रातृ॑व्येभ्यो॒ भ्रातृ॑व्येभ्य ए॒वैव भ्रातृ॑व्येभ्यः । \newline
3. भ्रातृ॑व्येभ्यः॒ प्र प्र भ्रातृ॑व्येभ्यो॒ भ्रातृ॑व्येभ्यः॒ प्र । \newline
4. प्र ह॑रन्ति हरन्ति॒ प्र प्र ह॑रन्ति । \newline
5. ह॒र॒न्ति॒ षो॒ड॒शि॒म थ्षो॑डशि॒म द्ध॑रन्ति हरन्ति षोडशि॒मत् । \newline
6. षो॒ड॒शि॒मद् द॑श॒मम् द॑श॒मꣳ षो॑डशि॒म थ्षो॑डशि॒मद् द॑श॒मम् । \newline
7. षो॒ड॒शि॒मदिति॑ षोडशि - मत् । \newline
8. द॒श॒म मह॒ रह॑र् दश॒मम् द॑श॒म महः॑ । \newline
9. अह॑र् भवति भव॒ त्यह॒ रह॑र् भवति । \newline
10. भ॒व॒ति॒ वज्रे॒ वज्रे॑ भवति भवति॒ वज्रे᳚ । \newline
11. वज्र॑ ए॒वैव वज्रे॒ वज्र॑ ए॒व । \newline
12. ए॒व वी॒र्यं॑ ॅवी॒र्य॑ मे॒वैव वी॒र्य᳚म् । \newline
13. वी॒र्य॑म् दधति दधति वी॒र्यं॑ ॅवी॒र्य॑म् दधति । \newline
14. द॒ध॒ति॒ द्वाद॑श॒ द्वाद॑श दधति दधति॒ द्वाद॑श । \newline
15. द्वाद॑श सप्तद॒शाः स॑प्तद॒शा द्वाद॑श॒ द्वाद॑श सप्तद॒शाः । \newline
16. स॒प्त॒द॒शा भ॑वन्ति भवन्ति सप्तद॒शाः स॑प्तद॒शा भ॑वन्ति । \newline
17. स॒प्त॒द॒शा इति॑ सप्त - द॒शाः । \newline
18. भ॒व॒न् त्य॒न्नाद्य॑स्या॒ न्नाद्य॑स्य भवन्ति भव न्त्य॒न्नाद्य॑स्य । \newline
19. अ॒न्नाद्य॒स्या व॑रुद्ध्या॒ अव॑रुद्ध्या अ॒न्नाद्य॑स्या॒ न्नाद्य॒स्या व॑रुद्ध्यै । \newline
20. अ॒न्नाद्य॒स्येत्य॑न्न - अद्य॑स्य । \newline
21. अव॑रुद्ध्या॒ अथो॒ अथो॒ अव॑रुद्ध्या॒ अव॑रुद्ध्या॒ अथो᳚ । \newline
22. अव॑रुद्ध्या॒ इत्यव॑ - रु॒द्ध्यै॒ । \newline
23. अथो॒ प्र प्राथो॒ अथो॒ प्र । \newline
24. अथो॒ इत्यथो᳚ । \newline
25. प्रैवैव प्र प्रैव । \newline
26. ए॒व तै स्तै रे॒वैव तैः । \newline
27. तैर् जा॑यन्ते जायन्ते॒ तै स्तैर् जा॑यन्ते । \newline
28. जा॒य॒न्ते॒ पृष्ठ्यः॒ पृष्ठ्यो॑ जायन्ते जायन्ते॒ पृष्ठ्यः॑ । \newline
29. पृष्ठ्य॑ ष्षड॒ह ष्ष॑ड॒हः पृष्ठ्यः॒ पृष्ठ्य॑ ष्षड॒हः । \newline
30. ष॒ड॒हो भ॑वति भवति षड॒ह ष्ष॑ड॒हो भ॑वति । \newline
31. ष॒ड॒ह इति॑ षट् - अ॒हः । \newline
32. भ॒व॒ति॒ षट् थ्षड् भ॑वति भवति॒ षट् । \newline
33. षड् वै वै षट् थ्षड् वै । \newline
34. वा ऋ॒तव॑ ऋ॒तवो॒ वै वा ऋ॒तवः॑ । \newline
35. ऋ॒तव॒ ष्षट् थ्षडृ॒तव॑ ऋ॒तव॒ ष्षट् । \newline
36. षट् पृ॒ष्ठानि॑ पृ॒ष्ठानि॒ षट् थ्षट् पृ॒ष्ठानि॑ । \newline
37. पृ॒ष्ठानि॑ पृ॒ष्ठैः पृ॒ष्ठैः पृ॒ष्ठानि॑ पृ॒ष्ठानि॑ पृ॒ष्ठैः । \newline
38. पृ॒ष्ठै रे॒वैव पृ॒ष्ठैः पृ॒ष्ठै रे॒व । \newline
39. ए॒व र्‌तू नृ॒तूने॒ वैव र्‌तून् । \newline
40. ऋ॒तू न॒न्वारो॑ह न्त्य॒न्वारो॑हन् त्यृ॒तू नृ॒तू न॒न्वारो॑हन्ति । \newline
41. अ॒न्वारो॑ह न्त्यृ॒तुभिर्॑. ऋ॒तुभि॑ र॒न्वारो॑ह न्त्य॒न्वारो॑ह न्त्यृ॒तुभिः॑ । \newline
42. अ॒न्वारो॑ह॒न्तीत्य॑नु - आरो॑हन्ति । \newline
43. ऋ॒तुभिः॑ संॅवथ्स॒रꣳ सं॑ॅवथ्स॒र मृ॒तुभिर्॑. ऋ॒तुभिः॑ संॅवथ्स॒रम् । \newline
44. ऋ॒तुभि॒रित्यृ॒तु - भिः॒ । \newline
45. सं॒ॅव॒थ्स॒रम् ते ते सं॑ॅवथ्स॒रꣳ सं॑ॅवथ्स॒रम् ते । \newline
46. सं॒ॅव॒थ्स॒रमिति॑ सं - व॒थ्स॒रम् । \newline
47. ते सं॑ॅवथ्स॒रे सं॑ॅवथ्स॒रे ते ते सं॑ॅवथ्स॒रे । \newline
48. सं॒ॅव॒थ्स॒र ए॒वैव सं॑ॅवथ्स॒रे सं॑ॅवथ्स॒र ए॒व । \newline
49. सं॒ॅव॒थ्स॒र इति॑ सं - व॒थ्स॒रे । \newline
50. ए॒व प्रति॒ प्रत्ये॒वैव प्रति॑ । \newline
51. प्रति॑ तिष्ठन्ति तिष्ठन्ति॒ प्रति॒ प्रति॑ तिष्ठन्ति । \newline
52. ति॒ष्ठ॒न्ति॒ द्वाद॑श॒ द्वाद॑श तिष्ठन्ति तिष्ठन्ति॒ द्वाद॑श । \newline
53. द्वाद॑ शैकविꣳ॒॒शा ए॑कविꣳ॒॒शा द्वाद॑श॒ द्वाद॑ शैकविꣳ॒॒शाः । \newline
54. ए॒क॒विꣳ॒॒शा भ॑वन्ति भव न्त्येकविꣳ॒॒शा ए॑कविꣳ॒॒शा भ॑वन्ति । \newline
55. ए॒क॒विꣳ॒॒शा इत्ये॑क - विꣳ॒॒शाः । \newline
56. भ॒व॒न्ति॒ प्रति॑ष्ठित्यै॒ प्रति॑ष्ठित्यै भवन्ति भवन्ति॒ प्रति॑ष्ठित्यै । \newline
57. प्रति॑ष्ठित्या॒ अथो॒ अथो॒ प्रति॑ष्ठित्यै॒ प्रति॑ष्ठित्या॒ अथो᳚ । \newline
58. प्रति॑ष्ठित्या॒ इति॒ प्रति॑ - स्थि॒त्यै॒ । \newline
59. अथो॒ रुचꣳ॒॒ रुच॒ मथो॒ अथो॒ रुच᳚म् । \newline
60. अथो॒ इत्यथो᳚ । \newline
61. रुच॑ मे॒वैव रुचꣳ॒॒ रुच॑ मे॒व । \newline
62. ए॒वात्मन् ना॒त्मन् ने॒वै वात्मन्न् । \newline
63. आ॒त्मन् द॑धते दधत आ॒त्मन् ना॒त्मन् द॑धते । \newline

\textbf{Ghana Paata } \newline

1. वज्र॑ मे॒वैव वज्रं॒ ॅवज्र॑ मे॒व भ्रातृ॑व्येभ्यो॒ भ्रातृ॑व्येभ्य ए॒व वज्रं॒ ॅवज्र॑ मे॒व भ्रातृ॑व्येभ्यः । \newline
2. ए॒व भ्रातृ॑व्येभ्यो॒ भ्रातृ॑व्येभ्य ए॒वैव भ्रातृ॑व्येभ्यः॒ प्र प्र भ्रातृ॑व्येभ्य ए॒वैव भ्रातृ॑व्येभ्यः॒ प्र । \newline
3. भ्रातृ॑व्येभ्यः॒ प्र प्र भ्रातृ॑व्येभ्यो॒ भ्रातृ॑व्येभ्यः॒ प्र ह॑रन्ति हरन्ति॒ प्र भ्रातृ॑व्येभ्यो॒ भ्रातृ॑व्येभ्यः॒ प्र ह॑रन्ति । \newline
4. प्र ह॑रन्ति हरन्ति॒ प्र प्र ह॑रन्ति षोडशि॒म थ्षो॑डशि॒म द्ध॑रन्ति॒ प्र प्र ह॑रन्ति षोडशि॒मत् । \newline
5. ह॒र॒न्ति॒ षो॒ड॒शि॒म थ्षो॑डशि॒म द्ध॑रन्ति हरन्ति षोडशि॒मद् द॑श॒मम् द॑श॒मꣳ षो॑डशि॒म
द्ध॑रन्ति हरन्ति षोडशि॒मद् द॑श॒मम् । \newline
6. षो॒ड॒शि॒मद् द॑श॒मम् द॑श॒मꣳ षो॑डशि॒म थ्षो॑डशि॒मद् द॑श॒म मह॒ रह॑र् दश॒मꣳ षो॑डशि॒म थ्षो॑डशि॒मद् द॑श॒म महः॑ । \newline
7. षो॒ड॒शि॒मदिति॑ षोडशि - मत् । \newline
8. द॒श॒म मह॒ रह॑र् दश॒मम् द॑श॒म मह॑र् भवति भव॒ त्यह॑र् दश॒मम् द॑श॒म मह॑र् भवति । \newline
9. अह॑र् भवति भव॒ त्यह॒ रह॑र् भवति॒ वज्रे॒ वज्रे॑ भव॒ त्यह॒ रह॑र् भवति॒ वज्रे᳚ । \newline
10. भ॒व॒ति॒ वज्रे॒ वज्रे॑ भवति भवति॒ वज्र॑ ए॒वैव वज्रे॑ भवति भवति॒ वज्र॑ ए॒व । \newline
11. वज्र॑ ए॒वैव वज्रे॒ वज्र॑ ए॒व वी॒र्यं॑ ॅवी॒र्य॑ मे॒व वज्रे॒ वज्र॑ ए॒व वी॒र्य᳚म् । \newline
12. ए॒व वी॒र्यं॑ ॅवी॒र्य॑ मे॒वैव वी॒र्य॑म् दधति दधति वी॒र्य॑ मे॒वैव वी॒र्य॑म् दधति । \newline
13. वी॒र्य॑म् दधति दधति वी॒र्यं॑ ॅवी॒र्य॑म् दधति॒ द्वाद॑श॒ द्वाद॑श दधति वी॒र्यं॑ ॅवी॒र्य॑म् दधति॒ द्वाद॑श । \newline
14. द॒ध॒ति॒ द्वाद॑श॒ द्वाद॑श दधति दधति॒ द्वाद॑श सप्तद॒शाः स॑प्तद॒शा द्वाद॑श दधति दधति॒ द्वाद॑श सप्तद॒शाः । \newline
15. द्वाद॑श सप्तद॒शाः स॑प्तद॒शा द्वाद॑श॒ द्वाद॑श सप्तद॒शा भ॑वन्ति भवन्ति सप्तद॒शा द्वाद॑श॒ द्वाद॑श सप्तद॒शा भ॑वन्ति । \newline
16. स॒प्त॒द॒शा भ॑वन्ति भवन्ति सप्तद॒शाः स॑प्तद॒शा भ॑व न्त्य॒न्नाद्य॑स्या॒ न्नाद्य॑स्य भवन्ति सप्तद॒शाः स॑प्तद॒शा भ॑व न्त्य॒न्नाद्य॑स्य । \newline
17. स॒प्त॒द॒शा इति॑ सप्त - द॒शाः । \newline
18. भ॒व॒ न्त्य॒न्नाद्य॑स्या॒ न्नाद्य॑स्य भवन्ति भव न्त्य॒न्नाद्य॒स्या व॑रुद्ध्या॒ अव॑रुद्ध्या अ॒न्नाद्य॑स्य भवन्ति भव न्त्य॒न्नाद्य॒स्या व॑रुद्ध्यै । \newline
19. अ॒न्नाद्य॒स्या व॑रुद्ध्या॒ अव॑रुद्ध्या अ॒न्नाद्य॑स्या॒ न्नाद्य॒स्या व॑रुद्ध्या॒ अथो॒ अथो॒ अव॑रुद्ध्या अ॒न्नाद्य॑स्या॒ न्नाद्य॒स्या व॑रुद्ध्या॒ अथो᳚ । \newline
20. अ॒न्नाद्य॒स्येत्य॑न्न - अद्य॑स्य । \newline
21. अव॑रुद्ध्या॒ अथो॒ अथो॒ अव॑रुद्ध्या॒ अव॑रुद्ध्या॒ अथो॒ प्र प्राथो॒ अव॑रुद्ध्या॒ अव॑रुद्ध्या॒ अथो॒ प्र । \newline
22. अव॑रुद्ध्या॒ इत्यव॑ - रु॒द्ध्यै॒ । \newline
23. अथो॒ प्र प्राथो॒ अथो॒ प्रैवैव प्राथो॒ अथो॒ प्रैव । \newline
24. अथो॒ इत्यथो᳚ । \newline
25. प्रैवैव प्र प्रैव तै स्तै रे॒व प्र प्रैव तैः । \newline
26. ए॒व तै स्तै रे॒वैव तैर् जा॑यन्ते जायन्ते॒ तै रे॒वैव तैर् जा॑यन्ते । \newline
27. तैर् जा॑यन्ते जायन्ते॒ तै स्तैर् जा॑यन्ते॒ पृष्ठ्यः॒ पृष्ठ्यो॑ जायन्ते॒ तै स्तैर् जा॑यन्ते॒ पृष्ठ्यः॑ । \newline
28. जा॒य॒न्ते॒ पृष्ठ्यः॒ पृष्ठ्यो॑ जायन्ते जायन्ते॒ पृष्ठ्य॑ ष्षड॒ह ष्ष॑ड॒हः पृष्ठ्यो॑ जायन्ते जायन्ते॒ पृष्ठ्य॑ ष्षड॒हः । \newline
29. पृष्ठ्य॑ ष्षड॒ह ष्ष॑ड॒हः पृष्ठ्यः॒ पृष्ठ्य॑ ष्षड॒हो भ॑वति भवति षड॒हः पृष्ठ्यः॒ पृष्ठ्य॑ ष्षड॒हो भ॑वति । \newline
30. ष॒ड॒हो भ॑वति भवति षड॒ह ष्ष॑ड॒हो भ॑वति॒ षट् थ्षड् भ॑वति षड॒ह ष्ष॑ड॒हो भ॑वति॒ षट् । \newline
31. ष॒ड॒ह इति॑ षट् - अ॒हः । \newline
32. भ॒व॒ति॒ षट् थ्षड् भ॑वति भवति॒ षड् वै वै षड् भ॑वति भवति॒ षड् वै । \newline
33. षड् वै वै षट् थ्षड् वा ऋ॒तव॑ ऋ॒तवो॒ वै षट् थ्षड् वा ऋ॒तवः॑ । \newline
34. वा ऋ॒तव॑ ऋ॒तवो॒ वै वा ऋ॒तव॒ ष्षट् थ्षडृ॒तवो॒ वै वा ऋ॒तव॒ ष्षट् । \newline
35. ऋ॒तव॒ ष्षट् थ्षडृ॒तव॑ ऋ॒तव॒ ष्षट् पृ॒ष्ठानि॑ पृ॒ष्ठानि॒ षडृ॒तव॑ ऋ॒तव॒ ष्षट् पृ॒ष्ठानि॑ । \newline
36. षट् पृ॒ष्ठानि॑ पृ॒ष्ठानि॒ षट् थ्षट् पृ॒ष्ठानि॑ पृ॒ष्ठैः पृ॒ष्ठैः पृ॒ष्ठानि॒ षट् थ्षट् पृ॒ष्ठानि॑ पृ॒ष्ठैः । \newline
37. पृ॒ष्ठानि॑ पृ॒ष्ठैः पृ॒ष्ठैः पृ॒ष्ठानि॑ पृ॒ष्ठानि॑ पृ॒ष्ठै रे॒वैव पृ॒ष्ठैः पृ॒ष्ठानि॑ पृ॒ष्ठानि॑ पृ॒ष्ठै रे॒व । \newline
38. पृ॒ष्ठै रे॒वैव पृ॒ष्ठैः पृ॒ष्ठै रे॒व र्‌तू नृ॒तू ने॒व पृ॒ष्ठैः पृ॒ष्ठै रे॒व र्‌तून् । \newline
39. ए॒व र्‌तू नृ॒तू ने॒वैव र्‌तू न॒न्वारो॑ह न्त्य॒न्वारो॑ह न्त्यृ॒तू ने॒वैव र्‌तू न॒न्वारो॑हन्ति । \newline
40. ऋ॒तू न॒न्वारो॑ह न्त्य॒न्वारो॑ह न्त्यृ॒तू नृ॒तू न॒न्वारो॑ह न्त्यृ॒तुभिर्॑. ऋ॒तुभि॑ र॒न्वारो॑ह न्त्यृ॒तू नृ॒तू न॒न्वारो॑ह न्त्यृ॒तुभिः॑ । \newline
41. अ॒न्वारो॑ह न्त्यृ॒तुभिर्॑. ऋ॒तुभि॑ र॒न्वारो॑ह न्त्य॒न्वारो॑ह न्त्यृ॒तुभिः॑ संॅवथ्स॒रꣳ सं॑ॅवथ्स॒र मृ॒तुभि॑ र॒न्वारो॑ह न्त्य॒न्वारो॑ह न्त्यृ॒तुभिः॑ संॅवथ्स॒रम् । \newline
42. अ॒न्वारो॑ह॒न्तीत्य॑नु - आरो॑हन्ति । \newline
43. ऋ॒तुभिः॑ संॅवथ्स॒रꣳ सं॑ॅवथ्स॒र मृ॒तुभिर्॑. ऋ॒तुभिः॑ संॅवथ्स॒रम् ते ते सं॑ॅवथ्स॒र मृ॒तुभिर्॑. ऋ॒तुभिः॑ संॅवथ्स॒रम् ते । \newline
44. ऋ॒तुभि॒रित्यृ॒तु - भिः॒ । \newline
45. सं॒ॅव॒थ्स॒रम् ते ते सं॑ॅवथ्स॒रꣳ सं॑ॅवथ्स॒रम् ते सं॑ॅवथ्स॒रे सं॑ॅवथ्स॒रे ते सं॑ॅवथ्स॒रꣳ सं॑ॅवथ्स॒रम् ते सं॑ॅवथ्स॒रे । \newline
46. सं॒ॅव॒थ्स॒रमिति॑ सं - व॒थ्स॒रम् । \newline
47. ते सं॑ॅवथ्स॒रे सं॑ॅवथ्स॒रे ते ते सं॑ॅवथ्स॒र ए॒वैव सं॑ॅवथ्स॒रे ते ते सं॑ॅवथ्स॒र ए॒व । \newline
48. सं॒ॅव॒थ्स॒र ए॒वैव सं॑ॅवथ्स॒रे सं॑ॅवथ्स॒र ए॒व प्रति॒ प्रत्ये॒व सं॑ॅवथ्स॒रे सं॑ॅवथ्स॒र ए॒व प्रति॑ । \newline
49. सं॒ॅव॒थ्स॒र इति॑ सं - व॒थ्स॒रे । \newline
50. ए॒व प्रति॒ प्रत्ये॒वैव प्रति॑ तिष्ठन्ति तिष्ठन्ति॒ प्रत्ये॒वैव प्रति॑ तिष्ठन्ति । \newline
51. प्रति॑ तिष्ठन्ति तिष्ठन्ति॒ प्रति॒ प्रति॑ तिष्ठन्ति॒ द्वाद॑श॒ द्वाद॑श तिष्ठन्ति॒ प्रति॒ प्रति॑ तिष्ठन्ति॒ द्वाद॑श । \newline
52. ति॒ष्ठ॒न्ति॒ द्वाद॑श॒ द्वाद॑श तिष्ठन्ति तिष्ठन्ति॒ द्वाद॑शैकविꣳ॒॒शा ए॑कविꣳ॒॒शा द्वाद॑श तिष्ठन्ति तिष्ठन्ति॒ द्वाद॑शैकविꣳ॒॒शाः । \newline
53. द्वाद॑शै कविꣳ॒॒शा ए॑कविꣳ॒॒शा द्वाद॑श॒ द्वाद॑शैकविꣳ॒॒शा भ॑वन्ति भव न्त्येकविꣳ॒॒शा द्वाद॑श॒ द्वाद॑शैकविꣳ॒॒शा भ॑वन्ति । \newline
54. ए॒क॒विꣳ॒॒शा भ॑वन्ति भव न्त्येकविꣳ॒॒शा ए॑कविꣳ॒॒शा भ॑वन्ति॒ प्रति॑ष्ठित्यै॒ प्रति॑ष्ठित्यै भव न्त्येकविꣳ॒॒शा ए॑कविꣳ॒॒शा भ॑वन्ति॒ प्रति॑ष्ठित्यै । \newline
55. ए॒क॒विꣳ॒॒शा इत्ये॑क - विꣳ॒॒शाः । \newline
56. भ॒व॒न्ति॒ प्रति॑ष्ठित्यै॒ प्रति॑ष्ठित्यै भवन्ति भवन्ति॒ प्रति॑ष्ठित्या॒ अथो॒ अथो॒ प्रति॑ष्ठित्यै भवन्ति भवन्ति॒ प्रति॑ष्ठित्या॒ अथो᳚ । \newline
57. प्रति॑ष्ठित्या॒ अथो॒ अथो॒ प्रति॑ष्ठित्यै॒ प्रति॑ष्ठित्या॒ अथो॒ रुचꣳ॒॒ रुच॒ मथो॒ प्रति॑ष्ठित्यै॒ प्रति॑ष्ठित्या॒ अथो॒ रुच᳚म् । \newline
58. प्रति॑ष्ठित्या॒ इति॒ प्रति॑ - स्थि॒त्यै॒ । \newline
59. अथो॒ रुचꣳ॒॒ रुच॒ मथो॒ अथो॒ रुच॑ मे॒वैव रुच॒ मथो॒ अथो॒ रुच॑ मे॒व । \newline
60. अथो॒ इत्यथो᳚ । \newline
61. रुच॑ मे॒वैव रुचꣳ॒॒ रुच॑ मे॒वात्मन् ना॒त्मन् ने॒व रुचꣳ॒॒ रुच॑ मे॒वात्मन्न् । \newline
62. ए॒वात्मन् ना॒त्मन् ने॒वै वात्मन् द॑धते दधत आ॒त्मन् ने॒वै वात्मन् द॑धते । \newline
63. आ॒त्मन् द॑धते दधत आ॒त्मन् ना॒त्मन् द॑धते ब॒हवो॑ ब॒हवो॑ दधत आ॒त्मन् ना॒त्मन् द॑धते ब॒हवः॑ । \newline
\pagebreak
\markright{ TS 7.4.7.3  \hfill https://www.vedavms.in \hfill}

\section{ TS 7.4.7.3 }

\textbf{TS 7.4.7.3 } \newline
\textbf{Samhita Paata} \newline

द॑धते ब॒हवः॑ षोड॒शिनो॑ भवन्ति॒ विजि॑त्यै॒ षडा᳚श्वि॒नानि॑ भवन्ति॒ षड्वा ऋ॒तव॑ ऋ॒तुष्वे॒व प्रति॑ तिष्ठन्त्यूनातिरि॒क्ता वा ए॒ता रात्र॑य ऊ॒नास्तद्-यदेक॑स्यै॒ न प॑ञ्चा॒शद-ति॑रिक्ता॒स्तद्-यद्-भूय॑सी-र॒ष्टाच॑त्वारिꣳशत ऊ॒नाच्च॒ खलु॒ वा अति॑रिक्ताच्च प्र॒जाप॑तिः॒ प्राजा॑यत॒ ये प्र॒जाका॑माः प॒शुका॑माः॒ स्युस्त ए॒ता आ॑सीर॒न् प्रैव जा॑यन्ते प्र॒जया॑ ( ) प॒शुभि॑र्वैरा॒जो वा ए॒ष य॒ज्ञो यदे॑कस्मा-न्नपञ्चा॒शो य ए॒वं ॅवि॒द्वाꣳस॑ एकस्मा-न्नपञ्चा॒शमास॑ते वि॒राज॑मे॒व ग॑च्छन्त्यन्ना॒दा भ॑वन्त्यति-रा॒त्राव॒भितो॑ भवतो॒ऽन्नाद्य॑स्य॒ परि॑गृहीत्यै ॥ \newline

\textbf{Pada Paata} \newline

द॒ध॒ते॒ । ब॒हवः॑ । षो॒ड॒शिनः॑ । भ॒व॒न्ति॒ । विजि॑त्या॒ इति॒ वि - जि॒त्यै॒ । षट् । आ॒श्वि॒नानि॑ । भ॒व॒न्ति॒ । षट् । वै । ऋ॒तवः॑ । ऋ॒तुषु॑ । ए॒व । प्रतीति॑ । ति॒ष्ठ॒न्ति॒ । ऊ॒ना॒ति॒रि॒क्ता इत्यू॑न - अ॒ति॒रि॒क्ताः । वै । ए॒ताः । रात्र॑यः । ऊ॒नाः । तत् । यत् । एक॑स्यै । न । प॒ञ्चा॒शत् । अति॑रिक्ता॒ इत्यति॑ - रि॒क्ताः॒ । तत् । यत् । भूय॑सीः । अ॒ष्टाच॑त्वारिꣳशत॒ इत्य॒ष्टा-च॒त्वा॒रिꣳ॒॒श॒तः॒ । ऊ॒नात् । च॒ । खलु॑ । वै । अति॑रिक्ता॒दित्यति॑-रि॒क्ता॒त् । च॒ । प्र॒जाप॑ति॒रिति॑ प्र॒जा - प॒तिः॒ । प्रेति॑ । अ॒जा॒य॒त॒ । ये । प्र॒जाका॑मा॒ इति॑ प्र॒जा - का॒माः॒ । प॒शुका॑मा॒ इति॑ प॒शु - का॒माः॒ । स्युः । ते । ए॒ताः । आ॒सी॒र॒न्न् । प्रेति॑ । ए॒व । जा॒य॒न्ते॒ । प्र॒जयेति॑ प्र - जया᳚ ( ) । प॒शुभि॒रिति॑ प॒शु - भिः॒ । वै॒रा॒जः । वै । ए॒षः । य॒ज्ञ्ः । यत् । ए॒क॒स्मा॒न्न॒प॒ञ्चा॒श इत्ये॑कस्मात् - न॒प॒ञ्चा॒शः । ये । ए॒वम् । वि॒द्वाꣳसः॑ । ए॒क॒स्मा॒न्न॒प॒ञ्चा॒शमित्ये॑कस्मात् - न॒प॒ञ्चा॒शम् । आस॑ते । वि॒राज॒मिति॑ वि - राज᳚म् । ए॒व । ग॒च्छ॒न्ति॒ । अ॒न्ना॒दा इत्य॑न्न-अ॒दाः । भ॒व॒न्ति॒ । अ॒ति॒रा॒त्रावित्य॑ति - रा॒त्रौ । अ॒भितः॑ । भ॒व॒तः॒ । अ॒न्नाद्य॒स्येत्य॑न्न-अद्य॑स्य । परि॑गृहीत्या॒ इति॒ परि॑-गृ॒ही॒त्यै॒ ॥  \newline


\textbf{Krama Paata} \newline

द॒ध॒ते॒ ब॒हवः॑ । ब॒हवः॑ षोड॒शिनः॑ । षो॒ड॒शिनो॑ भवन्ति । भ॒व॒न्ति॒ विजि॑त्यै । विजि॑त्यै॒ षट् । विजि॑त्या॒ इति॒ वि - जि॒त्यै॒ । षडा᳚श्वि॒नानि॑ । आ॒श्वि॒नानि॑ भवन्ति । भ॒व॒न्ति॒ षट् । षड् वै । वा ऋ॒तवः॑ । ऋ॒तव॑ ऋ॒तुषु॑ । ऋ॒तुष्वे॒व । ए॒व प्रति॑ । प्रति॑ तिष्ठन्ति । ति॒ष्ठ॒न्त्यू॒ना॒ति॒रि॒क्ताः । ऊ॒ना॒ति॒रि॒क्ता वै । ऊ॒ना॒ति॒रि॒क्ता इत्यू॑न - अ॒ति॒रि॒क्ताः । वा ए॒ताः । ए॒ता रात्र॑यः । रात्र॑य ऊ॒नाः । ऊ॒नास्तत् । तद् यत् । यदेक॑स्यै । एक॑स्यै॒ न । न प॑ञ्चा॒शत् । प॒ञ्चा॒शदति॑रिक्ताः । अति॑रिक्ता॒स्तत् । अति॑रिक्ता॒ इत्यति॑ - रि॒क्ताः॒ । तद् यत् । यद् भूय॑सीः । भूय॑सीर॒ष्टाच॑त्वारिꣳशतः । अ॒ष्टाच॑त्वारिꣳशत ऊ॒नात् । अ॒ष्टाच॑त्वारिꣳशत॒ इत्य॒ष्टा - च॒त्वा॒रिꣳ॒॒श॒तः॒ । ऊ॒नाच् च॑ । च॒ खलु॑ । खलु॒ वै । वा अति॑रिक्तात् । अति॑रिक्ताच् च । अति॑रिक्ता॒दित्यति॑ - रि॒क्ता॒त्॒ । च॒ प्र॒जाप॑तिः । प्र॒जाप॑तिः॒ प्र । प्र॒जाप॑ति॒रिति॑ प्र॒जा - प॒तिः॒ । प्राजा॑यत । अ॒जा॒य॒त॒ ये । ये प्र॒जाका॑माः । प्र॒जाका॑माः प॒शुका॑माः । प्र॒जाका॑मा॒ इति॑ प्र॒जा - का॒माः॒ । प॒शुका॑माः॒ स्युः । प॒शुका॑मा॒ इति॑ प॒शु - का॒माः॒ । स्युस्ते । त ए॒ताः । ए॒ता आ॑सीरन्न् । आ॒सी॒र॒न् प्र । प्रैव । ए॒व जा॑यन्ते । जा॒य॒न्ते॒ प्र॒जया᳚ ( ) । प्र॒जया॑ प॒शुभिः॑ । प्र॒जयेति॑ प्र - जया᳚ । प॒शुभि॑र् वैरा॒जः । प॒शुभि॒रिति॑ प॒शु - भिः॒ । वै॒रा॒जो वै । वा ए॒षः । ए॒ष य॒ज्ञ्ः । य॒ज्ञो यत् । यदे॑कस्मान्नपञ्चा॒शः । ए॒क॒स्मा॒न्न॒प॒ञ्चा॒शो ये । ए॒क॒स्मा॒न्न॒प॒ञ्चा॒श इत्ये॑कस्मात् - न॒प॒ञ्चा॒शः । य ए॒वम् । ए॒वम् ॅवि॒द्वाꣳसः॑ । वि॒द्वाꣳस॑ एकस्मान्नपञ्चा॒शम् । ए॒क॒स्मा॒न्न॒प॒ञ्चा॒शमास॑ते । ए॒क॒स्मा॒न्न॒प॒ञ्चा॒शमित्ये॑कस्मात् - न॒प॒ञ्चा॒शम् । आस॑ते वि॒राज᳚म् । वि॒राज॑मे॒व । वि॒राज॒मिति॑ वि - राज᳚म् । ए॒व ग॑च्छन्ति । ग॒च्छ॒न्त्य॒न्ना॒दाः । अ॒न्ना॒दा भ॑वन्ति । अ॒न्ना॒दा इत्य॑न्न - अ॒दाः । भ॒व॒न्त्य॒ति॒रा॒त्रौ । अ॒ति॒रा॒त्राव॒भितः॑ । अ॒ति॒रा॒त्रावित्य॑ति - रा॒त्रौ । अ॒भितो॑ भवतः । भ॒व॒तो॒ऽन्नाद्य॑स्य । अ॒न्नाद्य॑स्य॒ परि॑गृहीत्यै । अ॒न्नाद्य॒स्येत्य॑न्न - अद्य॑स्य । परि॑गृहीत्या॒ इति॒ परि॑ - गृ॒ही॒त्यै॒ । \newline

\textbf{Jatai Paata} \newline

1. द॒ध॒ते॒ ब॒हवो॑ ब॒हवो॑ दधते दधते ब॒हवः॑ । \newline
2. ब॒हव॑ ष्षोड॒शिन॑ ष्षोड॒शिनो॑ ब॒हवो॑ ब॒हव॑ ष्षोड॒शिनः॑ । \newline
3. षो॒ड॒शिनो॑ भवन्ति भवन्ति षोड॒शिन॑ ष्षोड॒शिनो॑ भवन्ति । \newline
4. भ॒व॒न्ति॒ विजि॑त्यै॒ विजि॑त्यै भवन्ति भवन्ति॒ विजि॑त्यै । \newline
5. विजि॑त्यै॒ षट् थ्षड् विजि॑त्यै॒ विजि॑त्यै॒ षट् । \newline
6. विजि॑त्या॒ इति॒ वि - जि॒त्यै॒ । \newline
7. षडा᳚श्वि॒ना न्या᳚श्वि॒नानि॒ षट् थ्षडा᳚श्वि॒नानि॑ । \newline
8. आ॒श्वि॒नानि॑ भवन्ति भव न्त्याश्वि॒ना न्या᳚श्वि॒नानि॑ भवन्ति । \newline
9. भ॒व॒न्ति॒ षट् थ्षड् भ॑वन्ति भवन्ति॒ षट् । \newline
10. षड् वै वै षट् थ्षड् वै । \newline
11. वा ऋ॒तव॑ ऋ॒तवो॒ वै वा ऋ॒तवः॑ । \newline
12. ऋ॒तव॑ ऋ॒तुष् वृ॒तुष् वृ॒तव॑ ऋ॒तव॑ ऋ॒तुषु॑ । \newline
13. ऋ॒तु ष्वे॒वैव र्‌तुष् वृ॒तु ष्वे॒व । \newline
14. ए॒व प्रति॒ प्रत्ये॒वैव प्रति॑ । \newline
15. प्रति॑ तिष्ठन्ति तिष्ठन्ति॒ प्रति॒ प्रति॑ तिष्ठन्ति । \newline
16. ति॒ष्ठ॒ न्त्यू॒ना॒ति॒रि॒क्ता ऊ॑नातिरि॒क्ता स्ति॑ष्ठन्ति तिष्ठ न्त्यूनातिरि॒क्ताः । \newline
17. ऊ॒ना॒ति॒रि॒क्ता वै वा ऊ॑नातिरि॒क्ता ऊ॑नातिरि॒क्ता वै । \newline
18. ऊ॒ना॒ति॒रि॒क्ता इत्यू॑न - अ॒ति॒रि॒क्ताः । \newline
19. वा ए॒ता ए॒ता वै वा ए॒ताः । \newline
20. ए॒ता रात्र॑यो॒ रात्र॑य ए॒ता ए॒ता रात्र॑यः । \newline
21. रात्र॑य ऊ॒ना ऊ॒ना रात्र॑यो॒ रात्र॑य ऊ॒नाः । \newline
22. ऊ॒ना स्तत् तदू॒ना ऊ॒ना स्तत् । \newline
23. तद् यद् यत् तत् तद् यत् । \newline
24. यदेक॑स्या॒ एक॑स्यै॒ यद् यदेक॑स्यै । \newline
25. एक॑स्यै॒ न नैक॑स्या॒ एक॑स्यै॒ न । \newline
26. न प॑ञ्चा॒शत् प॑ञ्चा॒शन् न न प॑ञ्चा॒शत् । \newline
27. प॒ञ्चा॒श दति॑रिक्ता॒ अति॑रिक्ताः पञ्चा॒शत् प॑ञ्चा॒श दति॑रिक्ताः । \newline
28. अति॑रिक्ता॒ स्तत् तदति॑रिक्ता॒ अति॑रिक्ता॒ स्तत् । \newline
29. अति॑रिक्ता॒ इत्यति॑ - रि॒क्ताः॒ । \newline
30. तद् यद् यत् तत् तद् यत् । \newline
31. यद् भूय॑सी॒र् भूय॑सी॒र् यद् यद् भूय॑सीः । \newline
32. भूय॑सी र॒ष्टाच॑त्वारिꣳशतो॒ ऽष्टाच॑त्वारिꣳशतो॒ भूय॑सी॒र् भूय॑सी र॒ष्टाच॑त्वारिꣳशतः । \newline
33. अ॒ष्टाच॑त्वारिꣳशत ऊ॒ना दू॒ना द॒ष्टाच॑त्वारिꣳशतो॒ ऽष्टाच॑त्वारिꣳशत ऊ॒नात् । \newline
34. अ॒ष्टाच॑त्वारिꣳशत॒ इत्य॒ष्टा - च॒त्वा॒रिꣳ॒॒श॒तः॒ । \newline
35. ऊ॒नाच् च॑ चो॒ना दू॒नाच् च॑ । \newline
36. च॒ खलु॒ खलु॑ च च॒ खलु॑ । \newline
37. खलु॒ वै वै खलु॒ खलु॒ वै । \newline
38. वा अति॑रिक्ता॒ दति॑रिक्ता॒द् वै वा अति॑रिक्तात् । \newline
39. अति॑रिक्ताच् च॒ चाति॑रिक्ता॒ दति॑रिक्ताच् च । \newline
40. अति॑रिक्ता॒दित्यति॑ - रि॒क्ता॒त् । \newline
41. च॒ प्र॒जाप॑तिः प्र॒जाप॑ति श्च च प्र॒जाप॑तिः । \newline
42. प्र॒जाप॑तिः॒ प्र प्र प्र॒जाप॑तिः प्र॒जाप॑तिः॒ प्र । \newline
43. प्र॒जाप॑ति॒रिति॑ प्र॒जा - प॒तिः॒ । \newline
44. प्रा जा॑यता जायत॒ प्र प्रा जा॑यत । \newline
45. अ॒जा॒य॒त॒ ये ये॑ ऽजायता जायत॒ ये । \newline
46. ये प्र॒जाका॑माः प्र॒जाका॑मा॒ ये ये प्र॒जाका॑माः । \newline
47. प्र॒जाका॑माः प॒शुका॑माः प॒शुका॑माः प्र॒जाका॑माः प्र॒जाका॑माः प॒शुका॑माः । \newline
48. प्र॒जाका॑मा॒ इति॑ प्र॒जा - का॒माः॒ । \newline
49. प॒शुका॑माः॒ स्युः स्युः प॒शुका॑माः प॒शुका॑माः॒ स्युः । \newline
50. प॒शुका॑मा॒ इति॑ प॒शु - का॒माः॒ । \newline
51. स्यु स्ते ते स्युः स्यु स्ते । \newline
52. त ए॒ता ए॒ता स्ते त ए॒ताः । \newline
53. ए॒ता आ॑सीरन् नासीरन् ने॒ता ए॒ता आ॑सीरन्न् । \newline
54. आ॒सी॒र॒न् प्र प्रासी॑रन् नासीर॒न् प्र । \newline
55. प्रैवैव प्र प्रैव । \newline
56. ए॒व जा॑यन्ते जायन्त ए॒वैव जा॑यन्ते । \newline
57. जा॒य॒न्ते॒ प्र॒जया᳚ प्र॒जया॑ जायन्ते जायन्ते प्र॒जया᳚ । \newline
58. प्र॒जया॑ प॒शुभिः॑ प॒शुभिः॑ प्र॒जया᳚ प्र॒जया॑ प॒शुभिः॑ । \newline
59. प्र॒जयेति॑ प्र - जया᳚ । \newline
60. प॒शुभि॑र् वैरा॒जो वै॑रा॒जः प॒शुभिः॑ प॒शुभि॑र् वैरा॒जः । \newline
61. प॒शुभि॒रिति॑ प॒शु - भिः॒ । \newline
62. वै॒रा॒जो वै वै वै॑रा॒जो वै॑रा॒जो वै । \newline
63. वा ए॒ष ए॒ष वै वा ए॒षः । \newline
64. ए॒ष य॒ज्ञो य॒ज्ञ् ए॒ष ए॒ष य॒ज्ञ्ः । \newline
65. य॒ज्ञो यद् यद् य॒ज्ञो य॒ज्ञो यत् । \newline
66. यदे॑कस्मान्नपञ्चा॒श ए॑कस्मान्नपञ्चा॒शो यद् यदे॑कस्मान्नपञ्चा॒शः । \newline
67. ए॒क॒स्मा॒न्न॒प॒ञ्चा॒शो ये य ए॑कस्मान्नपञ्चा॒श ए॑कस्मान्नपञ्चा॒शो ये । \newline
68. ए॒क॒स्मा॒न्न॒प॒ञ्चा॒श इत्ये॑कस्मात् - न॒प॒ञ्चा॒शः । \newline
69. य ए॒व मे॒वं ॅये य ए॒वम् । \newline
70. ए॒वं ॅवि॒द्वाꣳसो॑ वि॒द्वाꣳस॑ ए॒व मे॒वं ॅवि॒द्वाꣳसः॑ । \newline
71. वि॒द्वाꣳस॑ एकस्मान्नपञ्चा॒श मे॑कस्मान्नपञ्चा॒शं ॅवि॒द्वाꣳसो॑ वि॒द्वाꣳस॑ एकस्मान्नपञ्चा॒शम् । \newline
72. ए॒क॒स्मा॒न्न॒प॒ञ्चा॒श मास॑त॒ आस॑त एकस्मान्नपञ्चा॒श मे॑कस्मान्नपञ्चा॒श मास॑ते । \newline
73. ए॒क॒स्मा॒न्न॒प॒ञ्चा॒शमित्ये॑कस्मात् - न॒प॒ञ्चा॒शम् । \newline
74. आस॑ते वि॒राजं॑ ॅवि॒राज॒ मास॑त॒ आस॑ते वि॒राज᳚म् । \newline
75. वि॒राज॑ मे॒वैव वि॒राजं॑ ॅवि॒राज॑ मे॒व । \newline
76. वि॒राज॒मिति॑ वि - राज᳚म् । \newline
77. ए॒व ग॑च्छन्ति गच्छ न्त्ये॒वैव ग॑च्छन्ति । \newline
78. ग॒च्छ॒ न्त्य॒न्ना॒दा अ॑न्ना॒दा ग॑च्छन्ति गच्छ न्त्यन्ना॒दाः । \newline
79. अ॒न्ना॒दा भ॑वन्ति भव न्त्यन्ना॒दा अ॑न्ना॒दा भ॑वन्ति । \newline
80. अ॒न्ना॒दा इत्य॑न्न - अ॒दाः । \newline
81. भ॒व॒ न्त्य॒ति॒रा॒त्रा व॑तिरा॒त्रौ भ॑वन्ति भव न्त्यतिरा॒त्रौ । \newline
82. अ॒ति॒रा॒त्रा व॒भितो॒ ऽभितो॑ ऽतिरा॒त्रा व॑तिरा॒त्रा व॒भितः॑ । \newline
83. अ॒ति॒रा॒त्रावित्य॑ति - रा॒त्रौ । \newline
84. अ॒भितो॑ भवतो भवतो॒ ऽभितो॒ ऽभितो॑ भवतः । \newline
85. भ॒व॒तो॒ ऽन्नाद्य॑स्या॒ न्नाद्य॑स्य भवतो भवतो॒ ऽन्नाद्य॑स्य । \newline
86. अ॒न्नाद्य॑स्य॒ परि॑गृहीत्यै॒ परि॑गृहीत्या अ॒न्नाद्य॑स्या॒ न्नाद्य॑स्य॒ परि॑गृहीत्यै । \newline
87. अ॒न्नाद्य॒स्येत्य॑न्न - अद्य॑स्य । \newline
88. परि॑गृहीत्या॒ इति॒ परि॑ - गृ॒ही॒त्यै॒ । \newline

\textbf{Ghana Paata } \newline

1. द॒ध॒ते॒ ब॒हवो॑ ब॒हवो॑ दधते दधते ब॒हव॑ ष्षोड॒शिन॑ ष्षोड॒शिनो॑ ब॒हवो॑ दधते दधते ब॒हव॑ ष्षोड॒शिनः॑ । \newline
2. ब॒हव॑ ष्षोड॒शिन॑ ष्षोड॒शिनो॑ ब॒हवो॑ ब॒हव॑ ष्षोड॒शिनो॑ भवन्ति भवन्ति षोड॒शिनो॑ ब॒हवो॑ ब॒हव॑ ष्षोड॒शिनो॑ भवन्ति । \newline
3. षो॒ड॒शिनो॑ भवन्ति भवन्ति षोड॒शिन॑ ष्षोड॒शिनो॑ भवन्ति॒ विजि॑त्यै॒ विजि॑त्यै भवन्ति षोड॒शिन॑ ष्षोड॒शिनो॑ भवन्ति॒ विजि॑त्यै । \newline
4. भ॒व॒न्ति॒ विजि॑त्यै॒ विजि॑त्यै भवन्ति भवन्ति॒ विजि॑त्यै॒ षट् थ्षड् विजि॑त्यै भवन्ति भवन्ति॒ विजि॑त्यै॒ षट् । \newline
5. विजि॑त्यै॒ षट् थ्षड् विजि॑त्यै॒ विजि॑त्यै॒ षडा᳚श्वि॒नान्या᳚ श्वि॒नानि॒ षड् विजि॑त्यै॒ विजि॑त्यै॒ षडा᳚श्वि॒नानि॑ । \newline
6. विजि॑त्या॒ इति॒ वि - जि॒त्यै॒ । \newline
7. षडा᳚श्वि॒ना न्या᳚श्वि॒नानि॒ षट् थ्षडा᳚श्वि॒नानि॑ भवन्ति भव न्त्याश्वि॒नानि॒ षट् 
थ्षडा᳚श्वि॒नानि॑ भवन्ति । \newline
8. आ॒श्वि॒नानि॑ भवन्ति भव न्त्याश्वि॒ना न्या᳚श्वि॒नानि॑ भवन्ति॒ षट् थ्षड् भ॑व न्त्याश्वि॒ना न्या᳚श्वि॒नानि॑ भवन्ति॒ षट् । \newline
9. भ॒व॒न्ति॒ षट् थ्षड् भ॑वन्ति भवन्ति॒ षड् वै वै षड् भ॑वन्ति भवन्ति॒ षड् वै । \newline
10. षड् वै वै षट् थ्षड् वा ऋ॒तव॑ ऋ॒तवो॒ वै षट् थ्षड् वा ऋ॒तवः॑ । \newline
11. वा ऋ॒तव॑ ऋ॒तवो॒ वै वा ऋ॒तव॑ ऋ॒तुष् वृ॒तुष् वृ॒तवो॒ वै वा ऋ॒तव॑ ऋ॒तुषु॑ । \newline
12. ऋ॒तव॑ ऋ॒तुष् वृ॒तुष् वृ॒तव॑ ऋ॒तव॑ ऋ॒तुष् वे॒वैव र्‌तुष् वृ॒तव॑ ऋ॒तव॑ ऋ॒तुष् वे॒व । \newline
13. ऋ॒तुष् वे॒वैव र्‌तुष् वृ॒तुष् वे॒व प्रति॒ प्रत्ये॒व र्‌तुष् वृ॒तुष् वे॒व प्रति॑ । \newline
14. ए॒व प्रति॒ प्रत्ये॒वैव प्रति॑ तिष्ठन्ति तिष्ठन्ति॒ प्रत्ये॒वैव प्रति॑ तिष्ठन्ति । \newline
15. प्रति॑ तिष्ठन्ति तिष्ठन्ति॒ प्रति॒ प्रति॑ तिष्ठ न्त्यूनातिरि॒क्ता ऊ॑नातिरि॒क्ता स्ति॑ष्ठन्ति॒ प्रति॒ प्रति॑ तिष्ठ न्त्यूनातिरि॒क्ताः । \newline
16. ति॒ष्ठ॒ न्त्यू॒ना॒ति॒रि॒क्ता ऊ॑नातिरि॒क्ता स्ति॑ष्ठन्ति तिष्ठ न्त्यूनातिरि॒क्ता वै वा ऊ॑नातिरि॒क्ता स्ति॑ष्ठन्ति तिष्ठ न्त्यूनातिरि॒क्ता वै । \newline
17. ऊ॒ना॒ति॒रि॒क्ता वै वा ऊ॑नातिरि॒क्ता ऊ॑नातिरि॒क्ता वा ए॒ता ए॒ता वा ऊ॑नातिरि॒क्ता ऊ॑नातिरि॒क्ता वा ए॒ताः । \newline
18. ऊ॒ना॒ति॒रि॒क्ता इत्यू॑न - अ॒ति॒रि॒क्ताः । \newline
19. वा ए॒ता ए॒ता वै वा ए॒ता रात्र॑यो॒ रात्र॑य ए॒ता वै वा ए॒ता रात्र॑यः । \newline
20. ए॒ता रात्र॑यो॒ रात्र॑य ए॒ता ए॒ता रात्र॑य ऊ॒ना ऊ॒ना रात्र॑य ए॒ता ए॒ता रात्र॑य ऊ॒नाः । \newline
21. रात्र॑य ऊ॒ना ऊ॒ना रात्र॑यो॒ रात्र॑य ऊ॒ना स्तत् तदू॒ना रात्र॑यो॒ रात्र॑य ऊ॒ना स्तत् । \newline
22. ऊ॒ना स्तत् तदू॒ना ऊ॒ना स्तद् यद् यत् तदू॒ना ऊ॒ना स्तद् यत् । \newline
23. तद् यद् यत् तत् तद् यदेक॑स्या॒ एक॑स्यै॒ यत् तत् तद् यदेक॑स्यै । \newline
24. यदेक॑स्या॒ एक॑स्यै॒ यद् यदेक॑स्यै॒ न नैक॑स्यै॒ यद् यदेक॑स्यै॒ न । \newline
25. एक॑स्यै॒ न नैक॑स्या॒ एक॑स्यै॒ न प॑ञ्चा॒शत् प॑ञ्चा॒शन् नैक॑स्या॒ एक॑स्यै॒ न प॑ञ्चा॒शत् । \newline
26. न प॑ञ्चा॒शत् प॑ञ्चा॒शन् न न प॑ञ्चा॒श दति॑रिक्ता॒ अति॑रिक्ताः पञ्चा॒शन् न न प॑ञ्चा॒श दति॑रिक्ताः । \newline
27. प॒ञ्चा॒श दति॑रिक्ता॒ अति॑रिक्ताः पञ्चा॒शत् प॑ञ्चा॒श दति॑रिक्ता॒ स्तत् तदति॑रिक्ताः पञ्चा॒शत् प॑ञ्चा॒श दति॑रिक्ता॒ स्तत् । \newline
28. अति॑रिक्ता॒ स्तत् तदति॑रिक्ता॒ अति॑रिक्ता॒ स्तद् यद् यत् तदति॑रिक्ता॒ अति॑रिक्ता॒ स्तद् यत् । \newline
29. अति॑रिक्ता॒ इत्यति॑ - रि॒क्ताः॒ । \newline
30. तद् यद् यत् तत् तद् यद् भूय॑सी॒र् भूय॑सी॒र् यत् तत् तद् यद् भूय॑सीः । \newline
31. यद् भूय॑सी॒र् भूय॑सी॒र् यद् यद् भूय॑सी र॒ष्टाच॑त्वारिꣳशतो॒ ऽष्टाच॑त्वारिꣳशतो॒ भूय॑सी॒र् यद् यद् भूय॑सी र॒ष्टाच॑त्वारिꣳशतः । \newline
32. भूय॑सी र॒ष्टाच॑त्वारिꣳशतो॒ ऽष्टाच॑त्वारिꣳशतो॒ भूय॑सी॒र् भूय॑सी र॒ष्टाच॑त्वारिꣳशत ऊ॒ना दू॒ना द॒ष्टाच॑त्वारिꣳशतो॒ भूय॑सी॒र् भूय॑सी र॒ष्टाच॑त्वारिꣳशत ऊ॒नात् । \newline
33. अ॒ष्टाच॑त्वारिꣳशत ऊ॒ना दू॒ना द॒ष्टाच॑त्वारिꣳशतो॒ ऽष्टाच॑त्वारिꣳशत ऊ॒नाच् च॑ चो॒ना द॒ष्टाच॑त्वारिꣳशतो॒ ऽष्टाच॑त्वारिꣳशत ऊ॒नाच् च॑ । \newline
34. अ॒ष्टाच॑त्वारिꣳशत॒ इत्य॒ष्टा - च॒त्वा॒रिꣳ॒॒श॒तः॒ । \newline
35. ऊ॒नाच् च॑ चो॒ना दू॒नाच् च॒ खलु॒ खलु॑ चो॒ना दू॒नाच् च॒ खलु॑ । \newline
36. च॒ खलु॒ खलु॑ च च॒ खलु॒ वै वै खलु॑ च च॒ खलु॒ वै । \newline
37. खलु॒ वै वै खलु॒ खलु॒ वा अति॑रिक्ता॒ दति॑रिक्ता॒द् वै खलु॒ खलु॒ वा अति॑रिक्तात् । \newline
38. वा अति॑रिक्ता॒ दति॑रिक्ता॒द् वै वा अति॑रिक्ताच् च॒ चाति॑रिक्ता॒द् वै वा अति॑रिक्ताच् च । \newline
39. अति॑रिक्ताच् च॒ चाति॑रिक्ता॒ दति॑रिक्ताच् च प्र॒जाप॑तिः प्र॒जाप॑ति॒ श्चाति॑रिक्ता॒ दति॑रिक्ताच् च प्र॒जाप॑तिः । \newline
40. अति॑रिक्ता॒दित्यति॑ - रि॒क्ता॒त् । \newline
41. च॒ प्र॒जाप॑तिः प्र॒जाप॑ति श्च च प्र॒जाप॑तिः॒ प्र प्र प्र॒जाप॑ति श्च च प्र॒जाप॑तिः॒ प्र । \newline
42. प्र॒जाप॑तिः॒ प्र प्र प्र॒जाप॑तिः प्र॒जाप॑तिः॒ प्रा जा॑यता जायत॒ प्र प्र॒जाप॑तिः प्र॒जाप॑तिः॒ प्रा जा॑यत । \newline
43. प्र॒जाप॑ति॒रिति॑ प्र॒जा - प॒तिः॒ । \newline
44. प्रा जा॑यता जायत॒ प्र प्राजा॑यत॒ ये ये॑ ऽजायत॒ प्र प्रा जा॑यत॒ ये । \newline
45. अ॒जा॒य॒त॒ ये ये॑ ऽजायता जायत॒ ये प्र॒जाका॑माः प्र॒जाका॑मा॒ ये॑ ऽजायता जायत॒ ये प्र॒जाका॑माः । \newline
46. ये प्र॒जाका॑माः प्र॒जाका॑मा॒ ये ये प्र॒जाका॑माः प॒शुका॑माः प॒शुका॑माः प्र॒जाका॑मा॒ ये ये प्र॒जाका॑माः प॒शुका॑माः । \newline
47. प्र॒जाका॑माः प॒शुका॑माः प॒शुका॑माः प्र॒जाका॑माः प्र॒जाका॑माः प॒शुका॑माः॒ स्युः स्युः प॒शुका॑माः प्र॒जाका॑माः प्र॒जाका॑माः प॒शुका॑माः॒ स्युः । \newline
48. प्र॒जाका॑मा॒ इति॑ प्र॒जा - का॒माः॒ । \newline
49. प॒शुका॑माः॒ स्युः स्युः प॒शुका॑माः प॒शुका॑माः॒ स्यु स्ते ते स्युः प॒शुका॑माः प॒शुका॑माः॒ स्यु स्ते । \newline
50. प॒शुका॑मा॒ इति॑ प॒शु - का॒माः॒ । \newline
51. स्यु स्ते ते स्युः स्यु स्त ए॒ता ए॒ता स्ते स्युः स्यु स्त ए॒ताः । \newline
52. त ए॒ता ए॒ता स्ते त ए॒ता आ॑सीरन् नासीरन् ने॒ता स्ते त ए॒ता आ॑सीरन्न् । \newline
53. ए॒ता आ॑सीरन् नासीरन् ने॒ता ए॒ता आ॑सीर॒न् प्र प्रासी॑रन् ने॒ता ए॒ता आ॑सीर॒न् प्र । \newline
54. आ॒सी॒र॒न् प्र प्रासी॑रन् नासीर॒न् प्रैवैव प्रासी॑रन् नासीर॒न् प्रैव । \newline
55. प्रैवैव प्र प्रैव जा॑यन्ते जायन्त ए॒व प्र प्रैव जा॑यन्ते । \newline
56. ए॒व जा॑यन्ते जायन्त ए॒वैव जा॑यन्ते प्र॒जया᳚ प्र॒जया॑ जायन्त ए॒वैव जा॑यन्ते प्र॒जया᳚ । \newline
57. जा॒य॒न्ते॒ प्र॒जया᳚ प्र॒जया॑ जायन्ते जायन्ते प्र॒जया॑ प॒शुभिः॑ प॒शुभिः॑ प्र॒जया॑ जायन्ते जायन्ते प्र॒जया॑ प॒शुभिः॑ । \newline
58. प्र॒जया॑ प॒शुभिः॑ प॒शुभिः॑ प्र॒जया᳚ प्र॒जया॑ प॒शुभि॑र् वैरा॒जो वै॑रा॒जः प॒शुभिः॑ प्र॒जया᳚ प्र॒जया॑ प॒शुभि॑र् वैरा॒जः । \newline
59. प्र॒जयेति॑ प्र - जया᳚ । \newline
60. प॒शुभि॑र् वैरा॒जो वै॑रा॒जः प॒शुभिः॑ प॒शुभि॑र् वैरा॒जो वै वै वै॑रा॒जः प॒शुभिः॑ प॒शुभि॑र् वैरा॒जो वै । \newline
61. प॒शुभि॒रिति॑ प॒शु - भिः॒ । \newline
62. वै॒रा॒जो वै वै वै॑रा॒जो वै॑रा॒जो वा ए॒ष ए॒ष वै वै॑रा॒जो वै॑रा॒जो वा ए॒षः । \newline
63. वा ए॒ष ए॒ष वै वा ए॒ष य॒ज्ञो य॒ज्ञ् ए॒ष वै वा ए॒ष य॒ज्ञ्ः । \newline
64. ए॒ष य॒ज्ञो य॒ज्ञ् ए॒ष ए॒ष य॒ज्ञो यद् यद् य॒ज्ञ् ए॒ष ए॒ष य॒ज्ञो यत् । \newline
65. य॒ज्ञो यद् यद् य॒ज्ञो य॒ज्ञो यदे॑कस्मान्नपञ्चा॒श ए॑कस्मान्नपञ्चा॒शो यद् य॒ज्ञो य॒ज्ञो 
यदे॑कस्मान्नपञ्चा॒शः । \newline
66. यदे॑कस्मान्नपञ्चा॒श ए॑कस्मान्नपञ्चा॒शो यद् यदे॑कस्मान्नपञ्चा॒शो ये य ए॑कस्मान्नपञ्चा॒शो यद् 
यदे॑कस्मान्नपञ्चा॒शो ये । \newline
67. ए॒क॒स्मा॒न्न॒प॒ञ्चा॒शो ये य ए॑कस्मान्नपञ्चा॒श ए॑कस्मान्नपञ्चा॒शो य ए॒व मे॒वं ॅय ए॑कस्मान्नपञ्चा॒श ए॑कस्मान्नपञ्चा॒शो य ए॒वम् । \newline
68. ए॒क॒स्मा॒न्न॒प॒ञ्चा॒श इत्ये॑कस्मात् - न॒प॒ञ्चा॒शः । \newline
69. य ए॒व मे॒वं ॅये य ए॒वं ॅवि॒द्वाꣳसो॑ वि॒द्वाꣳस॑ ए॒वं ॅये य ए॒वं ॅवि॒द्वाꣳसः॑ । \newline
70. ए॒वं ॅवि॒द्वाꣳसो॑ वि॒द्वाꣳस॑ ए॒व मे॒वं ॅवि॒द्वाꣳस॑ एकस्मान्नपञ्चा॒श मे॑कस्मान्नपञ्चा॒शं ॅवि॒द्वाꣳस॑ ए॒व मे॒वं ॅवि॒द्वाꣳस॑ एकस्मान्नपञ्चा॒शम् । \newline
71. वि॒द्वाꣳस॑ एकस्मान्नपञ्चा॒श मे॑कस्मान्नपञ्चा॒शं ॅवि॒द्वाꣳसो॑ वि॒द्वाꣳस॑ एकस्मान्नपञ्चा॒श मास॑त॒ आस॑त एकस्मान्नपञ्चा॒शं ॅवि॒द्वाꣳसो॑ वि॒द्वाꣳस॑ एकस्मान्नपञ्चा॒श मास॑ते । \newline
72. ए॒क॒स्मा॒न्न॒प॒ञ्चा॒श मास॑त॒ आस॑त एकस्मान्नपञ्चा॒श मे॑कस्मान्नपञ्चा॒श मास॑ते वि॒राजं॑ ॅवि॒राज॒ मास॑त एकस्मान्नपञ्चा॒श मे॑कस्मान्नपञ्चा॒श मास॑ते वि॒राज᳚म् । \newline
73. ए॒क॒स्मा॒न्न॒प॒ञ्चा॒शमित्ये॑कस्मात् - न॒प॒ञ्चा॒शम् । \newline
74. आस॑ते वि॒राजं॑ ॅवि॒राज॒ मास॑त॒ आस॑ते वि॒राज॑ मे॒वैव वि॒राज॒ मास॑त॒ आस॑ते वि॒राज॑ मे॒व । \newline
75. वि॒राज॑ मे॒वैव वि॒राजं॑ ॅवि॒राज॑ मे॒व ग॑च्छन्ति गच्छ न्त्ये॒व वि॒राजं॑ ॅवि॒राज॑ मे॒व ग॑च्छन्ति । \newline
76. वि॒राज॒मिति॑ वि - राज᳚म् । \newline
77. ए॒व ग॑च्छन्ति गच्छ न्त्ये॒वैव ग॑च्छ न्त्यन्ना॒दा अ॑न्ना॒दा ग॑च्छ न्त्ये॒वैव ग॑च्छ न्त्यन्ना॒दाः । \newline
78. ग॒च्छ॒ न्त्य॒न्ना॒दा अ॑न्ना॒दा ग॑च्छन्ति गच्छ न्त्यन्ना॒दा भ॑वन्ति भव न्त्यन्ना॒दा ग॑च्छन्ति गच्छ न्त्यन्ना॒दा भ॑वन्ति । \newline
79. अ॒न्ना॒दा भ॑वन्ति भव न्त्यन्ना॒दा अ॑न्ना॒दा भ॑व न्त्यतिरा॒त्रा व॑तिरा॒त्रौ भ॑व न्त्यन्ना॒दा अ॑न्ना॒दा भ॑व न्त्यतिरा॒त्रौ । \newline
80. अ॒न्ना॒दा इत्य॑न्न - अ॒दाः । \newline
81. भ॒व॒ न्त्य॒ति॒रा॒त्रा व॑तिरा॒त्रौ भ॑वन्ति भव न्त्यतिरा॒त्रा व॒भितो॒ ऽभितो॑ ऽतिरा॒त्रौ भ॑वन्ति भव न्त्यतिरा॒त्रा व॒भितः॑ । \newline
82. अ॒ति॒रा॒त्रा व॒भितो॒ ऽभितो॑ ऽतिरा॒त्रा व॑तिरा॒त्रा व॒भितो॑ भवतो भवतो॒ ऽभितो॑ ऽतिरा॒त्रा व॑तिरा॒त्रा व॒भितो॑ भवतः । \newline
83. अ॒ति॒रा॒त्रावित्य॑ति - रा॒त्रौ । \newline
84. अ॒भितो॑ भवतो भवतो॒ ऽभितो॒ ऽभितो॑ भवतो॒ ऽन्नाद्य॑स्या॒ न्नाद्य॑स्य भवतो॒ ऽभितो॒ ऽभितो॑ भवतो॒ ऽन्नाद्य॑स्य । \newline
85. भ॒व॒तो॒ ऽन्नाद्य॑स्या॒ न्नाद्य॑स्य भवतो भवतो॒ ऽन्नाद्य॑स्य॒ परि॑गृहीत्यै॒ परि॑गृहीत्या अ॒न्नाद्य॑स्य भवतो भवतो॒ ऽन्नाद्य॑स्य॒ परि॑गृहीत्यै । \newline
86. अ॒न्नाद्य॑स्य॒ परि॑गृहीत्यै॒ परि॑गृहीत्या अ॒न्नाद्य॑स्या॒ न्नाद्य॑स्य॒ परि॑गृहीत्यै । \newline
87. अ॒न्नाद्य॒स्येत्य॑न्न - अद्य॑स्य । \newline
88. परि॑गृहीत्या॒ इति॒ परि॑ - गृ॒ही॒त्यै॒ । \newline
\pagebreak
\markright{ TS 7.4.8.1  \hfill https://www.vedavms.in \hfill}

\section{ TS 7.4.8.1 }

\textbf{TS 7.4.8.1 } \newline
\textbf{Samhita Paata} \newline

सं॒ॅव॒थ्स॒राय॑ दीक्षि॒ष्यमा॑णा एकाष्ट॒कायां᳚ दीक्षेरन्ने॒षा वै सं॑ॅवथ्स॒रस्य॒ पत्नी॒ यदे॑काष्ट॒कैतस्यां॒ ॅवा ए॒ष ए॒ताꣳ रात्रिं॑ ॅवसति सा॒क्षादे॒व सं॑ॅवथ्स॒रमा॒रभ्य॑ दीक्षन्त॒ आर्तं॒ ॅवा ए॒ते सं॑ॅवथ्स॒रस्या॒भि दी᳚क्षन्ते॒ य ए॑काष्ट॒कायां॒ दीक्ष॒न्ते ऽन्त॑नामानावृ॒तू भ॑वतो॒ व्य॑स्तं॒ ॅवा ए॒ते सं॑ॅवथ्स॒रस्या॒ऽभि दी᳚क्षन्ते॒ य ए॑काष्ट॒कायां॒ दीक्ष॒न्तेऽन्त॑नामानावृ॒तू भ॑वतः फल्गुनी पूर्णमा॒से दी᳚क्षेर॒न् मुखं॒ ॅवा ए॒तथ् - [  ] \newline

\textbf{Pada Paata} \newline

सं॒ॅव॒थ्स॒रायेति॑ सं - व॒थ्स॒राय॑ । दी॒क्षि॒ष्यमा॑णाः । ए॒का॒ष्ट॒काया॒मित्ये॑क - अ॒ष्ट॒काया᳚म् । दी॒क्षे॒र॒न्न् । ए॒षा । वै । सं॒ॅव॒थ्स॒रस्येति॑ सं - व॒थ्स॒रस्य॑ । पत्नी᳚ । यत् । ए॒का॒ष्ट॒केत्ये॑क-अ॒ष्ट॒का । ए॒तस्या᳚म् । वै । ए॒षः । ए॒ताम् । रात्रि᳚म् । व॒स॒ति॒ । सा॒क्षादिति॑ स - अ॒क्षात् । ए॒व । सं॒ॅव॒थ्स॒रमिति॑ सं -  व॒थ्स॒रम् । आ॒रभ्येत्या᳚ - रभ्य॑ । दी॒क्ष॒न्ते॒ । आर्त᳚म् । वै । ए॒ते । सं॒ॅव॒थ्स॒रस्येति॑ सं - व॒थ्स॒रस्य॑ । अ॒भीति॑ । दी॒क्ष॒न्ते॒ । ये । ए॒का॒ष्ट॒काया॒मित्ये॑क - अ॒ष्ट॒काया᳚म् । दीक्ष॑न्ते । अन्त॑नामाना॒वित्यन्त॑- ना॒मा॒नौ॒ । ऋ॒तू इति॑ । भ॒व॒तः॒ । व्य॑स्त॒मिति॒ वि-अ॒स्त॒म् । वै । ए॒ते । सं॒ॅव॒थ्स॒रस्येति॑ सं - व॒थ्स॒रस्य॑ । अ॒भीति॑ । दी॒क्ष॒न्ते॒ । ये । ए॒का॒ष्ट॒काया॒मित्ये॑क - अ॒ष्ट॒काया᳚म् । दीक्ष॑न्ते । अन्त॑नामाना॒वित्यन्त॑- ना॒मा॒नौ॒ । ऋ॒तू इति॑ । भ॒व॒तः॒ । फ॒ल्गु॒नी॒पू॒र्ण॒मा॒स इति॑ फल्गुनी- पू॒र्ण॒मा॒से । दी॒क्षे॒र॒न्न् । मुख᳚म् । वै । ए॒तत् ।  \newline


\textbf{Krama Paata} \newline

स॒म्ॅव॒थ्स॒राय॑ दीक्षि॒ष्यमा॑णाः । स॒म्ॅव॒थ्स॒रायेति॑ सम् - व॒थ्स॒राय॑ । दी॒क्षि॒ष्यमा॑णा एकाष्ट॒काया᳚म् । ए॒का॒ष्ट॒काया᳚म् दीक्षेरन्न् । ए॒का॒ष्ट॒काया॒मित्ये॑क - अ॒ष्ट॒काया᳚म् । दी॒क्षे॒र॒न्ने॒षा । ए॒षा वै । वै स॑म्ॅवथ्स॒रस्य॑ । स॒म्ॅव॒थ्स॒रस्य॒ पत्नी᳚ । स॒म्ॅव॒थ्स॒रस्ये॑ति सम् - व॒थ्स॒रस्य॑ । पत्नी॒ यत् । यदे॑काष्ट॒का । ए॒का॒ष्ट॒कैतस्या᳚म् । ए॒का॒ष्ट॒केत्ये॑क - अ॒ष्ट॒का । ए॒तस्या॒म् ॅवै । वा ए॒षः । ए॒ष ए॒ताम् । ए॒ताꣳ रात्रि᳚म् । रात्रि॑म् ॅवसति । व॒स॒ति॒ सा॒क्षात् । सा॒क्षादे॒व । सा॒क्षादिति॑ स - अ॒क्षात् । ए॒व स॑म्ॅवथ्स॒रम् । स॒म्ॅव॒थ्स॒रमा॒रभ्य॑ । स॒म्ॅव॒थ्स॒रमिति॑ सम् - व॒थ्स॒रम् । आ॒रभ्य॑ दीक्षन्ते । आ॒रभ्येत्या᳚ - रभ्य॑ । दी॒क्ष॒न्त॒ आर्त᳚म् । आर्त॒म् ॅवै । वा ए॒ते । ए॒ते स॑म्ॅवथ्स॒रस्य॑ । स॒म्ॅव॒थ्स॒रस्या॒भि । स॒म्ॅव॒थ्स॒रस्येति॑ सम् - व॒थ्स॒रस्य॑ । अ॒भि दी᳚क्षन्ते । दी॒क्ष॒न्ते॒ ये । य ए॑काष्ट॒काया᳚म् । ए॒का॒ष्ट॒काया॒म् दीक्ष॑न्ते । ए॒का॒ष्ट॒काया॒मित्ये॑क - अ॒ष्ट॒काया᳚म् । दीक्ष॒न्तेऽन्त॑नामानौ । अन्त॑नामनावृ॒तू । अन्त॑नामाना॒वित्यन्त॑ - ना॒मा॒नौ॒ । ऋ॒तू भ॑वतः । ऋ॒तू इत्यृ॒तू । भ॒व॒तो॒ व्य॑स्तम् । व्य॑स्त॒म् ॅवै । व्य॑स्त॒मिति॒ वि - अ॒स्त॒म् । वा ए॒ते । ए॒ते स॑म्ॅवथ्स॒रस्य॑ । स॒म्ॅव॒थ्स॒रस्या॒भि । स॒म्ॅव॒थ्स॒रस्येति॑ सम् - व॒थ्स॒रस्य॑ । अ॒भि दी᳚क्षन्ते । दी॒क्ष॒न्ते॒ ये । य ए॑काष्ट॒काया᳚म् । ए॒का॒ष्ट॒काया॒म् दीक्ष॑न्ते । ए॒का॒ष्टा॒काया॒मित्ये॑क - अ॒ष्ट॒काया᳚म् । दीक्ष॒न्तेऽन्त॑नामानौ । अन्त॑नामानावृ॒तू । अन्त॑नामाना॒वित्य॑न्त - ना॒मा॒नौ॒ । ऋ॒तू भ॑वतः । ऋ॒तू इत्यृ॒तू । भ॒व॒तः॒ फ॒ल्गु॒नी॒पू॒र्ण॒मा॒से । फ॒ल्गु॒नी॒पू॒र्ण॒मा॒से दी᳚क्षेरन्न् । फ॒ल्गु॒नी॒पू॒र्ण॒मा॒स इति॑ फल्गुनी - पू॒र्ण॒मा॒से । दी॒क्षे॒र॒न् मुख᳚म् । मुख॒म् ॅवै । वा ए॒तत् । ए॒तथ् स॑म्ॅवथ्स॒रस्य॑ \newline

\textbf{Jatai Paata} \newline

1. सं॒ॅव॒थ्स॒राय॑ दीक्षि॒ष्यमा॑णा दीक्षि॒ष्यमा॑णाः संॅवथ्स॒राय॑ संॅवथ्स॒राय॑ दीक्षि॒ष्यमा॑णाः । \newline
2. सं॒ॅव॒थ्स॒रायेति॑ सं - व॒थ्स॒राय॑ । \newline
3. दी॒क्षि॒ष्यमा॑णा एकाष्ट॒काया॑ मेकाष्ट॒काया᳚म् दीक्षि॒ष्यमा॑णा दीक्षि॒ष्यमा॑णा एकाष्ट॒काया᳚म् । \newline
4. ए॒का॒ष्ट॒काया᳚म् दीक्षेरन् दीक्षेरन् नेकाष्ट॒काया॑ मेकाष्ट॒काया᳚म् दीक्षेरन्न् । \newline
5. ए॒का॒ष्ट॒काया॒मित्ये॑क - अ॒ष्ट॒काया᳚म् । \newline
6. दी॒क्षे॒र॒न् ने॒षैषा दी᳚क्षेरन् दीक्षेरन् ने॒षा । \newline
7. ए॒षा वै वा ए॒षैषा वै । \newline
8. वै सं॑ॅवथ्स॒रस्य॑ संॅवथ्स॒रस्य॒ वै वै सं॑ॅवथ्स॒रस्य॑ । \newline
9. सं॒ॅव॒थ्स॒रस्य॒ पत्नी॒ पत्नी॑ संॅवथ्स॒रस्य॑ संॅवथ्स॒रस्य॒ पत्नी᳚ । \newline
10. सं॒ॅव॒थ्स॒रस्येति॑ सं - व॒थ्स॒रस्य॑ । \newline
11. पत्नी॒ यद् यत् पत्नी॒ पत्नी॒ यत् । \newline
12. यदे॑काष्ट॒ कैका᳚ष्ट॒का यद् यदे॑काष्ट॒का । \newline
13. ए॒का॒ष्ट॒ कैतस्या॑ मे॒तस्या॑ मेकाष्ट॒ कैका᳚ष्ट॒ कैतस्या᳚म् । \newline
14. ए॒का॒ष्ट॒केत्ये॑क - अ॒ष्ट॒का । \newline
15. ए॒तस्यां॒ ॅवै वा ए॒तस्या॑ मे॒तस्यां॒ ॅवै । \newline
16. वा ए॒ष ए॒ष वै वा ए॒षः । \newline
17. ए॒ष ए॒ता मे॒ता मे॒ष ए॒ष ए॒ताम् । \newline
18. ए॒ताꣳ रात्रिꣳ॒॒ रात्रि॑ मे॒ता मे॒ताꣳ रात्रि᳚म् । \newline
19. रात्रिं॑ ॅवसति वसति॒ रात्रिꣳ॒॒ रात्रिं॑ ॅवसति । \newline
20. व॒स॒ति॒ सा॒क्षाथ् सा॒क्षाद् व॑सति वसति सा॒क्षात् । \newline
21. सा॒क्षा दे॒वैव सा॒क्षाथ् सा॒क्षा दे॒व । \newline
22. सा॒क्षादिति॑ स - अ॒क्षात् । \newline
23. ए॒व सं॑ॅवथ्स॒रꣳ सं॑ॅवथ्स॒र मे॒वैव सं॑ॅवथ्स॒रम् । \newline
24. सं॒ॅव॒थ्स॒र मा॒रभ्या॒ रभ्य॑ संॅवथ्स॒रꣳ सं॑ॅवथ्स॒र मा॒रभ्य॑ । \newline
25. सं॒ॅव॒थ्स॒रमिति॑ सं - व॒थ्स॒रम् । \newline
26. आ॒रभ्य॑ दीक्षन्ते दीक्षन्त आ॒रभ्या॒ रभ्य॑ दीक्षन्ते । \newline
27. आ॒रभ्येत्या᳚ - रभ्य॑ । \newline
28. दी॒क्ष॒न्त॒ आर्त॒ मार्त॑म् दीक्षन्ते दीक्षन्त॒ आर्त᳚म् । \newline
29. आर्तं॒ ॅवै वा आर्त॒ मार्तं॒ ॅवै । \newline
30. वा ए॒त ए॒ते वै वा ए॒ते । \newline
31. ए॒ते सं॑ॅवथ्स॒रस्य॑ संॅवथ्स॒र स्यै॒त ए॒ते सं॑ॅवथ्स॒रस्य॑ । \newline
32. सं॒ॅव॒थ्स॒रस्या॒ भ्य॑भि सं॑ॅवथ्स॒रस्य॑ संॅवथ्स॒रस्या॒भि । \newline
33. सं॒ॅव॒थ्स॒रस्येति॑ सं - व॒थ्स॒रस्य॑ । \newline
34. अ॒भि दी᳚क्षन्ते दीक्षन्ते॒ ऽभ्य॑भि दी᳚क्षन्ते । \newline
35. दी॒क्ष॒न्ते॒ ये ये दी᳚क्षन्ते दीक्षन्ते॒ ये । \newline
36. य ए॑काष्ट॒काया॑ मेकाष्ट॒कायां॒ ॅये य ए॑काष्ट॒काया᳚म् । \newline
37. ए॒का॒ष्ट॒काया॒म् दीक्ष॑न्ते॒ दीक्ष॑न्त एकाष्ट॒काया॑ मेकाष्ट॒काया॒म् दीक्ष॑न्ते । \newline
38. ए॒का॒ष्ट॒काया॒मित्ये॑क - अ॒ष्ट॒काया᳚म् । \newline
39. दीक्ष॒न्ते ऽन्त॑नामाना॒ वन्त॑नामानौ॒ दीक्ष॑न्ते॒ दीक्ष॒न्ते ऽन्त॑नामानौ । \newline
40. अन्त॑नामाना वृ॒तू ऋ॒तू अन्त॑नामाना॒ वन्त॑नामाना वृ॒तू । \newline
41. अन्त॑नामाना॒वित्यन्त॑ - ना॒मा॒नौ॒ । \newline
42. ऋ॒तू भ॑वतो भवत ऋ॒तू ऋ॒तू भ॑वतः । \newline
43. ऋ॒तू इत्यृ॒तू । \newline
44. भ॒व॒तो॒ व्य॑स्तं॒ ॅव्य॑स्तम् भवतो भवतो॒ व्य॑स्तम् । \newline
45. व्य॑स्तं॒ ॅवै वै व्य॑स्तं॒ ॅव्य॑स्तं॒ ॅवै । \newline
46. व्य॑स्त॒मिति॒ वि - अ॒स्त॒म् । \newline
47. वा ए॒त ए॒ते वै वा ए॒ते । \newline
48. ए॒ते सं॑ॅवथ्स॒रस्य॑ संॅवथ्स॒र स्यै॒त ए॒ते सं॑ॅवथ्स॒रस्य॑ । \newline
49. सं॒ॅव॒थ्स॒र स्या॒भ्य॑भि सं॑ॅवथ्स॒रस्य॑ संॅवथ्स॒र स्या॒भि । \newline
50. सं॒ॅव॒थ्स॒रस्येति॑ सं - व॒थ्स॒रस्य॑ । \newline
51. अ॒भि दी᳚क्षन्ते दीक्षन्ते॒ ऽभ्य॑भि दी᳚क्षन्ते । \newline
52. दी॒क्ष॒न्ते॒ ये ये दी᳚क्षन्ते दीक्षन्ते॒ ये । \newline
53. य ए॑काष्ट॒काया॑ मेकाष्ट॒कायां॒ ॅये य ए॑काष्ट॒काया᳚म् । \newline
54. ए॒का॒ष्ट॒काया॒म् दीक्ष॑न्ते॒ दीक्ष॑न्त एकाष्ट॒काया॑ मेकाष्ट॒काया॒म् दीक्ष॑न्ते । \newline
55. ए॒का॒ष्ट॒काया॒मित्ये॑क - अ॒ष्ट॒काया᳚म् । \newline
56. दीक्ष॒न्ते ऽन्त॑नामाना॒ वन्त॑नामानौ॒ दीक्ष॑न्ते॒ दीक्ष॒न्ते ऽन्त॑नामानौ । \newline
57. अन्त॑नामाना वृ॒तू ऋ॒तू अन्त॑नामाना॒ वन्त॑नामाना वृ॒तू । \newline
58. अन्त॑नामाना॒वित्यन्त॑ - ना॒मा॒नौ॒ । \newline
59. ऋ॒तू भ॑वतो भवत ऋ॒तू ऋ॒तू भ॑वतः । \newline
60. ऋ॒तू इत्यृ॒तू । \newline
61. भ॒व॒तः॒ फ॒ल्गु॒नी॒पू॒र्ण॒मा॒से फ॑ल्गुनीपूर्णमा॒से भ॑वतो भवतः फल्गुनीपूर्णमा॒से । \newline
62. फ॒ल्गु॒नी॒पू॒र्ण॒मा॒से दी᳚क्षेरन् दीक्षेरन् फल्गुनीपूर्णमा॒से फ॑ल्गुनीपूर्णमा॒से दी᳚क्षेरन्न् । \newline
63. फ॒ल्गु॒नी॒पू॒र्ण॒मा॒स इति॑ फल्गुनी - पू॒र्ण॒मा॒से । \newline
64. दी॒क्षे॒र॒न् मुख॒म् मुख॑म् दीक्षेरन् दीक्षेर॒न् मुख᳚म् । \newline
65. मुखं॒ ॅवै वै मुख॒म् मुखं॒ ॅवै । \newline
66. वा ए॒त दे॒तद् वै वा ए॒तत् । \newline
67. ए॒तथ् सं॑ॅवथ्स॒रस्य॑ संॅवथ्स॒र स्यै॒त दे॒तथ् सं॑ॅवथ्स॒रस्य॑ । \newline

\textbf{Ghana Paata } \newline

1. सं॒ॅव॒थ्स॒राय॑ दीक्षि॒ष्यमा॑णा दीक्षि॒ष्यमा॑णाः संॅवथ्स॒राय॑ संॅवथ्स॒राय॑ दीक्षि॒ष्यमा॑णा एकाष्ट॒काया॑ मेकाष्ट॒काया᳚म् दीक्षि॒ष्यमा॑णाः संॅवथ्स॒राय॑ संॅवथ्स॒राय॑ दीक्षि॒ष्यमा॑णा एकाष्ट॒काया᳚म् । \newline
2. सं॒ॅव॒थ्स॒रायेति॑ सं - व॒थ्स॒राय॑ । \newline
3. दी॒क्षि॒ष्यमा॑णा एकाष्ट॒काया॑ मेकाष्ट॒काया᳚म् दीक्षि॒ष्यमा॑णा दीक्षि॒ष्यमा॑णा एकाष्ट॒काया᳚म् दीक्षेरन् दीक्षेरन् नेकाष्ट॒काया᳚म् दीक्षि॒ष्यमा॑णा दीक्षि॒ष्यमा॑णा एकाष्ट॒काया᳚म् दीक्षेरन्न् । \newline
4. ए॒का॒ष्ट॒काया᳚म् दीक्षेरन् दीक्षेरन् नेकाष्ट॒काया॑ मेकाष्ट॒काया᳚म् दीक्षेरन् ने॒षैषा दी᳚क्षेरन् नेकाष्ट॒काया॑ मेकाष्ट॒काया᳚म् दीक्षेरन् ने॒षा । \newline
5. ए॒का॒ष्ट॒काया॒मित्ये॑क - अ॒ष्ट॒काया᳚म् । \newline
6. दी॒क्षे॒र॒न् ने॒षैषा दी᳚क्षेरन् दीक्षेरन् ने॒षा वै वा ए॒षा दी᳚क्षेरन् दीक्षेरन् ने॒षा वै । \newline
7. ए॒षा वै वा ए॒षैषा वै सं॑ॅवथ्स॒रस्य॑ संॅवथ्स॒रस्य॒ वा ए॒षैषा वै सं॑ॅवथ्स॒रस्य॑ । \newline
8. वै सं॑ॅवथ्स॒रस्य॑ संॅवथ्स॒रस्य॒ वै वै सं॑ॅवथ्स॒रस्य॒ पत्नी॒ पत्नी॑ संॅवथ्स॒रस्य॒ वै वै सं॑ॅवथ्स॒रस्य॒ पत्नी᳚ । \newline
9. सं॒ॅव॒थ्स॒रस्य॒ पत्नी॒ पत्नी॑ संॅवथ्स॒रस्य॑ संॅवथ्स॒रस्य॒ पत्नी॒ यद् यत् पत्नी॑ संॅवथ्स॒रस्य॑ संॅवथ्स॒रस्य॒ पत्नी॒ यत् । \newline
10. सं॒ॅव॒थ्स॒रस्येति॑ सं - व॒थ्स॒रस्य॑ । \newline
11. पत्नी॒ यद् यत् पत्नी॒ पत्नी॒ यदे॑काष्ट॒ कैका᳚ष्ट॒का यत् पत्नी॒ पत्नी॒ यदे॑काष्ट॒का । \newline
12. यदे॑काष्ट॒ कैका᳚ष्ट॒का यद् यदे॑काष्ट॒ कैतस्या॑ मे॒तस्या॑ मेकाष्ट॒का यद् यदे॑काष्ट॒ कैतस्या᳚म् । \newline
13. ए॒का॒ष्ट॒ कैतस्या॑ मे॒तस्या॑ मेकाष्ट॒ कैका᳚ष्ट॒ कैतस्यां॒ ॅवै वा ए॒तस्या॑ मेकाष्ट॒ कैका᳚ष्ट॒
कैतस्यां॒ ॅवै । \newline
14. ए॒का॒ष्ट॒केत्ये॑क - अ॒ष्ट॒का । \newline
15. ए॒तस्यां॒ ॅवै वा ए॒तस्या॑ मे॒तस्यां॒ ॅवा ए॒ष ए॒ष वा ए॒तस्या॑ मे॒तस्यां॒ ॅवा ए॒षः । \newline
16. वा ए॒ष ए॒ष वै वा ए॒ष ए॒ता मे॒ता मे॒ष वै वा ए॒ष ए॒ताम् । \newline
17. ए॒ष ए॒ता मे॒ता मे॒ष ए॒ष ए॒ताꣳ रात्रिꣳ॒॒ रात्रि॑ मे॒ता मे॒ष ए॒ष ए॒ताꣳ रात्रि᳚म् । \newline
18. ए॒ताꣳ रात्रिꣳ॒॒ रात्रि॑ मे॒ता मे॒ताꣳ रात्रिं॑ ॅवसति वसति॒ रात्रि॑ मे॒ता मे॒ताꣳ रात्रिं॑ ॅवसति । \newline
19. रात्रिं॑ ॅवसति वसति॒ रात्रिꣳ॒॒ रात्रिं॑ ॅवसति सा॒क्षाथ् सा॒क्षाद् व॑सति॒ रात्रिꣳ॒॒ रात्रिं॑ ॅवसति सा॒क्षात् । \newline
20. व॒स॒ति॒ सा॒क्षाथ् सा॒क्षाद् व॑सति वसति सा॒क्षा दे॒वैव सा॒क्षाद् व॑सति वसति सा॒क्षा दे॒व । \newline
21. सा॒क्षा दे॒वैव सा॒क्षाथ् सा॒क्षा दे॒व सं॑ॅवथ्स॒रꣳ सं॑ॅवथ्स॒र मे॒व सा॒क्षाथ् सा॒क्षा दे॒व सं॑ॅवथ्स॒रम् । \newline
22. सा॒क्षादिति॑ स - अ॒क्षात् । \newline
23. ए॒व सं॑ॅवथ्स॒रꣳ सं॑ॅवथ्स॒र मे॒वैव सं॑ॅवथ्स॒र मा॒रभ्या॒ रभ्य॑ संॅवथ्स॒र मे॒वैव सं॑ॅवथ्स॒र मा॒रभ्य॑ । \newline
24. सं॒ॅव॒थ्स॒र मा॒रभ्या॒ रभ्य॑ संॅवथ्स॒रꣳ सं॑ॅवथ्स॒र मा॒रभ्य॑ दीक्षन्ते दीक्षन्त आ॒रभ्य॑ संॅवथ्स॒रꣳ सं॑ॅवथ्स॒र मा॒रभ्य॑ दीक्षन्ते । \newline
25. सं॒ॅव॒थ्स॒रमिति॑ सं - व॒थ्स॒रम् । \newline
26. आ॒रभ्य॑ दीक्षन्ते दीक्षन्त आ॒रभ्या॒ रभ्य॑ दीक्षन्त॒ आर्त॒ मार्त॑म् दीक्षन्त आ॒रभ्या॒ रभ्य॑ दीक्षन्त॒ आर्त᳚म् । \newline
27. आ॒रभ्येत्या᳚ - रभ्य॑ । \newline
28. दी॒क्ष॒न्त॒ आर्त॒ मार्त॑म् दीक्षन्ते दीक्षन्त॒ आर्तं॒ ॅवै वा आर्त॑म् दीक्षन्ते दीक्षन्त॒ आर्तं॒ ॅवै । \newline
29. आर्तं॒ ॅवै वा आर्त॒ मार्तं॒ ॅवा ए॒त ए॒ते वा आर्त॒ मार्तं॒ ॅवा ए॒ते । \newline
30. वा ए॒त ए॒ते वै वा ए॒ते सं॑ॅवथ्स॒रस्य॑ संॅवथ्स॒र स्यै॒ते वै वा ए॒ते सं॑ॅवथ्स॒रस्य॑ । \newline
31. ए॒ते सं॑ॅवथ्स॒रस्य॑ संॅवथ्स॒र स्यै॒त ए॒ते सं॑ॅवथ्स॒र स्या॒भ्य॑भि सं॑ॅवथ्स॒र स्यै॒त ए॒ते सं॑ॅवथ्स॒र स्या॒भि । \newline
32. सं॒ॅव॒थ्स॒र स्या॒भ्य॑भि सं॑ॅवथ्स॒रस्य॑ संॅवथ्स॒र स्या॒भि दी᳚क्षन्ते दीक्षन्ते॒ ऽभि सं॑ॅवथ्स॒रस्य॑ संॅवथ्स॒र स्या॒भि दी᳚क्षन्ते । \newline
33. सं॒ॅव॒थ्स॒रस्येति॑ सं - व॒थ्स॒रस्य॑ । \newline
34. अ॒भि दी᳚क्षन्ते दीक्षन्ते॒ ऽभ्य॑भि दी᳚क्षन्ते॒ ये ये दी᳚क्षन्ते॒ ऽभ्य॑भि दी᳚क्षन्ते॒ ये । \newline
35. दी॒क्ष॒न्ते॒ ये ये दी᳚क्षन्ते दीक्षन्ते॒ य ए॑काष्ट॒काया॑ मेकाष्ट॒कायां॒ ॅये दी᳚क्षन्ते दीक्षन्ते॒ य ए॑काष्ट॒काया᳚म् । \newline
36. य ए॑काष्ट॒काया॑ मेकाष्ट॒कायां॒ ॅये य ए॑काष्ट॒काया॒म् दीक्ष॑न्ते॒ दीक्ष॑न्त एकाष्ट॒कायां॒ ॅये य ए॑काष्ट॒काया॒म् दीक्ष॑न्ते । \newline
37. ए॒का॒ष्ट॒काया॒म् दीक्ष॑न्ते॒ दीक्ष॑न्त एकाष्ट॒काया॑ मेकाष्ट॒काया॒म् दीक्ष॒न्ते ऽन्त॑नामाना॒ वन्त॑नामानौ॒ दीक्ष॑न्त एकाष्ट॒काया॑ मेकाष्ट॒काया॒म् दीक्ष॒न्ते ऽन्त॑नामानौ । \newline
38. ए॒का॒ष्ट॒काया॒मित्ये॑क - अ॒ष्ट॒काया᳚म् । \newline
39. दीक्ष॒न्ते ऽन्त॑नामाना॒ वन्त॑नामानौ॒ दीक्ष॑न्ते॒ दीक्ष॒न्ते ऽन्त॑नामाना वृ॒तू ऋ॒तू अन्त॑नामानौ॒ दीक्ष॑न्ते॒ दीक्ष॒न्ते ऽन्त॑नामाना वृ॒तू । \newline
40. अन्त॑नामाना वृ॒तू ऋ॒तू अन्त॑नामाना॒ वन्त॑नामाना वृ॒तू भ॑वतो भवत ऋ॒तू अन्त॑नामाना॒ वन्त॑नामाना वृ॒तू भ॑वतः । \newline
41. अन्त॑नामाना॒वित्यन्त॑ - ना॒मा॒नौ॒ । \newline
42. ऋ॒तू भ॑वतो भवत ऋ॒तू ऋ॒तू भ॑वतो॒ व्य॑स्तं॒ ॅव्य॑स्तम् भवत ऋ॒तू ऋ॒तू भ॑वतो॒ व्य॑स्तम् । \newline
43. ऋ॒तू इत्यृ॒तू । \newline
44. भ॒व॒तो॒ व्य॑स्तं॒ ॅव्य॑स्तम् भवतो भवतो॒ व्य॑स्तं॒ ॅवै वै व्य॑स्तम् भवतो भवतो॒ व्य॑स्तं॒ ॅवै । \newline
45. व्य॑स्तं॒ ॅवै वै व्य॑स्तं॒ ॅव्य॑स्तं॒ ॅवा ए॒त ए॒ते वै व्य॑स्तं॒ ॅव्य॑स्तं॒ ॅवा ए॒ते । \newline
46. व्य॑स्त॒मिति॒ वि - अ॒स्त॒म् । \newline
47. वा ए॒त ए॒ते वै वा ए॒ते सं॑ॅवथ्स॒रस्य॑ संॅवथ्स॒र स्यै॒ते वै वा ए॒ते सं॑ॅवथ्स॒रस्य॑ । \newline
48. ए॒ते सं॑ॅवथ्स॒रस्य॑ संॅवथ्स॒र स्यै॒त ए॒ते सं॑ॅवथ्स॒र स्या॒भ्य॑भि सं॑ॅवथ्स॒र स्यै॒त ए॒ते सं॑ॅवथ्स॒र स्या॒भि । \newline
49. सं॒ॅव॒थ्स॒र स्या॒भ्य॑भि सं॑ॅवथ्स॒रस्य॑ संॅवथ्स॒र स्या॒भि दी᳚क्षन्ते दीक्षन्ते॒ ऽभि सं॑ॅवथ्स॒रस्य॑ संॅवथ्स॒र स्या॒भि दी᳚क्षन्ते । \newline
50. सं॒ॅव॒थ्स॒रस्येति॑ सं - व॒थ्स॒रस्य॑ । \newline
51. अ॒भि दी᳚क्षन्ते दीक्षन्ते॒ ऽभ्य॑भि दी᳚क्षन्ते॒ ये ये दी᳚क्षन्ते॒ ऽभ्य॑भि दी᳚क्षन्ते॒ ये । \newline
52. दी॒क्ष॒न्ते॒ ये ये दी᳚क्षन्ते दीक्षन्ते॒ य ए॑काष्ट॒काया॑ मेकाष्ट॒कायां॒ ॅये दी᳚क्षन्ते दीक्षन्ते॒ य ए॑काष्ट॒काया᳚म् । \newline
53. य ए॑काष्ट॒काया॑ मेकाष्ट॒कायां॒ ॅये य ए॑काष्ट॒काया॒म् दीक्ष॑न्ते॒ दीक्ष॑न्त एकाष्ट॒कायां॒ ॅये य ए॑काष्ट॒काया॒म् दीक्ष॑न्ते । \newline
54. ए॒का॒ष्ट॒काया॒म् दीक्ष॑न्ते॒ दीक्ष॑न्त एकाष्ट॒काया॑ मेकाष्ट॒काया॒म् दीक्ष॒न्ते ऽन्त॑नामाना॒ वन्त॑नामानौ॒ दीक्ष॑न्त एकाष्ट॒काया॑ मेकाष्ट॒काया॒म् दीक्ष॒न्ते ऽन्त॑नामानौ । \newline
55. ए॒का॒ष्ट॒काया॒मित्ये॑क - अ॒ष्ट॒काया᳚म् । \newline
56. दीक्ष॒न्ते ऽन्त॑नामाना॒ वन्त॑नामानौ॒ दीक्ष॑न्ते॒ दीक्ष॒न्ते ऽन्त॑नामाना वृ॒तू ऋ॒तू अन्त॑नामानौ॒ दीक्ष॑न्ते॒ दीक्ष॒न्ते ऽन्त॑नामाना वृ॒तू । \newline
57. अन्त॑नामाना वृ॒तू ऋ॒तू अन्त॑नामाना॒ वन्त॑नामाना वृ॒तू भ॑वतो भवत ऋ॒तू अन्त॑नामाना॒ वन्त॑नामाना वृ॒तू भ॑वतः । \newline
58. अन्त॑नामाना॒वित्यन्त॑ - ना॒मा॒नौ॒ । \newline
59. ऋ॒तू भ॑वतो भवत ऋ॒तू ऋ॒तू भ॑वतः फल्गुनीपूर्णमा॒से फ॑ल्गुनीपूर्णमा॒से भ॑वत ऋ॒तू ऋ॒तू भ॑वतः फल्गुनीपूर्णमा॒से । \newline
60. ऋ॒तू इत्यृ॒तू । \newline
61. भ॒व॒तः॒ फ॒ल्गु॒नी॒पू॒र्ण॒मा॒से फ॑ल्गुनीपूर्णमा॒से भ॑वतो भवतः फल्गुनीपूर्णमा॒से दी᳚क्षेरन् दीक्षेरन् फल्गुनीपूर्णमा॒से भ॑वतो भवतः फल्गुनीपूर्णमा॒से दी᳚क्षेरन्न् । \newline
62. फ॒ल्गु॒नी॒पू॒र्ण॒मा॒से दी᳚क्षेरन् दीक्षेरन् फल्गुनीपूर्णमा॒से फ॑ल्गुनीपूर्णमा॒से दी᳚क्षेर॒न् मुख॒म् मुख॑म् दीक्षेरन् फल्गुनीपूर्णमा॒से फ॑ल्गुनीपूर्णमा॒से दी᳚क्षेर॒न् मुख᳚म् । \newline
63. फ॒ल्गु॒नी॒पू॒र्ण॒मा॒स इति॑ फल्गुनी - पू॒र्ण॒मा॒से । \newline
64. दी॒क्षे॒र॒न् मुख॒म् मुख॑म् दीक्षेरन् दीक्षेर॒न् मुखं॒ ॅवै वै मुख॑म् दीक्षेरन् दीक्षेर॒न् मुखं॒ ॅवै । \newline
65. मुखं॒ ॅवै वै मुख॒म् मुखं॒ ॅवा ए॒त दे॒तद् वै मुख॒म् मुखं॒ ॅवा ए॒तत् । \newline
66. वा ए॒त दे॒तद् वै वा ए॒तथ् सं॑ॅवथ्स॒रस्य॑ संॅवथ्स॒र स्यै॒तद् वै वा ए॒तथ् सं॑ॅवथ्स॒रस्य॑ । \newline
67. ए॒तथ् सं॑ॅवथ्स॒रस्य॑ संॅवथ्स॒र स्यै॒त दे॒तथ् सं॑ॅवथ्स॒रस्य॒ यद् यथ् सं॑ॅवथ्स॒र स्यै॒त दे॒तथ् सं॑ॅवथ्स॒रस्य॒ यत् । \newline
\pagebreak
\markright{ TS 7.4.8.2  \hfill https://www.vedavms.in \hfill}

\section{ TS 7.4.8.2 }

\textbf{TS 7.4.8.2 } \newline
\textbf{Samhita Paata} \newline

सं॑ॅवथ्स॒रस्य॒ यत् फ॑ल्गुनी पूर्णमा॒सो मु॑ख॒त ए॒व सं॑ॅवथ्स॒रमा॒रभ्य॑ दीक्षन्ते॒ तस्यैकै॒व नि॒र्या यथ् साम्मे᳚घ्ये विषू॒वान्थ् स॒पंद्य॑ते चित्रापूर्णमा॒से दी᳚क्षेर॒न् मुखं॒ ॅवा ए॒तथ् सं॑ॅवथ्स॒रस्य॒ यच्चि॑त्रापूर्णमा॒सो मु॑ख॒त ए॒व सं॑ॅवथ्स॒रमा॒रभ्य॑ दीक्षन्ते॒ तस्य॒ न का च॒न नि॒र्या भ॑वति चतुर॒हे पु॒रस्ता᳚त् पौर्णमा॒स्यै दी᳚क्षेर॒न् तेषा॑मेकाष्ट॒कायां᳚ क्र॒यः सं प॑द्यते॒ तेनै॑काष्ट॒कां न छ॒म्बट् कु॑र्वन्ति॒ तेषां᳚ - [  ] \newline

\textbf{Pada Paata} \newline

सं॒ॅव॒थ्स॒रस्येति॑ सं - व॒थ्स॒रस्य॑ । यत् । फ॒ल्गु॒नी॒पू॒र्ण॒मा॒स इति॑ फल्गुनी-पू॒र्ण॒मा॒सः । मु॒ख॒तः । ए॒व । सं॒ॅव॒थ्स॒रमिति॑ सं-व॒थ्स॒रम् । आ॒रभ्येत्या᳚ - रभ्य॑ । दी॒क्ष॒न्ते॒ । तस्य॑ । एका᳚ । ए॒व । नि॒र्येति॑ निः - या । यथ् । सांमे᳚घ्य॒ इति॒ सां - मे॒घ्ये॒ । वि॒षू॒वानिति॑ विषु - वान् । स॒पंद्य॑त॒ इति॑ सं - पद्य॑ते । चि॒त्रा॒पू॒र्ण॒मा॒स इति॑ चित्रा - पू॒र्ण॒मा॒से । दी॒क्षे॒र॒न्न् । मुख᳚म् । वै । ए॒तत् । सं॒ॅव॒थ्स॒रस्येति॑ सं - व॒थ्स॒रस्य॑ । यत् । चि॒त्रा॒पू॒र्ण॒मा॒स इति॑ चित्रा - पू॒र्ण॒मा॒सः । मु॒ख॒तः । ए॒व । सं॒ॅव॒थ्स॒रमिति॑ सं - व॒थ्स॒रम् । आ॒रभ्येत्या᳚ - रभ्य॑ । दी॒क्ष॒न्ते॒ । तस्य॑ । न । का । च॒न । नि॒र्येति॑ निः - या । भ॒व॒ति॒ । च॒तु॒र॒ह इति॑ चतुः-अ॒हे । पु॒रस्ता᳚त् । पौ॒र्ण॒मा॒स्या इति॑ पौर्ण-मा॒स्यै । दी॒क्षे॒र॒न्न् । तेषा᳚म् । ए॒का॒ष्ट॒काया॒मित्ये॑क - अ॒ष्ट॒काया᳚म् । क्र॒यः । समिति॑ । प॒द्य॒ते॒ । तेन॑ । ए॒का॒ष्ट॒कामित्ये॑क-अ॒ष्ट॒काम् । न । छ॒म्बट् । कु॒र्व॒न्ति॒ । तेषा᳚म् ।  \newline


\textbf{Krama Paata} \newline

स॒म्ॅव॒थ्स॒रस्य॒ यत् । स॒म्ॅव॒थ्स॒रस्येति॑ सम् - व॒थ्स॒रस्य॑ । यत् फ॑ल्गुनीपूर्णमा॒सः । फ॒ल्गु॒नी॒पू॒र्ण॒मा॒सो मु॑ख॒तः । फ॒ल्गु॒नी॒पू॒र्ण॒मा॒स इति॑ फल्गुनी - पू॒र्ण॒मा॒सः । मु॒ख॒त ए॒व । ए॒व स॑म्ॅवथ्स॒रम् । स॒म्ॅव॒थ्स॒रमा॒रभ्य॑ । स॒म्ॅव॒थ्स॒रमिति॑ सम् - व॒थ्स॒रम् । आ॒रभ्य॑ दीक्षन्ते । आ॒रभ्येत्या᳚ - रभ्य॑ । दी॒क्ष॒न्ते॒ तस्य॑ । तस्यैका᳚ । एकै॒व । ए॒व नि॒र्या । नि॒र्या यत् । नि॒र्येति॑ निः - या । यथ् साम्मे᳚घ्ये । साम्मे᳚घ्ये विषू॒वान् । साम्मे᳚घ्य॒ इति॒ साम् - मे॒घ्ये॒ । वि॒षू॒वान्थ् स॒म्पद्य॑ते । वि॒षू॒वानिति॑ विषु - वान् । स॒म्पद्य॑ते चित्रापूर्णमा॒से । स॒म्पद्य॑त॒ इति॑ सम् - पद्य॑ते । चि॒त्रा॒पू॒र्ण॒मा॒से दी᳚क्षेरन्न् । चि॒त्रा॒पू॒र्ण॒मा॒स इति॑ चित्रा - पू॒र्ण॒मा॒से । दी॒क्षे॒र॒न् मुख᳚म् । मुख॒म् ॅवै । वा ए॒तत् । ए॒तथ् स॑म्ॅवथ्स॒रस्य॑ । स॒म्ॅव॒थ्स॒रस्य॒ यत् । स॒म्ॅव॒थ्स॒रस्येति॑ सम् - व॒थ्स॒रस्य॑ । यच् चि॑त्रापूर्णमा॒सः । चि॒त्रा॒पू॒र्ण॒मा॒सो मु॑ख॒तः । चि॒त्रा॒पू॒र्ण॒मा॒स इति॑ चित्रा - पू॒र्ण॒मा॒सः । मु॒ख॒त ए॒व । ए॒व स॑म्ॅवथ्स॒रम् । स॒म्ॅव॒थ्स॒रमा॒रभ्य॑ । स॒म्ॅव॒थ्स॒रमिति॑ सम् - व॒थ्स॒रम् । आ॒रभ्य॑ दीक्षन्ते । आ॒रभ्येत्या᳚ - रभ्य॑ । दी॒क्ष॒न्ते॒ तस्य॑ । तस्य॒ न । न का । का च॒न । च॒न नि॒र्या । नि॒र्या भ॑वति । नि॒र्येति॑ निः - या । भ॒व॒ति॒ च॒तु॒र॒हे । च॒तु॒र॒हे पु॒रस्ता᳚त् । च॒तु॒र॒ह इति॑ चतुः - अ॒हे । पु॒रस्ता᳚त् पौर्णमा॒स्यै । पौ॒र्ण॒मा॒स्यै दी᳚क्षेरन्न् । पौ॒र्ण॒मा॒स्या इति॑ पौर्ण - मा॒स्यै । दी॒क्षे॒र॒न् तेषा᳚म् । तेषा॑मेकाष्ट॒काया᳚म् । ए॒का॒ष्ट॒काया᳚म् क्र॒यः । ए॒का॒ष्ट॒काया॒मित्ये॑क - अ॒ष्ट॒काया᳚म् । क्र॒यः सम् । सम् प॑द्यते । प॒द्य॒ते॒ तेन॑ । तेनै॑काष्ट॒काम् । ए॒का॒ष्ट॒काम् न । ए॒का॒ष्ट॒कामित्ये॑क - अ॒ष्ट॒काम् । न छ॒म्बट् । छ॒म्बट् कु॑र्वन्ति । कु॒र्व॒न्ति॒ तेषा᳚म् ( ) । तेषा᳚म् पूर्वप॒क्षे \newline

\textbf{Jatai Paata} \newline

1. सं॒ॅव॒थ्स॒रस्य॒ यद् यथ् सं॑ॅवथ्स॒रस्य॑ संॅवथ्स॒रस्य॒ यत् । \newline
2. सं॒ॅव॒थ्स॒रस्येति॑ सं - व॒थ्स॒रस्य॑ । \newline
3. यत् फ॑ल्गुनीपूर्णमा॒सः फ॑ल्गुनीपूर्णमा॒सो यद् यत् फ॑ल्गुनीपूर्णमा॒सः । \newline
4. फ॒ल्गु॒नी॒पू॒र्ण॒मा॒सो मु॑ख॒तो मु॑ख॒तः फ॑ल्गुनीपूर्णमा॒सः फ॑ल्गुनीपूर्णमा॒सो मु॑ख॒तः । \newline
5. फ॒ल्गु॒नी॒पू॒र्ण॒मा॒स इति॑ फल्गुनी - पू॒र्ण॒मा॒सः । \newline
6. मु॒ख॒त ए॒वैव मु॑ख॒तो मु॑ख॒त ए॒व । \newline
7. ए॒व सं॑ॅवथ्स॒रꣳ सं॑ॅवथ्स॒र मे॒वैव सं॑ॅवथ्स॒रम् । \newline
8. सं॒ॅव॒थ्स॒र मा॒रभ्या॒ रभ्य॑ संॅवथ्स॒रꣳ सं॑ॅवथ्स॒र मा॒रभ्य॑ । \newline
9. सं॒ॅव॒थ्स॒रमिति॑ सं - व॒थ्स॒रम् । \newline
10. आ॒रभ्य॑ दीक्षन्ते दीक्षन्त आ॒रभ्या॒ रभ्य॑ दीक्षन्ते । \newline
11. आ॒रभ्येत्या᳚ - रभ्य॑ । \newline
12. दी॒क्ष॒न्ते॒ तस्य॒ तस्य॑ दीक्षन्ते दीक्षन्ते॒ तस्य॑ । \newline
13. तस्यै कैका॒ तस्य॒ तस्यैका᳚ । \newline
14. एकै॒ वैवै कैकै॒व । \newline
15. ए॒व नि॒र्या नि॒र्यै वैव नि॒र्या । \newline
16. नि॒र्या यद् यन् नि॒र्या नि॒र्या यत् । \newline
17. नि॒र्येति॑ निः - या । \newline
18. यथ् साम्मे᳚घ्ये॒ साम्मे᳚घ्ये॒ यद् यथ् साम्मे᳚घ्ये । \newline
19. साम्मे᳚घ्ये विषू॒वान्. वि॑षू॒वान् थ्साम्मे᳚घ्ये॒ साम्मे᳚घ्ये विषू॒वान् । \newline
20. साम्मे᳚घ्य॒ इति॒ सां - मे॒घ्ये॒ । \newline
21. वि॒षू॒वान् थ्सं॒पद्य॑ते सं॒पद्य॑ते विषू॒वान्. वि॑षू॒वान् थ्सं॒पद्य॑ते । \newline
22. वि॒षू॒वानिति॑ विषु - वान् । \newline
23. सं॒पद्य॑ते चित्रापूर्णमा॒से चि॑त्रापूर्णमा॒से सं॒पद्य॑ते सं॒पद्य॑ते चित्रापूर्णमा॒से । \newline
24. सं॒पद्य॑त॒ इति॑ सं - पद्य॑ते । \newline
25. चि॒त्रा॒पू॒र्ण॒मा॒से दी᳚क्षेरन् दीक्षेरꣳ श्चित्रापूर्णमा॒से चि॑त्रापूर्णमा॒से दी᳚क्षेरन्न् । \newline
26. चि॒त्रा॒पू॒र्ण॒मा॒स इति॑ चित्रा - पू॒र्ण॒मा॒से । \newline
27. दी॒क्षे॒र॒न् मुख॒म् मुख॑म् दीक्षेरन् दीक्षेर॒न् मुख᳚म् । \newline
28. मुखं॒ ॅवै वै मुख॒म् मुखं॒ ॅवै । \newline
29. वा ए॒त दे॒तद् वै वा ए॒तत् । \newline
30. ए॒तथ् सं॑ॅवथ्स॒रस्य॑ संॅवथ्स॒र स्यै॒त दे॒तथ् सं॑ॅवथ्स॒रस्य॑ । \newline
31. सं॒ॅव॒थ्स॒रस्य॒ यद् यथ् सं॑ॅवथ्स॒रस्य॑ संॅवथ्स॒रस्य॒ यत् । \newline
32. सं॒ॅव॒थ्स॒रस्येति॑ सं - व॒थ्स॒रस्य॑ । \newline
33. यच् चि॑त्रापूर्णमा॒स श्चि॑त्रापूर्णमा॒सो यद् यच् चि॑त्रापूर्णमा॒सः । \newline
34. चि॒त्रा॒पू॒र्ण॒मा॒सो मु॑ख॒तो मु॑ख॒त श्चि॑त्रापूर्णमा॒स श्चि॑त्रापूर्णमा॒सो मु॑ख॒तः । \newline
35. चि॒त्रा॒पू॒र्ण॒मा॒स इति॑ चित्रा - पू॒र्ण॒मा॒सः । \newline
36. मु॒ख॒त ए॒वैव मु॑ख॒तो मु॑ख॒त ए॒व । \newline
37. ए॒व सं॑ॅवथ्स॒रꣳ सं॑ॅवथ्स॒र मे॒वैव सं॑ॅवथ्स॒रम् । \newline
38. सं॒ॅव॒थ्स॒र मा॒रभ्या॒ रभ्य॑ संॅवथ्स॒रꣳ सं॑ॅवथ्स॒र मा॒रभ्य॑ । \newline
39. सं॒ॅव॒थ्स॒रमिति॑ सं - व॒थ्स॒रम् । \newline
40. आ॒रभ्य॑ दीक्षन्ते दीक्षन्त आ॒रभ्या॒ रभ्य॑ दीक्षन्ते । \newline
41. आ॒रभ्येत्या᳚ - रभ्य॑ । \newline
42. दी॒क्ष॒न्ते॒ तस्य॒ तस्य॑ दीक्षन्ते दीक्षन्ते॒ तस्य॑ । \newline
43. तस्य॒ न न तस्य॒ तस्य॒ न । \newline
44. न का का न न का । \newline
45. का च॒न च॒न का का च॒न । \newline
46. च॒न नि॒र्या नि॒र्या च॒न च॒न नि॒र्या । \newline
47. नि॒र्या भ॑वति भवति नि॒र्या नि॒र्या भ॑वति । \newline
48. नि॒र्येति॑ निः - या । \newline
49. भ॒व॒ति॒ च॒तु॒र॒हे च॑तुर॒हे भ॑वति भवति चतुर॒हे । \newline
50. च॒तु॒र॒हे पु॒रस्ता᳚त् पु॒रस्ता᳚च् चतुर॒हे च॑तुर॒हे पु॒रस्ता᳚त् । \newline
51. च॒तु॒र॒ह इति॑ चतुः - अ॒हे । \newline
52. पु॒रस्ता᳚त् पौर्णमा॒स्यै पौ᳚र्णमा॒स्यै पु॒रस्ता᳚त् पु॒रस्ता᳚त् पौर्णमा॒स्यै । \newline
53. पौ॒र्ण॒मा॒स्यै दी᳚क्षेरन् दीक्षेरन् पौर्णमा॒स्यै पौ᳚र्णमा॒स्यै दी᳚क्षेरन्न् । \newline
54. पौ॒र्ण॒मा॒स्या इति॑ पौर्ण - मा॒स्यै । \newline
55. दी॒क्षे॒र॒न् तेषा॒म् तेषा᳚म् दीक्षेरन् दीक्षेर॒न् तेषा᳚म् । \newline
56. तेषा॑ मेकाष्ट॒काया॑ मेकाष्ट॒काया॒म् तेषा॒म् तेषा॑ मेकाष्ट॒काया᳚म् । \newline
57. ए॒का॒ष्ट॒काया᳚म् क्र॒यः क्र॒य ए॑काष्ट॒काया॑ मेकाष्ट॒काया᳚म् क्र॒यः । \newline
58. ए॒का॒ष्ट॒काया॒मित्ये॑क - अ॒ष्ट॒काया᳚म् । \newline
59. क्र॒यः सꣳ सम् क्र॒यः क्र॒यः सम् । \newline
60. सम् प॑द्यते पद्यते॒ सꣳ सम् प॑द्यते । \newline
61. प॒द्य॒ते॒ तेन॒ तेन॑ पद्यते पद्यते॒ तेन॑ । \newline
62. तेनै॑ काष्ट॒का मे॑काष्ट॒काम् तेन॒ तेनै॑ काष्ट॒काम् । \newline
63. ए॒का॒ष्ट॒कान् न नैका᳚ष्ट॒का मे॑काष्ट॒कान् न । \newline
64. ए॒का॒ष्ट॒कामित्ये॑क - अ॒ष्ट॒काम् । \newline
65. न छ॒म्बट् छ॒म्बण् ण न छ॒म्बट् । \newline
66. छ॒म्बट् कु॑र्वन्ति कुर्वन्ति छ॒म्बट् छ॒म्बट् कु॑र्वन्ति । \newline
67. कु॒र्व॒न्ति॒ तेषा॒म् तेषा᳚म् कुर्वन्ति कुर्वन्ति॒ तेषा᳚म् । \newline
68. तेषा᳚म् पूर्वप॒क्षे पू᳚र्वप॒क्षे तेषा॒म् तेषा᳚म् पूर्वप॒क्षे । \newline

\textbf{Ghana Paata } \newline

1. सं॒ॅव॒थ्स॒रस्य॒ यद् यथ् सं॑ॅवथ्स॒रस्य॑ संॅवथ्स॒रस्य॒ यत् फ॑ल्गुनीपूर्णमा॒सः फ॑ल्गुनीपूर्णमा॒सो यथ् सं॑ॅवथ्स॒रस्य॑ संॅवथ्स॒रस्य॒ यत् फ॑ल्गुनीपूर्णमा॒सः । \newline
2. सं॒ॅव॒थ्स॒रस्येति॑ सं - व॒थ्स॒रस्य॑ । \newline
3. यत् फ॑ल्गुनीपूर्णमा॒सः फ॑ल्गुनीपूर्णमा॒सो यद् यत् फ॑ल्गुनीपूर्णमा॒सो मु॑ख॒तो मु॑ख॒तः फ॑ल्गुनीपूर्णमा॒सो यद् यत् फ॑ल्गुनीपूर्णमा॒सो मु॑ख॒तः । \newline
4. फ॒ल्गु॒नी॒पू॒र्ण॒मा॒सो मु॑ख॒तो मु॑ख॒तः फ॑ल्गुनीपूर्णमा॒सः फ॑ल्गुनीपूर्णमा॒सो मु॑ख॒त ए॒वैव मु॑ख॒तः फ॑ल्गुनीपूर्णमा॒सः फ॑ल्गुनीपूर्णमा॒सो मु॑ख॒त ए॒व । \newline
5. फ॒ल्गु॒नी॒पू॒र्ण॒मा॒स इति॑ फल्गुनी - पू॒र्ण॒मा॒सः । \newline
6. मु॒ख॒त ए॒वैव मु॑ख॒तो मु॑ख॒त ए॒व सं॑ॅवथ्स॒रꣳ सं॑ॅवथ्स॒र मे॒व मु॑ख॒तो मु॑ख॒त ए॒व सं॑ॅवथ्स॒रम् । \newline
7. ए॒व सं॑ॅवथ्स॒रꣳ सं॑ॅवथ्स॒र मे॒वैव सं॑ॅवथ्स॒र मा॒रभ्या॒ रभ्य॑ संॅवथ्स॒र मे॒वैव सं॑ॅवथ्स॒र मा॒रभ्य॑ । \newline
8. सं॒ॅव॒थ्स॒र मा॒रभ्या॒ रभ्य॑ संॅवथ्स॒रꣳ सं॑ॅवथ्स॒र मा॒रभ्य॑ दीक्षन्ते दीक्षन्त आ॒रभ्य॑ संॅवथ्स॒रꣳ सं॑ॅवथ्स॒र मा॒रभ्य॑ दीक्षन्ते । \newline
9. सं॒ॅव॒थ्स॒रमिति॑ सं - व॒थ्स॒रम् । \newline
10. आ॒रभ्य॑ दीक्षन्ते दीक्षन्त आ॒रभ्या॒ रभ्य॑ दीक्षन्ते॒ तस्य॒ तस्य॑ दीक्षन्त आ॒रभ्या॒ रभ्य॑ दीक्षन्ते॒ तस्य॑ । \newline
11. आ॒रभ्येत्या᳚ - रभ्य॑ । \newline
12. दी॒क्ष॒न्ते॒ तस्य॒ तस्य॑ दीक्षन्ते दीक्षन्ते॒ तस्यै कैका॒ तस्य॑ दीक्षन्ते दीक्षन्ते॒ तस्यैका᳚ । \newline
13. तस्यै कैका॒ तस्य॒ तस्यै कै॒वै वैका॒ तस्य॒ तस्यै कै॒व । \newline
14. एकै॒ वैवै कैकै॒व नि॒र्या नि॒र्यै वैकै कै॒व नि॒र्या । \newline
15. ए॒व नि॒र्या नि॒र्यै वैव नि॒र्या यद् यन् नि॒र्यै वैव नि॒र्या यत् । \newline
16. नि॒र्या यद् यन् नि॒र्या नि॒र्या यथ् साम्मे᳚घ्ये॒ साम्मे᳚घ्ये॒ यन् नि॒र्या नि॒र्या यथ् साम्मे᳚घ्ये । \newline
17. नि॒र्येति॑ निः - या । \newline
18. यथ् साम्मे᳚घ्ये॒ साम्मे᳚घ्ये॒ यद् यथ् साम्मे᳚घ्ये विषू॒वान्. वि॑षू॒वान् थ्साम्मे᳚घ्ये॒ यद् यथ् साम्मे᳚घ्ये विषू॒वान् । \newline
19. साम्मे᳚घ्ये विषू॒वान्. वि॑षू॒वान् थ्साम्मे᳚घ्ये॒ साम्मे᳚घ्ये विषू॒वान् थ्सं॒पद्य॑ते सं॒पद्य॑ते विषू॒वान् थ्साम्मे᳚घ्ये॒ साम्मे᳚घ्ये विषू॒वान् थ्सं॒पद्य॑ते । \newline
20. साम्मे᳚घ्य॒ इति॒ सां - मे॒घ्ये॒ । \newline
21. वि॒षू॒वान् थ्सं॒पद्य॑ते सं॒पद्य॑ते विषू॒वान्. वि॑षू॒वान् थ्सं॒पद्य॑ते चित्रापूर्णमा॒से चि॑त्रापूर्णमा॒से सं॒पद्य॑ते विषू॒वान्. वि॑षू॒वान् थ्सं॒पद्य॑ते चित्रापूर्णमा॒से । \newline
22. वि॒षू॒वानिति॑ विषु - वान् । \newline
23. सं॒पद्य॑ते चित्रापूर्णमा॒से चि॑त्रापूर्णमा॒से सं॒पद्य॑ते सं॒पद्य॑ते चित्रापूर्णमा॒से दी᳚क्षेरन् दीक्षेरꣳ श्चित्रापूर्णमा॒से सं॒पद्य॑ते सं॒पद्य॑ते चित्रापूर्णमा॒से दी᳚क्षेरन्न् । \newline
24. सं॒पद्य॑त॒ इति॑ सं - पद्य॑ते । \newline
25. चि॒त्रा॒पू॒र्ण॒मा॒से दी᳚क्षेरन् दीक्षेरꣳ श्चित्रापूर्णमा॒से चि॑त्रापूर्णमा॒से दी᳚क्षेर॒न् मुख॒म् मुख॑म् दीक्षेरꣳ श्चित्रापूर्णमा॒से चि॑त्रापूर्णमा॒से दी᳚क्षेर॒न् मुख᳚म् । \newline
26. चि॒त्रा॒पू॒र्ण॒मा॒स इति॑ चित्रा - पू॒र्ण॒मा॒से । \newline
27. दी॒क्षे॒र॒न् मुख॒म् मुख॑म् दीक्षेरन् दीक्षेर॒न् मुखं॒ ॅवै वै मुख॑म् दीक्षेरन् दीक्षेर॒न् मुखं॒ ॅवै । \newline
28. मुखं॒ ॅवै वै मुख॒म् मुखं॒ ॅवा ए॒त दे॒तद् वै मुख॒म् मुखं॒ ॅवा ए॒तत् । \newline
29. वा ए॒त दे॒तद् वै वा ए॒तथ् सं॑ॅवथ्स॒रस्य॑ संॅवथ्स॒र स्यै॒तद् वै वा ए॒तथ् सं॑ॅवथ्स॒रस्य॑ । \newline
30. ए॒तथ् सं॑ॅवथ्स॒रस्य॑ संॅवथ्स॒र स्यै॒त दे॒तथ् सं॑ॅवथ्स॒रस्य॒ यद् यथ् सं॑ॅवथ्स॒र स्यै॒त दे॒तथ् सं॑ॅवथ्स॒रस्य॒ यत् । \newline
31. सं॒ॅव॒थ्स॒रस्य॒ यद् यथ् सं॑ॅवथ्स॒रस्य॑ संॅवथ्स॒रस्य॒ यच् चि॑त्रापूर्णमा॒स श्चि॑त्रापूर्णमा॒सो यथ् सं॑ॅवथ्स॒रस्य॑ संॅवथ्स॒रस्य॒ यच् चि॑त्रापूर्णमा॒सः । \newline
32. सं॒ॅव॒थ्स॒रस्येति॑ सं - व॒थ्स॒रस्य॑ । \newline
33. यच् चि॑त्रापूर्णमा॒स श्चि॑त्रापूर्णमा॒सो यद् यच् चि॑त्रापूर्णमा॒सो मु॑ख॒तो मु॑ख॒त श्चि॑त्रापूर्णमा॒सो यद् यच् चि॑त्रापूर्णमा॒सो मु॑ख॒तः । \newline
34. चि॒त्रा॒पू॒र्ण॒मा॒सो मु॑ख॒तो मु॑ख॒त श्चि॑त्रापूर्णमा॒स श्चि॑त्रापूर्णमा॒सो मु॑ख॒त ए॒वैव मु॑ख॒त श्चि॑त्रापूर्णमा॒स श्चि॑त्रापूर्णमा॒सो मु॑ख॒त ए॒व । \newline
35. चि॒त्रा॒पू॒र्ण॒मा॒स इति॑ चित्रा - पू॒र्ण॒मा॒सः । \newline
36. मु॒ख॒त ए॒वैव मु॑ख॒तो मु॑ख॒त ए॒व सं॑ॅवथ्स॒रꣳ सं॑ॅवथ्स॒र मे॒व मु॑ख॒तो मु॑ख॒त ए॒व सं॑ॅवथ्स॒रम् । \newline
37. ए॒व सं॑ॅवथ्स॒रꣳ सं॑ॅवथ्स॒र मे॒वैव सं॑ॅवथ्स॒र मा॒रभ्या॒ रभ्य॑ संॅवथ्स॒र मे॒वैव सं॑ॅवथ्स॒र मा॒रभ्य॑ । \newline
38. सं॒ॅव॒थ्स॒र मा॒रभ्या॒ रभ्य॑ संॅवथ्स॒रꣳ सं॑ॅवथ्स॒र मा॒रभ्य॑ दीक्षन्ते दीक्षन्त आ॒रभ्य॑ संॅवथ्स॒रꣳ सं॑ॅवथ्स॒र मा॒रभ्य॑ दीक्षन्ते । \newline
39. सं॒ॅव॒थ्स॒रमिति॑ सं - व॒थ्स॒रम् । \newline
40. आ॒रभ्य॑ दीक्षन्ते दीक्षन्त आ॒रभ्या॒ रभ्य॑ दीक्षन्ते॒ तस्य॒ तस्य॑ दीक्षन्त आ॒रभ्या॒ रभ्य॑ दीक्षन्ते॒ तस्य॑ । \newline
41. आ॒रभ्येत्या᳚ - रभ्य॑ । \newline
42. दी॒क्ष॒न्ते॒ तस्य॒ तस्य॑ दीक्षन्ते दीक्षन्ते॒ तस्य॒ न न तस्य॑ दीक्षन्ते दीक्षन्ते॒ तस्य॒ न । \newline
43. तस्य॒ न न तस्य॒ तस्य॒ न का का न तस्य॒ तस्य॒ न का । \newline
44. न का का न न का च॒न च॒न का न न का च॒न । \newline
45. का च॒न च॒न का का च॒न नि॒र्या नि॒र्या च॒न का का च॒न नि॒र्या । \newline
46. च॒न नि॒र्या नि॒र्या च॒न च॒न नि॒र्या भ॑वति भवति नि॒र्या च॒न च॒न नि॒र्या भ॑वति । \newline
47. नि॒र्या भ॑वति भवति नि॒र्या नि॒र्या भ॑वति चतुर॒हे च॑तुर॒हे भ॑वति नि॒र्या नि॒र्या भ॑वति चतुर॒हे । \newline
48. नि॒र्येति॑ निः - या । \newline
49. भ॒व॒ति॒ च॒तु॒र॒हे च॑तुर॒हे भ॑वति भवति चतुर॒हे पु॒रस्ता᳚त् पु॒रस्ता᳚च् चतुर॒हे भ॑वति भवति चतुर॒हे पु॒रस्ता᳚त् । \newline
50. च॒तु॒र॒हे पु॒रस्ता᳚त् पु॒रस्ता᳚च् चतुर॒हे च॑तुर॒हे पु॒रस्ता᳚त् पौर्णमा॒स्यै पौ᳚र्णमा॒स्यै पु॒रस्ता᳚च् चतुर॒हे च॑तुर॒हे पु॒रस्ता᳚त् पौर्णमा॒स्यै । \newline
51. च॒तु॒र॒ह इति॑ चतुः - अ॒हे । \newline
52. पु॒रस्ता᳚त् पौर्णमा॒स्यै पौ᳚र्णमा॒स्यै पु॒रस्ता᳚त् पु॒रस्ता᳚त् पौर्णमा॒स्यै दी᳚क्षेरन् दीक्षेरन् पौर्णमा॒स्यै पु॒रस्ता᳚त् पु॒रस्ता᳚त् पौर्णमा॒स्यै दी᳚क्षेरन्न् । \newline
53. पौ॒र्ण॒मा॒स्यै दी᳚क्षेरन् दीक्षेरन् पौर्णमा॒स्यै पौ᳚र्णमा॒स्यै दी᳚क्षेर॒न् तेषा॒म् तेषा᳚म् दीक्षेरन् पौर्णमा॒स्यै पौ᳚र्णमा॒स्यै दी᳚क्षेर॒न् तेषा᳚म् । \newline
54. पौ॒र्ण॒मा॒स्या इति॑ पौर्ण - मा॒स्यै । \newline
55. दी॒क्षे॒र॒न् तेषा॒म् तेषा᳚म् दीक्षेरन् दीक्षेर॒न् तेषा॑ मेकाष्ट॒काया॑ मेकाष्ट॒काया॒म् तेषा᳚म् दीक्षेरन् दीक्षेर॒न् तेषा॑ मेकाष्ट॒काया᳚म् । \newline
56. तेषा॑ मेकाष्ट॒काया॑ मेकाष्ट॒काया॒म् तेषा॒म् तेषा॑ मेकाष्ट॒काया᳚म् क्र॒यः क्र॒य ए॑काष्ट॒काया॒म् तेषा॒म् तेषा॑ मेकाष्ट॒काया᳚म् क्र॒यः । \newline
57. ए॒का॒ष्ट॒काया᳚म् क्र॒यः क्र॒य ए॑काष्ट॒काया॑ मेकाष्ट॒काया᳚म् क्र॒यः सꣳ सम् क्र॒य ए॑काष्ट॒काया॑ मेकाष्ट॒काया᳚म् क्र॒यः सम् । \newline
58. ए॒का॒ष्ट॒काया॒मित्ये॑क - अ॒ष्ट॒काया᳚म् । \newline
59. क्र॒यः सꣳ सम् क्र॒यः क्र॒यः सम् प॑द्यते पद्यते॒ सम् क्र॒यः क्र॒यः सम् प॑द्यते । \newline
60. सम् प॑द्यते पद्यते॒ सꣳ सम् प॑द्यते॒ तेन॒ तेन॑ पद्यते॒ सꣳ सम् प॑द्यते॒ तेन॑ । \newline
61. प॒द्य॒ते॒ तेन॒ तेन॑ पद्यते पद्यते॒ तेनै॑काष्ट॒का मे॑काष्ट॒काम् तेन॑ पद्यते पद्यते॒ तेनै॑काष्ट॒काम् । \newline
62. तेनै॑काष्ट॒का मे॑काष्ट॒काम् तेन॒ तेनै॑काष्ट॒कान् न नैका᳚ष्ट॒काम् तेन॒ तेनै॑काष्ट॒कान् न । \newline
63. ए॒का॒ष्ट॒कान् न नैका᳚ष्ट॒का मे॑काष्ट॒कान् न छ॒म्बट् छ॒म्बण् णैका᳚ष्ट॒का मे॑काष्ट॒कान् न छ॒म्बट् । \newline
64. ए॒का॒ष्ट॒कामित्ये॑क - अ॒ष्ट॒काम् । \newline
65. न छ॒म्बट् छ॒म्बण् ण न छ॒म्बट् कु॑र्वन्ति कुर्वन्ति छ॒म्बण् ण न छ॒म्बट् कु॑र्वन्ति । \newline
66. छ॒म्बट् कु॑र्वन्ति कुर्वन्ति छ॒म्बट् छ॒म्बट् कु॑र्वन्ति॒ तेषा॒म् तेषा᳚म् कुर्वन्ति छ॒म्बट् छ॒म्बट् कु॑र्वन्ति॒ तेषा᳚म् । \newline
67. कु॒र्व॒न्ति॒ तेषा॒म् तेषा᳚म् कुर्वन्ति कुर्वन्ति॒ तेषा᳚म् पूर्वप॒क्षे पू᳚र्वप॒क्षे तेषा᳚म् कुर्वन्ति कुर्वन्ति॒ तेषा᳚म् पूर्वप॒क्षे । \newline
68. तेषा᳚म् पूर्वप॒क्षे पू᳚र्वप॒क्षे तेषा॒म् तेषा᳚म् पूर्वप॒क्षे सु॒त्या सु॒त्या पू᳚र्वप॒क्षे तेषा॒म् तेषा᳚म् पूर्वप॒क्षे सु॒त्या । \newline
\pagebreak
\markright{ TS 7.4.8.3  \hfill https://www.vedavms.in \hfill}

\section{ TS 7.4.8.3 }

\textbf{TS 7.4.8.3 } \newline
\textbf{Samhita Paata} \newline

पूर्वप॒क्षे सु॒त्या सं प॑द्यते पूर्वप॒क्षं मासा॑ अ॒भि सं प॑द्यन्ते॒ ते पू᳚र्वप॒क्ष उत् ति॑ष्ठन्ति॒ तानु॒त्तिष्ठ॑त॒ ओष॑धयो॒ वन॒स्पत॒योऽनूत् ति॑ष्ठन्ति॒ तान् क॑ल्या॒णी की॒र्तिरनूत् ति॑ष्ठ॒त्यरा᳚थ् सुरि॒मे यज॑माना॒ इति॒ तदनु॒ सर्वे॑ राद्ध्नुवन्ति ॥ \newline

\textbf{Pada Paata} \newline

पू॒र्व॒प॒क्ष इति॑ पूर्व - प॒क्षे । सु॒त्या । समिति॑ । प॒द्य॒ते॒ । पू॒र्व॒प॒क्षमिति॑ पूर्व - प॒क्षम् । मासाः᳚ । अ॒भि । समिति॑ । प॒द्य॒न्ते॒ । ते । पू॒र्व॒प॒क्ष इति॑ पूर्व-प॒क्षे । उदिति॑ । ति॒ष्ठ॒न्ति॒ । तान् । उ॒त्तिष्ठ॑त॒ इत्यु॑त्-तिष्ठ॑तः । ओष॑धयः । वन॒स्पत॑यः । अनु॑ । उदिति॑ । ति॒ष्ठ॒न्ति॒ । तान् । क॒ल्या॒णी । की॒र्तिः । अनु॑ । उदिति॑ । ति॒ष्ठ॒ति॒ । अरा᳚थ्सुः । इ॒मे । यज॑मानाः । इति॑ । तत् । अन्विति॑ । सर्वे᳚ । रा॒द्ध्नु॒व॒न्ति॒ ॥  \newline


\textbf{Krama Paata} \newline

पू॒र्व॒प॒क्षे सु॒त्या । पू॒र्व॒प॒क्ष इति॑ पूर्व - प॒क्षे । सु॒त्या सम् । सम् प॑द्यते । प॒द्य॒ते॒ पू॒र्व॒प॒क्षम् । पू॒र्व॒प॒क्षम् मासाः᳚ । पू॒र्व॒प॒क्षमिति॑ पूर्व - प॒क्षम् । मासा॑ अ॒भि । अ॒भि सम् । सम् प॑द्यन्ते । प॒द्य॒न्ते॒ ते । ते पू᳚र्वप॒क्षे । पू॒र्व॒प॒क्ष उत् । पू॒र्व॒प॒क्ष इति॑ पूर्व - प॒क्षे । उत् ति॑ष्ठन्ति । ति॒ष्ठ॒न्ति॒ तान् । तानु॒त्तिष्ठ॑तः । उ॒त्तिष्ठ॑त॒ ओष॑धयः । उ॒त्तिष्ठ॑त॒ इत्यु॑त् - तिष्ठ॑तः । ओष॑धयो॒ वन॒स्पत॑यः । वन॒स्पत॒योऽनु॑ । अनूत् । उत् ति॑ष्ठन्ति । ति॒ष्ठ॒न्ति॒ तान् । तान् क॑ल्या॒णी । क॒ल्या॒णी की॒र्तिः । की॒र्तिरनु॑ । अनूत् । उत् ति॑ष्ठति । ति॒ष्ठ॒त्यरा᳚थ्सुः । अरा᳚थ्सुरि॒मे । इ॒मे यज॑मानाः । यज॑माना॒ इति॑ । इति॒ तत् । तदनु॑ । अनु॒ सर्वे᳚ । सर्वे॑ राद्ध्नुवन्ति । रा॒द्ध्नु॒व॒न्तीति॑ राद्ध्नुवन्ति । \newline

\textbf{Jatai Paata} \newline

1. पू॒र्व॒प॒क्षे सु॒त्या सु॒त्या पू᳚र्वप॒क्षे पू᳚र्वप॒क्षे सु॒त्या । \newline
2. पू॒र्व॒प॒क्ष इति॑ पूर्व - प॒क्षे । \newline
3. सु॒त्या सꣳ सꣳ सु॒त्या सु॒त्या सम् । \newline
4. सम् प॑द्यते पद्यते॒ सꣳ सम् प॑द्यते । \newline
5. प॒द्य॒ते॒ पू॒र्व॒प॒क्षम् पू᳚र्वप॒क्षम् प॑द्यते पद्यते पूर्वप॒क्षम् । \newline
6. पू॒र्व॒प॒क्षम् मासा॒ मासाः᳚ पूर्वप॒क्षम् पू᳚र्वप॒क्षम् मासाः᳚ । \newline
7. पू॒र्व॒प॒क्षमिति॑ पूर्व - प॒क्षम् । \newline
8. मासा॑ अ॒भ्य॑भि मासा॒ मासा॑ अ॒भि । \newline
9. अ॒भि सꣳ स म॒भ्य॑भि सम् । \newline
10. सम् प॑द्यन्ते पद्यन्ते॒ सꣳ सम् प॑द्यन्ते । \newline
11. प॒द्य॒न्ते॒ ते ते प॑द्यन्ते पद्यन्ते॒ ते । \newline
12. ते पू᳚र्वप॒क्षे पू᳚र्वप॒क्षे ते ते पू᳚र्वप॒क्षे । \newline
13. पू॒र्व॒प॒क्ष उदुत् पू᳚र्वप॒क्षे पू᳚र्वप॒क्ष उत् । \newline
14. पू॒र्व॒प॒क्ष इति॑ पूर्व - प॒क्षे । \newline
15. उत् ति॑ष्ठन्ति तिष्ठ॒ न्त्युदुत् ति॑ष्ठन्ति । \newline
16. ति॒ष्ठ॒न्ति॒ ताꣳ स्ताꣳ स्ति॑ष्ठन्ति तिष्ठन्ति॒ तान् । \newline
17. तानु॒त्तिष्ठ॑त उ॒त्तिष्ठ॑त॒ स्ताꣳ स्ता नु॒त्तिष्ठ॑तः । \newline
18. उ॒त्तिष्ठ॑त॒ ओष॑धय॒ ओष॑धय उ॒त्तिष्ठ॑त उ॒त्तिष्ठ॑त॒ ओष॑धयः । \newline
19. उ॒त्तिष्ठ॑त॒ इत्यु॑त् - तिष्ठ॑तः । \newline
20. ओष॑धयो॒ वन॒स्पत॑यो॒ वन॒स्पत॑य॒ ओष॑धय॒ ओष॑धयो॒ वन॒स्पत॑यः । \newline
21. वन॒स्पत॒यो ऽन्वनु॒ वन॒स्पत॑यो॒ वन॒स्पत॒यो ऽनु॑ । \newline
22. अनू दुदन् वनूत् । \newline
23. उत् ति॑ष्ठन्ति तिष्ठ॒ न्त्युदुत् ति॑ष्ठन्ति । \newline
24. ति॒ष्ठ॒न्ति॒ ताꣳ स्ताꣳ स्ति॑ष्ठन्ति तिष्ठन्ति॒ तान् । \newline
25. तान् क॑ल्या॒णी क॑ल्या॒णी ताꣳ स्तान् क॑ल्या॒णी । \newline
26. क॒ल्या॒णी की॒र्तिः की॒र्तिः क॑ल्या॒णी क॑ल्या॒णी की॒र्तिः । \newline
27. की॒र्ति रन्वनु॑ की॒र्तिः की॒र्ति रनु॑ । \newline
28. अनू दुदन् वनूत् । \newline
29. उत् ति॑ष्ठति तिष्ठ॒ त्युदुत् ति॑ष्ठति । \newline
30. ति॒ष्ठ॒ त्यरा᳚थ्सु॒ ररा᳚थ्सु स्तिष्ठति तिष्ठ॒त्य रा᳚थ्सुः । \newline
31. अरा᳚थ्सु रि॒म इ॒मे ऽरा᳚थ्सु॒ ररा᳚थ्सु रि॒मे । \newline
32. इ॒मे यज॑माना॒ यज॑माना इ॒म इ॒मे यज॑मानाः । \newline
33. यज॑माना॒ इतीति॒ यज॑माना॒ यज॑माना॒ इति॑ । \newline
34. इति॒ तत् तदितीति॒ तत् । \newline
35. तदन् वनु॒ तत् तदनु॑ । \newline
36. अनु॒ सर्वे॒ सर्वे ऽन्वनु॒ सर्वे᳚ । \newline
37. सर्वे॑ राद्ध्नुवन्ति राद्ध्नुवन्ति॒ सर्वे॒ सर्वे॑ राद्ध्नुवन्ति । \newline
38. रा॒द्ध्नु॒व॒न्तीति॑ राद्ध्नुवन्ति । \newline

\textbf{Ghana Paata } \newline

1. पू॒र्व॒प॒क्षे सु॒त्या सु॒त्या पू᳚र्वप॒क्षे पू᳚र्वप॒क्षे सु॒त्या सꣳ सꣳ सु॒त्या पू᳚र्वप॒क्षे पू᳚र्वप॒क्षे सु॒त्या सम् । \newline
2. पू॒र्व॒प॒क्ष इति॑ पूर्व - प॒क्षे । \newline
3. सु॒त्या सꣳ सꣳ सु॒त्या सु॒त्या सम् प॑द्यते पद्यते॒ सꣳ सु॒त्या सु॒त्या सम् प॑द्यते । \newline
4. सम् प॑द्यते पद्यते॒ सꣳ सम् प॑द्यते पूर्वप॒क्षम् पू᳚र्वप॒क्षम् प॑द्यते॒ सꣳ सम् प॑द्यते पूर्वप॒क्षम् । \newline
5. प॒द्य॒ते॒ पू॒र्व॒प॒क्षम् पू᳚र्वप॒क्षम् प॑द्यते पद्यते पूर्वप॒क्षम् मासा॒ मासाः᳚ पूर्वप॒क्षम् प॑द्यते पद्यते पूर्वप॒क्षम् मासाः᳚ । \newline
6. पू॒र्व॒प॒क्षम् मासा॒ मासाः᳚ पूर्वप॒क्षम् पू᳚र्वप॒क्षम् मासा॑ अ॒भ्य॑भि मासाः᳚ पूर्वप॒क्षम् पू᳚र्वप॒क्षम् मासा॑ अ॒भि । \newline
7. पू॒र्व॒प॒क्षमिति॑ पूर्व - प॒क्षम् । \newline
8. मासा॑ अ॒भ्य॑भि मासा॒ मासा॑ अ॒भि सꣳ स म॒भि मासा॒ मासा॑ अ॒भि सम् । \newline
9. अ॒भि सꣳ स म॒भ्य॑भि सम् प॑द्यन्ते पद्यन्ते॒ स म॒भ्य॑भि सम् प॑द्यन्ते । \newline
10. सम् प॑द्यन्ते पद्यन्ते॒ सꣳ सम् प॑द्यन्ते॒ ते ते प॑द्यन्ते॒ सꣳ सम् प॑द्यन्ते॒ ते । \newline
11. प॒द्य॒न्ते॒ ते ते प॑द्यन्ते पद्यन्ते॒ ते पू᳚र्वप॒क्षे पू᳚र्वप॒क्षे ते प॑द्यन्ते पद्यन्ते॒ ते पू᳚र्वप॒क्षे । \newline
12. ते पू᳚र्वप॒क्षे पू᳚र्वप॒क्षे ते ते पू᳚र्वप॒क्ष उदुत् पू᳚र्वप॒क्षे ते ते पू᳚र्वप॒क्ष उत् । \newline
13. पू॒र्व॒प॒क्ष उदुत् पू᳚र्वप॒क्षे पू᳚र्वप॒क्ष उत् ति॑ष्ठन्ति तिष्ठ॒ न्त्युत् पू᳚र्वप॒क्षे पू᳚र्वप॒क्ष उत् ति॑ष्ठन्ति । \newline
14. पू॒र्व॒प॒क्ष इति॑ पूर्व - प॒क्षे । \newline
15. उत् ति॑ष्ठन्ति तिष्ठ॒ न्त्युदुत् ति॑ष्ठन्ति॒ ताꣳ स्ताꣳ स्ति॑ष्ठ॒ न्त्युदुत् ति॑ष्ठन्ति॒ तान् । \newline
16. ति॒ष्ठ॒न्ति॒ ताꣳ स्ताꣳ स्ति॑ष्ठन्ति तिष्ठन्ति॒ तानु॒त्तिष्ठ॑त उ॒त्तिष्ठ॑त॒ स्ताꣳ स्ति॑ष्ठन्ति 
तिष्ठन्ति॒ तानु॒त्तिष्ठ॑तः । \newline
17. तानु॒त्तिष्ठ॑त उ॒त्तिष्ठ॑त॒ स्ताꣳ स्ता नु॒त्तिष्ठ॑त॒ ओष॑धय॒ ओष॑धय उ॒त्तिष्ठ॑त॒ स्ताꣳ
स्तानु॒त्तिष्ठ॑त॒ ओष॑धयः । \newline
18. उ॒त्तिष्ठ॑त॒ ओष॑धय॒ ओष॑धय उ॒त्तिष्ठ॑त उ॒त्तिष्ठ॑त॒ ओष॑धयो॒ वन॒स्पत॑यो॒ वन॒स्पत॑य॒ ओष॑धय उ॒त्तिष्ठ॑त उ॒त्तिष्ठ॑त॒ ओष॑धयो॒ वन॒स्पत॑यः । \newline
19. उ॒त्तिष्ठ॑त॒ इत्यु॑त् - तिष्ठ॑तः । \newline
20. ओष॑धयो॒ वन॒स्पत॑यो॒ वन॒स्पत॑य॒ ओष॑धय॒ ओष॑धयो॒ वन॒स्पत॒यो ऽन्वनु॒ वन॒स्पत॑य॒ ओष॑धय॒ ओष॑धयो॒ वन॒स्पत॒यो ऽनु॑ । \newline
21. वन॒स्पत॒यो ऽन्वनु॒ वन॒स्पत॑यो॒ वन॒स्पत॒यो ऽनूदु दनु॒ वन॒स्पत॑यो॒ वन॒स्पत॒यो ऽनूत् । \newline
22. अनूदु दन् वनूत् ति॑ष्ठन्ति तिष्ठ॒ न्त्युदन् वनूत् ति॑ष्ठन्ति । \newline
23. उत् ति॑ष्ठन्ति तिष्ठ॒ न्त्युदुत् ति॑ष्ठन्ति॒ ताꣳ स्ताꣳ स्ति॑ष्ठ॒ न्त्युदुत् ति॑ष्ठन्ति॒ तान् । \newline
24. ति॒ष्ठ॒न्ति॒ ताꣳ स्ताꣳ स्ति॑ष्ठन्ति तिष्ठन्ति॒ तान् क॑ल्या॒णी क॑ल्या॒णी ताꣳ स्ति॑ष्ठन्ति तिष्ठन्ति॒ तान् क॑ल्या॒णी । \newline
25. तान् क॑ल्या॒णी क॑ल्या॒णी ताꣳ स्तान् क॑ल्या॒णी की॒र्तिः की॒र्तिः क॑ल्या॒णी ताꣳ स्तान् क॑ल्या॒णी की॒र्तिः । \newline
26. क॒ल्या॒णी की॒र्तिः की॒र्तिः क॑ल्या॒णी क॑ल्या॒णी की॒र्ति रन्वनु॑ की॒र्तिः क॑ल्या॒णी क॑ल्या॒णी की॒र्तिरनु॑ । \newline
27. की॒र्ति रन्वनु॑ की॒र्तिः की॒र्ति रनू दुदनु॑ की॒र्तिः की॒र्ति रनूत् । \newline
28. अनू दुदन् वनूत् ति॑ष्ठति तिष्ठ॒ त्युदन् वनूत् ति॑ष्ठति । \newline
29. उत् ति॑ष्ठति तिष्ठ॒ त्युदुत् ति॑ष्ठ॒ त्यरा᳚थ्सु॒ ररा᳚थ्सु स्तिष्ठ॒ त्युदुत् ति॑ष्ठ॒ त्यरा᳚थ्सुः । \newline
30. ति॒ष्ठ॒ त्यरा᳚थ्सु॒ ररा᳚थ्सु स्तिष्ठति तिष्ठ॒ त्यरा᳚थ्सु रि॒म इ॒मे ऽरा᳚थ्सु स्तिष्ठति तिष्ठ॒ त्यरा᳚थ्सु रि॒मे । \newline
31. अरा᳚थ्सु रि॒म इ॒मे ऽरा᳚थ्सु॒ ररा᳚थ्सु रि॒मे यज॑माना॒ यज॑माना इ॒मे ऽरा᳚थ्सु॒ ररा᳚थ्सु रि॒मे यज॑मानाः । \newline
32. इ॒मे यज॑माना॒ यज॑माना इ॒म इ॒मे यज॑माना॒ इतीति॒ यज॑माना इ॒म इ॒मे यज॑माना॒ इति॑ । \newline
33. यज॑माना॒ इतीति॒ यज॑माना॒ यज॑माना॒ इति॒ तत् तदिति॒ यज॑माना॒ यज॑माना॒ इति॒ तत् । \newline
34. इति॒ तत् तदितीति॒ तदन् वनु॒ तदितीति॒ तदनु॑ । \newline
35. तदन् वनु॒ तत् तदनु॒ सर्वे॒ सर्वे ऽनु॒ तत् तदनु॒ सर्वे᳚ । \newline
36. अनु॒ सर्वे॒ सर्वे ऽन्वनु॒ सर्वे॑ राद्ध्नुवन्ति राद्ध्नुवन्ति॒ सर्वे ऽन्वनु॒ सर्वे॑ राद्ध्नुवन्ति । \newline
37. सर्वे॑ राद्ध्नुवन्ति राद्ध्नुवन्ति॒ सर्वे॒ सर्वे॑ राद्ध्नुवन्ति । \newline
38. रा॒द्ध्नु॒व॒न्तीति॑ राद्ध्नुवन्ति । \newline
\pagebreak
\markright{ TS 7.4.9.1  \hfill https://www.vedavms.in \hfill}

\section{ TS 7.4.9.1 }

\textbf{TS 7.4.9.1 } \newline
\textbf{Samhita Paata} \newline

सु॒व॒र्गं ॅवा ए॒ते लो॒कं ॅय॑न्ति॒ ये स॒त्रमु॑प॒यन्त्य॒भीन्ध॑त ए॒व दी॒क्षाभि॑रा॒त्मानꣳ॑ श्रपयन्त उप॒सद्भि॒-र्द्वाभ्यां॒ ॅलोमाव॑ द्यन्ति॒ द्वाभ्यां॒ त्वचं॒ द्वाभ्या॒मसृ॒द् द्वाभ्यां᳚ माꣳ॒॒सं द्वाभ्या॒मस्थि॒ द्वाभ्यां᳚ म॒ज्जान॑मा॒त्मद॑क्षिणं॒ ॅवै स॒त्रमा॒त्मान॑मे॒व दक्षि॑णां नी॒त्वा सु॑व॒र्गं ॅलो॒कं ॅय॑न्ति॒ शिखा॒मनु॒ प्र व॑पन्त॒ ऋद्ध्या॒ अथो॒ रघी॑याꣳसः सुव॒र्गं ॅलो॒कम॑या॒मेति॑ ( ) ॥ \newline

\textbf{Pada Paata} \newline

सु॒व॒र्गमिति॑ सुवः - गम् । वै । ए॒ते । लो॒कम् । य॒न्ति॒ । ये । स॒त्रम् । उ॒प॒यन्तीत्यु॑प - यन्ति॑ । अ॒भीति॑ । इ॒न्ध॒ते॒ । ए॒व । दी॒क्षाभिः॑ । आ॒त्मान᳚म् । श्र॒प॒य॒न्ते॒ । उ॒प॒सद्भि॒रित्यु॑प॒सत् - भिः॒ । द्वाभ्या᳚म् । लोम॑ । अवेति॑ । द्य॒न्ति॒ । द्वाभ्या᳚म् । त्वच᳚म् । द्वाभ्या᳚म् । असृ॑त् । द्वाभ्या᳚म् । माꣳ॒॒सम् । द्वाभ्या᳚म् । अस्थि॑ । द्वाभ्या᳚म् । म॒ज्जान᳚म् । आ॒त्मद॑क्षिण॒मित्या॒त्म - द॒क्षि॒ण॒म् । वै । स॒त्रम् । आ॒त्मान᳚म् । ए॒व । दक्षि॑णाम् । नी॒त्वा । सु॒व॒र्गमिति॑ सुवः - गम् । लो॒कम् । य॒न्ति॒ । शिखा᳚म् । अनु॑ । प्रेति॑ । व॒प॒न्ते॒ । ऋद्ध्यै᳚ । अथो॒ इति॑ । रघी॑याꣳसः । सु॒व॒र्गमिति॑ सुवः - गम् । लो॒कम् । अ॒या॒म॒ । इति॑ ( ) ॥  \newline


\textbf{Krama Paata} \newline

सु॒व॒र्गम् ॅवै । सु॒व॒र्गमिति॑ सुवः - गम् । वा ए॒ते । ए॒ते लो॒कम् । लो॒कम् ॅय॑न्ति । य॒न्ति॒ ये । ये स॒त्रम् । स॒त्रमु॑प॒यन्ति॑ । उ॒प॒यन्त्य॒भि । उ॒प॒यन्तीत्यु॑प - यन्ति॑ । अ॒भीन्ध॑ते । इ॒न्ध॒त॒ ए॒व । ए॒व दी॒क्षाभिः॑ । दी॒क्षाभि॑रा॒त्मान᳚म् । आ॒त्मानꣳ॑ श्रपयन्ते । श्र॒प॒य॒न्त॒ उ॒प॒सद्‍भिः॑ । उ॒प॒सद्‍भि॒र् द्वाभ्या᳚म् । उ॒प॒सद्भि॒रित्यु॑प॒सत् - भिः॒ । द्वाभ्या॒म् ॅलोम॑ । लोमाव॑ । अव॑ द्यन्ति । द्य॒न्ति॒ द्वाभ्या᳚म् । द्वाभ्या॒म् त्वच᳚म् । त्वच॒म् द्वाभ्या᳚म् । द्वाभ्या॒मसृ॑त् । असृ॒द् द्वाभ्या᳚म् । द्वाभ्या᳚म् माꣳ॒॒सम् । माꣳ॒॒सम् द्वाभ्या᳚म् । द्वाभ्या॒मस्थि॑ । अस्थि॒ द्वाभ्या᳚म् । द्वाभ्या᳚म् म॒ज्जान᳚म् । म॒ज्जान॑मा॒त्मद॑क्षिणम् । आ॒त्मद॑क्षिण॒म् ॅवै । आ॒त्मद॑क्षिण॒मित्या॒त्म - द॒क्षि॒ण॒म् । वै स॒त्रम् । स॒त्रमा॒त्मान᳚म् । आ॒त्मान॑मे॒व । ए॒व दक्षि॑णाम् । दक्षि॑णाम् नी॒त्वा । नी॒त्वा सु॑व॒र्गम् । सु॒व॒र्गम् ॅलो॒कम् । सु॒व॒र्गमिति॑ सुवः - गम् । लो॒कम् ॅय॑न्ति । य॒न्ति॒ शिखा᳚म् । शिखा॒मनु॑ । अनु॒ प्र । प्र व॑पन्ते । व॒प॒न्त॒ ऋद्ध्यै᳚ । ऋद्ध्या॒ अथो᳚ । अथो॒ रघी॑याꣳसः । अथो॒ इत्यथो᳚ । रघी॑याꣳसः सुव॒र्गम् । सु॒व॒र्गम् ॅलो॒कम् । सु॒व॒र्गमिति॑ सुवः - गम् । लो॒कम॑याम । अ॒या॒मेति॑ । इतीतीति॑ । \newline

\textbf{Jatai Paata} \newline

1. सु॒व॒र्गं ॅवै वै सु॑व॒र्गꣳ सु॑व॒र्गं ॅवै । \newline
2. सु॒व॒र्गमिति॑ सुवः - गम् । \newline
3. वा ए॒त ए॒ते वै वा ए॒ते । \newline
4. ए॒ते लो॒कम् ॅलो॒क मे॒त ए॒ते लो॒कम् । \newline
5. लो॒कं ॅय॑न्ति यन्ति लो॒कम् ॅलो॒कं ॅय॑न्ति । \newline
6. य॒न्ति॒ ये ये य॑न्ति यन्ति॒ ये । \newline
7. ये स॒त्रꣳ स॒त्रं ॅये ये स॒त्रम् । \newline
8. स॒त्र मु॑प॒य न्त्यु॑प॒यन्ति॑ स॒त्रꣳ स॒त्र मु॑प॒यन्ति॑ । \newline
9. उ॒प॒य न्त्य॒भ्या᳚(1॒) भ्यु॑प॒य न्त्यु॑प॒य न्त्य॒भि । \newline
10. उ॒प॒यन्तीत्यु॑प - यन्ति॑ । \newline
11. अ॒भीन्ध॑त इन्धते॒ ऽभ्य॑भीन्ध॑ते । \newline
12. इ॒न्ध॒त॒ ए॒वैवेन्ध॑त इन्धत ए॒व । \newline
13. ए॒व दी॒क्षाभि॑र् दी॒क्षाभि॑ रे॒वैव दी॒क्षाभिः॑ । \newline
14. दी॒क्षाभि॑ रा॒त्मान॑ मा॒त्मान॑म् दी॒क्षाभि॑र् दी॒क्षाभि॑ रा॒त्मान᳚म् । \newline
15. आ॒त्मानꣳ॑ श्रपयन्ते श्रपयन्त आ॒त्मान॑ मा॒त्मानꣳ॑ श्रपयन्ते । \newline
16. श्र॒प॒य॒न्त॒ उ॒प॒सद्भि॑ रुप॒सद्भिः॑ श्रपयन्ते श्रपयन्त उप॒सद्भिः॑ । \newline
17. उ॒प॒सद्भि॒र् द्वाभ्या॒म् द्वाभ्या॑ मुप॒सद्भि॑ रुप॒सद्भि॒र् द्वाभ्या᳚म् । \newline
18. उ॒प॒सद्भि॒रित्यु॑प॒सत् - भिः॒ । \newline
19. द्वाभ्या॒म् ॅलोम॒ लोम॒ द्वाभ्या॒म् द्वाभ्या॒म् ॅलोम॑ । \newline
20. लोमा वाव॒ लोम॒ लोमाव॑ । \newline
21. अव॑ द्यन्ति द्य॒न्त्यवाव॑ द्यन्ति । \newline
22. द्य॒न्ति॒ द्वाभ्या॒म् द्वाभ्या᳚म् द्यन्ति द्यन्ति॒ द्वाभ्या᳚म् । \newline
23. द्वाभ्या॒म् त्वच॒म् त्वच॒म् द्वाभ्या॒म् द्वाभ्या॒म् त्वच᳚म् । \newline
24. त्वच॒म् द्वाभ्या॒म् द्वाभ्या॒म् त्वच॒म् त्वच॒म् द्वाभ्या᳚म् । \newline
25. द्वाभ्या॒ मसृ॒ दसृ॒द् द्वाभ्या॒म् द्वाभ्या॒ मसृ॑त् । \newline
26. असृ॒द् द्वाभ्या॒म् द्वाभ्या॒ मसृ॒ दसृ॒द् द्वाभ्या᳚म् । \newline
27. द्वाभ्या᳚म् माꣳ॒॒सम् माꣳ॒॒सम् द्वाभ्या॒म् द्वाभ्या᳚म् माꣳ॒॒सम् । \newline
28. माꣳ॒॒सम् द्वाभ्या॒म् द्वाभ्या᳚म् माꣳ॒॒सम् माꣳ॒॒सम् द्वाभ्या᳚म् । \newline
29. द्वाभ्या॒ मस्थ्य स्थि॒ द्वाभ्या॒म् द्वाभ्या॒ मस्थि॑ । \newline
30. अस्थि॒ द्वाभ्या॒म् द्वाभ्या॒ मस्थ्य स्थि॒ द्वाभ्या᳚म् । \newline
31. द्वाभ्या᳚म् म॒ज्जान॑म् म॒ज्जान॒म् द्वाभ्या॒म् द्वाभ्या᳚म् म॒ज्जान᳚म् । \newline
32. म॒ज्जान॑ मा॒त्मद॑क्षिण मा॒त्मद॑क्षिणम् म॒ज्जान॑म् म॒ज्जान॑ मा॒त्मद॑क्षिणम् । \newline
33. आ॒त्मद॑क्षिणं॒ ॅवै वा आ॒त्मद॑क्षिण मा॒त्मद॑क्षिणं॒ ॅवै । \newline
34. आ॒त्मद॑क्षिण॒मित्या॒त्म - द॒क्षि॒ण॒म् । \newline
35. वै स॒त्रꣳ स॒त्रं ॅवै वै स॒त्रम् । \newline
36. स॒त्र मा॒त्मान॑ मा॒त्मानꣳ॑ स॒त्रꣳ स॒त्र मा॒त्मान᳚म् । \newline
37. आ॒त्मान॑ मे॒वै वात्मान॑ मा॒त्मान॑ मे॒व । \newline
38. ए॒व दक्षि॑णा॒म् दक्षि॑णा मे॒वैव दक्षि॑णाम् । \newline
39. दक्षि॑णान् नी॒त्वा नी॒त्वा दक्षि॑णा॒म् दक्षि॑णान् नी॒त्वा । \newline
40. नी॒त्वा सु॑व॒र्गꣳ सु॑व॒र्गन् नी॒त्वा नी॒त्वा सु॑व॒र्गम् । \newline
41. सु॒व॒र्गम् ॅलो॒कम् ॅलो॒कꣳ सु॑व॒र्गꣳ सु॑व॒र्गम् ॅलो॒कम् । \newline
42. सु॒व॒र्गमिति॑ सुवः - गम् । \newline
43. लो॒कं ॅय॑न्ति यन्ति लो॒कम् ॅलो॒कं ॅय॑न्ति । \newline
44. य॒न्ति॒ शिखाꣳ॒॒ शिखां᳚ ॅयन्ति यन्ति॒ शिखा᳚म् । \newline
45. शिखा॒ मन्वनु॒ शिखाꣳ॒॒ शिखा॒ मनु॑ । \newline
46. अनु॒ प्र प्राण्वनु॒ प्र । \newline
47. प्र व॑पन्ते वपन्ते॒ प्र प्र व॑पन्ते । \newline
48. व॒प॒न्त॒ ऋद्ध्या॒ ऋद्ध्यै॑ वपन्ते वपन्त॒ ऋद्ध्यै᳚ । \newline
49. ऋद्ध्या॒ अथो॒ अथो॒ ऋद्ध्या॒ ऋद्ध्या॒ अथो᳚ । \newline
50. अथो॒ रघी॑याꣳसो॒ रघी॑याꣳ॒॒सो ऽथो॒ अथो॒ रघी॑याꣳसः । \newline
51. अथो॒ इत्यथो᳚ । \newline
52. रघी॑याꣳसः सुव॒र्गꣳ सु॑व॒र्गꣳ रघी॑याꣳसो॒ रघी॑याꣳसः सुव॒र्गम् । \newline
53. सु॒व॒र्गम् ॅलो॒कम् ॅलो॒कꣳ सु॑व॒र्गꣳ सु॑व॒र्गम् ॅलो॒कम् । \newline
54. सु॒व॒र्गमिति॑ सुवः - गम् । \newline
55. लो॒क म॑यामा याम लो॒कम् ॅलो॒क म॑याम । \newline
56. अ॒या॒ मेती त्य॑या माया॒ मेति॑ । \newline
57. इतीतीति॑ । \newline

\textbf{Ghana Paata } \newline

1. सु॒व॒र्गं ॅवै वै सु॑व॒र्गꣳ सु॑व॒र्गं ॅवा ए॒त ए॒ते वै सु॑व॒र्गꣳ सु॑व॒र्गं ॅवा ए॒ते । \newline
2. सु॒व॒र्गमिति॑ सुवः - गम् । \newline
3. वा ए॒त ए॒ते वै वा ए॒ते लो॒कम् ॅलो॒क मे॒ते वै वा ए॒ते लो॒कम् । \newline
4. ए॒ते लो॒कम् ॅलो॒क मे॒त ए॒ते लो॒कं ॅय॑न्ति यन्ति लो॒क मे॒त ए॒ते लो॒कं ॅय॑न्ति । \newline
5. लो॒कं ॅय॑न्ति यन्ति लो॒कम् ॅलो॒कं ॅय॑न्ति॒ ये ये य॑न्ति लो॒कम् ॅलो॒कं ॅय॑न्ति॒ ये । \newline
6. य॒न्ति॒ ये ये य॑न्ति यन्ति॒ ये स॒त्रꣳ स॒त्रं ॅये य॑न्ति यन्ति॒ ये स॒त्रम् । \newline
7. ये स॒त्रꣳ स॒त्रं ॅये ये स॒त्र मु॑प॒य न्त्यु॑प॒यन्ति॑ स॒त्रं ॅये ये स॒त्र मु॑प॒यन्ति॑ । \newline
8. स॒त्र मु॑प॒य न्त्यु॑प॒यन्ति॑ स॒त्रꣳ स॒त्र मु॑प॒य न्त्य॒भ्या᳚(1॒) भ्यु॑प॒यन्ति॑ स॒त्रꣳ स॒त्र मु॑प॒य न्त्य॒भि । \newline
9. उ॒प॒य न्त्य॒भ्या᳚(1॒) भ्यु॑प॒य न्त्यु॑प॒य न्त्य॒भीन्ध॑त इन्धते॒ ऽभ्यु॑प॒य न्त्यु॑प॒य न्त्य॒भीन्ध॑ते । \newline
10. उ॒प॒यन्तीत्यु॑प - यन्ति॑ । \newline
11. अ॒भीन्ध॑त इन्धते॒ ऽभ्य॑ भीन्ध॑त ए॒वैवेन्ध॑ते॒ ऽभ्य॑ भीन्ध॑त ए॒व । \newline
12. इ॒न्ध॒त॒ ए॒वैवेन्ध॑त इन्धत ए॒व दी॒क्षाभि॑र् दी॒क्षाभि॑ रे॒वेन्ध॑त इन्धत ए॒व दी॒क्षाभिः॑ । \newline
13. ए॒व दी॒क्षाभि॑र् दी॒क्षाभि॑ रे॒वैव दी॒क्षाभि॑ रा॒त्मान॑ मा॒त्मान॑म् दी॒क्षाभि॑ रे॒वैव दी॒क्षाभि॑ रा॒त्मान᳚म् । \newline
14. दी॒क्षाभि॑ रा॒त्मान॑ मा॒त्मान॑म् दी॒क्षाभि॑र् दी॒क्षाभि॑ रा॒त्मानꣳ॑ श्रपयन्ते श्रपयन्त आ॒त्मान॑म् दी॒क्षाभि॑र् दी॒क्षाभि॑ रा॒त्मानꣳ॑ श्रपयन्ते । \newline
15. आ॒त्मानꣳ॑ श्रपयन्ते श्रपयन्त आ॒त्मान॑ मा॒त्मानꣳ॑ श्रपयन्त उप॒सद्भि॑ रुप॒सद्भिः॑ श्रपयन्त आ॒त्मान॑ मा॒त्मानꣳ॑ श्रपयन्त उप॒सद्भिः॑ । \newline
16. श्र॒प॒य॒न्त॒ उ॒प॒सद्भि॑ रुप॒सद्भिः॑ श्रपयन्ते श्रपयन्त उप॒सद्भि॒र् द्वाभ्या॒म् द्वाभ्या॑ मुप॒सद्भिः॑ श्रपयन्ते श्रपयन्त उप॒सद्भि॒र् द्वाभ्या᳚म् । \newline
17. उ॒प॒सद्भि॒र् द्वाभ्या॒म् द्वाभ्या॑ मुप॒सद्भि॑ रुप॒सद्भि॒र् द्वाभ्या॒म् ॅलोम॒ लोम॒ द्वाभ्या॑ मुप॒सद्भि॑ रुप॒सद्भि॒र् द्वाभ्या॒म् ॅलोम॑ । \newline
18. उ॒प॒सद्भि॒रित्यु॑प॒सत् - भिः॒ । \newline
19. द्वाभ्या॒म् ॅलोम॒ लोम॒ द्वाभ्या॒म् द्वाभ्या॒म् ॅलोमावाव॒ लोम॒ द्वाभ्या॒म् द्वाभ्या॒म् ॅलोमाव॑ । \newline
20. लोमावाव॒ लोम॒ लोमाव॑ द्यन्ति द्य॒न्त्यव॒ लोम॒ लोमाव॑ द्यन्ति । \newline
21. अव॑ द्यन्ति द्य॒न्त्यवाव॑ द्यन्ति॒ द्वाभ्या॒म् द्वाभ्या᳚म् द्य॒न्त्यवाव॑ द्यन्ति॒ द्वाभ्या᳚म् । \newline
22. द्य॒न्ति॒ द्वाभ्या॒म् द्वाभ्या᳚म् द्यन्ति द्यन्ति॒ द्वाभ्या॒म् त्वच॒म् त्वच॒म् द्वाभ्या᳚म् द्यन्ति द्यन्ति॒ द्वाभ्या॒म् त्वच᳚म् । \newline
23. द्वाभ्या॒म् त्वच॒म् त्वच॒म् द्वाभ्या॒म् द्वाभ्या॒म् त्वच॒म् द्वाभ्या॒म् द्वाभ्या॒म् त्वच॒म् द्वाभ्या॒म् द्वाभ्या॒म् त्वच॒म् द्वाभ्या᳚म् । \newline
24. त्वच॒म् द्वाभ्या॒म् द्वाभ्या॒म् त्वच॒म् त्वच॒म् द्वाभ्या॒ मसृ॒ दसृ॒द् द्वाभ्या॒म् त्वच॒म् त्वच॒म् द्वाभ्या॒ मसृ॑त् । \newline
25. द्वाभ्या॒ मसृ॒ दसृ॒द् द्वाभ्या॒म् द्वाभ्या॒ मसृ॒द् द्वाभ्या॒म् द्वाभ्या॒ मसृ॒द् द्वाभ्या॒म् द्वाभ्या॒ मसृ॒द् द्वाभ्या᳚म् । \newline
26. असृ॒द् द्वाभ्या॒म् द्वाभ्या॒ मसृ॒ दसृ॒द् द्वाभ्या᳚म् माꣳ॒॒सम् माꣳ॒॒सम् द्वाभ्या॒ मसृ॒ दसृ॒द् द्वाभ्या᳚म् माꣳ॒॒सम् । \newline
27. द्वाभ्या᳚म् माꣳ॒॒सम् माꣳ॒॒सम् द्वाभ्या॒म् द्वाभ्या᳚म् माꣳ॒॒सम् द्वाभ्या॒म् द्वाभ्या᳚म् माꣳ॒॒सम् द्वाभ्या॒म् द्वाभ्या᳚म् माꣳ॒॒सम् द्वाभ्या᳚म् । \newline
28. माꣳ॒॒सम् द्वाभ्या॒म् द्वाभ्या᳚म् माꣳ॒॒सम् माꣳ॒॒सम् द्वाभ्या॒ मस्थ्यस्थि॒ द्वाभ्या᳚म् माꣳ॒॒सम् माꣳ॒॒सम् द्वाभ्या॒ मस्थि॑ । \newline
29. द्वाभ्या॒ मस्थ्यस्थि॒ द्वाभ्या॒म् द्वाभ्या॒ मस्थि॒ द्वाभ्या॒म् द्वाभ्या॒ मस्थि॒ द्वाभ्या॒म् द्वाभ्या॒ मस्थि॒ द्वाभ्या᳚म् । \newline
30. अस्थि॒ द्वाभ्या॒म् द्वाभ्या॒ मस्थ्यस्थि॒ द्वाभ्या᳚म् म॒ज्जान॑म् म॒ज्जान॒म् द्वाभ्या॒ मस्थ्यस्थि॒ द्वाभ्या᳚म् म॒ज्जान᳚म् । \newline
31. द्वाभ्या᳚म् म॒ज्जान॑म् म॒ज्जान॒म् द्वाभ्या॒म् द्वाभ्या᳚म् म॒ज्जान॑ मा॒त्मद॑क्षिण मा॒त्मद॑क्षिणम् म॒ज्जान॒म् द्वाभ्या॒म् द्वाभ्या᳚म् म॒ज्जान॑ मा॒त्मद॑क्षिणम् । \newline
32. म॒ज्जान॑ मा॒त्मद॑क्षिण मा॒त्मद॑क्षिणम् म॒ज्जान॑म् म॒ज्जान॑ मा॒त्मद॑क्षिणं॒ ॅवै वा आ॒त्मद॑क्षिणम् म॒ज्जान॑म् म॒ज्जान॑ मा॒त्मद॑क्षिणं॒ ॅवै । \newline
33. आ॒त्मद॑क्षिणं॒ ॅवै वा आ॒त्मद॑क्षिण मा॒त्मद॑क्षिणं॒ ॅवै स॒त्रꣳ स॒त्रं ॅवा आ॒त्मद॑क्षिण मा॒त्मद॑क्षिणं॒ ॅवै स॒त्रम् । \newline
34. आ॒त्मद॑क्षिण॒मित्या॒त्म - द॒क्षि॒ण॒म् । \newline
35. वै स॒त्रꣳ स॒त्रं ॅवै वै स॒त्र मा॒त्मान॑ मा॒त्मानꣳ॑ स॒त्रं ॅवै वै स॒त्र मा॒त्मान᳚म् । \newline
36. स॒त्र मा॒त्मान॑ मा॒त्मानꣳ॑ स॒त्रꣳ स॒त्र मा॒त्मान॑ मे॒वै वात्मानꣳ॑ स॒त्रꣳ स॒त्र मा॒त्मान॑ मे॒व । \newline
37. आ॒त्मान॑ मे॒वै वात्मान॑ मा॒त्मान॑ मे॒व दक्षि॑णा॒म् दक्षि॑णा मे॒वात्मान॑ मा॒त्मान॑ मे॒व दक्षि॑णाम् । \newline
38. ए॒व दक्षि॑णा॒म् दक्षि॑णा मे॒वैव दक्षि॑णान् नी॒त्वा नी॒त्वा दक्षि॑णा मे॒वैव दक्षि॑णान् नी॒त्वा । \newline
39. दक्षि॑णान् नी॒त्वा नी॒त्वा दक्षि॑णा॒म् दक्षि॑णान् नी॒त्वा सु॑व॒र्गꣳ सु॑व॒र्गन् नी॒त्वा दक्षि॑णा॒म् दक्षि॑णान् नी॒त्वा सु॑व॒र्गम् । \newline
40. नी॒त्वा सु॑व॒र्गꣳ सु॑व॒र्गन् नी॒त्वा नी॒त्वा सु॑व॒र्गम् ॅलो॒कम् ॅलो॒कꣳ सु॑व॒र्गन् नी॒त्वा नी॒त्वा सु॑व॒र्गम् ॅलो॒कम् । \newline
41. सु॒व॒र्गम् ॅलो॒कम् ॅलो॒कꣳ सु॑व॒र्गꣳ सु॑व॒र्गम् ॅलो॒कं ॅय॑न्ति यन्ति लो॒कꣳ सु॑व॒र्गꣳ सु॑व॒र्गम् ॅलो॒कं ॅय॑न्ति । \newline
42. सु॒व॒र्गमिति॑ सुवः - गम् । \newline
43. लो॒कं ॅय॑न्ति यन्ति लो॒कम् ॅलो॒कं ॅय॑न्ति॒ शिखाꣳ॒॒ शिखां᳚ ॅयन्ति लो॒कम् ॅलो॒कं ॅय॑न्ति॒ शिखा᳚म् । \newline
44. य॒न्ति॒ शिखाꣳ॒॒ शिखां᳚ ॅयन्ति यन्ति॒ शिखा॒ मन्वनु॒ शिखां᳚ ॅयन्ति यन्ति॒ शिखा॒ मनु॑ । \newline
45. शिखा॒ मन्वनु॒ शिखाꣳ॒॒ शिखा॒ मनु॒ प्र प्राणु॒ शिखाꣳ॒॒ शिखा॒ मनु॒ प्र । \newline
46. अनु॒ प्र प्राण्वनु॒ प्र व॑पन्ते वपन्ते॒ प्राण्वनु॒ प्र व॑पन्ते । \newline
47. प्र व॑पन्ते वपन्ते॒ प्र प्र व॑पन्त॒ ऋद्ध्या॒ ऋद्ध्यै॑ वपन्ते॒ प्र प्र व॑पन्त॒ ऋद्ध्यै᳚ । \newline
48. व॒प॒न्त॒ ऋद्ध्या॒ ऋद्ध्यै॑ वपन्ते वपन्त॒ ऋद्ध्या॒ अथो॒ अथो॒ ऋद्ध्यै॑ वपन्ते वपन्त॒ ऋद्ध्या॒ अथो᳚ । \newline
49. ऋद्ध्या॒ अथो॒ अथो॒ ऋद्ध्या॒ ऋद्ध्या॒ अथो॒ रघी॑याꣳसो॒ रघी॑याꣳ॒॒सो ऽथो॒ ऋद्ध्या॒ ऋद्ध्या॒ अथो॒ रघी॑याꣳसः । \newline
50. अथो॒ रघी॑याꣳसो॒ रघी॑याꣳ॒॒सो ऽथो॒ अथो॒ रघी॑याꣳसः सुव॒र्गꣳ सु॑व॒र्गꣳ रघी॑याꣳ॒॒सो ऽथो॒ अथो॒ रघी॑याꣳसः सुव॒र्गम् । \newline
51. अथो॒ इत्यथो᳚ । \newline
52. रघी॑याꣳसः सुव॒र्गꣳ सु॑व॒र्गꣳ रघी॑याꣳसो॒ रघी॑याꣳसः सुव॒र्गम् ॅलो॒कम् ॅलो॒कꣳ सु॑व॒र्गꣳ रघी॑याꣳसो॒ रघी॑याꣳसः सुव॒र्गम् ॅलो॒कम् । \newline
53. सु॒व॒र्गम् ॅलो॒कम् ॅलो॒कꣳ सु॑व॒र्गꣳ सु॑व॒र्गम् ॅलो॒क म॑यामा याम लो॒कꣳ सु॑व॒र्गꣳ सु॑व॒र्गम् ॅलो॒क म॑याम । \newline
54. सु॒व॒र्गमिति॑ सुवः - गम् । \newline
55. लो॒क म॑यामा याम लो॒कम् ॅलो॒क म॑या॒मे तीत्य॑याम लो॒कम् ॅलो॒क म॑या॒मेति॑ । \newline
56. अ॒या॒मे तीत्य॑ यामा या॒मेति॑ । \newline
57. इतीतीति॑ । \newline
\pagebreak
\markright{ TS 7.4.10.1  \hfill https://www.vedavms.in \hfill}

\section{ TS 7.4.10.1 }

\textbf{TS 7.4.10.1 } \newline
\textbf{Samhita Paata} \newline

ब्र॒ह्म॒वा॒दिनो॑ वदन्त्यतिरा॒त्रः प॑र॒मो य॑ज्ञ्क्रतू॒नां कस्मा॒त् तं प्र॑थ॒ममुप॑ य॒न्तीत्ये॒तद्वा अ॑ग्निष्टो॒मं प्र॑थ॒ममुप॑ य॒न्त्यथो॒क्थ्य॑मथ॑ षोड॒शिन॒-मथा॑तिरा॒त्र-म॑नुपू॒र्वमे॒वैतद्-य॑ज्ञ्क्र॒तूनु॒पेत्य॒ ताना॒लभ्य॑ परि॒गृह्य॒ सोम॑मे॒वैतत् पिब॑न्त आसते॒ ज्योति॑ष्टोमं प्रथ॒ममुप॑ यन्ति॒ ज्योति॑ष्टोमो॒ वै स्तोमा॑नां॒ मुखं॑ मुख॒त ए॒व स्तोमा॒न् प्र यु॑ञ्जते॒ ते - [  ] \newline

\textbf{Pada Paata} \newline

ब्र॒ह्म॒वा॒दिन॒ इति॑ ब्रह्म - वा॒दिनः॑ । व॒द॒न्ति॒ । अ॒ति॒रा॒त्र इत्य॑ति-रा॒त्रः । प॒र॒मः । य॒ज्ञ्॒क्र॒तू॒नामिति॑ यज्ञ्-क्र॒तू॒नाम् । कस्मा᳚त् । तम् । प्र॒थ॒मम् । उपेति॑ । य॒न्ति॒ । इति॑ । ए॒तत् । वै । अ॒ग्नि॒ष्टो॒ममित्य॑ग्नि - स्तो॒मम् । प्र॒थ॒मम् । उपेति॑ । य॒न्ति॒ । अथ॑ । उ॒क्थ्य᳚म् । अथ॑ । षो॒ड॒शिन᳚म् । अथ॑ । अ॒ति॒रा॒त्रमित्य॑ति - रा॒त्रम् । अ॒नु॒पू॒र्वमित्य॑नु - पू॒र्वम् । ए॒व । ए॒तत् । य॒ज्ञ्॒क्र॒तूनिति॑ यज्ञ् - क्र॒तून् । उ॒पेत्येत्यु॑प - इत्य॑ । तान् । आ॒लभ्येत्या᳚ - लभ्य॑ । प॒रि॒गृह्येति॑ परि - गृह्य॑ । सोम᳚म् । ए॒व । ए॒तत् । पिब॑न्तः । आ॒स॒ते॒ । ज्योति॑ष्टोम॒मिति॒ ज्योतिः॑ - स्तो॒म॒म् । प्र॒थ॒मम् । उपेति॑ । य॒न्ति॒ । ज्योति॑ष्टोम॒ इति॒ ज्योतिः॑ - स्तो॒मः॒ । वै । स्तोमा॑नाम् । मुख᳚म् । मु॒ख॒तः । ए॒व । स्तोमान्॑ । प्रेति॑ । यु॒ञ्ज॒ते॒ । ते ।  \newline


\textbf{Krama Paata} \newline

ब्र॒ह्म॒वा॒दिनो॑ वदन्ति । ब्र॒ह्म॒वा॒दिन॒ इति॑ ब्रह्म - वा॒दिनः॑ । व॒द॒न्त्य॒ति॒रा॒त्रः । अ॒ति॒रा॒त्रः प॑र॒मः । अ॒ति॒रा॒त्र इत्य॑ति - रा॒त्रः । प॒र॒मो य॑ज्ञ्क्रतू॒नाम् । य॒ज्ञ्॒क्र॒तू॒नाम् कस्मा᳚त् । य॒ज्ञ्॒क्र॒तू॒नामिति॑ यज्ञ् - क्र॒तू॒नाम् । कस्मा॒त् तम् । तम् प्र॑थ॒मम् । प्र॒थ॒ममुप॑ । उप॑ यन्ति । य॒न्तीति॑ । इत्ये॒तत् । ए॒तद् वै । वा अ॑ग्निष्टो॒मम् । अ॒ग्नि॒ष्टो॒मम् प्र॑थ॒मम् । अ॒ग्नि॒ष्टो॒ममित्य॑ग्नि - स्तो॒मम् । प्र॒थ॒ममुप॑ । उप॑ यन्ति । य॒न्त्यथ॑ । अथो॒क्थ्य᳚म् । उ॒क्थ्य॑मथ॑ । अथ॑ षोड॒शिन᳚म् । षो॒ड॒शिन॒मथ॑ । अथा॑तिरा॒त्रम् । अ॒ति॒रा॒त्रम॑नुपू॒र्वम् । अ॒ति॒रा॒त्रमित्य॑ति - रा॒त्रम् । अ॒नु॒पू॒र्वमे॒व । अ॒नु॒पू॒र्वमित्य॑नु - पू॒र्वम् । ए॒वैतत् । ए॒तद् य॑ज्ञ्क्र॒तून् । य॒ज्ञ्॒क्र॒तूनु॒पेत्य॑ । य॒ज्ञ्॒क्र॒तूनिति॑ यज्ञ् - क्र॒तून् । उ॒पेत्य॒ तान् । उ॒पेत्येत्यु॑प - इत्य॑ । ताना॒लभ्य॑ । आ॒लभ्य॑ परि॒गृह्य॑ । आ॒लभ्येत्या᳚ - लभ्य॑ । प॒रि॒गृह्य॒ सोम᳚म् । प॒रि॒गृह्येति॑ परि - गृह्य॑ । सोम॑मे॒व । ए॒वैतत् । ए॒तत् पिब॑न्तः । पिब॑न्त आसते । आ॒स॒ते॒ ज्योति॑ष्टोमम् । ज्योति॑ष्टोमम् प्रथ॒मम् । ज्योति॑ष्टोम॒मिति॒ ज्योतिः॑ - स्तो॒म॒म् । प्र॒थ॒ममुप॑ । उप॑ यन्ति । य॒न्ति॒ ज्योति॑ष्टोमः । ज्योति॑ष्टोमो॒ वै । ज्योति॑ष्टोम॒ इति॒ ज्योतिः॑ - स्तो॒मः॒ । वै स्तोमा॑नाम् । स्तोमा॑ना॒म् मुख᳚म् । मुख॑म् मुख॒तः । मु॒ख॒त ए॒व । ए॒व स्तोमान्॑ । स्तोमा॒न् प्र । प्र यु॑ञ्जते । यु॒ञ्ज॒ते॒ ते ( ) । ते सꣳस्तु॑ताः \newline

\textbf{Jatai Paata} \newline

1. ब्र॒ह्म॒वा॒दिनो॑ वदन्ति वदन्ति ब्रह्मवा॒दिनो᳚ ब्रह्मवा॒दिनो॑ वदन्ति । \newline
2. ब्र॒ह्म॒वा॒दिन॒ इति॑ ब्रह्म - वा॒दिनः॑ । \newline
3. व॒द॒ न्त्य॒ति॒रा॒त्रो॑ ऽतिरा॒त्रो व॑दन्ति वद न्त्यतिरा॒त्रः । \newline
4. अ॒ति॒रा॒त्रः प॑र॒मः प॑र॒मो॑ ऽतिरा॒त्रो॑ ऽतिरा॒त्रः प॑र॒मः । \newline
5. अ॒ति॒रा॒त्र इत्य॑ति - रा॒त्रः । \newline
6. प॒र॒मो य॑ज्ञ्क्रतू॒नां ॅय॑ज्ञ्क्रतू॒नाम् प॑र॒मः प॑र॒मो य॑ज्ञ्क्रतू॒नाम् । \newline
7. य॒ज्ञ्॒क्र॒तू॒नाम् कस्मा॒त् कस्मा᳚द् यज्ञ्क्रतू॒नां ॅय॑ज्ञ्क्रतू॒नाम् कस्मा᳚त् । \newline
8. य॒ज्ञ्॒क्र॒तू॒नामिति॑ यज्ञ् - क्र॒तू॒नाम् । \newline
9. कस्मा॒त् तम् तम् कस्मा॒त् कस्मा॒त् तम् । \newline
10. तम् प्र॑थ॒मम् प्र॑थ॒मम् तम् तम् प्र॑थ॒मम् । \newline
11. प्र॒थ॒म मुपोप॑ प्रथ॒मम् प्र॑थ॒म मुप॑ । \newline
12. उप॑ यन्ति य॒न्त्युपोप॑ यन्ति । \newline
13. य॒न्ती तीति॑ यन्ति य॒न्तीति॑ । \newline
14. इत्ये॒त दे॒त दिती त्ये॒तत् । \newline
15. ए॒तद् वै वा ए॒त दे॒तद् वै । \newline
16. वा अ॑ग्निष्टो॒म म॑ग्निष्टो॒मं ॅवै वा अ॑ग्निष्टो॒मम् । \newline
17. अ॒ग्नि॒ष्टो॒मम् प्र॑थ॒मम् प्र॑थ॒म म॑ग्निष्टो॒म म॑ग्निष्टो॒मम् प्र॑थ॒मम् । \newline
18. अ॒ग्नि॒ष्टो॒ममित्य॑ग्नि - स्तो॒मम् । \newline
19. प्र॒थ॒म मुपोप॑ प्रथ॒मम् प्र॑थ॒म मुप॑ । \newline
20. उप॑ यन्ति य॒न्त्युपोप॑ यन्ति । \newline
21. य॒न्त्यथाथ॑ यन्ति य॒न्त्यथ॑ । \newline
22. अथो॒क्थ्य॑ मु॒क्थ्य॑ मथाथो॒क्थ्य᳚म् । \newline
23. उ॒क्थ्य॑ मथा थो॒क्थ्य॑ मु॒क्थ्य॑ मथ॑ । \newline
24. अथ॑ षोड॒शिनꣳ॑ षोड॒शिन॒ मथाथ॑ षोड॒शिन᳚म् । \newline
25. षो॒ड॒शिन॒ मथाथ॑ षोड॒शिनꣳ॑ षोड॒शिन॒ मथ॑ । \newline
26. अथा॑ तिरा॒त्र म॑तिरा॒त्र मथाथा॑ तिरा॒त्रम् । \newline
27. अ॒ति॒रा॒त्र म॑नुपू॒र्व म॑नुपू॒र्व म॑तिरा॒त्र म॑तिरा॒त्र म॑नुपू॒र्वम् । \newline
28. अ॒ति॒रा॒त्रमित्य॑ति - रा॒त्रम् । \newline
29. अ॒नु॒पू॒र्व मे॒वै वानु॑पू॒र्व म॑नुपू॒र्व मे॒व । \newline
30. अ॒नु॒पू॒र्वमित्य॑नु - पू॒र्वम् । \newline
31. ए॒वैत दे॒त दे॒वै वैतत् । \newline
32. ए॒तद् य॑ज्ञ्क्र॒तून्. य॑ज्ञ्क्र॒तू ने॒त दे॒तद् य॑ज्ञ्क्र॒तून् । \newline
33. य॒ज्ञ्॒क्र॒तू नु॒पे त्यो॒पेत्य॑ यज्ञ्क्र॒तून्. य॑ज्ञ्क्र॒तू नु॒पेत्य॑ । \newline
34. य॒ज्ञ्॒क्र॒तूनिति॑ यज्ञ् - क्र॒तून् । \newline
35. उ॒पेत्य॒ ताꣳ स्तानु॒पे त्यो॒पेत्य॒ तान् । \newline
36. उ॒पेत्येत्यु॑प - इत्य॑ । \newline
37. ताना॒लभ्या॒ लभ्य॒ ताꣳ स्ताना॒ लभ्य॑ । \newline
38. आ॒लभ्य॑ परि॒गृह्य॑ परि॒गृह्या॒ लभ्या॒ लभ्य॑ परि॒गृह्य॑ । \newline
39. आ॒लभ्येत्या᳚ - लभ्य॑ । \newline
40. प॒रि॒गृह्य॒ सोमꣳ॒॒ सोम॑म् परि॒गृह्य॑ परि॒गृह्य॒ सोम᳚म् । \newline
41. प॒रि॒गृह्येति॑ परि - गृह्य॑ । \newline
42. सोम॑ मे॒वैव सोमꣳ॒॒ सोम॑ मे॒व । \newline
43. ए॒वैत दे॒त दे॒वैवैतत् । \newline
44. ए॒तत् पिब॑न्तः॒ पिब॑न्त ए॒त दे॒तत् पिब॑न्तः । \newline
45. पिब॑न्त आसत आसते॒ पिब॑न्तः॒ पिब॑न्त आसते । \newline
46. आ॒स॒ते॒ ज्योति॑ष्टोम॒म् ज्योति॑ष्टोम मासत आसते॒ ज्योति॑ष्टोमम् । \newline
47. ज्योति॑ष्टोमम् प्रथ॒मम् प्र॑थ॒मम् ज्योति॑ष्टोम॒म् ज्योति॑ष्टोमम् प्रथ॒मम् । \newline
48. ज्योति॑ष्टोम॒मिति॒ ज्योतिः॑ - स्तो॒म॒म् । \newline
49. प्र॒थ॒म मुपोप॑ प्रथ॒मम् प्र॑थ॒म मुप॑ । \newline
50. उप॑ यन्ति य॒न्त्युपोप॑ यन्ति । \newline
51. य॒न्ति॒ ज्योति॑ष्टोमो॒ ज्योति॑ष्टोमो यन्ति यन्ति॒ ज्योति॑ष्टोमः । \newline
52. ज्योति॑ष्टोमो॒ वै वै ज्योति॑ष्टोमो॒ ज्योति॑ष्टोमो॒ वै । \newline
53. ज्योति॑ष्टोम॒ इति॒ ज्योतिः॑ - स्तो॒मः॒ । \newline
54. वै स्तोमा॑नाꣳ॒॒ स्तोमा॑नां॒ ॅवै वै स्तोमा॑नाम् । \newline
55. स्तोमा॑ना॒म् मुख॒म् मुखꣳ॒॒ स्तोमा॑नाꣳ॒॒ स्तोमा॑ना॒म् मुख᳚म् । \newline
56. मुख॑म् मुख॒तो मु॑ख॒तो मुख॒म् मुख॑म् मुख॒तः । \newline
57. मु॒ख॒त ए॒वैव मु॑ख॒तो मु॑ख॒त ए॒व । \newline
58. ए॒व स्तोमा॒न् थ्स्तोमा॑ ने॒वैव स्तोमान्॑ । \newline
59. स्तोमा॒न् प्र प्र स्तोमा॒न् थ्स्तोमा॒न् प्र । \newline
60. प्र यु॑ञ्जते युञ्जते॒ प्र प्र यु॑ञ्जते । \newline
61. यु॒ञ्ज॒ते॒ ते ते यु॑ञ्जते युञ्जते॒ ते । \newline
62. ते सꣳस्तु॑ताः॒ सꣳस्तु॑ता॒ स्ते ते सꣳस्तु॑ताः । \newline

\textbf{Ghana Paata } \newline

1. ब्र॒ह्म॒वा॒दिनो॑ वदन्ति वदन्ति ब्रह्मवा॒दिनो᳚ ब्रह्मवा॒दिनो॑ वद न्त्यतिरा॒त्रो॑ ऽतिरा॒त्रो व॑दन्ति ब्रह्मवा॒दिनो᳚ ब्रह्मवा॒दिनो॑ वद न्त्यतिरा॒त्रः । \newline
2. ब्र॒ह्म॒वा॒दिन॒ इति॑ ब्रह्म - वा॒दिनः॑ । \newline
3. व॒द॒ न्त्य॒ति॒रा॒त्रो॑ ऽतिरा॒त्रो व॑दन्ति वद न्त्यतिरा॒त्रः प॑र॒मः प॑र॒मो॑ ऽतिरा॒त्रो व॑दन्ति वद न्त्यतिरा॒त्रः प॑र॒मः । \newline
4. अ॒ति॒रा॒त्रः प॑र॒मः प॑र॒मो॑ ऽतिरा॒त्रो॑ ऽतिरा॒त्रः प॑र॒मो य॑ज्ञ्क्रतू॒नां ॅय॑ज्ञ्क्रतू॒नाम् प॑र॒मो॑ ऽतिरा॒त्रो॑ ऽतिरा॒त्रः प॑र॒मो य॑ज्ञ्क्रतू॒नाम् । \newline
5. अ॒ति॒रा॒त्र इत्य॑ति - रा॒त्रः । \newline
6. प॒र॒मो य॑ज्ञ्क्रतू॒नां ॅय॑ज्ञ्क्रतू॒नाम् प॑र॒मः प॑र॒मो य॑ज्ञ्क्रतू॒नाम् कस्मा॒त् कस्मा᳚द् यज्ञ्क्रतू॒नाम् प॑र॒मः प॑र॒मो य॑ज्ञ्क्रतू॒नाम् कस्मा᳚त् । \newline
7. य॒ज्ञ्॒क्र॒तू॒नाम् कस्मा॒त् कस्मा᳚द् यज्ञ्क्रतू॒नां ॅय॑ज्ञ्क्रतू॒नाम् कस्मा॒त् तम् तम् कस्मा᳚द् यज्ञ्क्रतू॒नां ॅय॑ज्ञ्क्रतू॒नाम् कस्मा॒त् तम् । \newline
8. य॒ज्ञ्॒क्र॒तू॒नामिति॑ यज्ञ् - क्र॒तू॒नाम् । \newline
9. कस्मा॒त् तम् तम् कस्मा॒त् कस्मा॒त् तम् प्र॑थ॒मम् प्र॑थ॒मम् तम् कस्मा॒त् कस्मा॒त् तम् प्र॑थ॒मम् । \newline
10. तम् प्र॑थ॒मम् प्र॑थ॒मम् तम् तम् प्र॑थ॒म मुपोप॑ प्रथ॒मम् तम् तम् प्र॑थ॒म मुप॑ । \newline
11. प्र॒थ॒म मुपोप॑ प्रथ॒मम् प्र॑थ॒म मुप॑ यन्ति य॒न्त्युप॑ प्रथ॒मम् प्र॑थ॒म मुप॑ यन्ति । \newline
12. उप॑ यन्ति य॒न्त्युपोप॑ य॒न्तीतीति॑ य॒न्त्युपोप॑ य॒न्तीति॑ । \newline
13. य॒न्तीतीति॑ यन्ति य॒न्ती त्ये॒त दे॒त दिति॑ यन्ति य॒न्ती त्ये॒तत् । \newline
14. इत्ये॒त दे॒त दिती त्ये॒तद् वै वा ए॒त दिती त्ये॒तद् वै । \newline
15. ए॒तद् वै वा ए॒त दे॒तद् वा अ॑ग्निष्टो॒म म॑ग्निष्टो॒मं ॅवा ए॒त दे॒तद् वा अ॑ग्निष्टो॒मम् । \newline
16. वा अ॑ग्निष्टो॒म म॑ग्निष्टो॒मं ॅवै वा अ॑ग्निष्टो॒मम् प्र॑थ॒मम् प्र॑थ॒म म॑ग्निष्टो॒मं ॅवै वा अ॑ग्निष्टो॒मम् प्र॑थ॒मम् । \newline
17. अ॒ग्नि॒ष्टो॒मम् प्र॑थ॒मम् प्र॑थ॒म म॑ग्निष्टो॒म म॑ग्निष्टो॒मम् प्र॑थ॒म मुपोप॑ प्रथ॒म म॑ग्निष्टो॒म म॑ग्निष्टो॒मम् प्र॑थ॒म मुप॑ । \newline
18. अ॒ग्नि॒ष्टो॒ममित्य॑ग्नि - स्तो॒मम् । \newline
19. प्र॒थ॒म मुपोप॑ प्रथ॒मम् प्र॑थ॒म मुप॑ यन्ति य॒न्त्युप॑ प्रथ॒मम् प्र॑थ॒म मुप॑ यन्ति । \newline
20. उप॑ यन्ति य॒न्त्युपोप॑ य॒न्त्यथाथ॑ य॒न्त्युपोप॑ य॒न्त्यथ॑ । \newline
21. य॒न्त्यथाथ॑ यन्ति य॒न्त्यथो॒क्थ्य॑ मु॒क्थ्य॑ मथ॑ यन्ति य॒न्त्यथो॒क्थ्य᳚म् । \newline
22. अथो॒क्थ्य॑ मु॒क्थ्य॑ मथाथो॒क्थ्य॑ मथाथो॒क्थ्य॑ मथाथो॒क्थ्य॑ मथ॑ । \newline
23. उ॒क्थ्य॑ मथाथो॒क्थ्य॑ मु॒क्थ्य॑ मथ॑ षोड॒शिनꣳ॑ षोड॒शिन॒ मथो॒क्थ्य॑ मु॒क्थ्य॑ मथ॑ षोड॒शिन᳚म् । \newline
24. अथ॑ षोड॒शिनꣳ॑ षोड॒शिन॒ मथाथ॑ षोड॒शिन॒ मथाथ॑ षोड॒शिन॒ मथाथ॑ षोड॒शिन॒ मथ॑ । \newline
25. षो॒ड॒शिन॒ मथाथ॑ षोड॒शिनꣳ॑ षोड॒शिन॒ मथा॑तिरा॒त्र म॑तिरा॒त्र मथ॑ षोड॒शिनꣳ॑ षोड॒शिन॒ मथा॑तिरा॒त्रम् । \newline
26. अथा॑तिरा॒त्र म॑तिरा॒त्र मथाथा॑तिरा॒त्र म॑नुपू॒र्व म॑नुपू॒र्व म॑तिरा॒त्र मथाथा॑तिरा॒त्र म॑नुपू॒र्वम् । \newline
27. अ॒ति॒रा॒त्र म॑नुपू॒र्व म॑नुपू॒र्व म॑तिरा॒त्र म॑तिरा॒त्र म॑नुपू॒र्व मे॒वै वानु॑पू॒र्व म॑तिरा॒त्र म॑तिरा॒त्र म॑नुपू॒र्व मे॒व । \newline
28. अ॒ति॒रा॒त्रमित्य॑ति - रा॒त्रम् । \newline
29. अ॒नु॒पू॒र्व मे॒वै वानु॑पू॒र्व म॑नुपू॒र्व मे॒वैत दे॒त दे॒वा नु॑पू॒र्व म॑नुपू॒र्व मे॒वैतत् । \newline
30. अ॒नु॒पू॒र्वमित्य॑नु - पू॒र्वम् । \newline
31. ए॒वैत दे॒त दे॒वै वैतद् य॑ज्ञ्क्र॒तून्. य॑ज्ञ्क्र॒तू ने॒त दे॒वै वैतद् य॑ज्ञ्क्र॒तून् । \newline
32. ए॒तद् य॑ज्ञ्क्र॒तून्. य॑ज्ञ्क्र॒तू ने॒त दे॒तद् य॑ज्ञ्क्र॒तू नु॒पेत्यो॒पेत्य॑ यज्ञ्क्र॒तू ने॒त दे॒तद् य॑ज्ञ्क्र॒तू नु॒पेत्य॑ । \newline
33. य॒ज्ञ्॒क्र॒तू नु॒पे त्यो॒पेत्य॑ यज्ञ्क्र॒तून्. य॑ज्ञ्क्र॒तू नु॒पेत्य॒ ताꣳ स्ता नु॒पेत्य॑ यज्ञ्क्र॒तून्. य॑ज्ञ्क्र॒तू नु॒पेत्य॒ तान् । \newline
34. य॒ज्ञ्॒क्र॒तूनिति॑ यज्ञ् - क्र॒तून् । \newline
35. उ॒पेत्य॒ ताꣳ स्ता नु॒पे त्यो॒पेत्य॒ ताना॒लभ्या॒ लभ्य॒ तानु॒पे त्यो॒पेत्य॒ ताना॒लभ्य॑ । \newline
36. उ॒पेत्येत्यु॑प - इत्य॑ । \newline
37. ताना॒लभ्या॒ लभ्य॒ ताꣳ स्ताना॒लभ्य॑ परि॒गृह्य॑ परि॒गृह्या॒ लभ्य॒ ताꣳ स्ताना॒लभ्य॑ परि॒गृह्य॑ । \newline
38. आ॒लभ्य॑ परि॒गृह्य॑ परि॒गृह्या॒ लभ्या॒ लभ्य॑ परि॒गृह्य॒ सोमꣳ॒॒ सोम॑म् परि॒गृह्या॒ लभ्या॒ लभ्य॑ परि॒गृह्य॒ सोम᳚म् । \newline
39. आ॒लभ्येत्या᳚ - लभ्य॑ । \newline
40. प॒रि॒गृह्य॒ सोमꣳ॒॒ सोम॑म् परि॒गृह्य॑ परि॒गृह्य॒ सोम॑ मे॒वैव सोम॑म् परि॒गृह्य॑ परि॒गृह्य॒ सोम॑ मे॒व । \newline
41. प॒रि॒गृह्येति॑ परि - गृह्य॑ । \newline
42. सोम॑ मे॒वैव सोमꣳ॒॒ सोम॑ मे॒वै तदे॒ तदे॒व सोमꣳ॒॒ सोम॑ मे॒वैतत् । \newline
43. ए॒वै तदे॒ तदे॒वै वैतत् पिब॑न्तः॒ पिब॑न्त ए॒त दे॒वै वैतत् पिब॑न्तः । \newline
44. ए॒तत् पिब॑न्तः॒ पिब॑न्त ए॒त दे॒तत् पिब॑न्त आसत आसते॒ पिब॑न्त ए॒त दे॒तत् पिब॑न्त आसते । \newline
45. पिब॑न्त आसत आसते॒ पिब॑न्तः॒ पिब॑न्त आसते॒ ज्योति॑ष्टोम॒म् ज्योति॑ष्टोम मासते॒ पिब॑न्तः॒ पिब॑न्त आसते॒ ज्योति॑ष्टोमम् । \newline
46. आ॒स॒ते॒ ज्योति॑ष्टोम॒म् ज्योति॑ष्टोम मासत आसते॒ ज्योति॑ष्टोमम् प्रथ॒मम् प्र॑थ॒मम् ज्योति॑ष्टोम मासत आसते॒ ज्योति॑ष्टोमम् प्रथ॒मम् । \newline
47. ज्योति॑ष्टोमम् प्रथ॒मम् प्र॑थ॒मम् ज्योति॑ष्टोम॒म् ज्योति॑ष्टोमम् प्रथ॒म मुपोप॑ प्रथ॒मम् ज्योति॑ष्टोम॒म् ज्योति॑ष्टोमम् प्रथ॒म मुप॑ । \newline
48. ज्योति॑ष्टोम॒मिति॒ ज्योतिः॑ - स्तो॒म॒म् । \newline
49. प्र॒थ॒म मुपोप॑ प्रथ॒मम् प्र॑थ॒म मुप॑ यन्ति य॒न्त्युप॑ प्रथ॒मम् प्र॑थ॒म मुप॑ यन्ति । \newline
50. उप॑ यन्ति य॒न्त्युपोप॑ यन्ति॒ ज्योति॑ष्टोमो॒ ज्योति॑ष्टोमो य॒न्त्युपोप॑ यन्ति॒ ज्योति॑ष्टोमः । \newline
51. य॒न्ति॒ ज्योति॑ष्टोमो॒ ज्योति॑ष्टोमो यन्ति यन्ति॒ ज्योति॑ष्टोमो॒ वै वै ज्योति॑ष्टोमो यन्ति यन्ति॒ ज्योति॑ष्टोमो॒ वै । \newline
52. ज्योति॑ष्टोमो॒ वै वै ज्योति॑ष्टोमो॒ ज्योति॑ष्टोमो॒ वै स्तोमा॑नाꣳ॒॒ स्तोमा॑नां॒ ॅवै ज्योति॑ष्टोमो॒ ज्योति॑ष्टोमो॒ वै स्तोमा॑नाम् । \newline
53. ज्योति॑ष्टोम॒ इति॒ ज्योतिः॑ - स्तो॒मः॒ । \newline
54. वै स्तोमा॑नाꣳ॒॒ स्तोमा॑नां॒ ॅवै वै स्तोमा॑ना॒म् मुख॒म् मुखꣳ॒॒ स्तोमा॑नां॒ ॅवै वै स्तोमा॑ना॒म् मुख᳚म् । \newline
55. स्तोमा॑ना॒म् मुख॒म् मुखꣳ॒॒ स्तोमा॑नाꣳ॒॒ स्तोमा॑ना॒म् मुख॑म् मुख॒तो मु॑ख॒तो मुखꣳ॒॒ स्तोमा॑नाꣳ॒॒ स्तोमा॑ना॒म् मुख॑म् मुख॒तः । \newline
56. मुख॑म् मुख॒तो मु॑ख॒तो मुख॒म् मुख॑म् मुख॒त ए॒वैव मु॑ख॒तो मुख॒म् मुख॑म् मुख॒त ए॒व । \newline
57. मु॒ख॒त ए॒वैव मु॑ख॒तो मु॑ख॒त ए॒व स्तोमा॒न् थ्स्तोमा॑ ने॒व मु॑ख॒तो मु॑ख॒त ए॒व स्तोमान्॑ । \newline
58. ए॒व स्तोमा॒न् थ्स्तोमा॑ ने॒वैव स्तोमा॒न् प्र प्र स्तोमा॑ ने॒वैव स्तोमा॒न् प्र । \newline
59. स्तोमा॒न् प्र प्र स्तोमा॒न् थ्स्तोमा॒न् प्र यु॑ञ्जते युञ्जते॒ प्र स्तोमा॒न् थ्स्तोमा॒न् प्र यु॑ञ्जते । \newline
60. प्र यु॑ञ्जते युञ्जते॒ प्र प्र यु॑ञ्जते॒ ते ते यु॑ञ्जते॒ प्र प्र यु॑ञ्जते॒ ते । \newline
61. यु॒ञ्ज॒ते॒ ते ते यु॑ञ्जते युञ्जते॒ ते सꣳस्तु॑ताः॒ सꣳस्तु॑ता॒ स्ते यु॑ञ्जते युञ्जते॒ ते सꣳस्तु॑ताः । \newline
62. ते सꣳस्तु॑ताः॒ सꣳस्तु॑ता॒ स्ते ते सꣳस्तु॑ता वि॒राजं॑ ॅवि॒राजꣳ॒॒ सꣳस्तु॑ता॒ स्ते ते सꣳस्तु॑ता वि॒राज᳚म् । \newline
\pagebreak
\markright{ TS 7.4.10.2  \hfill https://www.vedavms.in \hfill}

\section{ TS 7.4.10.2 }

\textbf{TS 7.4.10.2 } \newline
\textbf{Samhita Paata} \newline

सꣳस्तु॑ता वि॒राज॑म॒भि सं प॑द्यन्ते॒ द्वे चर्चा॒वति॑ रिच्येते॒ एक॑या॒ गौरति॑रिक्त॒ एक॒याऽऽयु॑रू॒नः सु॑व॒र्गो वै लो॒को ज्योति॒रूर्ग्-वि॒राट्-थ्सु॑व॒र्गमे॒व तेन॑ लो॒कं ॅय॑न्ति रथन्त॒रं दिवा॒ भव॑ति रथन्त॒रं नक्त॒मित्या॑हु-र्ब्रह्मवा॒दिनः॒ केन॒ तदजा॒मीति॑ सौभ॒रं तृ॑तीयसव॒ने ब्र॑ह्मसा॒मं बृ॒हत् तन्म॑द्ध्य॒तो द॑धति॒ विधृ॑त्यै॒ तेनाजा॑मि ॥ \newline

\textbf{Pada Paata} \newline

सꣳस्तु॑ता॒ इति॒ सं - स्तु॒ताः॒ । वि॒राज॒मिति॑ वि - राज᳚म् । अ॒भि । समिति॑ । प॒द्य॒न्ते॒ । द्वे इति॑ । च॒ । ऋचौ᳚ । अतीति॑ । रि॒च्ये॒ते॒ इति॑ । एक॑या । गौः । अति॑रिक्त॒ इत्यति॑ - रि॒क्तः॒ । एक॑या । आयुः॑ । ऊ॒नः । सु॒व॒र्ग इति॑ सुवः - गः । वै । लो॒कः । ज्योतिः॑ । ऊर्क् । वि॒राडिति॑ वि - राट् । सु॒व॒र्गमिति॑ सुवः-गम् । ए॒व । तेन॑ । लो॒कम् । य॒न्ति॒ । र॒थ॒न्त॒रमिति॑ रथं - त॒रम् । दिवा᳚ । भव॑ति । र॒थ॒न्त॒रमिति॑ रथं-त॒रम् । नक्त᳚म् । इति॑ । आ॒हुः॒ । ब्र॒ह्म॒वा॒दिन॒ इति॑ ब्रह्म-वा॒दिनः॑ । केन॑ । तत् । अजा॑मि । इति॑ । सौ॒भ॒रम् । तृ॒ती॒य॒स॒व॒न इति॑ तृतीय - स॒व॒ने । ब्र॒ह्म॒सा॒ममिति॑ ब्रह्म - सा॒मम् । बृ॒हत् । तन् । म॒द्ध्य॒तः । द॒ध॒ति॒ । विधृ॑त्या॒ इति॒ वि - धृ॒त्यै॒ । तेन॑ । अजा॑मि ॥  \newline


\textbf{Krama Paata} \newline

सꣳस्तु॑ता वि॒राज᳚म् । सꣳस्तु॑ता॒ इति॒ सम् - स्तु॒ताः॒ । वि॒राज॑म॒भि । वि॒राज॒मिति॑ वि - राज᳚म् । अ॒भि सम् । सम् प॑द्यन्ते । प॒द्य॒न्ते॒ द्वे । द्वे च॑ । द्वे इति॒ द्वे । चर्चौ᳚ । ऋचा॒वति॑ । अति॑ रिच्येते । रि॒च्ये॒ते॒ एक॑या । रि॒च्ये॒ते॒ इति॑ रिच्येते । एक॑या॒ गौः । गौरति॑रिक्तः । अति॑रिक्त॒ एक॑या । अति॑रिक्त॒ इत्यति॑ - रि॒क्तः॒ । एक॒याऽऽयुः॑ । आयु॑रू॒नः । ऊ॒नः सु॑व॒र्गः । सु॒व॒र्गो वै । सु॒व॒र्ग इति॑ सुवः - गः । वै लो॒कः । लो॒को ज्योतिः॑ । ज्योति॒रूर्क् । ऊर्ग् वि॒राट् । वि॒राट्थ् सु॑व॒र्गम् । वि॒राडिति॑ वि - राट् । सु॒व॒र्गमे॒व । सु॒व॒र्गमिति॑ सुवः - गम् । ए॒व तेन॑ । तेन॑ लो॒कम् । लो॒कम् ॅय॑न्ति । य॒न्ति॒ र॒थ॒न्त॒रम् । र॒थ॒न्त॒रम् दिवा᳚ । र॒थ॒न्त॒रमिति॑ रथम् - त॒रम् । दिवा॒ भव॑ति । भव॑ति रथन्त॒रम् । र॒थ॒न्त॒रम् नक्त᳚म् । र॒थ॒न्त॒रमिति॑ रथम् - त॒रम् । नक्त॒मिति॑ । इत्या॑हुः । आ॒हु॒र् ब्र॒ह्म॒वा॒दिनः॑ । ब्र॒ह्म॒वा॒दिनः॒ केन॑ । ब्र॒ह्म॒वा॒दिन॒ इति॑ ब्रह्म - वा॒दिनः॑ । केन॒ तत् । तदजा॑मि । अजा॒मीति॑ । इति॑ सौभ॒रम् । सौ॒भ॒रम् तृ॑तीयसव॒ने । तृ॒ती॒य॒स॒व॒ने ब्र॑ह्मसा॒मम् । तृ॒ती॒य॒स॒व॒न इति॑ तृतीय - स॒व॒ने । ब्र॒ह्म॒सा॒मम् बृ॒हत् । ब्र॒ह्म॒सा॒ममिति॑ ब्रह्म - सा॒मम् । बृ॒हत् तत् । तन् म॑द्ध्य॒तः । म॒द्ध्य॒तो द॑धति । द॒ध॒ति॒ विधृ॑त्यै । विधृ॑त्यै॒ तेन॑ । विधृ॑त्या॒ इति॒ वि - धृ॒त्यै॒ । तेनाजा॑मि । अजा॒मीत्यजा॑मि । \newline

\textbf{Jatai Paata} \newline

1. सꣳस्तु॑ता वि॒राजं॑ ॅवि॒राजꣳ॒॒ सꣳस्तु॑ताः॒ सꣳस्तु॑ता वि॒राज᳚म् । \newline
2. सꣳस्तु॑ता॒ इति॒ सं - स्तु॒ताः॒ । \newline
3. वि॒राज॑ म॒भ्य॑भि वि॒राजं॑ ॅवि॒राज॑ म॒भि । \newline
4. वि॒राज॒मिति॑ वि - राज᳚म् । \newline
5. अ॒भि सꣳ स म॒भ्य॑भि सम् । \newline
6. सम् प॑द्यन्ते पद्यन्ते॒ सꣳ सम् प॑द्यन्ते । \newline
7. प॒द्य॒न्ते॒ द्वे द्वे प॑द्यन्ते पद्यन्ते॒ द्वे । \newline
8. द्वे च॑ च॒ द्वे द्वे च॑ । \newline
9. द्वे इति॒ द्वे । \newline
10. च र्‌चा॒ वृचौ॑ च॒ च र्‌चौ᳚ । \newline
11. ऋचा॒ वत्य त्यृचा॒ वृचा॒ वति॑ । \newline
12. अति॑ रिच्येते रिच्येते॒ अत्यति॑ रिच्येते । \newline
13. रि॒च्ये॒ते॒ एक॒ यैक॑या रिच्येते रिच्येते॒ एक॑या । \newline
14. रि॒च्ये॒ते॒ इति॑ रिच्येते । \newline
15. एक॑या॒ गौर् गौ रेक॒ यैक॑या॒ गौः । \newline
16. गौ रति॑रि॒क्तो ऽति॑रिक्तो॒ गौर् गौ रति॑रिक्तः । \newline
17. अति॑रिक्त॒ एक॒ यैक॒या ऽति॑रि॒क्तो ऽति॑रिक्त॒ एक॑या । \newline
18. अति॑रिक्त॒ इत्यति॑ - रि॒क्तः॒ । \newline
19. एक॒या ऽऽयु॒ रायु॒ रेक॒ यैक॒या ऽऽयुः॑ । \newline
20. आयु॑ रू॒न ऊ॒न आयु॒ रायु॑ रू॒नः । \newline
21. ऊ॒नः सु॑व॒र्गः सु॑व॒र्ग ऊ॒न ऊ॒नः सु॑व॒र्गः । \newline
22. सु॒व॒र्गो वै वै सु॑व॒र्गः सु॑व॒र्गो वै । \newline
23. सु॒व॒र्ग इति॑ सुवः - गः । \newline
24. वै लो॒को लो॒को वै वै लो॒कः । \newline
25. लो॒को ज्योति॒र् ज्योति॑र् लो॒को लो॒को ज्योतिः॑ । \newline
26. ज्योति॒ रूर् गूर्ग् ज्योति॒र् ज्योति॒ रूर्क् । \newline
27. ऊर्ग् वि॒राड् वि॒रा डूर् गूर्ग् वि॒राट् । \newline
28. वि॒राट् थ्सु॑व॒र्गꣳ सु॑व॒र्गं ॅवि॒राड् वि॒राट् थ्सु॑व॒र्गम् । \newline
29. वि॒राडिति॑ वि - राट् । \newline
30. सु॒व॒र्ग मे॒वैव सु॑व॒र्गꣳ सु॑व॒र्ग मे॒व । \newline
31. सु॒व॒र्गमिति॑ सुवः - गम् । \newline
32. ए॒व तेन॒ तेनै॒वैव तेन॑ । \newline
33. तेन॑ लो॒कम् ॅलो॒कम् तेन॒ तेन॑ लो॒कम् । \newline
34. लो॒कं ॅय॑न्ति यन्ति लो॒कम् ॅलो॒कं ॅय॑न्ति । \newline
35. य॒न्ति॒ र॒थ॒न्त॒रꣳ र॑थन्त॒रं ॅय॑न्ति यन्ति रथन्त॒रम् । \newline
36. र॒थ॒न्त॒रम् दिवा॒ दिवा॑ रथन्त॒रꣳ र॑थन्त॒रम् दिवा᳚ । \newline
37. र॒थ॒न्त॒रमिति॑ रथं - त॒रम् । \newline
38. दिवा॒ भव॑ति॒ भव॑ति॒ दिवा॒ दिवा॒ भव॑ति । \newline
39. भव॑ति रथन्त॒रꣳ र॑थन्त॒रम् भव॑ति॒ भव॑ति रथन्त॒रम् । \newline
40. र॒थ॒न्त॒रन् नक्त॒न् नक्तꣳ॑ रथन्त॒रꣳ र॑थन्त॒रन् नक्त᳚म् । \newline
41. र॒थ॒न्त॒रमिति॑ रथं - त॒रम् । \newline
42. नक्त॒ मितीति॒ नक्त॒न् नक्त॒ मिति॑ । \newline
43. इत्या॑हु राहु॒ रिती त्या॑हुः । \newline
44. आ॒हु॒र् ब्र॒ह्म॒वा॒दिनो᳚ ब्रह्मवा॒दिन॑ आहुराहुर् ब्रह्मवा॒दिनः॑ । \newline
45. ब्र॒ह्म॒वा॒दिनः॒ केन॒ केन॑ ब्रह्मवा॒दिनो᳚ ब्रह्मवा॒दिनः॒ केन॑ । \newline
46. ब्र॒ह्म॒वा॒दिन॒ इति॑ ब्रह्म - वा॒दिनः॑ । \newline
47. केन॒ तत् तत् केन॒ केन॒ तत् । \newline
48. तदजा॒ म्यजा॑मि॒ तत् तदजा॑मि । \newline
49. अजा॒ मीती त्यजा॒ म्यजा॒ मीति॑ । \newline
50. इति॑ सौभ॒रꣳ सौ॑भ॒र मितीति॑ सौभ॒रम् । \newline
51. सौ॒भ॒रम् तृ॑तीयसव॒ने तृ॑तीयसव॒ने सौ॑भ॒रꣳ सौ॑भ॒रम् तृ॑तीयसव॒ने । \newline
52. तृ॒ती॒य॒स॒व॒ने ब्र॑ह्मसा॒मम् ब्र॑ह्मसा॒मम् तृ॑तीयसव॒ने तृ॑तीयसव॒ने ब्र॑ह्मसा॒मम् । \newline
53. तृ॒ती॒य॒स॒व॒न इति॑ तृतीय - स॒व॒ने । \newline
54. ब्र॒ह्म॒सा॒मम् बृ॒हद् बृ॒हद् ब्र॑ह्मसा॒मम् ब्र॑ह्मसा॒मम् बृ॒हत् । \newline
55. ब्र॒ह्म॒सा॒ममिति॑ ब्रह्म - सा॒मम् । \newline
56. बृ॒हत् तत् तद् बृ॒हद् बृ॒हत् तत् । \newline
57. तन् म॑द्ध्य॒तो म॑द्ध्य॒त स्तत् तन् म॑द्ध्य॒तः । \newline
58. म॒द्ध्य॒तो द॑धति दधति मद्ध्य॒तो म॑द्ध्य॒तो द॑धति । \newline
59. द॒ध॒ति॒ विधृ॑त्यै॒ विधृ॑त्यै दधति दधति॒ विधृ॑त्यै । \newline
60. विधृ॑त्यै॒ तेन॒ तेन॒ विधृ॑त्यै॒ विधृ॑त्यै॒ तेन॑ । \newline
61. विधृ॑त्या॒ इति॒ वि - धृ॒त्यै॒ । \newline
62. तेना जा॒ म्यजा॑मि॒ तेन॒ तेना जा॑मि । \newline
63. अजा॒मीत्यजा॑मि । \newline

\textbf{Ghana Paata } \newline

1. सꣳस्तु॑ता वि॒राजं॑ ॅवि॒राजꣳ॒॒ सꣳस्तु॑ताः॒ सꣳस्तु॑ता वि॒राज॑ म॒भ्य॑भि वि॒राजꣳ॒॒ सꣳस्तु॑ताः॒ सꣳस्तु॑ता वि॒राज॑ म॒भि । \newline
2. सꣳस्तु॑ता॒ इति॒ सं - स्तु॒ताः॒ । \newline
3. वि॒राज॑ म॒भ्य॑भि वि॒राजं॑ ॅवि॒राज॑ म॒भि सꣳ स म॒भि वि॒राजं॑ ॅवि॒राज॑ म॒भि सम् । \newline
4. वि॒राज॒मिति॑ वि - राज᳚म् । \newline
5. अ॒भि सꣳ सम॒भ्य॑भि सम्प॑द्यन्ते पद्यन्ते॒ सम॒भ्य॑भि सम्प॑द्यन्ते । \newline
6. सम्प॑द्यन्ते पद्यन्ते॒ सꣳ सम्प॑द्यन्ते॒ द्वे द्वे प॑द्यन्ते॒ सꣳ सम्प॑द्यन्ते॒ द्वे । \newline
7. प॒द्य॒न्ते॒ द्वे द्वे प॑द्यन्ते पद्यन्ते॒ द्वे च॑ च॒ द्वे प॑द्यन्ते पद्यन्ते॒ द्वे च॑ । \newline
8. द्वे च॑ च॒ द्वे द्वे च र्‌चा॒ वृचौ॑ च॒ द्वे द्वे च र्‌चौ᳚ । \newline
9. द्वे इति॒ द्वे । \newline
10. च र्‌चा॒ वृचौ॑ च॒ च र्‌चा॒ वत्य त्यृचौ॑ च॒ च र्‌चा॒ वति॑ । \newline
11. ऋचा॒ वत्य त्यृचा॒ वृचा॒ वति॑ रिच्येते रिच्येते॒ अत्यृचा॒ वृचा॒ वति॑ रिच्येते । \newline
12. अति॑ रिच्येते रिच्येते॒ अत्यति॑ रिच्येते॒ एक॒यैक॑या रिच्येते॒ अत्यति॑ रिच्येते॒ एक॑या । \newline
13. रि॒च्ये॒ते॒ एक॒यैक॑या रिच्येते रिच्येते॒ एक॑या॒ गौर् गौ रेक॑या रिच्येते रिच्येते॒ एक॑या॒ गौः । \newline
14. रि॒च्ये॒ते॒ इति॑ रिच्येते । \newline
15. एक॑या॒ गौर् गौ रेक॒ यैक॑या॒ गौ रति॑रि॒क्तो ऽति॑रिक्तो॒ गौ रेक॒ यैक॑या॒ गौ रति॑रिक्तः । \newline
16. गौ रति॑रि॒क्तो ऽति॑रिक्तो॒ गौर् गौ रति॑रिक्त॒ एक॒ यैक॒या ऽति॑रिक्तो॒ गौर् गौ रति॑रिक्त॒ एक॑या । \newline
17. अति॑रिक्त॒ एक॒ यैक॒या ऽति॑रि॒क्तो ऽति॑रिक्त॒ एक॒या ऽऽयु॒ रायु॒ रेक॒या ऽति॑रि॒क्तो ऽति॑रिक्त॒ एक॒या ऽऽयुः॑ । \newline
18. अति॑रिक्त॒ इत्यति॑ - रि॒क्तः॒ । \newline
19. एक॒या ऽऽयु॒ रायु॒ रेक॒ यैक॒या ऽऽयु॑ रू॒न ऊ॒न आयु॒ रेक॒ यैक॒या ऽऽयु॑ रू॒नः । \newline
20. आयु॑ रू॒न ऊ॒न आयु॒ रायु॑ रू॒नः सु॑व॒र्गः सु॑व॒र्ग ऊ॒न आयु॒ रायु॑ रू॒नः सु॑व॒र्गः । \newline
21. ऊ॒नः सु॑व॒र्गः सु॑व॒र्ग ऊ॒न ऊ॒नः सु॑व॒र्गो वै वै सु॑व॒र्ग ऊ॒न ऊ॒नः सु॑व॒र्गो वै । \newline
22. सु॒व॒र्गो वै वै सु॑व॒र्गः सु॑व॒र्गो वै लो॒को लो॒को वै सु॑व॒र्गः सु॑व॒र्गो वै लो॒कः । \newline
23. सु॒व॒र्ग इति॑ सुवः - गः । \newline
24. वै लो॒को लो॒को वै वै लो॒को ज्योति॒र् ज्योति॑र् लो॒को वै वै लो॒को ज्योतिः॑ । \newline
25. लो॒को ज्योति॒र् ज्योति॑र् लो॒को लो॒को ज्योति॒ रूर् गूर्ग् ज्योति॑र् लो॒को लो॒को ज्योति॒रूर्क् । \newline
26. ज्योति॒ रूर् गूर्ग् ज्योति॒र् ज्योति॒ रूर्ग् वि॒राड् वि॒राडूर्ग् ज्योति॒र् ज्योति॒ रूर्ग् वि॒राट् । \newline
27. ऊर्ग् वि॒राड् वि॒रा डूर्गूर्ग् वि॒राट् थ्सु॑व॒र्गꣳ सु॑व॒र्गं ॅवि॒रा डूर् गूर्ग् वि॒राट् थ्सु॑व॒र्गम् । \newline
28. वि॒राट् थ्सु॑व॒र्गꣳ सु॑व॒र्गं ॅवि॒राड् वि॒राट् थ्सु॑व॒र्ग मे॒वैव सु॑व॒र्गं ॅवि॒राड् वि॒राट् थ्सु॑व॒र्ग मे॒व । \newline
29. वि॒राडिति॑ वि - राट् । \newline
30. सु॒व॒र्ग मे॒वैव सु॑व॒र्गꣳ सु॑व॒र्ग मे॒व तेन॒ तेनै॒व सु॑व॒र्गꣳ सु॑व॒र्ग मे॒व तेन॑ । \newline
31. सु॒व॒र्गमिति॑ सुवः - गम् । \newline
32. ए॒व तेन॒ तेनै॒ वैव तेन॑ लो॒कम् ॅलो॒कम् तेनै॒ वैव तेन॑ लो॒कम् । \newline
33. तेन॑ लो॒कम् ॅलो॒कम् तेन॒ तेन॑ लो॒कं ॅय॑न्ति यन्ति लो॒कम् तेन॒ तेन॑ लो॒कं ॅय॑न्ति । \newline
34. लो॒कं ॅय॑न्ति यन्ति लो॒कम् ॅलो॒कं ॅय॑न्ति रथन्त॒रꣳ र॑थन्त॒रं ॅय॑न्ति लो॒कम् ॅलो॒कं ॅय॑न्ति रथन्त॒रम् । \newline
35. य॒न्ति॒ र॒थ॒न्त॒रꣳ र॑थन्त॒रं ॅय॑न्ति यन्ति रथन्त॒रम् दिवा॒ दिवा॑ रथन्त॒रं ॅय॑न्ति यन्ति रथन्त॒रम् दिवा᳚ । \newline
36. र॒थ॒न्त॒रम् दिवा॒ दिवा॑ रथन्त॒रꣳ र॑थन्त॒रम् दिवा॒ भव॑ति॒ भव॑ति॒ दिवा॑ रथन्त॒रꣳ र॑थन्त॒रम् दिवा॒ भव॑ति । \newline
37. र॒थ॒न्त॒रमिति॑ रथं - त॒रम् । \newline
38. दिवा॒ भव॑ति॒ भव॑ति॒ दिवा॒ दिवा॒ भव॑ति रथन्त॒रꣳ र॑थन्त॒रम् भव॑ति॒ दिवा॒ दिवा॒ भव॑ति रथन्त॒रम् । \newline
39. भव॑ति रथन्त॒रꣳ र॑थन्त॒रम् भव॑ति॒ भव॑ति रथन्त॒रन् नक्त॒न् नक्तꣳ॑ रथन्त॒रम् भव॑ति॒ भव॑ति रथन्त॒रन् नक्त᳚म् । \newline
40. र॒थ॒न्त॒रन् नक्त॒न् नक्तꣳ॑ रथन्त॒रꣳ र॑थन्त॒रन् नक्त॒ मितीति॒ नक्तꣳ॑ रथन्त॒रꣳ र॑थन्त॒रन् नक्त॒ मिति॑ । \newline
41. र॒थ॒न्त॒रमिति॑ रथं - त॒रम् । \newline
42. नक्त॒ मितीति॒ नक्त॒न् नक्त॒ मित्या॑हु राहु॒ रिति॒ नक्त॒न् नक्त॒ मित्या॑हुः । \newline
43. इत्या॑हु राहु॒ रिती त्या॑हुर् ब्रह्मवा॒दिनो᳚ ब्रह्मवा॒दिन॑ आहु॒ रिती त्या॑हुर् ब्रह्मवा॒दिनः॑ । \newline
44. आ॒हु॒र् ब्र॒ह्म॒वा॒दिनो᳚ ब्रह्मवा॒दिन॑ आहु राहुर् ब्रह्मवा॒दिनः॒ केन॒ केन॑ ब्रह्मवा॒दिन॑ आहु राहुर् ब्रह्मवा॒दिनः॒ केन॑ । \newline
45. ब्र॒ह्म॒वा॒दिनः॒ केन॒ केन॑ ब्रह्मवा॒दिनो᳚ ब्रह्मवा॒दिनः॒ केन॒ तत् तत् केन॑ ब्रह्मवा॒दिनो᳚ ब्रह्मवा॒दिनः॒ केन॒ तत् । \newline
46. ब्र॒ह्म॒वा॒दिन॒ इति॑ ब्रह्म - वा॒दिनः॑ । \newline
47. केन॒ तत् तत् केन॒ केन॒ तदजा॒ म्यजा॑मि॒ तत् केन॒ केन॒ तदजा॑मि । \newline
48. तदजा॒ म्यजा॑मि॒ तत् तदजा॒ मीती त्यजा॑मि॒ तत् तदजा॒मीति॑ । \newline
49. अजा॒ मीती त्यजा॒ म्यजा॒ मीति॑ सौभ॒रꣳ सौ॑भ॒र मित्यजा॒ म्यजा॒ मीति॑ सौभ॒रम् । \newline
50. इति॑ सौभ॒रꣳ सौ॑भ॒र मितीति॑ सौभ॒रम् तृ॑तीयसव॒ने तृ॑तीयसव॒ने सौ॑भ॒र मितीति॑ सौभ॒रम् तृ॑तीयसव॒ने । \newline
51. सौ॒भ॒रम् तृ॑तीयसव॒ने तृ॑तीयसव॒ने सौ॑भ॒रꣳ सौ॑भ॒रम् तृ॑तीयसव॒ने ब्र॑ह्मसा॒मम् ब्र॑ह्मसा॒मम् तृ॑तीयसव॒ने सौ॑भ॒रꣳ सौ॑भ॒रम् तृ॑तीयसव॒ने ब्र॑ह्मसा॒मम् । \newline
52. तृ॒ती॒य॒स॒व॒ने ब्र॑ह्मसा॒मम् ब्र॑ह्मसा॒मम् तृ॑तीयसव॒ने तृ॑तीयसव॒ने ब्र॑ह्मसा॒मम् बृ॒हद् बृ॒हद् ब्र॑ह्मसा॒मम् तृ॑तीयसव॒ने तृ॑तीयसव॒ने ब्र॑ह्मसा॒मम् बृ॒हत् । \newline
53. तृ॒ती॒य॒स॒व॒न इति॑ तृतीय - स॒व॒ने । \newline
54. ब्र॒ह्म॒सा॒मम् बृ॒हद् बृ॒हद् ब्र॑ह्मसा॒मम् ब्र॑ह्मसा॒मम् बृ॒हत् तत् तद् बृ॒हद् ब्र॑ह्मसा॒मम् ब्र॑ह्मसा॒मम् बृ॒हत् तत् । \newline
55. ब्र॒ह्म॒सा॒ममिति॑ ब्रह्म - सा॒मम् । \newline
56. बृ॒हत् तत् तद् बृ॒हद् बृ॒हत् तन् म॑द्ध्य॒तो म॑द्ध्य॒त स्तद् बृ॒हद् बृ॒हत् तन् म॑द्ध्य॒तः । \newline
57. तन् म॑द्ध्य॒तो म॑द्ध्य॒त स्तत् तन् म॑द्ध्य॒तो द॑धति दधति मद्ध्य॒त स्तत् तन् म॑द्ध्य॒तो द॑धति । \newline
58. म॒द्ध्य॒तो द॑धति दधति मद्ध्य॒तो म॑द्ध्य॒तो द॑धति॒ विधृ॑त्यै॒ विधृ॑त्यै दधति मद्ध्य॒तो म॑द्ध्य॒तो द॑धति॒ विधृ॑त्यै । \newline
59. द॒ध॒ति॒ विधृ॑त्यै॒ विधृ॑त्यै दधति दधति॒ विधृ॑त्यै॒ तेन॒ तेन॒ विधृ॑त्यै दधति दधति॒ विधृ॑त्यै॒ तेन॑ । \newline
60. विधृ॑त्यै॒ तेन॒ तेन॒ विधृ॑त्यै॒ विधृ॑त्यै॒ तेनाजा॒ म्यजा॑मि॒ तेन॒ विधृ॑त्यै॒ विधृ॑त्यै॒ तेना जा॑मि । \newline
61. विधृ॑त्या॒ इति॒ वि - धृ॒त्यै॒ । \newline
62. तेनाजा॒ म्यजा॑मि॒ तेन॒ तेना जा॑मि । \newline
63. अजा॒मीत्यजा॑मि । \newline
\pagebreak
\markright{ TS 7.4.11.1  \hfill https://www.vedavms.in \hfill}

\section{ TS 7.4.11.1 }

\textbf{TS 7.4.11.1 } \newline
\textbf{Samhita Paata} \newline

ज्योति॑ष्टोमं प्रथ॒ममुप॑ यन्त्य॒स्मिन्ने॒व तेन॑ लो॒के प्रति॑ तिष्ठन्ति॒ गोष्टो॑मं द्वि॒तीय॒मुप॑ यन्त्य॒न्तरि॑क्ष ए॒व तेन॒ प्रति॑ तिष्ठ॒न्त्यायु॑ष्टोमं तृ॒तीय॒मुप॑ यन्त्य॒मुष्मि॑न्ने॒व तेन॑ लो॒के प्रति॑ तिष्ठन्ती॒यं ॅवाव ज्योति॑र॒न्तरि॑क्षं॒ गौर॒सावायु॒-र्यदे॒तान्थ्-स्तोमा॑-नुप॒यन्त्ये॒ष्वे॑व तल्लो॒केषु॑ स॒त्रिणः॑ प्रति॒ तिष्ठ॑न्तो यन्ति॒ ते सꣳस्तु॑ता वि॒राज॑ - [  ] \newline

\textbf{Pada Paata} \newline

ज्योति॑ष्टोम॒मिति॒ ज्योतिः॑ - स्तो॒म॒म् । प्र॒थ॒मम् । उपेति॑ । य॒न्ति॒ । अ॒स्मिन्न् । ए॒व । तेन॑ । लो॒के । प्रतीति॑ । ति॒ष्ठ॒न्ति॒ । गोष्टो॑म॒मिति॒ गो - स्तो॒म॒म् । द्वि॒तीय᳚म् । उपेति॑ । य॒न्ति॒ । अ॒न्तरि॑क्षे । ए॒व । तेन॑ । प्रतीति॑ । ति॒ष्ठ॒न्ति॒ । आयु॑ष्टोम॒मित्यायुः॑ - स्ता॒म॒म् । तृ॒तीय᳚म् । उपेति॑ । य॒न्ति॒ । अ॒मुष्मिन्न्॑ । ए॒व । तेन॑ । लो॒के । प्रतीति॑ । ति॒ष्ठ॒न्ति॒ । इ॒यम् । वाव । ज्योतिः॑ । अ॒न्तरि॑क्षम् । गौः । अ॒सौ । आयुः॑ । यत् । ए॒तान् । स्तोमान्॑ । उ॒प॒यन्तीत्यु॑प - यन्ति॑ । ए॒षु । ए॒व । तत् । लो॒केषु॑ । स॒त्रिणः॑ । प्र॒ति॒तिष्ठ॑न्त॒ इति॑ प्रति - तिष्ठ॑न्तः । य॒न्ति॒ । ते । सꣳस्तु॑ता॒ इति॒ सं - स्तु॒ताः॒ । वि॒राज॒मिति॑ वि - राज᳚म् ।  \newline


\textbf{Krama Paata} \newline

ज्योति॑ष्टोमम् प्रथ॒मम् । ज्योति॑ष्टोम॒मिति॒ ज्योतिः॑ - स्तो॒म॒म् । प्र॒थ॒ममुप॑ । उप॑ यन्ति । य॒न्त्य॒स्मिन्न् । अ॒स्मिन्ने॒व । ए॒व तेन॑ । तेन॑ लो॒के । लो॒के प्रति॑ । प्रति॑ तिष्ठन्ति । ति॒ष्ठ॒न्ति॒ गोष्टो॑मम् । गोष्टो॑मम् द्वि॒तीय᳚म् । गोष्टो॑म॒मिति॒ गो - स्तो॒म॒म् । द्वि॒तीय॒मुप॑ । उप॑ यन्ति । य॒न्त्य॒न्तरि॑क्षे । अ॒न्तरि॑क्ष ए॒व । ए॒व तेन॑ । तेन॒ प्रति॑ । प्रति॑ तिष्ठन्ति । ति॒ष्ठ॒न्त्यायु॑ष्टोमम् । आयु॑ष्टोमम् तृ॒तीय᳚म् । आयु॑ष्टोम॒मित्यायुः॑ - स्तो॒म॒म् । तृ॒तीय॒मुप॑ । उप॑ यन्ति । य॒न्त्य॒मुष्मिन्न्॑ । अ॒मुष्मि॑न्ने॒व । ए॒व तेन॑ । तेन॑ लो॒के । लो॒के प्रति॑ । प्रति॑ तिष्ठन्ति । ति॒ष्ठ॒न्ती॒यम् । इ॒यम् ॅवाव । वाव ज्योतिः॑ । ज्योति॑र॒न्तरि॑क्षम् । अ॒न्तरि॑क्ष॒म् गौः । गौर॒सौ । अ॒सावायुः॑ । आयु॒र् यत् । यदे॒तान् । ए॒तान्थ् स्तोमान्॑ । स्तोमा॑नुप॒यन्ति॑ । उ॒प॒यन्त्ये॒षु । उ॒प॒यन्तीत्यु॑प - यन्ति॑ । ए॒ष्वे॑व । ए॒व तत् । तल्लो॒केषु॑ । लो॒केषु॑ स॒त्रिणः॑ । स॒त्रिणः॑ प्रति॒तिष्ठ॑न्तः । प्र॒ति॒तिष्ठ॑न्तो यन्ति । प्र॒ति॒तिष्ठ॑न्त॒ इति॑ प्रति - तिष्ठ॑न्तः । य॒न्ति॒ ते । ते सꣳस्तु॑ताः । सꣳस्तु॑ता वि॒राज᳚म् । सꣳस्तु॑ता॒ इति॒ सम् - स्तु॒ताः॒ । वि॒राज॑म॒भि । वि॒राज॒मिति॑ वि - राज᳚म् \newline

\textbf{Jatai Paata} \newline

1. ज्योति॑ष्टोमम् प्रथ॒मम् प्र॑थ॒मम् ज्योति॑ष्टोम॒म् ज्योति॑ष्टोमम् प्रथ॒मम् । \newline
2. ज्योति॑ष्टोम॒मिति॒ ज्योतिः॑ - स्तो॒म॒म् । \newline
3. प्र॒थ॒म मुपोप॑ प्रथ॒मम् प्र॑थ॒म मुप॑ । \newline
4. उप॑ यन्ति य॒न्त्युपोप॑ यन्ति । \newline
5. य॒न्त्य॒स्मिन् न॒स्मिन्. य॑न्ति यन्त्य॒स्मिन्न् । \newline
6. अ॒स्मिन् ने॒वै वास्मिन् न॒स्मिन् ने॒व । \newline
7. ए॒व तेन॒ तेनै॒ वैव तेन॑ । \newline
8. तेन॑ लो॒के लो॒के तेन॒ तेन॑ लो॒के । \newline
9. लो॒के प्रति॒ प्रति॑ लो॒के लो॒के प्रति॑ । \newline
10. प्रति॑ तिष्ठन्ति तिष्ठन्ति॒ प्रति॒ प्रति॑ तिष्ठन्ति । \newline
11. ति॒ष्ठ॒न्ति॒ गोष्टो॑म॒म् गोष्टो॑मम् तिष्ठन्ति तिष्ठन्ति॒ गोष्टो॑मम् । \newline
12. गोष्टो॑मम् द्वि॒तीय॑म् द्वि॒तीय॒म् गोष्टो॑म॒म् गोष्टो॑मम् द्वि॒तीय᳚म् । \newline
13. गोष्टो॑म॒मिति॒ गो - स्तो॒म॒म् । \newline
14. द्वि॒तीय॒ मुपोप॑ द्वि॒तीय॑म् द्वि॒तीय॒ मुप॑ । \newline
15. उप॑ यन्ति य॒न्त्युपोप॑ यन्ति । \newline
16. य॒न्त्य॒न्तरि॑क्षे॒ ऽन्तरि॑क्षे यन्ति यन्त्य॒न्तरि॑क्षे । \newline
17. अ॒न्तरि॑क्ष ए॒वै वान्तरि॑क्षे॒ ऽन्तरि॑क्ष ए॒व । \newline
18. ए॒व तेन॒ तेनै॒ वैव तेन॑ । \newline
19. तेन॒ प्रति॒ प्रति॒ तेन॒ तेन॒ प्रति॑ । \newline
20. प्रति॑ तिष्ठन्ति तिष्ठन्ति॒ प्रति॒ प्रति॑ तिष्ठन्ति । \newline
21. ति॒ष्ठ॒ न्त्यायु॑ष्टोम॒ मायु॑ष्टोमम् तिष्ठन्ति तिष्ठ॒ न्त्यायु॑ष्टोमम् । \newline
22. आयु॑ष्टोमम् तृ॒तीय॑म् तृ॒तीय॒ मायु॑ष्टोम॒ मायु॑ष्टोमम् तृ॒तीय᳚म् । \newline
23. आयु॑ष्टोम॒मित्यायुः॑ - स्ता॒म॒म् । \newline
24. तृ॒तीय॒ मुपोप॑ तृ॒तीय॑म् तृ॒तीय॒ मुप॑ । \newline
25. उप॑ यन्ति य॒न्त्युपोप॑ यन्ति । \newline
26. य॒न्त्य॒मुष्मि॑न् न॒मुष्मि॑न्. यन्ति यन्त्य॒मुष्मिन्न्॑ । \newline
27. अ॒मुष्मि॑न् ने॒वै वामुष्मि॑न् न॒मुष्मि॑न् ने॒व । \newline
28. ए॒व तेन॒ तेनै॒ वैव तेन॑ । \newline
29. तेन॑ लो॒के लो॒के तेन॒ तेन॑ लो॒के । \newline
30. लो॒के प्रति॒ प्रति॑ लो॒के लो॒के प्रति॑ । \newline
31. प्रति॑ तिष्ठन्ति तिष्ठन्ति॒ प्रति॒ प्रति॑ तिष्ठन्ति । \newline
32. ति॒ष्ठ॒न्ती॒य मि॒यम् ति॑ष्ठन्ति तिष्ठन्ती॒यम् । \newline
33. इ॒यं ॅवाव वावेय मि॒यं ॅवाव । \newline
34. वाव ज्योति॒र् ज्योति॒र् वाव वाव ज्योतिः॑ । \newline
35. ज्योति॑ र॒न्तरि॑क्ष म॒न्तरि॑क्ष॒म् ज्योति॒र् ज्योति॑ र॒न्तरि॑क्षम् । \newline
36. अ॒न्तरि॑क्ष॒म् गौर् गौ र॒न्तरि॑क्ष म॒न्तरि॑क्ष॒म् गौः । \newline
37. गौ र॒सा व॒सौ गौर् गौ र॒सौ । \newline
38. अ॒सा वायु॒ रायु॑ र॒सा व॒सा वायुः॑ । \newline
39. आयु॒र् यद् यदायु॒ रायु॒र् यत् । \newline
40. यदे॒ता ने॒तान्. यद् यदे॒तान् । \newline
41. ए॒तान् थ्स्तोमा॒न् थ्स्तोमा॑ ने॒ता ने॒तान् थ्स्तोमान्॑ । \newline
42. स्तोमा॑ नुप॒य न्त्यु॑प॒यन्ति॒ स्तोमा॒न् थ्स्तोमा॑ नुप॒यन्ति॑ । \newline
43. उ॒प॒य न्त्ये॒ष्वे᳚(1॒)षू॑प॒य न्त्यु॑प॒य न्त्ये॒षु । \newline
44. उ॒प॒यन्तीत्यु॑प - यन्ति॑ । \newline
45. ए॒ष्वे॑वै वैष्वे᳚(1॒) ष्वे॑व । \newline
46. ए॒व तत् तदे॒ वैव तत् । \newline
47. तल् लो॒केषु॑ लो॒केषु॒ तत् तल् लो॒केषु॑ । \newline
48. लो॒केषु॑ स॒त्रिणः॑ स॒त्रिणो॑ लो॒केषु॑ लो॒केषु॑ स॒त्रिणः॑ । \newline
49. स॒त्रिणः॑ प्रति॒तिष्ठ॑न्तः प्रति॒तिष्ठ॑न्तः स॒त्रिणः॑ स॒त्रिणः॑ प्रति॒तिष्ठ॑न्तः । \newline
50. प्र॒ति॒तिष्ठ॑न्तो यन्ति यन्ति प्रति॒तिष्ठ॑न्तः प्रति॒तिष्ठ॑न्तो यन्ति । \newline
51. प्र॒ति॒तिष्ठ॑न्त॒ इति॑ प्रति - तिष्ठ॑न्तः । \newline
52. य॒न्ति॒ ते ते य॑न्ति यन्ति॒ ते । \newline
53. ते सꣳस्तु॑ताः॒ सꣳस्तु॑ता॒ स्ते ते सꣳस्तु॑ताः । \newline
54. सꣳस्तु॑ता वि॒राजं॑ ॅवि॒राजꣳ॒॒ सꣳस्तु॑ताः॒ सꣳस्तु॑ता वि॒राज᳚म् । \newline
55. सꣳस्तु॑ता॒ इति॒ सं - स्तु॒ताः॒ । \newline
56. वि॒राज॑ म॒भ्य॑भि वि॒राजं॑ ॅवि॒राज॑ म॒भि । \newline
57. वि॒राज॒मिति॑ वि - राज᳚म् । \newline

\textbf{Ghana Paata } \newline

1. ज्योति॑ष्टोमम् प्रथ॒मम् प्र॑थ॒मम् ज्योति॑ष्टोम॒म् ज्योति॑ष्टोमम् प्रथ॒म मुपोप॑ प्रथ॒मम् ज्योति॑ष्टोम॒म् ज्योति॑ष्टोमम् प्रथ॒म मुप॑ । \newline
2. ज्योति॑ष्टोम॒मिति॒ ज्योतिः॑ - स्तो॒म॒म् । \newline
3. प्र॒थ॒म मुपोप॑ प्रथ॒मम् प्र॑थ॒म मुप॑ यन्ति य॒न्त्युप॑ प्रथ॒मम् प्र॑थ॒म मुप॑ यन्ति । \newline
4. उप॑ यन्ति य॒न्त्युपोप॑ यन्त्य॒स्मिन् न॒स्मिन्. य॒न्त्युपोप॑ यन्त्य॒स्मिन्न् । \newline
5. य॒ न्त्य॒स्मिन् न॒स्मिन्. य॑न्ति यन्त्य॒स्मिन् ने॒वै वास्मिन्. य॑न्ति यन्त्य॒स्मिन् ने॒व । \newline
6. अ॒स्मिन् ने॒वै वास्मिन् न॒स्मिन् ने॒व तेन॒ तेनै॒ वास्मिन् न॒स्मिन् ने॒व तेन॑ । \newline
7. ए॒व तेन॒ तेनै॒ वैव तेन॑ लो॒के लो॒के तेनै॒ वैव तेन॑ लो॒के । \newline
8. तेन॑ लो॒के लो॒के तेन॒ तेन॑ लो॒के प्रति॒ प्रति॑ लो॒के तेन॒ तेन॑ लो॒के प्रति॑ । \newline
9. लो॒के प्रति॒ प्रति॑ लो॒के लो॒के प्रति॑ तिष्ठन्ति तिष्ठन्ति॒ प्रति॑ लो॒के लो॒के प्रति॑ तिष्ठन्ति । \newline
10. प्रति॑ तिष्ठन्ति तिष्ठन्ति॒ प्रति॒ प्रति॑ तिष्ठन्ति॒ गोष्टो॑म॒म् गोष्टो॑मम् तिष्ठन्ति॒ प्रति॒ प्रति॑ तिष्ठन्ति॒ गोष्टो॑मम् । \newline
11. ति॒ष्ठ॒न्ति॒ गोष्टो॑म॒म् गोष्टो॑मम् तिष्ठन्ति तिष्ठन्ति॒ गोष्टो॑मम् द्वि॒तीय॑म् द्वि॒तीय॒म् गोष्टो॑मम् तिष्ठन्ति तिष्ठन्ति॒ गोष्टो॑मम् द्वि॒तीय᳚म् । \newline
12. गोष्टो॑मम् द्वि॒तीय॑म् द्वि॒तीय॒म् गोष्टो॑म॒म् गोष्टो॑मम् द्वि॒तीय॒ मुपोप॑ द्वि॒तीय॒म् गोष्टो॑म॒म् गोष्टो॑मम् द्वि॒तीय॒ मुप॑ । \newline
13. गोष्टो॑म॒मिति॒ गो - स्तो॒म॒म् । \newline
14. द्वि॒तीय॒ मुपोप॑ द्वि॒तीय॑म् द्वि॒तीय॒ मुप॑ यन्ति य॒न्त्युप॑ द्वि॒तीय॑म् द्वि॒तीय॒ मुप॑ यन्ति । \newline
15. उप॑ यन्ति य॒न्त्युपोप॑ यन्त्य॒न्तरि॑क्षे॒ ऽन्तरि॑क्षे य॒न्त्युपोप॑ यन्त्य॒न्तरि॑क्षे । \newline
16. य॒न्त्य॒न्तरि॑क्षे॒ ऽन्तरि॑क्षे यन्ति यन्त्य॒न्तरि॑क्ष ए॒वै वान्तरि॑क्षे यन्ति यन्त्य॒न्तरि॑क्ष ए॒व । \newline
17. अ॒न्तरि॑क्ष ए॒वै वान्तरि॑क्षे॒ ऽन्तरि॑क्ष ए॒व तेन॒ तेनै॒ वान्तरि॑क्षे॒ ऽन्तरि॑क्ष ए॒व तेन॑ । \newline
18. ए॒व तेन॒ तेनै॒ वैव तेन॒ प्रति॒ प्रति॒ तेनै॒ वैव तेन॒ प्रति॑ । \newline
19. तेन॒ प्रति॒ प्रति॒ तेन॒ तेन॒ प्रति॑ तिष्ठन्ति तिष्ठन्ति॒ प्रति॒ तेन॒ तेन॒ प्रति॑ तिष्ठन्ति । \newline
20. प्रति॑ तिष्ठन्ति तिष्ठन्ति॒ प्रति॒ प्रति॑ तिष्ठ॒ न्त्यायु॑ष्टोम॒ मायु॑ष्टोमम् तिष्ठन्ति॒ प्रति॒ प्रति॑ तिष्ठ॒ न्त्यायु॑ष्टोमम् । \newline
21. ति॒ष्ठ॒ न्त्यायु॑ष्टोम॒ मायु॑ष्टोमम् तिष्ठन्ति तिष्ठ॒ न्त्यायु॑ष्टोमम् तृ॒तीय॑म् तृ॒तीय॒ मायु॑ष्टोमम् तिष्ठन्ति तिष्ठ॒ न्त्यायु॑ष्टोमम् तृ॒तीय᳚म् । \newline
22. आयु॑ष्टोमम् तृ॒तीय॑म् तृ॒तीय॒ मायु॑ष्टोम॒ मायु॑ष्टोमम् तृ॒तीय॒ मुपोप॑ तृ॒तीय॒ मायु॑ष्टोम॒ मायु॑ष्टोमम् तृ॒तीय॒ मुप॑ । \newline
23. आयु॑ष्टोम॒मित्यायुः॑ - स्ता॒म॒म् । \newline
24. तृ॒तीय॒ मुपोप॑ तृ॒तीय॑म् तृ॒तीय॒ मुप॑ यन्ति य॒न्त्युप॑ तृ॒तीय॑म् तृ॒तीय॒ मुप॑ यन्ति । \newline
25. उप॑ यन्ति य॒न्त्युपोप॑ य न्त्य॒मुष्मि॑न् न॒मुष्मि॑न्. य॒न्त्युपोप॑ यन्त्य॒मुष्मिन्न्॑ । \newline
26. य॒ न्त्य॒मुष्मि॑न् न॒मुष्मि॑न्. यन्ति यन्त्य॒मुष्मि॑न् ने॒वै वामुष्मि॑न्. यन्ति यन्त्य॒मुष्मि॑न् ने॒व । \newline
27. अ॒मुष्मि॑न् ने॒वै वामुष्मि॑न् न॒मुष्मि॑न् ने॒व तेन॒ तेनै॒ वामुष्मि॑न् न॒मुष्मि॑न् ने॒व तेन॑ । \newline
28. ए॒व तेन॒ तेनै॒वैव तेन॑ लो॒के लो॒के तेनै॒वैव तेन॑ लो॒के । \newline
29. तेन॑ लो॒के लो॒के तेन॒ तेन॑ लो॒के प्रति॒ प्रति॑ लो॒के तेन॒ तेन॑ लो॒के प्रति॑ । \newline
30. लो॒के प्रति॒ प्रति॑ लो॒के लो॒के प्रति॑ तिष्ठन्ति तिष्ठन्ति॒ प्रति॑ लो॒के लो॒के प्रति॑ तिष्ठन्ति । \newline
31. प्रति॑ तिष्ठन्ति तिष्ठन्ति॒ प्रति॒ प्रति॑ तिष्ठन्ती॒य मि॒यम् ति॑ष्ठन्ति॒ प्रति॒ प्रति॑ तिष्ठन्ती॒यम् । \newline
32. ति॒ष्ठ॒न्ती॒य मि॒यम् ति॑ष्ठन्ति तिष्ठन्ती॒यं ॅवाव वावेयम् ति॑ष्ठन्ति तिष्ठन्ती॒यं ॅवाव । \newline
33. इ॒यं ॅवाव वावेय मि॒यं ॅवाव ज्योति॒र् ज्योति॒र् वावेय मि॒यं ॅवाव ज्योतिः॑ । \newline
34. वाव ज्योति॒र् ज्योति॒र् वाव वाव ज्योति॑ र॒न्तरि॑क्ष म॒न्तरि॑क्ष॒म् ज्योति॒र् वाव वाव ज्योति॑ र॒न्तरि॑क्षम् । \newline
35. ज्योति॑ र॒न्तरि॑क्ष म॒न्तरि॑क्ष॒म् ज्योति॒र् ज्योति॑ र॒न्तरि॑क्ष॒म् गौर् गौ र॒न्तरि॑क्ष॒म् ज्योति॒र् ज्योति॑ र॒न्तरि॑क्ष॒म् गौः । \newline
36. अ॒न्तरि॑क्ष॒म् गौर् गौ र॒न्तरि॑क्ष म॒न्तरि॑क्ष॒म् गौ र॒सा व॒सौ गौ र॒न्तरि॑क्ष म॒न्तरि॑क्ष॒म् गौर॒सौ । \newline
37. गौ र॒सा व॒सौ गौर् गौ र॒सा वायु॒ रायु॑ र॒सौ गौर् गौ र॒सा वायुः॑ । \newline
38. अ॒सा वायु॒ रायु॑ र॒सा व॒सा वायु॒र् यद् यदायु॑ र॒सा व॒सा वायु॒र् यत् । \newline
39. आयु॒र् यद् यदायु॒ रायु॒र् यदे॒ता ने॒तान्. यदायु॒ रायु॒र् यदे॒तान् । \newline
40. यदे॒ता ने॒तान्. यद् यदे॒तान् थ्स्तोमा॒न् थ्स्तोमा॑ ने॒तान्. यद् यदे॒तान् थ्स्तोमान्॑ । \newline
41. ए॒तान् थ्स्तोमा॒न् थ्स्तोमा॑ ने॒ता ने॒तान् थ्स्तोमा॑ नुप॒य न्त्यु॑प॒यन्ति॒ स्तोमा॑ ने॒ता ने॒तान् थ्स्तोमा॑ नुप॒यन्ति॑ । \newline
42. स्तोमा॑ नुप॒य न्त्यु॑प॒यन्ति॒ स्तोमा॒न् थ्स्तोमा॑ नुप॒य न्त्ये॒ष्वे᳚(1॒) षू॑प॒यन्ति॒ स्तोमा॒न् थ्स्तोमा॑ नुप॒य न्त्ये॒षु । \newline
43. उ॒प॒य न्त्ये॒ष्वे᳚(1॒) षू॑प॒य न्त्यु॑प॒य न्त्ये॒ष्वे॑ वैवैषू॑ प॒य न्त्यु॑प॒य न्त्ये॒ष्वे॑व । \newline
44. उ॒प॒यन्तीत्यु॑प - यन्ति॑ । \newline
45. ए॒ष्वे॑ वैवै ष्वे᳚(1॒)ष्वे॑व तत् तदे॒ वैष्वे᳚(1॒) ष्वे॑व तत् । \newline
46. ए॒व तत् तदे॒ वैव तल्लो॒केषु॑ लो॒केषु॒ तदे॒ वैव तल्लो॒केषु॑ । \newline
47. तल्लो॒केषु॑ लो॒केषु॒ तत् तल्लो॒केषु॑ स॒त्रिणः॑ स॒त्रिणो॑ लो॒केषु॒ तत् तल्लो॒केषु॑ स॒त्रिणः॑ । \newline
48. लो॒केषु॑ स॒त्रिणः॑ स॒त्रिणो॑ लो॒केषु॑ लो॒केषु॑ स॒त्रिणः॑ प्रति॒तिष्ठ॑न्तः प्रति॒तिष्ठ॑न्तः स॒त्रिणो॑ लो॒केषु॑ लो॒केषु॑ स॒त्रिणः॑ प्रति॒तिष्ठ॑न्तः । \newline
49. स॒त्रिणः॑ प्रति॒तिष्ठ॑न्तः प्रति॒तिष्ठ॑न्तः स॒त्रिणः॑ स॒त्रिणः॑ प्रति॒तिष्ठ॑न्तो यन्ति यन्ति प्रति॒तिष्ठ॑न्तः स॒त्रिणः॑ स॒त्रिणः॑ प्रति॒तिष्ठ॑न्तो यन्ति । \newline
50. प्र॒ति॒तिष्ठ॑न्तो यन्ति यन्ति प्रति॒तिष्ठ॑न्तः प्रति॒तिष्ठ॑न्तो यन्ति॒ ते ते य॑न्ति प्रति॒तिष्ठ॑न्तः प्रति॒तिष्ठ॑न्तो यन्ति॒ ते । \newline
51. प्र॒ति॒तिष्ठ॑न्त॒ इति॑ प्रति - तिष्ठ॑न्तः । \newline
52. य॒न्ति॒ ते ते य॑न्ति यन्ति॒ ते सꣳस्तु॑ताः॒ सꣳस्तु॑ता॒ स्ते य॑न्ति यन्ति॒ ते सꣳस्तु॑ताः । \newline
53. ते सꣳस्तु॑ताः॒ सꣳस्तु॑ता॒ स्ते ते सꣳस्तु॑ता वि॒राजं॑ ॅवि॒राजꣳ॒॒ सꣳस्तु॑ता॒ स्ते ते सꣳस्तु॑ता वि॒राज᳚म् । \newline
54. सꣳस्तु॑ता वि॒राजं॑ ॅवि॒राजꣳ॒॒ सꣳस्तु॑ताः॒ सꣳस्तु॑ता वि॒राज॑ म॒भ्य॑भि वि॒राजꣳ॒॒ सꣳस्तु॑ताः॒ सꣳस्तु॑ता वि॒राज॑ म॒भि । \newline
55. सꣳस्तु॑ता॒ इति॒ सं - स्तु॒ताः॒ । \newline
56. वि॒राज॑ म॒भ्य॑भि वि॒राजं॑ ॅवि॒राज॑ म॒भि सꣳ सम॒भि वि॒राजं॑ ॅवि॒राज॑ म॒भि सम् । \newline
57. वि॒राज॒मिति॑ वि - राज᳚म् । \newline
\pagebreak
\markright{ TS 7.4.11.2  \hfill https://www.vedavms.in \hfill}

\section{ TS 7.4.11.2 }

\textbf{TS 7.4.11.2 } \newline
\textbf{Samhita Paata} \newline

-म॒भि संप॑द्यन्ते॒ द्वे चर्चा॒वति॑ रिच्येते॒ एक॑या॒ गौरति॑रिक्त॒ एक॒याऽऽयु॑रू॒नः सु॑व॒र्गो वै लो॒को ज्योति॒रूर्ग्-वि॒राडूर्ज॑-मे॒वाव॑ रुन्धते॒ ते न क्षु॒धा ऽऽर्ति॒मार्च्छ॒न्त्यक्षो॑धुका भवन्ति॒ क्षुथ् स॑बांधा इव॒ हि स॒त्रिणो᳚ ऽग्निष्टो॒माव॒भितः॑ प्र॒धी तावु॒क्थ्या॑ मद्ध्ये॒ नभ्यं॒ तत् तदे॒तत् प॑रि॒यद्-दे॑वच॒क्रं ॅयदे॒तेन॑ - [  ] \newline

\textbf{Pada Paata} \newline

अ॒भि । समिति॑ । प॒द्य॒न्ते॒ । द्वे इति॑ । च॒ । ऋचौ᳚ । अतीति॑ । रि॒च्ये॒ते॒ इति॑ । एक॑या । गौः । अति॑रिक्त॒ इत्यति॑ - रि॒क्तः॒ । एक॑या । आयुः॑ । ऊ॒नः । सु॒व॒र्ग इति॑ सुवः - गः । वै । लो॒कः । ज्योतिः॑ । ऊर्क् । वि॒राडिति॑ वि - राट् । ऊर्ज᳚म् । ए॒व । अवेति॑ । रु॒न्ध॒ते॒ । ते । न । क्षु॒धा । आर्ति᳚म् । एति॑ । ऋ॒च्छ॒न्ति॒ । अक्षो॑धुकाः । भ॒व॒न्ति॒ । क्षुथ्स॑म्बाधा॒ इति॒ क्षुत् - स॒म्बा॒धाः॒ । इ॒व॒ । हि । स॒त्रिणः॑ । अ॒ग्नि॒ष्टो॒मावित्य॑ग्नि - स्तो॒मौ । अ॒भितः॑ । प्र॒धी इति॑ प्र -धी । तौ । उ॒क्थ्याः᳚ । मद्ध्ये᳚ । नभ्य᳚म् । तत् । तत् । ए॒तत् । प॒रि॒यदिति॑ परि - यत् । दे॒व॒च॒क्रमिति॑ देव - च॒क्रम् । यत् । ए॒तेन॑ ।  \newline


\textbf{Krama Paata} \newline

अ॒भि सम् । सम् प॑द्यन्ते । प॒द्य॒न्ते॒ द्वे । द्वे च॑ । द्वे इति॒ द्वे । चर्चौ᳚ । ऋचा॒वति॑ । अति॑ रिच्येते । रि॒च्ये॒ते॒ एक॑या । रि॒च्ये॒ते॒ इति॑ रिच्येते । एक॑या॒ गौः । गौरति॑रिक्तः । अति॑रिक्त॒ एक॑या । अति॑रिक्त॒ इत्यति॑ - रि॒क्तः॒ । एक॒याऽऽयुः॑ । आयु॑रू॒नः । ऊ॒नः सु॑व॒र्गः । सु॒व॒र्गो वै । सु॒व॒र्ग इति॑ सुवः - गः । वै लो॒कः । लो॒को ज्योतिः॑ । ज्योति॒रूर्क् । ऊर्ग् वि॒राट् । वि॒राडूर्ज᳚म् । वि॒राडिति॑ वि - राट् । ऊर्ज॑मे॒व । ए॒वाव॑ । अव॑ रुन्धते । रु॒न्ध॒ते॒ ते । ते न । न क्षु॒धा । क्षु॒धाऽऽर्ति᳚म् । आर्ति॒मा । आर्च्छ॑न्ति । ऋ॒च्छ॒न्त्यक्षो॑धुकाः । अक्षो॑धुका भवन्ति । भ॒व॒न्ति॒ क्षुथ्स॑म्बाधाः । क्षुथ्स॑म्बाधा इव । क्षुथ्स॑म्बाधा॒ इति॒ क्षुत् - स॒म्बा॒धाः॒ । इ॒व॒ हि । हि स॒त्रिणः॑ । स॒त्रिणो᳚ऽग्निष्टो॒मौ । अ॒ग्नि॒ष्टो॒माव॒भितः॑ । अ॒ग्नि॒ष्टो॒मावित्य॑ग्नि - स्तो॒मौ । अ॒भितः॑ प्र॒धी । प्र॒धी तौ । प्र॒धी इति॑ प्र - धी । तावु॒क्थ्याः᳚ । उ॒क्थ्या॑ मद्ध्ये᳚ । मद्ध्ये॒ नभ्य᳚म् । नभ्य॒म् तत् । तत् तत् । तदे॒तत् । ए॒तत् प॑रि॒यत् । प॒रि॒यद् दे॑वच॒क्रम् । प॒रि॒यदिति॑ परि - यत् । दे॒व॒च॒क्रम् ॅयत् । दे॒व॒च॒क्रमिति॑ देव - च॒क्रम् । यदे॒तेन॑ । ए॒तेन॑ षड॒हेन॑ \newline

\textbf{Jatai Paata} \newline

1. अ॒भि सꣳ स म॒भ्य॑भि सम् । \newline
2. सम् प॑द्यन्ते पद्यन्ते॒ सꣳ सम् प॑द्यन्ते । \newline
3. प॒द्य॒न्ते॒ द्वे द्वे प॑द्यन्ते पद्यन्ते॒ द्वे । \newline
4. द्वे च॑ च॒ द्वे द्वे च॑ । \newline
5. द्वे इति॒ द्वे । \newline
6. च र्‌चा॒ वृचौ॑ च॒ च र्‌चौ᳚ । \newline
7. ऋचा॒ वत्य त्यृचा॒ वृचा॒ वति॑ । \newline
8. अति॑ रिच्येते रिच्येते॒ अत्यति॑ रिच्येते । \newline
9. रि॒च्ये॒ते॒ एक॒ यैक॑या रिच्येते रिच्येते॒ एक॑या । \newline
10. रि॒च्ये॒ते॒ इति॑ रिच्येते । \newline
11. एक॑या॒ गौर् गौ रेक॒ यैक॑या॒ गौः । \newline
12. गौ रति॑रि॒क्तो ऽति॑रिक्तो॒ गौर् गौ रति॑रिक्तः । \newline
13. अति॑रिक्त॒ एक॒ यैक॒या ऽति॑रि॒क्तो ऽति॑रिक्त॒ एक॑या । \newline
14. अति॑रिक्त॒ इत्यति॑ - रि॒क्तः॒ । \newline
15. एक॒या ऽऽयु॒ रायु॒ रेक॒यै क॒या ऽऽयुः॑ । \newline
16. आयु॑ रू॒न ऊ॒न आयु॒ रायु॑ रू॒नः । \newline
17. ऊ॒नः सु॑व॒र्गः सु॑व॒र्ग ऊ॒न ऊ॒नः सु॑व॒र्गः । \newline
18. सु॒व॒र्गो वै वै सु॑व॒र्गः सु॑व॒र्गो वै । \newline
19. सु॒व॒र्ग इति॑ सुवः - गः । \newline
20. वै लो॒को लो॒को वै वै लो॒कः । \newline
21. लो॒को ज्योति॒र् ज्योति॑र् लो॒को लो॒को ज्योतिः॑ । \newline
22. ज्योति॒ रूर् गूर्ग् ज्योति॒र् ज्योति॒ रूर्क् । \newline
23. ऊर्ग् वि॒राड् वि॒रा डूर् गूर्ग् वि॒राट् । \newline
24. वि॒रा डूर्ज॒ मूर्जं॑ ॅवि॒राड् वि॒रा डूर्ज᳚म् । \newline
25. वि॒राडिति॑ वि - राट् । \newline
26. ऊर्ज॑ मे॒वै वोर्ज॒ मूर्ज॑ मे॒व । \newline
27. ए॒वावा वै॒वै वाव॑ । \newline
28. अव॑ रुन्धते रुन्ध॒ते ऽवाव॑ रुन्धते । \newline
29. रु॒न्ध॒ते॒ ते ते रु॑न्धते रुन्धते॒ ते । \newline
30. ते न न ते ते न । \newline
31. न क्षु॒धा क्षु॒धा न न क्षु॒धा । \newline
32. क्षु॒धा ऽऽर्ति॒ मार्ति॑म् क्षु॒धा क्षु॒धा ऽऽर्ति᳚म् । \newline
33. आर्ति॒ मा ऽऽर्ति॒ मार्ति॒ मा । \newline
34. आर्च्छ॑ न्त्यृच्छ॒ न्त्यार्च्छ॑न्ति । \newline
35. ऋ॒च्छ॒ न्त्यक्षो॑धुका॒ अक्षो॑धुका ऋच्छ न्त्यृच्छ॒ न्त्यक्षो॑धुकाः । \newline
36. अक्षो॑धुका भवन्ति भव॒ न्त्यक्षो॑धुका॒ अक्षो॑धुका भवन्ति । \newline
37. भ॒व॒न्ति॒ क्षुथ्स॑म्बाधाः॒ क्षुथ्स॑म्बाधा भवन्ति भवन्ति॒ क्षुथ्स॑म्बाधाः । \newline
38. क्षुथ्स॑म्बाधा इवेव॒ क्षुथ्स॑म्बाधाः॒ क्षुथ्स॑म्बाधा इव । \newline
39. क्षुथ्स॑म्बाधा॒ इति॒ क्षुत् - स॒म्बा॒धाः॒ । \newline
40. इ॒व॒ हि हीवे॑व॒ हि । \newline
41. हि स॒त्रिणः॑ स॒त्रिणो॒ हि हि स॒त्रिणः॑ । \newline
42. स॒त्रिणो᳚ ऽग्निष्टो॒मा व॑ग्निष्टो॒मौ स॒त्रिणः॑ स॒त्रिणो᳚ ऽग्निष्टो॒मौ । \newline
43. अ॒ग्नि॒ष्टो॒मा व॒भितो॒ ऽभितो᳚ ऽग्निष्टो॒मा व॑ग्निष्टो॒मा व॒भितः॑ । \newline
44. अ॒ग्नि॒ष्टो॒मावित्य॑ग्नि - स्तो॒मौ । \newline
45. अ॒भितः॑ प्र॒धी प्र॒धी अ॒भितो॒ ऽभितः॑ प्र॒धी । \newline
46. प्र॒धी तौ तौ प्र॒धी प्र॒धी तौ । \newline
47. प्र॒धी इति॑ प्र - धी । \newline
48. ता वु॒क्थ्या॑ उ॒क्थ्या᳚ स्तौ ता वु॒क्थ्याः᳚ । \newline
49. उ॒क्थ्या॑ मद्ध्ये॒ मद्ध्य॑ उ॒क्थ्या॑ उ॒क्थ्या॑ मद्ध्ये᳚ । \newline
50. मद्ध्ये॒ नभ्य॒न् नभ्य॒म् मद्ध्ये॒ मद्ध्ये॒ नभ्य᳚म् । \newline
51. नभ्य॒म् तत् तन् नभ्य॒न् नभ्य॒म् तत् । \newline
52. तत् तत् । \newline
53. तदे॒त दे॒तत् तत् तदे॒तत् । \newline
54. ए॒तत् प॑रि॒यत् प॑रि॒य दे॒त दे॒तत् प॑रि॒यत् । \newline
55. प॒रि॒यद् दे॑वच॒क्रम् दे॑वच॒क्रम् प॑रि॒यत् प॑रि॒यद् दे॑वच॒क्रम् । \newline
56. प॒रि॒यदिति॑ परि - यत् । \newline
57. दे॒व॒च॒क्रं ॅयद् यद् दे॑वच॒क्रम् दे॑वच॒क्रं ॅयत् । \newline
58. दे॒व॒च॒क्रमिति॑ देव - च॒क्रम् । \newline
59. यदे॒ते नै॒तेन॒ यद् यदे॒तेन॑ । \newline
60. ए॒तेन॑ षड॒हेन॑ षड॒हे नै॒तेनै॒ तेन॑ षड॒हेन॑ । \newline

\textbf{Ghana Paata } \newline

1. अ॒भि सꣳ सम॒भ्य॑भि सम् प॑द्यन्ते पद्यन्ते॒ सम॒भ्य॑भि सम्प॑द्यन्ते । \newline
2. सम् प॑द्यन्ते पद्यन्ते॒ सꣳ सम् प॑द्यन्ते॒ द्वे द्वे प॑द्यन्ते॒ सꣳ सम् प॑द्यन्ते॒ द्वे । \newline
3. प॒द्य॒न्ते॒ द्वे द्वे प॑द्यन्ते पद्यन्ते॒ द्वे च॑ च॒ द्वे प॑द्यन्ते पद्यन्ते॒ द्वे च॑ । \newline
4. द्वे च॑ च॒ द्वे द्वे च र्‌चा॒ वृचौ॑ च॒ द्वे द्वे च र्‌चौ᳚ । \newline
5. द्वे इति॒ द्वे । \newline
6. च र्‌चा॒ वृचौ॑ च॒ च र्‌चा॒ वत्य त्यृचौ॑ च॒ च र्‌चा॒ वति॑ । \newline
7. ऋचा॒ वत्य त्यृचा॒ वृचा॒ वति॑ रिच्येते रिच्येते॒ अत्यृचा॒ वृचा॒ वति॑ रिच्येते । \newline
8. अति॑ रिच्येते रिच्येते॒ अत्यति॑ रिच्येते॒ एक॒यैक॑या रिच्येते॒ अत्यति॑ रिच्येते॒ एक॑या । \newline
9. रि॒च्ये॒ते॒ एक॒यैक॑या रिच्येते रिच्येते॒ एक॑या॒ गौर् गौ रेक॑या रिच्येते रिच्येते॒ एक॑या॒ गौः । \newline
10. रि॒च्ये॒ते॒ इति॑ रिच्येते । \newline
11. एक॑या॒ गौर् गौरेक॒ यैक॑या॒ गौ रति॑रि॒क्तो ऽति॑रिक्तो॒ गौरेक॒ यैक॑या॒ गौ रति॑रिक्तः । \newline
12. गौ रति॑रि॒क्तो ऽति॑रिक्तो॒ गौर् गौ रति॑रिक्त॒ एक॒यैक॒या ऽति॑रिक्तो॒ गौर् गौ रति॑रिक्त॒ एक॑या । \newline
13. अति॑रिक्त॒ एक॒यैक॒या ऽति॑रि॒क्तो ऽति॑रिक्त॒ एक॒या ऽऽयु॒ रायु॒ रेक॒या ऽति॑रि॒क्तो ऽति॑रिक्त॒ एक॒या ऽऽयुः॑ । \newline
14. अति॑रिक्त॒ इत्यति॑ - रि॒क्तः॒ । \newline
15. एक॒या ऽऽयु॒ रायु॒ रेक॒यैक॒या ऽऽयु॑ रू॒न ऊ॒न आयु॒ रेक॒यैक॒या ऽऽयु॑ रू॒नः । \newline
16. आयु॑ रू॒न ऊ॒न आयु॒ रायु॑ रू॒नः सु॑व॒र्गः सु॑व॒र्ग ऊ॒न आयु॒ रायु॑ रू॒नः सु॑व॒र्गः । \newline
17. ऊ॒नः सु॑व॒र्गः सु॑व॒र्ग ऊ॒न ऊ॒नः सु॑व॒र्गो वै वै सु॑व॒र्ग ऊ॒न ऊ॒नः सु॑व॒र्गो वै । \newline
18. सु॒व॒र्गो वै वै सु॑व॒र्गः सु॑व॒र्गो वै लो॒को लो॒को वै सु॑व॒र्गः सु॑व॒र्गो वै लो॒कः । \newline
19. सु॒व॒र्ग इति॑ सुवः - गः । \newline
20. वै लो॒को लो॒को वै वै लो॒को ज्योति॒र् ज्योति॑र् लो॒को वै वै लो॒को ज्योतिः॑ । \newline
21. लो॒को ज्योति॒र् ज्योति॑र् लो॒को लो॒को ज्योति॒ रूर् गूर्ग् ज्योति॑र् लो॒को लो॒को ज्योति॒ रूर्क् । \newline
22. ज्योति॒ रूर् गूर्ग् ज्योति॒र् ज्योति॒ रूर्ग् वि॒राड् वि॒रा डूर्ग् ज्योति॒र् ज्योति॒ रूर्ग् वि॒राट् । \newline
23. ऊर्ग् वि॒राड् वि॒रा डूर् गूर्ग् वि॒रा डूर्ज॒ मूर्जं॑ ॅवि॒रा डूर् गूर्ग् वि॒रा डूर्ज᳚म् । \newline
24. वि॒रा डूर्ज॒ मूर्जं॑ ॅवि॒राड् वि॒रा डूर्ज॑ मे॒वै वोर्जं॑ ॅवि॒राड् वि॒रा डूर्ज॑ मे॒व । \newline
25. वि॒राडिति॑ वि - राट् । \newline
26. ऊर्ज॑ मे॒वै वोर्ज॒ मूर्ज॑ मे॒वावा वै॒वोर्ज॒ मूर्ज॑ मे॒वाव॑ । \newline
27. ए॒वावा वै॒वै वाव॑ रुन्धते रुन्ध॒ते ऽवै॒वै वाव॑ रुन्धते । \newline
28. अव॑ रुन्धते रुन्ध॒ते ऽवाव॑ रुन्धते॒ ते ते रु॑न्ध॒ते ऽवाव॑ रुन्धते॒ ते । \newline
29. रु॒न्ध॒ते॒ ते ते रु॑न्धते रुन्धते॒ ते न न ते रु॑न्धते रुन्धते॒ ते न । \newline
30. ते न न ते ते न क्षु॒धा क्षु॒धा न ते ते न क्षु॒धा । \newline
31. न क्षु॒धा क्षु॒धा न न क्षु॒धा ऽऽर्ति॒ मार्ति॑म् क्षु॒धा न न क्षु॒धा ऽऽर्ति᳚म् । \newline
32. क्षु॒धा ऽऽर्ति॒ मार्ति॑म् क्षु॒धा क्षु॒धा ऽऽर्ति॒ मा ऽऽर्ति॑म् क्षु॒धा क्षु॒धा ऽऽर्ति॒ मा । \newline
33. आर्ति॒ मा ऽऽर्ति॒ मार्ति॒ मार्च्छ॑ न्त्यृच्छ॒ न्त्याऽऽर्ति॒ मार्ति॒ मार्च्छ॑न्ति । \newline
34. आर्च्छ॑ न्त्यृच्छ न्त्यार्च्छ॒ न्त्यक्षो॑धुका॒ अक्षो॑धुका ऋच्छ न्त्यार्च्छ॒ न्त्यक्षो॑धुकाः । \newline
35. ऋ॒च्छ॒ न्त्यक्षो॑धुका॒ अक्षो॑धुका ऋच्छ न्त्यृच्छ॒ न्त्यक्षो॑धुका भवन्ति भव॒ न्त्यक्षो॑धुका ऋच्छ न्त्यृच्छ॒ न्त्यक्षो॑धुका भवन्ति । \newline
36. अक्षो॑धुका भवन्ति भव॒ न्त्यक्षो॑धुका॒ अक्षो॑धुका भवन्ति॒ क्षुथ्स॑म्बाधाः॒ क्षुथ्स॑म्बाधा भव॒ न्त्यक्षो॑धुका॒ अक्षो॑धुका भवन्ति॒ क्षुथ्स॑म्बाधाः । \newline
37. भ॒व॒न्ति॒ क्षुथ्स॑म्बाधाः॒ क्षुथ्स॑म्बाधा भवन्ति भवन्ति॒ क्षुथ्स॑म्बाधा इवेव॒ क्षुथ्स॑म्बाधा भवन्ति भवन्ति॒ क्षुथ्स॑म्बाधा इव । \newline
38. क्षुथ्स॑म्बाधा इवेव॒ क्षुथ्स॑म्बाधाः॒ क्षुथ्स॑म्बाधा इव॒ हि हीव॒ क्षुथ्स॑म्बाधाः॒ क्षुथ्स॑म्बाधा इव॒ हि । \newline
39. क्षुथ्स॑म्बाधा॒ इति॒ क्षुत् - स॒म्बा॒धाः॒ । \newline
40. इ॒व॒ हि हीवे॑व॒ हि स॒त्रिणः॑ स॒त्रिणो॒ हीवे॑व॒ हि स॒त्रिणः॑ । \newline
41. हि स॒त्रिणः॑ स॒त्रिणो॒ हि हि स॒त्रिणो᳚ ऽग्निष्टो॒मा व॑ग्निष्टो॒मौ स॒त्रिणो॒ हि हि स॒त्रिणो᳚ ऽग्निष्टो॒मौ । \newline
42. स॒त्रिणो᳚ ऽग्निष्टो॒मा व॑ग्निष्टो॒मौ स॒त्रिणः॑ स॒त्रिणो᳚ ऽग्निष्टो॒मा व॒भितो॒ ऽभितो᳚ ऽग्निष्टो॒मौ स॒त्रिणः॑ स॒त्रिणो᳚ ऽग्निष्टो॒मा व॒भितः॑ । \newline
43. अ॒ग्नि॒ष्टो॒मा व॒भितो॒ ऽभितो᳚ ऽग्निष्टो॒मा व॑ग्निष्टो॒मा व॒भितः॑ प्र॒धी प्र॒धी अ॒भितो᳚ ऽग्निष्टो॒मा व॑ग्निष्टो॒मा व॒भितः॑ प्र॒धी । \newline
44. अ॒ग्नि॒ष्टो॒मावित्य॑ग्नि - स्तो॒मौ । \newline
45. अ॒भितः॑ प्र॒धी प्र॒धी अ॒भितो॒ ऽभितः॑ प्र॒धी तौ तौ प्र॒धी अ॒भितो॒ ऽभितः॑ प्र॒धी तौ । \newline
46. प्र॒धी तौ तौ प्र॒धी प्र॒धी ता वु॒क्थ्या॑ उ॒क्थ्या᳚ स्तौ प्र॒धी प्र॒धी ता वु॒क्थ्याः᳚ । \newline
47. प्र॒धी इति॑ प्र - धी । \newline
48. ता वु॒क्थ्या॑ उ॒क्थ्या᳚ स्तौ ता वु॒क्थ्या॑ मद्ध्ये॒ मद्ध्य॑ उ॒क्थ्या᳚ स्तौ ता वु॒क्थ्या॑ मद्ध्ये᳚ । \newline
49. उ॒क्थ्या॑ मद्ध्ये॒ मद्ध्य॑ उ॒क्थ्या॑ उ॒क्थ्या॑ मद्ध्ये॒ नभ्य॒न् नभ्य॒म् मद्ध्य॑ उ॒क्थ्या॑ उ॒क्थ्या॑ मद्ध्ये॒ नभ्य᳚म् । \newline
50. मद्ध्ये॒ नभ्य॒न् नभ्य॒म् मद्ध्ये॒ मद्ध्ये॒ नभ्य॒म् तत् तन् नभ्य॒म् मद्ध्ये॒ मद्ध्ये॒ नभ्य॒म् तत् । \newline
51. नभ्य॒म् तत् तन् नभ्य॒न् नभ्य॒म् तत् । \newline
52. तत् तत् । \newline
53. तदे॒त दे॒तत् तत् तदे॒तत् प॑रि॒यत् प॑रि॒य दे॒तत् तत् तदे॒तत् प॑रि॒यत् । \newline
54. ए॒तत् प॑रि॒यत् प॑रि॒य दे॒त दे॒तत् प॑रि॒यद् दे॑वच॒क्रम् दे॑वच॒क्रम् प॑रि॒य दे॒त दे॒तत् प॑रि॒यद् दे॑वच॒क्रम् । \newline
55. प॒रि॒यद् दे॑वच॒क्रम् दे॑वच॒क्रम् प॑रि॒यत् प॑रि॒यद् दे॑वच॒क्रं ॅयद् यद् दे॑वच॒क्रम् प॑रि॒यत् प॑रि॒यद् दे॑वच॒क्रं ॅयत् । \newline
56. प॒रि॒यदिति॑ परि - यत् । \newline
57. दे॒व॒च॒क्रं ॅयद् यद् दे॑वच॒क्रम् दे॑वच॒क्रं ॅयदे॒ते नै॒तेन॒ यद् दे॑वच॒क्रम् दे॑वच॒क्रं ॅयदे॒तेन॑ । \newline
58. दे॒व॒च॒क्रमिति॑ देव - च॒क्रम् । \newline
59. यदे॒ते नै॒तेन॒ यद् यदे॒तेन॑ षड॒हेन॑ षड॒हे नै॒तेन॒ यद् यदे॒तेन॑ षड॒हेन॑ । \newline
60. ए॒तेन॑ षड॒हेन॑ षड॒हे नै॒ते नै॒तेन॑ षड॒हेन॒ यन्ति॒ यन्ति॑ षड॒हे नै॒ते नै॒तेन॑ षड॒हेन॒ यन्ति॑ । \newline
\pagebreak
\markright{ TS 7.4.11.3  \hfill https://www.vedavms.in \hfill}

\section{ TS 7.4.11.3 }

\textbf{TS 7.4.11.3 } \newline
\textbf{Samhita Paata} \newline

षड॒हेन॒ यन्ति॑ देवच॒क्रमे॒व स॒मारो॑ह॒न्त्यरि॑ष्ट्यै॒ ते स्व॒स्ति सम॑श्नुवते षड॒हेन॑ यन्ति॒ षड्वा ऋ॒तव॑ ऋ॒तुष्वे॒व प्रति॑ तिष्ठन्त्युभ॒यतो᳚ ज्योतिषा यन्त्युभ॒यत॑ ए॒व सु॑व॒र्गे लो॒के प्र॑ति॒तिष्ठ॑न्तो यन्ति॒ द्वौ ष॑ड॒हौ भ॑वत॒स्तानि॒ द्वाद॒शाहा॑नि॒ सं प॑द्यन्ते द्वाद॒शो वै पुरु॑षो॒ द्वे स॒क्थ्यौ᳚ द्वौ बा॒हू आ॒त्मा च॒ शिर॑श्च च॒त्वार्यङ्गा॑नि॒ स्तनौ᳚ द्वाद॒शौ - [  ] \newline

\textbf{Pada Paata} \newline

ष॒ड॒हेनेति॑ षट् - अ॒हेन॑ । यन्ति॑ । दे॒व॒च॒क्रमिति॑ देव-च॒क्रम् । ए॒व । स॒मारो॑ह॒न्तीति॑ सं - आरो॑हन्ति । अरि॑ष्ट्यै । ते । स्व॒स्ति । समिति॑ । अ॒श्नु॒व॒ते॒ । ष॒ड॒हेनेति॑ षट् - अ॒हेन॑ । य॒न्ति॒ । षट् । वै । ऋ॒तवः॑ । ऋ॒तुषु॑ । ए॒व । प्रतीति॑ । ति॒ष्ठ॒न्ति॒ । उ॒भ॒यता᳚ज्योति॒षेत्यु॑भ॒यतः॑-ज्यो॒ति॒षा॒ । य॒न्ति॒ । उ॒भ॒यतः॑ । ए॒व । सु॒व॒र्ग इति॑ सुवः - गे । लो॒के । प्र॒ति॒तिष्ठ॑न्त॒ इति॑ प्रति-तिष्ठ॑न्तः । य॒न्ति॒ । द्वौ । ष॒ड॒हाविति॑ षट्-अ॒हौ । भ॒व॒तः॒ । तानि॑ । द्वाद॑श । अहा॑नि । समिति॑ । प॒द्य॒न्ते॒ । द्वा॒द॒शः । वै । पुरु॑षः । द्वे इति॑ । स॒क्थ्यौ᳚ । द्वौ । बा॒हू इति॑ । आ॒त्मा । च॒ । शिरः॑ । च॒ । च॒त्वारि॑ । अङ्गा॑नि । स्तनौ᳚ । द्वा॒द॒शौ ।  \newline


\textbf{Krama Paata} \newline

ष॒ड॒हेन॒ यन्ति॑ । ष॒ड॒हेनेति॑ षट् - अ॒हेन॑ । यन्ति॑ देवच॒क्रम् । दे॒व॒च॒क्रमे॒व । दे॒व॒च॒क्रमिति॑ देव - च॒क्रम् । ए॒व स॒मारो॑हन्ति । स॒मारो॑ह॒न्त्यरि॑ष्ट्‍यै । स॒मारो॑ह॒न्तीति॑ सम् - आरो॑हन्ति । अरि॑ष्ट्‍यै॒ ते । ते स्व॒स्ति । स्व॒स्ति सम् । सम॑श्ञुवते । अ॒श्ञु॒व॒ते॒ ष॒ड॒हेन॑ । ष॒ड॒हेन॑ यन्ति । ष॒ड॒हेनेति॑ षट् - अ॒हेन॑ । 
य॒न्ति॒ षट् । षड् वै । वा ऋ॒तवः॑ । ऋ॒तव॑ ऋ॒तुषु॑ । ऋ॒तुष्वे॒व । ए॒व प्रति॑ । प्रति॑ तिष्ठन्ति । ति॒ष्ठ॒न्त्यु॒भ॒यतो᳚ज्योतिषा । उ॒भ॒यतो᳚ज्योतिषा यन्ति । उ॒भ॒यतो᳚ज्योति॒षेत्यु॑भ॒यतः॑ - ज्यो॒ति॒षा॒ । य॒न्त्यु॒भ॒यतः॑ । उ॒भ॒यत॑ ए॒व । ए॒व सु॑व॒र्गे । सु॒व॒र्गे लो॒के । सु॒व॒र्ग इति॑ सुवः - गे । लो॒के प्र॑ति॒तिष्ठ॑न्तः । प्र॒ति॒तिष्ठ॑न्तो यन्ति । प्र॒ति॒तिष्ठ॑न्त॒ इति॑ प्रति - तिष्ठ॑न्तः । य॒न्ति॒ द्वौ । द्वौ ष॑ड॒हौ । ष॒ड॒हौ भ॑वतः । ष॒ड॒हाविति॑ षट् - अ॒हौ । भ॒व॒त॒स्तानि॑ । तानि॒ द्वाद॑श । द्वाद॒शाहा॑नि । अहा॑नि॒ सम् । सम् प॑द्यन्ते । प॒द्य॒न्ते॒ द्वा॒द॒शः । द्वा॒द॒शो वै । वै पुरु॑षः । पुरु॑षो॒ द्वे । द्वे स॒क्थ्यौ᳚ । द्वे इति॒ द्वे । स॒क्थ्यौ᳚ द्वौ । द्वौ बा॒हू । बा॒हू आ॒त्मा । बा॒हू इति॑ बा॒हू । आ॒त्मा च॑ । च॒ शिरः॑ । शिर॑श्च । च॒ च॒त्वारि॑ । च॒त्वार्यङ्‍गा॑नि । अङ्‍गा॑नि॒ स्तनौ᳚ । स्तनौ᳚ द्वाद॒शौ । द्वा॒द॒शौ तत् \newline

\textbf{Jatai Paata} \newline

1. ष॒ड॒हेन॒ यन्ति॒ यन्ति॑ षड॒हेन॑ षड॒हेन॒ यन्ति॑ । \newline
2. ष॒ड॒हेनेति॑ षट् - अ॒हेन॑ । \newline
3. यन्ति॑ देवच॒क्रम् दे॑वच॒क्रं ॅयन्ति॒ यन्ति॑ देवच॒क्रम् । \newline
4. दे॒व॒च॒क्र मे॒वैव दे॑वच॒क्रम् दे॑वच॒क्र मे॒व । \newline
5. दे॒व॒च॒क्रमिति॑ देव - च॒क्रम् । \newline
6. ए॒व स॒मारो॑हन्ति स॒मारो॑ह न्त्ये॒वैव स॒मारो॑हन्ति । \newline
7. स॒मारो॑ह॒ न्त्यरि॑ष्ट्या॒ अरि॑ष्ट्यै स॒मारो॑हन्ति स॒मारो॑ह॒ न्त्यरि॑ष्ट्यै । \newline
8. स॒मारो॑ह॒न्तीति॑ सं - आरो॑हन्ति । \newline
9. अरि॑ष्ट्यै॒ ते ते ऽरि॑ष्ट्या॒ अरि॑ष्ट्यै॒ ते । \newline
10. ते स्व॒स्ति स्व॒स्ति ते ते स्व॒स्ति । \newline
11. स्व॒स्ति सꣳ सꣳ स्व॒स्ति स्व॒स्ति सम् । \newline
12. स म॑श्ञुवते ऽश्ञुवते॒ सꣳ स म॑श्ञुवते । \newline
13. अ॒श्ञु॒व॒ते॒ ष॒ड॒हेन॑ षड॒हेना᳚ श्ञुवते ऽश्ञुवते षड॒हेन॑ । \newline
14. ष॒ड॒हेन॑ यन्ति यन्ति षड॒हेन॑ षड॒हेन॑ यन्ति । \newline
15. ष॒ड॒हेनेति॑ षट् - अ॒हेन॑ । \newline
16. य॒न्ति॒ षट् थ्षड् य॑न्ति यन्ति॒ षट् । \newline
17. षड् वै वै षट् थ्षड् वै । \newline
18. वा ऋ॒तव॑ ऋ॒तवो॒ वै वा ऋ॒तवः॑ । \newline
19. ऋ॒तव॑ ऋ॒तुष् वृ॒तुष् वृ॒तव॑ ऋ॒तव॑ ऋ॒तुषु॑ । \newline
20. ऋ॒तु ष्वे॒वैव र्तुष् वृ॒तु ष्वे॒व । \newline
21. ए॒व प्रति॒ प्रत्ये॒वैव प्रति॑ । \newline
22. प्रति॑ तिष्ठन्ति तिष्ठन्ति॒ प्रति॒ प्रति॑ तिष्ठन्ति । \newline
23. ति॒ष्ठ॒ न्त्यु॒भ॒यता᳚ज्योति षोभ॒यता᳚ज्योतिषा तिष्ठन्ति तिष्ठ न्त्युभ॒यता᳚ज्योतिषा । \newline
24. उ॒भ॒यता᳚ज्योतिषा यन्ति यन्त्युभ॒यता᳚ज्योति षोभ॒यता᳚ज्योतिषा यन्ति । \newline
25. उ॒भ॒यता᳚ज्योति॒षेत्यु॑भ॒यतः॑ - ज्यो॒ति॒षा॒ । \newline
26. य॒न्त्यु॒भ॒यत॑ उभ॒यतो॑ यन्ति यन्त्युभ॒यतः॑ । \newline
27. उ॒भ॒यत॑ ए॒वै वोभ॒यत॑ उभ॒यत॑ ए॒व । \newline
28. ए॒व सु॑व॒र्गे सु॑व॒र्ग ए॒वैव सु॑व॒र्गे । \newline
29. सु॒व॒र्गे लो॒के लो॒के सु॑व॒र्गे सु॑व॒र्गे लो॒के । \newline
30. सु॒व॒र्ग इति॑ सुवः - गे । \newline
31. लो॒के प्र॑ति॒तिष्ठ॑न्तः प्रति॒तिष्ठ॑न्तो लो॒के लो॒के प्र॑ति॒तिष्ठ॑न्तः । \newline
32. प्र॒ति॒तिष्ठ॑न्तो यन्ति यन्ति प्रति॒तिष्ठ॑न्तः प्रति॒तिष्ठ॑न्तो यन्ति । \newline
33. प्र॒ति॒तिष्ठ॑न्त॒ इति॑ प्रति - तिष्ठ॑न्तः । \newline
34. य॒न्ति॒ द्वौ द्वौ य॑न्ति यन्ति॒ द्वौ । \newline
35. द्वौ ष॑ड॒हौ ष॑ड॒हौ द्वौ द्वौ ष॑ड॒हौ । \newline
36. ष॒ड॒हौ भ॑वतो भवत ष्षड॒हौ ष॑ड॒हौ भ॑वतः । \newline
37. ष॒ड॒हाविति॑ षट् - अ॒हौ । \newline
38. भ॒व॒त॒ स्तानि॒ तानि॑ भवतो भवत॒ स्तानि॑ । \newline
39. तानि॒ द्वाद॑श॒ द्वाद॑श॒ तानि॒ तानि॒ द्वाद॑श । \newline
40. द्वाद॒शा हा॒ न्यहा॑नि॒ द्वाद॑श॒ द्वाद॒शा हा॑नि । \newline
41. अहा॑नि॒ सꣳ स महा॒ न्यहा॑नि॒ सम् । \newline
42. सम् प॑द्यन्ते पद्यन्ते॒ सꣳ सम् प॑द्यन्ते । \newline
43. प॒द्य॒न्ते॒ द्वा॒द॒शो द्वा॑द॒शः प॑द्यन्ते पद्यन्ते द्वाद॒शः । \newline
44. द्वा॒द॒शो वै वै द्वा॑द॒शो द्वा॑द॒शो वै । \newline
45. वै पुरु॑षः॒ पुरु॑षो॒ वै वै पुरु॑षः । \newline
46. पुरु॑षो॒ द्वे द्वे पुरु॑षः॒ पुरु॑षो॒ द्वे । \newline
47. द्वे स॒क्थ्यौ॑ स॒क्थ्यौ᳚ द्वे द्वे स॒क्थ्यौ᳚ । \newline
48. द्वे इति॒ द्वे । \newline
49. स॒क्थ्यौ᳚ द्वौ द्वौ स॒क्थ्यौ॑ स॒क्थ्यौ᳚ द्वौ । \newline
50. द्वौ बा॒हू बा॒हू द्वौ द्वौ बा॒हू । \newline
51. बा॒हू आ॒त्मा ऽऽत्मा बा॒हू बा॒हू आ॒त्मा । \newline
52. बा॒हू इति॑ बा॒हू । \newline
53. आ॒त्मा च॑ चा॒त्मा ऽऽत्मा च॑ । \newline
54. च॒ शिरः॒ शिर॑ श्च च॒ शिरः॑ । \newline
55. शिर॑ श्च च॒ शिरः॒ शिर॑ श्च । \newline
56. च॒ च॒त्वारि॑ च॒त्वारि॑ च च च॒त्वारि॑ । \newline
57. च॒त्वार्यङ्गा॒ न्यङ्गा॑नि च॒त्वारि॑ च॒त्वार्यङ्गा॑नि । \newline
58. अङ्गा॑नि॒ स्तनौ॒ स्तना॒ वङ्गा॒ न्यङ्गा॑नि॒ स्तनौ᳚ । \newline
59. स्तनौ᳚ द्वाद॒शौ द्वा॑द॒शौ स्तनौ॒ स्तनौ᳚ द्वाद॒शौ । \newline
60. द्वा॒द॒शौ तत् तद् द्वा॑द॒शौ द्वा॑द॒शौ तत् । \newline

\textbf{Ghana Paata } \newline

1. ष॒ड॒हेन॒ यन्ति॒ यन्ति॑ षड॒हेन॑ षड॒हेन॒ यन्ति॑ देवच॒क्रम् दे॑वच॒क्रं ॅयन्ति॑ षड॒हेन॑ षड॒हेन॒ यन्ति॑ देवच॒क्रम् । \newline
2. ष॒ड॒हेनेति॑ षट् - अ॒हेन॑ । \newline
3. यन्ति॑ देवच॒क्रम् दे॑वच॒क्रं ॅयन्ति॒ यन्ति॑ देवच॒क्र मे॒वैव दे॑वच॒क्रं ॅयन्ति॒ यन्ति॑ देवच॒क्र मे॒व । \newline
4. दे॒व॒च॒क्र मे॒वैव दे॑वच॒क्रम् दे॑वच॒क्र मे॒व स॒मारो॑हन्ति स॒मारो॑ह न्त्ये॒व दे॑वच॒क्रम् दे॑वच॒क्र मे॒व स॒मारो॑हन्ति । \newline
5. दे॒व॒च॒क्रमिति॑ देव - च॒क्रम् । \newline
6. ए॒व स॒मारो॑हन्ति स॒मारो॑ह न्त्ये॒वैव स॒मारो॑ह॒ न्त्यरि॑ष्ट्या॒ अरि॑ष्ट्यै स॒मारो॑ह न्त्ये॒वैव स॒मारो॑ह॒ न्त्यरि॑ष्ट्यै । \newline
7. स॒मारो॑ह॒ न्त्यरि॑ष्ट्या॒ अरि॑ष्ट्यै स॒मारो॑हन्ति स॒मारो॑ह॒ न्त्यरि॑ष्ट्यै॒ ते ते ऽरि॑ष्ट्यै स॒मारो॑हन्ति स॒मारो॑ह॒ न्त्यरि॑ष्ट्यै॒ ते । \newline
8. स॒मारो॑ह॒न्तीति॑ सं - आरो॑हन्ति । \newline
9. अरि॑ष्ट्यै॒ ते ते ऽरि॑ष्ट्या॒ अरि॑ष्ट्यै॒ ते स्व॒स्ति स्व॒स्ति ते ऽरि॑ष्ट्या॒ अरि॑ष्ट्यै॒ ते स्व॒स्ति । \newline
10. ते स्व॒स्ति स्व॒स्ति ते ते स्व॒स्ति सꣳ सꣳ स्व॒स्ति ते ते स्व॒स्ति सम् । \newline
11. स्व॒स्ति सꣳ सꣳ स्व॒स्ति स्व॒स्ति स म॑श्ञुवते ऽश्ञुवते॒ सꣳ स्व॒स्ति स्व॒स्ति स म॑श्ञुवते । \newline
12. स म॑श्ञुवते ऽश्ञुवते॒ सꣳ स म॑श्ञुवते षड॒हेन॑ षड॒हेना᳚ श्ञुवते॒ सꣳ स म॑श्ञुवते षड॒हेन॑ । \newline
13. अ॒श्ञु॒व॒ते॒ ष॒ड॒हेन॑ षड॒हेना᳚ श्ञुवते ऽश्ञुवते षड॒हेन॑ यन्ति यन्ति षड॒हेना᳚ श्ञुवते ऽश्ञुवते षड॒हेन॑ यन्ति । \newline
14. ष॒ड॒हेन॑ यन्ति यन्ति षड॒हेन॑ षड॒हेन॑ यन्ति॒ षट् थ्षड् य॑न्ति षड॒हेन॑ षड॒हेन॑ यन्ति॒ षट् । \newline
15. ष॒ड॒हेनेति॑ षट् - अ॒हेन॑ । \newline
16. य॒न्ति॒ षट् थ्षड् य॑न्ति यन्ति॒ षड् वै वै षड् य॑न्ति यन्ति॒ षड् वै । \newline
17. षड् वै वै षट् थ्षड् वा ऋ॒तव॑ ऋ॒तवो॒ वै षट् थ्षड् वा ऋ॒तवः॑ । \newline
18. वा ऋ॒तव॑ ऋ॒तवो॒ वै वा ऋ॒तव॑ ऋ॒तुष् वृ॒तुष् वृ॒तवो॒ वै वा ऋ॒तव॑ ऋ॒तुषु॑ । \newline
19. ऋ॒तव॑ ऋ॒तुष् वृ॒तुष् वृ॒तव॑ ऋ॒तव॑ ऋ॒तुष् वे॒वैव र्‌तुष् वृ॒तव॑ ऋ॒तव॑ ऋ॒तुष्वे॒व । \newline
20. ऋ॒तुष् वे॒वैव र्‌तुष् वृ॒तुष् वे॒व प्रति॒ प्रत्ये॒व र्‌तुष् वृ॒तुष् वे॒व प्रति॑ । \newline
21. ए॒व प्रति॒ प्रत्ये॒ वैव प्रति॑ तिष्ठन्ति तिष्ठन्ति॒ प्रत्ये॒ वैव प्रति॑ तिष्ठन्ति । \newline
22. प्रति॑ तिष्ठन्ति तिष्ठन्ति॒ प्रति॒ प्रति॑ तिष्ठ न्त्युभ॒यता᳚ज्योति षोभ॒यता᳚ज्योतिषा तिष्ठन्ति॒ प्रति॒ प्रति॑ तिष्ठ न्त्युभ॒यता᳚ज्योतिषा । \newline
23. ति॒ष्ठ॒ न्त्यु॒भ॒यता᳚ज्योति षोभ॒यता᳚ज्योतिषा तिष्ठन्ति तिष्ठ न्त्युभ॒यता᳚ज्योतिषा यन्ति यन्त्युभ॒यता᳚ज्योतिषा तिष्ठन्ति तिष्ठ न्त्युभ॒यता᳚ज्योतिषा यन्ति । \newline
24. उ॒भ॒यता᳚ज्योतिषा यन्ति यन्त्युभ॒यता᳚ज्योति षोभ॒यता᳚ज्योतिषा यन्त्युभ॒यत॑ उभ॒यतो॑ यन्त्युभ॒यता᳚ज्योति
षोभ॒यता᳚ज्योतिषा यन्त्युभ॒यतः॑ । \newline
25. उ॒भ॒यता᳚ज्योति॒षेत्यु॑भ॒यतः॑ - ज्यो॒ति॒षा॒ । \newline
26. य॒न्त्यु॒भ॒यत॑ उभ॒यतो॑ यन्ति यन्त्युभ॒यत॑ ए॒वै वोभ॒यतो॑ यन्ति यन्त्युभ॒यत॑ ए॒व । \newline
27. उ॒भ॒यत॑ ए॒वै वोभ॒यत॑ उभ॒यत॑ ए॒व सु॑व॒र्गे सु॑व॒र्ग ए॒वोभ॒यत॑ उभ॒यत॑ ए॒व सु॑व॒र्गे । \newline
28. ए॒व सु॑व॒र्गे सु॑व॒र्ग ए॒वैव सु॑व॒र्गे लो॒के लो॒के सु॑व॒र्ग ए॒वैव सु॑व॒र्गे लो॒के । \newline
29. सु॒व॒र्गे लो॒के लो॒के सु॑व॒र्गे सु॑व॒र्गे लो॒के प्र॑ति॒तिष्ठ॑न्तः प्रति॒तिष्ठ॑न्तो लो॒के सु॑व॒र्गे सु॑व॒र्गे लो॒के प्र॑ति॒तिष्ठ॑न्तः । \newline
30. सु॒व॒र्ग इति॑ सुवः - गे । \newline
31. लो॒के प्र॑ति॒तिष्ठ॑न्तः प्रति॒तिष्ठ॑न्तो लो॒के लो॒के प्र॑ति॒तिष्ठ॑न्तो यन्ति यन्ति प्रति॒तिष्ठ॑न्तो लो॒के लो॒के प्र॑ति॒तिष्ठ॑न्तो यन्ति । \newline
32. प्र॒ति॒तिष्ठ॑न्तो यन्ति यन्ति प्रति॒तिष्ठ॑न्तः प्रति॒तिष्ठ॑न्तो यन्ति॒ द्वौ द्वौ य॑न्ति प्रति॒तिष्ठ॑न्तः प्रति॒तिष्ठ॑न्तो यन्ति॒ द्वौ । \newline
33. प्र॒ति॒तिष्ठ॑न्त॒ इति॑ प्रति - तिष्ठ॑न्तः । \newline
34. य॒न्ति॒ द्वौ द्वौ य॑न्ति यन्ति॒ द्वौ ष॑ड॒हौ ष॑ड॒हौ द्वौ य॑न्ति यन्ति॒ द्वौ ष॑ड॒हौ । \newline
35. द्वौ ष॑ड॒हौ ष॑ड॒हौ द्वौ द्वौ ष॑ड॒हौ भ॑वतो भवत ष्षड॒हौ द्वौ द्वौ ष॑ड॒हौ भ॑वतः । \newline
36. ष॒ड॒हौ भ॑वतो भवत ष्षड॒हौ ष॑ड॒हौ भ॑वत॒ स्तानि॒ तानि॑ भवत ष्षड॒हौ ष॑ड॒हौ भ॑वत॒ स्तानि॑ । \newline
37. ष॒ड॒हाविति॑ षट् - अ॒हौ । \newline
38. भ॒व॒त॒ स्तानि॒ तानि॑ भवतो भवत॒ स्तानि॒ द्वाद॑श॒ द्वाद॑श॒ तानि॑ भवतो भवत॒ स्तानि॒ द्वाद॑श । \newline
39. तानि॒ द्वाद॑श॒ द्वाद॑श॒ तानि॒ तानि॒ द्वाद॒शा हा॒ न्यहा॑नि॒ द्वाद॑श॒ तानि॒ तानि॒ द्वाद॒शा हा॑नि । \newline
40. द्वाद॒शा हा॒ न्यहा॑नि॒ द्वाद॑श॒ द्वाद॒शा हा॑नि॒ सꣳ समहा॑नि॒ द्वाद॑श॒ द्वाद॒शा हा॑नि॒ सम् । \newline
41. अहा॑नि॒ सꣳ समहा॒ न्यहा॑नि॒ सम् प॑द्यन्ते पद्यन्ते॒ समहा॒ न्यहा॑नि॒ सम् प॑द्यन्ते । \newline
42. सम् प॑द्यन्ते पद्यन्ते॒ सꣳ सम् प॑द्यन्ते द्वाद॒शो द्वा॑द॒शः प॑द्यन्ते॒ सꣳ सम् प॑द्यन्ते द्वाद॒शः । \newline
43. प॒द्य॒न्ते॒ द्वा॒द॒शो द्वा॑द॒शः प॑द्यन्ते पद्यन्ते द्वाद॒शो वै वै द्वा॑द॒शः प॑द्यन्ते पद्यन्ते द्वाद॒शो वै । \newline
44. द्वा॒द॒शो वै वै द्वा॑द॒शो द्वा॑द॒शो वै पुरु॑षः॒ पुरु॑षो॒ वै द्वा॑द॒शो द्वा॑द॒शो वै पुरु॑षः । \newline
45. वै पुरु॑षः॒ पुरु॑षो॒ वै वै पुरु॑षो॒ द्वे द्वे पुरु॑षो॒ वै वै पुरु॑षो॒ द्वे । \newline
46. पुरु॑षो॒ द्वे द्वे पुरु॑षः॒ पुरु॑षो॒ द्वे स॒क्थ्यौ॑ स॒क्थ्यौ᳚ द्वे पुरु॑षः॒ पुरु॑षो॒ द्वे स॒क्थ्यौ᳚ । \newline
47. द्वे स॒क्थ्यौ॑ स॒क्थ्यौ᳚ द्वे द्वे स॒क्थ्यौ᳚ द्वौ द्वौ स॒क्थ्यौ᳚ द्वे द्वे स॒क्थ्यौ᳚ द्वौ । \newline
48. द्वे इति॒ द्वे । \newline
49. स॒क्थ्यौ᳚ द्वौ द्वौ स॒क्थ्यौ॑ स॒क्थ्यौ᳚ द्वौ बा॒हू बा॒हू द्वौ स॒क्थ्यौ॑ स॒क्थ्यौ᳚ द्वौ बा॒हू । \newline
50. द्वौ बा॒हू बा॒हू द्वौ द्वौ बा॒हू आ॒त्मा ऽऽत्मा बा॒हू द्वौ द्वौ बा॒हू आ॒त्मा । \newline
51. बा॒हू आ॒त्मा ऽऽत्मा बा॒हू बा॒हू आ॒त्मा च॑ चा॒त्मा बा॒हू बा॒हू आ॒त्मा च॑ । \newline
52. बा॒हू इति॑ बा॒हू । \newline
53. आ॒त्मा च॑ चा॒त्मा ऽऽत्मा च॒ शिरः॒ शिर॑ श्चा॒त्मा ऽऽत्मा च॒ शिरः॑ । \newline
54. च॒ शिरः॒ शिर॑ श्च च॒ शिर॑ श्च च॒ शिर॑श्च च॒ शिर॑ श्च । \newline
55. शिर॑ श्च च॒ शिरः॒ शिर॑ श्च च॒त्वारि॑ च॒त्वारि॑ च॒ शिरः॒ शिर॑ श्च च॒त्वारि॑ । \newline
56. च॒ च॒त्वारि॑ च॒त्वारि॑ च च च॒त्वार्यङ्गा॒ न्यङ्गा॑नि च॒त्वारि॑ च च च॒त्वार्यङ्गा॑नि । \newline
57. च॒त्वार्यङ्गा॒ न्यङ्गा॑नि च॒त्वारि॑ च॒त्वार्यङ्गा॑नि॒ स्तनौ॒ स्तना॒ वङ्गा॑नि च॒त्वारि॑ च॒त्वार्यङ्गा॑नि॒ स्तनौ᳚ । \newline
58. अङ्गा॑नि॒ स्तनौ॒ स्तना॒ वङ्गा॒ न्यङ्गा॑नि॒ स्तनौ᳚ द्वाद॒शौ द्वा॑द॒शौ स्तना॒ वङ्गा॒ न्यङ्गा॑नि॒ स्तनौ᳚ द्वाद॒शौ । \newline
59. स्तनौ᳚ द्वाद॒शौ द्वा॑द॒शौ स्तनौ॒ स्तनौ᳚ द्वाद॒शौ तत् तद् द्वा॑द॒शौ स्तनौ॒ स्तनौ᳚ द्वाद॒शौ तत् । \newline
60. द्वा॒द॒शौ तत् तद् द्वा॑द॒शौ द्वा॑द॒शौ तत् पुरु॑ष॒म् पुरु॑ष॒म् तद् द्वा॑द॒शौ द्वा॑द॒शौ तत् पुरु॑षम् । \newline
\pagebreak
\markright{ TS 7.4.11.4  \hfill https://www.vedavms.in \hfill}

\section{ TS 7.4.11.4 }

\textbf{TS 7.4.11.4 } \newline
\textbf{Samhita Paata} \newline

तत् पुरु॑ष॒मनु॑ प॒र्याव॑र्तन्ते॒ त्रयः॑ षड॒हा भ॑वन्ति॒ तान्य॒ष्टाद॒शाहा॑नि॒ सं प॑द्यन्ते॒ नवा॒न्यानि॒ नवा॒न्यानि॒ नव॒ वै पुरु॑षे प्रा॒णास्तत् प्रा॒णाननु॑ प॒र्याव॑र्तन्ते च॒त्वारः॑ षड॒हा भ॑वन्ति॒ तानि॒ चतु॑र्विꣳशति॒रहा॑नि॒ सं प॑द्यन्ते॒ चतु॑र्विꣳशतिरर्द्धमा॒साः सं॑ॅवथ्स॒रस्तथ् सं॑ॅवथ्स॒रमनु॑ प॒र्याव॑र्त॒न्ते ऽप्र॑तिष्ठितः संॅवथ्स॒र इति॒ खलु॒ वा आ॑हु॒र्वर्.षी॑यान् प्रति॒ष्ठाया॒ इत्ये॒ताव॒द्वै ( ) सं॑ॅवथ्स॒रस्य॒ ब्राह्म॑णं॒ ॅयाव॑न्मा॒सो मा॒सिमा᳚स्ये॒व प्र॑ति॒तिष्ठ॑न्तो यन्ति ॥ \newline

\textbf{Pada Paata} \newline

तत् । पुरु॑षम् । अन्विति॑ । प॒र्याव॑र्तन्त॒ इति॑ परि - आव॑र्तन्ते । त्रयः॑ । ष॒ड॒हा इति॑ षट् - अ॒हाः । भ॒व॒न्ति॒ । तानि॑ । अ॒ष्टाद॒शेत्य॒ष्टा - द॒श॒ । अहा॑नि । समिति॑ । प॒द्य॒न्ते॒ । नव॑ । अ॒न्यानि॑ । नव॑ । अ॒न्यानि॑ । नव॑ । वै । पुरु॑षे । प्रा॒णा इति॑ प्र - अ॒नाः । तत् । प्रा॒णानिति॑ प्र-अ॒नान् । अन्विति॑ । प॒र्याव॑र्तन्त॒ इति॑ परि - आव॑र्तन्ते । च॒त्वारः॑ । ष॒ड॒हा इति॑ षट्- अ॒हाः । भ॒व॒न्ति॒ । तानि॑ । चतु॑र्विꣳशति॒रिति॒ चतुः॑-विꣳ॒॒श॒तिः॒ । अहा॑नि । समिति॑ । प॒द्य॒न्ते॒ । चतु॑र्विꣳशति॒रिति॒ चतुः॑ - विꣳ॒॒श॒तिः॒ । अ॒द्‌र्ध॒मा॒सा इत्यद्‌र्ध - मा॒साः । सं॒ॅव॒थ्स॒र इति॑ सं - व॒थ्स॒रः । तत् । सं॒ॅव॒थ्स॒रमिति॑ सं - व॒थ्स॒रम् । अन्विति॑ । प॒र्याव॑र्तन्त॒ इति॑ परि - आव॑र्तन्ते । अप्र॑तिष्ठित॒ इत्यप्र॑ति - स्थि॒तः॒ । सं॒ॅव॒थ्स॒र इति॑ सं - व॒थ्स॒रः । इति॑ । खलु॑ । वै । आ॒हुः॒ । वर्.षी॑यान् । प्र॒ति॒ष्ठाया॒ इति॑ प्रति - स्थायाः᳚ । इति॑ । ए॒ताव॑त् । वै ( ) । सं॒ॅव॒थ्स॒रस्येति॑ सं - व॒थ्स॒रस्य॑ । ब्राह्म॑णम् । याव॑त् । मा॒सः । मा॒सिमा॒सीति॑ मा॒सि - मा॒सि॒ । ए॒व । प्र॒ति॒तिष्ठ॑न्त॒ इति॑ प्रति - तिष्ठ॑न्तः । य॒न्ति॒ ॥  \newline


\textbf{Krama Paata} \newline

तत् पुरु॑षम् । पुरु॑ष॒मनु॑ । अनु॑ प॒र्याव॑र्तन्ते । प॒र्याव॑र्तन्ते॒ त्रयः॑ । प॒र्याव॑र्तन्त॒ इति॑ परि - आव॑र्तन्ते । त्रयः॑ षड॒हाः । ष॒ड॒हा भ॑वन्ति । ष॒ड॒हा इति॑ षट् - अ॒हाः । भ॒व॒न्ति॒ तानि॑ । तान्य॒ष्टाद॑श । अ॒ष्टाद॒शाहा॑नि । अ॒ष्टाद॒शेत्य॒ष्टा - द॒श॒ । अहा॑नि॒ सम् । सम् प॑द्यन्ते । प॒द्य॒न्ते॒ नव॑ । नवा॒न्यानि॑ । अ॒न्यानि॒ नव॑ । नवा॒न्यानि॑ । अ॒न्यानि॒ नव॑ । नव॒ वै । वै पुरु॑षे । पुरु॑षे प्रा॒णाः । प्रा॒णास्तत् । प्रा॒णा इति॑ प्र - अ॒नाः । तत् प्रा॒णान् । प्रा॒णाननु॑ । प्रा॒णानिति॑ प्र - अ॒नान् । अनु॑ प॒र्याव॑र्तन्ते । प॒र्याव॑र्तन्ते च॒त्वारः॑ । प॒र्याव॑र्तन्त॒ इति॑ परि - आव॑र्तन्ते । च॒त्वारः॑ षड॒हाः । ष॒ड॒हा भ॑वन्ति । ष॒ड॒हा इति॑ षट् - अ॒हाः । भ॒व॒न्ति॒ तानि॑ । तानि॒ चतु॑र्विꣳशतिः । चतु॑र्विꣳशति॒रहा॑नि । चतु॑र्विꣳशति॒रिति॒ चतुः॑ - विꣳ॒॒श॒तिः॒ । अहा॑नि॒ सम् । सम् प॑द्यन्ते । प॒द्य॒न्ते॒ चतु॑र्विꣳशतिः । चतु॑र्विꣳशतिरर्द्धमा॒साः । चतु॑र्विꣳशति॒रिति॒ चतुः॑ - विꣳ॒॒श॒तिः॒ । अ॒र्द्ध॒मा॒साः स॑म्ॅवथ्स॒रः । अ॒र्द्ध॒मा॒सा इत्य॑र्द्ध - मा॒साः । स॒म्ॅव॒थ्स॒रस्तत् । स॒म्ॅव॒थ्स॒र इति॑ सम् - व॒थ्स॒रः । तथ् स॑म्ॅवथ्स॒रम् । स॒म्ॅव॒थ्स॒रमनु॑ । स॒म्ॅव॒थ्स॒रमिति॑ सम् - व॒थ्स॒रम् । अनु॑ प॒र्याव॑र्तन्ते । प॒र्याव॑र्त॒न्तेऽप्र॑तिष्ठितः । प॒र्याव॑र्तन्त॒ इति॑ परि - आव॑र्तन्ते । अप्र॑तिष्ठितः सम्ॅवथ्स॒रः । अप्र॑तिष्ठित॒ इत्यप्र॑ति - स्थि॒तः॒ । स॒म्ॅव॒थ्स॒र इति॑ । स॒म्ॅव॒थ्स॒र इति॑ सम् - व॒थ्स॒रः । इति॒ खलु॑ । खलु॒ वै । वा आ॑हुः । आ॒हु॒र् वर्.षी॑यान् । वर्.षी॑यान् प्रति॒ष्ठायाः᳚ । प्र॒ति॒ष्ठाया॒ इति॑ । प्र॒ति॒ष्ठाया॒ इति॑ प्रति - स्थायाः᳚ । इत्ये॒ताव॑त् । ए॒ताव॒द् वै ( ) । वै स॑म्ॅवथ्स॒रस्य॑ । स॒म्ॅव॒थ्स॒रस्य॒ ब्राह्म॑णम् । स॒म्ॅव॒थ्स॒रस्येति॑ सम् - व॒थ्स॒रस्य॑ । ब्राह्म॑ण॒म् ॅयाव॑त् । याव॑न् मा॒सः । मा॒सो मा॒सिमा॑सि । मा॒सिमा᳚स्ये॒व । मा॒सिमा॒सीति॑ मा॒सि - मा॒सि॒ । ए॒व प्र॑ति॒तिष्ठ॑न्तः । प्र॒ति॒तिष्ठ॑न्तो यन्ति । प्र॒ति॒तिष्ठ॑न्त॒ इति॑ प्रति - तिष्ठ॑न्तः । य॒न्तीति॑ यन्ति । \newline

\textbf{Jatai Paata} \newline

1. तत् पुरु॑ष॒म् पुरु॑ष॒म् तत् तत् पुरु॑षम् । \newline
2. पुरु॑ष॒ मन्वनु॒ पुरु॑ष॒म् पुरु॑ष॒ मनु॑ । \newline
3. अनु॑ प॒र्याव॑र्तन्ते प॒र्याव॑र्त॒न्ते ऽन्वनु॑ प॒र्याव॑र्तन्ते । \newline
4. प॒र्याव॑र्तन्ते॒ त्रय॒ स्त्रयः॑ प॒र्याव॑र्तन्ते प॒र्याव॑र्तन्ते॒ त्रयः॑ । \newline
5. प॒र्याव॑र्तन्त॒ इति॑ परि - आव॑र्तन्ते । \newline
6. त्रय॑ ष्षड॒हा ष्ष॑ड॒हा स्त्रय॒ स्त्रय॑ ष्षड॒हाः । \newline
7. ष॒ड॒हा भ॑वन्ति भवन्ति षड॒हा ष्ष॑ड॒हा भ॑वन्ति । \newline
8. ष॒ड॒हा इति॑ षट् - अ॒हाः । \newline
9. भ॒व॒न्ति॒ तानि॒ तानि॑ भवन्ति भवन्ति॒ तानि॑ । \newline
10. तान्य॒ष्टाद॑शा॒ ष्टाद॑श॒ तानि॒ तान्य॒ष्टाद॑श । \newline
11. अ॒ष्टाद॒शा हा॒ न्यहा᳚ न्य॒ष्टाद॑शा॒ ष्टाद॒शा हा॑नि । \newline
12. अ॒ष्टाद॒शेत्य॒ष्टा - द॒श॒ । \newline
13. अहा॑नि॒ सꣳ स महा॒ न्यहा॑नि॒ सम् । \newline
14. सम् प॑द्यन्ते पद्यन्ते॒ सꣳ सम् प॑द्यन्ते । \newline
15. प॒द्य॒न्ते॒ नव॒ नव॑ पद्यन्ते पद्यन्ते॒ नव॑ । \newline
16. नवा॒ न्यान्य॒ न्यानि॒ नव॒ नवा॒ न्यानि॑ । \newline
17. अ॒न्यानि॒ नव॒ नवा॒न्या न्य॒न्यानि॒ नव॑ । \newline
18. नवा॒ न्यान्य॒ न्यानि॒ नव॒ नवा॒ न्यानि॑ । \newline
19. अ॒न्यानि॒ नव॒ नवा॒न्या न्य॒न्यानि॒ नव॑ । \newline
20. नव॒ वै वै नव॒ नव॒ वै । \newline
21. वै पुरु॑षे॒ पुरु॑षे॒ वै वै पुरु॑षे । \newline
22. पुरु॑षे प्रा॒णाः प्रा॒णाः पुरु॑षे॒ पुरु॑षे प्रा॒णाः । \newline
23. प्रा॒णा स्तत् तत् प्रा॒णाः प्रा॒णा स्तत् । \newline
24. प्रा॒णा इति॑ प्र - अ॒नाः । \newline
25. तत् प्रा॒णान् प्रा॒णाꣳ स्तत् तत् प्रा॒णान् । \newline
26. प्रा॒णानन् वनु॑ प्रा॒णान् प्रा॒णाननु॑ । \newline
27. प्रा॒णानिति॑ प्र - अ॒नान् । \newline
28. अनु॑ प॒र्याव॑र्तन्ते प॒र्याव॑र्त॒न्ते ऽन्वनु॑ प॒र्याव॑र्तन्ते । \newline
29. प॒र्याव॑र्तन्ते च॒त्वार॑ श्च॒त्वारः॑ प॒र्याव॑र्तन्ते प॒र्याव॑र्तन्ते च॒त्वारः॑ । \newline
30. प॒र्याव॑र्तन्त॒ इति॑ परि - आव॑र्तन्ते । \newline
31. च॒त्वार॑ ष्षड॒हा ष्ष॑ड॒हा श्च॒त्वार॑ श्च॒त्वार॑ ष्षड॒हाः । \newline
32. ष॒ड॒हा भ॑वन्ति भवन्ति षड॒हा ष्ष॑ड॒हा भ॑वन्ति । \newline
33. ष॒ड॒हा इति॑ षट् - अ॒हाः । \newline
34. भ॒व॒न्ति॒ तानि॒ तानि॑ भवन्ति भवन्ति॒ तानि॑ । \newline
35. तानि॒ चतु॑र्विꣳशति॒ श्चतु॑र्विꣳशति॒ स्तानि॒ तानि॒ चतु॑र्विꣳशतिः । \newline
36. चतु॑र्विꣳशति॒ रहा॒ न्यहा॑नि॒ चतु॑र्विꣳशति॒ श्चतु॑र्विꣳशति॒ रहा॑नि । \newline
37. चतु॑र्विꣳशति॒रिति॒ चतुः॑ - विꣳ॒॒श॒तिः॒ । \newline
38. अहा॑नि॒ सꣳ स महा॒ न्यहा॑नि॒ सम् । \newline
39. सम् प॑द्यन्ते पद्यन्ते॒ सꣳ सम् प॑द्यन्ते । \newline
40. प॒द्य॒न्ते॒ चतु॑र्विꣳशति॒ श्चतु॑र्विꣳशतिः पद्यन्ते पद्यन्ते॒ चतु॑र्विꣳशतिः । \newline
41. चतु॑र्विꣳशति रर्द्धमा॒सा अ॑र्द्धमा॒सा श्चतु॑र्विꣳशति॒ श्चतु॑र्विꣳशति रर्द्धमा॒साः । \newline
42. चतु॑र्विꣳशति॒रिति॒ चतुः॑ - विꣳ॒॒श॒तिः॒ । \newline
43. अ॒र्द्ध॒मा॒साः सं॑ॅवथ्स॒रः सं॑ॅवथ्स॒रो᳚ ऽर्द्धमा॒सा अ॑र्द्धमा॒साः सं॑ॅवथ्स॒रः । \newline
44. अ॒र्द्ध॒मा॒सा इत्य॑र्द्ध - मा॒साः । \newline
45. सं॒ॅव॒थ्स॒र स्तत् तथ् सं॑ॅवथ्स॒रः सं॑ॅवथ्स॒र स्तत् । \newline
46. सं॒ॅव॒थ्स॒र इति॑ सं - व॒थ्स॒रः । \newline
47. तथ् सं॑ॅवथ्स॒रꣳ सं॑ॅवथ्स॒रम् तत् तथ् सं॑ॅवथ्स॒रम् । \newline
48. सं॒ॅव॒थ्स॒र मन्वनु॑ संॅवथ्स॒रꣳ सं॑ॅवथ्स॒र मनु॑ । \newline
49. सं॒ॅव॒थ्स॒रमिति॑ सं - व॒थ्स॒रम् । \newline
50. अनु॑ प॒र्याव॑र्तन्ते प॒र्याव॑र्त॒न्ते ऽन्वनु॑ प॒र्याव॑र्तन्ते । \newline
51. प॒र्याव॑र्त॒न्ते ऽप्र॑तिष्ठि॒तो ऽप्र॑तिष्ठितः प॒र्याव॑र्तन्ते प॒र्याव॑र्त॒न्ते ऽप्र॑तिष्ठितः । \newline
52. प॒र्याव॑र्तन्त॒ इति॑ परि - आव॑र्तन्ते । \newline
53. अप्र॑तिष्ठितः संॅवथ्स॒रः सं॑ॅवथ्स॒रो ऽप्र॑तिष्ठि॒तो ऽप्र॑तिष्ठितः संॅवथ्स॒रः । \newline
54. अप्र॑तिष्ठित॒ इत्यप्र॑ति - स्थि॒तः॒ । \newline
55. सं॒ॅव॒थ्स॒र इतीति॑ संॅवथ्स॒रः सं॑ॅवथ्स॒र इति॑ । \newline
56. सं॒ॅव॒थ्स॒र इति॑ सं - व॒थ्स॒रः । \newline
57. इति॒ खलु॒ खल्वितीति॒ खलु॑ । \newline
58. खलु॒ वै वै खलु॒ खलु॒ वै । \newline
59. वा आ॑हु राहु॒र् वै वा आ॑हुः । \newline
60. आ॒हु॒र् वर्.षी॑या॒न्॒. वर्.षी॑या नाहु राहु॒र् वर्.षी॑यान् । \newline
61. वर्.षी॑यान् प्रति॒ष्ठायाः᳚ प्रति॒ष्ठाया॒ वर्.षी॑या॒न्॒. वर्.षी॑यान् प्रति॒ष्ठायाः᳚ । \newline
62. प्र॒ति॒ष्ठाया॒ इतीति॑ प्रति॒ष्ठायाः᳚ प्रति॒ष्ठाया॒ इति॑ । \newline
63. प्र॒ति॒ष्ठाया॒ इति॑ प्रति - स्थायाः᳚ । \newline
64. इत्ये॒ताव॑ दे॒ताव॒ दिती त्ये॒ताव॑त् । \newline
65. ए॒ताव॒द् वै वा ए॒ताव॑ दे॒ताव॒द् वै । \newline
66. वै सं॑ॅवथ्स॒रस्य॑ संॅवथ्स॒रस्य॒ वै वै सं॑ॅवथ्स॒रस्य॑ । \newline
67. सं॒ॅव॒थ्स॒रस्य॒ ब्राह्म॑ण॒म् ब्राह्म॑णꣳ संॅवथ्स॒रस्य॑ संॅवथ्स॒रस्य॒ ब्राह्म॑णम् । \newline
68. सं॒ॅव॒थ्स॒रस्येति॑ सं - व॒थ्स॒रस्य॑ । \newline
69. ब्राह्म॑णं॒ ॅयाव॒द् याव॒द् ब्राह्म॑ण॒म् ब्राह्म॑णं॒ ॅयाव॑त् । \newline
70. याव॑न् मा॒सो मा॒सो याव॒द् याव॑न् मा॒सः । \newline
71. मा॒सो मा॒सिमा॑सि मा॒सिमा॑सि मा॒सो मा॒सो मा॒सिमा॑सि । \newline
72. मा॒सिमा᳚ स्ये॒वैव मा॒सिमा॑सि मा॒सिमा᳚ स्ये॒व । \newline
73. मा॒सिमा॒सीति॑ मा॒सि - मा॒सि॒ । \newline
74. ए॒व प्र॑ति॒तिष्ठ॑न्तः प्रति॒तिष्ठ॑न्त ए॒वैव प्र॑ति॒तिष्ठ॑न्तः । \newline
75. प्र॒ति॒तिष्ठ॑न्तो यन्ति यन्ति प्रति॒तिष्ठ॑न्तः प्रति॒तिष्ठ॑न्तो यन्ति । \newline
76. प्र॒ति॒तिष्ठ॑न्त॒ इति॑ प्रति - तिष्ठ॑न्तः । \newline
77. य॒न्तीति॑ यन्ति । \newline

\textbf{Ghana Paata } \newline

1. तत् पुरु॑ष॒म् पुरु॑ष॒म् तत् तत् पुरु॑ष॒ मन्वनु॒ पुरु॑ष॒म् तत् तत् पुरु॑ष॒ मनु॑ । \newline
2. पुरु॑ष॒ मन्वनु॒ पुरु॑ष॒म् पुरु॑ष॒ मनु॑ प॒र्याव॑र्तन्ते प॒र्याव॑र्त॒न्ते ऽनु॒ पुरु॑ष॒म् पुरु॑ष॒ मनु॑ प॒र्याव॑र्तन्ते । \newline
3. अनु॑ प॒र्याव॑र्तन्ते प॒र्याव॑र्त॒न्ते ऽन्वनु॑ प॒र्याव॑र्तन्ते॒ त्रय॒ स्त्रयः॑ प॒र्याव॑र्त॒न्ते ऽन्वनु॑ प॒र्याव॑र्तन्ते॒ त्रयः॑ । \newline
4. प॒र्याव॑र्तन्ते॒ त्रय॒ स्त्रयः॑ प॒र्याव॑र्तन्ते प॒र्याव॑र्तन्ते॒ त्रय॑ ष्षड॒हा ष्ष॑ड॒हा स्त्रयः॑ प॒र्याव॑र्तन्ते प॒र्याव॑र्तन्ते॒ त्रय॑ ष्षड॒हाः । \newline
5. प॒र्याव॑र्तन्त॒ इति॑ परि - आव॑र्तन्ते । \newline
6. त्रय॑ ष्षड॒हा ष्ष॑ड॒हा स्त्रय॒ स्त्रय॑ ष्षड॒हा भ॑वन्ति भवन्ति षड॒हा स्त्रय॒ स्त्रय॑ ष्षड॒हा भ॑वन्ति । \newline
7. ष॒ड॒हा भ॑वन्ति भवन्ति षड॒हा ष्ष॑ड॒हा भ॑वन्ति॒ तानि॒ तानि॑ भवन्ति षड॒हा ष्ष॑ड॒हा भ॑वन्ति॒ तानि॑ । \newline
8. ष॒ड॒हा इति॑ षट् - अ॒हाः । \newline
9. भ॒व॒न्ति॒ तानि॒ तानि॑ भवन्ति भवन्ति॒ तान्य॒ष्टाद॑शा॒ ष्टाद॑श॒ तानि॑ भवन्ति भवन्ति॒ तान्य॒ष्टाद॑श । \newline
10. तान्य॒ष्टाद॑शा॒ ष्टाद॑श॒ तानि॒ तान्य॒ष्टाद॒शा हा॒न्य हा᳚न्य॒ष्टाद॑श॒ तानि॒ तान्य॒ष्टाद॒शा हा॑नि । \newline
11. अ॒ष्टाद॒शा हा॒ न्यहा᳚ न्य॒ष्टाद॑शा॒ ष्टाद॒शाहा॑नि॒ सꣳ स महा᳚ न्य॒ष्टाद॑शा॒ ष्टाद॒शा हा॑नि॒ सम् । \newline
12. अ॒ष्टाद॒शेत्य॒ष्टा - द॒श॒ । \newline
13. अहा॑नि॒ सꣳ स महा॒ न्यहा॑नि॒ सम् प॑द्यन्ते पद्यन्ते॒ स महा॒ न्यहा॑नि॒ सम् प॑द्यन्ते । \newline
14. सम् प॑द्यन्ते पद्यन्ते॒ सꣳ सम् प॑द्यन्ते॒ नव॒ नव॑ पद्यन्ते॒ सꣳ सम् प॑द्यन्ते॒ नव॑ । \newline
15. प॒द्य॒न्ते॒ नव॒ नव॑ पद्यन्ते पद्यन्ते॒ नवा॒न्या न्य॒न्यानि॒ नव॑ पद्यन्ते पद्यन्ते॒ नवा॒न्यानि॑ । \newline
16. नवा॒ न्या न्य॒न्यानि॒ नव॒ नवा॒ न्यानि॒ नव॒ नवा॒ न्यानि॒ नव॒ नवा॒ न्यानि॒ नव॑ । \newline
17. अ॒न्यानि॒ नव॒ नवा॒न्या न्य॒न्यानि॒ नवा॒न्या न्य॒न्यानि॒ नवा॒न्या न्य॒न्यानि॒ नवा॒न्यानि॑ । \newline
18. नवा॒ न्यान्य॒ न्यानि॒ नव॒ नवा॒ न्यानि॒ नव॒ नवा॒ न्यानि॒ नव॒ नवा॒ न्यानि॒ नव॑ । \newline
19. अ॒न्यानि॒ नव॒ नवा॒न्या न्य॒न्यानि॒ नव॒ वै वै नवा॒न्या न्य॒न्यानि॒ नव॒ वै । \newline
20. नव॒ वै वै नव॒ नव॒ वै पुरु॑षे॒ पुरु॑षे॒ वै नव॒ नव॒ वै पुरु॑षे । \newline
21. वै पुरु॑षे॒ पुरु॑षे॒ वै वै पुरु॑षे प्रा॒णाः प्रा॒णाः पुरु॑षे॒ वै वै पुरु॑षे प्रा॒णाः । \newline
22. पुरु॑षे प्रा॒णाः प्रा॒णाः पुरु॑षे॒ पुरु॑षे प्रा॒णा स्तत् तत् प्रा॒णाः पुरु॑षे॒ पुरु॑षे प्रा॒णा स्तत् । \newline
23. प्रा॒णा स्तत् तत् प्रा॒णाः प्रा॒णा स्तत् प्रा॒णान् प्रा॒णाꣳ स्तत् प्रा॒णाः प्रा॒णा स्तत् प्रा॒णान् । \newline
24. प्रा॒णा इति॑ प्र - अ॒नाः । \newline
25. तत् प्रा॒णान् प्रा॒णाꣳ स्तत् तत् प्रा॒णान न्वनु॑ प्रा॒णाꣳ स्तत् तत् प्रा॒णाननु॑ । \newline
26. प्रा॒णान न्वनु॑ प्रा॒णान् प्रा॒णाननु॑ प॒र्याव॑र्तन्ते प॒र्याव॑र्त॒न्ते ऽनु॑ प्रा॒णान् प्रा॒णाननु॑ प॒र्याव॑र्तन्ते । \newline
27. प्रा॒णानिति॑ प्र - अ॒नान् । \newline
28. अनु॑ प॒र्याव॑र्तन्ते प॒र्याव॑र्त॒न्ते ऽन्वनु॑ प॒र्याव॑र्तन्ते च॒त्वार॑ श्च॒त्वारः॑ प॒र्याव॑र्त॒न्ते ऽन्वनु॑ प॒र्याव॑र्तन्ते च॒त्वारः॑ । \newline
29. प॒र्याव॑र्तन्ते च॒त्वार॑ श्च॒त्वारः॑ प॒र्याव॑र्तन्ते प॒र्याव॑र्तन्ते च॒त्वार॑ ष्षड॒हा ष्ष॑ड॒हा श्च॒त्वारः॑ प॒र्याव॑र्तन्ते प॒र्याव॑र्तन्ते च॒त्वार॑ ष्षड॒हाः । \newline
30. प॒र्याव॑र्तन्त॒ इति॑ परि - आव॑र्तन्ते । \newline
31. च॒त्वार॑ ष्षड॒हा ष्ष॑ड॒हा श्च॒त्वार॑ श्च॒त्वार॑ ष्षड॒हा भ॑वन्ति भवन्ति षड॒हा श्च॒त्वार॑ श्च॒त्वार॑ ष्षड॒हा भ॑वन्ति । \newline
32. ष॒ड॒हा भ॑वन्ति भवन्ति षड॒हा ष्ष॑ड॒हा भ॑वन्ति॒ तानि॒ तानि॑ भवन्ति षड॒हा ष्ष॑ड॒हा भ॑वन्ति॒ तानि॑ । \newline
33. ष॒ड॒हा इति॑ षट् - अ॒हाः । \newline
34. भ॒व॒न्ति॒ तानि॒ तानि॑ भवन्ति भवन्ति॒ तानि॒ चतु॑र्विꣳशति॒ श्चतु॑र्विꣳशति॒ स्तानि॑ भवन्ति भवन्ति॒ तानि॒ चतु॑र्विꣳशतिः । \newline
35. तानि॒ चतु॑र्विꣳशति॒ श्चतु॑र्विꣳशति॒ स्तानि॒ तानि॒ चतु॑र्विꣳशति॒ रहा॒ न्यहा॑नि॒ चतु॑र्विꣳशति॒ स्तानि॒ तानि॒ चतु॑र्विꣳशति॒ रहा॑नि । \newline
36. चतु॑र्विꣳशति॒ रहा॒ न्यहा॑नि॒ चतु॑र्विꣳशति॒ श्चतु॑र्विꣳशति॒ रहा॑नि॒ सꣳ समहा॑नि॒ चतु॑र्विꣳशति॒ श्चतु॑र्विꣳशति॒ रहा॑नि॒ सम् । \newline
37. चतु॑र्विꣳशति॒रिति॒ चतुः॑ - विꣳ॒॒श॒तिः॒ । \newline
38. अहा॑नि॒ सꣳ समहा॒ न्यहा॑नि॒ सम् प॑द्यन्ते पद्यन्ते॒ समहा॒ न्यहा॑नि॒ सम् प॑द्यन्ते । \newline
39. सम् प॑द्यन्ते पद्यन्ते॒ सꣳ सम् प॑द्यन्ते॒ चतु॑र्विꣳशति॒ श्चतु॑र्विꣳशतिः पद्यन्ते॒ सꣳ सम् प॑द्यन्ते॒ चतु॑र्विꣳशतिः । \newline
40. प॒द्य॒न्ते॒ चतु॑र्विꣳशति॒ श्चतु॑र्विꣳशतिः पद्यन्ते पद्यन्ते॒ चतु॑र्विꣳशति रर्द्धमा॒सा अ॑र्द्धमा॒सा श्चतु॑र्विꣳशतिः पद्यन्ते पद्यन्ते॒ चतु॑र्विꣳशति रर्द्धमा॒साः । \newline
41. चतु॑र्विꣳशति रर्द्धमा॒सा अ॑र्द्धमा॒सा श्चतु॑र्विꣳशति॒ श्चतु॑र्विꣳशति रर्द्धमा॒साः सं॑ॅवथ्स॒रः सं॑ॅवथ्स॒रो᳚ ऽर्द्धमा॒सा श्चतु॑र्विꣳशति॒ श्चतु॑र्विꣳशति रर्द्धमा॒साः सं॑ॅवथ्स॒रः । \newline
42. चतु॑र्विꣳशति॒रिति॒ चतुः॑ - विꣳ॒॒श॒तिः॒ । \newline
43. अ॒र्द्ध॒मा॒साः सं॑ॅवथ्स॒रः सं॑ॅवथ्स॒रो᳚ ऽर्द्धमा॒सा अ॑र्द्धमा॒साः सं॑ॅवथ्स॒र स्तत् तथ् सं॑ॅवथ्स॒रो᳚ ऽर्द्धमा॒सा अ॑र्द्धमा॒साः सं॑ॅवथ्स॒र स्तत् । \newline
44. अ॒र्द्ध॒मा॒सा इत्य॑र्द्ध - मा॒साः । \newline
45. सं॒ॅव॒थ्स॒र स्तत् तथ् सं॑ॅवथ्स॒रः सं॑ॅवथ्स॒र स्तथ् सं॑ॅवथ्स॒रꣳ सं॑ॅवथ्स॒रम् तथ् सं॑ॅवथ्स॒रः सं॑ॅवथ्स॒र स्तथ् सं॑ॅवथ्स॒रम् । \newline
46. सं॒ॅव॒थ्स॒र इति॑ सं - व॒थ्स॒रः । \newline
47. तथ् सं॑ॅवथ्स॒रꣳ सं॑ॅवथ्स॒रम् तत् तथ् सं॑ॅवथ्स॒र मन्वनु॑ संॅवथ्स॒रम् तत् तथ् सं॑ॅवथ्स॒र मनु॑ । \newline
48. सं॒ॅव॒थ्स॒र मन्वनु॑ संॅवथ्स॒रꣳ सं॑ॅवथ्स॒र मनु॑ प॒र्याव॑र्तन्ते प॒र्याव॑र्त॒न्ते ऽनु॑ संॅवथ्स॒रꣳ सं॑ॅवथ्स॒र मनु॑ प॒र्याव॑र्तन्ते । \newline
49. सं॒ॅव॒थ्स॒रमिति॑ सं - व॒थ्स॒रम् । \newline
50. अनु॑ प॒र्याव॑र्तन्ते प॒र्याव॑र्त॒न्ते ऽन्वनु॑ प॒र्याव॑र्त॒न्ते ऽप्र॑तिष्ठि॒तो ऽप्र॑तिष्ठितः प॒र्याव॑र्त॒न्ते ऽन्वनु॑ प॒र्याव॑र्त॒न्ते ऽप्र॑तिष्ठितः । \newline
51. प॒र्याव॑र्त॒न्ते ऽप्र॑तिष्ठि॒तो ऽप्र॑तिष्ठितः प॒र्याव॑र्तन्ते प॒र्याव॑र्त॒न्ते ऽप्र॑तिष्ठितः संॅवथ्स॒रः सं॑ॅवथ्स॒रो ऽप्र॑तिष्ठितः प॒र्याव॑र्तन्ते प॒र्याव॑र्त॒न्ते ऽप्र॑तिष्ठितः संॅवथ्स॒रः । \newline
52. प॒र्याव॑र्तन्त॒ इति॑ परि - आव॑र्तन्ते । \newline
53. अप्र॑तिष्ठितः संॅवथ्स॒रः सं॑ॅवथ्स॒रो ऽप्र॑तिष्ठि॒तो ऽप्र॑तिष्ठितः संॅवथ्स॒र इतीति॑ संॅवथ्स॒रो ऽप्र॑तिष्ठि॒तो ऽप्र॑तिष्ठितः संॅवथ्स॒र इति॑ । \newline
54. अप्र॑तिष्ठित॒ इत्यप्र॑ति - स्थि॒तः॒ । \newline
55. सं॒ॅव॒थ्स॒र इतीति॑ संॅवथ्स॒रः सं॑ॅवथ्स॒र इति॒ खलु॒ खल्विति॑ संॅवथ्स॒रः सं॑ॅवथ्स॒र इति॒ खलु॑ । \newline
56. सं॒ॅव॒थ्स॒र इति॑ सं - व॒थ्स॒रः । \newline
57. इति॒ खलु॒ खल्वितीति॒ खलु॒ वै वै खल्वितीति॒ खलु॒ वै । \newline
58. खलु॒ वै वै खलु॒ खलु॒ वा आ॑हु राहु॒र् वै खलु॒ खलु॒ वा आ॑हुः । \newline
59. वा आ॑हु राहु॒र् वै वा आ॑हु॒र् वर्.षी॑या॒न्॒. वर्.षी॑या नाहु॒र् वै वा आ॑हु॒र् वर्.षी॑यान् । \newline
60. आ॒हु॒र् वर्.षी॑या॒न्॒. वर्.षी॑या नाहु राहु॒र् वर्.षी॑यान् प्रति॒ष्ठायाः᳚ प्रति॒ष्ठाया॒ वर्.षी॑या नाहु राहु॒र् वर्.षी॑यान् प्रति॒ष्ठायाः᳚ । \newline
61. वर्.षी॑यान् प्रति॒ष्ठायाः᳚ प्रति॒ष्ठाया॒ वर्.षी॑या॒न्॒. वर्.षी॑यान् प्रति॒ष्ठाया॒ इतीति॑ प्रति॒ष्ठाया॒ वर्.षी॑या॒न्॒. वर्.षी॑यान् प्रति॒ष्ठाया॒ इति॑ । \newline
62. प्र॒ति॒ष्ठाया॒ इतीति॑ प्रति॒ष्ठायाः᳚ प्रति॒ष्ठाया॒ इत्ये॒ताव॑ दे॒ताव॒ दिति॑ प्रति॒ष्ठायाः᳚ प्रति॒ष्ठाया॒ इत्ये॒ताव॑त् । \newline
63. प्र॒ति॒ष्ठाया॒ इति॑ प्रति - स्थायाः᳚ । \newline
64. इत्ये॒ताव॑ दे॒ताव॒ दिती त्ये॒ताव॒द् वै वा ए॒ताव॒ दिती त्ये॒ताव॒द् वै । \newline
65. ए॒ताव॒द् वै वा ए॒ताव॑ दे॒ताव॒द् वै सं॑ॅवथ्स॒रस्य॑ संॅवथ्स॒रस्य॒ वा ए॒ताव॑ दे॒ताव॒द् वै सं॑ॅवथ्स॒रस्य॑ । \newline
66. वै सं॑ॅवथ्स॒रस्य॑ संॅवथ्स॒रस्य॒ वै वै सं॑ॅवथ्स॒रस्य॒ ब्राह्म॑ण॒म् ब्राह्म॑णꣳ संॅवथ्स॒रस्य॒ वै वै सं॑ॅवथ्स॒रस्य॒ ब्राह्म॑णम् । \newline
67. सं॒ॅव॒थ्स॒रस्य॒ ब्राह्म॑ण॒म् ब्राह्म॑णꣳ संॅवथ्स॒रस्य॑ संॅवथ्स॒रस्य॒ ब्राह्म॑णं॒ ॅयाव॒द् याव॒द् ब्राह्म॑णꣳ संॅवथ्स॒रस्य॑ संॅवथ्स॒रस्य॒ ब्राह्म॑णं॒ ॅयाव॑त् । \newline
68. सं॒ॅव॒थ्स॒रस्येति॑ सं - व॒थ्स॒रस्य॑ । \newline
69. ब्राह्म॑णं॒ ॅयाव॒द् याव॒द् ब्राह्म॑ण॒म् ब्राह्म॑णं॒ ॅयाव॑न् मा॒सो मा॒सो याव॒द् ब्राह्म॑ण॒म् ब्राह्म॑णं॒ ॅयाव॑न् मा॒सः । \newline
70. याव॑न् मा॒सो मा॒सो याव॒द् याव॑न् मा॒सो मा॒सिमा॑सि मा॒सिमा॑सि मा॒सो याव॒द् याव॑न् मा॒सो मा॒सिमा॑सि । \newline
71. मा॒सो मा॒सिमा॑सि मा॒सिमा॑सि मा॒सो मा॒सो मा॒सिमा᳚ स्ये॒वैव मा॒सिमा॑सि मा॒सो मा॒सो मा॒सिमा᳚ स्ये॒व । \newline
72. मा॒सिमा᳚ स्ये॒वैव मा॒सिमा॑सि मा॒सिमा᳚ स्ये॒व प्र॑ति॒तिष्ठ॑न्तः प्रति॒तिष्ठ॑न्त ए॒व मा॒सिमा॑सि मा॒सिमा᳚ स्ये॒व प्र॑ति॒तिष्ठ॑न्तः । \newline
73. मा॒सिमा॒सीति॑ मा॒सि - मा॒सि॒ । \newline
74. ए॒व प्र॑ति॒तिष्ठ॑न्तः प्रति॒तिष्ठ॑न्त ए॒वैव प्र॑ति॒तिष्ठ॑न्तो यन्ति यन्ति प्रति॒तिष्ठ॑न्त ए॒वैव प्र॑ति॒तिष्ठ॑न्तो यन्ति । \newline
75. प्र॒ति॒तिष्ठ॑न्तो यन्ति यन्ति प्रति॒तिष्ठ॑न्तः प्रति॒तिष्ठ॑न्तो यन्ति । \newline
76. प्र॒ति॒तिष्ठ॑न्त॒ इति॑ प्रति - तिष्ठ॑न्तः । \newline
77. य॒न्तीति॑ यन्ति । \newline
\pagebreak
\markright{ TS 7.4.12.1  \hfill https://www.vedavms.in \hfill}

\section{ TS 7.4.12.1 }

\textbf{TS 7.4.12.1 } \newline
\textbf{Samhita Paata} \newline

मे॒षस्त्वा॑ पच॒तैर॑वतु॒ लोहि॑तग्रीवः॒ छागैः᳚ शल्म॒लिर्वृद्ध्या॑ प॒र्णो ब्रह्म॑णा प्ल॒क्षो मेधे॑न न्य॒ग्रोध॑श्चम॒सैरु॑दु॒बंर॑ ऊ॒र्जा गा॑य॒त्री छन्दो॑भिस्त्रि॒वृथ् स्तोमै॒रव॑न्तीः॒ स्थाव॑न्तीस्त्वाऽवन्तु प्रि॒यं त्वा᳚ प्रि॒याणां॒ ॅवर्.षि॑ष्ठ॒माप्या॑नां निधी॒नां त्वा॑ निधि॒पतिꣳ॑ हवामहे वसो मम ॥ \newline

\textbf{Pada Paata} \newline

मे॒षः । त्वा॒ । प॒च॒तैः । अ॒व॒तु॒ । लोहि॑तग्रीव॒ इति॒ लोहि॑त - ग्री॒वः॒ । छागैः᳚ । श॒ल्म॒लिः । वृद्ध्या᳚ । प॒र्णः । ब्रह्म॑णा । प्ल॒क्षः । मेधे॑न । न्य॒ग्रोधः॑ । च॒म॒सैः । उ॒दु॒म्बरः॑ । ऊ॒र्जा । गा॒य॒त्री । छन्दो॑भि॒रिति॒ छन्दः॑ - भिः॒ । त्रि॒वृदिति॑ त्रि - वृत् । स्तोमैः᳚ । अव॑न्तीः । स्थ॒ । अव॑न्तीः । त्वा॒ । अ॒व॒न्तु॒ । प्रि॒यम् । त्वा॒ । प्रि॒याणा᳚म् । वर्.षि॑ष्ठम् । आप्या॑नाम् । नि॒धी॒नामिति॑ नि - धी॒नाम् । त्वा॒ । नि॒धि॒पति॒मिति॑ निधि - पति᳚म् । ह॒वा॒म॒हे॒ । व॒सो॒ इति॑ । म॒म॒ ॥  \newline


\textbf{Krama Paata} \newline

मे॒षस्त्वा᳚ । त्वा॒ प॒च॒तैः । प॒च॒तैर॑वतु । अ॒व॒तु॒ लोहि॑तग्रीवः । लोहि॑तग्रीव॒श्छागैः᳚ । लोहि॑तग्रीव॒ इति॒ लोहि॑त - ग्री॒वः॒ । छागैः᳚ शल्म॒लिः । श॒ल्म॒लिर् वृद्ध्या᳚ । वृद्ध्या॑ प॒र्णः । प॒र्णो ब्रह्म॑णा । ब्रह्म॑णा प्ल॒क्षः । प्ल॒क्षो मेधे॑न । मेधे॑न न्य॒ग्रोधः॑ । न्य॒ग्रोध॑श्चम॒सैः । च॒म॒सैरु॑दु॒म्बरः॑ । उ॒दु॒म्बर॑ ऊ॒र्जा । ऊ॒र्जा गा॑य॒त्री । गा॒य॒त्री छन्दो॑भिः । छन्दो॑भिस्त्रि॒वृत् । छन्दो॑भि॒रिति॒ छन्दः॑ - भिः॒ । त्रि॒वृथ् स्तोमैः᳚ । त्रि॒वृदिति॑ त्रि - वृत् । स्तोमै॒रव॑न्तीः । अव॑न्तीः स्थ । स्थाव॑न्तीः । अव॑न्तीस्त्वा । त्वा॒ऽव॒न्तु॒ । अ॒व॒न्तु॒ प्रि॒यम् । प्रि॒यम् त्वा᳚ । त्वा॒ प्रि॒याणा᳚म् । प्रि॒याणा॒म् ॅवर्.षि॑ष्ठम् । वर्.षि॑ष्ठ॒माप्या॑नाम् । आप्या॑नाम् निधी॒नाम् । नि॒धी॒नाम् त्वा᳚ । नि॒धी॒नामिति॑ नि - धी॒नाम् । त्वा॒ नि॒धि॒पति᳚म् । नि॒धि॒पतिꣳ॑ हवामहे । नि॒धि॒पति॒मिति॑ निधि - पति᳚म् । ह॒वा॒म॒हे॒ व॒सो॒ । व॒सो॒ म॒म॒ । व॒सो॒ इति॑ वसो । म॒मेति॑ मम । \newline

\textbf{Jatai Paata} \newline

1. मे॒ष स्त्वा᳚ त्वा मे॒षो मे॒ष स्त्वा᳚ । \newline
2. त्वा॒ प॒च॒तैः प॑च॒तै स्त्वा᳚ त्वा पच॒तैः । \newline
3. प॒च॒तै र॑व त्ववतु पच॒तैः प॑च॒तै र॑वतु । \newline
4. अ॒व॒तु॒ लोहि॑तग्रीवो॒ लोहि॑तग्रीवो ऽव त्ववतु॒ लोहि॑तग्रीवः । \newline
5. लोहि॑तग्रीव॒ श्छागै॒ श्छागै॒र् लोहि॑तग्रीवो॒ लोहि॑तग्रीव॒ श्छागैः᳚ । \newline
6. लोहि॑तग्रीव॒ इति॒ लोहि॑त - ग्री॒वः॒ । \newline
7. छागैः᳚ शल्म॒लिः श॑ल्म॒लि श्छागै॒ श्छागैः᳚ शल्म॒लिः । \newline
8. श॒ल्म॒लिर् वृद्ध्या॒ वृद्ध्या॑ शल्म॒लिः श॑ल्म॒लिर् वृद्ध्या᳚ । \newline
9. वृद्ध्या॑ प॒र्णः प॒र्णो वृद्ध्या॒ वृद्ध्या॑ प॒र्णः । \newline
10. प॒र्णो ब्रह्म॑णा॒ ब्रह्म॑णा प॒र्णः प॒र्णो ब्रह्म॑णा । \newline
11. ब्रह्म॑णा प्ल॒क्षः प्ल॒क्षो ब्रह्म॑णा॒ ब्रह्म॑णा प्ल॒क्षः । \newline
12. प्ल॒क्षो मेधे॑न॒ मेधे॑न प्ल॒क्षः प्ल॒क्षो मेधे॑न । \newline
13. मेधे॑न न्य॒ग्रोधो᳚ न्य॒ग्रोधो॒ मेधे॑न॒ मेधे॑न न्य॒ग्रोधः॑ । \newline
14. न्य॒ग्रोध॑ श्चम॒सै श्च॑म॒सैर् न्य॒ग्रोधो᳚ न्य॒ग्रोध॑ श्चम॒सैः । \newline
15. च॒म॒सै रु॑दु॒म्बर॑ उदु॒म्बर॑ श्चम॒सै श्च॑म॒सै रु॑दु॒म्बरः॑ । \newline
16. उ॒दु॒म्बर॑ ऊ॒र्जोर्जो दु॒म्बर॑ उदु॒म्बर॑ ऊ॒र्जा । \newline
17. ऊ॒र्जा गा॑य॒त्री गा॑य॒ त्र्यू᳚र्जोर्जा गा॑य॒त्री । \newline
18. गा॒य॒त्री छन्दो॑भि॒ श्छन्दो॑भिर् गाय॒त्री गा॑य॒त्री छन्दो॑भिः । \newline
19. छन्दो॑भि स्त्रि॒वृत् त्रि॒वृच् छन्दो॑भि॒ श्छन्दो॑भि स्त्रि॒वृत् । \newline
20. छन्दो॑भि॒रिति॒ छन्दः॑ - भिः॒ । \newline
21. त्रि॒वृथ् स्तोमैः॒ स्तोमै᳚ स्त्रि॒वृत् त्रि॒वृथ् स्तोमैः᳚ । \newline
22. त्रि॒वृदिति॑ त्रि - वृत् । \newline
23. स्तोमै॒ रव॑न्ती॒ रव॑न्तीः॒ स्तोमैः॒ स्तोमै॒ रव॑न्तीः । \newline
24. अव॑न्तीः स्थ॒ स्थाव॑न्ती॒ रव॑न्तीः स्थ । \newline
25. स्थाव॑न्ती॒ रव॑न्तीः स्थ॒ स्थाव॑न्तीः । \newline
26. अव॑न्ती स्त्वा॒ त्वा ऽव॑न्ती॒ रव॑न्ती स्त्वा । \newline
27. त्वा॒ ऽव॒न् त्व॒व॒न्तु॒ त्वा॒ त्वा॒ ऽव॒न्तु॒ । \newline
28. अ॒व॒न्तु॒ प्रि॒यम् प्रि॒य म॑वन् त्ववन्तु प्रि॒यम् । \newline
29. प्रि॒यम् त्वा᳚ त्वा प्रि॒यम् प्रि॒यम् त्वा᳚ । \newline
30. त्वा॒ प्रि॒याणा᳚म् प्रि॒याणा᳚म् त्वा त्वा प्रि॒याणा᳚म् । \newline
31. प्रि॒याणां॒ ॅवर्.षि॑ष्ठं॒ ॅवर्.षि॑ष्ठम् प्रि॒याणा᳚म् प्रि॒याणां॒ ॅवर्.षि॑ष्ठम् । \newline
32. वर्.षि॑ष्ठ॒ माप्या॑ना॒ माप्या॑नां॒ ॅवर्.षि॑ष्ठं॒ ॅवर्.षि॑ष्ठ॒ माप्या॑नाम् । \newline
33. आप्या॑नान् निधी॒नान् नि॑धी॒ना माप्या॑ना॒ माप्या॑नान् निधी॒नाम् । \newline
34. नि॒धी॒नाम् त्वा᳚ त्वा निधी॒नान् नि॑धी॒नाम् त्वा᳚ । \newline
35. नि॒धी॒नामिति॑ नि - धी॒नाम् । \newline
36. त्वा॒ नि॒धि॒पति॑न् निधि॒पति॑म् त्वा त्वा निधि॒पति᳚म् । \newline
37. नि॒धि॒पतिꣳ॑ हवामहे हवामहे निधि॒पति॑न् निधि॒पतिꣳ॑ हवामहे । \newline
38. नि॒धि॒पति॒मिति॑ निधि - पति᳚म् । \newline
39. ह॒वा॒म॒हे॒ व॒सो॒ व॒सो॒ ह॒वा॒म॒हे॒ ह॒वा॒म॒हे॒ व॒सो॒ । \newline
40. व॒सो॒ म॒म॒ म॒म॒ व॒सो॒ व॒सो॒ म॒म॒ । \newline
41. व॒सो॒ इति॑ वसो । \newline
42. म॒मेति॑ मम । \newline

\textbf{Ghana Paata } \newline

1. मे॒ष स्त्वा᳚ त्वा मे॒षो मे॒ष स्त्वा॑ पच॒तैः प॑च॒तै स्त्वा॑ मे॒षो मे॒ष स्त्वा॑ पच॒तैः । \newline
2. त्वा॒ प॒च॒तैः प॑च॒तै स्त्वा᳚ त्वा पच॒तै र॑व त्ववतु पच॒तै स्त्वा᳚ त्वा पच॒तै र॑वतु । \newline
3. प॒च॒तै र॑व त्ववतु पच॒तैः प॑च॒तै र॑वतु॒ लोहि॑तग्रीवो॒ लोहि॑तग्रीवो ऽवतु पच॒तैः प॑च॒तै र॑वतु॒ लोहि॑तग्रीवः । \newline
4. अ॒व॒तु॒ लोहि॑तग्रीवो॒ लोहि॑तग्रीवो ऽव त्ववतु॒ लोहि॑तग्रीव॒ श्छागै॒ श्छागै॒र् लोहि॑तग्रीवो ऽव त्ववतु॒ लोहि॑तग्रीव॒ श्छागैः᳚ । \newline
5. लोहि॑तग्रीव॒ श्छागै॒ श्छागै॒र् लोहि॑तग्रीवो॒ लोहि॑तग्रीव॒ श्छागैः᳚ शल्म॒लिः श॑ल्म॒लि श्छागै॒र् लोहि॑तग्रीवो॒ लोहि॑तग्रीव॒ श्छागैः᳚ शल्म॒लिः । \newline
6. लोहि॑तग्रीव॒ इति॒ लोहि॑त - ग्री॒वः॒ । \newline
7. छागैः᳚ शल्म॒लिः श॑ल्म॒लि श्छागै॒ श्छागैः᳚ शल्म॒लिर् वृद्ध्या॒ वृद्ध्या॑ शल्म॒लि श्छागै॒ श्छागैः᳚ शल्म॒लिर् वृद्ध्या᳚ । \newline
8. श॒ल्म॒लिर् वृद्ध्या॒ वृद्ध्या॑ शल्म॒लिः श॑ल्म॒लिर् वृद्ध्या॑ प॒र्णः प॒र्णो वृद्ध्या॑ शल्म॒लिः श॑ल्म॒लिर् वृद्ध्या॑ प॒र्णः । \newline
9. वृद्ध्या॑ प॒र्णः प॒र्णो वृद्ध्या॒ वृद्ध्या॑ प॒र्णो ब्रह्म॑णा॒ ब्रह्म॑णा प॒र्णो वृद्ध्या॒ वृद्ध्या॑ प॒र्णो ब्रह्म॑णा । \newline
10. प॒र्णो ब्रह्म॑णा॒ ब्रह्म॑णा प॒र्णः प॒र्णो ब्रह्म॑णा प्ल॒क्षः प्ल॒क्षो ब्रह्म॑णा प॒र्णः प॒र्णो ब्रह्म॑णा प्ल॒क्षः । \newline
11. ब्रह्म॑णा प्ल॒क्षः प्ल॒क्षो ब्रह्म॑णा॒ ब्रह्म॑णा प्ल॒क्षो मेधे॑न॒ मेधे॑न प्ल॒क्षो ब्रह्म॑णा॒ ब्रह्म॑णा प्ल॒क्षो मेधे॑न । \newline
12. प्ल॒क्षो मेधे॑न॒ मेधे॑न प्ल॒क्षः प्ल॒क्षो मेधे॑न न्य॒ग्रोधो᳚ न्य॒ग्रोधो॒ मेधे॑न प्ल॒क्षः प्ल॒क्षो मेधे॑न न्य॒ग्रोधः॑ । \newline
13. मेधे॑न न्य॒ग्रोधो᳚ न्य॒ग्रोधो॒ मेधे॑न॒ मेधे॑न न्य॒ग्रोध॑ श्चम॒सै श्च॑म॒सैर् न्य॒ग्रोधो॒ मेधे॑न॒ मेधे॑न न्य॒ग्रोध॑ श्चम॒सैः । \newline
14. न्य॒ग्रोध॑ श्चम॒सै श्च॑म॒सैर् न्य॒ग्रोधो᳚ न्य॒ग्रोध॑ श्चम॒सै रु॑दु॒म्बर॑ उदु॒म्बर॑ श्चम॒सैर् न्य॒ग्रोधो᳚ न्य॒ग्रोध॑ श्चम॒सै रु॑दु॒म्बरः॑ । \newline
15. च॒म॒सै रु॑दु॒म्बर॑ उदु॒म्बर॑ श्चम॒सै श्च॑म॒सै रु॑दु॒म्बर॑ ऊ॒र्जोर्जो दु॒म्बर॑ श्चम॒सै श्च॑म॒सै रु॑दु॒म्बर॑ ऊ॒र्जा । \newline
16. उ॒दु॒म्बर॑ ऊ॒र्जोर्जो दु॒म्बर॑ उदु॒म्बर॑ ऊ॒र्जा गा॑य॒त्री गा॑य॒ त्र्यू᳚र्जो दु॒म्बर॑ उदु॒म्बर॑ ऊ॒र्जा गा॑य॒त्री । \newline
17. ऊ॒र्जा गा॑य॒त्री गा॑य॒ त्र्यू᳚र्जोर्जा गा॑य॒त्री छन्दो॑भि॒ श्छन्दो॑भिर् गाय॒ त्र्यू᳚र्जोर्जा गा॑य॒त्री छन्दो॑भिः । \newline
18. गा॒य॒त्री छन्दो॑भि॒ श्छन्दो॑भिर् गाय॒त्री गा॑य॒त्री छन्दो॑भि स्त्रि॒वृत् त्रि॒वृच् छन्दो॑भिर् गाय॒त्री गा॑य॒त्री छन्दो॑भि स्त्रि॒वृत् । \newline
19. छन्दो॑भि स्त्रि॒वृत् त्रि॒वृच् छन्दो॑भि॒ श्छन्दो॑भि स्त्रि॒वृथ् स्तोमैः॒ स्तोमै᳚ स्त्रि॒वृच् छन्दो॑भि॒ श्छन्दो॑भि स्त्रि॒वृथ् स्तोमैः᳚ । \newline
20. छन्दो॑भि॒रिति॒ छन्दः॑ - भिः॒ । \newline
21. त्रि॒वृथ् स्तोमैः॒ स्तोमै᳚ स्त्रि॒वृत् त्रि॒वृथ् स्तोमै॒ रव॑न्ती॒ रव॑न्तीः॒ स्तोमै᳚ स्त्रि॒वृत् त्रि॒वृथ् स्तोमै॒ रव॑न्तीः । \newline
22. त्रि॒वृदिति॑ त्रि - वृत् । \newline
23. स्तोमै॒ रव॑न्ती॒ रव॑न्तीः॒ स्तोमैः॒ स्तोमै॒ रव॑न्तीः स्थ॒ स्थाव॑न्तीः॒ स्तोमैः॒ स्तोमै॒ रव॑न्तीः स्थ । \newline
24. अव॑न्तीः स्थ॒ स्थाव॑न्ती॒ रव॑न्तीः॒ स्थाव॑न्ती॒ रव॑न्तीः॒ स्थाव॑न्ती॒ रव॑न्तीः॒ स्थाव॑न्तीः । \newline
25. स्थाव॑न्ती॒ रव॑न्तीः स्थ॒ स्थाव॑न्ती स्त्वा॒ त्वा ऽव॑न्तीः स्थ॒ स्थाव॑न्ती स्त्वा । \newline
26. अव॑न्ती स्त्वा॒ त्वा ऽव॑न्ती॒ रव॑न्ती स्त्वा ऽवन् त्ववन्तु॒ त्वा ऽव॑न्ती॒ रव॑न्ती स्त्वा ऽवन्तु । \newline
27. त्वा॒ ऽव॒न् त्व॒व॒न्तु॒ त्वा॒ त्वा॒ ऽव॒न्तु॒ प्रि॒यम् प्रि॒य म॑वन्तु त्वा त्वा ऽवन्तु प्रि॒यम् । \newline
28. अ॒व॒न्तु॒ प्रि॒यम् प्रि॒य म॑वन् त्ववन्तु प्रि॒यम् त्वा᳚ त्वा प्रि॒य म॑वन् त्ववन्तु प्रि॒यम् त्वा᳚ । \newline
29. प्रि॒यम् त्वा᳚ त्वा प्रि॒यम् प्रि॒यम् त्वा᳚ प्रि॒याणा᳚म् प्रि॒याणा᳚म् त्वा प्रि॒यम् प्रि॒यम् त्वा᳚ प्रि॒याणा᳚म् । \newline
30. त्वा॒ प्रि॒याणा᳚म् प्रि॒याणा᳚म् त्वा त्वा प्रि॒याणां॒ ॅवर्.षि॑ष्ठं॒ ॅवर्.षि॑ष्ठम् प्रि॒याणा᳚म् त्वा त्वा प्रि॒याणां॒ ॅवर्.षि॑ष्ठम् । \newline
31. प्रि॒याणां॒ ॅवर्.षि॑ष्ठं॒ ॅवर्.षि॑ष्ठम् प्रि॒याणा᳚म् प्रि॒याणां॒ ॅवर्.षि॑ष्ठ॒ माप्या॑ना॒ माप्या॑नां॒ ॅवर्.षि॑ष्ठम् प्रि॒याणा᳚म् प्रि॒याणां॒ ॅवर्.षि॑ष्ठ॒ माप्या॑नाम् । \newline
32. वर्.षि॑ष्ठ॒ माप्या॑ना॒ माप्या॑नां॒ ॅवर्.षि॑ष्ठं॒ ॅवर्.षि॑ष्ठ॒ माप्या॑नान् निधी॒नान् नि॑धी॒ना माप्या॑नां॒ ॅवर्.षि॑ष्ठं॒ ॅवर्.षि॑ष्ठ॒ माप्या॑नान् निधी॒नाम् । \newline
33. आप्या॑नान् निधी॒नान् नि॑धी॒ना माप्या॑ना॒ माप्या॑नान् निधी॒नाम् त्वा᳚ त्वा निधी॒ना माप्या॑ना॒ माप्या॑नान् निधी॒नाम् त्वा᳚ । \newline
34. नि॒धी॒नाम् त्वा᳚ त्वा निधी॒नान् नि॑धी॒नाम् त्वा॑ निधि॒पति॑न् निधि॒पति॑म् त्वा निधी॒नान् नि॑धी॒नाम् त्वा॑ निधि॒पति᳚म् । \newline
35. नि॒धी॒नामिति॑ नि - धी॒नाम् । \newline
36. त्वा॒ नि॒धि॒पति॑न् निधि॒पति॑म् त्वा त्वा निधि॒पतिꣳ॑ हवामहे हवामहे निधि॒पति॑म् त्वा त्वा निधि॒पतिꣳ॑ हवामहे । \newline
37. नि॒धि॒पतिꣳ॑ हवामहे हवामहे निधि॒पति॑न् निधि॒पतिꣳ॑ हवामहे वसो वसो हवामहे निधि॒पति॑न् निधि॒पतिꣳ॑ हवामहे वसो । \newline
38. नि॒धि॒पति॒मिति॑ निधि - पति᳚म् । \newline
39. ह॒वा॒म॒हे॒ व॒सो॒ व॒सो॒ ह॒वा॒म॒हे॒ ह॒वा॒म॒हे॒ व॒सो॒ म॒म॒ म॒म॒ व॒सो॒ ह॒वा॒म॒हे॒ ह॒वा॒म॒हे॒ व॒सो॒ म॒म॒ । \newline
40. व॒सो॒ म॒म॒ म॒म॒ व॒सो॒ व॒सो॒ म॒म॒ । \newline
41. व॒सो॒ इति॑ वसो । \newline
42. म॒मेति॑ मम । \newline
\pagebreak
\markright{ TS 7.4.13.1  \hfill https://www.vedavms.in \hfill}

\section{ TS 7.4.13.1 }

\textbf{TS 7.4.13.1 } \newline
\textbf{Samhita Paata} \newline

कूप्या᳚भ्यः॒ स्वाहा॒ कूल्या᳚भ्यः॒ स्वाहा॑ विक॒र्या᳚भ्यः॒ स्वाहा॑ ऽव॒ट्या᳚भ्यः॒ स्वाहा॒ खन्या᳚भ्यः॒ स्वाहा॒ ह्रद्या᳚भ्यः॒ स्वाहा॒ सूद्या᳚भ्यः॒ स्वाहा॑ सर॒स्या᳚भ्यः॒ स्वाहा॑ वैश॒न्तीभ्यः॒ स्वाहा॑ पल्व॒ल्या᳚भ्यः॒ स्वाहा॒ वर्ष्या᳚भ्यः॒ स्वाहा॑ ऽव॒र्ष्याभ्यः॒ स्वाहा᳚ ह्रा॒दुनी᳚भ्यः॒ स्वाहा॒ पृष्वा᳚भ्यः॒ स्वाहा॒ स्यन्द॑मानाभ्यः॒ स्वाहा᳚ स्थाव॒राभ्यः॒ स्वाहा॑ नादे॒यीभ्यः॒ स्वाहा॑ सैन्ध॒वीभ्यः॒ स्वाहा॑ समु॒द्रिया᳚भ्यः॒ स्वाहा॒ सर्वा᳚भ्यः॒ स्वाहा᳚ ॥ \newline

\textbf{Pada Paata} \newline

कूप्या᳚भ्यः । स्वाहा᳚ । कूल्या᳚भ्यः । स्वाहा᳚ । वि॒क॒र्या᳚भ्य॒ इति॑ वि - क॒र्या᳚भ्यः । स्वाहा᳚ । अ॒व॒ट्या᳚भ्यः । स्वाहा᳚ । खन्या᳚भ्यः । स्वाहा᳚ । ह्रद्या᳚भ्यः । स्वाहा᳚ । सूद्या᳚भ्यः । स्वाहा᳚ । स॒र॒स्या᳚भ्यः । स्वाहा᳚ । वै॒श॒न्तीभ्यः॑ । स्वाहा᳚ । प॒ल्व॒ल्या᳚भ्यः । स्वाहा᳚ । वर्ष्या᳚भ्यः । स्वाहा᳚ । अ॒व॒र्ष्याभ्यः॑ । स्वाहा᳚ । ह्रा॒दुनी᳚भ्य॒ इति॑ ह्रा॒दुनि॑ - भ्यः॒ । स्वाहा᳚ । पृष्वा᳚भ्यः । स्वाहा᳚ । स्यन्द॑मानाभ्यः । स्वाहा᳚ । स्था॒व॒राभ्यः॑ । स्वाहा᳚ । ना॒दे॒यीभ्यः॑ । स्वाहा᳚ । सै॒न्ध॒वीभ्यः॑ । स्वाहा᳚ । स॒मु॒द्रिया᳚भ्यः । स्वाहा᳚ । सर्वा᳚भ्यः । स्वाहा᳚ ॥  \newline


\textbf{Krama Paata} \newline

कूप्या᳚भ्यः॒ स्वाहा᳚ । स्वाहा॒ कूल्या᳚भ्यः । कूल्या᳚भ्यः॒ स्वाहा᳚ । स्वाहा॑ विक॒र्या᳚भ्यः । वि॒क॒र्या᳚भ्यः॒ स्वाहा᳚ । वि॒क॒र्या᳚भ्य॒ इति॑ वि - क॒र्या᳚भ्यः । स्वाहा॑ऽव॒ट्‍या᳚भ्यः । अ॒व॒ट्‍या᳚भ्यः॒ स्वाहा᳚ । स्वाहा॒ खन्या᳚भ्यः । खन्या᳚भ्यः॒ स्वाहा᳚ । स्वाहा॒ ह्रद्या᳚भ्यः । ह्रद्या᳚भ्यः॒ स्वाहा᳚ । स्वाहा॒ सूद्या᳚भ्यः । सूद्या᳚भ्यः॒ स्वाहा᳚ । स्वाहा॑ सर॒स्या᳚भ्यः । स॒र॒स्या᳚भ्यः॒ स्वाहा᳚ । स्वाहा॑ वैश॒न्तीभ्यः॑ । वै॒श॒न्तीभ्यः॒ स्वाहा᳚ । स्वाहा॑ पल्व॒ल्या᳚भ्यः । प॒ल्व॒ल्या᳚भ्यः॒ स्वाहा᳚ । स्वाहा॒ वर्ष्या᳚भ्यः । वर्ष्या᳚भ्यः॒ स्वाहा᳚ । स्वाहा॑ऽव॒र्ष्याभ्यः॑ । अ॒व॒र्ष्याभ्यः॒ स्वाहा᳚ । स्वाहा᳚ ह्रा॒दुनी᳚भ्यः । ह्रा॒दुनी᳚भ्यः॒ स्वाहा᳚ । ह्रा॒दुनी᳚भ्य॒ इति॑ ह्रा॒दुनि॑ - भ्यः॒ । स्वाहा॒ पृष्वा᳚भ्यः । पृष्वा᳚भ्यः॒ स्वाहा᳚ । स्वाहा॒ स्यन्द॑मानाभ्यः । स्यन्द॑मानाभ्यः॒ स्वाहा᳚ । स्वाहा᳚ स्थाव॒राभ्यः॑ । स्था॒व॒राभ्यः॒ स्वाहा᳚ । स्वाहा॑ नादे॒यीभ्यः॑ । ना॒दे॒यीभ्यः॒ स्वाहा᳚ । स्वाहा॑ सैन्ध॒वीभ्यः॑ । सै॒न्ध॒वीभ्यः॒ स्वाहा᳚ । स्वाहा॑ समु॒द्रिया᳚भ्यः । स॒मु॒द्रिया᳚भ्यः॒ स्वाहा᳚ । स्वाहा॒ सर्वा᳚भ्यः । सर्वा᳚भ्यः॒ स्वाहा᳚ । स्वाहेति॒ स्वाहा᳚ । \newline

\textbf{Jatai Paata} \newline

1. कूप्या᳚भ्यः॒ स्वाहा॒ स्वाहा॒ कूप्या᳚भ्यः॒ कूप्या᳚भ्यः॒ स्वाहा᳚ । \newline
2. स्वाहा॒ कूल्या᳚भ्यः॒ कूल्या᳚भ्यः॒ स्वाहा॒ स्वाहा॒ कूल्या᳚भ्यः । \newline
3. कूल्या᳚भ्यः॒ स्वाहा॒ स्वाहा॒ कूल्या᳚भ्यः॒ कूल्या᳚भ्यः॒ स्वाहा᳚ । \newline
4. स्वाहा॑ विक॒र्या᳚भ्यो विक॒र्या᳚भ्यः॒ स्वाहा॒ स्वाहा॑ विक॒र्या᳚भ्यः । \newline
5. वि॒क॒र्या᳚भ्यः॒ स्वाहा॒ स्वाहा॑ विक॒र्या᳚भ्यो विक॒र्या᳚भ्यः॒ स्वाहा᳚ । \newline
6. वि॒क॒र्या᳚भ्य॒ इति॑ वि - क॒र्या᳚भ्यः । \newline
7. स्वाहा॑ ऽव॒ट्या᳚भ्यो ऽव॒ट्या᳚भ्यः॒ स्वाहा॒ स्वाहा॑ ऽव॒ट्या᳚भ्यः । \newline
8. अ॒व॒ट्या᳚भ्यः॒ स्वाहा॒ स्वाहा॑ ऽव॒ट्या᳚भ्यो ऽव॒ट्या᳚भ्यः॒ स्वाहा᳚ । \newline
9. स्वाहा॒ खन्या᳚भ्यः॒ खन्या᳚भ्यः॒ स्वाहा॒ स्वाहा॒ खन्या᳚भ्यः । \newline
10. खन्या᳚भ्यः॒ स्वाहा॒ स्वाहा॒ खन्या᳚भ्यः॒ खन्या᳚भ्यः॒ स्वाहा᳚ । \newline
11. स्वाहा॒ ह्रद्या᳚भ्यो॒ ह्रद्या᳚भ्यः॒ स्वाहा॒ स्वाहा॒ ह्रद्या᳚भ्यः । \newline
12. ह्रद्या᳚भ्यः॒ स्वाहा॒ स्वाहा॒ ह्रद्या᳚भ्यो॒ ह्रद्या᳚भ्यः॒ स्वाहा᳚ । \newline
13. स्वाहा॒ सूद्या᳚भ्यः॒ सूद्या᳚भ्यः॒ स्वाहा॒ स्वाहा॒ सूद्या᳚भ्यः । \newline
14. सूद्या᳚भ्यः॒ स्वाहा॒ स्वाहा॒ सूद्या᳚भ्यः॒ सूद्या᳚भ्यः॒ स्वाहा᳚ । \newline
15. स्वाहा॑ सर॒स्या᳚भ्यः सर॒स्या᳚भ्यः॒ स्वाहा॒ स्वाहा॑ सर॒स्या᳚भ्यः । \newline
16. स॒र॒स्या᳚भ्यः॒ स्वाहा॒ स्वाहा॑ सर॒स्या᳚भ्यः सर॒स्या᳚भ्यः॒ स्वाहा᳚ । \newline
17. स्वाहा॑ वैश॒न्तीभ्यो॑ वैश॒न्तीभ्यः॒ स्वाहा॒ स्वाहा॑ वैश॒न्तीभ्यः॑ । \newline
18. वै॒श॒न्तीभ्यः॒ स्वाहा॒ स्वाहा॑ वैश॒न्तीभ्यो॑ वैश॒न्तीभ्यः॒ स्वाहा᳚ । \newline
19. स्वाहा॑ पल्व॒ल्या᳚भ्यः पल्व॒ल्या᳚भ्यः॒ स्वाहा॒ स्वाहा॑ पल्व॒ल्या᳚भ्यः । \newline
20. प॒ल्व॒ल्या᳚भ्यः॒ स्वाहा॒ स्वाहा॑ पल्व॒ल्या᳚भ्यः पल्व॒ल्या᳚भ्यः॒ स्वाहा᳚ । \newline
21. स्वाहा॒ वर्ष्या᳚भ्यो॒ वर्ष्या᳚भ्यः॒ स्वाहा॒ स्वाहा॒ वर्ष्या᳚भ्यः । \newline
22. वर्ष्या᳚भ्यः॒ स्वाहा॒ स्वाहा॒ वर्ष्या᳚भ्यो॒ वर्ष्या᳚भ्यः॒ स्वाहा᳚ । \newline
23. स्वाहा॑ ऽव॒र्ष्याभ्यो॑ ऽव॒र्ष्याभ्यः॒ स्वाहा॒ स्वाहा॑ ऽव॒र्ष्याभ्यः॑ । \newline
24. अ॒व॒र्ष्याभ्यः॒ स्वाहा॒ स्वाहा॑ ऽव॒र्ष्याभ्यो॑ ऽव॒र्ष्याभ्यः॒ स्वाहा᳚ । \newline
25. स्वाहा᳚ ह्रा॒दुनी᳚भ्यो ह्रा॒दुनी᳚भ्यः॒ स्वाहा॒ स्वाहा᳚ ह्रा॒दुनी᳚भ्यः । \newline
26. ह्रा॒दुनी᳚भ्यः॒ स्वाहा॒ स्वाहा᳚ ह्रा॒दुनी᳚भ्यो ह्रा॒दुनी᳚भ्यः॒ स्वाहा᳚ । \newline
27. ह्रा॒दुनी᳚भ्य॒ इति॑ ह्रा॒दुनि॑ - भ्यः॒ । \newline
28. स्वाहा॒ पृष्वा᳚भ्यः॒ पृष्वा᳚भ्यः॒ स्वाहा॒ स्वाहा॒ पृष्वा᳚भ्यः । \newline
29. पृष्वा᳚भ्यः॒ स्वाहा॒ स्वाहा॒ पृष्वा᳚भ्यः॒ पृष्वा᳚भ्यः॒ स्वाहा᳚ । \newline
30. स्वाहा॒ स्यन्द॑मानाभ्यः॒ स्यन्द॑मानाभ्यः॒ स्वाहा॒ स्वाहा॒ स्यन्द॑मानाभ्यः । \newline
31. स्यन्द॑मानाभ्यः॒ स्वाहा॒ स्वाहा॒ स्यन्द॑मानाभ्यः॒ स्यन्द॑मानाभ्यः॒ स्वाहा᳚ । \newline
32. स्वाहा᳚ स्थाव॒राभ्यः॑ स्थाव॒राभ्यः॒ स्वाहा॒ स्वाहा᳚ स्थाव॒राभ्यः॑ । \newline
33. स्था॒व॒राभ्यः॒ स्वाहा॒ स्वाहा᳚ स्थाव॒राभ्यः॑ स्थाव॒राभ्यः॒ स्वाहा᳚ । \newline
34. स्वाहा॑ नादे॒यीभ्यो॑ नादे॒यीभ्यः॒ स्वाहा॒ स्वाहा॑ नादे॒यीभ्यः॑ । \newline
35. ना॒दे॒यीभ्यः॒ स्वाहा॒ स्वाहा॑ नादे॒यीभ्यो॑ नादे॒यीभ्यः॒ स्वाहा᳚ । \newline
36. स्वाहा॑ सैन्ध॒वीभ्यः॑ सैन्ध॒वीभ्यः॒ स्वाहा॒ स्वाहा॑ सैन्ध॒वीभ्यः॑ । \newline
37. सै॒न्ध॒वीभ्यः॒ स्वाहा॒ स्वाहा॑ सैन्ध॒वीभ्यः॑ सैन्ध॒वीभ्यः॒ स्वाहा᳚ । \newline
38. स्वाहा॑ समु॒द्रिया᳚भ्यः समु॒द्रिया᳚भ्यः॒ स्वाहा॒ स्वाहा॑ समु॒द्रिया᳚भ्यः । \newline
39. स॒मु॒द्रिया᳚भ्यः॒ स्वाहा॒ स्वाहा॑ समु॒द्रिया᳚भ्यः समु॒द्रिया᳚भ्यः॒ स्वाहा᳚ । \newline
40. स्वाहा॒ सर्वा᳚भ्यः॒ सर्वा᳚भ्यः॒ स्वाहा॒ स्वाहा॒ सर्वा᳚भ्यः । \newline
41. सर्वा᳚भ्यः॒ स्वाहा॒ स्वाहा॒ सर्वा᳚भ्यः॒ सर्वा᳚भ्यः॒ स्वाहा᳚ । \newline
42. स्वाहेति॒ स्वाहा᳚ । \newline

\textbf{Ghana Paata } \newline

1. कूप्या᳚भ्यः॒ स्वाहा॒ स्वाहा॒ कूप्या᳚भ्यः॒ कूप्या᳚भ्यः॒ स्वाहा॒ कूल्या᳚भ्यः॒ कूल्या᳚भ्यः॒ स्वाहा॒ कूप्या᳚भ्यः॒ कूप्या᳚भ्यः॒ स्वाहा॒ कूल्या᳚भ्यः । \newline
2. स्वाहा॒ कूल्या᳚भ्यः॒ कूल्या᳚भ्यः॒ स्वाहा॒ स्वाहा॒ कूल्या᳚भ्यः॒ स्वाहा॒ स्वाहा॒ कूल्या᳚भ्यः॒ स्वाहा॒ स्वाहा॒ कूल्या᳚भ्यः॒ स्वाहा᳚ । \newline
3. कूल्या᳚भ्यः॒ स्वाहा॒ स्वाहा॒ कूल्या᳚भ्यः॒ कूल्या᳚भ्यः॒ स्वाहा॑ विक॒र्या᳚भ्यो विक॒र्या᳚भ्यः॒ स्वाहा॒ कूल्या᳚भ्यः॒ कूल्या᳚भ्यः॒ स्वाहा॑ विक॒र्या᳚भ्यः । \newline
4. स्वाहा॑ विक॒र्या᳚भ्यो विक॒र्या᳚भ्यः॒ स्वाहा॒ स्वाहा॑ विक॒र्या᳚भ्यः॒ स्वाहा॒ स्वाहा॑ विक॒र्या᳚भ्यः॒ स्वाहा॒ स्वाहा॑ विक॒र्या᳚भ्यः॒ स्वाहा᳚ । \newline
5. वि॒क॒र्या᳚भ्यः॒ स्वाहा॒ स्वाहा॑ विक॒र्या᳚भ्यो विक॒र्या᳚भ्यः॒ स्वाहा॑ ऽव॒ट्या᳚भ्यो ऽव॒ट्या᳚भ्यः॒ स्वाहा॑ विक॒र्या᳚भ्यो विक॒र्या᳚भ्यः॒ स्वाहा॑ ऽव॒ट्या᳚भ्यः । \newline
6. वि॒क॒र्या᳚भ्य॒ इति॑ वि - क॒र्या᳚भ्यः । \newline
7. स्वाहा॑ ऽव॒ट्या᳚भ्यो ऽव॒ट्या᳚भ्यः॒ स्वाहा॒ स्वाहा॑ ऽव॒ट्या᳚भ्यः॒ स्वाहा॒ स्वाहा॑ ऽव॒ट्या᳚भ्यः॒ स्वाहा॒ स्वाहा॑ ऽव॒ट्या᳚भ्यः॒ स्वाहा᳚ । \newline
8. अ॒व॒ट्या᳚भ्यः॒ स्वाहा॒ स्वाहा॑ ऽव॒ट्या᳚भ्यो ऽव॒ट्या᳚भ्यः॒ स्वाहा॒ खन्या᳚भ्यः॒ खन्या᳚भ्यः॒ स्वाहा॑ ऽव॒ट्या᳚भ्यो ऽव॒ट्या᳚भ्यः॒ स्वाहा॒ खन्या᳚भ्यः । \newline
9. स्वाहा॒ खन्या᳚भ्यः॒ खन्या᳚भ्यः॒ स्वाहा॒ स्वाहा॒ खन्या᳚भ्यः॒ स्वाहा॒ स्वाहा॒ खन्या᳚भ्यः॒ स्वाहा॒ स्वाहा॒ खन्या᳚भ्यः॒ स्वाहा᳚ । \newline
10. खन्या᳚भ्यः॒ स्वाहा॒ स्वाहा॒ खन्या᳚भ्यः॒ खन्या᳚भ्यः॒ स्वाहा॒ ह्रद्या᳚भ्यो॒ ह्रद्या᳚भ्यः॒ स्वाहा॒ खन्या᳚भ्यः॒ खन्या᳚भ्यः॒ स्वाहा॒ ह्रद्या᳚भ्यः । \newline
11. स्वाहा॒ ह्रद्या᳚भ्यो॒ ह्रद्या᳚भ्यः॒ स्वाहा॒ स्वाहा॒ ह्रद्या᳚भ्यः॒ स्वाहा॒ स्वाहा॒ ह्रद्या᳚भ्यः॒ स्वाहा॒ स्वाहा॒ ह्रद्या᳚भ्यः॒ स्वाहा᳚ । \newline
12. ह्रद्या᳚भ्यः॒ स्वाहा॒ स्वाहा॒ ह्रद्या᳚भ्यो॒ ह्रद्या᳚भ्यः॒ स्वाहा॒ सूद्या᳚भ्यः॒ सूद्या᳚भ्यः॒ स्वाहा॒ ह्रद्या᳚भ्यो॒ ह्रद्या᳚भ्यः॒ स्वाहा॒ सूद्या᳚भ्यः । \newline
13. स्वाहा॒ सूद्या᳚भ्यः॒ सूद्या᳚भ्यः॒ स्वाहा॒ स्वाहा॒ सूद्या᳚भ्यः॒ स्वाहा॒ स्वाहा॒ सूद्या᳚भ्यः॒ स्वाहा॒ स्वाहा॒ सूद्या᳚भ्यः॒ स्वाहा᳚ । \newline
14. सूद्या᳚भ्यः॒ स्वाहा॒ स्वाहा॒ सूद्या᳚भ्यः॒ सूद्या᳚भ्यः॒ स्वाहा॑ सर॒स्या᳚भ्यः सर॒स्या᳚भ्यः॒ स्वाहा॒ सूद्या᳚भ्यः॒ सूद्या᳚भ्यः॒ स्वाहा॑ सर॒स्या᳚भ्यः । \newline
15. स्वाहा॑ सर॒स्या᳚भ्यः सर॒स्या᳚भ्यः॒ स्वाहा॒ स्वाहा॑ सर॒स्या᳚भ्यः॒ स्वाहा॒ स्वाहा॑ सर॒स्या᳚भ्यः॒ स्वाहा॒ स्वाहा॑ सर॒स्या᳚भ्यः॒ स्वाहा᳚ । \newline
16. स॒र॒स्या᳚भ्यः॒ स्वाहा॒ स्वाहा॑ सर॒स्या᳚भ्यः सर॒स्या᳚भ्यः॒ स्वाहा॑ वैश॒न्तीभ्यो॑ वैश॒न्तीभ्यः॒ स्वाहा॑ सर॒स्या᳚भ्यः सर॒स्या᳚भ्यः॒ स्वाहा॑ वैश॒न्तीभ्यः॑ । \newline
17. स्वाहा॑ वैश॒न्तीभ्यो॑ वैश॒न्तीभ्यः॒ स्वाहा॒ स्वाहा॑ वैश॒न्तीभ्यः॒ स्वाहा॒ स्वाहा॑ वैश॒न्तीभ्यः॒ स्वाहा॒ स्वाहा॑ वैश॒न्तीभ्यः॒ स्वाहा᳚ । \newline
18. वै॒श॒न्तीभ्यः॒ स्वाहा॒ स्वाहा॑ वैश॒न्तीभ्यो॑ वैश॒न्तीभ्यः॒ स्वाहा॑ पल्व॒ल्या᳚भ्यः पल्व॒ल्या᳚भ्यः॒ स्वाहा॑ वैश॒न्तीभ्यो॑ वैश॒न्तीभ्यः॒ स्वाहा॑ पल्व॒ल्या᳚भ्यः । \newline
19. स्वाहा॑ पल्व॒ल्या᳚भ्यः पल्व॒ल्या᳚भ्यः॒ स्वाहा॒ स्वाहा॑ पल्व॒ल्या᳚भ्यः॒ स्वाहा॒ स्वाहा॑ पल्व॒ल्या᳚भ्यः॒ स्वाहा॒ स्वाहा॑ पल्व॒ल्या᳚भ्यः॒ स्वाहा᳚ । \newline
20. प॒ल्व॒ल्या᳚भ्यः॒ स्वाहा॒ स्वाहा॑ पल्व॒ल्या᳚भ्यः पल्व॒ल्या᳚भ्यः॒ स्वाहा॒ वर्ष्या᳚भ्यो॒ वर्ष्या᳚भ्यः॒ स्वाहा॑ पल्व॒ल्या᳚भ्यः पल्व॒ल्या᳚भ्यः॒ स्वाहा॒ वर्ष्या᳚भ्यः । \newline
21. स्वाहा॒ वर्ष्या᳚भ्यो॒ वर्ष्या᳚भ्यः॒ स्वाहा॒ स्वाहा॒ वर्ष्या᳚भ्यः॒ स्वाहा॒ स्वाहा॒ वर्ष्या᳚भ्यः॒ स्वाहा॒ स्वाहा॒ वर्ष्या᳚भ्यः॒ स्वाहा᳚ । \newline
22. वर्ष्या᳚भ्यः॒ स्वाहा॒ स्वाहा॒ वर्ष्या᳚भ्यो॒ वर्ष्या᳚भ्यः॒ स्वाहा॑ ऽव॒र्ष्याभ्यो॑ ऽव॒र्ष्याभ्यः॒ स्वाहा॒ वर्ष्या᳚भ्यो॒ वर्ष्या᳚भ्यः॒ स्वाहा॑ ऽव॒र्ष्याभ्यः॑ । \newline
23. स्वाहा॑ ऽव॒र्ष्याभ्यो॑ ऽव॒र्ष्याभ्यः॒ स्वाहा॒ स्वाहा॑ ऽव॒र्ष्याभ्यः॒ स्वाहा॒ स्वाहा॑ ऽव॒र्ष्याभ्यः॒ स्वाहा॒ स्वाहा॑ ऽव॒र्ष्याभ्यः॒ स्वाहा᳚ । \newline
24. अ॒व॒र्ष्याभ्यः॒ स्वाहा॒ स्वाहा॑ ऽव॒र्ष्याभ्यो॑ ऽव॒र्ष्याभ्यः॒ स्वाहा᳚ ह्रा॒दुनी᳚भ्यो ह्रा॒दुनी᳚भ्यः॒ स्वाहा॑ ऽव॒र्ष्याभ्यो॑ ऽव॒र्ष्याभ्यः॒ स्वाहा᳚ ह्रा॒दुनी᳚भ्यः । \newline
25. स्वाहा᳚ ह्रा॒दुनी᳚भ्यो ह्रा॒दुनी᳚भ्यः॒ स्वाहा॒ स्वाहा᳚ ह्रा॒दुनी᳚भ्यः॒ स्वाहा॒ स्वाहा᳚ ह्रा॒दुनी᳚भ्यः॒ स्वाहा॒ स्वाहा᳚ ह्रा॒दुनी᳚भ्यः॒ स्वाहा᳚ । \newline
26. ह्रा॒दुनी᳚भ्यः॒ स्वाहा॒ स्वाहा᳚ ह्रा॒दुनी᳚भ्यो ह्रा॒दुनी᳚भ्यः॒ स्वाहा॒ पृष्वा᳚भ्यः॒ पृष्वा᳚भ्यः॒ स्वाहा᳚ ह्रा॒दुनी᳚भ्यो ह्रा॒दुनी᳚भ्यः॒ स्वाहा॒ पृष्वा᳚भ्यः । \newline
27. ह्रा॒दुनी᳚भ्य॒ इति॑ ह्रा॒दुनि॑ - भ्यः॒ । \newline
28. स्वाहा॒ पृष्वा᳚भ्यः॒ पृष्वा᳚भ्यः॒ स्वाहा॒ स्वाहा॒ पृष्वा᳚भ्यः॒ स्वाहा॒ स्वाहा॒ पृष्वा᳚भ्यः॒ स्वाहा॒ स्वाहा॒ पृष्वा᳚भ्यः॒ स्वाहा᳚ । \newline
29. पृष्वा᳚भ्यः॒ स्वाहा॒ स्वाहा॒ पृष्वा᳚भ्यः॒ पृष्वा᳚भ्यः॒ स्वाहा॒ स्यन्द॑मानाभ्यः॒ स्यन्द॑मानाभ्यः॒ स्वाहा॒ पृष्वा᳚भ्यः॒ पृष्वा᳚भ्यः॒ स्वाहा॒ स्यन्द॑मानाभ्यः । \newline
30. स्वाहा॒ स्यन्द॑मानाभ्यः॒ स्यन्द॑मानाभ्यः॒ स्वाहा॒ स्वाहा॒ स्यन्द॑मानाभ्यः॒ स्वाहा॒ स्वाहा॒ स्यन्द॑मानाभ्यः॒ स्वाहा॒ स्वाहा॒ स्यन्द॑मानाभ्यः॒ स्वाहा᳚ । \newline
31. स्यन्द॑मानाभ्यः॒ स्वाहा॒ स्वाहा॒ स्यन्द॑मानाभ्यः॒ स्यन्द॑मानाभ्यः॒ स्वाहा᳚ स्थाव॒राभ्यः॑ स्थाव॒राभ्यः॒ स्वाहा॒ स्यन्द॑मानाभ्यः॒ स्यन्द॑मानाभ्यः॒ स्वाहा᳚ स्थाव॒राभ्यः॑ । \newline
32. स्वाहा᳚ स्थाव॒राभ्यः॑ स्थाव॒राभ्यः॒ स्वाहा॒ स्वाहा᳚ स्थाव॒राभ्यः॒ स्वाहा॒ स्वाहा᳚ स्थाव॒राभ्यः॒ स्वाहा॒ स्वाहा᳚ स्थाव॒राभ्यः॒ स्वाहा᳚ । \newline
33. स्था॒व॒राभ्यः॒ स्वाहा॒ स्वाहा᳚ स्थाव॒राभ्यः॑ स्थाव॒राभ्यः॒ स्वाहा॑ नादे॒यीभ्यो॑ नादे॒यीभ्यः॒ स्वाहा᳚ स्थाव॒राभ्यः॑ स्थाव॒राभ्यः॒ स्वाहा॑ नादे॒यीभ्यः॑ । \newline
34. स्वाहा॑ नादे॒यीभ्यो॑ नादे॒यीभ्यः॒ स्वाहा॒ स्वाहा॑ नादे॒यीभ्यः॒ स्वाहा॒ स्वाहा॑ नादे॒यीभ्यः॒ स्वाहा॒ स्वाहा॑ नादे॒यीभ्यः॒ स्वाहा᳚ । \newline
35. ना॒दे॒यीभ्यः॒ स्वाहा॒ स्वाहा॑ नादे॒यीभ्यो॑ नादे॒यीभ्यः॒ स्वाहा॑ सैन्ध॒वीभ्यः॑ सैन्ध॒वीभ्यः॒ स्वाहा॑ नादे॒यीभ्यो॑ नादे॒यीभ्यः॒ स्वाहा॑ सैन्ध॒वीभ्यः॑ । \newline
36. स्वाहा॑ सैन्ध॒वीभ्यः॑ सैन्ध॒वीभ्यः॒ स्वाहा॒ स्वाहा॑ सैन्ध॒वीभ्यः॒ स्वाहा॒ स्वाहा॑ सैन्ध॒वीभ्यः॒ स्वाहा॒ स्वाहा॑ सैन्ध॒वीभ्यः॒ स्वाहा᳚ । \newline
37. सै॒न्ध॒वीभ्यः॒ स्वाहा॒ स्वाहा॑ सैन्ध॒वीभ्यः॑ सैन्ध॒वीभ्यः॒ स्वाहा॑ समु॒द्रिया᳚भ्यः समु॒द्रिया᳚भ्यः॒ स्वाहा॑ सैन्ध॒वीभ्यः॑ सैन्ध॒वीभ्यः॒ स्वाहा॑ समु॒द्रिया᳚भ्यः । \newline
38. स्वाहा॑ समु॒द्रिया᳚भ्यः समु॒द्रिया᳚भ्यः॒ स्वाहा॒ स्वाहा॑ समु॒द्रिया᳚भ्यः॒ स्वाहा॒ स्वाहा॑ समु॒द्रिया᳚भ्यः॒ स्वाहा॒ स्वाहा॑ समु॒द्रिया᳚भ्यः॒ स्वाहा᳚ । \newline
39. स॒मु॒द्रिया᳚भ्यः॒ स्वाहा॒ स्वाहा॑ समु॒द्रिया᳚भ्यः समु॒द्रिया᳚भ्यः॒ स्वाहा॒ सर्वा᳚भ्यः॒ सर्वा᳚भ्यः॒ स्वाहा॑ समु॒द्रिया᳚भ्यः समु॒द्रिया᳚भ्यः॒ स्वाहा॒ सर्वा᳚भ्यः । \newline
40. स्वाहा॒ सर्वा᳚भ्यः॒ सर्वा᳚भ्यः॒ स्वाहा॒ स्वाहा॒ सर्वा᳚भ्यः॒ स्वाहा॒ स्वाहा॒ सर्वा᳚भ्यः॒ स्वाहा॒ स्वाहा॒ सर्वा᳚भ्यः॒ स्वाहा᳚ । \newline
41. सर्वा᳚भ्यः॒ स्वाहा॒ स्वाहा॒ सर्वा᳚भ्यः॒ सर्वा᳚भ्यः॒ स्वाहा᳚ । \newline
42. स्वाहेति॒ स्वाहा᳚ । \newline
\pagebreak
\markright{ TS 7.4.14.1  \hfill https://www.vedavms.in \hfill}

\section{ TS 7.4.14.1 }

\textbf{TS 7.4.14.1 } \newline
\textbf{Samhita Paata} \newline

अ॒द्भ्यः स्वाहा॒ वह॑न्तीभ्यः॒ स्वाहा॑ परि॒वह॑न्तीभ्यः॒ स्वाहा॑ सम॒न्तं ॅवह॑न्तीभ्यः॒ स्वाहा॒ शीघ्रं॒ ॅवह॑न्तीभ्यः॒ स्वाहा॒ शीभं॒ ॅवह॑न्तीभ्यः॒ स्वाहो॒ग्रं ॅवह॑न्तीभ्यः॒ स्वाहा॑ भी॒मं ॅवह॑न्तीभ्यः॒ स्वाहाऽम्भो᳚भ्यः॒ स्वाहा॒ नभो᳚भ्यः॒ स्वाहा॒ महो᳚भ्यः॒ स्वाहा॒ सर्व॑स्मै॒ स्वाहा᳚ ॥ \newline

\textbf{Pada Paata} \newline

अ॒द्भ्य इत्य॑त् - भ्यः । स्वाहा᳚ । वह॑न्तीभ्यः । स्वाहा᳚ । प॒रि॒वह॑न्तीभ्य॒ इति॑ परि - वह॑न्तीभ्यः । स्वाहा᳚ । स॒म॒न्तमिति॑ सं-अ॒न्तम् । वह॑न्तीभ्यः । स्वाहा᳚ । शीघ्र᳚म् । वह॑न्तीभ्यः । स्वाहा᳚ । शीभ᳚म् । वह॑न्तीभ्यः । स्वाहा᳚ । उ॒ग्रम् । वह॑न्तीभ्यः । स्वाहा᳚ । भी॒मम् । वह॑न्तीभ्यः । स्वाहा᳚ । अम्भो᳚भ्य॒ इत्यम्भः॑-भ्यः॒ । स्वाहा᳚ । नभो᳚भ्य॒ इति॒ नभः॑ - भ्यः॒ । स्वाहा᳚ । महो᳚भ्य॒ इति॒ महः॑ - भ्यः॒ । स्वाहा᳚ । सर्व॑स्मै । स्वाहा᳚ ॥  \newline


\textbf{Krama Paata} \newline

अ॒द्भ्यः स्वाहा᳚ । अ॒द्भ्य इत्य॑त् - भ्यः । स्वाहा॒ वह॑न्तीभ्यः । वह॑न्तीभ्यः॒ स्वाहा᳚ । स्वाहा॑ परि॒वह॑न्तीभ्यः । 
प॒रि॒वह॑न्तीभ्यः॒ स्वाहा᳚ । प॒रि॒वह॑न्तीभ्य॒ इति॑ परि - वह॑न्तीभ्यः । स्वाहा॑ सम॒न्तम् । स॒म॒न्तम् ॅवह॑न्तीभ्यः । स॒म॒न्तमिति॑ सम् - अ॒न्तम् । वह॑न्तीभ्यः॒ स्वाहा᳚ । स्वाहा॒ शीघ्र᳚म् । शीघ्र॒म् ॅवह॑न्तीभ्यः । वह॑न्तीभ्यः॒ स्वाहा᳚ । स्वाहा॒ शीभ᳚म् । शीभ॒म् ॅवह॑न्तीभ्यः । वह॑न्तीभ्यः॒ स्वाहा᳚ । स्वाहो॒ग्रम् । उ॒ग्रम् ॅवह॑न्तीभ्यः । वह॑न्तीभ्यः॒ स्वाहा᳚ । स्वाहा॑ भी॒मम् । भी॒मम् ॅवह॑न्तीभ्यः । वह॑न्तीभ्यः॒ स्वाहा᳚ । स्वाहाऽम्भो᳚भ्यः । अम्भो᳚भ्यः॒ स्वाहा᳚ । अम्भो᳚भ्य॒ इत्यम्भः॑ - भ्यः॒ । स्वाहा॒ नभो᳚भ्यः । नभो᳚भ्यः॒ स्वाहा᳚ । नभो᳚भ्य॒ इति॒ नभः॑ - भ्यः॒ । स्वाहा॒ महो᳚भ्यः । महो᳚भ्यः॒ स्वाहा᳚ । महो᳚भ्य॒ इति॒ महः॑ - भ्यः॒ । स्वाहा॒ सर्व॑स्मै । सर्व॑स्मै॒ स्वाहा᳚ । स्वाहेति॒ स्वाहा᳚ । \newline

\textbf{Jatai Paata} \newline

1. अ॒द्भ्यः स्वाहा॒ स्वाहा॒ ऽद्भ्यो᳚ ऽद्भ्यः स्वाहा᳚ । \newline
2. अ॒द्भ्य इत्य॑त् - भ्यः । \newline
3. स्वाहा॒ वह॑न्तीभ्यो॒ वह॑न्तीभ्यः॒ स्वाहा॒ स्वाहा॒ वह॑न्तीभ्यः । \newline
4. वह॑न्तीभ्यः॒ स्वाहा॒ स्वाहा॒ वह॑न्तीभ्यो॒ वह॑न्तीभ्यः॒ स्वाहा᳚ । \newline
5. स्वाहा॑ परि॒वह॑न्तीभ्यः परि॒वह॑न्तीभ्यः॒ स्वाहा॒ स्वाहा॑ परि॒वह॑न्तीभ्यः । \newline
6. प॒रि॒वह॑न्तीभ्यः॒ स्वाहा॒ स्वाहा॑ परि॒वह॑न्तीभ्यः परि॒वह॑न्तीभ्यः॒ स्वाहा᳚ । \newline
7. प॒रि॒वह॑न्तीभ्य॒ इति॑ परि - वह॑न्तीभ्यः । \newline
8. स्वाहा॑ सम॒न्तꣳ स॑म॒न्तꣳ स्वाहा॒ स्वाहा॑ सम॒न्तम् । \newline
9. स॒म॒न्तं ॅवह॑न्तीभ्यो॒ वह॑न्तीभ्यः सम॒न्तꣳ स॑म॒न्तं ॅवह॑न्तीभ्यः । \newline
10. स॒म॒न्तमिति॑ सं - अ॒न्तम् । \newline
11. वह॑न्तीभ्यः॒ स्वाहा॒ स्वाहा॒ वह॑न्तीभ्यो॒ वह॑न्तीभ्यः॒ स्वाहा᳚ । \newline
12. स्वाहा॒ शीघ्रꣳ॒॒ शीघ्रꣳ॒॒ स्वाहा॒ स्वाहा॒ शीघ्र᳚म् । \newline
13. शीघ्रं॒ ॅवह॑न्तीभ्यो॒ वह॑न्तीभ्यः॒ शीघ्रꣳ॒॒ शीघ्रं॒ ॅवह॑न्तीभ्यः । \newline
14. वह॑न्तीभ्यः॒ स्वाहा॒ स्वाहा॒ वह॑न्तीभ्यो॒ वह॑न्तीभ्यः॒ स्वाहा᳚ । \newline
15. स्वाहा॒ शीभꣳ॒॒ शीभꣳ॒॒ स्वाहा॒ स्वाहा॒ शीभ᳚म् । \newline
16. शीभं॒ ॅवह॑न्तीभ्यो॒ वह॑न्तीभ्यः॒ शीभꣳ॒॒ शीभं॒ ॅवह॑न्तीभ्यः । \newline
17. वह॑न्तीभ्यः॒ स्वाहा॒ स्वाहा॒ वह॑न्तीभ्यो॒ वह॑न्तीभ्यः॒ स्वाहा᳚ । \newline
18. स्वाहो॒ग्र मु॒ग्रꣳ स्वाहा॒ स्वाहो॒ग्रम् । \newline
19. उ॒ग्रं ॅवह॑न्तीभ्यो॒ वह॑न्तीभ्य उ॒ग्र मु॒ग्रं ॅवह॑न्तीभ्यः । \newline
20. वह॑न्तीभ्यः॒ स्वाहा॒ स्वाहा॒ वह॑न्तीभ्यो॒ वह॑न्तीभ्यः॒ स्वाहा᳚ । \newline
21. स्वाहा॑ भी॒मम् भी॒मꣳ स्वाहा॒ स्वाहा॑ भी॒मम् । \newline
22. भी॒मं ॅवह॑न्तीभ्यो॒ वह॑न्तीभ्यो भी॒मम् भी॒मं ॅवह॑न्तीभ्यः । \newline
23. वह॑न्तीभ्यः॒ स्वाहा॒ स्वाहा॒ वह॑न्तीभ्यो॒ वह॑न्तीभ्यः॒ स्वाहा᳚ । \newline
24. स्वाहा ऽम्भो॒भ्यो ऽम्भो᳚भ्यः॒ स्वाहा॒ स्वाहा ऽम्भो᳚भ्यः । \newline
25. अम्भो᳚भ्यः॒ स्वाहा॒ स्वाहा ऽम्भो॒भ्यो ऽम्भो᳚भ्यः॒ स्वाहा᳚ । \newline
26. अम्भो᳚भ्य॒ इत्यम्भः॑ - भ्यः॒ । \newline
27. स्वाहा॒ नभो᳚भ्यो॒ नभो᳚भ्यः॒ स्वाहा॒ स्वाहा॒ नभो᳚भ्यः । \newline
28. नभो᳚भ्यः॒ स्वाहा॒ स्वाहा॒ नभो᳚भ्यो॒ नभो᳚भ्यः॒ स्वाहा᳚ । \newline
29. नभो᳚भ्य॒ इति॒ नभः॑ - भ्यः॒ । \newline
30. स्वाहा॒ महो᳚भ्यो॒ महो᳚भ्यः॒ स्वाहा॒ स्वाहा॒ महो᳚भ्यः । \newline
31. महो᳚भ्यः॒ स्वाहा॒ स्वाहा॒ महो᳚भ्यो॒ महो᳚भ्यः॒ स्वाहा᳚ । \newline
32. महो᳚भ्य॒ इति॒ महः॑ - भ्यः॒ । \newline
33. स्वाहा॒ सर्व॑स्मै॒ सर्व॑स्मै॒ स्वाहा॒ स्वाहा॒ सर्व॑स्मै । \newline
34. सर्व॑स्मै॒ स्वाहा॒ स्वाहा॒ सर्व॑स्मै॒ सर्व॑स्मै॒ स्वाहा᳚ । \newline
35. स्वाहेति॒ स्वाहा᳚ । \newline

\textbf{Ghana Paata } \newline

1. अ॒द्भ्यः स्वाहा॒ स्वाहा॒ ऽद्भ्यो᳚ ऽद्भ्यः स्वाहा॒ वह॑न्तीभ्यो॒ वह॑न्तीभ्यः॒ स्वाहा॒ ऽद्भ्यो᳚ ऽद्भ्यः स्वाहा॒ वह॑न्तीभ्यः । \newline
2. अ॒द्भ्य इत्य॑त् - भ्यः । \newline
3. स्वाहा॒ वह॑न्तीभ्यो॒ वह॑न्तीभ्यः॒ स्वाहा॒ स्वाहा॒ वह॑न्तीभ्यः॒ स्वाहा॒ स्वाहा॒ वह॑न्तीभ्यः॒ स्वाहा॒ स्वाहा॒ वह॑न्तीभ्यः॒ स्वाहा᳚ । \newline
4. वह॑न्तीभ्यः॒ स्वाहा॒ स्वाहा॒ वह॑न्तीभ्यो॒ वह॑न्तीभ्यः॒ स्वाहा॑ परि॒वह॑न्तीभ्यः परि॒वह॑न्तीभ्यः॒ स्वाहा॒ वह॑न्तीभ्यो॒ वह॑न्तीभ्यः॒ स्वाहा॑ परि॒वह॑न्तीभ्यः । \newline
5. स्वाहा॑ परि॒वह॑न्तीभ्यः परि॒वह॑न्तीभ्यः॒ स्वाहा॒ स्वाहा॑ परि॒वह॑न्तीभ्यः॒ स्वाहा॒ स्वाहा॑ परि॒वह॑न्तीभ्यः॒ स्वाहा॒ स्वाहा॑ परि॒वह॑न्तीभ्यः॒ स्वाहा᳚ । \newline
6. प॒रि॒वह॑न्तीभ्यः॒ स्वाहा॒ स्वाहा॑ परि॒वह॑न्तीभ्यः परि॒वह॑न्तीभ्यः॒ स्वाहा॑ सम॒न्तꣳ स॑म॒न्तꣳ स्वाहा॑ परि॒वह॑न्तीभ्यः परि॒वह॑न्तीभ्यः॒ स्वाहा॑ सम॒न्तम् । \newline
7. प॒रि॒वह॑न्तीभ्य॒ इति॑ परि - वह॑न्तीभ्यः । \newline
8. स्वाहा॑ सम॒न्तꣳ स॑म॒न्तꣳ स्वाहा॒ स्वाहा॑ सम॒न्तं ॅवह॑न्तीभ्यो॒ वह॑न्तीभ्यः सम॒न्तꣳ स्वाहा॒ स्वाहा॑ सम॒न्तं ॅवह॑न्तीभ्यः । \newline
9. स॒म॒न्तं ॅवह॑न्तीभ्यो॒ वह॑न्तीभ्यः सम॒न्तꣳ स॑म॒न्तं ॅवह॑न्तीभ्यः॒ स्वाहा॒ स्वाहा॒ वह॑न्तीभ्यः सम॒न्तꣳ स॑म॒न्तं ॅवह॑न्तीभ्यः॒ स्वाहा᳚ । \newline
10. स॒म॒न्तमिति॑ सं - अ॒न्तम् । \newline
11. वह॑न्तीभ्यः॒ स्वाहा॒ स्वाहा॒ वह॑न्तीभ्यो॒ वह॑न्तीभ्यः॒ स्वाहा॒ शीघ्रꣳ॒॒ शीघ्रꣳ॒॒ स्वाहा॒ वह॑न्तीभ्यो॒ वह॑न्तीभ्यः॒ स्वाहा॒ शीघ्र᳚म् । \newline
12. स्वाहा॒ शीघ्रꣳ॒॒ शीघ्रꣳ॒॒ स्वाहा॒ स्वाहा॒ शीघ्रं॒ ॅवह॑न्तीभ्यो॒ वह॑न्तीभ्यः॒ शीघ्रꣳ॒॒ स्वाहा॒ स्वाहा॒ शीघ्रं॒ ॅवह॑न्तीभ्यः । \newline
13. शीघ्रं॒ ॅवह॑न्तीभ्यो॒ वह॑न्तीभ्यः॒ शीघ्रꣳ॒॒ शीघ्रं॒ ॅवह॑न्तीभ्यः॒ स्वाहा॒ स्वाहा॒ वह॑न्तीभ्यः॒ शीघ्रꣳ॒॒ शीघ्रं॒ ॅवह॑न्तीभ्यः॒ स्वाहा᳚ । \newline
14. वह॑न्तीभ्यः॒ स्वाहा॒ स्वाहा॒ वह॑न्तीभ्यो॒ वह॑न्तीभ्यः॒ स्वाहा॒ शीभꣳ॒॒ शीभꣳ॒॒ स्वाहा॒ वह॑न्तीभ्यो॒ वह॑न्तीभ्यः॒ स्वाहा॒ शीभ᳚म् । \newline
15. स्वाहा॒ शीभꣳ॒॒ शीभꣳ॒॒ स्वाहा॒ स्वाहा॒ शीभं॒ ॅवह॑न्तीभ्यो॒ वह॑न्तीभ्यः॒ शीभꣳ॒॒ स्वाहा॒ स्वाहा॒ शीभं॒ ॅवह॑न्तीभ्यः । \newline
16. शीभं॒ ॅवह॑न्तीभ्यो॒ वह॑न्तीभ्यः॒ शीभꣳ॒॒ शीभं॒ ॅवह॑न्तीभ्यः॒ स्वाहा॒ स्वाहा॒ वह॑न्तीभ्यः॒ शीभꣳ॒॒ शीभं॒ ॅवह॑न्तीभ्यः॒ स्वाहा᳚ । \newline
17. वह॑न्तीभ्यः॒ स्वाहा॒ स्वाहा॒ वह॑न्तीभ्यो॒ वह॑न्तीभ्यः॒ स्वाहो॒ग्र मु॒ग्रꣳ स्वाहा॒ वह॑न्तीभ्यो॒ वह॑न्तीभ्यः॒ स्वाहो॒ग्रम् । \newline
18. स्वाहो॒ग्र मु॒ग्रꣳ स्वाहा॒ स्वाहो॒ग्रं ॅवह॑न्तीभ्यो॒ वह॑न्तीभ्य उ॒ग्रꣳ स्वाहा॒ स्वाहो॒ग्रं ॅवह॑न्तीभ्यः । \newline
19. उ॒ग्रं ॅवह॑न्तीभ्यो॒ वह॑न्तीभ्य उ॒ग्र मु॒ग्रं ॅवह॑न्तीभ्यः॒ स्वाहा॒ स्वाहा॒ वह॑न्तीभ्य उ॒ग्र मु॒ग्रं ॅवह॑न्तीभ्यः॒ स्वाहा᳚ । \newline
20. वह॑न्तीभ्यः॒ स्वाहा॒ स्वाहा॒ वह॑न्तीभ्यो॒ वह॑न्तीभ्यः॒ स्वाहा॑ भी॒मम् भी॒मꣳ स्वाहा॒ वह॑न्तीभ्यो॒ वह॑न्तीभ्यः॒ स्वाहा॑ भी॒मम् । \newline
21. स्वाहा॑ भी॒मम् भी॒मꣳ स्वाहा॒ स्वाहा॑ भी॒मं ॅवह॑न्तीभ्यो॒ वह॑न्तीभ्यो भी॒मꣳ स्वाहा॒ स्वाहा॑ भी॒मं ॅवह॑न्तीभ्यः । \newline
22. भी॒मं ॅवह॑न्तीभ्यो॒ वह॑न्तीभ्यो भी॒मम् भी॒मं ॅवह॑न्तीभ्यः॒ स्वाहा॒ स्वाहा॒ वह॑न्तीभ्यो भी॒मम् भी॒मं ॅवह॑न्तीभ्यः॒ स्वाहा᳚ । \newline
23. वह॑न्तीभ्यः॒ स्वाहा॒ स्वाहा॒ वह॑न्तीभ्यो॒ वह॑न्तीभ्यः॒ स्वाहा ऽम्भो॒भ्यो ऽम्भो᳚भ्यः॒ स्वाहा॒ वह॑न्तीभ्यो॒ वह॑न्तीभ्यः॒ स्वाहा ऽम्भो᳚भ्यः । \newline
24. स्वाहा ऽम्भो॒भ्यो ऽम्भो᳚भ्यः॒ स्वाहा॒ स्वाहा ऽम्भो᳚भ्यः॒ स्वाहा॒ स्वाहा ऽम्भो᳚भ्यः॒ स्वाहा॒ स्वाहा ऽम्भो᳚भ्यः॒ स्वाहा᳚ । \newline
25. अम्भो᳚भ्यः॒ स्वाहा॒ स्वाहा ऽम्भो॒भ्यो ऽम्भो᳚भ्यः॒ स्वाहा॒ नभो᳚भ्यो॒ नभो᳚भ्यः॒ स्वाहा ऽम्भो॒भ्यो ऽम्भो᳚भ्यः॒ स्वाहा॒ नभो᳚भ्यः । \newline
26. अम्भो᳚भ्य॒ इत्यम्भः॑ - भ्यः॒ । \newline
27. स्वाहा॒ नभो᳚भ्यो॒ नभो᳚भ्यः॒ स्वाहा॒ स्वाहा॒ नभो᳚भ्यः॒ स्वाहा॒ स्वाहा॒ नभो᳚भ्यः॒ स्वाहा॒ स्वाहा॒ नभो᳚भ्यः॒ स्वाहा᳚ । \newline
28. नभो᳚भ्यः॒ स्वाहा॒ स्वाहा॒ नभो᳚भ्यो॒ नभो᳚भ्यः॒ स्वाहा॒ महो᳚भ्यो॒ महो᳚भ्यः॒ स्वाहा॒ नभो᳚भ्यो॒ नभो᳚भ्यः॒ स्वाहा॒ महो᳚भ्यः । \newline
29. नभो᳚भ्य॒ इति॒ नभः॑ - भ्यः॒ । \newline
30. स्वाहा॒ महो᳚भ्यो॒ महो᳚भ्यः॒ स्वाहा॒ स्वाहा॒ महो᳚भ्यः॒ स्वाहा॒ स्वाहा॒ महो᳚भ्यः॒ स्वाहा॒ स्वाहा॒ महो᳚भ्यः॒ स्वाहा᳚ । \newline
31. महो᳚भ्यः॒ स्वाहा॒ स्वाहा॒ महो᳚भ्यो॒ महो᳚भ्यः॒ स्वाहा॒ सर्व॑स्मै॒ सर्व॑स्मै॒ स्वाहा॒ महो᳚भ्यो॒ महो᳚भ्यः॒ स्वाहा॒ सर्व॑स्मै । \newline
32. महो᳚भ्य॒ इति॒ महः॑ - भ्यः॒ । \newline
33. स्वाहा॒ सर्व॑स्मै॒ सर्व॑स्मै॒ स्वाहा॒ स्वाहा॒ सर्व॑स्मै॒ स्वाहा॒ स्वाहा॒ सर्व॑स्मै॒ स्वाहा॒ स्वाहा॒ सर्व॑स्मै॒ स्वाहा᳚ । \newline
34. सर्व॑स्मै॒ स्वाहा॒ स्वाहा॒ सर्व॑स्मै॒ सर्व॑स्मै॒ स्वाहा᳚ । \newline
35. स्वाहेति॒ स्वाहा᳚ । \newline
\pagebreak
\markright{ TS 7.4.15.1  \hfill https://www.vedavms.in \hfill}

\section{ TS 7.4.15.1 }

\textbf{TS 7.4.15.1 } \newline
\textbf{Samhita Paata} \newline

यो अर्व॑न्तं॒ जिघाꣳ॑सति॒ तम॒भ्य॑मीति॒ वरु॑णः ॥ प॒रो मर्तः॑ प॒रः श्वा ॥अ॒हं च॒ त्वं च॑ वृत्रह॒न्थ्सं ब॑भूव स॒निभ्य॒ आ । अ॒रा॒ती॒वा चि॑दद्रि॒वोऽनु॑ नौ शूर मꣳसतै भ॒द्रा इन्द्र॑स्य रा॒तयः॑ ॥अ॒भि क्रत्वे᳚न्द्र भू॒रध॒ ज्मन्न ते॑ विव्यङ्महि॒मानꣳ॒॒ रजाꣳ॑सि । स्वेना॒ हि वृ॒त्रꣳ शव॑सा ज॒घन्थ॒ न शत्रु॒रन्तं॑ ॅविविदद् ( ) यु॒धा ते᳚ ॥ \newline

\textbf{Pada Paata} \newline

यः । अर्व॑न्तम् । जिघाꣳ॑सति । तम् । अ॒भीति॑ । अ॒मी॒ति॒ । वरु॑णः ॥ प॒रः । मर्तः॑ । प॒रः । श्वा ॥ अ॒हम् । च॒ । त्वम् । च॒ । वृ॒त्र॒ह॒न्निति॑ वृत्र - ह॒न्न् । समिति॑ । ब॒भू॒व॒ । स॒निभ्य॒ इति॑ स॒नि - भ्यः॒ । आ ॥ अ॒रा॒ती॒वा । चि॒त् । अ॒द्रि॒व॒ इत्य॑द्रि - वः॒ । अन्विति॑ । नौ॒ । शू॒र॒ । मꣳ॒॒स॒तै॒ । भ॒द्राः । इन्द्र॑स्य । रा॒तयः॑ ॥ अ॒भीति॑ । क्रत्वा᳚ । इ॒न्द्र॒ । भूः॒ । अध॑ । ज्मन्न् । न । ते॒ । वि॒व्य॒क् । म॒हि॒मान᳚म् । रजाꣳ॑सि ॥ स्वेन॑ । हि । वृ॒त्रम् । शव॑सा । ज॒घन्थ॑ । न । शत्रुः॑ । अन्त᳚म् । वि॒वि॒द॒त् ( ) । यु॒धा । ते॒ ॥  \newline


\textbf{Krama Paata} \newline

यो अर्व॑न्तम् । अर्व॑न्त॒म् जिघाꣳ॑सति । जिघाꣳ॑सति॒ तम् । तम॒भि । अ॒भ्य॑मीति । अ॒मी॒ति॒ वरु॑णः । वरु॑ण॒ इति॒ वरु॑णः ॥ प॒रो मर्तः॑ । मर्तः॑ प॒रः । प॒रः श्वा । श्वेति॒ श्वा ॥ अ॒हम् च॑ । च॒ त्वम् । त्वम् च॑ । च॒ वृ॒त्र॒ह॒न्न्॒ । वृ॒त्र॒ह॒न्थ् सम् । वृ॒त्र॒ह॒न्निति॑ वृत्र - ह॒न्न्॒ । सम् ब॑भूव । ब॒भू॒व॒ स॒निभ्यः॑ । स॒निभ्य॒ आ । स॒निभ्य॒ इति॑ स॒नि - भ्यः॒ । एत्या ॥ अ॒रा॒ती॒वा चि॑त् । चि॒द॒द्रि॒वः॒ । अ॒द्रि॒वोऽनु॑ । अ॒द्रि॒व॒ इत्य॑द्रि - वः॒ । अनु॑ नौ । नौ॒ शू॒र॒ । शू॒र॒ मꣳ॒॒स॒तै॒ । मꣳ॒॒स॒तै॒ भ॒द्राः । भ॒द्रा इन्द्र॑स्य । इन्द्र॑स्य रा॒तयः॑ । रा॒तय॒ इति॑ रा॒तयः॑ ॥ अ॒भि क्रत्वा᳚ । क्रत्वे᳚न्द्र । इ॒न्द्र॒ भूः॒ । भू॒रध॑ । अध॒ ज्मन्न् । ज्मन् न । न ते᳚ । ते॒ वि॒व्य॒क्॒ । वि॒व्य॒ङ्‍ म॒हि॒मान᳚म् । म॒हि॒मानꣳ॒॒ रजाꣳ॑सि । रजाꣳ॒॒सीति॒ रजाꣳ॑सि ॥ स्वेना॒ हि । हि वृ॒त्रम् । वृ॒त्रꣳ शव॑सा । शव॑सा ज॒घन्थ॑ । ज॒घन्थ॒ न । न शत्रुः॑ । शत्रु॒रन्त᳚म् । अन्त॑म् ॅविविदत् ( ) । वि॒वि॒द॒द् यु॒धा । यु॒धा ते᳚ । त॒ इति॑ ते । \newline

\textbf{Jatai Paata} \newline

1. यो अर्व॑न्त॒ मर्व॑न्तं॒ ॅयो यो अर्व॑न्तम् । \newline
2. अर्व॑न्त॒म् जिघाꣳ॑सति॒ जिघाꣳ॑स॒ त्यर्व॑न्त॒ मर्व॑न्त॒म् जिघाꣳ॑सति । \newline
3. जिघाꣳ॑सति॒ तम् तम् जिघाꣳ॑सति॒ जिघाꣳ॑सति॒ तम् । \newline
4. त म॒भ्य॑भि तम् त म॒भि । \newline
5. अ॒भ्य॑मी त्यमी त्य॒भ्या᳚(1॒) भ्य॑मीति । \newline
6. अ॒मी॒ति॒ वरु॑णो॒ वरु॑णो ऽमीत्य मीति॒ वरु॑णः । \newline
7. वरु॑ण॒ इति॒ वरु॑णः । \newline
8. प॒रो मर्तो॒ मर्तः॑ प॒रः प॒रो मर्तः॑ । \newline
9. मर्तः॑ प॒रः प॒रो मर्तो॒ मर्तः॑ प॒रः । \newline
10. प॒रः श्वा श्वा प॒रः प॒रः श्वा । \newline
11. श्वेति॒ श्वा । \newline
12. अ॒हम् च॑ चा॒ह म॒हम् च॑ । \newline
13. च॒ त्वम् त्वम् च॑ च॒ त्वम् । \newline
14. त्वम् च॑ च॒ त्वम् त्वम् च॑ । \newline
15. च॒ वृ॒त्र॒ह॒न् वृ॒त्र॒हꣳ॒॒ श्च॒ च॒ वृ॒त्र॒ह॒न्न् । \newline
16. वृ॒त्र॒ह॒न् थ्सꣳ सं ॅवृ॑त्रहन् वृत्रह॒न् थ्सम् । \newline
17. वृ॒त्र॒ह॒न्निति॑ वृत्र - ह॒न्न् । \newline
18. सम् ब॑भूव बभूव॒ सꣳ सम् ब॑भूव । \newline
19. ब॒भू॒व॒ स॒निभ्यः॑ स॒निभ्यो॑ बभूव बभूव स॒निभ्यः॑ । \newline
20. स॒निभ्य॒ आ स॒निभ्यः॑ स॒निभ्य॒ आ । \newline
21. स॒निभ्य॒ इति॑ स॒नि - भ्यः॒ । \newline
22. एत्या । \newline
23. अ॒रा॒ती॒वा चि॑च् चिदराती॒वा ऽरा॑ती॒वा चि॑त् । \newline
24. चि॒द॒द्रि॒वो॒ ऽद्रि॒व॒ श्चि॒च् चि॒द॒द्रि॒वः॒ । \newline
25. अ॒द्रि॒वो ऽन्वन् व॑द्रिवो ऽद्रि॒वो ऽनु॑ । \newline
26. अ॒द्रि॒व॒ इत्य॑द्रि - वः॒ । \newline
27. अनु॑ नौ ना॒ वन् वनु॑ नौ । \newline
28. नौ॒ शू॒र॒ शू॒र॒ नौ॒ नौ॒ शू॒र॒ । \newline
29. शू॒र॒ मꣳ॒॒स॒तै॒ मꣳ॒॒स॒तै॒ शू॒र॒ शू॒र॒ मꣳ॒॒स॒तै॒ । \newline
30. मꣳ॒॒स॒तै॒ भ॒द्रा भ॒द्रा मꣳ॑सतै मꣳसतै भ॒द्राः । \newline
31. भ॒द्रा इन्द्र॒ स्येन्द्र॑स्य भ॒द्रा भ॒द्रा इन्द्र॑स्य । \newline
32. इन्द्र॑स्य रा॒तयो॑ रा॒तय॒ इन्द्र॒ स्येन्द्र॑स्य रा॒तयः॑ । \newline
33. रा॒तय॒ इति॑ रा॒तयः॑ । \newline
34. अ॒भि क्रत्वा॒ क्रत्वा॒ ऽभ्य॑भि क्रत्वा᳚ । \newline
35. क्रत्वे᳚ न्द्रे न्द्र॒ क्रत्वा॒ क्रत्वे᳚न्द्र । \newline
36. इ॒न्द्र॒ भू॒र् भू॒ रि॒न्द्रे॒न्द्र॒ भूः॒ । \newline
37. भू॒ रधाध॑ भूर् भू॒ रध॑ । \newline
38. अध॒ ज्मन् ज्मन् नधाध॒ ज्मन्न् । \newline
39. ज्मन् न न ज्मन् ज्मन् न । \newline
40. न ते॑ ते॒ न न ते᳚ । \newline
41. ते॒ वि॒व्य॒ग् वि॒व्य॒क् ते॒ ते॒ वि॒व्य॒क् । \newline
42. वि॒व्य॒ङ् म॒हि॒मान॑म् महि॒मानं॑ ॅविव्यग् विव्यङ् महि॒मान᳚म् । \newline
43. म॒हि॒मानꣳ॒॒ रजाꣳ॑सि॒ रजाꣳ॑सि महि॒मान॑म् महि॒मानꣳ॒॒ रजाꣳ॑सि । \newline
44. रजाꣳ॒॒सीति॒ रजाꣳ॑सि । \newline
45. स्वेन॒ हि हि स्वेन॒ स्वेन॒ हि । \newline
46. हि वृ॒त्रं ॅवृ॒त्रꣳ हि हि वृ॒त्रम् । \newline
47. वृ॒त्रꣳ शव॑सा॒ शव॑सा वृ॒त्रं ॅवृ॒त्रꣳ शव॑सा । \newline
48. शव॑सा ज॒घन्थ॑ ज॒घन्थ॒ शव॑सा॒ शव॑सा ज॒घन्थ॑ । \newline
49. ज॒घन्थ॒ न न ज॒घन्थ॑ ज॒घन्थ॒ न । \newline
50. न शत्रुः॒ शत्रु॒र् न न शत्रुः॑ । \newline
51. शत्रु॒ रन्त॒ मन्तꣳ॒॒ शत्रुः॒ शत्रु॒ रन्त᳚म् । \newline
52. अन्तं॑ ॅविविदद् विविद॒ दन्त॒ मन्तं॑ ॅविविदत् । \newline
53. वि॒वि॒द॒द् यु॒धा यु॒धा वि॑विदद् विविदद् यु॒धा । \newline
54. यु॒धा ते॑ ते यु॒धा यु॒धा ते᳚ । \newline
55. त॒ इति॑ ते । \newline

\textbf{Ghana Paata } \newline

1. यो अर्व॑न्त॒ मर्व॑न्तं॒ ॅयो यो अर्व॑न्त॒म् जिघाꣳ॑सति॒ जिघाꣳ॑स॒ त्यर्व॑न्तं॒ ॅयो यो अर्व॑न्त॒म् जिघाꣳ॑सति । \newline
2. अर्व॑न्त॒म् जिघाꣳ॑सति॒ जिघाꣳ॑स॒ त्यर्व॑न्त॒ मर्व॑न्त॒म् जिघाꣳ॑सति॒ तम् तम् जिघाꣳ॑स॒ त्यर्व॑न्त॒ मर्व॑न्त॒म् जिघाꣳ॑सति॒ तम् । \newline
3. जिघाꣳ॑सति॒ तम् तम् जिघाꣳ॑सति॒ जिघाꣳ॑सति॒ त म॒भ्य॑भि तम् जिघाꣳ॑सति॒ जिघाꣳ॑सति॒ त म॒भि । \newline
4. त म॒भ्य॑भि तम् त म॒भ्य॑मी त्यमी त्य॒भि तम् त म॒भ्य॑मीति । \newline
5. अ॒भ्य॑ मीत्यमी त्य॒भ्या᳚(1॒)भ्य॑मीति॒ वरु॑णो॒ वरु॑णो ऽमीत्य॒भ्या᳚(1॒) भ्य॑मीति॒ वरु॑णः । \newline
6. अ॒मी॒ति॒ वरु॑णो॒ वरु॑णो ऽमीत्यमीति॒ वरु॑णः । \newline
7. वरु॑ण॒ इति॒ वरु॑णः । \newline
8. प॒रो मर्तो॒ मर्तः॑ प॒रः प॒रो मर्तः॑ प॒रः प॒रो मर्तः॑ प॒रः प॒रो मर्तः॑ प॒रः । \newline
9. मर्तः॑ प॒रः प॒रो मर्तो॒ मर्तः॑ प॒रः श्वा श्वा प॒रो मर्तो॒ मर्तः॑ प॒रः श्वा । \newline
10. प॒रः श्वा श्वा प॒रः प॒रः श्वा । \newline
11. श्वेति॒ श्वा । \newline
12. अ॒हम् च॑ चा॒ह म॒हम् च॒ त्वम् त्वम् चा॒ह म॒हम् च॒ त्वम् । \newline
13. च॒ त्वम् त्वम् च॑ च॒ त्वम् च॑ च॒ त्वम् च॑ च॒ त्वम् च॑ । \newline
14. त्वम् च॑ च॒ त्वम् त्वम् च॑ वृत्रहन् वृत्रहꣳश्च॒ त्वम् त्वम् च॑ वृत्रहन्न् । \newline
15. च॒ वृ॒त्र॒ह॒न् वृ॒त्र॒हꣳ॒॒श्च॒ च॒ वृ॒त्र॒ह॒न् थ्सꣳ सं ॅवृ॑त्रहꣳश्च च वृत्रह॒न् थ्सम् । \newline
16. वृ॒त्र॒ह॒न् थ्सꣳ सं ॅवृ॑त्रहन् वृत्रह॒न् थ्सम् ब॑भूव बभूव॒ सं ॅवृ॑त्रहन् वृत्रह॒न् थ्सम् ब॑भूव । \newline
17. वृ॒त्र॒ह॒न्निति॑ वृत्र - ह॒न्न् । \newline
18. सम् ब॑भूव बभूव॒ सꣳ सम् ब॑भूव स॒निभ्यः॑ स॒निभ्यो॑ बभूव॒ सꣳ सम् ब॑भूव स॒निभ्यः॑ । \newline
19. ब॒भू॒व॒ स॒निभ्यः॑ स॒निभ्यो॑ बभूव बभूव स॒निभ्य॒ आ स॒निभ्यो॑ बभूव बभूव स॒निभ्य॒ आ । \newline
20. स॒निभ्य॒ आ स॒निभ्यः॑ स॒निभ्य॒ आ । \newline
21. स॒निभ्य॒ इति॑ स॒नि - भ्यः॒ । \newline
22. एत्या । \newline
23. अ॒रा॒ती॒वा चि॑च् चिदराती॒वा ऽरा॑ती॒वा चि॑दद्रिवो ऽद्रिव श्चिदराती॒वा ऽरा॑ती॒वा चि॑दद्रिवः । \newline
24. चि॒द॒द्रि॒वो॒ ऽद्रि॒व॒श्चि॒च् चि॒द॒द्रि॒वो ऽन्वन् व॑द्रिव श्चिच् चिदद्रि॒वो ऽनु॑ । \newline
25. अ॒द्रि॒वो ऽन्वन् व॑द्रिवो ऽद्रि॒वो ऽनु॑ नौ ना॒ वन् व॑द्रिवो ऽद्रि॒वो ऽनु॑ नौ । \newline
26. अ॒द्रि॒व॒ इत्य॑द्रि - वः॒ । \newline
27. अनु॑ नौ ना॒ वन् वनु॑ नौ शूर शूर ना॒ वन् वनु॑ नौ शूर । \newline
28. नौ॒ शू॒र॒ शू॒र॒ नौ॒ नौ॒ शू॒र॒ मꣳ॒॒स॒तै॒ मꣳ॒॒स॒तै॒ शू॒र॒ नौ॒ नौ॒ शू॒र॒ मꣳ॒॒स॒तै॒ । \newline
29. शू॒र॒ मꣳ॒॒स॒तै॒ मꣳ॒॒स॒तै॒ शू॒र॒ शू॒र॒ मꣳ॒॒स॒तै॒ भ॒द्रा भ॒द्रा मꣳ॑सतै शूर शूर मꣳसतै भ॒द्राः । \newline
30. मꣳ॒॒स॒तै॒ भ॒द्रा भ॒द्रा मꣳ॑सतै मꣳसतै भ॒द्रा इन्द्र॒ स्येन्द्र॑स्य भ॒द्रा मꣳ॑सतै मꣳसतै भ॒द्रा इन्द्र॑स्य । \newline
31. भ॒द्रा इन्द्र॒ स्येन्द्र॑स्य भ॒द्रा भ॒द्रा इन्द्र॑स्य रा॒तयो॑ रा॒तय॒ इन्द्र॑स्य भ॒द्रा भ॒द्रा इन्द्र॑स्य रा॒तयः॑ । \newline
32. इन्द्र॑स्य रा॒तयो॑ रा॒तय॒ इन्द्र॒ स्येन्द्र॑स्य रा॒तयः॑ । \newline
33. रा॒तय॒ इति॑ रा॒तयः॑ । \newline
34. अ॒भि क्रत्वा॒ क्रत्वा॒ ऽभ्य॑भि क्रत्वे᳚न्द्रे न्द्र॒ क्रत्वा॒ ऽभ्य॑भि क्रत्वे᳚न्द्र । \newline
35. क्रत्वे᳚न्द्रे न्द्र॒ क्रत्वा॒ क्रत्वे᳚न्द्र भूर् भूरिन्द्र॒ क्रत्वा॒ क्रत्वे᳚न्द्र भूः । \newline
36. इ॒न्द्र॒ भू॒र् भू॒रि॒न्द्रे॒ न्द्र॒ भू॒र धाध॑ भूरिन्द्रे न्द्र भू॒रध॑ । \newline
37. भू॒र धाध॑ भूर् भू॒रध॒ ज्मन् ज्मन् नध॑ भूर् भू॒रध॒ ज्मन्न् । \newline
38. अध॒ ज्मन् ज्मन् नधाध॒ ज्मन् न न ज्मन् नधाध॒ ज्मन् न । \newline
39. ज्मन् न न ज्मन् ज्मन् न ते॑ ते॒ न ज्मन् ज्मन् न ते᳚ । \newline
40. न ते॑ ते॒ न न ते॑ विव्यग् विव्यक् ते॒ न न ते॑ विव्यक् । \newline
41. ते॒ वि॒व्य॒ग् वि॒व्य॒क् ते॒ ते॒ वि॒व्य॒ङ् म॒हि॒मान॑म् महि॒मानं॑ ॅविव्यक् ते ते विव्यङ् महि॒मान᳚म् । \newline
42. वि॒व्य॒ङ् म॒हि॒मान॑म् महि॒मानं॑ ॅविव्यग् विव्यङ् महि॒मानꣳ॒॒ रजाꣳ॑सि॒ रजाꣳ॑सि महि॒मानं॑ ॅविव्यग् विव्यङ् महि॒मानꣳ॒॒ रजाꣳ॑सि । \newline
43. म॒हि॒मानꣳ॒॒ रजाꣳ॑सि॒ रजाꣳ॑सि महि॒मान॑म् महि॒मानꣳ॒॒ रजाꣳ॑सि । \newline
44. रजाꣳ॒॒सीति॒ रजाꣳ॑सि । \newline
45. स्वेन॒ हि हि स्वेन॒ स्वेन॒ हि वृ॒त्रं ॅवृ॒त्रꣳ हि स्वेन॒ स्वेन॒ हि वृ॒त्रम् । \newline
46. हि वृ॒त्रं ॅवृ॒त्रꣳ हि हि वृ॒त्रꣳ शव॑सा॒ शव॑सा वृ॒त्रꣳ हि हि वृ॒त्रꣳ शव॑सा । \newline
47. वृ॒त्रꣳ शव॑सा॒ शव॑सा वृ॒त्रं ॅवृ॒त्रꣳ शव॑सा ज॒घन्थ॑ ज॒घन्थ॒ शव॑सा वृ॒त्रं ॅवृ॒त्रꣳ शव॑सा ज॒घन्थ॑ । \newline
48. शव॑सा ज॒घन्थ॑ ज॒घन्थ॒ शव॑सा॒ शव॑सा ज॒घन्थ॒ न न ज॒घन्थ॒ शव॑सा॒ शव॑सा ज॒घन्थ॒ न । \newline
49. ज॒घन्थ॒ न न ज॒घन्थ॑ ज॒घन्थ॒ न शत्रुः॒ शत्रु॒र् न ज॒घन्थ॑ ज॒घन्थ॒ न शत्रुः॑ । \newline
50. न शत्रुः॒ शत्रु॒र् न न शत्रु॒ रन्त॒ मन्तꣳ॒॒ शत्रु॒र् न न शत्रु॒ रन्त᳚म् । \newline
51. शत्रु॒ रन्त॒ मन्तꣳ॒॒ शत्रुः॒ शत्रु॒ रन्तं॑ ॅविविदद् विविद॒ दन्तꣳ॒॒ शत्रुः॒ शत्रु॒ रन्तं॑ ॅविविदत् । \newline
52. अन्तं॑ ॅविविदद् विविद॒ दन्त॒ मन्तं॑ ॅविविदद् यु॒धा यु॒धा वि॑विद॒ दन्त॒ मन्तं॑ ॅविविदद् यु॒धा । \newline
53. वि॒वि॒द॒द् यु॒धा यु॒धा वि॑विदद् विविदद् यु॒धा ते॑ ते यु॒धा वि॑विदद् विविदद् यु॒धा ते᳚ । \newline
54. यु॒धा ते॑ ते यु॒धा यु॒धा ते᳚ । \newline
55. त॒ इति॑ ते । \newline
\pagebreak
\markright{ TS 7.4.16.1  \hfill https://www.vedavms.in \hfill}

\section{ TS 7.4.16.1 }

\textbf{TS 7.4.16.1 } \newline
\textbf{Samhita Paata} \newline

नमो॒ राज्ञे॒ नमो॒ वरु॑णाय॒ नमोऽश्वा॑य॒ नमः॑ प्र॒जाप॑तये॒ नमोऽधि॑पत॒ये ऽधि॑पतिर॒स्यधि॑पतिं मा कु॒र्वधि॑पतिर॒हं प्र॒जानां᳚ भूयासं॒ मां धे॑हि॒ मयि॑ धेह्यु॒पाकृ॑ताय॒ स्वाहा ऽऽल॑ब्धाय॒ स्वाहा॑ हु॒ताय॒ स्वाहा᳚ ॥ \newline

\textbf{Pada Paata} \newline

नमः॑ । राज्ञे᳚ । नमः॑ । वरु॑णाय । नमः॑ । अश्वा॑य । नमः॑ । प्र॒जाप॑तय॒ इति॑ प्र॒जा - प॒त॒ये॒ । नमः॑ । अधि॑पतय॒ इत्यधि॑ - प॒त॒ये॒ । अधि॑पति॒रित्यधि॑ - प॒तिः॒ । अ॒सि॒ । अधि॑पति॒मित्यधि॑ - प॒ति॒म् । मा॒ । कु॒रु॒ । अधि॑पति॒रित्यधि॑ - प॒तिः॒ । अ॒हम् । प्र॒जाना॒मिति॑ प्र - जाना᳚म् । भू॒या॒स॒म् । माम् । धे॒हि॒ । मयि॑ । धे॒हि॒ । उ॒पाकृ॑ता॒येत्यु॑प - आकृ॑ताय । स्वाहा᳚ । आल॑ब्धा॒येत्या - ल॒ब्धा॒य॒ । स्वाहा᳚ । हु॒ताय॑ । स्वाहा᳚ ॥  \newline


\textbf{Krama Paata} \newline

नमो॒ राज्ञे᳚ । राज्ञे॒ नमः॑ । नमो॒ वरु॑णाय । वरु॑णाय॒ नमः॑ । नमोऽश्वा॑य । अश्वा॑य॒ नमः॑ । नमः॑ प्र॒जाप॑तये । प्र॒जाप॑तये॒ नमः॑ । प्र॒जाप॑तय॒ इति॑ प्र॒जा - प॒त॒ये॒ । नमोऽधि॑पतये । अधि॑पत॒येऽधि॑पतिः । अधि॑पतय॒ इत्यधि॑ - प॒त॒ये॒ । अधि॑पतिरसि । अधि॑पति॒रित्यधि॑ - प॒तिः॒ । अ॒स्यधि॑पतिम् । अधि॑पतिम् मा । अधि॑पति॒मित्यधि॑ - प॒ति॒म् । मा॒ कु॒रु॒ । कु॒र्वधि॑पतिः । अधि॑पतिर॒हम् । अधि॑पति॒रित्यधि॑ - प॒तिः॒ । अ॒हम् प्र॒जाना᳚म् । प्र॒जाना᳚म् भूयासम् । प्र॒जाना॒मिति॑ प्र - जाना᳚म् । भू॒या॒स॒म् माम् । माम् धे॑हि । धे॒हि॒ मयि॑ । मयि॑ धेहि । धे॒ह्यु॒पाकृ॑ताय । उ॒पाकृ॑ताय॒ स्वाहा᳚ । उ॒पाकृ॑ता॒येत्यु॑प - आकृ॑ताय । स्वाहाऽऽल॑ब्धाय । आल॑ब्धाय॒ स्वाहा᳚ । आल॑ब्धा॒येत्या - ल॒ब्धा॒य॒ । स्वाहा॑ हु॒ताय॑ । हु॒ताय॒ स्वाहा᳚ । स्वाहेति॒ स्वाहा᳚ । \newline

\textbf{Jatai Paata} \newline

1. नमो॒ राज्ञे॒ राज्ञे॒ नमो॒ नमो॒ राज्ञे᳚ । \newline
2. राज्ञे॒ नमो॒ नमो॒ राज्ञे॒ राज्ञे॒ नमः॑ । \newline
3. नमो॒ वरु॑णाय॒ वरु॑णाय॒ नमो॒ नमो॒ वरु॑णाय । \newline
4. वरु॑णाय॒ नमो॒ नमो॒ वरु॑णाय॒ वरु॑णाय॒ नमः॑ । \newline
5. नमो ऽश्वा॒या श्वा॑य॒ नमो॒ नमो ऽश्वा॑य । \newline
6. अश्वा॑य॒ नमो॒ नमो ऽश्वा॒या श्वा॑य॒ नमः॑ । \newline
7. नमः॑ प्र॒जाप॑तये प्र॒जाप॑तये॒ नमो॒ नमः॑ प्र॒जाप॑तये । \newline
8. प्र॒जाप॑तये॒ नमो॒ नमः॑ प्र॒जाप॑तये प्र॒जाप॑तये॒ नमः॑ । \newline
9. प्र॒जाप॑तय॒ इति॑ प्र॒जा - प॒त॒ये॒ । \newline
10. नमो ऽधि॑पत॒ये ऽधि॑पतये॒ नमो॒ नमो ऽधि॑पतये । \newline
11. अधि॑पत॒ये ऽधि॑पति॒ रधि॑पति॒ रधि॑पत॒ये ऽधि॑पत॒ये ऽधि॑पतिः । \newline
12. अधि॑पतय॒ इत्यधि॑ - प॒त॒ये॒ । \newline
13. अधि॑पति रस्य॒स्य धि॑पति॒ रधि॑पति रसि । \newline
14. अधि॑पति॒रित्यधि॑ - प॒तिः॒ । \newline
15. अ॒स्य धि॑पति॒ मधि॑पति मस्य॒स्य धि॑पतिम् । \newline
16. अधि॑पतिम् मा॒ मा ऽधि॑पति॒ मधि॑पतिम् मा । \newline
17. अधि॑पति॒मित्यधि॑ - प॒ति॒म् । \newline
18. मा॒ कु॒रु॒ कु॒रु॒ मा॒ मा॒ कु॒रु॒ । \newline
19. कु॒र्वधि॑पति॒ रधि॑पतिः कुरु कु॒र्वधि॑पतिः । \newline
20. अधि॑पति र॒ह म॒ह मधि॑पति॒ रधि॑पति र॒हम् । \newline
21. अधि॑पति॒रित्यधि॑ - प॒तिः॒ । \newline
22. अ॒हम् प्र॒जाना᳚म् प्र॒जाना॑ म॒ह म॒हम् प्र॒जाना᳚म् । \newline
23. प्र॒जाना᳚म् भूयासम् भूयासम् प्र॒जाना᳚म् प्र॒जाना᳚म् भूयासम् । \newline
24. प्र॒जाना॒मिति॑ प्र - जाना᳚म् । \newline
25. भू॒या॒स॒म् माम् माम् भू॑यासम् भूयास॒म् माम् । \newline
26. माम् धे॑हि धेहि॒ माम् माम् धे॑हि । \newline
27. धे॒हि॒ मयि॒ मयि॑ धेहि धेहि॒ मयि॑ । \newline
28. मयि॑ धेहि धेहि॒ मयि॒ मयि॑ धेहि । \newline
29. धे॒ह्यु॒पाकृ॑ता यो॒पाकृ॑ताय धेहि धेह्यु॒पाकृ॑ताय । \newline
30. उ॒पाकृ॑ताय॒ स्वाहा॒ स्वाहो॒पाकृ॑ता यो॒पाकृ॑ताय॒ स्वाहा᳚ । \newline
31. उ॒पाकृ॑ता॒येत्यु॑प - आकृ॑ताय । \newline
32. स्वाहा ऽऽल॑ब्धा॒या ल॑ब्धाय॒ स्वाहा॒ स्वाहा ऽऽल॑ब्धाय । \newline
33. आल॑ब्धाय॒ स्वाहा॒ स्वाहा ऽऽल॑ब्धा॒या ल॑ब्धाय॒ स्वाहा᳚ । \newline
34. आल॑ब्धा॒येत्या - ल॒ब्धा॒य॒ । \newline
35. स्वाहा॑ हु॒ताय॑ हु॒ताय॒ स्वाहा॒ स्वाहा॑ हु॒ताय॑ । \newline
36. हु॒ताय॒ स्वाहा॒ स्वाहा॑ हु॒ताय॑ हु॒ताय॒ स्वाहा᳚ । \newline
37. स्वाहेति॒ स्वाहा᳚ । \newline

\textbf{Ghana Paata } \newline

1. नमो॒ राज्ञे॒ राज्ञे॒ नमो॒ नमो॒ राज्ञे॒ नमो॒ नमो॒ राज्ञे॒ नमो॒ नमो॒ राज्ञे॒ नमः॑ । \newline
2. राज्ञे॒ नमो॒ नमो॒ राज्ञे॒ राज्ञे॒ नमो॒ वरु॑णाय॒ वरु॑णाय॒ नमो॒ राज्ञे॒ राज्ञे॒ नमो॒ वरु॑णाय । \newline
3. नमो॒ वरु॑णाय॒ वरु॑णाय॒ नमो॒ नमो॒ वरु॑णाय॒ नमो॒ नमो॒ वरु॑णाय॒ नमो॒ नमो॒ वरु॑णाय॒ नमः॑ । \newline
4. वरु॑णाय॒ नमो॒ नमो॒ वरु॑णाय॒ वरु॑णाय॒ नमो ऽश्वा॒या श्वा॑य॒ नमो॒ वरु॑णाय॒ वरु॑णाय॒ नमो ऽश्वा॑य । \newline
5. नमो ऽश्वा॒या श्वा॑य॒ नमो॒ नमो ऽश्वा॑य॒ नमो॒ नमो ऽश्वा॑य॒ नमो॒ नमो ऽश्वा॑य॒ नमः॑ । \newline
6. अश्वा॑य॒ नमो॒ नमो ऽश्वा॒या श्वा॑य॒ नमः॑ प्र॒जाप॑तये प्र॒जाप॑तये॒ नमो ऽश्वा॒या श्वा॑य॒ नमः॑ प्र॒जाप॑तये । \newline
7. नमः॑ प्र॒जाप॑तये प्र॒जाप॑तये॒ नमो॒ नमः॑ प्र॒जाप॑तये॒ नमो॒ नमः॑ प्र॒जाप॑तये॒ नमो॒ नमः॑ प्र॒जाप॑तये॒ नमः॑ । \newline
8. प्र॒जाप॑तये॒ नमो॒ नमः॑ प्र॒जाप॑तये प्र॒जाप॑तये॒ नमो ऽधि॑पत॒ये ऽधि॑पतये॒ नमः॑ प्र॒जाप॑तये प्र॒जाप॑तये॒ नमो ऽधि॑पतये । \newline
9. प्र॒जाप॑तय॒ इति॑ प्र॒जा - प॒त॒ये॒ । \newline
10. नमो ऽधि॑पत॒ये ऽधि॑पतये॒ नमो॒ नमो ऽधि॑पत॒ये ऽधि॑पति॒ रधि॑पति॒ रधि॑पतये॒ नमो॒ नमो ऽधि॑पत॒ये ऽधि॑पतिः । \newline
11. अधि॑पत॒ये ऽधि॑पति॒ रधि॑पति॒ रधि॑पत॒ये ऽधि॑पत॒ये ऽधि॑पति रस्य॒ स्यधि॑पति॒ रधि॑पत॒ये ऽधि॑पत॒ये ऽधि॑पति रसि । \newline
12. अधि॑पतय॒ इत्यधि॑ - प॒त॒ये॒ । \newline
13. अधि॑पति रस्य॒ स्यधि॑पति॒ रधि॑पति र॒स्यधि॑पति॒ मधि॑पति म॒स्यधि॑पति॒ रधि॑पति र॒स्यधि॑पतिम् । \newline
14. अधि॑पति॒रित्यधि॑ - प॒तिः॒ । \newline
15. अ॒स्यधि॑पति॒ मधि॑पति मस्य॒ स्यधि॑पतिम् मा॒ मा ऽधि॑पति मस्य॒ स्यधि॑पतिम् मा । \newline
16. अधि॑पतिम् मा॒ मा ऽधि॑पति॒ मधि॑पतिम् मा कुरु कुरु॒ मा ऽधि॑पति॒ मधि॑पतिम् मा कुरु । \newline
17. अधि॑पति॒मित्यधि॑ - प॒ति॒म् । \newline
18. मा॒ कु॒रु॒ कु॒रु॒ मा॒ मा॒ कु॒र्वधि॑पति॒ रधि॑पतिः कुरु मा मा कु॒र्वधि॑पतिः । \newline
19. कु॒र्वधि॑पति॒ रधि॑पतिः कुरु कु॒र्वधि॑पति र॒ह म॒ह मधि॑पतिः कुरु कु॒र्वधि॑पति र॒हम् । \newline
20. अधि॑पति र॒ह म॒ह मधि॑पति॒ रधि॑पति र॒हम् प्र॒जाना᳚म् प्र॒जाना॑ म॒ह मधि॑पति॒ रधि॑पति र॒हम् प्र॒जाना᳚म् । \newline
21. अधि॑पति॒रित्यधि॑ - प॒तिः॒ । \newline
22. अ॒हम् प्र॒जाना᳚म् प्र॒जाना॑ म॒ह म॒हम् प्र॒जाना᳚म् भूयासम् भूयासम् प्र॒जाना॑ म॒ह म॒हम् प्र॒जाना᳚म् भूयासम् । \newline
23. प्र॒जाना᳚म् भूयासम् भूयासम् प्र॒जाना᳚म् प्र॒जाना᳚म् भूयास॒म् माम् माम् भू॑यासम् प्र॒जाना᳚म् प्र॒जाना᳚म् भूयास॒म् माम् । \newline
24. प्र॒जाना॒मिति॑ प्र - जाना᳚म् । \newline
25. भू॒या॒स॒म् माम् माम् भू॑यासम् भूयास॒म् माम् धे॑हि धेहि॒ माम् भू॑यासम् भूयास॒म् माम् धे॑हि । \newline
26. माम् धे॑हि धेहि॒ माम् माम् धे॑हि॒ मयि॒ मयि॑ धेहि॒ माम् माम् धे॑हि॒ मयि॑ । \newline
27. धे॒हि॒ मयि॒ मयि॑ धेहि धेहि॒ मयि॑ धेहि धेहि॒ मयि॑ धेहि धेहि॒ मयि॑ धेहि । \newline
28. मयि॑ धेहि धेहि॒ मयि॒ मयि॑ धेह्यु॒पाकृ॑ता यो॒पाकृ॑ताय धेहि॒ मयि॒ मयि॑ धेह्यु॒पाकृ॑ताय । \newline
29. धे॒ह्यु॒पाकृ॑ता यो॒पाकृ॑ताय धेहि धेह्यु॒पाकृ॑ताय॒ स्वाहा॒ स्वाहो॒पाकृ॑ताय धेहि धेह्यु॒पाकृ॑ताय॒ स्वाहा᳚ । \newline
30. उ॒पाकृ॑ताय॒ स्वाहा॒ स्वाहो॒पाकृ॑ता यो॒पाकृ॑ताय॒ स्वाहा ऽऽल॑ब्धा॒या ल॑ब्धाय॒ स्वाहो॒पाकृ॑ता
यो॒पाकृ॑ताय॒ स्वाहा ऽऽल॑ब्धाय । \newline
31. उ॒पाकृ॑ता॒येत्यु॑प - आकृ॑ताय । \newline
32. स्वाहा ऽऽल॑ब्धा॒या ल॑ब्धाय॒ स्वाहा॒ स्वाहा ऽऽल॑ब्धाय॒ स्वाहा॒ स्वाहा ऽऽल॑ब्धाय॒ स्वाहा॒ स्वाहा ऽऽल॑ब्धाय॒ स्वाहा᳚ । \newline
33. आल॑ब्धाय॒ स्वाहा॒ स्वाहा ऽऽल॑ब्धा॒या ल॑ब्धाय॒ स्वाहा॑ हु॒ताय॑ हु॒ताय॒ स्वाहा ऽऽल॑ब्धा॒या ल॑ब्धाय॒ स्वाहा॑ हु॒ताय॑ । \newline
34. आल॑ब्धा॒येत्या - ल॒ब्धा॒य॒ । \newline
35. स्वाहा॑ हु॒ताय॑ हु॒ताय॒ स्वाहा॒ स्वाहा॑ हु॒ताय॒ स्वाहा॒ स्वाहा॑ हु॒ताय॒ स्वाहा॒ स्वाहा॑ हु॒ताय॒ स्वाहा᳚ । \newline
36. हु॒ताय॒ स्वाहा॒ स्वाहा॑ हु॒ताय॑ हु॒ताय॒ स्वाहा᳚ । \newline
37. स्वाहेति॒ स्वाहा᳚ । \newline
\pagebreak
\markright{ TS 7.4.17.1  \hfill https://www.vedavms.in \hfill}

\section{ TS 7.4.17.1 }

\textbf{TS 7.4.17.1 } \newline
\textbf{Samhita Paata} \newline

म॒यो॒भूर्वातो॑ अ॒भि वा॑तू॒स्रा ऊर्ज॑स्वती॒रोष॑धी॒रा रि॑शन्तां । पीव॑स्वतीर्जी॒वध॑न्याः पिबन्त्वव॒साय॑ प॒द्वते॑ रुद्र मृड ॥याः सरू॑पा॒ विरू॑पा॒ एक॑रूपा॒ यासा॑म॒ग्निरिष्ट्या॒ नामा॑नि॒ वेद॑ । या अङ्गि॑रस॒स्तप॑से॒ह च॒क्रुस्ताभ्यः॑ पर्जन्य॒ महि॒ शर्म॑ यच्छ ॥या दे॒वेषु॑ त॒नुव॒मैर॑यन्त॒ यासाꣳ॒॒ सोमो॒ विश्वा॑ रू॒पाणि॒ वेद॑ । ता अ॒स्मभ्यं॒ पय॑सा॒ पिन्व॑मानाः प्र॒जाव॑तीरिन्द्र - [  ] \newline

\textbf{Pada Paata} \newline

म॒यो॒भूरिति॑ मयः - भूः । वातः॑ । अ॒भीति॑ । वा॒तु॒ । उ॒स्राः । ऊर्ज॑स्वतीः । ओष॑धीः । एति॑ । रि॒श॒न्ता॒म् ॥ पीव॑स्वतीः । जी॒वध॑न्या॒ इति॑ जी॒व - ध॒न्याः॒ । पि॒ब॒न्तु॒ । अ॒व॒साय॑ । प॒द्वत॒ इति॑ पत् - वते᳚ । रु॒द्र॒ । मृ॒ड॒ ॥ याः । सरू॑पा॒ इति॒ स - रू॒पाः॒ । विरू॑पा॒ इति॒ वि - रू॒पाः॒ । एक॑रूपा॒ इत्येक॑-रू॒पाः॒ । यासा᳚म् । अ॒ग्निः । इष्ट्या᳚ । नामा॑नि । वेद॑ ॥ याः । अङ्गि॑रसः । तप॑सा । इ॒ह । च॒क्रुः । ताभ्यः॑ । प॒र्ज॒न्य॒ । महि॑ । शर्म॑ । य॒च्छ॒ ॥ याः । दे॒वेषु॑ । त॒नुव᳚म् । ऐर॑यन्त । यासा᳚म् । सोमः॑ । विश्वा᳚ । रू॒पाणि॑ । वेद॑ ॥ ताः । अ॒स्मभ्य॒मित्य॒स्म - भ्य॒म् । पय॑सा । पिन्व॑मानाः । प्र॒जाव॑ती॒रिति॑ प्र॒जा - व॒तीः॒ । इ॒न्द्र॒ ।  \newline


\textbf{Krama Paata} \newline

म॒यो॒भूर् वातः॑ । म॒यो॒भूरिति॑ मयः - भूः । वातो॑ अ॒भि । अ॒भि वा॑तु । वा॒तू॒स्राः । उ॒स्रा ऊर्ज॑स्वतीः । ऊर्ज॑स्वती॒रोष॑धीः । ओष॑धी॒रा । आ रि॑शन्ताम् । रि॒श॒न्ता॒मिति॑ रिशन्ताम् ॥ पीव॑स्वतीर् जी॒वध॑न्याः । जी॒वध॑न्याः पिबन्तु । जी॒वध॑न्या॒ इति॑ जी॒व - ध॒न्याः॒ । पि॒ब॒न्त्व॒व॒साय॑ । अ॒व॒साय॑ प॒द्वते᳚ । प॒द्वते॑ रुद्र । प॒द्वत॒ इति॑ पत् - वते᳚ । रु॒द्र॒ मृ॒ड॒ । मृ॒डेति॑ मृड ॥ याः सरू॑पाः । सरू॑पा॒ विरू॑पाः । सरू॑पा॒ इति॒ स - रू॒पाः॒ । विरू॑पा॒ एक॑रूपाः । विरू॑पा॒ इति॒ वि - रू॒पाः॒ । एक॑रूपा॒ यासा᳚म् । एक॑रूपा॒ इत्येक॑ - रू॒पाः॒ । यासा॑म॒ग्निः । अ॒ग्निरिष्ट्‍या᳚ । इष्ट्‍या॒ नामा॑नि । नामा॑नि॒ वेद॑ । वेदेति॒ वेद॑ ॥ या अङ्‍गि॑रसः । अङ्‍गि॑रस॒स्तप॑सा । तप॑से॒ह । इ॒ह च॒क्रुः । च॒क्रुस्ताभ्यः॑ । ताभ्यः॑ पर्जन्य । प॒र्ज॒न्य॒ महि॑ । महि॒ शर्म॑ । शर्म॑ यच्छ । य॒च्छेति॑ यच्छ ॥ या दे॒वेषु॑ । दे॒वेषु॑ त॒नुव᳚म् । त॒नुव॒मैर॑यन्त । ऐर॑यन्त॒ यासा᳚म् । यासाꣳ॒॒ सोमः॑ । सोमो॒ विश्वा᳚ । विश्वा॑ रू॒पाणि॑ । रू॒पाणि॒ वेद॑ । वेदेति॒ वेद॑ ॥ ता अ॒स्मभ्य᳚म् । अ॒स्मभ्य॒म् पय॑सा । अ॒स्मभ्य॒मित्य॒स्म - भ्य॒म् । पय॑सा॒ पिन्व॑मानाः । पिन्व॑मानाः प्र॒जाव॑तीः । प्र॒जाव॑तीरिन्द्र । प्र॒जाव॑ती॒रिति॑ प्र॒जा - व॒तीः॒ । इ॒न्द्र॒ गो॒ष्ठे \newline

\textbf{Jatai Paata} \newline

1. म॒यो॒भूर् वातो॒ वातो॑ मयो॒भूर् म॑यो॒भूर् वातः॑ । \newline
2. म॒यो॒भूरिति॑ मयः - भूः । \newline
3. वातो॑ अ॒भ्य॑भि वातो॒ वातो॑ अ॒भि । \newline
4. अ॒भि वा॑तु वात्व॒ भ्य॑भि वा॑तु । \newline
5. वा॒तू॒स्रा उ॒स्रा वा॑तु वातू॒स्राः । \newline
6. उ॒स्रा ऊर्ज॑स्वती॒ रूर्ज॑स्वती रु॒स्रा उ॒स्रा ऊर्ज॑स्वतीः । \newline
7. ऊर्ज॑स्वती॒ रोष॑धी॒ रोष॑धी॒ रूर्ज॑स्वती॒ रूर्ज॑स्वती॒ रोष॑धीः । \newline
8. ओष॑धी॒ रौष॑धी॒ रोष॑धी॒रा । \newline
9. आ रि॑शन्ताꣳ रिशन्ता॒ मा रि॑शन्ताम् । \newline
10. रि॒श॒न्ता॒मिति॑ रिशन्ताम् । \newline
11. पीव॑स्वतीर् जी॒वध॑न्या जी॒वध॑न्याः॒ पीव॑स्वतीः॒ पीव॑स्वतीर् जी॒वध॑न्याः । \newline
12. जी॒वध॑न्याः पिबन्तु पिबन्तु जी॒वध॑न्या जी॒वध॑न्याः पिबन्तु । \newline
13. जी॒वध॑न्या॒ इति॑ जी॒व - ध॒न्याः॒ । \newline
14. पि॒ब॒ न्त्व॒व॒साया॑ व॒साय॑ पिबन्तु पिबन्त्व व॒साय॑ । \newline
15. अ॒व॒साय॑ प॒द्वते॑ प॒द्वते॑ ऽव॒साया॑ व॒साय॑ प॒द्वते᳚ । \newline
16. प॒द्वते॑ रुद्र रुद्र प॒द्वते॑ प॒द्वते॑ रुद्र । \newline
17. प॒द्वत॒ इति॑ पत् - वते᳚ । \newline
18. रु॒द्र॒ मृ॒ड॒ मृ॒ड॒ रु॒द्र॒ रु॒द्र॒ मृ॒ड॒ । \newline
19. मृ॒डेति॑ मृड । \newline
20. याः सरू॑पाः॒ सरू॑पा॒ या याः सरू॑पाः । \newline
21. सरू॑पा॒ विरू॑पा॒ विरू॑पाः॒ सरू॑पाः॒ सरू॑पा॒ विरू॑पाः । \newline
22. सरू॑पा॒ इति॒ स - रू॒पाः॒ । \newline
23. विरू॑पा॒ एक॑रूपा॒ एक॑रूपा॒ विरू॑पा॒ विरू॑पा॒ एक॑रूपाः । \newline
24. विरू॑पा॒ इति॒ वि - रू॒पाः॒ । \newline
25. एक॑रूपा॒ यासां॒ ॅयासा॒ मेक॑रूपा॒ एक॑रूपा॒ यासा᳚म् । \newline
26. एक॑रूपा॒ इत्येक॑ - रू॒पाः॒ । \newline
27. यासा॑ म॒ग्नि र॒ग्निर् यासां॒ ॅयासा॑ म॒ग्निः । \newline
28. अ॒ग्नि रिष्ट्येष्ट्या॒ ऽग्नि र॒ग्नि रिष्ट्या᳚ । \newline
29. इष्ट्या॒ नामा॑नि॒ नामा॒ नीष्ट्ये ष्ट्या॒ नामा॑नि । \newline
30. नामा॑नि॒ वेद॒ वेद॒ नामा॑नि॒ नामा॑नि॒ वेद॑ । \newline
31. वेदेति॒ वेद॑ । \newline
32. या अङ्गि॑र॒सो ऽङ्गि॑रसो॒ या या अङ्गि॑रसः । \newline
33. अङ्गि॑रस॒ स्तप॑सा॒ तप॒सा ऽङ्गि॑र॒सो ऽङ्गि॑रस॒ स्तप॑सा । \newline
34. तप॑से॒ हेह तप॑सा॒ तप॑से॒ह । \newline
35. इ॒ह च॒क्रु श्च॒क्रु रि॒हेह च॒क्रुः । \newline
36. च॒क्रु स्ताभ्य॒ स्ताभ्य॑ श्च॒क्रु श्च॒क्रु स्ताभ्यः॑ । \newline
37. ताभ्यः॑ पर्जन्य पर्जन्य॒ ताभ्य॒ स्ताभ्यः॑ पर्जन्य । \newline
38. प॒र्ज॒न्य॒ महि॒ महि॑ पर्जन्य पर्जन्य॒ महि॑ । \newline
39. महि॒ शर्म॒ शर्म॒ महि॒ महि॒ शर्म॑ । \newline
40. शर्म॑ यच्छ यच्छ॒ शर्म॒ शर्म॑ यच्छ । \newline
41. य॒च्छेति॑ यच्छ । \newline
42. या दे॒वेषु॑ दे॒वेषु॒ या या दे॒वेषु॑ । \newline
43. दे॒वेषु॑ त॒नुव॑म् त॒नुव॑म् दे॒वेषु॑ दे॒वेषु॑ त॒नुव᳚म् । \newline
44. त॒नुव॒ मैर॑य॒ न्तैर॑यन्त त॒नुव॑म् त॒नुव॒ मैर॑यन्त । \newline
45. ऐर॑यन्त॒ यासां॒ ॅयासा॒ मैर॑य॒ न्तैर॑यन्त॒ यासा᳚म् । \newline
46. यासाꣳ॒॒ सोमः॒ सोमो॒ यासां॒ ॅयासाꣳ॒॒ सोमः॑ । \newline
47. सोमो॒ विश्वा॒ विश्वा॒ सोमः॒ सोमो॒ विश्वा᳚ । \newline
48. विश्वा॑ रू॒पाणि॑ रू॒पाणि॒ विश्वा॒ विश्वा॑ रू॒पाणि॑ । \newline
49. रू॒पाणि॒ वेद॒ वेद॑ रू॒पाणि॑ रू॒पाणि॒ वेद॑ । \newline
50. वेदेति॒ वेद॑ । \newline
51. ता अ॒स्मभ्य॑ म॒स्मभ्य॒म् ता स्ता अ॒स्मभ्य᳚म् । \newline
52. अ॒स्मभ्य॒म् पय॑सा॒ पय॑सा॒ ऽस्मभ्य॑ म॒स्मभ्य॒म् पय॑सा । \newline
53. अ॒स्मभ्य॒मित्य॒स्म - भ्य॒म् । \newline
54. पय॑सा॒ पिन्व॑मानाः॒ पिन्व॑मानाः॒ पय॑सा॒ पय॑सा॒ पिन्व॑मानाः । \newline
55. पिन्व॑मानाः प्र॒जाव॑तीः प्र॒जाव॑तीः॒ पिन्व॑मानाः॒ पिन्व॑मानाः प्र॒जाव॑तीः । \newline
56. प्र॒जाव॑ती रिन्द्रेन्द्र प्र॒जाव॑तीः प्र॒जाव॑ती रिन्द्र । \newline
57. प्र॒जाव॑ती॒रिति॑ प्र॒जा - व॒तीः॒ । \newline
58. इ॒न्द्र॒ गो॒ष्ठे गो॒ष्ठ इ॑न्द्रेन्द्र गो॒ष्ठे । \newline

\textbf{Ghana Paata } \newline

1. म॒यो॒भूर् वातो॒ वातो॑ मयो॒भूर् म॑यो॒भूर् वातो॑ अ॒भ्य॑भि वातो॑ मयो॒भूर् म॑यो॒भूर् वातो॑ अ॒भि । \newline
2. म॒यो॒भूरिति॑ मयः - भूः । \newline
3. वातो॑ अ॒भ्य॑भि वातो॒ वातो॑ अ॒भि वा॑तु वात्व॒भि वातो॒ वातो॑ अ॒भि वा॑तु । \newline
4. अ॒भि वा॑तु वात्व॒भ्य॑भि वा॑तू॒स्रा उ॒स्रा वा᳚त्व॒भ्य॑भि वा॑तू॒स्राः । \newline
5. वा॒तू॒स्रा उ॒स्रा वा॑तु वातू॒स्रा ऊर्ज॑स्वती॒ रूर्ज॑स्वती रु॒स्रा वा॑तु वातू॒स्रा ऊर्ज॑स्वतीः । \newline
6. उ॒स्रा ऊर्ज॑स्वती॒ रूर्ज॑स्वती रु॒स्रा उ॒स्रा ऊर्ज॑स्वती॒ रोष॑धी॒ रोष॑धी॒ रूर्ज॑स्वती रु॒स्रा उ॒स्रा ऊर्ज॑स्वती॒ रोष॑धीः । \newline
7. ऊर्ज॑स्वती॒ रोष॑धी॒ रोष॑धी॒ रूर्ज॑स्वती॒ रूर्ज॑स्वती॒ रोष॑धी॒ रौष॑धी॒ रूर्ज॑स्वती॒ रूर्ज॑स्वती॒ रोष॑धी॒रा । \newline
8. ओष॑धी॒ रौष॑धी॒ रोष॑धी॒रा रि॑शन्ताꣳ रिशन्ता॒ मौष॑धी॒ रोष॑धी॒रा रि॑शन्ताम् । \newline
9. आ रि॑शन्ताꣳ रिशन्ता॒ मा रि॑शन्ताम् । \newline
10. रि॒श॒न्ता॒मिति॑ रिशन्ताम् । \newline
11. पीव॑स्वतीर् जी॒वध॑न्या जी॒वध॑न्याः॒ पीव॑स्वतीः॒ पीव॑स्वतीर् जी॒वध॑न्याः पिबन्तु पिबन्तु जी॒वध॑न्याः॒ पीव॑स्वतीः॒ पीव॑स्वतीर् जी॒वध॑न्याः पिबन्तु । \newline
12. जी॒वध॑न्याः पिबन्तु पिबन्तु जी॒वध॑न्या जी॒वध॑न्याः पिब न्त्वव॒साया॑ व॒साय॑ पिबन्तु जी॒वध॑न्या जी॒वध॑न्याः पिब न्त्वव॒साय॑ । \newline
13. जी॒वध॑न्या॒ इति॑ जी॒व - ध॒न्याः॒ । \newline
14. पि॒ब॒ न्त्व॒व॒साया॑ व॒साय॑ पिबन्तु पिब न्त्वव॒साय॑ प॒द्वते॑ प॒द्वते॑ ऽव॒साय॑ पिबन्तु पिब न्त्वव॒साय॑ प॒द्वते᳚ । \newline
15. अ॒व॒साय॑ प॒द्वते॑ प॒द्वते॑ ऽव॒साया॑ व॒साय॑ प॒द्वते॑ रुद्र रुद्र प॒द्वते॑ ऽव॒साया॑ व॒साय॑ प॒द्वते॑ रुद्र । \newline
16. प॒द्वते॑ रुद्र रुद्र प॒द्वते॑ प॒द्वते॑ रुद्र मृड मृड रुद्र प॒द्वते॑ प॒द्वते॑ रुद्र मृड । \newline
17. प॒द्वत॒ इति॑ पत् - वते᳚ । \newline
18. रु॒द्र॒ मृ॒ड॒ मृ॒ड॒ रु॒द्र॒ रु॒द्र॒ मृ॒ड॒ । \newline
19. मृ॒डेति॑ मृड । \newline
20. याः सरू॑पाः॒ सरू॑पा॒ या याः सरू॑पा॒ विरू॑पा॒ विरू॑पाः॒ सरू॑पा॒ या याः सरू॑पा॒ विरू॑पाः । \newline
21. सरू॑पा॒ विरू॑पा॒ विरू॑पाः॒ सरू॑पाः॒ सरू॑पा॒ विरू॑पा॒ एक॑रूपा॒ एक॑रूपा॒ विरू॑पाः॒ सरू॑पाः॒ सरू॑पा॒ विरू॑पा॒ एक॑रूपाः । \newline
22. सरू॑पा॒ इति॒ स - रू॒पाः॒ । \newline
23. विरू॑पा॒ एक॑रूपा॒ एक॑रूपा॒ विरू॑पा॒ विरू॑पा॒ एक॑रूपा॒ यासां॒ ॅयासा॒ मेक॑रूपा॒ विरू॑पा॒ विरू॑पा॒ एक॑रूपा॒ यासा᳚म् । \newline
24. विरू॑पा॒ इति॒ वि - रू॒पाः॒ । \newline
25. एक॑रूपा॒ यासां॒ ॅयासा॒ मेक॑रूपा॒ एक॑रूपा॒ यासा॑ म॒ग्नि र॒ग्निर् यासा॒ मेक॑रूपा॒ एक॑रूपा॒ यासा॑ म॒ग्निः । \newline
26. एक॑रूपा॒ इत्येक॑ - रू॒पाः॒ । \newline
27. यासा॑ म॒ग्नि र॒ग्निर् यासां॒ ॅयासा॑ म॒ग्नि रिष्ट्येष्ट्या॒ ऽग्निर् यासां॒ ॅयासा॑ म॒ग्नि रिष्ट्या᳚ । \newline
28. अ॒ग्नि रिष्ट्येष्ट्या॒ ऽग्नि र॒ग्नि रिष्ट्या॒ नामा॑नि॒ नामा॒ नीष्ट्या॒ ऽग्नि र॒ग्नि रिष्ट्या॒ नामा॑नि । \newline
29. इष्ट्या॒ नामा॑नि॒ नामा॒ नीष्ट्येष्ट्या॒ नामा॑नि॒ वेद॒ वेद॒ नामा॒ नीष्ट्येष्ट्या॒ नामा॑नि॒ वेद॑ । \newline
30. नामा॑नि॒ वेद॒ वेद॒ नामा॑नि॒ नामा॑नि॒ वेद॑ । \newline
31. वेदेति॒ वेद॑ । \newline
32. या अङ्गि॑र॒सो ऽङ्गि॑रसो॒ या या अङ्गि॑रस॒ स्तप॑सा॒ तप॒सा ऽङ्गि॑रसो॒ या या अङ्गि॑रस॒ स्तप॑सा । \newline
33. अङ्गि॑रस॒ स्तप॑सा॒ तप॒सा ऽङ्गि॑र॒सो ऽङ्गि॑रस॒ स्तप॑से॒हेह तप॒सा ऽङ्गि॑र॒सो ऽङ्गि॑रस॒ स्तप॑ से॒ह । \newline
34. तप॑से॒हेह तप॑सा॒ तप॑ से॒ह च॒क्रु श्च॒क्रु रि॒ह तप॑सा॒ तप॑ से॒ह च॒क्रुः । \newline
35. इ॒ह च॒क्रु श्च॒क्रु रि॒हे ह च॒क्रु स्ताभ्य॒ स्ताभ्य॑ श्च॒क्रु रि॒हेह च॒क्रु स्ताभ्यः॑ । \newline
36. च॒क्रु स्ताभ्य॒ स्ताभ्य॑ श्च॒क्रु श्च॒क्रु स्ताभ्यः॑ पर्जन्य पर्जन्य॒ ताभ्य॑ श्च॒क्रु श्च॒क्रु स्ताभ्यः॑ पर्जन्य । \newline
37. ताभ्यः॑ पर्जन्य पर्जन्य॒ ताभ्य॒ स्ताभ्यः॑ पर्जन्य॒ महि॒ महि॑ पर्जन्य॒ ताभ्य॒ स्ताभ्यः॑ पर्जन्य॒ महि॑ । \newline
38. प॒र्ज॒न्य॒ महि॒ महि॑ पर्जन्य पर्जन्य॒ महि॒ शर्म॒ शर्म॒ महि॑ पर्जन्य पर्जन्य॒ महि॒ शर्म॑ । \newline
39. महि॒ शर्म॒ शर्म॒ महि॒ महि॒ शर्म॑ यच्छ यच्छ॒ शर्म॒ महि॒ महि॒ शर्म॑ यच्छ । \newline
40. शर्म॑ यच्छ यच्छ॒ शर्म॒ शर्म॑ यच्छ । \newline
41. य॒च्छेति॑ यच्छ । \newline
42. या दे॒वेषु॑ दे॒वेषु॒ या या दे॒वेषु॑ त॒नुव॑म् त॒नुव॑म् दे॒वेषु॒ या या दे॒वेषु॑ त॒नुव᳚म् । \newline
43. दे॒वेषु॑ त॒नुव॑म् त॒नुव॑म् दे॒वेषु॑ दे॒वेषु॑ त॒नुव॒ मैर॑य॒ न्तैर॑यन्त त॒नुव॑म् दे॒वेषु॑ दे॒वेषु॑ त॒नुव॒ मैर॑यन्त । \newline
44. त॒नुव॒ मैर॑य॒ न्तैर॑यन्त त॒नुव॑म् त॒नुव॒ मैर॑यन्त॒ यासां॒ ॅयासा॒ मैर॑यन्त त॒नुव॑म् त॒नुव॒ मैर॑यन्त॒ यासा᳚म् । \newline
45. ऐर॑यन्त॒ यासां॒ ॅयासा॒ मैर॑य॒ न्तैर॑यन्त॒ यासाꣳ॒॒ सोमः॒ सोमो॒ यासा॒ मैर॑य॒ न्तैर॑यन्त॒ यासाꣳ॒॒ सोमः॑ । \newline
46. यासाꣳ॒॒ सोमः॒ सोमो॒ यासां॒ ॅयासाꣳ॒॒ सोमो॒ विश्वा॒ विश्वा॒ सोमो॒ यासां॒ ॅयासाꣳ॒॒ सोमो॒ विश्वा᳚ । \newline
47. सोमो॒ विश्वा॒ विश्वा॒ सोमः॒ सोमो॒ विश्वा॑ रू॒पाणि॑ रू॒पाणि॒ विश्वा॒ सोमः॒ सोमो॒ विश्वा॑ रू॒पाणि॑ । \newline
48. विश्वा॑ रू॒पाणि॑ रू॒पाणि॒ विश्वा॒ विश्वा॑ रू॒पाणि॒ वेद॒ वेद॑ रू॒पाणि॒ विश्वा॒ विश्वा॑ रू॒पाणि॒ वेद॑ । \newline
49. रू॒पाणि॒ वेद॒ वेद॑ रू॒पाणि॑ रू॒पाणि॒ वेद॑ । \newline
50. वेदेति॒ वेद॑ । \newline
51. ता अ॒स्मभ्य॑ म॒स्मभ्य॒म् ता स्ता अ॒स्मभ्य॒म् पय॑सा॒ पय॑सा॒ ऽस्मभ्य॒म् ता स्ता अ॒स्मभ्य॒म् पय॑सा । \newline
52. अ॒स्मभ्य॒म् पय॑सा॒ पय॑सा॒ ऽस्मभ्य॑ म॒स्मभ्य॒म् पय॑सा॒ पिन्व॑मानाः॒ पिन्व॑मानाः॒ पय॑सा॒ ऽस्मभ्य॑ म॒स्मभ्य॒म् पय॑सा॒ पिन्व॑मानाः । \newline
53. अ॒स्मभ्य॒मित्य॒स्म - भ्य॒म् । \newline
54. पय॑सा॒ पिन्व॑मानाः॒ पिन्व॑मानाः॒ पय॑सा॒ पय॑सा॒ पिन्व॑मानाः प्र॒जाव॑तीः प्र॒जाव॑तीः॒ पिन्व॑मानाः॒ पय॑सा॒ पय॑सा॒ पिन्व॑मानाः प्र॒जाव॑तीः । \newline
55. पिन्व॑मानाः प्र॒जाव॑तीः प्र॒जाव॑तीः॒ पिन्व॑मानाः॒ पिन्व॑मानाः प्र॒जाव॑ती रिन्द्रेन्द्र प्र॒जाव॑तीः॒ पिन्व॑मानाः॒ पिन्व॑मानाः प्र॒जाव॑ती रिन्द्र । \newline
56. प्र॒जाव॑ती रिन्द्रेन्द्र प्र॒जाव॑तीः प्र॒जाव॑ती रिन्द्र गो॒ष्ठे गो॒ष्ठ इ॑न्द्र प्र॒जाव॑तीः प्र॒जाव॑ती रिन्द्र गो॒ष्ठे । \newline
57. प्र॒जाव॑ती॒रिति॑ प्र॒जा - व॒तीः॒ । \newline
58. इ॒न्द्र॒ गो॒ष्ठे गो॒ष्ठ इ॑न्द्रेन्द्र गो॒ष्ठे रि॑रीहि रिरीहि गो॒ष्ठ इ॑न्द्रेन्द्र गो॒ष्ठे रि॑रीहि । \newline
\pagebreak
\markright{ TS 7.4.17.2  \hfill https://www.vedavms.in \hfill}

\section{ TS 7.4.17.2 }

\textbf{TS 7.4.17.2 } \newline
\textbf{Samhita Paata} \newline

गो॒ष्ठे रि॑रीहि ॥ प्र॒जाप॑ति॒र्मह्य॑मे॒ता ररा॑णो॒ विश्वै᳚र्दे॒वैः पि॒तृभिः॑ संॅविदा॒नः । शि॒वाः स॒तीरुप॑ नो गो॒ष्ठमाऽक॒स्तासां᳚ ॅव॒यं प्र॒जया॒ सꣳ स॑देम ॥इ॒ह धृतिः॒ स्वाहे॒ह विधृ॑तिः॒ स्वाहे॒ह रन्तिः॒ स्वाहे॒ह रम॑तिः॒ स्वाहा॑ म॒हीमू॒ षु1 सु॒त्रामा॑णं2 ॥ \newline

\textbf{Pada Paata} \newline

गो॒ष्ठ इति॑ गो - स्थे । रि॒री॒हि॒ ॥ प्र॒जाप॑ति॒रिति॑ प्र॒जा-प॒तिः॒ । मह्य᳚म् । ए॒ताः । ररा॑णः । विश्वैः᳚ । दे॒वैः । पि॒तृभि॒रिति॑ पि॒तृ - भिः॒ । सं॒ॅवि॒दा॒न इति॑ सं - वि॒दा॒नः ॥ शि॒वाः । स॒तीः । उपेति॑ । नः॒ । गो॒ष्ठमिति॑ गो - स्थम् । एति॑ । अ॒कः॒ । तासा᳚म् । व॒यम् । प्र॒जयेति॑ प्र-जया᳚ । समिति॑ । स॒दे॒म॒ ॥ इ॒ह । धृतिः॑ । स्वाहा᳚ । इ॒ह । विधृ॑ति॒रिति॒ वि - धृ॒तिः॒ । स्वाहा᳚ । इ॒ह । रन्तिः॑ । स्वाहा᳚ । इ॒ह । रम॑तिः । स्वाहा᳚ । म॒हीम् । उ॒ । स्विति॑ । सु॒त्रामा॑ण॒मिति॑ सु - त्रामा॑णम् ॥  \newline


\textbf{Krama Paata} \newline

गो॒ष्ठे रि॑रीहि । गो॒ष्ठ इति॑ गो - स्थे । रि॒री॒हीति॑ रिरीहि ॥ प्र॒जाप॑ति॒र् मह्य᳚म् । प्र॒जाप॑ति॒रिति॑ प्र॒जा - प॒तिः॒ । मह्य॑मे॒ताः । ए॒ता ररा॑णः । ररा॑णो॒ विश्वैः᳚ । विश्वै᳚र् दे॒वैः । दे॒वैः पि॒तृभिः॑ । पि॒तृभिः॑ सम्ॅविदा॒नः । पि॒तृभि॒रिति॑ पि॒तृ - भिः॒ । स॒म्ॅवि॒दा॒न इति॑ सम् - वि॒दा॒नः ॥ शि॒वाः स॒तीः । स॒तीरुप॑ । उप॑ नः । नो॒ गो॒ष्ठम् । गो॒ष्ठमा । गो॒ष्ठमिति॑ गो - स्थम् । आऽकः॑ । अ॒क॒स्तासा᳚म् । तासा᳚म् ॅव॒यम् । व॒यम् प्र॒जया᳚ । प्र॒जया॒ सम् । प्र॒जयेति॑ प्र - जया᳚ । सꣳ स॑देम । स॒दे॒मेति॑ सदेम ॥ इ॒ह धृतिः॑ । धृतिः॒ स्वाहा᳚ । स्वाहे॒ह । इ॒ह विधृ॑तिः । विधृ॑तिः॒ स्वाहा᳚ । विधृ॑ति॒रिति॒ वि - धृ॒तिः॒ । स्वाहे॒ह । इ॒ह रन्तिः॑ । रन्तिः॒ स्वाहा᳚ । स्वाहे॒ह । इ॒ह रम॑तिः । रम॑तिः॒ स्वाहा᳚ । स्वाहा॑ म॒हीम् । म॒हीमु॑ । उ॒षु । सु सु॒त्रामा॑णम् । सु॒त्रामा॑ण॒मिति॑ सु - त्रामा॑णम् । \newline

\textbf{Jatai Paata} \newline

1. गो॒ष्ठे रि॑रीहि रिरीहि गो॒ष्ठे गो॒ष्ठे रि॑रीहि । \newline
2. गो॒ष्ठ इति॑ गो - स्थे । \newline
3. रि॒री॒हीति॑ रिरीहि । \newline
4. प्र॒जाप॑ति॒र् मह्य॒म् मह्य॑म् प्र॒जाप॑तिः प्र॒जाप॑ति॒र् मह्य᳚म् । \newline
5. प्र॒जाप॑ति॒रिति॑ प्र॒जा - प॒तिः॒ । \newline
6. मह्य॑ मे॒ता ए॒ता मह्य॒म् मह्य॑ मे॒ताः । \newline
7. ए॒ता ररा॑णो॒ ररा॑ण ए॒ता ए॒ता ररा॑णः । \newline
8. ररा॑णो॒ विश्वै॒र् विश्वै॒ ररा॑णो॒ ररा॑णो॒ विश्वैः᳚ । \newline
9. विश्वै᳚र् दे॒वैर् दे॒वैर् विश्वै॒र् विश्वै᳚र् दे॒वैः । \newline
10. दे॒वैः पि॒तृभिः॑ पि॒तृभि॑र् दे॒वैर् दे॒वैः पि॒तृभिः॑ । \newline
11. पि॒तृभिः॑ संॅविदा॒नः सं॑ॅविदा॒नः पि॒तृभिः॑ पि॒तृभिः॑ संॅविदा॒नः । \newline
12. पि॒तृभि॒रिति॑ पि॒तृ - भिः॒ । \newline
13. सं॒ॅवि॒दा॒न इति॑ सं - वि॒दा॒नः । \newline
14. शि॒वाः स॒तीः स॒तीः शि॒वाः शि॒वाः स॒तीः । \newline
15. स॒ती रुपोप॑ स॒तीः स॒ती रुप॑ । \newline
16. उप॑ नो न॒ उपोप॑ नः । \newline
17. नो॒ गो॒ष्ठम् गो॒ष्ठन् नो॑ नो गो॒ष्ठम् । \newline
18. गो॒ष्ठ मा गो॒ष्ठम् गो॒ष्ठ मा । \newline
19. गो॒ष्ठमिति॑ गो - स्थम् । \newline
20. आ ऽक॑ रक॒ रा ऽकः॑ । \newline
21. अ॒क॒ स्तासा॒म् तासा॑ मक रक॒ स्तासा᳚म् । \newline
22. तासां᳚ ॅव॒यं ॅव॒यम् तासा॒म् तासां᳚ ॅव॒यम् । \newline
23. व॒यम् प्र॒जया᳚ प्र॒जया॑ व॒यं ॅव॒यम् प्र॒जया᳚ । \newline
24. प्र॒जया॒ सꣳ सम् प्र॒जया᳚ प्र॒जया॒ सम् । \newline
25. प्र॒जयेति॑ प्र - जया᳚ । \newline
26. सꣳ स॑देम सदेम॒ सꣳ सꣳ स॑देम । \newline
27. स॒दे॒मेति॑ सदेम । \newline
28. इ॒ह धृति॒र् धृति॑ रि॒हेह धृतिः॑ । \newline
29. धृतिः॒ स्वाहा॒ स्वाहा॒ धृति॒र् धृतिः॒ स्वाहा᳚ । \newline
30. स्वाहे॒ हेह स्वाहा॒ स्वाहे॒ह । \newline
31. इ॒ह विधृ॑ति॒र् विधृ॑ति रि॒हेह विधृ॑तिः । \newline
32. विधृ॑तिः॒ स्वाहा॒ स्वाहा॒ विधृ॑ति॒र् विधृ॑तिः॒ स्वाहा᳚ । \newline
33. विधृ॑ति॒रिति॒ वि - धृ॒तिः॒ । \newline
34. स्वाहे॒ हेह स्वाहा॒ स्वाहे॒ह । \newline
35. इ॒ह रन्ती॒ रन्ति॑ रि॒हेह रन्तिः॑ । \newline
36. रन्तिः॒ स्वाहा॒ स्वाहा॒ रन्ती॒ रन्तिः॒ स्वाहा᳚ । \newline
37. स्वाहे॒ हेह स्वाहा॒ स्वाहे॒ह । \newline
38. इ॒ह रम॑ती॒ रम॑ति रि॒हेह रम॑तिः । \newline
39. रम॑तिः॒ स्वाहा॒ स्वाहा॒ रम॑ती॒ रम॑तिः॒ स्वाहा᳚ । \newline
40. स्वाहा॑ म॒हीम् म॒हीꣳ स्वाहा॒ स्वाहा॑ म॒हीम् । \newline
41. म॒ही मु॑ वु म॒हीम् म॒ही मु॑ । \newline
42. ऊ॒ षु सू॑ षु । \newline
43. सु सु॒त्रामा॑णꣳ सु॒त्रामा॑णꣳ॒॒ सु सु सु॒त्रामा॑णम् । \newline
44. सु॒त्रामा॑ण॒मिति॑ सु - त्रामा॑णम् । \newline

\textbf{Ghana Paata } \newline

1. गो॒ष्ठे रि॑रीहि रिरीहि गो॒ष्ठे गो॒ष्ठे रि॑रीहि । \newline
2. गो॒ष्ठ इति॑ गो - स्थे । \newline
3. रि॒री॒हीति॑ रिरीहि । \newline
4. प्र॒जाप॑ति॒र् मह्य॒म् मह्य॑म् प्र॒जाप॑तिः प्र॒जाप॑ति॒र् मह्य॑ मे॒ता ए॒ता मह्य॑म् प्र॒जाप॑तिः प्र॒जाप॑ति॒र् मह्य॑ मे॒ताः । \newline
5. प्र॒जाप॑ति॒रिति॑ प्र॒जा - प॒तिः॒ । \newline
6. मह्य॑ मे॒ता ए॒ता मह्य॒म् मह्य॑ मे॒ता ररा॑णो॒ ररा॑ण ए॒ता मह्य॒म् मह्य॑ मे॒ता ररा॑णः । \newline
7. ए॒ता ररा॑णो॒ ररा॑ण ए॒ता ए॒ता ररा॑णो॒ विश्वै॒र् विश्वै॒ ररा॑ण ए॒ता ए॒ता ररा॑णो॒ विश्वैः᳚ । \newline
8. ररा॑णो॒ विश्वै॒र् विश्वै॒ ररा॑णो॒ ररा॑णो॒ विश्वै᳚र् दे॒वैर् दे॒वैर् विश्वै॒ ररा॑णो॒ ररा॑णो॒ विश्वै᳚र् दे॒वैः । \newline
9. विश्वै᳚र् दे॒वैर् दे॒वैर् विश्वै॒र् विश्वै᳚र् दे॒वैः पि॒तृभिः॑ पि॒तृभि॑र् दे॒वैर् विश्वै॒र् विश्वै᳚र् दे॒वैः पि॒तृभिः॑ । \newline
10. दे॒वैः पि॒तृभिः॑ पि॒तृभि॑र् दे॒वैर् दे॒वैः पि॒तृभिः॑ संॅविदा॒नः सं॑ॅविदा॒नः पि॒तृभि॑र् दे॒वैर् दे॒वैः पि॒तृभिः॑ संॅविदा॒नः । \newline
11. पि॒तृभिः॑ संॅविदा॒नः सं॑ॅविदा॒नः पि॒तृभिः॑ पि॒तृभिः॑ संॅविदा॒नः । \newline
12. पि॒तृभि॒रिति॑ पि॒तृ - भिः॒ । \newline
13. सं॒ॅवि॒दा॒न इति॑ सं - वि॒दा॒नः । \newline
14. शि॒वाः स॒तीः स॒तीः शि॒वाः शि॒वाः स॒ती रुपोप॑ स॒तीः शि॒वाः शि॒वाः स॒ती रुप॑ । \newline
15. स॒ती रुपोप॑ स॒तीः स॒ती रुप॑ नो न॒ उप॑ स॒तीः स॒ती रुप॑ नः । \newline
16. उप॑ नो न॒ उपोप॑ नो गो॒ष्ठम् गो॒ष्ठन् न॒ उपोप॑ नो गो॒ष्ठम् । \newline
17. नो॒ गो॒ष्ठम् गो॒ष्ठन् नो॑ नो गो॒ष्ठ मा गो॒ष्ठन् नो॑ नो गो॒ष्ठ मा । \newline
18. गो॒ष्ठ मा गो॒ष्ठम् गो॒ष्ठ मा ऽक॑ रक॒ रा गो॒ष्ठम् गो॒ष्ठ मा ऽकः॑ । \newline
19. गो॒ष्ठमिति॑ गो - स्थम् । \newline
20. आ ऽक॑ रक॒ रा ऽक॒ स्तासा॒म् तासा॑ मक॒ रा ऽक॒ स्तासा᳚म् । \newline
21. अ॒क॒ स्तासा॒म् तासा॑ मक रक॒ स्तासां᳚ ॅव॒यं ॅव॒यम् तासा॑ मक रक॒ स्तासां᳚ ॅव॒यम् । \newline
22. तासां᳚ ॅव॒यं ॅव॒यम् तासा॒म् तासां᳚ ॅव॒यम् प्र॒जया᳚ प्र॒जया॑ व॒यम् तासा॒म् तासां᳚ ॅव॒यम् प्र॒जया᳚ । \newline
23. व॒यम् प्र॒जया᳚ प्र॒जया॑ व॒यं ॅव॒यम् प्र॒जया॒ सꣳ सम् प्र॒जया॑ व॒यं ॅव॒यम् प्र॒जया॒ सम् । \newline
24. प्र॒जया॒ सꣳ सम् प्र॒जया᳚ प्र॒जया॒ सꣳ स॑देम सदेम॒ सम् प्र॒जया᳚ प्र॒जया॒ सꣳ स॑देम । \newline
25. प्र॒जयेति॑ प्र - जया᳚ । \newline
26. सꣳ स॑देम सदेम॒ सꣳ सꣳ स॑देम । \newline
27. स॒दे॒मेति॑ सदेम । \newline
28. इ॒ह धृति॒र् धृति॑ रि॒हेह धृतिः॒ स्वाहा॒ स्वाहा॒ धृति॑ रि॒हेह धृतिः॒ स्वाहा᳚ । \newline
29. धृतिः॒ स्वाहा॒ स्वाहा॒ धृति॒र् धृतिः॒ स्वाहे॒ हेह स्वाहा॒ धृति॒र् धृतिः॒ स्वाहे॒ह । \newline
30. स्वाहे॒ हेह स्वाहा॒ स्वाहे॒ह विधृ॑ति॒र् विधृ॑ति रि॒ह स्वाहा॒ स्वाहे॒ह विधृ॑तिः । \newline
31. इ॒ह विधृ॑ति॒र् विधृ॑ति रि॒हेह विधृ॑तिः॒ स्वाहा॒ स्वाहा॒ विधृ॑ति रि॒हेह विधृ॑तिः॒ स्वाहा᳚ । \newline
32. विधृ॑तिः॒ स्वाहा॒ स्वाहा॒ विधृ॑ति॒र् विधृ॑तिः॒ स्वाहे॒ हेह स्वाहा॒ विधृ॑ति॒र् विधृ॑तिः॒ स्वाहे॒ह । \newline
33. विधृ॑ति॒रिति॒ वि - धृ॒तिः॒ । \newline
34. स्वाहे॒ हेह स्वाहा॒ स्वाहे॒ह रन्ती॒ रन्ति॑ रि॒ह स्वाहा॒ स्वाहे॒ह रन्तिः॑ । \newline
35. इ॒ह रन्ती॒ रन्ति॑ रि॒हेह रन्तिः॒ स्वाहा॒ स्वाहा॒ रन्ति॑ रि॒हेह रन्तिः॒ स्वाहा᳚ । \newline
36. रन्तिः॒ स्वाहा॒ स्वाहा॒ रन्ती॒ रन्तिः॒ स्वाहे॒ हेह स्वाहा॒ रन्ती॒ रन्तिः॒ स्वाहे॒ह । \newline
37. स्वाहे॒ हेह स्वाहा॒ स्वाहे॒ह रम॑ती॒ रम॑ति रि॒ह स्वाहा॒ स्वाहे॒ह रम॑तिः । \newline
38. इ॒ह रम॑ती॒ रम॑ति रि॒हेह रम॑तिः॒ स्वाहा॒ स्वाहा॒ रम॑ति रि॒हेह रम॑तिः॒ स्वाहा᳚ । \newline
39. रम॑तिः॒ स्वाहा॒ स्वाहा॒ रम॑ती॒ रम॑तिः॒ स्वाहा॑ म॒हीम् म॒हीꣳ स्वाहा॒ रम॑ती॒ रम॑तिः॒ स्वाहा॑ म॒हीम् । \newline
40. स्वाहा॑ म॒हीम् म॒हीꣳ स्वाहा॒ स्वाहा॑ म॒ही मु॑ वु म॒हीꣳ स्वाहा॒ स्वाहा॑ म॒ही मु॑ । \newline
41. म॒ही मु॑ वु म॒हीम् म॒ही मू॒ षु सू॑ म॒हीम् म॒ही मू॒ षु । \newline
42. ऊ॒ षु सू॑ षु सु॒त्रामा॑णꣳ सु॒त्रामा॑णꣳ॒॒ सू॑ षु सु॒त्रामा॑णम् । \newline
43. सु सु॒त्रामा॑णꣳ सु॒त्रामा॑णꣳ॒॒ सु सु सु॒त्रामा॑णम् । \newline
44. सु॒त्रामा॑ण॒मिति॑ सु - त्रामा॑णम् । \newline
\pagebreak
\markright{ TS 7.4.18.1  \hfill https://www.vedavms.in \hfill}

\section{ TS 7.4.18.1 }

\textbf{TS 7.4.18.1 } \newline
\textbf{Samhita Paata} \newline

किꣳ स्वि॑दासीत् पू॒र्वचि॑त्तिः॒ किꣳ स्वि॑दासीद्-बृ॒हद्वयः॑ । किꣳ स्वि॑दासीत् पिशंगि॒ला किꣳ स्वि॑दासीत् पिलिप्पि॒ला ॥द्यौरा॑सीत् पू॒र्वचि॑त्ति॒रश्व॑ आसीद् बृ॒हद्वयः॑ । रात्रि॑रासीत् पिशङ्गि॒ला ऽवि॑रासीत् पिलिप्पि॒ला ॥ कः स्वि॑देका॒की च॑रति॒ क उ॑ स्विज्जायते॒ पुनः॑ । किꣳ स्वि॑द्धि॒मस्य॑ भेष॒जं किꣳ स्वि॑दा॒वप॑नं म॒हत् ॥ सूर्य॑ एका॒की च॑रति - [  ] \newline

\textbf{Pada Paata} \newline

किम् । स्वि॒त् । आ॒सी॒त् । पू॒र्वचि॑त्ति॒रिति॑ पू॒र्व - चि॒त्तिः॒ । किम् । स्वि॒त् । आ॒सी॒त् । बृ॒हत् । वयः॑ ॥ किम् । स्वि॒त् । आ॒सी॒त् । पि॒श॒ङ्गि॒ला । किम् । स्वि॒त् । आ॒सी॒त् । पि॒लि॒प्पि॒ला ॥ द्यौः । आ॒सी॒त् । पू॒र्वचि॑त्ति॒रिति॑ पू॒र्व - चि॒त्तिः॒ । अश्वः॑ । आ॒सी॒त् । बृ॒हत् । वयः॑ ॥ रात्रिः॑ । आ॒सी॒त् । पि॒श॒ङ्गि॒ला । अविः॑ । आ॒सी॒त् । पि॒लि॒प्पि॒ला ॥ कः । स्वि॒त् । ए॒का॒की । च॒र॒ति॒ । कः । उ॒ । स्वि॒त् । जा॒य॒ते॒ । पुनः॑ ॥ किम् । स्वि॒त् । हि॒मस्य॑ । भे॒ष॒जम् । किम् । स्वि॒त् । आ॒वप॑न॒मित्या᳚ - वप॑नम् । म॒हत् ॥ सूर्यः॑ । ए॒का॒की । च॒र॒ति॒ ।  \newline


\textbf{Krama Paata} \newline

किꣳ स्वि॑त् । स्वि॒दा॒सी॒त्॒ । आ॒सी॒त् पू॒र्वचि॑त्तिः । पू॒र्वचि॑त्तिः॒ किम् । पू॒र्वचि॑त्ति॒रिति॑ पू॒र्व - चि॒त्तिः॒ । किꣳ स्वि॑त् । स्वि॒दा॒सी॒त्॒ । आ॒सी॒द् बृ॒हत् । बृ॒हद् वयः॑ । वय॒ इति॒ वयः॑ ॥ किꣳ स्वि॑त् । स्वि॒दा॒सी॒त्॒ । आ॒सी॒त् पि॒श॒ङ्‍गि॒ला । पि॒श॒ङ्‍गि॒ला किम् । किꣳ स्वि॑त् । स्वि॒दा॒सी॒त्॒ । आ॒सी॒त् पि॒लि॒प्पि॒ला । पि॒लि॒प्पि॒लेति॑ पिलिप्पि॒ला ॥ द्यौरा॑सीत् । आ॒सी॒त् पू॒र्वचि॑त्तिः । पू॒र्वचि॑त्ति॒रश्वः॑ । पू॒र्वचि॑त्ति॒रिति॑ पू॒र्व - चि॒त्तिः॒ । अश्व॑ आसीत् । आ॒सी॒द् बृ॒हत् । बृ॒हद् वयः॑ । वय॒ इति॒ वयः॑ ॥ रात्रि॑रासीत् । आ॒सी॒त् पि॒श॒ङ्‍गि॒ला । पि॒श॒ङ्‍गि॒लाऽविः॑ । अवि॑रासीत् । आ॒सी॒त् पि॒लि॒प्पि॒ला । पि॒लि॒प्पि॒लेति॑ पिलिप्पि॒ला ॥ किꣳ स्वि॑त् । स्वि॒दे॒का॒की । ए॒का॒की च॑रति । च॒र॒ति॒ कः । क उ॑ । उ॒ स्वि॒त्॒ । स्वि॒ज् जा॒य॒ते॒ । जा॒य॒ते॒ पुनः॑ । पुन॒रिति॒ पुनः॑ ॥ किꣳ स्वि॑त् । स्वि॒द्‌धि॒मस्य॑ । हि॒मस्य॑ भेष॒जम् । भे॒ष॒जम् किम् । किꣳ स्वि॑त् । स्वि॒दा॒वप॑नम् । आ॒वप॑नम् म॒हत् । आ॒वप॑न॒मित्या᳚ - वप॑नम् । म॒हदिति॑ म॒हत् ॥ सूर्य॑ एका॒की । ए॒का॒की च॑रति ( ) । च॒र॒ति॒ च॒न्द्रमाः᳚ \newline

\textbf{Jatai Paata} \newline

1. किꣳ स्वि॑थ् स्वि॒त् किम् किꣳ स्वि॑त् । \newline
2. स्वि॒ दा॒सी॒ दा॒सी॒थ् स्वि॒थ् स्वि॒ दा॒सी॒त् । \newline
3. आ॒सी॒त् पू॒र्वचि॑त्तिः पू॒र्वचि॑त्ति रासी दासीत् पू॒र्वचि॑त्तिः । \newline
4. पू॒र्वचि॑त्तिः॒ किम् किम् पू॒र्वचि॑त्तिः पू॒र्वचि॑त्तिः॒ किम् । \newline
5. पू॒र्वचि॑त्ति॒रिति॑ पू॒र्व - चि॒त्तिः॒ । \newline
6. किꣳ स्वि॑थ् स्वि॒त् किम् किꣳ स्वि॑त् । \newline
7. स्वि॒ दा॒सी॒ दा॒सी॒थ् स्वि॒थ् स्वि॒ दा॒सी॒त् । \newline
8. आ॒सी॒द् बृ॒हद् बृ॒ह दा॑सी दासीद् बृ॒हत् । \newline
9. बृ॒हद् वयो॒ वयो॑ बृ॒हद् बृ॒हद् वयः॑ । \newline
10. वय॒ इति॒ वयः॑ । \newline
11. किꣳ स्वि॑थ् स्वि॒त् किम् किꣳ स्वि॑त् । \newline
12. स्वि॒ दा॒सी॒ दा॒सी॒थ् स्वि॒थ् स्वि॒ दा॒सी॒त् । \newline
13. आ॒सी॒त् पि॒श॒ङ्गि॒ला पि॑शङ्गि॒ला ऽऽसी॑ दासीत् पिशङ्गि॒ला । \newline
14. पि॒श॒ङ्गि॒ला किम् किम् पि॑शङ्गि॒ला पि॑शङ्गि॒ला किम् । \newline
15. किꣳ स्वि॑थ् स्वि॒त् किम् किꣳ स्वि॑त् । \newline
16. स्वि॒ दा॒सी॒ दा॒सी॒थ् स्वि॒थ् स्वि॒ दा॒सी॒त् । \newline
17. आ॒सी॒त् पि॒लि॒प्पि॒ला पि॑लिप्पि॒ला ऽऽसी॑ दासीत् पिलिप्पि॒ला । \newline
18. पि॒लि॒प्पि॒लेति॑ पिलिप्पि॒ला । \newline
19. द्यौ रा॑सी दासी॒द् द्यौर् द्यौ रा॑सीत् । \newline
20. आ॒सी॒त् पू॒र्वचि॑त्तिः पू॒र्वचि॑त्ति रासी दासीत् पू॒र्वचि॑त्तिः । \newline
21. पू॒र्वचि॑त्ति॒ रश्वो ऽश्वः॑ पू॒र्वचि॑त्तिः पू॒र्वचि॑त्ति॒ रश्वः॑ । \newline
22. पू॒र्वचि॑त्ति॒रिति॑ पू॒र्व - चि॒त्तिः॒ । \newline
23. अश्व॑ आसी दासी॒ दश्वो ऽश्व॑ आसीत् । \newline
24. आ॒सी॒द् बृ॒हद् बृ॒ह दा॑सी दासीद् बृ॒हत् । \newline
25. बृ॒हद् वयो॒ वयो॑ बृ॒हद् बृ॒हद् वयः॑ । \newline
26. वय॒ इति॒ वयः॑ । \newline
27. रात्रि॑ रासी दासी॒द् रात्री॒ रात्रि॑ रासीत् । \newline
28. आ॒सी॒त् पि॒श॒ङ्गि॒ला पि॑शङ्गि॒ला ऽऽसी॑ दासीत् पिशङ्गि॒ला । \newline
29. पि॒श॒ङ्गि॒ला ऽवि॒ रविः॑ पिशङ्गि॒ला पि॑शङ्गि॒ला ऽविः॑ । \newline
30. अवि॑ रासी दासी॒ दवि॒ रवि॑ रासीत् । \newline
31. आ॒सी॒त् पि॒लि॒प्पि॒ला पि॑लिप्पि॒ला ऽऽसी॑ दासीत् पिलिप्पि॒ला । \newline
32. पि॒लि॒प्पि॒लेति॑ पिलिप्पि॒ला । \newline
33. कः स्वि॑थ् स्वि॒त् कः कः स्वि॑त् । \newline
34. स्वि॒ दे॒का॒ क्ये॑का॒की स्वि॑थ् स्वि देका॒की । \newline
35. ए॒का॒की च॑रति चर त्येका॒ क्ये॑का॒की च॑रति । \newline
36. च॒र॒ति॒ कः क श्च॑रति चरति॒ कः । \newline
37. क उ॑ वु॒ कः क उ॑ । \newline
38. उ॒ स्वि॒थ् स्वि॒दु॒ वु॒ स्वि॒त् । \newline
39. स्वि॒ज् जा॒य॒ते॒ जा॒य॒ते॒ स्वि॒थ् स्वि॒ज् जा॒य॒ते॒ । \newline
40. जा॒य॒ते॒ पुनः॒ पुन॑र् जायते जायते॒ पुनः॑ । \newline
41. पुन॒रिति॒ पुनः॑ । \newline
42. किꣳ स्वि॑थ् स्वि॒त् किम् किꣳ स्वि॑त् । \newline
43. स्वि॒ द्धि॒मस्य॑ हि॒मस्य॑ स्विथ् स्वि द्धि॒मस्य॑ । \newline
44. हि॒मस्य॑ भेष॒जम् भे॑ष॒जꣳ हि॒मस्य॑ हि॒मस्य॑ भेष॒जम् । \newline
45. भे॒ष॒जम् किम् किम् भे॑ष॒जम् भे॑ष॒जम् किम् । \newline
46. किꣳ स्वि॑थ् स्वि॒त् किम् किꣳ स्वि॑त् । \newline
47. स्वि॒ दा॒वप॑न मा॒वप॑नꣳ स्विथ् स्वि दा॒वप॑नम् । \newline
48. आ॒वप॑नम् म॒हन् म॒ह दा॒वप॑न मा॒वप॑नम् म॒हत् । \newline
49. आ॒वप॑न॒मित्या᳚ - वप॑नम् । \newline
50. म॒हदिति॑ म॒हत् । \newline
51. सूर्य॑ एका॒ क्ये॑का॒की सूर्यः॒ सूर्य॑ एका॒की । \newline
52. ए॒का॒की च॑रति चर त्येका॒ क्ये॑का॒की च॑रति । \newline
53. च॒र॒ति॒ च॒न्द्रमा᳚ श्च॒न्द्रमा᳚ श्चरति चरति च॒न्द्रमाः᳚ । \newline

\textbf{Ghana Paata } \newline

1. किꣳ स्वि॑थ् स्वि॒त् किम् किꣳ स्वि॑दासी दासीथ् स्वि॒त् किम् किꣳ स्वि॑दासीत् । \newline
2. स्वि॒दा॒सी॒ दा॒सी॒थ् स्वि॒थ् स्वि॒दा॒सी॒त् पू॒र्वचि॑त्तिः पू॒र्वचि॑त्ति रासीथ् स्विथ् स्विदासीत् पू॒र्वचि॑त्तिः । \newline
3. आ॒सी॒त् पू॒र्वचि॑त्तिः पू॒र्वचि॑त्ति रासी दासीत् पू॒र्वचि॑त्तिः॒ किम् किम् पू॒र्वचि॑त्ति रासी दासीत् पू॒र्वचि॑त्तिः॒ किम् । \newline
4. पू॒र्वचि॑त्तिः॒ किम् किम् पू॒र्वचि॑त्तिः पू॒र्वचि॑त्तिः॒ किꣳ स्वि॑थ् स्वि॒त् किम् पू॒र्वचि॑त्तिः पू॒र्वचि॑त्तिः॒ किꣳ स्वि॑त् । \newline
5. पू॒र्वचि॑त्ति॒रिति॑ पू॒र्व - चि॒त्तिः॒ । \newline
6. किꣳ स्वि॑थ् स्वि॒त् किम् किꣳ स्वि॑दासी दासीथ् स्वि॒त् किम् किꣳ स्वि॑दासीत् । \newline
7. स्वि॒ दा॒सी॒ दा॒सी॒थ् स्वि॒थ् स्वि॒ दा॒सी॒द् बृ॒हद् बृ॒ह दा॑सीथ् स्विथ् स्वि दासीद् बृ॒हत् । \newline
8. आ॒सी॒द् बृ॒हद् बृ॒ह दा॑सी दासीद् बृ॒हद् वयो॒ वयो॑ बृ॒ह दा॑सी दासीद् बृ॒हद् वयः॑ । \newline
9. बृ॒हद् वयो॒ वयो॑ बृ॒हद् बृ॒हद् वयः॑ । \newline
10. वय॒ इति॒ वयः॑ । \newline
11. किꣳ स्वि॑थ् स्वि॒त् किम् किꣳ स्वि॑दासी दासीथ् स्वि॒त् किम् किꣳ स्वि॑दासीत् । \newline
12. स्वि॒ दा॒सी॒ दा॒सी॒थ् स्वि॒थ् स्वि॒ दा॒सी॒त् पि॒श॒ङ्गि॒ला पि॑शङ्गि॒ला ऽऽसी᳚थ् स्विथ् स्वि दासीत् पिशङ्गि॒ला । \newline
13. आ॒सी॒त् पि॒श॒ङ्गि॒ला पि॑शङ्गि॒ला ऽऽसी॑ दासीत् पिशङ्गि॒ला किम् किम् पि॑शङ्गि॒ला ऽऽसी॑ दासीत् पिशङ्गि॒ला किम् । \newline
14. पि॒श॒ङ्गि॒ला किम् किम् पि॑शङ्गि॒ला पि॑शङ्गि॒ला किꣳ स्वि॑थ् स्वि॒त् किम् पि॑शङ्गि॒ला पि॑शङ्गि॒ला किꣳ स्वि॑त् । \newline
15. किꣳ स्वि॑थ् स्वि॒त् किम् किꣳ स्वि॑दासी दासीथ् स्वि॒त् किम् किꣳ स्वि॑दासीत् । \newline
16. स्वि॒दा॒सी॒ दा॒सी॒थ् स्वि॒थ् स्वि॒ दा॒सी॒त् पि॒लि॒प्पि॒ला पि॑लिप्पि॒ला ऽऽसी᳚थ् स्विथ् स्विदासीत् पिलिप्पि॒ला । \newline
17. आ॒सी॒त् पि॒लि॒प्पि॒ला पि॑लिप्पि॒ला ऽऽसी॑दासीत् पिलिप्पि॒ला । \newline
18. पि॒लि॒प्पि॒लेति॑ पिलिप्पि॒ला । \newline
19. द्यौ रा॑सी दासी॒द् द्यौर् द्यौ रा॑सीत् पू॒र्वचि॑त्तिः पू॒र्वचि॑त्ति रासी॒द् द्यौर् द्यौ रा॑सीत् पू॒र्वचि॑त्तिः । \newline
20. आ॒सी॒त् पू॒र्वचि॑त्तिः पू॒र्वचि॑त्ति रासी दासीत् पू॒र्वचि॑त्ति॒ रश्वो ऽश्वः॑ पू॒र्वचि॑त्ति रासी दासीत् पू॒र्वचि॑त्ति॒ रश्वः॑ । \newline
21. पू॒र्वचि॑त्ति॒ रश्वो ऽश्वः॑ पू॒र्वचि॑त्तिः पू॒र्वचि॑त्ति॒ रश्व॑ आसी दासी॒ दश्वः॑ पू॒र्वचि॑त्तिः पू॒र्वचि॑त्ति॒ रश्व॑ आसीत् । \newline
22. पू॒र्वचि॑त्ति॒रिति॑ पू॒र्व - चि॒त्तिः॒ । \newline
23. अश्व॑ आसी दासी॒ दश्वो ऽश्व॑ आसीद् बृ॒हद् बृ॒ह दा॑सी॒ दश्वो ऽश्व॑ आसीद् बृ॒हत् । \newline
24. आ॒सी॒द् बृ॒हद् बृ॒ह दा॑सी दासीद् बृ॒हद् वयो॒ वयो॑ बृ॒ह दा॑सी दासीद् बृ॒हद् वयः॑ । \newline
25. बृ॒हद् वयो॒ वयो॑ बृ॒हद् बृ॒हद् वयः॑ । \newline
26. वय॒ इति॒ वयः॑ । \newline
27. रात्रि॑ रासी दासी॒द् रात्री॒ रात्रि॑ रासीत् पिशङ्गि॒ला पि॑शङ्गि॒ला ऽऽसी॒द् रात्री॒ रात्रि॑ रासीत् पिशङ्गि॒ला । \newline
28. आ॒सी॒त् पि॒श॒ङ्गि॒ला पि॑शङ्गि॒ला ऽऽसी॑ दासीत् पिशङ्गि॒ला ऽवि॒ रविः॑ पिशङ्गि॒ला ऽऽसी॑ दासीत् पिशङ्गि॒ला ऽविः॑ । \newline
29. पि॒श॒ङ्गि॒ला ऽवि॒ रविः॑ पिशङ्गि॒ला पि॑शङ्गि॒ला ऽवि॑रासी दासी॒ दविः॑ पिशङ्गि॒ला पि॑शङ्गि॒ला ऽवि॑ रासीत् । \newline
30. अवि॑ रासी दासी॒ दवि॒ रवि॑ रासीत् पिलिप्पि॒ला पि॑लिप्पि॒ला ऽऽसी॒ दवि॒ रवि॑ रासीत् पिलिप्पि॒ला । \newline
31. आ॒सी॒त् पि॒लि॒प्पि॒ला पि॑लिप्पि॒ला ऽऽसी॑ दासीत् पिलिप्पि॒ला । \newline
32. पि॒लि॒प्पि॒लेति॑ पिलिप्पि॒ला । \newline
33. कः स्वि॑थ् स्वि॒त् कः कः स्वि॑ देका॒ क्ये॑का॒की स्वि॒त् कः कः स्वि॑ देका॒की । \newline
34. स्वि॒ दे॒का॒ क्ये॑का॒की स्वि॑थ् स्वि देका॒की च॑रति चर त्येका॒की स्वि॑थ् स्वि देका॒की च॑रति । \newline
35. ए॒का॒की च॑रति चर त्येका॒ क्ये॑का॒की च॑रति॒ कः क श्च॑र त्येका॒ क्ये॑का॒की च॑रति॒ कः । \newline
36. च॒र॒ति॒ कः कश्च॑रति चरति॒ क उ॑ वु॒ कश्च॑रति चरति॒ क उ॑ । \newline
37. क उ॑ वु॒ कः क उ॑ स्विथ् स्विदु॒ कः क उ॑ स्वित् । \newline
38. उ॒ स्वि॒थ् स्वि॒दु॒ वु॒ स्वि॒ज् जा॒य॒ते॒ जा॒य॒ते॒ स्वि॒दु॒ वु॒ स्वि॒ज् जा॒य॒ते॒ । \newline
39. स्वि॒ज् जा॒य॒ते॒ जा॒य॒ते॒ स्वि॒थ् स्वि॒ज् जा॒य॒ते॒ पुनः॒ पुन॑र् जायते स्विथ् स्विज् जायते॒ पुनः॑ । \newline
40. जा॒य॒ते॒ पुनः॒ पुन॑र् जायते जायते॒ पुनः॑ । \newline
41. पुन॒रिति॒ पुनः॑ । \newline
42. किꣳ स्वि॑थ् स्वि॒त् किम् किꣳ स्वि॑ द्धि॒मस्य॑ हि॒मस्य॑ स्वि॒त् किम् किꣳ स्वि॑ द्धि॒मस्य॑ । \newline
43. स्वि॒द्धि॒मस्य॑ हि॒मस्य॑ स्विथ् स्विद्धि॒मस्य॑ भेष॒जम् भे॑ष॒जꣳ हि॒मस्य॑ स्विथ् स्विद्धि॒मस्य॑ भेष॒जम् । \newline
44. हि॒मस्य॑ भेष॒जम् भे॑ष॒जꣳ हि॒मस्य॑ हि॒मस्य॑ भेष॒जम् किम् किम् भे॑ष॒जꣳ हि॒मस्य॑ हि॒मस्य॑ भेष॒जम् किम् । \newline
45. भे॒ष॒जम् किम् किम् भे॑ष॒जम् भे॑ष॒जम् किꣳ स्वि॑थ् स्वि॒त् किम् भे॑ष॒जम् भे॑ष॒जम् किꣳ स्वि॑त् । \newline
46. किꣳ स्वि॑थ् स्वि॒त् किम् किꣳ स्वि॑ दा॒वप॑न मा॒वप॑नꣳ स्वि॒त् किम् किꣳ स्वि॑ दा॒वप॑नम् । \newline
47. स्वि॒ दा॒वप॑न मा॒वप॑नꣳ स्विथ् स्वि दा॒वप॑नम् म॒हन् म॒ह दा॒वप॑नꣳ स्विथ् स्वि दा॒वप॑नम् म॒हत् । \newline
48. आ॒वप॑नम् म॒हन् म॒ह दा॒वप॑न मा॒वप॑नम् म॒हत् । \newline
49. आ॒वप॑न॒मित्या᳚ - वप॑नम् । \newline
50. म॒हदिति॑ म॒हत् । \newline
51. सूर्य॑ एका॒ क्ये॑का॒की सूर्यः॒ सूर्य॑ एका॒की च॑रति चर त्येका॒की सूर्यः॒ सूर्य॑ एका॒की च॑रति । \newline
52. ए॒का॒की च॑रति चर त्येका॒ क्ये॑का॒की च॑रति च॒न्द्रमा᳚ श्च॒न्द्रमा᳚ श्चर त्येका॒ क्ये॑का॒की च॑रति च॒न्द्रमाः᳚ । \newline
53. च॒र॒ति॒ च॒न्द्रमा᳚ श्च॒न्द्रमा᳚ श्चरति चरति च॒न्द्रमा॑ जायते जायते च॒न्द्रमा᳚ श्चरति चरति च॒न्द्रमा॑ जायते । \newline
\pagebreak
\markright{ TS 7.4.18.2  \hfill https://www.vedavms.in \hfill}

\section{ TS 7.4.18.2 }

\textbf{TS 7.4.18.2 } \newline
\textbf{Samhita Paata} \newline

च॒न्द्रमा॑ जायते॒ पुनः॑ । अ॒ग्निर्. हि॒मस्य॑ भेष॒जं भूमि॑रा॒वप॑नं म॒हत् ॥ पृ॒च्छामि॑ त्वा॒ पर॒मन्तं॑ पृथि॒व्याः पृ॒च्छामि॑ त्वा॒ भुव॑नस्य॒ नाभिं᳚ । पृ॒च्छामि॑ त्वा॒ वृष्णो॒ अश्व॑स्य॒ रेतः॑ पृ॒च्छामि॑ वा॒चः प॑र॒मं ॅव्यो॑म ॥वेदि॑माहुः॒ पर॒मन्तं॑ पृथि॒व्या य॒ज्ञ्मा॑हु॒ र्भुव॑नस्य॒ नाभिं᳚ । सोम॑माहु॒र्वृष्णो॒ अश्व॑स्य॒ रेतो॒ ब्रह्मै॒व वा॒चः प॑र॒मं ॅव्यो॑म ॥ \newline

\textbf{Pada Paata} \newline

च॒न्द्रमाः᳚ । जा॒य॒ते॒ । पुनः॑ ॥ अ॒ग्निः । हि॒मस्य॑ । भे॒ष॒जम् । भूमिः॑ । आ॒वप॑न॒मित्या᳚ - वप॑नम् । म॒हत् ॥ पृ॒च्छामि॑ । त्वा॒ । पर᳚म् । अन्त᳚म् । पृ॒थि॒व्याः । पृ॒च्छामि॑ । त्वा॒ । भुव॑नस्य । नाभि᳚म् ॥ पृ॒च्छामि॑ । त्वा॒ । वृष्णः॑ । अश्व॑स्य । रेतः॑ । पृ॒च्छामि॑ । वा॒चः । प॒र॒मम् । व्यो॑मेति॒ वि - ओ॒म॒ ॥ वेदि᳚म् । आ॒हुः॒ । पर᳚म् । अन्त᳚म् । पृ॒थि॒व्याः । य॒ज्ञ्म् । आ॒हुः॒ । भुव॑नस्य । नाभि᳚म् ॥ सोम᳚म् । आ॒हुः॒ । वृष्णः॑ । अश्व॑स्य । रेतः॑ । ब्रह्म॑ । ए॒व । वा॒चः । प॒र॒मम् । व्यो॑मेति॒ वि - ओ॒म॒ ॥  \newline


\textbf{Krama Paata} \newline

च॒न्द्रमा॑ जायते । जा॒य॒ते॒ पुनः॑ । पुन॒रिति॒ पुनः॑ ॥ अ॒ग्निर्. हि॒मस्य॑ । हि॒मस्य॑ भेष॒जम् । भे॒ष॒जम् भूमिः॑ । भूमि॑रा॒वप॑नम् । आ॒वप॑नम् म॒हत् । आ॒वप॑न॒मित्या᳚ - वप॑नम् । म॒हदिति॑ म॒हत् ॥ पृ॒च्छामि॑ त्वा । त्वा॒ पर᳚म् । पर॒मन्त᳚म् । अन्त॑म् पृथि॒व्याः । पृ॒थि॒व्याः पृ॒च्छामि॑ । पृ॒च्छामि॑ त्वा । त्वा॒ भुव॑नस्य । भुव॑नस्य॒ नाभि᳚म् । नाभि॒मिति॒ नाभि᳚म् ॥ पृ॒च्छामि॑ त्वा । त्वा॒ वृष्णः॑ । वृष्णो॒ अश्व॑स्य । अश्व॑स्य॒ रेतः॑ । रेतः॑ पृ॒च्छामि॑ । पृ॒च्छामि॑ वा॒चः । वा॒चः प॑र॒मम् । प॒र॒मम् ॅव्यो॑म । व्यो॑मेति॒ वि - ओ॒म॒ ॥ वेदि॑माहुः । आ॒हुः॒ पर᳚म् । पर॒मन्त᳚म् । अन्त॑म् पृथि॒व्याः । पृ॒थि॒व्या य॒ज्ञ्म् । य॒ज्ञ्मा॑हुः । आ॒हु॒र् भुव॑नस्य । भुव॑नस्य॒ नाभि᳚म् । नाभि॒मिति॒ नाभि᳚म् ॥ सोम॑माहुः । आ॒हु॒र् वृष्णः॑ । वृष्णो॒ अश्व॑स्य । अश्व॑स्य॒ रेतः॑ । रेतो॒ ब्रह्म॑ । ब्रह्मै॒व । ए॒व वा॒चः । वा॒चः प॑र॒मम् । प॒र॒मम् ॅव्यो॑म । व्यो॑मेति॒ वि - ओ॒म॒ । \newline

\textbf{Jatai Paata} \newline

1. च॒न्द्रमा॑ जायते जायते च॒न्द्रमा᳚ श्च॒न्द्रमा॑ जायते । \newline
2. जा॒य॒ते॒ पुनः॒ पुन॑र् जायते जायते॒ पुनः॑ । \newline
3. पुन॒रिति॒ पुनः॑ । \newline
4. अ॒ग्निर्. हि॒मस्य॑ हि॒म स्या॒ग्नि र॒ग्निर्. हि॒मस्य॑ । \newline
5. हि॒मस्य॑ भेष॒जम् भे॑ष॒जꣳ हि॒मस्य॑ हि॒मस्य॑ भेष॒जम् । \newline
6. भे॒ष॒जम् भूमि॒र् भूमि॑र् भेष॒जम् भे॑ष॒जम् भूमिः॑ । \newline
7. भूमि॑ रा॒वप॑न मा॒वप॑न॒म् भूमि॒र् भूमि॑ रा॒वप॑नम् । \newline
8. आ॒वप॑नम् म॒हन् म॒ह दा॒वप॑न मा॒वप॑नम् म॒हत् । \newline
9. आ॒वप॑न॒मित्या᳚ - वप॑नम् । \newline
10. म॒हदिति॑ म॒हत् । \newline
11. पृ॒च्छामि॑ त्वा त्वा पृ॒च्छामि॑ पृ॒च्छामि॑ त्वा । \newline
12. त्वा॒ पर॒म् पर॑म् त्वा त्वा॒ पर᳚म् । \newline
13. पर॒ मन्त॒ मन्त॒म् पर॒म् पर॒ मन्त᳚म् । \newline
14. अन्त॑म् पृथि॒व्याः पृ॑थि॒व्या अन्त॒ मन्त॑म् पृथि॒व्याः । \newline
15. पृ॒थि॒व्याः पृ॒च्छामि॑ पृ॒च्छामि॑ पृथि॒व्याः पृ॑थि॒व्याः पृ॒च्छामि॑ । \newline
16. पृ॒च्छामि॑ त्वा त्वा पृ॒च्छामि॑ पृ॒च्छामि॑ त्वा । \newline
17. त्वा॒ भुव॑नस्य॒ भुव॑नस्य त्वा त्वा॒ भुव॑नस्य । \newline
18. भुव॑नस्य॒ नाभि॒न् नाभि॒म् भुव॑नस्य॒ भुव॑नस्य॒ नाभि᳚म् । \newline
19. नाभि॒मिति॒ नाभि᳚म् । \newline
20. पृ॒च्छामि॑ त्वा त्वा पृ॒च्छामि॑ पृ॒च्छामि॑ त्वा । \newline
21. त्वा॒ वृष्णो॒ वृष्ण॑ स्त्वा त्वा॒ वृष्णः॑ । \newline
22. वृष्णो॒ अश्व॒स्या श्व॑स्य॒ वृष्णो॒ वृष्णो॒ अश्व॑स्य । \newline
23. अश्व॑स्य॒ रेतो॒ रेतो ऽश्व॒स्या श्व॑स्य॒ रेतः॑ । \newline
24. रेतः॑ पृ॒च्छामि॑ पृ॒च्छामि॒ रेतो॒ रेतः॑ पृ॒च्छामि॑ । \newline
25. पृ॒च्छामि॑ वा॒चो वा॒चः पृ॒च्छामि॑ पृ॒च्छामि॑ वा॒चः । \newline
26. वा॒चः प॑र॒मम् प॑र॒मं ॅवा॒चो वा॒चः प॑र॒मम् । \newline
27. प॒र॒मं ॅव्यो॑म॒ व्यो॑म पर॒मम् प॑र॒मं ॅव्यो॑म । \newline
28. व्यो॑मेति॒ वि - ओ॒म॒ । \newline
29. वेदि॑ माहु राहु॒र् वेदिं॒ ॅवेदि॑ माहुः । \newline
30. आ॒हुः॒ पर॒म् पर॑ माहु राहुः॒ पर᳚म् । \newline
31. पर॒ मन्त॒ मन्त॒म् पर॒म् पर॒ मन्त᳚म् । \newline
32. अन्त॑म् पृथि॒व्याः पृ॑थि॒व्या अन्त॒ मन्त॑म् पृथि॒व्याः । \newline
33. पृ॒थि॒व्या य॒ज्ञ्ं ॅय॒ज्ञ्म् पृ॑थि॒व्याः पृ॑थि॒व्या य॒ज्ञ्म् । \newline
34. य॒ज्ञ् मा॑हु राहुर् य॒ज्ञ्ं ॅय॒ज्ञ् मा॑हुः । \newline
35. आ॒हु॒र् भुव॑नस्य॒ भुव॑नस्या हु राहु॒र् भुव॑नस्य । \newline
36. भुव॑नस्य॒ नाभि॒न् नाभि॒म् भुव॑नस्य॒ भुव॑नस्य॒ नाभि᳚म् । \newline
37. नाभि॒मिति॒ नाभि᳚म् । \newline
38. सोम॑ माहु राहुः॒ सोमꣳ॒॒ सोम॑ माहुः । \newline
39. आ॒हु॒र् वृष्णो॒ वृष्ण॑ आहु राहु॒र् वृष्णः॑ । \newline
40. वृष्णो॒ अश्व॒स्या श्व॑स्य॒ वृष्णो॒ वृष्णो॒ अश्व॑स्य । \newline
41. अश्व॑स्य॒ रेतो॒ रेतो ऽश्व॒स्या श्व॑स्य॒ रेतः॑ । \newline
42. रेतो॒ ब्रह्म॒ ब्रह्म॒ रेतो॒ रेतो॒ ब्रह्म॑ । \newline
43. ब्रह्मै॒वैव ब्रह्म॒ ब्रह्मै॒व । \newline
44. ए॒व वा॒चो वा॒च ए॒वैव वा॒चः । \newline
45. वा॒चः प॑र॒मम् प॑र॒मं ॅवा॒चो वा॒चः प॑र॒मम् । \newline
46. प॒र॒मं ॅव्यो॑म॒ व्यो॑म पर॒मम् प॑र॒मं ॅव्यो॑म । \newline
47. व्यो॑मेति॒ वि - ओ॒म॒ । \newline

\textbf{Ghana Paata } \newline

1. च॒न्द्रमा॑ जायते जायते च॒न्द्रमा᳚ श्च॒न्द्रमा॑ जायते॒ पुनः॒ पुन॑र् जायते च॒न्द्रमा᳚ श्च॒न्द्रमा॑ जायते॒ पुनः॑ । \newline
2. जा॒य॒ते॒ पुनः॒ पुन॑र् जायते जायते॒ पुनः॑ । \newline
3. पुन॒रिति॒ पुनः॑ । \newline
4. अ॒ग्निर्. हि॒मस्य॑ हि॒म स्या॒ग्नि र॒ग्निर्. हि॒मस्य॑ भेष॒जम् भे॑ष॒जꣳ हि॒म स्या॒ग्नि र॒ग्निर्. हि॒मस्य॑ भेष॒जम् । \newline
5. हि॒मस्य॑ भेष॒जम् भे॑ष॒जꣳ हि॒मस्य॑ हि॒मस्य॑ भेष॒जम् भूमि॒र् भूमि॑र् भेष॒जꣳ हि॒मस्य॑ हि॒मस्य॑ भेष॒जम् भूमिः॑ । \newline
6. भे॒ष॒जम् भूमि॒र् भूमि॑र् भेष॒जम् भे॑ष॒जम् भूमि॑ रा॒वप॑न मा॒वप॑न॒म् भूमि॑र् भेष॒जम् भे॑ष॒जम् भूमि॑ रा॒वप॑नम् । \newline
7. भूमि॑ रा॒वप॑न मा॒वप॑न॒म् भूमि॒र् भूमि॑ रा॒वप॑नम् म॒हन् म॒ह दा॒वप॑न॒म् भूमि॒र् भूमि॑ रा॒वप॑नम् म॒हत् । \newline
8. आ॒वप॑नम् म॒हन् म॒ह दा॒वप॑न मा॒वप॑नम् म॒हत् । \newline
9. आ॒वप॑न॒मित्या᳚ - वप॑नम् । \newline
10. म॒हदिति॑ म॒हत् । \newline
11. पृ॒च्छामि॑ त्वा त्वा पृ॒च्छामि॑ पृ॒च्छामि॑ त्वा॒ पर॒म् पर॑म् त्वा पृ॒च्छामि॑ पृ॒च्छामि॑ त्वा॒ पर᳚म् । \newline
12. त्वा॒ पर॒म् पर॑म् त्वा त्वा॒ पर॒ मन्त॒ मन्त॒म् पर॑म् त्वा त्वा॒ पर॒ मन्त᳚म् । \newline
13. पर॒ मन्त॒ मन्त॒म् पर॒म् पर॒ मन्त॑म् पृथि॒व्याः पृ॑थि॒व्या अन्त॒म् पर॒म् पर॒ मन्त॑म् पृथि॒व्याः । \newline
14. अन्त॑म् पृथि॒व्याः पृ॑थि॒व्या अन्त॒ मन्त॑म् पृथि॒व्याः पृ॒च्छामि॑ पृ॒च्छामि॑ पृथि॒व्या अन्त॒ मन्त॑म् पृथि॒व्याः पृ॒च्छामि॑ । \newline
15. पृ॒थि॒व्याः पृ॒च्छामि॑ पृ॒च्छामि॑ पृथि॒व्याः पृ॑थि॒व्याः पृ॒च्छामि॑ त्वा त्वा पृ॒च्छामि॑ पृथि॒व्याः पृ॑थि॒व्याः पृ॒च्छामि॑ त्वा । \newline
16. पृ॒च्छामि॑ त्वा त्वा पृ॒च्छामि॑ पृ॒च्छामि॑ त्वा॒ भुव॑नस्य॒ भुव॑नस्य त्वा पृ॒च्छामि॑ पृ॒च्छामि॑ त्वा॒ भुव॑नस्य । \newline
17. त्वा॒ भुव॑नस्य॒ भुव॑नस्य त्वा त्वा॒ भुव॑नस्य॒ नाभि॒न् नाभि॒म् भुव॑नस्य त्वा त्वा॒ भुव॑नस्य॒ नाभि᳚म् । \newline
18. भुव॑नस्य॒ नाभि॒न् नाभि॒म् भुव॑नस्य॒ भुव॑नस्य॒ नाभि᳚म् । \newline
19. नाभि॒मिति॒ नाभि᳚म् । \newline
20. पृ॒च्छामि॑ त्वा त्वा पृ॒च्छामि॑ पृ॒च्छामि॑ त्वा॒ वृष्णो॒ वृष्ण॑ स्त्वा पृ॒च्छामि॑ पृ॒च्छामि॑ त्वा॒ वृष्णः॑ । \newline
21. त्वा॒ वृष्णो॒ वृष्ण॑ स्त्वा त्वा॒ वृष्णो॒ अश्व॒स्या श्व॑स्य॒ वृष्ण॑ स्त्वा त्वा॒ वृष्णो॒ अश्व॑स्य । \newline
22. वृष्णो॒ अश्व॒स्या श्व॑स्य॒ वृष्णो॒ वृष्णो॒ अश्व॑स्य॒ रेतो॒ रेतो ऽश्व॑स्य॒ वृष्णो॒ वृष्णो॒ अश्व॑स्य॒ रेतः॑ । \newline
23. अश्व॑स्य॒ रेतो॒ रेतो ऽश्व॒स्या श्व॑स्य॒ रेतः॑ पृ॒च्छामि॑ पृ॒च्छामि॒ रेतो ऽश्व॒स्या श्व॑स्य॒ रेतः॑ पृ॒च्छामि॑ । \newline
24. रेतः॑ पृ॒च्छामि॑ पृ॒च्छामि॒ रेतो॒ रेतः॑ पृ॒च्छामि॑ वा॒चो वा॒चः पृ॒च्छामि॒ रेतो॒ रेतः॑ पृ॒च्छामि॑ वा॒चः । \newline
25. पृ॒च्छामि॑ वा॒चो वा॒चः पृ॒च्छामि॑ पृ॒च्छामि॑ वा॒चः प॑र॒मम् प॑र॒मं ॅवा॒चः पृ॒च्छामि॑ पृ॒च्छामि॑ वा॒चः प॑र॒मम् । \newline
26. वा॒चः प॑र॒मम् प॑र॒मं ॅवा॒चो वा॒चः प॑र॒मं ॅव्यो॑म॒ व्यो॑म पर॒मं ॅवा॒चो वा॒चः प॑र॒मं ॅव्यो॑म । \newline
27. प॒र॒मं ॅव्यो॑म॒ व्यो॑म पर॒मम् प॑र॒मं ॅव्यो॑म । \newline
28. व्यो॑मेति॒ वि - ओ॒म॒ । \newline
29. वेदि॑ माहु राहु॒र् वेदिं॒ ॅवेदि॑ माहुः॒ पर॒म् पर॑ माहु॒र् वेदिं॒ ॅवेदि॑ माहुः॒ पर᳚म् । \newline
30. आ॒हुः॒ पर॒म् पर॑ माहु राहुः॒ पर॒ मन्त॒ मन्त॒म् पर॑ माहु राहुः॒ पर॒ मन्त᳚म् । \newline
31. पर॒ मन्त॒ मन्त॒म् पर॒म् पर॒ मन्त॑म् पृथि॒व्याः पृ॑थि॒व्या अन्त॒म् पर॒म् पर॒ मन्त॑म् पृथि॒व्याः । \newline
32. अन्त॑म् पृथि॒व्याः पृ॑थि॒व्या अन्त॒ मन्त॑म् पृथि॒व्या य॒ज्ञ्ं ॅय॒ज्ञ्म् पृ॑थि॒व्या अन्त॒ मन्त॑म् पृथि॒व्या य॒ज्ञ्म् । \newline
33. पृ॒थि॒व्या य॒ज्ञ्ं ॅय॒ज्ञ्म् पृ॑थि॒व्याः पृ॑थि॒व्या य॒ज्ञ् मा॑हु राहुर् य॒ज्ञ्म् पृ॑थि॒व्याः पृ॑थि॒व्या य॒ज्ञ् मा॑हुः । \newline
34. य॒ज्ञ् मा॑हु राहुर् य॒ज्ञ्ं ॅय॒ज्ञ् मा॑हु॒र् भुव॑नस्य॒ भुव॑न स्याहुर् य॒ज्ञ्ं ॅय॒ज्ञ् मा॑हु॒र् भुव॑नस्य । \newline
35. आ॒हु॒र् भुव॑नस्य॒ भुव॑न स्याहु राहु॒र् भुव॑नस्य॒ नाभि॒न् नाभि॒म् भुव॑न स्याहु राहु॒र् भुव॑नस्य॒ नाभि᳚म् । \newline
36. भुव॑नस्य॒ नाभि॒न् नाभि॒म् भुव॑नस्य॒ भुव॑नस्य॒ नाभि᳚म् । \newline
37. नाभि॒मिति॒ नाभि᳚म् । \newline
38. सोम॑ माहु राहुः॒ सोमꣳ॒॒ सोम॑ माहु॒र् वृष्णो॒ वृष्ण॑ आहुः॒ सोमꣳ॒॒ सोम॑ माहु॒र् वृष्णः॑ । \newline
39. आ॒हु॒र् वृष्णो॒ वृष्ण॑ आहु राहु॒र् वृष्णो॒ अश्व॒स्या श्व॑स्य॒ वृष्ण॑ आहु राहु॒र् वृष्णो॒ अश्व॑स्य । \newline
40. वृष्णो॒ अश्व॒स्या श्व॑स्य॒ वृष्णो॒ वृष्णो॒ अश्व॑स्य॒ रेतो॒ रेतो ऽश्व॑स्य॒ वृष्णो॒ वृष्णो॒ अश्व॑स्य॒ रेतः॑ । \newline
41. अश्व॑स्य॒ रेतो॒ रेतो ऽश्व॒स्या श्व॑स्य॒ रेतो॒ ब्रह्म॒ ब्रह्म॒ रेतो ऽश्व॒स्या श्व॑स्य॒ रेतो॒ ब्रह्म॑ । \newline
42. रेतो॒ ब्रह्म॒ ब्रह्म॒ रेतो॒ रेतो॒ ब्रह्मै॒ वैव ब्रह्म॒ रेतो॒ रेतो॒ ब्रह्मै॒व । \newline
43. ब्रह्मै॒ वैव ब्रह्म॒ ब्रह्मै॒व वा॒चो वा॒च ए॒व ब्रह्म॒ ब्रह्मै॒व वा॒चः । \newline
44. ए॒व वा॒चो वा॒च ए॒वैव वा॒चः प॑र॒मम् प॑र॒मं ॅवा॒च ए॒वैव वा॒चः प॑र॒मम् । \newline
45. वा॒चः प॑र॒मम् प॑र॒मं ॅवा॒चो वा॒चः प॑र॒मं ॅव्यो॑म॒ व्यो॑म पर॒मं ॅवा॒चो वा॒चः प॑र॒मं ॅव्यो॑म । \newline
46. प॒र॒मं ॅव्यो॑म॒ व्यो॑म पर॒मम् प॑र॒मं ॅव्यो॑म । \newline
47. व्यो॑मेति॒ वि - ओ॒म॒ । \newline
\pagebreak
\markright{ TS 7.4.19.1  \hfill https://www.vedavms.in \hfill}

\section{ TS 7.4.19.1 }

\textbf{TS 7.4.19.1 } \newline
\textbf{Samhita Paata} \newline

अबें॒ अबां॒ल्यंबि॑के॒ न मा॑ नयति॒ कश्च॒न । स॒सस्त्य॑श्व॒कः ॥ सुभ॑गे॒ कांपी॑लवासिनि सुव॒र्गे लो॒के सं प्रोर्ण्वा॑थां ।आऽहम॑जानि गर्भ॒धमा त्वम॑जासि गर्भ॒धं ॥ तौ स॒ह च॒तुरः॑ प॒दः सं प्र सा॑रयावहै ॥वृषा॑ वाꣳ रेतो॒धा रेतो॑ दधा॒तूथ् स॒क्थ्यो᳚र्गृ॒दं धे᳚ह्य॒ञ्जिमुद॑ञ्जि॒मन्व॑ज । यः स्त्री॒णां जी॑व॒भोज॑नो॒ य आ॑सां - [  ] \newline

\textbf{Pada Paata} \newline

अबें᳚ । अम्बा॑लि । अम्बि॑के । न । मा॒ । न॒य॒ति॒ । कः । च॒न ॥ स॒सस्ति॑ । अ॒श्व॒कः ॥ सुभ॑ग॒ इति॒ सु - भ॒गे॒ । काम्पी॑लवासि॒नीति॒ काम्पी॑ल - वा॒सि॒नि॒ । सु॒व॒र्ग इति॑ सुवः-गे । लो॒के । सम् । प्रेति॑ । ऊ॒र्ण्वा॒था॒म् ॥ एति॑ । अ॒हम् । अ॒जा॒नि॒ । ग॒र्भ॒धमिति॑ गर्भ - धम् । एति॑ । त्वम् । अ॒जा॒सि॒ । ग॒र्भ॒धमिति॑ गर्भ - धम् ॥ तौ । स॒ह । च॒तुरः॑ । प॒दः । सम् । प्रेति॑ । सा॒र॒या॒व॒है॒ ॥ वृषा᳚ । वा॒म् । रे॒तो॒धा इति॑ रेतः - धाः । रेतः॑ । द॒धा॒तु॒ । उदिति॑ । स॒क्थ्योः᳚ । गृ॒दम् । धे॒हि॒ । अ॒ञ्जिम् । उद॑ञ्जि॒मित्युत्-अ॒ञ्जि॒म् । अन्विति॑ । अ॒ज॒ ॥ यः । स्त्री॒णाम् । जी॒व॒भोज॑न॒ इति॑ जीव - भोज॑नः । यः । आ॒सा॒म् ।  \newline


\textbf{Krama Paata} \newline

अम्बे॒ अम्बा॑लि । अम्बा॒ल्यम्बि॑के । अम्बि॑के॒ न । न मा᳚ । मा॒ न॒य॒ति॒ । न॒य॒ति॒ कः । कश्च॒न । च॒नेति॑ च॒न ॥ स॒सस्त्य॑श्व॒कः । अ॒श्व॒क इत्य॑श्व॒कः ॥ सुभ॑गे॒ काम्पी॑लवासिनि । सुभ॑ग॒ इति॒ सु - भ॒गे॒ । काम्पी॑लवासिनि सुव॒र्गे । काम्पी॑लवासि॒नीति॒ काम्पी॑ल - वा॒सि॒नि॒ । सु॒व॒र्गे लो॒के । सु॒व॒र्ग इति॑ सुवः - गे । लो॒के सम् । सम् प्र । प्रोर्ण्वा॑थाम् । ऊ॒र्ण्वा॒था॒मित्यू᳚र्ण्वाथाम् ॥ आऽहम् । अ॒हम॑जानि । अ॒जा॒नि॒ ग॒र्भ॒धम् । ग॒र्भ॒धमा । ग॒र्भ॒धमिति॑ गर्भ - धम् । आ त्वम् । त्वम॑जासि । अ॒जा॒सि॒ ग॒र्भ॒धम् । ग॒र्भ॒धमिति॑ गर्भ - धम् ॥ तौ स॒ह । स॒ह च॒तुरः॑ । च॒तुरः॑ प॒दः । प॒दः सम् । सम् प्र । प्र सा॑रयावहै । सा॒र॒या॒व॒हा॒ इति॑ सारयावहै ॥ वृषा॑ वाम् । वाꣳ॒॒ रे॒तो॒धाः । रे॒तो॒धा रेतः॑ । रे॒तो॒धा इति॑ रेतः - धाः । रेतो॑ दधातु । द॒धा॒तूत् । उथ् स॒क्थ्योः᳚ । स॒क्थ्यो᳚र् गृ॒दम् । गृ॒दम् धे॑हि । धे॒ह्य॒ञ्जिम् । अ॒ञ्जिमुद॑ञ्जिम् । उद॑ञ्जि॒मनु॑ । उद॑ञ्जि॒मित्युत् - अ॒ञ्जि॒म् । अन्व॑ज । अ॒जेत्य॑ज ॥ य स्त्री॒णाम् । स्त्री॒णाम् जी॑व॒भोज॑नः । जी॒व॒भोज॑नो॒ यः । जी॒व॒भोज॑न॒ इति॑ जीव - भोज॑नः । य आ॑साम् । आ॒सा॒म् बि॒ल॒धाव॑नः \newline

\textbf{Jatai Paata} \newline

1. अम्बे॒ अम्बा॒ ल्यम्बा॒ ल्यम्बे ऽम्बे॒ अम्बा॑लि । \newline
2. अम्बा॒ ल्यम्बि॒के ऽम्बि॑के॒ अम्बा॒ ल्यम्बा॒ ल्यम्बि॑के । \newline
3. अम्बि॑के॒ न नाम्बि॒के ऽम्बि॑के॒ न । \newline
4. न मा॑ मा॒ न न मा᳚ । \newline
5. मा॒ न॒य॒ति॒ न॒य॒ति॒ मा॒ मा॒ न॒य॒ति॒ । \newline
6. न॒य॒ति॒ कः को न॑यति नयति॒ कः । \newline
7. क श्च॒न च॒न कः क श्च॒न । \newline
8. च॒नेति॑ च॒न । \newline
9. स॒स स्त्य॑श्व॒को᳚ ऽश्व॒कः स॒सस्ति॑ स॒स स्त्य॑श्व॒कः । \newline
10. अ॒श्व॒क इत्य॑श्व॒कः । \newline
11. सुभ॑गे॒ काम्पी॑लवासिनि॒ काम्पी॑लवासिनि॒ सुभ॑गे॒ सुभ॑गे॒ काम्पी॑लवासिनि । \newline
12. सुभ॑ग॒ इति॒ सु - भ॒गे॒ । \newline
13. काम्पी॑लवासिनि सुव॒र्गे सु॑व॒र्गे काम्पी॑लवासिनि॒ काम्पी॑लवासिनि सुव॒र्गे । \newline
14. काम्पी॑लवासि॒नीति॒ काम्पी॑ल - वा॒सि॒नि॒ । \newline
15. सु॒व॒र्गे लो॒के लो॒के सु॑व॒र्गे सु॑व॒र्गे लो॒के । \newline
16. सु॒व॒र्ग इति॑ सुवः - गे । \newline
17. लो॒के सꣳ सम् ॅलो॒के लो॒के सम् । \newline
18. सम् प्र प्र सꣳ सम् प्र । \newline
19. प्रोर्ण्वा॑था मूर्ण्वाथा॒म् प्र प्रोर्ण्वा॑थाम् । \newline
20. ऊ॒र्ण्वा॒था॒मित्यू᳚र्ण्वाथाम् । \newline
21. आ ऽह म॒ह मा ऽहम् । \newline
22. अ॒ह म॑जा न्यजा न्य॒ह म॒ह म॑जानि । \newline
23. अ॒जा॒नि॒ ग॒र्भ॒धम् ग॑र्भ॒ध म॑जा न्यजानि गर्भ॒धम् । \newline
24. ग॒र्भ॒ध मा ग॑र्भ॒धम् ग॑र्भ॒ध मा । \newline
25. ग॒र्भ॒धमिति॑ गर्भ - धम् । \newline
26. आ त्वम् त्व मा त्वम् । \newline
27. त्व म॑जा स्यजासि॒ त्वम् त्व म॑जासि । \newline
28. अ॒जा॒सि॒ ग॒र्भ॒धम् ग॑र्भ॒ध म॑जा स्यजासि गर्भ॒धम् । \newline
29. ग॒र्भ॒धमिति॑ गर्भ - धम् । \newline
30. तौ स॒ह स॒ह तौ तौ स॒ह । \newline
31. स॒ह च॒तुर॑ श्च॒तुरः॑ स॒ह स॒ह च॒तुरः॑ । \newline
32. च॒तुरः॑ प॒दः प॒द श्च॒तुर॑ श्च॒तुरः॑ प॒दः । \newline
33. प॒दः सꣳ सम् प॒दः प॒दः सम् । \newline
34. सम् प्र प्र सꣳ सम् प्र । \newline
35. प्र सा॑रयावहै सारयावहै॒ प्र प्र सा॑रयावहै । \newline
36. सा॒र॒या॒व॒हा॒ इति॑ सारयावहै । \newline
37. वृषा॑ वां ॅवां॒ ॅवृषा॒ वृषा॑ वाम् । \newline
38. वाꣳ॒॒ रे॒तो॒धा रे॑तो॒धा वां᳚ ॅवाꣳ रेतो॒धाः । \newline
39. रे॒तो॒धा रेतो॒ रेतो॑ रेतो॒धा रे॑तो॒धा रेतः॑ । \newline
40. रे॒तो॒धा इति॑ रेतः - धाः । \newline
41. रेतो॑ दधातु दधातु॒ रेतो॒ रेतो॑ दधातु । \newline
42. द॒धा॒ तूदुद् द॑धातु दधा॒तूत् । \newline
43. उथ् स॒क्थ्योः᳚ स॒क्थ्यो॑ रुदुथ् स॒क्थ्योः᳚ । \newline
44. स॒क्थ्यो᳚र् गृ॒दम् गृ॒दꣳ स॒क्थ्योः᳚ स॒क्थ्यो᳚र् गृ॒दम् । \newline
45. गृ॒दम् धे॑हि धेहि गृ॒दम् गृ॒दम् धे॑हि । \newline
46. धे॒ह्य॒ञ्जि म॒ञ्जिम् धे॑हि धेह्य॒ञ्जिम् । \newline
47. अ॒ञ्जि मुद॑ञ्जि॒ मुद॑ञ्जि म॒ञ्जि म॒ञ्जि मुद॑ञ्जिम् । \newline
48. उद॑ञ्जि॒ मन् वनूद॑ञ्जि॒ मुद॑ञ्जि॒ मनु॑ । \newline
49. उद॑ञ्जि॒मित्युत् - अ॒ञ्जि॒म् । \newline
50. अन्व॑जा॒ जान्वन् व॑ज । \newline
51. अ॒जेत्य॑ज । \newline
52. यः स्त्री॒णाꣳ स्त्री॒णां ॅयो यः स्त्री॒णाम् । \newline
53. स्त्री॒णाम् जी॑व॒भोज॑नो जीव॒भोज॑नः स्त्री॒णाꣳ स्त्री॒णाम् जी॑व॒भोज॑नः । \newline
54. जी॒व॒भोज॑नो॒ यो यो जी॑व॒भोज॑नो जीव॒भोज॑नो॒ यः । \newline
55. जी॒व॒भोज॑न॒ इति॑ जीव - भोज॑नः । \newline
56. य आ॑सा मासां॒ ॅयो य आ॑साम् । \newline
57. आ॒सा॒म् बि॒ल॒धाव॑नो बिल॒धाव॑न आसा मासाम् बिल॒धाव॑नः । \newline

\textbf{Ghana Paata } \newline

1. अम्बे॒ अम्बा॒ ल्यम्बा॒ ल्यम्बे ऽम्बे॒ अम्बा॒ ल्यम्बि॒के ऽम्बि॑के॒ अम्बा॒ ल्यम्बे ऽम्बे॒ अम्बा॒ ल्यम्बि॑के । \newline
2. अम्बा॒ ल्यम्बि॒के ऽम्बि॑के॒ अम्बा॒ ल्यम्बा॒ ल्यम्बि॑के॒ न नाम्बि॑के॒ अम्बा॒ ल्यम्बा॒ ल्यम्बि॑के॒ न । \newline
3. अम्बि॑के॒ न नाम्बि॒के ऽम्बि॑के॒ न मा॑ मा॒ नाम्बि॒के ऽम्बि॑के॒ न मा᳚ । \newline
4. न मा॑ मा॒ न न मा॑ नयति नयति मा॒ न न मा॑ नयति । \newline
5. मा॒ न॒य॒ति॒ न॒य॒ति॒ मा॒ मा॒ न॒य॒ति॒ कः को न॑यति मा मा नयति॒ कः । \newline
6. न॒य॒ति॒ कः को न॑यति नयति॒ क श्च॒न च॒न को न॑यति नयति॒ क श्च॒न । \newline
7. क श्च॒न च॒न कः क श्च॒न । \newline
8. च॒नेति॑ च॒न । \newline
9. स॒स स्त्य॑श्व॒को᳚ ऽश्व॒कः स॒सस्ति॑ स॒स स्त्य॑श्व॒कः । \newline
10. अ॒श्व॒क इत्य॑श्व॒कः । \newline
11. सुभ॑गे॒ काम्पी॑लवासिनि॒ काम्पी॑लवासिनि॒ सुभ॑गे॒ सुभ॑गे॒ काम्पी॑लवासिनि सुव॒र्गे सु॑व॒र्गे काम्पी॑लवासिनि॒ सुभ॑गे॒ सुभ॑गे॒ काम्पी॑लवासिनि सुव॒र्गे । \newline
12. सुभ॑ग॒ इति॒ सु - भ॒गे॒ । \newline
13. काम्पी॑लवासिनि सुव॒र्गे सु॑व॒र्गे काम्पी॑लवासिनि॒ काम्पी॑लवासिनि सुव॒र्गे लो॒के लो॒के सु॑व॒र्गे काम्पी॑लवासिनि॒ काम्पी॑लवासिनि सुव॒र्गे लो॒के । \newline
14. काम्पी॑लवासि॒नीति॒ काम्पी॑ल - वा॒सि॒नि॒ । \newline
15. सु॒व॒र्गे लो॒के लो॒के सु॑व॒र्गे सु॑व॒र्गे लो॒के सꣳ सम् ॅलो॒के सु॑व॒र्गे सु॑व॒र्गे लो॒के सम् । \newline
16. सु॒व॒र्ग इति॑ सुवः - गे । \newline
17. लो॒के सꣳ सम् ॅलो॒के लो॒के सम् प्र प्र सम् ॅलो॒के लो॒के सम् प्र । \newline
18. सम् प्र प्र सꣳ सम् प्रोर्ण्वा॑था मूर्ण्वाथा॒म् प्र सꣳ सम् प्रोर्ण्वा॑थाम् । \newline
19. प्रोर्ण्वा॑था मूर्ण्वाथा॒म् प्र प्रोर्ण्वा॑थाम् । \newline
20. ऊ॒र्ण्वा॒था॒मित्यू᳚र्ण्वाथाम् । \newline
21. आ ऽह म॒ह मा ऽह म॑जा न्यजा न्य॒ह मा ऽह म॑जानि । \newline
22. अ॒ह म॑जा न्यजा न्य॒ह म॒ह म॑जानि गर्भ॒धम् ग॑र्भ॒ध म॑जा न्य॒ह म॒ह म॑जानि गर्भ॒धम् । \newline
23. अ॒जा॒नि॒ ग॒र्भ॒धम् ग॑र्भ॒ध म॑जा न्यजानि गर्भ॒ध मा ग॑र्भ॒ध म॑जा न्यजानि गर्भ॒ध मा । \newline
24. ग॒र्भ॒ध मा ग॑र्भ॒धम् ग॑र्भ॒ध मा त्वम् त्व मा ग॑र्भ॒धम् ग॑र्भ॒ध मा त्वम् । \newline
25. ग॒र्भ॒धमिति॑ गर्भ - धम् । \newline
26. आ त्वम् त्व मा त्व म॑जास्य जासि॒ त्व मा त्व म॑जासि । \newline
27. त्व म॑जास्य जासि॒ त्वम् त्व म॑जासि गर्भ॒धम् ग॑र्भ॒ध म॑जासि॒ त्वम् त्व म॑जासि गर्भ॒धम् । \newline
28. अ॒जा॒सि॒ ग॒र्भ॒धम् ग॑र्भ॒ध म॑जास्य जासि गर्भ॒धम् । \newline
29. ग॒र्भ॒धमिति॑ गर्भ - धम् । \newline
30. तौ स॒ह स॒ह तौ तौ स॒ह च॒तुर॑ श्च॒तुरः॑ स॒ह तौ तौ स॒ह च॒तुरः॑ । \newline
31. स॒ह च॒तुर॑ श्च॒तुरः॑ स॒ह स॒ह च॒तुरः॑ प॒दः प॒द श्च॒तुरः॑ स॒ह स॒ह च॒तुरः॑ प॒दः । \newline
32. च॒तुरः॑ प॒दः प॒द श्च॒तुर॑ श्च॒तुरः॑ प॒दः सꣳ सम् प॒द श्च॒तुर॑ श्च॒तुरः॑ प॒दः सम् । \newline
33. प॒दः सꣳ सम् प॒दः प॒दः सम् प्र प्र सम् प॒दः प॒दः सम् प्र । \newline
34. सम् प्र प्र सꣳ सम् प्र सा॑रयावहै सारयावहै॒ प्र सꣳ सम् प्र सा॑रयावहै । \newline
35. प्र सा॑रयावहै सारयावहै॒ प्र प्र सा॑रयावहै । \newline
36. सा॒र॒या॒व॒हा॒ इति॑ सारयावहै । \newline
37. वृषा॑ वां ॅवां॒ ॅवृषा॒ वृषा॑ वाꣳ रेतो॒धा रे॑तो॒धा वां॒ ॅवृषा॒ वृषा॑ वाꣳ रेतो॒धाः । \newline
38. वाꣳ॒॒ रे॒तो॒धा रे॑तो॒धा वां᳚ ॅवाꣳ रेतो॒धा रेतो॒ रेतो॑ रेतो॒धा वां᳚ ॅवाꣳ रेतो॒धा रेतः॑ । \newline
39. रे॒तो॒धा रेतो॒ रेतो॑ रेतो॒धा रे॑तो॒धा रेतो॑ दधातु दधातु॒ रेतो॑ रेतो॒धा रे॑तो॒धा रेतो॑ दधातु । \newline
40. रे॒तो॒धा इति॑ रेतः - धाः । \newline
41. रेतो॑ दधातु दधातु॒ रेतो॒ रेतो॑ दधा॒ तूदुद् द॑धातु॒ रेतो॒ रेतो॑ दधा॒तूत् । \newline
42. द॒धा॒ तूदुद् द॑धातु दधा॒तूथ् स॒क्थ्योः᳚ स॒क्थ्यो॑रुद् द॑धातु दधा॒तूथ् स॒क्थ्योः᳚ । \newline
43. उथ् स॒क्थ्योः᳚ स॒क्थ्यो॑ रुदुथ् स॒क्थ्यो᳚र् गृ॒दम् गृ॒दꣳ स॒क्थ्यो॑ रुदुथ् स॒क्थ्यो᳚र् गृ॒दम् । \newline
44. स॒क्थ्यो᳚र् गृ॒दम् गृ॒दꣳ स॒क्थ्योः᳚ स॒क्थ्यो᳚र् गृ॒दम् धे॑हि धेहि गृ॒दꣳ स॒क्थ्योः᳚ स॒क्थ्यो᳚र् गृ॒दम् धे॑हि । \newline
45. गृ॒दम् धे॑हि धेहि गृ॒दम् गृ॒दम् धे᳚ह्य॒ञ्जि म॒ञ्जिम् धे॑हि गृ॒दम् गृ॒दम् धे᳚ह्य॒ञ्जिम् । \newline
46. धे॒ह्य॒ञ्जि म॒ञ्जिम् धे॑हि धेह्य॒ञ्जि मुद॑ञ्जि॒ मुद॑ञ्जि म॒ञ्जिम् धे॑हि धेह्य॒ञ्जि मुद॑ञ्जिम् । \newline
47. अ॒ञ्जि मुद॑ञ्जि॒ मुद॑ञ्जि म॒ञ्जि म॒ञ्जि मुद॑ञ्जि॒ मन्व नूद॑ञ्जि म॒ञ्जि म॒ञ्जि मुद॑ञ्जि॒ मनु॑ । \newline
48. उद॑ञ्जि॒ मन्व नूद॑ञ्जि॒ मुद॑ञ्जि॒ मन् व॑जा॒ जानूद॑ञ्जि॒ मुद॑ञ्जि॒ मन्व॑ज । \newline
49. उद॑ञ्जि॒मित्युत् - अ॒ञ्जि॒म् । \newline
50. अन्व॑जा॒ जान् वन् व॑ज । \newline
51. अ॒जेत्य॑ज । \newline
52. यः स्त्री॒णाꣳ स्त्री॒णां ॅयो यः स्त्री॒णाम् जी॑व॒भोज॑नो जीव॒भोज॑नः स्त्री॒णां ॅयो यः स्त्री॒णाम् जी॑व॒भोज॑नः । \newline
53. स्त्री॒णाम् जी॑व॒भोज॑नो जीव॒भोज॑नः स्त्री॒णाꣳ स्त्री॒णाम् जी॑व॒भोज॑नो॒ यो यो जी॑व॒भोज॑नः स्त्री॒णाꣳ स्त्री॒णाम् जी॑व॒भोज॑नो॒ यः । \newline
54. जी॒व॒भोज॑नो॒ यो यो जी॑व॒भोज॑नो जीव॒भोज॑नो॒ य आ॑सा मासां॒ ॅयो जी॑व॒भोज॑नो जीव॒भोज॑नो॒ य आ॑साम् । \newline
55. जी॒व॒भोज॑न॒ इति॑ जीव - भोज॑नः । \newline
56. य आ॑सा मासां॒ ॅयो य आ॑साम् बिल॒धाव॑नो बिल॒धाव॑न आसां॒ ॅयो य आ॑साम् बिल॒धाव॑नः । \newline
57. आ॒सा॒म् बि॒ल॒धाव॑नो बिल॒धाव॑न आसा मासाम् बिल॒धाव॑नः । \newline
\pagebreak
\markright{ TS 7.4.19.2  \hfill https://www.vedavms.in \hfill}

\section{ TS 7.4.19.2 }

\textbf{TS 7.4.19.2 } \newline
\textbf{Samhita Paata} \newline

बिल॒धाव॑नः । प्रि॒यः स्त्री॒णाम॑पी॒च्यः॑ ॥ य आ॑सां कृ॒ष्णे लक्ष्म॑णि॒ सर्दि॑गृदिं प॒राव॑धीत् ॥अबें॒ अबां॒ल्यंबि॑के॒ न मा॑ यभति॒ कश्च॒न । स॒सस्त्य॑श्व॒कः ॥ऊ॒र्द्ध्वा-मे॑ना॒मुच्छ्र॑यताद्-वेणुभा॒रं गि॒रावि॑व । अथा᳚स्या॒ मद्ध्य॑मेधताꣳ शी॒ते वाते॑ पु॒नन्नि॑व ॥ अबें॒ अबां॒ल्यंबि॑के॒ न मा॑ यभति॒ कश्च॒न । स॒सस्त्य॑श्व॒कः ॥ यद्ध॑रि॒णी यव॒मत्ति॒ न - [  ] \newline

\textbf{Pada Paata} \newline

बि॒ल॒धाव॑न॒ इति॑ बिल - धाव॑नः ॥ प्रि॒यः । स्त्री॒णाम् । अ॒पी॒च्यः॑ ॥ यः । आ॒सा॒म् । कृ॒ष्णे । लक्ष्म॑णि । सर्दि॑गृदिम् । प॒राव॑धी॒दिति॑ परा - अव॑धीत् ॥ अम्बे᳚ । अम्बा॑लि । अम्बि॑के । न । मा॒ । य॒भ॒ति॒ । कः । च॒न ॥ स॒सस्ति॑ । अ॒श्व॒कः ॥ ऊ॒द्‌र्ध्वाम् । ए॒ना॒म् । उदिति॑ । श्र॒य॒ता॒त् । वे॒णु॒भा॒रमिति॑ वेणु-भा॒रम् । गि॒रौ । इ॒व॒ ॥ अथ॑ । अ॒स्याः॒ । मद्ध्य᳚म् । ए॒ध॒ता॒म् । शी॒ते । वाते᳚ । पु॒नन्न् । इ॒व॒ ॥ अम्बे᳚ । अम्बा॑लि । अम्बि॑के । न । मा॒ । य॒भ॒ति॒ । कः । च॒न ॥ स॒सस्ति॑ । अ॒श्व॒कः ॥ यत् । ह॒रि॒णी । यव᳚म् । अत्ति॑ । न ।  \newline


\textbf{Krama Paata} \newline

बि॒ल॒धाव॑न॒ इति॑ बिल - धाव॑नः ॥ प्रि॒यः स्त्री॒णाम् । स्त्री॒णाम॑पी॒च्यः॑ । अ॒पी॒च्य॑ इत्य॑पी॒च्यः॑ ॥ य आ॑साम् । आ॒सा॒म् कृ॒ष्णे । कृ॒ष्णे लक्ष्म॑णि । लक्ष्म॑णि॒ सर्दि॑गृदिम् । सर्दि॑गृदिम् प॒राऽव॑धीत् । प॒राव॑धी॒दिति॑ परा - अव॑धीत् ॥ अम्बे॒ अम्बा॑लि । अम्बा॒ल्यम्बि॑के । अम्बि॑के॒ न । न मा᳚ । मा॒ य॒भ॒ति॒ । य॒भ॒ति॒ कः । कश्च॒न । च॒नेति॑ च॒न ॥ 
स॒सस्त्य॑श्व॒कः । अ॒श्व॒क इत्य॑श्व॒कः ॥ ऊ॒र्द्ध्वामे॑नाम् । ए॒ना॒मुत् । उच्छ्र॑यतात् । श्र॒य॒ता॒द् वे॒णु॒भा॒रम् । वे॒णु॒भा॒रम् गि॒रौ । वे॒णु॒भा॒रमिति॑ वेणु - भा॒रम् । गि॒रावि॑व । इ॒वेती॑व ॥ अथा᳚ऽस्याः । अ॒स्या॒ मद्ध्य᳚म् । मद्ध्य॑मेधताम् । ए॒ध॒ताꣳ॒॒ शी॒ते । शी॒ते वाते᳚ । वाते॑ पु॒नन्न् । पु॒नन्नि॑व । इ॒वेती॑व ॥ अम्बे॒ अम्बा॑लि । अम्बा॒ल्यम्बि॑के । अम्बि॑के॒ न । न मा᳚ । मा॒ य॒भ॒ति॒ । य॒भ॒ति॒ कः । कश्च॒न । च॒नेति॑ च॒न ॥ स॒सस्त्य॑श्व॒कः । अ॒श्व॒क इत्य॑श्व॒कः ॥ यद्‍ध॑रि॒णी । ह॒रि॒णी यव᳚म् । यव॒मत्ति॑ । अत्ति॒ न । न पु॒ष्टम् \newline

\textbf{Jatai Paata} \newline

1. बि॒ल॒धाव॑न॒ इति॑ बिल - धाव॑नः । \newline
2. प्रि॒यः स्त्री॒णाꣳ स्त्री॒णाम् प्रि॒यः प्रि॒यः स्त्री॒णाम् । \newline
3. स्त्री॒णा म॑पी॒च्यो॑ ऽपी॒च्यः॑ स्त्री॒णाꣳ स्त्री॒णा म॑पी॒च्यः॑ । \newline
4. अ॒पी॒च्य॑ इत्य॑पी॒च्यः॑ । \newline
5. य आ॑सा मासां॒ ॅयो य आ॑साम् । \newline
6. आ॒सा॒म् कृ॒ष्णे कृ॒ष्ण आ॑सा मासाम् कृ॒ष्णे । \newline
7. कृ॒ष्णे लक्ष्म॑णि॒ लक्ष्म॑णि कृ॒ष्णे कृ॒ष्णे लक्ष्म॑णि । \newline
8. लक्ष्म॑णि॒ सर्दि॑गृदिꣳ॒॒ सर्दि॑गृदि॒म् ॅलक्ष्म॑णि॒ लक्ष्म॑णि॒ सर्दि॑गृदिम् । \newline
9. सर्दि॑गृदिम् प॒राव॑धीत् प॒राव॑धी॒थ् सर्दि॑गृदिꣳ॒॒ सर्दि॑गृदिम् प॒राव॑धीत् । \newline
10. प॒राव॑धी॒दिति॑ परा - अव॑धीत् । \newline
11. अम्बे॒ अम्बा॒ ल्यम्बा॒ ल्यम्बे ऽम्बे॒ अम्बा॑लि । \newline
12. अम्बा॒ ल्यम्बि॒के ऽम्बि॑के॒ अम्बा॒ ल्यम्बा॒ ल्यम्बि॑के । \newline
13. अम्बि॑के॒ न नाम्बि॒के ऽम्बि॑के॒ न । \newline
14. न मा॑ मा॒ न न मा᳚ । \newline
15. मा॒ य॒भ॒ति॒ य॒भ॒ति॒ मा॒ मा॒ य॒भ॒ति॒ । \newline
16. य॒भ॒ति॒ कः को य॑भति यभति॒ कः । \newline
17. क श्च॒न च॒न कः क श्च॒न । \newline
18. च॒नेति॑ च॒न । \newline
19. स॒स स्त्य॑श्व॒को᳚ ऽश्व॒कः स॒सस्ति॑ स॒स स्त्य॑श्व॒कः । \newline
20. अ॒श्व॒क इत्य॑श्व॒कः । \newline
21. ऊ॒र्द्ध्वा मे॑ना मेना मू॒र्द्ध्वा मू॒र्द्ध्वा मे॑नाम् । \newline
22. ए॒ना॒ मुदु दे॑ना मेना॒ मुत् । \newline
23. उच्छ्र॑यताच् छ्रयता॒ दुदुच् छ्र॑यतात् । \newline
24. श्र॒य॒ता॒द् वे॒णु॒भा॒रं ॅवे॑णुभा॒रꣳ श्र॑यताच् छ्रयताद् वेणुभा॒रम् । \newline
25. वे॒णु॒भा॒रम् गि॒रौ गि॒रौ वे॑णुभा॒रं ॅवे॑णुभा॒रम् गि॒रौ । \newline
26. वे॒णु॒भा॒रमिति॑ वेणु - भा॒रम् । \newline
27. गि॒रा वि॑वेव गि॒रौ गि॒रा वि॑व । \newline
28. इ॒वेती॑व । \newline
29. अथा᳚स्या अस्या॒ अथाथा᳚ स्याः । \newline
30. अ॒स्या॒ मद्ध्य॒म् मद्ध्य॑ मस्या अस्या॒ मद्ध्य᳚म् । \newline
31. मद्ध्य॑ मेधता मेधता॒म् मद्ध्य॒म् मद्ध्य॑ मेधताम् । \newline
32. ए॒ध॒ताꣳ॒॒ शी॒ते शी॒त ए॑धता मेधताꣳ शी॒ते । \newline
33. शी॒ते वाते॒ वाते॑ शी॒ते शी॒ते वाते᳚ । \newline
34. वाते॑ पु॒नन् पु॒नन्. वाते॒ वाते॑ पु॒नन्न् । \newline
35. पु॒नन् नि॑वेव पु॒नन् पु॒नन् नि॑व । \newline
36. इ॒वेती॑व । \newline
37. अम्बे॒ अम्बा॒ ल्यम्बा॒ ल्यम्बे ऽम्बे॒ अम्बा॑लि । \newline
38. अम्बा॒ ल्यम्बि॒के ऽम्बि॑के॒ अम्बा॒ ल्यम्बा॒ ल्यम्बि॑के । \newline
39. अम्बि॑के॒ न नाम्बि॒के ऽम्बि॑के॒ न । \newline
40. न मा॑ मा॒ न न मा᳚ । \newline
41. मा॒ य॒भ॒ति॒ य॒भ॒ति॒ मा॒ मा॒ य॒भ॒ति॒ । \newline
42. य॒भ॒ति॒ कः को य॑भति यभति॒ कः । \newline
43. क श्च॒न च॒न कः क श्च॒न । \newline
44. च॒नेति॑ च॒न । \newline
45. स॒स स्त्य॑श्व॒को᳚ ऽश्व॒कः स॒सस्ति॑ स॒स स्त्य॑श्व॒कः । \newline
46. अ॒श्व॒क इत्य॑श्व॒कः । \newline
47. यद्ध॑रि॒णी ह॑रि॒णी यद् यद्ध॑रि॒णी । \newline
48. ह॒रि॒णी यवं॒ ॅयवꣳ॑ हरि॒णी ह॑रि॒णी यव᳚म् । \newline
49. यव॒ मत्त्यत्ति॒ यवं॒ ॅयव॒ मत्ति॑ । \newline
50. अत्ति॒ न नात्त्यत्ति॒ न । \newline
51. न पु॒ष्टम् पु॒ष्टन् न न पु॒ष्टम् । \newline

\textbf{Ghana Paata } \newline

1. बि॒ल॒धाव॑न॒ इति॑ बिल - धाव॑नः । \newline
2. प्रि॒यः स्त्री॒णाꣳ स्त्री॒णाम् प्रि॒यः प्रि॒यः स्त्री॒णा म॑पी॒च्यो॑ ऽपी॒च्यः॑ स्त्री॒णाम् प्रि॒यः प्रि॒यः स्त्री॒णा म॑पी॒च्यः॑ । \newline
3. स्त्री॒णा म॑पी॒च्यो॑ ऽपी॒च्यः॑ स्त्री॒णाꣳ स्त्री॒णा म॑पी॒च्यः॑ । \newline
4. अ॒पी॒च्य॑ इत्य॑पी॒च्यः॑ । \newline
5. य आ॑सा मासां॒ ॅयो य आ॑साम् कृ॒ष्णे कृ॒ष्ण आ॑सां॒ ॅयो य आ॑साम् कृ॒ष्णे । \newline
6. आ॒सा॒म् कृ॒ष्णे कृ॒ष्ण आ॑सा मासाम् कृ॒ष्णे लक्ष्म॑णि॒ लक्ष्म॑णि कृ॒ष्ण आ॑सा मासाम् कृ॒ष्णे लक्ष्म॑णि । \newline
7. कृ॒ष्णे लक्ष्म॑णि॒ लक्ष्म॑णि कृ॒ष्णे कृ॒ष्णे लक्ष्म॑णि॒ सर्दि॑गृदिꣳ॒॒ सर्दि॑गृदि॒म् ॅलक्ष्म॑णि कृ॒ष्णे कृ॒ष्णे लक्ष्म॑णि॒ सर्दि॑गृदिम् । \newline
8. लक्ष्म॑णि॒ सर्दि॑गृदिꣳ॒॒ सर्दि॑गृदि॒म् ॅलक्ष्म॑णि॒ लक्ष्म॑णि॒ सर्दि॑गृदिम् प॒राव॑धीत् प॒राव॑धी॒थ् सर्दि॑गृदि॒म् ॅलक्ष्म॑णि॒ लक्ष्म॑णि॒ सर्दि॑गृदिम् प॒राव॑धीत् । \newline
9. सर्दि॑गृदिम् प॒राव॑धीत् प॒राव॑धी॒थ् सर्दि॑गृदिꣳ॒॒ सर्दि॑गृदिम् प॒राव॑धीत् । \newline
10. प॒राव॑धी॒दिति॑ परा - अव॑धीत् । \newline
11. अम्बे॒ अम्बा॒ ल्यम्बा॒ ल्यम्बे ऽम्बे॒ अम्बा॒ ल्यम्बि॒के ऽम्बि॑के॒ अम्बा॒ ल्यम्बे ऽम्बे॒ अम्बा॒ ल्यम्बि॑के । \newline
12. अम्बा॒ ल्यम्बि॒के ऽम्बि॑के॒ अम्बा॒ ल्यम्बा॒ ल्यम्बि॑के॒ न नाम्बि॑के॒ अम्बा॒ ल्यम्बा॒ ल्यम्बि॑के॒ न । \newline
13. अम्बि॑के॒ न नाम्बि॒के ऽम्बि॑के॒ न मा॑ मा॒ नाम्बि॒के ऽम्बि॑के॒ न मा᳚ । \newline
14. न मा॑ मा॒ न न मा॑ यभति यभति मा॒ न न मा॑ यभति । \newline
15. मा॒ य॒भ॒ति॒ य॒भ॒ति॒ मा॒ मा॒ य॒भ॒ति॒ कः को य॑भति मा मा यभति॒ कः । \newline
16. य॒भ॒ति॒ कः को य॑भति यभति॒ क श्च॒न च॒न को य॑भति यभति॒ क श्च॒न । \newline
17. क श्च॒न च॒न कः क श्च॒न । \newline
18. च॒नेति॑ च॒न । \newline
19. स॒स स्त्य॑श्व॒को᳚ ऽश्व॒कः स॒सस्ति॑ स॒स स्त्य॑श्व॒कः । \newline
20. अ॒श्व॒क इत्य॑श्व॒कः । \newline
21. ऊ॒र्द्ध्वा मे॑ना मेना मू॒र्द्ध्वा मू॒र्द्ध्वा मे॑ना॒ मुदु दे॑ना मू॒र्द्ध्वा मू॒र्द्ध्वा मे॑ना॒ मुत् । \newline
22. ए॒ना॒ मुदु दे॑ना मेना॒ मुच्छ्र॑यताच् छ्रयता॒ दुदे॑ना मेना॒ मुच्छ्र॑यतात् । \newline
23. उच्छ्र॑यताच् छ्रयता॒ दुदुच् छ्र॑यताद् वेणुभा॒रं ॅवे॑णुभा॒रꣳ श्र॑यता॒ दुदुच् छ्र॑यताद् वेणुभा॒रम् । \newline
24. श्र॒य॒ता॒द् वे॒णु॒भा॒रं ॅवे॑णुभा॒रꣳ श्र॑यताच् छ्रयताद् वेणुभा॒रम् गि॒रौ गि॒रौ वे॑णुभा॒रꣳ श्र॑यताच् छ्रयताद् वेणुभा॒रम् गि॒रौ । \newline
25. वे॒णु॒भा॒रम् गि॒रौ गि॒रौ वे॑णुभा॒रं ॅवे॑णुभा॒रम् गि॒रा वि॑वेव गि॒रौ वे॑णुभा॒रं ॅवे॑णुभा॒रम् गि॒रा वि॑व । \newline
26. वे॒णु॒भा॒रमिति॑ वेणु - भा॒रम् । \newline
27. गि॒रा वि॑वेव गि॒रौ गि॒रा वि॑व । \newline
28. इ॒वेती॑व । \newline
29. अथा᳚स्या अस्या॒ अथाथा᳚ स्या॒ मद्ध्य॒म् मद्ध्य॑ मस्या॒ अथाथा᳚ स्या॒ मद्ध्य᳚म् । \newline
30. अ॒स्या॒ मद्ध्य॒म् मद्ध्य॑ मस्या अस्या॒ मद्ध्य॑ मेधता मेधता॒म् मद्ध्य॑ मस्या अस्या॒ मद्ध्य॑ मेधताम् । \newline
31. मद्ध्य॑ मेधता मेधता॒म् मद्ध्य॒म् मद्ध्य॑ मेधताꣳ शी॒ते शी॒त ए॑धता॒म् मद्ध्य॒म् मद्ध्य॑ मेधताꣳ शी॒ते । \newline
32. ए॒ध॒ताꣳ॒॒ शी॒ते शी॒त ए॑धता मेधताꣳ शी॒ते वाते॒ वाते॑ शी॒त ए॑धता मेधताꣳ शी॒ते वाते᳚ । \newline
33. शी॒ते वाते॒ वाते॑ शी॒ते शी॒ते वाते॑ पु॒नन् पु॒नन्. वाते॑ शी॒ते शी॒ते वाते॑ पु॒नन्न् । \newline
34. वाते॑ पु॒नन् पु॒नन्. वाते॒ वाते॑ पु॒नन् नि॑वेव पु॒नन्. वाते॒ वाते॑ पु॒नन् नि॑व । \newline
35. पु॒नन् नि॑वेव पु॒नन् पु॒नन् नि॑व । \newline
36. इ॒वेती॑व । \newline
37. अम्बे॒ अम्बा॒ ल्यम्बा॒ ल्यम्बे ऽम्बे॒ अम्बा॒ ल्यम्बि॒के ऽम्बि॑के॒ अम्बा॒ ल्यम्बे ऽम्बे॒ अम्बा॒ ल्यम्बि॑के । \newline
38. अम्बा॒ ल्यम्बि॒के ऽम्बि॑के॒ अम्बा॒ ल्यम्बा॒ ल्यम्बि॑के॒ न नाम्बि॑के॒ अम्बा॒ ल्यम्बा॒ ल्यम्बि॑के॒ न । \newline
39. अम्बि॑के॒ न नाम्बि॒के ऽम्बि॑के॒ न मा॑ मा॒ नाम्बि॒के ऽम्बि॑के॒ न मा᳚ । \newline
40. न मा॑ मा॒ न न मा॑ यभति यभति मा॒ न न मा॑ यभति । \newline
41. मा॒ य॒भ॒ति॒ य॒भ॒ति॒ मा॒ मा॒ य॒भ॒ति॒ कः को य॑भति मा मा यभति॒ कः । \newline
42. य॒भ॒ति॒ कः को य॑भति यभति॒ क श्च॒न च॒न को य॑भति यभति॒ क श्च॒न । \newline
43. क श्च॒न च॒न कः क श्च॒न । \newline
44. च॒नेति॑ च॒न । \newline
45. स॒स स्त्य॑श्व॒को᳚ ऽश्व॒कः स॒सस्ति॑ स॒स स्त्य॑श्व॒कः । \newline
46. अ॒श्व॒क इत्य॑श्व॒कः । \newline
47. यद्ध॑रि॒णी ह॑रि॒णी यद् यद्ध॑रि॒णी यवं॒ ॅयवꣳ॑ हरि॒णी यद् यद्ध॑रि॒णी यव᳚म् । \newline
48. ह॒रि॒णी यवं॒ ॅयवꣳ॑ हरि॒णी ह॑रि॒णी यव॒ मत्त्यत्ति॒ यवꣳ॑ हरि॒णी ह॑रि॒णी यव॒ मत्ति॑ । \newline
49. यव॒ मत्त्यत्ति॒ यवं॒ ॅयव॒ मत्ति॒ न नात्ति॒ यवं॒ ॅयव॒ मत्ति॒ न । \newline
50. अत्ति॒ न नात्त्यत्ति॒ न पु॒ष्टम् पु॒ष्टन् नात्त्यत्ति॒ न पु॒ष्टम् । \newline
51. न पु॒ष्टम् पु॒ष्टन् न न पु॒ष्टम् प॒शु प॒शु पु॒ष्टन् न न पु॒ष्टम् प॒शु । \newline
\pagebreak
\markright{ TS 7.4.19.3  \hfill https://www.vedavms.in \hfill}

\section{ TS 7.4.19.3 }

\textbf{TS 7.4.19.3 } \newline
\textbf{Samhita Paata} \newline

पु॒ष्टं प॒शु म॑न्यते । शू॒द्रा यदर्य॑जारा॒ न पोषा॑य धनायति ॥अबें॒ अबां॒ल्यंबि॑के॒ न मा॑ यभति॒ कश्च॒न । स॒सस्त्य॑श्व॒कः ॥इ॒यं ॅय॒का श॑कुन्ति॒का ऽऽहल॒मिति॒ सर्प॑ति । आह॑तं ग॒भे पसो॒ नि ज॑ल्गुलीति॒ धाणि॑का ॥ अबें॒ अबां॒ल्यंबि॑के॒ न मा॑ यभति॒ कश्च॒न । स॒सस्त्य॑श्व॒कः ॥मा॒ता च॑ ते पि॒ता च॒ तेऽग्रं॑ ॅवृ॒क्षस्य॑ रोहतः ( ) । \newline

\textbf{Pada Paata} \newline

पु॒ष्टम् । प॒शु । म॒न्य॒ते॒ ॥ शू॒द्रा । यत् । अर्य॑जा॒रेत्यर्य॑ - जा॒रा॒ । न । पोषा॑य । ध॒ना॒य॒ति॒ ॥ अम्बे᳚ । अम्बा॑लि । अम्बि॑के । न । मा॒ । य॒भ॒ति॒ । कः । च॒न ॥ स॒सस्ति॑ । अ॒श्व॒कः ॥ इ॒यम् । य॒का । श॒कु॒न्ति॒का । आ॒हल॒मित्या᳚-हल᳚म् । इति॑ । सर्प॑ति ॥ आह॑त॒मित्या-ह॒त॒म् । ग॒भे । पसः॑ । नीति॑ । ज॒ल्गु॒ली॒ति॒ । धाणि॑का ॥ अम्बे᳚ । अम्बा॑लि । अम्बि॑के । न । मा॒ । य॒भ॒ति॒ । कः । च॒न ॥ स॒सस्ति॑ । अ॒श्व॒कः ॥ मा॒ता । च॒ । ते॒ । पि॒ता । च॒ । ते॒ । अग्र᳚म् । वृ॒क्षस्य॑ । रो॒ह॒तः॒ ( ) ॥  \newline


\textbf{Krama Paata} \newline

पु॒ष्टम् प॒शु । प॒शु म॑न्यते । म॒न्य॒त॒ इति॑ मन्यते ॥ शू॒द्रा यत् । यदर्य॑जारा । अर्य॑जारा॒ न । अर्य॑जा॒रेत्यर्य॑ - जा॒रा॒ । न पोषा॑य । पोषा॑य धनायति । ध॒ना॒य॒तीति॑ धनायति ॥ अम्बे॒ अम्बा॑लि । अम्बा॒ल्यम्बि॑के । अम्बि॑के॒ न । न मा᳚ । मा॒ य॒भ॒ति॒ । य॒भ॒ति॒ कः । कश्च॒न । च॒नेति॑ च॒न ॥ स॒सस्त्य॑श्व॒कः । अ॒श्व॒क इत्य॑श्व॒कः ॥ इ॒यम् ॅय॒का । य॒का श॑कुन्ति॒का । श॒कु॒न्ति॒काऽऽहल᳚म् । आ॒हल॒मिति॑ । आ॒हल॒मित्या᳚ - हल᳚म् । इति॒ सर्प॑ति । सर्प॒तीति॒ सर्प॑ति ॥ आह॑तम् ग॒भे । आह॑त॒मित्या - ह॒त॒म् । ग॒भे पसः॑ । पसो॒ नि । नि ज॑ल्गुलीति । ज॒ल्गु॒ली॒ति॒ धाणि॑का । धाणि॒केति॒ धाणि॑का ॥ अम्बे॒ अम्बा॑लि । अम्बा॒ल्यम्बि॑के । अम्बि॑के॒ न । न मा᳚ । मा॒ य॒भ॒ति॒ । य॒भ॒ति॒ कः । कश्च॒न । च॒नेति॑ च॒न ॥ स॒सस्त्य॑श्व॒कः । अ॒श्व॒क इत्य॑श्व॒कः ॥ मा॒ता च॑ । च॒ ते॒ । ते॒ पि॒ता । पि॒ता च॑ । च॒ ते॒ । तेऽग्र᳚म् । अग्र॑म् ॅवृ॒क्षस्य॑ । वृ॒क्षस्य॑ रोहतः । रो॒ह॒त॒ इति॑ रोहतः । \newline

\textbf{Jatai Paata} \newline

1. पु॒ष्टम् प॒शु प॒शु पु॒ष्टम् पु॒ष्टम् प॒शु । \newline
2. प॒शु म॑न्यते मन्यते प॒शु प॒शु म॑न्यते । \newline
3. म॒न्य॒त॒ इति॑ मन्यते । \newline
4. शू॒द्रा यद् यच्छू॒द्रा शू॒द्रा यत् । \newline
5. यदर्य॑जा॒रा ऽर्य॑जारा॒ यद् यदर्य॑जारा । \newline
6. अर्य॑जारा॒ न नार्य॑जा॒रा ऽर्य॑जारा॒ न । \newline
7. अर्य॑जा॒रेत्यर्य॑ - जा॒रा॒ । \newline
8. न पोषा॑य॒ पोषा॑य॒ न न पोषा॑य । \newline
9. पोषा॑य धनायति धनायति॒ पोषा॑य॒ पोषा॑य धनायति । \newline
10. ध॒ना॒य॒तीति॑ धनायति । \newline
11. अम्बे॒ अम्बा॒ ल्यम्बा॒ ल्यम्बे ऽम्बे॒ अम्बा॑लि । \newline
12. अम्बा॒ ल्यम्बि॒के ऽम्बि॑के॒ अम्बा॒ ल्यम्बा॒ ल्यम्बि॑के । \newline
13. अम्बि॑के॒ न नाम्बि॒के ऽम्बि॑के॒ न । \newline
14. न मा॑ मा॒ न न मा᳚ । \newline
15. मा॒ य॒भ॒ति॒ य॒भ॒ति॒ मा॒ मा॒ य॒भ॒ति॒ । \newline
16. य॒भ॒ति॒ कः को य॑भति यभति॒ कः । \newline
17. क श्च॒न च॒न कः क श्च॒न । \newline
18. च॒नेति॑ च॒न । \newline
19. स॒स स्त्य॑श्व॒को᳚ ऽश्व॒कः स॒सस्ति॑ स॒स स्त्य॑श्व॒कः । \newline
20. अ॒श्व॒क इत्य॑श्व॒कः । \newline
21. इ॒यं ॅय॒का य॒केय मि॒यं ॅय॒का । \newline
22. य॒का श॑कुन्ति॒का श॑कुन्ति॒का य॒का य॒का श॑कुन्ति॒का । \newline
23. श॒कु॒न्ति॒का ऽऽहल॑ मा॒हलꣳ॑ शकुन्ति॒का श॑कुन्ति॒का ऽऽहल᳚म् । \newline
24. आ॒हल॒ मितीत्या॒हल॑ मा॒हल॒ मिति॑ । \newline
25. आ॒हल॒मित्या᳚ - हल᳚म् । \newline
26. इति॒ सर्प॑ति॒ सर्प॒तीतीति॒ सर्प॑ति । \newline
27. सर्प॒तीति॒ सर्प॑ति । \newline
28. आह॑तम् ग॒भे ग॒भ आह॑त॒ माह॑तम् ग॒भे । \newline
29. आह॑त॒मित्या - ह॒त॒म् । \newline
30. ग॒भे पसः॒ पसो॑ ग॒भे ग॒भे पसः॑ । \newline
31. पसो॒ नि नि पसः॒ पसो॒ नि । \newline
32. नि ज॑ल्गुलीति जल्गुलीति॒ नि नि ज॑ल्गुलीति । \newline
33. ज॒ल्गु॒ली॒ति॒ धाणि॑का॒ धाणि॑का जल्गुलीति जल्गुलीति॒ धाणि॑का । \newline
34. धाणि॒केति॒ धाणि॑का । \newline
35. अम्बे॒ अम्बा॒ ल्यम्बा॒ ल्यम्बे ऽम्बे॒ अम्बा॑लि । \newline
36. अम्बा॒ ल्यम्बि॒के ऽम्बि॑के॒ अम्बा॒ ल्यम्बा॒ ल्यम्बि॑के । \newline
37. अम्बि॑के॒ न नाम्बि॒के ऽम्बि॑के॒ न । \newline
38. न मा॑ मा॒ न न मा᳚ । \newline
39. मा॒ य॒भ॒ति॒ य॒भ॒ति॒ मा॒ मा॒ य॒भ॒ति॒ । \newline
40. य॒भ॒ति॒ कः को य॑भति यभति॒ कः । \newline
41. क श्च॒न च॒न कः क श्च॒न । \newline
42. च॒नेति॑ च॒न । \newline
43. स॒स स्त्य॑श्व॒को᳚ ऽश्व॒कः स॒सस्ति॑ स॒स स्त्य॑श्व॒कः । \newline
44. अ॒श्व॒क इत्य॑श्व॒कः । \newline
45. मा॒ता च॑ च मा॒ता मा॒ता च॑ । \newline
46. च॒ ते॒ ते॒ च॒ च॒ ते॒ । \newline
47. ते॒ पि॒ता पि॒ता ते॑ ते पि॒ता । \newline
48. पि॒ता च॑ च पि॒ता पि॒ता च॑ । \newline
49. च॒ ते॒ ते॒ च॒ च॒ ते॒ । \newline
50. ते ऽग्र॒ मग्र॑म् ते॒ ते ऽग्र᳚म् । \newline
51. अग्रं॑ ॅवृ॒क्षस्य॑ वृ॒क्ष स्याग्र॒ मग्रं॑ ॅवृ॒क्षस्य॑ । \newline
52. वृ॒क्षस्य॑ रोहतो रोहतो वृ॒क्षस्य॑ वृ॒क्षस्य॑ रोहतः । \newline
53. रो॒ह॒त॒ इति॑ रोहतः । \newline

\textbf{Ghana Paata } \newline

1. पु॒ष्टम् प॒शु प॒शु पु॒ष्टम् पु॒ष्टम् प॒शु म॑न्यते मन्यते प॒शु पु॒ष्टम् पु॒ष्टम् प॒शु म॑न्यते । \newline
2. प॒शु म॑न्यते मन्यते प॒शु प॒शु म॑न्यते । \newline
3. म॒न्य॒त॒ इति॑ मन्यते । \newline
4. शू॒द्रा यद् यच्छू॒द्रा शू॒द्रा यदर्य॑जा॒रा ऽर्य॑जारा॒ यच्छू॒द्रा शू॒द्रा यदर्य॑जारा । \newline
5. यदर्य॑जा॒रा ऽर्य॑जारा॒ यद् यदर्य॑जारा॒ न नार्य॑जारा॒ यद् यदर्य॑जारा॒ न । \newline
6. अर्य॑जारा॒ न नार्य॑जा॒रा ऽर्य॑जारा॒ न पोषा॑य॒ पोषा॑य॒ नार्य॑जा॒रा ऽर्य॑जारा॒ न पोषा॑य । \newline
7. अर्य॑जा॒रेत्यर्य॑ - जा॒रा॒ । \newline
8. न पोषा॑य॒ पोषा॑य॒ न न पोषा॑य धनायति धनायति॒ पोषा॑य॒ न न पोषा॑य धनायति । \newline
9. पोषा॑य धनायति धनायति॒ पोषा॑य॒ पोषा॑य धनायति । \newline
10. ध॒ना॒य॒तीति॑ धनायति । \newline
11. अम्बे॒ अम्बा॒ ल्यम्बा॒ ल्यम्बे ऽम्बे॒ अम्बा॒ ल्यम्बि॒के ऽम्बि॑के॒ अम्बा॒ ल्यम्बे ऽम्बे॒ अम्बा॒ ल्यम्बि॑के । \newline
12. अम्बा॒ ल्यम्बि॒के ऽम्बि॑के॒ अम्बा॒ ल्यम्बा॒ ल्यम्बि॑के॒ न नाम्बि॑के॒ अम्बा॒ ल्यम्बा॒ ल्यम्बि॑के॒ न । \newline
13. अम्बि॑के॒ न नाम्बि॒के ऽम्बि॑के॒ न मा॑ मा॒ नाम्बि॒के ऽम्बि॑के॒ न मा᳚ । \newline
14. न मा॑ मा॒ न न मा॑ यभति यभति मा॒ न न मा॑ यभति । \newline
15. मा॒ य॒भ॒ति॒ य॒भ॒ति॒ मा॒ मा॒ य॒भ॒ति॒ कः को य॑भति मा मा यभति॒ कः । \newline
16. य॒भ॒ति॒ कः को य॑भति यभति॒ क श्च॒न च॒न को य॑भति यभति॒ क श्च॒न । \newline
17. क श्च॒न च॒न कः क श्च॒न । \newline
18. च॒नेति॑ च॒न । \newline
19. स॒स स्त्य॑श्व॒को᳚ ऽश्व॒कः स॒सस्ति॑ स॒स स्त्य॑श्व॒कः । \newline
20. अ॒श्व॒क इत्य॑श्व॒कः । \newline
21. इ॒यं ॅय॒का य॒केय मि॒यं ॅय॒का श॑कुन्ति॒का श॑कुन्ति॒का य॒केय मि॒यं ॅय॒का श॑कुन्ति॒का । \newline
22. य॒का श॑कुन्ति॒का श॑कुन्ति॒का य॒का य॒का श॑कुन्ति॒का ऽऽहल॑ मा॒हलꣳ॑ शकुन्ति॒का य॒का य॒का श॑कुन्ति॒का ऽऽहल᳚म् । \newline
23. श॒कु॒न्ति॒का ऽऽहल॑ मा॒हलꣳ॑ शकुन्ति॒का श॑कुन्ति॒का ऽऽहल॒ मितीत्या॒ हलꣳ॑ शकुन्ति॒का श॑कुन्ति॒का ऽऽहल॒ मिति॑ । \newline
24. आ॒हल॒ मितीत्या॒ हल॑ मा॒हल॒ मिति॒ सर्प॑ति॒ सर्प॒तीत्या॒ हल॑ मा॒हल॒ मिति॒ सर्प॑ति । \newline
25. आ॒हल॒मित्या᳚ - हल᳚म् । \newline
26. इति॒ सर्प॑ति॒ सर्प॒ती तीति॒ सर्प॑ति । \newline
27. सर्प॒तीति॒ सर्प॑ति । \newline
28. आह॑तम् ग॒भे ग॒भ आह॑त॒ माह॑तम् ग॒भे पसः॒ पसो॑ ग॒भ आह॑त॒ माह॑तम् ग॒भे पसः॑ । \newline
29. आह॑त॒मित्या - ह॒त॒म् । \newline
30. ग॒भे पसः॒ पसो॑ ग॒भे ग॒भे पसो॒ नि नि पसो॑ ग॒भे ग॒भे पसो॒ नि । \newline
31. पसो॒ नि नि पसः॒ पसो॒ नि ज॑ल्गुलीति जल्गुलीति॒ नि पसः॒ पसो॒ नि ज॑ल्गुलीति । \newline
32. नि ज॑ल्गुलीति जल्गुलीति॒ नि नि ज॑ल्गुलीति॒ धाणि॑का॒ धाणि॑का जल्गुलीति॒ नि नि ज॑ल्गुलीति॒ धाणि॑का । \newline
33. ज॒ल्गु॒ली॒ति॒ धाणि॑का॒ धाणि॑का जल्गुलीति जल्गुलीति॒ धाणि॑का । \newline
34. धाणि॒केति॒ धाणि॑का । \newline
35. अम्बे॒ अम्बा॒ ल्यम्बा॒ ल्यम्बे ऽम्बे॒ अम्बा॒ ल्यम्बि॒के ऽम्बि॑के॒ अम्बा॒ ल्यम्बे ऽम्बे॒ अम्बा॒ ल्यम्बि॑के । \newline
36. अम्बा॒ ल्यम्बि॒के ऽम्बि॑के॒ अम्बा॒ ल्यम्बा॒ ल्यम्बि॑के॒ न नाम्बि॑के॒ अम्बा॒ ल्यम्बा॒ ल्यम्बि॑के॒ न । \newline
37. अम्बि॑के॒ न नाम्बि॒के ऽम्बि॑के॒ न मा॑ मा॒ नाम्बि॒के ऽम्बि॑के॒ न मा᳚ । \newline
38. न मा॑ मा॒ न न मा॑ यभति यभति मा॒ न न मा॑ यभति । \newline
39. मा॒ य॒भ॒ति॒ य॒भ॒ति॒ मा॒ मा॒ य॒भ॒ति॒ कः को य॑भति मा मा यभति॒ कः । \newline
40. य॒भ॒ति॒ कः को य॑भति यभति॒ क श्च॒न च॒न को य॑भति यभति॒ क श्च॒न । \newline
41. क श्च॒न च॒न कः क श्च॒न । \newline
42. च॒नेति॑ च॒न । \newline
43. स॒स स्त्य॑श्व॒को᳚ ऽश्व॒कः स॒सस्ति॑ स॒स स्त्य॑श्व॒कः । \newline
44. अ॒श्व॒क इत्य॑श्व॒कः । \newline
45. मा॒ता च॑ च मा॒ता मा॒ता च॑ ते ते च मा॒ता मा॒ता च॑ ते । \newline
46. च॒ ते॒ ते॒ च॒ च॒ ते॒ पि॒ता पि॒ता ते॑ च च ते पि॒ता । \newline
47. ते॒ पि॒ता पि॒ता ते॑ ते पि॒ता च॑ च पि॒ता ते॑ ते पि॒ता च॑ । \newline
48. पि॒ता च॑ च पि॒ता पि॒ता च॑ ते ते च पि॒ता पि॒ता च॑ ते । \newline
49. च॒ ते॒ ते॒ च॒ च॒ ते ऽग्र॒ मग्र॑म् ते च च॒ ते ऽग्र᳚म् । \newline
50. ते ऽग्र॒ मग्र॑म् ते॒ ते ऽग्रं॑ ॅवृ॒क्षस्य॑ वृ॒क्ष स्याग्र॑म् ते॒ ते ऽग्रं॑ ॅवृ॒क्षस्य॑ । \newline
51. अग्रं॑ ॅवृ॒क्षस्य॑ वृ॒क्ष स्याग्र॒ मग्रं॑ ॅवृ॒क्षस्य॑ रोहतो रोहतो वृ॒क्ष स्याग्र॒ मग्रं॑ ॅवृ॒क्षस्य॑ रोहतः । \newline
52. वृ॒क्षस्य॑ रोहतो रोहतो वृ॒क्षस्य॑ वृ॒क्षस्य॑ रोहतः । \newline
53. रो॒ह॒त॒ इति॑ रोहतः । \newline
\pagebreak
\markright{ TS 7.4.19.4  \hfill https://www.vedavms.in \hfill}

\section{ TS 7.4.19.4 }

\textbf{TS 7.4.19.4 } \newline
\textbf{Samhita Paata} \newline

प्र सु॑ला॒मीति॑ ते पि॒ता ग॒भे मु॒ष्टिम॑तꣳसयत् ॥द॒धि॒क्राव्.ण्णो॑ अकारिषं जि॒ष्णोरश्व॑स्य वा॒जिनः॑ । सुर॒भि नो॒ मुखा॑ कर॒त् प्रण॒ आयूꣳ॑षि तारिषत् ॥आपो॒ हि ष्ठा म॑यो॒भुव॒स्ता न॑ ऊ॒र्जे द॑धातन । म॒हेरणा॑य॒ चक्ष॑से ॥यो वः॑ शि॒वत॑मो॒ रस॒स्तस्य॑ भाजयते॒ह नः॑ । उ॒श॒तीरि॑व मा॒तरः॑ ॥तस्मा॒ अरं॑ गमाम वो॒ यस्य॒ क्षया॑य॒ जिन्व॑थ ( ) । आपो॑ ज॒नय॑था च नः ॥ \newline

\textbf{Pada Paata} \newline

प्रेति॑ । सु॒ला॒मि॒ । इति॑ । ते॒ । पि॒ता । ग॒भे । मु॒ष्टिम् । अ॒तꣳ॒॒स॒य॒त् ॥ द॒धि॒क्राव्.ण्ण॒ इति॑ दधि-क्राव्.ण्णः॑ । अ॒का॒रि॒ष॒म् । जि॒ष्णोः । अश्व॑स्य । वा॒जिनः॑ ॥ सु॒र॒भि । नः॒ । मुखा᳚ । क॒र॒त् । प्रेति॑ । नः॒ । आयूꣳ॑षि । ता॒रि॒ष॒त् ॥ आपः॑ । हि । स्थ । म॒यो॒भुव॒ इति॑ मयः - भुवः॑ । ताः । नः॒ । ऊ॒र्जे । द॒धा॒त॒न॒ ॥ म॒हे । रणा॑य । चक्ष॑से ॥ यः । वः॒ । शि॒वत॑म॒ इति॑ शि॒व - त॒मः॒ । रसः॑ । तस्य॑ । भा॒ज॒य॒त॒ । इ॒ह । नः॒ ॥ उ॒श॒तीः । इ॒व॒ । मा॒तरः॑ ॥ तस्मै᳚ । अर᳚म् । ग॒मा॒म॒ । वः॒ । यस्य॑ । क्षया॑य । जिन्व॑थ ( ) ॥ आपः॑ । ज॒नय॑थ । च॒ । नः॒ ॥  \newline


\textbf{Krama Paata} \newline

प्र सु॑लामि । सु॒ला॒मीति॑ । इति॑ ते । ते॒ पि॒ता । पि॒ता ग॒भे । ग॒भे मु॒ष्टिम् । मु॒ष्टिम॑तꣳसयत् । अ॒तꣳ॒॒स॒य॒दित्य॑तꣳसयत् ॥ द॒धि॒क्राव्.ण्णो॑ अकारिषम् । द॒धि॒क्राव्.ण्ण॒ इति॑ दधि - क्राव्.ण्णः॑ । अ॒का॒रि॒ष॒म् जि॒ष्णोः । जि॒ष्णोरश्व॑स्य । अश्व॑स्य वा॒जिनः॑ । वा॒जिन॒ इति॑ वा॒जिनः॑ ॥ सु॒र॒भि नः॑ । नो॒ मुखा᳚ । मुखा॑ करत् । क॒र॒त् प्र । प्र णः॑ । न॒ आयूꣳ॑षि । आयूꣳ॑षि तारिषत् । ता॒रि॒ष॒दिति॑ तारिषत् ॥ आपो॒ हि । हि ष्ठ । स्था म॑यो॒भुवः॑ । म॒यो॒भुव॒स्ताः । म॒यो॒भुव॒ इति॑ मयः - भुवः॑ । ता नः॑ । न॒ ऊ॒र्जे । ऊ॒र्जे द॑धातन । द॒धा॒त॒नेति॑ दधातन ॥ म॒हे रणा॑य । रणा॑य॒ चक्ष॑से । चक्ष॑स॒ इति॒ चक्ष॑से ॥ यो वः॑ । वः॒ शि॒वत॑मः । शि॒वत॑मो॒ रसः॑ । शि॒वत॑म॒ इति॑ शि॒व - त॒मः॒ । रस॒स्तस्य॑ । तस्य॑ भाजयत । भा॒ज॒य॒ते॒ह । इ॒ह नः॑ । न॒ इति॑ नः ॥ उ॒श॒तीरि॑व । इ॒व॒ मा॒तरः॑ । मा॒तर॒ इति॑ मा॒तरः॑ ॥ तस्मा॒ अर᳚म् । अर॑म् गमाम । ग॒मा॒म॒ वः॒ । वो॒ यस्य॑ । यस्य॒ क्षया॑य । क्षया॑य॒ जिन्व॑थ ( ) । जिन्व॒थेति॒ जिन्व॑थ ॥ आपो॑ ज॒नय॑थ । ज॒नय॑था च । च॒ नः॒ । न॒ इति॑ नः । \newline

\textbf{Jatai Paata} \newline

1. प्र सु॑लामि सुलामि॒ प्र प्र सु॑लामि । \newline
2. सु॒ला॒मीतीति॑ सुलामि सुला॒मीति॑ । \newline
3. इति॑ ते त॒ इतीति॑ ते । \newline
4. ते॒ पि॒ता पि॒ता ते॑ ते पि॒ता । \newline
5. पि॒ता ग॒भे ग॒भे पि॒ता पि॒ता ग॒भे । \newline
6. ग॒भे मु॒ष्टिम् मु॒ष्टिम् ग॒भे ग॒भे मु॒ष्टिम् । \newline
7. मु॒ष्टि म॑तꣳसय दतꣳसयन् मु॒ष्टिम् मु॒ष्टि म॑तꣳसयत् । \newline
8. अ॒तꣳ॒॒स॒य॒दित्य॑सꣳसयत् । \newline
9. द॒धि॒क्राव्.ण्णो॑ अकारिष मकारिषम् दधि॒क्राव्.ण्णो॑ दधि॒क्राव्.ण्णो॑ अकारिषम् । \newline
10. द॒धि॒क्राव्.ण्ण॒ इति॑ दधि - क्राव्.ण्णः॑ । \newline
11. अ॒का॒रि॒ष॒म् जि॒ष्णोर् जि॒ष्णो र॑कारिष मकारिषम् जि॒ष्णोः । \newline
12. जि॒ष्णो रश्व॒स्या श्व॑स्य जि॒ष्णोर् जि॒ष्णो रश्व॑स्य । \newline
13. अश्व॑स्य वा॒जिनो॑ वा॒जिनो॒ अश्व॒स्या श्व॑स्य वा॒जिनः॑ । \newline
14. वा॒जिन॒ इति॑ वा॒जिनः॑ । \newline
15. सु॒र॒भि नो॑ नः सुर॒भि सु॑र॒भि नः॑ । \newline
16. नो॒ मुखा॒ मुखा॑ नो नो॒ मुखा᳚ । \newline
17. मुखा॑ करत् कर॒न् मुखा॒ मुखा॑ करत् । \newline
18. क॒र॒त् प्र प्र क॑रत् कर॒त् प्र । \newline
19. प्र णो॑ नः॒ प्र प्र णः॑ । \newline
20. न॒ आयूꣳ॒॒ ष्यायूꣳ॑षि नो न॒ आयूꣳ॑षि । \newline
21. आयूꣳ॑षि तारिषत् तारिष॒दा यूꣳ॒॒ ष्यायूꣳ॑षि तारिषत् । \newline
22. ता॒रि॒ष॒दिति॑ तारिषत् । \newline
23. आपो॒ हि ह्याप॒ आपो॒ हि । \newline
24. हि ष्ठ स्थ हि हि ष्ठ । \newline
25. स्था म॑यो॒भुवो॑ मयो॒भुवः॒ स्थ स्था म॑यो॒भुवः॑ । \newline
26. म॒यो॒भुव॒ स्ता स्ता म॑यो॒भुवो॑ मयो॒भुव॒ स्ताः । \newline
27. म॒यो॒भुव॒ इति॑ मयः - भुवः॑ । \newline
28. ता नो॑ न॒ स्ता स्ता नः॑ । \newline
29. न॒ ऊ॒र्ज ऊ॒र्जे नो॑ न ऊ॒र्जे । \newline
30. ऊ॒र्जे द॑धातन दधात नो॒र्ज ऊ॒र्जे द॑धातन । \newline
31. द॒धा॒त॒नेति॑ दधातन । \newline
32. म॒हे रणा॑य॒ रणा॑य म॒हे म॒हे रणा॑य । \newline
33. रणा॑य॒ चक्ष॑से॒ चक्ष॑से॒ रणा॑य॒ रणा॑य॒ चक्ष॑से । \newline
34. चक्ष॑स॒ इति॒ चक्ष॑से । \newline
35. यो वो॑ वो॒ यो यो वः॑ । \newline
36. वः॒ शि॒वत॑मः शि॒वत॑मो वो वः शि॒वत॑मः । \newline
37. शि॒वत॑मो॒ रसो॒ रसः॑ शि॒वत॑मः शि॒वत॑मो॒ रसः॑ । \newline
38. शि॒वत॑म॒ इति॑ शि॒व - त॒मः॒ । \newline
39. रस॒ स्तस्य॒ तस्य॒ रसो॒ रस॒ स्तस्य॑ । \newline
40. तस्य॑ भाजयत भाजयत॒ तस्य॒ तस्य॑ भाजयत । \newline
41. भा॒ज॒य॒ ते॒हेह भा॑जयत भाजय ते॒ह । \newline
42. इ॒ह नो॑ न इ॒हे ह नः॑ । \newline
43. न॒ इति॑ नः । \newline
44. उ॒श॒ती रि॑वे वोश॒ती रु॑श॒ती रि॑व । \newline
45. इ॒व॒ मा॒तरो॑ मा॒तर॑ इवेव मा॒तरः॑ । \newline
46. मा॒तर॒ इति॑ मा॒तरः॑ । \newline
47. तस्मा॒ अर॒ मर॒म् तस्मै॒ तस्मा॒ अर᳚म् । \newline
48. अर॑म् गमाम गमा॒ मार॒ मर॑म् गमाम । \newline
49. ग॒मा॒म॒ वो॒ वो॒ ग॒मा॒म॒ ग॒मा॒म॒ वः॒ । \newline
50. वो॒ यस्य॒ यस्य॑ वो वो॒ यस्य॑ । \newline
51. यस्य॒ क्षया॑य॒ क्षया॑य॒ यस्य॒ यस्य॒ क्षया॑य । \newline
52. क्षया॑य॒ जिन्व॑थ॒ जिन्व॑थ॒ क्षया॑य॒ क्षया॑य॒ जिन्व॑थ । \newline
53. जिन्व॒थेति॒ जिन्व॑थ । \newline
54. आपो॑ ज॒नय॑थ ज॒नय॒ थाप॒ आपो॑ ज॒नय॑थ । \newline
55. ज॒नय॑था च च ज॒नय॑थ ज॒नय॑था च । \newline
56. च॒ नो॒ न॒ श्च॒ च॒ नः॒ । \newline
57. न॒ इति॑ नः । \newline

\textbf{Ghana Paata } \newline

1. प्र सु॑लामि सुलामि॒ प्र प्र सु॑ला॒मीतीति॑ सुलामि॒ प्र प्र सु॑ला॒ मीति॑ । \newline
2. सु॒ला॒मीतीति॑ सुलामि सुला॒मीति॑ ते त॒ इति॑ सुलामि सुला॒मीति॑ ते । \newline
3. इति॑ ते त॒ इतीति॑ ते पि॒ता पि॒ता त॒ इतीति॑ ते पि॒ता । \newline
4. ते॒ पि॒ता पि॒ता ते॑ ते पि॒ता ग॒भे ग॒भे पि॒ता ते॑ ते पि॒ता ग॒भे । \newline
5. पि॒ता ग॒भे ग॒भे पि॒ता पि॒ता ग॒भे मु॒ष्टिम् मु॒ष्टिम् ग॒भे पि॒ता पि॒ता ग॒भे मु॒ष्टिम् । \newline
6. ग॒भे मु॒ष्टिम् मु॒ष्टिम् ग॒भे ग॒भे मु॒ष्टि म॑तꣳसय दतꣳसयन् मु॒ष्टिम् ग॒भे ग॒भे मु॒ष्टि म॑तꣳसयत् । \newline
7. मु॒ष्टि म॑तꣳसय दतꣳसयन् मु॒ष्टिम् मु॒ष्टि म॑तꣳसयत् । \newline
8. अ॒तꣳ॒॒स॒य॒दित्य॑सꣳसयत् । \newline
9. द॒धि॒क्राव्.ण्णो॑ अकारिष मकारिषम् दधि॒क्राव्.ण्णो॑ दधि॒क्राव्.ण्णो॑ अकारिषम् जि॒ष्णोर् जि॒ष्णो र॑कारिषम् दधि॒क्राव्.ण्णो॑ दधि॒क्राव्.ण्णो॑ अकारिषम् जि॒ष्णोः । \newline
10. द॒धि॒क्राव्.ण्ण॒ इति॑ दधि - क्राव्.ण्णः॑ । \newline
11. अ॒का॒रि॒ष॒म् जि॒ष्णोर् जि॒ष्णो र॑कारिष मकारिषम् जि॒ष्णो रश्व॒स्या श्व॑स्य जि॒ष्णो र॑कारिष मकारिषम् जि॒ष्णो रश्व॑स्य । \newline
12. जि॒ष्णो रश्व॒स्या श्व॑स्य जि॒ष्णोर् जि॒ष्णो रश्व॑स्य वा॒जिनो॑ वा॒जिनो॒ अश्व॑स्य जि॒ष्णोर् जि॒ष्णो रश्व॑स्य वा॒जिनः॑ । \newline
13. अश्व॑स्य वा॒जिनो॑ वा॒जिनो॒ अश्व॒स्या श्व॑स्य वा॒जिनः॑ । \newline
14. वा॒जिन॒ इति॑ वा॒जिनः॑ । \newline
15. सु॒र॒भि नो॑ नः सुर॒भि सु॑र॒भि नो॒ मुखा॒ मुखा॑ नः सुर॒भि सु॑र॒भि नो॒ मुखा᳚ । \newline
16. नो॒ मुखा॒ मुखा॑ नो नो॒ मुखा॑ करत् कर॒न् मुखा॑ नो नो॒ मुखा॑ करत् । \newline
17. मुखा॑ करत् कर॒न् मुखा॒ मुखा॑ कर॒त् प्र प्र क॑र॒न् मुखा॒ मुखा॑ कर॒त् प्र । \newline
18. क॒र॒त् प्र प्र क॑रत् कर॒त् प्र णो॑ नः॒ प्र क॑रत् कर॒त् प्र णः॑ । \newline
19. प्र णो॑ नः॒ प्र प्र ण॒ आयूꣳ॒॒ ष्यायूꣳ॑षि नः॒ प्र प्र ण॒ आयूꣳ॑षि । \newline
20. न॒ आयूꣳ॒॒ ष्यायूꣳ॑षि नो न॒ आयूꣳ॑षि तारिषत् तारिष॒ दायूꣳ॑षि नो न॒ आयूꣳ॑षि तारिषत् । \newline
21. आयूꣳ॑षि तारिषत् तारिष॒ दायूꣳ॒॒ ष्यायूꣳ॑षि तारिषत् । \newline
22. ता॒रि॒ष॒दिति॑ तारिषत् । \newline
23. आपो॒ हि ह्याप॒ आपो॒ हि ष्ठ स्थ ह्याप॒ आपो॒ हि ष्ठ । \newline
24. हि ष्ठ स्थ हि हि ष्ठा म॑यो॒भुवो॑ मयो॒भुवः॒ स्थ हि हि ष्ठा म॑यो॒भुवः॑ । \newline
25. स्था म॑यो॒भुवो॑ मयो॒भुवः॒ स्थ स्था म॑यो॒भुव॒ स्ता स्ता म॑यो॒भुवः॒ स्थ स्था म॑यो॒भुव॒ स्ताः । \newline
26. म॒यो॒भुव॒ स्ता स्ता म॑यो॒भुवो॑ मयो॒भुव॒ स्ता नो॑ न॒ स्ता म॑यो॒भुवो॑ मयो॒भुव॒ स्ता नः॑ । \newline
27. म॒यो॒भुव॒ इति॑ मयः - भुवः॑ । \newline
28. ता नो॑ न॒ स्ता स्ता न॑ ऊ॒र्ज ऊ॒र्जे न॒ स्ता स्ता न॑ ऊ॒र्जे । \newline
29. न॒ ऊ॒र्ज ऊ॒र्जे नो॑ न ऊ॒र्जे द॑धातन दधात नो॒र्जे नो॑ न ऊ॒र्जे द॑धातन । \newline
30. ऊ॒र्जे द॑धातन दधात नो॒र्ज ऊ॒र्जे द॑धातन । \newline
31. द॒धा॒त॒नेति॑ दधातन । \newline
32. म॒हे रणा॑य॒ रणा॑य म॒हे म॒हे रणा॑य॒ चक्ष॑से॒ चक्ष॑से॒ रणा॑य म॒हे म॒हे रणा॑य॒ चक्ष॑से । \newline
33. रणा॑य॒ चक्ष॑से॒ चक्ष॑से॒ रणा॑य॒ रणा॑य॒ चक्ष॑से । \newline
34. चक्ष॑स॒ इति॒ चक्ष॑से । \newline
35. यो वो॑ वो॒ यो यो वः॑ शि॒वत॑मः शि॒वत॑मो वो॒ यो यो वः॑ शि॒वत॑मः । \newline
36. वः॒ शि॒वत॑मः शि॒वत॑मो वो वः शि॒वत॑मो॒ रसो॒ रसः॑ शि॒वत॑मो वो वः शि॒वत॑मो॒ रसः॑ । \newline
37. शि॒वत॑मो॒ रसो॒ रसः॑ शि॒वत॑मः शि॒वत॑मो॒ रस॒ स्तस्य॒ तस्य॒ रसः॑ शि॒वत॑मः शि॒वत॑मो॒ रस॒ स्तस्य॑ । \newline
38. शि॒वत॑म॒ इति॑ शि॒व - त॒मः॒ । \newline
39. रस॒ स्तस्य॒ तस्य॒ रसो॒ रस॒ स्तस्य॑ भाजयत भाजयत॒ तस्य॒ रसो॒ रस॒ स्तस्य॑ भाजयत । \newline
40. तस्य॑ भाजयत भाजयत॒ तस्य॒ तस्य॑ भाजयते॒ हेह भा॑जयत॒ तस्य॒ तस्य॑ भाजयते॒ह । \newline
41. भा॒ज॒य॒ते॒ हेह भा॑जयत भाजयते॒ह नो॑ न इ॒ह भा॑जयत भाजयते॒ह नः॑ । \newline
42. इ॒ह नो॑ न इ॒हेह नः॑ । \newline
43. न॒ इति॑ नः । \newline
44. उ॒श॒ती रि॑वे वोश॒ती रु॑श॒ती रि॑व मा॒तरो॑ मा॒तर॑ इवोश॒ती रु॑श॒ती रि॑व मा॒तरः॑ । \newline
45. इ॒व॒ मा॒तरो॑ मा॒तर॑ इवेव मा॒तरः॑ । \newline
46. मा॒तर॒ इति॑ मा॒तरः॑ । \newline
47. तस्मा॒ अर॒ मर॒म् तस्मै॒ तस्मा॒ अर॑म् गमाम गमा॒ मार॒म् तस्मै॒ तस्मा॒ अर॑म् गमाम । \newline
48. अर॑म् गमाम गमा॒ मार॒ मर॑म् गमाम वो वो गमा॒ मार॒ मर॑म् गमाम वः । \newline
49. ग॒मा॒म॒ वो॒ वो॒ ग॒मा॒म॒ ग॒मा॒म॒ वो॒ यस्य॒ यस्य॑ वो गमाम गमाम वो॒ यस्य॑ । \newline
50. वो॒ यस्य॒ यस्य॑ वो वो॒ यस्य॒ क्षया॑य॒ क्षया॑य॒ यस्य॑ वो वो॒ यस्य॒ क्षया॑य । \newline
51. यस्य॒ क्षया॑य॒ क्षया॑य॒ यस्य॒ यस्य॒ क्षया॑य॒ जिन्व॑थ॒ जिन्व॑थ॒ क्षया॑य॒ यस्य॒ यस्य॒ क्षया॑य॒ जिन्व॑थ । \newline
52. क्षया॑य॒ जिन्व॑थ॒ जिन्व॑थ॒ क्षया॑य॒ क्षया॑य॒ जिन्व॑थ । \newline
53. जिन्व॒थेति॒ जिन्व॑थ । \newline
54. आपो॑ ज॒नय॑थ ज॒नय॒ थाप॒ आपो॑ ज॒नय॑था च च ज॒नय॒ थाप॒ आपो॑ ज॒नय॑था च । \newline
55. ज॒नय॑था च च ज॒नय॑थ ज॒नय॑था च नो नश्च ज॒नय॑थ ज॒नय॑था च नः । \newline
56. च॒ नो॒ न॒श्च॒ च॒ नः॒ । \newline
57. न॒ इति॑ नः । \newline
\pagebreak
\markright{ TS 7.4.20.1  \hfill https://www.vedavms.in \hfill}

\section{ TS 7.4.20.1 }

\textbf{TS 7.4.20.1 } \newline
\textbf{Samhita Paata} \newline

भूर्भुवः॒ सुव॒र्वस॑वस्त्वा ऽञ्जन्तु गाय॒त्रेण॒ छन्द॑सा रु॒द्रास्त्वा᳚ ऽञ्जन्तु॒ त्रैष्टु॑भेन॒ छन्द॑सा, ऽऽदि॒त्यास्त्वा᳚ऽञ्जन्तु॒ जाग॑तेन॒ छन्द॑सा॒ यद्-वातो॑ अ॒पो अग॑म॒दिन्द्र॑स्य त॒नुवं॑ प्रि॒यां । ए॒तꣳ स्तो॑तरे॒तेन॑ प॒था पुन॒रश्व॒मा व॑र्तयासि नः ॥ लाजी(3)ञ्छाची(3)न् यशो॑ म॒माॅ(4) । य॒व्यायै॑ ग॒व्याया॑ ए॒तद्-दे॑वा॒ अन्न॑मत्तै॒तदन्न॑मद्धि प्रजापते ॥ यु॒ञ्जन्ति॑ ब्र॒द्ध्न ( )-म॑रु॒षं चर॑न्तं॒ परि॑ त॒स्थुषः॑ । रोच॑न्ते रोच॒ना दि॒वि ॥ यु॒ञ्जन्त्य॑स्य॒ काम्या॒ हरी॒ विप॑क्षसा॒ रथे᳚ । शोणा॑ धृ॒ष्णू नृ॒वाह॑सा ॥ के॒तुं कृ॒ण्वन्न॑के॒तवे॒ पेशो॑ मर्या अपे॒शसे᳚ । समु॒षद्भि॑रजायथाः ॥ \newline

\textbf{Pada Paata} \newline

भूः । भुवः॑ । सुवः॑ । वस॑वः । त्वा॒ । अ॒ञ्ज॒न्तु॒ । गा॒य॒त्रेण॑ । छन्द॑सा । रु॒द्राः । त्वा॒ । अ॒ञ्ज॒न्तु॒ । त्रैष्टु॑भेन । छन्द॑सा । आ॒दि॒त्याः । त्वा॒ । अ॒ञ्ज॒न्तु॒ । जाग॑तेन । छन्द॑सा । यत् । वातः॑ । अ॒पः । अग॑मत् । इन्द्र॑स्य । त॒नुव᳚म् । प्रि॒याम् ॥ ए॒तम् । स्तो॒तः॒ । ए॒तेन॑ । प॒था । पुनः॑ । अश्व᳚म् । एति॑ । व॒र्त॒या॒सि॒ । नः॒ ॥ लाजी(3)न् । शाची(3)न् । यशः॑ । म॒मा(4)ॅम् ॥ य॒व्यायै᳚ । ग॒व्यायै᳚ । ए॒तत् । दे॒वाः॒ । अन्न᳚म् । अ॒त्त॒ । ए॒तत् । अन्न᳚म् । अ॒द्धि॒ । प्र॒जा॒प॒त॒ इति॑ प्रजा - प॒ते॒ ॥ यु॒ञ्जन्ति॑ । ब्र॒द्ध्नम् ( ) । अ॒रु॒षम् । चर॑न्तम् । परीति॑ । त॒स्थुषः॑ ॥ रोच॑न्ते । रो॒च॒ना । दि॒वि ॥ यु॒ञ्जन्ति॑ । अ॒स्य॒ । काम्या᳚ । हरी॒ इति॑ । विप॑क्ष॒सेति॒ वि - प॒क्ष॒सा॒ । रथे᳚ ॥ शोणा᳚ । धृ॒ष्णू इति॑ । नृ॒वाह॒सेति॑ नृ-वाह॑सा ॥ के॒तुम् । कृ॒ण्वन्न् । अ॒के॒तवे᳚ । पेशः॑ । म॒र्याः॒ । अ॒पे॒शसे᳚ ॥ समिति॑ । उ॒षद्भि॒रित्यु॒षत् - भिः॒ । अ॒जा॒य॒थाः॒ ॥  \newline


\textbf{Krama Paata} \newline

भूर् भुवः॑ । भुवः॒ सुवः॑ । सुव॒र् वस॑वः । वस॑वस्त्वा । त्वा॒ऽञ्ज॒न्तु॒ । अ॒ञ्ज॒न्तु॒ गा॒य॒त्रेण॑ । गा॒य॒त्रेण॒ छन्द॑सा । छन्द॑सा रु॒द्राः । रु॒द्रास्त्वा᳚ । त्वा॒ऽञ्ज॒न्तु॒ । अ॒ञ्ज॒न्तु॒ त्रैष्टु॑भेन । त्रैष्टु॑भेन॒ छन्द॑सा । छन्द॑साऽऽदि॒त्याः । आ॒दि॒त्यास्त्वा᳚ । त्वा॒ऽञ्ज॒न्तु॒ । अ॒ञ्ज॒न्तु॒ जाग॑तेन । जाग॑तेन॒ छन्द॑सा । छन्द॑सा॒ यत् । यद् वातः॑ । वातो॑ अ॒पः । अ॒पो अग॑मत् । अग॑म॒दिन्द्र॑स्य । इन्द्र॑स्य त॒नुव᳚म् । त॒नुव॑म् प्रि॒याम् । प्रि॒यामिति॑ प्रि॒याम् ॥ ए॒तꣳ स्तो॑तः । स्तो॒त॒रे॒तेन॑ । ए॒तेन॑ प॒था । प॒था पुनः॑ । पुन॒रश्व᳚म् । अश्व॒मा । आ व॑र्तयासि । व॒र्त॒या॒सि॒ नः॒ । न॒ इति॑ नः ॥ लाजी(3)ञ्छाची(3)न् । शाचीऽ(3)न्. यशः॑ । यशो॑ म॒माॅ(4) । म॒माॅ(4) इति॑ म॒माॅ(4) ॥ य॒व्यायै॑ ग॒व्यायै᳚ । ग॒व्याया॑ ए॒तत् । ए॒तद् दे॑वाः । दे॒वा॒ अन्न᳚म् । अन्न॑मत्त । अ॒त्तै॒तत् । ए॒तदन्न᳚म् । अन्न॑मद्धि । अ॒द्धि॒ प्र॒जा॒प॒ते॒ । प्र॒जा॒प॒त॒ इति॑ प्रजा - प॒ते॒ ॥ यु॒ञ्जन्ति॑ ब्र॒द्ध्नम् ( ) । ब्र॒द्ध्नम॑रु॒षम् । अ॒रु॒षम् चर॑न्तम् । चर॑न्त॒म् परि॑ । परि॑ त॒स्थुषः॑ । त॒स्थुष॒ इति॑ त॒स्थुषः॑ ॥ रोच॑न्ते रोच॒ना । रो॒च॒ना दि॒वि । दि॒वीति॑ दि॒वि ॥ यु॒ञ्जन्त्य॑स्य । अ॒स्य॒ काम्या᳚ । काम्या॒ हरी᳚ । हरी॒ विप॑क्षसा । हरी॒ इति॒ हरी᳚ । विप॑क्षसा॒ रथे᳚ । विप॑क्ष॒सेति॒ वि - प॒क्ष॒सा॒ ॥ रथ॒ इति॒ रथे᳚ ॥ शोणा॑ धृ॒ष्णू । धृ॒ष्णू नृ॒वाह॑सा । धृ॒ष्णू इति॑ धृ॒ष्णू । नृ॒वाह॒सेति॑ नृ - वाह॑सा ॥ के॒तुम् कृ॒ण्वन्न् । कृ॒ण्वन्न॑के॒तवे᳚ । अ॒के॒तवे॒ पेशः॑ । पेशो॑ मर्याः । म॒र्या॒ अ॒पे॒शसे᳚ । अ॒पे॒शस॒ इत्य॑पे॒शसे᳚ ॥ समु॒षद्‌भिः॑ । उ॒षद्‌भि॑रजायथाः । उ॒षद्‌भि॒रित्यु॒षत् - भिः॒ । अ॒जा॒य॒था॒ इत्य॑जायथाः । \newline

\textbf{Jatai Paata} \newline

1. भूर् भुवो॒ भुवो॒ भूर् भूर् भुवः॑ । \newline
2. भुवः॒ सुवः॒ सुव॒र् भुवो॒ भुवः॒ सुवः॑ । \newline
3. सुव॒र् वस॑वो॒ वस॑वः॒ सुवः॒ सुव॒र् वस॑वः । \newline
4. वस॑व स्त्वा त्वा॒ वस॑वो॒ वस॑व स्त्वा । \newline
5. त्वा॒ ऽञ्ज॒न्त्व॒ ञ्ज॒न्तु॒ त्वा॒ त्वा॒ ऽञ्ज॒न्तु॒ । \newline
6. अ॒ञ्ज॒न्तु॒ गा॒य॒त्रेण॑ गाय॒त्रेणा᳚ ञ्जन्त्व ञ्जन्तु गाय॒त्रेण॑ । \newline
7. गा॒य॒त्रेण॒ छन्द॑सा॒ छन्द॑सा गाय॒त्रेण॑ गाय॒त्रेण॒ छन्द॑सा । \newline
8. छन्द॑सा रु॒द्रा रु॒द्रा श्छन्द॑सा॒ छन्द॑सा रु॒द्राः । \newline
9. रु॒द्रा स्त्वा᳚ त्वा रु॒द्रा रु॒द्रा स्त्वा᳚ । \newline
10. त्वा॒ ऽञ्ज॒न्त्व॒ ञ्ज॒न्तु॒ त्वा॒ त्वा॒ ऽञ्ज॒न्तु॒ । \newline
11. अ॒ञ्ज॒न्तु॒ त्रैष्टु॑भेन॒ त्रैष्टु॑भेना ञ्ज न्त्वञ्जन्तु॒ त्रैष्टु॑भेन । \newline
12. त्रैष्टु॑भेन॒ छन्द॑सा॒ छन्द॑सा॒ त्रैष्टु॑भेन॒ त्रैष्टु॑भेन॒ छन्द॑सा । \newline
13. छन्द॑सा ऽऽदि॒त्या आ॑दि॒त्या श्छन्द॑सा॒ छन्द॑सा ऽऽदि॒त्याः । \newline
14. आ॒दि॒त्या स्त्वा᳚ त्वा ऽऽदि॒त्या आ॑दि॒त्या स्त्वा᳚ । \newline
15. त्वा॒ ऽञ्ज॒ न्त्व॒ञ्ज॒न्तु॒ त्वा॒ त्वा॒ ऽञ्ज॒न्तु॒ । \newline
16. अ॒ञ्ज॒न्तु॒ जाग॑तेन॒ जाग॑तेना ञ्ज न्त्वञ्जन्तु॒ जाग॑तेन । \newline
17. जाग॑तेन॒ छन्द॑सा॒ छन्द॑सा॒ जाग॑तेन॒ जाग॑तेन॒ छन्द॑सा । \newline
18. छन्द॑सा॒ यद् यच् छन्द॑सा॒ छन्द॑सा॒ यत् । \newline
19. यद् वातो॒ वातो॒ यद् यद् वातः॑ । \newline
20. वातो॑ अ॒पो॑ ऽपो वातो॒ वातो॑ अ॒पः । \newline
21. अ॒पो अग॑म॒ दग॑म द॒पो॑ ऽपो अग॑मत् । \newline
22. अग॑म॒ दिन्द्र॒ स्येन्द्र॒स्या ग॑म॒ दग॑म॒ दिन्द्र॑स्य । \newline
23. इन्द्र॑स्य त॒नुव॑म् त॒नुव॒ मिन्द्र॒ स्येन्द्र॑स्य त॒नुव᳚म् । \newline
24. त॒नुव॑म् प्रि॒याम् प्रि॒याम् त॒नुव॑म् त॒नुव॑म् प्रि॒याम् । \newline
25. प्रि॒यामिति॑ प्रि॒याम् । \newline
26. ए॒तꣳ स्तो॑तः स्तोत रे॒त मे॒तꣳ स्तो॑तः । \newline
27. स्तो॒त॒ रे॒ते नै॒तेन॑ स्तोतः स्तोत रे॒तेन॑ । \newline
28. ए॒तेन॑ प॒था प॒थै तेनै॒तेन॑ प॒था । \newline
29. प॒था पुनः॒ पुनः॑ प॒था प॒था पुनः॑ । \newline
30. पुन॒ रश्व॒ मश्व॒म् पुनः॒ पुन॒ रश्व᳚म् । \newline
31. अश्व॒ मा ऽश्व॒ मश्व॒ मा । \newline
32. आ व॑र्तयासि वर्तया॒स्या व॑र्तयासि । \newline
33. व॒र्त॒या॒सि॒ नो॒ नो॒ व॒र्त॒या॒सि॒ व॒र्त॒या॒सि॒ नः॒ । \newline
34. न॒ इति॑ नः । \newline
35. लाजी(3)ञ् छाची(3)ञ् छाची(3)न् लाजी(3)न् लाजी(3)ञ् छाची(3)न् । \newline
36. शाची(3)न्. यशो॒ यशः॒ शाची(3)ञ् छाची(3)न्. यशः॑ । \newline
37. यशो॑ म॒मा(4)ॅ म॒मा(4)ॅ यशो॒ यशो॑ म॒मा(4)ॅ । \newline
38. म॒माॅ(4) इति॑ म॒माॅ(4) । \newline
39. य॒व्यायै॑ ग॒व्यायै॑ ग॒व्यायै॑ य॒व्यायै॑ य॒व्यायै॑ ग॒व्यायै᳚ । \newline
40. ग॒व्याया॑ ए॒त दे॒तद् ग॒व्यायै॑ ग॒व्याया॑ ए॒तत् । \newline
41. ए॒तद् दे॑वा देवा ए॒त दे॒तद् दे॑वाः । \newline
42. दे॒वा॒ अन्न॒ मन्न॑म् देवा देवा॒ अन्न᳚म् । \newline
43. अन्न॑ मत्ता॒ त्तान्न॒ मन्न॑ मत्त । \newline
44. अ॒त्तै॒त दे॒त द॑त्तात् तै॒तत् । \newline
45. ए॒त दन्न॒ मन्न॑ मे॒त दे॒त दन्न᳚म् । \newline
46. अन्न॑ मद्ध्य॒ द्ध्यन्न॒ मन्न॑ मद्धि । \newline
47. अ॒द्धि॒ प्र॒जा॒प॒ते॒ प्र॒जा॒प॒ते॒ ऽद्ध्य॒ द्धि॒ प्र॒जा॒प॒ते॒ । \newline
48. प्र॒जा॒प॒त॒ इति॑ प्रजा - प॒ते॒ । \newline
49. यु॒ञ्जन्ति॑ ब्र॒द्ध्नम् ब्र॒द्ध्नं ॅयु॒ञ्जन्ति॑ यु॒ञ्जन्ति॑ ब्र॒द्ध्नम् । \newline
50. ब्र॒द्ध्न म॑रु॒ष म॑रु॒षम् ब्र॒द्ध्नम् ब्र॒द्ध्न म॑रु॒षम् । \newline
51. अ॒रु॒षम् चर॑न्त॒म् चर॑न्त मरु॒ष म॑रु॒षम् चर॑न्तम् । \newline
52. चर॑न्त॒म् परि॒ परि॒ चर॑न्त॒म् चर॑न्त॒म् परि॑ । \newline
53. परि॑ त॒स्थुष॑ स्त॒स्थुषः॒ परि॒ परि॑ त॒स्थुषः॑ । \newline
54. त॒स्थुष॒ इति॑ त॒स्थुषः॑ । \newline
55. रोच॑न्ते रोच॒ना रो॑च॒ना रोच॑न्ते॒ रोच॑न्ते रोच॒ना । \newline
56. रो॒च॒ना दि॒वि दि॒वि रो॑च॒ना रो॑च॒ना दि॒वि । \newline
57. दि॒वीति॑ दि॒वि । \newline
58. यु॒ञ्जन्त्य॑ स्यास्य यु॒ञ्जन्ति॑ यु॒ञ्ज न्त्य॑स्य । \newline
59. अ॒स्य॒ काम्या॒ काम्या᳚ ऽस्यास्य॒ काम्या᳚ । \newline
60. काम्या॒ हरी॒ हरी॒ काम्या॒ काम्या॒ हरी᳚ । \newline
61. हरी॒ विप॑क्षसा॒ विप॑क्षसा॒ हरी॒ हरी॒ विप॑क्षसा । \newline
62. हरी॒ इति॒ हरी᳚ । \newline
63. विप॑क्षसा॒ रथे॒ रथे॒ विप॑क्षसा॒ विप॑क्षसा॒ रथे᳚ । \newline
64. विप॑क्ष॒सेति॒ वि - प॒क्ष॒सा॒ । \newline
65. रथ॒ इति॒ रथे᳚ । \newline
66. शोणा॑ धृ॒ष्णू धृ॒ष्णू शोणा॒ शोणा॑ धृ॒ष्णू । \newline
67. धृ॒ष्णू नृ॒वाह॑सा नृ॒वाह॑सा धृ॒ष्णू धृ॒ष्णू नृ॒वाह॑सा । \newline
68. धृ॒ष्णू इति॑ धृ॒ष्णू । \newline
69. नृ॒वाह॒सेति॑ नृ - वाह॑सा । \newline
70. के॒तुम् कृ॒ण्वन् कृ॒ण्वन् के॒तुम् के॒तुम् कृ॒ण्वन्न् । \newline
71. कृ॒ण्वन् न॑के॒तवे॑ ऽके॒तवे॑ कृ॒ण्वन् कृ॒ण्वन् न॑के॒तवे᳚ । \newline
72. अ॒के॒तवे॒ पेशः॒ पेशो॑ ऽके॒तवे॑ ऽके॒तवे॒ पेशः॑ । \newline
73. पेशो॑ मर्या मर्याः॒ पेशः॒ पेशो॑ मर्याः । \newline
74. म॒र्या॒ अ॒पे॒शसे॑ ऽपे॒शसे॑ मर्या मर्या अपे॒शसे᳚ । \newline
75. अ॒पे॒शस॒ इत्य॑पे॒शसे᳚ । \newline
76. स मु॒षद्भि॑ रु॒षद्भिः॒ सꣳ स मु॒षद्भिः॑ । \newline
77. उ॒षद्भि॑ रजायथा अजायथा उ॒षद्भि॑ रु॒षद्भि॑ रजायथाः । \newline
78. उ॒षद्भि॒रित्यु॒षत् - भिः॒ । \newline
79. अ॒जा॒य॒था॒ इत्य॑जायथाः । \newline

\textbf{Ghana Paata } \newline

1. भूर् भुवो॒ भुवो॒ भूर् भूर् भुवः॒ सुवः॒ सुव॒र् भुवो॒ भूर् भूर् भुवः॒ सुवः॑ । \newline
2. भुवः॒ सुवः॒ सुव॒र् भुवो॒ भुवः॒ सुव॒र् वस॑वो॒ वस॑वः॒ सुव॒र् भुवो॒ भुवः॒ सुव॒र् वस॑वः । \newline
3. सुव॒र् वस॑वो॒ वस॑वः॒ सुवः॒ सुव॒र् वस॑व स्त्वा त्वा॒ वस॑वः॒ सुवः॒ सुव॒र् वस॑व स्त्वा । \newline
4. वस॑व स्त्वा त्वा॒ वस॑वो॒ वस॑व स्त्वा ऽञ्जन् त्वञ्जन्तु त्वा॒ वस॑वो॒ वस॑व स्त्वा ऽञ्जन्तु । \newline
5. त्वा॒ ऽञ्ज॒न् त्व॒ञ्ज॒न्तु॒ त्वा॒ त्वा॒ ऽञ्ज॒न्तु॒ गा॒य॒त्रेण॑ गाय॒त्रेणा᳚ ञ्जन्तु त्वा त्वा ऽञ्जन्तु गाय॒त्रेण॑ । \newline
6. अ॒ञ्ज॒न्तु॒ गा॒य॒त्रेण॑ गाय॒त्रेणा᳚ ञ्जन् त्वञ्जन्तु गाय॒त्रेण॒ छन्द॑सा॒ छन्द॑सा गाय॒त्रेणा᳚ ञ्जन् त्वञ्जन्तु गाय॒त्रेण॒ छन्द॑सा । \newline
7. गा॒य॒त्रेण॒ छन्द॑सा॒ छन्द॑सा गाय॒त्रेण॑ गाय॒त्रेण॒ छन्द॑सा रु॒द्रा रु॒द्रा श्छन्द॑सा गाय॒त्रेण॑ गाय॒त्रेण॒ छन्द॑सा रु॒द्राः । \newline
8. छन्द॑सा रु॒द्रा रु॒द्रा श्छन्द॑सा॒ छन्द॑सा रु॒द्रा स्त्वा᳚ त्वा रु॒द्रा श्छन्द॑सा॒ छन्द॑सा रु॒द्रा स्त्वा᳚ । \newline
9. रु॒द्रा स्त्वा᳚ त्वा रु॒द्रा रु॒द्रा स्त्वा᳚ ऽञ्जन् त्वञ्जन्तु त्वा रु॒द्रा रु॒द्रा स्त्वा᳚ ऽञ्जन्तु । \newline
10. त्वा॒ ऽञ्ज॒न् त्व॒ञ्ज॒न्तु॒ त्वा॒ त्वा॒ ऽञ्ज॒न्तु॒ त्रैष्टु॑भेन॒ त्रैष्टु॑भेना ञ्जन्तु त्वा त्वा ऽञ्जन्तु॒ त्रैष्टु॑भेन । \newline
11. अ॒ञ्ज॒न्तु॒ त्रैष्टु॑भेन॒ त्रैष्टु॑भेना ञ्जन् त्वञ्जन्तु॒ त्रैष्टु॑भेन॒ छन्द॑सा॒ छन्द॑सा॒ त्रैष्टु॑भेना ञ्जन् त्वञ्जन्तु॒ त्रैष्टु॑भेन॒ छन्द॑सा । \newline
12. त्रैष्टु॑भेन॒ छन्द॑सा॒ छन्द॑सा॒ त्रैष्टु॑भेन॒ त्रैष्टु॑भेन॒ छन्द॑सा ऽऽदि॒त्या आ॑दि॒त्या श्छन्द॑सा॒ त्रैष्टु॑भेन॒ त्रैष्टु॑भेन॒ छन्द॑सा ऽऽदि॒त्याः । \newline
13. छन्द॑सा ऽऽदि॒त्या आ॑दि॒त्या श्छन्द॑सा॒ छन्द॑सा ऽऽदि॒त्या स्त्वा᳚ त्वा ऽऽदि॒त्या श्छन्द॑सा॒ छन्द॑सा ऽऽदि॒त्या स्त्वा᳚ । \newline
14. आ॒दि॒त्या स्त्वा᳚ त्वा ऽऽदि॒त्या आ॑दि॒त्या स्त्वा᳚ ऽञ्जन् त्वञ्जन्तु त्वा ऽऽदि॒त्या आ॑दि॒त्या स्त्वा᳚ ऽञ्जन्तु । \newline
15. त्वा॒ ऽञ्ज॒न् त्व॒ञ्ज॒न्तु॒ त्वा॒ त्वा॒ ऽञ्ज॒न्तु॒ जाग॑तेन॒ जाग॑तेना ञ्जन्तु त्वा त्वा ऽञ्जन्तु॒ जाग॑तेन । \newline
16. अ॒ञ्ज॒न्तु॒ जाग॑तेन॒ जाग॑तेना ञ्जन् त्वञ्जन्तु॒ जाग॑तेन॒ छन्द॑सा॒ छन्द॑सा॒ जाग॑तेना ञ्जन्त्व ञ्जन्तु॒ जाग॑तेन॒ छन्द॑सा । \newline
17. जाग॑तेन॒ छन्द॑सा॒ छन्द॑सा॒ जाग॑तेन॒ जाग॑तेन॒ छन्द॑सा॒ यद् यच् छन्द॑सा॒ जाग॑तेन॒ जाग॑तेन॒ छन्द॑सा॒ यत् । \newline
18. छन्द॑सा॒ यद् यच् छन्द॑सा॒ छन्द॑सा॒ यद् वातो॒ वातो॒ यच् छन्द॑सा॒ छन्द॑सा॒ यद् वातः॑ । \newline
19. यद् वातो॒ वातो॒ यद् यद् वातो॑ अ॒पो॑ ऽपो वातो॒ यद् यद् वातो॑ अ॒पः । \newline
20. वातो॑ अ॒पो॑ ऽपो वातो॒ वातो॑ अ॒पो अग॑म॒ दग॑म द॒पो वातो॒ वातो॑ अ॒पो अग॑मत् । \newline
21. अ॒पो अग॑म॒ दग॑म द॒पो॑ ऽपो अग॑म॒ दिन्द्र॒ स्येन्द्र॒स्या ग॑म द॒पो॑ ऽपो अग॑म॒ दिन्द्र॑स्य । \newline
22. अग॑म॒ दिन्द्र॒ स्येन्द्र॒स्या ग॑म॒द ग॑म॒ दिन्द्र॑स्य त॒नुव॑म् त॒नुव॒ मिन्द्र॒स्या ग॑म॒ दग॑म॒ दिन्द्र॑स्य त॒नुव᳚म् । \newline
23. इन्द्र॑स्य त॒नुव॑म् त॒नुव॒ मिन्द्र॒ स्येन्द्र॑स्य त॒नुव॑म् प्रि॒याम् प्रि॒याम् त॒नुव॒ मिन्द्र॒ स्येन्द्र॑स्य त॒नुव॑म् प्रि॒याम् । \newline
24. त॒नुव॑म् प्रि॒याम् प्रि॒याम् त॒नुव॑म् त॒नुव॑म् प्रि॒याम् । \newline
25. प्रि॒यामिति॑ प्रि॒याम् । \newline
26. ए॒तꣳ स्तो॑तः स्तोत रे॒त मे॒तꣳ स्तो॑त रे॒तेनै॒तेन॑ स्तोत रे॒त मे॒तꣳ स्तो॑त रे॒तेन॑ । \newline
27. स्तो॒त॒ रे॒तेनै॒तेन॑ स्तोतः स्तोत रे॒तेन॑ प॒था प॒थैतेन॑ स्तोतः स्तोत रे॒तेन॑ प॒था । \newline
28. ए॒तेन॑ प॒था प॒थै तेनै॒तेन॑ प॒था पुनः॒ पुनः॑ प॒थै तेनै॒तेन॑ प॒था पुनः॑ । \newline
29. प॒था पुनः॒ पुनः॑ प॒था प॒था पुन॒ रश्व॒ मश्व॒म् पुनः॑ प॒था प॒था पुन॒ रश्व᳚म् । \newline
30. पुन॒ रश्व॒ मश्व॒म् पुनः॒ पुन॒ रश्व॒ मा ऽश्व॒म् पुनः॒ पुन॒ रश्व॒ मा । \newline
31. अश्व॒ मा ऽश्व॒ मश्व॒ मा व॑र्तयासि वर्तया॒स्या ऽश्व॒ मश्व॒ मा व॑र्तयासि । \newline
32. आ व॑र्तयासि वर्तया॒स्या व॑र्तयासि नो नो वर्तया॒स्या व॑र्तयासि नः । \newline
33. व॒र्त॒या॒सि॒ नो॒ नो॒ व॒र्त॒या॒सि॒ व॒र्त॒या॒सि॒ नः॒ । \newline
34. न॒ इति॑ नः । \newline
35. लाजी(3)ञ् छाची(3)ञ् छाची(3)न् लाजी(3)न् लाजी(3)ञ् छाची(3)न्. यशो॒ यशः॒ शाची(3)न् लाजी(3)न् लाजी(3)ञ् छाची(3)न्. यशः॑ । \newline
36. शाची(3)न्. यशो॒ यशः॒ शाची(3)ञ् छाची(3)न्. यशो॑ म॒मा(4)ॅ म॒मा(4)ॅ यशः॒ शाची(3)ञ् छाची(3)न्. यशो॑ म॒मा(4)ॅ । \newline
37. यशो॑ म॒मा(4)ॅ म॒मा(4)ॅ यशो॒ यशो॑ म॒मा(4)ॅ । \newline
38. म॒माॅ(4) इति॑ म॒माॅ(4) । \newline
39. य॒व्यायै॑ ग॒व्यायै॑ ग॒व्यायै॑ य॒व्यायै॑ य॒व्यायै॑ ग॒व्याया॑ ए॒त दे॒तद् ग॒व्यायै॑ य॒व्यायै॑ य॒व्यायै॑ ग॒व्याया॑ ए॒तत् । \newline
40. ग॒व्याया॑ ए॒त दे॒तद् ग॒व्यायै॑ ग॒व्याया॑ ए॒तद् दे॑वा देवा ए॒तद् ग॒व्यायै॑ ग॒व्याया॑ ए॒तद् दे॑वाः । \newline
41. ए॒तद् दे॑वा देवा ए॒त दे॒तद् दे॑वा॒ अन्न॒ मन्न॑म् देवा ए॒त दे॒तद् दे॑वा॒ अन्न᳚म् । \newline
42. दे॒वा॒ अन्न॒ मन्न॑म् देवा देवा॒ अन्न॑ मत्ता॒ त्तान्न॑म् देवा देवा॒ अन्न॑ मत्त । \newline
43. अन्न॑ मत्ता॒ त्तान्न॒ मन्न॑ मत्तै॒त दे॒त द॒त्तान्न॒ मन्न॑ मत्तै॒तत् । \newline
44. अ॒त्तै॒त दे॒त द॑त्तात्तै॒त दन्न॒ मन्न॑ मे॒त द॑त्ता त्तै॒त दन्न᳚म् । \newline
45. ए॒त दन्न॒ मन्न॑ मे॒त दे॒त दन्न॑ मद्ध्य॒ द्ध्यन्न॑ मे॒त दे॒त दन्न॑ मद्धि । \newline
46. अन्न॑ मद्ध्य॒ द्ध्यन्न॒ मन्न॑ मद्धि प्रजापते प्रजापते॒ ऽद्ध्यन्न॒ मन्न॑ मद्धि प्रजापते । \newline
47. अ॒द्धि॒ प्र॒जा॒प॒ते॒ प्र॒जा॒प॒ते॒ ऽद्ध्य॒द्धि॒ प्र॒जा॒प॒ते॒ । \newline
48. प्र॒जा॒प॒त॒ इति॑ प्रजा - प॒ते॒ । \newline
49. यु॒ञ्जन्ति॑ ब्र॒द्ध्नम् ब्र॒द्ध्नं ॅयु॒ञ्जन्ति॑ यु॒ञ्जन्ति॑ ब्र॒द्ध्न म॑रु॒ष म॑रु॒षम् ब्र॒द्ध्नं ॅयु॒ञ्जन्ति॑ यु॒ञ्जन्ति॑ ब्र॒द्ध्न म॑रु॒षम् । \newline
50. ब्र॒द्ध्न म॑रु॒ष म॑रु॒षम् ब्र॒द्ध्नम् ब्र॒द्ध्न म॑रु॒षम् चर॑न्त॒म् चर॑न्त मरु॒षम् ब्र॒द्ध्नम् ब्र॒द्ध्न म॑रु॒षम् चर॑न्तम् । \newline
51. अ॒रु॒षम् चर॑न्त॒म् चर॑न्त मरु॒ष म॑रु॒षम् चर॑न्त॒म् परि॒ परि॒ चर॑न्त मरु॒ष म॑रु॒षम् चर॑न्त॒म् परि॑ । \newline
52. चर॑न्त॒म् परि॒ परि॒ चर॑न्त॒म् चर॑न्त॒म् परि॑ त॒स्थुष॑ स्त॒स्थुषः॒ परि॒ चर॑न्त॒म् चर॑न्त॒म् परि॑ त॒स्थुषः॑ । \newline
53. परि॑ त॒स्थुष॑ स्त॒स्थुषः॒ परि॒ परि॑ त॒स्थुषः॑ । \newline
54. त॒स्थुष॒ इति॑ त॒स्थुषः॑ । \newline
55. रोच॑न्ते रोच॒ना रो॑च॒ना रोच॑न्ते॒ रोच॑न्ते रोच॒ना दि॒वि दि॒वि रो॑च॒ना रोच॑न्ते॒ रोच॑न्ते रोच॒ना दि॒वि । \newline
56. रो॒च॒ना दि॒वि दि॒वि रो॑च॒ना रो॑च॒ना दि॒वि । \newline
57. दि॒वीति॑ दि॒वि । \newline
58. यु॒ञ्ज न्त्य॑स्यास्य यु॒ञ्जन्ति॑ यु॒ञ्ज न्त्य॑स्य॒ काम्या॒ काम्या᳚ ऽस्य यु॒ञ्जन्ति॑ यु॒ञ्ज न्त्य॑स्य॒ काम्या᳚ । \newline
59. अ॒स्य॒ काम्या॒ काम्या᳚ ऽस्यास्य॒ काम्या॒ हरी॒ हरी॒ काम्या᳚ ऽस्यास्य॒ काम्या॒ हरी᳚ । \newline
60. काम्या॒ हरी॒ हरी॒ काम्या॒ काम्या॒ हरी॒ विप॑क्षसा॒ विप॑क्षसा॒ हरी॒ काम्या॒ काम्या॒ हरी॒ विप॑क्षसा । \newline
61. हरी॒ विप॑क्षसा॒ विप॑क्षसा॒ हरी॒ हरी॒ विप॑क्षसा॒ रथे॒ रथे॒ विप॑क्षसा॒ हरी॒ हरी॒ विप॑क्षसा॒ रथे᳚ । \newline
62. हरी॒ इति॒ हरी᳚ । \newline
63. विप॑क्षसा॒ रथे॒ रथे॒ विप॑क्षसा॒ विप॑क्षसा॒ रथे᳚ । \newline
64. विप॑क्ष॒सेति॒ वि - प॒क्ष॒सा॒ । \newline
65. रथ॒ इति॒ रथे᳚ । \newline
66. शोणा॑ धृ॒ष्णू धृ॒ष्णू शोणा॒ शोणा॑ धृ॒ष्णू नृ॒वाह॑सा नृ॒वाह॑सा धृ॒ष्णू शोणा॒ शोणा॑ धृ॒ष्णू नृ॒वाह॑सा । \newline
67. धृ॒ष्णू नृ॒वाह॑सा नृ॒वाह॑सा धृ॒ष्णू धृ॒ष्णू नृ॒वाह॑सा । \newline
68. धृ॒ष्णू इति॑ धृ॒ष्णू । \newline
69. नृ॒वाह॒सेति॑ नृ - वाह॑सा । \newline
70. के॒तुम् कृ॒ण्वन् कृ॒ण्वन् के॒तुम् के॒तुम् कृ॒ण्वन् न॑के॒तवे॑ ऽके॒तवे॑ कृ॒ण्वन् के॒तुम् के॒तुम् कृ॒ण्वन् न॑के॒तवे᳚ । \newline
71. कृ॒ण्वन् न॑के॒तवे॑ ऽके॒तवे॑ कृ॒ण्वन् कृ॒ण्वन् न॑के॒तवे॒ पेशः॒ पेशो॑ ऽके॒तवे॑ कृ॒ण्वन् कृ॒ण्वन् न॑के॒तवे॒ पेशः॑ । \newline
72. अ॒के॒तवे॒ पेशः॒ पेशो॑ ऽके॒तवे॑ ऽके॒तवे॒ पेशो॑ मर्या मर्याः॒ पेशो॑ ऽके॒तवे॑ ऽके॒तवे॒ पेशो॑ मर्याः । \newline
73. पेशो॑ मर्या मर्याः॒ पेशः॒ पेशो॑ मर्या अपे॒शसे॑ ऽपे॒शसे॑ मर्याः॒ पेशः॒ पेशो॑ मर्या अपे॒शसे᳚ । \newline
74. म॒र्या॒ अ॒पे॒शसे॑ ऽपे॒शसे॑ मर्या मर्या अपे॒शसे᳚ । \newline
75. अ॒पे॒शस॒ इत्य॑पे॒शसे᳚ । \newline
76. समु॒षद्भि॑ रु॒षद्भिः॒ सꣳ समु॒षद्भि॑ रजायथा अजायथा उ॒षद्भिः॒ सꣳ समु॒षद्भि॑ रजायथाः । \newline
77. उ॒षद्भि॑ रजायथा अजायथा उ॒षद्भि॑ रु॒षद्भि॑ रजायथाः । \newline
78. उ॒षद्भि॒रित्यु॒षत् - भिः॒ । \newline
79. अ॒जा॒य॒था॒ इत्य॑जायथाः । \newline
\pagebreak
\markright{ TS 7.4.21.1  \hfill https://www.vedavms.in \hfill}

\section{ TS 7.4.21.1 }

\textbf{TS 7.4.21.1 } \newline
\textbf{Samhita Paata} \newline

प्रा॒णाय॒ स्वाहा᳚ व्या॒नाय॒ स्वाहा॑ ऽपा॒नाय॒ स्वाहा॒ स्नाव॑भ्यः॒ स्वाहा॑ संता॒नेभ्यः॒ स्वाहा॒ परि॑संतानेभ्यः॒ स्वाहा॒ पर्व॑भ्यः॒ स्वाहा॑ स॒धांने᳚भ्यः॒ स्वाहा॒ शरी॑रेभ्यः॒ स्वाहा॑ य॒ज्ञाय॒ स्वाहा॒ दक्षि॑णाभ्यः॒ स्वाहा॑ सुव॒र्गाय॒ स्वाहा॑ लो॒काय॒ स्वाहा॒ सर्व॑स्मै॒ स्वाहा᳚ ॥ \newline

\textbf{Pada Paata} \newline

प्रा॒णायेति॑ प्र - अ॒नाय॑ । स्वाहा᳚ । व्या॒नायेति॑ वि - अ॒नाय॑ । स्वाहा᳚ । अ॒पा॒नायेत्य॑प - अ॒नाय॑ । स्वाहा᳚ । स्नाव॑भ्य॒ इति॒ स्नाव॑ - भ्यः॒ । स्वाहा᳚ । स॒न्ता॒नेभ्य॒ इति॑ सं - ता॒नेभ्यः॑ । स्वाहा᳚ । परि॑सन्तानेभ्य॒ इति॒ परि॑ - स॒न्ता॒ने॒भ्यः॒ । स्वाहा᳚ । पर्व॑भ्य॒ इति॒ पर्व॑-भ्यः॒ । स्वाहा᳚ । स॒धांने᳚भ्य॒ इति॑ सं - धाने᳚भ्यः । स्वाहा᳚ । शरी॑रेभ्यः । स्वाहा᳚ । य॒ज्ञाय॑ । स्वाहा᳚ । दक्षि॑णाभ्यः । स्वाहा᳚ । सु॒व॒र्गायेति॑ सुवः - गाय॑ । स्वाहा᳚ । लो॒काय॑ । स्वाहा᳚ । सर्व॑स्मै । स्वाहा᳚ ॥  \newline


\textbf{Krama Paata} \newline

प्रा॒णाय॒ स्वाहा᳚ । प्रा॒णायेति॑ प्र - अ॒नाय॑ । स्वाहा᳚ व्या॒नाय॑ । व्या॒नाय॒ स्वाहा᳚ । व्या॒नायेति॑ वि - अ॒नाय॑ । स्वाहा॑ऽपा॒नाय॑ । अ॒पा॒नाय॒ स्वाहा᳚ । अ॒पा॒नायेत्य॑प - अ॒नाय॑ । स्वाहा॒ स्नाव॑भ्यः । स्नाव॑भ्यः॒ स्वाहा᳚ । स्नाव॑भ्य॒ इति॒ स्नाव॑ - भ्यः॒ । स्वाहा॑ सन्ता॒नेभ्यः॑ । स॒न्ता॒नेभ्यः॒ स्वाहा᳚ । स॒न्ता॒नेभ्य॒ इति॑ सम् - ता॒नेभ्यः॑ । स्वाहा॒ परि॑सन्तानेभ्यः । परि॑सन्तानेभ्यः॒ स्वाहा᳚ । परि॑सन्तानेभ्य॒ इति॒ परि॑ - स॒न्ता॒ने॒भ्यः॒ । स्वाहा॒ पर्व॑भ्यः । पर्व॑भ्यः॒ स्वाहा᳚ । पर्व॑भ्य॒ इति॒ पर्व॑ - भ्यः॒ । स्वाहा॑ स॒न्धाने᳚भ्यः । स॒न्धाने᳚भ्यः॒ स्वाहा᳚ । स॒न्धाने᳚भ्य॒ इति॑ सम् - धाने᳚भ्यः । स्वाहा॒ शरी॑रेभ्यः । शरी॑रेभ्यः॒ स्वाहा᳚ । स्वाहा॑ य॒ज्ञाय॑ । य॒ज्ञाय॒ स्वाहा᳚ । स्वाहा॒ दक्षि॑णाभ्यः । दक्षि॑णाभ्यः॒ स्वाहा᳚ । स्वाहा॑ सुव॒र्गाय॑ । सु॒व॒र्गाय॒ स्वाहा᳚ । सु॒व॒र्गायेति॑ सुवः - गाय॑ । स्वाहा॑ लो॒काय॑ । लो॒काय॒ स्वाहा᳚ । स्वाहा॒ सर्व॑स्मै । सर्व॑स्मै॒ स्वाहा᳚ । स्वाहेति॒ स्वाहा᳚ । \newline

\textbf{Jatai Paata} \newline

1. प्रा॒णाय॒ स्वाहा॒ स्वाहा᳚ प्रा॒णाय॑ प्रा॒णाय॒ स्वाहा᳚ । \newline
2. प्रा॒णायेति॑ प्र - अ॒नाय॑ । \newline
3. स्वाहा᳚ व्या॒नाय॑ व्या॒नाय॒ स्वाहा॒ स्वाहा᳚ व्या॒नाय॑ । \newline
4. व्या॒नाय॒ स्वाहा॒ स्वाहा᳚ व्या॒नाय॑ व्या॒नाय॒ स्वाहा᳚ । \newline
5. व्या॒नायेति॑ वि - अ॒नाय॑ । \newline
6. स्वाहा॑ ऽपा॒नाया॑ पा॒नाय॒ स्वाहा॒ स्वाहा॑ ऽपा॒नाय॑ । \newline
7. अ॒पा॒नाय॒ स्वाहा॒ स्वाहा॑ ऽपा॒नाया॑ पा॒नाय॒ स्वाहा᳚ । \newline
8. अ॒पा॒नायेत्य॑प - अ॒नाय॑ । \newline
9. स्वाहा॒ स्नाव॑भ्यः॒ स्नाव॑भ्यः॒ स्वाहा॒ स्वाहा॒ स्नाव॑भ्यः । \newline
10. स्नाव॑भ्यः॒ स्वाहा॒ स्वाहा॒ स्नाव॑भ्यः॒ स्नाव॑भ्यः॒ स्वाहा᳚ । \newline
11. स्नाव॑भ्य॒ इति॒ स्नाव॑ - भ्यः॒ । \newline
12. स्वाहा॑ सन्ता॒नेभ्यः॑ सन्ता॒नेभ्यः॒ स्वाहा॒ स्वाहा॑ सन्ता॒नेभ्यः॑ । \newline
13. स॒न्ता॒नेभ्यः॒ स्वाहा॒ स्वाहा॑ सन्ता॒नेभ्यः॑ सन्ता॒नेभ्यः॒ स्वाहा᳚ । \newline
14. स॒न्ता॒नेभ्य॒ इति॑ सं - ता॒नेभ्यः॑ । \newline
15. स्वाहा॒ परि॑सन्तानेभ्यः॒ परि॑सन्तानेभ्यः॒ स्वाहा॒ स्वाहा॒ परि॑सन्तानेभ्यः । \newline
16. परि॑सन्तानेभ्यः॒ स्वाहा॒ स्वाहा॒ परि॑सन्तानेभ्यः॒ परि॑सन्तानेभ्यः॒ स्वाहा᳚ । \newline
17. परि॑सन्तानेभ्य॒ इति॒ परि॑ - स॒न्ता॒ने॒भ्यः॒ । \newline
18. स्वाहा॒ पर्व॑भ्यः॒ पर्व॑भ्यः॒ स्वाहा॒ स्वाहा॒ पर्व॑भ्यः । \newline
19. पर्व॑भ्यः॒ स्वाहा॒ स्वाहा॒ पर्व॑भ्यः॒ पर्व॑भ्यः॒ स्वाहा᳚ । \newline
20. पर्व॑भ्य॒ इति॒ पर्व॑ - भ्यः॒ । \newline
21. स्वाहा॑ स॒न्धाने᳚भ्यः स॒न्धाने᳚भ्यः॒ स्वाहा॒ स्वाहा॑ स॒न्धाने᳚भ्यः । \newline
22. स॒न्धाने᳚भ्यः॒ स्वाहा॒ स्वाहा॑ स॒न्धाने᳚भ्यः स॒न्धाने᳚भ्यः॒ स्वाहा᳚ । \newline
23. स॒न्धाने᳚भ्य॒ इति॑ सं - धाने᳚भ्यः । \newline
24. स्वाहा॒ शरी॑रेभ्यः॒ शरी॑रेभ्यः॒ स्वाहा॒ स्वाहा॒ शरी॑रेभ्यः । \newline
25. शरी॑रेभ्यः॒ स्वाहा॒ स्वाहा॒ शरी॑रेभ्यः॒ शरी॑रेभ्यः॒ स्वाहा᳚ । \newline
26. स्वाहा॑ य॒ज्ञाय॑ य॒ज्ञाय॒ स्वाहा॒ स्वाहा॑ य॒ज्ञाय॑ । \newline
27. य॒ज्ञाय॒ स्वाहा॒ स्वाहा॑ य॒ज्ञाय॑ य॒ज्ञाय॒ स्वाहा᳚ । \newline
28. स्वाहा॒ दक्षि॑णाभ्यो॒ दक्षि॑णाभ्यः॒ स्वाहा॒ स्वाहा॒ दक्षि॑णाभ्यः । \newline
29. दक्षि॑णाभ्यः॒ स्वाहा॒ स्वाहा॒ दक्षि॑णाभ्यो॒ दक्षि॑णाभ्यः॒ स्वाहा᳚ । \newline
30. स्वाहा॑ सुव॒र्गाय॑ सुव॒र्गाय॒ स्वाहा॒ स्वाहा॑ सुव॒र्गाय॑ । \newline
31. सु॒व॒र्गाय॒ स्वाहा॒ स्वाहा॑ सुव॒र्गाय॑ सुव॒र्गाय॒ स्वाहा᳚ । \newline
32. सु॒व॒र्गायेति॑ सुवः - गाय॑ । \newline
33. स्वाहा॑ लो॒काय॑ लो॒काय॒ स्वाहा॒ स्वाहा॑ लो॒काय॑ । \newline
34. लो॒काय॒ स्वाहा॒ स्वाहा॑ लो॒काय॑ लो॒काय॒ स्वाहा᳚ । \newline
35. स्वाहा॒ सर्व॑स्मै॒ सर्व॑स्मै॒ स्वाहा॒ स्वाहा॒ सर्व॑स्मै । \newline
36. सर्व॑स्मै॒ स्वाहा॒ स्वाहा॒ सर्व॑स्मै॒ सर्व॑स्मै॒ स्वाहा᳚ । \newline
37. स्वाहेति॒ स्वाहा᳚ । \newline

\textbf{Ghana Paata } \newline

1. प्रा॒णाय॒ स्वाहा॒ स्वाहा᳚ प्रा॒णाय॑ प्रा॒णाय॒ स्वाहा᳚ व्या॒नाय॑ व्या॒नाय॒ स्वाहा᳚ प्रा॒णाय॑ प्रा॒णाय॒ स्वाहा᳚ व्या॒नाय॑ । \newline
2. प्रा॒णायेति॑ प्र - अ॒नाय॑ । \newline
3. स्वाहा᳚ व्या॒नाय॑ व्या॒नाय॒ स्वाहा॒ स्वाहा᳚ व्या॒नाय॒ स्वाहा॒ स्वाहा᳚ व्या॒नाय॒ स्वाहा॒ स्वाहा᳚ व्या॒नाय॒ स्वाहा᳚ । \newline
4. व्या॒नाय॒ स्वाहा॒ स्वाहा᳚ व्या॒नाय॑ व्या॒नाय॒ स्वाहा॑ ऽपा॒नाया॑ पा॒नाय॒ स्वाहा᳚ व्या॒नाय॑ व्या॒नाय॒ स्वाहा॑ ऽपा॒नाय॑ । \newline
5. व्या॒नायेति॑ वि - अ॒नाय॑ । \newline
6. स्वाहा॑ ऽपा॒नाया॑ पा॒नाय॒ स्वाहा॒ स्वाहा॑ ऽपा॒नाय॒ स्वाहा॒ स्वाहा॑ ऽपा॒नाय॒ स्वाहा॒ स्वाहा॑ ऽपा॒नाय॒ स्वाहा᳚ । \newline
7. अ॒पा॒नाय॒ स्वाहा॒ स्वाहा॑ ऽपा॒नाया॑ पा॒नाय॒ स्वाहा॒ स्नाव॑भ्यः॒ स्नाव॑भ्यः॒ स्वाहा॑ ऽपा॒नाया॑ पा॒नाय॒ स्वाहा॒ स्नाव॑भ्यः । \newline
8. अ॒पा॒नायेत्य॑प - अ॒नाय॑ । \newline
9. स्वाहा॒ स्नाव॑भ्यः॒ स्नाव॑भ्यः॒ स्वाहा॒ स्वाहा॒ स्नाव॑भ्यः॒ स्वाहा॒ स्वाहा॒ स्नाव॑भ्यः॒ स्वाहा॒ स्वाहा॒ स्नाव॑भ्यः॒ स्वाहा᳚ । \newline
10. स्नाव॑भ्यः॒ स्वाहा॒ स्वाहा॒ स्नाव॑भ्यः॒ स्नाव॑भ्यः॒ स्वाहा॑ सन्ता॒नेभ्यः॑ सन्ता॒नेभ्यः॒ स्वाहा॒ स्नाव॑भ्यः॒ स्नाव॑भ्यः॒ स्वाहा॑ सन्ता॒नेभ्यः॑ । \newline
11. स्नाव॑भ्य॒ इति॒ स्नाव॑ - भ्यः॒ । \newline
12. स्वाहा॑ सन्ता॒नेभ्यः॑ सन्ता॒नेभ्यः॒ स्वाहा॒ स्वाहा॑ सन्ता॒नेभ्यः॒ स्वाहा॒ स्वाहा॑ सन्ता॒नेभ्यः॒ स्वाहा॒ स्वाहा॑ सन्ता॒नेभ्यः॒ स्वाहा᳚ । \newline
13. स॒न्ता॒नेभ्यः॒ स्वाहा॒ स्वाहा॑ सन्ता॒नेभ्यः॑ सन्ता॒नेभ्यः॒ स्वाहा॒ परि॑सन्तानेभ्यः॒ परि॑सन्तानेभ्यः॒ स्वाहा॑ सन्ता॒नेभ्यः॑ सन्ता॒नेभ्यः॒ स्वाहा॒ परि॑सन्तानेभ्यः । \newline
14. स॒न्ता॒नेभ्य॒ इति॑ सं - ता॒नेभ्यः॑ । \newline
15. स्वाहा॒ परि॑सन्तानेभ्यः॒ परि॑सन्तानेभ्यः॒ स्वाहा॒ स्वाहा॒ परि॑सन्तानेभ्यः॒ स्वाहा॒ स्वाहा॒ परि॑सन्तानेभ्यः॒ स्वाहा॒ स्वाहा॒ परि॑सन्तानेभ्यः॒ स्वाहा᳚ । \newline
16. परि॑सन्तानेभ्यः॒ स्वाहा॒ स्वाहा॒ परि॑सन्तानेभ्यः॒ परि॑सन्तानेभ्यः॒ स्वाहा॒ पर्व॑भ्यः॒ पर्व॑भ्यः॒ स्वाहा॒ परि॑सन्तानेभ्यः॒ परि॑सन्तानेभ्यः॒ स्वाहा॒ पर्व॑भ्यः । \newline
17. परि॑सन्तानेभ्य॒ इति॒ परि॑ - स॒न्ता॒ने॒भ्यः॒ । \newline
18. स्वाहा॒ पर्व॑भ्यः॒ पर्व॑भ्यः॒ स्वाहा॒ स्वाहा॒ पर्व॑भ्यः॒ स्वाहा॒ स्वाहा॒ पर्व॑भ्यः॒ स्वाहा॒ स्वाहा॒ पर्व॑भ्यः॒ स्वाहा᳚ । \newline
19. पर्व॑भ्यः॒ स्वाहा॒ स्वाहा॒ पर्व॑भ्यः॒ पर्व॑भ्यः॒ स्वाहा॑ स॒न्धाने᳚भ्यः स॒न्धाने᳚भ्यः॒ स्वाहा॒ पर्व॑भ्यः॒ पर्व॑भ्यः॒ स्वाहा॑ स॒न्धाने᳚भ्यः । \newline
20. पर्व॑भ्य॒ इति॒ पर्व॑ - भ्यः॒ । \newline
21. स्वाहा॑ स॒न्धाने᳚भ्यः स॒न्धाने᳚भ्यः॒ स्वाहा॒ स्वाहा॑ स॒न्धाने᳚भ्यः॒ स्वाहा॒ स्वाहा॑ स॒न्धाने᳚भ्यः॒ स्वाहा॒ स्वाहा॑ स॒न्धाने᳚भ्यः॒ स्वाहा᳚ । \newline
22. स॒न्धाने᳚भ्यः॒ स्वाहा॒ स्वाहा॑ स॒न्धाने᳚भ्यः स॒न्धाने᳚भ्यः॒ स्वाहा॒ शरी॑रेभ्यः॒ शरी॑रेभ्यः॒ स्वाहा॑ स॒न्धाने᳚भ्यः स॒न्धाने᳚भ्यः॒ स्वाहा॒ शरी॑रेभ्यः । \newline
23. स॒न्धाने᳚भ्य॒ इति॑ सं - धाने᳚भ्यः । \newline
24. स्वाहा॒ शरी॑रेभ्यः॒ शरी॑रेभ्यः॒ स्वाहा॒ स्वाहा॒ शरी॑रेभ्यः॒ स्वाहा॒ स्वाहा॒ शरी॑रेभ्यः॒ स्वाहा॒ स्वाहा॒ शरी॑रेभ्यः॒ स्वाहा᳚ । \newline
25. शरी॑रेभ्यः॒ स्वाहा॒ स्वाहा॒ शरी॑रेभ्यः॒ शरी॑रेभ्यः॒ स्वाहा॑ य॒ज्ञाय॑ य॒ज्ञाय॒ स्वाहा॒ शरी॑रेभ्यः॒ शरी॑रेभ्यः॒ स्वाहा॑ य॒ज्ञाय॑ । \newline
26. स्वाहा॑ य॒ज्ञाय॑ य॒ज्ञाय॒ स्वाहा॒ स्वाहा॑ य॒ज्ञाय॒ स्वाहा॒ स्वाहा॑ य॒ज्ञाय॒ स्वाहा॒ स्वाहा॑ य॒ज्ञाय॒ स्वाहा᳚ । \newline
27. य॒ज्ञाय॒ स्वाहा॒ स्वाहा॑ य॒ज्ञाय॑ य॒ज्ञाय॒ स्वाहा॒ दक्षि॑णाभ्यो॒ दक्षि॑णाभ्यः॒ स्वाहा॑ य॒ज्ञाय॑ य॒ज्ञाय॒ स्वाहा॒ दक्षि॑णाभ्यः । \newline
28. स्वाहा॒ दक्षि॑णाभ्यो॒ दक्षि॑णाभ्यः॒ स्वाहा॒ स्वाहा॒ दक्षि॑णाभ्यः॒ स्वाहा॒ स्वाहा॒ दक्षि॑णाभ्यः॒ स्वाहा॒ स्वाहा॒ दक्षि॑णाभ्यः॒ स्वाहा᳚ । \newline
29. दक्षि॑णाभ्यः॒ स्वाहा॒ स्वाहा॒ दक्षि॑णाभ्यो॒ दक्षि॑णाभ्यः॒ स्वाहा॑ सुव॒र्गाय॑ सुव॒र्गाय॒ स्वाहा॒ दक्षि॑णाभ्यो॒ दक्षि॑णाभ्यः॒ स्वाहा॑ सुव॒र्गाय॑ । \newline
30. स्वाहा॑ सुव॒र्गाय॑ सुव॒र्गाय॒ स्वाहा॒ स्वाहा॑ सुव॒र्गाय॒ स्वाहा॒ स्वाहा॑ सुव॒र्गाय॒ स्वाहा॒ स्वाहा॑ सुव॒र्गाय॒ स्वाहा᳚ । \newline
31. सु॒व॒र्गाय॒ स्वाहा॒ स्वाहा॑ सुव॒र्गाय॑ सुव॒र्गाय॒ स्वाहा॑ लो॒काय॑ लो॒काय॒ स्वाहा॑ सुव॒र्गाय॑ सुव॒र्गाय॒ स्वाहा॑ लो॒काय॑ । \newline
32. सु॒व॒र्गायेति॑ सुवः - गाय॑ । \newline
33. स्वाहा॑ लो॒काय॑ लो॒काय॒ स्वाहा॒ स्वाहा॑ लो॒काय॒ स्वाहा॒ स्वाहा॑ लो॒काय॒ स्वाहा॒ स्वाहा॑ लो॒काय॒ स्वाहा᳚ । \newline
34. लो॒काय॒ स्वाहा॒ स्वाहा॑ लो॒काय॑ लो॒काय॒ स्वाहा॒ सर्व॑स्मै॒ सर्व॑स्मै॒ स्वाहा॑ लो॒काय॑ लो॒काय॒ स्वाहा॒ सर्व॑स्मै । \newline
35. स्वाहा॒ सर्व॑स्मै॒ सर्व॑स्मै॒ स्वाहा॒ स्वाहा॒ सर्व॑स्मै॒ स्वाहा॒ स्वाहा॒ सर्व॑स्मै॒ स्वाहा॒ स्वाहा॒ सर्व॑स्मै॒ स्वाहा᳚ । \newline
36. सर्व॑स्मै॒ स्वाहा॒ स्वाहा॒ सर्व॑स्मै॒ सर्व॑स्मै॒ स्वाहा᳚ । \newline
37. स्वाहेति॒ स्वाहा᳚ । \newline
\pagebreak
\markright{ TS 7.4.22.1  \hfill https://www.vedavms.in \hfill}

\section{ TS 7.4.22.1 }

\textbf{TS 7.4.22.1 } \newline
\textbf{Samhita Paata} \newline

सि॒ताय॒ स्वाहा ऽसि॑ताय॒ स्वाहा॒ ऽभिहि॑ताय॒ स्वाहा ऽन॑भिहिताय॒ स्वाहा॑ यु॒क्ताय॒ स्वाहा ऽयु॑क्ताय॒ स्वाहा॒ सुयु॑क्ताय॒ स्वाहो -द्यु॑क्ताय॒ स्वाहा॒ विमु॑क्ताय॒ स्वाहा॒ प्रमु॑क्ताय॒ स्वाहा॒ वञ्च॑ते॒ स्वाहा॑ परि॒वञ्च॑ते॒ स्वाहा॑ सं॒ॅवञ्च॑ते॒ स्वाहा॑ ऽनु॒वञ्च॑ते॒ स्वाहो॒द् -वञ्च॑ते॒ स्वाहा॑ य॒ते स्वाहा॒ धाव॑ते॒ स्वाहा॒ तिष्ठ॑ते॒ स्वाहा॒ सर्व॑स्मै॒ स्वाहा᳚ ॥ \newline

\textbf{Pada Paata} \newline

सि॒ताय॑ । स्वाहा᳚ । असि॑ताय । स्वाहा᳚ । अ॒भिहि॑ता॒येत्य॒भि - हि॒ता॒य॒ । स्वाहा᳚ । अन॑भिहिता॒येत्यन॑भि - हि॒ता॒य॒ । स्वाहा᳚ । यु॒क्ताय॑ । स्वाहा᳚ । अयु॑क्ताय । स्वाहा᳚ । सुयु॑क्ता॒येति॒ सु - यु॒क्ता॒य॒ । स्वाहा᳚ । उद्यु॑क्ता॒येत्युत् - यु॒क्ता॒य॒ । स्वाहा᳚ । विमु॑क्ता॒येति॒ वि - मु॒क्ता॒य॒ । स्वाहा᳚ । प्रमु॑क्ता॒येति॒ प्र - मु॒क्ता॒य॒ । स्वाहा᳚ । वञ्च॑ते । स्वाहा᳚ । प॒रि॒वञ्च॑त॒ इति॑ परि - वञ्च॑ते । स्वाहा᳚ । सं॒ॅवञ्च॑त॒ इति॑ सं - वञ्च॑ते । स्वाहा᳚ । अ॒नु॒वञ्च॑त॒ इत्य॑नु - वञ्च॑ते । स्वाहा᳚ । उ॒द्वञ्च॑त॒ इत्यु॑त्-वञ्च॑ते । स्वाहा᳚ । य॒ते । स्वाहा᳚ । धाव॑ते । स्वाहा᳚ । तिष्ठ॑ते । स्वाहा᳚ । सर्व॑स्मै । स्वाहा᳚ ॥  \newline


\textbf{Krama Paata} \newline

सि॒ताय॒ स्वाहा᳚ । स्वाहाऽसि॑ताय । असि॑ताय॒ स्वाहा᳚ । स्वाहा॒ऽभिहि॑ताय । अ॒भिहि॑ताय॒ स्वाहा᳚ । अ॒भिहि॑ता॒येत्य॒भि - हि॒ता॒य॒ । स्वाहाऽन॑भिहिताय । अन॑भिहिताय॒ स्वाहा᳚ । अन॑भिहिता॒येत्य॑नभि - हि॒ता॒य॒ । स्वाहा॑ यु॒क्ताय॑ । यु॒क्ताय॒ स्वाहा᳚ । स्वाहाऽयु॑क्ताय । अयु॑क्ताय॒ स्वाहा᳚ । स्वाहा॒ सुयु॑क्ताय । सुयु॑क्ताय॒ स्वाहा᳚ । सुयु॑क्ता॒येति॒ सु - यु॒क्ता॒य॒ । स्वाहोद्यु॑क्ताय । उद्यु॑क्ताय॒ स्वाहा᳚ । उद्यु॑क्ता॒येत्युत् - यु॒क्ता॒य॒ । स्वाहा॒ विमु॑क्ताय । विमु॑क्ताय॒ स्वाहा᳚ । विमु॑क्ता॒येति॒ वि - मु॒क्ता॒य॒ । स्वाहा॒ प्रमु॑क्ताय । प्रमु॑क्ताय॒ स्वाहा᳚ । प्रमु॑क्ता॒येति॒ प्र - मु॒क्ता॒य॒ । स्वाहा॒ वञ्च॑ते । वञ्च॑ते॒ स्वाहा᳚ । स्वाहा॑ परि॒वञ्च॑ते । प॒रि॒वञ्च॑ते॒ स्वाहा᳚ । प॒रि॒वञ्च॑त॒ इति॑ परि - वञ्च॑ते । स्वाहा॑ स॒म्ॅवञ्च॑ते । स॒म्ॅवञ्च॑ते॒ स्वाहा᳚ । स॒म्ॅवञ्च॑त॒ इति॑ सम् - वञ्च॑ते । स्वाहा॑ऽनु॒वञ्च॑ते । अ॒नु॒वञ्च॑ते॒ स्वाहा᳚ । अ॒नु॒वञ्च॑त॒ इत्य॑नु - वञ्च॑ते । स्वाहो॒द्वञ्च॑ते । उ॒द्वञ्च॑ते॒ स्वाहा᳚ । उ॒द्वञ्च॑त॒ इत्यु॑त् - वञ्च॑ते । स्वाहा॑ य॒ते । य॒ते स्वाहा᳚ । स्वाहा॒ धाव॑ते । धाव॑ते॒ स्वाहा᳚ । स्वाहा॒ तिष्ठ॑ते । तिष्ठ॑ते॒ स्वाहा᳚ । स्वाहा॒ सर्व॑स्मै । सर्व॑स्मै॒ स्वाहा᳚ । स्वाहेति॒ स्वाहा᳚ । \newline

\textbf{Jatai Paata} \newline

1. सि॒ताय॒ स्वाहा॒ स्वाहा॑ सि॒ताय॑ सि॒ताय॒ स्वाहा᳚ । \newline
2. स्वाहा ऽसि॑ता॒या सि॑ताय॒ स्वाहा॒ स्वाहा ऽसि॑ताय । \newline
3. असि॑ताय॒ स्वाहा॒ स्वाहा ऽसि॑ता॒या सि॑ताय॒ स्वाहा᳚ । \newline
4. स्वाहा॒ ऽभिहि॑ताया॒ भिहि॑ताय॒ स्वाहा॒ स्वाहा॒ ऽभिहि॑ताय । \newline
5. अ॒भिहि॑ताय॒ स्वाहा॒ स्वाहा॒ ऽभिहि॑ताया॒ भिहि॑ताय॒ स्वाहा᳚ । \newline
6. अ॒भिहि॑ता॒येत्य॒भि - हि॒ता॒य॒ । \newline
7. स्वाहा ऽन॑भिहिता॒या न॑भिहिताय॒ स्वाहा॒ स्वाहा ऽन॑भिहिताय । \newline
8. अन॑भिहिताय॒ स्वाहा॒ स्वाहा ऽन॑भिहिता॒या न॑भिहिताय॒ स्वाहा᳚ । \newline
9. अन॑भिहिता॒येत्यन॑भि - हि॒ता॒य॒ । \newline
10. स्वाहा॑ यु॒क्ताय॑ यु॒क्ताय॒ स्वाहा॒ स्वाहा॑ यु॒क्ताय॑ । \newline
11. यु॒क्ताय॒ स्वाहा॒ स्वाहा॑ यु॒क्ताय॑ यु॒क्ताय॒ स्वाहा᳚ । \newline
12. स्वाहा ऽयु॑क्ता॒या यु॑क्ताय॒ स्वाहा॒ स्वाहा ऽयु॑क्ताय । \newline
13. अयु॑क्ताय॒ स्वाहा॒ स्वाहा ऽयु॑क्ता॒या यु॑क्ताय॒ स्वाहा᳚ । \newline
14. स्वाहा॒ सुयु॑क्ताय॒ सुयु॑क्ताय॒ स्वाहा॒ स्वाहा॒ सुयु॑क्ताय । \newline
15. सुयु॑क्ताय॒ स्वाहा॒ स्वाहा॒ सुयु॑क्ताय॒ सुयु॑क्ताय॒ स्वाहा᳚ । \newline
16. सुयु॑क्ता॒येति॒ सु - यु॒क्ता॒य॒ । \newline
17. स्वाहोद्यु॑क्ता॒ योद्यु॑क्ताय॒ स्वाहा॒ स्वाहोद्यु॑क्ताय । \newline
18. उद्यु॑क्ताय॒ स्वाहा॒ स्वाहोद्यु॑क्ता॒ योद्यु॑क्ताय॒ स्वाहा᳚ । \newline
19. उद्यु॑क्ता॒येत्युत् - यु॒क्ता॒य॒ । \newline
20. स्वाहा॒ विमु॑क्ताय॒ विमु॑क्ताय॒ स्वाहा॒ स्वाहा॒ विमु॑क्ताय । \newline
21. विमु॑क्ताय॒ स्वाहा॒ स्वाहा॒ विमु॑क्ताय॒ विमु॑क्ताय॒ स्वाहा᳚ । \newline
22. विमु॑क्ता॒येति॒ वि - मु॒क्ता॒य॒ । \newline
23. स्वाहा॒ प्रमु॑क्ताय॒ प्रमु॑क्ताय॒ स्वाहा॒ स्वाहा॒ प्रमु॑क्ताय । \newline
24. प्रमु॑क्ताय॒ स्वाहा॒ स्वाहा॒ प्रमु॑क्ताय॒ प्रमु॑क्ताय॒ स्वाहा᳚ । \newline
25. प्रमु॑क्ता॒येति॒ प्र - मु॒क्ता॒य॒ । \newline
26. स्वाहा॒ वञ्च॑ते॒ वञ्च॑ते॒ स्वाहा॒ स्वाहा॒ वञ्च॑ते । \newline
27. वञ्च॑ते॒ स्वाहा॒ स्वाहा॒ वञ्च॑ते॒ वञ्च॑ते॒ स्वाहा᳚ । \newline
28. स्वाहा॑ परि॒वञ्च॑ते परि॒वञ्च॑ते॒ स्वाहा॒ स्वाहा॑ परि॒वञ्च॑ते । \newline
29. प॒रि॒वञ्च॑ते॒ स्वाहा॒ स्वाहा॑ परि॒वञ्च॑ते परि॒वञ्च॑ते॒ स्वाहा᳚ । \newline
30. प॒रि॒वञ्च॑त॒ इति॑ परि - वञ्च॑ते । \newline
31. स्वाहा॑ सं॒ॅवञ्च॑ते सं॒ॅवञ्च॑ते॒ स्वाहा॒ स्वाहा॑ सं॒ॅवञ्च॑ते । \newline
32. सं॒ॅवञ्च॑ते॒ स्वाहा॒ स्वाहा॑ सं॒ॅवञ्च॑ते सं॒ॅवञ्च॑ते॒ स्वाहा᳚ । \newline
33. सं॒ॅवञ्च॑त॒ इति॑ सं - वञ्च॑ते । \newline
34. स्वाहा॑ ऽनु॒वञ्च॑ते ऽनु॒वञ्च॑ते॒ स्वाहा॒ स्वाहा॑ ऽनु॒वञ्च॑ते । \newline
35. अ॒नु॒वञ्च॑ते॒ स्वाहा॒ स्वाहा॑ ऽनु॒वञ्च॑ते ऽनु॒वञ्च॑ते॒ स्वाहा᳚ । \newline
36. अ॒नु॒वञ्च॑त॒ इत्य॑नु - वञ्च॑ते । \newline
37. स्वाहो॒द्वञ्च॑त उ॒द्वञ्च॑ते॒ स्वाहा॒ स्वाहो॒द्वञ्च॑ते । \newline
38. उ॒द्वञ्च॑ते॒ स्वाहा॒ स्वाहो॒द्वञ्च॑त उ॒द्वञ्च॑ते॒ स्वाहा᳚ । \newline
39. उ॒द्वञ्च॑त॒ इत्यु॑त् - वञ्च॑ते । \newline
40. स्वाहा॑ य॒ते य॒ते स्वाहा॒ स्वाहा॑ य॒ते । \newline
41. य॒ते स्वाहा॒ स्वाहा॑ य॒ते य॒ते स्वाहा᳚ । \newline
42. स्वाहा॒ धाव॑ते॒ धाव॑ते॒ स्वाहा॒ स्वाहा॒ धाव॑ते । \newline
43. धाव॑ते॒ स्वाहा॒ स्वाहा॒ धाव॑ते॒ धाव॑ते॒ स्वाहा᳚ । \newline
44. स्वाहा॒ तिष्ठ॑ते॒ तिष्ठ॑ते॒ स्वाहा॒ स्वाहा॒ तिष्ठ॑ते । \newline
45. तिष्ठ॑ते॒ स्वाहा॒ स्वाहा॒ तिष्ठ॑ते॒ तिष्ठ॑ते॒ स्वाहा᳚ । \newline
46. स्वाहा॒ सर्व॑स्मै॒ सर्व॑स्मै॒ स्वाहा॒ स्वाहा॒ सर्व॑स्मै । \newline
47. सर्व॑स्मै॒ स्वाहा॒ स्वाहा॒ सर्व॑स्मै॒ सर्व॑स्मै॒ स्वाहा᳚ । \newline
48. स्वाहेति॒ स्वाहा᳚ । \newline

\textbf{Ghana Paata } \newline

1. सि॒ताय॒ स्वाहा॒ स्वाहा॑ सि॒ताय॑ सि॒ताय॒ स्वाहा ऽसि॑ता॒या सि॑ताय॒ स्वाहा॑ सि॒ताय॑ सि॒ताय॒ स्वाहा ऽसि॑ताय । \newline
2. स्वाहा ऽसि॑ता॒या सि॑ताय॒ स्वाहा॒ स्वाहा ऽसि॑ताय॒ स्वाहा॒ स्वाहा ऽसि॑ताय॒ स्वाहा॒ स्वाहा ऽसि॑ताय॒ स्वाहा᳚ । \newline
3. असि॑ताय॒ स्वाहा॒ स्वाहा ऽसि॑ता॒या सि॑ताय॒ स्वाहा॒ ऽभिहि॑ताया॒ भिहि॑ताय॒ स्वाहा ऽसि॑ता॒या सि॑ताय॒ स्वाहा॒ ऽभिहि॑ताय । \newline
4. स्वाहा॒ ऽभिहि॑ताया॒ भिहि॑ताय॒ स्वाहा॒ स्वाहा॒ ऽभिहि॑ताय॒ स्वाहा॒ स्वाहा॒ ऽभिहि॑ताय॒ स्वाहा॒ स्वाहा॒ ऽभिहि॑ताय॒ स्वाहा᳚ । \newline
5. अ॒भिहि॑ताय॒ स्वाहा॒ स्वाहा॒ ऽभिहि॑ताया॒ भिहि॑ताय॒ स्वाहा ऽन॑भिहिता॒या न॑भिहिताय॒ स्वाहा॒ ऽभिहि॑ताया॒ भिहि॑ताय॒ स्वाहा ऽन॑भिहिताय । \newline
6. अ॒भिहि॑ता॒येत्य॒भि - हि॒ता॒य॒ । \newline
7. स्वाहा ऽन॑भिहिता॒या न॑भिहिताय॒ स्वाहा॒ स्वाहा ऽन॑भिहिताय॒ स्वाहा॒ स्वाहा ऽन॑भिहिताय॒ स्वाहा॒ स्वाहा ऽन॑भिहिताय॒ स्वाहा᳚ । \newline
8. अन॑भिहिताय॒ स्वाहा॒ स्वाहा ऽन॑भिहिता॒या न॑भिहिताय॒ स्वाहा॑ यु॒क्ताय॑ यु॒क्ताय॒ स्वाहा ऽन॑भिहिता॒या न॑भिहिताय॒ स्वाहा॑ यु॒क्ताय॑ । \newline
9. अन॑भिहिता॒येत्यन॑भि - हि॒ता॒य॒ । \newline
10. स्वाहा॑ यु॒क्ताय॑ यु॒क्ताय॒ स्वाहा॒ स्वाहा॑ यु॒क्ताय॒ स्वाहा॒ स्वाहा॑ यु॒क्ताय॒ स्वाहा॒ स्वाहा॑ यु॒क्ताय॒ स्वाहा᳚ । \newline
11. यु॒क्ताय॒ स्वाहा॒ स्वाहा॑ यु॒क्ताय॑ यु॒क्ताय॒ स्वाहा ऽयु॑क्ता॒या यु॑क्ताय॒ स्वाहा॑ यु॒क्ताय॑ यु॒क्ताय॒ स्वाहा ऽयु॑क्ताय । \newline
12. स्वाहा ऽयु॑क्ता॒या यु॑क्ताय॒ स्वाहा॒ स्वाहा ऽयु॑क्ताय॒ स्वाहा॒ स्वाहा ऽयु॑क्ताय॒ स्वाहा॒ स्वाहा ऽयु॑क्ताय॒ स्वाहा᳚ । \newline
13. अयु॑क्ताय॒ स्वाहा॒ स्वाहा ऽयु॑क्ता॒या यु॑क्ताय॒ स्वाहा॒ सुयु॑क्ताय॒ सुयु॑क्ताय॒ स्वाहा ऽयु॑क्ता॒या यु॑क्ताय॒ स्वाहा॒ सुयु॑क्ताय । \newline
14. स्वाहा॒ सुयु॑क्ताय॒ सुयु॑क्ताय॒ स्वाहा॒ स्वाहा॒ सुयु॑क्ताय॒ स्वाहा॒ स्वाहा॒ सुयु॑क्ताय॒ स्वाहा॒ स्वाहा॒ सुयु॑क्ताय॒ स्वाहा᳚ । \newline
15. सुयु॑क्ताय॒ स्वाहा॒ स्वाहा॒ सुयु॑क्ताय॒ सुयु॑क्ताय॒ स्वाहो द्यु॑क्ता॒यो द्यु॑क्ताय॒ स्वाहा॒ सुयु॑क्ताय॒ सुयु॑क्ताय॒ स्वाहो द्यु॑क्ताय । \newline
16. सुयु॑क्ता॒येति॒ सु - यु॒क्ता॒य॒ । \newline
17. स्वाहो द्यु॑क्ता॒यो द्यु॑क्ताय॒ स्वाहा॒ स्वाहो द्यु॑क्ताय॒ स्वाहा॒ स्वाहो द्यु॑क्ताय॒ स्वाहा॒ स्वाहो द्यु॑क्ताय॒ स्वाहा᳚ । \newline
18. उद्यु॑क्ताय॒ स्वाहा॒ स्वाहो द्यु॑क्ता॒यो द्यु॑क्ताय॒ स्वाहा॒ विमु॑क्ताय॒ विमु॑क्ताय॒ स्वाहो द्यु॑क्ता॒यो द्यु॑क्ताय॒ स्वाहा॒ विमु॑क्ताय । \newline
19. उद्यु॑क्ता॒येत्युत् - यु॒क्ता॒य॒ । \newline
20. स्वाहा॒ विमु॑क्ताय॒ विमु॑क्ताय॒ स्वाहा॒ स्वाहा॒ विमु॑क्ताय॒ स्वाहा॒ स्वाहा॒ विमु॑क्ताय॒ स्वाहा॒ स्वाहा॒ विमु॑क्ताय॒ स्वाहा᳚ । \newline
21. विमु॑क्ताय॒ स्वाहा॒ स्वाहा॒ विमु॑क्ताय॒ विमु॑क्ताय॒ स्वाहा॒ प्रमु॑क्ताय॒ प्रमु॑क्ताय॒ स्वाहा॒ विमु॑क्ताय॒ विमु॑क्ताय॒ स्वाहा॒ प्रमु॑क्ताय । \newline
22. विमु॑क्ता॒येति॒ वि - मु॒क्ता॒य॒ । \newline
23. स्वाहा॒ प्रमु॑क्ताय॒ प्रमु॑क्ताय॒ स्वाहा॒ स्वाहा॒ प्रमु॑क्ताय॒ स्वाहा॒ स्वाहा॒ प्रमु॑क्ताय॒ स्वाहा॒ स्वाहा॒ प्रमु॑क्ताय॒ स्वाहा᳚ । \newline
24. प्रमु॑क्ताय॒ स्वाहा॒ स्वाहा॒ प्रमु॑क्ताय॒ प्रमु॑क्ताय॒ स्वाहा॒ वञ्च॑ते॒ वञ्च॑ते॒ स्वाहा॒ प्रमु॑क्ताय॒ प्रमु॑क्ताय॒ स्वाहा॒ वञ्च॑ते । \newline
25. प्रमु॑क्ता॒येति॒ प्र - मु॒क्ता॒य॒ । \newline
26. स्वाहा॒ वञ्च॑ते॒ वञ्च॑ते॒ स्वाहा॒ स्वाहा॒ वञ्च॑ते॒ स्वाहा॒ स्वाहा॒ वञ्च॑ते॒ स्वाहा॒ स्वाहा॒ वञ्च॑ते॒ स्वाहा᳚ । \newline
27. वञ्च॑ते॒ स्वाहा॒ स्वाहा॒ वञ्च॑ते॒ वञ्च॑ते॒ स्वाहा॑ परि॒वञ्च॑ते परि॒वञ्च॑ते॒ स्वाहा॒ वञ्च॑ते॒ वञ्च॑ते॒ स्वाहा॑ परि॒वञ्च॑ते । \newline
28. स्वाहा॑ परि॒वञ्च॑ते परि॒वञ्च॑ते॒ स्वाहा॒ स्वाहा॑ परि॒वञ्च॑ते॒ स्वाहा॒ स्वाहा॑ परि॒वञ्च॑ते॒ स्वाहा॒ स्वाहा॑ परि॒वञ्च॑ते॒ स्वाहा᳚ । \newline
29. प॒रि॒वञ्च॑ते॒ स्वाहा॒ स्वाहा॑ परि॒वञ्च॑ते परि॒वञ्च॑ते॒ स्वाहा॑ सं॒ॅवञ्च॑ते सं॒ॅवञ्च॑ते॒ स्वाहा॑ परि॒वञ्च॑ते परि॒वञ्च॑ते॒ स्वाहा॑ सं॒ॅवञ्च॑ते । \newline
30. प॒रि॒वञ्च॑त॒ इति॑ परि - वञ्च॑ते । \newline
31. स्वाहा॑ सं॒ॅवञ्च॑ते सं॒ॅवञ्च॑ते॒ स्वाहा॒ स्वाहा॑ सं॒ॅवञ्च॑ते॒ स्वाहा॒ स्वाहा॑ सं॒ॅवञ्च॑ते॒ स्वाहा॒ स्वाहा॑ सं॒ॅवञ्च॑ते॒ स्वाहा᳚ । \newline
32. सं॒ॅवञ्च॑ते॒ स्वाहा॒ स्वाहा॑ सं॒ॅवञ्च॑ते सं॒ॅवञ्च॑ते॒ स्वाहा॑ ऽनु॒वञ्च॑ते ऽनु॒वञ्च॑ते॒ स्वाहा॑ सं॒ॅवञ्च॑ते सं॒ॅवञ्च॑ते॒ स्वाहा॑ ऽनु॒वञ्च॑ते । \newline
33. सं॒ॅवञ्च॑त॒ इति॑ सं - वञ्च॑ते । \newline
34. स्वाहा॑ ऽनु॒वञ्च॑ते ऽनु॒वञ्च॑ते॒ स्वाहा॒ स्वाहा॑ ऽनु॒वञ्च॑ते॒ स्वाहा॒ स्वाहा॑ ऽनु॒वञ्च॑ते॒ स्वाहा॒ स्वाहा॑ ऽनु॒वञ्च॑ते॒ स्वाहा᳚ । \newline
35. अ॒नु॒वञ्च॑ते॒ स्वाहा॒ स्वाहा॑ ऽनु॒वञ्च॑ते ऽनु॒वञ्च॑ते॒ स्वाहो॒द्वञ्च॑त उ॒द्वञ्च॑ते॒ स्वाहा॑ ऽनु॒वञ्च॑ते ऽनु॒वञ्च॑ते॒ स्वाहो॒द्वञ्च॑ते । \newline
36. अ॒नु॒वञ्च॑त॒ इत्य॑नु - वञ्च॑ते । \newline
37. स्वाहो॒द्वञ्च॑त उ॒द्वञ्च॑ते॒ स्वाहा॒ स्वाहो॒द्वञ्च॑ते॒ स्वाहा॒ स्वाहो॒द्वञ्च॑ते॒ स्वाहा॒ स्वाहो॒द्वञ्च॑ते॒ स्वाहा᳚ । \newline
38. उ॒द्वञ्च॑ते॒ स्वाहा॒ स्वाहो॒द्वञ्च॑त उ॒द्वञ्च॑ते॒ स्वाहा॑ य॒ते य॒ते स्वाहो॒द्वञ्च॑त उ॒द्वञ्च॑ते॒ स्वाहा॑ य॒ते । \newline
39. उ॒द्वञ्च॑त॒ इत्यु॑त् - वञ्च॑ते । \newline
40. स्वाहा॑ य॒ते य॒ते स्वाहा॒ स्वाहा॑ य॒ते स्वाहा॒ स्वाहा॑ य॒ते स्वाहा॒ स्वाहा॑ य॒ते स्वाहा᳚ । \newline
41. य॒ते स्वाहा॒ स्वाहा॑ य॒ते य॒ते स्वाहा॒ धाव॑ते॒ धाव॑ते॒ स्वाहा॑ य॒ते य॒ते स्वाहा॒ धाव॑ते । \newline
42. स्वाहा॒ धाव॑ते॒ धाव॑ते॒ स्वाहा॒ स्वाहा॒ धाव॑ते॒ स्वाहा॒ स्वाहा॒ धाव॑ते॒ स्वाहा॒ स्वाहा॒ धाव॑ते॒ स्वाहा᳚ । \newline
43. धाव॑ते॒ स्वाहा॒ स्वाहा॒ धाव॑ते॒ धाव॑ते॒ स्वाहा॒ तिष्ठ॑ते॒ तिष्ठ॑ते॒ स्वाहा॒ धाव॑ते॒ धाव॑ते॒ स्वाहा॒ तिष्ठ॑ते । \newline
44. स्वाहा॒ तिष्ठ॑ते॒ तिष्ठ॑ते॒ स्वाहा॒ स्वाहा॒ तिष्ठ॑ते॒ स्वाहा॒ स्वाहा॒ तिष्ठ॑ते॒ स्वाहा॒ स्वाहा॒ तिष्ठ॑ते॒ स्वाहा᳚ । \newline
45. तिष्ठ॑ते॒ स्वाहा॒ स्वाहा॒ तिष्ठ॑ते॒ तिष्ठ॑ते॒ स्वाहा॒ सर्व॑स्मै॒ सर्व॑स्मै॒ स्वाहा॒ तिष्ठ॑ते॒ तिष्ठ॑ते॒ स्वाहा॒ सर्व॑स्मै । \newline
46. स्वाहा॒ सर्व॑स्मै॒ सर्व॑स्मै॒ स्वाहा॒ स्वाहा॒ सर्व॑स्मै॒ स्वाहा॒ स्वाहा॒ सर्व॑स्मै॒ स्वाहा॒ स्वाहा॒ सर्व॑स्मै॒ स्वाहा᳚ । \newline
47. सर्व॑स्मै॒ स्वाहा॒ स्वाहा॒ सर्व॑स्मै॒ सर्व॑स्मै॒ स्वाहा᳚ । \newline
48. स्वाहेति॒ स्वाहा᳚ । \newline
\pagebreak


\end{document}